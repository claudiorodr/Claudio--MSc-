% \section{Results}

% This section presents results from the different set of experiments mentioned in chapter \ref{sub:experiments} \nameref{sub:experiments}. For every experiment, a table comparing every algorithm's displacement and turn error is present. And an error percentage for that shape. And lastly, a 2D and 3D visualization of the best performing algorithm for that experiment is present for every test.

% \subsection{Line}

% The line shape consisted of moving the inertial system in a straight line for a determined distance. Three-line distances were tested: 4, 16, and 28 meter. The results are shown below:

% \subsubsection{4 meter}

% For the 4-meter line experiment, the OLEQ algorithm which had the lowest displacement error with an average of 0.13 meters (3.24\% of error margin), and ROLEQ with an average of 0.24 meters of turn error (6.06\% of error margin).

% \begin{figure}[!h]
%     \centering
%     \begin{table}[H]
    \begin{center}
    \resizebox{1\linewidth}{!}{

        \begin{tabular}[t]{lcccc}
            \hline
            Algorithm                   & Displacement Error[$m$] & Displacement Error[\%]      & Turn Error[$m$]  & Turn Error[\%]             \\
            \hline 
            AngularRate            & 0.16  & 3.90 & 0.50 & 12.62              \\            AQUA            & 0.90  & 22.59 & 1.46 & 36.50              \\            Complementary            & 0.34  & 8.45 & 0.55 & 13.66              \\            Davenport            & 0.49  & 12.28 & 0.64 & 16.02              \\            EKF            & 1.01  & 25.28 & 1.35 & 33.73              \\            FAMC            & 0.17  & 4.25 & 0.50 & 12.57              \\            FLAE            & 0.49  & 12.29 & 0.64 & 16.04              \\            Fourati            & 0.32  & 8.01 & 0.54 & 13.39              \\            Madgwick            & 0.74  & 18.57 & 0.74 & 18.51              \\            Mahony            & 0.25  & 6.21 & 0.55 & 13.80              \\            OLEQ            & 0.13  & 3.24 & 0.41 & 10.28              \\            QUEST            & 2.14  & 53.59 & 2.07 & 51.69              \\            ROLEQ            & 0.16  & 4.06 & 0.24 & 6.06              \\            SAAM            & 0.34  & 8.53 & 0.55 & 13.82              \\            Tilt            & 0.34  & 8.53 & 0.55 & 13.82              \\
            \hline
            Average & 0.53 & 13.32 & 0.75 & 18.84
        \end{tabular}
        }
        \caption{Accelerometer Specifications. }
        \label{tab:accelerometer_specification}
    \end{center}
\end{table}
% \end{figure}

% \begin{figure}[!h]
%     \centering
%     \begin{subfigure}{0.49\textwidth}
%         \centering
%         \resizebox{1\linewidth}{!}{%% Creator: Matplotlib, PGF backend
%%
%% To include the figure in your LaTeX document, write
%%   \input{<filename>.pgf}
%%
%% Make sure the required packages are loaded in your preamble
%%   \usepackage{pgf}
%%
%% and, on pdftex
%%   \usepackage[utf8]{inputenc}\DeclareUnicodeCharacter{2212}{-}
%%
%% or, on luatex and xetex
%%   \usepackage{unicode-math}
%%
%% Figures using additional raster images can only be included by \input if
%% they are in the same directory as the main LaTeX file. For loading figures
%% from other directories you can use the `import` package
%%   \usepackage{import}
%%
%% and then include the figures with
%%   \import{<path to file>}{<filename>.pgf}
%%
%% Matplotlib used the following preamble
%%   \usepackage{fontspec}
%%
\begingroup%
\makeatletter%
\begin{pgfpicture}%
\pgfpathrectangle{\pgfpointorigin}{\pgfqpoint{5.737192in}{4.311000in}}%
\pgfusepath{use as bounding box, clip}%
\begin{pgfscope}%
\pgfsetbuttcap%
\pgfsetmiterjoin%
\definecolor{currentfill}{rgb}{1.000000,1.000000,1.000000}%
\pgfsetfillcolor{currentfill}%
\pgfsetlinewidth{0.000000pt}%
\definecolor{currentstroke}{rgb}{1.000000,1.000000,1.000000}%
\pgfsetstrokecolor{currentstroke}%
\pgfsetdash{}{0pt}%
\pgfpathmoveto{\pgfqpoint{0.000000in}{0.000000in}}%
\pgfpathlineto{\pgfqpoint{5.737192in}{0.000000in}}%
\pgfpathlineto{\pgfqpoint{5.737192in}{4.311000in}}%
\pgfpathlineto{\pgfqpoint{0.000000in}{4.311000in}}%
\pgfpathclose%
\pgfusepath{fill}%
\end{pgfscope}%
\begin{pgfscope}%
\pgfsetbuttcap%
\pgfsetmiterjoin%
\definecolor{currentfill}{rgb}{1.000000,1.000000,1.000000}%
\pgfsetfillcolor{currentfill}%
\pgfsetlinewidth{0.000000pt}%
\definecolor{currentstroke}{rgb}{0.000000,0.000000,0.000000}%
\pgfsetstrokecolor{currentstroke}%
\pgfsetstrokeopacity{0.000000}%
\pgfsetdash{}{0pt}%
\pgfpathmoveto{\pgfqpoint{0.677192in}{0.515000in}}%
\pgfpathlineto{\pgfqpoint{5.637192in}{0.515000in}}%
\pgfpathlineto{\pgfqpoint{5.637192in}{4.211000in}}%
\pgfpathlineto{\pgfqpoint{0.677192in}{4.211000in}}%
\pgfpathclose%
\pgfusepath{fill}%
\end{pgfscope}%
\begin{pgfscope}%
\pgfpathrectangle{\pgfqpoint{0.677192in}{0.515000in}}{\pgfqpoint{4.960000in}{3.696000in}}%
\pgfusepath{clip}%
\pgfsetbuttcap%
\pgfsetroundjoin%
\definecolor{currentfill}{rgb}{0.121569,0.466667,0.705882}%
\pgfsetfillcolor{currentfill}%
\pgfsetlinewidth{1.003750pt}%
\definecolor{currentstroke}{rgb}{0.121569,0.466667,0.705882}%
\pgfsetstrokecolor{currentstroke}%
\pgfsetdash{}{0pt}%
\pgfsys@defobject{currentmarker}{\pgfqpoint{-0.041667in}{-0.041667in}}{\pgfqpoint{0.041667in}{0.041667in}}{%
\pgfpathmoveto{\pgfqpoint{0.000000in}{-0.041667in}}%
\pgfpathcurveto{\pgfqpoint{0.011050in}{-0.041667in}}{\pgfqpoint{0.021649in}{-0.037276in}}{\pgfqpoint{0.029463in}{-0.029463in}}%
\pgfpathcurveto{\pgfqpoint{0.037276in}{-0.021649in}}{\pgfqpoint{0.041667in}{-0.011050in}}{\pgfqpoint{0.041667in}{0.000000in}}%
\pgfpathcurveto{\pgfqpoint{0.041667in}{0.011050in}}{\pgfqpoint{0.037276in}{0.021649in}}{\pgfqpoint{0.029463in}{0.029463in}}%
\pgfpathcurveto{\pgfqpoint{0.021649in}{0.037276in}}{\pgfqpoint{0.011050in}{0.041667in}}{\pgfqpoint{0.000000in}{0.041667in}}%
\pgfpathcurveto{\pgfqpoint{-0.011050in}{0.041667in}}{\pgfqpoint{-0.021649in}{0.037276in}}{\pgfqpoint{-0.029463in}{0.029463in}}%
\pgfpathcurveto{\pgfqpoint{-0.037276in}{0.021649in}}{\pgfqpoint{-0.041667in}{0.011050in}}{\pgfqpoint{-0.041667in}{0.000000in}}%
\pgfpathcurveto{\pgfqpoint{-0.041667in}{-0.011050in}}{\pgfqpoint{-0.037276in}{-0.021649in}}{\pgfqpoint{-0.029463in}{-0.029463in}}%
\pgfpathcurveto{\pgfqpoint{-0.021649in}{-0.037276in}}{\pgfqpoint{-0.011050in}{-0.041667in}}{\pgfqpoint{0.000000in}{-0.041667in}}%
\pgfpathclose%
\pgfusepath{stroke,fill}%
}%
\begin{pgfscope}%
\pgfsys@transformshift{0.926199in}{2.091149in}%
\pgfsys@useobject{currentmarker}{}%
\end{pgfscope}%
\begin{pgfscope}%
\pgfsys@transformshift{0.939301in}{2.087686in}%
\pgfsys@useobject{currentmarker}{}%
\end{pgfscope}%
\begin{pgfscope}%
\pgfsys@transformshift{0.961417in}{2.094800in}%
\pgfsys@useobject{currentmarker}{}%
\end{pgfscope}%
\begin{pgfscope}%
\pgfsys@transformshift{0.998322in}{2.096985in}%
\pgfsys@useobject{currentmarker}{}%
\end{pgfscope}%
\begin{pgfscope}%
\pgfsys@transformshift{1.044619in}{2.105697in}%
\pgfsys@useobject{currentmarker}{}%
\end{pgfscope}%
\begin{pgfscope}%
\pgfsys@transformshift{1.070144in}{2.110125in}%
\pgfsys@useobject{currentmarker}{}%
\end{pgfscope}%
\begin{pgfscope}%
\pgfsys@transformshift{1.084311in}{2.111994in}%
\pgfsys@useobject{currentmarker}{}%
\end{pgfscope}%
\begin{pgfscope}%
\pgfsys@transformshift{1.103609in}{2.117267in}%
\pgfsys@useobject{currentmarker}{}%
\end{pgfscope}%
\begin{pgfscope}%
\pgfsys@transformshift{1.142975in}{2.124424in}%
\pgfsys@useobject{currentmarker}{}%
\end{pgfscope}%
\begin{pgfscope}%
\pgfsys@transformshift{1.199251in}{2.135907in}%
\pgfsys@useobject{currentmarker}{}%
\end{pgfscope}%
\begin{pgfscope}%
\pgfsys@transformshift{1.263500in}{2.149798in}%
\pgfsys@useobject{currentmarker}{}%
\end{pgfscope}%
\begin{pgfscope}%
\pgfsys@transformshift{1.299631in}{2.150570in}%
\pgfsys@useobject{currentmarker}{}%
\end{pgfscope}%
\begin{pgfscope}%
\pgfsys@transformshift{1.319156in}{2.154485in}%
\pgfsys@useobject{currentmarker}{}%
\end{pgfscope}%
\begin{pgfscope}%
\pgfsys@transformshift{1.345430in}{2.154648in}%
\pgfsys@useobject{currentmarker}{}%
\end{pgfscope}%
\begin{pgfscope}%
\pgfsys@transformshift{1.383450in}{2.165203in}%
\pgfsys@useobject{currentmarker}{}%
\end{pgfscope}%
\begin{pgfscope}%
\pgfsys@transformshift{1.443262in}{2.170163in}%
\pgfsys@useobject{currentmarker}{}%
\end{pgfscope}%
\begin{pgfscope}%
\pgfsys@transformshift{1.512773in}{2.175245in}%
\pgfsys@useobject{currentmarker}{}%
\end{pgfscope}%
\begin{pgfscope}%
\pgfsys@transformshift{1.588247in}{2.179569in}%
\pgfsys@useobject{currentmarker}{}%
\end{pgfscope}%
\begin{pgfscope}%
\pgfsys@transformshift{1.676022in}{2.183406in}%
\pgfsys@useobject{currentmarker}{}%
\end{pgfscope}%
\begin{pgfscope}%
\pgfsys@transformshift{1.724295in}{2.185268in}%
\pgfsys@useobject{currentmarker}{}%
\end{pgfscope}%
\begin{pgfscope}%
\pgfsys@transformshift{1.780136in}{2.187024in}%
\pgfsys@useobject{currentmarker}{}%
\end{pgfscope}%
\begin{pgfscope}%
\pgfsys@transformshift{1.843874in}{2.198766in}%
\pgfsys@useobject{currentmarker}{}%
\end{pgfscope}%
\begin{pgfscope}%
\pgfsys@transformshift{1.918433in}{2.196881in}%
\pgfsys@useobject{currentmarker}{}%
\end{pgfscope}%
\begin{pgfscope}%
\pgfsys@transformshift{2.006609in}{2.221149in}%
\pgfsys@useobject{currentmarker}{}%
\end{pgfscope}%
\begin{pgfscope}%
\pgfsys@transformshift{2.112998in}{2.232023in}%
\pgfsys@useobject{currentmarker}{}%
\end{pgfscope}%
\begin{pgfscope}%
\pgfsys@transformshift{2.225343in}{2.253169in}%
\pgfsys@useobject{currentmarker}{}%
\end{pgfscope}%
\begin{pgfscope}%
\pgfsys@transformshift{2.343949in}{2.261908in}%
\pgfsys@useobject{currentmarker}{}%
\end{pgfscope}%
\begin{pgfscope}%
\pgfsys@transformshift{2.409434in}{2.266425in}%
\pgfsys@useobject{currentmarker}{}%
\end{pgfscope}%
\begin{pgfscope}%
\pgfsys@transformshift{2.481725in}{2.273613in}%
\pgfsys@useobject{currentmarker}{}%
\end{pgfscope}%
\begin{pgfscope}%
\pgfsys@transformshift{2.521336in}{2.278560in}%
\pgfsys@useobject{currentmarker}{}%
\end{pgfscope}%
\begin{pgfscope}%
\pgfsys@transformshift{2.569106in}{2.282596in}%
\pgfsys@useobject{currentmarker}{}%
\end{pgfscope}%
\begin{pgfscope}%
\pgfsys@transformshift{2.595125in}{2.286865in}%
\pgfsys@useobject{currentmarker}{}%
\end{pgfscope}%
\begin{pgfscope}%
\pgfsys@transformshift{2.632486in}{2.289128in}%
\pgfsys@useobject{currentmarker}{}%
\end{pgfscope}%
\begin{pgfscope}%
\pgfsys@transformshift{2.681619in}{2.296559in}%
\pgfsys@useobject{currentmarker}{}%
\end{pgfscope}%
\begin{pgfscope}%
\pgfsys@transformshift{2.747725in}{2.301441in}%
\pgfsys@useobject{currentmarker}{}%
\end{pgfscope}%
\begin{pgfscope}%
\pgfsys@transformshift{2.826059in}{2.305809in}%
\pgfsys@useobject{currentmarker}{}%
\end{pgfscope}%
\begin{pgfscope}%
\pgfsys@transformshift{2.912280in}{2.317639in}%
\pgfsys@useobject{currentmarker}{}%
\end{pgfscope}%
\begin{pgfscope}%
\pgfsys@transformshift{3.008126in}{2.320206in}%
\pgfsys@useobject{currentmarker}{}%
\end{pgfscope}%
\begin{pgfscope}%
\pgfsys@transformshift{3.114000in}{2.335286in}%
\pgfsys@useobject{currentmarker}{}%
\end{pgfscope}%
\begin{pgfscope}%
\pgfsys@transformshift{3.172614in}{2.333774in}%
\pgfsys@useobject{currentmarker}{}%
\end{pgfscope}%
\begin{pgfscope}%
\pgfsys@transformshift{3.204747in}{2.336712in}%
\pgfsys@useobject{currentmarker}{}%
\end{pgfscope}%
\begin{pgfscope}%
\pgfsys@transformshift{3.222412in}{2.337839in}%
\pgfsys@useobject{currentmarker}{}%
\end{pgfscope}%
\begin{pgfscope}%
\pgfsys@transformshift{3.232180in}{2.338152in}%
\pgfsys@useobject{currentmarker}{}%
\end{pgfscope}%
\begin{pgfscope}%
\pgfsys@transformshift{3.249270in}{2.339177in}%
\pgfsys@useobject{currentmarker}{}%
\end{pgfscope}%
\begin{pgfscope}%
\pgfsys@transformshift{3.272871in}{2.339738in}%
\pgfsys@useobject{currentmarker}{}%
\end{pgfscope}%
\begin{pgfscope}%
\pgfsys@transformshift{3.304528in}{2.343360in}%
\pgfsys@useobject{currentmarker}{}%
\end{pgfscope}%
\begin{pgfscope}%
\pgfsys@transformshift{3.354102in}{2.342515in}%
\pgfsys@useobject{currentmarker}{}%
\end{pgfscope}%
\begin{pgfscope}%
\pgfsys@transformshift{3.413687in}{2.350369in}%
\pgfsys@useobject{currentmarker}{}%
\end{pgfscope}%
\begin{pgfscope}%
\pgfsys@transformshift{3.482062in}{2.353474in}%
\pgfsys@useobject{currentmarker}{}%
\end{pgfscope}%
\begin{pgfscope}%
\pgfsys@transformshift{3.562393in}{2.359925in}%
\pgfsys@useobject{currentmarker}{}%
\end{pgfscope}%
\begin{pgfscope}%
\pgfsys@transformshift{3.606735in}{2.359403in}%
\pgfsys@useobject{currentmarker}{}%
\end{pgfscope}%
\begin{pgfscope}%
\pgfsys@transformshift{3.630987in}{2.361454in}%
\pgfsys@useobject{currentmarker}{}%
\end{pgfscope}%
\begin{pgfscope}%
\pgfsys@transformshift{3.665633in}{2.364212in}%
\pgfsys@useobject{currentmarker}{}%
\end{pgfscope}%
\begin{pgfscope}%
\pgfsys@transformshift{3.684625in}{2.366090in}%
\pgfsys@useobject{currentmarker}{}%
\end{pgfscope}%
\begin{pgfscope}%
\pgfsys@transformshift{3.709089in}{2.368815in}%
\pgfsys@useobject{currentmarker}{}%
\end{pgfscope}%
\begin{pgfscope}%
\pgfsys@transformshift{3.740497in}{2.372480in}%
\pgfsys@useobject{currentmarker}{}%
\end{pgfscope}%
\begin{pgfscope}%
\pgfsys@transformshift{3.757872in}{2.373786in}%
\pgfsys@useobject{currentmarker}{}%
\end{pgfscope}%
\begin{pgfscope}%
\pgfsys@transformshift{3.780173in}{2.377373in}%
\pgfsys@useobject{currentmarker}{}%
\end{pgfscope}%
\begin{pgfscope}%
\pgfsys@transformshift{3.808274in}{2.379858in}%
\pgfsys@useobject{currentmarker}{}%
\end{pgfscope}%
\begin{pgfscope}%
\pgfsys@transformshift{3.843098in}{2.384448in}%
\pgfsys@useobject{currentmarker}{}%
\end{pgfscope}%
\begin{pgfscope}%
\pgfsys@transformshift{3.883709in}{2.386368in}%
\pgfsys@useobject{currentmarker}{}%
\end{pgfscope}%
\begin{pgfscope}%
\pgfsys@transformshift{3.929272in}{2.392018in}%
\pgfsys@useobject{currentmarker}{}%
\end{pgfscope}%
\begin{pgfscope}%
\pgfsys@transformshift{3.981035in}{2.395843in}%
\pgfsys@useobject{currentmarker}{}%
\end{pgfscope}%
\begin{pgfscope}%
\pgfsys@transformshift{4.037292in}{2.404255in}%
\pgfsys@useobject{currentmarker}{}%
\end{pgfscope}%
\begin{pgfscope}%
\pgfsys@transformshift{4.100626in}{2.407701in}%
\pgfsys@useobject{currentmarker}{}%
\end{pgfscope}%
\begin{pgfscope}%
\pgfsys@transformshift{4.169356in}{2.418104in}%
\pgfsys@useobject{currentmarker}{}%
\end{pgfscope}%
\begin{pgfscope}%
\pgfsys@transformshift{4.244470in}{2.423685in}%
\pgfsys@useobject{currentmarker}{}%
\end{pgfscope}%
\begin{pgfscope}%
\pgfsys@transformshift{4.325591in}{2.434705in}%
\pgfsys@useobject{currentmarker}{}%
\end{pgfscope}%
\begin{pgfscope}%
\pgfsys@transformshift{4.412528in}{2.434546in}%
\pgfsys@useobject{currentmarker}{}%
\end{pgfscope}%
\begin{pgfscope}%
\pgfsys@transformshift{4.504055in}{2.445602in}%
\pgfsys@useobject{currentmarker}{}%
\end{pgfscope}%
\begin{pgfscope}%
\pgfsys@transformshift{4.602627in}{2.441811in}%
\pgfsys@useobject{currentmarker}{}%
\end{pgfscope}%
\begin{pgfscope}%
\pgfsys@transformshift{4.708086in}{2.452711in}%
\pgfsys@useobject{currentmarker}{}%
\end{pgfscope}%
\begin{pgfscope}%
\pgfsys@transformshift{4.766265in}{2.448148in}%
\pgfsys@useobject{currentmarker}{}%
\end{pgfscope}%
\begin{pgfscope}%
\pgfsys@transformshift{4.798297in}{2.449943in}%
\pgfsys@useobject{currentmarker}{}%
\end{pgfscope}%
\begin{pgfscope}%
\pgfsys@transformshift{4.815805in}{2.448056in}%
\pgfsys@useobject{currentmarker}{}%
\end{pgfscope}%
\begin{pgfscope}%
\pgfsys@transformshift{4.825450in}{2.448875in}%
\pgfsys@useobject{currentmarker}{}%
\end{pgfscope}%
\begin{pgfscope}%
\pgfsys@transformshift{4.830764in}{2.448362in}%
\pgfsys@useobject{currentmarker}{}%
\end{pgfscope}%
\begin{pgfscope}%
\pgfsys@transformshift{4.833673in}{2.448724in}%
\pgfsys@useobject{currentmarker}{}%
\end{pgfscope}%
\begin{pgfscope}%
\pgfsys@transformshift{4.835283in}{2.448620in}%
\pgfsys@useobject{currentmarker}{}%
\end{pgfscope}%
\begin{pgfscope}%
\pgfsys@transformshift{4.836171in}{2.448618in}%
\pgfsys@useobject{currentmarker}{}%
\end{pgfscope}%
\begin{pgfscope}%
\pgfsys@transformshift{4.836659in}{2.448600in}%
\pgfsys@useobject{currentmarker}{}%
\end{pgfscope}%
\begin{pgfscope}%
\pgfsys@transformshift{4.836927in}{2.448585in}%
\pgfsys@useobject{currentmarker}{}%
\end{pgfscope}%
\begin{pgfscope}%
\pgfsys@transformshift{4.837074in}{2.448588in}%
\pgfsys@useobject{currentmarker}{}%
\end{pgfscope}%
\begin{pgfscope}%
\pgfsys@transformshift{4.837156in}{2.448588in}%
\pgfsys@useobject{currentmarker}{}%
\end{pgfscope}%
\begin{pgfscope}%
\pgfsys@transformshift{4.837200in}{2.448589in}%
\pgfsys@useobject{currentmarker}{}%
\end{pgfscope}%
\begin{pgfscope}%
\pgfsys@transformshift{4.837225in}{2.448587in}%
\pgfsys@useobject{currentmarker}{}%
\end{pgfscope}%
\begin{pgfscope}%
\pgfsys@transformshift{4.837238in}{2.448587in}%
\pgfsys@useobject{currentmarker}{}%
\end{pgfscope}%
\begin{pgfscope}%
\pgfsys@transformshift{4.837246in}{2.448587in}%
\pgfsys@useobject{currentmarker}{}%
\end{pgfscope}%
\begin{pgfscope}%
\pgfsys@transformshift{4.837250in}{2.448587in}%
\pgfsys@useobject{currentmarker}{}%
\end{pgfscope}%
\begin{pgfscope}%
\pgfsys@transformshift{4.837252in}{2.448587in}%
\pgfsys@useobject{currentmarker}{}%
\end{pgfscope}%
\begin{pgfscope}%
\pgfsys@transformshift{4.837253in}{2.448587in}%
\pgfsys@useobject{currentmarker}{}%
\end{pgfscope}%
\begin{pgfscope}%
\pgfsys@transformshift{4.837254in}{2.448587in}%
\pgfsys@useobject{currentmarker}{}%
\end{pgfscope}%
\begin{pgfscope}%
\pgfsys@transformshift{4.837254in}{2.448587in}%
\pgfsys@useobject{currentmarker}{}%
\end{pgfscope}%
\begin{pgfscope}%
\pgfsys@transformshift{4.837255in}{2.448587in}%
\pgfsys@useobject{currentmarker}{}%
\end{pgfscope}%
\begin{pgfscope}%
\pgfsys@transformshift{4.837255in}{2.448587in}%
\pgfsys@useobject{currentmarker}{}%
\end{pgfscope}%
\begin{pgfscope}%
\pgfsys@transformshift{4.837255in}{2.448587in}%
\pgfsys@useobject{currentmarker}{}%
\end{pgfscope}%
\begin{pgfscope}%
\pgfsys@transformshift{4.837255in}{2.448587in}%
\pgfsys@useobject{currentmarker}{}%
\end{pgfscope}%
\begin{pgfscope}%
\pgfsys@transformshift{4.837255in}{2.448587in}%
\pgfsys@useobject{currentmarker}{}%
\end{pgfscope}%
\begin{pgfscope}%
\pgfsys@transformshift{4.837255in}{2.448587in}%
\pgfsys@useobject{currentmarker}{}%
\end{pgfscope}%
\begin{pgfscope}%
\pgfsys@transformshift{4.837255in}{2.448587in}%
\pgfsys@useobject{currentmarker}{}%
\end{pgfscope}%
\begin{pgfscope}%
\pgfsys@transformshift{4.837255in}{2.448587in}%
\pgfsys@useobject{currentmarker}{}%
\end{pgfscope}%
\begin{pgfscope}%
\pgfsys@transformshift{4.837255in}{2.448587in}%
\pgfsys@useobject{currentmarker}{}%
\end{pgfscope}%
\begin{pgfscope}%
\pgfsys@transformshift{4.837255in}{2.448587in}%
\pgfsys@useobject{currentmarker}{}%
\end{pgfscope}%
\begin{pgfscope}%
\pgfsys@transformshift{4.842645in}{2.448104in}%
\pgfsys@useobject{currentmarker}{}%
\end{pgfscope}%
\begin{pgfscope}%
\pgfsys@transformshift{4.845620in}{2.447985in}%
\pgfsys@useobject{currentmarker}{}%
\end{pgfscope}%
\begin{pgfscope}%
\pgfsys@transformshift{4.847259in}{2.447888in}%
\pgfsys@useobject{currentmarker}{}%
\end{pgfscope}%
\begin{pgfscope}%
\pgfsys@transformshift{4.848156in}{2.447823in}%
\pgfsys@useobject{currentmarker}{}%
\end{pgfscope}%
\begin{pgfscope}%
\pgfsys@transformshift{4.848650in}{2.447784in}%
\pgfsys@useobject{currentmarker}{}%
\end{pgfscope}%
\begin{pgfscope}%
\pgfsys@transformshift{4.848921in}{2.447776in}%
\pgfsys@useobject{currentmarker}{}%
\end{pgfscope}%
\begin{pgfscope}%
\pgfsys@transformshift{4.849071in}{2.447769in}%
\pgfsys@useobject{currentmarker}{}%
\end{pgfscope}%
\begin{pgfscope}%
\pgfsys@transformshift{4.849153in}{2.447768in}%
\pgfsys@useobject{currentmarker}{}%
\end{pgfscope}%
\begin{pgfscope}%
\pgfsys@transformshift{4.849198in}{2.447775in}%
\pgfsys@useobject{currentmarker}{}%
\end{pgfscope}%
\begin{pgfscope}%
\pgfsys@transformshift{4.849221in}{2.447785in}%
\pgfsys@useobject{currentmarker}{}%
\end{pgfscope}%
\begin{pgfscope}%
\pgfsys@transformshift{4.895228in}{2.461900in}%
\pgfsys@useobject{currentmarker}{}%
\end{pgfscope}%
\begin{pgfscope}%
\pgfsys@transformshift{4.957378in}{2.513327in}%
\pgfsys@useobject{currentmarker}{}%
\end{pgfscope}%
\begin{pgfscope}%
\pgfsys@transformshift{5.000452in}{2.502789in}%
\pgfsys@useobject{currentmarker}{}%
\end{pgfscope}%
\begin{pgfscope}%
\pgfsys@transformshift{5.069550in}{2.544416in}%
\pgfsys@useobject{currentmarker}{}%
\end{pgfscope}%
\begin{pgfscope}%
\pgfsys@transformshift{5.157270in}{2.541559in}%
\pgfsys@useobject{currentmarker}{}%
\end{pgfscope}%
\begin{pgfscope}%
\pgfsys@transformshift{5.246429in}{2.591786in}%
\pgfsys@useobject{currentmarker}{}%
\end{pgfscope}%
\begin{pgfscope}%
\pgfsys@transformshift{5.353165in}{2.624212in}%
\pgfsys@useobject{currentmarker}{}%
\end{pgfscope}%
\begin{pgfscope}%
\pgfsys@transformshift{5.411737in}{2.642006in}%
\pgfsys@useobject{currentmarker}{}%
\end{pgfscope}%
\end{pgfscope}%
\begin{pgfscope}%
\pgfsetbuttcap%
\pgfsetroundjoin%
\definecolor{currentfill}{rgb}{0.000000,0.000000,0.000000}%
\pgfsetfillcolor{currentfill}%
\pgfsetlinewidth{0.803000pt}%
\definecolor{currentstroke}{rgb}{0.000000,0.000000,0.000000}%
\pgfsetstrokecolor{currentstroke}%
\pgfsetdash{}{0pt}%
\pgfsys@defobject{currentmarker}{\pgfqpoint{0.000000in}{-0.048611in}}{\pgfqpoint{0.000000in}{0.000000in}}{%
\pgfpathmoveto{\pgfqpoint{0.000000in}{0.000000in}}%
\pgfpathlineto{\pgfqpoint{0.000000in}{-0.048611in}}%
\pgfusepath{stroke,fill}%
}%
\begin{pgfscope}%
\pgfsys@transformshift{0.902646in}{0.515000in}%
\pgfsys@useobject{currentmarker}{}%
\end{pgfscope}%
\end{pgfscope}%
\begin{pgfscope}%
\definecolor{textcolor}{rgb}{0.000000,0.000000,0.000000}%
\pgfsetstrokecolor{textcolor}%
\pgfsetfillcolor{textcolor}%
\pgftext[x=0.902646in,y=0.417777in,,top]{\color{textcolor}\rmfamily\fontsize{10.000000}{12.000000}\selectfont \(\displaystyle {0}\)}%
\end{pgfscope}%
\begin{pgfscope}%
\pgfsetbuttcap%
\pgfsetroundjoin%
\definecolor{currentfill}{rgb}{0.000000,0.000000,0.000000}%
\pgfsetfillcolor{currentfill}%
\pgfsetlinewidth{0.803000pt}%
\definecolor{currentstroke}{rgb}{0.000000,0.000000,0.000000}%
\pgfsetstrokecolor{currentstroke}%
\pgfsetdash{}{0pt}%
\pgfsys@defobject{currentmarker}{\pgfqpoint{0.000000in}{-0.048611in}}{\pgfqpoint{0.000000in}{0.000000in}}{%
\pgfpathmoveto{\pgfqpoint{0.000000in}{0.000000in}}%
\pgfpathlineto{\pgfqpoint{0.000000in}{-0.048611in}}%
\pgfusepath{stroke,fill}%
}%
\begin{pgfscope}%
\pgfsys@transformshift{1.824566in}{0.515000in}%
\pgfsys@useobject{currentmarker}{}%
\end{pgfscope}%
\end{pgfscope}%
\begin{pgfscope}%
\definecolor{textcolor}{rgb}{0.000000,0.000000,0.000000}%
\pgfsetstrokecolor{textcolor}%
\pgfsetfillcolor{textcolor}%
\pgftext[x=1.824566in,y=0.417777in,,top]{\color{textcolor}\rmfamily\fontsize{10.000000}{12.000000}\selectfont \(\displaystyle {1}\)}%
\end{pgfscope}%
\begin{pgfscope}%
\pgfsetbuttcap%
\pgfsetroundjoin%
\definecolor{currentfill}{rgb}{0.000000,0.000000,0.000000}%
\pgfsetfillcolor{currentfill}%
\pgfsetlinewidth{0.803000pt}%
\definecolor{currentstroke}{rgb}{0.000000,0.000000,0.000000}%
\pgfsetstrokecolor{currentstroke}%
\pgfsetdash{}{0pt}%
\pgfsys@defobject{currentmarker}{\pgfqpoint{0.000000in}{-0.048611in}}{\pgfqpoint{0.000000in}{0.000000in}}{%
\pgfpathmoveto{\pgfqpoint{0.000000in}{0.000000in}}%
\pgfpathlineto{\pgfqpoint{0.000000in}{-0.048611in}}%
\pgfusepath{stroke,fill}%
}%
\begin{pgfscope}%
\pgfsys@transformshift{2.746486in}{0.515000in}%
\pgfsys@useobject{currentmarker}{}%
\end{pgfscope}%
\end{pgfscope}%
\begin{pgfscope}%
\definecolor{textcolor}{rgb}{0.000000,0.000000,0.000000}%
\pgfsetstrokecolor{textcolor}%
\pgfsetfillcolor{textcolor}%
\pgftext[x=2.746486in,y=0.417777in,,top]{\color{textcolor}\rmfamily\fontsize{10.000000}{12.000000}\selectfont \(\displaystyle {2}\)}%
\end{pgfscope}%
\begin{pgfscope}%
\pgfsetbuttcap%
\pgfsetroundjoin%
\definecolor{currentfill}{rgb}{0.000000,0.000000,0.000000}%
\pgfsetfillcolor{currentfill}%
\pgfsetlinewidth{0.803000pt}%
\definecolor{currentstroke}{rgb}{0.000000,0.000000,0.000000}%
\pgfsetstrokecolor{currentstroke}%
\pgfsetdash{}{0pt}%
\pgfsys@defobject{currentmarker}{\pgfqpoint{0.000000in}{-0.048611in}}{\pgfqpoint{0.000000in}{0.000000in}}{%
\pgfpathmoveto{\pgfqpoint{0.000000in}{0.000000in}}%
\pgfpathlineto{\pgfqpoint{0.000000in}{-0.048611in}}%
\pgfusepath{stroke,fill}%
}%
\begin{pgfscope}%
\pgfsys@transformshift{3.668405in}{0.515000in}%
\pgfsys@useobject{currentmarker}{}%
\end{pgfscope}%
\end{pgfscope}%
\begin{pgfscope}%
\definecolor{textcolor}{rgb}{0.000000,0.000000,0.000000}%
\pgfsetstrokecolor{textcolor}%
\pgfsetfillcolor{textcolor}%
\pgftext[x=3.668405in,y=0.417777in,,top]{\color{textcolor}\rmfamily\fontsize{10.000000}{12.000000}\selectfont \(\displaystyle {3}\)}%
\end{pgfscope}%
\begin{pgfscope}%
\pgfsetbuttcap%
\pgfsetroundjoin%
\definecolor{currentfill}{rgb}{0.000000,0.000000,0.000000}%
\pgfsetfillcolor{currentfill}%
\pgfsetlinewidth{0.803000pt}%
\definecolor{currentstroke}{rgb}{0.000000,0.000000,0.000000}%
\pgfsetstrokecolor{currentstroke}%
\pgfsetdash{}{0pt}%
\pgfsys@defobject{currentmarker}{\pgfqpoint{0.000000in}{-0.048611in}}{\pgfqpoint{0.000000in}{0.000000in}}{%
\pgfpathmoveto{\pgfqpoint{0.000000in}{0.000000in}}%
\pgfpathlineto{\pgfqpoint{0.000000in}{-0.048611in}}%
\pgfusepath{stroke,fill}%
}%
\begin{pgfscope}%
\pgfsys@transformshift{4.590325in}{0.515000in}%
\pgfsys@useobject{currentmarker}{}%
\end{pgfscope}%
\end{pgfscope}%
\begin{pgfscope}%
\definecolor{textcolor}{rgb}{0.000000,0.000000,0.000000}%
\pgfsetstrokecolor{textcolor}%
\pgfsetfillcolor{textcolor}%
\pgftext[x=4.590325in,y=0.417777in,,top]{\color{textcolor}\rmfamily\fontsize{10.000000}{12.000000}\selectfont \(\displaystyle {4}\)}%
\end{pgfscope}%
\begin{pgfscope}%
\pgfsetbuttcap%
\pgfsetroundjoin%
\definecolor{currentfill}{rgb}{0.000000,0.000000,0.000000}%
\pgfsetfillcolor{currentfill}%
\pgfsetlinewidth{0.803000pt}%
\definecolor{currentstroke}{rgb}{0.000000,0.000000,0.000000}%
\pgfsetstrokecolor{currentstroke}%
\pgfsetdash{}{0pt}%
\pgfsys@defobject{currentmarker}{\pgfqpoint{0.000000in}{-0.048611in}}{\pgfqpoint{0.000000in}{0.000000in}}{%
\pgfpathmoveto{\pgfqpoint{0.000000in}{0.000000in}}%
\pgfpathlineto{\pgfqpoint{0.000000in}{-0.048611in}}%
\pgfusepath{stroke,fill}%
}%
\begin{pgfscope}%
\pgfsys@transformshift{5.512245in}{0.515000in}%
\pgfsys@useobject{currentmarker}{}%
\end{pgfscope}%
\end{pgfscope}%
\begin{pgfscope}%
\definecolor{textcolor}{rgb}{0.000000,0.000000,0.000000}%
\pgfsetstrokecolor{textcolor}%
\pgfsetfillcolor{textcolor}%
\pgftext[x=5.512245in,y=0.417777in,,top]{\color{textcolor}\rmfamily\fontsize{10.000000}{12.000000}\selectfont \(\displaystyle {5}\)}%
\end{pgfscope}%
\begin{pgfscope}%
\definecolor{textcolor}{rgb}{0.000000,0.000000,0.000000}%
\pgfsetstrokecolor{textcolor}%
\pgfsetfillcolor{textcolor}%
\pgftext[x=3.157192in,y=0.238889in,,top]{\color{textcolor}\rmfamily\fontsize{10.000000}{12.000000}\selectfont Position X [\(\displaystyle m\)]}%
\end{pgfscope}%
\begin{pgfscope}%
\pgfsetbuttcap%
\pgfsetroundjoin%
\definecolor{currentfill}{rgb}{0.000000,0.000000,0.000000}%
\pgfsetfillcolor{currentfill}%
\pgfsetlinewidth{0.803000pt}%
\definecolor{currentstroke}{rgb}{0.000000,0.000000,0.000000}%
\pgfsetstrokecolor{currentstroke}%
\pgfsetdash{}{0pt}%
\pgfsys@defobject{currentmarker}{\pgfqpoint{-0.048611in}{0.000000in}}{\pgfqpoint{-0.000000in}{0.000000in}}{%
\pgfpathmoveto{\pgfqpoint{-0.000000in}{0.000000in}}%
\pgfpathlineto{\pgfqpoint{-0.048611in}{0.000000in}}%
\pgfusepath{stroke,fill}%
}%
\begin{pgfscope}%
\pgfsys@transformshift{0.677192in}{0.701114in}%
\pgfsys@useobject{currentmarker}{}%
\end{pgfscope}%
\end{pgfscope}%
\begin{pgfscope}%
\definecolor{textcolor}{rgb}{0.000000,0.000000,0.000000}%
\pgfsetstrokecolor{textcolor}%
\pgfsetfillcolor{textcolor}%
\pgftext[x=0.294444in, y=0.652919in, left, base]{\color{textcolor}\rmfamily\fontsize{10.000000}{12.000000}\selectfont \(\displaystyle {−1.5}\)}%
\end{pgfscope}%
\begin{pgfscope}%
\pgfsetbuttcap%
\pgfsetroundjoin%
\definecolor{currentfill}{rgb}{0.000000,0.000000,0.000000}%
\pgfsetfillcolor{currentfill}%
\pgfsetlinewidth{0.803000pt}%
\definecolor{currentstroke}{rgb}{0.000000,0.000000,0.000000}%
\pgfsetstrokecolor{currentstroke}%
\pgfsetdash{}{0pt}%
\pgfsys@defobject{currentmarker}{\pgfqpoint{-0.048611in}{0.000000in}}{\pgfqpoint{-0.000000in}{0.000000in}}{%
\pgfpathmoveto{\pgfqpoint{-0.000000in}{0.000000in}}%
\pgfpathlineto{\pgfqpoint{-0.048611in}{0.000000in}}%
\pgfusepath{stroke,fill}%
}%
\begin{pgfscope}%
\pgfsys@transformshift{0.677192in}{1.162074in}%
\pgfsys@useobject{currentmarker}{}%
\end{pgfscope}%
\end{pgfscope}%
\begin{pgfscope}%
\definecolor{textcolor}{rgb}{0.000000,0.000000,0.000000}%
\pgfsetstrokecolor{textcolor}%
\pgfsetfillcolor{textcolor}%
\pgftext[x=0.294444in, y=1.113879in, left, base]{\color{textcolor}\rmfamily\fontsize{10.000000}{12.000000}\selectfont \(\displaystyle {−1.0}\)}%
\end{pgfscope}%
\begin{pgfscope}%
\pgfsetbuttcap%
\pgfsetroundjoin%
\definecolor{currentfill}{rgb}{0.000000,0.000000,0.000000}%
\pgfsetfillcolor{currentfill}%
\pgfsetlinewidth{0.803000pt}%
\definecolor{currentstroke}{rgb}{0.000000,0.000000,0.000000}%
\pgfsetstrokecolor{currentstroke}%
\pgfsetdash{}{0pt}%
\pgfsys@defobject{currentmarker}{\pgfqpoint{-0.048611in}{0.000000in}}{\pgfqpoint{-0.000000in}{0.000000in}}{%
\pgfpathmoveto{\pgfqpoint{-0.000000in}{0.000000in}}%
\pgfpathlineto{\pgfqpoint{-0.048611in}{0.000000in}}%
\pgfusepath{stroke,fill}%
}%
\begin{pgfscope}%
\pgfsys@transformshift{0.677192in}{1.623034in}%
\pgfsys@useobject{currentmarker}{}%
\end{pgfscope}%
\end{pgfscope}%
\begin{pgfscope}%
\definecolor{textcolor}{rgb}{0.000000,0.000000,0.000000}%
\pgfsetstrokecolor{textcolor}%
\pgfsetfillcolor{textcolor}%
\pgftext[x=0.294444in, y=1.574839in, left, base]{\color{textcolor}\rmfamily\fontsize{10.000000}{12.000000}\selectfont \(\displaystyle {−0.5}\)}%
\end{pgfscope}%
\begin{pgfscope}%
\pgfsetbuttcap%
\pgfsetroundjoin%
\definecolor{currentfill}{rgb}{0.000000,0.000000,0.000000}%
\pgfsetfillcolor{currentfill}%
\pgfsetlinewidth{0.803000pt}%
\definecolor{currentstroke}{rgb}{0.000000,0.000000,0.000000}%
\pgfsetstrokecolor{currentstroke}%
\pgfsetdash{}{0pt}%
\pgfsys@defobject{currentmarker}{\pgfqpoint{-0.048611in}{0.000000in}}{\pgfqpoint{-0.000000in}{0.000000in}}{%
\pgfpathmoveto{\pgfqpoint{-0.000000in}{0.000000in}}%
\pgfpathlineto{\pgfqpoint{-0.048611in}{0.000000in}}%
\pgfusepath{stroke,fill}%
}%
\begin{pgfscope}%
\pgfsys@transformshift{0.677192in}{2.083993in}%
\pgfsys@useobject{currentmarker}{}%
\end{pgfscope}%
\end{pgfscope}%
\begin{pgfscope}%
\definecolor{textcolor}{rgb}{0.000000,0.000000,0.000000}%
\pgfsetstrokecolor{textcolor}%
\pgfsetfillcolor{textcolor}%
\pgftext[x=0.402500in, y=2.035799in, left, base]{\color{textcolor}\rmfamily\fontsize{10.000000}{12.000000}\selectfont \(\displaystyle {0.0}\)}%
\end{pgfscope}%
\begin{pgfscope}%
\pgfsetbuttcap%
\pgfsetroundjoin%
\definecolor{currentfill}{rgb}{0.000000,0.000000,0.000000}%
\pgfsetfillcolor{currentfill}%
\pgfsetlinewidth{0.803000pt}%
\definecolor{currentstroke}{rgb}{0.000000,0.000000,0.000000}%
\pgfsetstrokecolor{currentstroke}%
\pgfsetdash{}{0pt}%
\pgfsys@defobject{currentmarker}{\pgfqpoint{-0.048611in}{0.000000in}}{\pgfqpoint{-0.000000in}{0.000000in}}{%
\pgfpathmoveto{\pgfqpoint{-0.000000in}{0.000000in}}%
\pgfpathlineto{\pgfqpoint{-0.048611in}{0.000000in}}%
\pgfusepath{stroke,fill}%
}%
\begin{pgfscope}%
\pgfsys@transformshift{0.677192in}{2.544953in}%
\pgfsys@useobject{currentmarker}{}%
\end{pgfscope}%
\end{pgfscope}%
\begin{pgfscope}%
\definecolor{textcolor}{rgb}{0.000000,0.000000,0.000000}%
\pgfsetstrokecolor{textcolor}%
\pgfsetfillcolor{textcolor}%
\pgftext[x=0.402500in, y=2.496759in, left, base]{\color{textcolor}\rmfamily\fontsize{10.000000}{12.000000}\selectfont \(\displaystyle {0.5}\)}%
\end{pgfscope}%
\begin{pgfscope}%
\pgfsetbuttcap%
\pgfsetroundjoin%
\definecolor{currentfill}{rgb}{0.000000,0.000000,0.000000}%
\pgfsetfillcolor{currentfill}%
\pgfsetlinewidth{0.803000pt}%
\definecolor{currentstroke}{rgb}{0.000000,0.000000,0.000000}%
\pgfsetstrokecolor{currentstroke}%
\pgfsetdash{}{0pt}%
\pgfsys@defobject{currentmarker}{\pgfqpoint{-0.048611in}{0.000000in}}{\pgfqpoint{-0.000000in}{0.000000in}}{%
\pgfpathmoveto{\pgfqpoint{-0.000000in}{0.000000in}}%
\pgfpathlineto{\pgfqpoint{-0.048611in}{0.000000in}}%
\pgfusepath{stroke,fill}%
}%
\begin{pgfscope}%
\pgfsys@transformshift{0.677192in}{3.005913in}%
\pgfsys@useobject{currentmarker}{}%
\end{pgfscope}%
\end{pgfscope}%
\begin{pgfscope}%
\definecolor{textcolor}{rgb}{0.000000,0.000000,0.000000}%
\pgfsetstrokecolor{textcolor}%
\pgfsetfillcolor{textcolor}%
\pgftext[x=0.402500in, y=2.957719in, left, base]{\color{textcolor}\rmfamily\fontsize{10.000000}{12.000000}\selectfont \(\displaystyle {1.0}\)}%
\end{pgfscope}%
\begin{pgfscope}%
\pgfsetbuttcap%
\pgfsetroundjoin%
\definecolor{currentfill}{rgb}{0.000000,0.000000,0.000000}%
\pgfsetfillcolor{currentfill}%
\pgfsetlinewidth{0.803000pt}%
\definecolor{currentstroke}{rgb}{0.000000,0.000000,0.000000}%
\pgfsetstrokecolor{currentstroke}%
\pgfsetdash{}{0pt}%
\pgfsys@defobject{currentmarker}{\pgfqpoint{-0.048611in}{0.000000in}}{\pgfqpoint{-0.000000in}{0.000000in}}{%
\pgfpathmoveto{\pgfqpoint{-0.000000in}{0.000000in}}%
\pgfpathlineto{\pgfqpoint{-0.048611in}{0.000000in}}%
\pgfusepath{stroke,fill}%
}%
\begin{pgfscope}%
\pgfsys@transformshift{0.677192in}{3.466873in}%
\pgfsys@useobject{currentmarker}{}%
\end{pgfscope}%
\end{pgfscope}%
\begin{pgfscope}%
\definecolor{textcolor}{rgb}{0.000000,0.000000,0.000000}%
\pgfsetstrokecolor{textcolor}%
\pgfsetfillcolor{textcolor}%
\pgftext[x=0.402500in, y=3.418679in, left, base]{\color{textcolor}\rmfamily\fontsize{10.000000}{12.000000}\selectfont \(\displaystyle {1.5}\)}%
\end{pgfscope}%
\begin{pgfscope}%
\pgfsetbuttcap%
\pgfsetroundjoin%
\definecolor{currentfill}{rgb}{0.000000,0.000000,0.000000}%
\pgfsetfillcolor{currentfill}%
\pgfsetlinewidth{0.803000pt}%
\definecolor{currentstroke}{rgb}{0.000000,0.000000,0.000000}%
\pgfsetstrokecolor{currentstroke}%
\pgfsetdash{}{0pt}%
\pgfsys@defobject{currentmarker}{\pgfqpoint{-0.048611in}{0.000000in}}{\pgfqpoint{-0.000000in}{0.000000in}}{%
\pgfpathmoveto{\pgfqpoint{-0.000000in}{0.000000in}}%
\pgfpathlineto{\pgfqpoint{-0.048611in}{0.000000in}}%
\pgfusepath{stroke,fill}%
}%
\begin{pgfscope}%
\pgfsys@transformshift{0.677192in}{3.927833in}%
\pgfsys@useobject{currentmarker}{}%
\end{pgfscope}%
\end{pgfscope}%
\begin{pgfscope}%
\definecolor{textcolor}{rgb}{0.000000,0.000000,0.000000}%
\pgfsetstrokecolor{textcolor}%
\pgfsetfillcolor{textcolor}%
\pgftext[x=0.402500in, y=3.879638in, left, base]{\color{textcolor}\rmfamily\fontsize{10.000000}{12.000000}\selectfont \(\displaystyle {2.0}\)}%
\end{pgfscope}%
\begin{pgfscope}%
\definecolor{textcolor}{rgb}{0.000000,0.000000,0.000000}%
\pgfsetstrokecolor{textcolor}%
\pgfsetfillcolor{textcolor}%
\pgftext[x=0.238889in,y=2.363000in,,bottom,rotate=90.000000]{\color{textcolor}\rmfamily\fontsize{10.000000}{12.000000}\selectfont Position Y [\(\displaystyle m\)]}%
\end{pgfscope}%
\begin{pgfscope}%
\pgfpathrectangle{\pgfqpoint{0.677192in}{0.515000in}}{\pgfqpoint{4.960000in}{3.696000in}}%
\pgfusepath{clip}%
\pgfsetrectcap%
\pgfsetroundjoin%
\pgfsetlinewidth{1.505625pt}%
\definecolor{currentstroke}{rgb}{0.121569,0.466667,0.705882}%
\pgfsetstrokecolor{currentstroke}%
\pgfsetdash{}{0pt}%
\pgfpathmoveto{\pgfqpoint{0.902646in}{2.083993in}}%
\pgfpathlineto{\pgfqpoint{4.590325in}{2.083993in}}%
\pgfusepath{stroke}%
\end{pgfscope}%
\begin{pgfscope}%
\pgfsetrectcap%
\pgfsetmiterjoin%
\pgfsetlinewidth{0.803000pt}%
\definecolor{currentstroke}{rgb}{0.000000,0.000000,0.000000}%
\pgfsetstrokecolor{currentstroke}%
\pgfsetdash{}{0pt}%
\pgfpathmoveto{\pgfqpoint{0.677192in}{0.515000in}}%
\pgfpathlineto{\pgfqpoint{0.677192in}{4.211000in}}%
\pgfusepath{stroke}%
\end{pgfscope}%
\begin{pgfscope}%
\pgfsetrectcap%
\pgfsetmiterjoin%
\pgfsetlinewidth{0.803000pt}%
\definecolor{currentstroke}{rgb}{0.000000,0.000000,0.000000}%
\pgfsetstrokecolor{currentstroke}%
\pgfsetdash{}{0pt}%
\pgfpathmoveto{\pgfqpoint{5.637192in}{0.515000in}}%
\pgfpathlineto{\pgfqpoint{5.637192in}{4.211000in}}%
\pgfusepath{stroke}%
\end{pgfscope}%
\begin{pgfscope}%
\pgfsetrectcap%
\pgfsetmiterjoin%
\pgfsetlinewidth{0.803000pt}%
\definecolor{currentstroke}{rgb}{0.000000,0.000000,0.000000}%
\pgfsetstrokecolor{currentstroke}%
\pgfsetdash{}{0pt}%
\pgfpathmoveto{\pgfqpoint{0.677192in}{0.515000in}}%
\pgfpathlineto{\pgfqpoint{5.637192in}{0.515000in}}%
\pgfusepath{stroke}%
\end{pgfscope}%
\begin{pgfscope}%
\pgfsetrectcap%
\pgfsetmiterjoin%
\pgfsetlinewidth{0.803000pt}%
\definecolor{currentstroke}{rgb}{0.000000,0.000000,0.000000}%
\pgfsetstrokecolor{currentstroke}%
\pgfsetdash{}{0pt}%
\pgfpathmoveto{\pgfqpoint{0.677192in}{4.211000in}}%
\pgfpathlineto{\pgfqpoint{5.637192in}{4.211000in}}%
\pgfusepath{stroke}%
\end{pgfscope}%
\begin{pgfscope}%
\pgfsetbuttcap%
\pgfsetmiterjoin%
\definecolor{currentfill}{rgb}{1.000000,1.000000,1.000000}%
\pgfsetfillcolor{currentfill}%
\pgfsetfillopacity{0.800000}%
\pgfsetlinewidth{1.003750pt}%
\definecolor{currentstroke}{rgb}{0.800000,0.800000,0.800000}%
\pgfsetstrokecolor{currentstroke}%
\pgfsetstrokeopacity{0.800000}%
\pgfsetdash{}{0pt}%
\pgfpathmoveto{\pgfqpoint{3.946080in}{3.712667in}}%
\pgfpathlineto{\pgfqpoint{5.539969in}{3.712667in}}%
\pgfpathquadraticcurveto{\pgfqpoint{5.567747in}{3.712667in}}{\pgfqpoint{5.567747in}{3.740444in}}%
\pgfpathlineto{\pgfqpoint{5.567747in}{4.113777in}}%
\pgfpathquadraticcurveto{\pgfqpoint{5.567747in}{4.141555in}}{\pgfqpoint{5.539969in}{4.141555in}}%
\pgfpathlineto{\pgfqpoint{3.946080in}{4.141555in}}%
\pgfpathquadraticcurveto{\pgfqpoint{3.918303in}{4.141555in}}{\pgfqpoint{3.918303in}{4.113777in}}%
\pgfpathlineto{\pgfqpoint{3.918303in}{3.740444in}}%
\pgfpathquadraticcurveto{\pgfqpoint{3.918303in}{3.712667in}}{\pgfqpoint{3.946080in}{3.712667in}}%
\pgfpathclose%
\pgfusepath{stroke,fill}%
\end{pgfscope}%
\begin{pgfscope}%
\pgfsetrectcap%
\pgfsetroundjoin%
\pgfsetlinewidth{1.505625pt}%
\definecolor{currentstroke}{rgb}{0.121569,0.466667,0.705882}%
\pgfsetstrokecolor{currentstroke}%
\pgfsetdash{}{0pt}%
\pgfpathmoveto{\pgfqpoint{3.973858in}{4.037388in}}%
\pgfpathlineto{\pgfqpoint{4.251636in}{4.037388in}}%
\pgfusepath{stroke}%
\end{pgfscope}%
\begin{pgfscope}%
\definecolor{textcolor}{rgb}{0.000000,0.000000,0.000000}%
\pgfsetstrokecolor{textcolor}%
\pgfsetfillcolor{textcolor}%
\pgftext[x=4.362747in,y=3.988777in,left,base]{\color{textcolor}\rmfamily\fontsize{10.000000}{12.000000}\selectfont Ground truth}%
\end{pgfscope}%
\begin{pgfscope}%
\pgfsetbuttcap%
\pgfsetroundjoin%
\definecolor{currentfill}{rgb}{0.121569,0.466667,0.705882}%
\pgfsetfillcolor{currentfill}%
\pgfsetlinewidth{1.003750pt}%
\definecolor{currentstroke}{rgb}{0.121569,0.466667,0.705882}%
\pgfsetstrokecolor{currentstroke}%
\pgfsetdash{}{0pt}%
\pgfsys@defobject{currentmarker}{\pgfqpoint{-0.041667in}{-0.041667in}}{\pgfqpoint{0.041667in}{0.041667in}}{%
\pgfpathmoveto{\pgfqpoint{0.000000in}{-0.041667in}}%
\pgfpathcurveto{\pgfqpoint{0.011050in}{-0.041667in}}{\pgfqpoint{0.021649in}{-0.037276in}}{\pgfqpoint{0.029463in}{-0.029463in}}%
\pgfpathcurveto{\pgfqpoint{0.037276in}{-0.021649in}}{\pgfqpoint{0.041667in}{-0.011050in}}{\pgfqpoint{0.041667in}{0.000000in}}%
\pgfpathcurveto{\pgfqpoint{0.041667in}{0.011050in}}{\pgfqpoint{0.037276in}{0.021649in}}{\pgfqpoint{0.029463in}{0.029463in}}%
\pgfpathcurveto{\pgfqpoint{0.021649in}{0.037276in}}{\pgfqpoint{0.011050in}{0.041667in}}{\pgfqpoint{0.000000in}{0.041667in}}%
\pgfpathcurveto{\pgfqpoint{-0.011050in}{0.041667in}}{\pgfqpoint{-0.021649in}{0.037276in}}{\pgfqpoint{-0.029463in}{0.029463in}}%
\pgfpathcurveto{\pgfqpoint{-0.037276in}{0.021649in}}{\pgfqpoint{-0.041667in}{0.011050in}}{\pgfqpoint{-0.041667in}{0.000000in}}%
\pgfpathcurveto{\pgfqpoint{-0.041667in}{-0.011050in}}{\pgfqpoint{-0.037276in}{-0.021649in}}{\pgfqpoint{-0.029463in}{-0.029463in}}%
\pgfpathcurveto{\pgfqpoint{-0.021649in}{-0.037276in}}{\pgfqpoint{-0.011050in}{-0.041667in}}{\pgfqpoint{0.000000in}{-0.041667in}}%
\pgfpathclose%
\pgfusepath{stroke,fill}%
}%
\begin{pgfscope}%
\pgfsys@transformshift{4.112747in}{3.831625in}%
\pgfsys@useobject{currentmarker}{}%
\end{pgfscope}%
\end{pgfscope}%
\begin{pgfscope}%
\definecolor{textcolor}{rgb}{0.000000,0.000000,0.000000}%
\pgfsetstrokecolor{textcolor}%
\pgfsetfillcolor{textcolor}%
\pgftext[x=4.362747in,y=3.795166in,left,base]{\color{textcolor}\rmfamily\fontsize{10.000000}{12.000000}\selectfont Estimated position}%
\end{pgfscope}%
\end{pgfpicture}%
\makeatother%
\endgroup%
}
%         \caption{ OLEQ's 3D position estimation had the lowest displacement error for the 4-meter line experiment. }
%         \label{fig:line4_2D}
%     \end{subfigure}
%     \begin{subfigure}{0.49\textwidth}
%         \centering
%         \resizebox{1\linewidth}{!}{%% Creator: Matplotlib, PGF backend
%%
%% To include the figure in your LaTeX document, write
%%   \input{<filename>.pgf}
%%
%% Make sure the required packages are loaded in your preamble
%%   \usepackage{pgf}
%%
%% and, on pdftex
%%   \usepackage[utf8]{inputenc}\DeclareUnicodeCharacter{2212}{-}
%%
%% or, on luatex and xetex
%%   \usepackage{unicode-math}
%%
%% Figures using additional raster images can only be included by \input if
%% they are in the same directory as the main LaTeX file. For loading figures
%% from other directories you can use the `import` package
%%   \usepackage{import}
%%
%% and then include the figures with
%%   \import{<path to file>}{<filename>.pgf}
%%
%% Matplotlib used the following preamble
%%   \usepackage{fontspec}
%%
\begingroup%
\makeatletter%
\begin{pgfpicture}%
\pgfpathrectangle{\pgfpointorigin}{\pgfqpoint{4.342355in}{4.209289in}}%
\pgfusepath{use as bounding box, clip}%
\begin{pgfscope}%
\pgfsetbuttcap%
\pgfsetmiterjoin%
\definecolor{currentfill}{rgb}{1.000000,1.000000,1.000000}%
\pgfsetfillcolor{currentfill}%
\pgfsetlinewidth{0.000000pt}%
\definecolor{currentstroke}{rgb}{1.000000,1.000000,1.000000}%
\pgfsetstrokecolor{currentstroke}%
\pgfsetdash{}{0pt}%
\pgfpathmoveto{\pgfqpoint{0.000000in}{-0.000000in}}%
\pgfpathlineto{\pgfqpoint{4.342355in}{-0.000000in}}%
\pgfpathlineto{\pgfqpoint{4.342355in}{4.209289in}}%
\pgfpathlineto{\pgfqpoint{0.000000in}{4.209289in}}%
\pgfpathclose%
\pgfusepath{fill}%
\end{pgfscope}%
\begin{pgfscope}%
\pgfsetbuttcap%
\pgfsetmiterjoin%
\definecolor{currentfill}{rgb}{1.000000,1.000000,1.000000}%
\pgfsetfillcolor{currentfill}%
\pgfsetlinewidth{0.000000pt}%
\definecolor{currentstroke}{rgb}{0.000000,0.000000,0.000000}%
\pgfsetstrokecolor{currentstroke}%
\pgfsetstrokeopacity{0.000000}%
\pgfsetdash{}{0pt}%
\pgfpathmoveto{\pgfqpoint{0.100000in}{0.212622in}}%
\pgfpathlineto{\pgfqpoint{3.796000in}{0.212622in}}%
\pgfpathlineto{\pgfqpoint{3.796000in}{3.908622in}}%
\pgfpathlineto{\pgfqpoint{0.100000in}{3.908622in}}%
\pgfpathclose%
\pgfusepath{fill}%
\end{pgfscope}%
\begin{pgfscope}%
\pgfsetbuttcap%
\pgfsetmiterjoin%
\definecolor{currentfill}{rgb}{0.950000,0.950000,0.950000}%
\pgfsetfillcolor{currentfill}%
\pgfsetfillopacity{0.500000}%
\pgfsetlinewidth{1.003750pt}%
\definecolor{currentstroke}{rgb}{0.950000,0.950000,0.950000}%
\pgfsetstrokecolor{currentstroke}%
\pgfsetstrokeopacity{0.500000}%
\pgfsetdash{}{0pt}%
\pgfpathmoveto{\pgfqpoint{0.379073in}{1.123938in}}%
\pgfpathlineto{\pgfqpoint{1.599613in}{2.147018in}}%
\pgfpathlineto{\pgfqpoint{1.582647in}{3.622484in}}%
\pgfpathlineto{\pgfqpoint{0.303698in}{2.689165in}}%
\pgfusepath{stroke,fill}%
\end{pgfscope}%
\begin{pgfscope}%
\pgfsetbuttcap%
\pgfsetmiterjoin%
\definecolor{currentfill}{rgb}{0.900000,0.900000,0.900000}%
\pgfsetfillcolor{currentfill}%
\pgfsetfillopacity{0.500000}%
\pgfsetlinewidth{1.003750pt}%
\definecolor{currentstroke}{rgb}{0.900000,0.900000,0.900000}%
\pgfsetstrokecolor{currentstroke}%
\pgfsetstrokeopacity{0.500000}%
\pgfsetdash{}{0pt}%
\pgfpathmoveto{\pgfqpoint{1.599613in}{2.147018in}}%
\pgfpathlineto{\pgfqpoint{3.558144in}{1.577751in}}%
\pgfpathlineto{\pgfqpoint{3.628038in}{3.104037in}}%
\pgfpathlineto{\pgfqpoint{1.582647in}{3.622484in}}%
\pgfusepath{stroke,fill}%
\end{pgfscope}%
\begin{pgfscope}%
\pgfsetbuttcap%
\pgfsetmiterjoin%
\definecolor{currentfill}{rgb}{0.925000,0.925000,0.925000}%
\pgfsetfillcolor{currentfill}%
\pgfsetfillopacity{0.500000}%
\pgfsetlinewidth{1.003750pt}%
\definecolor{currentstroke}{rgb}{0.925000,0.925000,0.925000}%
\pgfsetstrokecolor{currentstroke}%
\pgfsetstrokeopacity{0.500000}%
\pgfsetdash{}{0pt}%
\pgfpathmoveto{\pgfqpoint{0.379073in}{1.123938in}}%
\pgfpathlineto{\pgfqpoint{2.455212in}{0.445871in}}%
\pgfpathlineto{\pgfqpoint{3.558144in}{1.577751in}}%
\pgfpathlineto{\pgfqpoint{1.599613in}{2.147018in}}%
\pgfusepath{stroke,fill}%
\end{pgfscope}%
\begin{pgfscope}%
\pgfsetrectcap%
\pgfsetroundjoin%
\pgfsetlinewidth{0.803000pt}%
\definecolor{currentstroke}{rgb}{0.000000,0.000000,0.000000}%
\pgfsetstrokecolor{currentstroke}%
\pgfsetdash{}{0pt}%
\pgfpathmoveto{\pgfqpoint{0.379073in}{1.123938in}}%
\pgfpathlineto{\pgfqpoint{2.455212in}{0.445871in}}%
\pgfusepath{stroke}%
\end{pgfscope}%
\begin{pgfscope}%
\definecolor{textcolor}{rgb}{0.000000,0.000000,0.000000}%
\pgfsetstrokecolor{textcolor}%
\pgfsetfillcolor{textcolor}%
\pgftext[x=0.730374in, y=0.408886in, left, base,rotate=341.912962]{\color{textcolor}\rmfamily\fontsize{10.000000}{12.000000}\selectfont Position X [\(\displaystyle m\)]}%
\end{pgfscope}%
\begin{pgfscope}%
\pgfsetbuttcap%
\pgfsetroundjoin%
\pgfsetlinewidth{0.803000pt}%
\definecolor{currentstroke}{rgb}{0.690196,0.690196,0.690196}%
\pgfsetstrokecolor{currentstroke}%
\pgfsetdash{}{0pt}%
\pgfpathmoveto{\pgfqpoint{0.523763in}{1.076682in}}%
\pgfpathlineto{\pgfqpoint{1.736668in}{2.107182in}}%
\pgfpathlineto{\pgfqpoint{1.725499in}{3.586275in}}%
\pgfusepath{stroke}%
\end{pgfscope}%
\begin{pgfscope}%
\pgfsetbuttcap%
\pgfsetroundjoin%
\pgfsetlinewidth{0.803000pt}%
\definecolor{currentstroke}{rgb}{0.690196,0.690196,0.690196}%
\pgfsetstrokecolor{currentstroke}%
\pgfsetdash{}{0pt}%
\pgfpathmoveto{\pgfqpoint{0.905177in}{0.952112in}}%
\pgfpathlineto{\pgfqpoint{2.097553in}{2.002287in}}%
\pgfpathlineto{\pgfqpoint{2.101851in}{3.490881in}}%
\pgfusepath{stroke}%
\end{pgfscope}%
\begin{pgfscope}%
\pgfsetbuttcap%
\pgfsetroundjoin%
\pgfsetlinewidth{0.803000pt}%
\definecolor{currentstroke}{rgb}{0.690196,0.690196,0.690196}%
\pgfsetstrokecolor{currentstroke}%
\pgfsetdash{}{0pt}%
\pgfpathmoveto{\pgfqpoint{1.294228in}{0.825048in}}%
\pgfpathlineto{\pgfqpoint{2.465059in}{1.895467in}}%
\pgfpathlineto{\pgfqpoint{2.485410in}{3.393660in}}%
\pgfusepath{stroke}%
\end{pgfscope}%
\begin{pgfscope}%
\pgfsetbuttcap%
\pgfsetroundjoin%
\pgfsetlinewidth{0.803000pt}%
\definecolor{currentstroke}{rgb}{0.690196,0.690196,0.690196}%
\pgfsetstrokecolor{currentstroke}%
\pgfsetdash{}{0pt}%
\pgfpathmoveto{\pgfqpoint{1.691147in}{0.695414in}}%
\pgfpathlineto{\pgfqpoint{2.839372in}{1.786670in}}%
\pgfpathlineto{\pgfqpoint{2.876384in}{3.294559in}}%
\pgfusepath{stroke}%
\end{pgfscope}%
\begin{pgfscope}%
\pgfsetbuttcap%
\pgfsetroundjoin%
\pgfsetlinewidth{0.803000pt}%
\definecolor{currentstroke}{rgb}{0.690196,0.690196,0.690196}%
\pgfsetstrokecolor{currentstroke}%
\pgfsetdash{}{0pt}%
\pgfpathmoveto{\pgfqpoint{2.096175in}{0.563132in}}%
\pgfpathlineto{\pgfqpoint{3.220681in}{1.675838in}}%
\pgfpathlineto{\pgfqpoint{3.274992in}{3.193524in}}%
\pgfusepath{stroke}%
\end{pgfscope}%
\begin{pgfscope}%
\pgfsetrectcap%
\pgfsetroundjoin%
\pgfsetlinewidth{0.803000pt}%
\definecolor{currentstroke}{rgb}{0.000000,0.000000,0.000000}%
\pgfsetstrokecolor{currentstroke}%
\pgfsetdash{}{0pt}%
\pgfpathmoveto{\pgfqpoint{0.534325in}{1.085656in}}%
\pgfpathlineto{\pgfqpoint{0.502593in}{1.058696in}}%
\pgfusepath{stroke}%
\end{pgfscope}%
\begin{pgfscope}%
\definecolor{textcolor}{rgb}{0.000000,0.000000,0.000000}%
\pgfsetstrokecolor{textcolor}%
\pgfsetfillcolor{textcolor}%
\pgftext[x=0.419211in,y=0.858347in,,top]{\color{textcolor}\rmfamily\fontsize{10.000000}{12.000000}\selectfont \(\displaystyle {0}\)}%
\end{pgfscope}%
\begin{pgfscope}%
\pgfsetrectcap%
\pgfsetroundjoin%
\pgfsetlinewidth{0.803000pt}%
\definecolor{currentstroke}{rgb}{0.000000,0.000000,0.000000}%
\pgfsetstrokecolor{currentstroke}%
\pgfsetdash{}{0pt}%
\pgfpathmoveto{\pgfqpoint{0.915569in}{0.961265in}}%
\pgfpathlineto{\pgfqpoint{0.884348in}{0.933767in}}%
\pgfusepath{stroke}%
\end{pgfscope}%
\begin{pgfscope}%
\definecolor{textcolor}{rgb}{0.000000,0.000000,0.000000}%
\pgfsetstrokecolor{textcolor}%
\pgfsetfillcolor{textcolor}%
\pgftext[x=0.801018in,y=0.731137in,,top]{\color{textcolor}\rmfamily\fontsize{10.000000}{12.000000}\selectfont \(\displaystyle {1}\)}%
\end{pgfscope}%
\begin{pgfscope}%
\pgfsetrectcap%
\pgfsetroundjoin%
\pgfsetlinewidth{0.803000pt}%
\definecolor{currentstroke}{rgb}{0.000000,0.000000,0.000000}%
\pgfsetstrokecolor{currentstroke}%
\pgfsetdash{}{0pt}%
\pgfpathmoveto{\pgfqpoint{1.304441in}{0.834385in}}%
\pgfpathlineto{\pgfqpoint{1.273758in}{0.806334in}}%
\pgfusepath{stroke}%
\end{pgfscope}%
\begin{pgfscope}%
\definecolor{textcolor}{rgb}{0.000000,0.000000,0.000000}%
\pgfsetstrokecolor{textcolor}%
\pgfsetfillcolor{textcolor}%
\pgftext[x=1.190499in,y=0.601370in,,top]{\color{textcolor}\rmfamily\fontsize{10.000000}{12.000000}\selectfont \(\displaystyle {2}\)}%
\end{pgfscope}%
\begin{pgfscope}%
\pgfsetrectcap%
\pgfsetroundjoin%
\pgfsetlinewidth{0.803000pt}%
\definecolor{currentstroke}{rgb}{0.000000,0.000000,0.000000}%
\pgfsetstrokecolor{currentstroke}%
\pgfsetdash{}{0pt}%
\pgfpathmoveto{\pgfqpoint{1.701171in}{0.704941in}}%
\pgfpathlineto{\pgfqpoint{1.671055in}{0.676319in}}%
\pgfusepath{stroke}%
\end{pgfscope}%
\begin{pgfscope}%
\definecolor{textcolor}{rgb}{0.000000,0.000000,0.000000}%
\pgfsetstrokecolor{textcolor}%
\pgfsetfillcolor{textcolor}%
\pgftext[x=1.587886in,y=0.468969in,,top]{\color{textcolor}\rmfamily\fontsize{10.000000}{12.000000}\selectfont \(\displaystyle {3}\)}%
\end{pgfscope}%
\begin{pgfscope}%
\pgfsetrectcap%
\pgfsetroundjoin%
\pgfsetlinewidth{0.803000pt}%
\definecolor{currentstroke}{rgb}{0.000000,0.000000,0.000000}%
\pgfsetstrokecolor{currentstroke}%
\pgfsetdash{}{0pt}%
\pgfpathmoveto{\pgfqpoint{2.106000in}{0.572854in}}%
\pgfpathlineto{\pgfqpoint{2.076481in}{0.543645in}}%
\pgfusepath{stroke}%
\end{pgfscope}%
\begin{pgfscope}%
\definecolor{textcolor}{rgb}{0.000000,0.000000,0.000000}%
\pgfsetstrokecolor{textcolor}%
\pgfsetfillcolor{textcolor}%
\pgftext[x=1.993423in,y=0.333853in,,top]{\color{textcolor}\rmfamily\fontsize{10.000000}{12.000000}\selectfont \(\displaystyle {4}\)}%
\end{pgfscope}%
\begin{pgfscope}%
\pgfsetrectcap%
\pgfsetroundjoin%
\pgfsetlinewidth{0.803000pt}%
\definecolor{currentstroke}{rgb}{0.000000,0.000000,0.000000}%
\pgfsetstrokecolor{currentstroke}%
\pgfsetdash{}{0pt}%
\pgfpathmoveto{\pgfqpoint{3.558144in}{1.577751in}}%
\pgfpathlineto{\pgfqpoint{2.455212in}{0.445871in}}%
\pgfusepath{stroke}%
\end{pgfscope}%
\begin{pgfscope}%
\definecolor{textcolor}{rgb}{0.000000,0.000000,0.000000}%
\pgfsetstrokecolor{textcolor}%
\pgfsetfillcolor{textcolor}%
\pgftext[x=3.120747in, y=0.305657in, left, base,rotate=45.742112]{\color{textcolor}\rmfamily\fontsize{10.000000}{12.000000}\selectfont Position Y [\(\displaystyle m\)]}%
\end{pgfscope}%
\begin{pgfscope}%
\pgfsetbuttcap%
\pgfsetroundjoin%
\pgfsetlinewidth{0.803000pt}%
\definecolor{currentstroke}{rgb}{0.690196,0.690196,0.690196}%
\pgfsetstrokecolor{currentstroke}%
\pgfsetdash{}{0pt}%
\pgfpathmoveto{\pgfqpoint{0.332888in}{2.710466in}}%
\pgfpathlineto{\pgfqpoint{0.406827in}{1.147201in}}%
\pgfpathlineto{\pgfqpoint{2.480400in}{0.471720in}}%
\pgfusepath{stroke}%
\end{pgfscope}%
\begin{pgfscope}%
\pgfsetbuttcap%
\pgfsetroundjoin%
\pgfsetlinewidth{0.803000pt}%
\definecolor{currentstroke}{rgb}{0.690196,0.690196,0.690196}%
\pgfsetstrokecolor{currentstroke}%
\pgfsetdash{}{0pt}%
\pgfpathmoveto{\pgfqpoint{0.558770in}{2.875305in}}%
\pgfpathlineto{\pgfqpoint{0.621758in}{1.327361in}}%
\pgfpathlineto{\pgfqpoint{2.675289in}{0.671724in}}%
\pgfusepath{stroke}%
\end{pgfscope}%
\begin{pgfscope}%
\pgfsetbuttcap%
\pgfsetroundjoin%
\pgfsetlinewidth{0.803000pt}%
\definecolor{currentstroke}{rgb}{0.690196,0.690196,0.690196}%
\pgfsetstrokecolor{currentstroke}%
\pgfsetdash{}{0pt}%
\pgfpathmoveto{\pgfqpoint{0.777944in}{3.035248in}}%
\pgfpathlineto{\pgfqpoint{0.830580in}{1.502400in}}%
\pgfpathlineto{\pgfqpoint{2.864349in}{0.865746in}}%
\pgfusepath{stroke}%
\end{pgfscope}%
\begin{pgfscope}%
\pgfsetbuttcap%
\pgfsetroundjoin%
\pgfsetlinewidth{0.803000pt}%
\definecolor{currentstroke}{rgb}{0.690196,0.690196,0.690196}%
\pgfsetstrokecolor{currentstroke}%
\pgfsetdash{}{0pt}%
\pgfpathmoveto{\pgfqpoint{0.990703in}{3.190510in}}%
\pgfpathlineto{\pgfqpoint{1.033551in}{1.672534in}}%
\pgfpathlineto{\pgfqpoint{3.047837in}{1.054050in}}%
\pgfusepath{stroke}%
\end{pgfscope}%
\begin{pgfscope}%
\pgfsetbuttcap%
\pgfsetroundjoin%
\pgfsetlinewidth{0.803000pt}%
\definecolor{currentstroke}{rgb}{0.690196,0.690196,0.690196}%
\pgfsetstrokecolor{currentstroke}%
\pgfsetdash{}{0pt}%
\pgfpathmoveto{\pgfqpoint{1.197325in}{3.341293in}}%
\pgfpathlineto{\pgfqpoint{1.230913in}{1.837966in}}%
\pgfpathlineto{\pgfqpoint{3.225996in}{1.236885in}}%
\pgfusepath{stroke}%
\end{pgfscope}%
\begin{pgfscope}%
\pgfsetbuttcap%
\pgfsetroundjoin%
\pgfsetlinewidth{0.803000pt}%
\definecolor{currentstroke}{rgb}{0.690196,0.690196,0.690196}%
\pgfsetstrokecolor{currentstroke}%
\pgfsetdash{}{0pt}%
\pgfpathmoveto{\pgfqpoint{1.398071in}{3.487789in}}%
\pgfpathlineto{\pgfqpoint{1.422894in}{1.998889in}}%
\pgfpathlineto{\pgfqpoint{3.399056in}{1.414486in}}%
\pgfusepath{stroke}%
\end{pgfscope}%
\begin{pgfscope}%
\pgfsetrectcap%
\pgfsetroundjoin%
\pgfsetlinewidth{0.803000pt}%
\definecolor{currentstroke}{rgb}{0.000000,0.000000,0.000000}%
\pgfsetstrokecolor{currentstroke}%
\pgfsetdash{}{0pt}%
\pgfpathmoveto{\pgfqpoint{2.462922in}{0.477413in}}%
\pgfpathlineto{\pgfqpoint{2.515401in}{0.460318in}}%
\pgfusepath{stroke}%
\end{pgfscope}%
\begin{pgfscope}%
\definecolor{textcolor}{rgb}{0.000000,0.000000,0.000000}%
\pgfsetstrokecolor{textcolor}%
\pgfsetfillcolor{textcolor}%
\pgftext[x=2.659983in,y=0.284508in,,top]{\color{textcolor}\rmfamily\fontsize{10.000000}{12.000000}\selectfont \(\displaystyle {-0.3}\)}%
\end{pgfscope}%
\begin{pgfscope}%
\pgfsetrectcap%
\pgfsetroundjoin%
\pgfsetlinewidth{0.803000pt}%
\definecolor{currentstroke}{rgb}{0.000000,0.000000,0.000000}%
\pgfsetstrokecolor{currentstroke}%
\pgfsetdash{}{0pt}%
\pgfpathmoveto{\pgfqpoint{2.657994in}{0.677246in}}%
\pgfpathlineto{\pgfqpoint{2.709924in}{0.660666in}}%
\pgfusepath{stroke}%
\end{pgfscope}%
\begin{pgfscope}%
\definecolor{textcolor}{rgb}{0.000000,0.000000,0.000000}%
\pgfsetstrokecolor{textcolor}%
\pgfsetfillcolor{textcolor}%
\pgftext[x=2.852259in,y=0.487474in,,top]{\color{textcolor}\rmfamily\fontsize{10.000000}{12.000000}\selectfont \(\displaystyle {-0.2}\)}%
\end{pgfscope}%
\begin{pgfscope}%
\pgfsetrectcap%
\pgfsetroundjoin%
\pgfsetlinewidth{0.803000pt}%
\definecolor{currentstroke}{rgb}{0.000000,0.000000,0.000000}%
\pgfsetstrokecolor{currentstroke}%
\pgfsetdash{}{0pt}%
\pgfpathmoveto{\pgfqpoint{2.847233in}{0.871104in}}%
\pgfpathlineto{\pgfqpoint{2.898624in}{0.855016in}}%
\pgfusepath{stroke}%
\end{pgfscope}%
\begin{pgfscope}%
\definecolor{textcolor}{rgb}{0.000000,0.000000,0.000000}%
\pgfsetstrokecolor{textcolor}%
\pgfsetfillcolor{textcolor}%
\pgftext[x=3.038781in,y=0.684366in,,top]{\color{textcolor}\rmfamily\fontsize{10.000000}{12.000000}\selectfont \(\displaystyle {-0.1}\)}%
\end{pgfscope}%
\begin{pgfscope}%
\pgfsetrectcap%
\pgfsetroundjoin%
\pgfsetlinewidth{0.803000pt}%
\definecolor{currentstroke}{rgb}{0.000000,0.000000,0.000000}%
\pgfsetstrokecolor{currentstroke}%
\pgfsetdash{}{0pt}%
\pgfpathmoveto{\pgfqpoint{3.030897in}{1.059251in}}%
\pgfpathlineto{\pgfqpoint{3.081758in}{1.043634in}}%
\pgfusepath{stroke}%
\end{pgfscope}%
\begin{pgfscope}%
\definecolor{textcolor}{rgb}{0.000000,0.000000,0.000000}%
\pgfsetstrokecolor{textcolor}%
\pgfsetfillcolor{textcolor}%
\pgftext[x=3.219803in,y=0.875453in,,top]{\color{textcolor}\rmfamily\fontsize{10.000000}{12.000000}\selectfont \(\displaystyle {0.0}\)}%
\end{pgfscope}%
\begin{pgfscope}%
\pgfsetrectcap%
\pgfsetroundjoin%
\pgfsetlinewidth{0.803000pt}%
\definecolor{currentstroke}{rgb}{0.000000,0.000000,0.000000}%
\pgfsetstrokecolor{currentstroke}%
\pgfsetdash{}{0pt}%
\pgfpathmoveto{\pgfqpoint{3.209230in}{1.241936in}}%
\pgfpathlineto{\pgfqpoint{3.259570in}{1.226770in}}%
\pgfusepath{stroke}%
\end{pgfscope}%
\begin{pgfscope}%
\definecolor{textcolor}{rgb}{0.000000,0.000000,0.000000}%
\pgfsetstrokecolor{textcolor}%
\pgfsetfillcolor{textcolor}%
\pgftext[x=3.395564in,y=1.060986in,,top]{\color{textcolor}\rmfamily\fontsize{10.000000}{12.000000}\selectfont \(\displaystyle {0.1}\)}%
\end{pgfscope}%
\begin{pgfscope}%
\pgfsetrectcap%
\pgfsetroundjoin%
\pgfsetlinewidth{0.803000pt}%
\definecolor{currentstroke}{rgb}{0.000000,0.000000,0.000000}%
\pgfsetstrokecolor{currentstroke}%
\pgfsetdash{}{0pt}%
\pgfpathmoveto{\pgfqpoint{3.382460in}{1.419394in}}%
\pgfpathlineto{\pgfqpoint{3.432287in}{1.404659in}}%
\pgfusepath{stroke}%
\end{pgfscope}%
\begin{pgfscope}%
\definecolor{textcolor}{rgb}{0.000000,0.000000,0.000000}%
\pgfsetstrokecolor{textcolor}%
\pgfsetfillcolor{textcolor}%
\pgftext[x=3.566292in,y=1.241206in,,top]{\color{textcolor}\rmfamily\fontsize{10.000000}{12.000000}\selectfont \(\displaystyle {0.2}\)}%
\end{pgfscope}%
\begin{pgfscope}%
\pgfsetrectcap%
\pgfsetroundjoin%
\pgfsetlinewidth{0.803000pt}%
\definecolor{currentstroke}{rgb}{0.000000,0.000000,0.000000}%
\pgfsetstrokecolor{currentstroke}%
\pgfsetdash{}{0pt}%
\pgfpathmoveto{\pgfqpoint{3.558144in}{1.577751in}}%
\pgfpathlineto{\pgfqpoint{3.628038in}{3.104037in}}%
\pgfusepath{stroke}%
\end{pgfscope}%
\begin{pgfscope}%
\definecolor{textcolor}{rgb}{0.000000,0.000000,0.000000}%
\pgfsetstrokecolor{textcolor}%
\pgfsetfillcolor{textcolor}%
\pgftext[x=4.167903in, y=1.963517in, left, base,rotate=87.378092]{\color{textcolor}\rmfamily\fontsize{10.000000}{12.000000}\selectfont Position Z [\(\displaystyle m\)]}%
\end{pgfscope}%
\begin{pgfscope}%
\pgfsetbuttcap%
\pgfsetroundjoin%
\pgfsetlinewidth{0.803000pt}%
\definecolor{currentstroke}{rgb}{0.690196,0.690196,0.690196}%
\pgfsetstrokecolor{currentstroke}%
\pgfsetdash{}{0pt}%
\pgfpathmoveto{\pgfqpoint{3.562413in}{1.670968in}}%
\pgfpathlineto{\pgfqpoint{1.598575in}{2.237310in}}%
\pgfpathlineto{\pgfqpoint{0.374477in}{1.219382in}}%
\pgfusepath{stroke}%
\end{pgfscope}%
\begin{pgfscope}%
\pgfsetbuttcap%
\pgfsetroundjoin%
\pgfsetlinewidth{0.803000pt}%
\definecolor{currentstroke}{rgb}{0.690196,0.690196,0.690196}%
\pgfsetstrokecolor{currentstroke}%
\pgfsetdash{}{0pt}%
\pgfpathmoveto{\pgfqpoint{3.572604in}{1.893520in}}%
\pgfpathlineto{\pgfqpoint{1.596097in}{2.452785in}}%
\pgfpathlineto{\pgfqpoint{0.363500in}{1.447331in}}%
\pgfusepath{stroke}%
\end{pgfscope}%
\begin{pgfscope}%
\pgfsetbuttcap%
\pgfsetroundjoin%
\pgfsetlinewidth{0.803000pt}%
\definecolor{currentstroke}{rgb}{0.690196,0.690196,0.690196}%
\pgfsetstrokecolor{currentstroke}%
\pgfsetdash{}{0pt}%
\pgfpathmoveto{\pgfqpoint{3.582929in}{2.118991in}}%
\pgfpathlineto{\pgfqpoint{1.593589in}{2.670950in}}%
\pgfpathlineto{\pgfqpoint{0.352373in}{1.678383in}}%
\pgfusepath{stroke}%
\end{pgfscope}%
\begin{pgfscope}%
\pgfsetbuttcap%
\pgfsetroundjoin%
\pgfsetlinewidth{0.803000pt}%
\definecolor{currentstroke}{rgb}{0.690196,0.690196,0.690196}%
\pgfsetstrokecolor{currentstroke}%
\pgfsetdash{}{0pt}%
\pgfpathmoveto{\pgfqpoint{3.593391in}{2.347438in}}%
\pgfpathlineto{\pgfqpoint{1.591048in}{2.891856in}}%
\pgfpathlineto{\pgfqpoint{0.341094in}{1.912603in}}%
\pgfusepath{stroke}%
\end{pgfscope}%
\begin{pgfscope}%
\pgfsetbuttcap%
\pgfsetroundjoin%
\pgfsetlinewidth{0.803000pt}%
\definecolor{currentstroke}{rgb}{0.690196,0.690196,0.690196}%
\pgfsetstrokecolor{currentstroke}%
\pgfsetdash{}{0pt}%
\pgfpathmoveto{\pgfqpoint{3.603991in}{2.578921in}}%
\pgfpathlineto{\pgfqpoint{1.588476in}{3.115555in}}%
\pgfpathlineto{\pgfqpoint{0.329659in}{2.150055in}}%
\pgfusepath{stroke}%
\end{pgfscope}%
\begin{pgfscope}%
\pgfsetbuttcap%
\pgfsetroundjoin%
\pgfsetlinewidth{0.803000pt}%
\definecolor{currentstroke}{rgb}{0.690196,0.690196,0.690196}%
\pgfsetstrokecolor{currentstroke}%
\pgfsetdash{}{0pt}%
\pgfpathmoveto{\pgfqpoint{3.614733in}{2.813500in}}%
\pgfpathlineto{\pgfqpoint{1.585871in}{3.342101in}}%
\pgfpathlineto{\pgfqpoint{0.318065in}{2.390809in}}%
\pgfusepath{stroke}%
\end{pgfscope}%
\begin{pgfscope}%
\pgfsetbuttcap%
\pgfsetroundjoin%
\pgfsetlinewidth{0.803000pt}%
\definecolor{currentstroke}{rgb}{0.690196,0.690196,0.690196}%
\pgfsetstrokecolor{currentstroke}%
\pgfsetdash{}{0pt}%
\pgfpathmoveto{\pgfqpoint{3.625620in}{3.051238in}}%
\pgfpathlineto{\pgfqpoint{1.583232in}{3.571547in}}%
\pgfpathlineto{\pgfqpoint{0.306309in}{2.634931in}}%
\pgfusepath{stroke}%
\end{pgfscope}%
\begin{pgfscope}%
\pgfsetrectcap%
\pgfsetroundjoin%
\pgfsetlinewidth{0.803000pt}%
\definecolor{currentstroke}{rgb}{0.000000,0.000000,0.000000}%
\pgfsetstrokecolor{currentstroke}%
\pgfsetdash{}{0pt}%
\pgfpathmoveto{\pgfqpoint{3.545929in}{1.675722in}}%
\pgfpathlineto{\pgfqpoint{3.595421in}{1.661449in}}%
\pgfusepath{stroke}%
\end{pgfscope}%
\begin{pgfscope}%
\definecolor{textcolor}{rgb}{0.000000,0.000000,0.000000}%
\pgfsetstrokecolor{textcolor}%
\pgfsetfillcolor{textcolor}%
\pgftext[x=3.816545in,y=1.706967in,,top]{\color{textcolor}\rmfamily\fontsize{10.000000}{12.000000}\selectfont \(\displaystyle {0.0}\)}%
\end{pgfscope}%
\begin{pgfscope}%
\pgfsetrectcap%
\pgfsetroundjoin%
\pgfsetlinewidth{0.803000pt}%
\definecolor{currentstroke}{rgb}{0.000000,0.000000,0.000000}%
\pgfsetstrokecolor{currentstroke}%
\pgfsetdash{}{0pt}%
\pgfpathmoveto{\pgfqpoint{3.556009in}{1.898216in}}%
\pgfpathlineto{\pgfqpoint{3.605835in}{1.884117in}}%
\pgfusepath{stroke}%
\end{pgfscope}%
\begin{pgfscope}%
\definecolor{textcolor}{rgb}{0.000000,0.000000,0.000000}%
\pgfsetstrokecolor{textcolor}%
\pgfsetfillcolor{textcolor}%
\pgftext[x=3.828356in,y=1.929079in,,top]{\color{textcolor}\rmfamily\fontsize{10.000000}{12.000000}\selectfont \(\displaystyle {0.1}\)}%
\end{pgfscope}%
\begin{pgfscope}%
\pgfsetrectcap%
\pgfsetroundjoin%
\pgfsetlinewidth{0.803000pt}%
\definecolor{currentstroke}{rgb}{0.000000,0.000000,0.000000}%
\pgfsetstrokecolor{currentstroke}%
\pgfsetdash{}{0pt}%
\pgfpathmoveto{\pgfqpoint{3.566221in}{2.123627in}}%
\pgfpathlineto{\pgfqpoint{3.616387in}{2.109708in}}%
\pgfusepath{stroke}%
\end{pgfscope}%
\begin{pgfscope}%
\definecolor{textcolor}{rgb}{0.000000,0.000000,0.000000}%
\pgfsetstrokecolor{textcolor}%
\pgfsetfillcolor{textcolor}%
\pgftext[x=3.840321in,y=2.154096in,,top]{\color{textcolor}\rmfamily\fontsize{10.000000}{12.000000}\selectfont \(\displaystyle {0.2}\)}%
\end{pgfscope}%
\begin{pgfscope}%
\pgfsetrectcap%
\pgfsetroundjoin%
\pgfsetlinewidth{0.803000pt}%
\definecolor{currentstroke}{rgb}{0.000000,0.000000,0.000000}%
\pgfsetstrokecolor{currentstroke}%
\pgfsetdash{}{0pt}%
\pgfpathmoveto{\pgfqpoint{3.576568in}{2.352012in}}%
\pgfpathlineto{\pgfqpoint{3.627077in}{2.338279in}}%
\pgfusepath{stroke}%
\end{pgfscope}%
\begin{pgfscope}%
\definecolor{textcolor}{rgb}{0.000000,0.000000,0.000000}%
\pgfsetstrokecolor{textcolor}%
\pgfsetfillcolor{textcolor}%
\pgftext[x=3.852444in,y=2.382074in,,top]{\color{textcolor}\rmfamily\fontsize{10.000000}{12.000000}\selectfont \(\displaystyle {0.3}\)}%
\end{pgfscope}%
\begin{pgfscope}%
\pgfsetrectcap%
\pgfsetroundjoin%
\pgfsetlinewidth{0.803000pt}%
\definecolor{currentstroke}{rgb}{0.000000,0.000000,0.000000}%
\pgfsetstrokecolor{currentstroke}%
\pgfsetdash{}{0pt}%
\pgfpathmoveto{\pgfqpoint{3.587052in}{2.583431in}}%
\pgfpathlineto{\pgfqpoint{3.637910in}{2.569890in}}%
\pgfusepath{stroke}%
\end{pgfscope}%
\begin{pgfscope}%
\definecolor{textcolor}{rgb}{0.000000,0.000000,0.000000}%
\pgfsetstrokecolor{textcolor}%
\pgfsetfillcolor{textcolor}%
\pgftext[x=3.864728in,y=2.613071in,,top]{\color{textcolor}\rmfamily\fontsize{10.000000}{12.000000}\selectfont \(\displaystyle {0.4}\)}%
\end{pgfscope}%
\begin{pgfscope}%
\pgfsetrectcap%
\pgfsetroundjoin%
\pgfsetlinewidth{0.803000pt}%
\definecolor{currentstroke}{rgb}{0.000000,0.000000,0.000000}%
\pgfsetstrokecolor{currentstroke}%
\pgfsetdash{}{0pt}%
\pgfpathmoveto{\pgfqpoint{3.597676in}{2.817944in}}%
\pgfpathlineto{\pgfqpoint{3.648888in}{2.804601in}}%
\pgfusepath{stroke}%
\end{pgfscope}%
\begin{pgfscope}%
\definecolor{textcolor}{rgb}{0.000000,0.000000,0.000000}%
\pgfsetstrokecolor{textcolor}%
\pgfsetfillcolor{textcolor}%
\pgftext[x=3.877175in,y=2.847150in,,top]{\color{textcolor}\rmfamily\fontsize{10.000000}{12.000000}\selectfont \(\displaystyle {0.5}\)}%
\end{pgfscope}%
\begin{pgfscope}%
\pgfsetrectcap%
\pgfsetroundjoin%
\pgfsetlinewidth{0.803000pt}%
\definecolor{currentstroke}{rgb}{0.000000,0.000000,0.000000}%
\pgfsetstrokecolor{currentstroke}%
\pgfsetdash{}{0pt}%
\pgfpathmoveto{\pgfqpoint{3.608444in}{3.055614in}}%
\pgfpathlineto{\pgfqpoint{3.660014in}{3.042476in}}%
\pgfusepath{stroke}%
\end{pgfscope}%
\begin{pgfscope}%
\definecolor{textcolor}{rgb}{0.000000,0.000000,0.000000}%
\pgfsetstrokecolor{textcolor}%
\pgfsetfillcolor{textcolor}%
\pgftext[x=3.889789in,y=3.084371in,,top]{\color{textcolor}\rmfamily\fontsize{10.000000}{12.000000}\selectfont \(\displaystyle {0.6}\)}%
\end{pgfscope}%
\begin{pgfscope}%
\pgfpathrectangle{\pgfqpoint{0.100000in}{0.212622in}}{\pgfqpoint{3.696000in}{3.696000in}}%
\pgfusepath{clip}%
\pgfsetrectcap%
\pgfsetroundjoin%
\pgfsetlinewidth{1.505625pt}%
\definecolor{currentstroke}{rgb}{0.121569,0.466667,0.705882}%
\pgfsetstrokecolor{currentstroke}%
\pgfsetdash{}{0pt}%
\pgfpathmoveto{\pgfqpoint{1.171996in}{1.722294in}}%
\pgfpathlineto{\pgfqpoint{2.702167in}{1.256065in}}%
\pgfusepath{stroke}%
\end{pgfscope}%
\begin{pgfscope}%
\pgfpathrectangle{\pgfqpoint{0.100000in}{0.212622in}}{\pgfqpoint{3.696000in}{3.696000in}}%
\pgfusepath{clip}%
\pgfsetrectcap%
\pgfsetroundjoin%
\pgfsetlinewidth{1.505625pt}%
\definecolor{currentstroke}{rgb}{1.000000,0.000000,0.000000}%
\pgfsetstrokecolor{currentstroke}%
\pgfsetdash{}{0pt}%
\pgfpathmoveto{\pgfqpoint{1.211629in}{1.792567in}}%
\pgfpathlineto{\pgfqpoint{1.171996in}{1.722294in}}%
\pgfusepath{stroke}%
\end{pgfscope}%
\begin{pgfscope}%
\pgfpathrectangle{\pgfqpoint{0.100000in}{0.212622in}}{\pgfqpoint{3.696000in}{3.696000in}}%
\pgfusepath{clip}%
\pgfsetrectcap%
\pgfsetroundjoin%
\pgfsetlinewidth{1.505625pt}%
\definecolor{currentstroke}{rgb}{1.000000,0.000000,0.000000}%
\pgfsetstrokecolor{currentstroke}%
\pgfsetdash{}{0pt}%
\pgfpathmoveto{\pgfqpoint{2.508628in}{1.588200in}}%
\pgfpathlineto{\pgfqpoint{2.702167in}{1.256065in}}%
\pgfusepath{stroke}%
\end{pgfscope}%
\begin{pgfscope}%
\pgfpathrectangle{\pgfqpoint{0.100000in}{0.212622in}}{\pgfqpoint{3.696000in}{3.696000in}}%
\pgfusepath{clip}%
\pgfsetrectcap%
\pgfsetroundjoin%
\pgfsetlinewidth{1.505625pt}%
\definecolor{currentstroke}{rgb}{1.000000,0.000000,0.000000}%
\pgfsetstrokecolor{currentstroke}%
\pgfsetdash{}{0pt}%
\pgfpathmoveto{\pgfqpoint{3.419944in}{2.977213in}}%
\pgfpathlineto{\pgfqpoint{2.702167in}{1.256065in}}%
\pgfusepath{stroke}%
\end{pgfscope}%
\begin{pgfscope}%
\pgfpathrectangle{\pgfqpoint{0.100000in}{0.212622in}}{\pgfqpoint{3.696000in}{3.696000in}}%
\pgfusepath{clip}%
\pgfsetbuttcap%
\pgfsetroundjoin%
\definecolor{currentfill}{rgb}{1.000000,0.498039,0.054902}%
\pgfsetfillcolor{currentfill}%
\pgfsetfillopacity{0.300000}%
\pgfsetlinewidth{1.003750pt}%
\definecolor{currentstroke}{rgb}{1.000000,0.498039,0.054902}%
\pgfsetstrokecolor{currentstroke}%
\pgfsetstrokeopacity{0.300000}%
\pgfsetdash{}{0pt}%
\pgfpathmoveto{\pgfqpoint{1.211629in}{1.761510in}}%
\pgfpathcurveto{\pgfqpoint{1.219865in}{1.761510in}}{\pgfqpoint{1.227765in}{1.764782in}}{\pgfqpoint{1.233589in}{1.770606in}}%
\pgfpathcurveto{\pgfqpoint{1.239413in}{1.776430in}}{\pgfqpoint{1.242685in}{1.784330in}}{\pgfqpoint{1.242685in}{1.792567in}}%
\pgfpathcurveto{\pgfqpoint{1.242685in}{1.800803in}}{\pgfqpoint{1.239413in}{1.808703in}}{\pgfqpoint{1.233589in}{1.814527in}}%
\pgfpathcurveto{\pgfqpoint{1.227765in}{1.820351in}}{\pgfqpoint{1.219865in}{1.823623in}}{\pgfqpoint{1.211629in}{1.823623in}}%
\pgfpathcurveto{\pgfqpoint{1.203393in}{1.823623in}}{\pgfqpoint{1.195493in}{1.820351in}}{\pgfqpoint{1.189669in}{1.814527in}}%
\pgfpathcurveto{\pgfqpoint{1.183845in}{1.808703in}}{\pgfqpoint{1.180572in}{1.800803in}}{\pgfqpoint{1.180572in}{1.792567in}}%
\pgfpathcurveto{\pgfqpoint{1.180572in}{1.784330in}}{\pgfqpoint{1.183845in}{1.776430in}}{\pgfqpoint{1.189669in}{1.770606in}}%
\pgfpathcurveto{\pgfqpoint{1.195493in}{1.764782in}}{\pgfqpoint{1.203393in}{1.761510in}}{\pgfqpoint{1.211629in}{1.761510in}}%
\pgfpathclose%
\pgfusepath{stroke,fill}%
\end{pgfscope}%
\begin{pgfscope}%
\pgfpathrectangle{\pgfqpoint{0.100000in}{0.212622in}}{\pgfqpoint{3.696000in}{3.696000in}}%
\pgfusepath{clip}%
\pgfsetbuttcap%
\pgfsetroundjoin%
\definecolor{currentfill}{rgb}{1.000000,0.498039,0.054902}%
\pgfsetfillcolor{currentfill}%
\pgfsetfillopacity{0.747363}%
\pgfsetlinewidth{1.003750pt}%
\definecolor{currentstroke}{rgb}{1.000000,0.498039,0.054902}%
\pgfsetstrokecolor{currentstroke}%
\pgfsetstrokeopacity{0.747363}%
\pgfsetdash{}{0pt}%
\pgfpathmoveto{\pgfqpoint{3.419944in}{2.946157in}}%
\pgfpathcurveto{\pgfqpoint{3.428180in}{2.946157in}}{\pgfqpoint{3.436081in}{2.949429in}}{\pgfqpoint{3.441904in}{2.955253in}}%
\pgfpathcurveto{\pgfqpoint{3.447728in}{2.961077in}}{\pgfqpoint{3.451001in}{2.968977in}}{\pgfqpoint{3.451001in}{2.977213in}}%
\pgfpathcurveto{\pgfqpoint{3.451001in}{2.985449in}}{\pgfqpoint{3.447728in}{2.993349in}}{\pgfqpoint{3.441904in}{2.999173in}}%
\pgfpathcurveto{\pgfqpoint{3.436081in}{3.004997in}}{\pgfqpoint{3.428180in}{3.008270in}}{\pgfqpoint{3.419944in}{3.008270in}}%
\pgfpathcurveto{\pgfqpoint{3.411708in}{3.008270in}}{\pgfqpoint{3.403808in}{3.004997in}}{\pgfqpoint{3.397984in}{2.999173in}}%
\pgfpathcurveto{\pgfqpoint{3.392160in}{2.993349in}}{\pgfqpoint{3.388888in}{2.985449in}}{\pgfqpoint{3.388888in}{2.977213in}}%
\pgfpathcurveto{\pgfqpoint{3.388888in}{2.968977in}}{\pgfqpoint{3.392160in}{2.961077in}}{\pgfqpoint{3.397984in}{2.955253in}}%
\pgfpathcurveto{\pgfqpoint{3.403808in}{2.949429in}}{\pgfqpoint{3.411708in}{2.946157in}}{\pgfqpoint{3.419944in}{2.946157in}}%
\pgfpathclose%
\pgfusepath{stroke,fill}%
\end{pgfscope}%
\begin{pgfscope}%
\pgfpathrectangle{\pgfqpoint{0.100000in}{0.212622in}}{\pgfqpoint{3.696000in}{3.696000in}}%
\pgfusepath{clip}%
\pgfsetbuttcap%
\pgfsetroundjoin%
\definecolor{currentfill}{rgb}{1.000000,0.498039,0.054902}%
\pgfsetfillcolor{currentfill}%
\pgfsetlinewidth{1.003750pt}%
\definecolor{currentstroke}{rgb}{1.000000,0.498039,0.054902}%
\pgfsetstrokecolor{currentstroke}%
\pgfsetdash{}{0pt}%
\pgfpathmoveto{\pgfqpoint{2.508628in}{1.557144in}}%
\pgfpathcurveto{\pgfqpoint{2.516865in}{1.557144in}}{\pgfqpoint{2.524765in}{1.560416in}}{\pgfqpoint{2.530589in}{1.566240in}}%
\pgfpathcurveto{\pgfqpoint{2.536413in}{1.572064in}}{\pgfqpoint{2.539685in}{1.579964in}}{\pgfqpoint{2.539685in}{1.588200in}}%
\pgfpathcurveto{\pgfqpoint{2.539685in}{1.596436in}}{\pgfqpoint{2.536413in}{1.604336in}}{\pgfqpoint{2.530589in}{1.610160in}}%
\pgfpathcurveto{\pgfqpoint{2.524765in}{1.615984in}}{\pgfqpoint{2.516865in}{1.619257in}}{\pgfqpoint{2.508628in}{1.619257in}}%
\pgfpathcurveto{\pgfqpoint{2.500392in}{1.619257in}}{\pgfqpoint{2.492492in}{1.615984in}}{\pgfqpoint{2.486668in}{1.610160in}}%
\pgfpathcurveto{\pgfqpoint{2.480844in}{1.604336in}}{\pgfqpoint{2.477572in}{1.596436in}}{\pgfqpoint{2.477572in}{1.588200in}}%
\pgfpathcurveto{\pgfqpoint{2.477572in}{1.579964in}}{\pgfqpoint{2.480844in}{1.572064in}}{\pgfqpoint{2.486668in}{1.566240in}}%
\pgfpathcurveto{\pgfqpoint{2.492492in}{1.560416in}}{\pgfqpoint{2.500392in}{1.557144in}}{\pgfqpoint{2.508628in}{1.557144in}}%
\pgfpathclose%
\pgfusepath{stroke,fill}%
\end{pgfscope}%
\begin{pgfscope}%
\definecolor{textcolor}{rgb}{0.000000,0.000000,0.000000}%
\pgfsetstrokecolor{textcolor}%
\pgfsetfillcolor{textcolor}%
\pgftext[x=1.948000in,y=3.991956in,,base]{\color{textcolor}\rmfamily\fontsize{12.000000}{14.400000}\selectfont ROLEQ}%
\end{pgfscope}%
\begin{pgfscope}%
\pgfpathrectangle{\pgfqpoint{0.100000in}{0.212622in}}{\pgfqpoint{3.696000in}{3.696000in}}%
\pgfusepath{clip}%
\pgfsetbuttcap%
\pgfsetroundjoin%
\definecolor{currentfill}{rgb}{0.121569,0.466667,0.705882}%
\pgfsetfillcolor{currentfill}%
\pgfsetfillopacity{0.300000}%
\pgfsetlinewidth{1.003750pt}%
\definecolor{currentstroke}{rgb}{0.121569,0.466667,0.705882}%
\pgfsetstrokecolor{currentstroke}%
\pgfsetstrokeopacity{0.300000}%
\pgfsetdash{}{0pt}%
\pgfpathmoveto{\pgfqpoint{1.211629in}{1.761510in}}%
\pgfpathcurveto{\pgfqpoint{1.219865in}{1.761510in}}{\pgfqpoint{1.227765in}{1.764782in}}{\pgfqpoint{1.233589in}{1.770606in}}%
\pgfpathcurveto{\pgfqpoint{1.239413in}{1.776430in}}{\pgfqpoint{1.242685in}{1.784330in}}{\pgfqpoint{1.242685in}{1.792567in}}%
\pgfpathcurveto{\pgfqpoint{1.242685in}{1.800803in}}{\pgfqpoint{1.239413in}{1.808703in}}{\pgfqpoint{1.233589in}{1.814527in}}%
\pgfpathcurveto{\pgfqpoint{1.227765in}{1.820351in}}{\pgfqpoint{1.219865in}{1.823623in}}{\pgfqpoint{1.211629in}{1.823623in}}%
\pgfpathcurveto{\pgfqpoint{1.203393in}{1.823623in}}{\pgfqpoint{1.195493in}{1.820351in}}{\pgfqpoint{1.189669in}{1.814527in}}%
\pgfpathcurveto{\pgfqpoint{1.183845in}{1.808703in}}{\pgfqpoint{1.180572in}{1.800803in}}{\pgfqpoint{1.180572in}{1.792567in}}%
\pgfpathcurveto{\pgfqpoint{1.180572in}{1.784330in}}{\pgfqpoint{1.183845in}{1.776430in}}{\pgfqpoint{1.189669in}{1.770606in}}%
\pgfpathcurveto{\pgfqpoint{1.195493in}{1.764782in}}{\pgfqpoint{1.203393in}{1.761510in}}{\pgfqpoint{1.211629in}{1.761510in}}%
\pgfpathclose%
\pgfusepath{stroke,fill}%
\end{pgfscope}%
\begin{pgfscope}%
\pgfpathrectangle{\pgfqpoint{0.100000in}{0.212622in}}{\pgfqpoint{3.696000in}{3.696000in}}%
\pgfusepath{clip}%
\pgfsetbuttcap%
\pgfsetroundjoin%
\definecolor{currentfill}{rgb}{0.121569,0.466667,0.705882}%
\pgfsetfillcolor{currentfill}%
\pgfsetfillopacity{0.321969}%
\pgfsetlinewidth{1.003750pt}%
\definecolor{currentstroke}{rgb}{0.121569,0.466667,0.705882}%
\pgfsetstrokecolor{currentstroke}%
\pgfsetstrokeopacity{0.321969}%
\pgfsetdash{}{0pt}%
\pgfpathmoveto{\pgfqpoint{1.194179in}{1.768119in}}%
\pgfpathcurveto{\pgfqpoint{1.202416in}{1.768119in}}{\pgfqpoint{1.210316in}{1.771391in}}{\pgfqpoint{1.216140in}{1.777215in}}%
\pgfpathcurveto{\pgfqpoint{1.221964in}{1.783039in}}{\pgfqpoint{1.225236in}{1.790939in}}{\pgfqpoint{1.225236in}{1.799175in}}%
\pgfpathcurveto{\pgfqpoint{1.225236in}{1.807411in}}{\pgfqpoint{1.221964in}{1.815312in}}{\pgfqpoint{1.216140in}{1.821135in}}%
\pgfpathcurveto{\pgfqpoint{1.210316in}{1.826959in}}{\pgfqpoint{1.202416in}{1.830232in}}{\pgfqpoint{1.194179in}{1.830232in}}%
\pgfpathcurveto{\pgfqpoint{1.185943in}{1.830232in}}{\pgfqpoint{1.178043in}{1.826959in}}{\pgfqpoint{1.172219in}{1.821135in}}%
\pgfpathcurveto{\pgfqpoint{1.166395in}{1.815312in}}{\pgfqpoint{1.163123in}{1.807411in}}{\pgfqpoint{1.163123in}{1.799175in}}%
\pgfpathcurveto{\pgfqpoint{1.163123in}{1.790939in}}{\pgfqpoint{1.166395in}{1.783039in}}{\pgfqpoint{1.172219in}{1.777215in}}%
\pgfpathcurveto{\pgfqpoint{1.178043in}{1.771391in}}{\pgfqpoint{1.185943in}{1.768119in}}{\pgfqpoint{1.194179in}{1.768119in}}%
\pgfpathclose%
\pgfusepath{stroke,fill}%
\end{pgfscope}%
\begin{pgfscope}%
\pgfpathrectangle{\pgfqpoint{0.100000in}{0.212622in}}{\pgfqpoint{3.696000in}{3.696000in}}%
\pgfusepath{clip}%
\pgfsetbuttcap%
\pgfsetroundjoin%
\definecolor{currentfill}{rgb}{0.121569,0.466667,0.705882}%
\pgfsetfillcolor{currentfill}%
\pgfsetfillopacity{0.346407}%
\pgfsetlinewidth{1.003750pt}%
\definecolor{currentstroke}{rgb}{0.121569,0.466667,0.705882}%
\pgfsetstrokecolor{currentstroke}%
\pgfsetstrokeopacity{0.346407}%
\pgfsetdash{}{0pt}%
\pgfpathmoveto{\pgfqpoint{1.155614in}{1.763983in}}%
\pgfpathcurveto{\pgfqpoint{1.163850in}{1.763983in}}{\pgfqpoint{1.171750in}{1.767256in}}{\pgfqpoint{1.177574in}{1.773079in}}%
\pgfpathcurveto{\pgfqpoint{1.183398in}{1.778903in}}{\pgfqpoint{1.186670in}{1.786803in}}{\pgfqpoint{1.186670in}{1.795040in}}%
\pgfpathcurveto{\pgfqpoint{1.186670in}{1.803276in}}{\pgfqpoint{1.183398in}{1.811176in}}{\pgfqpoint{1.177574in}{1.817000in}}%
\pgfpathcurveto{\pgfqpoint{1.171750in}{1.822824in}}{\pgfqpoint{1.163850in}{1.826096in}}{\pgfqpoint{1.155614in}{1.826096in}}%
\pgfpathcurveto{\pgfqpoint{1.147377in}{1.826096in}}{\pgfqpoint{1.139477in}{1.822824in}}{\pgfqpoint{1.133653in}{1.817000in}}%
\pgfpathcurveto{\pgfqpoint{1.127829in}{1.811176in}}{\pgfqpoint{1.124557in}{1.803276in}}{\pgfqpoint{1.124557in}{1.795040in}}%
\pgfpathcurveto{\pgfqpoint{1.124557in}{1.786803in}}{\pgfqpoint{1.127829in}{1.778903in}}{\pgfqpoint{1.133653in}{1.773079in}}%
\pgfpathcurveto{\pgfqpoint{1.139477in}{1.767256in}}{\pgfqpoint{1.147377in}{1.763983in}}{\pgfqpoint{1.155614in}{1.763983in}}%
\pgfpathclose%
\pgfusepath{stroke,fill}%
\end{pgfscope}%
\begin{pgfscope}%
\pgfpathrectangle{\pgfqpoint{0.100000in}{0.212622in}}{\pgfqpoint{3.696000in}{3.696000in}}%
\pgfusepath{clip}%
\pgfsetbuttcap%
\pgfsetroundjoin%
\definecolor{currentfill}{rgb}{0.121569,0.466667,0.705882}%
\pgfsetfillcolor{currentfill}%
\pgfsetfillopacity{0.399296}%
\pgfsetlinewidth{1.003750pt}%
\definecolor{currentstroke}{rgb}{0.121569,0.466667,0.705882}%
\pgfsetstrokecolor{currentstroke}%
\pgfsetstrokeopacity{0.399296}%
\pgfsetdash{}{0pt}%
\pgfpathmoveto{\pgfqpoint{1.081984in}{1.757683in}}%
\pgfpathcurveto{\pgfqpoint{1.090220in}{1.757683in}}{\pgfqpoint{1.098120in}{1.760955in}}{\pgfqpoint{1.103944in}{1.766779in}}%
\pgfpathcurveto{\pgfqpoint{1.109768in}{1.772603in}}{\pgfqpoint{1.113040in}{1.780503in}}{\pgfqpoint{1.113040in}{1.788740in}}%
\pgfpathcurveto{\pgfqpoint{1.113040in}{1.796976in}}{\pgfqpoint{1.109768in}{1.804876in}}{\pgfqpoint{1.103944in}{1.810700in}}%
\pgfpathcurveto{\pgfqpoint{1.098120in}{1.816524in}}{\pgfqpoint{1.090220in}{1.819796in}}{\pgfqpoint{1.081984in}{1.819796in}}%
\pgfpathcurveto{\pgfqpoint{1.073747in}{1.819796in}}{\pgfqpoint{1.065847in}{1.816524in}}{\pgfqpoint{1.060023in}{1.810700in}}%
\pgfpathcurveto{\pgfqpoint{1.054200in}{1.804876in}}{\pgfqpoint{1.050927in}{1.796976in}}{\pgfqpoint{1.050927in}{1.788740in}}%
\pgfpathcurveto{\pgfqpoint{1.050927in}{1.780503in}}{\pgfqpoint{1.054200in}{1.772603in}}{\pgfqpoint{1.060023in}{1.766779in}}%
\pgfpathcurveto{\pgfqpoint{1.065847in}{1.760955in}}{\pgfqpoint{1.073747in}{1.757683in}}{\pgfqpoint{1.081984in}{1.757683in}}%
\pgfpathclose%
\pgfusepath{stroke,fill}%
\end{pgfscope}%
\begin{pgfscope}%
\pgfpathrectangle{\pgfqpoint{0.100000in}{0.212622in}}{\pgfqpoint{3.696000in}{3.696000in}}%
\pgfusepath{clip}%
\pgfsetbuttcap%
\pgfsetroundjoin%
\definecolor{currentfill}{rgb}{0.121569,0.466667,0.705882}%
\pgfsetfillcolor{currentfill}%
\pgfsetfillopacity{0.469359}%
\pgfsetlinewidth{1.003750pt}%
\definecolor{currentstroke}{rgb}{0.121569,0.466667,0.705882}%
\pgfsetstrokecolor{currentstroke}%
\pgfsetstrokeopacity{0.469359}%
\pgfsetdash{}{0pt}%
\pgfpathmoveto{\pgfqpoint{0.982802in}{1.735061in}}%
\pgfpathcurveto{\pgfqpoint{0.991039in}{1.735061in}}{\pgfqpoint{0.998939in}{1.738333in}}{\pgfqpoint{1.004763in}{1.744157in}}%
\pgfpathcurveto{\pgfqpoint{1.010587in}{1.749981in}}{\pgfqpoint{1.013859in}{1.757881in}}{\pgfqpoint{1.013859in}{1.766117in}}%
\pgfpathcurveto{\pgfqpoint{1.013859in}{1.774353in}}{\pgfqpoint{1.010587in}{1.782253in}}{\pgfqpoint{1.004763in}{1.788077in}}%
\pgfpathcurveto{\pgfqpoint{0.998939in}{1.793901in}}{\pgfqpoint{0.991039in}{1.797174in}}{\pgfqpoint{0.982802in}{1.797174in}}%
\pgfpathcurveto{\pgfqpoint{0.974566in}{1.797174in}}{\pgfqpoint{0.966666in}{1.793901in}}{\pgfqpoint{0.960842in}{1.788077in}}%
\pgfpathcurveto{\pgfqpoint{0.955018in}{1.782253in}}{\pgfqpoint{0.951746in}{1.774353in}}{\pgfqpoint{0.951746in}{1.766117in}}%
\pgfpathcurveto{\pgfqpoint{0.951746in}{1.757881in}}{\pgfqpoint{0.955018in}{1.749981in}}{\pgfqpoint{0.960842in}{1.744157in}}%
\pgfpathcurveto{\pgfqpoint{0.966666in}{1.738333in}}{\pgfqpoint{0.974566in}{1.735061in}}{\pgfqpoint{0.982802in}{1.735061in}}%
\pgfpathclose%
\pgfusepath{stroke,fill}%
\end{pgfscope}%
\begin{pgfscope}%
\pgfpathrectangle{\pgfqpoint{0.100000in}{0.212622in}}{\pgfqpoint{3.696000in}{3.696000in}}%
\pgfusepath{clip}%
\pgfsetbuttcap%
\pgfsetroundjoin%
\definecolor{currentfill}{rgb}{0.121569,0.466667,0.705882}%
\pgfsetfillcolor{currentfill}%
\pgfsetfillopacity{0.514215}%
\pgfsetlinewidth{1.003750pt}%
\definecolor{currentstroke}{rgb}{0.121569,0.466667,0.705882}%
\pgfsetstrokecolor{currentstroke}%
\pgfsetstrokeopacity{0.514215}%
\pgfsetdash{}{0pt}%
\pgfpathmoveto{\pgfqpoint{0.923548in}{1.719779in}}%
\pgfpathcurveto{\pgfqpoint{0.931785in}{1.719779in}}{\pgfqpoint{0.939685in}{1.723052in}}{\pgfqpoint{0.945509in}{1.728875in}}%
\pgfpathcurveto{\pgfqpoint{0.951333in}{1.734699in}}{\pgfqpoint{0.954605in}{1.742599in}}{\pgfqpoint{0.954605in}{1.750836in}}%
\pgfpathcurveto{\pgfqpoint{0.954605in}{1.759072in}}{\pgfqpoint{0.951333in}{1.766972in}}{\pgfqpoint{0.945509in}{1.772796in}}%
\pgfpathcurveto{\pgfqpoint{0.939685in}{1.778620in}}{\pgfqpoint{0.931785in}{1.781892in}}{\pgfqpoint{0.923548in}{1.781892in}}%
\pgfpathcurveto{\pgfqpoint{0.915312in}{1.781892in}}{\pgfqpoint{0.907412in}{1.778620in}}{\pgfqpoint{0.901588in}{1.772796in}}%
\pgfpathcurveto{\pgfqpoint{0.895764in}{1.766972in}}{\pgfqpoint{0.892492in}{1.759072in}}{\pgfqpoint{0.892492in}{1.750836in}}%
\pgfpathcurveto{\pgfqpoint{0.892492in}{1.742599in}}{\pgfqpoint{0.895764in}{1.734699in}}{\pgfqpoint{0.901588in}{1.728875in}}%
\pgfpathcurveto{\pgfqpoint{0.907412in}{1.723052in}}{\pgfqpoint{0.915312in}{1.719779in}}{\pgfqpoint{0.923548in}{1.719779in}}%
\pgfpathclose%
\pgfusepath{stroke,fill}%
\end{pgfscope}%
\begin{pgfscope}%
\pgfpathrectangle{\pgfqpoint{0.100000in}{0.212622in}}{\pgfqpoint{3.696000in}{3.696000in}}%
\pgfusepath{clip}%
\pgfsetbuttcap%
\pgfsetroundjoin%
\definecolor{currentfill}{rgb}{0.121569,0.466667,0.705882}%
\pgfsetfillcolor{currentfill}%
\pgfsetfillopacity{0.541264}%
\pgfsetlinewidth{1.003750pt}%
\definecolor{currentstroke}{rgb}{0.121569,0.466667,0.705882}%
\pgfsetstrokecolor{currentstroke}%
\pgfsetstrokeopacity{0.541264}%
\pgfsetdash{}{0pt}%
\pgfpathmoveto{\pgfqpoint{0.890784in}{1.711300in}}%
\pgfpathcurveto{\pgfqpoint{0.899020in}{1.711300in}}{\pgfqpoint{0.906920in}{1.714572in}}{\pgfqpoint{0.912744in}{1.720396in}}%
\pgfpathcurveto{\pgfqpoint{0.918568in}{1.726220in}}{\pgfqpoint{0.921841in}{1.734120in}}{\pgfqpoint{0.921841in}{1.742356in}}%
\pgfpathcurveto{\pgfqpoint{0.921841in}{1.750593in}}{\pgfqpoint{0.918568in}{1.758493in}}{\pgfqpoint{0.912744in}{1.764317in}}%
\pgfpathcurveto{\pgfqpoint{0.906920in}{1.770141in}}{\pgfqpoint{0.899020in}{1.773413in}}{\pgfqpoint{0.890784in}{1.773413in}}%
\pgfpathcurveto{\pgfqpoint{0.882548in}{1.773413in}}{\pgfqpoint{0.874648in}{1.770141in}}{\pgfqpoint{0.868824in}{1.764317in}}%
\pgfpathcurveto{\pgfqpoint{0.863000in}{1.758493in}}{\pgfqpoint{0.859728in}{1.750593in}}{\pgfqpoint{0.859728in}{1.742356in}}%
\pgfpathcurveto{\pgfqpoint{0.859728in}{1.734120in}}{\pgfqpoint{0.863000in}{1.726220in}}{\pgfqpoint{0.868824in}{1.720396in}}%
\pgfpathcurveto{\pgfqpoint{0.874648in}{1.714572in}}{\pgfqpoint{0.882548in}{1.711300in}}{\pgfqpoint{0.890784in}{1.711300in}}%
\pgfpathclose%
\pgfusepath{stroke,fill}%
\end{pgfscope}%
\begin{pgfscope}%
\pgfpathrectangle{\pgfqpoint{0.100000in}{0.212622in}}{\pgfqpoint{3.696000in}{3.696000in}}%
\pgfusepath{clip}%
\pgfsetbuttcap%
\pgfsetroundjoin%
\definecolor{currentfill}{rgb}{0.121569,0.466667,0.705882}%
\pgfsetfillcolor{currentfill}%
\pgfsetfillopacity{0.577640}%
\pgfsetlinewidth{1.003750pt}%
\definecolor{currentstroke}{rgb}{0.121569,0.466667,0.705882}%
\pgfsetstrokecolor{currentstroke}%
\pgfsetstrokeopacity{0.577640}%
\pgfsetdash{}{0pt}%
\pgfpathmoveto{\pgfqpoint{0.848736in}{1.696321in}}%
\pgfpathcurveto{\pgfqpoint{0.856973in}{1.696321in}}{\pgfqpoint{0.864873in}{1.699593in}}{\pgfqpoint{0.870697in}{1.705417in}}%
\pgfpathcurveto{\pgfqpoint{0.876521in}{1.711241in}}{\pgfqpoint{0.879793in}{1.719141in}}{\pgfqpoint{0.879793in}{1.727377in}}%
\pgfpathcurveto{\pgfqpoint{0.879793in}{1.735613in}}{\pgfqpoint{0.876521in}{1.743513in}}{\pgfqpoint{0.870697in}{1.749337in}}%
\pgfpathcurveto{\pgfqpoint{0.864873in}{1.755161in}}{\pgfqpoint{0.856973in}{1.758434in}}{\pgfqpoint{0.848736in}{1.758434in}}%
\pgfpathcurveto{\pgfqpoint{0.840500in}{1.758434in}}{\pgfqpoint{0.832600in}{1.755161in}}{\pgfqpoint{0.826776in}{1.749337in}}%
\pgfpathcurveto{\pgfqpoint{0.820952in}{1.743513in}}{\pgfqpoint{0.817680in}{1.735613in}}{\pgfqpoint{0.817680in}{1.727377in}}%
\pgfpathcurveto{\pgfqpoint{0.817680in}{1.719141in}}{\pgfqpoint{0.820952in}{1.711241in}}{\pgfqpoint{0.826776in}{1.705417in}}%
\pgfpathcurveto{\pgfqpoint{0.832600in}{1.699593in}}{\pgfqpoint{0.840500in}{1.696321in}}{\pgfqpoint{0.848736in}{1.696321in}}%
\pgfpathclose%
\pgfusepath{stroke,fill}%
\end{pgfscope}%
\begin{pgfscope}%
\pgfpathrectangle{\pgfqpoint{0.100000in}{0.212622in}}{\pgfqpoint{3.696000in}{3.696000in}}%
\pgfusepath{clip}%
\pgfsetbuttcap%
\pgfsetroundjoin%
\definecolor{currentfill}{rgb}{0.121569,0.466667,0.705882}%
\pgfsetfillcolor{currentfill}%
\pgfsetfillopacity{0.642855}%
\pgfsetlinewidth{1.003750pt}%
\definecolor{currentstroke}{rgb}{0.121569,0.466667,0.705882}%
\pgfsetstrokecolor{currentstroke}%
\pgfsetstrokeopacity{0.642855}%
\pgfsetdash{}{0pt}%
\pgfpathmoveto{\pgfqpoint{0.783327in}{1.676825in}}%
\pgfpathcurveto{\pgfqpoint{0.791563in}{1.676825in}}{\pgfqpoint{0.799463in}{1.680097in}}{\pgfqpoint{0.805287in}{1.685921in}}%
\pgfpathcurveto{\pgfqpoint{0.811111in}{1.691745in}}{\pgfqpoint{0.814383in}{1.699645in}}{\pgfqpoint{0.814383in}{1.707881in}}%
\pgfpathcurveto{\pgfqpoint{0.814383in}{1.716117in}}{\pgfqpoint{0.811111in}{1.724017in}}{\pgfqpoint{0.805287in}{1.729841in}}%
\pgfpathcurveto{\pgfqpoint{0.799463in}{1.735665in}}{\pgfqpoint{0.791563in}{1.738938in}}{\pgfqpoint{0.783327in}{1.738938in}}%
\pgfpathcurveto{\pgfqpoint{0.775091in}{1.738938in}}{\pgfqpoint{0.767191in}{1.735665in}}{\pgfqpoint{0.761367in}{1.729841in}}%
\pgfpathcurveto{\pgfqpoint{0.755543in}{1.724017in}}{\pgfqpoint{0.752270in}{1.716117in}}{\pgfqpoint{0.752270in}{1.707881in}}%
\pgfpathcurveto{\pgfqpoint{0.752270in}{1.699645in}}{\pgfqpoint{0.755543in}{1.691745in}}{\pgfqpoint{0.761367in}{1.685921in}}%
\pgfpathcurveto{\pgfqpoint{0.767191in}{1.680097in}}{\pgfqpoint{0.775091in}{1.676825in}}{\pgfqpoint{0.783327in}{1.676825in}}%
\pgfpathclose%
\pgfusepath{stroke,fill}%
\end{pgfscope}%
\begin{pgfscope}%
\pgfpathrectangle{\pgfqpoint{0.100000in}{0.212622in}}{\pgfqpoint{3.696000in}{3.696000in}}%
\pgfusepath{clip}%
\pgfsetbuttcap%
\pgfsetroundjoin%
\definecolor{currentfill}{rgb}{0.121569,0.466667,0.705882}%
\pgfsetfillcolor{currentfill}%
\pgfsetfillopacity{0.717406}%
\pgfsetlinewidth{1.003750pt}%
\definecolor{currentstroke}{rgb}{0.121569,0.466667,0.705882}%
\pgfsetstrokecolor{currentstroke}%
\pgfsetstrokeopacity{0.717406}%
\pgfsetdash{}{0pt}%
\pgfpathmoveto{\pgfqpoint{0.725323in}{1.672255in}}%
\pgfpathcurveto{\pgfqpoint{0.733559in}{1.672255in}}{\pgfqpoint{0.741459in}{1.675527in}}{\pgfqpoint{0.747283in}{1.681351in}}%
\pgfpathcurveto{\pgfqpoint{0.753107in}{1.687175in}}{\pgfqpoint{0.756379in}{1.695075in}}{\pgfqpoint{0.756379in}{1.703311in}}%
\pgfpathcurveto{\pgfqpoint{0.756379in}{1.711548in}}{\pgfqpoint{0.753107in}{1.719448in}}{\pgfqpoint{0.747283in}{1.725272in}}%
\pgfpathcurveto{\pgfqpoint{0.741459in}{1.731095in}}{\pgfqpoint{0.733559in}{1.734368in}}{\pgfqpoint{0.725323in}{1.734368in}}%
\pgfpathcurveto{\pgfqpoint{0.717087in}{1.734368in}}{\pgfqpoint{0.709187in}{1.731095in}}{\pgfqpoint{0.703363in}{1.725272in}}%
\pgfpathcurveto{\pgfqpoint{0.697539in}{1.719448in}}{\pgfqpoint{0.694266in}{1.711548in}}{\pgfqpoint{0.694266in}{1.703311in}}%
\pgfpathcurveto{\pgfqpoint{0.694266in}{1.695075in}}{\pgfqpoint{0.697539in}{1.687175in}}{\pgfqpoint{0.703363in}{1.681351in}}%
\pgfpathcurveto{\pgfqpoint{0.709187in}{1.675527in}}{\pgfqpoint{0.717087in}{1.672255in}}{\pgfqpoint{0.725323in}{1.672255in}}%
\pgfpathclose%
\pgfusepath{stroke,fill}%
\end{pgfscope}%
\begin{pgfscope}%
\pgfpathrectangle{\pgfqpoint{0.100000in}{0.212622in}}{\pgfqpoint{3.696000in}{3.696000in}}%
\pgfusepath{clip}%
\pgfsetbuttcap%
\pgfsetroundjoin%
\definecolor{currentfill}{rgb}{0.121569,0.466667,0.705882}%
\pgfsetfillcolor{currentfill}%
\pgfsetfillopacity{0.747362}%
\pgfsetlinewidth{1.003750pt}%
\definecolor{currentstroke}{rgb}{0.121569,0.466667,0.705882}%
\pgfsetstrokecolor{currentstroke}%
\pgfsetstrokeopacity{0.747362}%
\pgfsetdash{}{0pt}%
\pgfpathmoveto{\pgfqpoint{3.419944in}{2.946157in}}%
\pgfpathcurveto{\pgfqpoint{3.428180in}{2.946157in}}{\pgfqpoint{3.436081in}{2.949429in}}{\pgfqpoint{3.441904in}{2.955253in}}%
\pgfpathcurveto{\pgfqpoint{3.447728in}{2.961077in}}{\pgfqpoint{3.451001in}{2.968977in}}{\pgfqpoint{3.451001in}{2.977213in}}%
\pgfpathcurveto{\pgfqpoint{3.451001in}{2.985449in}}{\pgfqpoint{3.447728in}{2.993349in}}{\pgfqpoint{3.441904in}{2.999173in}}%
\pgfpathcurveto{\pgfqpoint{3.436081in}{3.004997in}}{\pgfqpoint{3.428180in}{3.008270in}}{\pgfqpoint{3.419944in}{3.008270in}}%
\pgfpathcurveto{\pgfqpoint{3.411708in}{3.008270in}}{\pgfqpoint{3.403808in}{3.004997in}}{\pgfqpoint{3.397984in}{2.999173in}}%
\pgfpathcurveto{\pgfqpoint{3.392160in}{2.993349in}}{\pgfqpoint{3.388888in}{2.985449in}}{\pgfqpoint{3.388888in}{2.977213in}}%
\pgfpathcurveto{\pgfqpoint{3.388888in}{2.968977in}}{\pgfqpoint{3.392160in}{2.961077in}}{\pgfqpoint{3.397984in}{2.955253in}}%
\pgfpathcurveto{\pgfqpoint{3.403808in}{2.949429in}}{\pgfqpoint{3.411708in}{2.946157in}}{\pgfqpoint{3.419944in}{2.946157in}}%
\pgfpathclose%
\pgfusepath{stroke,fill}%
\end{pgfscope}%
\begin{pgfscope}%
\pgfpathrectangle{\pgfqpoint{0.100000in}{0.212622in}}{\pgfqpoint{3.696000in}{3.696000in}}%
\pgfusepath{clip}%
\pgfsetbuttcap%
\pgfsetroundjoin%
\definecolor{currentfill}{rgb}{0.121569,0.466667,0.705882}%
\pgfsetfillcolor{currentfill}%
\pgfsetfillopacity{0.768958}%
\pgfsetlinewidth{1.003750pt}%
\definecolor{currentstroke}{rgb}{0.121569,0.466667,0.705882}%
\pgfsetstrokecolor{currentstroke}%
\pgfsetstrokeopacity{0.768958}%
\pgfsetdash{}{0pt}%
\pgfpathmoveto{\pgfqpoint{3.321396in}{2.765525in}}%
\pgfpathcurveto{\pgfqpoint{3.329633in}{2.765525in}}{\pgfqpoint{3.337533in}{2.768798in}}{\pgfqpoint{3.343357in}{2.774621in}}%
\pgfpathcurveto{\pgfqpoint{3.349181in}{2.780445in}}{\pgfqpoint{3.352453in}{2.788345in}}{\pgfqpoint{3.352453in}{2.796582in}}%
\pgfpathcurveto{\pgfqpoint{3.352453in}{2.804818in}}{\pgfqpoint{3.349181in}{2.812718in}}{\pgfqpoint{3.343357in}{2.818542in}}%
\pgfpathcurveto{\pgfqpoint{3.337533in}{2.824366in}}{\pgfqpoint{3.329633in}{2.827638in}}{\pgfqpoint{3.321396in}{2.827638in}}%
\pgfpathcurveto{\pgfqpoint{3.313160in}{2.827638in}}{\pgfqpoint{3.305260in}{2.824366in}}{\pgfqpoint{3.299436in}{2.818542in}}%
\pgfpathcurveto{\pgfqpoint{3.293612in}{2.812718in}}{\pgfqpoint{3.290340in}{2.804818in}}{\pgfqpoint{3.290340in}{2.796582in}}%
\pgfpathcurveto{\pgfqpoint{3.290340in}{2.788345in}}{\pgfqpoint{3.293612in}{2.780445in}}{\pgfqpoint{3.299436in}{2.774621in}}%
\pgfpathcurveto{\pgfqpoint{3.305260in}{2.768798in}}{\pgfqpoint{3.313160in}{2.765525in}}{\pgfqpoint{3.321396in}{2.765525in}}%
\pgfpathclose%
\pgfusepath{stroke,fill}%
\end{pgfscope}%
\begin{pgfscope}%
\pgfpathrectangle{\pgfqpoint{0.100000in}{0.212622in}}{\pgfqpoint{3.696000in}{3.696000in}}%
\pgfusepath{clip}%
\pgfsetbuttcap%
\pgfsetroundjoin%
\definecolor{currentfill}{rgb}{0.121569,0.466667,0.705882}%
\pgfsetfillcolor{currentfill}%
\pgfsetfillopacity{0.783577}%
\pgfsetlinewidth{1.003750pt}%
\definecolor{currentstroke}{rgb}{0.121569,0.466667,0.705882}%
\pgfsetstrokecolor{currentstroke}%
\pgfsetstrokeopacity{0.783577}%
\pgfsetdash{}{0pt}%
\pgfpathmoveto{\pgfqpoint{0.682212in}{1.661689in}}%
\pgfpathcurveto{\pgfqpoint{0.690449in}{1.661689in}}{\pgfqpoint{0.698349in}{1.664961in}}{\pgfqpoint{0.704173in}{1.670785in}}%
\pgfpathcurveto{\pgfqpoint{0.709997in}{1.676609in}}{\pgfqpoint{0.713269in}{1.684509in}}{\pgfqpoint{0.713269in}{1.692745in}}%
\pgfpathcurveto{\pgfqpoint{0.713269in}{1.700981in}}{\pgfqpoint{0.709997in}{1.708881in}}{\pgfqpoint{0.704173in}{1.714705in}}%
\pgfpathcurveto{\pgfqpoint{0.698349in}{1.720529in}}{\pgfqpoint{0.690449in}{1.723802in}}{\pgfqpoint{0.682212in}{1.723802in}}%
\pgfpathcurveto{\pgfqpoint{0.673976in}{1.723802in}}{\pgfqpoint{0.666076in}{1.720529in}}{\pgfqpoint{0.660252in}{1.714705in}}%
\pgfpathcurveto{\pgfqpoint{0.654428in}{1.708881in}}{\pgfqpoint{0.651156in}{1.700981in}}{\pgfqpoint{0.651156in}{1.692745in}}%
\pgfpathcurveto{\pgfqpoint{0.651156in}{1.684509in}}{\pgfqpoint{0.654428in}{1.676609in}}{\pgfqpoint{0.660252in}{1.670785in}}%
\pgfpathcurveto{\pgfqpoint{0.666076in}{1.664961in}}{\pgfqpoint{0.673976in}{1.661689in}}{\pgfqpoint{0.682212in}{1.661689in}}%
\pgfpathclose%
\pgfusepath{stroke,fill}%
\end{pgfscope}%
\begin{pgfscope}%
\pgfpathrectangle{\pgfqpoint{0.100000in}{0.212622in}}{\pgfqpoint{3.696000in}{3.696000in}}%
\pgfusepath{clip}%
\pgfsetbuttcap%
\pgfsetroundjoin%
\definecolor{currentfill}{rgb}{0.121569,0.466667,0.705882}%
\pgfsetfillcolor{currentfill}%
\pgfsetfillopacity{0.808673}%
\pgfsetlinewidth{1.003750pt}%
\definecolor{currentstroke}{rgb}{0.121569,0.466667,0.705882}%
\pgfsetstrokecolor{currentstroke}%
\pgfsetstrokeopacity{0.808673}%
\pgfsetdash{}{0pt}%
\pgfpathmoveto{\pgfqpoint{0.675304in}{1.665132in}}%
\pgfpathcurveto{\pgfqpoint{0.683540in}{1.665132in}}{\pgfqpoint{0.691440in}{1.668404in}}{\pgfqpoint{0.697264in}{1.674228in}}%
\pgfpathcurveto{\pgfqpoint{0.703088in}{1.680052in}}{\pgfqpoint{0.706360in}{1.687952in}}{\pgfqpoint{0.706360in}{1.696188in}}%
\pgfpathcurveto{\pgfqpoint{0.706360in}{1.704425in}}{\pgfqpoint{0.703088in}{1.712325in}}{\pgfqpoint{0.697264in}{1.718149in}}%
\pgfpathcurveto{\pgfqpoint{0.691440in}{1.723973in}}{\pgfqpoint{0.683540in}{1.727245in}}{\pgfqpoint{0.675304in}{1.727245in}}%
\pgfpathcurveto{\pgfqpoint{0.667068in}{1.727245in}}{\pgfqpoint{0.659167in}{1.723973in}}{\pgfqpoint{0.653344in}{1.718149in}}%
\pgfpathcurveto{\pgfqpoint{0.647520in}{1.712325in}}{\pgfqpoint{0.644247in}{1.704425in}}{\pgfqpoint{0.644247in}{1.696188in}}%
\pgfpathcurveto{\pgfqpoint{0.644247in}{1.687952in}}{\pgfqpoint{0.647520in}{1.680052in}}{\pgfqpoint{0.653344in}{1.674228in}}%
\pgfpathcurveto{\pgfqpoint{0.659167in}{1.668404in}}{\pgfqpoint{0.667068in}{1.665132in}}{\pgfqpoint{0.675304in}{1.665132in}}%
\pgfpathclose%
\pgfusepath{stroke,fill}%
\end{pgfscope}%
\begin{pgfscope}%
\pgfpathrectangle{\pgfqpoint{0.100000in}{0.212622in}}{\pgfqpoint{3.696000in}{3.696000in}}%
\pgfusepath{clip}%
\pgfsetbuttcap%
\pgfsetroundjoin%
\definecolor{currentfill}{rgb}{0.121569,0.466667,0.705882}%
\pgfsetfillcolor{currentfill}%
\pgfsetfillopacity{0.812626}%
\pgfsetlinewidth{1.003750pt}%
\definecolor{currentstroke}{rgb}{0.121569,0.466667,0.705882}%
\pgfsetstrokecolor{currentstroke}%
\pgfsetstrokeopacity{0.812626}%
\pgfsetdash{}{0pt}%
\pgfpathmoveto{\pgfqpoint{3.132867in}{2.411999in}}%
\pgfpathcurveto{\pgfqpoint{3.141104in}{2.411999in}}{\pgfqpoint{3.149004in}{2.415271in}}{\pgfqpoint{3.154828in}{2.421095in}}%
\pgfpathcurveto{\pgfqpoint{3.160651in}{2.426919in}}{\pgfqpoint{3.163924in}{2.434819in}}{\pgfqpoint{3.163924in}{2.443056in}}%
\pgfpathcurveto{\pgfqpoint{3.163924in}{2.451292in}}{\pgfqpoint{3.160651in}{2.459192in}}{\pgfqpoint{3.154828in}{2.465016in}}%
\pgfpathcurveto{\pgfqpoint{3.149004in}{2.470840in}}{\pgfqpoint{3.141104in}{2.474112in}}{\pgfqpoint{3.132867in}{2.474112in}}%
\pgfpathcurveto{\pgfqpoint{3.124631in}{2.474112in}}{\pgfqpoint{3.116731in}{2.470840in}}{\pgfqpoint{3.110907in}{2.465016in}}%
\pgfpathcurveto{\pgfqpoint{3.105083in}{2.459192in}}{\pgfqpoint{3.101811in}{2.451292in}}{\pgfqpoint{3.101811in}{2.443056in}}%
\pgfpathcurveto{\pgfqpoint{3.101811in}{2.434819in}}{\pgfqpoint{3.105083in}{2.426919in}}{\pgfqpoint{3.110907in}{2.421095in}}%
\pgfpathcurveto{\pgfqpoint{3.116731in}{2.415271in}}{\pgfqpoint{3.124631in}{2.411999in}}{\pgfqpoint{3.132867in}{2.411999in}}%
\pgfpathclose%
\pgfusepath{stroke,fill}%
\end{pgfscope}%
\begin{pgfscope}%
\pgfpathrectangle{\pgfqpoint{0.100000in}{0.212622in}}{\pgfqpoint{3.696000in}{3.696000in}}%
\pgfusepath{clip}%
\pgfsetbuttcap%
\pgfsetroundjoin%
\definecolor{currentfill}{rgb}{0.121569,0.466667,0.705882}%
\pgfsetfillcolor{currentfill}%
\pgfsetfillopacity{0.820150}%
\pgfsetlinewidth{1.003750pt}%
\definecolor{currentstroke}{rgb}{0.121569,0.466667,0.705882}%
\pgfsetstrokecolor{currentstroke}%
\pgfsetstrokeopacity{0.820150}%
\pgfsetdash{}{0pt}%
\pgfpathmoveto{\pgfqpoint{0.671713in}{1.658596in}}%
\pgfpathcurveto{\pgfqpoint{0.679949in}{1.658596in}}{\pgfqpoint{0.687849in}{1.661868in}}{\pgfqpoint{0.693673in}{1.667692in}}%
\pgfpathcurveto{\pgfqpoint{0.699497in}{1.673516in}}{\pgfqpoint{0.702769in}{1.681416in}}{\pgfqpoint{0.702769in}{1.689652in}}%
\pgfpathcurveto{\pgfqpoint{0.702769in}{1.697888in}}{\pgfqpoint{0.699497in}{1.705788in}}{\pgfqpoint{0.693673in}{1.711612in}}%
\pgfpathcurveto{\pgfqpoint{0.687849in}{1.717436in}}{\pgfqpoint{0.679949in}{1.720709in}}{\pgfqpoint{0.671713in}{1.720709in}}%
\pgfpathcurveto{\pgfqpoint{0.663476in}{1.720709in}}{\pgfqpoint{0.655576in}{1.717436in}}{\pgfqpoint{0.649753in}{1.711612in}}%
\pgfpathcurveto{\pgfqpoint{0.643929in}{1.705788in}}{\pgfqpoint{0.640656in}{1.697888in}}{\pgfqpoint{0.640656in}{1.689652in}}%
\pgfpathcurveto{\pgfqpoint{0.640656in}{1.681416in}}{\pgfqpoint{0.643929in}{1.673516in}}{\pgfqpoint{0.649753in}{1.667692in}}%
\pgfpathcurveto{\pgfqpoint{0.655576in}{1.661868in}}{\pgfqpoint{0.663476in}{1.658596in}}{\pgfqpoint{0.671713in}{1.658596in}}%
\pgfpathclose%
\pgfusepath{stroke,fill}%
\end{pgfscope}%
\begin{pgfscope}%
\pgfpathrectangle{\pgfqpoint{0.100000in}{0.212622in}}{\pgfqpoint{3.696000in}{3.696000in}}%
\pgfusepath{clip}%
\pgfsetbuttcap%
\pgfsetroundjoin%
\definecolor{currentfill}{rgb}{0.121569,0.466667,0.705882}%
\pgfsetfillcolor{currentfill}%
\pgfsetfillopacity{0.830103}%
\pgfsetlinewidth{1.003750pt}%
\definecolor{currentstroke}{rgb}{0.121569,0.466667,0.705882}%
\pgfsetstrokecolor{currentstroke}%
\pgfsetstrokeopacity{0.830103}%
\pgfsetdash{}{0pt}%
\pgfpathmoveto{\pgfqpoint{0.673716in}{1.652176in}}%
\pgfpathcurveto{\pgfqpoint{0.681952in}{1.652176in}}{\pgfqpoint{0.689852in}{1.655448in}}{\pgfqpoint{0.695676in}{1.661272in}}%
\pgfpathcurveto{\pgfqpoint{0.701500in}{1.667096in}}{\pgfqpoint{0.704773in}{1.674996in}}{\pgfqpoint{0.704773in}{1.683232in}}%
\pgfpathcurveto{\pgfqpoint{0.704773in}{1.691469in}}{\pgfqpoint{0.701500in}{1.699369in}}{\pgfqpoint{0.695676in}{1.705193in}}%
\pgfpathcurveto{\pgfqpoint{0.689852in}{1.711017in}}{\pgfqpoint{0.681952in}{1.714289in}}{\pgfqpoint{0.673716in}{1.714289in}}%
\pgfpathcurveto{\pgfqpoint{0.665480in}{1.714289in}}{\pgfqpoint{0.657580in}{1.711017in}}{\pgfqpoint{0.651756in}{1.705193in}}%
\pgfpathcurveto{\pgfqpoint{0.645932in}{1.699369in}}{\pgfqpoint{0.642660in}{1.691469in}}{\pgfqpoint{0.642660in}{1.683232in}}%
\pgfpathcurveto{\pgfqpoint{0.642660in}{1.674996in}}{\pgfqpoint{0.645932in}{1.667096in}}{\pgfqpoint{0.651756in}{1.661272in}}%
\pgfpathcurveto{\pgfqpoint{0.657580in}{1.655448in}}{\pgfqpoint{0.665480in}{1.652176in}}{\pgfqpoint{0.673716in}{1.652176in}}%
\pgfpathclose%
\pgfusepath{stroke,fill}%
\end{pgfscope}%
\begin{pgfscope}%
\pgfpathrectangle{\pgfqpoint{0.100000in}{0.212622in}}{\pgfqpoint{3.696000in}{3.696000in}}%
\pgfusepath{clip}%
\pgfsetbuttcap%
\pgfsetroundjoin%
\definecolor{currentfill}{rgb}{0.121569,0.466667,0.705882}%
\pgfsetfillcolor{currentfill}%
\pgfsetfillopacity{0.841525}%
\pgfsetlinewidth{1.003750pt}%
\definecolor{currentstroke}{rgb}{0.121569,0.466667,0.705882}%
\pgfsetstrokecolor{currentstroke}%
\pgfsetstrokeopacity{0.841525}%
\pgfsetdash{}{0pt}%
\pgfpathmoveto{\pgfqpoint{0.682239in}{1.647417in}}%
\pgfpathcurveto{\pgfqpoint{0.690475in}{1.647417in}}{\pgfqpoint{0.698375in}{1.650689in}}{\pgfqpoint{0.704199in}{1.656513in}}%
\pgfpathcurveto{\pgfqpoint{0.710023in}{1.662337in}}{\pgfqpoint{0.713295in}{1.670237in}}{\pgfqpoint{0.713295in}{1.678473in}}%
\pgfpathcurveto{\pgfqpoint{0.713295in}{1.686709in}}{\pgfqpoint{0.710023in}{1.694609in}}{\pgfqpoint{0.704199in}{1.700433in}}%
\pgfpathcurveto{\pgfqpoint{0.698375in}{1.706257in}}{\pgfqpoint{0.690475in}{1.709530in}}{\pgfqpoint{0.682239in}{1.709530in}}%
\pgfpathcurveto{\pgfqpoint{0.674003in}{1.709530in}}{\pgfqpoint{0.666103in}{1.706257in}}{\pgfqpoint{0.660279in}{1.700433in}}%
\pgfpathcurveto{\pgfqpoint{0.654455in}{1.694609in}}{\pgfqpoint{0.651182in}{1.686709in}}{\pgfqpoint{0.651182in}{1.678473in}}%
\pgfpathcurveto{\pgfqpoint{0.651182in}{1.670237in}}{\pgfqpoint{0.654455in}{1.662337in}}{\pgfqpoint{0.660279in}{1.656513in}}%
\pgfpathcurveto{\pgfqpoint{0.666103in}{1.650689in}}{\pgfqpoint{0.674003in}{1.647417in}}{\pgfqpoint{0.682239in}{1.647417in}}%
\pgfpathclose%
\pgfusepath{stroke,fill}%
\end{pgfscope}%
\begin{pgfscope}%
\pgfpathrectangle{\pgfqpoint{0.100000in}{0.212622in}}{\pgfqpoint{3.696000in}{3.696000in}}%
\pgfusepath{clip}%
\pgfsetbuttcap%
\pgfsetroundjoin%
\definecolor{currentfill}{rgb}{0.121569,0.466667,0.705882}%
\pgfsetfillcolor{currentfill}%
\pgfsetfillopacity{0.851040}%
\pgfsetlinewidth{1.003750pt}%
\definecolor{currentstroke}{rgb}{0.121569,0.466667,0.705882}%
\pgfsetstrokecolor{currentstroke}%
\pgfsetstrokeopacity{0.851040}%
\pgfsetdash{}{0pt}%
\pgfpathmoveto{\pgfqpoint{0.705901in}{1.646445in}}%
\pgfpathcurveto{\pgfqpoint{0.714137in}{1.646445in}}{\pgfqpoint{0.722037in}{1.649718in}}{\pgfqpoint{0.727861in}{1.655542in}}%
\pgfpathcurveto{\pgfqpoint{0.733685in}{1.661366in}}{\pgfqpoint{0.736957in}{1.669266in}}{\pgfqpoint{0.736957in}{1.677502in}}%
\pgfpathcurveto{\pgfqpoint{0.736957in}{1.685738in}}{\pgfqpoint{0.733685in}{1.693638in}}{\pgfqpoint{0.727861in}{1.699462in}}%
\pgfpathcurveto{\pgfqpoint{0.722037in}{1.705286in}}{\pgfqpoint{0.714137in}{1.708558in}}{\pgfqpoint{0.705901in}{1.708558in}}%
\pgfpathcurveto{\pgfqpoint{0.697665in}{1.708558in}}{\pgfqpoint{0.689765in}{1.705286in}}{\pgfqpoint{0.683941in}{1.699462in}}%
\pgfpathcurveto{\pgfqpoint{0.678117in}{1.693638in}}{\pgfqpoint{0.674844in}{1.685738in}}{\pgfqpoint{0.674844in}{1.677502in}}%
\pgfpathcurveto{\pgfqpoint{0.674844in}{1.669266in}}{\pgfqpoint{0.678117in}{1.661366in}}{\pgfqpoint{0.683941in}{1.655542in}}%
\pgfpathcurveto{\pgfqpoint{0.689765in}{1.649718in}}{\pgfqpoint{0.697665in}{1.646445in}}{\pgfqpoint{0.705901in}{1.646445in}}%
\pgfpathclose%
\pgfusepath{stroke,fill}%
\end{pgfscope}%
\begin{pgfscope}%
\pgfpathrectangle{\pgfqpoint{0.100000in}{0.212622in}}{\pgfqpoint{3.696000in}{3.696000in}}%
\pgfusepath{clip}%
\pgfsetbuttcap%
\pgfsetroundjoin%
\definecolor{currentfill}{rgb}{0.121569,0.466667,0.705882}%
\pgfsetfillcolor{currentfill}%
\pgfsetfillopacity{0.859906}%
\pgfsetlinewidth{1.003750pt}%
\definecolor{currentstroke}{rgb}{0.121569,0.466667,0.705882}%
\pgfsetstrokecolor{currentstroke}%
\pgfsetstrokeopacity{0.859906}%
\pgfsetdash{}{0pt}%
\pgfpathmoveto{\pgfqpoint{0.735898in}{1.645434in}}%
\pgfpathcurveto{\pgfqpoint{0.744134in}{1.645434in}}{\pgfqpoint{0.752034in}{1.648707in}}{\pgfqpoint{0.757858in}{1.654531in}}%
\pgfpathcurveto{\pgfqpoint{0.763682in}{1.660355in}}{\pgfqpoint{0.766955in}{1.668255in}}{\pgfqpoint{0.766955in}{1.676491in}}%
\pgfpathcurveto{\pgfqpoint{0.766955in}{1.684727in}}{\pgfqpoint{0.763682in}{1.692627in}}{\pgfqpoint{0.757858in}{1.698451in}}%
\pgfpathcurveto{\pgfqpoint{0.752034in}{1.704275in}}{\pgfqpoint{0.744134in}{1.707547in}}{\pgfqpoint{0.735898in}{1.707547in}}%
\pgfpathcurveto{\pgfqpoint{0.727662in}{1.707547in}}{\pgfqpoint{0.719762in}{1.704275in}}{\pgfqpoint{0.713938in}{1.698451in}}%
\pgfpathcurveto{\pgfqpoint{0.708114in}{1.692627in}}{\pgfqpoint{0.704842in}{1.684727in}}{\pgfqpoint{0.704842in}{1.676491in}}%
\pgfpathcurveto{\pgfqpoint{0.704842in}{1.668255in}}{\pgfqpoint{0.708114in}{1.660355in}}{\pgfqpoint{0.713938in}{1.654531in}}%
\pgfpathcurveto{\pgfqpoint{0.719762in}{1.648707in}}{\pgfqpoint{0.727662in}{1.645434in}}{\pgfqpoint{0.735898in}{1.645434in}}%
\pgfpathclose%
\pgfusepath{stroke,fill}%
\end{pgfscope}%
\begin{pgfscope}%
\pgfpathrectangle{\pgfqpoint{0.100000in}{0.212622in}}{\pgfqpoint{3.696000in}{3.696000in}}%
\pgfusepath{clip}%
\pgfsetbuttcap%
\pgfsetroundjoin%
\definecolor{currentfill}{rgb}{0.121569,0.466667,0.705882}%
\pgfsetfillcolor{currentfill}%
\pgfsetfillopacity{0.865166}%
\pgfsetlinewidth{1.003750pt}%
\definecolor{currentstroke}{rgb}{0.121569,0.466667,0.705882}%
\pgfsetstrokecolor{currentstroke}%
\pgfsetstrokeopacity{0.865166}%
\pgfsetdash{}{0pt}%
\pgfpathmoveto{\pgfqpoint{2.948493in}{2.075714in}}%
\pgfpathcurveto{\pgfqpoint{2.956729in}{2.075714in}}{\pgfqpoint{2.964629in}{2.078987in}}{\pgfqpoint{2.970453in}{2.084811in}}%
\pgfpathcurveto{\pgfqpoint{2.976277in}{2.090634in}}{\pgfqpoint{2.979549in}{2.098535in}}{\pgfqpoint{2.979549in}{2.106771in}}%
\pgfpathcurveto{\pgfqpoint{2.979549in}{2.115007in}}{\pgfqpoint{2.976277in}{2.122907in}}{\pgfqpoint{2.970453in}{2.128731in}}%
\pgfpathcurveto{\pgfqpoint{2.964629in}{2.134555in}}{\pgfqpoint{2.956729in}{2.137827in}}{\pgfqpoint{2.948493in}{2.137827in}}%
\pgfpathcurveto{\pgfqpoint{2.940256in}{2.137827in}}{\pgfqpoint{2.932356in}{2.134555in}}{\pgfqpoint{2.926532in}{2.128731in}}%
\pgfpathcurveto{\pgfqpoint{2.920708in}{2.122907in}}{\pgfqpoint{2.917436in}{2.115007in}}{\pgfqpoint{2.917436in}{2.106771in}}%
\pgfpathcurveto{\pgfqpoint{2.917436in}{2.098535in}}{\pgfqpoint{2.920708in}{2.090634in}}{\pgfqpoint{2.926532in}{2.084811in}}%
\pgfpathcurveto{\pgfqpoint{2.932356in}{2.078987in}}{\pgfqpoint{2.940256in}{2.075714in}}{\pgfqpoint{2.948493in}{2.075714in}}%
\pgfpathclose%
\pgfusepath{stroke,fill}%
\end{pgfscope}%
\begin{pgfscope}%
\pgfpathrectangle{\pgfqpoint{0.100000in}{0.212622in}}{\pgfqpoint{3.696000in}{3.696000in}}%
\pgfusepath{clip}%
\pgfsetbuttcap%
\pgfsetroundjoin%
\definecolor{currentfill}{rgb}{0.121569,0.466667,0.705882}%
\pgfsetfillcolor{currentfill}%
\pgfsetfillopacity{0.871119}%
\pgfsetlinewidth{1.003750pt}%
\definecolor{currentstroke}{rgb}{0.121569,0.466667,0.705882}%
\pgfsetstrokecolor{currentstroke}%
\pgfsetstrokeopacity{0.871119}%
\pgfsetdash{}{0pt}%
\pgfpathmoveto{\pgfqpoint{0.758820in}{1.617012in}}%
\pgfpathcurveto{\pgfqpoint{0.767056in}{1.617012in}}{\pgfqpoint{0.774956in}{1.620284in}}{\pgfqpoint{0.780780in}{1.626108in}}%
\pgfpathcurveto{\pgfqpoint{0.786604in}{1.631932in}}{\pgfqpoint{0.789876in}{1.639832in}}{\pgfqpoint{0.789876in}{1.648068in}}%
\pgfpathcurveto{\pgfqpoint{0.789876in}{1.656304in}}{\pgfqpoint{0.786604in}{1.664204in}}{\pgfqpoint{0.780780in}{1.670028in}}%
\pgfpathcurveto{\pgfqpoint{0.774956in}{1.675852in}}{\pgfqpoint{0.767056in}{1.679125in}}{\pgfqpoint{0.758820in}{1.679125in}}%
\pgfpathcurveto{\pgfqpoint{0.750584in}{1.679125in}}{\pgfqpoint{0.742683in}{1.675852in}}{\pgfqpoint{0.736860in}{1.670028in}}%
\pgfpathcurveto{\pgfqpoint{0.731036in}{1.664204in}}{\pgfqpoint{0.727763in}{1.656304in}}{\pgfqpoint{0.727763in}{1.648068in}}%
\pgfpathcurveto{\pgfqpoint{0.727763in}{1.639832in}}{\pgfqpoint{0.731036in}{1.631932in}}{\pgfqpoint{0.736860in}{1.626108in}}%
\pgfpathcurveto{\pgfqpoint{0.742683in}{1.620284in}}{\pgfqpoint{0.750584in}{1.617012in}}{\pgfqpoint{0.758820in}{1.617012in}}%
\pgfpathclose%
\pgfusepath{stroke,fill}%
\end{pgfscope}%
\begin{pgfscope}%
\pgfpathrectangle{\pgfqpoint{0.100000in}{0.212622in}}{\pgfqpoint{3.696000in}{3.696000in}}%
\pgfusepath{clip}%
\pgfsetbuttcap%
\pgfsetroundjoin%
\definecolor{currentfill}{rgb}{0.121569,0.466667,0.705882}%
\pgfsetfillcolor{currentfill}%
\pgfsetfillopacity{0.880278}%
\pgfsetlinewidth{1.003750pt}%
\definecolor{currentstroke}{rgb}{0.121569,0.466667,0.705882}%
\pgfsetstrokecolor{currentstroke}%
\pgfsetstrokeopacity{0.880278}%
\pgfsetdash{}{0pt}%
\pgfpathmoveto{\pgfqpoint{0.795512in}{1.603515in}}%
\pgfpathcurveto{\pgfqpoint{0.803748in}{1.603515in}}{\pgfqpoint{0.811648in}{1.606788in}}{\pgfqpoint{0.817472in}{1.612612in}}%
\pgfpathcurveto{\pgfqpoint{0.823296in}{1.618435in}}{\pgfqpoint{0.826569in}{1.626336in}}{\pgfqpoint{0.826569in}{1.634572in}}%
\pgfpathcurveto{\pgfqpoint{0.826569in}{1.642808in}}{\pgfqpoint{0.823296in}{1.650708in}}{\pgfqpoint{0.817472in}{1.656532in}}%
\pgfpathcurveto{\pgfqpoint{0.811648in}{1.662356in}}{\pgfqpoint{0.803748in}{1.665628in}}{\pgfqpoint{0.795512in}{1.665628in}}%
\pgfpathcurveto{\pgfqpoint{0.787276in}{1.665628in}}{\pgfqpoint{0.779376in}{1.662356in}}{\pgfqpoint{0.773552in}{1.656532in}}%
\pgfpathcurveto{\pgfqpoint{0.767728in}{1.650708in}}{\pgfqpoint{0.764456in}{1.642808in}}{\pgfqpoint{0.764456in}{1.634572in}}%
\pgfpathcurveto{\pgfqpoint{0.764456in}{1.626336in}}{\pgfqpoint{0.767728in}{1.618435in}}{\pgfqpoint{0.773552in}{1.612612in}}%
\pgfpathcurveto{\pgfqpoint{0.779376in}{1.606788in}}{\pgfqpoint{0.787276in}{1.603515in}}{\pgfqpoint{0.795512in}{1.603515in}}%
\pgfpathclose%
\pgfusepath{stroke,fill}%
\end{pgfscope}%
\begin{pgfscope}%
\pgfpathrectangle{\pgfqpoint{0.100000in}{0.212622in}}{\pgfqpoint{3.696000in}{3.696000in}}%
\pgfusepath{clip}%
\pgfsetbuttcap%
\pgfsetroundjoin%
\definecolor{currentfill}{rgb}{0.121569,0.466667,0.705882}%
\pgfsetfillcolor{currentfill}%
\pgfsetfillopacity{0.885122}%
\pgfsetlinewidth{1.003750pt}%
\definecolor{currentstroke}{rgb}{0.121569,0.466667,0.705882}%
\pgfsetstrokecolor{currentstroke}%
\pgfsetstrokeopacity{0.885122}%
\pgfsetdash{}{0pt}%
\pgfpathmoveto{\pgfqpoint{0.814895in}{1.592499in}}%
\pgfpathcurveto{\pgfqpoint{0.823131in}{1.592499in}}{\pgfqpoint{0.831032in}{1.595772in}}{\pgfqpoint{0.836855in}{1.601596in}}%
\pgfpathcurveto{\pgfqpoint{0.842679in}{1.607419in}}{\pgfqpoint{0.845952in}{1.615319in}}{\pgfqpoint{0.845952in}{1.623556in}}%
\pgfpathcurveto{\pgfqpoint{0.845952in}{1.631792in}}{\pgfqpoint{0.842679in}{1.639692in}}{\pgfqpoint{0.836855in}{1.645516in}}%
\pgfpathcurveto{\pgfqpoint{0.831032in}{1.651340in}}{\pgfqpoint{0.823131in}{1.654612in}}{\pgfqpoint{0.814895in}{1.654612in}}%
\pgfpathcurveto{\pgfqpoint{0.806659in}{1.654612in}}{\pgfqpoint{0.798759in}{1.651340in}}{\pgfqpoint{0.792935in}{1.645516in}}%
\pgfpathcurveto{\pgfqpoint{0.787111in}{1.639692in}}{\pgfqpoint{0.783839in}{1.631792in}}{\pgfqpoint{0.783839in}{1.623556in}}%
\pgfpathcurveto{\pgfqpoint{0.783839in}{1.615319in}}{\pgfqpoint{0.787111in}{1.607419in}}{\pgfqpoint{0.792935in}{1.601596in}}%
\pgfpathcurveto{\pgfqpoint{0.798759in}{1.595772in}}{\pgfqpoint{0.806659in}{1.592499in}}{\pgfqpoint{0.814895in}{1.592499in}}%
\pgfpathclose%
\pgfusepath{stroke,fill}%
\end{pgfscope}%
\begin{pgfscope}%
\pgfpathrectangle{\pgfqpoint{0.100000in}{0.212622in}}{\pgfqpoint{3.696000in}{3.696000in}}%
\pgfusepath{clip}%
\pgfsetbuttcap%
\pgfsetroundjoin%
\definecolor{currentfill}{rgb}{0.121569,0.466667,0.705882}%
\pgfsetfillcolor{currentfill}%
\pgfsetfillopacity{0.890217}%
\pgfsetlinewidth{1.003750pt}%
\definecolor{currentstroke}{rgb}{0.121569,0.466667,0.705882}%
\pgfsetstrokecolor{currentstroke}%
\pgfsetstrokeopacity{0.890217}%
\pgfsetdash{}{0pt}%
\pgfpathmoveto{\pgfqpoint{0.838194in}{1.580788in}}%
\pgfpathcurveto{\pgfqpoint{0.846430in}{1.580788in}}{\pgfqpoint{0.854330in}{1.584060in}}{\pgfqpoint{0.860154in}{1.589884in}}%
\pgfpathcurveto{\pgfqpoint{0.865978in}{1.595708in}}{\pgfqpoint{0.869251in}{1.603608in}}{\pgfqpoint{0.869251in}{1.611844in}}%
\pgfpathcurveto{\pgfqpoint{0.869251in}{1.620081in}}{\pgfqpoint{0.865978in}{1.627981in}}{\pgfqpoint{0.860154in}{1.633805in}}%
\pgfpathcurveto{\pgfqpoint{0.854330in}{1.639629in}}{\pgfqpoint{0.846430in}{1.642901in}}{\pgfqpoint{0.838194in}{1.642901in}}%
\pgfpathcurveto{\pgfqpoint{0.829958in}{1.642901in}}{\pgfqpoint{0.822058in}{1.639629in}}{\pgfqpoint{0.816234in}{1.633805in}}%
\pgfpathcurveto{\pgfqpoint{0.810410in}{1.627981in}}{\pgfqpoint{0.807138in}{1.620081in}}{\pgfqpoint{0.807138in}{1.611844in}}%
\pgfpathcurveto{\pgfqpoint{0.807138in}{1.603608in}}{\pgfqpoint{0.810410in}{1.595708in}}{\pgfqpoint{0.816234in}{1.589884in}}%
\pgfpathcurveto{\pgfqpoint{0.822058in}{1.584060in}}{\pgfqpoint{0.829958in}{1.580788in}}{\pgfqpoint{0.838194in}{1.580788in}}%
\pgfpathclose%
\pgfusepath{stroke,fill}%
\end{pgfscope}%
\begin{pgfscope}%
\pgfpathrectangle{\pgfqpoint{0.100000in}{0.212622in}}{\pgfqpoint{3.696000in}{3.696000in}}%
\pgfusepath{clip}%
\pgfsetbuttcap%
\pgfsetroundjoin%
\definecolor{currentfill}{rgb}{0.121569,0.466667,0.705882}%
\pgfsetfillcolor{currentfill}%
\pgfsetfillopacity{0.895200}%
\pgfsetlinewidth{1.003750pt}%
\definecolor{currentstroke}{rgb}{0.121569,0.466667,0.705882}%
\pgfsetstrokecolor{currentstroke}%
\pgfsetstrokeopacity{0.895200}%
\pgfsetdash{}{0pt}%
\pgfpathmoveto{\pgfqpoint{0.870507in}{1.581974in}}%
\pgfpathcurveto{\pgfqpoint{0.878744in}{1.581974in}}{\pgfqpoint{0.886644in}{1.585247in}}{\pgfqpoint{0.892468in}{1.591071in}}%
\pgfpathcurveto{\pgfqpoint{0.898291in}{1.596895in}}{\pgfqpoint{0.901564in}{1.604795in}}{\pgfqpoint{0.901564in}{1.613031in}}%
\pgfpathcurveto{\pgfqpoint{0.901564in}{1.621267in}}{\pgfqpoint{0.898291in}{1.629167in}}{\pgfqpoint{0.892468in}{1.634991in}}%
\pgfpathcurveto{\pgfqpoint{0.886644in}{1.640815in}}{\pgfqpoint{0.878744in}{1.644087in}}{\pgfqpoint{0.870507in}{1.644087in}}%
\pgfpathcurveto{\pgfqpoint{0.862271in}{1.644087in}}{\pgfqpoint{0.854371in}{1.640815in}}{\pgfqpoint{0.848547in}{1.634991in}}%
\pgfpathcurveto{\pgfqpoint{0.842723in}{1.629167in}}{\pgfqpoint{0.839451in}{1.621267in}}{\pgfqpoint{0.839451in}{1.613031in}}%
\pgfpathcurveto{\pgfqpoint{0.839451in}{1.604795in}}{\pgfqpoint{0.842723in}{1.596895in}}{\pgfqpoint{0.848547in}{1.591071in}}%
\pgfpathcurveto{\pgfqpoint{0.854371in}{1.585247in}}{\pgfqpoint{0.862271in}{1.581974in}}{\pgfqpoint{0.870507in}{1.581974in}}%
\pgfpathclose%
\pgfusepath{stroke,fill}%
\end{pgfscope}%
\begin{pgfscope}%
\pgfpathrectangle{\pgfqpoint{0.100000in}{0.212622in}}{\pgfqpoint{3.696000in}{3.696000in}}%
\pgfusepath{clip}%
\pgfsetbuttcap%
\pgfsetroundjoin%
\definecolor{currentfill}{rgb}{0.121569,0.466667,0.705882}%
\pgfsetfillcolor{currentfill}%
\pgfsetfillopacity{0.897321}%
\pgfsetlinewidth{1.003750pt}%
\definecolor{currentstroke}{rgb}{0.121569,0.466667,0.705882}%
\pgfsetstrokecolor{currentstroke}%
\pgfsetstrokeopacity{0.897321}%
\pgfsetdash{}{0pt}%
\pgfpathmoveto{\pgfqpoint{0.914090in}{1.592154in}}%
\pgfpathcurveto{\pgfqpoint{0.922326in}{1.592154in}}{\pgfqpoint{0.930226in}{1.595426in}}{\pgfqpoint{0.936050in}{1.601250in}}%
\pgfpathcurveto{\pgfqpoint{0.941874in}{1.607074in}}{\pgfqpoint{0.945146in}{1.614974in}}{\pgfqpoint{0.945146in}{1.623210in}}%
\pgfpathcurveto{\pgfqpoint{0.945146in}{1.631446in}}{\pgfqpoint{0.941874in}{1.639346in}}{\pgfqpoint{0.936050in}{1.645170in}}%
\pgfpathcurveto{\pgfqpoint{0.930226in}{1.650994in}}{\pgfqpoint{0.922326in}{1.654267in}}{\pgfqpoint{0.914090in}{1.654267in}}%
\pgfpathcurveto{\pgfqpoint{0.905854in}{1.654267in}}{\pgfqpoint{0.897953in}{1.650994in}}{\pgfqpoint{0.892130in}{1.645170in}}%
\pgfpathcurveto{\pgfqpoint{0.886306in}{1.639346in}}{\pgfqpoint{0.883033in}{1.631446in}}{\pgfqpoint{0.883033in}{1.623210in}}%
\pgfpathcurveto{\pgfqpoint{0.883033in}{1.614974in}}{\pgfqpoint{0.886306in}{1.607074in}}{\pgfqpoint{0.892130in}{1.601250in}}%
\pgfpathcurveto{\pgfqpoint{0.897953in}{1.595426in}}{\pgfqpoint{0.905854in}{1.592154in}}{\pgfqpoint{0.914090in}{1.592154in}}%
\pgfpathclose%
\pgfusepath{stroke,fill}%
\end{pgfscope}%
\begin{pgfscope}%
\pgfpathrectangle{\pgfqpoint{0.100000in}{0.212622in}}{\pgfqpoint{3.696000in}{3.696000in}}%
\pgfusepath{clip}%
\pgfsetbuttcap%
\pgfsetroundjoin%
\definecolor{currentfill}{rgb}{0.121569,0.466667,0.705882}%
\pgfsetfillcolor{currentfill}%
\pgfsetfillopacity{0.906532}%
\pgfsetlinewidth{1.003750pt}%
\definecolor{currentstroke}{rgb}{0.121569,0.466667,0.705882}%
\pgfsetstrokecolor{currentstroke}%
\pgfsetstrokeopacity{0.906532}%
\pgfsetdash{}{0pt}%
\pgfpathmoveto{\pgfqpoint{0.954231in}{1.582557in}}%
\pgfpathcurveto{\pgfqpoint{0.962467in}{1.582557in}}{\pgfqpoint{0.970367in}{1.585829in}}{\pgfqpoint{0.976191in}{1.591653in}}%
\pgfpathcurveto{\pgfqpoint{0.982015in}{1.597477in}}{\pgfqpoint{0.985287in}{1.605377in}}{\pgfqpoint{0.985287in}{1.613613in}}%
\pgfpathcurveto{\pgfqpoint{0.985287in}{1.621850in}}{\pgfqpoint{0.982015in}{1.629750in}}{\pgfqpoint{0.976191in}{1.635574in}}%
\pgfpathcurveto{\pgfqpoint{0.970367in}{1.641397in}}{\pgfqpoint{0.962467in}{1.644670in}}{\pgfqpoint{0.954231in}{1.644670in}}%
\pgfpathcurveto{\pgfqpoint{0.945994in}{1.644670in}}{\pgfqpoint{0.938094in}{1.641397in}}{\pgfqpoint{0.932270in}{1.635574in}}%
\pgfpathcurveto{\pgfqpoint{0.926446in}{1.629750in}}{\pgfqpoint{0.923174in}{1.621850in}}{\pgfqpoint{0.923174in}{1.613613in}}%
\pgfpathcurveto{\pgfqpoint{0.923174in}{1.605377in}}{\pgfqpoint{0.926446in}{1.597477in}}{\pgfqpoint{0.932270in}{1.591653in}}%
\pgfpathcurveto{\pgfqpoint{0.938094in}{1.585829in}}{\pgfqpoint{0.945994in}{1.582557in}}{\pgfqpoint{0.954231in}{1.582557in}}%
\pgfpathclose%
\pgfusepath{stroke,fill}%
\end{pgfscope}%
\begin{pgfscope}%
\pgfpathrectangle{\pgfqpoint{0.100000in}{0.212622in}}{\pgfqpoint{3.696000in}{3.696000in}}%
\pgfusepath{clip}%
\pgfsetbuttcap%
\pgfsetroundjoin%
\definecolor{currentfill}{rgb}{0.121569,0.466667,0.705882}%
\pgfsetfillcolor{currentfill}%
\pgfsetfillopacity{0.908450}%
\pgfsetlinewidth{1.003750pt}%
\definecolor{currentstroke}{rgb}{0.121569,0.466667,0.705882}%
\pgfsetstrokecolor{currentstroke}%
\pgfsetstrokeopacity{0.908450}%
\pgfsetdash{}{0pt}%
\pgfpathmoveto{\pgfqpoint{1.168808in}{1.655379in}}%
\pgfpathcurveto{\pgfqpoint{1.177045in}{1.655379in}}{\pgfqpoint{1.184945in}{1.658651in}}{\pgfqpoint{1.190769in}{1.664475in}}%
\pgfpathcurveto{\pgfqpoint{1.196592in}{1.670299in}}{\pgfqpoint{1.199865in}{1.678199in}}{\pgfqpoint{1.199865in}{1.686435in}}%
\pgfpathcurveto{\pgfqpoint{1.199865in}{1.694671in}}{\pgfqpoint{1.196592in}{1.702571in}}{\pgfqpoint{1.190769in}{1.708395in}}%
\pgfpathcurveto{\pgfqpoint{1.184945in}{1.714219in}}{\pgfqpoint{1.177045in}{1.717492in}}{\pgfqpoint{1.168808in}{1.717492in}}%
\pgfpathcurveto{\pgfqpoint{1.160572in}{1.717492in}}{\pgfqpoint{1.152672in}{1.714219in}}{\pgfqpoint{1.146848in}{1.708395in}}%
\pgfpathcurveto{\pgfqpoint{1.141024in}{1.702571in}}{\pgfqpoint{1.137752in}{1.694671in}}{\pgfqpoint{1.137752in}{1.686435in}}%
\pgfpathcurveto{\pgfqpoint{1.137752in}{1.678199in}}{\pgfqpoint{1.141024in}{1.670299in}}{\pgfqpoint{1.146848in}{1.664475in}}%
\pgfpathcurveto{\pgfqpoint{1.152672in}{1.658651in}}{\pgfqpoint{1.160572in}{1.655379in}}{\pgfqpoint{1.168808in}{1.655379in}}%
\pgfpathclose%
\pgfusepath{stroke,fill}%
\end{pgfscope}%
\begin{pgfscope}%
\pgfpathrectangle{\pgfqpoint{0.100000in}{0.212622in}}{\pgfqpoint{3.696000in}{3.696000in}}%
\pgfusepath{clip}%
\pgfsetbuttcap%
\pgfsetroundjoin%
\definecolor{currentfill}{rgb}{0.121569,0.466667,0.705882}%
\pgfsetfillcolor{currentfill}%
\pgfsetfillopacity{0.908879}%
\pgfsetlinewidth{1.003750pt}%
\definecolor{currentstroke}{rgb}{0.121569,0.466667,0.705882}%
\pgfsetstrokecolor{currentstroke}%
\pgfsetstrokeopacity{0.908879}%
\pgfsetdash{}{0pt}%
\pgfpathmoveto{\pgfqpoint{1.092599in}{1.629897in}}%
\pgfpathcurveto{\pgfqpoint{1.100835in}{1.629897in}}{\pgfqpoint{1.108735in}{1.633170in}}{\pgfqpoint{1.114559in}{1.638994in}}%
\pgfpathcurveto{\pgfqpoint{1.120383in}{1.644818in}}{\pgfqpoint{1.123656in}{1.652718in}}{\pgfqpoint{1.123656in}{1.660954in}}%
\pgfpathcurveto{\pgfqpoint{1.123656in}{1.669190in}}{\pgfqpoint{1.120383in}{1.677090in}}{\pgfqpoint{1.114559in}{1.682914in}}%
\pgfpathcurveto{\pgfqpoint{1.108735in}{1.688738in}}{\pgfqpoint{1.100835in}{1.692010in}}{\pgfqpoint{1.092599in}{1.692010in}}%
\pgfpathcurveto{\pgfqpoint{1.084363in}{1.692010in}}{\pgfqpoint{1.076463in}{1.688738in}}{\pgfqpoint{1.070639in}{1.682914in}}%
\pgfpathcurveto{\pgfqpoint{1.064815in}{1.677090in}}{\pgfqpoint{1.061543in}{1.669190in}}{\pgfqpoint{1.061543in}{1.660954in}}%
\pgfpathcurveto{\pgfqpoint{1.061543in}{1.652718in}}{\pgfqpoint{1.064815in}{1.644818in}}{\pgfqpoint{1.070639in}{1.638994in}}%
\pgfpathcurveto{\pgfqpoint{1.076463in}{1.633170in}}{\pgfqpoint{1.084363in}{1.629897in}}{\pgfqpoint{1.092599in}{1.629897in}}%
\pgfpathclose%
\pgfusepath{stroke,fill}%
\end{pgfscope}%
\begin{pgfscope}%
\pgfpathrectangle{\pgfqpoint{0.100000in}{0.212622in}}{\pgfqpoint{3.696000in}{3.696000in}}%
\pgfusepath{clip}%
\pgfsetbuttcap%
\pgfsetroundjoin%
\definecolor{currentfill}{rgb}{0.121569,0.466667,0.705882}%
\pgfsetfillcolor{currentfill}%
\pgfsetfillopacity{0.910452}%
\pgfsetlinewidth{1.003750pt}%
\definecolor{currentstroke}{rgb}{0.121569,0.466667,0.705882}%
\pgfsetstrokecolor{currentstroke}%
\pgfsetstrokeopacity{0.910452}%
\pgfsetdash{}{0pt}%
\pgfpathmoveto{\pgfqpoint{1.204129in}{1.653781in}}%
\pgfpathcurveto{\pgfqpoint{1.212365in}{1.653781in}}{\pgfqpoint{1.220265in}{1.657053in}}{\pgfqpoint{1.226089in}{1.662877in}}%
\pgfpathcurveto{\pgfqpoint{1.231913in}{1.668701in}}{\pgfqpoint{1.235185in}{1.676601in}}{\pgfqpoint{1.235185in}{1.684837in}}%
\pgfpathcurveto{\pgfqpoint{1.235185in}{1.693074in}}{\pgfqpoint{1.231913in}{1.700974in}}{\pgfqpoint{1.226089in}{1.706798in}}%
\pgfpathcurveto{\pgfqpoint{1.220265in}{1.712622in}}{\pgfqpoint{1.212365in}{1.715894in}}{\pgfqpoint{1.204129in}{1.715894in}}%
\pgfpathcurveto{\pgfqpoint{1.195893in}{1.715894in}}{\pgfqpoint{1.187993in}{1.712622in}}{\pgfqpoint{1.182169in}{1.706798in}}%
\pgfpathcurveto{\pgfqpoint{1.176345in}{1.700974in}}{\pgfqpoint{1.173072in}{1.693074in}}{\pgfqpoint{1.173072in}{1.684837in}}%
\pgfpathcurveto{\pgfqpoint{1.173072in}{1.676601in}}{\pgfqpoint{1.176345in}{1.668701in}}{\pgfqpoint{1.182169in}{1.662877in}}%
\pgfpathcurveto{\pgfqpoint{1.187993in}{1.657053in}}{\pgfqpoint{1.195893in}{1.653781in}}{\pgfqpoint{1.204129in}{1.653781in}}%
\pgfpathclose%
\pgfusepath{stroke,fill}%
\end{pgfscope}%
\begin{pgfscope}%
\pgfpathrectangle{\pgfqpoint{0.100000in}{0.212622in}}{\pgfqpoint{3.696000in}{3.696000in}}%
\pgfusepath{clip}%
\pgfsetbuttcap%
\pgfsetroundjoin%
\definecolor{currentfill}{rgb}{0.121569,0.466667,0.705882}%
\pgfsetfillcolor{currentfill}%
\pgfsetfillopacity{0.910548}%
\pgfsetlinewidth{1.003750pt}%
\definecolor{currentstroke}{rgb}{0.121569,0.466667,0.705882}%
\pgfsetstrokecolor{currentstroke}%
\pgfsetstrokeopacity{0.910548}%
\pgfsetdash{}{0pt}%
\pgfpathmoveto{\pgfqpoint{1.010854in}{1.579743in}}%
\pgfpathcurveto{\pgfqpoint{1.019090in}{1.579743in}}{\pgfqpoint{1.026990in}{1.583015in}}{\pgfqpoint{1.032814in}{1.588839in}}%
\pgfpathcurveto{\pgfqpoint{1.038638in}{1.594663in}}{\pgfqpoint{1.041910in}{1.602563in}}{\pgfqpoint{1.041910in}{1.610799in}}%
\pgfpathcurveto{\pgfqpoint{1.041910in}{1.619036in}}{\pgfqpoint{1.038638in}{1.626936in}}{\pgfqpoint{1.032814in}{1.632760in}}%
\pgfpathcurveto{\pgfqpoint{1.026990in}{1.638584in}}{\pgfqpoint{1.019090in}{1.641856in}}{\pgfqpoint{1.010854in}{1.641856in}}%
\pgfpathcurveto{\pgfqpoint{1.002617in}{1.641856in}}{\pgfqpoint{0.994717in}{1.638584in}}{\pgfqpoint{0.988893in}{1.632760in}}%
\pgfpathcurveto{\pgfqpoint{0.983070in}{1.626936in}}{\pgfqpoint{0.979797in}{1.619036in}}{\pgfqpoint{0.979797in}{1.610799in}}%
\pgfpathcurveto{\pgfqpoint{0.979797in}{1.602563in}}{\pgfqpoint{0.983070in}{1.594663in}}{\pgfqpoint{0.988893in}{1.588839in}}%
\pgfpathcurveto{\pgfqpoint{0.994717in}{1.583015in}}{\pgfqpoint{1.002617in}{1.579743in}}{\pgfqpoint{1.010854in}{1.579743in}}%
\pgfpathclose%
\pgfusepath{stroke,fill}%
\end{pgfscope}%
\begin{pgfscope}%
\pgfpathrectangle{\pgfqpoint{0.100000in}{0.212622in}}{\pgfqpoint{3.696000in}{3.696000in}}%
\pgfusepath{clip}%
\pgfsetbuttcap%
\pgfsetroundjoin%
\definecolor{currentfill}{rgb}{0.121569,0.466667,0.705882}%
\pgfsetfillcolor{currentfill}%
\pgfsetfillopacity{0.912374}%
\pgfsetlinewidth{1.003750pt}%
\definecolor{currentstroke}{rgb}{0.121569,0.466667,0.705882}%
\pgfsetstrokecolor{currentstroke}%
\pgfsetstrokeopacity{0.912374}%
\pgfsetdash{}{0pt}%
\pgfpathmoveto{\pgfqpoint{1.415119in}{1.713526in}}%
\pgfpathcurveto{\pgfqpoint{1.423355in}{1.713526in}}{\pgfqpoint{1.431255in}{1.716798in}}{\pgfqpoint{1.437079in}{1.722622in}}%
\pgfpathcurveto{\pgfqpoint{1.442903in}{1.728446in}}{\pgfqpoint{1.446175in}{1.736346in}}{\pgfqpoint{1.446175in}{1.744582in}}%
\pgfpathcurveto{\pgfqpoint{1.446175in}{1.752819in}}{\pgfqpoint{1.442903in}{1.760719in}}{\pgfqpoint{1.437079in}{1.766543in}}%
\pgfpathcurveto{\pgfqpoint{1.431255in}{1.772366in}}{\pgfqpoint{1.423355in}{1.775639in}}{\pgfqpoint{1.415119in}{1.775639in}}%
\pgfpathcurveto{\pgfqpoint{1.406883in}{1.775639in}}{\pgfqpoint{1.398983in}{1.772366in}}{\pgfqpoint{1.393159in}{1.766543in}}%
\pgfpathcurveto{\pgfqpoint{1.387335in}{1.760719in}}{\pgfqpoint{1.384062in}{1.752819in}}{\pgfqpoint{1.384062in}{1.744582in}}%
\pgfpathcurveto{\pgfqpoint{1.384062in}{1.736346in}}{\pgfqpoint{1.387335in}{1.728446in}}{\pgfqpoint{1.393159in}{1.722622in}}%
\pgfpathcurveto{\pgfqpoint{1.398983in}{1.716798in}}{\pgfqpoint{1.406883in}{1.713526in}}{\pgfqpoint{1.415119in}{1.713526in}}%
\pgfpathclose%
\pgfusepath{stroke,fill}%
\end{pgfscope}%
\begin{pgfscope}%
\pgfpathrectangle{\pgfqpoint{0.100000in}{0.212622in}}{\pgfqpoint{3.696000in}{3.696000in}}%
\pgfusepath{clip}%
\pgfsetbuttcap%
\pgfsetroundjoin%
\definecolor{currentfill}{rgb}{0.121569,0.466667,0.705882}%
\pgfsetfillcolor{currentfill}%
\pgfsetfillopacity{0.913338}%
\pgfsetlinewidth{1.003750pt}%
\definecolor{currentstroke}{rgb}{0.121569,0.466667,0.705882}%
\pgfsetstrokecolor{currentstroke}%
\pgfsetstrokeopacity{0.913338}%
\pgfsetdash{}{0pt}%
\pgfpathmoveto{\pgfqpoint{1.338210in}{1.681271in}}%
\pgfpathcurveto{\pgfqpoint{1.346446in}{1.681271in}}{\pgfqpoint{1.354346in}{1.684544in}}{\pgfqpoint{1.360170in}{1.690367in}}%
\pgfpathcurveto{\pgfqpoint{1.365994in}{1.696191in}}{\pgfqpoint{1.369266in}{1.704091in}}{\pgfqpoint{1.369266in}{1.712328in}}%
\pgfpathcurveto{\pgfqpoint{1.369266in}{1.720564in}}{\pgfqpoint{1.365994in}{1.728464in}}{\pgfqpoint{1.360170in}{1.734288in}}%
\pgfpathcurveto{\pgfqpoint{1.354346in}{1.740112in}}{\pgfqpoint{1.346446in}{1.743384in}}{\pgfqpoint{1.338210in}{1.743384in}}%
\pgfpathcurveto{\pgfqpoint{1.329973in}{1.743384in}}{\pgfqpoint{1.322073in}{1.740112in}}{\pgfqpoint{1.316249in}{1.734288in}}%
\pgfpathcurveto{\pgfqpoint{1.310425in}{1.728464in}}{\pgfqpoint{1.307153in}{1.720564in}}{\pgfqpoint{1.307153in}{1.712328in}}%
\pgfpathcurveto{\pgfqpoint{1.307153in}{1.704091in}}{\pgfqpoint{1.310425in}{1.696191in}}{\pgfqpoint{1.316249in}{1.690367in}}%
\pgfpathcurveto{\pgfqpoint{1.322073in}{1.684544in}}{\pgfqpoint{1.329973in}{1.681271in}}{\pgfqpoint{1.338210in}{1.681271in}}%
\pgfpathclose%
\pgfusepath{stroke,fill}%
\end{pgfscope}%
\begin{pgfscope}%
\pgfpathrectangle{\pgfqpoint{0.100000in}{0.212622in}}{\pgfqpoint{3.696000in}{3.696000in}}%
\pgfusepath{clip}%
\pgfsetbuttcap%
\pgfsetroundjoin%
\definecolor{currentfill}{rgb}{0.121569,0.466667,0.705882}%
\pgfsetfillcolor{currentfill}%
\pgfsetfillopacity{0.913635}%
\pgfsetlinewidth{1.003750pt}%
\definecolor{currentstroke}{rgb}{0.121569,0.466667,0.705882}%
\pgfsetstrokecolor{currentstroke}%
\pgfsetstrokeopacity{0.913635}%
\pgfsetdash{}{0pt}%
\pgfpathmoveto{\pgfqpoint{1.240920in}{1.647399in}}%
\pgfpathcurveto{\pgfqpoint{1.249156in}{1.647399in}}{\pgfqpoint{1.257057in}{1.650671in}}{\pgfqpoint{1.262880in}{1.656495in}}%
\pgfpathcurveto{\pgfqpoint{1.268704in}{1.662319in}}{\pgfqpoint{1.271977in}{1.670219in}}{\pgfqpoint{1.271977in}{1.678455in}}%
\pgfpathcurveto{\pgfqpoint{1.271977in}{1.686692in}}{\pgfqpoint{1.268704in}{1.694592in}}{\pgfqpoint{1.262880in}{1.700416in}}%
\pgfpathcurveto{\pgfqpoint{1.257057in}{1.706240in}}{\pgfqpoint{1.249156in}{1.709512in}}{\pgfqpoint{1.240920in}{1.709512in}}%
\pgfpathcurveto{\pgfqpoint{1.232684in}{1.709512in}}{\pgfqpoint{1.224784in}{1.706240in}}{\pgfqpoint{1.218960in}{1.700416in}}%
\pgfpathcurveto{\pgfqpoint{1.213136in}{1.694592in}}{\pgfqpoint{1.209864in}{1.686692in}}{\pgfqpoint{1.209864in}{1.678455in}}%
\pgfpathcurveto{\pgfqpoint{1.209864in}{1.670219in}}{\pgfqpoint{1.213136in}{1.662319in}}{\pgfqpoint{1.218960in}{1.656495in}}%
\pgfpathcurveto{\pgfqpoint{1.224784in}{1.650671in}}{\pgfqpoint{1.232684in}{1.647399in}}{\pgfqpoint{1.240920in}{1.647399in}}%
\pgfpathclose%
\pgfusepath{stroke,fill}%
\end{pgfscope}%
\begin{pgfscope}%
\pgfpathrectangle{\pgfqpoint{0.100000in}{0.212622in}}{\pgfqpoint{3.696000in}{3.696000in}}%
\pgfusepath{clip}%
\pgfsetbuttcap%
\pgfsetroundjoin%
\definecolor{currentfill}{rgb}{0.121569,0.466667,0.705882}%
\pgfsetfillcolor{currentfill}%
\pgfsetfillopacity{0.914325}%
\pgfsetlinewidth{1.003750pt}%
\definecolor{currentstroke}{rgb}{0.121569,0.466667,0.705882}%
\pgfsetstrokecolor{currentstroke}%
\pgfsetstrokeopacity{0.914325}%
\pgfsetdash{}{0pt}%
\pgfpathmoveto{\pgfqpoint{1.264873in}{1.653165in}}%
\pgfpathcurveto{\pgfqpoint{1.273109in}{1.653165in}}{\pgfqpoint{1.281009in}{1.656437in}}{\pgfqpoint{1.286833in}{1.662261in}}%
\pgfpathcurveto{\pgfqpoint{1.292657in}{1.668085in}}{\pgfqpoint{1.295929in}{1.675985in}}{\pgfqpoint{1.295929in}{1.684222in}}%
\pgfpathcurveto{\pgfqpoint{1.295929in}{1.692458in}}{\pgfqpoint{1.292657in}{1.700358in}}{\pgfqpoint{1.286833in}{1.706182in}}%
\pgfpathcurveto{\pgfqpoint{1.281009in}{1.712006in}}{\pgfqpoint{1.273109in}{1.715278in}}{\pgfqpoint{1.264873in}{1.715278in}}%
\pgfpathcurveto{\pgfqpoint{1.256637in}{1.715278in}}{\pgfqpoint{1.248737in}{1.712006in}}{\pgfqpoint{1.242913in}{1.706182in}}%
\pgfpathcurveto{\pgfqpoint{1.237089in}{1.700358in}}{\pgfqpoint{1.233816in}{1.692458in}}{\pgfqpoint{1.233816in}{1.684222in}}%
\pgfpathcurveto{\pgfqpoint{1.233816in}{1.675985in}}{\pgfqpoint{1.237089in}{1.668085in}}{\pgfqpoint{1.242913in}{1.662261in}}%
\pgfpathcurveto{\pgfqpoint{1.248737in}{1.656437in}}{\pgfqpoint{1.256637in}{1.653165in}}{\pgfqpoint{1.264873in}{1.653165in}}%
\pgfpathclose%
\pgfusepath{stroke,fill}%
\end{pgfscope}%
\begin{pgfscope}%
\pgfpathrectangle{\pgfqpoint{0.100000in}{0.212622in}}{\pgfqpoint{3.696000in}{3.696000in}}%
\pgfusepath{clip}%
\pgfsetbuttcap%
\pgfsetroundjoin%
\definecolor{currentfill}{rgb}{0.121569,0.466667,0.705882}%
\pgfsetfillcolor{currentfill}%
\pgfsetfillopacity{0.914367}%
\pgfsetlinewidth{1.003750pt}%
\definecolor{currentstroke}{rgb}{0.121569,0.466667,0.705882}%
\pgfsetstrokecolor{currentstroke}%
\pgfsetstrokeopacity{0.914367}%
\pgfsetdash{}{0pt}%
\pgfpathmoveto{\pgfqpoint{1.366096in}{1.682222in}}%
\pgfpathcurveto{\pgfqpoint{1.374332in}{1.682222in}}{\pgfqpoint{1.382232in}{1.685495in}}{\pgfqpoint{1.388056in}{1.691319in}}%
\pgfpathcurveto{\pgfqpoint{1.393880in}{1.697142in}}{\pgfqpoint{1.397153in}{1.705043in}}{\pgfqpoint{1.397153in}{1.713279in}}%
\pgfpathcurveto{\pgfqpoint{1.397153in}{1.721515in}}{\pgfqpoint{1.393880in}{1.729415in}}{\pgfqpoint{1.388056in}{1.735239in}}%
\pgfpathcurveto{\pgfqpoint{1.382232in}{1.741063in}}{\pgfqpoint{1.374332in}{1.744335in}}{\pgfqpoint{1.366096in}{1.744335in}}%
\pgfpathcurveto{\pgfqpoint{1.357860in}{1.744335in}}{\pgfqpoint{1.349960in}{1.741063in}}{\pgfqpoint{1.344136in}{1.735239in}}%
\pgfpathcurveto{\pgfqpoint{1.338312in}{1.729415in}}{\pgfqpoint{1.335040in}{1.721515in}}{\pgfqpoint{1.335040in}{1.713279in}}%
\pgfpathcurveto{\pgfqpoint{1.335040in}{1.705043in}}{\pgfqpoint{1.338312in}{1.697142in}}{\pgfqpoint{1.344136in}{1.691319in}}%
\pgfpathcurveto{\pgfqpoint{1.349960in}{1.685495in}}{\pgfqpoint{1.357860in}{1.682222in}}{\pgfqpoint{1.366096in}{1.682222in}}%
\pgfpathclose%
\pgfusepath{stroke,fill}%
\end{pgfscope}%
\begin{pgfscope}%
\pgfpathrectangle{\pgfqpoint{0.100000in}{0.212622in}}{\pgfqpoint{3.696000in}{3.696000in}}%
\pgfusepath{clip}%
\pgfsetbuttcap%
\pgfsetroundjoin%
\definecolor{currentfill}{rgb}{0.121569,0.466667,0.705882}%
\pgfsetfillcolor{currentfill}%
\pgfsetfillopacity{0.914590}%
\pgfsetlinewidth{1.003750pt}%
\definecolor{currentstroke}{rgb}{0.121569,0.466667,0.705882}%
\pgfsetstrokecolor{currentstroke}%
\pgfsetstrokeopacity{0.914590}%
\pgfsetdash{}{0pt}%
\pgfpathmoveto{\pgfqpoint{1.457251in}{1.710432in}}%
\pgfpathcurveto{\pgfqpoint{1.465487in}{1.710432in}}{\pgfqpoint{1.473387in}{1.713704in}}{\pgfqpoint{1.479211in}{1.719528in}}%
\pgfpathcurveto{\pgfqpoint{1.485035in}{1.725352in}}{\pgfqpoint{1.488307in}{1.733252in}}{\pgfqpoint{1.488307in}{1.741488in}}%
\pgfpathcurveto{\pgfqpoint{1.488307in}{1.749724in}}{\pgfqpoint{1.485035in}{1.757624in}}{\pgfqpoint{1.479211in}{1.763448in}}%
\pgfpathcurveto{\pgfqpoint{1.473387in}{1.769272in}}{\pgfqpoint{1.465487in}{1.772545in}}{\pgfqpoint{1.457251in}{1.772545in}}%
\pgfpathcurveto{\pgfqpoint{1.449015in}{1.772545in}}{\pgfqpoint{1.441115in}{1.769272in}}{\pgfqpoint{1.435291in}{1.763448in}}%
\pgfpathcurveto{\pgfqpoint{1.429467in}{1.757624in}}{\pgfqpoint{1.426194in}{1.749724in}}{\pgfqpoint{1.426194in}{1.741488in}}%
\pgfpathcurveto{\pgfqpoint{1.426194in}{1.733252in}}{\pgfqpoint{1.429467in}{1.725352in}}{\pgfqpoint{1.435291in}{1.719528in}}%
\pgfpathcurveto{\pgfqpoint{1.441115in}{1.713704in}}{\pgfqpoint{1.449015in}{1.710432in}}{\pgfqpoint{1.457251in}{1.710432in}}%
\pgfpathclose%
\pgfusepath{stroke,fill}%
\end{pgfscope}%
\begin{pgfscope}%
\pgfpathrectangle{\pgfqpoint{0.100000in}{0.212622in}}{\pgfqpoint{3.696000in}{3.696000in}}%
\pgfusepath{clip}%
\pgfsetbuttcap%
\pgfsetroundjoin%
\definecolor{currentfill}{rgb}{0.121569,0.466667,0.705882}%
\pgfsetfillcolor{currentfill}%
\pgfsetfillopacity{0.914804}%
\pgfsetlinewidth{1.003750pt}%
\definecolor{currentstroke}{rgb}{0.121569,0.466667,0.705882}%
\pgfsetstrokecolor{currentstroke}%
\pgfsetstrokeopacity{0.914804}%
\pgfsetdash{}{0pt}%
\pgfpathmoveto{\pgfqpoint{1.293771in}{1.659279in}}%
\pgfpathcurveto{\pgfqpoint{1.302007in}{1.659279in}}{\pgfqpoint{1.309907in}{1.662552in}}{\pgfqpoint{1.315731in}{1.668376in}}%
\pgfpathcurveto{\pgfqpoint{1.321555in}{1.674200in}}{\pgfqpoint{1.324827in}{1.682100in}}{\pgfqpoint{1.324827in}{1.690336in}}%
\pgfpathcurveto{\pgfqpoint{1.324827in}{1.698572in}}{\pgfqpoint{1.321555in}{1.706472in}}{\pgfqpoint{1.315731in}{1.712296in}}%
\pgfpathcurveto{\pgfqpoint{1.309907in}{1.718120in}}{\pgfqpoint{1.302007in}{1.721392in}}{\pgfqpoint{1.293771in}{1.721392in}}%
\pgfpathcurveto{\pgfqpoint{1.285535in}{1.721392in}}{\pgfqpoint{1.277635in}{1.718120in}}{\pgfqpoint{1.271811in}{1.712296in}}%
\pgfpathcurveto{\pgfqpoint{1.265987in}{1.706472in}}{\pgfqpoint{1.262714in}{1.698572in}}{\pgfqpoint{1.262714in}{1.690336in}}%
\pgfpathcurveto{\pgfqpoint{1.262714in}{1.682100in}}{\pgfqpoint{1.265987in}{1.674200in}}{\pgfqpoint{1.271811in}{1.668376in}}%
\pgfpathcurveto{\pgfqpoint{1.277635in}{1.662552in}}{\pgfqpoint{1.285535in}{1.659279in}}{\pgfqpoint{1.293771in}{1.659279in}}%
\pgfpathclose%
\pgfusepath{stroke,fill}%
\end{pgfscope}%
\begin{pgfscope}%
\pgfpathrectangle{\pgfqpoint{0.100000in}{0.212622in}}{\pgfqpoint{3.696000in}{3.696000in}}%
\pgfusepath{clip}%
\pgfsetbuttcap%
\pgfsetroundjoin%
\definecolor{currentfill}{rgb}{0.121569,0.466667,0.705882}%
\pgfsetfillcolor{currentfill}%
\pgfsetfillopacity{0.915137}%
\pgfsetlinewidth{1.003750pt}%
\definecolor{currentstroke}{rgb}{0.121569,0.466667,0.705882}%
\pgfsetstrokecolor{currentstroke}%
\pgfsetstrokeopacity{0.915137}%
\pgfsetdash{}{0pt}%
\pgfpathmoveto{\pgfqpoint{1.309094in}{1.660874in}}%
\pgfpathcurveto{\pgfqpoint{1.317330in}{1.660874in}}{\pgfqpoint{1.325230in}{1.664147in}}{\pgfqpoint{1.331054in}{1.669971in}}%
\pgfpathcurveto{\pgfqpoint{1.336878in}{1.675795in}}{\pgfqpoint{1.340150in}{1.683695in}}{\pgfqpoint{1.340150in}{1.691931in}}%
\pgfpathcurveto{\pgfqpoint{1.340150in}{1.700167in}}{\pgfqpoint{1.336878in}{1.708067in}}{\pgfqpoint{1.331054in}{1.713891in}}%
\pgfpathcurveto{\pgfqpoint{1.325230in}{1.719715in}}{\pgfqpoint{1.317330in}{1.722987in}}{\pgfqpoint{1.309094in}{1.722987in}}%
\pgfpathcurveto{\pgfqpoint{1.300858in}{1.722987in}}{\pgfqpoint{1.292958in}{1.719715in}}{\pgfqpoint{1.287134in}{1.713891in}}%
\pgfpathcurveto{\pgfqpoint{1.281310in}{1.708067in}}{\pgfqpoint{1.278037in}{1.700167in}}{\pgfqpoint{1.278037in}{1.691931in}}%
\pgfpathcurveto{\pgfqpoint{1.278037in}{1.683695in}}{\pgfqpoint{1.281310in}{1.675795in}}{\pgfqpoint{1.287134in}{1.669971in}}%
\pgfpathcurveto{\pgfqpoint{1.292958in}{1.664147in}}{\pgfqpoint{1.300858in}{1.660874in}}{\pgfqpoint{1.309094in}{1.660874in}}%
\pgfpathclose%
\pgfusepath{stroke,fill}%
\end{pgfscope}%
\begin{pgfscope}%
\pgfpathrectangle{\pgfqpoint{0.100000in}{0.212622in}}{\pgfqpoint{3.696000in}{3.696000in}}%
\pgfusepath{clip}%
\pgfsetbuttcap%
\pgfsetroundjoin%
\definecolor{currentfill}{rgb}{0.121569,0.466667,0.705882}%
\pgfsetfillcolor{currentfill}%
\pgfsetfillopacity{0.917226}%
\pgfsetlinewidth{1.003750pt}%
\definecolor{currentstroke}{rgb}{0.121569,0.466667,0.705882}%
\pgfsetstrokecolor{currentstroke}%
\pgfsetstrokeopacity{0.917226}%
\pgfsetdash{}{0pt}%
\pgfpathmoveto{\pgfqpoint{1.507984in}{1.721614in}}%
\pgfpathcurveto{\pgfqpoint{1.516220in}{1.721614in}}{\pgfqpoint{1.524120in}{1.724886in}}{\pgfqpoint{1.529944in}{1.730710in}}%
\pgfpathcurveto{\pgfqpoint{1.535768in}{1.736534in}}{\pgfqpoint{1.539040in}{1.744434in}}{\pgfqpoint{1.539040in}{1.752670in}}%
\pgfpathcurveto{\pgfqpoint{1.539040in}{1.760906in}}{\pgfqpoint{1.535768in}{1.768806in}}{\pgfqpoint{1.529944in}{1.774630in}}%
\pgfpathcurveto{\pgfqpoint{1.524120in}{1.780454in}}{\pgfqpoint{1.516220in}{1.783727in}}{\pgfqpoint{1.507984in}{1.783727in}}%
\pgfpathcurveto{\pgfqpoint{1.499748in}{1.783727in}}{\pgfqpoint{1.491847in}{1.780454in}}{\pgfqpoint{1.486024in}{1.774630in}}%
\pgfpathcurveto{\pgfqpoint{1.480200in}{1.768806in}}{\pgfqpoint{1.476927in}{1.760906in}}{\pgfqpoint{1.476927in}{1.752670in}}%
\pgfpathcurveto{\pgfqpoint{1.476927in}{1.744434in}}{\pgfqpoint{1.480200in}{1.736534in}}{\pgfqpoint{1.486024in}{1.730710in}}%
\pgfpathcurveto{\pgfqpoint{1.491847in}{1.724886in}}{\pgfqpoint{1.499748in}{1.721614in}}{\pgfqpoint{1.507984in}{1.721614in}}%
\pgfpathclose%
\pgfusepath{stroke,fill}%
\end{pgfscope}%
\begin{pgfscope}%
\pgfpathrectangle{\pgfqpoint{0.100000in}{0.212622in}}{\pgfqpoint{3.696000in}{3.696000in}}%
\pgfusepath{clip}%
\pgfsetbuttcap%
\pgfsetroundjoin%
\definecolor{currentfill}{rgb}{0.121569,0.466667,0.705882}%
\pgfsetfillcolor{currentfill}%
\pgfsetfillopacity{0.920170}%
\pgfsetlinewidth{1.003750pt}%
\definecolor{currentstroke}{rgb}{0.121569,0.466667,0.705882}%
\pgfsetstrokecolor{currentstroke}%
\pgfsetstrokeopacity{0.920170}%
\pgfsetdash{}{0pt}%
\pgfpathmoveto{\pgfqpoint{2.802497in}{1.875188in}}%
\pgfpathcurveto{\pgfqpoint{2.810733in}{1.875188in}}{\pgfqpoint{2.818633in}{1.878460in}}{\pgfqpoint{2.824457in}{1.884284in}}%
\pgfpathcurveto{\pgfqpoint{2.830281in}{1.890108in}}{\pgfqpoint{2.833553in}{1.898008in}}{\pgfqpoint{2.833553in}{1.906245in}}%
\pgfpathcurveto{\pgfqpoint{2.833553in}{1.914481in}}{\pgfqpoint{2.830281in}{1.922381in}}{\pgfqpoint{2.824457in}{1.928205in}}%
\pgfpathcurveto{\pgfqpoint{2.818633in}{1.934029in}}{\pgfqpoint{2.810733in}{1.937301in}}{\pgfqpoint{2.802497in}{1.937301in}}%
\pgfpathcurveto{\pgfqpoint{2.794261in}{1.937301in}}{\pgfqpoint{2.786361in}{1.934029in}}{\pgfqpoint{2.780537in}{1.928205in}}%
\pgfpathcurveto{\pgfqpoint{2.774713in}{1.922381in}}{\pgfqpoint{2.771440in}{1.914481in}}{\pgfqpoint{2.771440in}{1.906245in}}%
\pgfpathcurveto{\pgfqpoint{2.771440in}{1.898008in}}{\pgfqpoint{2.774713in}{1.890108in}}{\pgfqpoint{2.780537in}{1.884284in}}%
\pgfpathcurveto{\pgfqpoint{2.786361in}{1.878460in}}{\pgfqpoint{2.794261in}{1.875188in}}{\pgfqpoint{2.802497in}{1.875188in}}%
\pgfpathclose%
\pgfusepath{stroke,fill}%
\end{pgfscope}%
\begin{pgfscope}%
\pgfpathrectangle{\pgfqpoint{0.100000in}{0.212622in}}{\pgfqpoint{3.696000in}{3.696000in}}%
\pgfusepath{clip}%
\pgfsetbuttcap%
\pgfsetroundjoin%
\definecolor{currentfill}{rgb}{0.121569,0.466667,0.705882}%
\pgfsetfillcolor{currentfill}%
\pgfsetfillopacity{0.920587}%
\pgfsetlinewidth{1.003750pt}%
\definecolor{currentstroke}{rgb}{0.121569,0.466667,0.705882}%
\pgfsetstrokecolor{currentstroke}%
\pgfsetstrokeopacity{0.920587}%
\pgfsetdash{}{0pt}%
\pgfpathmoveto{\pgfqpoint{1.558462in}{1.716393in}}%
\pgfpathcurveto{\pgfqpoint{1.566698in}{1.716393in}}{\pgfqpoint{1.574598in}{1.719665in}}{\pgfqpoint{1.580422in}{1.725489in}}%
\pgfpathcurveto{\pgfqpoint{1.586246in}{1.731313in}}{\pgfqpoint{1.589518in}{1.739213in}}{\pgfqpoint{1.589518in}{1.747449in}}%
\pgfpathcurveto{\pgfqpoint{1.589518in}{1.755686in}}{\pgfqpoint{1.586246in}{1.763586in}}{\pgfqpoint{1.580422in}{1.769410in}}%
\pgfpathcurveto{\pgfqpoint{1.574598in}{1.775233in}}{\pgfqpoint{1.566698in}{1.778506in}}{\pgfqpoint{1.558462in}{1.778506in}}%
\pgfpathcurveto{\pgfqpoint{1.550226in}{1.778506in}}{\pgfqpoint{1.542326in}{1.775233in}}{\pgfqpoint{1.536502in}{1.769410in}}%
\pgfpathcurveto{\pgfqpoint{1.530678in}{1.763586in}}{\pgfqpoint{1.527405in}{1.755686in}}{\pgfqpoint{1.527405in}{1.747449in}}%
\pgfpathcurveto{\pgfqpoint{1.527405in}{1.739213in}}{\pgfqpoint{1.530678in}{1.731313in}}{\pgfqpoint{1.536502in}{1.725489in}}%
\pgfpathcurveto{\pgfqpoint{1.542326in}{1.719665in}}{\pgfqpoint{1.550226in}{1.716393in}}{\pgfqpoint{1.558462in}{1.716393in}}%
\pgfpathclose%
\pgfusepath{stroke,fill}%
\end{pgfscope}%
\begin{pgfscope}%
\pgfpathrectangle{\pgfqpoint{0.100000in}{0.212622in}}{\pgfqpoint{3.696000in}{3.696000in}}%
\pgfusepath{clip}%
\pgfsetbuttcap%
\pgfsetroundjoin%
\definecolor{currentfill}{rgb}{0.121569,0.466667,0.705882}%
\pgfsetfillcolor{currentfill}%
\pgfsetfillopacity{0.924754}%
\pgfsetlinewidth{1.003750pt}%
\definecolor{currentstroke}{rgb}{0.121569,0.466667,0.705882}%
\pgfsetstrokecolor{currentstroke}%
\pgfsetstrokeopacity{0.924754}%
\pgfsetdash{}{0pt}%
\pgfpathmoveto{\pgfqpoint{1.620719in}{1.732821in}}%
\pgfpathcurveto{\pgfqpoint{1.628955in}{1.732821in}}{\pgfqpoint{1.636856in}{1.736093in}}{\pgfqpoint{1.642679in}{1.741917in}}%
\pgfpathcurveto{\pgfqpoint{1.648503in}{1.747741in}}{\pgfqpoint{1.651776in}{1.755641in}}{\pgfqpoint{1.651776in}{1.763877in}}%
\pgfpathcurveto{\pgfqpoint{1.651776in}{1.772113in}}{\pgfqpoint{1.648503in}{1.780013in}}{\pgfqpoint{1.642679in}{1.785837in}}%
\pgfpathcurveto{\pgfqpoint{1.636856in}{1.791661in}}{\pgfqpoint{1.628955in}{1.794934in}}{\pgfqpoint{1.620719in}{1.794934in}}%
\pgfpathcurveto{\pgfqpoint{1.612483in}{1.794934in}}{\pgfqpoint{1.604583in}{1.791661in}}{\pgfqpoint{1.598759in}{1.785837in}}%
\pgfpathcurveto{\pgfqpoint{1.592935in}{1.780013in}}{\pgfqpoint{1.589663in}{1.772113in}}{\pgfqpoint{1.589663in}{1.763877in}}%
\pgfpathcurveto{\pgfqpoint{1.589663in}{1.755641in}}{\pgfqpoint{1.592935in}{1.747741in}}{\pgfqpoint{1.598759in}{1.741917in}}%
\pgfpathcurveto{\pgfqpoint{1.604583in}{1.736093in}}{\pgfqpoint{1.612483in}{1.732821in}}{\pgfqpoint{1.620719in}{1.732821in}}%
\pgfpathclose%
\pgfusepath{stroke,fill}%
\end{pgfscope}%
\begin{pgfscope}%
\pgfpathrectangle{\pgfqpoint{0.100000in}{0.212622in}}{\pgfqpoint{3.696000in}{3.696000in}}%
\pgfusepath{clip}%
\pgfsetbuttcap%
\pgfsetroundjoin%
\definecolor{currentfill}{rgb}{0.121569,0.466667,0.705882}%
\pgfsetfillcolor{currentfill}%
\pgfsetfillopacity{0.928202}%
\pgfsetlinewidth{1.003750pt}%
\definecolor{currentstroke}{rgb}{0.121569,0.466667,0.705882}%
\pgfsetstrokecolor{currentstroke}%
\pgfsetstrokeopacity{0.928202}%
\pgfsetdash{}{0pt}%
\pgfpathmoveto{\pgfqpoint{1.649559in}{1.727227in}}%
\pgfpathcurveto{\pgfqpoint{1.657795in}{1.727227in}}{\pgfqpoint{1.665695in}{1.730499in}}{\pgfqpoint{1.671519in}{1.736323in}}%
\pgfpathcurveto{\pgfqpoint{1.677343in}{1.742147in}}{\pgfqpoint{1.680615in}{1.750047in}}{\pgfqpoint{1.680615in}{1.758283in}}%
\pgfpathcurveto{\pgfqpoint{1.680615in}{1.766520in}}{\pgfqpoint{1.677343in}{1.774420in}}{\pgfqpoint{1.671519in}{1.780244in}}%
\pgfpathcurveto{\pgfqpoint{1.665695in}{1.786068in}}{\pgfqpoint{1.657795in}{1.789340in}}{\pgfqpoint{1.649559in}{1.789340in}}%
\pgfpathcurveto{\pgfqpoint{1.641323in}{1.789340in}}{\pgfqpoint{1.633423in}{1.786068in}}{\pgfqpoint{1.627599in}{1.780244in}}%
\pgfpathcurveto{\pgfqpoint{1.621775in}{1.774420in}}{\pgfqpoint{1.618502in}{1.766520in}}{\pgfqpoint{1.618502in}{1.758283in}}%
\pgfpathcurveto{\pgfqpoint{1.618502in}{1.750047in}}{\pgfqpoint{1.621775in}{1.742147in}}{\pgfqpoint{1.627599in}{1.736323in}}%
\pgfpathcurveto{\pgfqpoint{1.633423in}{1.730499in}}{\pgfqpoint{1.641323in}{1.727227in}}{\pgfqpoint{1.649559in}{1.727227in}}%
\pgfpathclose%
\pgfusepath{stroke,fill}%
\end{pgfscope}%
\begin{pgfscope}%
\pgfpathrectangle{\pgfqpoint{0.100000in}{0.212622in}}{\pgfqpoint{3.696000in}{3.696000in}}%
\pgfusepath{clip}%
\pgfsetbuttcap%
\pgfsetroundjoin%
\definecolor{currentfill}{rgb}{0.121569,0.466667,0.705882}%
\pgfsetfillcolor{currentfill}%
\pgfsetfillopacity{0.930838}%
\pgfsetlinewidth{1.003750pt}%
\definecolor{currentstroke}{rgb}{0.121569,0.466667,0.705882}%
\pgfsetstrokecolor{currentstroke}%
\pgfsetstrokeopacity{0.930838}%
\pgfsetdash{}{0pt}%
\pgfpathmoveto{\pgfqpoint{1.664652in}{1.723972in}}%
\pgfpathcurveto{\pgfqpoint{1.672888in}{1.723972in}}{\pgfqpoint{1.680788in}{1.727244in}}{\pgfqpoint{1.686612in}{1.733068in}}%
\pgfpathcurveto{\pgfqpoint{1.692436in}{1.738892in}}{\pgfqpoint{1.695708in}{1.746792in}}{\pgfqpoint{1.695708in}{1.755028in}}%
\pgfpathcurveto{\pgfqpoint{1.695708in}{1.763265in}}{\pgfqpoint{1.692436in}{1.771165in}}{\pgfqpoint{1.686612in}{1.776989in}}%
\pgfpathcurveto{\pgfqpoint{1.680788in}{1.782812in}}{\pgfqpoint{1.672888in}{1.786085in}}{\pgfqpoint{1.664652in}{1.786085in}}%
\pgfpathcurveto{\pgfqpoint{1.656416in}{1.786085in}}{\pgfqpoint{1.648516in}{1.782812in}}{\pgfqpoint{1.642692in}{1.776989in}}%
\pgfpathcurveto{\pgfqpoint{1.636868in}{1.771165in}}{\pgfqpoint{1.633595in}{1.763265in}}{\pgfqpoint{1.633595in}{1.755028in}}%
\pgfpathcurveto{\pgfqpoint{1.633595in}{1.746792in}}{\pgfqpoint{1.636868in}{1.738892in}}{\pgfqpoint{1.642692in}{1.733068in}}%
\pgfpathcurveto{\pgfqpoint{1.648516in}{1.727244in}}{\pgfqpoint{1.656416in}{1.723972in}}{\pgfqpoint{1.664652in}{1.723972in}}%
\pgfpathclose%
\pgfusepath{stroke,fill}%
\end{pgfscope}%
\begin{pgfscope}%
\pgfpathrectangle{\pgfqpoint{0.100000in}{0.212622in}}{\pgfqpoint{3.696000in}{3.696000in}}%
\pgfusepath{clip}%
\pgfsetbuttcap%
\pgfsetroundjoin%
\definecolor{currentfill}{rgb}{0.121569,0.466667,0.705882}%
\pgfsetfillcolor{currentfill}%
\pgfsetfillopacity{0.932827}%
\pgfsetlinewidth{1.003750pt}%
\definecolor{currentstroke}{rgb}{0.121569,0.466667,0.705882}%
\pgfsetstrokecolor{currentstroke}%
\pgfsetstrokeopacity{0.932827}%
\pgfsetdash{}{0pt}%
\pgfpathmoveto{\pgfqpoint{1.671032in}{1.717493in}}%
\pgfpathcurveto{\pgfqpoint{1.679269in}{1.717493in}}{\pgfqpoint{1.687169in}{1.720765in}}{\pgfqpoint{1.692993in}{1.726589in}}%
\pgfpathcurveto{\pgfqpoint{1.698816in}{1.732413in}}{\pgfqpoint{1.702089in}{1.740313in}}{\pgfqpoint{1.702089in}{1.748549in}}%
\pgfpathcurveto{\pgfqpoint{1.702089in}{1.756786in}}{\pgfqpoint{1.698816in}{1.764686in}}{\pgfqpoint{1.692993in}{1.770510in}}%
\pgfpathcurveto{\pgfqpoint{1.687169in}{1.776334in}}{\pgfqpoint{1.679269in}{1.779606in}}{\pgfqpoint{1.671032in}{1.779606in}}%
\pgfpathcurveto{\pgfqpoint{1.662796in}{1.779606in}}{\pgfqpoint{1.654896in}{1.776334in}}{\pgfqpoint{1.649072in}{1.770510in}}%
\pgfpathcurveto{\pgfqpoint{1.643248in}{1.764686in}}{\pgfqpoint{1.639976in}{1.756786in}}{\pgfqpoint{1.639976in}{1.748549in}}%
\pgfpathcurveto{\pgfqpoint{1.639976in}{1.740313in}}{\pgfqpoint{1.643248in}{1.732413in}}{\pgfqpoint{1.649072in}{1.726589in}}%
\pgfpathcurveto{\pgfqpoint{1.654896in}{1.720765in}}{\pgfqpoint{1.662796in}{1.717493in}}{\pgfqpoint{1.671032in}{1.717493in}}%
\pgfpathclose%
\pgfusepath{stroke,fill}%
\end{pgfscope}%
\begin{pgfscope}%
\pgfpathrectangle{\pgfqpoint{0.100000in}{0.212622in}}{\pgfqpoint{3.696000in}{3.696000in}}%
\pgfusepath{clip}%
\pgfsetbuttcap%
\pgfsetroundjoin%
\definecolor{currentfill}{rgb}{0.121569,0.466667,0.705882}%
\pgfsetfillcolor{currentfill}%
\pgfsetfillopacity{0.933150}%
\pgfsetlinewidth{1.003750pt}%
\definecolor{currentstroke}{rgb}{0.121569,0.466667,0.705882}%
\pgfsetstrokecolor{currentstroke}%
\pgfsetstrokeopacity{0.933150}%
\pgfsetdash{}{0pt}%
\pgfpathmoveto{\pgfqpoint{1.676617in}{1.718391in}}%
\pgfpathcurveto{\pgfqpoint{1.684853in}{1.718391in}}{\pgfqpoint{1.692753in}{1.721664in}}{\pgfqpoint{1.698577in}{1.727488in}}%
\pgfpathcurveto{\pgfqpoint{1.704401in}{1.733312in}}{\pgfqpoint{1.707674in}{1.741212in}}{\pgfqpoint{1.707674in}{1.749448in}}%
\pgfpathcurveto{\pgfqpoint{1.707674in}{1.757684in}}{\pgfqpoint{1.704401in}{1.765584in}}{\pgfqpoint{1.698577in}{1.771408in}}%
\pgfpathcurveto{\pgfqpoint{1.692753in}{1.777232in}}{\pgfqpoint{1.684853in}{1.780504in}}{\pgfqpoint{1.676617in}{1.780504in}}%
\pgfpathcurveto{\pgfqpoint{1.668381in}{1.780504in}}{\pgfqpoint{1.660481in}{1.777232in}}{\pgfqpoint{1.654657in}{1.771408in}}%
\pgfpathcurveto{\pgfqpoint{1.648833in}{1.765584in}}{\pgfqpoint{1.645561in}{1.757684in}}{\pgfqpoint{1.645561in}{1.749448in}}%
\pgfpathcurveto{\pgfqpoint{1.645561in}{1.741212in}}{\pgfqpoint{1.648833in}{1.733312in}}{\pgfqpoint{1.654657in}{1.727488in}}%
\pgfpathcurveto{\pgfqpoint{1.660481in}{1.721664in}}{\pgfqpoint{1.668381in}{1.718391in}}{\pgfqpoint{1.676617in}{1.718391in}}%
\pgfpathclose%
\pgfusepath{stroke,fill}%
\end{pgfscope}%
\begin{pgfscope}%
\pgfpathrectangle{\pgfqpoint{0.100000in}{0.212622in}}{\pgfqpoint{3.696000in}{3.696000in}}%
\pgfusepath{clip}%
\pgfsetbuttcap%
\pgfsetroundjoin%
\definecolor{currentfill}{rgb}{0.121569,0.466667,0.705882}%
\pgfsetfillcolor{currentfill}%
\pgfsetfillopacity{0.934455}%
\pgfsetlinewidth{1.003750pt}%
\definecolor{currentstroke}{rgb}{0.121569,0.466667,0.705882}%
\pgfsetstrokecolor{currentstroke}%
\pgfsetstrokeopacity{0.934455}%
\pgfsetdash{}{0pt}%
\pgfpathmoveto{\pgfqpoint{1.684711in}{1.716609in}}%
\pgfpathcurveto{\pgfqpoint{1.692948in}{1.716609in}}{\pgfqpoint{1.700848in}{1.719882in}}{\pgfqpoint{1.706671in}{1.725706in}}%
\pgfpathcurveto{\pgfqpoint{1.712495in}{1.731530in}}{\pgfqpoint{1.715768in}{1.739430in}}{\pgfqpoint{1.715768in}{1.747666in}}%
\pgfpathcurveto{\pgfqpoint{1.715768in}{1.755902in}}{\pgfqpoint{1.712495in}{1.763802in}}{\pgfqpoint{1.706671in}{1.769626in}}%
\pgfpathcurveto{\pgfqpoint{1.700848in}{1.775450in}}{\pgfqpoint{1.692948in}{1.778722in}}{\pgfqpoint{1.684711in}{1.778722in}}%
\pgfpathcurveto{\pgfqpoint{1.676475in}{1.778722in}}{\pgfqpoint{1.668575in}{1.775450in}}{\pgfqpoint{1.662751in}{1.769626in}}%
\pgfpathcurveto{\pgfqpoint{1.656927in}{1.763802in}}{\pgfqpoint{1.653655in}{1.755902in}}{\pgfqpoint{1.653655in}{1.747666in}}%
\pgfpathcurveto{\pgfqpoint{1.653655in}{1.739430in}}{\pgfqpoint{1.656927in}{1.731530in}}{\pgfqpoint{1.662751in}{1.725706in}}%
\pgfpathcurveto{\pgfqpoint{1.668575in}{1.719882in}}{\pgfqpoint{1.676475in}{1.716609in}}{\pgfqpoint{1.684711in}{1.716609in}}%
\pgfpathclose%
\pgfusepath{stroke,fill}%
\end{pgfscope}%
\begin{pgfscope}%
\pgfpathrectangle{\pgfqpoint{0.100000in}{0.212622in}}{\pgfqpoint{3.696000in}{3.696000in}}%
\pgfusepath{clip}%
\pgfsetbuttcap%
\pgfsetroundjoin%
\definecolor{currentfill}{rgb}{0.121569,0.466667,0.705882}%
\pgfsetfillcolor{currentfill}%
\pgfsetfillopacity{0.936573}%
\pgfsetlinewidth{1.003750pt}%
\definecolor{currentstroke}{rgb}{0.121569,0.466667,0.705882}%
\pgfsetstrokecolor{currentstroke}%
\pgfsetstrokeopacity{0.936573}%
\pgfsetdash{}{0pt}%
\pgfpathmoveto{\pgfqpoint{1.694840in}{1.711658in}}%
\pgfpathcurveto{\pgfqpoint{1.703076in}{1.711658in}}{\pgfqpoint{1.710976in}{1.714930in}}{\pgfqpoint{1.716800in}{1.720754in}}%
\pgfpathcurveto{\pgfqpoint{1.722624in}{1.726578in}}{\pgfqpoint{1.725897in}{1.734478in}}{\pgfqpoint{1.725897in}{1.742714in}}%
\pgfpathcurveto{\pgfqpoint{1.725897in}{1.750951in}}{\pgfqpoint{1.722624in}{1.758851in}}{\pgfqpoint{1.716800in}{1.764675in}}%
\pgfpathcurveto{\pgfqpoint{1.710976in}{1.770498in}}{\pgfqpoint{1.703076in}{1.773771in}}{\pgfqpoint{1.694840in}{1.773771in}}%
\pgfpathcurveto{\pgfqpoint{1.686604in}{1.773771in}}{\pgfqpoint{1.678704in}{1.770498in}}{\pgfqpoint{1.672880in}{1.764675in}}%
\pgfpathcurveto{\pgfqpoint{1.667056in}{1.758851in}}{\pgfqpoint{1.663784in}{1.750951in}}{\pgfqpoint{1.663784in}{1.742714in}}%
\pgfpathcurveto{\pgfqpoint{1.663784in}{1.734478in}}{\pgfqpoint{1.667056in}{1.726578in}}{\pgfqpoint{1.672880in}{1.720754in}}%
\pgfpathcurveto{\pgfqpoint{1.678704in}{1.714930in}}{\pgfqpoint{1.686604in}{1.711658in}}{\pgfqpoint{1.694840in}{1.711658in}}%
\pgfpathclose%
\pgfusepath{stroke,fill}%
\end{pgfscope}%
\begin{pgfscope}%
\pgfpathrectangle{\pgfqpoint{0.100000in}{0.212622in}}{\pgfqpoint{3.696000in}{3.696000in}}%
\pgfusepath{clip}%
\pgfsetbuttcap%
\pgfsetroundjoin%
\definecolor{currentfill}{rgb}{0.121569,0.466667,0.705882}%
\pgfsetfillcolor{currentfill}%
\pgfsetfillopacity{0.939078}%
\pgfsetlinewidth{1.003750pt}%
\definecolor{currentstroke}{rgb}{0.121569,0.466667,0.705882}%
\pgfsetstrokecolor{currentstroke}%
\pgfsetstrokeopacity{0.939078}%
\pgfsetdash{}{0pt}%
\pgfpathmoveto{\pgfqpoint{1.710296in}{1.709918in}}%
\pgfpathcurveto{\pgfqpoint{1.718532in}{1.709918in}}{\pgfqpoint{1.726432in}{1.713190in}}{\pgfqpoint{1.732256in}{1.719014in}}%
\pgfpathcurveto{\pgfqpoint{1.738080in}{1.724838in}}{\pgfqpoint{1.741352in}{1.732738in}}{\pgfqpoint{1.741352in}{1.740974in}}%
\pgfpathcurveto{\pgfqpoint{1.741352in}{1.749210in}}{\pgfqpoint{1.738080in}{1.757110in}}{\pgfqpoint{1.732256in}{1.762934in}}%
\pgfpathcurveto{\pgfqpoint{1.726432in}{1.768758in}}{\pgfqpoint{1.718532in}{1.772031in}}{\pgfqpoint{1.710296in}{1.772031in}}%
\pgfpathcurveto{\pgfqpoint{1.702059in}{1.772031in}}{\pgfqpoint{1.694159in}{1.768758in}}{\pgfqpoint{1.688335in}{1.762934in}}%
\pgfpathcurveto{\pgfqpoint{1.682511in}{1.757110in}}{\pgfqpoint{1.679239in}{1.749210in}}{\pgfqpoint{1.679239in}{1.740974in}}%
\pgfpathcurveto{\pgfqpoint{1.679239in}{1.732738in}}{\pgfqpoint{1.682511in}{1.724838in}}{\pgfqpoint{1.688335in}{1.719014in}}%
\pgfpathcurveto{\pgfqpoint{1.694159in}{1.713190in}}{\pgfqpoint{1.702059in}{1.709918in}}{\pgfqpoint{1.710296in}{1.709918in}}%
\pgfpathclose%
\pgfusepath{stroke,fill}%
\end{pgfscope}%
\begin{pgfscope}%
\pgfpathrectangle{\pgfqpoint{0.100000in}{0.212622in}}{\pgfqpoint{3.696000in}{3.696000in}}%
\pgfusepath{clip}%
\pgfsetbuttcap%
\pgfsetroundjoin%
\definecolor{currentfill}{rgb}{0.121569,0.466667,0.705882}%
\pgfsetfillcolor{currentfill}%
\pgfsetfillopacity{0.942794}%
\pgfsetlinewidth{1.003750pt}%
\definecolor{currentstroke}{rgb}{0.121569,0.466667,0.705882}%
\pgfsetstrokecolor{currentstroke}%
\pgfsetstrokeopacity{0.942794}%
\pgfsetdash{}{0pt}%
\pgfpathmoveto{\pgfqpoint{1.733444in}{1.703388in}}%
\pgfpathcurveto{\pgfqpoint{1.741680in}{1.703388in}}{\pgfqpoint{1.749580in}{1.706660in}}{\pgfqpoint{1.755404in}{1.712484in}}%
\pgfpathcurveto{\pgfqpoint{1.761228in}{1.718308in}}{\pgfqpoint{1.764500in}{1.726208in}}{\pgfqpoint{1.764500in}{1.734444in}}%
\pgfpathcurveto{\pgfqpoint{1.764500in}{1.742681in}}{\pgfqpoint{1.761228in}{1.750581in}}{\pgfqpoint{1.755404in}{1.756405in}}%
\pgfpathcurveto{\pgfqpoint{1.749580in}{1.762229in}}{\pgfqpoint{1.741680in}{1.765501in}}{\pgfqpoint{1.733444in}{1.765501in}}%
\pgfpathcurveto{\pgfqpoint{1.725207in}{1.765501in}}{\pgfqpoint{1.717307in}{1.762229in}}{\pgfqpoint{1.711483in}{1.756405in}}%
\pgfpathcurveto{\pgfqpoint{1.705660in}{1.750581in}}{\pgfqpoint{1.702387in}{1.742681in}}{\pgfqpoint{1.702387in}{1.734444in}}%
\pgfpathcurveto{\pgfqpoint{1.702387in}{1.726208in}}{\pgfqpoint{1.705660in}{1.718308in}}{\pgfqpoint{1.711483in}{1.712484in}}%
\pgfpathcurveto{\pgfqpoint{1.717307in}{1.706660in}}{\pgfqpoint{1.725207in}{1.703388in}}{\pgfqpoint{1.733444in}{1.703388in}}%
\pgfpathclose%
\pgfusepath{stroke,fill}%
\end{pgfscope}%
\begin{pgfscope}%
\pgfpathrectangle{\pgfqpoint{0.100000in}{0.212622in}}{\pgfqpoint{3.696000in}{3.696000in}}%
\pgfusepath{clip}%
\pgfsetbuttcap%
\pgfsetroundjoin%
\definecolor{currentfill}{rgb}{0.121569,0.466667,0.705882}%
\pgfsetfillcolor{currentfill}%
\pgfsetfillopacity{0.948801}%
\pgfsetlinewidth{1.003750pt}%
\definecolor{currentstroke}{rgb}{0.121569,0.466667,0.705882}%
\pgfsetstrokecolor{currentstroke}%
\pgfsetstrokeopacity{0.948801}%
\pgfsetdash{}{0pt}%
\pgfpathmoveto{\pgfqpoint{1.759662in}{1.694139in}}%
\pgfpathcurveto{\pgfqpoint{1.767898in}{1.694139in}}{\pgfqpoint{1.775798in}{1.697411in}}{\pgfqpoint{1.781622in}{1.703235in}}%
\pgfpathcurveto{\pgfqpoint{1.787446in}{1.709059in}}{\pgfqpoint{1.790719in}{1.716959in}}{\pgfqpoint{1.790719in}{1.725196in}}%
\pgfpathcurveto{\pgfqpoint{1.790719in}{1.733432in}}{\pgfqpoint{1.787446in}{1.741332in}}{\pgfqpoint{1.781622in}{1.747156in}}%
\pgfpathcurveto{\pgfqpoint{1.775798in}{1.752980in}}{\pgfqpoint{1.767898in}{1.756252in}}{\pgfqpoint{1.759662in}{1.756252in}}%
\pgfpathcurveto{\pgfqpoint{1.751426in}{1.756252in}}{\pgfqpoint{1.743526in}{1.752980in}}{\pgfqpoint{1.737702in}{1.747156in}}%
\pgfpathcurveto{\pgfqpoint{1.731878in}{1.741332in}}{\pgfqpoint{1.728606in}{1.733432in}}{\pgfqpoint{1.728606in}{1.725196in}}%
\pgfpathcurveto{\pgfqpoint{1.728606in}{1.716959in}}{\pgfqpoint{1.731878in}{1.709059in}}{\pgfqpoint{1.737702in}{1.703235in}}%
\pgfpathcurveto{\pgfqpoint{1.743526in}{1.697411in}}{\pgfqpoint{1.751426in}{1.694139in}}{\pgfqpoint{1.759662in}{1.694139in}}%
\pgfpathclose%
\pgfusepath{stroke,fill}%
\end{pgfscope}%
\begin{pgfscope}%
\pgfpathrectangle{\pgfqpoint{0.100000in}{0.212622in}}{\pgfqpoint{3.696000in}{3.696000in}}%
\pgfusepath{clip}%
\pgfsetbuttcap%
\pgfsetroundjoin%
\definecolor{currentfill}{rgb}{0.121569,0.466667,0.705882}%
\pgfsetfillcolor{currentfill}%
\pgfsetfillopacity{0.956741}%
\pgfsetlinewidth{1.003750pt}%
\definecolor{currentstroke}{rgb}{0.121569,0.466667,0.705882}%
\pgfsetstrokecolor{currentstroke}%
\pgfsetstrokeopacity{0.956741}%
\pgfsetdash{}{0pt}%
\pgfpathmoveto{\pgfqpoint{1.782897in}{1.664376in}}%
\pgfpathcurveto{\pgfqpoint{1.791134in}{1.664376in}}{\pgfqpoint{1.799034in}{1.667648in}}{\pgfqpoint{1.804857in}{1.673472in}}%
\pgfpathcurveto{\pgfqpoint{1.810681in}{1.679296in}}{\pgfqpoint{1.813954in}{1.687196in}}{\pgfqpoint{1.813954in}{1.695432in}}%
\pgfpathcurveto{\pgfqpoint{1.813954in}{1.703669in}}{\pgfqpoint{1.810681in}{1.711569in}}{\pgfqpoint{1.804857in}{1.717393in}}%
\pgfpathcurveto{\pgfqpoint{1.799034in}{1.723217in}}{\pgfqpoint{1.791134in}{1.726489in}}{\pgfqpoint{1.782897in}{1.726489in}}%
\pgfpathcurveto{\pgfqpoint{1.774661in}{1.726489in}}{\pgfqpoint{1.766761in}{1.723217in}}{\pgfqpoint{1.760937in}{1.717393in}}%
\pgfpathcurveto{\pgfqpoint{1.755113in}{1.711569in}}{\pgfqpoint{1.751841in}{1.703669in}}{\pgfqpoint{1.751841in}{1.695432in}}%
\pgfpathcurveto{\pgfqpoint{1.751841in}{1.687196in}}{\pgfqpoint{1.755113in}{1.679296in}}{\pgfqpoint{1.760937in}{1.673472in}}%
\pgfpathcurveto{\pgfqpoint{1.766761in}{1.667648in}}{\pgfqpoint{1.774661in}{1.664376in}}{\pgfqpoint{1.782897in}{1.664376in}}%
\pgfpathclose%
\pgfusepath{stroke,fill}%
\end{pgfscope}%
\begin{pgfscope}%
\pgfpathrectangle{\pgfqpoint{0.100000in}{0.212622in}}{\pgfqpoint{3.696000in}{3.696000in}}%
\pgfusepath{clip}%
\pgfsetbuttcap%
\pgfsetroundjoin%
\definecolor{currentfill}{rgb}{0.121569,0.466667,0.705882}%
\pgfsetfillcolor{currentfill}%
\pgfsetfillopacity{0.959930}%
\pgfsetlinewidth{1.003750pt}%
\definecolor{currentstroke}{rgb}{0.121569,0.466667,0.705882}%
\pgfsetstrokecolor{currentstroke}%
\pgfsetstrokeopacity{0.959930}%
\pgfsetdash{}{0pt}%
\pgfpathmoveto{\pgfqpoint{1.830401in}{1.678137in}}%
\pgfpathcurveto{\pgfqpoint{1.838638in}{1.678137in}}{\pgfqpoint{1.846538in}{1.681409in}}{\pgfqpoint{1.852362in}{1.687233in}}%
\pgfpathcurveto{\pgfqpoint{1.858185in}{1.693057in}}{\pgfqpoint{1.861458in}{1.700957in}}{\pgfqpoint{1.861458in}{1.709193in}}%
\pgfpathcurveto{\pgfqpoint{1.861458in}{1.717430in}}{\pgfqpoint{1.858185in}{1.725330in}}{\pgfqpoint{1.852362in}{1.731154in}}%
\pgfpathcurveto{\pgfqpoint{1.846538in}{1.736978in}}{\pgfqpoint{1.838638in}{1.740250in}}{\pgfqpoint{1.830401in}{1.740250in}}%
\pgfpathcurveto{\pgfqpoint{1.822165in}{1.740250in}}{\pgfqpoint{1.814265in}{1.736978in}}{\pgfqpoint{1.808441in}{1.731154in}}%
\pgfpathcurveto{\pgfqpoint{1.802617in}{1.725330in}}{\pgfqpoint{1.799345in}{1.717430in}}{\pgfqpoint{1.799345in}{1.709193in}}%
\pgfpathcurveto{\pgfqpoint{1.799345in}{1.700957in}}{\pgfqpoint{1.802617in}{1.693057in}}{\pgfqpoint{1.808441in}{1.687233in}}%
\pgfpathcurveto{\pgfqpoint{1.814265in}{1.681409in}}{\pgfqpoint{1.822165in}{1.678137in}}{\pgfqpoint{1.830401in}{1.678137in}}%
\pgfpathclose%
\pgfusepath{stroke,fill}%
\end{pgfscope}%
\begin{pgfscope}%
\pgfpathrectangle{\pgfqpoint{0.100000in}{0.212622in}}{\pgfqpoint{3.696000in}{3.696000in}}%
\pgfusepath{clip}%
\pgfsetbuttcap%
\pgfsetroundjoin%
\definecolor{currentfill}{rgb}{0.121569,0.466667,0.705882}%
\pgfsetfillcolor{currentfill}%
\pgfsetfillopacity{0.963951}%
\pgfsetlinewidth{1.003750pt}%
\definecolor{currentstroke}{rgb}{0.121569,0.466667,0.705882}%
\pgfsetstrokecolor{currentstroke}%
\pgfsetstrokeopacity{0.963951}%
\pgfsetdash{}{0pt}%
\pgfpathmoveto{\pgfqpoint{1.848896in}{1.666654in}}%
\pgfpathcurveto{\pgfqpoint{1.857132in}{1.666654in}}{\pgfqpoint{1.865032in}{1.669926in}}{\pgfqpoint{1.870856in}{1.675750in}}%
\pgfpathcurveto{\pgfqpoint{1.876680in}{1.681574in}}{\pgfqpoint{1.879952in}{1.689474in}}{\pgfqpoint{1.879952in}{1.697711in}}%
\pgfpathcurveto{\pgfqpoint{1.879952in}{1.705947in}}{\pgfqpoint{1.876680in}{1.713847in}}{\pgfqpoint{1.870856in}{1.719671in}}%
\pgfpathcurveto{\pgfqpoint{1.865032in}{1.725495in}}{\pgfqpoint{1.857132in}{1.728767in}}{\pgfqpoint{1.848896in}{1.728767in}}%
\pgfpathcurveto{\pgfqpoint{1.840659in}{1.728767in}}{\pgfqpoint{1.832759in}{1.725495in}}{\pgfqpoint{1.826935in}{1.719671in}}%
\pgfpathcurveto{\pgfqpoint{1.821111in}{1.713847in}}{\pgfqpoint{1.817839in}{1.705947in}}{\pgfqpoint{1.817839in}{1.697711in}}%
\pgfpathcurveto{\pgfqpoint{1.817839in}{1.689474in}}{\pgfqpoint{1.821111in}{1.681574in}}{\pgfqpoint{1.826935in}{1.675750in}}%
\pgfpathcurveto{\pgfqpoint{1.832759in}{1.669926in}}{\pgfqpoint{1.840659in}{1.666654in}}{\pgfqpoint{1.848896in}{1.666654in}}%
\pgfpathclose%
\pgfusepath{stroke,fill}%
\end{pgfscope}%
\begin{pgfscope}%
\pgfpathrectangle{\pgfqpoint{0.100000in}{0.212622in}}{\pgfqpoint{3.696000in}{3.696000in}}%
\pgfusepath{clip}%
\pgfsetbuttcap%
\pgfsetroundjoin%
\definecolor{currentfill}{rgb}{0.121569,0.466667,0.705882}%
\pgfsetfillcolor{currentfill}%
\pgfsetfillopacity{0.966157}%
\pgfsetlinewidth{1.003750pt}%
\definecolor{currentstroke}{rgb}{0.121569,0.466667,0.705882}%
\pgfsetstrokecolor{currentstroke}%
\pgfsetstrokeopacity{0.966157}%
\pgfsetdash{}{0pt}%
\pgfpathmoveto{\pgfqpoint{1.859511in}{1.661928in}}%
\pgfpathcurveto{\pgfqpoint{1.867748in}{1.661928in}}{\pgfqpoint{1.875648in}{1.665201in}}{\pgfqpoint{1.881472in}{1.671025in}}%
\pgfpathcurveto{\pgfqpoint{1.887296in}{1.676848in}}{\pgfqpoint{1.890568in}{1.684749in}}{\pgfqpoint{1.890568in}{1.692985in}}%
\pgfpathcurveto{\pgfqpoint{1.890568in}{1.701221in}}{\pgfqpoint{1.887296in}{1.709121in}}{\pgfqpoint{1.881472in}{1.714945in}}%
\pgfpathcurveto{\pgfqpoint{1.875648in}{1.720769in}}{\pgfqpoint{1.867748in}{1.724041in}}{\pgfqpoint{1.859511in}{1.724041in}}%
\pgfpathcurveto{\pgfqpoint{1.851275in}{1.724041in}}{\pgfqpoint{1.843375in}{1.720769in}}{\pgfqpoint{1.837551in}{1.714945in}}%
\pgfpathcurveto{\pgfqpoint{1.831727in}{1.709121in}}{\pgfqpoint{1.828455in}{1.701221in}}{\pgfqpoint{1.828455in}{1.692985in}}%
\pgfpathcurveto{\pgfqpoint{1.828455in}{1.684749in}}{\pgfqpoint{1.831727in}{1.676848in}}{\pgfqpoint{1.837551in}{1.671025in}}%
\pgfpathcurveto{\pgfqpoint{1.843375in}{1.665201in}}{\pgfqpoint{1.851275in}{1.661928in}}{\pgfqpoint{1.859511in}{1.661928in}}%
\pgfpathclose%
\pgfusepath{stroke,fill}%
\end{pgfscope}%
\begin{pgfscope}%
\pgfpathrectangle{\pgfqpoint{0.100000in}{0.212622in}}{\pgfqpoint{3.696000in}{3.696000in}}%
\pgfusepath{clip}%
\pgfsetbuttcap%
\pgfsetroundjoin%
\definecolor{currentfill}{rgb}{0.121569,0.466667,0.705882}%
\pgfsetfillcolor{currentfill}%
\pgfsetfillopacity{0.969340}%
\pgfsetlinewidth{1.003750pt}%
\definecolor{currentstroke}{rgb}{0.121569,0.466667,0.705882}%
\pgfsetstrokecolor{currentstroke}%
\pgfsetstrokeopacity{0.969340}%
\pgfsetdash{}{0pt}%
\pgfpathmoveto{\pgfqpoint{1.874491in}{1.654632in}}%
\pgfpathcurveto{\pgfqpoint{1.882727in}{1.654632in}}{\pgfqpoint{1.890627in}{1.657904in}}{\pgfqpoint{1.896451in}{1.663728in}}%
\pgfpathcurveto{\pgfqpoint{1.902275in}{1.669552in}}{\pgfqpoint{1.905547in}{1.677452in}}{\pgfqpoint{1.905547in}{1.685688in}}%
\pgfpathcurveto{\pgfqpoint{1.905547in}{1.693925in}}{\pgfqpoint{1.902275in}{1.701825in}}{\pgfqpoint{1.896451in}{1.707649in}}%
\pgfpathcurveto{\pgfqpoint{1.890627in}{1.713473in}}{\pgfqpoint{1.882727in}{1.716745in}}{\pgfqpoint{1.874491in}{1.716745in}}%
\pgfpathcurveto{\pgfqpoint{1.866254in}{1.716745in}}{\pgfqpoint{1.858354in}{1.713473in}}{\pgfqpoint{1.852530in}{1.707649in}}%
\pgfpathcurveto{\pgfqpoint{1.846707in}{1.701825in}}{\pgfqpoint{1.843434in}{1.693925in}}{\pgfqpoint{1.843434in}{1.685688in}}%
\pgfpathcurveto{\pgfqpoint{1.843434in}{1.677452in}}{\pgfqpoint{1.846707in}{1.669552in}}{\pgfqpoint{1.852530in}{1.663728in}}%
\pgfpathcurveto{\pgfqpoint{1.858354in}{1.657904in}}{\pgfqpoint{1.866254in}{1.654632in}}{\pgfqpoint{1.874491in}{1.654632in}}%
\pgfpathclose%
\pgfusepath{stroke,fill}%
\end{pgfscope}%
\begin{pgfscope}%
\pgfpathrectangle{\pgfqpoint{0.100000in}{0.212622in}}{\pgfqpoint{3.696000in}{3.696000in}}%
\pgfusepath{clip}%
\pgfsetbuttcap%
\pgfsetroundjoin%
\definecolor{currentfill}{rgb}{0.121569,0.466667,0.705882}%
\pgfsetfillcolor{currentfill}%
\pgfsetfillopacity{0.970253}%
\pgfsetlinewidth{1.003750pt}%
\definecolor{currentstroke}{rgb}{0.121569,0.466667,0.705882}%
\pgfsetstrokecolor{currentstroke}%
\pgfsetstrokeopacity{0.970253}%
\pgfsetdash{}{0pt}%
\pgfpathmoveto{\pgfqpoint{1.884548in}{1.654167in}}%
\pgfpathcurveto{\pgfqpoint{1.892785in}{1.654167in}}{\pgfqpoint{1.900685in}{1.657440in}}{\pgfqpoint{1.906509in}{1.663264in}}%
\pgfpathcurveto{\pgfqpoint{1.912332in}{1.669087in}}{\pgfqpoint{1.915605in}{1.676988in}}{\pgfqpoint{1.915605in}{1.685224in}}%
\pgfpathcurveto{\pgfqpoint{1.915605in}{1.693460in}}{\pgfqpoint{1.912332in}{1.701360in}}{\pgfqpoint{1.906509in}{1.707184in}}%
\pgfpathcurveto{\pgfqpoint{1.900685in}{1.713008in}}{\pgfqpoint{1.892785in}{1.716280in}}{\pgfqpoint{1.884548in}{1.716280in}}%
\pgfpathcurveto{\pgfqpoint{1.876312in}{1.716280in}}{\pgfqpoint{1.868412in}{1.713008in}}{\pgfqpoint{1.862588in}{1.707184in}}%
\pgfpathcurveto{\pgfqpoint{1.856764in}{1.701360in}}{\pgfqpoint{1.853492in}{1.693460in}}{\pgfqpoint{1.853492in}{1.685224in}}%
\pgfpathcurveto{\pgfqpoint{1.853492in}{1.676988in}}{\pgfqpoint{1.856764in}{1.669087in}}{\pgfqpoint{1.862588in}{1.663264in}}%
\pgfpathcurveto{\pgfqpoint{1.868412in}{1.657440in}}{\pgfqpoint{1.876312in}{1.654167in}}{\pgfqpoint{1.884548in}{1.654167in}}%
\pgfpathclose%
\pgfusepath{stroke,fill}%
\end{pgfscope}%
\begin{pgfscope}%
\pgfpathrectangle{\pgfqpoint{0.100000in}{0.212622in}}{\pgfqpoint{3.696000in}{3.696000in}}%
\pgfusepath{clip}%
\pgfsetbuttcap%
\pgfsetroundjoin%
\definecolor{currentfill}{rgb}{0.121569,0.466667,0.705882}%
\pgfsetfillcolor{currentfill}%
\pgfsetfillopacity{0.970610}%
\pgfsetlinewidth{1.003750pt}%
\definecolor{currentstroke}{rgb}{0.121569,0.466667,0.705882}%
\pgfsetstrokecolor{currentstroke}%
\pgfsetstrokeopacity{0.970610}%
\pgfsetdash{}{0pt}%
\pgfpathmoveto{\pgfqpoint{2.004755in}{1.669731in}}%
\pgfpathcurveto{\pgfqpoint{2.012992in}{1.669731in}}{\pgfqpoint{2.020892in}{1.673003in}}{\pgfqpoint{2.026716in}{1.678827in}}%
\pgfpathcurveto{\pgfqpoint{2.032540in}{1.684651in}}{\pgfqpoint{2.035812in}{1.692551in}}{\pgfqpoint{2.035812in}{1.700787in}}%
\pgfpathcurveto{\pgfqpoint{2.035812in}{1.709024in}}{\pgfqpoint{2.032540in}{1.716924in}}{\pgfqpoint{2.026716in}{1.722748in}}%
\pgfpathcurveto{\pgfqpoint{2.020892in}{1.728572in}}{\pgfqpoint{2.012992in}{1.731844in}}{\pgfqpoint{2.004755in}{1.731844in}}%
\pgfpathcurveto{\pgfqpoint{1.996519in}{1.731844in}}{\pgfqpoint{1.988619in}{1.728572in}}{\pgfqpoint{1.982795in}{1.722748in}}%
\pgfpathcurveto{\pgfqpoint{1.976971in}{1.716924in}}{\pgfqpoint{1.973699in}{1.709024in}}{\pgfqpoint{1.973699in}{1.700787in}}%
\pgfpathcurveto{\pgfqpoint{1.973699in}{1.692551in}}{\pgfqpoint{1.976971in}{1.684651in}}{\pgfqpoint{1.982795in}{1.678827in}}%
\pgfpathcurveto{\pgfqpoint{1.988619in}{1.673003in}}{\pgfqpoint{1.996519in}{1.669731in}}{\pgfqpoint{2.004755in}{1.669731in}}%
\pgfpathclose%
\pgfusepath{stroke,fill}%
\end{pgfscope}%
\begin{pgfscope}%
\pgfpathrectangle{\pgfqpoint{0.100000in}{0.212622in}}{\pgfqpoint{3.696000in}{3.696000in}}%
\pgfusepath{clip}%
\pgfsetbuttcap%
\pgfsetroundjoin%
\definecolor{currentfill}{rgb}{0.121569,0.466667,0.705882}%
\pgfsetfillcolor{currentfill}%
\pgfsetfillopacity{0.970925}%
\pgfsetlinewidth{1.003750pt}%
\definecolor{currentstroke}{rgb}{0.121569,0.466667,0.705882}%
\pgfsetstrokecolor{currentstroke}%
\pgfsetstrokeopacity{0.970925}%
\pgfsetdash{}{0pt}%
\pgfpathmoveto{\pgfqpoint{1.979985in}{1.666595in}}%
\pgfpathcurveto{\pgfqpoint{1.988222in}{1.666595in}}{\pgfqpoint{1.996122in}{1.669868in}}{\pgfqpoint{2.001946in}{1.675691in}}%
\pgfpathcurveto{\pgfqpoint{2.007770in}{1.681515in}}{\pgfqpoint{2.011042in}{1.689415in}}{\pgfqpoint{2.011042in}{1.697652in}}%
\pgfpathcurveto{\pgfqpoint{2.011042in}{1.705888in}}{\pgfqpoint{2.007770in}{1.713788in}}{\pgfqpoint{2.001946in}{1.719612in}}%
\pgfpathcurveto{\pgfqpoint{1.996122in}{1.725436in}}{\pgfqpoint{1.988222in}{1.728708in}}{\pgfqpoint{1.979985in}{1.728708in}}%
\pgfpathcurveto{\pgfqpoint{1.971749in}{1.728708in}}{\pgfqpoint{1.963849in}{1.725436in}}{\pgfqpoint{1.958025in}{1.719612in}}%
\pgfpathcurveto{\pgfqpoint{1.952201in}{1.713788in}}{\pgfqpoint{1.948929in}{1.705888in}}{\pgfqpoint{1.948929in}{1.697652in}}%
\pgfpathcurveto{\pgfqpoint{1.948929in}{1.689415in}}{\pgfqpoint{1.952201in}{1.681515in}}{\pgfqpoint{1.958025in}{1.675691in}}%
\pgfpathcurveto{\pgfqpoint{1.963849in}{1.669868in}}{\pgfqpoint{1.971749in}{1.666595in}}{\pgfqpoint{1.979985in}{1.666595in}}%
\pgfpathclose%
\pgfusepath{stroke,fill}%
\end{pgfscope}%
\begin{pgfscope}%
\pgfpathrectangle{\pgfqpoint{0.100000in}{0.212622in}}{\pgfqpoint{3.696000in}{3.696000in}}%
\pgfusepath{clip}%
\pgfsetbuttcap%
\pgfsetroundjoin%
\definecolor{currentfill}{rgb}{0.121569,0.466667,0.705882}%
\pgfsetfillcolor{currentfill}%
\pgfsetfillopacity{0.971179}%
\pgfsetlinewidth{1.003750pt}%
\definecolor{currentstroke}{rgb}{0.121569,0.466667,0.705882}%
\pgfsetstrokecolor{currentstroke}%
\pgfsetstrokeopacity{0.971179}%
\pgfsetdash{}{0pt}%
\pgfpathmoveto{\pgfqpoint{1.897742in}{1.653478in}}%
\pgfpathcurveto{\pgfqpoint{1.905978in}{1.653478in}}{\pgfqpoint{1.913878in}{1.656751in}}{\pgfqpoint{1.919702in}{1.662575in}}%
\pgfpathcurveto{\pgfqpoint{1.925526in}{1.668398in}}{\pgfqpoint{1.928798in}{1.676299in}}{\pgfqpoint{1.928798in}{1.684535in}}%
\pgfpathcurveto{\pgfqpoint{1.928798in}{1.692771in}}{\pgfqpoint{1.925526in}{1.700671in}}{\pgfqpoint{1.919702in}{1.706495in}}%
\pgfpathcurveto{\pgfqpoint{1.913878in}{1.712319in}}{\pgfqpoint{1.905978in}{1.715591in}}{\pgfqpoint{1.897742in}{1.715591in}}%
\pgfpathcurveto{\pgfqpoint{1.889506in}{1.715591in}}{\pgfqpoint{1.881606in}{1.712319in}}{\pgfqpoint{1.875782in}{1.706495in}}%
\pgfpathcurveto{\pgfqpoint{1.869958in}{1.700671in}}{\pgfqpoint{1.866685in}{1.692771in}}{\pgfqpoint{1.866685in}{1.684535in}}%
\pgfpathcurveto{\pgfqpoint{1.866685in}{1.676299in}}{\pgfqpoint{1.869958in}{1.668398in}}{\pgfqpoint{1.875782in}{1.662575in}}%
\pgfpathcurveto{\pgfqpoint{1.881606in}{1.656751in}}{\pgfqpoint{1.889506in}{1.653478in}}{\pgfqpoint{1.897742in}{1.653478in}}%
\pgfpathclose%
\pgfusepath{stroke,fill}%
\end{pgfscope}%
\begin{pgfscope}%
\pgfpathrectangle{\pgfqpoint{0.100000in}{0.212622in}}{\pgfqpoint{3.696000in}{3.696000in}}%
\pgfusepath{clip}%
\pgfsetbuttcap%
\pgfsetroundjoin%
\definecolor{currentfill}{rgb}{0.121569,0.466667,0.705882}%
\pgfsetfillcolor{currentfill}%
\pgfsetfillopacity{0.971464}%
\pgfsetlinewidth{1.003750pt}%
\definecolor{currentstroke}{rgb}{0.121569,0.466667,0.705882}%
\pgfsetstrokecolor{currentstroke}%
\pgfsetstrokeopacity{0.971464}%
\pgfsetdash{}{0pt}%
\pgfpathmoveto{\pgfqpoint{2.030150in}{1.668220in}}%
\pgfpathcurveto{\pgfqpoint{2.038386in}{1.668220in}}{\pgfqpoint{2.046286in}{1.671493in}}{\pgfqpoint{2.052110in}{1.677317in}}%
\pgfpathcurveto{\pgfqpoint{2.057934in}{1.683141in}}{\pgfqpoint{2.061206in}{1.691041in}}{\pgfqpoint{2.061206in}{1.699277in}}%
\pgfpathcurveto{\pgfqpoint{2.061206in}{1.707513in}}{\pgfqpoint{2.057934in}{1.715413in}}{\pgfqpoint{2.052110in}{1.721237in}}%
\pgfpathcurveto{\pgfqpoint{2.046286in}{1.727061in}}{\pgfqpoint{2.038386in}{1.730333in}}{\pgfqpoint{2.030150in}{1.730333in}}%
\pgfpathcurveto{\pgfqpoint{2.021914in}{1.730333in}}{\pgfqpoint{2.014014in}{1.727061in}}{\pgfqpoint{2.008190in}{1.721237in}}%
\pgfpathcurveto{\pgfqpoint{2.002366in}{1.715413in}}{\pgfqpoint{1.999093in}{1.707513in}}{\pgfqpoint{1.999093in}{1.699277in}}%
\pgfpathcurveto{\pgfqpoint{1.999093in}{1.691041in}}{\pgfqpoint{2.002366in}{1.683141in}}{\pgfqpoint{2.008190in}{1.677317in}}%
\pgfpathcurveto{\pgfqpoint{2.014014in}{1.671493in}}{\pgfqpoint{2.021914in}{1.668220in}}{\pgfqpoint{2.030150in}{1.668220in}}%
\pgfpathclose%
\pgfusepath{stroke,fill}%
\end{pgfscope}%
\begin{pgfscope}%
\pgfpathrectangle{\pgfqpoint{0.100000in}{0.212622in}}{\pgfqpoint{3.696000in}{3.696000in}}%
\pgfusepath{clip}%
\pgfsetbuttcap%
\pgfsetroundjoin%
\definecolor{currentfill}{rgb}{0.121569,0.466667,0.705882}%
\pgfsetfillcolor{currentfill}%
\pgfsetfillopacity{0.971522}%
\pgfsetlinewidth{1.003750pt}%
\definecolor{currentstroke}{rgb}{0.121569,0.466667,0.705882}%
\pgfsetstrokecolor{currentstroke}%
\pgfsetstrokeopacity{0.971522}%
\pgfsetdash{}{0pt}%
\pgfpathmoveto{\pgfqpoint{1.957661in}{1.661705in}}%
\pgfpathcurveto{\pgfqpoint{1.965897in}{1.661705in}}{\pgfqpoint{1.973797in}{1.664977in}}{\pgfqpoint{1.979621in}{1.670801in}}%
\pgfpathcurveto{\pgfqpoint{1.985445in}{1.676625in}}{\pgfqpoint{1.988717in}{1.684525in}}{\pgfqpoint{1.988717in}{1.692761in}}%
\pgfpathcurveto{\pgfqpoint{1.988717in}{1.700998in}}{\pgfqpoint{1.985445in}{1.708898in}}{\pgfqpoint{1.979621in}{1.714722in}}%
\pgfpathcurveto{\pgfqpoint{1.973797in}{1.720545in}}{\pgfqpoint{1.965897in}{1.723818in}}{\pgfqpoint{1.957661in}{1.723818in}}%
\pgfpathcurveto{\pgfqpoint{1.949424in}{1.723818in}}{\pgfqpoint{1.941524in}{1.720545in}}{\pgfqpoint{1.935701in}{1.714722in}}%
\pgfpathcurveto{\pgfqpoint{1.929877in}{1.708898in}}{\pgfqpoint{1.926604in}{1.700998in}}{\pgfqpoint{1.926604in}{1.692761in}}%
\pgfpathcurveto{\pgfqpoint{1.926604in}{1.684525in}}{\pgfqpoint{1.929877in}{1.676625in}}{\pgfqpoint{1.935701in}{1.670801in}}%
\pgfpathcurveto{\pgfqpoint{1.941524in}{1.664977in}}{\pgfqpoint{1.949424in}{1.661705in}}{\pgfqpoint{1.957661in}{1.661705in}}%
\pgfpathclose%
\pgfusepath{stroke,fill}%
\end{pgfscope}%
\begin{pgfscope}%
\pgfpathrectangle{\pgfqpoint{0.100000in}{0.212622in}}{\pgfqpoint{3.696000in}{3.696000in}}%
\pgfusepath{clip}%
\pgfsetbuttcap%
\pgfsetroundjoin%
\definecolor{currentfill}{rgb}{0.121569,0.466667,0.705882}%
\pgfsetfillcolor{currentfill}%
\pgfsetfillopacity{0.971625}%
\pgfsetlinewidth{1.003750pt}%
\definecolor{currentstroke}{rgb}{0.121569,0.466667,0.705882}%
\pgfsetstrokecolor{currentstroke}%
\pgfsetstrokeopacity{0.971625}%
\pgfsetdash{}{0pt}%
\pgfpathmoveto{\pgfqpoint{2.315921in}{1.682647in}}%
\pgfpathcurveto{\pgfqpoint{2.324158in}{1.682647in}}{\pgfqpoint{2.332058in}{1.685919in}}{\pgfqpoint{2.337882in}{1.691743in}}%
\pgfpathcurveto{\pgfqpoint{2.343706in}{1.697567in}}{\pgfqpoint{2.346978in}{1.705467in}}{\pgfqpoint{2.346978in}{1.713703in}}%
\pgfpathcurveto{\pgfqpoint{2.346978in}{1.721940in}}{\pgfqpoint{2.343706in}{1.729840in}}{\pgfqpoint{2.337882in}{1.735664in}}%
\pgfpathcurveto{\pgfqpoint{2.332058in}{1.741487in}}{\pgfqpoint{2.324158in}{1.744760in}}{\pgfqpoint{2.315921in}{1.744760in}}%
\pgfpathcurveto{\pgfqpoint{2.307685in}{1.744760in}}{\pgfqpoint{2.299785in}{1.741487in}}{\pgfqpoint{2.293961in}{1.735664in}}%
\pgfpathcurveto{\pgfqpoint{2.288137in}{1.729840in}}{\pgfqpoint{2.284865in}{1.721940in}}{\pgfqpoint{2.284865in}{1.713703in}}%
\pgfpathcurveto{\pgfqpoint{2.284865in}{1.705467in}}{\pgfqpoint{2.288137in}{1.697567in}}{\pgfqpoint{2.293961in}{1.691743in}}%
\pgfpathcurveto{\pgfqpoint{2.299785in}{1.685919in}}{\pgfqpoint{2.307685in}{1.682647in}}{\pgfqpoint{2.315921in}{1.682647in}}%
\pgfpathclose%
\pgfusepath{stroke,fill}%
\end{pgfscope}%
\begin{pgfscope}%
\pgfpathrectangle{\pgfqpoint{0.100000in}{0.212622in}}{\pgfqpoint{3.696000in}{3.696000in}}%
\pgfusepath{clip}%
\pgfsetbuttcap%
\pgfsetroundjoin%
\definecolor{currentfill}{rgb}{0.121569,0.466667,0.705882}%
\pgfsetfillcolor{currentfill}%
\pgfsetfillopacity{0.971650}%
\pgfsetlinewidth{1.003750pt}%
\definecolor{currentstroke}{rgb}{0.121569,0.466667,0.705882}%
\pgfsetstrokecolor{currentstroke}%
\pgfsetstrokeopacity{0.971650}%
\pgfsetdash{}{0pt}%
\pgfpathmoveto{\pgfqpoint{1.926554in}{1.656342in}}%
\pgfpathcurveto{\pgfqpoint{1.934791in}{1.656342in}}{\pgfqpoint{1.942691in}{1.659614in}}{\pgfqpoint{1.948515in}{1.665438in}}%
\pgfpathcurveto{\pgfqpoint{1.954339in}{1.671262in}}{\pgfqpoint{1.957611in}{1.679162in}}{\pgfqpoint{1.957611in}{1.687398in}}%
\pgfpathcurveto{\pgfqpoint{1.957611in}{1.695635in}}{\pgfqpoint{1.954339in}{1.703535in}}{\pgfqpoint{1.948515in}{1.709358in}}%
\pgfpathcurveto{\pgfqpoint{1.942691in}{1.715182in}}{\pgfqpoint{1.934791in}{1.718455in}}{\pgfqpoint{1.926554in}{1.718455in}}%
\pgfpathcurveto{\pgfqpoint{1.918318in}{1.718455in}}{\pgfqpoint{1.910418in}{1.715182in}}{\pgfqpoint{1.904594in}{1.709358in}}%
\pgfpathcurveto{\pgfqpoint{1.898770in}{1.703535in}}{\pgfqpoint{1.895498in}{1.695635in}}{\pgfqpoint{1.895498in}{1.687398in}}%
\pgfpathcurveto{\pgfqpoint{1.895498in}{1.679162in}}{\pgfqpoint{1.898770in}{1.671262in}}{\pgfqpoint{1.904594in}{1.665438in}}%
\pgfpathcurveto{\pgfqpoint{1.910418in}{1.659614in}}{\pgfqpoint{1.918318in}{1.656342in}}{\pgfqpoint{1.926554in}{1.656342in}}%
\pgfpathclose%
\pgfusepath{stroke,fill}%
\end{pgfscope}%
\begin{pgfscope}%
\pgfpathrectangle{\pgfqpoint{0.100000in}{0.212622in}}{\pgfqpoint{3.696000in}{3.696000in}}%
\pgfusepath{clip}%
\pgfsetbuttcap%
\pgfsetroundjoin%
\definecolor{currentfill}{rgb}{0.121569,0.466667,0.705882}%
\pgfsetfillcolor{currentfill}%
\pgfsetfillopacity{0.971670}%
\pgfsetlinewidth{1.003750pt}%
\definecolor{currentstroke}{rgb}{0.121569,0.466667,0.705882}%
\pgfsetstrokecolor{currentstroke}%
\pgfsetstrokeopacity{0.971670}%
\pgfsetdash{}{0pt}%
\pgfpathmoveto{\pgfqpoint{1.915969in}{1.654725in}}%
\pgfpathcurveto{\pgfqpoint{1.924206in}{1.654725in}}{\pgfqpoint{1.932106in}{1.657997in}}{\pgfqpoint{1.937929in}{1.663821in}}%
\pgfpathcurveto{\pgfqpoint{1.943753in}{1.669645in}}{\pgfqpoint{1.947026in}{1.677545in}}{\pgfqpoint{1.947026in}{1.685782in}}%
\pgfpathcurveto{\pgfqpoint{1.947026in}{1.694018in}}{\pgfqpoint{1.943753in}{1.701918in}}{\pgfqpoint{1.937929in}{1.707742in}}%
\pgfpathcurveto{\pgfqpoint{1.932106in}{1.713566in}}{\pgfqpoint{1.924206in}{1.716838in}}{\pgfqpoint{1.915969in}{1.716838in}}%
\pgfpathcurveto{\pgfqpoint{1.907733in}{1.716838in}}{\pgfqpoint{1.899833in}{1.713566in}}{\pgfqpoint{1.894009in}{1.707742in}}%
\pgfpathcurveto{\pgfqpoint{1.888185in}{1.701918in}}{\pgfqpoint{1.884913in}{1.694018in}}{\pgfqpoint{1.884913in}{1.685782in}}%
\pgfpathcurveto{\pgfqpoint{1.884913in}{1.677545in}}{\pgfqpoint{1.888185in}{1.669645in}}{\pgfqpoint{1.894009in}{1.663821in}}%
\pgfpathcurveto{\pgfqpoint{1.899833in}{1.657997in}}{\pgfqpoint{1.907733in}{1.654725in}}{\pgfqpoint{1.915969in}{1.654725in}}%
\pgfpathclose%
\pgfusepath{stroke,fill}%
\end{pgfscope}%
\begin{pgfscope}%
\pgfpathrectangle{\pgfqpoint{0.100000in}{0.212622in}}{\pgfqpoint{3.696000in}{3.696000in}}%
\pgfusepath{clip}%
\pgfsetbuttcap%
\pgfsetroundjoin%
\definecolor{currentfill}{rgb}{0.121569,0.466667,0.705882}%
\pgfsetfillcolor{currentfill}%
\pgfsetfillopacity{0.971911}%
\pgfsetlinewidth{1.003750pt}%
\definecolor{currentstroke}{rgb}{0.121569,0.466667,0.705882}%
\pgfsetstrokecolor{currentstroke}%
\pgfsetstrokeopacity{0.971911}%
\pgfsetdash{}{0pt}%
\pgfpathmoveto{\pgfqpoint{1.939673in}{1.657274in}}%
\pgfpathcurveto{\pgfqpoint{1.947910in}{1.657274in}}{\pgfqpoint{1.955810in}{1.660547in}}{\pgfqpoint{1.961634in}{1.666371in}}%
\pgfpathcurveto{\pgfqpoint{1.967458in}{1.672195in}}{\pgfqpoint{1.970730in}{1.680095in}}{\pgfqpoint{1.970730in}{1.688331in}}%
\pgfpathcurveto{\pgfqpoint{1.970730in}{1.696567in}}{\pgfqpoint{1.967458in}{1.704467in}}{\pgfqpoint{1.961634in}{1.710291in}}%
\pgfpathcurveto{\pgfqpoint{1.955810in}{1.716115in}}{\pgfqpoint{1.947910in}{1.719387in}}{\pgfqpoint{1.939673in}{1.719387in}}%
\pgfpathcurveto{\pgfqpoint{1.931437in}{1.719387in}}{\pgfqpoint{1.923537in}{1.716115in}}{\pgfqpoint{1.917713in}{1.710291in}}%
\pgfpathcurveto{\pgfqpoint{1.911889in}{1.704467in}}{\pgfqpoint{1.908617in}{1.696567in}}{\pgfqpoint{1.908617in}{1.688331in}}%
\pgfpathcurveto{\pgfqpoint{1.908617in}{1.680095in}}{\pgfqpoint{1.911889in}{1.672195in}}{\pgfqpoint{1.917713in}{1.666371in}}%
\pgfpathcurveto{\pgfqpoint{1.923537in}{1.660547in}}{\pgfqpoint{1.931437in}{1.657274in}}{\pgfqpoint{1.939673in}{1.657274in}}%
\pgfpathclose%
\pgfusepath{stroke,fill}%
\end{pgfscope}%
\begin{pgfscope}%
\pgfpathrectangle{\pgfqpoint{0.100000in}{0.212622in}}{\pgfqpoint{3.696000in}{3.696000in}}%
\pgfusepath{clip}%
\pgfsetbuttcap%
\pgfsetroundjoin%
\definecolor{currentfill}{rgb}{0.121569,0.466667,0.705882}%
\pgfsetfillcolor{currentfill}%
\pgfsetfillopacity{0.971999}%
\pgfsetlinewidth{1.003750pt}%
\definecolor{currentstroke}{rgb}{0.121569,0.466667,0.705882}%
\pgfsetstrokecolor{currentstroke}%
\pgfsetstrokeopacity{0.971999}%
\pgfsetdash{}{0pt}%
\pgfpathmoveto{\pgfqpoint{2.131797in}{1.676174in}}%
\pgfpathcurveto{\pgfqpoint{2.140033in}{1.676174in}}{\pgfqpoint{2.147933in}{1.679447in}}{\pgfqpoint{2.153757in}{1.685270in}}%
\pgfpathcurveto{\pgfqpoint{2.159581in}{1.691094in}}{\pgfqpoint{2.162853in}{1.698994in}}{\pgfqpoint{2.162853in}{1.707231in}}%
\pgfpathcurveto{\pgfqpoint{2.162853in}{1.715467in}}{\pgfqpoint{2.159581in}{1.723367in}}{\pgfqpoint{2.153757in}{1.729191in}}%
\pgfpathcurveto{\pgfqpoint{2.147933in}{1.735015in}}{\pgfqpoint{2.140033in}{1.738287in}}{\pgfqpoint{2.131797in}{1.738287in}}%
\pgfpathcurveto{\pgfqpoint{2.123560in}{1.738287in}}{\pgfqpoint{2.115660in}{1.735015in}}{\pgfqpoint{2.109836in}{1.729191in}}%
\pgfpathcurveto{\pgfqpoint{2.104012in}{1.723367in}}{\pgfqpoint{2.100740in}{1.715467in}}{\pgfqpoint{2.100740in}{1.707231in}}%
\pgfpathcurveto{\pgfqpoint{2.100740in}{1.698994in}}{\pgfqpoint{2.104012in}{1.691094in}}{\pgfqpoint{2.109836in}{1.685270in}}%
\pgfpathcurveto{\pgfqpoint{2.115660in}{1.679447in}}{\pgfqpoint{2.123560in}{1.676174in}}{\pgfqpoint{2.131797in}{1.676174in}}%
\pgfpathclose%
\pgfusepath{stroke,fill}%
\end{pgfscope}%
\begin{pgfscope}%
\pgfpathrectangle{\pgfqpoint{0.100000in}{0.212622in}}{\pgfqpoint{3.696000in}{3.696000in}}%
\pgfusepath{clip}%
\pgfsetbuttcap%
\pgfsetroundjoin%
\definecolor{currentfill}{rgb}{0.121569,0.466667,0.705882}%
\pgfsetfillcolor{currentfill}%
\pgfsetfillopacity{0.972435}%
\pgfsetlinewidth{1.003750pt}%
\definecolor{currentstroke}{rgb}{0.121569,0.466667,0.705882}%
\pgfsetstrokecolor{currentstroke}%
\pgfsetstrokeopacity{0.972435}%
\pgfsetdash{}{0pt}%
\pgfpathmoveto{\pgfqpoint{2.263361in}{1.677777in}}%
\pgfpathcurveto{\pgfqpoint{2.271598in}{1.677777in}}{\pgfqpoint{2.279498in}{1.681049in}}{\pgfqpoint{2.285322in}{1.686873in}}%
\pgfpathcurveto{\pgfqpoint{2.291145in}{1.692697in}}{\pgfqpoint{2.294418in}{1.700597in}}{\pgfqpoint{2.294418in}{1.708834in}}%
\pgfpathcurveto{\pgfqpoint{2.294418in}{1.717070in}}{\pgfqpoint{2.291145in}{1.724970in}}{\pgfqpoint{2.285322in}{1.730794in}}%
\pgfpathcurveto{\pgfqpoint{2.279498in}{1.736618in}}{\pgfqpoint{2.271598in}{1.739890in}}{\pgfqpoint{2.263361in}{1.739890in}}%
\pgfpathcurveto{\pgfqpoint{2.255125in}{1.739890in}}{\pgfqpoint{2.247225in}{1.736618in}}{\pgfqpoint{2.241401in}{1.730794in}}%
\pgfpathcurveto{\pgfqpoint{2.235577in}{1.724970in}}{\pgfqpoint{2.232305in}{1.717070in}}{\pgfqpoint{2.232305in}{1.708834in}}%
\pgfpathcurveto{\pgfqpoint{2.232305in}{1.700597in}}{\pgfqpoint{2.235577in}{1.692697in}}{\pgfqpoint{2.241401in}{1.686873in}}%
\pgfpathcurveto{\pgfqpoint{2.247225in}{1.681049in}}{\pgfqpoint{2.255125in}{1.677777in}}{\pgfqpoint{2.263361in}{1.677777in}}%
\pgfpathclose%
\pgfusepath{stroke,fill}%
\end{pgfscope}%
\begin{pgfscope}%
\pgfpathrectangle{\pgfqpoint{0.100000in}{0.212622in}}{\pgfqpoint{3.696000in}{3.696000in}}%
\pgfusepath{clip}%
\pgfsetbuttcap%
\pgfsetroundjoin%
\definecolor{currentfill}{rgb}{0.121569,0.466667,0.705882}%
\pgfsetfillcolor{currentfill}%
\pgfsetfillopacity{0.972471}%
\pgfsetlinewidth{1.003750pt}%
\definecolor{currentstroke}{rgb}{0.121569,0.466667,0.705882}%
\pgfsetstrokecolor{currentstroke}%
\pgfsetstrokeopacity{0.972471}%
\pgfsetdash{}{0pt}%
\pgfpathmoveto{\pgfqpoint{2.058584in}{1.665669in}}%
\pgfpathcurveto{\pgfqpoint{2.066821in}{1.665669in}}{\pgfqpoint{2.074721in}{1.668942in}}{\pgfqpoint{2.080545in}{1.674766in}}%
\pgfpathcurveto{\pgfqpoint{2.086369in}{1.680589in}}{\pgfqpoint{2.089641in}{1.688490in}}{\pgfqpoint{2.089641in}{1.696726in}}%
\pgfpathcurveto{\pgfqpoint{2.089641in}{1.704962in}}{\pgfqpoint{2.086369in}{1.712862in}}{\pgfqpoint{2.080545in}{1.718686in}}%
\pgfpathcurveto{\pgfqpoint{2.074721in}{1.724510in}}{\pgfqpoint{2.066821in}{1.727782in}}{\pgfqpoint{2.058584in}{1.727782in}}%
\pgfpathcurveto{\pgfqpoint{2.050348in}{1.727782in}}{\pgfqpoint{2.042448in}{1.724510in}}{\pgfqpoint{2.036624in}{1.718686in}}%
\pgfpathcurveto{\pgfqpoint{2.030800in}{1.712862in}}{\pgfqpoint{2.027528in}{1.704962in}}{\pgfqpoint{2.027528in}{1.696726in}}%
\pgfpathcurveto{\pgfqpoint{2.027528in}{1.688490in}}{\pgfqpoint{2.030800in}{1.680589in}}{\pgfqpoint{2.036624in}{1.674766in}}%
\pgfpathcurveto{\pgfqpoint{2.042448in}{1.668942in}}{\pgfqpoint{2.050348in}{1.665669in}}{\pgfqpoint{2.058584in}{1.665669in}}%
\pgfpathclose%
\pgfusepath{stroke,fill}%
\end{pgfscope}%
\begin{pgfscope}%
\pgfpathrectangle{\pgfqpoint{0.100000in}{0.212622in}}{\pgfqpoint{3.696000in}{3.696000in}}%
\pgfusepath{clip}%
\pgfsetbuttcap%
\pgfsetroundjoin%
\definecolor{currentfill}{rgb}{0.121569,0.466667,0.705882}%
\pgfsetfillcolor{currentfill}%
\pgfsetfillopacity{0.972740}%
\pgfsetlinewidth{1.003750pt}%
\definecolor{currentstroke}{rgb}{0.121569,0.466667,0.705882}%
\pgfsetstrokecolor{currentstroke}%
\pgfsetstrokeopacity{0.972740}%
\pgfsetdash{}{0pt}%
\pgfpathmoveto{\pgfqpoint{2.091893in}{1.667640in}}%
\pgfpathcurveto{\pgfqpoint{2.100129in}{1.667640in}}{\pgfqpoint{2.108029in}{1.670912in}}{\pgfqpoint{2.113853in}{1.676736in}}%
\pgfpathcurveto{\pgfqpoint{2.119677in}{1.682560in}}{\pgfqpoint{2.122950in}{1.690460in}}{\pgfqpoint{2.122950in}{1.698696in}}%
\pgfpathcurveto{\pgfqpoint{2.122950in}{1.706933in}}{\pgfqpoint{2.119677in}{1.714833in}}{\pgfqpoint{2.113853in}{1.720657in}}%
\pgfpathcurveto{\pgfqpoint{2.108029in}{1.726480in}}{\pgfqpoint{2.100129in}{1.729753in}}{\pgfqpoint{2.091893in}{1.729753in}}%
\pgfpathcurveto{\pgfqpoint{2.083657in}{1.729753in}}{\pgfqpoint{2.075757in}{1.726480in}}{\pgfqpoint{2.069933in}{1.720657in}}%
\pgfpathcurveto{\pgfqpoint{2.064109in}{1.714833in}}{\pgfqpoint{2.060837in}{1.706933in}}{\pgfqpoint{2.060837in}{1.698696in}}%
\pgfpathcurveto{\pgfqpoint{2.060837in}{1.690460in}}{\pgfqpoint{2.064109in}{1.682560in}}{\pgfqpoint{2.069933in}{1.676736in}}%
\pgfpathcurveto{\pgfqpoint{2.075757in}{1.670912in}}{\pgfqpoint{2.083657in}{1.667640in}}{\pgfqpoint{2.091893in}{1.667640in}}%
\pgfpathclose%
\pgfusepath{stroke,fill}%
\end{pgfscope}%
\begin{pgfscope}%
\pgfpathrectangle{\pgfqpoint{0.100000in}{0.212622in}}{\pgfqpoint{3.696000in}{3.696000in}}%
\pgfusepath{clip}%
\pgfsetbuttcap%
\pgfsetroundjoin%
\definecolor{currentfill}{rgb}{0.121569,0.466667,0.705882}%
\pgfsetfillcolor{currentfill}%
\pgfsetfillopacity{0.972748}%
\pgfsetlinewidth{1.003750pt}%
\definecolor{currentstroke}{rgb}{0.121569,0.466667,0.705882}%
\pgfsetstrokecolor{currentstroke}%
\pgfsetstrokeopacity{0.972748}%
\pgfsetdash{}{0pt}%
\pgfpathmoveto{\pgfqpoint{2.215053in}{1.675938in}}%
\pgfpathcurveto{\pgfqpoint{2.223289in}{1.675938in}}{\pgfqpoint{2.231189in}{1.679210in}}{\pgfqpoint{2.237013in}{1.685034in}}%
\pgfpathcurveto{\pgfqpoint{2.242837in}{1.690858in}}{\pgfqpoint{2.246109in}{1.698758in}}{\pgfqpoint{2.246109in}{1.706994in}}%
\pgfpathcurveto{\pgfqpoint{2.246109in}{1.715231in}}{\pgfqpoint{2.242837in}{1.723131in}}{\pgfqpoint{2.237013in}{1.728955in}}%
\pgfpathcurveto{\pgfqpoint{2.231189in}{1.734779in}}{\pgfqpoint{2.223289in}{1.738051in}}{\pgfqpoint{2.215053in}{1.738051in}}%
\pgfpathcurveto{\pgfqpoint{2.206817in}{1.738051in}}{\pgfqpoint{2.198917in}{1.734779in}}{\pgfqpoint{2.193093in}{1.728955in}}%
\pgfpathcurveto{\pgfqpoint{2.187269in}{1.723131in}}{\pgfqpoint{2.183996in}{1.715231in}}{\pgfqpoint{2.183996in}{1.706994in}}%
\pgfpathcurveto{\pgfqpoint{2.183996in}{1.698758in}}{\pgfqpoint{2.187269in}{1.690858in}}{\pgfqpoint{2.193093in}{1.685034in}}%
\pgfpathcurveto{\pgfqpoint{2.198917in}{1.679210in}}{\pgfqpoint{2.206817in}{1.675938in}}{\pgfqpoint{2.215053in}{1.675938in}}%
\pgfpathclose%
\pgfusepath{stroke,fill}%
\end{pgfscope}%
\begin{pgfscope}%
\pgfpathrectangle{\pgfqpoint{0.100000in}{0.212622in}}{\pgfqpoint{3.696000in}{3.696000in}}%
\pgfusepath{clip}%
\pgfsetbuttcap%
\pgfsetroundjoin%
\definecolor{currentfill}{rgb}{0.121569,0.466667,0.705882}%
\pgfsetfillcolor{currentfill}%
\pgfsetfillopacity{0.973230}%
\pgfsetlinewidth{1.003750pt}%
\definecolor{currentstroke}{rgb}{0.121569,0.466667,0.705882}%
\pgfsetstrokecolor{currentstroke}%
\pgfsetstrokeopacity{0.973230}%
\pgfsetdash{}{0pt}%
\pgfpathmoveto{\pgfqpoint{2.169728in}{1.671782in}}%
\pgfpathcurveto{\pgfqpoint{2.177964in}{1.671782in}}{\pgfqpoint{2.185864in}{1.675054in}}{\pgfqpoint{2.191688in}{1.680878in}}%
\pgfpathcurveto{\pgfqpoint{2.197512in}{1.686702in}}{\pgfqpoint{2.200784in}{1.694602in}}{\pgfqpoint{2.200784in}{1.702838in}}%
\pgfpathcurveto{\pgfqpoint{2.200784in}{1.711074in}}{\pgfqpoint{2.197512in}{1.718974in}}{\pgfqpoint{2.191688in}{1.724798in}}%
\pgfpathcurveto{\pgfqpoint{2.185864in}{1.730622in}}{\pgfqpoint{2.177964in}{1.733895in}}{\pgfqpoint{2.169728in}{1.733895in}}%
\pgfpathcurveto{\pgfqpoint{2.161491in}{1.733895in}}{\pgfqpoint{2.153591in}{1.730622in}}{\pgfqpoint{2.147767in}{1.724798in}}%
\pgfpathcurveto{\pgfqpoint{2.141943in}{1.718974in}}{\pgfqpoint{2.138671in}{1.711074in}}{\pgfqpoint{2.138671in}{1.702838in}}%
\pgfpathcurveto{\pgfqpoint{2.138671in}{1.694602in}}{\pgfqpoint{2.141943in}{1.686702in}}{\pgfqpoint{2.147767in}{1.680878in}}%
\pgfpathcurveto{\pgfqpoint{2.153591in}{1.675054in}}{\pgfqpoint{2.161491in}{1.671782in}}{\pgfqpoint{2.169728in}{1.671782in}}%
\pgfpathclose%
\pgfusepath{stroke,fill}%
\end{pgfscope}%
\begin{pgfscope}%
\pgfpathrectangle{\pgfqpoint{0.100000in}{0.212622in}}{\pgfqpoint{3.696000in}{3.696000in}}%
\pgfusepath{clip}%
\pgfsetbuttcap%
\pgfsetroundjoin%
\definecolor{currentfill}{rgb}{0.121569,0.466667,0.705882}%
\pgfsetfillcolor{currentfill}%
\pgfsetfillopacity{0.974228}%
\pgfsetlinewidth{1.003750pt}%
\definecolor{currentstroke}{rgb}{0.121569,0.466667,0.705882}%
\pgfsetstrokecolor{currentstroke}%
\pgfsetstrokeopacity{0.974228}%
\pgfsetdash{}{0pt}%
\pgfpathmoveto{\pgfqpoint{2.659052in}{1.662856in}}%
\pgfpathcurveto{\pgfqpoint{2.667289in}{1.662856in}}{\pgfqpoint{2.675189in}{1.666128in}}{\pgfqpoint{2.681013in}{1.671952in}}%
\pgfpathcurveto{\pgfqpoint{2.686836in}{1.677776in}}{\pgfqpoint{2.690109in}{1.685676in}}{\pgfqpoint{2.690109in}{1.693912in}}%
\pgfpathcurveto{\pgfqpoint{2.690109in}{1.702148in}}{\pgfqpoint{2.686836in}{1.710048in}}{\pgfqpoint{2.681013in}{1.715872in}}%
\pgfpathcurveto{\pgfqpoint{2.675189in}{1.721696in}}{\pgfqpoint{2.667289in}{1.724969in}}{\pgfqpoint{2.659052in}{1.724969in}}%
\pgfpathcurveto{\pgfqpoint{2.650816in}{1.724969in}}{\pgfqpoint{2.642916in}{1.721696in}}{\pgfqpoint{2.637092in}{1.715872in}}%
\pgfpathcurveto{\pgfqpoint{2.631268in}{1.710048in}}{\pgfqpoint{2.627996in}{1.702148in}}{\pgfqpoint{2.627996in}{1.693912in}}%
\pgfpathcurveto{\pgfqpoint{2.627996in}{1.685676in}}{\pgfqpoint{2.631268in}{1.677776in}}{\pgfqpoint{2.637092in}{1.671952in}}%
\pgfpathcurveto{\pgfqpoint{2.642916in}{1.666128in}}{\pgfqpoint{2.650816in}{1.662856in}}{\pgfqpoint{2.659052in}{1.662856in}}%
\pgfpathclose%
\pgfusepath{stroke,fill}%
\end{pgfscope}%
\begin{pgfscope}%
\pgfpathrectangle{\pgfqpoint{0.100000in}{0.212622in}}{\pgfqpoint{3.696000in}{3.696000in}}%
\pgfusepath{clip}%
\pgfsetbuttcap%
\pgfsetroundjoin%
\definecolor{currentfill}{rgb}{0.121569,0.466667,0.705882}%
\pgfsetfillcolor{currentfill}%
\pgfsetfillopacity{0.975425}%
\pgfsetlinewidth{1.003750pt}%
\definecolor{currentstroke}{rgb}{0.121569,0.466667,0.705882}%
\pgfsetstrokecolor{currentstroke}%
\pgfsetstrokeopacity{0.975425}%
\pgfsetdash{}{0pt}%
\pgfpathmoveto{\pgfqpoint{2.360203in}{1.662631in}}%
\pgfpathcurveto{\pgfqpoint{2.368439in}{1.662631in}}{\pgfqpoint{2.376339in}{1.665903in}}{\pgfqpoint{2.382163in}{1.671727in}}%
\pgfpathcurveto{\pgfqpoint{2.387987in}{1.677551in}}{\pgfqpoint{2.391259in}{1.685451in}}{\pgfqpoint{2.391259in}{1.693687in}}%
\pgfpathcurveto{\pgfqpoint{2.391259in}{1.701923in}}{\pgfqpoint{2.387987in}{1.709823in}}{\pgfqpoint{2.382163in}{1.715647in}}%
\pgfpathcurveto{\pgfqpoint{2.376339in}{1.721471in}}{\pgfqpoint{2.368439in}{1.724744in}}{\pgfqpoint{2.360203in}{1.724744in}}%
\pgfpathcurveto{\pgfqpoint{2.351966in}{1.724744in}}{\pgfqpoint{2.344066in}{1.721471in}}{\pgfqpoint{2.338242in}{1.715647in}}%
\pgfpathcurveto{\pgfqpoint{2.332418in}{1.709823in}}{\pgfqpoint{2.329146in}{1.701923in}}{\pgfqpoint{2.329146in}{1.693687in}}%
\pgfpathcurveto{\pgfqpoint{2.329146in}{1.685451in}}{\pgfqpoint{2.332418in}{1.677551in}}{\pgfqpoint{2.338242in}{1.671727in}}%
\pgfpathcurveto{\pgfqpoint{2.344066in}{1.665903in}}{\pgfqpoint{2.351966in}{1.662631in}}{\pgfqpoint{2.360203in}{1.662631in}}%
\pgfpathclose%
\pgfusepath{stroke,fill}%
\end{pgfscope}%
\begin{pgfscope}%
\pgfpathrectangle{\pgfqpoint{0.100000in}{0.212622in}}{\pgfqpoint{3.696000in}{3.696000in}}%
\pgfusepath{clip}%
\pgfsetbuttcap%
\pgfsetroundjoin%
\definecolor{currentfill}{rgb}{0.121569,0.466667,0.705882}%
\pgfsetfillcolor{currentfill}%
\pgfsetfillopacity{0.978898}%
\pgfsetlinewidth{1.003750pt}%
\definecolor{currentstroke}{rgb}{0.121569,0.466667,0.705882}%
\pgfsetstrokecolor{currentstroke}%
\pgfsetstrokeopacity{0.978898}%
\pgfsetdash{}{0pt}%
\pgfpathmoveto{\pgfqpoint{2.408736in}{1.643176in}}%
\pgfpathcurveto{\pgfqpoint{2.416972in}{1.643176in}}{\pgfqpoint{2.424872in}{1.646448in}}{\pgfqpoint{2.430696in}{1.652272in}}%
\pgfpathcurveto{\pgfqpoint{2.436520in}{1.658096in}}{\pgfqpoint{2.439792in}{1.665996in}}{\pgfqpoint{2.439792in}{1.674232in}}%
\pgfpathcurveto{\pgfqpoint{2.439792in}{1.682469in}}{\pgfqpoint{2.436520in}{1.690369in}}{\pgfqpoint{2.430696in}{1.696193in}}%
\pgfpathcurveto{\pgfqpoint{2.424872in}{1.702016in}}{\pgfqpoint{2.416972in}{1.705289in}}{\pgfqpoint{2.408736in}{1.705289in}}%
\pgfpathcurveto{\pgfqpoint{2.400500in}{1.705289in}}{\pgfqpoint{2.392600in}{1.702016in}}{\pgfqpoint{2.386776in}{1.696193in}}%
\pgfpathcurveto{\pgfqpoint{2.380952in}{1.690369in}}{\pgfqpoint{2.377679in}{1.682469in}}{\pgfqpoint{2.377679in}{1.674232in}}%
\pgfpathcurveto{\pgfqpoint{2.377679in}{1.665996in}}{\pgfqpoint{2.380952in}{1.658096in}}{\pgfqpoint{2.386776in}{1.652272in}}%
\pgfpathcurveto{\pgfqpoint{2.392600in}{1.646448in}}{\pgfqpoint{2.400500in}{1.643176in}}{\pgfqpoint{2.408736in}{1.643176in}}%
\pgfpathclose%
\pgfusepath{stroke,fill}%
\end{pgfscope}%
\begin{pgfscope}%
\pgfpathrectangle{\pgfqpoint{0.100000in}{0.212622in}}{\pgfqpoint{3.696000in}{3.696000in}}%
\pgfusepath{clip}%
\pgfsetbuttcap%
\pgfsetroundjoin%
\definecolor{currentfill}{rgb}{0.121569,0.466667,0.705882}%
\pgfsetfillcolor{currentfill}%
\pgfsetfillopacity{0.983406}%
\pgfsetlinewidth{1.003750pt}%
\definecolor{currentstroke}{rgb}{0.121569,0.466667,0.705882}%
\pgfsetstrokecolor{currentstroke}%
\pgfsetstrokeopacity{0.983406}%
\pgfsetdash{}{0pt}%
\pgfpathmoveto{\pgfqpoint{2.645918in}{1.735252in}}%
\pgfpathcurveto{\pgfqpoint{2.654154in}{1.735252in}}{\pgfqpoint{2.662054in}{1.738525in}}{\pgfqpoint{2.667878in}{1.744349in}}%
\pgfpathcurveto{\pgfqpoint{2.673702in}{1.750173in}}{\pgfqpoint{2.676974in}{1.758073in}}{\pgfqpoint{2.676974in}{1.766309in}}%
\pgfpathcurveto{\pgfqpoint{2.676974in}{1.774545in}}{\pgfqpoint{2.673702in}{1.782445in}}{\pgfqpoint{2.667878in}{1.788269in}}%
\pgfpathcurveto{\pgfqpoint{2.662054in}{1.794093in}}{\pgfqpoint{2.654154in}{1.797365in}}{\pgfqpoint{2.645918in}{1.797365in}}%
\pgfpathcurveto{\pgfqpoint{2.637681in}{1.797365in}}{\pgfqpoint{2.629781in}{1.794093in}}{\pgfqpoint{2.623958in}{1.788269in}}%
\pgfpathcurveto{\pgfqpoint{2.618134in}{1.782445in}}{\pgfqpoint{2.614861in}{1.774545in}}{\pgfqpoint{2.614861in}{1.766309in}}%
\pgfpathcurveto{\pgfqpoint{2.614861in}{1.758073in}}{\pgfqpoint{2.618134in}{1.750173in}}{\pgfqpoint{2.623958in}{1.744349in}}%
\pgfpathcurveto{\pgfqpoint{2.629781in}{1.738525in}}{\pgfqpoint{2.637681in}{1.735252in}}{\pgfqpoint{2.645918in}{1.735252in}}%
\pgfpathclose%
\pgfusepath{stroke,fill}%
\end{pgfscope}%
\begin{pgfscope}%
\pgfpathrectangle{\pgfqpoint{0.100000in}{0.212622in}}{\pgfqpoint{3.696000in}{3.696000in}}%
\pgfusepath{clip}%
\pgfsetbuttcap%
\pgfsetroundjoin%
\definecolor{currentfill}{rgb}{0.121569,0.466667,0.705882}%
\pgfsetfillcolor{currentfill}%
\pgfsetfillopacity{0.985780}%
\pgfsetlinewidth{1.003750pt}%
\definecolor{currentstroke}{rgb}{0.121569,0.466667,0.705882}%
\pgfsetstrokecolor{currentstroke}%
\pgfsetstrokeopacity{0.985780}%
\pgfsetdash{}{0pt}%
\pgfpathmoveto{\pgfqpoint{2.454731in}{1.610702in}}%
\pgfpathcurveto{\pgfqpoint{2.462967in}{1.610702in}}{\pgfqpoint{2.470867in}{1.613974in}}{\pgfqpoint{2.476691in}{1.619798in}}%
\pgfpathcurveto{\pgfqpoint{2.482515in}{1.625622in}}{\pgfqpoint{2.485787in}{1.633522in}}{\pgfqpoint{2.485787in}{1.641758in}}%
\pgfpathcurveto{\pgfqpoint{2.485787in}{1.649995in}}{\pgfqpoint{2.482515in}{1.657895in}}{\pgfqpoint{2.476691in}{1.663719in}}%
\pgfpathcurveto{\pgfqpoint{2.470867in}{1.669543in}}{\pgfqpoint{2.462967in}{1.672815in}}{\pgfqpoint{2.454731in}{1.672815in}}%
\pgfpathcurveto{\pgfqpoint{2.446495in}{1.672815in}}{\pgfqpoint{2.438594in}{1.669543in}}{\pgfqpoint{2.432771in}{1.663719in}}%
\pgfpathcurveto{\pgfqpoint{2.426947in}{1.657895in}}{\pgfqpoint{2.423674in}{1.649995in}}{\pgfqpoint{2.423674in}{1.641758in}}%
\pgfpathcurveto{\pgfqpoint{2.423674in}{1.633522in}}{\pgfqpoint{2.426947in}{1.625622in}}{\pgfqpoint{2.432771in}{1.619798in}}%
\pgfpathcurveto{\pgfqpoint{2.438594in}{1.613974in}}{\pgfqpoint{2.446495in}{1.610702in}}{\pgfqpoint{2.454731in}{1.610702in}}%
\pgfpathclose%
\pgfusepath{stroke,fill}%
\end{pgfscope}%
\begin{pgfscope}%
\pgfpathrectangle{\pgfqpoint{0.100000in}{0.212622in}}{\pgfqpoint{3.696000in}{3.696000in}}%
\pgfusepath{clip}%
\pgfsetbuttcap%
\pgfsetroundjoin%
\definecolor{currentfill}{rgb}{0.121569,0.466667,0.705882}%
\pgfsetfillcolor{currentfill}%
\pgfsetfillopacity{0.988882}%
\pgfsetlinewidth{1.003750pt}%
\definecolor{currentstroke}{rgb}{0.121569,0.466667,0.705882}%
\pgfsetstrokecolor{currentstroke}%
\pgfsetstrokeopacity{0.988882}%
\pgfsetdash{}{0pt}%
\pgfpathmoveto{\pgfqpoint{2.482814in}{1.600136in}}%
\pgfpathcurveto{\pgfqpoint{2.491050in}{1.600136in}}{\pgfqpoint{2.498950in}{1.603409in}}{\pgfqpoint{2.504774in}{1.609233in}}%
\pgfpathcurveto{\pgfqpoint{2.510598in}{1.615056in}}{\pgfqpoint{2.513870in}{1.622957in}}{\pgfqpoint{2.513870in}{1.631193in}}%
\pgfpathcurveto{\pgfqpoint{2.513870in}{1.639429in}}{\pgfqpoint{2.510598in}{1.647329in}}{\pgfqpoint{2.504774in}{1.653153in}}%
\pgfpathcurveto{\pgfqpoint{2.498950in}{1.658977in}}{\pgfqpoint{2.491050in}{1.662249in}}{\pgfqpoint{2.482814in}{1.662249in}}%
\pgfpathcurveto{\pgfqpoint{2.474577in}{1.662249in}}{\pgfqpoint{2.466677in}{1.658977in}}{\pgfqpoint{2.460853in}{1.653153in}}%
\pgfpathcurveto{\pgfqpoint{2.455030in}{1.647329in}}{\pgfqpoint{2.451757in}{1.639429in}}{\pgfqpoint{2.451757in}{1.631193in}}%
\pgfpathcurveto{\pgfqpoint{2.451757in}{1.622957in}}{\pgfqpoint{2.455030in}{1.615056in}}{\pgfqpoint{2.460853in}{1.609233in}}%
\pgfpathcurveto{\pgfqpoint{2.466677in}{1.603409in}}{\pgfqpoint{2.474577in}{1.600136in}}{\pgfqpoint{2.482814in}{1.600136in}}%
\pgfpathclose%
\pgfusepath{stroke,fill}%
\end{pgfscope}%
\begin{pgfscope}%
\pgfpathrectangle{\pgfqpoint{0.100000in}{0.212622in}}{\pgfqpoint{3.696000in}{3.696000in}}%
\pgfusepath{clip}%
\pgfsetbuttcap%
\pgfsetroundjoin%
\definecolor{currentfill}{rgb}{0.121569,0.466667,0.705882}%
\pgfsetfillcolor{currentfill}%
\pgfsetfillopacity{0.990296}%
\pgfsetlinewidth{1.003750pt}%
\definecolor{currentstroke}{rgb}{0.121569,0.466667,0.705882}%
\pgfsetstrokecolor{currentstroke}%
\pgfsetstrokeopacity{0.990296}%
\pgfsetdash{}{0pt}%
\pgfpathmoveto{\pgfqpoint{2.575480in}{1.672052in}}%
\pgfpathcurveto{\pgfqpoint{2.583716in}{1.672052in}}{\pgfqpoint{2.591616in}{1.675324in}}{\pgfqpoint{2.597440in}{1.681148in}}%
\pgfpathcurveto{\pgfqpoint{2.603264in}{1.686972in}}{\pgfqpoint{2.606536in}{1.694872in}}{\pgfqpoint{2.606536in}{1.703108in}}%
\pgfpathcurveto{\pgfqpoint{2.606536in}{1.711345in}}{\pgfqpoint{2.603264in}{1.719245in}}{\pgfqpoint{2.597440in}{1.725069in}}%
\pgfpathcurveto{\pgfqpoint{2.591616in}{1.730893in}}{\pgfqpoint{2.583716in}{1.734165in}}{\pgfqpoint{2.575480in}{1.734165in}}%
\pgfpathcurveto{\pgfqpoint{2.567243in}{1.734165in}}{\pgfqpoint{2.559343in}{1.730893in}}{\pgfqpoint{2.553519in}{1.725069in}}%
\pgfpathcurveto{\pgfqpoint{2.547695in}{1.719245in}}{\pgfqpoint{2.544423in}{1.711345in}}{\pgfqpoint{2.544423in}{1.703108in}}%
\pgfpathcurveto{\pgfqpoint{2.544423in}{1.694872in}}{\pgfqpoint{2.547695in}{1.686972in}}{\pgfqpoint{2.553519in}{1.681148in}}%
\pgfpathcurveto{\pgfqpoint{2.559343in}{1.675324in}}{\pgfqpoint{2.567243in}{1.672052in}}{\pgfqpoint{2.575480in}{1.672052in}}%
\pgfpathclose%
\pgfusepath{stroke,fill}%
\end{pgfscope}%
\begin{pgfscope}%
\pgfpathrectangle{\pgfqpoint{0.100000in}{0.212622in}}{\pgfqpoint{3.696000in}{3.696000in}}%
\pgfusepath{clip}%
\pgfsetbuttcap%
\pgfsetroundjoin%
\definecolor{currentfill}{rgb}{0.121569,0.466667,0.705882}%
\pgfsetfillcolor{currentfill}%
\pgfsetfillopacity{0.992684}%
\pgfsetlinewidth{1.003750pt}%
\definecolor{currentstroke}{rgb}{0.121569,0.466667,0.705882}%
\pgfsetstrokecolor{currentstroke}%
\pgfsetstrokeopacity{0.992684}%
\pgfsetdash{}{0pt}%
\pgfpathmoveto{\pgfqpoint{2.493170in}{1.583403in}}%
\pgfpathcurveto{\pgfqpoint{2.501406in}{1.583403in}}{\pgfqpoint{2.509306in}{1.586675in}}{\pgfqpoint{2.515130in}{1.592499in}}%
\pgfpathcurveto{\pgfqpoint{2.520954in}{1.598323in}}{\pgfqpoint{2.524227in}{1.606223in}}{\pgfqpoint{2.524227in}{1.614459in}}%
\pgfpathcurveto{\pgfqpoint{2.524227in}{1.622696in}}{\pgfqpoint{2.520954in}{1.630596in}}{\pgfqpoint{2.515130in}{1.636420in}}%
\pgfpathcurveto{\pgfqpoint{2.509306in}{1.642244in}}{\pgfqpoint{2.501406in}{1.645516in}}{\pgfqpoint{2.493170in}{1.645516in}}%
\pgfpathcurveto{\pgfqpoint{2.484934in}{1.645516in}}{\pgfqpoint{2.477034in}{1.642244in}}{\pgfqpoint{2.471210in}{1.636420in}}%
\pgfpathcurveto{\pgfqpoint{2.465386in}{1.630596in}}{\pgfqpoint{2.462114in}{1.622696in}}{\pgfqpoint{2.462114in}{1.614459in}}%
\pgfpathcurveto{\pgfqpoint{2.462114in}{1.606223in}}{\pgfqpoint{2.465386in}{1.598323in}}{\pgfqpoint{2.471210in}{1.592499in}}%
\pgfpathcurveto{\pgfqpoint{2.477034in}{1.586675in}}{\pgfqpoint{2.484934in}{1.583403in}}{\pgfqpoint{2.493170in}{1.583403in}}%
\pgfpathclose%
\pgfusepath{stroke,fill}%
\end{pgfscope}%
\begin{pgfscope}%
\pgfpathrectangle{\pgfqpoint{0.100000in}{0.212622in}}{\pgfqpoint{3.696000in}{3.696000in}}%
\pgfusepath{clip}%
\pgfsetbuttcap%
\pgfsetroundjoin%
\definecolor{currentfill}{rgb}{0.121569,0.466667,0.705882}%
\pgfsetfillcolor{currentfill}%
\pgfsetfillopacity{0.994506}%
\pgfsetlinewidth{1.003750pt}%
\definecolor{currentstroke}{rgb}{0.121569,0.466667,0.705882}%
\pgfsetstrokecolor{currentstroke}%
\pgfsetstrokeopacity{0.994506}%
\pgfsetdash{}{0pt}%
\pgfpathmoveto{\pgfqpoint{2.499894in}{1.576922in}}%
\pgfpathcurveto{\pgfqpoint{2.508130in}{1.576922in}}{\pgfqpoint{2.516030in}{1.580194in}}{\pgfqpoint{2.521854in}{1.586018in}}%
\pgfpathcurveto{\pgfqpoint{2.527678in}{1.591842in}}{\pgfqpoint{2.530950in}{1.599742in}}{\pgfqpoint{2.530950in}{1.607978in}}%
\pgfpathcurveto{\pgfqpoint{2.530950in}{1.616214in}}{\pgfqpoint{2.527678in}{1.624114in}}{\pgfqpoint{2.521854in}{1.629938in}}%
\pgfpathcurveto{\pgfqpoint{2.516030in}{1.635762in}}{\pgfqpoint{2.508130in}{1.639035in}}{\pgfqpoint{2.499894in}{1.639035in}}%
\pgfpathcurveto{\pgfqpoint{2.491658in}{1.639035in}}{\pgfqpoint{2.483757in}{1.635762in}}{\pgfqpoint{2.477934in}{1.629938in}}%
\pgfpathcurveto{\pgfqpoint{2.472110in}{1.624114in}}{\pgfqpoint{2.468837in}{1.616214in}}{\pgfqpoint{2.468837in}{1.607978in}}%
\pgfpathcurveto{\pgfqpoint{2.468837in}{1.599742in}}{\pgfqpoint{2.472110in}{1.591842in}}{\pgfqpoint{2.477934in}{1.586018in}}%
\pgfpathcurveto{\pgfqpoint{2.483757in}{1.580194in}}{\pgfqpoint{2.491658in}{1.576922in}}{\pgfqpoint{2.499894in}{1.576922in}}%
\pgfpathclose%
\pgfusepath{stroke,fill}%
\end{pgfscope}%
\begin{pgfscope}%
\pgfpathrectangle{\pgfqpoint{0.100000in}{0.212622in}}{\pgfqpoint{3.696000in}{3.696000in}}%
\pgfusepath{clip}%
\pgfsetbuttcap%
\pgfsetroundjoin%
\definecolor{currentfill}{rgb}{0.121569,0.466667,0.705882}%
\pgfsetfillcolor{currentfill}%
\pgfsetfillopacity{0.996060}%
\pgfsetlinewidth{1.003750pt}%
\definecolor{currentstroke}{rgb}{0.121569,0.466667,0.705882}%
\pgfsetstrokecolor{currentstroke}%
\pgfsetstrokeopacity{0.996060}%
\pgfsetdash{}{0pt}%
\pgfpathmoveto{\pgfqpoint{2.502305in}{1.570720in}}%
\pgfpathcurveto{\pgfqpoint{2.510541in}{1.570720in}}{\pgfqpoint{2.518442in}{1.573993in}}{\pgfqpoint{2.524265in}{1.579816in}}%
\pgfpathcurveto{\pgfqpoint{2.530089in}{1.585640in}}{\pgfqpoint{2.533362in}{1.593540in}}{\pgfqpoint{2.533362in}{1.601777in}}%
\pgfpathcurveto{\pgfqpoint{2.533362in}{1.610013in}}{\pgfqpoint{2.530089in}{1.617913in}}{\pgfqpoint{2.524265in}{1.623737in}}%
\pgfpathcurveto{\pgfqpoint{2.518442in}{1.629561in}}{\pgfqpoint{2.510541in}{1.632833in}}{\pgfqpoint{2.502305in}{1.632833in}}%
\pgfpathcurveto{\pgfqpoint{2.494069in}{1.632833in}}{\pgfqpoint{2.486169in}{1.629561in}}{\pgfqpoint{2.480345in}{1.623737in}}%
\pgfpathcurveto{\pgfqpoint{2.474521in}{1.617913in}}{\pgfqpoint{2.471249in}{1.610013in}}{\pgfqpoint{2.471249in}{1.601777in}}%
\pgfpathcurveto{\pgfqpoint{2.471249in}{1.593540in}}{\pgfqpoint{2.474521in}{1.585640in}}{\pgfqpoint{2.480345in}{1.579816in}}%
\pgfpathcurveto{\pgfqpoint{2.486169in}{1.573993in}}{\pgfqpoint{2.494069in}{1.570720in}}{\pgfqpoint{2.502305in}{1.570720in}}%
\pgfpathclose%
\pgfusepath{stroke,fill}%
\end{pgfscope}%
\begin{pgfscope}%
\pgfpathrectangle{\pgfqpoint{0.100000in}{0.212622in}}{\pgfqpoint{3.696000in}{3.696000in}}%
\pgfusepath{clip}%
\pgfsetbuttcap%
\pgfsetroundjoin%
\definecolor{currentfill}{rgb}{0.121569,0.466667,0.705882}%
\pgfsetfillcolor{currentfill}%
\pgfsetfillopacity{0.996635}%
\pgfsetlinewidth{1.003750pt}%
\definecolor{currentstroke}{rgb}{0.121569,0.466667,0.705882}%
\pgfsetstrokecolor{currentstroke}%
\pgfsetstrokeopacity{0.996635}%
\pgfsetdash{}{0pt}%
\pgfpathmoveto{\pgfqpoint{2.504402in}{1.569022in}}%
\pgfpathcurveto{\pgfqpoint{2.512638in}{1.569022in}}{\pgfqpoint{2.520538in}{1.572295in}}{\pgfqpoint{2.526362in}{1.578119in}}%
\pgfpathcurveto{\pgfqpoint{2.532186in}{1.583943in}}{\pgfqpoint{2.535458in}{1.591843in}}{\pgfqpoint{2.535458in}{1.600079in}}%
\pgfpathcurveto{\pgfqpoint{2.535458in}{1.608315in}}{\pgfqpoint{2.532186in}{1.616215in}}{\pgfqpoint{2.526362in}{1.622039in}}%
\pgfpathcurveto{\pgfqpoint{2.520538in}{1.627863in}}{\pgfqpoint{2.512638in}{1.631135in}}{\pgfqpoint{2.504402in}{1.631135in}}%
\pgfpathcurveto{\pgfqpoint{2.496165in}{1.631135in}}{\pgfqpoint{2.488265in}{1.627863in}}{\pgfqpoint{2.482441in}{1.622039in}}%
\pgfpathcurveto{\pgfqpoint{2.476617in}{1.616215in}}{\pgfqpoint{2.473345in}{1.608315in}}{\pgfqpoint{2.473345in}{1.600079in}}%
\pgfpathcurveto{\pgfqpoint{2.473345in}{1.591843in}}{\pgfqpoint{2.476617in}{1.583943in}}{\pgfqpoint{2.482441in}{1.578119in}}%
\pgfpathcurveto{\pgfqpoint{2.488265in}{1.572295in}}{\pgfqpoint{2.496165in}{1.569022in}}{\pgfqpoint{2.504402in}{1.569022in}}%
\pgfpathclose%
\pgfusepath{stroke,fill}%
\end{pgfscope}%
\begin{pgfscope}%
\pgfpathrectangle{\pgfqpoint{0.100000in}{0.212622in}}{\pgfqpoint{3.696000in}{3.696000in}}%
\pgfusepath{clip}%
\pgfsetbuttcap%
\pgfsetroundjoin%
\definecolor{currentfill}{rgb}{0.121569,0.466667,0.705882}%
\pgfsetfillcolor{currentfill}%
\pgfsetfillopacity{0.997081}%
\pgfsetlinewidth{1.003750pt}%
\definecolor{currentstroke}{rgb}{0.121569,0.466667,0.705882}%
\pgfsetstrokecolor{currentstroke}%
\pgfsetstrokeopacity{0.997081}%
\pgfsetdash{}{0pt}%
\pgfpathmoveto{\pgfqpoint{2.505240in}{1.567429in}}%
\pgfpathcurveto{\pgfqpoint{2.513477in}{1.567429in}}{\pgfqpoint{2.521377in}{1.570701in}}{\pgfqpoint{2.527201in}{1.576525in}}%
\pgfpathcurveto{\pgfqpoint{2.533025in}{1.582349in}}{\pgfqpoint{2.536297in}{1.590249in}}{\pgfqpoint{2.536297in}{1.598485in}}%
\pgfpathcurveto{\pgfqpoint{2.536297in}{1.606722in}}{\pgfqpoint{2.533025in}{1.614622in}}{\pgfqpoint{2.527201in}{1.620446in}}%
\pgfpathcurveto{\pgfqpoint{2.521377in}{1.626270in}}{\pgfqpoint{2.513477in}{1.629542in}}{\pgfqpoint{2.505240in}{1.629542in}}%
\pgfpathcurveto{\pgfqpoint{2.497004in}{1.629542in}}{\pgfqpoint{2.489104in}{1.626270in}}{\pgfqpoint{2.483280in}{1.620446in}}%
\pgfpathcurveto{\pgfqpoint{2.477456in}{1.614622in}}{\pgfqpoint{2.474184in}{1.606722in}}{\pgfqpoint{2.474184in}{1.598485in}}%
\pgfpathcurveto{\pgfqpoint{2.474184in}{1.590249in}}{\pgfqpoint{2.477456in}{1.582349in}}{\pgfqpoint{2.483280in}{1.576525in}}%
\pgfpathcurveto{\pgfqpoint{2.489104in}{1.570701in}}{\pgfqpoint{2.497004in}{1.567429in}}{\pgfqpoint{2.505240in}{1.567429in}}%
\pgfpathclose%
\pgfusepath{stroke,fill}%
\end{pgfscope}%
\begin{pgfscope}%
\pgfpathrectangle{\pgfqpoint{0.100000in}{0.212622in}}{\pgfqpoint{3.696000in}{3.696000in}}%
\pgfusepath{clip}%
\pgfsetbuttcap%
\pgfsetroundjoin%
\definecolor{currentfill}{rgb}{0.121569,0.466667,0.705882}%
\pgfsetfillcolor{currentfill}%
\pgfsetfillopacity{0.997222}%
\pgfsetlinewidth{1.003750pt}%
\definecolor{currentstroke}{rgb}{0.121569,0.466667,0.705882}%
\pgfsetstrokecolor{currentstroke}%
\pgfsetstrokeopacity{0.997222}%
\pgfsetdash{}{0pt}%
\pgfpathmoveto{\pgfqpoint{2.505962in}{1.567118in}}%
\pgfpathcurveto{\pgfqpoint{2.514199in}{1.567118in}}{\pgfqpoint{2.522099in}{1.570390in}}{\pgfqpoint{2.527923in}{1.576214in}}%
\pgfpathcurveto{\pgfqpoint{2.533746in}{1.582038in}}{\pgfqpoint{2.537019in}{1.589938in}}{\pgfqpoint{2.537019in}{1.598175in}}%
\pgfpathcurveto{\pgfqpoint{2.537019in}{1.606411in}}{\pgfqpoint{2.533746in}{1.614311in}}{\pgfqpoint{2.527923in}{1.620135in}}%
\pgfpathcurveto{\pgfqpoint{2.522099in}{1.625959in}}{\pgfqpoint{2.514199in}{1.629231in}}{\pgfqpoint{2.505962in}{1.629231in}}%
\pgfpathcurveto{\pgfqpoint{2.497726in}{1.629231in}}{\pgfqpoint{2.489826in}{1.625959in}}{\pgfqpoint{2.484002in}{1.620135in}}%
\pgfpathcurveto{\pgfqpoint{2.478178in}{1.614311in}}{\pgfqpoint{2.474906in}{1.606411in}}{\pgfqpoint{2.474906in}{1.598175in}}%
\pgfpathcurveto{\pgfqpoint{2.474906in}{1.589938in}}{\pgfqpoint{2.478178in}{1.582038in}}{\pgfqpoint{2.484002in}{1.576214in}}%
\pgfpathcurveto{\pgfqpoint{2.489826in}{1.570390in}}{\pgfqpoint{2.497726in}{1.567118in}}{\pgfqpoint{2.505962in}{1.567118in}}%
\pgfpathclose%
\pgfusepath{stroke,fill}%
\end{pgfscope}%
\begin{pgfscope}%
\pgfpathrectangle{\pgfqpoint{0.100000in}{0.212622in}}{\pgfqpoint{3.696000in}{3.696000in}}%
\pgfusepath{clip}%
\pgfsetbuttcap%
\pgfsetroundjoin%
\definecolor{currentfill}{rgb}{0.121569,0.466667,0.705882}%
\pgfsetfillcolor{currentfill}%
\pgfsetfillopacity{0.997332}%
\pgfsetlinewidth{1.003750pt}%
\definecolor{currentstroke}{rgb}{0.121569,0.466667,0.705882}%
\pgfsetstrokecolor{currentstroke}%
\pgfsetstrokeopacity{0.997332}%
\pgfsetdash{}{0pt}%
\pgfpathmoveto{\pgfqpoint{2.506273in}{1.566751in}}%
\pgfpathcurveto{\pgfqpoint{2.514509in}{1.566751in}}{\pgfqpoint{2.522409in}{1.570023in}}{\pgfqpoint{2.528233in}{1.575847in}}%
\pgfpathcurveto{\pgfqpoint{2.534057in}{1.581671in}}{\pgfqpoint{2.537329in}{1.589571in}}{\pgfqpoint{2.537329in}{1.597808in}}%
\pgfpathcurveto{\pgfqpoint{2.537329in}{1.606044in}}{\pgfqpoint{2.534057in}{1.613944in}}{\pgfqpoint{2.528233in}{1.619768in}}%
\pgfpathcurveto{\pgfqpoint{2.522409in}{1.625592in}}{\pgfqpoint{2.514509in}{1.628864in}}{\pgfqpoint{2.506273in}{1.628864in}}%
\pgfpathcurveto{\pgfqpoint{2.498037in}{1.628864in}}{\pgfqpoint{2.490137in}{1.625592in}}{\pgfqpoint{2.484313in}{1.619768in}}%
\pgfpathcurveto{\pgfqpoint{2.478489in}{1.613944in}}{\pgfqpoint{2.475216in}{1.606044in}}{\pgfqpoint{2.475216in}{1.597808in}}%
\pgfpathcurveto{\pgfqpoint{2.475216in}{1.589571in}}{\pgfqpoint{2.478489in}{1.581671in}}{\pgfqpoint{2.484313in}{1.575847in}}%
\pgfpathcurveto{\pgfqpoint{2.490137in}{1.570023in}}{\pgfqpoint{2.498037in}{1.566751in}}{\pgfqpoint{2.506273in}{1.566751in}}%
\pgfpathclose%
\pgfusepath{stroke,fill}%
\end{pgfscope}%
\begin{pgfscope}%
\pgfpathrectangle{\pgfqpoint{0.100000in}{0.212622in}}{\pgfqpoint{3.696000in}{3.696000in}}%
\pgfusepath{clip}%
\pgfsetbuttcap%
\pgfsetroundjoin%
\definecolor{currentfill}{rgb}{0.121569,0.466667,0.705882}%
\pgfsetfillcolor{currentfill}%
\pgfsetfillopacity{0.997410}%
\pgfsetlinewidth{1.003750pt}%
\definecolor{currentstroke}{rgb}{0.121569,0.466667,0.705882}%
\pgfsetstrokecolor{currentstroke}%
\pgfsetstrokeopacity{0.997410}%
\pgfsetdash{}{0pt}%
\pgfpathmoveto{\pgfqpoint{2.506402in}{1.566459in}}%
\pgfpathcurveto{\pgfqpoint{2.514638in}{1.566459in}}{\pgfqpoint{2.522538in}{1.569732in}}{\pgfqpoint{2.528362in}{1.575555in}}%
\pgfpathcurveto{\pgfqpoint{2.534186in}{1.581379in}}{\pgfqpoint{2.537458in}{1.589279in}}{\pgfqpoint{2.537458in}{1.597516in}}%
\pgfpathcurveto{\pgfqpoint{2.537458in}{1.605752in}}{\pgfqpoint{2.534186in}{1.613652in}}{\pgfqpoint{2.528362in}{1.619476in}}%
\pgfpathcurveto{\pgfqpoint{2.522538in}{1.625300in}}{\pgfqpoint{2.514638in}{1.628572in}}{\pgfqpoint{2.506402in}{1.628572in}}%
\pgfpathcurveto{\pgfqpoint{2.498166in}{1.628572in}}{\pgfqpoint{2.490266in}{1.625300in}}{\pgfqpoint{2.484442in}{1.619476in}}%
\pgfpathcurveto{\pgfqpoint{2.478618in}{1.613652in}}{\pgfqpoint{2.475345in}{1.605752in}}{\pgfqpoint{2.475345in}{1.597516in}}%
\pgfpathcurveto{\pgfqpoint{2.475345in}{1.589279in}}{\pgfqpoint{2.478618in}{1.581379in}}{\pgfqpoint{2.484442in}{1.575555in}}%
\pgfpathcurveto{\pgfqpoint{2.490266in}{1.569732in}}{\pgfqpoint{2.498166in}{1.566459in}}{\pgfqpoint{2.506402in}{1.566459in}}%
\pgfpathclose%
\pgfusepath{stroke,fill}%
\end{pgfscope}%
\begin{pgfscope}%
\pgfpathrectangle{\pgfqpoint{0.100000in}{0.212622in}}{\pgfqpoint{3.696000in}{3.696000in}}%
\pgfusepath{clip}%
\pgfsetbuttcap%
\pgfsetroundjoin%
\definecolor{currentfill}{rgb}{0.121569,0.466667,0.705882}%
\pgfsetfillcolor{currentfill}%
\pgfsetfillopacity{0.997449}%
\pgfsetlinewidth{1.003750pt}%
\definecolor{currentstroke}{rgb}{0.121569,0.466667,0.705882}%
\pgfsetstrokecolor{currentstroke}%
\pgfsetstrokeopacity{0.997449}%
\pgfsetdash{}{0pt}%
\pgfpathmoveto{\pgfqpoint{2.506484in}{1.566328in}}%
\pgfpathcurveto{\pgfqpoint{2.514720in}{1.566328in}}{\pgfqpoint{2.522620in}{1.569600in}}{\pgfqpoint{2.528444in}{1.575424in}}%
\pgfpathcurveto{\pgfqpoint{2.534268in}{1.581248in}}{\pgfqpoint{2.537541in}{1.589148in}}{\pgfqpoint{2.537541in}{1.597385in}}%
\pgfpathcurveto{\pgfqpoint{2.537541in}{1.605621in}}{\pgfqpoint{2.534268in}{1.613521in}}{\pgfqpoint{2.528444in}{1.619345in}}%
\pgfpathcurveto{\pgfqpoint{2.522620in}{1.625169in}}{\pgfqpoint{2.514720in}{1.628441in}}{\pgfqpoint{2.506484in}{1.628441in}}%
\pgfpathcurveto{\pgfqpoint{2.498248in}{1.628441in}}{\pgfqpoint{2.490348in}{1.625169in}}{\pgfqpoint{2.484524in}{1.619345in}}%
\pgfpathcurveto{\pgfqpoint{2.478700in}{1.613521in}}{\pgfqpoint{2.475428in}{1.605621in}}{\pgfqpoint{2.475428in}{1.597385in}}%
\pgfpathcurveto{\pgfqpoint{2.475428in}{1.589148in}}{\pgfqpoint{2.478700in}{1.581248in}}{\pgfqpoint{2.484524in}{1.575424in}}%
\pgfpathcurveto{\pgfqpoint{2.490348in}{1.569600in}}{\pgfqpoint{2.498248in}{1.566328in}}{\pgfqpoint{2.506484in}{1.566328in}}%
\pgfpathclose%
\pgfusepath{stroke,fill}%
\end{pgfscope}%
\begin{pgfscope}%
\pgfpathrectangle{\pgfqpoint{0.100000in}{0.212622in}}{\pgfqpoint{3.696000in}{3.696000in}}%
\pgfusepath{clip}%
\pgfsetbuttcap%
\pgfsetroundjoin%
\definecolor{currentfill}{rgb}{0.121569,0.466667,0.705882}%
\pgfsetfillcolor{currentfill}%
\pgfsetfillopacity{0.997476}%
\pgfsetlinewidth{1.003750pt}%
\definecolor{currentstroke}{rgb}{0.121569,0.466667,0.705882}%
\pgfsetstrokecolor{currentstroke}%
\pgfsetstrokeopacity{0.997476}%
\pgfsetdash{}{0pt}%
\pgfpathmoveto{\pgfqpoint{2.506517in}{1.566230in}}%
\pgfpathcurveto{\pgfqpoint{2.514753in}{1.566230in}}{\pgfqpoint{2.522653in}{1.569502in}}{\pgfqpoint{2.528477in}{1.575326in}}%
\pgfpathcurveto{\pgfqpoint{2.534301in}{1.581150in}}{\pgfqpoint{2.537573in}{1.589050in}}{\pgfqpoint{2.537573in}{1.597286in}}%
\pgfpathcurveto{\pgfqpoint{2.537573in}{1.605523in}}{\pgfqpoint{2.534301in}{1.613423in}}{\pgfqpoint{2.528477in}{1.619247in}}%
\pgfpathcurveto{\pgfqpoint{2.522653in}{1.625070in}}{\pgfqpoint{2.514753in}{1.628343in}}{\pgfqpoint{2.506517in}{1.628343in}}%
\pgfpathcurveto{\pgfqpoint{2.498280in}{1.628343in}}{\pgfqpoint{2.490380in}{1.625070in}}{\pgfqpoint{2.484556in}{1.619247in}}%
\pgfpathcurveto{\pgfqpoint{2.478732in}{1.613423in}}{\pgfqpoint{2.475460in}{1.605523in}}{\pgfqpoint{2.475460in}{1.597286in}}%
\pgfpathcurveto{\pgfqpoint{2.475460in}{1.589050in}}{\pgfqpoint{2.478732in}{1.581150in}}{\pgfqpoint{2.484556in}{1.575326in}}%
\pgfpathcurveto{\pgfqpoint{2.490380in}{1.569502in}}{\pgfqpoint{2.498280in}{1.566230in}}{\pgfqpoint{2.506517in}{1.566230in}}%
\pgfpathclose%
\pgfusepath{stroke,fill}%
\end{pgfscope}%
\begin{pgfscope}%
\pgfpathrectangle{\pgfqpoint{0.100000in}{0.212622in}}{\pgfqpoint{3.696000in}{3.696000in}}%
\pgfusepath{clip}%
\pgfsetbuttcap%
\pgfsetroundjoin%
\definecolor{currentfill}{rgb}{0.121569,0.466667,0.705882}%
\pgfsetfillcolor{currentfill}%
\pgfsetfillopacity{0.997488}%
\pgfsetlinewidth{1.003750pt}%
\definecolor{currentstroke}{rgb}{0.121569,0.466667,0.705882}%
\pgfsetstrokecolor{currentstroke}%
\pgfsetstrokeopacity{0.997488}%
\pgfsetdash{}{0pt}%
\pgfpathmoveto{\pgfqpoint{2.506544in}{1.566197in}}%
\pgfpathcurveto{\pgfqpoint{2.514780in}{1.566197in}}{\pgfqpoint{2.522681in}{1.569470in}}{\pgfqpoint{2.528504in}{1.575294in}}%
\pgfpathcurveto{\pgfqpoint{2.534328in}{1.581118in}}{\pgfqpoint{2.537601in}{1.589018in}}{\pgfqpoint{2.537601in}{1.597254in}}%
\pgfpathcurveto{\pgfqpoint{2.537601in}{1.605490in}}{\pgfqpoint{2.534328in}{1.613390in}}{\pgfqpoint{2.528504in}{1.619214in}}%
\pgfpathcurveto{\pgfqpoint{2.522681in}{1.625038in}}{\pgfqpoint{2.514780in}{1.628310in}}{\pgfqpoint{2.506544in}{1.628310in}}%
\pgfpathcurveto{\pgfqpoint{2.498308in}{1.628310in}}{\pgfqpoint{2.490408in}{1.625038in}}{\pgfqpoint{2.484584in}{1.619214in}}%
\pgfpathcurveto{\pgfqpoint{2.478760in}{1.613390in}}{\pgfqpoint{2.475488in}{1.605490in}}{\pgfqpoint{2.475488in}{1.597254in}}%
\pgfpathcurveto{\pgfqpoint{2.475488in}{1.589018in}}{\pgfqpoint{2.478760in}{1.581118in}}{\pgfqpoint{2.484584in}{1.575294in}}%
\pgfpathcurveto{\pgfqpoint{2.490408in}{1.569470in}}{\pgfqpoint{2.498308in}{1.566197in}}{\pgfqpoint{2.506544in}{1.566197in}}%
\pgfpathclose%
\pgfusepath{stroke,fill}%
\end{pgfscope}%
\begin{pgfscope}%
\pgfpathrectangle{\pgfqpoint{0.100000in}{0.212622in}}{\pgfqpoint{3.696000in}{3.696000in}}%
\pgfusepath{clip}%
\pgfsetbuttcap%
\pgfsetroundjoin%
\definecolor{currentfill}{rgb}{0.121569,0.466667,0.705882}%
\pgfsetfillcolor{currentfill}%
\pgfsetfillopacity{0.997494}%
\pgfsetlinewidth{1.003750pt}%
\definecolor{currentstroke}{rgb}{0.121569,0.466667,0.705882}%
\pgfsetstrokecolor{currentstroke}%
\pgfsetstrokeopacity{0.997494}%
\pgfsetdash{}{0pt}%
\pgfpathmoveto{\pgfqpoint{2.506560in}{1.566181in}}%
\pgfpathcurveto{\pgfqpoint{2.514796in}{1.566181in}}{\pgfqpoint{2.522696in}{1.569453in}}{\pgfqpoint{2.528520in}{1.575277in}}%
\pgfpathcurveto{\pgfqpoint{2.534344in}{1.581101in}}{\pgfqpoint{2.537617in}{1.589001in}}{\pgfqpoint{2.537617in}{1.597237in}}%
\pgfpathcurveto{\pgfqpoint{2.537617in}{1.605473in}}{\pgfqpoint{2.534344in}{1.613373in}}{\pgfqpoint{2.528520in}{1.619197in}}%
\pgfpathcurveto{\pgfqpoint{2.522696in}{1.625021in}}{\pgfqpoint{2.514796in}{1.628294in}}{\pgfqpoint{2.506560in}{1.628294in}}%
\pgfpathcurveto{\pgfqpoint{2.498324in}{1.628294in}}{\pgfqpoint{2.490424in}{1.625021in}}{\pgfqpoint{2.484600in}{1.619197in}}%
\pgfpathcurveto{\pgfqpoint{2.478776in}{1.613373in}}{\pgfqpoint{2.475504in}{1.605473in}}{\pgfqpoint{2.475504in}{1.597237in}}%
\pgfpathcurveto{\pgfqpoint{2.475504in}{1.589001in}}{\pgfqpoint{2.478776in}{1.581101in}}{\pgfqpoint{2.484600in}{1.575277in}}%
\pgfpathcurveto{\pgfqpoint{2.490424in}{1.569453in}}{\pgfqpoint{2.498324in}{1.566181in}}{\pgfqpoint{2.506560in}{1.566181in}}%
\pgfpathclose%
\pgfusepath{stroke,fill}%
\end{pgfscope}%
\begin{pgfscope}%
\pgfpathrectangle{\pgfqpoint{0.100000in}{0.212622in}}{\pgfqpoint{3.696000in}{3.696000in}}%
\pgfusepath{clip}%
\pgfsetbuttcap%
\pgfsetroundjoin%
\definecolor{currentfill}{rgb}{0.121569,0.466667,0.705882}%
\pgfsetfillcolor{currentfill}%
\pgfsetfillopacity{0.997497}%
\pgfsetlinewidth{1.003750pt}%
\definecolor{currentstroke}{rgb}{0.121569,0.466667,0.705882}%
\pgfsetstrokecolor{currentstroke}%
\pgfsetstrokeopacity{0.997497}%
\pgfsetdash{}{0pt}%
\pgfpathmoveto{\pgfqpoint{2.506568in}{1.566171in}}%
\pgfpathcurveto{\pgfqpoint{2.514805in}{1.566171in}}{\pgfqpoint{2.522705in}{1.569443in}}{\pgfqpoint{2.528529in}{1.575267in}}%
\pgfpathcurveto{\pgfqpoint{2.534353in}{1.581091in}}{\pgfqpoint{2.537625in}{1.588991in}}{\pgfqpoint{2.537625in}{1.597227in}}%
\pgfpathcurveto{\pgfqpoint{2.537625in}{1.605464in}}{\pgfqpoint{2.534353in}{1.613364in}}{\pgfqpoint{2.528529in}{1.619188in}}%
\pgfpathcurveto{\pgfqpoint{2.522705in}{1.625011in}}{\pgfqpoint{2.514805in}{1.628284in}}{\pgfqpoint{2.506568in}{1.628284in}}%
\pgfpathcurveto{\pgfqpoint{2.498332in}{1.628284in}}{\pgfqpoint{2.490432in}{1.625011in}}{\pgfqpoint{2.484608in}{1.619188in}}%
\pgfpathcurveto{\pgfqpoint{2.478784in}{1.613364in}}{\pgfqpoint{2.475512in}{1.605464in}}{\pgfqpoint{2.475512in}{1.597227in}}%
\pgfpathcurveto{\pgfqpoint{2.475512in}{1.588991in}}{\pgfqpoint{2.478784in}{1.581091in}}{\pgfqpoint{2.484608in}{1.575267in}}%
\pgfpathcurveto{\pgfqpoint{2.490432in}{1.569443in}}{\pgfqpoint{2.498332in}{1.566171in}}{\pgfqpoint{2.506568in}{1.566171in}}%
\pgfpathclose%
\pgfusepath{stroke,fill}%
\end{pgfscope}%
\begin{pgfscope}%
\pgfpathrectangle{\pgfqpoint{0.100000in}{0.212622in}}{\pgfqpoint{3.696000in}{3.696000in}}%
\pgfusepath{clip}%
\pgfsetbuttcap%
\pgfsetroundjoin%
\definecolor{currentfill}{rgb}{0.121569,0.466667,0.705882}%
\pgfsetfillcolor{currentfill}%
\pgfsetfillopacity{0.997499}%
\pgfsetlinewidth{1.003750pt}%
\definecolor{currentstroke}{rgb}{0.121569,0.466667,0.705882}%
\pgfsetstrokecolor{currentstroke}%
\pgfsetstrokeopacity{0.997499}%
\pgfsetdash{}{0pt}%
\pgfpathmoveto{\pgfqpoint{2.506572in}{1.566162in}}%
\pgfpathcurveto{\pgfqpoint{2.514808in}{1.566162in}}{\pgfqpoint{2.522708in}{1.569434in}}{\pgfqpoint{2.528532in}{1.575258in}}%
\pgfpathcurveto{\pgfqpoint{2.534356in}{1.581082in}}{\pgfqpoint{2.537628in}{1.588982in}}{\pgfqpoint{2.537628in}{1.597218in}}%
\pgfpathcurveto{\pgfqpoint{2.537628in}{1.605455in}}{\pgfqpoint{2.534356in}{1.613355in}}{\pgfqpoint{2.528532in}{1.619179in}}%
\pgfpathcurveto{\pgfqpoint{2.522708in}{1.625003in}}{\pgfqpoint{2.514808in}{1.628275in}}{\pgfqpoint{2.506572in}{1.628275in}}%
\pgfpathcurveto{\pgfqpoint{2.498335in}{1.628275in}}{\pgfqpoint{2.490435in}{1.625003in}}{\pgfqpoint{2.484611in}{1.619179in}}%
\pgfpathcurveto{\pgfqpoint{2.478787in}{1.613355in}}{\pgfqpoint{2.475515in}{1.605455in}}{\pgfqpoint{2.475515in}{1.597218in}}%
\pgfpathcurveto{\pgfqpoint{2.475515in}{1.588982in}}{\pgfqpoint{2.478787in}{1.581082in}}{\pgfqpoint{2.484611in}{1.575258in}}%
\pgfpathcurveto{\pgfqpoint{2.490435in}{1.569434in}}{\pgfqpoint{2.498335in}{1.566162in}}{\pgfqpoint{2.506572in}{1.566162in}}%
\pgfpathclose%
\pgfusepath{stroke,fill}%
\end{pgfscope}%
\begin{pgfscope}%
\pgfpathrectangle{\pgfqpoint{0.100000in}{0.212622in}}{\pgfqpoint{3.696000in}{3.696000in}}%
\pgfusepath{clip}%
\pgfsetbuttcap%
\pgfsetroundjoin%
\definecolor{currentfill}{rgb}{0.121569,0.466667,0.705882}%
\pgfsetfillcolor{currentfill}%
\pgfsetfillopacity{0.997500}%
\pgfsetlinewidth{1.003750pt}%
\definecolor{currentstroke}{rgb}{0.121569,0.466667,0.705882}%
\pgfsetstrokecolor{currentstroke}%
\pgfsetstrokeopacity{0.997500}%
\pgfsetdash{}{0pt}%
\pgfpathmoveto{\pgfqpoint{2.506574in}{1.566159in}}%
\pgfpathcurveto{\pgfqpoint{2.514810in}{1.566159in}}{\pgfqpoint{2.522710in}{1.569431in}}{\pgfqpoint{2.528534in}{1.575255in}}%
\pgfpathcurveto{\pgfqpoint{2.534358in}{1.581079in}}{\pgfqpoint{2.537630in}{1.588979in}}{\pgfqpoint{2.537630in}{1.597215in}}%
\pgfpathcurveto{\pgfqpoint{2.537630in}{1.605451in}}{\pgfqpoint{2.534358in}{1.613351in}}{\pgfqpoint{2.528534in}{1.619175in}}%
\pgfpathcurveto{\pgfqpoint{2.522710in}{1.624999in}}{\pgfqpoint{2.514810in}{1.628272in}}{\pgfqpoint{2.506574in}{1.628272in}}%
\pgfpathcurveto{\pgfqpoint{2.498338in}{1.628272in}}{\pgfqpoint{2.490438in}{1.624999in}}{\pgfqpoint{2.484614in}{1.619175in}}%
\pgfpathcurveto{\pgfqpoint{2.478790in}{1.613351in}}{\pgfqpoint{2.475517in}{1.605451in}}{\pgfqpoint{2.475517in}{1.597215in}}%
\pgfpathcurveto{\pgfqpoint{2.475517in}{1.588979in}}{\pgfqpoint{2.478790in}{1.581079in}}{\pgfqpoint{2.484614in}{1.575255in}}%
\pgfpathcurveto{\pgfqpoint{2.490438in}{1.569431in}}{\pgfqpoint{2.498338in}{1.566159in}}{\pgfqpoint{2.506574in}{1.566159in}}%
\pgfpathclose%
\pgfusepath{stroke,fill}%
\end{pgfscope}%
\begin{pgfscope}%
\pgfpathrectangle{\pgfqpoint{0.100000in}{0.212622in}}{\pgfqpoint{3.696000in}{3.696000in}}%
\pgfusepath{clip}%
\pgfsetbuttcap%
\pgfsetroundjoin%
\definecolor{currentfill}{rgb}{0.121569,0.466667,0.705882}%
\pgfsetfillcolor{currentfill}%
\pgfsetfillopacity{0.997501}%
\pgfsetlinewidth{1.003750pt}%
\definecolor{currentstroke}{rgb}{0.121569,0.466667,0.705882}%
\pgfsetstrokecolor{currentstroke}%
\pgfsetstrokeopacity{0.997501}%
\pgfsetdash{}{0pt}%
\pgfpathmoveto{\pgfqpoint{2.506575in}{1.566157in}}%
\pgfpathcurveto{\pgfqpoint{2.514812in}{1.566157in}}{\pgfqpoint{2.522712in}{1.569429in}}{\pgfqpoint{2.528536in}{1.575253in}}%
\pgfpathcurveto{\pgfqpoint{2.534360in}{1.581077in}}{\pgfqpoint{2.537632in}{1.588977in}}{\pgfqpoint{2.537632in}{1.597213in}}%
\pgfpathcurveto{\pgfqpoint{2.537632in}{1.605450in}}{\pgfqpoint{2.534360in}{1.613350in}}{\pgfqpoint{2.528536in}{1.619174in}}%
\pgfpathcurveto{\pgfqpoint{2.522712in}{1.624997in}}{\pgfqpoint{2.514812in}{1.628270in}}{\pgfqpoint{2.506575in}{1.628270in}}%
\pgfpathcurveto{\pgfqpoint{2.498339in}{1.628270in}}{\pgfqpoint{2.490439in}{1.624997in}}{\pgfqpoint{2.484615in}{1.619174in}}%
\pgfpathcurveto{\pgfqpoint{2.478791in}{1.613350in}}{\pgfqpoint{2.475519in}{1.605450in}}{\pgfqpoint{2.475519in}{1.597213in}}%
\pgfpathcurveto{\pgfqpoint{2.475519in}{1.588977in}}{\pgfqpoint{2.478791in}{1.581077in}}{\pgfqpoint{2.484615in}{1.575253in}}%
\pgfpathcurveto{\pgfqpoint{2.490439in}{1.569429in}}{\pgfqpoint{2.498339in}{1.566157in}}{\pgfqpoint{2.506575in}{1.566157in}}%
\pgfpathclose%
\pgfusepath{stroke,fill}%
\end{pgfscope}%
\begin{pgfscope}%
\pgfpathrectangle{\pgfqpoint{0.100000in}{0.212622in}}{\pgfqpoint{3.696000in}{3.696000in}}%
\pgfusepath{clip}%
\pgfsetbuttcap%
\pgfsetroundjoin%
\definecolor{currentfill}{rgb}{0.121569,0.466667,0.705882}%
\pgfsetfillcolor{currentfill}%
\pgfsetfillopacity{0.997501}%
\pgfsetlinewidth{1.003750pt}%
\definecolor{currentstroke}{rgb}{0.121569,0.466667,0.705882}%
\pgfsetstrokecolor{currentstroke}%
\pgfsetstrokeopacity{0.997501}%
\pgfsetdash{}{0pt}%
\pgfpathmoveto{\pgfqpoint{2.506576in}{1.566155in}}%
\pgfpathcurveto{\pgfqpoint{2.514812in}{1.566155in}}{\pgfqpoint{2.522712in}{1.569428in}}{\pgfqpoint{2.528536in}{1.575252in}}%
\pgfpathcurveto{\pgfqpoint{2.534360in}{1.581076in}}{\pgfqpoint{2.537632in}{1.588976in}}{\pgfqpoint{2.537632in}{1.597212in}}%
\pgfpathcurveto{\pgfqpoint{2.537632in}{1.605448in}}{\pgfqpoint{2.534360in}{1.613348in}}{\pgfqpoint{2.528536in}{1.619172in}}%
\pgfpathcurveto{\pgfqpoint{2.522712in}{1.624996in}}{\pgfqpoint{2.514812in}{1.628268in}}{\pgfqpoint{2.506576in}{1.628268in}}%
\pgfpathcurveto{\pgfqpoint{2.498340in}{1.628268in}}{\pgfqpoint{2.490440in}{1.624996in}}{\pgfqpoint{2.484616in}{1.619172in}}%
\pgfpathcurveto{\pgfqpoint{2.478792in}{1.613348in}}{\pgfqpoint{2.475519in}{1.605448in}}{\pgfqpoint{2.475519in}{1.597212in}}%
\pgfpathcurveto{\pgfqpoint{2.475519in}{1.588976in}}{\pgfqpoint{2.478792in}{1.581076in}}{\pgfqpoint{2.484616in}{1.575252in}}%
\pgfpathcurveto{\pgfqpoint{2.490440in}{1.569428in}}{\pgfqpoint{2.498340in}{1.566155in}}{\pgfqpoint{2.506576in}{1.566155in}}%
\pgfpathclose%
\pgfusepath{stroke,fill}%
\end{pgfscope}%
\begin{pgfscope}%
\pgfpathrectangle{\pgfqpoint{0.100000in}{0.212622in}}{\pgfqpoint{3.696000in}{3.696000in}}%
\pgfusepath{clip}%
\pgfsetbuttcap%
\pgfsetroundjoin%
\definecolor{currentfill}{rgb}{0.121569,0.466667,0.705882}%
\pgfsetfillcolor{currentfill}%
\pgfsetfillopacity{0.997502}%
\pgfsetlinewidth{1.003750pt}%
\definecolor{currentstroke}{rgb}{0.121569,0.466667,0.705882}%
\pgfsetstrokecolor{currentstroke}%
\pgfsetstrokeopacity{0.997502}%
\pgfsetdash{}{0pt}%
\pgfpathmoveto{\pgfqpoint{2.506576in}{1.566155in}}%
\pgfpathcurveto{\pgfqpoint{2.514813in}{1.566155in}}{\pgfqpoint{2.522713in}{1.569427in}}{\pgfqpoint{2.528537in}{1.575251in}}%
\pgfpathcurveto{\pgfqpoint{2.534360in}{1.581075in}}{\pgfqpoint{2.537633in}{1.588975in}}{\pgfqpoint{2.537633in}{1.597211in}}%
\pgfpathcurveto{\pgfqpoint{2.537633in}{1.605448in}}{\pgfqpoint{2.534360in}{1.613348in}}{\pgfqpoint{2.528537in}{1.619172in}}%
\pgfpathcurveto{\pgfqpoint{2.522713in}{1.624996in}}{\pgfqpoint{2.514813in}{1.628268in}}{\pgfqpoint{2.506576in}{1.628268in}}%
\pgfpathcurveto{\pgfqpoint{2.498340in}{1.628268in}}{\pgfqpoint{2.490440in}{1.624996in}}{\pgfqpoint{2.484616in}{1.619172in}}%
\pgfpathcurveto{\pgfqpoint{2.478792in}{1.613348in}}{\pgfqpoint{2.475520in}{1.605448in}}{\pgfqpoint{2.475520in}{1.597211in}}%
\pgfpathcurveto{\pgfqpoint{2.475520in}{1.588975in}}{\pgfqpoint{2.478792in}{1.581075in}}{\pgfqpoint{2.484616in}{1.575251in}}%
\pgfpathcurveto{\pgfqpoint{2.490440in}{1.569427in}}{\pgfqpoint{2.498340in}{1.566155in}}{\pgfqpoint{2.506576in}{1.566155in}}%
\pgfpathclose%
\pgfusepath{stroke,fill}%
\end{pgfscope}%
\begin{pgfscope}%
\pgfpathrectangle{\pgfqpoint{0.100000in}{0.212622in}}{\pgfqpoint{3.696000in}{3.696000in}}%
\pgfusepath{clip}%
\pgfsetbuttcap%
\pgfsetroundjoin%
\definecolor{currentfill}{rgb}{0.121569,0.466667,0.705882}%
\pgfsetfillcolor{currentfill}%
\pgfsetfillopacity{0.997502}%
\pgfsetlinewidth{1.003750pt}%
\definecolor{currentstroke}{rgb}{0.121569,0.466667,0.705882}%
\pgfsetstrokecolor{currentstroke}%
\pgfsetstrokeopacity{0.997502}%
\pgfsetdash{}{0pt}%
\pgfpathmoveto{\pgfqpoint{2.506577in}{1.566155in}}%
\pgfpathcurveto{\pgfqpoint{2.514813in}{1.566155in}}{\pgfqpoint{2.522713in}{1.569427in}}{\pgfqpoint{2.528537in}{1.575251in}}%
\pgfpathcurveto{\pgfqpoint{2.534361in}{1.581075in}}{\pgfqpoint{2.537633in}{1.588975in}}{\pgfqpoint{2.537633in}{1.597211in}}%
\pgfpathcurveto{\pgfqpoint{2.537633in}{1.605447in}}{\pgfqpoint{2.534361in}{1.613347in}}{\pgfqpoint{2.528537in}{1.619171in}}%
\pgfpathcurveto{\pgfqpoint{2.522713in}{1.624995in}}{\pgfqpoint{2.514813in}{1.628268in}}{\pgfqpoint{2.506577in}{1.628268in}}%
\pgfpathcurveto{\pgfqpoint{2.498340in}{1.628268in}}{\pgfqpoint{2.490440in}{1.624995in}}{\pgfqpoint{2.484616in}{1.619171in}}%
\pgfpathcurveto{\pgfqpoint{2.478792in}{1.613347in}}{\pgfqpoint{2.475520in}{1.605447in}}{\pgfqpoint{2.475520in}{1.597211in}}%
\pgfpathcurveto{\pgfqpoint{2.475520in}{1.588975in}}{\pgfqpoint{2.478792in}{1.581075in}}{\pgfqpoint{2.484616in}{1.575251in}}%
\pgfpathcurveto{\pgfqpoint{2.490440in}{1.569427in}}{\pgfqpoint{2.498340in}{1.566155in}}{\pgfqpoint{2.506577in}{1.566155in}}%
\pgfpathclose%
\pgfusepath{stroke,fill}%
\end{pgfscope}%
\begin{pgfscope}%
\pgfpathrectangle{\pgfqpoint{0.100000in}{0.212622in}}{\pgfqpoint{3.696000in}{3.696000in}}%
\pgfusepath{clip}%
\pgfsetbuttcap%
\pgfsetroundjoin%
\definecolor{currentfill}{rgb}{0.121569,0.466667,0.705882}%
\pgfsetfillcolor{currentfill}%
\pgfsetfillopacity{0.997502}%
\pgfsetlinewidth{1.003750pt}%
\definecolor{currentstroke}{rgb}{0.121569,0.466667,0.705882}%
\pgfsetstrokecolor{currentstroke}%
\pgfsetstrokeopacity{0.997502}%
\pgfsetdash{}{0pt}%
\pgfpathmoveto{\pgfqpoint{2.506577in}{1.566154in}}%
\pgfpathcurveto{\pgfqpoint{2.514813in}{1.566154in}}{\pgfqpoint{2.522713in}{1.569427in}}{\pgfqpoint{2.528537in}{1.575251in}}%
\pgfpathcurveto{\pgfqpoint{2.534361in}{1.581075in}}{\pgfqpoint{2.537633in}{1.588975in}}{\pgfqpoint{2.537633in}{1.597211in}}%
\pgfpathcurveto{\pgfqpoint{2.537633in}{1.605447in}}{\pgfqpoint{2.534361in}{1.613347in}}{\pgfqpoint{2.528537in}{1.619171in}}%
\pgfpathcurveto{\pgfqpoint{2.522713in}{1.624995in}}{\pgfqpoint{2.514813in}{1.628267in}}{\pgfqpoint{2.506577in}{1.628267in}}%
\pgfpathcurveto{\pgfqpoint{2.498340in}{1.628267in}}{\pgfqpoint{2.490440in}{1.624995in}}{\pgfqpoint{2.484616in}{1.619171in}}%
\pgfpathcurveto{\pgfqpoint{2.478792in}{1.613347in}}{\pgfqpoint{2.475520in}{1.605447in}}{\pgfqpoint{2.475520in}{1.597211in}}%
\pgfpathcurveto{\pgfqpoint{2.475520in}{1.588975in}}{\pgfqpoint{2.478792in}{1.581075in}}{\pgfqpoint{2.484616in}{1.575251in}}%
\pgfpathcurveto{\pgfqpoint{2.490440in}{1.569427in}}{\pgfqpoint{2.498340in}{1.566154in}}{\pgfqpoint{2.506577in}{1.566154in}}%
\pgfpathclose%
\pgfusepath{stroke,fill}%
\end{pgfscope}%
\begin{pgfscope}%
\pgfpathrectangle{\pgfqpoint{0.100000in}{0.212622in}}{\pgfqpoint{3.696000in}{3.696000in}}%
\pgfusepath{clip}%
\pgfsetbuttcap%
\pgfsetroundjoin%
\definecolor{currentfill}{rgb}{0.121569,0.466667,0.705882}%
\pgfsetfillcolor{currentfill}%
\pgfsetfillopacity{0.997502}%
\pgfsetlinewidth{1.003750pt}%
\definecolor{currentstroke}{rgb}{0.121569,0.466667,0.705882}%
\pgfsetstrokecolor{currentstroke}%
\pgfsetstrokeopacity{0.997502}%
\pgfsetdash{}{0pt}%
\pgfpathmoveto{\pgfqpoint{2.506577in}{1.566154in}}%
\pgfpathcurveto{\pgfqpoint{2.514813in}{1.566154in}}{\pgfqpoint{2.522713in}{1.569427in}}{\pgfqpoint{2.528537in}{1.575251in}}%
\pgfpathcurveto{\pgfqpoint{2.534361in}{1.581075in}}{\pgfqpoint{2.537633in}{1.588975in}}{\pgfqpoint{2.537633in}{1.597211in}}%
\pgfpathcurveto{\pgfqpoint{2.537633in}{1.605447in}}{\pgfqpoint{2.534361in}{1.613347in}}{\pgfqpoint{2.528537in}{1.619171in}}%
\pgfpathcurveto{\pgfqpoint{2.522713in}{1.624995in}}{\pgfqpoint{2.514813in}{1.628267in}}{\pgfqpoint{2.506577in}{1.628267in}}%
\pgfpathcurveto{\pgfqpoint{2.498340in}{1.628267in}}{\pgfqpoint{2.490440in}{1.624995in}}{\pgfqpoint{2.484616in}{1.619171in}}%
\pgfpathcurveto{\pgfqpoint{2.478793in}{1.613347in}}{\pgfqpoint{2.475520in}{1.605447in}}{\pgfqpoint{2.475520in}{1.597211in}}%
\pgfpathcurveto{\pgfqpoint{2.475520in}{1.588975in}}{\pgfqpoint{2.478793in}{1.581075in}}{\pgfqpoint{2.484616in}{1.575251in}}%
\pgfpathcurveto{\pgfqpoint{2.490440in}{1.569427in}}{\pgfqpoint{2.498340in}{1.566154in}}{\pgfqpoint{2.506577in}{1.566154in}}%
\pgfpathclose%
\pgfusepath{stroke,fill}%
\end{pgfscope}%
\begin{pgfscope}%
\pgfpathrectangle{\pgfqpoint{0.100000in}{0.212622in}}{\pgfqpoint{3.696000in}{3.696000in}}%
\pgfusepath{clip}%
\pgfsetbuttcap%
\pgfsetroundjoin%
\definecolor{currentfill}{rgb}{0.121569,0.466667,0.705882}%
\pgfsetfillcolor{currentfill}%
\pgfsetfillopacity{0.997502}%
\pgfsetlinewidth{1.003750pt}%
\definecolor{currentstroke}{rgb}{0.121569,0.466667,0.705882}%
\pgfsetstrokecolor{currentstroke}%
\pgfsetstrokeopacity{0.997502}%
\pgfsetdash{}{0pt}%
\pgfpathmoveto{\pgfqpoint{2.506577in}{1.566154in}}%
\pgfpathcurveto{\pgfqpoint{2.514813in}{1.566154in}}{\pgfqpoint{2.522713in}{1.569427in}}{\pgfqpoint{2.528537in}{1.575251in}}%
\pgfpathcurveto{\pgfqpoint{2.534361in}{1.581075in}}{\pgfqpoint{2.537633in}{1.588975in}}{\pgfqpoint{2.537633in}{1.597211in}}%
\pgfpathcurveto{\pgfqpoint{2.537633in}{1.605447in}}{\pgfqpoint{2.534361in}{1.613347in}}{\pgfqpoint{2.528537in}{1.619171in}}%
\pgfpathcurveto{\pgfqpoint{2.522713in}{1.624995in}}{\pgfqpoint{2.514813in}{1.628267in}}{\pgfqpoint{2.506577in}{1.628267in}}%
\pgfpathcurveto{\pgfqpoint{2.498340in}{1.628267in}}{\pgfqpoint{2.490440in}{1.624995in}}{\pgfqpoint{2.484617in}{1.619171in}}%
\pgfpathcurveto{\pgfqpoint{2.478793in}{1.613347in}}{\pgfqpoint{2.475520in}{1.605447in}}{\pgfqpoint{2.475520in}{1.597211in}}%
\pgfpathcurveto{\pgfqpoint{2.475520in}{1.588975in}}{\pgfqpoint{2.478793in}{1.581075in}}{\pgfqpoint{2.484617in}{1.575251in}}%
\pgfpathcurveto{\pgfqpoint{2.490440in}{1.569427in}}{\pgfqpoint{2.498340in}{1.566154in}}{\pgfqpoint{2.506577in}{1.566154in}}%
\pgfpathclose%
\pgfusepath{stroke,fill}%
\end{pgfscope}%
\begin{pgfscope}%
\pgfpathrectangle{\pgfqpoint{0.100000in}{0.212622in}}{\pgfqpoint{3.696000in}{3.696000in}}%
\pgfusepath{clip}%
\pgfsetbuttcap%
\pgfsetroundjoin%
\definecolor{currentfill}{rgb}{0.121569,0.466667,0.705882}%
\pgfsetfillcolor{currentfill}%
\pgfsetfillopacity{0.997502}%
\pgfsetlinewidth{1.003750pt}%
\definecolor{currentstroke}{rgb}{0.121569,0.466667,0.705882}%
\pgfsetstrokecolor{currentstroke}%
\pgfsetstrokeopacity{0.997502}%
\pgfsetdash{}{0pt}%
\pgfpathmoveto{\pgfqpoint{2.506577in}{1.566154in}}%
\pgfpathcurveto{\pgfqpoint{2.514813in}{1.566154in}}{\pgfqpoint{2.522713in}{1.569427in}}{\pgfqpoint{2.528537in}{1.575251in}}%
\pgfpathcurveto{\pgfqpoint{2.534361in}{1.581075in}}{\pgfqpoint{2.537633in}{1.588975in}}{\pgfqpoint{2.537633in}{1.597211in}}%
\pgfpathcurveto{\pgfqpoint{2.537633in}{1.605447in}}{\pgfqpoint{2.534361in}{1.613347in}}{\pgfqpoint{2.528537in}{1.619171in}}%
\pgfpathcurveto{\pgfqpoint{2.522713in}{1.624995in}}{\pgfqpoint{2.514813in}{1.628267in}}{\pgfqpoint{2.506577in}{1.628267in}}%
\pgfpathcurveto{\pgfqpoint{2.498341in}{1.628267in}}{\pgfqpoint{2.490440in}{1.624995in}}{\pgfqpoint{2.484617in}{1.619171in}}%
\pgfpathcurveto{\pgfqpoint{2.478793in}{1.613347in}}{\pgfqpoint{2.475520in}{1.605447in}}{\pgfqpoint{2.475520in}{1.597211in}}%
\pgfpathcurveto{\pgfqpoint{2.475520in}{1.588975in}}{\pgfqpoint{2.478793in}{1.581075in}}{\pgfqpoint{2.484617in}{1.575251in}}%
\pgfpathcurveto{\pgfqpoint{2.490440in}{1.569427in}}{\pgfqpoint{2.498341in}{1.566154in}}{\pgfqpoint{2.506577in}{1.566154in}}%
\pgfpathclose%
\pgfusepath{stroke,fill}%
\end{pgfscope}%
\begin{pgfscope}%
\pgfpathrectangle{\pgfqpoint{0.100000in}{0.212622in}}{\pgfqpoint{3.696000in}{3.696000in}}%
\pgfusepath{clip}%
\pgfsetbuttcap%
\pgfsetroundjoin%
\definecolor{currentfill}{rgb}{0.121569,0.466667,0.705882}%
\pgfsetfillcolor{currentfill}%
\pgfsetfillopacity{0.997502}%
\pgfsetlinewidth{1.003750pt}%
\definecolor{currentstroke}{rgb}{0.121569,0.466667,0.705882}%
\pgfsetstrokecolor{currentstroke}%
\pgfsetstrokeopacity{0.997502}%
\pgfsetdash{}{0pt}%
\pgfpathmoveto{\pgfqpoint{2.506577in}{1.566154in}}%
\pgfpathcurveto{\pgfqpoint{2.514813in}{1.566154in}}{\pgfqpoint{2.522713in}{1.569427in}}{\pgfqpoint{2.528537in}{1.575251in}}%
\pgfpathcurveto{\pgfqpoint{2.534361in}{1.581074in}}{\pgfqpoint{2.537633in}{1.588975in}}{\pgfqpoint{2.537633in}{1.597211in}}%
\pgfpathcurveto{\pgfqpoint{2.537633in}{1.605447in}}{\pgfqpoint{2.534361in}{1.613347in}}{\pgfqpoint{2.528537in}{1.619171in}}%
\pgfpathcurveto{\pgfqpoint{2.522713in}{1.624995in}}{\pgfqpoint{2.514813in}{1.628267in}}{\pgfqpoint{2.506577in}{1.628267in}}%
\pgfpathcurveto{\pgfqpoint{2.498341in}{1.628267in}}{\pgfqpoint{2.490440in}{1.624995in}}{\pgfqpoint{2.484617in}{1.619171in}}%
\pgfpathcurveto{\pgfqpoint{2.478793in}{1.613347in}}{\pgfqpoint{2.475520in}{1.605447in}}{\pgfqpoint{2.475520in}{1.597211in}}%
\pgfpathcurveto{\pgfqpoint{2.475520in}{1.588975in}}{\pgfqpoint{2.478793in}{1.581074in}}{\pgfqpoint{2.484617in}{1.575251in}}%
\pgfpathcurveto{\pgfqpoint{2.490440in}{1.569427in}}{\pgfqpoint{2.498341in}{1.566154in}}{\pgfqpoint{2.506577in}{1.566154in}}%
\pgfpathclose%
\pgfusepath{stroke,fill}%
\end{pgfscope}%
\begin{pgfscope}%
\pgfpathrectangle{\pgfqpoint{0.100000in}{0.212622in}}{\pgfqpoint{3.696000in}{3.696000in}}%
\pgfusepath{clip}%
\pgfsetbuttcap%
\pgfsetroundjoin%
\definecolor{currentfill}{rgb}{0.121569,0.466667,0.705882}%
\pgfsetfillcolor{currentfill}%
\pgfsetfillopacity{0.997502}%
\pgfsetlinewidth{1.003750pt}%
\definecolor{currentstroke}{rgb}{0.121569,0.466667,0.705882}%
\pgfsetstrokecolor{currentstroke}%
\pgfsetstrokeopacity{0.997502}%
\pgfsetdash{}{0pt}%
\pgfpathmoveto{\pgfqpoint{2.506577in}{1.566154in}}%
\pgfpathcurveto{\pgfqpoint{2.514813in}{1.566154in}}{\pgfqpoint{2.522713in}{1.569427in}}{\pgfqpoint{2.528537in}{1.575251in}}%
\pgfpathcurveto{\pgfqpoint{2.534361in}{1.581074in}}{\pgfqpoint{2.537633in}{1.588975in}}{\pgfqpoint{2.537633in}{1.597211in}}%
\pgfpathcurveto{\pgfqpoint{2.537633in}{1.605447in}}{\pgfqpoint{2.534361in}{1.613347in}}{\pgfqpoint{2.528537in}{1.619171in}}%
\pgfpathcurveto{\pgfqpoint{2.522713in}{1.624995in}}{\pgfqpoint{2.514813in}{1.628267in}}{\pgfqpoint{2.506577in}{1.628267in}}%
\pgfpathcurveto{\pgfqpoint{2.498341in}{1.628267in}}{\pgfqpoint{2.490440in}{1.624995in}}{\pgfqpoint{2.484617in}{1.619171in}}%
\pgfpathcurveto{\pgfqpoint{2.478793in}{1.613347in}}{\pgfqpoint{2.475520in}{1.605447in}}{\pgfqpoint{2.475520in}{1.597211in}}%
\pgfpathcurveto{\pgfqpoint{2.475520in}{1.588975in}}{\pgfqpoint{2.478793in}{1.581074in}}{\pgfqpoint{2.484617in}{1.575251in}}%
\pgfpathcurveto{\pgfqpoint{2.490440in}{1.569427in}}{\pgfqpoint{2.498341in}{1.566154in}}{\pgfqpoint{2.506577in}{1.566154in}}%
\pgfpathclose%
\pgfusepath{stroke,fill}%
\end{pgfscope}%
\begin{pgfscope}%
\pgfpathrectangle{\pgfqpoint{0.100000in}{0.212622in}}{\pgfqpoint{3.696000in}{3.696000in}}%
\pgfusepath{clip}%
\pgfsetbuttcap%
\pgfsetroundjoin%
\definecolor{currentfill}{rgb}{0.121569,0.466667,0.705882}%
\pgfsetfillcolor{currentfill}%
\pgfsetfillopacity{0.997502}%
\pgfsetlinewidth{1.003750pt}%
\definecolor{currentstroke}{rgb}{0.121569,0.466667,0.705882}%
\pgfsetstrokecolor{currentstroke}%
\pgfsetstrokeopacity{0.997502}%
\pgfsetdash{}{0pt}%
\pgfpathmoveto{\pgfqpoint{2.506577in}{1.566154in}}%
\pgfpathcurveto{\pgfqpoint{2.514813in}{1.566154in}}{\pgfqpoint{2.522713in}{1.569427in}}{\pgfqpoint{2.528537in}{1.575251in}}%
\pgfpathcurveto{\pgfqpoint{2.534361in}{1.581074in}}{\pgfqpoint{2.537633in}{1.588975in}}{\pgfqpoint{2.537633in}{1.597211in}}%
\pgfpathcurveto{\pgfqpoint{2.537633in}{1.605447in}}{\pgfqpoint{2.534361in}{1.613347in}}{\pgfqpoint{2.528537in}{1.619171in}}%
\pgfpathcurveto{\pgfqpoint{2.522713in}{1.624995in}}{\pgfqpoint{2.514813in}{1.628267in}}{\pgfqpoint{2.506577in}{1.628267in}}%
\pgfpathcurveto{\pgfqpoint{2.498341in}{1.628267in}}{\pgfqpoint{2.490440in}{1.624995in}}{\pgfqpoint{2.484617in}{1.619171in}}%
\pgfpathcurveto{\pgfqpoint{2.478793in}{1.613347in}}{\pgfqpoint{2.475520in}{1.605447in}}{\pgfqpoint{2.475520in}{1.597211in}}%
\pgfpathcurveto{\pgfqpoint{2.475520in}{1.588975in}}{\pgfqpoint{2.478793in}{1.581074in}}{\pgfqpoint{2.484617in}{1.575251in}}%
\pgfpathcurveto{\pgfqpoint{2.490440in}{1.569427in}}{\pgfqpoint{2.498341in}{1.566154in}}{\pgfqpoint{2.506577in}{1.566154in}}%
\pgfpathclose%
\pgfusepath{stroke,fill}%
\end{pgfscope}%
\begin{pgfscope}%
\pgfpathrectangle{\pgfqpoint{0.100000in}{0.212622in}}{\pgfqpoint{3.696000in}{3.696000in}}%
\pgfusepath{clip}%
\pgfsetbuttcap%
\pgfsetroundjoin%
\definecolor{currentfill}{rgb}{0.121569,0.466667,0.705882}%
\pgfsetfillcolor{currentfill}%
\pgfsetfillopacity{0.997502}%
\pgfsetlinewidth{1.003750pt}%
\definecolor{currentstroke}{rgb}{0.121569,0.466667,0.705882}%
\pgfsetstrokecolor{currentstroke}%
\pgfsetstrokeopacity{0.997502}%
\pgfsetdash{}{0pt}%
\pgfpathmoveto{\pgfqpoint{2.506577in}{1.566154in}}%
\pgfpathcurveto{\pgfqpoint{2.514813in}{1.566154in}}{\pgfqpoint{2.522713in}{1.569427in}}{\pgfqpoint{2.528537in}{1.575251in}}%
\pgfpathcurveto{\pgfqpoint{2.534361in}{1.581074in}}{\pgfqpoint{2.537633in}{1.588975in}}{\pgfqpoint{2.537633in}{1.597211in}}%
\pgfpathcurveto{\pgfqpoint{2.537633in}{1.605447in}}{\pgfqpoint{2.534361in}{1.613347in}}{\pgfqpoint{2.528537in}{1.619171in}}%
\pgfpathcurveto{\pgfqpoint{2.522713in}{1.624995in}}{\pgfqpoint{2.514813in}{1.628267in}}{\pgfqpoint{2.506577in}{1.628267in}}%
\pgfpathcurveto{\pgfqpoint{2.498341in}{1.628267in}}{\pgfqpoint{2.490440in}{1.624995in}}{\pgfqpoint{2.484617in}{1.619171in}}%
\pgfpathcurveto{\pgfqpoint{2.478793in}{1.613347in}}{\pgfqpoint{2.475520in}{1.605447in}}{\pgfqpoint{2.475520in}{1.597211in}}%
\pgfpathcurveto{\pgfqpoint{2.475520in}{1.588975in}}{\pgfqpoint{2.478793in}{1.581074in}}{\pgfqpoint{2.484617in}{1.575251in}}%
\pgfpathcurveto{\pgfqpoint{2.490440in}{1.569427in}}{\pgfqpoint{2.498341in}{1.566154in}}{\pgfqpoint{2.506577in}{1.566154in}}%
\pgfpathclose%
\pgfusepath{stroke,fill}%
\end{pgfscope}%
\begin{pgfscope}%
\pgfpathrectangle{\pgfqpoint{0.100000in}{0.212622in}}{\pgfqpoint{3.696000in}{3.696000in}}%
\pgfusepath{clip}%
\pgfsetbuttcap%
\pgfsetroundjoin%
\definecolor{currentfill}{rgb}{0.121569,0.466667,0.705882}%
\pgfsetfillcolor{currentfill}%
\pgfsetfillopacity{0.997502}%
\pgfsetlinewidth{1.003750pt}%
\definecolor{currentstroke}{rgb}{0.121569,0.466667,0.705882}%
\pgfsetstrokecolor{currentstroke}%
\pgfsetstrokeopacity{0.997502}%
\pgfsetdash{}{0pt}%
\pgfpathmoveto{\pgfqpoint{2.506577in}{1.566154in}}%
\pgfpathcurveto{\pgfqpoint{2.514813in}{1.566154in}}{\pgfqpoint{2.522713in}{1.569427in}}{\pgfqpoint{2.528537in}{1.575251in}}%
\pgfpathcurveto{\pgfqpoint{2.534361in}{1.581074in}}{\pgfqpoint{2.537633in}{1.588975in}}{\pgfqpoint{2.537633in}{1.597211in}}%
\pgfpathcurveto{\pgfqpoint{2.537633in}{1.605447in}}{\pgfqpoint{2.534361in}{1.613347in}}{\pgfqpoint{2.528537in}{1.619171in}}%
\pgfpathcurveto{\pgfqpoint{2.522713in}{1.624995in}}{\pgfqpoint{2.514813in}{1.628267in}}{\pgfqpoint{2.506577in}{1.628267in}}%
\pgfpathcurveto{\pgfqpoint{2.498341in}{1.628267in}}{\pgfqpoint{2.490440in}{1.624995in}}{\pgfqpoint{2.484617in}{1.619171in}}%
\pgfpathcurveto{\pgfqpoint{2.478793in}{1.613347in}}{\pgfqpoint{2.475520in}{1.605447in}}{\pgfqpoint{2.475520in}{1.597211in}}%
\pgfpathcurveto{\pgfqpoint{2.475520in}{1.588975in}}{\pgfqpoint{2.478793in}{1.581074in}}{\pgfqpoint{2.484617in}{1.575251in}}%
\pgfpathcurveto{\pgfqpoint{2.490440in}{1.569427in}}{\pgfqpoint{2.498341in}{1.566154in}}{\pgfqpoint{2.506577in}{1.566154in}}%
\pgfpathclose%
\pgfusepath{stroke,fill}%
\end{pgfscope}%
\begin{pgfscope}%
\pgfpathrectangle{\pgfqpoint{0.100000in}{0.212622in}}{\pgfqpoint{3.696000in}{3.696000in}}%
\pgfusepath{clip}%
\pgfsetbuttcap%
\pgfsetroundjoin%
\definecolor{currentfill}{rgb}{0.121569,0.466667,0.705882}%
\pgfsetfillcolor{currentfill}%
\pgfsetfillopacity{0.997502}%
\pgfsetlinewidth{1.003750pt}%
\definecolor{currentstroke}{rgb}{0.121569,0.466667,0.705882}%
\pgfsetstrokecolor{currentstroke}%
\pgfsetstrokeopacity{0.997502}%
\pgfsetdash{}{0pt}%
\pgfpathmoveto{\pgfqpoint{2.506577in}{1.566154in}}%
\pgfpathcurveto{\pgfqpoint{2.514813in}{1.566154in}}{\pgfqpoint{2.522713in}{1.569427in}}{\pgfqpoint{2.528537in}{1.575251in}}%
\pgfpathcurveto{\pgfqpoint{2.534361in}{1.581074in}}{\pgfqpoint{2.537633in}{1.588975in}}{\pgfqpoint{2.537633in}{1.597211in}}%
\pgfpathcurveto{\pgfqpoint{2.537633in}{1.605447in}}{\pgfqpoint{2.534361in}{1.613347in}}{\pgfqpoint{2.528537in}{1.619171in}}%
\pgfpathcurveto{\pgfqpoint{2.522713in}{1.624995in}}{\pgfqpoint{2.514813in}{1.628267in}}{\pgfqpoint{2.506577in}{1.628267in}}%
\pgfpathcurveto{\pgfqpoint{2.498341in}{1.628267in}}{\pgfqpoint{2.490440in}{1.624995in}}{\pgfqpoint{2.484617in}{1.619171in}}%
\pgfpathcurveto{\pgfqpoint{2.478793in}{1.613347in}}{\pgfqpoint{2.475520in}{1.605447in}}{\pgfqpoint{2.475520in}{1.597211in}}%
\pgfpathcurveto{\pgfqpoint{2.475520in}{1.588975in}}{\pgfqpoint{2.478793in}{1.581074in}}{\pgfqpoint{2.484617in}{1.575251in}}%
\pgfpathcurveto{\pgfqpoint{2.490440in}{1.569427in}}{\pgfqpoint{2.498341in}{1.566154in}}{\pgfqpoint{2.506577in}{1.566154in}}%
\pgfpathclose%
\pgfusepath{stroke,fill}%
\end{pgfscope}%
\begin{pgfscope}%
\pgfpathrectangle{\pgfqpoint{0.100000in}{0.212622in}}{\pgfqpoint{3.696000in}{3.696000in}}%
\pgfusepath{clip}%
\pgfsetbuttcap%
\pgfsetroundjoin%
\definecolor{currentfill}{rgb}{0.121569,0.466667,0.705882}%
\pgfsetfillcolor{currentfill}%
\pgfsetfillopacity{0.997502}%
\pgfsetlinewidth{1.003750pt}%
\definecolor{currentstroke}{rgb}{0.121569,0.466667,0.705882}%
\pgfsetstrokecolor{currentstroke}%
\pgfsetstrokeopacity{0.997502}%
\pgfsetdash{}{0pt}%
\pgfpathmoveto{\pgfqpoint{2.506577in}{1.566154in}}%
\pgfpathcurveto{\pgfqpoint{2.514813in}{1.566154in}}{\pgfqpoint{2.522713in}{1.569427in}}{\pgfqpoint{2.528537in}{1.575251in}}%
\pgfpathcurveto{\pgfqpoint{2.534361in}{1.581074in}}{\pgfqpoint{2.537633in}{1.588975in}}{\pgfqpoint{2.537633in}{1.597211in}}%
\pgfpathcurveto{\pgfqpoint{2.537633in}{1.605447in}}{\pgfqpoint{2.534361in}{1.613347in}}{\pgfqpoint{2.528537in}{1.619171in}}%
\pgfpathcurveto{\pgfqpoint{2.522713in}{1.624995in}}{\pgfqpoint{2.514813in}{1.628267in}}{\pgfqpoint{2.506577in}{1.628267in}}%
\pgfpathcurveto{\pgfqpoint{2.498341in}{1.628267in}}{\pgfqpoint{2.490440in}{1.624995in}}{\pgfqpoint{2.484617in}{1.619171in}}%
\pgfpathcurveto{\pgfqpoint{2.478793in}{1.613347in}}{\pgfqpoint{2.475520in}{1.605447in}}{\pgfqpoint{2.475520in}{1.597211in}}%
\pgfpathcurveto{\pgfqpoint{2.475520in}{1.588975in}}{\pgfqpoint{2.478793in}{1.581074in}}{\pgfqpoint{2.484617in}{1.575251in}}%
\pgfpathcurveto{\pgfqpoint{2.490440in}{1.569427in}}{\pgfqpoint{2.498341in}{1.566154in}}{\pgfqpoint{2.506577in}{1.566154in}}%
\pgfpathclose%
\pgfusepath{stroke,fill}%
\end{pgfscope}%
\begin{pgfscope}%
\pgfpathrectangle{\pgfqpoint{0.100000in}{0.212622in}}{\pgfqpoint{3.696000in}{3.696000in}}%
\pgfusepath{clip}%
\pgfsetbuttcap%
\pgfsetroundjoin%
\definecolor{currentfill}{rgb}{0.121569,0.466667,0.705882}%
\pgfsetfillcolor{currentfill}%
\pgfsetfillopacity{0.998607}%
\pgfsetlinewidth{1.003750pt}%
\definecolor{currentstroke}{rgb}{0.121569,0.466667,0.705882}%
\pgfsetstrokecolor{currentstroke}%
\pgfsetstrokeopacity{0.998607}%
\pgfsetdash{}{0pt}%
\pgfpathmoveto{\pgfqpoint{2.507481in}{1.561932in}}%
\pgfpathcurveto{\pgfqpoint{2.515717in}{1.561932in}}{\pgfqpoint{2.523618in}{1.565204in}}{\pgfqpoint{2.529441in}{1.571028in}}%
\pgfpathcurveto{\pgfqpoint{2.535265in}{1.576852in}}{\pgfqpoint{2.538538in}{1.584752in}}{\pgfqpoint{2.538538in}{1.592988in}}%
\pgfpathcurveto{\pgfqpoint{2.538538in}{1.601224in}}{\pgfqpoint{2.535265in}{1.609124in}}{\pgfqpoint{2.529441in}{1.614948in}}%
\pgfpathcurveto{\pgfqpoint{2.523618in}{1.620772in}}{\pgfqpoint{2.515717in}{1.624045in}}{\pgfqpoint{2.507481in}{1.624045in}}%
\pgfpathcurveto{\pgfqpoint{2.499245in}{1.624045in}}{\pgfqpoint{2.491345in}{1.620772in}}{\pgfqpoint{2.485521in}{1.614948in}}%
\pgfpathcurveto{\pgfqpoint{2.479697in}{1.609124in}}{\pgfqpoint{2.476425in}{1.601224in}}{\pgfqpoint{2.476425in}{1.592988in}}%
\pgfpathcurveto{\pgfqpoint{2.476425in}{1.584752in}}{\pgfqpoint{2.479697in}{1.576852in}}{\pgfqpoint{2.485521in}{1.571028in}}%
\pgfpathcurveto{\pgfqpoint{2.491345in}{1.565204in}}{\pgfqpoint{2.499245in}{1.561932in}}{\pgfqpoint{2.507481in}{1.561932in}}%
\pgfpathclose%
\pgfusepath{stroke,fill}%
\end{pgfscope}%
\begin{pgfscope}%
\pgfpathrectangle{\pgfqpoint{0.100000in}{0.212622in}}{\pgfqpoint{3.696000in}{3.696000in}}%
\pgfusepath{clip}%
\pgfsetbuttcap%
\pgfsetroundjoin%
\definecolor{currentfill}{rgb}{0.121569,0.466667,0.705882}%
\pgfsetfillcolor{currentfill}%
\pgfsetfillopacity{0.999263}%
\pgfsetlinewidth{1.003750pt}%
\definecolor{currentstroke}{rgb}{0.121569,0.466667,0.705882}%
\pgfsetstrokecolor{currentstroke}%
\pgfsetstrokeopacity{0.999263}%
\pgfsetdash{}{0pt}%
\pgfpathmoveto{\pgfqpoint{2.507902in}{1.559504in}}%
\pgfpathcurveto{\pgfqpoint{2.516138in}{1.559504in}}{\pgfqpoint{2.524038in}{1.562776in}}{\pgfqpoint{2.529862in}{1.568600in}}%
\pgfpathcurveto{\pgfqpoint{2.535686in}{1.574424in}}{\pgfqpoint{2.538958in}{1.582324in}}{\pgfqpoint{2.538958in}{1.590560in}}%
\pgfpathcurveto{\pgfqpoint{2.538958in}{1.598796in}}{\pgfqpoint{2.535686in}{1.606697in}}{\pgfqpoint{2.529862in}{1.612520in}}%
\pgfpathcurveto{\pgfqpoint{2.524038in}{1.618344in}}{\pgfqpoint{2.516138in}{1.621617in}}{\pgfqpoint{2.507902in}{1.621617in}}%
\pgfpathcurveto{\pgfqpoint{2.499666in}{1.621617in}}{\pgfqpoint{2.491766in}{1.618344in}}{\pgfqpoint{2.485942in}{1.612520in}}%
\pgfpathcurveto{\pgfqpoint{2.480118in}{1.606697in}}{\pgfqpoint{2.476845in}{1.598796in}}{\pgfqpoint{2.476845in}{1.590560in}}%
\pgfpathcurveto{\pgfqpoint{2.476845in}{1.582324in}}{\pgfqpoint{2.480118in}{1.574424in}}{\pgfqpoint{2.485942in}{1.568600in}}%
\pgfpathcurveto{\pgfqpoint{2.491766in}{1.562776in}}{\pgfqpoint{2.499666in}{1.559504in}}{\pgfqpoint{2.507902in}{1.559504in}}%
\pgfpathclose%
\pgfusepath{stroke,fill}%
\end{pgfscope}%
\begin{pgfscope}%
\pgfpathrectangle{\pgfqpoint{0.100000in}{0.212622in}}{\pgfqpoint{3.696000in}{3.696000in}}%
\pgfusepath{clip}%
\pgfsetbuttcap%
\pgfsetroundjoin%
\definecolor{currentfill}{rgb}{0.121569,0.466667,0.705882}%
\pgfsetfillcolor{currentfill}%
\pgfsetfillopacity{0.999586}%
\pgfsetlinewidth{1.003750pt}%
\definecolor{currentstroke}{rgb}{0.121569,0.466667,0.705882}%
\pgfsetstrokecolor{currentstroke}%
\pgfsetstrokeopacity{0.999586}%
\pgfsetdash{}{0pt}%
\pgfpathmoveto{\pgfqpoint{2.508264in}{1.558481in}}%
\pgfpathcurveto{\pgfqpoint{2.516500in}{1.558481in}}{\pgfqpoint{2.524401in}{1.561753in}}{\pgfqpoint{2.530224in}{1.567577in}}%
\pgfpathcurveto{\pgfqpoint{2.536048in}{1.573401in}}{\pgfqpoint{2.539321in}{1.581301in}}{\pgfqpoint{2.539321in}{1.589537in}}%
\pgfpathcurveto{\pgfqpoint{2.539321in}{1.597774in}}{\pgfqpoint{2.536048in}{1.605674in}}{\pgfqpoint{2.530224in}{1.611498in}}%
\pgfpathcurveto{\pgfqpoint{2.524401in}{1.617322in}}{\pgfqpoint{2.516500in}{1.620594in}}{\pgfqpoint{2.508264in}{1.620594in}}%
\pgfpathcurveto{\pgfqpoint{2.500028in}{1.620594in}}{\pgfqpoint{2.492128in}{1.617322in}}{\pgfqpoint{2.486304in}{1.611498in}}%
\pgfpathcurveto{\pgfqpoint{2.480480in}{1.605674in}}{\pgfqpoint{2.477208in}{1.597774in}}{\pgfqpoint{2.477208in}{1.589537in}}%
\pgfpathcurveto{\pgfqpoint{2.477208in}{1.581301in}}{\pgfqpoint{2.480480in}{1.573401in}}{\pgfqpoint{2.486304in}{1.567577in}}%
\pgfpathcurveto{\pgfqpoint{2.492128in}{1.561753in}}{\pgfqpoint{2.500028in}{1.558481in}}{\pgfqpoint{2.508264in}{1.558481in}}%
\pgfpathclose%
\pgfusepath{stroke,fill}%
\end{pgfscope}%
\begin{pgfscope}%
\pgfpathrectangle{\pgfqpoint{0.100000in}{0.212622in}}{\pgfqpoint{3.696000in}{3.696000in}}%
\pgfusepath{clip}%
\pgfsetbuttcap%
\pgfsetroundjoin%
\definecolor{currentfill}{rgb}{0.121569,0.466667,0.705882}%
\pgfsetfillcolor{currentfill}%
\pgfsetfillopacity{0.999780}%
\pgfsetlinewidth{1.003750pt}%
\definecolor{currentstroke}{rgb}{0.121569,0.466667,0.705882}%
\pgfsetstrokecolor{currentstroke}%
\pgfsetstrokeopacity{0.999780}%
\pgfsetdash{}{0pt}%
\pgfpathmoveto{\pgfqpoint{2.508399in}{1.557759in}}%
\pgfpathcurveto{\pgfqpoint{2.516636in}{1.557759in}}{\pgfqpoint{2.524536in}{1.561032in}}{\pgfqpoint{2.530360in}{1.566856in}}%
\pgfpathcurveto{\pgfqpoint{2.536183in}{1.572679in}}{\pgfqpoint{2.539456in}{1.580580in}}{\pgfqpoint{2.539456in}{1.588816in}}%
\pgfpathcurveto{\pgfqpoint{2.539456in}{1.597052in}}{\pgfqpoint{2.536183in}{1.604952in}}{\pgfqpoint{2.530360in}{1.610776in}}%
\pgfpathcurveto{\pgfqpoint{2.524536in}{1.616600in}}{\pgfqpoint{2.516636in}{1.619872in}}{\pgfqpoint{2.508399in}{1.619872in}}%
\pgfpathcurveto{\pgfqpoint{2.500163in}{1.619872in}}{\pgfqpoint{2.492263in}{1.616600in}}{\pgfqpoint{2.486439in}{1.610776in}}%
\pgfpathcurveto{\pgfqpoint{2.480615in}{1.604952in}}{\pgfqpoint{2.477343in}{1.597052in}}{\pgfqpoint{2.477343in}{1.588816in}}%
\pgfpathcurveto{\pgfqpoint{2.477343in}{1.580580in}}{\pgfqpoint{2.480615in}{1.572679in}}{\pgfqpoint{2.486439in}{1.566856in}}%
\pgfpathcurveto{\pgfqpoint{2.492263in}{1.561032in}}{\pgfqpoint{2.500163in}{1.557759in}}{\pgfqpoint{2.508399in}{1.557759in}}%
\pgfpathclose%
\pgfusepath{stroke,fill}%
\end{pgfscope}%
\begin{pgfscope}%
\pgfpathrectangle{\pgfqpoint{0.100000in}{0.212622in}}{\pgfqpoint{3.696000in}{3.696000in}}%
\pgfusepath{clip}%
\pgfsetbuttcap%
\pgfsetroundjoin%
\definecolor{currentfill}{rgb}{0.121569,0.466667,0.705882}%
\pgfsetfillcolor{currentfill}%
\pgfsetfillopacity{0.999877}%
\pgfsetlinewidth{1.003750pt}%
\definecolor{currentstroke}{rgb}{0.121569,0.466667,0.705882}%
\pgfsetstrokecolor{currentstroke}%
\pgfsetstrokeopacity{0.999877}%
\pgfsetdash{}{0pt}%
\pgfpathmoveto{\pgfqpoint{2.508520in}{1.557487in}}%
\pgfpathcurveto{\pgfqpoint{2.516756in}{1.557487in}}{\pgfqpoint{2.524656in}{1.560759in}}{\pgfqpoint{2.530480in}{1.566583in}}%
\pgfpathcurveto{\pgfqpoint{2.536304in}{1.572407in}}{\pgfqpoint{2.539576in}{1.580307in}}{\pgfqpoint{2.539576in}{1.588544in}}%
\pgfpathcurveto{\pgfqpoint{2.539576in}{1.596780in}}{\pgfqpoint{2.536304in}{1.604680in}}{\pgfqpoint{2.530480in}{1.610504in}}%
\pgfpathcurveto{\pgfqpoint{2.524656in}{1.616328in}}{\pgfqpoint{2.516756in}{1.619600in}}{\pgfqpoint{2.508520in}{1.619600in}}%
\pgfpathcurveto{\pgfqpoint{2.500283in}{1.619600in}}{\pgfqpoint{2.492383in}{1.616328in}}{\pgfqpoint{2.486560in}{1.610504in}}%
\pgfpathcurveto{\pgfqpoint{2.480736in}{1.604680in}}{\pgfqpoint{2.477463in}{1.596780in}}{\pgfqpoint{2.477463in}{1.588544in}}%
\pgfpathcurveto{\pgfqpoint{2.477463in}{1.580307in}}{\pgfqpoint{2.480736in}{1.572407in}}{\pgfqpoint{2.486560in}{1.566583in}}%
\pgfpathcurveto{\pgfqpoint{2.492383in}{1.560759in}}{\pgfqpoint{2.500283in}{1.557487in}}{\pgfqpoint{2.508520in}{1.557487in}}%
\pgfpathclose%
\pgfusepath{stroke,fill}%
\end{pgfscope}%
\begin{pgfscope}%
\pgfpathrectangle{\pgfqpoint{0.100000in}{0.212622in}}{\pgfqpoint{3.696000in}{3.696000in}}%
\pgfusepath{clip}%
\pgfsetbuttcap%
\pgfsetroundjoin%
\definecolor{currentfill}{rgb}{0.121569,0.466667,0.705882}%
\pgfsetfillcolor{currentfill}%
\pgfsetfillopacity{0.999941}%
\pgfsetlinewidth{1.003750pt}%
\definecolor{currentstroke}{rgb}{0.121569,0.466667,0.705882}%
\pgfsetstrokecolor{currentstroke}%
\pgfsetstrokeopacity{0.999941}%
\pgfsetdash{}{0pt}%
\pgfpathmoveto{\pgfqpoint{2.508563in}{1.557293in}}%
\pgfpathcurveto{\pgfqpoint{2.516799in}{1.557293in}}{\pgfqpoint{2.524699in}{1.560565in}}{\pgfqpoint{2.530523in}{1.566389in}}%
\pgfpathcurveto{\pgfqpoint{2.536347in}{1.572213in}}{\pgfqpoint{2.539619in}{1.580113in}}{\pgfqpoint{2.539619in}{1.588349in}}%
\pgfpathcurveto{\pgfqpoint{2.539619in}{1.596585in}}{\pgfqpoint{2.536347in}{1.604485in}}{\pgfqpoint{2.530523in}{1.610309in}}%
\pgfpathcurveto{\pgfqpoint{2.524699in}{1.616133in}}{\pgfqpoint{2.516799in}{1.619406in}}{\pgfqpoint{2.508563in}{1.619406in}}%
\pgfpathcurveto{\pgfqpoint{2.500326in}{1.619406in}}{\pgfqpoint{2.492426in}{1.616133in}}{\pgfqpoint{2.486602in}{1.610309in}}%
\pgfpathcurveto{\pgfqpoint{2.480778in}{1.604485in}}{\pgfqpoint{2.477506in}{1.596585in}}{\pgfqpoint{2.477506in}{1.588349in}}%
\pgfpathcurveto{\pgfqpoint{2.477506in}{1.580113in}}{\pgfqpoint{2.480778in}{1.572213in}}{\pgfqpoint{2.486602in}{1.566389in}}%
\pgfpathcurveto{\pgfqpoint{2.492426in}{1.560565in}}{\pgfqpoint{2.500326in}{1.557293in}}{\pgfqpoint{2.508563in}{1.557293in}}%
\pgfpathclose%
\pgfusepath{stroke,fill}%
\end{pgfscope}%
\begin{pgfscope}%
\pgfpathrectangle{\pgfqpoint{0.100000in}{0.212622in}}{\pgfqpoint{3.696000in}{3.696000in}}%
\pgfusepath{clip}%
\pgfsetbuttcap%
\pgfsetroundjoin%
\definecolor{currentfill}{rgb}{0.121569,0.466667,0.705882}%
\pgfsetfillcolor{currentfill}%
\pgfsetfillopacity{0.999975}%
\pgfsetlinewidth{1.003750pt}%
\definecolor{currentstroke}{rgb}{0.121569,0.466667,0.705882}%
\pgfsetstrokecolor{currentstroke}%
\pgfsetstrokeopacity{0.999975}%
\pgfsetdash{}{0pt}%
\pgfpathmoveto{\pgfqpoint{2.508588in}{1.557186in}}%
\pgfpathcurveto{\pgfqpoint{2.516824in}{1.557186in}}{\pgfqpoint{2.524724in}{1.560458in}}{\pgfqpoint{2.530548in}{1.566282in}}%
\pgfpathcurveto{\pgfqpoint{2.536372in}{1.572106in}}{\pgfqpoint{2.539644in}{1.580006in}}{\pgfqpoint{2.539644in}{1.588242in}}%
\pgfpathcurveto{\pgfqpoint{2.539644in}{1.596478in}}{\pgfqpoint{2.536372in}{1.604379in}}{\pgfqpoint{2.530548in}{1.610202in}}%
\pgfpathcurveto{\pgfqpoint{2.524724in}{1.616026in}}{\pgfqpoint{2.516824in}{1.619299in}}{\pgfqpoint{2.508588in}{1.619299in}}%
\pgfpathcurveto{\pgfqpoint{2.500351in}{1.619299in}}{\pgfqpoint{2.492451in}{1.616026in}}{\pgfqpoint{2.486627in}{1.610202in}}%
\pgfpathcurveto{\pgfqpoint{2.480803in}{1.604379in}}{\pgfqpoint{2.477531in}{1.596478in}}{\pgfqpoint{2.477531in}{1.588242in}}%
\pgfpathcurveto{\pgfqpoint{2.477531in}{1.580006in}}{\pgfqpoint{2.480803in}{1.572106in}}{\pgfqpoint{2.486627in}{1.566282in}}%
\pgfpathcurveto{\pgfqpoint{2.492451in}{1.560458in}}{\pgfqpoint{2.500351in}{1.557186in}}{\pgfqpoint{2.508588in}{1.557186in}}%
\pgfpathclose%
\pgfusepath{stroke,fill}%
\end{pgfscope}%
\begin{pgfscope}%
\pgfpathrectangle{\pgfqpoint{0.100000in}{0.212622in}}{\pgfqpoint{3.696000in}{3.696000in}}%
\pgfusepath{clip}%
\pgfsetbuttcap%
\pgfsetroundjoin%
\definecolor{currentfill}{rgb}{0.121569,0.466667,0.705882}%
\pgfsetfillcolor{currentfill}%
\pgfsetfillopacity{0.999990}%
\pgfsetlinewidth{1.003750pt}%
\definecolor{currentstroke}{rgb}{0.121569,0.466667,0.705882}%
\pgfsetstrokecolor{currentstroke}%
\pgfsetstrokeopacity{0.999990}%
\pgfsetdash{}{0pt}%
\pgfpathmoveto{\pgfqpoint{2.508611in}{1.557150in}}%
\pgfpathcurveto{\pgfqpoint{2.516847in}{1.557150in}}{\pgfqpoint{2.524747in}{1.560422in}}{\pgfqpoint{2.530571in}{1.566246in}}%
\pgfpathcurveto{\pgfqpoint{2.536395in}{1.572070in}}{\pgfqpoint{2.539668in}{1.579970in}}{\pgfqpoint{2.539668in}{1.588207in}}%
\pgfpathcurveto{\pgfqpoint{2.539668in}{1.596443in}}{\pgfqpoint{2.536395in}{1.604343in}}{\pgfqpoint{2.530571in}{1.610167in}}%
\pgfpathcurveto{\pgfqpoint{2.524747in}{1.615991in}}{\pgfqpoint{2.516847in}{1.619263in}}{\pgfqpoint{2.508611in}{1.619263in}}%
\pgfpathcurveto{\pgfqpoint{2.500375in}{1.619263in}}{\pgfqpoint{2.492475in}{1.615991in}}{\pgfqpoint{2.486651in}{1.610167in}}%
\pgfpathcurveto{\pgfqpoint{2.480827in}{1.604343in}}{\pgfqpoint{2.477555in}{1.596443in}}{\pgfqpoint{2.477555in}{1.588207in}}%
\pgfpathcurveto{\pgfqpoint{2.477555in}{1.579970in}}{\pgfqpoint{2.480827in}{1.572070in}}{\pgfqpoint{2.486651in}{1.566246in}}%
\pgfpathcurveto{\pgfqpoint{2.492475in}{1.560422in}}{\pgfqpoint{2.500375in}{1.557150in}}{\pgfqpoint{2.508611in}{1.557150in}}%
\pgfpathclose%
\pgfusepath{stroke,fill}%
\end{pgfscope}%
\begin{pgfscope}%
\pgfpathrectangle{\pgfqpoint{0.100000in}{0.212622in}}{\pgfqpoint{3.696000in}{3.696000in}}%
\pgfusepath{clip}%
\pgfsetbuttcap%
\pgfsetroundjoin%
\definecolor{currentfill}{rgb}{0.121569,0.466667,0.705882}%
\pgfsetfillcolor{currentfill}%
\pgfsetfillopacity{0.999998}%
\pgfsetlinewidth{1.003750pt}%
\definecolor{currentstroke}{rgb}{0.121569,0.466667,0.705882}%
\pgfsetstrokecolor{currentstroke}%
\pgfsetstrokeopacity{0.999998}%
\pgfsetdash{}{0pt}%
\pgfpathmoveto{\pgfqpoint{2.508628in}{1.557144in}}%
\pgfpathcurveto{\pgfqpoint{2.516865in}{1.557144in}}{\pgfqpoint{2.524765in}{1.560416in}}{\pgfqpoint{2.530589in}{1.566240in}}%
\pgfpathcurveto{\pgfqpoint{2.536413in}{1.572064in}}{\pgfqpoint{2.539685in}{1.579964in}}{\pgfqpoint{2.539685in}{1.588200in}}%
\pgfpathcurveto{\pgfqpoint{2.539685in}{1.596436in}}{\pgfqpoint{2.536413in}{1.604336in}}{\pgfqpoint{2.530589in}{1.610160in}}%
\pgfpathcurveto{\pgfqpoint{2.524765in}{1.615984in}}{\pgfqpoint{2.516865in}{1.619257in}}{\pgfqpoint{2.508628in}{1.619257in}}%
\pgfpathcurveto{\pgfqpoint{2.500392in}{1.619257in}}{\pgfqpoint{2.492492in}{1.615984in}}{\pgfqpoint{2.486668in}{1.610160in}}%
\pgfpathcurveto{\pgfqpoint{2.480844in}{1.604336in}}{\pgfqpoint{2.477572in}{1.596436in}}{\pgfqpoint{2.477572in}{1.588200in}}%
\pgfpathcurveto{\pgfqpoint{2.477572in}{1.579964in}}{\pgfqpoint{2.480844in}{1.572064in}}{\pgfqpoint{2.486668in}{1.566240in}}%
\pgfpathcurveto{\pgfqpoint{2.492492in}{1.560416in}}{\pgfqpoint{2.500392in}{1.557144in}}{\pgfqpoint{2.508628in}{1.557144in}}%
\pgfpathclose%
\pgfusepath{stroke,fill}%
\end{pgfscope}%
\begin{pgfscope}%
\pgfpathrectangle{\pgfqpoint{0.100000in}{0.212622in}}{\pgfqpoint{3.696000in}{3.696000in}}%
\pgfusepath{clip}%
\pgfsetbuttcap%
\pgfsetroundjoin%
\definecolor{currentfill}{rgb}{0.121569,0.466667,0.705882}%
\pgfsetfillcolor{currentfill}%
\pgfsetlinewidth{1.003750pt}%
\definecolor{currentstroke}{rgb}{0.121569,0.466667,0.705882}%
\pgfsetstrokecolor{currentstroke}%
\pgfsetdash{}{0pt}%
\pgfpathmoveto{\pgfqpoint{2.508644in}{1.557152in}}%
\pgfpathcurveto{\pgfqpoint{2.516880in}{1.557152in}}{\pgfqpoint{2.524780in}{1.560425in}}{\pgfqpoint{2.530604in}{1.566249in}}%
\pgfpathcurveto{\pgfqpoint{2.536428in}{1.572073in}}{\pgfqpoint{2.539700in}{1.579973in}}{\pgfqpoint{2.539700in}{1.588209in}}%
\pgfpathcurveto{\pgfqpoint{2.539700in}{1.596445in}}{\pgfqpoint{2.536428in}{1.604345in}}{\pgfqpoint{2.530604in}{1.610169in}}%
\pgfpathcurveto{\pgfqpoint{2.524780in}{1.615993in}}{\pgfqpoint{2.516880in}{1.619265in}}{\pgfqpoint{2.508644in}{1.619265in}}%
\pgfpathcurveto{\pgfqpoint{2.500408in}{1.619265in}}{\pgfqpoint{2.492508in}{1.615993in}}{\pgfqpoint{2.486684in}{1.610169in}}%
\pgfpathcurveto{\pgfqpoint{2.480860in}{1.604345in}}{\pgfqpoint{2.477587in}{1.596445in}}{\pgfqpoint{2.477587in}{1.588209in}}%
\pgfpathcurveto{\pgfqpoint{2.477587in}{1.579973in}}{\pgfqpoint{2.480860in}{1.572073in}}{\pgfqpoint{2.486684in}{1.566249in}}%
\pgfpathcurveto{\pgfqpoint{2.492508in}{1.560425in}}{\pgfqpoint{2.500408in}{1.557152in}}{\pgfqpoint{2.508644in}{1.557152in}}%
\pgfpathclose%
\pgfusepath{stroke,fill}%
\end{pgfscope}%
\begin{pgfscope}%
\pgfsetbuttcap%
\pgfsetmiterjoin%
\definecolor{currentfill}{rgb}{1.000000,1.000000,1.000000}%
\pgfsetfillcolor{currentfill}%
\pgfsetfillopacity{0.800000}%
\pgfsetlinewidth{1.003750pt}%
\definecolor{currentstroke}{rgb}{0.800000,0.800000,0.800000}%
\pgfsetstrokecolor{currentstroke}%
\pgfsetstrokeopacity{0.800000}%
\pgfsetdash{}{0pt}%
\pgfpathmoveto{\pgfqpoint{2.104889in}{3.216678in}}%
\pgfpathlineto{\pgfqpoint{3.698778in}{3.216678in}}%
\pgfpathquadraticcurveto{\pgfqpoint{3.726556in}{3.216678in}}{\pgfqpoint{3.726556in}{3.244456in}}%
\pgfpathlineto{\pgfqpoint{3.726556in}{3.811400in}}%
\pgfpathquadraticcurveto{\pgfqpoint{3.726556in}{3.839178in}}{\pgfqpoint{3.698778in}{3.839178in}}%
\pgfpathlineto{\pgfqpoint{2.104889in}{3.839178in}}%
\pgfpathquadraticcurveto{\pgfqpoint{2.077111in}{3.839178in}}{\pgfqpoint{2.077111in}{3.811400in}}%
\pgfpathlineto{\pgfqpoint{2.077111in}{3.244456in}}%
\pgfpathquadraticcurveto{\pgfqpoint{2.077111in}{3.216678in}}{\pgfqpoint{2.104889in}{3.216678in}}%
\pgfpathclose%
\pgfusepath{stroke,fill}%
\end{pgfscope}%
\begin{pgfscope}%
\pgfsetrectcap%
\pgfsetroundjoin%
\pgfsetlinewidth{1.505625pt}%
\definecolor{currentstroke}{rgb}{0.121569,0.466667,0.705882}%
\pgfsetstrokecolor{currentstroke}%
\pgfsetdash{}{0pt}%
\pgfpathmoveto{\pgfqpoint{2.132667in}{3.735011in}}%
\pgfpathlineto{\pgfqpoint{2.410444in}{3.735011in}}%
\pgfusepath{stroke}%
\end{pgfscope}%
\begin{pgfscope}%
\definecolor{textcolor}{rgb}{0.000000,0.000000,0.000000}%
\pgfsetstrokecolor{textcolor}%
\pgfsetfillcolor{textcolor}%
\pgftext[x=2.521555in,y=3.686400in,left,base]{\color{textcolor}\rmfamily\fontsize{10.000000}{12.000000}\selectfont Ground truth}%
\end{pgfscope}%
\begin{pgfscope}%
\pgfsetbuttcap%
\pgfsetroundjoin%
\definecolor{currentfill}{rgb}{0.121569,0.466667,0.705882}%
\pgfsetfillcolor{currentfill}%
\pgfsetlinewidth{1.003750pt}%
\definecolor{currentstroke}{rgb}{0.121569,0.466667,0.705882}%
\pgfsetstrokecolor{currentstroke}%
\pgfsetdash{}{0pt}%
\pgfsys@defobject{currentmarker}{\pgfqpoint{-0.031056in}{-0.031056in}}{\pgfqpoint{0.031056in}{0.031056in}}{%
\pgfpathmoveto{\pgfqpoint{0.000000in}{-0.031056in}}%
\pgfpathcurveto{\pgfqpoint{0.008236in}{-0.031056in}}{\pgfqpoint{0.016136in}{-0.027784in}}{\pgfqpoint{0.021960in}{-0.021960in}}%
\pgfpathcurveto{\pgfqpoint{0.027784in}{-0.016136in}}{\pgfqpoint{0.031056in}{-0.008236in}}{\pgfqpoint{0.031056in}{0.000000in}}%
\pgfpathcurveto{\pgfqpoint{0.031056in}{0.008236in}}{\pgfqpoint{0.027784in}{0.016136in}}{\pgfqpoint{0.021960in}{0.021960in}}%
\pgfpathcurveto{\pgfqpoint{0.016136in}{0.027784in}}{\pgfqpoint{0.008236in}{0.031056in}}{\pgfqpoint{0.000000in}{0.031056in}}%
\pgfpathcurveto{\pgfqpoint{-0.008236in}{0.031056in}}{\pgfqpoint{-0.016136in}{0.027784in}}{\pgfqpoint{-0.021960in}{0.021960in}}%
\pgfpathcurveto{\pgfqpoint{-0.027784in}{0.016136in}}{\pgfqpoint{-0.031056in}{0.008236in}}{\pgfqpoint{-0.031056in}{0.000000in}}%
\pgfpathcurveto{\pgfqpoint{-0.031056in}{-0.008236in}}{\pgfqpoint{-0.027784in}{-0.016136in}}{\pgfqpoint{-0.021960in}{-0.021960in}}%
\pgfpathcurveto{\pgfqpoint{-0.016136in}{-0.027784in}}{\pgfqpoint{-0.008236in}{-0.031056in}}{\pgfqpoint{0.000000in}{-0.031056in}}%
\pgfpathclose%
\pgfusepath{stroke,fill}%
}%
\begin{pgfscope}%
\pgfsys@transformshift{2.271555in}{3.529248in}%
\pgfsys@useobject{currentmarker}{}%
\end{pgfscope}%
\end{pgfscope}%
\begin{pgfscope}%
\definecolor{textcolor}{rgb}{0.000000,0.000000,0.000000}%
\pgfsetstrokecolor{textcolor}%
\pgfsetfillcolor{textcolor}%
\pgftext[x=2.521555in,y=3.492789in,left,base]{\color{textcolor}\rmfamily\fontsize{10.000000}{12.000000}\selectfont Estimated position}%
\end{pgfscope}%
\begin{pgfscope}%
\pgfsetbuttcap%
\pgfsetroundjoin%
\definecolor{currentfill}{rgb}{1.000000,0.498039,0.054902}%
\pgfsetfillcolor{currentfill}%
\pgfsetlinewidth{1.003750pt}%
\definecolor{currentstroke}{rgb}{1.000000,0.498039,0.054902}%
\pgfsetstrokecolor{currentstroke}%
\pgfsetdash{}{0pt}%
\pgfsys@defobject{currentmarker}{\pgfqpoint{-0.031056in}{-0.031056in}}{\pgfqpoint{0.031056in}{0.031056in}}{%
\pgfpathmoveto{\pgfqpoint{0.000000in}{-0.031056in}}%
\pgfpathcurveto{\pgfqpoint{0.008236in}{-0.031056in}}{\pgfqpoint{0.016136in}{-0.027784in}}{\pgfqpoint{0.021960in}{-0.021960in}}%
\pgfpathcurveto{\pgfqpoint{0.027784in}{-0.016136in}}{\pgfqpoint{0.031056in}{-0.008236in}}{\pgfqpoint{0.031056in}{0.000000in}}%
\pgfpathcurveto{\pgfqpoint{0.031056in}{0.008236in}}{\pgfqpoint{0.027784in}{0.016136in}}{\pgfqpoint{0.021960in}{0.021960in}}%
\pgfpathcurveto{\pgfqpoint{0.016136in}{0.027784in}}{\pgfqpoint{0.008236in}{0.031056in}}{\pgfqpoint{0.000000in}{0.031056in}}%
\pgfpathcurveto{\pgfqpoint{-0.008236in}{0.031056in}}{\pgfqpoint{-0.016136in}{0.027784in}}{\pgfqpoint{-0.021960in}{0.021960in}}%
\pgfpathcurveto{\pgfqpoint{-0.027784in}{0.016136in}}{\pgfqpoint{-0.031056in}{0.008236in}}{\pgfqpoint{-0.031056in}{0.000000in}}%
\pgfpathcurveto{\pgfqpoint{-0.031056in}{-0.008236in}}{\pgfqpoint{-0.027784in}{-0.016136in}}{\pgfqpoint{-0.021960in}{-0.021960in}}%
\pgfpathcurveto{\pgfqpoint{-0.016136in}{-0.027784in}}{\pgfqpoint{-0.008236in}{-0.031056in}}{\pgfqpoint{0.000000in}{-0.031056in}}%
\pgfpathclose%
\pgfusepath{stroke,fill}%
}%
\begin{pgfscope}%
\pgfsys@transformshift{2.271555in}{3.335637in}%
\pgfsys@useobject{currentmarker}{}%
\end{pgfscope}%
\end{pgfscope}%
\begin{pgfscope}%
\definecolor{textcolor}{rgb}{0.000000,0.000000,0.000000}%
\pgfsetstrokecolor{textcolor}%
\pgfsetfillcolor{textcolor}%
\pgftext[x=2.521555in,y=3.299178in,left,base]{\color{textcolor}\rmfamily\fontsize{10.000000}{12.000000}\selectfont Estimated turn}%
\end{pgfscope}%
\end{pgfpicture}%
\makeatother%
\endgroup%
}
%         \caption{ ROLEQ's 3D position estimation had the lowest turn error for the 4-meter line experiment. }
%         \label{fig:line4_3D}
%     \end{subfigure}
%     \caption{Position estimation by the best performing algorithms in the 4-meter line experiment.}
%     \label{fig:line4}
% \end{figure}

% \subsubsection{16 meter}
% For the 16-meter line experiment, the Mahony algorithm which had the lowest displacement error with an average of 0.48 meters (16\% of error margin), and ROLEQ with an average of 0.24 meters of turn error (7\% of error margin).

% \begin{figure}[!h]
%     \centering
%     \begin{table}[H]
    \begin{center}
        \resizebox{1\linewidth}{!}{
            \begin{tabular}[t]{lcccc}
                \hline
                Algorithm   & Displacement Error[$m$] & Displacement Error[\%] & Turn Error[$m$] & Turn Error[\%] \\
                \hline
                AngularRate & 0.89                    & 5.59                   & 3.97            & 24.78          \\            AQUA            & 1.47  & 9.21 & 4.23 & 26.46              \\            Complementary            & 0.50  & 3.09 & 2.16 & 13.47              \\            Davenport            & 0.51  & 3.17 & 2.16 & 13.52              \\            EKF            & 0.72  & 4.53 & 2.22 & 13.86              \\            FAMC            & 0.43  & 2.69 & 2.16 & 13.49              \\            FLAE            & 0.51  & 3.17 & 2.16 & 13.52              \\            Fourati            & 1.09  & 6.83 & 4.80 & 29.98              \\            Madgwick            & 0.49  & 3.05 & 2.15 & 13.47              \\            Mahony            & 0.48  & 3.00 & 2.11 & 13.21              \\            OLEQ            & 0.68  & 4.28 & 3.54 & 22.14              \\            QUEST            & 1.67  & 10.44 & 3.79 & 23.72              \\            ROLEQ            & 0.72  & 4.51 & 3.56 & 22.26              \\            SAAM            & 0.48  & 3.02 & 2.14 & 13.35              \\            Tilt            & 0.48  & 3.02 & 2.14 & 13.35              \\
                \hline
                Average     & 0.74                    & 4.64                   & 2.89            & 18.04
            \end{tabular}
        }
        \caption{Accelerometer Specifications. }
        \label{tab:accelerometer_specification}
    \end{center}
\end{table}
% \end{figure}

% \begin{figure}[!h]
%     \centering
%     \begin{subfigure}{0.49\textwidth}
%         \centering
%         \resizebox{1\linewidth}{!}{%% Creator: Matplotlib, PGF backend
%%
%% To include the figure in your LaTeX document, write
%%   \input{<filename>.pgf}
%%
%% Make sure the required packages are loaded in your preamble
%%   \usepackage{pgf}
%%
%% and, on pdftex
%%   \usepackage[utf8]{inputenc}\DeclareUnicodeCharacter{2212}{-}
%%
%% or, on luatex and xetex
%%   \usepackage{unicode-math}
%%
%% Figures using additional raster images can only be included by \input if
%% they are in the same directory as the main LaTeX file. For loading figures
%% from other directories you can use the `import` package
%%   \usepackage{import}
%%
%% and then include the figures with
%%   \import{<path to file>}{<filename>.pgf}
%%
%% Matplotlib used the following preamble
%%   \usepackage{fontspec}
%%
\begingroup%
\makeatletter%
\begin{pgfpicture}%
\pgfpathrectangle{\pgfpointorigin}{\pgfqpoint{4.342355in}{4.207622in}}%
\pgfusepath{use as bounding box, clip}%
\begin{pgfscope}%
\pgfsetbuttcap%
\pgfsetmiterjoin%
\definecolor{currentfill}{rgb}{1.000000,1.000000,1.000000}%
\pgfsetfillcolor{currentfill}%
\pgfsetlinewidth{0.000000pt}%
\definecolor{currentstroke}{rgb}{1.000000,1.000000,1.000000}%
\pgfsetstrokecolor{currentstroke}%
\pgfsetdash{}{0pt}%
\pgfpathmoveto{\pgfqpoint{0.000000in}{0.000000in}}%
\pgfpathlineto{\pgfqpoint{4.342355in}{0.000000in}}%
\pgfpathlineto{\pgfqpoint{4.342355in}{4.207622in}}%
\pgfpathlineto{\pgfqpoint{0.000000in}{4.207622in}}%
\pgfpathclose%
\pgfusepath{fill}%
\end{pgfscope}%
\begin{pgfscope}%
\pgfsetbuttcap%
\pgfsetmiterjoin%
\definecolor{currentfill}{rgb}{1.000000,1.000000,1.000000}%
\pgfsetfillcolor{currentfill}%
\pgfsetlinewidth{0.000000pt}%
\definecolor{currentstroke}{rgb}{0.000000,0.000000,0.000000}%
\pgfsetstrokecolor{currentstroke}%
\pgfsetstrokeopacity{0.000000}%
\pgfsetdash{}{0pt}%
\pgfpathmoveto{\pgfqpoint{0.100000in}{0.212622in}}%
\pgfpathlineto{\pgfqpoint{3.796000in}{0.212622in}}%
\pgfpathlineto{\pgfqpoint{3.796000in}{3.908622in}}%
\pgfpathlineto{\pgfqpoint{0.100000in}{3.908622in}}%
\pgfpathclose%
\pgfusepath{fill}%
\end{pgfscope}%
\begin{pgfscope}%
\pgfsetbuttcap%
\pgfsetmiterjoin%
\definecolor{currentfill}{rgb}{0.950000,0.950000,0.950000}%
\pgfsetfillcolor{currentfill}%
\pgfsetfillopacity{0.500000}%
\pgfsetlinewidth{1.003750pt}%
\definecolor{currentstroke}{rgb}{0.950000,0.950000,0.950000}%
\pgfsetstrokecolor{currentstroke}%
\pgfsetstrokeopacity{0.500000}%
\pgfsetdash{}{0pt}%
\pgfpathmoveto{\pgfqpoint{0.379073in}{1.123938in}}%
\pgfpathlineto{\pgfqpoint{1.599613in}{2.147018in}}%
\pgfpathlineto{\pgfqpoint{1.582647in}{3.622484in}}%
\pgfpathlineto{\pgfqpoint{0.303698in}{2.689165in}}%
\pgfusepath{stroke,fill}%
\end{pgfscope}%
\begin{pgfscope}%
\pgfsetbuttcap%
\pgfsetmiterjoin%
\definecolor{currentfill}{rgb}{0.900000,0.900000,0.900000}%
\pgfsetfillcolor{currentfill}%
\pgfsetfillopacity{0.500000}%
\pgfsetlinewidth{1.003750pt}%
\definecolor{currentstroke}{rgb}{0.900000,0.900000,0.900000}%
\pgfsetstrokecolor{currentstroke}%
\pgfsetstrokeopacity{0.500000}%
\pgfsetdash{}{0pt}%
\pgfpathmoveto{\pgfqpoint{1.599613in}{2.147018in}}%
\pgfpathlineto{\pgfqpoint{3.558144in}{1.577751in}}%
\pgfpathlineto{\pgfqpoint{3.628038in}{3.104037in}}%
\pgfpathlineto{\pgfqpoint{1.582647in}{3.622484in}}%
\pgfusepath{stroke,fill}%
\end{pgfscope}%
\begin{pgfscope}%
\pgfsetbuttcap%
\pgfsetmiterjoin%
\definecolor{currentfill}{rgb}{0.925000,0.925000,0.925000}%
\pgfsetfillcolor{currentfill}%
\pgfsetfillopacity{0.500000}%
\pgfsetlinewidth{1.003750pt}%
\definecolor{currentstroke}{rgb}{0.925000,0.925000,0.925000}%
\pgfsetstrokecolor{currentstroke}%
\pgfsetstrokeopacity{0.500000}%
\pgfsetdash{}{0pt}%
\pgfpathmoveto{\pgfqpoint{0.379073in}{1.123938in}}%
\pgfpathlineto{\pgfqpoint{2.455212in}{0.445871in}}%
\pgfpathlineto{\pgfqpoint{3.558144in}{1.577751in}}%
\pgfpathlineto{\pgfqpoint{1.599613in}{2.147018in}}%
\pgfusepath{stroke,fill}%
\end{pgfscope}%
\begin{pgfscope}%
\pgfsetrectcap%
\pgfsetroundjoin%
\pgfsetlinewidth{0.803000pt}%
\definecolor{currentstroke}{rgb}{0.000000,0.000000,0.000000}%
\pgfsetstrokecolor{currentstroke}%
\pgfsetdash{}{0pt}%
\pgfpathmoveto{\pgfqpoint{0.379073in}{1.123938in}}%
\pgfpathlineto{\pgfqpoint{2.455212in}{0.445871in}}%
\pgfusepath{stroke}%
\end{pgfscope}%
\begin{pgfscope}%
\definecolor{textcolor}{rgb}{0.000000,0.000000,0.000000}%
\pgfsetstrokecolor{textcolor}%
\pgfsetfillcolor{textcolor}%
\pgftext[x=0.730374in, y=0.408886in, left, base,rotate=341.912962]{\color{textcolor}\rmfamily\fontsize{10.000000}{12.000000}\selectfont Position X [\(\displaystyle m\)]}%
\end{pgfscope}%
\begin{pgfscope}%
\pgfsetbuttcap%
\pgfsetroundjoin%
\pgfsetlinewidth{0.803000pt}%
\definecolor{currentstroke}{rgb}{0.690196,0.690196,0.690196}%
\pgfsetstrokecolor{currentstroke}%
\pgfsetdash{}{0pt}%
\pgfpathmoveto{\pgfqpoint{0.504815in}{1.082870in}}%
\pgfpathlineto{\pgfqpoint{1.718725in}{2.112397in}}%
\pgfpathlineto{\pgfqpoint{1.706795in}{3.591016in}}%
\pgfusepath{stroke}%
\end{pgfscope}%
\begin{pgfscope}%
\pgfsetbuttcap%
\pgfsetroundjoin%
\pgfsetlinewidth{0.803000pt}%
\definecolor{currentstroke}{rgb}{0.690196,0.690196,0.690196}%
\pgfsetstrokecolor{currentstroke}%
\pgfsetdash{}{0pt}%
\pgfpathmoveto{\pgfqpoint{0.935376in}{0.942249in}}%
\pgfpathlineto{\pgfqpoint{2.126101in}{1.993989in}}%
\pgfpathlineto{\pgfqpoint{2.131635in}{3.483331in}}%
\pgfusepath{stroke}%
\end{pgfscope}%
\begin{pgfscope}%
\pgfsetbuttcap%
\pgfsetroundjoin%
\pgfsetlinewidth{0.803000pt}%
\definecolor{currentstroke}{rgb}{0.690196,0.690196,0.690196}%
\pgfsetstrokecolor{currentstroke}%
\pgfsetdash{}{0pt}%
\pgfpathmoveto{\pgfqpoint{1.375685in}{0.798444in}}%
\pgfpathlineto{\pgfqpoint{2.541928in}{1.873125in}}%
\pgfpathlineto{\pgfqpoint{2.565674in}{3.373315in}}%
\pgfusepath{stroke}%
\end{pgfscope}%
\begin{pgfscope}%
\pgfsetbuttcap%
\pgfsetroundjoin%
\pgfsetlinewidth{0.803000pt}%
\definecolor{currentstroke}{rgb}{0.690196,0.690196,0.690196}%
\pgfsetstrokecolor{currentstroke}%
\pgfsetdash{}{0pt}%
\pgfpathmoveto{\pgfqpoint{1.826077in}{0.651346in}}%
\pgfpathlineto{\pgfqpoint{2.966473in}{1.749726in}}%
\pgfpathlineto{\pgfqpoint{3.009215in}{3.260890in}}%
\pgfusepath{stroke}%
\end{pgfscope}%
\begin{pgfscope}%
\pgfsetbuttcap%
\pgfsetroundjoin%
\pgfsetlinewidth{0.803000pt}%
\definecolor{currentstroke}{rgb}{0.690196,0.690196,0.690196}%
\pgfsetstrokecolor{currentstroke}%
\pgfsetdash{}{0pt}%
\pgfpathmoveto{\pgfqpoint{2.286903in}{0.500841in}}%
\pgfpathlineto{\pgfqpoint{3.400012in}{1.623713in}}%
\pgfpathlineto{\pgfqpoint{3.462572in}{3.145977in}}%
\pgfusepath{stroke}%
\end{pgfscope}%
\begin{pgfscope}%
\pgfsetrectcap%
\pgfsetroundjoin%
\pgfsetlinewidth{0.803000pt}%
\definecolor{currentstroke}{rgb}{0.000000,0.000000,0.000000}%
\pgfsetstrokecolor{currentstroke}%
\pgfsetdash{}{0pt}%
\pgfpathmoveto{\pgfqpoint{0.515386in}{1.091835in}}%
\pgfpathlineto{\pgfqpoint{0.483629in}{1.064902in}}%
\pgfusepath{stroke}%
\end{pgfscope}%
\begin{pgfscope}%
\definecolor{textcolor}{rgb}{0.000000,0.000000,0.000000}%
\pgfsetstrokecolor{textcolor}%
\pgfsetfillcolor{textcolor}%
\pgftext[x=0.400245in,y=0.864666in,,top]{\color{textcolor}\rmfamily\fontsize{10.000000}{12.000000}\selectfont \(\displaystyle {0}\)}%
\end{pgfscope}%
\begin{pgfscope}%
\pgfsetrectcap%
\pgfsetroundjoin%
\pgfsetlinewidth{0.803000pt}%
\definecolor{currentstroke}{rgb}{0.000000,0.000000,0.000000}%
\pgfsetstrokecolor{currentstroke}%
\pgfsetdash{}{0pt}%
\pgfpathmoveto{\pgfqpoint{0.945754in}{0.951416in}}%
\pgfpathlineto{\pgfqpoint{0.914574in}{0.923876in}}%
\pgfusepath{stroke}%
\end{pgfscope}%
\begin{pgfscope}%
\definecolor{textcolor}{rgb}{0.000000,0.000000,0.000000}%
\pgfsetstrokecolor{textcolor}%
\pgfsetfillcolor{textcolor}%
\pgftext[x=0.831249in,y=0.721065in,,top]{\color{textcolor}\rmfamily\fontsize{10.000000}{12.000000}\selectfont \(\displaystyle {5}\)}%
\end{pgfscope}%
\begin{pgfscope}%
\pgfsetrectcap%
\pgfsetroundjoin%
\pgfsetlinewidth{0.803000pt}%
\definecolor{currentstroke}{rgb}{0.000000,0.000000,0.000000}%
\pgfsetstrokecolor{currentstroke}%
\pgfsetdash{}{0pt}%
\pgfpathmoveto{\pgfqpoint{1.385859in}{0.807820in}}%
\pgfpathlineto{\pgfqpoint{1.355291in}{0.779652in}}%
\pgfusepath{stroke}%
\end{pgfscope}%
\begin{pgfscope}%
\definecolor{textcolor}{rgb}{0.000000,0.000000,0.000000}%
\pgfsetstrokecolor{textcolor}%
\pgfsetfillcolor{textcolor}%
\pgftext[x=1.272049in,y=0.574200in,,top]{\color{textcolor}\rmfamily\fontsize{10.000000}{12.000000}\selectfont \(\displaystyle {10}\)}%
\end{pgfscope}%
\begin{pgfscope}%
\pgfsetrectcap%
\pgfsetroundjoin%
\pgfsetlinewidth{0.803000pt}%
\definecolor{currentstroke}{rgb}{0.000000,0.000000,0.000000}%
\pgfsetstrokecolor{currentstroke}%
\pgfsetdash{}{0pt}%
\pgfpathmoveto{\pgfqpoint{1.836035in}{0.660938in}}%
\pgfpathlineto{\pgfqpoint{1.806116in}{0.632121in}}%
\pgfusepath{stroke}%
\end{pgfscope}%
\begin{pgfscope}%
\definecolor{textcolor}{rgb}{0.000000,0.000000,0.000000}%
\pgfsetstrokecolor{textcolor}%
\pgfsetfillcolor{textcolor}%
\pgftext[x=1.722982in,y=0.423958in,,top]{\color{textcolor}\rmfamily\fontsize{10.000000}{12.000000}\selectfont \(\displaystyle {15}\)}%
\end{pgfscope}%
\begin{pgfscope}%
\pgfsetrectcap%
\pgfsetroundjoin%
\pgfsetlinewidth{0.803000pt}%
\definecolor{currentstroke}{rgb}{0.000000,0.000000,0.000000}%
\pgfsetstrokecolor{currentstroke}%
\pgfsetdash{}{0pt}%
\pgfpathmoveto{\pgfqpoint{2.296632in}{0.510655in}}%
\pgfpathlineto{\pgfqpoint{2.267400in}{0.481167in}}%
\pgfusepath{stroke}%
\end{pgfscope}%
\begin{pgfscope}%
\definecolor{textcolor}{rgb}{0.000000,0.000000,0.000000}%
\pgfsetstrokecolor{textcolor}%
\pgfsetfillcolor{textcolor}%
\pgftext[x=2.184402in,y=0.270223in,,top]{\color{textcolor}\rmfamily\fontsize{10.000000}{12.000000}\selectfont \(\displaystyle {20}\)}%
\end{pgfscope}%
\begin{pgfscope}%
\pgfsetrectcap%
\pgfsetroundjoin%
\pgfsetlinewidth{0.803000pt}%
\definecolor{currentstroke}{rgb}{0.000000,0.000000,0.000000}%
\pgfsetstrokecolor{currentstroke}%
\pgfsetdash{}{0pt}%
\pgfpathmoveto{\pgfqpoint{3.558144in}{1.577751in}}%
\pgfpathlineto{\pgfqpoint{2.455212in}{0.445871in}}%
\pgfusepath{stroke}%
\end{pgfscope}%
\begin{pgfscope}%
\definecolor{textcolor}{rgb}{0.000000,0.000000,0.000000}%
\pgfsetstrokecolor{textcolor}%
\pgfsetfillcolor{textcolor}%
\pgftext[x=3.120747in, y=0.305657in, left, base,rotate=45.742112]{\color{textcolor}\rmfamily\fontsize{10.000000}{12.000000}\selectfont Position Y [\(\displaystyle m\)]}%
\end{pgfscope}%
\begin{pgfscope}%
\pgfsetbuttcap%
\pgfsetroundjoin%
\pgfsetlinewidth{0.803000pt}%
\definecolor{currentstroke}{rgb}{0.690196,0.690196,0.690196}%
\pgfsetstrokecolor{currentstroke}%
\pgfsetdash{}{0pt}%
\pgfpathmoveto{\pgfqpoint{0.331607in}{2.709532in}}%
\pgfpathlineto{\pgfqpoint{0.405609in}{1.146180in}}%
\pgfpathlineto{\pgfqpoint{2.479295in}{0.470585in}}%
\pgfusepath{stroke}%
\end{pgfscope}%
\begin{pgfscope}%
\pgfsetbuttcap%
\pgfsetroundjoin%
\pgfsetlinewidth{0.803000pt}%
\definecolor{currentstroke}{rgb}{0.690196,0.690196,0.690196}%
\pgfsetstrokecolor{currentstroke}%
\pgfsetdash{}{0pt}%
\pgfpathmoveto{\pgfqpoint{0.496905in}{2.830159in}}%
\pgfpathlineto{\pgfqpoint{0.562864in}{1.277994in}}%
\pgfpathlineto{\pgfqpoint{2.621917in}{0.616951in}}%
\pgfusepath{stroke}%
\end{pgfscope}%
\begin{pgfscope}%
\pgfsetbuttcap%
\pgfsetroundjoin%
\pgfsetlinewidth{0.803000pt}%
\definecolor{currentstroke}{rgb}{0.690196,0.690196,0.690196}%
\pgfsetstrokecolor{currentstroke}%
\pgfsetdash{}{0pt}%
\pgfpathmoveto{\pgfqpoint{0.658597in}{2.948154in}}%
\pgfpathlineto{\pgfqpoint{0.716836in}{1.407057in}}%
\pgfpathlineto{\pgfqpoint{2.761405in}{0.760100in}}%
\pgfusepath{stroke}%
\end{pgfscope}%
\begin{pgfscope}%
\pgfsetbuttcap%
\pgfsetroundjoin%
\pgfsetlinewidth{0.803000pt}%
\definecolor{currentstroke}{rgb}{0.690196,0.690196,0.690196}%
\pgfsetstrokecolor{currentstroke}%
\pgfsetdash{}{0pt}%
\pgfpathmoveto{\pgfqpoint{0.816799in}{3.063603in}}%
\pgfpathlineto{\pgfqpoint{0.867628in}{1.533454in}}%
\pgfpathlineto{\pgfqpoint{2.897861in}{0.900137in}}%
\pgfusepath{stroke}%
\end{pgfscope}%
\begin{pgfscope}%
\pgfsetbuttcap%
\pgfsetroundjoin%
\pgfsetlinewidth{0.803000pt}%
\definecolor{currentstroke}{rgb}{0.690196,0.690196,0.690196}%
\pgfsetstrokecolor{currentstroke}%
\pgfsetdash{}{0pt}%
\pgfpathmoveto{\pgfqpoint{0.971622in}{3.176586in}}%
\pgfpathlineto{\pgfqpoint{1.015337in}{1.657267in}}%
\pgfpathlineto{\pgfqpoint{3.031383in}{1.037163in}}%
\pgfusepath{stroke}%
\end{pgfscope}%
\begin{pgfscope}%
\pgfsetbuttcap%
\pgfsetroundjoin%
\pgfsetlinewidth{0.803000pt}%
\definecolor{currentstroke}{rgb}{0.690196,0.690196,0.690196}%
\pgfsetstrokecolor{currentstroke}%
\pgfsetdash{}{0pt}%
\pgfpathmoveto{\pgfqpoint{1.123174in}{3.287181in}}%
\pgfpathlineto{\pgfqpoint{1.160057in}{1.778574in}}%
\pgfpathlineto{\pgfqpoint{3.162064in}{1.171275in}}%
\pgfusepath{stroke}%
\end{pgfscope}%
\begin{pgfscope}%
\pgfsetbuttcap%
\pgfsetroundjoin%
\pgfsetlinewidth{0.803000pt}%
\definecolor{currentstroke}{rgb}{0.690196,0.690196,0.690196}%
\pgfsetstrokecolor{currentstroke}%
\pgfsetdash{}{0pt}%
\pgfpathmoveto{\pgfqpoint{1.271557in}{3.395465in}}%
\pgfpathlineto{\pgfqpoint{1.301877in}{1.897450in}}%
\pgfpathlineto{\pgfqpoint{3.289994in}{1.302563in}}%
\pgfusepath{stroke}%
\end{pgfscope}%
\begin{pgfscope}%
\pgfsetbuttcap%
\pgfsetroundjoin%
\pgfsetlinewidth{0.803000pt}%
\definecolor{currentstroke}{rgb}{0.690196,0.690196,0.690196}%
\pgfsetstrokecolor{currentstroke}%
\pgfsetdash{}{0pt}%
\pgfpathmoveto{\pgfqpoint{1.416870in}{3.501507in}}%
\pgfpathlineto{\pgfqpoint{1.440884in}{2.013968in}}%
\pgfpathlineto{\pgfqpoint{3.415260in}{1.431116in}}%
\pgfusepath{stroke}%
\end{pgfscope}%
\begin{pgfscope}%
\pgfsetbuttcap%
\pgfsetroundjoin%
\pgfsetlinewidth{0.803000pt}%
\definecolor{currentstroke}{rgb}{0.690196,0.690196,0.690196}%
\pgfsetstrokecolor{currentstroke}%
\pgfsetdash{}{0pt}%
\pgfpathmoveto{\pgfqpoint{1.559207in}{3.605378in}}%
\pgfpathlineto{\pgfqpoint{1.577160in}{2.128198in}}%
\pgfpathlineto{\pgfqpoint{3.537943in}{1.557019in}}%
\pgfusepath{stroke}%
\end{pgfscope}%
\begin{pgfscope}%
\pgfsetrectcap%
\pgfsetroundjoin%
\pgfsetlinewidth{0.803000pt}%
\definecolor{currentstroke}{rgb}{0.000000,0.000000,0.000000}%
\pgfsetstrokecolor{currentstroke}%
\pgfsetdash{}{0pt}%
\pgfpathmoveto{\pgfqpoint{2.461816in}{0.476280in}}%
\pgfpathlineto{\pgfqpoint{2.514297in}{0.459182in}}%
\pgfusepath{stroke}%
\end{pgfscope}%
\begin{pgfscope}%
\definecolor{textcolor}{rgb}{0.000000,0.000000,0.000000}%
\pgfsetstrokecolor{textcolor}%
\pgfsetfillcolor{textcolor}%
\pgftext[x=2.658892in,y=0.283356in,,top]{\color{textcolor}\rmfamily\fontsize{10.000000}{12.000000}\selectfont \(\displaystyle {−2.0}\)}%
\end{pgfscope}%
\begin{pgfscope}%
\pgfsetrectcap%
\pgfsetroundjoin%
\pgfsetlinewidth{0.803000pt}%
\definecolor{currentstroke}{rgb}{0.000000,0.000000,0.000000}%
\pgfsetstrokecolor{currentstroke}%
\pgfsetdash{}{0pt}%
\pgfpathmoveto{\pgfqpoint{2.604571in}{0.622519in}}%
\pgfpathlineto{\pgfqpoint{2.656652in}{0.605799in}}%
\pgfusepath{stroke}%
\end{pgfscope}%
\begin{pgfscope}%
\definecolor{textcolor}{rgb}{0.000000,0.000000,0.000000}%
\pgfsetstrokecolor{textcolor}%
\pgfsetfillcolor{textcolor}%
\pgftext[x=2.799603in,y=0.431890in,,top]{\color{textcolor}\rmfamily\fontsize{10.000000}{12.000000}\selectfont \(\displaystyle {−1.5}\)}%
\end{pgfscope}%
\begin{pgfscope}%
\pgfsetrectcap%
\pgfsetroundjoin%
\pgfsetlinewidth{0.803000pt}%
\definecolor{currentstroke}{rgb}{0.000000,0.000000,0.000000}%
\pgfsetstrokecolor{currentstroke}%
\pgfsetdash{}{0pt}%
\pgfpathmoveto{\pgfqpoint{2.744191in}{0.765547in}}%
\pgfpathlineto{\pgfqpoint{2.795876in}{0.749192in}}%
\pgfusepath{stroke}%
\end{pgfscope}%
\begin{pgfscope}%
\definecolor{textcolor}{rgb}{0.000000,0.000000,0.000000}%
\pgfsetstrokecolor{textcolor}%
\pgfsetfillcolor{textcolor}%
\pgftext[x=2.937219in,y=0.577158in,,top]{\color{textcolor}\rmfamily\fontsize{10.000000}{12.000000}\selectfont \(\displaystyle {−1.0}\)}%
\end{pgfscope}%
\begin{pgfscope}%
\pgfsetrectcap%
\pgfsetroundjoin%
\pgfsetlinewidth{0.803000pt}%
\definecolor{currentstroke}{rgb}{0.000000,0.000000,0.000000}%
\pgfsetstrokecolor{currentstroke}%
\pgfsetdash{}{0pt}%
\pgfpathmoveto{\pgfqpoint{2.880777in}{0.905466in}}%
\pgfpathlineto{\pgfqpoint{2.932072in}{0.889465in}}%
\pgfusepath{stroke}%
\end{pgfscope}%
\begin{pgfscope}%
\definecolor{textcolor}{rgb}{0.000000,0.000000,0.000000}%
\pgfsetstrokecolor{textcolor}%
\pgfsetfillcolor{textcolor}%
\pgftext[x=3.071843in,y=0.719266in,,top]{\color{textcolor}\rmfamily\fontsize{10.000000}{12.000000}\selectfont \(\displaystyle {−0.5}\)}%
\end{pgfscope}%
\begin{pgfscope}%
\pgfsetrectcap%
\pgfsetroundjoin%
\pgfsetlinewidth{0.803000pt}%
\definecolor{currentstroke}{rgb}{0.000000,0.000000,0.000000}%
\pgfsetstrokecolor{currentstroke}%
\pgfsetdash{}{0pt}%
\pgfpathmoveto{\pgfqpoint{3.014427in}{1.042379in}}%
\pgfpathlineto{\pgfqpoint{3.065336in}{1.026720in}}%
\pgfusepath{stroke}%
\end{pgfscope}%
\begin{pgfscope}%
\definecolor{textcolor}{rgb}{0.000000,0.000000,0.000000}%
\pgfsetstrokecolor{textcolor}%
\pgfsetfillcolor{textcolor}%
\pgftext[x=3.203570in,y=0.858317in,,top]{\color{textcolor}\rmfamily\fontsize{10.000000}{12.000000}\selectfont \(\displaystyle {0.0}\)}%
\end{pgfscope}%
\begin{pgfscope}%
\pgfsetrectcap%
\pgfsetroundjoin%
\pgfsetlinewidth{0.803000pt}%
\definecolor{currentstroke}{rgb}{0.000000,0.000000,0.000000}%
\pgfsetstrokecolor{currentstroke}%
\pgfsetdash{}{0pt}%
\pgfpathmoveto{\pgfqpoint{3.145235in}{1.176380in}}%
\pgfpathlineto{\pgfqpoint{3.195763in}{1.161052in}}%
\pgfusepath{stroke}%
\end{pgfscope}%
\begin{pgfscope}%
\definecolor{textcolor}{rgb}{0.000000,0.000000,0.000000}%
\pgfsetstrokecolor{textcolor}%
\pgfsetfillcolor{textcolor}%
\pgftext[x=3.332493in,y=0.994408in,,top]{\color{textcolor}\rmfamily\fontsize{10.000000}{12.000000}\selectfont \(\displaystyle {0.5}\)}%
\end{pgfscope}%
\begin{pgfscope}%
\pgfsetrectcap%
\pgfsetroundjoin%
\pgfsetlinewidth{0.803000pt}%
\definecolor{currentstroke}{rgb}{0.000000,0.000000,0.000000}%
\pgfsetstrokecolor{currentstroke}%
\pgfsetdash{}{0pt}%
\pgfpathmoveto{\pgfqpoint{3.273291in}{1.307561in}}%
\pgfpathlineto{\pgfqpoint{3.323442in}{1.292554in}}%
\pgfusepath{stroke}%
\end{pgfscope}%
\begin{pgfscope}%
\definecolor{textcolor}{rgb}{0.000000,0.000000,0.000000}%
\pgfsetstrokecolor{textcolor}%
\pgfsetfillcolor{textcolor}%
\pgftext[x=3.458700in,y=1.127633in,,top]{\color{textcolor}\rmfamily\fontsize{10.000000}{12.000000}\selectfont \(\displaystyle {1.0}\)}%
\end{pgfscope}%
\begin{pgfscope}%
\pgfsetrectcap%
\pgfsetroundjoin%
\pgfsetlinewidth{0.803000pt}%
\definecolor{currentstroke}{rgb}{0.000000,0.000000,0.000000}%
\pgfsetstrokecolor{currentstroke}%
\pgfsetdash{}{0pt}%
\pgfpathmoveto{\pgfqpoint{3.398680in}{1.436010in}}%
\pgfpathlineto{\pgfqpoint{3.448459in}{1.421315in}}%
\pgfusepath{stroke}%
\end{pgfscope}%
\begin{pgfscope}%
\definecolor{textcolor}{rgb}{0.000000,0.000000,0.000000}%
\pgfsetstrokecolor{textcolor}%
\pgfsetfillcolor{textcolor}%
\pgftext[x=3.582278in,y=1.258080in,,top]{\color{textcolor}\rmfamily\fontsize{10.000000}{12.000000}\selectfont \(\displaystyle {1.5}\)}%
\end{pgfscope}%
\begin{pgfscope}%
\pgfsetrectcap%
\pgfsetroundjoin%
\pgfsetlinewidth{0.803000pt}%
\definecolor{currentstroke}{rgb}{0.000000,0.000000,0.000000}%
\pgfsetstrokecolor{currentstroke}%
\pgfsetdash{}{0pt}%
\pgfpathmoveto{\pgfqpoint{3.521485in}{1.561813in}}%
\pgfpathlineto{\pgfqpoint{3.570897in}{1.547419in}}%
\pgfusepath{stroke}%
\end{pgfscope}%
\begin{pgfscope}%
\definecolor{textcolor}{rgb}{0.000000,0.000000,0.000000}%
\pgfsetstrokecolor{textcolor}%
\pgfsetfillcolor{textcolor}%
\pgftext[x=3.703306in,y=1.385837in,,top]{\color{textcolor}\rmfamily\fontsize{10.000000}{12.000000}\selectfont \(\displaystyle {2.0}\)}%
\end{pgfscope}%
\begin{pgfscope}%
\pgfsetrectcap%
\pgfsetroundjoin%
\pgfsetlinewidth{0.803000pt}%
\definecolor{currentstroke}{rgb}{0.000000,0.000000,0.000000}%
\pgfsetstrokecolor{currentstroke}%
\pgfsetdash{}{0pt}%
\pgfpathmoveto{\pgfqpoint{3.558144in}{1.577751in}}%
\pgfpathlineto{\pgfqpoint{3.628038in}{3.104037in}}%
\pgfusepath{stroke}%
\end{pgfscope}%
\begin{pgfscope}%
\definecolor{textcolor}{rgb}{0.000000,0.000000,0.000000}%
\pgfsetstrokecolor{textcolor}%
\pgfsetfillcolor{textcolor}%
\pgftext[x=4.167903in, y=1.963517in, left, base,rotate=87.378092]{\color{textcolor}\rmfamily\fontsize{10.000000}{12.000000}\selectfont Position Z [\(\displaystyle m\)]}%
\end{pgfscope}%
\begin{pgfscope}%
\pgfsetbuttcap%
\pgfsetroundjoin%
\pgfsetlinewidth{0.803000pt}%
\definecolor{currentstroke}{rgb}{0.690196,0.690196,0.690196}%
\pgfsetstrokecolor{currentstroke}%
\pgfsetdash{}{0pt}%
\pgfpathmoveto{\pgfqpoint{3.559484in}{1.606993in}}%
\pgfpathlineto{\pgfqpoint{1.599288in}{2.175345in}}%
\pgfpathlineto{\pgfqpoint{0.377632in}{1.153876in}}%
\pgfusepath{stroke}%
\end{pgfscope}%
\begin{pgfscope}%
\pgfsetbuttcap%
\pgfsetroundjoin%
\pgfsetlinewidth{0.803000pt}%
\definecolor{currentstroke}{rgb}{0.690196,0.690196,0.690196}%
\pgfsetstrokecolor{currentstroke}%
\pgfsetdash{}{0pt}%
\pgfpathmoveto{\pgfqpoint{3.567566in}{1.783504in}}%
\pgfpathlineto{\pgfqpoint{1.597322in}{2.346284in}}%
\pgfpathlineto{\pgfqpoint{0.368927in}{1.334633in}}%
\pgfusepath{stroke}%
\end{pgfscope}%
\begin{pgfscope}%
\pgfsetbuttcap%
\pgfsetroundjoin%
\pgfsetlinewidth{0.803000pt}%
\definecolor{currentstroke}{rgb}{0.690196,0.690196,0.690196}%
\pgfsetstrokecolor{currentstroke}%
\pgfsetdash{}{0pt}%
\pgfpathmoveto{\pgfqpoint{3.575734in}{1.961852in}}%
\pgfpathlineto{\pgfqpoint{1.595337in}{2.518917in}}%
\pgfpathlineto{\pgfqpoint{0.360129in}{1.517342in}}%
\pgfusepath{stroke}%
\end{pgfscope}%
\begin{pgfscope}%
\pgfsetbuttcap%
\pgfsetroundjoin%
\pgfsetlinewidth{0.803000pt}%
\definecolor{currentstroke}{rgb}{0.690196,0.690196,0.690196}%
\pgfsetstrokecolor{currentstroke}%
\pgfsetdash{}{0pt}%
\pgfpathmoveto{\pgfqpoint{3.583986in}{2.142066in}}%
\pgfpathlineto{\pgfqpoint{1.593332in}{2.693270in}}%
\pgfpathlineto{\pgfqpoint{0.351234in}{1.702036in}}%
\pgfusepath{stroke}%
\end{pgfscope}%
\begin{pgfscope}%
\pgfsetbuttcap%
\pgfsetroundjoin%
\pgfsetlinewidth{0.803000pt}%
\definecolor{currentstroke}{rgb}{0.690196,0.690196,0.690196}%
\pgfsetstrokecolor{currentstroke}%
\pgfsetdash{}{0pt}%
\pgfpathmoveto{\pgfqpoint{3.592325in}{2.324175in}}%
\pgfpathlineto{\pgfqpoint{1.591307in}{2.869367in}}%
\pgfpathlineto{\pgfqpoint{0.342243in}{1.888746in}}%
\pgfusepath{stroke}%
\end{pgfscope}%
\begin{pgfscope}%
\pgfsetbuttcap%
\pgfsetroundjoin%
\pgfsetlinewidth{0.803000pt}%
\definecolor{currentstroke}{rgb}{0.690196,0.690196,0.690196}%
\pgfsetstrokecolor{currentstroke}%
\pgfsetdash{}{0pt}%
\pgfpathmoveto{\pgfqpoint{3.600753in}{2.508209in}}%
\pgfpathlineto{\pgfqpoint{1.589262in}{3.047236in}}%
\pgfpathlineto{\pgfqpoint{0.333153in}{2.077507in}}%
\pgfusepath{stroke}%
\end{pgfscope}%
\begin{pgfscope}%
\pgfsetbuttcap%
\pgfsetroundjoin%
\pgfsetlinewidth{0.803000pt}%
\definecolor{currentstroke}{rgb}{0.690196,0.690196,0.690196}%
\pgfsetstrokecolor{currentstroke}%
\pgfsetdash{}{0pt}%
\pgfpathmoveto{\pgfqpoint{3.609270in}{2.694199in}}%
\pgfpathlineto{\pgfqpoint{1.587196in}{3.226903in}}%
\pgfpathlineto{\pgfqpoint{0.323963in}{2.268352in}}%
\pgfusepath{stroke}%
\end{pgfscope}%
\begin{pgfscope}%
\pgfsetbuttcap%
\pgfsetroundjoin%
\pgfsetlinewidth{0.803000pt}%
\definecolor{currentstroke}{rgb}{0.690196,0.690196,0.690196}%
\pgfsetstrokecolor{currentstroke}%
\pgfsetdash{}{0pt}%
\pgfpathmoveto{\pgfqpoint{3.617878in}{2.882176in}}%
\pgfpathlineto{\pgfqpoint{1.585109in}{3.408397in}}%
\pgfpathlineto{\pgfqpoint{0.314670in}{2.461315in}}%
\pgfusepath{stroke}%
\end{pgfscope}%
\begin{pgfscope}%
\pgfsetbuttcap%
\pgfsetroundjoin%
\pgfsetlinewidth{0.803000pt}%
\definecolor{currentstroke}{rgb}{0.690196,0.690196,0.690196}%
\pgfsetstrokecolor{currentstroke}%
\pgfsetdash{}{0pt}%
\pgfpathmoveto{\pgfqpoint{3.626578in}{3.072172in}}%
\pgfpathlineto{\pgfqpoint{1.583000in}{3.591744in}}%
\pgfpathlineto{\pgfqpoint{0.305274in}{2.656433in}}%
\pgfusepath{stroke}%
\end{pgfscope}%
\begin{pgfscope}%
\pgfsetrectcap%
\pgfsetroundjoin%
\pgfsetlinewidth{0.803000pt}%
\definecolor{currentstroke}{rgb}{0.000000,0.000000,0.000000}%
\pgfsetstrokecolor{currentstroke}%
\pgfsetdash{}{0pt}%
\pgfpathmoveto{\pgfqpoint{3.543031in}{1.611763in}}%
\pgfpathlineto{\pgfqpoint{3.592427in}{1.597441in}}%
\pgfusepath{stroke}%
\end{pgfscope}%
\begin{pgfscope}%
\definecolor{textcolor}{rgb}{0.000000,0.000000,0.000000}%
\pgfsetstrokecolor{textcolor}%
\pgfsetfillcolor{textcolor}%
\pgftext[x=3.813150in,y=1.643116in,,top]{\color{textcolor}\rmfamily\fontsize{10.000000}{12.000000}\selectfont \(\displaystyle {0.0}\)}%
\end{pgfscope}%
\begin{pgfscope}%
\pgfsetrectcap%
\pgfsetroundjoin%
\pgfsetlinewidth{0.803000pt}%
\definecolor{currentstroke}{rgb}{0.000000,0.000000,0.000000}%
\pgfsetstrokecolor{currentstroke}%
\pgfsetdash{}{0pt}%
\pgfpathmoveto{\pgfqpoint{3.551026in}{1.788229in}}%
\pgfpathlineto{\pgfqpoint{3.600687in}{1.774043in}}%
\pgfusepath{stroke}%
\end{pgfscope}%
\begin{pgfscope}%
\definecolor{textcolor}{rgb}{0.000000,0.000000,0.000000}%
\pgfsetstrokecolor{textcolor}%
\pgfsetfillcolor{textcolor}%
\pgftext[x=3.822517in,y=1.819282in,,top]{\color{textcolor}\rmfamily\fontsize{10.000000}{12.000000}\selectfont \(\displaystyle {0.5}\)}%
\end{pgfscope}%
\begin{pgfscope}%
\pgfsetrectcap%
\pgfsetroundjoin%
\pgfsetlinewidth{0.803000pt}%
\definecolor{currentstroke}{rgb}{0.000000,0.000000,0.000000}%
\pgfsetstrokecolor{currentstroke}%
\pgfsetdash{}{0pt}%
\pgfpathmoveto{\pgfqpoint{3.559104in}{1.966530in}}%
\pgfpathlineto{\pgfqpoint{3.609033in}{1.952485in}}%
\pgfusepath{stroke}%
\end{pgfscope}%
\begin{pgfscope}%
\definecolor{textcolor}{rgb}{0.000000,0.000000,0.000000}%
\pgfsetstrokecolor{textcolor}%
\pgfsetfillcolor{textcolor}%
\pgftext[x=3.831982in,y=1.997275in,,top]{\color{textcolor}\rmfamily\fontsize{10.000000}{12.000000}\selectfont \(\displaystyle {1.0}\)}%
\end{pgfscope}%
\begin{pgfscope}%
\pgfsetrectcap%
\pgfsetroundjoin%
\pgfsetlinewidth{0.803000pt}%
\definecolor{currentstroke}{rgb}{0.000000,0.000000,0.000000}%
\pgfsetstrokecolor{currentstroke}%
\pgfsetdash{}{0pt}%
\pgfpathmoveto{\pgfqpoint{3.567266in}{2.146696in}}%
\pgfpathlineto{\pgfqpoint{3.617467in}{2.132795in}}%
\pgfusepath{stroke}%
\end{pgfscope}%
\begin{pgfscope}%
\definecolor{textcolor}{rgb}{0.000000,0.000000,0.000000}%
\pgfsetstrokecolor{textcolor}%
\pgfsetfillcolor{textcolor}%
\pgftext[x=3.841546in,y=2.177124in,,top]{\color{textcolor}\rmfamily\fontsize{10.000000}{12.000000}\selectfont \(\displaystyle {1.5}\)}%
\end{pgfscope}%
\begin{pgfscope}%
\pgfsetrectcap%
\pgfsetroundjoin%
\pgfsetlinewidth{0.803000pt}%
\definecolor{currentstroke}{rgb}{0.000000,0.000000,0.000000}%
\pgfsetstrokecolor{currentstroke}%
\pgfsetdash{}{0pt}%
\pgfpathmoveto{\pgfqpoint{3.575514in}{2.328755in}}%
\pgfpathlineto{\pgfqpoint{3.625989in}{2.315003in}}%
\pgfusepath{stroke}%
\end{pgfscope}%
\begin{pgfscope}%
\definecolor{textcolor}{rgb}{0.000000,0.000000,0.000000}%
\pgfsetstrokecolor{textcolor}%
\pgfsetfillcolor{textcolor}%
\pgftext[x=3.851210in,y=2.358859in,,top]{\color{textcolor}\rmfamily\fontsize{10.000000}{12.000000}\selectfont \(\displaystyle {2.0}\)}%
\end{pgfscope}%
\begin{pgfscope}%
\pgfsetrectcap%
\pgfsetroundjoin%
\pgfsetlinewidth{0.803000pt}%
\definecolor{currentstroke}{rgb}{0.000000,0.000000,0.000000}%
\pgfsetstrokecolor{currentstroke}%
\pgfsetdash{}{0pt}%
\pgfpathmoveto{\pgfqpoint{3.583849in}{2.512738in}}%
\pgfpathlineto{\pgfqpoint{3.634601in}{2.499138in}}%
\pgfusepath{stroke}%
\end{pgfscope}%
\begin{pgfscope}%
\definecolor{textcolor}{rgb}{0.000000,0.000000,0.000000}%
\pgfsetstrokecolor{textcolor}%
\pgfsetfillcolor{textcolor}%
\pgftext[x=3.860975in,y=2.542509in,,top]{\color{textcolor}\rmfamily\fontsize{10.000000}{12.000000}\selectfont \(\displaystyle {2.5}\)}%
\end{pgfscope}%
\begin{pgfscope}%
\pgfsetrectcap%
\pgfsetroundjoin%
\pgfsetlinewidth{0.803000pt}%
\definecolor{currentstroke}{rgb}{0.000000,0.000000,0.000000}%
\pgfsetstrokecolor{currentstroke}%
\pgfsetdash{}{0pt}%
\pgfpathmoveto{\pgfqpoint{3.592273in}{2.698676in}}%
\pgfpathlineto{\pgfqpoint{3.643305in}{2.685232in}}%
\pgfusepath{stroke}%
\end{pgfscope}%
\begin{pgfscope}%
\definecolor{textcolor}{rgb}{0.000000,0.000000,0.000000}%
\pgfsetstrokecolor{textcolor}%
\pgfsetfillcolor{textcolor}%
\pgftext[x=3.870844in,y=2.728104in,,top]{\color{textcolor}\rmfamily\fontsize{10.000000}{12.000000}\selectfont \(\displaystyle {3.0}\)}%
\end{pgfscope}%
\begin{pgfscope}%
\pgfsetrectcap%
\pgfsetroundjoin%
\pgfsetlinewidth{0.803000pt}%
\definecolor{currentstroke}{rgb}{0.000000,0.000000,0.000000}%
\pgfsetstrokecolor{currentstroke}%
\pgfsetdash{}{0pt}%
\pgfpathmoveto{\pgfqpoint{3.600787in}{2.886600in}}%
\pgfpathlineto{\pgfqpoint{3.652102in}{2.873316in}}%
\pgfusepath{stroke}%
\end{pgfscope}%
\begin{pgfscope}%
\definecolor{textcolor}{rgb}{0.000000,0.000000,0.000000}%
\pgfsetstrokecolor{textcolor}%
\pgfsetfillcolor{textcolor}%
\pgftext[x=3.880819in,y=2.915677in,,top]{\color{textcolor}\rmfamily\fontsize{10.000000}{12.000000}\selectfont \(\displaystyle {3.5}\)}%
\end{pgfscope}%
\begin{pgfscope}%
\pgfsetrectcap%
\pgfsetroundjoin%
\pgfsetlinewidth{0.803000pt}%
\definecolor{currentstroke}{rgb}{0.000000,0.000000,0.000000}%
\pgfsetstrokecolor{currentstroke}%
\pgfsetdash{}{0pt}%
\pgfpathmoveto{\pgfqpoint{3.609392in}{3.076542in}}%
\pgfpathlineto{\pgfqpoint{3.660994in}{3.063422in}}%
\pgfusepath{stroke}%
\end{pgfscope}%
\begin{pgfscope}%
\definecolor{textcolor}{rgb}{0.000000,0.000000,0.000000}%
\pgfsetstrokecolor{textcolor}%
\pgfsetfillcolor{textcolor}%
\pgftext[x=3.890900in,y=3.105258in,,top]{\color{textcolor}\rmfamily\fontsize{10.000000}{12.000000}\selectfont \(\displaystyle {4.0}\)}%
\end{pgfscope}%
\begin{pgfscope}%
\pgfpathrectangle{\pgfqpoint{0.100000in}{0.212622in}}{\pgfqpoint{3.696000in}{3.696000in}}%
\pgfusepath{clip}%
\pgfsetrectcap%
\pgfsetroundjoin%
\pgfsetlinewidth{1.505625pt}%
\definecolor{currentstroke}{rgb}{0.121569,0.466667,0.705882}%
\pgfsetstrokecolor{currentstroke}%
\pgfsetdash{}{0pt}%
\pgfpathmoveto{\pgfqpoint{1.136968in}{1.648805in}}%
\pgfpathlineto{\pgfqpoint{2.510478in}{1.227352in}}%
\pgfusepath{stroke}%
\end{pgfscope}%
\begin{pgfscope}%
\pgfpathrectangle{\pgfqpoint{0.100000in}{0.212622in}}{\pgfqpoint{3.696000in}{3.696000in}}%
\pgfusepath{clip}%
\pgfsetrectcap%
\pgfsetroundjoin%
\pgfsetlinewidth{1.505625pt}%
\definecolor{currentstroke}{rgb}{1.000000,0.000000,0.000000}%
\pgfsetstrokecolor{currentstroke}%
\pgfsetdash{}{0pt}%
\pgfpathmoveto{\pgfqpoint{1.136977in}{1.648806in}}%
\pgfpathlineto{\pgfqpoint{1.136968in}{1.648805in}}%
\pgfusepath{stroke}%
\end{pgfscope}%
\begin{pgfscope}%
\pgfpathrectangle{\pgfqpoint{0.100000in}{0.212622in}}{\pgfqpoint{3.696000in}{3.696000in}}%
\pgfusepath{clip}%
\pgfsetrectcap%
\pgfsetroundjoin%
\pgfsetlinewidth{1.505625pt}%
\definecolor{currentstroke}{rgb}{1.000000,0.000000,0.000000}%
\pgfsetstrokecolor{currentstroke}%
\pgfsetdash{}{0pt}%
\pgfpathmoveto{\pgfqpoint{1.136979in}{1.648805in}}%
\pgfpathlineto{\pgfqpoint{1.136968in}{1.648805in}}%
\pgfusepath{stroke}%
\end{pgfscope}%
\begin{pgfscope}%
\pgfpathrectangle{\pgfqpoint{0.100000in}{0.212622in}}{\pgfqpoint{3.696000in}{3.696000in}}%
\pgfusepath{clip}%
\pgfsetrectcap%
\pgfsetroundjoin%
\pgfsetlinewidth{1.505625pt}%
\definecolor{currentstroke}{rgb}{1.000000,0.000000,0.000000}%
\pgfsetstrokecolor{currentstroke}%
\pgfsetdash{}{0pt}%
\pgfpathmoveto{\pgfqpoint{1.136981in}{1.648804in}}%
\pgfpathlineto{\pgfqpoint{1.136968in}{1.648805in}}%
\pgfusepath{stroke}%
\end{pgfscope}%
\begin{pgfscope}%
\pgfpathrectangle{\pgfqpoint{0.100000in}{0.212622in}}{\pgfqpoint{3.696000in}{3.696000in}}%
\pgfusepath{clip}%
\pgfsetrectcap%
\pgfsetroundjoin%
\pgfsetlinewidth{1.505625pt}%
\definecolor{currentstroke}{rgb}{1.000000,0.000000,0.000000}%
\pgfsetstrokecolor{currentstroke}%
\pgfsetdash{}{0pt}%
\pgfpathmoveto{\pgfqpoint{1.136982in}{1.648804in}}%
\pgfpathlineto{\pgfqpoint{1.136968in}{1.648805in}}%
\pgfusepath{stroke}%
\end{pgfscope}%
\begin{pgfscope}%
\pgfpathrectangle{\pgfqpoint{0.100000in}{0.212622in}}{\pgfqpoint{3.696000in}{3.696000in}}%
\pgfusepath{clip}%
\pgfsetrectcap%
\pgfsetroundjoin%
\pgfsetlinewidth{1.505625pt}%
\definecolor{currentstroke}{rgb}{1.000000,0.000000,0.000000}%
\pgfsetstrokecolor{currentstroke}%
\pgfsetdash{}{0pt}%
\pgfpathmoveto{\pgfqpoint{1.136982in}{1.648803in}}%
\pgfpathlineto{\pgfqpoint{1.136968in}{1.648805in}}%
\pgfusepath{stroke}%
\end{pgfscope}%
\begin{pgfscope}%
\pgfpathrectangle{\pgfqpoint{0.100000in}{0.212622in}}{\pgfqpoint{3.696000in}{3.696000in}}%
\pgfusepath{clip}%
\pgfsetrectcap%
\pgfsetroundjoin%
\pgfsetlinewidth{1.505625pt}%
\definecolor{currentstroke}{rgb}{1.000000,0.000000,0.000000}%
\pgfsetstrokecolor{currentstroke}%
\pgfsetdash{}{0pt}%
\pgfpathmoveto{\pgfqpoint{1.136982in}{1.648803in}}%
\pgfpathlineto{\pgfqpoint{1.136968in}{1.648805in}}%
\pgfusepath{stroke}%
\end{pgfscope}%
\begin{pgfscope}%
\pgfpathrectangle{\pgfqpoint{0.100000in}{0.212622in}}{\pgfqpoint{3.696000in}{3.696000in}}%
\pgfusepath{clip}%
\pgfsetrectcap%
\pgfsetroundjoin%
\pgfsetlinewidth{1.505625pt}%
\definecolor{currentstroke}{rgb}{1.000000,0.000000,0.000000}%
\pgfsetstrokecolor{currentstroke}%
\pgfsetdash{}{0pt}%
\pgfpathmoveto{\pgfqpoint{1.136982in}{1.648803in}}%
\pgfpathlineto{\pgfqpoint{1.136968in}{1.648805in}}%
\pgfusepath{stroke}%
\end{pgfscope}%
\begin{pgfscope}%
\pgfpathrectangle{\pgfqpoint{0.100000in}{0.212622in}}{\pgfqpoint{3.696000in}{3.696000in}}%
\pgfusepath{clip}%
\pgfsetrectcap%
\pgfsetroundjoin%
\pgfsetlinewidth{1.505625pt}%
\definecolor{currentstroke}{rgb}{1.000000,0.000000,0.000000}%
\pgfsetstrokecolor{currentstroke}%
\pgfsetdash{}{0pt}%
\pgfpathmoveto{\pgfqpoint{1.136982in}{1.648803in}}%
\pgfpathlineto{\pgfqpoint{1.136968in}{1.648805in}}%
\pgfusepath{stroke}%
\end{pgfscope}%
\begin{pgfscope}%
\pgfpathrectangle{\pgfqpoint{0.100000in}{0.212622in}}{\pgfqpoint{3.696000in}{3.696000in}}%
\pgfusepath{clip}%
\pgfsetrectcap%
\pgfsetroundjoin%
\pgfsetlinewidth{1.505625pt}%
\definecolor{currentstroke}{rgb}{1.000000,0.000000,0.000000}%
\pgfsetstrokecolor{currentstroke}%
\pgfsetdash{}{0pt}%
\pgfpathmoveto{\pgfqpoint{1.136982in}{1.648803in}}%
\pgfpathlineto{\pgfqpoint{1.136968in}{1.648805in}}%
\pgfusepath{stroke}%
\end{pgfscope}%
\begin{pgfscope}%
\pgfpathrectangle{\pgfqpoint{0.100000in}{0.212622in}}{\pgfqpoint{3.696000in}{3.696000in}}%
\pgfusepath{clip}%
\pgfsetrectcap%
\pgfsetroundjoin%
\pgfsetlinewidth{1.505625pt}%
\definecolor{currentstroke}{rgb}{1.000000,0.000000,0.000000}%
\pgfsetstrokecolor{currentstroke}%
\pgfsetdash{}{0pt}%
\pgfpathmoveto{\pgfqpoint{1.136982in}{1.648803in}}%
\pgfpathlineto{\pgfqpoint{1.136968in}{1.648805in}}%
\pgfusepath{stroke}%
\end{pgfscope}%
\begin{pgfscope}%
\pgfpathrectangle{\pgfqpoint{0.100000in}{0.212622in}}{\pgfqpoint{3.696000in}{3.696000in}}%
\pgfusepath{clip}%
\pgfsetrectcap%
\pgfsetroundjoin%
\pgfsetlinewidth{1.505625pt}%
\definecolor{currentstroke}{rgb}{1.000000,0.000000,0.000000}%
\pgfsetstrokecolor{currentstroke}%
\pgfsetdash{}{0pt}%
\pgfpathmoveto{\pgfqpoint{1.136982in}{1.648803in}}%
\pgfpathlineto{\pgfqpoint{1.136968in}{1.648805in}}%
\pgfusepath{stroke}%
\end{pgfscope}%
\begin{pgfscope}%
\pgfpathrectangle{\pgfqpoint{0.100000in}{0.212622in}}{\pgfqpoint{3.696000in}{3.696000in}}%
\pgfusepath{clip}%
\pgfsetrectcap%
\pgfsetroundjoin%
\pgfsetlinewidth{1.505625pt}%
\definecolor{currentstroke}{rgb}{1.000000,0.000000,0.000000}%
\pgfsetstrokecolor{currentstroke}%
\pgfsetdash{}{0pt}%
\pgfpathmoveto{\pgfqpoint{1.136982in}{1.648803in}}%
\pgfpathlineto{\pgfqpoint{1.136968in}{1.648805in}}%
\pgfusepath{stroke}%
\end{pgfscope}%
\begin{pgfscope}%
\pgfpathrectangle{\pgfqpoint{0.100000in}{0.212622in}}{\pgfqpoint{3.696000in}{3.696000in}}%
\pgfusepath{clip}%
\pgfsetrectcap%
\pgfsetroundjoin%
\pgfsetlinewidth{1.505625pt}%
\definecolor{currentstroke}{rgb}{1.000000,0.000000,0.000000}%
\pgfsetstrokecolor{currentstroke}%
\pgfsetdash{}{0pt}%
\pgfpathmoveto{\pgfqpoint{1.136982in}{1.648803in}}%
\pgfpathlineto{\pgfqpoint{1.136968in}{1.648805in}}%
\pgfusepath{stroke}%
\end{pgfscope}%
\begin{pgfscope}%
\pgfpathrectangle{\pgfqpoint{0.100000in}{0.212622in}}{\pgfqpoint{3.696000in}{3.696000in}}%
\pgfusepath{clip}%
\pgfsetrectcap%
\pgfsetroundjoin%
\pgfsetlinewidth{1.505625pt}%
\definecolor{currentstroke}{rgb}{1.000000,0.000000,0.000000}%
\pgfsetstrokecolor{currentstroke}%
\pgfsetdash{}{0pt}%
\pgfpathmoveto{\pgfqpoint{1.136982in}{1.648803in}}%
\pgfpathlineto{\pgfqpoint{1.136968in}{1.648805in}}%
\pgfusepath{stroke}%
\end{pgfscope}%
\begin{pgfscope}%
\pgfpathrectangle{\pgfqpoint{0.100000in}{0.212622in}}{\pgfqpoint{3.696000in}{3.696000in}}%
\pgfusepath{clip}%
\pgfsetrectcap%
\pgfsetroundjoin%
\pgfsetlinewidth{1.505625pt}%
\definecolor{currentstroke}{rgb}{1.000000,0.000000,0.000000}%
\pgfsetstrokecolor{currentstroke}%
\pgfsetdash{}{0pt}%
\pgfpathmoveto{\pgfqpoint{1.136982in}{1.648803in}}%
\pgfpathlineto{\pgfqpoint{1.136968in}{1.648805in}}%
\pgfusepath{stroke}%
\end{pgfscope}%
\begin{pgfscope}%
\pgfpathrectangle{\pgfqpoint{0.100000in}{0.212622in}}{\pgfqpoint{3.696000in}{3.696000in}}%
\pgfusepath{clip}%
\pgfsetrectcap%
\pgfsetroundjoin%
\pgfsetlinewidth{1.505625pt}%
\definecolor{currentstroke}{rgb}{1.000000,0.000000,0.000000}%
\pgfsetstrokecolor{currentstroke}%
\pgfsetdash{}{0pt}%
\pgfpathmoveto{\pgfqpoint{1.136982in}{1.648803in}}%
\pgfpathlineto{\pgfqpoint{1.136968in}{1.648805in}}%
\pgfusepath{stroke}%
\end{pgfscope}%
\begin{pgfscope}%
\pgfpathrectangle{\pgfqpoint{0.100000in}{0.212622in}}{\pgfqpoint{3.696000in}{3.696000in}}%
\pgfusepath{clip}%
\pgfsetrectcap%
\pgfsetroundjoin%
\pgfsetlinewidth{1.505625pt}%
\definecolor{currentstroke}{rgb}{1.000000,0.000000,0.000000}%
\pgfsetstrokecolor{currentstroke}%
\pgfsetdash{}{0pt}%
\pgfpathmoveto{\pgfqpoint{1.136982in}{1.648803in}}%
\pgfpathlineto{\pgfqpoint{1.136968in}{1.648805in}}%
\pgfusepath{stroke}%
\end{pgfscope}%
\begin{pgfscope}%
\pgfpathrectangle{\pgfqpoint{0.100000in}{0.212622in}}{\pgfqpoint{3.696000in}{3.696000in}}%
\pgfusepath{clip}%
\pgfsetrectcap%
\pgfsetroundjoin%
\pgfsetlinewidth{1.505625pt}%
\definecolor{currentstroke}{rgb}{1.000000,0.000000,0.000000}%
\pgfsetstrokecolor{currentstroke}%
\pgfsetdash{}{0pt}%
\pgfpathmoveto{\pgfqpoint{1.136982in}{1.648803in}}%
\pgfpathlineto{\pgfqpoint{1.136968in}{1.648805in}}%
\pgfusepath{stroke}%
\end{pgfscope}%
\begin{pgfscope}%
\pgfpathrectangle{\pgfqpoint{0.100000in}{0.212622in}}{\pgfqpoint{3.696000in}{3.696000in}}%
\pgfusepath{clip}%
\pgfsetrectcap%
\pgfsetroundjoin%
\pgfsetlinewidth{1.505625pt}%
\definecolor{currentstroke}{rgb}{1.000000,0.000000,0.000000}%
\pgfsetstrokecolor{currentstroke}%
\pgfsetdash{}{0pt}%
\pgfpathmoveto{\pgfqpoint{1.136982in}{1.648803in}}%
\pgfpathlineto{\pgfqpoint{1.136968in}{1.648805in}}%
\pgfusepath{stroke}%
\end{pgfscope}%
\begin{pgfscope}%
\pgfpathrectangle{\pgfqpoint{0.100000in}{0.212622in}}{\pgfqpoint{3.696000in}{3.696000in}}%
\pgfusepath{clip}%
\pgfsetrectcap%
\pgfsetroundjoin%
\pgfsetlinewidth{1.505625pt}%
\definecolor{currentstroke}{rgb}{1.000000,0.000000,0.000000}%
\pgfsetstrokecolor{currentstroke}%
\pgfsetdash{}{0pt}%
\pgfpathmoveto{\pgfqpoint{1.136982in}{1.648803in}}%
\pgfpathlineto{\pgfqpoint{1.136968in}{1.648805in}}%
\pgfusepath{stroke}%
\end{pgfscope}%
\begin{pgfscope}%
\pgfpathrectangle{\pgfqpoint{0.100000in}{0.212622in}}{\pgfqpoint{3.696000in}{3.696000in}}%
\pgfusepath{clip}%
\pgfsetrectcap%
\pgfsetroundjoin%
\pgfsetlinewidth{1.505625pt}%
\definecolor{currentstroke}{rgb}{1.000000,0.000000,0.000000}%
\pgfsetstrokecolor{currentstroke}%
\pgfsetdash{}{0pt}%
\pgfpathmoveto{\pgfqpoint{1.136982in}{1.648803in}}%
\pgfpathlineto{\pgfqpoint{1.136968in}{1.648805in}}%
\pgfusepath{stroke}%
\end{pgfscope}%
\begin{pgfscope}%
\pgfpathrectangle{\pgfqpoint{0.100000in}{0.212622in}}{\pgfqpoint{3.696000in}{3.696000in}}%
\pgfusepath{clip}%
\pgfsetrectcap%
\pgfsetroundjoin%
\pgfsetlinewidth{1.505625pt}%
\definecolor{currentstroke}{rgb}{1.000000,0.000000,0.000000}%
\pgfsetstrokecolor{currentstroke}%
\pgfsetdash{}{0pt}%
\pgfpathmoveto{\pgfqpoint{1.136982in}{1.648803in}}%
\pgfpathlineto{\pgfqpoint{1.136968in}{1.648805in}}%
\pgfusepath{stroke}%
\end{pgfscope}%
\begin{pgfscope}%
\pgfpathrectangle{\pgfqpoint{0.100000in}{0.212622in}}{\pgfqpoint{3.696000in}{3.696000in}}%
\pgfusepath{clip}%
\pgfsetrectcap%
\pgfsetroundjoin%
\pgfsetlinewidth{1.505625pt}%
\definecolor{currentstroke}{rgb}{1.000000,0.000000,0.000000}%
\pgfsetstrokecolor{currentstroke}%
\pgfsetdash{}{0pt}%
\pgfpathmoveto{\pgfqpoint{1.136982in}{1.648803in}}%
\pgfpathlineto{\pgfqpoint{1.136968in}{1.648805in}}%
\pgfusepath{stroke}%
\end{pgfscope}%
\begin{pgfscope}%
\pgfpathrectangle{\pgfqpoint{0.100000in}{0.212622in}}{\pgfqpoint{3.696000in}{3.696000in}}%
\pgfusepath{clip}%
\pgfsetrectcap%
\pgfsetroundjoin%
\pgfsetlinewidth{1.505625pt}%
\definecolor{currentstroke}{rgb}{1.000000,0.000000,0.000000}%
\pgfsetstrokecolor{currentstroke}%
\pgfsetdash{}{0pt}%
\pgfpathmoveto{\pgfqpoint{1.136982in}{1.648803in}}%
\pgfpathlineto{\pgfqpoint{1.136968in}{1.648805in}}%
\pgfusepath{stroke}%
\end{pgfscope}%
\begin{pgfscope}%
\pgfpathrectangle{\pgfqpoint{0.100000in}{0.212622in}}{\pgfqpoint{3.696000in}{3.696000in}}%
\pgfusepath{clip}%
\pgfsetrectcap%
\pgfsetroundjoin%
\pgfsetlinewidth{1.505625pt}%
\definecolor{currentstroke}{rgb}{1.000000,0.000000,0.000000}%
\pgfsetstrokecolor{currentstroke}%
\pgfsetdash{}{0pt}%
\pgfpathmoveto{\pgfqpoint{1.136982in}{1.648803in}}%
\pgfpathlineto{\pgfqpoint{1.136968in}{1.648805in}}%
\pgfusepath{stroke}%
\end{pgfscope}%
\begin{pgfscope}%
\pgfpathrectangle{\pgfqpoint{0.100000in}{0.212622in}}{\pgfqpoint{3.696000in}{3.696000in}}%
\pgfusepath{clip}%
\pgfsetrectcap%
\pgfsetroundjoin%
\pgfsetlinewidth{1.505625pt}%
\definecolor{currentstroke}{rgb}{1.000000,0.000000,0.000000}%
\pgfsetstrokecolor{currentstroke}%
\pgfsetdash{}{0pt}%
\pgfpathmoveto{\pgfqpoint{1.136982in}{1.648803in}}%
\pgfpathlineto{\pgfqpoint{1.136968in}{1.648805in}}%
\pgfusepath{stroke}%
\end{pgfscope}%
\begin{pgfscope}%
\pgfpathrectangle{\pgfqpoint{0.100000in}{0.212622in}}{\pgfqpoint{3.696000in}{3.696000in}}%
\pgfusepath{clip}%
\pgfsetrectcap%
\pgfsetroundjoin%
\pgfsetlinewidth{1.505625pt}%
\definecolor{currentstroke}{rgb}{1.000000,0.000000,0.000000}%
\pgfsetstrokecolor{currentstroke}%
\pgfsetdash{}{0pt}%
\pgfpathmoveto{\pgfqpoint{1.136982in}{1.648803in}}%
\pgfpathlineto{\pgfqpoint{1.136968in}{1.648805in}}%
\pgfusepath{stroke}%
\end{pgfscope}%
\begin{pgfscope}%
\pgfpathrectangle{\pgfqpoint{0.100000in}{0.212622in}}{\pgfqpoint{3.696000in}{3.696000in}}%
\pgfusepath{clip}%
\pgfsetrectcap%
\pgfsetroundjoin%
\pgfsetlinewidth{1.505625pt}%
\definecolor{currentstroke}{rgb}{1.000000,0.000000,0.000000}%
\pgfsetstrokecolor{currentstroke}%
\pgfsetdash{}{0pt}%
\pgfpathmoveto{\pgfqpoint{1.136982in}{1.648803in}}%
\pgfpathlineto{\pgfqpoint{1.136968in}{1.648805in}}%
\pgfusepath{stroke}%
\end{pgfscope}%
\begin{pgfscope}%
\pgfpathrectangle{\pgfqpoint{0.100000in}{0.212622in}}{\pgfqpoint{3.696000in}{3.696000in}}%
\pgfusepath{clip}%
\pgfsetrectcap%
\pgfsetroundjoin%
\pgfsetlinewidth{1.505625pt}%
\definecolor{currentstroke}{rgb}{1.000000,0.000000,0.000000}%
\pgfsetstrokecolor{currentstroke}%
\pgfsetdash{}{0pt}%
\pgfpathmoveto{\pgfqpoint{1.136982in}{1.648803in}}%
\pgfpathlineto{\pgfqpoint{1.136968in}{1.648805in}}%
\pgfusepath{stroke}%
\end{pgfscope}%
\begin{pgfscope}%
\pgfpathrectangle{\pgfqpoint{0.100000in}{0.212622in}}{\pgfqpoint{3.696000in}{3.696000in}}%
\pgfusepath{clip}%
\pgfsetrectcap%
\pgfsetroundjoin%
\pgfsetlinewidth{1.505625pt}%
\definecolor{currentstroke}{rgb}{1.000000,0.000000,0.000000}%
\pgfsetstrokecolor{currentstroke}%
\pgfsetdash{}{0pt}%
\pgfpathmoveto{\pgfqpoint{1.136982in}{1.648803in}}%
\pgfpathlineto{\pgfqpoint{1.136968in}{1.648805in}}%
\pgfusepath{stroke}%
\end{pgfscope}%
\begin{pgfscope}%
\pgfpathrectangle{\pgfqpoint{0.100000in}{0.212622in}}{\pgfqpoint{3.696000in}{3.696000in}}%
\pgfusepath{clip}%
\pgfsetrectcap%
\pgfsetroundjoin%
\pgfsetlinewidth{1.505625pt}%
\definecolor{currentstroke}{rgb}{1.000000,0.000000,0.000000}%
\pgfsetstrokecolor{currentstroke}%
\pgfsetdash{}{0pt}%
\pgfpathmoveto{\pgfqpoint{1.136982in}{1.648803in}}%
\pgfpathlineto{\pgfqpoint{1.136968in}{1.648805in}}%
\pgfusepath{stroke}%
\end{pgfscope}%
\begin{pgfscope}%
\pgfpathrectangle{\pgfqpoint{0.100000in}{0.212622in}}{\pgfqpoint{3.696000in}{3.696000in}}%
\pgfusepath{clip}%
\pgfsetrectcap%
\pgfsetroundjoin%
\pgfsetlinewidth{1.505625pt}%
\definecolor{currentstroke}{rgb}{1.000000,0.000000,0.000000}%
\pgfsetstrokecolor{currentstroke}%
\pgfsetdash{}{0pt}%
\pgfpathmoveto{\pgfqpoint{1.136982in}{1.648803in}}%
\pgfpathlineto{\pgfqpoint{1.136968in}{1.648805in}}%
\pgfusepath{stroke}%
\end{pgfscope}%
\begin{pgfscope}%
\pgfpathrectangle{\pgfqpoint{0.100000in}{0.212622in}}{\pgfqpoint{3.696000in}{3.696000in}}%
\pgfusepath{clip}%
\pgfsetrectcap%
\pgfsetroundjoin%
\pgfsetlinewidth{1.505625pt}%
\definecolor{currentstroke}{rgb}{1.000000,0.000000,0.000000}%
\pgfsetstrokecolor{currentstroke}%
\pgfsetdash{}{0pt}%
\pgfpathmoveto{\pgfqpoint{1.136982in}{1.648803in}}%
\pgfpathlineto{\pgfqpoint{1.136968in}{1.648805in}}%
\pgfusepath{stroke}%
\end{pgfscope}%
\begin{pgfscope}%
\pgfpathrectangle{\pgfqpoint{0.100000in}{0.212622in}}{\pgfqpoint{3.696000in}{3.696000in}}%
\pgfusepath{clip}%
\pgfsetrectcap%
\pgfsetroundjoin%
\pgfsetlinewidth{1.505625pt}%
\definecolor{currentstroke}{rgb}{1.000000,0.000000,0.000000}%
\pgfsetstrokecolor{currentstroke}%
\pgfsetdash{}{0pt}%
\pgfpathmoveto{\pgfqpoint{1.136982in}{1.648803in}}%
\pgfpathlineto{\pgfqpoint{1.136968in}{1.648805in}}%
\pgfusepath{stroke}%
\end{pgfscope}%
\begin{pgfscope}%
\pgfpathrectangle{\pgfqpoint{0.100000in}{0.212622in}}{\pgfqpoint{3.696000in}{3.696000in}}%
\pgfusepath{clip}%
\pgfsetrectcap%
\pgfsetroundjoin%
\pgfsetlinewidth{1.505625pt}%
\definecolor{currentstroke}{rgb}{1.000000,0.000000,0.000000}%
\pgfsetstrokecolor{currentstroke}%
\pgfsetdash{}{0pt}%
\pgfpathmoveto{\pgfqpoint{1.136982in}{1.648803in}}%
\pgfpathlineto{\pgfqpoint{1.136968in}{1.648805in}}%
\pgfusepath{stroke}%
\end{pgfscope}%
\begin{pgfscope}%
\pgfpathrectangle{\pgfqpoint{0.100000in}{0.212622in}}{\pgfqpoint{3.696000in}{3.696000in}}%
\pgfusepath{clip}%
\pgfsetrectcap%
\pgfsetroundjoin%
\pgfsetlinewidth{1.505625pt}%
\definecolor{currentstroke}{rgb}{1.000000,0.000000,0.000000}%
\pgfsetstrokecolor{currentstroke}%
\pgfsetdash{}{0pt}%
\pgfpathmoveto{\pgfqpoint{1.136982in}{1.648803in}}%
\pgfpathlineto{\pgfqpoint{1.136968in}{1.648805in}}%
\pgfusepath{stroke}%
\end{pgfscope}%
\begin{pgfscope}%
\pgfpathrectangle{\pgfqpoint{0.100000in}{0.212622in}}{\pgfqpoint{3.696000in}{3.696000in}}%
\pgfusepath{clip}%
\pgfsetrectcap%
\pgfsetroundjoin%
\pgfsetlinewidth{1.505625pt}%
\definecolor{currentstroke}{rgb}{1.000000,0.000000,0.000000}%
\pgfsetstrokecolor{currentstroke}%
\pgfsetdash{}{0pt}%
\pgfpathmoveto{\pgfqpoint{1.136982in}{1.648803in}}%
\pgfpathlineto{\pgfqpoint{1.136968in}{1.648805in}}%
\pgfusepath{stroke}%
\end{pgfscope}%
\begin{pgfscope}%
\pgfpathrectangle{\pgfqpoint{0.100000in}{0.212622in}}{\pgfqpoint{3.696000in}{3.696000in}}%
\pgfusepath{clip}%
\pgfsetrectcap%
\pgfsetroundjoin%
\pgfsetlinewidth{1.505625pt}%
\definecolor{currentstroke}{rgb}{1.000000,0.000000,0.000000}%
\pgfsetstrokecolor{currentstroke}%
\pgfsetdash{}{0pt}%
\pgfpathmoveto{\pgfqpoint{1.136982in}{1.648803in}}%
\pgfpathlineto{\pgfqpoint{1.136968in}{1.648805in}}%
\pgfusepath{stroke}%
\end{pgfscope}%
\begin{pgfscope}%
\pgfpathrectangle{\pgfqpoint{0.100000in}{0.212622in}}{\pgfqpoint{3.696000in}{3.696000in}}%
\pgfusepath{clip}%
\pgfsetrectcap%
\pgfsetroundjoin%
\pgfsetlinewidth{1.505625pt}%
\definecolor{currentstroke}{rgb}{1.000000,0.000000,0.000000}%
\pgfsetstrokecolor{currentstroke}%
\pgfsetdash{}{0pt}%
\pgfpathmoveto{\pgfqpoint{1.136982in}{1.648803in}}%
\pgfpathlineto{\pgfqpoint{1.136968in}{1.648805in}}%
\pgfusepath{stroke}%
\end{pgfscope}%
\begin{pgfscope}%
\pgfpathrectangle{\pgfqpoint{0.100000in}{0.212622in}}{\pgfqpoint{3.696000in}{3.696000in}}%
\pgfusepath{clip}%
\pgfsetrectcap%
\pgfsetroundjoin%
\pgfsetlinewidth{1.505625pt}%
\definecolor{currentstroke}{rgb}{1.000000,0.000000,0.000000}%
\pgfsetstrokecolor{currentstroke}%
\pgfsetdash{}{0pt}%
\pgfpathmoveto{\pgfqpoint{1.137356in}{1.648620in}}%
\pgfpathlineto{\pgfqpoint{1.136968in}{1.648805in}}%
\pgfusepath{stroke}%
\end{pgfscope}%
\begin{pgfscope}%
\pgfpathrectangle{\pgfqpoint{0.100000in}{0.212622in}}{\pgfqpoint{3.696000in}{3.696000in}}%
\pgfusepath{clip}%
\pgfsetrectcap%
\pgfsetroundjoin%
\pgfsetlinewidth{1.505625pt}%
\definecolor{currentstroke}{rgb}{1.000000,0.000000,0.000000}%
\pgfsetstrokecolor{currentstroke}%
\pgfsetdash{}{0pt}%
\pgfpathmoveto{\pgfqpoint{1.137571in}{1.648528in}}%
\pgfpathlineto{\pgfqpoint{1.136968in}{1.648805in}}%
\pgfusepath{stroke}%
\end{pgfscope}%
\begin{pgfscope}%
\pgfpathrectangle{\pgfqpoint{0.100000in}{0.212622in}}{\pgfqpoint{3.696000in}{3.696000in}}%
\pgfusepath{clip}%
\pgfsetrectcap%
\pgfsetroundjoin%
\pgfsetlinewidth{1.505625pt}%
\definecolor{currentstroke}{rgb}{1.000000,0.000000,0.000000}%
\pgfsetstrokecolor{currentstroke}%
\pgfsetdash{}{0pt}%
\pgfpathmoveto{\pgfqpoint{1.137687in}{1.648476in}}%
\pgfpathlineto{\pgfqpoint{1.136968in}{1.648805in}}%
\pgfusepath{stroke}%
\end{pgfscope}%
\begin{pgfscope}%
\pgfpathrectangle{\pgfqpoint{0.100000in}{0.212622in}}{\pgfqpoint{3.696000in}{3.696000in}}%
\pgfusepath{clip}%
\pgfsetrectcap%
\pgfsetroundjoin%
\pgfsetlinewidth{1.505625pt}%
\definecolor{currentstroke}{rgb}{1.000000,0.000000,0.000000}%
\pgfsetstrokecolor{currentstroke}%
\pgfsetdash{}{0pt}%
\pgfpathmoveto{\pgfqpoint{1.137755in}{1.648451in}}%
\pgfpathlineto{\pgfqpoint{1.136968in}{1.648805in}}%
\pgfusepath{stroke}%
\end{pgfscope}%
\begin{pgfscope}%
\pgfpathrectangle{\pgfqpoint{0.100000in}{0.212622in}}{\pgfqpoint{3.696000in}{3.696000in}}%
\pgfusepath{clip}%
\pgfsetrectcap%
\pgfsetroundjoin%
\pgfsetlinewidth{1.505625pt}%
\definecolor{currentstroke}{rgb}{1.000000,0.000000,0.000000}%
\pgfsetstrokecolor{currentstroke}%
\pgfsetdash{}{0pt}%
\pgfpathmoveto{\pgfqpoint{1.137793in}{1.648437in}}%
\pgfpathlineto{\pgfqpoint{1.136968in}{1.648805in}}%
\pgfusepath{stroke}%
\end{pgfscope}%
\begin{pgfscope}%
\pgfpathrectangle{\pgfqpoint{0.100000in}{0.212622in}}{\pgfqpoint{3.696000in}{3.696000in}}%
\pgfusepath{clip}%
\pgfsetrectcap%
\pgfsetroundjoin%
\pgfsetlinewidth{1.505625pt}%
\definecolor{currentstroke}{rgb}{1.000000,0.000000,0.000000}%
\pgfsetstrokecolor{currentstroke}%
\pgfsetdash{}{0pt}%
\pgfpathmoveto{\pgfqpoint{1.137814in}{1.648430in}}%
\pgfpathlineto{\pgfqpoint{1.136968in}{1.648805in}}%
\pgfusepath{stroke}%
\end{pgfscope}%
\begin{pgfscope}%
\pgfpathrectangle{\pgfqpoint{0.100000in}{0.212622in}}{\pgfqpoint{3.696000in}{3.696000in}}%
\pgfusepath{clip}%
\pgfsetrectcap%
\pgfsetroundjoin%
\pgfsetlinewidth{1.505625pt}%
\definecolor{currentstroke}{rgb}{1.000000,0.000000,0.000000}%
\pgfsetstrokecolor{currentstroke}%
\pgfsetdash{}{0pt}%
\pgfpathmoveto{\pgfqpoint{1.137826in}{1.648426in}}%
\pgfpathlineto{\pgfqpoint{1.136968in}{1.648805in}}%
\pgfusepath{stroke}%
\end{pgfscope}%
\begin{pgfscope}%
\pgfpathrectangle{\pgfqpoint{0.100000in}{0.212622in}}{\pgfqpoint{3.696000in}{3.696000in}}%
\pgfusepath{clip}%
\pgfsetrectcap%
\pgfsetroundjoin%
\pgfsetlinewidth{1.505625pt}%
\definecolor{currentstroke}{rgb}{1.000000,0.000000,0.000000}%
\pgfsetstrokecolor{currentstroke}%
\pgfsetdash{}{0pt}%
\pgfpathmoveto{\pgfqpoint{1.137833in}{1.648424in}}%
\pgfpathlineto{\pgfqpoint{1.136968in}{1.648805in}}%
\pgfusepath{stroke}%
\end{pgfscope}%
\begin{pgfscope}%
\pgfpathrectangle{\pgfqpoint{0.100000in}{0.212622in}}{\pgfqpoint{3.696000in}{3.696000in}}%
\pgfusepath{clip}%
\pgfsetrectcap%
\pgfsetroundjoin%
\pgfsetlinewidth{1.505625pt}%
\definecolor{currentstroke}{rgb}{1.000000,0.000000,0.000000}%
\pgfsetstrokecolor{currentstroke}%
\pgfsetdash{}{0pt}%
\pgfpathmoveto{\pgfqpoint{1.137836in}{1.648423in}}%
\pgfpathlineto{\pgfqpoint{1.136968in}{1.648805in}}%
\pgfusepath{stroke}%
\end{pgfscope}%
\begin{pgfscope}%
\pgfpathrectangle{\pgfqpoint{0.100000in}{0.212622in}}{\pgfqpoint{3.696000in}{3.696000in}}%
\pgfusepath{clip}%
\pgfsetrectcap%
\pgfsetroundjoin%
\pgfsetlinewidth{1.505625pt}%
\definecolor{currentstroke}{rgb}{1.000000,0.000000,0.000000}%
\pgfsetstrokecolor{currentstroke}%
\pgfsetdash{}{0pt}%
\pgfpathmoveto{\pgfqpoint{1.137838in}{1.648423in}}%
\pgfpathlineto{\pgfqpoint{1.136968in}{1.648805in}}%
\pgfusepath{stroke}%
\end{pgfscope}%
\begin{pgfscope}%
\pgfpathrectangle{\pgfqpoint{0.100000in}{0.212622in}}{\pgfqpoint{3.696000in}{3.696000in}}%
\pgfusepath{clip}%
\pgfsetrectcap%
\pgfsetroundjoin%
\pgfsetlinewidth{1.505625pt}%
\definecolor{currentstroke}{rgb}{1.000000,0.000000,0.000000}%
\pgfsetstrokecolor{currentstroke}%
\pgfsetdash{}{0pt}%
\pgfpathmoveto{\pgfqpoint{1.137839in}{1.648422in}}%
\pgfpathlineto{\pgfqpoint{1.136968in}{1.648805in}}%
\pgfusepath{stroke}%
\end{pgfscope}%
\begin{pgfscope}%
\pgfpathrectangle{\pgfqpoint{0.100000in}{0.212622in}}{\pgfqpoint{3.696000in}{3.696000in}}%
\pgfusepath{clip}%
\pgfsetrectcap%
\pgfsetroundjoin%
\pgfsetlinewidth{1.505625pt}%
\definecolor{currentstroke}{rgb}{1.000000,0.000000,0.000000}%
\pgfsetstrokecolor{currentstroke}%
\pgfsetdash{}{0pt}%
\pgfpathmoveto{\pgfqpoint{1.137840in}{1.648422in}}%
\pgfpathlineto{\pgfqpoint{1.136968in}{1.648805in}}%
\pgfusepath{stroke}%
\end{pgfscope}%
\begin{pgfscope}%
\pgfpathrectangle{\pgfqpoint{0.100000in}{0.212622in}}{\pgfqpoint{3.696000in}{3.696000in}}%
\pgfusepath{clip}%
\pgfsetrectcap%
\pgfsetroundjoin%
\pgfsetlinewidth{1.505625pt}%
\definecolor{currentstroke}{rgb}{1.000000,0.000000,0.000000}%
\pgfsetstrokecolor{currentstroke}%
\pgfsetdash{}{0pt}%
\pgfpathmoveto{\pgfqpoint{1.137840in}{1.648422in}}%
\pgfpathlineto{\pgfqpoint{1.136968in}{1.648805in}}%
\pgfusepath{stroke}%
\end{pgfscope}%
\begin{pgfscope}%
\pgfpathrectangle{\pgfqpoint{0.100000in}{0.212622in}}{\pgfqpoint{3.696000in}{3.696000in}}%
\pgfusepath{clip}%
\pgfsetrectcap%
\pgfsetroundjoin%
\pgfsetlinewidth{1.505625pt}%
\definecolor{currentstroke}{rgb}{1.000000,0.000000,0.000000}%
\pgfsetstrokecolor{currentstroke}%
\pgfsetdash{}{0pt}%
\pgfpathmoveto{\pgfqpoint{1.137840in}{1.648422in}}%
\pgfpathlineto{\pgfqpoint{1.136968in}{1.648805in}}%
\pgfusepath{stroke}%
\end{pgfscope}%
\begin{pgfscope}%
\pgfpathrectangle{\pgfqpoint{0.100000in}{0.212622in}}{\pgfqpoint{3.696000in}{3.696000in}}%
\pgfusepath{clip}%
\pgfsetrectcap%
\pgfsetroundjoin%
\pgfsetlinewidth{1.505625pt}%
\definecolor{currentstroke}{rgb}{1.000000,0.000000,0.000000}%
\pgfsetstrokecolor{currentstroke}%
\pgfsetdash{}{0pt}%
\pgfpathmoveto{\pgfqpoint{1.137840in}{1.648422in}}%
\pgfpathlineto{\pgfqpoint{1.136968in}{1.648805in}}%
\pgfusepath{stroke}%
\end{pgfscope}%
\begin{pgfscope}%
\pgfpathrectangle{\pgfqpoint{0.100000in}{0.212622in}}{\pgfqpoint{3.696000in}{3.696000in}}%
\pgfusepath{clip}%
\pgfsetrectcap%
\pgfsetroundjoin%
\pgfsetlinewidth{1.505625pt}%
\definecolor{currentstroke}{rgb}{1.000000,0.000000,0.000000}%
\pgfsetstrokecolor{currentstroke}%
\pgfsetdash{}{0pt}%
\pgfpathmoveto{\pgfqpoint{1.137840in}{1.648422in}}%
\pgfpathlineto{\pgfqpoint{1.136968in}{1.648805in}}%
\pgfusepath{stroke}%
\end{pgfscope}%
\begin{pgfscope}%
\pgfpathrectangle{\pgfqpoint{0.100000in}{0.212622in}}{\pgfqpoint{3.696000in}{3.696000in}}%
\pgfusepath{clip}%
\pgfsetrectcap%
\pgfsetroundjoin%
\pgfsetlinewidth{1.505625pt}%
\definecolor{currentstroke}{rgb}{1.000000,0.000000,0.000000}%
\pgfsetstrokecolor{currentstroke}%
\pgfsetdash{}{0pt}%
\pgfpathmoveto{\pgfqpoint{1.137840in}{1.648422in}}%
\pgfpathlineto{\pgfqpoint{1.136968in}{1.648805in}}%
\pgfusepath{stroke}%
\end{pgfscope}%
\begin{pgfscope}%
\pgfpathrectangle{\pgfqpoint{0.100000in}{0.212622in}}{\pgfqpoint{3.696000in}{3.696000in}}%
\pgfusepath{clip}%
\pgfsetrectcap%
\pgfsetroundjoin%
\pgfsetlinewidth{1.505625pt}%
\definecolor{currentstroke}{rgb}{1.000000,0.000000,0.000000}%
\pgfsetstrokecolor{currentstroke}%
\pgfsetdash{}{0pt}%
\pgfpathmoveto{\pgfqpoint{1.137841in}{1.648422in}}%
\pgfpathlineto{\pgfqpoint{1.136968in}{1.648805in}}%
\pgfusepath{stroke}%
\end{pgfscope}%
\begin{pgfscope}%
\pgfpathrectangle{\pgfqpoint{0.100000in}{0.212622in}}{\pgfqpoint{3.696000in}{3.696000in}}%
\pgfusepath{clip}%
\pgfsetrectcap%
\pgfsetroundjoin%
\pgfsetlinewidth{1.505625pt}%
\definecolor{currentstroke}{rgb}{1.000000,0.000000,0.000000}%
\pgfsetstrokecolor{currentstroke}%
\pgfsetdash{}{0pt}%
\pgfpathmoveto{\pgfqpoint{1.137841in}{1.648422in}}%
\pgfpathlineto{\pgfqpoint{1.136968in}{1.648805in}}%
\pgfusepath{stroke}%
\end{pgfscope}%
\begin{pgfscope}%
\pgfpathrectangle{\pgfqpoint{0.100000in}{0.212622in}}{\pgfqpoint{3.696000in}{3.696000in}}%
\pgfusepath{clip}%
\pgfsetrectcap%
\pgfsetroundjoin%
\pgfsetlinewidth{1.505625pt}%
\definecolor{currentstroke}{rgb}{1.000000,0.000000,0.000000}%
\pgfsetstrokecolor{currentstroke}%
\pgfsetdash{}{0pt}%
\pgfpathmoveto{\pgfqpoint{1.137841in}{1.648422in}}%
\pgfpathlineto{\pgfqpoint{1.136968in}{1.648805in}}%
\pgfusepath{stroke}%
\end{pgfscope}%
\begin{pgfscope}%
\pgfpathrectangle{\pgfqpoint{0.100000in}{0.212622in}}{\pgfqpoint{3.696000in}{3.696000in}}%
\pgfusepath{clip}%
\pgfsetrectcap%
\pgfsetroundjoin%
\pgfsetlinewidth{1.505625pt}%
\definecolor{currentstroke}{rgb}{1.000000,0.000000,0.000000}%
\pgfsetstrokecolor{currentstroke}%
\pgfsetdash{}{0pt}%
\pgfpathmoveto{\pgfqpoint{1.137841in}{1.648422in}}%
\pgfpathlineto{\pgfqpoint{1.136968in}{1.648805in}}%
\pgfusepath{stroke}%
\end{pgfscope}%
\begin{pgfscope}%
\pgfpathrectangle{\pgfqpoint{0.100000in}{0.212622in}}{\pgfqpoint{3.696000in}{3.696000in}}%
\pgfusepath{clip}%
\pgfsetrectcap%
\pgfsetroundjoin%
\pgfsetlinewidth{1.505625pt}%
\definecolor{currentstroke}{rgb}{1.000000,0.000000,0.000000}%
\pgfsetstrokecolor{currentstroke}%
\pgfsetdash{}{0pt}%
\pgfpathmoveto{\pgfqpoint{1.137841in}{1.648422in}}%
\pgfpathlineto{\pgfqpoint{1.136968in}{1.648805in}}%
\pgfusepath{stroke}%
\end{pgfscope}%
\begin{pgfscope}%
\pgfpathrectangle{\pgfqpoint{0.100000in}{0.212622in}}{\pgfqpoint{3.696000in}{3.696000in}}%
\pgfusepath{clip}%
\pgfsetrectcap%
\pgfsetroundjoin%
\pgfsetlinewidth{1.505625pt}%
\definecolor{currentstroke}{rgb}{1.000000,0.000000,0.000000}%
\pgfsetstrokecolor{currentstroke}%
\pgfsetdash{}{0pt}%
\pgfpathmoveto{\pgfqpoint{1.137841in}{1.648422in}}%
\pgfpathlineto{\pgfqpoint{1.136968in}{1.648805in}}%
\pgfusepath{stroke}%
\end{pgfscope}%
\begin{pgfscope}%
\pgfpathrectangle{\pgfqpoint{0.100000in}{0.212622in}}{\pgfqpoint{3.696000in}{3.696000in}}%
\pgfusepath{clip}%
\pgfsetrectcap%
\pgfsetroundjoin%
\pgfsetlinewidth{1.505625pt}%
\definecolor{currentstroke}{rgb}{1.000000,0.000000,0.000000}%
\pgfsetstrokecolor{currentstroke}%
\pgfsetdash{}{0pt}%
\pgfpathmoveto{\pgfqpoint{1.137841in}{1.648422in}}%
\pgfpathlineto{\pgfqpoint{1.136968in}{1.648805in}}%
\pgfusepath{stroke}%
\end{pgfscope}%
\begin{pgfscope}%
\pgfpathrectangle{\pgfqpoint{0.100000in}{0.212622in}}{\pgfqpoint{3.696000in}{3.696000in}}%
\pgfusepath{clip}%
\pgfsetrectcap%
\pgfsetroundjoin%
\pgfsetlinewidth{1.505625pt}%
\definecolor{currentstroke}{rgb}{1.000000,0.000000,0.000000}%
\pgfsetstrokecolor{currentstroke}%
\pgfsetdash{}{0pt}%
\pgfpathmoveto{\pgfqpoint{1.137841in}{1.648422in}}%
\pgfpathlineto{\pgfqpoint{1.136968in}{1.648805in}}%
\pgfusepath{stroke}%
\end{pgfscope}%
\begin{pgfscope}%
\pgfpathrectangle{\pgfqpoint{0.100000in}{0.212622in}}{\pgfqpoint{3.696000in}{3.696000in}}%
\pgfusepath{clip}%
\pgfsetrectcap%
\pgfsetroundjoin%
\pgfsetlinewidth{1.505625pt}%
\definecolor{currentstroke}{rgb}{1.000000,0.000000,0.000000}%
\pgfsetstrokecolor{currentstroke}%
\pgfsetdash{}{0pt}%
\pgfpathmoveto{\pgfqpoint{1.137841in}{1.648422in}}%
\pgfpathlineto{\pgfqpoint{1.136968in}{1.648805in}}%
\pgfusepath{stroke}%
\end{pgfscope}%
\begin{pgfscope}%
\pgfpathrectangle{\pgfqpoint{0.100000in}{0.212622in}}{\pgfqpoint{3.696000in}{3.696000in}}%
\pgfusepath{clip}%
\pgfsetrectcap%
\pgfsetroundjoin%
\pgfsetlinewidth{1.505625pt}%
\definecolor{currentstroke}{rgb}{1.000000,0.000000,0.000000}%
\pgfsetstrokecolor{currentstroke}%
\pgfsetdash{}{0pt}%
\pgfpathmoveto{\pgfqpoint{1.137841in}{1.648422in}}%
\pgfpathlineto{\pgfqpoint{1.136968in}{1.648805in}}%
\pgfusepath{stroke}%
\end{pgfscope}%
\begin{pgfscope}%
\pgfpathrectangle{\pgfqpoint{0.100000in}{0.212622in}}{\pgfqpoint{3.696000in}{3.696000in}}%
\pgfusepath{clip}%
\pgfsetrectcap%
\pgfsetroundjoin%
\pgfsetlinewidth{1.505625pt}%
\definecolor{currentstroke}{rgb}{1.000000,0.000000,0.000000}%
\pgfsetstrokecolor{currentstroke}%
\pgfsetdash{}{0pt}%
\pgfpathmoveto{\pgfqpoint{1.137841in}{1.648422in}}%
\pgfpathlineto{\pgfqpoint{1.136968in}{1.648805in}}%
\pgfusepath{stroke}%
\end{pgfscope}%
\begin{pgfscope}%
\pgfpathrectangle{\pgfqpoint{0.100000in}{0.212622in}}{\pgfqpoint{3.696000in}{3.696000in}}%
\pgfusepath{clip}%
\pgfsetrectcap%
\pgfsetroundjoin%
\pgfsetlinewidth{1.505625pt}%
\definecolor{currentstroke}{rgb}{1.000000,0.000000,0.000000}%
\pgfsetstrokecolor{currentstroke}%
\pgfsetdash{}{0pt}%
\pgfpathmoveto{\pgfqpoint{1.137841in}{1.648422in}}%
\pgfpathlineto{\pgfqpoint{1.136968in}{1.648805in}}%
\pgfusepath{stroke}%
\end{pgfscope}%
\begin{pgfscope}%
\pgfpathrectangle{\pgfqpoint{0.100000in}{0.212622in}}{\pgfqpoint{3.696000in}{3.696000in}}%
\pgfusepath{clip}%
\pgfsetrectcap%
\pgfsetroundjoin%
\pgfsetlinewidth{1.505625pt}%
\definecolor{currentstroke}{rgb}{1.000000,0.000000,0.000000}%
\pgfsetstrokecolor{currentstroke}%
\pgfsetdash{}{0pt}%
\pgfpathmoveto{\pgfqpoint{1.137841in}{1.648422in}}%
\pgfpathlineto{\pgfqpoint{1.136968in}{1.648805in}}%
\pgfusepath{stroke}%
\end{pgfscope}%
\begin{pgfscope}%
\pgfpathrectangle{\pgfqpoint{0.100000in}{0.212622in}}{\pgfqpoint{3.696000in}{3.696000in}}%
\pgfusepath{clip}%
\pgfsetrectcap%
\pgfsetroundjoin%
\pgfsetlinewidth{1.505625pt}%
\definecolor{currentstroke}{rgb}{1.000000,0.000000,0.000000}%
\pgfsetstrokecolor{currentstroke}%
\pgfsetdash{}{0pt}%
\pgfpathmoveto{\pgfqpoint{1.137841in}{1.648422in}}%
\pgfpathlineto{\pgfqpoint{1.136968in}{1.648805in}}%
\pgfusepath{stroke}%
\end{pgfscope}%
\begin{pgfscope}%
\pgfpathrectangle{\pgfqpoint{0.100000in}{0.212622in}}{\pgfqpoint{3.696000in}{3.696000in}}%
\pgfusepath{clip}%
\pgfsetrectcap%
\pgfsetroundjoin%
\pgfsetlinewidth{1.505625pt}%
\definecolor{currentstroke}{rgb}{1.000000,0.000000,0.000000}%
\pgfsetstrokecolor{currentstroke}%
\pgfsetdash{}{0pt}%
\pgfpathmoveto{\pgfqpoint{1.137841in}{1.648422in}}%
\pgfpathlineto{\pgfqpoint{1.136968in}{1.648805in}}%
\pgfusepath{stroke}%
\end{pgfscope}%
\begin{pgfscope}%
\pgfpathrectangle{\pgfqpoint{0.100000in}{0.212622in}}{\pgfqpoint{3.696000in}{3.696000in}}%
\pgfusepath{clip}%
\pgfsetrectcap%
\pgfsetroundjoin%
\pgfsetlinewidth{1.505625pt}%
\definecolor{currentstroke}{rgb}{1.000000,0.000000,0.000000}%
\pgfsetstrokecolor{currentstroke}%
\pgfsetdash{}{0pt}%
\pgfpathmoveto{\pgfqpoint{1.137841in}{1.648422in}}%
\pgfpathlineto{\pgfqpoint{1.136968in}{1.648805in}}%
\pgfusepath{stroke}%
\end{pgfscope}%
\begin{pgfscope}%
\pgfpathrectangle{\pgfqpoint{0.100000in}{0.212622in}}{\pgfqpoint{3.696000in}{3.696000in}}%
\pgfusepath{clip}%
\pgfsetrectcap%
\pgfsetroundjoin%
\pgfsetlinewidth{1.505625pt}%
\definecolor{currentstroke}{rgb}{1.000000,0.000000,0.000000}%
\pgfsetstrokecolor{currentstroke}%
\pgfsetdash{}{0pt}%
\pgfpathmoveto{\pgfqpoint{1.137841in}{1.648422in}}%
\pgfpathlineto{\pgfqpoint{1.136968in}{1.648805in}}%
\pgfusepath{stroke}%
\end{pgfscope}%
\begin{pgfscope}%
\pgfpathrectangle{\pgfqpoint{0.100000in}{0.212622in}}{\pgfqpoint{3.696000in}{3.696000in}}%
\pgfusepath{clip}%
\pgfsetrectcap%
\pgfsetroundjoin%
\pgfsetlinewidth{1.505625pt}%
\definecolor{currentstroke}{rgb}{1.000000,0.000000,0.000000}%
\pgfsetstrokecolor{currentstroke}%
\pgfsetdash{}{0pt}%
\pgfpathmoveto{\pgfqpoint{1.137841in}{1.648422in}}%
\pgfpathlineto{\pgfqpoint{1.136968in}{1.648805in}}%
\pgfusepath{stroke}%
\end{pgfscope}%
\begin{pgfscope}%
\pgfpathrectangle{\pgfqpoint{0.100000in}{0.212622in}}{\pgfqpoint{3.696000in}{3.696000in}}%
\pgfusepath{clip}%
\pgfsetrectcap%
\pgfsetroundjoin%
\pgfsetlinewidth{1.505625pt}%
\definecolor{currentstroke}{rgb}{1.000000,0.000000,0.000000}%
\pgfsetstrokecolor{currentstroke}%
\pgfsetdash{}{0pt}%
\pgfpathmoveto{\pgfqpoint{1.137841in}{1.648422in}}%
\pgfpathlineto{\pgfqpoint{1.136968in}{1.648805in}}%
\pgfusepath{stroke}%
\end{pgfscope}%
\begin{pgfscope}%
\pgfpathrectangle{\pgfqpoint{0.100000in}{0.212622in}}{\pgfqpoint{3.696000in}{3.696000in}}%
\pgfusepath{clip}%
\pgfsetrectcap%
\pgfsetroundjoin%
\pgfsetlinewidth{1.505625pt}%
\definecolor{currentstroke}{rgb}{1.000000,0.000000,0.000000}%
\pgfsetstrokecolor{currentstroke}%
\pgfsetdash{}{0pt}%
\pgfpathmoveto{\pgfqpoint{1.137841in}{1.648422in}}%
\pgfpathlineto{\pgfqpoint{1.136968in}{1.648805in}}%
\pgfusepath{stroke}%
\end{pgfscope}%
\begin{pgfscope}%
\pgfpathrectangle{\pgfqpoint{0.100000in}{0.212622in}}{\pgfqpoint{3.696000in}{3.696000in}}%
\pgfusepath{clip}%
\pgfsetrectcap%
\pgfsetroundjoin%
\pgfsetlinewidth{1.505625pt}%
\definecolor{currentstroke}{rgb}{1.000000,0.000000,0.000000}%
\pgfsetstrokecolor{currentstroke}%
\pgfsetdash{}{0pt}%
\pgfpathmoveto{\pgfqpoint{1.137841in}{1.648422in}}%
\pgfpathlineto{\pgfqpoint{1.136968in}{1.648805in}}%
\pgfusepath{stroke}%
\end{pgfscope}%
\begin{pgfscope}%
\pgfpathrectangle{\pgfqpoint{0.100000in}{0.212622in}}{\pgfqpoint{3.696000in}{3.696000in}}%
\pgfusepath{clip}%
\pgfsetrectcap%
\pgfsetroundjoin%
\pgfsetlinewidth{1.505625pt}%
\definecolor{currentstroke}{rgb}{1.000000,0.000000,0.000000}%
\pgfsetstrokecolor{currentstroke}%
\pgfsetdash{}{0pt}%
\pgfpathmoveto{\pgfqpoint{1.137841in}{1.648422in}}%
\pgfpathlineto{\pgfqpoint{1.136968in}{1.648805in}}%
\pgfusepath{stroke}%
\end{pgfscope}%
\begin{pgfscope}%
\pgfpathrectangle{\pgfqpoint{0.100000in}{0.212622in}}{\pgfqpoint{3.696000in}{3.696000in}}%
\pgfusepath{clip}%
\pgfsetrectcap%
\pgfsetroundjoin%
\pgfsetlinewidth{1.505625pt}%
\definecolor{currentstroke}{rgb}{1.000000,0.000000,0.000000}%
\pgfsetstrokecolor{currentstroke}%
\pgfsetdash{}{0pt}%
\pgfpathmoveto{\pgfqpoint{1.137841in}{1.648422in}}%
\pgfpathlineto{\pgfqpoint{1.136968in}{1.648805in}}%
\pgfusepath{stroke}%
\end{pgfscope}%
\begin{pgfscope}%
\pgfpathrectangle{\pgfqpoint{0.100000in}{0.212622in}}{\pgfqpoint{3.696000in}{3.696000in}}%
\pgfusepath{clip}%
\pgfsetrectcap%
\pgfsetroundjoin%
\pgfsetlinewidth{1.505625pt}%
\definecolor{currentstroke}{rgb}{1.000000,0.000000,0.000000}%
\pgfsetstrokecolor{currentstroke}%
\pgfsetdash{}{0pt}%
\pgfpathmoveto{\pgfqpoint{1.137841in}{1.648422in}}%
\pgfpathlineto{\pgfqpoint{1.136968in}{1.648805in}}%
\pgfusepath{stroke}%
\end{pgfscope}%
\begin{pgfscope}%
\pgfpathrectangle{\pgfqpoint{0.100000in}{0.212622in}}{\pgfqpoint{3.696000in}{3.696000in}}%
\pgfusepath{clip}%
\pgfsetrectcap%
\pgfsetroundjoin%
\pgfsetlinewidth{1.505625pt}%
\definecolor{currentstroke}{rgb}{1.000000,0.000000,0.000000}%
\pgfsetstrokecolor{currentstroke}%
\pgfsetdash{}{0pt}%
\pgfpathmoveto{\pgfqpoint{1.137841in}{1.648422in}}%
\pgfpathlineto{\pgfqpoint{1.136968in}{1.648805in}}%
\pgfusepath{stroke}%
\end{pgfscope}%
\begin{pgfscope}%
\pgfpathrectangle{\pgfqpoint{0.100000in}{0.212622in}}{\pgfqpoint{3.696000in}{3.696000in}}%
\pgfusepath{clip}%
\pgfsetrectcap%
\pgfsetroundjoin%
\pgfsetlinewidth{1.505625pt}%
\definecolor{currentstroke}{rgb}{1.000000,0.000000,0.000000}%
\pgfsetstrokecolor{currentstroke}%
\pgfsetdash{}{0pt}%
\pgfpathmoveto{\pgfqpoint{1.137841in}{1.648422in}}%
\pgfpathlineto{\pgfqpoint{1.136968in}{1.648805in}}%
\pgfusepath{stroke}%
\end{pgfscope}%
\begin{pgfscope}%
\pgfpathrectangle{\pgfqpoint{0.100000in}{0.212622in}}{\pgfqpoint{3.696000in}{3.696000in}}%
\pgfusepath{clip}%
\pgfsetrectcap%
\pgfsetroundjoin%
\pgfsetlinewidth{1.505625pt}%
\definecolor{currentstroke}{rgb}{1.000000,0.000000,0.000000}%
\pgfsetstrokecolor{currentstroke}%
\pgfsetdash{}{0pt}%
\pgfpathmoveto{\pgfqpoint{1.137841in}{1.648422in}}%
\pgfpathlineto{\pgfqpoint{1.136968in}{1.648805in}}%
\pgfusepath{stroke}%
\end{pgfscope}%
\begin{pgfscope}%
\pgfpathrectangle{\pgfqpoint{0.100000in}{0.212622in}}{\pgfqpoint{3.696000in}{3.696000in}}%
\pgfusepath{clip}%
\pgfsetrectcap%
\pgfsetroundjoin%
\pgfsetlinewidth{1.505625pt}%
\definecolor{currentstroke}{rgb}{1.000000,0.000000,0.000000}%
\pgfsetstrokecolor{currentstroke}%
\pgfsetdash{}{0pt}%
\pgfpathmoveto{\pgfqpoint{1.137841in}{1.648422in}}%
\pgfpathlineto{\pgfqpoint{1.136968in}{1.648805in}}%
\pgfusepath{stroke}%
\end{pgfscope}%
\begin{pgfscope}%
\pgfpathrectangle{\pgfqpoint{0.100000in}{0.212622in}}{\pgfqpoint{3.696000in}{3.696000in}}%
\pgfusepath{clip}%
\pgfsetrectcap%
\pgfsetroundjoin%
\pgfsetlinewidth{1.505625pt}%
\definecolor{currentstroke}{rgb}{1.000000,0.000000,0.000000}%
\pgfsetstrokecolor{currentstroke}%
\pgfsetdash{}{0pt}%
\pgfpathmoveto{\pgfqpoint{1.137841in}{1.648422in}}%
\pgfpathlineto{\pgfqpoint{1.136968in}{1.648805in}}%
\pgfusepath{stroke}%
\end{pgfscope}%
\begin{pgfscope}%
\pgfpathrectangle{\pgfqpoint{0.100000in}{0.212622in}}{\pgfqpoint{3.696000in}{3.696000in}}%
\pgfusepath{clip}%
\pgfsetrectcap%
\pgfsetroundjoin%
\pgfsetlinewidth{1.505625pt}%
\definecolor{currentstroke}{rgb}{1.000000,0.000000,0.000000}%
\pgfsetstrokecolor{currentstroke}%
\pgfsetdash{}{0pt}%
\pgfpathmoveto{\pgfqpoint{1.137841in}{1.648422in}}%
\pgfpathlineto{\pgfqpoint{1.136968in}{1.648805in}}%
\pgfusepath{stroke}%
\end{pgfscope}%
\begin{pgfscope}%
\pgfpathrectangle{\pgfqpoint{0.100000in}{0.212622in}}{\pgfqpoint{3.696000in}{3.696000in}}%
\pgfusepath{clip}%
\pgfsetrectcap%
\pgfsetroundjoin%
\pgfsetlinewidth{1.505625pt}%
\definecolor{currentstroke}{rgb}{1.000000,0.000000,0.000000}%
\pgfsetstrokecolor{currentstroke}%
\pgfsetdash{}{0pt}%
\pgfpathmoveto{\pgfqpoint{1.137841in}{1.648422in}}%
\pgfpathlineto{\pgfqpoint{1.136968in}{1.648805in}}%
\pgfusepath{stroke}%
\end{pgfscope}%
\begin{pgfscope}%
\pgfpathrectangle{\pgfqpoint{0.100000in}{0.212622in}}{\pgfqpoint{3.696000in}{3.696000in}}%
\pgfusepath{clip}%
\pgfsetrectcap%
\pgfsetroundjoin%
\pgfsetlinewidth{1.505625pt}%
\definecolor{currentstroke}{rgb}{1.000000,0.000000,0.000000}%
\pgfsetstrokecolor{currentstroke}%
\pgfsetdash{}{0pt}%
\pgfpathmoveto{\pgfqpoint{1.137841in}{1.648422in}}%
\pgfpathlineto{\pgfqpoint{1.136968in}{1.648805in}}%
\pgfusepath{stroke}%
\end{pgfscope}%
\begin{pgfscope}%
\pgfpathrectangle{\pgfqpoint{0.100000in}{0.212622in}}{\pgfqpoint{3.696000in}{3.696000in}}%
\pgfusepath{clip}%
\pgfsetrectcap%
\pgfsetroundjoin%
\pgfsetlinewidth{1.505625pt}%
\definecolor{currentstroke}{rgb}{1.000000,0.000000,0.000000}%
\pgfsetstrokecolor{currentstroke}%
\pgfsetdash{}{0pt}%
\pgfpathmoveto{\pgfqpoint{1.137841in}{1.648422in}}%
\pgfpathlineto{\pgfqpoint{1.136968in}{1.648805in}}%
\pgfusepath{stroke}%
\end{pgfscope}%
\begin{pgfscope}%
\pgfpathrectangle{\pgfqpoint{0.100000in}{0.212622in}}{\pgfqpoint{3.696000in}{3.696000in}}%
\pgfusepath{clip}%
\pgfsetrectcap%
\pgfsetroundjoin%
\pgfsetlinewidth{1.505625pt}%
\definecolor{currentstroke}{rgb}{1.000000,0.000000,0.000000}%
\pgfsetstrokecolor{currentstroke}%
\pgfsetdash{}{0pt}%
\pgfpathmoveto{\pgfqpoint{1.137841in}{1.648422in}}%
\pgfpathlineto{\pgfqpoint{1.136968in}{1.648805in}}%
\pgfusepath{stroke}%
\end{pgfscope}%
\begin{pgfscope}%
\pgfpathrectangle{\pgfqpoint{0.100000in}{0.212622in}}{\pgfqpoint{3.696000in}{3.696000in}}%
\pgfusepath{clip}%
\pgfsetrectcap%
\pgfsetroundjoin%
\pgfsetlinewidth{1.505625pt}%
\definecolor{currentstroke}{rgb}{1.000000,0.000000,0.000000}%
\pgfsetstrokecolor{currentstroke}%
\pgfsetdash{}{0pt}%
\pgfpathmoveto{\pgfqpoint{1.137841in}{1.648422in}}%
\pgfpathlineto{\pgfqpoint{1.136968in}{1.648805in}}%
\pgfusepath{stroke}%
\end{pgfscope}%
\begin{pgfscope}%
\pgfpathrectangle{\pgfqpoint{0.100000in}{0.212622in}}{\pgfqpoint{3.696000in}{3.696000in}}%
\pgfusepath{clip}%
\pgfsetrectcap%
\pgfsetroundjoin%
\pgfsetlinewidth{1.505625pt}%
\definecolor{currentstroke}{rgb}{1.000000,0.000000,0.000000}%
\pgfsetstrokecolor{currentstroke}%
\pgfsetdash{}{0pt}%
\pgfpathmoveto{\pgfqpoint{1.137841in}{1.648422in}}%
\pgfpathlineto{\pgfqpoint{1.136968in}{1.648805in}}%
\pgfusepath{stroke}%
\end{pgfscope}%
\begin{pgfscope}%
\pgfpathrectangle{\pgfqpoint{0.100000in}{0.212622in}}{\pgfqpoint{3.696000in}{3.696000in}}%
\pgfusepath{clip}%
\pgfsetrectcap%
\pgfsetroundjoin%
\pgfsetlinewidth{1.505625pt}%
\definecolor{currentstroke}{rgb}{1.000000,0.000000,0.000000}%
\pgfsetstrokecolor{currentstroke}%
\pgfsetdash{}{0pt}%
\pgfpathmoveto{\pgfqpoint{1.137841in}{1.648422in}}%
\pgfpathlineto{\pgfqpoint{1.136968in}{1.648805in}}%
\pgfusepath{stroke}%
\end{pgfscope}%
\begin{pgfscope}%
\pgfpathrectangle{\pgfqpoint{0.100000in}{0.212622in}}{\pgfqpoint{3.696000in}{3.696000in}}%
\pgfusepath{clip}%
\pgfsetrectcap%
\pgfsetroundjoin%
\pgfsetlinewidth{1.505625pt}%
\definecolor{currentstroke}{rgb}{1.000000,0.000000,0.000000}%
\pgfsetstrokecolor{currentstroke}%
\pgfsetdash{}{0pt}%
\pgfpathmoveto{\pgfqpoint{1.137841in}{1.648422in}}%
\pgfpathlineto{\pgfqpoint{1.136968in}{1.648805in}}%
\pgfusepath{stroke}%
\end{pgfscope}%
\begin{pgfscope}%
\pgfpathrectangle{\pgfqpoint{0.100000in}{0.212622in}}{\pgfqpoint{3.696000in}{3.696000in}}%
\pgfusepath{clip}%
\pgfsetrectcap%
\pgfsetroundjoin%
\pgfsetlinewidth{1.505625pt}%
\definecolor{currentstroke}{rgb}{1.000000,0.000000,0.000000}%
\pgfsetstrokecolor{currentstroke}%
\pgfsetdash{}{0pt}%
\pgfpathmoveto{\pgfqpoint{1.137841in}{1.648422in}}%
\pgfpathlineto{\pgfqpoint{1.136968in}{1.648805in}}%
\pgfusepath{stroke}%
\end{pgfscope}%
\begin{pgfscope}%
\pgfpathrectangle{\pgfqpoint{0.100000in}{0.212622in}}{\pgfqpoint{3.696000in}{3.696000in}}%
\pgfusepath{clip}%
\pgfsetrectcap%
\pgfsetroundjoin%
\pgfsetlinewidth{1.505625pt}%
\definecolor{currentstroke}{rgb}{1.000000,0.000000,0.000000}%
\pgfsetstrokecolor{currentstroke}%
\pgfsetdash{}{0pt}%
\pgfpathmoveto{\pgfqpoint{1.137841in}{1.648422in}}%
\pgfpathlineto{\pgfqpoint{1.136968in}{1.648805in}}%
\pgfusepath{stroke}%
\end{pgfscope}%
\begin{pgfscope}%
\pgfpathrectangle{\pgfqpoint{0.100000in}{0.212622in}}{\pgfqpoint{3.696000in}{3.696000in}}%
\pgfusepath{clip}%
\pgfsetrectcap%
\pgfsetroundjoin%
\pgfsetlinewidth{1.505625pt}%
\definecolor{currentstroke}{rgb}{1.000000,0.000000,0.000000}%
\pgfsetstrokecolor{currentstroke}%
\pgfsetdash{}{0pt}%
\pgfpathmoveto{\pgfqpoint{1.137841in}{1.648422in}}%
\pgfpathlineto{\pgfqpoint{1.136968in}{1.648805in}}%
\pgfusepath{stroke}%
\end{pgfscope}%
\begin{pgfscope}%
\pgfpathrectangle{\pgfqpoint{0.100000in}{0.212622in}}{\pgfqpoint{3.696000in}{3.696000in}}%
\pgfusepath{clip}%
\pgfsetrectcap%
\pgfsetroundjoin%
\pgfsetlinewidth{1.505625pt}%
\definecolor{currentstroke}{rgb}{1.000000,0.000000,0.000000}%
\pgfsetstrokecolor{currentstroke}%
\pgfsetdash{}{0pt}%
\pgfpathmoveto{\pgfqpoint{1.137841in}{1.648422in}}%
\pgfpathlineto{\pgfqpoint{1.136968in}{1.648805in}}%
\pgfusepath{stroke}%
\end{pgfscope}%
\begin{pgfscope}%
\pgfpathrectangle{\pgfqpoint{0.100000in}{0.212622in}}{\pgfqpoint{3.696000in}{3.696000in}}%
\pgfusepath{clip}%
\pgfsetrectcap%
\pgfsetroundjoin%
\pgfsetlinewidth{1.505625pt}%
\definecolor{currentstroke}{rgb}{1.000000,0.000000,0.000000}%
\pgfsetstrokecolor{currentstroke}%
\pgfsetdash{}{0pt}%
\pgfpathmoveto{\pgfqpoint{1.137841in}{1.648422in}}%
\pgfpathlineto{\pgfqpoint{1.136968in}{1.648805in}}%
\pgfusepath{stroke}%
\end{pgfscope}%
\begin{pgfscope}%
\pgfpathrectangle{\pgfqpoint{0.100000in}{0.212622in}}{\pgfqpoint{3.696000in}{3.696000in}}%
\pgfusepath{clip}%
\pgfsetrectcap%
\pgfsetroundjoin%
\pgfsetlinewidth{1.505625pt}%
\definecolor{currentstroke}{rgb}{1.000000,0.000000,0.000000}%
\pgfsetstrokecolor{currentstroke}%
\pgfsetdash{}{0pt}%
\pgfpathmoveto{\pgfqpoint{1.137841in}{1.648422in}}%
\pgfpathlineto{\pgfqpoint{1.136968in}{1.648805in}}%
\pgfusepath{stroke}%
\end{pgfscope}%
\begin{pgfscope}%
\pgfpathrectangle{\pgfqpoint{0.100000in}{0.212622in}}{\pgfqpoint{3.696000in}{3.696000in}}%
\pgfusepath{clip}%
\pgfsetrectcap%
\pgfsetroundjoin%
\pgfsetlinewidth{1.505625pt}%
\definecolor{currentstroke}{rgb}{1.000000,0.000000,0.000000}%
\pgfsetstrokecolor{currentstroke}%
\pgfsetdash{}{0pt}%
\pgfpathmoveto{\pgfqpoint{1.137841in}{1.648422in}}%
\pgfpathlineto{\pgfqpoint{1.136968in}{1.648805in}}%
\pgfusepath{stroke}%
\end{pgfscope}%
\begin{pgfscope}%
\pgfpathrectangle{\pgfqpoint{0.100000in}{0.212622in}}{\pgfqpoint{3.696000in}{3.696000in}}%
\pgfusepath{clip}%
\pgfsetrectcap%
\pgfsetroundjoin%
\pgfsetlinewidth{1.505625pt}%
\definecolor{currentstroke}{rgb}{1.000000,0.000000,0.000000}%
\pgfsetstrokecolor{currentstroke}%
\pgfsetdash{}{0pt}%
\pgfpathmoveto{\pgfqpoint{1.137841in}{1.648422in}}%
\pgfpathlineto{\pgfqpoint{1.136968in}{1.648805in}}%
\pgfusepath{stroke}%
\end{pgfscope}%
\begin{pgfscope}%
\pgfpathrectangle{\pgfqpoint{0.100000in}{0.212622in}}{\pgfqpoint{3.696000in}{3.696000in}}%
\pgfusepath{clip}%
\pgfsetrectcap%
\pgfsetroundjoin%
\pgfsetlinewidth{1.505625pt}%
\definecolor{currentstroke}{rgb}{1.000000,0.000000,0.000000}%
\pgfsetstrokecolor{currentstroke}%
\pgfsetdash{}{0pt}%
\pgfpathmoveto{\pgfqpoint{1.137841in}{1.648422in}}%
\pgfpathlineto{\pgfqpoint{1.136968in}{1.648805in}}%
\pgfusepath{stroke}%
\end{pgfscope}%
\begin{pgfscope}%
\pgfpathrectangle{\pgfqpoint{0.100000in}{0.212622in}}{\pgfqpoint{3.696000in}{3.696000in}}%
\pgfusepath{clip}%
\pgfsetrectcap%
\pgfsetroundjoin%
\pgfsetlinewidth{1.505625pt}%
\definecolor{currentstroke}{rgb}{1.000000,0.000000,0.000000}%
\pgfsetstrokecolor{currentstroke}%
\pgfsetdash{}{0pt}%
\pgfpathmoveto{\pgfqpoint{1.137841in}{1.648422in}}%
\pgfpathlineto{\pgfqpoint{1.136968in}{1.648805in}}%
\pgfusepath{stroke}%
\end{pgfscope}%
\begin{pgfscope}%
\pgfpathrectangle{\pgfqpoint{0.100000in}{0.212622in}}{\pgfqpoint{3.696000in}{3.696000in}}%
\pgfusepath{clip}%
\pgfsetrectcap%
\pgfsetroundjoin%
\pgfsetlinewidth{1.505625pt}%
\definecolor{currentstroke}{rgb}{1.000000,0.000000,0.000000}%
\pgfsetstrokecolor{currentstroke}%
\pgfsetdash{}{0pt}%
\pgfpathmoveto{\pgfqpoint{1.137841in}{1.648422in}}%
\pgfpathlineto{\pgfqpoint{1.136968in}{1.648805in}}%
\pgfusepath{stroke}%
\end{pgfscope}%
\begin{pgfscope}%
\pgfpathrectangle{\pgfqpoint{0.100000in}{0.212622in}}{\pgfqpoint{3.696000in}{3.696000in}}%
\pgfusepath{clip}%
\pgfsetrectcap%
\pgfsetroundjoin%
\pgfsetlinewidth{1.505625pt}%
\definecolor{currentstroke}{rgb}{1.000000,0.000000,0.000000}%
\pgfsetstrokecolor{currentstroke}%
\pgfsetdash{}{0pt}%
\pgfpathmoveto{\pgfqpoint{1.137841in}{1.648422in}}%
\pgfpathlineto{\pgfqpoint{1.136968in}{1.648805in}}%
\pgfusepath{stroke}%
\end{pgfscope}%
\begin{pgfscope}%
\pgfpathrectangle{\pgfqpoint{0.100000in}{0.212622in}}{\pgfqpoint{3.696000in}{3.696000in}}%
\pgfusepath{clip}%
\pgfsetrectcap%
\pgfsetroundjoin%
\pgfsetlinewidth{1.505625pt}%
\definecolor{currentstroke}{rgb}{1.000000,0.000000,0.000000}%
\pgfsetstrokecolor{currentstroke}%
\pgfsetdash{}{0pt}%
\pgfpathmoveto{\pgfqpoint{1.137841in}{1.648422in}}%
\pgfpathlineto{\pgfqpoint{1.136968in}{1.648805in}}%
\pgfusepath{stroke}%
\end{pgfscope}%
\begin{pgfscope}%
\pgfpathrectangle{\pgfqpoint{0.100000in}{0.212622in}}{\pgfqpoint{3.696000in}{3.696000in}}%
\pgfusepath{clip}%
\pgfsetrectcap%
\pgfsetroundjoin%
\pgfsetlinewidth{1.505625pt}%
\definecolor{currentstroke}{rgb}{1.000000,0.000000,0.000000}%
\pgfsetstrokecolor{currentstroke}%
\pgfsetdash{}{0pt}%
\pgfpathmoveto{\pgfqpoint{1.137841in}{1.648422in}}%
\pgfpathlineto{\pgfqpoint{1.136968in}{1.648805in}}%
\pgfusepath{stroke}%
\end{pgfscope}%
\begin{pgfscope}%
\pgfpathrectangle{\pgfqpoint{0.100000in}{0.212622in}}{\pgfqpoint{3.696000in}{3.696000in}}%
\pgfusepath{clip}%
\pgfsetrectcap%
\pgfsetroundjoin%
\pgfsetlinewidth{1.505625pt}%
\definecolor{currentstroke}{rgb}{1.000000,0.000000,0.000000}%
\pgfsetstrokecolor{currentstroke}%
\pgfsetdash{}{0pt}%
\pgfpathmoveto{\pgfqpoint{1.137841in}{1.648422in}}%
\pgfpathlineto{\pgfqpoint{1.136968in}{1.648805in}}%
\pgfusepath{stroke}%
\end{pgfscope}%
\begin{pgfscope}%
\pgfpathrectangle{\pgfqpoint{0.100000in}{0.212622in}}{\pgfqpoint{3.696000in}{3.696000in}}%
\pgfusepath{clip}%
\pgfsetrectcap%
\pgfsetroundjoin%
\pgfsetlinewidth{1.505625pt}%
\definecolor{currentstroke}{rgb}{1.000000,0.000000,0.000000}%
\pgfsetstrokecolor{currentstroke}%
\pgfsetdash{}{0pt}%
\pgfpathmoveto{\pgfqpoint{1.137841in}{1.648422in}}%
\pgfpathlineto{\pgfqpoint{1.136968in}{1.648805in}}%
\pgfusepath{stroke}%
\end{pgfscope}%
\begin{pgfscope}%
\pgfpathrectangle{\pgfqpoint{0.100000in}{0.212622in}}{\pgfqpoint{3.696000in}{3.696000in}}%
\pgfusepath{clip}%
\pgfsetrectcap%
\pgfsetroundjoin%
\pgfsetlinewidth{1.505625pt}%
\definecolor{currentstroke}{rgb}{1.000000,0.000000,0.000000}%
\pgfsetstrokecolor{currentstroke}%
\pgfsetdash{}{0pt}%
\pgfpathmoveto{\pgfqpoint{1.137841in}{1.648422in}}%
\pgfpathlineto{\pgfqpoint{1.136968in}{1.648805in}}%
\pgfusepath{stroke}%
\end{pgfscope}%
\begin{pgfscope}%
\pgfpathrectangle{\pgfqpoint{0.100000in}{0.212622in}}{\pgfqpoint{3.696000in}{3.696000in}}%
\pgfusepath{clip}%
\pgfsetrectcap%
\pgfsetroundjoin%
\pgfsetlinewidth{1.505625pt}%
\definecolor{currentstroke}{rgb}{1.000000,0.000000,0.000000}%
\pgfsetstrokecolor{currentstroke}%
\pgfsetdash{}{0pt}%
\pgfpathmoveto{\pgfqpoint{1.137841in}{1.648422in}}%
\pgfpathlineto{\pgfqpoint{1.136968in}{1.648805in}}%
\pgfusepath{stroke}%
\end{pgfscope}%
\begin{pgfscope}%
\pgfpathrectangle{\pgfqpoint{0.100000in}{0.212622in}}{\pgfqpoint{3.696000in}{3.696000in}}%
\pgfusepath{clip}%
\pgfsetrectcap%
\pgfsetroundjoin%
\pgfsetlinewidth{1.505625pt}%
\definecolor{currentstroke}{rgb}{1.000000,0.000000,0.000000}%
\pgfsetstrokecolor{currentstroke}%
\pgfsetdash{}{0pt}%
\pgfpathmoveto{\pgfqpoint{1.137841in}{1.648422in}}%
\pgfpathlineto{\pgfqpoint{1.136968in}{1.648805in}}%
\pgfusepath{stroke}%
\end{pgfscope}%
\begin{pgfscope}%
\pgfpathrectangle{\pgfqpoint{0.100000in}{0.212622in}}{\pgfqpoint{3.696000in}{3.696000in}}%
\pgfusepath{clip}%
\pgfsetrectcap%
\pgfsetroundjoin%
\pgfsetlinewidth{1.505625pt}%
\definecolor{currentstroke}{rgb}{1.000000,0.000000,0.000000}%
\pgfsetstrokecolor{currentstroke}%
\pgfsetdash{}{0pt}%
\pgfpathmoveto{\pgfqpoint{1.137841in}{1.648422in}}%
\pgfpathlineto{\pgfqpoint{1.136968in}{1.648805in}}%
\pgfusepath{stroke}%
\end{pgfscope}%
\begin{pgfscope}%
\pgfpathrectangle{\pgfqpoint{0.100000in}{0.212622in}}{\pgfqpoint{3.696000in}{3.696000in}}%
\pgfusepath{clip}%
\pgfsetrectcap%
\pgfsetroundjoin%
\pgfsetlinewidth{1.505625pt}%
\definecolor{currentstroke}{rgb}{1.000000,0.000000,0.000000}%
\pgfsetstrokecolor{currentstroke}%
\pgfsetdash{}{0pt}%
\pgfpathmoveto{\pgfqpoint{1.137841in}{1.648422in}}%
\pgfpathlineto{\pgfqpoint{1.136968in}{1.648805in}}%
\pgfusepath{stroke}%
\end{pgfscope}%
\begin{pgfscope}%
\pgfpathrectangle{\pgfqpoint{0.100000in}{0.212622in}}{\pgfqpoint{3.696000in}{3.696000in}}%
\pgfusepath{clip}%
\pgfsetrectcap%
\pgfsetroundjoin%
\pgfsetlinewidth{1.505625pt}%
\definecolor{currentstroke}{rgb}{1.000000,0.000000,0.000000}%
\pgfsetstrokecolor{currentstroke}%
\pgfsetdash{}{0pt}%
\pgfpathmoveto{\pgfqpoint{1.137841in}{1.648422in}}%
\pgfpathlineto{\pgfqpoint{1.136968in}{1.648805in}}%
\pgfusepath{stroke}%
\end{pgfscope}%
\begin{pgfscope}%
\pgfpathrectangle{\pgfqpoint{0.100000in}{0.212622in}}{\pgfqpoint{3.696000in}{3.696000in}}%
\pgfusepath{clip}%
\pgfsetrectcap%
\pgfsetroundjoin%
\pgfsetlinewidth{1.505625pt}%
\definecolor{currentstroke}{rgb}{1.000000,0.000000,0.000000}%
\pgfsetstrokecolor{currentstroke}%
\pgfsetdash{}{0pt}%
\pgfpathmoveto{\pgfqpoint{1.137841in}{1.648422in}}%
\pgfpathlineto{\pgfqpoint{1.136968in}{1.648805in}}%
\pgfusepath{stroke}%
\end{pgfscope}%
\begin{pgfscope}%
\pgfpathrectangle{\pgfqpoint{0.100000in}{0.212622in}}{\pgfqpoint{3.696000in}{3.696000in}}%
\pgfusepath{clip}%
\pgfsetrectcap%
\pgfsetroundjoin%
\pgfsetlinewidth{1.505625pt}%
\definecolor{currentstroke}{rgb}{1.000000,0.000000,0.000000}%
\pgfsetstrokecolor{currentstroke}%
\pgfsetdash{}{0pt}%
\pgfpathmoveto{\pgfqpoint{1.137841in}{1.648422in}}%
\pgfpathlineto{\pgfqpoint{1.136968in}{1.648805in}}%
\pgfusepath{stroke}%
\end{pgfscope}%
\begin{pgfscope}%
\pgfpathrectangle{\pgfqpoint{0.100000in}{0.212622in}}{\pgfqpoint{3.696000in}{3.696000in}}%
\pgfusepath{clip}%
\pgfsetrectcap%
\pgfsetroundjoin%
\pgfsetlinewidth{1.505625pt}%
\definecolor{currentstroke}{rgb}{1.000000,0.000000,0.000000}%
\pgfsetstrokecolor{currentstroke}%
\pgfsetdash{}{0pt}%
\pgfpathmoveto{\pgfqpoint{1.137841in}{1.648422in}}%
\pgfpathlineto{\pgfqpoint{1.136968in}{1.648805in}}%
\pgfusepath{stroke}%
\end{pgfscope}%
\begin{pgfscope}%
\pgfpathrectangle{\pgfqpoint{0.100000in}{0.212622in}}{\pgfqpoint{3.696000in}{3.696000in}}%
\pgfusepath{clip}%
\pgfsetrectcap%
\pgfsetroundjoin%
\pgfsetlinewidth{1.505625pt}%
\definecolor{currentstroke}{rgb}{1.000000,0.000000,0.000000}%
\pgfsetstrokecolor{currentstroke}%
\pgfsetdash{}{0pt}%
\pgfpathmoveto{\pgfqpoint{1.137841in}{1.648422in}}%
\pgfpathlineto{\pgfqpoint{1.136968in}{1.648805in}}%
\pgfusepath{stroke}%
\end{pgfscope}%
\begin{pgfscope}%
\pgfpathrectangle{\pgfqpoint{0.100000in}{0.212622in}}{\pgfqpoint{3.696000in}{3.696000in}}%
\pgfusepath{clip}%
\pgfsetrectcap%
\pgfsetroundjoin%
\pgfsetlinewidth{1.505625pt}%
\definecolor{currentstroke}{rgb}{1.000000,0.000000,0.000000}%
\pgfsetstrokecolor{currentstroke}%
\pgfsetdash{}{0pt}%
\pgfpathmoveto{\pgfqpoint{1.137841in}{1.648422in}}%
\pgfpathlineto{\pgfqpoint{1.136968in}{1.648805in}}%
\pgfusepath{stroke}%
\end{pgfscope}%
\begin{pgfscope}%
\pgfpathrectangle{\pgfqpoint{0.100000in}{0.212622in}}{\pgfqpoint{3.696000in}{3.696000in}}%
\pgfusepath{clip}%
\pgfsetrectcap%
\pgfsetroundjoin%
\pgfsetlinewidth{1.505625pt}%
\definecolor{currentstroke}{rgb}{1.000000,0.000000,0.000000}%
\pgfsetstrokecolor{currentstroke}%
\pgfsetdash{}{0pt}%
\pgfpathmoveto{\pgfqpoint{1.138330in}{1.648297in}}%
\pgfpathlineto{\pgfqpoint{1.136968in}{1.648805in}}%
\pgfusepath{stroke}%
\end{pgfscope}%
\begin{pgfscope}%
\pgfpathrectangle{\pgfqpoint{0.100000in}{0.212622in}}{\pgfqpoint{3.696000in}{3.696000in}}%
\pgfusepath{clip}%
\pgfsetrectcap%
\pgfsetroundjoin%
\pgfsetlinewidth{1.505625pt}%
\definecolor{currentstroke}{rgb}{1.000000,0.000000,0.000000}%
\pgfsetstrokecolor{currentstroke}%
\pgfsetdash{}{0pt}%
\pgfpathmoveto{\pgfqpoint{1.138592in}{1.648222in}}%
\pgfpathlineto{\pgfqpoint{1.136968in}{1.648805in}}%
\pgfusepath{stroke}%
\end{pgfscope}%
\begin{pgfscope}%
\pgfpathrectangle{\pgfqpoint{0.100000in}{0.212622in}}{\pgfqpoint{3.696000in}{3.696000in}}%
\pgfusepath{clip}%
\pgfsetrectcap%
\pgfsetroundjoin%
\pgfsetlinewidth{1.505625pt}%
\definecolor{currentstroke}{rgb}{1.000000,0.000000,0.000000}%
\pgfsetstrokecolor{currentstroke}%
\pgfsetdash{}{0pt}%
\pgfpathmoveto{\pgfqpoint{1.138738in}{1.648182in}}%
\pgfpathlineto{\pgfqpoint{1.136968in}{1.648805in}}%
\pgfusepath{stroke}%
\end{pgfscope}%
\begin{pgfscope}%
\pgfpathrectangle{\pgfqpoint{0.100000in}{0.212622in}}{\pgfqpoint{3.696000in}{3.696000in}}%
\pgfusepath{clip}%
\pgfsetrectcap%
\pgfsetroundjoin%
\pgfsetlinewidth{1.505625pt}%
\definecolor{currentstroke}{rgb}{1.000000,0.000000,0.000000}%
\pgfsetstrokecolor{currentstroke}%
\pgfsetdash{}{0pt}%
\pgfpathmoveto{\pgfqpoint{1.138819in}{1.648161in}}%
\pgfpathlineto{\pgfqpoint{1.136968in}{1.648805in}}%
\pgfusepath{stroke}%
\end{pgfscope}%
\begin{pgfscope}%
\pgfpathrectangle{\pgfqpoint{0.100000in}{0.212622in}}{\pgfqpoint{3.696000in}{3.696000in}}%
\pgfusepath{clip}%
\pgfsetrectcap%
\pgfsetroundjoin%
\pgfsetlinewidth{1.505625pt}%
\definecolor{currentstroke}{rgb}{1.000000,0.000000,0.000000}%
\pgfsetstrokecolor{currentstroke}%
\pgfsetdash{}{0pt}%
\pgfpathmoveto{\pgfqpoint{1.138864in}{1.648149in}}%
\pgfpathlineto{\pgfqpoint{1.136968in}{1.648805in}}%
\pgfusepath{stroke}%
\end{pgfscope}%
\begin{pgfscope}%
\pgfpathrectangle{\pgfqpoint{0.100000in}{0.212622in}}{\pgfqpoint{3.696000in}{3.696000in}}%
\pgfusepath{clip}%
\pgfsetrectcap%
\pgfsetroundjoin%
\pgfsetlinewidth{1.505625pt}%
\definecolor{currentstroke}{rgb}{1.000000,0.000000,0.000000}%
\pgfsetstrokecolor{currentstroke}%
\pgfsetdash{}{0pt}%
\pgfpathmoveto{\pgfqpoint{1.138888in}{1.648142in}}%
\pgfpathlineto{\pgfqpoint{1.136968in}{1.648805in}}%
\pgfusepath{stroke}%
\end{pgfscope}%
\begin{pgfscope}%
\pgfpathrectangle{\pgfqpoint{0.100000in}{0.212622in}}{\pgfqpoint{3.696000in}{3.696000in}}%
\pgfusepath{clip}%
\pgfsetrectcap%
\pgfsetroundjoin%
\pgfsetlinewidth{1.505625pt}%
\definecolor{currentstroke}{rgb}{1.000000,0.000000,0.000000}%
\pgfsetstrokecolor{currentstroke}%
\pgfsetdash{}{0pt}%
\pgfpathmoveto{\pgfqpoint{1.138901in}{1.648138in}}%
\pgfpathlineto{\pgfqpoint{1.136968in}{1.648805in}}%
\pgfusepath{stroke}%
\end{pgfscope}%
\begin{pgfscope}%
\pgfpathrectangle{\pgfqpoint{0.100000in}{0.212622in}}{\pgfqpoint{3.696000in}{3.696000in}}%
\pgfusepath{clip}%
\pgfsetrectcap%
\pgfsetroundjoin%
\pgfsetlinewidth{1.505625pt}%
\definecolor{currentstroke}{rgb}{1.000000,0.000000,0.000000}%
\pgfsetstrokecolor{currentstroke}%
\pgfsetdash{}{0pt}%
\pgfpathmoveto{\pgfqpoint{1.138908in}{1.648136in}}%
\pgfpathlineto{\pgfqpoint{1.136968in}{1.648805in}}%
\pgfusepath{stroke}%
\end{pgfscope}%
\begin{pgfscope}%
\pgfpathrectangle{\pgfqpoint{0.100000in}{0.212622in}}{\pgfqpoint{3.696000in}{3.696000in}}%
\pgfusepath{clip}%
\pgfsetrectcap%
\pgfsetroundjoin%
\pgfsetlinewidth{1.505625pt}%
\definecolor{currentstroke}{rgb}{1.000000,0.000000,0.000000}%
\pgfsetstrokecolor{currentstroke}%
\pgfsetdash{}{0pt}%
\pgfpathmoveto{\pgfqpoint{1.138912in}{1.648135in}}%
\pgfpathlineto{\pgfqpoint{1.136968in}{1.648805in}}%
\pgfusepath{stroke}%
\end{pgfscope}%
\begin{pgfscope}%
\pgfpathrectangle{\pgfqpoint{0.100000in}{0.212622in}}{\pgfqpoint{3.696000in}{3.696000in}}%
\pgfusepath{clip}%
\pgfsetrectcap%
\pgfsetroundjoin%
\pgfsetlinewidth{1.505625pt}%
\definecolor{currentstroke}{rgb}{1.000000,0.000000,0.000000}%
\pgfsetstrokecolor{currentstroke}%
\pgfsetdash{}{0pt}%
\pgfpathmoveto{\pgfqpoint{1.138914in}{1.648135in}}%
\pgfpathlineto{\pgfqpoint{1.136968in}{1.648805in}}%
\pgfusepath{stroke}%
\end{pgfscope}%
\begin{pgfscope}%
\pgfpathrectangle{\pgfqpoint{0.100000in}{0.212622in}}{\pgfqpoint{3.696000in}{3.696000in}}%
\pgfusepath{clip}%
\pgfsetrectcap%
\pgfsetroundjoin%
\pgfsetlinewidth{1.505625pt}%
\definecolor{currentstroke}{rgb}{1.000000,0.000000,0.000000}%
\pgfsetstrokecolor{currentstroke}%
\pgfsetdash{}{0pt}%
\pgfpathmoveto{\pgfqpoint{1.138915in}{1.648134in}}%
\pgfpathlineto{\pgfqpoint{1.136968in}{1.648805in}}%
\pgfusepath{stroke}%
\end{pgfscope}%
\begin{pgfscope}%
\pgfpathrectangle{\pgfqpoint{0.100000in}{0.212622in}}{\pgfqpoint{3.696000in}{3.696000in}}%
\pgfusepath{clip}%
\pgfsetrectcap%
\pgfsetroundjoin%
\pgfsetlinewidth{1.505625pt}%
\definecolor{currentstroke}{rgb}{1.000000,0.000000,0.000000}%
\pgfsetstrokecolor{currentstroke}%
\pgfsetdash{}{0pt}%
\pgfpathmoveto{\pgfqpoint{1.138916in}{1.648134in}}%
\pgfpathlineto{\pgfqpoint{1.136968in}{1.648805in}}%
\pgfusepath{stroke}%
\end{pgfscope}%
\begin{pgfscope}%
\pgfpathrectangle{\pgfqpoint{0.100000in}{0.212622in}}{\pgfqpoint{3.696000in}{3.696000in}}%
\pgfusepath{clip}%
\pgfsetrectcap%
\pgfsetroundjoin%
\pgfsetlinewidth{1.505625pt}%
\definecolor{currentstroke}{rgb}{1.000000,0.000000,0.000000}%
\pgfsetstrokecolor{currentstroke}%
\pgfsetdash{}{0pt}%
\pgfpathmoveto{\pgfqpoint{1.138916in}{1.648134in}}%
\pgfpathlineto{\pgfqpoint{1.136968in}{1.648805in}}%
\pgfusepath{stroke}%
\end{pgfscope}%
\begin{pgfscope}%
\pgfpathrectangle{\pgfqpoint{0.100000in}{0.212622in}}{\pgfqpoint{3.696000in}{3.696000in}}%
\pgfusepath{clip}%
\pgfsetrectcap%
\pgfsetroundjoin%
\pgfsetlinewidth{1.505625pt}%
\definecolor{currentstroke}{rgb}{1.000000,0.000000,0.000000}%
\pgfsetstrokecolor{currentstroke}%
\pgfsetdash{}{0pt}%
\pgfpathmoveto{\pgfqpoint{1.138917in}{1.648134in}}%
\pgfpathlineto{\pgfqpoint{1.136968in}{1.648805in}}%
\pgfusepath{stroke}%
\end{pgfscope}%
\begin{pgfscope}%
\pgfpathrectangle{\pgfqpoint{0.100000in}{0.212622in}}{\pgfqpoint{3.696000in}{3.696000in}}%
\pgfusepath{clip}%
\pgfsetrectcap%
\pgfsetroundjoin%
\pgfsetlinewidth{1.505625pt}%
\definecolor{currentstroke}{rgb}{1.000000,0.000000,0.000000}%
\pgfsetstrokecolor{currentstroke}%
\pgfsetdash{}{0pt}%
\pgfpathmoveto{\pgfqpoint{1.138917in}{1.648134in}}%
\pgfpathlineto{\pgfqpoint{1.136968in}{1.648805in}}%
\pgfusepath{stroke}%
\end{pgfscope}%
\begin{pgfscope}%
\pgfpathrectangle{\pgfqpoint{0.100000in}{0.212622in}}{\pgfqpoint{3.696000in}{3.696000in}}%
\pgfusepath{clip}%
\pgfsetrectcap%
\pgfsetroundjoin%
\pgfsetlinewidth{1.505625pt}%
\definecolor{currentstroke}{rgb}{1.000000,0.000000,0.000000}%
\pgfsetstrokecolor{currentstroke}%
\pgfsetdash{}{0pt}%
\pgfpathmoveto{\pgfqpoint{1.138917in}{1.648134in}}%
\pgfpathlineto{\pgfqpoint{1.136968in}{1.648805in}}%
\pgfusepath{stroke}%
\end{pgfscope}%
\begin{pgfscope}%
\pgfpathrectangle{\pgfqpoint{0.100000in}{0.212622in}}{\pgfqpoint{3.696000in}{3.696000in}}%
\pgfusepath{clip}%
\pgfsetrectcap%
\pgfsetroundjoin%
\pgfsetlinewidth{1.505625pt}%
\definecolor{currentstroke}{rgb}{1.000000,0.000000,0.000000}%
\pgfsetstrokecolor{currentstroke}%
\pgfsetdash{}{0pt}%
\pgfpathmoveto{\pgfqpoint{1.138917in}{1.648134in}}%
\pgfpathlineto{\pgfqpoint{1.136968in}{1.648805in}}%
\pgfusepath{stroke}%
\end{pgfscope}%
\begin{pgfscope}%
\pgfpathrectangle{\pgfqpoint{0.100000in}{0.212622in}}{\pgfqpoint{3.696000in}{3.696000in}}%
\pgfusepath{clip}%
\pgfsetrectcap%
\pgfsetroundjoin%
\pgfsetlinewidth{1.505625pt}%
\definecolor{currentstroke}{rgb}{1.000000,0.000000,0.000000}%
\pgfsetstrokecolor{currentstroke}%
\pgfsetdash{}{0pt}%
\pgfpathmoveto{\pgfqpoint{1.138917in}{1.648134in}}%
\pgfpathlineto{\pgfqpoint{1.136968in}{1.648805in}}%
\pgfusepath{stroke}%
\end{pgfscope}%
\begin{pgfscope}%
\pgfpathrectangle{\pgfqpoint{0.100000in}{0.212622in}}{\pgfqpoint{3.696000in}{3.696000in}}%
\pgfusepath{clip}%
\pgfsetrectcap%
\pgfsetroundjoin%
\pgfsetlinewidth{1.505625pt}%
\definecolor{currentstroke}{rgb}{1.000000,0.000000,0.000000}%
\pgfsetstrokecolor{currentstroke}%
\pgfsetdash{}{0pt}%
\pgfpathmoveto{\pgfqpoint{1.138917in}{1.648134in}}%
\pgfpathlineto{\pgfqpoint{1.136968in}{1.648805in}}%
\pgfusepath{stroke}%
\end{pgfscope}%
\begin{pgfscope}%
\pgfpathrectangle{\pgfqpoint{0.100000in}{0.212622in}}{\pgfqpoint{3.696000in}{3.696000in}}%
\pgfusepath{clip}%
\pgfsetrectcap%
\pgfsetroundjoin%
\pgfsetlinewidth{1.505625pt}%
\definecolor{currentstroke}{rgb}{1.000000,0.000000,0.000000}%
\pgfsetstrokecolor{currentstroke}%
\pgfsetdash{}{0pt}%
\pgfpathmoveto{\pgfqpoint{1.138917in}{1.648134in}}%
\pgfpathlineto{\pgfqpoint{1.136968in}{1.648805in}}%
\pgfusepath{stroke}%
\end{pgfscope}%
\begin{pgfscope}%
\pgfpathrectangle{\pgfqpoint{0.100000in}{0.212622in}}{\pgfqpoint{3.696000in}{3.696000in}}%
\pgfusepath{clip}%
\pgfsetrectcap%
\pgfsetroundjoin%
\pgfsetlinewidth{1.505625pt}%
\definecolor{currentstroke}{rgb}{1.000000,0.000000,0.000000}%
\pgfsetstrokecolor{currentstroke}%
\pgfsetdash{}{0pt}%
\pgfpathmoveto{\pgfqpoint{1.138917in}{1.648134in}}%
\pgfpathlineto{\pgfqpoint{1.136968in}{1.648805in}}%
\pgfusepath{stroke}%
\end{pgfscope}%
\begin{pgfscope}%
\pgfpathrectangle{\pgfqpoint{0.100000in}{0.212622in}}{\pgfqpoint{3.696000in}{3.696000in}}%
\pgfusepath{clip}%
\pgfsetrectcap%
\pgfsetroundjoin%
\pgfsetlinewidth{1.505625pt}%
\definecolor{currentstroke}{rgb}{1.000000,0.000000,0.000000}%
\pgfsetstrokecolor{currentstroke}%
\pgfsetdash{}{0pt}%
\pgfpathmoveto{\pgfqpoint{1.138917in}{1.648134in}}%
\pgfpathlineto{\pgfqpoint{1.136968in}{1.648805in}}%
\pgfusepath{stroke}%
\end{pgfscope}%
\begin{pgfscope}%
\pgfpathrectangle{\pgfqpoint{0.100000in}{0.212622in}}{\pgfqpoint{3.696000in}{3.696000in}}%
\pgfusepath{clip}%
\pgfsetrectcap%
\pgfsetroundjoin%
\pgfsetlinewidth{1.505625pt}%
\definecolor{currentstroke}{rgb}{1.000000,0.000000,0.000000}%
\pgfsetstrokecolor{currentstroke}%
\pgfsetdash{}{0pt}%
\pgfpathmoveto{\pgfqpoint{1.138917in}{1.648134in}}%
\pgfpathlineto{\pgfqpoint{1.136968in}{1.648805in}}%
\pgfusepath{stroke}%
\end{pgfscope}%
\begin{pgfscope}%
\pgfpathrectangle{\pgfqpoint{0.100000in}{0.212622in}}{\pgfqpoint{3.696000in}{3.696000in}}%
\pgfusepath{clip}%
\pgfsetrectcap%
\pgfsetroundjoin%
\pgfsetlinewidth{1.505625pt}%
\definecolor{currentstroke}{rgb}{1.000000,0.000000,0.000000}%
\pgfsetstrokecolor{currentstroke}%
\pgfsetdash{}{0pt}%
\pgfpathmoveto{\pgfqpoint{1.138917in}{1.648134in}}%
\pgfpathlineto{\pgfqpoint{1.136968in}{1.648805in}}%
\pgfusepath{stroke}%
\end{pgfscope}%
\begin{pgfscope}%
\pgfpathrectangle{\pgfqpoint{0.100000in}{0.212622in}}{\pgfqpoint{3.696000in}{3.696000in}}%
\pgfusepath{clip}%
\pgfsetrectcap%
\pgfsetroundjoin%
\pgfsetlinewidth{1.505625pt}%
\definecolor{currentstroke}{rgb}{1.000000,0.000000,0.000000}%
\pgfsetstrokecolor{currentstroke}%
\pgfsetdash{}{0pt}%
\pgfpathmoveto{\pgfqpoint{1.138917in}{1.648134in}}%
\pgfpathlineto{\pgfqpoint{1.136968in}{1.648805in}}%
\pgfusepath{stroke}%
\end{pgfscope}%
\begin{pgfscope}%
\pgfpathrectangle{\pgfqpoint{0.100000in}{0.212622in}}{\pgfqpoint{3.696000in}{3.696000in}}%
\pgfusepath{clip}%
\pgfsetrectcap%
\pgfsetroundjoin%
\pgfsetlinewidth{1.505625pt}%
\definecolor{currentstroke}{rgb}{1.000000,0.000000,0.000000}%
\pgfsetstrokecolor{currentstroke}%
\pgfsetdash{}{0pt}%
\pgfpathmoveto{\pgfqpoint{1.138917in}{1.648134in}}%
\pgfpathlineto{\pgfqpoint{1.136968in}{1.648805in}}%
\pgfusepath{stroke}%
\end{pgfscope}%
\begin{pgfscope}%
\pgfpathrectangle{\pgfqpoint{0.100000in}{0.212622in}}{\pgfqpoint{3.696000in}{3.696000in}}%
\pgfusepath{clip}%
\pgfsetrectcap%
\pgfsetroundjoin%
\pgfsetlinewidth{1.505625pt}%
\definecolor{currentstroke}{rgb}{1.000000,0.000000,0.000000}%
\pgfsetstrokecolor{currentstroke}%
\pgfsetdash{}{0pt}%
\pgfpathmoveto{\pgfqpoint{1.138917in}{1.648134in}}%
\pgfpathlineto{\pgfqpoint{1.136968in}{1.648805in}}%
\pgfusepath{stroke}%
\end{pgfscope}%
\begin{pgfscope}%
\pgfpathrectangle{\pgfqpoint{0.100000in}{0.212622in}}{\pgfqpoint{3.696000in}{3.696000in}}%
\pgfusepath{clip}%
\pgfsetrectcap%
\pgfsetroundjoin%
\pgfsetlinewidth{1.505625pt}%
\definecolor{currentstroke}{rgb}{1.000000,0.000000,0.000000}%
\pgfsetstrokecolor{currentstroke}%
\pgfsetdash{}{0pt}%
\pgfpathmoveto{\pgfqpoint{1.138917in}{1.648134in}}%
\pgfpathlineto{\pgfqpoint{1.136968in}{1.648805in}}%
\pgfusepath{stroke}%
\end{pgfscope}%
\begin{pgfscope}%
\pgfpathrectangle{\pgfqpoint{0.100000in}{0.212622in}}{\pgfqpoint{3.696000in}{3.696000in}}%
\pgfusepath{clip}%
\pgfsetrectcap%
\pgfsetroundjoin%
\pgfsetlinewidth{1.505625pt}%
\definecolor{currentstroke}{rgb}{1.000000,0.000000,0.000000}%
\pgfsetstrokecolor{currentstroke}%
\pgfsetdash{}{0pt}%
\pgfpathmoveto{\pgfqpoint{1.138917in}{1.648134in}}%
\pgfpathlineto{\pgfqpoint{1.136968in}{1.648805in}}%
\pgfusepath{stroke}%
\end{pgfscope}%
\begin{pgfscope}%
\pgfpathrectangle{\pgfqpoint{0.100000in}{0.212622in}}{\pgfqpoint{3.696000in}{3.696000in}}%
\pgfusepath{clip}%
\pgfsetrectcap%
\pgfsetroundjoin%
\pgfsetlinewidth{1.505625pt}%
\definecolor{currentstroke}{rgb}{1.000000,0.000000,0.000000}%
\pgfsetstrokecolor{currentstroke}%
\pgfsetdash{}{0pt}%
\pgfpathmoveto{\pgfqpoint{1.138917in}{1.648134in}}%
\pgfpathlineto{\pgfqpoint{1.136968in}{1.648805in}}%
\pgfusepath{stroke}%
\end{pgfscope}%
\begin{pgfscope}%
\pgfpathrectangle{\pgfqpoint{0.100000in}{0.212622in}}{\pgfqpoint{3.696000in}{3.696000in}}%
\pgfusepath{clip}%
\pgfsetrectcap%
\pgfsetroundjoin%
\pgfsetlinewidth{1.505625pt}%
\definecolor{currentstroke}{rgb}{1.000000,0.000000,0.000000}%
\pgfsetstrokecolor{currentstroke}%
\pgfsetdash{}{0pt}%
\pgfpathmoveto{\pgfqpoint{1.138917in}{1.648134in}}%
\pgfpathlineto{\pgfqpoint{1.136968in}{1.648805in}}%
\pgfusepath{stroke}%
\end{pgfscope}%
\begin{pgfscope}%
\pgfpathrectangle{\pgfqpoint{0.100000in}{0.212622in}}{\pgfqpoint{3.696000in}{3.696000in}}%
\pgfusepath{clip}%
\pgfsetrectcap%
\pgfsetroundjoin%
\pgfsetlinewidth{1.505625pt}%
\definecolor{currentstroke}{rgb}{1.000000,0.000000,0.000000}%
\pgfsetstrokecolor{currentstroke}%
\pgfsetdash{}{0pt}%
\pgfpathmoveto{\pgfqpoint{1.138917in}{1.648134in}}%
\pgfpathlineto{\pgfqpoint{1.136968in}{1.648805in}}%
\pgfusepath{stroke}%
\end{pgfscope}%
\begin{pgfscope}%
\pgfpathrectangle{\pgfqpoint{0.100000in}{0.212622in}}{\pgfqpoint{3.696000in}{3.696000in}}%
\pgfusepath{clip}%
\pgfsetrectcap%
\pgfsetroundjoin%
\pgfsetlinewidth{1.505625pt}%
\definecolor{currentstroke}{rgb}{1.000000,0.000000,0.000000}%
\pgfsetstrokecolor{currentstroke}%
\pgfsetdash{}{0pt}%
\pgfpathmoveto{\pgfqpoint{1.138917in}{1.648134in}}%
\pgfpathlineto{\pgfqpoint{1.136968in}{1.648805in}}%
\pgfusepath{stroke}%
\end{pgfscope}%
\begin{pgfscope}%
\pgfpathrectangle{\pgfqpoint{0.100000in}{0.212622in}}{\pgfqpoint{3.696000in}{3.696000in}}%
\pgfusepath{clip}%
\pgfsetrectcap%
\pgfsetroundjoin%
\pgfsetlinewidth{1.505625pt}%
\definecolor{currentstroke}{rgb}{1.000000,0.000000,0.000000}%
\pgfsetstrokecolor{currentstroke}%
\pgfsetdash{}{0pt}%
\pgfpathmoveto{\pgfqpoint{1.138917in}{1.648134in}}%
\pgfpathlineto{\pgfqpoint{1.136968in}{1.648805in}}%
\pgfusepath{stroke}%
\end{pgfscope}%
\begin{pgfscope}%
\pgfpathrectangle{\pgfqpoint{0.100000in}{0.212622in}}{\pgfqpoint{3.696000in}{3.696000in}}%
\pgfusepath{clip}%
\pgfsetrectcap%
\pgfsetroundjoin%
\pgfsetlinewidth{1.505625pt}%
\definecolor{currentstroke}{rgb}{1.000000,0.000000,0.000000}%
\pgfsetstrokecolor{currentstroke}%
\pgfsetdash{}{0pt}%
\pgfpathmoveto{\pgfqpoint{1.138917in}{1.648134in}}%
\pgfpathlineto{\pgfqpoint{1.136968in}{1.648805in}}%
\pgfusepath{stroke}%
\end{pgfscope}%
\begin{pgfscope}%
\pgfpathrectangle{\pgfqpoint{0.100000in}{0.212622in}}{\pgfqpoint{3.696000in}{3.696000in}}%
\pgfusepath{clip}%
\pgfsetrectcap%
\pgfsetroundjoin%
\pgfsetlinewidth{1.505625pt}%
\definecolor{currentstroke}{rgb}{1.000000,0.000000,0.000000}%
\pgfsetstrokecolor{currentstroke}%
\pgfsetdash{}{0pt}%
\pgfpathmoveto{\pgfqpoint{1.138917in}{1.648134in}}%
\pgfpathlineto{\pgfqpoint{1.136968in}{1.648805in}}%
\pgfusepath{stroke}%
\end{pgfscope}%
\begin{pgfscope}%
\pgfpathrectangle{\pgfqpoint{0.100000in}{0.212622in}}{\pgfqpoint{3.696000in}{3.696000in}}%
\pgfusepath{clip}%
\pgfsetrectcap%
\pgfsetroundjoin%
\pgfsetlinewidth{1.505625pt}%
\definecolor{currentstroke}{rgb}{1.000000,0.000000,0.000000}%
\pgfsetstrokecolor{currentstroke}%
\pgfsetdash{}{0pt}%
\pgfpathmoveto{\pgfqpoint{1.138917in}{1.648134in}}%
\pgfpathlineto{\pgfqpoint{1.136968in}{1.648805in}}%
\pgfusepath{stroke}%
\end{pgfscope}%
\begin{pgfscope}%
\pgfpathrectangle{\pgfqpoint{0.100000in}{0.212622in}}{\pgfqpoint{3.696000in}{3.696000in}}%
\pgfusepath{clip}%
\pgfsetrectcap%
\pgfsetroundjoin%
\pgfsetlinewidth{1.505625pt}%
\definecolor{currentstroke}{rgb}{1.000000,0.000000,0.000000}%
\pgfsetstrokecolor{currentstroke}%
\pgfsetdash{}{0pt}%
\pgfpathmoveto{\pgfqpoint{1.138917in}{1.648134in}}%
\pgfpathlineto{\pgfqpoint{1.136968in}{1.648805in}}%
\pgfusepath{stroke}%
\end{pgfscope}%
\begin{pgfscope}%
\pgfpathrectangle{\pgfqpoint{0.100000in}{0.212622in}}{\pgfqpoint{3.696000in}{3.696000in}}%
\pgfusepath{clip}%
\pgfsetrectcap%
\pgfsetroundjoin%
\pgfsetlinewidth{1.505625pt}%
\definecolor{currentstroke}{rgb}{1.000000,0.000000,0.000000}%
\pgfsetstrokecolor{currentstroke}%
\pgfsetdash{}{0pt}%
\pgfpathmoveto{\pgfqpoint{1.138917in}{1.648134in}}%
\pgfpathlineto{\pgfqpoint{1.136968in}{1.648805in}}%
\pgfusepath{stroke}%
\end{pgfscope}%
\begin{pgfscope}%
\pgfpathrectangle{\pgfqpoint{0.100000in}{0.212622in}}{\pgfqpoint{3.696000in}{3.696000in}}%
\pgfusepath{clip}%
\pgfsetrectcap%
\pgfsetroundjoin%
\pgfsetlinewidth{1.505625pt}%
\definecolor{currentstroke}{rgb}{1.000000,0.000000,0.000000}%
\pgfsetstrokecolor{currentstroke}%
\pgfsetdash{}{0pt}%
\pgfpathmoveto{\pgfqpoint{1.138917in}{1.648134in}}%
\pgfpathlineto{\pgfqpoint{1.136968in}{1.648805in}}%
\pgfusepath{stroke}%
\end{pgfscope}%
\begin{pgfscope}%
\pgfpathrectangle{\pgfqpoint{0.100000in}{0.212622in}}{\pgfqpoint{3.696000in}{3.696000in}}%
\pgfusepath{clip}%
\pgfsetrectcap%
\pgfsetroundjoin%
\pgfsetlinewidth{1.505625pt}%
\definecolor{currentstroke}{rgb}{1.000000,0.000000,0.000000}%
\pgfsetstrokecolor{currentstroke}%
\pgfsetdash{}{0pt}%
\pgfpathmoveto{\pgfqpoint{1.138917in}{1.648134in}}%
\pgfpathlineto{\pgfqpoint{1.136968in}{1.648805in}}%
\pgfusepath{stroke}%
\end{pgfscope}%
\begin{pgfscope}%
\pgfpathrectangle{\pgfqpoint{0.100000in}{0.212622in}}{\pgfqpoint{3.696000in}{3.696000in}}%
\pgfusepath{clip}%
\pgfsetrectcap%
\pgfsetroundjoin%
\pgfsetlinewidth{1.505625pt}%
\definecolor{currentstroke}{rgb}{1.000000,0.000000,0.000000}%
\pgfsetstrokecolor{currentstroke}%
\pgfsetdash{}{0pt}%
\pgfpathmoveto{\pgfqpoint{1.138917in}{1.648134in}}%
\pgfpathlineto{\pgfqpoint{1.136968in}{1.648805in}}%
\pgfusepath{stroke}%
\end{pgfscope}%
\begin{pgfscope}%
\pgfpathrectangle{\pgfqpoint{0.100000in}{0.212622in}}{\pgfqpoint{3.696000in}{3.696000in}}%
\pgfusepath{clip}%
\pgfsetrectcap%
\pgfsetroundjoin%
\pgfsetlinewidth{1.505625pt}%
\definecolor{currentstroke}{rgb}{1.000000,0.000000,0.000000}%
\pgfsetstrokecolor{currentstroke}%
\pgfsetdash{}{0pt}%
\pgfpathmoveto{\pgfqpoint{1.138917in}{1.648134in}}%
\pgfpathlineto{\pgfqpoint{1.136968in}{1.648805in}}%
\pgfusepath{stroke}%
\end{pgfscope}%
\begin{pgfscope}%
\pgfpathrectangle{\pgfqpoint{0.100000in}{0.212622in}}{\pgfqpoint{3.696000in}{3.696000in}}%
\pgfusepath{clip}%
\pgfsetrectcap%
\pgfsetroundjoin%
\pgfsetlinewidth{1.505625pt}%
\definecolor{currentstroke}{rgb}{1.000000,0.000000,0.000000}%
\pgfsetstrokecolor{currentstroke}%
\pgfsetdash{}{0pt}%
\pgfpathmoveto{\pgfqpoint{1.138917in}{1.648134in}}%
\pgfpathlineto{\pgfqpoint{1.136968in}{1.648805in}}%
\pgfusepath{stroke}%
\end{pgfscope}%
\begin{pgfscope}%
\pgfpathrectangle{\pgfqpoint{0.100000in}{0.212622in}}{\pgfqpoint{3.696000in}{3.696000in}}%
\pgfusepath{clip}%
\pgfsetrectcap%
\pgfsetroundjoin%
\pgfsetlinewidth{1.505625pt}%
\definecolor{currentstroke}{rgb}{1.000000,0.000000,0.000000}%
\pgfsetstrokecolor{currentstroke}%
\pgfsetdash{}{0pt}%
\pgfpathmoveto{\pgfqpoint{1.138917in}{1.648134in}}%
\pgfpathlineto{\pgfqpoint{1.136968in}{1.648805in}}%
\pgfusepath{stroke}%
\end{pgfscope}%
\begin{pgfscope}%
\pgfpathrectangle{\pgfqpoint{0.100000in}{0.212622in}}{\pgfqpoint{3.696000in}{3.696000in}}%
\pgfusepath{clip}%
\pgfsetrectcap%
\pgfsetroundjoin%
\pgfsetlinewidth{1.505625pt}%
\definecolor{currentstroke}{rgb}{1.000000,0.000000,0.000000}%
\pgfsetstrokecolor{currentstroke}%
\pgfsetdash{}{0pt}%
\pgfpathmoveto{\pgfqpoint{1.138917in}{1.648134in}}%
\pgfpathlineto{\pgfqpoint{1.136968in}{1.648805in}}%
\pgfusepath{stroke}%
\end{pgfscope}%
\begin{pgfscope}%
\pgfpathrectangle{\pgfqpoint{0.100000in}{0.212622in}}{\pgfqpoint{3.696000in}{3.696000in}}%
\pgfusepath{clip}%
\pgfsetrectcap%
\pgfsetroundjoin%
\pgfsetlinewidth{1.505625pt}%
\definecolor{currentstroke}{rgb}{1.000000,0.000000,0.000000}%
\pgfsetstrokecolor{currentstroke}%
\pgfsetdash{}{0pt}%
\pgfpathmoveto{\pgfqpoint{1.138917in}{1.648134in}}%
\pgfpathlineto{\pgfqpoint{1.136968in}{1.648805in}}%
\pgfusepath{stroke}%
\end{pgfscope}%
\begin{pgfscope}%
\pgfpathrectangle{\pgfqpoint{0.100000in}{0.212622in}}{\pgfqpoint{3.696000in}{3.696000in}}%
\pgfusepath{clip}%
\pgfsetrectcap%
\pgfsetroundjoin%
\pgfsetlinewidth{1.505625pt}%
\definecolor{currentstroke}{rgb}{1.000000,0.000000,0.000000}%
\pgfsetstrokecolor{currentstroke}%
\pgfsetdash{}{0pt}%
\pgfpathmoveto{\pgfqpoint{1.138917in}{1.648134in}}%
\pgfpathlineto{\pgfqpoint{1.136968in}{1.648805in}}%
\pgfusepath{stroke}%
\end{pgfscope}%
\begin{pgfscope}%
\pgfpathrectangle{\pgfqpoint{0.100000in}{0.212622in}}{\pgfqpoint{3.696000in}{3.696000in}}%
\pgfusepath{clip}%
\pgfsetrectcap%
\pgfsetroundjoin%
\pgfsetlinewidth{1.505625pt}%
\definecolor{currentstroke}{rgb}{1.000000,0.000000,0.000000}%
\pgfsetstrokecolor{currentstroke}%
\pgfsetdash{}{0pt}%
\pgfpathmoveto{\pgfqpoint{1.138917in}{1.648134in}}%
\pgfpathlineto{\pgfqpoint{1.136968in}{1.648805in}}%
\pgfusepath{stroke}%
\end{pgfscope}%
\begin{pgfscope}%
\pgfpathrectangle{\pgfqpoint{0.100000in}{0.212622in}}{\pgfqpoint{3.696000in}{3.696000in}}%
\pgfusepath{clip}%
\pgfsetrectcap%
\pgfsetroundjoin%
\pgfsetlinewidth{1.505625pt}%
\definecolor{currentstroke}{rgb}{1.000000,0.000000,0.000000}%
\pgfsetstrokecolor{currentstroke}%
\pgfsetdash{}{0pt}%
\pgfpathmoveto{\pgfqpoint{1.138917in}{1.648134in}}%
\pgfpathlineto{\pgfqpoint{1.136968in}{1.648805in}}%
\pgfusepath{stroke}%
\end{pgfscope}%
\begin{pgfscope}%
\pgfpathrectangle{\pgfqpoint{0.100000in}{0.212622in}}{\pgfqpoint{3.696000in}{3.696000in}}%
\pgfusepath{clip}%
\pgfsetrectcap%
\pgfsetroundjoin%
\pgfsetlinewidth{1.505625pt}%
\definecolor{currentstroke}{rgb}{1.000000,0.000000,0.000000}%
\pgfsetstrokecolor{currentstroke}%
\pgfsetdash{}{0pt}%
\pgfpathmoveto{\pgfqpoint{1.138917in}{1.648134in}}%
\pgfpathlineto{\pgfqpoint{1.136968in}{1.648805in}}%
\pgfusepath{stroke}%
\end{pgfscope}%
\begin{pgfscope}%
\pgfpathrectangle{\pgfqpoint{0.100000in}{0.212622in}}{\pgfqpoint{3.696000in}{3.696000in}}%
\pgfusepath{clip}%
\pgfsetrectcap%
\pgfsetroundjoin%
\pgfsetlinewidth{1.505625pt}%
\definecolor{currentstroke}{rgb}{1.000000,0.000000,0.000000}%
\pgfsetstrokecolor{currentstroke}%
\pgfsetdash{}{0pt}%
\pgfpathmoveto{\pgfqpoint{1.138917in}{1.648134in}}%
\pgfpathlineto{\pgfqpoint{1.136968in}{1.648805in}}%
\pgfusepath{stroke}%
\end{pgfscope}%
\begin{pgfscope}%
\pgfpathrectangle{\pgfqpoint{0.100000in}{0.212622in}}{\pgfqpoint{3.696000in}{3.696000in}}%
\pgfusepath{clip}%
\pgfsetrectcap%
\pgfsetroundjoin%
\pgfsetlinewidth{1.505625pt}%
\definecolor{currentstroke}{rgb}{1.000000,0.000000,0.000000}%
\pgfsetstrokecolor{currentstroke}%
\pgfsetdash{}{0pt}%
\pgfpathmoveto{\pgfqpoint{1.138917in}{1.648134in}}%
\pgfpathlineto{\pgfqpoint{1.136968in}{1.648805in}}%
\pgfusepath{stroke}%
\end{pgfscope}%
\begin{pgfscope}%
\pgfpathrectangle{\pgfqpoint{0.100000in}{0.212622in}}{\pgfqpoint{3.696000in}{3.696000in}}%
\pgfusepath{clip}%
\pgfsetrectcap%
\pgfsetroundjoin%
\pgfsetlinewidth{1.505625pt}%
\definecolor{currentstroke}{rgb}{1.000000,0.000000,0.000000}%
\pgfsetstrokecolor{currentstroke}%
\pgfsetdash{}{0pt}%
\pgfpathmoveto{\pgfqpoint{1.138917in}{1.648134in}}%
\pgfpathlineto{\pgfqpoint{1.136968in}{1.648805in}}%
\pgfusepath{stroke}%
\end{pgfscope}%
\begin{pgfscope}%
\pgfpathrectangle{\pgfqpoint{0.100000in}{0.212622in}}{\pgfqpoint{3.696000in}{3.696000in}}%
\pgfusepath{clip}%
\pgfsetrectcap%
\pgfsetroundjoin%
\pgfsetlinewidth{1.505625pt}%
\definecolor{currentstroke}{rgb}{1.000000,0.000000,0.000000}%
\pgfsetstrokecolor{currentstroke}%
\pgfsetdash{}{0pt}%
\pgfpathmoveto{\pgfqpoint{1.138917in}{1.648134in}}%
\pgfpathlineto{\pgfqpoint{1.136968in}{1.648805in}}%
\pgfusepath{stroke}%
\end{pgfscope}%
\begin{pgfscope}%
\pgfpathrectangle{\pgfqpoint{0.100000in}{0.212622in}}{\pgfqpoint{3.696000in}{3.696000in}}%
\pgfusepath{clip}%
\pgfsetrectcap%
\pgfsetroundjoin%
\pgfsetlinewidth{1.505625pt}%
\definecolor{currentstroke}{rgb}{1.000000,0.000000,0.000000}%
\pgfsetstrokecolor{currentstroke}%
\pgfsetdash{}{0pt}%
\pgfpathmoveto{\pgfqpoint{1.138917in}{1.648134in}}%
\pgfpathlineto{\pgfqpoint{1.136968in}{1.648805in}}%
\pgfusepath{stroke}%
\end{pgfscope}%
\begin{pgfscope}%
\pgfpathrectangle{\pgfqpoint{0.100000in}{0.212622in}}{\pgfqpoint{3.696000in}{3.696000in}}%
\pgfusepath{clip}%
\pgfsetrectcap%
\pgfsetroundjoin%
\pgfsetlinewidth{1.505625pt}%
\definecolor{currentstroke}{rgb}{1.000000,0.000000,0.000000}%
\pgfsetstrokecolor{currentstroke}%
\pgfsetdash{}{0pt}%
\pgfpathmoveto{\pgfqpoint{1.138917in}{1.648134in}}%
\pgfpathlineto{\pgfqpoint{1.136968in}{1.648805in}}%
\pgfusepath{stroke}%
\end{pgfscope}%
\begin{pgfscope}%
\pgfpathrectangle{\pgfqpoint{0.100000in}{0.212622in}}{\pgfqpoint{3.696000in}{3.696000in}}%
\pgfusepath{clip}%
\pgfsetrectcap%
\pgfsetroundjoin%
\pgfsetlinewidth{1.505625pt}%
\definecolor{currentstroke}{rgb}{1.000000,0.000000,0.000000}%
\pgfsetstrokecolor{currentstroke}%
\pgfsetdash{}{0pt}%
\pgfpathmoveto{\pgfqpoint{1.138917in}{1.648134in}}%
\pgfpathlineto{\pgfqpoint{1.136968in}{1.648805in}}%
\pgfusepath{stroke}%
\end{pgfscope}%
\begin{pgfscope}%
\pgfpathrectangle{\pgfqpoint{0.100000in}{0.212622in}}{\pgfqpoint{3.696000in}{3.696000in}}%
\pgfusepath{clip}%
\pgfsetrectcap%
\pgfsetroundjoin%
\pgfsetlinewidth{1.505625pt}%
\definecolor{currentstroke}{rgb}{1.000000,0.000000,0.000000}%
\pgfsetstrokecolor{currentstroke}%
\pgfsetdash{}{0pt}%
\pgfpathmoveto{\pgfqpoint{1.138917in}{1.648134in}}%
\pgfpathlineto{\pgfqpoint{1.136968in}{1.648805in}}%
\pgfusepath{stroke}%
\end{pgfscope}%
\begin{pgfscope}%
\pgfpathrectangle{\pgfqpoint{0.100000in}{0.212622in}}{\pgfqpoint{3.696000in}{3.696000in}}%
\pgfusepath{clip}%
\pgfsetrectcap%
\pgfsetroundjoin%
\pgfsetlinewidth{1.505625pt}%
\definecolor{currentstroke}{rgb}{1.000000,0.000000,0.000000}%
\pgfsetstrokecolor{currentstroke}%
\pgfsetdash{}{0pt}%
\pgfpathmoveto{\pgfqpoint{1.138917in}{1.648134in}}%
\pgfpathlineto{\pgfqpoint{1.136968in}{1.648805in}}%
\pgfusepath{stroke}%
\end{pgfscope}%
\begin{pgfscope}%
\pgfpathrectangle{\pgfqpoint{0.100000in}{0.212622in}}{\pgfqpoint{3.696000in}{3.696000in}}%
\pgfusepath{clip}%
\pgfsetrectcap%
\pgfsetroundjoin%
\pgfsetlinewidth{1.505625pt}%
\definecolor{currentstroke}{rgb}{1.000000,0.000000,0.000000}%
\pgfsetstrokecolor{currentstroke}%
\pgfsetdash{}{0pt}%
\pgfpathmoveto{\pgfqpoint{1.138917in}{1.648134in}}%
\pgfpathlineto{\pgfqpoint{1.136968in}{1.648805in}}%
\pgfusepath{stroke}%
\end{pgfscope}%
\begin{pgfscope}%
\pgfpathrectangle{\pgfqpoint{0.100000in}{0.212622in}}{\pgfqpoint{3.696000in}{3.696000in}}%
\pgfusepath{clip}%
\pgfsetrectcap%
\pgfsetroundjoin%
\pgfsetlinewidth{1.505625pt}%
\definecolor{currentstroke}{rgb}{1.000000,0.000000,0.000000}%
\pgfsetstrokecolor{currentstroke}%
\pgfsetdash{}{0pt}%
\pgfpathmoveto{\pgfqpoint{1.138917in}{1.648134in}}%
\pgfpathlineto{\pgfqpoint{1.136968in}{1.648805in}}%
\pgfusepath{stroke}%
\end{pgfscope}%
\begin{pgfscope}%
\pgfpathrectangle{\pgfqpoint{0.100000in}{0.212622in}}{\pgfqpoint{3.696000in}{3.696000in}}%
\pgfusepath{clip}%
\pgfsetrectcap%
\pgfsetroundjoin%
\pgfsetlinewidth{1.505625pt}%
\definecolor{currentstroke}{rgb}{1.000000,0.000000,0.000000}%
\pgfsetstrokecolor{currentstroke}%
\pgfsetdash{}{0pt}%
\pgfpathmoveto{\pgfqpoint{1.138917in}{1.648134in}}%
\pgfpathlineto{\pgfqpoint{1.136968in}{1.648805in}}%
\pgfusepath{stroke}%
\end{pgfscope}%
\begin{pgfscope}%
\pgfpathrectangle{\pgfqpoint{0.100000in}{0.212622in}}{\pgfqpoint{3.696000in}{3.696000in}}%
\pgfusepath{clip}%
\pgfsetrectcap%
\pgfsetroundjoin%
\pgfsetlinewidth{1.505625pt}%
\definecolor{currentstroke}{rgb}{1.000000,0.000000,0.000000}%
\pgfsetstrokecolor{currentstroke}%
\pgfsetdash{}{0pt}%
\pgfpathmoveto{\pgfqpoint{1.139756in}{1.647903in}}%
\pgfpathlineto{\pgfqpoint{1.136968in}{1.648805in}}%
\pgfusepath{stroke}%
\end{pgfscope}%
\begin{pgfscope}%
\pgfpathrectangle{\pgfqpoint{0.100000in}{0.212622in}}{\pgfqpoint{3.696000in}{3.696000in}}%
\pgfusepath{clip}%
\pgfsetrectcap%
\pgfsetroundjoin%
\pgfsetlinewidth{1.505625pt}%
\definecolor{currentstroke}{rgb}{1.000000,0.000000,0.000000}%
\pgfsetstrokecolor{currentstroke}%
\pgfsetdash{}{0pt}%
\pgfpathmoveto{\pgfqpoint{1.140196in}{1.647758in}}%
\pgfpathlineto{\pgfqpoint{1.136968in}{1.648805in}}%
\pgfusepath{stroke}%
\end{pgfscope}%
\begin{pgfscope}%
\pgfpathrectangle{\pgfqpoint{0.100000in}{0.212622in}}{\pgfqpoint{3.696000in}{3.696000in}}%
\pgfusepath{clip}%
\pgfsetrectcap%
\pgfsetroundjoin%
\pgfsetlinewidth{1.505625pt}%
\definecolor{currentstroke}{rgb}{1.000000,0.000000,0.000000}%
\pgfsetstrokecolor{currentstroke}%
\pgfsetdash{}{0pt}%
\pgfpathmoveto{\pgfqpoint{1.140440in}{1.647681in}}%
\pgfpathlineto{\pgfqpoint{1.136968in}{1.648805in}}%
\pgfusepath{stroke}%
\end{pgfscope}%
\begin{pgfscope}%
\pgfpathrectangle{\pgfqpoint{0.100000in}{0.212622in}}{\pgfqpoint{3.696000in}{3.696000in}}%
\pgfusepath{clip}%
\pgfsetrectcap%
\pgfsetroundjoin%
\pgfsetlinewidth{1.505625pt}%
\definecolor{currentstroke}{rgb}{1.000000,0.000000,0.000000}%
\pgfsetstrokecolor{currentstroke}%
\pgfsetdash{}{0pt}%
\pgfpathmoveto{\pgfqpoint{1.141298in}{1.647402in}}%
\pgfpathlineto{\pgfqpoint{1.136968in}{1.648805in}}%
\pgfusepath{stroke}%
\end{pgfscope}%
\begin{pgfscope}%
\pgfpathrectangle{\pgfqpoint{0.100000in}{0.212622in}}{\pgfqpoint{3.696000in}{3.696000in}}%
\pgfusepath{clip}%
\pgfsetrectcap%
\pgfsetroundjoin%
\pgfsetlinewidth{1.505625pt}%
\definecolor{currentstroke}{rgb}{1.000000,0.000000,0.000000}%
\pgfsetstrokecolor{currentstroke}%
\pgfsetdash{}{0pt}%
\pgfpathmoveto{\pgfqpoint{1.141693in}{1.647161in}}%
\pgfpathlineto{\pgfqpoint{1.136968in}{1.648805in}}%
\pgfusepath{stroke}%
\end{pgfscope}%
\begin{pgfscope}%
\pgfpathrectangle{\pgfqpoint{0.100000in}{0.212622in}}{\pgfqpoint{3.696000in}{3.696000in}}%
\pgfusepath{clip}%
\pgfsetrectcap%
\pgfsetroundjoin%
\pgfsetlinewidth{1.505625pt}%
\definecolor{currentstroke}{rgb}{1.000000,0.000000,0.000000}%
\pgfsetstrokecolor{currentstroke}%
\pgfsetdash{}{0pt}%
\pgfpathmoveto{\pgfqpoint{1.142590in}{1.646812in}}%
\pgfpathlineto{\pgfqpoint{1.136968in}{1.648805in}}%
\pgfusepath{stroke}%
\end{pgfscope}%
\begin{pgfscope}%
\pgfpathrectangle{\pgfqpoint{0.100000in}{0.212622in}}{\pgfqpoint{3.696000in}{3.696000in}}%
\pgfusepath{clip}%
\pgfsetrectcap%
\pgfsetroundjoin%
\pgfsetlinewidth{1.505625pt}%
\definecolor{currentstroke}{rgb}{1.000000,0.000000,0.000000}%
\pgfsetstrokecolor{currentstroke}%
\pgfsetdash{}{0pt}%
\pgfpathmoveto{\pgfqpoint{1.143890in}{1.646182in}}%
\pgfpathlineto{\pgfqpoint{1.150376in}{1.644691in}}%
\pgfusepath{stroke}%
\end{pgfscope}%
\begin{pgfscope}%
\pgfpathrectangle{\pgfqpoint{0.100000in}{0.212622in}}{\pgfqpoint{3.696000in}{3.696000in}}%
\pgfusepath{clip}%
\pgfsetrectcap%
\pgfsetroundjoin%
\pgfsetlinewidth{1.505625pt}%
\definecolor{currentstroke}{rgb}{1.000000,0.000000,0.000000}%
\pgfsetstrokecolor{currentstroke}%
\pgfsetdash{}{0pt}%
\pgfpathmoveto{\pgfqpoint{1.144634in}{1.645862in}}%
\pgfpathlineto{\pgfqpoint{1.150376in}{1.644691in}}%
\pgfusepath{stroke}%
\end{pgfscope}%
\begin{pgfscope}%
\pgfpathrectangle{\pgfqpoint{0.100000in}{0.212622in}}{\pgfqpoint{3.696000in}{3.696000in}}%
\pgfusepath{clip}%
\pgfsetrectcap%
\pgfsetroundjoin%
\pgfsetlinewidth{1.505625pt}%
\definecolor{currentstroke}{rgb}{1.000000,0.000000,0.000000}%
\pgfsetstrokecolor{currentstroke}%
\pgfsetdash{}{0pt}%
\pgfpathmoveto{\pgfqpoint{1.146067in}{1.645347in}}%
\pgfpathlineto{\pgfqpoint{1.150376in}{1.644691in}}%
\pgfusepath{stroke}%
\end{pgfscope}%
\begin{pgfscope}%
\pgfpathrectangle{\pgfqpoint{0.100000in}{0.212622in}}{\pgfqpoint{3.696000in}{3.696000in}}%
\pgfusepath{clip}%
\pgfsetrectcap%
\pgfsetroundjoin%
\pgfsetlinewidth{1.505625pt}%
\definecolor{currentstroke}{rgb}{1.000000,0.000000,0.000000}%
\pgfsetstrokecolor{currentstroke}%
\pgfsetdash{}{0pt}%
\pgfpathmoveto{\pgfqpoint{1.147806in}{1.644465in}}%
\pgfpathlineto{\pgfqpoint{1.150376in}{1.644691in}}%
\pgfusepath{stroke}%
\end{pgfscope}%
\begin{pgfscope}%
\pgfpathrectangle{\pgfqpoint{0.100000in}{0.212622in}}{\pgfqpoint{3.696000in}{3.696000in}}%
\pgfusepath{clip}%
\pgfsetrectcap%
\pgfsetroundjoin%
\pgfsetlinewidth{1.505625pt}%
\definecolor{currentstroke}{rgb}{1.000000,0.000000,0.000000}%
\pgfsetstrokecolor{currentstroke}%
\pgfsetdash{}{0pt}%
\pgfpathmoveto{\pgfqpoint{1.148800in}{1.644036in}}%
\pgfpathlineto{\pgfqpoint{1.150376in}{1.644691in}}%
\pgfusepath{stroke}%
\end{pgfscope}%
\begin{pgfscope}%
\pgfpathrectangle{\pgfqpoint{0.100000in}{0.212622in}}{\pgfqpoint{3.696000in}{3.696000in}}%
\pgfusepath{clip}%
\pgfsetrectcap%
\pgfsetroundjoin%
\pgfsetlinewidth{1.505625pt}%
\definecolor{currentstroke}{rgb}{1.000000,0.000000,0.000000}%
\pgfsetstrokecolor{currentstroke}%
\pgfsetdash{}{0pt}%
\pgfpathmoveto{\pgfqpoint{1.150437in}{1.643341in}}%
\pgfpathlineto{\pgfqpoint{1.150376in}{1.644691in}}%
\pgfusepath{stroke}%
\end{pgfscope}%
\begin{pgfscope}%
\pgfpathrectangle{\pgfqpoint{0.100000in}{0.212622in}}{\pgfqpoint{3.696000in}{3.696000in}}%
\pgfusepath{clip}%
\pgfsetrectcap%
\pgfsetroundjoin%
\pgfsetlinewidth{1.505625pt}%
\definecolor{currentstroke}{rgb}{1.000000,0.000000,0.000000}%
\pgfsetstrokecolor{currentstroke}%
\pgfsetdash{}{0pt}%
\pgfpathmoveto{\pgfqpoint{1.152695in}{1.642169in}}%
\pgfpathlineto{\pgfqpoint{1.150376in}{1.644691in}}%
\pgfusepath{stroke}%
\end{pgfscope}%
\begin{pgfscope}%
\pgfpathrectangle{\pgfqpoint{0.100000in}{0.212622in}}{\pgfqpoint{3.696000in}{3.696000in}}%
\pgfusepath{clip}%
\pgfsetrectcap%
\pgfsetroundjoin%
\pgfsetlinewidth{1.505625pt}%
\definecolor{currentstroke}{rgb}{1.000000,0.000000,0.000000}%
\pgfsetstrokecolor{currentstroke}%
\pgfsetdash{}{0pt}%
\pgfpathmoveto{\pgfqpoint{1.154015in}{1.641641in}}%
\pgfpathlineto{\pgfqpoint{1.150376in}{1.644691in}}%
\pgfusepath{stroke}%
\end{pgfscope}%
\begin{pgfscope}%
\pgfpathrectangle{\pgfqpoint{0.100000in}{0.212622in}}{\pgfqpoint{3.696000in}{3.696000in}}%
\pgfusepath{clip}%
\pgfsetrectcap%
\pgfsetroundjoin%
\pgfsetlinewidth{1.505625pt}%
\definecolor{currentstroke}{rgb}{1.000000,0.000000,0.000000}%
\pgfsetstrokecolor{currentstroke}%
\pgfsetdash{}{0pt}%
\pgfpathmoveto{\pgfqpoint{1.155592in}{1.640821in}}%
\pgfpathlineto{\pgfqpoint{1.163793in}{1.640574in}}%
\pgfusepath{stroke}%
\end{pgfscope}%
\begin{pgfscope}%
\pgfpathrectangle{\pgfqpoint{0.100000in}{0.212622in}}{\pgfqpoint{3.696000in}{3.696000in}}%
\pgfusepath{clip}%
\pgfsetrectcap%
\pgfsetroundjoin%
\pgfsetlinewidth{1.505625pt}%
\definecolor{currentstroke}{rgb}{1.000000,0.000000,0.000000}%
\pgfsetstrokecolor{currentstroke}%
\pgfsetdash{}{0pt}%
\pgfpathmoveto{\pgfqpoint{1.157810in}{1.639756in}}%
\pgfpathlineto{\pgfqpoint{1.163793in}{1.640574in}}%
\pgfusepath{stroke}%
\end{pgfscope}%
\begin{pgfscope}%
\pgfpathrectangle{\pgfqpoint{0.100000in}{0.212622in}}{\pgfqpoint{3.696000in}{3.696000in}}%
\pgfusepath{clip}%
\pgfsetrectcap%
\pgfsetroundjoin%
\pgfsetlinewidth{1.505625pt}%
\definecolor{currentstroke}{rgb}{1.000000,0.000000,0.000000}%
\pgfsetstrokecolor{currentstroke}%
\pgfsetdash{}{0pt}%
\pgfpathmoveto{\pgfqpoint{1.158889in}{1.639052in}}%
\pgfpathlineto{\pgfqpoint{1.163793in}{1.640574in}}%
\pgfusepath{stroke}%
\end{pgfscope}%
\begin{pgfscope}%
\pgfpathrectangle{\pgfqpoint{0.100000in}{0.212622in}}{\pgfqpoint{3.696000in}{3.696000in}}%
\pgfusepath{clip}%
\pgfsetrectcap%
\pgfsetroundjoin%
\pgfsetlinewidth{1.505625pt}%
\definecolor{currentstroke}{rgb}{1.000000,0.000000,0.000000}%
\pgfsetstrokecolor{currentstroke}%
\pgfsetdash{}{0pt}%
\pgfpathmoveto{\pgfqpoint{1.162180in}{1.637574in}}%
\pgfpathlineto{\pgfqpoint{1.163793in}{1.640574in}}%
\pgfusepath{stroke}%
\end{pgfscope}%
\begin{pgfscope}%
\pgfpathrectangle{\pgfqpoint{0.100000in}{0.212622in}}{\pgfqpoint{3.696000in}{3.696000in}}%
\pgfusepath{clip}%
\pgfsetrectcap%
\pgfsetroundjoin%
\pgfsetlinewidth{1.505625pt}%
\definecolor{currentstroke}{rgb}{1.000000,0.000000,0.000000}%
\pgfsetstrokecolor{currentstroke}%
\pgfsetdash{}{0pt}%
\pgfpathmoveto{\pgfqpoint{1.166068in}{1.635699in}}%
\pgfpathlineto{\pgfqpoint{1.163793in}{1.640574in}}%
\pgfusepath{stroke}%
\end{pgfscope}%
\begin{pgfscope}%
\pgfpathrectangle{\pgfqpoint{0.100000in}{0.212622in}}{\pgfqpoint{3.696000in}{3.696000in}}%
\pgfusepath{clip}%
\pgfsetrectcap%
\pgfsetroundjoin%
\pgfsetlinewidth{1.505625pt}%
\definecolor{currentstroke}{rgb}{1.000000,0.000000,0.000000}%
\pgfsetstrokecolor{currentstroke}%
\pgfsetdash{}{0pt}%
\pgfpathmoveto{\pgfqpoint{1.168215in}{1.634730in}}%
\pgfpathlineto{\pgfqpoint{1.177219in}{1.636454in}}%
\pgfusepath{stroke}%
\end{pgfscope}%
\begin{pgfscope}%
\pgfpathrectangle{\pgfqpoint{0.100000in}{0.212622in}}{\pgfqpoint{3.696000in}{3.696000in}}%
\pgfusepath{clip}%
\pgfsetrectcap%
\pgfsetroundjoin%
\pgfsetlinewidth{1.505625pt}%
\definecolor{currentstroke}{rgb}{1.000000,0.000000,0.000000}%
\pgfsetstrokecolor{currentstroke}%
\pgfsetdash{}{0pt}%
\pgfpathmoveto{\pgfqpoint{1.169489in}{1.634278in}}%
\pgfpathlineto{\pgfqpoint{1.177219in}{1.636454in}}%
\pgfusepath{stroke}%
\end{pgfscope}%
\begin{pgfscope}%
\pgfpathrectangle{\pgfqpoint{0.100000in}{0.212622in}}{\pgfqpoint{3.696000in}{3.696000in}}%
\pgfusepath{clip}%
\pgfsetrectcap%
\pgfsetroundjoin%
\pgfsetlinewidth{1.505625pt}%
\definecolor{currentstroke}{rgb}{1.000000,0.000000,0.000000}%
\pgfsetstrokecolor{currentstroke}%
\pgfsetdash{}{0pt}%
\pgfpathmoveto{\pgfqpoint{1.171168in}{1.633540in}}%
\pgfpathlineto{\pgfqpoint{1.177219in}{1.636454in}}%
\pgfusepath{stroke}%
\end{pgfscope}%
\begin{pgfscope}%
\pgfpathrectangle{\pgfqpoint{0.100000in}{0.212622in}}{\pgfqpoint{3.696000in}{3.696000in}}%
\pgfusepath{clip}%
\pgfsetrectcap%
\pgfsetroundjoin%
\pgfsetlinewidth{1.505625pt}%
\definecolor{currentstroke}{rgb}{1.000000,0.000000,0.000000}%
\pgfsetstrokecolor{currentstroke}%
\pgfsetdash{}{0pt}%
\pgfpathmoveto{\pgfqpoint{1.172083in}{1.633141in}}%
\pgfpathlineto{\pgfqpoint{1.177219in}{1.636454in}}%
\pgfusepath{stroke}%
\end{pgfscope}%
\begin{pgfscope}%
\pgfpathrectangle{\pgfqpoint{0.100000in}{0.212622in}}{\pgfqpoint{3.696000in}{3.696000in}}%
\pgfusepath{clip}%
\pgfsetrectcap%
\pgfsetroundjoin%
\pgfsetlinewidth{1.505625pt}%
\definecolor{currentstroke}{rgb}{1.000000,0.000000,0.000000}%
\pgfsetstrokecolor{currentstroke}%
\pgfsetdash{}{0pt}%
\pgfpathmoveto{\pgfqpoint{1.172486in}{1.632819in}}%
\pgfpathlineto{\pgfqpoint{1.177219in}{1.636454in}}%
\pgfusepath{stroke}%
\end{pgfscope}%
\begin{pgfscope}%
\pgfpathrectangle{\pgfqpoint{0.100000in}{0.212622in}}{\pgfqpoint{3.696000in}{3.696000in}}%
\pgfusepath{clip}%
\pgfsetrectcap%
\pgfsetroundjoin%
\pgfsetlinewidth{1.505625pt}%
\definecolor{currentstroke}{rgb}{1.000000,0.000000,0.000000}%
\pgfsetstrokecolor{currentstroke}%
\pgfsetdash{}{0pt}%
\pgfpathmoveto{\pgfqpoint{1.172751in}{1.632689in}}%
\pgfpathlineto{\pgfqpoint{1.177219in}{1.636454in}}%
\pgfusepath{stroke}%
\end{pgfscope}%
\begin{pgfscope}%
\pgfpathrectangle{\pgfqpoint{0.100000in}{0.212622in}}{\pgfqpoint{3.696000in}{3.696000in}}%
\pgfusepath{clip}%
\pgfsetrectcap%
\pgfsetroundjoin%
\pgfsetlinewidth{1.505625pt}%
\definecolor{currentstroke}{rgb}{1.000000,0.000000,0.000000}%
\pgfsetstrokecolor{currentstroke}%
\pgfsetdash{}{0pt}%
\pgfpathmoveto{\pgfqpoint{1.173601in}{1.632336in}}%
\pgfpathlineto{\pgfqpoint{1.177219in}{1.636454in}}%
\pgfusepath{stroke}%
\end{pgfscope}%
\begin{pgfscope}%
\pgfpathrectangle{\pgfqpoint{0.100000in}{0.212622in}}{\pgfqpoint{3.696000in}{3.696000in}}%
\pgfusepath{clip}%
\pgfsetrectcap%
\pgfsetroundjoin%
\pgfsetlinewidth{1.505625pt}%
\definecolor{currentstroke}{rgb}{1.000000,0.000000,0.000000}%
\pgfsetstrokecolor{currentstroke}%
\pgfsetdash{}{0pt}%
\pgfpathmoveto{\pgfqpoint{1.174021in}{1.632093in}}%
\pgfpathlineto{\pgfqpoint{1.177219in}{1.636454in}}%
\pgfusepath{stroke}%
\end{pgfscope}%
\begin{pgfscope}%
\pgfpathrectangle{\pgfqpoint{0.100000in}{0.212622in}}{\pgfqpoint{3.696000in}{3.696000in}}%
\pgfusepath{clip}%
\pgfsetrectcap%
\pgfsetroundjoin%
\pgfsetlinewidth{1.505625pt}%
\definecolor{currentstroke}{rgb}{1.000000,0.000000,0.000000}%
\pgfsetstrokecolor{currentstroke}%
\pgfsetdash{}{0pt}%
\pgfpathmoveto{\pgfqpoint{1.174269in}{1.631973in}}%
\pgfpathlineto{\pgfqpoint{1.177219in}{1.636454in}}%
\pgfusepath{stroke}%
\end{pgfscope}%
\begin{pgfscope}%
\pgfpathrectangle{\pgfqpoint{0.100000in}{0.212622in}}{\pgfqpoint{3.696000in}{3.696000in}}%
\pgfusepath{clip}%
\pgfsetrectcap%
\pgfsetroundjoin%
\pgfsetlinewidth{1.505625pt}%
\definecolor{currentstroke}{rgb}{1.000000,0.000000,0.000000}%
\pgfsetstrokecolor{currentstroke}%
\pgfsetdash{}{0pt}%
\pgfpathmoveto{\pgfqpoint{1.175099in}{1.631644in}}%
\pgfpathlineto{\pgfqpoint{1.177219in}{1.636454in}}%
\pgfusepath{stroke}%
\end{pgfscope}%
\begin{pgfscope}%
\pgfpathrectangle{\pgfqpoint{0.100000in}{0.212622in}}{\pgfqpoint{3.696000in}{3.696000in}}%
\pgfusepath{clip}%
\pgfsetrectcap%
\pgfsetroundjoin%
\pgfsetlinewidth{1.505625pt}%
\definecolor{currentstroke}{rgb}{1.000000,0.000000,0.000000}%
\pgfsetstrokecolor{currentstroke}%
\pgfsetdash{}{0pt}%
\pgfpathmoveto{\pgfqpoint{1.175553in}{1.631472in}}%
\pgfpathlineto{\pgfqpoint{1.177219in}{1.636454in}}%
\pgfusepath{stroke}%
\end{pgfscope}%
\begin{pgfscope}%
\pgfpathrectangle{\pgfqpoint{0.100000in}{0.212622in}}{\pgfqpoint{3.696000in}{3.696000in}}%
\pgfusepath{clip}%
\pgfsetrectcap%
\pgfsetroundjoin%
\pgfsetlinewidth{1.505625pt}%
\definecolor{currentstroke}{rgb}{1.000000,0.000000,0.000000}%
\pgfsetstrokecolor{currentstroke}%
\pgfsetdash{}{0pt}%
\pgfpathmoveto{\pgfqpoint{1.175776in}{1.631345in}}%
\pgfpathlineto{\pgfqpoint{1.177219in}{1.636454in}}%
\pgfusepath{stroke}%
\end{pgfscope}%
\begin{pgfscope}%
\pgfpathrectangle{\pgfqpoint{0.100000in}{0.212622in}}{\pgfqpoint{3.696000in}{3.696000in}}%
\pgfusepath{clip}%
\pgfsetrectcap%
\pgfsetroundjoin%
\pgfsetlinewidth{1.505625pt}%
\definecolor{currentstroke}{rgb}{1.000000,0.000000,0.000000}%
\pgfsetstrokecolor{currentstroke}%
\pgfsetdash{}{0pt}%
\pgfpathmoveto{\pgfqpoint{1.175918in}{1.631296in}}%
\pgfpathlineto{\pgfqpoint{1.177219in}{1.636454in}}%
\pgfusepath{stroke}%
\end{pgfscope}%
\begin{pgfscope}%
\pgfpathrectangle{\pgfqpoint{0.100000in}{0.212622in}}{\pgfqpoint{3.696000in}{3.696000in}}%
\pgfusepath{clip}%
\pgfsetrectcap%
\pgfsetroundjoin%
\pgfsetlinewidth{1.505625pt}%
\definecolor{currentstroke}{rgb}{1.000000,0.000000,0.000000}%
\pgfsetstrokecolor{currentstroke}%
\pgfsetdash{}{0pt}%
\pgfpathmoveto{\pgfqpoint{1.178043in}{1.630568in}}%
\pgfpathlineto{\pgfqpoint{1.177219in}{1.636454in}}%
\pgfusepath{stroke}%
\end{pgfscope}%
\begin{pgfscope}%
\pgfpathrectangle{\pgfqpoint{0.100000in}{0.212622in}}{\pgfqpoint{3.696000in}{3.696000in}}%
\pgfusepath{clip}%
\pgfsetrectcap%
\pgfsetroundjoin%
\pgfsetlinewidth{1.505625pt}%
\definecolor{currentstroke}{rgb}{1.000000,0.000000,0.000000}%
\pgfsetstrokecolor{currentstroke}%
\pgfsetdash{}{0pt}%
\pgfpathmoveto{\pgfqpoint{1.180345in}{1.629357in}}%
\pgfpathlineto{\pgfqpoint{1.190654in}{1.632332in}}%
\pgfusepath{stroke}%
\end{pgfscope}%
\begin{pgfscope}%
\pgfpathrectangle{\pgfqpoint{0.100000in}{0.212622in}}{\pgfqpoint{3.696000in}{3.696000in}}%
\pgfusepath{clip}%
\pgfsetrectcap%
\pgfsetroundjoin%
\pgfsetlinewidth{1.505625pt}%
\definecolor{currentstroke}{rgb}{1.000000,0.000000,0.000000}%
\pgfsetstrokecolor{currentstroke}%
\pgfsetdash{}{0pt}%
\pgfpathmoveto{\pgfqpoint{1.183584in}{1.628342in}}%
\pgfpathlineto{\pgfqpoint{1.190654in}{1.632332in}}%
\pgfusepath{stroke}%
\end{pgfscope}%
\begin{pgfscope}%
\pgfpathrectangle{\pgfqpoint{0.100000in}{0.212622in}}{\pgfqpoint{3.696000in}{3.696000in}}%
\pgfusepath{clip}%
\pgfsetrectcap%
\pgfsetroundjoin%
\pgfsetlinewidth{1.505625pt}%
\definecolor{currentstroke}{rgb}{1.000000,0.000000,0.000000}%
\pgfsetstrokecolor{currentstroke}%
\pgfsetdash{}{0pt}%
\pgfpathmoveto{\pgfqpoint{1.185301in}{1.627732in}}%
\pgfpathlineto{\pgfqpoint{1.190654in}{1.632332in}}%
\pgfusepath{stroke}%
\end{pgfscope}%
\begin{pgfscope}%
\pgfpathrectangle{\pgfqpoint{0.100000in}{0.212622in}}{\pgfqpoint{3.696000in}{3.696000in}}%
\pgfusepath{clip}%
\pgfsetrectcap%
\pgfsetroundjoin%
\pgfsetlinewidth{1.505625pt}%
\definecolor{currentstroke}{rgb}{1.000000,0.000000,0.000000}%
\pgfsetstrokecolor{currentstroke}%
\pgfsetdash{}{0pt}%
\pgfpathmoveto{\pgfqpoint{1.187673in}{1.626976in}}%
\pgfpathlineto{\pgfqpoint{1.190654in}{1.632332in}}%
\pgfusepath{stroke}%
\end{pgfscope}%
\begin{pgfscope}%
\pgfpathrectangle{\pgfqpoint{0.100000in}{0.212622in}}{\pgfqpoint{3.696000in}{3.696000in}}%
\pgfusepath{clip}%
\pgfsetrectcap%
\pgfsetroundjoin%
\pgfsetlinewidth{1.505625pt}%
\definecolor{currentstroke}{rgb}{1.000000,0.000000,0.000000}%
\pgfsetstrokecolor{currentstroke}%
\pgfsetdash{}{0pt}%
\pgfpathmoveto{\pgfqpoint{1.188867in}{1.626449in}}%
\pgfpathlineto{\pgfqpoint{1.190654in}{1.632332in}}%
\pgfusepath{stroke}%
\end{pgfscope}%
\begin{pgfscope}%
\pgfpathrectangle{\pgfqpoint{0.100000in}{0.212622in}}{\pgfqpoint{3.696000in}{3.696000in}}%
\pgfusepath{clip}%
\pgfsetrectcap%
\pgfsetroundjoin%
\pgfsetlinewidth{1.505625pt}%
\definecolor{currentstroke}{rgb}{1.000000,0.000000,0.000000}%
\pgfsetstrokecolor{currentstroke}%
\pgfsetdash{}{0pt}%
\pgfpathmoveto{\pgfqpoint{1.190706in}{1.625710in}}%
\pgfpathlineto{\pgfqpoint{1.190654in}{1.632332in}}%
\pgfusepath{stroke}%
\end{pgfscope}%
\begin{pgfscope}%
\pgfpathrectangle{\pgfqpoint{0.100000in}{0.212622in}}{\pgfqpoint{3.696000in}{3.696000in}}%
\pgfusepath{clip}%
\pgfsetrectcap%
\pgfsetroundjoin%
\pgfsetlinewidth{1.505625pt}%
\definecolor{currentstroke}{rgb}{1.000000,0.000000,0.000000}%
\pgfsetstrokecolor{currentstroke}%
\pgfsetdash{}{0pt}%
\pgfpathmoveto{\pgfqpoint{1.193047in}{1.624776in}}%
\pgfpathlineto{\pgfqpoint{1.204098in}{1.628206in}}%
\pgfusepath{stroke}%
\end{pgfscope}%
\begin{pgfscope}%
\pgfpathrectangle{\pgfqpoint{0.100000in}{0.212622in}}{\pgfqpoint{3.696000in}{3.696000in}}%
\pgfusepath{clip}%
\pgfsetrectcap%
\pgfsetroundjoin%
\pgfsetlinewidth{1.505625pt}%
\definecolor{currentstroke}{rgb}{1.000000,0.000000,0.000000}%
\pgfsetstrokecolor{currentstroke}%
\pgfsetdash{}{0pt}%
\pgfpathmoveto{\pgfqpoint{1.194187in}{1.624104in}}%
\pgfpathlineto{\pgfqpoint{1.204098in}{1.628206in}}%
\pgfusepath{stroke}%
\end{pgfscope}%
\begin{pgfscope}%
\pgfpathrectangle{\pgfqpoint{0.100000in}{0.212622in}}{\pgfqpoint{3.696000in}{3.696000in}}%
\pgfusepath{clip}%
\pgfsetrectcap%
\pgfsetroundjoin%
\pgfsetlinewidth{1.505625pt}%
\definecolor{currentstroke}{rgb}{1.000000,0.000000,0.000000}%
\pgfsetstrokecolor{currentstroke}%
\pgfsetdash{}{0pt}%
\pgfpathmoveto{\pgfqpoint{1.194911in}{1.623824in}}%
\pgfpathlineto{\pgfqpoint{1.204098in}{1.628206in}}%
\pgfusepath{stroke}%
\end{pgfscope}%
\begin{pgfscope}%
\pgfpathrectangle{\pgfqpoint{0.100000in}{0.212622in}}{\pgfqpoint{3.696000in}{3.696000in}}%
\pgfusepath{clip}%
\pgfsetrectcap%
\pgfsetroundjoin%
\pgfsetlinewidth{1.505625pt}%
\definecolor{currentstroke}{rgb}{1.000000,0.000000,0.000000}%
\pgfsetstrokecolor{currentstroke}%
\pgfsetdash{}{0pt}%
\pgfpathmoveto{\pgfqpoint{1.196071in}{1.623245in}}%
\pgfpathlineto{\pgfqpoint{1.204098in}{1.628206in}}%
\pgfusepath{stroke}%
\end{pgfscope}%
\begin{pgfscope}%
\pgfpathrectangle{\pgfqpoint{0.100000in}{0.212622in}}{\pgfqpoint{3.696000in}{3.696000in}}%
\pgfusepath{clip}%
\pgfsetrectcap%
\pgfsetroundjoin%
\pgfsetlinewidth{1.505625pt}%
\definecolor{currentstroke}{rgb}{1.000000,0.000000,0.000000}%
\pgfsetstrokecolor{currentstroke}%
\pgfsetdash{}{0pt}%
\pgfpathmoveto{\pgfqpoint{1.196570in}{1.622748in}}%
\pgfpathlineto{\pgfqpoint{1.204098in}{1.628206in}}%
\pgfusepath{stroke}%
\end{pgfscope}%
\begin{pgfscope}%
\pgfpathrectangle{\pgfqpoint{0.100000in}{0.212622in}}{\pgfqpoint{3.696000in}{3.696000in}}%
\pgfusepath{clip}%
\pgfsetrectcap%
\pgfsetroundjoin%
\pgfsetlinewidth{1.505625pt}%
\definecolor{currentstroke}{rgb}{1.000000,0.000000,0.000000}%
\pgfsetstrokecolor{currentstroke}%
\pgfsetdash{}{0pt}%
\pgfpathmoveto{\pgfqpoint{1.196928in}{1.622568in}}%
\pgfpathlineto{\pgfqpoint{1.204098in}{1.628206in}}%
\pgfusepath{stroke}%
\end{pgfscope}%
\begin{pgfscope}%
\pgfpathrectangle{\pgfqpoint{0.100000in}{0.212622in}}{\pgfqpoint{3.696000in}{3.696000in}}%
\pgfusepath{clip}%
\pgfsetrectcap%
\pgfsetroundjoin%
\pgfsetlinewidth{1.505625pt}%
\definecolor{currentstroke}{rgb}{1.000000,0.000000,0.000000}%
\pgfsetstrokecolor{currentstroke}%
\pgfsetdash{}{0pt}%
\pgfpathmoveto{\pgfqpoint{1.198258in}{1.621949in}}%
\pgfpathlineto{\pgfqpoint{1.204098in}{1.628206in}}%
\pgfusepath{stroke}%
\end{pgfscope}%
\begin{pgfscope}%
\pgfpathrectangle{\pgfqpoint{0.100000in}{0.212622in}}{\pgfqpoint{3.696000in}{3.696000in}}%
\pgfusepath{clip}%
\pgfsetrectcap%
\pgfsetroundjoin%
\pgfsetlinewidth{1.505625pt}%
\definecolor{currentstroke}{rgb}{1.000000,0.000000,0.000000}%
\pgfsetstrokecolor{currentstroke}%
\pgfsetdash{}{0pt}%
\pgfpathmoveto{\pgfqpoint{1.199043in}{1.621664in}}%
\pgfpathlineto{\pgfqpoint{1.204098in}{1.628206in}}%
\pgfusepath{stroke}%
\end{pgfscope}%
\begin{pgfscope}%
\pgfpathrectangle{\pgfqpoint{0.100000in}{0.212622in}}{\pgfqpoint{3.696000in}{3.696000in}}%
\pgfusepath{clip}%
\pgfsetrectcap%
\pgfsetroundjoin%
\pgfsetlinewidth{1.505625pt}%
\definecolor{currentstroke}{rgb}{1.000000,0.000000,0.000000}%
\pgfsetstrokecolor{currentstroke}%
\pgfsetdash{}{0pt}%
\pgfpathmoveto{\pgfqpoint{1.199457in}{1.621489in}}%
\pgfpathlineto{\pgfqpoint{1.204098in}{1.628206in}}%
\pgfusepath{stroke}%
\end{pgfscope}%
\begin{pgfscope}%
\pgfpathrectangle{\pgfqpoint{0.100000in}{0.212622in}}{\pgfqpoint{3.696000in}{3.696000in}}%
\pgfusepath{clip}%
\pgfsetrectcap%
\pgfsetroundjoin%
\pgfsetlinewidth{1.505625pt}%
\definecolor{currentstroke}{rgb}{1.000000,0.000000,0.000000}%
\pgfsetstrokecolor{currentstroke}%
\pgfsetdash{}{0pt}%
\pgfpathmoveto{\pgfqpoint{1.200446in}{1.621140in}}%
\pgfpathlineto{\pgfqpoint{1.204098in}{1.628206in}}%
\pgfusepath{stroke}%
\end{pgfscope}%
\begin{pgfscope}%
\pgfpathrectangle{\pgfqpoint{0.100000in}{0.212622in}}{\pgfqpoint{3.696000in}{3.696000in}}%
\pgfusepath{clip}%
\pgfsetrectcap%
\pgfsetroundjoin%
\pgfsetlinewidth{1.505625pt}%
\definecolor{currentstroke}{rgb}{1.000000,0.000000,0.000000}%
\pgfsetstrokecolor{currentstroke}%
\pgfsetdash{}{0pt}%
\pgfpathmoveto{\pgfqpoint{1.200970in}{1.620928in}}%
\pgfpathlineto{\pgfqpoint{1.204098in}{1.628206in}}%
\pgfusepath{stroke}%
\end{pgfscope}%
\begin{pgfscope}%
\pgfpathrectangle{\pgfqpoint{0.100000in}{0.212622in}}{\pgfqpoint{3.696000in}{3.696000in}}%
\pgfusepath{clip}%
\pgfsetrectcap%
\pgfsetroundjoin%
\pgfsetlinewidth{1.505625pt}%
\definecolor{currentstroke}{rgb}{1.000000,0.000000,0.000000}%
\pgfsetstrokecolor{currentstroke}%
\pgfsetdash{}{0pt}%
\pgfpathmoveto{\pgfqpoint{1.202208in}{1.620433in}}%
\pgfpathlineto{\pgfqpoint{1.204098in}{1.628206in}}%
\pgfusepath{stroke}%
\end{pgfscope}%
\begin{pgfscope}%
\pgfpathrectangle{\pgfqpoint{0.100000in}{0.212622in}}{\pgfqpoint{3.696000in}{3.696000in}}%
\pgfusepath{clip}%
\pgfsetrectcap%
\pgfsetroundjoin%
\pgfsetlinewidth{1.505625pt}%
\definecolor{currentstroke}{rgb}{1.000000,0.000000,0.000000}%
\pgfsetstrokecolor{currentstroke}%
\pgfsetdash{}{0pt}%
\pgfpathmoveto{\pgfqpoint{1.204351in}{1.619714in}}%
\pgfpathlineto{\pgfqpoint{1.204098in}{1.628206in}}%
\pgfusepath{stroke}%
\end{pgfscope}%
\begin{pgfscope}%
\pgfpathrectangle{\pgfqpoint{0.100000in}{0.212622in}}{\pgfqpoint{3.696000in}{3.696000in}}%
\pgfusepath{clip}%
\pgfsetrectcap%
\pgfsetroundjoin%
\pgfsetlinewidth{1.505625pt}%
\definecolor{currentstroke}{rgb}{1.000000,0.000000,0.000000}%
\pgfsetstrokecolor{currentstroke}%
\pgfsetdash{}{0pt}%
\pgfpathmoveto{\pgfqpoint{1.205427in}{1.619189in}}%
\pgfpathlineto{\pgfqpoint{1.217552in}{1.624078in}}%
\pgfusepath{stroke}%
\end{pgfscope}%
\begin{pgfscope}%
\pgfpathrectangle{\pgfqpoint{0.100000in}{0.212622in}}{\pgfqpoint{3.696000in}{3.696000in}}%
\pgfusepath{clip}%
\pgfsetrectcap%
\pgfsetroundjoin%
\pgfsetlinewidth{1.505625pt}%
\definecolor{currentstroke}{rgb}{1.000000,0.000000,0.000000}%
\pgfsetstrokecolor{currentstroke}%
\pgfsetdash{}{0pt}%
\pgfpathmoveto{\pgfqpoint{1.208073in}{1.618041in}}%
\pgfpathlineto{\pgfqpoint{1.217552in}{1.624078in}}%
\pgfusepath{stroke}%
\end{pgfscope}%
\begin{pgfscope}%
\pgfpathrectangle{\pgfqpoint{0.100000in}{0.212622in}}{\pgfqpoint{3.696000in}{3.696000in}}%
\pgfusepath{clip}%
\pgfsetrectcap%
\pgfsetroundjoin%
\pgfsetlinewidth{1.505625pt}%
\definecolor{currentstroke}{rgb}{1.000000,0.000000,0.000000}%
\pgfsetstrokecolor{currentstroke}%
\pgfsetdash{}{0pt}%
\pgfpathmoveto{\pgfqpoint{1.212067in}{1.616673in}}%
\pgfpathlineto{\pgfqpoint{1.217552in}{1.624078in}}%
\pgfusepath{stroke}%
\end{pgfscope}%
\begin{pgfscope}%
\pgfpathrectangle{\pgfqpoint{0.100000in}{0.212622in}}{\pgfqpoint{3.696000in}{3.696000in}}%
\pgfusepath{clip}%
\pgfsetrectcap%
\pgfsetroundjoin%
\pgfsetlinewidth{1.505625pt}%
\definecolor{currentstroke}{rgb}{1.000000,0.000000,0.000000}%
\pgfsetstrokecolor{currentstroke}%
\pgfsetdash{}{0pt}%
\pgfpathmoveto{\pgfqpoint{1.214318in}{1.616046in}}%
\pgfpathlineto{\pgfqpoint{1.217552in}{1.624078in}}%
\pgfusepath{stroke}%
\end{pgfscope}%
\begin{pgfscope}%
\pgfpathrectangle{\pgfqpoint{0.100000in}{0.212622in}}{\pgfqpoint{3.696000in}{3.696000in}}%
\pgfusepath{clip}%
\pgfsetrectcap%
\pgfsetroundjoin%
\pgfsetlinewidth{1.505625pt}%
\definecolor{currentstroke}{rgb}{1.000000,0.000000,0.000000}%
\pgfsetstrokecolor{currentstroke}%
\pgfsetdash{}{0pt}%
\pgfpathmoveto{\pgfqpoint{1.215504in}{1.615616in}}%
\pgfpathlineto{\pgfqpoint{1.217552in}{1.624078in}}%
\pgfusepath{stroke}%
\end{pgfscope}%
\begin{pgfscope}%
\pgfpathrectangle{\pgfqpoint{0.100000in}{0.212622in}}{\pgfqpoint{3.696000in}{3.696000in}}%
\pgfusepath{clip}%
\pgfsetrectcap%
\pgfsetroundjoin%
\pgfsetlinewidth{1.505625pt}%
\definecolor{currentstroke}{rgb}{1.000000,0.000000,0.000000}%
\pgfsetstrokecolor{currentstroke}%
\pgfsetdash{}{0pt}%
\pgfpathmoveto{\pgfqpoint{1.219803in}{1.614149in}}%
\pgfpathlineto{\pgfqpoint{1.231015in}{1.619947in}}%
\pgfusepath{stroke}%
\end{pgfscope}%
\begin{pgfscope}%
\pgfpathrectangle{\pgfqpoint{0.100000in}{0.212622in}}{\pgfqpoint{3.696000in}{3.696000in}}%
\pgfusepath{clip}%
\pgfsetrectcap%
\pgfsetroundjoin%
\pgfsetlinewidth{1.505625pt}%
\definecolor{currentstroke}{rgb}{1.000000,0.000000,0.000000}%
\pgfsetstrokecolor{currentstroke}%
\pgfsetdash{}{0pt}%
\pgfpathmoveto{\pgfqpoint{1.225168in}{1.612008in}}%
\pgfpathlineto{\pgfqpoint{1.231015in}{1.619947in}}%
\pgfusepath{stroke}%
\end{pgfscope}%
\begin{pgfscope}%
\pgfpathrectangle{\pgfqpoint{0.100000in}{0.212622in}}{\pgfqpoint{3.696000in}{3.696000in}}%
\pgfusepath{clip}%
\pgfsetrectcap%
\pgfsetroundjoin%
\pgfsetlinewidth{1.505625pt}%
\definecolor{currentstroke}{rgb}{1.000000,0.000000,0.000000}%
\pgfsetstrokecolor{currentstroke}%
\pgfsetdash{}{0pt}%
\pgfpathmoveto{\pgfqpoint{1.228200in}{1.610901in}}%
\pgfpathlineto{\pgfqpoint{1.231015in}{1.619947in}}%
\pgfusepath{stroke}%
\end{pgfscope}%
\begin{pgfscope}%
\pgfpathrectangle{\pgfqpoint{0.100000in}{0.212622in}}{\pgfqpoint{3.696000in}{3.696000in}}%
\pgfusepath{clip}%
\pgfsetrectcap%
\pgfsetroundjoin%
\pgfsetlinewidth{1.505625pt}%
\definecolor{currentstroke}{rgb}{1.000000,0.000000,0.000000}%
\pgfsetstrokecolor{currentstroke}%
\pgfsetdash{}{0pt}%
\pgfpathmoveto{\pgfqpoint{1.229895in}{1.610343in}}%
\pgfpathlineto{\pgfqpoint{1.231015in}{1.619947in}}%
\pgfusepath{stroke}%
\end{pgfscope}%
\begin{pgfscope}%
\pgfpathrectangle{\pgfqpoint{0.100000in}{0.212622in}}{\pgfqpoint{3.696000in}{3.696000in}}%
\pgfusepath{clip}%
\pgfsetrectcap%
\pgfsetroundjoin%
\pgfsetlinewidth{1.505625pt}%
\definecolor{currentstroke}{rgb}{1.000000,0.000000,0.000000}%
\pgfsetstrokecolor{currentstroke}%
\pgfsetdash{}{0pt}%
\pgfpathmoveto{\pgfqpoint{1.232542in}{1.609466in}}%
\pgfpathlineto{\pgfqpoint{1.244487in}{1.615813in}}%
\pgfusepath{stroke}%
\end{pgfscope}%
\begin{pgfscope}%
\pgfpathrectangle{\pgfqpoint{0.100000in}{0.212622in}}{\pgfqpoint{3.696000in}{3.696000in}}%
\pgfusepath{clip}%
\pgfsetrectcap%
\pgfsetroundjoin%
\pgfsetlinewidth{1.505625pt}%
\definecolor{currentstroke}{rgb}{1.000000,0.000000,0.000000}%
\pgfsetstrokecolor{currentstroke}%
\pgfsetdash{}{0pt}%
\pgfpathmoveto{\pgfqpoint{1.235853in}{1.608327in}}%
\pgfpathlineto{\pgfqpoint{1.244487in}{1.615813in}}%
\pgfusepath{stroke}%
\end{pgfscope}%
\begin{pgfscope}%
\pgfpathrectangle{\pgfqpoint{0.100000in}{0.212622in}}{\pgfqpoint{3.696000in}{3.696000in}}%
\pgfusepath{clip}%
\pgfsetrectcap%
\pgfsetroundjoin%
\pgfsetlinewidth{1.505625pt}%
\definecolor{currentstroke}{rgb}{1.000000,0.000000,0.000000}%
\pgfsetstrokecolor{currentstroke}%
\pgfsetdash{}{0pt}%
\pgfpathmoveto{\pgfqpoint{1.240879in}{1.606250in}}%
\pgfpathlineto{\pgfqpoint{1.244487in}{1.615813in}}%
\pgfusepath{stroke}%
\end{pgfscope}%
\begin{pgfscope}%
\pgfpathrectangle{\pgfqpoint{0.100000in}{0.212622in}}{\pgfqpoint{3.696000in}{3.696000in}}%
\pgfusepath{clip}%
\pgfsetrectcap%
\pgfsetroundjoin%
\pgfsetlinewidth{1.505625pt}%
\definecolor{currentstroke}{rgb}{1.000000,0.000000,0.000000}%
\pgfsetstrokecolor{currentstroke}%
\pgfsetdash{}{0pt}%
\pgfpathmoveto{\pgfqpoint{1.244373in}{1.601389in}}%
\pgfpathlineto{\pgfqpoint{1.257968in}{1.611677in}}%
\pgfusepath{stroke}%
\end{pgfscope}%
\begin{pgfscope}%
\pgfpathrectangle{\pgfqpoint{0.100000in}{0.212622in}}{\pgfqpoint{3.696000in}{3.696000in}}%
\pgfusepath{clip}%
\pgfsetrectcap%
\pgfsetroundjoin%
\pgfsetlinewidth{1.505625pt}%
\definecolor{currentstroke}{rgb}{1.000000,0.000000,0.000000}%
\pgfsetstrokecolor{currentstroke}%
\pgfsetdash{}{0pt}%
\pgfpathmoveto{\pgfqpoint{1.247155in}{1.599635in}}%
\pgfpathlineto{\pgfqpoint{1.257968in}{1.611677in}}%
\pgfusepath{stroke}%
\end{pgfscope}%
\begin{pgfscope}%
\pgfpathrectangle{\pgfqpoint{0.100000in}{0.212622in}}{\pgfqpoint{3.696000in}{3.696000in}}%
\pgfusepath{clip}%
\pgfsetrectcap%
\pgfsetroundjoin%
\pgfsetlinewidth{1.505625pt}%
\definecolor{currentstroke}{rgb}{1.000000,0.000000,0.000000}%
\pgfsetstrokecolor{currentstroke}%
\pgfsetdash{}{0pt}%
\pgfpathmoveto{\pgfqpoint{1.251482in}{1.597923in}}%
\pgfpathlineto{\pgfqpoint{1.257968in}{1.611677in}}%
\pgfusepath{stroke}%
\end{pgfscope}%
\begin{pgfscope}%
\pgfpathrectangle{\pgfqpoint{0.100000in}{0.212622in}}{\pgfqpoint{3.696000in}{3.696000in}}%
\pgfusepath{clip}%
\pgfsetrectcap%
\pgfsetroundjoin%
\pgfsetlinewidth{1.505625pt}%
\definecolor{currentstroke}{rgb}{1.000000,0.000000,0.000000}%
\pgfsetstrokecolor{currentstroke}%
\pgfsetdash{}{0pt}%
\pgfpathmoveto{\pgfqpoint{1.253704in}{1.596819in}}%
\pgfpathlineto{\pgfqpoint{1.271459in}{1.607537in}}%
\pgfusepath{stroke}%
\end{pgfscope}%
\begin{pgfscope}%
\pgfpathrectangle{\pgfqpoint{0.100000in}{0.212622in}}{\pgfqpoint{3.696000in}{3.696000in}}%
\pgfusepath{clip}%
\pgfsetrectcap%
\pgfsetroundjoin%
\pgfsetlinewidth{1.505625pt}%
\definecolor{currentstroke}{rgb}{1.000000,0.000000,0.000000}%
\pgfsetstrokecolor{currentstroke}%
\pgfsetdash{}{0pt}%
\pgfpathmoveto{\pgfqpoint{1.254767in}{1.596036in}}%
\pgfpathlineto{\pgfqpoint{1.271459in}{1.607537in}}%
\pgfusepath{stroke}%
\end{pgfscope}%
\begin{pgfscope}%
\pgfpathrectangle{\pgfqpoint{0.100000in}{0.212622in}}{\pgfqpoint{3.696000in}{3.696000in}}%
\pgfusepath{clip}%
\pgfsetrectcap%
\pgfsetroundjoin%
\pgfsetlinewidth{1.505625pt}%
\definecolor{currentstroke}{rgb}{1.000000,0.000000,0.000000}%
\pgfsetstrokecolor{currentstroke}%
\pgfsetdash{}{0pt}%
\pgfpathmoveto{\pgfqpoint{1.257489in}{1.594796in}}%
\pgfpathlineto{\pgfqpoint{1.271459in}{1.607537in}}%
\pgfusepath{stroke}%
\end{pgfscope}%
\begin{pgfscope}%
\pgfpathrectangle{\pgfqpoint{0.100000in}{0.212622in}}{\pgfqpoint{3.696000in}{3.696000in}}%
\pgfusepath{clip}%
\pgfsetrectcap%
\pgfsetroundjoin%
\pgfsetlinewidth{1.505625pt}%
\definecolor{currentstroke}{rgb}{1.000000,0.000000,0.000000}%
\pgfsetstrokecolor{currentstroke}%
\pgfsetdash{}{0pt}%
\pgfpathmoveto{\pgfqpoint{1.258769in}{1.593939in}}%
\pgfpathlineto{\pgfqpoint{1.271459in}{1.607537in}}%
\pgfusepath{stroke}%
\end{pgfscope}%
\begin{pgfscope}%
\pgfpathrectangle{\pgfqpoint{0.100000in}{0.212622in}}{\pgfqpoint{3.696000in}{3.696000in}}%
\pgfusepath{clip}%
\pgfsetrectcap%
\pgfsetroundjoin%
\pgfsetlinewidth{1.505625pt}%
\definecolor{currentstroke}{rgb}{1.000000,0.000000,0.000000}%
\pgfsetstrokecolor{currentstroke}%
\pgfsetdash{}{0pt}%
\pgfpathmoveto{\pgfqpoint{1.261013in}{1.593292in}}%
\pgfpathlineto{\pgfqpoint{1.271459in}{1.607537in}}%
\pgfusepath{stroke}%
\end{pgfscope}%
\begin{pgfscope}%
\pgfpathrectangle{\pgfqpoint{0.100000in}{0.212622in}}{\pgfqpoint{3.696000in}{3.696000in}}%
\pgfusepath{clip}%
\pgfsetrectcap%
\pgfsetroundjoin%
\pgfsetlinewidth{1.505625pt}%
\definecolor{currentstroke}{rgb}{1.000000,0.000000,0.000000}%
\pgfsetstrokecolor{currentstroke}%
\pgfsetdash{}{0pt}%
\pgfpathmoveto{\pgfqpoint{1.264385in}{1.592117in}}%
\pgfpathlineto{\pgfqpoint{1.271459in}{1.607537in}}%
\pgfusepath{stroke}%
\end{pgfscope}%
\begin{pgfscope}%
\pgfpathrectangle{\pgfqpoint{0.100000in}{0.212622in}}{\pgfqpoint{3.696000in}{3.696000in}}%
\pgfusepath{clip}%
\pgfsetrectcap%
\pgfsetroundjoin%
\pgfsetlinewidth{1.505625pt}%
\definecolor{currentstroke}{rgb}{1.000000,0.000000,0.000000}%
\pgfsetstrokecolor{currentstroke}%
\pgfsetdash{}{0pt}%
\pgfpathmoveto{\pgfqpoint{1.266192in}{1.591423in}}%
\pgfpathlineto{\pgfqpoint{1.284959in}{1.603395in}}%
\pgfusepath{stroke}%
\end{pgfscope}%
\begin{pgfscope}%
\pgfpathrectangle{\pgfqpoint{0.100000in}{0.212622in}}{\pgfqpoint{3.696000in}{3.696000in}}%
\pgfusepath{clip}%
\pgfsetrectcap%
\pgfsetroundjoin%
\pgfsetlinewidth{1.505625pt}%
\definecolor{currentstroke}{rgb}{1.000000,0.000000,0.000000}%
\pgfsetstrokecolor{currentstroke}%
\pgfsetdash{}{0pt}%
\pgfpathmoveto{\pgfqpoint{1.268886in}{1.590196in}}%
\pgfpathlineto{\pgfqpoint{1.284959in}{1.603395in}}%
\pgfusepath{stroke}%
\end{pgfscope}%
\begin{pgfscope}%
\pgfpathrectangle{\pgfqpoint{0.100000in}{0.212622in}}{\pgfqpoint{3.696000in}{3.696000in}}%
\pgfusepath{clip}%
\pgfsetrectcap%
\pgfsetroundjoin%
\pgfsetlinewidth{1.505625pt}%
\definecolor{currentstroke}{rgb}{1.000000,0.000000,0.000000}%
\pgfsetstrokecolor{currentstroke}%
\pgfsetdash{}{0pt}%
\pgfpathmoveto{\pgfqpoint{1.272956in}{1.588873in}}%
\pgfpathlineto{\pgfqpoint{1.284959in}{1.603395in}}%
\pgfusepath{stroke}%
\end{pgfscope}%
\begin{pgfscope}%
\pgfpathrectangle{\pgfqpoint{0.100000in}{0.212622in}}{\pgfqpoint{3.696000in}{3.696000in}}%
\pgfusepath{clip}%
\pgfsetrectcap%
\pgfsetroundjoin%
\pgfsetlinewidth{1.505625pt}%
\definecolor{currentstroke}{rgb}{1.000000,0.000000,0.000000}%
\pgfsetstrokecolor{currentstroke}%
\pgfsetdash{}{0pt}%
\pgfpathmoveto{\pgfqpoint{1.275005in}{1.588012in}}%
\pgfpathlineto{\pgfqpoint{1.284959in}{1.603395in}}%
\pgfusepath{stroke}%
\end{pgfscope}%
\begin{pgfscope}%
\pgfpathrectangle{\pgfqpoint{0.100000in}{0.212622in}}{\pgfqpoint{3.696000in}{3.696000in}}%
\pgfusepath{clip}%
\pgfsetrectcap%
\pgfsetroundjoin%
\pgfsetlinewidth{1.505625pt}%
\definecolor{currentstroke}{rgb}{1.000000,0.000000,0.000000}%
\pgfsetstrokecolor{currentstroke}%
\pgfsetdash{}{0pt}%
\pgfpathmoveto{\pgfqpoint{1.278385in}{1.586875in}}%
\pgfpathlineto{\pgfqpoint{1.284959in}{1.603395in}}%
\pgfusepath{stroke}%
\end{pgfscope}%
\begin{pgfscope}%
\pgfpathrectangle{\pgfqpoint{0.100000in}{0.212622in}}{\pgfqpoint{3.696000in}{3.696000in}}%
\pgfusepath{clip}%
\pgfsetrectcap%
\pgfsetroundjoin%
\pgfsetlinewidth{1.505625pt}%
\definecolor{currentstroke}{rgb}{1.000000,0.000000,0.000000}%
\pgfsetstrokecolor{currentstroke}%
\pgfsetdash{}{0pt}%
\pgfpathmoveto{\pgfqpoint{1.283043in}{1.585564in}}%
\pgfpathlineto{\pgfqpoint{1.298468in}{1.599250in}}%
\pgfusepath{stroke}%
\end{pgfscope}%
\begin{pgfscope}%
\pgfpathrectangle{\pgfqpoint{0.100000in}{0.212622in}}{\pgfqpoint{3.696000in}{3.696000in}}%
\pgfusepath{clip}%
\pgfsetrectcap%
\pgfsetroundjoin%
\pgfsetlinewidth{1.505625pt}%
\definecolor{currentstroke}{rgb}{1.000000,0.000000,0.000000}%
\pgfsetstrokecolor{currentstroke}%
\pgfsetdash{}{0pt}%
\pgfpathmoveto{\pgfqpoint{1.288022in}{1.583711in}}%
\pgfpathlineto{\pgfqpoint{1.298468in}{1.599250in}}%
\pgfusepath{stroke}%
\end{pgfscope}%
\begin{pgfscope}%
\pgfpathrectangle{\pgfqpoint{0.100000in}{0.212622in}}{\pgfqpoint{3.696000in}{3.696000in}}%
\pgfusepath{clip}%
\pgfsetrectcap%
\pgfsetroundjoin%
\pgfsetlinewidth{1.505625pt}%
\definecolor{currentstroke}{rgb}{1.000000,0.000000,0.000000}%
\pgfsetstrokecolor{currentstroke}%
\pgfsetdash{}{0pt}%
\pgfpathmoveto{\pgfqpoint{1.293777in}{1.580758in}}%
\pgfpathlineto{\pgfqpoint{1.311987in}{1.595101in}}%
\pgfusepath{stroke}%
\end{pgfscope}%
\begin{pgfscope}%
\pgfpathrectangle{\pgfqpoint{0.100000in}{0.212622in}}{\pgfqpoint{3.696000in}{3.696000in}}%
\pgfusepath{clip}%
\pgfsetrectcap%
\pgfsetroundjoin%
\pgfsetlinewidth{1.505625pt}%
\definecolor{currentstroke}{rgb}{1.000000,0.000000,0.000000}%
\pgfsetstrokecolor{currentstroke}%
\pgfsetdash{}{0pt}%
\pgfpathmoveto{\pgfqpoint{1.300640in}{1.578177in}}%
\pgfpathlineto{\pgfqpoint{1.311987in}{1.595101in}}%
\pgfusepath{stroke}%
\end{pgfscope}%
\begin{pgfscope}%
\pgfpathrectangle{\pgfqpoint{0.100000in}{0.212622in}}{\pgfqpoint{3.696000in}{3.696000in}}%
\pgfusepath{clip}%
\pgfsetrectcap%
\pgfsetroundjoin%
\pgfsetlinewidth{1.505625pt}%
\definecolor{currentstroke}{rgb}{1.000000,0.000000,0.000000}%
\pgfsetstrokecolor{currentstroke}%
\pgfsetdash{}{0pt}%
\pgfpathmoveto{\pgfqpoint{1.308108in}{1.575496in}}%
\pgfpathlineto{\pgfqpoint{1.325515in}{1.590951in}}%
\pgfusepath{stroke}%
\end{pgfscope}%
\begin{pgfscope}%
\pgfpathrectangle{\pgfqpoint{0.100000in}{0.212622in}}{\pgfqpoint{3.696000in}{3.696000in}}%
\pgfusepath{clip}%
\pgfsetrectcap%
\pgfsetroundjoin%
\pgfsetlinewidth{1.505625pt}%
\definecolor{currentstroke}{rgb}{1.000000,0.000000,0.000000}%
\pgfsetstrokecolor{currentstroke}%
\pgfsetdash{}{0pt}%
\pgfpathmoveto{\pgfqpoint{1.317786in}{1.572295in}}%
\pgfpathlineto{\pgfqpoint{1.325515in}{1.590951in}}%
\pgfusepath{stroke}%
\end{pgfscope}%
\begin{pgfscope}%
\pgfpathrectangle{\pgfqpoint{0.100000in}{0.212622in}}{\pgfqpoint{3.696000in}{3.696000in}}%
\pgfusepath{clip}%
\pgfsetrectcap%
\pgfsetroundjoin%
\pgfsetlinewidth{1.505625pt}%
\definecolor{currentstroke}{rgb}{1.000000,0.000000,0.000000}%
\pgfsetstrokecolor{currentstroke}%
\pgfsetdash{}{0pt}%
\pgfpathmoveto{\pgfqpoint{1.328676in}{1.568776in}}%
\pgfpathlineto{\pgfqpoint{1.339052in}{1.586797in}}%
\pgfusepath{stroke}%
\end{pgfscope}%
\begin{pgfscope}%
\pgfpathrectangle{\pgfqpoint{0.100000in}{0.212622in}}{\pgfqpoint{3.696000in}{3.696000in}}%
\pgfusepath{clip}%
\pgfsetrectcap%
\pgfsetroundjoin%
\pgfsetlinewidth{1.505625pt}%
\definecolor{currentstroke}{rgb}{1.000000,0.000000,0.000000}%
\pgfsetstrokecolor{currentstroke}%
\pgfsetdash{}{0pt}%
\pgfpathmoveto{\pgfqpoint{1.334487in}{1.566786in}}%
\pgfpathlineto{\pgfqpoint{1.352599in}{1.582640in}}%
\pgfusepath{stroke}%
\end{pgfscope}%
\begin{pgfscope}%
\pgfpathrectangle{\pgfqpoint{0.100000in}{0.212622in}}{\pgfqpoint{3.696000in}{3.696000in}}%
\pgfusepath{clip}%
\pgfsetrectcap%
\pgfsetroundjoin%
\pgfsetlinewidth{1.505625pt}%
\definecolor{currentstroke}{rgb}{1.000000,0.000000,0.000000}%
\pgfsetstrokecolor{currentstroke}%
\pgfsetdash{}{0pt}%
\pgfpathmoveto{\pgfqpoint{1.340750in}{1.564206in}}%
\pgfpathlineto{\pgfqpoint{1.352599in}{1.582640in}}%
\pgfusepath{stroke}%
\end{pgfscope}%
\begin{pgfscope}%
\pgfpathrectangle{\pgfqpoint{0.100000in}{0.212622in}}{\pgfqpoint{3.696000in}{3.696000in}}%
\pgfusepath{clip}%
\pgfsetrectcap%
\pgfsetroundjoin%
\pgfsetlinewidth{1.505625pt}%
\definecolor{currentstroke}{rgb}{1.000000,0.000000,0.000000}%
\pgfsetstrokecolor{currentstroke}%
\pgfsetdash{}{0pt}%
\pgfpathmoveto{\pgfqpoint{1.348759in}{1.561922in}}%
\pgfpathlineto{\pgfqpoint{1.366155in}{1.578481in}}%
\pgfusepath{stroke}%
\end{pgfscope}%
\begin{pgfscope}%
\pgfpathrectangle{\pgfqpoint{0.100000in}{0.212622in}}{\pgfqpoint{3.696000in}{3.696000in}}%
\pgfusepath{clip}%
\pgfsetrectcap%
\pgfsetroundjoin%
\pgfsetlinewidth{1.505625pt}%
\definecolor{currentstroke}{rgb}{1.000000,0.000000,0.000000}%
\pgfsetstrokecolor{currentstroke}%
\pgfsetdash{}{0pt}%
\pgfpathmoveto{\pgfqpoint{1.357371in}{1.559496in}}%
\pgfpathlineto{\pgfqpoint{1.366155in}{1.578481in}}%
\pgfusepath{stroke}%
\end{pgfscope}%
\begin{pgfscope}%
\pgfpathrectangle{\pgfqpoint{0.100000in}{0.212622in}}{\pgfqpoint{3.696000in}{3.696000in}}%
\pgfusepath{clip}%
\pgfsetrectcap%
\pgfsetroundjoin%
\pgfsetlinewidth{1.505625pt}%
\definecolor{currentstroke}{rgb}{1.000000,0.000000,0.000000}%
\pgfsetstrokecolor{currentstroke}%
\pgfsetdash{}{0pt}%
\pgfpathmoveto{\pgfqpoint{1.364863in}{1.554097in}}%
\pgfpathlineto{\pgfqpoint{1.379720in}{1.574318in}}%
\pgfusepath{stroke}%
\end{pgfscope}%
\begin{pgfscope}%
\pgfpathrectangle{\pgfqpoint{0.100000in}{0.212622in}}{\pgfqpoint{3.696000in}{3.696000in}}%
\pgfusepath{clip}%
\pgfsetrectcap%
\pgfsetroundjoin%
\pgfsetlinewidth{1.505625pt}%
\definecolor{currentstroke}{rgb}{1.000000,0.000000,0.000000}%
\pgfsetstrokecolor{currentstroke}%
\pgfsetdash{}{0pt}%
\pgfpathmoveto{\pgfqpoint{1.375330in}{1.550804in}}%
\pgfpathlineto{\pgfqpoint{1.393294in}{1.570153in}}%
\pgfusepath{stroke}%
\end{pgfscope}%
\begin{pgfscope}%
\pgfpathrectangle{\pgfqpoint{0.100000in}{0.212622in}}{\pgfqpoint{3.696000in}{3.696000in}}%
\pgfusepath{clip}%
\pgfsetrectcap%
\pgfsetroundjoin%
\pgfsetlinewidth{1.505625pt}%
\definecolor{currentstroke}{rgb}{1.000000,0.000000,0.000000}%
\pgfsetstrokecolor{currentstroke}%
\pgfsetdash{}{0pt}%
\pgfpathmoveto{\pgfqpoint{1.381371in}{1.549361in}}%
\pgfpathlineto{\pgfqpoint{1.393294in}{1.570153in}}%
\pgfusepath{stroke}%
\end{pgfscope}%
\begin{pgfscope}%
\pgfpathrectangle{\pgfqpoint{0.100000in}{0.212622in}}{\pgfqpoint{3.696000in}{3.696000in}}%
\pgfusepath{clip}%
\pgfsetrectcap%
\pgfsetroundjoin%
\pgfsetlinewidth{1.505625pt}%
\definecolor{currentstroke}{rgb}{1.000000,0.000000,0.000000}%
\pgfsetstrokecolor{currentstroke}%
\pgfsetdash{}{0pt}%
\pgfpathmoveto{\pgfqpoint{1.384149in}{1.547968in}}%
\pgfpathlineto{\pgfqpoint{1.393294in}{1.570153in}}%
\pgfusepath{stroke}%
\end{pgfscope}%
\begin{pgfscope}%
\pgfpathrectangle{\pgfqpoint{0.100000in}{0.212622in}}{\pgfqpoint{3.696000in}{3.696000in}}%
\pgfusepath{clip}%
\pgfsetrectcap%
\pgfsetroundjoin%
\pgfsetlinewidth{1.505625pt}%
\definecolor{currentstroke}{rgb}{1.000000,0.000000,0.000000}%
\pgfsetstrokecolor{currentstroke}%
\pgfsetdash{}{0pt}%
\pgfpathmoveto{\pgfqpoint{1.388581in}{1.546809in}}%
\pgfpathlineto{\pgfqpoint{1.406878in}{1.565985in}}%
\pgfusepath{stroke}%
\end{pgfscope}%
\begin{pgfscope}%
\pgfpathrectangle{\pgfqpoint{0.100000in}{0.212622in}}{\pgfqpoint{3.696000in}{3.696000in}}%
\pgfusepath{clip}%
\pgfsetrectcap%
\pgfsetroundjoin%
\pgfsetlinewidth{1.505625pt}%
\definecolor{currentstroke}{rgb}{1.000000,0.000000,0.000000}%
\pgfsetstrokecolor{currentstroke}%
\pgfsetdash{}{0pt}%
\pgfpathmoveto{\pgfqpoint{1.390889in}{1.546108in}}%
\pgfpathlineto{\pgfqpoint{1.406878in}{1.565985in}}%
\pgfusepath{stroke}%
\end{pgfscope}%
\begin{pgfscope}%
\pgfpathrectangle{\pgfqpoint{0.100000in}{0.212622in}}{\pgfqpoint{3.696000in}{3.696000in}}%
\pgfusepath{clip}%
\pgfsetrectcap%
\pgfsetroundjoin%
\pgfsetlinewidth{1.505625pt}%
\definecolor{currentstroke}{rgb}{1.000000,0.000000,0.000000}%
\pgfsetstrokecolor{currentstroke}%
\pgfsetdash{}{0pt}%
\pgfpathmoveto{\pgfqpoint{1.393747in}{1.545435in}}%
\pgfpathlineto{\pgfqpoint{1.406878in}{1.565985in}}%
\pgfusepath{stroke}%
\end{pgfscope}%
\begin{pgfscope}%
\pgfpathrectangle{\pgfqpoint{0.100000in}{0.212622in}}{\pgfqpoint{3.696000in}{3.696000in}}%
\pgfusepath{clip}%
\pgfsetrectcap%
\pgfsetroundjoin%
\pgfsetlinewidth{1.505625pt}%
\definecolor{currentstroke}{rgb}{1.000000,0.000000,0.000000}%
\pgfsetstrokecolor{currentstroke}%
\pgfsetdash{}{0pt}%
\pgfpathmoveto{\pgfqpoint{1.397288in}{1.544293in}}%
\pgfpathlineto{\pgfqpoint{1.406878in}{1.565985in}}%
\pgfusepath{stroke}%
\end{pgfscope}%
\begin{pgfscope}%
\pgfpathrectangle{\pgfqpoint{0.100000in}{0.212622in}}{\pgfqpoint{3.696000in}{3.696000in}}%
\pgfusepath{clip}%
\pgfsetrectcap%
\pgfsetroundjoin%
\pgfsetlinewidth{1.505625pt}%
\definecolor{currentstroke}{rgb}{1.000000,0.000000,0.000000}%
\pgfsetstrokecolor{currentstroke}%
\pgfsetdash{}{0pt}%
\pgfpathmoveto{\pgfqpoint{1.399332in}{1.543832in}}%
\pgfpathlineto{\pgfqpoint{1.406878in}{1.565985in}}%
\pgfusepath{stroke}%
\end{pgfscope}%
\begin{pgfscope}%
\pgfpathrectangle{\pgfqpoint{0.100000in}{0.212622in}}{\pgfqpoint{3.696000in}{3.696000in}}%
\pgfusepath{clip}%
\pgfsetrectcap%
\pgfsetroundjoin%
\pgfsetlinewidth{1.505625pt}%
\definecolor{currentstroke}{rgb}{1.000000,0.000000,0.000000}%
\pgfsetstrokecolor{currentstroke}%
\pgfsetdash{}{0pt}%
\pgfpathmoveto{\pgfqpoint{1.400456in}{1.543567in}}%
\pgfpathlineto{\pgfqpoint{1.420472in}{1.561814in}}%
\pgfusepath{stroke}%
\end{pgfscope}%
\begin{pgfscope}%
\pgfpathrectangle{\pgfqpoint{0.100000in}{0.212622in}}{\pgfqpoint{3.696000in}{3.696000in}}%
\pgfusepath{clip}%
\pgfsetrectcap%
\pgfsetroundjoin%
\pgfsetlinewidth{1.505625pt}%
\definecolor{currentstroke}{rgb}{1.000000,0.000000,0.000000}%
\pgfsetstrokecolor{currentstroke}%
\pgfsetdash{}{0pt}%
\pgfpathmoveto{\pgfqpoint{1.403250in}{1.542751in}}%
\pgfpathlineto{\pgfqpoint{1.420472in}{1.561814in}}%
\pgfusepath{stroke}%
\end{pgfscope}%
\begin{pgfscope}%
\pgfpathrectangle{\pgfqpoint{0.100000in}{0.212622in}}{\pgfqpoint{3.696000in}{3.696000in}}%
\pgfusepath{clip}%
\pgfsetrectcap%
\pgfsetroundjoin%
\pgfsetlinewidth{1.505625pt}%
\definecolor{currentstroke}{rgb}{1.000000,0.000000,0.000000}%
\pgfsetstrokecolor{currentstroke}%
\pgfsetdash{}{0pt}%
\pgfpathmoveto{\pgfqpoint{1.407035in}{1.541459in}}%
\pgfpathlineto{\pgfqpoint{1.420472in}{1.561814in}}%
\pgfusepath{stroke}%
\end{pgfscope}%
\begin{pgfscope}%
\pgfpathrectangle{\pgfqpoint{0.100000in}{0.212622in}}{\pgfqpoint{3.696000in}{3.696000in}}%
\pgfusepath{clip}%
\pgfsetrectcap%
\pgfsetroundjoin%
\pgfsetlinewidth{1.505625pt}%
\definecolor{currentstroke}{rgb}{1.000000,0.000000,0.000000}%
\pgfsetstrokecolor{currentstroke}%
\pgfsetdash{}{0pt}%
\pgfpathmoveto{\pgfqpoint{1.411276in}{1.539928in}}%
\pgfpathlineto{\pgfqpoint{1.420472in}{1.561814in}}%
\pgfusepath{stroke}%
\end{pgfscope}%
\begin{pgfscope}%
\pgfpathrectangle{\pgfqpoint{0.100000in}{0.212622in}}{\pgfqpoint{3.696000in}{3.696000in}}%
\pgfusepath{clip}%
\pgfsetrectcap%
\pgfsetroundjoin%
\pgfsetlinewidth{1.505625pt}%
\definecolor{currentstroke}{rgb}{1.000000,0.000000,0.000000}%
\pgfsetstrokecolor{currentstroke}%
\pgfsetdash{}{0pt}%
\pgfpathmoveto{\pgfqpoint{1.416242in}{1.538210in}}%
\pgfpathlineto{\pgfqpoint{1.434074in}{1.557640in}}%
\pgfusepath{stroke}%
\end{pgfscope}%
\begin{pgfscope}%
\pgfpathrectangle{\pgfqpoint{0.100000in}{0.212622in}}{\pgfqpoint{3.696000in}{3.696000in}}%
\pgfusepath{clip}%
\pgfsetrectcap%
\pgfsetroundjoin%
\pgfsetlinewidth{1.505625pt}%
\definecolor{currentstroke}{rgb}{1.000000,0.000000,0.000000}%
\pgfsetstrokecolor{currentstroke}%
\pgfsetdash{}{0pt}%
\pgfpathmoveto{\pgfqpoint{1.422204in}{1.534918in}}%
\pgfpathlineto{\pgfqpoint{1.434074in}{1.557640in}}%
\pgfusepath{stroke}%
\end{pgfscope}%
\begin{pgfscope}%
\pgfpathrectangle{\pgfqpoint{0.100000in}{0.212622in}}{\pgfqpoint{3.696000in}{3.696000in}}%
\pgfusepath{clip}%
\pgfsetrectcap%
\pgfsetroundjoin%
\pgfsetlinewidth{1.505625pt}%
\definecolor{currentstroke}{rgb}{1.000000,0.000000,0.000000}%
\pgfsetstrokecolor{currentstroke}%
\pgfsetdash{}{0pt}%
\pgfpathmoveto{\pgfqpoint{1.429932in}{1.532569in}}%
\pgfpathlineto{\pgfqpoint{1.447687in}{1.553463in}}%
\pgfusepath{stroke}%
\end{pgfscope}%
\begin{pgfscope}%
\pgfpathrectangle{\pgfqpoint{0.100000in}{0.212622in}}{\pgfqpoint{3.696000in}{3.696000in}}%
\pgfusepath{clip}%
\pgfsetrectcap%
\pgfsetroundjoin%
\pgfsetlinewidth{1.505625pt}%
\definecolor{currentstroke}{rgb}{1.000000,0.000000,0.000000}%
\pgfsetstrokecolor{currentstroke}%
\pgfsetdash{}{0pt}%
\pgfpathmoveto{\pgfqpoint{1.433843in}{1.530977in}}%
\pgfpathlineto{\pgfqpoint{1.447687in}{1.553463in}}%
\pgfusepath{stroke}%
\end{pgfscope}%
\begin{pgfscope}%
\pgfpathrectangle{\pgfqpoint{0.100000in}{0.212622in}}{\pgfqpoint{3.696000in}{3.696000in}}%
\pgfusepath{clip}%
\pgfsetrectcap%
\pgfsetroundjoin%
\pgfsetlinewidth{1.505625pt}%
\definecolor{currentstroke}{rgb}{1.000000,0.000000,0.000000}%
\pgfsetstrokecolor{currentstroke}%
\pgfsetdash{}{0pt}%
\pgfpathmoveto{\pgfqpoint{1.436148in}{1.530265in}}%
\pgfpathlineto{\pgfqpoint{1.447687in}{1.553463in}}%
\pgfusepath{stroke}%
\end{pgfscope}%
\begin{pgfscope}%
\pgfpathrectangle{\pgfqpoint{0.100000in}{0.212622in}}{\pgfqpoint{3.696000in}{3.696000in}}%
\pgfusepath{clip}%
\pgfsetrectcap%
\pgfsetroundjoin%
\pgfsetlinewidth{1.505625pt}%
\definecolor{currentstroke}{rgb}{1.000000,0.000000,0.000000}%
\pgfsetstrokecolor{currentstroke}%
\pgfsetdash{}{0pt}%
\pgfpathmoveto{\pgfqpoint{1.439753in}{1.529148in}}%
\pgfpathlineto{\pgfqpoint{1.461308in}{1.549283in}}%
\pgfusepath{stroke}%
\end{pgfscope}%
\begin{pgfscope}%
\pgfpathrectangle{\pgfqpoint{0.100000in}{0.212622in}}{\pgfqpoint{3.696000in}{3.696000in}}%
\pgfusepath{clip}%
\pgfsetrectcap%
\pgfsetroundjoin%
\pgfsetlinewidth{1.505625pt}%
\definecolor{currentstroke}{rgb}{1.000000,0.000000,0.000000}%
\pgfsetstrokecolor{currentstroke}%
\pgfsetdash{}{0pt}%
\pgfpathmoveto{\pgfqpoint{1.443896in}{1.527629in}}%
\pgfpathlineto{\pgfqpoint{1.461308in}{1.549283in}}%
\pgfusepath{stroke}%
\end{pgfscope}%
\begin{pgfscope}%
\pgfpathrectangle{\pgfqpoint{0.100000in}{0.212622in}}{\pgfqpoint{3.696000in}{3.696000in}}%
\pgfusepath{clip}%
\pgfsetrectcap%
\pgfsetroundjoin%
\pgfsetlinewidth{1.505625pt}%
\definecolor{currentstroke}{rgb}{1.000000,0.000000,0.000000}%
\pgfsetstrokecolor{currentstroke}%
\pgfsetdash{}{0pt}%
\pgfpathmoveto{\pgfqpoint{1.446179in}{1.526817in}}%
\pgfpathlineto{\pgfqpoint{1.461308in}{1.549283in}}%
\pgfusepath{stroke}%
\end{pgfscope}%
\begin{pgfscope}%
\pgfpathrectangle{\pgfqpoint{0.100000in}{0.212622in}}{\pgfqpoint{3.696000in}{3.696000in}}%
\pgfusepath{clip}%
\pgfsetrectcap%
\pgfsetroundjoin%
\pgfsetlinewidth{1.505625pt}%
\definecolor{currentstroke}{rgb}{1.000000,0.000000,0.000000}%
\pgfsetstrokecolor{currentstroke}%
\pgfsetdash{}{0pt}%
\pgfpathmoveto{\pgfqpoint{1.450397in}{1.525244in}}%
\pgfpathlineto{\pgfqpoint{1.461308in}{1.549283in}}%
\pgfusepath{stroke}%
\end{pgfscope}%
\begin{pgfscope}%
\pgfpathrectangle{\pgfqpoint{0.100000in}{0.212622in}}{\pgfqpoint{3.696000in}{3.696000in}}%
\pgfusepath{clip}%
\pgfsetrectcap%
\pgfsetroundjoin%
\pgfsetlinewidth{1.505625pt}%
\definecolor{currentstroke}{rgb}{1.000000,0.000000,0.000000}%
\pgfsetstrokecolor{currentstroke}%
\pgfsetdash{}{0pt}%
\pgfpathmoveto{\pgfqpoint{1.455510in}{1.523321in}}%
\pgfpathlineto{\pgfqpoint{1.474939in}{1.545101in}}%
\pgfusepath{stroke}%
\end{pgfscope}%
\begin{pgfscope}%
\pgfpathrectangle{\pgfqpoint{0.100000in}{0.212622in}}{\pgfqpoint{3.696000in}{3.696000in}}%
\pgfusepath{clip}%
\pgfsetrectcap%
\pgfsetroundjoin%
\pgfsetlinewidth{1.505625pt}%
\definecolor{currentstroke}{rgb}{1.000000,0.000000,0.000000}%
\pgfsetstrokecolor{currentstroke}%
\pgfsetdash{}{0pt}%
\pgfpathmoveto{\pgfqpoint{1.461599in}{1.521466in}}%
\pgfpathlineto{\pgfqpoint{1.474939in}{1.545101in}}%
\pgfusepath{stroke}%
\end{pgfscope}%
\begin{pgfscope}%
\pgfpathrectangle{\pgfqpoint{0.100000in}{0.212622in}}{\pgfqpoint{3.696000in}{3.696000in}}%
\pgfusepath{clip}%
\pgfsetrectcap%
\pgfsetroundjoin%
\pgfsetlinewidth{1.505625pt}%
\definecolor{currentstroke}{rgb}{1.000000,0.000000,0.000000}%
\pgfsetstrokecolor{currentstroke}%
\pgfsetdash{}{0pt}%
\pgfpathmoveto{\pgfqpoint{1.464531in}{1.520135in}}%
\pgfpathlineto{\pgfqpoint{1.474939in}{1.545101in}}%
\pgfusepath{stroke}%
\end{pgfscope}%
\begin{pgfscope}%
\pgfpathrectangle{\pgfqpoint{0.100000in}{0.212622in}}{\pgfqpoint{3.696000in}{3.696000in}}%
\pgfusepath{clip}%
\pgfsetrectcap%
\pgfsetroundjoin%
\pgfsetlinewidth{1.505625pt}%
\definecolor{currentstroke}{rgb}{1.000000,0.000000,0.000000}%
\pgfsetstrokecolor{currentstroke}%
\pgfsetdash{}{0pt}%
\pgfpathmoveto{\pgfqpoint{1.469156in}{1.518428in}}%
\pgfpathlineto{\pgfqpoint{1.488580in}{1.540915in}}%
\pgfusepath{stroke}%
\end{pgfscope}%
\begin{pgfscope}%
\pgfpathrectangle{\pgfqpoint{0.100000in}{0.212622in}}{\pgfqpoint{3.696000in}{3.696000in}}%
\pgfusepath{clip}%
\pgfsetrectcap%
\pgfsetroundjoin%
\pgfsetlinewidth{1.505625pt}%
\definecolor{currentstroke}{rgb}{1.000000,0.000000,0.000000}%
\pgfsetstrokecolor{currentstroke}%
\pgfsetdash{}{0pt}%
\pgfpathmoveto{\pgfqpoint{1.475033in}{1.515737in}}%
\pgfpathlineto{\pgfqpoint{1.488580in}{1.540915in}}%
\pgfusepath{stroke}%
\end{pgfscope}%
\begin{pgfscope}%
\pgfpathrectangle{\pgfqpoint{0.100000in}{0.212622in}}{\pgfqpoint{3.696000in}{3.696000in}}%
\pgfusepath{clip}%
\pgfsetrectcap%
\pgfsetroundjoin%
\pgfsetlinewidth{1.505625pt}%
\definecolor{currentstroke}{rgb}{1.000000,0.000000,0.000000}%
\pgfsetstrokecolor{currentstroke}%
\pgfsetdash{}{0pt}%
\pgfpathmoveto{\pgfqpoint{1.482151in}{1.513001in}}%
\pgfpathlineto{\pgfqpoint{1.502229in}{1.536727in}}%
\pgfusepath{stroke}%
\end{pgfscope}%
\begin{pgfscope}%
\pgfpathrectangle{\pgfqpoint{0.100000in}{0.212622in}}{\pgfqpoint{3.696000in}{3.696000in}}%
\pgfusepath{clip}%
\pgfsetrectcap%
\pgfsetroundjoin%
\pgfsetlinewidth{1.505625pt}%
\definecolor{currentstroke}{rgb}{1.000000,0.000000,0.000000}%
\pgfsetstrokecolor{currentstroke}%
\pgfsetdash{}{0pt}%
\pgfpathmoveto{\pgfqpoint{1.486266in}{1.511626in}}%
\pgfpathlineto{\pgfqpoint{1.502229in}{1.536727in}}%
\pgfusepath{stroke}%
\end{pgfscope}%
\begin{pgfscope}%
\pgfpathrectangle{\pgfqpoint{0.100000in}{0.212622in}}{\pgfqpoint{3.696000in}{3.696000in}}%
\pgfusepath{clip}%
\pgfsetrectcap%
\pgfsetroundjoin%
\pgfsetlinewidth{1.505625pt}%
\definecolor{currentstroke}{rgb}{1.000000,0.000000,0.000000}%
\pgfsetstrokecolor{currentstroke}%
\pgfsetdash{}{0pt}%
\pgfpathmoveto{\pgfqpoint{1.492352in}{1.509492in}}%
\pgfpathlineto{\pgfqpoint{1.515889in}{1.532536in}}%
\pgfusepath{stroke}%
\end{pgfscope}%
\begin{pgfscope}%
\pgfpathrectangle{\pgfqpoint{0.100000in}{0.212622in}}{\pgfqpoint{3.696000in}{3.696000in}}%
\pgfusepath{clip}%
\pgfsetrectcap%
\pgfsetroundjoin%
\pgfsetlinewidth{1.505625pt}%
\definecolor{currentstroke}{rgb}{1.000000,0.000000,0.000000}%
\pgfsetstrokecolor{currentstroke}%
\pgfsetdash{}{0pt}%
\pgfpathmoveto{\pgfqpoint{1.499198in}{1.506726in}}%
\pgfpathlineto{\pgfqpoint{1.515889in}{1.532536in}}%
\pgfusepath{stroke}%
\end{pgfscope}%
\begin{pgfscope}%
\pgfpathrectangle{\pgfqpoint{0.100000in}{0.212622in}}{\pgfqpoint{3.696000in}{3.696000in}}%
\pgfusepath{clip}%
\pgfsetrectcap%
\pgfsetroundjoin%
\pgfsetlinewidth{1.505625pt}%
\definecolor{currentstroke}{rgb}{1.000000,0.000000,0.000000}%
\pgfsetstrokecolor{currentstroke}%
\pgfsetdash{}{0pt}%
\pgfpathmoveto{\pgfqpoint{1.503166in}{1.505381in}}%
\pgfpathlineto{\pgfqpoint{1.515889in}{1.532536in}}%
\pgfusepath{stroke}%
\end{pgfscope}%
\begin{pgfscope}%
\pgfpathrectangle{\pgfqpoint{0.100000in}{0.212622in}}{\pgfqpoint{3.696000in}{3.696000in}}%
\pgfusepath{clip}%
\pgfsetrectcap%
\pgfsetroundjoin%
\pgfsetlinewidth{1.505625pt}%
\definecolor{currentstroke}{rgb}{1.000000,0.000000,0.000000}%
\pgfsetstrokecolor{currentstroke}%
\pgfsetdash{}{0pt}%
\pgfpathmoveto{\pgfqpoint{1.508067in}{1.503746in}}%
\pgfpathlineto{\pgfqpoint{1.529557in}{1.528341in}}%
\pgfusepath{stroke}%
\end{pgfscope}%
\begin{pgfscope}%
\pgfpathrectangle{\pgfqpoint{0.100000in}{0.212622in}}{\pgfqpoint{3.696000in}{3.696000in}}%
\pgfusepath{clip}%
\pgfsetrectcap%
\pgfsetroundjoin%
\pgfsetlinewidth{1.505625pt}%
\definecolor{currentstroke}{rgb}{1.000000,0.000000,0.000000}%
\pgfsetstrokecolor{currentstroke}%
\pgfsetdash{}{0pt}%
\pgfpathmoveto{\pgfqpoint{1.513632in}{1.501097in}}%
\pgfpathlineto{\pgfqpoint{1.529557in}{1.528341in}}%
\pgfusepath{stroke}%
\end{pgfscope}%
\begin{pgfscope}%
\pgfpathrectangle{\pgfqpoint{0.100000in}{0.212622in}}{\pgfqpoint{3.696000in}{3.696000in}}%
\pgfusepath{clip}%
\pgfsetrectcap%
\pgfsetroundjoin%
\pgfsetlinewidth{1.505625pt}%
\definecolor{currentstroke}{rgb}{1.000000,0.000000,0.000000}%
\pgfsetstrokecolor{currentstroke}%
\pgfsetdash{}{0pt}%
\pgfpathmoveto{\pgfqpoint{1.520076in}{1.498076in}}%
\pgfpathlineto{\pgfqpoint{1.543236in}{1.524144in}}%
\pgfusepath{stroke}%
\end{pgfscope}%
\begin{pgfscope}%
\pgfpathrectangle{\pgfqpoint{0.100000in}{0.212622in}}{\pgfqpoint{3.696000in}{3.696000in}}%
\pgfusepath{clip}%
\pgfsetrectcap%
\pgfsetroundjoin%
\pgfsetlinewidth{1.505625pt}%
\definecolor{currentstroke}{rgb}{1.000000,0.000000,0.000000}%
\pgfsetstrokecolor{currentstroke}%
\pgfsetdash{}{0pt}%
\pgfpathmoveto{\pgfqpoint{1.523842in}{1.496747in}}%
\pgfpathlineto{\pgfqpoint{1.543236in}{1.524144in}}%
\pgfusepath{stroke}%
\end{pgfscope}%
\begin{pgfscope}%
\pgfpathrectangle{\pgfqpoint{0.100000in}{0.212622in}}{\pgfqpoint{3.696000in}{3.696000in}}%
\pgfusepath{clip}%
\pgfsetrectcap%
\pgfsetroundjoin%
\pgfsetlinewidth{1.505625pt}%
\definecolor{currentstroke}{rgb}{1.000000,0.000000,0.000000}%
\pgfsetstrokecolor{currentstroke}%
\pgfsetdash{}{0pt}%
\pgfpathmoveto{\pgfqpoint{1.528114in}{1.495197in}}%
\pgfpathlineto{\pgfqpoint{1.543236in}{1.524144in}}%
\pgfusepath{stroke}%
\end{pgfscope}%
\begin{pgfscope}%
\pgfpathrectangle{\pgfqpoint{0.100000in}{0.212622in}}{\pgfqpoint{3.696000in}{3.696000in}}%
\pgfusepath{clip}%
\pgfsetrectcap%
\pgfsetroundjoin%
\pgfsetlinewidth{1.505625pt}%
\definecolor{currentstroke}{rgb}{1.000000,0.000000,0.000000}%
\pgfsetstrokecolor{currentstroke}%
\pgfsetdash{}{0pt}%
\pgfpathmoveto{\pgfqpoint{1.534216in}{1.493226in}}%
\pgfpathlineto{\pgfqpoint{1.556923in}{1.519944in}}%
\pgfusepath{stroke}%
\end{pgfscope}%
\begin{pgfscope}%
\pgfpathrectangle{\pgfqpoint{0.100000in}{0.212622in}}{\pgfqpoint{3.696000in}{3.696000in}}%
\pgfusepath{clip}%
\pgfsetrectcap%
\pgfsetroundjoin%
\pgfsetlinewidth{1.505625pt}%
\definecolor{currentstroke}{rgb}{1.000000,0.000000,0.000000}%
\pgfsetstrokecolor{currentstroke}%
\pgfsetdash{}{0pt}%
\pgfpathmoveto{\pgfqpoint{1.541386in}{1.490662in}}%
\pgfpathlineto{\pgfqpoint{1.556923in}{1.519944in}}%
\pgfusepath{stroke}%
\end{pgfscope}%
\begin{pgfscope}%
\pgfpathrectangle{\pgfqpoint{0.100000in}{0.212622in}}{\pgfqpoint{3.696000in}{3.696000in}}%
\pgfusepath{clip}%
\pgfsetrectcap%
\pgfsetroundjoin%
\pgfsetlinewidth{1.505625pt}%
\definecolor{currentstroke}{rgb}{1.000000,0.000000,0.000000}%
\pgfsetstrokecolor{currentstroke}%
\pgfsetdash{}{0pt}%
\pgfpathmoveto{\pgfqpoint{1.545205in}{1.489092in}}%
\pgfpathlineto{\pgfqpoint{1.570621in}{1.515741in}}%
\pgfusepath{stroke}%
\end{pgfscope}%
\begin{pgfscope}%
\pgfpathrectangle{\pgfqpoint{0.100000in}{0.212622in}}{\pgfqpoint{3.696000in}{3.696000in}}%
\pgfusepath{clip}%
\pgfsetrectcap%
\pgfsetroundjoin%
\pgfsetlinewidth{1.505625pt}%
\definecolor{currentstroke}{rgb}{1.000000,0.000000,0.000000}%
\pgfsetstrokecolor{currentstroke}%
\pgfsetdash{}{0pt}%
\pgfpathmoveto{\pgfqpoint{1.549715in}{1.487325in}}%
\pgfpathlineto{\pgfqpoint{1.570621in}{1.515741in}}%
\pgfusepath{stroke}%
\end{pgfscope}%
\begin{pgfscope}%
\pgfpathrectangle{\pgfqpoint{0.100000in}{0.212622in}}{\pgfqpoint{3.696000in}{3.696000in}}%
\pgfusepath{clip}%
\pgfsetrectcap%
\pgfsetroundjoin%
\pgfsetlinewidth{1.505625pt}%
\definecolor{currentstroke}{rgb}{1.000000,0.000000,0.000000}%
\pgfsetstrokecolor{currentstroke}%
\pgfsetdash{}{0pt}%
\pgfpathmoveto{\pgfqpoint{1.555380in}{1.485433in}}%
\pgfpathlineto{\pgfqpoint{1.570621in}{1.515741in}}%
\pgfusepath{stroke}%
\end{pgfscope}%
\begin{pgfscope}%
\pgfpathrectangle{\pgfqpoint{0.100000in}{0.212622in}}{\pgfqpoint{3.696000in}{3.696000in}}%
\pgfusepath{clip}%
\pgfsetrectcap%
\pgfsetroundjoin%
\pgfsetlinewidth{1.505625pt}%
\definecolor{currentstroke}{rgb}{1.000000,0.000000,0.000000}%
\pgfsetstrokecolor{currentstroke}%
\pgfsetdash{}{0pt}%
\pgfpathmoveto{\pgfqpoint{1.561832in}{1.482877in}}%
\pgfpathlineto{\pgfqpoint{1.584327in}{1.511536in}}%
\pgfusepath{stroke}%
\end{pgfscope}%
\begin{pgfscope}%
\pgfpathrectangle{\pgfqpoint{0.100000in}{0.212622in}}{\pgfqpoint{3.696000in}{3.696000in}}%
\pgfusepath{clip}%
\pgfsetrectcap%
\pgfsetroundjoin%
\pgfsetlinewidth{1.505625pt}%
\definecolor{currentstroke}{rgb}{1.000000,0.000000,0.000000}%
\pgfsetstrokecolor{currentstroke}%
\pgfsetdash{}{0pt}%
\pgfpathmoveto{\pgfqpoint{1.565231in}{1.481405in}}%
\pgfpathlineto{\pgfqpoint{1.584327in}{1.511536in}}%
\pgfusepath{stroke}%
\end{pgfscope}%
\begin{pgfscope}%
\pgfpathrectangle{\pgfqpoint{0.100000in}{0.212622in}}{\pgfqpoint{3.696000in}{3.696000in}}%
\pgfusepath{clip}%
\pgfsetrectcap%
\pgfsetroundjoin%
\pgfsetlinewidth{1.505625pt}%
\definecolor{currentstroke}{rgb}{1.000000,0.000000,0.000000}%
\pgfsetstrokecolor{currentstroke}%
\pgfsetdash{}{0pt}%
\pgfpathmoveto{\pgfqpoint{1.567333in}{1.480837in}}%
\pgfpathlineto{\pgfqpoint{1.584327in}{1.511536in}}%
\pgfusepath{stroke}%
\end{pgfscope}%
\begin{pgfscope}%
\pgfpathrectangle{\pgfqpoint{0.100000in}{0.212622in}}{\pgfqpoint{3.696000in}{3.696000in}}%
\pgfusepath{clip}%
\pgfsetrectcap%
\pgfsetroundjoin%
\pgfsetlinewidth{1.505625pt}%
\definecolor{currentstroke}{rgb}{1.000000,0.000000,0.000000}%
\pgfsetstrokecolor{currentstroke}%
\pgfsetdash{}{0pt}%
\pgfpathmoveto{\pgfqpoint{1.570507in}{1.479888in}}%
\pgfpathlineto{\pgfqpoint{1.584327in}{1.511536in}}%
\pgfusepath{stroke}%
\end{pgfscope}%
\begin{pgfscope}%
\pgfpathrectangle{\pgfqpoint{0.100000in}{0.212622in}}{\pgfqpoint{3.696000in}{3.696000in}}%
\pgfusepath{clip}%
\pgfsetrectcap%
\pgfsetroundjoin%
\pgfsetlinewidth{1.505625pt}%
\definecolor{currentstroke}{rgb}{1.000000,0.000000,0.000000}%
\pgfsetstrokecolor{currentstroke}%
\pgfsetdash{}{0pt}%
\pgfpathmoveto{\pgfqpoint{1.574589in}{1.478758in}}%
\pgfpathlineto{\pgfqpoint{1.598043in}{1.507327in}}%
\pgfusepath{stroke}%
\end{pgfscope}%
\begin{pgfscope}%
\pgfpathrectangle{\pgfqpoint{0.100000in}{0.212622in}}{\pgfqpoint{3.696000in}{3.696000in}}%
\pgfusepath{clip}%
\pgfsetrectcap%
\pgfsetroundjoin%
\pgfsetlinewidth{1.505625pt}%
\definecolor{currentstroke}{rgb}{1.000000,0.000000,0.000000}%
\pgfsetstrokecolor{currentstroke}%
\pgfsetdash{}{0pt}%
\pgfpathmoveto{\pgfqpoint{1.576823in}{1.478036in}}%
\pgfpathlineto{\pgfqpoint{1.598043in}{1.507327in}}%
\pgfusepath{stroke}%
\end{pgfscope}%
\begin{pgfscope}%
\pgfpathrectangle{\pgfqpoint{0.100000in}{0.212622in}}{\pgfqpoint{3.696000in}{3.696000in}}%
\pgfusepath{clip}%
\pgfsetrectcap%
\pgfsetroundjoin%
\pgfsetlinewidth{1.505625pt}%
\definecolor{currentstroke}{rgb}{1.000000,0.000000,0.000000}%
\pgfsetstrokecolor{currentstroke}%
\pgfsetdash{}{0pt}%
\pgfpathmoveto{\pgfqpoint{1.579793in}{1.476979in}}%
\pgfpathlineto{\pgfqpoint{1.598043in}{1.507327in}}%
\pgfusepath{stroke}%
\end{pgfscope}%
\begin{pgfscope}%
\pgfpathrectangle{\pgfqpoint{0.100000in}{0.212622in}}{\pgfqpoint{3.696000in}{3.696000in}}%
\pgfusepath{clip}%
\pgfsetrectcap%
\pgfsetroundjoin%
\pgfsetlinewidth{1.505625pt}%
\definecolor{currentstroke}{rgb}{1.000000,0.000000,0.000000}%
\pgfsetstrokecolor{currentstroke}%
\pgfsetdash{}{0pt}%
\pgfpathmoveto{\pgfqpoint{1.583599in}{1.475339in}}%
\pgfpathlineto{\pgfqpoint{1.598043in}{1.507327in}}%
\pgfusepath{stroke}%
\end{pgfscope}%
\begin{pgfscope}%
\pgfpathrectangle{\pgfqpoint{0.100000in}{0.212622in}}{\pgfqpoint{3.696000in}{3.696000in}}%
\pgfusepath{clip}%
\pgfsetrectcap%
\pgfsetroundjoin%
\pgfsetlinewidth{1.505625pt}%
\definecolor{currentstroke}{rgb}{1.000000,0.000000,0.000000}%
\pgfsetstrokecolor{currentstroke}%
\pgfsetdash{}{0pt}%
\pgfpathmoveto{\pgfqpoint{1.588226in}{1.473638in}}%
\pgfpathlineto{\pgfqpoint{1.611769in}{1.503115in}}%
\pgfusepath{stroke}%
\end{pgfscope}%
\begin{pgfscope}%
\pgfpathrectangle{\pgfqpoint{0.100000in}{0.212622in}}{\pgfqpoint{3.696000in}{3.696000in}}%
\pgfusepath{clip}%
\pgfsetrectcap%
\pgfsetroundjoin%
\pgfsetlinewidth{1.505625pt}%
\definecolor{currentstroke}{rgb}{1.000000,0.000000,0.000000}%
\pgfsetstrokecolor{currentstroke}%
\pgfsetdash{}{0pt}%
\pgfpathmoveto{\pgfqpoint{1.590804in}{1.472790in}}%
\pgfpathlineto{\pgfqpoint{1.611769in}{1.503115in}}%
\pgfusepath{stroke}%
\end{pgfscope}%
\begin{pgfscope}%
\pgfpathrectangle{\pgfqpoint{0.100000in}{0.212622in}}{\pgfqpoint{3.696000in}{3.696000in}}%
\pgfusepath{clip}%
\pgfsetrectcap%
\pgfsetroundjoin%
\pgfsetlinewidth{1.505625pt}%
\definecolor{currentstroke}{rgb}{1.000000,0.000000,0.000000}%
\pgfsetstrokecolor{currentstroke}%
\pgfsetdash{}{0pt}%
\pgfpathmoveto{\pgfqpoint{1.592233in}{1.472360in}}%
\pgfpathlineto{\pgfqpoint{1.611769in}{1.503115in}}%
\pgfusepath{stroke}%
\end{pgfscope}%
\begin{pgfscope}%
\pgfpathrectangle{\pgfqpoint{0.100000in}{0.212622in}}{\pgfqpoint{3.696000in}{3.696000in}}%
\pgfusepath{clip}%
\pgfsetrectcap%
\pgfsetroundjoin%
\pgfsetlinewidth{1.505625pt}%
\definecolor{currentstroke}{rgb}{1.000000,0.000000,0.000000}%
\pgfsetstrokecolor{currentstroke}%
\pgfsetdash{}{0pt}%
\pgfpathmoveto{\pgfqpoint{1.595873in}{1.471186in}}%
\pgfpathlineto{\pgfqpoint{1.611769in}{1.503115in}}%
\pgfusepath{stroke}%
\end{pgfscope}%
\begin{pgfscope}%
\pgfpathrectangle{\pgfqpoint{0.100000in}{0.212622in}}{\pgfqpoint{3.696000in}{3.696000in}}%
\pgfusepath{clip}%
\pgfsetrectcap%
\pgfsetroundjoin%
\pgfsetlinewidth{1.505625pt}%
\definecolor{currentstroke}{rgb}{1.000000,0.000000,0.000000}%
\pgfsetstrokecolor{currentstroke}%
\pgfsetdash{}{0pt}%
\pgfpathmoveto{\pgfqpoint{1.599915in}{1.469657in}}%
\pgfpathlineto{\pgfqpoint{1.625504in}{1.498901in}}%
\pgfusepath{stroke}%
\end{pgfscope}%
\begin{pgfscope}%
\pgfpathrectangle{\pgfqpoint{0.100000in}{0.212622in}}{\pgfqpoint{3.696000in}{3.696000in}}%
\pgfusepath{clip}%
\pgfsetrectcap%
\pgfsetroundjoin%
\pgfsetlinewidth{1.505625pt}%
\definecolor{currentstroke}{rgb}{1.000000,0.000000,0.000000}%
\pgfsetstrokecolor{currentstroke}%
\pgfsetdash{}{0pt}%
\pgfpathmoveto{\pgfqpoint{1.602125in}{1.468799in}}%
\pgfpathlineto{\pgfqpoint{1.625504in}{1.498901in}}%
\pgfusepath{stroke}%
\end{pgfscope}%
\begin{pgfscope}%
\pgfpathrectangle{\pgfqpoint{0.100000in}{0.212622in}}{\pgfqpoint{3.696000in}{3.696000in}}%
\pgfusepath{clip}%
\pgfsetrectcap%
\pgfsetroundjoin%
\pgfsetlinewidth{1.505625pt}%
\definecolor{currentstroke}{rgb}{1.000000,0.000000,0.000000}%
\pgfsetstrokecolor{currentstroke}%
\pgfsetdash{}{0pt}%
\pgfpathmoveto{\pgfqpoint{1.604872in}{1.467800in}}%
\pgfpathlineto{\pgfqpoint{1.625504in}{1.498901in}}%
\pgfusepath{stroke}%
\end{pgfscope}%
\begin{pgfscope}%
\pgfpathrectangle{\pgfqpoint{0.100000in}{0.212622in}}{\pgfqpoint{3.696000in}{3.696000in}}%
\pgfusepath{clip}%
\pgfsetrectcap%
\pgfsetroundjoin%
\pgfsetlinewidth{1.505625pt}%
\definecolor{currentstroke}{rgb}{1.000000,0.000000,0.000000}%
\pgfsetstrokecolor{currentstroke}%
\pgfsetdash{}{0pt}%
\pgfpathmoveto{\pgfqpoint{1.608752in}{1.466254in}}%
\pgfpathlineto{\pgfqpoint{1.625504in}{1.498901in}}%
\pgfusepath{stroke}%
\end{pgfscope}%
\begin{pgfscope}%
\pgfpathrectangle{\pgfqpoint{0.100000in}{0.212622in}}{\pgfqpoint{3.696000in}{3.696000in}}%
\pgfusepath{clip}%
\pgfsetrectcap%
\pgfsetroundjoin%
\pgfsetlinewidth{1.505625pt}%
\definecolor{currentstroke}{rgb}{1.000000,0.000000,0.000000}%
\pgfsetstrokecolor{currentstroke}%
\pgfsetdash{}{0pt}%
\pgfpathmoveto{\pgfqpoint{1.613789in}{1.464229in}}%
\pgfpathlineto{\pgfqpoint{1.639249in}{1.494683in}}%
\pgfusepath{stroke}%
\end{pgfscope}%
\begin{pgfscope}%
\pgfpathrectangle{\pgfqpoint{0.100000in}{0.212622in}}{\pgfqpoint{3.696000in}{3.696000in}}%
\pgfusepath{clip}%
\pgfsetrectcap%
\pgfsetroundjoin%
\pgfsetlinewidth{1.505625pt}%
\definecolor{currentstroke}{rgb}{1.000000,0.000000,0.000000}%
\pgfsetstrokecolor{currentstroke}%
\pgfsetdash{}{0pt}%
\pgfpathmoveto{\pgfqpoint{1.616505in}{1.463111in}}%
\pgfpathlineto{\pgfqpoint{1.639249in}{1.494683in}}%
\pgfusepath{stroke}%
\end{pgfscope}%
\begin{pgfscope}%
\pgfpathrectangle{\pgfqpoint{0.100000in}{0.212622in}}{\pgfqpoint{3.696000in}{3.696000in}}%
\pgfusepath{clip}%
\pgfsetrectcap%
\pgfsetroundjoin%
\pgfsetlinewidth{1.505625pt}%
\definecolor{currentstroke}{rgb}{1.000000,0.000000,0.000000}%
\pgfsetstrokecolor{currentstroke}%
\pgfsetdash{}{0pt}%
\pgfpathmoveto{\pgfqpoint{1.618090in}{1.462613in}}%
\pgfpathlineto{\pgfqpoint{1.639249in}{1.494683in}}%
\pgfusepath{stroke}%
\end{pgfscope}%
\begin{pgfscope}%
\pgfpathrectangle{\pgfqpoint{0.100000in}{0.212622in}}{\pgfqpoint{3.696000in}{3.696000in}}%
\pgfusepath{clip}%
\pgfsetrectcap%
\pgfsetroundjoin%
\pgfsetlinewidth{1.505625pt}%
\definecolor{currentstroke}{rgb}{1.000000,0.000000,0.000000}%
\pgfsetstrokecolor{currentstroke}%
\pgfsetdash{}{0pt}%
\pgfpathmoveto{\pgfqpoint{1.620843in}{1.461819in}}%
\pgfpathlineto{\pgfqpoint{1.639249in}{1.494683in}}%
\pgfusepath{stroke}%
\end{pgfscope}%
\begin{pgfscope}%
\pgfpathrectangle{\pgfqpoint{0.100000in}{0.212622in}}{\pgfqpoint{3.696000in}{3.696000in}}%
\pgfusepath{clip}%
\pgfsetrectcap%
\pgfsetroundjoin%
\pgfsetlinewidth{1.505625pt}%
\definecolor{currentstroke}{rgb}{1.000000,0.000000,0.000000}%
\pgfsetstrokecolor{currentstroke}%
\pgfsetdash{}{0pt}%
\pgfpathmoveto{\pgfqpoint{1.624912in}{1.460514in}}%
\pgfpathlineto{\pgfqpoint{1.653004in}{1.490463in}}%
\pgfusepath{stroke}%
\end{pgfscope}%
\begin{pgfscope}%
\pgfpathrectangle{\pgfqpoint{0.100000in}{0.212622in}}{\pgfqpoint{3.696000in}{3.696000in}}%
\pgfusepath{clip}%
\pgfsetrectcap%
\pgfsetroundjoin%
\pgfsetlinewidth{1.505625pt}%
\definecolor{currentstroke}{rgb}{1.000000,0.000000,0.000000}%
\pgfsetstrokecolor{currentstroke}%
\pgfsetdash{}{0pt}%
\pgfpathmoveto{\pgfqpoint{1.629496in}{1.458788in}}%
\pgfpathlineto{\pgfqpoint{1.653004in}{1.490463in}}%
\pgfusepath{stroke}%
\end{pgfscope}%
\begin{pgfscope}%
\pgfpathrectangle{\pgfqpoint{0.100000in}{0.212622in}}{\pgfqpoint{3.696000in}{3.696000in}}%
\pgfusepath{clip}%
\pgfsetrectcap%
\pgfsetroundjoin%
\pgfsetlinewidth{1.505625pt}%
\definecolor{currentstroke}{rgb}{1.000000,0.000000,0.000000}%
\pgfsetstrokecolor{currentstroke}%
\pgfsetdash{}{0pt}%
\pgfpathmoveto{\pgfqpoint{1.631937in}{1.457726in}}%
\pgfpathlineto{\pgfqpoint{1.653004in}{1.490463in}}%
\pgfusepath{stroke}%
\end{pgfscope}%
\begin{pgfscope}%
\pgfpathrectangle{\pgfqpoint{0.100000in}{0.212622in}}{\pgfqpoint{3.696000in}{3.696000in}}%
\pgfusepath{clip}%
\pgfsetrectcap%
\pgfsetroundjoin%
\pgfsetlinewidth{1.505625pt}%
\definecolor{currentstroke}{rgb}{1.000000,0.000000,0.000000}%
\pgfsetstrokecolor{currentstroke}%
\pgfsetdash{}{0pt}%
\pgfpathmoveto{\pgfqpoint{1.633394in}{1.457285in}}%
\pgfpathlineto{\pgfqpoint{1.653004in}{1.490463in}}%
\pgfusepath{stroke}%
\end{pgfscope}%
\begin{pgfscope}%
\pgfpathrectangle{\pgfqpoint{0.100000in}{0.212622in}}{\pgfqpoint{3.696000in}{3.696000in}}%
\pgfusepath{clip}%
\pgfsetrectcap%
\pgfsetroundjoin%
\pgfsetlinewidth{1.505625pt}%
\definecolor{currentstroke}{rgb}{1.000000,0.000000,0.000000}%
\pgfsetstrokecolor{currentstroke}%
\pgfsetdash{}{0pt}%
\pgfpathmoveto{\pgfqpoint{1.636205in}{1.456444in}}%
\pgfpathlineto{\pgfqpoint{1.653004in}{1.490463in}}%
\pgfusepath{stroke}%
\end{pgfscope}%
\begin{pgfscope}%
\pgfpathrectangle{\pgfqpoint{0.100000in}{0.212622in}}{\pgfqpoint{3.696000in}{3.696000in}}%
\pgfusepath{clip}%
\pgfsetrectcap%
\pgfsetroundjoin%
\pgfsetlinewidth{1.505625pt}%
\definecolor{currentstroke}{rgb}{1.000000,0.000000,0.000000}%
\pgfsetstrokecolor{currentstroke}%
\pgfsetdash{}{0pt}%
\pgfpathmoveto{\pgfqpoint{1.639443in}{1.455276in}}%
\pgfpathlineto{\pgfqpoint{1.666767in}{1.486239in}}%
\pgfusepath{stroke}%
\end{pgfscope}%
\begin{pgfscope}%
\pgfpathrectangle{\pgfqpoint{0.100000in}{0.212622in}}{\pgfqpoint{3.696000in}{3.696000in}}%
\pgfusepath{clip}%
\pgfsetrectcap%
\pgfsetroundjoin%
\pgfsetlinewidth{1.505625pt}%
\definecolor{currentstroke}{rgb}{1.000000,0.000000,0.000000}%
\pgfsetstrokecolor{currentstroke}%
\pgfsetdash{}{0pt}%
\pgfpathmoveto{\pgfqpoint{1.641248in}{1.454620in}}%
\pgfpathlineto{\pgfqpoint{1.666767in}{1.486239in}}%
\pgfusepath{stroke}%
\end{pgfscope}%
\begin{pgfscope}%
\pgfpathrectangle{\pgfqpoint{0.100000in}{0.212622in}}{\pgfqpoint{3.696000in}{3.696000in}}%
\pgfusepath{clip}%
\pgfsetrectcap%
\pgfsetroundjoin%
\pgfsetlinewidth{1.505625pt}%
\definecolor{currentstroke}{rgb}{1.000000,0.000000,0.000000}%
\pgfsetstrokecolor{currentstroke}%
\pgfsetdash{}{0pt}%
\pgfpathmoveto{\pgfqpoint{1.644029in}{1.453659in}}%
\pgfpathlineto{\pgfqpoint{1.666767in}{1.486239in}}%
\pgfusepath{stroke}%
\end{pgfscope}%
\begin{pgfscope}%
\pgfpathrectangle{\pgfqpoint{0.100000in}{0.212622in}}{\pgfqpoint{3.696000in}{3.696000in}}%
\pgfusepath{clip}%
\pgfsetrectcap%
\pgfsetroundjoin%
\pgfsetlinewidth{1.505625pt}%
\definecolor{currentstroke}{rgb}{1.000000,0.000000,0.000000}%
\pgfsetstrokecolor{currentstroke}%
\pgfsetdash{}{0pt}%
\pgfpathmoveto{\pgfqpoint{1.647992in}{1.452140in}}%
\pgfpathlineto{\pgfqpoint{1.666767in}{1.486239in}}%
\pgfusepath{stroke}%
\end{pgfscope}%
\begin{pgfscope}%
\pgfpathrectangle{\pgfqpoint{0.100000in}{0.212622in}}{\pgfqpoint{3.696000in}{3.696000in}}%
\pgfusepath{clip}%
\pgfsetrectcap%
\pgfsetroundjoin%
\pgfsetlinewidth{1.505625pt}%
\definecolor{currentstroke}{rgb}{1.000000,0.000000,0.000000}%
\pgfsetstrokecolor{currentstroke}%
\pgfsetdash{}{0pt}%
\pgfpathmoveto{\pgfqpoint{1.652929in}{1.450788in}}%
\pgfpathlineto{\pgfqpoint{1.680541in}{1.482013in}}%
\pgfusepath{stroke}%
\end{pgfscope}%
\begin{pgfscope}%
\pgfpathrectangle{\pgfqpoint{0.100000in}{0.212622in}}{\pgfqpoint{3.696000in}{3.696000in}}%
\pgfusepath{clip}%
\pgfsetrectcap%
\pgfsetroundjoin%
\pgfsetlinewidth{1.505625pt}%
\definecolor{currentstroke}{rgb}{1.000000,0.000000,0.000000}%
\pgfsetstrokecolor{currentstroke}%
\pgfsetdash{}{0pt}%
\pgfpathmoveto{\pgfqpoint{1.655595in}{1.450009in}}%
\pgfpathlineto{\pgfqpoint{1.680541in}{1.482013in}}%
\pgfusepath{stroke}%
\end{pgfscope}%
\begin{pgfscope}%
\pgfpathrectangle{\pgfqpoint{0.100000in}{0.212622in}}{\pgfqpoint{3.696000in}{3.696000in}}%
\pgfusepath{clip}%
\pgfsetrectcap%
\pgfsetroundjoin%
\pgfsetlinewidth{1.505625pt}%
\definecolor{currentstroke}{rgb}{1.000000,0.000000,0.000000}%
\pgfsetstrokecolor{currentstroke}%
\pgfsetdash{}{0pt}%
\pgfpathmoveto{\pgfqpoint{1.657101in}{1.449618in}}%
\pgfpathlineto{\pgfqpoint{1.680541in}{1.482013in}}%
\pgfusepath{stroke}%
\end{pgfscope}%
\begin{pgfscope}%
\pgfpathrectangle{\pgfqpoint{0.100000in}{0.212622in}}{\pgfqpoint{3.696000in}{3.696000in}}%
\pgfusepath{clip}%
\pgfsetrectcap%
\pgfsetroundjoin%
\pgfsetlinewidth{1.505625pt}%
\definecolor{currentstroke}{rgb}{1.000000,0.000000,0.000000}%
\pgfsetstrokecolor{currentstroke}%
\pgfsetdash{}{0pt}%
\pgfpathmoveto{\pgfqpoint{1.660593in}{1.448630in}}%
\pgfpathlineto{\pgfqpoint{1.680541in}{1.482013in}}%
\pgfusepath{stroke}%
\end{pgfscope}%
\begin{pgfscope}%
\pgfpathrectangle{\pgfqpoint{0.100000in}{0.212622in}}{\pgfqpoint{3.696000in}{3.696000in}}%
\pgfusepath{clip}%
\pgfsetrectcap%
\pgfsetroundjoin%
\pgfsetlinewidth{1.505625pt}%
\definecolor{currentstroke}{rgb}{1.000000,0.000000,0.000000}%
\pgfsetstrokecolor{currentstroke}%
\pgfsetdash{}{0pt}%
\pgfpathmoveto{\pgfqpoint{1.666320in}{1.447750in}}%
\pgfpathlineto{\pgfqpoint{1.694324in}{1.477784in}}%
\pgfusepath{stroke}%
\end{pgfscope}%
\begin{pgfscope}%
\pgfpathrectangle{\pgfqpoint{0.100000in}{0.212622in}}{\pgfqpoint{3.696000in}{3.696000in}}%
\pgfusepath{clip}%
\pgfsetrectcap%
\pgfsetroundjoin%
\pgfsetlinewidth{1.505625pt}%
\definecolor{currentstroke}{rgb}{1.000000,0.000000,0.000000}%
\pgfsetstrokecolor{currentstroke}%
\pgfsetdash{}{0pt}%
\pgfpathmoveto{\pgfqpoint{1.672380in}{1.446368in}}%
\pgfpathlineto{\pgfqpoint{1.694324in}{1.477784in}}%
\pgfusepath{stroke}%
\end{pgfscope}%
\begin{pgfscope}%
\pgfpathrectangle{\pgfqpoint{0.100000in}{0.212622in}}{\pgfqpoint{3.696000in}{3.696000in}}%
\pgfusepath{clip}%
\pgfsetrectcap%
\pgfsetroundjoin%
\pgfsetlinewidth{1.505625pt}%
\definecolor{currentstroke}{rgb}{1.000000,0.000000,0.000000}%
\pgfsetstrokecolor{currentstroke}%
\pgfsetdash{}{0pt}%
\pgfpathmoveto{\pgfqpoint{1.675380in}{1.445158in}}%
\pgfpathlineto{\pgfqpoint{1.694324in}{1.477784in}}%
\pgfusepath{stroke}%
\end{pgfscope}%
\begin{pgfscope}%
\pgfpathrectangle{\pgfqpoint{0.100000in}{0.212622in}}{\pgfqpoint{3.696000in}{3.696000in}}%
\pgfusepath{clip}%
\pgfsetrectcap%
\pgfsetroundjoin%
\pgfsetlinewidth{1.505625pt}%
\definecolor{currentstroke}{rgb}{1.000000,0.000000,0.000000}%
\pgfsetstrokecolor{currentstroke}%
\pgfsetdash{}{0pt}%
\pgfpathmoveto{\pgfqpoint{1.677198in}{1.444733in}}%
\pgfpathlineto{\pgfqpoint{1.694324in}{1.477784in}}%
\pgfusepath{stroke}%
\end{pgfscope}%
\begin{pgfscope}%
\pgfpathrectangle{\pgfqpoint{0.100000in}{0.212622in}}{\pgfqpoint{3.696000in}{3.696000in}}%
\pgfusepath{clip}%
\pgfsetrectcap%
\pgfsetroundjoin%
\pgfsetlinewidth{1.505625pt}%
\definecolor{currentstroke}{rgb}{1.000000,0.000000,0.000000}%
\pgfsetstrokecolor{currentstroke}%
\pgfsetdash{}{0pt}%
\pgfpathmoveto{\pgfqpoint{1.680275in}{1.443919in}}%
\pgfpathlineto{\pgfqpoint{1.708117in}{1.473552in}}%
\pgfusepath{stroke}%
\end{pgfscope}%
\begin{pgfscope}%
\pgfpathrectangle{\pgfqpoint{0.100000in}{0.212622in}}{\pgfqpoint{3.696000in}{3.696000in}}%
\pgfusepath{clip}%
\pgfsetrectcap%
\pgfsetroundjoin%
\pgfsetlinewidth{1.505625pt}%
\definecolor{currentstroke}{rgb}{1.000000,0.000000,0.000000}%
\pgfsetstrokecolor{currentstroke}%
\pgfsetdash{}{0pt}%
\pgfpathmoveto{\pgfqpoint{1.685008in}{1.443051in}}%
\pgfpathlineto{\pgfqpoint{1.708117in}{1.473552in}}%
\pgfusepath{stroke}%
\end{pgfscope}%
\begin{pgfscope}%
\pgfpathrectangle{\pgfqpoint{0.100000in}{0.212622in}}{\pgfqpoint{3.696000in}{3.696000in}}%
\pgfusepath{clip}%
\pgfsetrectcap%
\pgfsetroundjoin%
\pgfsetlinewidth{1.505625pt}%
\definecolor{currentstroke}{rgb}{1.000000,0.000000,0.000000}%
\pgfsetstrokecolor{currentstroke}%
\pgfsetdash{}{0pt}%
\pgfpathmoveto{\pgfqpoint{1.689801in}{1.441210in}}%
\pgfpathlineto{\pgfqpoint{1.708117in}{1.473552in}}%
\pgfusepath{stroke}%
\end{pgfscope}%
\begin{pgfscope}%
\pgfpathrectangle{\pgfqpoint{0.100000in}{0.212622in}}{\pgfqpoint{3.696000in}{3.696000in}}%
\pgfusepath{clip}%
\pgfsetrectcap%
\pgfsetroundjoin%
\pgfsetlinewidth{1.505625pt}%
\definecolor{currentstroke}{rgb}{1.000000,0.000000,0.000000}%
\pgfsetstrokecolor{currentstroke}%
\pgfsetdash{}{0pt}%
\pgfpathmoveto{\pgfqpoint{1.692242in}{1.440068in}}%
\pgfpathlineto{\pgfqpoint{1.708117in}{1.473552in}}%
\pgfusepath{stroke}%
\end{pgfscope}%
\begin{pgfscope}%
\pgfpathrectangle{\pgfqpoint{0.100000in}{0.212622in}}{\pgfqpoint{3.696000in}{3.696000in}}%
\pgfusepath{clip}%
\pgfsetrectcap%
\pgfsetroundjoin%
\pgfsetlinewidth{1.505625pt}%
\definecolor{currentstroke}{rgb}{1.000000,0.000000,0.000000}%
\pgfsetstrokecolor{currentstroke}%
\pgfsetdash{}{0pt}%
\pgfpathmoveto{\pgfqpoint{1.693728in}{1.439617in}}%
\pgfpathlineto{\pgfqpoint{1.708117in}{1.473552in}}%
\pgfusepath{stroke}%
\end{pgfscope}%
\begin{pgfscope}%
\pgfpathrectangle{\pgfqpoint{0.100000in}{0.212622in}}{\pgfqpoint{3.696000in}{3.696000in}}%
\pgfusepath{clip}%
\pgfsetrectcap%
\pgfsetroundjoin%
\pgfsetlinewidth{1.505625pt}%
\definecolor{currentstroke}{rgb}{1.000000,0.000000,0.000000}%
\pgfsetstrokecolor{currentstroke}%
\pgfsetdash{}{0pt}%
\pgfpathmoveto{\pgfqpoint{1.696802in}{1.438710in}}%
\pgfpathlineto{\pgfqpoint{1.721919in}{1.469317in}}%
\pgfusepath{stroke}%
\end{pgfscope}%
\begin{pgfscope}%
\pgfpathrectangle{\pgfqpoint{0.100000in}{0.212622in}}{\pgfqpoint{3.696000in}{3.696000in}}%
\pgfusepath{clip}%
\pgfsetrectcap%
\pgfsetroundjoin%
\pgfsetlinewidth{1.505625pt}%
\definecolor{currentstroke}{rgb}{1.000000,0.000000,0.000000}%
\pgfsetstrokecolor{currentstroke}%
\pgfsetdash{}{0pt}%
\pgfpathmoveto{\pgfqpoint{1.701335in}{1.437093in}}%
\pgfpathlineto{\pgfqpoint{1.721919in}{1.469317in}}%
\pgfusepath{stroke}%
\end{pgfscope}%
\begin{pgfscope}%
\pgfpathrectangle{\pgfqpoint{0.100000in}{0.212622in}}{\pgfqpoint{3.696000in}{3.696000in}}%
\pgfusepath{clip}%
\pgfsetrectcap%
\pgfsetroundjoin%
\pgfsetlinewidth{1.505625pt}%
\definecolor{currentstroke}{rgb}{1.000000,0.000000,0.000000}%
\pgfsetstrokecolor{currentstroke}%
\pgfsetdash{}{0pt}%
\pgfpathmoveto{\pgfqpoint{1.706617in}{1.434881in}}%
\pgfpathlineto{\pgfqpoint{1.721919in}{1.469317in}}%
\pgfusepath{stroke}%
\end{pgfscope}%
\begin{pgfscope}%
\pgfpathrectangle{\pgfqpoint{0.100000in}{0.212622in}}{\pgfqpoint{3.696000in}{3.696000in}}%
\pgfusepath{clip}%
\pgfsetrectcap%
\pgfsetroundjoin%
\pgfsetlinewidth{1.505625pt}%
\definecolor{currentstroke}{rgb}{1.000000,0.000000,0.000000}%
\pgfsetstrokecolor{currentstroke}%
\pgfsetdash{}{0pt}%
\pgfpathmoveto{\pgfqpoint{1.713440in}{1.433580in}}%
\pgfpathlineto{\pgfqpoint{1.735731in}{1.465079in}}%
\pgfusepath{stroke}%
\end{pgfscope}%
\begin{pgfscope}%
\pgfpathrectangle{\pgfqpoint{0.100000in}{0.212622in}}{\pgfqpoint{3.696000in}{3.696000in}}%
\pgfusepath{clip}%
\pgfsetrectcap%
\pgfsetroundjoin%
\pgfsetlinewidth{1.505625pt}%
\definecolor{currentstroke}{rgb}{1.000000,0.000000,0.000000}%
\pgfsetstrokecolor{currentstroke}%
\pgfsetdash{}{0pt}%
\pgfpathmoveto{\pgfqpoint{1.716861in}{1.432436in}}%
\pgfpathlineto{\pgfqpoint{1.735731in}{1.465079in}}%
\pgfusepath{stroke}%
\end{pgfscope}%
\begin{pgfscope}%
\pgfpathrectangle{\pgfqpoint{0.100000in}{0.212622in}}{\pgfqpoint{3.696000in}{3.696000in}}%
\pgfusepath{clip}%
\pgfsetrectcap%
\pgfsetroundjoin%
\pgfsetlinewidth{1.505625pt}%
\definecolor{currentstroke}{rgb}{1.000000,0.000000,0.000000}%
\pgfsetstrokecolor{currentstroke}%
\pgfsetdash{}{0pt}%
\pgfpathmoveto{\pgfqpoint{1.721211in}{1.431047in}}%
\pgfpathlineto{\pgfqpoint{1.735731in}{1.465079in}}%
\pgfusepath{stroke}%
\end{pgfscope}%
\begin{pgfscope}%
\pgfpathrectangle{\pgfqpoint{0.100000in}{0.212622in}}{\pgfqpoint{3.696000in}{3.696000in}}%
\pgfusepath{clip}%
\pgfsetrectcap%
\pgfsetroundjoin%
\pgfsetlinewidth{1.505625pt}%
\definecolor{currentstroke}{rgb}{1.000000,0.000000,0.000000}%
\pgfsetstrokecolor{currentstroke}%
\pgfsetdash{}{0pt}%
\pgfpathmoveto{\pgfqpoint{1.726459in}{1.428438in}}%
\pgfpathlineto{\pgfqpoint{1.749552in}{1.460837in}}%
\pgfusepath{stroke}%
\end{pgfscope}%
\begin{pgfscope}%
\pgfpathrectangle{\pgfqpoint{0.100000in}{0.212622in}}{\pgfqpoint{3.696000in}{3.696000in}}%
\pgfusepath{clip}%
\pgfsetrectcap%
\pgfsetroundjoin%
\pgfsetlinewidth{1.505625pt}%
\definecolor{currentstroke}{rgb}{1.000000,0.000000,0.000000}%
\pgfsetstrokecolor{currentstroke}%
\pgfsetdash{}{0pt}%
\pgfpathmoveto{\pgfqpoint{1.734037in}{1.426399in}}%
\pgfpathlineto{\pgfqpoint{1.749552in}{1.460837in}}%
\pgfusepath{stroke}%
\end{pgfscope}%
\begin{pgfscope}%
\pgfpathrectangle{\pgfqpoint{0.100000in}{0.212622in}}{\pgfqpoint{3.696000in}{3.696000in}}%
\pgfusepath{clip}%
\pgfsetrectcap%
\pgfsetroundjoin%
\pgfsetlinewidth{1.505625pt}%
\definecolor{currentstroke}{rgb}{1.000000,0.000000,0.000000}%
\pgfsetstrokecolor{currentstroke}%
\pgfsetdash{}{0pt}%
\pgfpathmoveto{\pgfqpoint{1.742195in}{1.423575in}}%
\pgfpathlineto{\pgfqpoint{1.763383in}{1.456594in}}%
\pgfusepath{stroke}%
\end{pgfscope}%
\begin{pgfscope}%
\pgfpathrectangle{\pgfqpoint{0.100000in}{0.212622in}}{\pgfqpoint{3.696000in}{3.696000in}}%
\pgfusepath{clip}%
\pgfsetrectcap%
\pgfsetroundjoin%
\pgfsetlinewidth{1.505625pt}%
\definecolor{currentstroke}{rgb}{1.000000,0.000000,0.000000}%
\pgfsetstrokecolor{currentstroke}%
\pgfsetdash{}{0pt}%
\pgfpathmoveto{\pgfqpoint{1.750552in}{1.420203in}}%
\pgfpathlineto{\pgfqpoint{1.777224in}{1.452347in}}%
\pgfusepath{stroke}%
\end{pgfscope}%
\begin{pgfscope}%
\pgfpathrectangle{\pgfqpoint{0.100000in}{0.212622in}}{\pgfqpoint{3.696000in}{3.696000in}}%
\pgfusepath{clip}%
\pgfsetrectcap%
\pgfsetroundjoin%
\pgfsetlinewidth{1.505625pt}%
\definecolor{currentstroke}{rgb}{1.000000,0.000000,0.000000}%
\pgfsetstrokecolor{currentstroke}%
\pgfsetdash{}{0pt}%
\pgfpathmoveto{\pgfqpoint{1.755359in}{1.418567in}}%
\pgfpathlineto{\pgfqpoint{1.777224in}{1.452347in}}%
\pgfusepath{stroke}%
\end{pgfscope}%
\begin{pgfscope}%
\pgfpathrectangle{\pgfqpoint{0.100000in}{0.212622in}}{\pgfqpoint{3.696000in}{3.696000in}}%
\pgfusepath{clip}%
\pgfsetrectcap%
\pgfsetroundjoin%
\pgfsetlinewidth{1.505625pt}%
\definecolor{currentstroke}{rgb}{1.000000,0.000000,0.000000}%
\pgfsetstrokecolor{currentstroke}%
\pgfsetdash{}{0pt}%
\pgfpathmoveto{\pgfqpoint{1.758167in}{1.417854in}}%
\pgfpathlineto{\pgfqpoint{1.777224in}{1.452347in}}%
\pgfusepath{stroke}%
\end{pgfscope}%
\begin{pgfscope}%
\pgfpathrectangle{\pgfqpoint{0.100000in}{0.212622in}}{\pgfqpoint{3.696000in}{3.696000in}}%
\pgfusepath{clip}%
\pgfsetrectcap%
\pgfsetroundjoin%
\pgfsetlinewidth{1.505625pt}%
\definecolor{currentstroke}{rgb}{1.000000,0.000000,0.000000}%
\pgfsetstrokecolor{currentstroke}%
\pgfsetdash{}{0pt}%
\pgfpathmoveto{\pgfqpoint{1.762054in}{1.416445in}}%
\pgfpathlineto{\pgfqpoint{1.777224in}{1.452347in}}%
\pgfusepath{stroke}%
\end{pgfscope}%
\begin{pgfscope}%
\pgfpathrectangle{\pgfqpoint{0.100000in}{0.212622in}}{\pgfqpoint{3.696000in}{3.696000in}}%
\pgfusepath{clip}%
\pgfsetrectcap%
\pgfsetroundjoin%
\pgfsetlinewidth{1.505625pt}%
\definecolor{currentstroke}{rgb}{1.000000,0.000000,0.000000}%
\pgfsetstrokecolor{currentstroke}%
\pgfsetdash{}{0pt}%
\pgfpathmoveto{\pgfqpoint{1.767058in}{1.415102in}}%
\pgfpathlineto{\pgfqpoint{1.791074in}{1.448097in}}%
\pgfusepath{stroke}%
\end{pgfscope}%
\begin{pgfscope}%
\pgfpathrectangle{\pgfqpoint{0.100000in}{0.212622in}}{\pgfqpoint{3.696000in}{3.696000in}}%
\pgfusepath{clip}%
\pgfsetrectcap%
\pgfsetroundjoin%
\pgfsetlinewidth{1.505625pt}%
\definecolor{currentstroke}{rgb}{1.000000,0.000000,0.000000}%
\pgfsetstrokecolor{currentstroke}%
\pgfsetdash{}{0pt}%
\pgfpathmoveto{\pgfqpoint{1.769722in}{1.414250in}}%
\pgfpathlineto{\pgfqpoint{1.791074in}{1.448097in}}%
\pgfusepath{stroke}%
\end{pgfscope}%
\begin{pgfscope}%
\pgfpathrectangle{\pgfqpoint{0.100000in}{0.212622in}}{\pgfqpoint{3.696000in}{3.696000in}}%
\pgfusepath{clip}%
\pgfsetrectcap%
\pgfsetroundjoin%
\pgfsetlinewidth{1.505625pt}%
\definecolor{currentstroke}{rgb}{1.000000,0.000000,0.000000}%
\pgfsetstrokecolor{currentstroke}%
\pgfsetdash{}{0pt}%
\pgfpathmoveto{\pgfqpoint{1.772662in}{1.413007in}}%
\pgfpathlineto{\pgfqpoint{1.791074in}{1.448097in}}%
\pgfusepath{stroke}%
\end{pgfscope}%
\begin{pgfscope}%
\pgfpathrectangle{\pgfqpoint{0.100000in}{0.212622in}}{\pgfqpoint{3.696000in}{3.696000in}}%
\pgfusepath{clip}%
\pgfsetrectcap%
\pgfsetroundjoin%
\pgfsetlinewidth{1.505625pt}%
\definecolor{currentstroke}{rgb}{1.000000,0.000000,0.000000}%
\pgfsetstrokecolor{currentstroke}%
\pgfsetdash{}{0pt}%
\pgfpathmoveto{\pgfqpoint{1.776761in}{1.411480in}}%
\pgfpathlineto{\pgfqpoint{1.804934in}{1.443844in}}%
\pgfusepath{stroke}%
\end{pgfscope}%
\begin{pgfscope}%
\pgfpathrectangle{\pgfqpoint{0.100000in}{0.212622in}}{\pgfqpoint{3.696000in}{3.696000in}}%
\pgfusepath{clip}%
\pgfsetrectcap%
\pgfsetroundjoin%
\pgfsetlinewidth{1.505625pt}%
\definecolor{currentstroke}{rgb}{1.000000,0.000000,0.000000}%
\pgfsetstrokecolor{currentstroke}%
\pgfsetdash{}{0pt}%
\pgfpathmoveto{\pgfqpoint{1.782952in}{1.410037in}}%
\pgfpathlineto{\pgfqpoint{1.804934in}{1.443844in}}%
\pgfusepath{stroke}%
\end{pgfscope}%
\begin{pgfscope}%
\pgfpathrectangle{\pgfqpoint{0.100000in}{0.212622in}}{\pgfqpoint{3.696000in}{3.696000in}}%
\pgfusepath{clip}%
\pgfsetrectcap%
\pgfsetroundjoin%
\pgfsetlinewidth{1.505625pt}%
\definecolor{currentstroke}{rgb}{1.000000,0.000000,0.000000}%
\pgfsetstrokecolor{currentstroke}%
\pgfsetdash{}{0pt}%
\pgfpathmoveto{\pgfqpoint{1.789768in}{1.407840in}}%
\pgfpathlineto{\pgfqpoint{1.804934in}{1.443844in}}%
\pgfusepath{stroke}%
\end{pgfscope}%
\begin{pgfscope}%
\pgfpathrectangle{\pgfqpoint{0.100000in}{0.212622in}}{\pgfqpoint{3.696000in}{3.696000in}}%
\pgfusepath{clip}%
\pgfsetrectcap%
\pgfsetroundjoin%
\pgfsetlinewidth{1.505625pt}%
\definecolor{currentstroke}{rgb}{1.000000,0.000000,0.000000}%
\pgfsetstrokecolor{currentstroke}%
\pgfsetdash{}{0pt}%
\pgfpathmoveto{\pgfqpoint{1.797053in}{1.405176in}}%
\pgfpathlineto{\pgfqpoint{1.818804in}{1.439588in}}%
\pgfusepath{stroke}%
\end{pgfscope}%
\begin{pgfscope}%
\pgfpathrectangle{\pgfqpoint{0.100000in}{0.212622in}}{\pgfqpoint{3.696000in}{3.696000in}}%
\pgfusepath{clip}%
\pgfsetrectcap%
\pgfsetroundjoin%
\pgfsetlinewidth{1.505625pt}%
\definecolor{currentstroke}{rgb}{1.000000,0.000000,0.000000}%
\pgfsetstrokecolor{currentstroke}%
\pgfsetdash{}{0pt}%
\pgfpathmoveto{\pgfqpoint{1.804539in}{1.402007in}}%
\pgfpathlineto{\pgfqpoint{1.832683in}{1.435329in}}%
\pgfusepath{stroke}%
\end{pgfscope}%
\begin{pgfscope}%
\pgfpathrectangle{\pgfqpoint{0.100000in}{0.212622in}}{\pgfqpoint{3.696000in}{3.696000in}}%
\pgfusepath{clip}%
\pgfsetrectcap%
\pgfsetroundjoin%
\pgfsetlinewidth{1.505625pt}%
\definecolor{currentstroke}{rgb}{1.000000,0.000000,0.000000}%
\pgfsetstrokecolor{currentstroke}%
\pgfsetdash{}{0pt}%
\pgfpathmoveto{\pgfqpoint{1.808720in}{1.400378in}}%
\pgfpathlineto{\pgfqpoint{1.832683in}{1.435329in}}%
\pgfusepath{stroke}%
\end{pgfscope}%
\begin{pgfscope}%
\pgfpathrectangle{\pgfqpoint{0.100000in}{0.212622in}}{\pgfqpoint{3.696000in}{3.696000in}}%
\pgfusepath{clip}%
\pgfsetrectcap%
\pgfsetroundjoin%
\pgfsetlinewidth{1.505625pt}%
\definecolor{currentstroke}{rgb}{1.000000,0.000000,0.000000}%
\pgfsetstrokecolor{currentstroke}%
\pgfsetdash{}{0pt}%
\pgfpathmoveto{\pgfqpoint{1.814740in}{1.398762in}}%
\pgfpathlineto{\pgfqpoint{1.832683in}{1.435329in}}%
\pgfusepath{stroke}%
\end{pgfscope}%
\begin{pgfscope}%
\pgfpathrectangle{\pgfqpoint{0.100000in}{0.212622in}}{\pgfqpoint{3.696000in}{3.696000in}}%
\pgfusepath{clip}%
\pgfsetrectcap%
\pgfsetroundjoin%
\pgfsetlinewidth{1.505625pt}%
\definecolor{currentstroke}{rgb}{1.000000,0.000000,0.000000}%
\pgfsetstrokecolor{currentstroke}%
\pgfsetdash{}{0pt}%
\pgfpathmoveto{\pgfqpoint{1.822685in}{1.397727in}}%
\pgfpathlineto{\pgfqpoint{1.846572in}{1.431067in}}%
\pgfusepath{stroke}%
\end{pgfscope}%
\begin{pgfscope}%
\pgfpathrectangle{\pgfqpoint{0.100000in}{0.212622in}}{\pgfqpoint{3.696000in}{3.696000in}}%
\pgfusepath{clip}%
\pgfsetrectcap%
\pgfsetroundjoin%
\pgfsetlinewidth{1.505625pt}%
\definecolor{currentstroke}{rgb}{1.000000,0.000000,0.000000}%
\pgfsetstrokecolor{currentstroke}%
\pgfsetdash{}{0pt}%
\pgfpathmoveto{\pgfqpoint{1.830797in}{1.394983in}}%
\pgfpathlineto{\pgfqpoint{1.846572in}{1.431067in}}%
\pgfusepath{stroke}%
\end{pgfscope}%
\begin{pgfscope}%
\pgfpathrectangle{\pgfqpoint{0.100000in}{0.212622in}}{\pgfqpoint{3.696000in}{3.696000in}}%
\pgfusepath{clip}%
\pgfsetrectcap%
\pgfsetroundjoin%
\pgfsetlinewidth{1.505625pt}%
\definecolor{currentstroke}{rgb}{1.000000,0.000000,0.000000}%
\pgfsetstrokecolor{currentstroke}%
\pgfsetdash{}{0pt}%
\pgfpathmoveto{\pgfqpoint{1.839952in}{1.392889in}}%
\pgfpathlineto{\pgfqpoint{1.860471in}{1.426803in}}%
\pgfusepath{stroke}%
\end{pgfscope}%
\begin{pgfscope}%
\pgfpathrectangle{\pgfqpoint{0.100000in}{0.212622in}}{\pgfqpoint{3.696000in}{3.696000in}}%
\pgfusepath{clip}%
\pgfsetrectcap%
\pgfsetroundjoin%
\pgfsetlinewidth{1.505625pt}%
\definecolor{currentstroke}{rgb}{1.000000,0.000000,0.000000}%
\pgfsetstrokecolor{currentstroke}%
\pgfsetdash{}{0pt}%
\pgfpathmoveto{\pgfqpoint{1.844981in}{1.391677in}}%
\pgfpathlineto{\pgfqpoint{1.860471in}{1.426803in}}%
\pgfusepath{stroke}%
\end{pgfscope}%
\begin{pgfscope}%
\pgfpathrectangle{\pgfqpoint{0.100000in}{0.212622in}}{\pgfqpoint{3.696000in}{3.696000in}}%
\pgfusepath{clip}%
\pgfsetrectcap%
\pgfsetroundjoin%
\pgfsetlinewidth{1.505625pt}%
\definecolor{currentstroke}{rgb}{1.000000,0.000000,0.000000}%
\pgfsetstrokecolor{currentstroke}%
\pgfsetdash{}{0pt}%
\pgfpathmoveto{\pgfqpoint{1.851298in}{1.389540in}}%
\pgfpathlineto{\pgfqpoint{1.874380in}{1.422535in}}%
\pgfusepath{stroke}%
\end{pgfscope}%
\begin{pgfscope}%
\pgfpathrectangle{\pgfqpoint{0.100000in}{0.212622in}}{\pgfqpoint{3.696000in}{3.696000in}}%
\pgfusepath{clip}%
\pgfsetrectcap%
\pgfsetroundjoin%
\pgfsetlinewidth{1.505625pt}%
\definecolor{currentstroke}{rgb}{1.000000,0.000000,0.000000}%
\pgfsetstrokecolor{currentstroke}%
\pgfsetdash{}{0pt}%
\pgfpathmoveto{\pgfqpoint{1.858997in}{1.387468in}}%
\pgfpathlineto{\pgfqpoint{1.874380in}{1.422535in}}%
\pgfusepath{stroke}%
\end{pgfscope}%
\begin{pgfscope}%
\pgfpathrectangle{\pgfqpoint{0.100000in}{0.212622in}}{\pgfqpoint{3.696000in}{3.696000in}}%
\pgfusepath{clip}%
\pgfsetrectcap%
\pgfsetroundjoin%
\pgfsetlinewidth{1.505625pt}%
\definecolor{currentstroke}{rgb}{1.000000,0.000000,0.000000}%
\pgfsetstrokecolor{currentstroke}%
\pgfsetdash{}{0pt}%
\pgfpathmoveto{\pgfqpoint{1.867034in}{1.384263in}}%
\pgfpathlineto{\pgfqpoint{1.888298in}{1.418264in}}%
\pgfusepath{stroke}%
\end{pgfscope}%
\begin{pgfscope}%
\pgfpathrectangle{\pgfqpoint{0.100000in}{0.212622in}}{\pgfqpoint{3.696000in}{3.696000in}}%
\pgfusepath{clip}%
\pgfsetrectcap%
\pgfsetroundjoin%
\pgfsetlinewidth{1.505625pt}%
\definecolor{currentstroke}{rgb}{1.000000,0.000000,0.000000}%
\pgfsetstrokecolor{currentstroke}%
\pgfsetdash{}{0pt}%
\pgfpathmoveto{\pgfqpoint{1.871606in}{1.382837in}}%
\pgfpathlineto{\pgfqpoint{1.888298in}{1.418264in}}%
\pgfusepath{stroke}%
\end{pgfscope}%
\begin{pgfscope}%
\pgfpathrectangle{\pgfqpoint{0.100000in}{0.212622in}}{\pgfqpoint{3.696000in}{3.696000in}}%
\pgfusepath{clip}%
\pgfsetrectcap%
\pgfsetroundjoin%
\pgfsetlinewidth{1.505625pt}%
\definecolor{currentstroke}{rgb}{1.000000,0.000000,0.000000}%
\pgfsetstrokecolor{currentstroke}%
\pgfsetdash{}{0pt}%
\pgfpathmoveto{\pgfqpoint{1.874206in}{1.382128in}}%
\pgfpathlineto{\pgfqpoint{1.888298in}{1.418264in}}%
\pgfusepath{stroke}%
\end{pgfscope}%
\begin{pgfscope}%
\pgfpathrectangle{\pgfqpoint{0.100000in}{0.212622in}}{\pgfqpoint{3.696000in}{3.696000in}}%
\pgfusepath{clip}%
\pgfsetrectcap%
\pgfsetroundjoin%
\pgfsetlinewidth{1.505625pt}%
\definecolor{currentstroke}{rgb}{1.000000,0.000000,0.000000}%
\pgfsetstrokecolor{currentstroke}%
\pgfsetdash{}{0pt}%
\pgfpathmoveto{\pgfqpoint{1.878035in}{1.380916in}}%
\pgfpathlineto{\pgfqpoint{1.902226in}{1.413991in}}%
\pgfusepath{stroke}%
\end{pgfscope}%
\begin{pgfscope}%
\pgfpathrectangle{\pgfqpoint{0.100000in}{0.212622in}}{\pgfqpoint{3.696000in}{3.696000in}}%
\pgfusepath{clip}%
\pgfsetrectcap%
\pgfsetroundjoin%
\pgfsetlinewidth{1.505625pt}%
\definecolor{currentstroke}{rgb}{1.000000,0.000000,0.000000}%
\pgfsetstrokecolor{currentstroke}%
\pgfsetdash{}{0pt}%
\pgfpathmoveto{\pgfqpoint{1.883690in}{1.379787in}}%
\pgfpathlineto{\pgfqpoint{1.902226in}{1.413991in}}%
\pgfusepath{stroke}%
\end{pgfscope}%
\begin{pgfscope}%
\pgfpathrectangle{\pgfqpoint{0.100000in}{0.212622in}}{\pgfqpoint{3.696000in}{3.696000in}}%
\pgfusepath{clip}%
\pgfsetrectcap%
\pgfsetroundjoin%
\pgfsetlinewidth{1.505625pt}%
\definecolor{currentstroke}{rgb}{1.000000,0.000000,0.000000}%
\pgfsetstrokecolor{currentstroke}%
\pgfsetdash{}{0pt}%
\pgfpathmoveto{\pgfqpoint{1.890218in}{1.377647in}}%
\pgfpathlineto{\pgfqpoint{1.916163in}{1.409714in}}%
\pgfusepath{stroke}%
\end{pgfscope}%
\begin{pgfscope}%
\pgfpathrectangle{\pgfqpoint{0.100000in}{0.212622in}}{\pgfqpoint{3.696000in}{3.696000in}}%
\pgfusepath{clip}%
\pgfsetrectcap%
\pgfsetroundjoin%
\pgfsetlinewidth{1.505625pt}%
\definecolor{currentstroke}{rgb}{1.000000,0.000000,0.000000}%
\pgfsetstrokecolor{currentstroke}%
\pgfsetdash{}{0pt}%
\pgfpathmoveto{\pgfqpoint{1.897722in}{1.376182in}}%
\pgfpathlineto{\pgfqpoint{1.916163in}{1.409714in}}%
\pgfusepath{stroke}%
\end{pgfscope}%
\begin{pgfscope}%
\pgfpathrectangle{\pgfqpoint{0.100000in}{0.212622in}}{\pgfqpoint{3.696000in}{3.696000in}}%
\pgfusepath{clip}%
\pgfsetrectcap%
\pgfsetroundjoin%
\pgfsetlinewidth{1.505625pt}%
\definecolor{currentstroke}{rgb}{1.000000,0.000000,0.000000}%
\pgfsetstrokecolor{currentstroke}%
\pgfsetdash{}{0pt}%
\pgfpathmoveto{\pgfqpoint{1.901772in}{1.375216in}}%
\pgfpathlineto{\pgfqpoint{1.916163in}{1.409714in}}%
\pgfusepath{stroke}%
\end{pgfscope}%
\begin{pgfscope}%
\pgfpathrectangle{\pgfqpoint{0.100000in}{0.212622in}}{\pgfqpoint{3.696000in}{3.696000in}}%
\pgfusepath{clip}%
\pgfsetrectcap%
\pgfsetroundjoin%
\pgfsetlinewidth{1.505625pt}%
\definecolor{currentstroke}{rgb}{1.000000,0.000000,0.000000}%
\pgfsetstrokecolor{currentstroke}%
\pgfsetdash{}{0pt}%
\pgfpathmoveto{\pgfqpoint{1.906957in}{1.373592in}}%
\pgfpathlineto{\pgfqpoint{1.930111in}{1.405434in}}%
\pgfusepath{stroke}%
\end{pgfscope}%
\begin{pgfscope}%
\pgfpathrectangle{\pgfqpoint{0.100000in}{0.212622in}}{\pgfqpoint{3.696000in}{3.696000in}}%
\pgfusepath{clip}%
\pgfsetrectcap%
\pgfsetroundjoin%
\pgfsetlinewidth{1.505625pt}%
\definecolor{currentstroke}{rgb}{1.000000,0.000000,0.000000}%
\pgfsetstrokecolor{currentstroke}%
\pgfsetdash{}{0pt}%
\pgfpathmoveto{\pgfqpoint{1.913538in}{1.371707in}}%
\pgfpathlineto{\pgfqpoint{1.930111in}{1.405434in}}%
\pgfusepath{stroke}%
\end{pgfscope}%
\begin{pgfscope}%
\pgfpathrectangle{\pgfqpoint{0.100000in}{0.212622in}}{\pgfqpoint{3.696000in}{3.696000in}}%
\pgfusepath{clip}%
\pgfsetrectcap%
\pgfsetroundjoin%
\pgfsetlinewidth{1.505625pt}%
\definecolor{currentstroke}{rgb}{1.000000,0.000000,0.000000}%
\pgfsetstrokecolor{currentstroke}%
\pgfsetdash{}{0pt}%
\pgfpathmoveto{\pgfqpoint{1.916868in}{1.370343in}}%
\pgfpathlineto{\pgfqpoint{1.930111in}{1.405434in}}%
\pgfusepath{stroke}%
\end{pgfscope}%
\begin{pgfscope}%
\pgfpathrectangle{\pgfqpoint{0.100000in}{0.212622in}}{\pgfqpoint{3.696000in}{3.696000in}}%
\pgfusepath{clip}%
\pgfsetrectcap%
\pgfsetroundjoin%
\pgfsetlinewidth{1.505625pt}%
\definecolor{currentstroke}{rgb}{1.000000,0.000000,0.000000}%
\pgfsetstrokecolor{currentstroke}%
\pgfsetdash{}{0pt}%
\pgfpathmoveto{\pgfqpoint{1.921077in}{1.369207in}}%
\pgfpathlineto{\pgfqpoint{1.944068in}{1.401152in}}%
\pgfusepath{stroke}%
\end{pgfscope}%
\begin{pgfscope}%
\pgfpathrectangle{\pgfqpoint{0.100000in}{0.212622in}}{\pgfqpoint{3.696000in}{3.696000in}}%
\pgfusepath{clip}%
\pgfsetrectcap%
\pgfsetroundjoin%
\pgfsetlinewidth{1.505625pt}%
\definecolor{currentstroke}{rgb}{1.000000,0.000000,0.000000}%
\pgfsetstrokecolor{currentstroke}%
\pgfsetdash{}{0pt}%
\pgfpathmoveto{\pgfqpoint{1.923416in}{1.368557in}}%
\pgfpathlineto{\pgfqpoint{1.944068in}{1.401152in}}%
\pgfusepath{stroke}%
\end{pgfscope}%
\begin{pgfscope}%
\pgfpathrectangle{\pgfqpoint{0.100000in}{0.212622in}}{\pgfqpoint{3.696000in}{3.696000in}}%
\pgfusepath{clip}%
\pgfsetrectcap%
\pgfsetroundjoin%
\pgfsetlinewidth{1.505625pt}%
\definecolor{currentstroke}{rgb}{1.000000,0.000000,0.000000}%
\pgfsetstrokecolor{currentstroke}%
\pgfsetdash{}{0pt}%
\pgfpathmoveto{\pgfqpoint{1.926685in}{1.367528in}}%
\pgfpathlineto{\pgfqpoint{1.944068in}{1.401152in}}%
\pgfusepath{stroke}%
\end{pgfscope}%
\begin{pgfscope}%
\pgfpathrectangle{\pgfqpoint{0.100000in}{0.212622in}}{\pgfqpoint{3.696000in}{3.696000in}}%
\pgfusepath{clip}%
\pgfsetrectcap%
\pgfsetroundjoin%
\pgfsetlinewidth{1.505625pt}%
\definecolor{currentstroke}{rgb}{1.000000,0.000000,0.000000}%
\pgfsetstrokecolor{currentstroke}%
\pgfsetdash{}{0pt}%
\pgfpathmoveto{\pgfqpoint{1.930961in}{1.365879in}}%
\pgfpathlineto{\pgfqpoint{1.944068in}{1.401152in}}%
\pgfusepath{stroke}%
\end{pgfscope}%
\begin{pgfscope}%
\pgfpathrectangle{\pgfqpoint{0.100000in}{0.212622in}}{\pgfqpoint{3.696000in}{3.696000in}}%
\pgfusepath{clip}%
\pgfsetrectcap%
\pgfsetroundjoin%
\pgfsetlinewidth{1.505625pt}%
\definecolor{currentstroke}{rgb}{1.000000,0.000000,0.000000}%
\pgfsetstrokecolor{currentstroke}%
\pgfsetdash{}{0pt}%
\pgfpathmoveto{\pgfqpoint{1.936904in}{1.364236in}}%
\pgfpathlineto{\pgfqpoint{1.958035in}{1.396866in}}%
\pgfusepath{stroke}%
\end{pgfscope}%
\begin{pgfscope}%
\pgfpathrectangle{\pgfqpoint{0.100000in}{0.212622in}}{\pgfqpoint{3.696000in}{3.696000in}}%
\pgfusepath{clip}%
\pgfsetrectcap%
\pgfsetroundjoin%
\pgfsetlinewidth{1.505625pt}%
\definecolor{currentstroke}{rgb}{1.000000,0.000000,0.000000}%
\pgfsetstrokecolor{currentstroke}%
\pgfsetdash{}{0pt}%
\pgfpathmoveto{\pgfqpoint{1.943421in}{1.362338in}}%
\pgfpathlineto{\pgfqpoint{1.958035in}{1.396866in}}%
\pgfusepath{stroke}%
\end{pgfscope}%
\begin{pgfscope}%
\pgfpathrectangle{\pgfqpoint{0.100000in}{0.212622in}}{\pgfqpoint{3.696000in}{3.696000in}}%
\pgfusepath{clip}%
\pgfsetrectcap%
\pgfsetroundjoin%
\pgfsetlinewidth{1.505625pt}%
\definecolor{currentstroke}{rgb}{1.000000,0.000000,0.000000}%
\pgfsetstrokecolor{currentstroke}%
\pgfsetdash{}{0pt}%
\pgfpathmoveto{\pgfqpoint{1.950046in}{1.359663in}}%
\pgfpathlineto{\pgfqpoint{1.972012in}{1.392577in}}%
\pgfusepath{stroke}%
\end{pgfscope}%
\begin{pgfscope}%
\pgfpathrectangle{\pgfqpoint{0.100000in}{0.212622in}}{\pgfqpoint{3.696000in}{3.696000in}}%
\pgfusepath{clip}%
\pgfsetrectcap%
\pgfsetroundjoin%
\pgfsetlinewidth{1.505625pt}%
\definecolor{currentstroke}{rgb}{1.000000,0.000000,0.000000}%
\pgfsetstrokecolor{currentstroke}%
\pgfsetdash{}{0pt}%
\pgfpathmoveto{\pgfqpoint{1.957185in}{1.356220in}}%
\pgfpathlineto{\pgfqpoint{1.972012in}{1.392577in}}%
\pgfusepath{stroke}%
\end{pgfscope}%
\begin{pgfscope}%
\pgfpathrectangle{\pgfqpoint{0.100000in}{0.212622in}}{\pgfqpoint{3.696000in}{3.696000in}}%
\pgfusepath{clip}%
\pgfsetrectcap%
\pgfsetroundjoin%
\pgfsetlinewidth{1.505625pt}%
\definecolor{currentstroke}{rgb}{1.000000,0.000000,0.000000}%
\pgfsetstrokecolor{currentstroke}%
\pgfsetdash{}{0pt}%
\pgfpathmoveto{\pgfqpoint{1.966026in}{1.352941in}}%
\pgfpathlineto{\pgfqpoint{1.985998in}{1.388286in}}%
\pgfusepath{stroke}%
\end{pgfscope}%
\begin{pgfscope}%
\pgfpathrectangle{\pgfqpoint{0.100000in}{0.212622in}}{\pgfqpoint{3.696000in}{3.696000in}}%
\pgfusepath{clip}%
\pgfsetrectcap%
\pgfsetroundjoin%
\pgfsetlinewidth{1.505625pt}%
\definecolor{currentstroke}{rgb}{1.000000,0.000000,0.000000}%
\pgfsetstrokecolor{currentstroke}%
\pgfsetdash{}{0pt}%
\pgfpathmoveto{\pgfqpoint{1.975820in}{1.349807in}}%
\pgfpathlineto{\pgfqpoint{1.999995in}{1.383991in}}%
\pgfusepath{stroke}%
\end{pgfscope}%
\begin{pgfscope}%
\pgfpathrectangle{\pgfqpoint{0.100000in}{0.212622in}}{\pgfqpoint{3.696000in}{3.696000in}}%
\pgfusepath{clip}%
\pgfsetrectcap%
\pgfsetroundjoin%
\pgfsetlinewidth{1.505625pt}%
\definecolor{currentstroke}{rgb}{1.000000,0.000000,0.000000}%
\pgfsetstrokecolor{currentstroke}%
\pgfsetdash{}{0pt}%
\pgfpathmoveto{\pgfqpoint{1.981244in}{1.348005in}}%
\pgfpathlineto{\pgfqpoint{1.999995in}{1.383991in}}%
\pgfusepath{stroke}%
\end{pgfscope}%
\begin{pgfscope}%
\pgfpathrectangle{\pgfqpoint{0.100000in}{0.212622in}}{\pgfqpoint{3.696000in}{3.696000in}}%
\pgfusepath{clip}%
\pgfsetrectcap%
\pgfsetroundjoin%
\pgfsetlinewidth{1.505625pt}%
\definecolor{currentstroke}{rgb}{1.000000,0.000000,0.000000}%
\pgfsetstrokecolor{currentstroke}%
\pgfsetdash{}{0pt}%
\pgfpathmoveto{\pgfqpoint{1.984205in}{1.346942in}}%
\pgfpathlineto{\pgfqpoint{1.999995in}{1.383991in}}%
\pgfusepath{stroke}%
\end{pgfscope}%
\begin{pgfscope}%
\pgfpathrectangle{\pgfqpoint{0.100000in}{0.212622in}}{\pgfqpoint{3.696000in}{3.696000in}}%
\pgfusepath{clip}%
\pgfsetrectcap%
\pgfsetroundjoin%
\pgfsetlinewidth{1.505625pt}%
\definecolor{currentstroke}{rgb}{1.000000,0.000000,0.000000}%
\pgfsetstrokecolor{currentstroke}%
\pgfsetdash{}{0pt}%
\pgfpathmoveto{\pgfqpoint{1.987511in}{1.345383in}}%
\pgfpathlineto{\pgfqpoint{2.014001in}{1.379693in}}%
\pgfusepath{stroke}%
\end{pgfscope}%
\begin{pgfscope}%
\pgfpathrectangle{\pgfqpoint{0.100000in}{0.212622in}}{\pgfqpoint{3.696000in}{3.696000in}}%
\pgfusepath{clip}%
\pgfsetrectcap%
\pgfsetroundjoin%
\pgfsetlinewidth{1.505625pt}%
\definecolor{currentstroke}{rgb}{1.000000,0.000000,0.000000}%
\pgfsetstrokecolor{currentstroke}%
\pgfsetdash{}{0pt}%
\pgfpathmoveto{\pgfqpoint{1.991875in}{1.343990in}}%
\pgfpathlineto{\pgfqpoint{2.014001in}{1.379693in}}%
\pgfusepath{stroke}%
\end{pgfscope}%
\begin{pgfscope}%
\pgfpathrectangle{\pgfqpoint{0.100000in}{0.212622in}}{\pgfqpoint{3.696000in}{3.696000in}}%
\pgfusepath{clip}%
\pgfsetrectcap%
\pgfsetroundjoin%
\pgfsetlinewidth{1.505625pt}%
\definecolor{currentstroke}{rgb}{1.000000,0.000000,0.000000}%
\pgfsetstrokecolor{currentstroke}%
\pgfsetdash{}{0pt}%
\pgfpathmoveto{\pgfqpoint{1.997870in}{1.342504in}}%
\pgfpathlineto{\pgfqpoint{2.014001in}{1.379693in}}%
\pgfusepath{stroke}%
\end{pgfscope}%
\begin{pgfscope}%
\pgfpathrectangle{\pgfqpoint{0.100000in}{0.212622in}}{\pgfqpoint{3.696000in}{3.696000in}}%
\pgfusepath{clip}%
\pgfsetrectcap%
\pgfsetroundjoin%
\pgfsetlinewidth{1.505625pt}%
\definecolor{currentstroke}{rgb}{1.000000,0.000000,0.000000}%
\pgfsetstrokecolor{currentstroke}%
\pgfsetdash{}{0pt}%
\pgfpathmoveto{\pgfqpoint{2.006122in}{1.341173in}}%
\pgfpathlineto{\pgfqpoint{2.028017in}{1.375392in}}%
\pgfusepath{stroke}%
\end{pgfscope}%
\begin{pgfscope}%
\pgfpathrectangle{\pgfqpoint{0.100000in}{0.212622in}}{\pgfqpoint{3.696000in}{3.696000in}}%
\pgfusepath{clip}%
\pgfsetrectcap%
\pgfsetroundjoin%
\pgfsetlinewidth{1.505625pt}%
\definecolor{currentstroke}{rgb}{1.000000,0.000000,0.000000}%
\pgfsetstrokecolor{currentstroke}%
\pgfsetdash{}{0pt}%
\pgfpathmoveto{\pgfqpoint{2.014355in}{1.339241in}}%
\pgfpathlineto{\pgfqpoint{2.028017in}{1.375392in}}%
\pgfusepath{stroke}%
\end{pgfscope}%
\begin{pgfscope}%
\pgfpathrectangle{\pgfqpoint{0.100000in}{0.212622in}}{\pgfqpoint{3.696000in}{3.696000in}}%
\pgfusepath{clip}%
\pgfsetrectcap%
\pgfsetroundjoin%
\pgfsetlinewidth{1.505625pt}%
\definecolor{currentstroke}{rgb}{1.000000,0.000000,0.000000}%
\pgfsetstrokecolor{currentstroke}%
\pgfsetdash{}{0pt}%
\pgfpathmoveto{\pgfqpoint{2.018651in}{1.337825in}}%
\pgfpathlineto{\pgfqpoint{2.042042in}{1.371089in}}%
\pgfusepath{stroke}%
\end{pgfscope}%
\begin{pgfscope}%
\pgfpathrectangle{\pgfqpoint{0.100000in}{0.212622in}}{\pgfqpoint{3.696000in}{3.696000in}}%
\pgfusepath{clip}%
\pgfsetrectcap%
\pgfsetroundjoin%
\pgfsetlinewidth{1.505625pt}%
\definecolor{currentstroke}{rgb}{1.000000,0.000000,0.000000}%
\pgfsetstrokecolor{currentstroke}%
\pgfsetdash{}{0pt}%
\pgfpathmoveto{\pgfqpoint{2.021145in}{1.337187in}}%
\pgfpathlineto{\pgfqpoint{2.042042in}{1.371089in}}%
\pgfusepath{stroke}%
\end{pgfscope}%
\begin{pgfscope}%
\pgfpathrectangle{\pgfqpoint{0.100000in}{0.212622in}}{\pgfqpoint{3.696000in}{3.696000in}}%
\pgfusepath{clip}%
\pgfsetrectcap%
\pgfsetroundjoin%
\pgfsetlinewidth{1.505625pt}%
\definecolor{currentstroke}{rgb}{1.000000,0.000000,0.000000}%
\pgfsetstrokecolor{currentstroke}%
\pgfsetdash{}{0pt}%
\pgfpathmoveto{\pgfqpoint{2.024514in}{1.335963in}}%
\pgfpathlineto{\pgfqpoint{2.042042in}{1.371089in}}%
\pgfusepath{stroke}%
\end{pgfscope}%
\begin{pgfscope}%
\pgfpathrectangle{\pgfqpoint{0.100000in}{0.212622in}}{\pgfqpoint{3.696000in}{3.696000in}}%
\pgfusepath{clip}%
\pgfsetrectcap%
\pgfsetroundjoin%
\pgfsetlinewidth{1.505625pt}%
\definecolor{currentstroke}{rgb}{1.000000,0.000000,0.000000}%
\pgfsetstrokecolor{currentstroke}%
\pgfsetdash{}{0pt}%
\pgfpathmoveto{\pgfqpoint{2.030169in}{1.334067in}}%
\pgfpathlineto{\pgfqpoint{2.056078in}{1.366782in}}%
\pgfusepath{stroke}%
\end{pgfscope}%
\begin{pgfscope}%
\pgfpathrectangle{\pgfqpoint{0.100000in}{0.212622in}}{\pgfqpoint{3.696000in}{3.696000in}}%
\pgfusepath{clip}%
\pgfsetrectcap%
\pgfsetroundjoin%
\pgfsetlinewidth{1.505625pt}%
\definecolor{currentstroke}{rgb}{1.000000,0.000000,0.000000}%
\pgfsetstrokecolor{currentstroke}%
\pgfsetdash{}{0pt}%
\pgfpathmoveto{\pgfqpoint{2.037097in}{1.331981in}}%
\pgfpathlineto{\pgfqpoint{2.056078in}{1.366782in}}%
\pgfusepath{stroke}%
\end{pgfscope}%
\begin{pgfscope}%
\pgfpathrectangle{\pgfqpoint{0.100000in}{0.212622in}}{\pgfqpoint{3.696000in}{3.696000in}}%
\pgfusepath{clip}%
\pgfsetrectcap%
\pgfsetroundjoin%
\pgfsetlinewidth{1.505625pt}%
\definecolor{currentstroke}{rgb}{1.000000,0.000000,0.000000}%
\pgfsetstrokecolor{currentstroke}%
\pgfsetdash{}{0pt}%
\pgfpathmoveto{\pgfqpoint{2.044351in}{1.329559in}}%
\pgfpathlineto{\pgfqpoint{2.070124in}{1.362472in}}%
\pgfusepath{stroke}%
\end{pgfscope}%
\begin{pgfscope}%
\pgfpathrectangle{\pgfqpoint{0.100000in}{0.212622in}}{\pgfqpoint{3.696000in}{3.696000in}}%
\pgfusepath{clip}%
\pgfsetrectcap%
\pgfsetroundjoin%
\pgfsetlinewidth{1.505625pt}%
\definecolor{currentstroke}{rgb}{1.000000,0.000000,0.000000}%
\pgfsetstrokecolor{currentstroke}%
\pgfsetdash{}{0pt}%
\pgfpathmoveto{\pgfqpoint{2.048276in}{1.328039in}}%
\pgfpathlineto{\pgfqpoint{2.070124in}{1.362472in}}%
\pgfusepath{stroke}%
\end{pgfscope}%
\begin{pgfscope}%
\pgfpathrectangle{\pgfqpoint{0.100000in}{0.212622in}}{\pgfqpoint{3.696000in}{3.696000in}}%
\pgfusepath{clip}%
\pgfsetrectcap%
\pgfsetroundjoin%
\pgfsetlinewidth{1.505625pt}%
\definecolor{currentstroke}{rgb}{1.000000,0.000000,0.000000}%
\pgfsetstrokecolor{currentstroke}%
\pgfsetdash{}{0pt}%
\pgfpathmoveto{\pgfqpoint{2.053228in}{1.326294in}}%
\pgfpathlineto{\pgfqpoint{2.070124in}{1.362472in}}%
\pgfusepath{stroke}%
\end{pgfscope}%
\begin{pgfscope}%
\pgfpathrectangle{\pgfqpoint{0.100000in}{0.212622in}}{\pgfqpoint{3.696000in}{3.696000in}}%
\pgfusepath{clip}%
\pgfsetrectcap%
\pgfsetroundjoin%
\pgfsetlinewidth{1.505625pt}%
\definecolor{currentstroke}{rgb}{1.000000,0.000000,0.000000}%
\pgfsetstrokecolor{currentstroke}%
\pgfsetdash{}{0pt}%
\pgfpathmoveto{\pgfqpoint{2.058771in}{1.324319in}}%
\pgfpathlineto{\pgfqpoint{2.084179in}{1.358159in}}%
\pgfusepath{stroke}%
\end{pgfscope}%
\begin{pgfscope}%
\pgfpathrectangle{\pgfqpoint{0.100000in}{0.212622in}}{\pgfqpoint{3.696000in}{3.696000in}}%
\pgfusepath{clip}%
\pgfsetrectcap%
\pgfsetroundjoin%
\pgfsetlinewidth{1.505625pt}%
\definecolor{currentstroke}{rgb}{1.000000,0.000000,0.000000}%
\pgfsetstrokecolor{currentstroke}%
\pgfsetdash{}{0pt}%
\pgfpathmoveto{\pgfqpoint{2.065539in}{1.322277in}}%
\pgfpathlineto{\pgfqpoint{2.084179in}{1.358159in}}%
\pgfusepath{stroke}%
\end{pgfscope}%
\begin{pgfscope}%
\pgfpathrectangle{\pgfqpoint{0.100000in}{0.212622in}}{\pgfqpoint{3.696000in}{3.696000in}}%
\pgfusepath{clip}%
\pgfsetrectcap%
\pgfsetroundjoin%
\pgfsetlinewidth{1.505625pt}%
\definecolor{currentstroke}{rgb}{1.000000,0.000000,0.000000}%
\pgfsetstrokecolor{currentstroke}%
\pgfsetdash{}{0pt}%
\pgfpathmoveto{\pgfqpoint{2.073269in}{1.320091in}}%
\pgfpathlineto{\pgfqpoint{2.098244in}{1.353844in}}%
\pgfusepath{stroke}%
\end{pgfscope}%
\begin{pgfscope}%
\pgfpathrectangle{\pgfqpoint{0.100000in}{0.212622in}}{\pgfqpoint{3.696000in}{3.696000in}}%
\pgfusepath{clip}%
\pgfsetrectcap%
\pgfsetroundjoin%
\pgfsetlinewidth{1.505625pt}%
\definecolor{currentstroke}{rgb}{1.000000,0.000000,0.000000}%
\pgfsetstrokecolor{currentstroke}%
\pgfsetdash{}{0pt}%
\pgfpathmoveto{\pgfqpoint{2.081111in}{1.317391in}}%
\pgfpathlineto{\pgfqpoint{2.098244in}{1.353844in}}%
\pgfusepath{stroke}%
\end{pgfscope}%
\begin{pgfscope}%
\pgfpathrectangle{\pgfqpoint{0.100000in}{0.212622in}}{\pgfqpoint{3.696000in}{3.696000in}}%
\pgfusepath{clip}%
\pgfsetrectcap%
\pgfsetroundjoin%
\pgfsetlinewidth{1.505625pt}%
\definecolor{currentstroke}{rgb}{1.000000,0.000000,0.000000}%
\pgfsetstrokecolor{currentstroke}%
\pgfsetdash{}{0pt}%
\pgfpathmoveto{\pgfqpoint{2.085253in}{1.315646in}}%
\pgfpathlineto{\pgfqpoint{2.112319in}{1.349525in}}%
\pgfusepath{stroke}%
\end{pgfscope}%
\begin{pgfscope}%
\pgfpathrectangle{\pgfqpoint{0.100000in}{0.212622in}}{\pgfqpoint{3.696000in}{3.696000in}}%
\pgfusepath{clip}%
\pgfsetrectcap%
\pgfsetroundjoin%
\pgfsetlinewidth{1.505625pt}%
\definecolor{currentstroke}{rgb}{1.000000,0.000000,0.000000}%
\pgfsetstrokecolor{currentstroke}%
\pgfsetdash{}{0pt}%
\pgfpathmoveto{\pgfqpoint{2.090604in}{1.313949in}}%
\pgfpathlineto{\pgfqpoint{2.112319in}{1.349525in}}%
\pgfusepath{stroke}%
\end{pgfscope}%
\begin{pgfscope}%
\pgfpathrectangle{\pgfqpoint{0.100000in}{0.212622in}}{\pgfqpoint{3.696000in}{3.696000in}}%
\pgfusepath{clip}%
\pgfsetrectcap%
\pgfsetroundjoin%
\pgfsetlinewidth{1.505625pt}%
\definecolor{currentstroke}{rgb}{1.000000,0.000000,0.000000}%
\pgfsetstrokecolor{currentstroke}%
\pgfsetdash{}{0pt}%
\pgfpathmoveto{\pgfqpoint{2.097110in}{1.312356in}}%
\pgfpathlineto{\pgfqpoint{2.112319in}{1.349525in}}%
\pgfusepath{stroke}%
\end{pgfscope}%
\begin{pgfscope}%
\pgfpathrectangle{\pgfqpoint{0.100000in}{0.212622in}}{\pgfqpoint{3.696000in}{3.696000in}}%
\pgfusepath{clip}%
\pgfsetrectcap%
\pgfsetroundjoin%
\pgfsetlinewidth{1.505625pt}%
\definecolor{currentstroke}{rgb}{1.000000,0.000000,0.000000}%
\pgfsetstrokecolor{currentstroke}%
\pgfsetdash{}{0pt}%
\pgfpathmoveto{\pgfqpoint{2.103804in}{1.309219in}}%
\pgfpathlineto{\pgfqpoint{2.126404in}{1.345203in}}%
\pgfusepath{stroke}%
\end{pgfscope}%
\begin{pgfscope}%
\pgfpathrectangle{\pgfqpoint{0.100000in}{0.212622in}}{\pgfqpoint{3.696000in}{3.696000in}}%
\pgfusepath{clip}%
\pgfsetrectcap%
\pgfsetroundjoin%
\pgfsetlinewidth{1.505625pt}%
\definecolor{currentstroke}{rgb}{1.000000,0.000000,0.000000}%
\pgfsetstrokecolor{currentstroke}%
\pgfsetdash{}{0pt}%
\pgfpathmoveto{\pgfqpoint{2.112114in}{1.306238in}}%
\pgfpathlineto{\pgfqpoint{2.126404in}{1.345203in}}%
\pgfusepath{stroke}%
\end{pgfscope}%
\begin{pgfscope}%
\pgfpathrectangle{\pgfqpoint{0.100000in}{0.212622in}}{\pgfqpoint{3.696000in}{3.696000in}}%
\pgfusepath{clip}%
\pgfsetrectcap%
\pgfsetroundjoin%
\pgfsetlinewidth{1.505625pt}%
\definecolor{currentstroke}{rgb}{1.000000,0.000000,0.000000}%
\pgfsetstrokecolor{currentstroke}%
\pgfsetdash{}{0pt}%
\pgfpathmoveto{\pgfqpoint{2.120827in}{1.302949in}}%
\pgfpathlineto{\pgfqpoint{2.140499in}{1.340878in}}%
\pgfusepath{stroke}%
\end{pgfscope}%
\begin{pgfscope}%
\pgfpathrectangle{\pgfqpoint{0.100000in}{0.212622in}}{\pgfqpoint{3.696000in}{3.696000in}}%
\pgfusepath{clip}%
\pgfsetrectcap%
\pgfsetroundjoin%
\pgfsetlinewidth{1.505625pt}%
\definecolor{currentstroke}{rgb}{1.000000,0.000000,0.000000}%
\pgfsetstrokecolor{currentstroke}%
\pgfsetdash{}{0pt}%
\pgfpathmoveto{\pgfqpoint{2.125679in}{1.301125in}}%
\pgfpathlineto{\pgfqpoint{2.140499in}{1.340878in}}%
\pgfusepath{stroke}%
\end{pgfscope}%
\begin{pgfscope}%
\pgfpathrectangle{\pgfqpoint{0.100000in}{0.212622in}}{\pgfqpoint{3.696000in}{3.696000in}}%
\pgfusepath{clip}%
\pgfsetrectcap%
\pgfsetroundjoin%
\pgfsetlinewidth{1.505625pt}%
\definecolor{currentstroke}{rgb}{1.000000,0.000000,0.000000}%
\pgfsetstrokecolor{currentstroke}%
\pgfsetdash{}{0pt}%
\pgfpathmoveto{\pgfqpoint{2.131674in}{1.299214in}}%
\pgfpathlineto{\pgfqpoint{2.154604in}{1.336550in}}%
\pgfusepath{stroke}%
\end{pgfscope}%
\begin{pgfscope}%
\pgfpathrectangle{\pgfqpoint{0.100000in}{0.212622in}}{\pgfqpoint{3.696000in}{3.696000in}}%
\pgfusepath{clip}%
\pgfsetrectcap%
\pgfsetroundjoin%
\pgfsetlinewidth{1.505625pt}%
\definecolor{currentstroke}{rgb}{1.000000,0.000000,0.000000}%
\pgfsetstrokecolor{currentstroke}%
\pgfsetdash{}{0pt}%
\pgfpathmoveto{\pgfqpoint{2.139256in}{1.297080in}}%
\pgfpathlineto{\pgfqpoint{2.154604in}{1.336550in}}%
\pgfusepath{stroke}%
\end{pgfscope}%
\begin{pgfscope}%
\pgfpathrectangle{\pgfqpoint{0.100000in}{0.212622in}}{\pgfqpoint{3.696000in}{3.696000in}}%
\pgfusepath{clip}%
\pgfsetrectcap%
\pgfsetroundjoin%
\pgfsetlinewidth{1.505625pt}%
\definecolor{currentstroke}{rgb}{1.000000,0.000000,0.000000}%
\pgfsetstrokecolor{currentstroke}%
\pgfsetdash{}{0pt}%
\pgfpathmoveto{\pgfqpoint{2.147499in}{1.294056in}}%
\pgfpathlineto{\pgfqpoint{2.168719in}{1.332219in}}%
\pgfusepath{stroke}%
\end{pgfscope}%
\begin{pgfscope}%
\pgfpathrectangle{\pgfqpoint{0.100000in}{0.212622in}}{\pgfqpoint{3.696000in}{3.696000in}}%
\pgfusepath{clip}%
\pgfsetrectcap%
\pgfsetroundjoin%
\pgfsetlinewidth{1.505625pt}%
\definecolor{currentstroke}{rgb}{1.000000,0.000000,0.000000}%
\pgfsetstrokecolor{currentstroke}%
\pgfsetdash{}{0pt}%
\pgfpathmoveto{\pgfqpoint{2.156308in}{1.290678in}}%
\pgfpathlineto{\pgfqpoint{2.182843in}{1.327885in}}%
\pgfusepath{stroke}%
\end{pgfscope}%
\begin{pgfscope}%
\pgfpathrectangle{\pgfqpoint{0.100000in}{0.212622in}}{\pgfqpoint{3.696000in}{3.696000in}}%
\pgfusepath{clip}%
\pgfsetrectcap%
\pgfsetroundjoin%
\pgfsetlinewidth{1.505625pt}%
\definecolor{currentstroke}{rgb}{1.000000,0.000000,0.000000}%
\pgfsetstrokecolor{currentstroke}%
\pgfsetdash{}{0pt}%
\pgfpathmoveto{\pgfqpoint{2.166235in}{1.287840in}}%
\pgfpathlineto{\pgfqpoint{2.182843in}{1.327885in}}%
\pgfusepath{stroke}%
\end{pgfscope}%
\begin{pgfscope}%
\pgfpathrectangle{\pgfqpoint{0.100000in}{0.212622in}}{\pgfqpoint{3.696000in}{3.696000in}}%
\pgfusepath{clip}%
\pgfsetrectcap%
\pgfsetroundjoin%
\pgfsetlinewidth{1.505625pt}%
\definecolor{currentstroke}{rgb}{1.000000,0.000000,0.000000}%
\pgfsetstrokecolor{currentstroke}%
\pgfsetdash{}{0pt}%
\pgfpathmoveto{\pgfqpoint{2.171509in}{1.285948in}}%
\pgfpathlineto{\pgfqpoint{2.196978in}{1.323548in}}%
\pgfusepath{stroke}%
\end{pgfscope}%
\begin{pgfscope}%
\pgfpathrectangle{\pgfqpoint{0.100000in}{0.212622in}}{\pgfqpoint{3.696000in}{3.696000in}}%
\pgfusepath{clip}%
\pgfsetrectcap%
\pgfsetroundjoin%
\pgfsetlinewidth{1.505625pt}%
\definecolor{currentstroke}{rgb}{1.000000,0.000000,0.000000}%
\pgfsetstrokecolor{currentstroke}%
\pgfsetdash{}{0pt}%
\pgfpathmoveto{\pgfqpoint{2.174701in}{1.285282in}}%
\pgfpathlineto{\pgfqpoint{2.196978in}{1.323548in}}%
\pgfusepath{stroke}%
\end{pgfscope}%
\begin{pgfscope}%
\pgfpathrectangle{\pgfqpoint{0.100000in}{0.212622in}}{\pgfqpoint{3.696000in}{3.696000in}}%
\pgfusepath{clip}%
\pgfsetrectcap%
\pgfsetroundjoin%
\pgfsetlinewidth{1.505625pt}%
\definecolor{currentstroke}{rgb}{1.000000,0.000000,0.000000}%
\pgfsetstrokecolor{currentstroke}%
\pgfsetdash{}{0pt}%
\pgfpathmoveto{\pgfqpoint{2.178689in}{1.284010in}}%
\pgfpathlineto{\pgfqpoint{2.196978in}{1.323548in}}%
\pgfusepath{stroke}%
\end{pgfscope}%
\begin{pgfscope}%
\pgfpathrectangle{\pgfqpoint{0.100000in}{0.212622in}}{\pgfqpoint{3.696000in}{3.696000in}}%
\pgfusepath{clip}%
\pgfsetrectcap%
\pgfsetroundjoin%
\pgfsetlinewidth{1.505625pt}%
\definecolor{currentstroke}{rgb}{1.000000,0.000000,0.000000}%
\pgfsetstrokecolor{currentstroke}%
\pgfsetdash{}{0pt}%
\pgfpathmoveto{\pgfqpoint{2.184289in}{1.282009in}}%
\pgfpathlineto{\pgfqpoint{2.211123in}{1.319208in}}%
\pgfusepath{stroke}%
\end{pgfscope}%
\begin{pgfscope}%
\pgfpathrectangle{\pgfqpoint{0.100000in}{0.212622in}}{\pgfqpoint{3.696000in}{3.696000in}}%
\pgfusepath{clip}%
\pgfsetrectcap%
\pgfsetroundjoin%
\pgfsetlinewidth{1.505625pt}%
\definecolor{currentstroke}{rgb}{1.000000,0.000000,0.000000}%
\pgfsetstrokecolor{currentstroke}%
\pgfsetdash{}{0pt}%
\pgfpathmoveto{\pgfqpoint{2.191928in}{1.280056in}}%
\pgfpathlineto{\pgfqpoint{2.211123in}{1.319208in}}%
\pgfusepath{stroke}%
\end{pgfscope}%
\begin{pgfscope}%
\pgfpathrectangle{\pgfqpoint{0.100000in}{0.212622in}}{\pgfqpoint{3.696000in}{3.696000in}}%
\pgfusepath{clip}%
\pgfsetrectcap%
\pgfsetroundjoin%
\pgfsetlinewidth{1.505625pt}%
\definecolor{currentstroke}{rgb}{1.000000,0.000000,0.000000}%
\pgfsetstrokecolor{currentstroke}%
\pgfsetdash{}{0pt}%
\pgfpathmoveto{\pgfqpoint{2.200281in}{1.277275in}}%
\pgfpathlineto{\pgfqpoint{2.225277in}{1.314864in}}%
\pgfusepath{stroke}%
\end{pgfscope}%
\begin{pgfscope}%
\pgfpathrectangle{\pgfqpoint{0.100000in}{0.212622in}}{\pgfqpoint{3.696000in}{3.696000in}}%
\pgfusepath{clip}%
\pgfsetrectcap%
\pgfsetroundjoin%
\pgfsetlinewidth{1.505625pt}%
\definecolor{currentstroke}{rgb}{1.000000,0.000000,0.000000}%
\pgfsetstrokecolor{currentstroke}%
\pgfsetdash{}{0pt}%
\pgfpathmoveto{\pgfqpoint{2.210073in}{1.275589in}}%
\pgfpathlineto{\pgfqpoint{2.225277in}{1.314864in}}%
\pgfusepath{stroke}%
\end{pgfscope}%
\begin{pgfscope}%
\pgfpathrectangle{\pgfqpoint{0.100000in}{0.212622in}}{\pgfqpoint{3.696000in}{3.696000in}}%
\pgfusepath{clip}%
\pgfsetrectcap%
\pgfsetroundjoin%
\pgfsetlinewidth{1.505625pt}%
\definecolor{currentstroke}{rgb}{1.000000,0.000000,0.000000}%
\pgfsetstrokecolor{currentstroke}%
\pgfsetdash{}{0pt}%
\pgfpathmoveto{\pgfqpoint{2.215143in}{1.274160in}}%
\pgfpathlineto{\pgfqpoint{2.239442in}{1.310518in}}%
\pgfusepath{stroke}%
\end{pgfscope}%
\begin{pgfscope}%
\pgfpathrectangle{\pgfqpoint{0.100000in}{0.212622in}}{\pgfqpoint{3.696000in}{3.696000in}}%
\pgfusepath{clip}%
\pgfsetrectcap%
\pgfsetroundjoin%
\pgfsetlinewidth{1.505625pt}%
\definecolor{currentstroke}{rgb}{1.000000,0.000000,0.000000}%
\pgfsetstrokecolor{currentstroke}%
\pgfsetdash{}{0pt}%
\pgfpathmoveto{\pgfqpoint{2.217981in}{1.273365in}}%
\pgfpathlineto{\pgfqpoint{2.239442in}{1.310518in}}%
\pgfusepath{stroke}%
\end{pgfscope}%
\begin{pgfscope}%
\pgfpathrectangle{\pgfqpoint{0.100000in}{0.212622in}}{\pgfqpoint{3.696000in}{3.696000in}}%
\pgfusepath{clip}%
\pgfsetrectcap%
\pgfsetroundjoin%
\pgfsetlinewidth{1.505625pt}%
\definecolor{currentstroke}{rgb}{1.000000,0.000000,0.000000}%
\pgfsetstrokecolor{currentstroke}%
\pgfsetdash{}{0pt}%
\pgfpathmoveto{\pgfqpoint{2.221658in}{1.272154in}}%
\pgfpathlineto{\pgfqpoint{2.239442in}{1.310518in}}%
\pgfusepath{stroke}%
\end{pgfscope}%
\begin{pgfscope}%
\pgfpathrectangle{\pgfqpoint{0.100000in}{0.212622in}}{\pgfqpoint{3.696000in}{3.696000in}}%
\pgfusepath{clip}%
\pgfsetrectcap%
\pgfsetroundjoin%
\pgfsetlinewidth{1.505625pt}%
\definecolor{currentstroke}{rgb}{1.000000,0.000000,0.000000}%
\pgfsetstrokecolor{currentstroke}%
\pgfsetdash{}{0pt}%
\pgfpathmoveto{\pgfqpoint{2.226833in}{1.270547in}}%
\pgfpathlineto{\pgfqpoint{2.253616in}{1.306169in}}%
\pgfusepath{stroke}%
\end{pgfscope}%
\begin{pgfscope}%
\pgfpathrectangle{\pgfqpoint{0.100000in}{0.212622in}}{\pgfqpoint{3.696000in}{3.696000in}}%
\pgfusepath{clip}%
\pgfsetrectcap%
\pgfsetroundjoin%
\pgfsetlinewidth{1.505625pt}%
\definecolor{currentstroke}{rgb}{1.000000,0.000000,0.000000}%
\pgfsetstrokecolor{currentstroke}%
\pgfsetdash{}{0pt}%
\pgfpathmoveto{\pgfqpoint{2.233157in}{1.268675in}}%
\pgfpathlineto{\pgfqpoint{2.253616in}{1.306169in}}%
\pgfusepath{stroke}%
\end{pgfscope}%
\begin{pgfscope}%
\pgfpathrectangle{\pgfqpoint{0.100000in}{0.212622in}}{\pgfqpoint{3.696000in}{3.696000in}}%
\pgfusepath{clip}%
\pgfsetrectcap%
\pgfsetroundjoin%
\pgfsetlinewidth{1.505625pt}%
\definecolor{currentstroke}{rgb}{1.000000,0.000000,0.000000}%
\pgfsetstrokecolor{currentstroke}%
\pgfsetdash{}{0pt}%
\pgfpathmoveto{\pgfqpoint{2.241203in}{1.267407in}}%
\pgfpathlineto{\pgfqpoint{2.267801in}{1.301816in}}%
\pgfusepath{stroke}%
\end{pgfscope}%
\begin{pgfscope}%
\pgfpathrectangle{\pgfqpoint{0.100000in}{0.212622in}}{\pgfqpoint{3.696000in}{3.696000in}}%
\pgfusepath{clip}%
\pgfsetrectcap%
\pgfsetroundjoin%
\pgfsetlinewidth{1.505625pt}%
\definecolor{currentstroke}{rgb}{1.000000,0.000000,0.000000}%
\pgfsetstrokecolor{currentstroke}%
\pgfsetdash{}{0pt}%
\pgfpathmoveto{\pgfqpoint{2.249254in}{1.264503in}}%
\pgfpathlineto{\pgfqpoint{2.267801in}{1.301816in}}%
\pgfusepath{stroke}%
\end{pgfscope}%
\begin{pgfscope}%
\pgfpathrectangle{\pgfqpoint{0.100000in}{0.212622in}}{\pgfqpoint{3.696000in}{3.696000in}}%
\pgfusepath{clip}%
\pgfsetrectcap%
\pgfsetroundjoin%
\pgfsetlinewidth{1.505625pt}%
\definecolor{currentstroke}{rgb}{1.000000,0.000000,0.000000}%
\pgfsetstrokecolor{currentstroke}%
\pgfsetdash{}{0pt}%
\pgfpathmoveto{\pgfqpoint{2.257920in}{1.261335in}}%
\pgfpathlineto{\pgfqpoint{2.281995in}{1.297461in}}%
\pgfusepath{stroke}%
\end{pgfscope}%
\begin{pgfscope}%
\pgfpathrectangle{\pgfqpoint{0.100000in}{0.212622in}}{\pgfqpoint{3.696000in}{3.696000in}}%
\pgfusepath{clip}%
\pgfsetrectcap%
\pgfsetroundjoin%
\pgfsetlinewidth{1.505625pt}%
\definecolor{currentstroke}{rgb}{1.000000,0.000000,0.000000}%
\pgfsetstrokecolor{currentstroke}%
\pgfsetdash{}{0pt}%
\pgfpathmoveto{\pgfqpoint{2.262783in}{1.259739in}}%
\pgfpathlineto{\pgfqpoint{2.281995in}{1.297461in}}%
\pgfusepath{stroke}%
\end{pgfscope}%
\begin{pgfscope}%
\pgfpathrectangle{\pgfqpoint{0.100000in}{0.212622in}}{\pgfqpoint{3.696000in}{3.696000in}}%
\pgfusepath{clip}%
\pgfsetrectcap%
\pgfsetroundjoin%
\pgfsetlinewidth{1.505625pt}%
\definecolor{currentstroke}{rgb}{1.000000,0.000000,0.000000}%
\pgfsetstrokecolor{currentstroke}%
\pgfsetdash{}{0pt}%
\pgfpathmoveto{\pgfqpoint{2.268784in}{1.258036in}}%
\pgfpathlineto{\pgfqpoint{2.281995in}{1.297461in}}%
\pgfusepath{stroke}%
\end{pgfscope}%
\begin{pgfscope}%
\pgfpathrectangle{\pgfqpoint{0.100000in}{0.212622in}}{\pgfqpoint{3.696000in}{3.696000in}}%
\pgfusepath{clip}%
\pgfsetrectcap%
\pgfsetroundjoin%
\pgfsetlinewidth{1.505625pt}%
\definecolor{currentstroke}{rgb}{1.000000,0.000000,0.000000}%
\pgfsetstrokecolor{currentstroke}%
\pgfsetdash{}{0pt}%
\pgfpathmoveto{\pgfqpoint{2.275635in}{1.255608in}}%
\pgfpathlineto{\pgfqpoint{2.296200in}{1.293102in}}%
\pgfusepath{stroke}%
\end{pgfscope}%
\begin{pgfscope}%
\pgfpathrectangle{\pgfqpoint{0.100000in}{0.212622in}}{\pgfqpoint{3.696000in}{3.696000in}}%
\pgfusepath{clip}%
\pgfsetrectcap%
\pgfsetroundjoin%
\pgfsetlinewidth{1.505625pt}%
\definecolor{currentstroke}{rgb}{1.000000,0.000000,0.000000}%
\pgfsetstrokecolor{currentstroke}%
\pgfsetdash{}{0pt}%
\pgfpathmoveto{\pgfqpoint{2.283768in}{1.253249in}}%
\pgfpathlineto{\pgfqpoint{2.310414in}{1.288741in}}%
\pgfusepath{stroke}%
\end{pgfscope}%
\begin{pgfscope}%
\pgfpathrectangle{\pgfqpoint{0.100000in}{0.212622in}}{\pgfqpoint{3.696000in}{3.696000in}}%
\pgfusepath{clip}%
\pgfsetrectcap%
\pgfsetroundjoin%
\pgfsetlinewidth{1.505625pt}%
\definecolor{currentstroke}{rgb}{1.000000,0.000000,0.000000}%
\pgfsetstrokecolor{currentstroke}%
\pgfsetdash{}{0pt}%
\pgfpathmoveto{\pgfqpoint{2.293716in}{1.251008in}}%
\pgfpathlineto{\pgfqpoint{2.310414in}{1.288741in}}%
\pgfusepath{stroke}%
\end{pgfscope}%
\begin{pgfscope}%
\pgfpathrectangle{\pgfqpoint{0.100000in}{0.212622in}}{\pgfqpoint{3.696000in}{3.696000in}}%
\pgfusepath{clip}%
\pgfsetrectcap%
\pgfsetroundjoin%
\pgfsetlinewidth{1.505625pt}%
\definecolor{currentstroke}{rgb}{1.000000,0.000000,0.000000}%
\pgfsetstrokecolor{currentstroke}%
\pgfsetdash{}{0pt}%
\pgfpathmoveto{\pgfqpoint{2.304100in}{1.248479in}}%
\pgfpathlineto{\pgfqpoint{2.324639in}{1.284376in}}%
\pgfusepath{stroke}%
\end{pgfscope}%
\begin{pgfscope}%
\pgfpathrectangle{\pgfqpoint{0.100000in}{0.212622in}}{\pgfqpoint{3.696000in}{3.696000in}}%
\pgfusepath{clip}%
\pgfsetrectcap%
\pgfsetroundjoin%
\pgfsetlinewidth{1.505625pt}%
\definecolor{currentstroke}{rgb}{1.000000,0.000000,0.000000}%
\pgfsetstrokecolor{currentstroke}%
\pgfsetdash{}{0pt}%
\pgfpathmoveto{\pgfqpoint{2.309710in}{1.247069in}}%
\pgfpathlineto{\pgfqpoint{2.324639in}{1.284376in}}%
\pgfusepath{stroke}%
\end{pgfscope}%
\begin{pgfscope}%
\pgfpathrectangle{\pgfqpoint{0.100000in}{0.212622in}}{\pgfqpoint{3.696000in}{3.696000in}}%
\pgfusepath{clip}%
\pgfsetrectcap%
\pgfsetroundjoin%
\pgfsetlinewidth{1.505625pt}%
\definecolor{currentstroke}{rgb}{1.000000,0.000000,0.000000}%
\pgfsetstrokecolor{currentstroke}%
\pgfsetdash{}{0pt}%
\pgfpathmoveto{\pgfqpoint{2.312650in}{1.246024in}}%
\pgfpathlineto{\pgfqpoint{2.324639in}{1.284376in}}%
\pgfusepath{stroke}%
\end{pgfscope}%
\begin{pgfscope}%
\pgfpathrectangle{\pgfqpoint{0.100000in}{0.212622in}}{\pgfqpoint{3.696000in}{3.696000in}}%
\pgfusepath{clip}%
\pgfsetrectcap%
\pgfsetroundjoin%
\pgfsetlinewidth{1.505625pt}%
\definecolor{currentstroke}{rgb}{1.000000,0.000000,0.000000}%
\pgfsetstrokecolor{currentstroke}%
\pgfsetdash{}{0pt}%
\pgfpathmoveto{\pgfqpoint{2.316309in}{1.244757in}}%
\pgfpathlineto{\pgfqpoint{2.338874in}{1.280008in}}%
\pgfusepath{stroke}%
\end{pgfscope}%
\begin{pgfscope}%
\pgfpathrectangle{\pgfqpoint{0.100000in}{0.212622in}}{\pgfqpoint{3.696000in}{3.696000in}}%
\pgfusepath{clip}%
\pgfsetrectcap%
\pgfsetroundjoin%
\pgfsetlinewidth{1.505625pt}%
\definecolor{currentstroke}{rgb}{1.000000,0.000000,0.000000}%
\pgfsetstrokecolor{currentstroke}%
\pgfsetdash{}{0pt}%
\pgfpathmoveto{\pgfqpoint{2.320623in}{1.243650in}}%
\pgfpathlineto{\pgfqpoint{2.338874in}{1.280008in}}%
\pgfusepath{stroke}%
\end{pgfscope}%
\begin{pgfscope}%
\pgfpathrectangle{\pgfqpoint{0.100000in}{0.212622in}}{\pgfqpoint{3.696000in}{3.696000in}}%
\pgfusepath{clip}%
\pgfsetrectcap%
\pgfsetroundjoin%
\pgfsetlinewidth{1.505625pt}%
\definecolor{currentstroke}{rgb}{1.000000,0.000000,0.000000}%
\pgfsetstrokecolor{currentstroke}%
\pgfsetdash{}{0pt}%
\pgfpathmoveto{\pgfqpoint{2.326371in}{1.241880in}}%
\pgfpathlineto{\pgfqpoint{2.338874in}{1.280008in}}%
\pgfusepath{stroke}%
\end{pgfscope}%
\begin{pgfscope}%
\pgfpathrectangle{\pgfqpoint{0.100000in}{0.212622in}}{\pgfqpoint{3.696000in}{3.696000in}}%
\pgfusepath{clip}%
\pgfsetrectcap%
\pgfsetroundjoin%
\pgfsetlinewidth{1.505625pt}%
\definecolor{currentstroke}{rgb}{1.000000,0.000000,0.000000}%
\pgfsetstrokecolor{currentstroke}%
\pgfsetdash{}{0pt}%
\pgfpathmoveto{\pgfqpoint{2.333345in}{1.239772in}}%
\pgfpathlineto{\pgfqpoint{2.353118in}{1.275637in}}%
\pgfusepath{stroke}%
\end{pgfscope}%
\begin{pgfscope}%
\pgfpathrectangle{\pgfqpoint{0.100000in}{0.212622in}}{\pgfqpoint{3.696000in}{3.696000in}}%
\pgfusepath{clip}%
\pgfsetrectcap%
\pgfsetroundjoin%
\pgfsetlinewidth{1.505625pt}%
\definecolor{currentstroke}{rgb}{1.000000,0.000000,0.000000}%
\pgfsetstrokecolor{currentstroke}%
\pgfsetdash{}{0pt}%
\pgfpathmoveto{\pgfqpoint{2.341534in}{1.237859in}}%
\pgfpathlineto{\pgfqpoint{2.353118in}{1.275637in}}%
\pgfusepath{stroke}%
\end{pgfscope}%
\begin{pgfscope}%
\pgfpathrectangle{\pgfqpoint{0.100000in}{0.212622in}}{\pgfqpoint{3.696000in}{3.696000in}}%
\pgfusepath{clip}%
\pgfsetrectcap%
\pgfsetroundjoin%
\pgfsetlinewidth{1.505625pt}%
\definecolor{currentstroke}{rgb}{1.000000,0.000000,0.000000}%
\pgfsetstrokecolor{currentstroke}%
\pgfsetdash{}{0pt}%
\pgfpathmoveto{\pgfqpoint{2.345603in}{1.236279in}}%
\pgfpathlineto{\pgfqpoint{2.367373in}{1.271263in}}%
\pgfusepath{stroke}%
\end{pgfscope}%
\begin{pgfscope}%
\pgfpathrectangle{\pgfqpoint{0.100000in}{0.212622in}}{\pgfqpoint{3.696000in}{3.696000in}}%
\pgfusepath{clip}%
\pgfsetrectcap%
\pgfsetroundjoin%
\pgfsetlinewidth{1.505625pt}%
\definecolor{currentstroke}{rgb}{1.000000,0.000000,0.000000}%
\pgfsetstrokecolor{currentstroke}%
\pgfsetdash{}{0pt}%
\pgfpathmoveto{\pgfqpoint{2.348065in}{1.235676in}}%
\pgfpathlineto{\pgfqpoint{2.367373in}{1.271263in}}%
\pgfusepath{stroke}%
\end{pgfscope}%
\begin{pgfscope}%
\pgfpathrectangle{\pgfqpoint{0.100000in}{0.212622in}}{\pgfqpoint{3.696000in}{3.696000in}}%
\pgfusepath{clip}%
\pgfsetrectcap%
\pgfsetroundjoin%
\pgfsetlinewidth{1.505625pt}%
\definecolor{currentstroke}{rgb}{1.000000,0.000000,0.000000}%
\pgfsetstrokecolor{currentstroke}%
\pgfsetdash{}{0pt}%
\pgfpathmoveto{\pgfqpoint{2.349361in}{1.235275in}}%
\pgfpathlineto{\pgfqpoint{2.367373in}{1.271263in}}%
\pgfusepath{stroke}%
\end{pgfscope}%
\begin{pgfscope}%
\pgfpathrectangle{\pgfqpoint{0.100000in}{0.212622in}}{\pgfqpoint{3.696000in}{3.696000in}}%
\pgfusepath{clip}%
\pgfsetrectcap%
\pgfsetroundjoin%
\pgfsetlinewidth{1.505625pt}%
\definecolor{currentstroke}{rgb}{1.000000,0.000000,0.000000}%
\pgfsetstrokecolor{currentstroke}%
\pgfsetdash{}{0pt}%
\pgfpathmoveto{\pgfqpoint{2.350101in}{1.235085in}}%
\pgfpathlineto{\pgfqpoint{2.367373in}{1.271263in}}%
\pgfusepath{stroke}%
\end{pgfscope}%
\begin{pgfscope}%
\pgfpathrectangle{\pgfqpoint{0.100000in}{0.212622in}}{\pgfqpoint{3.696000in}{3.696000in}}%
\pgfusepath{clip}%
\pgfsetrectcap%
\pgfsetroundjoin%
\pgfsetlinewidth{1.505625pt}%
\definecolor{currentstroke}{rgb}{1.000000,0.000000,0.000000}%
\pgfsetstrokecolor{currentstroke}%
\pgfsetdash{}{0pt}%
\pgfpathmoveto{\pgfqpoint{2.351900in}{1.234510in}}%
\pgfpathlineto{\pgfqpoint{2.367373in}{1.271263in}}%
\pgfusepath{stroke}%
\end{pgfscope}%
\begin{pgfscope}%
\pgfpathrectangle{\pgfqpoint{0.100000in}{0.212622in}}{\pgfqpoint{3.696000in}{3.696000in}}%
\pgfusepath{clip}%
\pgfsetrectcap%
\pgfsetroundjoin%
\pgfsetlinewidth{1.505625pt}%
\definecolor{currentstroke}{rgb}{1.000000,0.000000,0.000000}%
\pgfsetstrokecolor{currentstroke}%
\pgfsetdash{}{0pt}%
\pgfpathmoveto{\pgfqpoint{2.355221in}{1.233352in}}%
\pgfpathlineto{\pgfqpoint{2.367373in}{1.271263in}}%
\pgfusepath{stroke}%
\end{pgfscope}%
\begin{pgfscope}%
\pgfpathrectangle{\pgfqpoint{0.100000in}{0.212622in}}{\pgfqpoint{3.696000in}{3.696000in}}%
\pgfusepath{clip}%
\pgfsetrectcap%
\pgfsetroundjoin%
\pgfsetlinewidth{1.505625pt}%
\definecolor{currentstroke}{rgb}{1.000000,0.000000,0.000000}%
\pgfsetstrokecolor{currentstroke}%
\pgfsetdash{}{0pt}%
\pgfpathmoveto{\pgfqpoint{2.359683in}{1.232044in}}%
\pgfpathlineto{\pgfqpoint{2.381638in}{1.266886in}}%
\pgfusepath{stroke}%
\end{pgfscope}%
\begin{pgfscope}%
\pgfpathrectangle{\pgfqpoint{0.100000in}{0.212622in}}{\pgfqpoint{3.696000in}{3.696000in}}%
\pgfusepath{clip}%
\pgfsetrectcap%
\pgfsetroundjoin%
\pgfsetlinewidth{1.505625pt}%
\definecolor{currentstroke}{rgb}{1.000000,0.000000,0.000000}%
\pgfsetstrokecolor{currentstroke}%
\pgfsetdash{}{0pt}%
\pgfpathmoveto{\pgfqpoint{2.362185in}{1.231391in}}%
\pgfpathlineto{\pgfqpoint{2.381638in}{1.266886in}}%
\pgfusepath{stroke}%
\end{pgfscope}%
\begin{pgfscope}%
\pgfpathrectangle{\pgfqpoint{0.100000in}{0.212622in}}{\pgfqpoint{3.696000in}{3.696000in}}%
\pgfusepath{clip}%
\pgfsetrectcap%
\pgfsetroundjoin%
\pgfsetlinewidth{1.505625pt}%
\definecolor{currentstroke}{rgb}{1.000000,0.000000,0.000000}%
\pgfsetstrokecolor{currentstroke}%
\pgfsetdash{}{0pt}%
\pgfpathmoveto{\pgfqpoint{2.366216in}{1.230511in}}%
\pgfpathlineto{\pgfqpoint{2.381638in}{1.266886in}}%
\pgfusepath{stroke}%
\end{pgfscope}%
\begin{pgfscope}%
\pgfpathrectangle{\pgfqpoint{0.100000in}{0.212622in}}{\pgfqpoint{3.696000in}{3.696000in}}%
\pgfusepath{clip}%
\pgfsetrectcap%
\pgfsetroundjoin%
\pgfsetlinewidth{1.505625pt}%
\definecolor{currentstroke}{rgb}{1.000000,0.000000,0.000000}%
\pgfsetstrokecolor{currentstroke}%
\pgfsetdash{}{0pt}%
\pgfpathmoveto{\pgfqpoint{2.370825in}{1.229400in}}%
\pgfpathlineto{\pgfqpoint{2.381638in}{1.266886in}}%
\pgfusepath{stroke}%
\end{pgfscope}%
\begin{pgfscope}%
\pgfpathrectangle{\pgfqpoint{0.100000in}{0.212622in}}{\pgfqpoint{3.696000in}{3.696000in}}%
\pgfusepath{clip}%
\pgfsetrectcap%
\pgfsetroundjoin%
\pgfsetlinewidth{1.505625pt}%
\definecolor{currentstroke}{rgb}{1.000000,0.000000,0.000000}%
\pgfsetstrokecolor{currentstroke}%
\pgfsetdash{}{0pt}%
\pgfpathmoveto{\pgfqpoint{2.375837in}{1.228082in}}%
\pgfpathlineto{\pgfqpoint{2.395913in}{1.262506in}}%
\pgfusepath{stroke}%
\end{pgfscope}%
\begin{pgfscope}%
\pgfpathrectangle{\pgfqpoint{0.100000in}{0.212622in}}{\pgfqpoint{3.696000in}{3.696000in}}%
\pgfusepath{clip}%
\pgfsetrectcap%
\pgfsetroundjoin%
\pgfsetlinewidth{1.505625pt}%
\definecolor{currentstroke}{rgb}{1.000000,0.000000,0.000000}%
\pgfsetstrokecolor{currentstroke}%
\pgfsetdash{}{0pt}%
\pgfpathmoveto{\pgfqpoint{2.378645in}{1.227426in}}%
\pgfpathlineto{\pgfqpoint{2.395913in}{1.262506in}}%
\pgfusepath{stroke}%
\end{pgfscope}%
\begin{pgfscope}%
\pgfpathrectangle{\pgfqpoint{0.100000in}{0.212622in}}{\pgfqpoint{3.696000in}{3.696000in}}%
\pgfusepath{clip}%
\pgfsetrectcap%
\pgfsetroundjoin%
\pgfsetlinewidth{1.505625pt}%
\definecolor{currentstroke}{rgb}{1.000000,0.000000,0.000000}%
\pgfsetstrokecolor{currentstroke}%
\pgfsetdash{}{0pt}%
\pgfpathmoveto{\pgfqpoint{2.380153in}{1.227015in}}%
\pgfpathlineto{\pgfqpoint{2.395913in}{1.262506in}}%
\pgfusepath{stroke}%
\end{pgfscope}%
\begin{pgfscope}%
\pgfpathrectangle{\pgfqpoint{0.100000in}{0.212622in}}{\pgfqpoint{3.696000in}{3.696000in}}%
\pgfusepath{clip}%
\pgfsetrectcap%
\pgfsetroundjoin%
\pgfsetlinewidth{1.505625pt}%
\definecolor{currentstroke}{rgb}{1.000000,0.000000,0.000000}%
\pgfsetstrokecolor{currentstroke}%
\pgfsetdash{}{0pt}%
\pgfpathmoveto{\pgfqpoint{2.382475in}{1.226096in}}%
\pgfpathlineto{\pgfqpoint{2.395913in}{1.262506in}}%
\pgfusepath{stroke}%
\end{pgfscope}%
\begin{pgfscope}%
\pgfpathrectangle{\pgfqpoint{0.100000in}{0.212622in}}{\pgfqpoint{3.696000in}{3.696000in}}%
\pgfusepath{clip}%
\pgfsetrectcap%
\pgfsetroundjoin%
\pgfsetlinewidth{1.505625pt}%
\definecolor{currentstroke}{rgb}{1.000000,0.000000,0.000000}%
\pgfsetstrokecolor{currentstroke}%
\pgfsetdash{}{0pt}%
\pgfpathmoveto{\pgfqpoint{2.385689in}{1.224891in}}%
\pgfpathlineto{\pgfqpoint{2.395913in}{1.262506in}}%
\pgfusepath{stroke}%
\end{pgfscope}%
\begin{pgfscope}%
\pgfpathrectangle{\pgfqpoint{0.100000in}{0.212622in}}{\pgfqpoint{3.696000in}{3.696000in}}%
\pgfusepath{clip}%
\pgfsetrectcap%
\pgfsetroundjoin%
\pgfsetlinewidth{1.505625pt}%
\definecolor{currentstroke}{rgb}{1.000000,0.000000,0.000000}%
\pgfsetstrokecolor{currentstroke}%
\pgfsetdash{}{0pt}%
\pgfpathmoveto{\pgfqpoint{2.390232in}{1.223742in}}%
\pgfpathlineto{\pgfqpoint{2.410198in}{1.258123in}}%
\pgfusepath{stroke}%
\end{pgfscope}%
\begin{pgfscope}%
\pgfpathrectangle{\pgfqpoint{0.100000in}{0.212622in}}{\pgfqpoint{3.696000in}{3.696000in}}%
\pgfusepath{clip}%
\pgfsetrectcap%
\pgfsetroundjoin%
\pgfsetlinewidth{1.505625pt}%
\definecolor{currentstroke}{rgb}{1.000000,0.000000,0.000000}%
\pgfsetstrokecolor{currentstroke}%
\pgfsetdash{}{0pt}%
\pgfpathmoveto{\pgfqpoint{2.396122in}{1.222029in}}%
\pgfpathlineto{\pgfqpoint{2.410198in}{1.258123in}}%
\pgfusepath{stroke}%
\end{pgfscope}%
\begin{pgfscope}%
\pgfpathrectangle{\pgfqpoint{0.100000in}{0.212622in}}{\pgfqpoint{3.696000in}{3.696000in}}%
\pgfusepath{clip}%
\pgfsetrectcap%
\pgfsetroundjoin%
\pgfsetlinewidth{1.505625pt}%
\definecolor{currentstroke}{rgb}{1.000000,0.000000,0.000000}%
\pgfsetstrokecolor{currentstroke}%
\pgfsetdash{}{0pt}%
\pgfpathmoveto{\pgfqpoint{2.402811in}{1.220385in}}%
\pgfpathlineto{\pgfqpoint{2.424493in}{1.253736in}}%
\pgfusepath{stroke}%
\end{pgfscope}%
\begin{pgfscope}%
\pgfpathrectangle{\pgfqpoint{0.100000in}{0.212622in}}{\pgfqpoint{3.696000in}{3.696000in}}%
\pgfusepath{clip}%
\pgfsetrectcap%
\pgfsetroundjoin%
\pgfsetlinewidth{1.505625pt}%
\definecolor{currentstroke}{rgb}{1.000000,0.000000,0.000000}%
\pgfsetstrokecolor{currentstroke}%
\pgfsetdash{}{0pt}%
\pgfpathmoveto{\pgfqpoint{2.406328in}{1.219299in}}%
\pgfpathlineto{\pgfqpoint{2.424493in}{1.253736in}}%
\pgfusepath{stroke}%
\end{pgfscope}%
\begin{pgfscope}%
\pgfpathrectangle{\pgfqpoint{0.100000in}{0.212622in}}{\pgfqpoint{3.696000in}{3.696000in}}%
\pgfusepath{clip}%
\pgfsetrectcap%
\pgfsetroundjoin%
\pgfsetlinewidth{1.505625pt}%
\definecolor{currentstroke}{rgb}{1.000000,0.000000,0.000000}%
\pgfsetstrokecolor{currentstroke}%
\pgfsetdash{}{0pt}%
\pgfpathmoveto{\pgfqpoint{2.410263in}{1.217809in}}%
\pgfpathlineto{\pgfqpoint{2.424493in}{1.253736in}}%
\pgfusepath{stroke}%
\end{pgfscope}%
\begin{pgfscope}%
\pgfpathrectangle{\pgfqpoint{0.100000in}{0.212622in}}{\pgfqpoint{3.696000in}{3.696000in}}%
\pgfusepath{clip}%
\pgfsetrectcap%
\pgfsetroundjoin%
\pgfsetlinewidth{1.505625pt}%
\definecolor{currentstroke}{rgb}{1.000000,0.000000,0.000000}%
\pgfsetstrokecolor{currentstroke}%
\pgfsetdash{}{0pt}%
\pgfpathmoveto{\pgfqpoint{2.415013in}{1.215928in}}%
\pgfpathlineto{\pgfqpoint{2.438799in}{1.249347in}}%
\pgfusepath{stroke}%
\end{pgfscope}%
\begin{pgfscope}%
\pgfpathrectangle{\pgfqpoint{0.100000in}{0.212622in}}{\pgfqpoint{3.696000in}{3.696000in}}%
\pgfusepath{clip}%
\pgfsetrectcap%
\pgfsetroundjoin%
\pgfsetlinewidth{1.505625pt}%
\definecolor{currentstroke}{rgb}{1.000000,0.000000,0.000000}%
\pgfsetstrokecolor{currentstroke}%
\pgfsetdash{}{0pt}%
\pgfpathmoveto{\pgfqpoint{2.417901in}{1.215273in}}%
\pgfpathlineto{\pgfqpoint{2.438799in}{1.249347in}}%
\pgfusepath{stroke}%
\end{pgfscope}%
\begin{pgfscope}%
\pgfpathrectangle{\pgfqpoint{0.100000in}{0.212622in}}{\pgfqpoint{3.696000in}{3.696000in}}%
\pgfusepath{clip}%
\pgfsetrectcap%
\pgfsetroundjoin%
\pgfsetlinewidth{1.505625pt}%
\definecolor{currentstroke}{rgb}{1.000000,0.000000,0.000000}%
\pgfsetstrokecolor{currentstroke}%
\pgfsetdash{}{0pt}%
\pgfpathmoveto{\pgfqpoint{2.422191in}{1.214240in}}%
\pgfpathlineto{\pgfqpoint{2.438799in}{1.249347in}}%
\pgfusepath{stroke}%
\end{pgfscope}%
\begin{pgfscope}%
\pgfpathrectangle{\pgfqpoint{0.100000in}{0.212622in}}{\pgfqpoint{3.696000in}{3.696000in}}%
\pgfusepath{clip}%
\pgfsetrectcap%
\pgfsetroundjoin%
\pgfsetlinewidth{1.505625pt}%
\definecolor{currentstroke}{rgb}{1.000000,0.000000,0.000000}%
\pgfsetstrokecolor{currentstroke}%
\pgfsetdash{}{0pt}%
\pgfpathmoveto{\pgfqpoint{2.428171in}{1.212053in}}%
\pgfpathlineto{\pgfqpoint{2.438799in}{1.249347in}}%
\pgfusepath{stroke}%
\end{pgfscope}%
\begin{pgfscope}%
\pgfpathrectangle{\pgfqpoint{0.100000in}{0.212622in}}{\pgfqpoint{3.696000in}{3.696000in}}%
\pgfusepath{clip}%
\pgfsetrectcap%
\pgfsetroundjoin%
\pgfsetlinewidth{1.505625pt}%
\definecolor{currentstroke}{rgb}{1.000000,0.000000,0.000000}%
\pgfsetstrokecolor{currentstroke}%
\pgfsetdash{}{0pt}%
\pgfpathmoveto{\pgfqpoint{2.435806in}{1.210274in}}%
\pgfpathlineto{\pgfqpoint{2.453114in}{1.244954in}}%
\pgfusepath{stroke}%
\end{pgfscope}%
\begin{pgfscope}%
\pgfpathrectangle{\pgfqpoint{0.100000in}{0.212622in}}{\pgfqpoint{3.696000in}{3.696000in}}%
\pgfusepath{clip}%
\pgfsetrectcap%
\pgfsetroundjoin%
\pgfsetlinewidth{1.505625pt}%
\definecolor{currentstroke}{rgb}{1.000000,0.000000,0.000000}%
\pgfsetstrokecolor{currentstroke}%
\pgfsetdash{}{0pt}%
\pgfpathmoveto{\pgfqpoint{2.439885in}{1.209178in}}%
\pgfpathlineto{\pgfqpoint{2.453114in}{1.244954in}}%
\pgfusepath{stroke}%
\end{pgfscope}%
\begin{pgfscope}%
\pgfpathrectangle{\pgfqpoint{0.100000in}{0.212622in}}{\pgfqpoint{3.696000in}{3.696000in}}%
\pgfusepath{clip}%
\pgfsetrectcap%
\pgfsetroundjoin%
\pgfsetlinewidth{1.505625pt}%
\definecolor{currentstroke}{rgb}{1.000000,0.000000,0.000000}%
\pgfsetstrokecolor{currentstroke}%
\pgfsetdash{}{0pt}%
\pgfpathmoveto{\pgfqpoint{2.442133in}{1.208551in}}%
\pgfpathlineto{\pgfqpoint{2.453114in}{1.244954in}}%
\pgfusepath{stroke}%
\end{pgfscope}%
\begin{pgfscope}%
\pgfpathrectangle{\pgfqpoint{0.100000in}{0.212622in}}{\pgfqpoint{3.696000in}{3.696000in}}%
\pgfusepath{clip}%
\pgfsetrectcap%
\pgfsetroundjoin%
\pgfsetlinewidth{1.505625pt}%
\definecolor{currentstroke}{rgb}{1.000000,0.000000,0.000000}%
\pgfsetstrokecolor{currentstroke}%
\pgfsetdash{}{0pt}%
\pgfpathmoveto{\pgfqpoint{2.443294in}{1.208105in}}%
\pgfpathlineto{\pgfqpoint{2.453114in}{1.244954in}}%
\pgfusepath{stroke}%
\end{pgfscope}%
\begin{pgfscope}%
\pgfpathrectangle{\pgfqpoint{0.100000in}{0.212622in}}{\pgfqpoint{3.696000in}{3.696000in}}%
\pgfusepath{clip}%
\pgfsetrectcap%
\pgfsetroundjoin%
\pgfsetlinewidth{1.505625pt}%
\definecolor{currentstroke}{rgb}{1.000000,0.000000,0.000000}%
\pgfsetstrokecolor{currentstroke}%
\pgfsetdash{}{0pt}%
\pgfpathmoveto{\pgfqpoint{2.445469in}{1.207650in}}%
\pgfpathlineto{\pgfqpoint{2.467440in}{1.240558in}}%
\pgfusepath{stroke}%
\end{pgfscope}%
\begin{pgfscope}%
\pgfpathrectangle{\pgfqpoint{0.100000in}{0.212622in}}{\pgfqpoint{3.696000in}{3.696000in}}%
\pgfusepath{clip}%
\pgfsetrectcap%
\pgfsetroundjoin%
\pgfsetlinewidth{1.505625pt}%
\definecolor{currentstroke}{rgb}{1.000000,0.000000,0.000000}%
\pgfsetstrokecolor{currentstroke}%
\pgfsetdash{}{0pt}%
\pgfpathmoveto{\pgfqpoint{2.448726in}{1.206600in}}%
\pgfpathlineto{\pgfqpoint{2.467440in}{1.240558in}}%
\pgfusepath{stroke}%
\end{pgfscope}%
\begin{pgfscope}%
\pgfpathrectangle{\pgfqpoint{0.100000in}{0.212622in}}{\pgfqpoint{3.696000in}{3.696000in}}%
\pgfusepath{clip}%
\pgfsetrectcap%
\pgfsetroundjoin%
\pgfsetlinewidth{1.505625pt}%
\definecolor{currentstroke}{rgb}{1.000000,0.000000,0.000000}%
\pgfsetstrokecolor{currentstroke}%
\pgfsetdash{}{0pt}%
\pgfpathmoveto{\pgfqpoint{2.453882in}{1.205042in}}%
\pgfpathlineto{\pgfqpoint{2.467440in}{1.240558in}}%
\pgfusepath{stroke}%
\end{pgfscope}%
\begin{pgfscope}%
\pgfpathrectangle{\pgfqpoint{0.100000in}{0.212622in}}{\pgfqpoint{3.696000in}{3.696000in}}%
\pgfusepath{clip}%
\pgfsetrectcap%
\pgfsetroundjoin%
\pgfsetlinewidth{1.505625pt}%
\definecolor{currentstroke}{rgb}{1.000000,0.000000,0.000000}%
\pgfsetstrokecolor{currentstroke}%
\pgfsetdash{}{0pt}%
\pgfpathmoveto{\pgfqpoint{2.460385in}{1.202965in}}%
\pgfpathlineto{\pgfqpoint{2.481776in}{1.236159in}}%
\pgfusepath{stroke}%
\end{pgfscope}%
\begin{pgfscope}%
\pgfpathrectangle{\pgfqpoint{0.100000in}{0.212622in}}{\pgfqpoint{3.696000in}{3.696000in}}%
\pgfusepath{clip}%
\pgfsetrectcap%
\pgfsetroundjoin%
\pgfsetlinewidth{1.505625pt}%
\definecolor{currentstroke}{rgb}{1.000000,0.000000,0.000000}%
\pgfsetstrokecolor{currentstroke}%
\pgfsetdash{}{0pt}%
\pgfpathmoveto{\pgfqpoint{2.467462in}{1.200438in}}%
\pgfpathlineto{\pgfqpoint{2.481776in}{1.236159in}}%
\pgfusepath{stroke}%
\end{pgfscope}%
\begin{pgfscope}%
\pgfpathrectangle{\pgfqpoint{0.100000in}{0.212622in}}{\pgfqpoint{3.696000in}{3.696000in}}%
\pgfusepath{clip}%
\pgfsetrectcap%
\pgfsetroundjoin%
\pgfsetlinewidth{1.505625pt}%
\definecolor{currentstroke}{rgb}{1.000000,0.000000,0.000000}%
\pgfsetstrokecolor{currentstroke}%
\pgfsetdash{}{0pt}%
\pgfpathmoveto{\pgfqpoint{2.471619in}{1.199265in}}%
\pgfpathlineto{\pgfqpoint{2.481776in}{1.236159in}}%
\pgfusepath{stroke}%
\end{pgfscope}%
\begin{pgfscope}%
\pgfpathrectangle{\pgfqpoint{0.100000in}{0.212622in}}{\pgfqpoint{3.696000in}{3.696000in}}%
\pgfusepath{clip}%
\pgfsetrectcap%
\pgfsetroundjoin%
\pgfsetlinewidth{1.505625pt}%
\definecolor{currentstroke}{rgb}{1.000000,0.000000,0.000000}%
\pgfsetstrokecolor{currentstroke}%
\pgfsetdash{}{0pt}%
\pgfpathmoveto{\pgfqpoint{2.473826in}{1.198551in}}%
\pgfpathlineto{\pgfqpoint{2.496122in}{1.231757in}}%
\pgfusepath{stroke}%
\end{pgfscope}%
\begin{pgfscope}%
\pgfpathrectangle{\pgfqpoint{0.100000in}{0.212622in}}{\pgfqpoint{3.696000in}{3.696000in}}%
\pgfusepath{clip}%
\pgfsetrectcap%
\pgfsetroundjoin%
\pgfsetlinewidth{1.505625pt}%
\definecolor{currentstroke}{rgb}{1.000000,0.000000,0.000000}%
\pgfsetstrokecolor{currentstroke}%
\pgfsetdash{}{0pt}%
\pgfpathmoveto{\pgfqpoint{2.474977in}{1.198117in}}%
\pgfpathlineto{\pgfqpoint{2.496122in}{1.231757in}}%
\pgfusepath{stroke}%
\end{pgfscope}%
\begin{pgfscope}%
\pgfpathrectangle{\pgfqpoint{0.100000in}{0.212622in}}{\pgfqpoint{3.696000in}{3.696000in}}%
\pgfusepath{clip}%
\pgfsetrectcap%
\pgfsetroundjoin%
\pgfsetlinewidth{1.505625pt}%
\definecolor{currentstroke}{rgb}{1.000000,0.000000,0.000000}%
\pgfsetstrokecolor{currentstroke}%
\pgfsetdash{}{0pt}%
\pgfpathmoveto{\pgfqpoint{2.477704in}{1.197374in}}%
\pgfpathlineto{\pgfqpoint{2.496122in}{1.231757in}}%
\pgfusepath{stroke}%
\end{pgfscope}%
\begin{pgfscope}%
\pgfpathrectangle{\pgfqpoint{0.100000in}{0.212622in}}{\pgfqpoint{3.696000in}{3.696000in}}%
\pgfusepath{clip}%
\pgfsetrectcap%
\pgfsetroundjoin%
\pgfsetlinewidth{1.505625pt}%
\definecolor{currentstroke}{rgb}{1.000000,0.000000,0.000000}%
\pgfsetstrokecolor{currentstroke}%
\pgfsetdash{}{0pt}%
\pgfpathmoveto{\pgfqpoint{2.482889in}{1.195677in}}%
\pgfpathlineto{\pgfqpoint{2.496122in}{1.231757in}}%
\pgfusepath{stroke}%
\end{pgfscope}%
\begin{pgfscope}%
\pgfpathrectangle{\pgfqpoint{0.100000in}{0.212622in}}{\pgfqpoint{3.696000in}{3.696000in}}%
\pgfusepath{clip}%
\pgfsetrectcap%
\pgfsetroundjoin%
\pgfsetlinewidth{1.505625pt}%
\definecolor{currentstroke}{rgb}{1.000000,0.000000,0.000000}%
\pgfsetstrokecolor{currentstroke}%
\pgfsetdash{}{0pt}%
\pgfpathmoveto{\pgfqpoint{2.490247in}{1.193559in}}%
\pgfpathlineto{\pgfqpoint{2.510478in}{1.227352in}}%
\pgfusepath{stroke}%
\end{pgfscope}%
\begin{pgfscope}%
\pgfpathrectangle{\pgfqpoint{0.100000in}{0.212622in}}{\pgfqpoint{3.696000in}{3.696000in}}%
\pgfusepath{clip}%
\pgfsetrectcap%
\pgfsetroundjoin%
\pgfsetlinewidth{1.505625pt}%
\definecolor{currentstroke}{rgb}{1.000000,0.000000,0.000000}%
\pgfsetstrokecolor{currentstroke}%
\pgfsetdash{}{0pt}%
\pgfpathmoveto{\pgfqpoint{2.498296in}{1.191309in}}%
\pgfpathlineto{\pgfqpoint{2.510478in}{1.227352in}}%
\pgfusepath{stroke}%
\end{pgfscope}%
\begin{pgfscope}%
\pgfpathrectangle{\pgfqpoint{0.100000in}{0.212622in}}{\pgfqpoint{3.696000in}{3.696000in}}%
\pgfusepath{clip}%
\pgfsetrectcap%
\pgfsetroundjoin%
\pgfsetlinewidth{1.505625pt}%
\definecolor{currentstroke}{rgb}{1.000000,0.000000,0.000000}%
\pgfsetstrokecolor{currentstroke}%
\pgfsetdash{}{0pt}%
\pgfpathmoveto{\pgfqpoint{2.502768in}{1.190073in}}%
\pgfpathlineto{\pgfqpoint{2.510478in}{1.227352in}}%
\pgfusepath{stroke}%
\end{pgfscope}%
\begin{pgfscope}%
\pgfpathrectangle{\pgfqpoint{0.100000in}{0.212622in}}{\pgfqpoint{3.696000in}{3.696000in}}%
\pgfusepath{clip}%
\pgfsetrectcap%
\pgfsetroundjoin%
\pgfsetlinewidth{1.505625pt}%
\definecolor{currentstroke}{rgb}{1.000000,0.000000,0.000000}%
\pgfsetstrokecolor{currentstroke}%
\pgfsetdash{}{0pt}%
\pgfpathmoveto{\pgfqpoint{2.505161in}{1.189350in}}%
\pgfpathlineto{\pgfqpoint{2.510478in}{1.227352in}}%
\pgfusepath{stroke}%
\end{pgfscope}%
\begin{pgfscope}%
\pgfpathrectangle{\pgfqpoint{0.100000in}{0.212622in}}{\pgfqpoint{3.696000in}{3.696000in}}%
\pgfusepath{clip}%
\pgfsetrectcap%
\pgfsetroundjoin%
\pgfsetlinewidth{1.505625pt}%
\definecolor{currentstroke}{rgb}{1.000000,0.000000,0.000000}%
\pgfsetstrokecolor{currentstroke}%
\pgfsetdash{}{0pt}%
\pgfpathmoveto{\pgfqpoint{2.508013in}{1.188472in}}%
\pgfpathlineto{\pgfqpoint{2.510478in}{1.227352in}}%
\pgfusepath{stroke}%
\end{pgfscope}%
\begin{pgfscope}%
\pgfpathrectangle{\pgfqpoint{0.100000in}{0.212622in}}{\pgfqpoint{3.696000in}{3.696000in}}%
\pgfusepath{clip}%
\pgfsetrectcap%
\pgfsetroundjoin%
\pgfsetlinewidth{1.505625pt}%
\definecolor{currentstroke}{rgb}{1.000000,0.000000,0.000000}%
\pgfsetstrokecolor{currentstroke}%
\pgfsetdash{}{0pt}%
\pgfpathmoveto{\pgfqpoint{2.511885in}{1.187102in}}%
\pgfpathlineto{\pgfqpoint{2.510478in}{1.227352in}}%
\pgfusepath{stroke}%
\end{pgfscope}%
\begin{pgfscope}%
\pgfpathrectangle{\pgfqpoint{0.100000in}{0.212622in}}{\pgfqpoint{3.696000in}{3.696000in}}%
\pgfusepath{clip}%
\pgfsetrectcap%
\pgfsetroundjoin%
\pgfsetlinewidth{1.505625pt}%
\definecolor{currentstroke}{rgb}{1.000000,0.000000,0.000000}%
\pgfsetstrokecolor{currentstroke}%
\pgfsetdash{}{0pt}%
\pgfpathmoveto{\pgfqpoint{2.516413in}{1.185209in}}%
\pgfpathlineto{\pgfqpoint{2.510478in}{1.227352in}}%
\pgfusepath{stroke}%
\end{pgfscope}%
\begin{pgfscope}%
\pgfpathrectangle{\pgfqpoint{0.100000in}{0.212622in}}{\pgfqpoint{3.696000in}{3.696000in}}%
\pgfusepath{clip}%
\pgfsetrectcap%
\pgfsetroundjoin%
\pgfsetlinewidth{1.505625pt}%
\definecolor{currentstroke}{rgb}{1.000000,0.000000,0.000000}%
\pgfsetstrokecolor{currentstroke}%
\pgfsetdash{}{0pt}%
\pgfpathmoveto{\pgfqpoint{2.522546in}{1.183542in}}%
\pgfpathlineto{\pgfqpoint{2.510478in}{1.227352in}}%
\pgfusepath{stroke}%
\end{pgfscope}%
\begin{pgfscope}%
\pgfpathrectangle{\pgfqpoint{0.100000in}{0.212622in}}{\pgfqpoint{3.696000in}{3.696000in}}%
\pgfusepath{clip}%
\pgfsetrectcap%
\pgfsetroundjoin%
\pgfsetlinewidth{1.505625pt}%
\definecolor{currentstroke}{rgb}{1.000000,0.000000,0.000000}%
\pgfsetstrokecolor{currentstroke}%
\pgfsetdash{}{0pt}%
\pgfpathmoveto{\pgfqpoint{2.530022in}{1.181328in}}%
\pgfpathlineto{\pgfqpoint{2.510478in}{1.227352in}}%
\pgfusepath{stroke}%
\end{pgfscope}%
\begin{pgfscope}%
\pgfpathrectangle{\pgfqpoint{0.100000in}{0.212622in}}{\pgfqpoint{3.696000in}{3.696000in}}%
\pgfusepath{clip}%
\pgfsetrectcap%
\pgfsetroundjoin%
\pgfsetlinewidth{1.505625pt}%
\definecolor{currentstroke}{rgb}{1.000000,0.000000,0.000000}%
\pgfsetstrokecolor{currentstroke}%
\pgfsetdash{}{0pt}%
\pgfpathmoveto{\pgfqpoint{2.539084in}{1.178714in}}%
\pgfpathlineto{\pgfqpoint{2.510478in}{1.227352in}}%
\pgfusepath{stroke}%
\end{pgfscope}%
\begin{pgfscope}%
\pgfpathrectangle{\pgfqpoint{0.100000in}{0.212622in}}{\pgfqpoint{3.696000in}{3.696000in}}%
\pgfusepath{clip}%
\pgfsetrectcap%
\pgfsetroundjoin%
\pgfsetlinewidth{1.505625pt}%
\definecolor{currentstroke}{rgb}{1.000000,0.000000,0.000000}%
\pgfsetstrokecolor{currentstroke}%
\pgfsetdash{}{0pt}%
\pgfpathmoveto{\pgfqpoint{2.549298in}{1.175838in}}%
\pgfpathlineto{\pgfqpoint{2.510478in}{1.227352in}}%
\pgfusepath{stroke}%
\end{pgfscope}%
\begin{pgfscope}%
\pgfpathrectangle{\pgfqpoint{0.100000in}{0.212622in}}{\pgfqpoint{3.696000in}{3.696000in}}%
\pgfusepath{clip}%
\pgfsetrectcap%
\pgfsetroundjoin%
\pgfsetlinewidth{1.505625pt}%
\definecolor{currentstroke}{rgb}{1.000000,0.000000,0.000000}%
\pgfsetstrokecolor{currentstroke}%
\pgfsetdash{}{0pt}%
\pgfpathmoveto{\pgfqpoint{2.554740in}{1.173904in}}%
\pgfpathlineto{\pgfqpoint{2.510478in}{1.227352in}}%
\pgfusepath{stroke}%
\end{pgfscope}%
\begin{pgfscope}%
\pgfpathrectangle{\pgfqpoint{0.100000in}{0.212622in}}{\pgfqpoint{3.696000in}{3.696000in}}%
\pgfusepath{clip}%
\pgfsetrectcap%
\pgfsetroundjoin%
\pgfsetlinewidth{1.505625pt}%
\definecolor{currentstroke}{rgb}{1.000000,0.000000,0.000000}%
\pgfsetstrokecolor{currentstroke}%
\pgfsetdash{}{0pt}%
\pgfpathmoveto{\pgfqpoint{2.560479in}{1.171629in}}%
\pgfpathlineto{\pgfqpoint{2.510478in}{1.227352in}}%
\pgfusepath{stroke}%
\end{pgfscope}%
\begin{pgfscope}%
\pgfpathrectangle{\pgfqpoint{0.100000in}{0.212622in}}{\pgfqpoint{3.696000in}{3.696000in}}%
\pgfusepath{clip}%
\pgfsetrectcap%
\pgfsetroundjoin%
\pgfsetlinewidth{1.505625pt}%
\definecolor{currentstroke}{rgb}{1.000000,0.000000,0.000000}%
\pgfsetstrokecolor{currentstroke}%
\pgfsetdash{}{0pt}%
\pgfpathmoveto{\pgfqpoint{2.567448in}{1.169513in}}%
\pgfpathlineto{\pgfqpoint{2.510478in}{1.227352in}}%
\pgfusepath{stroke}%
\end{pgfscope}%
\begin{pgfscope}%
\pgfpathrectangle{\pgfqpoint{0.100000in}{0.212622in}}{\pgfqpoint{3.696000in}{3.696000in}}%
\pgfusepath{clip}%
\pgfsetrectcap%
\pgfsetroundjoin%
\pgfsetlinewidth{1.505625pt}%
\definecolor{currentstroke}{rgb}{1.000000,0.000000,0.000000}%
\pgfsetstrokecolor{currentstroke}%
\pgfsetdash{}{0pt}%
\pgfpathmoveto{\pgfqpoint{2.575062in}{1.166490in}}%
\pgfpathlineto{\pgfqpoint{2.510478in}{1.227352in}}%
\pgfusepath{stroke}%
\end{pgfscope}%
\begin{pgfscope}%
\pgfpathrectangle{\pgfqpoint{0.100000in}{0.212622in}}{\pgfqpoint{3.696000in}{3.696000in}}%
\pgfusepath{clip}%
\pgfsetrectcap%
\pgfsetroundjoin%
\pgfsetlinewidth{1.505625pt}%
\definecolor{currentstroke}{rgb}{1.000000,0.000000,0.000000}%
\pgfsetstrokecolor{currentstroke}%
\pgfsetdash{}{0pt}%
\pgfpathmoveto{\pgfqpoint{2.584396in}{1.162586in}}%
\pgfpathlineto{\pgfqpoint{2.510478in}{1.227352in}}%
\pgfusepath{stroke}%
\end{pgfscope}%
\begin{pgfscope}%
\pgfpathrectangle{\pgfqpoint{0.100000in}{0.212622in}}{\pgfqpoint{3.696000in}{3.696000in}}%
\pgfusepath{clip}%
\pgfsetrectcap%
\pgfsetroundjoin%
\pgfsetlinewidth{1.505625pt}%
\definecolor{currentstroke}{rgb}{1.000000,0.000000,0.000000}%
\pgfsetstrokecolor{currentstroke}%
\pgfsetdash{}{0pt}%
\pgfpathmoveto{\pgfqpoint{2.596381in}{1.159556in}}%
\pgfpathlineto{\pgfqpoint{2.510478in}{1.227352in}}%
\pgfusepath{stroke}%
\end{pgfscope}%
\begin{pgfscope}%
\pgfpathrectangle{\pgfqpoint{0.100000in}{0.212622in}}{\pgfqpoint{3.696000in}{3.696000in}}%
\pgfusepath{clip}%
\pgfsetrectcap%
\pgfsetroundjoin%
\pgfsetlinewidth{1.505625pt}%
\definecolor{currentstroke}{rgb}{1.000000,0.000000,0.000000}%
\pgfsetstrokecolor{currentstroke}%
\pgfsetdash{}{0pt}%
\pgfpathmoveto{\pgfqpoint{2.609261in}{1.155817in}}%
\pgfpathlineto{\pgfqpoint{2.510478in}{1.227352in}}%
\pgfusepath{stroke}%
\end{pgfscope}%
\begin{pgfscope}%
\pgfpathrectangle{\pgfqpoint{0.100000in}{0.212622in}}{\pgfqpoint{3.696000in}{3.696000in}}%
\pgfusepath{clip}%
\pgfsetrectcap%
\pgfsetroundjoin%
\pgfsetlinewidth{1.505625pt}%
\definecolor{currentstroke}{rgb}{1.000000,0.000000,0.000000}%
\pgfsetstrokecolor{currentstroke}%
\pgfsetdash{}{0pt}%
\pgfpathmoveto{\pgfqpoint{2.623050in}{1.152356in}}%
\pgfpathlineto{\pgfqpoint{2.510478in}{1.227352in}}%
\pgfusepath{stroke}%
\end{pgfscope}%
\begin{pgfscope}%
\pgfpathrectangle{\pgfqpoint{0.100000in}{0.212622in}}{\pgfqpoint{3.696000in}{3.696000in}}%
\pgfusepath{clip}%
\pgfsetrectcap%
\pgfsetroundjoin%
\pgfsetlinewidth{1.505625pt}%
\definecolor{currentstroke}{rgb}{1.000000,0.000000,0.000000}%
\pgfsetstrokecolor{currentstroke}%
\pgfsetdash{}{0pt}%
\pgfpathmoveto{\pgfqpoint{2.638258in}{1.149397in}}%
\pgfpathlineto{\pgfqpoint{2.510478in}{1.227352in}}%
\pgfusepath{stroke}%
\end{pgfscope}%
\begin{pgfscope}%
\pgfpathrectangle{\pgfqpoint{0.100000in}{0.212622in}}{\pgfqpoint{3.696000in}{3.696000in}}%
\pgfusepath{clip}%
\pgfsetrectcap%
\pgfsetroundjoin%
\pgfsetlinewidth{1.505625pt}%
\definecolor{currentstroke}{rgb}{1.000000,0.000000,0.000000}%
\pgfsetstrokecolor{currentstroke}%
\pgfsetdash{}{0pt}%
\pgfpathmoveto{\pgfqpoint{2.653376in}{1.145142in}}%
\pgfpathlineto{\pgfqpoint{2.510478in}{1.227352in}}%
\pgfusepath{stroke}%
\end{pgfscope}%
\begin{pgfscope}%
\pgfpathrectangle{\pgfqpoint{0.100000in}{0.212622in}}{\pgfqpoint{3.696000in}{3.696000in}}%
\pgfusepath{clip}%
\pgfsetrectcap%
\pgfsetroundjoin%
\pgfsetlinewidth{1.505625pt}%
\definecolor{currentstroke}{rgb}{1.000000,0.000000,0.000000}%
\pgfsetstrokecolor{currentstroke}%
\pgfsetdash{}{0pt}%
\pgfpathmoveto{\pgfqpoint{2.661491in}{1.142878in}}%
\pgfpathlineto{\pgfqpoint{2.510478in}{1.227352in}}%
\pgfusepath{stroke}%
\end{pgfscope}%
\begin{pgfscope}%
\pgfpathrectangle{\pgfqpoint{0.100000in}{0.212622in}}{\pgfqpoint{3.696000in}{3.696000in}}%
\pgfusepath{clip}%
\pgfsetrectcap%
\pgfsetroundjoin%
\pgfsetlinewidth{1.505625pt}%
\definecolor{currentstroke}{rgb}{1.000000,0.000000,0.000000}%
\pgfsetstrokecolor{currentstroke}%
\pgfsetdash{}{0pt}%
\pgfpathmoveto{\pgfqpoint{2.665919in}{1.141384in}}%
\pgfpathlineto{\pgfqpoint{2.510478in}{1.227352in}}%
\pgfusepath{stroke}%
\end{pgfscope}%
\begin{pgfscope}%
\pgfpathrectangle{\pgfqpoint{0.100000in}{0.212622in}}{\pgfqpoint{3.696000in}{3.696000in}}%
\pgfusepath{clip}%
\pgfsetrectcap%
\pgfsetroundjoin%
\pgfsetlinewidth{1.505625pt}%
\definecolor{currentstroke}{rgb}{1.000000,0.000000,0.000000}%
\pgfsetstrokecolor{currentstroke}%
\pgfsetdash{}{0pt}%
\pgfpathmoveto{\pgfqpoint{2.670942in}{1.139442in}}%
\pgfpathlineto{\pgfqpoint{2.510478in}{1.227352in}}%
\pgfusepath{stroke}%
\end{pgfscope}%
\begin{pgfscope}%
\pgfpathrectangle{\pgfqpoint{0.100000in}{0.212622in}}{\pgfqpoint{3.696000in}{3.696000in}}%
\pgfusepath{clip}%
\pgfsetrectcap%
\pgfsetroundjoin%
\pgfsetlinewidth{1.505625pt}%
\definecolor{currentstroke}{rgb}{1.000000,0.000000,0.000000}%
\pgfsetstrokecolor{currentstroke}%
\pgfsetdash{}{0pt}%
\pgfpathmoveto{\pgfqpoint{2.676907in}{1.137103in}}%
\pgfpathlineto{\pgfqpoint{2.510478in}{1.227352in}}%
\pgfusepath{stroke}%
\end{pgfscope}%
\begin{pgfscope}%
\pgfpathrectangle{\pgfqpoint{0.100000in}{0.212622in}}{\pgfqpoint{3.696000in}{3.696000in}}%
\pgfusepath{clip}%
\pgfsetrectcap%
\pgfsetroundjoin%
\pgfsetlinewidth{1.505625pt}%
\definecolor{currentstroke}{rgb}{1.000000,0.000000,0.000000}%
\pgfsetstrokecolor{currentstroke}%
\pgfsetdash{}{0pt}%
\pgfpathmoveto{\pgfqpoint{2.684508in}{1.134962in}}%
\pgfpathlineto{\pgfqpoint{2.510478in}{1.227352in}}%
\pgfusepath{stroke}%
\end{pgfscope}%
\begin{pgfscope}%
\pgfpathrectangle{\pgfqpoint{0.100000in}{0.212622in}}{\pgfqpoint{3.696000in}{3.696000in}}%
\pgfusepath{clip}%
\pgfsetrectcap%
\pgfsetroundjoin%
\pgfsetlinewidth{1.505625pt}%
\definecolor{currentstroke}{rgb}{1.000000,0.000000,0.000000}%
\pgfsetstrokecolor{currentstroke}%
\pgfsetdash{}{0pt}%
\pgfpathmoveto{\pgfqpoint{2.693234in}{1.131517in}}%
\pgfpathlineto{\pgfqpoint{2.510478in}{1.227352in}}%
\pgfusepath{stroke}%
\end{pgfscope}%
\begin{pgfscope}%
\pgfpathrectangle{\pgfqpoint{0.100000in}{0.212622in}}{\pgfqpoint{3.696000in}{3.696000in}}%
\pgfusepath{clip}%
\pgfsetrectcap%
\pgfsetroundjoin%
\pgfsetlinewidth{1.505625pt}%
\definecolor{currentstroke}{rgb}{1.000000,0.000000,0.000000}%
\pgfsetstrokecolor{currentstroke}%
\pgfsetdash{}{0pt}%
\pgfpathmoveto{\pgfqpoint{2.703290in}{1.128164in}}%
\pgfpathlineto{\pgfqpoint{2.510478in}{1.227352in}}%
\pgfusepath{stroke}%
\end{pgfscope}%
\begin{pgfscope}%
\pgfpathrectangle{\pgfqpoint{0.100000in}{0.212622in}}{\pgfqpoint{3.696000in}{3.696000in}}%
\pgfusepath{clip}%
\pgfsetrectcap%
\pgfsetroundjoin%
\pgfsetlinewidth{1.505625pt}%
\definecolor{currentstroke}{rgb}{1.000000,0.000000,0.000000}%
\pgfsetstrokecolor{currentstroke}%
\pgfsetdash{}{0pt}%
\pgfpathmoveto{\pgfqpoint{2.714945in}{1.125609in}}%
\pgfpathlineto{\pgfqpoint{2.510478in}{1.227352in}}%
\pgfusepath{stroke}%
\end{pgfscope}%
\begin{pgfscope}%
\pgfpathrectangle{\pgfqpoint{0.100000in}{0.212622in}}{\pgfqpoint{3.696000in}{3.696000in}}%
\pgfusepath{clip}%
\pgfsetrectcap%
\pgfsetroundjoin%
\pgfsetlinewidth{1.505625pt}%
\definecolor{currentstroke}{rgb}{1.000000,0.000000,0.000000}%
\pgfsetstrokecolor{currentstroke}%
\pgfsetdash{}{0pt}%
\pgfpathmoveto{\pgfqpoint{2.726927in}{1.121895in}}%
\pgfpathlineto{\pgfqpoint{2.510478in}{1.227352in}}%
\pgfusepath{stroke}%
\end{pgfscope}%
\begin{pgfscope}%
\pgfpathrectangle{\pgfqpoint{0.100000in}{0.212622in}}{\pgfqpoint{3.696000in}{3.696000in}}%
\pgfusepath{clip}%
\pgfsetrectcap%
\pgfsetroundjoin%
\pgfsetlinewidth{1.505625pt}%
\definecolor{currentstroke}{rgb}{1.000000,0.000000,0.000000}%
\pgfsetstrokecolor{currentstroke}%
\pgfsetdash{}{0pt}%
\pgfpathmoveto{\pgfqpoint{2.733080in}{1.119596in}}%
\pgfpathlineto{\pgfqpoint{2.510478in}{1.227352in}}%
\pgfusepath{stroke}%
\end{pgfscope}%
\begin{pgfscope}%
\pgfpathrectangle{\pgfqpoint{0.100000in}{0.212622in}}{\pgfqpoint{3.696000in}{3.696000in}}%
\pgfusepath{clip}%
\pgfsetrectcap%
\pgfsetroundjoin%
\pgfsetlinewidth{1.505625pt}%
\definecolor{currentstroke}{rgb}{1.000000,0.000000,0.000000}%
\pgfsetstrokecolor{currentstroke}%
\pgfsetdash{}{0pt}%
\pgfpathmoveto{\pgfqpoint{2.736671in}{1.118460in}}%
\pgfpathlineto{\pgfqpoint{2.510478in}{1.227352in}}%
\pgfusepath{stroke}%
\end{pgfscope}%
\begin{pgfscope}%
\pgfpathrectangle{\pgfqpoint{0.100000in}{0.212622in}}{\pgfqpoint{3.696000in}{3.696000in}}%
\pgfusepath{clip}%
\pgfsetrectcap%
\pgfsetroundjoin%
\pgfsetlinewidth{1.505625pt}%
\definecolor{currentstroke}{rgb}{1.000000,0.000000,0.000000}%
\pgfsetstrokecolor{currentstroke}%
\pgfsetdash{}{0pt}%
\pgfpathmoveto{\pgfqpoint{2.740511in}{1.116779in}}%
\pgfpathlineto{\pgfqpoint{2.510478in}{1.227352in}}%
\pgfusepath{stroke}%
\end{pgfscope}%
\begin{pgfscope}%
\pgfpathrectangle{\pgfqpoint{0.100000in}{0.212622in}}{\pgfqpoint{3.696000in}{3.696000in}}%
\pgfusepath{clip}%
\pgfsetrectcap%
\pgfsetroundjoin%
\pgfsetlinewidth{1.505625pt}%
\definecolor{currentstroke}{rgb}{1.000000,0.000000,0.000000}%
\pgfsetstrokecolor{currentstroke}%
\pgfsetdash{}{0pt}%
\pgfpathmoveto{\pgfqpoint{2.742724in}{1.115979in}}%
\pgfpathlineto{\pgfqpoint{2.510478in}{1.227352in}}%
\pgfusepath{stroke}%
\end{pgfscope}%
\begin{pgfscope}%
\pgfpathrectangle{\pgfqpoint{0.100000in}{0.212622in}}{\pgfqpoint{3.696000in}{3.696000in}}%
\pgfusepath{clip}%
\pgfsetrectcap%
\pgfsetroundjoin%
\pgfsetlinewidth{1.505625pt}%
\definecolor{currentstroke}{rgb}{1.000000,0.000000,0.000000}%
\pgfsetstrokecolor{currentstroke}%
\pgfsetdash{}{0pt}%
\pgfpathmoveto{\pgfqpoint{2.746317in}{1.114661in}}%
\pgfpathlineto{\pgfqpoint{2.510478in}{1.227352in}}%
\pgfusepath{stroke}%
\end{pgfscope}%
\begin{pgfscope}%
\pgfpathrectangle{\pgfqpoint{0.100000in}{0.212622in}}{\pgfqpoint{3.696000in}{3.696000in}}%
\pgfusepath{clip}%
\pgfsetrectcap%
\pgfsetroundjoin%
\pgfsetlinewidth{1.505625pt}%
\definecolor{currentstroke}{rgb}{1.000000,0.000000,0.000000}%
\pgfsetstrokecolor{currentstroke}%
\pgfsetdash{}{0pt}%
\pgfpathmoveto{\pgfqpoint{2.750844in}{1.112729in}}%
\pgfpathlineto{\pgfqpoint{2.510478in}{1.227352in}}%
\pgfusepath{stroke}%
\end{pgfscope}%
\begin{pgfscope}%
\pgfpathrectangle{\pgfqpoint{0.100000in}{0.212622in}}{\pgfqpoint{3.696000in}{3.696000in}}%
\pgfusepath{clip}%
\pgfsetrectcap%
\pgfsetroundjoin%
\pgfsetlinewidth{1.505625pt}%
\definecolor{currentstroke}{rgb}{1.000000,0.000000,0.000000}%
\pgfsetstrokecolor{currentstroke}%
\pgfsetdash{}{0pt}%
\pgfpathmoveto{\pgfqpoint{2.757000in}{1.110990in}}%
\pgfpathlineto{\pgfqpoint{2.510478in}{1.227352in}}%
\pgfusepath{stroke}%
\end{pgfscope}%
\begin{pgfscope}%
\pgfpathrectangle{\pgfqpoint{0.100000in}{0.212622in}}{\pgfqpoint{3.696000in}{3.696000in}}%
\pgfusepath{clip}%
\pgfsetrectcap%
\pgfsetroundjoin%
\pgfsetlinewidth{1.505625pt}%
\definecolor{currentstroke}{rgb}{1.000000,0.000000,0.000000}%
\pgfsetstrokecolor{currentstroke}%
\pgfsetdash{}{0pt}%
\pgfpathmoveto{\pgfqpoint{2.763965in}{1.108855in}}%
\pgfpathlineto{\pgfqpoint{2.510478in}{1.227352in}}%
\pgfusepath{stroke}%
\end{pgfscope}%
\begin{pgfscope}%
\pgfpathrectangle{\pgfqpoint{0.100000in}{0.212622in}}{\pgfqpoint{3.696000in}{3.696000in}}%
\pgfusepath{clip}%
\pgfsetrectcap%
\pgfsetroundjoin%
\pgfsetlinewidth{1.505625pt}%
\definecolor{currentstroke}{rgb}{1.000000,0.000000,0.000000}%
\pgfsetstrokecolor{currentstroke}%
\pgfsetdash{}{0pt}%
\pgfpathmoveto{\pgfqpoint{2.767810in}{1.107678in}}%
\pgfpathlineto{\pgfqpoint{2.510478in}{1.227352in}}%
\pgfusepath{stroke}%
\end{pgfscope}%
\begin{pgfscope}%
\pgfpathrectangle{\pgfqpoint{0.100000in}{0.212622in}}{\pgfqpoint{3.696000in}{3.696000in}}%
\pgfusepath{clip}%
\pgfsetrectcap%
\pgfsetroundjoin%
\pgfsetlinewidth{1.505625pt}%
\definecolor{currentstroke}{rgb}{1.000000,0.000000,0.000000}%
\pgfsetstrokecolor{currentstroke}%
\pgfsetdash{}{0pt}%
\pgfpathmoveto{\pgfqpoint{2.769932in}{1.107049in}}%
\pgfpathlineto{\pgfqpoint{2.510478in}{1.227352in}}%
\pgfusepath{stroke}%
\end{pgfscope}%
\begin{pgfscope}%
\pgfpathrectangle{\pgfqpoint{0.100000in}{0.212622in}}{\pgfqpoint{3.696000in}{3.696000in}}%
\pgfusepath{clip}%
\pgfsetrectcap%
\pgfsetroundjoin%
\pgfsetlinewidth{1.505625pt}%
\definecolor{currentstroke}{rgb}{1.000000,0.000000,0.000000}%
\pgfsetstrokecolor{currentstroke}%
\pgfsetdash{}{0pt}%
\pgfpathmoveto{\pgfqpoint{2.771079in}{1.106663in}}%
\pgfpathlineto{\pgfqpoint{2.510478in}{1.227352in}}%
\pgfusepath{stroke}%
\end{pgfscope}%
\begin{pgfscope}%
\pgfpathrectangle{\pgfqpoint{0.100000in}{0.212622in}}{\pgfqpoint{3.696000in}{3.696000in}}%
\pgfusepath{clip}%
\pgfsetrectcap%
\pgfsetroundjoin%
\pgfsetlinewidth{1.505625pt}%
\definecolor{currentstroke}{rgb}{1.000000,0.000000,0.000000}%
\pgfsetstrokecolor{currentstroke}%
\pgfsetdash{}{0pt}%
\pgfpathmoveto{\pgfqpoint{2.773419in}{1.105785in}}%
\pgfpathlineto{\pgfqpoint{2.510478in}{1.227352in}}%
\pgfusepath{stroke}%
\end{pgfscope}%
\begin{pgfscope}%
\pgfpathrectangle{\pgfqpoint{0.100000in}{0.212622in}}{\pgfqpoint{3.696000in}{3.696000in}}%
\pgfusepath{clip}%
\pgfsetrectcap%
\pgfsetroundjoin%
\pgfsetlinewidth{1.505625pt}%
\definecolor{currentstroke}{rgb}{1.000000,0.000000,0.000000}%
\pgfsetstrokecolor{currentstroke}%
\pgfsetdash{}{0pt}%
\pgfpathmoveto{\pgfqpoint{2.776677in}{1.104338in}}%
\pgfpathlineto{\pgfqpoint{2.510478in}{1.227352in}}%
\pgfusepath{stroke}%
\end{pgfscope}%
\begin{pgfscope}%
\pgfpathrectangle{\pgfqpoint{0.100000in}{0.212622in}}{\pgfqpoint{3.696000in}{3.696000in}}%
\pgfusepath{clip}%
\pgfsetrectcap%
\pgfsetroundjoin%
\pgfsetlinewidth{1.505625pt}%
\definecolor{currentstroke}{rgb}{1.000000,0.000000,0.000000}%
\pgfsetstrokecolor{currentstroke}%
\pgfsetdash{}{0pt}%
\pgfpathmoveto{\pgfqpoint{2.781334in}{1.103280in}}%
\pgfpathlineto{\pgfqpoint{2.510478in}{1.227352in}}%
\pgfusepath{stroke}%
\end{pgfscope}%
\begin{pgfscope}%
\pgfpathrectangle{\pgfqpoint{0.100000in}{0.212622in}}{\pgfqpoint{3.696000in}{3.696000in}}%
\pgfusepath{clip}%
\pgfsetrectcap%
\pgfsetroundjoin%
\pgfsetlinewidth{1.505625pt}%
\definecolor{currentstroke}{rgb}{1.000000,0.000000,0.000000}%
\pgfsetstrokecolor{currentstroke}%
\pgfsetdash{}{0pt}%
\pgfpathmoveto{\pgfqpoint{2.787054in}{1.101785in}}%
\pgfpathlineto{\pgfqpoint{2.510478in}{1.227352in}}%
\pgfusepath{stroke}%
\end{pgfscope}%
\begin{pgfscope}%
\pgfpathrectangle{\pgfqpoint{0.100000in}{0.212622in}}{\pgfqpoint{3.696000in}{3.696000in}}%
\pgfusepath{clip}%
\pgfsetrectcap%
\pgfsetroundjoin%
\pgfsetlinewidth{1.505625pt}%
\definecolor{currentstroke}{rgb}{1.000000,0.000000,0.000000}%
\pgfsetstrokecolor{currentstroke}%
\pgfsetdash{}{0pt}%
\pgfpathmoveto{\pgfqpoint{2.793432in}{1.099583in}}%
\pgfpathlineto{\pgfqpoint{2.510478in}{1.227352in}}%
\pgfusepath{stroke}%
\end{pgfscope}%
\begin{pgfscope}%
\pgfpathrectangle{\pgfqpoint{0.100000in}{0.212622in}}{\pgfqpoint{3.696000in}{3.696000in}}%
\pgfusepath{clip}%
\pgfsetrectcap%
\pgfsetroundjoin%
\pgfsetlinewidth{1.505625pt}%
\definecolor{currentstroke}{rgb}{1.000000,0.000000,0.000000}%
\pgfsetstrokecolor{currentstroke}%
\pgfsetdash{}{0pt}%
\pgfpathmoveto{\pgfqpoint{2.796969in}{1.098522in}}%
\pgfpathlineto{\pgfqpoint{2.510478in}{1.227352in}}%
\pgfusepath{stroke}%
\end{pgfscope}%
\begin{pgfscope}%
\pgfpathrectangle{\pgfqpoint{0.100000in}{0.212622in}}{\pgfqpoint{3.696000in}{3.696000in}}%
\pgfusepath{clip}%
\pgfsetrectcap%
\pgfsetroundjoin%
\pgfsetlinewidth{1.505625pt}%
\definecolor{currentstroke}{rgb}{1.000000,0.000000,0.000000}%
\pgfsetstrokecolor{currentstroke}%
\pgfsetdash{}{0pt}%
\pgfpathmoveto{\pgfqpoint{2.798835in}{1.097787in}}%
\pgfpathlineto{\pgfqpoint{2.510478in}{1.227352in}}%
\pgfusepath{stroke}%
\end{pgfscope}%
\begin{pgfscope}%
\pgfpathrectangle{\pgfqpoint{0.100000in}{0.212622in}}{\pgfqpoint{3.696000in}{3.696000in}}%
\pgfusepath{clip}%
\pgfsetrectcap%
\pgfsetroundjoin%
\pgfsetlinewidth{1.505625pt}%
\definecolor{currentstroke}{rgb}{1.000000,0.000000,0.000000}%
\pgfsetstrokecolor{currentstroke}%
\pgfsetdash{}{0pt}%
\pgfpathmoveto{\pgfqpoint{2.801245in}{1.096672in}}%
\pgfpathlineto{\pgfqpoint{2.510478in}{1.227352in}}%
\pgfusepath{stroke}%
\end{pgfscope}%
\begin{pgfscope}%
\pgfpathrectangle{\pgfqpoint{0.100000in}{0.212622in}}{\pgfqpoint{3.696000in}{3.696000in}}%
\pgfusepath{clip}%
\pgfsetrectcap%
\pgfsetroundjoin%
\pgfsetlinewidth{1.505625pt}%
\definecolor{currentstroke}{rgb}{1.000000,0.000000,0.000000}%
\pgfsetstrokecolor{currentstroke}%
\pgfsetdash{}{0pt}%
\pgfpathmoveto{\pgfqpoint{2.804249in}{1.095580in}}%
\pgfpathlineto{\pgfqpoint{2.510478in}{1.227352in}}%
\pgfusepath{stroke}%
\end{pgfscope}%
\begin{pgfscope}%
\pgfpathrectangle{\pgfqpoint{0.100000in}{0.212622in}}{\pgfqpoint{3.696000in}{3.696000in}}%
\pgfusepath{clip}%
\pgfsetrectcap%
\pgfsetroundjoin%
\pgfsetlinewidth{1.505625pt}%
\definecolor{currentstroke}{rgb}{1.000000,0.000000,0.000000}%
\pgfsetstrokecolor{currentstroke}%
\pgfsetdash{}{0pt}%
\pgfpathmoveto{\pgfqpoint{2.808694in}{1.094196in}}%
\pgfpathlineto{\pgfqpoint{2.510478in}{1.227352in}}%
\pgfusepath{stroke}%
\end{pgfscope}%
\begin{pgfscope}%
\pgfpathrectangle{\pgfqpoint{0.100000in}{0.212622in}}{\pgfqpoint{3.696000in}{3.696000in}}%
\pgfusepath{clip}%
\pgfsetrectcap%
\pgfsetroundjoin%
\pgfsetlinewidth{1.505625pt}%
\definecolor{currentstroke}{rgb}{1.000000,0.000000,0.000000}%
\pgfsetstrokecolor{currentstroke}%
\pgfsetdash{}{0pt}%
\pgfpathmoveto{\pgfqpoint{2.813888in}{1.092443in}}%
\pgfpathlineto{\pgfqpoint{2.510478in}{1.227352in}}%
\pgfusepath{stroke}%
\end{pgfscope}%
\begin{pgfscope}%
\pgfpathrectangle{\pgfqpoint{0.100000in}{0.212622in}}{\pgfqpoint{3.696000in}{3.696000in}}%
\pgfusepath{clip}%
\pgfsetrectcap%
\pgfsetroundjoin%
\pgfsetlinewidth{1.505625pt}%
\definecolor{currentstroke}{rgb}{1.000000,0.000000,0.000000}%
\pgfsetstrokecolor{currentstroke}%
\pgfsetdash{}{0pt}%
\pgfpathmoveto{\pgfqpoint{2.820022in}{1.090390in}}%
\pgfpathlineto{\pgfqpoint{2.510478in}{1.227352in}}%
\pgfusepath{stroke}%
\end{pgfscope}%
\begin{pgfscope}%
\pgfpathrectangle{\pgfqpoint{0.100000in}{0.212622in}}{\pgfqpoint{3.696000in}{3.696000in}}%
\pgfusepath{clip}%
\pgfsetrectcap%
\pgfsetroundjoin%
\pgfsetlinewidth{1.505625pt}%
\definecolor{currentstroke}{rgb}{1.000000,0.000000,0.000000}%
\pgfsetstrokecolor{currentstroke}%
\pgfsetdash{}{0pt}%
\pgfpathmoveto{\pgfqpoint{2.826963in}{1.088210in}}%
\pgfpathlineto{\pgfqpoint{2.510478in}{1.227352in}}%
\pgfusepath{stroke}%
\end{pgfscope}%
\begin{pgfscope}%
\pgfpathrectangle{\pgfqpoint{0.100000in}{0.212622in}}{\pgfqpoint{3.696000in}{3.696000in}}%
\pgfusepath{clip}%
\pgfsetrectcap%
\pgfsetroundjoin%
\pgfsetlinewidth{1.505625pt}%
\definecolor{currentstroke}{rgb}{1.000000,0.000000,0.000000}%
\pgfsetstrokecolor{currentstroke}%
\pgfsetdash{}{0pt}%
\pgfpathmoveto{\pgfqpoint{2.830801in}{1.087022in}}%
\pgfpathlineto{\pgfqpoint{2.510478in}{1.227352in}}%
\pgfusepath{stroke}%
\end{pgfscope}%
\begin{pgfscope}%
\pgfpathrectangle{\pgfqpoint{0.100000in}{0.212622in}}{\pgfqpoint{3.696000in}{3.696000in}}%
\pgfusepath{clip}%
\pgfsetrectcap%
\pgfsetroundjoin%
\pgfsetlinewidth{1.505625pt}%
\definecolor{currentstroke}{rgb}{1.000000,0.000000,0.000000}%
\pgfsetstrokecolor{currentstroke}%
\pgfsetdash{}{0pt}%
\pgfpathmoveto{\pgfqpoint{2.832705in}{1.086117in}}%
\pgfpathlineto{\pgfqpoint{2.510478in}{1.227352in}}%
\pgfusepath{stroke}%
\end{pgfscope}%
\begin{pgfscope}%
\pgfpathrectangle{\pgfqpoint{0.100000in}{0.212622in}}{\pgfqpoint{3.696000in}{3.696000in}}%
\pgfusepath{clip}%
\pgfsetrectcap%
\pgfsetroundjoin%
\pgfsetlinewidth{1.505625pt}%
\definecolor{currentstroke}{rgb}{1.000000,0.000000,0.000000}%
\pgfsetstrokecolor{currentstroke}%
\pgfsetdash{}{0pt}%
\pgfpathmoveto{\pgfqpoint{2.835182in}{1.084991in}}%
\pgfpathlineto{\pgfqpoint{2.510478in}{1.227352in}}%
\pgfusepath{stroke}%
\end{pgfscope}%
\begin{pgfscope}%
\pgfpathrectangle{\pgfqpoint{0.100000in}{0.212622in}}{\pgfqpoint{3.696000in}{3.696000in}}%
\pgfusepath{clip}%
\pgfsetrectcap%
\pgfsetroundjoin%
\pgfsetlinewidth{1.505625pt}%
\definecolor{currentstroke}{rgb}{1.000000,0.000000,0.000000}%
\pgfsetstrokecolor{currentstroke}%
\pgfsetdash{}{0pt}%
\pgfpathmoveto{\pgfqpoint{2.838515in}{1.083776in}}%
\pgfpathlineto{\pgfqpoint{2.510478in}{1.227352in}}%
\pgfusepath{stroke}%
\end{pgfscope}%
\begin{pgfscope}%
\pgfpathrectangle{\pgfqpoint{0.100000in}{0.212622in}}{\pgfqpoint{3.696000in}{3.696000in}}%
\pgfusepath{clip}%
\pgfsetrectcap%
\pgfsetroundjoin%
\pgfsetlinewidth{1.505625pt}%
\definecolor{currentstroke}{rgb}{1.000000,0.000000,0.000000}%
\pgfsetstrokecolor{currentstroke}%
\pgfsetdash{}{0pt}%
\pgfpathmoveto{\pgfqpoint{2.840353in}{1.083132in}}%
\pgfpathlineto{\pgfqpoint{2.510478in}{1.227352in}}%
\pgfusepath{stroke}%
\end{pgfscope}%
\begin{pgfscope}%
\pgfpathrectangle{\pgfqpoint{0.100000in}{0.212622in}}{\pgfqpoint{3.696000in}{3.696000in}}%
\pgfusepath{clip}%
\pgfsetrectcap%
\pgfsetroundjoin%
\pgfsetlinewidth{1.505625pt}%
\definecolor{currentstroke}{rgb}{1.000000,0.000000,0.000000}%
\pgfsetstrokecolor{currentstroke}%
\pgfsetdash{}{0pt}%
\pgfpathmoveto{\pgfqpoint{2.841386in}{1.082807in}}%
\pgfpathlineto{\pgfqpoint{2.510478in}{1.227352in}}%
\pgfusepath{stroke}%
\end{pgfscope}%
\begin{pgfscope}%
\pgfpathrectangle{\pgfqpoint{0.100000in}{0.212622in}}{\pgfqpoint{3.696000in}{3.696000in}}%
\pgfusepath{clip}%
\pgfsetrectcap%
\pgfsetroundjoin%
\pgfsetlinewidth{1.505625pt}%
\definecolor{currentstroke}{rgb}{1.000000,0.000000,0.000000}%
\pgfsetstrokecolor{currentstroke}%
\pgfsetdash{}{0pt}%
\pgfpathmoveto{\pgfqpoint{2.841864in}{1.082519in}}%
\pgfpathlineto{\pgfqpoint{2.510478in}{1.227352in}}%
\pgfusepath{stroke}%
\end{pgfscope}%
\begin{pgfscope}%
\pgfpathrectangle{\pgfqpoint{0.100000in}{0.212622in}}{\pgfqpoint{3.696000in}{3.696000in}}%
\pgfusepath{clip}%
\pgfsetrectcap%
\pgfsetroundjoin%
\pgfsetlinewidth{1.505625pt}%
\definecolor{currentstroke}{rgb}{1.000000,0.000000,0.000000}%
\pgfsetstrokecolor{currentstroke}%
\pgfsetdash{}{0pt}%
\pgfpathmoveto{\pgfqpoint{2.842151in}{1.082391in}}%
\pgfpathlineto{\pgfqpoint{2.510478in}{1.227352in}}%
\pgfusepath{stroke}%
\end{pgfscope}%
\begin{pgfscope}%
\pgfpathrectangle{\pgfqpoint{0.100000in}{0.212622in}}{\pgfqpoint{3.696000in}{3.696000in}}%
\pgfusepath{clip}%
\pgfsetrectcap%
\pgfsetroundjoin%
\pgfsetlinewidth{1.505625pt}%
\definecolor{currentstroke}{rgb}{1.000000,0.000000,0.000000}%
\pgfsetstrokecolor{currentstroke}%
\pgfsetdash{}{0pt}%
\pgfpathmoveto{\pgfqpoint{2.843348in}{1.081935in}}%
\pgfpathlineto{\pgfqpoint{2.510478in}{1.227352in}}%
\pgfusepath{stroke}%
\end{pgfscope}%
\begin{pgfscope}%
\pgfpathrectangle{\pgfqpoint{0.100000in}{0.212622in}}{\pgfqpoint{3.696000in}{3.696000in}}%
\pgfusepath{clip}%
\pgfsetrectcap%
\pgfsetroundjoin%
\pgfsetlinewidth{1.505625pt}%
\definecolor{currentstroke}{rgb}{1.000000,0.000000,0.000000}%
\pgfsetstrokecolor{currentstroke}%
\pgfsetdash{}{0pt}%
\pgfpathmoveto{\pgfqpoint{2.845731in}{1.080937in}}%
\pgfpathlineto{\pgfqpoint{2.510478in}{1.227352in}}%
\pgfusepath{stroke}%
\end{pgfscope}%
\begin{pgfscope}%
\pgfpathrectangle{\pgfqpoint{0.100000in}{0.212622in}}{\pgfqpoint{3.696000in}{3.696000in}}%
\pgfusepath{clip}%
\pgfsetrectcap%
\pgfsetroundjoin%
\pgfsetlinewidth{1.505625pt}%
\definecolor{currentstroke}{rgb}{1.000000,0.000000,0.000000}%
\pgfsetstrokecolor{currentstroke}%
\pgfsetdash{}{0pt}%
\pgfpathmoveto{\pgfqpoint{2.848945in}{1.079850in}}%
\pgfpathlineto{\pgfqpoint{2.510478in}{1.227352in}}%
\pgfusepath{stroke}%
\end{pgfscope}%
\begin{pgfscope}%
\pgfpathrectangle{\pgfqpoint{0.100000in}{0.212622in}}{\pgfqpoint{3.696000in}{3.696000in}}%
\pgfusepath{clip}%
\pgfsetrectcap%
\pgfsetroundjoin%
\pgfsetlinewidth{1.505625pt}%
\definecolor{currentstroke}{rgb}{1.000000,0.000000,0.000000}%
\pgfsetstrokecolor{currentstroke}%
\pgfsetdash{}{0pt}%
\pgfpathmoveto{\pgfqpoint{2.852659in}{1.078370in}}%
\pgfpathlineto{\pgfqpoint{2.510478in}{1.227352in}}%
\pgfusepath{stroke}%
\end{pgfscope}%
\begin{pgfscope}%
\pgfpathrectangle{\pgfqpoint{0.100000in}{0.212622in}}{\pgfqpoint{3.696000in}{3.696000in}}%
\pgfusepath{clip}%
\pgfsetrectcap%
\pgfsetroundjoin%
\pgfsetlinewidth{1.505625pt}%
\definecolor{currentstroke}{rgb}{1.000000,0.000000,0.000000}%
\pgfsetstrokecolor{currentstroke}%
\pgfsetdash{}{0pt}%
\pgfpathmoveto{\pgfqpoint{2.857708in}{1.076613in}}%
\pgfpathlineto{\pgfqpoint{2.510478in}{1.227352in}}%
\pgfusepath{stroke}%
\end{pgfscope}%
\begin{pgfscope}%
\pgfpathrectangle{\pgfqpoint{0.100000in}{0.212622in}}{\pgfqpoint{3.696000in}{3.696000in}}%
\pgfusepath{clip}%
\pgfsetrectcap%
\pgfsetroundjoin%
\pgfsetlinewidth{1.505625pt}%
\definecolor{currentstroke}{rgb}{1.000000,0.000000,0.000000}%
\pgfsetstrokecolor{currentstroke}%
\pgfsetdash{}{0pt}%
\pgfpathmoveto{\pgfqpoint{2.864464in}{1.073048in}}%
\pgfpathlineto{\pgfqpoint{2.510478in}{1.227352in}}%
\pgfusepath{stroke}%
\end{pgfscope}%
\begin{pgfscope}%
\pgfpathrectangle{\pgfqpoint{0.100000in}{0.212622in}}{\pgfqpoint{3.696000in}{3.696000in}}%
\pgfusepath{clip}%
\pgfsetrectcap%
\pgfsetroundjoin%
\pgfsetlinewidth{1.505625pt}%
\definecolor{currentstroke}{rgb}{1.000000,0.000000,0.000000}%
\pgfsetstrokecolor{currentstroke}%
\pgfsetdash{}{0pt}%
\pgfpathmoveto{\pgfqpoint{2.868256in}{1.071085in}}%
\pgfpathlineto{\pgfqpoint{2.510478in}{1.227352in}}%
\pgfusepath{stroke}%
\end{pgfscope}%
\begin{pgfscope}%
\pgfpathrectangle{\pgfqpoint{0.100000in}{0.212622in}}{\pgfqpoint{3.696000in}{3.696000in}}%
\pgfusepath{clip}%
\pgfsetrectcap%
\pgfsetroundjoin%
\pgfsetlinewidth{1.505625pt}%
\definecolor{currentstroke}{rgb}{1.000000,0.000000,0.000000}%
\pgfsetstrokecolor{currentstroke}%
\pgfsetdash{}{0pt}%
\pgfpathmoveto{\pgfqpoint{2.870461in}{1.070163in}}%
\pgfpathlineto{\pgfqpoint{2.510478in}{1.227352in}}%
\pgfusepath{stroke}%
\end{pgfscope}%
\begin{pgfscope}%
\pgfpathrectangle{\pgfqpoint{0.100000in}{0.212622in}}{\pgfqpoint{3.696000in}{3.696000in}}%
\pgfusepath{clip}%
\pgfsetrectcap%
\pgfsetroundjoin%
\pgfsetlinewidth{1.505625pt}%
\definecolor{currentstroke}{rgb}{1.000000,0.000000,0.000000}%
\pgfsetstrokecolor{currentstroke}%
\pgfsetdash{}{0pt}%
\pgfpathmoveto{\pgfqpoint{2.871680in}{1.069677in}}%
\pgfpathlineto{\pgfqpoint{2.510478in}{1.227352in}}%
\pgfusepath{stroke}%
\end{pgfscope}%
\begin{pgfscope}%
\pgfpathrectangle{\pgfqpoint{0.100000in}{0.212622in}}{\pgfqpoint{3.696000in}{3.696000in}}%
\pgfusepath{clip}%
\pgfsetrectcap%
\pgfsetroundjoin%
\pgfsetlinewidth{1.505625pt}%
\definecolor{currentstroke}{rgb}{1.000000,0.000000,0.000000}%
\pgfsetstrokecolor{currentstroke}%
\pgfsetdash{}{0pt}%
\pgfpathmoveto{\pgfqpoint{2.872324in}{1.069383in}}%
\pgfpathlineto{\pgfqpoint{2.510478in}{1.227352in}}%
\pgfusepath{stroke}%
\end{pgfscope}%
\begin{pgfscope}%
\pgfpathrectangle{\pgfqpoint{0.100000in}{0.212622in}}{\pgfqpoint{3.696000in}{3.696000in}}%
\pgfusepath{clip}%
\pgfsetrectcap%
\pgfsetroundjoin%
\pgfsetlinewidth{1.505625pt}%
\definecolor{currentstroke}{rgb}{1.000000,0.000000,0.000000}%
\pgfsetstrokecolor{currentstroke}%
\pgfsetdash{}{0pt}%
\pgfpathmoveto{\pgfqpoint{2.872691in}{1.069234in}}%
\pgfpathlineto{\pgfqpoint{2.510478in}{1.227352in}}%
\pgfusepath{stroke}%
\end{pgfscope}%
\begin{pgfscope}%
\pgfpathrectangle{\pgfqpoint{0.100000in}{0.212622in}}{\pgfqpoint{3.696000in}{3.696000in}}%
\pgfusepath{clip}%
\pgfsetrectcap%
\pgfsetroundjoin%
\pgfsetlinewidth{1.505625pt}%
\definecolor{currentstroke}{rgb}{1.000000,0.000000,0.000000}%
\pgfsetstrokecolor{currentstroke}%
\pgfsetdash{}{0pt}%
\pgfpathmoveto{\pgfqpoint{2.873507in}{1.068942in}}%
\pgfpathlineto{\pgfqpoint{2.510478in}{1.227352in}}%
\pgfusepath{stroke}%
\end{pgfscope}%
\begin{pgfscope}%
\pgfpathrectangle{\pgfqpoint{0.100000in}{0.212622in}}{\pgfqpoint{3.696000in}{3.696000in}}%
\pgfusepath{clip}%
\pgfsetrectcap%
\pgfsetroundjoin%
\pgfsetlinewidth{1.505625pt}%
\definecolor{currentstroke}{rgb}{1.000000,0.000000,0.000000}%
\pgfsetstrokecolor{currentstroke}%
\pgfsetdash{}{0pt}%
\pgfpathmoveto{\pgfqpoint{2.874849in}{1.068534in}}%
\pgfpathlineto{\pgfqpoint{2.510478in}{1.227352in}}%
\pgfusepath{stroke}%
\end{pgfscope}%
\begin{pgfscope}%
\pgfpathrectangle{\pgfqpoint{0.100000in}{0.212622in}}{\pgfqpoint{3.696000in}{3.696000in}}%
\pgfusepath{clip}%
\pgfsetrectcap%
\pgfsetroundjoin%
\pgfsetlinewidth{1.505625pt}%
\definecolor{currentstroke}{rgb}{1.000000,0.000000,0.000000}%
\pgfsetstrokecolor{currentstroke}%
\pgfsetdash{}{0pt}%
\pgfpathmoveto{\pgfqpoint{2.875517in}{1.068186in}}%
\pgfpathlineto{\pgfqpoint{2.510478in}{1.227352in}}%
\pgfusepath{stroke}%
\end{pgfscope}%
\begin{pgfscope}%
\pgfpathrectangle{\pgfqpoint{0.100000in}{0.212622in}}{\pgfqpoint{3.696000in}{3.696000in}}%
\pgfusepath{clip}%
\pgfsetrectcap%
\pgfsetroundjoin%
\pgfsetlinewidth{1.505625pt}%
\definecolor{currentstroke}{rgb}{1.000000,0.000000,0.000000}%
\pgfsetstrokecolor{currentstroke}%
\pgfsetdash{}{0pt}%
\pgfpathmoveto{\pgfqpoint{2.877219in}{1.068103in}}%
\pgfpathlineto{\pgfqpoint{2.510478in}{1.227352in}}%
\pgfusepath{stroke}%
\end{pgfscope}%
\begin{pgfscope}%
\pgfpathrectangle{\pgfqpoint{0.100000in}{0.212622in}}{\pgfqpoint{3.696000in}{3.696000in}}%
\pgfusepath{clip}%
\pgfsetrectcap%
\pgfsetroundjoin%
\pgfsetlinewidth{1.505625pt}%
\definecolor{currentstroke}{rgb}{1.000000,0.000000,0.000000}%
\pgfsetstrokecolor{currentstroke}%
\pgfsetdash{}{0pt}%
\pgfpathmoveto{\pgfqpoint{2.877975in}{1.067803in}}%
\pgfpathlineto{\pgfqpoint{2.510478in}{1.227352in}}%
\pgfusepath{stroke}%
\end{pgfscope}%
\begin{pgfscope}%
\pgfpathrectangle{\pgfqpoint{0.100000in}{0.212622in}}{\pgfqpoint{3.696000in}{3.696000in}}%
\pgfusepath{clip}%
\pgfsetrectcap%
\pgfsetroundjoin%
\pgfsetlinewidth{1.505625pt}%
\definecolor{currentstroke}{rgb}{1.000000,0.000000,0.000000}%
\pgfsetstrokecolor{currentstroke}%
\pgfsetdash{}{0pt}%
\pgfpathmoveto{\pgfqpoint{2.878507in}{1.067794in}}%
\pgfpathlineto{\pgfqpoint{2.510478in}{1.227352in}}%
\pgfusepath{stroke}%
\end{pgfscope}%
\begin{pgfscope}%
\pgfpathrectangle{\pgfqpoint{0.100000in}{0.212622in}}{\pgfqpoint{3.696000in}{3.696000in}}%
\pgfusepath{clip}%
\pgfsetrectcap%
\pgfsetroundjoin%
\pgfsetlinewidth{1.505625pt}%
\definecolor{currentstroke}{rgb}{1.000000,0.000000,0.000000}%
\pgfsetstrokecolor{currentstroke}%
\pgfsetdash{}{0pt}%
\pgfpathmoveto{\pgfqpoint{2.879668in}{1.067807in}}%
\pgfpathlineto{\pgfqpoint{2.510478in}{1.227352in}}%
\pgfusepath{stroke}%
\end{pgfscope}%
\begin{pgfscope}%
\pgfpathrectangle{\pgfqpoint{0.100000in}{0.212622in}}{\pgfqpoint{3.696000in}{3.696000in}}%
\pgfusepath{clip}%
\pgfsetrectcap%
\pgfsetroundjoin%
\pgfsetlinewidth{1.505625pt}%
\definecolor{currentstroke}{rgb}{1.000000,0.000000,0.000000}%
\pgfsetstrokecolor{currentstroke}%
\pgfsetdash{}{0pt}%
\pgfpathmoveto{\pgfqpoint{2.880227in}{1.067705in}}%
\pgfpathlineto{\pgfqpoint{2.510478in}{1.227352in}}%
\pgfusepath{stroke}%
\end{pgfscope}%
\begin{pgfscope}%
\pgfpathrectangle{\pgfqpoint{0.100000in}{0.212622in}}{\pgfqpoint{3.696000in}{3.696000in}}%
\pgfusepath{clip}%
\pgfsetrectcap%
\pgfsetroundjoin%
\pgfsetlinewidth{1.505625pt}%
\definecolor{currentstroke}{rgb}{1.000000,0.000000,0.000000}%
\pgfsetstrokecolor{currentstroke}%
\pgfsetdash{}{0pt}%
\pgfpathmoveto{\pgfqpoint{2.880599in}{1.067726in}}%
\pgfpathlineto{\pgfqpoint{2.510478in}{1.227352in}}%
\pgfusepath{stroke}%
\end{pgfscope}%
\begin{pgfscope}%
\pgfpathrectangle{\pgfqpoint{0.100000in}{0.212622in}}{\pgfqpoint{3.696000in}{3.696000in}}%
\pgfusepath{clip}%
\pgfsetrectcap%
\pgfsetroundjoin%
\pgfsetlinewidth{1.505625pt}%
\definecolor{currentstroke}{rgb}{1.000000,0.000000,0.000000}%
\pgfsetstrokecolor{currentstroke}%
\pgfsetdash{}{0pt}%
\pgfpathmoveto{\pgfqpoint{2.880811in}{1.067752in}}%
\pgfpathlineto{\pgfqpoint{2.510478in}{1.227352in}}%
\pgfusepath{stroke}%
\end{pgfscope}%
\begin{pgfscope}%
\pgfpathrectangle{\pgfqpoint{0.100000in}{0.212622in}}{\pgfqpoint{3.696000in}{3.696000in}}%
\pgfusepath{clip}%
\pgfsetbuttcap%
\pgfsetroundjoin%
\definecolor{currentfill}{rgb}{0.121569,0.466667,0.705882}%
\pgfsetfillcolor{currentfill}%
\pgfsetfillopacity{0.300000}%
\pgfsetlinewidth{1.003750pt}%
\definecolor{currentstroke}{rgb}{0.121569,0.466667,0.705882}%
\pgfsetstrokecolor{currentstroke}%
\pgfsetstrokeopacity{0.300000}%
\pgfsetdash{}{0pt}%
\pgfpathmoveto{\pgfqpoint{1.136977in}{1.617749in}}%
\pgfpathcurveto{\pgfqpoint{1.145213in}{1.617749in}}{\pgfqpoint{1.153113in}{1.621021in}}{\pgfqpoint{1.158937in}{1.626845in}}%
\pgfpathcurveto{\pgfqpoint{1.164761in}{1.632669in}}{\pgfqpoint{1.168033in}{1.640569in}}{\pgfqpoint{1.168033in}{1.648806in}}%
\pgfpathcurveto{\pgfqpoint{1.168033in}{1.657042in}}{\pgfqpoint{1.164761in}{1.664942in}}{\pgfqpoint{1.158937in}{1.670766in}}%
\pgfpathcurveto{\pgfqpoint{1.153113in}{1.676590in}}{\pgfqpoint{1.145213in}{1.679862in}}{\pgfqpoint{1.136977in}{1.679862in}}%
\pgfpathcurveto{\pgfqpoint{1.128740in}{1.679862in}}{\pgfqpoint{1.120840in}{1.676590in}}{\pgfqpoint{1.115016in}{1.670766in}}%
\pgfpathcurveto{\pgfqpoint{1.109193in}{1.664942in}}{\pgfqpoint{1.105920in}{1.657042in}}{\pgfqpoint{1.105920in}{1.648806in}}%
\pgfpathcurveto{\pgfqpoint{1.105920in}{1.640569in}}{\pgfqpoint{1.109193in}{1.632669in}}{\pgfqpoint{1.115016in}{1.626845in}}%
\pgfpathcurveto{\pgfqpoint{1.120840in}{1.621021in}}{\pgfqpoint{1.128740in}{1.617749in}}{\pgfqpoint{1.136977in}{1.617749in}}%
\pgfpathclose%
\pgfusepath{stroke,fill}%
\end{pgfscope}%
\begin{pgfscope}%
\pgfpathrectangle{\pgfqpoint{0.100000in}{0.212622in}}{\pgfqpoint{3.696000in}{3.696000in}}%
\pgfusepath{clip}%
\pgfsetbuttcap%
\pgfsetroundjoin%
\definecolor{currentfill}{rgb}{0.121569,0.466667,0.705882}%
\pgfsetfillcolor{currentfill}%
\pgfsetfillopacity{0.300001}%
\pgfsetlinewidth{1.003750pt}%
\definecolor{currentstroke}{rgb}{0.121569,0.466667,0.705882}%
\pgfsetstrokecolor{currentstroke}%
\pgfsetstrokeopacity{0.300001}%
\pgfsetdash{}{0pt}%
\pgfpathmoveto{\pgfqpoint{1.136979in}{1.617748in}}%
\pgfpathcurveto{\pgfqpoint{1.145216in}{1.617748in}}{\pgfqpoint{1.153116in}{1.621020in}}{\pgfqpoint{1.158940in}{1.626844in}}%
\pgfpathcurveto{\pgfqpoint{1.164764in}{1.632668in}}{\pgfqpoint{1.168036in}{1.640568in}}{\pgfqpoint{1.168036in}{1.648805in}}%
\pgfpathcurveto{\pgfqpoint{1.168036in}{1.657041in}}{\pgfqpoint{1.164764in}{1.664941in}}{\pgfqpoint{1.158940in}{1.670765in}}%
\pgfpathcurveto{\pgfqpoint{1.153116in}{1.676589in}}{\pgfqpoint{1.145216in}{1.679861in}}{\pgfqpoint{1.136979in}{1.679861in}}%
\pgfpathcurveto{\pgfqpoint{1.128743in}{1.679861in}}{\pgfqpoint{1.120843in}{1.676589in}}{\pgfqpoint{1.115019in}{1.670765in}}%
\pgfpathcurveto{\pgfqpoint{1.109195in}{1.664941in}}{\pgfqpoint{1.105923in}{1.657041in}}{\pgfqpoint{1.105923in}{1.648805in}}%
\pgfpathcurveto{\pgfqpoint{1.105923in}{1.640568in}}{\pgfqpoint{1.109195in}{1.632668in}}{\pgfqpoint{1.115019in}{1.626844in}}%
\pgfpathcurveto{\pgfqpoint{1.120843in}{1.621020in}}{\pgfqpoint{1.128743in}{1.617748in}}{\pgfqpoint{1.136979in}{1.617748in}}%
\pgfpathclose%
\pgfusepath{stroke,fill}%
\end{pgfscope}%
\begin{pgfscope}%
\pgfpathrectangle{\pgfqpoint{0.100000in}{0.212622in}}{\pgfqpoint{3.696000in}{3.696000in}}%
\pgfusepath{clip}%
\pgfsetbuttcap%
\pgfsetroundjoin%
\definecolor{currentfill}{rgb}{0.121569,0.466667,0.705882}%
\pgfsetfillcolor{currentfill}%
\pgfsetfillopacity{0.300002}%
\pgfsetlinewidth{1.003750pt}%
\definecolor{currentstroke}{rgb}{0.121569,0.466667,0.705882}%
\pgfsetstrokecolor{currentstroke}%
\pgfsetstrokeopacity{0.300002}%
\pgfsetdash{}{0pt}%
\pgfpathmoveto{\pgfqpoint{1.136981in}{1.617747in}}%
\pgfpathcurveto{\pgfqpoint{1.145217in}{1.617747in}}{\pgfqpoint{1.153117in}{1.621020in}}{\pgfqpoint{1.158941in}{1.626844in}}%
\pgfpathcurveto{\pgfqpoint{1.164765in}{1.632668in}}{\pgfqpoint{1.168037in}{1.640568in}}{\pgfqpoint{1.168037in}{1.648804in}}%
\pgfpathcurveto{\pgfqpoint{1.168037in}{1.657040in}}{\pgfqpoint{1.164765in}{1.664940in}}{\pgfqpoint{1.158941in}{1.670764in}}%
\pgfpathcurveto{\pgfqpoint{1.153117in}{1.676588in}}{\pgfqpoint{1.145217in}{1.679860in}}{\pgfqpoint{1.136981in}{1.679860in}}%
\pgfpathcurveto{\pgfqpoint{1.128744in}{1.679860in}}{\pgfqpoint{1.120844in}{1.676588in}}{\pgfqpoint{1.115020in}{1.670764in}}%
\pgfpathcurveto{\pgfqpoint{1.109197in}{1.664940in}}{\pgfqpoint{1.105924in}{1.657040in}}{\pgfqpoint{1.105924in}{1.648804in}}%
\pgfpathcurveto{\pgfqpoint{1.105924in}{1.640568in}}{\pgfqpoint{1.109197in}{1.632668in}}{\pgfqpoint{1.115020in}{1.626844in}}%
\pgfpathcurveto{\pgfqpoint{1.120844in}{1.621020in}}{\pgfqpoint{1.128744in}{1.617747in}}{\pgfqpoint{1.136981in}{1.617747in}}%
\pgfpathclose%
\pgfusepath{stroke,fill}%
\end{pgfscope}%
\begin{pgfscope}%
\pgfpathrectangle{\pgfqpoint{0.100000in}{0.212622in}}{\pgfqpoint{3.696000in}{3.696000in}}%
\pgfusepath{clip}%
\pgfsetbuttcap%
\pgfsetroundjoin%
\definecolor{currentfill}{rgb}{0.121569,0.466667,0.705882}%
\pgfsetfillcolor{currentfill}%
\pgfsetfillopacity{0.300003}%
\pgfsetlinewidth{1.003750pt}%
\definecolor{currentstroke}{rgb}{0.121569,0.466667,0.705882}%
\pgfsetstrokecolor{currentstroke}%
\pgfsetstrokeopacity{0.300003}%
\pgfsetdash{}{0pt}%
\pgfpathmoveto{\pgfqpoint{1.136982in}{1.617747in}}%
\pgfpathcurveto{\pgfqpoint{1.145218in}{1.617747in}}{\pgfqpoint{1.153118in}{1.621019in}}{\pgfqpoint{1.158942in}{1.626843in}}%
\pgfpathcurveto{\pgfqpoint{1.164766in}{1.632667in}}{\pgfqpoint{1.168038in}{1.640567in}}{\pgfqpoint{1.168038in}{1.648804in}}%
\pgfpathcurveto{\pgfqpoint{1.168038in}{1.657040in}}{\pgfqpoint{1.164766in}{1.664940in}}{\pgfqpoint{1.158942in}{1.670764in}}%
\pgfpathcurveto{\pgfqpoint{1.153118in}{1.676588in}}{\pgfqpoint{1.145218in}{1.679860in}}{\pgfqpoint{1.136982in}{1.679860in}}%
\pgfpathcurveto{\pgfqpoint{1.128745in}{1.679860in}}{\pgfqpoint{1.120845in}{1.676588in}}{\pgfqpoint{1.115021in}{1.670764in}}%
\pgfpathcurveto{\pgfqpoint{1.109197in}{1.664940in}}{\pgfqpoint{1.105925in}{1.657040in}}{\pgfqpoint{1.105925in}{1.648804in}}%
\pgfpathcurveto{\pgfqpoint{1.105925in}{1.640567in}}{\pgfqpoint{1.109197in}{1.632667in}}{\pgfqpoint{1.115021in}{1.626843in}}%
\pgfpathcurveto{\pgfqpoint{1.120845in}{1.621019in}}{\pgfqpoint{1.128745in}{1.617747in}}{\pgfqpoint{1.136982in}{1.617747in}}%
\pgfpathclose%
\pgfusepath{stroke,fill}%
\end{pgfscope}%
\begin{pgfscope}%
\pgfpathrectangle{\pgfqpoint{0.100000in}{0.212622in}}{\pgfqpoint{3.696000in}{3.696000in}}%
\pgfusepath{clip}%
\pgfsetbuttcap%
\pgfsetroundjoin%
\definecolor{currentfill}{rgb}{0.121569,0.466667,0.705882}%
\pgfsetfillcolor{currentfill}%
\pgfsetfillopacity{0.300003}%
\pgfsetlinewidth{1.003750pt}%
\definecolor{currentstroke}{rgb}{0.121569,0.466667,0.705882}%
\pgfsetstrokecolor{currentstroke}%
\pgfsetstrokeopacity{0.300003}%
\pgfsetdash{}{0pt}%
\pgfpathmoveto{\pgfqpoint{1.136982in}{1.617747in}}%
\pgfpathcurveto{\pgfqpoint{1.145218in}{1.617747in}}{\pgfqpoint{1.153118in}{1.621019in}}{\pgfqpoint{1.158942in}{1.626843in}}%
\pgfpathcurveto{\pgfqpoint{1.164766in}{1.632667in}}{\pgfqpoint{1.168038in}{1.640567in}}{\pgfqpoint{1.168038in}{1.648803in}}%
\pgfpathcurveto{\pgfqpoint{1.168038in}{1.657040in}}{\pgfqpoint{1.164766in}{1.664940in}}{\pgfqpoint{1.158942in}{1.670764in}}%
\pgfpathcurveto{\pgfqpoint{1.153118in}{1.676588in}}{\pgfqpoint{1.145218in}{1.679860in}}{\pgfqpoint{1.136982in}{1.679860in}}%
\pgfpathcurveto{\pgfqpoint{1.128746in}{1.679860in}}{\pgfqpoint{1.120846in}{1.676588in}}{\pgfqpoint{1.115022in}{1.670764in}}%
\pgfpathcurveto{\pgfqpoint{1.109198in}{1.664940in}}{\pgfqpoint{1.105925in}{1.657040in}}{\pgfqpoint{1.105925in}{1.648803in}}%
\pgfpathcurveto{\pgfqpoint{1.105925in}{1.640567in}}{\pgfqpoint{1.109198in}{1.632667in}}{\pgfqpoint{1.115022in}{1.626843in}}%
\pgfpathcurveto{\pgfqpoint{1.120846in}{1.621019in}}{\pgfqpoint{1.128746in}{1.617747in}}{\pgfqpoint{1.136982in}{1.617747in}}%
\pgfpathclose%
\pgfusepath{stroke,fill}%
\end{pgfscope}%
\begin{pgfscope}%
\pgfpathrectangle{\pgfqpoint{0.100000in}{0.212622in}}{\pgfqpoint{3.696000in}{3.696000in}}%
\pgfusepath{clip}%
\pgfsetbuttcap%
\pgfsetroundjoin%
\definecolor{currentfill}{rgb}{0.121569,0.466667,0.705882}%
\pgfsetfillcolor{currentfill}%
\pgfsetfillopacity{0.300003}%
\pgfsetlinewidth{1.003750pt}%
\definecolor{currentstroke}{rgb}{0.121569,0.466667,0.705882}%
\pgfsetstrokecolor{currentstroke}%
\pgfsetstrokeopacity{0.300003}%
\pgfsetdash{}{0pt}%
\pgfpathmoveto{\pgfqpoint{1.136982in}{1.617747in}}%
\pgfpathcurveto{\pgfqpoint{1.145218in}{1.617747in}}{\pgfqpoint{1.153118in}{1.621019in}}{\pgfqpoint{1.158942in}{1.626843in}}%
\pgfpathcurveto{\pgfqpoint{1.164766in}{1.632667in}}{\pgfqpoint{1.168039in}{1.640567in}}{\pgfqpoint{1.168039in}{1.648803in}}%
\pgfpathcurveto{\pgfqpoint{1.168039in}{1.657039in}}{\pgfqpoint{1.164766in}{1.664940in}}{\pgfqpoint{1.158942in}{1.670763in}}%
\pgfpathcurveto{\pgfqpoint{1.153118in}{1.676587in}}{\pgfqpoint{1.145218in}{1.679860in}}{\pgfqpoint{1.136982in}{1.679860in}}%
\pgfpathcurveto{\pgfqpoint{1.128746in}{1.679860in}}{\pgfqpoint{1.120846in}{1.676587in}}{\pgfqpoint{1.115022in}{1.670763in}}%
\pgfpathcurveto{\pgfqpoint{1.109198in}{1.664940in}}{\pgfqpoint{1.105926in}{1.657039in}}{\pgfqpoint{1.105926in}{1.648803in}}%
\pgfpathcurveto{\pgfqpoint{1.105926in}{1.640567in}}{\pgfqpoint{1.109198in}{1.632667in}}{\pgfqpoint{1.115022in}{1.626843in}}%
\pgfpathcurveto{\pgfqpoint{1.120846in}{1.621019in}}{\pgfqpoint{1.128746in}{1.617747in}}{\pgfqpoint{1.136982in}{1.617747in}}%
\pgfpathclose%
\pgfusepath{stroke,fill}%
\end{pgfscope}%
\begin{pgfscope}%
\pgfpathrectangle{\pgfqpoint{0.100000in}{0.212622in}}{\pgfqpoint{3.696000in}{3.696000in}}%
\pgfusepath{clip}%
\pgfsetbuttcap%
\pgfsetroundjoin%
\definecolor{currentfill}{rgb}{0.121569,0.466667,0.705882}%
\pgfsetfillcolor{currentfill}%
\pgfsetfillopacity{0.300003}%
\pgfsetlinewidth{1.003750pt}%
\definecolor{currentstroke}{rgb}{0.121569,0.466667,0.705882}%
\pgfsetstrokecolor{currentstroke}%
\pgfsetstrokeopacity{0.300003}%
\pgfsetdash{}{0pt}%
\pgfpathmoveto{\pgfqpoint{1.136982in}{1.617747in}}%
\pgfpathcurveto{\pgfqpoint{1.145219in}{1.617747in}}{\pgfqpoint{1.153119in}{1.621019in}}{\pgfqpoint{1.158943in}{1.626843in}}%
\pgfpathcurveto{\pgfqpoint{1.164766in}{1.632667in}}{\pgfqpoint{1.168039in}{1.640567in}}{\pgfqpoint{1.168039in}{1.648803in}}%
\pgfpathcurveto{\pgfqpoint{1.168039in}{1.657039in}}{\pgfqpoint{1.164766in}{1.664939in}}{\pgfqpoint{1.158943in}{1.670763in}}%
\pgfpathcurveto{\pgfqpoint{1.153119in}{1.676587in}}{\pgfqpoint{1.145219in}{1.679860in}}{\pgfqpoint{1.136982in}{1.679860in}}%
\pgfpathcurveto{\pgfqpoint{1.128746in}{1.679860in}}{\pgfqpoint{1.120846in}{1.676587in}}{\pgfqpoint{1.115022in}{1.670763in}}%
\pgfpathcurveto{\pgfqpoint{1.109198in}{1.664939in}}{\pgfqpoint{1.105926in}{1.657039in}}{\pgfqpoint{1.105926in}{1.648803in}}%
\pgfpathcurveto{\pgfqpoint{1.105926in}{1.640567in}}{\pgfqpoint{1.109198in}{1.632667in}}{\pgfqpoint{1.115022in}{1.626843in}}%
\pgfpathcurveto{\pgfqpoint{1.120846in}{1.621019in}}{\pgfqpoint{1.128746in}{1.617747in}}{\pgfqpoint{1.136982in}{1.617747in}}%
\pgfpathclose%
\pgfusepath{stroke,fill}%
\end{pgfscope}%
\begin{pgfscope}%
\pgfpathrectangle{\pgfqpoint{0.100000in}{0.212622in}}{\pgfqpoint{3.696000in}{3.696000in}}%
\pgfusepath{clip}%
\pgfsetbuttcap%
\pgfsetroundjoin%
\definecolor{currentfill}{rgb}{0.121569,0.466667,0.705882}%
\pgfsetfillcolor{currentfill}%
\pgfsetfillopacity{0.300003}%
\pgfsetlinewidth{1.003750pt}%
\definecolor{currentstroke}{rgb}{0.121569,0.466667,0.705882}%
\pgfsetstrokecolor{currentstroke}%
\pgfsetstrokeopacity{0.300003}%
\pgfsetdash{}{0pt}%
\pgfpathmoveto{\pgfqpoint{1.136982in}{1.617747in}}%
\pgfpathcurveto{\pgfqpoint{1.145219in}{1.617747in}}{\pgfqpoint{1.153119in}{1.621019in}}{\pgfqpoint{1.158943in}{1.626843in}}%
\pgfpathcurveto{\pgfqpoint{1.164767in}{1.632667in}}{\pgfqpoint{1.168039in}{1.640567in}}{\pgfqpoint{1.168039in}{1.648803in}}%
\pgfpathcurveto{\pgfqpoint{1.168039in}{1.657039in}}{\pgfqpoint{1.164767in}{1.664939in}}{\pgfqpoint{1.158943in}{1.670763in}}%
\pgfpathcurveto{\pgfqpoint{1.153119in}{1.676587in}}{\pgfqpoint{1.145219in}{1.679860in}}{\pgfqpoint{1.136982in}{1.679860in}}%
\pgfpathcurveto{\pgfqpoint{1.128746in}{1.679860in}}{\pgfqpoint{1.120846in}{1.676587in}}{\pgfqpoint{1.115022in}{1.670763in}}%
\pgfpathcurveto{\pgfqpoint{1.109198in}{1.664939in}}{\pgfqpoint{1.105926in}{1.657039in}}{\pgfqpoint{1.105926in}{1.648803in}}%
\pgfpathcurveto{\pgfqpoint{1.105926in}{1.640567in}}{\pgfqpoint{1.109198in}{1.632667in}}{\pgfqpoint{1.115022in}{1.626843in}}%
\pgfpathcurveto{\pgfqpoint{1.120846in}{1.621019in}}{\pgfqpoint{1.128746in}{1.617747in}}{\pgfqpoint{1.136982in}{1.617747in}}%
\pgfpathclose%
\pgfusepath{stroke,fill}%
\end{pgfscope}%
\begin{pgfscope}%
\pgfpathrectangle{\pgfqpoint{0.100000in}{0.212622in}}{\pgfqpoint{3.696000in}{3.696000in}}%
\pgfusepath{clip}%
\pgfsetbuttcap%
\pgfsetroundjoin%
\definecolor{currentfill}{rgb}{0.121569,0.466667,0.705882}%
\pgfsetfillcolor{currentfill}%
\pgfsetfillopacity{0.300003}%
\pgfsetlinewidth{1.003750pt}%
\definecolor{currentstroke}{rgb}{0.121569,0.466667,0.705882}%
\pgfsetstrokecolor{currentstroke}%
\pgfsetstrokeopacity{0.300003}%
\pgfsetdash{}{0pt}%
\pgfpathmoveto{\pgfqpoint{1.136982in}{1.617747in}}%
\pgfpathcurveto{\pgfqpoint{1.145219in}{1.617747in}}{\pgfqpoint{1.153119in}{1.621019in}}{\pgfqpoint{1.158943in}{1.626843in}}%
\pgfpathcurveto{\pgfqpoint{1.164767in}{1.632667in}}{\pgfqpoint{1.168039in}{1.640567in}}{\pgfqpoint{1.168039in}{1.648803in}}%
\pgfpathcurveto{\pgfqpoint{1.168039in}{1.657039in}}{\pgfqpoint{1.164767in}{1.664939in}}{\pgfqpoint{1.158943in}{1.670763in}}%
\pgfpathcurveto{\pgfqpoint{1.153119in}{1.676587in}}{\pgfqpoint{1.145219in}{1.679860in}}{\pgfqpoint{1.136982in}{1.679860in}}%
\pgfpathcurveto{\pgfqpoint{1.128746in}{1.679860in}}{\pgfqpoint{1.120846in}{1.676587in}}{\pgfqpoint{1.115022in}{1.670763in}}%
\pgfpathcurveto{\pgfqpoint{1.109198in}{1.664939in}}{\pgfqpoint{1.105926in}{1.657039in}}{\pgfqpoint{1.105926in}{1.648803in}}%
\pgfpathcurveto{\pgfqpoint{1.105926in}{1.640567in}}{\pgfqpoint{1.109198in}{1.632667in}}{\pgfqpoint{1.115022in}{1.626843in}}%
\pgfpathcurveto{\pgfqpoint{1.120846in}{1.621019in}}{\pgfqpoint{1.128746in}{1.617747in}}{\pgfqpoint{1.136982in}{1.617747in}}%
\pgfpathclose%
\pgfusepath{stroke,fill}%
\end{pgfscope}%
\begin{pgfscope}%
\pgfpathrectangle{\pgfqpoint{0.100000in}{0.212622in}}{\pgfqpoint{3.696000in}{3.696000in}}%
\pgfusepath{clip}%
\pgfsetbuttcap%
\pgfsetroundjoin%
\definecolor{currentfill}{rgb}{0.121569,0.466667,0.705882}%
\pgfsetfillcolor{currentfill}%
\pgfsetfillopacity{0.300003}%
\pgfsetlinewidth{1.003750pt}%
\definecolor{currentstroke}{rgb}{0.121569,0.466667,0.705882}%
\pgfsetstrokecolor{currentstroke}%
\pgfsetstrokeopacity{0.300003}%
\pgfsetdash{}{0pt}%
\pgfpathmoveto{\pgfqpoint{1.136982in}{1.617747in}}%
\pgfpathcurveto{\pgfqpoint{1.145219in}{1.617747in}}{\pgfqpoint{1.153119in}{1.621019in}}{\pgfqpoint{1.158943in}{1.626843in}}%
\pgfpathcurveto{\pgfqpoint{1.164767in}{1.632667in}}{\pgfqpoint{1.168039in}{1.640567in}}{\pgfqpoint{1.168039in}{1.648803in}}%
\pgfpathcurveto{\pgfqpoint{1.168039in}{1.657039in}}{\pgfqpoint{1.164767in}{1.664939in}}{\pgfqpoint{1.158943in}{1.670763in}}%
\pgfpathcurveto{\pgfqpoint{1.153119in}{1.676587in}}{\pgfqpoint{1.145219in}{1.679860in}}{\pgfqpoint{1.136982in}{1.679860in}}%
\pgfpathcurveto{\pgfqpoint{1.128746in}{1.679860in}}{\pgfqpoint{1.120846in}{1.676587in}}{\pgfqpoint{1.115022in}{1.670763in}}%
\pgfpathcurveto{\pgfqpoint{1.109198in}{1.664939in}}{\pgfqpoint{1.105926in}{1.657039in}}{\pgfqpoint{1.105926in}{1.648803in}}%
\pgfpathcurveto{\pgfqpoint{1.105926in}{1.640567in}}{\pgfqpoint{1.109198in}{1.632667in}}{\pgfqpoint{1.115022in}{1.626843in}}%
\pgfpathcurveto{\pgfqpoint{1.120846in}{1.621019in}}{\pgfqpoint{1.128746in}{1.617747in}}{\pgfqpoint{1.136982in}{1.617747in}}%
\pgfpathclose%
\pgfusepath{stroke,fill}%
\end{pgfscope}%
\begin{pgfscope}%
\pgfpathrectangle{\pgfqpoint{0.100000in}{0.212622in}}{\pgfqpoint{3.696000in}{3.696000in}}%
\pgfusepath{clip}%
\pgfsetbuttcap%
\pgfsetroundjoin%
\definecolor{currentfill}{rgb}{0.121569,0.466667,0.705882}%
\pgfsetfillcolor{currentfill}%
\pgfsetfillopacity{0.300003}%
\pgfsetlinewidth{1.003750pt}%
\definecolor{currentstroke}{rgb}{0.121569,0.466667,0.705882}%
\pgfsetstrokecolor{currentstroke}%
\pgfsetstrokeopacity{0.300003}%
\pgfsetdash{}{0pt}%
\pgfpathmoveto{\pgfqpoint{1.136982in}{1.617747in}}%
\pgfpathcurveto{\pgfqpoint{1.145219in}{1.617747in}}{\pgfqpoint{1.153119in}{1.621019in}}{\pgfqpoint{1.158943in}{1.626843in}}%
\pgfpathcurveto{\pgfqpoint{1.164767in}{1.632667in}}{\pgfqpoint{1.168039in}{1.640567in}}{\pgfqpoint{1.168039in}{1.648803in}}%
\pgfpathcurveto{\pgfqpoint{1.168039in}{1.657039in}}{\pgfqpoint{1.164767in}{1.664939in}}{\pgfqpoint{1.158943in}{1.670763in}}%
\pgfpathcurveto{\pgfqpoint{1.153119in}{1.676587in}}{\pgfqpoint{1.145219in}{1.679860in}}{\pgfqpoint{1.136982in}{1.679860in}}%
\pgfpathcurveto{\pgfqpoint{1.128746in}{1.679860in}}{\pgfqpoint{1.120846in}{1.676587in}}{\pgfqpoint{1.115022in}{1.670763in}}%
\pgfpathcurveto{\pgfqpoint{1.109198in}{1.664939in}}{\pgfqpoint{1.105926in}{1.657039in}}{\pgfqpoint{1.105926in}{1.648803in}}%
\pgfpathcurveto{\pgfqpoint{1.105926in}{1.640567in}}{\pgfqpoint{1.109198in}{1.632667in}}{\pgfqpoint{1.115022in}{1.626843in}}%
\pgfpathcurveto{\pgfqpoint{1.120846in}{1.621019in}}{\pgfqpoint{1.128746in}{1.617747in}}{\pgfqpoint{1.136982in}{1.617747in}}%
\pgfpathclose%
\pgfusepath{stroke,fill}%
\end{pgfscope}%
\begin{pgfscope}%
\pgfpathrectangle{\pgfqpoint{0.100000in}{0.212622in}}{\pgfqpoint{3.696000in}{3.696000in}}%
\pgfusepath{clip}%
\pgfsetbuttcap%
\pgfsetroundjoin%
\definecolor{currentfill}{rgb}{0.121569,0.466667,0.705882}%
\pgfsetfillcolor{currentfill}%
\pgfsetfillopacity{0.300003}%
\pgfsetlinewidth{1.003750pt}%
\definecolor{currentstroke}{rgb}{0.121569,0.466667,0.705882}%
\pgfsetstrokecolor{currentstroke}%
\pgfsetstrokeopacity{0.300003}%
\pgfsetdash{}{0pt}%
\pgfpathmoveto{\pgfqpoint{1.136982in}{1.617747in}}%
\pgfpathcurveto{\pgfqpoint{1.145219in}{1.617747in}}{\pgfqpoint{1.153119in}{1.621019in}}{\pgfqpoint{1.158943in}{1.626843in}}%
\pgfpathcurveto{\pgfqpoint{1.164767in}{1.632667in}}{\pgfqpoint{1.168039in}{1.640567in}}{\pgfqpoint{1.168039in}{1.648803in}}%
\pgfpathcurveto{\pgfqpoint{1.168039in}{1.657039in}}{\pgfqpoint{1.164767in}{1.664939in}}{\pgfqpoint{1.158943in}{1.670763in}}%
\pgfpathcurveto{\pgfqpoint{1.153119in}{1.676587in}}{\pgfqpoint{1.145219in}{1.679860in}}{\pgfqpoint{1.136982in}{1.679860in}}%
\pgfpathcurveto{\pgfqpoint{1.128746in}{1.679860in}}{\pgfqpoint{1.120846in}{1.676587in}}{\pgfqpoint{1.115022in}{1.670763in}}%
\pgfpathcurveto{\pgfqpoint{1.109198in}{1.664939in}}{\pgfqpoint{1.105926in}{1.657039in}}{\pgfqpoint{1.105926in}{1.648803in}}%
\pgfpathcurveto{\pgfqpoint{1.105926in}{1.640567in}}{\pgfqpoint{1.109198in}{1.632667in}}{\pgfqpoint{1.115022in}{1.626843in}}%
\pgfpathcurveto{\pgfqpoint{1.120846in}{1.621019in}}{\pgfqpoint{1.128746in}{1.617747in}}{\pgfqpoint{1.136982in}{1.617747in}}%
\pgfpathclose%
\pgfusepath{stroke,fill}%
\end{pgfscope}%
\begin{pgfscope}%
\pgfpathrectangle{\pgfqpoint{0.100000in}{0.212622in}}{\pgfqpoint{3.696000in}{3.696000in}}%
\pgfusepath{clip}%
\pgfsetbuttcap%
\pgfsetroundjoin%
\definecolor{currentfill}{rgb}{0.121569,0.466667,0.705882}%
\pgfsetfillcolor{currentfill}%
\pgfsetfillopacity{0.300003}%
\pgfsetlinewidth{1.003750pt}%
\definecolor{currentstroke}{rgb}{0.121569,0.466667,0.705882}%
\pgfsetstrokecolor{currentstroke}%
\pgfsetstrokeopacity{0.300003}%
\pgfsetdash{}{0pt}%
\pgfpathmoveto{\pgfqpoint{1.136982in}{1.617747in}}%
\pgfpathcurveto{\pgfqpoint{1.145219in}{1.617747in}}{\pgfqpoint{1.153119in}{1.621019in}}{\pgfqpoint{1.158943in}{1.626843in}}%
\pgfpathcurveto{\pgfqpoint{1.164767in}{1.632667in}}{\pgfqpoint{1.168039in}{1.640567in}}{\pgfqpoint{1.168039in}{1.648803in}}%
\pgfpathcurveto{\pgfqpoint{1.168039in}{1.657039in}}{\pgfqpoint{1.164767in}{1.664939in}}{\pgfqpoint{1.158943in}{1.670763in}}%
\pgfpathcurveto{\pgfqpoint{1.153119in}{1.676587in}}{\pgfqpoint{1.145219in}{1.679860in}}{\pgfqpoint{1.136982in}{1.679860in}}%
\pgfpathcurveto{\pgfqpoint{1.128746in}{1.679860in}}{\pgfqpoint{1.120846in}{1.676587in}}{\pgfqpoint{1.115022in}{1.670763in}}%
\pgfpathcurveto{\pgfqpoint{1.109198in}{1.664939in}}{\pgfqpoint{1.105926in}{1.657039in}}{\pgfqpoint{1.105926in}{1.648803in}}%
\pgfpathcurveto{\pgfqpoint{1.105926in}{1.640567in}}{\pgfqpoint{1.109198in}{1.632667in}}{\pgfqpoint{1.115022in}{1.626843in}}%
\pgfpathcurveto{\pgfqpoint{1.120846in}{1.621019in}}{\pgfqpoint{1.128746in}{1.617747in}}{\pgfqpoint{1.136982in}{1.617747in}}%
\pgfpathclose%
\pgfusepath{stroke,fill}%
\end{pgfscope}%
\begin{pgfscope}%
\pgfpathrectangle{\pgfqpoint{0.100000in}{0.212622in}}{\pgfqpoint{3.696000in}{3.696000in}}%
\pgfusepath{clip}%
\pgfsetbuttcap%
\pgfsetroundjoin%
\definecolor{currentfill}{rgb}{0.121569,0.466667,0.705882}%
\pgfsetfillcolor{currentfill}%
\pgfsetfillopacity{0.300003}%
\pgfsetlinewidth{1.003750pt}%
\definecolor{currentstroke}{rgb}{0.121569,0.466667,0.705882}%
\pgfsetstrokecolor{currentstroke}%
\pgfsetstrokeopacity{0.300003}%
\pgfsetdash{}{0pt}%
\pgfpathmoveto{\pgfqpoint{1.136982in}{1.617747in}}%
\pgfpathcurveto{\pgfqpoint{1.145219in}{1.617747in}}{\pgfqpoint{1.153119in}{1.621019in}}{\pgfqpoint{1.158943in}{1.626843in}}%
\pgfpathcurveto{\pgfqpoint{1.164767in}{1.632667in}}{\pgfqpoint{1.168039in}{1.640567in}}{\pgfqpoint{1.168039in}{1.648803in}}%
\pgfpathcurveto{\pgfqpoint{1.168039in}{1.657039in}}{\pgfqpoint{1.164767in}{1.664939in}}{\pgfqpoint{1.158943in}{1.670763in}}%
\pgfpathcurveto{\pgfqpoint{1.153119in}{1.676587in}}{\pgfqpoint{1.145219in}{1.679860in}}{\pgfqpoint{1.136982in}{1.679860in}}%
\pgfpathcurveto{\pgfqpoint{1.128746in}{1.679860in}}{\pgfqpoint{1.120846in}{1.676587in}}{\pgfqpoint{1.115022in}{1.670763in}}%
\pgfpathcurveto{\pgfqpoint{1.109198in}{1.664939in}}{\pgfqpoint{1.105926in}{1.657039in}}{\pgfqpoint{1.105926in}{1.648803in}}%
\pgfpathcurveto{\pgfqpoint{1.105926in}{1.640567in}}{\pgfqpoint{1.109198in}{1.632667in}}{\pgfqpoint{1.115022in}{1.626843in}}%
\pgfpathcurveto{\pgfqpoint{1.120846in}{1.621019in}}{\pgfqpoint{1.128746in}{1.617747in}}{\pgfqpoint{1.136982in}{1.617747in}}%
\pgfpathclose%
\pgfusepath{stroke,fill}%
\end{pgfscope}%
\begin{pgfscope}%
\pgfpathrectangle{\pgfqpoint{0.100000in}{0.212622in}}{\pgfqpoint{3.696000in}{3.696000in}}%
\pgfusepath{clip}%
\pgfsetbuttcap%
\pgfsetroundjoin%
\definecolor{currentfill}{rgb}{0.121569,0.466667,0.705882}%
\pgfsetfillcolor{currentfill}%
\pgfsetfillopacity{0.300003}%
\pgfsetlinewidth{1.003750pt}%
\definecolor{currentstroke}{rgb}{0.121569,0.466667,0.705882}%
\pgfsetstrokecolor{currentstroke}%
\pgfsetstrokeopacity{0.300003}%
\pgfsetdash{}{0pt}%
\pgfpathmoveto{\pgfqpoint{1.136982in}{1.617747in}}%
\pgfpathcurveto{\pgfqpoint{1.145219in}{1.617747in}}{\pgfqpoint{1.153119in}{1.621019in}}{\pgfqpoint{1.158943in}{1.626843in}}%
\pgfpathcurveto{\pgfqpoint{1.164767in}{1.632667in}}{\pgfqpoint{1.168039in}{1.640567in}}{\pgfqpoint{1.168039in}{1.648803in}}%
\pgfpathcurveto{\pgfqpoint{1.168039in}{1.657039in}}{\pgfqpoint{1.164767in}{1.664939in}}{\pgfqpoint{1.158943in}{1.670763in}}%
\pgfpathcurveto{\pgfqpoint{1.153119in}{1.676587in}}{\pgfqpoint{1.145219in}{1.679860in}}{\pgfqpoint{1.136982in}{1.679860in}}%
\pgfpathcurveto{\pgfqpoint{1.128746in}{1.679860in}}{\pgfqpoint{1.120846in}{1.676587in}}{\pgfqpoint{1.115022in}{1.670763in}}%
\pgfpathcurveto{\pgfqpoint{1.109198in}{1.664939in}}{\pgfqpoint{1.105926in}{1.657039in}}{\pgfqpoint{1.105926in}{1.648803in}}%
\pgfpathcurveto{\pgfqpoint{1.105926in}{1.640567in}}{\pgfqpoint{1.109198in}{1.632667in}}{\pgfqpoint{1.115022in}{1.626843in}}%
\pgfpathcurveto{\pgfqpoint{1.120846in}{1.621019in}}{\pgfqpoint{1.128746in}{1.617747in}}{\pgfqpoint{1.136982in}{1.617747in}}%
\pgfpathclose%
\pgfusepath{stroke,fill}%
\end{pgfscope}%
\begin{pgfscope}%
\pgfpathrectangle{\pgfqpoint{0.100000in}{0.212622in}}{\pgfqpoint{3.696000in}{3.696000in}}%
\pgfusepath{clip}%
\pgfsetbuttcap%
\pgfsetroundjoin%
\definecolor{currentfill}{rgb}{0.121569,0.466667,0.705882}%
\pgfsetfillcolor{currentfill}%
\pgfsetfillopacity{0.300003}%
\pgfsetlinewidth{1.003750pt}%
\definecolor{currentstroke}{rgb}{0.121569,0.466667,0.705882}%
\pgfsetstrokecolor{currentstroke}%
\pgfsetstrokeopacity{0.300003}%
\pgfsetdash{}{0pt}%
\pgfpathmoveto{\pgfqpoint{1.136982in}{1.617747in}}%
\pgfpathcurveto{\pgfqpoint{1.145219in}{1.617747in}}{\pgfqpoint{1.153119in}{1.621019in}}{\pgfqpoint{1.158943in}{1.626843in}}%
\pgfpathcurveto{\pgfqpoint{1.164767in}{1.632667in}}{\pgfqpoint{1.168039in}{1.640567in}}{\pgfqpoint{1.168039in}{1.648803in}}%
\pgfpathcurveto{\pgfqpoint{1.168039in}{1.657039in}}{\pgfqpoint{1.164767in}{1.664939in}}{\pgfqpoint{1.158943in}{1.670763in}}%
\pgfpathcurveto{\pgfqpoint{1.153119in}{1.676587in}}{\pgfqpoint{1.145219in}{1.679860in}}{\pgfqpoint{1.136982in}{1.679860in}}%
\pgfpathcurveto{\pgfqpoint{1.128746in}{1.679860in}}{\pgfqpoint{1.120846in}{1.676587in}}{\pgfqpoint{1.115022in}{1.670763in}}%
\pgfpathcurveto{\pgfqpoint{1.109198in}{1.664939in}}{\pgfqpoint{1.105926in}{1.657039in}}{\pgfqpoint{1.105926in}{1.648803in}}%
\pgfpathcurveto{\pgfqpoint{1.105926in}{1.640567in}}{\pgfqpoint{1.109198in}{1.632667in}}{\pgfqpoint{1.115022in}{1.626843in}}%
\pgfpathcurveto{\pgfqpoint{1.120846in}{1.621019in}}{\pgfqpoint{1.128746in}{1.617747in}}{\pgfqpoint{1.136982in}{1.617747in}}%
\pgfpathclose%
\pgfusepath{stroke,fill}%
\end{pgfscope}%
\begin{pgfscope}%
\pgfpathrectangle{\pgfqpoint{0.100000in}{0.212622in}}{\pgfqpoint{3.696000in}{3.696000in}}%
\pgfusepath{clip}%
\pgfsetbuttcap%
\pgfsetroundjoin%
\definecolor{currentfill}{rgb}{0.121569,0.466667,0.705882}%
\pgfsetfillcolor{currentfill}%
\pgfsetfillopacity{0.300003}%
\pgfsetlinewidth{1.003750pt}%
\definecolor{currentstroke}{rgb}{0.121569,0.466667,0.705882}%
\pgfsetstrokecolor{currentstroke}%
\pgfsetstrokeopacity{0.300003}%
\pgfsetdash{}{0pt}%
\pgfpathmoveto{\pgfqpoint{1.136982in}{1.617747in}}%
\pgfpathcurveto{\pgfqpoint{1.145219in}{1.617747in}}{\pgfqpoint{1.153119in}{1.621019in}}{\pgfqpoint{1.158943in}{1.626843in}}%
\pgfpathcurveto{\pgfqpoint{1.164767in}{1.632667in}}{\pgfqpoint{1.168039in}{1.640567in}}{\pgfqpoint{1.168039in}{1.648803in}}%
\pgfpathcurveto{\pgfqpoint{1.168039in}{1.657039in}}{\pgfqpoint{1.164767in}{1.664939in}}{\pgfqpoint{1.158943in}{1.670763in}}%
\pgfpathcurveto{\pgfqpoint{1.153119in}{1.676587in}}{\pgfqpoint{1.145219in}{1.679860in}}{\pgfqpoint{1.136982in}{1.679860in}}%
\pgfpathcurveto{\pgfqpoint{1.128746in}{1.679860in}}{\pgfqpoint{1.120846in}{1.676587in}}{\pgfqpoint{1.115022in}{1.670763in}}%
\pgfpathcurveto{\pgfqpoint{1.109198in}{1.664939in}}{\pgfqpoint{1.105926in}{1.657039in}}{\pgfqpoint{1.105926in}{1.648803in}}%
\pgfpathcurveto{\pgfqpoint{1.105926in}{1.640567in}}{\pgfqpoint{1.109198in}{1.632667in}}{\pgfqpoint{1.115022in}{1.626843in}}%
\pgfpathcurveto{\pgfqpoint{1.120846in}{1.621019in}}{\pgfqpoint{1.128746in}{1.617747in}}{\pgfqpoint{1.136982in}{1.617747in}}%
\pgfpathclose%
\pgfusepath{stroke,fill}%
\end{pgfscope}%
\begin{pgfscope}%
\pgfpathrectangle{\pgfqpoint{0.100000in}{0.212622in}}{\pgfqpoint{3.696000in}{3.696000in}}%
\pgfusepath{clip}%
\pgfsetbuttcap%
\pgfsetroundjoin%
\definecolor{currentfill}{rgb}{0.121569,0.466667,0.705882}%
\pgfsetfillcolor{currentfill}%
\pgfsetfillopacity{0.300003}%
\pgfsetlinewidth{1.003750pt}%
\definecolor{currentstroke}{rgb}{0.121569,0.466667,0.705882}%
\pgfsetstrokecolor{currentstroke}%
\pgfsetstrokeopacity{0.300003}%
\pgfsetdash{}{0pt}%
\pgfpathmoveto{\pgfqpoint{1.136982in}{1.617747in}}%
\pgfpathcurveto{\pgfqpoint{1.145219in}{1.617747in}}{\pgfqpoint{1.153119in}{1.621019in}}{\pgfqpoint{1.158943in}{1.626843in}}%
\pgfpathcurveto{\pgfqpoint{1.164767in}{1.632667in}}{\pgfqpoint{1.168039in}{1.640567in}}{\pgfqpoint{1.168039in}{1.648803in}}%
\pgfpathcurveto{\pgfqpoint{1.168039in}{1.657039in}}{\pgfqpoint{1.164767in}{1.664939in}}{\pgfqpoint{1.158943in}{1.670763in}}%
\pgfpathcurveto{\pgfqpoint{1.153119in}{1.676587in}}{\pgfqpoint{1.145219in}{1.679860in}}{\pgfqpoint{1.136982in}{1.679860in}}%
\pgfpathcurveto{\pgfqpoint{1.128746in}{1.679860in}}{\pgfqpoint{1.120846in}{1.676587in}}{\pgfqpoint{1.115022in}{1.670763in}}%
\pgfpathcurveto{\pgfqpoint{1.109198in}{1.664939in}}{\pgfqpoint{1.105926in}{1.657039in}}{\pgfqpoint{1.105926in}{1.648803in}}%
\pgfpathcurveto{\pgfqpoint{1.105926in}{1.640567in}}{\pgfqpoint{1.109198in}{1.632667in}}{\pgfqpoint{1.115022in}{1.626843in}}%
\pgfpathcurveto{\pgfqpoint{1.120846in}{1.621019in}}{\pgfqpoint{1.128746in}{1.617747in}}{\pgfqpoint{1.136982in}{1.617747in}}%
\pgfpathclose%
\pgfusepath{stroke,fill}%
\end{pgfscope}%
\begin{pgfscope}%
\pgfpathrectangle{\pgfqpoint{0.100000in}{0.212622in}}{\pgfqpoint{3.696000in}{3.696000in}}%
\pgfusepath{clip}%
\pgfsetbuttcap%
\pgfsetroundjoin%
\definecolor{currentfill}{rgb}{0.121569,0.466667,0.705882}%
\pgfsetfillcolor{currentfill}%
\pgfsetfillopacity{0.300003}%
\pgfsetlinewidth{1.003750pt}%
\definecolor{currentstroke}{rgb}{0.121569,0.466667,0.705882}%
\pgfsetstrokecolor{currentstroke}%
\pgfsetstrokeopacity{0.300003}%
\pgfsetdash{}{0pt}%
\pgfpathmoveto{\pgfqpoint{1.136982in}{1.617747in}}%
\pgfpathcurveto{\pgfqpoint{1.145219in}{1.617747in}}{\pgfqpoint{1.153119in}{1.621019in}}{\pgfqpoint{1.158943in}{1.626843in}}%
\pgfpathcurveto{\pgfqpoint{1.164767in}{1.632667in}}{\pgfqpoint{1.168039in}{1.640567in}}{\pgfqpoint{1.168039in}{1.648803in}}%
\pgfpathcurveto{\pgfqpoint{1.168039in}{1.657039in}}{\pgfqpoint{1.164767in}{1.664939in}}{\pgfqpoint{1.158943in}{1.670763in}}%
\pgfpathcurveto{\pgfqpoint{1.153119in}{1.676587in}}{\pgfqpoint{1.145219in}{1.679860in}}{\pgfqpoint{1.136982in}{1.679860in}}%
\pgfpathcurveto{\pgfqpoint{1.128746in}{1.679860in}}{\pgfqpoint{1.120846in}{1.676587in}}{\pgfqpoint{1.115022in}{1.670763in}}%
\pgfpathcurveto{\pgfqpoint{1.109198in}{1.664939in}}{\pgfqpoint{1.105926in}{1.657039in}}{\pgfqpoint{1.105926in}{1.648803in}}%
\pgfpathcurveto{\pgfqpoint{1.105926in}{1.640567in}}{\pgfqpoint{1.109198in}{1.632667in}}{\pgfqpoint{1.115022in}{1.626843in}}%
\pgfpathcurveto{\pgfqpoint{1.120846in}{1.621019in}}{\pgfqpoint{1.128746in}{1.617747in}}{\pgfqpoint{1.136982in}{1.617747in}}%
\pgfpathclose%
\pgfusepath{stroke,fill}%
\end{pgfscope}%
\begin{pgfscope}%
\pgfpathrectangle{\pgfqpoint{0.100000in}{0.212622in}}{\pgfqpoint{3.696000in}{3.696000in}}%
\pgfusepath{clip}%
\pgfsetbuttcap%
\pgfsetroundjoin%
\definecolor{currentfill}{rgb}{0.121569,0.466667,0.705882}%
\pgfsetfillcolor{currentfill}%
\pgfsetfillopacity{0.300003}%
\pgfsetlinewidth{1.003750pt}%
\definecolor{currentstroke}{rgb}{0.121569,0.466667,0.705882}%
\pgfsetstrokecolor{currentstroke}%
\pgfsetstrokeopacity{0.300003}%
\pgfsetdash{}{0pt}%
\pgfpathmoveto{\pgfqpoint{1.136982in}{1.617747in}}%
\pgfpathcurveto{\pgfqpoint{1.145219in}{1.617747in}}{\pgfqpoint{1.153119in}{1.621019in}}{\pgfqpoint{1.158943in}{1.626843in}}%
\pgfpathcurveto{\pgfqpoint{1.164767in}{1.632667in}}{\pgfqpoint{1.168039in}{1.640567in}}{\pgfqpoint{1.168039in}{1.648803in}}%
\pgfpathcurveto{\pgfqpoint{1.168039in}{1.657039in}}{\pgfqpoint{1.164767in}{1.664939in}}{\pgfqpoint{1.158943in}{1.670763in}}%
\pgfpathcurveto{\pgfqpoint{1.153119in}{1.676587in}}{\pgfqpoint{1.145219in}{1.679860in}}{\pgfqpoint{1.136982in}{1.679860in}}%
\pgfpathcurveto{\pgfqpoint{1.128746in}{1.679860in}}{\pgfqpoint{1.120846in}{1.676587in}}{\pgfqpoint{1.115022in}{1.670763in}}%
\pgfpathcurveto{\pgfqpoint{1.109198in}{1.664939in}}{\pgfqpoint{1.105926in}{1.657039in}}{\pgfqpoint{1.105926in}{1.648803in}}%
\pgfpathcurveto{\pgfqpoint{1.105926in}{1.640567in}}{\pgfqpoint{1.109198in}{1.632667in}}{\pgfqpoint{1.115022in}{1.626843in}}%
\pgfpathcurveto{\pgfqpoint{1.120846in}{1.621019in}}{\pgfqpoint{1.128746in}{1.617747in}}{\pgfqpoint{1.136982in}{1.617747in}}%
\pgfpathclose%
\pgfusepath{stroke,fill}%
\end{pgfscope}%
\begin{pgfscope}%
\pgfpathrectangle{\pgfqpoint{0.100000in}{0.212622in}}{\pgfqpoint{3.696000in}{3.696000in}}%
\pgfusepath{clip}%
\pgfsetbuttcap%
\pgfsetroundjoin%
\definecolor{currentfill}{rgb}{0.121569,0.466667,0.705882}%
\pgfsetfillcolor{currentfill}%
\pgfsetfillopacity{0.300003}%
\pgfsetlinewidth{1.003750pt}%
\definecolor{currentstroke}{rgb}{0.121569,0.466667,0.705882}%
\pgfsetstrokecolor{currentstroke}%
\pgfsetstrokeopacity{0.300003}%
\pgfsetdash{}{0pt}%
\pgfpathmoveto{\pgfqpoint{1.136982in}{1.617747in}}%
\pgfpathcurveto{\pgfqpoint{1.145219in}{1.617747in}}{\pgfqpoint{1.153119in}{1.621019in}}{\pgfqpoint{1.158943in}{1.626843in}}%
\pgfpathcurveto{\pgfqpoint{1.164767in}{1.632667in}}{\pgfqpoint{1.168039in}{1.640567in}}{\pgfqpoint{1.168039in}{1.648803in}}%
\pgfpathcurveto{\pgfqpoint{1.168039in}{1.657039in}}{\pgfqpoint{1.164767in}{1.664939in}}{\pgfqpoint{1.158943in}{1.670763in}}%
\pgfpathcurveto{\pgfqpoint{1.153119in}{1.676587in}}{\pgfqpoint{1.145219in}{1.679860in}}{\pgfqpoint{1.136982in}{1.679860in}}%
\pgfpathcurveto{\pgfqpoint{1.128746in}{1.679860in}}{\pgfqpoint{1.120846in}{1.676587in}}{\pgfqpoint{1.115022in}{1.670763in}}%
\pgfpathcurveto{\pgfqpoint{1.109198in}{1.664939in}}{\pgfqpoint{1.105926in}{1.657039in}}{\pgfqpoint{1.105926in}{1.648803in}}%
\pgfpathcurveto{\pgfqpoint{1.105926in}{1.640567in}}{\pgfqpoint{1.109198in}{1.632667in}}{\pgfqpoint{1.115022in}{1.626843in}}%
\pgfpathcurveto{\pgfqpoint{1.120846in}{1.621019in}}{\pgfqpoint{1.128746in}{1.617747in}}{\pgfqpoint{1.136982in}{1.617747in}}%
\pgfpathclose%
\pgfusepath{stroke,fill}%
\end{pgfscope}%
\begin{pgfscope}%
\pgfpathrectangle{\pgfqpoint{0.100000in}{0.212622in}}{\pgfqpoint{3.696000in}{3.696000in}}%
\pgfusepath{clip}%
\pgfsetbuttcap%
\pgfsetroundjoin%
\definecolor{currentfill}{rgb}{0.121569,0.466667,0.705882}%
\pgfsetfillcolor{currentfill}%
\pgfsetfillopacity{0.300003}%
\pgfsetlinewidth{1.003750pt}%
\definecolor{currentstroke}{rgb}{0.121569,0.466667,0.705882}%
\pgfsetstrokecolor{currentstroke}%
\pgfsetstrokeopacity{0.300003}%
\pgfsetdash{}{0pt}%
\pgfpathmoveto{\pgfqpoint{1.136982in}{1.617747in}}%
\pgfpathcurveto{\pgfqpoint{1.145219in}{1.617747in}}{\pgfqpoint{1.153119in}{1.621019in}}{\pgfqpoint{1.158943in}{1.626843in}}%
\pgfpathcurveto{\pgfqpoint{1.164767in}{1.632667in}}{\pgfqpoint{1.168039in}{1.640567in}}{\pgfqpoint{1.168039in}{1.648803in}}%
\pgfpathcurveto{\pgfqpoint{1.168039in}{1.657039in}}{\pgfqpoint{1.164767in}{1.664939in}}{\pgfqpoint{1.158943in}{1.670763in}}%
\pgfpathcurveto{\pgfqpoint{1.153119in}{1.676587in}}{\pgfqpoint{1.145219in}{1.679860in}}{\pgfqpoint{1.136982in}{1.679860in}}%
\pgfpathcurveto{\pgfqpoint{1.128746in}{1.679860in}}{\pgfqpoint{1.120846in}{1.676587in}}{\pgfqpoint{1.115022in}{1.670763in}}%
\pgfpathcurveto{\pgfqpoint{1.109198in}{1.664939in}}{\pgfqpoint{1.105926in}{1.657039in}}{\pgfqpoint{1.105926in}{1.648803in}}%
\pgfpathcurveto{\pgfqpoint{1.105926in}{1.640567in}}{\pgfqpoint{1.109198in}{1.632667in}}{\pgfqpoint{1.115022in}{1.626843in}}%
\pgfpathcurveto{\pgfqpoint{1.120846in}{1.621019in}}{\pgfqpoint{1.128746in}{1.617747in}}{\pgfqpoint{1.136982in}{1.617747in}}%
\pgfpathclose%
\pgfusepath{stroke,fill}%
\end{pgfscope}%
\begin{pgfscope}%
\pgfpathrectangle{\pgfqpoint{0.100000in}{0.212622in}}{\pgfqpoint{3.696000in}{3.696000in}}%
\pgfusepath{clip}%
\pgfsetbuttcap%
\pgfsetroundjoin%
\definecolor{currentfill}{rgb}{0.121569,0.466667,0.705882}%
\pgfsetfillcolor{currentfill}%
\pgfsetfillopacity{0.300003}%
\pgfsetlinewidth{1.003750pt}%
\definecolor{currentstroke}{rgb}{0.121569,0.466667,0.705882}%
\pgfsetstrokecolor{currentstroke}%
\pgfsetstrokeopacity{0.300003}%
\pgfsetdash{}{0pt}%
\pgfpathmoveto{\pgfqpoint{1.136982in}{1.617747in}}%
\pgfpathcurveto{\pgfqpoint{1.145219in}{1.617747in}}{\pgfqpoint{1.153119in}{1.621019in}}{\pgfqpoint{1.158943in}{1.626843in}}%
\pgfpathcurveto{\pgfqpoint{1.164767in}{1.632667in}}{\pgfqpoint{1.168039in}{1.640567in}}{\pgfqpoint{1.168039in}{1.648803in}}%
\pgfpathcurveto{\pgfqpoint{1.168039in}{1.657039in}}{\pgfqpoint{1.164767in}{1.664939in}}{\pgfqpoint{1.158943in}{1.670763in}}%
\pgfpathcurveto{\pgfqpoint{1.153119in}{1.676587in}}{\pgfqpoint{1.145219in}{1.679860in}}{\pgfqpoint{1.136982in}{1.679860in}}%
\pgfpathcurveto{\pgfqpoint{1.128746in}{1.679860in}}{\pgfqpoint{1.120846in}{1.676587in}}{\pgfqpoint{1.115022in}{1.670763in}}%
\pgfpathcurveto{\pgfqpoint{1.109198in}{1.664939in}}{\pgfqpoint{1.105926in}{1.657039in}}{\pgfqpoint{1.105926in}{1.648803in}}%
\pgfpathcurveto{\pgfqpoint{1.105926in}{1.640567in}}{\pgfqpoint{1.109198in}{1.632667in}}{\pgfqpoint{1.115022in}{1.626843in}}%
\pgfpathcurveto{\pgfqpoint{1.120846in}{1.621019in}}{\pgfqpoint{1.128746in}{1.617747in}}{\pgfqpoint{1.136982in}{1.617747in}}%
\pgfpathclose%
\pgfusepath{stroke,fill}%
\end{pgfscope}%
\begin{pgfscope}%
\pgfpathrectangle{\pgfqpoint{0.100000in}{0.212622in}}{\pgfqpoint{3.696000in}{3.696000in}}%
\pgfusepath{clip}%
\pgfsetbuttcap%
\pgfsetroundjoin%
\definecolor{currentfill}{rgb}{0.121569,0.466667,0.705882}%
\pgfsetfillcolor{currentfill}%
\pgfsetfillopacity{0.300003}%
\pgfsetlinewidth{1.003750pt}%
\definecolor{currentstroke}{rgb}{0.121569,0.466667,0.705882}%
\pgfsetstrokecolor{currentstroke}%
\pgfsetstrokeopacity{0.300003}%
\pgfsetdash{}{0pt}%
\pgfpathmoveto{\pgfqpoint{1.136982in}{1.617747in}}%
\pgfpathcurveto{\pgfqpoint{1.145219in}{1.617747in}}{\pgfqpoint{1.153119in}{1.621019in}}{\pgfqpoint{1.158943in}{1.626843in}}%
\pgfpathcurveto{\pgfqpoint{1.164767in}{1.632667in}}{\pgfqpoint{1.168039in}{1.640567in}}{\pgfqpoint{1.168039in}{1.648803in}}%
\pgfpathcurveto{\pgfqpoint{1.168039in}{1.657039in}}{\pgfqpoint{1.164767in}{1.664939in}}{\pgfqpoint{1.158943in}{1.670763in}}%
\pgfpathcurveto{\pgfqpoint{1.153119in}{1.676587in}}{\pgfqpoint{1.145219in}{1.679860in}}{\pgfqpoint{1.136982in}{1.679860in}}%
\pgfpathcurveto{\pgfqpoint{1.128746in}{1.679860in}}{\pgfqpoint{1.120846in}{1.676587in}}{\pgfqpoint{1.115022in}{1.670763in}}%
\pgfpathcurveto{\pgfqpoint{1.109198in}{1.664939in}}{\pgfqpoint{1.105926in}{1.657039in}}{\pgfqpoint{1.105926in}{1.648803in}}%
\pgfpathcurveto{\pgfqpoint{1.105926in}{1.640567in}}{\pgfqpoint{1.109198in}{1.632667in}}{\pgfqpoint{1.115022in}{1.626843in}}%
\pgfpathcurveto{\pgfqpoint{1.120846in}{1.621019in}}{\pgfqpoint{1.128746in}{1.617747in}}{\pgfqpoint{1.136982in}{1.617747in}}%
\pgfpathclose%
\pgfusepath{stroke,fill}%
\end{pgfscope}%
\begin{pgfscope}%
\pgfpathrectangle{\pgfqpoint{0.100000in}{0.212622in}}{\pgfqpoint{3.696000in}{3.696000in}}%
\pgfusepath{clip}%
\pgfsetbuttcap%
\pgfsetroundjoin%
\definecolor{currentfill}{rgb}{0.121569,0.466667,0.705882}%
\pgfsetfillcolor{currentfill}%
\pgfsetfillopacity{0.300003}%
\pgfsetlinewidth{1.003750pt}%
\definecolor{currentstroke}{rgb}{0.121569,0.466667,0.705882}%
\pgfsetstrokecolor{currentstroke}%
\pgfsetstrokeopacity{0.300003}%
\pgfsetdash{}{0pt}%
\pgfpathmoveto{\pgfqpoint{1.136982in}{1.617747in}}%
\pgfpathcurveto{\pgfqpoint{1.145219in}{1.617747in}}{\pgfqpoint{1.153119in}{1.621019in}}{\pgfqpoint{1.158943in}{1.626843in}}%
\pgfpathcurveto{\pgfqpoint{1.164767in}{1.632667in}}{\pgfqpoint{1.168039in}{1.640567in}}{\pgfqpoint{1.168039in}{1.648803in}}%
\pgfpathcurveto{\pgfqpoint{1.168039in}{1.657039in}}{\pgfqpoint{1.164767in}{1.664939in}}{\pgfqpoint{1.158943in}{1.670763in}}%
\pgfpathcurveto{\pgfqpoint{1.153119in}{1.676587in}}{\pgfqpoint{1.145219in}{1.679860in}}{\pgfqpoint{1.136982in}{1.679860in}}%
\pgfpathcurveto{\pgfqpoint{1.128746in}{1.679860in}}{\pgfqpoint{1.120846in}{1.676587in}}{\pgfqpoint{1.115022in}{1.670763in}}%
\pgfpathcurveto{\pgfqpoint{1.109198in}{1.664939in}}{\pgfqpoint{1.105926in}{1.657039in}}{\pgfqpoint{1.105926in}{1.648803in}}%
\pgfpathcurveto{\pgfqpoint{1.105926in}{1.640567in}}{\pgfqpoint{1.109198in}{1.632667in}}{\pgfqpoint{1.115022in}{1.626843in}}%
\pgfpathcurveto{\pgfqpoint{1.120846in}{1.621019in}}{\pgfqpoint{1.128746in}{1.617747in}}{\pgfqpoint{1.136982in}{1.617747in}}%
\pgfpathclose%
\pgfusepath{stroke,fill}%
\end{pgfscope}%
\begin{pgfscope}%
\pgfpathrectangle{\pgfqpoint{0.100000in}{0.212622in}}{\pgfqpoint{3.696000in}{3.696000in}}%
\pgfusepath{clip}%
\pgfsetbuttcap%
\pgfsetroundjoin%
\definecolor{currentfill}{rgb}{0.121569,0.466667,0.705882}%
\pgfsetfillcolor{currentfill}%
\pgfsetfillopacity{0.300003}%
\pgfsetlinewidth{1.003750pt}%
\definecolor{currentstroke}{rgb}{0.121569,0.466667,0.705882}%
\pgfsetstrokecolor{currentstroke}%
\pgfsetstrokeopacity{0.300003}%
\pgfsetdash{}{0pt}%
\pgfpathmoveto{\pgfqpoint{1.136982in}{1.617747in}}%
\pgfpathcurveto{\pgfqpoint{1.145219in}{1.617747in}}{\pgfqpoint{1.153119in}{1.621019in}}{\pgfqpoint{1.158943in}{1.626843in}}%
\pgfpathcurveto{\pgfqpoint{1.164767in}{1.632667in}}{\pgfqpoint{1.168039in}{1.640567in}}{\pgfqpoint{1.168039in}{1.648803in}}%
\pgfpathcurveto{\pgfqpoint{1.168039in}{1.657039in}}{\pgfqpoint{1.164767in}{1.664939in}}{\pgfqpoint{1.158943in}{1.670763in}}%
\pgfpathcurveto{\pgfqpoint{1.153119in}{1.676587in}}{\pgfqpoint{1.145219in}{1.679860in}}{\pgfqpoint{1.136982in}{1.679860in}}%
\pgfpathcurveto{\pgfqpoint{1.128746in}{1.679860in}}{\pgfqpoint{1.120846in}{1.676587in}}{\pgfqpoint{1.115022in}{1.670763in}}%
\pgfpathcurveto{\pgfqpoint{1.109198in}{1.664939in}}{\pgfqpoint{1.105926in}{1.657039in}}{\pgfqpoint{1.105926in}{1.648803in}}%
\pgfpathcurveto{\pgfqpoint{1.105926in}{1.640567in}}{\pgfqpoint{1.109198in}{1.632667in}}{\pgfqpoint{1.115022in}{1.626843in}}%
\pgfpathcurveto{\pgfqpoint{1.120846in}{1.621019in}}{\pgfqpoint{1.128746in}{1.617747in}}{\pgfqpoint{1.136982in}{1.617747in}}%
\pgfpathclose%
\pgfusepath{stroke,fill}%
\end{pgfscope}%
\begin{pgfscope}%
\pgfpathrectangle{\pgfqpoint{0.100000in}{0.212622in}}{\pgfqpoint{3.696000in}{3.696000in}}%
\pgfusepath{clip}%
\pgfsetbuttcap%
\pgfsetroundjoin%
\definecolor{currentfill}{rgb}{0.121569,0.466667,0.705882}%
\pgfsetfillcolor{currentfill}%
\pgfsetfillopacity{0.300003}%
\pgfsetlinewidth{1.003750pt}%
\definecolor{currentstroke}{rgb}{0.121569,0.466667,0.705882}%
\pgfsetstrokecolor{currentstroke}%
\pgfsetstrokeopacity{0.300003}%
\pgfsetdash{}{0pt}%
\pgfpathmoveto{\pgfqpoint{1.136982in}{1.617747in}}%
\pgfpathcurveto{\pgfqpoint{1.145219in}{1.617747in}}{\pgfqpoint{1.153119in}{1.621019in}}{\pgfqpoint{1.158943in}{1.626843in}}%
\pgfpathcurveto{\pgfqpoint{1.164767in}{1.632667in}}{\pgfqpoint{1.168039in}{1.640567in}}{\pgfqpoint{1.168039in}{1.648803in}}%
\pgfpathcurveto{\pgfqpoint{1.168039in}{1.657039in}}{\pgfqpoint{1.164767in}{1.664939in}}{\pgfqpoint{1.158943in}{1.670763in}}%
\pgfpathcurveto{\pgfqpoint{1.153119in}{1.676587in}}{\pgfqpoint{1.145219in}{1.679860in}}{\pgfqpoint{1.136982in}{1.679860in}}%
\pgfpathcurveto{\pgfqpoint{1.128746in}{1.679860in}}{\pgfqpoint{1.120846in}{1.676587in}}{\pgfqpoint{1.115022in}{1.670763in}}%
\pgfpathcurveto{\pgfqpoint{1.109198in}{1.664939in}}{\pgfqpoint{1.105926in}{1.657039in}}{\pgfqpoint{1.105926in}{1.648803in}}%
\pgfpathcurveto{\pgfqpoint{1.105926in}{1.640567in}}{\pgfqpoint{1.109198in}{1.632667in}}{\pgfqpoint{1.115022in}{1.626843in}}%
\pgfpathcurveto{\pgfqpoint{1.120846in}{1.621019in}}{\pgfqpoint{1.128746in}{1.617747in}}{\pgfqpoint{1.136982in}{1.617747in}}%
\pgfpathclose%
\pgfusepath{stroke,fill}%
\end{pgfscope}%
\begin{pgfscope}%
\pgfpathrectangle{\pgfqpoint{0.100000in}{0.212622in}}{\pgfqpoint{3.696000in}{3.696000in}}%
\pgfusepath{clip}%
\pgfsetbuttcap%
\pgfsetroundjoin%
\definecolor{currentfill}{rgb}{0.121569,0.466667,0.705882}%
\pgfsetfillcolor{currentfill}%
\pgfsetfillopacity{0.300003}%
\pgfsetlinewidth{1.003750pt}%
\definecolor{currentstroke}{rgb}{0.121569,0.466667,0.705882}%
\pgfsetstrokecolor{currentstroke}%
\pgfsetstrokeopacity{0.300003}%
\pgfsetdash{}{0pt}%
\pgfpathmoveto{\pgfqpoint{1.136982in}{1.617747in}}%
\pgfpathcurveto{\pgfqpoint{1.145219in}{1.617747in}}{\pgfqpoint{1.153119in}{1.621019in}}{\pgfqpoint{1.158943in}{1.626843in}}%
\pgfpathcurveto{\pgfqpoint{1.164767in}{1.632667in}}{\pgfqpoint{1.168039in}{1.640567in}}{\pgfqpoint{1.168039in}{1.648803in}}%
\pgfpathcurveto{\pgfqpoint{1.168039in}{1.657039in}}{\pgfqpoint{1.164767in}{1.664939in}}{\pgfqpoint{1.158943in}{1.670763in}}%
\pgfpathcurveto{\pgfqpoint{1.153119in}{1.676587in}}{\pgfqpoint{1.145219in}{1.679860in}}{\pgfqpoint{1.136982in}{1.679860in}}%
\pgfpathcurveto{\pgfqpoint{1.128746in}{1.679860in}}{\pgfqpoint{1.120846in}{1.676587in}}{\pgfqpoint{1.115022in}{1.670763in}}%
\pgfpathcurveto{\pgfqpoint{1.109198in}{1.664939in}}{\pgfqpoint{1.105926in}{1.657039in}}{\pgfqpoint{1.105926in}{1.648803in}}%
\pgfpathcurveto{\pgfqpoint{1.105926in}{1.640567in}}{\pgfqpoint{1.109198in}{1.632667in}}{\pgfqpoint{1.115022in}{1.626843in}}%
\pgfpathcurveto{\pgfqpoint{1.120846in}{1.621019in}}{\pgfqpoint{1.128746in}{1.617747in}}{\pgfqpoint{1.136982in}{1.617747in}}%
\pgfpathclose%
\pgfusepath{stroke,fill}%
\end{pgfscope}%
\begin{pgfscope}%
\pgfpathrectangle{\pgfqpoint{0.100000in}{0.212622in}}{\pgfqpoint{3.696000in}{3.696000in}}%
\pgfusepath{clip}%
\pgfsetbuttcap%
\pgfsetroundjoin%
\definecolor{currentfill}{rgb}{0.121569,0.466667,0.705882}%
\pgfsetfillcolor{currentfill}%
\pgfsetfillopacity{0.300003}%
\pgfsetlinewidth{1.003750pt}%
\definecolor{currentstroke}{rgb}{0.121569,0.466667,0.705882}%
\pgfsetstrokecolor{currentstroke}%
\pgfsetstrokeopacity{0.300003}%
\pgfsetdash{}{0pt}%
\pgfpathmoveto{\pgfqpoint{1.136982in}{1.617747in}}%
\pgfpathcurveto{\pgfqpoint{1.145219in}{1.617747in}}{\pgfqpoint{1.153119in}{1.621019in}}{\pgfqpoint{1.158943in}{1.626843in}}%
\pgfpathcurveto{\pgfqpoint{1.164767in}{1.632667in}}{\pgfqpoint{1.168039in}{1.640567in}}{\pgfqpoint{1.168039in}{1.648803in}}%
\pgfpathcurveto{\pgfqpoint{1.168039in}{1.657039in}}{\pgfqpoint{1.164767in}{1.664939in}}{\pgfqpoint{1.158943in}{1.670763in}}%
\pgfpathcurveto{\pgfqpoint{1.153119in}{1.676587in}}{\pgfqpoint{1.145219in}{1.679860in}}{\pgfqpoint{1.136982in}{1.679860in}}%
\pgfpathcurveto{\pgfqpoint{1.128746in}{1.679860in}}{\pgfqpoint{1.120846in}{1.676587in}}{\pgfqpoint{1.115022in}{1.670763in}}%
\pgfpathcurveto{\pgfqpoint{1.109198in}{1.664939in}}{\pgfqpoint{1.105926in}{1.657039in}}{\pgfqpoint{1.105926in}{1.648803in}}%
\pgfpathcurveto{\pgfqpoint{1.105926in}{1.640567in}}{\pgfqpoint{1.109198in}{1.632667in}}{\pgfqpoint{1.115022in}{1.626843in}}%
\pgfpathcurveto{\pgfqpoint{1.120846in}{1.621019in}}{\pgfqpoint{1.128746in}{1.617747in}}{\pgfqpoint{1.136982in}{1.617747in}}%
\pgfpathclose%
\pgfusepath{stroke,fill}%
\end{pgfscope}%
\begin{pgfscope}%
\pgfpathrectangle{\pgfqpoint{0.100000in}{0.212622in}}{\pgfqpoint{3.696000in}{3.696000in}}%
\pgfusepath{clip}%
\pgfsetbuttcap%
\pgfsetroundjoin%
\definecolor{currentfill}{rgb}{0.121569,0.466667,0.705882}%
\pgfsetfillcolor{currentfill}%
\pgfsetfillopacity{0.300003}%
\pgfsetlinewidth{1.003750pt}%
\definecolor{currentstroke}{rgb}{0.121569,0.466667,0.705882}%
\pgfsetstrokecolor{currentstroke}%
\pgfsetstrokeopacity{0.300003}%
\pgfsetdash{}{0pt}%
\pgfpathmoveto{\pgfqpoint{1.136982in}{1.617747in}}%
\pgfpathcurveto{\pgfqpoint{1.145219in}{1.617747in}}{\pgfqpoint{1.153119in}{1.621019in}}{\pgfqpoint{1.158943in}{1.626843in}}%
\pgfpathcurveto{\pgfqpoint{1.164767in}{1.632667in}}{\pgfqpoint{1.168039in}{1.640567in}}{\pgfqpoint{1.168039in}{1.648803in}}%
\pgfpathcurveto{\pgfqpoint{1.168039in}{1.657039in}}{\pgfqpoint{1.164767in}{1.664939in}}{\pgfqpoint{1.158943in}{1.670763in}}%
\pgfpathcurveto{\pgfqpoint{1.153119in}{1.676587in}}{\pgfqpoint{1.145219in}{1.679860in}}{\pgfqpoint{1.136982in}{1.679860in}}%
\pgfpathcurveto{\pgfqpoint{1.128746in}{1.679860in}}{\pgfqpoint{1.120846in}{1.676587in}}{\pgfqpoint{1.115022in}{1.670763in}}%
\pgfpathcurveto{\pgfqpoint{1.109198in}{1.664939in}}{\pgfqpoint{1.105926in}{1.657039in}}{\pgfqpoint{1.105926in}{1.648803in}}%
\pgfpathcurveto{\pgfqpoint{1.105926in}{1.640567in}}{\pgfqpoint{1.109198in}{1.632667in}}{\pgfqpoint{1.115022in}{1.626843in}}%
\pgfpathcurveto{\pgfqpoint{1.120846in}{1.621019in}}{\pgfqpoint{1.128746in}{1.617747in}}{\pgfqpoint{1.136982in}{1.617747in}}%
\pgfpathclose%
\pgfusepath{stroke,fill}%
\end{pgfscope}%
\begin{pgfscope}%
\pgfpathrectangle{\pgfqpoint{0.100000in}{0.212622in}}{\pgfqpoint{3.696000in}{3.696000in}}%
\pgfusepath{clip}%
\pgfsetbuttcap%
\pgfsetroundjoin%
\definecolor{currentfill}{rgb}{0.121569,0.466667,0.705882}%
\pgfsetfillcolor{currentfill}%
\pgfsetfillopacity{0.300003}%
\pgfsetlinewidth{1.003750pt}%
\definecolor{currentstroke}{rgb}{0.121569,0.466667,0.705882}%
\pgfsetstrokecolor{currentstroke}%
\pgfsetstrokeopacity{0.300003}%
\pgfsetdash{}{0pt}%
\pgfpathmoveto{\pgfqpoint{1.136982in}{1.617747in}}%
\pgfpathcurveto{\pgfqpoint{1.145219in}{1.617747in}}{\pgfqpoint{1.153119in}{1.621019in}}{\pgfqpoint{1.158943in}{1.626843in}}%
\pgfpathcurveto{\pgfqpoint{1.164767in}{1.632667in}}{\pgfqpoint{1.168039in}{1.640567in}}{\pgfqpoint{1.168039in}{1.648803in}}%
\pgfpathcurveto{\pgfqpoint{1.168039in}{1.657039in}}{\pgfqpoint{1.164767in}{1.664939in}}{\pgfqpoint{1.158943in}{1.670763in}}%
\pgfpathcurveto{\pgfqpoint{1.153119in}{1.676587in}}{\pgfqpoint{1.145219in}{1.679860in}}{\pgfqpoint{1.136982in}{1.679860in}}%
\pgfpathcurveto{\pgfqpoint{1.128746in}{1.679860in}}{\pgfqpoint{1.120846in}{1.676587in}}{\pgfqpoint{1.115022in}{1.670763in}}%
\pgfpathcurveto{\pgfqpoint{1.109198in}{1.664939in}}{\pgfqpoint{1.105926in}{1.657039in}}{\pgfqpoint{1.105926in}{1.648803in}}%
\pgfpathcurveto{\pgfqpoint{1.105926in}{1.640567in}}{\pgfqpoint{1.109198in}{1.632667in}}{\pgfqpoint{1.115022in}{1.626843in}}%
\pgfpathcurveto{\pgfqpoint{1.120846in}{1.621019in}}{\pgfqpoint{1.128746in}{1.617747in}}{\pgfqpoint{1.136982in}{1.617747in}}%
\pgfpathclose%
\pgfusepath{stroke,fill}%
\end{pgfscope}%
\begin{pgfscope}%
\pgfpathrectangle{\pgfqpoint{0.100000in}{0.212622in}}{\pgfqpoint{3.696000in}{3.696000in}}%
\pgfusepath{clip}%
\pgfsetbuttcap%
\pgfsetroundjoin%
\definecolor{currentfill}{rgb}{0.121569,0.466667,0.705882}%
\pgfsetfillcolor{currentfill}%
\pgfsetfillopacity{0.300003}%
\pgfsetlinewidth{1.003750pt}%
\definecolor{currentstroke}{rgb}{0.121569,0.466667,0.705882}%
\pgfsetstrokecolor{currentstroke}%
\pgfsetstrokeopacity{0.300003}%
\pgfsetdash{}{0pt}%
\pgfpathmoveto{\pgfqpoint{1.136982in}{1.617747in}}%
\pgfpathcurveto{\pgfqpoint{1.145219in}{1.617747in}}{\pgfqpoint{1.153119in}{1.621019in}}{\pgfqpoint{1.158943in}{1.626843in}}%
\pgfpathcurveto{\pgfqpoint{1.164767in}{1.632667in}}{\pgfqpoint{1.168039in}{1.640567in}}{\pgfqpoint{1.168039in}{1.648803in}}%
\pgfpathcurveto{\pgfqpoint{1.168039in}{1.657039in}}{\pgfqpoint{1.164767in}{1.664939in}}{\pgfqpoint{1.158943in}{1.670763in}}%
\pgfpathcurveto{\pgfqpoint{1.153119in}{1.676587in}}{\pgfqpoint{1.145219in}{1.679860in}}{\pgfqpoint{1.136982in}{1.679860in}}%
\pgfpathcurveto{\pgfqpoint{1.128746in}{1.679860in}}{\pgfqpoint{1.120846in}{1.676587in}}{\pgfqpoint{1.115022in}{1.670763in}}%
\pgfpathcurveto{\pgfqpoint{1.109198in}{1.664939in}}{\pgfqpoint{1.105926in}{1.657039in}}{\pgfqpoint{1.105926in}{1.648803in}}%
\pgfpathcurveto{\pgfqpoint{1.105926in}{1.640567in}}{\pgfqpoint{1.109198in}{1.632667in}}{\pgfqpoint{1.115022in}{1.626843in}}%
\pgfpathcurveto{\pgfqpoint{1.120846in}{1.621019in}}{\pgfqpoint{1.128746in}{1.617747in}}{\pgfqpoint{1.136982in}{1.617747in}}%
\pgfpathclose%
\pgfusepath{stroke,fill}%
\end{pgfscope}%
\begin{pgfscope}%
\pgfpathrectangle{\pgfqpoint{0.100000in}{0.212622in}}{\pgfqpoint{3.696000in}{3.696000in}}%
\pgfusepath{clip}%
\pgfsetbuttcap%
\pgfsetroundjoin%
\definecolor{currentfill}{rgb}{0.121569,0.466667,0.705882}%
\pgfsetfillcolor{currentfill}%
\pgfsetfillopacity{0.300003}%
\pgfsetlinewidth{1.003750pt}%
\definecolor{currentstroke}{rgb}{0.121569,0.466667,0.705882}%
\pgfsetstrokecolor{currentstroke}%
\pgfsetstrokeopacity{0.300003}%
\pgfsetdash{}{0pt}%
\pgfpathmoveto{\pgfqpoint{1.136982in}{1.617747in}}%
\pgfpathcurveto{\pgfqpoint{1.145219in}{1.617747in}}{\pgfqpoint{1.153119in}{1.621019in}}{\pgfqpoint{1.158943in}{1.626843in}}%
\pgfpathcurveto{\pgfqpoint{1.164767in}{1.632667in}}{\pgfqpoint{1.168039in}{1.640567in}}{\pgfqpoint{1.168039in}{1.648803in}}%
\pgfpathcurveto{\pgfqpoint{1.168039in}{1.657039in}}{\pgfqpoint{1.164767in}{1.664939in}}{\pgfqpoint{1.158943in}{1.670763in}}%
\pgfpathcurveto{\pgfqpoint{1.153119in}{1.676587in}}{\pgfqpoint{1.145219in}{1.679860in}}{\pgfqpoint{1.136982in}{1.679860in}}%
\pgfpathcurveto{\pgfqpoint{1.128746in}{1.679860in}}{\pgfqpoint{1.120846in}{1.676587in}}{\pgfqpoint{1.115022in}{1.670763in}}%
\pgfpathcurveto{\pgfqpoint{1.109198in}{1.664939in}}{\pgfqpoint{1.105926in}{1.657039in}}{\pgfqpoint{1.105926in}{1.648803in}}%
\pgfpathcurveto{\pgfqpoint{1.105926in}{1.640567in}}{\pgfqpoint{1.109198in}{1.632667in}}{\pgfqpoint{1.115022in}{1.626843in}}%
\pgfpathcurveto{\pgfqpoint{1.120846in}{1.621019in}}{\pgfqpoint{1.128746in}{1.617747in}}{\pgfqpoint{1.136982in}{1.617747in}}%
\pgfpathclose%
\pgfusepath{stroke,fill}%
\end{pgfscope}%
\begin{pgfscope}%
\pgfpathrectangle{\pgfqpoint{0.100000in}{0.212622in}}{\pgfqpoint{3.696000in}{3.696000in}}%
\pgfusepath{clip}%
\pgfsetbuttcap%
\pgfsetroundjoin%
\definecolor{currentfill}{rgb}{0.121569,0.466667,0.705882}%
\pgfsetfillcolor{currentfill}%
\pgfsetfillopacity{0.300003}%
\pgfsetlinewidth{1.003750pt}%
\definecolor{currentstroke}{rgb}{0.121569,0.466667,0.705882}%
\pgfsetstrokecolor{currentstroke}%
\pgfsetstrokeopacity{0.300003}%
\pgfsetdash{}{0pt}%
\pgfpathmoveto{\pgfqpoint{1.136982in}{1.617747in}}%
\pgfpathcurveto{\pgfqpoint{1.145219in}{1.617747in}}{\pgfqpoint{1.153119in}{1.621019in}}{\pgfqpoint{1.158943in}{1.626843in}}%
\pgfpathcurveto{\pgfqpoint{1.164767in}{1.632667in}}{\pgfqpoint{1.168039in}{1.640567in}}{\pgfqpoint{1.168039in}{1.648803in}}%
\pgfpathcurveto{\pgfqpoint{1.168039in}{1.657039in}}{\pgfqpoint{1.164767in}{1.664939in}}{\pgfqpoint{1.158943in}{1.670763in}}%
\pgfpathcurveto{\pgfqpoint{1.153119in}{1.676587in}}{\pgfqpoint{1.145219in}{1.679860in}}{\pgfqpoint{1.136982in}{1.679860in}}%
\pgfpathcurveto{\pgfqpoint{1.128746in}{1.679860in}}{\pgfqpoint{1.120846in}{1.676587in}}{\pgfqpoint{1.115022in}{1.670763in}}%
\pgfpathcurveto{\pgfqpoint{1.109198in}{1.664939in}}{\pgfqpoint{1.105926in}{1.657039in}}{\pgfqpoint{1.105926in}{1.648803in}}%
\pgfpathcurveto{\pgfqpoint{1.105926in}{1.640567in}}{\pgfqpoint{1.109198in}{1.632667in}}{\pgfqpoint{1.115022in}{1.626843in}}%
\pgfpathcurveto{\pgfqpoint{1.120846in}{1.621019in}}{\pgfqpoint{1.128746in}{1.617747in}}{\pgfqpoint{1.136982in}{1.617747in}}%
\pgfpathclose%
\pgfusepath{stroke,fill}%
\end{pgfscope}%
\begin{pgfscope}%
\pgfpathrectangle{\pgfqpoint{0.100000in}{0.212622in}}{\pgfqpoint{3.696000in}{3.696000in}}%
\pgfusepath{clip}%
\pgfsetbuttcap%
\pgfsetroundjoin%
\definecolor{currentfill}{rgb}{0.121569,0.466667,0.705882}%
\pgfsetfillcolor{currentfill}%
\pgfsetfillopacity{0.300003}%
\pgfsetlinewidth{1.003750pt}%
\definecolor{currentstroke}{rgb}{0.121569,0.466667,0.705882}%
\pgfsetstrokecolor{currentstroke}%
\pgfsetstrokeopacity{0.300003}%
\pgfsetdash{}{0pt}%
\pgfpathmoveto{\pgfqpoint{1.136982in}{1.617747in}}%
\pgfpathcurveto{\pgfqpoint{1.145219in}{1.617747in}}{\pgfqpoint{1.153119in}{1.621019in}}{\pgfqpoint{1.158943in}{1.626843in}}%
\pgfpathcurveto{\pgfqpoint{1.164767in}{1.632667in}}{\pgfqpoint{1.168039in}{1.640567in}}{\pgfqpoint{1.168039in}{1.648803in}}%
\pgfpathcurveto{\pgfqpoint{1.168039in}{1.657039in}}{\pgfqpoint{1.164767in}{1.664939in}}{\pgfqpoint{1.158943in}{1.670763in}}%
\pgfpathcurveto{\pgfqpoint{1.153119in}{1.676587in}}{\pgfqpoint{1.145219in}{1.679860in}}{\pgfqpoint{1.136982in}{1.679860in}}%
\pgfpathcurveto{\pgfqpoint{1.128746in}{1.679860in}}{\pgfqpoint{1.120846in}{1.676587in}}{\pgfqpoint{1.115022in}{1.670763in}}%
\pgfpathcurveto{\pgfqpoint{1.109198in}{1.664939in}}{\pgfqpoint{1.105926in}{1.657039in}}{\pgfqpoint{1.105926in}{1.648803in}}%
\pgfpathcurveto{\pgfqpoint{1.105926in}{1.640567in}}{\pgfqpoint{1.109198in}{1.632667in}}{\pgfqpoint{1.115022in}{1.626843in}}%
\pgfpathcurveto{\pgfqpoint{1.120846in}{1.621019in}}{\pgfqpoint{1.128746in}{1.617747in}}{\pgfqpoint{1.136982in}{1.617747in}}%
\pgfpathclose%
\pgfusepath{stroke,fill}%
\end{pgfscope}%
\begin{pgfscope}%
\pgfpathrectangle{\pgfqpoint{0.100000in}{0.212622in}}{\pgfqpoint{3.696000in}{3.696000in}}%
\pgfusepath{clip}%
\pgfsetbuttcap%
\pgfsetroundjoin%
\definecolor{currentfill}{rgb}{0.121569,0.466667,0.705882}%
\pgfsetfillcolor{currentfill}%
\pgfsetfillopacity{0.300003}%
\pgfsetlinewidth{1.003750pt}%
\definecolor{currentstroke}{rgb}{0.121569,0.466667,0.705882}%
\pgfsetstrokecolor{currentstroke}%
\pgfsetstrokeopacity{0.300003}%
\pgfsetdash{}{0pt}%
\pgfpathmoveto{\pgfqpoint{1.136982in}{1.617747in}}%
\pgfpathcurveto{\pgfqpoint{1.145219in}{1.617747in}}{\pgfqpoint{1.153119in}{1.621019in}}{\pgfqpoint{1.158943in}{1.626843in}}%
\pgfpathcurveto{\pgfqpoint{1.164767in}{1.632667in}}{\pgfqpoint{1.168039in}{1.640567in}}{\pgfqpoint{1.168039in}{1.648803in}}%
\pgfpathcurveto{\pgfqpoint{1.168039in}{1.657039in}}{\pgfqpoint{1.164767in}{1.664939in}}{\pgfqpoint{1.158943in}{1.670763in}}%
\pgfpathcurveto{\pgfqpoint{1.153119in}{1.676587in}}{\pgfqpoint{1.145219in}{1.679860in}}{\pgfqpoint{1.136982in}{1.679860in}}%
\pgfpathcurveto{\pgfqpoint{1.128746in}{1.679860in}}{\pgfqpoint{1.120846in}{1.676587in}}{\pgfqpoint{1.115022in}{1.670763in}}%
\pgfpathcurveto{\pgfqpoint{1.109198in}{1.664939in}}{\pgfqpoint{1.105926in}{1.657039in}}{\pgfqpoint{1.105926in}{1.648803in}}%
\pgfpathcurveto{\pgfqpoint{1.105926in}{1.640567in}}{\pgfqpoint{1.109198in}{1.632667in}}{\pgfqpoint{1.115022in}{1.626843in}}%
\pgfpathcurveto{\pgfqpoint{1.120846in}{1.621019in}}{\pgfqpoint{1.128746in}{1.617747in}}{\pgfqpoint{1.136982in}{1.617747in}}%
\pgfpathclose%
\pgfusepath{stroke,fill}%
\end{pgfscope}%
\begin{pgfscope}%
\pgfpathrectangle{\pgfqpoint{0.100000in}{0.212622in}}{\pgfqpoint{3.696000in}{3.696000in}}%
\pgfusepath{clip}%
\pgfsetbuttcap%
\pgfsetroundjoin%
\definecolor{currentfill}{rgb}{0.121569,0.466667,0.705882}%
\pgfsetfillcolor{currentfill}%
\pgfsetfillopacity{0.300003}%
\pgfsetlinewidth{1.003750pt}%
\definecolor{currentstroke}{rgb}{0.121569,0.466667,0.705882}%
\pgfsetstrokecolor{currentstroke}%
\pgfsetstrokeopacity{0.300003}%
\pgfsetdash{}{0pt}%
\pgfpathmoveto{\pgfqpoint{1.136982in}{1.617747in}}%
\pgfpathcurveto{\pgfqpoint{1.145219in}{1.617747in}}{\pgfqpoint{1.153119in}{1.621019in}}{\pgfqpoint{1.158943in}{1.626843in}}%
\pgfpathcurveto{\pgfqpoint{1.164767in}{1.632667in}}{\pgfqpoint{1.168039in}{1.640567in}}{\pgfqpoint{1.168039in}{1.648803in}}%
\pgfpathcurveto{\pgfqpoint{1.168039in}{1.657039in}}{\pgfqpoint{1.164767in}{1.664939in}}{\pgfqpoint{1.158943in}{1.670763in}}%
\pgfpathcurveto{\pgfqpoint{1.153119in}{1.676587in}}{\pgfqpoint{1.145219in}{1.679860in}}{\pgfqpoint{1.136982in}{1.679860in}}%
\pgfpathcurveto{\pgfqpoint{1.128746in}{1.679860in}}{\pgfqpoint{1.120846in}{1.676587in}}{\pgfqpoint{1.115022in}{1.670763in}}%
\pgfpathcurveto{\pgfqpoint{1.109198in}{1.664939in}}{\pgfqpoint{1.105926in}{1.657039in}}{\pgfqpoint{1.105926in}{1.648803in}}%
\pgfpathcurveto{\pgfqpoint{1.105926in}{1.640567in}}{\pgfqpoint{1.109198in}{1.632667in}}{\pgfqpoint{1.115022in}{1.626843in}}%
\pgfpathcurveto{\pgfqpoint{1.120846in}{1.621019in}}{\pgfqpoint{1.128746in}{1.617747in}}{\pgfqpoint{1.136982in}{1.617747in}}%
\pgfpathclose%
\pgfusepath{stroke,fill}%
\end{pgfscope}%
\begin{pgfscope}%
\pgfpathrectangle{\pgfqpoint{0.100000in}{0.212622in}}{\pgfqpoint{3.696000in}{3.696000in}}%
\pgfusepath{clip}%
\pgfsetbuttcap%
\pgfsetroundjoin%
\definecolor{currentfill}{rgb}{0.121569,0.466667,0.705882}%
\pgfsetfillcolor{currentfill}%
\pgfsetfillopacity{0.300003}%
\pgfsetlinewidth{1.003750pt}%
\definecolor{currentstroke}{rgb}{0.121569,0.466667,0.705882}%
\pgfsetstrokecolor{currentstroke}%
\pgfsetstrokeopacity{0.300003}%
\pgfsetdash{}{0pt}%
\pgfpathmoveto{\pgfqpoint{1.136982in}{1.617747in}}%
\pgfpathcurveto{\pgfqpoint{1.145219in}{1.617747in}}{\pgfqpoint{1.153119in}{1.621019in}}{\pgfqpoint{1.158943in}{1.626843in}}%
\pgfpathcurveto{\pgfqpoint{1.164767in}{1.632667in}}{\pgfqpoint{1.168039in}{1.640567in}}{\pgfqpoint{1.168039in}{1.648803in}}%
\pgfpathcurveto{\pgfqpoint{1.168039in}{1.657039in}}{\pgfqpoint{1.164767in}{1.664939in}}{\pgfqpoint{1.158943in}{1.670763in}}%
\pgfpathcurveto{\pgfqpoint{1.153119in}{1.676587in}}{\pgfqpoint{1.145219in}{1.679860in}}{\pgfqpoint{1.136982in}{1.679860in}}%
\pgfpathcurveto{\pgfqpoint{1.128746in}{1.679860in}}{\pgfqpoint{1.120846in}{1.676587in}}{\pgfqpoint{1.115022in}{1.670763in}}%
\pgfpathcurveto{\pgfqpoint{1.109198in}{1.664939in}}{\pgfqpoint{1.105926in}{1.657039in}}{\pgfqpoint{1.105926in}{1.648803in}}%
\pgfpathcurveto{\pgfqpoint{1.105926in}{1.640567in}}{\pgfqpoint{1.109198in}{1.632667in}}{\pgfqpoint{1.115022in}{1.626843in}}%
\pgfpathcurveto{\pgfqpoint{1.120846in}{1.621019in}}{\pgfqpoint{1.128746in}{1.617747in}}{\pgfqpoint{1.136982in}{1.617747in}}%
\pgfpathclose%
\pgfusepath{stroke,fill}%
\end{pgfscope}%
\begin{pgfscope}%
\pgfpathrectangle{\pgfqpoint{0.100000in}{0.212622in}}{\pgfqpoint{3.696000in}{3.696000in}}%
\pgfusepath{clip}%
\pgfsetbuttcap%
\pgfsetroundjoin%
\definecolor{currentfill}{rgb}{0.121569,0.466667,0.705882}%
\pgfsetfillcolor{currentfill}%
\pgfsetfillopacity{0.300003}%
\pgfsetlinewidth{1.003750pt}%
\definecolor{currentstroke}{rgb}{0.121569,0.466667,0.705882}%
\pgfsetstrokecolor{currentstroke}%
\pgfsetstrokeopacity{0.300003}%
\pgfsetdash{}{0pt}%
\pgfpathmoveto{\pgfqpoint{1.136982in}{1.617747in}}%
\pgfpathcurveto{\pgfqpoint{1.145219in}{1.617747in}}{\pgfqpoint{1.153119in}{1.621019in}}{\pgfqpoint{1.158943in}{1.626843in}}%
\pgfpathcurveto{\pgfqpoint{1.164767in}{1.632667in}}{\pgfqpoint{1.168039in}{1.640567in}}{\pgfqpoint{1.168039in}{1.648803in}}%
\pgfpathcurveto{\pgfqpoint{1.168039in}{1.657039in}}{\pgfqpoint{1.164767in}{1.664939in}}{\pgfqpoint{1.158943in}{1.670763in}}%
\pgfpathcurveto{\pgfqpoint{1.153119in}{1.676587in}}{\pgfqpoint{1.145219in}{1.679860in}}{\pgfqpoint{1.136982in}{1.679860in}}%
\pgfpathcurveto{\pgfqpoint{1.128746in}{1.679860in}}{\pgfqpoint{1.120846in}{1.676587in}}{\pgfqpoint{1.115022in}{1.670763in}}%
\pgfpathcurveto{\pgfqpoint{1.109198in}{1.664939in}}{\pgfqpoint{1.105926in}{1.657039in}}{\pgfqpoint{1.105926in}{1.648803in}}%
\pgfpathcurveto{\pgfqpoint{1.105926in}{1.640567in}}{\pgfqpoint{1.109198in}{1.632667in}}{\pgfqpoint{1.115022in}{1.626843in}}%
\pgfpathcurveto{\pgfqpoint{1.120846in}{1.621019in}}{\pgfqpoint{1.128746in}{1.617747in}}{\pgfqpoint{1.136982in}{1.617747in}}%
\pgfpathclose%
\pgfusepath{stroke,fill}%
\end{pgfscope}%
\begin{pgfscope}%
\pgfpathrectangle{\pgfqpoint{0.100000in}{0.212622in}}{\pgfqpoint{3.696000in}{3.696000in}}%
\pgfusepath{clip}%
\pgfsetbuttcap%
\pgfsetroundjoin%
\definecolor{currentfill}{rgb}{0.121569,0.466667,0.705882}%
\pgfsetfillcolor{currentfill}%
\pgfsetfillopacity{0.300221}%
\pgfsetlinewidth{1.003750pt}%
\definecolor{currentstroke}{rgb}{0.121569,0.466667,0.705882}%
\pgfsetstrokecolor{currentstroke}%
\pgfsetstrokeopacity{0.300221}%
\pgfsetdash{}{0pt}%
\pgfpathmoveto{\pgfqpoint{1.137356in}{1.617563in}}%
\pgfpathcurveto{\pgfqpoint{1.145592in}{1.617563in}}{\pgfqpoint{1.153492in}{1.620836in}}{\pgfqpoint{1.159316in}{1.626660in}}%
\pgfpathcurveto{\pgfqpoint{1.165140in}{1.632484in}}{\pgfqpoint{1.168412in}{1.640384in}}{\pgfqpoint{1.168412in}{1.648620in}}%
\pgfpathcurveto{\pgfqpoint{1.168412in}{1.656856in}}{\pgfqpoint{1.165140in}{1.664756in}}{\pgfqpoint{1.159316in}{1.670580in}}%
\pgfpathcurveto{\pgfqpoint{1.153492in}{1.676404in}}{\pgfqpoint{1.145592in}{1.679676in}}{\pgfqpoint{1.137356in}{1.679676in}}%
\pgfpathcurveto{\pgfqpoint{1.129120in}{1.679676in}}{\pgfqpoint{1.121220in}{1.676404in}}{\pgfqpoint{1.115396in}{1.670580in}}%
\pgfpathcurveto{\pgfqpoint{1.109572in}{1.664756in}}{\pgfqpoint{1.106299in}{1.656856in}}{\pgfqpoint{1.106299in}{1.648620in}}%
\pgfpathcurveto{\pgfqpoint{1.106299in}{1.640384in}}{\pgfqpoint{1.109572in}{1.632484in}}{\pgfqpoint{1.115396in}{1.626660in}}%
\pgfpathcurveto{\pgfqpoint{1.121220in}{1.620836in}}{\pgfqpoint{1.129120in}{1.617563in}}{\pgfqpoint{1.137356in}{1.617563in}}%
\pgfpathclose%
\pgfusepath{stroke,fill}%
\end{pgfscope}%
\begin{pgfscope}%
\pgfpathrectangle{\pgfqpoint{0.100000in}{0.212622in}}{\pgfqpoint{3.696000in}{3.696000in}}%
\pgfusepath{clip}%
\pgfsetbuttcap%
\pgfsetroundjoin%
\definecolor{currentfill}{rgb}{0.121569,0.466667,0.705882}%
\pgfsetfillcolor{currentfill}%
\pgfsetfillopacity{0.300331}%
\pgfsetlinewidth{1.003750pt}%
\definecolor{currentstroke}{rgb}{0.121569,0.466667,0.705882}%
\pgfsetstrokecolor{currentstroke}%
\pgfsetstrokeopacity{0.300331}%
\pgfsetdash{}{0pt}%
\pgfpathmoveto{\pgfqpoint{1.137571in}{1.617472in}}%
\pgfpathcurveto{\pgfqpoint{1.145807in}{1.617472in}}{\pgfqpoint{1.153707in}{1.620744in}}{\pgfqpoint{1.159531in}{1.626568in}}%
\pgfpathcurveto{\pgfqpoint{1.165355in}{1.632392in}}{\pgfqpoint{1.168627in}{1.640292in}}{\pgfqpoint{1.168627in}{1.648528in}}%
\pgfpathcurveto{\pgfqpoint{1.168627in}{1.656765in}}{\pgfqpoint{1.165355in}{1.664665in}}{\pgfqpoint{1.159531in}{1.670489in}}%
\pgfpathcurveto{\pgfqpoint{1.153707in}{1.676313in}}{\pgfqpoint{1.145807in}{1.679585in}}{\pgfqpoint{1.137571in}{1.679585in}}%
\pgfpathcurveto{\pgfqpoint{1.129334in}{1.679585in}}{\pgfqpoint{1.121434in}{1.676313in}}{\pgfqpoint{1.115610in}{1.670489in}}%
\pgfpathcurveto{\pgfqpoint{1.109787in}{1.664665in}}{\pgfqpoint{1.106514in}{1.656765in}}{\pgfqpoint{1.106514in}{1.648528in}}%
\pgfpathcurveto{\pgfqpoint{1.106514in}{1.640292in}}{\pgfqpoint{1.109787in}{1.632392in}}{\pgfqpoint{1.115610in}{1.626568in}}%
\pgfpathcurveto{\pgfqpoint{1.121434in}{1.620744in}}{\pgfqpoint{1.129334in}{1.617472in}}{\pgfqpoint{1.137571in}{1.617472in}}%
\pgfpathclose%
\pgfusepath{stroke,fill}%
\end{pgfscope}%
\begin{pgfscope}%
\pgfpathrectangle{\pgfqpoint{0.100000in}{0.212622in}}{\pgfqpoint{3.696000in}{3.696000in}}%
\pgfusepath{clip}%
\pgfsetbuttcap%
\pgfsetroundjoin%
\definecolor{currentfill}{rgb}{0.121569,0.466667,0.705882}%
\pgfsetfillcolor{currentfill}%
\pgfsetfillopacity{0.300394}%
\pgfsetlinewidth{1.003750pt}%
\definecolor{currentstroke}{rgb}{0.121569,0.466667,0.705882}%
\pgfsetstrokecolor{currentstroke}%
\pgfsetstrokeopacity{0.300394}%
\pgfsetdash{}{0pt}%
\pgfpathmoveto{\pgfqpoint{1.137687in}{1.617419in}}%
\pgfpathcurveto{\pgfqpoint{1.145923in}{1.617419in}}{\pgfqpoint{1.153823in}{1.620692in}}{\pgfqpoint{1.159647in}{1.626516in}}%
\pgfpathcurveto{\pgfqpoint{1.165471in}{1.632340in}}{\pgfqpoint{1.168743in}{1.640240in}}{\pgfqpoint{1.168743in}{1.648476in}}%
\pgfpathcurveto{\pgfqpoint{1.168743in}{1.656712in}}{\pgfqpoint{1.165471in}{1.664612in}}{\pgfqpoint{1.159647in}{1.670436in}}%
\pgfpathcurveto{\pgfqpoint{1.153823in}{1.676260in}}{\pgfqpoint{1.145923in}{1.679532in}}{\pgfqpoint{1.137687in}{1.679532in}}%
\pgfpathcurveto{\pgfqpoint{1.129451in}{1.679532in}}{\pgfqpoint{1.121551in}{1.676260in}}{\pgfqpoint{1.115727in}{1.670436in}}%
\pgfpathcurveto{\pgfqpoint{1.109903in}{1.664612in}}{\pgfqpoint{1.106630in}{1.656712in}}{\pgfqpoint{1.106630in}{1.648476in}}%
\pgfpathcurveto{\pgfqpoint{1.106630in}{1.640240in}}{\pgfqpoint{1.109903in}{1.632340in}}{\pgfqpoint{1.115727in}{1.626516in}}%
\pgfpathcurveto{\pgfqpoint{1.121551in}{1.620692in}}{\pgfqpoint{1.129451in}{1.617419in}}{\pgfqpoint{1.137687in}{1.617419in}}%
\pgfpathclose%
\pgfusepath{stroke,fill}%
\end{pgfscope}%
\begin{pgfscope}%
\pgfpathrectangle{\pgfqpoint{0.100000in}{0.212622in}}{\pgfqpoint{3.696000in}{3.696000in}}%
\pgfusepath{clip}%
\pgfsetbuttcap%
\pgfsetroundjoin%
\definecolor{currentfill}{rgb}{0.121569,0.466667,0.705882}%
\pgfsetfillcolor{currentfill}%
\pgfsetfillopacity{0.300424}%
\pgfsetlinewidth{1.003750pt}%
\definecolor{currentstroke}{rgb}{0.121569,0.466667,0.705882}%
\pgfsetstrokecolor{currentstroke}%
\pgfsetstrokeopacity{0.300424}%
\pgfsetdash{}{0pt}%
\pgfpathmoveto{\pgfqpoint{1.137755in}{1.617394in}}%
\pgfpathcurveto{\pgfqpoint{1.145991in}{1.617394in}}{\pgfqpoint{1.153891in}{1.620666in}}{\pgfqpoint{1.159715in}{1.626490in}}%
\pgfpathcurveto{\pgfqpoint{1.165539in}{1.632314in}}{\pgfqpoint{1.168811in}{1.640214in}}{\pgfqpoint{1.168811in}{1.648451in}}%
\pgfpathcurveto{\pgfqpoint{1.168811in}{1.656687in}}{\pgfqpoint{1.165539in}{1.664587in}}{\pgfqpoint{1.159715in}{1.670411in}}%
\pgfpathcurveto{\pgfqpoint{1.153891in}{1.676235in}}{\pgfqpoint{1.145991in}{1.679507in}}{\pgfqpoint{1.137755in}{1.679507in}}%
\pgfpathcurveto{\pgfqpoint{1.129519in}{1.679507in}}{\pgfqpoint{1.121619in}{1.676235in}}{\pgfqpoint{1.115795in}{1.670411in}}%
\pgfpathcurveto{\pgfqpoint{1.109971in}{1.664587in}}{\pgfqpoint{1.106698in}{1.656687in}}{\pgfqpoint{1.106698in}{1.648451in}}%
\pgfpathcurveto{\pgfqpoint{1.106698in}{1.640214in}}{\pgfqpoint{1.109971in}{1.632314in}}{\pgfqpoint{1.115795in}{1.626490in}}%
\pgfpathcurveto{\pgfqpoint{1.121619in}{1.620666in}}{\pgfqpoint{1.129519in}{1.617394in}}{\pgfqpoint{1.137755in}{1.617394in}}%
\pgfpathclose%
\pgfusepath{stroke,fill}%
\end{pgfscope}%
\begin{pgfscope}%
\pgfpathrectangle{\pgfqpoint{0.100000in}{0.212622in}}{\pgfqpoint{3.696000in}{3.696000in}}%
\pgfusepath{clip}%
\pgfsetbuttcap%
\pgfsetroundjoin%
\definecolor{currentfill}{rgb}{0.121569,0.466667,0.705882}%
\pgfsetfillcolor{currentfill}%
\pgfsetfillopacity{0.300440}%
\pgfsetlinewidth{1.003750pt}%
\definecolor{currentstroke}{rgb}{0.121569,0.466667,0.705882}%
\pgfsetstrokecolor{currentstroke}%
\pgfsetstrokeopacity{0.300440}%
\pgfsetdash{}{0pt}%
\pgfpathmoveto{\pgfqpoint{1.137793in}{1.617381in}}%
\pgfpathcurveto{\pgfqpoint{1.146029in}{1.617381in}}{\pgfqpoint{1.153929in}{1.620653in}}{\pgfqpoint{1.159753in}{1.626477in}}%
\pgfpathcurveto{\pgfqpoint{1.165577in}{1.632301in}}{\pgfqpoint{1.168850in}{1.640201in}}{\pgfqpoint{1.168850in}{1.648437in}}%
\pgfpathcurveto{\pgfqpoint{1.168850in}{1.656674in}}{\pgfqpoint{1.165577in}{1.664574in}}{\pgfqpoint{1.159753in}{1.670398in}}%
\pgfpathcurveto{\pgfqpoint{1.153929in}{1.676222in}}{\pgfqpoint{1.146029in}{1.679494in}}{\pgfqpoint{1.137793in}{1.679494in}}%
\pgfpathcurveto{\pgfqpoint{1.129557in}{1.679494in}}{\pgfqpoint{1.121657in}{1.676222in}}{\pgfqpoint{1.115833in}{1.670398in}}%
\pgfpathcurveto{\pgfqpoint{1.110009in}{1.664574in}}{\pgfqpoint{1.106737in}{1.656674in}}{\pgfqpoint{1.106737in}{1.648437in}}%
\pgfpathcurveto{\pgfqpoint{1.106737in}{1.640201in}}{\pgfqpoint{1.110009in}{1.632301in}}{\pgfqpoint{1.115833in}{1.626477in}}%
\pgfpathcurveto{\pgfqpoint{1.121657in}{1.620653in}}{\pgfqpoint{1.129557in}{1.617381in}}{\pgfqpoint{1.137793in}{1.617381in}}%
\pgfpathclose%
\pgfusepath{stroke,fill}%
\end{pgfscope}%
\begin{pgfscope}%
\pgfpathrectangle{\pgfqpoint{0.100000in}{0.212622in}}{\pgfqpoint{3.696000in}{3.696000in}}%
\pgfusepath{clip}%
\pgfsetbuttcap%
\pgfsetroundjoin%
\definecolor{currentfill}{rgb}{0.121569,0.466667,0.705882}%
\pgfsetfillcolor{currentfill}%
\pgfsetfillopacity{0.300449}%
\pgfsetlinewidth{1.003750pt}%
\definecolor{currentstroke}{rgb}{0.121569,0.466667,0.705882}%
\pgfsetstrokecolor{currentstroke}%
\pgfsetstrokeopacity{0.300449}%
\pgfsetdash{}{0pt}%
\pgfpathmoveto{\pgfqpoint{1.137814in}{1.617373in}}%
\pgfpathcurveto{\pgfqpoint{1.146050in}{1.617373in}}{\pgfqpoint{1.153950in}{1.620646in}}{\pgfqpoint{1.159774in}{1.626470in}}%
\pgfpathcurveto{\pgfqpoint{1.165598in}{1.632294in}}{\pgfqpoint{1.168870in}{1.640194in}}{\pgfqpoint{1.168870in}{1.648430in}}%
\pgfpathcurveto{\pgfqpoint{1.168870in}{1.656666in}}{\pgfqpoint{1.165598in}{1.664566in}}{\pgfqpoint{1.159774in}{1.670390in}}%
\pgfpathcurveto{\pgfqpoint{1.153950in}{1.676214in}}{\pgfqpoint{1.146050in}{1.679486in}}{\pgfqpoint{1.137814in}{1.679486in}}%
\pgfpathcurveto{\pgfqpoint{1.129577in}{1.679486in}}{\pgfqpoint{1.121677in}{1.676214in}}{\pgfqpoint{1.115853in}{1.670390in}}%
\pgfpathcurveto{\pgfqpoint{1.110029in}{1.664566in}}{\pgfqpoint{1.106757in}{1.656666in}}{\pgfqpoint{1.106757in}{1.648430in}}%
\pgfpathcurveto{\pgfqpoint{1.106757in}{1.640194in}}{\pgfqpoint{1.110029in}{1.632294in}}{\pgfqpoint{1.115853in}{1.626470in}}%
\pgfpathcurveto{\pgfqpoint{1.121677in}{1.620646in}}{\pgfqpoint{1.129577in}{1.617373in}}{\pgfqpoint{1.137814in}{1.617373in}}%
\pgfpathclose%
\pgfusepath{stroke,fill}%
\end{pgfscope}%
\begin{pgfscope}%
\pgfpathrectangle{\pgfqpoint{0.100000in}{0.212622in}}{\pgfqpoint{3.696000in}{3.696000in}}%
\pgfusepath{clip}%
\pgfsetbuttcap%
\pgfsetroundjoin%
\definecolor{currentfill}{rgb}{0.121569,0.466667,0.705882}%
\pgfsetfillcolor{currentfill}%
\pgfsetfillopacity{0.300453}%
\pgfsetlinewidth{1.003750pt}%
\definecolor{currentstroke}{rgb}{0.121569,0.466667,0.705882}%
\pgfsetstrokecolor{currentstroke}%
\pgfsetstrokeopacity{0.300453}%
\pgfsetdash{}{0pt}%
\pgfpathmoveto{\pgfqpoint{1.137826in}{1.617370in}}%
\pgfpathcurveto{\pgfqpoint{1.146062in}{1.617370in}}{\pgfqpoint{1.153962in}{1.620642in}}{\pgfqpoint{1.159786in}{1.626466in}}%
\pgfpathcurveto{\pgfqpoint{1.165610in}{1.632290in}}{\pgfqpoint{1.168882in}{1.640190in}}{\pgfqpoint{1.168882in}{1.648426in}}%
\pgfpathcurveto{\pgfqpoint{1.168882in}{1.656663in}}{\pgfqpoint{1.165610in}{1.664563in}}{\pgfqpoint{1.159786in}{1.670387in}}%
\pgfpathcurveto{\pgfqpoint{1.153962in}{1.676211in}}{\pgfqpoint{1.146062in}{1.679483in}}{\pgfqpoint{1.137826in}{1.679483in}}%
\pgfpathcurveto{\pgfqpoint{1.129590in}{1.679483in}}{\pgfqpoint{1.121690in}{1.676211in}}{\pgfqpoint{1.115866in}{1.670387in}}%
\pgfpathcurveto{\pgfqpoint{1.110042in}{1.664563in}}{\pgfqpoint{1.106769in}{1.656663in}}{\pgfqpoint{1.106769in}{1.648426in}}%
\pgfpathcurveto{\pgfqpoint{1.106769in}{1.640190in}}{\pgfqpoint{1.110042in}{1.632290in}}{\pgfqpoint{1.115866in}{1.626466in}}%
\pgfpathcurveto{\pgfqpoint{1.121690in}{1.620642in}}{\pgfqpoint{1.129590in}{1.617370in}}{\pgfqpoint{1.137826in}{1.617370in}}%
\pgfpathclose%
\pgfusepath{stroke,fill}%
\end{pgfscope}%
\begin{pgfscope}%
\pgfpathrectangle{\pgfqpoint{0.100000in}{0.212622in}}{\pgfqpoint{3.696000in}{3.696000in}}%
\pgfusepath{clip}%
\pgfsetbuttcap%
\pgfsetroundjoin%
\definecolor{currentfill}{rgb}{0.121569,0.466667,0.705882}%
\pgfsetfillcolor{currentfill}%
\pgfsetfillopacity{0.300455}%
\pgfsetlinewidth{1.003750pt}%
\definecolor{currentstroke}{rgb}{0.121569,0.466667,0.705882}%
\pgfsetstrokecolor{currentstroke}%
\pgfsetstrokeopacity{0.300455}%
\pgfsetdash{}{0pt}%
\pgfpathmoveto{\pgfqpoint{1.137833in}{1.617368in}}%
\pgfpathcurveto{\pgfqpoint{1.146069in}{1.617368in}}{\pgfqpoint{1.153969in}{1.620640in}}{\pgfqpoint{1.159793in}{1.626464in}}%
\pgfpathcurveto{\pgfqpoint{1.165617in}{1.632288in}}{\pgfqpoint{1.168889in}{1.640188in}}{\pgfqpoint{1.168889in}{1.648424in}}%
\pgfpathcurveto{\pgfqpoint{1.168889in}{1.656661in}}{\pgfqpoint{1.165617in}{1.664561in}}{\pgfqpoint{1.159793in}{1.670385in}}%
\pgfpathcurveto{\pgfqpoint{1.153969in}{1.676209in}}{\pgfqpoint{1.146069in}{1.679481in}}{\pgfqpoint{1.137833in}{1.679481in}}%
\pgfpathcurveto{\pgfqpoint{1.129596in}{1.679481in}}{\pgfqpoint{1.121696in}{1.676209in}}{\pgfqpoint{1.115872in}{1.670385in}}%
\pgfpathcurveto{\pgfqpoint{1.110048in}{1.664561in}}{\pgfqpoint{1.106776in}{1.656661in}}{\pgfqpoint{1.106776in}{1.648424in}}%
\pgfpathcurveto{\pgfqpoint{1.106776in}{1.640188in}}{\pgfqpoint{1.110048in}{1.632288in}}{\pgfqpoint{1.115872in}{1.626464in}}%
\pgfpathcurveto{\pgfqpoint{1.121696in}{1.620640in}}{\pgfqpoint{1.129596in}{1.617368in}}{\pgfqpoint{1.137833in}{1.617368in}}%
\pgfpathclose%
\pgfusepath{stroke,fill}%
\end{pgfscope}%
\begin{pgfscope}%
\pgfpathrectangle{\pgfqpoint{0.100000in}{0.212622in}}{\pgfqpoint{3.696000in}{3.696000in}}%
\pgfusepath{clip}%
\pgfsetbuttcap%
\pgfsetroundjoin%
\definecolor{currentfill}{rgb}{0.121569,0.466667,0.705882}%
\pgfsetfillcolor{currentfill}%
\pgfsetfillopacity{0.300457}%
\pgfsetlinewidth{1.003750pt}%
\definecolor{currentstroke}{rgb}{0.121569,0.466667,0.705882}%
\pgfsetstrokecolor{currentstroke}%
\pgfsetstrokeopacity{0.300457}%
\pgfsetdash{}{0pt}%
\pgfpathmoveto{\pgfqpoint{1.137836in}{1.617367in}}%
\pgfpathcurveto{\pgfqpoint{1.146072in}{1.617367in}}{\pgfqpoint{1.153972in}{1.620639in}}{\pgfqpoint{1.159796in}{1.626463in}}%
\pgfpathcurveto{\pgfqpoint{1.165620in}{1.632287in}}{\pgfqpoint{1.168893in}{1.640187in}}{\pgfqpoint{1.168893in}{1.648423in}}%
\pgfpathcurveto{\pgfqpoint{1.168893in}{1.656660in}}{\pgfqpoint{1.165620in}{1.664560in}}{\pgfqpoint{1.159796in}{1.670384in}}%
\pgfpathcurveto{\pgfqpoint{1.153972in}{1.676208in}}{\pgfqpoint{1.146072in}{1.679480in}}{\pgfqpoint{1.137836in}{1.679480in}}%
\pgfpathcurveto{\pgfqpoint{1.129600in}{1.679480in}}{\pgfqpoint{1.121700in}{1.676208in}}{\pgfqpoint{1.115876in}{1.670384in}}%
\pgfpathcurveto{\pgfqpoint{1.110052in}{1.664560in}}{\pgfqpoint{1.106780in}{1.656660in}}{\pgfqpoint{1.106780in}{1.648423in}}%
\pgfpathcurveto{\pgfqpoint{1.106780in}{1.640187in}}{\pgfqpoint{1.110052in}{1.632287in}}{\pgfqpoint{1.115876in}{1.626463in}}%
\pgfpathcurveto{\pgfqpoint{1.121700in}{1.620639in}}{\pgfqpoint{1.129600in}{1.617367in}}{\pgfqpoint{1.137836in}{1.617367in}}%
\pgfpathclose%
\pgfusepath{stroke,fill}%
\end{pgfscope}%
\begin{pgfscope}%
\pgfpathrectangle{\pgfqpoint{0.100000in}{0.212622in}}{\pgfqpoint{3.696000in}{3.696000in}}%
\pgfusepath{clip}%
\pgfsetbuttcap%
\pgfsetroundjoin%
\definecolor{currentfill}{rgb}{0.121569,0.466667,0.705882}%
\pgfsetfillcolor{currentfill}%
\pgfsetfillopacity{0.300457}%
\pgfsetlinewidth{1.003750pt}%
\definecolor{currentstroke}{rgb}{0.121569,0.466667,0.705882}%
\pgfsetstrokecolor{currentstroke}%
\pgfsetstrokeopacity{0.300457}%
\pgfsetdash{}{0pt}%
\pgfpathmoveto{\pgfqpoint{1.137838in}{1.617366in}}%
\pgfpathcurveto{\pgfqpoint{1.146074in}{1.617366in}}{\pgfqpoint{1.153974in}{1.620639in}}{\pgfqpoint{1.159798in}{1.626463in}}%
\pgfpathcurveto{\pgfqpoint{1.165622in}{1.632286in}}{\pgfqpoint{1.168895in}{1.640187in}}{\pgfqpoint{1.168895in}{1.648423in}}%
\pgfpathcurveto{\pgfqpoint{1.168895in}{1.656659in}}{\pgfqpoint{1.165622in}{1.664559in}}{\pgfqpoint{1.159798in}{1.670383in}}%
\pgfpathcurveto{\pgfqpoint{1.153974in}{1.676207in}}{\pgfqpoint{1.146074in}{1.679479in}}{\pgfqpoint{1.137838in}{1.679479in}}%
\pgfpathcurveto{\pgfqpoint{1.129602in}{1.679479in}}{\pgfqpoint{1.121702in}{1.676207in}}{\pgfqpoint{1.115878in}{1.670383in}}%
\pgfpathcurveto{\pgfqpoint{1.110054in}{1.664559in}}{\pgfqpoint{1.106782in}{1.656659in}}{\pgfqpoint{1.106782in}{1.648423in}}%
\pgfpathcurveto{\pgfqpoint{1.106782in}{1.640187in}}{\pgfqpoint{1.110054in}{1.632286in}}{\pgfqpoint{1.115878in}{1.626463in}}%
\pgfpathcurveto{\pgfqpoint{1.121702in}{1.620639in}}{\pgfqpoint{1.129602in}{1.617366in}}{\pgfqpoint{1.137838in}{1.617366in}}%
\pgfpathclose%
\pgfusepath{stroke,fill}%
\end{pgfscope}%
\begin{pgfscope}%
\pgfpathrectangle{\pgfqpoint{0.100000in}{0.212622in}}{\pgfqpoint{3.696000in}{3.696000in}}%
\pgfusepath{clip}%
\pgfsetbuttcap%
\pgfsetroundjoin%
\definecolor{currentfill}{rgb}{0.121569,0.466667,0.705882}%
\pgfsetfillcolor{currentfill}%
\pgfsetfillopacity{0.300458}%
\pgfsetlinewidth{1.003750pt}%
\definecolor{currentstroke}{rgb}{0.121569,0.466667,0.705882}%
\pgfsetstrokecolor{currentstroke}%
\pgfsetstrokeopacity{0.300458}%
\pgfsetdash{}{0pt}%
\pgfpathmoveto{\pgfqpoint{1.137839in}{1.617366in}}%
\pgfpathcurveto{\pgfqpoint{1.146075in}{1.617366in}}{\pgfqpoint{1.153976in}{1.620638in}}{\pgfqpoint{1.159799in}{1.626462in}}%
\pgfpathcurveto{\pgfqpoint{1.165623in}{1.632286in}}{\pgfqpoint{1.168896in}{1.640186in}}{\pgfqpoint{1.168896in}{1.648422in}}%
\pgfpathcurveto{\pgfqpoint{1.168896in}{1.656659in}}{\pgfqpoint{1.165623in}{1.664559in}}{\pgfqpoint{1.159799in}{1.670383in}}%
\pgfpathcurveto{\pgfqpoint{1.153976in}{1.676207in}}{\pgfqpoint{1.146075in}{1.679479in}}{\pgfqpoint{1.137839in}{1.679479in}}%
\pgfpathcurveto{\pgfqpoint{1.129603in}{1.679479in}}{\pgfqpoint{1.121703in}{1.676207in}}{\pgfqpoint{1.115879in}{1.670383in}}%
\pgfpathcurveto{\pgfqpoint{1.110055in}{1.664559in}}{\pgfqpoint{1.106783in}{1.656659in}}{\pgfqpoint{1.106783in}{1.648422in}}%
\pgfpathcurveto{\pgfqpoint{1.106783in}{1.640186in}}{\pgfqpoint{1.110055in}{1.632286in}}{\pgfqpoint{1.115879in}{1.626462in}}%
\pgfpathcurveto{\pgfqpoint{1.121703in}{1.620638in}}{\pgfqpoint{1.129603in}{1.617366in}}{\pgfqpoint{1.137839in}{1.617366in}}%
\pgfpathclose%
\pgfusepath{stroke,fill}%
\end{pgfscope}%
\begin{pgfscope}%
\pgfpathrectangle{\pgfqpoint{0.100000in}{0.212622in}}{\pgfqpoint{3.696000in}{3.696000in}}%
\pgfusepath{clip}%
\pgfsetbuttcap%
\pgfsetroundjoin%
\definecolor{currentfill}{rgb}{0.121569,0.466667,0.705882}%
\pgfsetfillcolor{currentfill}%
\pgfsetfillopacity{0.300458}%
\pgfsetlinewidth{1.003750pt}%
\definecolor{currentstroke}{rgb}{0.121569,0.466667,0.705882}%
\pgfsetstrokecolor{currentstroke}%
\pgfsetstrokeopacity{0.300458}%
\pgfsetdash{}{0pt}%
\pgfpathmoveto{\pgfqpoint{1.137840in}{1.617366in}}%
\pgfpathcurveto{\pgfqpoint{1.146076in}{1.617366in}}{\pgfqpoint{1.153976in}{1.620638in}}{\pgfqpoint{1.159800in}{1.626462in}}%
\pgfpathcurveto{\pgfqpoint{1.165624in}{1.632286in}}{\pgfqpoint{1.168896in}{1.640186in}}{\pgfqpoint{1.168896in}{1.648422in}}%
\pgfpathcurveto{\pgfqpoint{1.168896in}{1.656659in}}{\pgfqpoint{1.165624in}{1.664559in}}{\pgfqpoint{1.159800in}{1.670382in}}%
\pgfpathcurveto{\pgfqpoint{1.153976in}{1.676206in}}{\pgfqpoint{1.146076in}{1.679479in}}{\pgfqpoint{1.137840in}{1.679479in}}%
\pgfpathcurveto{\pgfqpoint{1.129603in}{1.679479in}}{\pgfqpoint{1.121703in}{1.676206in}}{\pgfqpoint{1.115880in}{1.670382in}}%
\pgfpathcurveto{\pgfqpoint{1.110056in}{1.664559in}}{\pgfqpoint{1.106783in}{1.656659in}}{\pgfqpoint{1.106783in}{1.648422in}}%
\pgfpathcurveto{\pgfqpoint{1.106783in}{1.640186in}}{\pgfqpoint{1.110056in}{1.632286in}}{\pgfqpoint{1.115880in}{1.626462in}}%
\pgfpathcurveto{\pgfqpoint{1.121703in}{1.620638in}}{\pgfqpoint{1.129603in}{1.617366in}}{\pgfqpoint{1.137840in}{1.617366in}}%
\pgfpathclose%
\pgfusepath{stroke,fill}%
\end{pgfscope}%
\begin{pgfscope}%
\pgfpathrectangle{\pgfqpoint{0.100000in}{0.212622in}}{\pgfqpoint{3.696000in}{3.696000in}}%
\pgfusepath{clip}%
\pgfsetbuttcap%
\pgfsetroundjoin%
\definecolor{currentfill}{rgb}{0.121569,0.466667,0.705882}%
\pgfsetfillcolor{currentfill}%
\pgfsetfillopacity{0.300458}%
\pgfsetlinewidth{1.003750pt}%
\definecolor{currentstroke}{rgb}{0.121569,0.466667,0.705882}%
\pgfsetstrokecolor{currentstroke}%
\pgfsetstrokeopacity{0.300458}%
\pgfsetdash{}{0pt}%
\pgfpathmoveto{\pgfqpoint{1.137840in}{1.617366in}}%
\pgfpathcurveto{\pgfqpoint{1.146076in}{1.617366in}}{\pgfqpoint{1.153976in}{1.620638in}}{\pgfqpoint{1.159800in}{1.626462in}}%
\pgfpathcurveto{\pgfqpoint{1.165624in}{1.632286in}}{\pgfqpoint{1.168897in}{1.640186in}}{\pgfqpoint{1.168897in}{1.648422in}}%
\pgfpathcurveto{\pgfqpoint{1.168897in}{1.656658in}}{\pgfqpoint{1.165624in}{1.664558in}}{\pgfqpoint{1.159800in}{1.670382in}}%
\pgfpathcurveto{\pgfqpoint{1.153976in}{1.676206in}}{\pgfqpoint{1.146076in}{1.679479in}}{\pgfqpoint{1.137840in}{1.679479in}}%
\pgfpathcurveto{\pgfqpoint{1.129604in}{1.679479in}}{\pgfqpoint{1.121704in}{1.676206in}}{\pgfqpoint{1.115880in}{1.670382in}}%
\pgfpathcurveto{\pgfqpoint{1.110056in}{1.664558in}}{\pgfqpoint{1.106784in}{1.656658in}}{\pgfqpoint{1.106784in}{1.648422in}}%
\pgfpathcurveto{\pgfqpoint{1.106784in}{1.640186in}}{\pgfqpoint{1.110056in}{1.632286in}}{\pgfqpoint{1.115880in}{1.626462in}}%
\pgfpathcurveto{\pgfqpoint{1.121704in}{1.620638in}}{\pgfqpoint{1.129604in}{1.617366in}}{\pgfqpoint{1.137840in}{1.617366in}}%
\pgfpathclose%
\pgfusepath{stroke,fill}%
\end{pgfscope}%
\begin{pgfscope}%
\pgfpathrectangle{\pgfqpoint{0.100000in}{0.212622in}}{\pgfqpoint{3.696000in}{3.696000in}}%
\pgfusepath{clip}%
\pgfsetbuttcap%
\pgfsetroundjoin%
\definecolor{currentfill}{rgb}{0.121569,0.466667,0.705882}%
\pgfsetfillcolor{currentfill}%
\pgfsetfillopacity{0.300458}%
\pgfsetlinewidth{1.003750pt}%
\definecolor{currentstroke}{rgb}{0.121569,0.466667,0.705882}%
\pgfsetstrokecolor{currentstroke}%
\pgfsetstrokeopacity{0.300458}%
\pgfsetdash{}{0pt}%
\pgfpathmoveto{\pgfqpoint{1.137840in}{1.617366in}}%
\pgfpathcurveto{\pgfqpoint{1.146077in}{1.617366in}}{\pgfqpoint{1.153977in}{1.620638in}}{\pgfqpoint{1.159801in}{1.626462in}}%
\pgfpathcurveto{\pgfqpoint{1.165624in}{1.632286in}}{\pgfqpoint{1.168897in}{1.640186in}}{\pgfqpoint{1.168897in}{1.648422in}}%
\pgfpathcurveto{\pgfqpoint{1.168897in}{1.656658in}}{\pgfqpoint{1.165624in}{1.664558in}}{\pgfqpoint{1.159801in}{1.670382in}}%
\pgfpathcurveto{\pgfqpoint{1.153977in}{1.676206in}}{\pgfqpoint{1.146077in}{1.679479in}}{\pgfqpoint{1.137840in}{1.679479in}}%
\pgfpathcurveto{\pgfqpoint{1.129604in}{1.679479in}}{\pgfqpoint{1.121704in}{1.676206in}}{\pgfqpoint{1.115880in}{1.670382in}}%
\pgfpathcurveto{\pgfqpoint{1.110056in}{1.664558in}}{\pgfqpoint{1.106784in}{1.656658in}}{\pgfqpoint{1.106784in}{1.648422in}}%
\pgfpathcurveto{\pgfqpoint{1.106784in}{1.640186in}}{\pgfqpoint{1.110056in}{1.632286in}}{\pgfqpoint{1.115880in}{1.626462in}}%
\pgfpathcurveto{\pgfqpoint{1.121704in}{1.620638in}}{\pgfqpoint{1.129604in}{1.617366in}}{\pgfqpoint{1.137840in}{1.617366in}}%
\pgfpathclose%
\pgfusepath{stroke,fill}%
\end{pgfscope}%
\begin{pgfscope}%
\pgfpathrectangle{\pgfqpoint{0.100000in}{0.212622in}}{\pgfqpoint{3.696000in}{3.696000in}}%
\pgfusepath{clip}%
\pgfsetbuttcap%
\pgfsetroundjoin%
\definecolor{currentfill}{rgb}{0.121569,0.466667,0.705882}%
\pgfsetfillcolor{currentfill}%
\pgfsetfillopacity{0.300458}%
\pgfsetlinewidth{1.003750pt}%
\definecolor{currentstroke}{rgb}{0.121569,0.466667,0.705882}%
\pgfsetstrokecolor{currentstroke}%
\pgfsetstrokeopacity{0.300458}%
\pgfsetdash{}{0pt}%
\pgfpathmoveto{\pgfqpoint{1.137840in}{1.617366in}}%
\pgfpathcurveto{\pgfqpoint{1.146077in}{1.617366in}}{\pgfqpoint{1.153977in}{1.620638in}}{\pgfqpoint{1.159801in}{1.626462in}}%
\pgfpathcurveto{\pgfqpoint{1.165625in}{1.632286in}}{\pgfqpoint{1.168897in}{1.640186in}}{\pgfqpoint{1.168897in}{1.648422in}}%
\pgfpathcurveto{\pgfqpoint{1.168897in}{1.656658in}}{\pgfqpoint{1.165625in}{1.664558in}}{\pgfqpoint{1.159801in}{1.670382in}}%
\pgfpathcurveto{\pgfqpoint{1.153977in}{1.676206in}}{\pgfqpoint{1.146077in}{1.679479in}}{\pgfqpoint{1.137840in}{1.679479in}}%
\pgfpathcurveto{\pgfqpoint{1.129604in}{1.679479in}}{\pgfqpoint{1.121704in}{1.676206in}}{\pgfqpoint{1.115880in}{1.670382in}}%
\pgfpathcurveto{\pgfqpoint{1.110056in}{1.664558in}}{\pgfqpoint{1.106784in}{1.656658in}}{\pgfqpoint{1.106784in}{1.648422in}}%
\pgfpathcurveto{\pgfqpoint{1.106784in}{1.640186in}}{\pgfqpoint{1.110056in}{1.632286in}}{\pgfqpoint{1.115880in}{1.626462in}}%
\pgfpathcurveto{\pgfqpoint{1.121704in}{1.620638in}}{\pgfqpoint{1.129604in}{1.617366in}}{\pgfqpoint{1.137840in}{1.617366in}}%
\pgfpathclose%
\pgfusepath{stroke,fill}%
\end{pgfscope}%
\begin{pgfscope}%
\pgfpathrectangle{\pgfqpoint{0.100000in}{0.212622in}}{\pgfqpoint{3.696000in}{3.696000in}}%
\pgfusepath{clip}%
\pgfsetbuttcap%
\pgfsetroundjoin%
\definecolor{currentfill}{rgb}{0.121569,0.466667,0.705882}%
\pgfsetfillcolor{currentfill}%
\pgfsetfillopacity{0.300458}%
\pgfsetlinewidth{1.003750pt}%
\definecolor{currentstroke}{rgb}{0.121569,0.466667,0.705882}%
\pgfsetstrokecolor{currentstroke}%
\pgfsetstrokeopacity{0.300458}%
\pgfsetdash{}{0pt}%
\pgfpathmoveto{\pgfqpoint{1.137840in}{1.617366in}}%
\pgfpathcurveto{\pgfqpoint{1.146077in}{1.617366in}}{\pgfqpoint{1.153977in}{1.620638in}}{\pgfqpoint{1.159801in}{1.626462in}}%
\pgfpathcurveto{\pgfqpoint{1.165625in}{1.632286in}}{\pgfqpoint{1.168897in}{1.640186in}}{\pgfqpoint{1.168897in}{1.648422in}}%
\pgfpathcurveto{\pgfqpoint{1.168897in}{1.656658in}}{\pgfqpoint{1.165625in}{1.664558in}}{\pgfqpoint{1.159801in}{1.670382in}}%
\pgfpathcurveto{\pgfqpoint{1.153977in}{1.676206in}}{\pgfqpoint{1.146077in}{1.679479in}}{\pgfqpoint{1.137840in}{1.679479in}}%
\pgfpathcurveto{\pgfqpoint{1.129604in}{1.679479in}}{\pgfqpoint{1.121704in}{1.676206in}}{\pgfqpoint{1.115880in}{1.670382in}}%
\pgfpathcurveto{\pgfqpoint{1.110056in}{1.664558in}}{\pgfqpoint{1.106784in}{1.656658in}}{\pgfqpoint{1.106784in}{1.648422in}}%
\pgfpathcurveto{\pgfqpoint{1.106784in}{1.640186in}}{\pgfqpoint{1.110056in}{1.632286in}}{\pgfqpoint{1.115880in}{1.626462in}}%
\pgfpathcurveto{\pgfqpoint{1.121704in}{1.620638in}}{\pgfqpoint{1.129604in}{1.617366in}}{\pgfqpoint{1.137840in}{1.617366in}}%
\pgfpathclose%
\pgfusepath{stroke,fill}%
\end{pgfscope}%
\begin{pgfscope}%
\pgfpathrectangle{\pgfqpoint{0.100000in}{0.212622in}}{\pgfqpoint{3.696000in}{3.696000in}}%
\pgfusepath{clip}%
\pgfsetbuttcap%
\pgfsetroundjoin%
\definecolor{currentfill}{rgb}{0.121569,0.466667,0.705882}%
\pgfsetfillcolor{currentfill}%
\pgfsetfillopacity{0.300458}%
\pgfsetlinewidth{1.003750pt}%
\definecolor{currentstroke}{rgb}{0.121569,0.466667,0.705882}%
\pgfsetstrokecolor{currentstroke}%
\pgfsetstrokeopacity{0.300458}%
\pgfsetdash{}{0pt}%
\pgfpathmoveto{\pgfqpoint{1.137840in}{1.617366in}}%
\pgfpathcurveto{\pgfqpoint{1.146077in}{1.617366in}}{\pgfqpoint{1.153977in}{1.620638in}}{\pgfqpoint{1.159801in}{1.626462in}}%
\pgfpathcurveto{\pgfqpoint{1.165625in}{1.632286in}}{\pgfqpoint{1.168897in}{1.640186in}}{\pgfqpoint{1.168897in}{1.648422in}}%
\pgfpathcurveto{\pgfqpoint{1.168897in}{1.656658in}}{\pgfqpoint{1.165625in}{1.664558in}}{\pgfqpoint{1.159801in}{1.670382in}}%
\pgfpathcurveto{\pgfqpoint{1.153977in}{1.676206in}}{\pgfqpoint{1.146077in}{1.679479in}}{\pgfqpoint{1.137840in}{1.679479in}}%
\pgfpathcurveto{\pgfqpoint{1.129604in}{1.679479in}}{\pgfqpoint{1.121704in}{1.676206in}}{\pgfqpoint{1.115880in}{1.670382in}}%
\pgfpathcurveto{\pgfqpoint{1.110056in}{1.664558in}}{\pgfqpoint{1.106784in}{1.656658in}}{\pgfqpoint{1.106784in}{1.648422in}}%
\pgfpathcurveto{\pgfqpoint{1.106784in}{1.640186in}}{\pgfqpoint{1.110056in}{1.632286in}}{\pgfqpoint{1.115880in}{1.626462in}}%
\pgfpathcurveto{\pgfqpoint{1.121704in}{1.620638in}}{\pgfqpoint{1.129604in}{1.617366in}}{\pgfqpoint{1.137840in}{1.617366in}}%
\pgfpathclose%
\pgfusepath{stroke,fill}%
\end{pgfscope}%
\begin{pgfscope}%
\pgfpathrectangle{\pgfqpoint{0.100000in}{0.212622in}}{\pgfqpoint{3.696000in}{3.696000in}}%
\pgfusepath{clip}%
\pgfsetbuttcap%
\pgfsetroundjoin%
\definecolor{currentfill}{rgb}{0.121569,0.466667,0.705882}%
\pgfsetfillcolor{currentfill}%
\pgfsetfillopacity{0.300458}%
\pgfsetlinewidth{1.003750pt}%
\definecolor{currentstroke}{rgb}{0.121569,0.466667,0.705882}%
\pgfsetstrokecolor{currentstroke}%
\pgfsetstrokeopacity{0.300458}%
\pgfsetdash{}{0pt}%
\pgfpathmoveto{\pgfqpoint{1.137841in}{1.617366in}}%
\pgfpathcurveto{\pgfqpoint{1.146077in}{1.617366in}}{\pgfqpoint{1.153977in}{1.620638in}}{\pgfqpoint{1.159801in}{1.626462in}}%
\pgfpathcurveto{\pgfqpoint{1.165625in}{1.632286in}}{\pgfqpoint{1.168897in}{1.640186in}}{\pgfqpoint{1.168897in}{1.648422in}}%
\pgfpathcurveto{\pgfqpoint{1.168897in}{1.656658in}}{\pgfqpoint{1.165625in}{1.664558in}}{\pgfqpoint{1.159801in}{1.670382in}}%
\pgfpathcurveto{\pgfqpoint{1.153977in}{1.676206in}}{\pgfqpoint{1.146077in}{1.679479in}}{\pgfqpoint{1.137841in}{1.679479in}}%
\pgfpathcurveto{\pgfqpoint{1.129604in}{1.679479in}}{\pgfqpoint{1.121704in}{1.676206in}}{\pgfqpoint{1.115880in}{1.670382in}}%
\pgfpathcurveto{\pgfqpoint{1.110056in}{1.664558in}}{\pgfqpoint{1.106784in}{1.656658in}}{\pgfqpoint{1.106784in}{1.648422in}}%
\pgfpathcurveto{\pgfqpoint{1.106784in}{1.640186in}}{\pgfqpoint{1.110056in}{1.632286in}}{\pgfqpoint{1.115880in}{1.626462in}}%
\pgfpathcurveto{\pgfqpoint{1.121704in}{1.620638in}}{\pgfqpoint{1.129604in}{1.617366in}}{\pgfqpoint{1.137841in}{1.617366in}}%
\pgfpathclose%
\pgfusepath{stroke,fill}%
\end{pgfscope}%
\begin{pgfscope}%
\pgfpathrectangle{\pgfqpoint{0.100000in}{0.212622in}}{\pgfqpoint{3.696000in}{3.696000in}}%
\pgfusepath{clip}%
\pgfsetbuttcap%
\pgfsetroundjoin%
\definecolor{currentfill}{rgb}{0.121569,0.466667,0.705882}%
\pgfsetfillcolor{currentfill}%
\pgfsetfillopacity{0.300458}%
\pgfsetlinewidth{1.003750pt}%
\definecolor{currentstroke}{rgb}{0.121569,0.466667,0.705882}%
\pgfsetstrokecolor{currentstroke}%
\pgfsetstrokeopacity{0.300458}%
\pgfsetdash{}{0pt}%
\pgfpathmoveto{\pgfqpoint{1.137841in}{1.617366in}}%
\pgfpathcurveto{\pgfqpoint{1.146077in}{1.617366in}}{\pgfqpoint{1.153977in}{1.620638in}}{\pgfqpoint{1.159801in}{1.626462in}}%
\pgfpathcurveto{\pgfqpoint{1.165625in}{1.632286in}}{\pgfqpoint{1.168897in}{1.640186in}}{\pgfqpoint{1.168897in}{1.648422in}}%
\pgfpathcurveto{\pgfqpoint{1.168897in}{1.656658in}}{\pgfqpoint{1.165625in}{1.664558in}}{\pgfqpoint{1.159801in}{1.670382in}}%
\pgfpathcurveto{\pgfqpoint{1.153977in}{1.676206in}}{\pgfqpoint{1.146077in}{1.679479in}}{\pgfqpoint{1.137841in}{1.679479in}}%
\pgfpathcurveto{\pgfqpoint{1.129604in}{1.679479in}}{\pgfqpoint{1.121704in}{1.676206in}}{\pgfqpoint{1.115880in}{1.670382in}}%
\pgfpathcurveto{\pgfqpoint{1.110056in}{1.664558in}}{\pgfqpoint{1.106784in}{1.656658in}}{\pgfqpoint{1.106784in}{1.648422in}}%
\pgfpathcurveto{\pgfqpoint{1.106784in}{1.640186in}}{\pgfqpoint{1.110056in}{1.632286in}}{\pgfqpoint{1.115880in}{1.626462in}}%
\pgfpathcurveto{\pgfqpoint{1.121704in}{1.620638in}}{\pgfqpoint{1.129604in}{1.617366in}}{\pgfqpoint{1.137841in}{1.617366in}}%
\pgfpathclose%
\pgfusepath{stroke,fill}%
\end{pgfscope}%
\begin{pgfscope}%
\pgfpathrectangle{\pgfqpoint{0.100000in}{0.212622in}}{\pgfqpoint{3.696000in}{3.696000in}}%
\pgfusepath{clip}%
\pgfsetbuttcap%
\pgfsetroundjoin%
\definecolor{currentfill}{rgb}{0.121569,0.466667,0.705882}%
\pgfsetfillcolor{currentfill}%
\pgfsetfillopacity{0.300458}%
\pgfsetlinewidth{1.003750pt}%
\definecolor{currentstroke}{rgb}{0.121569,0.466667,0.705882}%
\pgfsetstrokecolor{currentstroke}%
\pgfsetstrokeopacity{0.300458}%
\pgfsetdash{}{0pt}%
\pgfpathmoveto{\pgfqpoint{1.137841in}{1.617366in}}%
\pgfpathcurveto{\pgfqpoint{1.146077in}{1.617366in}}{\pgfqpoint{1.153977in}{1.620638in}}{\pgfqpoint{1.159801in}{1.626462in}}%
\pgfpathcurveto{\pgfqpoint{1.165625in}{1.632286in}}{\pgfqpoint{1.168897in}{1.640186in}}{\pgfqpoint{1.168897in}{1.648422in}}%
\pgfpathcurveto{\pgfqpoint{1.168897in}{1.656658in}}{\pgfqpoint{1.165625in}{1.664558in}}{\pgfqpoint{1.159801in}{1.670382in}}%
\pgfpathcurveto{\pgfqpoint{1.153977in}{1.676206in}}{\pgfqpoint{1.146077in}{1.679479in}}{\pgfqpoint{1.137841in}{1.679479in}}%
\pgfpathcurveto{\pgfqpoint{1.129604in}{1.679479in}}{\pgfqpoint{1.121704in}{1.676206in}}{\pgfqpoint{1.115880in}{1.670382in}}%
\pgfpathcurveto{\pgfqpoint{1.110056in}{1.664558in}}{\pgfqpoint{1.106784in}{1.656658in}}{\pgfqpoint{1.106784in}{1.648422in}}%
\pgfpathcurveto{\pgfqpoint{1.106784in}{1.640186in}}{\pgfqpoint{1.110056in}{1.632286in}}{\pgfqpoint{1.115880in}{1.626462in}}%
\pgfpathcurveto{\pgfqpoint{1.121704in}{1.620638in}}{\pgfqpoint{1.129604in}{1.617366in}}{\pgfqpoint{1.137841in}{1.617366in}}%
\pgfpathclose%
\pgfusepath{stroke,fill}%
\end{pgfscope}%
\begin{pgfscope}%
\pgfpathrectangle{\pgfqpoint{0.100000in}{0.212622in}}{\pgfqpoint{3.696000in}{3.696000in}}%
\pgfusepath{clip}%
\pgfsetbuttcap%
\pgfsetroundjoin%
\definecolor{currentfill}{rgb}{0.121569,0.466667,0.705882}%
\pgfsetfillcolor{currentfill}%
\pgfsetfillopacity{0.300458}%
\pgfsetlinewidth{1.003750pt}%
\definecolor{currentstroke}{rgb}{0.121569,0.466667,0.705882}%
\pgfsetstrokecolor{currentstroke}%
\pgfsetstrokeopacity{0.300458}%
\pgfsetdash{}{0pt}%
\pgfpathmoveto{\pgfqpoint{1.137841in}{1.617366in}}%
\pgfpathcurveto{\pgfqpoint{1.146077in}{1.617366in}}{\pgfqpoint{1.153977in}{1.620638in}}{\pgfqpoint{1.159801in}{1.626462in}}%
\pgfpathcurveto{\pgfqpoint{1.165625in}{1.632286in}}{\pgfqpoint{1.168897in}{1.640186in}}{\pgfqpoint{1.168897in}{1.648422in}}%
\pgfpathcurveto{\pgfqpoint{1.168897in}{1.656658in}}{\pgfqpoint{1.165625in}{1.664558in}}{\pgfqpoint{1.159801in}{1.670382in}}%
\pgfpathcurveto{\pgfqpoint{1.153977in}{1.676206in}}{\pgfqpoint{1.146077in}{1.679479in}}{\pgfqpoint{1.137841in}{1.679479in}}%
\pgfpathcurveto{\pgfqpoint{1.129604in}{1.679479in}}{\pgfqpoint{1.121704in}{1.676206in}}{\pgfqpoint{1.115880in}{1.670382in}}%
\pgfpathcurveto{\pgfqpoint{1.110056in}{1.664558in}}{\pgfqpoint{1.106784in}{1.656658in}}{\pgfqpoint{1.106784in}{1.648422in}}%
\pgfpathcurveto{\pgfqpoint{1.106784in}{1.640186in}}{\pgfqpoint{1.110056in}{1.632286in}}{\pgfqpoint{1.115880in}{1.626462in}}%
\pgfpathcurveto{\pgfqpoint{1.121704in}{1.620638in}}{\pgfqpoint{1.129604in}{1.617366in}}{\pgfqpoint{1.137841in}{1.617366in}}%
\pgfpathclose%
\pgfusepath{stroke,fill}%
\end{pgfscope}%
\begin{pgfscope}%
\pgfpathrectangle{\pgfqpoint{0.100000in}{0.212622in}}{\pgfqpoint{3.696000in}{3.696000in}}%
\pgfusepath{clip}%
\pgfsetbuttcap%
\pgfsetroundjoin%
\definecolor{currentfill}{rgb}{0.121569,0.466667,0.705882}%
\pgfsetfillcolor{currentfill}%
\pgfsetfillopacity{0.300458}%
\pgfsetlinewidth{1.003750pt}%
\definecolor{currentstroke}{rgb}{0.121569,0.466667,0.705882}%
\pgfsetstrokecolor{currentstroke}%
\pgfsetstrokeopacity{0.300458}%
\pgfsetdash{}{0pt}%
\pgfpathmoveto{\pgfqpoint{1.137841in}{1.617366in}}%
\pgfpathcurveto{\pgfqpoint{1.146077in}{1.617366in}}{\pgfqpoint{1.153977in}{1.620638in}}{\pgfqpoint{1.159801in}{1.626462in}}%
\pgfpathcurveto{\pgfqpoint{1.165625in}{1.632286in}}{\pgfqpoint{1.168897in}{1.640186in}}{\pgfqpoint{1.168897in}{1.648422in}}%
\pgfpathcurveto{\pgfqpoint{1.168897in}{1.656658in}}{\pgfqpoint{1.165625in}{1.664558in}}{\pgfqpoint{1.159801in}{1.670382in}}%
\pgfpathcurveto{\pgfqpoint{1.153977in}{1.676206in}}{\pgfqpoint{1.146077in}{1.679479in}}{\pgfqpoint{1.137841in}{1.679479in}}%
\pgfpathcurveto{\pgfqpoint{1.129604in}{1.679479in}}{\pgfqpoint{1.121704in}{1.676206in}}{\pgfqpoint{1.115880in}{1.670382in}}%
\pgfpathcurveto{\pgfqpoint{1.110056in}{1.664558in}}{\pgfqpoint{1.106784in}{1.656658in}}{\pgfqpoint{1.106784in}{1.648422in}}%
\pgfpathcurveto{\pgfqpoint{1.106784in}{1.640186in}}{\pgfqpoint{1.110056in}{1.632286in}}{\pgfqpoint{1.115880in}{1.626462in}}%
\pgfpathcurveto{\pgfqpoint{1.121704in}{1.620638in}}{\pgfqpoint{1.129604in}{1.617366in}}{\pgfqpoint{1.137841in}{1.617366in}}%
\pgfpathclose%
\pgfusepath{stroke,fill}%
\end{pgfscope}%
\begin{pgfscope}%
\pgfpathrectangle{\pgfqpoint{0.100000in}{0.212622in}}{\pgfqpoint{3.696000in}{3.696000in}}%
\pgfusepath{clip}%
\pgfsetbuttcap%
\pgfsetroundjoin%
\definecolor{currentfill}{rgb}{0.121569,0.466667,0.705882}%
\pgfsetfillcolor{currentfill}%
\pgfsetfillopacity{0.300458}%
\pgfsetlinewidth{1.003750pt}%
\definecolor{currentstroke}{rgb}{0.121569,0.466667,0.705882}%
\pgfsetstrokecolor{currentstroke}%
\pgfsetstrokeopacity{0.300458}%
\pgfsetdash{}{0pt}%
\pgfpathmoveto{\pgfqpoint{1.137841in}{1.617366in}}%
\pgfpathcurveto{\pgfqpoint{1.146077in}{1.617366in}}{\pgfqpoint{1.153977in}{1.620638in}}{\pgfqpoint{1.159801in}{1.626462in}}%
\pgfpathcurveto{\pgfqpoint{1.165625in}{1.632286in}}{\pgfqpoint{1.168897in}{1.640186in}}{\pgfqpoint{1.168897in}{1.648422in}}%
\pgfpathcurveto{\pgfqpoint{1.168897in}{1.656658in}}{\pgfqpoint{1.165625in}{1.664558in}}{\pgfqpoint{1.159801in}{1.670382in}}%
\pgfpathcurveto{\pgfqpoint{1.153977in}{1.676206in}}{\pgfqpoint{1.146077in}{1.679479in}}{\pgfqpoint{1.137841in}{1.679479in}}%
\pgfpathcurveto{\pgfqpoint{1.129604in}{1.679479in}}{\pgfqpoint{1.121704in}{1.676206in}}{\pgfqpoint{1.115880in}{1.670382in}}%
\pgfpathcurveto{\pgfqpoint{1.110056in}{1.664558in}}{\pgfqpoint{1.106784in}{1.656658in}}{\pgfqpoint{1.106784in}{1.648422in}}%
\pgfpathcurveto{\pgfqpoint{1.106784in}{1.640186in}}{\pgfqpoint{1.110056in}{1.632286in}}{\pgfqpoint{1.115880in}{1.626462in}}%
\pgfpathcurveto{\pgfqpoint{1.121704in}{1.620638in}}{\pgfqpoint{1.129604in}{1.617366in}}{\pgfqpoint{1.137841in}{1.617366in}}%
\pgfpathclose%
\pgfusepath{stroke,fill}%
\end{pgfscope}%
\begin{pgfscope}%
\pgfpathrectangle{\pgfqpoint{0.100000in}{0.212622in}}{\pgfqpoint{3.696000in}{3.696000in}}%
\pgfusepath{clip}%
\pgfsetbuttcap%
\pgfsetroundjoin%
\definecolor{currentfill}{rgb}{0.121569,0.466667,0.705882}%
\pgfsetfillcolor{currentfill}%
\pgfsetfillopacity{0.300458}%
\pgfsetlinewidth{1.003750pt}%
\definecolor{currentstroke}{rgb}{0.121569,0.466667,0.705882}%
\pgfsetstrokecolor{currentstroke}%
\pgfsetstrokeopacity{0.300458}%
\pgfsetdash{}{0pt}%
\pgfpathmoveto{\pgfqpoint{1.137841in}{1.617366in}}%
\pgfpathcurveto{\pgfqpoint{1.146077in}{1.617366in}}{\pgfqpoint{1.153977in}{1.620638in}}{\pgfqpoint{1.159801in}{1.626462in}}%
\pgfpathcurveto{\pgfqpoint{1.165625in}{1.632286in}}{\pgfqpoint{1.168897in}{1.640186in}}{\pgfqpoint{1.168897in}{1.648422in}}%
\pgfpathcurveto{\pgfqpoint{1.168897in}{1.656658in}}{\pgfqpoint{1.165625in}{1.664558in}}{\pgfqpoint{1.159801in}{1.670382in}}%
\pgfpathcurveto{\pgfqpoint{1.153977in}{1.676206in}}{\pgfqpoint{1.146077in}{1.679479in}}{\pgfqpoint{1.137841in}{1.679479in}}%
\pgfpathcurveto{\pgfqpoint{1.129604in}{1.679479in}}{\pgfqpoint{1.121704in}{1.676206in}}{\pgfqpoint{1.115880in}{1.670382in}}%
\pgfpathcurveto{\pgfqpoint{1.110056in}{1.664558in}}{\pgfqpoint{1.106784in}{1.656658in}}{\pgfqpoint{1.106784in}{1.648422in}}%
\pgfpathcurveto{\pgfqpoint{1.106784in}{1.640186in}}{\pgfqpoint{1.110056in}{1.632286in}}{\pgfqpoint{1.115880in}{1.626462in}}%
\pgfpathcurveto{\pgfqpoint{1.121704in}{1.620638in}}{\pgfqpoint{1.129604in}{1.617366in}}{\pgfqpoint{1.137841in}{1.617366in}}%
\pgfpathclose%
\pgfusepath{stroke,fill}%
\end{pgfscope}%
\begin{pgfscope}%
\pgfpathrectangle{\pgfqpoint{0.100000in}{0.212622in}}{\pgfqpoint{3.696000in}{3.696000in}}%
\pgfusepath{clip}%
\pgfsetbuttcap%
\pgfsetroundjoin%
\definecolor{currentfill}{rgb}{0.121569,0.466667,0.705882}%
\pgfsetfillcolor{currentfill}%
\pgfsetfillopacity{0.300458}%
\pgfsetlinewidth{1.003750pt}%
\definecolor{currentstroke}{rgb}{0.121569,0.466667,0.705882}%
\pgfsetstrokecolor{currentstroke}%
\pgfsetstrokeopacity{0.300458}%
\pgfsetdash{}{0pt}%
\pgfpathmoveto{\pgfqpoint{1.137841in}{1.617366in}}%
\pgfpathcurveto{\pgfqpoint{1.146077in}{1.617366in}}{\pgfqpoint{1.153977in}{1.620638in}}{\pgfqpoint{1.159801in}{1.626462in}}%
\pgfpathcurveto{\pgfqpoint{1.165625in}{1.632286in}}{\pgfqpoint{1.168897in}{1.640186in}}{\pgfqpoint{1.168897in}{1.648422in}}%
\pgfpathcurveto{\pgfqpoint{1.168897in}{1.656658in}}{\pgfqpoint{1.165625in}{1.664558in}}{\pgfqpoint{1.159801in}{1.670382in}}%
\pgfpathcurveto{\pgfqpoint{1.153977in}{1.676206in}}{\pgfqpoint{1.146077in}{1.679479in}}{\pgfqpoint{1.137841in}{1.679479in}}%
\pgfpathcurveto{\pgfqpoint{1.129604in}{1.679479in}}{\pgfqpoint{1.121704in}{1.676206in}}{\pgfqpoint{1.115880in}{1.670382in}}%
\pgfpathcurveto{\pgfqpoint{1.110056in}{1.664558in}}{\pgfqpoint{1.106784in}{1.656658in}}{\pgfqpoint{1.106784in}{1.648422in}}%
\pgfpathcurveto{\pgfqpoint{1.106784in}{1.640186in}}{\pgfqpoint{1.110056in}{1.632286in}}{\pgfqpoint{1.115880in}{1.626462in}}%
\pgfpathcurveto{\pgfqpoint{1.121704in}{1.620638in}}{\pgfqpoint{1.129604in}{1.617366in}}{\pgfqpoint{1.137841in}{1.617366in}}%
\pgfpathclose%
\pgfusepath{stroke,fill}%
\end{pgfscope}%
\begin{pgfscope}%
\pgfpathrectangle{\pgfqpoint{0.100000in}{0.212622in}}{\pgfqpoint{3.696000in}{3.696000in}}%
\pgfusepath{clip}%
\pgfsetbuttcap%
\pgfsetroundjoin%
\definecolor{currentfill}{rgb}{0.121569,0.466667,0.705882}%
\pgfsetfillcolor{currentfill}%
\pgfsetfillopacity{0.300458}%
\pgfsetlinewidth{1.003750pt}%
\definecolor{currentstroke}{rgb}{0.121569,0.466667,0.705882}%
\pgfsetstrokecolor{currentstroke}%
\pgfsetstrokeopacity{0.300458}%
\pgfsetdash{}{0pt}%
\pgfpathmoveto{\pgfqpoint{1.137841in}{1.617366in}}%
\pgfpathcurveto{\pgfqpoint{1.146077in}{1.617366in}}{\pgfqpoint{1.153977in}{1.620638in}}{\pgfqpoint{1.159801in}{1.626462in}}%
\pgfpathcurveto{\pgfqpoint{1.165625in}{1.632286in}}{\pgfqpoint{1.168897in}{1.640186in}}{\pgfqpoint{1.168897in}{1.648422in}}%
\pgfpathcurveto{\pgfqpoint{1.168897in}{1.656658in}}{\pgfqpoint{1.165625in}{1.664558in}}{\pgfqpoint{1.159801in}{1.670382in}}%
\pgfpathcurveto{\pgfqpoint{1.153977in}{1.676206in}}{\pgfqpoint{1.146077in}{1.679479in}}{\pgfqpoint{1.137841in}{1.679479in}}%
\pgfpathcurveto{\pgfqpoint{1.129604in}{1.679479in}}{\pgfqpoint{1.121704in}{1.676206in}}{\pgfqpoint{1.115880in}{1.670382in}}%
\pgfpathcurveto{\pgfqpoint{1.110056in}{1.664558in}}{\pgfqpoint{1.106784in}{1.656658in}}{\pgfqpoint{1.106784in}{1.648422in}}%
\pgfpathcurveto{\pgfqpoint{1.106784in}{1.640186in}}{\pgfqpoint{1.110056in}{1.632286in}}{\pgfqpoint{1.115880in}{1.626462in}}%
\pgfpathcurveto{\pgfqpoint{1.121704in}{1.620638in}}{\pgfqpoint{1.129604in}{1.617366in}}{\pgfqpoint{1.137841in}{1.617366in}}%
\pgfpathclose%
\pgfusepath{stroke,fill}%
\end{pgfscope}%
\begin{pgfscope}%
\pgfpathrectangle{\pgfqpoint{0.100000in}{0.212622in}}{\pgfqpoint{3.696000in}{3.696000in}}%
\pgfusepath{clip}%
\pgfsetbuttcap%
\pgfsetroundjoin%
\definecolor{currentfill}{rgb}{0.121569,0.466667,0.705882}%
\pgfsetfillcolor{currentfill}%
\pgfsetfillopacity{0.300458}%
\pgfsetlinewidth{1.003750pt}%
\definecolor{currentstroke}{rgb}{0.121569,0.466667,0.705882}%
\pgfsetstrokecolor{currentstroke}%
\pgfsetstrokeopacity{0.300458}%
\pgfsetdash{}{0pt}%
\pgfpathmoveto{\pgfqpoint{1.137841in}{1.617366in}}%
\pgfpathcurveto{\pgfqpoint{1.146077in}{1.617366in}}{\pgfqpoint{1.153977in}{1.620638in}}{\pgfqpoint{1.159801in}{1.626462in}}%
\pgfpathcurveto{\pgfqpoint{1.165625in}{1.632286in}}{\pgfqpoint{1.168897in}{1.640186in}}{\pgfqpoint{1.168897in}{1.648422in}}%
\pgfpathcurveto{\pgfqpoint{1.168897in}{1.656658in}}{\pgfqpoint{1.165625in}{1.664558in}}{\pgfqpoint{1.159801in}{1.670382in}}%
\pgfpathcurveto{\pgfqpoint{1.153977in}{1.676206in}}{\pgfqpoint{1.146077in}{1.679479in}}{\pgfqpoint{1.137841in}{1.679479in}}%
\pgfpathcurveto{\pgfqpoint{1.129604in}{1.679479in}}{\pgfqpoint{1.121704in}{1.676206in}}{\pgfqpoint{1.115880in}{1.670382in}}%
\pgfpathcurveto{\pgfqpoint{1.110056in}{1.664558in}}{\pgfqpoint{1.106784in}{1.656658in}}{\pgfqpoint{1.106784in}{1.648422in}}%
\pgfpathcurveto{\pgfqpoint{1.106784in}{1.640186in}}{\pgfqpoint{1.110056in}{1.632286in}}{\pgfqpoint{1.115880in}{1.626462in}}%
\pgfpathcurveto{\pgfqpoint{1.121704in}{1.620638in}}{\pgfqpoint{1.129604in}{1.617366in}}{\pgfqpoint{1.137841in}{1.617366in}}%
\pgfpathclose%
\pgfusepath{stroke,fill}%
\end{pgfscope}%
\begin{pgfscope}%
\pgfpathrectangle{\pgfqpoint{0.100000in}{0.212622in}}{\pgfqpoint{3.696000in}{3.696000in}}%
\pgfusepath{clip}%
\pgfsetbuttcap%
\pgfsetroundjoin%
\definecolor{currentfill}{rgb}{0.121569,0.466667,0.705882}%
\pgfsetfillcolor{currentfill}%
\pgfsetfillopacity{0.300458}%
\pgfsetlinewidth{1.003750pt}%
\definecolor{currentstroke}{rgb}{0.121569,0.466667,0.705882}%
\pgfsetstrokecolor{currentstroke}%
\pgfsetstrokeopacity{0.300458}%
\pgfsetdash{}{0pt}%
\pgfpathmoveto{\pgfqpoint{1.137841in}{1.617366in}}%
\pgfpathcurveto{\pgfqpoint{1.146077in}{1.617366in}}{\pgfqpoint{1.153977in}{1.620638in}}{\pgfqpoint{1.159801in}{1.626462in}}%
\pgfpathcurveto{\pgfqpoint{1.165625in}{1.632286in}}{\pgfqpoint{1.168897in}{1.640186in}}{\pgfqpoint{1.168897in}{1.648422in}}%
\pgfpathcurveto{\pgfqpoint{1.168897in}{1.656658in}}{\pgfqpoint{1.165625in}{1.664558in}}{\pgfqpoint{1.159801in}{1.670382in}}%
\pgfpathcurveto{\pgfqpoint{1.153977in}{1.676206in}}{\pgfqpoint{1.146077in}{1.679479in}}{\pgfqpoint{1.137841in}{1.679479in}}%
\pgfpathcurveto{\pgfqpoint{1.129604in}{1.679479in}}{\pgfqpoint{1.121704in}{1.676206in}}{\pgfqpoint{1.115880in}{1.670382in}}%
\pgfpathcurveto{\pgfqpoint{1.110056in}{1.664558in}}{\pgfqpoint{1.106784in}{1.656658in}}{\pgfqpoint{1.106784in}{1.648422in}}%
\pgfpathcurveto{\pgfqpoint{1.106784in}{1.640186in}}{\pgfqpoint{1.110056in}{1.632286in}}{\pgfqpoint{1.115880in}{1.626462in}}%
\pgfpathcurveto{\pgfqpoint{1.121704in}{1.620638in}}{\pgfqpoint{1.129604in}{1.617366in}}{\pgfqpoint{1.137841in}{1.617366in}}%
\pgfpathclose%
\pgfusepath{stroke,fill}%
\end{pgfscope}%
\begin{pgfscope}%
\pgfpathrectangle{\pgfqpoint{0.100000in}{0.212622in}}{\pgfqpoint{3.696000in}{3.696000in}}%
\pgfusepath{clip}%
\pgfsetbuttcap%
\pgfsetroundjoin%
\definecolor{currentfill}{rgb}{0.121569,0.466667,0.705882}%
\pgfsetfillcolor{currentfill}%
\pgfsetfillopacity{0.300458}%
\pgfsetlinewidth{1.003750pt}%
\definecolor{currentstroke}{rgb}{0.121569,0.466667,0.705882}%
\pgfsetstrokecolor{currentstroke}%
\pgfsetstrokeopacity{0.300458}%
\pgfsetdash{}{0pt}%
\pgfpathmoveto{\pgfqpoint{1.137841in}{1.617366in}}%
\pgfpathcurveto{\pgfqpoint{1.146077in}{1.617366in}}{\pgfqpoint{1.153977in}{1.620638in}}{\pgfqpoint{1.159801in}{1.626462in}}%
\pgfpathcurveto{\pgfqpoint{1.165625in}{1.632286in}}{\pgfqpoint{1.168897in}{1.640186in}}{\pgfqpoint{1.168897in}{1.648422in}}%
\pgfpathcurveto{\pgfqpoint{1.168897in}{1.656658in}}{\pgfqpoint{1.165625in}{1.664558in}}{\pgfqpoint{1.159801in}{1.670382in}}%
\pgfpathcurveto{\pgfqpoint{1.153977in}{1.676206in}}{\pgfqpoint{1.146077in}{1.679479in}}{\pgfqpoint{1.137841in}{1.679479in}}%
\pgfpathcurveto{\pgfqpoint{1.129604in}{1.679479in}}{\pgfqpoint{1.121704in}{1.676206in}}{\pgfqpoint{1.115880in}{1.670382in}}%
\pgfpathcurveto{\pgfqpoint{1.110056in}{1.664558in}}{\pgfqpoint{1.106784in}{1.656658in}}{\pgfqpoint{1.106784in}{1.648422in}}%
\pgfpathcurveto{\pgfqpoint{1.106784in}{1.640186in}}{\pgfqpoint{1.110056in}{1.632286in}}{\pgfqpoint{1.115880in}{1.626462in}}%
\pgfpathcurveto{\pgfqpoint{1.121704in}{1.620638in}}{\pgfqpoint{1.129604in}{1.617366in}}{\pgfqpoint{1.137841in}{1.617366in}}%
\pgfpathclose%
\pgfusepath{stroke,fill}%
\end{pgfscope}%
\begin{pgfscope}%
\pgfpathrectangle{\pgfqpoint{0.100000in}{0.212622in}}{\pgfqpoint{3.696000in}{3.696000in}}%
\pgfusepath{clip}%
\pgfsetbuttcap%
\pgfsetroundjoin%
\definecolor{currentfill}{rgb}{0.121569,0.466667,0.705882}%
\pgfsetfillcolor{currentfill}%
\pgfsetfillopacity{0.300458}%
\pgfsetlinewidth{1.003750pt}%
\definecolor{currentstroke}{rgb}{0.121569,0.466667,0.705882}%
\pgfsetstrokecolor{currentstroke}%
\pgfsetstrokeopacity{0.300458}%
\pgfsetdash{}{0pt}%
\pgfpathmoveto{\pgfqpoint{1.137841in}{1.617366in}}%
\pgfpathcurveto{\pgfqpoint{1.146077in}{1.617366in}}{\pgfqpoint{1.153977in}{1.620638in}}{\pgfqpoint{1.159801in}{1.626462in}}%
\pgfpathcurveto{\pgfqpoint{1.165625in}{1.632286in}}{\pgfqpoint{1.168897in}{1.640186in}}{\pgfqpoint{1.168897in}{1.648422in}}%
\pgfpathcurveto{\pgfqpoint{1.168897in}{1.656658in}}{\pgfqpoint{1.165625in}{1.664558in}}{\pgfqpoint{1.159801in}{1.670382in}}%
\pgfpathcurveto{\pgfqpoint{1.153977in}{1.676206in}}{\pgfqpoint{1.146077in}{1.679479in}}{\pgfqpoint{1.137841in}{1.679479in}}%
\pgfpathcurveto{\pgfqpoint{1.129604in}{1.679479in}}{\pgfqpoint{1.121704in}{1.676206in}}{\pgfqpoint{1.115880in}{1.670382in}}%
\pgfpathcurveto{\pgfqpoint{1.110056in}{1.664558in}}{\pgfqpoint{1.106784in}{1.656658in}}{\pgfqpoint{1.106784in}{1.648422in}}%
\pgfpathcurveto{\pgfqpoint{1.106784in}{1.640186in}}{\pgfqpoint{1.110056in}{1.632286in}}{\pgfqpoint{1.115880in}{1.626462in}}%
\pgfpathcurveto{\pgfqpoint{1.121704in}{1.620638in}}{\pgfqpoint{1.129604in}{1.617366in}}{\pgfqpoint{1.137841in}{1.617366in}}%
\pgfpathclose%
\pgfusepath{stroke,fill}%
\end{pgfscope}%
\begin{pgfscope}%
\pgfpathrectangle{\pgfqpoint{0.100000in}{0.212622in}}{\pgfqpoint{3.696000in}{3.696000in}}%
\pgfusepath{clip}%
\pgfsetbuttcap%
\pgfsetroundjoin%
\definecolor{currentfill}{rgb}{0.121569,0.466667,0.705882}%
\pgfsetfillcolor{currentfill}%
\pgfsetfillopacity{0.300458}%
\pgfsetlinewidth{1.003750pt}%
\definecolor{currentstroke}{rgb}{0.121569,0.466667,0.705882}%
\pgfsetstrokecolor{currentstroke}%
\pgfsetstrokeopacity{0.300458}%
\pgfsetdash{}{0pt}%
\pgfpathmoveto{\pgfqpoint{1.137841in}{1.617366in}}%
\pgfpathcurveto{\pgfqpoint{1.146077in}{1.617366in}}{\pgfqpoint{1.153977in}{1.620638in}}{\pgfqpoint{1.159801in}{1.626462in}}%
\pgfpathcurveto{\pgfqpoint{1.165625in}{1.632286in}}{\pgfqpoint{1.168897in}{1.640186in}}{\pgfqpoint{1.168897in}{1.648422in}}%
\pgfpathcurveto{\pgfqpoint{1.168897in}{1.656658in}}{\pgfqpoint{1.165625in}{1.664558in}}{\pgfqpoint{1.159801in}{1.670382in}}%
\pgfpathcurveto{\pgfqpoint{1.153977in}{1.676206in}}{\pgfqpoint{1.146077in}{1.679479in}}{\pgfqpoint{1.137841in}{1.679479in}}%
\pgfpathcurveto{\pgfqpoint{1.129604in}{1.679479in}}{\pgfqpoint{1.121704in}{1.676206in}}{\pgfqpoint{1.115880in}{1.670382in}}%
\pgfpathcurveto{\pgfqpoint{1.110056in}{1.664558in}}{\pgfqpoint{1.106784in}{1.656658in}}{\pgfqpoint{1.106784in}{1.648422in}}%
\pgfpathcurveto{\pgfqpoint{1.106784in}{1.640186in}}{\pgfqpoint{1.110056in}{1.632286in}}{\pgfqpoint{1.115880in}{1.626462in}}%
\pgfpathcurveto{\pgfqpoint{1.121704in}{1.620638in}}{\pgfqpoint{1.129604in}{1.617366in}}{\pgfqpoint{1.137841in}{1.617366in}}%
\pgfpathclose%
\pgfusepath{stroke,fill}%
\end{pgfscope}%
\begin{pgfscope}%
\pgfpathrectangle{\pgfqpoint{0.100000in}{0.212622in}}{\pgfqpoint{3.696000in}{3.696000in}}%
\pgfusepath{clip}%
\pgfsetbuttcap%
\pgfsetroundjoin%
\definecolor{currentfill}{rgb}{0.121569,0.466667,0.705882}%
\pgfsetfillcolor{currentfill}%
\pgfsetfillopacity{0.300458}%
\pgfsetlinewidth{1.003750pt}%
\definecolor{currentstroke}{rgb}{0.121569,0.466667,0.705882}%
\pgfsetstrokecolor{currentstroke}%
\pgfsetstrokeopacity{0.300458}%
\pgfsetdash{}{0pt}%
\pgfpathmoveto{\pgfqpoint{1.137841in}{1.617366in}}%
\pgfpathcurveto{\pgfqpoint{1.146077in}{1.617366in}}{\pgfqpoint{1.153977in}{1.620638in}}{\pgfqpoint{1.159801in}{1.626462in}}%
\pgfpathcurveto{\pgfqpoint{1.165625in}{1.632286in}}{\pgfqpoint{1.168897in}{1.640186in}}{\pgfqpoint{1.168897in}{1.648422in}}%
\pgfpathcurveto{\pgfqpoint{1.168897in}{1.656658in}}{\pgfqpoint{1.165625in}{1.664558in}}{\pgfqpoint{1.159801in}{1.670382in}}%
\pgfpathcurveto{\pgfqpoint{1.153977in}{1.676206in}}{\pgfqpoint{1.146077in}{1.679479in}}{\pgfqpoint{1.137841in}{1.679479in}}%
\pgfpathcurveto{\pgfqpoint{1.129604in}{1.679479in}}{\pgfqpoint{1.121704in}{1.676206in}}{\pgfqpoint{1.115880in}{1.670382in}}%
\pgfpathcurveto{\pgfqpoint{1.110056in}{1.664558in}}{\pgfqpoint{1.106784in}{1.656658in}}{\pgfqpoint{1.106784in}{1.648422in}}%
\pgfpathcurveto{\pgfqpoint{1.106784in}{1.640186in}}{\pgfqpoint{1.110056in}{1.632286in}}{\pgfqpoint{1.115880in}{1.626462in}}%
\pgfpathcurveto{\pgfqpoint{1.121704in}{1.620638in}}{\pgfqpoint{1.129604in}{1.617366in}}{\pgfqpoint{1.137841in}{1.617366in}}%
\pgfpathclose%
\pgfusepath{stroke,fill}%
\end{pgfscope}%
\begin{pgfscope}%
\pgfpathrectangle{\pgfqpoint{0.100000in}{0.212622in}}{\pgfqpoint{3.696000in}{3.696000in}}%
\pgfusepath{clip}%
\pgfsetbuttcap%
\pgfsetroundjoin%
\definecolor{currentfill}{rgb}{0.121569,0.466667,0.705882}%
\pgfsetfillcolor{currentfill}%
\pgfsetfillopacity{0.300458}%
\pgfsetlinewidth{1.003750pt}%
\definecolor{currentstroke}{rgb}{0.121569,0.466667,0.705882}%
\pgfsetstrokecolor{currentstroke}%
\pgfsetstrokeopacity{0.300458}%
\pgfsetdash{}{0pt}%
\pgfpathmoveto{\pgfqpoint{1.137841in}{1.617366in}}%
\pgfpathcurveto{\pgfqpoint{1.146077in}{1.617366in}}{\pgfqpoint{1.153977in}{1.620638in}}{\pgfqpoint{1.159801in}{1.626462in}}%
\pgfpathcurveto{\pgfqpoint{1.165625in}{1.632286in}}{\pgfqpoint{1.168897in}{1.640186in}}{\pgfqpoint{1.168897in}{1.648422in}}%
\pgfpathcurveto{\pgfqpoint{1.168897in}{1.656658in}}{\pgfqpoint{1.165625in}{1.664558in}}{\pgfqpoint{1.159801in}{1.670382in}}%
\pgfpathcurveto{\pgfqpoint{1.153977in}{1.676206in}}{\pgfqpoint{1.146077in}{1.679479in}}{\pgfqpoint{1.137841in}{1.679479in}}%
\pgfpathcurveto{\pgfqpoint{1.129604in}{1.679479in}}{\pgfqpoint{1.121704in}{1.676206in}}{\pgfqpoint{1.115880in}{1.670382in}}%
\pgfpathcurveto{\pgfqpoint{1.110056in}{1.664558in}}{\pgfqpoint{1.106784in}{1.656658in}}{\pgfqpoint{1.106784in}{1.648422in}}%
\pgfpathcurveto{\pgfqpoint{1.106784in}{1.640186in}}{\pgfqpoint{1.110056in}{1.632286in}}{\pgfqpoint{1.115880in}{1.626462in}}%
\pgfpathcurveto{\pgfqpoint{1.121704in}{1.620638in}}{\pgfqpoint{1.129604in}{1.617366in}}{\pgfqpoint{1.137841in}{1.617366in}}%
\pgfpathclose%
\pgfusepath{stroke,fill}%
\end{pgfscope}%
\begin{pgfscope}%
\pgfpathrectangle{\pgfqpoint{0.100000in}{0.212622in}}{\pgfqpoint{3.696000in}{3.696000in}}%
\pgfusepath{clip}%
\pgfsetbuttcap%
\pgfsetroundjoin%
\definecolor{currentfill}{rgb}{0.121569,0.466667,0.705882}%
\pgfsetfillcolor{currentfill}%
\pgfsetfillopacity{0.300458}%
\pgfsetlinewidth{1.003750pt}%
\definecolor{currentstroke}{rgb}{0.121569,0.466667,0.705882}%
\pgfsetstrokecolor{currentstroke}%
\pgfsetstrokeopacity{0.300458}%
\pgfsetdash{}{0pt}%
\pgfpathmoveto{\pgfqpoint{1.137841in}{1.617366in}}%
\pgfpathcurveto{\pgfqpoint{1.146077in}{1.617366in}}{\pgfqpoint{1.153977in}{1.620638in}}{\pgfqpoint{1.159801in}{1.626462in}}%
\pgfpathcurveto{\pgfqpoint{1.165625in}{1.632286in}}{\pgfqpoint{1.168897in}{1.640186in}}{\pgfqpoint{1.168897in}{1.648422in}}%
\pgfpathcurveto{\pgfqpoint{1.168897in}{1.656658in}}{\pgfqpoint{1.165625in}{1.664558in}}{\pgfqpoint{1.159801in}{1.670382in}}%
\pgfpathcurveto{\pgfqpoint{1.153977in}{1.676206in}}{\pgfqpoint{1.146077in}{1.679479in}}{\pgfqpoint{1.137841in}{1.679479in}}%
\pgfpathcurveto{\pgfqpoint{1.129604in}{1.679479in}}{\pgfqpoint{1.121704in}{1.676206in}}{\pgfqpoint{1.115880in}{1.670382in}}%
\pgfpathcurveto{\pgfqpoint{1.110056in}{1.664558in}}{\pgfqpoint{1.106784in}{1.656658in}}{\pgfqpoint{1.106784in}{1.648422in}}%
\pgfpathcurveto{\pgfqpoint{1.106784in}{1.640186in}}{\pgfqpoint{1.110056in}{1.632286in}}{\pgfqpoint{1.115880in}{1.626462in}}%
\pgfpathcurveto{\pgfqpoint{1.121704in}{1.620638in}}{\pgfqpoint{1.129604in}{1.617366in}}{\pgfqpoint{1.137841in}{1.617366in}}%
\pgfpathclose%
\pgfusepath{stroke,fill}%
\end{pgfscope}%
\begin{pgfscope}%
\pgfpathrectangle{\pgfqpoint{0.100000in}{0.212622in}}{\pgfqpoint{3.696000in}{3.696000in}}%
\pgfusepath{clip}%
\pgfsetbuttcap%
\pgfsetroundjoin%
\definecolor{currentfill}{rgb}{0.121569,0.466667,0.705882}%
\pgfsetfillcolor{currentfill}%
\pgfsetfillopacity{0.300458}%
\pgfsetlinewidth{1.003750pt}%
\definecolor{currentstroke}{rgb}{0.121569,0.466667,0.705882}%
\pgfsetstrokecolor{currentstroke}%
\pgfsetstrokeopacity{0.300458}%
\pgfsetdash{}{0pt}%
\pgfpathmoveto{\pgfqpoint{1.137841in}{1.617366in}}%
\pgfpathcurveto{\pgfqpoint{1.146077in}{1.617366in}}{\pgfqpoint{1.153977in}{1.620638in}}{\pgfqpoint{1.159801in}{1.626462in}}%
\pgfpathcurveto{\pgfqpoint{1.165625in}{1.632286in}}{\pgfqpoint{1.168897in}{1.640186in}}{\pgfqpoint{1.168897in}{1.648422in}}%
\pgfpathcurveto{\pgfqpoint{1.168897in}{1.656658in}}{\pgfqpoint{1.165625in}{1.664558in}}{\pgfqpoint{1.159801in}{1.670382in}}%
\pgfpathcurveto{\pgfqpoint{1.153977in}{1.676206in}}{\pgfqpoint{1.146077in}{1.679479in}}{\pgfqpoint{1.137841in}{1.679479in}}%
\pgfpathcurveto{\pgfqpoint{1.129604in}{1.679479in}}{\pgfqpoint{1.121704in}{1.676206in}}{\pgfqpoint{1.115880in}{1.670382in}}%
\pgfpathcurveto{\pgfqpoint{1.110056in}{1.664558in}}{\pgfqpoint{1.106784in}{1.656658in}}{\pgfqpoint{1.106784in}{1.648422in}}%
\pgfpathcurveto{\pgfqpoint{1.106784in}{1.640186in}}{\pgfqpoint{1.110056in}{1.632286in}}{\pgfqpoint{1.115880in}{1.626462in}}%
\pgfpathcurveto{\pgfqpoint{1.121704in}{1.620638in}}{\pgfqpoint{1.129604in}{1.617366in}}{\pgfqpoint{1.137841in}{1.617366in}}%
\pgfpathclose%
\pgfusepath{stroke,fill}%
\end{pgfscope}%
\begin{pgfscope}%
\pgfpathrectangle{\pgfqpoint{0.100000in}{0.212622in}}{\pgfqpoint{3.696000in}{3.696000in}}%
\pgfusepath{clip}%
\pgfsetbuttcap%
\pgfsetroundjoin%
\definecolor{currentfill}{rgb}{0.121569,0.466667,0.705882}%
\pgfsetfillcolor{currentfill}%
\pgfsetfillopacity{0.300458}%
\pgfsetlinewidth{1.003750pt}%
\definecolor{currentstroke}{rgb}{0.121569,0.466667,0.705882}%
\pgfsetstrokecolor{currentstroke}%
\pgfsetstrokeopacity{0.300458}%
\pgfsetdash{}{0pt}%
\pgfpathmoveto{\pgfqpoint{1.137841in}{1.617366in}}%
\pgfpathcurveto{\pgfqpoint{1.146077in}{1.617366in}}{\pgfqpoint{1.153977in}{1.620638in}}{\pgfqpoint{1.159801in}{1.626462in}}%
\pgfpathcurveto{\pgfqpoint{1.165625in}{1.632286in}}{\pgfqpoint{1.168897in}{1.640186in}}{\pgfqpoint{1.168897in}{1.648422in}}%
\pgfpathcurveto{\pgfqpoint{1.168897in}{1.656658in}}{\pgfqpoint{1.165625in}{1.664558in}}{\pgfqpoint{1.159801in}{1.670382in}}%
\pgfpathcurveto{\pgfqpoint{1.153977in}{1.676206in}}{\pgfqpoint{1.146077in}{1.679479in}}{\pgfqpoint{1.137841in}{1.679479in}}%
\pgfpathcurveto{\pgfqpoint{1.129604in}{1.679479in}}{\pgfqpoint{1.121704in}{1.676206in}}{\pgfqpoint{1.115880in}{1.670382in}}%
\pgfpathcurveto{\pgfqpoint{1.110056in}{1.664558in}}{\pgfqpoint{1.106784in}{1.656658in}}{\pgfqpoint{1.106784in}{1.648422in}}%
\pgfpathcurveto{\pgfqpoint{1.106784in}{1.640186in}}{\pgfqpoint{1.110056in}{1.632286in}}{\pgfqpoint{1.115880in}{1.626462in}}%
\pgfpathcurveto{\pgfqpoint{1.121704in}{1.620638in}}{\pgfqpoint{1.129604in}{1.617366in}}{\pgfqpoint{1.137841in}{1.617366in}}%
\pgfpathclose%
\pgfusepath{stroke,fill}%
\end{pgfscope}%
\begin{pgfscope}%
\pgfpathrectangle{\pgfqpoint{0.100000in}{0.212622in}}{\pgfqpoint{3.696000in}{3.696000in}}%
\pgfusepath{clip}%
\pgfsetbuttcap%
\pgfsetroundjoin%
\definecolor{currentfill}{rgb}{0.121569,0.466667,0.705882}%
\pgfsetfillcolor{currentfill}%
\pgfsetfillopacity{0.300458}%
\pgfsetlinewidth{1.003750pt}%
\definecolor{currentstroke}{rgb}{0.121569,0.466667,0.705882}%
\pgfsetstrokecolor{currentstroke}%
\pgfsetstrokeopacity{0.300458}%
\pgfsetdash{}{0pt}%
\pgfpathmoveto{\pgfqpoint{1.137841in}{1.617366in}}%
\pgfpathcurveto{\pgfqpoint{1.146077in}{1.617366in}}{\pgfqpoint{1.153977in}{1.620638in}}{\pgfqpoint{1.159801in}{1.626462in}}%
\pgfpathcurveto{\pgfqpoint{1.165625in}{1.632286in}}{\pgfqpoint{1.168897in}{1.640186in}}{\pgfqpoint{1.168897in}{1.648422in}}%
\pgfpathcurveto{\pgfqpoint{1.168897in}{1.656658in}}{\pgfqpoint{1.165625in}{1.664558in}}{\pgfqpoint{1.159801in}{1.670382in}}%
\pgfpathcurveto{\pgfqpoint{1.153977in}{1.676206in}}{\pgfqpoint{1.146077in}{1.679479in}}{\pgfqpoint{1.137841in}{1.679479in}}%
\pgfpathcurveto{\pgfqpoint{1.129604in}{1.679479in}}{\pgfqpoint{1.121704in}{1.676206in}}{\pgfqpoint{1.115880in}{1.670382in}}%
\pgfpathcurveto{\pgfqpoint{1.110056in}{1.664558in}}{\pgfqpoint{1.106784in}{1.656658in}}{\pgfqpoint{1.106784in}{1.648422in}}%
\pgfpathcurveto{\pgfqpoint{1.106784in}{1.640186in}}{\pgfqpoint{1.110056in}{1.632286in}}{\pgfqpoint{1.115880in}{1.626462in}}%
\pgfpathcurveto{\pgfqpoint{1.121704in}{1.620638in}}{\pgfqpoint{1.129604in}{1.617366in}}{\pgfqpoint{1.137841in}{1.617366in}}%
\pgfpathclose%
\pgfusepath{stroke,fill}%
\end{pgfscope}%
\begin{pgfscope}%
\pgfpathrectangle{\pgfqpoint{0.100000in}{0.212622in}}{\pgfqpoint{3.696000in}{3.696000in}}%
\pgfusepath{clip}%
\pgfsetbuttcap%
\pgfsetroundjoin%
\definecolor{currentfill}{rgb}{0.121569,0.466667,0.705882}%
\pgfsetfillcolor{currentfill}%
\pgfsetfillopacity{0.300458}%
\pgfsetlinewidth{1.003750pt}%
\definecolor{currentstroke}{rgb}{0.121569,0.466667,0.705882}%
\pgfsetstrokecolor{currentstroke}%
\pgfsetstrokeopacity{0.300458}%
\pgfsetdash{}{0pt}%
\pgfpathmoveto{\pgfqpoint{1.137841in}{1.617366in}}%
\pgfpathcurveto{\pgfqpoint{1.146077in}{1.617366in}}{\pgfqpoint{1.153977in}{1.620638in}}{\pgfqpoint{1.159801in}{1.626462in}}%
\pgfpathcurveto{\pgfqpoint{1.165625in}{1.632286in}}{\pgfqpoint{1.168897in}{1.640186in}}{\pgfqpoint{1.168897in}{1.648422in}}%
\pgfpathcurveto{\pgfqpoint{1.168897in}{1.656658in}}{\pgfqpoint{1.165625in}{1.664558in}}{\pgfqpoint{1.159801in}{1.670382in}}%
\pgfpathcurveto{\pgfqpoint{1.153977in}{1.676206in}}{\pgfqpoint{1.146077in}{1.679479in}}{\pgfqpoint{1.137841in}{1.679479in}}%
\pgfpathcurveto{\pgfqpoint{1.129604in}{1.679479in}}{\pgfqpoint{1.121704in}{1.676206in}}{\pgfqpoint{1.115880in}{1.670382in}}%
\pgfpathcurveto{\pgfqpoint{1.110056in}{1.664558in}}{\pgfqpoint{1.106784in}{1.656658in}}{\pgfqpoint{1.106784in}{1.648422in}}%
\pgfpathcurveto{\pgfqpoint{1.106784in}{1.640186in}}{\pgfqpoint{1.110056in}{1.632286in}}{\pgfqpoint{1.115880in}{1.626462in}}%
\pgfpathcurveto{\pgfqpoint{1.121704in}{1.620638in}}{\pgfqpoint{1.129604in}{1.617366in}}{\pgfqpoint{1.137841in}{1.617366in}}%
\pgfpathclose%
\pgfusepath{stroke,fill}%
\end{pgfscope}%
\begin{pgfscope}%
\pgfpathrectangle{\pgfqpoint{0.100000in}{0.212622in}}{\pgfqpoint{3.696000in}{3.696000in}}%
\pgfusepath{clip}%
\pgfsetbuttcap%
\pgfsetroundjoin%
\definecolor{currentfill}{rgb}{0.121569,0.466667,0.705882}%
\pgfsetfillcolor{currentfill}%
\pgfsetfillopacity{0.300458}%
\pgfsetlinewidth{1.003750pt}%
\definecolor{currentstroke}{rgb}{0.121569,0.466667,0.705882}%
\pgfsetstrokecolor{currentstroke}%
\pgfsetstrokeopacity{0.300458}%
\pgfsetdash{}{0pt}%
\pgfpathmoveto{\pgfqpoint{1.137841in}{1.617366in}}%
\pgfpathcurveto{\pgfqpoint{1.146077in}{1.617366in}}{\pgfqpoint{1.153977in}{1.620638in}}{\pgfqpoint{1.159801in}{1.626462in}}%
\pgfpathcurveto{\pgfqpoint{1.165625in}{1.632286in}}{\pgfqpoint{1.168897in}{1.640186in}}{\pgfqpoint{1.168897in}{1.648422in}}%
\pgfpathcurveto{\pgfqpoint{1.168897in}{1.656658in}}{\pgfqpoint{1.165625in}{1.664558in}}{\pgfqpoint{1.159801in}{1.670382in}}%
\pgfpathcurveto{\pgfqpoint{1.153977in}{1.676206in}}{\pgfqpoint{1.146077in}{1.679479in}}{\pgfqpoint{1.137841in}{1.679479in}}%
\pgfpathcurveto{\pgfqpoint{1.129604in}{1.679479in}}{\pgfqpoint{1.121704in}{1.676206in}}{\pgfqpoint{1.115880in}{1.670382in}}%
\pgfpathcurveto{\pgfqpoint{1.110056in}{1.664558in}}{\pgfqpoint{1.106784in}{1.656658in}}{\pgfqpoint{1.106784in}{1.648422in}}%
\pgfpathcurveto{\pgfqpoint{1.106784in}{1.640186in}}{\pgfqpoint{1.110056in}{1.632286in}}{\pgfqpoint{1.115880in}{1.626462in}}%
\pgfpathcurveto{\pgfqpoint{1.121704in}{1.620638in}}{\pgfqpoint{1.129604in}{1.617366in}}{\pgfqpoint{1.137841in}{1.617366in}}%
\pgfpathclose%
\pgfusepath{stroke,fill}%
\end{pgfscope}%
\begin{pgfscope}%
\pgfpathrectangle{\pgfqpoint{0.100000in}{0.212622in}}{\pgfqpoint{3.696000in}{3.696000in}}%
\pgfusepath{clip}%
\pgfsetbuttcap%
\pgfsetroundjoin%
\definecolor{currentfill}{rgb}{0.121569,0.466667,0.705882}%
\pgfsetfillcolor{currentfill}%
\pgfsetfillopacity{0.300458}%
\pgfsetlinewidth{1.003750pt}%
\definecolor{currentstroke}{rgb}{0.121569,0.466667,0.705882}%
\pgfsetstrokecolor{currentstroke}%
\pgfsetstrokeopacity{0.300458}%
\pgfsetdash{}{0pt}%
\pgfpathmoveto{\pgfqpoint{1.137841in}{1.617366in}}%
\pgfpathcurveto{\pgfqpoint{1.146077in}{1.617366in}}{\pgfqpoint{1.153977in}{1.620638in}}{\pgfqpoint{1.159801in}{1.626462in}}%
\pgfpathcurveto{\pgfqpoint{1.165625in}{1.632286in}}{\pgfqpoint{1.168897in}{1.640186in}}{\pgfqpoint{1.168897in}{1.648422in}}%
\pgfpathcurveto{\pgfqpoint{1.168897in}{1.656658in}}{\pgfqpoint{1.165625in}{1.664558in}}{\pgfqpoint{1.159801in}{1.670382in}}%
\pgfpathcurveto{\pgfqpoint{1.153977in}{1.676206in}}{\pgfqpoint{1.146077in}{1.679479in}}{\pgfqpoint{1.137841in}{1.679479in}}%
\pgfpathcurveto{\pgfqpoint{1.129604in}{1.679479in}}{\pgfqpoint{1.121704in}{1.676206in}}{\pgfqpoint{1.115880in}{1.670382in}}%
\pgfpathcurveto{\pgfqpoint{1.110056in}{1.664558in}}{\pgfqpoint{1.106784in}{1.656658in}}{\pgfqpoint{1.106784in}{1.648422in}}%
\pgfpathcurveto{\pgfqpoint{1.106784in}{1.640186in}}{\pgfqpoint{1.110056in}{1.632286in}}{\pgfqpoint{1.115880in}{1.626462in}}%
\pgfpathcurveto{\pgfqpoint{1.121704in}{1.620638in}}{\pgfqpoint{1.129604in}{1.617366in}}{\pgfqpoint{1.137841in}{1.617366in}}%
\pgfpathclose%
\pgfusepath{stroke,fill}%
\end{pgfscope}%
\begin{pgfscope}%
\pgfpathrectangle{\pgfqpoint{0.100000in}{0.212622in}}{\pgfqpoint{3.696000in}{3.696000in}}%
\pgfusepath{clip}%
\pgfsetbuttcap%
\pgfsetroundjoin%
\definecolor{currentfill}{rgb}{0.121569,0.466667,0.705882}%
\pgfsetfillcolor{currentfill}%
\pgfsetfillopacity{0.300458}%
\pgfsetlinewidth{1.003750pt}%
\definecolor{currentstroke}{rgb}{0.121569,0.466667,0.705882}%
\pgfsetstrokecolor{currentstroke}%
\pgfsetstrokeopacity{0.300458}%
\pgfsetdash{}{0pt}%
\pgfpathmoveto{\pgfqpoint{1.137841in}{1.617366in}}%
\pgfpathcurveto{\pgfqpoint{1.146077in}{1.617366in}}{\pgfqpoint{1.153977in}{1.620638in}}{\pgfqpoint{1.159801in}{1.626462in}}%
\pgfpathcurveto{\pgfqpoint{1.165625in}{1.632286in}}{\pgfqpoint{1.168897in}{1.640186in}}{\pgfqpoint{1.168897in}{1.648422in}}%
\pgfpathcurveto{\pgfqpoint{1.168897in}{1.656658in}}{\pgfqpoint{1.165625in}{1.664558in}}{\pgfqpoint{1.159801in}{1.670382in}}%
\pgfpathcurveto{\pgfqpoint{1.153977in}{1.676206in}}{\pgfqpoint{1.146077in}{1.679479in}}{\pgfqpoint{1.137841in}{1.679479in}}%
\pgfpathcurveto{\pgfqpoint{1.129604in}{1.679479in}}{\pgfqpoint{1.121704in}{1.676206in}}{\pgfqpoint{1.115880in}{1.670382in}}%
\pgfpathcurveto{\pgfqpoint{1.110056in}{1.664558in}}{\pgfqpoint{1.106784in}{1.656658in}}{\pgfqpoint{1.106784in}{1.648422in}}%
\pgfpathcurveto{\pgfqpoint{1.106784in}{1.640186in}}{\pgfqpoint{1.110056in}{1.632286in}}{\pgfqpoint{1.115880in}{1.626462in}}%
\pgfpathcurveto{\pgfqpoint{1.121704in}{1.620638in}}{\pgfqpoint{1.129604in}{1.617366in}}{\pgfqpoint{1.137841in}{1.617366in}}%
\pgfpathclose%
\pgfusepath{stroke,fill}%
\end{pgfscope}%
\begin{pgfscope}%
\pgfpathrectangle{\pgfqpoint{0.100000in}{0.212622in}}{\pgfqpoint{3.696000in}{3.696000in}}%
\pgfusepath{clip}%
\pgfsetbuttcap%
\pgfsetroundjoin%
\definecolor{currentfill}{rgb}{0.121569,0.466667,0.705882}%
\pgfsetfillcolor{currentfill}%
\pgfsetfillopacity{0.300458}%
\pgfsetlinewidth{1.003750pt}%
\definecolor{currentstroke}{rgb}{0.121569,0.466667,0.705882}%
\pgfsetstrokecolor{currentstroke}%
\pgfsetstrokeopacity{0.300458}%
\pgfsetdash{}{0pt}%
\pgfpathmoveto{\pgfqpoint{1.137841in}{1.617366in}}%
\pgfpathcurveto{\pgfqpoint{1.146077in}{1.617366in}}{\pgfqpoint{1.153977in}{1.620638in}}{\pgfqpoint{1.159801in}{1.626462in}}%
\pgfpathcurveto{\pgfqpoint{1.165625in}{1.632286in}}{\pgfqpoint{1.168897in}{1.640186in}}{\pgfqpoint{1.168897in}{1.648422in}}%
\pgfpathcurveto{\pgfqpoint{1.168897in}{1.656658in}}{\pgfqpoint{1.165625in}{1.664558in}}{\pgfqpoint{1.159801in}{1.670382in}}%
\pgfpathcurveto{\pgfqpoint{1.153977in}{1.676206in}}{\pgfqpoint{1.146077in}{1.679479in}}{\pgfqpoint{1.137841in}{1.679479in}}%
\pgfpathcurveto{\pgfqpoint{1.129604in}{1.679479in}}{\pgfqpoint{1.121704in}{1.676206in}}{\pgfqpoint{1.115880in}{1.670382in}}%
\pgfpathcurveto{\pgfqpoint{1.110056in}{1.664558in}}{\pgfqpoint{1.106784in}{1.656658in}}{\pgfqpoint{1.106784in}{1.648422in}}%
\pgfpathcurveto{\pgfqpoint{1.106784in}{1.640186in}}{\pgfqpoint{1.110056in}{1.632286in}}{\pgfqpoint{1.115880in}{1.626462in}}%
\pgfpathcurveto{\pgfqpoint{1.121704in}{1.620638in}}{\pgfqpoint{1.129604in}{1.617366in}}{\pgfqpoint{1.137841in}{1.617366in}}%
\pgfpathclose%
\pgfusepath{stroke,fill}%
\end{pgfscope}%
\begin{pgfscope}%
\pgfpathrectangle{\pgfqpoint{0.100000in}{0.212622in}}{\pgfqpoint{3.696000in}{3.696000in}}%
\pgfusepath{clip}%
\pgfsetbuttcap%
\pgfsetroundjoin%
\definecolor{currentfill}{rgb}{0.121569,0.466667,0.705882}%
\pgfsetfillcolor{currentfill}%
\pgfsetfillopacity{0.300458}%
\pgfsetlinewidth{1.003750pt}%
\definecolor{currentstroke}{rgb}{0.121569,0.466667,0.705882}%
\pgfsetstrokecolor{currentstroke}%
\pgfsetstrokeopacity{0.300458}%
\pgfsetdash{}{0pt}%
\pgfpathmoveto{\pgfqpoint{1.137841in}{1.617366in}}%
\pgfpathcurveto{\pgfqpoint{1.146077in}{1.617366in}}{\pgfqpoint{1.153977in}{1.620638in}}{\pgfqpoint{1.159801in}{1.626462in}}%
\pgfpathcurveto{\pgfqpoint{1.165625in}{1.632286in}}{\pgfqpoint{1.168897in}{1.640186in}}{\pgfqpoint{1.168897in}{1.648422in}}%
\pgfpathcurveto{\pgfqpoint{1.168897in}{1.656658in}}{\pgfqpoint{1.165625in}{1.664558in}}{\pgfqpoint{1.159801in}{1.670382in}}%
\pgfpathcurveto{\pgfqpoint{1.153977in}{1.676206in}}{\pgfqpoint{1.146077in}{1.679479in}}{\pgfqpoint{1.137841in}{1.679479in}}%
\pgfpathcurveto{\pgfqpoint{1.129604in}{1.679479in}}{\pgfqpoint{1.121704in}{1.676206in}}{\pgfqpoint{1.115880in}{1.670382in}}%
\pgfpathcurveto{\pgfqpoint{1.110056in}{1.664558in}}{\pgfqpoint{1.106784in}{1.656658in}}{\pgfqpoint{1.106784in}{1.648422in}}%
\pgfpathcurveto{\pgfqpoint{1.106784in}{1.640186in}}{\pgfqpoint{1.110056in}{1.632286in}}{\pgfqpoint{1.115880in}{1.626462in}}%
\pgfpathcurveto{\pgfqpoint{1.121704in}{1.620638in}}{\pgfqpoint{1.129604in}{1.617366in}}{\pgfqpoint{1.137841in}{1.617366in}}%
\pgfpathclose%
\pgfusepath{stroke,fill}%
\end{pgfscope}%
\begin{pgfscope}%
\pgfpathrectangle{\pgfqpoint{0.100000in}{0.212622in}}{\pgfqpoint{3.696000in}{3.696000in}}%
\pgfusepath{clip}%
\pgfsetbuttcap%
\pgfsetroundjoin%
\definecolor{currentfill}{rgb}{0.121569,0.466667,0.705882}%
\pgfsetfillcolor{currentfill}%
\pgfsetfillopacity{0.300458}%
\pgfsetlinewidth{1.003750pt}%
\definecolor{currentstroke}{rgb}{0.121569,0.466667,0.705882}%
\pgfsetstrokecolor{currentstroke}%
\pgfsetstrokeopacity{0.300458}%
\pgfsetdash{}{0pt}%
\pgfpathmoveto{\pgfqpoint{1.137841in}{1.617366in}}%
\pgfpathcurveto{\pgfqpoint{1.146077in}{1.617366in}}{\pgfqpoint{1.153977in}{1.620638in}}{\pgfqpoint{1.159801in}{1.626462in}}%
\pgfpathcurveto{\pgfqpoint{1.165625in}{1.632286in}}{\pgfqpoint{1.168897in}{1.640186in}}{\pgfqpoint{1.168897in}{1.648422in}}%
\pgfpathcurveto{\pgfqpoint{1.168897in}{1.656658in}}{\pgfqpoint{1.165625in}{1.664558in}}{\pgfqpoint{1.159801in}{1.670382in}}%
\pgfpathcurveto{\pgfqpoint{1.153977in}{1.676206in}}{\pgfqpoint{1.146077in}{1.679479in}}{\pgfqpoint{1.137841in}{1.679479in}}%
\pgfpathcurveto{\pgfqpoint{1.129604in}{1.679479in}}{\pgfqpoint{1.121704in}{1.676206in}}{\pgfqpoint{1.115880in}{1.670382in}}%
\pgfpathcurveto{\pgfqpoint{1.110056in}{1.664558in}}{\pgfqpoint{1.106784in}{1.656658in}}{\pgfqpoint{1.106784in}{1.648422in}}%
\pgfpathcurveto{\pgfqpoint{1.106784in}{1.640186in}}{\pgfqpoint{1.110056in}{1.632286in}}{\pgfqpoint{1.115880in}{1.626462in}}%
\pgfpathcurveto{\pgfqpoint{1.121704in}{1.620638in}}{\pgfqpoint{1.129604in}{1.617366in}}{\pgfqpoint{1.137841in}{1.617366in}}%
\pgfpathclose%
\pgfusepath{stroke,fill}%
\end{pgfscope}%
\begin{pgfscope}%
\pgfpathrectangle{\pgfqpoint{0.100000in}{0.212622in}}{\pgfqpoint{3.696000in}{3.696000in}}%
\pgfusepath{clip}%
\pgfsetbuttcap%
\pgfsetroundjoin%
\definecolor{currentfill}{rgb}{0.121569,0.466667,0.705882}%
\pgfsetfillcolor{currentfill}%
\pgfsetfillopacity{0.300458}%
\pgfsetlinewidth{1.003750pt}%
\definecolor{currentstroke}{rgb}{0.121569,0.466667,0.705882}%
\pgfsetstrokecolor{currentstroke}%
\pgfsetstrokeopacity{0.300458}%
\pgfsetdash{}{0pt}%
\pgfpathmoveto{\pgfqpoint{1.137841in}{1.617366in}}%
\pgfpathcurveto{\pgfqpoint{1.146077in}{1.617366in}}{\pgfqpoint{1.153977in}{1.620638in}}{\pgfqpoint{1.159801in}{1.626462in}}%
\pgfpathcurveto{\pgfqpoint{1.165625in}{1.632286in}}{\pgfqpoint{1.168897in}{1.640186in}}{\pgfqpoint{1.168897in}{1.648422in}}%
\pgfpathcurveto{\pgfqpoint{1.168897in}{1.656658in}}{\pgfqpoint{1.165625in}{1.664558in}}{\pgfqpoint{1.159801in}{1.670382in}}%
\pgfpathcurveto{\pgfqpoint{1.153977in}{1.676206in}}{\pgfqpoint{1.146077in}{1.679479in}}{\pgfqpoint{1.137841in}{1.679479in}}%
\pgfpathcurveto{\pgfqpoint{1.129604in}{1.679479in}}{\pgfqpoint{1.121704in}{1.676206in}}{\pgfqpoint{1.115880in}{1.670382in}}%
\pgfpathcurveto{\pgfqpoint{1.110056in}{1.664558in}}{\pgfqpoint{1.106784in}{1.656658in}}{\pgfqpoint{1.106784in}{1.648422in}}%
\pgfpathcurveto{\pgfqpoint{1.106784in}{1.640186in}}{\pgfqpoint{1.110056in}{1.632286in}}{\pgfqpoint{1.115880in}{1.626462in}}%
\pgfpathcurveto{\pgfqpoint{1.121704in}{1.620638in}}{\pgfqpoint{1.129604in}{1.617366in}}{\pgfqpoint{1.137841in}{1.617366in}}%
\pgfpathclose%
\pgfusepath{stroke,fill}%
\end{pgfscope}%
\begin{pgfscope}%
\pgfpathrectangle{\pgfqpoint{0.100000in}{0.212622in}}{\pgfqpoint{3.696000in}{3.696000in}}%
\pgfusepath{clip}%
\pgfsetbuttcap%
\pgfsetroundjoin%
\definecolor{currentfill}{rgb}{0.121569,0.466667,0.705882}%
\pgfsetfillcolor{currentfill}%
\pgfsetfillopacity{0.300458}%
\pgfsetlinewidth{1.003750pt}%
\definecolor{currentstroke}{rgb}{0.121569,0.466667,0.705882}%
\pgfsetstrokecolor{currentstroke}%
\pgfsetstrokeopacity{0.300458}%
\pgfsetdash{}{0pt}%
\pgfpathmoveto{\pgfqpoint{1.137841in}{1.617366in}}%
\pgfpathcurveto{\pgfqpoint{1.146077in}{1.617366in}}{\pgfqpoint{1.153977in}{1.620638in}}{\pgfqpoint{1.159801in}{1.626462in}}%
\pgfpathcurveto{\pgfqpoint{1.165625in}{1.632286in}}{\pgfqpoint{1.168897in}{1.640186in}}{\pgfqpoint{1.168897in}{1.648422in}}%
\pgfpathcurveto{\pgfqpoint{1.168897in}{1.656658in}}{\pgfqpoint{1.165625in}{1.664558in}}{\pgfqpoint{1.159801in}{1.670382in}}%
\pgfpathcurveto{\pgfqpoint{1.153977in}{1.676206in}}{\pgfqpoint{1.146077in}{1.679479in}}{\pgfqpoint{1.137841in}{1.679479in}}%
\pgfpathcurveto{\pgfqpoint{1.129604in}{1.679479in}}{\pgfqpoint{1.121704in}{1.676206in}}{\pgfqpoint{1.115880in}{1.670382in}}%
\pgfpathcurveto{\pgfqpoint{1.110056in}{1.664558in}}{\pgfqpoint{1.106784in}{1.656658in}}{\pgfqpoint{1.106784in}{1.648422in}}%
\pgfpathcurveto{\pgfqpoint{1.106784in}{1.640186in}}{\pgfqpoint{1.110056in}{1.632286in}}{\pgfqpoint{1.115880in}{1.626462in}}%
\pgfpathcurveto{\pgfqpoint{1.121704in}{1.620638in}}{\pgfqpoint{1.129604in}{1.617366in}}{\pgfqpoint{1.137841in}{1.617366in}}%
\pgfpathclose%
\pgfusepath{stroke,fill}%
\end{pgfscope}%
\begin{pgfscope}%
\pgfpathrectangle{\pgfqpoint{0.100000in}{0.212622in}}{\pgfqpoint{3.696000in}{3.696000in}}%
\pgfusepath{clip}%
\pgfsetbuttcap%
\pgfsetroundjoin%
\definecolor{currentfill}{rgb}{0.121569,0.466667,0.705882}%
\pgfsetfillcolor{currentfill}%
\pgfsetfillopacity{0.300458}%
\pgfsetlinewidth{1.003750pt}%
\definecolor{currentstroke}{rgb}{0.121569,0.466667,0.705882}%
\pgfsetstrokecolor{currentstroke}%
\pgfsetstrokeopacity{0.300458}%
\pgfsetdash{}{0pt}%
\pgfpathmoveto{\pgfqpoint{1.137841in}{1.617366in}}%
\pgfpathcurveto{\pgfqpoint{1.146077in}{1.617366in}}{\pgfqpoint{1.153977in}{1.620638in}}{\pgfqpoint{1.159801in}{1.626462in}}%
\pgfpathcurveto{\pgfqpoint{1.165625in}{1.632286in}}{\pgfqpoint{1.168897in}{1.640186in}}{\pgfqpoint{1.168897in}{1.648422in}}%
\pgfpathcurveto{\pgfqpoint{1.168897in}{1.656658in}}{\pgfqpoint{1.165625in}{1.664558in}}{\pgfqpoint{1.159801in}{1.670382in}}%
\pgfpathcurveto{\pgfqpoint{1.153977in}{1.676206in}}{\pgfqpoint{1.146077in}{1.679479in}}{\pgfqpoint{1.137841in}{1.679479in}}%
\pgfpathcurveto{\pgfqpoint{1.129604in}{1.679479in}}{\pgfqpoint{1.121704in}{1.676206in}}{\pgfqpoint{1.115880in}{1.670382in}}%
\pgfpathcurveto{\pgfqpoint{1.110056in}{1.664558in}}{\pgfqpoint{1.106784in}{1.656658in}}{\pgfqpoint{1.106784in}{1.648422in}}%
\pgfpathcurveto{\pgfqpoint{1.106784in}{1.640186in}}{\pgfqpoint{1.110056in}{1.632286in}}{\pgfqpoint{1.115880in}{1.626462in}}%
\pgfpathcurveto{\pgfqpoint{1.121704in}{1.620638in}}{\pgfqpoint{1.129604in}{1.617366in}}{\pgfqpoint{1.137841in}{1.617366in}}%
\pgfpathclose%
\pgfusepath{stroke,fill}%
\end{pgfscope}%
\begin{pgfscope}%
\pgfpathrectangle{\pgfqpoint{0.100000in}{0.212622in}}{\pgfqpoint{3.696000in}{3.696000in}}%
\pgfusepath{clip}%
\pgfsetbuttcap%
\pgfsetroundjoin%
\definecolor{currentfill}{rgb}{0.121569,0.466667,0.705882}%
\pgfsetfillcolor{currentfill}%
\pgfsetfillopacity{0.300458}%
\pgfsetlinewidth{1.003750pt}%
\definecolor{currentstroke}{rgb}{0.121569,0.466667,0.705882}%
\pgfsetstrokecolor{currentstroke}%
\pgfsetstrokeopacity{0.300458}%
\pgfsetdash{}{0pt}%
\pgfpathmoveto{\pgfqpoint{1.137841in}{1.617366in}}%
\pgfpathcurveto{\pgfqpoint{1.146077in}{1.617366in}}{\pgfqpoint{1.153977in}{1.620638in}}{\pgfqpoint{1.159801in}{1.626462in}}%
\pgfpathcurveto{\pgfqpoint{1.165625in}{1.632286in}}{\pgfqpoint{1.168897in}{1.640186in}}{\pgfqpoint{1.168897in}{1.648422in}}%
\pgfpathcurveto{\pgfqpoint{1.168897in}{1.656658in}}{\pgfqpoint{1.165625in}{1.664558in}}{\pgfqpoint{1.159801in}{1.670382in}}%
\pgfpathcurveto{\pgfqpoint{1.153977in}{1.676206in}}{\pgfqpoint{1.146077in}{1.679479in}}{\pgfqpoint{1.137841in}{1.679479in}}%
\pgfpathcurveto{\pgfqpoint{1.129604in}{1.679479in}}{\pgfqpoint{1.121704in}{1.676206in}}{\pgfqpoint{1.115880in}{1.670382in}}%
\pgfpathcurveto{\pgfqpoint{1.110056in}{1.664558in}}{\pgfqpoint{1.106784in}{1.656658in}}{\pgfqpoint{1.106784in}{1.648422in}}%
\pgfpathcurveto{\pgfqpoint{1.106784in}{1.640186in}}{\pgfqpoint{1.110056in}{1.632286in}}{\pgfqpoint{1.115880in}{1.626462in}}%
\pgfpathcurveto{\pgfqpoint{1.121704in}{1.620638in}}{\pgfqpoint{1.129604in}{1.617366in}}{\pgfqpoint{1.137841in}{1.617366in}}%
\pgfpathclose%
\pgfusepath{stroke,fill}%
\end{pgfscope}%
\begin{pgfscope}%
\pgfpathrectangle{\pgfqpoint{0.100000in}{0.212622in}}{\pgfqpoint{3.696000in}{3.696000in}}%
\pgfusepath{clip}%
\pgfsetbuttcap%
\pgfsetroundjoin%
\definecolor{currentfill}{rgb}{0.121569,0.466667,0.705882}%
\pgfsetfillcolor{currentfill}%
\pgfsetfillopacity{0.300458}%
\pgfsetlinewidth{1.003750pt}%
\definecolor{currentstroke}{rgb}{0.121569,0.466667,0.705882}%
\pgfsetstrokecolor{currentstroke}%
\pgfsetstrokeopacity{0.300458}%
\pgfsetdash{}{0pt}%
\pgfpathmoveto{\pgfqpoint{1.137841in}{1.617366in}}%
\pgfpathcurveto{\pgfqpoint{1.146077in}{1.617366in}}{\pgfqpoint{1.153977in}{1.620638in}}{\pgfqpoint{1.159801in}{1.626462in}}%
\pgfpathcurveto{\pgfqpoint{1.165625in}{1.632286in}}{\pgfqpoint{1.168897in}{1.640186in}}{\pgfqpoint{1.168897in}{1.648422in}}%
\pgfpathcurveto{\pgfqpoint{1.168897in}{1.656658in}}{\pgfqpoint{1.165625in}{1.664558in}}{\pgfqpoint{1.159801in}{1.670382in}}%
\pgfpathcurveto{\pgfqpoint{1.153977in}{1.676206in}}{\pgfqpoint{1.146077in}{1.679479in}}{\pgfqpoint{1.137841in}{1.679479in}}%
\pgfpathcurveto{\pgfqpoint{1.129604in}{1.679479in}}{\pgfqpoint{1.121704in}{1.676206in}}{\pgfqpoint{1.115880in}{1.670382in}}%
\pgfpathcurveto{\pgfqpoint{1.110056in}{1.664558in}}{\pgfqpoint{1.106784in}{1.656658in}}{\pgfqpoint{1.106784in}{1.648422in}}%
\pgfpathcurveto{\pgfqpoint{1.106784in}{1.640186in}}{\pgfqpoint{1.110056in}{1.632286in}}{\pgfqpoint{1.115880in}{1.626462in}}%
\pgfpathcurveto{\pgfqpoint{1.121704in}{1.620638in}}{\pgfqpoint{1.129604in}{1.617366in}}{\pgfqpoint{1.137841in}{1.617366in}}%
\pgfpathclose%
\pgfusepath{stroke,fill}%
\end{pgfscope}%
\begin{pgfscope}%
\pgfpathrectangle{\pgfqpoint{0.100000in}{0.212622in}}{\pgfqpoint{3.696000in}{3.696000in}}%
\pgfusepath{clip}%
\pgfsetbuttcap%
\pgfsetroundjoin%
\definecolor{currentfill}{rgb}{0.121569,0.466667,0.705882}%
\pgfsetfillcolor{currentfill}%
\pgfsetfillopacity{0.300458}%
\pgfsetlinewidth{1.003750pt}%
\definecolor{currentstroke}{rgb}{0.121569,0.466667,0.705882}%
\pgfsetstrokecolor{currentstroke}%
\pgfsetstrokeopacity{0.300458}%
\pgfsetdash{}{0pt}%
\pgfpathmoveto{\pgfqpoint{1.137841in}{1.617366in}}%
\pgfpathcurveto{\pgfqpoint{1.146077in}{1.617366in}}{\pgfqpoint{1.153977in}{1.620638in}}{\pgfqpoint{1.159801in}{1.626462in}}%
\pgfpathcurveto{\pgfqpoint{1.165625in}{1.632286in}}{\pgfqpoint{1.168897in}{1.640186in}}{\pgfqpoint{1.168897in}{1.648422in}}%
\pgfpathcurveto{\pgfqpoint{1.168897in}{1.656658in}}{\pgfqpoint{1.165625in}{1.664558in}}{\pgfqpoint{1.159801in}{1.670382in}}%
\pgfpathcurveto{\pgfqpoint{1.153977in}{1.676206in}}{\pgfqpoint{1.146077in}{1.679479in}}{\pgfqpoint{1.137841in}{1.679479in}}%
\pgfpathcurveto{\pgfqpoint{1.129604in}{1.679479in}}{\pgfqpoint{1.121704in}{1.676206in}}{\pgfqpoint{1.115880in}{1.670382in}}%
\pgfpathcurveto{\pgfqpoint{1.110056in}{1.664558in}}{\pgfqpoint{1.106784in}{1.656658in}}{\pgfqpoint{1.106784in}{1.648422in}}%
\pgfpathcurveto{\pgfqpoint{1.106784in}{1.640186in}}{\pgfqpoint{1.110056in}{1.632286in}}{\pgfqpoint{1.115880in}{1.626462in}}%
\pgfpathcurveto{\pgfqpoint{1.121704in}{1.620638in}}{\pgfqpoint{1.129604in}{1.617366in}}{\pgfqpoint{1.137841in}{1.617366in}}%
\pgfpathclose%
\pgfusepath{stroke,fill}%
\end{pgfscope}%
\begin{pgfscope}%
\pgfpathrectangle{\pgfqpoint{0.100000in}{0.212622in}}{\pgfqpoint{3.696000in}{3.696000in}}%
\pgfusepath{clip}%
\pgfsetbuttcap%
\pgfsetroundjoin%
\definecolor{currentfill}{rgb}{0.121569,0.466667,0.705882}%
\pgfsetfillcolor{currentfill}%
\pgfsetfillopacity{0.300458}%
\pgfsetlinewidth{1.003750pt}%
\definecolor{currentstroke}{rgb}{0.121569,0.466667,0.705882}%
\pgfsetstrokecolor{currentstroke}%
\pgfsetstrokeopacity{0.300458}%
\pgfsetdash{}{0pt}%
\pgfpathmoveto{\pgfqpoint{1.137841in}{1.617366in}}%
\pgfpathcurveto{\pgfqpoint{1.146077in}{1.617366in}}{\pgfqpoint{1.153977in}{1.620638in}}{\pgfqpoint{1.159801in}{1.626462in}}%
\pgfpathcurveto{\pgfqpoint{1.165625in}{1.632286in}}{\pgfqpoint{1.168897in}{1.640186in}}{\pgfqpoint{1.168897in}{1.648422in}}%
\pgfpathcurveto{\pgfqpoint{1.168897in}{1.656658in}}{\pgfqpoint{1.165625in}{1.664558in}}{\pgfqpoint{1.159801in}{1.670382in}}%
\pgfpathcurveto{\pgfqpoint{1.153977in}{1.676206in}}{\pgfqpoint{1.146077in}{1.679479in}}{\pgfqpoint{1.137841in}{1.679479in}}%
\pgfpathcurveto{\pgfqpoint{1.129604in}{1.679479in}}{\pgfqpoint{1.121704in}{1.676206in}}{\pgfqpoint{1.115880in}{1.670382in}}%
\pgfpathcurveto{\pgfqpoint{1.110056in}{1.664558in}}{\pgfqpoint{1.106784in}{1.656658in}}{\pgfqpoint{1.106784in}{1.648422in}}%
\pgfpathcurveto{\pgfqpoint{1.106784in}{1.640186in}}{\pgfqpoint{1.110056in}{1.632286in}}{\pgfqpoint{1.115880in}{1.626462in}}%
\pgfpathcurveto{\pgfqpoint{1.121704in}{1.620638in}}{\pgfqpoint{1.129604in}{1.617366in}}{\pgfqpoint{1.137841in}{1.617366in}}%
\pgfpathclose%
\pgfusepath{stroke,fill}%
\end{pgfscope}%
\begin{pgfscope}%
\pgfpathrectangle{\pgfqpoint{0.100000in}{0.212622in}}{\pgfqpoint{3.696000in}{3.696000in}}%
\pgfusepath{clip}%
\pgfsetbuttcap%
\pgfsetroundjoin%
\definecolor{currentfill}{rgb}{0.121569,0.466667,0.705882}%
\pgfsetfillcolor{currentfill}%
\pgfsetfillopacity{0.300458}%
\pgfsetlinewidth{1.003750pt}%
\definecolor{currentstroke}{rgb}{0.121569,0.466667,0.705882}%
\pgfsetstrokecolor{currentstroke}%
\pgfsetstrokeopacity{0.300458}%
\pgfsetdash{}{0pt}%
\pgfpathmoveto{\pgfqpoint{1.137841in}{1.617366in}}%
\pgfpathcurveto{\pgfqpoint{1.146077in}{1.617366in}}{\pgfqpoint{1.153977in}{1.620638in}}{\pgfqpoint{1.159801in}{1.626462in}}%
\pgfpathcurveto{\pgfqpoint{1.165625in}{1.632286in}}{\pgfqpoint{1.168897in}{1.640186in}}{\pgfqpoint{1.168897in}{1.648422in}}%
\pgfpathcurveto{\pgfqpoint{1.168897in}{1.656658in}}{\pgfqpoint{1.165625in}{1.664558in}}{\pgfqpoint{1.159801in}{1.670382in}}%
\pgfpathcurveto{\pgfqpoint{1.153977in}{1.676206in}}{\pgfqpoint{1.146077in}{1.679479in}}{\pgfqpoint{1.137841in}{1.679479in}}%
\pgfpathcurveto{\pgfqpoint{1.129604in}{1.679479in}}{\pgfqpoint{1.121704in}{1.676206in}}{\pgfqpoint{1.115880in}{1.670382in}}%
\pgfpathcurveto{\pgfqpoint{1.110056in}{1.664558in}}{\pgfqpoint{1.106784in}{1.656658in}}{\pgfqpoint{1.106784in}{1.648422in}}%
\pgfpathcurveto{\pgfqpoint{1.106784in}{1.640186in}}{\pgfqpoint{1.110056in}{1.632286in}}{\pgfqpoint{1.115880in}{1.626462in}}%
\pgfpathcurveto{\pgfqpoint{1.121704in}{1.620638in}}{\pgfqpoint{1.129604in}{1.617366in}}{\pgfqpoint{1.137841in}{1.617366in}}%
\pgfpathclose%
\pgfusepath{stroke,fill}%
\end{pgfscope}%
\begin{pgfscope}%
\pgfpathrectangle{\pgfqpoint{0.100000in}{0.212622in}}{\pgfqpoint{3.696000in}{3.696000in}}%
\pgfusepath{clip}%
\pgfsetbuttcap%
\pgfsetroundjoin%
\definecolor{currentfill}{rgb}{0.121569,0.466667,0.705882}%
\pgfsetfillcolor{currentfill}%
\pgfsetfillopacity{0.300458}%
\pgfsetlinewidth{1.003750pt}%
\definecolor{currentstroke}{rgb}{0.121569,0.466667,0.705882}%
\pgfsetstrokecolor{currentstroke}%
\pgfsetstrokeopacity{0.300458}%
\pgfsetdash{}{0pt}%
\pgfpathmoveto{\pgfqpoint{1.137841in}{1.617366in}}%
\pgfpathcurveto{\pgfqpoint{1.146077in}{1.617366in}}{\pgfqpoint{1.153977in}{1.620638in}}{\pgfqpoint{1.159801in}{1.626462in}}%
\pgfpathcurveto{\pgfqpoint{1.165625in}{1.632286in}}{\pgfqpoint{1.168897in}{1.640186in}}{\pgfqpoint{1.168897in}{1.648422in}}%
\pgfpathcurveto{\pgfqpoint{1.168897in}{1.656658in}}{\pgfqpoint{1.165625in}{1.664558in}}{\pgfqpoint{1.159801in}{1.670382in}}%
\pgfpathcurveto{\pgfqpoint{1.153977in}{1.676206in}}{\pgfqpoint{1.146077in}{1.679479in}}{\pgfqpoint{1.137841in}{1.679479in}}%
\pgfpathcurveto{\pgfqpoint{1.129604in}{1.679479in}}{\pgfqpoint{1.121704in}{1.676206in}}{\pgfqpoint{1.115880in}{1.670382in}}%
\pgfpathcurveto{\pgfqpoint{1.110056in}{1.664558in}}{\pgfqpoint{1.106784in}{1.656658in}}{\pgfqpoint{1.106784in}{1.648422in}}%
\pgfpathcurveto{\pgfqpoint{1.106784in}{1.640186in}}{\pgfqpoint{1.110056in}{1.632286in}}{\pgfqpoint{1.115880in}{1.626462in}}%
\pgfpathcurveto{\pgfqpoint{1.121704in}{1.620638in}}{\pgfqpoint{1.129604in}{1.617366in}}{\pgfqpoint{1.137841in}{1.617366in}}%
\pgfpathclose%
\pgfusepath{stroke,fill}%
\end{pgfscope}%
\begin{pgfscope}%
\pgfpathrectangle{\pgfqpoint{0.100000in}{0.212622in}}{\pgfqpoint{3.696000in}{3.696000in}}%
\pgfusepath{clip}%
\pgfsetbuttcap%
\pgfsetroundjoin%
\definecolor{currentfill}{rgb}{0.121569,0.466667,0.705882}%
\pgfsetfillcolor{currentfill}%
\pgfsetfillopacity{0.300458}%
\pgfsetlinewidth{1.003750pt}%
\definecolor{currentstroke}{rgb}{0.121569,0.466667,0.705882}%
\pgfsetstrokecolor{currentstroke}%
\pgfsetstrokeopacity{0.300458}%
\pgfsetdash{}{0pt}%
\pgfpathmoveto{\pgfqpoint{1.137841in}{1.617366in}}%
\pgfpathcurveto{\pgfqpoint{1.146077in}{1.617366in}}{\pgfqpoint{1.153977in}{1.620638in}}{\pgfqpoint{1.159801in}{1.626462in}}%
\pgfpathcurveto{\pgfqpoint{1.165625in}{1.632286in}}{\pgfqpoint{1.168897in}{1.640186in}}{\pgfqpoint{1.168897in}{1.648422in}}%
\pgfpathcurveto{\pgfqpoint{1.168897in}{1.656658in}}{\pgfqpoint{1.165625in}{1.664558in}}{\pgfqpoint{1.159801in}{1.670382in}}%
\pgfpathcurveto{\pgfqpoint{1.153977in}{1.676206in}}{\pgfqpoint{1.146077in}{1.679479in}}{\pgfqpoint{1.137841in}{1.679479in}}%
\pgfpathcurveto{\pgfqpoint{1.129604in}{1.679479in}}{\pgfqpoint{1.121704in}{1.676206in}}{\pgfqpoint{1.115880in}{1.670382in}}%
\pgfpathcurveto{\pgfqpoint{1.110056in}{1.664558in}}{\pgfqpoint{1.106784in}{1.656658in}}{\pgfqpoint{1.106784in}{1.648422in}}%
\pgfpathcurveto{\pgfqpoint{1.106784in}{1.640186in}}{\pgfqpoint{1.110056in}{1.632286in}}{\pgfqpoint{1.115880in}{1.626462in}}%
\pgfpathcurveto{\pgfqpoint{1.121704in}{1.620638in}}{\pgfqpoint{1.129604in}{1.617366in}}{\pgfqpoint{1.137841in}{1.617366in}}%
\pgfpathclose%
\pgfusepath{stroke,fill}%
\end{pgfscope}%
\begin{pgfscope}%
\pgfpathrectangle{\pgfqpoint{0.100000in}{0.212622in}}{\pgfqpoint{3.696000in}{3.696000in}}%
\pgfusepath{clip}%
\pgfsetbuttcap%
\pgfsetroundjoin%
\definecolor{currentfill}{rgb}{0.121569,0.466667,0.705882}%
\pgfsetfillcolor{currentfill}%
\pgfsetfillopacity{0.300458}%
\pgfsetlinewidth{1.003750pt}%
\definecolor{currentstroke}{rgb}{0.121569,0.466667,0.705882}%
\pgfsetstrokecolor{currentstroke}%
\pgfsetstrokeopacity{0.300458}%
\pgfsetdash{}{0pt}%
\pgfpathmoveto{\pgfqpoint{1.137841in}{1.617366in}}%
\pgfpathcurveto{\pgfqpoint{1.146077in}{1.617366in}}{\pgfqpoint{1.153977in}{1.620638in}}{\pgfqpoint{1.159801in}{1.626462in}}%
\pgfpathcurveto{\pgfqpoint{1.165625in}{1.632286in}}{\pgfqpoint{1.168897in}{1.640186in}}{\pgfqpoint{1.168897in}{1.648422in}}%
\pgfpathcurveto{\pgfqpoint{1.168897in}{1.656658in}}{\pgfqpoint{1.165625in}{1.664558in}}{\pgfqpoint{1.159801in}{1.670382in}}%
\pgfpathcurveto{\pgfqpoint{1.153977in}{1.676206in}}{\pgfqpoint{1.146077in}{1.679479in}}{\pgfqpoint{1.137841in}{1.679479in}}%
\pgfpathcurveto{\pgfqpoint{1.129604in}{1.679479in}}{\pgfqpoint{1.121704in}{1.676206in}}{\pgfqpoint{1.115880in}{1.670382in}}%
\pgfpathcurveto{\pgfqpoint{1.110056in}{1.664558in}}{\pgfqpoint{1.106784in}{1.656658in}}{\pgfqpoint{1.106784in}{1.648422in}}%
\pgfpathcurveto{\pgfqpoint{1.106784in}{1.640186in}}{\pgfqpoint{1.110056in}{1.632286in}}{\pgfqpoint{1.115880in}{1.626462in}}%
\pgfpathcurveto{\pgfqpoint{1.121704in}{1.620638in}}{\pgfqpoint{1.129604in}{1.617366in}}{\pgfqpoint{1.137841in}{1.617366in}}%
\pgfpathclose%
\pgfusepath{stroke,fill}%
\end{pgfscope}%
\begin{pgfscope}%
\pgfpathrectangle{\pgfqpoint{0.100000in}{0.212622in}}{\pgfqpoint{3.696000in}{3.696000in}}%
\pgfusepath{clip}%
\pgfsetbuttcap%
\pgfsetroundjoin%
\definecolor{currentfill}{rgb}{0.121569,0.466667,0.705882}%
\pgfsetfillcolor{currentfill}%
\pgfsetfillopacity{0.300458}%
\pgfsetlinewidth{1.003750pt}%
\definecolor{currentstroke}{rgb}{0.121569,0.466667,0.705882}%
\pgfsetstrokecolor{currentstroke}%
\pgfsetstrokeopacity{0.300458}%
\pgfsetdash{}{0pt}%
\pgfpathmoveto{\pgfqpoint{1.137841in}{1.617366in}}%
\pgfpathcurveto{\pgfqpoint{1.146077in}{1.617366in}}{\pgfqpoint{1.153977in}{1.620638in}}{\pgfqpoint{1.159801in}{1.626462in}}%
\pgfpathcurveto{\pgfqpoint{1.165625in}{1.632286in}}{\pgfqpoint{1.168897in}{1.640186in}}{\pgfqpoint{1.168897in}{1.648422in}}%
\pgfpathcurveto{\pgfqpoint{1.168897in}{1.656658in}}{\pgfqpoint{1.165625in}{1.664558in}}{\pgfqpoint{1.159801in}{1.670382in}}%
\pgfpathcurveto{\pgfqpoint{1.153977in}{1.676206in}}{\pgfqpoint{1.146077in}{1.679479in}}{\pgfqpoint{1.137841in}{1.679479in}}%
\pgfpathcurveto{\pgfqpoint{1.129604in}{1.679479in}}{\pgfqpoint{1.121704in}{1.676206in}}{\pgfqpoint{1.115880in}{1.670382in}}%
\pgfpathcurveto{\pgfqpoint{1.110056in}{1.664558in}}{\pgfqpoint{1.106784in}{1.656658in}}{\pgfqpoint{1.106784in}{1.648422in}}%
\pgfpathcurveto{\pgfqpoint{1.106784in}{1.640186in}}{\pgfqpoint{1.110056in}{1.632286in}}{\pgfqpoint{1.115880in}{1.626462in}}%
\pgfpathcurveto{\pgfqpoint{1.121704in}{1.620638in}}{\pgfqpoint{1.129604in}{1.617366in}}{\pgfqpoint{1.137841in}{1.617366in}}%
\pgfpathclose%
\pgfusepath{stroke,fill}%
\end{pgfscope}%
\begin{pgfscope}%
\pgfpathrectangle{\pgfqpoint{0.100000in}{0.212622in}}{\pgfqpoint{3.696000in}{3.696000in}}%
\pgfusepath{clip}%
\pgfsetbuttcap%
\pgfsetroundjoin%
\definecolor{currentfill}{rgb}{0.121569,0.466667,0.705882}%
\pgfsetfillcolor{currentfill}%
\pgfsetfillopacity{0.300458}%
\pgfsetlinewidth{1.003750pt}%
\definecolor{currentstroke}{rgb}{0.121569,0.466667,0.705882}%
\pgfsetstrokecolor{currentstroke}%
\pgfsetstrokeopacity{0.300458}%
\pgfsetdash{}{0pt}%
\pgfpathmoveto{\pgfqpoint{1.137841in}{1.617366in}}%
\pgfpathcurveto{\pgfqpoint{1.146077in}{1.617366in}}{\pgfqpoint{1.153977in}{1.620638in}}{\pgfqpoint{1.159801in}{1.626462in}}%
\pgfpathcurveto{\pgfqpoint{1.165625in}{1.632286in}}{\pgfqpoint{1.168897in}{1.640186in}}{\pgfqpoint{1.168897in}{1.648422in}}%
\pgfpathcurveto{\pgfqpoint{1.168897in}{1.656658in}}{\pgfqpoint{1.165625in}{1.664558in}}{\pgfqpoint{1.159801in}{1.670382in}}%
\pgfpathcurveto{\pgfqpoint{1.153977in}{1.676206in}}{\pgfqpoint{1.146077in}{1.679479in}}{\pgfqpoint{1.137841in}{1.679479in}}%
\pgfpathcurveto{\pgfqpoint{1.129604in}{1.679479in}}{\pgfqpoint{1.121704in}{1.676206in}}{\pgfqpoint{1.115880in}{1.670382in}}%
\pgfpathcurveto{\pgfqpoint{1.110056in}{1.664558in}}{\pgfqpoint{1.106784in}{1.656658in}}{\pgfqpoint{1.106784in}{1.648422in}}%
\pgfpathcurveto{\pgfqpoint{1.106784in}{1.640186in}}{\pgfqpoint{1.110056in}{1.632286in}}{\pgfqpoint{1.115880in}{1.626462in}}%
\pgfpathcurveto{\pgfqpoint{1.121704in}{1.620638in}}{\pgfqpoint{1.129604in}{1.617366in}}{\pgfqpoint{1.137841in}{1.617366in}}%
\pgfpathclose%
\pgfusepath{stroke,fill}%
\end{pgfscope}%
\begin{pgfscope}%
\pgfpathrectangle{\pgfqpoint{0.100000in}{0.212622in}}{\pgfqpoint{3.696000in}{3.696000in}}%
\pgfusepath{clip}%
\pgfsetbuttcap%
\pgfsetroundjoin%
\definecolor{currentfill}{rgb}{0.121569,0.466667,0.705882}%
\pgfsetfillcolor{currentfill}%
\pgfsetfillopacity{0.300458}%
\pgfsetlinewidth{1.003750pt}%
\definecolor{currentstroke}{rgb}{0.121569,0.466667,0.705882}%
\pgfsetstrokecolor{currentstroke}%
\pgfsetstrokeopacity{0.300458}%
\pgfsetdash{}{0pt}%
\pgfpathmoveto{\pgfqpoint{1.137841in}{1.617366in}}%
\pgfpathcurveto{\pgfqpoint{1.146077in}{1.617366in}}{\pgfqpoint{1.153977in}{1.620638in}}{\pgfqpoint{1.159801in}{1.626462in}}%
\pgfpathcurveto{\pgfqpoint{1.165625in}{1.632286in}}{\pgfqpoint{1.168897in}{1.640186in}}{\pgfqpoint{1.168897in}{1.648422in}}%
\pgfpathcurveto{\pgfqpoint{1.168897in}{1.656658in}}{\pgfqpoint{1.165625in}{1.664558in}}{\pgfqpoint{1.159801in}{1.670382in}}%
\pgfpathcurveto{\pgfqpoint{1.153977in}{1.676206in}}{\pgfqpoint{1.146077in}{1.679479in}}{\pgfqpoint{1.137841in}{1.679479in}}%
\pgfpathcurveto{\pgfqpoint{1.129604in}{1.679479in}}{\pgfqpoint{1.121704in}{1.676206in}}{\pgfqpoint{1.115880in}{1.670382in}}%
\pgfpathcurveto{\pgfqpoint{1.110056in}{1.664558in}}{\pgfqpoint{1.106784in}{1.656658in}}{\pgfqpoint{1.106784in}{1.648422in}}%
\pgfpathcurveto{\pgfqpoint{1.106784in}{1.640186in}}{\pgfqpoint{1.110056in}{1.632286in}}{\pgfqpoint{1.115880in}{1.626462in}}%
\pgfpathcurveto{\pgfqpoint{1.121704in}{1.620638in}}{\pgfqpoint{1.129604in}{1.617366in}}{\pgfqpoint{1.137841in}{1.617366in}}%
\pgfpathclose%
\pgfusepath{stroke,fill}%
\end{pgfscope}%
\begin{pgfscope}%
\pgfpathrectangle{\pgfqpoint{0.100000in}{0.212622in}}{\pgfqpoint{3.696000in}{3.696000in}}%
\pgfusepath{clip}%
\pgfsetbuttcap%
\pgfsetroundjoin%
\definecolor{currentfill}{rgb}{0.121569,0.466667,0.705882}%
\pgfsetfillcolor{currentfill}%
\pgfsetfillopacity{0.300458}%
\pgfsetlinewidth{1.003750pt}%
\definecolor{currentstroke}{rgb}{0.121569,0.466667,0.705882}%
\pgfsetstrokecolor{currentstroke}%
\pgfsetstrokeopacity{0.300458}%
\pgfsetdash{}{0pt}%
\pgfpathmoveto{\pgfqpoint{1.137841in}{1.617366in}}%
\pgfpathcurveto{\pgfqpoint{1.146077in}{1.617366in}}{\pgfqpoint{1.153977in}{1.620638in}}{\pgfqpoint{1.159801in}{1.626462in}}%
\pgfpathcurveto{\pgfqpoint{1.165625in}{1.632286in}}{\pgfqpoint{1.168897in}{1.640186in}}{\pgfqpoint{1.168897in}{1.648422in}}%
\pgfpathcurveto{\pgfqpoint{1.168897in}{1.656658in}}{\pgfqpoint{1.165625in}{1.664558in}}{\pgfqpoint{1.159801in}{1.670382in}}%
\pgfpathcurveto{\pgfqpoint{1.153977in}{1.676206in}}{\pgfqpoint{1.146077in}{1.679479in}}{\pgfqpoint{1.137841in}{1.679479in}}%
\pgfpathcurveto{\pgfqpoint{1.129604in}{1.679479in}}{\pgfqpoint{1.121704in}{1.676206in}}{\pgfqpoint{1.115880in}{1.670382in}}%
\pgfpathcurveto{\pgfqpoint{1.110056in}{1.664558in}}{\pgfqpoint{1.106784in}{1.656658in}}{\pgfqpoint{1.106784in}{1.648422in}}%
\pgfpathcurveto{\pgfqpoint{1.106784in}{1.640186in}}{\pgfqpoint{1.110056in}{1.632286in}}{\pgfqpoint{1.115880in}{1.626462in}}%
\pgfpathcurveto{\pgfqpoint{1.121704in}{1.620638in}}{\pgfqpoint{1.129604in}{1.617366in}}{\pgfqpoint{1.137841in}{1.617366in}}%
\pgfpathclose%
\pgfusepath{stroke,fill}%
\end{pgfscope}%
\begin{pgfscope}%
\pgfpathrectangle{\pgfqpoint{0.100000in}{0.212622in}}{\pgfqpoint{3.696000in}{3.696000in}}%
\pgfusepath{clip}%
\pgfsetbuttcap%
\pgfsetroundjoin%
\definecolor{currentfill}{rgb}{0.121569,0.466667,0.705882}%
\pgfsetfillcolor{currentfill}%
\pgfsetfillopacity{0.300458}%
\pgfsetlinewidth{1.003750pt}%
\definecolor{currentstroke}{rgb}{0.121569,0.466667,0.705882}%
\pgfsetstrokecolor{currentstroke}%
\pgfsetstrokeopacity{0.300458}%
\pgfsetdash{}{0pt}%
\pgfpathmoveto{\pgfqpoint{1.137841in}{1.617366in}}%
\pgfpathcurveto{\pgfqpoint{1.146077in}{1.617366in}}{\pgfqpoint{1.153977in}{1.620638in}}{\pgfqpoint{1.159801in}{1.626462in}}%
\pgfpathcurveto{\pgfqpoint{1.165625in}{1.632286in}}{\pgfqpoint{1.168897in}{1.640186in}}{\pgfqpoint{1.168897in}{1.648422in}}%
\pgfpathcurveto{\pgfqpoint{1.168897in}{1.656658in}}{\pgfqpoint{1.165625in}{1.664558in}}{\pgfqpoint{1.159801in}{1.670382in}}%
\pgfpathcurveto{\pgfqpoint{1.153977in}{1.676206in}}{\pgfqpoint{1.146077in}{1.679479in}}{\pgfqpoint{1.137841in}{1.679479in}}%
\pgfpathcurveto{\pgfqpoint{1.129604in}{1.679479in}}{\pgfqpoint{1.121704in}{1.676206in}}{\pgfqpoint{1.115880in}{1.670382in}}%
\pgfpathcurveto{\pgfqpoint{1.110056in}{1.664558in}}{\pgfqpoint{1.106784in}{1.656658in}}{\pgfqpoint{1.106784in}{1.648422in}}%
\pgfpathcurveto{\pgfqpoint{1.106784in}{1.640186in}}{\pgfqpoint{1.110056in}{1.632286in}}{\pgfqpoint{1.115880in}{1.626462in}}%
\pgfpathcurveto{\pgfqpoint{1.121704in}{1.620638in}}{\pgfqpoint{1.129604in}{1.617366in}}{\pgfqpoint{1.137841in}{1.617366in}}%
\pgfpathclose%
\pgfusepath{stroke,fill}%
\end{pgfscope}%
\begin{pgfscope}%
\pgfpathrectangle{\pgfqpoint{0.100000in}{0.212622in}}{\pgfqpoint{3.696000in}{3.696000in}}%
\pgfusepath{clip}%
\pgfsetbuttcap%
\pgfsetroundjoin%
\definecolor{currentfill}{rgb}{0.121569,0.466667,0.705882}%
\pgfsetfillcolor{currentfill}%
\pgfsetfillopacity{0.300458}%
\pgfsetlinewidth{1.003750pt}%
\definecolor{currentstroke}{rgb}{0.121569,0.466667,0.705882}%
\pgfsetstrokecolor{currentstroke}%
\pgfsetstrokeopacity{0.300458}%
\pgfsetdash{}{0pt}%
\pgfpathmoveto{\pgfqpoint{1.137841in}{1.617366in}}%
\pgfpathcurveto{\pgfqpoint{1.146077in}{1.617366in}}{\pgfqpoint{1.153977in}{1.620638in}}{\pgfqpoint{1.159801in}{1.626462in}}%
\pgfpathcurveto{\pgfqpoint{1.165625in}{1.632286in}}{\pgfqpoint{1.168897in}{1.640186in}}{\pgfqpoint{1.168897in}{1.648422in}}%
\pgfpathcurveto{\pgfqpoint{1.168897in}{1.656658in}}{\pgfqpoint{1.165625in}{1.664558in}}{\pgfqpoint{1.159801in}{1.670382in}}%
\pgfpathcurveto{\pgfqpoint{1.153977in}{1.676206in}}{\pgfqpoint{1.146077in}{1.679479in}}{\pgfqpoint{1.137841in}{1.679479in}}%
\pgfpathcurveto{\pgfqpoint{1.129604in}{1.679479in}}{\pgfqpoint{1.121704in}{1.676206in}}{\pgfqpoint{1.115880in}{1.670382in}}%
\pgfpathcurveto{\pgfqpoint{1.110056in}{1.664558in}}{\pgfqpoint{1.106784in}{1.656658in}}{\pgfqpoint{1.106784in}{1.648422in}}%
\pgfpathcurveto{\pgfqpoint{1.106784in}{1.640186in}}{\pgfqpoint{1.110056in}{1.632286in}}{\pgfqpoint{1.115880in}{1.626462in}}%
\pgfpathcurveto{\pgfqpoint{1.121704in}{1.620638in}}{\pgfqpoint{1.129604in}{1.617366in}}{\pgfqpoint{1.137841in}{1.617366in}}%
\pgfpathclose%
\pgfusepath{stroke,fill}%
\end{pgfscope}%
\begin{pgfscope}%
\pgfpathrectangle{\pgfqpoint{0.100000in}{0.212622in}}{\pgfqpoint{3.696000in}{3.696000in}}%
\pgfusepath{clip}%
\pgfsetbuttcap%
\pgfsetroundjoin%
\definecolor{currentfill}{rgb}{0.121569,0.466667,0.705882}%
\pgfsetfillcolor{currentfill}%
\pgfsetfillopacity{0.300458}%
\pgfsetlinewidth{1.003750pt}%
\definecolor{currentstroke}{rgb}{0.121569,0.466667,0.705882}%
\pgfsetstrokecolor{currentstroke}%
\pgfsetstrokeopacity{0.300458}%
\pgfsetdash{}{0pt}%
\pgfpathmoveto{\pgfqpoint{1.137841in}{1.617366in}}%
\pgfpathcurveto{\pgfqpoint{1.146077in}{1.617366in}}{\pgfqpoint{1.153977in}{1.620638in}}{\pgfqpoint{1.159801in}{1.626462in}}%
\pgfpathcurveto{\pgfqpoint{1.165625in}{1.632286in}}{\pgfqpoint{1.168897in}{1.640186in}}{\pgfqpoint{1.168897in}{1.648422in}}%
\pgfpathcurveto{\pgfqpoint{1.168897in}{1.656658in}}{\pgfqpoint{1.165625in}{1.664558in}}{\pgfqpoint{1.159801in}{1.670382in}}%
\pgfpathcurveto{\pgfqpoint{1.153977in}{1.676206in}}{\pgfqpoint{1.146077in}{1.679479in}}{\pgfqpoint{1.137841in}{1.679479in}}%
\pgfpathcurveto{\pgfqpoint{1.129604in}{1.679479in}}{\pgfqpoint{1.121704in}{1.676206in}}{\pgfqpoint{1.115880in}{1.670382in}}%
\pgfpathcurveto{\pgfqpoint{1.110056in}{1.664558in}}{\pgfqpoint{1.106784in}{1.656658in}}{\pgfqpoint{1.106784in}{1.648422in}}%
\pgfpathcurveto{\pgfqpoint{1.106784in}{1.640186in}}{\pgfqpoint{1.110056in}{1.632286in}}{\pgfqpoint{1.115880in}{1.626462in}}%
\pgfpathcurveto{\pgfqpoint{1.121704in}{1.620638in}}{\pgfqpoint{1.129604in}{1.617366in}}{\pgfqpoint{1.137841in}{1.617366in}}%
\pgfpathclose%
\pgfusepath{stroke,fill}%
\end{pgfscope}%
\begin{pgfscope}%
\pgfpathrectangle{\pgfqpoint{0.100000in}{0.212622in}}{\pgfqpoint{3.696000in}{3.696000in}}%
\pgfusepath{clip}%
\pgfsetbuttcap%
\pgfsetroundjoin%
\definecolor{currentfill}{rgb}{0.121569,0.466667,0.705882}%
\pgfsetfillcolor{currentfill}%
\pgfsetfillopacity{0.300458}%
\pgfsetlinewidth{1.003750pt}%
\definecolor{currentstroke}{rgb}{0.121569,0.466667,0.705882}%
\pgfsetstrokecolor{currentstroke}%
\pgfsetstrokeopacity{0.300458}%
\pgfsetdash{}{0pt}%
\pgfpathmoveto{\pgfqpoint{1.137841in}{1.617366in}}%
\pgfpathcurveto{\pgfqpoint{1.146077in}{1.617366in}}{\pgfqpoint{1.153977in}{1.620638in}}{\pgfqpoint{1.159801in}{1.626462in}}%
\pgfpathcurveto{\pgfqpoint{1.165625in}{1.632286in}}{\pgfqpoint{1.168897in}{1.640186in}}{\pgfqpoint{1.168897in}{1.648422in}}%
\pgfpathcurveto{\pgfqpoint{1.168897in}{1.656658in}}{\pgfqpoint{1.165625in}{1.664558in}}{\pgfqpoint{1.159801in}{1.670382in}}%
\pgfpathcurveto{\pgfqpoint{1.153977in}{1.676206in}}{\pgfqpoint{1.146077in}{1.679479in}}{\pgfqpoint{1.137841in}{1.679479in}}%
\pgfpathcurveto{\pgfqpoint{1.129604in}{1.679479in}}{\pgfqpoint{1.121704in}{1.676206in}}{\pgfqpoint{1.115880in}{1.670382in}}%
\pgfpathcurveto{\pgfqpoint{1.110056in}{1.664558in}}{\pgfqpoint{1.106784in}{1.656658in}}{\pgfqpoint{1.106784in}{1.648422in}}%
\pgfpathcurveto{\pgfqpoint{1.106784in}{1.640186in}}{\pgfqpoint{1.110056in}{1.632286in}}{\pgfqpoint{1.115880in}{1.626462in}}%
\pgfpathcurveto{\pgfqpoint{1.121704in}{1.620638in}}{\pgfqpoint{1.129604in}{1.617366in}}{\pgfqpoint{1.137841in}{1.617366in}}%
\pgfpathclose%
\pgfusepath{stroke,fill}%
\end{pgfscope}%
\begin{pgfscope}%
\pgfpathrectangle{\pgfqpoint{0.100000in}{0.212622in}}{\pgfqpoint{3.696000in}{3.696000in}}%
\pgfusepath{clip}%
\pgfsetbuttcap%
\pgfsetroundjoin%
\definecolor{currentfill}{rgb}{0.121569,0.466667,0.705882}%
\pgfsetfillcolor{currentfill}%
\pgfsetfillopacity{0.300458}%
\pgfsetlinewidth{1.003750pt}%
\definecolor{currentstroke}{rgb}{0.121569,0.466667,0.705882}%
\pgfsetstrokecolor{currentstroke}%
\pgfsetstrokeopacity{0.300458}%
\pgfsetdash{}{0pt}%
\pgfpathmoveto{\pgfqpoint{1.137841in}{1.617366in}}%
\pgfpathcurveto{\pgfqpoint{1.146077in}{1.617366in}}{\pgfqpoint{1.153977in}{1.620638in}}{\pgfqpoint{1.159801in}{1.626462in}}%
\pgfpathcurveto{\pgfqpoint{1.165625in}{1.632286in}}{\pgfqpoint{1.168897in}{1.640186in}}{\pgfqpoint{1.168897in}{1.648422in}}%
\pgfpathcurveto{\pgfqpoint{1.168897in}{1.656658in}}{\pgfqpoint{1.165625in}{1.664558in}}{\pgfqpoint{1.159801in}{1.670382in}}%
\pgfpathcurveto{\pgfqpoint{1.153977in}{1.676206in}}{\pgfqpoint{1.146077in}{1.679479in}}{\pgfqpoint{1.137841in}{1.679479in}}%
\pgfpathcurveto{\pgfqpoint{1.129604in}{1.679479in}}{\pgfqpoint{1.121704in}{1.676206in}}{\pgfqpoint{1.115880in}{1.670382in}}%
\pgfpathcurveto{\pgfqpoint{1.110056in}{1.664558in}}{\pgfqpoint{1.106784in}{1.656658in}}{\pgfqpoint{1.106784in}{1.648422in}}%
\pgfpathcurveto{\pgfqpoint{1.106784in}{1.640186in}}{\pgfqpoint{1.110056in}{1.632286in}}{\pgfqpoint{1.115880in}{1.626462in}}%
\pgfpathcurveto{\pgfqpoint{1.121704in}{1.620638in}}{\pgfqpoint{1.129604in}{1.617366in}}{\pgfqpoint{1.137841in}{1.617366in}}%
\pgfpathclose%
\pgfusepath{stroke,fill}%
\end{pgfscope}%
\begin{pgfscope}%
\pgfpathrectangle{\pgfqpoint{0.100000in}{0.212622in}}{\pgfqpoint{3.696000in}{3.696000in}}%
\pgfusepath{clip}%
\pgfsetbuttcap%
\pgfsetroundjoin%
\definecolor{currentfill}{rgb}{0.121569,0.466667,0.705882}%
\pgfsetfillcolor{currentfill}%
\pgfsetfillopacity{0.300458}%
\pgfsetlinewidth{1.003750pt}%
\definecolor{currentstroke}{rgb}{0.121569,0.466667,0.705882}%
\pgfsetstrokecolor{currentstroke}%
\pgfsetstrokeopacity{0.300458}%
\pgfsetdash{}{0pt}%
\pgfpathmoveto{\pgfqpoint{1.137841in}{1.617366in}}%
\pgfpathcurveto{\pgfqpoint{1.146077in}{1.617366in}}{\pgfqpoint{1.153977in}{1.620638in}}{\pgfqpoint{1.159801in}{1.626462in}}%
\pgfpathcurveto{\pgfqpoint{1.165625in}{1.632286in}}{\pgfqpoint{1.168897in}{1.640186in}}{\pgfqpoint{1.168897in}{1.648422in}}%
\pgfpathcurveto{\pgfqpoint{1.168897in}{1.656658in}}{\pgfqpoint{1.165625in}{1.664558in}}{\pgfqpoint{1.159801in}{1.670382in}}%
\pgfpathcurveto{\pgfqpoint{1.153977in}{1.676206in}}{\pgfqpoint{1.146077in}{1.679479in}}{\pgfqpoint{1.137841in}{1.679479in}}%
\pgfpathcurveto{\pgfqpoint{1.129604in}{1.679479in}}{\pgfqpoint{1.121704in}{1.676206in}}{\pgfqpoint{1.115880in}{1.670382in}}%
\pgfpathcurveto{\pgfqpoint{1.110056in}{1.664558in}}{\pgfqpoint{1.106784in}{1.656658in}}{\pgfqpoint{1.106784in}{1.648422in}}%
\pgfpathcurveto{\pgfqpoint{1.106784in}{1.640186in}}{\pgfqpoint{1.110056in}{1.632286in}}{\pgfqpoint{1.115880in}{1.626462in}}%
\pgfpathcurveto{\pgfqpoint{1.121704in}{1.620638in}}{\pgfqpoint{1.129604in}{1.617366in}}{\pgfqpoint{1.137841in}{1.617366in}}%
\pgfpathclose%
\pgfusepath{stroke,fill}%
\end{pgfscope}%
\begin{pgfscope}%
\pgfpathrectangle{\pgfqpoint{0.100000in}{0.212622in}}{\pgfqpoint{3.696000in}{3.696000in}}%
\pgfusepath{clip}%
\pgfsetbuttcap%
\pgfsetroundjoin%
\definecolor{currentfill}{rgb}{0.121569,0.466667,0.705882}%
\pgfsetfillcolor{currentfill}%
\pgfsetfillopacity{0.300458}%
\pgfsetlinewidth{1.003750pt}%
\definecolor{currentstroke}{rgb}{0.121569,0.466667,0.705882}%
\pgfsetstrokecolor{currentstroke}%
\pgfsetstrokeopacity{0.300458}%
\pgfsetdash{}{0pt}%
\pgfpathmoveto{\pgfqpoint{1.137841in}{1.617366in}}%
\pgfpathcurveto{\pgfqpoint{1.146077in}{1.617366in}}{\pgfqpoint{1.153977in}{1.620638in}}{\pgfqpoint{1.159801in}{1.626462in}}%
\pgfpathcurveto{\pgfqpoint{1.165625in}{1.632286in}}{\pgfqpoint{1.168897in}{1.640186in}}{\pgfqpoint{1.168897in}{1.648422in}}%
\pgfpathcurveto{\pgfqpoint{1.168897in}{1.656658in}}{\pgfqpoint{1.165625in}{1.664558in}}{\pgfqpoint{1.159801in}{1.670382in}}%
\pgfpathcurveto{\pgfqpoint{1.153977in}{1.676206in}}{\pgfqpoint{1.146077in}{1.679479in}}{\pgfqpoint{1.137841in}{1.679479in}}%
\pgfpathcurveto{\pgfqpoint{1.129604in}{1.679479in}}{\pgfqpoint{1.121704in}{1.676206in}}{\pgfqpoint{1.115880in}{1.670382in}}%
\pgfpathcurveto{\pgfqpoint{1.110056in}{1.664558in}}{\pgfqpoint{1.106784in}{1.656658in}}{\pgfqpoint{1.106784in}{1.648422in}}%
\pgfpathcurveto{\pgfqpoint{1.106784in}{1.640186in}}{\pgfqpoint{1.110056in}{1.632286in}}{\pgfqpoint{1.115880in}{1.626462in}}%
\pgfpathcurveto{\pgfqpoint{1.121704in}{1.620638in}}{\pgfqpoint{1.129604in}{1.617366in}}{\pgfqpoint{1.137841in}{1.617366in}}%
\pgfpathclose%
\pgfusepath{stroke,fill}%
\end{pgfscope}%
\begin{pgfscope}%
\pgfpathrectangle{\pgfqpoint{0.100000in}{0.212622in}}{\pgfqpoint{3.696000in}{3.696000in}}%
\pgfusepath{clip}%
\pgfsetbuttcap%
\pgfsetroundjoin%
\definecolor{currentfill}{rgb}{0.121569,0.466667,0.705882}%
\pgfsetfillcolor{currentfill}%
\pgfsetfillopacity{0.300458}%
\pgfsetlinewidth{1.003750pt}%
\definecolor{currentstroke}{rgb}{0.121569,0.466667,0.705882}%
\pgfsetstrokecolor{currentstroke}%
\pgfsetstrokeopacity{0.300458}%
\pgfsetdash{}{0pt}%
\pgfpathmoveto{\pgfqpoint{1.137841in}{1.617366in}}%
\pgfpathcurveto{\pgfqpoint{1.146077in}{1.617366in}}{\pgfqpoint{1.153977in}{1.620638in}}{\pgfqpoint{1.159801in}{1.626462in}}%
\pgfpathcurveto{\pgfqpoint{1.165625in}{1.632286in}}{\pgfqpoint{1.168897in}{1.640186in}}{\pgfqpoint{1.168897in}{1.648422in}}%
\pgfpathcurveto{\pgfqpoint{1.168897in}{1.656658in}}{\pgfqpoint{1.165625in}{1.664558in}}{\pgfqpoint{1.159801in}{1.670382in}}%
\pgfpathcurveto{\pgfqpoint{1.153977in}{1.676206in}}{\pgfqpoint{1.146077in}{1.679479in}}{\pgfqpoint{1.137841in}{1.679479in}}%
\pgfpathcurveto{\pgfqpoint{1.129604in}{1.679479in}}{\pgfqpoint{1.121704in}{1.676206in}}{\pgfqpoint{1.115880in}{1.670382in}}%
\pgfpathcurveto{\pgfqpoint{1.110056in}{1.664558in}}{\pgfqpoint{1.106784in}{1.656658in}}{\pgfqpoint{1.106784in}{1.648422in}}%
\pgfpathcurveto{\pgfqpoint{1.106784in}{1.640186in}}{\pgfqpoint{1.110056in}{1.632286in}}{\pgfqpoint{1.115880in}{1.626462in}}%
\pgfpathcurveto{\pgfqpoint{1.121704in}{1.620638in}}{\pgfqpoint{1.129604in}{1.617366in}}{\pgfqpoint{1.137841in}{1.617366in}}%
\pgfpathclose%
\pgfusepath{stroke,fill}%
\end{pgfscope}%
\begin{pgfscope}%
\pgfpathrectangle{\pgfqpoint{0.100000in}{0.212622in}}{\pgfqpoint{3.696000in}{3.696000in}}%
\pgfusepath{clip}%
\pgfsetbuttcap%
\pgfsetroundjoin%
\definecolor{currentfill}{rgb}{0.121569,0.466667,0.705882}%
\pgfsetfillcolor{currentfill}%
\pgfsetfillopacity{0.300458}%
\pgfsetlinewidth{1.003750pt}%
\definecolor{currentstroke}{rgb}{0.121569,0.466667,0.705882}%
\pgfsetstrokecolor{currentstroke}%
\pgfsetstrokeopacity{0.300458}%
\pgfsetdash{}{0pt}%
\pgfpathmoveto{\pgfqpoint{1.137841in}{1.617366in}}%
\pgfpathcurveto{\pgfqpoint{1.146077in}{1.617366in}}{\pgfqpoint{1.153977in}{1.620638in}}{\pgfqpoint{1.159801in}{1.626462in}}%
\pgfpathcurveto{\pgfqpoint{1.165625in}{1.632286in}}{\pgfqpoint{1.168897in}{1.640186in}}{\pgfqpoint{1.168897in}{1.648422in}}%
\pgfpathcurveto{\pgfqpoint{1.168897in}{1.656658in}}{\pgfqpoint{1.165625in}{1.664558in}}{\pgfqpoint{1.159801in}{1.670382in}}%
\pgfpathcurveto{\pgfqpoint{1.153977in}{1.676206in}}{\pgfqpoint{1.146077in}{1.679479in}}{\pgfqpoint{1.137841in}{1.679479in}}%
\pgfpathcurveto{\pgfqpoint{1.129604in}{1.679479in}}{\pgfqpoint{1.121704in}{1.676206in}}{\pgfqpoint{1.115880in}{1.670382in}}%
\pgfpathcurveto{\pgfqpoint{1.110056in}{1.664558in}}{\pgfqpoint{1.106784in}{1.656658in}}{\pgfqpoint{1.106784in}{1.648422in}}%
\pgfpathcurveto{\pgfqpoint{1.106784in}{1.640186in}}{\pgfqpoint{1.110056in}{1.632286in}}{\pgfqpoint{1.115880in}{1.626462in}}%
\pgfpathcurveto{\pgfqpoint{1.121704in}{1.620638in}}{\pgfqpoint{1.129604in}{1.617366in}}{\pgfqpoint{1.137841in}{1.617366in}}%
\pgfpathclose%
\pgfusepath{stroke,fill}%
\end{pgfscope}%
\begin{pgfscope}%
\pgfpathrectangle{\pgfqpoint{0.100000in}{0.212622in}}{\pgfqpoint{3.696000in}{3.696000in}}%
\pgfusepath{clip}%
\pgfsetbuttcap%
\pgfsetroundjoin%
\definecolor{currentfill}{rgb}{0.121569,0.466667,0.705882}%
\pgfsetfillcolor{currentfill}%
\pgfsetfillopacity{0.300458}%
\pgfsetlinewidth{1.003750pt}%
\definecolor{currentstroke}{rgb}{0.121569,0.466667,0.705882}%
\pgfsetstrokecolor{currentstroke}%
\pgfsetstrokeopacity{0.300458}%
\pgfsetdash{}{0pt}%
\pgfpathmoveto{\pgfqpoint{1.137841in}{1.617366in}}%
\pgfpathcurveto{\pgfqpoint{1.146077in}{1.617366in}}{\pgfqpoint{1.153977in}{1.620638in}}{\pgfqpoint{1.159801in}{1.626462in}}%
\pgfpathcurveto{\pgfqpoint{1.165625in}{1.632286in}}{\pgfqpoint{1.168897in}{1.640186in}}{\pgfqpoint{1.168897in}{1.648422in}}%
\pgfpathcurveto{\pgfqpoint{1.168897in}{1.656658in}}{\pgfqpoint{1.165625in}{1.664558in}}{\pgfqpoint{1.159801in}{1.670382in}}%
\pgfpathcurveto{\pgfqpoint{1.153977in}{1.676206in}}{\pgfqpoint{1.146077in}{1.679479in}}{\pgfqpoint{1.137841in}{1.679479in}}%
\pgfpathcurveto{\pgfqpoint{1.129604in}{1.679479in}}{\pgfqpoint{1.121704in}{1.676206in}}{\pgfqpoint{1.115880in}{1.670382in}}%
\pgfpathcurveto{\pgfqpoint{1.110056in}{1.664558in}}{\pgfqpoint{1.106784in}{1.656658in}}{\pgfqpoint{1.106784in}{1.648422in}}%
\pgfpathcurveto{\pgfqpoint{1.106784in}{1.640186in}}{\pgfqpoint{1.110056in}{1.632286in}}{\pgfqpoint{1.115880in}{1.626462in}}%
\pgfpathcurveto{\pgfqpoint{1.121704in}{1.620638in}}{\pgfqpoint{1.129604in}{1.617366in}}{\pgfqpoint{1.137841in}{1.617366in}}%
\pgfpathclose%
\pgfusepath{stroke,fill}%
\end{pgfscope}%
\begin{pgfscope}%
\pgfpathrectangle{\pgfqpoint{0.100000in}{0.212622in}}{\pgfqpoint{3.696000in}{3.696000in}}%
\pgfusepath{clip}%
\pgfsetbuttcap%
\pgfsetroundjoin%
\definecolor{currentfill}{rgb}{0.121569,0.466667,0.705882}%
\pgfsetfillcolor{currentfill}%
\pgfsetfillopacity{0.300458}%
\pgfsetlinewidth{1.003750pt}%
\definecolor{currentstroke}{rgb}{0.121569,0.466667,0.705882}%
\pgfsetstrokecolor{currentstroke}%
\pgfsetstrokeopacity{0.300458}%
\pgfsetdash{}{0pt}%
\pgfpathmoveto{\pgfqpoint{1.137841in}{1.617366in}}%
\pgfpathcurveto{\pgfqpoint{1.146077in}{1.617366in}}{\pgfqpoint{1.153977in}{1.620638in}}{\pgfqpoint{1.159801in}{1.626462in}}%
\pgfpathcurveto{\pgfqpoint{1.165625in}{1.632286in}}{\pgfqpoint{1.168897in}{1.640186in}}{\pgfqpoint{1.168897in}{1.648422in}}%
\pgfpathcurveto{\pgfqpoint{1.168897in}{1.656658in}}{\pgfqpoint{1.165625in}{1.664558in}}{\pgfqpoint{1.159801in}{1.670382in}}%
\pgfpathcurveto{\pgfqpoint{1.153977in}{1.676206in}}{\pgfqpoint{1.146077in}{1.679479in}}{\pgfqpoint{1.137841in}{1.679479in}}%
\pgfpathcurveto{\pgfqpoint{1.129604in}{1.679479in}}{\pgfqpoint{1.121704in}{1.676206in}}{\pgfqpoint{1.115880in}{1.670382in}}%
\pgfpathcurveto{\pgfqpoint{1.110056in}{1.664558in}}{\pgfqpoint{1.106784in}{1.656658in}}{\pgfqpoint{1.106784in}{1.648422in}}%
\pgfpathcurveto{\pgfqpoint{1.106784in}{1.640186in}}{\pgfqpoint{1.110056in}{1.632286in}}{\pgfqpoint{1.115880in}{1.626462in}}%
\pgfpathcurveto{\pgfqpoint{1.121704in}{1.620638in}}{\pgfqpoint{1.129604in}{1.617366in}}{\pgfqpoint{1.137841in}{1.617366in}}%
\pgfpathclose%
\pgfusepath{stroke,fill}%
\end{pgfscope}%
\begin{pgfscope}%
\pgfpathrectangle{\pgfqpoint{0.100000in}{0.212622in}}{\pgfqpoint{3.696000in}{3.696000in}}%
\pgfusepath{clip}%
\pgfsetbuttcap%
\pgfsetroundjoin%
\definecolor{currentfill}{rgb}{0.121569,0.466667,0.705882}%
\pgfsetfillcolor{currentfill}%
\pgfsetfillopacity{0.300458}%
\pgfsetlinewidth{1.003750pt}%
\definecolor{currentstroke}{rgb}{0.121569,0.466667,0.705882}%
\pgfsetstrokecolor{currentstroke}%
\pgfsetstrokeopacity{0.300458}%
\pgfsetdash{}{0pt}%
\pgfpathmoveto{\pgfqpoint{1.137841in}{1.617366in}}%
\pgfpathcurveto{\pgfqpoint{1.146077in}{1.617366in}}{\pgfqpoint{1.153977in}{1.620638in}}{\pgfqpoint{1.159801in}{1.626462in}}%
\pgfpathcurveto{\pgfqpoint{1.165625in}{1.632286in}}{\pgfqpoint{1.168897in}{1.640186in}}{\pgfqpoint{1.168897in}{1.648422in}}%
\pgfpathcurveto{\pgfqpoint{1.168897in}{1.656658in}}{\pgfqpoint{1.165625in}{1.664558in}}{\pgfqpoint{1.159801in}{1.670382in}}%
\pgfpathcurveto{\pgfqpoint{1.153977in}{1.676206in}}{\pgfqpoint{1.146077in}{1.679479in}}{\pgfqpoint{1.137841in}{1.679479in}}%
\pgfpathcurveto{\pgfqpoint{1.129604in}{1.679479in}}{\pgfqpoint{1.121704in}{1.676206in}}{\pgfqpoint{1.115880in}{1.670382in}}%
\pgfpathcurveto{\pgfqpoint{1.110056in}{1.664558in}}{\pgfqpoint{1.106784in}{1.656658in}}{\pgfqpoint{1.106784in}{1.648422in}}%
\pgfpathcurveto{\pgfqpoint{1.106784in}{1.640186in}}{\pgfqpoint{1.110056in}{1.632286in}}{\pgfqpoint{1.115880in}{1.626462in}}%
\pgfpathcurveto{\pgfqpoint{1.121704in}{1.620638in}}{\pgfqpoint{1.129604in}{1.617366in}}{\pgfqpoint{1.137841in}{1.617366in}}%
\pgfpathclose%
\pgfusepath{stroke,fill}%
\end{pgfscope}%
\begin{pgfscope}%
\pgfpathrectangle{\pgfqpoint{0.100000in}{0.212622in}}{\pgfqpoint{3.696000in}{3.696000in}}%
\pgfusepath{clip}%
\pgfsetbuttcap%
\pgfsetroundjoin%
\definecolor{currentfill}{rgb}{0.121569,0.466667,0.705882}%
\pgfsetfillcolor{currentfill}%
\pgfsetfillopacity{0.300458}%
\pgfsetlinewidth{1.003750pt}%
\definecolor{currentstroke}{rgb}{0.121569,0.466667,0.705882}%
\pgfsetstrokecolor{currentstroke}%
\pgfsetstrokeopacity{0.300458}%
\pgfsetdash{}{0pt}%
\pgfpathmoveto{\pgfqpoint{1.137841in}{1.617366in}}%
\pgfpathcurveto{\pgfqpoint{1.146077in}{1.617366in}}{\pgfqpoint{1.153977in}{1.620638in}}{\pgfqpoint{1.159801in}{1.626462in}}%
\pgfpathcurveto{\pgfqpoint{1.165625in}{1.632286in}}{\pgfqpoint{1.168897in}{1.640186in}}{\pgfqpoint{1.168897in}{1.648422in}}%
\pgfpathcurveto{\pgfqpoint{1.168897in}{1.656658in}}{\pgfqpoint{1.165625in}{1.664558in}}{\pgfqpoint{1.159801in}{1.670382in}}%
\pgfpathcurveto{\pgfqpoint{1.153977in}{1.676206in}}{\pgfqpoint{1.146077in}{1.679479in}}{\pgfqpoint{1.137841in}{1.679479in}}%
\pgfpathcurveto{\pgfqpoint{1.129604in}{1.679479in}}{\pgfqpoint{1.121704in}{1.676206in}}{\pgfqpoint{1.115880in}{1.670382in}}%
\pgfpathcurveto{\pgfqpoint{1.110056in}{1.664558in}}{\pgfqpoint{1.106784in}{1.656658in}}{\pgfqpoint{1.106784in}{1.648422in}}%
\pgfpathcurveto{\pgfqpoint{1.106784in}{1.640186in}}{\pgfqpoint{1.110056in}{1.632286in}}{\pgfqpoint{1.115880in}{1.626462in}}%
\pgfpathcurveto{\pgfqpoint{1.121704in}{1.620638in}}{\pgfqpoint{1.129604in}{1.617366in}}{\pgfqpoint{1.137841in}{1.617366in}}%
\pgfpathclose%
\pgfusepath{stroke,fill}%
\end{pgfscope}%
\begin{pgfscope}%
\pgfpathrectangle{\pgfqpoint{0.100000in}{0.212622in}}{\pgfqpoint{3.696000in}{3.696000in}}%
\pgfusepath{clip}%
\pgfsetbuttcap%
\pgfsetroundjoin%
\definecolor{currentfill}{rgb}{0.121569,0.466667,0.705882}%
\pgfsetfillcolor{currentfill}%
\pgfsetfillopacity{0.300458}%
\pgfsetlinewidth{1.003750pt}%
\definecolor{currentstroke}{rgb}{0.121569,0.466667,0.705882}%
\pgfsetstrokecolor{currentstroke}%
\pgfsetstrokeopacity{0.300458}%
\pgfsetdash{}{0pt}%
\pgfpathmoveto{\pgfqpoint{1.137841in}{1.617366in}}%
\pgfpathcurveto{\pgfqpoint{1.146077in}{1.617366in}}{\pgfqpoint{1.153977in}{1.620638in}}{\pgfqpoint{1.159801in}{1.626462in}}%
\pgfpathcurveto{\pgfqpoint{1.165625in}{1.632286in}}{\pgfqpoint{1.168897in}{1.640186in}}{\pgfqpoint{1.168897in}{1.648422in}}%
\pgfpathcurveto{\pgfqpoint{1.168897in}{1.656658in}}{\pgfqpoint{1.165625in}{1.664558in}}{\pgfqpoint{1.159801in}{1.670382in}}%
\pgfpathcurveto{\pgfqpoint{1.153977in}{1.676206in}}{\pgfqpoint{1.146077in}{1.679479in}}{\pgfqpoint{1.137841in}{1.679479in}}%
\pgfpathcurveto{\pgfqpoint{1.129604in}{1.679479in}}{\pgfqpoint{1.121704in}{1.676206in}}{\pgfqpoint{1.115880in}{1.670382in}}%
\pgfpathcurveto{\pgfqpoint{1.110056in}{1.664558in}}{\pgfqpoint{1.106784in}{1.656658in}}{\pgfqpoint{1.106784in}{1.648422in}}%
\pgfpathcurveto{\pgfqpoint{1.106784in}{1.640186in}}{\pgfqpoint{1.110056in}{1.632286in}}{\pgfqpoint{1.115880in}{1.626462in}}%
\pgfpathcurveto{\pgfqpoint{1.121704in}{1.620638in}}{\pgfqpoint{1.129604in}{1.617366in}}{\pgfqpoint{1.137841in}{1.617366in}}%
\pgfpathclose%
\pgfusepath{stroke,fill}%
\end{pgfscope}%
\begin{pgfscope}%
\pgfpathrectangle{\pgfqpoint{0.100000in}{0.212622in}}{\pgfqpoint{3.696000in}{3.696000in}}%
\pgfusepath{clip}%
\pgfsetbuttcap%
\pgfsetroundjoin%
\definecolor{currentfill}{rgb}{0.121569,0.466667,0.705882}%
\pgfsetfillcolor{currentfill}%
\pgfsetfillopacity{0.300458}%
\pgfsetlinewidth{1.003750pt}%
\definecolor{currentstroke}{rgb}{0.121569,0.466667,0.705882}%
\pgfsetstrokecolor{currentstroke}%
\pgfsetstrokeopacity{0.300458}%
\pgfsetdash{}{0pt}%
\pgfpathmoveto{\pgfqpoint{1.137841in}{1.617366in}}%
\pgfpathcurveto{\pgfqpoint{1.146077in}{1.617366in}}{\pgfqpoint{1.153977in}{1.620638in}}{\pgfqpoint{1.159801in}{1.626462in}}%
\pgfpathcurveto{\pgfqpoint{1.165625in}{1.632286in}}{\pgfqpoint{1.168897in}{1.640186in}}{\pgfqpoint{1.168897in}{1.648422in}}%
\pgfpathcurveto{\pgfqpoint{1.168897in}{1.656658in}}{\pgfqpoint{1.165625in}{1.664558in}}{\pgfqpoint{1.159801in}{1.670382in}}%
\pgfpathcurveto{\pgfqpoint{1.153977in}{1.676206in}}{\pgfqpoint{1.146077in}{1.679479in}}{\pgfqpoint{1.137841in}{1.679479in}}%
\pgfpathcurveto{\pgfqpoint{1.129604in}{1.679479in}}{\pgfqpoint{1.121704in}{1.676206in}}{\pgfqpoint{1.115880in}{1.670382in}}%
\pgfpathcurveto{\pgfqpoint{1.110056in}{1.664558in}}{\pgfqpoint{1.106784in}{1.656658in}}{\pgfqpoint{1.106784in}{1.648422in}}%
\pgfpathcurveto{\pgfqpoint{1.106784in}{1.640186in}}{\pgfqpoint{1.110056in}{1.632286in}}{\pgfqpoint{1.115880in}{1.626462in}}%
\pgfpathcurveto{\pgfqpoint{1.121704in}{1.620638in}}{\pgfqpoint{1.129604in}{1.617366in}}{\pgfqpoint{1.137841in}{1.617366in}}%
\pgfpathclose%
\pgfusepath{stroke,fill}%
\end{pgfscope}%
\begin{pgfscope}%
\pgfpathrectangle{\pgfqpoint{0.100000in}{0.212622in}}{\pgfqpoint{3.696000in}{3.696000in}}%
\pgfusepath{clip}%
\pgfsetbuttcap%
\pgfsetroundjoin%
\definecolor{currentfill}{rgb}{0.121569,0.466667,0.705882}%
\pgfsetfillcolor{currentfill}%
\pgfsetfillopacity{0.300458}%
\pgfsetlinewidth{1.003750pt}%
\definecolor{currentstroke}{rgb}{0.121569,0.466667,0.705882}%
\pgfsetstrokecolor{currentstroke}%
\pgfsetstrokeopacity{0.300458}%
\pgfsetdash{}{0pt}%
\pgfpathmoveto{\pgfqpoint{1.137841in}{1.617366in}}%
\pgfpathcurveto{\pgfqpoint{1.146077in}{1.617366in}}{\pgfqpoint{1.153977in}{1.620638in}}{\pgfqpoint{1.159801in}{1.626462in}}%
\pgfpathcurveto{\pgfqpoint{1.165625in}{1.632286in}}{\pgfqpoint{1.168897in}{1.640186in}}{\pgfqpoint{1.168897in}{1.648422in}}%
\pgfpathcurveto{\pgfqpoint{1.168897in}{1.656658in}}{\pgfqpoint{1.165625in}{1.664558in}}{\pgfqpoint{1.159801in}{1.670382in}}%
\pgfpathcurveto{\pgfqpoint{1.153977in}{1.676206in}}{\pgfqpoint{1.146077in}{1.679479in}}{\pgfqpoint{1.137841in}{1.679479in}}%
\pgfpathcurveto{\pgfqpoint{1.129604in}{1.679479in}}{\pgfqpoint{1.121704in}{1.676206in}}{\pgfqpoint{1.115880in}{1.670382in}}%
\pgfpathcurveto{\pgfqpoint{1.110056in}{1.664558in}}{\pgfqpoint{1.106784in}{1.656658in}}{\pgfqpoint{1.106784in}{1.648422in}}%
\pgfpathcurveto{\pgfqpoint{1.106784in}{1.640186in}}{\pgfqpoint{1.110056in}{1.632286in}}{\pgfqpoint{1.115880in}{1.626462in}}%
\pgfpathcurveto{\pgfqpoint{1.121704in}{1.620638in}}{\pgfqpoint{1.129604in}{1.617366in}}{\pgfqpoint{1.137841in}{1.617366in}}%
\pgfpathclose%
\pgfusepath{stroke,fill}%
\end{pgfscope}%
\begin{pgfscope}%
\pgfpathrectangle{\pgfqpoint{0.100000in}{0.212622in}}{\pgfqpoint{3.696000in}{3.696000in}}%
\pgfusepath{clip}%
\pgfsetbuttcap%
\pgfsetroundjoin%
\definecolor{currentfill}{rgb}{0.121569,0.466667,0.705882}%
\pgfsetfillcolor{currentfill}%
\pgfsetfillopacity{0.300458}%
\pgfsetlinewidth{1.003750pt}%
\definecolor{currentstroke}{rgb}{0.121569,0.466667,0.705882}%
\pgfsetstrokecolor{currentstroke}%
\pgfsetstrokeopacity{0.300458}%
\pgfsetdash{}{0pt}%
\pgfpathmoveto{\pgfqpoint{1.137841in}{1.617366in}}%
\pgfpathcurveto{\pgfqpoint{1.146077in}{1.617366in}}{\pgfqpoint{1.153977in}{1.620638in}}{\pgfqpoint{1.159801in}{1.626462in}}%
\pgfpathcurveto{\pgfqpoint{1.165625in}{1.632286in}}{\pgfqpoint{1.168897in}{1.640186in}}{\pgfqpoint{1.168897in}{1.648422in}}%
\pgfpathcurveto{\pgfqpoint{1.168897in}{1.656658in}}{\pgfqpoint{1.165625in}{1.664558in}}{\pgfqpoint{1.159801in}{1.670382in}}%
\pgfpathcurveto{\pgfqpoint{1.153977in}{1.676206in}}{\pgfqpoint{1.146077in}{1.679479in}}{\pgfqpoint{1.137841in}{1.679479in}}%
\pgfpathcurveto{\pgfqpoint{1.129604in}{1.679479in}}{\pgfqpoint{1.121704in}{1.676206in}}{\pgfqpoint{1.115880in}{1.670382in}}%
\pgfpathcurveto{\pgfqpoint{1.110056in}{1.664558in}}{\pgfqpoint{1.106784in}{1.656658in}}{\pgfqpoint{1.106784in}{1.648422in}}%
\pgfpathcurveto{\pgfqpoint{1.106784in}{1.640186in}}{\pgfqpoint{1.110056in}{1.632286in}}{\pgfqpoint{1.115880in}{1.626462in}}%
\pgfpathcurveto{\pgfqpoint{1.121704in}{1.620638in}}{\pgfqpoint{1.129604in}{1.617366in}}{\pgfqpoint{1.137841in}{1.617366in}}%
\pgfpathclose%
\pgfusepath{stroke,fill}%
\end{pgfscope}%
\begin{pgfscope}%
\pgfpathrectangle{\pgfqpoint{0.100000in}{0.212622in}}{\pgfqpoint{3.696000in}{3.696000in}}%
\pgfusepath{clip}%
\pgfsetbuttcap%
\pgfsetroundjoin%
\definecolor{currentfill}{rgb}{0.121569,0.466667,0.705882}%
\pgfsetfillcolor{currentfill}%
\pgfsetfillopacity{0.300458}%
\pgfsetlinewidth{1.003750pt}%
\definecolor{currentstroke}{rgb}{0.121569,0.466667,0.705882}%
\pgfsetstrokecolor{currentstroke}%
\pgfsetstrokeopacity{0.300458}%
\pgfsetdash{}{0pt}%
\pgfpathmoveto{\pgfqpoint{1.137841in}{1.617366in}}%
\pgfpathcurveto{\pgfqpoint{1.146077in}{1.617366in}}{\pgfqpoint{1.153977in}{1.620638in}}{\pgfqpoint{1.159801in}{1.626462in}}%
\pgfpathcurveto{\pgfqpoint{1.165625in}{1.632286in}}{\pgfqpoint{1.168897in}{1.640186in}}{\pgfqpoint{1.168897in}{1.648422in}}%
\pgfpathcurveto{\pgfqpoint{1.168897in}{1.656658in}}{\pgfqpoint{1.165625in}{1.664558in}}{\pgfqpoint{1.159801in}{1.670382in}}%
\pgfpathcurveto{\pgfqpoint{1.153977in}{1.676206in}}{\pgfqpoint{1.146077in}{1.679479in}}{\pgfqpoint{1.137841in}{1.679479in}}%
\pgfpathcurveto{\pgfqpoint{1.129604in}{1.679479in}}{\pgfqpoint{1.121704in}{1.676206in}}{\pgfqpoint{1.115880in}{1.670382in}}%
\pgfpathcurveto{\pgfqpoint{1.110056in}{1.664558in}}{\pgfqpoint{1.106784in}{1.656658in}}{\pgfqpoint{1.106784in}{1.648422in}}%
\pgfpathcurveto{\pgfqpoint{1.106784in}{1.640186in}}{\pgfqpoint{1.110056in}{1.632286in}}{\pgfqpoint{1.115880in}{1.626462in}}%
\pgfpathcurveto{\pgfqpoint{1.121704in}{1.620638in}}{\pgfqpoint{1.129604in}{1.617366in}}{\pgfqpoint{1.137841in}{1.617366in}}%
\pgfpathclose%
\pgfusepath{stroke,fill}%
\end{pgfscope}%
\begin{pgfscope}%
\pgfpathrectangle{\pgfqpoint{0.100000in}{0.212622in}}{\pgfqpoint{3.696000in}{3.696000in}}%
\pgfusepath{clip}%
\pgfsetbuttcap%
\pgfsetroundjoin%
\definecolor{currentfill}{rgb}{0.121569,0.466667,0.705882}%
\pgfsetfillcolor{currentfill}%
\pgfsetfillopacity{0.300458}%
\pgfsetlinewidth{1.003750pt}%
\definecolor{currentstroke}{rgb}{0.121569,0.466667,0.705882}%
\pgfsetstrokecolor{currentstroke}%
\pgfsetstrokeopacity{0.300458}%
\pgfsetdash{}{0pt}%
\pgfpathmoveto{\pgfqpoint{1.137841in}{1.617366in}}%
\pgfpathcurveto{\pgfqpoint{1.146077in}{1.617366in}}{\pgfqpoint{1.153977in}{1.620638in}}{\pgfqpoint{1.159801in}{1.626462in}}%
\pgfpathcurveto{\pgfqpoint{1.165625in}{1.632286in}}{\pgfqpoint{1.168897in}{1.640186in}}{\pgfqpoint{1.168897in}{1.648422in}}%
\pgfpathcurveto{\pgfqpoint{1.168897in}{1.656658in}}{\pgfqpoint{1.165625in}{1.664558in}}{\pgfqpoint{1.159801in}{1.670382in}}%
\pgfpathcurveto{\pgfqpoint{1.153977in}{1.676206in}}{\pgfqpoint{1.146077in}{1.679479in}}{\pgfqpoint{1.137841in}{1.679479in}}%
\pgfpathcurveto{\pgfqpoint{1.129604in}{1.679479in}}{\pgfqpoint{1.121704in}{1.676206in}}{\pgfqpoint{1.115880in}{1.670382in}}%
\pgfpathcurveto{\pgfqpoint{1.110056in}{1.664558in}}{\pgfqpoint{1.106784in}{1.656658in}}{\pgfqpoint{1.106784in}{1.648422in}}%
\pgfpathcurveto{\pgfqpoint{1.106784in}{1.640186in}}{\pgfqpoint{1.110056in}{1.632286in}}{\pgfqpoint{1.115880in}{1.626462in}}%
\pgfpathcurveto{\pgfqpoint{1.121704in}{1.620638in}}{\pgfqpoint{1.129604in}{1.617366in}}{\pgfqpoint{1.137841in}{1.617366in}}%
\pgfpathclose%
\pgfusepath{stroke,fill}%
\end{pgfscope}%
\begin{pgfscope}%
\pgfpathrectangle{\pgfqpoint{0.100000in}{0.212622in}}{\pgfqpoint{3.696000in}{3.696000in}}%
\pgfusepath{clip}%
\pgfsetbuttcap%
\pgfsetroundjoin%
\definecolor{currentfill}{rgb}{0.121569,0.466667,0.705882}%
\pgfsetfillcolor{currentfill}%
\pgfsetfillopacity{0.300458}%
\pgfsetlinewidth{1.003750pt}%
\definecolor{currentstroke}{rgb}{0.121569,0.466667,0.705882}%
\pgfsetstrokecolor{currentstroke}%
\pgfsetstrokeopacity{0.300458}%
\pgfsetdash{}{0pt}%
\pgfpathmoveto{\pgfqpoint{1.137841in}{1.617366in}}%
\pgfpathcurveto{\pgfqpoint{1.146077in}{1.617366in}}{\pgfqpoint{1.153977in}{1.620638in}}{\pgfqpoint{1.159801in}{1.626462in}}%
\pgfpathcurveto{\pgfqpoint{1.165625in}{1.632286in}}{\pgfqpoint{1.168897in}{1.640186in}}{\pgfqpoint{1.168897in}{1.648422in}}%
\pgfpathcurveto{\pgfqpoint{1.168897in}{1.656658in}}{\pgfqpoint{1.165625in}{1.664558in}}{\pgfqpoint{1.159801in}{1.670382in}}%
\pgfpathcurveto{\pgfqpoint{1.153977in}{1.676206in}}{\pgfqpoint{1.146077in}{1.679479in}}{\pgfqpoint{1.137841in}{1.679479in}}%
\pgfpathcurveto{\pgfqpoint{1.129604in}{1.679479in}}{\pgfqpoint{1.121704in}{1.676206in}}{\pgfqpoint{1.115880in}{1.670382in}}%
\pgfpathcurveto{\pgfqpoint{1.110056in}{1.664558in}}{\pgfqpoint{1.106784in}{1.656658in}}{\pgfqpoint{1.106784in}{1.648422in}}%
\pgfpathcurveto{\pgfqpoint{1.106784in}{1.640186in}}{\pgfqpoint{1.110056in}{1.632286in}}{\pgfqpoint{1.115880in}{1.626462in}}%
\pgfpathcurveto{\pgfqpoint{1.121704in}{1.620638in}}{\pgfqpoint{1.129604in}{1.617366in}}{\pgfqpoint{1.137841in}{1.617366in}}%
\pgfpathclose%
\pgfusepath{stroke,fill}%
\end{pgfscope}%
\begin{pgfscope}%
\pgfpathrectangle{\pgfqpoint{0.100000in}{0.212622in}}{\pgfqpoint{3.696000in}{3.696000in}}%
\pgfusepath{clip}%
\pgfsetbuttcap%
\pgfsetroundjoin%
\definecolor{currentfill}{rgb}{0.121569,0.466667,0.705882}%
\pgfsetfillcolor{currentfill}%
\pgfsetfillopacity{0.300458}%
\pgfsetlinewidth{1.003750pt}%
\definecolor{currentstroke}{rgb}{0.121569,0.466667,0.705882}%
\pgfsetstrokecolor{currentstroke}%
\pgfsetstrokeopacity{0.300458}%
\pgfsetdash{}{0pt}%
\pgfpathmoveto{\pgfqpoint{1.137841in}{1.617366in}}%
\pgfpathcurveto{\pgfqpoint{1.146077in}{1.617366in}}{\pgfqpoint{1.153977in}{1.620638in}}{\pgfqpoint{1.159801in}{1.626462in}}%
\pgfpathcurveto{\pgfqpoint{1.165625in}{1.632286in}}{\pgfqpoint{1.168897in}{1.640186in}}{\pgfqpoint{1.168897in}{1.648422in}}%
\pgfpathcurveto{\pgfqpoint{1.168897in}{1.656658in}}{\pgfqpoint{1.165625in}{1.664558in}}{\pgfqpoint{1.159801in}{1.670382in}}%
\pgfpathcurveto{\pgfqpoint{1.153977in}{1.676206in}}{\pgfqpoint{1.146077in}{1.679479in}}{\pgfqpoint{1.137841in}{1.679479in}}%
\pgfpathcurveto{\pgfqpoint{1.129604in}{1.679479in}}{\pgfqpoint{1.121704in}{1.676206in}}{\pgfqpoint{1.115880in}{1.670382in}}%
\pgfpathcurveto{\pgfqpoint{1.110056in}{1.664558in}}{\pgfqpoint{1.106784in}{1.656658in}}{\pgfqpoint{1.106784in}{1.648422in}}%
\pgfpathcurveto{\pgfqpoint{1.106784in}{1.640186in}}{\pgfqpoint{1.110056in}{1.632286in}}{\pgfqpoint{1.115880in}{1.626462in}}%
\pgfpathcurveto{\pgfqpoint{1.121704in}{1.620638in}}{\pgfqpoint{1.129604in}{1.617366in}}{\pgfqpoint{1.137841in}{1.617366in}}%
\pgfpathclose%
\pgfusepath{stroke,fill}%
\end{pgfscope}%
\begin{pgfscope}%
\pgfpathrectangle{\pgfqpoint{0.100000in}{0.212622in}}{\pgfqpoint{3.696000in}{3.696000in}}%
\pgfusepath{clip}%
\pgfsetbuttcap%
\pgfsetroundjoin%
\definecolor{currentfill}{rgb}{0.121569,0.466667,0.705882}%
\pgfsetfillcolor{currentfill}%
\pgfsetfillopacity{0.300458}%
\pgfsetlinewidth{1.003750pt}%
\definecolor{currentstroke}{rgb}{0.121569,0.466667,0.705882}%
\pgfsetstrokecolor{currentstroke}%
\pgfsetstrokeopacity{0.300458}%
\pgfsetdash{}{0pt}%
\pgfpathmoveto{\pgfqpoint{1.137841in}{1.617366in}}%
\pgfpathcurveto{\pgfqpoint{1.146077in}{1.617366in}}{\pgfqpoint{1.153977in}{1.620638in}}{\pgfqpoint{1.159801in}{1.626462in}}%
\pgfpathcurveto{\pgfqpoint{1.165625in}{1.632286in}}{\pgfqpoint{1.168897in}{1.640186in}}{\pgfqpoint{1.168897in}{1.648422in}}%
\pgfpathcurveto{\pgfqpoint{1.168897in}{1.656658in}}{\pgfqpoint{1.165625in}{1.664558in}}{\pgfqpoint{1.159801in}{1.670382in}}%
\pgfpathcurveto{\pgfqpoint{1.153977in}{1.676206in}}{\pgfqpoint{1.146077in}{1.679479in}}{\pgfqpoint{1.137841in}{1.679479in}}%
\pgfpathcurveto{\pgfqpoint{1.129604in}{1.679479in}}{\pgfqpoint{1.121704in}{1.676206in}}{\pgfqpoint{1.115880in}{1.670382in}}%
\pgfpathcurveto{\pgfqpoint{1.110056in}{1.664558in}}{\pgfqpoint{1.106784in}{1.656658in}}{\pgfqpoint{1.106784in}{1.648422in}}%
\pgfpathcurveto{\pgfqpoint{1.106784in}{1.640186in}}{\pgfqpoint{1.110056in}{1.632286in}}{\pgfqpoint{1.115880in}{1.626462in}}%
\pgfpathcurveto{\pgfqpoint{1.121704in}{1.620638in}}{\pgfqpoint{1.129604in}{1.617366in}}{\pgfqpoint{1.137841in}{1.617366in}}%
\pgfpathclose%
\pgfusepath{stroke,fill}%
\end{pgfscope}%
\begin{pgfscope}%
\pgfpathrectangle{\pgfqpoint{0.100000in}{0.212622in}}{\pgfqpoint{3.696000in}{3.696000in}}%
\pgfusepath{clip}%
\pgfsetbuttcap%
\pgfsetroundjoin%
\definecolor{currentfill}{rgb}{0.121569,0.466667,0.705882}%
\pgfsetfillcolor{currentfill}%
\pgfsetfillopacity{0.300458}%
\pgfsetlinewidth{1.003750pt}%
\definecolor{currentstroke}{rgb}{0.121569,0.466667,0.705882}%
\pgfsetstrokecolor{currentstroke}%
\pgfsetstrokeopacity{0.300458}%
\pgfsetdash{}{0pt}%
\pgfpathmoveto{\pgfqpoint{1.137841in}{1.617366in}}%
\pgfpathcurveto{\pgfqpoint{1.146077in}{1.617366in}}{\pgfqpoint{1.153977in}{1.620638in}}{\pgfqpoint{1.159801in}{1.626462in}}%
\pgfpathcurveto{\pgfqpoint{1.165625in}{1.632286in}}{\pgfqpoint{1.168897in}{1.640186in}}{\pgfqpoint{1.168897in}{1.648422in}}%
\pgfpathcurveto{\pgfqpoint{1.168897in}{1.656658in}}{\pgfqpoint{1.165625in}{1.664558in}}{\pgfqpoint{1.159801in}{1.670382in}}%
\pgfpathcurveto{\pgfqpoint{1.153977in}{1.676206in}}{\pgfqpoint{1.146077in}{1.679479in}}{\pgfqpoint{1.137841in}{1.679479in}}%
\pgfpathcurveto{\pgfqpoint{1.129604in}{1.679479in}}{\pgfqpoint{1.121704in}{1.676206in}}{\pgfqpoint{1.115880in}{1.670382in}}%
\pgfpathcurveto{\pgfqpoint{1.110056in}{1.664558in}}{\pgfqpoint{1.106784in}{1.656658in}}{\pgfqpoint{1.106784in}{1.648422in}}%
\pgfpathcurveto{\pgfqpoint{1.106784in}{1.640186in}}{\pgfqpoint{1.110056in}{1.632286in}}{\pgfqpoint{1.115880in}{1.626462in}}%
\pgfpathcurveto{\pgfqpoint{1.121704in}{1.620638in}}{\pgfqpoint{1.129604in}{1.617366in}}{\pgfqpoint{1.137841in}{1.617366in}}%
\pgfpathclose%
\pgfusepath{stroke,fill}%
\end{pgfscope}%
\begin{pgfscope}%
\pgfpathrectangle{\pgfqpoint{0.100000in}{0.212622in}}{\pgfqpoint{3.696000in}{3.696000in}}%
\pgfusepath{clip}%
\pgfsetbuttcap%
\pgfsetroundjoin%
\definecolor{currentfill}{rgb}{0.121569,0.466667,0.705882}%
\pgfsetfillcolor{currentfill}%
\pgfsetfillopacity{0.300606}%
\pgfsetlinewidth{1.003750pt}%
\definecolor{currentstroke}{rgb}{0.121569,0.466667,0.705882}%
\pgfsetstrokecolor{currentstroke}%
\pgfsetstrokeopacity{0.300606}%
\pgfsetdash{}{0pt}%
\pgfpathmoveto{\pgfqpoint{1.138330in}{1.617240in}}%
\pgfpathcurveto{\pgfqpoint{1.146566in}{1.617240in}}{\pgfqpoint{1.154466in}{1.620513in}}{\pgfqpoint{1.160290in}{1.626336in}}%
\pgfpathcurveto{\pgfqpoint{1.166114in}{1.632160in}}{\pgfqpoint{1.169387in}{1.640060in}}{\pgfqpoint{1.169387in}{1.648297in}}%
\pgfpathcurveto{\pgfqpoint{1.169387in}{1.656533in}}{\pgfqpoint{1.166114in}{1.664433in}}{\pgfqpoint{1.160290in}{1.670257in}}%
\pgfpathcurveto{\pgfqpoint{1.154466in}{1.676081in}}{\pgfqpoint{1.146566in}{1.679353in}}{\pgfqpoint{1.138330in}{1.679353in}}%
\pgfpathcurveto{\pgfqpoint{1.130094in}{1.679353in}}{\pgfqpoint{1.122194in}{1.676081in}}{\pgfqpoint{1.116370in}{1.670257in}}%
\pgfpathcurveto{\pgfqpoint{1.110546in}{1.664433in}}{\pgfqpoint{1.107274in}{1.656533in}}{\pgfqpoint{1.107274in}{1.648297in}}%
\pgfpathcurveto{\pgfqpoint{1.107274in}{1.640060in}}{\pgfqpoint{1.110546in}{1.632160in}}{\pgfqpoint{1.116370in}{1.626336in}}%
\pgfpathcurveto{\pgfqpoint{1.122194in}{1.620513in}}{\pgfqpoint{1.130094in}{1.617240in}}{\pgfqpoint{1.138330in}{1.617240in}}%
\pgfpathclose%
\pgfusepath{stroke,fill}%
\end{pgfscope}%
\begin{pgfscope}%
\pgfpathrectangle{\pgfqpoint{0.100000in}{0.212622in}}{\pgfqpoint{3.696000in}{3.696000in}}%
\pgfusepath{clip}%
\pgfsetbuttcap%
\pgfsetroundjoin%
\definecolor{currentfill}{rgb}{0.121569,0.466667,0.705882}%
\pgfsetfillcolor{currentfill}%
\pgfsetfillopacity{0.300695}%
\pgfsetlinewidth{1.003750pt}%
\definecolor{currentstroke}{rgb}{0.121569,0.466667,0.705882}%
\pgfsetstrokecolor{currentstroke}%
\pgfsetstrokeopacity{0.300695}%
\pgfsetdash{}{0pt}%
\pgfpathmoveto{\pgfqpoint{1.138592in}{1.617165in}}%
\pgfpathcurveto{\pgfqpoint{1.146828in}{1.617165in}}{\pgfqpoint{1.154728in}{1.620438in}}{\pgfqpoint{1.160552in}{1.626262in}}%
\pgfpathcurveto{\pgfqpoint{1.166376in}{1.632085in}}{\pgfqpoint{1.169649in}{1.639986in}}{\pgfqpoint{1.169649in}{1.648222in}}%
\pgfpathcurveto{\pgfqpoint{1.169649in}{1.656458in}}{\pgfqpoint{1.166376in}{1.664358in}}{\pgfqpoint{1.160552in}{1.670182in}}%
\pgfpathcurveto{\pgfqpoint{1.154728in}{1.676006in}}{\pgfqpoint{1.146828in}{1.679278in}}{\pgfqpoint{1.138592in}{1.679278in}}%
\pgfpathcurveto{\pgfqpoint{1.130356in}{1.679278in}}{\pgfqpoint{1.122456in}{1.676006in}}{\pgfqpoint{1.116632in}{1.670182in}}%
\pgfpathcurveto{\pgfqpoint{1.110808in}{1.664358in}}{\pgfqpoint{1.107536in}{1.656458in}}{\pgfqpoint{1.107536in}{1.648222in}}%
\pgfpathcurveto{\pgfqpoint{1.107536in}{1.639986in}}{\pgfqpoint{1.110808in}{1.632085in}}{\pgfqpoint{1.116632in}{1.626262in}}%
\pgfpathcurveto{\pgfqpoint{1.122456in}{1.620438in}}{\pgfqpoint{1.130356in}{1.617165in}}{\pgfqpoint{1.138592in}{1.617165in}}%
\pgfpathclose%
\pgfusepath{stroke,fill}%
\end{pgfscope}%
\begin{pgfscope}%
\pgfpathrectangle{\pgfqpoint{0.100000in}{0.212622in}}{\pgfqpoint{3.696000in}{3.696000in}}%
\pgfusepath{clip}%
\pgfsetbuttcap%
\pgfsetroundjoin%
\definecolor{currentfill}{rgb}{0.121569,0.466667,0.705882}%
\pgfsetfillcolor{currentfill}%
\pgfsetfillopacity{0.300742}%
\pgfsetlinewidth{1.003750pt}%
\definecolor{currentstroke}{rgb}{0.121569,0.466667,0.705882}%
\pgfsetstrokecolor{currentstroke}%
\pgfsetstrokeopacity{0.300742}%
\pgfsetdash{}{0pt}%
\pgfpathmoveto{\pgfqpoint{1.138738in}{1.617126in}}%
\pgfpathcurveto{\pgfqpoint{1.146974in}{1.617126in}}{\pgfqpoint{1.154874in}{1.620398in}}{\pgfqpoint{1.160698in}{1.626222in}}%
\pgfpathcurveto{\pgfqpoint{1.166522in}{1.632046in}}{\pgfqpoint{1.169795in}{1.639946in}}{\pgfqpoint{1.169795in}{1.648182in}}%
\pgfpathcurveto{\pgfqpoint{1.169795in}{1.656418in}}{\pgfqpoint{1.166522in}{1.664318in}}{\pgfqpoint{1.160698in}{1.670142in}}%
\pgfpathcurveto{\pgfqpoint{1.154874in}{1.675966in}}{\pgfqpoint{1.146974in}{1.679239in}}{\pgfqpoint{1.138738in}{1.679239in}}%
\pgfpathcurveto{\pgfqpoint{1.130502in}{1.679239in}}{\pgfqpoint{1.122602in}{1.675966in}}{\pgfqpoint{1.116778in}{1.670142in}}%
\pgfpathcurveto{\pgfqpoint{1.110954in}{1.664318in}}{\pgfqpoint{1.107682in}{1.656418in}}{\pgfqpoint{1.107682in}{1.648182in}}%
\pgfpathcurveto{\pgfqpoint{1.107682in}{1.639946in}}{\pgfqpoint{1.110954in}{1.632046in}}{\pgfqpoint{1.116778in}{1.626222in}}%
\pgfpathcurveto{\pgfqpoint{1.122602in}{1.620398in}}{\pgfqpoint{1.130502in}{1.617126in}}{\pgfqpoint{1.138738in}{1.617126in}}%
\pgfpathclose%
\pgfusepath{stroke,fill}%
\end{pgfscope}%
\begin{pgfscope}%
\pgfpathrectangle{\pgfqpoint{0.100000in}{0.212622in}}{\pgfqpoint{3.696000in}{3.696000in}}%
\pgfusepath{clip}%
\pgfsetbuttcap%
\pgfsetroundjoin%
\definecolor{currentfill}{rgb}{0.121569,0.466667,0.705882}%
\pgfsetfillcolor{currentfill}%
\pgfsetfillopacity{0.300767}%
\pgfsetlinewidth{1.003750pt}%
\definecolor{currentstroke}{rgb}{0.121569,0.466667,0.705882}%
\pgfsetstrokecolor{currentstroke}%
\pgfsetstrokeopacity{0.300767}%
\pgfsetdash{}{0pt}%
\pgfpathmoveto{\pgfqpoint{1.138819in}{1.617104in}}%
\pgfpathcurveto{\pgfqpoint{1.147055in}{1.617104in}}{\pgfqpoint{1.154955in}{1.620376in}}{\pgfqpoint{1.160779in}{1.626200in}}%
\pgfpathcurveto{\pgfqpoint{1.166603in}{1.632024in}}{\pgfqpoint{1.169875in}{1.639924in}}{\pgfqpoint{1.169875in}{1.648161in}}%
\pgfpathcurveto{\pgfqpoint{1.169875in}{1.656397in}}{\pgfqpoint{1.166603in}{1.664297in}}{\pgfqpoint{1.160779in}{1.670121in}}%
\pgfpathcurveto{\pgfqpoint{1.154955in}{1.675945in}}{\pgfqpoint{1.147055in}{1.679217in}}{\pgfqpoint{1.138819in}{1.679217in}}%
\pgfpathcurveto{\pgfqpoint{1.130582in}{1.679217in}}{\pgfqpoint{1.122682in}{1.675945in}}{\pgfqpoint{1.116858in}{1.670121in}}%
\pgfpathcurveto{\pgfqpoint{1.111034in}{1.664297in}}{\pgfqpoint{1.107762in}{1.656397in}}{\pgfqpoint{1.107762in}{1.648161in}}%
\pgfpathcurveto{\pgfqpoint{1.107762in}{1.639924in}}{\pgfqpoint{1.111034in}{1.632024in}}{\pgfqpoint{1.116858in}{1.626200in}}%
\pgfpathcurveto{\pgfqpoint{1.122682in}{1.620376in}}{\pgfqpoint{1.130582in}{1.617104in}}{\pgfqpoint{1.138819in}{1.617104in}}%
\pgfpathclose%
\pgfusepath{stroke,fill}%
\end{pgfscope}%
\begin{pgfscope}%
\pgfpathrectangle{\pgfqpoint{0.100000in}{0.212622in}}{\pgfqpoint{3.696000in}{3.696000in}}%
\pgfusepath{clip}%
\pgfsetbuttcap%
\pgfsetroundjoin%
\definecolor{currentfill}{rgb}{0.121569,0.466667,0.705882}%
\pgfsetfillcolor{currentfill}%
\pgfsetfillopacity{0.300780}%
\pgfsetlinewidth{1.003750pt}%
\definecolor{currentstroke}{rgb}{0.121569,0.466667,0.705882}%
\pgfsetstrokecolor{currentstroke}%
\pgfsetstrokeopacity{0.300780}%
\pgfsetdash{}{0pt}%
\pgfpathmoveto{\pgfqpoint{1.138864in}{1.617093in}}%
\pgfpathcurveto{\pgfqpoint{1.147100in}{1.617093in}}{\pgfqpoint{1.155000in}{1.620365in}}{\pgfqpoint{1.160824in}{1.626189in}}%
\pgfpathcurveto{\pgfqpoint{1.166648in}{1.632013in}}{\pgfqpoint{1.169920in}{1.639913in}}{\pgfqpoint{1.169920in}{1.648149in}}%
\pgfpathcurveto{\pgfqpoint{1.169920in}{1.656386in}}{\pgfqpoint{1.166648in}{1.664286in}}{\pgfqpoint{1.160824in}{1.670110in}}%
\pgfpathcurveto{\pgfqpoint{1.155000in}{1.675934in}}{\pgfqpoint{1.147100in}{1.679206in}}{\pgfqpoint{1.138864in}{1.679206in}}%
\pgfpathcurveto{\pgfqpoint{1.130628in}{1.679206in}}{\pgfqpoint{1.122727in}{1.675934in}}{\pgfqpoint{1.116904in}{1.670110in}}%
\pgfpathcurveto{\pgfqpoint{1.111080in}{1.664286in}}{\pgfqpoint{1.107807in}{1.656386in}}{\pgfqpoint{1.107807in}{1.648149in}}%
\pgfpathcurveto{\pgfqpoint{1.107807in}{1.639913in}}{\pgfqpoint{1.111080in}{1.632013in}}{\pgfqpoint{1.116904in}{1.626189in}}%
\pgfpathcurveto{\pgfqpoint{1.122727in}{1.620365in}}{\pgfqpoint{1.130628in}{1.617093in}}{\pgfqpoint{1.138864in}{1.617093in}}%
\pgfpathclose%
\pgfusepath{stroke,fill}%
\end{pgfscope}%
\begin{pgfscope}%
\pgfpathrectangle{\pgfqpoint{0.100000in}{0.212622in}}{\pgfqpoint{3.696000in}{3.696000in}}%
\pgfusepath{clip}%
\pgfsetbuttcap%
\pgfsetroundjoin%
\definecolor{currentfill}{rgb}{0.121569,0.466667,0.705882}%
\pgfsetfillcolor{currentfill}%
\pgfsetfillopacity{0.300789}%
\pgfsetlinewidth{1.003750pt}%
\definecolor{currentstroke}{rgb}{0.121569,0.466667,0.705882}%
\pgfsetstrokecolor{currentstroke}%
\pgfsetstrokeopacity{0.300789}%
\pgfsetdash{}{0pt}%
\pgfpathmoveto{\pgfqpoint{1.138888in}{1.617086in}}%
\pgfpathcurveto{\pgfqpoint{1.147124in}{1.617086in}}{\pgfqpoint{1.155024in}{1.620358in}}{\pgfqpoint{1.160848in}{1.626182in}}%
\pgfpathcurveto{\pgfqpoint{1.166672in}{1.632006in}}{\pgfqpoint{1.169944in}{1.639906in}}{\pgfqpoint{1.169944in}{1.648142in}}%
\pgfpathcurveto{\pgfqpoint{1.169944in}{1.656379in}}{\pgfqpoint{1.166672in}{1.664279in}}{\pgfqpoint{1.160848in}{1.670103in}}%
\pgfpathcurveto{\pgfqpoint{1.155024in}{1.675927in}}{\pgfqpoint{1.147124in}{1.679199in}}{\pgfqpoint{1.138888in}{1.679199in}}%
\pgfpathcurveto{\pgfqpoint{1.130651in}{1.679199in}}{\pgfqpoint{1.122751in}{1.675927in}}{\pgfqpoint{1.116927in}{1.670103in}}%
\pgfpathcurveto{\pgfqpoint{1.111103in}{1.664279in}}{\pgfqpoint{1.107831in}{1.656379in}}{\pgfqpoint{1.107831in}{1.648142in}}%
\pgfpathcurveto{\pgfqpoint{1.107831in}{1.639906in}}{\pgfqpoint{1.111103in}{1.632006in}}{\pgfqpoint{1.116927in}{1.626182in}}%
\pgfpathcurveto{\pgfqpoint{1.122751in}{1.620358in}}{\pgfqpoint{1.130651in}{1.617086in}}{\pgfqpoint{1.138888in}{1.617086in}}%
\pgfpathclose%
\pgfusepath{stroke,fill}%
\end{pgfscope}%
\begin{pgfscope}%
\pgfpathrectangle{\pgfqpoint{0.100000in}{0.212622in}}{\pgfqpoint{3.696000in}{3.696000in}}%
\pgfusepath{clip}%
\pgfsetbuttcap%
\pgfsetroundjoin%
\definecolor{currentfill}{rgb}{0.121569,0.466667,0.705882}%
\pgfsetfillcolor{currentfill}%
\pgfsetfillopacity{0.300793}%
\pgfsetlinewidth{1.003750pt}%
\definecolor{currentstroke}{rgb}{0.121569,0.466667,0.705882}%
\pgfsetstrokecolor{currentstroke}%
\pgfsetstrokeopacity{0.300793}%
\pgfsetdash{}{0pt}%
\pgfpathmoveto{\pgfqpoint{1.138901in}{1.617082in}}%
\pgfpathcurveto{\pgfqpoint{1.147137in}{1.617082in}}{\pgfqpoint{1.155037in}{1.620354in}}{\pgfqpoint{1.160861in}{1.626178in}}%
\pgfpathcurveto{\pgfqpoint{1.166685in}{1.632002in}}{\pgfqpoint{1.169957in}{1.639902in}}{\pgfqpoint{1.169957in}{1.648138in}}%
\pgfpathcurveto{\pgfqpoint{1.169957in}{1.656375in}}{\pgfqpoint{1.166685in}{1.664275in}}{\pgfqpoint{1.160861in}{1.670099in}}%
\pgfpathcurveto{\pgfqpoint{1.155037in}{1.675923in}}{\pgfqpoint{1.147137in}{1.679195in}}{\pgfqpoint{1.138901in}{1.679195in}}%
\pgfpathcurveto{\pgfqpoint{1.130664in}{1.679195in}}{\pgfqpoint{1.122764in}{1.675923in}}{\pgfqpoint{1.116940in}{1.670099in}}%
\pgfpathcurveto{\pgfqpoint{1.111116in}{1.664275in}}{\pgfqpoint{1.107844in}{1.656375in}}{\pgfqpoint{1.107844in}{1.648138in}}%
\pgfpathcurveto{\pgfqpoint{1.107844in}{1.639902in}}{\pgfqpoint{1.111116in}{1.632002in}}{\pgfqpoint{1.116940in}{1.626178in}}%
\pgfpathcurveto{\pgfqpoint{1.122764in}{1.620354in}}{\pgfqpoint{1.130664in}{1.617082in}}{\pgfqpoint{1.138901in}{1.617082in}}%
\pgfpathclose%
\pgfusepath{stroke,fill}%
\end{pgfscope}%
\begin{pgfscope}%
\pgfpathrectangle{\pgfqpoint{0.100000in}{0.212622in}}{\pgfqpoint{3.696000in}{3.696000in}}%
\pgfusepath{clip}%
\pgfsetbuttcap%
\pgfsetroundjoin%
\definecolor{currentfill}{rgb}{0.121569,0.466667,0.705882}%
\pgfsetfillcolor{currentfill}%
\pgfsetfillopacity{0.300796}%
\pgfsetlinewidth{1.003750pt}%
\definecolor{currentstroke}{rgb}{0.121569,0.466667,0.705882}%
\pgfsetstrokecolor{currentstroke}%
\pgfsetstrokeopacity{0.300796}%
\pgfsetdash{}{0pt}%
\pgfpathmoveto{\pgfqpoint{1.138908in}{1.617080in}}%
\pgfpathcurveto{\pgfqpoint{1.147144in}{1.617080in}}{\pgfqpoint{1.155044in}{1.620352in}}{\pgfqpoint{1.160868in}{1.626176in}}%
\pgfpathcurveto{\pgfqpoint{1.166692in}{1.632000in}}{\pgfqpoint{1.169964in}{1.639900in}}{\pgfqpoint{1.169964in}{1.648136in}}%
\pgfpathcurveto{\pgfqpoint{1.169964in}{1.656372in}}{\pgfqpoint{1.166692in}{1.664272in}}{\pgfqpoint{1.160868in}{1.670096in}}%
\pgfpathcurveto{\pgfqpoint{1.155044in}{1.675920in}}{\pgfqpoint{1.147144in}{1.679193in}}{\pgfqpoint{1.138908in}{1.679193in}}%
\pgfpathcurveto{\pgfqpoint{1.130671in}{1.679193in}}{\pgfqpoint{1.122771in}{1.675920in}}{\pgfqpoint{1.116947in}{1.670096in}}%
\pgfpathcurveto{\pgfqpoint{1.111123in}{1.664272in}}{\pgfqpoint{1.107851in}{1.656372in}}{\pgfqpoint{1.107851in}{1.648136in}}%
\pgfpathcurveto{\pgfqpoint{1.107851in}{1.639900in}}{\pgfqpoint{1.111123in}{1.632000in}}{\pgfqpoint{1.116947in}{1.626176in}}%
\pgfpathcurveto{\pgfqpoint{1.122771in}{1.620352in}}{\pgfqpoint{1.130671in}{1.617080in}}{\pgfqpoint{1.138908in}{1.617080in}}%
\pgfpathclose%
\pgfusepath{stroke,fill}%
\end{pgfscope}%
\begin{pgfscope}%
\pgfpathrectangle{\pgfqpoint{0.100000in}{0.212622in}}{\pgfqpoint{3.696000in}{3.696000in}}%
\pgfusepath{clip}%
\pgfsetbuttcap%
\pgfsetroundjoin%
\definecolor{currentfill}{rgb}{0.121569,0.466667,0.705882}%
\pgfsetfillcolor{currentfill}%
\pgfsetfillopacity{0.300797}%
\pgfsetlinewidth{1.003750pt}%
\definecolor{currentstroke}{rgb}{0.121569,0.466667,0.705882}%
\pgfsetstrokecolor{currentstroke}%
\pgfsetstrokeopacity{0.300797}%
\pgfsetdash{}{0pt}%
\pgfpathmoveto{\pgfqpoint{1.138912in}{1.617079in}}%
\pgfpathcurveto{\pgfqpoint{1.147148in}{1.617079in}}{\pgfqpoint{1.155048in}{1.620351in}}{\pgfqpoint{1.160872in}{1.626175in}}%
\pgfpathcurveto{\pgfqpoint{1.166696in}{1.631999in}}{\pgfqpoint{1.169968in}{1.639899in}}{\pgfqpoint{1.169968in}{1.648135in}}%
\pgfpathcurveto{\pgfqpoint{1.169968in}{1.656371in}}{\pgfqpoint{1.166696in}{1.664272in}}{\pgfqpoint{1.160872in}{1.670095in}}%
\pgfpathcurveto{\pgfqpoint{1.155048in}{1.675919in}}{\pgfqpoint{1.147148in}{1.679192in}}{\pgfqpoint{1.138912in}{1.679192in}}%
\pgfpathcurveto{\pgfqpoint{1.130676in}{1.679192in}}{\pgfqpoint{1.122775in}{1.675919in}}{\pgfqpoint{1.116952in}{1.670095in}}%
\pgfpathcurveto{\pgfqpoint{1.111128in}{1.664272in}}{\pgfqpoint{1.107855in}{1.656371in}}{\pgfqpoint{1.107855in}{1.648135in}}%
\pgfpathcurveto{\pgfqpoint{1.107855in}{1.639899in}}{\pgfqpoint{1.111128in}{1.631999in}}{\pgfqpoint{1.116952in}{1.626175in}}%
\pgfpathcurveto{\pgfqpoint{1.122775in}{1.620351in}}{\pgfqpoint{1.130676in}{1.617079in}}{\pgfqpoint{1.138912in}{1.617079in}}%
\pgfpathclose%
\pgfusepath{stroke,fill}%
\end{pgfscope}%
\begin{pgfscope}%
\pgfpathrectangle{\pgfqpoint{0.100000in}{0.212622in}}{\pgfqpoint{3.696000in}{3.696000in}}%
\pgfusepath{clip}%
\pgfsetbuttcap%
\pgfsetroundjoin%
\definecolor{currentfill}{rgb}{0.121569,0.466667,0.705882}%
\pgfsetfillcolor{currentfill}%
\pgfsetfillopacity{0.300798}%
\pgfsetlinewidth{1.003750pt}%
\definecolor{currentstroke}{rgb}{0.121569,0.466667,0.705882}%
\pgfsetstrokecolor{currentstroke}%
\pgfsetstrokeopacity{0.300798}%
\pgfsetdash{}{0pt}%
\pgfpathmoveto{\pgfqpoint{1.138914in}{1.617078in}}%
\pgfpathcurveto{\pgfqpoint{1.147150in}{1.617078in}}{\pgfqpoint{1.155050in}{1.620350in}}{\pgfqpoint{1.160874in}{1.626174in}}%
\pgfpathcurveto{\pgfqpoint{1.166698in}{1.631998in}}{\pgfqpoint{1.169971in}{1.639898in}}{\pgfqpoint{1.169971in}{1.648135in}}%
\pgfpathcurveto{\pgfqpoint{1.169971in}{1.656371in}}{\pgfqpoint{1.166698in}{1.664271in}}{\pgfqpoint{1.160874in}{1.670095in}}%
\pgfpathcurveto{\pgfqpoint{1.155050in}{1.675919in}}{\pgfqpoint{1.147150in}{1.679191in}}{\pgfqpoint{1.138914in}{1.679191in}}%
\pgfpathcurveto{\pgfqpoint{1.130678in}{1.679191in}}{\pgfqpoint{1.122778in}{1.675919in}}{\pgfqpoint{1.116954in}{1.670095in}}%
\pgfpathcurveto{\pgfqpoint{1.111130in}{1.664271in}}{\pgfqpoint{1.107858in}{1.656371in}}{\pgfqpoint{1.107858in}{1.648135in}}%
\pgfpathcurveto{\pgfqpoint{1.107858in}{1.639898in}}{\pgfqpoint{1.111130in}{1.631998in}}{\pgfqpoint{1.116954in}{1.626174in}}%
\pgfpathcurveto{\pgfqpoint{1.122778in}{1.620350in}}{\pgfqpoint{1.130678in}{1.617078in}}{\pgfqpoint{1.138914in}{1.617078in}}%
\pgfpathclose%
\pgfusepath{stroke,fill}%
\end{pgfscope}%
\begin{pgfscope}%
\pgfpathrectangle{\pgfqpoint{0.100000in}{0.212622in}}{\pgfqpoint{3.696000in}{3.696000in}}%
\pgfusepath{clip}%
\pgfsetbuttcap%
\pgfsetroundjoin%
\definecolor{currentfill}{rgb}{0.121569,0.466667,0.705882}%
\pgfsetfillcolor{currentfill}%
\pgfsetfillopacity{0.300798}%
\pgfsetlinewidth{1.003750pt}%
\definecolor{currentstroke}{rgb}{0.121569,0.466667,0.705882}%
\pgfsetstrokecolor{currentstroke}%
\pgfsetstrokeopacity{0.300798}%
\pgfsetdash{}{0pt}%
\pgfpathmoveto{\pgfqpoint{1.138915in}{1.617078in}}%
\pgfpathcurveto{\pgfqpoint{1.147152in}{1.617078in}}{\pgfqpoint{1.155052in}{1.620350in}}{\pgfqpoint{1.160876in}{1.626174in}}%
\pgfpathcurveto{\pgfqpoint{1.166700in}{1.631998in}}{\pgfqpoint{1.169972in}{1.639898in}}{\pgfqpoint{1.169972in}{1.648134in}}%
\pgfpathcurveto{\pgfqpoint{1.169972in}{1.656371in}}{\pgfqpoint{1.166700in}{1.664271in}}{\pgfqpoint{1.160876in}{1.670095in}}%
\pgfpathcurveto{\pgfqpoint{1.155052in}{1.675919in}}{\pgfqpoint{1.147152in}{1.679191in}}{\pgfqpoint{1.138915in}{1.679191in}}%
\pgfpathcurveto{\pgfqpoint{1.130679in}{1.679191in}}{\pgfqpoint{1.122779in}{1.675919in}}{\pgfqpoint{1.116955in}{1.670095in}}%
\pgfpathcurveto{\pgfqpoint{1.111131in}{1.664271in}}{\pgfqpoint{1.107859in}{1.656371in}}{\pgfqpoint{1.107859in}{1.648134in}}%
\pgfpathcurveto{\pgfqpoint{1.107859in}{1.639898in}}{\pgfqpoint{1.111131in}{1.631998in}}{\pgfqpoint{1.116955in}{1.626174in}}%
\pgfpathcurveto{\pgfqpoint{1.122779in}{1.620350in}}{\pgfqpoint{1.130679in}{1.617078in}}{\pgfqpoint{1.138915in}{1.617078in}}%
\pgfpathclose%
\pgfusepath{stroke,fill}%
\end{pgfscope}%
\begin{pgfscope}%
\pgfpathrectangle{\pgfqpoint{0.100000in}{0.212622in}}{\pgfqpoint{3.696000in}{3.696000in}}%
\pgfusepath{clip}%
\pgfsetbuttcap%
\pgfsetroundjoin%
\definecolor{currentfill}{rgb}{0.121569,0.466667,0.705882}%
\pgfsetfillcolor{currentfill}%
\pgfsetfillopacity{0.300798}%
\pgfsetlinewidth{1.003750pt}%
\definecolor{currentstroke}{rgb}{0.121569,0.466667,0.705882}%
\pgfsetstrokecolor{currentstroke}%
\pgfsetstrokeopacity{0.300798}%
\pgfsetdash{}{0pt}%
\pgfpathmoveto{\pgfqpoint{1.138916in}{1.617078in}}%
\pgfpathcurveto{\pgfqpoint{1.147152in}{1.617078in}}{\pgfqpoint{1.155052in}{1.620350in}}{\pgfqpoint{1.160876in}{1.626174in}}%
\pgfpathcurveto{\pgfqpoint{1.166700in}{1.631998in}}{\pgfqpoint{1.169973in}{1.639898in}}{\pgfqpoint{1.169973in}{1.648134in}}%
\pgfpathcurveto{\pgfqpoint{1.169973in}{1.656370in}}{\pgfqpoint{1.166700in}{1.664271in}}{\pgfqpoint{1.160876in}{1.670094in}}%
\pgfpathcurveto{\pgfqpoint{1.155052in}{1.675918in}}{\pgfqpoint{1.147152in}{1.679191in}}{\pgfqpoint{1.138916in}{1.679191in}}%
\pgfpathcurveto{\pgfqpoint{1.130680in}{1.679191in}}{\pgfqpoint{1.122780in}{1.675918in}}{\pgfqpoint{1.116956in}{1.670094in}}%
\pgfpathcurveto{\pgfqpoint{1.111132in}{1.664271in}}{\pgfqpoint{1.107860in}{1.656370in}}{\pgfqpoint{1.107860in}{1.648134in}}%
\pgfpathcurveto{\pgfqpoint{1.107860in}{1.639898in}}{\pgfqpoint{1.111132in}{1.631998in}}{\pgfqpoint{1.116956in}{1.626174in}}%
\pgfpathcurveto{\pgfqpoint{1.122780in}{1.620350in}}{\pgfqpoint{1.130680in}{1.617078in}}{\pgfqpoint{1.138916in}{1.617078in}}%
\pgfpathclose%
\pgfusepath{stroke,fill}%
\end{pgfscope}%
\begin{pgfscope}%
\pgfpathrectangle{\pgfqpoint{0.100000in}{0.212622in}}{\pgfqpoint{3.696000in}{3.696000in}}%
\pgfusepath{clip}%
\pgfsetbuttcap%
\pgfsetroundjoin%
\definecolor{currentfill}{rgb}{0.121569,0.466667,0.705882}%
\pgfsetfillcolor{currentfill}%
\pgfsetfillopacity{0.300799}%
\pgfsetlinewidth{1.003750pt}%
\definecolor{currentstroke}{rgb}{0.121569,0.466667,0.705882}%
\pgfsetstrokecolor{currentstroke}%
\pgfsetstrokeopacity{0.300799}%
\pgfsetdash{}{0pt}%
\pgfpathmoveto{\pgfqpoint{1.138916in}{1.617078in}}%
\pgfpathcurveto{\pgfqpoint{1.147153in}{1.617078in}}{\pgfqpoint{1.155053in}{1.620350in}}{\pgfqpoint{1.160877in}{1.626174in}}%
\pgfpathcurveto{\pgfqpoint{1.166701in}{1.631998in}}{\pgfqpoint{1.169973in}{1.639898in}}{\pgfqpoint{1.169973in}{1.648134in}}%
\pgfpathcurveto{\pgfqpoint{1.169973in}{1.656370in}}{\pgfqpoint{1.166701in}{1.664270in}}{\pgfqpoint{1.160877in}{1.670094in}}%
\pgfpathcurveto{\pgfqpoint{1.155053in}{1.675918in}}{\pgfqpoint{1.147153in}{1.679191in}}{\pgfqpoint{1.138916in}{1.679191in}}%
\pgfpathcurveto{\pgfqpoint{1.130680in}{1.679191in}}{\pgfqpoint{1.122780in}{1.675918in}}{\pgfqpoint{1.116956in}{1.670094in}}%
\pgfpathcurveto{\pgfqpoint{1.111132in}{1.664270in}}{\pgfqpoint{1.107860in}{1.656370in}}{\pgfqpoint{1.107860in}{1.648134in}}%
\pgfpathcurveto{\pgfqpoint{1.107860in}{1.639898in}}{\pgfqpoint{1.111132in}{1.631998in}}{\pgfqpoint{1.116956in}{1.626174in}}%
\pgfpathcurveto{\pgfqpoint{1.122780in}{1.620350in}}{\pgfqpoint{1.130680in}{1.617078in}}{\pgfqpoint{1.138916in}{1.617078in}}%
\pgfpathclose%
\pgfusepath{stroke,fill}%
\end{pgfscope}%
\begin{pgfscope}%
\pgfpathrectangle{\pgfqpoint{0.100000in}{0.212622in}}{\pgfqpoint{3.696000in}{3.696000in}}%
\pgfusepath{clip}%
\pgfsetbuttcap%
\pgfsetroundjoin%
\definecolor{currentfill}{rgb}{0.121569,0.466667,0.705882}%
\pgfsetfillcolor{currentfill}%
\pgfsetfillopacity{0.300799}%
\pgfsetlinewidth{1.003750pt}%
\definecolor{currentstroke}{rgb}{0.121569,0.466667,0.705882}%
\pgfsetstrokecolor{currentstroke}%
\pgfsetstrokeopacity{0.300799}%
\pgfsetdash{}{0pt}%
\pgfpathmoveto{\pgfqpoint{1.138917in}{1.617078in}}%
\pgfpathcurveto{\pgfqpoint{1.147153in}{1.617078in}}{\pgfqpoint{1.155053in}{1.620350in}}{\pgfqpoint{1.160877in}{1.626174in}}%
\pgfpathcurveto{\pgfqpoint{1.166701in}{1.631998in}}{\pgfqpoint{1.169973in}{1.639898in}}{\pgfqpoint{1.169973in}{1.648134in}}%
\pgfpathcurveto{\pgfqpoint{1.169973in}{1.656370in}}{\pgfqpoint{1.166701in}{1.664270in}}{\pgfqpoint{1.160877in}{1.670094in}}%
\pgfpathcurveto{\pgfqpoint{1.155053in}{1.675918in}}{\pgfqpoint{1.147153in}{1.679191in}}{\pgfqpoint{1.138917in}{1.679191in}}%
\pgfpathcurveto{\pgfqpoint{1.130680in}{1.679191in}}{\pgfqpoint{1.122780in}{1.675918in}}{\pgfqpoint{1.116956in}{1.670094in}}%
\pgfpathcurveto{\pgfqpoint{1.111132in}{1.664270in}}{\pgfqpoint{1.107860in}{1.656370in}}{\pgfqpoint{1.107860in}{1.648134in}}%
\pgfpathcurveto{\pgfqpoint{1.107860in}{1.639898in}}{\pgfqpoint{1.111132in}{1.631998in}}{\pgfqpoint{1.116956in}{1.626174in}}%
\pgfpathcurveto{\pgfqpoint{1.122780in}{1.620350in}}{\pgfqpoint{1.130680in}{1.617078in}}{\pgfqpoint{1.138917in}{1.617078in}}%
\pgfpathclose%
\pgfusepath{stroke,fill}%
\end{pgfscope}%
\begin{pgfscope}%
\pgfpathrectangle{\pgfqpoint{0.100000in}{0.212622in}}{\pgfqpoint{3.696000in}{3.696000in}}%
\pgfusepath{clip}%
\pgfsetbuttcap%
\pgfsetroundjoin%
\definecolor{currentfill}{rgb}{0.121569,0.466667,0.705882}%
\pgfsetfillcolor{currentfill}%
\pgfsetfillopacity{0.300799}%
\pgfsetlinewidth{1.003750pt}%
\definecolor{currentstroke}{rgb}{0.121569,0.466667,0.705882}%
\pgfsetstrokecolor{currentstroke}%
\pgfsetstrokeopacity{0.300799}%
\pgfsetdash{}{0pt}%
\pgfpathmoveto{\pgfqpoint{1.138917in}{1.617078in}}%
\pgfpathcurveto{\pgfqpoint{1.147153in}{1.617078in}}{\pgfqpoint{1.155053in}{1.620350in}}{\pgfqpoint{1.160877in}{1.626174in}}%
\pgfpathcurveto{\pgfqpoint{1.166701in}{1.631998in}}{\pgfqpoint{1.169973in}{1.639898in}}{\pgfqpoint{1.169973in}{1.648134in}}%
\pgfpathcurveto{\pgfqpoint{1.169973in}{1.656370in}}{\pgfqpoint{1.166701in}{1.664270in}}{\pgfqpoint{1.160877in}{1.670094in}}%
\pgfpathcurveto{\pgfqpoint{1.155053in}{1.675918in}}{\pgfqpoint{1.147153in}{1.679191in}}{\pgfqpoint{1.138917in}{1.679191in}}%
\pgfpathcurveto{\pgfqpoint{1.130680in}{1.679191in}}{\pgfqpoint{1.122780in}{1.675918in}}{\pgfqpoint{1.116956in}{1.670094in}}%
\pgfpathcurveto{\pgfqpoint{1.111133in}{1.664270in}}{\pgfqpoint{1.107860in}{1.656370in}}{\pgfqpoint{1.107860in}{1.648134in}}%
\pgfpathcurveto{\pgfqpoint{1.107860in}{1.639898in}}{\pgfqpoint{1.111133in}{1.631998in}}{\pgfqpoint{1.116956in}{1.626174in}}%
\pgfpathcurveto{\pgfqpoint{1.122780in}{1.620350in}}{\pgfqpoint{1.130680in}{1.617078in}}{\pgfqpoint{1.138917in}{1.617078in}}%
\pgfpathclose%
\pgfusepath{stroke,fill}%
\end{pgfscope}%
\begin{pgfscope}%
\pgfpathrectangle{\pgfqpoint{0.100000in}{0.212622in}}{\pgfqpoint{3.696000in}{3.696000in}}%
\pgfusepath{clip}%
\pgfsetbuttcap%
\pgfsetroundjoin%
\definecolor{currentfill}{rgb}{0.121569,0.466667,0.705882}%
\pgfsetfillcolor{currentfill}%
\pgfsetfillopacity{0.300799}%
\pgfsetlinewidth{1.003750pt}%
\definecolor{currentstroke}{rgb}{0.121569,0.466667,0.705882}%
\pgfsetstrokecolor{currentstroke}%
\pgfsetstrokeopacity{0.300799}%
\pgfsetdash{}{0pt}%
\pgfpathmoveto{\pgfqpoint{1.138917in}{1.617078in}}%
\pgfpathcurveto{\pgfqpoint{1.147153in}{1.617078in}}{\pgfqpoint{1.155053in}{1.620350in}}{\pgfqpoint{1.160877in}{1.626174in}}%
\pgfpathcurveto{\pgfqpoint{1.166701in}{1.631998in}}{\pgfqpoint{1.169973in}{1.639898in}}{\pgfqpoint{1.169973in}{1.648134in}}%
\pgfpathcurveto{\pgfqpoint{1.169973in}{1.656370in}}{\pgfqpoint{1.166701in}{1.664270in}}{\pgfqpoint{1.160877in}{1.670094in}}%
\pgfpathcurveto{\pgfqpoint{1.155053in}{1.675918in}}{\pgfqpoint{1.147153in}{1.679191in}}{\pgfqpoint{1.138917in}{1.679191in}}%
\pgfpathcurveto{\pgfqpoint{1.130681in}{1.679191in}}{\pgfqpoint{1.122780in}{1.675918in}}{\pgfqpoint{1.116957in}{1.670094in}}%
\pgfpathcurveto{\pgfqpoint{1.111133in}{1.664270in}}{\pgfqpoint{1.107860in}{1.656370in}}{\pgfqpoint{1.107860in}{1.648134in}}%
\pgfpathcurveto{\pgfqpoint{1.107860in}{1.639898in}}{\pgfqpoint{1.111133in}{1.631998in}}{\pgfqpoint{1.116957in}{1.626174in}}%
\pgfpathcurveto{\pgfqpoint{1.122780in}{1.620350in}}{\pgfqpoint{1.130681in}{1.617078in}}{\pgfqpoint{1.138917in}{1.617078in}}%
\pgfpathclose%
\pgfusepath{stroke,fill}%
\end{pgfscope}%
\begin{pgfscope}%
\pgfpathrectangle{\pgfqpoint{0.100000in}{0.212622in}}{\pgfqpoint{3.696000in}{3.696000in}}%
\pgfusepath{clip}%
\pgfsetbuttcap%
\pgfsetroundjoin%
\definecolor{currentfill}{rgb}{0.121569,0.466667,0.705882}%
\pgfsetfillcolor{currentfill}%
\pgfsetfillopacity{0.300799}%
\pgfsetlinewidth{1.003750pt}%
\definecolor{currentstroke}{rgb}{0.121569,0.466667,0.705882}%
\pgfsetstrokecolor{currentstroke}%
\pgfsetstrokeopacity{0.300799}%
\pgfsetdash{}{0pt}%
\pgfpathmoveto{\pgfqpoint{1.138917in}{1.617077in}}%
\pgfpathcurveto{\pgfqpoint{1.147153in}{1.617077in}}{\pgfqpoint{1.155053in}{1.620350in}}{\pgfqpoint{1.160877in}{1.626174in}}%
\pgfpathcurveto{\pgfqpoint{1.166701in}{1.631998in}}{\pgfqpoint{1.169973in}{1.639898in}}{\pgfqpoint{1.169973in}{1.648134in}}%
\pgfpathcurveto{\pgfqpoint{1.169973in}{1.656370in}}{\pgfqpoint{1.166701in}{1.664270in}}{\pgfqpoint{1.160877in}{1.670094in}}%
\pgfpathcurveto{\pgfqpoint{1.155053in}{1.675918in}}{\pgfqpoint{1.147153in}{1.679190in}}{\pgfqpoint{1.138917in}{1.679190in}}%
\pgfpathcurveto{\pgfqpoint{1.130681in}{1.679190in}}{\pgfqpoint{1.122781in}{1.675918in}}{\pgfqpoint{1.116957in}{1.670094in}}%
\pgfpathcurveto{\pgfqpoint{1.111133in}{1.664270in}}{\pgfqpoint{1.107860in}{1.656370in}}{\pgfqpoint{1.107860in}{1.648134in}}%
\pgfpathcurveto{\pgfqpoint{1.107860in}{1.639898in}}{\pgfqpoint{1.111133in}{1.631998in}}{\pgfqpoint{1.116957in}{1.626174in}}%
\pgfpathcurveto{\pgfqpoint{1.122781in}{1.620350in}}{\pgfqpoint{1.130681in}{1.617077in}}{\pgfqpoint{1.138917in}{1.617077in}}%
\pgfpathclose%
\pgfusepath{stroke,fill}%
\end{pgfscope}%
\begin{pgfscope}%
\pgfpathrectangle{\pgfqpoint{0.100000in}{0.212622in}}{\pgfqpoint{3.696000in}{3.696000in}}%
\pgfusepath{clip}%
\pgfsetbuttcap%
\pgfsetroundjoin%
\definecolor{currentfill}{rgb}{0.121569,0.466667,0.705882}%
\pgfsetfillcolor{currentfill}%
\pgfsetfillopacity{0.300799}%
\pgfsetlinewidth{1.003750pt}%
\definecolor{currentstroke}{rgb}{0.121569,0.466667,0.705882}%
\pgfsetstrokecolor{currentstroke}%
\pgfsetstrokeopacity{0.300799}%
\pgfsetdash{}{0pt}%
\pgfpathmoveto{\pgfqpoint{1.138917in}{1.617077in}}%
\pgfpathcurveto{\pgfqpoint{1.147153in}{1.617077in}}{\pgfqpoint{1.155053in}{1.620350in}}{\pgfqpoint{1.160877in}{1.626174in}}%
\pgfpathcurveto{\pgfqpoint{1.166701in}{1.631998in}}{\pgfqpoint{1.169973in}{1.639898in}}{\pgfqpoint{1.169973in}{1.648134in}}%
\pgfpathcurveto{\pgfqpoint{1.169973in}{1.656370in}}{\pgfqpoint{1.166701in}{1.664270in}}{\pgfqpoint{1.160877in}{1.670094in}}%
\pgfpathcurveto{\pgfqpoint{1.155053in}{1.675918in}}{\pgfqpoint{1.147153in}{1.679190in}}{\pgfqpoint{1.138917in}{1.679190in}}%
\pgfpathcurveto{\pgfqpoint{1.130681in}{1.679190in}}{\pgfqpoint{1.122781in}{1.675918in}}{\pgfqpoint{1.116957in}{1.670094in}}%
\pgfpathcurveto{\pgfqpoint{1.111133in}{1.664270in}}{\pgfqpoint{1.107860in}{1.656370in}}{\pgfqpoint{1.107860in}{1.648134in}}%
\pgfpathcurveto{\pgfqpoint{1.107860in}{1.639898in}}{\pgfqpoint{1.111133in}{1.631998in}}{\pgfqpoint{1.116957in}{1.626174in}}%
\pgfpathcurveto{\pgfqpoint{1.122781in}{1.620350in}}{\pgfqpoint{1.130681in}{1.617077in}}{\pgfqpoint{1.138917in}{1.617077in}}%
\pgfpathclose%
\pgfusepath{stroke,fill}%
\end{pgfscope}%
\begin{pgfscope}%
\pgfpathrectangle{\pgfqpoint{0.100000in}{0.212622in}}{\pgfqpoint{3.696000in}{3.696000in}}%
\pgfusepath{clip}%
\pgfsetbuttcap%
\pgfsetroundjoin%
\definecolor{currentfill}{rgb}{0.121569,0.466667,0.705882}%
\pgfsetfillcolor{currentfill}%
\pgfsetfillopacity{0.300799}%
\pgfsetlinewidth{1.003750pt}%
\definecolor{currentstroke}{rgb}{0.121569,0.466667,0.705882}%
\pgfsetstrokecolor{currentstroke}%
\pgfsetstrokeopacity{0.300799}%
\pgfsetdash{}{0pt}%
\pgfpathmoveto{\pgfqpoint{1.138917in}{1.617077in}}%
\pgfpathcurveto{\pgfqpoint{1.147153in}{1.617077in}}{\pgfqpoint{1.155053in}{1.620350in}}{\pgfqpoint{1.160877in}{1.626174in}}%
\pgfpathcurveto{\pgfqpoint{1.166701in}{1.631998in}}{\pgfqpoint{1.169973in}{1.639898in}}{\pgfqpoint{1.169973in}{1.648134in}}%
\pgfpathcurveto{\pgfqpoint{1.169973in}{1.656370in}}{\pgfqpoint{1.166701in}{1.664270in}}{\pgfqpoint{1.160877in}{1.670094in}}%
\pgfpathcurveto{\pgfqpoint{1.155053in}{1.675918in}}{\pgfqpoint{1.147153in}{1.679190in}}{\pgfqpoint{1.138917in}{1.679190in}}%
\pgfpathcurveto{\pgfqpoint{1.130681in}{1.679190in}}{\pgfqpoint{1.122781in}{1.675918in}}{\pgfqpoint{1.116957in}{1.670094in}}%
\pgfpathcurveto{\pgfqpoint{1.111133in}{1.664270in}}{\pgfqpoint{1.107860in}{1.656370in}}{\pgfqpoint{1.107860in}{1.648134in}}%
\pgfpathcurveto{\pgfqpoint{1.107860in}{1.639898in}}{\pgfqpoint{1.111133in}{1.631998in}}{\pgfqpoint{1.116957in}{1.626174in}}%
\pgfpathcurveto{\pgfqpoint{1.122781in}{1.620350in}}{\pgfqpoint{1.130681in}{1.617077in}}{\pgfqpoint{1.138917in}{1.617077in}}%
\pgfpathclose%
\pgfusepath{stroke,fill}%
\end{pgfscope}%
\begin{pgfscope}%
\pgfpathrectangle{\pgfqpoint{0.100000in}{0.212622in}}{\pgfqpoint{3.696000in}{3.696000in}}%
\pgfusepath{clip}%
\pgfsetbuttcap%
\pgfsetroundjoin%
\definecolor{currentfill}{rgb}{0.121569,0.466667,0.705882}%
\pgfsetfillcolor{currentfill}%
\pgfsetfillopacity{0.300799}%
\pgfsetlinewidth{1.003750pt}%
\definecolor{currentstroke}{rgb}{0.121569,0.466667,0.705882}%
\pgfsetstrokecolor{currentstroke}%
\pgfsetstrokeopacity{0.300799}%
\pgfsetdash{}{0pt}%
\pgfpathmoveto{\pgfqpoint{1.138917in}{1.617077in}}%
\pgfpathcurveto{\pgfqpoint{1.147153in}{1.617077in}}{\pgfqpoint{1.155053in}{1.620350in}}{\pgfqpoint{1.160877in}{1.626174in}}%
\pgfpathcurveto{\pgfqpoint{1.166701in}{1.631998in}}{\pgfqpoint{1.169973in}{1.639898in}}{\pgfqpoint{1.169973in}{1.648134in}}%
\pgfpathcurveto{\pgfqpoint{1.169973in}{1.656370in}}{\pgfqpoint{1.166701in}{1.664270in}}{\pgfqpoint{1.160877in}{1.670094in}}%
\pgfpathcurveto{\pgfqpoint{1.155053in}{1.675918in}}{\pgfqpoint{1.147153in}{1.679190in}}{\pgfqpoint{1.138917in}{1.679190in}}%
\pgfpathcurveto{\pgfqpoint{1.130681in}{1.679190in}}{\pgfqpoint{1.122781in}{1.675918in}}{\pgfqpoint{1.116957in}{1.670094in}}%
\pgfpathcurveto{\pgfqpoint{1.111133in}{1.664270in}}{\pgfqpoint{1.107860in}{1.656370in}}{\pgfqpoint{1.107860in}{1.648134in}}%
\pgfpathcurveto{\pgfqpoint{1.107860in}{1.639898in}}{\pgfqpoint{1.111133in}{1.631998in}}{\pgfqpoint{1.116957in}{1.626174in}}%
\pgfpathcurveto{\pgfqpoint{1.122781in}{1.620350in}}{\pgfqpoint{1.130681in}{1.617077in}}{\pgfqpoint{1.138917in}{1.617077in}}%
\pgfpathclose%
\pgfusepath{stroke,fill}%
\end{pgfscope}%
\begin{pgfscope}%
\pgfpathrectangle{\pgfqpoint{0.100000in}{0.212622in}}{\pgfqpoint{3.696000in}{3.696000in}}%
\pgfusepath{clip}%
\pgfsetbuttcap%
\pgfsetroundjoin%
\definecolor{currentfill}{rgb}{0.121569,0.466667,0.705882}%
\pgfsetfillcolor{currentfill}%
\pgfsetfillopacity{0.300799}%
\pgfsetlinewidth{1.003750pt}%
\definecolor{currentstroke}{rgb}{0.121569,0.466667,0.705882}%
\pgfsetstrokecolor{currentstroke}%
\pgfsetstrokeopacity{0.300799}%
\pgfsetdash{}{0pt}%
\pgfpathmoveto{\pgfqpoint{1.138917in}{1.617077in}}%
\pgfpathcurveto{\pgfqpoint{1.147153in}{1.617077in}}{\pgfqpoint{1.155053in}{1.620350in}}{\pgfqpoint{1.160877in}{1.626174in}}%
\pgfpathcurveto{\pgfqpoint{1.166701in}{1.631998in}}{\pgfqpoint{1.169973in}{1.639898in}}{\pgfqpoint{1.169973in}{1.648134in}}%
\pgfpathcurveto{\pgfqpoint{1.169973in}{1.656370in}}{\pgfqpoint{1.166701in}{1.664270in}}{\pgfqpoint{1.160877in}{1.670094in}}%
\pgfpathcurveto{\pgfqpoint{1.155053in}{1.675918in}}{\pgfqpoint{1.147153in}{1.679190in}}{\pgfqpoint{1.138917in}{1.679190in}}%
\pgfpathcurveto{\pgfqpoint{1.130681in}{1.679190in}}{\pgfqpoint{1.122781in}{1.675918in}}{\pgfqpoint{1.116957in}{1.670094in}}%
\pgfpathcurveto{\pgfqpoint{1.111133in}{1.664270in}}{\pgfqpoint{1.107860in}{1.656370in}}{\pgfqpoint{1.107860in}{1.648134in}}%
\pgfpathcurveto{\pgfqpoint{1.107860in}{1.639898in}}{\pgfqpoint{1.111133in}{1.631998in}}{\pgfqpoint{1.116957in}{1.626174in}}%
\pgfpathcurveto{\pgfqpoint{1.122781in}{1.620350in}}{\pgfqpoint{1.130681in}{1.617077in}}{\pgfqpoint{1.138917in}{1.617077in}}%
\pgfpathclose%
\pgfusepath{stroke,fill}%
\end{pgfscope}%
\begin{pgfscope}%
\pgfpathrectangle{\pgfqpoint{0.100000in}{0.212622in}}{\pgfqpoint{3.696000in}{3.696000in}}%
\pgfusepath{clip}%
\pgfsetbuttcap%
\pgfsetroundjoin%
\definecolor{currentfill}{rgb}{0.121569,0.466667,0.705882}%
\pgfsetfillcolor{currentfill}%
\pgfsetfillopacity{0.300799}%
\pgfsetlinewidth{1.003750pt}%
\definecolor{currentstroke}{rgb}{0.121569,0.466667,0.705882}%
\pgfsetstrokecolor{currentstroke}%
\pgfsetstrokeopacity{0.300799}%
\pgfsetdash{}{0pt}%
\pgfpathmoveto{\pgfqpoint{1.138917in}{1.617077in}}%
\pgfpathcurveto{\pgfqpoint{1.147153in}{1.617077in}}{\pgfqpoint{1.155053in}{1.620350in}}{\pgfqpoint{1.160877in}{1.626174in}}%
\pgfpathcurveto{\pgfqpoint{1.166701in}{1.631998in}}{\pgfqpoint{1.169973in}{1.639898in}}{\pgfqpoint{1.169973in}{1.648134in}}%
\pgfpathcurveto{\pgfqpoint{1.169973in}{1.656370in}}{\pgfqpoint{1.166701in}{1.664270in}}{\pgfqpoint{1.160877in}{1.670094in}}%
\pgfpathcurveto{\pgfqpoint{1.155053in}{1.675918in}}{\pgfqpoint{1.147153in}{1.679190in}}{\pgfqpoint{1.138917in}{1.679190in}}%
\pgfpathcurveto{\pgfqpoint{1.130681in}{1.679190in}}{\pgfqpoint{1.122781in}{1.675918in}}{\pgfqpoint{1.116957in}{1.670094in}}%
\pgfpathcurveto{\pgfqpoint{1.111133in}{1.664270in}}{\pgfqpoint{1.107860in}{1.656370in}}{\pgfqpoint{1.107860in}{1.648134in}}%
\pgfpathcurveto{\pgfqpoint{1.107860in}{1.639898in}}{\pgfqpoint{1.111133in}{1.631998in}}{\pgfqpoint{1.116957in}{1.626174in}}%
\pgfpathcurveto{\pgfqpoint{1.122781in}{1.620350in}}{\pgfqpoint{1.130681in}{1.617077in}}{\pgfqpoint{1.138917in}{1.617077in}}%
\pgfpathclose%
\pgfusepath{stroke,fill}%
\end{pgfscope}%
\begin{pgfscope}%
\pgfpathrectangle{\pgfqpoint{0.100000in}{0.212622in}}{\pgfqpoint{3.696000in}{3.696000in}}%
\pgfusepath{clip}%
\pgfsetbuttcap%
\pgfsetroundjoin%
\definecolor{currentfill}{rgb}{0.121569,0.466667,0.705882}%
\pgfsetfillcolor{currentfill}%
\pgfsetfillopacity{0.300799}%
\pgfsetlinewidth{1.003750pt}%
\definecolor{currentstroke}{rgb}{0.121569,0.466667,0.705882}%
\pgfsetstrokecolor{currentstroke}%
\pgfsetstrokeopacity{0.300799}%
\pgfsetdash{}{0pt}%
\pgfpathmoveto{\pgfqpoint{1.138917in}{1.617077in}}%
\pgfpathcurveto{\pgfqpoint{1.147153in}{1.617077in}}{\pgfqpoint{1.155053in}{1.620350in}}{\pgfqpoint{1.160877in}{1.626174in}}%
\pgfpathcurveto{\pgfqpoint{1.166701in}{1.631998in}}{\pgfqpoint{1.169973in}{1.639898in}}{\pgfqpoint{1.169973in}{1.648134in}}%
\pgfpathcurveto{\pgfqpoint{1.169973in}{1.656370in}}{\pgfqpoint{1.166701in}{1.664270in}}{\pgfqpoint{1.160877in}{1.670094in}}%
\pgfpathcurveto{\pgfqpoint{1.155053in}{1.675918in}}{\pgfqpoint{1.147153in}{1.679190in}}{\pgfqpoint{1.138917in}{1.679190in}}%
\pgfpathcurveto{\pgfqpoint{1.130681in}{1.679190in}}{\pgfqpoint{1.122781in}{1.675918in}}{\pgfqpoint{1.116957in}{1.670094in}}%
\pgfpathcurveto{\pgfqpoint{1.111133in}{1.664270in}}{\pgfqpoint{1.107860in}{1.656370in}}{\pgfqpoint{1.107860in}{1.648134in}}%
\pgfpathcurveto{\pgfqpoint{1.107860in}{1.639898in}}{\pgfqpoint{1.111133in}{1.631998in}}{\pgfqpoint{1.116957in}{1.626174in}}%
\pgfpathcurveto{\pgfqpoint{1.122781in}{1.620350in}}{\pgfqpoint{1.130681in}{1.617077in}}{\pgfqpoint{1.138917in}{1.617077in}}%
\pgfpathclose%
\pgfusepath{stroke,fill}%
\end{pgfscope}%
\begin{pgfscope}%
\pgfpathrectangle{\pgfqpoint{0.100000in}{0.212622in}}{\pgfqpoint{3.696000in}{3.696000in}}%
\pgfusepath{clip}%
\pgfsetbuttcap%
\pgfsetroundjoin%
\definecolor{currentfill}{rgb}{0.121569,0.466667,0.705882}%
\pgfsetfillcolor{currentfill}%
\pgfsetfillopacity{0.300799}%
\pgfsetlinewidth{1.003750pt}%
\definecolor{currentstroke}{rgb}{0.121569,0.466667,0.705882}%
\pgfsetstrokecolor{currentstroke}%
\pgfsetstrokeopacity{0.300799}%
\pgfsetdash{}{0pt}%
\pgfpathmoveto{\pgfqpoint{1.138917in}{1.617077in}}%
\pgfpathcurveto{\pgfqpoint{1.147153in}{1.617077in}}{\pgfqpoint{1.155053in}{1.620350in}}{\pgfqpoint{1.160877in}{1.626174in}}%
\pgfpathcurveto{\pgfqpoint{1.166701in}{1.631998in}}{\pgfqpoint{1.169973in}{1.639898in}}{\pgfqpoint{1.169973in}{1.648134in}}%
\pgfpathcurveto{\pgfqpoint{1.169973in}{1.656370in}}{\pgfqpoint{1.166701in}{1.664270in}}{\pgfqpoint{1.160877in}{1.670094in}}%
\pgfpathcurveto{\pgfqpoint{1.155053in}{1.675918in}}{\pgfqpoint{1.147153in}{1.679190in}}{\pgfqpoint{1.138917in}{1.679190in}}%
\pgfpathcurveto{\pgfqpoint{1.130681in}{1.679190in}}{\pgfqpoint{1.122781in}{1.675918in}}{\pgfqpoint{1.116957in}{1.670094in}}%
\pgfpathcurveto{\pgfqpoint{1.111133in}{1.664270in}}{\pgfqpoint{1.107860in}{1.656370in}}{\pgfqpoint{1.107860in}{1.648134in}}%
\pgfpathcurveto{\pgfqpoint{1.107860in}{1.639898in}}{\pgfqpoint{1.111133in}{1.631998in}}{\pgfqpoint{1.116957in}{1.626174in}}%
\pgfpathcurveto{\pgfqpoint{1.122781in}{1.620350in}}{\pgfqpoint{1.130681in}{1.617077in}}{\pgfqpoint{1.138917in}{1.617077in}}%
\pgfpathclose%
\pgfusepath{stroke,fill}%
\end{pgfscope}%
\begin{pgfscope}%
\pgfpathrectangle{\pgfqpoint{0.100000in}{0.212622in}}{\pgfqpoint{3.696000in}{3.696000in}}%
\pgfusepath{clip}%
\pgfsetbuttcap%
\pgfsetroundjoin%
\definecolor{currentfill}{rgb}{0.121569,0.466667,0.705882}%
\pgfsetfillcolor{currentfill}%
\pgfsetfillopacity{0.300799}%
\pgfsetlinewidth{1.003750pt}%
\definecolor{currentstroke}{rgb}{0.121569,0.466667,0.705882}%
\pgfsetstrokecolor{currentstroke}%
\pgfsetstrokeopacity{0.300799}%
\pgfsetdash{}{0pt}%
\pgfpathmoveto{\pgfqpoint{1.138917in}{1.617077in}}%
\pgfpathcurveto{\pgfqpoint{1.147153in}{1.617077in}}{\pgfqpoint{1.155053in}{1.620350in}}{\pgfqpoint{1.160877in}{1.626174in}}%
\pgfpathcurveto{\pgfqpoint{1.166701in}{1.631998in}}{\pgfqpoint{1.169973in}{1.639898in}}{\pgfqpoint{1.169973in}{1.648134in}}%
\pgfpathcurveto{\pgfqpoint{1.169973in}{1.656370in}}{\pgfqpoint{1.166701in}{1.664270in}}{\pgfqpoint{1.160877in}{1.670094in}}%
\pgfpathcurveto{\pgfqpoint{1.155053in}{1.675918in}}{\pgfqpoint{1.147153in}{1.679190in}}{\pgfqpoint{1.138917in}{1.679190in}}%
\pgfpathcurveto{\pgfqpoint{1.130681in}{1.679190in}}{\pgfqpoint{1.122781in}{1.675918in}}{\pgfqpoint{1.116957in}{1.670094in}}%
\pgfpathcurveto{\pgfqpoint{1.111133in}{1.664270in}}{\pgfqpoint{1.107860in}{1.656370in}}{\pgfqpoint{1.107860in}{1.648134in}}%
\pgfpathcurveto{\pgfqpoint{1.107860in}{1.639898in}}{\pgfqpoint{1.111133in}{1.631998in}}{\pgfqpoint{1.116957in}{1.626174in}}%
\pgfpathcurveto{\pgfqpoint{1.122781in}{1.620350in}}{\pgfqpoint{1.130681in}{1.617077in}}{\pgfqpoint{1.138917in}{1.617077in}}%
\pgfpathclose%
\pgfusepath{stroke,fill}%
\end{pgfscope}%
\begin{pgfscope}%
\pgfpathrectangle{\pgfqpoint{0.100000in}{0.212622in}}{\pgfqpoint{3.696000in}{3.696000in}}%
\pgfusepath{clip}%
\pgfsetbuttcap%
\pgfsetroundjoin%
\definecolor{currentfill}{rgb}{0.121569,0.466667,0.705882}%
\pgfsetfillcolor{currentfill}%
\pgfsetfillopacity{0.300799}%
\pgfsetlinewidth{1.003750pt}%
\definecolor{currentstroke}{rgb}{0.121569,0.466667,0.705882}%
\pgfsetstrokecolor{currentstroke}%
\pgfsetstrokeopacity{0.300799}%
\pgfsetdash{}{0pt}%
\pgfpathmoveto{\pgfqpoint{1.138917in}{1.617077in}}%
\pgfpathcurveto{\pgfqpoint{1.147153in}{1.617077in}}{\pgfqpoint{1.155053in}{1.620350in}}{\pgfqpoint{1.160877in}{1.626174in}}%
\pgfpathcurveto{\pgfqpoint{1.166701in}{1.631998in}}{\pgfqpoint{1.169973in}{1.639898in}}{\pgfqpoint{1.169973in}{1.648134in}}%
\pgfpathcurveto{\pgfqpoint{1.169973in}{1.656370in}}{\pgfqpoint{1.166701in}{1.664270in}}{\pgfqpoint{1.160877in}{1.670094in}}%
\pgfpathcurveto{\pgfqpoint{1.155053in}{1.675918in}}{\pgfqpoint{1.147153in}{1.679190in}}{\pgfqpoint{1.138917in}{1.679190in}}%
\pgfpathcurveto{\pgfqpoint{1.130681in}{1.679190in}}{\pgfqpoint{1.122781in}{1.675918in}}{\pgfqpoint{1.116957in}{1.670094in}}%
\pgfpathcurveto{\pgfqpoint{1.111133in}{1.664270in}}{\pgfqpoint{1.107860in}{1.656370in}}{\pgfqpoint{1.107860in}{1.648134in}}%
\pgfpathcurveto{\pgfqpoint{1.107860in}{1.639898in}}{\pgfqpoint{1.111133in}{1.631998in}}{\pgfqpoint{1.116957in}{1.626174in}}%
\pgfpathcurveto{\pgfqpoint{1.122781in}{1.620350in}}{\pgfqpoint{1.130681in}{1.617077in}}{\pgfqpoint{1.138917in}{1.617077in}}%
\pgfpathclose%
\pgfusepath{stroke,fill}%
\end{pgfscope}%
\begin{pgfscope}%
\pgfpathrectangle{\pgfqpoint{0.100000in}{0.212622in}}{\pgfqpoint{3.696000in}{3.696000in}}%
\pgfusepath{clip}%
\pgfsetbuttcap%
\pgfsetroundjoin%
\definecolor{currentfill}{rgb}{0.121569,0.466667,0.705882}%
\pgfsetfillcolor{currentfill}%
\pgfsetfillopacity{0.300799}%
\pgfsetlinewidth{1.003750pt}%
\definecolor{currentstroke}{rgb}{0.121569,0.466667,0.705882}%
\pgfsetstrokecolor{currentstroke}%
\pgfsetstrokeopacity{0.300799}%
\pgfsetdash{}{0pt}%
\pgfpathmoveto{\pgfqpoint{1.138917in}{1.617077in}}%
\pgfpathcurveto{\pgfqpoint{1.147153in}{1.617077in}}{\pgfqpoint{1.155053in}{1.620350in}}{\pgfqpoint{1.160877in}{1.626174in}}%
\pgfpathcurveto{\pgfqpoint{1.166701in}{1.631998in}}{\pgfqpoint{1.169973in}{1.639898in}}{\pgfqpoint{1.169973in}{1.648134in}}%
\pgfpathcurveto{\pgfqpoint{1.169973in}{1.656370in}}{\pgfqpoint{1.166701in}{1.664270in}}{\pgfqpoint{1.160877in}{1.670094in}}%
\pgfpathcurveto{\pgfqpoint{1.155053in}{1.675918in}}{\pgfqpoint{1.147153in}{1.679190in}}{\pgfqpoint{1.138917in}{1.679190in}}%
\pgfpathcurveto{\pgfqpoint{1.130681in}{1.679190in}}{\pgfqpoint{1.122781in}{1.675918in}}{\pgfqpoint{1.116957in}{1.670094in}}%
\pgfpathcurveto{\pgfqpoint{1.111133in}{1.664270in}}{\pgfqpoint{1.107860in}{1.656370in}}{\pgfqpoint{1.107860in}{1.648134in}}%
\pgfpathcurveto{\pgfqpoint{1.107860in}{1.639898in}}{\pgfqpoint{1.111133in}{1.631998in}}{\pgfqpoint{1.116957in}{1.626174in}}%
\pgfpathcurveto{\pgfqpoint{1.122781in}{1.620350in}}{\pgfqpoint{1.130681in}{1.617077in}}{\pgfqpoint{1.138917in}{1.617077in}}%
\pgfpathclose%
\pgfusepath{stroke,fill}%
\end{pgfscope}%
\begin{pgfscope}%
\pgfpathrectangle{\pgfqpoint{0.100000in}{0.212622in}}{\pgfqpoint{3.696000in}{3.696000in}}%
\pgfusepath{clip}%
\pgfsetbuttcap%
\pgfsetroundjoin%
\definecolor{currentfill}{rgb}{0.121569,0.466667,0.705882}%
\pgfsetfillcolor{currentfill}%
\pgfsetfillopacity{0.300799}%
\pgfsetlinewidth{1.003750pt}%
\definecolor{currentstroke}{rgb}{0.121569,0.466667,0.705882}%
\pgfsetstrokecolor{currentstroke}%
\pgfsetstrokeopacity{0.300799}%
\pgfsetdash{}{0pt}%
\pgfpathmoveto{\pgfqpoint{1.138917in}{1.617077in}}%
\pgfpathcurveto{\pgfqpoint{1.147153in}{1.617077in}}{\pgfqpoint{1.155053in}{1.620350in}}{\pgfqpoint{1.160877in}{1.626174in}}%
\pgfpathcurveto{\pgfqpoint{1.166701in}{1.631998in}}{\pgfqpoint{1.169973in}{1.639898in}}{\pgfqpoint{1.169973in}{1.648134in}}%
\pgfpathcurveto{\pgfqpoint{1.169973in}{1.656370in}}{\pgfqpoint{1.166701in}{1.664270in}}{\pgfqpoint{1.160877in}{1.670094in}}%
\pgfpathcurveto{\pgfqpoint{1.155053in}{1.675918in}}{\pgfqpoint{1.147153in}{1.679190in}}{\pgfqpoint{1.138917in}{1.679190in}}%
\pgfpathcurveto{\pgfqpoint{1.130681in}{1.679190in}}{\pgfqpoint{1.122781in}{1.675918in}}{\pgfqpoint{1.116957in}{1.670094in}}%
\pgfpathcurveto{\pgfqpoint{1.111133in}{1.664270in}}{\pgfqpoint{1.107860in}{1.656370in}}{\pgfqpoint{1.107860in}{1.648134in}}%
\pgfpathcurveto{\pgfqpoint{1.107860in}{1.639898in}}{\pgfqpoint{1.111133in}{1.631998in}}{\pgfqpoint{1.116957in}{1.626174in}}%
\pgfpathcurveto{\pgfqpoint{1.122781in}{1.620350in}}{\pgfqpoint{1.130681in}{1.617077in}}{\pgfqpoint{1.138917in}{1.617077in}}%
\pgfpathclose%
\pgfusepath{stroke,fill}%
\end{pgfscope}%
\begin{pgfscope}%
\pgfpathrectangle{\pgfqpoint{0.100000in}{0.212622in}}{\pgfqpoint{3.696000in}{3.696000in}}%
\pgfusepath{clip}%
\pgfsetbuttcap%
\pgfsetroundjoin%
\definecolor{currentfill}{rgb}{0.121569,0.466667,0.705882}%
\pgfsetfillcolor{currentfill}%
\pgfsetfillopacity{0.300799}%
\pgfsetlinewidth{1.003750pt}%
\definecolor{currentstroke}{rgb}{0.121569,0.466667,0.705882}%
\pgfsetstrokecolor{currentstroke}%
\pgfsetstrokeopacity{0.300799}%
\pgfsetdash{}{0pt}%
\pgfpathmoveto{\pgfqpoint{1.138917in}{1.617077in}}%
\pgfpathcurveto{\pgfqpoint{1.147153in}{1.617077in}}{\pgfqpoint{1.155053in}{1.620350in}}{\pgfqpoint{1.160877in}{1.626174in}}%
\pgfpathcurveto{\pgfqpoint{1.166701in}{1.631998in}}{\pgfqpoint{1.169973in}{1.639898in}}{\pgfqpoint{1.169973in}{1.648134in}}%
\pgfpathcurveto{\pgfqpoint{1.169973in}{1.656370in}}{\pgfqpoint{1.166701in}{1.664270in}}{\pgfqpoint{1.160877in}{1.670094in}}%
\pgfpathcurveto{\pgfqpoint{1.155053in}{1.675918in}}{\pgfqpoint{1.147153in}{1.679190in}}{\pgfqpoint{1.138917in}{1.679190in}}%
\pgfpathcurveto{\pgfqpoint{1.130681in}{1.679190in}}{\pgfqpoint{1.122781in}{1.675918in}}{\pgfqpoint{1.116957in}{1.670094in}}%
\pgfpathcurveto{\pgfqpoint{1.111133in}{1.664270in}}{\pgfqpoint{1.107860in}{1.656370in}}{\pgfqpoint{1.107860in}{1.648134in}}%
\pgfpathcurveto{\pgfqpoint{1.107860in}{1.639898in}}{\pgfqpoint{1.111133in}{1.631998in}}{\pgfqpoint{1.116957in}{1.626174in}}%
\pgfpathcurveto{\pgfqpoint{1.122781in}{1.620350in}}{\pgfqpoint{1.130681in}{1.617077in}}{\pgfqpoint{1.138917in}{1.617077in}}%
\pgfpathclose%
\pgfusepath{stroke,fill}%
\end{pgfscope}%
\begin{pgfscope}%
\pgfpathrectangle{\pgfqpoint{0.100000in}{0.212622in}}{\pgfqpoint{3.696000in}{3.696000in}}%
\pgfusepath{clip}%
\pgfsetbuttcap%
\pgfsetroundjoin%
\definecolor{currentfill}{rgb}{0.121569,0.466667,0.705882}%
\pgfsetfillcolor{currentfill}%
\pgfsetfillopacity{0.300799}%
\pgfsetlinewidth{1.003750pt}%
\definecolor{currentstroke}{rgb}{0.121569,0.466667,0.705882}%
\pgfsetstrokecolor{currentstroke}%
\pgfsetstrokeopacity{0.300799}%
\pgfsetdash{}{0pt}%
\pgfpathmoveto{\pgfqpoint{1.138917in}{1.617077in}}%
\pgfpathcurveto{\pgfqpoint{1.147153in}{1.617077in}}{\pgfqpoint{1.155053in}{1.620350in}}{\pgfqpoint{1.160877in}{1.626174in}}%
\pgfpathcurveto{\pgfqpoint{1.166701in}{1.631998in}}{\pgfqpoint{1.169973in}{1.639898in}}{\pgfqpoint{1.169973in}{1.648134in}}%
\pgfpathcurveto{\pgfqpoint{1.169973in}{1.656370in}}{\pgfqpoint{1.166701in}{1.664270in}}{\pgfqpoint{1.160877in}{1.670094in}}%
\pgfpathcurveto{\pgfqpoint{1.155053in}{1.675918in}}{\pgfqpoint{1.147153in}{1.679190in}}{\pgfqpoint{1.138917in}{1.679190in}}%
\pgfpathcurveto{\pgfqpoint{1.130681in}{1.679190in}}{\pgfqpoint{1.122781in}{1.675918in}}{\pgfqpoint{1.116957in}{1.670094in}}%
\pgfpathcurveto{\pgfqpoint{1.111133in}{1.664270in}}{\pgfqpoint{1.107860in}{1.656370in}}{\pgfqpoint{1.107860in}{1.648134in}}%
\pgfpathcurveto{\pgfqpoint{1.107860in}{1.639898in}}{\pgfqpoint{1.111133in}{1.631998in}}{\pgfqpoint{1.116957in}{1.626174in}}%
\pgfpathcurveto{\pgfqpoint{1.122781in}{1.620350in}}{\pgfqpoint{1.130681in}{1.617077in}}{\pgfqpoint{1.138917in}{1.617077in}}%
\pgfpathclose%
\pgfusepath{stroke,fill}%
\end{pgfscope}%
\begin{pgfscope}%
\pgfpathrectangle{\pgfqpoint{0.100000in}{0.212622in}}{\pgfqpoint{3.696000in}{3.696000in}}%
\pgfusepath{clip}%
\pgfsetbuttcap%
\pgfsetroundjoin%
\definecolor{currentfill}{rgb}{0.121569,0.466667,0.705882}%
\pgfsetfillcolor{currentfill}%
\pgfsetfillopacity{0.300799}%
\pgfsetlinewidth{1.003750pt}%
\definecolor{currentstroke}{rgb}{0.121569,0.466667,0.705882}%
\pgfsetstrokecolor{currentstroke}%
\pgfsetstrokeopacity{0.300799}%
\pgfsetdash{}{0pt}%
\pgfpathmoveto{\pgfqpoint{1.138917in}{1.617077in}}%
\pgfpathcurveto{\pgfqpoint{1.147153in}{1.617077in}}{\pgfqpoint{1.155053in}{1.620350in}}{\pgfqpoint{1.160877in}{1.626174in}}%
\pgfpathcurveto{\pgfqpoint{1.166701in}{1.631998in}}{\pgfqpoint{1.169973in}{1.639898in}}{\pgfqpoint{1.169973in}{1.648134in}}%
\pgfpathcurveto{\pgfqpoint{1.169973in}{1.656370in}}{\pgfqpoint{1.166701in}{1.664270in}}{\pgfqpoint{1.160877in}{1.670094in}}%
\pgfpathcurveto{\pgfqpoint{1.155053in}{1.675918in}}{\pgfqpoint{1.147153in}{1.679190in}}{\pgfqpoint{1.138917in}{1.679190in}}%
\pgfpathcurveto{\pgfqpoint{1.130681in}{1.679190in}}{\pgfqpoint{1.122781in}{1.675918in}}{\pgfqpoint{1.116957in}{1.670094in}}%
\pgfpathcurveto{\pgfqpoint{1.111133in}{1.664270in}}{\pgfqpoint{1.107860in}{1.656370in}}{\pgfqpoint{1.107860in}{1.648134in}}%
\pgfpathcurveto{\pgfqpoint{1.107860in}{1.639898in}}{\pgfqpoint{1.111133in}{1.631998in}}{\pgfqpoint{1.116957in}{1.626174in}}%
\pgfpathcurveto{\pgfqpoint{1.122781in}{1.620350in}}{\pgfqpoint{1.130681in}{1.617077in}}{\pgfqpoint{1.138917in}{1.617077in}}%
\pgfpathclose%
\pgfusepath{stroke,fill}%
\end{pgfscope}%
\begin{pgfscope}%
\pgfpathrectangle{\pgfqpoint{0.100000in}{0.212622in}}{\pgfqpoint{3.696000in}{3.696000in}}%
\pgfusepath{clip}%
\pgfsetbuttcap%
\pgfsetroundjoin%
\definecolor{currentfill}{rgb}{0.121569,0.466667,0.705882}%
\pgfsetfillcolor{currentfill}%
\pgfsetfillopacity{0.300799}%
\pgfsetlinewidth{1.003750pt}%
\definecolor{currentstroke}{rgb}{0.121569,0.466667,0.705882}%
\pgfsetstrokecolor{currentstroke}%
\pgfsetstrokeopacity{0.300799}%
\pgfsetdash{}{0pt}%
\pgfpathmoveto{\pgfqpoint{1.138917in}{1.617077in}}%
\pgfpathcurveto{\pgfqpoint{1.147153in}{1.617077in}}{\pgfqpoint{1.155053in}{1.620350in}}{\pgfqpoint{1.160877in}{1.626174in}}%
\pgfpathcurveto{\pgfqpoint{1.166701in}{1.631998in}}{\pgfqpoint{1.169973in}{1.639898in}}{\pgfqpoint{1.169973in}{1.648134in}}%
\pgfpathcurveto{\pgfqpoint{1.169973in}{1.656370in}}{\pgfqpoint{1.166701in}{1.664270in}}{\pgfqpoint{1.160877in}{1.670094in}}%
\pgfpathcurveto{\pgfqpoint{1.155053in}{1.675918in}}{\pgfqpoint{1.147153in}{1.679190in}}{\pgfqpoint{1.138917in}{1.679190in}}%
\pgfpathcurveto{\pgfqpoint{1.130681in}{1.679190in}}{\pgfqpoint{1.122781in}{1.675918in}}{\pgfqpoint{1.116957in}{1.670094in}}%
\pgfpathcurveto{\pgfqpoint{1.111133in}{1.664270in}}{\pgfqpoint{1.107860in}{1.656370in}}{\pgfqpoint{1.107860in}{1.648134in}}%
\pgfpathcurveto{\pgfqpoint{1.107860in}{1.639898in}}{\pgfqpoint{1.111133in}{1.631998in}}{\pgfqpoint{1.116957in}{1.626174in}}%
\pgfpathcurveto{\pgfqpoint{1.122781in}{1.620350in}}{\pgfqpoint{1.130681in}{1.617077in}}{\pgfqpoint{1.138917in}{1.617077in}}%
\pgfpathclose%
\pgfusepath{stroke,fill}%
\end{pgfscope}%
\begin{pgfscope}%
\pgfpathrectangle{\pgfqpoint{0.100000in}{0.212622in}}{\pgfqpoint{3.696000in}{3.696000in}}%
\pgfusepath{clip}%
\pgfsetbuttcap%
\pgfsetroundjoin%
\definecolor{currentfill}{rgb}{0.121569,0.466667,0.705882}%
\pgfsetfillcolor{currentfill}%
\pgfsetfillopacity{0.300799}%
\pgfsetlinewidth{1.003750pt}%
\definecolor{currentstroke}{rgb}{0.121569,0.466667,0.705882}%
\pgfsetstrokecolor{currentstroke}%
\pgfsetstrokeopacity{0.300799}%
\pgfsetdash{}{0pt}%
\pgfpathmoveto{\pgfqpoint{1.138917in}{1.617077in}}%
\pgfpathcurveto{\pgfqpoint{1.147153in}{1.617077in}}{\pgfqpoint{1.155053in}{1.620350in}}{\pgfqpoint{1.160877in}{1.626174in}}%
\pgfpathcurveto{\pgfqpoint{1.166701in}{1.631998in}}{\pgfqpoint{1.169973in}{1.639898in}}{\pgfqpoint{1.169973in}{1.648134in}}%
\pgfpathcurveto{\pgfqpoint{1.169973in}{1.656370in}}{\pgfqpoint{1.166701in}{1.664270in}}{\pgfqpoint{1.160877in}{1.670094in}}%
\pgfpathcurveto{\pgfqpoint{1.155053in}{1.675918in}}{\pgfqpoint{1.147153in}{1.679190in}}{\pgfqpoint{1.138917in}{1.679190in}}%
\pgfpathcurveto{\pgfqpoint{1.130681in}{1.679190in}}{\pgfqpoint{1.122781in}{1.675918in}}{\pgfqpoint{1.116957in}{1.670094in}}%
\pgfpathcurveto{\pgfqpoint{1.111133in}{1.664270in}}{\pgfqpoint{1.107860in}{1.656370in}}{\pgfqpoint{1.107860in}{1.648134in}}%
\pgfpathcurveto{\pgfqpoint{1.107860in}{1.639898in}}{\pgfqpoint{1.111133in}{1.631998in}}{\pgfqpoint{1.116957in}{1.626174in}}%
\pgfpathcurveto{\pgfqpoint{1.122781in}{1.620350in}}{\pgfqpoint{1.130681in}{1.617077in}}{\pgfqpoint{1.138917in}{1.617077in}}%
\pgfpathclose%
\pgfusepath{stroke,fill}%
\end{pgfscope}%
\begin{pgfscope}%
\pgfpathrectangle{\pgfqpoint{0.100000in}{0.212622in}}{\pgfqpoint{3.696000in}{3.696000in}}%
\pgfusepath{clip}%
\pgfsetbuttcap%
\pgfsetroundjoin%
\definecolor{currentfill}{rgb}{0.121569,0.466667,0.705882}%
\pgfsetfillcolor{currentfill}%
\pgfsetfillopacity{0.300799}%
\pgfsetlinewidth{1.003750pt}%
\definecolor{currentstroke}{rgb}{0.121569,0.466667,0.705882}%
\pgfsetstrokecolor{currentstroke}%
\pgfsetstrokeopacity{0.300799}%
\pgfsetdash{}{0pt}%
\pgfpathmoveto{\pgfqpoint{1.138917in}{1.617077in}}%
\pgfpathcurveto{\pgfqpoint{1.147153in}{1.617077in}}{\pgfqpoint{1.155053in}{1.620350in}}{\pgfqpoint{1.160877in}{1.626174in}}%
\pgfpathcurveto{\pgfqpoint{1.166701in}{1.631998in}}{\pgfqpoint{1.169973in}{1.639898in}}{\pgfqpoint{1.169973in}{1.648134in}}%
\pgfpathcurveto{\pgfqpoint{1.169973in}{1.656370in}}{\pgfqpoint{1.166701in}{1.664270in}}{\pgfqpoint{1.160877in}{1.670094in}}%
\pgfpathcurveto{\pgfqpoint{1.155053in}{1.675918in}}{\pgfqpoint{1.147153in}{1.679190in}}{\pgfqpoint{1.138917in}{1.679190in}}%
\pgfpathcurveto{\pgfqpoint{1.130681in}{1.679190in}}{\pgfqpoint{1.122781in}{1.675918in}}{\pgfqpoint{1.116957in}{1.670094in}}%
\pgfpathcurveto{\pgfqpoint{1.111133in}{1.664270in}}{\pgfqpoint{1.107860in}{1.656370in}}{\pgfqpoint{1.107860in}{1.648134in}}%
\pgfpathcurveto{\pgfqpoint{1.107860in}{1.639898in}}{\pgfqpoint{1.111133in}{1.631998in}}{\pgfqpoint{1.116957in}{1.626174in}}%
\pgfpathcurveto{\pgfqpoint{1.122781in}{1.620350in}}{\pgfqpoint{1.130681in}{1.617077in}}{\pgfqpoint{1.138917in}{1.617077in}}%
\pgfpathclose%
\pgfusepath{stroke,fill}%
\end{pgfscope}%
\begin{pgfscope}%
\pgfpathrectangle{\pgfqpoint{0.100000in}{0.212622in}}{\pgfqpoint{3.696000in}{3.696000in}}%
\pgfusepath{clip}%
\pgfsetbuttcap%
\pgfsetroundjoin%
\definecolor{currentfill}{rgb}{0.121569,0.466667,0.705882}%
\pgfsetfillcolor{currentfill}%
\pgfsetfillopacity{0.300799}%
\pgfsetlinewidth{1.003750pt}%
\definecolor{currentstroke}{rgb}{0.121569,0.466667,0.705882}%
\pgfsetstrokecolor{currentstroke}%
\pgfsetstrokeopacity{0.300799}%
\pgfsetdash{}{0pt}%
\pgfpathmoveto{\pgfqpoint{1.138917in}{1.617077in}}%
\pgfpathcurveto{\pgfqpoint{1.147153in}{1.617077in}}{\pgfqpoint{1.155053in}{1.620350in}}{\pgfqpoint{1.160877in}{1.626174in}}%
\pgfpathcurveto{\pgfqpoint{1.166701in}{1.631998in}}{\pgfqpoint{1.169973in}{1.639898in}}{\pgfqpoint{1.169973in}{1.648134in}}%
\pgfpathcurveto{\pgfqpoint{1.169973in}{1.656370in}}{\pgfqpoint{1.166701in}{1.664270in}}{\pgfqpoint{1.160877in}{1.670094in}}%
\pgfpathcurveto{\pgfqpoint{1.155053in}{1.675918in}}{\pgfqpoint{1.147153in}{1.679190in}}{\pgfqpoint{1.138917in}{1.679190in}}%
\pgfpathcurveto{\pgfqpoint{1.130681in}{1.679190in}}{\pgfqpoint{1.122781in}{1.675918in}}{\pgfqpoint{1.116957in}{1.670094in}}%
\pgfpathcurveto{\pgfqpoint{1.111133in}{1.664270in}}{\pgfqpoint{1.107860in}{1.656370in}}{\pgfqpoint{1.107860in}{1.648134in}}%
\pgfpathcurveto{\pgfqpoint{1.107860in}{1.639898in}}{\pgfqpoint{1.111133in}{1.631998in}}{\pgfqpoint{1.116957in}{1.626174in}}%
\pgfpathcurveto{\pgfqpoint{1.122781in}{1.620350in}}{\pgfqpoint{1.130681in}{1.617077in}}{\pgfqpoint{1.138917in}{1.617077in}}%
\pgfpathclose%
\pgfusepath{stroke,fill}%
\end{pgfscope}%
\begin{pgfscope}%
\pgfpathrectangle{\pgfqpoint{0.100000in}{0.212622in}}{\pgfqpoint{3.696000in}{3.696000in}}%
\pgfusepath{clip}%
\pgfsetbuttcap%
\pgfsetroundjoin%
\definecolor{currentfill}{rgb}{0.121569,0.466667,0.705882}%
\pgfsetfillcolor{currentfill}%
\pgfsetfillopacity{0.300799}%
\pgfsetlinewidth{1.003750pt}%
\definecolor{currentstroke}{rgb}{0.121569,0.466667,0.705882}%
\pgfsetstrokecolor{currentstroke}%
\pgfsetstrokeopacity{0.300799}%
\pgfsetdash{}{0pt}%
\pgfpathmoveto{\pgfqpoint{1.138917in}{1.617077in}}%
\pgfpathcurveto{\pgfqpoint{1.147153in}{1.617077in}}{\pgfqpoint{1.155053in}{1.620350in}}{\pgfqpoint{1.160877in}{1.626174in}}%
\pgfpathcurveto{\pgfqpoint{1.166701in}{1.631998in}}{\pgfqpoint{1.169973in}{1.639898in}}{\pgfqpoint{1.169973in}{1.648134in}}%
\pgfpathcurveto{\pgfqpoint{1.169973in}{1.656370in}}{\pgfqpoint{1.166701in}{1.664270in}}{\pgfqpoint{1.160877in}{1.670094in}}%
\pgfpathcurveto{\pgfqpoint{1.155053in}{1.675918in}}{\pgfqpoint{1.147153in}{1.679190in}}{\pgfqpoint{1.138917in}{1.679190in}}%
\pgfpathcurveto{\pgfqpoint{1.130681in}{1.679190in}}{\pgfqpoint{1.122781in}{1.675918in}}{\pgfqpoint{1.116957in}{1.670094in}}%
\pgfpathcurveto{\pgfqpoint{1.111133in}{1.664270in}}{\pgfqpoint{1.107860in}{1.656370in}}{\pgfqpoint{1.107860in}{1.648134in}}%
\pgfpathcurveto{\pgfqpoint{1.107860in}{1.639898in}}{\pgfqpoint{1.111133in}{1.631998in}}{\pgfqpoint{1.116957in}{1.626174in}}%
\pgfpathcurveto{\pgfqpoint{1.122781in}{1.620350in}}{\pgfqpoint{1.130681in}{1.617077in}}{\pgfqpoint{1.138917in}{1.617077in}}%
\pgfpathclose%
\pgfusepath{stroke,fill}%
\end{pgfscope}%
\begin{pgfscope}%
\pgfpathrectangle{\pgfqpoint{0.100000in}{0.212622in}}{\pgfqpoint{3.696000in}{3.696000in}}%
\pgfusepath{clip}%
\pgfsetbuttcap%
\pgfsetroundjoin%
\definecolor{currentfill}{rgb}{0.121569,0.466667,0.705882}%
\pgfsetfillcolor{currentfill}%
\pgfsetfillopacity{0.300799}%
\pgfsetlinewidth{1.003750pt}%
\definecolor{currentstroke}{rgb}{0.121569,0.466667,0.705882}%
\pgfsetstrokecolor{currentstroke}%
\pgfsetstrokeopacity{0.300799}%
\pgfsetdash{}{0pt}%
\pgfpathmoveto{\pgfqpoint{1.138917in}{1.617077in}}%
\pgfpathcurveto{\pgfqpoint{1.147153in}{1.617077in}}{\pgfqpoint{1.155053in}{1.620350in}}{\pgfqpoint{1.160877in}{1.626174in}}%
\pgfpathcurveto{\pgfqpoint{1.166701in}{1.631998in}}{\pgfqpoint{1.169973in}{1.639898in}}{\pgfqpoint{1.169973in}{1.648134in}}%
\pgfpathcurveto{\pgfqpoint{1.169973in}{1.656370in}}{\pgfqpoint{1.166701in}{1.664270in}}{\pgfqpoint{1.160877in}{1.670094in}}%
\pgfpathcurveto{\pgfqpoint{1.155053in}{1.675918in}}{\pgfqpoint{1.147153in}{1.679190in}}{\pgfqpoint{1.138917in}{1.679190in}}%
\pgfpathcurveto{\pgfqpoint{1.130681in}{1.679190in}}{\pgfqpoint{1.122781in}{1.675918in}}{\pgfqpoint{1.116957in}{1.670094in}}%
\pgfpathcurveto{\pgfqpoint{1.111133in}{1.664270in}}{\pgfqpoint{1.107860in}{1.656370in}}{\pgfqpoint{1.107860in}{1.648134in}}%
\pgfpathcurveto{\pgfqpoint{1.107860in}{1.639898in}}{\pgfqpoint{1.111133in}{1.631998in}}{\pgfqpoint{1.116957in}{1.626174in}}%
\pgfpathcurveto{\pgfqpoint{1.122781in}{1.620350in}}{\pgfqpoint{1.130681in}{1.617077in}}{\pgfqpoint{1.138917in}{1.617077in}}%
\pgfpathclose%
\pgfusepath{stroke,fill}%
\end{pgfscope}%
\begin{pgfscope}%
\pgfpathrectangle{\pgfqpoint{0.100000in}{0.212622in}}{\pgfqpoint{3.696000in}{3.696000in}}%
\pgfusepath{clip}%
\pgfsetbuttcap%
\pgfsetroundjoin%
\definecolor{currentfill}{rgb}{0.121569,0.466667,0.705882}%
\pgfsetfillcolor{currentfill}%
\pgfsetfillopacity{0.300799}%
\pgfsetlinewidth{1.003750pt}%
\definecolor{currentstroke}{rgb}{0.121569,0.466667,0.705882}%
\pgfsetstrokecolor{currentstroke}%
\pgfsetstrokeopacity{0.300799}%
\pgfsetdash{}{0pt}%
\pgfpathmoveto{\pgfqpoint{1.138917in}{1.617077in}}%
\pgfpathcurveto{\pgfqpoint{1.147153in}{1.617077in}}{\pgfqpoint{1.155053in}{1.620350in}}{\pgfqpoint{1.160877in}{1.626174in}}%
\pgfpathcurveto{\pgfqpoint{1.166701in}{1.631998in}}{\pgfqpoint{1.169973in}{1.639898in}}{\pgfqpoint{1.169973in}{1.648134in}}%
\pgfpathcurveto{\pgfqpoint{1.169973in}{1.656370in}}{\pgfqpoint{1.166701in}{1.664270in}}{\pgfqpoint{1.160877in}{1.670094in}}%
\pgfpathcurveto{\pgfqpoint{1.155053in}{1.675918in}}{\pgfqpoint{1.147153in}{1.679190in}}{\pgfqpoint{1.138917in}{1.679190in}}%
\pgfpathcurveto{\pgfqpoint{1.130681in}{1.679190in}}{\pgfqpoint{1.122781in}{1.675918in}}{\pgfqpoint{1.116957in}{1.670094in}}%
\pgfpathcurveto{\pgfqpoint{1.111133in}{1.664270in}}{\pgfqpoint{1.107860in}{1.656370in}}{\pgfqpoint{1.107860in}{1.648134in}}%
\pgfpathcurveto{\pgfqpoint{1.107860in}{1.639898in}}{\pgfqpoint{1.111133in}{1.631998in}}{\pgfqpoint{1.116957in}{1.626174in}}%
\pgfpathcurveto{\pgfqpoint{1.122781in}{1.620350in}}{\pgfqpoint{1.130681in}{1.617077in}}{\pgfqpoint{1.138917in}{1.617077in}}%
\pgfpathclose%
\pgfusepath{stroke,fill}%
\end{pgfscope}%
\begin{pgfscope}%
\pgfpathrectangle{\pgfqpoint{0.100000in}{0.212622in}}{\pgfqpoint{3.696000in}{3.696000in}}%
\pgfusepath{clip}%
\pgfsetbuttcap%
\pgfsetroundjoin%
\definecolor{currentfill}{rgb}{0.121569,0.466667,0.705882}%
\pgfsetfillcolor{currentfill}%
\pgfsetfillopacity{0.300799}%
\pgfsetlinewidth{1.003750pt}%
\definecolor{currentstroke}{rgb}{0.121569,0.466667,0.705882}%
\pgfsetstrokecolor{currentstroke}%
\pgfsetstrokeopacity{0.300799}%
\pgfsetdash{}{0pt}%
\pgfpathmoveto{\pgfqpoint{1.138917in}{1.617077in}}%
\pgfpathcurveto{\pgfqpoint{1.147153in}{1.617077in}}{\pgfqpoint{1.155053in}{1.620350in}}{\pgfqpoint{1.160877in}{1.626174in}}%
\pgfpathcurveto{\pgfqpoint{1.166701in}{1.631998in}}{\pgfqpoint{1.169973in}{1.639898in}}{\pgfqpoint{1.169973in}{1.648134in}}%
\pgfpathcurveto{\pgfqpoint{1.169973in}{1.656370in}}{\pgfqpoint{1.166701in}{1.664270in}}{\pgfqpoint{1.160877in}{1.670094in}}%
\pgfpathcurveto{\pgfqpoint{1.155053in}{1.675918in}}{\pgfqpoint{1.147153in}{1.679190in}}{\pgfqpoint{1.138917in}{1.679190in}}%
\pgfpathcurveto{\pgfqpoint{1.130681in}{1.679190in}}{\pgfqpoint{1.122781in}{1.675918in}}{\pgfqpoint{1.116957in}{1.670094in}}%
\pgfpathcurveto{\pgfqpoint{1.111133in}{1.664270in}}{\pgfqpoint{1.107860in}{1.656370in}}{\pgfqpoint{1.107860in}{1.648134in}}%
\pgfpathcurveto{\pgfqpoint{1.107860in}{1.639898in}}{\pgfqpoint{1.111133in}{1.631998in}}{\pgfqpoint{1.116957in}{1.626174in}}%
\pgfpathcurveto{\pgfqpoint{1.122781in}{1.620350in}}{\pgfqpoint{1.130681in}{1.617077in}}{\pgfqpoint{1.138917in}{1.617077in}}%
\pgfpathclose%
\pgfusepath{stroke,fill}%
\end{pgfscope}%
\begin{pgfscope}%
\pgfpathrectangle{\pgfqpoint{0.100000in}{0.212622in}}{\pgfqpoint{3.696000in}{3.696000in}}%
\pgfusepath{clip}%
\pgfsetbuttcap%
\pgfsetroundjoin%
\definecolor{currentfill}{rgb}{0.121569,0.466667,0.705882}%
\pgfsetfillcolor{currentfill}%
\pgfsetfillopacity{0.300799}%
\pgfsetlinewidth{1.003750pt}%
\definecolor{currentstroke}{rgb}{0.121569,0.466667,0.705882}%
\pgfsetstrokecolor{currentstroke}%
\pgfsetstrokeopacity{0.300799}%
\pgfsetdash{}{0pt}%
\pgfpathmoveto{\pgfqpoint{1.138917in}{1.617077in}}%
\pgfpathcurveto{\pgfqpoint{1.147153in}{1.617077in}}{\pgfqpoint{1.155053in}{1.620350in}}{\pgfqpoint{1.160877in}{1.626174in}}%
\pgfpathcurveto{\pgfqpoint{1.166701in}{1.631998in}}{\pgfqpoint{1.169973in}{1.639898in}}{\pgfqpoint{1.169973in}{1.648134in}}%
\pgfpathcurveto{\pgfqpoint{1.169973in}{1.656370in}}{\pgfqpoint{1.166701in}{1.664270in}}{\pgfqpoint{1.160877in}{1.670094in}}%
\pgfpathcurveto{\pgfqpoint{1.155053in}{1.675918in}}{\pgfqpoint{1.147153in}{1.679190in}}{\pgfqpoint{1.138917in}{1.679190in}}%
\pgfpathcurveto{\pgfqpoint{1.130681in}{1.679190in}}{\pgfqpoint{1.122781in}{1.675918in}}{\pgfqpoint{1.116957in}{1.670094in}}%
\pgfpathcurveto{\pgfqpoint{1.111133in}{1.664270in}}{\pgfqpoint{1.107860in}{1.656370in}}{\pgfqpoint{1.107860in}{1.648134in}}%
\pgfpathcurveto{\pgfqpoint{1.107860in}{1.639898in}}{\pgfqpoint{1.111133in}{1.631998in}}{\pgfqpoint{1.116957in}{1.626174in}}%
\pgfpathcurveto{\pgfqpoint{1.122781in}{1.620350in}}{\pgfqpoint{1.130681in}{1.617077in}}{\pgfqpoint{1.138917in}{1.617077in}}%
\pgfpathclose%
\pgfusepath{stroke,fill}%
\end{pgfscope}%
\begin{pgfscope}%
\pgfpathrectangle{\pgfqpoint{0.100000in}{0.212622in}}{\pgfqpoint{3.696000in}{3.696000in}}%
\pgfusepath{clip}%
\pgfsetbuttcap%
\pgfsetroundjoin%
\definecolor{currentfill}{rgb}{0.121569,0.466667,0.705882}%
\pgfsetfillcolor{currentfill}%
\pgfsetfillopacity{0.300799}%
\pgfsetlinewidth{1.003750pt}%
\definecolor{currentstroke}{rgb}{0.121569,0.466667,0.705882}%
\pgfsetstrokecolor{currentstroke}%
\pgfsetstrokeopacity{0.300799}%
\pgfsetdash{}{0pt}%
\pgfpathmoveto{\pgfqpoint{1.138917in}{1.617077in}}%
\pgfpathcurveto{\pgfqpoint{1.147153in}{1.617077in}}{\pgfqpoint{1.155053in}{1.620350in}}{\pgfqpoint{1.160877in}{1.626174in}}%
\pgfpathcurveto{\pgfqpoint{1.166701in}{1.631998in}}{\pgfqpoint{1.169973in}{1.639898in}}{\pgfqpoint{1.169973in}{1.648134in}}%
\pgfpathcurveto{\pgfqpoint{1.169973in}{1.656370in}}{\pgfqpoint{1.166701in}{1.664270in}}{\pgfqpoint{1.160877in}{1.670094in}}%
\pgfpathcurveto{\pgfqpoint{1.155053in}{1.675918in}}{\pgfqpoint{1.147153in}{1.679190in}}{\pgfqpoint{1.138917in}{1.679190in}}%
\pgfpathcurveto{\pgfqpoint{1.130681in}{1.679190in}}{\pgfqpoint{1.122781in}{1.675918in}}{\pgfqpoint{1.116957in}{1.670094in}}%
\pgfpathcurveto{\pgfqpoint{1.111133in}{1.664270in}}{\pgfqpoint{1.107860in}{1.656370in}}{\pgfqpoint{1.107860in}{1.648134in}}%
\pgfpathcurveto{\pgfqpoint{1.107860in}{1.639898in}}{\pgfqpoint{1.111133in}{1.631998in}}{\pgfqpoint{1.116957in}{1.626174in}}%
\pgfpathcurveto{\pgfqpoint{1.122781in}{1.620350in}}{\pgfqpoint{1.130681in}{1.617077in}}{\pgfqpoint{1.138917in}{1.617077in}}%
\pgfpathclose%
\pgfusepath{stroke,fill}%
\end{pgfscope}%
\begin{pgfscope}%
\pgfpathrectangle{\pgfqpoint{0.100000in}{0.212622in}}{\pgfqpoint{3.696000in}{3.696000in}}%
\pgfusepath{clip}%
\pgfsetbuttcap%
\pgfsetroundjoin%
\definecolor{currentfill}{rgb}{0.121569,0.466667,0.705882}%
\pgfsetfillcolor{currentfill}%
\pgfsetfillopacity{0.300799}%
\pgfsetlinewidth{1.003750pt}%
\definecolor{currentstroke}{rgb}{0.121569,0.466667,0.705882}%
\pgfsetstrokecolor{currentstroke}%
\pgfsetstrokeopacity{0.300799}%
\pgfsetdash{}{0pt}%
\pgfpathmoveto{\pgfqpoint{1.138917in}{1.617077in}}%
\pgfpathcurveto{\pgfqpoint{1.147153in}{1.617077in}}{\pgfqpoint{1.155053in}{1.620350in}}{\pgfqpoint{1.160877in}{1.626174in}}%
\pgfpathcurveto{\pgfqpoint{1.166701in}{1.631998in}}{\pgfqpoint{1.169973in}{1.639898in}}{\pgfqpoint{1.169973in}{1.648134in}}%
\pgfpathcurveto{\pgfqpoint{1.169973in}{1.656370in}}{\pgfqpoint{1.166701in}{1.664270in}}{\pgfqpoint{1.160877in}{1.670094in}}%
\pgfpathcurveto{\pgfqpoint{1.155053in}{1.675918in}}{\pgfqpoint{1.147153in}{1.679190in}}{\pgfqpoint{1.138917in}{1.679190in}}%
\pgfpathcurveto{\pgfqpoint{1.130681in}{1.679190in}}{\pgfqpoint{1.122781in}{1.675918in}}{\pgfqpoint{1.116957in}{1.670094in}}%
\pgfpathcurveto{\pgfqpoint{1.111133in}{1.664270in}}{\pgfqpoint{1.107860in}{1.656370in}}{\pgfqpoint{1.107860in}{1.648134in}}%
\pgfpathcurveto{\pgfqpoint{1.107860in}{1.639898in}}{\pgfqpoint{1.111133in}{1.631998in}}{\pgfqpoint{1.116957in}{1.626174in}}%
\pgfpathcurveto{\pgfqpoint{1.122781in}{1.620350in}}{\pgfqpoint{1.130681in}{1.617077in}}{\pgfqpoint{1.138917in}{1.617077in}}%
\pgfpathclose%
\pgfusepath{stroke,fill}%
\end{pgfscope}%
\begin{pgfscope}%
\pgfpathrectangle{\pgfqpoint{0.100000in}{0.212622in}}{\pgfqpoint{3.696000in}{3.696000in}}%
\pgfusepath{clip}%
\pgfsetbuttcap%
\pgfsetroundjoin%
\definecolor{currentfill}{rgb}{0.121569,0.466667,0.705882}%
\pgfsetfillcolor{currentfill}%
\pgfsetfillopacity{0.300799}%
\pgfsetlinewidth{1.003750pt}%
\definecolor{currentstroke}{rgb}{0.121569,0.466667,0.705882}%
\pgfsetstrokecolor{currentstroke}%
\pgfsetstrokeopacity{0.300799}%
\pgfsetdash{}{0pt}%
\pgfpathmoveto{\pgfqpoint{1.138917in}{1.617077in}}%
\pgfpathcurveto{\pgfqpoint{1.147153in}{1.617077in}}{\pgfqpoint{1.155053in}{1.620350in}}{\pgfqpoint{1.160877in}{1.626174in}}%
\pgfpathcurveto{\pgfqpoint{1.166701in}{1.631998in}}{\pgfqpoint{1.169973in}{1.639898in}}{\pgfqpoint{1.169973in}{1.648134in}}%
\pgfpathcurveto{\pgfqpoint{1.169973in}{1.656370in}}{\pgfqpoint{1.166701in}{1.664270in}}{\pgfqpoint{1.160877in}{1.670094in}}%
\pgfpathcurveto{\pgfqpoint{1.155053in}{1.675918in}}{\pgfqpoint{1.147153in}{1.679190in}}{\pgfqpoint{1.138917in}{1.679190in}}%
\pgfpathcurveto{\pgfqpoint{1.130681in}{1.679190in}}{\pgfqpoint{1.122781in}{1.675918in}}{\pgfqpoint{1.116957in}{1.670094in}}%
\pgfpathcurveto{\pgfqpoint{1.111133in}{1.664270in}}{\pgfqpoint{1.107860in}{1.656370in}}{\pgfqpoint{1.107860in}{1.648134in}}%
\pgfpathcurveto{\pgfqpoint{1.107860in}{1.639898in}}{\pgfqpoint{1.111133in}{1.631998in}}{\pgfqpoint{1.116957in}{1.626174in}}%
\pgfpathcurveto{\pgfqpoint{1.122781in}{1.620350in}}{\pgfqpoint{1.130681in}{1.617077in}}{\pgfqpoint{1.138917in}{1.617077in}}%
\pgfpathclose%
\pgfusepath{stroke,fill}%
\end{pgfscope}%
\begin{pgfscope}%
\pgfpathrectangle{\pgfqpoint{0.100000in}{0.212622in}}{\pgfqpoint{3.696000in}{3.696000in}}%
\pgfusepath{clip}%
\pgfsetbuttcap%
\pgfsetroundjoin%
\definecolor{currentfill}{rgb}{0.121569,0.466667,0.705882}%
\pgfsetfillcolor{currentfill}%
\pgfsetfillopacity{0.300799}%
\pgfsetlinewidth{1.003750pt}%
\definecolor{currentstroke}{rgb}{0.121569,0.466667,0.705882}%
\pgfsetstrokecolor{currentstroke}%
\pgfsetstrokeopacity{0.300799}%
\pgfsetdash{}{0pt}%
\pgfpathmoveto{\pgfqpoint{1.138917in}{1.617077in}}%
\pgfpathcurveto{\pgfqpoint{1.147153in}{1.617077in}}{\pgfqpoint{1.155053in}{1.620350in}}{\pgfqpoint{1.160877in}{1.626174in}}%
\pgfpathcurveto{\pgfqpoint{1.166701in}{1.631998in}}{\pgfqpoint{1.169973in}{1.639898in}}{\pgfqpoint{1.169973in}{1.648134in}}%
\pgfpathcurveto{\pgfqpoint{1.169973in}{1.656370in}}{\pgfqpoint{1.166701in}{1.664270in}}{\pgfqpoint{1.160877in}{1.670094in}}%
\pgfpathcurveto{\pgfqpoint{1.155053in}{1.675918in}}{\pgfqpoint{1.147153in}{1.679190in}}{\pgfqpoint{1.138917in}{1.679190in}}%
\pgfpathcurveto{\pgfqpoint{1.130681in}{1.679190in}}{\pgfqpoint{1.122781in}{1.675918in}}{\pgfqpoint{1.116957in}{1.670094in}}%
\pgfpathcurveto{\pgfqpoint{1.111133in}{1.664270in}}{\pgfqpoint{1.107860in}{1.656370in}}{\pgfqpoint{1.107860in}{1.648134in}}%
\pgfpathcurveto{\pgfqpoint{1.107860in}{1.639898in}}{\pgfqpoint{1.111133in}{1.631998in}}{\pgfqpoint{1.116957in}{1.626174in}}%
\pgfpathcurveto{\pgfqpoint{1.122781in}{1.620350in}}{\pgfqpoint{1.130681in}{1.617077in}}{\pgfqpoint{1.138917in}{1.617077in}}%
\pgfpathclose%
\pgfusepath{stroke,fill}%
\end{pgfscope}%
\begin{pgfscope}%
\pgfpathrectangle{\pgfqpoint{0.100000in}{0.212622in}}{\pgfqpoint{3.696000in}{3.696000in}}%
\pgfusepath{clip}%
\pgfsetbuttcap%
\pgfsetroundjoin%
\definecolor{currentfill}{rgb}{0.121569,0.466667,0.705882}%
\pgfsetfillcolor{currentfill}%
\pgfsetfillopacity{0.300799}%
\pgfsetlinewidth{1.003750pt}%
\definecolor{currentstroke}{rgb}{0.121569,0.466667,0.705882}%
\pgfsetstrokecolor{currentstroke}%
\pgfsetstrokeopacity{0.300799}%
\pgfsetdash{}{0pt}%
\pgfpathmoveto{\pgfqpoint{1.138917in}{1.617077in}}%
\pgfpathcurveto{\pgfqpoint{1.147153in}{1.617077in}}{\pgfqpoint{1.155053in}{1.620350in}}{\pgfqpoint{1.160877in}{1.626174in}}%
\pgfpathcurveto{\pgfqpoint{1.166701in}{1.631998in}}{\pgfqpoint{1.169973in}{1.639898in}}{\pgfqpoint{1.169973in}{1.648134in}}%
\pgfpathcurveto{\pgfqpoint{1.169973in}{1.656370in}}{\pgfqpoint{1.166701in}{1.664270in}}{\pgfqpoint{1.160877in}{1.670094in}}%
\pgfpathcurveto{\pgfqpoint{1.155053in}{1.675918in}}{\pgfqpoint{1.147153in}{1.679190in}}{\pgfqpoint{1.138917in}{1.679190in}}%
\pgfpathcurveto{\pgfqpoint{1.130681in}{1.679190in}}{\pgfqpoint{1.122781in}{1.675918in}}{\pgfqpoint{1.116957in}{1.670094in}}%
\pgfpathcurveto{\pgfqpoint{1.111133in}{1.664270in}}{\pgfqpoint{1.107860in}{1.656370in}}{\pgfqpoint{1.107860in}{1.648134in}}%
\pgfpathcurveto{\pgfqpoint{1.107860in}{1.639898in}}{\pgfqpoint{1.111133in}{1.631998in}}{\pgfqpoint{1.116957in}{1.626174in}}%
\pgfpathcurveto{\pgfqpoint{1.122781in}{1.620350in}}{\pgfqpoint{1.130681in}{1.617077in}}{\pgfqpoint{1.138917in}{1.617077in}}%
\pgfpathclose%
\pgfusepath{stroke,fill}%
\end{pgfscope}%
\begin{pgfscope}%
\pgfpathrectangle{\pgfqpoint{0.100000in}{0.212622in}}{\pgfqpoint{3.696000in}{3.696000in}}%
\pgfusepath{clip}%
\pgfsetbuttcap%
\pgfsetroundjoin%
\definecolor{currentfill}{rgb}{0.121569,0.466667,0.705882}%
\pgfsetfillcolor{currentfill}%
\pgfsetfillopacity{0.300799}%
\pgfsetlinewidth{1.003750pt}%
\definecolor{currentstroke}{rgb}{0.121569,0.466667,0.705882}%
\pgfsetstrokecolor{currentstroke}%
\pgfsetstrokeopacity{0.300799}%
\pgfsetdash{}{0pt}%
\pgfpathmoveto{\pgfqpoint{1.138917in}{1.617077in}}%
\pgfpathcurveto{\pgfqpoint{1.147153in}{1.617077in}}{\pgfqpoint{1.155053in}{1.620350in}}{\pgfqpoint{1.160877in}{1.626174in}}%
\pgfpathcurveto{\pgfqpoint{1.166701in}{1.631998in}}{\pgfqpoint{1.169973in}{1.639898in}}{\pgfqpoint{1.169973in}{1.648134in}}%
\pgfpathcurveto{\pgfqpoint{1.169973in}{1.656370in}}{\pgfqpoint{1.166701in}{1.664270in}}{\pgfqpoint{1.160877in}{1.670094in}}%
\pgfpathcurveto{\pgfqpoint{1.155053in}{1.675918in}}{\pgfqpoint{1.147153in}{1.679190in}}{\pgfqpoint{1.138917in}{1.679190in}}%
\pgfpathcurveto{\pgfqpoint{1.130681in}{1.679190in}}{\pgfqpoint{1.122781in}{1.675918in}}{\pgfqpoint{1.116957in}{1.670094in}}%
\pgfpathcurveto{\pgfqpoint{1.111133in}{1.664270in}}{\pgfqpoint{1.107860in}{1.656370in}}{\pgfqpoint{1.107860in}{1.648134in}}%
\pgfpathcurveto{\pgfqpoint{1.107860in}{1.639898in}}{\pgfqpoint{1.111133in}{1.631998in}}{\pgfqpoint{1.116957in}{1.626174in}}%
\pgfpathcurveto{\pgfqpoint{1.122781in}{1.620350in}}{\pgfqpoint{1.130681in}{1.617077in}}{\pgfqpoint{1.138917in}{1.617077in}}%
\pgfpathclose%
\pgfusepath{stroke,fill}%
\end{pgfscope}%
\begin{pgfscope}%
\pgfpathrectangle{\pgfqpoint{0.100000in}{0.212622in}}{\pgfqpoint{3.696000in}{3.696000in}}%
\pgfusepath{clip}%
\pgfsetbuttcap%
\pgfsetroundjoin%
\definecolor{currentfill}{rgb}{0.121569,0.466667,0.705882}%
\pgfsetfillcolor{currentfill}%
\pgfsetfillopacity{0.300799}%
\pgfsetlinewidth{1.003750pt}%
\definecolor{currentstroke}{rgb}{0.121569,0.466667,0.705882}%
\pgfsetstrokecolor{currentstroke}%
\pgfsetstrokeopacity{0.300799}%
\pgfsetdash{}{0pt}%
\pgfpathmoveto{\pgfqpoint{1.138917in}{1.617077in}}%
\pgfpathcurveto{\pgfqpoint{1.147153in}{1.617077in}}{\pgfqpoint{1.155053in}{1.620350in}}{\pgfqpoint{1.160877in}{1.626174in}}%
\pgfpathcurveto{\pgfqpoint{1.166701in}{1.631998in}}{\pgfqpoint{1.169973in}{1.639898in}}{\pgfqpoint{1.169973in}{1.648134in}}%
\pgfpathcurveto{\pgfqpoint{1.169973in}{1.656370in}}{\pgfqpoint{1.166701in}{1.664270in}}{\pgfqpoint{1.160877in}{1.670094in}}%
\pgfpathcurveto{\pgfqpoint{1.155053in}{1.675918in}}{\pgfqpoint{1.147153in}{1.679190in}}{\pgfqpoint{1.138917in}{1.679190in}}%
\pgfpathcurveto{\pgfqpoint{1.130681in}{1.679190in}}{\pgfqpoint{1.122781in}{1.675918in}}{\pgfqpoint{1.116957in}{1.670094in}}%
\pgfpathcurveto{\pgfqpoint{1.111133in}{1.664270in}}{\pgfqpoint{1.107860in}{1.656370in}}{\pgfqpoint{1.107860in}{1.648134in}}%
\pgfpathcurveto{\pgfqpoint{1.107860in}{1.639898in}}{\pgfqpoint{1.111133in}{1.631998in}}{\pgfqpoint{1.116957in}{1.626174in}}%
\pgfpathcurveto{\pgfqpoint{1.122781in}{1.620350in}}{\pgfqpoint{1.130681in}{1.617077in}}{\pgfqpoint{1.138917in}{1.617077in}}%
\pgfpathclose%
\pgfusepath{stroke,fill}%
\end{pgfscope}%
\begin{pgfscope}%
\pgfpathrectangle{\pgfqpoint{0.100000in}{0.212622in}}{\pgfqpoint{3.696000in}{3.696000in}}%
\pgfusepath{clip}%
\pgfsetbuttcap%
\pgfsetroundjoin%
\definecolor{currentfill}{rgb}{0.121569,0.466667,0.705882}%
\pgfsetfillcolor{currentfill}%
\pgfsetfillopacity{0.300799}%
\pgfsetlinewidth{1.003750pt}%
\definecolor{currentstroke}{rgb}{0.121569,0.466667,0.705882}%
\pgfsetstrokecolor{currentstroke}%
\pgfsetstrokeopacity{0.300799}%
\pgfsetdash{}{0pt}%
\pgfpathmoveto{\pgfqpoint{1.138917in}{1.617077in}}%
\pgfpathcurveto{\pgfqpoint{1.147153in}{1.617077in}}{\pgfqpoint{1.155053in}{1.620350in}}{\pgfqpoint{1.160877in}{1.626174in}}%
\pgfpathcurveto{\pgfqpoint{1.166701in}{1.631998in}}{\pgfqpoint{1.169973in}{1.639898in}}{\pgfqpoint{1.169973in}{1.648134in}}%
\pgfpathcurveto{\pgfqpoint{1.169973in}{1.656370in}}{\pgfqpoint{1.166701in}{1.664270in}}{\pgfqpoint{1.160877in}{1.670094in}}%
\pgfpathcurveto{\pgfqpoint{1.155053in}{1.675918in}}{\pgfqpoint{1.147153in}{1.679190in}}{\pgfqpoint{1.138917in}{1.679190in}}%
\pgfpathcurveto{\pgfqpoint{1.130681in}{1.679190in}}{\pgfqpoint{1.122781in}{1.675918in}}{\pgfqpoint{1.116957in}{1.670094in}}%
\pgfpathcurveto{\pgfqpoint{1.111133in}{1.664270in}}{\pgfqpoint{1.107860in}{1.656370in}}{\pgfqpoint{1.107860in}{1.648134in}}%
\pgfpathcurveto{\pgfqpoint{1.107860in}{1.639898in}}{\pgfqpoint{1.111133in}{1.631998in}}{\pgfqpoint{1.116957in}{1.626174in}}%
\pgfpathcurveto{\pgfqpoint{1.122781in}{1.620350in}}{\pgfqpoint{1.130681in}{1.617077in}}{\pgfqpoint{1.138917in}{1.617077in}}%
\pgfpathclose%
\pgfusepath{stroke,fill}%
\end{pgfscope}%
\begin{pgfscope}%
\pgfpathrectangle{\pgfqpoint{0.100000in}{0.212622in}}{\pgfqpoint{3.696000in}{3.696000in}}%
\pgfusepath{clip}%
\pgfsetbuttcap%
\pgfsetroundjoin%
\definecolor{currentfill}{rgb}{0.121569,0.466667,0.705882}%
\pgfsetfillcolor{currentfill}%
\pgfsetfillopacity{0.300799}%
\pgfsetlinewidth{1.003750pt}%
\definecolor{currentstroke}{rgb}{0.121569,0.466667,0.705882}%
\pgfsetstrokecolor{currentstroke}%
\pgfsetstrokeopacity{0.300799}%
\pgfsetdash{}{0pt}%
\pgfpathmoveto{\pgfqpoint{1.138917in}{1.617077in}}%
\pgfpathcurveto{\pgfqpoint{1.147153in}{1.617077in}}{\pgfqpoint{1.155053in}{1.620350in}}{\pgfqpoint{1.160877in}{1.626174in}}%
\pgfpathcurveto{\pgfqpoint{1.166701in}{1.631998in}}{\pgfqpoint{1.169973in}{1.639898in}}{\pgfqpoint{1.169973in}{1.648134in}}%
\pgfpathcurveto{\pgfqpoint{1.169973in}{1.656370in}}{\pgfqpoint{1.166701in}{1.664270in}}{\pgfqpoint{1.160877in}{1.670094in}}%
\pgfpathcurveto{\pgfqpoint{1.155053in}{1.675918in}}{\pgfqpoint{1.147153in}{1.679190in}}{\pgfqpoint{1.138917in}{1.679190in}}%
\pgfpathcurveto{\pgfqpoint{1.130681in}{1.679190in}}{\pgfqpoint{1.122781in}{1.675918in}}{\pgfqpoint{1.116957in}{1.670094in}}%
\pgfpathcurveto{\pgfqpoint{1.111133in}{1.664270in}}{\pgfqpoint{1.107860in}{1.656370in}}{\pgfqpoint{1.107860in}{1.648134in}}%
\pgfpathcurveto{\pgfqpoint{1.107860in}{1.639898in}}{\pgfqpoint{1.111133in}{1.631998in}}{\pgfqpoint{1.116957in}{1.626174in}}%
\pgfpathcurveto{\pgfqpoint{1.122781in}{1.620350in}}{\pgfqpoint{1.130681in}{1.617077in}}{\pgfqpoint{1.138917in}{1.617077in}}%
\pgfpathclose%
\pgfusepath{stroke,fill}%
\end{pgfscope}%
\begin{pgfscope}%
\pgfpathrectangle{\pgfqpoint{0.100000in}{0.212622in}}{\pgfqpoint{3.696000in}{3.696000in}}%
\pgfusepath{clip}%
\pgfsetbuttcap%
\pgfsetroundjoin%
\definecolor{currentfill}{rgb}{0.121569,0.466667,0.705882}%
\pgfsetfillcolor{currentfill}%
\pgfsetfillopacity{0.300799}%
\pgfsetlinewidth{1.003750pt}%
\definecolor{currentstroke}{rgb}{0.121569,0.466667,0.705882}%
\pgfsetstrokecolor{currentstroke}%
\pgfsetstrokeopacity{0.300799}%
\pgfsetdash{}{0pt}%
\pgfpathmoveto{\pgfqpoint{1.138917in}{1.617077in}}%
\pgfpathcurveto{\pgfqpoint{1.147153in}{1.617077in}}{\pgfqpoint{1.155053in}{1.620350in}}{\pgfqpoint{1.160877in}{1.626174in}}%
\pgfpathcurveto{\pgfqpoint{1.166701in}{1.631998in}}{\pgfqpoint{1.169973in}{1.639898in}}{\pgfqpoint{1.169973in}{1.648134in}}%
\pgfpathcurveto{\pgfqpoint{1.169973in}{1.656370in}}{\pgfqpoint{1.166701in}{1.664270in}}{\pgfqpoint{1.160877in}{1.670094in}}%
\pgfpathcurveto{\pgfqpoint{1.155053in}{1.675918in}}{\pgfqpoint{1.147153in}{1.679190in}}{\pgfqpoint{1.138917in}{1.679190in}}%
\pgfpathcurveto{\pgfqpoint{1.130681in}{1.679190in}}{\pgfqpoint{1.122781in}{1.675918in}}{\pgfqpoint{1.116957in}{1.670094in}}%
\pgfpathcurveto{\pgfqpoint{1.111133in}{1.664270in}}{\pgfqpoint{1.107860in}{1.656370in}}{\pgfqpoint{1.107860in}{1.648134in}}%
\pgfpathcurveto{\pgfqpoint{1.107860in}{1.639898in}}{\pgfqpoint{1.111133in}{1.631998in}}{\pgfqpoint{1.116957in}{1.626174in}}%
\pgfpathcurveto{\pgfqpoint{1.122781in}{1.620350in}}{\pgfqpoint{1.130681in}{1.617077in}}{\pgfqpoint{1.138917in}{1.617077in}}%
\pgfpathclose%
\pgfusepath{stroke,fill}%
\end{pgfscope}%
\begin{pgfscope}%
\pgfpathrectangle{\pgfqpoint{0.100000in}{0.212622in}}{\pgfqpoint{3.696000in}{3.696000in}}%
\pgfusepath{clip}%
\pgfsetbuttcap%
\pgfsetroundjoin%
\definecolor{currentfill}{rgb}{0.121569,0.466667,0.705882}%
\pgfsetfillcolor{currentfill}%
\pgfsetfillopacity{0.300799}%
\pgfsetlinewidth{1.003750pt}%
\definecolor{currentstroke}{rgb}{0.121569,0.466667,0.705882}%
\pgfsetstrokecolor{currentstroke}%
\pgfsetstrokeopacity{0.300799}%
\pgfsetdash{}{0pt}%
\pgfpathmoveto{\pgfqpoint{1.138917in}{1.617077in}}%
\pgfpathcurveto{\pgfqpoint{1.147153in}{1.617077in}}{\pgfqpoint{1.155053in}{1.620350in}}{\pgfqpoint{1.160877in}{1.626174in}}%
\pgfpathcurveto{\pgfqpoint{1.166701in}{1.631998in}}{\pgfqpoint{1.169973in}{1.639898in}}{\pgfqpoint{1.169973in}{1.648134in}}%
\pgfpathcurveto{\pgfqpoint{1.169973in}{1.656370in}}{\pgfqpoint{1.166701in}{1.664270in}}{\pgfqpoint{1.160877in}{1.670094in}}%
\pgfpathcurveto{\pgfqpoint{1.155053in}{1.675918in}}{\pgfqpoint{1.147153in}{1.679190in}}{\pgfqpoint{1.138917in}{1.679190in}}%
\pgfpathcurveto{\pgfqpoint{1.130681in}{1.679190in}}{\pgfqpoint{1.122781in}{1.675918in}}{\pgfqpoint{1.116957in}{1.670094in}}%
\pgfpathcurveto{\pgfqpoint{1.111133in}{1.664270in}}{\pgfqpoint{1.107860in}{1.656370in}}{\pgfqpoint{1.107860in}{1.648134in}}%
\pgfpathcurveto{\pgfqpoint{1.107860in}{1.639898in}}{\pgfqpoint{1.111133in}{1.631998in}}{\pgfqpoint{1.116957in}{1.626174in}}%
\pgfpathcurveto{\pgfqpoint{1.122781in}{1.620350in}}{\pgfqpoint{1.130681in}{1.617077in}}{\pgfqpoint{1.138917in}{1.617077in}}%
\pgfpathclose%
\pgfusepath{stroke,fill}%
\end{pgfscope}%
\begin{pgfscope}%
\pgfpathrectangle{\pgfqpoint{0.100000in}{0.212622in}}{\pgfqpoint{3.696000in}{3.696000in}}%
\pgfusepath{clip}%
\pgfsetbuttcap%
\pgfsetroundjoin%
\definecolor{currentfill}{rgb}{0.121569,0.466667,0.705882}%
\pgfsetfillcolor{currentfill}%
\pgfsetfillopacity{0.300799}%
\pgfsetlinewidth{1.003750pt}%
\definecolor{currentstroke}{rgb}{0.121569,0.466667,0.705882}%
\pgfsetstrokecolor{currentstroke}%
\pgfsetstrokeopacity{0.300799}%
\pgfsetdash{}{0pt}%
\pgfpathmoveto{\pgfqpoint{1.138917in}{1.617077in}}%
\pgfpathcurveto{\pgfqpoint{1.147153in}{1.617077in}}{\pgfqpoint{1.155053in}{1.620350in}}{\pgfqpoint{1.160877in}{1.626174in}}%
\pgfpathcurveto{\pgfqpoint{1.166701in}{1.631998in}}{\pgfqpoint{1.169973in}{1.639898in}}{\pgfqpoint{1.169973in}{1.648134in}}%
\pgfpathcurveto{\pgfqpoint{1.169973in}{1.656370in}}{\pgfqpoint{1.166701in}{1.664270in}}{\pgfqpoint{1.160877in}{1.670094in}}%
\pgfpathcurveto{\pgfqpoint{1.155053in}{1.675918in}}{\pgfqpoint{1.147153in}{1.679190in}}{\pgfqpoint{1.138917in}{1.679190in}}%
\pgfpathcurveto{\pgfqpoint{1.130681in}{1.679190in}}{\pgfqpoint{1.122781in}{1.675918in}}{\pgfqpoint{1.116957in}{1.670094in}}%
\pgfpathcurveto{\pgfqpoint{1.111133in}{1.664270in}}{\pgfqpoint{1.107860in}{1.656370in}}{\pgfqpoint{1.107860in}{1.648134in}}%
\pgfpathcurveto{\pgfqpoint{1.107860in}{1.639898in}}{\pgfqpoint{1.111133in}{1.631998in}}{\pgfqpoint{1.116957in}{1.626174in}}%
\pgfpathcurveto{\pgfqpoint{1.122781in}{1.620350in}}{\pgfqpoint{1.130681in}{1.617077in}}{\pgfqpoint{1.138917in}{1.617077in}}%
\pgfpathclose%
\pgfusepath{stroke,fill}%
\end{pgfscope}%
\begin{pgfscope}%
\pgfpathrectangle{\pgfqpoint{0.100000in}{0.212622in}}{\pgfqpoint{3.696000in}{3.696000in}}%
\pgfusepath{clip}%
\pgfsetbuttcap%
\pgfsetroundjoin%
\definecolor{currentfill}{rgb}{0.121569,0.466667,0.705882}%
\pgfsetfillcolor{currentfill}%
\pgfsetfillopacity{0.300799}%
\pgfsetlinewidth{1.003750pt}%
\definecolor{currentstroke}{rgb}{0.121569,0.466667,0.705882}%
\pgfsetstrokecolor{currentstroke}%
\pgfsetstrokeopacity{0.300799}%
\pgfsetdash{}{0pt}%
\pgfpathmoveto{\pgfqpoint{1.138917in}{1.617077in}}%
\pgfpathcurveto{\pgfqpoint{1.147153in}{1.617077in}}{\pgfqpoint{1.155053in}{1.620350in}}{\pgfqpoint{1.160877in}{1.626174in}}%
\pgfpathcurveto{\pgfqpoint{1.166701in}{1.631998in}}{\pgfqpoint{1.169973in}{1.639898in}}{\pgfqpoint{1.169973in}{1.648134in}}%
\pgfpathcurveto{\pgfqpoint{1.169973in}{1.656370in}}{\pgfqpoint{1.166701in}{1.664270in}}{\pgfqpoint{1.160877in}{1.670094in}}%
\pgfpathcurveto{\pgfqpoint{1.155053in}{1.675918in}}{\pgfqpoint{1.147153in}{1.679190in}}{\pgfqpoint{1.138917in}{1.679190in}}%
\pgfpathcurveto{\pgfqpoint{1.130681in}{1.679190in}}{\pgfqpoint{1.122781in}{1.675918in}}{\pgfqpoint{1.116957in}{1.670094in}}%
\pgfpathcurveto{\pgfqpoint{1.111133in}{1.664270in}}{\pgfqpoint{1.107860in}{1.656370in}}{\pgfqpoint{1.107860in}{1.648134in}}%
\pgfpathcurveto{\pgfqpoint{1.107860in}{1.639898in}}{\pgfqpoint{1.111133in}{1.631998in}}{\pgfqpoint{1.116957in}{1.626174in}}%
\pgfpathcurveto{\pgfqpoint{1.122781in}{1.620350in}}{\pgfqpoint{1.130681in}{1.617077in}}{\pgfqpoint{1.138917in}{1.617077in}}%
\pgfpathclose%
\pgfusepath{stroke,fill}%
\end{pgfscope}%
\begin{pgfscope}%
\pgfpathrectangle{\pgfqpoint{0.100000in}{0.212622in}}{\pgfqpoint{3.696000in}{3.696000in}}%
\pgfusepath{clip}%
\pgfsetbuttcap%
\pgfsetroundjoin%
\definecolor{currentfill}{rgb}{0.121569,0.466667,0.705882}%
\pgfsetfillcolor{currentfill}%
\pgfsetfillopacity{0.300799}%
\pgfsetlinewidth{1.003750pt}%
\definecolor{currentstroke}{rgb}{0.121569,0.466667,0.705882}%
\pgfsetstrokecolor{currentstroke}%
\pgfsetstrokeopacity{0.300799}%
\pgfsetdash{}{0pt}%
\pgfpathmoveto{\pgfqpoint{1.138917in}{1.617077in}}%
\pgfpathcurveto{\pgfqpoint{1.147153in}{1.617077in}}{\pgfqpoint{1.155053in}{1.620350in}}{\pgfqpoint{1.160877in}{1.626174in}}%
\pgfpathcurveto{\pgfqpoint{1.166701in}{1.631998in}}{\pgfqpoint{1.169973in}{1.639898in}}{\pgfqpoint{1.169973in}{1.648134in}}%
\pgfpathcurveto{\pgfqpoint{1.169973in}{1.656370in}}{\pgfqpoint{1.166701in}{1.664270in}}{\pgfqpoint{1.160877in}{1.670094in}}%
\pgfpathcurveto{\pgfqpoint{1.155053in}{1.675918in}}{\pgfqpoint{1.147153in}{1.679190in}}{\pgfqpoint{1.138917in}{1.679190in}}%
\pgfpathcurveto{\pgfqpoint{1.130681in}{1.679190in}}{\pgfqpoint{1.122781in}{1.675918in}}{\pgfqpoint{1.116957in}{1.670094in}}%
\pgfpathcurveto{\pgfqpoint{1.111133in}{1.664270in}}{\pgfqpoint{1.107860in}{1.656370in}}{\pgfqpoint{1.107860in}{1.648134in}}%
\pgfpathcurveto{\pgfqpoint{1.107860in}{1.639898in}}{\pgfqpoint{1.111133in}{1.631998in}}{\pgfqpoint{1.116957in}{1.626174in}}%
\pgfpathcurveto{\pgfqpoint{1.122781in}{1.620350in}}{\pgfqpoint{1.130681in}{1.617077in}}{\pgfqpoint{1.138917in}{1.617077in}}%
\pgfpathclose%
\pgfusepath{stroke,fill}%
\end{pgfscope}%
\begin{pgfscope}%
\pgfpathrectangle{\pgfqpoint{0.100000in}{0.212622in}}{\pgfqpoint{3.696000in}{3.696000in}}%
\pgfusepath{clip}%
\pgfsetbuttcap%
\pgfsetroundjoin%
\definecolor{currentfill}{rgb}{0.121569,0.466667,0.705882}%
\pgfsetfillcolor{currentfill}%
\pgfsetfillopacity{0.300799}%
\pgfsetlinewidth{1.003750pt}%
\definecolor{currentstroke}{rgb}{0.121569,0.466667,0.705882}%
\pgfsetstrokecolor{currentstroke}%
\pgfsetstrokeopacity{0.300799}%
\pgfsetdash{}{0pt}%
\pgfpathmoveto{\pgfqpoint{1.138917in}{1.617077in}}%
\pgfpathcurveto{\pgfqpoint{1.147153in}{1.617077in}}{\pgfqpoint{1.155053in}{1.620350in}}{\pgfqpoint{1.160877in}{1.626174in}}%
\pgfpathcurveto{\pgfqpoint{1.166701in}{1.631998in}}{\pgfqpoint{1.169973in}{1.639898in}}{\pgfqpoint{1.169973in}{1.648134in}}%
\pgfpathcurveto{\pgfqpoint{1.169973in}{1.656370in}}{\pgfqpoint{1.166701in}{1.664270in}}{\pgfqpoint{1.160877in}{1.670094in}}%
\pgfpathcurveto{\pgfqpoint{1.155053in}{1.675918in}}{\pgfqpoint{1.147153in}{1.679190in}}{\pgfqpoint{1.138917in}{1.679190in}}%
\pgfpathcurveto{\pgfqpoint{1.130681in}{1.679190in}}{\pgfqpoint{1.122781in}{1.675918in}}{\pgfqpoint{1.116957in}{1.670094in}}%
\pgfpathcurveto{\pgfqpoint{1.111133in}{1.664270in}}{\pgfqpoint{1.107860in}{1.656370in}}{\pgfqpoint{1.107860in}{1.648134in}}%
\pgfpathcurveto{\pgfqpoint{1.107860in}{1.639898in}}{\pgfqpoint{1.111133in}{1.631998in}}{\pgfqpoint{1.116957in}{1.626174in}}%
\pgfpathcurveto{\pgfqpoint{1.122781in}{1.620350in}}{\pgfqpoint{1.130681in}{1.617077in}}{\pgfqpoint{1.138917in}{1.617077in}}%
\pgfpathclose%
\pgfusepath{stroke,fill}%
\end{pgfscope}%
\begin{pgfscope}%
\pgfpathrectangle{\pgfqpoint{0.100000in}{0.212622in}}{\pgfqpoint{3.696000in}{3.696000in}}%
\pgfusepath{clip}%
\pgfsetbuttcap%
\pgfsetroundjoin%
\definecolor{currentfill}{rgb}{0.121569,0.466667,0.705882}%
\pgfsetfillcolor{currentfill}%
\pgfsetfillopacity{0.300799}%
\pgfsetlinewidth{1.003750pt}%
\definecolor{currentstroke}{rgb}{0.121569,0.466667,0.705882}%
\pgfsetstrokecolor{currentstroke}%
\pgfsetstrokeopacity{0.300799}%
\pgfsetdash{}{0pt}%
\pgfpathmoveto{\pgfqpoint{1.138917in}{1.617077in}}%
\pgfpathcurveto{\pgfqpoint{1.147153in}{1.617077in}}{\pgfqpoint{1.155053in}{1.620350in}}{\pgfqpoint{1.160877in}{1.626174in}}%
\pgfpathcurveto{\pgfqpoint{1.166701in}{1.631998in}}{\pgfqpoint{1.169973in}{1.639898in}}{\pgfqpoint{1.169973in}{1.648134in}}%
\pgfpathcurveto{\pgfqpoint{1.169973in}{1.656370in}}{\pgfqpoint{1.166701in}{1.664270in}}{\pgfqpoint{1.160877in}{1.670094in}}%
\pgfpathcurveto{\pgfqpoint{1.155053in}{1.675918in}}{\pgfqpoint{1.147153in}{1.679190in}}{\pgfqpoint{1.138917in}{1.679190in}}%
\pgfpathcurveto{\pgfqpoint{1.130681in}{1.679190in}}{\pgfqpoint{1.122781in}{1.675918in}}{\pgfqpoint{1.116957in}{1.670094in}}%
\pgfpathcurveto{\pgfqpoint{1.111133in}{1.664270in}}{\pgfqpoint{1.107860in}{1.656370in}}{\pgfqpoint{1.107860in}{1.648134in}}%
\pgfpathcurveto{\pgfqpoint{1.107860in}{1.639898in}}{\pgfqpoint{1.111133in}{1.631998in}}{\pgfqpoint{1.116957in}{1.626174in}}%
\pgfpathcurveto{\pgfqpoint{1.122781in}{1.620350in}}{\pgfqpoint{1.130681in}{1.617077in}}{\pgfqpoint{1.138917in}{1.617077in}}%
\pgfpathclose%
\pgfusepath{stroke,fill}%
\end{pgfscope}%
\begin{pgfscope}%
\pgfpathrectangle{\pgfqpoint{0.100000in}{0.212622in}}{\pgfqpoint{3.696000in}{3.696000in}}%
\pgfusepath{clip}%
\pgfsetbuttcap%
\pgfsetroundjoin%
\definecolor{currentfill}{rgb}{0.121569,0.466667,0.705882}%
\pgfsetfillcolor{currentfill}%
\pgfsetfillopacity{0.300799}%
\pgfsetlinewidth{1.003750pt}%
\definecolor{currentstroke}{rgb}{0.121569,0.466667,0.705882}%
\pgfsetstrokecolor{currentstroke}%
\pgfsetstrokeopacity{0.300799}%
\pgfsetdash{}{0pt}%
\pgfpathmoveto{\pgfqpoint{1.138917in}{1.617077in}}%
\pgfpathcurveto{\pgfqpoint{1.147153in}{1.617077in}}{\pgfqpoint{1.155053in}{1.620350in}}{\pgfqpoint{1.160877in}{1.626174in}}%
\pgfpathcurveto{\pgfqpoint{1.166701in}{1.631998in}}{\pgfqpoint{1.169973in}{1.639898in}}{\pgfqpoint{1.169973in}{1.648134in}}%
\pgfpathcurveto{\pgfqpoint{1.169973in}{1.656370in}}{\pgfqpoint{1.166701in}{1.664270in}}{\pgfqpoint{1.160877in}{1.670094in}}%
\pgfpathcurveto{\pgfqpoint{1.155053in}{1.675918in}}{\pgfqpoint{1.147153in}{1.679190in}}{\pgfqpoint{1.138917in}{1.679190in}}%
\pgfpathcurveto{\pgfqpoint{1.130681in}{1.679190in}}{\pgfqpoint{1.122781in}{1.675918in}}{\pgfqpoint{1.116957in}{1.670094in}}%
\pgfpathcurveto{\pgfqpoint{1.111133in}{1.664270in}}{\pgfqpoint{1.107860in}{1.656370in}}{\pgfqpoint{1.107860in}{1.648134in}}%
\pgfpathcurveto{\pgfqpoint{1.107860in}{1.639898in}}{\pgfqpoint{1.111133in}{1.631998in}}{\pgfqpoint{1.116957in}{1.626174in}}%
\pgfpathcurveto{\pgfqpoint{1.122781in}{1.620350in}}{\pgfqpoint{1.130681in}{1.617077in}}{\pgfqpoint{1.138917in}{1.617077in}}%
\pgfpathclose%
\pgfusepath{stroke,fill}%
\end{pgfscope}%
\begin{pgfscope}%
\pgfpathrectangle{\pgfqpoint{0.100000in}{0.212622in}}{\pgfqpoint{3.696000in}{3.696000in}}%
\pgfusepath{clip}%
\pgfsetbuttcap%
\pgfsetroundjoin%
\definecolor{currentfill}{rgb}{0.121569,0.466667,0.705882}%
\pgfsetfillcolor{currentfill}%
\pgfsetfillopacity{0.300799}%
\pgfsetlinewidth{1.003750pt}%
\definecolor{currentstroke}{rgb}{0.121569,0.466667,0.705882}%
\pgfsetstrokecolor{currentstroke}%
\pgfsetstrokeopacity{0.300799}%
\pgfsetdash{}{0pt}%
\pgfpathmoveto{\pgfqpoint{1.138917in}{1.617077in}}%
\pgfpathcurveto{\pgfqpoint{1.147153in}{1.617077in}}{\pgfqpoint{1.155053in}{1.620350in}}{\pgfqpoint{1.160877in}{1.626174in}}%
\pgfpathcurveto{\pgfqpoint{1.166701in}{1.631998in}}{\pgfqpoint{1.169973in}{1.639898in}}{\pgfqpoint{1.169973in}{1.648134in}}%
\pgfpathcurveto{\pgfqpoint{1.169973in}{1.656370in}}{\pgfqpoint{1.166701in}{1.664270in}}{\pgfqpoint{1.160877in}{1.670094in}}%
\pgfpathcurveto{\pgfqpoint{1.155053in}{1.675918in}}{\pgfqpoint{1.147153in}{1.679190in}}{\pgfqpoint{1.138917in}{1.679190in}}%
\pgfpathcurveto{\pgfqpoint{1.130681in}{1.679190in}}{\pgfqpoint{1.122781in}{1.675918in}}{\pgfqpoint{1.116957in}{1.670094in}}%
\pgfpathcurveto{\pgfqpoint{1.111133in}{1.664270in}}{\pgfqpoint{1.107860in}{1.656370in}}{\pgfqpoint{1.107860in}{1.648134in}}%
\pgfpathcurveto{\pgfqpoint{1.107860in}{1.639898in}}{\pgfqpoint{1.111133in}{1.631998in}}{\pgfqpoint{1.116957in}{1.626174in}}%
\pgfpathcurveto{\pgfqpoint{1.122781in}{1.620350in}}{\pgfqpoint{1.130681in}{1.617077in}}{\pgfqpoint{1.138917in}{1.617077in}}%
\pgfpathclose%
\pgfusepath{stroke,fill}%
\end{pgfscope}%
\begin{pgfscope}%
\pgfpathrectangle{\pgfqpoint{0.100000in}{0.212622in}}{\pgfqpoint{3.696000in}{3.696000in}}%
\pgfusepath{clip}%
\pgfsetbuttcap%
\pgfsetroundjoin%
\definecolor{currentfill}{rgb}{0.121569,0.466667,0.705882}%
\pgfsetfillcolor{currentfill}%
\pgfsetfillopacity{0.300799}%
\pgfsetlinewidth{1.003750pt}%
\definecolor{currentstroke}{rgb}{0.121569,0.466667,0.705882}%
\pgfsetstrokecolor{currentstroke}%
\pgfsetstrokeopacity{0.300799}%
\pgfsetdash{}{0pt}%
\pgfpathmoveto{\pgfqpoint{1.138917in}{1.617077in}}%
\pgfpathcurveto{\pgfqpoint{1.147153in}{1.617077in}}{\pgfqpoint{1.155053in}{1.620350in}}{\pgfqpoint{1.160877in}{1.626174in}}%
\pgfpathcurveto{\pgfqpoint{1.166701in}{1.631998in}}{\pgfqpoint{1.169973in}{1.639898in}}{\pgfqpoint{1.169973in}{1.648134in}}%
\pgfpathcurveto{\pgfqpoint{1.169973in}{1.656370in}}{\pgfqpoint{1.166701in}{1.664270in}}{\pgfqpoint{1.160877in}{1.670094in}}%
\pgfpathcurveto{\pgfqpoint{1.155053in}{1.675918in}}{\pgfqpoint{1.147153in}{1.679190in}}{\pgfqpoint{1.138917in}{1.679190in}}%
\pgfpathcurveto{\pgfqpoint{1.130681in}{1.679190in}}{\pgfqpoint{1.122781in}{1.675918in}}{\pgfqpoint{1.116957in}{1.670094in}}%
\pgfpathcurveto{\pgfqpoint{1.111133in}{1.664270in}}{\pgfqpoint{1.107860in}{1.656370in}}{\pgfqpoint{1.107860in}{1.648134in}}%
\pgfpathcurveto{\pgfqpoint{1.107860in}{1.639898in}}{\pgfqpoint{1.111133in}{1.631998in}}{\pgfqpoint{1.116957in}{1.626174in}}%
\pgfpathcurveto{\pgfqpoint{1.122781in}{1.620350in}}{\pgfqpoint{1.130681in}{1.617077in}}{\pgfqpoint{1.138917in}{1.617077in}}%
\pgfpathclose%
\pgfusepath{stroke,fill}%
\end{pgfscope}%
\begin{pgfscope}%
\pgfpathrectangle{\pgfqpoint{0.100000in}{0.212622in}}{\pgfqpoint{3.696000in}{3.696000in}}%
\pgfusepath{clip}%
\pgfsetbuttcap%
\pgfsetroundjoin%
\definecolor{currentfill}{rgb}{0.121569,0.466667,0.705882}%
\pgfsetfillcolor{currentfill}%
\pgfsetfillopacity{0.300799}%
\pgfsetlinewidth{1.003750pt}%
\definecolor{currentstroke}{rgb}{0.121569,0.466667,0.705882}%
\pgfsetstrokecolor{currentstroke}%
\pgfsetstrokeopacity{0.300799}%
\pgfsetdash{}{0pt}%
\pgfpathmoveto{\pgfqpoint{1.138917in}{1.617077in}}%
\pgfpathcurveto{\pgfqpoint{1.147153in}{1.617077in}}{\pgfqpoint{1.155053in}{1.620350in}}{\pgfqpoint{1.160877in}{1.626174in}}%
\pgfpathcurveto{\pgfqpoint{1.166701in}{1.631998in}}{\pgfqpoint{1.169973in}{1.639898in}}{\pgfqpoint{1.169973in}{1.648134in}}%
\pgfpathcurveto{\pgfqpoint{1.169973in}{1.656370in}}{\pgfqpoint{1.166701in}{1.664270in}}{\pgfqpoint{1.160877in}{1.670094in}}%
\pgfpathcurveto{\pgfqpoint{1.155053in}{1.675918in}}{\pgfqpoint{1.147153in}{1.679190in}}{\pgfqpoint{1.138917in}{1.679190in}}%
\pgfpathcurveto{\pgfqpoint{1.130681in}{1.679190in}}{\pgfqpoint{1.122781in}{1.675918in}}{\pgfqpoint{1.116957in}{1.670094in}}%
\pgfpathcurveto{\pgfqpoint{1.111133in}{1.664270in}}{\pgfqpoint{1.107860in}{1.656370in}}{\pgfqpoint{1.107860in}{1.648134in}}%
\pgfpathcurveto{\pgfqpoint{1.107860in}{1.639898in}}{\pgfqpoint{1.111133in}{1.631998in}}{\pgfqpoint{1.116957in}{1.626174in}}%
\pgfpathcurveto{\pgfqpoint{1.122781in}{1.620350in}}{\pgfqpoint{1.130681in}{1.617077in}}{\pgfqpoint{1.138917in}{1.617077in}}%
\pgfpathclose%
\pgfusepath{stroke,fill}%
\end{pgfscope}%
\begin{pgfscope}%
\pgfpathrectangle{\pgfqpoint{0.100000in}{0.212622in}}{\pgfqpoint{3.696000in}{3.696000in}}%
\pgfusepath{clip}%
\pgfsetbuttcap%
\pgfsetroundjoin%
\definecolor{currentfill}{rgb}{0.121569,0.466667,0.705882}%
\pgfsetfillcolor{currentfill}%
\pgfsetfillopacity{0.300799}%
\pgfsetlinewidth{1.003750pt}%
\definecolor{currentstroke}{rgb}{0.121569,0.466667,0.705882}%
\pgfsetstrokecolor{currentstroke}%
\pgfsetstrokeopacity{0.300799}%
\pgfsetdash{}{0pt}%
\pgfpathmoveto{\pgfqpoint{1.138917in}{1.617077in}}%
\pgfpathcurveto{\pgfqpoint{1.147153in}{1.617077in}}{\pgfqpoint{1.155053in}{1.620350in}}{\pgfqpoint{1.160877in}{1.626174in}}%
\pgfpathcurveto{\pgfqpoint{1.166701in}{1.631998in}}{\pgfqpoint{1.169973in}{1.639898in}}{\pgfqpoint{1.169973in}{1.648134in}}%
\pgfpathcurveto{\pgfqpoint{1.169973in}{1.656370in}}{\pgfqpoint{1.166701in}{1.664270in}}{\pgfqpoint{1.160877in}{1.670094in}}%
\pgfpathcurveto{\pgfqpoint{1.155053in}{1.675918in}}{\pgfqpoint{1.147153in}{1.679190in}}{\pgfqpoint{1.138917in}{1.679190in}}%
\pgfpathcurveto{\pgfqpoint{1.130681in}{1.679190in}}{\pgfqpoint{1.122781in}{1.675918in}}{\pgfqpoint{1.116957in}{1.670094in}}%
\pgfpathcurveto{\pgfqpoint{1.111133in}{1.664270in}}{\pgfqpoint{1.107860in}{1.656370in}}{\pgfqpoint{1.107860in}{1.648134in}}%
\pgfpathcurveto{\pgfqpoint{1.107860in}{1.639898in}}{\pgfqpoint{1.111133in}{1.631998in}}{\pgfqpoint{1.116957in}{1.626174in}}%
\pgfpathcurveto{\pgfqpoint{1.122781in}{1.620350in}}{\pgfqpoint{1.130681in}{1.617077in}}{\pgfqpoint{1.138917in}{1.617077in}}%
\pgfpathclose%
\pgfusepath{stroke,fill}%
\end{pgfscope}%
\begin{pgfscope}%
\pgfpathrectangle{\pgfqpoint{0.100000in}{0.212622in}}{\pgfqpoint{3.696000in}{3.696000in}}%
\pgfusepath{clip}%
\pgfsetbuttcap%
\pgfsetroundjoin%
\definecolor{currentfill}{rgb}{0.121569,0.466667,0.705882}%
\pgfsetfillcolor{currentfill}%
\pgfsetfillopacity{0.300799}%
\pgfsetlinewidth{1.003750pt}%
\definecolor{currentstroke}{rgb}{0.121569,0.466667,0.705882}%
\pgfsetstrokecolor{currentstroke}%
\pgfsetstrokeopacity{0.300799}%
\pgfsetdash{}{0pt}%
\pgfpathmoveto{\pgfqpoint{1.138917in}{1.617077in}}%
\pgfpathcurveto{\pgfqpoint{1.147153in}{1.617077in}}{\pgfqpoint{1.155053in}{1.620350in}}{\pgfqpoint{1.160877in}{1.626174in}}%
\pgfpathcurveto{\pgfqpoint{1.166701in}{1.631998in}}{\pgfqpoint{1.169973in}{1.639898in}}{\pgfqpoint{1.169973in}{1.648134in}}%
\pgfpathcurveto{\pgfqpoint{1.169973in}{1.656370in}}{\pgfqpoint{1.166701in}{1.664270in}}{\pgfqpoint{1.160877in}{1.670094in}}%
\pgfpathcurveto{\pgfqpoint{1.155053in}{1.675918in}}{\pgfqpoint{1.147153in}{1.679190in}}{\pgfqpoint{1.138917in}{1.679190in}}%
\pgfpathcurveto{\pgfqpoint{1.130681in}{1.679190in}}{\pgfqpoint{1.122781in}{1.675918in}}{\pgfqpoint{1.116957in}{1.670094in}}%
\pgfpathcurveto{\pgfqpoint{1.111133in}{1.664270in}}{\pgfqpoint{1.107860in}{1.656370in}}{\pgfqpoint{1.107860in}{1.648134in}}%
\pgfpathcurveto{\pgfqpoint{1.107860in}{1.639898in}}{\pgfqpoint{1.111133in}{1.631998in}}{\pgfqpoint{1.116957in}{1.626174in}}%
\pgfpathcurveto{\pgfqpoint{1.122781in}{1.620350in}}{\pgfqpoint{1.130681in}{1.617077in}}{\pgfqpoint{1.138917in}{1.617077in}}%
\pgfpathclose%
\pgfusepath{stroke,fill}%
\end{pgfscope}%
\begin{pgfscope}%
\pgfpathrectangle{\pgfqpoint{0.100000in}{0.212622in}}{\pgfqpoint{3.696000in}{3.696000in}}%
\pgfusepath{clip}%
\pgfsetbuttcap%
\pgfsetroundjoin%
\definecolor{currentfill}{rgb}{0.121569,0.466667,0.705882}%
\pgfsetfillcolor{currentfill}%
\pgfsetfillopacity{0.300799}%
\pgfsetlinewidth{1.003750pt}%
\definecolor{currentstroke}{rgb}{0.121569,0.466667,0.705882}%
\pgfsetstrokecolor{currentstroke}%
\pgfsetstrokeopacity{0.300799}%
\pgfsetdash{}{0pt}%
\pgfpathmoveto{\pgfqpoint{1.138917in}{1.617077in}}%
\pgfpathcurveto{\pgfqpoint{1.147153in}{1.617077in}}{\pgfqpoint{1.155053in}{1.620350in}}{\pgfqpoint{1.160877in}{1.626174in}}%
\pgfpathcurveto{\pgfqpoint{1.166701in}{1.631998in}}{\pgfqpoint{1.169973in}{1.639898in}}{\pgfqpoint{1.169973in}{1.648134in}}%
\pgfpathcurveto{\pgfqpoint{1.169973in}{1.656370in}}{\pgfqpoint{1.166701in}{1.664270in}}{\pgfqpoint{1.160877in}{1.670094in}}%
\pgfpathcurveto{\pgfqpoint{1.155053in}{1.675918in}}{\pgfqpoint{1.147153in}{1.679190in}}{\pgfqpoint{1.138917in}{1.679190in}}%
\pgfpathcurveto{\pgfqpoint{1.130681in}{1.679190in}}{\pgfqpoint{1.122781in}{1.675918in}}{\pgfqpoint{1.116957in}{1.670094in}}%
\pgfpathcurveto{\pgfqpoint{1.111133in}{1.664270in}}{\pgfqpoint{1.107860in}{1.656370in}}{\pgfqpoint{1.107860in}{1.648134in}}%
\pgfpathcurveto{\pgfqpoint{1.107860in}{1.639898in}}{\pgfqpoint{1.111133in}{1.631998in}}{\pgfqpoint{1.116957in}{1.626174in}}%
\pgfpathcurveto{\pgfqpoint{1.122781in}{1.620350in}}{\pgfqpoint{1.130681in}{1.617077in}}{\pgfqpoint{1.138917in}{1.617077in}}%
\pgfpathclose%
\pgfusepath{stroke,fill}%
\end{pgfscope}%
\begin{pgfscope}%
\pgfpathrectangle{\pgfqpoint{0.100000in}{0.212622in}}{\pgfqpoint{3.696000in}{3.696000in}}%
\pgfusepath{clip}%
\pgfsetbuttcap%
\pgfsetroundjoin%
\definecolor{currentfill}{rgb}{0.121569,0.466667,0.705882}%
\pgfsetfillcolor{currentfill}%
\pgfsetfillopacity{0.301061}%
\pgfsetlinewidth{1.003750pt}%
\definecolor{currentstroke}{rgb}{0.121569,0.466667,0.705882}%
\pgfsetstrokecolor{currentstroke}%
\pgfsetstrokeopacity{0.301061}%
\pgfsetdash{}{0pt}%
\pgfpathmoveto{\pgfqpoint{1.139756in}{1.616846in}}%
\pgfpathcurveto{\pgfqpoint{1.147993in}{1.616846in}}{\pgfqpoint{1.155893in}{1.620118in}}{\pgfqpoint{1.161717in}{1.625942in}}%
\pgfpathcurveto{\pgfqpoint{1.167540in}{1.631766in}}{\pgfqpoint{1.170813in}{1.639666in}}{\pgfqpoint{1.170813in}{1.647903in}}%
\pgfpathcurveto{\pgfqpoint{1.170813in}{1.656139in}}{\pgfqpoint{1.167540in}{1.664039in}}{\pgfqpoint{1.161717in}{1.669863in}}%
\pgfpathcurveto{\pgfqpoint{1.155893in}{1.675687in}}{\pgfqpoint{1.147993in}{1.678959in}}{\pgfqpoint{1.139756in}{1.678959in}}%
\pgfpathcurveto{\pgfqpoint{1.131520in}{1.678959in}}{\pgfqpoint{1.123620in}{1.675687in}}{\pgfqpoint{1.117796in}{1.669863in}}%
\pgfpathcurveto{\pgfqpoint{1.111972in}{1.664039in}}{\pgfqpoint{1.108700in}{1.656139in}}{\pgfqpoint{1.108700in}{1.647903in}}%
\pgfpathcurveto{\pgfqpoint{1.108700in}{1.639666in}}{\pgfqpoint{1.111972in}{1.631766in}}{\pgfqpoint{1.117796in}{1.625942in}}%
\pgfpathcurveto{\pgfqpoint{1.123620in}{1.620118in}}{\pgfqpoint{1.131520in}{1.616846in}}{\pgfqpoint{1.139756in}{1.616846in}}%
\pgfpathclose%
\pgfusepath{stroke,fill}%
\end{pgfscope}%
\begin{pgfscope}%
\pgfpathrectangle{\pgfqpoint{0.100000in}{0.212622in}}{\pgfqpoint{3.696000in}{3.696000in}}%
\pgfusepath{clip}%
\pgfsetbuttcap%
\pgfsetroundjoin%
\definecolor{currentfill}{rgb}{0.121569,0.466667,0.705882}%
\pgfsetfillcolor{currentfill}%
\pgfsetfillopacity{0.301229}%
\pgfsetlinewidth{1.003750pt}%
\definecolor{currentstroke}{rgb}{0.121569,0.466667,0.705882}%
\pgfsetstrokecolor{currentstroke}%
\pgfsetstrokeopacity{0.301229}%
\pgfsetdash{}{0pt}%
\pgfpathmoveto{\pgfqpoint{1.140196in}{1.616701in}}%
\pgfpathcurveto{\pgfqpoint{1.148433in}{1.616701in}}{\pgfqpoint{1.156333in}{1.619973in}}{\pgfqpoint{1.162157in}{1.625797in}}%
\pgfpathcurveto{\pgfqpoint{1.167981in}{1.631621in}}{\pgfqpoint{1.171253in}{1.639521in}}{\pgfqpoint{1.171253in}{1.647758in}}%
\pgfpathcurveto{\pgfqpoint{1.171253in}{1.655994in}}{\pgfqpoint{1.167981in}{1.663894in}}{\pgfqpoint{1.162157in}{1.669718in}}%
\pgfpathcurveto{\pgfqpoint{1.156333in}{1.675542in}}{\pgfqpoint{1.148433in}{1.678814in}}{\pgfqpoint{1.140196in}{1.678814in}}%
\pgfpathcurveto{\pgfqpoint{1.131960in}{1.678814in}}{\pgfqpoint{1.124060in}{1.675542in}}{\pgfqpoint{1.118236in}{1.669718in}}%
\pgfpathcurveto{\pgfqpoint{1.112412in}{1.663894in}}{\pgfqpoint{1.109140in}{1.655994in}}{\pgfqpoint{1.109140in}{1.647758in}}%
\pgfpathcurveto{\pgfqpoint{1.109140in}{1.639521in}}{\pgfqpoint{1.112412in}{1.631621in}}{\pgfqpoint{1.118236in}{1.625797in}}%
\pgfpathcurveto{\pgfqpoint{1.124060in}{1.619973in}}{\pgfqpoint{1.131960in}{1.616701in}}{\pgfqpoint{1.140196in}{1.616701in}}%
\pgfpathclose%
\pgfusepath{stroke,fill}%
\end{pgfscope}%
\begin{pgfscope}%
\pgfpathrectangle{\pgfqpoint{0.100000in}{0.212622in}}{\pgfqpoint{3.696000in}{3.696000in}}%
\pgfusepath{clip}%
\pgfsetbuttcap%
\pgfsetroundjoin%
\definecolor{currentfill}{rgb}{0.121569,0.466667,0.705882}%
\pgfsetfillcolor{currentfill}%
\pgfsetfillopacity{0.301321}%
\pgfsetlinewidth{1.003750pt}%
\definecolor{currentstroke}{rgb}{0.121569,0.466667,0.705882}%
\pgfsetstrokecolor{currentstroke}%
\pgfsetstrokeopacity{0.301321}%
\pgfsetdash{}{0pt}%
\pgfpathmoveto{\pgfqpoint{1.140440in}{1.616624in}}%
\pgfpathcurveto{\pgfqpoint{1.148676in}{1.616624in}}{\pgfqpoint{1.156576in}{1.619897in}}{\pgfqpoint{1.162400in}{1.625721in}}%
\pgfpathcurveto{\pgfqpoint{1.168224in}{1.631545in}}{\pgfqpoint{1.171496in}{1.639445in}}{\pgfqpoint{1.171496in}{1.647681in}}%
\pgfpathcurveto{\pgfqpoint{1.171496in}{1.655917in}}{\pgfqpoint{1.168224in}{1.663817in}}{\pgfqpoint{1.162400in}{1.669641in}}%
\pgfpathcurveto{\pgfqpoint{1.156576in}{1.675465in}}{\pgfqpoint{1.148676in}{1.678737in}}{\pgfqpoint{1.140440in}{1.678737in}}%
\pgfpathcurveto{\pgfqpoint{1.132203in}{1.678737in}}{\pgfqpoint{1.124303in}{1.675465in}}{\pgfqpoint{1.118479in}{1.669641in}}%
\pgfpathcurveto{\pgfqpoint{1.112655in}{1.663817in}}{\pgfqpoint{1.109383in}{1.655917in}}{\pgfqpoint{1.109383in}{1.647681in}}%
\pgfpathcurveto{\pgfqpoint{1.109383in}{1.639445in}}{\pgfqpoint{1.112655in}{1.631545in}}{\pgfqpoint{1.118479in}{1.625721in}}%
\pgfpathcurveto{\pgfqpoint{1.124303in}{1.619897in}}{\pgfqpoint{1.132203in}{1.616624in}}{\pgfqpoint{1.140440in}{1.616624in}}%
\pgfpathclose%
\pgfusepath{stroke,fill}%
\end{pgfscope}%
\begin{pgfscope}%
\pgfpathrectangle{\pgfqpoint{0.100000in}{0.212622in}}{\pgfqpoint{3.696000in}{3.696000in}}%
\pgfusepath{clip}%
\pgfsetbuttcap%
\pgfsetroundjoin%
\definecolor{currentfill}{rgb}{0.121569,0.466667,0.705882}%
\pgfsetfillcolor{currentfill}%
\pgfsetfillopacity{0.301646}%
\pgfsetlinewidth{1.003750pt}%
\definecolor{currentstroke}{rgb}{0.121569,0.466667,0.705882}%
\pgfsetstrokecolor{currentstroke}%
\pgfsetstrokeopacity{0.301646}%
\pgfsetdash{}{0pt}%
\pgfpathmoveto{\pgfqpoint{1.141298in}{1.616345in}}%
\pgfpathcurveto{\pgfqpoint{1.149535in}{1.616345in}}{\pgfqpoint{1.157435in}{1.619618in}}{\pgfqpoint{1.163259in}{1.625442in}}%
\pgfpathcurveto{\pgfqpoint{1.169082in}{1.631266in}}{\pgfqpoint{1.172355in}{1.639166in}}{\pgfqpoint{1.172355in}{1.647402in}}%
\pgfpathcurveto{\pgfqpoint{1.172355in}{1.655638in}}{\pgfqpoint{1.169082in}{1.663538in}}{\pgfqpoint{1.163259in}{1.669362in}}%
\pgfpathcurveto{\pgfqpoint{1.157435in}{1.675186in}}{\pgfqpoint{1.149535in}{1.678458in}}{\pgfqpoint{1.141298in}{1.678458in}}%
\pgfpathcurveto{\pgfqpoint{1.133062in}{1.678458in}}{\pgfqpoint{1.125162in}{1.675186in}}{\pgfqpoint{1.119338in}{1.669362in}}%
\pgfpathcurveto{\pgfqpoint{1.113514in}{1.663538in}}{\pgfqpoint{1.110242in}{1.655638in}}{\pgfqpoint{1.110242in}{1.647402in}}%
\pgfpathcurveto{\pgfqpoint{1.110242in}{1.639166in}}{\pgfqpoint{1.113514in}{1.631266in}}{\pgfqpoint{1.119338in}{1.625442in}}%
\pgfpathcurveto{\pgfqpoint{1.125162in}{1.619618in}}{\pgfqpoint{1.133062in}{1.616345in}}{\pgfqpoint{1.141298in}{1.616345in}}%
\pgfpathclose%
\pgfusepath{stroke,fill}%
\end{pgfscope}%
\begin{pgfscope}%
\pgfpathrectangle{\pgfqpoint{0.100000in}{0.212622in}}{\pgfqpoint{3.696000in}{3.696000in}}%
\pgfusepath{clip}%
\pgfsetbuttcap%
\pgfsetroundjoin%
\definecolor{currentfill}{rgb}{0.121569,0.466667,0.705882}%
\pgfsetfillcolor{currentfill}%
\pgfsetfillopacity{0.301899}%
\pgfsetlinewidth{1.003750pt}%
\definecolor{currentstroke}{rgb}{0.121569,0.466667,0.705882}%
\pgfsetstrokecolor{currentstroke}%
\pgfsetstrokeopacity{0.301899}%
\pgfsetdash{}{0pt}%
\pgfpathmoveto{\pgfqpoint{1.141693in}{1.616104in}}%
\pgfpathcurveto{\pgfqpoint{1.149929in}{1.616104in}}{\pgfqpoint{1.157829in}{1.619376in}}{\pgfqpoint{1.163653in}{1.625200in}}%
\pgfpathcurveto{\pgfqpoint{1.169477in}{1.631024in}}{\pgfqpoint{1.172749in}{1.638924in}}{\pgfqpoint{1.172749in}{1.647161in}}%
\pgfpathcurveto{\pgfqpoint{1.172749in}{1.655397in}}{\pgfqpoint{1.169477in}{1.663297in}}{\pgfqpoint{1.163653in}{1.669121in}}%
\pgfpathcurveto{\pgfqpoint{1.157829in}{1.674945in}}{\pgfqpoint{1.149929in}{1.678217in}}{\pgfqpoint{1.141693in}{1.678217in}}%
\pgfpathcurveto{\pgfqpoint{1.133456in}{1.678217in}}{\pgfqpoint{1.125556in}{1.674945in}}{\pgfqpoint{1.119732in}{1.669121in}}%
\pgfpathcurveto{\pgfqpoint{1.113909in}{1.663297in}}{\pgfqpoint{1.110636in}{1.655397in}}{\pgfqpoint{1.110636in}{1.647161in}}%
\pgfpathcurveto{\pgfqpoint{1.110636in}{1.638924in}}{\pgfqpoint{1.113909in}{1.631024in}}{\pgfqpoint{1.119732in}{1.625200in}}%
\pgfpathcurveto{\pgfqpoint{1.125556in}{1.619376in}}{\pgfqpoint{1.133456in}{1.616104in}}{\pgfqpoint{1.141693in}{1.616104in}}%
\pgfpathclose%
\pgfusepath{stroke,fill}%
\end{pgfscope}%
\begin{pgfscope}%
\pgfpathrectangle{\pgfqpoint{0.100000in}{0.212622in}}{\pgfqpoint{3.696000in}{3.696000in}}%
\pgfusepath{clip}%
\pgfsetbuttcap%
\pgfsetroundjoin%
\definecolor{currentfill}{rgb}{0.121569,0.466667,0.705882}%
\pgfsetfillcolor{currentfill}%
\pgfsetfillopacity{0.302338}%
\pgfsetlinewidth{1.003750pt}%
\definecolor{currentstroke}{rgb}{0.121569,0.466667,0.705882}%
\pgfsetstrokecolor{currentstroke}%
\pgfsetstrokeopacity{0.302338}%
\pgfsetdash{}{0pt}%
\pgfpathmoveto{\pgfqpoint{1.142590in}{1.615756in}}%
\pgfpathcurveto{\pgfqpoint{1.150826in}{1.615756in}}{\pgfqpoint{1.158726in}{1.619028in}}{\pgfqpoint{1.164550in}{1.624852in}}%
\pgfpathcurveto{\pgfqpoint{1.170374in}{1.630676in}}{\pgfqpoint{1.173646in}{1.638576in}}{\pgfqpoint{1.173646in}{1.646812in}}%
\pgfpathcurveto{\pgfqpoint{1.173646in}{1.655048in}}{\pgfqpoint{1.170374in}{1.662948in}}{\pgfqpoint{1.164550in}{1.668772in}}%
\pgfpathcurveto{\pgfqpoint{1.158726in}{1.674596in}}{\pgfqpoint{1.150826in}{1.677869in}}{\pgfqpoint{1.142590in}{1.677869in}}%
\pgfpathcurveto{\pgfqpoint{1.134354in}{1.677869in}}{\pgfqpoint{1.126454in}{1.674596in}}{\pgfqpoint{1.120630in}{1.668772in}}%
\pgfpathcurveto{\pgfqpoint{1.114806in}{1.662948in}}{\pgfqpoint{1.111533in}{1.655048in}}{\pgfqpoint{1.111533in}{1.646812in}}%
\pgfpathcurveto{\pgfqpoint{1.111533in}{1.638576in}}{\pgfqpoint{1.114806in}{1.630676in}}{\pgfqpoint{1.120630in}{1.624852in}}%
\pgfpathcurveto{\pgfqpoint{1.126454in}{1.619028in}}{\pgfqpoint{1.134354in}{1.615756in}}{\pgfqpoint{1.142590in}{1.615756in}}%
\pgfpathclose%
\pgfusepath{stroke,fill}%
\end{pgfscope}%
\begin{pgfscope}%
\pgfpathrectangle{\pgfqpoint{0.100000in}{0.212622in}}{\pgfqpoint{3.696000in}{3.696000in}}%
\pgfusepath{clip}%
\pgfsetbuttcap%
\pgfsetroundjoin%
\definecolor{currentfill}{rgb}{0.121569,0.466667,0.705882}%
\pgfsetfillcolor{currentfill}%
\pgfsetfillopacity{0.303057}%
\pgfsetlinewidth{1.003750pt}%
\definecolor{currentstroke}{rgb}{0.121569,0.466667,0.705882}%
\pgfsetstrokecolor{currentstroke}%
\pgfsetstrokeopacity{0.303057}%
\pgfsetdash{}{0pt}%
\pgfpathmoveto{\pgfqpoint{1.143890in}{1.615125in}}%
\pgfpathcurveto{\pgfqpoint{1.152126in}{1.615125in}}{\pgfqpoint{1.160026in}{1.618398in}}{\pgfqpoint{1.165850in}{1.624222in}}%
\pgfpathcurveto{\pgfqpoint{1.171674in}{1.630046in}}{\pgfqpoint{1.174947in}{1.637946in}}{\pgfqpoint{1.174947in}{1.646182in}}%
\pgfpathcurveto{\pgfqpoint{1.174947in}{1.654418in}}{\pgfqpoint{1.171674in}{1.662318in}}{\pgfqpoint{1.165850in}{1.668142in}}%
\pgfpathcurveto{\pgfqpoint{1.160026in}{1.673966in}}{\pgfqpoint{1.152126in}{1.677238in}}{\pgfqpoint{1.143890in}{1.677238in}}%
\pgfpathcurveto{\pgfqpoint{1.135654in}{1.677238in}}{\pgfqpoint{1.127754in}{1.673966in}}{\pgfqpoint{1.121930in}{1.668142in}}%
\pgfpathcurveto{\pgfqpoint{1.116106in}{1.662318in}}{\pgfqpoint{1.112834in}{1.654418in}}{\pgfqpoint{1.112834in}{1.646182in}}%
\pgfpathcurveto{\pgfqpoint{1.112834in}{1.637946in}}{\pgfqpoint{1.116106in}{1.630046in}}{\pgfqpoint{1.121930in}{1.624222in}}%
\pgfpathcurveto{\pgfqpoint{1.127754in}{1.618398in}}{\pgfqpoint{1.135654in}{1.615125in}}{\pgfqpoint{1.143890in}{1.615125in}}%
\pgfpathclose%
\pgfusepath{stroke,fill}%
\end{pgfscope}%
\begin{pgfscope}%
\pgfpathrectangle{\pgfqpoint{0.100000in}{0.212622in}}{\pgfqpoint{3.696000in}{3.696000in}}%
\pgfusepath{clip}%
\pgfsetbuttcap%
\pgfsetroundjoin%
\definecolor{currentfill}{rgb}{0.121569,0.466667,0.705882}%
\pgfsetfillcolor{currentfill}%
\pgfsetfillopacity{0.303423}%
\pgfsetlinewidth{1.003750pt}%
\definecolor{currentstroke}{rgb}{0.121569,0.466667,0.705882}%
\pgfsetstrokecolor{currentstroke}%
\pgfsetstrokeopacity{0.303423}%
\pgfsetdash{}{0pt}%
\pgfpathmoveto{\pgfqpoint{1.144634in}{1.614805in}}%
\pgfpathcurveto{\pgfqpoint{1.152870in}{1.614805in}}{\pgfqpoint{1.160770in}{1.618078in}}{\pgfqpoint{1.166594in}{1.623902in}}%
\pgfpathcurveto{\pgfqpoint{1.172418in}{1.629725in}}{\pgfqpoint{1.175690in}{1.637626in}}{\pgfqpoint{1.175690in}{1.645862in}}%
\pgfpathcurveto{\pgfqpoint{1.175690in}{1.654098in}}{\pgfqpoint{1.172418in}{1.661998in}}{\pgfqpoint{1.166594in}{1.667822in}}%
\pgfpathcurveto{\pgfqpoint{1.160770in}{1.673646in}}{\pgfqpoint{1.152870in}{1.676918in}}{\pgfqpoint{1.144634in}{1.676918in}}%
\pgfpathcurveto{\pgfqpoint{1.136398in}{1.676918in}}{\pgfqpoint{1.128497in}{1.673646in}}{\pgfqpoint{1.122674in}{1.667822in}}%
\pgfpathcurveto{\pgfqpoint{1.116850in}{1.661998in}}{\pgfqpoint{1.113577in}{1.654098in}}{\pgfqpoint{1.113577in}{1.645862in}}%
\pgfpathcurveto{\pgfqpoint{1.113577in}{1.637626in}}{\pgfqpoint{1.116850in}{1.629725in}}{\pgfqpoint{1.122674in}{1.623902in}}%
\pgfpathcurveto{\pgfqpoint{1.128497in}{1.618078in}}{\pgfqpoint{1.136398in}{1.614805in}}{\pgfqpoint{1.144634in}{1.614805in}}%
\pgfpathclose%
\pgfusepath{stroke,fill}%
\end{pgfscope}%
\begin{pgfscope}%
\pgfpathrectangle{\pgfqpoint{0.100000in}{0.212622in}}{\pgfqpoint{3.696000in}{3.696000in}}%
\pgfusepath{clip}%
\pgfsetbuttcap%
\pgfsetroundjoin%
\definecolor{currentfill}{rgb}{0.121569,0.466667,0.705882}%
\pgfsetfillcolor{currentfill}%
\pgfsetfillopacity{0.304085}%
\pgfsetlinewidth{1.003750pt}%
\definecolor{currentstroke}{rgb}{0.121569,0.466667,0.705882}%
\pgfsetstrokecolor{currentstroke}%
\pgfsetstrokeopacity{0.304085}%
\pgfsetdash{}{0pt}%
\pgfpathmoveto{\pgfqpoint{1.146067in}{1.614291in}}%
\pgfpathcurveto{\pgfqpoint{1.154303in}{1.614291in}}{\pgfqpoint{1.162203in}{1.617563in}}{\pgfqpoint{1.168027in}{1.623387in}}%
\pgfpathcurveto{\pgfqpoint{1.173851in}{1.629211in}}{\pgfqpoint{1.177124in}{1.637111in}}{\pgfqpoint{1.177124in}{1.645347in}}%
\pgfpathcurveto{\pgfqpoint{1.177124in}{1.653583in}}{\pgfqpoint{1.173851in}{1.661483in}}{\pgfqpoint{1.168027in}{1.667307in}}%
\pgfpathcurveto{\pgfqpoint{1.162203in}{1.673131in}}{\pgfqpoint{1.154303in}{1.676404in}}{\pgfqpoint{1.146067in}{1.676404in}}%
\pgfpathcurveto{\pgfqpoint{1.137831in}{1.676404in}}{\pgfqpoint{1.129931in}{1.673131in}}{\pgfqpoint{1.124107in}{1.667307in}}%
\pgfpathcurveto{\pgfqpoint{1.118283in}{1.661483in}}{\pgfqpoint{1.115011in}{1.653583in}}{\pgfqpoint{1.115011in}{1.645347in}}%
\pgfpathcurveto{\pgfqpoint{1.115011in}{1.637111in}}{\pgfqpoint{1.118283in}{1.629211in}}{\pgfqpoint{1.124107in}{1.623387in}}%
\pgfpathcurveto{\pgfqpoint{1.129931in}{1.617563in}}{\pgfqpoint{1.137831in}{1.614291in}}{\pgfqpoint{1.146067in}{1.614291in}}%
\pgfpathclose%
\pgfusepath{stroke,fill}%
\end{pgfscope}%
\begin{pgfscope}%
\pgfpathrectangle{\pgfqpoint{0.100000in}{0.212622in}}{\pgfqpoint{3.696000in}{3.696000in}}%
\pgfusepath{clip}%
\pgfsetbuttcap%
\pgfsetroundjoin%
\definecolor{currentfill}{rgb}{0.121569,0.466667,0.705882}%
\pgfsetfillcolor{currentfill}%
\pgfsetfillopacity{0.305085}%
\pgfsetlinewidth{1.003750pt}%
\definecolor{currentstroke}{rgb}{0.121569,0.466667,0.705882}%
\pgfsetstrokecolor{currentstroke}%
\pgfsetstrokeopacity{0.305085}%
\pgfsetdash{}{0pt}%
\pgfpathmoveto{\pgfqpoint{1.147806in}{1.613408in}}%
\pgfpathcurveto{\pgfqpoint{1.156043in}{1.613408in}}{\pgfqpoint{1.163943in}{1.616681in}}{\pgfqpoint{1.169767in}{1.622505in}}%
\pgfpathcurveto{\pgfqpoint{1.175591in}{1.628328in}}{\pgfqpoint{1.178863in}{1.636229in}}{\pgfqpoint{1.178863in}{1.644465in}}%
\pgfpathcurveto{\pgfqpoint{1.178863in}{1.652701in}}{\pgfqpoint{1.175591in}{1.660601in}}{\pgfqpoint{1.169767in}{1.666425in}}%
\pgfpathcurveto{\pgfqpoint{1.163943in}{1.672249in}}{\pgfqpoint{1.156043in}{1.675521in}}{\pgfqpoint{1.147806in}{1.675521in}}%
\pgfpathcurveto{\pgfqpoint{1.139570in}{1.675521in}}{\pgfqpoint{1.131670in}{1.672249in}}{\pgfqpoint{1.125846in}{1.666425in}}%
\pgfpathcurveto{\pgfqpoint{1.120022in}{1.660601in}}{\pgfqpoint{1.116750in}{1.652701in}}{\pgfqpoint{1.116750in}{1.644465in}}%
\pgfpathcurveto{\pgfqpoint{1.116750in}{1.636229in}}{\pgfqpoint{1.120022in}{1.628328in}}{\pgfqpoint{1.125846in}{1.622505in}}%
\pgfpathcurveto{\pgfqpoint{1.131670in}{1.616681in}}{\pgfqpoint{1.139570in}{1.613408in}}{\pgfqpoint{1.147806in}{1.613408in}}%
\pgfpathclose%
\pgfusepath{stroke,fill}%
\end{pgfscope}%
\begin{pgfscope}%
\pgfpathrectangle{\pgfqpoint{0.100000in}{0.212622in}}{\pgfqpoint{3.696000in}{3.696000in}}%
\pgfusepath{clip}%
\pgfsetbuttcap%
\pgfsetroundjoin%
\definecolor{currentfill}{rgb}{0.121569,0.466667,0.705882}%
\pgfsetfillcolor{currentfill}%
\pgfsetfillopacity{0.305607}%
\pgfsetlinewidth{1.003750pt}%
\definecolor{currentstroke}{rgb}{0.121569,0.466667,0.705882}%
\pgfsetstrokecolor{currentstroke}%
\pgfsetstrokeopacity{0.305607}%
\pgfsetdash{}{0pt}%
\pgfpathmoveto{\pgfqpoint{1.148800in}{1.612980in}}%
\pgfpathcurveto{\pgfqpoint{1.157036in}{1.612980in}}{\pgfqpoint{1.164936in}{1.616252in}}{\pgfqpoint{1.170760in}{1.622076in}}%
\pgfpathcurveto{\pgfqpoint{1.176584in}{1.627900in}}{\pgfqpoint{1.179856in}{1.635800in}}{\pgfqpoint{1.179856in}{1.644036in}}%
\pgfpathcurveto{\pgfqpoint{1.179856in}{1.652273in}}{\pgfqpoint{1.176584in}{1.660173in}}{\pgfqpoint{1.170760in}{1.665997in}}%
\pgfpathcurveto{\pgfqpoint{1.164936in}{1.671820in}}{\pgfqpoint{1.157036in}{1.675093in}}{\pgfqpoint{1.148800in}{1.675093in}}%
\pgfpathcurveto{\pgfqpoint{1.140563in}{1.675093in}}{\pgfqpoint{1.132663in}{1.671820in}}{\pgfqpoint{1.126839in}{1.665997in}}%
\pgfpathcurveto{\pgfqpoint{1.121015in}{1.660173in}}{\pgfqpoint{1.117743in}{1.652273in}}{\pgfqpoint{1.117743in}{1.644036in}}%
\pgfpathcurveto{\pgfqpoint{1.117743in}{1.635800in}}{\pgfqpoint{1.121015in}{1.627900in}}{\pgfqpoint{1.126839in}{1.622076in}}%
\pgfpathcurveto{\pgfqpoint{1.132663in}{1.616252in}}{\pgfqpoint{1.140563in}{1.612980in}}{\pgfqpoint{1.148800in}{1.612980in}}%
\pgfpathclose%
\pgfusepath{stroke,fill}%
\end{pgfscope}%
\begin{pgfscope}%
\pgfpathrectangle{\pgfqpoint{0.100000in}{0.212622in}}{\pgfqpoint{3.696000in}{3.696000in}}%
\pgfusepath{clip}%
\pgfsetbuttcap%
\pgfsetroundjoin%
\definecolor{currentfill}{rgb}{0.121569,0.466667,0.705882}%
\pgfsetfillcolor{currentfill}%
\pgfsetfillopacity{0.306459}%
\pgfsetlinewidth{1.003750pt}%
\definecolor{currentstroke}{rgb}{0.121569,0.466667,0.705882}%
\pgfsetstrokecolor{currentstroke}%
\pgfsetstrokeopacity{0.306459}%
\pgfsetdash{}{0pt}%
\pgfpathmoveto{\pgfqpoint{1.150437in}{1.612284in}}%
\pgfpathcurveto{\pgfqpoint{1.158674in}{1.612284in}}{\pgfqpoint{1.166574in}{1.615556in}}{\pgfqpoint{1.172398in}{1.621380in}}%
\pgfpathcurveto{\pgfqpoint{1.178222in}{1.627204in}}{\pgfqpoint{1.181494in}{1.635104in}}{\pgfqpoint{1.181494in}{1.643341in}}%
\pgfpathcurveto{\pgfqpoint{1.181494in}{1.651577in}}{\pgfqpoint{1.178222in}{1.659477in}}{\pgfqpoint{1.172398in}{1.665301in}}%
\pgfpathcurveto{\pgfqpoint{1.166574in}{1.671125in}}{\pgfqpoint{1.158674in}{1.674397in}}{\pgfqpoint{1.150437in}{1.674397in}}%
\pgfpathcurveto{\pgfqpoint{1.142201in}{1.674397in}}{\pgfqpoint{1.134301in}{1.671125in}}{\pgfqpoint{1.128477in}{1.665301in}}%
\pgfpathcurveto{\pgfqpoint{1.122653in}{1.659477in}}{\pgfqpoint{1.119381in}{1.651577in}}{\pgfqpoint{1.119381in}{1.643341in}}%
\pgfpathcurveto{\pgfqpoint{1.119381in}{1.635104in}}{\pgfqpoint{1.122653in}{1.627204in}}{\pgfqpoint{1.128477in}{1.621380in}}%
\pgfpathcurveto{\pgfqpoint{1.134301in}{1.615556in}}{\pgfqpoint{1.142201in}{1.612284in}}{\pgfqpoint{1.150437in}{1.612284in}}%
\pgfpathclose%
\pgfusepath{stroke,fill}%
\end{pgfscope}%
\begin{pgfscope}%
\pgfpathrectangle{\pgfqpoint{0.100000in}{0.212622in}}{\pgfqpoint{3.696000in}{3.696000in}}%
\pgfusepath{clip}%
\pgfsetbuttcap%
\pgfsetroundjoin%
\definecolor{currentfill}{rgb}{0.121569,0.466667,0.705882}%
\pgfsetfillcolor{currentfill}%
\pgfsetfillopacity{0.307753}%
\pgfsetlinewidth{1.003750pt}%
\definecolor{currentstroke}{rgb}{0.121569,0.466667,0.705882}%
\pgfsetstrokecolor{currentstroke}%
\pgfsetstrokeopacity{0.307753}%
\pgfsetdash{}{0pt}%
\pgfpathmoveto{\pgfqpoint{1.152695in}{1.611112in}}%
\pgfpathcurveto{\pgfqpoint{1.160931in}{1.611112in}}{\pgfqpoint{1.168831in}{1.614385in}}{\pgfqpoint{1.174655in}{1.620209in}}%
\pgfpathcurveto{\pgfqpoint{1.180479in}{1.626033in}}{\pgfqpoint{1.183751in}{1.633933in}}{\pgfqpoint{1.183751in}{1.642169in}}%
\pgfpathcurveto{\pgfqpoint{1.183751in}{1.650405in}}{\pgfqpoint{1.180479in}{1.658305in}}{\pgfqpoint{1.174655in}{1.664129in}}%
\pgfpathcurveto{\pgfqpoint{1.168831in}{1.669953in}}{\pgfqpoint{1.160931in}{1.673225in}}{\pgfqpoint{1.152695in}{1.673225in}}%
\pgfpathcurveto{\pgfqpoint{1.144459in}{1.673225in}}{\pgfqpoint{1.136559in}{1.669953in}}{\pgfqpoint{1.130735in}{1.664129in}}%
\pgfpathcurveto{\pgfqpoint{1.124911in}{1.658305in}}{\pgfqpoint{1.121638in}{1.650405in}}{\pgfqpoint{1.121638in}{1.642169in}}%
\pgfpathcurveto{\pgfqpoint{1.121638in}{1.633933in}}{\pgfqpoint{1.124911in}{1.626033in}}{\pgfqpoint{1.130735in}{1.620209in}}%
\pgfpathcurveto{\pgfqpoint{1.136559in}{1.614385in}}{\pgfqpoint{1.144459in}{1.611112in}}{\pgfqpoint{1.152695in}{1.611112in}}%
\pgfpathclose%
\pgfusepath{stroke,fill}%
\end{pgfscope}%
\begin{pgfscope}%
\pgfpathrectangle{\pgfqpoint{0.100000in}{0.212622in}}{\pgfqpoint{3.696000in}{3.696000in}}%
\pgfusepath{clip}%
\pgfsetbuttcap%
\pgfsetroundjoin%
\definecolor{currentfill}{rgb}{0.121569,0.466667,0.705882}%
\pgfsetfillcolor{currentfill}%
\pgfsetfillopacity{0.308402}%
\pgfsetlinewidth{1.003750pt}%
\definecolor{currentstroke}{rgb}{0.121569,0.466667,0.705882}%
\pgfsetstrokecolor{currentstroke}%
\pgfsetstrokeopacity{0.308402}%
\pgfsetdash{}{0pt}%
\pgfpathmoveto{\pgfqpoint{1.154015in}{1.610585in}}%
\pgfpathcurveto{\pgfqpoint{1.162252in}{1.610585in}}{\pgfqpoint{1.170152in}{1.613857in}}{\pgfqpoint{1.175976in}{1.619681in}}%
\pgfpathcurveto{\pgfqpoint{1.181800in}{1.625505in}}{\pgfqpoint{1.185072in}{1.633405in}}{\pgfqpoint{1.185072in}{1.641641in}}%
\pgfpathcurveto{\pgfqpoint{1.185072in}{1.649877in}}{\pgfqpoint{1.181800in}{1.657777in}}{\pgfqpoint{1.175976in}{1.663601in}}%
\pgfpathcurveto{\pgfqpoint{1.170152in}{1.669425in}}{\pgfqpoint{1.162252in}{1.672698in}}{\pgfqpoint{1.154015in}{1.672698in}}%
\pgfpathcurveto{\pgfqpoint{1.145779in}{1.672698in}}{\pgfqpoint{1.137879in}{1.669425in}}{\pgfqpoint{1.132055in}{1.663601in}}%
\pgfpathcurveto{\pgfqpoint{1.126231in}{1.657777in}}{\pgfqpoint{1.122959in}{1.649877in}}{\pgfqpoint{1.122959in}{1.641641in}}%
\pgfpathcurveto{\pgfqpoint{1.122959in}{1.633405in}}{\pgfqpoint{1.126231in}{1.625505in}}{\pgfqpoint{1.132055in}{1.619681in}}%
\pgfpathcurveto{\pgfqpoint{1.137879in}{1.613857in}}{\pgfqpoint{1.145779in}{1.610585in}}{\pgfqpoint{1.154015in}{1.610585in}}%
\pgfpathclose%
\pgfusepath{stroke,fill}%
\end{pgfscope}%
\begin{pgfscope}%
\pgfpathrectangle{\pgfqpoint{0.100000in}{0.212622in}}{\pgfqpoint{3.696000in}{3.696000in}}%
\pgfusepath{clip}%
\pgfsetbuttcap%
\pgfsetroundjoin%
\definecolor{currentfill}{rgb}{0.121569,0.466667,0.705882}%
\pgfsetfillcolor{currentfill}%
\pgfsetfillopacity{0.309395}%
\pgfsetlinewidth{1.003750pt}%
\definecolor{currentstroke}{rgb}{0.121569,0.466667,0.705882}%
\pgfsetstrokecolor{currentstroke}%
\pgfsetstrokeopacity{0.309395}%
\pgfsetdash{}{0pt}%
\pgfpathmoveto{\pgfqpoint{1.155592in}{1.609765in}}%
\pgfpathcurveto{\pgfqpoint{1.163828in}{1.609765in}}{\pgfqpoint{1.171728in}{1.613037in}}{\pgfqpoint{1.177552in}{1.618861in}}%
\pgfpathcurveto{\pgfqpoint{1.183376in}{1.624685in}}{\pgfqpoint{1.186648in}{1.632585in}}{\pgfqpoint{1.186648in}{1.640821in}}%
\pgfpathcurveto{\pgfqpoint{1.186648in}{1.649058in}}{\pgfqpoint{1.183376in}{1.656958in}}{\pgfqpoint{1.177552in}{1.662782in}}%
\pgfpathcurveto{\pgfqpoint{1.171728in}{1.668605in}}{\pgfqpoint{1.163828in}{1.671878in}}{\pgfqpoint{1.155592in}{1.671878in}}%
\pgfpathcurveto{\pgfqpoint{1.147355in}{1.671878in}}{\pgfqpoint{1.139455in}{1.668605in}}{\pgfqpoint{1.133631in}{1.662782in}}%
\pgfpathcurveto{\pgfqpoint{1.127808in}{1.656958in}}{\pgfqpoint{1.124535in}{1.649058in}}{\pgfqpoint{1.124535in}{1.640821in}}%
\pgfpathcurveto{\pgfqpoint{1.124535in}{1.632585in}}{\pgfqpoint{1.127808in}{1.624685in}}{\pgfqpoint{1.133631in}{1.618861in}}%
\pgfpathcurveto{\pgfqpoint{1.139455in}{1.613037in}}{\pgfqpoint{1.147355in}{1.609765in}}{\pgfqpoint{1.155592in}{1.609765in}}%
\pgfpathclose%
\pgfusepath{stroke,fill}%
\end{pgfscope}%
\begin{pgfscope}%
\pgfpathrectangle{\pgfqpoint{0.100000in}{0.212622in}}{\pgfqpoint{3.696000in}{3.696000in}}%
\pgfusepath{clip}%
\pgfsetbuttcap%
\pgfsetroundjoin%
\definecolor{currentfill}{rgb}{0.121569,0.466667,0.705882}%
\pgfsetfillcolor{currentfill}%
\pgfsetfillopacity{0.310649}%
\pgfsetlinewidth{1.003750pt}%
\definecolor{currentstroke}{rgb}{0.121569,0.466667,0.705882}%
\pgfsetstrokecolor{currentstroke}%
\pgfsetstrokeopacity{0.310649}%
\pgfsetdash{}{0pt}%
\pgfpathmoveto{\pgfqpoint{1.157810in}{1.608700in}}%
\pgfpathcurveto{\pgfqpoint{1.166047in}{1.608700in}}{\pgfqpoint{1.173947in}{1.611972in}}{\pgfqpoint{1.179771in}{1.617796in}}%
\pgfpathcurveto{\pgfqpoint{1.185595in}{1.623620in}}{\pgfqpoint{1.188867in}{1.631520in}}{\pgfqpoint{1.188867in}{1.639756in}}%
\pgfpathcurveto{\pgfqpoint{1.188867in}{1.647993in}}{\pgfqpoint{1.185595in}{1.655893in}}{\pgfqpoint{1.179771in}{1.661717in}}%
\pgfpathcurveto{\pgfqpoint{1.173947in}{1.667540in}}{\pgfqpoint{1.166047in}{1.670813in}}{\pgfqpoint{1.157810in}{1.670813in}}%
\pgfpathcurveto{\pgfqpoint{1.149574in}{1.670813in}}{\pgfqpoint{1.141674in}{1.667540in}}{\pgfqpoint{1.135850in}{1.661717in}}%
\pgfpathcurveto{\pgfqpoint{1.130026in}{1.655893in}}{\pgfqpoint{1.126754in}{1.647993in}}{\pgfqpoint{1.126754in}{1.639756in}}%
\pgfpathcurveto{\pgfqpoint{1.126754in}{1.631520in}}{\pgfqpoint{1.130026in}{1.623620in}}{\pgfqpoint{1.135850in}{1.617796in}}%
\pgfpathcurveto{\pgfqpoint{1.141674in}{1.611972in}}{\pgfqpoint{1.149574in}{1.608700in}}{\pgfqpoint{1.157810in}{1.608700in}}%
\pgfpathclose%
\pgfusepath{stroke,fill}%
\end{pgfscope}%
\begin{pgfscope}%
\pgfpathrectangle{\pgfqpoint{0.100000in}{0.212622in}}{\pgfqpoint{3.696000in}{3.696000in}}%
\pgfusepath{clip}%
\pgfsetbuttcap%
\pgfsetroundjoin%
\definecolor{currentfill}{rgb}{0.121569,0.466667,0.705882}%
\pgfsetfillcolor{currentfill}%
\pgfsetfillopacity{0.311490}%
\pgfsetlinewidth{1.003750pt}%
\definecolor{currentstroke}{rgb}{0.121569,0.466667,0.705882}%
\pgfsetstrokecolor{currentstroke}%
\pgfsetstrokeopacity{0.311490}%
\pgfsetdash{}{0pt}%
\pgfpathmoveto{\pgfqpoint{1.158889in}{1.607996in}}%
\pgfpathcurveto{\pgfqpoint{1.167126in}{1.607996in}}{\pgfqpoint{1.175026in}{1.611268in}}{\pgfqpoint{1.180850in}{1.617092in}}%
\pgfpathcurveto{\pgfqpoint{1.186674in}{1.622916in}}{\pgfqpoint{1.189946in}{1.630816in}}{\pgfqpoint{1.189946in}{1.639052in}}%
\pgfpathcurveto{\pgfqpoint{1.189946in}{1.647289in}}{\pgfqpoint{1.186674in}{1.655189in}}{\pgfqpoint{1.180850in}{1.661013in}}%
\pgfpathcurveto{\pgfqpoint{1.175026in}{1.666836in}}{\pgfqpoint{1.167126in}{1.670109in}}{\pgfqpoint{1.158889in}{1.670109in}}%
\pgfpathcurveto{\pgfqpoint{1.150653in}{1.670109in}}{\pgfqpoint{1.142753in}{1.666836in}}{\pgfqpoint{1.136929in}{1.661013in}}%
\pgfpathcurveto{\pgfqpoint{1.131105in}{1.655189in}}{\pgfqpoint{1.127833in}{1.647289in}}{\pgfqpoint{1.127833in}{1.639052in}}%
\pgfpathcurveto{\pgfqpoint{1.127833in}{1.630816in}}{\pgfqpoint{1.131105in}{1.622916in}}{\pgfqpoint{1.136929in}{1.617092in}}%
\pgfpathcurveto{\pgfqpoint{1.142753in}{1.611268in}}{\pgfqpoint{1.150653in}{1.607996in}}{\pgfqpoint{1.158889in}{1.607996in}}%
\pgfpathclose%
\pgfusepath{stroke,fill}%
\end{pgfscope}%
\begin{pgfscope}%
\pgfpathrectangle{\pgfqpoint{0.100000in}{0.212622in}}{\pgfqpoint{3.696000in}{3.696000in}}%
\pgfusepath{clip}%
\pgfsetbuttcap%
\pgfsetroundjoin%
\definecolor{currentfill}{rgb}{0.121569,0.466667,0.705882}%
\pgfsetfillcolor{currentfill}%
\pgfsetfillopacity{0.313335}%
\pgfsetlinewidth{1.003750pt}%
\definecolor{currentstroke}{rgb}{0.121569,0.466667,0.705882}%
\pgfsetstrokecolor{currentstroke}%
\pgfsetstrokeopacity{0.313335}%
\pgfsetdash{}{0pt}%
\pgfpathmoveto{\pgfqpoint{1.162180in}{1.606518in}}%
\pgfpathcurveto{\pgfqpoint{1.170417in}{1.606518in}}{\pgfqpoint{1.178317in}{1.609790in}}{\pgfqpoint{1.184141in}{1.615614in}}%
\pgfpathcurveto{\pgfqpoint{1.189964in}{1.621438in}}{\pgfqpoint{1.193237in}{1.629338in}}{\pgfqpoint{1.193237in}{1.637574in}}%
\pgfpathcurveto{\pgfqpoint{1.193237in}{1.645811in}}{\pgfqpoint{1.189964in}{1.653711in}}{\pgfqpoint{1.184141in}{1.659535in}}%
\pgfpathcurveto{\pgfqpoint{1.178317in}{1.665359in}}{\pgfqpoint{1.170417in}{1.668631in}}{\pgfqpoint{1.162180in}{1.668631in}}%
\pgfpathcurveto{\pgfqpoint{1.153944in}{1.668631in}}{\pgfqpoint{1.146044in}{1.665359in}}{\pgfqpoint{1.140220in}{1.659535in}}%
\pgfpathcurveto{\pgfqpoint{1.134396in}{1.653711in}}{\pgfqpoint{1.131124in}{1.645811in}}{\pgfqpoint{1.131124in}{1.637574in}}%
\pgfpathcurveto{\pgfqpoint{1.131124in}{1.629338in}}{\pgfqpoint{1.134396in}{1.621438in}}{\pgfqpoint{1.140220in}{1.615614in}}%
\pgfpathcurveto{\pgfqpoint{1.146044in}{1.609790in}}{\pgfqpoint{1.153944in}{1.606518in}}{\pgfqpoint{1.162180in}{1.606518in}}%
\pgfpathclose%
\pgfusepath{stroke,fill}%
\end{pgfscope}%
\begin{pgfscope}%
\pgfpathrectangle{\pgfqpoint{0.100000in}{0.212622in}}{\pgfqpoint{3.696000in}{3.696000in}}%
\pgfusepath{clip}%
\pgfsetbuttcap%
\pgfsetroundjoin%
\definecolor{currentfill}{rgb}{0.121569,0.466667,0.705882}%
\pgfsetfillcolor{currentfill}%
\pgfsetfillopacity{0.315535}%
\pgfsetlinewidth{1.003750pt}%
\definecolor{currentstroke}{rgb}{0.121569,0.466667,0.705882}%
\pgfsetstrokecolor{currentstroke}%
\pgfsetstrokeopacity{0.315535}%
\pgfsetdash{}{0pt}%
\pgfpathmoveto{\pgfqpoint{1.166068in}{1.604642in}}%
\pgfpathcurveto{\pgfqpoint{1.174304in}{1.604642in}}{\pgfqpoint{1.182204in}{1.607914in}}{\pgfqpoint{1.188028in}{1.613738in}}%
\pgfpathcurveto{\pgfqpoint{1.193852in}{1.619562in}}{\pgfqpoint{1.197125in}{1.627462in}}{\pgfqpoint{1.197125in}{1.635699in}}%
\pgfpathcurveto{\pgfqpoint{1.197125in}{1.643935in}}{\pgfqpoint{1.193852in}{1.651835in}}{\pgfqpoint{1.188028in}{1.657659in}}%
\pgfpathcurveto{\pgfqpoint{1.182204in}{1.663483in}}{\pgfqpoint{1.174304in}{1.666755in}}{\pgfqpoint{1.166068in}{1.666755in}}%
\pgfpathcurveto{\pgfqpoint{1.157832in}{1.666755in}}{\pgfqpoint{1.149932in}{1.663483in}}{\pgfqpoint{1.144108in}{1.657659in}}%
\pgfpathcurveto{\pgfqpoint{1.138284in}{1.651835in}}{\pgfqpoint{1.135012in}{1.643935in}}{\pgfqpoint{1.135012in}{1.635699in}}%
\pgfpathcurveto{\pgfqpoint{1.135012in}{1.627462in}}{\pgfqpoint{1.138284in}{1.619562in}}{\pgfqpoint{1.144108in}{1.613738in}}%
\pgfpathcurveto{\pgfqpoint{1.149932in}{1.607914in}}{\pgfqpoint{1.157832in}{1.604642in}}{\pgfqpoint{1.166068in}{1.604642in}}%
\pgfpathclose%
\pgfusepath{stroke,fill}%
\end{pgfscope}%
\begin{pgfscope}%
\pgfpathrectangle{\pgfqpoint{0.100000in}{0.212622in}}{\pgfqpoint{3.696000in}{3.696000in}}%
\pgfusepath{clip}%
\pgfsetbuttcap%
\pgfsetroundjoin%
\definecolor{currentfill}{rgb}{0.121569,0.466667,0.705882}%
\pgfsetfillcolor{currentfill}%
\pgfsetfillopacity{0.316762}%
\pgfsetlinewidth{1.003750pt}%
\definecolor{currentstroke}{rgb}{0.121569,0.466667,0.705882}%
\pgfsetstrokecolor{currentstroke}%
\pgfsetstrokeopacity{0.316762}%
\pgfsetdash{}{0pt}%
\pgfpathmoveto{\pgfqpoint{1.168215in}{1.603674in}}%
\pgfpathcurveto{\pgfqpoint{1.176451in}{1.603674in}}{\pgfqpoint{1.184351in}{1.606946in}}{\pgfqpoint{1.190175in}{1.612770in}}%
\pgfpathcurveto{\pgfqpoint{1.195999in}{1.618594in}}{\pgfqpoint{1.199271in}{1.626494in}}{\pgfqpoint{1.199271in}{1.634730in}}%
\pgfpathcurveto{\pgfqpoint{1.199271in}{1.642966in}}{\pgfqpoint{1.195999in}{1.650867in}}{\pgfqpoint{1.190175in}{1.656690in}}%
\pgfpathcurveto{\pgfqpoint{1.184351in}{1.662514in}}{\pgfqpoint{1.176451in}{1.665787in}}{\pgfqpoint{1.168215in}{1.665787in}}%
\pgfpathcurveto{\pgfqpoint{1.159978in}{1.665787in}}{\pgfqpoint{1.152078in}{1.662514in}}{\pgfqpoint{1.146254in}{1.656690in}}%
\pgfpathcurveto{\pgfqpoint{1.140430in}{1.650867in}}{\pgfqpoint{1.137158in}{1.642966in}}{\pgfqpoint{1.137158in}{1.634730in}}%
\pgfpathcurveto{\pgfqpoint{1.137158in}{1.626494in}}{\pgfqpoint{1.140430in}{1.618594in}}{\pgfqpoint{1.146254in}{1.612770in}}%
\pgfpathcurveto{\pgfqpoint{1.152078in}{1.606946in}}{\pgfqpoint{1.159978in}{1.603674in}}{\pgfqpoint{1.168215in}{1.603674in}}%
\pgfpathclose%
\pgfusepath{stroke,fill}%
\end{pgfscope}%
\begin{pgfscope}%
\pgfpathrectangle{\pgfqpoint{0.100000in}{0.212622in}}{\pgfqpoint{3.696000in}{3.696000in}}%
\pgfusepath{clip}%
\pgfsetbuttcap%
\pgfsetroundjoin%
\definecolor{currentfill}{rgb}{0.121569,0.466667,0.705882}%
\pgfsetfillcolor{currentfill}%
\pgfsetfillopacity{0.317336}%
\pgfsetlinewidth{1.003750pt}%
\definecolor{currentstroke}{rgb}{0.121569,0.466667,0.705882}%
\pgfsetstrokecolor{currentstroke}%
\pgfsetstrokeopacity{0.317336}%
\pgfsetdash{}{0pt}%
\pgfpathmoveto{\pgfqpoint{1.169489in}{1.603222in}}%
\pgfpathcurveto{\pgfqpoint{1.177725in}{1.603222in}}{\pgfqpoint{1.185625in}{1.606494in}}{\pgfqpoint{1.191449in}{1.612318in}}%
\pgfpathcurveto{\pgfqpoint{1.197273in}{1.618142in}}{\pgfqpoint{1.200545in}{1.626042in}}{\pgfqpoint{1.200545in}{1.634278in}}%
\pgfpathcurveto{\pgfqpoint{1.200545in}{1.642514in}}{\pgfqpoint{1.197273in}{1.650414in}}{\pgfqpoint{1.191449in}{1.656238in}}%
\pgfpathcurveto{\pgfqpoint{1.185625in}{1.662062in}}{\pgfqpoint{1.177725in}{1.665335in}}{\pgfqpoint{1.169489in}{1.665335in}}%
\pgfpathcurveto{\pgfqpoint{1.161252in}{1.665335in}}{\pgfqpoint{1.153352in}{1.662062in}}{\pgfqpoint{1.147528in}{1.656238in}}%
\pgfpathcurveto{\pgfqpoint{1.141704in}{1.650414in}}{\pgfqpoint{1.138432in}{1.642514in}}{\pgfqpoint{1.138432in}{1.634278in}}%
\pgfpathcurveto{\pgfqpoint{1.138432in}{1.626042in}}{\pgfqpoint{1.141704in}{1.618142in}}{\pgfqpoint{1.147528in}{1.612318in}}%
\pgfpathcurveto{\pgfqpoint{1.153352in}{1.606494in}}{\pgfqpoint{1.161252in}{1.603222in}}{\pgfqpoint{1.169489in}{1.603222in}}%
\pgfpathclose%
\pgfusepath{stroke,fill}%
\end{pgfscope}%
\begin{pgfscope}%
\pgfpathrectangle{\pgfqpoint{0.100000in}{0.212622in}}{\pgfqpoint{3.696000in}{3.696000in}}%
\pgfusepath{clip}%
\pgfsetbuttcap%
\pgfsetroundjoin%
\definecolor{currentfill}{rgb}{0.121569,0.466667,0.705882}%
\pgfsetfillcolor{currentfill}%
\pgfsetfillopacity{0.318226}%
\pgfsetlinewidth{1.003750pt}%
\definecolor{currentstroke}{rgb}{0.121569,0.466667,0.705882}%
\pgfsetstrokecolor{currentstroke}%
\pgfsetstrokeopacity{0.318226}%
\pgfsetdash{}{0pt}%
\pgfpathmoveto{\pgfqpoint{1.171168in}{1.602483in}}%
\pgfpathcurveto{\pgfqpoint{1.179405in}{1.602483in}}{\pgfqpoint{1.187305in}{1.605756in}}{\pgfqpoint{1.193129in}{1.611580in}}%
\pgfpathcurveto{\pgfqpoint{1.198952in}{1.617404in}}{\pgfqpoint{1.202225in}{1.625304in}}{\pgfqpoint{1.202225in}{1.633540in}}%
\pgfpathcurveto{\pgfqpoint{1.202225in}{1.641776in}}{\pgfqpoint{1.198952in}{1.649676in}}{\pgfqpoint{1.193129in}{1.655500in}}%
\pgfpathcurveto{\pgfqpoint{1.187305in}{1.661324in}}{\pgfqpoint{1.179405in}{1.664596in}}{\pgfqpoint{1.171168in}{1.664596in}}%
\pgfpathcurveto{\pgfqpoint{1.162932in}{1.664596in}}{\pgfqpoint{1.155032in}{1.661324in}}{\pgfqpoint{1.149208in}{1.655500in}}%
\pgfpathcurveto{\pgfqpoint{1.143384in}{1.649676in}}{\pgfqpoint{1.140112in}{1.641776in}}{\pgfqpoint{1.140112in}{1.633540in}}%
\pgfpathcurveto{\pgfqpoint{1.140112in}{1.625304in}}{\pgfqpoint{1.143384in}{1.617404in}}{\pgfqpoint{1.149208in}{1.611580in}}%
\pgfpathcurveto{\pgfqpoint{1.155032in}{1.605756in}}{\pgfqpoint{1.162932in}{1.602483in}}{\pgfqpoint{1.171168in}{1.602483in}}%
\pgfpathclose%
\pgfusepath{stroke,fill}%
\end{pgfscope}%
\begin{pgfscope}%
\pgfpathrectangle{\pgfqpoint{0.100000in}{0.212622in}}{\pgfqpoint{3.696000in}{3.696000in}}%
\pgfusepath{clip}%
\pgfsetbuttcap%
\pgfsetroundjoin%
\definecolor{currentfill}{rgb}{0.121569,0.466667,0.705882}%
\pgfsetfillcolor{currentfill}%
\pgfsetfillopacity{0.318731}%
\pgfsetlinewidth{1.003750pt}%
\definecolor{currentstroke}{rgb}{0.121569,0.466667,0.705882}%
\pgfsetstrokecolor{currentstroke}%
\pgfsetstrokeopacity{0.318731}%
\pgfsetdash{}{0pt}%
\pgfpathmoveto{\pgfqpoint{1.172083in}{1.602085in}}%
\pgfpathcurveto{\pgfqpoint{1.180319in}{1.602085in}}{\pgfqpoint{1.188220in}{1.605357in}}{\pgfqpoint{1.194043in}{1.611181in}}%
\pgfpathcurveto{\pgfqpoint{1.199867in}{1.617005in}}{\pgfqpoint{1.203140in}{1.624905in}}{\pgfqpoint{1.203140in}{1.633141in}}%
\pgfpathcurveto{\pgfqpoint{1.203140in}{1.641378in}}{\pgfqpoint{1.199867in}{1.649278in}}{\pgfqpoint{1.194043in}{1.655102in}}%
\pgfpathcurveto{\pgfqpoint{1.188220in}{1.660926in}}{\pgfqpoint{1.180319in}{1.664198in}}{\pgfqpoint{1.172083in}{1.664198in}}%
\pgfpathcurveto{\pgfqpoint{1.163847in}{1.664198in}}{\pgfqpoint{1.155947in}{1.660926in}}{\pgfqpoint{1.150123in}{1.655102in}}%
\pgfpathcurveto{\pgfqpoint{1.144299in}{1.649278in}}{\pgfqpoint{1.141027in}{1.641378in}}{\pgfqpoint{1.141027in}{1.633141in}}%
\pgfpathcurveto{\pgfqpoint{1.141027in}{1.624905in}}{\pgfqpoint{1.144299in}{1.617005in}}{\pgfqpoint{1.150123in}{1.611181in}}%
\pgfpathcurveto{\pgfqpoint{1.155947in}{1.605357in}}{\pgfqpoint{1.163847in}{1.602085in}}{\pgfqpoint{1.172083in}{1.602085in}}%
\pgfpathclose%
\pgfusepath{stroke,fill}%
\end{pgfscope}%
\begin{pgfscope}%
\pgfpathrectangle{\pgfqpoint{0.100000in}{0.212622in}}{\pgfqpoint{3.696000in}{3.696000in}}%
\pgfusepath{clip}%
\pgfsetbuttcap%
\pgfsetroundjoin%
\definecolor{currentfill}{rgb}{0.121569,0.466667,0.705882}%
\pgfsetfillcolor{currentfill}%
\pgfsetfillopacity{0.319107}%
\pgfsetlinewidth{1.003750pt}%
\definecolor{currentstroke}{rgb}{0.121569,0.466667,0.705882}%
\pgfsetstrokecolor{currentstroke}%
\pgfsetstrokeopacity{0.319107}%
\pgfsetdash{}{0pt}%
\pgfpathmoveto{\pgfqpoint{1.172486in}{1.601763in}}%
\pgfpathcurveto{\pgfqpoint{1.180722in}{1.601763in}}{\pgfqpoint{1.188622in}{1.605035in}}{\pgfqpoint{1.194446in}{1.610859in}}%
\pgfpathcurveto{\pgfqpoint{1.200270in}{1.616683in}}{\pgfqpoint{1.203542in}{1.624583in}}{\pgfqpoint{1.203542in}{1.632819in}}%
\pgfpathcurveto{\pgfqpoint{1.203542in}{1.641056in}}{\pgfqpoint{1.200270in}{1.648956in}}{\pgfqpoint{1.194446in}{1.654780in}}%
\pgfpathcurveto{\pgfqpoint{1.188622in}{1.660604in}}{\pgfqpoint{1.180722in}{1.663876in}}{\pgfqpoint{1.172486in}{1.663876in}}%
\pgfpathcurveto{\pgfqpoint{1.164250in}{1.663876in}}{\pgfqpoint{1.156350in}{1.660604in}}{\pgfqpoint{1.150526in}{1.654780in}}%
\pgfpathcurveto{\pgfqpoint{1.144702in}{1.648956in}}{\pgfqpoint{1.141429in}{1.641056in}}{\pgfqpoint{1.141429in}{1.632819in}}%
\pgfpathcurveto{\pgfqpoint{1.141429in}{1.624583in}}{\pgfqpoint{1.144702in}{1.616683in}}{\pgfqpoint{1.150526in}{1.610859in}}%
\pgfpathcurveto{\pgfqpoint{1.156350in}{1.605035in}}{\pgfqpoint{1.164250in}{1.601763in}}{\pgfqpoint{1.172486in}{1.601763in}}%
\pgfpathclose%
\pgfusepath{stroke,fill}%
\end{pgfscope}%
\begin{pgfscope}%
\pgfpathrectangle{\pgfqpoint{0.100000in}{0.212622in}}{\pgfqpoint{3.696000in}{3.696000in}}%
\pgfusepath{clip}%
\pgfsetbuttcap%
\pgfsetroundjoin%
\definecolor{currentfill}{rgb}{0.121569,0.466667,0.705882}%
\pgfsetfillcolor{currentfill}%
\pgfsetfillopacity{0.319273}%
\pgfsetlinewidth{1.003750pt}%
\definecolor{currentstroke}{rgb}{0.121569,0.466667,0.705882}%
\pgfsetstrokecolor{currentstroke}%
\pgfsetstrokeopacity{0.319273}%
\pgfsetdash{}{0pt}%
\pgfpathmoveto{\pgfqpoint{1.172751in}{1.601633in}}%
\pgfpathcurveto{\pgfqpoint{1.180987in}{1.601633in}}{\pgfqpoint{1.188887in}{1.604905in}}{\pgfqpoint{1.194711in}{1.610729in}}%
\pgfpathcurveto{\pgfqpoint{1.200535in}{1.616553in}}{\pgfqpoint{1.203807in}{1.624453in}}{\pgfqpoint{1.203807in}{1.632689in}}%
\pgfpathcurveto{\pgfqpoint{1.203807in}{1.640925in}}{\pgfqpoint{1.200535in}{1.648825in}}{\pgfqpoint{1.194711in}{1.654649in}}%
\pgfpathcurveto{\pgfqpoint{1.188887in}{1.660473in}}{\pgfqpoint{1.180987in}{1.663746in}}{\pgfqpoint{1.172751in}{1.663746in}}%
\pgfpathcurveto{\pgfqpoint{1.164514in}{1.663746in}}{\pgfqpoint{1.156614in}{1.660473in}}{\pgfqpoint{1.150790in}{1.654649in}}%
\pgfpathcurveto{\pgfqpoint{1.144966in}{1.648825in}}{\pgfqpoint{1.141694in}{1.640925in}}{\pgfqpoint{1.141694in}{1.632689in}}%
\pgfpathcurveto{\pgfqpoint{1.141694in}{1.624453in}}{\pgfqpoint{1.144966in}{1.616553in}}{\pgfqpoint{1.150790in}{1.610729in}}%
\pgfpathcurveto{\pgfqpoint{1.156614in}{1.604905in}}{\pgfqpoint{1.164514in}{1.601633in}}{\pgfqpoint{1.172751in}{1.601633in}}%
\pgfpathclose%
\pgfusepath{stroke,fill}%
\end{pgfscope}%
\begin{pgfscope}%
\pgfpathrectangle{\pgfqpoint{0.100000in}{0.212622in}}{\pgfqpoint{3.696000in}{3.696000in}}%
\pgfusepath{clip}%
\pgfsetbuttcap%
\pgfsetroundjoin%
\definecolor{currentfill}{rgb}{0.121569,0.466667,0.705882}%
\pgfsetfillcolor{currentfill}%
\pgfsetfillopacity{0.319702}%
\pgfsetlinewidth{1.003750pt}%
\definecolor{currentstroke}{rgb}{0.121569,0.466667,0.705882}%
\pgfsetstrokecolor{currentstroke}%
\pgfsetstrokeopacity{0.319702}%
\pgfsetdash{}{0pt}%
\pgfpathmoveto{\pgfqpoint{1.173601in}{1.601279in}}%
\pgfpathcurveto{\pgfqpoint{1.181837in}{1.601279in}}{\pgfqpoint{1.189737in}{1.604551in}}{\pgfqpoint{1.195561in}{1.610375in}}%
\pgfpathcurveto{\pgfqpoint{1.201385in}{1.616199in}}{\pgfqpoint{1.204658in}{1.624099in}}{\pgfqpoint{1.204658in}{1.632336in}}%
\pgfpathcurveto{\pgfqpoint{1.204658in}{1.640572in}}{\pgfqpoint{1.201385in}{1.648472in}}{\pgfqpoint{1.195561in}{1.654296in}}%
\pgfpathcurveto{\pgfqpoint{1.189737in}{1.660120in}}{\pgfqpoint{1.181837in}{1.663392in}}{\pgfqpoint{1.173601in}{1.663392in}}%
\pgfpathcurveto{\pgfqpoint{1.165365in}{1.663392in}}{\pgfqpoint{1.157465in}{1.660120in}}{\pgfqpoint{1.151641in}{1.654296in}}%
\pgfpathcurveto{\pgfqpoint{1.145817in}{1.648472in}}{\pgfqpoint{1.142545in}{1.640572in}}{\pgfqpoint{1.142545in}{1.632336in}}%
\pgfpathcurveto{\pgfqpoint{1.142545in}{1.624099in}}{\pgfqpoint{1.145817in}{1.616199in}}{\pgfqpoint{1.151641in}{1.610375in}}%
\pgfpathcurveto{\pgfqpoint{1.157465in}{1.604551in}}{\pgfqpoint{1.165365in}{1.601279in}}{\pgfqpoint{1.173601in}{1.601279in}}%
\pgfpathclose%
\pgfusepath{stroke,fill}%
\end{pgfscope}%
\begin{pgfscope}%
\pgfpathrectangle{\pgfqpoint{0.100000in}{0.212622in}}{\pgfqpoint{3.696000in}{3.696000in}}%
\pgfusepath{clip}%
\pgfsetbuttcap%
\pgfsetroundjoin%
\definecolor{currentfill}{rgb}{0.121569,0.466667,0.705882}%
\pgfsetfillcolor{currentfill}%
\pgfsetfillopacity{0.319986}%
\pgfsetlinewidth{1.003750pt}%
\definecolor{currentstroke}{rgb}{0.121569,0.466667,0.705882}%
\pgfsetstrokecolor{currentstroke}%
\pgfsetstrokeopacity{0.319986}%
\pgfsetdash{}{0pt}%
\pgfpathmoveto{\pgfqpoint{1.174021in}{1.601036in}}%
\pgfpathcurveto{\pgfqpoint{1.182257in}{1.601036in}}{\pgfqpoint{1.190157in}{1.604308in}}{\pgfqpoint{1.195981in}{1.610132in}}%
\pgfpathcurveto{\pgfqpoint{1.201805in}{1.615956in}}{\pgfqpoint{1.205078in}{1.623856in}}{\pgfqpoint{1.205078in}{1.632093in}}%
\pgfpathcurveto{\pgfqpoint{1.205078in}{1.640329in}}{\pgfqpoint{1.201805in}{1.648229in}}{\pgfqpoint{1.195981in}{1.654053in}}%
\pgfpathcurveto{\pgfqpoint{1.190157in}{1.659877in}}{\pgfqpoint{1.182257in}{1.663149in}}{\pgfqpoint{1.174021in}{1.663149in}}%
\pgfpathcurveto{\pgfqpoint{1.165785in}{1.663149in}}{\pgfqpoint{1.157885in}{1.659877in}}{\pgfqpoint{1.152061in}{1.654053in}}%
\pgfpathcurveto{\pgfqpoint{1.146237in}{1.648229in}}{\pgfqpoint{1.142965in}{1.640329in}}{\pgfqpoint{1.142965in}{1.632093in}}%
\pgfpathcurveto{\pgfqpoint{1.142965in}{1.623856in}}{\pgfqpoint{1.146237in}{1.615956in}}{\pgfqpoint{1.152061in}{1.610132in}}%
\pgfpathcurveto{\pgfqpoint{1.157885in}{1.604308in}}{\pgfqpoint{1.165785in}{1.601036in}}{\pgfqpoint{1.174021in}{1.601036in}}%
\pgfpathclose%
\pgfusepath{stroke,fill}%
\end{pgfscope}%
\begin{pgfscope}%
\pgfpathrectangle{\pgfqpoint{0.100000in}{0.212622in}}{\pgfqpoint{3.696000in}{3.696000in}}%
\pgfusepath{clip}%
\pgfsetbuttcap%
\pgfsetroundjoin%
\definecolor{currentfill}{rgb}{0.121569,0.466667,0.705882}%
\pgfsetfillcolor{currentfill}%
\pgfsetfillopacity{0.320124}%
\pgfsetlinewidth{1.003750pt}%
\definecolor{currentstroke}{rgb}{0.121569,0.466667,0.705882}%
\pgfsetstrokecolor{currentstroke}%
\pgfsetstrokeopacity{0.320124}%
\pgfsetdash{}{0pt}%
\pgfpathmoveto{\pgfqpoint{1.174269in}{1.600917in}}%
\pgfpathcurveto{\pgfqpoint{1.182505in}{1.600917in}}{\pgfqpoint{1.190405in}{1.604189in}}{\pgfqpoint{1.196229in}{1.610013in}}%
\pgfpathcurveto{\pgfqpoint{1.202053in}{1.615837in}}{\pgfqpoint{1.205325in}{1.623737in}}{\pgfqpoint{1.205325in}{1.631973in}}%
\pgfpathcurveto{\pgfqpoint{1.205325in}{1.640210in}}{\pgfqpoint{1.202053in}{1.648110in}}{\pgfqpoint{1.196229in}{1.653934in}}%
\pgfpathcurveto{\pgfqpoint{1.190405in}{1.659758in}}{\pgfqpoint{1.182505in}{1.663030in}}{\pgfqpoint{1.174269in}{1.663030in}}%
\pgfpathcurveto{\pgfqpoint{1.166032in}{1.663030in}}{\pgfqpoint{1.158132in}{1.659758in}}{\pgfqpoint{1.152308in}{1.653934in}}%
\pgfpathcurveto{\pgfqpoint{1.146484in}{1.648110in}}{\pgfqpoint{1.143212in}{1.640210in}}{\pgfqpoint{1.143212in}{1.631973in}}%
\pgfpathcurveto{\pgfqpoint{1.143212in}{1.623737in}}{\pgfqpoint{1.146484in}{1.615837in}}{\pgfqpoint{1.152308in}{1.610013in}}%
\pgfpathcurveto{\pgfqpoint{1.158132in}{1.604189in}}{\pgfqpoint{1.166032in}{1.600917in}}{\pgfqpoint{1.174269in}{1.600917in}}%
\pgfpathclose%
\pgfusepath{stroke,fill}%
\end{pgfscope}%
\begin{pgfscope}%
\pgfpathrectangle{\pgfqpoint{0.100000in}{0.212622in}}{\pgfqpoint{3.696000in}{3.696000in}}%
\pgfusepath{clip}%
\pgfsetbuttcap%
\pgfsetroundjoin%
\definecolor{currentfill}{rgb}{0.121569,0.466667,0.705882}%
\pgfsetfillcolor{currentfill}%
\pgfsetfillopacity{0.320500}%
\pgfsetlinewidth{1.003750pt}%
\definecolor{currentstroke}{rgb}{0.121569,0.466667,0.705882}%
\pgfsetstrokecolor{currentstroke}%
\pgfsetstrokeopacity{0.320500}%
\pgfsetdash{}{0pt}%
\pgfpathmoveto{\pgfqpoint{1.175099in}{1.600588in}}%
\pgfpathcurveto{\pgfqpoint{1.183335in}{1.600588in}}{\pgfqpoint{1.191235in}{1.603860in}}{\pgfqpoint{1.197059in}{1.609684in}}%
\pgfpathcurveto{\pgfqpoint{1.202883in}{1.615508in}}{\pgfqpoint{1.206155in}{1.623408in}}{\pgfqpoint{1.206155in}{1.631644in}}%
\pgfpathcurveto{\pgfqpoint{1.206155in}{1.639881in}}{\pgfqpoint{1.202883in}{1.647781in}}{\pgfqpoint{1.197059in}{1.653605in}}%
\pgfpathcurveto{\pgfqpoint{1.191235in}{1.659429in}}{\pgfqpoint{1.183335in}{1.662701in}}{\pgfqpoint{1.175099in}{1.662701in}}%
\pgfpathcurveto{\pgfqpoint{1.166863in}{1.662701in}}{\pgfqpoint{1.158963in}{1.659429in}}{\pgfqpoint{1.153139in}{1.653605in}}%
\pgfpathcurveto{\pgfqpoint{1.147315in}{1.647781in}}{\pgfqpoint{1.144042in}{1.639881in}}{\pgfqpoint{1.144042in}{1.631644in}}%
\pgfpathcurveto{\pgfqpoint{1.144042in}{1.623408in}}{\pgfqpoint{1.147315in}{1.615508in}}{\pgfqpoint{1.153139in}{1.609684in}}%
\pgfpathcurveto{\pgfqpoint{1.158963in}{1.603860in}}{\pgfqpoint{1.166863in}{1.600588in}}{\pgfqpoint{1.175099in}{1.600588in}}%
\pgfpathclose%
\pgfusepath{stroke,fill}%
\end{pgfscope}%
\begin{pgfscope}%
\pgfpathrectangle{\pgfqpoint{0.100000in}{0.212622in}}{\pgfqpoint{3.696000in}{3.696000in}}%
\pgfusepath{clip}%
\pgfsetbuttcap%
\pgfsetroundjoin%
\definecolor{currentfill}{rgb}{0.121569,0.466667,0.705882}%
\pgfsetfillcolor{currentfill}%
\pgfsetfillopacity{0.320713}%
\pgfsetlinewidth{1.003750pt}%
\definecolor{currentstroke}{rgb}{0.121569,0.466667,0.705882}%
\pgfsetstrokecolor{currentstroke}%
\pgfsetstrokeopacity{0.320713}%
\pgfsetdash{}{0pt}%
\pgfpathmoveto{\pgfqpoint{1.175553in}{1.600416in}}%
\pgfpathcurveto{\pgfqpoint{1.183790in}{1.600416in}}{\pgfqpoint{1.191690in}{1.603688in}}{\pgfqpoint{1.197514in}{1.609512in}}%
\pgfpathcurveto{\pgfqpoint{1.203338in}{1.615336in}}{\pgfqpoint{1.206610in}{1.623236in}}{\pgfqpoint{1.206610in}{1.631472in}}%
\pgfpathcurveto{\pgfqpoint{1.206610in}{1.639708in}}{\pgfqpoint{1.203338in}{1.647609in}}{\pgfqpoint{1.197514in}{1.653432in}}%
\pgfpathcurveto{\pgfqpoint{1.191690in}{1.659256in}}{\pgfqpoint{1.183790in}{1.662529in}}{\pgfqpoint{1.175553in}{1.662529in}}%
\pgfpathcurveto{\pgfqpoint{1.167317in}{1.662529in}}{\pgfqpoint{1.159417in}{1.659256in}}{\pgfqpoint{1.153593in}{1.653432in}}%
\pgfpathcurveto{\pgfqpoint{1.147769in}{1.647609in}}{\pgfqpoint{1.144497in}{1.639708in}}{\pgfqpoint{1.144497in}{1.631472in}}%
\pgfpathcurveto{\pgfqpoint{1.144497in}{1.623236in}}{\pgfqpoint{1.147769in}{1.615336in}}{\pgfqpoint{1.153593in}{1.609512in}}%
\pgfpathcurveto{\pgfqpoint{1.159417in}{1.603688in}}{\pgfqpoint{1.167317in}{1.600416in}}{\pgfqpoint{1.175553in}{1.600416in}}%
\pgfpathclose%
\pgfusepath{stroke,fill}%
\end{pgfscope}%
\begin{pgfscope}%
\pgfpathrectangle{\pgfqpoint{0.100000in}{0.212622in}}{\pgfqpoint{3.696000in}{3.696000in}}%
\pgfusepath{clip}%
\pgfsetbuttcap%
\pgfsetroundjoin%
\definecolor{currentfill}{rgb}{0.121569,0.466667,0.705882}%
\pgfsetfillcolor{currentfill}%
\pgfsetfillopacity{0.320857}%
\pgfsetlinewidth{1.003750pt}%
\definecolor{currentstroke}{rgb}{0.121569,0.466667,0.705882}%
\pgfsetstrokecolor{currentstroke}%
\pgfsetstrokeopacity{0.320857}%
\pgfsetdash{}{0pt}%
\pgfpathmoveto{\pgfqpoint{1.175776in}{1.600288in}}%
\pgfpathcurveto{\pgfqpoint{1.184012in}{1.600288in}}{\pgfqpoint{1.191912in}{1.603561in}}{\pgfqpoint{1.197736in}{1.609385in}}%
\pgfpathcurveto{\pgfqpoint{1.203560in}{1.615209in}}{\pgfqpoint{1.206832in}{1.623109in}}{\pgfqpoint{1.206832in}{1.631345in}}%
\pgfpathcurveto{\pgfqpoint{1.206832in}{1.639581in}}{\pgfqpoint{1.203560in}{1.647481in}}{\pgfqpoint{1.197736in}{1.653305in}}%
\pgfpathcurveto{\pgfqpoint{1.191912in}{1.659129in}}{\pgfqpoint{1.184012in}{1.662401in}}{\pgfqpoint{1.175776in}{1.662401in}}%
\pgfpathcurveto{\pgfqpoint{1.167539in}{1.662401in}}{\pgfqpoint{1.159639in}{1.659129in}}{\pgfqpoint{1.153815in}{1.653305in}}%
\pgfpathcurveto{\pgfqpoint{1.147991in}{1.647481in}}{\pgfqpoint{1.144719in}{1.639581in}}{\pgfqpoint{1.144719in}{1.631345in}}%
\pgfpathcurveto{\pgfqpoint{1.144719in}{1.623109in}}{\pgfqpoint{1.147991in}{1.615209in}}{\pgfqpoint{1.153815in}{1.609385in}}%
\pgfpathcurveto{\pgfqpoint{1.159639in}{1.603561in}}{\pgfqpoint{1.167539in}{1.600288in}}{\pgfqpoint{1.175776in}{1.600288in}}%
\pgfpathclose%
\pgfusepath{stroke,fill}%
\end{pgfscope}%
\begin{pgfscope}%
\pgfpathrectangle{\pgfqpoint{0.100000in}{0.212622in}}{\pgfqpoint{3.696000in}{3.696000in}}%
\pgfusepath{clip}%
\pgfsetbuttcap%
\pgfsetroundjoin%
\definecolor{currentfill}{rgb}{0.121569,0.466667,0.705882}%
\pgfsetfillcolor{currentfill}%
\pgfsetfillopacity{0.320916}%
\pgfsetlinewidth{1.003750pt}%
\definecolor{currentstroke}{rgb}{0.121569,0.466667,0.705882}%
\pgfsetstrokecolor{currentstroke}%
\pgfsetstrokeopacity{0.320916}%
\pgfsetdash{}{0pt}%
\pgfpathmoveto{\pgfqpoint{1.175918in}{1.600240in}}%
\pgfpathcurveto{\pgfqpoint{1.184154in}{1.600240in}}{\pgfqpoint{1.192054in}{1.603512in}}{\pgfqpoint{1.197878in}{1.609336in}}%
\pgfpathcurveto{\pgfqpoint{1.203702in}{1.615160in}}{\pgfqpoint{1.206974in}{1.623060in}}{\pgfqpoint{1.206974in}{1.631296in}}%
\pgfpathcurveto{\pgfqpoint{1.206974in}{1.639532in}}{\pgfqpoint{1.203702in}{1.647433in}}{\pgfqpoint{1.197878in}{1.653256in}}%
\pgfpathcurveto{\pgfqpoint{1.192054in}{1.659080in}}{\pgfqpoint{1.184154in}{1.662353in}}{\pgfqpoint{1.175918in}{1.662353in}}%
\pgfpathcurveto{\pgfqpoint{1.167681in}{1.662353in}}{\pgfqpoint{1.159781in}{1.659080in}}{\pgfqpoint{1.153957in}{1.653256in}}%
\pgfpathcurveto{\pgfqpoint{1.148134in}{1.647433in}}{\pgfqpoint{1.144861in}{1.639532in}}{\pgfqpoint{1.144861in}{1.631296in}}%
\pgfpathcurveto{\pgfqpoint{1.144861in}{1.623060in}}{\pgfqpoint{1.148134in}{1.615160in}}{\pgfqpoint{1.153957in}{1.609336in}}%
\pgfpathcurveto{\pgfqpoint{1.159781in}{1.603512in}}{\pgfqpoint{1.167681in}{1.600240in}}{\pgfqpoint{1.175918in}{1.600240in}}%
\pgfpathclose%
\pgfusepath{stroke,fill}%
\end{pgfscope}%
\begin{pgfscope}%
\pgfpathrectangle{\pgfqpoint{0.100000in}{0.212622in}}{\pgfqpoint{3.696000in}{3.696000in}}%
\pgfusepath{clip}%
\pgfsetbuttcap%
\pgfsetroundjoin%
\definecolor{currentfill}{rgb}{0.121569,0.466667,0.705882}%
\pgfsetfillcolor{currentfill}%
\pgfsetfillopacity{0.321762}%
\pgfsetlinewidth{1.003750pt}%
\definecolor{currentstroke}{rgb}{0.121569,0.466667,0.705882}%
\pgfsetstrokecolor{currentstroke}%
\pgfsetstrokeopacity{0.321762}%
\pgfsetdash{}{0pt}%
\pgfpathmoveto{\pgfqpoint{1.178043in}{1.599511in}}%
\pgfpathcurveto{\pgfqpoint{1.186279in}{1.599511in}}{\pgfqpoint{1.194179in}{1.602784in}}{\pgfqpoint{1.200003in}{1.608608in}}%
\pgfpathcurveto{\pgfqpoint{1.205827in}{1.614431in}}{\pgfqpoint{1.209100in}{1.622332in}}{\pgfqpoint{1.209100in}{1.630568in}}%
\pgfpathcurveto{\pgfqpoint{1.209100in}{1.638804in}}{\pgfqpoint{1.205827in}{1.646704in}}{\pgfqpoint{1.200003in}{1.652528in}}%
\pgfpathcurveto{\pgfqpoint{1.194179in}{1.658352in}}{\pgfqpoint{1.186279in}{1.661624in}}{\pgfqpoint{1.178043in}{1.661624in}}%
\pgfpathcurveto{\pgfqpoint{1.169807in}{1.661624in}}{\pgfqpoint{1.161907in}{1.658352in}}{\pgfqpoint{1.156083in}{1.652528in}}%
\pgfpathcurveto{\pgfqpoint{1.150259in}{1.646704in}}{\pgfqpoint{1.146987in}{1.638804in}}{\pgfqpoint{1.146987in}{1.630568in}}%
\pgfpathcurveto{\pgfqpoint{1.146987in}{1.622332in}}{\pgfqpoint{1.150259in}{1.614431in}}{\pgfqpoint{1.156083in}{1.608608in}}%
\pgfpathcurveto{\pgfqpoint{1.161907in}{1.602784in}}{\pgfqpoint{1.169807in}{1.599511in}}{\pgfqpoint{1.178043in}{1.599511in}}%
\pgfpathclose%
\pgfusepath{stroke,fill}%
\end{pgfscope}%
\begin{pgfscope}%
\pgfpathrectangle{\pgfqpoint{0.100000in}{0.212622in}}{\pgfqpoint{3.696000in}{3.696000in}}%
\pgfusepath{clip}%
\pgfsetbuttcap%
\pgfsetroundjoin%
\definecolor{currentfill}{rgb}{0.121569,0.466667,0.705882}%
\pgfsetfillcolor{currentfill}%
\pgfsetfillopacity{0.323173}%
\pgfsetlinewidth{1.003750pt}%
\definecolor{currentstroke}{rgb}{0.121569,0.466667,0.705882}%
\pgfsetstrokecolor{currentstroke}%
\pgfsetstrokeopacity{0.323173}%
\pgfsetdash{}{0pt}%
\pgfpathmoveto{\pgfqpoint{1.180345in}{1.598300in}}%
\pgfpathcurveto{\pgfqpoint{1.188581in}{1.598300in}}{\pgfqpoint{1.196481in}{1.601573in}}{\pgfqpoint{1.202305in}{1.607396in}}%
\pgfpathcurveto{\pgfqpoint{1.208129in}{1.613220in}}{\pgfqpoint{1.211401in}{1.621120in}}{\pgfqpoint{1.211401in}{1.629357in}}%
\pgfpathcurveto{\pgfqpoint{1.211401in}{1.637593in}}{\pgfqpoint{1.208129in}{1.645493in}}{\pgfqpoint{1.202305in}{1.651317in}}%
\pgfpathcurveto{\pgfqpoint{1.196481in}{1.657141in}}{\pgfqpoint{1.188581in}{1.660413in}}{\pgfqpoint{1.180345in}{1.660413in}}%
\pgfpathcurveto{\pgfqpoint{1.172109in}{1.660413in}}{\pgfqpoint{1.164209in}{1.657141in}}{\pgfqpoint{1.158385in}{1.651317in}}%
\pgfpathcurveto{\pgfqpoint{1.152561in}{1.645493in}}{\pgfqpoint{1.149288in}{1.637593in}}{\pgfqpoint{1.149288in}{1.629357in}}%
\pgfpathcurveto{\pgfqpoint{1.149288in}{1.621120in}}{\pgfqpoint{1.152561in}{1.613220in}}{\pgfqpoint{1.158385in}{1.607396in}}%
\pgfpathcurveto{\pgfqpoint{1.164209in}{1.601573in}}{\pgfqpoint{1.172109in}{1.598300in}}{\pgfqpoint{1.180345in}{1.598300in}}%
\pgfpathclose%
\pgfusepath{stroke,fill}%
\end{pgfscope}%
\begin{pgfscope}%
\pgfpathrectangle{\pgfqpoint{0.100000in}{0.212622in}}{\pgfqpoint{3.696000in}{3.696000in}}%
\pgfusepath{clip}%
\pgfsetbuttcap%
\pgfsetroundjoin%
\definecolor{currentfill}{rgb}{0.121569,0.466667,0.705882}%
\pgfsetfillcolor{currentfill}%
\pgfsetfillopacity{0.324332}%
\pgfsetlinewidth{1.003750pt}%
\definecolor{currentstroke}{rgb}{0.121569,0.466667,0.705882}%
\pgfsetstrokecolor{currentstroke}%
\pgfsetstrokeopacity{0.324332}%
\pgfsetdash{}{0pt}%
\pgfpathmoveto{\pgfqpoint{1.183584in}{1.597285in}}%
\pgfpathcurveto{\pgfqpoint{1.191820in}{1.597285in}}{\pgfqpoint{1.199720in}{1.600558in}}{\pgfqpoint{1.205544in}{1.606382in}}%
\pgfpathcurveto{\pgfqpoint{1.211368in}{1.612205in}}{\pgfqpoint{1.214641in}{1.620106in}}{\pgfqpoint{1.214641in}{1.628342in}}%
\pgfpathcurveto{\pgfqpoint{1.214641in}{1.636578in}}{\pgfqpoint{1.211368in}{1.644478in}}{\pgfqpoint{1.205544in}{1.650302in}}%
\pgfpathcurveto{\pgfqpoint{1.199720in}{1.656126in}}{\pgfqpoint{1.191820in}{1.659398in}}{\pgfqpoint{1.183584in}{1.659398in}}%
\pgfpathcurveto{\pgfqpoint{1.175348in}{1.659398in}}{\pgfqpoint{1.167448in}{1.656126in}}{\pgfqpoint{1.161624in}{1.650302in}}%
\pgfpathcurveto{\pgfqpoint{1.155800in}{1.644478in}}{\pgfqpoint{1.152528in}{1.636578in}}{\pgfqpoint{1.152528in}{1.628342in}}%
\pgfpathcurveto{\pgfqpoint{1.152528in}{1.620106in}}{\pgfqpoint{1.155800in}{1.612205in}}{\pgfqpoint{1.161624in}{1.606382in}}%
\pgfpathcurveto{\pgfqpoint{1.167448in}{1.600558in}}{\pgfqpoint{1.175348in}{1.597285in}}{\pgfqpoint{1.183584in}{1.597285in}}%
\pgfpathclose%
\pgfusepath{stroke,fill}%
\end{pgfscope}%
\begin{pgfscope}%
\pgfpathrectangle{\pgfqpoint{0.100000in}{0.212622in}}{\pgfqpoint{3.696000in}{3.696000in}}%
\pgfusepath{clip}%
\pgfsetbuttcap%
\pgfsetroundjoin%
\definecolor{currentfill}{rgb}{0.121569,0.466667,0.705882}%
\pgfsetfillcolor{currentfill}%
\pgfsetfillopacity{0.325042}%
\pgfsetlinewidth{1.003750pt}%
\definecolor{currentstroke}{rgb}{0.121569,0.466667,0.705882}%
\pgfsetstrokecolor{currentstroke}%
\pgfsetstrokeopacity{0.325042}%
\pgfsetdash{}{0pt}%
\pgfpathmoveto{\pgfqpoint{1.185301in}{1.596675in}}%
\pgfpathcurveto{\pgfqpoint{1.193537in}{1.596675in}}{\pgfqpoint{1.201437in}{1.599948in}}{\pgfqpoint{1.207261in}{1.605772in}}%
\pgfpathcurveto{\pgfqpoint{1.213085in}{1.611596in}}{\pgfqpoint{1.216357in}{1.619496in}}{\pgfqpoint{1.216357in}{1.627732in}}%
\pgfpathcurveto{\pgfqpoint{1.216357in}{1.635968in}}{\pgfqpoint{1.213085in}{1.643868in}}{\pgfqpoint{1.207261in}{1.649692in}}%
\pgfpathcurveto{\pgfqpoint{1.201437in}{1.655516in}}{\pgfqpoint{1.193537in}{1.658788in}}{\pgfqpoint{1.185301in}{1.658788in}}%
\pgfpathcurveto{\pgfqpoint{1.177064in}{1.658788in}}{\pgfqpoint{1.169164in}{1.655516in}}{\pgfqpoint{1.163341in}{1.649692in}}%
\pgfpathcurveto{\pgfqpoint{1.157517in}{1.643868in}}{\pgfqpoint{1.154244in}{1.635968in}}{\pgfqpoint{1.154244in}{1.627732in}}%
\pgfpathcurveto{\pgfqpoint{1.154244in}{1.619496in}}{\pgfqpoint{1.157517in}{1.611596in}}{\pgfqpoint{1.163341in}{1.605772in}}%
\pgfpathcurveto{\pgfqpoint{1.169164in}{1.599948in}}{\pgfqpoint{1.177064in}{1.596675in}}{\pgfqpoint{1.185301in}{1.596675in}}%
\pgfpathclose%
\pgfusepath{stroke,fill}%
\end{pgfscope}%
\begin{pgfscope}%
\pgfpathrectangle{\pgfqpoint{0.100000in}{0.212622in}}{\pgfqpoint{3.696000in}{3.696000in}}%
\pgfusepath{clip}%
\pgfsetbuttcap%
\pgfsetroundjoin%
\definecolor{currentfill}{rgb}{0.121569,0.466667,0.705882}%
\pgfsetfillcolor{currentfill}%
\pgfsetfillopacity{0.325986}%
\pgfsetlinewidth{1.003750pt}%
\definecolor{currentstroke}{rgb}{0.121569,0.466667,0.705882}%
\pgfsetstrokecolor{currentstroke}%
\pgfsetstrokeopacity{0.325986}%
\pgfsetdash{}{0pt}%
\pgfpathmoveto{\pgfqpoint{1.187673in}{1.595920in}}%
\pgfpathcurveto{\pgfqpoint{1.195909in}{1.595920in}}{\pgfqpoint{1.203809in}{1.599192in}}{\pgfqpoint{1.209633in}{1.605016in}}%
\pgfpathcurveto{\pgfqpoint{1.215457in}{1.610840in}}{\pgfqpoint{1.218730in}{1.618740in}}{\pgfqpoint{1.218730in}{1.626976in}}%
\pgfpathcurveto{\pgfqpoint{1.218730in}{1.635213in}}{\pgfqpoint{1.215457in}{1.643113in}}{\pgfqpoint{1.209633in}{1.648937in}}%
\pgfpathcurveto{\pgfqpoint{1.203809in}{1.654761in}}{\pgfqpoint{1.195909in}{1.658033in}}{\pgfqpoint{1.187673in}{1.658033in}}%
\pgfpathcurveto{\pgfqpoint{1.179437in}{1.658033in}}{\pgfqpoint{1.171537in}{1.654761in}}{\pgfqpoint{1.165713in}{1.648937in}}%
\pgfpathcurveto{\pgfqpoint{1.159889in}{1.643113in}}{\pgfqpoint{1.156617in}{1.635213in}}{\pgfqpoint{1.156617in}{1.626976in}}%
\pgfpathcurveto{\pgfqpoint{1.156617in}{1.618740in}}{\pgfqpoint{1.159889in}{1.610840in}}{\pgfqpoint{1.165713in}{1.605016in}}%
\pgfpathcurveto{\pgfqpoint{1.171537in}{1.599192in}}{\pgfqpoint{1.179437in}{1.595920in}}{\pgfqpoint{1.187673in}{1.595920in}}%
\pgfpathclose%
\pgfusepath{stroke,fill}%
\end{pgfscope}%
\begin{pgfscope}%
\pgfpathrectangle{\pgfqpoint{0.100000in}{0.212622in}}{\pgfqpoint{3.696000in}{3.696000in}}%
\pgfusepath{clip}%
\pgfsetbuttcap%
\pgfsetroundjoin%
\definecolor{currentfill}{rgb}{0.121569,0.466667,0.705882}%
\pgfsetfillcolor{currentfill}%
\pgfsetfillopacity{0.326618}%
\pgfsetlinewidth{1.003750pt}%
\definecolor{currentstroke}{rgb}{0.121569,0.466667,0.705882}%
\pgfsetstrokecolor{currentstroke}%
\pgfsetstrokeopacity{0.326618}%
\pgfsetdash{}{0pt}%
\pgfpathmoveto{\pgfqpoint{1.188867in}{1.595392in}}%
\pgfpathcurveto{\pgfqpoint{1.197103in}{1.595392in}}{\pgfqpoint{1.205003in}{1.598665in}}{\pgfqpoint{1.210827in}{1.604489in}}%
\pgfpathcurveto{\pgfqpoint{1.216651in}{1.610313in}}{\pgfqpoint{1.219924in}{1.618213in}}{\pgfqpoint{1.219924in}{1.626449in}}%
\pgfpathcurveto{\pgfqpoint{1.219924in}{1.634685in}}{\pgfqpoint{1.216651in}{1.642585in}}{\pgfqpoint{1.210827in}{1.648409in}}%
\pgfpathcurveto{\pgfqpoint{1.205003in}{1.654233in}}{\pgfqpoint{1.197103in}{1.657505in}}{\pgfqpoint{1.188867in}{1.657505in}}%
\pgfpathcurveto{\pgfqpoint{1.180631in}{1.657505in}}{\pgfqpoint{1.172731in}{1.654233in}}{\pgfqpoint{1.166907in}{1.648409in}}%
\pgfpathcurveto{\pgfqpoint{1.161083in}{1.642585in}}{\pgfqpoint{1.157811in}{1.634685in}}{\pgfqpoint{1.157811in}{1.626449in}}%
\pgfpathcurveto{\pgfqpoint{1.157811in}{1.618213in}}{\pgfqpoint{1.161083in}{1.610313in}}{\pgfqpoint{1.166907in}{1.604489in}}%
\pgfpathcurveto{\pgfqpoint{1.172731in}{1.598665in}}{\pgfqpoint{1.180631in}{1.595392in}}{\pgfqpoint{1.188867in}{1.595392in}}%
\pgfpathclose%
\pgfusepath{stroke,fill}%
\end{pgfscope}%
\begin{pgfscope}%
\pgfpathrectangle{\pgfqpoint{0.100000in}{0.212622in}}{\pgfqpoint{3.696000in}{3.696000in}}%
\pgfusepath{clip}%
\pgfsetbuttcap%
\pgfsetroundjoin%
\definecolor{currentfill}{rgb}{0.121569,0.466667,0.705882}%
\pgfsetfillcolor{currentfill}%
\pgfsetfillopacity{0.327503}%
\pgfsetlinewidth{1.003750pt}%
\definecolor{currentstroke}{rgb}{0.121569,0.466667,0.705882}%
\pgfsetstrokecolor{currentstroke}%
\pgfsetstrokeopacity{0.327503}%
\pgfsetdash{}{0pt}%
\pgfpathmoveto{\pgfqpoint{1.190706in}{1.594653in}}%
\pgfpathcurveto{\pgfqpoint{1.198943in}{1.594653in}}{\pgfqpoint{1.206843in}{1.597926in}}{\pgfqpoint{1.212667in}{1.603749in}}%
\pgfpathcurveto{\pgfqpoint{1.218491in}{1.609573in}}{\pgfqpoint{1.221763in}{1.617473in}}{\pgfqpoint{1.221763in}{1.625710in}}%
\pgfpathcurveto{\pgfqpoint{1.221763in}{1.633946in}}{\pgfqpoint{1.218491in}{1.641846in}}{\pgfqpoint{1.212667in}{1.647670in}}%
\pgfpathcurveto{\pgfqpoint{1.206843in}{1.653494in}}{\pgfqpoint{1.198943in}{1.656766in}}{\pgfqpoint{1.190706in}{1.656766in}}%
\pgfpathcurveto{\pgfqpoint{1.182470in}{1.656766in}}{\pgfqpoint{1.174570in}{1.653494in}}{\pgfqpoint{1.168746in}{1.647670in}}%
\pgfpathcurveto{\pgfqpoint{1.162922in}{1.641846in}}{\pgfqpoint{1.159650in}{1.633946in}}{\pgfqpoint{1.159650in}{1.625710in}}%
\pgfpathcurveto{\pgfqpoint{1.159650in}{1.617473in}}{\pgfqpoint{1.162922in}{1.609573in}}{\pgfqpoint{1.168746in}{1.603749in}}%
\pgfpathcurveto{\pgfqpoint{1.174570in}{1.597926in}}{\pgfqpoint{1.182470in}{1.594653in}}{\pgfqpoint{1.190706in}{1.594653in}}%
\pgfpathclose%
\pgfusepath{stroke,fill}%
\end{pgfscope}%
\begin{pgfscope}%
\pgfpathrectangle{\pgfqpoint{0.100000in}{0.212622in}}{\pgfqpoint{3.696000in}{3.696000in}}%
\pgfusepath{clip}%
\pgfsetbuttcap%
\pgfsetroundjoin%
\definecolor{currentfill}{rgb}{0.121569,0.466667,0.705882}%
\pgfsetfillcolor{currentfill}%
\pgfsetfillopacity{0.328604}%
\pgfsetlinewidth{1.003750pt}%
\definecolor{currentstroke}{rgb}{0.121569,0.466667,0.705882}%
\pgfsetstrokecolor{currentstroke}%
\pgfsetstrokeopacity{0.328604}%
\pgfsetdash{}{0pt}%
\pgfpathmoveto{\pgfqpoint{1.193047in}{1.593719in}}%
\pgfpathcurveto{\pgfqpoint{1.201284in}{1.593719in}}{\pgfqpoint{1.209184in}{1.596992in}}{\pgfqpoint{1.215008in}{1.602815in}}%
\pgfpathcurveto{\pgfqpoint{1.220832in}{1.608639in}}{\pgfqpoint{1.224104in}{1.616539in}}{\pgfqpoint{1.224104in}{1.624776in}}%
\pgfpathcurveto{\pgfqpoint{1.224104in}{1.633012in}}{\pgfqpoint{1.220832in}{1.640912in}}{\pgfqpoint{1.215008in}{1.646736in}}%
\pgfpathcurveto{\pgfqpoint{1.209184in}{1.652560in}}{\pgfqpoint{1.201284in}{1.655832in}}{\pgfqpoint{1.193047in}{1.655832in}}%
\pgfpathcurveto{\pgfqpoint{1.184811in}{1.655832in}}{\pgfqpoint{1.176911in}{1.652560in}}{\pgfqpoint{1.171087in}{1.646736in}}%
\pgfpathcurveto{\pgfqpoint{1.165263in}{1.640912in}}{\pgfqpoint{1.161991in}{1.633012in}}{\pgfqpoint{1.161991in}{1.624776in}}%
\pgfpathcurveto{\pgfqpoint{1.161991in}{1.616539in}}{\pgfqpoint{1.165263in}{1.608639in}}{\pgfqpoint{1.171087in}{1.602815in}}%
\pgfpathcurveto{\pgfqpoint{1.176911in}{1.596992in}}{\pgfqpoint{1.184811in}{1.593719in}}{\pgfqpoint{1.193047in}{1.593719in}}%
\pgfpathclose%
\pgfusepath{stroke,fill}%
\end{pgfscope}%
\begin{pgfscope}%
\pgfpathrectangle{\pgfqpoint{0.100000in}{0.212622in}}{\pgfqpoint{3.696000in}{3.696000in}}%
\pgfusepath{clip}%
\pgfsetbuttcap%
\pgfsetroundjoin%
\definecolor{currentfill}{rgb}{0.121569,0.466667,0.705882}%
\pgfsetfillcolor{currentfill}%
\pgfsetfillopacity{0.329354}%
\pgfsetlinewidth{1.003750pt}%
\definecolor{currentstroke}{rgb}{0.121569,0.466667,0.705882}%
\pgfsetstrokecolor{currentstroke}%
\pgfsetstrokeopacity{0.329354}%
\pgfsetdash{}{0pt}%
\pgfpathmoveto{\pgfqpoint{1.194187in}{1.593047in}}%
\pgfpathcurveto{\pgfqpoint{1.202423in}{1.593047in}}{\pgfqpoint{1.210323in}{1.596320in}}{\pgfqpoint{1.216147in}{1.602144in}}%
\pgfpathcurveto{\pgfqpoint{1.221971in}{1.607967in}}{\pgfqpoint{1.225243in}{1.615867in}}{\pgfqpoint{1.225243in}{1.624104in}}%
\pgfpathcurveto{\pgfqpoint{1.225243in}{1.632340in}}{\pgfqpoint{1.221971in}{1.640240in}}{\pgfqpoint{1.216147in}{1.646064in}}%
\pgfpathcurveto{\pgfqpoint{1.210323in}{1.651888in}}{\pgfqpoint{1.202423in}{1.655160in}}{\pgfqpoint{1.194187in}{1.655160in}}%
\pgfpathcurveto{\pgfqpoint{1.185950in}{1.655160in}}{\pgfqpoint{1.178050in}{1.651888in}}{\pgfqpoint{1.172226in}{1.646064in}}%
\pgfpathcurveto{\pgfqpoint{1.166402in}{1.640240in}}{\pgfqpoint{1.163130in}{1.632340in}}{\pgfqpoint{1.163130in}{1.624104in}}%
\pgfpathcurveto{\pgfqpoint{1.163130in}{1.615867in}}{\pgfqpoint{1.166402in}{1.607967in}}{\pgfqpoint{1.172226in}{1.602144in}}%
\pgfpathcurveto{\pgfqpoint{1.178050in}{1.596320in}}{\pgfqpoint{1.185950in}{1.593047in}}{\pgfqpoint{1.194187in}{1.593047in}}%
\pgfpathclose%
\pgfusepath{stroke,fill}%
\end{pgfscope}%
\begin{pgfscope}%
\pgfpathrectangle{\pgfqpoint{0.100000in}{0.212622in}}{\pgfqpoint{3.696000in}{3.696000in}}%
\pgfusepath{clip}%
\pgfsetbuttcap%
\pgfsetroundjoin%
\definecolor{currentfill}{rgb}{0.121569,0.466667,0.705882}%
\pgfsetfillcolor{currentfill}%
\pgfsetfillopacity{0.329665}%
\pgfsetlinewidth{1.003750pt}%
\definecolor{currentstroke}{rgb}{0.121569,0.466667,0.705882}%
\pgfsetstrokecolor{currentstroke}%
\pgfsetstrokeopacity{0.329665}%
\pgfsetdash{}{0pt}%
\pgfpathmoveto{\pgfqpoint{1.194911in}{1.592767in}}%
\pgfpathcurveto{\pgfqpoint{1.203148in}{1.592767in}}{\pgfqpoint{1.211048in}{1.596040in}}{\pgfqpoint{1.216872in}{1.601864in}}%
\pgfpathcurveto{\pgfqpoint{1.222696in}{1.607688in}}{\pgfqpoint{1.225968in}{1.615588in}}{\pgfqpoint{1.225968in}{1.623824in}}%
\pgfpathcurveto{\pgfqpoint{1.225968in}{1.632060in}}{\pgfqpoint{1.222696in}{1.639960in}}{\pgfqpoint{1.216872in}{1.645784in}}%
\pgfpathcurveto{\pgfqpoint{1.211048in}{1.651608in}}{\pgfqpoint{1.203148in}{1.654880in}}{\pgfqpoint{1.194911in}{1.654880in}}%
\pgfpathcurveto{\pgfqpoint{1.186675in}{1.654880in}}{\pgfqpoint{1.178775in}{1.651608in}}{\pgfqpoint{1.172951in}{1.645784in}}%
\pgfpathcurveto{\pgfqpoint{1.167127in}{1.639960in}}{\pgfqpoint{1.163855in}{1.632060in}}{\pgfqpoint{1.163855in}{1.623824in}}%
\pgfpathcurveto{\pgfqpoint{1.163855in}{1.615588in}}{\pgfqpoint{1.167127in}{1.607688in}}{\pgfqpoint{1.172951in}{1.601864in}}%
\pgfpathcurveto{\pgfqpoint{1.178775in}{1.596040in}}{\pgfqpoint{1.186675in}{1.592767in}}{\pgfqpoint{1.194911in}{1.592767in}}%
\pgfpathclose%
\pgfusepath{stroke,fill}%
\end{pgfscope}%
\begin{pgfscope}%
\pgfpathrectangle{\pgfqpoint{0.100000in}{0.212622in}}{\pgfqpoint{3.696000in}{3.696000in}}%
\pgfusepath{clip}%
\pgfsetbuttcap%
\pgfsetroundjoin%
\definecolor{currentfill}{rgb}{0.121569,0.466667,0.705882}%
\pgfsetfillcolor{currentfill}%
\pgfsetfillopacity{0.330335}%
\pgfsetlinewidth{1.003750pt}%
\definecolor{currentstroke}{rgb}{0.121569,0.466667,0.705882}%
\pgfsetstrokecolor{currentstroke}%
\pgfsetstrokeopacity{0.330335}%
\pgfsetdash{}{0pt}%
\pgfpathmoveto{\pgfqpoint{1.196071in}{1.592188in}}%
\pgfpathcurveto{\pgfqpoint{1.204307in}{1.592188in}}{\pgfqpoint{1.212207in}{1.595461in}}{\pgfqpoint{1.218031in}{1.601285in}}%
\pgfpathcurveto{\pgfqpoint{1.223855in}{1.607109in}}{\pgfqpoint{1.227128in}{1.615009in}}{\pgfqpoint{1.227128in}{1.623245in}}%
\pgfpathcurveto{\pgfqpoint{1.227128in}{1.631481in}}{\pgfqpoint{1.223855in}{1.639381in}}{\pgfqpoint{1.218031in}{1.645205in}}%
\pgfpathcurveto{\pgfqpoint{1.212207in}{1.651029in}}{\pgfqpoint{1.204307in}{1.654301in}}{\pgfqpoint{1.196071in}{1.654301in}}%
\pgfpathcurveto{\pgfqpoint{1.187835in}{1.654301in}}{\pgfqpoint{1.179935in}{1.651029in}}{\pgfqpoint{1.174111in}{1.645205in}}%
\pgfpathcurveto{\pgfqpoint{1.168287in}{1.639381in}}{\pgfqpoint{1.165015in}{1.631481in}}{\pgfqpoint{1.165015in}{1.623245in}}%
\pgfpathcurveto{\pgfqpoint{1.165015in}{1.615009in}}{\pgfqpoint{1.168287in}{1.607109in}}{\pgfqpoint{1.174111in}{1.601285in}}%
\pgfpathcurveto{\pgfqpoint{1.179935in}{1.595461in}}{\pgfqpoint{1.187835in}{1.592188in}}{\pgfqpoint{1.196071in}{1.592188in}}%
\pgfpathclose%
\pgfusepath{stroke,fill}%
\end{pgfscope}%
\begin{pgfscope}%
\pgfpathrectangle{\pgfqpoint{0.100000in}{0.212622in}}{\pgfqpoint{3.696000in}{3.696000in}}%
\pgfusepath{clip}%
\pgfsetbuttcap%
\pgfsetroundjoin%
\definecolor{currentfill}{rgb}{0.121569,0.466667,0.705882}%
\pgfsetfillcolor{currentfill}%
\pgfsetfillopacity{0.330825}%
\pgfsetlinewidth{1.003750pt}%
\definecolor{currentstroke}{rgb}{0.121569,0.466667,0.705882}%
\pgfsetstrokecolor{currentstroke}%
\pgfsetstrokeopacity{0.330825}%
\pgfsetdash{}{0pt}%
\pgfpathmoveto{\pgfqpoint{1.196570in}{1.591691in}}%
\pgfpathcurveto{\pgfqpoint{1.204806in}{1.591691in}}{\pgfqpoint{1.212706in}{1.594963in}}{\pgfqpoint{1.218530in}{1.600787in}}%
\pgfpathcurveto{\pgfqpoint{1.224354in}{1.606611in}}{\pgfqpoint{1.227627in}{1.614511in}}{\pgfqpoint{1.227627in}{1.622748in}}%
\pgfpathcurveto{\pgfqpoint{1.227627in}{1.630984in}}{\pgfqpoint{1.224354in}{1.638884in}}{\pgfqpoint{1.218530in}{1.644708in}}%
\pgfpathcurveto{\pgfqpoint{1.212706in}{1.650532in}}{\pgfqpoint{1.204806in}{1.653804in}}{\pgfqpoint{1.196570in}{1.653804in}}%
\pgfpathcurveto{\pgfqpoint{1.188334in}{1.653804in}}{\pgfqpoint{1.180434in}{1.650532in}}{\pgfqpoint{1.174610in}{1.644708in}}%
\pgfpathcurveto{\pgfqpoint{1.168786in}{1.638884in}}{\pgfqpoint{1.165514in}{1.630984in}}{\pgfqpoint{1.165514in}{1.622748in}}%
\pgfpathcurveto{\pgfqpoint{1.165514in}{1.614511in}}{\pgfqpoint{1.168786in}{1.606611in}}{\pgfqpoint{1.174610in}{1.600787in}}%
\pgfpathcurveto{\pgfqpoint{1.180434in}{1.594963in}}{\pgfqpoint{1.188334in}{1.591691in}}{\pgfqpoint{1.196570in}{1.591691in}}%
\pgfpathclose%
\pgfusepath{stroke,fill}%
\end{pgfscope}%
\begin{pgfscope}%
\pgfpathrectangle{\pgfqpoint{0.100000in}{0.212622in}}{\pgfqpoint{3.696000in}{3.696000in}}%
\pgfusepath{clip}%
\pgfsetbuttcap%
\pgfsetroundjoin%
\definecolor{currentfill}{rgb}{0.121569,0.466667,0.705882}%
\pgfsetfillcolor{currentfill}%
\pgfsetfillopacity{0.331016}%
\pgfsetlinewidth{1.003750pt}%
\definecolor{currentstroke}{rgb}{0.121569,0.466667,0.705882}%
\pgfsetstrokecolor{currentstroke}%
\pgfsetstrokeopacity{0.331016}%
\pgfsetdash{}{0pt}%
\pgfpathmoveto{\pgfqpoint{1.196928in}{1.591512in}}%
\pgfpathcurveto{\pgfqpoint{1.205164in}{1.591512in}}{\pgfqpoint{1.213064in}{1.594784in}}{\pgfqpoint{1.218888in}{1.600608in}}%
\pgfpathcurveto{\pgfqpoint{1.224712in}{1.606432in}}{\pgfqpoint{1.227984in}{1.614332in}}{\pgfqpoint{1.227984in}{1.622568in}}%
\pgfpathcurveto{\pgfqpoint{1.227984in}{1.630805in}}{\pgfqpoint{1.224712in}{1.638705in}}{\pgfqpoint{1.218888in}{1.644529in}}%
\pgfpathcurveto{\pgfqpoint{1.213064in}{1.650353in}}{\pgfqpoint{1.205164in}{1.653625in}}{\pgfqpoint{1.196928in}{1.653625in}}%
\pgfpathcurveto{\pgfqpoint{1.188691in}{1.653625in}}{\pgfqpoint{1.180791in}{1.650353in}}{\pgfqpoint{1.174967in}{1.644529in}}%
\pgfpathcurveto{\pgfqpoint{1.169143in}{1.638705in}}{\pgfqpoint{1.165871in}{1.630805in}}{\pgfqpoint{1.165871in}{1.622568in}}%
\pgfpathcurveto{\pgfqpoint{1.165871in}{1.614332in}}{\pgfqpoint{1.169143in}{1.606432in}}{\pgfqpoint{1.174967in}{1.600608in}}%
\pgfpathcurveto{\pgfqpoint{1.180791in}{1.594784in}}{\pgfqpoint{1.188691in}{1.591512in}}{\pgfqpoint{1.196928in}{1.591512in}}%
\pgfpathclose%
\pgfusepath{stroke,fill}%
\end{pgfscope}%
\begin{pgfscope}%
\pgfpathrectangle{\pgfqpoint{0.100000in}{0.212622in}}{\pgfqpoint{3.696000in}{3.696000in}}%
\pgfusepath{clip}%
\pgfsetbuttcap%
\pgfsetroundjoin%
\definecolor{currentfill}{rgb}{0.121569,0.466667,0.705882}%
\pgfsetfillcolor{currentfill}%
\pgfsetfillopacity{0.331756}%
\pgfsetlinewidth{1.003750pt}%
\definecolor{currentstroke}{rgb}{0.121569,0.466667,0.705882}%
\pgfsetstrokecolor{currentstroke}%
\pgfsetstrokeopacity{0.331756}%
\pgfsetdash{}{0pt}%
\pgfpathmoveto{\pgfqpoint{1.198258in}{1.590892in}}%
\pgfpathcurveto{\pgfqpoint{1.206494in}{1.590892in}}{\pgfqpoint{1.214394in}{1.594165in}}{\pgfqpoint{1.220218in}{1.599988in}}%
\pgfpathcurveto{\pgfqpoint{1.226042in}{1.605812in}}{\pgfqpoint{1.229314in}{1.613712in}}{\pgfqpoint{1.229314in}{1.621949in}}%
\pgfpathcurveto{\pgfqpoint{1.229314in}{1.630185in}}{\pgfqpoint{1.226042in}{1.638085in}}{\pgfqpoint{1.220218in}{1.643909in}}%
\pgfpathcurveto{\pgfqpoint{1.214394in}{1.649733in}}{\pgfqpoint{1.206494in}{1.653005in}}{\pgfqpoint{1.198258in}{1.653005in}}%
\pgfpathcurveto{\pgfqpoint{1.190021in}{1.653005in}}{\pgfqpoint{1.182121in}{1.649733in}}{\pgfqpoint{1.176297in}{1.643909in}}%
\pgfpathcurveto{\pgfqpoint{1.170473in}{1.638085in}}{\pgfqpoint{1.167201in}{1.630185in}}{\pgfqpoint{1.167201in}{1.621949in}}%
\pgfpathcurveto{\pgfqpoint{1.167201in}{1.613712in}}{\pgfqpoint{1.170473in}{1.605812in}}{\pgfqpoint{1.176297in}{1.599988in}}%
\pgfpathcurveto{\pgfqpoint{1.182121in}{1.594165in}}{\pgfqpoint{1.190021in}{1.590892in}}{\pgfqpoint{1.198258in}{1.590892in}}%
\pgfpathclose%
\pgfusepath{stroke,fill}%
\end{pgfscope}%
\begin{pgfscope}%
\pgfpathrectangle{\pgfqpoint{0.100000in}{0.212622in}}{\pgfqpoint{3.696000in}{3.696000in}}%
\pgfusepath{clip}%
\pgfsetbuttcap%
\pgfsetroundjoin%
\definecolor{currentfill}{rgb}{0.121569,0.466667,0.705882}%
\pgfsetfillcolor{currentfill}%
\pgfsetfillopacity{0.332109}%
\pgfsetlinewidth{1.003750pt}%
\definecolor{currentstroke}{rgb}{0.121569,0.466667,0.705882}%
\pgfsetstrokecolor{currentstroke}%
\pgfsetstrokeopacity{0.332109}%
\pgfsetdash{}{0pt}%
\pgfpathmoveto{\pgfqpoint{1.199043in}{1.590607in}}%
\pgfpathcurveto{\pgfqpoint{1.207280in}{1.590607in}}{\pgfqpoint{1.215180in}{1.593880in}}{\pgfqpoint{1.221004in}{1.599704in}}%
\pgfpathcurveto{\pgfqpoint{1.226828in}{1.605528in}}{\pgfqpoint{1.230100in}{1.613428in}}{\pgfqpoint{1.230100in}{1.621664in}}%
\pgfpathcurveto{\pgfqpoint{1.230100in}{1.629900in}}{\pgfqpoint{1.226828in}{1.637800in}}{\pgfqpoint{1.221004in}{1.643624in}}%
\pgfpathcurveto{\pgfqpoint{1.215180in}{1.649448in}}{\pgfqpoint{1.207280in}{1.652720in}}{\pgfqpoint{1.199043in}{1.652720in}}%
\pgfpathcurveto{\pgfqpoint{1.190807in}{1.652720in}}{\pgfqpoint{1.182907in}{1.649448in}}{\pgfqpoint{1.177083in}{1.643624in}}%
\pgfpathcurveto{\pgfqpoint{1.171259in}{1.637800in}}{\pgfqpoint{1.167987in}{1.629900in}}{\pgfqpoint{1.167987in}{1.621664in}}%
\pgfpathcurveto{\pgfqpoint{1.167987in}{1.613428in}}{\pgfqpoint{1.171259in}{1.605528in}}{\pgfqpoint{1.177083in}{1.599704in}}%
\pgfpathcurveto{\pgfqpoint{1.182907in}{1.593880in}}{\pgfqpoint{1.190807in}{1.590607in}}{\pgfqpoint{1.199043in}{1.590607in}}%
\pgfpathclose%
\pgfusepath{stroke,fill}%
\end{pgfscope}%
\begin{pgfscope}%
\pgfpathrectangle{\pgfqpoint{0.100000in}{0.212622in}}{\pgfqpoint{3.696000in}{3.696000in}}%
\pgfusepath{clip}%
\pgfsetbuttcap%
\pgfsetroundjoin%
\definecolor{currentfill}{rgb}{0.121569,0.466667,0.705882}%
\pgfsetfillcolor{currentfill}%
\pgfsetfillopacity{0.332321}%
\pgfsetlinewidth{1.003750pt}%
\definecolor{currentstroke}{rgb}{0.121569,0.466667,0.705882}%
\pgfsetstrokecolor{currentstroke}%
\pgfsetstrokeopacity{0.332321}%
\pgfsetdash{}{0pt}%
\pgfpathmoveto{\pgfqpoint{1.199457in}{1.590432in}}%
\pgfpathcurveto{\pgfqpoint{1.207693in}{1.590432in}}{\pgfqpoint{1.215594in}{1.593705in}}{\pgfqpoint{1.221417in}{1.599529in}}%
\pgfpathcurveto{\pgfqpoint{1.227241in}{1.605353in}}{\pgfqpoint{1.230514in}{1.613253in}}{\pgfqpoint{1.230514in}{1.621489in}}%
\pgfpathcurveto{\pgfqpoint{1.230514in}{1.629725in}}{\pgfqpoint{1.227241in}{1.637625in}}{\pgfqpoint{1.221417in}{1.643449in}}%
\pgfpathcurveto{\pgfqpoint{1.215594in}{1.649273in}}{\pgfqpoint{1.207693in}{1.652545in}}{\pgfqpoint{1.199457in}{1.652545in}}%
\pgfpathcurveto{\pgfqpoint{1.191221in}{1.652545in}}{\pgfqpoint{1.183321in}{1.649273in}}{\pgfqpoint{1.177497in}{1.643449in}}%
\pgfpathcurveto{\pgfqpoint{1.171673in}{1.637625in}}{\pgfqpoint{1.168401in}{1.629725in}}{\pgfqpoint{1.168401in}{1.621489in}}%
\pgfpathcurveto{\pgfqpoint{1.168401in}{1.613253in}}{\pgfqpoint{1.171673in}{1.605353in}}{\pgfqpoint{1.177497in}{1.599529in}}%
\pgfpathcurveto{\pgfqpoint{1.183321in}{1.593705in}}{\pgfqpoint{1.191221in}{1.590432in}}{\pgfqpoint{1.199457in}{1.590432in}}%
\pgfpathclose%
\pgfusepath{stroke,fill}%
\end{pgfscope}%
\begin{pgfscope}%
\pgfpathrectangle{\pgfqpoint{0.100000in}{0.212622in}}{\pgfqpoint{3.696000in}{3.696000in}}%
\pgfusepath{clip}%
\pgfsetbuttcap%
\pgfsetroundjoin%
\definecolor{currentfill}{rgb}{0.121569,0.466667,0.705882}%
\pgfsetfillcolor{currentfill}%
\pgfsetfillopacity{0.332761}%
\pgfsetlinewidth{1.003750pt}%
\definecolor{currentstroke}{rgb}{0.121569,0.466667,0.705882}%
\pgfsetstrokecolor{currentstroke}%
\pgfsetstrokeopacity{0.332761}%
\pgfsetdash{}{0pt}%
\pgfpathmoveto{\pgfqpoint{1.200446in}{1.590083in}}%
\pgfpathcurveto{\pgfqpoint{1.208682in}{1.590083in}}{\pgfqpoint{1.216582in}{1.593355in}}{\pgfqpoint{1.222406in}{1.599179in}}%
\pgfpathcurveto{\pgfqpoint{1.228230in}{1.605003in}}{\pgfqpoint{1.231502in}{1.612903in}}{\pgfqpoint{1.231502in}{1.621140in}}%
\pgfpathcurveto{\pgfqpoint{1.231502in}{1.629376in}}{\pgfqpoint{1.228230in}{1.637276in}}{\pgfqpoint{1.222406in}{1.643100in}}%
\pgfpathcurveto{\pgfqpoint{1.216582in}{1.648924in}}{\pgfqpoint{1.208682in}{1.652196in}}{\pgfqpoint{1.200446in}{1.652196in}}%
\pgfpathcurveto{\pgfqpoint{1.192210in}{1.652196in}}{\pgfqpoint{1.184310in}{1.648924in}}{\pgfqpoint{1.178486in}{1.643100in}}%
\pgfpathcurveto{\pgfqpoint{1.172662in}{1.637276in}}{\pgfqpoint{1.169389in}{1.629376in}}{\pgfqpoint{1.169389in}{1.621140in}}%
\pgfpathcurveto{\pgfqpoint{1.169389in}{1.612903in}}{\pgfqpoint{1.172662in}{1.605003in}}{\pgfqpoint{1.178486in}{1.599179in}}%
\pgfpathcurveto{\pgfqpoint{1.184310in}{1.593355in}}{\pgfqpoint{1.192210in}{1.590083in}}{\pgfqpoint{1.200446in}{1.590083in}}%
\pgfpathclose%
\pgfusepath{stroke,fill}%
\end{pgfscope}%
\begin{pgfscope}%
\pgfpathrectangle{\pgfqpoint{0.100000in}{0.212622in}}{\pgfqpoint{3.696000in}{3.696000in}}%
\pgfusepath{clip}%
\pgfsetbuttcap%
\pgfsetroundjoin%
\definecolor{currentfill}{rgb}{0.121569,0.466667,0.705882}%
\pgfsetfillcolor{currentfill}%
\pgfsetfillopacity{0.333023}%
\pgfsetlinewidth{1.003750pt}%
\definecolor{currentstroke}{rgb}{0.121569,0.466667,0.705882}%
\pgfsetstrokecolor{currentstroke}%
\pgfsetstrokeopacity{0.333023}%
\pgfsetdash{}{0pt}%
\pgfpathmoveto{\pgfqpoint{1.200970in}{1.589872in}}%
\pgfpathcurveto{\pgfqpoint{1.209207in}{1.589872in}}{\pgfqpoint{1.217107in}{1.593144in}}{\pgfqpoint{1.222931in}{1.598968in}}%
\pgfpathcurveto{\pgfqpoint{1.228755in}{1.604792in}}{\pgfqpoint{1.232027in}{1.612692in}}{\pgfqpoint{1.232027in}{1.620928in}}%
\pgfpathcurveto{\pgfqpoint{1.232027in}{1.629165in}}{\pgfqpoint{1.228755in}{1.637065in}}{\pgfqpoint{1.222931in}{1.642889in}}%
\pgfpathcurveto{\pgfqpoint{1.217107in}{1.648712in}}{\pgfqpoint{1.209207in}{1.651985in}}{\pgfqpoint{1.200970in}{1.651985in}}%
\pgfpathcurveto{\pgfqpoint{1.192734in}{1.651985in}}{\pgfqpoint{1.184834in}{1.648712in}}{\pgfqpoint{1.179010in}{1.642889in}}%
\pgfpathcurveto{\pgfqpoint{1.173186in}{1.637065in}}{\pgfqpoint{1.169914in}{1.629165in}}{\pgfqpoint{1.169914in}{1.620928in}}%
\pgfpathcurveto{\pgfqpoint{1.169914in}{1.612692in}}{\pgfqpoint{1.173186in}{1.604792in}}{\pgfqpoint{1.179010in}{1.598968in}}%
\pgfpathcurveto{\pgfqpoint{1.184834in}{1.593144in}}{\pgfqpoint{1.192734in}{1.589872in}}{\pgfqpoint{1.200970in}{1.589872in}}%
\pgfpathclose%
\pgfusepath{stroke,fill}%
\end{pgfscope}%
\begin{pgfscope}%
\pgfpathrectangle{\pgfqpoint{0.100000in}{0.212622in}}{\pgfqpoint{3.696000in}{3.696000in}}%
\pgfusepath{clip}%
\pgfsetbuttcap%
\pgfsetroundjoin%
\definecolor{currentfill}{rgb}{0.121569,0.466667,0.705882}%
\pgfsetfillcolor{currentfill}%
\pgfsetfillopacity{0.333538}%
\pgfsetlinewidth{1.003750pt}%
\definecolor{currentstroke}{rgb}{0.121569,0.466667,0.705882}%
\pgfsetstrokecolor{currentstroke}%
\pgfsetstrokeopacity{0.333538}%
\pgfsetdash{}{0pt}%
\pgfpathmoveto{\pgfqpoint{1.202208in}{1.589377in}}%
\pgfpathcurveto{\pgfqpoint{1.210444in}{1.589377in}}{\pgfqpoint{1.218344in}{1.592649in}}{\pgfqpoint{1.224168in}{1.598473in}}%
\pgfpathcurveto{\pgfqpoint{1.229992in}{1.604297in}}{\pgfqpoint{1.233264in}{1.612197in}}{\pgfqpoint{1.233264in}{1.620433in}}%
\pgfpathcurveto{\pgfqpoint{1.233264in}{1.628670in}}{\pgfqpoint{1.229992in}{1.636570in}}{\pgfqpoint{1.224168in}{1.642394in}}%
\pgfpathcurveto{\pgfqpoint{1.218344in}{1.648218in}}{\pgfqpoint{1.210444in}{1.651490in}}{\pgfqpoint{1.202208in}{1.651490in}}%
\pgfpathcurveto{\pgfqpoint{1.193972in}{1.651490in}}{\pgfqpoint{1.186072in}{1.648218in}}{\pgfqpoint{1.180248in}{1.642394in}}%
\pgfpathcurveto{\pgfqpoint{1.174424in}{1.636570in}}{\pgfqpoint{1.171151in}{1.628670in}}{\pgfqpoint{1.171151in}{1.620433in}}%
\pgfpathcurveto{\pgfqpoint{1.171151in}{1.612197in}}{\pgfqpoint{1.174424in}{1.604297in}}{\pgfqpoint{1.180248in}{1.598473in}}%
\pgfpathcurveto{\pgfqpoint{1.186072in}{1.592649in}}{\pgfqpoint{1.193972in}{1.589377in}}{\pgfqpoint{1.202208in}{1.589377in}}%
\pgfpathclose%
\pgfusepath{stroke,fill}%
\end{pgfscope}%
\begin{pgfscope}%
\pgfpathrectangle{\pgfqpoint{0.100000in}{0.212622in}}{\pgfqpoint{3.696000in}{3.696000in}}%
\pgfusepath{clip}%
\pgfsetbuttcap%
\pgfsetroundjoin%
\definecolor{currentfill}{rgb}{0.121569,0.466667,0.705882}%
\pgfsetfillcolor{currentfill}%
\pgfsetfillopacity{0.334394}%
\pgfsetlinewidth{1.003750pt}%
\definecolor{currentstroke}{rgb}{0.121569,0.466667,0.705882}%
\pgfsetstrokecolor{currentstroke}%
\pgfsetstrokeopacity{0.334394}%
\pgfsetdash{}{0pt}%
\pgfpathmoveto{\pgfqpoint{1.204351in}{1.588658in}}%
\pgfpathcurveto{\pgfqpoint{1.212588in}{1.588658in}}{\pgfqpoint{1.220488in}{1.591930in}}{\pgfqpoint{1.226312in}{1.597754in}}%
\pgfpathcurveto{\pgfqpoint{1.232136in}{1.603578in}}{\pgfqpoint{1.235408in}{1.611478in}}{\pgfqpoint{1.235408in}{1.619714in}}%
\pgfpathcurveto{\pgfqpoint{1.235408in}{1.627951in}}{\pgfqpoint{1.232136in}{1.635851in}}{\pgfqpoint{1.226312in}{1.641674in}}%
\pgfpathcurveto{\pgfqpoint{1.220488in}{1.647498in}}{\pgfqpoint{1.212588in}{1.650771in}}{\pgfqpoint{1.204351in}{1.650771in}}%
\pgfpathcurveto{\pgfqpoint{1.196115in}{1.650771in}}{\pgfqpoint{1.188215in}{1.647498in}}{\pgfqpoint{1.182391in}{1.641674in}}%
\pgfpathcurveto{\pgfqpoint{1.176567in}{1.635851in}}{\pgfqpoint{1.173295in}{1.627951in}}{\pgfqpoint{1.173295in}{1.619714in}}%
\pgfpathcurveto{\pgfqpoint{1.173295in}{1.611478in}}{\pgfqpoint{1.176567in}{1.603578in}}{\pgfqpoint{1.182391in}{1.597754in}}%
\pgfpathcurveto{\pgfqpoint{1.188215in}{1.591930in}}{\pgfqpoint{1.196115in}{1.588658in}}{\pgfqpoint{1.204351in}{1.588658in}}%
\pgfpathclose%
\pgfusepath{stroke,fill}%
\end{pgfscope}%
\begin{pgfscope}%
\pgfpathrectangle{\pgfqpoint{0.100000in}{0.212622in}}{\pgfqpoint{3.696000in}{3.696000in}}%
\pgfusepath{clip}%
\pgfsetbuttcap%
\pgfsetroundjoin%
\definecolor{currentfill}{rgb}{0.121569,0.466667,0.705882}%
\pgfsetfillcolor{currentfill}%
\pgfsetfillopacity{0.334960}%
\pgfsetlinewidth{1.003750pt}%
\definecolor{currentstroke}{rgb}{0.121569,0.466667,0.705882}%
\pgfsetstrokecolor{currentstroke}%
\pgfsetstrokeopacity{0.334960}%
\pgfsetdash{}{0pt}%
\pgfpathmoveto{\pgfqpoint{1.205427in}{1.588132in}}%
\pgfpathcurveto{\pgfqpoint{1.213663in}{1.588132in}}{\pgfqpoint{1.221563in}{1.591405in}}{\pgfqpoint{1.227387in}{1.597229in}}%
\pgfpathcurveto{\pgfqpoint{1.233211in}{1.603053in}}{\pgfqpoint{1.236483in}{1.610953in}}{\pgfqpoint{1.236483in}{1.619189in}}%
\pgfpathcurveto{\pgfqpoint{1.236483in}{1.627425in}}{\pgfqpoint{1.233211in}{1.635325in}}{\pgfqpoint{1.227387in}{1.641149in}}%
\pgfpathcurveto{\pgfqpoint{1.221563in}{1.646973in}}{\pgfqpoint{1.213663in}{1.650245in}}{\pgfqpoint{1.205427in}{1.650245in}}%
\pgfpathcurveto{\pgfqpoint{1.197191in}{1.650245in}}{\pgfqpoint{1.189291in}{1.646973in}}{\pgfqpoint{1.183467in}{1.641149in}}%
\pgfpathcurveto{\pgfqpoint{1.177643in}{1.635325in}}{\pgfqpoint{1.174370in}{1.627425in}}{\pgfqpoint{1.174370in}{1.619189in}}%
\pgfpathcurveto{\pgfqpoint{1.174370in}{1.610953in}}{\pgfqpoint{1.177643in}{1.603053in}}{\pgfqpoint{1.183467in}{1.597229in}}%
\pgfpathcurveto{\pgfqpoint{1.189291in}{1.591405in}}{\pgfqpoint{1.197191in}{1.588132in}}{\pgfqpoint{1.205427in}{1.588132in}}%
\pgfpathclose%
\pgfusepath{stroke,fill}%
\end{pgfscope}%
\begin{pgfscope}%
\pgfpathrectangle{\pgfqpoint{0.100000in}{0.212622in}}{\pgfqpoint{3.696000in}{3.696000in}}%
\pgfusepath{clip}%
\pgfsetbuttcap%
\pgfsetroundjoin%
\definecolor{currentfill}{rgb}{0.121569,0.466667,0.705882}%
\pgfsetfillcolor{currentfill}%
\pgfsetfillopacity{0.336290}%
\pgfsetlinewidth{1.003750pt}%
\definecolor{currentstroke}{rgb}{0.121569,0.466667,0.705882}%
\pgfsetstrokecolor{currentstroke}%
\pgfsetstrokeopacity{0.336290}%
\pgfsetdash{}{0pt}%
\pgfpathmoveto{\pgfqpoint{1.208073in}{1.586985in}}%
\pgfpathcurveto{\pgfqpoint{1.216309in}{1.586985in}}{\pgfqpoint{1.224209in}{1.590257in}}{\pgfqpoint{1.230033in}{1.596081in}}%
\pgfpathcurveto{\pgfqpoint{1.235857in}{1.601905in}}{\pgfqpoint{1.239129in}{1.609805in}}{\pgfqpoint{1.239129in}{1.618041in}}%
\pgfpathcurveto{\pgfqpoint{1.239129in}{1.626278in}}{\pgfqpoint{1.235857in}{1.634178in}}{\pgfqpoint{1.230033in}{1.640002in}}%
\pgfpathcurveto{\pgfqpoint{1.224209in}{1.645826in}}{\pgfqpoint{1.216309in}{1.649098in}}{\pgfqpoint{1.208073in}{1.649098in}}%
\pgfpathcurveto{\pgfqpoint{1.199836in}{1.649098in}}{\pgfqpoint{1.191936in}{1.645826in}}{\pgfqpoint{1.186112in}{1.640002in}}%
\pgfpathcurveto{\pgfqpoint{1.180288in}{1.634178in}}{\pgfqpoint{1.177016in}{1.626278in}}{\pgfqpoint{1.177016in}{1.618041in}}%
\pgfpathcurveto{\pgfqpoint{1.177016in}{1.609805in}}{\pgfqpoint{1.180288in}{1.601905in}}{\pgfqpoint{1.186112in}{1.596081in}}%
\pgfpathcurveto{\pgfqpoint{1.191936in}{1.590257in}}{\pgfqpoint{1.199836in}{1.586985in}}{\pgfqpoint{1.208073in}{1.586985in}}%
\pgfpathclose%
\pgfusepath{stroke,fill}%
\end{pgfscope}%
\begin{pgfscope}%
\pgfpathrectangle{\pgfqpoint{0.100000in}{0.212622in}}{\pgfqpoint{3.696000in}{3.696000in}}%
\pgfusepath{clip}%
\pgfsetbuttcap%
\pgfsetroundjoin%
\definecolor{currentfill}{rgb}{0.121569,0.466667,0.705882}%
\pgfsetfillcolor{currentfill}%
\pgfsetfillopacity{0.337867}%
\pgfsetlinewidth{1.003750pt}%
\definecolor{currentstroke}{rgb}{0.121569,0.466667,0.705882}%
\pgfsetstrokecolor{currentstroke}%
\pgfsetstrokeopacity{0.337867}%
\pgfsetdash{}{0pt}%
\pgfpathmoveto{\pgfqpoint{1.212067in}{1.585616in}}%
\pgfpathcurveto{\pgfqpoint{1.220303in}{1.585616in}}{\pgfqpoint{1.228203in}{1.588888in}}{\pgfqpoint{1.234027in}{1.594712in}}%
\pgfpathcurveto{\pgfqpoint{1.239851in}{1.600536in}}{\pgfqpoint{1.243123in}{1.608436in}}{\pgfqpoint{1.243123in}{1.616673in}}%
\pgfpathcurveto{\pgfqpoint{1.243123in}{1.624909in}}{\pgfqpoint{1.239851in}{1.632809in}}{\pgfqpoint{1.234027in}{1.638633in}}%
\pgfpathcurveto{\pgfqpoint{1.228203in}{1.644457in}}{\pgfqpoint{1.220303in}{1.647729in}}{\pgfqpoint{1.212067in}{1.647729in}}%
\pgfpathcurveto{\pgfqpoint{1.203831in}{1.647729in}}{\pgfqpoint{1.195931in}{1.644457in}}{\pgfqpoint{1.190107in}{1.638633in}}%
\pgfpathcurveto{\pgfqpoint{1.184283in}{1.632809in}}{\pgfqpoint{1.181010in}{1.624909in}}{\pgfqpoint{1.181010in}{1.616673in}}%
\pgfpathcurveto{\pgfqpoint{1.181010in}{1.608436in}}{\pgfqpoint{1.184283in}{1.600536in}}{\pgfqpoint{1.190107in}{1.594712in}}%
\pgfpathcurveto{\pgfqpoint{1.195931in}{1.588888in}}{\pgfqpoint{1.203831in}{1.585616in}}{\pgfqpoint{1.212067in}{1.585616in}}%
\pgfpathclose%
\pgfusepath{stroke,fill}%
\end{pgfscope}%
\begin{pgfscope}%
\pgfpathrectangle{\pgfqpoint{0.100000in}{0.212622in}}{\pgfqpoint{3.696000in}{3.696000in}}%
\pgfusepath{clip}%
\pgfsetbuttcap%
\pgfsetroundjoin%
\definecolor{currentfill}{rgb}{0.121569,0.466667,0.705882}%
\pgfsetfillcolor{currentfill}%
\pgfsetfillopacity{0.338710}%
\pgfsetlinewidth{1.003750pt}%
\definecolor{currentstroke}{rgb}{0.121569,0.466667,0.705882}%
\pgfsetstrokecolor{currentstroke}%
\pgfsetstrokeopacity{0.338710}%
\pgfsetdash{}{0pt}%
\pgfpathmoveto{\pgfqpoint{1.214318in}{1.584990in}}%
\pgfpathcurveto{\pgfqpoint{1.222554in}{1.584990in}}{\pgfqpoint{1.230454in}{1.588262in}}{\pgfqpoint{1.236278in}{1.594086in}}%
\pgfpathcurveto{\pgfqpoint{1.242102in}{1.599910in}}{\pgfqpoint{1.245374in}{1.607810in}}{\pgfqpoint{1.245374in}{1.616046in}}%
\pgfpathcurveto{\pgfqpoint{1.245374in}{1.624282in}}{\pgfqpoint{1.242102in}{1.632182in}}{\pgfqpoint{1.236278in}{1.638006in}}%
\pgfpathcurveto{\pgfqpoint{1.230454in}{1.643830in}}{\pgfqpoint{1.222554in}{1.647103in}}{\pgfqpoint{1.214318in}{1.647103in}}%
\pgfpathcurveto{\pgfqpoint{1.206082in}{1.647103in}}{\pgfqpoint{1.198182in}{1.643830in}}{\pgfqpoint{1.192358in}{1.638006in}}%
\pgfpathcurveto{\pgfqpoint{1.186534in}{1.632182in}}{\pgfqpoint{1.183261in}{1.624282in}}{\pgfqpoint{1.183261in}{1.616046in}}%
\pgfpathcurveto{\pgfqpoint{1.183261in}{1.607810in}}{\pgfqpoint{1.186534in}{1.599910in}}{\pgfqpoint{1.192358in}{1.594086in}}%
\pgfpathcurveto{\pgfqpoint{1.198182in}{1.588262in}}{\pgfqpoint{1.206082in}{1.584990in}}{\pgfqpoint{1.214318in}{1.584990in}}%
\pgfpathclose%
\pgfusepath{stroke,fill}%
\end{pgfscope}%
\begin{pgfscope}%
\pgfpathrectangle{\pgfqpoint{0.100000in}{0.212622in}}{\pgfqpoint{3.696000in}{3.696000in}}%
\pgfusepath{clip}%
\pgfsetbuttcap%
\pgfsetroundjoin%
\definecolor{currentfill}{rgb}{0.121569,0.466667,0.705882}%
\pgfsetfillcolor{currentfill}%
\pgfsetfillopacity{0.339212}%
\pgfsetlinewidth{1.003750pt}%
\definecolor{currentstroke}{rgb}{0.121569,0.466667,0.705882}%
\pgfsetstrokecolor{currentstroke}%
\pgfsetstrokeopacity{0.339212}%
\pgfsetdash{}{0pt}%
\pgfpathmoveto{\pgfqpoint{1.215504in}{1.584559in}}%
\pgfpathcurveto{\pgfqpoint{1.223740in}{1.584559in}}{\pgfqpoint{1.231640in}{1.587831in}}{\pgfqpoint{1.237464in}{1.593655in}}%
\pgfpathcurveto{\pgfqpoint{1.243288in}{1.599479in}}{\pgfqpoint{1.246561in}{1.607379in}}{\pgfqpoint{1.246561in}{1.615616in}}%
\pgfpathcurveto{\pgfqpoint{1.246561in}{1.623852in}}{\pgfqpoint{1.243288in}{1.631752in}}{\pgfqpoint{1.237464in}{1.637576in}}%
\pgfpathcurveto{\pgfqpoint{1.231640in}{1.643400in}}{\pgfqpoint{1.223740in}{1.646672in}}{\pgfqpoint{1.215504in}{1.646672in}}%
\pgfpathcurveto{\pgfqpoint{1.207268in}{1.646672in}}{\pgfqpoint{1.199368in}{1.643400in}}{\pgfqpoint{1.193544in}{1.637576in}}%
\pgfpathcurveto{\pgfqpoint{1.187720in}{1.631752in}}{\pgfqpoint{1.184448in}{1.623852in}}{\pgfqpoint{1.184448in}{1.615616in}}%
\pgfpathcurveto{\pgfqpoint{1.184448in}{1.607379in}}{\pgfqpoint{1.187720in}{1.599479in}}{\pgfqpoint{1.193544in}{1.593655in}}%
\pgfpathcurveto{\pgfqpoint{1.199368in}{1.587831in}}{\pgfqpoint{1.207268in}{1.584559in}}{\pgfqpoint{1.215504in}{1.584559in}}%
\pgfpathclose%
\pgfusepath{stroke,fill}%
\end{pgfscope}%
\begin{pgfscope}%
\pgfpathrectangle{\pgfqpoint{0.100000in}{0.212622in}}{\pgfqpoint{3.696000in}{3.696000in}}%
\pgfusepath{clip}%
\pgfsetbuttcap%
\pgfsetroundjoin%
\definecolor{currentfill}{rgb}{0.121569,0.466667,0.705882}%
\pgfsetfillcolor{currentfill}%
\pgfsetfillopacity{0.341093}%
\pgfsetlinewidth{1.003750pt}%
\definecolor{currentstroke}{rgb}{0.121569,0.466667,0.705882}%
\pgfsetstrokecolor{currentstroke}%
\pgfsetstrokeopacity{0.341093}%
\pgfsetdash{}{0pt}%
\pgfpathmoveto{\pgfqpoint{1.219803in}{1.583093in}}%
\pgfpathcurveto{\pgfqpoint{1.228039in}{1.583093in}}{\pgfqpoint{1.235940in}{1.586365in}}{\pgfqpoint{1.241763in}{1.592189in}}%
\pgfpathcurveto{\pgfqpoint{1.247587in}{1.598013in}}{\pgfqpoint{1.250860in}{1.605913in}}{\pgfqpoint{1.250860in}{1.614149in}}%
\pgfpathcurveto{\pgfqpoint{1.250860in}{1.622385in}}{\pgfqpoint{1.247587in}{1.630286in}}{\pgfqpoint{1.241763in}{1.636109in}}%
\pgfpathcurveto{\pgfqpoint{1.235940in}{1.641933in}}{\pgfqpoint{1.228039in}{1.645206in}}{\pgfqpoint{1.219803in}{1.645206in}}%
\pgfpathcurveto{\pgfqpoint{1.211567in}{1.645206in}}{\pgfqpoint{1.203667in}{1.641933in}}{\pgfqpoint{1.197843in}{1.636109in}}%
\pgfpathcurveto{\pgfqpoint{1.192019in}{1.630286in}}{\pgfqpoint{1.188747in}{1.622385in}}{\pgfqpoint{1.188747in}{1.614149in}}%
\pgfpathcurveto{\pgfqpoint{1.188747in}{1.605913in}}{\pgfqpoint{1.192019in}{1.598013in}}{\pgfqpoint{1.197843in}{1.592189in}}%
\pgfpathcurveto{\pgfqpoint{1.203667in}{1.586365in}}{\pgfqpoint{1.211567in}{1.583093in}}{\pgfqpoint{1.219803in}{1.583093in}}%
\pgfpathclose%
\pgfusepath{stroke,fill}%
\end{pgfscope}%
\begin{pgfscope}%
\pgfpathrectangle{\pgfqpoint{0.100000in}{0.212622in}}{\pgfqpoint{3.696000in}{3.696000in}}%
\pgfusepath{clip}%
\pgfsetbuttcap%
\pgfsetroundjoin%
\definecolor{currentfill}{rgb}{0.121569,0.466667,0.705882}%
\pgfsetfillcolor{currentfill}%
\pgfsetfillopacity{0.343636}%
\pgfsetlinewidth{1.003750pt}%
\definecolor{currentstroke}{rgb}{0.121569,0.466667,0.705882}%
\pgfsetstrokecolor{currentstroke}%
\pgfsetstrokeopacity{0.343636}%
\pgfsetdash{}{0pt}%
\pgfpathmoveto{\pgfqpoint{1.225168in}{1.580952in}}%
\pgfpathcurveto{\pgfqpoint{1.233405in}{1.580952in}}{\pgfqpoint{1.241305in}{1.584224in}}{\pgfqpoint{1.247129in}{1.590048in}}%
\pgfpathcurveto{\pgfqpoint{1.252953in}{1.595872in}}{\pgfqpoint{1.256225in}{1.603772in}}{\pgfqpoint{1.256225in}{1.612008in}}%
\pgfpathcurveto{\pgfqpoint{1.256225in}{1.620245in}}{\pgfqpoint{1.252953in}{1.628145in}}{\pgfqpoint{1.247129in}{1.633969in}}%
\pgfpathcurveto{\pgfqpoint{1.241305in}{1.639792in}}{\pgfqpoint{1.233405in}{1.643065in}}{\pgfqpoint{1.225168in}{1.643065in}}%
\pgfpathcurveto{\pgfqpoint{1.216932in}{1.643065in}}{\pgfqpoint{1.209032in}{1.639792in}}{\pgfqpoint{1.203208in}{1.633969in}}%
\pgfpathcurveto{\pgfqpoint{1.197384in}{1.628145in}}{\pgfqpoint{1.194112in}{1.620245in}}{\pgfqpoint{1.194112in}{1.612008in}}%
\pgfpathcurveto{\pgfqpoint{1.194112in}{1.603772in}}{\pgfqpoint{1.197384in}{1.595872in}}{\pgfqpoint{1.203208in}{1.590048in}}%
\pgfpathcurveto{\pgfqpoint{1.209032in}{1.584224in}}{\pgfqpoint{1.216932in}{1.580952in}}{\pgfqpoint{1.225168in}{1.580952in}}%
\pgfpathclose%
\pgfusepath{stroke,fill}%
\end{pgfscope}%
\begin{pgfscope}%
\pgfpathrectangle{\pgfqpoint{0.100000in}{0.212622in}}{\pgfqpoint{3.696000in}{3.696000in}}%
\pgfusepath{clip}%
\pgfsetbuttcap%
\pgfsetroundjoin%
\definecolor{currentfill}{rgb}{0.121569,0.466667,0.705882}%
\pgfsetfillcolor{currentfill}%
\pgfsetfillopacity{0.344950}%
\pgfsetlinewidth{1.003750pt}%
\definecolor{currentstroke}{rgb}{0.121569,0.466667,0.705882}%
\pgfsetstrokecolor{currentstroke}%
\pgfsetstrokeopacity{0.344950}%
\pgfsetdash{}{0pt}%
\pgfpathmoveto{\pgfqpoint{1.228200in}{1.579845in}}%
\pgfpathcurveto{\pgfqpoint{1.236436in}{1.579845in}}{\pgfqpoint{1.244336in}{1.583117in}}{\pgfqpoint{1.250160in}{1.588941in}}%
\pgfpathcurveto{\pgfqpoint{1.255984in}{1.594765in}}{\pgfqpoint{1.259257in}{1.602665in}}{\pgfqpoint{1.259257in}{1.610901in}}%
\pgfpathcurveto{\pgfqpoint{1.259257in}{1.619138in}}{\pgfqpoint{1.255984in}{1.627038in}}{\pgfqpoint{1.250160in}{1.632862in}}%
\pgfpathcurveto{\pgfqpoint{1.244336in}{1.638686in}}{\pgfqpoint{1.236436in}{1.641958in}}{\pgfqpoint{1.228200in}{1.641958in}}%
\pgfpathcurveto{\pgfqpoint{1.219964in}{1.641958in}}{\pgfqpoint{1.212064in}{1.638686in}}{\pgfqpoint{1.206240in}{1.632862in}}%
\pgfpathcurveto{\pgfqpoint{1.200416in}{1.627038in}}{\pgfqpoint{1.197144in}{1.619138in}}{\pgfqpoint{1.197144in}{1.610901in}}%
\pgfpathcurveto{\pgfqpoint{1.197144in}{1.602665in}}{\pgfqpoint{1.200416in}{1.594765in}}{\pgfqpoint{1.206240in}{1.588941in}}%
\pgfpathcurveto{\pgfqpoint{1.212064in}{1.583117in}}{\pgfqpoint{1.219964in}{1.579845in}}{\pgfqpoint{1.228200in}{1.579845in}}%
\pgfpathclose%
\pgfusepath{stroke,fill}%
\end{pgfscope}%
\begin{pgfscope}%
\pgfpathrectangle{\pgfqpoint{0.100000in}{0.212622in}}{\pgfqpoint{3.696000in}{3.696000in}}%
\pgfusepath{clip}%
\pgfsetbuttcap%
\pgfsetroundjoin%
\definecolor{currentfill}{rgb}{0.121569,0.466667,0.705882}%
\pgfsetfillcolor{currentfill}%
\pgfsetfillopacity{0.345655}%
\pgfsetlinewidth{1.003750pt}%
\definecolor{currentstroke}{rgb}{0.121569,0.466667,0.705882}%
\pgfsetstrokecolor{currentstroke}%
\pgfsetstrokeopacity{0.345655}%
\pgfsetdash{}{0pt}%
\pgfpathmoveto{\pgfqpoint{1.229895in}{1.579287in}}%
\pgfpathcurveto{\pgfqpoint{1.238131in}{1.579287in}}{\pgfqpoint{1.246031in}{1.582559in}}{\pgfqpoint{1.251855in}{1.588383in}}%
\pgfpathcurveto{\pgfqpoint{1.257679in}{1.594207in}}{\pgfqpoint{1.260952in}{1.602107in}}{\pgfqpoint{1.260952in}{1.610343in}}%
\pgfpathcurveto{\pgfqpoint{1.260952in}{1.618580in}}{\pgfqpoint{1.257679in}{1.626480in}}{\pgfqpoint{1.251855in}{1.632304in}}%
\pgfpathcurveto{\pgfqpoint{1.246031in}{1.638128in}}{\pgfqpoint{1.238131in}{1.641400in}}{\pgfqpoint{1.229895in}{1.641400in}}%
\pgfpathcurveto{\pgfqpoint{1.221659in}{1.641400in}}{\pgfqpoint{1.213759in}{1.638128in}}{\pgfqpoint{1.207935in}{1.632304in}}%
\pgfpathcurveto{\pgfqpoint{1.202111in}{1.626480in}}{\pgfqpoint{1.198839in}{1.618580in}}{\pgfqpoint{1.198839in}{1.610343in}}%
\pgfpathcurveto{\pgfqpoint{1.198839in}{1.602107in}}{\pgfqpoint{1.202111in}{1.594207in}}{\pgfqpoint{1.207935in}{1.588383in}}%
\pgfpathcurveto{\pgfqpoint{1.213759in}{1.582559in}}{\pgfqpoint{1.221659in}{1.579287in}}{\pgfqpoint{1.229895in}{1.579287in}}%
\pgfpathclose%
\pgfusepath{stroke,fill}%
\end{pgfscope}%
\begin{pgfscope}%
\pgfpathrectangle{\pgfqpoint{0.100000in}{0.212622in}}{\pgfqpoint{3.696000in}{3.696000in}}%
\pgfusepath{clip}%
\pgfsetbuttcap%
\pgfsetroundjoin%
\definecolor{currentfill}{rgb}{0.121569,0.466667,0.705882}%
\pgfsetfillcolor{currentfill}%
\pgfsetfillopacity{0.346814}%
\pgfsetlinewidth{1.003750pt}%
\definecolor{currentstroke}{rgb}{0.121569,0.466667,0.705882}%
\pgfsetstrokecolor{currentstroke}%
\pgfsetstrokeopacity{0.346814}%
\pgfsetdash{}{0pt}%
\pgfpathmoveto{\pgfqpoint{1.232542in}{1.578409in}}%
\pgfpathcurveto{\pgfqpoint{1.240779in}{1.578409in}}{\pgfqpoint{1.248679in}{1.581682in}}{\pgfqpoint{1.254503in}{1.587506in}}%
\pgfpathcurveto{\pgfqpoint{1.260327in}{1.593330in}}{\pgfqpoint{1.263599in}{1.601230in}}{\pgfqpoint{1.263599in}{1.609466in}}%
\pgfpathcurveto{\pgfqpoint{1.263599in}{1.617702in}}{\pgfqpoint{1.260327in}{1.625602in}}{\pgfqpoint{1.254503in}{1.631426in}}%
\pgfpathcurveto{\pgfqpoint{1.248679in}{1.637250in}}{\pgfqpoint{1.240779in}{1.640522in}}{\pgfqpoint{1.232542in}{1.640522in}}%
\pgfpathcurveto{\pgfqpoint{1.224306in}{1.640522in}}{\pgfqpoint{1.216406in}{1.637250in}}{\pgfqpoint{1.210582in}{1.631426in}}%
\pgfpathcurveto{\pgfqpoint{1.204758in}{1.625602in}}{\pgfqpoint{1.201486in}{1.617702in}}{\pgfqpoint{1.201486in}{1.609466in}}%
\pgfpathcurveto{\pgfqpoint{1.201486in}{1.601230in}}{\pgfqpoint{1.204758in}{1.593330in}}{\pgfqpoint{1.210582in}{1.587506in}}%
\pgfpathcurveto{\pgfqpoint{1.216406in}{1.581682in}}{\pgfqpoint{1.224306in}{1.578409in}}{\pgfqpoint{1.232542in}{1.578409in}}%
\pgfpathclose%
\pgfusepath{stroke,fill}%
\end{pgfscope}%
\begin{pgfscope}%
\pgfpathrectangle{\pgfqpoint{0.100000in}{0.212622in}}{\pgfqpoint{3.696000in}{3.696000in}}%
\pgfusepath{clip}%
\pgfsetbuttcap%
\pgfsetroundjoin%
\definecolor{currentfill}{rgb}{0.121569,0.466667,0.705882}%
\pgfsetfillcolor{currentfill}%
\pgfsetfillopacity{0.348246}%
\pgfsetlinewidth{1.003750pt}%
\definecolor{currentstroke}{rgb}{0.121569,0.466667,0.705882}%
\pgfsetstrokecolor{currentstroke}%
\pgfsetstrokeopacity{0.348246}%
\pgfsetdash{}{0pt}%
\pgfpathmoveto{\pgfqpoint{1.235853in}{1.577270in}}%
\pgfpathcurveto{\pgfqpoint{1.244089in}{1.577270in}}{\pgfqpoint{1.251989in}{1.580543in}}{\pgfqpoint{1.257813in}{1.586366in}}%
\pgfpathcurveto{\pgfqpoint{1.263637in}{1.592190in}}{\pgfqpoint{1.266909in}{1.600090in}}{\pgfqpoint{1.266909in}{1.608327in}}%
\pgfpathcurveto{\pgfqpoint{1.266909in}{1.616563in}}{\pgfqpoint{1.263637in}{1.624463in}}{\pgfqpoint{1.257813in}{1.630287in}}%
\pgfpathcurveto{\pgfqpoint{1.251989in}{1.636111in}}{\pgfqpoint{1.244089in}{1.639383in}}{\pgfqpoint{1.235853in}{1.639383in}}%
\pgfpathcurveto{\pgfqpoint{1.227617in}{1.639383in}}{\pgfqpoint{1.219717in}{1.636111in}}{\pgfqpoint{1.213893in}{1.630287in}}%
\pgfpathcurveto{\pgfqpoint{1.208069in}{1.624463in}}{\pgfqpoint{1.204796in}{1.616563in}}{\pgfqpoint{1.204796in}{1.608327in}}%
\pgfpathcurveto{\pgfqpoint{1.204796in}{1.600090in}}{\pgfqpoint{1.208069in}{1.592190in}}{\pgfqpoint{1.213893in}{1.586366in}}%
\pgfpathcurveto{\pgfqpoint{1.219717in}{1.580543in}}{\pgfqpoint{1.227617in}{1.577270in}}{\pgfqpoint{1.235853in}{1.577270in}}%
\pgfpathclose%
\pgfusepath{stroke,fill}%
\end{pgfscope}%
\begin{pgfscope}%
\pgfpathrectangle{\pgfqpoint{0.100000in}{0.212622in}}{\pgfqpoint{3.696000in}{3.696000in}}%
\pgfusepath{clip}%
\pgfsetbuttcap%
\pgfsetroundjoin%
\definecolor{currentfill}{rgb}{0.121569,0.466667,0.705882}%
\pgfsetfillcolor{currentfill}%
\pgfsetfillopacity{0.350559}%
\pgfsetlinewidth{1.003750pt}%
\definecolor{currentstroke}{rgb}{0.121569,0.466667,0.705882}%
\pgfsetstrokecolor{currentstroke}%
\pgfsetstrokeopacity{0.350559}%
\pgfsetdash{}{0pt}%
\pgfpathmoveto{\pgfqpoint{1.240879in}{1.575193in}}%
\pgfpathcurveto{\pgfqpoint{1.249116in}{1.575193in}}{\pgfqpoint{1.257016in}{1.578466in}}{\pgfqpoint{1.262840in}{1.584290in}}%
\pgfpathcurveto{\pgfqpoint{1.268664in}{1.590113in}}{\pgfqpoint{1.271936in}{1.598014in}}{\pgfqpoint{1.271936in}{1.606250in}}%
\pgfpathcurveto{\pgfqpoint{1.271936in}{1.614486in}}{\pgfqpoint{1.268664in}{1.622386in}}{\pgfqpoint{1.262840in}{1.628210in}}%
\pgfpathcurveto{\pgfqpoint{1.257016in}{1.634034in}}{\pgfqpoint{1.249116in}{1.637306in}}{\pgfqpoint{1.240879in}{1.637306in}}%
\pgfpathcurveto{\pgfqpoint{1.232643in}{1.637306in}}{\pgfqpoint{1.224743in}{1.634034in}}{\pgfqpoint{1.218919in}{1.628210in}}%
\pgfpathcurveto{\pgfqpoint{1.213095in}{1.622386in}}{\pgfqpoint{1.209823in}{1.614486in}}{\pgfqpoint{1.209823in}{1.606250in}}%
\pgfpathcurveto{\pgfqpoint{1.209823in}{1.598014in}}{\pgfqpoint{1.213095in}{1.590113in}}{\pgfqpoint{1.218919in}{1.584290in}}%
\pgfpathcurveto{\pgfqpoint{1.224743in}{1.578466in}}{\pgfqpoint{1.232643in}{1.575193in}}{\pgfqpoint{1.240879in}{1.575193in}}%
\pgfpathclose%
\pgfusepath{stroke,fill}%
\end{pgfscope}%
\begin{pgfscope}%
\pgfpathrectangle{\pgfqpoint{0.100000in}{0.212622in}}{\pgfqpoint{3.696000in}{3.696000in}}%
\pgfusepath{clip}%
\pgfsetbuttcap%
\pgfsetroundjoin%
\definecolor{currentfill}{rgb}{0.121569,0.466667,0.705882}%
\pgfsetfillcolor{currentfill}%
\pgfsetfillopacity{0.355723}%
\pgfsetlinewidth{1.003750pt}%
\definecolor{currentstroke}{rgb}{0.121569,0.466667,0.705882}%
\pgfsetstrokecolor{currentstroke}%
\pgfsetstrokeopacity{0.355723}%
\pgfsetdash{}{0pt}%
\pgfpathmoveto{\pgfqpoint{1.244373in}{1.570333in}}%
\pgfpathcurveto{\pgfqpoint{1.252609in}{1.570333in}}{\pgfqpoint{1.260509in}{1.573605in}}{\pgfqpoint{1.266333in}{1.579429in}}%
\pgfpathcurveto{\pgfqpoint{1.272157in}{1.585253in}}{\pgfqpoint{1.275429in}{1.593153in}}{\pgfqpoint{1.275429in}{1.601389in}}%
\pgfpathcurveto{\pgfqpoint{1.275429in}{1.609626in}}{\pgfqpoint{1.272157in}{1.617526in}}{\pgfqpoint{1.266333in}{1.623350in}}%
\pgfpathcurveto{\pgfqpoint{1.260509in}{1.629174in}}{\pgfqpoint{1.252609in}{1.632446in}}{\pgfqpoint{1.244373in}{1.632446in}}%
\pgfpathcurveto{\pgfqpoint{1.236137in}{1.632446in}}{\pgfqpoint{1.228236in}{1.629174in}}{\pgfqpoint{1.222413in}{1.623350in}}%
\pgfpathcurveto{\pgfqpoint{1.216589in}{1.617526in}}{\pgfqpoint{1.213316in}{1.609626in}}{\pgfqpoint{1.213316in}{1.601389in}}%
\pgfpathcurveto{\pgfqpoint{1.213316in}{1.593153in}}{\pgfqpoint{1.216589in}{1.585253in}}{\pgfqpoint{1.222413in}{1.579429in}}%
\pgfpathcurveto{\pgfqpoint{1.228236in}{1.573605in}}{\pgfqpoint{1.236137in}{1.570333in}}{\pgfqpoint{1.244373in}{1.570333in}}%
\pgfpathclose%
\pgfusepath{stroke,fill}%
\end{pgfscope}%
\begin{pgfscope}%
\pgfpathrectangle{\pgfqpoint{0.100000in}{0.212622in}}{\pgfqpoint{3.696000in}{3.696000in}}%
\pgfusepath{clip}%
\pgfsetbuttcap%
\pgfsetroundjoin%
\definecolor{currentfill}{rgb}{0.121569,0.466667,0.705882}%
\pgfsetfillcolor{currentfill}%
\pgfsetfillopacity{0.357744}%
\pgfsetlinewidth{1.003750pt}%
\definecolor{currentstroke}{rgb}{0.121569,0.466667,0.705882}%
\pgfsetstrokecolor{currentstroke}%
\pgfsetstrokeopacity{0.357744}%
\pgfsetdash{}{0pt}%
\pgfpathmoveto{\pgfqpoint{1.247155in}{1.568578in}}%
\pgfpathcurveto{\pgfqpoint{1.255391in}{1.568578in}}{\pgfqpoint{1.263291in}{1.571850in}}{\pgfqpoint{1.269115in}{1.577674in}}%
\pgfpathcurveto{\pgfqpoint{1.274939in}{1.583498in}}{\pgfqpoint{1.278211in}{1.591398in}}{\pgfqpoint{1.278211in}{1.599635in}}%
\pgfpathcurveto{\pgfqpoint{1.278211in}{1.607871in}}{\pgfqpoint{1.274939in}{1.615771in}}{\pgfqpoint{1.269115in}{1.621595in}}%
\pgfpathcurveto{\pgfqpoint{1.263291in}{1.627419in}}{\pgfqpoint{1.255391in}{1.630691in}}{\pgfqpoint{1.247155in}{1.630691in}}%
\pgfpathcurveto{\pgfqpoint{1.238919in}{1.630691in}}{\pgfqpoint{1.231019in}{1.627419in}}{\pgfqpoint{1.225195in}{1.621595in}}%
\pgfpathcurveto{\pgfqpoint{1.219371in}{1.615771in}}{\pgfqpoint{1.216098in}{1.607871in}}{\pgfqpoint{1.216098in}{1.599635in}}%
\pgfpathcurveto{\pgfqpoint{1.216098in}{1.591398in}}{\pgfqpoint{1.219371in}{1.583498in}}{\pgfqpoint{1.225195in}{1.577674in}}%
\pgfpathcurveto{\pgfqpoint{1.231019in}{1.571850in}}{\pgfqpoint{1.238919in}{1.568578in}}{\pgfqpoint{1.247155in}{1.568578in}}%
\pgfpathclose%
\pgfusepath{stroke,fill}%
\end{pgfscope}%
\begin{pgfscope}%
\pgfpathrectangle{\pgfqpoint{0.100000in}{0.212622in}}{\pgfqpoint{3.696000in}{3.696000in}}%
\pgfusepath{clip}%
\pgfsetbuttcap%
\pgfsetroundjoin%
\definecolor{currentfill}{rgb}{0.121569,0.466667,0.705882}%
\pgfsetfillcolor{currentfill}%
\pgfsetfillopacity{0.359822}%
\pgfsetlinewidth{1.003750pt}%
\definecolor{currentstroke}{rgb}{0.121569,0.466667,0.705882}%
\pgfsetstrokecolor{currentstroke}%
\pgfsetstrokeopacity{0.359822}%
\pgfsetdash{}{0pt}%
\pgfpathmoveto{\pgfqpoint{1.251482in}{1.566866in}}%
\pgfpathcurveto{\pgfqpoint{1.259718in}{1.566866in}}{\pgfqpoint{1.267618in}{1.570139in}}{\pgfqpoint{1.273442in}{1.575962in}}%
\pgfpathcurveto{\pgfqpoint{1.279266in}{1.581786in}}{\pgfqpoint{1.282539in}{1.589686in}}{\pgfqpoint{1.282539in}{1.597923in}}%
\pgfpathcurveto{\pgfqpoint{1.282539in}{1.606159in}}{\pgfqpoint{1.279266in}{1.614059in}}{\pgfqpoint{1.273442in}{1.619883in}}%
\pgfpathcurveto{\pgfqpoint{1.267618in}{1.625707in}}{\pgfqpoint{1.259718in}{1.628979in}}{\pgfqpoint{1.251482in}{1.628979in}}%
\pgfpathcurveto{\pgfqpoint{1.243246in}{1.628979in}}{\pgfqpoint{1.235346in}{1.625707in}}{\pgfqpoint{1.229522in}{1.619883in}}%
\pgfpathcurveto{\pgfqpoint{1.223698in}{1.614059in}}{\pgfqpoint{1.220426in}{1.606159in}}{\pgfqpoint{1.220426in}{1.597923in}}%
\pgfpathcurveto{\pgfqpoint{1.220426in}{1.589686in}}{\pgfqpoint{1.223698in}{1.581786in}}{\pgfqpoint{1.229522in}{1.575962in}}%
\pgfpathcurveto{\pgfqpoint{1.235346in}{1.570139in}}{\pgfqpoint{1.243246in}{1.566866in}}{\pgfqpoint{1.251482in}{1.566866in}}%
\pgfpathclose%
\pgfusepath{stroke,fill}%
\end{pgfscope}%
\begin{pgfscope}%
\pgfpathrectangle{\pgfqpoint{0.100000in}{0.212622in}}{\pgfqpoint{3.696000in}{3.696000in}}%
\pgfusepath{clip}%
\pgfsetbuttcap%
\pgfsetroundjoin%
\definecolor{currentfill}{rgb}{0.121569,0.466667,0.705882}%
\pgfsetfillcolor{currentfill}%
\pgfsetfillopacity{0.361125}%
\pgfsetlinewidth{1.003750pt}%
\definecolor{currentstroke}{rgb}{0.121569,0.466667,0.705882}%
\pgfsetstrokecolor{currentstroke}%
\pgfsetstrokeopacity{0.361125}%
\pgfsetdash{}{0pt}%
\pgfpathmoveto{\pgfqpoint{1.253704in}{1.565762in}}%
\pgfpathcurveto{\pgfqpoint{1.261941in}{1.565762in}}{\pgfqpoint{1.269841in}{1.569034in}}{\pgfqpoint{1.275665in}{1.574858in}}%
\pgfpathcurveto{\pgfqpoint{1.281489in}{1.580682in}}{\pgfqpoint{1.284761in}{1.588582in}}{\pgfqpoint{1.284761in}{1.596819in}}%
\pgfpathcurveto{\pgfqpoint{1.284761in}{1.605055in}}{\pgfqpoint{1.281489in}{1.612955in}}{\pgfqpoint{1.275665in}{1.618779in}}%
\pgfpathcurveto{\pgfqpoint{1.269841in}{1.624603in}}{\pgfqpoint{1.261941in}{1.627875in}}{\pgfqpoint{1.253704in}{1.627875in}}%
\pgfpathcurveto{\pgfqpoint{1.245468in}{1.627875in}}{\pgfqpoint{1.237568in}{1.624603in}}{\pgfqpoint{1.231744in}{1.618779in}}%
\pgfpathcurveto{\pgfqpoint{1.225920in}{1.612955in}}{\pgfqpoint{1.222648in}{1.605055in}}{\pgfqpoint{1.222648in}{1.596819in}}%
\pgfpathcurveto{\pgfqpoint{1.222648in}{1.588582in}}{\pgfqpoint{1.225920in}{1.580682in}}{\pgfqpoint{1.231744in}{1.574858in}}%
\pgfpathcurveto{\pgfqpoint{1.237568in}{1.569034in}}{\pgfqpoint{1.245468in}{1.565762in}}{\pgfqpoint{1.253704in}{1.565762in}}%
\pgfpathclose%
\pgfusepath{stroke,fill}%
\end{pgfscope}%
\begin{pgfscope}%
\pgfpathrectangle{\pgfqpoint{0.100000in}{0.212622in}}{\pgfqpoint{3.696000in}{3.696000in}}%
\pgfusepath{clip}%
\pgfsetbuttcap%
\pgfsetroundjoin%
\definecolor{currentfill}{rgb}{0.121569,0.466667,0.705882}%
\pgfsetfillcolor{currentfill}%
\pgfsetfillopacity{0.361995}%
\pgfsetlinewidth{1.003750pt}%
\definecolor{currentstroke}{rgb}{0.121569,0.466667,0.705882}%
\pgfsetstrokecolor{currentstroke}%
\pgfsetstrokeopacity{0.361995}%
\pgfsetdash{}{0pt}%
\pgfpathmoveto{\pgfqpoint{1.254767in}{1.564980in}}%
\pgfpathcurveto{\pgfqpoint{1.263003in}{1.564980in}}{\pgfqpoint{1.270904in}{1.568252in}}{\pgfqpoint{1.276727in}{1.574076in}}%
\pgfpathcurveto{\pgfqpoint{1.282551in}{1.579900in}}{\pgfqpoint{1.285824in}{1.587800in}}{\pgfqpoint{1.285824in}{1.596036in}}%
\pgfpathcurveto{\pgfqpoint{1.285824in}{1.604272in}}{\pgfqpoint{1.282551in}{1.612173in}}{\pgfqpoint{1.276727in}{1.617996in}}%
\pgfpathcurveto{\pgfqpoint{1.270904in}{1.623820in}}{\pgfqpoint{1.263003in}{1.627093in}}{\pgfqpoint{1.254767in}{1.627093in}}%
\pgfpathcurveto{\pgfqpoint{1.246531in}{1.627093in}}{\pgfqpoint{1.238631in}{1.623820in}}{\pgfqpoint{1.232807in}{1.617996in}}%
\pgfpathcurveto{\pgfqpoint{1.226983in}{1.612173in}}{\pgfqpoint{1.223711in}{1.604272in}}{\pgfqpoint{1.223711in}{1.596036in}}%
\pgfpathcurveto{\pgfqpoint{1.223711in}{1.587800in}}{\pgfqpoint{1.226983in}{1.579900in}}{\pgfqpoint{1.232807in}{1.574076in}}%
\pgfpathcurveto{\pgfqpoint{1.238631in}{1.568252in}}{\pgfqpoint{1.246531in}{1.564980in}}{\pgfqpoint{1.254767in}{1.564980in}}%
\pgfpathclose%
\pgfusepath{stroke,fill}%
\end{pgfscope}%
\begin{pgfscope}%
\pgfpathrectangle{\pgfqpoint{0.100000in}{0.212622in}}{\pgfqpoint{3.696000in}{3.696000in}}%
\pgfusepath{clip}%
\pgfsetbuttcap%
\pgfsetroundjoin%
\definecolor{currentfill}{rgb}{0.121569,0.466667,0.705882}%
\pgfsetfillcolor{currentfill}%
\pgfsetfillopacity{0.363283}%
\pgfsetlinewidth{1.003750pt}%
\definecolor{currentstroke}{rgb}{0.121569,0.466667,0.705882}%
\pgfsetstrokecolor{currentstroke}%
\pgfsetstrokeopacity{0.363283}%
\pgfsetdash{}{0pt}%
\pgfpathmoveto{\pgfqpoint{1.257489in}{1.563739in}}%
\pgfpathcurveto{\pgfqpoint{1.265725in}{1.563739in}}{\pgfqpoint{1.273625in}{1.567012in}}{\pgfqpoint{1.279449in}{1.572836in}}%
\pgfpathcurveto{\pgfqpoint{1.285273in}{1.578659in}}{\pgfqpoint{1.288546in}{1.586560in}}{\pgfqpoint{1.288546in}{1.594796in}}%
\pgfpathcurveto{\pgfqpoint{1.288546in}{1.603032in}}{\pgfqpoint{1.285273in}{1.610932in}}{\pgfqpoint{1.279449in}{1.616756in}}%
\pgfpathcurveto{\pgfqpoint{1.273625in}{1.622580in}}{\pgfqpoint{1.265725in}{1.625852in}}{\pgfqpoint{1.257489in}{1.625852in}}%
\pgfpathcurveto{\pgfqpoint{1.249253in}{1.625852in}}{\pgfqpoint{1.241353in}{1.622580in}}{\pgfqpoint{1.235529in}{1.616756in}}%
\pgfpathcurveto{\pgfqpoint{1.229705in}{1.610932in}}{\pgfqpoint{1.226433in}{1.603032in}}{\pgfqpoint{1.226433in}{1.594796in}}%
\pgfpathcurveto{\pgfqpoint{1.226433in}{1.586560in}}{\pgfqpoint{1.229705in}{1.578659in}}{\pgfqpoint{1.235529in}{1.572836in}}%
\pgfpathcurveto{\pgfqpoint{1.241353in}{1.567012in}}{\pgfqpoint{1.249253in}{1.563739in}}{\pgfqpoint{1.257489in}{1.563739in}}%
\pgfpathclose%
\pgfusepath{stroke,fill}%
\end{pgfscope}%
\begin{pgfscope}%
\pgfpathrectangle{\pgfqpoint{0.100000in}{0.212622in}}{\pgfqpoint{3.696000in}{3.696000in}}%
\pgfusepath{clip}%
\pgfsetbuttcap%
\pgfsetroundjoin%
\definecolor{currentfill}{rgb}{0.121569,0.466667,0.705882}%
\pgfsetfillcolor{currentfill}%
\pgfsetfillopacity{0.364230}%
\pgfsetlinewidth{1.003750pt}%
\definecolor{currentstroke}{rgb}{0.121569,0.466667,0.705882}%
\pgfsetstrokecolor{currentstroke}%
\pgfsetstrokeopacity{0.364230}%
\pgfsetdash{}{0pt}%
\pgfpathmoveto{\pgfqpoint{1.258769in}{1.562883in}}%
\pgfpathcurveto{\pgfqpoint{1.267005in}{1.562883in}}{\pgfqpoint{1.274905in}{1.566155in}}{\pgfqpoint{1.280729in}{1.571979in}}%
\pgfpathcurveto{\pgfqpoint{1.286553in}{1.577803in}}{\pgfqpoint{1.289826in}{1.585703in}}{\pgfqpoint{1.289826in}{1.593939in}}%
\pgfpathcurveto{\pgfqpoint{1.289826in}{1.602176in}}{\pgfqpoint{1.286553in}{1.610076in}}{\pgfqpoint{1.280729in}{1.615900in}}%
\pgfpathcurveto{\pgfqpoint{1.274905in}{1.621724in}}{\pgfqpoint{1.267005in}{1.624996in}}{\pgfqpoint{1.258769in}{1.624996in}}%
\pgfpathcurveto{\pgfqpoint{1.250533in}{1.624996in}}{\pgfqpoint{1.242633in}{1.621724in}}{\pgfqpoint{1.236809in}{1.615900in}}%
\pgfpathcurveto{\pgfqpoint{1.230985in}{1.610076in}}{\pgfqpoint{1.227713in}{1.602176in}}{\pgfqpoint{1.227713in}{1.593939in}}%
\pgfpathcurveto{\pgfqpoint{1.227713in}{1.585703in}}{\pgfqpoint{1.230985in}{1.577803in}}{\pgfqpoint{1.236809in}{1.571979in}}%
\pgfpathcurveto{\pgfqpoint{1.242633in}{1.566155in}}{\pgfqpoint{1.250533in}{1.562883in}}{\pgfqpoint{1.258769in}{1.562883in}}%
\pgfpathclose%
\pgfusepath{stroke,fill}%
\end{pgfscope}%
\begin{pgfscope}%
\pgfpathrectangle{\pgfqpoint{0.100000in}{0.212622in}}{\pgfqpoint{3.696000in}{3.696000in}}%
\pgfusepath{clip}%
\pgfsetbuttcap%
\pgfsetroundjoin%
\definecolor{currentfill}{rgb}{0.121569,0.466667,0.705882}%
\pgfsetfillcolor{currentfill}%
\pgfsetfillopacity{0.365011}%
\pgfsetlinewidth{1.003750pt}%
\definecolor{currentstroke}{rgb}{0.121569,0.466667,0.705882}%
\pgfsetstrokecolor{currentstroke}%
\pgfsetstrokeopacity{0.365011}%
\pgfsetdash{}{0pt}%
\pgfpathmoveto{\pgfqpoint{1.261013in}{1.562235in}}%
\pgfpathcurveto{\pgfqpoint{1.269249in}{1.562235in}}{\pgfqpoint{1.277149in}{1.565508in}}{\pgfqpoint{1.282973in}{1.571331in}}%
\pgfpathcurveto{\pgfqpoint{1.288797in}{1.577155in}}{\pgfqpoint{1.292069in}{1.585055in}}{\pgfqpoint{1.292069in}{1.593292in}}%
\pgfpathcurveto{\pgfqpoint{1.292069in}{1.601528in}}{\pgfqpoint{1.288797in}{1.609428in}}{\pgfqpoint{1.282973in}{1.615252in}}%
\pgfpathcurveto{\pgfqpoint{1.277149in}{1.621076in}}{\pgfqpoint{1.269249in}{1.624348in}}{\pgfqpoint{1.261013in}{1.624348in}}%
\pgfpathcurveto{\pgfqpoint{1.252776in}{1.624348in}}{\pgfqpoint{1.244876in}{1.621076in}}{\pgfqpoint{1.239052in}{1.615252in}}%
\pgfpathcurveto{\pgfqpoint{1.233228in}{1.609428in}}{\pgfqpoint{1.229956in}{1.601528in}}{\pgfqpoint{1.229956in}{1.593292in}}%
\pgfpathcurveto{\pgfqpoint{1.229956in}{1.585055in}}{\pgfqpoint{1.233228in}{1.577155in}}{\pgfqpoint{1.239052in}{1.571331in}}%
\pgfpathcurveto{\pgfqpoint{1.244876in}{1.565508in}}{\pgfqpoint{1.252776in}{1.562235in}}{\pgfqpoint{1.261013in}{1.562235in}}%
\pgfpathclose%
\pgfusepath{stroke,fill}%
\end{pgfscope}%
\begin{pgfscope}%
\pgfpathrectangle{\pgfqpoint{0.100000in}{0.212622in}}{\pgfqpoint{3.696000in}{3.696000in}}%
\pgfusepath{clip}%
\pgfsetbuttcap%
\pgfsetroundjoin%
\definecolor{currentfill}{rgb}{0.121569,0.466667,0.705882}%
\pgfsetfillcolor{currentfill}%
\pgfsetfillopacity{0.366299}%
\pgfsetlinewidth{1.003750pt}%
\definecolor{currentstroke}{rgb}{0.121569,0.466667,0.705882}%
\pgfsetstrokecolor{currentstroke}%
\pgfsetstrokeopacity{0.366299}%
\pgfsetdash{}{0pt}%
\pgfpathmoveto{\pgfqpoint{1.264385in}{1.561060in}}%
\pgfpathcurveto{\pgfqpoint{1.272621in}{1.561060in}}{\pgfqpoint{1.280521in}{1.564333in}}{\pgfqpoint{1.286345in}{1.570157in}}%
\pgfpathcurveto{\pgfqpoint{1.292169in}{1.575981in}}{\pgfqpoint{1.295442in}{1.583881in}}{\pgfqpoint{1.295442in}{1.592117in}}%
\pgfpathcurveto{\pgfqpoint{1.295442in}{1.600353in}}{\pgfqpoint{1.292169in}{1.608253in}}{\pgfqpoint{1.286345in}{1.614077in}}%
\pgfpathcurveto{\pgfqpoint{1.280521in}{1.619901in}}{\pgfqpoint{1.272621in}{1.623173in}}{\pgfqpoint{1.264385in}{1.623173in}}%
\pgfpathcurveto{\pgfqpoint{1.256149in}{1.623173in}}{\pgfqpoint{1.248249in}{1.619901in}}{\pgfqpoint{1.242425in}{1.614077in}}%
\pgfpathcurveto{\pgfqpoint{1.236601in}{1.608253in}}{\pgfqpoint{1.233329in}{1.600353in}}{\pgfqpoint{1.233329in}{1.592117in}}%
\pgfpathcurveto{\pgfqpoint{1.233329in}{1.583881in}}{\pgfqpoint{1.236601in}{1.575981in}}{\pgfqpoint{1.242425in}{1.570157in}}%
\pgfpathcurveto{\pgfqpoint{1.248249in}{1.564333in}}{\pgfqpoint{1.256149in}{1.561060in}}{\pgfqpoint{1.264385in}{1.561060in}}%
\pgfpathclose%
\pgfusepath{stroke,fill}%
\end{pgfscope}%
\begin{pgfscope}%
\pgfpathrectangle{\pgfqpoint{0.100000in}{0.212622in}}{\pgfqpoint{3.696000in}{3.696000in}}%
\pgfusepath{clip}%
\pgfsetbuttcap%
\pgfsetroundjoin%
\definecolor{currentfill}{rgb}{0.121569,0.466667,0.705882}%
\pgfsetfillcolor{currentfill}%
\pgfsetfillopacity{0.367058}%
\pgfsetlinewidth{1.003750pt}%
\definecolor{currentstroke}{rgb}{0.121569,0.466667,0.705882}%
\pgfsetstrokecolor{currentstroke}%
\pgfsetstrokeopacity{0.367058}%
\pgfsetdash{}{0pt}%
\pgfpathmoveto{\pgfqpoint{1.266192in}{1.560366in}}%
\pgfpathcurveto{\pgfqpoint{1.274428in}{1.560366in}}{\pgfqpoint{1.282328in}{1.563638in}}{\pgfqpoint{1.288152in}{1.569462in}}%
\pgfpathcurveto{\pgfqpoint{1.293976in}{1.575286in}}{\pgfqpoint{1.297248in}{1.583186in}}{\pgfqpoint{1.297248in}{1.591423in}}%
\pgfpathcurveto{\pgfqpoint{1.297248in}{1.599659in}}{\pgfqpoint{1.293976in}{1.607559in}}{\pgfqpoint{1.288152in}{1.613383in}}%
\pgfpathcurveto{\pgfqpoint{1.282328in}{1.619207in}}{\pgfqpoint{1.274428in}{1.622479in}}{\pgfqpoint{1.266192in}{1.622479in}}%
\pgfpathcurveto{\pgfqpoint{1.257956in}{1.622479in}}{\pgfqpoint{1.250055in}{1.619207in}}{\pgfqpoint{1.244232in}{1.613383in}}%
\pgfpathcurveto{\pgfqpoint{1.238408in}{1.607559in}}{\pgfqpoint{1.235135in}{1.599659in}}{\pgfqpoint{1.235135in}{1.591423in}}%
\pgfpathcurveto{\pgfqpoint{1.235135in}{1.583186in}}{\pgfqpoint{1.238408in}{1.575286in}}{\pgfqpoint{1.244232in}{1.569462in}}%
\pgfpathcurveto{\pgfqpoint{1.250055in}{1.563638in}}{\pgfqpoint{1.257956in}{1.560366in}}{\pgfqpoint{1.266192in}{1.560366in}}%
\pgfpathclose%
\pgfusepath{stroke,fill}%
\end{pgfscope}%
\begin{pgfscope}%
\pgfpathrectangle{\pgfqpoint{0.100000in}{0.212622in}}{\pgfqpoint{3.696000in}{3.696000in}}%
\pgfusepath{clip}%
\pgfsetbuttcap%
\pgfsetroundjoin%
\definecolor{currentfill}{rgb}{0.121569,0.466667,0.705882}%
\pgfsetfillcolor{currentfill}%
\pgfsetfillopacity{0.368398}%
\pgfsetlinewidth{1.003750pt}%
\definecolor{currentstroke}{rgb}{0.121569,0.466667,0.705882}%
\pgfsetstrokecolor{currentstroke}%
\pgfsetstrokeopacity{0.368398}%
\pgfsetdash{}{0pt}%
\pgfpathmoveto{\pgfqpoint{1.268886in}{1.559139in}}%
\pgfpathcurveto{\pgfqpoint{1.277122in}{1.559139in}}{\pgfqpoint{1.285022in}{1.562412in}}{\pgfqpoint{1.290846in}{1.568235in}}%
\pgfpathcurveto{\pgfqpoint{1.296670in}{1.574059in}}{\pgfqpoint{1.299942in}{1.581959in}}{\pgfqpoint{1.299942in}{1.590196in}}%
\pgfpathcurveto{\pgfqpoint{1.299942in}{1.598432in}}{\pgfqpoint{1.296670in}{1.606332in}}{\pgfqpoint{1.290846in}{1.612156in}}%
\pgfpathcurveto{\pgfqpoint{1.285022in}{1.617980in}}{\pgfqpoint{1.277122in}{1.621252in}}{\pgfqpoint{1.268886in}{1.621252in}}%
\pgfpathcurveto{\pgfqpoint{1.260649in}{1.621252in}}{\pgfqpoint{1.252749in}{1.617980in}}{\pgfqpoint{1.246925in}{1.612156in}}%
\pgfpathcurveto{\pgfqpoint{1.241101in}{1.606332in}}{\pgfqpoint{1.237829in}{1.598432in}}{\pgfqpoint{1.237829in}{1.590196in}}%
\pgfpathcurveto{\pgfqpoint{1.237829in}{1.581959in}}{\pgfqpoint{1.241101in}{1.574059in}}{\pgfqpoint{1.246925in}{1.568235in}}%
\pgfpathcurveto{\pgfqpoint{1.252749in}{1.562412in}}{\pgfqpoint{1.260649in}{1.559139in}}{\pgfqpoint{1.268886in}{1.559139in}}%
\pgfpathclose%
\pgfusepath{stroke,fill}%
\end{pgfscope}%
\begin{pgfscope}%
\pgfpathrectangle{\pgfqpoint{0.100000in}{0.212622in}}{\pgfqpoint{3.696000in}{3.696000in}}%
\pgfusepath{clip}%
\pgfsetbuttcap%
\pgfsetroundjoin%
\definecolor{currentfill}{rgb}{0.121569,0.466667,0.705882}%
\pgfsetfillcolor{currentfill}%
\pgfsetfillopacity{0.369761}%
\pgfsetlinewidth{1.003750pt}%
\definecolor{currentstroke}{rgb}{0.121569,0.466667,0.705882}%
\pgfsetstrokecolor{currentstroke}%
\pgfsetstrokeopacity{0.369761}%
\pgfsetdash{}{0pt}%
\pgfpathmoveto{\pgfqpoint{1.272956in}{1.557817in}}%
\pgfpathcurveto{\pgfqpoint{1.281193in}{1.557817in}}{\pgfqpoint{1.289093in}{1.561089in}}{\pgfqpoint{1.294917in}{1.566913in}}%
\pgfpathcurveto{\pgfqpoint{1.300740in}{1.572737in}}{\pgfqpoint{1.304013in}{1.580637in}}{\pgfqpoint{1.304013in}{1.588873in}}%
\pgfpathcurveto{\pgfqpoint{1.304013in}{1.597110in}}{\pgfqpoint{1.300740in}{1.605010in}}{\pgfqpoint{1.294917in}{1.610834in}}%
\pgfpathcurveto{\pgfqpoint{1.289093in}{1.616658in}}{\pgfqpoint{1.281193in}{1.619930in}}{\pgfqpoint{1.272956in}{1.619930in}}%
\pgfpathcurveto{\pgfqpoint{1.264720in}{1.619930in}}{\pgfqpoint{1.256820in}{1.616658in}}{\pgfqpoint{1.250996in}{1.610834in}}%
\pgfpathcurveto{\pgfqpoint{1.245172in}{1.605010in}}{\pgfqpoint{1.241900in}{1.597110in}}{\pgfqpoint{1.241900in}{1.588873in}}%
\pgfpathcurveto{\pgfqpoint{1.241900in}{1.580637in}}{\pgfqpoint{1.245172in}{1.572737in}}{\pgfqpoint{1.250996in}{1.566913in}}%
\pgfpathcurveto{\pgfqpoint{1.256820in}{1.561089in}}{\pgfqpoint{1.264720in}{1.557817in}}{\pgfqpoint{1.272956in}{1.557817in}}%
\pgfpathclose%
\pgfusepath{stroke,fill}%
\end{pgfscope}%
\begin{pgfscope}%
\pgfpathrectangle{\pgfqpoint{0.100000in}{0.212622in}}{\pgfqpoint{3.696000in}{3.696000in}}%
\pgfusepath{clip}%
\pgfsetbuttcap%
\pgfsetroundjoin%
\definecolor{currentfill}{rgb}{0.121569,0.466667,0.705882}%
\pgfsetfillcolor{currentfill}%
\pgfsetfillopacity{0.370732}%
\pgfsetlinewidth{1.003750pt}%
\definecolor{currentstroke}{rgb}{0.121569,0.466667,0.705882}%
\pgfsetstrokecolor{currentstroke}%
\pgfsetstrokeopacity{0.370732}%
\pgfsetdash{}{0pt}%
\pgfpathmoveto{\pgfqpoint{1.275005in}{1.556956in}}%
\pgfpathcurveto{\pgfqpoint{1.283241in}{1.556956in}}{\pgfqpoint{1.291141in}{1.560228in}}{\pgfqpoint{1.296965in}{1.566052in}}%
\pgfpathcurveto{\pgfqpoint{1.302789in}{1.571876in}}{\pgfqpoint{1.306061in}{1.579776in}}{\pgfqpoint{1.306061in}{1.588012in}}%
\pgfpathcurveto{\pgfqpoint{1.306061in}{1.596249in}}{\pgfqpoint{1.302789in}{1.604149in}}{\pgfqpoint{1.296965in}{1.609973in}}%
\pgfpathcurveto{\pgfqpoint{1.291141in}{1.615797in}}{\pgfqpoint{1.283241in}{1.619069in}}{\pgfqpoint{1.275005in}{1.619069in}}%
\pgfpathcurveto{\pgfqpoint{1.266768in}{1.619069in}}{\pgfqpoint{1.258868in}{1.615797in}}{\pgfqpoint{1.253044in}{1.609973in}}%
\pgfpathcurveto{\pgfqpoint{1.247221in}{1.604149in}}{\pgfqpoint{1.243948in}{1.596249in}}{\pgfqpoint{1.243948in}{1.588012in}}%
\pgfpathcurveto{\pgfqpoint{1.243948in}{1.579776in}}{\pgfqpoint{1.247221in}{1.571876in}}{\pgfqpoint{1.253044in}{1.566052in}}%
\pgfpathcurveto{\pgfqpoint{1.258868in}{1.560228in}}{\pgfqpoint{1.266768in}{1.556956in}}{\pgfqpoint{1.275005in}{1.556956in}}%
\pgfpathclose%
\pgfusepath{stroke,fill}%
\end{pgfscope}%
\begin{pgfscope}%
\pgfpathrectangle{\pgfqpoint{0.100000in}{0.212622in}}{\pgfqpoint{3.696000in}{3.696000in}}%
\pgfusepath{clip}%
\pgfsetbuttcap%
\pgfsetroundjoin%
\definecolor{currentfill}{rgb}{0.121569,0.466667,0.705882}%
\pgfsetfillcolor{currentfill}%
\pgfsetfillopacity{0.372133}%
\pgfsetlinewidth{1.003750pt}%
\definecolor{currentstroke}{rgb}{0.121569,0.466667,0.705882}%
\pgfsetstrokecolor{currentstroke}%
\pgfsetstrokeopacity{0.372133}%
\pgfsetdash{}{0pt}%
\pgfpathmoveto{\pgfqpoint{1.278385in}{1.555818in}}%
\pgfpathcurveto{\pgfqpoint{1.286622in}{1.555818in}}{\pgfqpoint{1.294522in}{1.559091in}}{\pgfqpoint{1.300346in}{1.564914in}}%
\pgfpathcurveto{\pgfqpoint{1.306170in}{1.570738in}}{\pgfqpoint{1.309442in}{1.578638in}}{\pgfqpoint{1.309442in}{1.586875in}}%
\pgfpathcurveto{\pgfqpoint{1.309442in}{1.595111in}}{\pgfqpoint{1.306170in}{1.603011in}}{\pgfqpoint{1.300346in}{1.608835in}}%
\pgfpathcurveto{\pgfqpoint{1.294522in}{1.614659in}}{\pgfqpoint{1.286622in}{1.617931in}}{\pgfqpoint{1.278385in}{1.617931in}}%
\pgfpathcurveto{\pgfqpoint{1.270149in}{1.617931in}}{\pgfqpoint{1.262249in}{1.614659in}}{\pgfqpoint{1.256425in}{1.608835in}}%
\pgfpathcurveto{\pgfqpoint{1.250601in}{1.603011in}}{\pgfqpoint{1.247329in}{1.595111in}}{\pgfqpoint{1.247329in}{1.586875in}}%
\pgfpathcurveto{\pgfqpoint{1.247329in}{1.578638in}}{\pgfqpoint{1.250601in}{1.570738in}}{\pgfqpoint{1.256425in}{1.564914in}}%
\pgfpathcurveto{\pgfqpoint{1.262249in}{1.559091in}}{\pgfqpoint{1.270149in}{1.555818in}}{\pgfqpoint{1.278385in}{1.555818in}}%
\pgfpathclose%
\pgfusepath{stroke,fill}%
\end{pgfscope}%
\begin{pgfscope}%
\pgfpathrectangle{\pgfqpoint{0.100000in}{0.212622in}}{\pgfqpoint{3.696000in}{3.696000in}}%
\pgfusepath{clip}%
\pgfsetbuttcap%
\pgfsetroundjoin%
\definecolor{currentfill}{rgb}{0.121569,0.466667,0.705882}%
\pgfsetfillcolor{currentfill}%
\pgfsetfillopacity{0.373562}%
\pgfsetlinewidth{1.003750pt}%
\definecolor{currentstroke}{rgb}{0.121569,0.466667,0.705882}%
\pgfsetstrokecolor{currentstroke}%
\pgfsetstrokeopacity{0.373562}%
\pgfsetdash{}{0pt}%
\pgfpathmoveto{\pgfqpoint{1.283043in}{1.554507in}}%
\pgfpathcurveto{\pgfqpoint{1.291280in}{1.554507in}}{\pgfqpoint{1.299180in}{1.557779in}}{\pgfqpoint{1.305004in}{1.563603in}}%
\pgfpathcurveto{\pgfqpoint{1.310828in}{1.569427in}}{\pgfqpoint{1.314100in}{1.577327in}}{\pgfqpoint{1.314100in}{1.585564in}}%
\pgfpathcurveto{\pgfqpoint{1.314100in}{1.593800in}}{\pgfqpoint{1.310828in}{1.601700in}}{\pgfqpoint{1.305004in}{1.607524in}}%
\pgfpathcurveto{\pgfqpoint{1.299180in}{1.613348in}}{\pgfqpoint{1.291280in}{1.616620in}}{\pgfqpoint{1.283043in}{1.616620in}}%
\pgfpathcurveto{\pgfqpoint{1.274807in}{1.616620in}}{\pgfqpoint{1.266907in}{1.613348in}}{\pgfqpoint{1.261083in}{1.607524in}}%
\pgfpathcurveto{\pgfqpoint{1.255259in}{1.601700in}}{\pgfqpoint{1.251987in}{1.593800in}}{\pgfqpoint{1.251987in}{1.585564in}}%
\pgfpathcurveto{\pgfqpoint{1.251987in}{1.577327in}}{\pgfqpoint{1.255259in}{1.569427in}}{\pgfqpoint{1.261083in}{1.563603in}}%
\pgfpathcurveto{\pgfqpoint{1.266907in}{1.557779in}}{\pgfqpoint{1.274807in}{1.554507in}}{\pgfqpoint{1.283043in}{1.554507in}}%
\pgfpathclose%
\pgfusepath{stroke,fill}%
\end{pgfscope}%
\begin{pgfscope}%
\pgfpathrectangle{\pgfqpoint{0.100000in}{0.212622in}}{\pgfqpoint{3.696000in}{3.696000in}}%
\pgfusepath{clip}%
\pgfsetbuttcap%
\pgfsetroundjoin%
\definecolor{currentfill}{rgb}{0.121569,0.466667,0.705882}%
\pgfsetfillcolor{currentfill}%
\pgfsetfillopacity{0.375741}%
\pgfsetlinewidth{1.003750pt}%
\definecolor{currentstroke}{rgb}{0.121569,0.466667,0.705882}%
\pgfsetstrokecolor{currentstroke}%
\pgfsetstrokeopacity{0.375741}%
\pgfsetdash{}{0pt}%
\pgfpathmoveto{\pgfqpoint{1.288022in}{1.552655in}}%
\pgfpathcurveto{\pgfqpoint{1.296259in}{1.552655in}}{\pgfqpoint{1.304159in}{1.555927in}}{\pgfqpoint{1.309983in}{1.561751in}}%
\pgfpathcurveto{\pgfqpoint{1.315807in}{1.567575in}}{\pgfqpoint{1.319079in}{1.575475in}}{\pgfqpoint{1.319079in}{1.583711in}}%
\pgfpathcurveto{\pgfqpoint{1.319079in}{1.591947in}}{\pgfqpoint{1.315807in}{1.599848in}}{\pgfqpoint{1.309983in}{1.605671in}}%
\pgfpathcurveto{\pgfqpoint{1.304159in}{1.611495in}}{\pgfqpoint{1.296259in}{1.614768in}}{\pgfqpoint{1.288022in}{1.614768in}}%
\pgfpathcurveto{\pgfqpoint{1.279786in}{1.614768in}}{\pgfqpoint{1.271886in}{1.611495in}}{\pgfqpoint{1.266062in}{1.605671in}}%
\pgfpathcurveto{\pgfqpoint{1.260238in}{1.599848in}}{\pgfqpoint{1.256966in}{1.591947in}}{\pgfqpoint{1.256966in}{1.583711in}}%
\pgfpathcurveto{\pgfqpoint{1.256966in}{1.575475in}}{\pgfqpoint{1.260238in}{1.567575in}}{\pgfqpoint{1.266062in}{1.561751in}}%
\pgfpathcurveto{\pgfqpoint{1.271886in}{1.555927in}}{\pgfqpoint{1.279786in}{1.552655in}}{\pgfqpoint{1.288022in}{1.552655in}}%
\pgfpathclose%
\pgfusepath{stroke,fill}%
\end{pgfscope}%
\begin{pgfscope}%
\pgfpathrectangle{\pgfqpoint{0.100000in}{0.212622in}}{\pgfqpoint{3.696000in}{3.696000in}}%
\pgfusepath{clip}%
\pgfsetbuttcap%
\pgfsetroundjoin%
\definecolor{currentfill}{rgb}{0.121569,0.466667,0.705882}%
\pgfsetfillcolor{currentfill}%
\pgfsetfillopacity{0.378941}%
\pgfsetlinewidth{1.003750pt}%
\definecolor{currentstroke}{rgb}{0.121569,0.466667,0.705882}%
\pgfsetstrokecolor{currentstroke}%
\pgfsetstrokeopacity{0.378941}%
\pgfsetdash{}{0pt}%
\pgfpathmoveto{\pgfqpoint{1.293777in}{1.549701in}}%
\pgfpathcurveto{\pgfqpoint{1.302014in}{1.549701in}}{\pgfqpoint{1.309914in}{1.552974in}}{\pgfqpoint{1.315738in}{1.558797in}}%
\pgfpathcurveto{\pgfqpoint{1.321562in}{1.564621in}}{\pgfqpoint{1.324834in}{1.572521in}}{\pgfqpoint{1.324834in}{1.580758in}}%
\pgfpathcurveto{\pgfqpoint{1.324834in}{1.588994in}}{\pgfqpoint{1.321562in}{1.596894in}}{\pgfqpoint{1.315738in}{1.602718in}}%
\pgfpathcurveto{\pgfqpoint{1.309914in}{1.608542in}}{\pgfqpoint{1.302014in}{1.611814in}}{\pgfqpoint{1.293777in}{1.611814in}}%
\pgfpathcurveto{\pgfqpoint{1.285541in}{1.611814in}}{\pgfqpoint{1.277641in}{1.608542in}}{\pgfqpoint{1.271817in}{1.602718in}}%
\pgfpathcurveto{\pgfqpoint{1.265993in}{1.596894in}}{\pgfqpoint{1.262721in}{1.588994in}}{\pgfqpoint{1.262721in}{1.580758in}}%
\pgfpathcurveto{\pgfqpoint{1.262721in}{1.572521in}}{\pgfqpoint{1.265993in}{1.564621in}}{\pgfqpoint{1.271817in}{1.558797in}}%
\pgfpathcurveto{\pgfqpoint{1.277641in}{1.552974in}}{\pgfqpoint{1.285541in}{1.549701in}}{\pgfqpoint{1.293777in}{1.549701in}}%
\pgfpathclose%
\pgfusepath{stroke,fill}%
\end{pgfscope}%
\begin{pgfscope}%
\pgfpathrectangle{\pgfqpoint{0.100000in}{0.212622in}}{\pgfqpoint{3.696000in}{3.696000in}}%
\pgfusepath{clip}%
\pgfsetbuttcap%
\pgfsetroundjoin%
\definecolor{currentfill}{rgb}{0.121569,0.466667,0.705882}%
\pgfsetfillcolor{currentfill}%
\pgfsetfillopacity{0.381746}%
\pgfsetlinewidth{1.003750pt}%
\definecolor{currentstroke}{rgb}{0.121569,0.466667,0.705882}%
\pgfsetstrokecolor{currentstroke}%
\pgfsetstrokeopacity{0.381746}%
\pgfsetdash{}{0pt}%
\pgfpathmoveto{\pgfqpoint{1.300640in}{1.547121in}}%
\pgfpathcurveto{\pgfqpoint{1.308876in}{1.547121in}}{\pgfqpoint{1.316776in}{1.550393in}}{\pgfqpoint{1.322600in}{1.556217in}}%
\pgfpathcurveto{\pgfqpoint{1.328424in}{1.562041in}}{\pgfqpoint{1.331696in}{1.569941in}}{\pgfqpoint{1.331696in}{1.578177in}}%
\pgfpathcurveto{\pgfqpoint{1.331696in}{1.586414in}}{\pgfqpoint{1.328424in}{1.594314in}}{\pgfqpoint{1.322600in}{1.600138in}}%
\pgfpathcurveto{\pgfqpoint{1.316776in}{1.605961in}}{\pgfqpoint{1.308876in}{1.609234in}}{\pgfqpoint{1.300640in}{1.609234in}}%
\pgfpathcurveto{\pgfqpoint{1.292404in}{1.609234in}}{\pgfqpoint{1.284504in}{1.605961in}}{\pgfqpoint{1.278680in}{1.600138in}}%
\pgfpathcurveto{\pgfqpoint{1.272856in}{1.594314in}}{\pgfqpoint{1.269583in}{1.586414in}}{\pgfqpoint{1.269583in}{1.578177in}}%
\pgfpathcurveto{\pgfqpoint{1.269583in}{1.569941in}}{\pgfqpoint{1.272856in}{1.562041in}}{\pgfqpoint{1.278680in}{1.556217in}}%
\pgfpathcurveto{\pgfqpoint{1.284504in}{1.550393in}}{\pgfqpoint{1.292404in}{1.547121in}}{\pgfqpoint{1.300640in}{1.547121in}}%
\pgfpathclose%
\pgfusepath{stroke,fill}%
\end{pgfscope}%
\begin{pgfscope}%
\pgfpathrectangle{\pgfqpoint{0.100000in}{0.212622in}}{\pgfqpoint{3.696000in}{3.696000in}}%
\pgfusepath{clip}%
\pgfsetbuttcap%
\pgfsetroundjoin%
\definecolor{currentfill}{rgb}{0.121569,0.466667,0.705882}%
\pgfsetfillcolor{currentfill}%
\pgfsetfillopacity{0.384846}%
\pgfsetlinewidth{1.003750pt}%
\definecolor{currentstroke}{rgb}{0.121569,0.466667,0.705882}%
\pgfsetstrokecolor{currentstroke}%
\pgfsetstrokeopacity{0.384846}%
\pgfsetdash{}{0pt}%
\pgfpathmoveto{\pgfqpoint{1.308108in}{1.544439in}}%
\pgfpathcurveto{\pgfqpoint{1.316345in}{1.544439in}}{\pgfqpoint{1.324245in}{1.547712in}}{\pgfqpoint{1.330069in}{1.553535in}}%
\pgfpathcurveto{\pgfqpoint{1.335893in}{1.559359in}}{\pgfqpoint{1.339165in}{1.567259in}}{\pgfqpoint{1.339165in}{1.575496in}}%
\pgfpathcurveto{\pgfqpoint{1.339165in}{1.583732in}}{\pgfqpoint{1.335893in}{1.591632in}}{\pgfqpoint{1.330069in}{1.597456in}}%
\pgfpathcurveto{\pgfqpoint{1.324245in}{1.603280in}}{\pgfqpoint{1.316345in}{1.606552in}}{\pgfqpoint{1.308108in}{1.606552in}}%
\pgfpathcurveto{\pgfqpoint{1.299872in}{1.606552in}}{\pgfqpoint{1.291972in}{1.603280in}}{\pgfqpoint{1.286148in}{1.597456in}}%
\pgfpathcurveto{\pgfqpoint{1.280324in}{1.591632in}}{\pgfqpoint{1.277052in}{1.583732in}}{\pgfqpoint{1.277052in}{1.575496in}}%
\pgfpathcurveto{\pgfqpoint{1.277052in}{1.567259in}}{\pgfqpoint{1.280324in}{1.559359in}}{\pgfqpoint{1.286148in}{1.553535in}}%
\pgfpathcurveto{\pgfqpoint{1.291972in}{1.547712in}}{\pgfqpoint{1.299872in}{1.544439in}}{\pgfqpoint{1.308108in}{1.544439in}}%
\pgfpathclose%
\pgfusepath{stroke,fill}%
\end{pgfscope}%
\begin{pgfscope}%
\pgfpathrectangle{\pgfqpoint{0.100000in}{0.212622in}}{\pgfqpoint{3.696000in}{3.696000in}}%
\pgfusepath{clip}%
\pgfsetbuttcap%
\pgfsetroundjoin%
\definecolor{currentfill}{rgb}{0.121569,0.466667,0.705882}%
\pgfsetfillcolor{currentfill}%
\pgfsetfillopacity{0.388683}%
\pgfsetlinewidth{1.003750pt}%
\definecolor{currentstroke}{rgb}{0.121569,0.466667,0.705882}%
\pgfsetstrokecolor{currentstroke}%
\pgfsetstrokeopacity{0.388683}%
\pgfsetdash{}{0pt}%
\pgfpathmoveto{\pgfqpoint{1.317786in}{1.541239in}}%
\pgfpathcurveto{\pgfqpoint{1.326023in}{1.541239in}}{\pgfqpoint{1.333923in}{1.544511in}}{\pgfqpoint{1.339747in}{1.550335in}}%
\pgfpathcurveto{\pgfqpoint{1.345571in}{1.556159in}}{\pgfqpoint{1.348843in}{1.564059in}}{\pgfqpoint{1.348843in}{1.572295in}}%
\pgfpathcurveto{\pgfqpoint{1.348843in}{1.580532in}}{\pgfqpoint{1.345571in}{1.588432in}}{\pgfqpoint{1.339747in}{1.594256in}}%
\pgfpathcurveto{\pgfqpoint{1.333923in}{1.600079in}}{\pgfqpoint{1.326023in}{1.603352in}}{\pgfqpoint{1.317786in}{1.603352in}}%
\pgfpathcurveto{\pgfqpoint{1.309550in}{1.603352in}}{\pgfqpoint{1.301650in}{1.600079in}}{\pgfqpoint{1.295826in}{1.594256in}}%
\pgfpathcurveto{\pgfqpoint{1.290002in}{1.588432in}}{\pgfqpoint{1.286730in}{1.580532in}}{\pgfqpoint{1.286730in}{1.572295in}}%
\pgfpathcurveto{\pgfqpoint{1.286730in}{1.564059in}}{\pgfqpoint{1.290002in}{1.556159in}}{\pgfqpoint{1.295826in}{1.550335in}}%
\pgfpathcurveto{\pgfqpoint{1.301650in}{1.544511in}}{\pgfqpoint{1.309550in}{1.541239in}}{\pgfqpoint{1.317786in}{1.541239in}}%
\pgfpathclose%
\pgfusepath{stroke,fill}%
\end{pgfscope}%
\begin{pgfscope}%
\pgfpathrectangle{\pgfqpoint{0.100000in}{0.212622in}}{\pgfqpoint{3.696000in}{3.696000in}}%
\pgfusepath{clip}%
\pgfsetbuttcap%
\pgfsetroundjoin%
\definecolor{currentfill}{rgb}{0.121569,0.466667,0.705882}%
\pgfsetfillcolor{currentfill}%
\pgfsetfillopacity{0.392407}%
\pgfsetlinewidth{1.003750pt}%
\definecolor{currentstroke}{rgb}{0.121569,0.466667,0.705882}%
\pgfsetstrokecolor{currentstroke}%
\pgfsetstrokeopacity{0.392407}%
\pgfsetdash{}{0pt}%
\pgfpathmoveto{\pgfqpoint{1.328676in}{1.537720in}}%
\pgfpathcurveto{\pgfqpoint{1.336912in}{1.537720in}}{\pgfqpoint{1.344812in}{1.540992in}}{\pgfqpoint{1.350636in}{1.546816in}}%
\pgfpathcurveto{\pgfqpoint{1.356460in}{1.552640in}}{\pgfqpoint{1.359733in}{1.560540in}}{\pgfqpoint{1.359733in}{1.568776in}}%
\pgfpathcurveto{\pgfqpoint{1.359733in}{1.577012in}}{\pgfqpoint{1.356460in}{1.584912in}}{\pgfqpoint{1.350636in}{1.590736in}}%
\pgfpathcurveto{\pgfqpoint{1.344812in}{1.596560in}}{\pgfqpoint{1.336912in}{1.599833in}}{\pgfqpoint{1.328676in}{1.599833in}}%
\pgfpathcurveto{\pgfqpoint{1.320440in}{1.599833in}}{\pgfqpoint{1.312540in}{1.596560in}}{\pgfqpoint{1.306716in}{1.590736in}}%
\pgfpathcurveto{\pgfqpoint{1.300892in}{1.584912in}}{\pgfqpoint{1.297620in}{1.577012in}}{\pgfqpoint{1.297620in}{1.568776in}}%
\pgfpathcurveto{\pgfqpoint{1.297620in}{1.560540in}}{\pgfqpoint{1.300892in}{1.552640in}}{\pgfqpoint{1.306716in}{1.546816in}}%
\pgfpathcurveto{\pgfqpoint{1.312540in}{1.540992in}}{\pgfqpoint{1.320440in}{1.537720in}}{\pgfqpoint{1.328676in}{1.537720in}}%
\pgfpathclose%
\pgfusepath{stroke,fill}%
\end{pgfscope}%
\begin{pgfscope}%
\pgfpathrectangle{\pgfqpoint{0.100000in}{0.212622in}}{\pgfqpoint{3.696000in}{3.696000in}}%
\pgfusepath{clip}%
\pgfsetbuttcap%
\pgfsetroundjoin%
\definecolor{currentfill}{rgb}{0.121569,0.466667,0.705882}%
\pgfsetfillcolor{currentfill}%
\pgfsetfillopacity{0.394698}%
\pgfsetlinewidth{1.003750pt}%
\definecolor{currentstroke}{rgb}{0.121569,0.466667,0.705882}%
\pgfsetstrokecolor{currentstroke}%
\pgfsetstrokeopacity{0.394698}%
\pgfsetdash{}{0pt}%
\pgfpathmoveto{\pgfqpoint{1.334487in}{1.535729in}}%
\pgfpathcurveto{\pgfqpoint{1.342724in}{1.535729in}}{\pgfqpoint{1.350624in}{1.539001in}}{\pgfqpoint{1.356448in}{1.544825in}}%
\pgfpathcurveto{\pgfqpoint{1.362271in}{1.550649in}}{\pgfqpoint{1.365544in}{1.558549in}}{\pgfqpoint{1.365544in}{1.566786in}}%
\pgfpathcurveto{\pgfqpoint{1.365544in}{1.575022in}}{\pgfqpoint{1.362271in}{1.582922in}}{\pgfqpoint{1.356448in}{1.588746in}}%
\pgfpathcurveto{\pgfqpoint{1.350624in}{1.594570in}}{\pgfqpoint{1.342724in}{1.597842in}}{\pgfqpoint{1.334487in}{1.597842in}}%
\pgfpathcurveto{\pgfqpoint{1.326251in}{1.597842in}}{\pgfqpoint{1.318351in}{1.594570in}}{\pgfqpoint{1.312527in}{1.588746in}}%
\pgfpathcurveto{\pgfqpoint{1.306703in}{1.582922in}}{\pgfqpoint{1.303431in}{1.575022in}}{\pgfqpoint{1.303431in}{1.566786in}}%
\pgfpathcurveto{\pgfqpoint{1.303431in}{1.558549in}}{\pgfqpoint{1.306703in}{1.550649in}}{\pgfqpoint{1.312527in}{1.544825in}}%
\pgfpathcurveto{\pgfqpoint{1.318351in}{1.539001in}}{\pgfqpoint{1.326251in}{1.535729in}}{\pgfqpoint{1.334487in}{1.535729in}}%
\pgfpathclose%
\pgfusepath{stroke,fill}%
\end{pgfscope}%
\begin{pgfscope}%
\pgfpathrectangle{\pgfqpoint{0.100000in}{0.212622in}}{\pgfqpoint{3.696000in}{3.696000in}}%
\pgfusepath{clip}%
\pgfsetbuttcap%
\pgfsetroundjoin%
\definecolor{currentfill}{rgb}{0.121569,0.466667,0.705882}%
\pgfsetfillcolor{currentfill}%
\pgfsetfillopacity{0.397705}%
\pgfsetlinewidth{1.003750pt}%
\definecolor{currentstroke}{rgb}{0.121569,0.466667,0.705882}%
\pgfsetstrokecolor{currentstroke}%
\pgfsetstrokeopacity{0.397705}%
\pgfsetdash{}{0pt}%
\pgfpathmoveto{\pgfqpoint{1.340750in}{1.533150in}}%
\pgfpathcurveto{\pgfqpoint{1.348986in}{1.533150in}}{\pgfqpoint{1.356886in}{1.536422in}}{\pgfqpoint{1.362710in}{1.542246in}}%
\pgfpathcurveto{\pgfqpoint{1.368534in}{1.548070in}}{\pgfqpoint{1.371806in}{1.555970in}}{\pgfqpoint{1.371806in}{1.564206in}}%
\pgfpathcurveto{\pgfqpoint{1.371806in}{1.572442in}}{\pgfqpoint{1.368534in}{1.580342in}}{\pgfqpoint{1.362710in}{1.586166in}}%
\pgfpathcurveto{\pgfqpoint{1.356886in}{1.591990in}}{\pgfqpoint{1.348986in}{1.595263in}}{\pgfqpoint{1.340750in}{1.595263in}}%
\pgfpathcurveto{\pgfqpoint{1.332513in}{1.595263in}}{\pgfqpoint{1.324613in}{1.591990in}}{\pgfqpoint{1.318789in}{1.586166in}}%
\pgfpathcurveto{\pgfqpoint{1.312965in}{1.580342in}}{\pgfqpoint{1.309693in}{1.572442in}}{\pgfqpoint{1.309693in}{1.564206in}}%
\pgfpathcurveto{\pgfqpoint{1.309693in}{1.555970in}}{\pgfqpoint{1.312965in}{1.548070in}}{\pgfqpoint{1.318789in}{1.542246in}}%
\pgfpathcurveto{\pgfqpoint{1.324613in}{1.536422in}}{\pgfqpoint{1.332513in}{1.533150in}}{\pgfqpoint{1.340750in}{1.533150in}}%
\pgfpathclose%
\pgfusepath{stroke,fill}%
\end{pgfscope}%
\begin{pgfscope}%
\pgfpathrectangle{\pgfqpoint{0.100000in}{0.212622in}}{\pgfqpoint{3.696000in}{3.696000in}}%
\pgfusepath{clip}%
\pgfsetbuttcap%
\pgfsetroundjoin%
\definecolor{currentfill}{rgb}{0.121569,0.466667,0.705882}%
\pgfsetfillcolor{currentfill}%
\pgfsetfillopacity{0.400367}%
\pgfsetlinewidth{1.003750pt}%
\definecolor{currentstroke}{rgb}{0.121569,0.466667,0.705882}%
\pgfsetstrokecolor{currentstroke}%
\pgfsetstrokeopacity{0.400367}%
\pgfsetdash{}{0pt}%
\pgfpathmoveto{\pgfqpoint{1.348759in}{1.530865in}}%
\pgfpathcurveto{\pgfqpoint{1.356996in}{1.530865in}}{\pgfqpoint{1.364896in}{1.534138in}}{\pgfqpoint{1.370720in}{1.539962in}}%
\pgfpathcurveto{\pgfqpoint{1.376544in}{1.545786in}}{\pgfqpoint{1.379816in}{1.553686in}}{\pgfqpoint{1.379816in}{1.561922in}}%
\pgfpathcurveto{\pgfqpoint{1.379816in}{1.570158in}}{\pgfqpoint{1.376544in}{1.578058in}}{\pgfqpoint{1.370720in}{1.583882in}}%
\pgfpathcurveto{\pgfqpoint{1.364896in}{1.589706in}}{\pgfqpoint{1.356996in}{1.592978in}}{\pgfqpoint{1.348759in}{1.592978in}}%
\pgfpathcurveto{\pgfqpoint{1.340523in}{1.592978in}}{\pgfqpoint{1.332623in}{1.589706in}}{\pgfqpoint{1.326799in}{1.583882in}}%
\pgfpathcurveto{\pgfqpoint{1.320975in}{1.578058in}}{\pgfqpoint{1.317703in}{1.570158in}}{\pgfqpoint{1.317703in}{1.561922in}}%
\pgfpathcurveto{\pgfqpoint{1.317703in}{1.553686in}}{\pgfqpoint{1.320975in}{1.545786in}}{\pgfqpoint{1.326799in}{1.539962in}}%
\pgfpathcurveto{\pgfqpoint{1.332623in}{1.534138in}}{\pgfqpoint{1.340523in}{1.530865in}}{\pgfqpoint{1.348759in}{1.530865in}}%
\pgfpathclose%
\pgfusepath{stroke,fill}%
\end{pgfscope}%
\begin{pgfscope}%
\pgfpathrectangle{\pgfqpoint{0.100000in}{0.212622in}}{\pgfqpoint{3.696000in}{3.696000in}}%
\pgfusepath{clip}%
\pgfsetbuttcap%
\pgfsetroundjoin%
\definecolor{currentfill}{rgb}{0.121569,0.466667,0.705882}%
\pgfsetfillcolor{currentfill}%
\pgfsetfillopacity{0.403168}%
\pgfsetlinewidth{1.003750pt}%
\definecolor{currentstroke}{rgb}{0.121569,0.466667,0.705882}%
\pgfsetstrokecolor{currentstroke}%
\pgfsetstrokeopacity{0.403168}%
\pgfsetdash{}{0pt}%
\pgfpathmoveto{\pgfqpoint{1.357371in}{1.528439in}}%
\pgfpathcurveto{\pgfqpoint{1.365608in}{1.528439in}}{\pgfqpoint{1.373508in}{1.531711in}}{\pgfqpoint{1.379332in}{1.537535in}}%
\pgfpathcurveto{\pgfqpoint{1.385155in}{1.543359in}}{\pgfqpoint{1.388428in}{1.551259in}}{\pgfqpoint{1.388428in}{1.559496in}}%
\pgfpathcurveto{\pgfqpoint{1.388428in}{1.567732in}}{\pgfqpoint{1.385155in}{1.575632in}}{\pgfqpoint{1.379332in}{1.581456in}}%
\pgfpathcurveto{\pgfqpoint{1.373508in}{1.587280in}}{\pgfqpoint{1.365608in}{1.590552in}}{\pgfqpoint{1.357371in}{1.590552in}}%
\pgfpathcurveto{\pgfqpoint{1.349135in}{1.590552in}}{\pgfqpoint{1.341235in}{1.587280in}}{\pgfqpoint{1.335411in}{1.581456in}}%
\pgfpathcurveto{\pgfqpoint{1.329587in}{1.575632in}}{\pgfqpoint{1.326315in}{1.567732in}}{\pgfqpoint{1.326315in}{1.559496in}}%
\pgfpathcurveto{\pgfqpoint{1.326315in}{1.551259in}}{\pgfqpoint{1.329587in}{1.543359in}}{\pgfqpoint{1.335411in}{1.537535in}}%
\pgfpathcurveto{\pgfqpoint{1.341235in}{1.531711in}}{\pgfqpoint{1.349135in}{1.528439in}}{\pgfqpoint{1.357371in}{1.528439in}}%
\pgfpathclose%
\pgfusepath{stroke,fill}%
\end{pgfscope}%
\begin{pgfscope}%
\pgfpathrectangle{\pgfqpoint{0.100000in}{0.212622in}}{\pgfqpoint{3.696000in}{3.696000in}}%
\pgfusepath{clip}%
\pgfsetbuttcap%
\pgfsetroundjoin%
\definecolor{currentfill}{rgb}{0.121569,0.466667,0.705882}%
\pgfsetfillcolor{currentfill}%
\pgfsetfillopacity{0.408926}%
\pgfsetlinewidth{1.003750pt}%
\definecolor{currentstroke}{rgb}{0.121569,0.466667,0.705882}%
\pgfsetstrokecolor{currentstroke}%
\pgfsetstrokeopacity{0.408926}%
\pgfsetdash{}{0pt}%
\pgfpathmoveto{\pgfqpoint{1.364863in}{1.523041in}}%
\pgfpathcurveto{\pgfqpoint{1.373099in}{1.523041in}}{\pgfqpoint{1.380999in}{1.526313in}}{\pgfqpoint{1.386823in}{1.532137in}}%
\pgfpathcurveto{\pgfqpoint{1.392647in}{1.537961in}}{\pgfqpoint{1.395919in}{1.545861in}}{\pgfqpoint{1.395919in}{1.554097in}}%
\pgfpathcurveto{\pgfqpoint{1.395919in}{1.562334in}}{\pgfqpoint{1.392647in}{1.570234in}}{\pgfqpoint{1.386823in}{1.576058in}}%
\pgfpathcurveto{\pgfqpoint{1.380999in}{1.581882in}}{\pgfqpoint{1.373099in}{1.585154in}}{\pgfqpoint{1.364863in}{1.585154in}}%
\pgfpathcurveto{\pgfqpoint{1.356626in}{1.585154in}}{\pgfqpoint{1.348726in}{1.581882in}}{\pgfqpoint{1.342902in}{1.576058in}}%
\pgfpathcurveto{\pgfqpoint{1.337078in}{1.570234in}}{\pgfqpoint{1.333806in}{1.562334in}}{\pgfqpoint{1.333806in}{1.554097in}}%
\pgfpathcurveto{\pgfqpoint{1.333806in}{1.545861in}}{\pgfqpoint{1.337078in}{1.537961in}}{\pgfqpoint{1.342902in}{1.532137in}}%
\pgfpathcurveto{\pgfqpoint{1.348726in}{1.526313in}}{\pgfqpoint{1.356626in}{1.523041in}}{\pgfqpoint{1.364863in}{1.523041in}}%
\pgfpathclose%
\pgfusepath{stroke,fill}%
\end{pgfscope}%
\begin{pgfscope}%
\pgfpathrectangle{\pgfqpoint{0.100000in}{0.212622in}}{\pgfqpoint{3.696000in}{3.696000in}}%
\pgfusepath{clip}%
\pgfsetbuttcap%
\pgfsetroundjoin%
\definecolor{currentfill}{rgb}{0.121569,0.466667,0.705882}%
\pgfsetfillcolor{currentfill}%
\pgfsetfillopacity{0.412605}%
\pgfsetlinewidth{1.003750pt}%
\definecolor{currentstroke}{rgb}{0.121569,0.466667,0.705882}%
\pgfsetstrokecolor{currentstroke}%
\pgfsetstrokeopacity{0.412605}%
\pgfsetdash{}{0pt}%
\pgfpathmoveto{\pgfqpoint{1.375330in}{1.519748in}}%
\pgfpathcurveto{\pgfqpoint{1.383567in}{1.519748in}}{\pgfqpoint{1.391467in}{1.523020in}}{\pgfqpoint{1.397291in}{1.528844in}}%
\pgfpathcurveto{\pgfqpoint{1.403115in}{1.534668in}}{\pgfqpoint{1.406387in}{1.542568in}}{\pgfqpoint{1.406387in}{1.550804in}}%
\pgfpathcurveto{\pgfqpoint{1.406387in}{1.559040in}}{\pgfqpoint{1.403115in}{1.566941in}}{\pgfqpoint{1.397291in}{1.572764in}}%
\pgfpathcurveto{\pgfqpoint{1.391467in}{1.578588in}}{\pgfqpoint{1.383567in}{1.581861in}}{\pgfqpoint{1.375330in}{1.581861in}}%
\pgfpathcurveto{\pgfqpoint{1.367094in}{1.581861in}}{\pgfqpoint{1.359194in}{1.578588in}}{\pgfqpoint{1.353370in}{1.572764in}}%
\pgfpathcurveto{\pgfqpoint{1.347546in}{1.566941in}}{\pgfqpoint{1.344274in}{1.559040in}}{\pgfqpoint{1.344274in}{1.550804in}}%
\pgfpathcurveto{\pgfqpoint{1.344274in}{1.542568in}}{\pgfqpoint{1.347546in}{1.534668in}}{\pgfqpoint{1.353370in}{1.528844in}}%
\pgfpathcurveto{\pgfqpoint{1.359194in}{1.523020in}}{\pgfqpoint{1.367094in}{1.519748in}}{\pgfqpoint{1.375330in}{1.519748in}}%
\pgfpathclose%
\pgfusepath{stroke,fill}%
\end{pgfscope}%
\begin{pgfscope}%
\pgfpathrectangle{\pgfqpoint{0.100000in}{0.212622in}}{\pgfqpoint{3.696000in}{3.696000in}}%
\pgfusepath{clip}%
\pgfsetbuttcap%
\pgfsetroundjoin%
\definecolor{currentfill}{rgb}{0.121569,0.466667,0.705882}%
\pgfsetfillcolor{currentfill}%
\pgfsetfillopacity{0.414369}%
\pgfsetlinewidth{1.003750pt}%
\definecolor{currentstroke}{rgb}{0.121569,0.466667,0.705882}%
\pgfsetstrokecolor{currentstroke}%
\pgfsetstrokeopacity{0.414369}%
\pgfsetdash{}{0pt}%
\pgfpathmoveto{\pgfqpoint{1.381371in}{1.518304in}}%
\pgfpathcurveto{\pgfqpoint{1.389607in}{1.518304in}}{\pgfqpoint{1.397507in}{1.521577in}}{\pgfqpoint{1.403331in}{1.527401in}}%
\pgfpathcurveto{\pgfqpoint{1.409155in}{1.533225in}}{\pgfqpoint{1.412427in}{1.541125in}}{\pgfqpoint{1.412427in}{1.549361in}}%
\pgfpathcurveto{\pgfqpoint{1.412427in}{1.557597in}}{\pgfqpoint{1.409155in}{1.565497in}}{\pgfqpoint{1.403331in}{1.571321in}}%
\pgfpathcurveto{\pgfqpoint{1.397507in}{1.577145in}}{\pgfqpoint{1.389607in}{1.580417in}}{\pgfqpoint{1.381371in}{1.580417in}}%
\pgfpathcurveto{\pgfqpoint{1.373134in}{1.580417in}}{\pgfqpoint{1.365234in}{1.577145in}}{\pgfqpoint{1.359410in}{1.571321in}}%
\pgfpathcurveto{\pgfqpoint{1.353586in}{1.565497in}}{\pgfqpoint{1.350314in}{1.557597in}}{\pgfqpoint{1.350314in}{1.549361in}}%
\pgfpathcurveto{\pgfqpoint{1.350314in}{1.541125in}}{\pgfqpoint{1.353586in}{1.533225in}}{\pgfqpoint{1.359410in}{1.527401in}}%
\pgfpathcurveto{\pgfqpoint{1.365234in}{1.521577in}}{\pgfqpoint{1.373134in}{1.518304in}}{\pgfqpoint{1.381371in}{1.518304in}}%
\pgfpathclose%
\pgfusepath{stroke,fill}%
\end{pgfscope}%
\begin{pgfscope}%
\pgfpathrectangle{\pgfqpoint{0.100000in}{0.212622in}}{\pgfqpoint{3.696000in}{3.696000in}}%
\pgfusepath{clip}%
\pgfsetbuttcap%
\pgfsetroundjoin%
\definecolor{currentfill}{rgb}{0.121569,0.466667,0.705882}%
\pgfsetfillcolor{currentfill}%
\pgfsetfillopacity{0.415888}%
\pgfsetlinewidth{1.003750pt}%
\definecolor{currentstroke}{rgb}{0.121569,0.466667,0.705882}%
\pgfsetstrokecolor{currentstroke}%
\pgfsetstrokeopacity{0.415888}%
\pgfsetdash{}{0pt}%
\pgfpathmoveto{\pgfqpoint{1.384149in}{1.516911in}}%
\pgfpathcurveto{\pgfqpoint{1.392385in}{1.516911in}}{\pgfqpoint{1.400285in}{1.520184in}}{\pgfqpoint{1.406109in}{1.526008in}}%
\pgfpathcurveto{\pgfqpoint{1.411933in}{1.531832in}}{\pgfqpoint{1.415205in}{1.539732in}}{\pgfqpoint{1.415205in}{1.547968in}}%
\pgfpathcurveto{\pgfqpoint{1.415205in}{1.556204in}}{\pgfqpoint{1.411933in}{1.564104in}}{\pgfqpoint{1.406109in}{1.569928in}}%
\pgfpathcurveto{\pgfqpoint{1.400285in}{1.575752in}}{\pgfqpoint{1.392385in}{1.579024in}}{\pgfqpoint{1.384149in}{1.579024in}}%
\pgfpathcurveto{\pgfqpoint{1.375912in}{1.579024in}}{\pgfqpoint{1.368012in}{1.575752in}}{\pgfqpoint{1.362188in}{1.569928in}}%
\pgfpathcurveto{\pgfqpoint{1.356364in}{1.564104in}}{\pgfqpoint{1.353092in}{1.556204in}}{\pgfqpoint{1.353092in}{1.547968in}}%
\pgfpathcurveto{\pgfqpoint{1.353092in}{1.539732in}}{\pgfqpoint{1.356364in}{1.531832in}}{\pgfqpoint{1.362188in}{1.526008in}}%
\pgfpathcurveto{\pgfqpoint{1.368012in}{1.520184in}}{\pgfqpoint{1.375912in}{1.516911in}}{\pgfqpoint{1.384149in}{1.516911in}}%
\pgfpathclose%
\pgfusepath{stroke,fill}%
\end{pgfscope}%
\begin{pgfscope}%
\pgfpathrectangle{\pgfqpoint{0.100000in}{0.212622in}}{\pgfqpoint{3.696000in}{3.696000in}}%
\pgfusepath{clip}%
\pgfsetbuttcap%
\pgfsetroundjoin%
\definecolor{currentfill}{rgb}{0.121569,0.466667,0.705882}%
\pgfsetfillcolor{currentfill}%
\pgfsetfillopacity{0.417158}%
\pgfsetlinewidth{1.003750pt}%
\definecolor{currentstroke}{rgb}{0.121569,0.466667,0.705882}%
\pgfsetstrokecolor{currentstroke}%
\pgfsetstrokeopacity{0.417158}%
\pgfsetdash{}{0pt}%
\pgfpathmoveto{\pgfqpoint{1.388581in}{1.515752in}}%
\pgfpathcurveto{\pgfqpoint{1.396818in}{1.515752in}}{\pgfqpoint{1.404718in}{1.519025in}}{\pgfqpoint{1.410542in}{1.524848in}}%
\pgfpathcurveto{\pgfqpoint{1.416365in}{1.530672in}}{\pgfqpoint{1.419638in}{1.538572in}}{\pgfqpoint{1.419638in}{1.546809in}}%
\pgfpathcurveto{\pgfqpoint{1.419638in}{1.555045in}}{\pgfqpoint{1.416365in}{1.562945in}}{\pgfqpoint{1.410542in}{1.568769in}}%
\pgfpathcurveto{\pgfqpoint{1.404718in}{1.574593in}}{\pgfqpoint{1.396818in}{1.577865in}}{\pgfqpoint{1.388581in}{1.577865in}}%
\pgfpathcurveto{\pgfqpoint{1.380345in}{1.577865in}}{\pgfqpoint{1.372445in}{1.574593in}}{\pgfqpoint{1.366621in}{1.568769in}}%
\pgfpathcurveto{\pgfqpoint{1.360797in}{1.562945in}}{\pgfqpoint{1.357525in}{1.555045in}}{\pgfqpoint{1.357525in}{1.546809in}}%
\pgfpathcurveto{\pgfqpoint{1.357525in}{1.538572in}}{\pgfqpoint{1.360797in}{1.530672in}}{\pgfqpoint{1.366621in}{1.524848in}}%
\pgfpathcurveto{\pgfqpoint{1.372445in}{1.519025in}}{\pgfqpoint{1.380345in}{1.515752in}}{\pgfqpoint{1.388581in}{1.515752in}}%
\pgfpathclose%
\pgfusepath{stroke,fill}%
\end{pgfscope}%
\begin{pgfscope}%
\pgfpathrectangle{\pgfqpoint{0.100000in}{0.212622in}}{\pgfqpoint{3.696000in}{3.696000in}}%
\pgfusepath{clip}%
\pgfsetbuttcap%
\pgfsetroundjoin%
\definecolor{currentfill}{rgb}{0.121569,0.466667,0.705882}%
\pgfsetfillcolor{currentfill}%
\pgfsetfillopacity{0.418025}%
\pgfsetlinewidth{1.003750pt}%
\definecolor{currentstroke}{rgb}{0.121569,0.466667,0.705882}%
\pgfsetstrokecolor{currentstroke}%
\pgfsetstrokeopacity{0.418025}%
\pgfsetdash{}{0pt}%
\pgfpathmoveto{\pgfqpoint{1.390889in}{1.515052in}}%
\pgfpathcurveto{\pgfqpoint{1.399125in}{1.515052in}}{\pgfqpoint{1.407025in}{1.518324in}}{\pgfqpoint{1.412849in}{1.524148in}}%
\pgfpathcurveto{\pgfqpoint{1.418673in}{1.529972in}}{\pgfqpoint{1.421945in}{1.537872in}}{\pgfqpoint{1.421945in}{1.546108in}}%
\pgfpathcurveto{\pgfqpoint{1.421945in}{1.554344in}}{\pgfqpoint{1.418673in}{1.562245in}}{\pgfqpoint{1.412849in}{1.568068in}}%
\pgfpathcurveto{\pgfqpoint{1.407025in}{1.573892in}}{\pgfqpoint{1.399125in}{1.577165in}}{\pgfqpoint{1.390889in}{1.577165in}}%
\pgfpathcurveto{\pgfqpoint{1.382652in}{1.577165in}}{\pgfqpoint{1.374752in}{1.573892in}}{\pgfqpoint{1.368928in}{1.568068in}}%
\pgfpathcurveto{\pgfqpoint{1.363104in}{1.562245in}}{\pgfqpoint{1.359832in}{1.554344in}}{\pgfqpoint{1.359832in}{1.546108in}}%
\pgfpathcurveto{\pgfqpoint{1.359832in}{1.537872in}}{\pgfqpoint{1.363104in}{1.529972in}}{\pgfqpoint{1.368928in}{1.524148in}}%
\pgfpathcurveto{\pgfqpoint{1.374752in}{1.518324in}}{\pgfqpoint{1.382652in}{1.515052in}}{\pgfqpoint{1.390889in}{1.515052in}}%
\pgfpathclose%
\pgfusepath{stroke,fill}%
\end{pgfscope}%
\begin{pgfscope}%
\pgfpathrectangle{\pgfqpoint{0.100000in}{0.212622in}}{\pgfqpoint{3.696000in}{3.696000in}}%
\pgfusepath{clip}%
\pgfsetbuttcap%
\pgfsetroundjoin%
\definecolor{currentfill}{rgb}{0.121569,0.466667,0.705882}%
\pgfsetfillcolor{currentfill}%
\pgfsetfillopacity{0.418961}%
\pgfsetlinewidth{1.003750pt}%
\definecolor{currentstroke}{rgb}{0.121569,0.466667,0.705882}%
\pgfsetstrokecolor{currentstroke}%
\pgfsetstrokeopacity{0.418961}%
\pgfsetdash{}{0pt}%
\pgfpathmoveto{\pgfqpoint{1.393747in}{1.514379in}}%
\pgfpathcurveto{\pgfqpoint{1.401983in}{1.514379in}}{\pgfqpoint{1.409884in}{1.517651in}}{\pgfqpoint{1.415707in}{1.523475in}}%
\pgfpathcurveto{\pgfqpoint{1.421531in}{1.529299in}}{\pgfqpoint{1.424804in}{1.537199in}}{\pgfqpoint{1.424804in}{1.545435in}}%
\pgfpathcurveto{\pgfqpoint{1.424804in}{1.553671in}}{\pgfqpoint{1.421531in}{1.561571in}}{\pgfqpoint{1.415707in}{1.567395in}}%
\pgfpathcurveto{\pgfqpoint{1.409884in}{1.573219in}}{\pgfqpoint{1.401983in}{1.576492in}}{\pgfqpoint{1.393747in}{1.576492in}}%
\pgfpathcurveto{\pgfqpoint{1.385511in}{1.576492in}}{\pgfqpoint{1.377611in}{1.573219in}}{\pgfqpoint{1.371787in}{1.567395in}}%
\pgfpathcurveto{\pgfqpoint{1.365963in}{1.561571in}}{\pgfqpoint{1.362691in}{1.553671in}}{\pgfqpoint{1.362691in}{1.545435in}}%
\pgfpathcurveto{\pgfqpoint{1.362691in}{1.537199in}}{\pgfqpoint{1.365963in}{1.529299in}}{\pgfqpoint{1.371787in}{1.523475in}}%
\pgfpathcurveto{\pgfqpoint{1.377611in}{1.517651in}}{\pgfqpoint{1.385511in}{1.514379in}}{\pgfqpoint{1.393747in}{1.514379in}}%
\pgfpathclose%
\pgfusepath{stroke,fill}%
\end{pgfscope}%
\begin{pgfscope}%
\pgfpathrectangle{\pgfqpoint{0.100000in}{0.212622in}}{\pgfqpoint{3.696000in}{3.696000in}}%
\pgfusepath{clip}%
\pgfsetbuttcap%
\pgfsetroundjoin%
\definecolor{currentfill}{rgb}{0.121569,0.466667,0.705882}%
\pgfsetfillcolor{currentfill}%
\pgfsetfillopacity{0.420251}%
\pgfsetlinewidth{1.003750pt}%
\definecolor{currentstroke}{rgb}{0.121569,0.466667,0.705882}%
\pgfsetstrokecolor{currentstroke}%
\pgfsetstrokeopacity{0.420251}%
\pgfsetdash{}{0pt}%
\pgfpathmoveto{\pgfqpoint{1.397288in}{1.513237in}}%
\pgfpathcurveto{\pgfqpoint{1.405524in}{1.513237in}}{\pgfqpoint{1.413424in}{1.516509in}}{\pgfqpoint{1.419248in}{1.522333in}}%
\pgfpathcurveto{\pgfqpoint{1.425072in}{1.528157in}}{\pgfqpoint{1.428344in}{1.536057in}}{\pgfqpoint{1.428344in}{1.544293in}}%
\pgfpathcurveto{\pgfqpoint{1.428344in}{1.552529in}}{\pgfqpoint{1.425072in}{1.560429in}}{\pgfqpoint{1.419248in}{1.566253in}}%
\pgfpathcurveto{\pgfqpoint{1.413424in}{1.572077in}}{\pgfqpoint{1.405524in}{1.575350in}}{\pgfqpoint{1.397288in}{1.575350in}}%
\pgfpathcurveto{\pgfqpoint{1.389051in}{1.575350in}}{\pgfqpoint{1.381151in}{1.572077in}}{\pgfqpoint{1.375327in}{1.566253in}}%
\pgfpathcurveto{\pgfqpoint{1.369503in}{1.560429in}}{\pgfqpoint{1.366231in}{1.552529in}}{\pgfqpoint{1.366231in}{1.544293in}}%
\pgfpathcurveto{\pgfqpoint{1.366231in}{1.536057in}}{\pgfqpoint{1.369503in}{1.528157in}}{\pgfqpoint{1.375327in}{1.522333in}}%
\pgfpathcurveto{\pgfqpoint{1.381151in}{1.516509in}}{\pgfqpoint{1.389051in}{1.513237in}}{\pgfqpoint{1.397288in}{1.513237in}}%
\pgfpathclose%
\pgfusepath{stroke,fill}%
\end{pgfscope}%
\begin{pgfscope}%
\pgfpathrectangle{\pgfqpoint{0.100000in}{0.212622in}}{\pgfqpoint{3.696000in}{3.696000in}}%
\pgfusepath{clip}%
\pgfsetbuttcap%
\pgfsetroundjoin%
\definecolor{currentfill}{rgb}{0.121569,0.466667,0.705882}%
\pgfsetfillcolor{currentfill}%
\pgfsetfillopacity{0.420888}%
\pgfsetlinewidth{1.003750pt}%
\definecolor{currentstroke}{rgb}{0.121569,0.466667,0.705882}%
\pgfsetstrokecolor{currentstroke}%
\pgfsetstrokeopacity{0.420888}%
\pgfsetdash{}{0pt}%
\pgfpathmoveto{\pgfqpoint{1.399332in}{1.512775in}}%
\pgfpathcurveto{\pgfqpoint{1.407568in}{1.512775in}}{\pgfqpoint{1.415468in}{1.516048in}}{\pgfqpoint{1.421292in}{1.521872in}}%
\pgfpathcurveto{\pgfqpoint{1.427116in}{1.527696in}}{\pgfqpoint{1.430389in}{1.535596in}}{\pgfqpoint{1.430389in}{1.543832in}}%
\pgfpathcurveto{\pgfqpoint{1.430389in}{1.552068in}}{\pgfqpoint{1.427116in}{1.559968in}}{\pgfqpoint{1.421292in}{1.565792in}}%
\pgfpathcurveto{\pgfqpoint{1.415468in}{1.571616in}}{\pgfqpoint{1.407568in}{1.574888in}}{\pgfqpoint{1.399332in}{1.574888in}}%
\pgfpathcurveto{\pgfqpoint{1.391096in}{1.574888in}}{\pgfqpoint{1.383196in}{1.571616in}}{\pgfqpoint{1.377372in}{1.565792in}}%
\pgfpathcurveto{\pgfqpoint{1.371548in}{1.559968in}}{\pgfqpoint{1.368276in}{1.552068in}}{\pgfqpoint{1.368276in}{1.543832in}}%
\pgfpathcurveto{\pgfqpoint{1.368276in}{1.535596in}}{\pgfqpoint{1.371548in}{1.527696in}}{\pgfqpoint{1.377372in}{1.521872in}}%
\pgfpathcurveto{\pgfqpoint{1.383196in}{1.516048in}}{\pgfqpoint{1.391096in}{1.512775in}}{\pgfqpoint{1.399332in}{1.512775in}}%
\pgfpathclose%
\pgfusepath{stroke,fill}%
\end{pgfscope}%
\begin{pgfscope}%
\pgfpathrectangle{\pgfqpoint{0.100000in}{0.212622in}}{\pgfqpoint{3.696000in}{3.696000in}}%
\pgfusepath{clip}%
\pgfsetbuttcap%
\pgfsetroundjoin%
\definecolor{currentfill}{rgb}{0.121569,0.466667,0.705882}%
\pgfsetfillcolor{currentfill}%
\pgfsetfillopacity{0.421234}%
\pgfsetlinewidth{1.003750pt}%
\definecolor{currentstroke}{rgb}{0.121569,0.466667,0.705882}%
\pgfsetstrokecolor{currentstroke}%
\pgfsetstrokeopacity{0.421234}%
\pgfsetdash{}{0pt}%
\pgfpathmoveto{\pgfqpoint{1.400456in}{1.512510in}}%
\pgfpathcurveto{\pgfqpoint{1.408692in}{1.512510in}}{\pgfqpoint{1.416592in}{1.515783in}}{\pgfqpoint{1.422416in}{1.521607in}}%
\pgfpathcurveto{\pgfqpoint{1.428240in}{1.527431in}}{\pgfqpoint{1.431512in}{1.535331in}}{\pgfqpoint{1.431512in}{1.543567in}}%
\pgfpathcurveto{\pgfqpoint{1.431512in}{1.551803in}}{\pgfqpoint{1.428240in}{1.559703in}}{\pgfqpoint{1.422416in}{1.565527in}}%
\pgfpathcurveto{\pgfqpoint{1.416592in}{1.571351in}}{\pgfqpoint{1.408692in}{1.574623in}}{\pgfqpoint{1.400456in}{1.574623in}}%
\pgfpathcurveto{\pgfqpoint{1.392220in}{1.574623in}}{\pgfqpoint{1.384320in}{1.571351in}}{\pgfqpoint{1.378496in}{1.565527in}}%
\pgfpathcurveto{\pgfqpoint{1.372672in}{1.559703in}}{\pgfqpoint{1.369399in}{1.551803in}}{\pgfqpoint{1.369399in}{1.543567in}}%
\pgfpathcurveto{\pgfqpoint{1.369399in}{1.535331in}}{\pgfqpoint{1.372672in}{1.527431in}}{\pgfqpoint{1.378496in}{1.521607in}}%
\pgfpathcurveto{\pgfqpoint{1.384320in}{1.515783in}}{\pgfqpoint{1.392220in}{1.512510in}}{\pgfqpoint{1.400456in}{1.512510in}}%
\pgfpathclose%
\pgfusepath{stroke,fill}%
\end{pgfscope}%
\begin{pgfscope}%
\pgfpathrectangle{\pgfqpoint{0.100000in}{0.212622in}}{\pgfqpoint{3.696000in}{3.696000in}}%
\pgfusepath{clip}%
\pgfsetbuttcap%
\pgfsetroundjoin%
\definecolor{currentfill}{rgb}{0.121569,0.466667,0.705882}%
\pgfsetfillcolor{currentfill}%
\pgfsetfillopacity{0.422181}%
\pgfsetlinewidth{1.003750pt}%
\definecolor{currentstroke}{rgb}{0.121569,0.466667,0.705882}%
\pgfsetstrokecolor{currentstroke}%
\pgfsetstrokeopacity{0.422181}%
\pgfsetdash{}{0pt}%
\pgfpathmoveto{\pgfqpoint{1.403250in}{1.511694in}}%
\pgfpathcurveto{\pgfqpoint{1.411487in}{1.511694in}}{\pgfqpoint{1.419387in}{1.514967in}}{\pgfqpoint{1.425211in}{1.520790in}}%
\pgfpathcurveto{\pgfqpoint{1.431035in}{1.526614in}}{\pgfqpoint{1.434307in}{1.534514in}}{\pgfqpoint{1.434307in}{1.542751in}}%
\pgfpathcurveto{\pgfqpoint{1.434307in}{1.550987in}}{\pgfqpoint{1.431035in}{1.558887in}}{\pgfqpoint{1.425211in}{1.564711in}}%
\pgfpathcurveto{\pgfqpoint{1.419387in}{1.570535in}}{\pgfqpoint{1.411487in}{1.573807in}}{\pgfqpoint{1.403250in}{1.573807in}}%
\pgfpathcurveto{\pgfqpoint{1.395014in}{1.573807in}}{\pgfqpoint{1.387114in}{1.570535in}}{\pgfqpoint{1.381290in}{1.564711in}}%
\pgfpathcurveto{\pgfqpoint{1.375466in}{1.558887in}}{\pgfqpoint{1.372194in}{1.550987in}}{\pgfqpoint{1.372194in}{1.542751in}}%
\pgfpathcurveto{\pgfqpoint{1.372194in}{1.534514in}}{\pgfqpoint{1.375466in}{1.526614in}}{\pgfqpoint{1.381290in}{1.520790in}}%
\pgfpathcurveto{\pgfqpoint{1.387114in}{1.514967in}}{\pgfqpoint{1.395014in}{1.511694in}}{\pgfqpoint{1.403250in}{1.511694in}}%
\pgfpathclose%
\pgfusepath{stroke,fill}%
\end{pgfscope}%
\begin{pgfscope}%
\pgfpathrectangle{\pgfqpoint{0.100000in}{0.212622in}}{\pgfqpoint{3.696000in}{3.696000in}}%
\pgfusepath{clip}%
\pgfsetbuttcap%
\pgfsetroundjoin%
\definecolor{currentfill}{rgb}{0.121569,0.466667,0.705882}%
\pgfsetfillcolor{currentfill}%
\pgfsetfillopacity{0.423798}%
\pgfsetlinewidth{1.003750pt}%
\definecolor{currentstroke}{rgb}{0.121569,0.466667,0.705882}%
\pgfsetstrokecolor{currentstroke}%
\pgfsetstrokeopacity{0.423798}%
\pgfsetdash{}{0pt}%
\pgfpathmoveto{\pgfqpoint{1.407035in}{1.510402in}}%
\pgfpathcurveto{\pgfqpoint{1.415271in}{1.510402in}}{\pgfqpoint{1.423171in}{1.513675in}}{\pgfqpoint{1.428995in}{1.519499in}}%
\pgfpathcurveto{\pgfqpoint{1.434819in}{1.525322in}}{\pgfqpoint{1.438091in}{1.533223in}}{\pgfqpoint{1.438091in}{1.541459in}}%
\pgfpathcurveto{\pgfqpoint{1.438091in}{1.549695in}}{\pgfqpoint{1.434819in}{1.557595in}}{\pgfqpoint{1.428995in}{1.563419in}}%
\pgfpathcurveto{\pgfqpoint{1.423171in}{1.569243in}}{\pgfqpoint{1.415271in}{1.572515in}}{\pgfqpoint{1.407035in}{1.572515in}}%
\pgfpathcurveto{\pgfqpoint{1.398798in}{1.572515in}}{\pgfqpoint{1.390898in}{1.569243in}}{\pgfqpoint{1.385074in}{1.563419in}}%
\pgfpathcurveto{\pgfqpoint{1.379250in}{1.557595in}}{\pgfqpoint{1.375978in}{1.549695in}}{\pgfqpoint{1.375978in}{1.541459in}}%
\pgfpathcurveto{\pgfqpoint{1.375978in}{1.533223in}}{\pgfqpoint{1.379250in}{1.525322in}}{\pgfqpoint{1.385074in}{1.519499in}}%
\pgfpathcurveto{\pgfqpoint{1.390898in}{1.513675in}}{\pgfqpoint{1.398798in}{1.510402in}}{\pgfqpoint{1.407035in}{1.510402in}}%
\pgfpathclose%
\pgfusepath{stroke,fill}%
\end{pgfscope}%
\begin{pgfscope}%
\pgfpathrectangle{\pgfqpoint{0.100000in}{0.212622in}}{\pgfqpoint{3.696000in}{3.696000in}}%
\pgfusepath{clip}%
\pgfsetbuttcap%
\pgfsetroundjoin%
\definecolor{currentfill}{rgb}{0.121569,0.466667,0.705882}%
\pgfsetfillcolor{currentfill}%
\pgfsetfillopacity{0.425484}%
\pgfsetlinewidth{1.003750pt}%
\definecolor{currentstroke}{rgb}{0.121569,0.466667,0.705882}%
\pgfsetstrokecolor{currentstroke}%
\pgfsetstrokeopacity{0.425484}%
\pgfsetdash{}{0pt}%
\pgfpathmoveto{\pgfqpoint{1.411276in}{1.508871in}}%
\pgfpathcurveto{\pgfqpoint{1.419512in}{1.508871in}}{\pgfqpoint{1.427412in}{1.512144in}}{\pgfqpoint{1.433236in}{1.517967in}}%
\pgfpathcurveto{\pgfqpoint{1.439060in}{1.523791in}}{\pgfqpoint{1.442332in}{1.531691in}}{\pgfqpoint{1.442332in}{1.539928in}}%
\pgfpathcurveto{\pgfqpoint{1.442332in}{1.548164in}}{\pgfqpoint{1.439060in}{1.556064in}}{\pgfqpoint{1.433236in}{1.561888in}}%
\pgfpathcurveto{\pgfqpoint{1.427412in}{1.567712in}}{\pgfqpoint{1.419512in}{1.570984in}}{\pgfqpoint{1.411276in}{1.570984in}}%
\pgfpathcurveto{\pgfqpoint{1.403039in}{1.570984in}}{\pgfqpoint{1.395139in}{1.567712in}}{\pgfqpoint{1.389315in}{1.561888in}}%
\pgfpathcurveto{\pgfqpoint{1.383491in}{1.556064in}}{\pgfqpoint{1.380219in}{1.548164in}}{\pgfqpoint{1.380219in}{1.539928in}}%
\pgfpathcurveto{\pgfqpoint{1.380219in}{1.531691in}}{\pgfqpoint{1.383491in}{1.523791in}}{\pgfqpoint{1.389315in}{1.517967in}}%
\pgfpathcurveto{\pgfqpoint{1.395139in}{1.512144in}}{\pgfqpoint{1.403039in}{1.508871in}}{\pgfqpoint{1.411276in}{1.508871in}}%
\pgfpathclose%
\pgfusepath{stroke,fill}%
\end{pgfscope}%
\begin{pgfscope}%
\pgfpathrectangle{\pgfqpoint{0.100000in}{0.212622in}}{\pgfqpoint{3.696000in}{3.696000in}}%
\pgfusepath{clip}%
\pgfsetbuttcap%
\pgfsetroundjoin%
\definecolor{currentfill}{rgb}{0.121569,0.466667,0.705882}%
\pgfsetfillcolor{currentfill}%
\pgfsetfillopacity{0.427416}%
\pgfsetlinewidth{1.003750pt}%
\definecolor{currentstroke}{rgb}{0.121569,0.466667,0.705882}%
\pgfsetstrokecolor{currentstroke}%
\pgfsetstrokeopacity{0.427416}%
\pgfsetdash{}{0pt}%
\pgfpathmoveto{\pgfqpoint{1.416242in}{1.507154in}}%
\pgfpathcurveto{\pgfqpoint{1.424479in}{1.507154in}}{\pgfqpoint{1.432379in}{1.510426in}}{\pgfqpoint{1.438203in}{1.516250in}}%
\pgfpathcurveto{\pgfqpoint{1.444027in}{1.522074in}}{\pgfqpoint{1.447299in}{1.529974in}}{\pgfqpoint{1.447299in}{1.538210in}}%
\pgfpathcurveto{\pgfqpoint{1.447299in}{1.546447in}}{\pgfqpoint{1.444027in}{1.554347in}}{\pgfqpoint{1.438203in}{1.560171in}}%
\pgfpathcurveto{\pgfqpoint{1.432379in}{1.565995in}}{\pgfqpoint{1.424479in}{1.569267in}}{\pgfqpoint{1.416242in}{1.569267in}}%
\pgfpathcurveto{\pgfqpoint{1.408006in}{1.569267in}}{\pgfqpoint{1.400106in}{1.565995in}}{\pgfqpoint{1.394282in}{1.560171in}}%
\pgfpathcurveto{\pgfqpoint{1.388458in}{1.554347in}}{\pgfqpoint{1.385186in}{1.546447in}}{\pgfqpoint{1.385186in}{1.538210in}}%
\pgfpathcurveto{\pgfqpoint{1.385186in}{1.529974in}}{\pgfqpoint{1.388458in}{1.522074in}}{\pgfqpoint{1.394282in}{1.516250in}}%
\pgfpathcurveto{\pgfqpoint{1.400106in}{1.510426in}}{\pgfqpoint{1.408006in}{1.507154in}}{\pgfqpoint{1.416242in}{1.507154in}}%
\pgfpathclose%
\pgfusepath{stroke,fill}%
\end{pgfscope}%
\begin{pgfscope}%
\pgfpathrectangle{\pgfqpoint{0.100000in}{0.212622in}}{\pgfqpoint{3.696000in}{3.696000in}}%
\pgfusepath{clip}%
\pgfsetbuttcap%
\pgfsetroundjoin%
\definecolor{currentfill}{rgb}{0.121569,0.466667,0.705882}%
\pgfsetfillcolor{currentfill}%
\pgfsetfillopacity{0.431100}%
\pgfsetlinewidth{1.003750pt}%
\definecolor{currentstroke}{rgb}{0.121569,0.466667,0.705882}%
\pgfsetstrokecolor{currentstroke}%
\pgfsetstrokeopacity{0.431100}%
\pgfsetdash{}{0pt}%
\pgfpathmoveto{\pgfqpoint{1.422204in}{1.503861in}}%
\pgfpathcurveto{\pgfqpoint{1.430440in}{1.503861in}}{\pgfqpoint{1.438341in}{1.507133in}}{\pgfqpoint{1.444164in}{1.512957in}}%
\pgfpathcurveto{\pgfqpoint{1.449988in}{1.518781in}}{\pgfqpoint{1.453261in}{1.526681in}}{\pgfqpoint{1.453261in}{1.534918in}}%
\pgfpathcurveto{\pgfqpoint{1.453261in}{1.543154in}}{\pgfqpoint{1.449988in}{1.551054in}}{\pgfqpoint{1.444164in}{1.556878in}}%
\pgfpathcurveto{\pgfqpoint{1.438341in}{1.562702in}}{\pgfqpoint{1.430440in}{1.565974in}}{\pgfqpoint{1.422204in}{1.565974in}}%
\pgfpathcurveto{\pgfqpoint{1.413968in}{1.565974in}}{\pgfqpoint{1.406068in}{1.562702in}}{\pgfqpoint{1.400244in}{1.556878in}}%
\pgfpathcurveto{\pgfqpoint{1.394420in}{1.551054in}}{\pgfqpoint{1.391148in}{1.543154in}}{\pgfqpoint{1.391148in}{1.534918in}}%
\pgfpathcurveto{\pgfqpoint{1.391148in}{1.526681in}}{\pgfqpoint{1.394420in}{1.518781in}}{\pgfqpoint{1.400244in}{1.512957in}}%
\pgfpathcurveto{\pgfqpoint{1.406068in}{1.507133in}}{\pgfqpoint{1.413968in}{1.503861in}}{\pgfqpoint{1.422204in}{1.503861in}}%
\pgfpathclose%
\pgfusepath{stroke,fill}%
\end{pgfscope}%
\begin{pgfscope}%
\pgfpathrectangle{\pgfqpoint{0.100000in}{0.212622in}}{\pgfqpoint{3.696000in}{3.696000in}}%
\pgfusepath{clip}%
\pgfsetbuttcap%
\pgfsetroundjoin%
\definecolor{currentfill}{rgb}{0.121569,0.466667,0.705882}%
\pgfsetfillcolor{currentfill}%
\pgfsetfillopacity{0.433802}%
\pgfsetlinewidth{1.003750pt}%
\definecolor{currentstroke}{rgb}{0.121569,0.466667,0.705882}%
\pgfsetstrokecolor{currentstroke}%
\pgfsetstrokeopacity{0.433802}%
\pgfsetdash{}{0pt}%
\pgfpathmoveto{\pgfqpoint{1.429932in}{1.501512in}}%
\pgfpathcurveto{\pgfqpoint{1.438168in}{1.501512in}}{\pgfqpoint{1.446068in}{1.504784in}}{\pgfqpoint{1.451892in}{1.510608in}}%
\pgfpathcurveto{\pgfqpoint{1.457716in}{1.516432in}}{\pgfqpoint{1.460989in}{1.524332in}}{\pgfqpoint{1.460989in}{1.532569in}}%
\pgfpathcurveto{\pgfqpoint{1.460989in}{1.540805in}}{\pgfqpoint{1.457716in}{1.548705in}}{\pgfqpoint{1.451892in}{1.554529in}}%
\pgfpathcurveto{\pgfqpoint{1.446068in}{1.560353in}}{\pgfqpoint{1.438168in}{1.563625in}}{\pgfqpoint{1.429932in}{1.563625in}}%
\pgfpathcurveto{\pgfqpoint{1.421696in}{1.563625in}}{\pgfqpoint{1.413796in}{1.560353in}}{\pgfqpoint{1.407972in}{1.554529in}}%
\pgfpathcurveto{\pgfqpoint{1.402148in}{1.548705in}}{\pgfqpoint{1.398876in}{1.540805in}}{\pgfqpoint{1.398876in}{1.532569in}}%
\pgfpathcurveto{\pgfqpoint{1.398876in}{1.524332in}}{\pgfqpoint{1.402148in}{1.516432in}}{\pgfqpoint{1.407972in}{1.510608in}}%
\pgfpathcurveto{\pgfqpoint{1.413796in}{1.504784in}}{\pgfqpoint{1.421696in}{1.501512in}}{\pgfqpoint{1.429932in}{1.501512in}}%
\pgfpathclose%
\pgfusepath{stroke,fill}%
\end{pgfscope}%
\begin{pgfscope}%
\pgfpathrectangle{\pgfqpoint{0.100000in}{0.212622in}}{\pgfqpoint{3.696000in}{3.696000in}}%
\pgfusepath{clip}%
\pgfsetbuttcap%
\pgfsetroundjoin%
\definecolor{currentfill}{rgb}{0.121569,0.466667,0.705882}%
\pgfsetfillcolor{currentfill}%
\pgfsetfillopacity{0.435665}%
\pgfsetlinewidth{1.003750pt}%
\definecolor{currentstroke}{rgb}{0.121569,0.466667,0.705882}%
\pgfsetstrokecolor{currentstroke}%
\pgfsetstrokeopacity{0.435665}%
\pgfsetdash{}{0pt}%
\pgfpathmoveto{\pgfqpoint{1.433843in}{1.499920in}}%
\pgfpathcurveto{\pgfqpoint{1.442080in}{1.499920in}}{\pgfqpoint{1.449980in}{1.503193in}}{\pgfqpoint{1.455804in}{1.509017in}}%
\pgfpathcurveto{\pgfqpoint{1.461628in}{1.514841in}}{\pgfqpoint{1.464900in}{1.522741in}}{\pgfqpoint{1.464900in}{1.530977in}}%
\pgfpathcurveto{\pgfqpoint{1.464900in}{1.539213in}}{\pgfqpoint{1.461628in}{1.547113in}}{\pgfqpoint{1.455804in}{1.552937in}}%
\pgfpathcurveto{\pgfqpoint{1.449980in}{1.558761in}}{\pgfqpoint{1.442080in}{1.562033in}}{\pgfqpoint{1.433843in}{1.562033in}}%
\pgfpathcurveto{\pgfqpoint{1.425607in}{1.562033in}}{\pgfqpoint{1.417707in}{1.558761in}}{\pgfqpoint{1.411883in}{1.552937in}}%
\pgfpathcurveto{\pgfqpoint{1.406059in}{1.547113in}}{\pgfqpoint{1.402787in}{1.539213in}}{\pgfqpoint{1.402787in}{1.530977in}}%
\pgfpathcurveto{\pgfqpoint{1.402787in}{1.522741in}}{\pgfqpoint{1.406059in}{1.514841in}}{\pgfqpoint{1.411883in}{1.509017in}}%
\pgfpathcurveto{\pgfqpoint{1.417707in}{1.503193in}}{\pgfqpoint{1.425607in}{1.499920in}}{\pgfqpoint{1.433843in}{1.499920in}}%
\pgfpathclose%
\pgfusepath{stroke,fill}%
\end{pgfscope}%
\begin{pgfscope}%
\pgfpathrectangle{\pgfqpoint{0.100000in}{0.212622in}}{\pgfqpoint{3.696000in}{3.696000in}}%
\pgfusepath{clip}%
\pgfsetbuttcap%
\pgfsetroundjoin%
\definecolor{currentfill}{rgb}{0.121569,0.466667,0.705882}%
\pgfsetfillcolor{currentfill}%
\pgfsetfillopacity{0.436534}%
\pgfsetlinewidth{1.003750pt}%
\definecolor{currentstroke}{rgb}{0.121569,0.466667,0.705882}%
\pgfsetstrokecolor{currentstroke}%
\pgfsetstrokeopacity{0.436534}%
\pgfsetdash{}{0pt}%
\pgfpathmoveto{\pgfqpoint{1.436148in}{1.499208in}}%
\pgfpathcurveto{\pgfqpoint{1.444384in}{1.499208in}}{\pgfqpoint{1.452284in}{1.502481in}}{\pgfqpoint{1.458108in}{1.508305in}}%
\pgfpathcurveto{\pgfqpoint{1.463932in}{1.514129in}}{\pgfqpoint{1.467204in}{1.522029in}}{\pgfqpoint{1.467204in}{1.530265in}}%
\pgfpathcurveto{\pgfqpoint{1.467204in}{1.538501in}}{\pgfqpoint{1.463932in}{1.546401in}}{\pgfqpoint{1.458108in}{1.552225in}}%
\pgfpathcurveto{\pgfqpoint{1.452284in}{1.558049in}}{\pgfqpoint{1.444384in}{1.561321in}}{\pgfqpoint{1.436148in}{1.561321in}}%
\pgfpathcurveto{\pgfqpoint{1.427911in}{1.561321in}}{\pgfqpoint{1.420011in}{1.558049in}}{\pgfqpoint{1.414187in}{1.552225in}}%
\pgfpathcurveto{\pgfqpoint{1.408363in}{1.546401in}}{\pgfqpoint{1.405091in}{1.538501in}}{\pgfqpoint{1.405091in}{1.530265in}}%
\pgfpathcurveto{\pgfqpoint{1.405091in}{1.522029in}}{\pgfqpoint{1.408363in}{1.514129in}}{\pgfqpoint{1.414187in}{1.508305in}}%
\pgfpathcurveto{\pgfqpoint{1.420011in}{1.502481in}}{\pgfqpoint{1.427911in}{1.499208in}}{\pgfqpoint{1.436148in}{1.499208in}}%
\pgfpathclose%
\pgfusepath{stroke,fill}%
\end{pgfscope}%
\begin{pgfscope}%
\pgfpathrectangle{\pgfqpoint{0.100000in}{0.212622in}}{\pgfqpoint{3.696000in}{3.696000in}}%
\pgfusepath{clip}%
\pgfsetbuttcap%
\pgfsetroundjoin%
\definecolor{currentfill}{rgb}{0.121569,0.466667,0.705882}%
\pgfsetfillcolor{currentfill}%
\pgfsetfillopacity{0.437847}%
\pgfsetlinewidth{1.003750pt}%
\definecolor{currentstroke}{rgb}{0.121569,0.466667,0.705882}%
\pgfsetstrokecolor{currentstroke}%
\pgfsetstrokeopacity{0.437847}%
\pgfsetdash{}{0pt}%
\pgfpathmoveto{\pgfqpoint{1.439753in}{1.498092in}}%
\pgfpathcurveto{\pgfqpoint{1.447989in}{1.498092in}}{\pgfqpoint{1.455889in}{1.501364in}}{\pgfqpoint{1.461713in}{1.507188in}}%
\pgfpathcurveto{\pgfqpoint{1.467537in}{1.513012in}}{\pgfqpoint{1.470809in}{1.520912in}}{\pgfqpoint{1.470809in}{1.529148in}}%
\pgfpathcurveto{\pgfqpoint{1.470809in}{1.537385in}}{\pgfqpoint{1.467537in}{1.545285in}}{\pgfqpoint{1.461713in}{1.551109in}}%
\pgfpathcurveto{\pgfqpoint{1.455889in}{1.556932in}}{\pgfqpoint{1.447989in}{1.560205in}}{\pgfqpoint{1.439753in}{1.560205in}}%
\pgfpathcurveto{\pgfqpoint{1.431517in}{1.560205in}}{\pgfqpoint{1.423616in}{1.556932in}}{\pgfqpoint{1.417793in}{1.551109in}}%
\pgfpathcurveto{\pgfqpoint{1.411969in}{1.545285in}}{\pgfqpoint{1.408696in}{1.537385in}}{\pgfqpoint{1.408696in}{1.529148in}}%
\pgfpathcurveto{\pgfqpoint{1.408696in}{1.520912in}}{\pgfqpoint{1.411969in}{1.513012in}}{\pgfqpoint{1.417793in}{1.507188in}}%
\pgfpathcurveto{\pgfqpoint{1.423616in}{1.501364in}}{\pgfqpoint{1.431517in}{1.498092in}}{\pgfqpoint{1.439753in}{1.498092in}}%
\pgfpathclose%
\pgfusepath{stroke,fill}%
\end{pgfscope}%
\begin{pgfscope}%
\pgfpathrectangle{\pgfqpoint{0.100000in}{0.212622in}}{\pgfqpoint{3.696000in}{3.696000in}}%
\pgfusepath{clip}%
\pgfsetbuttcap%
\pgfsetroundjoin%
\definecolor{currentfill}{rgb}{0.121569,0.466667,0.705882}%
\pgfsetfillcolor{currentfill}%
\pgfsetfillopacity{0.439600}%
\pgfsetlinewidth{1.003750pt}%
\definecolor{currentstroke}{rgb}{0.121569,0.466667,0.705882}%
\pgfsetstrokecolor{currentstroke}%
\pgfsetstrokeopacity{0.439600}%
\pgfsetdash{}{0pt}%
\pgfpathmoveto{\pgfqpoint{1.443896in}{1.496573in}}%
\pgfpathcurveto{\pgfqpoint{1.452132in}{1.496573in}}{\pgfqpoint{1.460032in}{1.499845in}}{\pgfqpoint{1.465856in}{1.505669in}}%
\pgfpathcurveto{\pgfqpoint{1.471680in}{1.511493in}}{\pgfqpoint{1.474952in}{1.519393in}}{\pgfqpoint{1.474952in}{1.527629in}}%
\pgfpathcurveto{\pgfqpoint{1.474952in}{1.535866in}}{\pgfqpoint{1.471680in}{1.543766in}}{\pgfqpoint{1.465856in}{1.549590in}}%
\pgfpathcurveto{\pgfqpoint{1.460032in}{1.555414in}}{\pgfqpoint{1.452132in}{1.558686in}}{\pgfqpoint{1.443896in}{1.558686in}}%
\pgfpathcurveto{\pgfqpoint{1.435659in}{1.558686in}}{\pgfqpoint{1.427759in}{1.555414in}}{\pgfqpoint{1.421935in}{1.549590in}}%
\pgfpathcurveto{\pgfqpoint{1.416111in}{1.543766in}}{\pgfqpoint{1.412839in}{1.535866in}}{\pgfqpoint{1.412839in}{1.527629in}}%
\pgfpathcurveto{\pgfqpoint{1.412839in}{1.519393in}}{\pgfqpoint{1.416111in}{1.511493in}}{\pgfqpoint{1.421935in}{1.505669in}}%
\pgfpathcurveto{\pgfqpoint{1.427759in}{1.499845in}}{\pgfqpoint{1.435659in}{1.496573in}}{\pgfqpoint{1.443896in}{1.496573in}}%
\pgfpathclose%
\pgfusepath{stroke,fill}%
\end{pgfscope}%
\begin{pgfscope}%
\pgfpathrectangle{\pgfqpoint{0.100000in}{0.212622in}}{\pgfqpoint{3.696000in}{3.696000in}}%
\pgfusepath{clip}%
\pgfsetbuttcap%
\pgfsetroundjoin%
\definecolor{currentfill}{rgb}{0.121569,0.466667,0.705882}%
\pgfsetfillcolor{currentfill}%
\pgfsetfillopacity{0.440568}%
\pgfsetlinewidth{1.003750pt}%
\definecolor{currentstroke}{rgb}{0.121569,0.466667,0.705882}%
\pgfsetstrokecolor{currentstroke}%
\pgfsetstrokeopacity{0.440568}%
\pgfsetdash{}{0pt}%
\pgfpathmoveto{\pgfqpoint{1.446179in}{1.495760in}}%
\pgfpathcurveto{\pgfqpoint{1.454416in}{1.495760in}}{\pgfqpoint{1.462316in}{1.499033in}}{\pgfqpoint{1.468140in}{1.504857in}}%
\pgfpathcurveto{\pgfqpoint{1.473964in}{1.510681in}}{\pgfqpoint{1.477236in}{1.518581in}}{\pgfqpoint{1.477236in}{1.526817in}}%
\pgfpathcurveto{\pgfqpoint{1.477236in}{1.535053in}}{\pgfqpoint{1.473964in}{1.542953in}}{\pgfqpoint{1.468140in}{1.548777in}}%
\pgfpathcurveto{\pgfqpoint{1.462316in}{1.554601in}}{\pgfqpoint{1.454416in}{1.557873in}}{\pgfqpoint{1.446179in}{1.557873in}}%
\pgfpathcurveto{\pgfqpoint{1.437943in}{1.557873in}}{\pgfqpoint{1.430043in}{1.554601in}}{\pgfqpoint{1.424219in}{1.548777in}}%
\pgfpathcurveto{\pgfqpoint{1.418395in}{1.542953in}}{\pgfqpoint{1.415123in}{1.535053in}}{\pgfqpoint{1.415123in}{1.526817in}}%
\pgfpathcurveto{\pgfqpoint{1.415123in}{1.518581in}}{\pgfqpoint{1.418395in}{1.510681in}}{\pgfqpoint{1.424219in}{1.504857in}}%
\pgfpathcurveto{\pgfqpoint{1.430043in}{1.499033in}}{\pgfqpoint{1.437943in}{1.495760in}}{\pgfqpoint{1.446179in}{1.495760in}}%
\pgfpathclose%
\pgfusepath{stroke,fill}%
\end{pgfscope}%
\begin{pgfscope}%
\pgfpathrectangle{\pgfqpoint{0.100000in}{0.212622in}}{\pgfqpoint{3.696000in}{3.696000in}}%
\pgfusepath{clip}%
\pgfsetbuttcap%
\pgfsetroundjoin%
\definecolor{currentfill}{rgb}{0.121569,0.466667,0.705882}%
\pgfsetfillcolor{currentfill}%
\pgfsetfillopacity{0.442382}%
\pgfsetlinewidth{1.003750pt}%
\definecolor{currentstroke}{rgb}{0.121569,0.466667,0.705882}%
\pgfsetstrokecolor{currentstroke}%
\pgfsetstrokeopacity{0.442382}%
\pgfsetdash{}{0pt}%
\pgfpathmoveto{\pgfqpoint{1.450397in}{1.494187in}}%
\pgfpathcurveto{\pgfqpoint{1.458633in}{1.494187in}}{\pgfqpoint{1.466533in}{1.497459in}}{\pgfqpoint{1.472357in}{1.503283in}}%
\pgfpathcurveto{\pgfqpoint{1.478181in}{1.509107in}}{\pgfqpoint{1.481453in}{1.517007in}}{\pgfqpoint{1.481453in}{1.525244in}}%
\pgfpathcurveto{\pgfqpoint{1.481453in}{1.533480in}}{\pgfqpoint{1.478181in}{1.541380in}}{\pgfqpoint{1.472357in}{1.547204in}}%
\pgfpathcurveto{\pgfqpoint{1.466533in}{1.553028in}}{\pgfqpoint{1.458633in}{1.556300in}}{\pgfqpoint{1.450397in}{1.556300in}}%
\pgfpathcurveto{\pgfqpoint{1.442160in}{1.556300in}}{\pgfqpoint{1.434260in}{1.553028in}}{\pgfqpoint{1.428436in}{1.547204in}}%
\pgfpathcurveto{\pgfqpoint{1.422613in}{1.541380in}}{\pgfqpoint{1.419340in}{1.533480in}}{\pgfqpoint{1.419340in}{1.525244in}}%
\pgfpathcurveto{\pgfqpoint{1.419340in}{1.517007in}}{\pgfqpoint{1.422613in}{1.509107in}}{\pgfqpoint{1.428436in}{1.503283in}}%
\pgfpathcurveto{\pgfqpoint{1.434260in}{1.497459in}}{\pgfqpoint{1.442160in}{1.494187in}}{\pgfqpoint{1.450397in}{1.494187in}}%
\pgfpathclose%
\pgfusepath{stroke,fill}%
\end{pgfscope}%
\begin{pgfscope}%
\pgfpathrectangle{\pgfqpoint{0.100000in}{0.212622in}}{\pgfqpoint{3.696000in}{3.696000in}}%
\pgfusepath{clip}%
\pgfsetbuttcap%
\pgfsetroundjoin%
\definecolor{currentfill}{rgb}{0.121569,0.466667,0.705882}%
\pgfsetfillcolor{currentfill}%
\pgfsetfillopacity{0.444723}%
\pgfsetlinewidth{1.003750pt}%
\definecolor{currentstroke}{rgb}{0.121569,0.466667,0.705882}%
\pgfsetstrokecolor{currentstroke}%
\pgfsetstrokeopacity{0.444723}%
\pgfsetdash{}{0pt}%
\pgfpathmoveto{\pgfqpoint{1.455510in}{1.492265in}}%
\pgfpathcurveto{\pgfqpoint{1.463746in}{1.492265in}}{\pgfqpoint{1.471646in}{1.495537in}}{\pgfqpoint{1.477470in}{1.501361in}}%
\pgfpathcurveto{\pgfqpoint{1.483294in}{1.507185in}}{\pgfqpoint{1.486566in}{1.515085in}}{\pgfqpoint{1.486566in}{1.523321in}}%
\pgfpathcurveto{\pgfqpoint{1.486566in}{1.531558in}}{\pgfqpoint{1.483294in}{1.539458in}}{\pgfqpoint{1.477470in}{1.545282in}}%
\pgfpathcurveto{\pgfqpoint{1.471646in}{1.551106in}}{\pgfqpoint{1.463746in}{1.554378in}}{\pgfqpoint{1.455510in}{1.554378in}}%
\pgfpathcurveto{\pgfqpoint{1.447273in}{1.554378in}}{\pgfqpoint{1.439373in}{1.551106in}}{\pgfqpoint{1.433549in}{1.545282in}}%
\pgfpathcurveto{\pgfqpoint{1.427725in}{1.539458in}}{\pgfqpoint{1.424453in}{1.531558in}}{\pgfqpoint{1.424453in}{1.523321in}}%
\pgfpathcurveto{\pgfqpoint{1.424453in}{1.515085in}}{\pgfqpoint{1.427725in}{1.507185in}}{\pgfqpoint{1.433549in}{1.501361in}}%
\pgfpathcurveto{\pgfqpoint{1.439373in}{1.495537in}}{\pgfqpoint{1.447273in}{1.492265in}}{\pgfqpoint{1.455510in}{1.492265in}}%
\pgfpathclose%
\pgfusepath{stroke,fill}%
\end{pgfscope}%
\begin{pgfscope}%
\pgfpathrectangle{\pgfqpoint{0.100000in}{0.212622in}}{\pgfqpoint{3.696000in}{3.696000in}}%
\pgfusepath{clip}%
\pgfsetbuttcap%
\pgfsetroundjoin%
\definecolor{currentfill}{rgb}{0.121569,0.466667,0.705882}%
\pgfsetfillcolor{currentfill}%
\pgfsetfillopacity{0.446536}%
\pgfsetlinewidth{1.003750pt}%
\definecolor{currentstroke}{rgb}{0.121569,0.466667,0.705882}%
\pgfsetstrokecolor{currentstroke}%
\pgfsetstrokeopacity{0.446536}%
\pgfsetdash{}{0pt}%
\pgfpathmoveto{\pgfqpoint{1.461599in}{1.490410in}}%
\pgfpathcurveto{\pgfqpoint{1.469835in}{1.490410in}}{\pgfqpoint{1.477736in}{1.493682in}}{\pgfqpoint{1.483559in}{1.499506in}}%
\pgfpathcurveto{\pgfqpoint{1.489383in}{1.505330in}}{\pgfqpoint{1.492656in}{1.513230in}}{\pgfqpoint{1.492656in}{1.521466in}}%
\pgfpathcurveto{\pgfqpoint{1.492656in}{1.529703in}}{\pgfqpoint{1.489383in}{1.537603in}}{\pgfqpoint{1.483559in}{1.543427in}}%
\pgfpathcurveto{\pgfqpoint{1.477736in}{1.549251in}}{\pgfqpoint{1.469835in}{1.552523in}}{\pgfqpoint{1.461599in}{1.552523in}}%
\pgfpathcurveto{\pgfqpoint{1.453363in}{1.552523in}}{\pgfqpoint{1.445463in}{1.549251in}}{\pgfqpoint{1.439639in}{1.543427in}}%
\pgfpathcurveto{\pgfqpoint{1.433815in}{1.537603in}}{\pgfqpoint{1.430543in}{1.529703in}}{\pgfqpoint{1.430543in}{1.521466in}}%
\pgfpathcurveto{\pgfqpoint{1.430543in}{1.513230in}}{\pgfqpoint{1.433815in}{1.505330in}}{\pgfqpoint{1.439639in}{1.499506in}}%
\pgfpathcurveto{\pgfqpoint{1.445463in}{1.493682in}}{\pgfqpoint{1.453363in}{1.490410in}}{\pgfqpoint{1.461599in}{1.490410in}}%
\pgfpathclose%
\pgfusepath{stroke,fill}%
\end{pgfscope}%
\begin{pgfscope}%
\pgfpathrectangle{\pgfqpoint{0.100000in}{0.212622in}}{\pgfqpoint{3.696000in}{3.696000in}}%
\pgfusepath{clip}%
\pgfsetbuttcap%
\pgfsetroundjoin%
\definecolor{currentfill}{rgb}{0.121569,0.466667,0.705882}%
\pgfsetfillcolor{currentfill}%
\pgfsetfillopacity{0.448024}%
\pgfsetlinewidth{1.003750pt}%
\definecolor{currentstroke}{rgb}{0.121569,0.466667,0.705882}%
\pgfsetstrokecolor{currentstroke}%
\pgfsetstrokeopacity{0.448024}%
\pgfsetdash{}{0pt}%
\pgfpathmoveto{\pgfqpoint{1.464531in}{1.489079in}}%
\pgfpathcurveto{\pgfqpoint{1.472767in}{1.489079in}}{\pgfqpoint{1.480667in}{1.492351in}}{\pgfqpoint{1.486491in}{1.498175in}}%
\pgfpathcurveto{\pgfqpoint{1.492315in}{1.503999in}}{\pgfqpoint{1.495587in}{1.511899in}}{\pgfqpoint{1.495587in}{1.520135in}}%
\pgfpathcurveto{\pgfqpoint{1.495587in}{1.528372in}}{\pgfqpoint{1.492315in}{1.536272in}}{\pgfqpoint{1.486491in}{1.542096in}}%
\pgfpathcurveto{\pgfqpoint{1.480667in}{1.547920in}}{\pgfqpoint{1.472767in}{1.551192in}}{\pgfqpoint{1.464531in}{1.551192in}}%
\pgfpathcurveto{\pgfqpoint{1.456294in}{1.551192in}}{\pgfqpoint{1.448394in}{1.547920in}}{\pgfqpoint{1.442570in}{1.542096in}}%
\pgfpathcurveto{\pgfqpoint{1.436746in}{1.536272in}}{\pgfqpoint{1.433474in}{1.528372in}}{\pgfqpoint{1.433474in}{1.520135in}}%
\pgfpathcurveto{\pgfqpoint{1.433474in}{1.511899in}}{\pgfqpoint{1.436746in}{1.503999in}}{\pgfqpoint{1.442570in}{1.498175in}}%
\pgfpathcurveto{\pgfqpoint{1.448394in}{1.492351in}}{\pgfqpoint{1.456294in}{1.489079in}}{\pgfqpoint{1.464531in}{1.489079in}}%
\pgfpathclose%
\pgfusepath{stroke,fill}%
\end{pgfscope}%
\begin{pgfscope}%
\pgfpathrectangle{\pgfqpoint{0.100000in}{0.212622in}}{\pgfqpoint{3.696000in}{3.696000in}}%
\pgfusepath{clip}%
\pgfsetbuttcap%
\pgfsetroundjoin%
\definecolor{currentfill}{rgb}{0.121569,0.466667,0.705882}%
\pgfsetfillcolor{currentfill}%
\pgfsetfillopacity{0.450130}%
\pgfsetlinewidth{1.003750pt}%
\definecolor{currentstroke}{rgb}{0.121569,0.466667,0.705882}%
\pgfsetstrokecolor{currentstroke}%
\pgfsetstrokeopacity{0.450130}%
\pgfsetdash{}{0pt}%
\pgfpathmoveto{\pgfqpoint{1.469156in}{1.487371in}}%
\pgfpathcurveto{\pgfqpoint{1.477392in}{1.487371in}}{\pgfqpoint{1.485292in}{1.490644in}}{\pgfqpoint{1.491116in}{1.496468in}}%
\pgfpathcurveto{\pgfqpoint{1.496940in}{1.502292in}}{\pgfqpoint{1.500212in}{1.510192in}}{\pgfqpoint{1.500212in}{1.518428in}}%
\pgfpathcurveto{\pgfqpoint{1.500212in}{1.526664in}}{\pgfqpoint{1.496940in}{1.534564in}}{\pgfqpoint{1.491116in}{1.540388in}}%
\pgfpathcurveto{\pgfqpoint{1.485292in}{1.546212in}}{\pgfqpoint{1.477392in}{1.549484in}}{\pgfqpoint{1.469156in}{1.549484in}}%
\pgfpathcurveto{\pgfqpoint{1.460920in}{1.549484in}}{\pgfqpoint{1.453020in}{1.546212in}}{\pgfqpoint{1.447196in}{1.540388in}}%
\pgfpathcurveto{\pgfqpoint{1.441372in}{1.534564in}}{\pgfqpoint{1.438099in}{1.526664in}}{\pgfqpoint{1.438099in}{1.518428in}}%
\pgfpathcurveto{\pgfqpoint{1.438099in}{1.510192in}}{\pgfqpoint{1.441372in}{1.502292in}}{\pgfqpoint{1.447196in}{1.496468in}}%
\pgfpathcurveto{\pgfqpoint{1.453020in}{1.490644in}}{\pgfqpoint{1.460920in}{1.487371in}}{\pgfqpoint{1.469156in}{1.487371in}}%
\pgfpathclose%
\pgfusepath{stroke,fill}%
\end{pgfscope}%
\begin{pgfscope}%
\pgfpathrectangle{\pgfqpoint{0.100000in}{0.212622in}}{\pgfqpoint{3.696000in}{3.696000in}}%
\pgfusepath{clip}%
\pgfsetbuttcap%
\pgfsetroundjoin%
\definecolor{currentfill}{rgb}{0.121569,0.466667,0.705882}%
\pgfsetfillcolor{currentfill}%
\pgfsetfillopacity{0.453244}%
\pgfsetlinewidth{1.003750pt}%
\definecolor{currentstroke}{rgb}{0.121569,0.466667,0.705882}%
\pgfsetstrokecolor{currentstroke}%
\pgfsetstrokeopacity{0.453244}%
\pgfsetdash{}{0pt}%
\pgfpathmoveto{\pgfqpoint{1.475033in}{1.484681in}}%
\pgfpathcurveto{\pgfqpoint{1.483269in}{1.484681in}}{\pgfqpoint{1.491169in}{1.487953in}}{\pgfqpoint{1.496993in}{1.493777in}}%
\pgfpathcurveto{\pgfqpoint{1.502817in}{1.499601in}}{\pgfqpoint{1.506089in}{1.507501in}}{\pgfqpoint{1.506089in}{1.515737in}}%
\pgfpathcurveto{\pgfqpoint{1.506089in}{1.523974in}}{\pgfqpoint{1.502817in}{1.531874in}}{\pgfqpoint{1.496993in}{1.537697in}}%
\pgfpathcurveto{\pgfqpoint{1.491169in}{1.543521in}}{\pgfqpoint{1.483269in}{1.546794in}}{\pgfqpoint{1.475033in}{1.546794in}}%
\pgfpathcurveto{\pgfqpoint{1.466797in}{1.546794in}}{\pgfqpoint{1.458897in}{1.543521in}}{\pgfqpoint{1.453073in}{1.537697in}}%
\pgfpathcurveto{\pgfqpoint{1.447249in}{1.531874in}}{\pgfqpoint{1.443976in}{1.523974in}}{\pgfqpoint{1.443976in}{1.515737in}}%
\pgfpathcurveto{\pgfqpoint{1.443976in}{1.507501in}}{\pgfqpoint{1.447249in}{1.499601in}}{\pgfqpoint{1.453073in}{1.493777in}}%
\pgfpathcurveto{\pgfqpoint{1.458897in}{1.487953in}}{\pgfqpoint{1.466797in}{1.484681in}}{\pgfqpoint{1.475033in}{1.484681in}}%
\pgfpathclose%
\pgfusepath{stroke,fill}%
\end{pgfscope}%
\begin{pgfscope}%
\pgfpathrectangle{\pgfqpoint{0.100000in}{0.212622in}}{\pgfqpoint{3.696000in}{3.696000in}}%
\pgfusepath{clip}%
\pgfsetbuttcap%
\pgfsetroundjoin%
\definecolor{currentfill}{rgb}{0.121569,0.466667,0.705882}%
\pgfsetfillcolor{currentfill}%
\pgfsetfillopacity{0.456661}%
\pgfsetlinewidth{1.003750pt}%
\definecolor{currentstroke}{rgb}{0.121569,0.466667,0.705882}%
\pgfsetstrokecolor{currentstroke}%
\pgfsetstrokeopacity{0.456661}%
\pgfsetdash{}{0pt}%
\pgfpathmoveto{\pgfqpoint{1.482151in}{1.481944in}}%
\pgfpathcurveto{\pgfqpoint{1.490388in}{1.481944in}}{\pgfqpoint{1.498288in}{1.485217in}}{\pgfqpoint{1.504112in}{1.491041in}}%
\pgfpathcurveto{\pgfqpoint{1.509936in}{1.496865in}}{\pgfqpoint{1.513208in}{1.504765in}}{\pgfqpoint{1.513208in}{1.513001in}}%
\pgfpathcurveto{\pgfqpoint{1.513208in}{1.521237in}}{\pgfqpoint{1.509936in}{1.529137in}}{\pgfqpoint{1.504112in}{1.534961in}}%
\pgfpathcurveto{\pgfqpoint{1.498288in}{1.540785in}}{\pgfqpoint{1.490388in}{1.544057in}}{\pgfqpoint{1.482151in}{1.544057in}}%
\pgfpathcurveto{\pgfqpoint{1.473915in}{1.544057in}}{\pgfqpoint{1.466015in}{1.540785in}}{\pgfqpoint{1.460191in}{1.534961in}}%
\pgfpathcurveto{\pgfqpoint{1.454367in}{1.529137in}}{\pgfqpoint{1.451095in}{1.521237in}}{\pgfqpoint{1.451095in}{1.513001in}}%
\pgfpathcurveto{\pgfqpoint{1.451095in}{1.504765in}}{\pgfqpoint{1.454367in}{1.496865in}}{\pgfqpoint{1.460191in}{1.491041in}}%
\pgfpathcurveto{\pgfqpoint{1.466015in}{1.485217in}}{\pgfqpoint{1.473915in}{1.481944in}}{\pgfqpoint{1.482151in}{1.481944in}}%
\pgfpathclose%
\pgfusepath{stroke,fill}%
\end{pgfscope}%
\begin{pgfscope}%
\pgfpathrectangle{\pgfqpoint{0.100000in}{0.212622in}}{\pgfqpoint{3.696000in}{3.696000in}}%
\pgfusepath{clip}%
\pgfsetbuttcap%
\pgfsetroundjoin%
\definecolor{currentfill}{rgb}{0.121569,0.466667,0.705882}%
\pgfsetfillcolor{currentfill}%
\pgfsetfillopacity{0.458302}%
\pgfsetlinewidth{1.003750pt}%
\definecolor{currentstroke}{rgb}{0.121569,0.466667,0.705882}%
\pgfsetstrokecolor{currentstroke}%
\pgfsetstrokeopacity{0.458302}%
\pgfsetdash{}{0pt}%
\pgfpathmoveto{\pgfqpoint{1.486266in}{1.480570in}}%
\pgfpathcurveto{\pgfqpoint{1.494502in}{1.480570in}}{\pgfqpoint{1.502402in}{1.483842in}}{\pgfqpoint{1.508226in}{1.489666in}}%
\pgfpathcurveto{\pgfqpoint{1.514050in}{1.495490in}}{\pgfqpoint{1.517323in}{1.503390in}}{\pgfqpoint{1.517323in}{1.511626in}}%
\pgfpathcurveto{\pgfqpoint{1.517323in}{1.519863in}}{\pgfqpoint{1.514050in}{1.527763in}}{\pgfqpoint{1.508226in}{1.533587in}}%
\pgfpathcurveto{\pgfqpoint{1.502402in}{1.539411in}}{\pgfqpoint{1.494502in}{1.542683in}}{\pgfqpoint{1.486266in}{1.542683in}}%
\pgfpathcurveto{\pgfqpoint{1.478030in}{1.542683in}}{\pgfqpoint{1.470130in}{1.539411in}}{\pgfqpoint{1.464306in}{1.533587in}}%
\pgfpathcurveto{\pgfqpoint{1.458482in}{1.527763in}}{\pgfqpoint{1.455210in}{1.519863in}}{\pgfqpoint{1.455210in}{1.511626in}}%
\pgfpathcurveto{\pgfqpoint{1.455210in}{1.503390in}}{\pgfqpoint{1.458482in}{1.495490in}}{\pgfqpoint{1.464306in}{1.489666in}}%
\pgfpathcurveto{\pgfqpoint{1.470130in}{1.483842in}}{\pgfqpoint{1.478030in}{1.480570in}}{\pgfqpoint{1.486266in}{1.480570in}}%
\pgfpathclose%
\pgfusepath{stroke,fill}%
\end{pgfscope}%
\begin{pgfscope}%
\pgfpathrectangle{\pgfqpoint{0.100000in}{0.212622in}}{\pgfqpoint{3.696000in}{3.696000in}}%
\pgfusepath{clip}%
\pgfsetbuttcap%
\pgfsetroundjoin%
\definecolor{currentfill}{rgb}{0.121569,0.466667,0.705882}%
\pgfsetfillcolor{currentfill}%
\pgfsetfillopacity{0.460661}%
\pgfsetlinewidth{1.003750pt}%
\definecolor{currentstroke}{rgb}{0.121569,0.466667,0.705882}%
\pgfsetstrokecolor{currentstroke}%
\pgfsetstrokeopacity{0.460661}%
\pgfsetdash{}{0pt}%
\pgfpathmoveto{\pgfqpoint{1.492352in}{1.478436in}}%
\pgfpathcurveto{\pgfqpoint{1.500588in}{1.478436in}}{\pgfqpoint{1.508488in}{1.481708in}}{\pgfqpoint{1.514312in}{1.487532in}}%
\pgfpathcurveto{\pgfqpoint{1.520136in}{1.493356in}}{\pgfqpoint{1.523408in}{1.501256in}}{\pgfqpoint{1.523408in}{1.509492in}}%
\pgfpathcurveto{\pgfqpoint{1.523408in}{1.517728in}}{\pgfqpoint{1.520136in}{1.525628in}}{\pgfqpoint{1.514312in}{1.531452in}}%
\pgfpathcurveto{\pgfqpoint{1.508488in}{1.537276in}}{\pgfqpoint{1.500588in}{1.540549in}}{\pgfqpoint{1.492352in}{1.540549in}}%
\pgfpathcurveto{\pgfqpoint{1.484115in}{1.540549in}}{\pgfqpoint{1.476215in}{1.537276in}}{\pgfqpoint{1.470391in}{1.531452in}}%
\pgfpathcurveto{\pgfqpoint{1.464567in}{1.525628in}}{\pgfqpoint{1.461295in}{1.517728in}}{\pgfqpoint{1.461295in}{1.509492in}}%
\pgfpathcurveto{\pgfqpoint{1.461295in}{1.501256in}}{\pgfqpoint{1.464567in}{1.493356in}}{\pgfqpoint{1.470391in}{1.487532in}}%
\pgfpathcurveto{\pgfqpoint{1.476215in}{1.481708in}}{\pgfqpoint{1.484115in}{1.478436in}}{\pgfqpoint{1.492352in}{1.478436in}}%
\pgfpathclose%
\pgfusepath{stroke,fill}%
\end{pgfscope}%
\begin{pgfscope}%
\pgfpathrectangle{\pgfqpoint{0.100000in}{0.212622in}}{\pgfqpoint{3.696000in}{3.696000in}}%
\pgfusepath{clip}%
\pgfsetbuttcap%
\pgfsetroundjoin%
\definecolor{currentfill}{rgb}{0.121569,0.466667,0.705882}%
\pgfsetfillcolor{currentfill}%
\pgfsetfillopacity{0.463906}%
\pgfsetlinewidth{1.003750pt}%
\definecolor{currentstroke}{rgb}{0.121569,0.466667,0.705882}%
\pgfsetstrokecolor{currentstroke}%
\pgfsetstrokeopacity{0.463906}%
\pgfsetdash{}{0pt}%
\pgfpathmoveto{\pgfqpoint{1.499198in}{1.475669in}}%
\pgfpathcurveto{\pgfqpoint{1.507434in}{1.475669in}}{\pgfqpoint{1.515334in}{1.478942in}}{\pgfqpoint{1.521158in}{1.484766in}}%
\pgfpathcurveto{\pgfqpoint{1.526982in}{1.490589in}}{\pgfqpoint{1.530254in}{1.498489in}}{\pgfqpoint{1.530254in}{1.506726in}}%
\pgfpathcurveto{\pgfqpoint{1.530254in}{1.514962in}}{\pgfqpoint{1.526982in}{1.522862in}}{\pgfqpoint{1.521158in}{1.528686in}}%
\pgfpathcurveto{\pgfqpoint{1.515334in}{1.534510in}}{\pgfqpoint{1.507434in}{1.537782in}}{\pgfqpoint{1.499198in}{1.537782in}}%
\pgfpathcurveto{\pgfqpoint{1.490961in}{1.537782in}}{\pgfqpoint{1.483061in}{1.534510in}}{\pgfqpoint{1.477237in}{1.528686in}}%
\pgfpathcurveto{\pgfqpoint{1.471414in}{1.522862in}}{\pgfqpoint{1.468141in}{1.514962in}}{\pgfqpoint{1.468141in}{1.506726in}}%
\pgfpathcurveto{\pgfqpoint{1.468141in}{1.498489in}}{\pgfqpoint{1.471414in}{1.490589in}}{\pgfqpoint{1.477237in}{1.484766in}}%
\pgfpathcurveto{\pgfqpoint{1.483061in}{1.478942in}}{\pgfqpoint{1.490961in}{1.475669in}}{\pgfqpoint{1.499198in}{1.475669in}}%
\pgfpathclose%
\pgfusepath{stroke,fill}%
\end{pgfscope}%
\begin{pgfscope}%
\pgfpathrectangle{\pgfqpoint{0.100000in}{0.212622in}}{\pgfqpoint{3.696000in}{3.696000in}}%
\pgfusepath{clip}%
\pgfsetbuttcap%
\pgfsetroundjoin%
\definecolor{currentfill}{rgb}{0.121569,0.466667,0.705882}%
\pgfsetfillcolor{currentfill}%
\pgfsetfillopacity{0.465466}%
\pgfsetlinewidth{1.003750pt}%
\definecolor{currentstroke}{rgb}{0.121569,0.466667,0.705882}%
\pgfsetstrokecolor{currentstroke}%
\pgfsetstrokeopacity{0.465466}%
\pgfsetdash{}{0pt}%
\pgfpathmoveto{\pgfqpoint{1.503166in}{1.474325in}}%
\pgfpathcurveto{\pgfqpoint{1.511403in}{1.474325in}}{\pgfqpoint{1.519303in}{1.477597in}}{\pgfqpoint{1.525127in}{1.483421in}}%
\pgfpathcurveto{\pgfqpoint{1.530951in}{1.489245in}}{\pgfqpoint{1.534223in}{1.497145in}}{\pgfqpoint{1.534223in}{1.505381in}}%
\pgfpathcurveto{\pgfqpoint{1.534223in}{1.513617in}}{\pgfqpoint{1.530951in}{1.521517in}}{\pgfqpoint{1.525127in}{1.527341in}}%
\pgfpathcurveto{\pgfqpoint{1.519303in}{1.533165in}}{\pgfqpoint{1.511403in}{1.536438in}}{\pgfqpoint{1.503166in}{1.536438in}}%
\pgfpathcurveto{\pgfqpoint{1.494930in}{1.536438in}}{\pgfqpoint{1.487030in}{1.533165in}}{\pgfqpoint{1.481206in}{1.527341in}}%
\pgfpathcurveto{\pgfqpoint{1.475382in}{1.521517in}}{\pgfqpoint{1.472110in}{1.513617in}}{\pgfqpoint{1.472110in}{1.505381in}}%
\pgfpathcurveto{\pgfqpoint{1.472110in}{1.497145in}}{\pgfqpoint{1.475382in}{1.489245in}}{\pgfqpoint{1.481206in}{1.483421in}}%
\pgfpathcurveto{\pgfqpoint{1.487030in}{1.477597in}}{\pgfqpoint{1.494930in}{1.474325in}}{\pgfqpoint{1.503166in}{1.474325in}}%
\pgfpathclose%
\pgfusepath{stroke,fill}%
\end{pgfscope}%
\begin{pgfscope}%
\pgfpathrectangle{\pgfqpoint{0.100000in}{0.212622in}}{\pgfqpoint{3.696000in}{3.696000in}}%
\pgfusepath{clip}%
\pgfsetbuttcap%
\pgfsetroundjoin%
\definecolor{currentfill}{rgb}{0.121569,0.466667,0.705882}%
\pgfsetfillcolor{currentfill}%
\pgfsetfillopacity{0.467271}%
\pgfsetlinewidth{1.003750pt}%
\definecolor{currentstroke}{rgb}{0.121569,0.466667,0.705882}%
\pgfsetstrokecolor{currentstroke}%
\pgfsetstrokeopacity{0.467271}%
\pgfsetdash{}{0pt}%
\pgfpathmoveto{\pgfqpoint{1.508067in}{1.472689in}}%
\pgfpathcurveto{\pgfqpoint{1.516303in}{1.472689in}}{\pgfqpoint{1.524203in}{1.475962in}}{\pgfqpoint{1.530027in}{1.481786in}}%
\pgfpathcurveto{\pgfqpoint{1.535851in}{1.487610in}}{\pgfqpoint{1.539123in}{1.495510in}}{\pgfqpoint{1.539123in}{1.503746in}}%
\pgfpathcurveto{\pgfqpoint{1.539123in}{1.511982in}}{\pgfqpoint{1.535851in}{1.519882in}}{\pgfqpoint{1.530027in}{1.525706in}}%
\pgfpathcurveto{\pgfqpoint{1.524203in}{1.531530in}}{\pgfqpoint{1.516303in}{1.534802in}}{\pgfqpoint{1.508067in}{1.534802in}}%
\pgfpathcurveto{\pgfqpoint{1.499831in}{1.534802in}}{\pgfqpoint{1.491931in}{1.531530in}}{\pgfqpoint{1.486107in}{1.525706in}}%
\pgfpathcurveto{\pgfqpoint{1.480283in}{1.519882in}}{\pgfqpoint{1.477010in}{1.511982in}}{\pgfqpoint{1.477010in}{1.503746in}}%
\pgfpathcurveto{\pgfqpoint{1.477010in}{1.495510in}}{\pgfqpoint{1.480283in}{1.487610in}}{\pgfqpoint{1.486107in}{1.481786in}}%
\pgfpathcurveto{\pgfqpoint{1.491931in}{1.475962in}}{\pgfqpoint{1.499831in}{1.472689in}}{\pgfqpoint{1.508067in}{1.472689in}}%
\pgfpathclose%
\pgfusepath{stroke,fill}%
\end{pgfscope}%
\begin{pgfscope}%
\pgfpathrectangle{\pgfqpoint{0.100000in}{0.212622in}}{\pgfqpoint{3.696000in}{3.696000in}}%
\pgfusepath{clip}%
\pgfsetbuttcap%
\pgfsetroundjoin%
\definecolor{currentfill}{rgb}{0.121569,0.466667,0.705882}%
\pgfsetfillcolor{currentfill}%
\pgfsetfillopacity{0.470318}%
\pgfsetlinewidth{1.003750pt}%
\definecolor{currentstroke}{rgb}{0.121569,0.466667,0.705882}%
\pgfsetstrokecolor{currentstroke}%
\pgfsetstrokeopacity{0.470318}%
\pgfsetdash{}{0pt}%
\pgfpathmoveto{\pgfqpoint{1.513632in}{1.470040in}}%
\pgfpathcurveto{\pgfqpoint{1.521868in}{1.470040in}}{\pgfqpoint{1.529768in}{1.473313in}}{\pgfqpoint{1.535592in}{1.479137in}}%
\pgfpathcurveto{\pgfqpoint{1.541416in}{1.484961in}}{\pgfqpoint{1.544688in}{1.492861in}}{\pgfqpoint{1.544688in}{1.501097in}}%
\pgfpathcurveto{\pgfqpoint{1.544688in}{1.509333in}}{\pgfqpoint{1.541416in}{1.517233in}}{\pgfqpoint{1.535592in}{1.523057in}}%
\pgfpathcurveto{\pgfqpoint{1.529768in}{1.528881in}}{\pgfqpoint{1.521868in}{1.532153in}}{\pgfqpoint{1.513632in}{1.532153in}}%
\pgfpathcurveto{\pgfqpoint{1.505395in}{1.532153in}}{\pgfqpoint{1.497495in}{1.528881in}}{\pgfqpoint{1.491671in}{1.523057in}}%
\pgfpathcurveto{\pgfqpoint{1.485847in}{1.517233in}}{\pgfqpoint{1.482575in}{1.509333in}}{\pgfqpoint{1.482575in}{1.501097in}}%
\pgfpathcurveto{\pgfqpoint{1.482575in}{1.492861in}}{\pgfqpoint{1.485847in}{1.484961in}}{\pgfqpoint{1.491671in}{1.479137in}}%
\pgfpathcurveto{\pgfqpoint{1.497495in}{1.473313in}}{\pgfqpoint{1.505395in}{1.470040in}}{\pgfqpoint{1.513632in}{1.470040in}}%
\pgfpathclose%
\pgfusepath{stroke,fill}%
\end{pgfscope}%
\begin{pgfscope}%
\pgfpathrectangle{\pgfqpoint{0.100000in}{0.212622in}}{\pgfqpoint{3.696000in}{3.696000in}}%
\pgfusepath{clip}%
\pgfsetbuttcap%
\pgfsetroundjoin%
\definecolor{currentfill}{rgb}{0.121569,0.466667,0.705882}%
\pgfsetfillcolor{currentfill}%
\pgfsetfillopacity{0.473608}%
\pgfsetlinewidth{1.003750pt}%
\definecolor{currentstroke}{rgb}{0.121569,0.466667,0.705882}%
\pgfsetstrokecolor{currentstroke}%
\pgfsetstrokeopacity{0.473608}%
\pgfsetdash{}{0pt}%
\pgfpathmoveto{\pgfqpoint{1.520076in}{1.467020in}}%
\pgfpathcurveto{\pgfqpoint{1.528312in}{1.467020in}}{\pgfqpoint{1.536212in}{1.470292in}}{\pgfqpoint{1.542036in}{1.476116in}}%
\pgfpathcurveto{\pgfqpoint{1.547860in}{1.481940in}}{\pgfqpoint{1.551133in}{1.489840in}}{\pgfqpoint{1.551133in}{1.498076in}}%
\pgfpathcurveto{\pgfqpoint{1.551133in}{1.506313in}}{\pgfqpoint{1.547860in}{1.514213in}}{\pgfqpoint{1.542036in}{1.520037in}}%
\pgfpathcurveto{\pgfqpoint{1.536212in}{1.525861in}}{\pgfqpoint{1.528312in}{1.529133in}}{\pgfqpoint{1.520076in}{1.529133in}}%
\pgfpathcurveto{\pgfqpoint{1.511840in}{1.529133in}}{\pgfqpoint{1.503940in}{1.525861in}}{\pgfqpoint{1.498116in}{1.520037in}}%
\pgfpathcurveto{\pgfqpoint{1.492292in}{1.514213in}}{\pgfqpoint{1.489020in}{1.506313in}}{\pgfqpoint{1.489020in}{1.498076in}}%
\pgfpathcurveto{\pgfqpoint{1.489020in}{1.489840in}}{\pgfqpoint{1.492292in}{1.481940in}}{\pgfqpoint{1.498116in}{1.476116in}}%
\pgfpathcurveto{\pgfqpoint{1.503940in}{1.470292in}}{\pgfqpoint{1.511840in}{1.467020in}}{\pgfqpoint{1.520076in}{1.467020in}}%
\pgfpathclose%
\pgfusepath{stroke,fill}%
\end{pgfscope}%
\begin{pgfscope}%
\pgfpathrectangle{\pgfqpoint{0.100000in}{0.212622in}}{\pgfqpoint{3.696000in}{3.696000in}}%
\pgfusepath{clip}%
\pgfsetbuttcap%
\pgfsetroundjoin%
\definecolor{currentfill}{rgb}{0.121569,0.466667,0.705882}%
\pgfsetfillcolor{currentfill}%
\pgfsetfillopacity{0.475232}%
\pgfsetlinewidth{1.003750pt}%
\definecolor{currentstroke}{rgb}{0.121569,0.466667,0.705882}%
\pgfsetstrokecolor{currentstroke}%
\pgfsetstrokeopacity{0.475232}%
\pgfsetdash{}{0pt}%
\pgfpathmoveto{\pgfqpoint{1.523842in}{1.465690in}}%
\pgfpathcurveto{\pgfqpoint{1.532079in}{1.465690in}}{\pgfqpoint{1.539979in}{1.468963in}}{\pgfqpoint{1.545803in}{1.474786in}}%
\pgfpathcurveto{\pgfqpoint{1.551627in}{1.480610in}}{\pgfqpoint{1.554899in}{1.488510in}}{\pgfqpoint{1.554899in}{1.496747in}}%
\pgfpathcurveto{\pgfqpoint{1.554899in}{1.504983in}}{\pgfqpoint{1.551627in}{1.512883in}}{\pgfqpoint{1.545803in}{1.518707in}}%
\pgfpathcurveto{\pgfqpoint{1.539979in}{1.524531in}}{\pgfqpoint{1.532079in}{1.527803in}}{\pgfqpoint{1.523842in}{1.527803in}}%
\pgfpathcurveto{\pgfqpoint{1.515606in}{1.527803in}}{\pgfqpoint{1.507706in}{1.524531in}}{\pgfqpoint{1.501882in}{1.518707in}}%
\pgfpathcurveto{\pgfqpoint{1.496058in}{1.512883in}}{\pgfqpoint{1.492786in}{1.504983in}}{\pgfqpoint{1.492786in}{1.496747in}}%
\pgfpathcurveto{\pgfqpoint{1.492786in}{1.488510in}}{\pgfqpoint{1.496058in}{1.480610in}}{\pgfqpoint{1.501882in}{1.474786in}}%
\pgfpathcurveto{\pgfqpoint{1.507706in}{1.468963in}}{\pgfqpoint{1.515606in}{1.465690in}}{\pgfqpoint{1.523842in}{1.465690in}}%
\pgfpathclose%
\pgfusepath{stroke,fill}%
\end{pgfscope}%
\begin{pgfscope}%
\pgfpathrectangle{\pgfqpoint{0.100000in}{0.212622in}}{\pgfqpoint{3.696000in}{3.696000in}}%
\pgfusepath{clip}%
\pgfsetbuttcap%
\pgfsetroundjoin%
\definecolor{currentfill}{rgb}{0.121569,0.466667,0.705882}%
\pgfsetfillcolor{currentfill}%
\pgfsetfillopacity{0.477172}%
\pgfsetlinewidth{1.003750pt}%
\definecolor{currentstroke}{rgb}{0.121569,0.466667,0.705882}%
\pgfsetstrokecolor{currentstroke}%
\pgfsetstrokeopacity{0.477172}%
\pgfsetdash{}{0pt}%
\pgfpathmoveto{\pgfqpoint{1.528114in}{1.464141in}}%
\pgfpathcurveto{\pgfqpoint{1.536350in}{1.464141in}}{\pgfqpoint{1.544250in}{1.467413in}}{\pgfqpoint{1.550074in}{1.473237in}}%
\pgfpathcurveto{\pgfqpoint{1.555898in}{1.479061in}}{\pgfqpoint{1.559170in}{1.486961in}}{\pgfqpoint{1.559170in}{1.495197in}}%
\pgfpathcurveto{\pgfqpoint{1.559170in}{1.503434in}}{\pgfqpoint{1.555898in}{1.511334in}}{\pgfqpoint{1.550074in}{1.517158in}}%
\pgfpathcurveto{\pgfqpoint{1.544250in}{1.522982in}}{\pgfqpoint{1.536350in}{1.526254in}}{\pgfqpoint{1.528114in}{1.526254in}}%
\pgfpathcurveto{\pgfqpoint{1.519878in}{1.526254in}}{\pgfqpoint{1.511978in}{1.522982in}}{\pgfqpoint{1.506154in}{1.517158in}}%
\pgfpathcurveto{\pgfqpoint{1.500330in}{1.511334in}}{\pgfqpoint{1.497057in}{1.503434in}}{\pgfqpoint{1.497057in}{1.495197in}}%
\pgfpathcurveto{\pgfqpoint{1.497057in}{1.486961in}}{\pgfqpoint{1.500330in}{1.479061in}}{\pgfqpoint{1.506154in}{1.473237in}}%
\pgfpathcurveto{\pgfqpoint{1.511978in}{1.467413in}}{\pgfqpoint{1.519878in}{1.464141in}}{\pgfqpoint{1.528114in}{1.464141in}}%
\pgfpathclose%
\pgfusepath{stroke,fill}%
\end{pgfscope}%
\begin{pgfscope}%
\pgfpathrectangle{\pgfqpoint{0.100000in}{0.212622in}}{\pgfqpoint{3.696000in}{3.696000in}}%
\pgfusepath{clip}%
\pgfsetbuttcap%
\pgfsetroundjoin%
\definecolor{currentfill}{rgb}{0.121569,0.466667,0.705882}%
\pgfsetfillcolor{currentfill}%
\pgfsetfillopacity{0.479452}%
\pgfsetlinewidth{1.003750pt}%
\definecolor{currentstroke}{rgb}{0.121569,0.466667,0.705882}%
\pgfsetstrokecolor{currentstroke}%
\pgfsetstrokeopacity{0.479452}%
\pgfsetdash{}{0pt}%
\pgfpathmoveto{\pgfqpoint{1.534216in}{1.462169in}}%
\pgfpathcurveto{\pgfqpoint{1.542452in}{1.462169in}}{\pgfqpoint{1.550352in}{1.465442in}}{\pgfqpoint{1.556176in}{1.471266in}}%
\pgfpathcurveto{\pgfqpoint{1.562000in}{1.477090in}}{\pgfqpoint{1.565273in}{1.484990in}}{\pgfqpoint{1.565273in}{1.493226in}}%
\pgfpathcurveto{\pgfqpoint{1.565273in}{1.501462in}}{\pgfqpoint{1.562000in}{1.509362in}}{\pgfqpoint{1.556176in}{1.515186in}}%
\pgfpathcurveto{\pgfqpoint{1.550352in}{1.521010in}}{\pgfqpoint{1.542452in}{1.524282in}}{\pgfqpoint{1.534216in}{1.524282in}}%
\pgfpathcurveto{\pgfqpoint{1.525980in}{1.524282in}}{\pgfqpoint{1.518080in}{1.521010in}}{\pgfqpoint{1.512256in}{1.515186in}}%
\pgfpathcurveto{\pgfqpoint{1.506432in}{1.509362in}}{\pgfqpoint{1.503160in}{1.501462in}}{\pgfqpoint{1.503160in}{1.493226in}}%
\pgfpathcurveto{\pgfqpoint{1.503160in}{1.484990in}}{\pgfqpoint{1.506432in}{1.477090in}}{\pgfqpoint{1.512256in}{1.471266in}}%
\pgfpathcurveto{\pgfqpoint{1.518080in}{1.465442in}}{\pgfqpoint{1.525980in}{1.462169in}}{\pgfqpoint{1.534216in}{1.462169in}}%
\pgfpathclose%
\pgfusepath{stroke,fill}%
\end{pgfscope}%
\begin{pgfscope}%
\pgfpathrectangle{\pgfqpoint{0.100000in}{0.212622in}}{\pgfqpoint{3.696000in}{3.696000in}}%
\pgfusepath{clip}%
\pgfsetbuttcap%
\pgfsetroundjoin%
\definecolor{currentfill}{rgb}{0.121569,0.466667,0.705882}%
\pgfsetfillcolor{currentfill}%
\pgfsetfillopacity{0.482642}%
\pgfsetlinewidth{1.003750pt}%
\definecolor{currentstroke}{rgb}{0.121569,0.466667,0.705882}%
\pgfsetstrokecolor{currentstroke}%
\pgfsetstrokeopacity{0.482642}%
\pgfsetdash{}{0pt}%
\pgfpathmoveto{\pgfqpoint{1.541386in}{1.459605in}}%
\pgfpathcurveto{\pgfqpoint{1.549622in}{1.459605in}}{\pgfqpoint{1.557522in}{1.462878in}}{\pgfqpoint{1.563346in}{1.468702in}}%
\pgfpathcurveto{\pgfqpoint{1.569170in}{1.474525in}}{\pgfqpoint{1.572442in}{1.482425in}}{\pgfqpoint{1.572442in}{1.490662in}}%
\pgfpathcurveto{\pgfqpoint{1.572442in}{1.498898in}}{\pgfqpoint{1.569170in}{1.506798in}}{\pgfqpoint{1.563346in}{1.512622in}}%
\pgfpathcurveto{\pgfqpoint{1.557522in}{1.518446in}}{\pgfqpoint{1.549622in}{1.521718in}}{\pgfqpoint{1.541386in}{1.521718in}}%
\pgfpathcurveto{\pgfqpoint{1.533149in}{1.521718in}}{\pgfqpoint{1.525249in}{1.518446in}}{\pgfqpoint{1.519425in}{1.512622in}}%
\pgfpathcurveto{\pgfqpoint{1.513601in}{1.506798in}}{\pgfqpoint{1.510329in}{1.498898in}}{\pgfqpoint{1.510329in}{1.490662in}}%
\pgfpathcurveto{\pgfqpoint{1.510329in}{1.482425in}}{\pgfqpoint{1.513601in}{1.474525in}}{\pgfqpoint{1.519425in}{1.468702in}}%
\pgfpathcurveto{\pgfqpoint{1.525249in}{1.462878in}}{\pgfqpoint{1.533149in}{1.459605in}}{\pgfqpoint{1.541386in}{1.459605in}}%
\pgfpathclose%
\pgfusepath{stroke,fill}%
\end{pgfscope}%
\begin{pgfscope}%
\pgfpathrectangle{\pgfqpoint{0.100000in}{0.212622in}}{\pgfqpoint{3.696000in}{3.696000in}}%
\pgfusepath{clip}%
\pgfsetbuttcap%
\pgfsetroundjoin%
\definecolor{currentfill}{rgb}{0.121569,0.466667,0.705882}%
\pgfsetfillcolor{currentfill}%
\pgfsetfillopacity{0.484515}%
\pgfsetlinewidth{1.003750pt}%
\definecolor{currentstroke}{rgb}{0.121569,0.466667,0.705882}%
\pgfsetstrokecolor{currentstroke}%
\pgfsetstrokeopacity{0.484515}%
\pgfsetdash{}{0pt}%
\pgfpathmoveto{\pgfqpoint{1.545205in}{1.458036in}}%
\pgfpathcurveto{\pgfqpoint{1.553441in}{1.458036in}}{\pgfqpoint{1.561341in}{1.461308in}}{\pgfqpoint{1.567165in}{1.467132in}}%
\pgfpathcurveto{\pgfqpoint{1.572989in}{1.472956in}}{\pgfqpoint{1.576261in}{1.480856in}}{\pgfqpoint{1.576261in}{1.489092in}}%
\pgfpathcurveto{\pgfqpoint{1.576261in}{1.497328in}}{\pgfqpoint{1.572989in}{1.505228in}}{\pgfqpoint{1.567165in}{1.511052in}}%
\pgfpathcurveto{\pgfqpoint{1.561341in}{1.516876in}}{\pgfqpoint{1.553441in}{1.520149in}}{\pgfqpoint{1.545205in}{1.520149in}}%
\pgfpathcurveto{\pgfqpoint{1.536969in}{1.520149in}}{\pgfqpoint{1.529069in}{1.516876in}}{\pgfqpoint{1.523245in}{1.511052in}}%
\pgfpathcurveto{\pgfqpoint{1.517421in}{1.505228in}}{\pgfqpoint{1.514148in}{1.497328in}}{\pgfqpoint{1.514148in}{1.489092in}}%
\pgfpathcurveto{\pgfqpoint{1.514148in}{1.480856in}}{\pgfqpoint{1.517421in}{1.472956in}}{\pgfqpoint{1.523245in}{1.467132in}}%
\pgfpathcurveto{\pgfqpoint{1.529069in}{1.461308in}}{\pgfqpoint{1.536969in}{1.458036in}}{\pgfqpoint{1.545205in}{1.458036in}}%
\pgfpathclose%
\pgfusepath{stroke,fill}%
\end{pgfscope}%
\begin{pgfscope}%
\pgfpathrectangle{\pgfqpoint{0.100000in}{0.212622in}}{\pgfqpoint{3.696000in}{3.696000in}}%
\pgfusepath{clip}%
\pgfsetbuttcap%
\pgfsetroundjoin%
\definecolor{currentfill}{rgb}{0.121569,0.466667,0.705882}%
\pgfsetfillcolor{currentfill}%
\pgfsetfillopacity{0.486557}%
\pgfsetlinewidth{1.003750pt}%
\definecolor{currentstroke}{rgb}{0.121569,0.466667,0.705882}%
\pgfsetstrokecolor{currentstroke}%
\pgfsetstrokeopacity{0.486557}%
\pgfsetdash{}{0pt}%
\pgfpathmoveto{\pgfqpoint{1.549715in}{1.456268in}}%
\pgfpathcurveto{\pgfqpoint{1.557951in}{1.456268in}}{\pgfqpoint{1.565851in}{1.459541in}}{\pgfqpoint{1.571675in}{1.465365in}}%
\pgfpathcurveto{\pgfqpoint{1.577499in}{1.471189in}}{\pgfqpoint{1.580771in}{1.479089in}}{\pgfqpoint{1.580771in}{1.487325in}}%
\pgfpathcurveto{\pgfqpoint{1.580771in}{1.495561in}}{\pgfqpoint{1.577499in}{1.503461in}}{\pgfqpoint{1.571675in}{1.509285in}}%
\pgfpathcurveto{\pgfqpoint{1.565851in}{1.515109in}}{\pgfqpoint{1.557951in}{1.518381in}}{\pgfqpoint{1.549715in}{1.518381in}}%
\pgfpathcurveto{\pgfqpoint{1.541478in}{1.518381in}}{\pgfqpoint{1.533578in}{1.515109in}}{\pgfqpoint{1.527754in}{1.509285in}}%
\pgfpathcurveto{\pgfqpoint{1.521931in}{1.503461in}}{\pgfqpoint{1.518658in}{1.495561in}}{\pgfqpoint{1.518658in}{1.487325in}}%
\pgfpathcurveto{\pgfqpoint{1.518658in}{1.479089in}}{\pgfqpoint{1.521931in}{1.471189in}}{\pgfqpoint{1.527754in}{1.465365in}}%
\pgfpathcurveto{\pgfqpoint{1.533578in}{1.459541in}}{\pgfqpoint{1.541478in}{1.456268in}}{\pgfqpoint{1.549715in}{1.456268in}}%
\pgfpathclose%
\pgfusepath{stroke,fill}%
\end{pgfscope}%
\begin{pgfscope}%
\pgfpathrectangle{\pgfqpoint{0.100000in}{0.212622in}}{\pgfqpoint{3.696000in}{3.696000in}}%
\pgfusepath{clip}%
\pgfsetbuttcap%
\pgfsetroundjoin%
\definecolor{currentfill}{rgb}{0.121569,0.466667,0.705882}%
\pgfsetfillcolor{currentfill}%
\pgfsetfillopacity{0.488888}%
\pgfsetlinewidth{1.003750pt}%
\definecolor{currentstroke}{rgb}{0.121569,0.466667,0.705882}%
\pgfsetstrokecolor{currentstroke}%
\pgfsetstrokeopacity{0.488888}%
\pgfsetdash{}{0pt}%
\pgfpathmoveto{\pgfqpoint{1.555380in}{1.454377in}}%
\pgfpathcurveto{\pgfqpoint{1.563617in}{1.454377in}}{\pgfqpoint{1.571517in}{1.457649in}}{\pgfqpoint{1.577341in}{1.463473in}}%
\pgfpathcurveto{\pgfqpoint{1.583165in}{1.469297in}}{\pgfqpoint{1.586437in}{1.477197in}}{\pgfqpoint{1.586437in}{1.485433in}}%
\pgfpathcurveto{\pgfqpoint{1.586437in}{1.493669in}}{\pgfqpoint{1.583165in}{1.501569in}}{\pgfqpoint{1.577341in}{1.507393in}}%
\pgfpathcurveto{\pgfqpoint{1.571517in}{1.513217in}}{\pgfqpoint{1.563617in}{1.516490in}}{\pgfqpoint{1.555380in}{1.516490in}}%
\pgfpathcurveto{\pgfqpoint{1.547144in}{1.516490in}}{\pgfqpoint{1.539244in}{1.513217in}}{\pgfqpoint{1.533420in}{1.507393in}}%
\pgfpathcurveto{\pgfqpoint{1.527596in}{1.501569in}}{\pgfqpoint{1.524324in}{1.493669in}}{\pgfqpoint{1.524324in}{1.485433in}}%
\pgfpathcurveto{\pgfqpoint{1.524324in}{1.477197in}}{\pgfqpoint{1.527596in}{1.469297in}}{\pgfqpoint{1.533420in}{1.463473in}}%
\pgfpathcurveto{\pgfqpoint{1.539244in}{1.457649in}}{\pgfqpoint{1.547144in}{1.454377in}}{\pgfqpoint{1.555380in}{1.454377in}}%
\pgfpathclose%
\pgfusepath{stroke,fill}%
\end{pgfscope}%
\begin{pgfscope}%
\pgfpathrectangle{\pgfqpoint{0.100000in}{0.212622in}}{\pgfqpoint{3.696000in}{3.696000in}}%
\pgfusepath{clip}%
\pgfsetbuttcap%
\pgfsetroundjoin%
\definecolor{currentfill}{rgb}{0.121569,0.466667,0.705882}%
\pgfsetfillcolor{currentfill}%
\pgfsetfillopacity{0.491796}%
\pgfsetlinewidth{1.003750pt}%
\definecolor{currentstroke}{rgb}{0.121569,0.466667,0.705882}%
\pgfsetstrokecolor{currentstroke}%
\pgfsetstrokeopacity{0.491796}%
\pgfsetdash{}{0pt}%
\pgfpathmoveto{\pgfqpoint{1.561832in}{1.451821in}}%
\pgfpathcurveto{\pgfqpoint{1.570068in}{1.451821in}}{\pgfqpoint{1.577968in}{1.455093in}}{\pgfqpoint{1.583792in}{1.460917in}}%
\pgfpathcurveto{\pgfqpoint{1.589616in}{1.466741in}}{\pgfqpoint{1.592888in}{1.474641in}}{\pgfqpoint{1.592888in}{1.482877in}}%
\pgfpathcurveto{\pgfqpoint{1.592888in}{1.491113in}}{\pgfqpoint{1.589616in}{1.499014in}}{\pgfqpoint{1.583792in}{1.504837in}}%
\pgfpathcurveto{\pgfqpoint{1.577968in}{1.510661in}}{\pgfqpoint{1.570068in}{1.513934in}}{\pgfqpoint{1.561832in}{1.513934in}}%
\pgfpathcurveto{\pgfqpoint{1.553596in}{1.513934in}}{\pgfqpoint{1.545695in}{1.510661in}}{\pgfqpoint{1.539872in}{1.504837in}}%
\pgfpathcurveto{\pgfqpoint{1.534048in}{1.499014in}}{\pgfqpoint{1.530775in}{1.491113in}}{\pgfqpoint{1.530775in}{1.482877in}}%
\pgfpathcurveto{\pgfqpoint{1.530775in}{1.474641in}}{\pgfqpoint{1.534048in}{1.466741in}}{\pgfqpoint{1.539872in}{1.460917in}}%
\pgfpathcurveto{\pgfqpoint{1.545695in}{1.455093in}}{\pgfqpoint{1.553596in}{1.451821in}}{\pgfqpoint{1.561832in}{1.451821in}}%
\pgfpathclose%
\pgfusepath{stroke,fill}%
\end{pgfscope}%
\begin{pgfscope}%
\pgfpathrectangle{\pgfqpoint{0.100000in}{0.212622in}}{\pgfqpoint{3.696000in}{3.696000in}}%
\pgfusepath{clip}%
\pgfsetbuttcap%
\pgfsetroundjoin%
\definecolor{currentfill}{rgb}{0.121569,0.466667,0.705882}%
\pgfsetfillcolor{currentfill}%
\pgfsetfillopacity{0.493592}%
\pgfsetlinewidth{1.003750pt}%
\definecolor{currentstroke}{rgb}{0.121569,0.466667,0.705882}%
\pgfsetstrokecolor{currentstroke}%
\pgfsetstrokeopacity{0.493592}%
\pgfsetdash{}{0pt}%
\pgfpathmoveto{\pgfqpoint{1.565231in}{1.450348in}}%
\pgfpathcurveto{\pgfqpoint{1.573468in}{1.450348in}}{\pgfqpoint{1.581368in}{1.453621in}}{\pgfqpoint{1.587192in}{1.459445in}}%
\pgfpathcurveto{\pgfqpoint{1.593016in}{1.465269in}}{\pgfqpoint{1.596288in}{1.473169in}}{\pgfqpoint{1.596288in}{1.481405in}}%
\pgfpathcurveto{\pgfqpoint{1.596288in}{1.489641in}}{\pgfqpoint{1.593016in}{1.497541in}}{\pgfqpoint{1.587192in}{1.503365in}}%
\pgfpathcurveto{\pgfqpoint{1.581368in}{1.509189in}}{\pgfqpoint{1.573468in}{1.512461in}}{\pgfqpoint{1.565231in}{1.512461in}}%
\pgfpathcurveto{\pgfqpoint{1.556995in}{1.512461in}}{\pgfqpoint{1.549095in}{1.509189in}}{\pgfqpoint{1.543271in}{1.503365in}}%
\pgfpathcurveto{\pgfqpoint{1.537447in}{1.497541in}}{\pgfqpoint{1.534175in}{1.489641in}}{\pgfqpoint{1.534175in}{1.481405in}}%
\pgfpathcurveto{\pgfqpoint{1.534175in}{1.473169in}}{\pgfqpoint{1.537447in}{1.465269in}}{\pgfqpoint{1.543271in}{1.459445in}}%
\pgfpathcurveto{\pgfqpoint{1.549095in}{1.453621in}}{\pgfqpoint{1.556995in}{1.450348in}}{\pgfqpoint{1.565231in}{1.450348in}}%
\pgfpathclose%
\pgfusepath{stroke,fill}%
\end{pgfscope}%
\begin{pgfscope}%
\pgfpathrectangle{\pgfqpoint{0.100000in}{0.212622in}}{\pgfqpoint{3.696000in}{3.696000in}}%
\pgfusepath{clip}%
\pgfsetbuttcap%
\pgfsetroundjoin%
\definecolor{currentfill}{rgb}{0.121569,0.466667,0.705882}%
\pgfsetfillcolor{currentfill}%
\pgfsetfillopacity{0.494336}%
\pgfsetlinewidth{1.003750pt}%
\definecolor{currentstroke}{rgb}{0.121569,0.466667,0.705882}%
\pgfsetstrokecolor{currentstroke}%
\pgfsetstrokeopacity{0.494336}%
\pgfsetdash{}{0pt}%
\pgfpathmoveto{\pgfqpoint{1.567333in}{1.449781in}}%
\pgfpathcurveto{\pgfqpoint{1.575569in}{1.449781in}}{\pgfqpoint{1.583469in}{1.453053in}}{\pgfqpoint{1.589293in}{1.458877in}}%
\pgfpathcurveto{\pgfqpoint{1.595117in}{1.464701in}}{\pgfqpoint{1.598390in}{1.472601in}}{\pgfqpoint{1.598390in}{1.480837in}}%
\pgfpathcurveto{\pgfqpoint{1.598390in}{1.489074in}}{\pgfqpoint{1.595117in}{1.496974in}}{\pgfqpoint{1.589293in}{1.502798in}}%
\pgfpathcurveto{\pgfqpoint{1.583469in}{1.508621in}}{\pgfqpoint{1.575569in}{1.511894in}}{\pgfqpoint{1.567333in}{1.511894in}}%
\pgfpathcurveto{\pgfqpoint{1.559097in}{1.511894in}}{\pgfqpoint{1.551197in}{1.508621in}}{\pgfqpoint{1.545373in}{1.502798in}}%
\pgfpathcurveto{\pgfqpoint{1.539549in}{1.496974in}}{\pgfqpoint{1.536277in}{1.489074in}}{\pgfqpoint{1.536277in}{1.480837in}}%
\pgfpathcurveto{\pgfqpoint{1.536277in}{1.472601in}}{\pgfqpoint{1.539549in}{1.464701in}}{\pgfqpoint{1.545373in}{1.458877in}}%
\pgfpathcurveto{\pgfqpoint{1.551197in}{1.453053in}}{\pgfqpoint{1.559097in}{1.449781in}}{\pgfqpoint{1.567333in}{1.449781in}}%
\pgfpathclose%
\pgfusepath{stroke,fill}%
\end{pgfscope}%
\begin{pgfscope}%
\pgfpathrectangle{\pgfqpoint{0.100000in}{0.212622in}}{\pgfqpoint{3.696000in}{3.696000in}}%
\pgfusepath{clip}%
\pgfsetbuttcap%
\pgfsetroundjoin%
\definecolor{currentfill}{rgb}{0.121569,0.466667,0.705882}%
\pgfsetfillcolor{currentfill}%
\pgfsetfillopacity{0.495610}%
\pgfsetlinewidth{1.003750pt}%
\definecolor{currentstroke}{rgb}{0.121569,0.466667,0.705882}%
\pgfsetstrokecolor{currentstroke}%
\pgfsetstrokeopacity{0.495610}%
\pgfsetdash{}{0pt}%
\pgfpathmoveto{\pgfqpoint{1.570507in}{1.448831in}}%
\pgfpathcurveto{\pgfqpoint{1.578744in}{1.448831in}}{\pgfqpoint{1.586644in}{1.452104in}}{\pgfqpoint{1.592468in}{1.457928in}}%
\pgfpathcurveto{\pgfqpoint{1.598292in}{1.463751in}}{\pgfqpoint{1.601564in}{1.471652in}}{\pgfqpoint{1.601564in}{1.479888in}}%
\pgfpathcurveto{\pgfqpoint{1.601564in}{1.488124in}}{\pgfqpoint{1.598292in}{1.496024in}}{\pgfqpoint{1.592468in}{1.501848in}}%
\pgfpathcurveto{\pgfqpoint{1.586644in}{1.507672in}}{\pgfqpoint{1.578744in}{1.510944in}}{\pgfqpoint{1.570507in}{1.510944in}}%
\pgfpathcurveto{\pgfqpoint{1.562271in}{1.510944in}}{\pgfqpoint{1.554371in}{1.507672in}}{\pgfqpoint{1.548547in}{1.501848in}}%
\pgfpathcurveto{\pgfqpoint{1.542723in}{1.496024in}}{\pgfqpoint{1.539451in}{1.488124in}}{\pgfqpoint{1.539451in}{1.479888in}}%
\pgfpathcurveto{\pgfqpoint{1.539451in}{1.471652in}}{\pgfqpoint{1.542723in}{1.463751in}}{\pgfqpoint{1.548547in}{1.457928in}}%
\pgfpathcurveto{\pgfqpoint{1.554371in}{1.452104in}}{\pgfqpoint{1.562271in}{1.448831in}}{\pgfqpoint{1.570507in}{1.448831in}}%
\pgfpathclose%
\pgfusepath{stroke,fill}%
\end{pgfscope}%
\begin{pgfscope}%
\pgfpathrectangle{\pgfqpoint{0.100000in}{0.212622in}}{\pgfqpoint{3.696000in}{3.696000in}}%
\pgfusepath{clip}%
\pgfsetbuttcap%
\pgfsetroundjoin%
\definecolor{currentfill}{rgb}{0.121569,0.466667,0.705882}%
\pgfsetfillcolor{currentfill}%
\pgfsetfillopacity{0.497142}%
\pgfsetlinewidth{1.003750pt}%
\definecolor{currentstroke}{rgb}{0.121569,0.466667,0.705882}%
\pgfsetstrokecolor{currentstroke}%
\pgfsetstrokeopacity{0.497142}%
\pgfsetdash{}{0pt}%
\pgfpathmoveto{\pgfqpoint{1.574589in}{1.447702in}}%
\pgfpathcurveto{\pgfqpoint{1.582826in}{1.447702in}}{\pgfqpoint{1.590726in}{1.450974in}}{\pgfqpoint{1.596550in}{1.456798in}}%
\pgfpathcurveto{\pgfqpoint{1.602374in}{1.462622in}}{\pgfqpoint{1.605646in}{1.470522in}}{\pgfqpoint{1.605646in}{1.478758in}}%
\pgfpathcurveto{\pgfqpoint{1.605646in}{1.486995in}}{\pgfqpoint{1.602374in}{1.494895in}}{\pgfqpoint{1.596550in}{1.500719in}}%
\pgfpathcurveto{\pgfqpoint{1.590726in}{1.506543in}}{\pgfqpoint{1.582826in}{1.509815in}}{\pgfqpoint{1.574589in}{1.509815in}}%
\pgfpathcurveto{\pgfqpoint{1.566353in}{1.509815in}}{\pgfqpoint{1.558453in}{1.506543in}}{\pgfqpoint{1.552629in}{1.500719in}}%
\pgfpathcurveto{\pgfqpoint{1.546805in}{1.494895in}}{\pgfqpoint{1.543533in}{1.486995in}}{\pgfqpoint{1.543533in}{1.478758in}}%
\pgfpathcurveto{\pgfqpoint{1.543533in}{1.470522in}}{\pgfqpoint{1.546805in}{1.462622in}}{\pgfqpoint{1.552629in}{1.456798in}}%
\pgfpathcurveto{\pgfqpoint{1.558453in}{1.450974in}}{\pgfqpoint{1.566353in}{1.447702in}}{\pgfqpoint{1.574589in}{1.447702in}}%
\pgfpathclose%
\pgfusepath{stroke,fill}%
\end{pgfscope}%
\begin{pgfscope}%
\pgfpathrectangle{\pgfqpoint{0.100000in}{0.212622in}}{\pgfqpoint{3.696000in}{3.696000in}}%
\pgfusepath{clip}%
\pgfsetbuttcap%
\pgfsetroundjoin%
\definecolor{currentfill}{rgb}{0.121569,0.466667,0.705882}%
\pgfsetfillcolor{currentfill}%
\pgfsetfillopacity{0.497958}%
\pgfsetlinewidth{1.003750pt}%
\definecolor{currentstroke}{rgb}{0.121569,0.466667,0.705882}%
\pgfsetstrokecolor{currentstroke}%
\pgfsetstrokeopacity{0.497958}%
\pgfsetdash{}{0pt}%
\pgfpathmoveto{\pgfqpoint{1.576823in}{1.446979in}}%
\pgfpathcurveto{\pgfqpoint{1.585059in}{1.446979in}}{\pgfqpoint{1.592959in}{1.450251in}}{\pgfqpoint{1.598783in}{1.456075in}}%
\pgfpathcurveto{\pgfqpoint{1.604607in}{1.461899in}}{\pgfqpoint{1.607879in}{1.469799in}}{\pgfqpoint{1.607879in}{1.478036in}}%
\pgfpathcurveto{\pgfqpoint{1.607879in}{1.486272in}}{\pgfqpoint{1.604607in}{1.494172in}}{\pgfqpoint{1.598783in}{1.499996in}}%
\pgfpathcurveto{\pgfqpoint{1.592959in}{1.505820in}}{\pgfqpoint{1.585059in}{1.509092in}}{\pgfqpoint{1.576823in}{1.509092in}}%
\pgfpathcurveto{\pgfqpoint{1.568586in}{1.509092in}}{\pgfqpoint{1.560686in}{1.505820in}}{\pgfqpoint{1.554862in}{1.499996in}}%
\pgfpathcurveto{\pgfqpoint{1.549039in}{1.494172in}}{\pgfqpoint{1.545766in}{1.486272in}}{\pgfqpoint{1.545766in}{1.478036in}}%
\pgfpathcurveto{\pgfqpoint{1.545766in}{1.469799in}}{\pgfqpoint{1.549039in}{1.461899in}}{\pgfqpoint{1.554862in}{1.456075in}}%
\pgfpathcurveto{\pgfqpoint{1.560686in}{1.450251in}}{\pgfqpoint{1.568586in}{1.446979in}}{\pgfqpoint{1.576823in}{1.446979in}}%
\pgfpathclose%
\pgfusepath{stroke,fill}%
\end{pgfscope}%
\begin{pgfscope}%
\pgfpathrectangle{\pgfqpoint{0.100000in}{0.212622in}}{\pgfqpoint{3.696000in}{3.696000in}}%
\pgfusepath{clip}%
\pgfsetbuttcap%
\pgfsetroundjoin%
\definecolor{currentfill}{rgb}{0.121569,0.466667,0.705882}%
\pgfsetfillcolor{currentfill}%
\pgfsetfillopacity{0.499213}%
\pgfsetlinewidth{1.003750pt}%
\definecolor{currentstroke}{rgb}{0.121569,0.466667,0.705882}%
\pgfsetstrokecolor{currentstroke}%
\pgfsetstrokeopacity{0.499213}%
\pgfsetdash{}{0pt}%
\pgfpathmoveto{\pgfqpoint{1.579793in}{1.445922in}}%
\pgfpathcurveto{\pgfqpoint{1.588030in}{1.445922in}}{\pgfqpoint{1.595930in}{1.449195in}}{\pgfqpoint{1.601754in}{1.455018in}}%
\pgfpathcurveto{\pgfqpoint{1.607578in}{1.460842in}}{\pgfqpoint{1.610850in}{1.468742in}}{\pgfqpoint{1.610850in}{1.476979in}}%
\pgfpathcurveto{\pgfqpoint{1.610850in}{1.485215in}}{\pgfqpoint{1.607578in}{1.493115in}}{\pgfqpoint{1.601754in}{1.498939in}}%
\pgfpathcurveto{\pgfqpoint{1.595930in}{1.504763in}}{\pgfqpoint{1.588030in}{1.508035in}}{\pgfqpoint{1.579793in}{1.508035in}}%
\pgfpathcurveto{\pgfqpoint{1.571557in}{1.508035in}}{\pgfqpoint{1.563657in}{1.504763in}}{\pgfqpoint{1.557833in}{1.498939in}}%
\pgfpathcurveto{\pgfqpoint{1.552009in}{1.493115in}}{\pgfqpoint{1.548737in}{1.485215in}}{\pgfqpoint{1.548737in}{1.476979in}}%
\pgfpathcurveto{\pgfqpoint{1.548737in}{1.468742in}}{\pgfqpoint{1.552009in}{1.460842in}}{\pgfqpoint{1.557833in}{1.455018in}}%
\pgfpathcurveto{\pgfqpoint{1.563657in}{1.449195in}}{\pgfqpoint{1.571557in}{1.445922in}}{\pgfqpoint{1.579793in}{1.445922in}}%
\pgfpathclose%
\pgfusepath{stroke,fill}%
\end{pgfscope}%
\begin{pgfscope}%
\pgfpathrectangle{\pgfqpoint{0.100000in}{0.212622in}}{\pgfqpoint{3.696000in}{3.696000in}}%
\pgfusepath{clip}%
\pgfsetbuttcap%
\pgfsetroundjoin%
\definecolor{currentfill}{rgb}{0.121569,0.466667,0.705882}%
\pgfsetfillcolor{currentfill}%
\pgfsetfillopacity{0.501160}%
\pgfsetlinewidth{1.003750pt}%
\definecolor{currentstroke}{rgb}{0.121569,0.466667,0.705882}%
\pgfsetstrokecolor{currentstroke}%
\pgfsetstrokeopacity{0.501160}%
\pgfsetdash{}{0pt}%
\pgfpathmoveto{\pgfqpoint{1.583599in}{1.444282in}}%
\pgfpathcurveto{\pgfqpoint{1.591836in}{1.444282in}}{\pgfqpoint{1.599736in}{1.447555in}}{\pgfqpoint{1.605560in}{1.453378in}}%
\pgfpathcurveto{\pgfqpoint{1.611384in}{1.459202in}}{\pgfqpoint{1.614656in}{1.467102in}}{\pgfqpoint{1.614656in}{1.475339in}}%
\pgfpathcurveto{\pgfqpoint{1.614656in}{1.483575in}}{\pgfqpoint{1.611384in}{1.491475in}}{\pgfqpoint{1.605560in}{1.497299in}}%
\pgfpathcurveto{\pgfqpoint{1.599736in}{1.503123in}}{\pgfqpoint{1.591836in}{1.506395in}}{\pgfqpoint{1.583599in}{1.506395in}}%
\pgfpathcurveto{\pgfqpoint{1.575363in}{1.506395in}}{\pgfqpoint{1.567463in}{1.503123in}}{\pgfqpoint{1.561639in}{1.497299in}}%
\pgfpathcurveto{\pgfqpoint{1.555815in}{1.491475in}}{\pgfqpoint{1.552543in}{1.483575in}}{\pgfqpoint{1.552543in}{1.475339in}}%
\pgfpathcurveto{\pgfqpoint{1.552543in}{1.467102in}}{\pgfqpoint{1.555815in}{1.459202in}}{\pgfqpoint{1.561639in}{1.453378in}}%
\pgfpathcurveto{\pgfqpoint{1.567463in}{1.447555in}}{\pgfqpoint{1.575363in}{1.444282in}}{\pgfqpoint{1.583599in}{1.444282in}}%
\pgfpathclose%
\pgfusepath{stroke,fill}%
\end{pgfscope}%
\begin{pgfscope}%
\pgfpathrectangle{\pgfqpoint{0.100000in}{0.212622in}}{\pgfqpoint{3.696000in}{3.696000in}}%
\pgfusepath{clip}%
\pgfsetbuttcap%
\pgfsetroundjoin%
\definecolor{currentfill}{rgb}{0.121569,0.466667,0.705882}%
\pgfsetfillcolor{currentfill}%
\pgfsetfillopacity{0.503008}%
\pgfsetlinewidth{1.003750pt}%
\definecolor{currentstroke}{rgb}{0.121569,0.466667,0.705882}%
\pgfsetstrokecolor{currentstroke}%
\pgfsetstrokeopacity{0.503008}%
\pgfsetdash{}{0pt}%
\pgfpathmoveto{\pgfqpoint{1.588226in}{1.442582in}}%
\pgfpathcurveto{\pgfqpoint{1.596462in}{1.442582in}}{\pgfqpoint{1.604362in}{1.445854in}}{\pgfqpoint{1.610186in}{1.451678in}}%
\pgfpathcurveto{\pgfqpoint{1.616010in}{1.457502in}}{\pgfqpoint{1.619283in}{1.465402in}}{\pgfqpoint{1.619283in}{1.473638in}}%
\pgfpathcurveto{\pgfqpoint{1.619283in}{1.481875in}}{\pgfqpoint{1.616010in}{1.489775in}}{\pgfqpoint{1.610186in}{1.495599in}}%
\pgfpathcurveto{\pgfqpoint{1.604362in}{1.501422in}}{\pgfqpoint{1.596462in}{1.504695in}}{\pgfqpoint{1.588226in}{1.504695in}}%
\pgfpathcurveto{\pgfqpoint{1.579990in}{1.504695in}}{\pgfqpoint{1.572090in}{1.501422in}}{\pgfqpoint{1.566266in}{1.495599in}}%
\pgfpathcurveto{\pgfqpoint{1.560442in}{1.489775in}}{\pgfqpoint{1.557170in}{1.481875in}}{\pgfqpoint{1.557170in}{1.473638in}}%
\pgfpathcurveto{\pgfqpoint{1.557170in}{1.465402in}}{\pgfqpoint{1.560442in}{1.457502in}}{\pgfqpoint{1.566266in}{1.451678in}}%
\pgfpathcurveto{\pgfqpoint{1.572090in}{1.445854in}}{\pgfqpoint{1.579990in}{1.442582in}}{\pgfqpoint{1.588226in}{1.442582in}}%
\pgfpathclose%
\pgfusepath{stroke,fill}%
\end{pgfscope}%
\begin{pgfscope}%
\pgfpathrectangle{\pgfqpoint{0.100000in}{0.212622in}}{\pgfqpoint{3.696000in}{3.696000in}}%
\pgfusepath{clip}%
\pgfsetbuttcap%
\pgfsetroundjoin%
\definecolor{currentfill}{rgb}{0.121569,0.466667,0.705882}%
\pgfsetfillcolor{currentfill}%
\pgfsetfillopacity{0.504013}%
\pgfsetlinewidth{1.003750pt}%
\definecolor{currentstroke}{rgb}{0.121569,0.466667,0.705882}%
\pgfsetstrokecolor{currentstroke}%
\pgfsetstrokeopacity{0.504013}%
\pgfsetdash{}{0pt}%
\pgfpathmoveto{\pgfqpoint{1.590804in}{1.441734in}}%
\pgfpathcurveto{\pgfqpoint{1.599040in}{1.441734in}}{\pgfqpoint{1.606940in}{1.445006in}}{\pgfqpoint{1.612764in}{1.450830in}}%
\pgfpathcurveto{\pgfqpoint{1.618588in}{1.456654in}}{\pgfqpoint{1.621860in}{1.464554in}}{\pgfqpoint{1.621860in}{1.472790in}}%
\pgfpathcurveto{\pgfqpoint{1.621860in}{1.481027in}}{\pgfqpoint{1.618588in}{1.488927in}}{\pgfqpoint{1.612764in}{1.494750in}}%
\pgfpathcurveto{\pgfqpoint{1.606940in}{1.500574in}}{\pgfqpoint{1.599040in}{1.503847in}}{\pgfqpoint{1.590804in}{1.503847in}}%
\pgfpathcurveto{\pgfqpoint{1.582568in}{1.503847in}}{\pgfqpoint{1.574668in}{1.500574in}}{\pgfqpoint{1.568844in}{1.494750in}}%
\pgfpathcurveto{\pgfqpoint{1.563020in}{1.488927in}}{\pgfqpoint{1.559747in}{1.481027in}}{\pgfqpoint{1.559747in}{1.472790in}}%
\pgfpathcurveto{\pgfqpoint{1.559747in}{1.464554in}}{\pgfqpoint{1.563020in}{1.456654in}}{\pgfqpoint{1.568844in}{1.450830in}}%
\pgfpathcurveto{\pgfqpoint{1.574668in}{1.445006in}}{\pgfqpoint{1.582568in}{1.441734in}}{\pgfqpoint{1.590804in}{1.441734in}}%
\pgfpathclose%
\pgfusepath{stroke,fill}%
\end{pgfscope}%
\begin{pgfscope}%
\pgfpathrectangle{\pgfqpoint{0.100000in}{0.212622in}}{\pgfqpoint{3.696000in}{3.696000in}}%
\pgfusepath{clip}%
\pgfsetbuttcap%
\pgfsetroundjoin%
\definecolor{currentfill}{rgb}{0.121569,0.466667,0.705882}%
\pgfsetfillcolor{currentfill}%
\pgfsetfillopacity{0.504564}%
\pgfsetlinewidth{1.003750pt}%
\definecolor{currentstroke}{rgb}{0.121569,0.466667,0.705882}%
\pgfsetstrokecolor{currentstroke}%
\pgfsetstrokeopacity{0.504564}%
\pgfsetdash{}{0pt}%
\pgfpathmoveto{\pgfqpoint{1.592233in}{1.441303in}}%
\pgfpathcurveto{\pgfqpoint{1.600469in}{1.441303in}}{\pgfqpoint{1.608369in}{1.444575in}}{\pgfqpoint{1.614193in}{1.450399in}}%
\pgfpathcurveto{\pgfqpoint{1.620017in}{1.456223in}}{\pgfqpoint{1.623290in}{1.464123in}}{\pgfqpoint{1.623290in}{1.472360in}}%
\pgfpathcurveto{\pgfqpoint{1.623290in}{1.480596in}}{\pgfqpoint{1.620017in}{1.488496in}}{\pgfqpoint{1.614193in}{1.494320in}}%
\pgfpathcurveto{\pgfqpoint{1.608369in}{1.500144in}}{\pgfqpoint{1.600469in}{1.503416in}}{\pgfqpoint{1.592233in}{1.503416in}}%
\pgfpathcurveto{\pgfqpoint{1.583997in}{1.503416in}}{\pgfqpoint{1.576097in}{1.500144in}}{\pgfqpoint{1.570273in}{1.494320in}}%
\pgfpathcurveto{\pgfqpoint{1.564449in}{1.488496in}}{\pgfqpoint{1.561177in}{1.480596in}}{\pgfqpoint{1.561177in}{1.472360in}}%
\pgfpathcurveto{\pgfqpoint{1.561177in}{1.464123in}}{\pgfqpoint{1.564449in}{1.456223in}}{\pgfqpoint{1.570273in}{1.450399in}}%
\pgfpathcurveto{\pgfqpoint{1.576097in}{1.444575in}}{\pgfqpoint{1.583997in}{1.441303in}}{\pgfqpoint{1.592233in}{1.441303in}}%
\pgfpathclose%
\pgfusepath{stroke,fill}%
\end{pgfscope}%
\begin{pgfscope}%
\pgfpathrectangle{\pgfqpoint{0.100000in}{0.212622in}}{\pgfqpoint{3.696000in}{3.696000in}}%
\pgfusepath{clip}%
\pgfsetbuttcap%
\pgfsetroundjoin%
\definecolor{currentfill}{rgb}{0.121569,0.466667,0.705882}%
\pgfsetfillcolor{currentfill}%
\pgfsetfillopacity{0.505965}%
\pgfsetlinewidth{1.003750pt}%
\definecolor{currentstroke}{rgb}{0.121569,0.466667,0.705882}%
\pgfsetstrokecolor{currentstroke}%
\pgfsetstrokeopacity{0.505965}%
\pgfsetdash{}{0pt}%
\pgfpathmoveto{\pgfqpoint{1.595873in}{1.440129in}}%
\pgfpathcurveto{\pgfqpoint{1.604109in}{1.440129in}}{\pgfqpoint{1.612009in}{1.443402in}}{\pgfqpoint{1.617833in}{1.449226in}}%
\pgfpathcurveto{\pgfqpoint{1.623657in}{1.455049in}}{\pgfqpoint{1.626929in}{1.462950in}}{\pgfqpoint{1.626929in}{1.471186in}}%
\pgfpathcurveto{\pgfqpoint{1.626929in}{1.479422in}}{\pgfqpoint{1.623657in}{1.487322in}}{\pgfqpoint{1.617833in}{1.493146in}}%
\pgfpathcurveto{\pgfqpoint{1.612009in}{1.498970in}}{\pgfqpoint{1.604109in}{1.502242in}}{\pgfqpoint{1.595873in}{1.502242in}}%
\pgfpathcurveto{\pgfqpoint{1.587637in}{1.502242in}}{\pgfqpoint{1.579737in}{1.498970in}}{\pgfqpoint{1.573913in}{1.493146in}}%
\pgfpathcurveto{\pgfqpoint{1.568089in}{1.487322in}}{\pgfqpoint{1.564816in}{1.479422in}}{\pgfqpoint{1.564816in}{1.471186in}}%
\pgfpathcurveto{\pgfqpoint{1.564816in}{1.462950in}}{\pgfqpoint{1.568089in}{1.455049in}}{\pgfqpoint{1.573913in}{1.449226in}}%
\pgfpathcurveto{\pgfqpoint{1.579737in}{1.443402in}}{\pgfqpoint{1.587637in}{1.440129in}}{\pgfqpoint{1.595873in}{1.440129in}}%
\pgfpathclose%
\pgfusepath{stroke,fill}%
\end{pgfscope}%
\begin{pgfscope}%
\pgfpathrectangle{\pgfqpoint{0.100000in}{0.212622in}}{\pgfqpoint{3.696000in}{3.696000in}}%
\pgfusepath{clip}%
\pgfsetbuttcap%
\pgfsetroundjoin%
\definecolor{currentfill}{rgb}{0.121569,0.466667,0.705882}%
\pgfsetfillcolor{currentfill}%
\pgfsetfillopacity{0.507778}%
\pgfsetlinewidth{1.003750pt}%
\definecolor{currentstroke}{rgb}{0.121569,0.466667,0.705882}%
\pgfsetstrokecolor{currentstroke}%
\pgfsetstrokeopacity{0.507778}%
\pgfsetdash{}{0pt}%
\pgfpathmoveto{\pgfqpoint{1.599915in}{1.438601in}}%
\pgfpathcurveto{\pgfqpoint{1.608151in}{1.438601in}}{\pgfqpoint{1.616052in}{1.441873in}}{\pgfqpoint{1.621875in}{1.447697in}}%
\pgfpathcurveto{\pgfqpoint{1.627699in}{1.453521in}}{\pgfqpoint{1.630972in}{1.461421in}}{\pgfqpoint{1.630972in}{1.469657in}}%
\pgfpathcurveto{\pgfqpoint{1.630972in}{1.477893in}}{\pgfqpoint{1.627699in}{1.485793in}}{\pgfqpoint{1.621875in}{1.491617in}}%
\pgfpathcurveto{\pgfqpoint{1.616052in}{1.497441in}}{\pgfqpoint{1.608151in}{1.500714in}}{\pgfqpoint{1.599915in}{1.500714in}}%
\pgfpathcurveto{\pgfqpoint{1.591679in}{1.500714in}}{\pgfqpoint{1.583779in}{1.497441in}}{\pgfqpoint{1.577955in}{1.491617in}}%
\pgfpathcurveto{\pgfqpoint{1.572131in}{1.485793in}}{\pgfqpoint{1.568859in}{1.477893in}}{\pgfqpoint{1.568859in}{1.469657in}}%
\pgfpathcurveto{\pgfqpoint{1.568859in}{1.461421in}}{\pgfqpoint{1.572131in}{1.453521in}}{\pgfqpoint{1.577955in}{1.447697in}}%
\pgfpathcurveto{\pgfqpoint{1.583779in}{1.441873in}}{\pgfqpoint{1.591679in}{1.438601in}}{\pgfqpoint{1.599915in}{1.438601in}}%
\pgfpathclose%
\pgfusepath{stroke,fill}%
\end{pgfscope}%
\begin{pgfscope}%
\pgfpathrectangle{\pgfqpoint{0.100000in}{0.212622in}}{\pgfqpoint{3.696000in}{3.696000in}}%
\pgfusepath{clip}%
\pgfsetbuttcap%
\pgfsetroundjoin%
\definecolor{currentfill}{rgb}{0.121569,0.466667,0.705882}%
\pgfsetfillcolor{currentfill}%
\pgfsetfillopacity{0.508788}%
\pgfsetlinewidth{1.003750pt}%
\definecolor{currentstroke}{rgb}{0.121569,0.466667,0.705882}%
\pgfsetstrokecolor{currentstroke}%
\pgfsetstrokeopacity{0.508788}%
\pgfsetdash{}{0pt}%
\pgfpathmoveto{\pgfqpoint{1.602125in}{1.437743in}}%
\pgfpathcurveto{\pgfqpoint{1.610362in}{1.437743in}}{\pgfqpoint{1.618262in}{1.441015in}}{\pgfqpoint{1.624085in}{1.446839in}}%
\pgfpathcurveto{\pgfqpoint{1.629909in}{1.452663in}}{\pgfqpoint{1.633182in}{1.460563in}}{\pgfqpoint{1.633182in}{1.468799in}}%
\pgfpathcurveto{\pgfqpoint{1.633182in}{1.477036in}}{\pgfqpoint{1.629909in}{1.484936in}}{\pgfqpoint{1.624085in}{1.490760in}}%
\pgfpathcurveto{\pgfqpoint{1.618262in}{1.496583in}}{\pgfqpoint{1.610362in}{1.499856in}}{\pgfqpoint{1.602125in}{1.499856in}}%
\pgfpathcurveto{\pgfqpoint{1.593889in}{1.499856in}}{\pgfqpoint{1.585989in}{1.496583in}}{\pgfqpoint{1.580165in}{1.490760in}}%
\pgfpathcurveto{\pgfqpoint{1.574341in}{1.484936in}}{\pgfqpoint{1.571069in}{1.477036in}}{\pgfqpoint{1.571069in}{1.468799in}}%
\pgfpathcurveto{\pgfqpoint{1.571069in}{1.460563in}}{\pgfqpoint{1.574341in}{1.452663in}}{\pgfqpoint{1.580165in}{1.446839in}}%
\pgfpathcurveto{\pgfqpoint{1.585989in}{1.441015in}}{\pgfqpoint{1.593889in}{1.437743in}}{\pgfqpoint{1.602125in}{1.437743in}}%
\pgfpathclose%
\pgfusepath{stroke,fill}%
\end{pgfscope}%
\begin{pgfscope}%
\pgfpathrectangle{\pgfqpoint{0.100000in}{0.212622in}}{\pgfqpoint{3.696000in}{3.696000in}}%
\pgfusepath{clip}%
\pgfsetbuttcap%
\pgfsetroundjoin%
\definecolor{currentfill}{rgb}{0.121569,0.466667,0.705882}%
\pgfsetfillcolor{currentfill}%
\pgfsetfillopacity{0.509994}%
\pgfsetlinewidth{1.003750pt}%
\definecolor{currentstroke}{rgb}{0.121569,0.466667,0.705882}%
\pgfsetstrokecolor{currentstroke}%
\pgfsetstrokeopacity{0.509994}%
\pgfsetdash{}{0pt}%
\pgfpathmoveto{\pgfqpoint{1.604872in}{1.436744in}}%
\pgfpathcurveto{\pgfqpoint{1.613109in}{1.436744in}}{\pgfqpoint{1.621009in}{1.440016in}}{\pgfqpoint{1.626833in}{1.445840in}}%
\pgfpathcurveto{\pgfqpoint{1.632656in}{1.451664in}}{\pgfqpoint{1.635929in}{1.459564in}}{\pgfqpoint{1.635929in}{1.467800in}}%
\pgfpathcurveto{\pgfqpoint{1.635929in}{1.476037in}}{\pgfqpoint{1.632656in}{1.483937in}}{\pgfqpoint{1.626833in}{1.489761in}}%
\pgfpathcurveto{\pgfqpoint{1.621009in}{1.495585in}}{\pgfqpoint{1.613109in}{1.498857in}}{\pgfqpoint{1.604872in}{1.498857in}}%
\pgfpathcurveto{\pgfqpoint{1.596636in}{1.498857in}}{\pgfqpoint{1.588736in}{1.495585in}}{\pgfqpoint{1.582912in}{1.489761in}}%
\pgfpathcurveto{\pgfqpoint{1.577088in}{1.483937in}}{\pgfqpoint{1.573816in}{1.476037in}}{\pgfqpoint{1.573816in}{1.467800in}}%
\pgfpathcurveto{\pgfqpoint{1.573816in}{1.459564in}}{\pgfqpoint{1.577088in}{1.451664in}}{\pgfqpoint{1.582912in}{1.445840in}}%
\pgfpathcurveto{\pgfqpoint{1.588736in}{1.440016in}}{\pgfqpoint{1.596636in}{1.436744in}}{\pgfqpoint{1.604872in}{1.436744in}}%
\pgfpathclose%
\pgfusepath{stroke,fill}%
\end{pgfscope}%
\begin{pgfscope}%
\pgfpathrectangle{\pgfqpoint{0.100000in}{0.212622in}}{\pgfqpoint{3.696000in}{3.696000in}}%
\pgfusepath{clip}%
\pgfsetbuttcap%
\pgfsetroundjoin%
\definecolor{currentfill}{rgb}{0.121569,0.466667,0.705882}%
\pgfsetfillcolor{currentfill}%
\pgfsetfillopacity{0.511927}%
\pgfsetlinewidth{1.003750pt}%
\definecolor{currentstroke}{rgb}{0.121569,0.466667,0.705882}%
\pgfsetstrokecolor{currentstroke}%
\pgfsetstrokeopacity{0.511927}%
\pgfsetdash{}{0pt}%
\pgfpathmoveto{\pgfqpoint{1.608752in}{1.435197in}}%
\pgfpathcurveto{\pgfqpoint{1.616988in}{1.435197in}}{\pgfqpoint{1.624888in}{1.438470in}}{\pgfqpoint{1.630712in}{1.444294in}}%
\pgfpathcurveto{\pgfqpoint{1.636536in}{1.450117in}}{\pgfqpoint{1.639808in}{1.458017in}}{\pgfqpoint{1.639808in}{1.466254in}}%
\pgfpathcurveto{\pgfqpoint{1.639808in}{1.474490in}}{\pgfqpoint{1.636536in}{1.482390in}}{\pgfqpoint{1.630712in}{1.488214in}}%
\pgfpathcurveto{\pgfqpoint{1.624888in}{1.494038in}}{\pgfqpoint{1.616988in}{1.497310in}}{\pgfqpoint{1.608752in}{1.497310in}}%
\pgfpathcurveto{\pgfqpoint{1.600515in}{1.497310in}}{\pgfqpoint{1.592615in}{1.494038in}}{\pgfqpoint{1.586791in}{1.488214in}}%
\pgfpathcurveto{\pgfqpoint{1.580967in}{1.482390in}}{\pgfqpoint{1.577695in}{1.474490in}}{\pgfqpoint{1.577695in}{1.466254in}}%
\pgfpathcurveto{\pgfqpoint{1.577695in}{1.458017in}}{\pgfqpoint{1.580967in}{1.450117in}}{\pgfqpoint{1.586791in}{1.444294in}}%
\pgfpathcurveto{\pgfqpoint{1.592615in}{1.438470in}}{\pgfqpoint{1.600515in}{1.435197in}}{\pgfqpoint{1.608752in}{1.435197in}}%
\pgfpathclose%
\pgfusepath{stroke,fill}%
\end{pgfscope}%
\begin{pgfscope}%
\pgfpathrectangle{\pgfqpoint{0.100000in}{0.212622in}}{\pgfqpoint{3.696000in}{3.696000in}}%
\pgfusepath{clip}%
\pgfsetbuttcap%
\pgfsetroundjoin%
\definecolor{currentfill}{rgb}{0.121569,0.466667,0.705882}%
\pgfsetfillcolor{currentfill}%
\pgfsetfillopacity{0.514078}%
\pgfsetlinewidth{1.003750pt}%
\definecolor{currentstroke}{rgb}{0.121569,0.466667,0.705882}%
\pgfsetstrokecolor{currentstroke}%
\pgfsetstrokeopacity{0.514078}%
\pgfsetdash{}{0pt}%
\pgfpathmoveto{\pgfqpoint{1.613789in}{1.433172in}}%
\pgfpathcurveto{\pgfqpoint{1.622025in}{1.433172in}}{\pgfqpoint{1.629925in}{1.436444in}}{\pgfqpoint{1.635749in}{1.442268in}}%
\pgfpathcurveto{\pgfqpoint{1.641573in}{1.448092in}}{\pgfqpoint{1.644845in}{1.455992in}}{\pgfqpoint{1.644845in}{1.464229in}}%
\pgfpathcurveto{\pgfqpoint{1.644845in}{1.472465in}}{\pgfqpoint{1.641573in}{1.480365in}}{\pgfqpoint{1.635749in}{1.486189in}}%
\pgfpathcurveto{\pgfqpoint{1.629925in}{1.492013in}}{\pgfqpoint{1.622025in}{1.495285in}}{\pgfqpoint{1.613789in}{1.495285in}}%
\pgfpathcurveto{\pgfqpoint{1.605552in}{1.495285in}}{\pgfqpoint{1.597652in}{1.492013in}}{\pgfqpoint{1.591828in}{1.486189in}}%
\pgfpathcurveto{\pgfqpoint{1.586004in}{1.480365in}}{\pgfqpoint{1.582732in}{1.472465in}}{\pgfqpoint{1.582732in}{1.464229in}}%
\pgfpathcurveto{\pgfqpoint{1.582732in}{1.455992in}}{\pgfqpoint{1.586004in}{1.448092in}}{\pgfqpoint{1.591828in}{1.442268in}}%
\pgfpathcurveto{\pgfqpoint{1.597652in}{1.436444in}}{\pgfqpoint{1.605552in}{1.433172in}}{\pgfqpoint{1.613789in}{1.433172in}}%
\pgfpathclose%
\pgfusepath{stroke,fill}%
\end{pgfscope}%
\begin{pgfscope}%
\pgfpathrectangle{\pgfqpoint{0.100000in}{0.212622in}}{\pgfqpoint{3.696000in}{3.696000in}}%
\pgfusepath{clip}%
\pgfsetbuttcap%
\pgfsetroundjoin%
\definecolor{currentfill}{rgb}{0.121569,0.466667,0.705882}%
\pgfsetfillcolor{currentfill}%
\pgfsetfillopacity{0.515341}%
\pgfsetlinewidth{1.003750pt}%
\definecolor{currentstroke}{rgb}{0.121569,0.466667,0.705882}%
\pgfsetstrokecolor{currentstroke}%
\pgfsetstrokeopacity{0.515341}%
\pgfsetdash{}{0pt}%
\pgfpathmoveto{\pgfqpoint{1.616505in}{1.432055in}}%
\pgfpathcurveto{\pgfqpoint{1.624742in}{1.432055in}}{\pgfqpoint{1.632642in}{1.435327in}}{\pgfqpoint{1.638466in}{1.441151in}}%
\pgfpathcurveto{\pgfqpoint{1.644290in}{1.446975in}}{\pgfqpoint{1.647562in}{1.454875in}}{\pgfqpoint{1.647562in}{1.463111in}}%
\pgfpathcurveto{\pgfqpoint{1.647562in}{1.471347in}}{\pgfqpoint{1.644290in}{1.479247in}}{\pgfqpoint{1.638466in}{1.485071in}}%
\pgfpathcurveto{\pgfqpoint{1.632642in}{1.490895in}}{\pgfqpoint{1.624742in}{1.494168in}}{\pgfqpoint{1.616505in}{1.494168in}}%
\pgfpathcurveto{\pgfqpoint{1.608269in}{1.494168in}}{\pgfqpoint{1.600369in}{1.490895in}}{\pgfqpoint{1.594545in}{1.485071in}}%
\pgfpathcurveto{\pgfqpoint{1.588721in}{1.479247in}}{\pgfqpoint{1.585449in}{1.471347in}}{\pgfqpoint{1.585449in}{1.463111in}}%
\pgfpathcurveto{\pgfqpoint{1.585449in}{1.454875in}}{\pgfqpoint{1.588721in}{1.446975in}}{\pgfqpoint{1.594545in}{1.441151in}}%
\pgfpathcurveto{\pgfqpoint{1.600369in}{1.435327in}}{\pgfqpoint{1.608269in}{1.432055in}}{\pgfqpoint{1.616505in}{1.432055in}}%
\pgfpathclose%
\pgfusepath{stroke,fill}%
\end{pgfscope}%
\begin{pgfscope}%
\pgfpathrectangle{\pgfqpoint{0.100000in}{0.212622in}}{\pgfqpoint{3.696000in}{3.696000in}}%
\pgfusepath{clip}%
\pgfsetbuttcap%
\pgfsetroundjoin%
\definecolor{currentfill}{rgb}{0.121569,0.466667,0.705882}%
\pgfsetfillcolor{currentfill}%
\pgfsetfillopacity{0.515951}%
\pgfsetlinewidth{1.003750pt}%
\definecolor{currentstroke}{rgb}{0.121569,0.466667,0.705882}%
\pgfsetstrokecolor{currentstroke}%
\pgfsetstrokeopacity{0.515951}%
\pgfsetdash{}{0pt}%
\pgfpathmoveto{\pgfqpoint{1.618090in}{1.431557in}}%
\pgfpathcurveto{\pgfqpoint{1.626326in}{1.431557in}}{\pgfqpoint{1.634226in}{1.434829in}}{\pgfqpoint{1.640050in}{1.440653in}}%
\pgfpathcurveto{\pgfqpoint{1.645874in}{1.446477in}}{\pgfqpoint{1.649146in}{1.454377in}}{\pgfqpoint{1.649146in}{1.462613in}}%
\pgfpathcurveto{\pgfqpoint{1.649146in}{1.470850in}}{\pgfqpoint{1.645874in}{1.478750in}}{\pgfqpoint{1.640050in}{1.484574in}}%
\pgfpathcurveto{\pgfqpoint{1.634226in}{1.490398in}}{\pgfqpoint{1.626326in}{1.493670in}}{\pgfqpoint{1.618090in}{1.493670in}}%
\pgfpathcurveto{\pgfqpoint{1.609854in}{1.493670in}}{\pgfqpoint{1.601954in}{1.490398in}}{\pgfqpoint{1.596130in}{1.484574in}}%
\pgfpathcurveto{\pgfqpoint{1.590306in}{1.478750in}}{\pgfqpoint{1.587033in}{1.470850in}}{\pgfqpoint{1.587033in}{1.462613in}}%
\pgfpathcurveto{\pgfqpoint{1.587033in}{1.454377in}}{\pgfqpoint{1.590306in}{1.446477in}}{\pgfqpoint{1.596130in}{1.440653in}}%
\pgfpathcurveto{\pgfqpoint{1.601954in}{1.434829in}}{\pgfqpoint{1.609854in}{1.431557in}}{\pgfqpoint{1.618090in}{1.431557in}}%
\pgfpathclose%
\pgfusepath{stroke,fill}%
\end{pgfscope}%
\begin{pgfscope}%
\pgfpathrectangle{\pgfqpoint{0.100000in}{0.212622in}}{\pgfqpoint{3.696000in}{3.696000in}}%
\pgfusepath{clip}%
\pgfsetbuttcap%
\pgfsetroundjoin%
\definecolor{currentfill}{rgb}{0.121569,0.466667,0.705882}%
\pgfsetfillcolor{currentfill}%
\pgfsetfillopacity{0.516964}%
\pgfsetlinewidth{1.003750pt}%
\definecolor{currentstroke}{rgb}{0.121569,0.466667,0.705882}%
\pgfsetstrokecolor{currentstroke}%
\pgfsetstrokeopacity{0.516964}%
\pgfsetdash{}{0pt}%
\pgfpathmoveto{\pgfqpoint{1.620843in}{1.430762in}}%
\pgfpathcurveto{\pgfqpoint{1.629080in}{1.430762in}}{\pgfqpoint{1.636980in}{1.434035in}}{\pgfqpoint{1.642804in}{1.439859in}}%
\pgfpathcurveto{\pgfqpoint{1.648627in}{1.445683in}}{\pgfqpoint{1.651900in}{1.453583in}}{\pgfqpoint{1.651900in}{1.461819in}}%
\pgfpathcurveto{\pgfqpoint{1.651900in}{1.470055in}}{\pgfqpoint{1.648627in}{1.477955in}}{\pgfqpoint{1.642804in}{1.483779in}}%
\pgfpathcurveto{\pgfqpoint{1.636980in}{1.489603in}}{\pgfqpoint{1.629080in}{1.492875in}}{\pgfqpoint{1.620843in}{1.492875in}}%
\pgfpathcurveto{\pgfqpoint{1.612607in}{1.492875in}}{\pgfqpoint{1.604707in}{1.489603in}}{\pgfqpoint{1.598883in}{1.483779in}}%
\pgfpathcurveto{\pgfqpoint{1.593059in}{1.477955in}}{\pgfqpoint{1.589787in}{1.470055in}}{\pgfqpoint{1.589787in}{1.461819in}}%
\pgfpathcurveto{\pgfqpoint{1.589787in}{1.453583in}}{\pgfqpoint{1.593059in}{1.445683in}}{\pgfqpoint{1.598883in}{1.439859in}}%
\pgfpathcurveto{\pgfqpoint{1.604707in}{1.434035in}}{\pgfqpoint{1.612607in}{1.430762in}}{\pgfqpoint{1.620843in}{1.430762in}}%
\pgfpathclose%
\pgfusepath{stroke,fill}%
\end{pgfscope}%
\begin{pgfscope}%
\pgfpathrectangle{\pgfqpoint{0.100000in}{0.212622in}}{\pgfqpoint{3.696000in}{3.696000in}}%
\pgfusepath{clip}%
\pgfsetbuttcap%
\pgfsetroundjoin%
\definecolor{currentfill}{rgb}{0.121569,0.466667,0.705882}%
\pgfsetfillcolor{currentfill}%
\pgfsetfillopacity{0.518589}%
\pgfsetlinewidth{1.003750pt}%
\definecolor{currentstroke}{rgb}{0.121569,0.466667,0.705882}%
\pgfsetstrokecolor{currentstroke}%
\pgfsetstrokeopacity{0.518589}%
\pgfsetdash{}{0pt}%
\pgfpathmoveto{\pgfqpoint{1.624912in}{1.429458in}}%
\pgfpathcurveto{\pgfqpoint{1.633148in}{1.429458in}}{\pgfqpoint{1.641048in}{1.432730in}}{\pgfqpoint{1.646872in}{1.438554in}}%
\pgfpathcurveto{\pgfqpoint{1.652696in}{1.444378in}}{\pgfqpoint{1.655968in}{1.452278in}}{\pgfqpoint{1.655968in}{1.460514in}}%
\pgfpathcurveto{\pgfqpoint{1.655968in}{1.468751in}}{\pgfqpoint{1.652696in}{1.476651in}}{\pgfqpoint{1.646872in}{1.482475in}}%
\pgfpathcurveto{\pgfqpoint{1.641048in}{1.488298in}}{\pgfqpoint{1.633148in}{1.491571in}}{\pgfqpoint{1.624912in}{1.491571in}}%
\pgfpathcurveto{\pgfqpoint{1.616675in}{1.491571in}}{\pgfqpoint{1.608775in}{1.488298in}}{\pgfqpoint{1.602951in}{1.482475in}}%
\pgfpathcurveto{\pgfqpoint{1.597127in}{1.476651in}}{\pgfqpoint{1.593855in}{1.468751in}}{\pgfqpoint{1.593855in}{1.460514in}}%
\pgfpathcurveto{\pgfqpoint{1.593855in}{1.452278in}}{\pgfqpoint{1.597127in}{1.444378in}}{\pgfqpoint{1.602951in}{1.438554in}}%
\pgfpathcurveto{\pgfqpoint{1.608775in}{1.432730in}}{\pgfqpoint{1.616675in}{1.429458in}}{\pgfqpoint{1.624912in}{1.429458in}}%
\pgfpathclose%
\pgfusepath{stroke,fill}%
\end{pgfscope}%
\begin{pgfscope}%
\pgfpathrectangle{\pgfqpoint{0.100000in}{0.212622in}}{\pgfqpoint{3.696000in}{3.696000in}}%
\pgfusepath{clip}%
\pgfsetbuttcap%
\pgfsetroundjoin%
\definecolor{currentfill}{rgb}{0.121569,0.466667,0.705882}%
\pgfsetfillcolor{currentfill}%
\pgfsetfillopacity{0.520762}%
\pgfsetlinewidth{1.003750pt}%
\definecolor{currentstroke}{rgb}{0.121569,0.466667,0.705882}%
\pgfsetstrokecolor{currentstroke}%
\pgfsetstrokeopacity{0.520762}%
\pgfsetdash{}{0pt}%
\pgfpathmoveto{\pgfqpoint{1.629496in}{1.427731in}}%
\pgfpathcurveto{\pgfqpoint{1.637732in}{1.427731in}}{\pgfqpoint{1.645632in}{1.431004in}}{\pgfqpoint{1.651456in}{1.436828in}}%
\pgfpathcurveto{\pgfqpoint{1.657280in}{1.442651in}}{\pgfqpoint{1.660552in}{1.450551in}}{\pgfqpoint{1.660552in}{1.458788in}}%
\pgfpathcurveto{\pgfqpoint{1.660552in}{1.467024in}}{\pgfqpoint{1.657280in}{1.474924in}}{\pgfqpoint{1.651456in}{1.480748in}}%
\pgfpathcurveto{\pgfqpoint{1.645632in}{1.486572in}}{\pgfqpoint{1.637732in}{1.489844in}}{\pgfqpoint{1.629496in}{1.489844in}}%
\pgfpathcurveto{\pgfqpoint{1.621260in}{1.489844in}}{\pgfqpoint{1.613360in}{1.486572in}}{\pgfqpoint{1.607536in}{1.480748in}}%
\pgfpathcurveto{\pgfqpoint{1.601712in}{1.474924in}}{\pgfqpoint{1.598439in}{1.467024in}}{\pgfqpoint{1.598439in}{1.458788in}}%
\pgfpathcurveto{\pgfqpoint{1.598439in}{1.450551in}}{\pgfqpoint{1.601712in}{1.442651in}}{\pgfqpoint{1.607536in}{1.436828in}}%
\pgfpathcurveto{\pgfqpoint{1.613360in}{1.431004in}}{\pgfqpoint{1.621260in}{1.427731in}}{\pgfqpoint{1.629496in}{1.427731in}}%
\pgfpathclose%
\pgfusepath{stroke,fill}%
\end{pgfscope}%
\begin{pgfscope}%
\pgfpathrectangle{\pgfqpoint{0.100000in}{0.212622in}}{\pgfqpoint{3.696000in}{3.696000in}}%
\pgfusepath{clip}%
\pgfsetbuttcap%
\pgfsetroundjoin%
\definecolor{currentfill}{rgb}{0.121569,0.466667,0.705882}%
\pgfsetfillcolor{currentfill}%
\pgfsetfillopacity{0.522030}%
\pgfsetlinewidth{1.003750pt}%
\definecolor{currentstroke}{rgb}{0.121569,0.466667,0.705882}%
\pgfsetstrokecolor{currentstroke}%
\pgfsetstrokeopacity{0.522030}%
\pgfsetdash{}{0pt}%
\pgfpathmoveto{\pgfqpoint{1.631937in}{1.426669in}}%
\pgfpathcurveto{\pgfqpoint{1.640173in}{1.426669in}}{\pgfqpoint{1.648073in}{1.429942in}}{\pgfqpoint{1.653897in}{1.435766in}}%
\pgfpathcurveto{\pgfqpoint{1.659721in}{1.441589in}}{\pgfqpoint{1.662994in}{1.449490in}}{\pgfqpoint{1.662994in}{1.457726in}}%
\pgfpathcurveto{\pgfqpoint{1.662994in}{1.465962in}}{\pgfqpoint{1.659721in}{1.473862in}}{\pgfqpoint{1.653897in}{1.479686in}}%
\pgfpathcurveto{\pgfqpoint{1.648073in}{1.485510in}}{\pgfqpoint{1.640173in}{1.488782in}}{\pgfqpoint{1.631937in}{1.488782in}}%
\pgfpathcurveto{\pgfqpoint{1.623701in}{1.488782in}}{\pgfqpoint{1.615801in}{1.485510in}}{\pgfqpoint{1.609977in}{1.479686in}}%
\pgfpathcurveto{\pgfqpoint{1.604153in}{1.473862in}}{\pgfqpoint{1.600881in}{1.465962in}}{\pgfqpoint{1.600881in}{1.457726in}}%
\pgfpathcurveto{\pgfqpoint{1.600881in}{1.449490in}}{\pgfqpoint{1.604153in}{1.441589in}}{\pgfqpoint{1.609977in}{1.435766in}}%
\pgfpathcurveto{\pgfqpoint{1.615801in}{1.429942in}}{\pgfqpoint{1.623701in}{1.426669in}}{\pgfqpoint{1.631937in}{1.426669in}}%
\pgfpathclose%
\pgfusepath{stroke,fill}%
\end{pgfscope}%
\begin{pgfscope}%
\pgfpathrectangle{\pgfqpoint{0.100000in}{0.212622in}}{\pgfqpoint{3.696000in}{3.696000in}}%
\pgfusepath{clip}%
\pgfsetbuttcap%
\pgfsetroundjoin%
\definecolor{currentfill}{rgb}{0.121569,0.466667,0.705882}%
\pgfsetfillcolor{currentfill}%
\pgfsetfillopacity{0.522618}%
\pgfsetlinewidth{1.003750pt}%
\definecolor{currentstroke}{rgb}{0.121569,0.466667,0.705882}%
\pgfsetstrokecolor{currentstroke}%
\pgfsetstrokeopacity{0.522618}%
\pgfsetdash{}{0pt}%
\pgfpathmoveto{\pgfqpoint{1.633394in}{1.426229in}}%
\pgfpathcurveto{\pgfqpoint{1.641630in}{1.426229in}}{\pgfqpoint{1.649530in}{1.429501in}}{\pgfqpoint{1.655354in}{1.435325in}}%
\pgfpathcurveto{\pgfqpoint{1.661178in}{1.441149in}}{\pgfqpoint{1.664450in}{1.449049in}}{\pgfqpoint{1.664450in}{1.457285in}}%
\pgfpathcurveto{\pgfqpoint{1.664450in}{1.465522in}}{\pgfqpoint{1.661178in}{1.473422in}}{\pgfqpoint{1.655354in}{1.479246in}}%
\pgfpathcurveto{\pgfqpoint{1.649530in}{1.485070in}}{\pgfqpoint{1.641630in}{1.488342in}}{\pgfqpoint{1.633394in}{1.488342in}}%
\pgfpathcurveto{\pgfqpoint{1.625157in}{1.488342in}}{\pgfqpoint{1.617257in}{1.485070in}}{\pgfqpoint{1.611433in}{1.479246in}}%
\pgfpathcurveto{\pgfqpoint{1.605609in}{1.473422in}}{\pgfqpoint{1.602337in}{1.465522in}}{\pgfqpoint{1.602337in}{1.457285in}}%
\pgfpathcurveto{\pgfqpoint{1.602337in}{1.449049in}}{\pgfqpoint{1.605609in}{1.441149in}}{\pgfqpoint{1.611433in}{1.435325in}}%
\pgfpathcurveto{\pgfqpoint{1.617257in}{1.429501in}}{\pgfqpoint{1.625157in}{1.426229in}}{\pgfqpoint{1.633394in}{1.426229in}}%
\pgfpathclose%
\pgfusepath{stroke,fill}%
\end{pgfscope}%
\begin{pgfscope}%
\pgfpathrectangle{\pgfqpoint{0.100000in}{0.212622in}}{\pgfqpoint{3.696000in}{3.696000in}}%
\pgfusepath{clip}%
\pgfsetbuttcap%
\pgfsetroundjoin%
\definecolor{currentfill}{rgb}{0.121569,0.466667,0.705882}%
\pgfsetfillcolor{currentfill}%
\pgfsetfillopacity{0.523715}%
\pgfsetlinewidth{1.003750pt}%
\definecolor{currentstroke}{rgb}{0.121569,0.466667,0.705882}%
\pgfsetstrokecolor{currentstroke}%
\pgfsetstrokeopacity{0.523715}%
\pgfsetdash{}{0pt}%
\pgfpathmoveto{\pgfqpoint{1.636205in}{1.425387in}}%
\pgfpathcurveto{\pgfqpoint{1.644441in}{1.425387in}}{\pgfqpoint{1.652341in}{1.428659in}}{\pgfqpoint{1.658165in}{1.434483in}}%
\pgfpathcurveto{\pgfqpoint{1.663989in}{1.440307in}}{\pgfqpoint{1.667262in}{1.448207in}}{\pgfqpoint{1.667262in}{1.456444in}}%
\pgfpathcurveto{\pgfqpoint{1.667262in}{1.464680in}}{\pgfqpoint{1.663989in}{1.472580in}}{\pgfqpoint{1.658165in}{1.478404in}}%
\pgfpathcurveto{\pgfqpoint{1.652341in}{1.484228in}}{\pgfqpoint{1.644441in}{1.487500in}}{\pgfqpoint{1.636205in}{1.487500in}}%
\pgfpathcurveto{\pgfqpoint{1.627969in}{1.487500in}}{\pgfqpoint{1.620069in}{1.484228in}}{\pgfqpoint{1.614245in}{1.478404in}}%
\pgfpathcurveto{\pgfqpoint{1.608421in}{1.472580in}}{\pgfqpoint{1.605149in}{1.464680in}}{\pgfqpoint{1.605149in}{1.456444in}}%
\pgfpathcurveto{\pgfqpoint{1.605149in}{1.448207in}}{\pgfqpoint{1.608421in}{1.440307in}}{\pgfqpoint{1.614245in}{1.434483in}}%
\pgfpathcurveto{\pgfqpoint{1.620069in}{1.428659in}}{\pgfqpoint{1.627969in}{1.425387in}}{\pgfqpoint{1.636205in}{1.425387in}}%
\pgfpathclose%
\pgfusepath{stroke,fill}%
\end{pgfscope}%
\begin{pgfscope}%
\pgfpathrectangle{\pgfqpoint{0.100000in}{0.212622in}}{\pgfqpoint{3.696000in}{3.696000in}}%
\pgfusepath{clip}%
\pgfsetbuttcap%
\pgfsetroundjoin%
\definecolor{currentfill}{rgb}{0.121569,0.466667,0.705882}%
\pgfsetfillcolor{currentfill}%
\pgfsetfillopacity{0.525172}%
\pgfsetlinewidth{1.003750pt}%
\definecolor{currentstroke}{rgb}{0.121569,0.466667,0.705882}%
\pgfsetstrokecolor{currentstroke}%
\pgfsetstrokeopacity{0.525172}%
\pgfsetdash{}{0pt}%
\pgfpathmoveto{\pgfqpoint{1.639443in}{1.424219in}}%
\pgfpathcurveto{\pgfqpoint{1.647679in}{1.424219in}}{\pgfqpoint{1.655579in}{1.427491in}}{\pgfqpoint{1.661403in}{1.433315in}}%
\pgfpathcurveto{\pgfqpoint{1.667227in}{1.439139in}}{\pgfqpoint{1.670499in}{1.447039in}}{\pgfqpoint{1.670499in}{1.455276in}}%
\pgfpathcurveto{\pgfqpoint{1.670499in}{1.463512in}}{\pgfqpoint{1.667227in}{1.471412in}}{\pgfqpoint{1.661403in}{1.477236in}}%
\pgfpathcurveto{\pgfqpoint{1.655579in}{1.483060in}}{\pgfqpoint{1.647679in}{1.486332in}}{\pgfqpoint{1.639443in}{1.486332in}}%
\pgfpathcurveto{\pgfqpoint{1.631206in}{1.486332in}}{\pgfqpoint{1.623306in}{1.483060in}}{\pgfqpoint{1.617482in}{1.477236in}}%
\pgfpathcurveto{\pgfqpoint{1.611658in}{1.471412in}}{\pgfqpoint{1.608386in}{1.463512in}}{\pgfqpoint{1.608386in}{1.455276in}}%
\pgfpathcurveto{\pgfqpoint{1.608386in}{1.447039in}}{\pgfqpoint{1.611658in}{1.439139in}}{\pgfqpoint{1.617482in}{1.433315in}}%
\pgfpathcurveto{\pgfqpoint{1.623306in}{1.427491in}}{\pgfqpoint{1.631206in}{1.424219in}}{\pgfqpoint{1.639443in}{1.424219in}}%
\pgfpathclose%
\pgfusepath{stroke,fill}%
\end{pgfscope}%
\begin{pgfscope}%
\pgfpathrectangle{\pgfqpoint{0.100000in}{0.212622in}}{\pgfqpoint{3.696000in}{3.696000in}}%
\pgfusepath{clip}%
\pgfsetbuttcap%
\pgfsetroundjoin%
\definecolor{currentfill}{rgb}{0.121569,0.466667,0.705882}%
\pgfsetfillcolor{currentfill}%
\pgfsetfillopacity{0.525929}%
\pgfsetlinewidth{1.003750pt}%
\definecolor{currentstroke}{rgb}{0.121569,0.466667,0.705882}%
\pgfsetstrokecolor{currentstroke}%
\pgfsetstrokeopacity{0.525929}%
\pgfsetdash{}{0pt}%
\pgfpathmoveto{\pgfqpoint{1.641248in}{1.423564in}}%
\pgfpathcurveto{\pgfqpoint{1.649484in}{1.423564in}}{\pgfqpoint{1.657384in}{1.426836in}}{\pgfqpoint{1.663208in}{1.432660in}}%
\pgfpathcurveto{\pgfqpoint{1.669032in}{1.438484in}}{\pgfqpoint{1.672304in}{1.446384in}}{\pgfqpoint{1.672304in}{1.454620in}}%
\pgfpathcurveto{\pgfqpoint{1.672304in}{1.462856in}}{\pgfqpoint{1.669032in}{1.470756in}}{\pgfqpoint{1.663208in}{1.476580in}}%
\pgfpathcurveto{\pgfqpoint{1.657384in}{1.482404in}}{\pgfqpoint{1.649484in}{1.485677in}}{\pgfqpoint{1.641248in}{1.485677in}}%
\pgfpathcurveto{\pgfqpoint{1.633012in}{1.485677in}}{\pgfqpoint{1.625112in}{1.482404in}}{\pgfqpoint{1.619288in}{1.476580in}}%
\pgfpathcurveto{\pgfqpoint{1.613464in}{1.470756in}}{\pgfqpoint{1.610191in}{1.462856in}}{\pgfqpoint{1.610191in}{1.454620in}}%
\pgfpathcurveto{\pgfqpoint{1.610191in}{1.446384in}}{\pgfqpoint{1.613464in}{1.438484in}}{\pgfqpoint{1.619288in}{1.432660in}}%
\pgfpathcurveto{\pgfqpoint{1.625112in}{1.426836in}}{\pgfqpoint{1.633012in}{1.423564in}}{\pgfqpoint{1.641248in}{1.423564in}}%
\pgfpathclose%
\pgfusepath{stroke,fill}%
\end{pgfscope}%
\begin{pgfscope}%
\pgfpathrectangle{\pgfqpoint{0.100000in}{0.212622in}}{\pgfqpoint{3.696000in}{3.696000in}}%
\pgfusepath{clip}%
\pgfsetbuttcap%
\pgfsetroundjoin%
\definecolor{currentfill}{rgb}{0.121569,0.466667,0.705882}%
\pgfsetfillcolor{currentfill}%
\pgfsetfillopacity{0.526990}%
\pgfsetlinewidth{1.003750pt}%
\definecolor{currentstroke}{rgb}{0.121569,0.466667,0.705882}%
\pgfsetstrokecolor{currentstroke}%
\pgfsetstrokeopacity{0.526990}%
\pgfsetdash{}{0pt}%
\pgfpathmoveto{\pgfqpoint{1.644029in}{1.422602in}}%
\pgfpathcurveto{\pgfqpoint{1.652265in}{1.422602in}}{\pgfqpoint{1.660165in}{1.425875in}}{\pgfqpoint{1.665989in}{1.431698in}}%
\pgfpathcurveto{\pgfqpoint{1.671813in}{1.437522in}}{\pgfqpoint{1.675085in}{1.445422in}}{\pgfqpoint{1.675085in}{1.453659in}}%
\pgfpathcurveto{\pgfqpoint{1.675085in}{1.461895in}}{\pgfqpoint{1.671813in}{1.469795in}}{\pgfqpoint{1.665989in}{1.475619in}}%
\pgfpathcurveto{\pgfqpoint{1.660165in}{1.481443in}}{\pgfqpoint{1.652265in}{1.484715in}}{\pgfqpoint{1.644029in}{1.484715in}}%
\pgfpathcurveto{\pgfqpoint{1.635793in}{1.484715in}}{\pgfqpoint{1.627893in}{1.481443in}}{\pgfqpoint{1.622069in}{1.475619in}}%
\pgfpathcurveto{\pgfqpoint{1.616245in}{1.469795in}}{\pgfqpoint{1.612972in}{1.461895in}}{\pgfqpoint{1.612972in}{1.453659in}}%
\pgfpathcurveto{\pgfqpoint{1.612972in}{1.445422in}}{\pgfqpoint{1.616245in}{1.437522in}}{\pgfqpoint{1.622069in}{1.431698in}}%
\pgfpathcurveto{\pgfqpoint{1.627893in}{1.425875in}}{\pgfqpoint{1.635793in}{1.422602in}}{\pgfqpoint{1.644029in}{1.422602in}}%
\pgfpathclose%
\pgfusepath{stroke,fill}%
\end{pgfscope}%
\begin{pgfscope}%
\pgfpathrectangle{\pgfqpoint{0.100000in}{0.212622in}}{\pgfqpoint{3.696000in}{3.696000in}}%
\pgfusepath{clip}%
\pgfsetbuttcap%
\pgfsetroundjoin%
\definecolor{currentfill}{rgb}{0.121569,0.466667,0.705882}%
\pgfsetfillcolor{currentfill}%
\pgfsetfillopacity{0.528893}%
\pgfsetlinewidth{1.003750pt}%
\definecolor{currentstroke}{rgb}{0.121569,0.466667,0.705882}%
\pgfsetstrokecolor{currentstroke}%
\pgfsetstrokeopacity{0.528893}%
\pgfsetdash{}{0pt}%
\pgfpathmoveto{\pgfqpoint{1.647992in}{1.421084in}}%
\pgfpathcurveto{\pgfqpoint{1.656229in}{1.421084in}}{\pgfqpoint{1.664129in}{1.424356in}}{\pgfqpoint{1.669953in}{1.430180in}}%
\pgfpathcurveto{\pgfqpoint{1.675776in}{1.436004in}}{\pgfqpoint{1.679049in}{1.443904in}}{\pgfqpoint{1.679049in}{1.452140in}}%
\pgfpathcurveto{\pgfqpoint{1.679049in}{1.460377in}}{\pgfqpoint{1.675776in}{1.468277in}}{\pgfqpoint{1.669953in}{1.474101in}}%
\pgfpathcurveto{\pgfqpoint{1.664129in}{1.479925in}}{\pgfqpoint{1.656229in}{1.483197in}}{\pgfqpoint{1.647992in}{1.483197in}}%
\pgfpathcurveto{\pgfqpoint{1.639756in}{1.483197in}}{\pgfqpoint{1.631856in}{1.479925in}}{\pgfqpoint{1.626032in}{1.474101in}}%
\pgfpathcurveto{\pgfqpoint{1.620208in}{1.468277in}}{\pgfqpoint{1.616936in}{1.460377in}}{\pgfqpoint{1.616936in}{1.452140in}}%
\pgfpathcurveto{\pgfqpoint{1.616936in}{1.443904in}}{\pgfqpoint{1.620208in}{1.436004in}}{\pgfqpoint{1.626032in}{1.430180in}}%
\pgfpathcurveto{\pgfqpoint{1.631856in}{1.424356in}}{\pgfqpoint{1.639756in}{1.421084in}}{\pgfqpoint{1.647992in}{1.421084in}}%
\pgfpathclose%
\pgfusepath{stroke,fill}%
\end{pgfscope}%
\begin{pgfscope}%
\pgfpathrectangle{\pgfqpoint{0.100000in}{0.212622in}}{\pgfqpoint{3.696000in}{3.696000in}}%
\pgfusepath{clip}%
\pgfsetbuttcap%
\pgfsetroundjoin%
\definecolor{currentfill}{rgb}{0.121569,0.466667,0.705882}%
\pgfsetfillcolor{currentfill}%
\pgfsetfillopacity{0.530578}%
\pgfsetlinewidth{1.003750pt}%
\definecolor{currentstroke}{rgb}{0.121569,0.466667,0.705882}%
\pgfsetstrokecolor{currentstroke}%
\pgfsetstrokeopacity{0.530578}%
\pgfsetdash{}{0pt}%
\pgfpathmoveto{\pgfqpoint{1.652929in}{1.419731in}}%
\pgfpathcurveto{\pgfqpoint{1.661165in}{1.419731in}}{\pgfqpoint{1.669066in}{1.423004in}}{\pgfqpoint{1.674889in}{1.428828in}}%
\pgfpathcurveto{\pgfqpoint{1.680713in}{1.434652in}}{\pgfqpoint{1.683986in}{1.442552in}}{\pgfqpoint{1.683986in}{1.450788in}}%
\pgfpathcurveto{\pgfqpoint{1.683986in}{1.459024in}}{\pgfqpoint{1.680713in}{1.466924in}}{\pgfqpoint{1.674889in}{1.472748in}}%
\pgfpathcurveto{\pgfqpoint{1.669066in}{1.478572in}}{\pgfqpoint{1.661165in}{1.481844in}}{\pgfqpoint{1.652929in}{1.481844in}}%
\pgfpathcurveto{\pgfqpoint{1.644693in}{1.481844in}}{\pgfqpoint{1.636793in}{1.478572in}}{\pgfqpoint{1.630969in}{1.472748in}}%
\pgfpathcurveto{\pgfqpoint{1.625145in}{1.466924in}}{\pgfqpoint{1.621873in}{1.459024in}}{\pgfqpoint{1.621873in}{1.450788in}}%
\pgfpathcurveto{\pgfqpoint{1.621873in}{1.442552in}}{\pgfqpoint{1.625145in}{1.434652in}}{\pgfqpoint{1.630969in}{1.428828in}}%
\pgfpathcurveto{\pgfqpoint{1.636793in}{1.423004in}}{\pgfqpoint{1.644693in}{1.419731in}}{\pgfqpoint{1.652929in}{1.419731in}}%
\pgfpathclose%
\pgfusepath{stroke,fill}%
\end{pgfscope}%
\begin{pgfscope}%
\pgfpathrectangle{\pgfqpoint{0.100000in}{0.212622in}}{\pgfqpoint{3.696000in}{3.696000in}}%
\pgfusepath{clip}%
\pgfsetbuttcap%
\pgfsetroundjoin%
\definecolor{currentfill}{rgb}{0.121569,0.466667,0.705882}%
\pgfsetfillcolor{currentfill}%
\pgfsetfillopacity{0.531567}%
\pgfsetlinewidth{1.003750pt}%
\definecolor{currentstroke}{rgb}{0.121569,0.466667,0.705882}%
\pgfsetstrokecolor{currentstroke}%
\pgfsetstrokeopacity{0.531567}%
\pgfsetdash{}{0pt}%
\pgfpathmoveto{\pgfqpoint{1.655595in}{1.418952in}}%
\pgfpathcurveto{\pgfqpoint{1.663831in}{1.418952in}}{\pgfqpoint{1.671731in}{1.422224in}}{\pgfqpoint{1.677555in}{1.428048in}}%
\pgfpathcurveto{\pgfqpoint{1.683379in}{1.433872in}}{\pgfqpoint{1.686651in}{1.441772in}}{\pgfqpoint{1.686651in}{1.450009in}}%
\pgfpathcurveto{\pgfqpoint{1.686651in}{1.458245in}}{\pgfqpoint{1.683379in}{1.466145in}}{\pgfqpoint{1.677555in}{1.471969in}}%
\pgfpathcurveto{\pgfqpoint{1.671731in}{1.477793in}}{\pgfqpoint{1.663831in}{1.481065in}}{\pgfqpoint{1.655595in}{1.481065in}}%
\pgfpathcurveto{\pgfqpoint{1.647358in}{1.481065in}}{\pgfqpoint{1.639458in}{1.477793in}}{\pgfqpoint{1.633634in}{1.471969in}}%
\pgfpathcurveto{\pgfqpoint{1.627811in}{1.466145in}}{\pgfqpoint{1.624538in}{1.458245in}}{\pgfqpoint{1.624538in}{1.450009in}}%
\pgfpathcurveto{\pgfqpoint{1.624538in}{1.441772in}}{\pgfqpoint{1.627811in}{1.433872in}}{\pgfqpoint{1.633634in}{1.428048in}}%
\pgfpathcurveto{\pgfqpoint{1.639458in}{1.422224in}}{\pgfqpoint{1.647358in}{1.418952in}}{\pgfqpoint{1.655595in}{1.418952in}}%
\pgfpathclose%
\pgfusepath{stroke,fill}%
\end{pgfscope}%
\begin{pgfscope}%
\pgfpathrectangle{\pgfqpoint{0.100000in}{0.212622in}}{\pgfqpoint{3.696000in}{3.696000in}}%
\pgfusepath{clip}%
\pgfsetbuttcap%
\pgfsetroundjoin%
\definecolor{currentfill}{rgb}{0.121569,0.466667,0.705882}%
\pgfsetfillcolor{currentfill}%
\pgfsetfillopacity{0.532066}%
\pgfsetlinewidth{1.003750pt}%
\definecolor{currentstroke}{rgb}{0.121569,0.466667,0.705882}%
\pgfsetstrokecolor{currentstroke}%
\pgfsetstrokeopacity{0.532066}%
\pgfsetdash{}{0pt}%
\pgfpathmoveto{\pgfqpoint{1.657101in}{1.418561in}}%
\pgfpathcurveto{\pgfqpoint{1.665337in}{1.418561in}}{\pgfqpoint{1.673237in}{1.421833in}}{\pgfqpoint{1.679061in}{1.427657in}}%
\pgfpathcurveto{\pgfqpoint{1.684885in}{1.433481in}}{\pgfqpoint{1.688158in}{1.441381in}}{\pgfqpoint{1.688158in}{1.449618in}}%
\pgfpathcurveto{\pgfqpoint{1.688158in}{1.457854in}}{\pgfqpoint{1.684885in}{1.465754in}}{\pgfqpoint{1.679061in}{1.471578in}}%
\pgfpathcurveto{\pgfqpoint{1.673237in}{1.477402in}}{\pgfqpoint{1.665337in}{1.480674in}}{\pgfqpoint{1.657101in}{1.480674in}}%
\pgfpathcurveto{\pgfqpoint{1.648865in}{1.480674in}}{\pgfqpoint{1.640965in}{1.477402in}}{\pgfqpoint{1.635141in}{1.471578in}}%
\pgfpathcurveto{\pgfqpoint{1.629317in}{1.465754in}}{\pgfqpoint{1.626045in}{1.457854in}}{\pgfqpoint{1.626045in}{1.449618in}}%
\pgfpathcurveto{\pgfqpoint{1.626045in}{1.441381in}}{\pgfqpoint{1.629317in}{1.433481in}}{\pgfqpoint{1.635141in}{1.427657in}}%
\pgfpathcurveto{\pgfqpoint{1.640965in}{1.421833in}}{\pgfqpoint{1.648865in}{1.418561in}}{\pgfqpoint{1.657101in}{1.418561in}}%
\pgfpathclose%
\pgfusepath{stroke,fill}%
\end{pgfscope}%
\begin{pgfscope}%
\pgfpathrectangle{\pgfqpoint{0.100000in}{0.212622in}}{\pgfqpoint{3.696000in}{3.696000in}}%
\pgfusepath{clip}%
\pgfsetbuttcap%
\pgfsetroundjoin%
\definecolor{currentfill}{rgb}{0.121569,0.466667,0.705882}%
\pgfsetfillcolor{currentfill}%
\pgfsetfillopacity{0.533217}%
\pgfsetlinewidth{1.003750pt}%
\definecolor{currentstroke}{rgb}{0.121569,0.466667,0.705882}%
\pgfsetstrokecolor{currentstroke}%
\pgfsetstrokeopacity{0.533217}%
\pgfsetdash{}{0pt}%
\pgfpathmoveto{\pgfqpoint{1.660593in}{1.417573in}}%
\pgfpathcurveto{\pgfqpoint{1.668829in}{1.417573in}}{\pgfqpoint{1.676730in}{1.420845in}}{\pgfqpoint{1.682553in}{1.426669in}}%
\pgfpathcurveto{\pgfqpoint{1.688377in}{1.432493in}}{\pgfqpoint{1.691650in}{1.440393in}}{\pgfqpoint{1.691650in}{1.448630in}}%
\pgfpathcurveto{\pgfqpoint{1.691650in}{1.456866in}}{\pgfqpoint{1.688377in}{1.464766in}}{\pgfqpoint{1.682553in}{1.470590in}}%
\pgfpathcurveto{\pgfqpoint{1.676730in}{1.476414in}}{\pgfqpoint{1.668829in}{1.479686in}}{\pgfqpoint{1.660593in}{1.479686in}}%
\pgfpathcurveto{\pgfqpoint{1.652357in}{1.479686in}}{\pgfqpoint{1.644457in}{1.476414in}}{\pgfqpoint{1.638633in}{1.470590in}}%
\pgfpathcurveto{\pgfqpoint{1.632809in}{1.464766in}}{\pgfqpoint{1.629537in}{1.456866in}}{\pgfqpoint{1.629537in}{1.448630in}}%
\pgfpathcurveto{\pgfqpoint{1.629537in}{1.440393in}}{\pgfqpoint{1.632809in}{1.432493in}}{\pgfqpoint{1.638633in}{1.426669in}}%
\pgfpathcurveto{\pgfqpoint{1.644457in}{1.420845in}}{\pgfqpoint{1.652357in}{1.417573in}}{\pgfqpoint{1.660593in}{1.417573in}}%
\pgfpathclose%
\pgfusepath{stroke,fill}%
\end{pgfscope}%
\begin{pgfscope}%
\pgfpathrectangle{\pgfqpoint{0.100000in}{0.212622in}}{\pgfqpoint{3.696000in}{3.696000in}}%
\pgfusepath{clip}%
\pgfsetbuttcap%
\pgfsetroundjoin%
\definecolor{currentfill}{rgb}{0.121569,0.466667,0.705882}%
\pgfsetfillcolor{currentfill}%
\pgfsetfillopacity{0.534605}%
\pgfsetlinewidth{1.003750pt}%
\definecolor{currentstroke}{rgb}{0.121569,0.466667,0.705882}%
\pgfsetstrokecolor{currentstroke}%
\pgfsetstrokeopacity{0.534605}%
\pgfsetdash{}{0pt}%
\pgfpathmoveto{\pgfqpoint{1.666320in}{1.416694in}}%
\pgfpathcurveto{\pgfqpoint{1.674557in}{1.416694in}}{\pgfqpoint{1.682457in}{1.419966in}}{\pgfqpoint{1.688281in}{1.425790in}}%
\pgfpathcurveto{\pgfqpoint{1.694105in}{1.431614in}}{\pgfqpoint{1.697377in}{1.439514in}}{\pgfqpoint{1.697377in}{1.447750in}}%
\pgfpathcurveto{\pgfqpoint{1.697377in}{1.455986in}}{\pgfqpoint{1.694105in}{1.463886in}}{\pgfqpoint{1.688281in}{1.469710in}}%
\pgfpathcurveto{\pgfqpoint{1.682457in}{1.475534in}}{\pgfqpoint{1.674557in}{1.478807in}}{\pgfqpoint{1.666320in}{1.478807in}}%
\pgfpathcurveto{\pgfqpoint{1.658084in}{1.478807in}}{\pgfqpoint{1.650184in}{1.475534in}}{\pgfqpoint{1.644360in}{1.469710in}}%
\pgfpathcurveto{\pgfqpoint{1.638536in}{1.463886in}}{\pgfqpoint{1.635264in}{1.455986in}}{\pgfqpoint{1.635264in}{1.447750in}}%
\pgfpathcurveto{\pgfqpoint{1.635264in}{1.439514in}}{\pgfqpoint{1.638536in}{1.431614in}}{\pgfqpoint{1.644360in}{1.425790in}}%
\pgfpathcurveto{\pgfqpoint{1.650184in}{1.419966in}}{\pgfqpoint{1.658084in}{1.416694in}}{\pgfqpoint{1.666320in}{1.416694in}}%
\pgfpathclose%
\pgfusepath{stroke,fill}%
\end{pgfscope}%
\begin{pgfscope}%
\pgfpathrectangle{\pgfqpoint{0.100000in}{0.212622in}}{\pgfqpoint{3.696000in}{3.696000in}}%
\pgfusepath{clip}%
\pgfsetbuttcap%
\pgfsetroundjoin%
\definecolor{currentfill}{rgb}{0.121569,0.466667,0.705882}%
\pgfsetfillcolor{currentfill}%
\pgfsetfillopacity{0.536496}%
\pgfsetlinewidth{1.003750pt}%
\definecolor{currentstroke}{rgb}{0.121569,0.466667,0.705882}%
\pgfsetstrokecolor{currentstroke}%
\pgfsetstrokeopacity{0.536496}%
\pgfsetdash{}{0pt}%
\pgfpathmoveto{\pgfqpoint{1.672380in}{1.415312in}}%
\pgfpathcurveto{\pgfqpoint{1.680617in}{1.415312in}}{\pgfqpoint{1.688517in}{1.418584in}}{\pgfqpoint{1.694341in}{1.424408in}}%
\pgfpathcurveto{\pgfqpoint{1.700165in}{1.430232in}}{\pgfqpoint{1.703437in}{1.438132in}}{\pgfqpoint{1.703437in}{1.446368in}}%
\pgfpathcurveto{\pgfqpoint{1.703437in}{1.454604in}}{\pgfqpoint{1.700165in}{1.462504in}}{\pgfqpoint{1.694341in}{1.468328in}}%
\pgfpathcurveto{\pgfqpoint{1.688517in}{1.474152in}}{\pgfqpoint{1.680617in}{1.477425in}}{\pgfqpoint{1.672380in}{1.477425in}}%
\pgfpathcurveto{\pgfqpoint{1.664144in}{1.477425in}}{\pgfqpoint{1.656244in}{1.474152in}}{\pgfqpoint{1.650420in}{1.468328in}}%
\pgfpathcurveto{\pgfqpoint{1.644596in}{1.462504in}}{\pgfqpoint{1.641324in}{1.454604in}}{\pgfqpoint{1.641324in}{1.446368in}}%
\pgfpathcurveto{\pgfqpoint{1.641324in}{1.438132in}}{\pgfqpoint{1.644596in}{1.430232in}}{\pgfqpoint{1.650420in}{1.424408in}}%
\pgfpathcurveto{\pgfqpoint{1.656244in}{1.418584in}}{\pgfqpoint{1.664144in}{1.415312in}}{\pgfqpoint{1.672380in}{1.415312in}}%
\pgfpathclose%
\pgfusepath{stroke,fill}%
\end{pgfscope}%
\begin{pgfscope}%
\pgfpathrectangle{\pgfqpoint{0.100000in}{0.212622in}}{\pgfqpoint{3.696000in}{3.696000in}}%
\pgfusepath{clip}%
\pgfsetbuttcap%
\pgfsetroundjoin%
\definecolor{currentfill}{rgb}{0.121569,0.466667,0.705882}%
\pgfsetfillcolor{currentfill}%
\pgfsetfillopacity{0.537850}%
\pgfsetlinewidth{1.003750pt}%
\definecolor{currentstroke}{rgb}{0.121569,0.466667,0.705882}%
\pgfsetstrokecolor{currentstroke}%
\pgfsetstrokeopacity{0.537850}%
\pgfsetdash{}{0pt}%
\pgfpathmoveto{\pgfqpoint{1.675380in}{1.414101in}}%
\pgfpathcurveto{\pgfqpoint{1.683616in}{1.414101in}}{\pgfqpoint{1.691516in}{1.417374in}}{\pgfqpoint{1.697340in}{1.423198in}}%
\pgfpathcurveto{\pgfqpoint{1.703164in}{1.429021in}}{\pgfqpoint{1.706436in}{1.436921in}}{\pgfqpoint{1.706436in}{1.445158in}}%
\pgfpathcurveto{\pgfqpoint{1.706436in}{1.453394in}}{\pgfqpoint{1.703164in}{1.461294in}}{\pgfqpoint{1.697340in}{1.467118in}}%
\pgfpathcurveto{\pgfqpoint{1.691516in}{1.472942in}}{\pgfqpoint{1.683616in}{1.476214in}}{\pgfqpoint{1.675380in}{1.476214in}}%
\pgfpathcurveto{\pgfqpoint{1.667143in}{1.476214in}}{\pgfqpoint{1.659243in}{1.472942in}}{\pgfqpoint{1.653419in}{1.467118in}}%
\pgfpathcurveto{\pgfqpoint{1.647595in}{1.461294in}}{\pgfqpoint{1.644323in}{1.453394in}}{\pgfqpoint{1.644323in}{1.445158in}}%
\pgfpathcurveto{\pgfqpoint{1.644323in}{1.436921in}}{\pgfqpoint{1.647595in}{1.429021in}}{\pgfqpoint{1.653419in}{1.423198in}}%
\pgfpathcurveto{\pgfqpoint{1.659243in}{1.417374in}}{\pgfqpoint{1.667143in}{1.414101in}}{\pgfqpoint{1.675380in}{1.414101in}}%
\pgfpathclose%
\pgfusepath{stroke,fill}%
\end{pgfscope}%
\begin{pgfscope}%
\pgfpathrectangle{\pgfqpoint{0.100000in}{0.212622in}}{\pgfqpoint{3.696000in}{3.696000in}}%
\pgfusepath{clip}%
\pgfsetbuttcap%
\pgfsetroundjoin%
\definecolor{currentfill}{rgb}{0.121569,0.466667,0.705882}%
\pgfsetfillcolor{currentfill}%
\pgfsetfillopacity{0.538444}%
\pgfsetlinewidth{1.003750pt}%
\definecolor{currentstroke}{rgb}{0.121569,0.466667,0.705882}%
\pgfsetstrokecolor{currentstroke}%
\pgfsetstrokeopacity{0.538444}%
\pgfsetdash{}{0pt}%
\pgfpathmoveto{\pgfqpoint{1.677198in}{1.413677in}}%
\pgfpathcurveto{\pgfqpoint{1.685434in}{1.413677in}}{\pgfqpoint{1.693334in}{1.416949in}}{\pgfqpoint{1.699158in}{1.422773in}}%
\pgfpathcurveto{\pgfqpoint{1.704982in}{1.428597in}}{\pgfqpoint{1.708254in}{1.436497in}}{\pgfqpoint{1.708254in}{1.444733in}}%
\pgfpathcurveto{\pgfqpoint{1.708254in}{1.452970in}}{\pgfqpoint{1.704982in}{1.460870in}}{\pgfqpoint{1.699158in}{1.466694in}}%
\pgfpathcurveto{\pgfqpoint{1.693334in}{1.472517in}}{\pgfqpoint{1.685434in}{1.475790in}}{\pgfqpoint{1.677198in}{1.475790in}}%
\pgfpathcurveto{\pgfqpoint{1.668961in}{1.475790in}}{\pgfqpoint{1.661061in}{1.472517in}}{\pgfqpoint{1.655237in}{1.466694in}}%
\pgfpathcurveto{\pgfqpoint{1.649413in}{1.460870in}}{\pgfqpoint{1.646141in}{1.452970in}}{\pgfqpoint{1.646141in}{1.444733in}}%
\pgfpathcurveto{\pgfqpoint{1.646141in}{1.436497in}}{\pgfqpoint{1.649413in}{1.428597in}}{\pgfqpoint{1.655237in}{1.422773in}}%
\pgfpathcurveto{\pgfqpoint{1.661061in}{1.416949in}}{\pgfqpoint{1.668961in}{1.413677in}}{\pgfqpoint{1.677198in}{1.413677in}}%
\pgfpathclose%
\pgfusepath{stroke,fill}%
\end{pgfscope}%
\begin{pgfscope}%
\pgfpathrectangle{\pgfqpoint{0.100000in}{0.212622in}}{\pgfqpoint{3.696000in}{3.696000in}}%
\pgfusepath{clip}%
\pgfsetbuttcap%
\pgfsetroundjoin%
\definecolor{currentfill}{rgb}{0.121569,0.466667,0.705882}%
\pgfsetfillcolor{currentfill}%
\pgfsetfillopacity{0.539452}%
\pgfsetlinewidth{1.003750pt}%
\definecolor{currentstroke}{rgb}{0.121569,0.466667,0.705882}%
\pgfsetstrokecolor{currentstroke}%
\pgfsetstrokeopacity{0.539452}%
\pgfsetdash{}{0pt}%
\pgfpathmoveto{\pgfqpoint{1.680275in}{1.412863in}}%
\pgfpathcurveto{\pgfqpoint{1.688511in}{1.412863in}}{\pgfqpoint{1.696411in}{1.416135in}}{\pgfqpoint{1.702235in}{1.421959in}}%
\pgfpathcurveto{\pgfqpoint{1.708059in}{1.427783in}}{\pgfqpoint{1.711332in}{1.435683in}}{\pgfqpoint{1.711332in}{1.443919in}}%
\pgfpathcurveto{\pgfqpoint{1.711332in}{1.452156in}}{\pgfqpoint{1.708059in}{1.460056in}}{\pgfqpoint{1.702235in}{1.465880in}}%
\pgfpathcurveto{\pgfqpoint{1.696411in}{1.471704in}}{\pgfqpoint{1.688511in}{1.474976in}}{\pgfqpoint{1.680275in}{1.474976in}}%
\pgfpathcurveto{\pgfqpoint{1.672039in}{1.474976in}}{\pgfqpoint{1.664139in}{1.471704in}}{\pgfqpoint{1.658315in}{1.465880in}}%
\pgfpathcurveto{\pgfqpoint{1.652491in}{1.460056in}}{\pgfqpoint{1.649219in}{1.452156in}}{\pgfqpoint{1.649219in}{1.443919in}}%
\pgfpathcurveto{\pgfqpoint{1.649219in}{1.435683in}}{\pgfqpoint{1.652491in}{1.427783in}}{\pgfqpoint{1.658315in}{1.421959in}}%
\pgfpathcurveto{\pgfqpoint{1.664139in}{1.416135in}}{\pgfqpoint{1.672039in}{1.412863in}}{\pgfqpoint{1.680275in}{1.412863in}}%
\pgfpathclose%
\pgfusepath{stroke,fill}%
\end{pgfscope}%
\begin{pgfscope}%
\pgfpathrectangle{\pgfqpoint{0.100000in}{0.212622in}}{\pgfqpoint{3.696000in}{3.696000in}}%
\pgfusepath{clip}%
\pgfsetbuttcap%
\pgfsetroundjoin%
\definecolor{currentfill}{rgb}{0.121569,0.466667,0.705882}%
\pgfsetfillcolor{currentfill}%
\pgfsetfillopacity{0.540743}%
\pgfsetlinewidth{1.003750pt}%
\definecolor{currentstroke}{rgb}{0.121569,0.466667,0.705882}%
\pgfsetstrokecolor{currentstroke}%
\pgfsetstrokeopacity{0.540743}%
\pgfsetdash{}{0pt}%
\pgfpathmoveto{\pgfqpoint{1.685008in}{1.411995in}}%
\pgfpathcurveto{\pgfqpoint{1.693244in}{1.411995in}}{\pgfqpoint{1.701144in}{1.415267in}}{\pgfqpoint{1.706968in}{1.421091in}}%
\pgfpathcurveto{\pgfqpoint{1.712792in}{1.426915in}}{\pgfqpoint{1.716064in}{1.434815in}}{\pgfqpoint{1.716064in}{1.443051in}}%
\pgfpathcurveto{\pgfqpoint{1.716064in}{1.451288in}}{\pgfqpoint{1.712792in}{1.459188in}}{\pgfqpoint{1.706968in}{1.465012in}}%
\pgfpathcurveto{\pgfqpoint{1.701144in}{1.470836in}}{\pgfqpoint{1.693244in}{1.474108in}}{\pgfqpoint{1.685008in}{1.474108in}}%
\pgfpathcurveto{\pgfqpoint{1.676772in}{1.474108in}}{\pgfqpoint{1.668872in}{1.470836in}}{\pgfqpoint{1.663048in}{1.465012in}}%
\pgfpathcurveto{\pgfqpoint{1.657224in}{1.459188in}}{\pgfqpoint{1.653951in}{1.451288in}}{\pgfqpoint{1.653951in}{1.443051in}}%
\pgfpathcurveto{\pgfqpoint{1.653951in}{1.434815in}}{\pgfqpoint{1.657224in}{1.426915in}}{\pgfqpoint{1.663048in}{1.421091in}}%
\pgfpathcurveto{\pgfqpoint{1.668872in}{1.415267in}}{\pgfqpoint{1.676772in}{1.411995in}}{\pgfqpoint{1.685008in}{1.411995in}}%
\pgfpathclose%
\pgfusepath{stroke,fill}%
\end{pgfscope}%
\begin{pgfscope}%
\pgfpathrectangle{\pgfqpoint{0.100000in}{0.212622in}}{\pgfqpoint{3.696000in}{3.696000in}}%
\pgfusepath{clip}%
\pgfsetbuttcap%
\pgfsetroundjoin%
\definecolor{currentfill}{rgb}{0.121569,0.466667,0.705882}%
\pgfsetfillcolor{currentfill}%
\pgfsetfillopacity{0.542688}%
\pgfsetlinewidth{1.003750pt}%
\definecolor{currentstroke}{rgb}{0.121569,0.466667,0.705882}%
\pgfsetstrokecolor{currentstroke}%
\pgfsetstrokeopacity{0.542688}%
\pgfsetdash{}{0pt}%
\pgfpathmoveto{\pgfqpoint{1.689801in}{1.410154in}}%
\pgfpathcurveto{\pgfqpoint{1.698037in}{1.410154in}}{\pgfqpoint{1.705937in}{1.413426in}}{\pgfqpoint{1.711761in}{1.419250in}}%
\pgfpathcurveto{\pgfqpoint{1.717585in}{1.425074in}}{\pgfqpoint{1.720857in}{1.432974in}}{\pgfqpoint{1.720857in}{1.441210in}}%
\pgfpathcurveto{\pgfqpoint{1.720857in}{1.449447in}}{\pgfqpoint{1.717585in}{1.457347in}}{\pgfqpoint{1.711761in}{1.463171in}}%
\pgfpathcurveto{\pgfqpoint{1.705937in}{1.468995in}}{\pgfqpoint{1.698037in}{1.472267in}}{\pgfqpoint{1.689801in}{1.472267in}}%
\pgfpathcurveto{\pgfqpoint{1.681564in}{1.472267in}}{\pgfqpoint{1.673664in}{1.468995in}}{\pgfqpoint{1.667840in}{1.463171in}}%
\pgfpathcurveto{\pgfqpoint{1.662016in}{1.457347in}}{\pgfqpoint{1.658744in}{1.449447in}}{\pgfqpoint{1.658744in}{1.441210in}}%
\pgfpathcurveto{\pgfqpoint{1.658744in}{1.432974in}}{\pgfqpoint{1.662016in}{1.425074in}}{\pgfqpoint{1.667840in}{1.419250in}}%
\pgfpathcurveto{\pgfqpoint{1.673664in}{1.413426in}}{\pgfqpoint{1.681564in}{1.410154in}}{\pgfqpoint{1.689801in}{1.410154in}}%
\pgfpathclose%
\pgfusepath{stroke,fill}%
\end{pgfscope}%
\begin{pgfscope}%
\pgfpathrectangle{\pgfqpoint{0.100000in}{0.212622in}}{\pgfqpoint{3.696000in}{3.696000in}}%
\pgfusepath{clip}%
\pgfsetbuttcap%
\pgfsetroundjoin%
\definecolor{currentfill}{rgb}{0.121569,0.466667,0.705882}%
\pgfsetfillcolor{currentfill}%
\pgfsetfillopacity{0.543999}%
\pgfsetlinewidth{1.003750pt}%
\definecolor{currentstroke}{rgb}{0.121569,0.466667,0.705882}%
\pgfsetstrokecolor{currentstroke}%
\pgfsetstrokeopacity{0.543999}%
\pgfsetdash{}{0pt}%
\pgfpathmoveto{\pgfqpoint{1.692242in}{1.409011in}}%
\pgfpathcurveto{\pgfqpoint{1.700478in}{1.409011in}}{\pgfqpoint{1.708378in}{1.412284in}}{\pgfqpoint{1.714202in}{1.418108in}}%
\pgfpathcurveto{\pgfqpoint{1.720026in}{1.423932in}}{\pgfqpoint{1.723299in}{1.431832in}}{\pgfqpoint{1.723299in}{1.440068in}}%
\pgfpathcurveto{\pgfqpoint{1.723299in}{1.448304in}}{\pgfqpoint{1.720026in}{1.456204in}}{\pgfqpoint{1.714202in}{1.462028in}}%
\pgfpathcurveto{\pgfqpoint{1.708378in}{1.467852in}}{\pgfqpoint{1.700478in}{1.471124in}}{\pgfqpoint{1.692242in}{1.471124in}}%
\pgfpathcurveto{\pgfqpoint{1.684006in}{1.471124in}}{\pgfqpoint{1.676106in}{1.467852in}}{\pgfqpoint{1.670282in}{1.462028in}}%
\pgfpathcurveto{\pgfqpoint{1.664458in}{1.456204in}}{\pgfqpoint{1.661186in}{1.448304in}}{\pgfqpoint{1.661186in}{1.440068in}}%
\pgfpathcurveto{\pgfqpoint{1.661186in}{1.431832in}}{\pgfqpoint{1.664458in}{1.423932in}}{\pgfqpoint{1.670282in}{1.418108in}}%
\pgfpathcurveto{\pgfqpoint{1.676106in}{1.412284in}}{\pgfqpoint{1.684006in}{1.409011in}}{\pgfqpoint{1.692242in}{1.409011in}}%
\pgfpathclose%
\pgfusepath{stroke,fill}%
\end{pgfscope}%
\begin{pgfscope}%
\pgfpathrectangle{\pgfqpoint{0.100000in}{0.212622in}}{\pgfqpoint{3.696000in}{3.696000in}}%
\pgfusepath{clip}%
\pgfsetbuttcap%
\pgfsetroundjoin%
\definecolor{currentfill}{rgb}{0.121569,0.466667,0.705882}%
\pgfsetfillcolor{currentfill}%
\pgfsetfillopacity{0.544580}%
\pgfsetlinewidth{1.003750pt}%
\definecolor{currentstroke}{rgb}{0.121569,0.466667,0.705882}%
\pgfsetstrokecolor{currentstroke}%
\pgfsetstrokeopacity{0.544580}%
\pgfsetdash{}{0pt}%
\pgfpathmoveto{\pgfqpoint{1.693728in}{1.408560in}}%
\pgfpathcurveto{\pgfqpoint{1.701964in}{1.408560in}}{\pgfqpoint{1.709864in}{1.411832in}}{\pgfqpoint{1.715688in}{1.417656in}}%
\pgfpathcurveto{\pgfqpoint{1.721512in}{1.423480in}}{\pgfqpoint{1.724784in}{1.431380in}}{\pgfqpoint{1.724784in}{1.439617in}}%
\pgfpathcurveto{\pgfqpoint{1.724784in}{1.447853in}}{\pgfqpoint{1.721512in}{1.455753in}}{\pgfqpoint{1.715688in}{1.461577in}}%
\pgfpathcurveto{\pgfqpoint{1.709864in}{1.467401in}}{\pgfqpoint{1.701964in}{1.470673in}}{\pgfqpoint{1.693728in}{1.470673in}}%
\pgfpathcurveto{\pgfqpoint{1.685492in}{1.470673in}}{\pgfqpoint{1.677592in}{1.467401in}}{\pgfqpoint{1.671768in}{1.461577in}}%
\pgfpathcurveto{\pgfqpoint{1.665944in}{1.455753in}}{\pgfqpoint{1.662671in}{1.447853in}}{\pgfqpoint{1.662671in}{1.439617in}}%
\pgfpathcurveto{\pgfqpoint{1.662671in}{1.431380in}}{\pgfqpoint{1.665944in}{1.423480in}}{\pgfqpoint{1.671768in}{1.417656in}}%
\pgfpathcurveto{\pgfqpoint{1.677592in}{1.411832in}}{\pgfqpoint{1.685492in}{1.408560in}}{\pgfqpoint{1.693728in}{1.408560in}}%
\pgfpathclose%
\pgfusepath{stroke,fill}%
\end{pgfscope}%
\begin{pgfscope}%
\pgfpathrectangle{\pgfqpoint{0.100000in}{0.212622in}}{\pgfqpoint{3.696000in}{3.696000in}}%
\pgfusepath{clip}%
\pgfsetbuttcap%
\pgfsetroundjoin%
\definecolor{currentfill}{rgb}{0.121569,0.466667,0.705882}%
\pgfsetfillcolor{currentfill}%
\pgfsetfillopacity{0.545670}%
\pgfsetlinewidth{1.003750pt}%
\definecolor{currentstroke}{rgb}{0.121569,0.466667,0.705882}%
\pgfsetstrokecolor{currentstroke}%
\pgfsetstrokeopacity{0.545670}%
\pgfsetdash{}{0pt}%
\pgfpathmoveto{\pgfqpoint{1.696802in}{1.407653in}}%
\pgfpathcurveto{\pgfqpoint{1.705039in}{1.407653in}}{\pgfqpoint{1.712939in}{1.410925in}}{\pgfqpoint{1.718763in}{1.416749in}}%
\pgfpathcurveto{\pgfqpoint{1.724587in}{1.422573in}}{\pgfqpoint{1.727859in}{1.430473in}}{\pgfqpoint{1.727859in}{1.438710in}}%
\pgfpathcurveto{\pgfqpoint{1.727859in}{1.446946in}}{\pgfqpoint{1.724587in}{1.454846in}}{\pgfqpoint{1.718763in}{1.460670in}}%
\pgfpathcurveto{\pgfqpoint{1.712939in}{1.466494in}}{\pgfqpoint{1.705039in}{1.469766in}}{\pgfqpoint{1.696802in}{1.469766in}}%
\pgfpathcurveto{\pgfqpoint{1.688566in}{1.469766in}}{\pgfqpoint{1.680666in}{1.466494in}}{\pgfqpoint{1.674842in}{1.460670in}}%
\pgfpathcurveto{\pgfqpoint{1.669018in}{1.454846in}}{\pgfqpoint{1.665746in}{1.446946in}}{\pgfqpoint{1.665746in}{1.438710in}}%
\pgfpathcurveto{\pgfqpoint{1.665746in}{1.430473in}}{\pgfqpoint{1.669018in}{1.422573in}}{\pgfqpoint{1.674842in}{1.416749in}}%
\pgfpathcurveto{\pgfqpoint{1.680666in}{1.410925in}}{\pgfqpoint{1.688566in}{1.407653in}}{\pgfqpoint{1.696802in}{1.407653in}}%
\pgfpathclose%
\pgfusepath{stroke,fill}%
\end{pgfscope}%
\begin{pgfscope}%
\pgfpathrectangle{\pgfqpoint{0.100000in}{0.212622in}}{\pgfqpoint{3.696000in}{3.696000in}}%
\pgfusepath{clip}%
\pgfsetbuttcap%
\pgfsetroundjoin%
\definecolor{currentfill}{rgb}{0.121569,0.466667,0.705882}%
\pgfsetfillcolor{currentfill}%
\pgfsetfillopacity{0.547626}%
\pgfsetlinewidth{1.003750pt}%
\definecolor{currentstroke}{rgb}{0.121569,0.466667,0.705882}%
\pgfsetstrokecolor{currentstroke}%
\pgfsetstrokeopacity{0.547626}%
\pgfsetdash{}{0pt}%
\pgfpathmoveto{\pgfqpoint{1.701335in}{1.406037in}}%
\pgfpathcurveto{\pgfqpoint{1.709571in}{1.406037in}}{\pgfqpoint{1.717471in}{1.409309in}}{\pgfqpoint{1.723295in}{1.415133in}}%
\pgfpathcurveto{\pgfqpoint{1.729119in}{1.420957in}}{\pgfqpoint{1.732391in}{1.428857in}}{\pgfqpoint{1.732391in}{1.437093in}}%
\pgfpathcurveto{\pgfqpoint{1.732391in}{1.445330in}}{\pgfqpoint{1.729119in}{1.453230in}}{\pgfqpoint{1.723295in}{1.459053in}}%
\pgfpathcurveto{\pgfqpoint{1.717471in}{1.464877in}}{\pgfqpoint{1.709571in}{1.468150in}}{\pgfqpoint{1.701335in}{1.468150in}}%
\pgfpathcurveto{\pgfqpoint{1.693099in}{1.468150in}}{\pgfqpoint{1.685199in}{1.464877in}}{\pgfqpoint{1.679375in}{1.459053in}}%
\pgfpathcurveto{\pgfqpoint{1.673551in}{1.453230in}}{\pgfqpoint{1.670278in}{1.445330in}}{\pgfqpoint{1.670278in}{1.437093in}}%
\pgfpathcurveto{\pgfqpoint{1.670278in}{1.428857in}}{\pgfqpoint{1.673551in}{1.420957in}}{\pgfqpoint{1.679375in}{1.415133in}}%
\pgfpathcurveto{\pgfqpoint{1.685199in}{1.409309in}}{\pgfqpoint{1.693099in}{1.406037in}}{\pgfqpoint{1.701335in}{1.406037in}}%
\pgfpathclose%
\pgfusepath{stroke,fill}%
\end{pgfscope}%
\begin{pgfscope}%
\pgfpathrectangle{\pgfqpoint{0.100000in}{0.212622in}}{\pgfqpoint{3.696000in}{3.696000in}}%
\pgfusepath{clip}%
\pgfsetbuttcap%
\pgfsetroundjoin%
\definecolor{currentfill}{rgb}{0.121569,0.466667,0.705882}%
\pgfsetfillcolor{currentfill}%
\pgfsetfillopacity{0.550125}%
\pgfsetlinewidth{1.003750pt}%
\definecolor{currentstroke}{rgb}{0.121569,0.466667,0.705882}%
\pgfsetstrokecolor{currentstroke}%
\pgfsetstrokeopacity{0.550125}%
\pgfsetdash{}{0pt}%
\pgfpathmoveto{\pgfqpoint{1.706617in}{1.403825in}}%
\pgfpathcurveto{\pgfqpoint{1.714853in}{1.403825in}}{\pgfqpoint{1.722753in}{1.407097in}}{\pgfqpoint{1.728577in}{1.412921in}}%
\pgfpathcurveto{\pgfqpoint{1.734401in}{1.418745in}}{\pgfqpoint{1.737673in}{1.426645in}}{\pgfqpoint{1.737673in}{1.434881in}}%
\pgfpathcurveto{\pgfqpoint{1.737673in}{1.443118in}}{\pgfqpoint{1.734401in}{1.451018in}}{\pgfqpoint{1.728577in}{1.456842in}}%
\pgfpathcurveto{\pgfqpoint{1.722753in}{1.462666in}}{\pgfqpoint{1.714853in}{1.465938in}}{\pgfqpoint{1.706617in}{1.465938in}}%
\pgfpathcurveto{\pgfqpoint{1.698381in}{1.465938in}}{\pgfqpoint{1.690481in}{1.462666in}}{\pgfqpoint{1.684657in}{1.456842in}}%
\pgfpathcurveto{\pgfqpoint{1.678833in}{1.451018in}}{\pgfqpoint{1.675560in}{1.443118in}}{\pgfqpoint{1.675560in}{1.434881in}}%
\pgfpathcurveto{\pgfqpoint{1.675560in}{1.426645in}}{\pgfqpoint{1.678833in}{1.418745in}}{\pgfqpoint{1.684657in}{1.412921in}}%
\pgfpathcurveto{\pgfqpoint{1.690481in}{1.407097in}}{\pgfqpoint{1.698381in}{1.403825in}}{\pgfqpoint{1.706617in}{1.403825in}}%
\pgfpathclose%
\pgfusepath{stroke,fill}%
\end{pgfscope}%
\begin{pgfscope}%
\pgfpathrectangle{\pgfqpoint{0.100000in}{0.212622in}}{\pgfqpoint{3.696000in}{3.696000in}}%
\pgfusepath{clip}%
\pgfsetbuttcap%
\pgfsetroundjoin%
\definecolor{currentfill}{rgb}{0.121569,0.466667,0.705882}%
\pgfsetfillcolor{currentfill}%
\pgfsetfillopacity{0.552144}%
\pgfsetlinewidth{1.003750pt}%
\definecolor{currentstroke}{rgb}{0.121569,0.466667,0.705882}%
\pgfsetstrokecolor{currentstroke}%
\pgfsetstrokeopacity{0.552144}%
\pgfsetdash{}{0pt}%
\pgfpathmoveto{\pgfqpoint{1.713440in}{1.402524in}}%
\pgfpathcurveto{\pgfqpoint{1.721677in}{1.402524in}}{\pgfqpoint{1.729577in}{1.405796in}}{\pgfqpoint{1.735401in}{1.411620in}}%
\pgfpathcurveto{\pgfqpoint{1.741224in}{1.417444in}}{\pgfqpoint{1.744497in}{1.425344in}}{\pgfqpoint{1.744497in}{1.433580in}}%
\pgfpathcurveto{\pgfqpoint{1.744497in}{1.441817in}}{\pgfqpoint{1.741224in}{1.449717in}}{\pgfqpoint{1.735401in}{1.455541in}}%
\pgfpathcurveto{\pgfqpoint{1.729577in}{1.461365in}}{\pgfqpoint{1.721677in}{1.464637in}}{\pgfqpoint{1.713440in}{1.464637in}}%
\pgfpathcurveto{\pgfqpoint{1.705204in}{1.464637in}}{\pgfqpoint{1.697304in}{1.461365in}}{\pgfqpoint{1.691480in}{1.455541in}}%
\pgfpathcurveto{\pgfqpoint{1.685656in}{1.449717in}}{\pgfqpoint{1.682384in}{1.441817in}}{\pgfqpoint{1.682384in}{1.433580in}}%
\pgfpathcurveto{\pgfqpoint{1.682384in}{1.425344in}}{\pgfqpoint{1.685656in}{1.417444in}}{\pgfqpoint{1.691480in}{1.411620in}}%
\pgfpathcurveto{\pgfqpoint{1.697304in}{1.405796in}}{\pgfqpoint{1.705204in}{1.402524in}}{\pgfqpoint{1.713440in}{1.402524in}}%
\pgfpathclose%
\pgfusepath{stroke,fill}%
\end{pgfscope}%
\begin{pgfscope}%
\pgfpathrectangle{\pgfqpoint{0.100000in}{0.212622in}}{\pgfqpoint{3.696000in}{3.696000in}}%
\pgfusepath{clip}%
\pgfsetbuttcap%
\pgfsetroundjoin%
\definecolor{currentfill}{rgb}{0.121569,0.466667,0.705882}%
\pgfsetfillcolor{currentfill}%
\pgfsetfillopacity{0.553580}%
\pgfsetlinewidth{1.003750pt}%
\definecolor{currentstroke}{rgb}{0.121569,0.466667,0.705882}%
\pgfsetstrokecolor{currentstroke}%
\pgfsetstrokeopacity{0.553580}%
\pgfsetdash{}{0pt}%
\pgfpathmoveto{\pgfqpoint{1.716861in}{1.401379in}}%
\pgfpathcurveto{\pgfqpoint{1.725097in}{1.401379in}}{\pgfqpoint{1.732997in}{1.404651in}}{\pgfqpoint{1.738821in}{1.410475in}}%
\pgfpathcurveto{\pgfqpoint{1.744645in}{1.416299in}}{\pgfqpoint{1.747918in}{1.424199in}}{\pgfqpoint{1.747918in}{1.432436in}}%
\pgfpathcurveto{\pgfqpoint{1.747918in}{1.440672in}}{\pgfqpoint{1.744645in}{1.448572in}}{\pgfqpoint{1.738821in}{1.454396in}}%
\pgfpathcurveto{\pgfqpoint{1.732997in}{1.460220in}}{\pgfqpoint{1.725097in}{1.463492in}}{\pgfqpoint{1.716861in}{1.463492in}}%
\pgfpathcurveto{\pgfqpoint{1.708625in}{1.463492in}}{\pgfqpoint{1.700725in}{1.460220in}}{\pgfqpoint{1.694901in}{1.454396in}}%
\pgfpathcurveto{\pgfqpoint{1.689077in}{1.448572in}}{\pgfqpoint{1.685805in}{1.440672in}}{\pgfqpoint{1.685805in}{1.432436in}}%
\pgfpathcurveto{\pgfqpoint{1.685805in}{1.424199in}}{\pgfqpoint{1.689077in}{1.416299in}}{\pgfqpoint{1.694901in}{1.410475in}}%
\pgfpathcurveto{\pgfqpoint{1.700725in}{1.404651in}}{\pgfqpoint{1.708625in}{1.401379in}}{\pgfqpoint{1.716861in}{1.401379in}}%
\pgfpathclose%
\pgfusepath{stroke,fill}%
\end{pgfscope}%
\begin{pgfscope}%
\pgfpathrectangle{\pgfqpoint{0.100000in}{0.212622in}}{\pgfqpoint{3.696000in}{3.696000in}}%
\pgfusepath{clip}%
\pgfsetbuttcap%
\pgfsetroundjoin%
\definecolor{currentfill}{rgb}{0.121569,0.466667,0.705882}%
\pgfsetfillcolor{currentfill}%
\pgfsetfillopacity{0.555137}%
\pgfsetlinewidth{1.003750pt}%
\definecolor{currentstroke}{rgb}{0.121569,0.466667,0.705882}%
\pgfsetstrokecolor{currentstroke}%
\pgfsetstrokeopacity{0.555137}%
\pgfsetdash{}{0pt}%
\pgfpathmoveto{\pgfqpoint{1.721211in}{1.399991in}}%
\pgfpathcurveto{\pgfqpoint{1.729448in}{1.399991in}}{\pgfqpoint{1.737348in}{1.403263in}}{\pgfqpoint{1.743172in}{1.409087in}}%
\pgfpathcurveto{\pgfqpoint{1.748995in}{1.414911in}}{\pgfqpoint{1.752268in}{1.422811in}}{\pgfqpoint{1.752268in}{1.431047in}}%
\pgfpathcurveto{\pgfqpoint{1.752268in}{1.439284in}}{\pgfqpoint{1.748995in}{1.447184in}}{\pgfqpoint{1.743172in}{1.453008in}}%
\pgfpathcurveto{\pgfqpoint{1.737348in}{1.458832in}}{\pgfqpoint{1.729448in}{1.462104in}}{\pgfqpoint{1.721211in}{1.462104in}}%
\pgfpathcurveto{\pgfqpoint{1.712975in}{1.462104in}}{\pgfqpoint{1.705075in}{1.458832in}}{\pgfqpoint{1.699251in}{1.453008in}}%
\pgfpathcurveto{\pgfqpoint{1.693427in}{1.447184in}}{\pgfqpoint{1.690155in}{1.439284in}}{\pgfqpoint{1.690155in}{1.431047in}}%
\pgfpathcurveto{\pgfqpoint{1.690155in}{1.422811in}}{\pgfqpoint{1.693427in}{1.414911in}}{\pgfqpoint{1.699251in}{1.409087in}}%
\pgfpathcurveto{\pgfqpoint{1.705075in}{1.403263in}}{\pgfqpoint{1.712975in}{1.399991in}}{\pgfqpoint{1.721211in}{1.399991in}}%
\pgfpathclose%
\pgfusepath{stroke,fill}%
\end{pgfscope}%
\begin{pgfscope}%
\pgfpathrectangle{\pgfqpoint{0.100000in}{0.212622in}}{\pgfqpoint{3.696000in}{3.696000in}}%
\pgfusepath{clip}%
\pgfsetbuttcap%
\pgfsetroundjoin%
\definecolor{currentfill}{rgb}{0.121569,0.466667,0.705882}%
\pgfsetfillcolor{currentfill}%
\pgfsetfillopacity{0.558155}%
\pgfsetlinewidth{1.003750pt}%
\definecolor{currentstroke}{rgb}{0.121569,0.466667,0.705882}%
\pgfsetstrokecolor{currentstroke}%
\pgfsetstrokeopacity{0.558155}%
\pgfsetdash{}{0pt}%
\pgfpathmoveto{\pgfqpoint{1.726459in}{1.397382in}}%
\pgfpathcurveto{\pgfqpoint{1.734695in}{1.397382in}}{\pgfqpoint{1.742596in}{1.400654in}}{\pgfqpoint{1.748419in}{1.406478in}}%
\pgfpathcurveto{\pgfqpoint{1.754243in}{1.412302in}}{\pgfqpoint{1.757516in}{1.420202in}}{\pgfqpoint{1.757516in}{1.428438in}}%
\pgfpathcurveto{\pgfqpoint{1.757516in}{1.436675in}}{\pgfqpoint{1.754243in}{1.444575in}}{\pgfqpoint{1.748419in}{1.450399in}}%
\pgfpathcurveto{\pgfqpoint{1.742596in}{1.456223in}}{\pgfqpoint{1.734695in}{1.459495in}}{\pgfqpoint{1.726459in}{1.459495in}}%
\pgfpathcurveto{\pgfqpoint{1.718223in}{1.459495in}}{\pgfqpoint{1.710323in}{1.456223in}}{\pgfqpoint{1.704499in}{1.450399in}}%
\pgfpathcurveto{\pgfqpoint{1.698675in}{1.444575in}}{\pgfqpoint{1.695403in}{1.436675in}}{\pgfqpoint{1.695403in}{1.428438in}}%
\pgfpathcurveto{\pgfqpoint{1.695403in}{1.420202in}}{\pgfqpoint{1.698675in}{1.412302in}}{\pgfqpoint{1.704499in}{1.406478in}}%
\pgfpathcurveto{\pgfqpoint{1.710323in}{1.400654in}}{\pgfqpoint{1.718223in}{1.397382in}}{\pgfqpoint{1.726459in}{1.397382in}}%
\pgfpathclose%
\pgfusepath{stroke,fill}%
\end{pgfscope}%
\begin{pgfscope}%
\pgfpathrectangle{\pgfqpoint{0.100000in}{0.212622in}}{\pgfqpoint{3.696000in}{3.696000in}}%
\pgfusepath{clip}%
\pgfsetbuttcap%
\pgfsetroundjoin%
\definecolor{currentfill}{rgb}{0.121569,0.466667,0.705882}%
\pgfsetfillcolor{currentfill}%
\pgfsetfillopacity{0.560619}%
\pgfsetlinewidth{1.003750pt}%
\definecolor{currentstroke}{rgb}{0.121569,0.466667,0.705882}%
\pgfsetstrokecolor{currentstroke}%
\pgfsetstrokeopacity{0.560619}%
\pgfsetdash{}{0pt}%
\pgfpathmoveto{\pgfqpoint{1.734037in}{1.395342in}}%
\pgfpathcurveto{\pgfqpoint{1.742273in}{1.395342in}}{\pgfqpoint{1.750173in}{1.398614in}}{\pgfqpoint{1.755997in}{1.404438in}}%
\pgfpathcurveto{\pgfqpoint{1.761821in}{1.410262in}}{\pgfqpoint{1.765093in}{1.418162in}}{\pgfqpoint{1.765093in}{1.426399in}}%
\pgfpathcurveto{\pgfqpoint{1.765093in}{1.434635in}}{\pgfqpoint{1.761821in}{1.442535in}}{\pgfqpoint{1.755997in}{1.448359in}}%
\pgfpathcurveto{\pgfqpoint{1.750173in}{1.454183in}}{\pgfqpoint{1.742273in}{1.457455in}}{\pgfqpoint{1.734037in}{1.457455in}}%
\pgfpathcurveto{\pgfqpoint{1.725801in}{1.457455in}}{\pgfqpoint{1.717901in}{1.454183in}}{\pgfqpoint{1.712077in}{1.448359in}}%
\pgfpathcurveto{\pgfqpoint{1.706253in}{1.442535in}}{\pgfqpoint{1.702980in}{1.434635in}}{\pgfqpoint{1.702980in}{1.426399in}}%
\pgfpathcurveto{\pgfqpoint{1.702980in}{1.418162in}}{\pgfqpoint{1.706253in}{1.410262in}}{\pgfqpoint{1.712077in}{1.404438in}}%
\pgfpathcurveto{\pgfqpoint{1.717901in}{1.398614in}}{\pgfqpoint{1.725801in}{1.395342in}}{\pgfqpoint{1.734037in}{1.395342in}}%
\pgfpathclose%
\pgfusepath{stroke,fill}%
\end{pgfscope}%
\begin{pgfscope}%
\pgfpathrectangle{\pgfqpoint{0.100000in}{0.212622in}}{\pgfqpoint{3.696000in}{3.696000in}}%
\pgfusepath{clip}%
\pgfsetbuttcap%
\pgfsetroundjoin%
\definecolor{currentfill}{rgb}{0.121569,0.466667,0.705882}%
\pgfsetfillcolor{currentfill}%
\pgfsetfillopacity{0.564163}%
\pgfsetlinewidth{1.003750pt}%
\definecolor{currentstroke}{rgb}{0.121569,0.466667,0.705882}%
\pgfsetstrokecolor{currentstroke}%
\pgfsetstrokeopacity{0.564163}%
\pgfsetdash{}{0pt}%
\pgfpathmoveto{\pgfqpoint{1.742195in}{1.392519in}}%
\pgfpathcurveto{\pgfqpoint{1.750431in}{1.392519in}}{\pgfqpoint{1.758331in}{1.395791in}}{\pgfqpoint{1.764155in}{1.401615in}}%
\pgfpathcurveto{\pgfqpoint{1.769979in}{1.407439in}}{\pgfqpoint{1.773251in}{1.415339in}}{\pgfqpoint{1.773251in}{1.423575in}}%
\pgfpathcurveto{\pgfqpoint{1.773251in}{1.431811in}}{\pgfqpoint{1.769979in}{1.439711in}}{\pgfqpoint{1.764155in}{1.445535in}}%
\pgfpathcurveto{\pgfqpoint{1.758331in}{1.451359in}}{\pgfqpoint{1.750431in}{1.454632in}}{\pgfqpoint{1.742195in}{1.454632in}}%
\pgfpathcurveto{\pgfqpoint{1.733959in}{1.454632in}}{\pgfqpoint{1.726059in}{1.451359in}}{\pgfqpoint{1.720235in}{1.445535in}}%
\pgfpathcurveto{\pgfqpoint{1.714411in}{1.439711in}}{\pgfqpoint{1.711138in}{1.431811in}}{\pgfqpoint{1.711138in}{1.423575in}}%
\pgfpathcurveto{\pgfqpoint{1.711138in}{1.415339in}}{\pgfqpoint{1.714411in}{1.407439in}}{\pgfqpoint{1.720235in}{1.401615in}}%
\pgfpathcurveto{\pgfqpoint{1.726059in}{1.395791in}}{\pgfqpoint{1.733959in}{1.392519in}}{\pgfqpoint{1.742195in}{1.392519in}}%
\pgfpathclose%
\pgfusepath{stroke,fill}%
\end{pgfscope}%
\begin{pgfscope}%
\pgfpathrectangle{\pgfqpoint{0.100000in}{0.212622in}}{\pgfqpoint{3.696000in}{3.696000in}}%
\pgfusepath{clip}%
\pgfsetbuttcap%
\pgfsetroundjoin%
\definecolor{currentfill}{rgb}{0.121569,0.466667,0.705882}%
\pgfsetfillcolor{currentfill}%
\pgfsetfillopacity{0.568191}%
\pgfsetlinewidth{1.003750pt}%
\definecolor{currentstroke}{rgb}{0.121569,0.466667,0.705882}%
\pgfsetstrokecolor{currentstroke}%
\pgfsetstrokeopacity{0.568191}%
\pgfsetdash{}{0pt}%
\pgfpathmoveto{\pgfqpoint{1.750552in}{1.389146in}}%
\pgfpathcurveto{\pgfqpoint{1.758788in}{1.389146in}}{\pgfqpoint{1.766688in}{1.392418in}}{\pgfqpoint{1.772512in}{1.398242in}}%
\pgfpathcurveto{\pgfqpoint{1.778336in}{1.404066in}}{\pgfqpoint{1.781608in}{1.411966in}}{\pgfqpoint{1.781608in}{1.420203in}}%
\pgfpathcurveto{\pgfqpoint{1.781608in}{1.428439in}}{\pgfqpoint{1.778336in}{1.436339in}}{\pgfqpoint{1.772512in}{1.442163in}}%
\pgfpathcurveto{\pgfqpoint{1.766688in}{1.447987in}}{\pgfqpoint{1.758788in}{1.451259in}}{\pgfqpoint{1.750552in}{1.451259in}}%
\pgfpathcurveto{\pgfqpoint{1.742316in}{1.451259in}}{\pgfqpoint{1.734416in}{1.447987in}}{\pgfqpoint{1.728592in}{1.442163in}}%
\pgfpathcurveto{\pgfqpoint{1.722768in}{1.436339in}}{\pgfqpoint{1.719495in}{1.428439in}}{\pgfqpoint{1.719495in}{1.420203in}}%
\pgfpathcurveto{\pgfqpoint{1.719495in}{1.411966in}}{\pgfqpoint{1.722768in}{1.404066in}}{\pgfqpoint{1.728592in}{1.398242in}}%
\pgfpathcurveto{\pgfqpoint{1.734416in}{1.392418in}}{\pgfqpoint{1.742316in}{1.389146in}}{\pgfqpoint{1.750552in}{1.389146in}}%
\pgfpathclose%
\pgfusepath{stroke,fill}%
\end{pgfscope}%
\begin{pgfscope}%
\pgfpathrectangle{\pgfqpoint{0.100000in}{0.212622in}}{\pgfqpoint{3.696000in}{3.696000in}}%
\pgfusepath{clip}%
\pgfsetbuttcap%
\pgfsetroundjoin%
\definecolor{currentfill}{rgb}{0.121569,0.466667,0.705882}%
\pgfsetfillcolor{currentfill}%
\pgfsetfillopacity{0.570181}%
\pgfsetlinewidth{1.003750pt}%
\definecolor{currentstroke}{rgb}{0.121569,0.466667,0.705882}%
\pgfsetstrokecolor{currentstroke}%
\pgfsetstrokeopacity{0.570181}%
\pgfsetdash{}{0pt}%
\pgfpathmoveto{\pgfqpoint{1.755359in}{1.387510in}}%
\pgfpathcurveto{\pgfqpoint{1.763595in}{1.387510in}}{\pgfqpoint{1.771495in}{1.390782in}}{\pgfqpoint{1.777319in}{1.396606in}}%
\pgfpathcurveto{\pgfqpoint{1.783143in}{1.402430in}}{\pgfqpoint{1.786416in}{1.410330in}}{\pgfqpoint{1.786416in}{1.418567in}}%
\pgfpathcurveto{\pgfqpoint{1.786416in}{1.426803in}}{\pgfqpoint{1.783143in}{1.434703in}}{\pgfqpoint{1.777319in}{1.440527in}}%
\pgfpathcurveto{\pgfqpoint{1.771495in}{1.446351in}}{\pgfqpoint{1.763595in}{1.449623in}}{\pgfqpoint{1.755359in}{1.449623in}}%
\pgfpathcurveto{\pgfqpoint{1.747123in}{1.449623in}}{\pgfqpoint{1.739223in}{1.446351in}}{\pgfqpoint{1.733399in}{1.440527in}}%
\pgfpathcurveto{\pgfqpoint{1.727575in}{1.434703in}}{\pgfqpoint{1.724303in}{1.426803in}}{\pgfqpoint{1.724303in}{1.418567in}}%
\pgfpathcurveto{\pgfqpoint{1.724303in}{1.410330in}}{\pgfqpoint{1.727575in}{1.402430in}}{\pgfqpoint{1.733399in}{1.396606in}}%
\pgfpathcurveto{\pgfqpoint{1.739223in}{1.390782in}}{\pgfqpoint{1.747123in}{1.387510in}}{\pgfqpoint{1.755359in}{1.387510in}}%
\pgfpathclose%
\pgfusepath{stroke,fill}%
\end{pgfscope}%
\begin{pgfscope}%
\pgfpathrectangle{\pgfqpoint{0.100000in}{0.212622in}}{\pgfqpoint{3.696000in}{3.696000in}}%
\pgfusepath{clip}%
\pgfsetbuttcap%
\pgfsetroundjoin%
\definecolor{currentfill}{rgb}{0.121569,0.466667,0.705882}%
\pgfsetfillcolor{currentfill}%
\pgfsetfillopacity{0.571106}%
\pgfsetlinewidth{1.003750pt}%
\definecolor{currentstroke}{rgb}{0.121569,0.466667,0.705882}%
\pgfsetstrokecolor{currentstroke}%
\pgfsetstrokeopacity{0.571106}%
\pgfsetdash{}{0pt}%
\pgfpathmoveto{\pgfqpoint{1.758167in}{1.386797in}}%
\pgfpathcurveto{\pgfqpoint{1.766403in}{1.386797in}}{\pgfqpoint{1.774303in}{1.390070in}}{\pgfqpoint{1.780127in}{1.395894in}}%
\pgfpathcurveto{\pgfqpoint{1.785951in}{1.401718in}}{\pgfqpoint{1.789224in}{1.409618in}}{\pgfqpoint{1.789224in}{1.417854in}}%
\pgfpathcurveto{\pgfqpoint{1.789224in}{1.426090in}}{\pgfqpoint{1.785951in}{1.433990in}}{\pgfqpoint{1.780127in}{1.439814in}}%
\pgfpathcurveto{\pgfqpoint{1.774303in}{1.445638in}}{\pgfqpoint{1.766403in}{1.448910in}}{\pgfqpoint{1.758167in}{1.448910in}}%
\pgfpathcurveto{\pgfqpoint{1.749931in}{1.448910in}}{\pgfqpoint{1.742031in}{1.445638in}}{\pgfqpoint{1.736207in}{1.439814in}}%
\pgfpathcurveto{\pgfqpoint{1.730383in}{1.433990in}}{\pgfqpoint{1.727111in}{1.426090in}}{\pgfqpoint{1.727111in}{1.417854in}}%
\pgfpathcurveto{\pgfqpoint{1.727111in}{1.409618in}}{\pgfqpoint{1.730383in}{1.401718in}}{\pgfqpoint{1.736207in}{1.395894in}}%
\pgfpathcurveto{\pgfqpoint{1.742031in}{1.390070in}}{\pgfqpoint{1.749931in}{1.386797in}}{\pgfqpoint{1.758167in}{1.386797in}}%
\pgfpathclose%
\pgfusepath{stroke,fill}%
\end{pgfscope}%
\begin{pgfscope}%
\pgfpathrectangle{\pgfqpoint{0.100000in}{0.212622in}}{\pgfqpoint{3.696000in}{3.696000in}}%
\pgfusepath{clip}%
\pgfsetbuttcap%
\pgfsetroundjoin%
\definecolor{currentfill}{rgb}{0.121569,0.466667,0.705882}%
\pgfsetfillcolor{currentfill}%
\pgfsetfillopacity{0.572833}%
\pgfsetlinewidth{1.003750pt}%
\definecolor{currentstroke}{rgb}{0.121569,0.466667,0.705882}%
\pgfsetstrokecolor{currentstroke}%
\pgfsetstrokeopacity{0.572833}%
\pgfsetdash{}{0pt}%
\pgfpathmoveto{\pgfqpoint{1.762054in}{1.385388in}}%
\pgfpathcurveto{\pgfqpoint{1.770290in}{1.385388in}}{\pgfqpoint{1.778191in}{1.388660in}}{\pgfqpoint{1.784014in}{1.394484in}}%
\pgfpathcurveto{\pgfqpoint{1.789838in}{1.400308in}}{\pgfqpoint{1.793111in}{1.408208in}}{\pgfqpoint{1.793111in}{1.416445in}}%
\pgfpathcurveto{\pgfqpoint{1.793111in}{1.424681in}}{\pgfqpoint{1.789838in}{1.432581in}}{\pgfqpoint{1.784014in}{1.438405in}}%
\pgfpathcurveto{\pgfqpoint{1.778191in}{1.444229in}}{\pgfqpoint{1.770290in}{1.447501in}}{\pgfqpoint{1.762054in}{1.447501in}}%
\pgfpathcurveto{\pgfqpoint{1.753818in}{1.447501in}}{\pgfqpoint{1.745918in}{1.444229in}}{\pgfqpoint{1.740094in}{1.438405in}}%
\pgfpathcurveto{\pgfqpoint{1.734270in}{1.432581in}}{\pgfqpoint{1.730998in}{1.424681in}}{\pgfqpoint{1.730998in}{1.416445in}}%
\pgfpathcurveto{\pgfqpoint{1.730998in}{1.408208in}}{\pgfqpoint{1.734270in}{1.400308in}}{\pgfqpoint{1.740094in}{1.394484in}}%
\pgfpathcurveto{\pgfqpoint{1.745918in}{1.388660in}}{\pgfqpoint{1.753818in}{1.385388in}}{\pgfqpoint{1.762054in}{1.385388in}}%
\pgfpathclose%
\pgfusepath{stroke,fill}%
\end{pgfscope}%
\begin{pgfscope}%
\pgfpathrectangle{\pgfqpoint{0.100000in}{0.212622in}}{\pgfqpoint{3.696000in}{3.696000in}}%
\pgfusepath{clip}%
\pgfsetbuttcap%
\pgfsetroundjoin%
\definecolor{currentfill}{rgb}{0.121569,0.466667,0.705882}%
\pgfsetfillcolor{currentfill}%
\pgfsetfillopacity{0.574522}%
\pgfsetlinewidth{1.003750pt}%
\definecolor{currentstroke}{rgb}{0.121569,0.466667,0.705882}%
\pgfsetstrokecolor{currentstroke}%
\pgfsetstrokeopacity{0.574522}%
\pgfsetdash{}{0pt}%
\pgfpathmoveto{\pgfqpoint{1.767058in}{1.384045in}}%
\pgfpathcurveto{\pgfqpoint{1.775294in}{1.384045in}}{\pgfqpoint{1.783194in}{1.387318in}}{\pgfqpoint{1.789018in}{1.393141in}}%
\pgfpathcurveto{\pgfqpoint{1.794842in}{1.398965in}}{\pgfqpoint{1.798115in}{1.406865in}}{\pgfqpoint{1.798115in}{1.415102in}}%
\pgfpathcurveto{\pgfqpoint{1.798115in}{1.423338in}}{\pgfqpoint{1.794842in}{1.431238in}}{\pgfqpoint{1.789018in}{1.437062in}}%
\pgfpathcurveto{\pgfqpoint{1.783194in}{1.442886in}}{\pgfqpoint{1.775294in}{1.446158in}}{\pgfqpoint{1.767058in}{1.446158in}}%
\pgfpathcurveto{\pgfqpoint{1.758822in}{1.446158in}}{\pgfqpoint{1.750922in}{1.442886in}}{\pgfqpoint{1.745098in}{1.437062in}}%
\pgfpathcurveto{\pgfqpoint{1.739274in}{1.431238in}}{\pgfqpoint{1.736002in}{1.423338in}}{\pgfqpoint{1.736002in}{1.415102in}}%
\pgfpathcurveto{\pgfqpoint{1.736002in}{1.406865in}}{\pgfqpoint{1.739274in}{1.398965in}}{\pgfqpoint{1.745098in}{1.393141in}}%
\pgfpathcurveto{\pgfqpoint{1.750922in}{1.387318in}}{\pgfqpoint{1.758822in}{1.384045in}}{\pgfqpoint{1.767058in}{1.384045in}}%
\pgfpathclose%
\pgfusepath{stroke,fill}%
\end{pgfscope}%
\begin{pgfscope}%
\pgfpathrectangle{\pgfqpoint{0.100000in}{0.212622in}}{\pgfqpoint{3.696000in}{3.696000in}}%
\pgfusepath{clip}%
\pgfsetbuttcap%
\pgfsetroundjoin%
\definecolor{currentfill}{rgb}{0.121569,0.466667,0.705882}%
\pgfsetfillcolor{currentfill}%
\pgfsetfillopacity{0.575539}%
\pgfsetlinewidth{1.003750pt}%
\definecolor{currentstroke}{rgb}{0.121569,0.466667,0.705882}%
\pgfsetstrokecolor{currentstroke}%
\pgfsetstrokeopacity{0.575539}%
\pgfsetdash{}{0pt}%
\pgfpathmoveto{\pgfqpoint{1.769722in}{1.383194in}}%
\pgfpathcurveto{\pgfqpoint{1.777958in}{1.383194in}}{\pgfqpoint{1.785858in}{1.386466in}}{\pgfqpoint{1.791682in}{1.392290in}}%
\pgfpathcurveto{\pgfqpoint{1.797506in}{1.398114in}}{\pgfqpoint{1.800779in}{1.406014in}}{\pgfqpoint{1.800779in}{1.414250in}}%
\pgfpathcurveto{\pgfqpoint{1.800779in}{1.422487in}}{\pgfqpoint{1.797506in}{1.430387in}}{\pgfqpoint{1.791682in}{1.436211in}}%
\pgfpathcurveto{\pgfqpoint{1.785858in}{1.442035in}}{\pgfqpoint{1.777958in}{1.445307in}}{\pgfqpoint{1.769722in}{1.445307in}}%
\pgfpathcurveto{\pgfqpoint{1.761486in}{1.445307in}}{\pgfqpoint{1.753586in}{1.442035in}}{\pgfqpoint{1.747762in}{1.436211in}}%
\pgfpathcurveto{\pgfqpoint{1.741938in}{1.430387in}}{\pgfqpoint{1.738666in}{1.422487in}}{\pgfqpoint{1.738666in}{1.414250in}}%
\pgfpathcurveto{\pgfqpoint{1.738666in}{1.406014in}}{\pgfqpoint{1.741938in}{1.398114in}}{\pgfqpoint{1.747762in}{1.392290in}}%
\pgfpathcurveto{\pgfqpoint{1.753586in}{1.386466in}}{\pgfqpoint{1.761486in}{1.383194in}}{\pgfqpoint{1.769722in}{1.383194in}}%
\pgfpathclose%
\pgfusepath{stroke,fill}%
\end{pgfscope}%
\begin{pgfscope}%
\pgfpathrectangle{\pgfqpoint{0.100000in}{0.212622in}}{\pgfqpoint{3.696000in}{3.696000in}}%
\pgfusepath{clip}%
\pgfsetbuttcap%
\pgfsetroundjoin%
\definecolor{currentfill}{rgb}{0.121569,0.466667,0.705882}%
\pgfsetfillcolor{currentfill}%
\pgfsetfillopacity{0.576931}%
\pgfsetlinewidth{1.003750pt}%
\definecolor{currentstroke}{rgb}{0.121569,0.466667,0.705882}%
\pgfsetstrokecolor{currentstroke}%
\pgfsetstrokeopacity{0.576931}%
\pgfsetdash{}{0pt}%
\pgfpathmoveto{\pgfqpoint{1.772662in}{1.381950in}}%
\pgfpathcurveto{\pgfqpoint{1.780898in}{1.381950in}}{\pgfqpoint{1.788798in}{1.385222in}}{\pgfqpoint{1.794622in}{1.391046in}}%
\pgfpathcurveto{\pgfqpoint{1.800446in}{1.396870in}}{\pgfqpoint{1.803718in}{1.404770in}}{\pgfqpoint{1.803718in}{1.413007in}}%
\pgfpathcurveto{\pgfqpoint{1.803718in}{1.421243in}}{\pgfqpoint{1.800446in}{1.429143in}}{\pgfqpoint{1.794622in}{1.434967in}}%
\pgfpathcurveto{\pgfqpoint{1.788798in}{1.440791in}}{\pgfqpoint{1.780898in}{1.444063in}}{\pgfqpoint{1.772662in}{1.444063in}}%
\pgfpathcurveto{\pgfqpoint{1.764426in}{1.444063in}}{\pgfqpoint{1.756526in}{1.440791in}}{\pgfqpoint{1.750702in}{1.434967in}}%
\pgfpathcurveto{\pgfqpoint{1.744878in}{1.429143in}}{\pgfqpoint{1.741605in}{1.421243in}}{\pgfqpoint{1.741605in}{1.413007in}}%
\pgfpathcurveto{\pgfqpoint{1.741605in}{1.404770in}}{\pgfqpoint{1.744878in}{1.396870in}}{\pgfqpoint{1.750702in}{1.391046in}}%
\pgfpathcurveto{\pgfqpoint{1.756526in}{1.385222in}}{\pgfqpoint{1.764426in}{1.381950in}}{\pgfqpoint{1.772662in}{1.381950in}}%
\pgfpathclose%
\pgfusepath{stroke,fill}%
\end{pgfscope}%
\begin{pgfscope}%
\pgfpathrectangle{\pgfqpoint{0.100000in}{0.212622in}}{\pgfqpoint{3.696000in}{3.696000in}}%
\pgfusepath{clip}%
\pgfsetbuttcap%
\pgfsetroundjoin%
\definecolor{currentfill}{rgb}{0.121569,0.466667,0.705882}%
\pgfsetfillcolor{currentfill}%
\pgfsetfillopacity{0.578749}%
\pgfsetlinewidth{1.003750pt}%
\definecolor{currentstroke}{rgb}{0.121569,0.466667,0.705882}%
\pgfsetstrokecolor{currentstroke}%
\pgfsetstrokeopacity{0.578749}%
\pgfsetdash{}{0pt}%
\pgfpathmoveto{\pgfqpoint{1.776761in}{1.380424in}}%
\pgfpathcurveto{\pgfqpoint{1.784998in}{1.380424in}}{\pgfqpoint{1.792898in}{1.383696in}}{\pgfqpoint{1.798722in}{1.389520in}}%
\pgfpathcurveto{\pgfqpoint{1.804546in}{1.395344in}}{\pgfqpoint{1.807818in}{1.403244in}}{\pgfqpoint{1.807818in}{1.411480in}}%
\pgfpathcurveto{\pgfqpoint{1.807818in}{1.419717in}}{\pgfqpoint{1.804546in}{1.427617in}}{\pgfqpoint{1.798722in}{1.433441in}}%
\pgfpathcurveto{\pgfqpoint{1.792898in}{1.439265in}}{\pgfqpoint{1.784998in}{1.442537in}}{\pgfqpoint{1.776761in}{1.442537in}}%
\pgfpathcurveto{\pgfqpoint{1.768525in}{1.442537in}}{\pgfqpoint{1.760625in}{1.439265in}}{\pgfqpoint{1.754801in}{1.433441in}}%
\pgfpathcurveto{\pgfqpoint{1.748977in}{1.427617in}}{\pgfqpoint{1.745705in}{1.419717in}}{\pgfqpoint{1.745705in}{1.411480in}}%
\pgfpathcurveto{\pgfqpoint{1.745705in}{1.403244in}}{\pgfqpoint{1.748977in}{1.395344in}}{\pgfqpoint{1.754801in}{1.389520in}}%
\pgfpathcurveto{\pgfqpoint{1.760625in}{1.383696in}}{\pgfqpoint{1.768525in}{1.380424in}}{\pgfqpoint{1.776761in}{1.380424in}}%
\pgfpathclose%
\pgfusepath{stroke,fill}%
\end{pgfscope}%
\begin{pgfscope}%
\pgfpathrectangle{\pgfqpoint{0.100000in}{0.212622in}}{\pgfqpoint{3.696000in}{3.696000in}}%
\pgfusepath{clip}%
\pgfsetbuttcap%
\pgfsetroundjoin%
\definecolor{currentfill}{rgb}{0.121569,0.466667,0.705882}%
\pgfsetfillcolor{currentfill}%
\pgfsetfillopacity{0.580513}%
\pgfsetlinewidth{1.003750pt}%
\definecolor{currentstroke}{rgb}{0.121569,0.466667,0.705882}%
\pgfsetstrokecolor{currentstroke}%
\pgfsetstrokeopacity{0.580513}%
\pgfsetdash{}{0pt}%
\pgfpathmoveto{\pgfqpoint{1.782952in}{1.378981in}}%
\pgfpathcurveto{\pgfqpoint{1.791189in}{1.378981in}}{\pgfqpoint{1.799089in}{1.382253in}}{\pgfqpoint{1.804913in}{1.388077in}}%
\pgfpathcurveto{\pgfqpoint{1.810737in}{1.393901in}}{\pgfqpoint{1.814009in}{1.401801in}}{\pgfqpoint{1.814009in}{1.410037in}}%
\pgfpathcurveto{\pgfqpoint{1.814009in}{1.418274in}}{\pgfqpoint{1.810737in}{1.426174in}}{\pgfqpoint{1.804913in}{1.431998in}}%
\pgfpathcurveto{\pgfqpoint{1.799089in}{1.437821in}}{\pgfqpoint{1.791189in}{1.441094in}}{\pgfqpoint{1.782952in}{1.441094in}}%
\pgfpathcurveto{\pgfqpoint{1.774716in}{1.441094in}}{\pgfqpoint{1.766816in}{1.437821in}}{\pgfqpoint{1.760992in}{1.431998in}}%
\pgfpathcurveto{\pgfqpoint{1.755168in}{1.426174in}}{\pgfqpoint{1.751896in}{1.418274in}}{\pgfqpoint{1.751896in}{1.410037in}}%
\pgfpathcurveto{\pgfqpoint{1.751896in}{1.401801in}}{\pgfqpoint{1.755168in}{1.393901in}}{\pgfqpoint{1.760992in}{1.388077in}}%
\pgfpathcurveto{\pgfqpoint{1.766816in}{1.382253in}}{\pgfqpoint{1.774716in}{1.378981in}}{\pgfqpoint{1.782952in}{1.378981in}}%
\pgfpathclose%
\pgfusepath{stroke,fill}%
\end{pgfscope}%
\begin{pgfscope}%
\pgfpathrectangle{\pgfqpoint{0.100000in}{0.212622in}}{\pgfqpoint{3.696000in}{3.696000in}}%
\pgfusepath{clip}%
\pgfsetbuttcap%
\pgfsetroundjoin%
\definecolor{currentfill}{rgb}{0.121569,0.466667,0.705882}%
\pgfsetfillcolor{currentfill}%
\pgfsetfillopacity{0.583310}%
\pgfsetlinewidth{1.003750pt}%
\definecolor{currentstroke}{rgb}{0.121569,0.466667,0.705882}%
\pgfsetstrokecolor{currentstroke}%
\pgfsetstrokeopacity{0.583310}%
\pgfsetdash{}{0pt}%
\pgfpathmoveto{\pgfqpoint{1.789768in}{1.376784in}}%
\pgfpathcurveto{\pgfqpoint{1.798004in}{1.376784in}}{\pgfqpoint{1.805904in}{1.380056in}}{\pgfqpoint{1.811728in}{1.385880in}}%
\pgfpathcurveto{\pgfqpoint{1.817552in}{1.391704in}}{\pgfqpoint{1.820824in}{1.399604in}}{\pgfqpoint{1.820824in}{1.407840in}}%
\pgfpathcurveto{\pgfqpoint{1.820824in}{1.416076in}}{\pgfqpoint{1.817552in}{1.423976in}}{\pgfqpoint{1.811728in}{1.429800in}}%
\pgfpathcurveto{\pgfqpoint{1.805904in}{1.435624in}}{\pgfqpoint{1.798004in}{1.438897in}}{\pgfqpoint{1.789768in}{1.438897in}}%
\pgfpathcurveto{\pgfqpoint{1.781531in}{1.438897in}}{\pgfqpoint{1.773631in}{1.435624in}}{\pgfqpoint{1.767807in}{1.429800in}}%
\pgfpathcurveto{\pgfqpoint{1.761983in}{1.423976in}}{\pgfqpoint{1.758711in}{1.416076in}}{\pgfqpoint{1.758711in}{1.407840in}}%
\pgfpathcurveto{\pgfqpoint{1.758711in}{1.399604in}}{\pgfqpoint{1.761983in}{1.391704in}}{\pgfqpoint{1.767807in}{1.385880in}}%
\pgfpathcurveto{\pgfqpoint{1.773631in}{1.380056in}}{\pgfqpoint{1.781531in}{1.376784in}}{\pgfqpoint{1.789768in}{1.376784in}}%
\pgfpathclose%
\pgfusepath{stroke,fill}%
\end{pgfscope}%
\begin{pgfscope}%
\pgfpathrectangle{\pgfqpoint{0.100000in}{0.212622in}}{\pgfqpoint{3.696000in}{3.696000in}}%
\pgfusepath{clip}%
\pgfsetbuttcap%
\pgfsetroundjoin%
\definecolor{currentfill}{rgb}{0.121569,0.466667,0.705882}%
\pgfsetfillcolor{currentfill}%
\pgfsetfillopacity{0.586376}%
\pgfsetlinewidth{1.003750pt}%
\definecolor{currentstroke}{rgb}{0.121569,0.466667,0.705882}%
\pgfsetstrokecolor{currentstroke}%
\pgfsetstrokeopacity{0.586376}%
\pgfsetdash{}{0pt}%
\pgfpathmoveto{\pgfqpoint{1.797053in}{1.374120in}}%
\pgfpathcurveto{\pgfqpoint{1.805289in}{1.374120in}}{\pgfqpoint{1.813189in}{1.377392in}}{\pgfqpoint{1.819013in}{1.383216in}}%
\pgfpathcurveto{\pgfqpoint{1.824837in}{1.389040in}}{\pgfqpoint{1.828109in}{1.396940in}}{\pgfqpoint{1.828109in}{1.405176in}}%
\pgfpathcurveto{\pgfqpoint{1.828109in}{1.413413in}}{\pgfqpoint{1.824837in}{1.421313in}}{\pgfqpoint{1.819013in}{1.427137in}}%
\pgfpathcurveto{\pgfqpoint{1.813189in}{1.432960in}}{\pgfqpoint{1.805289in}{1.436233in}}{\pgfqpoint{1.797053in}{1.436233in}}%
\pgfpathcurveto{\pgfqpoint{1.788817in}{1.436233in}}{\pgfqpoint{1.780917in}{1.432960in}}{\pgfqpoint{1.775093in}{1.427137in}}%
\pgfpathcurveto{\pgfqpoint{1.769269in}{1.421313in}}{\pgfqpoint{1.765996in}{1.413413in}}{\pgfqpoint{1.765996in}{1.405176in}}%
\pgfpathcurveto{\pgfqpoint{1.765996in}{1.396940in}}{\pgfqpoint{1.769269in}{1.389040in}}{\pgfqpoint{1.775093in}{1.383216in}}%
\pgfpathcurveto{\pgfqpoint{1.780917in}{1.377392in}}{\pgfqpoint{1.788817in}{1.374120in}}{\pgfqpoint{1.797053in}{1.374120in}}%
\pgfpathclose%
\pgfusepath{stroke,fill}%
\end{pgfscope}%
\begin{pgfscope}%
\pgfpathrectangle{\pgfqpoint{0.100000in}{0.212622in}}{\pgfqpoint{3.696000in}{3.696000in}}%
\pgfusepath{clip}%
\pgfsetbuttcap%
\pgfsetroundjoin%
\definecolor{currentfill}{rgb}{0.121569,0.466667,0.705882}%
\pgfsetfillcolor{currentfill}%
\pgfsetfillopacity{0.589885}%
\pgfsetlinewidth{1.003750pt}%
\definecolor{currentstroke}{rgb}{0.121569,0.466667,0.705882}%
\pgfsetstrokecolor{currentstroke}%
\pgfsetstrokeopacity{0.589885}%
\pgfsetdash{}{0pt}%
\pgfpathmoveto{\pgfqpoint{1.804539in}{1.370951in}}%
\pgfpathcurveto{\pgfqpoint{1.812775in}{1.370951in}}{\pgfqpoint{1.820675in}{1.374223in}}{\pgfqpoint{1.826499in}{1.380047in}}%
\pgfpathcurveto{\pgfqpoint{1.832323in}{1.385871in}}{\pgfqpoint{1.835595in}{1.393771in}}{\pgfqpoint{1.835595in}{1.402007in}}%
\pgfpathcurveto{\pgfqpoint{1.835595in}{1.410244in}}{\pgfqpoint{1.832323in}{1.418144in}}{\pgfqpoint{1.826499in}{1.423968in}}%
\pgfpathcurveto{\pgfqpoint{1.820675in}{1.429792in}}{\pgfqpoint{1.812775in}{1.433064in}}{\pgfqpoint{1.804539in}{1.433064in}}%
\pgfpathcurveto{\pgfqpoint{1.796303in}{1.433064in}}{\pgfqpoint{1.788402in}{1.429792in}}{\pgfqpoint{1.782579in}{1.423968in}}%
\pgfpathcurveto{\pgfqpoint{1.776755in}{1.418144in}}{\pgfqpoint{1.773482in}{1.410244in}}{\pgfqpoint{1.773482in}{1.402007in}}%
\pgfpathcurveto{\pgfqpoint{1.773482in}{1.393771in}}{\pgfqpoint{1.776755in}{1.385871in}}{\pgfqpoint{1.782579in}{1.380047in}}%
\pgfpathcurveto{\pgfqpoint{1.788402in}{1.374223in}}{\pgfqpoint{1.796303in}{1.370951in}}{\pgfqpoint{1.804539in}{1.370951in}}%
\pgfpathclose%
\pgfusepath{stroke,fill}%
\end{pgfscope}%
\begin{pgfscope}%
\pgfpathrectangle{\pgfqpoint{0.100000in}{0.212622in}}{\pgfqpoint{3.696000in}{3.696000in}}%
\pgfusepath{clip}%
\pgfsetbuttcap%
\pgfsetroundjoin%
\definecolor{currentfill}{rgb}{0.121569,0.466667,0.705882}%
\pgfsetfillcolor{currentfill}%
\pgfsetfillopacity{0.591770}%
\pgfsetlinewidth{1.003750pt}%
\definecolor{currentstroke}{rgb}{0.121569,0.466667,0.705882}%
\pgfsetstrokecolor{currentstroke}%
\pgfsetstrokeopacity{0.591770}%
\pgfsetdash{}{0pt}%
\pgfpathmoveto{\pgfqpoint{1.808720in}{1.369321in}}%
\pgfpathcurveto{\pgfqpoint{1.816956in}{1.369321in}}{\pgfqpoint{1.824856in}{1.372594in}}{\pgfqpoint{1.830680in}{1.378417in}}%
\pgfpathcurveto{\pgfqpoint{1.836504in}{1.384241in}}{\pgfqpoint{1.839776in}{1.392141in}}{\pgfqpoint{1.839776in}{1.400378in}}%
\pgfpathcurveto{\pgfqpoint{1.839776in}{1.408614in}}{\pgfqpoint{1.836504in}{1.416514in}}{\pgfqpoint{1.830680in}{1.422338in}}%
\pgfpathcurveto{\pgfqpoint{1.824856in}{1.428162in}}{\pgfqpoint{1.816956in}{1.431434in}}{\pgfqpoint{1.808720in}{1.431434in}}%
\pgfpathcurveto{\pgfqpoint{1.800484in}{1.431434in}}{\pgfqpoint{1.792584in}{1.428162in}}{\pgfqpoint{1.786760in}{1.422338in}}%
\pgfpathcurveto{\pgfqpoint{1.780936in}{1.416514in}}{\pgfqpoint{1.777663in}{1.408614in}}{\pgfqpoint{1.777663in}{1.400378in}}%
\pgfpathcurveto{\pgfqpoint{1.777663in}{1.392141in}}{\pgfqpoint{1.780936in}{1.384241in}}{\pgfqpoint{1.786760in}{1.378417in}}%
\pgfpathcurveto{\pgfqpoint{1.792584in}{1.372594in}}{\pgfqpoint{1.800484in}{1.369321in}}{\pgfqpoint{1.808720in}{1.369321in}}%
\pgfpathclose%
\pgfusepath{stroke,fill}%
\end{pgfscope}%
\begin{pgfscope}%
\pgfpathrectangle{\pgfqpoint{0.100000in}{0.212622in}}{\pgfqpoint{3.696000in}{3.696000in}}%
\pgfusepath{clip}%
\pgfsetbuttcap%
\pgfsetroundjoin%
\definecolor{currentfill}{rgb}{0.121569,0.466667,0.705882}%
\pgfsetfillcolor{currentfill}%
\pgfsetfillopacity{0.593664}%
\pgfsetlinewidth{1.003750pt}%
\definecolor{currentstroke}{rgb}{0.121569,0.466667,0.705882}%
\pgfsetstrokecolor{currentstroke}%
\pgfsetstrokeopacity{0.593664}%
\pgfsetdash{}{0pt}%
\pgfpathmoveto{\pgfqpoint{1.814740in}{1.367706in}}%
\pgfpathcurveto{\pgfqpoint{1.822976in}{1.367706in}}{\pgfqpoint{1.830876in}{1.370978in}}{\pgfqpoint{1.836700in}{1.376802in}}%
\pgfpathcurveto{\pgfqpoint{1.842524in}{1.382626in}}{\pgfqpoint{1.845796in}{1.390526in}}{\pgfqpoint{1.845796in}{1.398762in}}%
\pgfpathcurveto{\pgfqpoint{1.845796in}{1.406998in}}{\pgfqpoint{1.842524in}{1.414898in}}{\pgfqpoint{1.836700in}{1.420722in}}%
\pgfpathcurveto{\pgfqpoint{1.830876in}{1.426546in}}{\pgfqpoint{1.822976in}{1.429819in}}{\pgfqpoint{1.814740in}{1.429819in}}%
\pgfpathcurveto{\pgfqpoint{1.806504in}{1.429819in}}{\pgfqpoint{1.798604in}{1.426546in}}{\pgfqpoint{1.792780in}{1.420722in}}%
\pgfpathcurveto{\pgfqpoint{1.786956in}{1.414898in}}{\pgfqpoint{1.783683in}{1.406998in}}{\pgfqpoint{1.783683in}{1.398762in}}%
\pgfpathcurveto{\pgfqpoint{1.783683in}{1.390526in}}{\pgfqpoint{1.786956in}{1.382626in}}{\pgfqpoint{1.792780in}{1.376802in}}%
\pgfpathcurveto{\pgfqpoint{1.798604in}{1.370978in}}{\pgfqpoint{1.806504in}{1.367706in}}{\pgfqpoint{1.814740in}{1.367706in}}%
\pgfpathclose%
\pgfusepath{stroke,fill}%
\end{pgfscope}%
\begin{pgfscope}%
\pgfpathrectangle{\pgfqpoint{0.100000in}{0.212622in}}{\pgfqpoint{3.696000in}{3.696000in}}%
\pgfusepath{clip}%
\pgfsetbuttcap%
\pgfsetroundjoin%
\definecolor{currentfill}{rgb}{0.121569,0.466667,0.705882}%
\pgfsetfillcolor{currentfill}%
\pgfsetfillopacity{0.595392}%
\pgfsetlinewidth{1.003750pt}%
\definecolor{currentstroke}{rgb}{0.121569,0.466667,0.705882}%
\pgfsetstrokecolor{currentstroke}%
\pgfsetstrokeopacity{0.595392}%
\pgfsetdash{}{0pt}%
\pgfpathmoveto{\pgfqpoint{1.822685in}{1.366670in}}%
\pgfpathcurveto{\pgfqpoint{1.830921in}{1.366670in}}{\pgfqpoint{1.838821in}{1.369942in}}{\pgfqpoint{1.844645in}{1.375766in}}%
\pgfpathcurveto{\pgfqpoint{1.850469in}{1.381590in}}{\pgfqpoint{1.853741in}{1.389490in}}{\pgfqpoint{1.853741in}{1.397727in}}%
\pgfpathcurveto{\pgfqpoint{1.853741in}{1.405963in}}{\pgfqpoint{1.850469in}{1.413863in}}{\pgfqpoint{1.844645in}{1.419687in}}%
\pgfpathcurveto{\pgfqpoint{1.838821in}{1.425511in}}{\pgfqpoint{1.830921in}{1.428783in}}{\pgfqpoint{1.822685in}{1.428783in}}%
\pgfpathcurveto{\pgfqpoint{1.814448in}{1.428783in}}{\pgfqpoint{1.806548in}{1.425511in}}{\pgfqpoint{1.800724in}{1.419687in}}%
\pgfpathcurveto{\pgfqpoint{1.794900in}{1.413863in}}{\pgfqpoint{1.791628in}{1.405963in}}{\pgfqpoint{1.791628in}{1.397727in}}%
\pgfpathcurveto{\pgfqpoint{1.791628in}{1.389490in}}{\pgfqpoint{1.794900in}{1.381590in}}{\pgfqpoint{1.800724in}{1.375766in}}%
\pgfpathcurveto{\pgfqpoint{1.806548in}{1.369942in}}{\pgfqpoint{1.814448in}{1.366670in}}{\pgfqpoint{1.822685in}{1.366670in}}%
\pgfpathclose%
\pgfusepath{stroke,fill}%
\end{pgfscope}%
\begin{pgfscope}%
\pgfpathrectangle{\pgfqpoint{0.100000in}{0.212622in}}{\pgfqpoint{3.696000in}{3.696000in}}%
\pgfusepath{clip}%
\pgfsetbuttcap%
\pgfsetroundjoin%
\definecolor{currentfill}{rgb}{0.121569,0.466667,0.705882}%
\pgfsetfillcolor{currentfill}%
\pgfsetfillopacity{0.598496}%
\pgfsetlinewidth{1.003750pt}%
\definecolor{currentstroke}{rgb}{0.121569,0.466667,0.705882}%
\pgfsetstrokecolor{currentstroke}%
\pgfsetstrokeopacity{0.598496}%
\pgfsetdash{}{0pt}%
\pgfpathmoveto{\pgfqpoint{1.830797in}{1.363926in}}%
\pgfpathcurveto{\pgfqpoint{1.839033in}{1.363926in}}{\pgfqpoint{1.846933in}{1.367199in}}{\pgfqpoint{1.852757in}{1.373023in}}%
\pgfpathcurveto{\pgfqpoint{1.858581in}{1.378847in}}{\pgfqpoint{1.861853in}{1.386747in}}{\pgfqpoint{1.861853in}{1.394983in}}%
\pgfpathcurveto{\pgfqpoint{1.861853in}{1.403219in}}{\pgfqpoint{1.858581in}{1.411119in}}{\pgfqpoint{1.852757in}{1.416943in}}%
\pgfpathcurveto{\pgfqpoint{1.846933in}{1.422767in}}{\pgfqpoint{1.839033in}{1.426039in}}{\pgfqpoint{1.830797in}{1.426039in}}%
\pgfpathcurveto{\pgfqpoint{1.822561in}{1.426039in}}{\pgfqpoint{1.814661in}{1.422767in}}{\pgfqpoint{1.808837in}{1.416943in}}%
\pgfpathcurveto{\pgfqpoint{1.803013in}{1.411119in}}{\pgfqpoint{1.799740in}{1.403219in}}{\pgfqpoint{1.799740in}{1.394983in}}%
\pgfpathcurveto{\pgfqpoint{1.799740in}{1.386747in}}{\pgfqpoint{1.803013in}{1.378847in}}{\pgfqpoint{1.808837in}{1.373023in}}%
\pgfpathcurveto{\pgfqpoint{1.814661in}{1.367199in}}{\pgfqpoint{1.822561in}{1.363926in}}{\pgfqpoint{1.830797in}{1.363926in}}%
\pgfpathclose%
\pgfusepath{stroke,fill}%
\end{pgfscope}%
\begin{pgfscope}%
\pgfpathrectangle{\pgfqpoint{0.100000in}{0.212622in}}{\pgfqpoint{3.696000in}{3.696000in}}%
\pgfusepath{clip}%
\pgfsetbuttcap%
\pgfsetroundjoin%
\definecolor{currentfill}{rgb}{0.121569,0.466667,0.705882}%
\pgfsetfillcolor{currentfill}%
\pgfsetfillopacity{0.601290}%
\pgfsetlinewidth{1.003750pt}%
\definecolor{currentstroke}{rgb}{0.121569,0.466667,0.705882}%
\pgfsetstrokecolor{currentstroke}%
\pgfsetstrokeopacity{0.601290}%
\pgfsetdash{}{0pt}%
\pgfpathmoveto{\pgfqpoint{1.839952in}{1.361832in}}%
\pgfpathcurveto{\pgfqpoint{1.848189in}{1.361832in}}{\pgfqpoint{1.856089in}{1.365104in}}{\pgfqpoint{1.861913in}{1.370928in}}%
\pgfpathcurveto{\pgfqpoint{1.867737in}{1.376752in}}{\pgfqpoint{1.871009in}{1.384652in}}{\pgfqpoint{1.871009in}{1.392889in}}%
\pgfpathcurveto{\pgfqpoint{1.871009in}{1.401125in}}{\pgfqpoint{1.867737in}{1.409025in}}{\pgfqpoint{1.861913in}{1.414849in}}%
\pgfpathcurveto{\pgfqpoint{1.856089in}{1.420673in}}{\pgfqpoint{1.848189in}{1.423945in}}{\pgfqpoint{1.839952in}{1.423945in}}%
\pgfpathcurveto{\pgfqpoint{1.831716in}{1.423945in}}{\pgfqpoint{1.823816in}{1.420673in}}{\pgfqpoint{1.817992in}{1.414849in}}%
\pgfpathcurveto{\pgfqpoint{1.812168in}{1.409025in}}{\pgfqpoint{1.808896in}{1.401125in}}{\pgfqpoint{1.808896in}{1.392889in}}%
\pgfpathcurveto{\pgfqpoint{1.808896in}{1.384652in}}{\pgfqpoint{1.812168in}{1.376752in}}{\pgfqpoint{1.817992in}{1.370928in}}%
\pgfpathcurveto{\pgfqpoint{1.823816in}{1.365104in}}{\pgfqpoint{1.831716in}{1.361832in}}{\pgfqpoint{1.839952in}{1.361832in}}%
\pgfpathclose%
\pgfusepath{stroke,fill}%
\end{pgfscope}%
\begin{pgfscope}%
\pgfpathrectangle{\pgfqpoint{0.100000in}{0.212622in}}{\pgfqpoint{3.696000in}{3.696000in}}%
\pgfusepath{clip}%
\pgfsetbuttcap%
\pgfsetroundjoin%
\definecolor{currentfill}{rgb}{0.121569,0.466667,0.705882}%
\pgfsetfillcolor{currentfill}%
\pgfsetfillopacity{0.602812}%
\pgfsetlinewidth{1.003750pt}%
\definecolor{currentstroke}{rgb}{0.121569,0.466667,0.705882}%
\pgfsetstrokecolor{currentstroke}%
\pgfsetstrokeopacity{0.602812}%
\pgfsetdash{}{0pt}%
\pgfpathmoveto{\pgfqpoint{1.844981in}{1.360620in}}%
\pgfpathcurveto{\pgfqpoint{1.853218in}{1.360620in}}{\pgfqpoint{1.861118in}{1.363893in}}{\pgfqpoint{1.866942in}{1.369717in}}%
\pgfpathcurveto{\pgfqpoint{1.872766in}{1.375541in}}{\pgfqpoint{1.876038in}{1.383441in}}{\pgfqpoint{1.876038in}{1.391677in}}%
\pgfpathcurveto{\pgfqpoint{1.876038in}{1.399913in}}{\pgfqpoint{1.872766in}{1.407813in}}{\pgfqpoint{1.866942in}{1.413637in}}%
\pgfpathcurveto{\pgfqpoint{1.861118in}{1.419461in}}{\pgfqpoint{1.853218in}{1.422733in}}{\pgfqpoint{1.844981in}{1.422733in}}%
\pgfpathcurveto{\pgfqpoint{1.836745in}{1.422733in}}{\pgfqpoint{1.828845in}{1.419461in}}{\pgfqpoint{1.823021in}{1.413637in}}%
\pgfpathcurveto{\pgfqpoint{1.817197in}{1.407813in}}{\pgfqpoint{1.813925in}{1.399913in}}{\pgfqpoint{1.813925in}{1.391677in}}%
\pgfpathcurveto{\pgfqpoint{1.813925in}{1.383441in}}{\pgfqpoint{1.817197in}{1.375541in}}{\pgfqpoint{1.823021in}{1.369717in}}%
\pgfpathcurveto{\pgfqpoint{1.828845in}{1.363893in}}{\pgfqpoint{1.836745in}{1.360620in}}{\pgfqpoint{1.844981in}{1.360620in}}%
\pgfpathclose%
\pgfusepath{stroke,fill}%
\end{pgfscope}%
\begin{pgfscope}%
\pgfpathrectangle{\pgfqpoint{0.100000in}{0.212622in}}{\pgfqpoint{3.696000in}{3.696000in}}%
\pgfusepath{clip}%
\pgfsetbuttcap%
\pgfsetroundjoin%
\definecolor{currentfill}{rgb}{0.121569,0.466667,0.705882}%
\pgfsetfillcolor{currentfill}%
\pgfsetfillopacity{0.605329}%
\pgfsetlinewidth{1.003750pt}%
\definecolor{currentstroke}{rgb}{0.121569,0.466667,0.705882}%
\pgfsetstrokecolor{currentstroke}%
\pgfsetstrokeopacity{0.605329}%
\pgfsetdash{}{0pt}%
\pgfpathmoveto{\pgfqpoint{1.851298in}{1.358483in}}%
\pgfpathcurveto{\pgfqpoint{1.859534in}{1.358483in}}{\pgfqpoint{1.867434in}{1.361756in}}{\pgfqpoint{1.873258in}{1.367580in}}%
\pgfpathcurveto{\pgfqpoint{1.879082in}{1.373404in}}{\pgfqpoint{1.882354in}{1.381304in}}{\pgfqpoint{1.882354in}{1.389540in}}%
\pgfpathcurveto{\pgfqpoint{1.882354in}{1.397776in}}{\pgfqpoint{1.879082in}{1.405676in}}{\pgfqpoint{1.873258in}{1.411500in}}%
\pgfpathcurveto{\pgfqpoint{1.867434in}{1.417324in}}{\pgfqpoint{1.859534in}{1.420596in}}{\pgfqpoint{1.851298in}{1.420596in}}%
\pgfpathcurveto{\pgfqpoint{1.843061in}{1.420596in}}{\pgfqpoint{1.835161in}{1.417324in}}{\pgfqpoint{1.829337in}{1.411500in}}%
\pgfpathcurveto{\pgfqpoint{1.823513in}{1.405676in}}{\pgfqpoint{1.820241in}{1.397776in}}{\pgfqpoint{1.820241in}{1.389540in}}%
\pgfpathcurveto{\pgfqpoint{1.820241in}{1.381304in}}{\pgfqpoint{1.823513in}{1.373404in}}{\pgfqpoint{1.829337in}{1.367580in}}%
\pgfpathcurveto{\pgfqpoint{1.835161in}{1.361756in}}{\pgfqpoint{1.843061in}{1.358483in}}{\pgfqpoint{1.851298in}{1.358483in}}%
\pgfpathclose%
\pgfusepath{stroke,fill}%
\end{pgfscope}%
\begin{pgfscope}%
\pgfpathrectangle{\pgfqpoint{0.100000in}{0.212622in}}{\pgfqpoint{3.696000in}{3.696000in}}%
\pgfusepath{clip}%
\pgfsetbuttcap%
\pgfsetroundjoin%
\definecolor{currentfill}{rgb}{0.121569,0.466667,0.705882}%
\pgfsetfillcolor{currentfill}%
\pgfsetfillopacity{0.607915}%
\pgfsetlinewidth{1.003750pt}%
\definecolor{currentstroke}{rgb}{0.121569,0.466667,0.705882}%
\pgfsetstrokecolor{currentstroke}%
\pgfsetstrokeopacity{0.607915}%
\pgfsetdash{}{0pt}%
\pgfpathmoveto{\pgfqpoint{1.858997in}{1.356412in}}%
\pgfpathcurveto{\pgfqpoint{1.867233in}{1.356412in}}{\pgfqpoint{1.875133in}{1.359684in}}{\pgfqpoint{1.880957in}{1.365508in}}%
\pgfpathcurveto{\pgfqpoint{1.886781in}{1.371332in}}{\pgfqpoint{1.890053in}{1.379232in}}{\pgfqpoint{1.890053in}{1.387468in}}%
\pgfpathcurveto{\pgfqpoint{1.890053in}{1.395704in}}{\pgfqpoint{1.886781in}{1.403605in}}{\pgfqpoint{1.880957in}{1.409428in}}%
\pgfpathcurveto{\pgfqpoint{1.875133in}{1.415252in}}{\pgfqpoint{1.867233in}{1.418525in}}{\pgfqpoint{1.858997in}{1.418525in}}%
\pgfpathcurveto{\pgfqpoint{1.850760in}{1.418525in}}{\pgfqpoint{1.842860in}{1.415252in}}{\pgfqpoint{1.837036in}{1.409428in}}%
\pgfpathcurveto{\pgfqpoint{1.831212in}{1.403605in}}{\pgfqpoint{1.827940in}{1.395704in}}{\pgfqpoint{1.827940in}{1.387468in}}%
\pgfpathcurveto{\pgfqpoint{1.827940in}{1.379232in}}{\pgfqpoint{1.831212in}{1.371332in}}{\pgfqpoint{1.837036in}{1.365508in}}%
\pgfpathcurveto{\pgfqpoint{1.842860in}{1.359684in}}{\pgfqpoint{1.850760in}{1.356412in}}{\pgfqpoint{1.858997in}{1.356412in}}%
\pgfpathclose%
\pgfusepath{stroke,fill}%
\end{pgfscope}%
\begin{pgfscope}%
\pgfpathrectangle{\pgfqpoint{0.100000in}{0.212622in}}{\pgfqpoint{3.696000in}{3.696000in}}%
\pgfusepath{clip}%
\pgfsetbuttcap%
\pgfsetroundjoin%
\definecolor{currentfill}{rgb}{0.121569,0.466667,0.705882}%
\pgfsetfillcolor{currentfill}%
\pgfsetfillopacity{0.611172}%
\pgfsetlinewidth{1.003750pt}%
\definecolor{currentstroke}{rgb}{0.121569,0.466667,0.705882}%
\pgfsetstrokecolor{currentstroke}%
\pgfsetstrokeopacity{0.611172}%
\pgfsetdash{}{0pt}%
\pgfpathmoveto{\pgfqpoint{1.867034in}{1.353207in}}%
\pgfpathcurveto{\pgfqpoint{1.875270in}{1.353207in}}{\pgfqpoint{1.883170in}{1.356479in}}{\pgfqpoint{1.888994in}{1.362303in}}%
\pgfpathcurveto{\pgfqpoint{1.894818in}{1.368127in}}{\pgfqpoint{1.898091in}{1.376027in}}{\pgfqpoint{1.898091in}{1.384263in}}%
\pgfpathcurveto{\pgfqpoint{1.898091in}{1.392500in}}{\pgfqpoint{1.894818in}{1.400400in}}{\pgfqpoint{1.888994in}{1.406224in}}%
\pgfpathcurveto{\pgfqpoint{1.883170in}{1.412048in}}{\pgfqpoint{1.875270in}{1.415320in}}{\pgfqpoint{1.867034in}{1.415320in}}%
\pgfpathcurveto{\pgfqpoint{1.858798in}{1.415320in}}{\pgfqpoint{1.850898in}{1.412048in}}{\pgfqpoint{1.845074in}{1.406224in}}%
\pgfpathcurveto{\pgfqpoint{1.839250in}{1.400400in}}{\pgfqpoint{1.835978in}{1.392500in}}{\pgfqpoint{1.835978in}{1.384263in}}%
\pgfpathcurveto{\pgfqpoint{1.835978in}{1.376027in}}{\pgfqpoint{1.839250in}{1.368127in}}{\pgfqpoint{1.845074in}{1.362303in}}%
\pgfpathcurveto{\pgfqpoint{1.850898in}{1.356479in}}{\pgfqpoint{1.858798in}{1.353207in}}{\pgfqpoint{1.867034in}{1.353207in}}%
\pgfpathclose%
\pgfusepath{stroke,fill}%
\end{pgfscope}%
\begin{pgfscope}%
\pgfpathrectangle{\pgfqpoint{0.100000in}{0.212622in}}{\pgfqpoint{3.696000in}{3.696000in}}%
\pgfusepath{clip}%
\pgfsetbuttcap%
\pgfsetroundjoin%
\definecolor{currentfill}{rgb}{0.121569,0.466667,0.705882}%
\pgfsetfillcolor{currentfill}%
\pgfsetfillopacity{0.612882}%
\pgfsetlinewidth{1.003750pt}%
\definecolor{currentstroke}{rgb}{0.121569,0.466667,0.705882}%
\pgfsetstrokecolor{currentstroke}%
\pgfsetstrokeopacity{0.612882}%
\pgfsetdash{}{0pt}%
\pgfpathmoveto{\pgfqpoint{1.871606in}{1.351781in}}%
\pgfpathcurveto{\pgfqpoint{1.879842in}{1.351781in}}{\pgfqpoint{1.887742in}{1.355053in}}{\pgfqpoint{1.893566in}{1.360877in}}%
\pgfpathcurveto{\pgfqpoint{1.899390in}{1.366701in}}{\pgfqpoint{1.902662in}{1.374601in}}{\pgfqpoint{1.902662in}{1.382837in}}%
\pgfpathcurveto{\pgfqpoint{1.902662in}{1.391073in}}{\pgfqpoint{1.899390in}{1.398973in}}{\pgfqpoint{1.893566in}{1.404797in}}%
\pgfpathcurveto{\pgfqpoint{1.887742in}{1.410621in}}{\pgfqpoint{1.879842in}{1.413894in}}{\pgfqpoint{1.871606in}{1.413894in}}%
\pgfpathcurveto{\pgfqpoint{1.863370in}{1.413894in}}{\pgfqpoint{1.855469in}{1.410621in}}{\pgfqpoint{1.849646in}{1.404797in}}%
\pgfpathcurveto{\pgfqpoint{1.843822in}{1.398973in}}{\pgfqpoint{1.840549in}{1.391073in}}{\pgfqpoint{1.840549in}{1.382837in}}%
\pgfpathcurveto{\pgfqpoint{1.840549in}{1.374601in}}{\pgfqpoint{1.843822in}{1.366701in}}{\pgfqpoint{1.849646in}{1.360877in}}%
\pgfpathcurveto{\pgfqpoint{1.855469in}{1.355053in}}{\pgfqpoint{1.863370in}{1.351781in}}{\pgfqpoint{1.871606in}{1.351781in}}%
\pgfpathclose%
\pgfusepath{stroke,fill}%
\end{pgfscope}%
\begin{pgfscope}%
\pgfpathrectangle{\pgfqpoint{0.100000in}{0.212622in}}{\pgfqpoint{3.696000in}{3.696000in}}%
\pgfusepath{clip}%
\pgfsetbuttcap%
\pgfsetroundjoin%
\definecolor{currentfill}{rgb}{0.121569,0.466667,0.705882}%
\pgfsetfillcolor{currentfill}%
\pgfsetfillopacity{0.613722}%
\pgfsetlinewidth{1.003750pt}%
\definecolor{currentstroke}{rgb}{0.121569,0.466667,0.705882}%
\pgfsetstrokecolor{currentstroke}%
\pgfsetstrokeopacity{0.613722}%
\pgfsetdash{}{0pt}%
\pgfpathmoveto{\pgfqpoint{1.874206in}{1.351071in}}%
\pgfpathcurveto{\pgfqpoint{1.882442in}{1.351071in}}{\pgfqpoint{1.890342in}{1.354343in}}{\pgfqpoint{1.896166in}{1.360167in}}%
\pgfpathcurveto{\pgfqpoint{1.901990in}{1.365991in}}{\pgfqpoint{1.905262in}{1.373891in}}{\pgfqpoint{1.905262in}{1.382128in}}%
\pgfpathcurveto{\pgfqpoint{1.905262in}{1.390364in}}{\pgfqpoint{1.901990in}{1.398264in}}{\pgfqpoint{1.896166in}{1.404088in}}%
\pgfpathcurveto{\pgfqpoint{1.890342in}{1.409912in}}{\pgfqpoint{1.882442in}{1.413184in}}{\pgfqpoint{1.874206in}{1.413184in}}%
\pgfpathcurveto{\pgfqpoint{1.865969in}{1.413184in}}{\pgfqpoint{1.858069in}{1.409912in}}{\pgfqpoint{1.852245in}{1.404088in}}%
\pgfpathcurveto{\pgfqpoint{1.846421in}{1.398264in}}{\pgfqpoint{1.843149in}{1.390364in}}{\pgfqpoint{1.843149in}{1.382128in}}%
\pgfpathcurveto{\pgfqpoint{1.843149in}{1.373891in}}{\pgfqpoint{1.846421in}{1.365991in}}{\pgfqpoint{1.852245in}{1.360167in}}%
\pgfpathcurveto{\pgfqpoint{1.858069in}{1.354343in}}{\pgfqpoint{1.865969in}{1.351071in}}{\pgfqpoint{1.874206in}{1.351071in}}%
\pgfpathclose%
\pgfusepath{stroke,fill}%
\end{pgfscope}%
\begin{pgfscope}%
\pgfpathrectangle{\pgfqpoint{0.100000in}{0.212622in}}{\pgfqpoint{3.696000in}{3.696000in}}%
\pgfusepath{clip}%
\pgfsetbuttcap%
\pgfsetroundjoin%
\definecolor{currentfill}{rgb}{0.121569,0.466667,0.705882}%
\pgfsetfillcolor{currentfill}%
\pgfsetfillopacity{0.615142}%
\pgfsetlinewidth{1.003750pt}%
\definecolor{currentstroke}{rgb}{0.121569,0.466667,0.705882}%
\pgfsetstrokecolor{currentstroke}%
\pgfsetstrokeopacity{0.615142}%
\pgfsetdash{}{0pt}%
\pgfpathmoveto{\pgfqpoint{1.878035in}{1.349859in}}%
\pgfpathcurveto{\pgfqpoint{1.886271in}{1.349859in}}{\pgfqpoint{1.894171in}{1.353132in}}{\pgfqpoint{1.899995in}{1.358955in}}%
\pgfpathcurveto{\pgfqpoint{1.905819in}{1.364779in}}{\pgfqpoint{1.909091in}{1.372679in}}{\pgfqpoint{1.909091in}{1.380916in}}%
\pgfpathcurveto{\pgfqpoint{1.909091in}{1.389152in}}{\pgfqpoint{1.905819in}{1.397052in}}{\pgfqpoint{1.899995in}{1.402876in}}%
\pgfpathcurveto{\pgfqpoint{1.894171in}{1.408700in}}{\pgfqpoint{1.886271in}{1.411972in}}{\pgfqpoint{1.878035in}{1.411972in}}%
\pgfpathcurveto{\pgfqpoint{1.869799in}{1.411972in}}{\pgfqpoint{1.861899in}{1.408700in}}{\pgfqpoint{1.856075in}{1.402876in}}%
\pgfpathcurveto{\pgfqpoint{1.850251in}{1.397052in}}{\pgfqpoint{1.846978in}{1.389152in}}{\pgfqpoint{1.846978in}{1.380916in}}%
\pgfpathcurveto{\pgfqpoint{1.846978in}{1.372679in}}{\pgfqpoint{1.850251in}{1.364779in}}{\pgfqpoint{1.856075in}{1.358955in}}%
\pgfpathcurveto{\pgfqpoint{1.861899in}{1.353132in}}{\pgfqpoint{1.869799in}{1.349859in}}{\pgfqpoint{1.878035in}{1.349859in}}%
\pgfpathclose%
\pgfusepath{stroke,fill}%
\end{pgfscope}%
\begin{pgfscope}%
\pgfpathrectangle{\pgfqpoint{0.100000in}{0.212622in}}{\pgfqpoint{3.696000in}{3.696000in}}%
\pgfusepath{clip}%
\pgfsetbuttcap%
\pgfsetroundjoin%
\definecolor{currentfill}{rgb}{0.121569,0.466667,0.705882}%
\pgfsetfillcolor{currentfill}%
\pgfsetfillopacity{0.616699}%
\pgfsetlinewidth{1.003750pt}%
\definecolor{currentstroke}{rgb}{0.121569,0.466667,0.705882}%
\pgfsetstrokecolor{currentstroke}%
\pgfsetstrokeopacity{0.616699}%
\pgfsetdash{}{0pt}%
\pgfpathmoveto{\pgfqpoint{1.883690in}{1.348731in}}%
\pgfpathcurveto{\pgfqpoint{1.891926in}{1.348731in}}{\pgfqpoint{1.899826in}{1.352003in}}{\pgfqpoint{1.905650in}{1.357827in}}%
\pgfpathcurveto{\pgfqpoint{1.911474in}{1.363651in}}{\pgfqpoint{1.914746in}{1.371551in}}{\pgfqpoint{1.914746in}{1.379787in}}%
\pgfpathcurveto{\pgfqpoint{1.914746in}{1.388023in}}{\pgfqpoint{1.911474in}{1.395923in}}{\pgfqpoint{1.905650in}{1.401747in}}%
\pgfpathcurveto{\pgfqpoint{1.899826in}{1.407571in}}{\pgfqpoint{1.891926in}{1.410844in}}{\pgfqpoint{1.883690in}{1.410844in}}%
\pgfpathcurveto{\pgfqpoint{1.875454in}{1.410844in}}{\pgfqpoint{1.867554in}{1.407571in}}{\pgfqpoint{1.861730in}{1.401747in}}%
\pgfpathcurveto{\pgfqpoint{1.855906in}{1.395923in}}{\pgfqpoint{1.852633in}{1.388023in}}{\pgfqpoint{1.852633in}{1.379787in}}%
\pgfpathcurveto{\pgfqpoint{1.852633in}{1.371551in}}{\pgfqpoint{1.855906in}{1.363651in}}{\pgfqpoint{1.861730in}{1.357827in}}%
\pgfpathcurveto{\pgfqpoint{1.867554in}{1.352003in}}{\pgfqpoint{1.875454in}{1.348731in}}{\pgfqpoint{1.883690in}{1.348731in}}%
\pgfpathclose%
\pgfusepath{stroke,fill}%
\end{pgfscope}%
\begin{pgfscope}%
\pgfpathrectangle{\pgfqpoint{0.100000in}{0.212622in}}{\pgfqpoint{3.696000in}{3.696000in}}%
\pgfusepath{clip}%
\pgfsetbuttcap%
\pgfsetroundjoin%
\definecolor{currentfill}{rgb}{0.121569,0.466667,0.705882}%
\pgfsetfillcolor{currentfill}%
\pgfsetfillopacity{0.619129}%
\pgfsetlinewidth{1.003750pt}%
\definecolor{currentstroke}{rgb}{0.121569,0.466667,0.705882}%
\pgfsetstrokecolor{currentstroke}%
\pgfsetstrokeopacity{0.619129}%
\pgfsetdash{}{0pt}%
\pgfpathmoveto{\pgfqpoint{1.890218in}{1.346590in}}%
\pgfpathcurveto{\pgfqpoint{1.898455in}{1.346590in}}{\pgfqpoint{1.906355in}{1.349862in}}{\pgfqpoint{1.912179in}{1.355686in}}%
\pgfpathcurveto{\pgfqpoint{1.918003in}{1.361510in}}{\pgfqpoint{1.921275in}{1.369410in}}{\pgfqpoint{1.921275in}{1.377647in}}%
\pgfpathcurveto{\pgfqpoint{1.921275in}{1.385883in}}{\pgfqpoint{1.918003in}{1.393783in}}{\pgfqpoint{1.912179in}{1.399607in}}%
\pgfpathcurveto{\pgfqpoint{1.906355in}{1.405431in}}{\pgfqpoint{1.898455in}{1.408703in}}{\pgfqpoint{1.890218in}{1.408703in}}%
\pgfpathcurveto{\pgfqpoint{1.881982in}{1.408703in}}{\pgfqpoint{1.874082in}{1.405431in}}{\pgfqpoint{1.868258in}{1.399607in}}%
\pgfpathcurveto{\pgfqpoint{1.862434in}{1.393783in}}{\pgfqpoint{1.859162in}{1.385883in}}{\pgfqpoint{1.859162in}{1.377647in}}%
\pgfpathcurveto{\pgfqpoint{1.859162in}{1.369410in}}{\pgfqpoint{1.862434in}{1.361510in}}{\pgfqpoint{1.868258in}{1.355686in}}%
\pgfpathcurveto{\pgfqpoint{1.874082in}{1.349862in}}{\pgfqpoint{1.881982in}{1.346590in}}{\pgfqpoint{1.890218in}{1.346590in}}%
\pgfpathclose%
\pgfusepath{stroke,fill}%
\end{pgfscope}%
\begin{pgfscope}%
\pgfpathrectangle{\pgfqpoint{0.100000in}{0.212622in}}{\pgfqpoint{3.696000in}{3.696000in}}%
\pgfusepath{clip}%
\pgfsetbuttcap%
\pgfsetroundjoin%
\definecolor{currentfill}{rgb}{0.121569,0.466667,0.705882}%
\pgfsetfillcolor{currentfill}%
\pgfsetfillopacity{0.621405}%
\pgfsetlinewidth{1.003750pt}%
\definecolor{currentstroke}{rgb}{0.121569,0.466667,0.705882}%
\pgfsetstrokecolor{currentstroke}%
\pgfsetstrokeopacity{0.621405}%
\pgfsetdash{}{0pt}%
\pgfpathmoveto{\pgfqpoint{1.897722in}{1.345125in}}%
\pgfpathcurveto{\pgfqpoint{1.905958in}{1.345125in}}{\pgfqpoint{1.913858in}{1.348398in}}{\pgfqpoint{1.919682in}{1.354222in}}%
\pgfpathcurveto{\pgfqpoint{1.925506in}{1.360045in}}{\pgfqpoint{1.928778in}{1.367946in}}{\pgfqpoint{1.928778in}{1.376182in}}%
\pgfpathcurveto{\pgfqpoint{1.928778in}{1.384418in}}{\pgfqpoint{1.925506in}{1.392318in}}{\pgfqpoint{1.919682in}{1.398142in}}%
\pgfpathcurveto{\pgfqpoint{1.913858in}{1.403966in}}{\pgfqpoint{1.905958in}{1.407238in}}{\pgfqpoint{1.897722in}{1.407238in}}%
\pgfpathcurveto{\pgfqpoint{1.889485in}{1.407238in}}{\pgfqpoint{1.881585in}{1.403966in}}{\pgfqpoint{1.875761in}{1.398142in}}%
\pgfpathcurveto{\pgfqpoint{1.869937in}{1.392318in}}{\pgfqpoint{1.866665in}{1.384418in}}{\pgfqpoint{1.866665in}{1.376182in}}%
\pgfpathcurveto{\pgfqpoint{1.866665in}{1.367946in}}{\pgfqpoint{1.869937in}{1.360045in}}{\pgfqpoint{1.875761in}{1.354222in}}%
\pgfpathcurveto{\pgfqpoint{1.881585in}{1.348398in}}{\pgfqpoint{1.889485in}{1.345125in}}{\pgfqpoint{1.897722in}{1.345125in}}%
\pgfpathclose%
\pgfusepath{stroke,fill}%
\end{pgfscope}%
\begin{pgfscope}%
\pgfpathrectangle{\pgfqpoint{0.100000in}{0.212622in}}{\pgfqpoint{3.696000in}{3.696000in}}%
\pgfusepath{clip}%
\pgfsetbuttcap%
\pgfsetroundjoin%
\definecolor{currentfill}{rgb}{0.121569,0.466667,0.705882}%
\pgfsetfillcolor{currentfill}%
\pgfsetfillopacity{0.622707}%
\pgfsetlinewidth{1.003750pt}%
\definecolor{currentstroke}{rgb}{0.121569,0.466667,0.705882}%
\pgfsetstrokecolor{currentstroke}%
\pgfsetstrokeopacity{0.622707}%
\pgfsetdash{}{0pt}%
\pgfpathmoveto{\pgfqpoint{1.901772in}{1.344160in}}%
\pgfpathcurveto{\pgfqpoint{1.910008in}{1.344160in}}{\pgfqpoint{1.917908in}{1.347432in}}{\pgfqpoint{1.923732in}{1.353256in}}%
\pgfpathcurveto{\pgfqpoint{1.929556in}{1.359080in}}{\pgfqpoint{1.932828in}{1.366980in}}{\pgfqpoint{1.932828in}{1.375216in}}%
\pgfpathcurveto{\pgfqpoint{1.932828in}{1.383452in}}{\pgfqpoint{1.929556in}{1.391353in}}{\pgfqpoint{1.923732in}{1.397176in}}%
\pgfpathcurveto{\pgfqpoint{1.917908in}{1.403000in}}{\pgfqpoint{1.910008in}{1.406273in}}{\pgfqpoint{1.901772in}{1.406273in}}%
\pgfpathcurveto{\pgfqpoint{1.893535in}{1.406273in}}{\pgfqpoint{1.885635in}{1.403000in}}{\pgfqpoint{1.879811in}{1.397176in}}%
\pgfpathcurveto{\pgfqpoint{1.873987in}{1.391353in}}{\pgfqpoint{1.870715in}{1.383452in}}{\pgfqpoint{1.870715in}{1.375216in}}%
\pgfpathcurveto{\pgfqpoint{1.870715in}{1.366980in}}{\pgfqpoint{1.873987in}{1.359080in}}{\pgfqpoint{1.879811in}{1.353256in}}%
\pgfpathcurveto{\pgfqpoint{1.885635in}{1.347432in}}{\pgfqpoint{1.893535in}{1.344160in}}{\pgfqpoint{1.901772in}{1.344160in}}%
\pgfpathclose%
\pgfusepath{stroke,fill}%
\end{pgfscope}%
\begin{pgfscope}%
\pgfpathrectangle{\pgfqpoint{0.100000in}{0.212622in}}{\pgfqpoint{3.696000in}{3.696000in}}%
\pgfusepath{clip}%
\pgfsetbuttcap%
\pgfsetroundjoin%
\definecolor{currentfill}{rgb}{0.121569,0.466667,0.705882}%
\pgfsetfillcolor{currentfill}%
\pgfsetfillopacity{0.624577}%
\pgfsetlinewidth{1.003750pt}%
\definecolor{currentstroke}{rgb}{0.121569,0.466667,0.705882}%
\pgfsetstrokecolor{currentstroke}%
\pgfsetstrokeopacity{0.624577}%
\pgfsetdash{}{0pt}%
\pgfpathmoveto{\pgfqpoint{1.906957in}{1.342535in}}%
\pgfpathcurveto{\pgfqpoint{1.915194in}{1.342535in}}{\pgfqpoint{1.923094in}{1.345807in}}{\pgfqpoint{1.928918in}{1.351631in}}%
\pgfpathcurveto{\pgfqpoint{1.934742in}{1.357455in}}{\pgfqpoint{1.938014in}{1.365355in}}{\pgfqpoint{1.938014in}{1.373592in}}%
\pgfpathcurveto{\pgfqpoint{1.938014in}{1.381828in}}{\pgfqpoint{1.934742in}{1.389728in}}{\pgfqpoint{1.928918in}{1.395552in}}%
\pgfpathcurveto{\pgfqpoint{1.923094in}{1.401376in}}{\pgfqpoint{1.915194in}{1.404648in}}{\pgfqpoint{1.906957in}{1.404648in}}%
\pgfpathcurveto{\pgfqpoint{1.898721in}{1.404648in}}{\pgfqpoint{1.890821in}{1.401376in}}{\pgfqpoint{1.884997in}{1.395552in}}%
\pgfpathcurveto{\pgfqpoint{1.879173in}{1.389728in}}{\pgfqpoint{1.875901in}{1.381828in}}{\pgfqpoint{1.875901in}{1.373592in}}%
\pgfpathcurveto{\pgfqpoint{1.875901in}{1.365355in}}{\pgfqpoint{1.879173in}{1.357455in}}{\pgfqpoint{1.884997in}{1.351631in}}%
\pgfpathcurveto{\pgfqpoint{1.890821in}{1.345807in}}{\pgfqpoint{1.898721in}{1.342535in}}{\pgfqpoint{1.906957in}{1.342535in}}%
\pgfpathclose%
\pgfusepath{stroke,fill}%
\end{pgfscope}%
\begin{pgfscope}%
\pgfpathrectangle{\pgfqpoint{0.100000in}{0.212622in}}{\pgfqpoint{3.696000in}{3.696000in}}%
\pgfusepath{clip}%
\pgfsetbuttcap%
\pgfsetroundjoin%
\definecolor{currentfill}{rgb}{0.121569,0.466667,0.705882}%
\pgfsetfillcolor{currentfill}%
\pgfsetfillopacity{0.626808}%
\pgfsetlinewidth{1.003750pt}%
\definecolor{currentstroke}{rgb}{0.121569,0.466667,0.705882}%
\pgfsetstrokecolor{currentstroke}%
\pgfsetstrokeopacity{0.626808}%
\pgfsetdash{}{0pt}%
\pgfpathmoveto{\pgfqpoint{1.913538in}{1.340650in}}%
\pgfpathcurveto{\pgfqpoint{1.921774in}{1.340650in}}{\pgfqpoint{1.929674in}{1.343923in}}{\pgfqpoint{1.935498in}{1.349747in}}%
\pgfpathcurveto{\pgfqpoint{1.941322in}{1.355570in}}{\pgfqpoint{1.944595in}{1.363471in}}{\pgfqpoint{1.944595in}{1.371707in}}%
\pgfpathcurveto{\pgfqpoint{1.944595in}{1.379943in}}{\pgfqpoint{1.941322in}{1.387843in}}{\pgfqpoint{1.935498in}{1.393667in}}%
\pgfpathcurveto{\pgfqpoint{1.929674in}{1.399491in}}{\pgfqpoint{1.921774in}{1.402763in}}{\pgfqpoint{1.913538in}{1.402763in}}%
\pgfpathcurveto{\pgfqpoint{1.905302in}{1.402763in}}{\pgfqpoint{1.897402in}{1.399491in}}{\pgfqpoint{1.891578in}{1.393667in}}%
\pgfpathcurveto{\pgfqpoint{1.885754in}{1.387843in}}{\pgfqpoint{1.882482in}{1.379943in}}{\pgfqpoint{1.882482in}{1.371707in}}%
\pgfpathcurveto{\pgfqpoint{1.882482in}{1.363471in}}{\pgfqpoint{1.885754in}{1.355570in}}{\pgfqpoint{1.891578in}{1.349747in}}%
\pgfpathcurveto{\pgfqpoint{1.897402in}{1.343923in}}{\pgfqpoint{1.905302in}{1.340650in}}{\pgfqpoint{1.913538in}{1.340650in}}%
\pgfpathclose%
\pgfusepath{stroke,fill}%
\end{pgfscope}%
\begin{pgfscope}%
\pgfpathrectangle{\pgfqpoint{0.100000in}{0.212622in}}{\pgfqpoint{3.696000in}{3.696000in}}%
\pgfusepath{clip}%
\pgfsetbuttcap%
\pgfsetroundjoin%
\definecolor{currentfill}{rgb}{0.121569,0.466667,0.705882}%
\pgfsetfillcolor{currentfill}%
\pgfsetfillopacity{0.628346}%
\pgfsetlinewidth{1.003750pt}%
\definecolor{currentstroke}{rgb}{0.121569,0.466667,0.705882}%
\pgfsetstrokecolor{currentstroke}%
\pgfsetstrokeopacity{0.628346}%
\pgfsetdash{}{0pt}%
\pgfpathmoveto{\pgfqpoint{1.916868in}{1.339287in}}%
\pgfpathcurveto{\pgfqpoint{1.925105in}{1.339287in}}{\pgfqpoint{1.933005in}{1.342559in}}{\pgfqpoint{1.938829in}{1.348383in}}%
\pgfpathcurveto{\pgfqpoint{1.944653in}{1.354207in}}{\pgfqpoint{1.947925in}{1.362107in}}{\pgfqpoint{1.947925in}{1.370343in}}%
\pgfpathcurveto{\pgfqpoint{1.947925in}{1.378579in}}{\pgfqpoint{1.944653in}{1.386479in}}{\pgfqpoint{1.938829in}{1.392303in}}%
\pgfpathcurveto{\pgfqpoint{1.933005in}{1.398127in}}{\pgfqpoint{1.925105in}{1.401400in}}{\pgfqpoint{1.916868in}{1.401400in}}%
\pgfpathcurveto{\pgfqpoint{1.908632in}{1.401400in}}{\pgfqpoint{1.900732in}{1.398127in}}{\pgfqpoint{1.894908in}{1.392303in}}%
\pgfpathcurveto{\pgfqpoint{1.889084in}{1.386479in}}{\pgfqpoint{1.885812in}{1.378579in}}{\pgfqpoint{1.885812in}{1.370343in}}%
\pgfpathcurveto{\pgfqpoint{1.885812in}{1.362107in}}{\pgfqpoint{1.889084in}{1.354207in}}{\pgfqpoint{1.894908in}{1.348383in}}%
\pgfpathcurveto{\pgfqpoint{1.900732in}{1.342559in}}{\pgfqpoint{1.908632in}{1.339287in}}{\pgfqpoint{1.916868in}{1.339287in}}%
\pgfpathclose%
\pgfusepath{stroke,fill}%
\end{pgfscope}%
\begin{pgfscope}%
\pgfpathrectangle{\pgfqpoint{0.100000in}{0.212622in}}{\pgfqpoint{3.696000in}{3.696000in}}%
\pgfusepath{clip}%
\pgfsetbuttcap%
\pgfsetroundjoin%
\definecolor{currentfill}{rgb}{0.121569,0.466667,0.705882}%
\pgfsetfillcolor{currentfill}%
\pgfsetfillopacity{0.629902}%
\pgfsetlinewidth{1.003750pt}%
\definecolor{currentstroke}{rgb}{0.121569,0.466667,0.705882}%
\pgfsetstrokecolor{currentstroke}%
\pgfsetstrokeopacity{0.629902}%
\pgfsetdash{}{0pt}%
\pgfpathmoveto{\pgfqpoint{1.921077in}{1.338150in}}%
\pgfpathcurveto{\pgfqpoint{1.929313in}{1.338150in}}{\pgfqpoint{1.937213in}{1.341423in}}{\pgfqpoint{1.943037in}{1.347247in}}%
\pgfpathcurveto{\pgfqpoint{1.948861in}{1.353071in}}{\pgfqpoint{1.952134in}{1.360971in}}{\pgfqpoint{1.952134in}{1.369207in}}%
\pgfpathcurveto{\pgfqpoint{1.952134in}{1.377443in}}{\pgfqpoint{1.948861in}{1.385343in}}{\pgfqpoint{1.943037in}{1.391167in}}%
\pgfpathcurveto{\pgfqpoint{1.937213in}{1.396991in}}{\pgfqpoint{1.929313in}{1.400263in}}{\pgfqpoint{1.921077in}{1.400263in}}%
\pgfpathcurveto{\pgfqpoint{1.912841in}{1.400263in}}{\pgfqpoint{1.904941in}{1.396991in}}{\pgfqpoint{1.899117in}{1.391167in}}%
\pgfpathcurveto{\pgfqpoint{1.893293in}{1.385343in}}{\pgfqpoint{1.890021in}{1.377443in}}{\pgfqpoint{1.890021in}{1.369207in}}%
\pgfpathcurveto{\pgfqpoint{1.890021in}{1.360971in}}{\pgfqpoint{1.893293in}{1.353071in}}{\pgfqpoint{1.899117in}{1.347247in}}%
\pgfpathcurveto{\pgfqpoint{1.904941in}{1.341423in}}{\pgfqpoint{1.912841in}{1.338150in}}{\pgfqpoint{1.921077in}{1.338150in}}%
\pgfpathclose%
\pgfusepath{stroke,fill}%
\end{pgfscope}%
\begin{pgfscope}%
\pgfpathrectangle{\pgfqpoint{0.100000in}{0.212622in}}{\pgfqpoint{3.696000in}{3.696000in}}%
\pgfusepath{clip}%
\pgfsetbuttcap%
\pgfsetroundjoin%
\definecolor{currentfill}{rgb}{0.121569,0.466667,0.705882}%
\pgfsetfillcolor{currentfill}%
\pgfsetfillopacity{0.630708}%
\pgfsetlinewidth{1.003750pt}%
\definecolor{currentstroke}{rgb}{0.121569,0.466667,0.705882}%
\pgfsetstrokecolor{currentstroke}%
\pgfsetstrokeopacity{0.630708}%
\pgfsetdash{}{0pt}%
\pgfpathmoveto{\pgfqpoint{1.923416in}{1.337500in}}%
\pgfpathcurveto{\pgfqpoint{1.931652in}{1.337500in}}{\pgfqpoint{1.939552in}{1.340773in}}{\pgfqpoint{1.945376in}{1.346597in}}%
\pgfpathcurveto{\pgfqpoint{1.951200in}{1.352421in}}{\pgfqpoint{1.954472in}{1.360321in}}{\pgfqpoint{1.954472in}{1.368557in}}%
\pgfpathcurveto{\pgfqpoint{1.954472in}{1.376793in}}{\pgfqpoint{1.951200in}{1.384693in}}{\pgfqpoint{1.945376in}{1.390517in}}%
\pgfpathcurveto{\pgfqpoint{1.939552in}{1.396341in}}{\pgfqpoint{1.931652in}{1.399613in}}{\pgfqpoint{1.923416in}{1.399613in}}%
\pgfpathcurveto{\pgfqpoint{1.915180in}{1.399613in}}{\pgfqpoint{1.907279in}{1.396341in}}{\pgfqpoint{1.901456in}{1.390517in}}%
\pgfpathcurveto{\pgfqpoint{1.895632in}{1.384693in}}{\pgfqpoint{1.892359in}{1.376793in}}{\pgfqpoint{1.892359in}{1.368557in}}%
\pgfpathcurveto{\pgfqpoint{1.892359in}{1.360321in}}{\pgfqpoint{1.895632in}{1.352421in}}{\pgfqpoint{1.901456in}{1.346597in}}%
\pgfpathcurveto{\pgfqpoint{1.907279in}{1.340773in}}{\pgfqpoint{1.915180in}{1.337500in}}{\pgfqpoint{1.923416in}{1.337500in}}%
\pgfpathclose%
\pgfusepath{stroke,fill}%
\end{pgfscope}%
\begin{pgfscope}%
\pgfpathrectangle{\pgfqpoint{0.100000in}{0.212622in}}{\pgfqpoint{3.696000in}{3.696000in}}%
\pgfusepath{clip}%
\pgfsetbuttcap%
\pgfsetroundjoin%
\definecolor{currentfill}{rgb}{0.121569,0.466667,0.705882}%
\pgfsetfillcolor{currentfill}%
\pgfsetfillopacity{0.631875}%
\pgfsetlinewidth{1.003750pt}%
\definecolor{currentstroke}{rgb}{0.121569,0.466667,0.705882}%
\pgfsetstrokecolor{currentstroke}%
\pgfsetstrokeopacity{0.631875}%
\pgfsetdash{}{0pt}%
\pgfpathmoveto{\pgfqpoint{1.926685in}{1.336472in}}%
\pgfpathcurveto{\pgfqpoint{1.934921in}{1.336472in}}{\pgfqpoint{1.942821in}{1.339744in}}{\pgfqpoint{1.948645in}{1.345568in}}%
\pgfpathcurveto{\pgfqpoint{1.954469in}{1.351392in}}{\pgfqpoint{1.957741in}{1.359292in}}{\pgfqpoint{1.957741in}{1.367528in}}%
\pgfpathcurveto{\pgfqpoint{1.957741in}{1.375764in}}{\pgfqpoint{1.954469in}{1.383664in}}{\pgfqpoint{1.948645in}{1.389488in}}%
\pgfpathcurveto{\pgfqpoint{1.942821in}{1.395312in}}{\pgfqpoint{1.934921in}{1.398585in}}{\pgfqpoint{1.926685in}{1.398585in}}%
\pgfpathcurveto{\pgfqpoint{1.918449in}{1.398585in}}{\pgfqpoint{1.910549in}{1.395312in}}{\pgfqpoint{1.904725in}{1.389488in}}%
\pgfpathcurveto{\pgfqpoint{1.898901in}{1.383664in}}{\pgfqpoint{1.895628in}{1.375764in}}{\pgfqpoint{1.895628in}{1.367528in}}%
\pgfpathcurveto{\pgfqpoint{1.895628in}{1.359292in}}{\pgfqpoint{1.898901in}{1.351392in}}{\pgfqpoint{1.904725in}{1.345568in}}%
\pgfpathcurveto{\pgfqpoint{1.910549in}{1.339744in}}{\pgfqpoint{1.918449in}{1.336472in}}{\pgfqpoint{1.926685in}{1.336472in}}%
\pgfpathclose%
\pgfusepath{stroke,fill}%
\end{pgfscope}%
\begin{pgfscope}%
\pgfpathrectangle{\pgfqpoint{0.100000in}{0.212622in}}{\pgfqpoint{3.696000in}{3.696000in}}%
\pgfusepath{clip}%
\pgfsetbuttcap%
\pgfsetroundjoin%
\definecolor{currentfill}{rgb}{0.121569,0.466667,0.705882}%
\pgfsetfillcolor{currentfill}%
\pgfsetfillopacity{0.633873}%
\pgfsetlinewidth{1.003750pt}%
\definecolor{currentstroke}{rgb}{0.121569,0.466667,0.705882}%
\pgfsetstrokecolor{currentstroke}%
\pgfsetstrokeopacity{0.633873}%
\pgfsetdash{}{0pt}%
\pgfpathmoveto{\pgfqpoint{1.930961in}{1.334822in}}%
\pgfpathcurveto{\pgfqpoint{1.939197in}{1.334822in}}{\pgfqpoint{1.947097in}{1.338094in}}{\pgfqpoint{1.952921in}{1.343918in}}%
\pgfpathcurveto{\pgfqpoint{1.958745in}{1.349742in}}{\pgfqpoint{1.962017in}{1.357642in}}{\pgfqpoint{1.962017in}{1.365879in}}%
\pgfpathcurveto{\pgfqpoint{1.962017in}{1.374115in}}{\pgfqpoint{1.958745in}{1.382015in}}{\pgfqpoint{1.952921in}{1.387839in}}%
\pgfpathcurveto{\pgfqpoint{1.947097in}{1.393663in}}{\pgfqpoint{1.939197in}{1.396935in}}{\pgfqpoint{1.930961in}{1.396935in}}%
\pgfpathcurveto{\pgfqpoint{1.922724in}{1.396935in}}{\pgfqpoint{1.914824in}{1.393663in}}{\pgfqpoint{1.909001in}{1.387839in}}%
\pgfpathcurveto{\pgfqpoint{1.903177in}{1.382015in}}{\pgfqpoint{1.899904in}{1.374115in}}{\pgfqpoint{1.899904in}{1.365879in}}%
\pgfpathcurveto{\pgfqpoint{1.899904in}{1.357642in}}{\pgfqpoint{1.903177in}{1.349742in}}{\pgfqpoint{1.909001in}{1.343918in}}%
\pgfpathcurveto{\pgfqpoint{1.914824in}{1.338094in}}{\pgfqpoint{1.922724in}{1.334822in}}{\pgfqpoint{1.930961in}{1.334822in}}%
\pgfpathclose%
\pgfusepath{stroke,fill}%
\end{pgfscope}%
\begin{pgfscope}%
\pgfpathrectangle{\pgfqpoint{0.100000in}{0.212622in}}{\pgfqpoint{3.696000in}{3.696000in}}%
\pgfusepath{clip}%
\pgfsetbuttcap%
\pgfsetroundjoin%
\definecolor{currentfill}{rgb}{0.121569,0.466667,0.705882}%
\pgfsetfillcolor{currentfill}%
\pgfsetfillopacity{0.635860}%
\pgfsetlinewidth{1.003750pt}%
\definecolor{currentstroke}{rgb}{0.121569,0.466667,0.705882}%
\pgfsetstrokecolor{currentstroke}%
\pgfsetstrokeopacity{0.635860}%
\pgfsetdash{}{0pt}%
\pgfpathmoveto{\pgfqpoint{1.936904in}{1.333180in}}%
\pgfpathcurveto{\pgfqpoint{1.945140in}{1.333180in}}{\pgfqpoint{1.953040in}{1.336452in}}{\pgfqpoint{1.958864in}{1.342276in}}%
\pgfpathcurveto{\pgfqpoint{1.964688in}{1.348100in}}{\pgfqpoint{1.967960in}{1.356000in}}{\pgfqpoint{1.967960in}{1.364236in}}%
\pgfpathcurveto{\pgfqpoint{1.967960in}{1.372473in}}{\pgfqpoint{1.964688in}{1.380373in}}{\pgfqpoint{1.958864in}{1.386197in}}%
\pgfpathcurveto{\pgfqpoint{1.953040in}{1.392021in}}{\pgfqpoint{1.945140in}{1.395293in}}{\pgfqpoint{1.936904in}{1.395293in}}%
\pgfpathcurveto{\pgfqpoint{1.928667in}{1.395293in}}{\pgfqpoint{1.920767in}{1.392021in}}{\pgfqpoint{1.914943in}{1.386197in}}%
\pgfpathcurveto{\pgfqpoint{1.909119in}{1.380373in}}{\pgfqpoint{1.905847in}{1.372473in}}{\pgfqpoint{1.905847in}{1.364236in}}%
\pgfpathcurveto{\pgfqpoint{1.905847in}{1.356000in}}{\pgfqpoint{1.909119in}{1.348100in}}{\pgfqpoint{1.914943in}{1.342276in}}%
\pgfpathcurveto{\pgfqpoint{1.920767in}{1.336452in}}{\pgfqpoint{1.928667in}{1.333180in}}{\pgfqpoint{1.936904in}{1.333180in}}%
\pgfpathclose%
\pgfusepath{stroke,fill}%
\end{pgfscope}%
\begin{pgfscope}%
\pgfpathrectangle{\pgfqpoint{0.100000in}{0.212622in}}{\pgfqpoint{3.696000in}{3.696000in}}%
\pgfusepath{clip}%
\pgfsetbuttcap%
\pgfsetroundjoin%
\definecolor{currentfill}{rgb}{0.121569,0.466667,0.705882}%
\pgfsetfillcolor{currentfill}%
\pgfsetfillopacity{0.638377}%
\pgfsetlinewidth{1.003750pt}%
\definecolor{currentstroke}{rgb}{0.121569,0.466667,0.705882}%
\pgfsetstrokecolor{currentstroke}%
\pgfsetstrokeopacity{0.638377}%
\pgfsetdash{}{0pt}%
\pgfpathmoveto{\pgfqpoint{1.943421in}{1.331281in}}%
\pgfpathcurveto{\pgfqpoint{1.951657in}{1.331281in}}{\pgfqpoint{1.959557in}{1.334554in}}{\pgfqpoint{1.965381in}{1.340378in}}%
\pgfpathcurveto{\pgfqpoint{1.971205in}{1.346202in}}{\pgfqpoint{1.974477in}{1.354102in}}{\pgfqpoint{1.974477in}{1.362338in}}%
\pgfpathcurveto{\pgfqpoint{1.974477in}{1.370574in}}{\pgfqpoint{1.971205in}{1.378474in}}{\pgfqpoint{1.965381in}{1.384298in}}%
\pgfpathcurveto{\pgfqpoint{1.959557in}{1.390122in}}{\pgfqpoint{1.951657in}{1.393394in}}{\pgfqpoint{1.943421in}{1.393394in}}%
\pgfpathcurveto{\pgfqpoint{1.935184in}{1.393394in}}{\pgfqpoint{1.927284in}{1.390122in}}{\pgfqpoint{1.921460in}{1.384298in}}%
\pgfpathcurveto{\pgfqpoint{1.915637in}{1.378474in}}{\pgfqpoint{1.912364in}{1.370574in}}{\pgfqpoint{1.912364in}{1.362338in}}%
\pgfpathcurveto{\pgfqpoint{1.912364in}{1.354102in}}{\pgfqpoint{1.915637in}{1.346202in}}{\pgfqpoint{1.921460in}{1.340378in}}%
\pgfpathcurveto{\pgfqpoint{1.927284in}{1.334554in}}{\pgfqpoint{1.935184in}{1.331281in}}{\pgfqpoint{1.943421in}{1.331281in}}%
\pgfpathclose%
\pgfusepath{stroke,fill}%
\end{pgfscope}%
\begin{pgfscope}%
\pgfpathrectangle{\pgfqpoint{0.100000in}{0.212622in}}{\pgfqpoint{3.696000in}{3.696000in}}%
\pgfusepath{clip}%
\pgfsetbuttcap%
\pgfsetroundjoin%
\definecolor{currentfill}{rgb}{0.121569,0.466667,0.705882}%
\pgfsetfillcolor{currentfill}%
\pgfsetfillopacity{0.641340}%
\pgfsetlinewidth{1.003750pt}%
\definecolor{currentstroke}{rgb}{0.121569,0.466667,0.705882}%
\pgfsetstrokecolor{currentstroke}%
\pgfsetstrokeopacity{0.641340}%
\pgfsetdash{}{0pt}%
\pgfpathmoveto{\pgfqpoint{1.950046in}{1.328607in}}%
\pgfpathcurveto{\pgfqpoint{1.958282in}{1.328607in}}{\pgfqpoint{1.966182in}{1.331879in}}{\pgfqpoint{1.972006in}{1.337703in}}%
\pgfpathcurveto{\pgfqpoint{1.977830in}{1.343527in}}{\pgfqpoint{1.981102in}{1.351427in}}{\pgfqpoint{1.981102in}{1.359663in}}%
\pgfpathcurveto{\pgfqpoint{1.981102in}{1.367900in}}{\pgfqpoint{1.977830in}{1.375800in}}{\pgfqpoint{1.972006in}{1.381624in}}%
\pgfpathcurveto{\pgfqpoint{1.966182in}{1.387447in}}{\pgfqpoint{1.958282in}{1.390720in}}{\pgfqpoint{1.950046in}{1.390720in}}%
\pgfpathcurveto{\pgfqpoint{1.941810in}{1.390720in}}{\pgfqpoint{1.933910in}{1.387447in}}{\pgfqpoint{1.928086in}{1.381624in}}%
\pgfpathcurveto{\pgfqpoint{1.922262in}{1.375800in}}{\pgfqpoint{1.918989in}{1.367900in}}{\pgfqpoint{1.918989in}{1.359663in}}%
\pgfpathcurveto{\pgfqpoint{1.918989in}{1.351427in}}{\pgfqpoint{1.922262in}{1.343527in}}{\pgfqpoint{1.928086in}{1.337703in}}%
\pgfpathcurveto{\pgfqpoint{1.933910in}{1.331879in}}{\pgfqpoint{1.941810in}{1.328607in}}{\pgfqpoint{1.950046in}{1.328607in}}%
\pgfpathclose%
\pgfusepath{stroke,fill}%
\end{pgfscope}%
\begin{pgfscope}%
\pgfpathrectangle{\pgfqpoint{0.100000in}{0.212622in}}{\pgfqpoint{3.696000in}{3.696000in}}%
\pgfusepath{clip}%
\pgfsetbuttcap%
\pgfsetroundjoin%
\definecolor{currentfill}{rgb}{0.121569,0.466667,0.705882}%
\pgfsetfillcolor{currentfill}%
\pgfsetfillopacity{0.645170}%
\pgfsetlinewidth{1.003750pt}%
\definecolor{currentstroke}{rgb}{0.121569,0.466667,0.705882}%
\pgfsetstrokecolor{currentstroke}%
\pgfsetstrokeopacity{0.645170}%
\pgfsetdash{}{0pt}%
\pgfpathmoveto{\pgfqpoint{1.957185in}{1.325164in}}%
\pgfpathcurveto{\pgfqpoint{1.965421in}{1.325164in}}{\pgfqpoint{1.973321in}{1.328436in}}{\pgfqpoint{1.979145in}{1.334260in}}%
\pgfpathcurveto{\pgfqpoint{1.984969in}{1.340084in}}{\pgfqpoint{1.988241in}{1.347984in}}{\pgfqpoint{1.988241in}{1.356220in}}%
\pgfpathcurveto{\pgfqpoint{1.988241in}{1.364456in}}{\pgfqpoint{1.984969in}{1.372356in}}{\pgfqpoint{1.979145in}{1.378180in}}%
\pgfpathcurveto{\pgfqpoint{1.973321in}{1.384004in}}{\pgfqpoint{1.965421in}{1.387277in}}{\pgfqpoint{1.957185in}{1.387277in}}%
\pgfpathcurveto{\pgfqpoint{1.948948in}{1.387277in}}{\pgfqpoint{1.941048in}{1.384004in}}{\pgfqpoint{1.935224in}{1.378180in}}%
\pgfpathcurveto{\pgfqpoint{1.929401in}{1.372356in}}{\pgfqpoint{1.926128in}{1.364456in}}{\pgfqpoint{1.926128in}{1.356220in}}%
\pgfpathcurveto{\pgfqpoint{1.926128in}{1.347984in}}{\pgfqpoint{1.929401in}{1.340084in}}{\pgfqpoint{1.935224in}{1.334260in}}%
\pgfpathcurveto{\pgfqpoint{1.941048in}{1.328436in}}{\pgfqpoint{1.948948in}{1.325164in}}{\pgfqpoint{1.957185in}{1.325164in}}%
\pgfpathclose%
\pgfusepath{stroke,fill}%
\end{pgfscope}%
\begin{pgfscope}%
\pgfpathrectangle{\pgfqpoint{0.100000in}{0.212622in}}{\pgfqpoint{3.696000in}{3.696000in}}%
\pgfusepath{clip}%
\pgfsetbuttcap%
\pgfsetroundjoin%
\definecolor{currentfill}{rgb}{0.121569,0.466667,0.705882}%
\pgfsetfillcolor{currentfill}%
\pgfsetfillopacity{0.649059}%
\pgfsetlinewidth{1.003750pt}%
\definecolor{currentstroke}{rgb}{0.121569,0.466667,0.705882}%
\pgfsetstrokecolor{currentstroke}%
\pgfsetstrokeopacity{0.649059}%
\pgfsetdash{}{0pt}%
\pgfpathmoveto{\pgfqpoint{1.966026in}{1.321884in}}%
\pgfpathcurveto{\pgfqpoint{1.974262in}{1.321884in}}{\pgfqpoint{1.982162in}{1.325156in}}{\pgfqpoint{1.987986in}{1.330980in}}%
\pgfpathcurveto{\pgfqpoint{1.993810in}{1.336804in}}{\pgfqpoint{1.997082in}{1.344704in}}{\pgfqpoint{1.997082in}{1.352941in}}%
\pgfpathcurveto{\pgfqpoint{1.997082in}{1.361177in}}{\pgfqpoint{1.993810in}{1.369077in}}{\pgfqpoint{1.987986in}{1.374901in}}%
\pgfpathcurveto{\pgfqpoint{1.982162in}{1.380725in}}{\pgfqpoint{1.974262in}{1.383997in}}{\pgfqpoint{1.966026in}{1.383997in}}%
\pgfpathcurveto{\pgfqpoint{1.957789in}{1.383997in}}{\pgfqpoint{1.949889in}{1.380725in}}{\pgfqpoint{1.944065in}{1.374901in}}%
\pgfpathcurveto{\pgfqpoint{1.938241in}{1.369077in}}{\pgfqpoint{1.934969in}{1.361177in}}{\pgfqpoint{1.934969in}{1.352941in}}%
\pgfpathcurveto{\pgfqpoint{1.934969in}{1.344704in}}{\pgfqpoint{1.938241in}{1.336804in}}{\pgfqpoint{1.944065in}{1.330980in}}%
\pgfpathcurveto{\pgfqpoint{1.949889in}{1.325156in}}{\pgfqpoint{1.957789in}{1.321884in}}{\pgfqpoint{1.966026in}{1.321884in}}%
\pgfpathclose%
\pgfusepath{stroke,fill}%
\end{pgfscope}%
\begin{pgfscope}%
\pgfpathrectangle{\pgfqpoint{0.100000in}{0.212622in}}{\pgfqpoint{3.696000in}{3.696000in}}%
\pgfusepath{clip}%
\pgfsetbuttcap%
\pgfsetroundjoin%
\definecolor{currentfill}{rgb}{0.121569,0.466667,0.705882}%
\pgfsetfillcolor{currentfill}%
\pgfsetfillopacity{0.653168}%
\pgfsetlinewidth{1.003750pt}%
\definecolor{currentstroke}{rgb}{0.121569,0.466667,0.705882}%
\pgfsetstrokecolor{currentstroke}%
\pgfsetstrokeopacity{0.653168}%
\pgfsetdash{}{0pt}%
\pgfpathmoveto{\pgfqpoint{1.975820in}{1.318751in}}%
\pgfpathcurveto{\pgfqpoint{1.984056in}{1.318751in}}{\pgfqpoint{1.991956in}{1.322023in}}{\pgfqpoint{1.997780in}{1.327847in}}%
\pgfpathcurveto{\pgfqpoint{2.003604in}{1.333671in}}{\pgfqpoint{2.006876in}{1.341571in}}{\pgfqpoint{2.006876in}{1.349807in}}%
\pgfpathcurveto{\pgfqpoint{2.006876in}{1.358044in}}{\pgfqpoint{2.003604in}{1.365944in}}{\pgfqpoint{1.997780in}{1.371768in}}%
\pgfpathcurveto{\pgfqpoint{1.991956in}{1.377592in}}{\pgfqpoint{1.984056in}{1.380864in}}{\pgfqpoint{1.975820in}{1.380864in}}%
\pgfpathcurveto{\pgfqpoint{1.967584in}{1.380864in}}{\pgfqpoint{1.959683in}{1.377592in}}{\pgfqpoint{1.953860in}{1.371768in}}%
\pgfpathcurveto{\pgfqpoint{1.948036in}{1.365944in}}{\pgfqpoint{1.944763in}{1.358044in}}{\pgfqpoint{1.944763in}{1.349807in}}%
\pgfpathcurveto{\pgfqpoint{1.944763in}{1.341571in}}{\pgfqpoint{1.948036in}{1.333671in}}{\pgfqpoint{1.953860in}{1.327847in}}%
\pgfpathcurveto{\pgfqpoint{1.959683in}{1.322023in}}{\pgfqpoint{1.967584in}{1.318751in}}{\pgfqpoint{1.975820in}{1.318751in}}%
\pgfpathclose%
\pgfusepath{stroke,fill}%
\end{pgfscope}%
\begin{pgfscope}%
\pgfpathrectangle{\pgfqpoint{0.100000in}{0.212622in}}{\pgfqpoint{3.696000in}{3.696000in}}%
\pgfusepath{clip}%
\pgfsetbuttcap%
\pgfsetroundjoin%
\definecolor{currentfill}{rgb}{0.121569,0.466667,0.705882}%
\pgfsetfillcolor{currentfill}%
\pgfsetfillopacity{0.655335}%
\pgfsetlinewidth{1.003750pt}%
\definecolor{currentstroke}{rgb}{0.121569,0.466667,0.705882}%
\pgfsetstrokecolor{currentstroke}%
\pgfsetstrokeopacity{0.655335}%
\pgfsetdash{}{0pt}%
\pgfpathmoveto{\pgfqpoint{1.981244in}{1.316948in}}%
\pgfpathcurveto{\pgfqpoint{1.989480in}{1.316948in}}{\pgfqpoint{1.997380in}{1.320221in}}{\pgfqpoint{2.003204in}{1.326045in}}%
\pgfpathcurveto{\pgfqpoint{2.009028in}{1.331868in}}{\pgfqpoint{2.012301in}{1.339769in}}{\pgfqpoint{2.012301in}{1.348005in}}%
\pgfpathcurveto{\pgfqpoint{2.012301in}{1.356241in}}{\pgfqpoint{2.009028in}{1.364141in}}{\pgfqpoint{2.003204in}{1.369965in}}%
\pgfpathcurveto{\pgfqpoint{1.997380in}{1.375789in}}{\pgfqpoint{1.989480in}{1.379061in}}{\pgfqpoint{1.981244in}{1.379061in}}%
\pgfpathcurveto{\pgfqpoint{1.973008in}{1.379061in}}{\pgfqpoint{1.965108in}{1.375789in}}{\pgfqpoint{1.959284in}{1.369965in}}%
\pgfpathcurveto{\pgfqpoint{1.953460in}{1.364141in}}{\pgfqpoint{1.950188in}{1.356241in}}{\pgfqpoint{1.950188in}{1.348005in}}%
\pgfpathcurveto{\pgfqpoint{1.950188in}{1.339769in}}{\pgfqpoint{1.953460in}{1.331868in}}{\pgfqpoint{1.959284in}{1.326045in}}%
\pgfpathcurveto{\pgfqpoint{1.965108in}{1.320221in}}{\pgfqpoint{1.973008in}{1.316948in}}{\pgfqpoint{1.981244in}{1.316948in}}%
\pgfpathclose%
\pgfusepath{stroke,fill}%
\end{pgfscope}%
\begin{pgfscope}%
\pgfpathrectangle{\pgfqpoint{0.100000in}{0.212622in}}{\pgfqpoint{3.696000in}{3.696000in}}%
\pgfusepath{clip}%
\pgfsetbuttcap%
\pgfsetroundjoin%
\definecolor{currentfill}{rgb}{0.121569,0.466667,0.705882}%
\pgfsetfillcolor{currentfill}%
\pgfsetfillopacity{0.656533}%
\pgfsetlinewidth{1.003750pt}%
\definecolor{currentstroke}{rgb}{0.121569,0.466667,0.705882}%
\pgfsetstrokecolor{currentstroke}%
\pgfsetstrokeopacity{0.656533}%
\pgfsetdash{}{0pt}%
\pgfpathmoveto{\pgfqpoint{1.984205in}{1.315886in}}%
\pgfpathcurveto{\pgfqpoint{1.992441in}{1.315886in}}{\pgfqpoint{2.000341in}{1.319158in}}{\pgfqpoint{2.006165in}{1.324982in}}%
\pgfpathcurveto{\pgfqpoint{2.011989in}{1.330806in}}{\pgfqpoint{2.015261in}{1.338706in}}{\pgfqpoint{2.015261in}{1.346942in}}%
\pgfpathcurveto{\pgfqpoint{2.015261in}{1.355179in}}{\pgfqpoint{2.011989in}{1.363079in}}{\pgfqpoint{2.006165in}{1.368903in}}%
\pgfpathcurveto{\pgfqpoint{2.000341in}{1.374726in}}{\pgfqpoint{1.992441in}{1.377999in}}{\pgfqpoint{1.984205in}{1.377999in}}%
\pgfpathcurveto{\pgfqpoint{1.975968in}{1.377999in}}{\pgfqpoint{1.968068in}{1.374726in}}{\pgfqpoint{1.962244in}{1.368903in}}%
\pgfpathcurveto{\pgfqpoint{1.956421in}{1.363079in}}{\pgfqpoint{1.953148in}{1.355179in}}{\pgfqpoint{1.953148in}{1.346942in}}%
\pgfpathcurveto{\pgfqpoint{1.953148in}{1.338706in}}{\pgfqpoint{1.956421in}{1.330806in}}{\pgfqpoint{1.962244in}{1.324982in}}%
\pgfpathcurveto{\pgfqpoint{1.968068in}{1.319158in}}{\pgfqpoint{1.975968in}{1.315886in}}{\pgfqpoint{1.984205in}{1.315886in}}%
\pgfpathclose%
\pgfusepath{stroke,fill}%
\end{pgfscope}%
\begin{pgfscope}%
\pgfpathrectangle{\pgfqpoint{0.100000in}{0.212622in}}{\pgfqpoint{3.696000in}{3.696000in}}%
\pgfusepath{clip}%
\pgfsetbuttcap%
\pgfsetroundjoin%
\definecolor{currentfill}{rgb}{0.121569,0.466667,0.705882}%
\pgfsetfillcolor{currentfill}%
\pgfsetfillopacity{0.658265}%
\pgfsetlinewidth{1.003750pt}%
\definecolor{currentstroke}{rgb}{0.121569,0.466667,0.705882}%
\pgfsetstrokecolor{currentstroke}%
\pgfsetstrokeopacity{0.658265}%
\pgfsetdash{}{0pt}%
\pgfpathmoveto{\pgfqpoint{1.987511in}{1.314326in}}%
\pgfpathcurveto{\pgfqpoint{1.995747in}{1.314326in}}{\pgfqpoint{2.003647in}{1.317599in}}{\pgfqpoint{2.009471in}{1.323422in}}%
\pgfpathcurveto{\pgfqpoint{2.015295in}{1.329246in}}{\pgfqpoint{2.018567in}{1.337146in}}{\pgfqpoint{2.018567in}{1.345383in}}%
\pgfpathcurveto{\pgfqpoint{2.018567in}{1.353619in}}{\pgfqpoint{2.015295in}{1.361519in}}{\pgfqpoint{2.009471in}{1.367343in}}%
\pgfpathcurveto{\pgfqpoint{2.003647in}{1.373167in}}{\pgfqpoint{1.995747in}{1.376439in}}{\pgfqpoint{1.987511in}{1.376439in}}%
\pgfpathcurveto{\pgfqpoint{1.979274in}{1.376439in}}{\pgfqpoint{1.971374in}{1.373167in}}{\pgfqpoint{1.965550in}{1.367343in}}%
\pgfpathcurveto{\pgfqpoint{1.959727in}{1.361519in}}{\pgfqpoint{1.956454in}{1.353619in}}{\pgfqpoint{1.956454in}{1.345383in}}%
\pgfpathcurveto{\pgfqpoint{1.956454in}{1.337146in}}{\pgfqpoint{1.959727in}{1.329246in}}{\pgfqpoint{1.965550in}{1.323422in}}%
\pgfpathcurveto{\pgfqpoint{1.971374in}{1.317599in}}{\pgfqpoint{1.979274in}{1.314326in}}{\pgfqpoint{1.987511in}{1.314326in}}%
\pgfpathclose%
\pgfusepath{stroke,fill}%
\end{pgfscope}%
\begin{pgfscope}%
\pgfpathrectangle{\pgfqpoint{0.100000in}{0.212622in}}{\pgfqpoint{3.696000in}{3.696000in}}%
\pgfusepath{clip}%
\pgfsetbuttcap%
\pgfsetroundjoin%
\definecolor{currentfill}{rgb}{0.121569,0.466667,0.705882}%
\pgfsetfillcolor{currentfill}%
\pgfsetfillopacity{0.659936}%
\pgfsetlinewidth{1.003750pt}%
\definecolor{currentstroke}{rgb}{0.121569,0.466667,0.705882}%
\pgfsetstrokecolor{currentstroke}%
\pgfsetstrokeopacity{0.659936}%
\pgfsetdash{}{0pt}%
\pgfpathmoveto{\pgfqpoint{1.991875in}{1.312934in}}%
\pgfpathcurveto{\pgfqpoint{2.000112in}{1.312934in}}{\pgfqpoint{2.008012in}{1.316206in}}{\pgfqpoint{2.013836in}{1.322030in}}%
\pgfpathcurveto{\pgfqpoint{2.019660in}{1.327854in}}{\pgfqpoint{2.022932in}{1.335754in}}{\pgfqpoint{2.022932in}{1.343990in}}%
\pgfpathcurveto{\pgfqpoint{2.022932in}{1.352226in}}{\pgfqpoint{2.019660in}{1.360126in}}{\pgfqpoint{2.013836in}{1.365950in}}%
\pgfpathcurveto{\pgfqpoint{2.008012in}{1.371774in}}{\pgfqpoint{2.000112in}{1.375047in}}{\pgfqpoint{1.991875in}{1.375047in}}%
\pgfpathcurveto{\pgfqpoint{1.983639in}{1.375047in}}{\pgfqpoint{1.975739in}{1.371774in}}{\pgfqpoint{1.969915in}{1.365950in}}%
\pgfpathcurveto{\pgfqpoint{1.964091in}{1.360126in}}{\pgfqpoint{1.960819in}{1.352226in}}{\pgfqpoint{1.960819in}{1.343990in}}%
\pgfpathcurveto{\pgfqpoint{1.960819in}{1.335754in}}{\pgfqpoint{1.964091in}{1.327854in}}{\pgfqpoint{1.969915in}{1.322030in}}%
\pgfpathcurveto{\pgfqpoint{1.975739in}{1.316206in}}{\pgfqpoint{1.983639in}{1.312934in}}{\pgfqpoint{1.991875in}{1.312934in}}%
\pgfpathclose%
\pgfusepath{stroke,fill}%
\end{pgfscope}%
\begin{pgfscope}%
\pgfpathrectangle{\pgfqpoint{0.100000in}{0.212622in}}{\pgfqpoint{3.696000in}{3.696000in}}%
\pgfusepath{clip}%
\pgfsetbuttcap%
\pgfsetroundjoin%
\definecolor{currentfill}{rgb}{0.121569,0.466667,0.705882}%
\pgfsetfillcolor{currentfill}%
\pgfsetfillopacity{0.661849}%
\pgfsetlinewidth{1.003750pt}%
\definecolor{currentstroke}{rgb}{0.121569,0.466667,0.705882}%
\pgfsetstrokecolor{currentstroke}%
\pgfsetstrokeopacity{0.661849}%
\pgfsetdash{}{0pt}%
\pgfpathmoveto{\pgfqpoint{1.997870in}{1.311447in}}%
\pgfpathcurveto{\pgfqpoint{2.006107in}{1.311447in}}{\pgfqpoint{2.014007in}{1.314719in}}{\pgfqpoint{2.019831in}{1.320543in}}%
\pgfpathcurveto{\pgfqpoint{2.025655in}{1.326367in}}{\pgfqpoint{2.028927in}{1.334267in}}{\pgfqpoint{2.028927in}{1.342504in}}%
\pgfpathcurveto{\pgfqpoint{2.028927in}{1.350740in}}{\pgfqpoint{2.025655in}{1.358640in}}{\pgfqpoint{2.019831in}{1.364464in}}%
\pgfpathcurveto{\pgfqpoint{2.014007in}{1.370288in}}{\pgfqpoint{2.006107in}{1.373560in}}{\pgfqpoint{1.997870in}{1.373560in}}%
\pgfpathcurveto{\pgfqpoint{1.989634in}{1.373560in}}{\pgfqpoint{1.981734in}{1.370288in}}{\pgfqpoint{1.975910in}{1.364464in}}%
\pgfpathcurveto{\pgfqpoint{1.970086in}{1.358640in}}{\pgfqpoint{1.966814in}{1.350740in}}{\pgfqpoint{1.966814in}{1.342504in}}%
\pgfpathcurveto{\pgfqpoint{1.966814in}{1.334267in}}{\pgfqpoint{1.970086in}{1.326367in}}{\pgfqpoint{1.975910in}{1.320543in}}%
\pgfpathcurveto{\pgfqpoint{1.981734in}{1.314719in}}{\pgfqpoint{1.989634in}{1.311447in}}{\pgfqpoint{1.997870in}{1.311447in}}%
\pgfpathclose%
\pgfusepath{stroke,fill}%
\end{pgfscope}%
\begin{pgfscope}%
\pgfpathrectangle{\pgfqpoint{0.100000in}{0.212622in}}{\pgfqpoint{3.696000in}{3.696000in}}%
\pgfusepath{clip}%
\pgfsetbuttcap%
\pgfsetroundjoin%
\definecolor{currentfill}{rgb}{0.121569,0.466667,0.705882}%
\pgfsetfillcolor{currentfill}%
\pgfsetfillopacity{0.663732}%
\pgfsetlinewidth{1.003750pt}%
\definecolor{currentstroke}{rgb}{0.121569,0.466667,0.705882}%
\pgfsetstrokecolor{currentstroke}%
\pgfsetstrokeopacity{0.663732}%
\pgfsetdash{}{0pt}%
\pgfpathmoveto{\pgfqpoint{2.006122in}{1.310117in}}%
\pgfpathcurveto{\pgfqpoint{2.014358in}{1.310117in}}{\pgfqpoint{2.022258in}{1.313389in}}{\pgfqpoint{2.028082in}{1.319213in}}%
\pgfpathcurveto{\pgfqpoint{2.033906in}{1.325037in}}{\pgfqpoint{2.037178in}{1.332937in}}{\pgfqpoint{2.037178in}{1.341173in}}%
\pgfpathcurveto{\pgfqpoint{2.037178in}{1.349409in}}{\pgfqpoint{2.033906in}{1.357310in}}{\pgfqpoint{2.028082in}{1.363133in}}%
\pgfpathcurveto{\pgfqpoint{2.022258in}{1.368957in}}{\pgfqpoint{2.014358in}{1.372230in}}{\pgfqpoint{2.006122in}{1.372230in}}%
\pgfpathcurveto{\pgfqpoint{1.997886in}{1.372230in}}{\pgfqpoint{1.989986in}{1.368957in}}{\pgfqpoint{1.984162in}{1.363133in}}%
\pgfpathcurveto{\pgfqpoint{1.978338in}{1.357310in}}{\pgfqpoint{1.975065in}{1.349409in}}{\pgfqpoint{1.975065in}{1.341173in}}%
\pgfpathcurveto{\pgfqpoint{1.975065in}{1.332937in}}{\pgfqpoint{1.978338in}{1.325037in}}{\pgfqpoint{1.984162in}{1.319213in}}%
\pgfpathcurveto{\pgfqpoint{1.989986in}{1.313389in}}{\pgfqpoint{1.997886in}{1.310117in}}{\pgfqpoint{2.006122in}{1.310117in}}%
\pgfpathclose%
\pgfusepath{stroke,fill}%
\end{pgfscope}%
\begin{pgfscope}%
\pgfpathrectangle{\pgfqpoint{0.100000in}{0.212622in}}{\pgfqpoint{3.696000in}{3.696000in}}%
\pgfusepath{clip}%
\pgfsetbuttcap%
\pgfsetroundjoin%
\definecolor{currentfill}{rgb}{0.121569,0.466667,0.705882}%
\pgfsetfillcolor{currentfill}%
\pgfsetfillopacity{0.666409}%
\pgfsetlinewidth{1.003750pt}%
\definecolor{currentstroke}{rgb}{0.121569,0.466667,0.705882}%
\pgfsetstrokecolor{currentstroke}%
\pgfsetstrokeopacity{0.666409}%
\pgfsetdash{}{0pt}%
\pgfpathmoveto{\pgfqpoint{2.014355in}{1.308185in}}%
\pgfpathcurveto{\pgfqpoint{2.022591in}{1.308185in}}{\pgfqpoint{2.030492in}{1.311457in}}{\pgfqpoint{2.036315in}{1.317281in}}%
\pgfpathcurveto{\pgfqpoint{2.042139in}{1.323105in}}{\pgfqpoint{2.045412in}{1.331005in}}{\pgfqpoint{2.045412in}{1.339241in}}%
\pgfpathcurveto{\pgfqpoint{2.045412in}{1.347477in}}{\pgfqpoint{2.042139in}{1.355377in}}{\pgfqpoint{2.036315in}{1.361201in}}%
\pgfpathcurveto{\pgfqpoint{2.030492in}{1.367025in}}{\pgfqpoint{2.022591in}{1.370298in}}{\pgfqpoint{2.014355in}{1.370298in}}%
\pgfpathcurveto{\pgfqpoint{2.006119in}{1.370298in}}{\pgfqpoint{1.998219in}{1.367025in}}{\pgfqpoint{1.992395in}{1.361201in}}%
\pgfpathcurveto{\pgfqpoint{1.986571in}{1.355377in}}{\pgfqpoint{1.983299in}{1.347477in}}{\pgfqpoint{1.983299in}{1.339241in}}%
\pgfpathcurveto{\pgfqpoint{1.983299in}{1.331005in}}{\pgfqpoint{1.986571in}{1.323105in}}{\pgfqpoint{1.992395in}{1.317281in}}%
\pgfpathcurveto{\pgfqpoint{1.998219in}{1.311457in}}{\pgfqpoint{2.006119in}{1.308185in}}{\pgfqpoint{2.014355in}{1.308185in}}%
\pgfpathclose%
\pgfusepath{stroke,fill}%
\end{pgfscope}%
\begin{pgfscope}%
\pgfpathrectangle{\pgfqpoint{0.100000in}{0.212622in}}{\pgfqpoint{3.696000in}{3.696000in}}%
\pgfusepath{clip}%
\pgfsetbuttcap%
\pgfsetroundjoin%
\definecolor{currentfill}{rgb}{0.121569,0.466667,0.705882}%
\pgfsetfillcolor{currentfill}%
\pgfsetfillopacity{0.668094}%
\pgfsetlinewidth{1.003750pt}%
\definecolor{currentstroke}{rgb}{0.121569,0.466667,0.705882}%
\pgfsetstrokecolor{currentstroke}%
\pgfsetstrokeopacity{0.668094}%
\pgfsetdash{}{0pt}%
\pgfpathmoveto{\pgfqpoint{2.018651in}{1.306768in}}%
\pgfpathcurveto{\pgfqpoint{2.026887in}{1.306768in}}{\pgfqpoint{2.034787in}{1.310040in}}{\pgfqpoint{2.040611in}{1.315864in}}%
\pgfpathcurveto{\pgfqpoint{2.046435in}{1.321688in}}{\pgfqpoint{2.049707in}{1.329588in}}{\pgfqpoint{2.049707in}{1.337825in}}%
\pgfpathcurveto{\pgfqpoint{2.049707in}{1.346061in}}{\pgfqpoint{2.046435in}{1.353961in}}{\pgfqpoint{2.040611in}{1.359785in}}%
\pgfpathcurveto{\pgfqpoint{2.034787in}{1.365609in}}{\pgfqpoint{2.026887in}{1.368881in}}{\pgfqpoint{2.018651in}{1.368881in}}%
\pgfpathcurveto{\pgfqpoint{2.010415in}{1.368881in}}{\pgfqpoint{2.002515in}{1.365609in}}{\pgfqpoint{1.996691in}{1.359785in}}%
\pgfpathcurveto{\pgfqpoint{1.990867in}{1.353961in}}{\pgfqpoint{1.987594in}{1.346061in}}{\pgfqpoint{1.987594in}{1.337825in}}%
\pgfpathcurveto{\pgfqpoint{1.987594in}{1.329588in}}{\pgfqpoint{1.990867in}{1.321688in}}{\pgfqpoint{1.996691in}{1.315864in}}%
\pgfpathcurveto{\pgfqpoint{2.002515in}{1.310040in}}{\pgfqpoint{2.010415in}{1.306768in}}{\pgfqpoint{2.018651in}{1.306768in}}%
\pgfpathclose%
\pgfusepath{stroke,fill}%
\end{pgfscope}%
\begin{pgfscope}%
\pgfpathrectangle{\pgfqpoint{0.100000in}{0.212622in}}{\pgfqpoint{3.696000in}{3.696000in}}%
\pgfusepath{clip}%
\pgfsetbuttcap%
\pgfsetroundjoin%
\definecolor{currentfill}{rgb}{0.121569,0.466667,0.705882}%
\pgfsetfillcolor{currentfill}%
\pgfsetfillopacity{0.668876}%
\pgfsetlinewidth{1.003750pt}%
\definecolor{currentstroke}{rgb}{0.121569,0.466667,0.705882}%
\pgfsetstrokecolor{currentstroke}%
\pgfsetstrokeopacity{0.668876}%
\pgfsetdash{}{0pt}%
\pgfpathmoveto{\pgfqpoint{2.021145in}{1.306131in}}%
\pgfpathcurveto{\pgfqpoint{2.029381in}{1.306131in}}{\pgfqpoint{2.037281in}{1.309403in}}{\pgfqpoint{2.043105in}{1.315227in}}%
\pgfpathcurveto{\pgfqpoint{2.048929in}{1.321051in}}{\pgfqpoint{2.052201in}{1.328951in}}{\pgfqpoint{2.052201in}{1.337187in}}%
\pgfpathcurveto{\pgfqpoint{2.052201in}{1.345424in}}{\pgfqpoint{2.048929in}{1.353324in}}{\pgfqpoint{2.043105in}{1.359148in}}%
\pgfpathcurveto{\pgfqpoint{2.037281in}{1.364971in}}{\pgfqpoint{2.029381in}{1.368244in}}{\pgfqpoint{2.021145in}{1.368244in}}%
\pgfpathcurveto{\pgfqpoint{2.012909in}{1.368244in}}{\pgfqpoint{2.005008in}{1.364971in}}{\pgfqpoint{1.999185in}{1.359148in}}%
\pgfpathcurveto{\pgfqpoint{1.993361in}{1.353324in}}{\pgfqpoint{1.990088in}{1.345424in}}{\pgfqpoint{1.990088in}{1.337187in}}%
\pgfpathcurveto{\pgfqpoint{1.990088in}{1.328951in}}{\pgfqpoint{1.993361in}{1.321051in}}{\pgfqpoint{1.999185in}{1.315227in}}%
\pgfpathcurveto{\pgfqpoint{2.005008in}{1.309403in}}{\pgfqpoint{2.012909in}{1.306131in}}{\pgfqpoint{2.021145in}{1.306131in}}%
\pgfpathclose%
\pgfusepath{stroke,fill}%
\end{pgfscope}%
\begin{pgfscope}%
\pgfpathrectangle{\pgfqpoint{0.100000in}{0.212622in}}{\pgfqpoint{3.696000in}{3.696000in}}%
\pgfusepath{clip}%
\pgfsetbuttcap%
\pgfsetroundjoin%
\definecolor{currentfill}{rgb}{0.121569,0.466667,0.705882}%
\pgfsetfillcolor{currentfill}%
\pgfsetfillopacity{0.670322}%
\pgfsetlinewidth{1.003750pt}%
\definecolor{currentstroke}{rgb}{0.121569,0.466667,0.705882}%
\pgfsetstrokecolor{currentstroke}%
\pgfsetstrokeopacity{0.670322}%
\pgfsetdash{}{0pt}%
\pgfpathmoveto{\pgfqpoint{2.024514in}{1.304907in}}%
\pgfpathcurveto{\pgfqpoint{2.032750in}{1.304907in}}{\pgfqpoint{2.040650in}{1.308179in}}{\pgfqpoint{2.046474in}{1.314003in}}%
\pgfpathcurveto{\pgfqpoint{2.052298in}{1.319827in}}{\pgfqpoint{2.055571in}{1.327727in}}{\pgfqpoint{2.055571in}{1.335963in}}%
\pgfpathcurveto{\pgfqpoint{2.055571in}{1.344199in}}{\pgfqpoint{2.052298in}{1.352099in}}{\pgfqpoint{2.046474in}{1.357923in}}%
\pgfpathcurveto{\pgfqpoint{2.040650in}{1.363747in}}{\pgfqpoint{2.032750in}{1.367020in}}{\pgfqpoint{2.024514in}{1.367020in}}%
\pgfpathcurveto{\pgfqpoint{2.016278in}{1.367020in}}{\pgfqpoint{2.008378in}{1.363747in}}{\pgfqpoint{2.002554in}{1.357923in}}%
\pgfpathcurveto{\pgfqpoint{1.996730in}{1.352099in}}{\pgfqpoint{1.993458in}{1.344199in}}{\pgfqpoint{1.993458in}{1.335963in}}%
\pgfpathcurveto{\pgfqpoint{1.993458in}{1.327727in}}{\pgfqpoint{1.996730in}{1.319827in}}{\pgfqpoint{2.002554in}{1.314003in}}%
\pgfpathcurveto{\pgfqpoint{2.008378in}{1.308179in}}{\pgfqpoint{2.016278in}{1.304907in}}{\pgfqpoint{2.024514in}{1.304907in}}%
\pgfpathclose%
\pgfusepath{stroke,fill}%
\end{pgfscope}%
\begin{pgfscope}%
\pgfpathrectangle{\pgfqpoint{0.100000in}{0.212622in}}{\pgfqpoint{3.696000in}{3.696000in}}%
\pgfusepath{clip}%
\pgfsetbuttcap%
\pgfsetroundjoin%
\definecolor{currentfill}{rgb}{0.121569,0.466667,0.705882}%
\pgfsetfillcolor{currentfill}%
\pgfsetfillopacity{0.672640}%
\pgfsetlinewidth{1.003750pt}%
\definecolor{currentstroke}{rgb}{0.121569,0.466667,0.705882}%
\pgfsetstrokecolor{currentstroke}%
\pgfsetstrokeopacity{0.672640}%
\pgfsetdash{}{0pt}%
\pgfpathmoveto{\pgfqpoint{2.030169in}{1.303011in}}%
\pgfpathcurveto{\pgfqpoint{2.038405in}{1.303011in}}{\pgfqpoint{2.046305in}{1.306283in}}{\pgfqpoint{2.052129in}{1.312107in}}%
\pgfpathcurveto{\pgfqpoint{2.057953in}{1.317931in}}{\pgfqpoint{2.061225in}{1.325831in}}{\pgfqpoint{2.061225in}{1.334067in}}%
\pgfpathcurveto{\pgfqpoint{2.061225in}{1.342303in}}{\pgfqpoint{2.057953in}{1.350203in}}{\pgfqpoint{2.052129in}{1.356027in}}%
\pgfpathcurveto{\pgfqpoint{2.046305in}{1.361851in}}{\pgfqpoint{2.038405in}{1.365124in}}{\pgfqpoint{2.030169in}{1.365124in}}%
\pgfpathcurveto{\pgfqpoint{2.021933in}{1.365124in}}{\pgfqpoint{2.014033in}{1.361851in}}{\pgfqpoint{2.008209in}{1.356027in}}%
\pgfpathcurveto{\pgfqpoint{2.002385in}{1.350203in}}{\pgfqpoint{1.999112in}{1.342303in}}{\pgfqpoint{1.999112in}{1.334067in}}%
\pgfpathcurveto{\pgfqpoint{1.999112in}{1.325831in}}{\pgfqpoint{2.002385in}{1.317931in}}{\pgfqpoint{2.008209in}{1.312107in}}%
\pgfpathcurveto{\pgfqpoint{2.014033in}{1.306283in}}{\pgfqpoint{2.021933in}{1.303011in}}{\pgfqpoint{2.030169in}{1.303011in}}%
\pgfpathclose%
\pgfusepath{stroke,fill}%
\end{pgfscope}%
\begin{pgfscope}%
\pgfpathrectangle{\pgfqpoint{0.100000in}{0.212622in}}{\pgfqpoint{3.696000in}{3.696000in}}%
\pgfusepath{clip}%
\pgfsetbuttcap%
\pgfsetroundjoin%
\definecolor{currentfill}{rgb}{0.121569,0.466667,0.705882}%
\pgfsetfillcolor{currentfill}%
\pgfsetfillopacity{0.675193}%
\pgfsetlinewidth{1.003750pt}%
\definecolor{currentstroke}{rgb}{0.121569,0.466667,0.705882}%
\pgfsetstrokecolor{currentstroke}%
\pgfsetstrokeopacity{0.675193}%
\pgfsetdash{}{0pt}%
\pgfpathmoveto{\pgfqpoint{2.037097in}{1.300925in}}%
\pgfpathcurveto{\pgfqpoint{2.045333in}{1.300925in}}{\pgfqpoint{2.053234in}{1.304197in}}{\pgfqpoint{2.059057in}{1.310021in}}%
\pgfpathcurveto{\pgfqpoint{2.064881in}{1.315845in}}{\pgfqpoint{2.068154in}{1.323745in}}{\pgfqpoint{2.068154in}{1.331981in}}%
\pgfpathcurveto{\pgfqpoint{2.068154in}{1.340217in}}{\pgfqpoint{2.064881in}{1.348117in}}{\pgfqpoint{2.059057in}{1.353941in}}%
\pgfpathcurveto{\pgfqpoint{2.053234in}{1.359765in}}{\pgfqpoint{2.045333in}{1.363038in}}{\pgfqpoint{2.037097in}{1.363038in}}%
\pgfpathcurveto{\pgfqpoint{2.028861in}{1.363038in}}{\pgfqpoint{2.020961in}{1.359765in}}{\pgfqpoint{2.015137in}{1.353941in}}%
\pgfpathcurveto{\pgfqpoint{2.009313in}{1.348117in}}{\pgfqpoint{2.006041in}{1.340217in}}{\pgfqpoint{2.006041in}{1.331981in}}%
\pgfpathcurveto{\pgfqpoint{2.006041in}{1.323745in}}{\pgfqpoint{2.009313in}{1.315845in}}{\pgfqpoint{2.015137in}{1.310021in}}%
\pgfpathcurveto{\pgfqpoint{2.020961in}{1.304197in}}{\pgfqpoint{2.028861in}{1.300925in}}{\pgfqpoint{2.037097in}{1.300925in}}%
\pgfpathclose%
\pgfusepath{stroke,fill}%
\end{pgfscope}%
\begin{pgfscope}%
\pgfpathrectangle{\pgfqpoint{0.100000in}{0.212622in}}{\pgfqpoint{3.696000in}{3.696000in}}%
\pgfusepath{clip}%
\pgfsetbuttcap%
\pgfsetroundjoin%
\definecolor{currentfill}{rgb}{0.121569,0.466667,0.705882}%
\pgfsetfillcolor{currentfill}%
\pgfsetfillopacity{0.678444}%
\pgfsetlinewidth{1.003750pt}%
\definecolor{currentstroke}{rgb}{0.121569,0.466667,0.705882}%
\pgfsetstrokecolor{currentstroke}%
\pgfsetstrokeopacity{0.678444}%
\pgfsetdash{}{0pt}%
\pgfpathmoveto{\pgfqpoint{2.044351in}{1.298502in}}%
\pgfpathcurveto{\pgfqpoint{2.052587in}{1.298502in}}{\pgfqpoint{2.060487in}{1.301775in}}{\pgfqpoint{2.066311in}{1.307598in}}%
\pgfpathcurveto{\pgfqpoint{2.072135in}{1.313422in}}{\pgfqpoint{2.075407in}{1.321322in}}{\pgfqpoint{2.075407in}{1.329559in}}%
\pgfpathcurveto{\pgfqpoint{2.075407in}{1.337795in}}{\pgfqpoint{2.072135in}{1.345695in}}{\pgfqpoint{2.066311in}{1.351519in}}%
\pgfpathcurveto{\pgfqpoint{2.060487in}{1.357343in}}{\pgfqpoint{2.052587in}{1.360615in}}{\pgfqpoint{2.044351in}{1.360615in}}%
\pgfpathcurveto{\pgfqpoint{2.036114in}{1.360615in}}{\pgfqpoint{2.028214in}{1.357343in}}{\pgfqpoint{2.022390in}{1.351519in}}%
\pgfpathcurveto{\pgfqpoint{2.016566in}{1.345695in}}{\pgfqpoint{2.013294in}{1.337795in}}{\pgfqpoint{2.013294in}{1.329559in}}%
\pgfpathcurveto{\pgfqpoint{2.013294in}{1.321322in}}{\pgfqpoint{2.016566in}{1.313422in}}{\pgfqpoint{2.022390in}{1.307598in}}%
\pgfpathcurveto{\pgfqpoint{2.028214in}{1.301775in}}{\pgfqpoint{2.036114in}{1.298502in}}{\pgfqpoint{2.044351in}{1.298502in}}%
\pgfpathclose%
\pgfusepath{stroke,fill}%
\end{pgfscope}%
\begin{pgfscope}%
\pgfpathrectangle{\pgfqpoint{0.100000in}{0.212622in}}{\pgfqpoint{3.696000in}{3.696000in}}%
\pgfusepath{clip}%
\pgfsetbuttcap%
\pgfsetroundjoin%
\definecolor{currentfill}{rgb}{0.121569,0.466667,0.705882}%
\pgfsetfillcolor{currentfill}%
\pgfsetfillopacity{0.680250}%
\pgfsetlinewidth{1.003750pt}%
\definecolor{currentstroke}{rgb}{0.121569,0.466667,0.705882}%
\pgfsetstrokecolor{currentstroke}%
\pgfsetstrokeopacity{0.680250}%
\pgfsetdash{}{0pt}%
\pgfpathmoveto{\pgfqpoint{2.048276in}{1.296982in}}%
\pgfpathcurveto{\pgfqpoint{2.056513in}{1.296982in}}{\pgfqpoint{2.064413in}{1.300255in}}{\pgfqpoint{2.070237in}{1.306079in}}%
\pgfpathcurveto{\pgfqpoint{2.076061in}{1.311903in}}{\pgfqpoint{2.079333in}{1.319803in}}{\pgfqpoint{2.079333in}{1.328039in}}%
\pgfpathcurveto{\pgfqpoint{2.079333in}{1.336275in}}{\pgfqpoint{2.076061in}{1.344175in}}{\pgfqpoint{2.070237in}{1.349999in}}%
\pgfpathcurveto{\pgfqpoint{2.064413in}{1.355823in}}{\pgfqpoint{2.056513in}{1.359095in}}{\pgfqpoint{2.048276in}{1.359095in}}%
\pgfpathcurveto{\pgfqpoint{2.040040in}{1.359095in}}{\pgfqpoint{2.032140in}{1.355823in}}{\pgfqpoint{2.026316in}{1.349999in}}%
\pgfpathcurveto{\pgfqpoint{2.020492in}{1.344175in}}{\pgfqpoint{2.017220in}{1.336275in}}{\pgfqpoint{2.017220in}{1.328039in}}%
\pgfpathcurveto{\pgfqpoint{2.017220in}{1.319803in}}{\pgfqpoint{2.020492in}{1.311903in}}{\pgfqpoint{2.026316in}{1.306079in}}%
\pgfpathcurveto{\pgfqpoint{2.032140in}{1.300255in}}{\pgfqpoint{2.040040in}{1.296982in}}{\pgfqpoint{2.048276in}{1.296982in}}%
\pgfpathclose%
\pgfusepath{stroke,fill}%
\end{pgfscope}%
\begin{pgfscope}%
\pgfpathrectangle{\pgfqpoint{0.100000in}{0.212622in}}{\pgfqpoint{3.696000in}{3.696000in}}%
\pgfusepath{clip}%
\pgfsetbuttcap%
\pgfsetroundjoin%
\definecolor{currentfill}{rgb}{0.121569,0.466667,0.705882}%
\pgfsetfillcolor{currentfill}%
\pgfsetfillopacity{0.682288}%
\pgfsetlinewidth{1.003750pt}%
\definecolor{currentstroke}{rgb}{0.121569,0.466667,0.705882}%
\pgfsetstrokecolor{currentstroke}%
\pgfsetstrokeopacity{0.682288}%
\pgfsetdash{}{0pt}%
\pgfpathmoveto{\pgfqpoint{2.053228in}{1.295238in}}%
\pgfpathcurveto{\pgfqpoint{2.061465in}{1.295238in}}{\pgfqpoint{2.069365in}{1.298510in}}{\pgfqpoint{2.075189in}{1.304334in}}%
\pgfpathcurveto{\pgfqpoint{2.081013in}{1.310158in}}{\pgfqpoint{2.084285in}{1.318058in}}{\pgfqpoint{2.084285in}{1.326294in}}%
\pgfpathcurveto{\pgfqpoint{2.084285in}{1.334531in}}{\pgfqpoint{2.081013in}{1.342431in}}{\pgfqpoint{2.075189in}{1.348255in}}%
\pgfpathcurveto{\pgfqpoint{2.069365in}{1.354078in}}{\pgfqpoint{2.061465in}{1.357351in}}{\pgfqpoint{2.053228in}{1.357351in}}%
\pgfpathcurveto{\pgfqpoint{2.044992in}{1.357351in}}{\pgfqpoint{2.037092in}{1.354078in}}{\pgfqpoint{2.031268in}{1.348255in}}%
\pgfpathcurveto{\pgfqpoint{2.025444in}{1.342431in}}{\pgfqpoint{2.022172in}{1.334531in}}{\pgfqpoint{2.022172in}{1.326294in}}%
\pgfpathcurveto{\pgfqpoint{2.022172in}{1.318058in}}{\pgfqpoint{2.025444in}{1.310158in}}{\pgfqpoint{2.031268in}{1.304334in}}%
\pgfpathcurveto{\pgfqpoint{2.037092in}{1.298510in}}{\pgfqpoint{2.044992in}{1.295238in}}{\pgfqpoint{2.053228in}{1.295238in}}%
\pgfpathclose%
\pgfusepath{stroke,fill}%
\end{pgfscope}%
\begin{pgfscope}%
\pgfpathrectangle{\pgfqpoint{0.100000in}{0.212622in}}{\pgfqpoint{3.696000in}{3.696000in}}%
\pgfusepath{clip}%
\pgfsetbuttcap%
\pgfsetroundjoin%
\definecolor{currentfill}{rgb}{0.121569,0.466667,0.705882}%
\pgfsetfillcolor{currentfill}%
\pgfsetfillopacity{0.684727}%
\pgfsetlinewidth{1.003750pt}%
\definecolor{currentstroke}{rgb}{0.121569,0.466667,0.705882}%
\pgfsetstrokecolor{currentstroke}%
\pgfsetstrokeopacity{0.684727}%
\pgfsetdash{}{0pt}%
\pgfpathmoveto{\pgfqpoint{2.058771in}{1.293262in}}%
\pgfpathcurveto{\pgfqpoint{2.067007in}{1.293262in}}{\pgfqpoint{2.074907in}{1.296534in}}{\pgfqpoint{2.080731in}{1.302358in}}%
\pgfpathcurveto{\pgfqpoint{2.086555in}{1.308182in}}{\pgfqpoint{2.089827in}{1.316082in}}{\pgfqpoint{2.089827in}{1.324319in}}%
\pgfpathcurveto{\pgfqpoint{2.089827in}{1.332555in}}{\pgfqpoint{2.086555in}{1.340455in}}{\pgfqpoint{2.080731in}{1.346279in}}%
\pgfpathcurveto{\pgfqpoint{2.074907in}{1.352103in}}{\pgfqpoint{2.067007in}{1.355375in}}{\pgfqpoint{2.058771in}{1.355375in}}%
\pgfpathcurveto{\pgfqpoint{2.050535in}{1.355375in}}{\pgfqpoint{2.042635in}{1.352103in}}{\pgfqpoint{2.036811in}{1.346279in}}%
\pgfpathcurveto{\pgfqpoint{2.030987in}{1.340455in}}{\pgfqpoint{2.027714in}{1.332555in}}{\pgfqpoint{2.027714in}{1.324319in}}%
\pgfpathcurveto{\pgfqpoint{2.027714in}{1.316082in}}{\pgfqpoint{2.030987in}{1.308182in}}{\pgfqpoint{2.036811in}{1.302358in}}%
\pgfpathcurveto{\pgfqpoint{2.042635in}{1.296534in}}{\pgfqpoint{2.050535in}{1.293262in}}{\pgfqpoint{2.058771in}{1.293262in}}%
\pgfpathclose%
\pgfusepath{stroke,fill}%
\end{pgfscope}%
\begin{pgfscope}%
\pgfpathrectangle{\pgfqpoint{0.100000in}{0.212622in}}{\pgfqpoint{3.696000in}{3.696000in}}%
\pgfusepath{clip}%
\pgfsetbuttcap%
\pgfsetroundjoin%
\definecolor{currentfill}{rgb}{0.121569,0.466667,0.705882}%
\pgfsetfillcolor{currentfill}%
\pgfsetfillopacity{0.687245}%
\pgfsetlinewidth{1.003750pt}%
\definecolor{currentstroke}{rgb}{0.121569,0.466667,0.705882}%
\pgfsetstrokecolor{currentstroke}%
\pgfsetstrokeopacity{0.687245}%
\pgfsetdash{}{0pt}%
\pgfpathmoveto{\pgfqpoint{2.065539in}{1.291220in}}%
\pgfpathcurveto{\pgfqpoint{2.073775in}{1.291220in}}{\pgfqpoint{2.081675in}{1.294493in}}{\pgfqpoint{2.087499in}{1.300317in}}%
\pgfpathcurveto{\pgfqpoint{2.093323in}{1.306140in}}{\pgfqpoint{2.096596in}{1.314040in}}{\pgfqpoint{2.096596in}{1.322277in}}%
\pgfpathcurveto{\pgfqpoint{2.096596in}{1.330513in}}{\pgfqpoint{2.093323in}{1.338413in}}{\pgfqpoint{2.087499in}{1.344237in}}%
\pgfpathcurveto{\pgfqpoint{2.081675in}{1.350061in}}{\pgfqpoint{2.073775in}{1.353333in}}{\pgfqpoint{2.065539in}{1.353333in}}%
\pgfpathcurveto{\pgfqpoint{2.057303in}{1.353333in}}{\pgfqpoint{2.049403in}{1.350061in}}{\pgfqpoint{2.043579in}{1.344237in}}%
\pgfpathcurveto{\pgfqpoint{2.037755in}{1.338413in}}{\pgfqpoint{2.034483in}{1.330513in}}{\pgfqpoint{2.034483in}{1.322277in}}%
\pgfpathcurveto{\pgfqpoint{2.034483in}{1.314040in}}{\pgfqpoint{2.037755in}{1.306140in}}{\pgfqpoint{2.043579in}{1.300317in}}%
\pgfpathcurveto{\pgfqpoint{2.049403in}{1.294493in}}{\pgfqpoint{2.057303in}{1.291220in}}{\pgfqpoint{2.065539in}{1.291220in}}%
\pgfpathclose%
\pgfusepath{stroke,fill}%
\end{pgfscope}%
\begin{pgfscope}%
\pgfpathrectangle{\pgfqpoint{0.100000in}{0.212622in}}{\pgfqpoint{3.696000in}{3.696000in}}%
\pgfusepath{clip}%
\pgfsetbuttcap%
\pgfsetroundjoin%
\definecolor{currentfill}{rgb}{0.121569,0.466667,0.705882}%
\pgfsetfillcolor{currentfill}%
\pgfsetfillopacity{0.690134}%
\pgfsetlinewidth{1.003750pt}%
\definecolor{currentstroke}{rgb}{0.121569,0.466667,0.705882}%
\pgfsetstrokecolor{currentstroke}%
\pgfsetstrokeopacity{0.690134}%
\pgfsetdash{}{0pt}%
\pgfpathmoveto{\pgfqpoint{2.073269in}{1.289034in}}%
\pgfpathcurveto{\pgfqpoint{2.081505in}{1.289034in}}{\pgfqpoint{2.089405in}{1.292306in}}{\pgfqpoint{2.095229in}{1.298130in}}%
\pgfpathcurveto{\pgfqpoint{2.101053in}{1.303954in}}{\pgfqpoint{2.104325in}{1.311854in}}{\pgfqpoint{2.104325in}{1.320091in}}%
\pgfpathcurveto{\pgfqpoint{2.104325in}{1.328327in}}{\pgfqpoint{2.101053in}{1.336227in}}{\pgfqpoint{2.095229in}{1.342051in}}%
\pgfpathcurveto{\pgfqpoint{2.089405in}{1.347875in}}{\pgfqpoint{2.081505in}{1.351147in}}{\pgfqpoint{2.073269in}{1.351147in}}%
\pgfpathcurveto{\pgfqpoint{2.065032in}{1.351147in}}{\pgfqpoint{2.057132in}{1.347875in}}{\pgfqpoint{2.051308in}{1.342051in}}%
\pgfpathcurveto{\pgfqpoint{2.045484in}{1.336227in}}{\pgfqpoint{2.042212in}{1.328327in}}{\pgfqpoint{2.042212in}{1.320091in}}%
\pgfpathcurveto{\pgfqpoint{2.042212in}{1.311854in}}{\pgfqpoint{2.045484in}{1.303954in}}{\pgfqpoint{2.051308in}{1.298130in}}%
\pgfpathcurveto{\pgfqpoint{2.057132in}{1.292306in}}{\pgfqpoint{2.065032in}{1.289034in}}{\pgfqpoint{2.073269in}{1.289034in}}%
\pgfpathclose%
\pgfusepath{stroke,fill}%
\end{pgfscope}%
\begin{pgfscope}%
\pgfpathrectangle{\pgfqpoint{0.100000in}{0.212622in}}{\pgfqpoint{3.696000in}{3.696000in}}%
\pgfusepath{clip}%
\pgfsetbuttcap%
\pgfsetroundjoin%
\definecolor{currentfill}{rgb}{0.121569,0.466667,0.705882}%
\pgfsetfillcolor{currentfill}%
\pgfsetfillopacity{0.693620}%
\pgfsetlinewidth{1.003750pt}%
\definecolor{currentstroke}{rgb}{0.121569,0.466667,0.705882}%
\pgfsetstrokecolor{currentstroke}%
\pgfsetstrokeopacity{0.693620}%
\pgfsetdash{}{0pt}%
\pgfpathmoveto{\pgfqpoint{2.081111in}{1.286334in}}%
\pgfpathcurveto{\pgfqpoint{2.089347in}{1.286334in}}{\pgfqpoint{2.097247in}{1.289607in}}{\pgfqpoint{2.103071in}{1.295431in}}%
\pgfpathcurveto{\pgfqpoint{2.108895in}{1.301255in}}{\pgfqpoint{2.112168in}{1.309155in}}{\pgfqpoint{2.112168in}{1.317391in}}%
\pgfpathcurveto{\pgfqpoint{2.112168in}{1.325627in}}{\pgfqpoint{2.108895in}{1.333527in}}{\pgfqpoint{2.103071in}{1.339351in}}%
\pgfpathcurveto{\pgfqpoint{2.097247in}{1.345175in}}{\pgfqpoint{2.089347in}{1.348447in}}{\pgfqpoint{2.081111in}{1.348447in}}%
\pgfpathcurveto{\pgfqpoint{2.072875in}{1.348447in}}{\pgfqpoint{2.064975in}{1.345175in}}{\pgfqpoint{2.059151in}{1.339351in}}%
\pgfpathcurveto{\pgfqpoint{2.053327in}{1.333527in}}{\pgfqpoint{2.050055in}{1.325627in}}{\pgfqpoint{2.050055in}{1.317391in}}%
\pgfpathcurveto{\pgfqpoint{2.050055in}{1.309155in}}{\pgfqpoint{2.053327in}{1.301255in}}{\pgfqpoint{2.059151in}{1.295431in}}%
\pgfpathcurveto{\pgfqpoint{2.064975in}{1.289607in}}{\pgfqpoint{2.072875in}{1.286334in}}{\pgfqpoint{2.081111in}{1.286334in}}%
\pgfpathclose%
\pgfusepath{stroke,fill}%
\end{pgfscope}%
\begin{pgfscope}%
\pgfpathrectangle{\pgfqpoint{0.100000in}{0.212622in}}{\pgfqpoint{3.696000in}{3.696000in}}%
\pgfusepath{clip}%
\pgfsetbuttcap%
\pgfsetroundjoin%
\definecolor{currentfill}{rgb}{0.121569,0.466667,0.705882}%
\pgfsetfillcolor{currentfill}%
\pgfsetfillopacity{0.695696}%
\pgfsetlinewidth{1.003750pt}%
\definecolor{currentstroke}{rgb}{0.121569,0.466667,0.705882}%
\pgfsetstrokecolor{currentstroke}%
\pgfsetstrokeopacity{0.695696}%
\pgfsetdash{}{0pt}%
\pgfpathmoveto{\pgfqpoint{2.085253in}{1.284589in}}%
\pgfpathcurveto{\pgfqpoint{2.093489in}{1.284589in}}{\pgfqpoint{2.101389in}{1.287862in}}{\pgfqpoint{2.107213in}{1.293686in}}%
\pgfpathcurveto{\pgfqpoint{2.113037in}{1.299510in}}{\pgfqpoint{2.116309in}{1.307410in}}{\pgfqpoint{2.116309in}{1.315646in}}%
\pgfpathcurveto{\pgfqpoint{2.116309in}{1.323882in}}{\pgfqpoint{2.113037in}{1.331782in}}{\pgfqpoint{2.107213in}{1.337606in}}%
\pgfpathcurveto{\pgfqpoint{2.101389in}{1.343430in}}{\pgfqpoint{2.093489in}{1.346702in}}{\pgfqpoint{2.085253in}{1.346702in}}%
\pgfpathcurveto{\pgfqpoint{2.077017in}{1.346702in}}{\pgfqpoint{2.069116in}{1.343430in}}{\pgfqpoint{2.063293in}{1.337606in}}%
\pgfpathcurveto{\pgfqpoint{2.057469in}{1.331782in}}{\pgfqpoint{2.054196in}{1.323882in}}{\pgfqpoint{2.054196in}{1.315646in}}%
\pgfpathcurveto{\pgfqpoint{2.054196in}{1.307410in}}{\pgfqpoint{2.057469in}{1.299510in}}{\pgfqpoint{2.063293in}{1.293686in}}%
\pgfpathcurveto{\pgfqpoint{2.069116in}{1.287862in}}{\pgfqpoint{2.077017in}{1.284589in}}{\pgfqpoint{2.085253in}{1.284589in}}%
\pgfpathclose%
\pgfusepath{stroke,fill}%
\end{pgfscope}%
\begin{pgfscope}%
\pgfpathrectangle{\pgfqpoint{0.100000in}{0.212622in}}{\pgfqpoint{3.696000in}{3.696000in}}%
\pgfusepath{clip}%
\pgfsetbuttcap%
\pgfsetroundjoin%
\definecolor{currentfill}{rgb}{0.121569,0.466667,0.705882}%
\pgfsetfillcolor{currentfill}%
\pgfsetfillopacity{0.697799}%
\pgfsetlinewidth{1.003750pt}%
\definecolor{currentstroke}{rgb}{0.121569,0.466667,0.705882}%
\pgfsetstrokecolor{currentstroke}%
\pgfsetstrokeopacity{0.697799}%
\pgfsetdash{}{0pt}%
\pgfpathmoveto{\pgfqpoint{2.090604in}{1.282893in}}%
\pgfpathcurveto{\pgfqpoint{2.098841in}{1.282893in}}{\pgfqpoint{2.106741in}{1.286165in}}{\pgfqpoint{2.112564in}{1.291989in}}%
\pgfpathcurveto{\pgfqpoint{2.118388in}{1.297813in}}{\pgfqpoint{2.121661in}{1.305713in}}{\pgfqpoint{2.121661in}{1.313949in}}%
\pgfpathcurveto{\pgfqpoint{2.121661in}{1.322186in}}{\pgfqpoint{2.118388in}{1.330086in}}{\pgfqpoint{2.112564in}{1.335910in}}%
\pgfpathcurveto{\pgfqpoint{2.106741in}{1.341734in}}{\pgfqpoint{2.098841in}{1.345006in}}{\pgfqpoint{2.090604in}{1.345006in}}%
\pgfpathcurveto{\pgfqpoint{2.082368in}{1.345006in}}{\pgfqpoint{2.074468in}{1.341734in}}{\pgfqpoint{2.068644in}{1.335910in}}%
\pgfpathcurveto{\pgfqpoint{2.062820in}{1.330086in}}{\pgfqpoint{2.059548in}{1.322186in}}{\pgfqpoint{2.059548in}{1.313949in}}%
\pgfpathcurveto{\pgfqpoint{2.059548in}{1.305713in}}{\pgfqpoint{2.062820in}{1.297813in}}{\pgfqpoint{2.068644in}{1.291989in}}%
\pgfpathcurveto{\pgfqpoint{2.074468in}{1.286165in}}{\pgfqpoint{2.082368in}{1.282893in}}{\pgfqpoint{2.090604in}{1.282893in}}%
\pgfpathclose%
\pgfusepath{stroke,fill}%
\end{pgfscope}%
\begin{pgfscope}%
\pgfpathrectangle{\pgfqpoint{0.100000in}{0.212622in}}{\pgfqpoint{3.696000in}{3.696000in}}%
\pgfusepath{clip}%
\pgfsetbuttcap%
\pgfsetroundjoin%
\definecolor{currentfill}{rgb}{0.121569,0.466667,0.705882}%
\pgfsetfillcolor{currentfill}%
\pgfsetfillopacity{0.699904}%
\pgfsetlinewidth{1.003750pt}%
\definecolor{currentstroke}{rgb}{0.121569,0.466667,0.705882}%
\pgfsetstrokecolor{currentstroke}%
\pgfsetstrokeopacity{0.699904}%
\pgfsetdash{}{0pt}%
\pgfpathmoveto{\pgfqpoint{2.097110in}{1.281299in}}%
\pgfpathcurveto{\pgfqpoint{2.105346in}{1.281299in}}{\pgfqpoint{2.113246in}{1.284572in}}{\pgfqpoint{2.119070in}{1.290395in}}%
\pgfpathcurveto{\pgfqpoint{2.124894in}{1.296219in}}{\pgfqpoint{2.128166in}{1.304119in}}{\pgfqpoint{2.128166in}{1.312356in}}%
\pgfpathcurveto{\pgfqpoint{2.128166in}{1.320592in}}{\pgfqpoint{2.124894in}{1.328492in}}{\pgfqpoint{2.119070in}{1.334316in}}%
\pgfpathcurveto{\pgfqpoint{2.113246in}{1.340140in}}{\pgfqpoint{2.105346in}{1.343412in}}{\pgfqpoint{2.097110in}{1.343412in}}%
\pgfpathcurveto{\pgfqpoint{2.088873in}{1.343412in}}{\pgfqpoint{2.080973in}{1.340140in}}{\pgfqpoint{2.075149in}{1.334316in}}%
\pgfpathcurveto{\pgfqpoint{2.069325in}{1.328492in}}{\pgfqpoint{2.066053in}{1.320592in}}{\pgfqpoint{2.066053in}{1.312356in}}%
\pgfpathcurveto{\pgfqpoint{2.066053in}{1.304119in}}{\pgfqpoint{2.069325in}{1.296219in}}{\pgfqpoint{2.075149in}{1.290395in}}%
\pgfpathcurveto{\pgfqpoint{2.080973in}{1.284572in}}{\pgfqpoint{2.088873in}{1.281299in}}{\pgfqpoint{2.097110in}{1.281299in}}%
\pgfpathclose%
\pgfusepath{stroke,fill}%
\end{pgfscope}%
\begin{pgfscope}%
\pgfpathrectangle{\pgfqpoint{0.100000in}{0.212622in}}{\pgfqpoint{3.696000in}{3.696000in}}%
\pgfusepath{clip}%
\pgfsetbuttcap%
\pgfsetroundjoin%
\definecolor{currentfill}{rgb}{0.121569,0.466667,0.705882}%
\pgfsetfillcolor{currentfill}%
\pgfsetfillopacity{0.703488}%
\pgfsetlinewidth{1.003750pt}%
\definecolor{currentstroke}{rgb}{0.121569,0.466667,0.705882}%
\pgfsetstrokecolor{currentstroke}%
\pgfsetstrokeopacity{0.703488}%
\pgfsetdash{}{0pt}%
\pgfpathmoveto{\pgfqpoint{2.103804in}{1.278162in}}%
\pgfpathcurveto{\pgfqpoint{2.112041in}{1.278162in}}{\pgfqpoint{2.119941in}{1.281435in}}{\pgfqpoint{2.125765in}{1.287258in}}%
\pgfpathcurveto{\pgfqpoint{2.131588in}{1.293082in}}{\pgfqpoint{2.134861in}{1.300982in}}{\pgfqpoint{2.134861in}{1.309219in}}%
\pgfpathcurveto{\pgfqpoint{2.134861in}{1.317455in}}{\pgfqpoint{2.131588in}{1.325355in}}{\pgfqpoint{2.125765in}{1.331179in}}%
\pgfpathcurveto{\pgfqpoint{2.119941in}{1.337003in}}{\pgfqpoint{2.112041in}{1.340275in}}{\pgfqpoint{2.103804in}{1.340275in}}%
\pgfpathcurveto{\pgfqpoint{2.095568in}{1.340275in}}{\pgfqpoint{2.087668in}{1.337003in}}{\pgfqpoint{2.081844in}{1.331179in}}%
\pgfpathcurveto{\pgfqpoint{2.076020in}{1.325355in}}{\pgfqpoint{2.072748in}{1.317455in}}{\pgfqpoint{2.072748in}{1.309219in}}%
\pgfpathcurveto{\pgfqpoint{2.072748in}{1.300982in}}{\pgfqpoint{2.076020in}{1.293082in}}{\pgfqpoint{2.081844in}{1.287258in}}%
\pgfpathcurveto{\pgfqpoint{2.087668in}{1.281435in}}{\pgfqpoint{2.095568in}{1.278162in}}{\pgfqpoint{2.103804in}{1.278162in}}%
\pgfpathclose%
\pgfusepath{stroke,fill}%
\end{pgfscope}%
\begin{pgfscope}%
\pgfpathrectangle{\pgfqpoint{0.100000in}{0.212622in}}{\pgfqpoint{3.696000in}{3.696000in}}%
\pgfusepath{clip}%
\pgfsetbuttcap%
\pgfsetroundjoin%
\definecolor{currentfill}{rgb}{0.121569,0.466667,0.705882}%
\pgfsetfillcolor{currentfill}%
\pgfsetfillopacity{0.706953}%
\pgfsetlinewidth{1.003750pt}%
\definecolor{currentstroke}{rgb}{0.121569,0.466667,0.705882}%
\pgfsetstrokecolor{currentstroke}%
\pgfsetstrokeopacity{0.706953}%
\pgfsetdash{}{0pt}%
\pgfpathmoveto{\pgfqpoint{2.112114in}{1.275181in}}%
\pgfpathcurveto{\pgfqpoint{2.120350in}{1.275181in}}{\pgfqpoint{2.128251in}{1.278454in}}{\pgfqpoint{2.134074in}{1.284278in}}%
\pgfpathcurveto{\pgfqpoint{2.139898in}{1.290102in}}{\pgfqpoint{2.143171in}{1.298002in}}{\pgfqpoint{2.143171in}{1.306238in}}%
\pgfpathcurveto{\pgfqpoint{2.143171in}{1.314474in}}{\pgfqpoint{2.139898in}{1.322374in}}{\pgfqpoint{2.134074in}{1.328198in}}%
\pgfpathcurveto{\pgfqpoint{2.128251in}{1.334022in}}{\pgfqpoint{2.120350in}{1.337294in}}{\pgfqpoint{2.112114in}{1.337294in}}%
\pgfpathcurveto{\pgfqpoint{2.103878in}{1.337294in}}{\pgfqpoint{2.095978in}{1.334022in}}{\pgfqpoint{2.090154in}{1.328198in}}%
\pgfpathcurveto{\pgfqpoint{2.084330in}{1.322374in}}{\pgfqpoint{2.081058in}{1.314474in}}{\pgfqpoint{2.081058in}{1.306238in}}%
\pgfpathcurveto{\pgfqpoint{2.081058in}{1.298002in}}{\pgfqpoint{2.084330in}{1.290102in}}{\pgfqpoint{2.090154in}{1.284278in}}%
\pgfpathcurveto{\pgfqpoint{2.095978in}{1.278454in}}{\pgfqpoint{2.103878in}{1.275181in}}{\pgfqpoint{2.112114in}{1.275181in}}%
\pgfpathclose%
\pgfusepath{stroke,fill}%
\end{pgfscope}%
\begin{pgfscope}%
\pgfpathrectangle{\pgfqpoint{0.100000in}{0.212622in}}{\pgfqpoint{3.696000in}{3.696000in}}%
\pgfusepath{clip}%
\pgfsetbuttcap%
\pgfsetroundjoin%
\definecolor{currentfill}{rgb}{0.121569,0.466667,0.705882}%
\pgfsetfillcolor{currentfill}%
\pgfsetfillopacity{0.711012}%
\pgfsetlinewidth{1.003750pt}%
\definecolor{currentstroke}{rgb}{0.121569,0.466667,0.705882}%
\pgfsetstrokecolor{currentstroke}%
\pgfsetstrokeopacity{0.711012}%
\pgfsetdash{}{0pt}%
\pgfpathmoveto{\pgfqpoint{2.120827in}{1.271893in}}%
\pgfpathcurveto{\pgfqpoint{2.129063in}{1.271893in}}{\pgfqpoint{2.136963in}{1.275165in}}{\pgfqpoint{2.142787in}{1.280989in}}%
\pgfpathcurveto{\pgfqpoint{2.148611in}{1.286813in}}{\pgfqpoint{2.151883in}{1.294713in}}{\pgfqpoint{2.151883in}{1.302949in}}%
\pgfpathcurveto{\pgfqpoint{2.151883in}{1.311185in}}{\pgfqpoint{2.148611in}{1.319085in}}{\pgfqpoint{2.142787in}{1.324909in}}%
\pgfpathcurveto{\pgfqpoint{2.136963in}{1.330733in}}{\pgfqpoint{2.129063in}{1.334006in}}{\pgfqpoint{2.120827in}{1.334006in}}%
\pgfpathcurveto{\pgfqpoint{2.112591in}{1.334006in}}{\pgfqpoint{2.104691in}{1.330733in}}{\pgfqpoint{2.098867in}{1.324909in}}%
\pgfpathcurveto{\pgfqpoint{2.093043in}{1.319085in}}{\pgfqpoint{2.089770in}{1.311185in}}{\pgfqpoint{2.089770in}{1.302949in}}%
\pgfpathcurveto{\pgfqpoint{2.089770in}{1.294713in}}{\pgfqpoint{2.093043in}{1.286813in}}{\pgfqpoint{2.098867in}{1.280989in}}%
\pgfpathcurveto{\pgfqpoint{2.104691in}{1.275165in}}{\pgfqpoint{2.112591in}{1.271893in}}{\pgfqpoint{2.120827in}{1.271893in}}%
\pgfpathclose%
\pgfusepath{stroke,fill}%
\end{pgfscope}%
\begin{pgfscope}%
\pgfpathrectangle{\pgfqpoint{0.100000in}{0.212622in}}{\pgfqpoint{3.696000in}{3.696000in}}%
\pgfusepath{clip}%
\pgfsetbuttcap%
\pgfsetroundjoin%
\definecolor{currentfill}{rgb}{0.121569,0.466667,0.705882}%
\pgfsetfillcolor{currentfill}%
\pgfsetfillopacity{0.713144}%
\pgfsetlinewidth{1.003750pt}%
\definecolor{currentstroke}{rgb}{0.121569,0.466667,0.705882}%
\pgfsetstrokecolor{currentstroke}%
\pgfsetstrokeopacity{0.713144}%
\pgfsetdash{}{0pt}%
\pgfpathmoveto{\pgfqpoint{2.125679in}{1.270069in}}%
\pgfpathcurveto{\pgfqpoint{2.133915in}{1.270069in}}{\pgfqpoint{2.141815in}{1.273341in}}{\pgfqpoint{2.147639in}{1.279165in}}%
\pgfpathcurveto{\pgfqpoint{2.153463in}{1.284989in}}{\pgfqpoint{2.156735in}{1.292889in}}{\pgfqpoint{2.156735in}{1.301125in}}%
\pgfpathcurveto{\pgfqpoint{2.156735in}{1.309361in}}{\pgfqpoint{2.153463in}{1.317261in}}{\pgfqpoint{2.147639in}{1.323085in}}%
\pgfpathcurveto{\pgfqpoint{2.141815in}{1.328909in}}{\pgfqpoint{2.133915in}{1.332182in}}{\pgfqpoint{2.125679in}{1.332182in}}%
\pgfpathcurveto{\pgfqpoint{2.117443in}{1.332182in}}{\pgfqpoint{2.109543in}{1.328909in}}{\pgfqpoint{2.103719in}{1.323085in}}%
\pgfpathcurveto{\pgfqpoint{2.097895in}{1.317261in}}{\pgfqpoint{2.094622in}{1.309361in}}{\pgfqpoint{2.094622in}{1.301125in}}%
\pgfpathcurveto{\pgfqpoint{2.094622in}{1.292889in}}{\pgfqpoint{2.097895in}{1.284989in}}{\pgfqpoint{2.103719in}{1.279165in}}%
\pgfpathcurveto{\pgfqpoint{2.109543in}{1.273341in}}{\pgfqpoint{2.117443in}{1.270069in}}{\pgfqpoint{2.125679in}{1.270069in}}%
\pgfpathclose%
\pgfusepath{stroke,fill}%
\end{pgfscope}%
\begin{pgfscope}%
\pgfpathrectangle{\pgfqpoint{0.100000in}{0.212622in}}{\pgfqpoint{3.696000in}{3.696000in}}%
\pgfusepath{clip}%
\pgfsetbuttcap%
\pgfsetroundjoin%
\definecolor{currentfill}{rgb}{0.121569,0.466667,0.705882}%
\pgfsetfillcolor{currentfill}%
\pgfsetfillopacity{0.715445}%
\pgfsetlinewidth{1.003750pt}%
\definecolor{currentstroke}{rgb}{0.121569,0.466667,0.705882}%
\pgfsetstrokecolor{currentstroke}%
\pgfsetstrokeopacity{0.715445}%
\pgfsetdash{}{0pt}%
\pgfpathmoveto{\pgfqpoint{2.131674in}{1.268157in}}%
\pgfpathcurveto{\pgfqpoint{2.139910in}{1.268157in}}{\pgfqpoint{2.147810in}{1.271429in}}{\pgfqpoint{2.153634in}{1.277253in}}%
\pgfpathcurveto{\pgfqpoint{2.159458in}{1.283077in}}{\pgfqpoint{2.162731in}{1.290977in}}{\pgfqpoint{2.162731in}{1.299214in}}%
\pgfpathcurveto{\pgfqpoint{2.162731in}{1.307450in}}{\pgfqpoint{2.159458in}{1.315350in}}{\pgfqpoint{2.153634in}{1.321174in}}%
\pgfpathcurveto{\pgfqpoint{2.147810in}{1.326998in}}{\pgfqpoint{2.139910in}{1.330270in}}{\pgfqpoint{2.131674in}{1.330270in}}%
\pgfpathcurveto{\pgfqpoint{2.123438in}{1.330270in}}{\pgfqpoint{2.115538in}{1.326998in}}{\pgfqpoint{2.109714in}{1.321174in}}%
\pgfpathcurveto{\pgfqpoint{2.103890in}{1.315350in}}{\pgfqpoint{2.100618in}{1.307450in}}{\pgfqpoint{2.100618in}{1.299214in}}%
\pgfpathcurveto{\pgfqpoint{2.100618in}{1.290977in}}{\pgfqpoint{2.103890in}{1.283077in}}{\pgfqpoint{2.109714in}{1.277253in}}%
\pgfpathcurveto{\pgfqpoint{2.115538in}{1.271429in}}{\pgfqpoint{2.123438in}{1.268157in}}{\pgfqpoint{2.131674in}{1.268157in}}%
\pgfpathclose%
\pgfusepath{stroke,fill}%
\end{pgfscope}%
\begin{pgfscope}%
\pgfpathrectangle{\pgfqpoint{0.100000in}{0.212622in}}{\pgfqpoint{3.696000in}{3.696000in}}%
\pgfusepath{clip}%
\pgfsetbuttcap%
\pgfsetroundjoin%
\definecolor{currentfill}{rgb}{0.121569,0.466667,0.705882}%
\pgfsetfillcolor{currentfill}%
\pgfsetfillopacity{0.718193}%
\pgfsetlinewidth{1.003750pt}%
\definecolor{currentstroke}{rgb}{0.121569,0.466667,0.705882}%
\pgfsetstrokecolor{currentstroke}%
\pgfsetstrokeopacity{0.718193}%
\pgfsetdash{}{0pt}%
\pgfpathmoveto{\pgfqpoint{2.139256in}{1.266024in}}%
\pgfpathcurveto{\pgfqpoint{2.147492in}{1.266024in}}{\pgfqpoint{2.155392in}{1.269296in}}{\pgfqpoint{2.161216in}{1.275120in}}%
\pgfpathcurveto{\pgfqpoint{2.167040in}{1.280944in}}{\pgfqpoint{2.170312in}{1.288844in}}{\pgfqpoint{2.170312in}{1.297080in}}%
\pgfpathcurveto{\pgfqpoint{2.170312in}{1.305316in}}{\pgfqpoint{2.167040in}{1.313216in}}{\pgfqpoint{2.161216in}{1.319040in}}%
\pgfpathcurveto{\pgfqpoint{2.155392in}{1.324864in}}{\pgfqpoint{2.147492in}{1.328137in}}{\pgfqpoint{2.139256in}{1.328137in}}%
\pgfpathcurveto{\pgfqpoint{2.131019in}{1.328137in}}{\pgfqpoint{2.123119in}{1.324864in}}{\pgfqpoint{2.117296in}{1.319040in}}%
\pgfpathcurveto{\pgfqpoint{2.111472in}{1.313216in}}{\pgfqpoint{2.108199in}{1.305316in}}{\pgfqpoint{2.108199in}{1.297080in}}%
\pgfpathcurveto{\pgfqpoint{2.108199in}{1.288844in}}{\pgfqpoint{2.111472in}{1.280944in}}{\pgfqpoint{2.117296in}{1.275120in}}%
\pgfpathcurveto{\pgfqpoint{2.123119in}{1.269296in}}{\pgfqpoint{2.131019in}{1.266024in}}{\pgfqpoint{2.139256in}{1.266024in}}%
\pgfpathclose%
\pgfusepath{stroke,fill}%
\end{pgfscope}%
\begin{pgfscope}%
\pgfpathrectangle{\pgfqpoint{0.100000in}{0.212622in}}{\pgfqpoint{3.696000in}{3.696000in}}%
\pgfusepath{clip}%
\pgfsetbuttcap%
\pgfsetroundjoin%
\definecolor{currentfill}{rgb}{0.121569,0.466667,0.705882}%
\pgfsetfillcolor{currentfill}%
\pgfsetfillopacity{0.721795}%
\pgfsetlinewidth{1.003750pt}%
\definecolor{currentstroke}{rgb}{0.121569,0.466667,0.705882}%
\pgfsetstrokecolor{currentstroke}%
\pgfsetstrokeopacity{0.721795}%
\pgfsetdash{}{0pt}%
\pgfpathmoveto{\pgfqpoint{2.147499in}{1.263000in}}%
\pgfpathcurveto{\pgfqpoint{2.155735in}{1.263000in}}{\pgfqpoint{2.163635in}{1.266272in}}{\pgfqpoint{2.169459in}{1.272096in}}%
\pgfpathcurveto{\pgfqpoint{2.175283in}{1.277920in}}{\pgfqpoint{2.178555in}{1.285820in}}{\pgfqpoint{2.178555in}{1.294056in}}%
\pgfpathcurveto{\pgfqpoint{2.178555in}{1.302292in}}{\pgfqpoint{2.175283in}{1.310192in}}{\pgfqpoint{2.169459in}{1.316016in}}%
\pgfpathcurveto{\pgfqpoint{2.163635in}{1.321840in}}{\pgfqpoint{2.155735in}{1.325113in}}{\pgfqpoint{2.147499in}{1.325113in}}%
\pgfpathcurveto{\pgfqpoint{2.139263in}{1.325113in}}{\pgfqpoint{2.131363in}{1.321840in}}{\pgfqpoint{2.125539in}{1.316016in}}%
\pgfpathcurveto{\pgfqpoint{2.119715in}{1.310192in}}{\pgfqpoint{2.116442in}{1.302292in}}{\pgfqpoint{2.116442in}{1.294056in}}%
\pgfpathcurveto{\pgfqpoint{2.116442in}{1.285820in}}{\pgfqpoint{2.119715in}{1.277920in}}{\pgfqpoint{2.125539in}{1.272096in}}%
\pgfpathcurveto{\pgfqpoint{2.131363in}{1.266272in}}{\pgfqpoint{2.139263in}{1.263000in}}{\pgfqpoint{2.147499in}{1.263000in}}%
\pgfpathclose%
\pgfusepath{stroke,fill}%
\end{pgfscope}%
\begin{pgfscope}%
\pgfpathrectangle{\pgfqpoint{0.100000in}{0.212622in}}{\pgfqpoint{3.696000in}{3.696000in}}%
\pgfusepath{clip}%
\pgfsetbuttcap%
\pgfsetroundjoin%
\definecolor{currentfill}{rgb}{0.121569,0.466667,0.705882}%
\pgfsetfillcolor{currentfill}%
\pgfsetfillopacity{0.725931}%
\pgfsetlinewidth{1.003750pt}%
\definecolor{currentstroke}{rgb}{0.121569,0.466667,0.705882}%
\pgfsetstrokecolor{currentstroke}%
\pgfsetstrokeopacity{0.725931}%
\pgfsetdash{}{0pt}%
\pgfpathmoveto{\pgfqpoint{2.156308in}{1.259622in}}%
\pgfpathcurveto{\pgfqpoint{2.164544in}{1.259622in}}{\pgfqpoint{2.172445in}{1.262894in}}{\pgfqpoint{2.178268in}{1.268718in}}%
\pgfpathcurveto{\pgfqpoint{2.184092in}{1.274542in}}{\pgfqpoint{2.187365in}{1.282442in}}{\pgfqpoint{2.187365in}{1.290678in}}%
\pgfpathcurveto{\pgfqpoint{2.187365in}{1.298914in}}{\pgfqpoint{2.184092in}{1.306815in}}{\pgfqpoint{2.178268in}{1.312638in}}%
\pgfpathcurveto{\pgfqpoint{2.172445in}{1.318462in}}{\pgfqpoint{2.164544in}{1.321735in}}{\pgfqpoint{2.156308in}{1.321735in}}%
\pgfpathcurveto{\pgfqpoint{2.148072in}{1.321735in}}{\pgfqpoint{2.140172in}{1.318462in}}{\pgfqpoint{2.134348in}{1.312638in}}%
\pgfpathcurveto{\pgfqpoint{2.128524in}{1.306815in}}{\pgfqpoint{2.125252in}{1.298914in}}{\pgfqpoint{2.125252in}{1.290678in}}%
\pgfpathcurveto{\pgfqpoint{2.125252in}{1.282442in}}{\pgfqpoint{2.128524in}{1.274542in}}{\pgfqpoint{2.134348in}{1.268718in}}%
\pgfpathcurveto{\pgfqpoint{2.140172in}{1.262894in}}{\pgfqpoint{2.148072in}{1.259622in}}{\pgfqpoint{2.156308in}{1.259622in}}%
\pgfpathclose%
\pgfusepath{stroke,fill}%
\end{pgfscope}%
\begin{pgfscope}%
\pgfpathrectangle{\pgfqpoint{0.100000in}{0.212622in}}{\pgfqpoint{3.696000in}{3.696000in}}%
\pgfusepath{clip}%
\pgfsetbuttcap%
\pgfsetroundjoin%
\definecolor{currentfill}{rgb}{0.121569,0.466667,0.705882}%
\pgfsetfillcolor{currentfill}%
\pgfsetfillopacity{0.729576}%
\pgfsetlinewidth{1.003750pt}%
\definecolor{currentstroke}{rgb}{0.121569,0.466667,0.705882}%
\pgfsetstrokecolor{currentstroke}%
\pgfsetstrokeopacity{0.729576}%
\pgfsetdash{}{0pt}%
\pgfpathmoveto{\pgfqpoint{2.166235in}{1.256783in}}%
\pgfpathcurveto{\pgfqpoint{2.174471in}{1.256783in}}{\pgfqpoint{2.182371in}{1.260056in}}{\pgfqpoint{2.188195in}{1.265880in}}%
\pgfpathcurveto{\pgfqpoint{2.194019in}{1.271704in}}{\pgfqpoint{2.197292in}{1.279604in}}{\pgfqpoint{2.197292in}{1.287840in}}%
\pgfpathcurveto{\pgfqpoint{2.197292in}{1.296076in}}{\pgfqpoint{2.194019in}{1.303976in}}{\pgfqpoint{2.188195in}{1.309800in}}%
\pgfpathcurveto{\pgfqpoint{2.182371in}{1.315624in}}{\pgfqpoint{2.174471in}{1.318896in}}{\pgfqpoint{2.166235in}{1.318896in}}%
\pgfpathcurveto{\pgfqpoint{2.157999in}{1.318896in}}{\pgfqpoint{2.150099in}{1.315624in}}{\pgfqpoint{2.144275in}{1.309800in}}%
\pgfpathcurveto{\pgfqpoint{2.138451in}{1.303976in}}{\pgfqpoint{2.135179in}{1.296076in}}{\pgfqpoint{2.135179in}{1.287840in}}%
\pgfpathcurveto{\pgfqpoint{2.135179in}{1.279604in}}{\pgfqpoint{2.138451in}{1.271704in}}{\pgfqpoint{2.144275in}{1.265880in}}%
\pgfpathcurveto{\pgfqpoint{2.150099in}{1.260056in}}{\pgfqpoint{2.157999in}{1.256783in}}{\pgfqpoint{2.166235in}{1.256783in}}%
\pgfpathclose%
\pgfusepath{stroke,fill}%
\end{pgfscope}%
\begin{pgfscope}%
\pgfpathrectangle{\pgfqpoint{0.100000in}{0.212622in}}{\pgfqpoint{3.696000in}{3.696000in}}%
\pgfusepath{clip}%
\pgfsetbuttcap%
\pgfsetroundjoin%
\definecolor{currentfill}{rgb}{0.121569,0.466667,0.705882}%
\pgfsetfillcolor{currentfill}%
\pgfsetfillopacity{0.731733}%
\pgfsetlinewidth{1.003750pt}%
\definecolor{currentstroke}{rgb}{0.121569,0.466667,0.705882}%
\pgfsetstrokecolor{currentstroke}%
\pgfsetstrokeopacity{0.731733}%
\pgfsetdash{}{0pt}%
\pgfpathmoveto{\pgfqpoint{2.171509in}{1.254892in}}%
\pgfpathcurveto{\pgfqpoint{2.179746in}{1.254892in}}{\pgfqpoint{2.187646in}{1.258164in}}{\pgfqpoint{2.193470in}{1.263988in}}%
\pgfpathcurveto{\pgfqpoint{2.199294in}{1.269812in}}{\pgfqpoint{2.202566in}{1.277712in}}{\pgfqpoint{2.202566in}{1.285948in}}%
\pgfpathcurveto{\pgfqpoint{2.202566in}{1.294185in}}{\pgfqpoint{2.199294in}{1.302085in}}{\pgfqpoint{2.193470in}{1.307909in}}%
\pgfpathcurveto{\pgfqpoint{2.187646in}{1.313732in}}{\pgfqpoint{2.179746in}{1.317005in}}{\pgfqpoint{2.171509in}{1.317005in}}%
\pgfpathcurveto{\pgfqpoint{2.163273in}{1.317005in}}{\pgfqpoint{2.155373in}{1.313732in}}{\pgfqpoint{2.149549in}{1.307909in}}%
\pgfpathcurveto{\pgfqpoint{2.143725in}{1.302085in}}{\pgfqpoint{2.140453in}{1.294185in}}{\pgfqpoint{2.140453in}{1.285948in}}%
\pgfpathcurveto{\pgfqpoint{2.140453in}{1.277712in}}{\pgfqpoint{2.143725in}{1.269812in}}{\pgfqpoint{2.149549in}{1.263988in}}%
\pgfpathcurveto{\pgfqpoint{2.155373in}{1.258164in}}{\pgfqpoint{2.163273in}{1.254892in}}{\pgfqpoint{2.171509in}{1.254892in}}%
\pgfpathclose%
\pgfusepath{stroke,fill}%
\end{pgfscope}%
\begin{pgfscope}%
\pgfpathrectangle{\pgfqpoint{0.100000in}{0.212622in}}{\pgfqpoint{3.696000in}{3.696000in}}%
\pgfusepath{clip}%
\pgfsetbuttcap%
\pgfsetroundjoin%
\definecolor{currentfill}{rgb}{0.121569,0.466667,0.705882}%
\pgfsetfillcolor{currentfill}%
\pgfsetfillopacity{0.732618}%
\pgfsetlinewidth{1.003750pt}%
\definecolor{currentstroke}{rgb}{0.121569,0.466667,0.705882}%
\pgfsetstrokecolor{currentstroke}%
\pgfsetstrokeopacity{0.732618}%
\pgfsetdash{}{0pt}%
\pgfpathmoveto{\pgfqpoint{2.174701in}{1.254225in}}%
\pgfpathcurveto{\pgfqpoint{2.182937in}{1.254225in}}{\pgfqpoint{2.190837in}{1.257497in}}{\pgfqpoint{2.196661in}{1.263321in}}%
\pgfpathcurveto{\pgfqpoint{2.202485in}{1.269145in}}{\pgfqpoint{2.205757in}{1.277045in}}{\pgfqpoint{2.205757in}{1.285282in}}%
\pgfpathcurveto{\pgfqpoint{2.205757in}{1.293518in}}{\pgfqpoint{2.202485in}{1.301418in}}{\pgfqpoint{2.196661in}{1.307242in}}%
\pgfpathcurveto{\pgfqpoint{2.190837in}{1.313066in}}{\pgfqpoint{2.182937in}{1.316338in}}{\pgfqpoint{2.174701in}{1.316338in}}%
\pgfpathcurveto{\pgfqpoint{2.166464in}{1.316338in}}{\pgfqpoint{2.158564in}{1.313066in}}{\pgfqpoint{2.152740in}{1.307242in}}%
\pgfpathcurveto{\pgfqpoint{2.146917in}{1.301418in}}{\pgfqpoint{2.143644in}{1.293518in}}{\pgfqpoint{2.143644in}{1.285282in}}%
\pgfpathcurveto{\pgfqpoint{2.143644in}{1.277045in}}{\pgfqpoint{2.146917in}{1.269145in}}{\pgfqpoint{2.152740in}{1.263321in}}%
\pgfpathcurveto{\pgfqpoint{2.158564in}{1.257497in}}{\pgfqpoint{2.166464in}{1.254225in}}{\pgfqpoint{2.174701in}{1.254225in}}%
\pgfpathclose%
\pgfusepath{stroke,fill}%
\end{pgfscope}%
\begin{pgfscope}%
\pgfpathrectangle{\pgfqpoint{0.100000in}{0.212622in}}{\pgfqpoint{3.696000in}{3.696000in}}%
\pgfusepath{clip}%
\pgfsetbuttcap%
\pgfsetroundjoin%
\definecolor{currentfill}{rgb}{0.121569,0.466667,0.705882}%
\pgfsetfillcolor{currentfill}%
\pgfsetfillopacity{0.734090}%
\pgfsetlinewidth{1.003750pt}%
\definecolor{currentstroke}{rgb}{0.121569,0.466667,0.705882}%
\pgfsetstrokecolor{currentstroke}%
\pgfsetstrokeopacity{0.734090}%
\pgfsetdash{}{0pt}%
\pgfpathmoveto{\pgfqpoint{2.178689in}{1.252954in}}%
\pgfpathcurveto{\pgfqpoint{2.186925in}{1.252954in}}{\pgfqpoint{2.194825in}{1.256226in}}{\pgfqpoint{2.200649in}{1.262050in}}%
\pgfpathcurveto{\pgfqpoint{2.206473in}{1.267874in}}{\pgfqpoint{2.209746in}{1.275774in}}{\pgfqpoint{2.209746in}{1.284010in}}%
\pgfpathcurveto{\pgfqpoint{2.209746in}{1.292246in}}{\pgfqpoint{2.206473in}{1.300146in}}{\pgfqpoint{2.200649in}{1.305970in}}%
\pgfpathcurveto{\pgfqpoint{2.194825in}{1.311794in}}{\pgfqpoint{2.186925in}{1.315067in}}{\pgfqpoint{2.178689in}{1.315067in}}%
\pgfpathcurveto{\pgfqpoint{2.170453in}{1.315067in}}{\pgfqpoint{2.162553in}{1.311794in}}{\pgfqpoint{2.156729in}{1.305970in}}%
\pgfpathcurveto{\pgfqpoint{2.150905in}{1.300146in}}{\pgfqpoint{2.147633in}{1.292246in}}{\pgfqpoint{2.147633in}{1.284010in}}%
\pgfpathcurveto{\pgfqpoint{2.147633in}{1.275774in}}{\pgfqpoint{2.150905in}{1.267874in}}{\pgfqpoint{2.156729in}{1.262050in}}%
\pgfpathcurveto{\pgfqpoint{2.162553in}{1.256226in}}{\pgfqpoint{2.170453in}{1.252954in}}{\pgfqpoint{2.178689in}{1.252954in}}%
\pgfpathclose%
\pgfusepath{stroke,fill}%
\end{pgfscope}%
\begin{pgfscope}%
\pgfpathrectangle{\pgfqpoint{0.100000in}{0.212622in}}{\pgfqpoint{3.696000in}{3.696000in}}%
\pgfusepath{clip}%
\pgfsetbuttcap%
\pgfsetroundjoin%
\definecolor{currentfill}{rgb}{0.121569,0.466667,0.705882}%
\pgfsetfillcolor{currentfill}%
\pgfsetfillopacity{0.736545}%
\pgfsetlinewidth{1.003750pt}%
\definecolor{currentstroke}{rgb}{0.121569,0.466667,0.705882}%
\pgfsetstrokecolor{currentstroke}%
\pgfsetstrokeopacity{0.736545}%
\pgfsetdash{}{0pt}%
\pgfpathmoveto{\pgfqpoint{2.184289in}{1.250953in}}%
\pgfpathcurveto{\pgfqpoint{2.192525in}{1.250953in}}{\pgfqpoint{2.200425in}{1.254225in}}{\pgfqpoint{2.206249in}{1.260049in}}%
\pgfpathcurveto{\pgfqpoint{2.212073in}{1.265873in}}{\pgfqpoint{2.215345in}{1.273773in}}{\pgfqpoint{2.215345in}{1.282009in}}%
\pgfpathcurveto{\pgfqpoint{2.215345in}{1.290246in}}{\pgfqpoint{2.212073in}{1.298146in}}{\pgfqpoint{2.206249in}{1.303970in}}%
\pgfpathcurveto{\pgfqpoint{2.200425in}{1.309794in}}{\pgfqpoint{2.192525in}{1.313066in}}{\pgfqpoint{2.184289in}{1.313066in}}%
\pgfpathcurveto{\pgfqpoint{2.176053in}{1.313066in}}{\pgfqpoint{2.168153in}{1.309794in}}{\pgfqpoint{2.162329in}{1.303970in}}%
\pgfpathcurveto{\pgfqpoint{2.156505in}{1.298146in}}{\pgfqpoint{2.153232in}{1.290246in}}{\pgfqpoint{2.153232in}{1.282009in}}%
\pgfpathcurveto{\pgfqpoint{2.153232in}{1.273773in}}{\pgfqpoint{2.156505in}{1.265873in}}{\pgfqpoint{2.162329in}{1.260049in}}%
\pgfpathcurveto{\pgfqpoint{2.168153in}{1.254225in}}{\pgfqpoint{2.176053in}{1.250953in}}{\pgfqpoint{2.184289in}{1.250953in}}%
\pgfpathclose%
\pgfusepath{stroke,fill}%
\end{pgfscope}%
\begin{pgfscope}%
\pgfpathrectangle{\pgfqpoint{0.100000in}{0.212622in}}{\pgfqpoint{3.696000in}{3.696000in}}%
\pgfusepath{clip}%
\pgfsetbuttcap%
\pgfsetroundjoin%
\definecolor{currentfill}{rgb}{0.121569,0.466667,0.705882}%
\pgfsetfillcolor{currentfill}%
\pgfsetfillopacity{0.739059}%
\pgfsetlinewidth{1.003750pt}%
\definecolor{currentstroke}{rgb}{0.121569,0.466667,0.705882}%
\pgfsetstrokecolor{currentstroke}%
\pgfsetstrokeopacity{0.739059}%
\pgfsetdash{}{0pt}%
\pgfpathmoveto{\pgfqpoint{2.191928in}{1.249000in}}%
\pgfpathcurveto{\pgfqpoint{2.200164in}{1.249000in}}{\pgfqpoint{2.208064in}{1.252272in}}{\pgfqpoint{2.213888in}{1.258096in}}%
\pgfpathcurveto{\pgfqpoint{2.219712in}{1.263920in}}{\pgfqpoint{2.222984in}{1.271820in}}{\pgfqpoint{2.222984in}{1.280056in}}%
\pgfpathcurveto{\pgfqpoint{2.222984in}{1.288293in}}{\pgfqpoint{2.219712in}{1.296193in}}{\pgfqpoint{2.213888in}{1.302016in}}%
\pgfpathcurveto{\pgfqpoint{2.208064in}{1.307840in}}{\pgfqpoint{2.200164in}{1.311113in}}{\pgfqpoint{2.191928in}{1.311113in}}%
\pgfpathcurveto{\pgfqpoint{2.183691in}{1.311113in}}{\pgfqpoint{2.175791in}{1.307840in}}{\pgfqpoint{2.169967in}{1.302016in}}%
\pgfpathcurveto{\pgfqpoint{2.164143in}{1.296193in}}{\pgfqpoint{2.160871in}{1.288293in}}{\pgfqpoint{2.160871in}{1.280056in}}%
\pgfpathcurveto{\pgfqpoint{2.160871in}{1.271820in}}{\pgfqpoint{2.164143in}{1.263920in}}{\pgfqpoint{2.169967in}{1.258096in}}%
\pgfpathcurveto{\pgfqpoint{2.175791in}{1.252272in}}{\pgfqpoint{2.183691in}{1.249000in}}{\pgfqpoint{2.191928in}{1.249000in}}%
\pgfpathclose%
\pgfusepath{stroke,fill}%
\end{pgfscope}%
\begin{pgfscope}%
\pgfpathrectangle{\pgfqpoint{0.100000in}{0.212622in}}{\pgfqpoint{3.696000in}{3.696000in}}%
\pgfusepath{clip}%
\pgfsetbuttcap%
\pgfsetroundjoin%
\definecolor{currentfill}{rgb}{0.121569,0.466667,0.705882}%
\pgfsetfillcolor{currentfill}%
\pgfsetfillopacity{0.742516}%
\pgfsetlinewidth{1.003750pt}%
\definecolor{currentstroke}{rgb}{0.121569,0.466667,0.705882}%
\pgfsetstrokecolor{currentstroke}%
\pgfsetstrokeopacity{0.742516}%
\pgfsetdash{}{0pt}%
\pgfpathmoveto{\pgfqpoint{2.200281in}{1.246218in}}%
\pgfpathcurveto{\pgfqpoint{2.208517in}{1.246218in}}{\pgfqpoint{2.216417in}{1.249491in}}{\pgfqpoint{2.222241in}{1.255314in}}%
\pgfpathcurveto{\pgfqpoint{2.228065in}{1.261138in}}{\pgfqpoint{2.231337in}{1.269038in}}{\pgfqpoint{2.231337in}{1.277275in}}%
\pgfpathcurveto{\pgfqpoint{2.231337in}{1.285511in}}{\pgfqpoint{2.228065in}{1.293411in}}{\pgfqpoint{2.222241in}{1.299235in}}%
\pgfpathcurveto{\pgfqpoint{2.216417in}{1.305059in}}{\pgfqpoint{2.208517in}{1.308331in}}{\pgfqpoint{2.200281in}{1.308331in}}%
\pgfpathcurveto{\pgfqpoint{2.192045in}{1.308331in}}{\pgfqpoint{2.184145in}{1.305059in}}{\pgfqpoint{2.178321in}{1.299235in}}%
\pgfpathcurveto{\pgfqpoint{2.172497in}{1.293411in}}{\pgfqpoint{2.169224in}{1.285511in}}{\pgfqpoint{2.169224in}{1.277275in}}%
\pgfpathcurveto{\pgfqpoint{2.169224in}{1.269038in}}{\pgfqpoint{2.172497in}{1.261138in}}{\pgfqpoint{2.178321in}{1.255314in}}%
\pgfpathcurveto{\pgfqpoint{2.184145in}{1.249491in}}{\pgfqpoint{2.192045in}{1.246218in}}{\pgfqpoint{2.200281in}{1.246218in}}%
\pgfpathclose%
\pgfusepath{stroke,fill}%
\end{pgfscope}%
\begin{pgfscope}%
\pgfpathrectangle{\pgfqpoint{0.100000in}{0.212622in}}{\pgfqpoint{3.696000in}{3.696000in}}%
\pgfusepath{clip}%
\pgfsetbuttcap%
\pgfsetroundjoin%
\definecolor{currentfill}{rgb}{0.121569,0.466667,0.705882}%
\pgfsetfillcolor{currentfill}%
\pgfsetfillopacity{0.745337}%
\pgfsetlinewidth{1.003750pt}%
\definecolor{currentstroke}{rgb}{0.121569,0.466667,0.705882}%
\pgfsetstrokecolor{currentstroke}%
\pgfsetstrokeopacity{0.745337}%
\pgfsetdash{}{0pt}%
\pgfpathmoveto{\pgfqpoint{2.210073in}{1.244533in}}%
\pgfpathcurveto{\pgfqpoint{2.218309in}{1.244533in}}{\pgfqpoint{2.226209in}{1.247805in}}{\pgfqpoint{2.232033in}{1.253629in}}%
\pgfpathcurveto{\pgfqpoint{2.237857in}{1.259453in}}{\pgfqpoint{2.241129in}{1.267353in}}{\pgfqpoint{2.241129in}{1.275589in}}%
\pgfpathcurveto{\pgfqpoint{2.241129in}{1.283826in}}{\pgfqpoint{2.237857in}{1.291726in}}{\pgfqpoint{2.232033in}{1.297550in}}%
\pgfpathcurveto{\pgfqpoint{2.226209in}{1.303374in}}{\pgfqpoint{2.218309in}{1.306646in}}{\pgfqpoint{2.210073in}{1.306646in}}%
\pgfpathcurveto{\pgfqpoint{2.201836in}{1.306646in}}{\pgfqpoint{2.193936in}{1.303374in}}{\pgfqpoint{2.188112in}{1.297550in}}%
\pgfpathcurveto{\pgfqpoint{2.182288in}{1.291726in}}{\pgfqpoint{2.179016in}{1.283826in}}{\pgfqpoint{2.179016in}{1.275589in}}%
\pgfpathcurveto{\pgfqpoint{2.179016in}{1.267353in}}{\pgfqpoint{2.182288in}{1.259453in}}{\pgfqpoint{2.188112in}{1.253629in}}%
\pgfpathcurveto{\pgfqpoint{2.193936in}{1.247805in}}{\pgfqpoint{2.201836in}{1.244533in}}{\pgfqpoint{2.210073in}{1.244533in}}%
\pgfpathclose%
\pgfusepath{stroke,fill}%
\end{pgfscope}%
\begin{pgfscope}%
\pgfpathrectangle{\pgfqpoint{0.100000in}{0.212622in}}{\pgfqpoint{3.696000in}{3.696000in}}%
\pgfusepath{clip}%
\pgfsetbuttcap%
\pgfsetroundjoin%
\definecolor{currentfill}{rgb}{0.121569,0.466667,0.705882}%
\pgfsetfillcolor{currentfill}%
\pgfsetfillopacity{0.747176}%
\pgfsetlinewidth{1.003750pt}%
\definecolor{currentstroke}{rgb}{0.121569,0.466667,0.705882}%
\pgfsetstrokecolor{currentstroke}%
\pgfsetstrokeopacity{0.747176}%
\pgfsetdash{}{0pt}%
\pgfpathmoveto{\pgfqpoint{2.215143in}{1.243104in}}%
\pgfpathcurveto{\pgfqpoint{2.223379in}{1.243104in}}{\pgfqpoint{2.231280in}{1.246376in}}{\pgfqpoint{2.237103in}{1.252200in}}%
\pgfpathcurveto{\pgfqpoint{2.242927in}{1.258024in}}{\pgfqpoint{2.246200in}{1.265924in}}{\pgfqpoint{2.246200in}{1.274160in}}%
\pgfpathcurveto{\pgfqpoint{2.246200in}{1.282396in}}{\pgfqpoint{2.242927in}{1.290296in}}{\pgfqpoint{2.237103in}{1.296120in}}%
\pgfpathcurveto{\pgfqpoint{2.231280in}{1.301944in}}{\pgfqpoint{2.223379in}{1.305217in}}{\pgfqpoint{2.215143in}{1.305217in}}%
\pgfpathcurveto{\pgfqpoint{2.206907in}{1.305217in}}{\pgfqpoint{2.199007in}{1.301944in}}{\pgfqpoint{2.193183in}{1.296120in}}%
\pgfpathcurveto{\pgfqpoint{2.187359in}{1.290296in}}{\pgfqpoint{2.184087in}{1.282396in}}{\pgfqpoint{2.184087in}{1.274160in}}%
\pgfpathcurveto{\pgfqpoint{2.184087in}{1.265924in}}{\pgfqpoint{2.187359in}{1.258024in}}{\pgfqpoint{2.193183in}{1.252200in}}%
\pgfpathcurveto{\pgfqpoint{2.199007in}{1.246376in}}{\pgfqpoint{2.206907in}{1.243104in}}{\pgfqpoint{2.215143in}{1.243104in}}%
\pgfpathclose%
\pgfusepath{stroke,fill}%
\end{pgfscope}%
\begin{pgfscope}%
\pgfpathrectangle{\pgfqpoint{0.100000in}{0.212622in}}{\pgfqpoint{3.696000in}{3.696000in}}%
\pgfusepath{clip}%
\pgfsetbuttcap%
\pgfsetroundjoin%
\definecolor{currentfill}{rgb}{0.121569,0.466667,0.705882}%
\pgfsetfillcolor{currentfill}%
\pgfsetfillopacity{0.748103}%
\pgfsetlinewidth{1.003750pt}%
\definecolor{currentstroke}{rgb}{0.121569,0.466667,0.705882}%
\pgfsetstrokecolor{currentstroke}%
\pgfsetstrokeopacity{0.748103}%
\pgfsetdash{}{0pt}%
\pgfpathmoveto{\pgfqpoint{2.217981in}{1.242308in}}%
\pgfpathcurveto{\pgfqpoint{2.226217in}{1.242308in}}{\pgfqpoint{2.234117in}{1.245580in}}{\pgfqpoint{2.239941in}{1.251404in}}%
\pgfpathcurveto{\pgfqpoint{2.245765in}{1.257228in}}{\pgfqpoint{2.249037in}{1.265128in}}{\pgfqpoint{2.249037in}{1.273365in}}%
\pgfpathcurveto{\pgfqpoint{2.249037in}{1.281601in}}{\pgfqpoint{2.245765in}{1.289501in}}{\pgfqpoint{2.239941in}{1.295325in}}%
\pgfpathcurveto{\pgfqpoint{2.234117in}{1.301149in}}{\pgfqpoint{2.226217in}{1.304421in}}{\pgfqpoint{2.217981in}{1.304421in}}%
\pgfpathcurveto{\pgfqpoint{2.209745in}{1.304421in}}{\pgfqpoint{2.201845in}{1.301149in}}{\pgfqpoint{2.196021in}{1.295325in}}%
\pgfpathcurveto{\pgfqpoint{2.190197in}{1.289501in}}{\pgfqpoint{2.186924in}{1.281601in}}{\pgfqpoint{2.186924in}{1.273365in}}%
\pgfpathcurveto{\pgfqpoint{2.186924in}{1.265128in}}{\pgfqpoint{2.190197in}{1.257228in}}{\pgfqpoint{2.196021in}{1.251404in}}%
\pgfpathcurveto{\pgfqpoint{2.201845in}{1.245580in}}{\pgfqpoint{2.209745in}{1.242308in}}{\pgfqpoint{2.217981in}{1.242308in}}%
\pgfpathclose%
\pgfusepath{stroke,fill}%
\end{pgfscope}%
\begin{pgfscope}%
\pgfpathrectangle{\pgfqpoint{0.100000in}{0.212622in}}{\pgfqpoint{3.696000in}{3.696000in}}%
\pgfusepath{clip}%
\pgfsetbuttcap%
\pgfsetroundjoin%
\definecolor{currentfill}{rgb}{0.121569,0.466667,0.705882}%
\pgfsetfillcolor{currentfill}%
\pgfsetfillopacity{0.749541}%
\pgfsetlinewidth{1.003750pt}%
\definecolor{currentstroke}{rgb}{0.121569,0.466667,0.705882}%
\pgfsetstrokecolor{currentstroke}%
\pgfsetstrokeopacity{0.749541}%
\pgfsetdash{}{0pt}%
\pgfpathmoveto{\pgfqpoint{2.221658in}{1.241097in}}%
\pgfpathcurveto{\pgfqpoint{2.229895in}{1.241097in}}{\pgfqpoint{2.237795in}{1.244370in}}{\pgfqpoint{2.243619in}{1.250194in}}%
\pgfpathcurveto{\pgfqpoint{2.249443in}{1.256018in}}{\pgfqpoint{2.252715in}{1.263918in}}{\pgfqpoint{2.252715in}{1.272154in}}%
\pgfpathcurveto{\pgfqpoint{2.252715in}{1.280390in}}{\pgfqpoint{2.249443in}{1.288290in}}{\pgfqpoint{2.243619in}{1.294114in}}%
\pgfpathcurveto{\pgfqpoint{2.237795in}{1.299938in}}{\pgfqpoint{2.229895in}{1.303210in}}{\pgfqpoint{2.221658in}{1.303210in}}%
\pgfpathcurveto{\pgfqpoint{2.213422in}{1.303210in}}{\pgfqpoint{2.205522in}{1.299938in}}{\pgfqpoint{2.199698in}{1.294114in}}%
\pgfpathcurveto{\pgfqpoint{2.193874in}{1.288290in}}{\pgfqpoint{2.190602in}{1.280390in}}{\pgfqpoint{2.190602in}{1.272154in}}%
\pgfpathcurveto{\pgfqpoint{2.190602in}{1.263918in}}{\pgfqpoint{2.193874in}{1.256018in}}{\pgfqpoint{2.199698in}{1.250194in}}%
\pgfpathcurveto{\pgfqpoint{2.205522in}{1.244370in}}{\pgfqpoint{2.213422in}{1.241097in}}{\pgfqpoint{2.221658in}{1.241097in}}%
\pgfpathclose%
\pgfusepath{stroke,fill}%
\end{pgfscope}%
\begin{pgfscope}%
\pgfpathrectangle{\pgfqpoint{0.100000in}{0.212622in}}{\pgfqpoint{3.696000in}{3.696000in}}%
\pgfusepath{clip}%
\pgfsetbuttcap%
\pgfsetroundjoin%
\definecolor{currentfill}{rgb}{0.121569,0.466667,0.705882}%
\pgfsetfillcolor{currentfill}%
\pgfsetfillopacity{0.751371}%
\pgfsetlinewidth{1.003750pt}%
\definecolor{currentstroke}{rgb}{0.121569,0.466667,0.705882}%
\pgfsetstrokecolor{currentstroke}%
\pgfsetstrokeopacity{0.751371}%
\pgfsetdash{}{0pt}%
\pgfpathmoveto{\pgfqpoint{2.226833in}{1.239491in}}%
\pgfpathcurveto{\pgfqpoint{2.235069in}{1.239491in}}{\pgfqpoint{2.242969in}{1.242763in}}{\pgfqpoint{2.248793in}{1.248587in}}%
\pgfpathcurveto{\pgfqpoint{2.254617in}{1.254411in}}{\pgfqpoint{2.257889in}{1.262311in}}{\pgfqpoint{2.257889in}{1.270547in}}%
\pgfpathcurveto{\pgfqpoint{2.257889in}{1.278784in}}{\pgfqpoint{2.254617in}{1.286684in}}{\pgfqpoint{2.248793in}{1.292508in}}%
\pgfpathcurveto{\pgfqpoint{2.242969in}{1.298332in}}{\pgfqpoint{2.235069in}{1.301604in}}{\pgfqpoint{2.226833in}{1.301604in}}%
\pgfpathcurveto{\pgfqpoint{2.218597in}{1.301604in}}{\pgfqpoint{2.210697in}{1.298332in}}{\pgfqpoint{2.204873in}{1.292508in}}%
\pgfpathcurveto{\pgfqpoint{2.199049in}{1.286684in}}{\pgfqpoint{2.195776in}{1.278784in}}{\pgfqpoint{2.195776in}{1.270547in}}%
\pgfpathcurveto{\pgfqpoint{2.195776in}{1.262311in}}{\pgfqpoint{2.199049in}{1.254411in}}{\pgfqpoint{2.204873in}{1.248587in}}%
\pgfpathcurveto{\pgfqpoint{2.210697in}{1.242763in}}{\pgfqpoint{2.218597in}{1.239491in}}{\pgfqpoint{2.226833in}{1.239491in}}%
\pgfpathclose%
\pgfusepath{stroke,fill}%
\end{pgfscope}%
\begin{pgfscope}%
\pgfpathrectangle{\pgfqpoint{0.100000in}{0.212622in}}{\pgfqpoint{3.696000in}{3.696000in}}%
\pgfusepath{clip}%
\pgfsetbuttcap%
\pgfsetroundjoin%
\definecolor{currentfill}{rgb}{0.121569,0.466667,0.705882}%
\pgfsetfillcolor{currentfill}%
\pgfsetfillopacity{0.753833}%
\pgfsetlinewidth{1.003750pt}%
\definecolor{currentstroke}{rgb}{0.121569,0.466667,0.705882}%
\pgfsetstrokecolor{currentstroke}%
\pgfsetstrokeopacity{0.753833}%
\pgfsetdash{}{0pt}%
\pgfpathmoveto{\pgfqpoint{2.233157in}{1.237618in}}%
\pgfpathcurveto{\pgfqpoint{2.241393in}{1.237618in}}{\pgfqpoint{2.249293in}{1.240890in}}{\pgfqpoint{2.255117in}{1.246714in}}%
\pgfpathcurveto{\pgfqpoint{2.260941in}{1.252538in}}{\pgfqpoint{2.264213in}{1.260438in}}{\pgfqpoint{2.264213in}{1.268675in}}%
\pgfpathcurveto{\pgfqpoint{2.264213in}{1.276911in}}{\pgfqpoint{2.260941in}{1.284811in}}{\pgfqpoint{2.255117in}{1.290635in}}%
\pgfpathcurveto{\pgfqpoint{2.249293in}{1.296459in}}{\pgfqpoint{2.241393in}{1.299731in}}{\pgfqpoint{2.233157in}{1.299731in}}%
\pgfpathcurveto{\pgfqpoint{2.224920in}{1.299731in}}{\pgfqpoint{2.217020in}{1.296459in}}{\pgfqpoint{2.211196in}{1.290635in}}%
\pgfpathcurveto{\pgfqpoint{2.205372in}{1.284811in}}{\pgfqpoint{2.202100in}{1.276911in}}{\pgfqpoint{2.202100in}{1.268675in}}%
\pgfpathcurveto{\pgfqpoint{2.202100in}{1.260438in}}{\pgfqpoint{2.205372in}{1.252538in}}{\pgfqpoint{2.211196in}{1.246714in}}%
\pgfpathcurveto{\pgfqpoint{2.217020in}{1.240890in}}{\pgfqpoint{2.224920in}{1.237618in}}{\pgfqpoint{2.233157in}{1.237618in}}%
\pgfpathclose%
\pgfusepath{stroke,fill}%
\end{pgfscope}%
\begin{pgfscope}%
\pgfpathrectangle{\pgfqpoint{0.100000in}{0.212622in}}{\pgfqpoint{3.696000in}{3.696000in}}%
\pgfusepath{clip}%
\pgfsetbuttcap%
\pgfsetroundjoin%
\definecolor{currentfill}{rgb}{0.121569,0.466667,0.705882}%
\pgfsetfillcolor{currentfill}%
\pgfsetfillopacity{0.755927}%
\pgfsetlinewidth{1.003750pt}%
\definecolor{currentstroke}{rgb}{0.121569,0.466667,0.705882}%
\pgfsetstrokecolor{currentstroke}%
\pgfsetstrokeopacity{0.755927}%
\pgfsetdash{}{0pt}%
\pgfpathmoveto{\pgfqpoint{2.241203in}{1.236350in}}%
\pgfpathcurveto{\pgfqpoint{2.249440in}{1.236350in}}{\pgfqpoint{2.257340in}{1.239623in}}{\pgfqpoint{2.263164in}{1.245447in}}%
\pgfpathcurveto{\pgfqpoint{2.268988in}{1.251271in}}{\pgfqpoint{2.272260in}{1.259171in}}{\pgfqpoint{2.272260in}{1.267407in}}%
\pgfpathcurveto{\pgfqpoint{2.272260in}{1.275643in}}{\pgfqpoint{2.268988in}{1.283543in}}{\pgfqpoint{2.263164in}{1.289367in}}%
\pgfpathcurveto{\pgfqpoint{2.257340in}{1.295191in}}{\pgfqpoint{2.249440in}{1.298463in}}{\pgfqpoint{2.241203in}{1.298463in}}%
\pgfpathcurveto{\pgfqpoint{2.232967in}{1.298463in}}{\pgfqpoint{2.225067in}{1.295191in}}{\pgfqpoint{2.219243in}{1.289367in}}%
\pgfpathcurveto{\pgfqpoint{2.213419in}{1.283543in}}{\pgfqpoint{2.210147in}{1.275643in}}{\pgfqpoint{2.210147in}{1.267407in}}%
\pgfpathcurveto{\pgfqpoint{2.210147in}{1.259171in}}{\pgfqpoint{2.213419in}{1.251271in}}{\pgfqpoint{2.219243in}{1.245447in}}%
\pgfpathcurveto{\pgfqpoint{2.225067in}{1.239623in}}{\pgfqpoint{2.232967in}{1.236350in}}{\pgfqpoint{2.241203in}{1.236350in}}%
\pgfpathclose%
\pgfusepath{stroke,fill}%
\end{pgfscope}%
\begin{pgfscope}%
\pgfpathrectangle{\pgfqpoint{0.100000in}{0.212622in}}{\pgfqpoint{3.696000in}{3.696000in}}%
\pgfusepath{clip}%
\pgfsetbuttcap%
\pgfsetroundjoin%
\definecolor{currentfill}{rgb}{0.121569,0.466667,0.705882}%
\pgfsetfillcolor{currentfill}%
\pgfsetfillopacity{0.759498}%
\pgfsetlinewidth{1.003750pt}%
\definecolor{currentstroke}{rgb}{0.121569,0.466667,0.705882}%
\pgfsetstrokecolor{currentstroke}%
\pgfsetstrokeopacity{0.759498}%
\pgfsetdash{}{0pt}%
\pgfpathmoveto{\pgfqpoint{2.249254in}{1.233447in}}%
\pgfpathcurveto{\pgfqpoint{2.257490in}{1.233447in}}{\pgfqpoint{2.265390in}{1.236719in}}{\pgfqpoint{2.271214in}{1.242543in}}%
\pgfpathcurveto{\pgfqpoint{2.277038in}{1.248367in}}{\pgfqpoint{2.280310in}{1.256267in}}{\pgfqpoint{2.280310in}{1.264503in}}%
\pgfpathcurveto{\pgfqpoint{2.280310in}{1.272740in}}{\pgfqpoint{2.277038in}{1.280640in}}{\pgfqpoint{2.271214in}{1.286464in}}%
\pgfpathcurveto{\pgfqpoint{2.265390in}{1.292287in}}{\pgfqpoint{2.257490in}{1.295560in}}{\pgfqpoint{2.249254in}{1.295560in}}%
\pgfpathcurveto{\pgfqpoint{2.241018in}{1.295560in}}{\pgfqpoint{2.233118in}{1.292287in}}{\pgfqpoint{2.227294in}{1.286464in}}%
\pgfpathcurveto{\pgfqpoint{2.221470in}{1.280640in}}{\pgfqpoint{2.218197in}{1.272740in}}{\pgfqpoint{2.218197in}{1.264503in}}%
\pgfpathcurveto{\pgfqpoint{2.218197in}{1.256267in}}{\pgfqpoint{2.221470in}{1.248367in}}{\pgfqpoint{2.227294in}{1.242543in}}%
\pgfpathcurveto{\pgfqpoint{2.233118in}{1.236719in}}{\pgfqpoint{2.241018in}{1.233447in}}{\pgfqpoint{2.249254in}{1.233447in}}%
\pgfpathclose%
\pgfusepath{stroke,fill}%
\end{pgfscope}%
\begin{pgfscope}%
\pgfpathrectangle{\pgfqpoint{0.100000in}{0.212622in}}{\pgfqpoint{3.696000in}{3.696000in}}%
\pgfusepath{clip}%
\pgfsetbuttcap%
\pgfsetroundjoin%
\definecolor{currentfill}{rgb}{0.121569,0.466667,0.705882}%
\pgfsetfillcolor{currentfill}%
\pgfsetfillopacity{0.763208}%
\pgfsetlinewidth{1.003750pt}%
\definecolor{currentstroke}{rgb}{0.121569,0.466667,0.705882}%
\pgfsetstrokecolor{currentstroke}%
\pgfsetstrokeopacity{0.763208}%
\pgfsetdash{}{0pt}%
\pgfpathmoveto{\pgfqpoint{2.257920in}{1.230278in}}%
\pgfpathcurveto{\pgfqpoint{2.266157in}{1.230278in}}{\pgfqpoint{2.274057in}{1.233551in}}{\pgfqpoint{2.279881in}{1.239375in}}%
\pgfpathcurveto{\pgfqpoint{2.285705in}{1.245199in}}{\pgfqpoint{2.288977in}{1.253099in}}{\pgfqpoint{2.288977in}{1.261335in}}%
\pgfpathcurveto{\pgfqpoint{2.288977in}{1.269571in}}{\pgfqpoint{2.285705in}{1.277471in}}{\pgfqpoint{2.279881in}{1.283295in}}%
\pgfpathcurveto{\pgfqpoint{2.274057in}{1.289119in}}{\pgfqpoint{2.266157in}{1.292391in}}{\pgfqpoint{2.257920in}{1.292391in}}%
\pgfpathcurveto{\pgfqpoint{2.249684in}{1.292391in}}{\pgfqpoint{2.241784in}{1.289119in}}{\pgfqpoint{2.235960in}{1.283295in}}%
\pgfpathcurveto{\pgfqpoint{2.230136in}{1.277471in}}{\pgfqpoint{2.226864in}{1.269571in}}{\pgfqpoint{2.226864in}{1.261335in}}%
\pgfpathcurveto{\pgfqpoint{2.226864in}{1.253099in}}{\pgfqpoint{2.230136in}{1.245199in}}{\pgfqpoint{2.235960in}{1.239375in}}%
\pgfpathcurveto{\pgfqpoint{2.241784in}{1.233551in}}{\pgfqpoint{2.249684in}{1.230278in}}{\pgfqpoint{2.257920in}{1.230278in}}%
\pgfpathclose%
\pgfusepath{stroke,fill}%
\end{pgfscope}%
\begin{pgfscope}%
\pgfpathrectangle{\pgfqpoint{0.100000in}{0.212622in}}{\pgfqpoint{3.696000in}{3.696000in}}%
\pgfusepath{clip}%
\pgfsetbuttcap%
\pgfsetroundjoin%
\definecolor{currentfill}{rgb}{0.121569,0.466667,0.705882}%
\pgfsetfillcolor{currentfill}%
\pgfsetfillopacity{0.765164}%
\pgfsetlinewidth{1.003750pt}%
\definecolor{currentstroke}{rgb}{0.121569,0.466667,0.705882}%
\pgfsetstrokecolor{currentstroke}%
\pgfsetstrokeopacity{0.765164}%
\pgfsetdash{}{0pt}%
\pgfpathmoveto{\pgfqpoint{2.262783in}{1.228683in}}%
\pgfpathcurveto{\pgfqpoint{2.271019in}{1.228683in}}{\pgfqpoint{2.278919in}{1.231955in}}{\pgfqpoint{2.284743in}{1.237779in}}%
\pgfpathcurveto{\pgfqpoint{2.290567in}{1.243603in}}{\pgfqpoint{2.293839in}{1.251503in}}{\pgfqpoint{2.293839in}{1.259739in}}%
\pgfpathcurveto{\pgfqpoint{2.293839in}{1.267975in}}{\pgfqpoint{2.290567in}{1.275875in}}{\pgfqpoint{2.284743in}{1.281699in}}%
\pgfpathcurveto{\pgfqpoint{2.278919in}{1.287523in}}{\pgfqpoint{2.271019in}{1.290796in}}{\pgfqpoint{2.262783in}{1.290796in}}%
\pgfpathcurveto{\pgfqpoint{2.254546in}{1.290796in}}{\pgfqpoint{2.246646in}{1.287523in}}{\pgfqpoint{2.240823in}{1.281699in}}%
\pgfpathcurveto{\pgfqpoint{2.234999in}{1.275875in}}{\pgfqpoint{2.231726in}{1.267975in}}{\pgfqpoint{2.231726in}{1.259739in}}%
\pgfpathcurveto{\pgfqpoint{2.231726in}{1.251503in}}{\pgfqpoint{2.234999in}{1.243603in}}{\pgfqpoint{2.240823in}{1.237779in}}%
\pgfpathcurveto{\pgfqpoint{2.246646in}{1.231955in}}{\pgfqpoint{2.254546in}{1.228683in}}{\pgfqpoint{2.262783in}{1.228683in}}%
\pgfpathclose%
\pgfusepath{stroke,fill}%
\end{pgfscope}%
\begin{pgfscope}%
\pgfpathrectangle{\pgfqpoint{0.100000in}{0.212622in}}{\pgfqpoint{3.696000in}{3.696000in}}%
\pgfusepath{clip}%
\pgfsetbuttcap%
\pgfsetroundjoin%
\definecolor{currentfill}{rgb}{0.121569,0.466667,0.705882}%
\pgfsetfillcolor{currentfill}%
\pgfsetfillopacity{0.767145}%
\pgfsetlinewidth{1.003750pt}%
\definecolor{currentstroke}{rgb}{0.121569,0.466667,0.705882}%
\pgfsetstrokecolor{currentstroke}%
\pgfsetstrokeopacity{0.767145}%
\pgfsetdash{}{0pt}%
\pgfpathmoveto{\pgfqpoint{2.268784in}{1.226980in}}%
\pgfpathcurveto{\pgfqpoint{2.277020in}{1.226980in}}{\pgfqpoint{2.284920in}{1.230252in}}{\pgfqpoint{2.290744in}{1.236076in}}%
\pgfpathcurveto{\pgfqpoint{2.296568in}{1.241900in}}{\pgfqpoint{2.299840in}{1.249800in}}{\pgfqpoint{2.299840in}{1.258036in}}%
\pgfpathcurveto{\pgfqpoint{2.299840in}{1.266273in}}{\pgfqpoint{2.296568in}{1.274173in}}{\pgfqpoint{2.290744in}{1.279997in}}%
\pgfpathcurveto{\pgfqpoint{2.284920in}{1.285821in}}{\pgfqpoint{2.277020in}{1.289093in}}{\pgfqpoint{2.268784in}{1.289093in}}%
\pgfpathcurveto{\pgfqpoint{2.260547in}{1.289093in}}{\pgfqpoint{2.252647in}{1.285821in}}{\pgfqpoint{2.246823in}{1.279997in}}%
\pgfpathcurveto{\pgfqpoint{2.241000in}{1.274173in}}{\pgfqpoint{2.237727in}{1.266273in}}{\pgfqpoint{2.237727in}{1.258036in}}%
\pgfpathcurveto{\pgfqpoint{2.237727in}{1.249800in}}{\pgfqpoint{2.241000in}{1.241900in}}{\pgfqpoint{2.246823in}{1.236076in}}%
\pgfpathcurveto{\pgfqpoint{2.252647in}{1.230252in}}{\pgfqpoint{2.260547in}{1.226980in}}{\pgfqpoint{2.268784in}{1.226980in}}%
\pgfpathclose%
\pgfusepath{stroke,fill}%
\end{pgfscope}%
\begin{pgfscope}%
\pgfpathrectangle{\pgfqpoint{0.100000in}{0.212622in}}{\pgfqpoint{3.696000in}{3.696000in}}%
\pgfusepath{clip}%
\pgfsetbuttcap%
\pgfsetroundjoin%
\definecolor{currentfill}{rgb}{0.121569,0.466667,0.705882}%
\pgfsetfillcolor{currentfill}%
\pgfsetfillopacity{0.770025}%
\pgfsetlinewidth{1.003750pt}%
\definecolor{currentstroke}{rgb}{0.121569,0.466667,0.705882}%
\pgfsetstrokecolor{currentstroke}%
\pgfsetstrokeopacity{0.770025}%
\pgfsetdash{}{0pt}%
\pgfpathmoveto{\pgfqpoint{2.275635in}{1.224551in}}%
\pgfpathcurveto{\pgfqpoint{2.283871in}{1.224551in}}{\pgfqpoint{2.291771in}{1.227823in}}{\pgfqpoint{2.297595in}{1.233647in}}%
\pgfpathcurveto{\pgfqpoint{2.303419in}{1.239471in}}{\pgfqpoint{2.306691in}{1.247371in}}{\pgfqpoint{2.306691in}{1.255608in}}%
\pgfpathcurveto{\pgfqpoint{2.306691in}{1.263844in}}{\pgfqpoint{2.303419in}{1.271744in}}{\pgfqpoint{2.297595in}{1.277568in}}%
\pgfpathcurveto{\pgfqpoint{2.291771in}{1.283392in}}{\pgfqpoint{2.283871in}{1.286664in}}{\pgfqpoint{2.275635in}{1.286664in}}%
\pgfpathcurveto{\pgfqpoint{2.267398in}{1.286664in}}{\pgfqpoint{2.259498in}{1.283392in}}{\pgfqpoint{2.253674in}{1.277568in}}%
\pgfpathcurveto{\pgfqpoint{2.247851in}{1.271744in}}{\pgfqpoint{2.244578in}{1.263844in}}{\pgfqpoint{2.244578in}{1.255608in}}%
\pgfpathcurveto{\pgfqpoint{2.244578in}{1.247371in}}{\pgfqpoint{2.247851in}{1.239471in}}{\pgfqpoint{2.253674in}{1.233647in}}%
\pgfpathcurveto{\pgfqpoint{2.259498in}{1.227823in}}{\pgfqpoint{2.267398in}{1.224551in}}{\pgfqpoint{2.275635in}{1.224551in}}%
\pgfpathclose%
\pgfusepath{stroke,fill}%
\end{pgfscope}%
\begin{pgfscope}%
\pgfpathrectangle{\pgfqpoint{0.100000in}{0.212622in}}{\pgfqpoint{3.696000in}{3.696000in}}%
\pgfusepath{clip}%
\pgfsetbuttcap%
\pgfsetroundjoin%
\definecolor{currentfill}{rgb}{0.121569,0.466667,0.705882}%
\pgfsetfillcolor{currentfill}%
\pgfsetfillopacity{0.773022}%
\pgfsetlinewidth{1.003750pt}%
\definecolor{currentstroke}{rgb}{0.121569,0.466667,0.705882}%
\pgfsetstrokecolor{currentstroke}%
\pgfsetstrokeopacity{0.773022}%
\pgfsetdash{}{0pt}%
\pgfpathmoveto{\pgfqpoint{2.283768in}{1.222193in}}%
\pgfpathcurveto{\pgfqpoint{2.292005in}{1.222193in}}{\pgfqpoint{2.299905in}{1.225465in}}{\pgfqpoint{2.305729in}{1.231289in}}%
\pgfpathcurveto{\pgfqpoint{2.311553in}{1.237113in}}{\pgfqpoint{2.314825in}{1.245013in}}{\pgfqpoint{2.314825in}{1.253249in}}%
\pgfpathcurveto{\pgfqpoint{2.314825in}{1.261485in}}{\pgfqpoint{2.311553in}{1.269386in}}{\pgfqpoint{2.305729in}{1.275209in}}%
\pgfpathcurveto{\pgfqpoint{2.299905in}{1.281033in}}{\pgfqpoint{2.292005in}{1.284306in}}{\pgfqpoint{2.283768in}{1.284306in}}%
\pgfpathcurveto{\pgfqpoint{2.275532in}{1.284306in}}{\pgfqpoint{2.267632in}{1.281033in}}{\pgfqpoint{2.261808in}{1.275209in}}%
\pgfpathcurveto{\pgfqpoint{2.255984in}{1.269386in}}{\pgfqpoint{2.252712in}{1.261485in}}{\pgfqpoint{2.252712in}{1.253249in}}%
\pgfpathcurveto{\pgfqpoint{2.252712in}{1.245013in}}{\pgfqpoint{2.255984in}{1.237113in}}{\pgfqpoint{2.261808in}{1.231289in}}%
\pgfpathcurveto{\pgfqpoint{2.267632in}{1.225465in}}{\pgfqpoint{2.275532in}{1.222193in}}{\pgfqpoint{2.283768in}{1.222193in}}%
\pgfpathclose%
\pgfusepath{stroke,fill}%
\end{pgfscope}%
\begin{pgfscope}%
\pgfpathrectangle{\pgfqpoint{0.100000in}{0.212622in}}{\pgfqpoint{3.696000in}{3.696000in}}%
\pgfusepath{clip}%
\pgfsetbuttcap%
\pgfsetroundjoin%
\definecolor{currentfill}{rgb}{0.121569,0.466667,0.705882}%
\pgfsetfillcolor{currentfill}%
\pgfsetfillopacity{0.776024}%
\pgfsetlinewidth{1.003750pt}%
\definecolor{currentstroke}{rgb}{0.121569,0.466667,0.705882}%
\pgfsetstrokecolor{currentstroke}%
\pgfsetstrokeopacity{0.776024}%
\pgfsetdash{}{0pt}%
\pgfpathmoveto{\pgfqpoint{2.293716in}{1.219951in}}%
\pgfpathcurveto{\pgfqpoint{2.301953in}{1.219951in}}{\pgfqpoint{2.309853in}{1.223224in}}{\pgfqpoint{2.315677in}{1.229048in}}%
\pgfpathcurveto{\pgfqpoint{2.321501in}{1.234871in}}{\pgfqpoint{2.324773in}{1.242772in}}{\pgfqpoint{2.324773in}{1.251008in}}%
\pgfpathcurveto{\pgfqpoint{2.324773in}{1.259244in}}{\pgfqpoint{2.321501in}{1.267144in}}{\pgfqpoint{2.315677in}{1.272968in}}%
\pgfpathcurveto{\pgfqpoint{2.309853in}{1.278792in}}{\pgfqpoint{2.301953in}{1.282064in}}{\pgfqpoint{2.293716in}{1.282064in}}%
\pgfpathcurveto{\pgfqpoint{2.285480in}{1.282064in}}{\pgfqpoint{2.277580in}{1.278792in}}{\pgfqpoint{2.271756in}{1.272968in}}%
\pgfpathcurveto{\pgfqpoint{2.265932in}{1.267144in}}{\pgfqpoint{2.262660in}{1.259244in}}{\pgfqpoint{2.262660in}{1.251008in}}%
\pgfpathcurveto{\pgfqpoint{2.262660in}{1.242772in}}{\pgfqpoint{2.265932in}{1.234871in}}{\pgfqpoint{2.271756in}{1.229048in}}%
\pgfpathcurveto{\pgfqpoint{2.277580in}{1.223224in}}{\pgfqpoint{2.285480in}{1.219951in}}{\pgfqpoint{2.293716in}{1.219951in}}%
\pgfpathclose%
\pgfusepath{stroke,fill}%
\end{pgfscope}%
\begin{pgfscope}%
\pgfpathrectangle{\pgfqpoint{0.100000in}{0.212622in}}{\pgfqpoint{3.696000in}{3.696000in}}%
\pgfusepath{clip}%
\pgfsetbuttcap%
\pgfsetroundjoin%
\definecolor{currentfill}{rgb}{0.121569,0.466667,0.705882}%
\pgfsetfillcolor{currentfill}%
\pgfsetfillopacity{0.779285}%
\pgfsetlinewidth{1.003750pt}%
\definecolor{currentstroke}{rgb}{0.121569,0.466667,0.705882}%
\pgfsetstrokecolor{currentstroke}%
\pgfsetstrokeopacity{0.779285}%
\pgfsetdash{}{0pt}%
\pgfpathmoveto{\pgfqpoint{2.304100in}{1.217423in}}%
\pgfpathcurveto{\pgfqpoint{2.312336in}{1.217423in}}{\pgfqpoint{2.320236in}{1.220695in}}{\pgfqpoint{2.326060in}{1.226519in}}%
\pgfpathcurveto{\pgfqpoint{2.331884in}{1.232343in}}{\pgfqpoint{2.335156in}{1.240243in}}{\pgfqpoint{2.335156in}{1.248479in}}%
\pgfpathcurveto{\pgfqpoint{2.335156in}{1.256715in}}{\pgfqpoint{2.331884in}{1.264615in}}{\pgfqpoint{2.326060in}{1.270439in}}%
\pgfpathcurveto{\pgfqpoint{2.320236in}{1.276263in}}{\pgfqpoint{2.312336in}{1.279536in}}{\pgfqpoint{2.304100in}{1.279536in}}%
\pgfpathcurveto{\pgfqpoint{2.295863in}{1.279536in}}{\pgfqpoint{2.287963in}{1.276263in}}{\pgfqpoint{2.282139in}{1.270439in}}%
\pgfpathcurveto{\pgfqpoint{2.276316in}{1.264615in}}{\pgfqpoint{2.273043in}{1.256715in}}{\pgfqpoint{2.273043in}{1.248479in}}%
\pgfpathcurveto{\pgfqpoint{2.273043in}{1.240243in}}{\pgfqpoint{2.276316in}{1.232343in}}{\pgfqpoint{2.282139in}{1.226519in}}%
\pgfpathcurveto{\pgfqpoint{2.287963in}{1.220695in}}{\pgfqpoint{2.295863in}{1.217423in}}{\pgfqpoint{2.304100in}{1.217423in}}%
\pgfpathclose%
\pgfusepath{stroke,fill}%
\end{pgfscope}%
\begin{pgfscope}%
\pgfpathrectangle{\pgfqpoint{0.100000in}{0.212622in}}{\pgfqpoint{3.696000in}{3.696000in}}%
\pgfusepath{clip}%
\pgfsetbuttcap%
\pgfsetroundjoin%
\definecolor{currentfill}{rgb}{0.121569,0.466667,0.705882}%
\pgfsetfillcolor{currentfill}%
\pgfsetfillopacity{0.781243}%
\pgfsetlinewidth{1.003750pt}%
\definecolor{currentstroke}{rgb}{0.121569,0.466667,0.705882}%
\pgfsetstrokecolor{currentstroke}%
\pgfsetstrokeopacity{0.781243}%
\pgfsetdash{}{0pt}%
\pgfpathmoveto{\pgfqpoint{2.309710in}{1.216012in}}%
\pgfpathcurveto{\pgfqpoint{2.317946in}{1.216012in}}{\pgfqpoint{2.325846in}{1.219285in}}{\pgfqpoint{2.331670in}{1.225109in}}%
\pgfpathcurveto{\pgfqpoint{2.337494in}{1.230933in}}{\pgfqpoint{2.340767in}{1.238833in}}{\pgfqpoint{2.340767in}{1.247069in}}%
\pgfpathcurveto{\pgfqpoint{2.340767in}{1.255305in}}{\pgfqpoint{2.337494in}{1.263205in}}{\pgfqpoint{2.331670in}{1.269029in}}%
\pgfpathcurveto{\pgfqpoint{2.325846in}{1.274853in}}{\pgfqpoint{2.317946in}{1.278125in}}{\pgfqpoint{2.309710in}{1.278125in}}%
\pgfpathcurveto{\pgfqpoint{2.301474in}{1.278125in}}{\pgfqpoint{2.293574in}{1.274853in}}{\pgfqpoint{2.287750in}{1.269029in}}%
\pgfpathcurveto{\pgfqpoint{2.281926in}{1.263205in}}{\pgfqpoint{2.278654in}{1.255305in}}{\pgfqpoint{2.278654in}{1.247069in}}%
\pgfpathcurveto{\pgfqpoint{2.278654in}{1.238833in}}{\pgfqpoint{2.281926in}{1.230933in}}{\pgfqpoint{2.287750in}{1.225109in}}%
\pgfpathcurveto{\pgfqpoint{2.293574in}{1.219285in}}{\pgfqpoint{2.301474in}{1.216012in}}{\pgfqpoint{2.309710in}{1.216012in}}%
\pgfpathclose%
\pgfusepath{stroke,fill}%
\end{pgfscope}%
\begin{pgfscope}%
\pgfpathrectangle{\pgfqpoint{0.100000in}{0.212622in}}{\pgfqpoint{3.696000in}{3.696000in}}%
\pgfusepath{clip}%
\pgfsetbuttcap%
\pgfsetroundjoin%
\definecolor{currentfill}{rgb}{0.121569,0.466667,0.705882}%
\pgfsetfillcolor{currentfill}%
\pgfsetfillopacity{0.782436}%
\pgfsetlinewidth{1.003750pt}%
\definecolor{currentstroke}{rgb}{0.121569,0.466667,0.705882}%
\pgfsetstrokecolor{currentstroke}%
\pgfsetstrokeopacity{0.782436}%
\pgfsetdash{}{0pt}%
\pgfpathmoveto{\pgfqpoint{2.312650in}{1.214967in}}%
\pgfpathcurveto{\pgfqpoint{2.320886in}{1.214967in}}{\pgfqpoint{2.328786in}{1.218240in}}{\pgfqpoint{2.334610in}{1.224064in}}%
\pgfpathcurveto{\pgfqpoint{2.340434in}{1.229888in}}{\pgfqpoint{2.343707in}{1.237788in}}{\pgfqpoint{2.343707in}{1.246024in}}%
\pgfpathcurveto{\pgfqpoint{2.343707in}{1.254260in}}{\pgfqpoint{2.340434in}{1.262160in}}{\pgfqpoint{2.334610in}{1.267984in}}%
\pgfpathcurveto{\pgfqpoint{2.328786in}{1.273808in}}{\pgfqpoint{2.320886in}{1.277080in}}{\pgfqpoint{2.312650in}{1.277080in}}%
\pgfpathcurveto{\pgfqpoint{2.304414in}{1.277080in}}{\pgfqpoint{2.296514in}{1.273808in}}{\pgfqpoint{2.290690in}{1.267984in}}%
\pgfpathcurveto{\pgfqpoint{2.284866in}{1.262160in}}{\pgfqpoint{2.281594in}{1.254260in}}{\pgfqpoint{2.281594in}{1.246024in}}%
\pgfpathcurveto{\pgfqpoint{2.281594in}{1.237788in}}{\pgfqpoint{2.284866in}{1.229888in}}{\pgfqpoint{2.290690in}{1.224064in}}%
\pgfpathcurveto{\pgfqpoint{2.296514in}{1.218240in}}{\pgfqpoint{2.304414in}{1.214967in}}{\pgfqpoint{2.312650in}{1.214967in}}%
\pgfpathclose%
\pgfusepath{stroke,fill}%
\end{pgfscope}%
\begin{pgfscope}%
\pgfpathrectangle{\pgfqpoint{0.100000in}{0.212622in}}{\pgfqpoint{3.696000in}{3.696000in}}%
\pgfusepath{clip}%
\pgfsetbuttcap%
\pgfsetroundjoin%
\definecolor{currentfill}{rgb}{0.121569,0.466667,0.705882}%
\pgfsetfillcolor{currentfill}%
\pgfsetfillopacity{0.783799}%
\pgfsetlinewidth{1.003750pt}%
\definecolor{currentstroke}{rgb}{0.121569,0.466667,0.705882}%
\pgfsetstrokecolor{currentstroke}%
\pgfsetstrokeopacity{0.783799}%
\pgfsetdash{}{0pt}%
\pgfpathmoveto{\pgfqpoint{2.316309in}{1.213700in}}%
\pgfpathcurveto{\pgfqpoint{2.324545in}{1.213700in}}{\pgfqpoint{2.332445in}{1.216972in}}{\pgfqpoint{2.338269in}{1.222796in}}%
\pgfpathcurveto{\pgfqpoint{2.344093in}{1.228620in}}{\pgfqpoint{2.347365in}{1.236520in}}{\pgfqpoint{2.347365in}{1.244757in}}%
\pgfpathcurveto{\pgfqpoint{2.347365in}{1.252993in}}{\pgfqpoint{2.344093in}{1.260893in}}{\pgfqpoint{2.338269in}{1.266717in}}%
\pgfpathcurveto{\pgfqpoint{2.332445in}{1.272541in}}{\pgfqpoint{2.324545in}{1.275813in}}{\pgfqpoint{2.316309in}{1.275813in}}%
\pgfpathcurveto{\pgfqpoint{2.308072in}{1.275813in}}{\pgfqpoint{2.300172in}{1.272541in}}{\pgfqpoint{2.294348in}{1.266717in}}%
\pgfpathcurveto{\pgfqpoint{2.288524in}{1.260893in}}{\pgfqpoint{2.285252in}{1.252993in}}{\pgfqpoint{2.285252in}{1.244757in}}%
\pgfpathcurveto{\pgfqpoint{2.285252in}{1.236520in}}{\pgfqpoint{2.288524in}{1.228620in}}{\pgfqpoint{2.294348in}{1.222796in}}%
\pgfpathcurveto{\pgfqpoint{2.300172in}{1.216972in}}{\pgfqpoint{2.308072in}{1.213700in}}{\pgfqpoint{2.316309in}{1.213700in}}%
\pgfpathclose%
\pgfusepath{stroke,fill}%
\end{pgfscope}%
\begin{pgfscope}%
\pgfpathrectangle{\pgfqpoint{0.100000in}{0.212622in}}{\pgfqpoint{3.696000in}{3.696000in}}%
\pgfusepath{clip}%
\pgfsetbuttcap%
\pgfsetroundjoin%
\definecolor{currentfill}{rgb}{0.121569,0.466667,0.705882}%
\pgfsetfillcolor{currentfill}%
\pgfsetfillopacity{0.785154}%
\pgfsetlinewidth{1.003750pt}%
\definecolor{currentstroke}{rgb}{0.121569,0.466667,0.705882}%
\pgfsetstrokecolor{currentstroke}%
\pgfsetstrokeopacity{0.785154}%
\pgfsetdash{}{0pt}%
\pgfpathmoveto{\pgfqpoint{2.320623in}{1.212594in}}%
\pgfpathcurveto{\pgfqpoint{2.328859in}{1.212594in}}{\pgfqpoint{2.336759in}{1.215866in}}{\pgfqpoint{2.342583in}{1.221690in}}%
\pgfpathcurveto{\pgfqpoint{2.348407in}{1.227514in}}{\pgfqpoint{2.351679in}{1.235414in}}{\pgfqpoint{2.351679in}{1.243650in}}%
\pgfpathcurveto{\pgfqpoint{2.351679in}{1.251887in}}{\pgfqpoint{2.348407in}{1.259787in}}{\pgfqpoint{2.342583in}{1.265611in}}%
\pgfpathcurveto{\pgfqpoint{2.336759in}{1.271435in}}{\pgfqpoint{2.328859in}{1.274707in}}{\pgfqpoint{2.320623in}{1.274707in}}%
\pgfpathcurveto{\pgfqpoint{2.312386in}{1.274707in}}{\pgfqpoint{2.304486in}{1.271435in}}{\pgfqpoint{2.298662in}{1.265611in}}%
\pgfpathcurveto{\pgfqpoint{2.292839in}{1.259787in}}{\pgfqpoint{2.289566in}{1.251887in}}{\pgfqpoint{2.289566in}{1.243650in}}%
\pgfpathcurveto{\pgfqpoint{2.289566in}{1.235414in}}{\pgfqpoint{2.292839in}{1.227514in}}{\pgfqpoint{2.298662in}{1.221690in}}%
\pgfpathcurveto{\pgfqpoint{2.304486in}{1.215866in}}{\pgfqpoint{2.312386in}{1.212594in}}{\pgfqpoint{2.320623in}{1.212594in}}%
\pgfpathclose%
\pgfusepath{stroke,fill}%
\end{pgfscope}%
\begin{pgfscope}%
\pgfpathrectangle{\pgfqpoint{0.100000in}{0.212622in}}{\pgfqpoint{3.696000in}{3.696000in}}%
\pgfusepath{clip}%
\pgfsetbuttcap%
\pgfsetroundjoin%
\definecolor{currentfill}{rgb}{0.121569,0.466667,0.705882}%
\pgfsetfillcolor{currentfill}%
\pgfsetfillopacity{0.787297}%
\pgfsetlinewidth{1.003750pt}%
\definecolor{currentstroke}{rgb}{0.121569,0.466667,0.705882}%
\pgfsetstrokecolor{currentstroke}%
\pgfsetstrokeopacity{0.787297}%
\pgfsetdash{}{0pt}%
\pgfpathmoveto{\pgfqpoint{2.326371in}{1.210824in}}%
\pgfpathcurveto{\pgfqpoint{2.334607in}{1.210824in}}{\pgfqpoint{2.342507in}{1.214096in}}{\pgfqpoint{2.348331in}{1.219920in}}%
\pgfpathcurveto{\pgfqpoint{2.354155in}{1.225744in}}{\pgfqpoint{2.357427in}{1.233644in}}{\pgfqpoint{2.357427in}{1.241880in}}%
\pgfpathcurveto{\pgfqpoint{2.357427in}{1.250117in}}{\pgfqpoint{2.354155in}{1.258017in}}{\pgfqpoint{2.348331in}{1.263840in}}%
\pgfpathcurveto{\pgfqpoint{2.342507in}{1.269664in}}{\pgfqpoint{2.334607in}{1.272937in}}{\pgfqpoint{2.326371in}{1.272937in}}%
\pgfpathcurveto{\pgfqpoint{2.318135in}{1.272937in}}{\pgfqpoint{2.310235in}{1.269664in}}{\pgfqpoint{2.304411in}{1.263840in}}%
\pgfpathcurveto{\pgfqpoint{2.298587in}{1.258017in}}{\pgfqpoint{2.295314in}{1.250117in}}{\pgfqpoint{2.295314in}{1.241880in}}%
\pgfpathcurveto{\pgfqpoint{2.295314in}{1.233644in}}{\pgfqpoint{2.298587in}{1.225744in}}{\pgfqpoint{2.304411in}{1.219920in}}%
\pgfpathcurveto{\pgfqpoint{2.310235in}{1.214096in}}{\pgfqpoint{2.318135in}{1.210824in}}{\pgfqpoint{2.326371in}{1.210824in}}%
\pgfpathclose%
\pgfusepath{stroke,fill}%
\end{pgfscope}%
\begin{pgfscope}%
\pgfpathrectangle{\pgfqpoint{0.100000in}{0.212622in}}{\pgfqpoint{3.696000in}{3.696000in}}%
\pgfusepath{clip}%
\pgfsetbuttcap%
\pgfsetroundjoin%
\definecolor{currentfill}{rgb}{0.121569,0.466667,0.705882}%
\pgfsetfillcolor{currentfill}%
\pgfsetfillopacity{0.789862}%
\pgfsetlinewidth{1.003750pt}%
\definecolor{currentstroke}{rgb}{0.121569,0.466667,0.705882}%
\pgfsetstrokecolor{currentstroke}%
\pgfsetstrokeopacity{0.789862}%
\pgfsetdash{}{0pt}%
\pgfpathmoveto{\pgfqpoint{2.333345in}{1.208716in}}%
\pgfpathcurveto{\pgfqpoint{2.341581in}{1.208716in}}{\pgfqpoint{2.349481in}{1.211988in}}{\pgfqpoint{2.355305in}{1.217812in}}%
\pgfpathcurveto{\pgfqpoint{2.361129in}{1.223636in}}{\pgfqpoint{2.364401in}{1.231536in}}{\pgfqpoint{2.364401in}{1.239772in}}%
\pgfpathcurveto{\pgfqpoint{2.364401in}{1.248008in}}{\pgfqpoint{2.361129in}{1.255908in}}{\pgfqpoint{2.355305in}{1.261732in}}%
\pgfpathcurveto{\pgfqpoint{2.349481in}{1.267556in}}{\pgfqpoint{2.341581in}{1.270829in}}{\pgfqpoint{2.333345in}{1.270829in}}%
\pgfpathcurveto{\pgfqpoint{2.325108in}{1.270829in}}{\pgfqpoint{2.317208in}{1.267556in}}{\pgfqpoint{2.311384in}{1.261732in}}%
\pgfpathcurveto{\pgfqpoint{2.305560in}{1.255908in}}{\pgfqpoint{2.302288in}{1.248008in}}{\pgfqpoint{2.302288in}{1.239772in}}%
\pgfpathcurveto{\pgfqpoint{2.302288in}{1.231536in}}{\pgfqpoint{2.305560in}{1.223636in}}{\pgfqpoint{2.311384in}{1.217812in}}%
\pgfpathcurveto{\pgfqpoint{2.317208in}{1.211988in}}{\pgfqpoint{2.325108in}{1.208716in}}{\pgfqpoint{2.333345in}{1.208716in}}%
\pgfpathclose%
\pgfusepath{stroke,fill}%
\end{pgfscope}%
\begin{pgfscope}%
\pgfpathrectangle{\pgfqpoint{0.100000in}{0.212622in}}{\pgfqpoint{3.696000in}{3.696000in}}%
\pgfusepath{clip}%
\pgfsetbuttcap%
\pgfsetroundjoin%
\definecolor{currentfill}{rgb}{0.121569,0.466667,0.705882}%
\pgfsetfillcolor{currentfill}%
\pgfsetfillopacity{0.792350}%
\pgfsetlinewidth{1.003750pt}%
\definecolor{currentstroke}{rgb}{0.121569,0.466667,0.705882}%
\pgfsetstrokecolor{currentstroke}%
\pgfsetstrokeopacity{0.792350}%
\pgfsetdash{}{0pt}%
\pgfpathmoveto{\pgfqpoint{2.341534in}{1.206803in}}%
\pgfpathcurveto{\pgfqpoint{2.349770in}{1.206803in}}{\pgfqpoint{2.357670in}{1.210075in}}{\pgfqpoint{2.363494in}{1.215899in}}%
\pgfpathcurveto{\pgfqpoint{2.369318in}{1.221723in}}{\pgfqpoint{2.372590in}{1.229623in}}{\pgfqpoint{2.372590in}{1.237859in}}%
\pgfpathcurveto{\pgfqpoint{2.372590in}{1.246095in}}{\pgfqpoint{2.369318in}{1.253996in}}{\pgfqpoint{2.363494in}{1.259819in}}%
\pgfpathcurveto{\pgfqpoint{2.357670in}{1.265643in}}{\pgfqpoint{2.349770in}{1.268916in}}{\pgfqpoint{2.341534in}{1.268916in}}%
\pgfpathcurveto{\pgfqpoint{2.333297in}{1.268916in}}{\pgfqpoint{2.325397in}{1.265643in}}{\pgfqpoint{2.319573in}{1.259819in}}%
\pgfpathcurveto{\pgfqpoint{2.313749in}{1.253996in}}{\pgfqpoint{2.310477in}{1.246095in}}{\pgfqpoint{2.310477in}{1.237859in}}%
\pgfpathcurveto{\pgfqpoint{2.310477in}{1.229623in}}{\pgfqpoint{2.313749in}{1.221723in}}{\pgfqpoint{2.319573in}{1.215899in}}%
\pgfpathcurveto{\pgfqpoint{2.325397in}{1.210075in}}{\pgfqpoint{2.333297in}{1.206803in}}{\pgfqpoint{2.341534in}{1.206803in}}%
\pgfpathclose%
\pgfusepath{stroke,fill}%
\end{pgfscope}%
\begin{pgfscope}%
\pgfpathrectangle{\pgfqpoint{0.100000in}{0.212622in}}{\pgfqpoint{3.696000in}{3.696000in}}%
\pgfusepath{clip}%
\pgfsetbuttcap%
\pgfsetroundjoin%
\definecolor{currentfill}{rgb}{0.121569,0.466667,0.705882}%
\pgfsetfillcolor{currentfill}%
\pgfsetfillopacity{0.794198}%
\pgfsetlinewidth{1.003750pt}%
\definecolor{currentstroke}{rgb}{0.121569,0.466667,0.705882}%
\pgfsetstrokecolor{currentstroke}%
\pgfsetstrokeopacity{0.794198}%
\pgfsetdash{}{0pt}%
\pgfpathmoveto{\pgfqpoint{2.345603in}{1.205223in}}%
\pgfpathcurveto{\pgfqpoint{2.353839in}{1.205223in}}{\pgfqpoint{2.361739in}{1.208495in}}{\pgfqpoint{2.367563in}{1.214319in}}%
\pgfpathcurveto{\pgfqpoint{2.373387in}{1.220143in}}{\pgfqpoint{2.376659in}{1.228043in}}{\pgfqpoint{2.376659in}{1.236279in}}%
\pgfpathcurveto{\pgfqpoint{2.376659in}{1.244516in}}{\pgfqpoint{2.373387in}{1.252416in}}{\pgfqpoint{2.367563in}{1.258240in}}%
\pgfpathcurveto{\pgfqpoint{2.361739in}{1.264064in}}{\pgfqpoint{2.353839in}{1.267336in}}{\pgfqpoint{2.345603in}{1.267336in}}%
\pgfpathcurveto{\pgfqpoint{2.337366in}{1.267336in}}{\pgfqpoint{2.329466in}{1.264064in}}{\pgfqpoint{2.323642in}{1.258240in}}%
\pgfpathcurveto{\pgfqpoint{2.317818in}{1.252416in}}{\pgfqpoint{2.314546in}{1.244516in}}{\pgfqpoint{2.314546in}{1.236279in}}%
\pgfpathcurveto{\pgfqpoint{2.314546in}{1.228043in}}{\pgfqpoint{2.317818in}{1.220143in}}{\pgfqpoint{2.323642in}{1.214319in}}%
\pgfpathcurveto{\pgfqpoint{2.329466in}{1.208495in}}{\pgfqpoint{2.337366in}{1.205223in}}{\pgfqpoint{2.345603in}{1.205223in}}%
\pgfpathclose%
\pgfusepath{stroke,fill}%
\end{pgfscope}%
\begin{pgfscope}%
\pgfpathrectangle{\pgfqpoint{0.100000in}{0.212622in}}{\pgfqpoint{3.696000in}{3.696000in}}%
\pgfusepath{clip}%
\pgfsetbuttcap%
\pgfsetroundjoin%
\definecolor{currentfill}{rgb}{0.121569,0.466667,0.705882}%
\pgfsetfillcolor{currentfill}%
\pgfsetfillopacity{0.794966}%
\pgfsetlinewidth{1.003750pt}%
\definecolor{currentstroke}{rgb}{0.121569,0.466667,0.705882}%
\pgfsetstrokecolor{currentstroke}%
\pgfsetstrokeopacity{0.794966}%
\pgfsetdash{}{0pt}%
\pgfpathmoveto{\pgfqpoint{2.348065in}{1.204620in}}%
\pgfpathcurveto{\pgfqpoint{2.356301in}{1.204620in}}{\pgfqpoint{2.364201in}{1.207892in}}{\pgfqpoint{2.370025in}{1.213716in}}%
\pgfpathcurveto{\pgfqpoint{2.375849in}{1.219540in}}{\pgfqpoint{2.379121in}{1.227440in}}{\pgfqpoint{2.379121in}{1.235676in}}%
\pgfpathcurveto{\pgfqpoint{2.379121in}{1.243913in}}{\pgfqpoint{2.375849in}{1.251813in}}{\pgfqpoint{2.370025in}{1.257637in}}%
\pgfpathcurveto{\pgfqpoint{2.364201in}{1.263461in}}{\pgfqpoint{2.356301in}{1.266733in}}{\pgfqpoint{2.348065in}{1.266733in}}%
\pgfpathcurveto{\pgfqpoint{2.339829in}{1.266733in}}{\pgfqpoint{2.331929in}{1.263461in}}{\pgfqpoint{2.326105in}{1.257637in}}%
\pgfpathcurveto{\pgfqpoint{2.320281in}{1.251813in}}{\pgfqpoint{2.317008in}{1.243913in}}{\pgfqpoint{2.317008in}{1.235676in}}%
\pgfpathcurveto{\pgfqpoint{2.317008in}{1.227440in}}{\pgfqpoint{2.320281in}{1.219540in}}{\pgfqpoint{2.326105in}{1.213716in}}%
\pgfpathcurveto{\pgfqpoint{2.331929in}{1.207892in}}{\pgfqpoint{2.339829in}{1.204620in}}{\pgfqpoint{2.348065in}{1.204620in}}%
\pgfpathclose%
\pgfusepath{stroke,fill}%
\end{pgfscope}%
\begin{pgfscope}%
\pgfpathrectangle{\pgfqpoint{0.100000in}{0.212622in}}{\pgfqpoint{3.696000in}{3.696000in}}%
\pgfusepath{clip}%
\pgfsetbuttcap%
\pgfsetroundjoin%
\definecolor{currentfill}{rgb}{0.121569,0.466667,0.705882}%
\pgfsetfillcolor{currentfill}%
\pgfsetfillopacity{0.795454}%
\pgfsetlinewidth{1.003750pt}%
\definecolor{currentstroke}{rgb}{0.121569,0.466667,0.705882}%
\pgfsetstrokecolor{currentstroke}%
\pgfsetstrokeopacity{0.795454}%
\pgfsetdash{}{0pt}%
\pgfpathmoveto{\pgfqpoint{2.349361in}{1.204219in}}%
\pgfpathcurveto{\pgfqpoint{2.357597in}{1.204219in}}{\pgfqpoint{2.365497in}{1.207491in}}{\pgfqpoint{2.371321in}{1.213315in}}%
\pgfpathcurveto{\pgfqpoint{2.377145in}{1.219139in}}{\pgfqpoint{2.380417in}{1.227039in}}{\pgfqpoint{2.380417in}{1.235275in}}%
\pgfpathcurveto{\pgfqpoint{2.380417in}{1.243511in}}{\pgfqpoint{2.377145in}{1.251412in}}{\pgfqpoint{2.371321in}{1.257235in}}%
\pgfpathcurveto{\pgfqpoint{2.365497in}{1.263059in}}{\pgfqpoint{2.357597in}{1.266332in}}{\pgfqpoint{2.349361in}{1.266332in}}%
\pgfpathcurveto{\pgfqpoint{2.341125in}{1.266332in}}{\pgfqpoint{2.333224in}{1.263059in}}{\pgfqpoint{2.327401in}{1.257235in}}%
\pgfpathcurveto{\pgfqpoint{2.321577in}{1.251412in}}{\pgfqpoint{2.318304in}{1.243511in}}{\pgfqpoint{2.318304in}{1.235275in}}%
\pgfpathcurveto{\pgfqpoint{2.318304in}{1.227039in}}{\pgfqpoint{2.321577in}{1.219139in}}{\pgfqpoint{2.327401in}{1.213315in}}%
\pgfpathcurveto{\pgfqpoint{2.333224in}{1.207491in}}{\pgfqpoint{2.341125in}{1.204219in}}{\pgfqpoint{2.349361in}{1.204219in}}%
\pgfpathclose%
\pgfusepath{stroke,fill}%
\end{pgfscope}%
\begin{pgfscope}%
\pgfpathrectangle{\pgfqpoint{0.100000in}{0.212622in}}{\pgfqpoint{3.696000in}{3.696000in}}%
\pgfusepath{clip}%
\pgfsetbuttcap%
\pgfsetroundjoin%
\definecolor{currentfill}{rgb}{0.121569,0.466667,0.705882}%
\pgfsetfillcolor{currentfill}%
\pgfsetfillopacity{0.795691}%
\pgfsetlinewidth{1.003750pt}%
\definecolor{currentstroke}{rgb}{0.121569,0.466667,0.705882}%
\pgfsetstrokecolor{currentstroke}%
\pgfsetstrokeopacity{0.795691}%
\pgfsetdash{}{0pt}%
\pgfpathmoveto{\pgfqpoint{2.350101in}{1.204029in}}%
\pgfpathcurveto{\pgfqpoint{2.358337in}{1.204029in}}{\pgfqpoint{2.366237in}{1.207301in}}{\pgfqpoint{2.372061in}{1.213125in}}%
\pgfpathcurveto{\pgfqpoint{2.377885in}{1.218949in}}{\pgfqpoint{2.381157in}{1.226849in}}{\pgfqpoint{2.381157in}{1.235085in}}%
\pgfpathcurveto{\pgfqpoint{2.381157in}{1.243321in}}{\pgfqpoint{2.377885in}{1.251221in}}{\pgfqpoint{2.372061in}{1.257045in}}%
\pgfpathcurveto{\pgfqpoint{2.366237in}{1.262869in}}{\pgfqpoint{2.358337in}{1.266142in}}{\pgfqpoint{2.350101in}{1.266142in}}%
\pgfpathcurveto{\pgfqpoint{2.341865in}{1.266142in}}{\pgfqpoint{2.333964in}{1.262869in}}{\pgfqpoint{2.328141in}{1.257045in}}%
\pgfpathcurveto{\pgfqpoint{2.322317in}{1.251221in}}{\pgfqpoint{2.319044in}{1.243321in}}{\pgfqpoint{2.319044in}{1.235085in}}%
\pgfpathcurveto{\pgfqpoint{2.319044in}{1.226849in}}{\pgfqpoint{2.322317in}{1.218949in}}{\pgfqpoint{2.328141in}{1.213125in}}%
\pgfpathcurveto{\pgfqpoint{2.333964in}{1.207301in}}{\pgfqpoint{2.341865in}{1.204029in}}{\pgfqpoint{2.350101in}{1.204029in}}%
\pgfpathclose%
\pgfusepath{stroke,fill}%
\end{pgfscope}%
\begin{pgfscope}%
\pgfpathrectangle{\pgfqpoint{0.100000in}{0.212622in}}{\pgfqpoint{3.696000in}{3.696000in}}%
\pgfusepath{clip}%
\pgfsetbuttcap%
\pgfsetroundjoin%
\definecolor{currentfill}{rgb}{0.121569,0.466667,0.705882}%
\pgfsetfillcolor{currentfill}%
\pgfsetfillopacity{0.796388}%
\pgfsetlinewidth{1.003750pt}%
\definecolor{currentstroke}{rgb}{0.121569,0.466667,0.705882}%
\pgfsetstrokecolor{currentstroke}%
\pgfsetstrokeopacity{0.796388}%
\pgfsetdash{}{0pt}%
\pgfpathmoveto{\pgfqpoint{2.351900in}{1.203453in}}%
\pgfpathcurveto{\pgfqpoint{2.360137in}{1.203453in}}{\pgfqpoint{2.368037in}{1.206726in}}{\pgfqpoint{2.373861in}{1.212550in}}%
\pgfpathcurveto{\pgfqpoint{2.379685in}{1.218374in}}{\pgfqpoint{2.382957in}{1.226274in}}{\pgfqpoint{2.382957in}{1.234510in}}%
\pgfpathcurveto{\pgfqpoint{2.382957in}{1.242746in}}{\pgfqpoint{2.379685in}{1.250646in}}{\pgfqpoint{2.373861in}{1.256470in}}%
\pgfpathcurveto{\pgfqpoint{2.368037in}{1.262294in}}{\pgfqpoint{2.360137in}{1.265566in}}{\pgfqpoint{2.351900in}{1.265566in}}%
\pgfpathcurveto{\pgfqpoint{2.343664in}{1.265566in}}{\pgfqpoint{2.335764in}{1.262294in}}{\pgfqpoint{2.329940in}{1.256470in}}%
\pgfpathcurveto{\pgfqpoint{2.324116in}{1.250646in}}{\pgfqpoint{2.320844in}{1.242746in}}{\pgfqpoint{2.320844in}{1.234510in}}%
\pgfpathcurveto{\pgfqpoint{2.320844in}{1.226274in}}{\pgfqpoint{2.324116in}{1.218374in}}{\pgfqpoint{2.329940in}{1.212550in}}%
\pgfpathcurveto{\pgfqpoint{2.335764in}{1.206726in}}{\pgfqpoint{2.343664in}{1.203453in}}{\pgfqpoint{2.351900in}{1.203453in}}%
\pgfpathclose%
\pgfusepath{stroke,fill}%
\end{pgfscope}%
\begin{pgfscope}%
\pgfpathrectangle{\pgfqpoint{0.100000in}{0.212622in}}{\pgfqpoint{3.696000in}{3.696000in}}%
\pgfusepath{clip}%
\pgfsetbuttcap%
\pgfsetroundjoin%
\definecolor{currentfill}{rgb}{0.121569,0.466667,0.705882}%
\pgfsetfillcolor{currentfill}%
\pgfsetfillopacity{0.797729}%
\pgfsetlinewidth{1.003750pt}%
\definecolor{currentstroke}{rgb}{0.121569,0.466667,0.705882}%
\pgfsetstrokecolor{currentstroke}%
\pgfsetstrokeopacity{0.797729}%
\pgfsetdash{}{0pt}%
\pgfpathmoveto{\pgfqpoint{2.355221in}{1.202296in}}%
\pgfpathcurveto{\pgfqpoint{2.363457in}{1.202296in}}{\pgfqpoint{2.371357in}{1.205568in}}{\pgfqpoint{2.377181in}{1.211392in}}%
\pgfpathcurveto{\pgfqpoint{2.383005in}{1.217216in}}{\pgfqpoint{2.386278in}{1.225116in}}{\pgfqpoint{2.386278in}{1.233352in}}%
\pgfpathcurveto{\pgfqpoint{2.386278in}{1.241589in}}{\pgfqpoint{2.383005in}{1.249489in}}{\pgfqpoint{2.377181in}{1.255313in}}%
\pgfpathcurveto{\pgfqpoint{2.371357in}{1.261137in}}{\pgfqpoint{2.363457in}{1.264409in}}{\pgfqpoint{2.355221in}{1.264409in}}%
\pgfpathcurveto{\pgfqpoint{2.346985in}{1.264409in}}{\pgfqpoint{2.339085in}{1.261137in}}{\pgfqpoint{2.333261in}{1.255313in}}%
\pgfpathcurveto{\pgfqpoint{2.327437in}{1.249489in}}{\pgfqpoint{2.324165in}{1.241589in}}{\pgfqpoint{2.324165in}{1.233352in}}%
\pgfpathcurveto{\pgfqpoint{2.324165in}{1.225116in}}{\pgfqpoint{2.327437in}{1.217216in}}{\pgfqpoint{2.333261in}{1.211392in}}%
\pgfpathcurveto{\pgfqpoint{2.339085in}{1.205568in}}{\pgfqpoint{2.346985in}{1.202296in}}{\pgfqpoint{2.355221in}{1.202296in}}%
\pgfpathclose%
\pgfusepath{stroke,fill}%
\end{pgfscope}%
\begin{pgfscope}%
\pgfpathrectangle{\pgfqpoint{0.100000in}{0.212622in}}{\pgfqpoint{3.696000in}{3.696000in}}%
\pgfusepath{clip}%
\pgfsetbuttcap%
\pgfsetroundjoin%
\definecolor{currentfill}{rgb}{0.121569,0.466667,0.705882}%
\pgfsetfillcolor{currentfill}%
\pgfsetfillopacity{0.799300}%
\pgfsetlinewidth{1.003750pt}%
\definecolor{currentstroke}{rgb}{0.121569,0.466667,0.705882}%
\pgfsetstrokecolor{currentstroke}%
\pgfsetstrokeopacity{0.799300}%
\pgfsetdash{}{0pt}%
\pgfpathmoveto{\pgfqpoint{2.359683in}{1.200987in}}%
\pgfpathcurveto{\pgfqpoint{2.367919in}{1.200987in}}{\pgfqpoint{2.375819in}{1.204260in}}{\pgfqpoint{2.381643in}{1.210084in}}%
\pgfpathcurveto{\pgfqpoint{2.387467in}{1.215908in}}{\pgfqpoint{2.390739in}{1.223808in}}{\pgfqpoint{2.390739in}{1.232044in}}%
\pgfpathcurveto{\pgfqpoint{2.390739in}{1.240280in}}{\pgfqpoint{2.387467in}{1.248180in}}{\pgfqpoint{2.381643in}{1.254004in}}%
\pgfpathcurveto{\pgfqpoint{2.375819in}{1.259828in}}{\pgfqpoint{2.367919in}{1.263100in}}{\pgfqpoint{2.359683in}{1.263100in}}%
\pgfpathcurveto{\pgfqpoint{2.351447in}{1.263100in}}{\pgfqpoint{2.343547in}{1.259828in}}{\pgfqpoint{2.337723in}{1.254004in}}%
\pgfpathcurveto{\pgfqpoint{2.331899in}{1.248180in}}{\pgfqpoint{2.328626in}{1.240280in}}{\pgfqpoint{2.328626in}{1.232044in}}%
\pgfpathcurveto{\pgfqpoint{2.328626in}{1.223808in}}{\pgfqpoint{2.331899in}{1.215908in}}{\pgfqpoint{2.337723in}{1.210084in}}%
\pgfpathcurveto{\pgfqpoint{2.343547in}{1.204260in}}{\pgfqpoint{2.351447in}{1.200987in}}{\pgfqpoint{2.359683in}{1.200987in}}%
\pgfpathclose%
\pgfusepath{stroke,fill}%
\end{pgfscope}%
\begin{pgfscope}%
\pgfpathrectangle{\pgfqpoint{0.100000in}{0.212622in}}{\pgfqpoint{3.696000in}{3.696000in}}%
\pgfusepath{clip}%
\pgfsetbuttcap%
\pgfsetroundjoin%
\definecolor{currentfill}{rgb}{0.121569,0.466667,0.705882}%
\pgfsetfillcolor{currentfill}%
\pgfsetfillopacity{0.800116}%
\pgfsetlinewidth{1.003750pt}%
\definecolor{currentstroke}{rgb}{0.121569,0.466667,0.705882}%
\pgfsetstrokecolor{currentstroke}%
\pgfsetstrokeopacity{0.800116}%
\pgfsetdash{}{0pt}%
\pgfpathmoveto{\pgfqpoint{2.362185in}{1.200334in}}%
\pgfpathcurveto{\pgfqpoint{2.370422in}{1.200334in}}{\pgfqpoint{2.378322in}{1.203607in}}{\pgfqpoint{2.384146in}{1.209430in}}%
\pgfpathcurveto{\pgfqpoint{2.389970in}{1.215254in}}{\pgfqpoint{2.393242in}{1.223154in}}{\pgfqpoint{2.393242in}{1.231391in}}%
\pgfpathcurveto{\pgfqpoint{2.393242in}{1.239627in}}{\pgfqpoint{2.389970in}{1.247527in}}{\pgfqpoint{2.384146in}{1.253351in}}%
\pgfpathcurveto{\pgfqpoint{2.378322in}{1.259175in}}{\pgfqpoint{2.370422in}{1.262447in}}{\pgfqpoint{2.362185in}{1.262447in}}%
\pgfpathcurveto{\pgfqpoint{2.353949in}{1.262447in}}{\pgfqpoint{2.346049in}{1.259175in}}{\pgfqpoint{2.340225in}{1.253351in}}%
\pgfpathcurveto{\pgfqpoint{2.334401in}{1.247527in}}{\pgfqpoint{2.331129in}{1.239627in}}{\pgfqpoint{2.331129in}{1.231391in}}%
\pgfpathcurveto{\pgfqpoint{2.331129in}{1.223154in}}{\pgfqpoint{2.334401in}{1.215254in}}{\pgfqpoint{2.340225in}{1.209430in}}%
\pgfpathcurveto{\pgfqpoint{2.346049in}{1.203607in}}{\pgfqpoint{2.353949in}{1.200334in}}{\pgfqpoint{2.362185in}{1.200334in}}%
\pgfpathclose%
\pgfusepath{stroke,fill}%
\end{pgfscope}%
\begin{pgfscope}%
\pgfpathrectangle{\pgfqpoint{0.100000in}{0.212622in}}{\pgfqpoint{3.696000in}{3.696000in}}%
\pgfusepath{clip}%
\pgfsetbuttcap%
\pgfsetroundjoin%
\definecolor{currentfill}{rgb}{0.121569,0.466667,0.705882}%
\pgfsetfillcolor{currentfill}%
\pgfsetfillopacity{0.801263}%
\pgfsetlinewidth{1.003750pt}%
\definecolor{currentstroke}{rgb}{0.121569,0.466667,0.705882}%
\pgfsetstrokecolor{currentstroke}%
\pgfsetstrokeopacity{0.801263}%
\pgfsetdash{}{0pt}%
\pgfpathmoveto{\pgfqpoint{2.366216in}{1.199455in}}%
\pgfpathcurveto{\pgfqpoint{2.374452in}{1.199455in}}{\pgfqpoint{2.382352in}{1.202727in}}{\pgfqpoint{2.388176in}{1.208551in}}%
\pgfpathcurveto{\pgfqpoint{2.394000in}{1.214375in}}{\pgfqpoint{2.397272in}{1.222275in}}{\pgfqpoint{2.397272in}{1.230511in}}%
\pgfpathcurveto{\pgfqpoint{2.397272in}{1.238747in}}{\pgfqpoint{2.394000in}{1.246647in}}{\pgfqpoint{2.388176in}{1.252471in}}%
\pgfpathcurveto{\pgfqpoint{2.382352in}{1.258295in}}{\pgfqpoint{2.374452in}{1.261568in}}{\pgfqpoint{2.366216in}{1.261568in}}%
\pgfpathcurveto{\pgfqpoint{2.357980in}{1.261568in}}{\pgfqpoint{2.350080in}{1.258295in}}{\pgfqpoint{2.344256in}{1.252471in}}%
\pgfpathcurveto{\pgfqpoint{2.338432in}{1.246647in}}{\pgfqpoint{2.335159in}{1.238747in}}{\pgfqpoint{2.335159in}{1.230511in}}%
\pgfpathcurveto{\pgfqpoint{2.335159in}{1.222275in}}{\pgfqpoint{2.338432in}{1.214375in}}{\pgfqpoint{2.344256in}{1.208551in}}%
\pgfpathcurveto{\pgfqpoint{2.350080in}{1.202727in}}{\pgfqpoint{2.357980in}{1.199455in}}{\pgfqpoint{2.366216in}{1.199455in}}%
\pgfpathclose%
\pgfusepath{stroke,fill}%
\end{pgfscope}%
\begin{pgfscope}%
\pgfpathrectangle{\pgfqpoint{0.100000in}{0.212622in}}{\pgfqpoint{3.696000in}{3.696000in}}%
\pgfusepath{clip}%
\pgfsetbuttcap%
\pgfsetroundjoin%
\definecolor{currentfill}{rgb}{0.121569,0.466667,0.705882}%
\pgfsetfillcolor{currentfill}%
\pgfsetfillopacity{0.802656}%
\pgfsetlinewidth{1.003750pt}%
\definecolor{currentstroke}{rgb}{0.121569,0.466667,0.705882}%
\pgfsetstrokecolor{currentstroke}%
\pgfsetstrokeopacity{0.802656}%
\pgfsetdash{}{0pt}%
\pgfpathmoveto{\pgfqpoint{2.370825in}{1.198343in}}%
\pgfpathcurveto{\pgfqpoint{2.379062in}{1.198343in}}{\pgfqpoint{2.386962in}{1.201616in}}{\pgfqpoint{2.392786in}{1.207439in}}%
\pgfpathcurveto{\pgfqpoint{2.398610in}{1.213263in}}{\pgfqpoint{2.401882in}{1.221163in}}{\pgfqpoint{2.401882in}{1.229400in}}%
\pgfpathcurveto{\pgfqpoint{2.401882in}{1.237636in}}{\pgfqpoint{2.398610in}{1.245536in}}{\pgfqpoint{2.392786in}{1.251360in}}%
\pgfpathcurveto{\pgfqpoint{2.386962in}{1.257184in}}{\pgfqpoint{2.379062in}{1.260456in}}{\pgfqpoint{2.370825in}{1.260456in}}%
\pgfpathcurveto{\pgfqpoint{2.362589in}{1.260456in}}{\pgfqpoint{2.354689in}{1.257184in}}{\pgfqpoint{2.348865in}{1.251360in}}%
\pgfpathcurveto{\pgfqpoint{2.343041in}{1.245536in}}{\pgfqpoint{2.339769in}{1.237636in}}{\pgfqpoint{2.339769in}{1.229400in}}%
\pgfpathcurveto{\pgfqpoint{2.339769in}{1.221163in}}{\pgfqpoint{2.343041in}{1.213263in}}{\pgfqpoint{2.348865in}{1.207439in}}%
\pgfpathcurveto{\pgfqpoint{2.354689in}{1.201616in}}{\pgfqpoint{2.362589in}{1.198343in}}{\pgfqpoint{2.370825in}{1.198343in}}%
\pgfpathclose%
\pgfusepath{stroke,fill}%
\end{pgfscope}%
\begin{pgfscope}%
\pgfpathrectangle{\pgfqpoint{0.100000in}{0.212622in}}{\pgfqpoint{3.696000in}{3.696000in}}%
\pgfusepath{clip}%
\pgfsetbuttcap%
\pgfsetroundjoin%
\definecolor{currentfill}{rgb}{0.121569,0.466667,0.705882}%
\pgfsetfillcolor{currentfill}%
\pgfsetfillopacity{0.804292}%
\pgfsetlinewidth{1.003750pt}%
\definecolor{currentstroke}{rgb}{0.121569,0.466667,0.705882}%
\pgfsetstrokecolor{currentstroke}%
\pgfsetstrokeopacity{0.804292}%
\pgfsetdash{}{0pt}%
\pgfpathmoveto{\pgfqpoint{2.375837in}{1.197026in}}%
\pgfpathcurveto{\pgfqpoint{2.384073in}{1.197026in}}{\pgfqpoint{2.391973in}{1.200298in}}{\pgfqpoint{2.397797in}{1.206122in}}%
\pgfpathcurveto{\pgfqpoint{2.403621in}{1.211946in}}{\pgfqpoint{2.406893in}{1.219846in}}{\pgfqpoint{2.406893in}{1.228082in}}%
\pgfpathcurveto{\pgfqpoint{2.406893in}{1.236318in}}{\pgfqpoint{2.403621in}{1.244219in}}{\pgfqpoint{2.397797in}{1.250042in}}%
\pgfpathcurveto{\pgfqpoint{2.391973in}{1.255866in}}{\pgfqpoint{2.384073in}{1.259139in}}{\pgfqpoint{2.375837in}{1.259139in}}%
\pgfpathcurveto{\pgfqpoint{2.367601in}{1.259139in}}{\pgfqpoint{2.359700in}{1.255866in}}{\pgfqpoint{2.353877in}{1.250042in}}%
\pgfpathcurveto{\pgfqpoint{2.348053in}{1.244219in}}{\pgfqpoint{2.344780in}{1.236318in}}{\pgfqpoint{2.344780in}{1.228082in}}%
\pgfpathcurveto{\pgfqpoint{2.344780in}{1.219846in}}{\pgfqpoint{2.348053in}{1.211946in}}{\pgfqpoint{2.353877in}{1.206122in}}%
\pgfpathcurveto{\pgfqpoint{2.359700in}{1.200298in}}{\pgfqpoint{2.367601in}{1.197026in}}{\pgfqpoint{2.375837in}{1.197026in}}%
\pgfpathclose%
\pgfusepath{stroke,fill}%
\end{pgfscope}%
\begin{pgfscope}%
\pgfpathrectangle{\pgfqpoint{0.100000in}{0.212622in}}{\pgfqpoint{3.696000in}{3.696000in}}%
\pgfusepath{clip}%
\pgfsetbuttcap%
\pgfsetroundjoin%
\definecolor{currentfill}{rgb}{0.121569,0.466667,0.705882}%
\pgfsetfillcolor{currentfill}%
\pgfsetfillopacity{0.805138}%
\pgfsetlinewidth{1.003750pt}%
\definecolor{currentstroke}{rgb}{0.121569,0.466667,0.705882}%
\pgfsetstrokecolor{currentstroke}%
\pgfsetstrokeopacity{0.805138}%
\pgfsetdash{}{0pt}%
\pgfpathmoveto{\pgfqpoint{2.378645in}{1.196369in}}%
\pgfpathcurveto{\pgfqpoint{2.386881in}{1.196369in}}{\pgfqpoint{2.394781in}{1.199641in}}{\pgfqpoint{2.400605in}{1.205465in}}%
\pgfpathcurveto{\pgfqpoint{2.406429in}{1.211289in}}{\pgfqpoint{2.409701in}{1.219189in}}{\pgfqpoint{2.409701in}{1.227426in}}%
\pgfpathcurveto{\pgfqpoint{2.409701in}{1.235662in}}{\pgfqpoint{2.406429in}{1.243562in}}{\pgfqpoint{2.400605in}{1.249386in}}%
\pgfpathcurveto{\pgfqpoint{2.394781in}{1.255210in}}{\pgfqpoint{2.386881in}{1.258482in}}{\pgfqpoint{2.378645in}{1.258482in}}%
\pgfpathcurveto{\pgfqpoint{2.370409in}{1.258482in}}{\pgfqpoint{2.362509in}{1.255210in}}{\pgfqpoint{2.356685in}{1.249386in}}%
\pgfpathcurveto{\pgfqpoint{2.350861in}{1.243562in}}{\pgfqpoint{2.347588in}{1.235662in}}{\pgfqpoint{2.347588in}{1.227426in}}%
\pgfpathcurveto{\pgfqpoint{2.347588in}{1.219189in}}{\pgfqpoint{2.350861in}{1.211289in}}{\pgfqpoint{2.356685in}{1.205465in}}%
\pgfpathcurveto{\pgfqpoint{2.362509in}{1.199641in}}{\pgfqpoint{2.370409in}{1.196369in}}{\pgfqpoint{2.378645in}{1.196369in}}%
\pgfpathclose%
\pgfusepath{stroke,fill}%
\end{pgfscope}%
\begin{pgfscope}%
\pgfpathrectangle{\pgfqpoint{0.100000in}{0.212622in}}{\pgfqpoint{3.696000in}{3.696000in}}%
\pgfusepath{clip}%
\pgfsetbuttcap%
\pgfsetroundjoin%
\definecolor{currentfill}{rgb}{0.121569,0.466667,0.705882}%
\pgfsetfillcolor{currentfill}%
\pgfsetfillopacity{0.805641}%
\pgfsetlinewidth{1.003750pt}%
\definecolor{currentstroke}{rgb}{0.121569,0.466667,0.705882}%
\pgfsetstrokecolor{currentstroke}%
\pgfsetstrokeopacity{0.805641}%
\pgfsetdash{}{0pt}%
\pgfpathmoveto{\pgfqpoint{2.380153in}{1.195958in}}%
\pgfpathcurveto{\pgfqpoint{2.388389in}{1.195958in}}{\pgfqpoint{2.396289in}{1.199230in}}{\pgfqpoint{2.402113in}{1.205054in}}%
\pgfpathcurveto{\pgfqpoint{2.407937in}{1.210878in}}{\pgfqpoint{2.411209in}{1.218778in}}{\pgfqpoint{2.411209in}{1.227015in}}%
\pgfpathcurveto{\pgfqpoint{2.411209in}{1.235251in}}{\pgfqpoint{2.407937in}{1.243151in}}{\pgfqpoint{2.402113in}{1.248975in}}%
\pgfpathcurveto{\pgfqpoint{2.396289in}{1.254799in}}{\pgfqpoint{2.388389in}{1.258071in}}{\pgfqpoint{2.380153in}{1.258071in}}%
\pgfpathcurveto{\pgfqpoint{2.371916in}{1.258071in}}{\pgfqpoint{2.364016in}{1.254799in}}{\pgfqpoint{2.358192in}{1.248975in}}%
\pgfpathcurveto{\pgfqpoint{2.352368in}{1.243151in}}{\pgfqpoint{2.349096in}{1.235251in}}{\pgfqpoint{2.349096in}{1.227015in}}%
\pgfpathcurveto{\pgfqpoint{2.349096in}{1.218778in}}{\pgfqpoint{2.352368in}{1.210878in}}{\pgfqpoint{2.358192in}{1.205054in}}%
\pgfpathcurveto{\pgfqpoint{2.364016in}{1.199230in}}{\pgfqpoint{2.371916in}{1.195958in}}{\pgfqpoint{2.380153in}{1.195958in}}%
\pgfpathclose%
\pgfusepath{stroke,fill}%
\end{pgfscope}%
\begin{pgfscope}%
\pgfpathrectangle{\pgfqpoint{0.100000in}{0.212622in}}{\pgfqpoint{3.696000in}{3.696000in}}%
\pgfusepath{clip}%
\pgfsetbuttcap%
\pgfsetroundjoin%
\definecolor{currentfill}{rgb}{0.121569,0.466667,0.705882}%
\pgfsetfillcolor{currentfill}%
\pgfsetfillopacity{0.806746}%
\pgfsetlinewidth{1.003750pt}%
\definecolor{currentstroke}{rgb}{0.121569,0.466667,0.705882}%
\pgfsetstrokecolor{currentstroke}%
\pgfsetstrokeopacity{0.806746}%
\pgfsetdash{}{0pt}%
\pgfpathmoveto{\pgfqpoint{2.382475in}{1.195040in}}%
\pgfpathcurveto{\pgfqpoint{2.390712in}{1.195040in}}{\pgfqpoint{2.398612in}{1.198312in}}{\pgfqpoint{2.404436in}{1.204136in}}%
\pgfpathcurveto{\pgfqpoint{2.410260in}{1.209960in}}{\pgfqpoint{2.413532in}{1.217860in}}{\pgfqpoint{2.413532in}{1.226096in}}%
\pgfpathcurveto{\pgfqpoint{2.413532in}{1.234332in}}{\pgfqpoint{2.410260in}{1.242232in}}{\pgfqpoint{2.404436in}{1.248056in}}%
\pgfpathcurveto{\pgfqpoint{2.398612in}{1.253880in}}{\pgfqpoint{2.390712in}{1.257153in}}{\pgfqpoint{2.382475in}{1.257153in}}%
\pgfpathcurveto{\pgfqpoint{2.374239in}{1.257153in}}{\pgfqpoint{2.366339in}{1.253880in}}{\pgfqpoint{2.360515in}{1.248056in}}%
\pgfpathcurveto{\pgfqpoint{2.354691in}{1.242232in}}{\pgfqpoint{2.351419in}{1.234332in}}{\pgfqpoint{2.351419in}{1.226096in}}%
\pgfpathcurveto{\pgfqpoint{2.351419in}{1.217860in}}{\pgfqpoint{2.354691in}{1.209960in}}{\pgfqpoint{2.360515in}{1.204136in}}%
\pgfpathcurveto{\pgfqpoint{2.366339in}{1.198312in}}{\pgfqpoint{2.374239in}{1.195040in}}{\pgfqpoint{2.382475in}{1.195040in}}%
\pgfpathclose%
\pgfusepath{stroke,fill}%
\end{pgfscope}%
\begin{pgfscope}%
\pgfpathrectangle{\pgfqpoint{0.100000in}{0.212622in}}{\pgfqpoint{3.696000in}{3.696000in}}%
\pgfusepath{clip}%
\pgfsetbuttcap%
\pgfsetroundjoin%
\definecolor{currentfill}{rgb}{0.121569,0.466667,0.705882}%
\pgfsetfillcolor{currentfill}%
\pgfsetfillopacity{0.808132}%
\pgfsetlinewidth{1.003750pt}%
\definecolor{currentstroke}{rgb}{0.121569,0.466667,0.705882}%
\pgfsetstrokecolor{currentstroke}%
\pgfsetstrokeopacity{0.808132}%
\pgfsetdash{}{0pt}%
\pgfpathmoveto{\pgfqpoint{2.385689in}{1.193835in}}%
\pgfpathcurveto{\pgfqpoint{2.393925in}{1.193835in}}{\pgfqpoint{2.401825in}{1.197107in}}{\pgfqpoint{2.407649in}{1.202931in}}%
\pgfpathcurveto{\pgfqpoint{2.413473in}{1.208755in}}{\pgfqpoint{2.416746in}{1.216655in}}{\pgfqpoint{2.416746in}{1.224891in}}%
\pgfpathcurveto{\pgfqpoint{2.416746in}{1.233128in}}{\pgfqpoint{2.413473in}{1.241028in}}{\pgfqpoint{2.407649in}{1.246852in}}%
\pgfpathcurveto{\pgfqpoint{2.401825in}{1.252676in}}{\pgfqpoint{2.393925in}{1.255948in}}{\pgfqpoint{2.385689in}{1.255948in}}%
\pgfpathcurveto{\pgfqpoint{2.377453in}{1.255948in}}{\pgfqpoint{2.369553in}{1.252676in}}{\pgfqpoint{2.363729in}{1.246852in}}%
\pgfpathcurveto{\pgfqpoint{2.357905in}{1.241028in}}{\pgfqpoint{2.354633in}{1.233128in}}{\pgfqpoint{2.354633in}{1.224891in}}%
\pgfpathcurveto{\pgfqpoint{2.354633in}{1.216655in}}{\pgfqpoint{2.357905in}{1.208755in}}{\pgfqpoint{2.363729in}{1.202931in}}%
\pgfpathcurveto{\pgfqpoint{2.369553in}{1.197107in}}{\pgfqpoint{2.377453in}{1.193835in}}{\pgfqpoint{2.385689in}{1.193835in}}%
\pgfpathclose%
\pgfusepath{stroke,fill}%
\end{pgfscope}%
\begin{pgfscope}%
\pgfpathrectangle{\pgfqpoint{0.100000in}{0.212622in}}{\pgfqpoint{3.696000in}{3.696000in}}%
\pgfusepath{clip}%
\pgfsetbuttcap%
\pgfsetroundjoin%
\definecolor{currentfill}{rgb}{0.121569,0.466667,0.705882}%
\pgfsetfillcolor{currentfill}%
\pgfsetfillopacity{0.809602}%
\pgfsetlinewidth{1.003750pt}%
\definecolor{currentstroke}{rgb}{0.121569,0.466667,0.705882}%
\pgfsetstrokecolor{currentstroke}%
\pgfsetstrokeopacity{0.809602}%
\pgfsetdash{}{0pt}%
\pgfpathmoveto{\pgfqpoint{2.390232in}{1.192685in}}%
\pgfpathcurveto{\pgfqpoint{2.398468in}{1.192685in}}{\pgfqpoint{2.406369in}{1.195957in}}{\pgfqpoint{2.412192in}{1.201781in}}%
\pgfpathcurveto{\pgfqpoint{2.418016in}{1.207605in}}{\pgfqpoint{2.421289in}{1.215505in}}{\pgfqpoint{2.421289in}{1.223742in}}%
\pgfpathcurveto{\pgfqpoint{2.421289in}{1.231978in}}{\pgfqpoint{2.418016in}{1.239878in}}{\pgfqpoint{2.412192in}{1.245702in}}%
\pgfpathcurveto{\pgfqpoint{2.406369in}{1.251526in}}{\pgfqpoint{2.398468in}{1.254798in}}{\pgfqpoint{2.390232in}{1.254798in}}%
\pgfpathcurveto{\pgfqpoint{2.381996in}{1.254798in}}{\pgfqpoint{2.374096in}{1.251526in}}{\pgfqpoint{2.368272in}{1.245702in}}%
\pgfpathcurveto{\pgfqpoint{2.362448in}{1.239878in}}{\pgfqpoint{2.359176in}{1.231978in}}{\pgfqpoint{2.359176in}{1.223742in}}%
\pgfpathcurveto{\pgfqpoint{2.359176in}{1.215505in}}{\pgfqpoint{2.362448in}{1.207605in}}{\pgfqpoint{2.368272in}{1.201781in}}%
\pgfpathcurveto{\pgfqpoint{2.374096in}{1.195957in}}{\pgfqpoint{2.381996in}{1.192685in}}{\pgfqpoint{2.390232in}{1.192685in}}%
\pgfpathclose%
\pgfusepath{stroke,fill}%
\end{pgfscope}%
\begin{pgfscope}%
\pgfpathrectangle{\pgfqpoint{0.100000in}{0.212622in}}{\pgfqpoint{3.696000in}{3.696000in}}%
\pgfusepath{clip}%
\pgfsetbuttcap%
\pgfsetroundjoin%
\definecolor{currentfill}{rgb}{0.121569,0.466667,0.705882}%
\pgfsetfillcolor{currentfill}%
\pgfsetfillopacity{0.811799}%
\pgfsetlinewidth{1.003750pt}%
\definecolor{currentstroke}{rgb}{0.121569,0.466667,0.705882}%
\pgfsetstrokecolor{currentstroke}%
\pgfsetstrokeopacity{0.811799}%
\pgfsetdash{}{0pt}%
\pgfpathmoveto{\pgfqpoint{2.396122in}{1.190973in}}%
\pgfpathcurveto{\pgfqpoint{2.404358in}{1.190973in}}{\pgfqpoint{2.412258in}{1.194245in}}{\pgfqpoint{2.418082in}{1.200069in}}%
\pgfpathcurveto{\pgfqpoint{2.423906in}{1.205893in}}{\pgfqpoint{2.427178in}{1.213793in}}{\pgfqpoint{2.427178in}{1.222029in}}%
\pgfpathcurveto{\pgfqpoint{2.427178in}{1.230266in}}{\pgfqpoint{2.423906in}{1.238166in}}{\pgfqpoint{2.418082in}{1.243990in}}%
\pgfpathcurveto{\pgfqpoint{2.412258in}{1.249813in}}{\pgfqpoint{2.404358in}{1.253086in}}{\pgfqpoint{2.396122in}{1.253086in}}%
\pgfpathcurveto{\pgfqpoint{2.387885in}{1.253086in}}{\pgfqpoint{2.379985in}{1.249813in}}{\pgfqpoint{2.374161in}{1.243990in}}%
\pgfpathcurveto{\pgfqpoint{2.368337in}{1.238166in}}{\pgfqpoint{2.365065in}{1.230266in}}{\pgfqpoint{2.365065in}{1.222029in}}%
\pgfpathcurveto{\pgfqpoint{2.365065in}{1.213793in}}{\pgfqpoint{2.368337in}{1.205893in}}{\pgfqpoint{2.374161in}{1.200069in}}%
\pgfpathcurveto{\pgfqpoint{2.379985in}{1.194245in}}{\pgfqpoint{2.387885in}{1.190973in}}{\pgfqpoint{2.396122in}{1.190973in}}%
\pgfpathclose%
\pgfusepath{stroke,fill}%
\end{pgfscope}%
\begin{pgfscope}%
\pgfpathrectangle{\pgfqpoint{0.100000in}{0.212622in}}{\pgfqpoint{3.696000in}{3.696000in}}%
\pgfusepath{clip}%
\pgfsetbuttcap%
\pgfsetroundjoin%
\definecolor{currentfill}{rgb}{0.121569,0.466667,0.705882}%
\pgfsetfillcolor{currentfill}%
\pgfsetfillopacity{0.813927}%
\pgfsetlinewidth{1.003750pt}%
\definecolor{currentstroke}{rgb}{0.121569,0.466667,0.705882}%
\pgfsetstrokecolor{currentstroke}%
\pgfsetstrokeopacity{0.813927}%
\pgfsetdash{}{0pt}%
\pgfpathmoveto{\pgfqpoint{2.402811in}{1.189329in}}%
\pgfpathcurveto{\pgfqpoint{2.411047in}{1.189329in}}{\pgfqpoint{2.418947in}{1.192601in}}{\pgfqpoint{2.424771in}{1.198425in}}%
\pgfpathcurveto{\pgfqpoint{2.430595in}{1.204249in}}{\pgfqpoint{2.433867in}{1.212149in}}{\pgfqpoint{2.433867in}{1.220385in}}%
\pgfpathcurveto{\pgfqpoint{2.433867in}{1.228621in}}{\pgfqpoint{2.430595in}{1.236522in}}{\pgfqpoint{2.424771in}{1.242345in}}%
\pgfpathcurveto{\pgfqpoint{2.418947in}{1.248169in}}{\pgfqpoint{2.411047in}{1.251442in}}{\pgfqpoint{2.402811in}{1.251442in}}%
\pgfpathcurveto{\pgfqpoint{2.394574in}{1.251442in}}{\pgfqpoint{2.386674in}{1.248169in}}{\pgfqpoint{2.380850in}{1.242345in}}%
\pgfpathcurveto{\pgfqpoint{2.375027in}{1.236522in}}{\pgfqpoint{2.371754in}{1.228621in}}{\pgfqpoint{2.371754in}{1.220385in}}%
\pgfpathcurveto{\pgfqpoint{2.371754in}{1.212149in}}{\pgfqpoint{2.375027in}{1.204249in}}{\pgfqpoint{2.380850in}{1.198425in}}%
\pgfpathcurveto{\pgfqpoint{2.386674in}{1.192601in}}{\pgfqpoint{2.394574in}{1.189329in}}{\pgfqpoint{2.402811in}{1.189329in}}%
\pgfpathclose%
\pgfusepath{stroke,fill}%
\end{pgfscope}%
\begin{pgfscope}%
\pgfpathrectangle{\pgfqpoint{0.100000in}{0.212622in}}{\pgfqpoint{3.696000in}{3.696000in}}%
\pgfusepath{clip}%
\pgfsetbuttcap%
\pgfsetroundjoin%
\definecolor{currentfill}{rgb}{0.121569,0.466667,0.705882}%
\pgfsetfillcolor{currentfill}%
\pgfsetfillopacity{0.815287}%
\pgfsetlinewidth{1.003750pt}%
\definecolor{currentstroke}{rgb}{0.121569,0.466667,0.705882}%
\pgfsetstrokecolor{currentstroke}%
\pgfsetstrokeopacity{0.815287}%
\pgfsetdash{}{0pt}%
\pgfpathmoveto{\pgfqpoint{2.406328in}{1.188243in}}%
\pgfpathcurveto{\pgfqpoint{2.414564in}{1.188243in}}{\pgfqpoint{2.422464in}{1.191515in}}{\pgfqpoint{2.428288in}{1.197339in}}%
\pgfpathcurveto{\pgfqpoint{2.434112in}{1.203163in}}{\pgfqpoint{2.437384in}{1.211063in}}{\pgfqpoint{2.437384in}{1.219299in}}%
\pgfpathcurveto{\pgfqpoint{2.437384in}{1.227536in}}{\pgfqpoint{2.434112in}{1.235436in}}{\pgfqpoint{2.428288in}{1.241260in}}%
\pgfpathcurveto{\pgfqpoint{2.422464in}{1.247084in}}{\pgfqpoint{2.414564in}{1.250356in}}{\pgfqpoint{2.406328in}{1.250356in}}%
\pgfpathcurveto{\pgfqpoint{2.398091in}{1.250356in}}{\pgfqpoint{2.390191in}{1.247084in}}{\pgfqpoint{2.384367in}{1.241260in}}%
\pgfpathcurveto{\pgfqpoint{2.378543in}{1.235436in}}{\pgfqpoint{2.375271in}{1.227536in}}{\pgfqpoint{2.375271in}{1.219299in}}%
\pgfpathcurveto{\pgfqpoint{2.375271in}{1.211063in}}{\pgfqpoint{2.378543in}{1.203163in}}{\pgfqpoint{2.384367in}{1.197339in}}%
\pgfpathcurveto{\pgfqpoint{2.390191in}{1.191515in}}{\pgfqpoint{2.398091in}{1.188243in}}{\pgfqpoint{2.406328in}{1.188243in}}%
\pgfpathclose%
\pgfusepath{stroke,fill}%
\end{pgfscope}%
\begin{pgfscope}%
\pgfpathrectangle{\pgfqpoint{0.100000in}{0.212622in}}{\pgfqpoint{3.696000in}{3.696000in}}%
\pgfusepath{clip}%
\pgfsetbuttcap%
\pgfsetroundjoin%
\definecolor{currentfill}{rgb}{0.121569,0.466667,0.705882}%
\pgfsetfillcolor{currentfill}%
\pgfsetfillopacity{0.816995}%
\pgfsetlinewidth{1.003750pt}%
\definecolor{currentstroke}{rgb}{0.121569,0.466667,0.705882}%
\pgfsetstrokecolor{currentstroke}%
\pgfsetstrokeopacity{0.816995}%
\pgfsetdash{}{0pt}%
\pgfpathmoveto{\pgfqpoint{2.410263in}{1.186753in}}%
\pgfpathcurveto{\pgfqpoint{2.418499in}{1.186753in}}{\pgfqpoint{2.426399in}{1.190025in}}{\pgfqpoint{2.432223in}{1.195849in}}%
\pgfpathcurveto{\pgfqpoint{2.438047in}{1.201673in}}{\pgfqpoint{2.441319in}{1.209573in}}{\pgfqpoint{2.441319in}{1.217809in}}%
\pgfpathcurveto{\pgfqpoint{2.441319in}{1.226046in}}{\pgfqpoint{2.438047in}{1.233946in}}{\pgfqpoint{2.432223in}{1.239770in}}%
\pgfpathcurveto{\pgfqpoint{2.426399in}{1.245593in}}{\pgfqpoint{2.418499in}{1.248866in}}{\pgfqpoint{2.410263in}{1.248866in}}%
\pgfpathcurveto{\pgfqpoint{2.402027in}{1.248866in}}{\pgfqpoint{2.394127in}{1.245593in}}{\pgfqpoint{2.388303in}{1.239770in}}%
\pgfpathcurveto{\pgfqpoint{2.382479in}{1.233946in}}{\pgfqpoint{2.379206in}{1.226046in}}{\pgfqpoint{2.379206in}{1.217809in}}%
\pgfpathcurveto{\pgfqpoint{2.379206in}{1.209573in}}{\pgfqpoint{2.382479in}{1.201673in}}{\pgfqpoint{2.388303in}{1.195849in}}%
\pgfpathcurveto{\pgfqpoint{2.394127in}{1.190025in}}{\pgfqpoint{2.402027in}{1.186753in}}{\pgfqpoint{2.410263in}{1.186753in}}%
\pgfpathclose%
\pgfusepath{stroke,fill}%
\end{pgfscope}%
\begin{pgfscope}%
\pgfpathrectangle{\pgfqpoint{0.100000in}{0.212622in}}{\pgfqpoint{3.696000in}{3.696000in}}%
\pgfusepath{clip}%
\pgfsetbuttcap%
\pgfsetroundjoin%
\definecolor{currentfill}{rgb}{0.121569,0.466667,0.705882}%
\pgfsetfillcolor{currentfill}%
\pgfsetfillopacity{0.819142}%
\pgfsetlinewidth{1.003750pt}%
\definecolor{currentstroke}{rgb}{0.121569,0.466667,0.705882}%
\pgfsetstrokecolor{currentstroke}%
\pgfsetstrokeopacity{0.819142}%
\pgfsetdash{}{0pt}%
\pgfpathmoveto{\pgfqpoint{2.415013in}{1.184871in}}%
\pgfpathcurveto{\pgfqpoint{2.423249in}{1.184871in}}{\pgfqpoint{2.431149in}{1.188144in}}{\pgfqpoint{2.436973in}{1.193968in}}%
\pgfpathcurveto{\pgfqpoint{2.442797in}{1.199792in}}{\pgfqpoint{2.446069in}{1.207692in}}{\pgfqpoint{2.446069in}{1.215928in}}%
\pgfpathcurveto{\pgfqpoint{2.446069in}{1.224164in}}{\pgfqpoint{2.442797in}{1.232064in}}{\pgfqpoint{2.436973in}{1.237888in}}%
\pgfpathcurveto{\pgfqpoint{2.431149in}{1.243712in}}{\pgfqpoint{2.423249in}{1.246984in}}{\pgfqpoint{2.415013in}{1.246984in}}%
\pgfpathcurveto{\pgfqpoint{2.406777in}{1.246984in}}{\pgfqpoint{2.398877in}{1.243712in}}{\pgfqpoint{2.393053in}{1.237888in}}%
\pgfpathcurveto{\pgfqpoint{2.387229in}{1.232064in}}{\pgfqpoint{2.383956in}{1.224164in}}{\pgfqpoint{2.383956in}{1.215928in}}%
\pgfpathcurveto{\pgfqpoint{2.383956in}{1.207692in}}{\pgfqpoint{2.387229in}{1.199792in}}{\pgfqpoint{2.393053in}{1.193968in}}%
\pgfpathcurveto{\pgfqpoint{2.398877in}{1.188144in}}{\pgfqpoint{2.406777in}{1.184871in}}{\pgfqpoint{2.415013in}{1.184871in}}%
\pgfpathclose%
\pgfusepath{stroke,fill}%
\end{pgfscope}%
\begin{pgfscope}%
\pgfpathrectangle{\pgfqpoint{0.100000in}{0.212622in}}{\pgfqpoint{3.696000in}{3.696000in}}%
\pgfusepath{clip}%
\pgfsetbuttcap%
\pgfsetroundjoin%
\definecolor{currentfill}{rgb}{0.121569,0.466667,0.705882}%
\pgfsetfillcolor{currentfill}%
\pgfsetfillopacity{0.820040}%
\pgfsetlinewidth{1.003750pt}%
\definecolor{currentstroke}{rgb}{0.121569,0.466667,0.705882}%
\pgfsetstrokecolor{currentstroke}%
\pgfsetstrokeopacity{0.820040}%
\pgfsetdash{}{0pt}%
\pgfpathmoveto{\pgfqpoint{2.417901in}{1.184217in}}%
\pgfpathcurveto{\pgfqpoint{2.426138in}{1.184217in}}{\pgfqpoint{2.434038in}{1.187489in}}{\pgfqpoint{2.439862in}{1.193313in}}%
\pgfpathcurveto{\pgfqpoint{2.445685in}{1.199137in}}{\pgfqpoint{2.448958in}{1.207037in}}{\pgfqpoint{2.448958in}{1.215273in}}%
\pgfpathcurveto{\pgfqpoint{2.448958in}{1.223509in}}{\pgfqpoint{2.445685in}{1.231410in}}{\pgfqpoint{2.439862in}{1.237233in}}%
\pgfpathcurveto{\pgfqpoint{2.434038in}{1.243057in}}{\pgfqpoint{2.426138in}{1.246330in}}{\pgfqpoint{2.417901in}{1.246330in}}%
\pgfpathcurveto{\pgfqpoint{2.409665in}{1.246330in}}{\pgfqpoint{2.401765in}{1.243057in}}{\pgfqpoint{2.395941in}{1.237233in}}%
\pgfpathcurveto{\pgfqpoint{2.390117in}{1.231410in}}{\pgfqpoint{2.386845in}{1.223509in}}{\pgfqpoint{2.386845in}{1.215273in}}%
\pgfpathcurveto{\pgfqpoint{2.386845in}{1.207037in}}{\pgfqpoint{2.390117in}{1.199137in}}{\pgfqpoint{2.395941in}{1.193313in}}%
\pgfpathcurveto{\pgfqpoint{2.401765in}{1.187489in}}{\pgfqpoint{2.409665in}{1.184217in}}{\pgfqpoint{2.417901in}{1.184217in}}%
\pgfpathclose%
\pgfusepath{stroke,fill}%
\end{pgfscope}%
\begin{pgfscope}%
\pgfpathrectangle{\pgfqpoint{0.100000in}{0.212622in}}{\pgfqpoint{3.696000in}{3.696000in}}%
\pgfusepath{clip}%
\pgfsetbuttcap%
\pgfsetroundjoin%
\definecolor{currentfill}{rgb}{0.121569,0.466667,0.705882}%
\pgfsetfillcolor{currentfill}%
\pgfsetfillopacity{0.821232}%
\pgfsetlinewidth{1.003750pt}%
\definecolor{currentstroke}{rgb}{0.121569,0.466667,0.705882}%
\pgfsetstrokecolor{currentstroke}%
\pgfsetstrokeopacity{0.821232}%
\pgfsetdash{}{0pt}%
\pgfpathmoveto{\pgfqpoint{2.422191in}{1.183184in}}%
\pgfpathcurveto{\pgfqpoint{2.430427in}{1.183184in}}{\pgfqpoint{2.438327in}{1.186456in}}{\pgfqpoint{2.444151in}{1.192280in}}%
\pgfpathcurveto{\pgfqpoint{2.449975in}{1.198104in}}{\pgfqpoint{2.453247in}{1.206004in}}{\pgfqpoint{2.453247in}{1.214240in}}%
\pgfpathcurveto{\pgfqpoint{2.453247in}{1.222477in}}{\pgfqpoint{2.449975in}{1.230377in}}{\pgfqpoint{2.444151in}{1.236201in}}%
\pgfpathcurveto{\pgfqpoint{2.438327in}{1.242025in}}{\pgfqpoint{2.430427in}{1.245297in}}{\pgfqpoint{2.422191in}{1.245297in}}%
\pgfpathcurveto{\pgfqpoint{2.413955in}{1.245297in}}{\pgfqpoint{2.406055in}{1.242025in}}{\pgfqpoint{2.400231in}{1.236201in}}%
\pgfpathcurveto{\pgfqpoint{2.394407in}{1.230377in}}{\pgfqpoint{2.391134in}{1.222477in}}{\pgfqpoint{2.391134in}{1.214240in}}%
\pgfpathcurveto{\pgfqpoint{2.391134in}{1.206004in}}{\pgfqpoint{2.394407in}{1.198104in}}{\pgfqpoint{2.400231in}{1.192280in}}%
\pgfpathcurveto{\pgfqpoint{2.406055in}{1.186456in}}{\pgfqpoint{2.413955in}{1.183184in}}{\pgfqpoint{2.422191in}{1.183184in}}%
\pgfpathclose%
\pgfusepath{stroke,fill}%
\end{pgfscope}%
\begin{pgfscope}%
\pgfpathrectangle{\pgfqpoint{0.100000in}{0.212622in}}{\pgfqpoint{3.696000in}{3.696000in}}%
\pgfusepath{clip}%
\pgfsetbuttcap%
\pgfsetroundjoin%
\definecolor{currentfill}{rgb}{0.121569,0.466667,0.705882}%
\pgfsetfillcolor{currentfill}%
\pgfsetfillopacity{0.823825}%
\pgfsetlinewidth{1.003750pt}%
\definecolor{currentstroke}{rgb}{0.121569,0.466667,0.705882}%
\pgfsetstrokecolor{currentstroke}%
\pgfsetstrokeopacity{0.823825}%
\pgfsetdash{}{0pt}%
\pgfpathmoveto{\pgfqpoint{2.428171in}{1.180996in}}%
\pgfpathcurveto{\pgfqpoint{2.436407in}{1.180996in}}{\pgfqpoint{2.444307in}{1.184268in}}{\pgfqpoint{2.450131in}{1.190092in}}%
\pgfpathcurveto{\pgfqpoint{2.455955in}{1.195916in}}{\pgfqpoint{2.459227in}{1.203816in}}{\pgfqpoint{2.459227in}{1.212053in}}%
\pgfpathcurveto{\pgfqpoint{2.459227in}{1.220289in}}{\pgfqpoint{2.455955in}{1.228189in}}{\pgfqpoint{2.450131in}{1.234013in}}%
\pgfpathcurveto{\pgfqpoint{2.444307in}{1.239837in}}{\pgfqpoint{2.436407in}{1.243109in}}{\pgfqpoint{2.428171in}{1.243109in}}%
\pgfpathcurveto{\pgfqpoint{2.419934in}{1.243109in}}{\pgfqpoint{2.412034in}{1.239837in}}{\pgfqpoint{2.406210in}{1.234013in}}%
\pgfpathcurveto{\pgfqpoint{2.400386in}{1.228189in}}{\pgfqpoint{2.397114in}{1.220289in}}{\pgfqpoint{2.397114in}{1.212053in}}%
\pgfpathcurveto{\pgfqpoint{2.397114in}{1.203816in}}{\pgfqpoint{2.400386in}{1.195916in}}{\pgfqpoint{2.406210in}{1.190092in}}%
\pgfpathcurveto{\pgfqpoint{2.412034in}{1.184268in}}{\pgfqpoint{2.419934in}{1.180996in}}{\pgfqpoint{2.428171in}{1.180996in}}%
\pgfpathclose%
\pgfusepath{stroke,fill}%
\end{pgfscope}%
\begin{pgfscope}%
\pgfpathrectangle{\pgfqpoint{0.100000in}{0.212622in}}{\pgfqpoint{3.696000in}{3.696000in}}%
\pgfusepath{clip}%
\pgfsetbuttcap%
\pgfsetroundjoin%
\definecolor{currentfill}{rgb}{0.121569,0.466667,0.705882}%
\pgfsetfillcolor{currentfill}%
\pgfsetfillopacity{0.826136}%
\pgfsetlinewidth{1.003750pt}%
\definecolor{currentstroke}{rgb}{0.121569,0.466667,0.705882}%
\pgfsetstrokecolor{currentstroke}%
\pgfsetstrokeopacity{0.826136}%
\pgfsetdash{}{0pt}%
\pgfpathmoveto{\pgfqpoint{2.435806in}{1.179218in}}%
\pgfpathcurveto{\pgfqpoint{2.444042in}{1.179218in}}{\pgfqpoint{2.451943in}{1.182490in}}{\pgfqpoint{2.457766in}{1.188314in}}%
\pgfpathcurveto{\pgfqpoint{2.463590in}{1.194138in}}{\pgfqpoint{2.466863in}{1.202038in}}{\pgfqpoint{2.466863in}{1.210274in}}%
\pgfpathcurveto{\pgfqpoint{2.466863in}{1.218511in}}{\pgfqpoint{2.463590in}{1.226411in}}{\pgfqpoint{2.457766in}{1.232235in}}%
\pgfpathcurveto{\pgfqpoint{2.451943in}{1.238059in}}{\pgfqpoint{2.444042in}{1.241331in}}{\pgfqpoint{2.435806in}{1.241331in}}%
\pgfpathcurveto{\pgfqpoint{2.427570in}{1.241331in}}{\pgfqpoint{2.419670in}{1.238059in}}{\pgfqpoint{2.413846in}{1.232235in}}%
\pgfpathcurveto{\pgfqpoint{2.408022in}{1.226411in}}{\pgfqpoint{2.404750in}{1.218511in}}{\pgfqpoint{2.404750in}{1.210274in}}%
\pgfpathcurveto{\pgfqpoint{2.404750in}{1.202038in}}{\pgfqpoint{2.408022in}{1.194138in}}{\pgfqpoint{2.413846in}{1.188314in}}%
\pgfpathcurveto{\pgfqpoint{2.419670in}{1.182490in}}{\pgfqpoint{2.427570in}{1.179218in}}{\pgfqpoint{2.435806in}{1.179218in}}%
\pgfpathclose%
\pgfusepath{stroke,fill}%
\end{pgfscope}%
\begin{pgfscope}%
\pgfpathrectangle{\pgfqpoint{0.100000in}{0.212622in}}{\pgfqpoint{3.696000in}{3.696000in}}%
\pgfusepath{clip}%
\pgfsetbuttcap%
\pgfsetroundjoin%
\definecolor{currentfill}{rgb}{0.121569,0.466667,0.705882}%
\pgfsetfillcolor{currentfill}%
\pgfsetfillopacity{0.827558}%
\pgfsetlinewidth{1.003750pt}%
\definecolor{currentstroke}{rgb}{0.121569,0.466667,0.705882}%
\pgfsetstrokecolor{currentstroke}%
\pgfsetstrokeopacity{0.827558}%
\pgfsetdash{}{0pt}%
\pgfpathmoveto{\pgfqpoint{2.439885in}{1.178121in}}%
\pgfpathcurveto{\pgfqpoint{2.448121in}{1.178121in}}{\pgfqpoint{2.456021in}{1.181394in}}{\pgfqpoint{2.461845in}{1.187218in}}%
\pgfpathcurveto{\pgfqpoint{2.467669in}{1.193042in}}{\pgfqpoint{2.470942in}{1.200942in}}{\pgfqpoint{2.470942in}{1.209178in}}%
\pgfpathcurveto{\pgfqpoint{2.470942in}{1.217414in}}{\pgfqpoint{2.467669in}{1.225314in}}{\pgfqpoint{2.461845in}{1.231138in}}%
\pgfpathcurveto{\pgfqpoint{2.456021in}{1.236962in}}{\pgfqpoint{2.448121in}{1.240234in}}{\pgfqpoint{2.439885in}{1.240234in}}%
\pgfpathcurveto{\pgfqpoint{2.431649in}{1.240234in}}{\pgfqpoint{2.423749in}{1.236962in}}{\pgfqpoint{2.417925in}{1.231138in}}%
\pgfpathcurveto{\pgfqpoint{2.412101in}{1.225314in}}{\pgfqpoint{2.408829in}{1.217414in}}{\pgfqpoint{2.408829in}{1.209178in}}%
\pgfpathcurveto{\pgfqpoint{2.408829in}{1.200942in}}{\pgfqpoint{2.412101in}{1.193042in}}{\pgfqpoint{2.417925in}{1.187218in}}%
\pgfpathcurveto{\pgfqpoint{2.423749in}{1.181394in}}{\pgfqpoint{2.431649in}{1.178121in}}{\pgfqpoint{2.439885in}{1.178121in}}%
\pgfpathclose%
\pgfusepath{stroke,fill}%
\end{pgfscope}%
\begin{pgfscope}%
\pgfpathrectangle{\pgfqpoint{0.100000in}{0.212622in}}{\pgfqpoint{3.696000in}{3.696000in}}%
\pgfusepath{clip}%
\pgfsetbuttcap%
\pgfsetroundjoin%
\definecolor{currentfill}{rgb}{0.121569,0.466667,0.705882}%
\pgfsetfillcolor{currentfill}%
\pgfsetfillopacity{0.828321}%
\pgfsetlinewidth{1.003750pt}%
\definecolor{currentstroke}{rgb}{0.121569,0.466667,0.705882}%
\pgfsetstrokecolor{currentstroke}%
\pgfsetstrokeopacity{0.828321}%
\pgfsetdash{}{0pt}%
\pgfpathmoveto{\pgfqpoint{2.442133in}{1.177495in}}%
\pgfpathcurveto{\pgfqpoint{2.450369in}{1.177495in}}{\pgfqpoint{2.458269in}{1.180767in}}{\pgfqpoint{2.464093in}{1.186591in}}%
\pgfpathcurveto{\pgfqpoint{2.469917in}{1.192415in}}{\pgfqpoint{2.473189in}{1.200315in}}{\pgfqpoint{2.473189in}{1.208551in}}%
\pgfpathcurveto{\pgfqpoint{2.473189in}{1.216788in}}{\pgfqpoint{2.469917in}{1.224688in}}{\pgfqpoint{2.464093in}{1.230512in}}%
\pgfpathcurveto{\pgfqpoint{2.458269in}{1.236335in}}{\pgfqpoint{2.450369in}{1.239608in}}{\pgfqpoint{2.442133in}{1.239608in}}%
\pgfpathcurveto{\pgfqpoint{2.433897in}{1.239608in}}{\pgfqpoint{2.425997in}{1.236335in}}{\pgfqpoint{2.420173in}{1.230512in}}%
\pgfpathcurveto{\pgfqpoint{2.414349in}{1.224688in}}{\pgfqpoint{2.411076in}{1.216788in}}{\pgfqpoint{2.411076in}{1.208551in}}%
\pgfpathcurveto{\pgfqpoint{2.411076in}{1.200315in}}{\pgfqpoint{2.414349in}{1.192415in}}{\pgfqpoint{2.420173in}{1.186591in}}%
\pgfpathcurveto{\pgfqpoint{2.425997in}{1.180767in}}{\pgfqpoint{2.433897in}{1.177495in}}{\pgfqpoint{2.442133in}{1.177495in}}%
\pgfpathclose%
\pgfusepath{stroke,fill}%
\end{pgfscope}%
\begin{pgfscope}%
\pgfpathrectangle{\pgfqpoint{0.100000in}{0.212622in}}{\pgfqpoint{3.696000in}{3.696000in}}%
\pgfusepath{clip}%
\pgfsetbuttcap%
\pgfsetroundjoin%
\definecolor{currentfill}{rgb}{0.121569,0.466667,0.705882}%
\pgfsetfillcolor{currentfill}%
\pgfsetfillopacity{0.828820}%
\pgfsetlinewidth{1.003750pt}%
\definecolor{currentstroke}{rgb}{0.121569,0.466667,0.705882}%
\pgfsetstrokecolor{currentstroke}%
\pgfsetstrokeopacity{0.828820}%
\pgfsetdash{}{0pt}%
\pgfpathmoveto{\pgfqpoint{2.443294in}{1.177048in}}%
\pgfpathcurveto{\pgfqpoint{2.451530in}{1.177048in}}{\pgfqpoint{2.459430in}{1.180321in}}{\pgfqpoint{2.465254in}{1.186145in}}%
\pgfpathcurveto{\pgfqpoint{2.471078in}{1.191968in}}{\pgfqpoint{2.474350in}{1.199869in}}{\pgfqpoint{2.474350in}{1.208105in}}%
\pgfpathcurveto{\pgfqpoint{2.474350in}{1.216341in}}{\pgfqpoint{2.471078in}{1.224241in}}{\pgfqpoint{2.465254in}{1.230065in}}%
\pgfpathcurveto{\pgfqpoint{2.459430in}{1.235889in}}{\pgfqpoint{2.451530in}{1.239161in}}{\pgfqpoint{2.443294in}{1.239161in}}%
\pgfpathcurveto{\pgfqpoint{2.435057in}{1.239161in}}{\pgfqpoint{2.427157in}{1.235889in}}{\pgfqpoint{2.421333in}{1.230065in}}%
\pgfpathcurveto{\pgfqpoint{2.415510in}{1.224241in}}{\pgfqpoint{2.412237in}{1.216341in}}{\pgfqpoint{2.412237in}{1.208105in}}%
\pgfpathcurveto{\pgfqpoint{2.412237in}{1.199869in}}{\pgfqpoint{2.415510in}{1.191968in}}{\pgfqpoint{2.421333in}{1.186145in}}%
\pgfpathcurveto{\pgfqpoint{2.427157in}{1.180321in}}{\pgfqpoint{2.435057in}{1.177048in}}{\pgfqpoint{2.443294in}{1.177048in}}%
\pgfpathclose%
\pgfusepath{stroke,fill}%
\end{pgfscope}%
\begin{pgfscope}%
\pgfpathrectangle{\pgfqpoint{0.100000in}{0.212622in}}{\pgfqpoint{3.696000in}{3.696000in}}%
\pgfusepath{clip}%
\pgfsetbuttcap%
\pgfsetroundjoin%
\definecolor{currentfill}{rgb}{0.121569,0.466667,0.705882}%
\pgfsetfillcolor{currentfill}%
\pgfsetfillopacity{0.829430}%
\pgfsetlinewidth{1.003750pt}%
\definecolor{currentstroke}{rgb}{0.121569,0.466667,0.705882}%
\pgfsetstrokecolor{currentstroke}%
\pgfsetstrokeopacity{0.829430}%
\pgfsetdash{}{0pt}%
\pgfpathmoveto{\pgfqpoint{2.445469in}{1.176594in}}%
\pgfpathcurveto{\pgfqpoint{2.453705in}{1.176594in}}{\pgfqpoint{2.461605in}{1.179866in}}{\pgfqpoint{2.467429in}{1.185690in}}%
\pgfpathcurveto{\pgfqpoint{2.473253in}{1.191514in}}{\pgfqpoint{2.476525in}{1.199414in}}{\pgfqpoint{2.476525in}{1.207650in}}%
\pgfpathcurveto{\pgfqpoint{2.476525in}{1.215887in}}{\pgfqpoint{2.473253in}{1.223787in}}{\pgfqpoint{2.467429in}{1.229610in}}%
\pgfpathcurveto{\pgfqpoint{2.461605in}{1.235434in}}{\pgfqpoint{2.453705in}{1.238707in}}{\pgfqpoint{2.445469in}{1.238707in}}%
\pgfpathcurveto{\pgfqpoint{2.437233in}{1.238707in}}{\pgfqpoint{2.429333in}{1.235434in}}{\pgfqpoint{2.423509in}{1.229610in}}%
\pgfpathcurveto{\pgfqpoint{2.417685in}{1.223787in}}{\pgfqpoint{2.414412in}{1.215887in}}{\pgfqpoint{2.414412in}{1.207650in}}%
\pgfpathcurveto{\pgfqpoint{2.414412in}{1.199414in}}{\pgfqpoint{2.417685in}{1.191514in}}{\pgfqpoint{2.423509in}{1.185690in}}%
\pgfpathcurveto{\pgfqpoint{2.429333in}{1.179866in}}{\pgfqpoint{2.437233in}{1.176594in}}{\pgfqpoint{2.445469in}{1.176594in}}%
\pgfpathclose%
\pgfusepath{stroke,fill}%
\end{pgfscope}%
\begin{pgfscope}%
\pgfpathrectangle{\pgfqpoint{0.100000in}{0.212622in}}{\pgfqpoint{3.696000in}{3.696000in}}%
\pgfusepath{clip}%
\pgfsetbuttcap%
\pgfsetroundjoin%
\definecolor{currentfill}{rgb}{0.121569,0.466667,0.705882}%
\pgfsetfillcolor{currentfill}%
\pgfsetfillopacity{0.830729}%
\pgfsetlinewidth{1.003750pt}%
\definecolor{currentstroke}{rgb}{0.121569,0.466667,0.705882}%
\pgfsetstrokecolor{currentstroke}%
\pgfsetstrokeopacity{0.830729}%
\pgfsetdash{}{0pt}%
\pgfpathmoveto{\pgfqpoint{2.448726in}{1.175544in}}%
\pgfpathcurveto{\pgfqpoint{2.456962in}{1.175544in}}{\pgfqpoint{2.464862in}{1.178816in}}{\pgfqpoint{2.470686in}{1.184640in}}%
\pgfpathcurveto{\pgfqpoint{2.476510in}{1.190464in}}{\pgfqpoint{2.479782in}{1.198364in}}{\pgfqpoint{2.479782in}{1.206600in}}%
\pgfpathcurveto{\pgfqpoint{2.479782in}{1.214837in}}{\pgfqpoint{2.476510in}{1.222737in}}{\pgfqpoint{2.470686in}{1.228560in}}%
\pgfpathcurveto{\pgfqpoint{2.464862in}{1.234384in}}{\pgfqpoint{2.456962in}{1.237657in}}{\pgfqpoint{2.448726in}{1.237657in}}%
\pgfpathcurveto{\pgfqpoint{2.440490in}{1.237657in}}{\pgfqpoint{2.432589in}{1.234384in}}{\pgfqpoint{2.426766in}{1.228560in}}%
\pgfpathcurveto{\pgfqpoint{2.420942in}{1.222737in}}{\pgfqpoint{2.417669in}{1.214837in}}{\pgfqpoint{2.417669in}{1.206600in}}%
\pgfpathcurveto{\pgfqpoint{2.417669in}{1.198364in}}{\pgfqpoint{2.420942in}{1.190464in}}{\pgfqpoint{2.426766in}{1.184640in}}%
\pgfpathcurveto{\pgfqpoint{2.432589in}{1.178816in}}{\pgfqpoint{2.440490in}{1.175544in}}{\pgfqpoint{2.448726in}{1.175544in}}%
\pgfpathclose%
\pgfusepath{stroke,fill}%
\end{pgfscope}%
\begin{pgfscope}%
\pgfpathrectangle{\pgfqpoint{0.100000in}{0.212622in}}{\pgfqpoint{3.696000in}{3.696000in}}%
\pgfusepath{clip}%
\pgfsetbuttcap%
\pgfsetroundjoin%
\definecolor{currentfill}{rgb}{0.121569,0.466667,0.705882}%
\pgfsetfillcolor{currentfill}%
\pgfsetfillopacity{0.832707}%
\pgfsetlinewidth{1.003750pt}%
\definecolor{currentstroke}{rgb}{0.121569,0.466667,0.705882}%
\pgfsetstrokecolor{currentstroke}%
\pgfsetstrokeopacity{0.832707}%
\pgfsetdash{}{0pt}%
\pgfpathmoveto{\pgfqpoint{2.453882in}{1.173986in}}%
\pgfpathcurveto{\pgfqpoint{2.462118in}{1.173986in}}{\pgfqpoint{2.470018in}{1.177258in}}{\pgfqpoint{2.475842in}{1.183082in}}%
\pgfpathcurveto{\pgfqpoint{2.481666in}{1.188906in}}{\pgfqpoint{2.484938in}{1.196806in}}{\pgfqpoint{2.484938in}{1.205042in}}%
\pgfpathcurveto{\pgfqpoint{2.484938in}{1.213279in}}{\pgfqpoint{2.481666in}{1.221179in}}{\pgfqpoint{2.475842in}{1.227003in}}%
\pgfpathcurveto{\pgfqpoint{2.470018in}{1.232827in}}{\pgfqpoint{2.462118in}{1.236099in}}{\pgfqpoint{2.453882in}{1.236099in}}%
\pgfpathcurveto{\pgfqpoint{2.445645in}{1.236099in}}{\pgfqpoint{2.437745in}{1.232827in}}{\pgfqpoint{2.431921in}{1.227003in}}%
\pgfpathcurveto{\pgfqpoint{2.426097in}{1.221179in}}{\pgfqpoint{2.422825in}{1.213279in}}{\pgfqpoint{2.422825in}{1.205042in}}%
\pgfpathcurveto{\pgfqpoint{2.422825in}{1.196806in}}{\pgfqpoint{2.426097in}{1.188906in}}{\pgfqpoint{2.431921in}{1.183082in}}%
\pgfpathcurveto{\pgfqpoint{2.437745in}{1.177258in}}{\pgfqpoint{2.445645in}{1.173986in}}{\pgfqpoint{2.453882in}{1.173986in}}%
\pgfpathclose%
\pgfusepath{stroke,fill}%
\end{pgfscope}%
\begin{pgfscope}%
\pgfpathrectangle{\pgfqpoint{0.100000in}{0.212622in}}{\pgfqpoint{3.696000in}{3.696000in}}%
\pgfusepath{clip}%
\pgfsetbuttcap%
\pgfsetroundjoin%
\definecolor{currentfill}{rgb}{0.121569,0.466667,0.705882}%
\pgfsetfillcolor{currentfill}%
\pgfsetfillopacity{0.834973}%
\pgfsetlinewidth{1.003750pt}%
\definecolor{currentstroke}{rgb}{0.121569,0.466667,0.705882}%
\pgfsetstrokecolor{currentstroke}%
\pgfsetstrokeopacity{0.834973}%
\pgfsetdash{}{0pt}%
\pgfpathmoveto{\pgfqpoint{2.460385in}{1.171908in}}%
\pgfpathcurveto{\pgfqpoint{2.468621in}{1.171908in}}{\pgfqpoint{2.476521in}{1.175180in}}{\pgfqpoint{2.482345in}{1.181004in}}%
\pgfpathcurveto{\pgfqpoint{2.488169in}{1.186828in}}{\pgfqpoint{2.491442in}{1.194728in}}{\pgfqpoint{2.491442in}{1.202965in}}%
\pgfpathcurveto{\pgfqpoint{2.491442in}{1.211201in}}{\pgfqpoint{2.488169in}{1.219101in}}{\pgfqpoint{2.482345in}{1.224925in}}%
\pgfpathcurveto{\pgfqpoint{2.476521in}{1.230749in}}{\pgfqpoint{2.468621in}{1.234021in}}{\pgfqpoint{2.460385in}{1.234021in}}%
\pgfpathcurveto{\pgfqpoint{2.452149in}{1.234021in}}{\pgfqpoint{2.444249in}{1.230749in}}{\pgfqpoint{2.438425in}{1.224925in}}%
\pgfpathcurveto{\pgfqpoint{2.432601in}{1.219101in}}{\pgfqpoint{2.429329in}{1.211201in}}{\pgfqpoint{2.429329in}{1.202965in}}%
\pgfpathcurveto{\pgfqpoint{2.429329in}{1.194728in}}{\pgfqpoint{2.432601in}{1.186828in}}{\pgfqpoint{2.438425in}{1.181004in}}%
\pgfpathcurveto{\pgfqpoint{2.444249in}{1.175180in}}{\pgfqpoint{2.452149in}{1.171908in}}{\pgfqpoint{2.460385in}{1.171908in}}%
\pgfpathclose%
\pgfusepath{stroke,fill}%
\end{pgfscope}%
\begin{pgfscope}%
\pgfpathrectangle{\pgfqpoint{0.100000in}{0.212622in}}{\pgfqpoint{3.696000in}{3.696000in}}%
\pgfusepath{clip}%
\pgfsetbuttcap%
\pgfsetroundjoin%
\definecolor{currentfill}{rgb}{0.121569,0.466667,0.705882}%
\pgfsetfillcolor{currentfill}%
\pgfsetfillopacity{0.838044}%
\pgfsetlinewidth{1.003750pt}%
\definecolor{currentstroke}{rgb}{0.121569,0.466667,0.705882}%
\pgfsetstrokecolor{currentstroke}%
\pgfsetstrokeopacity{0.838044}%
\pgfsetdash{}{0pt}%
\pgfpathmoveto{\pgfqpoint{2.467462in}{1.169382in}}%
\pgfpathcurveto{\pgfqpoint{2.475698in}{1.169382in}}{\pgfqpoint{2.483598in}{1.172654in}}{\pgfqpoint{2.489422in}{1.178478in}}%
\pgfpathcurveto{\pgfqpoint{2.495246in}{1.184302in}}{\pgfqpoint{2.498518in}{1.192202in}}{\pgfqpoint{2.498518in}{1.200438in}}%
\pgfpathcurveto{\pgfqpoint{2.498518in}{1.208674in}}{\pgfqpoint{2.495246in}{1.216574in}}{\pgfqpoint{2.489422in}{1.222398in}}%
\pgfpathcurveto{\pgfqpoint{2.483598in}{1.228222in}}{\pgfqpoint{2.475698in}{1.231495in}}{\pgfqpoint{2.467462in}{1.231495in}}%
\pgfpathcurveto{\pgfqpoint{2.459225in}{1.231495in}}{\pgfqpoint{2.451325in}{1.228222in}}{\pgfqpoint{2.445501in}{1.222398in}}%
\pgfpathcurveto{\pgfqpoint{2.439677in}{1.216574in}}{\pgfqpoint{2.436405in}{1.208674in}}{\pgfqpoint{2.436405in}{1.200438in}}%
\pgfpathcurveto{\pgfqpoint{2.436405in}{1.192202in}}{\pgfqpoint{2.439677in}{1.184302in}}{\pgfqpoint{2.445501in}{1.178478in}}%
\pgfpathcurveto{\pgfqpoint{2.451325in}{1.172654in}}{\pgfqpoint{2.459225in}{1.169382in}}{\pgfqpoint{2.467462in}{1.169382in}}%
\pgfpathclose%
\pgfusepath{stroke,fill}%
\end{pgfscope}%
\begin{pgfscope}%
\pgfpathrectangle{\pgfqpoint{0.100000in}{0.212622in}}{\pgfqpoint{3.696000in}{3.696000in}}%
\pgfusepath{clip}%
\pgfsetbuttcap%
\pgfsetroundjoin%
\definecolor{currentfill}{rgb}{0.121569,0.466667,0.705882}%
\pgfsetfillcolor{currentfill}%
\pgfsetfillopacity{0.839387}%
\pgfsetlinewidth{1.003750pt}%
\definecolor{currentstroke}{rgb}{0.121569,0.466667,0.705882}%
\pgfsetstrokecolor{currentstroke}%
\pgfsetstrokeopacity{0.839387}%
\pgfsetdash{}{0pt}%
\pgfpathmoveto{\pgfqpoint{2.471619in}{1.168208in}}%
\pgfpathcurveto{\pgfqpoint{2.479855in}{1.168208in}}{\pgfqpoint{2.487756in}{1.171481in}}{\pgfqpoint{2.493579in}{1.177305in}}%
\pgfpathcurveto{\pgfqpoint{2.499403in}{1.183128in}}{\pgfqpoint{2.502676in}{1.191029in}}{\pgfqpoint{2.502676in}{1.199265in}}%
\pgfpathcurveto{\pgfqpoint{2.502676in}{1.207501in}}{\pgfqpoint{2.499403in}{1.215401in}}{\pgfqpoint{2.493579in}{1.221225in}}%
\pgfpathcurveto{\pgfqpoint{2.487756in}{1.227049in}}{\pgfqpoint{2.479855in}{1.230321in}}{\pgfqpoint{2.471619in}{1.230321in}}%
\pgfpathcurveto{\pgfqpoint{2.463383in}{1.230321in}}{\pgfqpoint{2.455483in}{1.227049in}}{\pgfqpoint{2.449659in}{1.221225in}}%
\pgfpathcurveto{\pgfqpoint{2.443835in}{1.215401in}}{\pgfqpoint{2.440563in}{1.207501in}}{\pgfqpoint{2.440563in}{1.199265in}}%
\pgfpathcurveto{\pgfqpoint{2.440563in}{1.191029in}}{\pgfqpoint{2.443835in}{1.183128in}}{\pgfqpoint{2.449659in}{1.177305in}}%
\pgfpathcurveto{\pgfqpoint{2.455483in}{1.171481in}}{\pgfqpoint{2.463383in}{1.168208in}}{\pgfqpoint{2.471619in}{1.168208in}}%
\pgfpathclose%
\pgfusepath{stroke,fill}%
\end{pgfscope}%
\begin{pgfscope}%
\pgfpathrectangle{\pgfqpoint{0.100000in}{0.212622in}}{\pgfqpoint{3.696000in}{3.696000in}}%
\pgfusepath{clip}%
\pgfsetbuttcap%
\pgfsetroundjoin%
\definecolor{currentfill}{rgb}{0.121569,0.466667,0.705882}%
\pgfsetfillcolor{currentfill}%
\pgfsetfillopacity{0.840230}%
\pgfsetlinewidth{1.003750pt}%
\definecolor{currentstroke}{rgb}{0.121569,0.466667,0.705882}%
\pgfsetstrokecolor{currentstroke}%
\pgfsetstrokeopacity{0.840230}%
\pgfsetdash{}{0pt}%
\pgfpathmoveto{\pgfqpoint{2.473826in}{1.167495in}}%
\pgfpathcurveto{\pgfqpoint{2.482062in}{1.167495in}}{\pgfqpoint{2.489962in}{1.170767in}}{\pgfqpoint{2.495786in}{1.176591in}}%
\pgfpathcurveto{\pgfqpoint{2.501610in}{1.182415in}}{\pgfqpoint{2.504882in}{1.190315in}}{\pgfqpoint{2.504882in}{1.198551in}}%
\pgfpathcurveto{\pgfqpoint{2.504882in}{1.206788in}}{\pgfqpoint{2.501610in}{1.214688in}}{\pgfqpoint{2.495786in}{1.220512in}}%
\pgfpathcurveto{\pgfqpoint{2.489962in}{1.226336in}}{\pgfqpoint{2.482062in}{1.229608in}}{\pgfqpoint{2.473826in}{1.229608in}}%
\pgfpathcurveto{\pgfqpoint{2.465590in}{1.229608in}}{\pgfqpoint{2.457689in}{1.226336in}}{\pgfqpoint{2.451866in}{1.220512in}}%
\pgfpathcurveto{\pgfqpoint{2.446042in}{1.214688in}}{\pgfqpoint{2.442769in}{1.206788in}}{\pgfqpoint{2.442769in}{1.198551in}}%
\pgfpathcurveto{\pgfqpoint{2.442769in}{1.190315in}}{\pgfqpoint{2.446042in}{1.182415in}}{\pgfqpoint{2.451866in}{1.176591in}}%
\pgfpathcurveto{\pgfqpoint{2.457689in}{1.170767in}}{\pgfqpoint{2.465590in}{1.167495in}}{\pgfqpoint{2.473826in}{1.167495in}}%
\pgfpathclose%
\pgfusepath{stroke,fill}%
\end{pgfscope}%
\begin{pgfscope}%
\pgfpathrectangle{\pgfqpoint{0.100000in}{0.212622in}}{\pgfqpoint{3.696000in}{3.696000in}}%
\pgfusepath{clip}%
\pgfsetbuttcap%
\pgfsetroundjoin%
\definecolor{currentfill}{rgb}{0.121569,0.466667,0.705882}%
\pgfsetfillcolor{currentfill}%
\pgfsetfillopacity{0.840779}%
\pgfsetlinewidth{1.003750pt}%
\definecolor{currentstroke}{rgb}{0.121569,0.466667,0.705882}%
\pgfsetstrokecolor{currentstroke}%
\pgfsetstrokeopacity{0.840779}%
\pgfsetdash{}{0pt}%
\pgfpathmoveto{\pgfqpoint{2.474977in}{1.167061in}}%
\pgfpathcurveto{\pgfqpoint{2.483214in}{1.167061in}}{\pgfqpoint{2.491114in}{1.170333in}}{\pgfqpoint{2.496938in}{1.176157in}}%
\pgfpathcurveto{\pgfqpoint{2.502762in}{1.181981in}}{\pgfqpoint{2.506034in}{1.189881in}}{\pgfqpoint{2.506034in}{1.198117in}}%
\pgfpathcurveto{\pgfqpoint{2.506034in}{1.206354in}}{\pgfqpoint{2.502762in}{1.214254in}}{\pgfqpoint{2.496938in}{1.220078in}}%
\pgfpathcurveto{\pgfqpoint{2.491114in}{1.225902in}}{\pgfqpoint{2.483214in}{1.229174in}}{\pgfqpoint{2.474977in}{1.229174in}}%
\pgfpathcurveto{\pgfqpoint{2.466741in}{1.229174in}}{\pgfqpoint{2.458841in}{1.225902in}}{\pgfqpoint{2.453017in}{1.220078in}}%
\pgfpathcurveto{\pgfqpoint{2.447193in}{1.214254in}}{\pgfqpoint{2.443921in}{1.206354in}}{\pgfqpoint{2.443921in}{1.198117in}}%
\pgfpathcurveto{\pgfqpoint{2.443921in}{1.189881in}}{\pgfqpoint{2.447193in}{1.181981in}}{\pgfqpoint{2.453017in}{1.176157in}}%
\pgfpathcurveto{\pgfqpoint{2.458841in}{1.170333in}}{\pgfqpoint{2.466741in}{1.167061in}}{\pgfqpoint{2.474977in}{1.167061in}}%
\pgfpathclose%
\pgfusepath{stroke,fill}%
\end{pgfscope}%
\begin{pgfscope}%
\pgfpathrectangle{\pgfqpoint{0.100000in}{0.212622in}}{\pgfqpoint{3.696000in}{3.696000in}}%
\pgfusepath{clip}%
\pgfsetbuttcap%
\pgfsetroundjoin%
\definecolor{currentfill}{rgb}{0.121569,0.466667,0.705882}%
\pgfsetfillcolor{currentfill}%
\pgfsetfillopacity{0.841708}%
\pgfsetlinewidth{1.003750pt}%
\definecolor{currentstroke}{rgb}{0.121569,0.466667,0.705882}%
\pgfsetstrokecolor{currentstroke}%
\pgfsetstrokeopacity{0.841708}%
\pgfsetdash{}{0pt}%
\pgfpathmoveto{\pgfqpoint{2.477704in}{1.166317in}}%
\pgfpathcurveto{\pgfqpoint{2.485940in}{1.166317in}}{\pgfqpoint{2.493840in}{1.169590in}}{\pgfqpoint{2.499664in}{1.175414in}}%
\pgfpathcurveto{\pgfqpoint{2.505488in}{1.181238in}}{\pgfqpoint{2.508760in}{1.189138in}}{\pgfqpoint{2.508760in}{1.197374in}}%
\pgfpathcurveto{\pgfqpoint{2.508760in}{1.205610in}}{\pgfqpoint{2.505488in}{1.213510in}}{\pgfqpoint{2.499664in}{1.219334in}}%
\pgfpathcurveto{\pgfqpoint{2.493840in}{1.225158in}}{\pgfqpoint{2.485940in}{1.228430in}}{\pgfqpoint{2.477704in}{1.228430in}}%
\pgfpathcurveto{\pgfqpoint{2.469467in}{1.228430in}}{\pgfqpoint{2.461567in}{1.225158in}}{\pgfqpoint{2.455743in}{1.219334in}}%
\pgfpathcurveto{\pgfqpoint{2.449920in}{1.213510in}}{\pgfqpoint{2.446647in}{1.205610in}}{\pgfqpoint{2.446647in}{1.197374in}}%
\pgfpathcurveto{\pgfqpoint{2.446647in}{1.189138in}}{\pgfqpoint{2.449920in}{1.181238in}}{\pgfqpoint{2.455743in}{1.175414in}}%
\pgfpathcurveto{\pgfqpoint{2.461567in}{1.169590in}}{\pgfqpoint{2.469467in}{1.166317in}}{\pgfqpoint{2.477704in}{1.166317in}}%
\pgfpathclose%
\pgfusepath{stroke,fill}%
\end{pgfscope}%
\begin{pgfscope}%
\pgfpathrectangle{\pgfqpoint{0.100000in}{0.212622in}}{\pgfqpoint{3.696000in}{3.696000in}}%
\pgfusepath{clip}%
\pgfsetbuttcap%
\pgfsetroundjoin%
\definecolor{currentfill}{rgb}{0.121569,0.466667,0.705882}%
\pgfsetfillcolor{currentfill}%
\pgfsetfillopacity{0.843761}%
\pgfsetlinewidth{1.003750pt}%
\definecolor{currentstroke}{rgb}{0.121569,0.466667,0.705882}%
\pgfsetstrokecolor{currentstroke}%
\pgfsetstrokeopacity{0.843761}%
\pgfsetdash{}{0pt}%
\pgfpathmoveto{\pgfqpoint{2.482889in}{1.164621in}}%
\pgfpathcurveto{\pgfqpoint{2.491125in}{1.164621in}}{\pgfqpoint{2.499025in}{1.167893in}}{\pgfqpoint{2.504849in}{1.173717in}}%
\pgfpathcurveto{\pgfqpoint{2.510673in}{1.179541in}}{\pgfqpoint{2.513945in}{1.187441in}}{\pgfqpoint{2.513945in}{1.195677in}}%
\pgfpathcurveto{\pgfqpoint{2.513945in}{1.203914in}}{\pgfqpoint{2.510673in}{1.211814in}}{\pgfqpoint{2.504849in}{1.217638in}}%
\pgfpathcurveto{\pgfqpoint{2.499025in}{1.223462in}}{\pgfqpoint{2.491125in}{1.226734in}}{\pgfqpoint{2.482889in}{1.226734in}}%
\pgfpathcurveto{\pgfqpoint{2.474653in}{1.226734in}}{\pgfqpoint{2.466753in}{1.223462in}}{\pgfqpoint{2.460929in}{1.217638in}}%
\pgfpathcurveto{\pgfqpoint{2.455105in}{1.211814in}}{\pgfqpoint{2.451832in}{1.203914in}}{\pgfqpoint{2.451832in}{1.195677in}}%
\pgfpathcurveto{\pgfqpoint{2.451832in}{1.187441in}}{\pgfqpoint{2.455105in}{1.179541in}}{\pgfqpoint{2.460929in}{1.173717in}}%
\pgfpathcurveto{\pgfqpoint{2.466753in}{1.167893in}}{\pgfqpoint{2.474653in}{1.164621in}}{\pgfqpoint{2.482889in}{1.164621in}}%
\pgfpathclose%
\pgfusepath{stroke,fill}%
\end{pgfscope}%
\begin{pgfscope}%
\pgfpathrectangle{\pgfqpoint{0.100000in}{0.212622in}}{\pgfqpoint{3.696000in}{3.696000in}}%
\pgfusepath{clip}%
\pgfsetbuttcap%
\pgfsetroundjoin%
\definecolor{currentfill}{rgb}{0.121569,0.466667,0.705882}%
\pgfsetfillcolor{currentfill}%
\pgfsetfillopacity{0.846185}%
\pgfsetlinewidth{1.003750pt}%
\definecolor{currentstroke}{rgb}{0.121569,0.466667,0.705882}%
\pgfsetstrokecolor{currentstroke}%
\pgfsetstrokeopacity{0.846185}%
\pgfsetdash{}{0pt}%
\pgfpathmoveto{\pgfqpoint{2.490247in}{1.162503in}}%
\pgfpathcurveto{\pgfqpoint{2.498484in}{1.162503in}}{\pgfqpoint{2.506384in}{1.165775in}}{\pgfqpoint{2.512208in}{1.171599in}}%
\pgfpathcurveto{\pgfqpoint{2.518031in}{1.177423in}}{\pgfqpoint{2.521304in}{1.185323in}}{\pgfqpoint{2.521304in}{1.193559in}}%
\pgfpathcurveto{\pgfqpoint{2.521304in}{1.201796in}}{\pgfqpoint{2.518031in}{1.209696in}}{\pgfqpoint{2.512208in}{1.215520in}}%
\pgfpathcurveto{\pgfqpoint{2.506384in}{1.221344in}}{\pgfqpoint{2.498484in}{1.224616in}}{\pgfqpoint{2.490247in}{1.224616in}}%
\pgfpathcurveto{\pgfqpoint{2.482011in}{1.224616in}}{\pgfqpoint{2.474111in}{1.221344in}}{\pgfqpoint{2.468287in}{1.215520in}}%
\pgfpathcurveto{\pgfqpoint{2.462463in}{1.209696in}}{\pgfqpoint{2.459191in}{1.201796in}}{\pgfqpoint{2.459191in}{1.193559in}}%
\pgfpathcurveto{\pgfqpoint{2.459191in}{1.185323in}}{\pgfqpoint{2.462463in}{1.177423in}}{\pgfqpoint{2.468287in}{1.171599in}}%
\pgfpathcurveto{\pgfqpoint{2.474111in}{1.165775in}}{\pgfqpoint{2.482011in}{1.162503in}}{\pgfqpoint{2.490247in}{1.162503in}}%
\pgfpathclose%
\pgfusepath{stroke,fill}%
\end{pgfscope}%
\begin{pgfscope}%
\pgfpathrectangle{\pgfqpoint{0.100000in}{0.212622in}}{\pgfqpoint{3.696000in}{3.696000in}}%
\pgfusepath{clip}%
\pgfsetbuttcap%
\pgfsetroundjoin%
\definecolor{currentfill}{rgb}{0.121569,0.466667,0.705882}%
\pgfsetfillcolor{currentfill}%
\pgfsetfillopacity{0.849089}%
\pgfsetlinewidth{1.003750pt}%
\definecolor{currentstroke}{rgb}{0.121569,0.466667,0.705882}%
\pgfsetstrokecolor{currentstroke}%
\pgfsetstrokeopacity{0.849089}%
\pgfsetdash{}{0pt}%
\pgfpathmoveto{\pgfqpoint{2.498296in}{1.160252in}}%
\pgfpathcurveto{\pgfqpoint{2.506532in}{1.160252in}}{\pgfqpoint{2.514433in}{1.163525in}}{\pgfqpoint{2.520256in}{1.169349in}}%
\pgfpathcurveto{\pgfqpoint{2.526080in}{1.175173in}}{\pgfqpoint{2.529353in}{1.183073in}}{\pgfqpoint{2.529353in}{1.191309in}}%
\pgfpathcurveto{\pgfqpoint{2.529353in}{1.199545in}}{\pgfqpoint{2.526080in}{1.207445in}}{\pgfqpoint{2.520256in}{1.213269in}}%
\pgfpathcurveto{\pgfqpoint{2.514433in}{1.219093in}}{\pgfqpoint{2.506532in}{1.222365in}}{\pgfqpoint{2.498296in}{1.222365in}}%
\pgfpathcurveto{\pgfqpoint{2.490060in}{1.222365in}}{\pgfqpoint{2.482160in}{1.219093in}}{\pgfqpoint{2.476336in}{1.213269in}}%
\pgfpathcurveto{\pgfqpoint{2.470512in}{1.207445in}}{\pgfqpoint{2.467240in}{1.199545in}}{\pgfqpoint{2.467240in}{1.191309in}}%
\pgfpathcurveto{\pgfqpoint{2.467240in}{1.183073in}}{\pgfqpoint{2.470512in}{1.175173in}}{\pgfqpoint{2.476336in}{1.169349in}}%
\pgfpathcurveto{\pgfqpoint{2.482160in}{1.163525in}}{\pgfqpoint{2.490060in}{1.160252in}}{\pgfqpoint{2.498296in}{1.160252in}}%
\pgfpathclose%
\pgfusepath{stroke,fill}%
\end{pgfscope}%
\begin{pgfscope}%
\pgfpathrectangle{\pgfqpoint{0.100000in}{0.212622in}}{\pgfqpoint{3.696000in}{3.696000in}}%
\pgfusepath{clip}%
\pgfsetbuttcap%
\pgfsetroundjoin%
\definecolor{currentfill}{rgb}{0.121569,0.466667,0.705882}%
\pgfsetfillcolor{currentfill}%
\pgfsetfillopacity{0.850611}%
\pgfsetlinewidth{1.003750pt}%
\definecolor{currentstroke}{rgb}{0.121569,0.466667,0.705882}%
\pgfsetstrokecolor{currentstroke}%
\pgfsetstrokeopacity{0.850611}%
\pgfsetdash{}{0pt}%
\pgfpathmoveto{\pgfqpoint{2.502768in}{1.159016in}}%
\pgfpathcurveto{\pgfqpoint{2.511004in}{1.159016in}}{\pgfqpoint{2.518905in}{1.162289in}}{\pgfqpoint{2.524728in}{1.168113in}}%
\pgfpathcurveto{\pgfqpoint{2.530552in}{1.173937in}}{\pgfqpoint{2.533825in}{1.181837in}}{\pgfqpoint{2.533825in}{1.190073in}}%
\pgfpathcurveto{\pgfqpoint{2.533825in}{1.198309in}}{\pgfqpoint{2.530552in}{1.206209in}}{\pgfqpoint{2.524728in}{1.212033in}}%
\pgfpathcurveto{\pgfqpoint{2.518905in}{1.217857in}}{\pgfqpoint{2.511004in}{1.221129in}}{\pgfqpoint{2.502768in}{1.221129in}}%
\pgfpathcurveto{\pgfqpoint{2.494532in}{1.221129in}}{\pgfqpoint{2.486632in}{1.217857in}}{\pgfqpoint{2.480808in}{1.212033in}}%
\pgfpathcurveto{\pgfqpoint{2.474984in}{1.206209in}}{\pgfqpoint{2.471712in}{1.198309in}}{\pgfqpoint{2.471712in}{1.190073in}}%
\pgfpathcurveto{\pgfqpoint{2.471712in}{1.181837in}}{\pgfqpoint{2.474984in}{1.173937in}}{\pgfqpoint{2.480808in}{1.168113in}}%
\pgfpathcurveto{\pgfqpoint{2.486632in}{1.162289in}}{\pgfqpoint{2.494532in}{1.159016in}}{\pgfqpoint{2.502768in}{1.159016in}}%
\pgfpathclose%
\pgfusepath{stroke,fill}%
\end{pgfscope}%
\begin{pgfscope}%
\pgfpathrectangle{\pgfqpoint{0.100000in}{0.212622in}}{\pgfqpoint{3.696000in}{3.696000in}}%
\pgfusepath{clip}%
\pgfsetbuttcap%
\pgfsetroundjoin%
\definecolor{currentfill}{rgb}{0.121569,0.466667,0.705882}%
\pgfsetfillcolor{currentfill}%
\pgfsetfillopacity{0.851543}%
\pgfsetlinewidth{1.003750pt}%
\definecolor{currentstroke}{rgb}{0.121569,0.466667,0.705882}%
\pgfsetstrokecolor{currentstroke}%
\pgfsetstrokeopacity{0.851543}%
\pgfsetdash{}{0pt}%
\pgfpathmoveto{\pgfqpoint{2.505161in}{1.158294in}}%
\pgfpathcurveto{\pgfqpoint{2.513398in}{1.158294in}}{\pgfqpoint{2.521298in}{1.161566in}}{\pgfqpoint{2.527121in}{1.167390in}}%
\pgfpathcurveto{\pgfqpoint{2.532945in}{1.173214in}}{\pgfqpoint{2.536218in}{1.181114in}}{\pgfqpoint{2.536218in}{1.189350in}}%
\pgfpathcurveto{\pgfqpoint{2.536218in}{1.197586in}}{\pgfqpoint{2.532945in}{1.205486in}}{\pgfqpoint{2.527121in}{1.211310in}}%
\pgfpathcurveto{\pgfqpoint{2.521298in}{1.217134in}}{\pgfqpoint{2.513398in}{1.220407in}}{\pgfqpoint{2.505161in}{1.220407in}}%
\pgfpathcurveto{\pgfqpoint{2.496925in}{1.220407in}}{\pgfqpoint{2.489025in}{1.217134in}}{\pgfqpoint{2.483201in}{1.211310in}}%
\pgfpathcurveto{\pgfqpoint{2.477377in}{1.205486in}}{\pgfqpoint{2.474105in}{1.197586in}}{\pgfqpoint{2.474105in}{1.189350in}}%
\pgfpathcurveto{\pgfqpoint{2.474105in}{1.181114in}}{\pgfqpoint{2.477377in}{1.173214in}}{\pgfqpoint{2.483201in}{1.167390in}}%
\pgfpathcurveto{\pgfqpoint{2.489025in}{1.161566in}}{\pgfqpoint{2.496925in}{1.158294in}}{\pgfqpoint{2.505161in}{1.158294in}}%
\pgfpathclose%
\pgfusepath{stroke,fill}%
\end{pgfscope}%
\begin{pgfscope}%
\pgfpathrectangle{\pgfqpoint{0.100000in}{0.212622in}}{\pgfqpoint{3.696000in}{3.696000in}}%
\pgfusepath{clip}%
\pgfsetbuttcap%
\pgfsetroundjoin%
\definecolor{currentfill}{rgb}{0.121569,0.466667,0.705882}%
\pgfsetfillcolor{currentfill}%
\pgfsetfillopacity{0.852624}%
\pgfsetlinewidth{1.003750pt}%
\definecolor{currentstroke}{rgb}{0.121569,0.466667,0.705882}%
\pgfsetstrokecolor{currentstroke}%
\pgfsetstrokeopacity{0.852624}%
\pgfsetdash{}{0pt}%
\pgfpathmoveto{\pgfqpoint{2.508013in}{1.157415in}}%
\pgfpathcurveto{\pgfqpoint{2.516249in}{1.157415in}}{\pgfqpoint{2.524149in}{1.160688in}}{\pgfqpoint{2.529973in}{1.166511in}}%
\pgfpathcurveto{\pgfqpoint{2.535797in}{1.172335in}}{\pgfqpoint{2.539069in}{1.180235in}}{\pgfqpoint{2.539069in}{1.188472in}}%
\pgfpathcurveto{\pgfqpoint{2.539069in}{1.196708in}}{\pgfqpoint{2.535797in}{1.204608in}}{\pgfqpoint{2.529973in}{1.210432in}}%
\pgfpathcurveto{\pgfqpoint{2.524149in}{1.216256in}}{\pgfqpoint{2.516249in}{1.219528in}}{\pgfqpoint{2.508013in}{1.219528in}}%
\pgfpathcurveto{\pgfqpoint{2.499777in}{1.219528in}}{\pgfqpoint{2.491876in}{1.216256in}}{\pgfqpoint{2.486053in}{1.210432in}}%
\pgfpathcurveto{\pgfqpoint{2.480229in}{1.204608in}}{\pgfqpoint{2.476956in}{1.196708in}}{\pgfqpoint{2.476956in}{1.188472in}}%
\pgfpathcurveto{\pgfqpoint{2.476956in}{1.180235in}}{\pgfqpoint{2.480229in}{1.172335in}}{\pgfqpoint{2.486053in}{1.166511in}}%
\pgfpathcurveto{\pgfqpoint{2.491876in}{1.160688in}}{\pgfqpoint{2.499777in}{1.157415in}}{\pgfqpoint{2.508013in}{1.157415in}}%
\pgfpathclose%
\pgfusepath{stroke,fill}%
\end{pgfscope}%
\begin{pgfscope}%
\pgfpathrectangle{\pgfqpoint{0.100000in}{0.212622in}}{\pgfqpoint{3.696000in}{3.696000in}}%
\pgfusepath{clip}%
\pgfsetbuttcap%
\pgfsetroundjoin%
\definecolor{currentfill}{rgb}{0.121569,0.466667,0.705882}%
\pgfsetfillcolor{currentfill}%
\pgfsetfillopacity{0.854183}%
\pgfsetlinewidth{1.003750pt}%
\definecolor{currentstroke}{rgb}{0.121569,0.466667,0.705882}%
\pgfsetstrokecolor{currentstroke}%
\pgfsetstrokeopacity{0.854183}%
\pgfsetdash{}{0pt}%
\pgfpathmoveto{\pgfqpoint{2.511885in}{1.156046in}}%
\pgfpathcurveto{\pgfqpoint{2.520122in}{1.156046in}}{\pgfqpoint{2.528022in}{1.159318in}}{\pgfqpoint{2.533846in}{1.165142in}}%
\pgfpathcurveto{\pgfqpoint{2.539670in}{1.170966in}}{\pgfqpoint{2.542942in}{1.178866in}}{\pgfqpoint{2.542942in}{1.187102in}}%
\pgfpathcurveto{\pgfqpoint{2.542942in}{1.195338in}}{\pgfqpoint{2.539670in}{1.203238in}}{\pgfqpoint{2.533846in}{1.209062in}}%
\pgfpathcurveto{\pgfqpoint{2.528022in}{1.214886in}}{\pgfqpoint{2.520122in}{1.218159in}}{\pgfqpoint{2.511885in}{1.218159in}}%
\pgfpathcurveto{\pgfqpoint{2.503649in}{1.218159in}}{\pgfqpoint{2.495749in}{1.214886in}}{\pgfqpoint{2.489925in}{1.209062in}}%
\pgfpathcurveto{\pgfqpoint{2.484101in}{1.203238in}}{\pgfqpoint{2.480829in}{1.195338in}}{\pgfqpoint{2.480829in}{1.187102in}}%
\pgfpathcurveto{\pgfqpoint{2.480829in}{1.178866in}}{\pgfqpoint{2.484101in}{1.170966in}}{\pgfqpoint{2.489925in}{1.165142in}}%
\pgfpathcurveto{\pgfqpoint{2.495749in}{1.159318in}}{\pgfqpoint{2.503649in}{1.156046in}}{\pgfqpoint{2.511885in}{1.156046in}}%
\pgfpathclose%
\pgfusepath{stroke,fill}%
\end{pgfscope}%
\begin{pgfscope}%
\pgfpathrectangle{\pgfqpoint{0.100000in}{0.212622in}}{\pgfqpoint{3.696000in}{3.696000in}}%
\pgfusepath{clip}%
\pgfsetbuttcap%
\pgfsetroundjoin%
\definecolor{currentfill}{rgb}{0.121569,0.466667,0.705882}%
\pgfsetfillcolor{currentfill}%
\pgfsetfillopacity{0.856387}%
\pgfsetlinewidth{1.003750pt}%
\definecolor{currentstroke}{rgb}{0.121569,0.466667,0.705882}%
\pgfsetstrokecolor{currentstroke}%
\pgfsetstrokeopacity{0.856387}%
\pgfsetdash{}{0pt}%
\pgfpathmoveto{\pgfqpoint{2.516413in}{1.154152in}}%
\pgfpathcurveto{\pgfqpoint{2.524649in}{1.154152in}}{\pgfqpoint{2.532550in}{1.157424in}}{\pgfqpoint{2.538373in}{1.163248in}}%
\pgfpathcurveto{\pgfqpoint{2.544197in}{1.169072in}}{\pgfqpoint{2.547470in}{1.176972in}}{\pgfqpoint{2.547470in}{1.185209in}}%
\pgfpathcurveto{\pgfqpoint{2.547470in}{1.193445in}}{\pgfqpoint{2.544197in}{1.201345in}}{\pgfqpoint{2.538373in}{1.207169in}}%
\pgfpathcurveto{\pgfqpoint{2.532550in}{1.212993in}}{\pgfqpoint{2.524649in}{1.216265in}}{\pgfqpoint{2.516413in}{1.216265in}}%
\pgfpathcurveto{\pgfqpoint{2.508177in}{1.216265in}}{\pgfqpoint{2.500277in}{1.212993in}}{\pgfqpoint{2.494453in}{1.207169in}}%
\pgfpathcurveto{\pgfqpoint{2.488629in}{1.201345in}}{\pgfqpoint{2.485357in}{1.193445in}}{\pgfqpoint{2.485357in}{1.185209in}}%
\pgfpathcurveto{\pgfqpoint{2.485357in}{1.176972in}}{\pgfqpoint{2.488629in}{1.169072in}}{\pgfqpoint{2.494453in}{1.163248in}}%
\pgfpathcurveto{\pgfqpoint{2.500277in}{1.157424in}}{\pgfqpoint{2.508177in}{1.154152in}}{\pgfqpoint{2.516413in}{1.154152in}}%
\pgfpathclose%
\pgfusepath{stroke,fill}%
\end{pgfscope}%
\begin{pgfscope}%
\pgfpathrectangle{\pgfqpoint{0.100000in}{0.212622in}}{\pgfqpoint{3.696000in}{3.696000in}}%
\pgfusepath{clip}%
\pgfsetbuttcap%
\pgfsetroundjoin%
\definecolor{currentfill}{rgb}{0.121569,0.466667,0.705882}%
\pgfsetfillcolor{currentfill}%
\pgfsetfillopacity{0.858548}%
\pgfsetlinewidth{1.003750pt}%
\definecolor{currentstroke}{rgb}{0.121569,0.466667,0.705882}%
\pgfsetstrokecolor{currentstroke}%
\pgfsetstrokeopacity{0.858548}%
\pgfsetdash{}{0pt}%
\pgfpathmoveto{\pgfqpoint{2.522546in}{1.152485in}}%
\pgfpathcurveto{\pgfqpoint{2.530782in}{1.152485in}}{\pgfqpoint{2.538682in}{1.155758in}}{\pgfqpoint{2.544506in}{1.161582in}}%
\pgfpathcurveto{\pgfqpoint{2.550330in}{1.167405in}}{\pgfqpoint{2.553602in}{1.175305in}}{\pgfqpoint{2.553602in}{1.183542in}}%
\pgfpathcurveto{\pgfqpoint{2.553602in}{1.191778in}}{\pgfqpoint{2.550330in}{1.199678in}}{\pgfqpoint{2.544506in}{1.205502in}}%
\pgfpathcurveto{\pgfqpoint{2.538682in}{1.211326in}}{\pgfqpoint{2.530782in}{1.214598in}}{\pgfqpoint{2.522546in}{1.214598in}}%
\pgfpathcurveto{\pgfqpoint{2.514309in}{1.214598in}}{\pgfqpoint{2.506409in}{1.211326in}}{\pgfqpoint{2.500585in}{1.205502in}}%
\pgfpathcurveto{\pgfqpoint{2.494761in}{1.199678in}}{\pgfqpoint{2.491489in}{1.191778in}}{\pgfqpoint{2.491489in}{1.183542in}}%
\pgfpathcurveto{\pgfqpoint{2.491489in}{1.175305in}}{\pgfqpoint{2.494761in}{1.167405in}}{\pgfqpoint{2.500585in}{1.161582in}}%
\pgfpathcurveto{\pgfqpoint{2.506409in}{1.155758in}}{\pgfqpoint{2.514309in}{1.152485in}}{\pgfqpoint{2.522546in}{1.152485in}}%
\pgfpathclose%
\pgfusepath{stroke,fill}%
\end{pgfscope}%
\begin{pgfscope}%
\pgfpathrectangle{\pgfqpoint{0.100000in}{0.212622in}}{\pgfqpoint{3.696000in}{3.696000in}}%
\pgfusepath{clip}%
\pgfsetbuttcap%
\pgfsetroundjoin%
\definecolor{currentfill}{rgb}{0.121569,0.466667,0.705882}%
\pgfsetfillcolor{currentfill}%
\pgfsetfillopacity{0.861441}%
\pgfsetlinewidth{1.003750pt}%
\definecolor{currentstroke}{rgb}{0.121569,0.466667,0.705882}%
\pgfsetstrokecolor{currentstroke}%
\pgfsetstrokeopacity{0.861441}%
\pgfsetdash{}{0pt}%
\pgfpathmoveto{\pgfqpoint{2.530022in}{1.150271in}}%
\pgfpathcurveto{\pgfqpoint{2.538258in}{1.150271in}}{\pgfqpoint{2.546158in}{1.153544in}}{\pgfqpoint{2.551982in}{1.159368in}}%
\pgfpathcurveto{\pgfqpoint{2.557806in}{1.165192in}}{\pgfqpoint{2.561078in}{1.173092in}}{\pgfqpoint{2.561078in}{1.181328in}}%
\pgfpathcurveto{\pgfqpoint{2.561078in}{1.189564in}}{\pgfqpoint{2.557806in}{1.197464in}}{\pgfqpoint{2.551982in}{1.203288in}}%
\pgfpathcurveto{\pgfqpoint{2.546158in}{1.209112in}}{\pgfqpoint{2.538258in}{1.212384in}}{\pgfqpoint{2.530022in}{1.212384in}}%
\pgfpathcurveto{\pgfqpoint{2.521785in}{1.212384in}}{\pgfqpoint{2.513885in}{1.209112in}}{\pgfqpoint{2.508061in}{1.203288in}}%
\pgfpathcurveto{\pgfqpoint{2.502237in}{1.197464in}}{\pgfqpoint{2.498965in}{1.189564in}}{\pgfqpoint{2.498965in}{1.181328in}}%
\pgfpathcurveto{\pgfqpoint{2.498965in}{1.173092in}}{\pgfqpoint{2.502237in}{1.165192in}}{\pgfqpoint{2.508061in}{1.159368in}}%
\pgfpathcurveto{\pgfqpoint{2.513885in}{1.153544in}}{\pgfqpoint{2.521785in}{1.150271in}}{\pgfqpoint{2.530022in}{1.150271in}}%
\pgfpathclose%
\pgfusepath{stroke,fill}%
\end{pgfscope}%
\begin{pgfscope}%
\pgfpathrectangle{\pgfqpoint{0.100000in}{0.212622in}}{\pgfqpoint{3.696000in}{3.696000in}}%
\pgfusepath{clip}%
\pgfsetbuttcap%
\pgfsetroundjoin%
\definecolor{currentfill}{rgb}{0.121569,0.466667,0.705882}%
\pgfsetfillcolor{currentfill}%
\pgfsetfillopacity{0.864368}%
\pgfsetlinewidth{1.003750pt}%
\definecolor{currentstroke}{rgb}{0.121569,0.466667,0.705882}%
\pgfsetstrokecolor{currentstroke}%
\pgfsetstrokeopacity{0.864368}%
\pgfsetdash{}{0pt}%
\pgfpathmoveto{\pgfqpoint{2.539084in}{1.147657in}}%
\pgfpathcurveto{\pgfqpoint{2.547321in}{1.147657in}}{\pgfqpoint{2.555221in}{1.150930in}}{\pgfqpoint{2.561045in}{1.156753in}}%
\pgfpathcurveto{\pgfqpoint{2.566869in}{1.162577in}}{\pgfqpoint{2.570141in}{1.170477in}}{\pgfqpoint{2.570141in}{1.178714in}}%
\pgfpathcurveto{\pgfqpoint{2.570141in}{1.186950in}}{\pgfqpoint{2.566869in}{1.194850in}}{\pgfqpoint{2.561045in}{1.200674in}}%
\pgfpathcurveto{\pgfqpoint{2.555221in}{1.206498in}}{\pgfqpoint{2.547321in}{1.209770in}}{\pgfqpoint{2.539084in}{1.209770in}}%
\pgfpathcurveto{\pgfqpoint{2.530848in}{1.209770in}}{\pgfqpoint{2.522948in}{1.206498in}}{\pgfqpoint{2.517124in}{1.200674in}}%
\pgfpathcurveto{\pgfqpoint{2.511300in}{1.194850in}}{\pgfqpoint{2.508028in}{1.186950in}}{\pgfqpoint{2.508028in}{1.178714in}}%
\pgfpathcurveto{\pgfqpoint{2.508028in}{1.170477in}}{\pgfqpoint{2.511300in}{1.162577in}}{\pgfqpoint{2.517124in}{1.156753in}}%
\pgfpathcurveto{\pgfqpoint{2.522948in}{1.150930in}}{\pgfqpoint{2.530848in}{1.147657in}}{\pgfqpoint{2.539084in}{1.147657in}}%
\pgfpathclose%
\pgfusepath{stroke,fill}%
\end{pgfscope}%
\begin{pgfscope}%
\pgfpathrectangle{\pgfqpoint{0.100000in}{0.212622in}}{\pgfqpoint{3.696000in}{3.696000in}}%
\pgfusepath{clip}%
\pgfsetbuttcap%
\pgfsetroundjoin%
\definecolor{currentfill}{rgb}{0.121569,0.466667,0.705882}%
\pgfsetfillcolor{currentfill}%
\pgfsetfillopacity{0.868030}%
\pgfsetlinewidth{1.003750pt}%
\definecolor{currentstroke}{rgb}{0.121569,0.466667,0.705882}%
\pgfsetstrokecolor{currentstroke}%
\pgfsetstrokeopacity{0.868030}%
\pgfsetdash{}{0pt}%
\pgfpathmoveto{\pgfqpoint{2.549298in}{1.144781in}}%
\pgfpathcurveto{\pgfqpoint{2.557534in}{1.144781in}}{\pgfqpoint{2.565434in}{1.148054in}}{\pgfqpoint{2.571258in}{1.153878in}}%
\pgfpathcurveto{\pgfqpoint{2.577082in}{1.159702in}}{\pgfqpoint{2.580354in}{1.167602in}}{\pgfqpoint{2.580354in}{1.175838in}}%
\pgfpathcurveto{\pgfqpoint{2.580354in}{1.184074in}}{\pgfqpoint{2.577082in}{1.191974in}}{\pgfqpoint{2.571258in}{1.197798in}}%
\pgfpathcurveto{\pgfqpoint{2.565434in}{1.203622in}}{\pgfqpoint{2.557534in}{1.206894in}}{\pgfqpoint{2.549298in}{1.206894in}}%
\pgfpathcurveto{\pgfqpoint{2.541061in}{1.206894in}}{\pgfqpoint{2.533161in}{1.203622in}}{\pgfqpoint{2.527338in}{1.197798in}}%
\pgfpathcurveto{\pgfqpoint{2.521514in}{1.191974in}}{\pgfqpoint{2.518241in}{1.184074in}}{\pgfqpoint{2.518241in}{1.175838in}}%
\pgfpathcurveto{\pgfqpoint{2.518241in}{1.167602in}}{\pgfqpoint{2.521514in}{1.159702in}}{\pgfqpoint{2.527338in}{1.153878in}}%
\pgfpathcurveto{\pgfqpoint{2.533161in}{1.148054in}}{\pgfqpoint{2.541061in}{1.144781in}}{\pgfqpoint{2.549298in}{1.144781in}}%
\pgfpathclose%
\pgfusepath{stroke,fill}%
\end{pgfscope}%
\begin{pgfscope}%
\pgfpathrectangle{\pgfqpoint{0.100000in}{0.212622in}}{\pgfqpoint{3.696000in}{3.696000in}}%
\pgfusepath{clip}%
\pgfsetbuttcap%
\pgfsetroundjoin%
\definecolor{currentfill}{rgb}{0.121569,0.466667,0.705882}%
\pgfsetfillcolor{currentfill}%
\pgfsetfillopacity{0.870177}%
\pgfsetlinewidth{1.003750pt}%
\definecolor{currentstroke}{rgb}{0.121569,0.466667,0.705882}%
\pgfsetstrokecolor{currentstroke}%
\pgfsetstrokeopacity{0.870177}%
\pgfsetdash{}{0pt}%
\pgfpathmoveto{\pgfqpoint{2.554740in}{1.142848in}}%
\pgfpathcurveto{\pgfqpoint{2.562976in}{1.142848in}}{\pgfqpoint{2.570876in}{1.146120in}}{\pgfqpoint{2.576700in}{1.151944in}}%
\pgfpathcurveto{\pgfqpoint{2.582524in}{1.157768in}}{\pgfqpoint{2.585796in}{1.165668in}}{\pgfqpoint{2.585796in}{1.173904in}}%
\pgfpathcurveto{\pgfqpoint{2.585796in}{1.182141in}}{\pgfqpoint{2.582524in}{1.190041in}}{\pgfqpoint{2.576700in}{1.195865in}}%
\pgfpathcurveto{\pgfqpoint{2.570876in}{1.201689in}}{\pgfqpoint{2.562976in}{1.204961in}}{\pgfqpoint{2.554740in}{1.204961in}}%
\pgfpathcurveto{\pgfqpoint{2.546503in}{1.204961in}}{\pgfqpoint{2.538603in}{1.201689in}}{\pgfqpoint{2.532779in}{1.195865in}}%
\pgfpathcurveto{\pgfqpoint{2.526955in}{1.190041in}}{\pgfqpoint{2.523683in}{1.182141in}}{\pgfqpoint{2.523683in}{1.173904in}}%
\pgfpathcurveto{\pgfqpoint{2.523683in}{1.165668in}}{\pgfqpoint{2.526955in}{1.157768in}}{\pgfqpoint{2.532779in}{1.151944in}}%
\pgfpathcurveto{\pgfqpoint{2.538603in}{1.146120in}}{\pgfqpoint{2.546503in}{1.142848in}}{\pgfqpoint{2.554740in}{1.142848in}}%
\pgfpathclose%
\pgfusepath{stroke,fill}%
\end{pgfscope}%
\begin{pgfscope}%
\pgfpathrectangle{\pgfqpoint{0.100000in}{0.212622in}}{\pgfqpoint{3.696000in}{3.696000in}}%
\pgfusepath{clip}%
\pgfsetbuttcap%
\pgfsetroundjoin%
\definecolor{currentfill}{rgb}{0.121569,0.466667,0.705882}%
\pgfsetfillcolor{currentfill}%
\pgfsetfillopacity{0.872796}%
\pgfsetlinewidth{1.003750pt}%
\definecolor{currentstroke}{rgb}{0.121569,0.466667,0.705882}%
\pgfsetstrokecolor{currentstroke}%
\pgfsetstrokeopacity{0.872796}%
\pgfsetdash{}{0pt}%
\pgfpathmoveto{\pgfqpoint{2.560479in}{1.140573in}}%
\pgfpathcurveto{\pgfqpoint{2.568715in}{1.140573in}}{\pgfqpoint{2.576615in}{1.143845in}}{\pgfqpoint{2.582439in}{1.149669in}}%
\pgfpathcurveto{\pgfqpoint{2.588263in}{1.155493in}}{\pgfqpoint{2.591536in}{1.163393in}}{\pgfqpoint{2.591536in}{1.171629in}}%
\pgfpathcurveto{\pgfqpoint{2.591536in}{1.179865in}}{\pgfqpoint{2.588263in}{1.187765in}}{\pgfqpoint{2.582439in}{1.193589in}}%
\pgfpathcurveto{\pgfqpoint{2.576615in}{1.199413in}}{\pgfqpoint{2.568715in}{1.202686in}}{\pgfqpoint{2.560479in}{1.202686in}}%
\pgfpathcurveto{\pgfqpoint{2.552243in}{1.202686in}}{\pgfqpoint{2.544343in}{1.199413in}}{\pgfqpoint{2.538519in}{1.193589in}}%
\pgfpathcurveto{\pgfqpoint{2.532695in}{1.187765in}}{\pgfqpoint{2.529423in}{1.179865in}}{\pgfqpoint{2.529423in}{1.171629in}}%
\pgfpathcurveto{\pgfqpoint{2.529423in}{1.163393in}}{\pgfqpoint{2.532695in}{1.155493in}}{\pgfqpoint{2.538519in}{1.149669in}}%
\pgfpathcurveto{\pgfqpoint{2.544343in}{1.143845in}}{\pgfqpoint{2.552243in}{1.140573in}}{\pgfqpoint{2.560479in}{1.140573in}}%
\pgfpathclose%
\pgfusepath{stroke,fill}%
\end{pgfscope}%
\begin{pgfscope}%
\pgfpathrectangle{\pgfqpoint{0.100000in}{0.212622in}}{\pgfqpoint{3.696000in}{3.696000in}}%
\pgfusepath{clip}%
\pgfsetbuttcap%
\pgfsetroundjoin%
\definecolor{currentfill}{rgb}{0.121569,0.466667,0.705882}%
\pgfsetfillcolor{currentfill}%
\pgfsetfillopacity{0.875394}%
\pgfsetlinewidth{1.003750pt}%
\definecolor{currentstroke}{rgb}{0.121569,0.466667,0.705882}%
\pgfsetstrokecolor{currentstroke}%
\pgfsetstrokeopacity{0.875394}%
\pgfsetdash{}{0pt}%
\pgfpathmoveto{\pgfqpoint{2.567448in}{1.138456in}}%
\pgfpathcurveto{\pgfqpoint{2.575685in}{1.138456in}}{\pgfqpoint{2.583585in}{1.141729in}}{\pgfqpoint{2.589409in}{1.147552in}}%
\pgfpathcurveto{\pgfqpoint{2.595232in}{1.153376in}}{\pgfqpoint{2.598505in}{1.161276in}}{\pgfqpoint{2.598505in}{1.169513in}}%
\pgfpathcurveto{\pgfqpoint{2.598505in}{1.177749in}}{\pgfqpoint{2.595232in}{1.185649in}}{\pgfqpoint{2.589409in}{1.191473in}}%
\pgfpathcurveto{\pgfqpoint{2.583585in}{1.197297in}}{\pgfqpoint{2.575685in}{1.200569in}}{\pgfqpoint{2.567448in}{1.200569in}}%
\pgfpathcurveto{\pgfqpoint{2.559212in}{1.200569in}}{\pgfqpoint{2.551312in}{1.197297in}}{\pgfqpoint{2.545488in}{1.191473in}}%
\pgfpathcurveto{\pgfqpoint{2.539664in}{1.185649in}}{\pgfqpoint{2.536392in}{1.177749in}}{\pgfqpoint{2.536392in}{1.169513in}}%
\pgfpathcurveto{\pgfqpoint{2.536392in}{1.161276in}}{\pgfqpoint{2.539664in}{1.153376in}}{\pgfqpoint{2.545488in}{1.147552in}}%
\pgfpathcurveto{\pgfqpoint{2.551312in}{1.141729in}}{\pgfqpoint{2.559212in}{1.138456in}}{\pgfqpoint{2.567448in}{1.138456in}}%
\pgfpathclose%
\pgfusepath{stroke,fill}%
\end{pgfscope}%
\begin{pgfscope}%
\pgfpathrectangle{\pgfqpoint{0.100000in}{0.212622in}}{\pgfqpoint{3.696000in}{3.696000in}}%
\pgfusepath{clip}%
\pgfsetbuttcap%
\pgfsetroundjoin%
\definecolor{currentfill}{rgb}{0.121569,0.466667,0.705882}%
\pgfsetfillcolor{currentfill}%
\pgfsetfillopacity{0.878886}%
\pgfsetlinewidth{1.003750pt}%
\definecolor{currentstroke}{rgb}{0.121569,0.466667,0.705882}%
\pgfsetstrokecolor{currentstroke}%
\pgfsetstrokeopacity{0.878886}%
\pgfsetdash{}{0pt}%
\pgfpathmoveto{\pgfqpoint{2.575062in}{1.135433in}}%
\pgfpathcurveto{\pgfqpoint{2.583298in}{1.135433in}}{\pgfqpoint{2.591198in}{1.138706in}}{\pgfqpoint{2.597022in}{1.144529in}}%
\pgfpathcurveto{\pgfqpoint{2.602846in}{1.150353in}}{\pgfqpoint{2.606118in}{1.158253in}}{\pgfqpoint{2.606118in}{1.166490in}}%
\pgfpathcurveto{\pgfqpoint{2.606118in}{1.174726in}}{\pgfqpoint{2.602846in}{1.182626in}}{\pgfqpoint{2.597022in}{1.188450in}}%
\pgfpathcurveto{\pgfqpoint{2.591198in}{1.194274in}}{\pgfqpoint{2.583298in}{1.197546in}}{\pgfqpoint{2.575062in}{1.197546in}}%
\pgfpathcurveto{\pgfqpoint{2.566825in}{1.197546in}}{\pgfqpoint{2.558925in}{1.194274in}}{\pgfqpoint{2.553101in}{1.188450in}}%
\pgfpathcurveto{\pgfqpoint{2.547277in}{1.182626in}}{\pgfqpoint{2.544005in}{1.174726in}}{\pgfqpoint{2.544005in}{1.166490in}}%
\pgfpathcurveto{\pgfqpoint{2.544005in}{1.158253in}}{\pgfqpoint{2.547277in}{1.150353in}}{\pgfqpoint{2.553101in}{1.144529in}}%
\pgfpathcurveto{\pgfqpoint{2.558925in}{1.138706in}}{\pgfqpoint{2.566825in}{1.135433in}}{\pgfqpoint{2.575062in}{1.135433in}}%
\pgfpathclose%
\pgfusepath{stroke,fill}%
\end{pgfscope}%
\begin{pgfscope}%
\pgfpathrectangle{\pgfqpoint{0.100000in}{0.212622in}}{\pgfqpoint{3.696000in}{3.696000in}}%
\pgfusepath{clip}%
\pgfsetbuttcap%
\pgfsetroundjoin%
\definecolor{currentfill}{rgb}{0.121569,0.466667,0.705882}%
\pgfsetfillcolor{currentfill}%
\pgfsetfillopacity{0.883530}%
\pgfsetlinewidth{1.003750pt}%
\definecolor{currentstroke}{rgb}{0.121569,0.466667,0.705882}%
\pgfsetstrokecolor{currentstroke}%
\pgfsetstrokeopacity{0.883530}%
\pgfsetdash{}{0pt}%
\pgfpathmoveto{\pgfqpoint{2.584396in}{1.131529in}}%
\pgfpathcurveto{\pgfqpoint{2.592632in}{1.131529in}}{\pgfqpoint{2.600532in}{1.134802in}}{\pgfqpoint{2.606356in}{1.140625in}}%
\pgfpathcurveto{\pgfqpoint{2.612180in}{1.146449in}}{\pgfqpoint{2.615453in}{1.154349in}}{\pgfqpoint{2.615453in}{1.162586in}}%
\pgfpathcurveto{\pgfqpoint{2.615453in}{1.170822in}}{\pgfqpoint{2.612180in}{1.178722in}}{\pgfqpoint{2.606356in}{1.184546in}}%
\pgfpathcurveto{\pgfqpoint{2.600532in}{1.190370in}}{\pgfqpoint{2.592632in}{1.193642in}}{\pgfqpoint{2.584396in}{1.193642in}}%
\pgfpathcurveto{\pgfqpoint{2.576160in}{1.193642in}}{\pgfqpoint{2.568260in}{1.190370in}}{\pgfqpoint{2.562436in}{1.184546in}}%
\pgfpathcurveto{\pgfqpoint{2.556612in}{1.178722in}}{\pgfqpoint{2.553340in}{1.170822in}}{\pgfqpoint{2.553340in}{1.162586in}}%
\pgfpathcurveto{\pgfqpoint{2.553340in}{1.154349in}}{\pgfqpoint{2.556612in}{1.146449in}}{\pgfqpoint{2.562436in}{1.140625in}}%
\pgfpathcurveto{\pgfqpoint{2.568260in}{1.134802in}}{\pgfqpoint{2.576160in}{1.131529in}}{\pgfqpoint{2.584396in}{1.131529in}}%
\pgfpathclose%
\pgfusepath{stroke,fill}%
\end{pgfscope}%
\begin{pgfscope}%
\pgfpathrectangle{\pgfqpoint{0.100000in}{0.212622in}}{\pgfqpoint{3.696000in}{3.696000in}}%
\pgfusepath{clip}%
\pgfsetbuttcap%
\pgfsetroundjoin%
\definecolor{currentfill}{rgb}{0.121569,0.466667,0.705882}%
\pgfsetfillcolor{currentfill}%
\pgfsetfillopacity{0.887807}%
\pgfsetlinewidth{1.003750pt}%
\definecolor{currentstroke}{rgb}{0.121569,0.466667,0.705882}%
\pgfsetstrokecolor{currentstroke}%
\pgfsetstrokeopacity{0.887807}%
\pgfsetdash{}{0pt}%
\pgfpathmoveto{\pgfqpoint{2.596381in}{1.128500in}}%
\pgfpathcurveto{\pgfqpoint{2.604617in}{1.128500in}}{\pgfqpoint{2.612517in}{1.131772in}}{\pgfqpoint{2.618341in}{1.137596in}}%
\pgfpathcurveto{\pgfqpoint{2.624165in}{1.143420in}}{\pgfqpoint{2.627437in}{1.151320in}}{\pgfqpoint{2.627437in}{1.159556in}}%
\pgfpathcurveto{\pgfqpoint{2.627437in}{1.167793in}}{\pgfqpoint{2.624165in}{1.175693in}}{\pgfqpoint{2.618341in}{1.181517in}}%
\pgfpathcurveto{\pgfqpoint{2.612517in}{1.187341in}}{\pgfqpoint{2.604617in}{1.190613in}}{\pgfqpoint{2.596381in}{1.190613in}}%
\pgfpathcurveto{\pgfqpoint{2.588144in}{1.190613in}}{\pgfqpoint{2.580244in}{1.187341in}}{\pgfqpoint{2.574420in}{1.181517in}}%
\pgfpathcurveto{\pgfqpoint{2.568596in}{1.175693in}}{\pgfqpoint{2.565324in}{1.167793in}}{\pgfqpoint{2.565324in}{1.159556in}}%
\pgfpathcurveto{\pgfqpoint{2.565324in}{1.151320in}}{\pgfqpoint{2.568596in}{1.143420in}}{\pgfqpoint{2.574420in}{1.137596in}}%
\pgfpathcurveto{\pgfqpoint{2.580244in}{1.131772in}}{\pgfqpoint{2.588144in}{1.128500in}}{\pgfqpoint{2.596381in}{1.128500in}}%
\pgfpathclose%
\pgfusepath{stroke,fill}%
\end{pgfscope}%
\begin{pgfscope}%
\pgfpathrectangle{\pgfqpoint{0.100000in}{0.212622in}}{\pgfqpoint{3.696000in}{3.696000in}}%
\pgfusepath{clip}%
\pgfsetbuttcap%
\pgfsetroundjoin%
\definecolor{currentfill}{rgb}{0.121569,0.466667,0.705882}%
\pgfsetfillcolor{currentfill}%
\pgfsetfillopacity{0.892456}%
\pgfsetlinewidth{1.003750pt}%
\definecolor{currentstroke}{rgb}{0.121569,0.466667,0.705882}%
\pgfsetstrokecolor{currentstroke}%
\pgfsetstrokeopacity{0.892456}%
\pgfsetdash{}{0pt}%
\pgfpathmoveto{\pgfqpoint{2.609261in}{1.124761in}}%
\pgfpathcurveto{\pgfqpoint{2.617497in}{1.124761in}}{\pgfqpoint{2.625397in}{1.128033in}}{\pgfqpoint{2.631221in}{1.133857in}}%
\pgfpathcurveto{\pgfqpoint{2.637045in}{1.139681in}}{\pgfqpoint{2.640317in}{1.147581in}}{\pgfqpoint{2.640317in}{1.155817in}}%
\pgfpathcurveto{\pgfqpoint{2.640317in}{1.164054in}}{\pgfqpoint{2.637045in}{1.171954in}}{\pgfqpoint{2.631221in}{1.177778in}}%
\pgfpathcurveto{\pgfqpoint{2.625397in}{1.183602in}}{\pgfqpoint{2.617497in}{1.186874in}}{\pgfqpoint{2.609261in}{1.186874in}}%
\pgfpathcurveto{\pgfqpoint{2.601024in}{1.186874in}}{\pgfqpoint{2.593124in}{1.183602in}}{\pgfqpoint{2.587300in}{1.177778in}}%
\pgfpathcurveto{\pgfqpoint{2.581476in}{1.171954in}}{\pgfqpoint{2.578204in}{1.164054in}}{\pgfqpoint{2.578204in}{1.155817in}}%
\pgfpathcurveto{\pgfqpoint{2.578204in}{1.147581in}}{\pgfqpoint{2.581476in}{1.139681in}}{\pgfqpoint{2.587300in}{1.133857in}}%
\pgfpathcurveto{\pgfqpoint{2.593124in}{1.128033in}}{\pgfqpoint{2.601024in}{1.124761in}}{\pgfqpoint{2.609261in}{1.124761in}}%
\pgfpathclose%
\pgfusepath{stroke,fill}%
\end{pgfscope}%
\begin{pgfscope}%
\pgfpathrectangle{\pgfqpoint{0.100000in}{0.212622in}}{\pgfqpoint{3.696000in}{3.696000in}}%
\pgfusepath{clip}%
\pgfsetbuttcap%
\pgfsetroundjoin%
\definecolor{currentfill}{rgb}{0.121569,0.466667,0.705882}%
\pgfsetfillcolor{currentfill}%
\pgfsetfillopacity{0.897036}%
\pgfsetlinewidth{1.003750pt}%
\definecolor{currentstroke}{rgb}{0.121569,0.466667,0.705882}%
\pgfsetstrokecolor{currentstroke}%
\pgfsetstrokeopacity{0.897036}%
\pgfsetdash{}{0pt}%
\pgfpathmoveto{\pgfqpoint{2.623050in}{1.121299in}}%
\pgfpathcurveto{\pgfqpoint{2.631287in}{1.121299in}}{\pgfqpoint{2.639187in}{1.124572in}}{\pgfqpoint{2.645011in}{1.130396in}}%
\pgfpathcurveto{\pgfqpoint{2.650835in}{1.136220in}}{\pgfqpoint{2.654107in}{1.144120in}}{\pgfqpoint{2.654107in}{1.152356in}}%
\pgfpathcurveto{\pgfqpoint{2.654107in}{1.160592in}}{\pgfqpoint{2.650835in}{1.168492in}}{\pgfqpoint{2.645011in}{1.174316in}}%
\pgfpathcurveto{\pgfqpoint{2.639187in}{1.180140in}}{\pgfqpoint{2.631287in}{1.183412in}}{\pgfqpoint{2.623050in}{1.183412in}}%
\pgfpathcurveto{\pgfqpoint{2.614814in}{1.183412in}}{\pgfqpoint{2.606914in}{1.180140in}}{\pgfqpoint{2.601090in}{1.174316in}}%
\pgfpathcurveto{\pgfqpoint{2.595266in}{1.168492in}}{\pgfqpoint{2.591994in}{1.160592in}}{\pgfqpoint{2.591994in}{1.152356in}}%
\pgfpathcurveto{\pgfqpoint{2.591994in}{1.144120in}}{\pgfqpoint{2.595266in}{1.136220in}}{\pgfqpoint{2.601090in}{1.130396in}}%
\pgfpathcurveto{\pgfqpoint{2.606914in}{1.124572in}}{\pgfqpoint{2.614814in}{1.121299in}}{\pgfqpoint{2.623050in}{1.121299in}}%
\pgfpathclose%
\pgfusepath{stroke,fill}%
\end{pgfscope}%
\begin{pgfscope}%
\pgfpathrectangle{\pgfqpoint{0.100000in}{0.212622in}}{\pgfqpoint{3.696000in}{3.696000in}}%
\pgfusepath{clip}%
\pgfsetbuttcap%
\pgfsetroundjoin%
\definecolor{currentfill}{rgb}{0.121569,0.466667,0.705882}%
\pgfsetfillcolor{currentfill}%
\pgfsetfillopacity{0.901261}%
\pgfsetlinewidth{1.003750pt}%
\definecolor{currentstroke}{rgb}{0.121569,0.466667,0.705882}%
\pgfsetstrokecolor{currentstroke}%
\pgfsetstrokeopacity{0.901261}%
\pgfsetdash{}{0pt}%
\pgfpathmoveto{\pgfqpoint{2.638258in}{1.118341in}}%
\pgfpathcurveto{\pgfqpoint{2.646494in}{1.118341in}}{\pgfqpoint{2.654394in}{1.121613in}}{\pgfqpoint{2.660218in}{1.127437in}}%
\pgfpathcurveto{\pgfqpoint{2.666042in}{1.133261in}}{\pgfqpoint{2.669314in}{1.141161in}}{\pgfqpoint{2.669314in}{1.149397in}}%
\pgfpathcurveto{\pgfqpoint{2.669314in}{1.157633in}}{\pgfqpoint{2.666042in}{1.165533in}}{\pgfqpoint{2.660218in}{1.171357in}}%
\pgfpathcurveto{\pgfqpoint{2.654394in}{1.177181in}}{\pgfqpoint{2.646494in}{1.180454in}}{\pgfqpoint{2.638258in}{1.180454in}}%
\pgfpathcurveto{\pgfqpoint{2.630022in}{1.180454in}}{\pgfqpoint{2.622121in}{1.177181in}}{\pgfqpoint{2.616298in}{1.171357in}}%
\pgfpathcurveto{\pgfqpoint{2.610474in}{1.165533in}}{\pgfqpoint{2.607201in}{1.157633in}}{\pgfqpoint{2.607201in}{1.149397in}}%
\pgfpathcurveto{\pgfqpoint{2.607201in}{1.141161in}}{\pgfqpoint{2.610474in}{1.133261in}}{\pgfqpoint{2.616298in}{1.127437in}}%
\pgfpathcurveto{\pgfqpoint{2.622121in}{1.121613in}}{\pgfqpoint{2.630022in}{1.118341in}}{\pgfqpoint{2.638258in}{1.118341in}}%
\pgfpathclose%
\pgfusepath{stroke,fill}%
\end{pgfscope}%
\begin{pgfscope}%
\pgfpathrectangle{\pgfqpoint{0.100000in}{0.212622in}}{\pgfqpoint{3.696000in}{3.696000in}}%
\pgfusepath{clip}%
\pgfsetbuttcap%
\pgfsetroundjoin%
\definecolor{currentfill}{rgb}{0.121569,0.466667,0.705882}%
\pgfsetfillcolor{currentfill}%
\pgfsetfillopacity{0.906277}%
\pgfsetlinewidth{1.003750pt}%
\definecolor{currentstroke}{rgb}{0.121569,0.466667,0.705882}%
\pgfsetstrokecolor{currentstroke}%
\pgfsetstrokeopacity{0.906277}%
\pgfsetdash{}{0pt}%
\pgfpathmoveto{\pgfqpoint{2.653376in}{1.114086in}}%
\pgfpathcurveto{\pgfqpoint{2.661613in}{1.114086in}}{\pgfqpoint{2.669513in}{1.117358in}}{\pgfqpoint{2.675337in}{1.123182in}}%
\pgfpathcurveto{\pgfqpoint{2.681161in}{1.129006in}}{\pgfqpoint{2.684433in}{1.136906in}}{\pgfqpoint{2.684433in}{1.145142in}}%
\pgfpathcurveto{\pgfqpoint{2.684433in}{1.153379in}}{\pgfqpoint{2.681161in}{1.161279in}}{\pgfqpoint{2.675337in}{1.167103in}}%
\pgfpathcurveto{\pgfqpoint{2.669513in}{1.172926in}}{\pgfqpoint{2.661613in}{1.176199in}}{\pgfqpoint{2.653376in}{1.176199in}}%
\pgfpathcurveto{\pgfqpoint{2.645140in}{1.176199in}}{\pgfqpoint{2.637240in}{1.172926in}}{\pgfqpoint{2.631416in}{1.167103in}}%
\pgfpathcurveto{\pgfqpoint{2.625592in}{1.161279in}}{\pgfqpoint{2.622320in}{1.153379in}}{\pgfqpoint{2.622320in}{1.145142in}}%
\pgfpathcurveto{\pgfqpoint{2.622320in}{1.136906in}}{\pgfqpoint{2.625592in}{1.129006in}}{\pgfqpoint{2.631416in}{1.123182in}}%
\pgfpathcurveto{\pgfqpoint{2.637240in}{1.117358in}}{\pgfqpoint{2.645140in}{1.114086in}}{\pgfqpoint{2.653376in}{1.114086in}}%
\pgfpathclose%
\pgfusepath{stroke,fill}%
\end{pgfscope}%
\begin{pgfscope}%
\pgfpathrectangle{\pgfqpoint{0.100000in}{0.212622in}}{\pgfqpoint{3.696000in}{3.696000in}}%
\pgfusepath{clip}%
\pgfsetbuttcap%
\pgfsetroundjoin%
\definecolor{currentfill}{rgb}{0.121569,0.466667,0.705882}%
\pgfsetfillcolor{currentfill}%
\pgfsetfillopacity{0.909438}%
\pgfsetlinewidth{1.003750pt}%
\definecolor{currentstroke}{rgb}{0.121569,0.466667,0.705882}%
\pgfsetstrokecolor{currentstroke}%
\pgfsetstrokeopacity{0.909438}%
\pgfsetdash{}{0pt}%
\pgfpathmoveto{\pgfqpoint{2.661491in}{1.111822in}}%
\pgfpathcurveto{\pgfqpoint{2.669727in}{1.111822in}}{\pgfqpoint{2.677628in}{1.115094in}}{\pgfqpoint{2.683451in}{1.120918in}}%
\pgfpathcurveto{\pgfqpoint{2.689275in}{1.126742in}}{\pgfqpoint{2.692548in}{1.134642in}}{\pgfqpoint{2.692548in}{1.142878in}}%
\pgfpathcurveto{\pgfqpoint{2.692548in}{1.151115in}}{\pgfqpoint{2.689275in}{1.159015in}}{\pgfqpoint{2.683451in}{1.164839in}}%
\pgfpathcurveto{\pgfqpoint{2.677628in}{1.170663in}}{\pgfqpoint{2.669727in}{1.173935in}}{\pgfqpoint{2.661491in}{1.173935in}}%
\pgfpathcurveto{\pgfqpoint{2.653255in}{1.173935in}}{\pgfqpoint{2.645355in}{1.170663in}}{\pgfqpoint{2.639531in}{1.164839in}}%
\pgfpathcurveto{\pgfqpoint{2.633707in}{1.159015in}}{\pgfqpoint{2.630435in}{1.151115in}}{\pgfqpoint{2.630435in}{1.142878in}}%
\pgfpathcurveto{\pgfqpoint{2.630435in}{1.134642in}}{\pgfqpoint{2.633707in}{1.126742in}}{\pgfqpoint{2.639531in}{1.120918in}}%
\pgfpathcurveto{\pgfqpoint{2.645355in}{1.115094in}}{\pgfqpoint{2.653255in}{1.111822in}}{\pgfqpoint{2.661491in}{1.111822in}}%
\pgfpathclose%
\pgfusepath{stroke,fill}%
\end{pgfscope}%
\begin{pgfscope}%
\pgfpathrectangle{\pgfqpoint{0.100000in}{0.212622in}}{\pgfqpoint{3.696000in}{3.696000in}}%
\pgfusepath{clip}%
\pgfsetbuttcap%
\pgfsetroundjoin%
\definecolor{currentfill}{rgb}{0.121569,0.466667,0.705882}%
\pgfsetfillcolor{currentfill}%
\pgfsetfillopacity{0.911115}%
\pgfsetlinewidth{1.003750pt}%
\definecolor{currentstroke}{rgb}{0.121569,0.466667,0.705882}%
\pgfsetstrokecolor{currentstroke}%
\pgfsetstrokeopacity{0.911115}%
\pgfsetdash{}{0pt}%
\pgfpathmoveto{\pgfqpoint{2.665919in}{1.110328in}}%
\pgfpathcurveto{\pgfqpoint{2.674155in}{1.110328in}}{\pgfqpoint{2.682055in}{1.113600in}}{\pgfqpoint{2.687879in}{1.119424in}}%
\pgfpathcurveto{\pgfqpoint{2.693703in}{1.125248in}}{\pgfqpoint{2.696976in}{1.133148in}}{\pgfqpoint{2.696976in}{1.141384in}}%
\pgfpathcurveto{\pgfqpoint{2.696976in}{1.149621in}}{\pgfqpoint{2.693703in}{1.157521in}}{\pgfqpoint{2.687879in}{1.163345in}}%
\pgfpathcurveto{\pgfqpoint{2.682055in}{1.169169in}}{\pgfqpoint{2.674155in}{1.172441in}}{\pgfqpoint{2.665919in}{1.172441in}}%
\pgfpathcurveto{\pgfqpoint{2.657683in}{1.172441in}}{\pgfqpoint{2.649783in}{1.169169in}}{\pgfqpoint{2.643959in}{1.163345in}}%
\pgfpathcurveto{\pgfqpoint{2.638135in}{1.157521in}}{\pgfqpoint{2.634863in}{1.149621in}}{\pgfqpoint{2.634863in}{1.141384in}}%
\pgfpathcurveto{\pgfqpoint{2.634863in}{1.133148in}}{\pgfqpoint{2.638135in}{1.125248in}}{\pgfqpoint{2.643959in}{1.119424in}}%
\pgfpathcurveto{\pgfqpoint{2.649783in}{1.113600in}}{\pgfqpoint{2.657683in}{1.110328in}}{\pgfqpoint{2.665919in}{1.110328in}}%
\pgfpathclose%
\pgfusepath{stroke,fill}%
\end{pgfscope}%
\begin{pgfscope}%
\pgfpathrectangle{\pgfqpoint{0.100000in}{0.212622in}}{\pgfqpoint{3.696000in}{3.696000in}}%
\pgfusepath{clip}%
\pgfsetbuttcap%
\pgfsetroundjoin%
\definecolor{currentfill}{rgb}{0.121569,0.466667,0.705882}%
\pgfsetfillcolor{currentfill}%
\pgfsetfillopacity{0.913262}%
\pgfsetlinewidth{1.003750pt}%
\definecolor{currentstroke}{rgb}{0.121569,0.466667,0.705882}%
\pgfsetstrokecolor{currentstroke}%
\pgfsetstrokeopacity{0.913262}%
\pgfsetdash{}{0pt}%
\pgfpathmoveto{\pgfqpoint{2.670942in}{1.108385in}}%
\pgfpathcurveto{\pgfqpoint{2.679178in}{1.108385in}}{\pgfqpoint{2.687078in}{1.111658in}}{\pgfqpoint{2.692902in}{1.117482in}}%
\pgfpathcurveto{\pgfqpoint{2.698726in}{1.123306in}}{\pgfqpoint{2.701998in}{1.131206in}}{\pgfqpoint{2.701998in}{1.139442in}}%
\pgfpathcurveto{\pgfqpoint{2.701998in}{1.147678in}}{\pgfqpoint{2.698726in}{1.155578in}}{\pgfqpoint{2.692902in}{1.161402in}}%
\pgfpathcurveto{\pgfqpoint{2.687078in}{1.167226in}}{\pgfqpoint{2.679178in}{1.170498in}}{\pgfqpoint{2.670942in}{1.170498in}}%
\pgfpathcurveto{\pgfqpoint{2.662705in}{1.170498in}}{\pgfqpoint{2.654805in}{1.167226in}}{\pgfqpoint{2.648981in}{1.161402in}}%
\pgfpathcurveto{\pgfqpoint{2.643157in}{1.155578in}}{\pgfqpoint{2.639885in}{1.147678in}}{\pgfqpoint{2.639885in}{1.139442in}}%
\pgfpathcurveto{\pgfqpoint{2.639885in}{1.131206in}}{\pgfqpoint{2.643157in}{1.123306in}}{\pgfqpoint{2.648981in}{1.117482in}}%
\pgfpathcurveto{\pgfqpoint{2.654805in}{1.111658in}}{\pgfqpoint{2.662705in}{1.108385in}}{\pgfqpoint{2.670942in}{1.108385in}}%
\pgfpathclose%
\pgfusepath{stroke,fill}%
\end{pgfscope}%
\begin{pgfscope}%
\pgfpathrectangle{\pgfqpoint{0.100000in}{0.212622in}}{\pgfqpoint{3.696000in}{3.696000in}}%
\pgfusepath{clip}%
\pgfsetbuttcap%
\pgfsetroundjoin%
\definecolor{currentfill}{rgb}{0.121569,0.466667,0.705882}%
\pgfsetfillcolor{currentfill}%
\pgfsetfillopacity{0.915979}%
\pgfsetlinewidth{1.003750pt}%
\definecolor{currentstroke}{rgb}{0.121569,0.466667,0.705882}%
\pgfsetstrokecolor{currentstroke}%
\pgfsetstrokeopacity{0.915979}%
\pgfsetdash{}{0pt}%
\pgfpathmoveto{\pgfqpoint{2.676907in}{1.106046in}}%
\pgfpathcurveto{\pgfqpoint{2.685143in}{1.106046in}}{\pgfqpoint{2.693044in}{1.109319in}}{\pgfqpoint{2.698867in}{1.115143in}}%
\pgfpathcurveto{\pgfqpoint{2.704691in}{1.120966in}}{\pgfqpoint{2.707964in}{1.128866in}}{\pgfqpoint{2.707964in}{1.137103in}}%
\pgfpathcurveto{\pgfqpoint{2.707964in}{1.145339in}}{\pgfqpoint{2.704691in}{1.153239in}}{\pgfqpoint{2.698867in}{1.159063in}}%
\pgfpathcurveto{\pgfqpoint{2.693044in}{1.164887in}}{\pgfqpoint{2.685143in}{1.168159in}}{\pgfqpoint{2.676907in}{1.168159in}}%
\pgfpathcurveto{\pgfqpoint{2.668671in}{1.168159in}}{\pgfqpoint{2.660771in}{1.164887in}}{\pgfqpoint{2.654947in}{1.159063in}}%
\pgfpathcurveto{\pgfqpoint{2.649123in}{1.153239in}}{\pgfqpoint{2.645851in}{1.145339in}}{\pgfqpoint{2.645851in}{1.137103in}}%
\pgfpathcurveto{\pgfqpoint{2.645851in}{1.128866in}}{\pgfqpoint{2.649123in}{1.120966in}}{\pgfqpoint{2.654947in}{1.115143in}}%
\pgfpathcurveto{\pgfqpoint{2.660771in}{1.109319in}}{\pgfqpoint{2.668671in}{1.106046in}}{\pgfqpoint{2.676907in}{1.106046in}}%
\pgfpathclose%
\pgfusepath{stroke,fill}%
\end{pgfscope}%
\begin{pgfscope}%
\pgfpathrectangle{\pgfqpoint{0.100000in}{0.212622in}}{\pgfqpoint{3.696000in}{3.696000in}}%
\pgfusepath{clip}%
\pgfsetbuttcap%
\pgfsetroundjoin%
\definecolor{currentfill}{rgb}{0.121569,0.466667,0.705882}%
\pgfsetfillcolor{currentfill}%
\pgfsetfillopacity{0.918708}%
\pgfsetlinewidth{1.003750pt}%
\definecolor{currentstroke}{rgb}{0.121569,0.466667,0.705882}%
\pgfsetstrokecolor{currentstroke}%
\pgfsetstrokeopacity{0.918708}%
\pgfsetdash{}{0pt}%
\pgfpathmoveto{\pgfqpoint{2.684508in}{1.103905in}}%
\pgfpathcurveto{\pgfqpoint{2.692745in}{1.103905in}}{\pgfqpoint{2.700645in}{1.107177in}}{\pgfqpoint{2.706469in}{1.113001in}}%
\pgfpathcurveto{\pgfqpoint{2.712293in}{1.118825in}}{\pgfqpoint{2.715565in}{1.126725in}}{\pgfqpoint{2.715565in}{1.134962in}}%
\pgfpathcurveto{\pgfqpoint{2.715565in}{1.143198in}}{\pgfqpoint{2.712293in}{1.151098in}}{\pgfqpoint{2.706469in}{1.156922in}}%
\pgfpathcurveto{\pgfqpoint{2.700645in}{1.162746in}}{\pgfqpoint{2.692745in}{1.166018in}}{\pgfqpoint{2.684508in}{1.166018in}}%
\pgfpathcurveto{\pgfqpoint{2.676272in}{1.166018in}}{\pgfqpoint{2.668372in}{1.162746in}}{\pgfqpoint{2.662548in}{1.156922in}}%
\pgfpathcurveto{\pgfqpoint{2.656724in}{1.151098in}}{\pgfqpoint{2.653452in}{1.143198in}}{\pgfqpoint{2.653452in}{1.134962in}}%
\pgfpathcurveto{\pgfqpoint{2.653452in}{1.126725in}}{\pgfqpoint{2.656724in}{1.118825in}}{\pgfqpoint{2.662548in}{1.113001in}}%
\pgfpathcurveto{\pgfqpoint{2.668372in}{1.107177in}}{\pgfqpoint{2.676272in}{1.103905in}}{\pgfqpoint{2.684508in}{1.103905in}}%
\pgfpathclose%
\pgfusepath{stroke,fill}%
\end{pgfscope}%
\begin{pgfscope}%
\pgfpathrectangle{\pgfqpoint{0.100000in}{0.212622in}}{\pgfqpoint{3.696000in}{3.696000in}}%
\pgfusepath{clip}%
\pgfsetbuttcap%
\pgfsetroundjoin%
\definecolor{currentfill}{rgb}{0.121569,0.466667,0.705882}%
\pgfsetfillcolor{currentfill}%
\pgfsetfillopacity{0.922790}%
\pgfsetlinewidth{1.003750pt}%
\definecolor{currentstroke}{rgb}{0.121569,0.466667,0.705882}%
\pgfsetstrokecolor{currentstroke}%
\pgfsetstrokeopacity{0.922790}%
\pgfsetdash{}{0pt}%
\pgfpathmoveto{\pgfqpoint{2.693234in}{1.100461in}}%
\pgfpathcurveto{\pgfqpoint{2.701470in}{1.100461in}}{\pgfqpoint{2.709371in}{1.103733in}}{\pgfqpoint{2.715194in}{1.109557in}}%
\pgfpathcurveto{\pgfqpoint{2.721018in}{1.115381in}}{\pgfqpoint{2.724291in}{1.123281in}}{\pgfqpoint{2.724291in}{1.131517in}}%
\pgfpathcurveto{\pgfqpoint{2.724291in}{1.139754in}}{\pgfqpoint{2.721018in}{1.147654in}}{\pgfqpoint{2.715194in}{1.153478in}}%
\pgfpathcurveto{\pgfqpoint{2.709371in}{1.159302in}}{\pgfqpoint{2.701470in}{1.162574in}}{\pgfqpoint{2.693234in}{1.162574in}}%
\pgfpathcurveto{\pgfqpoint{2.684998in}{1.162574in}}{\pgfqpoint{2.677098in}{1.159302in}}{\pgfqpoint{2.671274in}{1.153478in}}%
\pgfpathcurveto{\pgfqpoint{2.665450in}{1.147654in}}{\pgfqpoint{2.662178in}{1.139754in}}{\pgfqpoint{2.662178in}{1.131517in}}%
\pgfpathcurveto{\pgfqpoint{2.662178in}{1.123281in}}{\pgfqpoint{2.665450in}{1.115381in}}{\pgfqpoint{2.671274in}{1.109557in}}%
\pgfpathcurveto{\pgfqpoint{2.677098in}{1.103733in}}{\pgfqpoint{2.684998in}{1.100461in}}{\pgfqpoint{2.693234in}{1.100461in}}%
\pgfpathclose%
\pgfusepath{stroke,fill}%
\end{pgfscope}%
\begin{pgfscope}%
\pgfpathrectangle{\pgfqpoint{0.100000in}{0.212622in}}{\pgfqpoint{3.696000in}{3.696000in}}%
\pgfusepath{clip}%
\pgfsetbuttcap%
\pgfsetroundjoin%
\definecolor{currentfill}{rgb}{0.121569,0.466667,0.705882}%
\pgfsetfillcolor{currentfill}%
\pgfsetfillopacity{0.926928}%
\pgfsetlinewidth{1.003750pt}%
\definecolor{currentstroke}{rgb}{0.121569,0.466667,0.705882}%
\pgfsetstrokecolor{currentstroke}%
\pgfsetstrokeopacity{0.926928}%
\pgfsetdash{}{0pt}%
\pgfpathmoveto{\pgfqpoint{2.703290in}{1.097108in}}%
\pgfpathcurveto{\pgfqpoint{2.711526in}{1.097108in}}{\pgfqpoint{2.719426in}{1.100380in}}{\pgfqpoint{2.725250in}{1.106204in}}%
\pgfpathcurveto{\pgfqpoint{2.731074in}{1.112028in}}{\pgfqpoint{2.734346in}{1.119928in}}{\pgfqpoint{2.734346in}{1.128164in}}%
\pgfpathcurveto{\pgfqpoint{2.734346in}{1.136400in}}{\pgfqpoint{2.731074in}{1.144300in}}{\pgfqpoint{2.725250in}{1.150124in}}%
\pgfpathcurveto{\pgfqpoint{2.719426in}{1.155948in}}{\pgfqpoint{2.711526in}{1.159221in}}{\pgfqpoint{2.703290in}{1.159221in}}%
\pgfpathcurveto{\pgfqpoint{2.695053in}{1.159221in}}{\pgfqpoint{2.687153in}{1.155948in}}{\pgfqpoint{2.681329in}{1.150124in}}%
\pgfpathcurveto{\pgfqpoint{2.675505in}{1.144300in}}{\pgfqpoint{2.672233in}{1.136400in}}{\pgfqpoint{2.672233in}{1.128164in}}%
\pgfpathcurveto{\pgfqpoint{2.672233in}{1.119928in}}{\pgfqpoint{2.675505in}{1.112028in}}{\pgfqpoint{2.681329in}{1.106204in}}%
\pgfpathcurveto{\pgfqpoint{2.687153in}{1.100380in}}{\pgfqpoint{2.695053in}{1.097108in}}{\pgfqpoint{2.703290in}{1.097108in}}%
\pgfpathclose%
\pgfusepath{stroke,fill}%
\end{pgfscope}%
\begin{pgfscope}%
\pgfpathrectangle{\pgfqpoint{0.100000in}{0.212622in}}{\pgfqpoint{3.696000in}{3.696000in}}%
\pgfusepath{clip}%
\pgfsetbuttcap%
\pgfsetroundjoin%
\definecolor{currentfill}{rgb}{0.121569,0.466667,0.705882}%
\pgfsetfillcolor{currentfill}%
\pgfsetfillopacity{0.930378}%
\pgfsetlinewidth{1.003750pt}%
\definecolor{currentstroke}{rgb}{0.121569,0.466667,0.705882}%
\pgfsetstrokecolor{currentstroke}%
\pgfsetstrokeopacity{0.930378}%
\pgfsetdash{}{0pt}%
\pgfpathmoveto{\pgfqpoint{2.714945in}{1.094553in}}%
\pgfpathcurveto{\pgfqpoint{2.723181in}{1.094553in}}{\pgfqpoint{2.731081in}{1.097825in}}{\pgfqpoint{2.736905in}{1.103649in}}%
\pgfpathcurveto{\pgfqpoint{2.742729in}{1.109473in}}{\pgfqpoint{2.746001in}{1.117373in}}{\pgfqpoint{2.746001in}{1.125609in}}%
\pgfpathcurveto{\pgfqpoint{2.746001in}{1.133846in}}{\pgfqpoint{2.742729in}{1.141746in}}{\pgfqpoint{2.736905in}{1.147570in}}%
\pgfpathcurveto{\pgfqpoint{2.731081in}{1.153394in}}{\pgfqpoint{2.723181in}{1.156666in}}{\pgfqpoint{2.714945in}{1.156666in}}%
\pgfpathcurveto{\pgfqpoint{2.706709in}{1.156666in}}{\pgfqpoint{2.698809in}{1.153394in}}{\pgfqpoint{2.692985in}{1.147570in}}%
\pgfpathcurveto{\pgfqpoint{2.687161in}{1.141746in}}{\pgfqpoint{2.683888in}{1.133846in}}{\pgfqpoint{2.683888in}{1.125609in}}%
\pgfpathcurveto{\pgfqpoint{2.683888in}{1.117373in}}{\pgfqpoint{2.687161in}{1.109473in}}{\pgfqpoint{2.692985in}{1.103649in}}%
\pgfpathcurveto{\pgfqpoint{2.698809in}{1.097825in}}{\pgfqpoint{2.706709in}{1.094553in}}{\pgfqpoint{2.714945in}{1.094553in}}%
\pgfpathclose%
\pgfusepath{stroke,fill}%
\end{pgfscope}%
\begin{pgfscope}%
\pgfpathrectangle{\pgfqpoint{0.100000in}{0.212622in}}{\pgfqpoint{3.696000in}{3.696000in}}%
\pgfusepath{clip}%
\pgfsetbuttcap%
\pgfsetroundjoin%
\definecolor{currentfill}{rgb}{0.121569,0.466667,0.705882}%
\pgfsetfillcolor{currentfill}%
\pgfsetfillopacity{0.934713}%
\pgfsetlinewidth{1.003750pt}%
\definecolor{currentstroke}{rgb}{0.121569,0.466667,0.705882}%
\pgfsetstrokecolor{currentstroke}%
\pgfsetstrokeopacity{0.934713}%
\pgfsetdash{}{0pt}%
\pgfpathmoveto{\pgfqpoint{2.726927in}{1.090839in}}%
\pgfpathcurveto{\pgfqpoint{2.735163in}{1.090839in}}{\pgfqpoint{2.743064in}{1.094111in}}{\pgfqpoint{2.748887in}{1.099935in}}%
\pgfpathcurveto{\pgfqpoint{2.754711in}{1.105759in}}{\pgfqpoint{2.757984in}{1.113659in}}{\pgfqpoint{2.757984in}{1.121895in}}%
\pgfpathcurveto{\pgfqpoint{2.757984in}{1.130132in}}{\pgfqpoint{2.754711in}{1.138032in}}{\pgfqpoint{2.748887in}{1.143856in}}%
\pgfpathcurveto{\pgfqpoint{2.743064in}{1.149680in}}{\pgfqpoint{2.735163in}{1.152952in}}{\pgfqpoint{2.726927in}{1.152952in}}%
\pgfpathcurveto{\pgfqpoint{2.718691in}{1.152952in}}{\pgfqpoint{2.710791in}{1.149680in}}{\pgfqpoint{2.704967in}{1.143856in}}%
\pgfpathcurveto{\pgfqpoint{2.699143in}{1.138032in}}{\pgfqpoint{2.695871in}{1.130132in}}{\pgfqpoint{2.695871in}{1.121895in}}%
\pgfpathcurveto{\pgfqpoint{2.695871in}{1.113659in}}{\pgfqpoint{2.699143in}{1.105759in}}{\pgfqpoint{2.704967in}{1.099935in}}%
\pgfpathcurveto{\pgfqpoint{2.710791in}{1.094111in}}{\pgfqpoint{2.718691in}{1.090839in}}{\pgfqpoint{2.726927in}{1.090839in}}%
\pgfpathclose%
\pgfusepath{stroke,fill}%
\end{pgfscope}%
\begin{pgfscope}%
\pgfpathrectangle{\pgfqpoint{0.100000in}{0.212622in}}{\pgfqpoint{3.696000in}{3.696000in}}%
\pgfusepath{clip}%
\pgfsetbuttcap%
\pgfsetroundjoin%
\definecolor{currentfill}{rgb}{0.121569,0.466667,0.705882}%
\pgfsetfillcolor{currentfill}%
\pgfsetfillopacity{0.937749}%
\pgfsetlinewidth{1.003750pt}%
\definecolor{currentstroke}{rgb}{0.121569,0.466667,0.705882}%
\pgfsetstrokecolor{currentstroke}%
\pgfsetstrokeopacity{0.937749}%
\pgfsetdash{}{0pt}%
\pgfpathmoveto{\pgfqpoint{2.733080in}{1.088539in}}%
\pgfpathcurveto{\pgfqpoint{2.741316in}{1.088539in}}{\pgfqpoint{2.749216in}{1.091811in}}{\pgfqpoint{2.755040in}{1.097635in}}%
\pgfpathcurveto{\pgfqpoint{2.760864in}{1.103459in}}{\pgfqpoint{2.764136in}{1.111359in}}{\pgfqpoint{2.764136in}{1.119596in}}%
\pgfpathcurveto{\pgfqpoint{2.764136in}{1.127832in}}{\pgfqpoint{2.760864in}{1.135732in}}{\pgfqpoint{2.755040in}{1.141556in}}%
\pgfpathcurveto{\pgfqpoint{2.749216in}{1.147380in}}{\pgfqpoint{2.741316in}{1.150652in}}{\pgfqpoint{2.733080in}{1.150652in}}%
\pgfpathcurveto{\pgfqpoint{2.724843in}{1.150652in}}{\pgfqpoint{2.716943in}{1.147380in}}{\pgfqpoint{2.711119in}{1.141556in}}%
\pgfpathcurveto{\pgfqpoint{2.705295in}{1.135732in}}{\pgfqpoint{2.702023in}{1.127832in}}{\pgfqpoint{2.702023in}{1.119596in}}%
\pgfpathcurveto{\pgfqpoint{2.702023in}{1.111359in}}{\pgfqpoint{2.705295in}{1.103459in}}{\pgfqpoint{2.711119in}{1.097635in}}%
\pgfpathcurveto{\pgfqpoint{2.716943in}{1.091811in}}{\pgfqpoint{2.724843in}{1.088539in}}{\pgfqpoint{2.733080in}{1.088539in}}%
\pgfpathclose%
\pgfusepath{stroke,fill}%
\end{pgfscope}%
\begin{pgfscope}%
\pgfpathrectangle{\pgfqpoint{0.100000in}{0.212622in}}{\pgfqpoint{3.696000in}{3.696000in}}%
\pgfusepath{clip}%
\pgfsetbuttcap%
\pgfsetroundjoin%
\definecolor{currentfill}{rgb}{0.121569,0.466667,0.705882}%
\pgfsetfillcolor{currentfill}%
\pgfsetfillopacity{0.939118}%
\pgfsetlinewidth{1.003750pt}%
\definecolor{currentstroke}{rgb}{0.121569,0.466667,0.705882}%
\pgfsetstrokecolor{currentstroke}%
\pgfsetstrokeopacity{0.939118}%
\pgfsetdash{}{0pt}%
\pgfpathmoveto{\pgfqpoint{2.736671in}{1.087404in}}%
\pgfpathcurveto{\pgfqpoint{2.744908in}{1.087404in}}{\pgfqpoint{2.752808in}{1.090676in}}{\pgfqpoint{2.758632in}{1.096500in}}%
\pgfpathcurveto{\pgfqpoint{2.764456in}{1.102324in}}{\pgfqpoint{2.767728in}{1.110224in}}{\pgfqpoint{2.767728in}{1.118460in}}%
\pgfpathcurveto{\pgfqpoint{2.767728in}{1.126696in}}{\pgfqpoint{2.764456in}{1.134596in}}{\pgfqpoint{2.758632in}{1.140420in}}%
\pgfpathcurveto{\pgfqpoint{2.752808in}{1.146244in}}{\pgfqpoint{2.744908in}{1.149517in}}{\pgfqpoint{2.736671in}{1.149517in}}%
\pgfpathcurveto{\pgfqpoint{2.728435in}{1.149517in}}{\pgfqpoint{2.720535in}{1.146244in}}{\pgfqpoint{2.714711in}{1.140420in}}%
\pgfpathcurveto{\pgfqpoint{2.708887in}{1.134596in}}{\pgfqpoint{2.705615in}{1.126696in}}{\pgfqpoint{2.705615in}{1.118460in}}%
\pgfpathcurveto{\pgfqpoint{2.705615in}{1.110224in}}{\pgfqpoint{2.708887in}{1.102324in}}{\pgfqpoint{2.714711in}{1.096500in}}%
\pgfpathcurveto{\pgfqpoint{2.720535in}{1.090676in}}{\pgfqpoint{2.728435in}{1.087404in}}{\pgfqpoint{2.736671in}{1.087404in}}%
\pgfpathclose%
\pgfusepath{stroke,fill}%
\end{pgfscope}%
\begin{pgfscope}%
\pgfpathrectangle{\pgfqpoint{0.100000in}{0.212622in}}{\pgfqpoint{3.696000in}{3.696000in}}%
\pgfusepath{clip}%
\pgfsetbuttcap%
\pgfsetroundjoin%
\definecolor{currentfill}{rgb}{0.121569,0.466667,0.705882}%
\pgfsetfillcolor{currentfill}%
\pgfsetfillopacity{0.941042}%
\pgfsetlinewidth{1.003750pt}%
\definecolor{currentstroke}{rgb}{0.121569,0.466667,0.705882}%
\pgfsetstrokecolor{currentstroke}%
\pgfsetstrokeopacity{0.941042}%
\pgfsetdash{}{0pt}%
\pgfpathmoveto{\pgfqpoint{2.740511in}{1.085723in}}%
\pgfpathcurveto{\pgfqpoint{2.748747in}{1.085723in}}{\pgfqpoint{2.756647in}{1.088995in}}{\pgfqpoint{2.762471in}{1.094819in}}%
\pgfpathcurveto{\pgfqpoint{2.768295in}{1.100643in}}{\pgfqpoint{2.771567in}{1.108543in}}{\pgfqpoint{2.771567in}{1.116779in}}%
\pgfpathcurveto{\pgfqpoint{2.771567in}{1.125015in}}{\pgfqpoint{2.768295in}{1.132915in}}{\pgfqpoint{2.762471in}{1.138739in}}%
\pgfpathcurveto{\pgfqpoint{2.756647in}{1.144563in}}{\pgfqpoint{2.748747in}{1.147836in}}{\pgfqpoint{2.740511in}{1.147836in}}%
\pgfpathcurveto{\pgfqpoint{2.732274in}{1.147836in}}{\pgfqpoint{2.724374in}{1.144563in}}{\pgfqpoint{2.718550in}{1.138739in}}%
\pgfpathcurveto{\pgfqpoint{2.712726in}{1.132915in}}{\pgfqpoint{2.709454in}{1.125015in}}{\pgfqpoint{2.709454in}{1.116779in}}%
\pgfpathcurveto{\pgfqpoint{2.709454in}{1.108543in}}{\pgfqpoint{2.712726in}{1.100643in}}{\pgfqpoint{2.718550in}{1.094819in}}%
\pgfpathcurveto{\pgfqpoint{2.724374in}{1.088995in}}{\pgfqpoint{2.732274in}{1.085723in}}{\pgfqpoint{2.740511in}{1.085723in}}%
\pgfpathclose%
\pgfusepath{stroke,fill}%
\end{pgfscope}%
\begin{pgfscope}%
\pgfpathrectangle{\pgfqpoint{0.100000in}{0.212622in}}{\pgfqpoint{3.696000in}{3.696000in}}%
\pgfusepath{clip}%
\pgfsetbuttcap%
\pgfsetroundjoin%
\definecolor{currentfill}{rgb}{0.121569,0.466667,0.705882}%
\pgfsetfillcolor{currentfill}%
\pgfsetfillopacity{0.941984}%
\pgfsetlinewidth{1.003750pt}%
\definecolor{currentstroke}{rgb}{0.121569,0.466667,0.705882}%
\pgfsetstrokecolor{currentstroke}%
\pgfsetstrokeopacity{0.941984}%
\pgfsetdash{}{0pt}%
\pgfpathmoveto{\pgfqpoint{2.742724in}{1.084922in}}%
\pgfpathcurveto{\pgfqpoint{2.750960in}{1.084922in}}{\pgfqpoint{2.758860in}{1.088194in}}{\pgfqpoint{2.764684in}{1.094018in}}%
\pgfpathcurveto{\pgfqpoint{2.770508in}{1.099842in}}{\pgfqpoint{2.773781in}{1.107742in}}{\pgfqpoint{2.773781in}{1.115979in}}%
\pgfpathcurveto{\pgfqpoint{2.773781in}{1.124215in}}{\pgfqpoint{2.770508in}{1.132115in}}{\pgfqpoint{2.764684in}{1.137939in}}%
\pgfpathcurveto{\pgfqpoint{2.758860in}{1.143763in}}{\pgfqpoint{2.750960in}{1.147035in}}{\pgfqpoint{2.742724in}{1.147035in}}%
\pgfpathcurveto{\pgfqpoint{2.734488in}{1.147035in}}{\pgfqpoint{2.726588in}{1.143763in}}{\pgfqpoint{2.720764in}{1.137939in}}%
\pgfpathcurveto{\pgfqpoint{2.714940in}{1.132115in}}{\pgfqpoint{2.711668in}{1.124215in}}{\pgfqpoint{2.711668in}{1.115979in}}%
\pgfpathcurveto{\pgfqpoint{2.711668in}{1.107742in}}{\pgfqpoint{2.714940in}{1.099842in}}{\pgfqpoint{2.720764in}{1.094018in}}%
\pgfpathcurveto{\pgfqpoint{2.726588in}{1.088194in}}{\pgfqpoint{2.734488in}{1.084922in}}{\pgfqpoint{2.742724in}{1.084922in}}%
\pgfpathclose%
\pgfusepath{stroke,fill}%
\end{pgfscope}%
\begin{pgfscope}%
\pgfpathrectangle{\pgfqpoint{0.100000in}{0.212622in}}{\pgfqpoint{3.696000in}{3.696000in}}%
\pgfusepath{clip}%
\pgfsetbuttcap%
\pgfsetroundjoin%
\definecolor{currentfill}{rgb}{0.121569,0.466667,0.705882}%
\pgfsetfillcolor{currentfill}%
\pgfsetfillopacity{0.943561}%
\pgfsetlinewidth{1.003750pt}%
\definecolor{currentstroke}{rgb}{0.121569,0.466667,0.705882}%
\pgfsetstrokecolor{currentstroke}%
\pgfsetstrokeopacity{0.943561}%
\pgfsetdash{}{0pt}%
\pgfpathmoveto{\pgfqpoint{2.746317in}{1.083604in}}%
\pgfpathcurveto{\pgfqpoint{2.754554in}{1.083604in}}{\pgfqpoint{2.762454in}{1.086876in}}{\pgfqpoint{2.768278in}{1.092700in}}%
\pgfpathcurveto{\pgfqpoint{2.774102in}{1.098524in}}{\pgfqpoint{2.777374in}{1.106424in}}{\pgfqpoint{2.777374in}{1.114661in}}%
\pgfpathcurveto{\pgfqpoint{2.777374in}{1.122897in}}{\pgfqpoint{2.774102in}{1.130797in}}{\pgfqpoint{2.768278in}{1.136621in}}%
\pgfpathcurveto{\pgfqpoint{2.762454in}{1.142445in}}{\pgfqpoint{2.754554in}{1.145717in}}{\pgfqpoint{2.746317in}{1.145717in}}%
\pgfpathcurveto{\pgfqpoint{2.738081in}{1.145717in}}{\pgfqpoint{2.730181in}{1.142445in}}{\pgfqpoint{2.724357in}{1.136621in}}%
\pgfpathcurveto{\pgfqpoint{2.718533in}{1.130797in}}{\pgfqpoint{2.715261in}{1.122897in}}{\pgfqpoint{2.715261in}{1.114661in}}%
\pgfpathcurveto{\pgfqpoint{2.715261in}{1.106424in}}{\pgfqpoint{2.718533in}{1.098524in}}{\pgfqpoint{2.724357in}{1.092700in}}%
\pgfpathcurveto{\pgfqpoint{2.730181in}{1.086876in}}{\pgfqpoint{2.738081in}{1.083604in}}{\pgfqpoint{2.746317in}{1.083604in}}%
\pgfpathclose%
\pgfusepath{stroke,fill}%
\end{pgfscope}%
\begin{pgfscope}%
\pgfpathrectangle{\pgfqpoint{0.100000in}{0.212622in}}{\pgfqpoint{3.696000in}{3.696000in}}%
\pgfusepath{clip}%
\pgfsetbuttcap%
\pgfsetroundjoin%
\definecolor{currentfill}{rgb}{0.121569,0.466667,0.705882}%
\pgfsetfillcolor{currentfill}%
\pgfsetfillopacity{0.945807}%
\pgfsetlinewidth{1.003750pt}%
\definecolor{currentstroke}{rgb}{0.121569,0.466667,0.705882}%
\pgfsetstrokecolor{currentstroke}%
\pgfsetstrokeopacity{0.945807}%
\pgfsetdash{}{0pt}%
\pgfpathmoveto{\pgfqpoint{2.750844in}{1.081672in}}%
\pgfpathcurveto{\pgfqpoint{2.759080in}{1.081672in}}{\pgfqpoint{2.766980in}{1.084944in}}{\pgfqpoint{2.772804in}{1.090768in}}%
\pgfpathcurveto{\pgfqpoint{2.778628in}{1.096592in}}{\pgfqpoint{2.781900in}{1.104492in}}{\pgfqpoint{2.781900in}{1.112729in}}%
\pgfpathcurveto{\pgfqpoint{2.781900in}{1.120965in}}{\pgfqpoint{2.778628in}{1.128865in}}{\pgfqpoint{2.772804in}{1.134689in}}%
\pgfpathcurveto{\pgfqpoint{2.766980in}{1.140513in}}{\pgfqpoint{2.759080in}{1.143785in}}{\pgfqpoint{2.750844in}{1.143785in}}%
\pgfpathcurveto{\pgfqpoint{2.742607in}{1.143785in}}{\pgfqpoint{2.734707in}{1.140513in}}{\pgfqpoint{2.728883in}{1.134689in}}%
\pgfpathcurveto{\pgfqpoint{2.723059in}{1.128865in}}{\pgfqpoint{2.719787in}{1.120965in}}{\pgfqpoint{2.719787in}{1.112729in}}%
\pgfpathcurveto{\pgfqpoint{2.719787in}{1.104492in}}{\pgfqpoint{2.723059in}{1.096592in}}{\pgfqpoint{2.728883in}{1.090768in}}%
\pgfpathcurveto{\pgfqpoint{2.734707in}{1.084944in}}{\pgfqpoint{2.742607in}{1.081672in}}{\pgfqpoint{2.750844in}{1.081672in}}%
\pgfpathclose%
\pgfusepath{stroke,fill}%
\end{pgfscope}%
\begin{pgfscope}%
\pgfpathrectangle{\pgfqpoint{0.100000in}{0.212622in}}{\pgfqpoint{3.696000in}{3.696000in}}%
\pgfusepath{clip}%
\pgfsetbuttcap%
\pgfsetroundjoin%
\definecolor{currentfill}{rgb}{0.121569,0.466667,0.705882}%
\pgfsetfillcolor{currentfill}%
\pgfsetfillopacity{0.948035}%
\pgfsetlinewidth{1.003750pt}%
\definecolor{currentstroke}{rgb}{0.121569,0.466667,0.705882}%
\pgfsetstrokecolor{currentstroke}%
\pgfsetstrokeopacity{0.948035}%
\pgfsetdash{}{0pt}%
\pgfpathmoveto{\pgfqpoint{2.757000in}{1.079934in}}%
\pgfpathcurveto{\pgfqpoint{2.765236in}{1.079934in}}{\pgfqpoint{2.773136in}{1.083206in}}{\pgfqpoint{2.778960in}{1.089030in}}%
\pgfpathcurveto{\pgfqpoint{2.784784in}{1.094854in}}{\pgfqpoint{2.788056in}{1.102754in}}{\pgfqpoint{2.788056in}{1.110990in}}%
\pgfpathcurveto{\pgfqpoint{2.788056in}{1.119227in}}{\pgfqpoint{2.784784in}{1.127127in}}{\pgfqpoint{2.778960in}{1.132951in}}%
\pgfpathcurveto{\pgfqpoint{2.773136in}{1.138774in}}{\pgfqpoint{2.765236in}{1.142047in}}{\pgfqpoint{2.757000in}{1.142047in}}%
\pgfpathcurveto{\pgfqpoint{2.748764in}{1.142047in}}{\pgfqpoint{2.740863in}{1.138774in}}{\pgfqpoint{2.735040in}{1.132951in}}%
\pgfpathcurveto{\pgfqpoint{2.729216in}{1.127127in}}{\pgfqpoint{2.725943in}{1.119227in}}{\pgfqpoint{2.725943in}{1.110990in}}%
\pgfpathcurveto{\pgfqpoint{2.725943in}{1.102754in}}{\pgfqpoint{2.729216in}{1.094854in}}{\pgfqpoint{2.735040in}{1.089030in}}%
\pgfpathcurveto{\pgfqpoint{2.740863in}{1.083206in}}{\pgfqpoint{2.748764in}{1.079934in}}{\pgfqpoint{2.757000in}{1.079934in}}%
\pgfpathclose%
\pgfusepath{stroke,fill}%
\end{pgfscope}%
\begin{pgfscope}%
\pgfpathrectangle{\pgfqpoint{0.100000in}{0.212622in}}{\pgfqpoint{3.696000in}{3.696000in}}%
\pgfusepath{clip}%
\pgfsetbuttcap%
\pgfsetroundjoin%
\definecolor{currentfill}{rgb}{0.121569,0.466667,0.705882}%
\pgfsetfillcolor{currentfill}%
\pgfsetfillopacity{0.950686}%
\pgfsetlinewidth{1.003750pt}%
\definecolor{currentstroke}{rgb}{0.121569,0.466667,0.705882}%
\pgfsetstrokecolor{currentstroke}%
\pgfsetstrokeopacity{0.950686}%
\pgfsetdash{}{0pt}%
\pgfpathmoveto{\pgfqpoint{2.763965in}{1.077799in}}%
\pgfpathcurveto{\pgfqpoint{2.772201in}{1.077799in}}{\pgfqpoint{2.780102in}{1.081071in}}{\pgfqpoint{2.785925in}{1.086895in}}%
\pgfpathcurveto{\pgfqpoint{2.791749in}{1.092719in}}{\pgfqpoint{2.795022in}{1.100619in}}{\pgfqpoint{2.795022in}{1.108855in}}%
\pgfpathcurveto{\pgfqpoint{2.795022in}{1.117091in}}{\pgfqpoint{2.791749in}{1.124991in}}{\pgfqpoint{2.785925in}{1.130815in}}%
\pgfpathcurveto{\pgfqpoint{2.780102in}{1.136639in}}{\pgfqpoint{2.772201in}{1.139912in}}{\pgfqpoint{2.763965in}{1.139912in}}%
\pgfpathcurveto{\pgfqpoint{2.755729in}{1.139912in}}{\pgfqpoint{2.747829in}{1.136639in}}{\pgfqpoint{2.742005in}{1.130815in}}%
\pgfpathcurveto{\pgfqpoint{2.736181in}{1.124991in}}{\pgfqpoint{2.732909in}{1.117091in}}{\pgfqpoint{2.732909in}{1.108855in}}%
\pgfpathcurveto{\pgfqpoint{2.732909in}{1.100619in}}{\pgfqpoint{2.736181in}{1.092719in}}{\pgfqpoint{2.742005in}{1.086895in}}%
\pgfpathcurveto{\pgfqpoint{2.747829in}{1.081071in}}{\pgfqpoint{2.755729in}{1.077799in}}{\pgfqpoint{2.763965in}{1.077799in}}%
\pgfpathclose%
\pgfusepath{stroke,fill}%
\end{pgfscope}%
\begin{pgfscope}%
\pgfpathrectangle{\pgfqpoint{0.100000in}{0.212622in}}{\pgfqpoint{3.696000in}{3.696000in}}%
\pgfusepath{clip}%
\pgfsetbuttcap%
\pgfsetroundjoin%
\definecolor{currentfill}{rgb}{0.121569,0.466667,0.705882}%
\pgfsetfillcolor{currentfill}%
\pgfsetfillopacity{0.952120}%
\pgfsetlinewidth{1.003750pt}%
\definecolor{currentstroke}{rgb}{0.121569,0.466667,0.705882}%
\pgfsetstrokecolor{currentstroke}%
\pgfsetstrokeopacity{0.952120}%
\pgfsetdash{}{0pt}%
\pgfpathmoveto{\pgfqpoint{2.767810in}{1.076621in}}%
\pgfpathcurveto{\pgfqpoint{2.776046in}{1.076621in}}{\pgfqpoint{2.783946in}{1.079894in}}{\pgfqpoint{2.789770in}{1.085717in}}%
\pgfpathcurveto{\pgfqpoint{2.795594in}{1.091541in}}{\pgfqpoint{2.798866in}{1.099441in}}{\pgfqpoint{2.798866in}{1.107678in}}%
\pgfpathcurveto{\pgfqpoint{2.798866in}{1.115914in}}{\pgfqpoint{2.795594in}{1.123814in}}{\pgfqpoint{2.789770in}{1.129638in}}%
\pgfpathcurveto{\pgfqpoint{2.783946in}{1.135462in}}{\pgfqpoint{2.776046in}{1.138734in}}{\pgfqpoint{2.767810in}{1.138734in}}%
\pgfpathcurveto{\pgfqpoint{2.759573in}{1.138734in}}{\pgfqpoint{2.751673in}{1.135462in}}{\pgfqpoint{2.745849in}{1.129638in}}%
\pgfpathcurveto{\pgfqpoint{2.740025in}{1.123814in}}{\pgfqpoint{2.736753in}{1.115914in}}{\pgfqpoint{2.736753in}{1.107678in}}%
\pgfpathcurveto{\pgfqpoint{2.736753in}{1.099441in}}{\pgfqpoint{2.740025in}{1.091541in}}{\pgfqpoint{2.745849in}{1.085717in}}%
\pgfpathcurveto{\pgfqpoint{2.751673in}{1.079894in}}{\pgfqpoint{2.759573in}{1.076621in}}{\pgfqpoint{2.767810in}{1.076621in}}%
\pgfpathclose%
\pgfusepath{stroke,fill}%
\end{pgfscope}%
\begin{pgfscope}%
\pgfpathrectangle{\pgfqpoint{0.100000in}{0.212622in}}{\pgfqpoint{3.696000in}{3.696000in}}%
\pgfusepath{clip}%
\pgfsetbuttcap%
\pgfsetroundjoin%
\definecolor{currentfill}{rgb}{0.121569,0.466667,0.705882}%
\pgfsetfillcolor{currentfill}%
\pgfsetfillopacity{0.952906}%
\pgfsetlinewidth{1.003750pt}%
\definecolor{currentstroke}{rgb}{0.121569,0.466667,0.705882}%
\pgfsetstrokecolor{currentstroke}%
\pgfsetstrokeopacity{0.952906}%
\pgfsetdash{}{0pt}%
\pgfpathmoveto{\pgfqpoint{2.769932in}{1.075993in}}%
\pgfpathcurveto{\pgfqpoint{2.778168in}{1.075993in}}{\pgfqpoint{2.786068in}{1.079265in}}{\pgfqpoint{2.791892in}{1.085089in}}%
\pgfpathcurveto{\pgfqpoint{2.797716in}{1.090913in}}{\pgfqpoint{2.800988in}{1.098813in}}{\pgfqpoint{2.800988in}{1.107049in}}%
\pgfpathcurveto{\pgfqpoint{2.800988in}{1.115285in}}{\pgfqpoint{2.797716in}{1.123185in}}{\pgfqpoint{2.791892in}{1.129009in}}%
\pgfpathcurveto{\pgfqpoint{2.786068in}{1.134833in}}{\pgfqpoint{2.778168in}{1.138106in}}{\pgfqpoint{2.769932in}{1.138106in}}%
\pgfpathcurveto{\pgfqpoint{2.761695in}{1.138106in}}{\pgfqpoint{2.753795in}{1.134833in}}{\pgfqpoint{2.747971in}{1.129009in}}%
\pgfpathcurveto{\pgfqpoint{2.742148in}{1.123185in}}{\pgfqpoint{2.738875in}{1.115285in}}{\pgfqpoint{2.738875in}{1.107049in}}%
\pgfpathcurveto{\pgfqpoint{2.738875in}{1.098813in}}{\pgfqpoint{2.742148in}{1.090913in}}{\pgfqpoint{2.747971in}{1.085089in}}%
\pgfpathcurveto{\pgfqpoint{2.753795in}{1.079265in}}{\pgfqpoint{2.761695in}{1.075993in}}{\pgfqpoint{2.769932in}{1.075993in}}%
\pgfpathclose%
\pgfusepath{stroke,fill}%
\end{pgfscope}%
\begin{pgfscope}%
\pgfpathrectangle{\pgfqpoint{0.100000in}{0.212622in}}{\pgfqpoint{3.696000in}{3.696000in}}%
\pgfusepath{clip}%
\pgfsetbuttcap%
\pgfsetroundjoin%
\definecolor{currentfill}{rgb}{0.121569,0.466667,0.705882}%
\pgfsetfillcolor{currentfill}%
\pgfsetfillopacity{0.953353}%
\pgfsetlinewidth{1.003750pt}%
\definecolor{currentstroke}{rgb}{0.121569,0.466667,0.705882}%
\pgfsetstrokecolor{currentstroke}%
\pgfsetstrokeopacity{0.953353}%
\pgfsetdash{}{0pt}%
\pgfpathmoveto{\pgfqpoint{2.771079in}{1.075607in}}%
\pgfpathcurveto{\pgfqpoint{2.779315in}{1.075607in}}{\pgfqpoint{2.787215in}{1.078879in}}{\pgfqpoint{2.793039in}{1.084703in}}%
\pgfpathcurveto{\pgfqpoint{2.798863in}{1.090527in}}{\pgfqpoint{2.802135in}{1.098427in}}{\pgfqpoint{2.802135in}{1.106663in}}%
\pgfpathcurveto{\pgfqpoint{2.802135in}{1.114899in}}{\pgfqpoint{2.798863in}{1.122799in}}{\pgfqpoint{2.793039in}{1.128623in}}%
\pgfpathcurveto{\pgfqpoint{2.787215in}{1.134447in}}{\pgfqpoint{2.779315in}{1.137720in}}{\pgfqpoint{2.771079in}{1.137720in}}%
\pgfpathcurveto{\pgfqpoint{2.762843in}{1.137720in}}{\pgfqpoint{2.754943in}{1.134447in}}{\pgfqpoint{2.749119in}{1.128623in}}%
\pgfpathcurveto{\pgfqpoint{2.743295in}{1.122799in}}{\pgfqpoint{2.740022in}{1.114899in}}{\pgfqpoint{2.740022in}{1.106663in}}%
\pgfpathcurveto{\pgfqpoint{2.740022in}{1.098427in}}{\pgfqpoint{2.743295in}{1.090527in}}{\pgfqpoint{2.749119in}{1.084703in}}%
\pgfpathcurveto{\pgfqpoint{2.754943in}{1.078879in}}{\pgfqpoint{2.762843in}{1.075607in}}{\pgfqpoint{2.771079in}{1.075607in}}%
\pgfpathclose%
\pgfusepath{stroke,fill}%
\end{pgfscope}%
\begin{pgfscope}%
\pgfpathrectangle{\pgfqpoint{0.100000in}{0.212622in}}{\pgfqpoint{3.696000in}{3.696000in}}%
\pgfusepath{clip}%
\pgfsetbuttcap%
\pgfsetroundjoin%
\definecolor{currentfill}{rgb}{0.121569,0.466667,0.705882}%
\pgfsetfillcolor{currentfill}%
\pgfsetfillopacity{0.954383}%
\pgfsetlinewidth{1.003750pt}%
\definecolor{currentstroke}{rgb}{0.121569,0.466667,0.705882}%
\pgfsetstrokecolor{currentstroke}%
\pgfsetstrokeopacity{0.954383}%
\pgfsetdash{}{0pt}%
\pgfpathmoveto{\pgfqpoint{2.773419in}{1.074729in}}%
\pgfpathcurveto{\pgfqpoint{2.781655in}{1.074729in}}{\pgfqpoint{2.789555in}{1.078001in}}{\pgfqpoint{2.795379in}{1.083825in}}%
\pgfpathcurveto{\pgfqpoint{2.801203in}{1.089649in}}{\pgfqpoint{2.804475in}{1.097549in}}{\pgfqpoint{2.804475in}{1.105785in}}%
\pgfpathcurveto{\pgfqpoint{2.804475in}{1.114022in}}{\pgfqpoint{2.801203in}{1.121922in}}{\pgfqpoint{2.795379in}{1.127746in}}%
\pgfpathcurveto{\pgfqpoint{2.789555in}{1.133569in}}{\pgfqpoint{2.781655in}{1.136842in}}{\pgfqpoint{2.773419in}{1.136842in}}%
\pgfpathcurveto{\pgfqpoint{2.765183in}{1.136842in}}{\pgfqpoint{2.757283in}{1.133569in}}{\pgfqpoint{2.751459in}{1.127746in}}%
\pgfpathcurveto{\pgfqpoint{2.745635in}{1.121922in}}{\pgfqpoint{2.742362in}{1.114022in}}{\pgfqpoint{2.742362in}{1.105785in}}%
\pgfpathcurveto{\pgfqpoint{2.742362in}{1.097549in}}{\pgfqpoint{2.745635in}{1.089649in}}{\pgfqpoint{2.751459in}{1.083825in}}%
\pgfpathcurveto{\pgfqpoint{2.757283in}{1.078001in}}{\pgfqpoint{2.765183in}{1.074729in}}{\pgfqpoint{2.773419in}{1.074729in}}%
\pgfpathclose%
\pgfusepath{stroke,fill}%
\end{pgfscope}%
\begin{pgfscope}%
\pgfpathrectangle{\pgfqpoint{0.100000in}{0.212622in}}{\pgfqpoint{3.696000in}{3.696000in}}%
\pgfusepath{clip}%
\pgfsetbuttcap%
\pgfsetroundjoin%
\definecolor{currentfill}{rgb}{0.121569,0.466667,0.705882}%
\pgfsetfillcolor{currentfill}%
\pgfsetfillopacity{0.956122}%
\pgfsetlinewidth{1.003750pt}%
\definecolor{currentstroke}{rgb}{0.121569,0.466667,0.705882}%
\pgfsetstrokecolor{currentstroke}%
\pgfsetstrokeopacity{0.956122}%
\pgfsetdash{}{0pt}%
\pgfpathmoveto{\pgfqpoint{2.776677in}{1.073282in}}%
\pgfpathcurveto{\pgfqpoint{2.784913in}{1.073282in}}{\pgfqpoint{2.792813in}{1.076554in}}{\pgfqpoint{2.798637in}{1.082378in}}%
\pgfpathcurveto{\pgfqpoint{2.804461in}{1.088202in}}{\pgfqpoint{2.807733in}{1.096102in}}{\pgfqpoint{2.807733in}{1.104338in}}%
\pgfpathcurveto{\pgfqpoint{2.807733in}{1.112575in}}{\pgfqpoint{2.804461in}{1.120475in}}{\pgfqpoint{2.798637in}{1.126299in}}%
\pgfpathcurveto{\pgfqpoint{2.792813in}{1.132122in}}{\pgfqpoint{2.784913in}{1.135395in}}{\pgfqpoint{2.776677in}{1.135395in}}%
\pgfpathcurveto{\pgfqpoint{2.768440in}{1.135395in}}{\pgfqpoint{2.760540in}{1.132122in}}{\pgfqpoint{2.754716in}{1.126299in}}%
\pgfpathcurveto{\pgfqpoint{2.748892in}{1.120475in}}{\pgfqpoint{2.745620in}{1.112575in}}{\pgfqpoint{2.745620in}{1.104338in}}%
\pgfpathcurveto{\pgfqpoint{2.745620in}{1.096102in}}{\pgfqpoint{2.748892in}{1.088202in}}{\pgfqpoint{2.754716in}{1.082378in}}%
\pgfpathcurveto{\pgfqpoint{2.760540in}{1.076554in}}{\pgfqpoint{2.768440in}{1.073282in}}{\pgfqpoint{2.776677in}{1.073282in}}%
\pgfpathclose%
\pgfusepath{stroke,fill}%
\end{pgfscope}%
\begin{pgfscope}%
\pgfpathrectangle{\pgfqpoint{0.100000in}{0.212622in}}{\pgfqpoint{3.696000in}{3.696000in}}%
\pgfusepath{clip}%
\pgfsetbuttcap%
\pgfsetroundjoin%
\definecolor{currentfill}{rgb}{0.121569,0.466667,0.705882}%
\pgfsetfillcolor{currentfill}%
\pgfsetfillopacity{0.957462}%
\pgfsetlinewidth{1.003750pt}%
\definecolor{currentstroke}{rgb}{0.121569,0.466667,0.705882}%
\pgfsetstrokecolor{currentstroke}%
\pgfsetstrokeopacity{0.957462}%
\pgfsetdash{}{0pt}%
\pgfpathmoveto{\pgfqpoint{2.781334in}{1.072223in}}%
\pgfpathcurveto{\pgfqpoint{2.789570in}{1.072223in}}{\pgfqpoint{2.797471in}{1.075496in}}{\pgfqpoint{2.803294in}{1.081320in}}%
\pgfpathcurveto{\pgfqpoint{2.809118in}{1.087144in}}{\pgfqpoint{2.812391in}{1.095044in}}{\pgfqpoint{2.812391in}{1.103280in}}%
\pgfpathcurveto{\pgfqpoint{2.812391in}{1.111516in}}{\pgfqpoint{2.809118in}{1.119416in}}{\pgfqpoint{2.803294in}{1.125240in}}%
\pgfpathcurveto{\pgfqpoint{2.797471in}{1.131064in}}{\pgfqpoint{2.789570in}{1.134336in}}{\pgfqpoint{2.781334in}{1.134336in}}%
\pgfpathcurveto{\pgfqpoint{2.773098in}{1.134336in}}{\pgfqpoint{2.765198in}{1.131064in}}{\pgfqpoint{2.759374in}{1.125240in}}%
\pgfpathcurveto{\pgfqpoint{2.753550in}{1.119416in}}{\pgfqpoint{2.750278in}{1.111516in}}{\pgfqpoint{2.750278in}{1.103280in}}%
\pgfpathcurveto{\pgfqpoint{2.750278in}{1.095044in}}{\pgfqpoint{2.753550in}{1.087144in}}{\pgfqpoint{2.759374in}{1.081320in}}%
\pgfpathcurveto{\pgfqpoint{2.765198in}{1.075496in}}{\pgfqpoint{2.773098in}{1.072223in}}{\pgfqpoint{2.781334in}{1.072223in}}%
\pgfpathclose%
\pgfusepath{stroke,fill}%
\end{pgfscope}%
\begin{pgfscope}%
\pgfpathrectangle{\pgfqpoint{0.100000in}{0.212622in}}{\pgfqpoint{3.696000in}{3.696000in}}%
\pgfusepath{clip}%
\pgfsetbuttcap%
\pgfsetroundjoin%
\definecolor{currentfill}{rgb}{0.121569,0.466667,0.705882}%
\pgfsetfillcolor{currentfill}%
\pgfsetfillopacity{0.959428}%
\pgfsetlinewidth{1.003750pt}%
\definecolor{currentstroke}{rgb}{0.121569,0.466667,0.705882}%
\pgfsetstrokecolor{currentstroke}%
\pgfsetstrokeopacity{0.959428}%
\pgfsetdash{}{0pt}%
\pgfpathmoveto{\pgfqpoint{2.787054in}{1.070729in}}%
\pgfpathcurveto{\pgfqpoint{2.795290in}{1.070729in}}{\pgfqpoint{2.803190in}{1.074001in}}{\pgfqpoint{2.809014in}{1.079825in}}%
\pgfpathcurveto{\pgfqpoint{2.814838in}{1.085649in}}{\pgfqpoint{2.818110in}{1.093549in}}{\pgfqpoint{2.818110in}{1.101785in}}%
\pgfpathcurveto{\pgfqpoint{2.818110in}{1.110021in}}{\pgfqpoint{2.814838in}{1.117921in}}{\pgfqpoint{2.809014in}{1.123745in}}%
\pgfpathcurveto{\pgfqpoint{2.803190in}{1.129569in}}{\pgfqpoint{2.795290in}{1.132842in}}{\pgfqpoint{2.787054in}{1.132842in}}%
\pgfpathcurveto{\pgfqpoint{2.778817in}{1.132842in}}{\pgfqpoint{2.770917in}{1.129569in}}{\pgfqpoint{2.765094in}{1.123745in}}%
\pgfpathcurveto{\pgfqpoint{2.759270in}{1.117921in}}{\pgfqpoint{2.755997in}{1.110021in}}{\pgfqpoint{2.755997in}{1.101785in}}%
\pgfpathcurveto{\pgfqpoint{2.755997in}{1.093549in}}{\pgfqpoint{2.759270in}{1.085649in}}{\pgfqpoint{2.765094in}{1.079825in}}%
\pgfpathcurveto{\pgfqpoint{2.770917in}{1.074001in}}{\pgfqpoint{2.778817in}{1.070729in}}{\pgfqpoint{2.787054in}{1.070729in}}%
\pgfpathclose%
\pgfusepath{stroke,fill}%
\end{pgfscope}%
\begin{pgfscope}%
\pgfpathrectangle{\pgfqpoint{0.100000in}{0.212622in}}{\pgfqpoint{3.696000in}{3.696000in}}%
\pgfusepath{clip}%
\pgfsetbuttcap%
\pgfsetroundjoin%
\definecolor{currentfill}{rgb}{0.121569,0.466667,0.705882}%
\pgfsetfillcolor{currentfill}%
\pgfsetfillopacity{0.961894}%
\pgfsetlinewidth{1.003750pt}%
\definecolor{currentstroke}{rgb}{0.121569,0.466667,0.705882}%
\pgfsetstrokecolor{currentstroke}%
\pgfsetstrokeopacity{0.961894}%
\pgfsetdash{}{0pt}%
\pgfpathmoveto{\pgfqpoint{2.793432in}{1.068527in}}%
\pgfpathcurveto{\pgfqpoint{2.801668in}{1.068527in}}{\pgfqpoint{2.809568in}{1.071799in}}{\pgfqpoint{2.815392in}{1.077623in}}%
\pgfpathcurveto{\pgfqpoint{2.821216in}{1.083447in}}{\pgfqpoint{2.824488in}{1.091347in}}{\pgfqpoint{2.824488in}{1.099583in}}%
\pgfpathcurveto{\pgfqpoint{2.824488in}{1.107819in}}{\pgfqpoint{2.821216in}{1.115719in}}{\pgfqpoint{2.815392in}{1.121543in}}%
\pgfpathcurveto{\pgfqpoint{2.809568in}{1.127367in}}{\pgfqpoint{2.801668in}{1.130640in}}{\pgfqpoint{2.793432in}{1.130640in}}%
\pgfpathcurveto{\pgfqpoint{2.785195in}{1.130640in}}{\pgfqpoint{2.777295in}{1.127367in}}{\pgfqpoint{2.771471in}{1.121543in}}%
\pgfpathcurveto{\pgfqpoint{2.765647in}{1.115719in}}{\pgfqpoint{2.762375in}{1.107819in}}{\pgfqpoint{2.762375in}{1.099583in}}%
\pgfpathcurveto{\pgfqpoint{2.762375in}{1.091347in}}{\pgfqpoint{2.765647in}{1.083447in}}{\pgfqpoint{2.771471in}{1.077623in}}%
\pgfpathcurveto{\pgfqpoint{2.777295in}{1.071799in}}{\pgfqpoint{2.785195in}{1.068527in}}{\pgfqpoint{2.793432in}{1.068527in}}%
\pgfpathclose%
\pgfusepath{stroke,fill}%
\end{pgfscope}%
\begin{pgfscope}%
\pgfpathrectangle{\pgfqpoint{0.100000in}{0.212622in}}{\pgfqpoint{3.696000in}{3.696000in}}%
\pgfusepath{clip}%
\pgfsetbuttcap%
\pgfsetroundjoin%
\definecolor{currentfill}{rgb}{0.121569,0.466667,0.705882}%
\pgfsetfillcolor{currentfill}%
\pgfsetfillopacity{0.963278}%
\pgfsetlinewidth{1.003750pt}%
\definecolor{currentstroke}{rgb}{0.121569,0.466667,0.705882}%
\pgfsetstrokecolor{currentstroke}%
\pgfsetstrokeopacity{0.963278}%
\pgfsetdash{}{0pt}%
\pgfpathmoveto{\pgfqpoint{2.796969in}{1.067466in}}%
\pgfpathcurveto{\pgfqpoint{2.805205in}{1.067466in}}{\pgfqpoint{2.813105in}{1.070738in}}{\pgfqpoint{2.818929in}{1.076562in}}%
\pgfpathcurveto{\pgfqpoint{2.824753in}{1.082386in}}{\pgfqpoint{2.828025in}{1.090286in}}{\pgfqpoint{2.828025in}{1.098522in}}%
\pgfpathcurveto{\pgfqpoint{2.828025in}{1.106759in}}{\pgfqpoint{2.824753in}{1.114659in}}{\pgfqpoint{2.818929in}{1.120483in}}%
\pgfpathcurveto{\pgfqpoint{2.813105in}{1.126307in}}{\pgfqpoint{2.805205in}{1.129579in}}{\pgfqpoint{2.796969in}{1.129579in}}%
\pgfpathcurveto{\pgfqpoint{2.788733in}{1.129579in}}{\pgfqpoint{2.780833in}{1.126307in}}{\pgfqpoint{2.775009in}{1.120483in}}%
\pgfpathcurveto{\pgfqpoint{2.769185in}{1.114659in}}{\pgfqpoint{2.765912in}{1.106759in}}{\pgfqpoint{2.765912in}{1.098522in}}%
\pgfpathcurveto{\pgfqpoint{2.765912in}{1.090286in}}{\pgfqpoint{2.769185in}{1.082386in}}{\pgfqpoint{2.775009in}{1.076562in}}%
\pgfpathcurveto{\pgfqpoint{2.780833in}{1.070738in}}{\pgfqpoint{2.788733in}{1.067466in}}{\pgfqpoint{2.796969in}{1.067466in}}%
\pgfpathclose%
\pgfusepath{stroke,fill}%
\end{pgfscope}%
\begin{pgfscope}%
\pgfpathrectangle{\pgfqpoint{0.100000in}{0.212622in}}{\pgfqpoint{3.696000in}{3.696000in}}%
\pgfusepath{clip}%
\pgfsetbuttcap%
\pgfsetroundjoin%
\definecolor{currentfill}{rgb}{0.121569,0.466667,0.705882}%
\pgfsetfillcolor{currentfill}%
\pgfsetfillopacity{0.964103}%
\pgfsetlinewidth{1.003750pt}%
\definecolor{currentstroke}{rgb}{0.121569,0.466667,0.705882}%
\pgfsetstrokecolor{currentstroke}%
\pgfsetstrokeopacity{0.964103}%
\pgfsetdash{}{0pt}%
\pgfpathmoveto{\pgfqpoint{2.798835in}{1.066730in}}%
\pgfpathcurveto{\pgfqpoint{2.807071in}{1.066730in}}{\pgfqpoint{2.814971in}{1.070003in}}{\pgfqpoint{2.820795in}{1.075827in}}%
\pgfpathcurveto{\pgfqpoint{2.826619in}{1.081650in}}{\pgfqpoint{2.829891in}{1.089551in}}{\pgfqpoint{2.829891in}{1.097787in}}%
\pgfpathcurveto{\pgfqpoint{2.829891in}{1.106023in}}{\pgfqpoint{2.826619in}{1.113923in}}{\pgfqpoint{2.820795in}{1.119747in}}%
\pgfpathcurveto{\pgfqpoint{2.814971in}{1.125571in}}{\pgfqpoint{2.807071in}{1.128843in}}{\pgfqpoint{2.798835in}{1.128843in}}%
\pgfpathcurveto{\pgfqpoint{2.790598in}{1.128843in}}{\pgfqpoint{2.782698in}{1.125571in}}{\pgfqpoint{2.776874in}{1.119747in}}%
\pgfpathcurveto{\pgfqpoint{2.771051in}{1.113923in}}{\pgfqpoint{2.767778in}{1.106023in}}{\pgfqpoint{2.767778in}{1.097787in}}%
\pgfpathcurveto{\pgfqpoint{2.767778in}{1.089551in}}{\pgfqpoint{2.771051in}{1.081650in}}{\pgfqpoint{2.776874in}{1.075827in}}%
\pgfpathcurveto{\pgfqpoint{2.782698in}{1.070003in}}{\pgfqpoint{2.790598in}{1.066730in}}{\pgfqpoint{2.798835in}{1.066730in}}%
\pgfpathclose%
\pgfusepath{stroke,fill}%
\end{pgfscope}%
\begin{pgfscope}%
\pgfpathrectangle{\pgfqpoint{0.100000in}{0.212622in}}{\pgfqpoint{3.696000in}{3.696000in}}%
\pgfusepath{clip}%
\pgfsetbuttcap%
\pgfsetroundjoin%
\definecolor{currentfill}{rgb}{0.121569,0.466667,0.705882}%
\pgfsetfillcolor{currentfill}%
\pgfsetfillopacity{0.965373}%
\pgfsetlinewidth{1.003750pt}%
\definecolor{currentstroke}{rgb}{0.121569,0.466667,0.705882}%
\pgfsetstrokecolor{currentstroke}%
\pgfsetstrokeopacity{0.965373}%
\pgfsetdash{}{0pt}%
\pgfpathmoveto{\pgfqpoint{2.801245in}{1.065615in}}%
\pgfpathcurveto{\pgfqpoint{2.809482in}{1.065615in}}{\pgfqpoint{2.817382in}{1.068888in}}{\pgfqpoint{2.823206in}{1.074712in}}%
\pgfpathcurveto{\pgfqpoint{2.829030in}{1.080535in}}{\pgfqpoint{2.832302in}{1.088436in}}{\pgfqpoint{2.832302in}{1.096672in}}%
\pgfpathcurveto{\pgfqpoint{2.832302in}{1.104908in}}{\pgfqpoint{2.829030in}{1.112808in}}{\pgfqpoint{2.823206in}{1.118632in}}%
\pgfpathcurveto{\pgfqpoint{2.817382in}{1.124456in}}{\pgfqpoint{2.809482in}{1.127728in}}{\pgfqpoint{2.801245in}{1.127728in}}%
\pgfpathcurveto{\pgfqpoint{2.793009in}{1.127728in}}{\pgfqpoint{2.785109in}{1.124456in}}{\pgfqpoint{2.779285in}{1.118632in}}%
\pgfpathcurveto{\pgfqpoint{2.773461in}{1.112808in}}{\pgfqpoint{2.770189in}{1.104908in}}{\pgfqpoint{2.770189in}{1.096672in}}%
\pgfpathcurveto{\pgfqpoint{2.770189in}{1.088436in}}{\pgfqpoint{2.773461in}{1.080535in}}{\pgfqpoint{2.779285in}{1.074712in}}%
\pgfpathcurveto{\pgfqpoint{2.785109in}{1.068888in}}{\pgfqpoint{2.793009in}{1.065615in}}{\pgfqpoint{2.801245in}{1.065615in}}%
\pgfpathclose%
\pgfusepath{stroke,fill}%
\end{pgfscope}%
\begin{pgfscope}%
\pgfpathrectangle{\pgfqpoint{0.100000in}{0.212622in}}{\pgfqpoint{3.696000in}{3.696000in}}%
\pgfusepath{clip}%
\pgfsetbuttcap%
\pgfsetroundjoin%
\definecolor{currentfill}{rgb}{0.121569,0.466667,0.705882}%
\pgfsetfillcolor{currentfill}%
\pgfsetfillopacity{0.966720}%
\pgfsetlinewidth{1.003750pt}%
\definecolor{currentstroke}{rgb}{0.121569,0.466667,0.705882}%
\pgfsetstrokecolor{currentstroke}%
\pgfsetstrokeopacity{0.966720}%
\pgfsetdash{}{0pt}%
\pgfpathmoveto{\pgfqpoint{2.804249in}{1.064523in}}%
\pgfpathcurveto{\pgfqpoint{2.812486in}{1.064523in}}{\pgfqpoint{2.820386in}{1.067795in}}{\pgfqpoint{2.826210in}{1.073619in}}%
\pgfpathcurveto{\pgfqpoint{2.832033in}{1.079443in}}{\pgfqpoint{2.835306in}{1.087343in}}{\pgfqpoint{2.835306in}{1.095580in}}%
\pgfpathcurveto{\pgfqpoint{2.835306in}{1.103816in}}{\pgfqpoint{2.832033in}{1.111716in}}{\pgfqpoint{2.826210in}{1.117540in}}%
\pgfpathcurveto{\pgfqpoint{2.820386in}{1.123364in}}{\pgfqpoint{2.812486in}{1.126636in}}{\pgfqpoint{2.804249in}{1.126636in}}%
\pgfpathcurveto{\pgfqpoint{2.796013in}{1.126636in}}{\pgfqpoint{2.788113in}{1.123364in}}{\pgfqpoint{2.782289in}{1.117540in}}%
\pgfpathcurveto{\pgfqpoint{2.776465in}{1.111716in}}{\pgfqpoint{2.773193in}{1.103816in}}{\pgfqpoint{2.773193in}{1.095580in}}%
\pgfpathcurveto{\pgfqpoint{2.773193in}{1.087343in}}{\pgfqpoint{2.776465in}{1.079443in}}{\pgfqpoint{2.782289in}{1.073619in}}%
\pgfpathcurveto{\pgfqpoint{2.788113in}{1.067795in}}{\pgfqpoint{2.796013in}{1.064523in}}{\pgfqpoint{2.804249in}{1.064523in}}%
\pgfpathclose%
\pgfusepath{stroke,fill}%
\end{pgfscope}%
\begin{pgfscope}%
\pgfpathrectangle{\pgfqpoint{0.100000in}{0.212622in}}{\pgfqpoint{3.696000in}{3.696000in}}%
\pgfusepath{clip}%
\pgfsetbuttcap%
\pgfsetroundjoin%
\definecolor{currentfill}{rgb}{0.121569,0.466667,0.705882}%
\pgfsetfillcolor{currentfill}%
\pgfsetfillopacity{0.968486}%
\pgfsetlinewidth{1.003750pt}%
\definecolor{currentstroke}{rgb}{0.121569,0.466667,0.705882}%
\pgfsetstrokecolor{currentstroke}%
\pgfsetstrokeopacity{0.968486}%
\pgfsetdash{}{0pt}%
\pgfpathmoveto{\pgfqpoint{2.808694in}{1.063140in}}%
\pgfpathcurveto{\pgfqpoint{2.816930in}{1.063140in}}{\pgfqpoint{2.824830in}{1.066412in}}{\pgfqpoint{2.830654in}{1.072236in}}%
\pgfpathcurveto{\pgfqpoint{2.836478in}{1.078060in}}{\pgfqpoint{2.839750in}{1.085960in}}{\pgfqpoint{2.839750in}{1.094196in}}%
\pgfpathcurveto{\pgfqpoint{2.839750in}{1.102432in}}{\pgfqpoint{2.836478in}{1.110332in}}{\pgfqpoint{2.830654in}{1.116156in}}%
\pgfpathcurveto{\pgfqpoint{2.824830in}{1.121980in}}{\pgfqpoint{2.816930in}{1.125253in}}{\pgfqpoint{2.808694in}{1.125253in}}%
\pgfpathcurveto{\pgfqpoint{2.800457in}{1.125253in}}{\pgfqpoint{2.792557in}{1.121980in}}{\pgfqpoint{2.786734in}{1.116156in}}%
\pgfpathcurveto{\pgfqpoint{2.780910in}{1.110332in}}{\pgfqpoint{2.777637in}{1.102432in}}{\pgfqpoint{2.777637in}{1.094196in}}%
\pgfpathcurveto{\pgfqpoint{2.777637in}{1.085960in}}{\pgfqpoint{2.780910in}{1.078060in}}{\pgfqpoint{2.786734in}{1.072236in}}%
\pgfpathcurveto{\pgfqpoint{2.792557in}{1.066412in}}{\pgfqpoint{2.800457in}{1.063140in}}{\pgfqpoint{2.808694in}{1.063140in}}%
\pgfpathclose%
\pgfusepath{stroke,fill}%
\end{pgfscope}%
\begin{pgfscope}%
\pgfpathrectangle{\pgfqpoint{0.100000in}{0.212622in}}{\pgfqpoint{3.696000in}{3.696000in}}%
\pgfusepath{clip}%
\pgfsetbuttcap%
\pgfsetroundjoin%
\definecolor{currentfill}{rgb}{0.121569,0.466667,0.705882}%
\pgfsetfillcolor{currentfill}%
\pgfsetfillopacity{0.970668}%
\pgfsetlinewidth{1.003750pt}%
\definecolor{currentstroke}{rgb}{0.121569,0.466667,0.705882}%
\pgfsetstrokecolor{currentstroke}%
\pgfsetstrokeopacity{0.970668}%
\pgfsetdash{}{0pt}%
\pgfpathmoveto{\pgfqpoint{2.813888in}{1.061387in}}%
\pgfpathcurveto{\pgfqpoint{2.822125in}{1.061387in}}{\pgfqpoint{2.830025in}{1.064659in}}{\pgfqpoint{2.835849in}{1.070483in}}%
\pgfpathcurveto{\pgfqpoint{2.841673in}{1.076307in}}{\pgfqpoint{2.844945in}{1.084207in}}{\pgfqpoint{2.844945in}{1.092443in}}%
\pgfpathcurveto{\pgfqpoint{2.844945in}{1.100679in}}{\pgfqpoint{2.841673in}{1.108579in}}{\pgfqpoint{2.835849in}{1.114403in}}%
\pgfpathcurveto{\pgfqpoint{2.830025in}{1.120227in}}{\pgfqpoint{2.822125in}{1.123500in}}{\pgfqpoint{2.813888in}{1.123500in}}%
\pgfpathcurveto{\pgfqpoint{2.805652in}{1.123500in}}{\pgfqpoint{2.797752in}{1.120227in}}{\pgfqpoint{2.791928in}{1.114403in}}%
\pgfpathcurveto{\pgfqpoint{2.786104in}{1.108579in}}{\pgfqpoint{2.782832in}{1.100679in}}{\pgfqpoint{2.782832in}{1.092443in}}%
\pgfpathcurveto{\pgfqpoint{2.782832in}{1.084207in}}{\pgfqpoint{2.786104in}{1.076307in}}{\pgfqpoint{2.791928in}{1.070483in}}%
\pgfpathcurveto{\pgfqpoint{2.797752in}{1.064659in}}{\pgfqpoint{2.805652in}{1.061387in}}{\pgfqpoint{2.813888in}{1.061387in}}%
\pgfpathclose%
\pgfusepath{stroke,fill}%
\end{pgfscope}%
\begin{pgfscope}%
\pgfpathrectangle{\pgfqpoint{0.100000in}{0.212622in}}{\pgfqpoint{3.696000in}{3.696000in}}%
\pgfusepath{clip}%
\pgfsetbuttcap%
\pgfsetroundjoin%
\definecolor{currentfill}{rgb}{0.121569,0.466667,0.705882}%
\pgfsetfillcolor{currentfill}%
\pgfsetfillopacity{0.973070}%
\pgfsetlinewidth{1.003750pt}%
\definecolor{currentstroke}{rgb}{0.121569,0.466667,0.705882}%
\pgfsetstrokecolor{currentstroke}%
\pgfsetstrokeopacity{0.973070}%
\pgfsetdash{}{0pt}%
\pgfpathmoveto{\pgfqpoint{2.820022in}{1.059334in}}%
\pgfpathcurveto{\pgfqpoint{2.828258in}{1.059334in}}{\pgfqpoint{2.836158in}{1.062606in}}{\pgfqpoint{2.841982in}{1.068430in}}%
\pgfpathcurveto{\pgfqpoint{2.847806in}{1.074254in}}{\pgfqpoint{2.851079in}{1.082154in}}{\pgfqpoint{2.851079in}{1.090390in}}%
\pgfpathcurveto{\pgfqpoint{2.851079in}{1.098626in}}{\pgfqpoint{2.847806in}{1.106526in}}{\pgfqpoint{2.841982in}{1.112350in}}%
\pgfpathcurveto{\pgfqpoint{2.836158in}{1.118174in}}{\pgfqpoint{2.828258in}{1.121447in}}{\pgfqpoint{2.820022in}{1.121447in}}%
\pgfpathcurveto{\pgfqpoint{2.811786in}{1.121447in}}{\pgfqpoint{2.803886in}{1.118174in}}{\pgfqpoint{2.798062in}{1.112350in}}%
\pgfpathcurveto{\pgfqpoint{2.792238in}{1.106526in}}{\pgfqpoint{2.788966in}{1.098626in}}{\pgfqpoint{2.788966in}{1.090390in}}%
\pgfpathcurveto{\pgfqpoint{2.788966in}{1.082154in}}{\pgfqpoint{2.792238in}{1.074254in}}{\pgfqpoint{2.798062in}{1.068430in}}%
\pgfpathcurveto{\pgfqpoint{2.803886in}{1.062606in}}{\pgfqpoint{2.811786in}{1.059334in}}{\pgfqpoint{2.820022in}{1.059334in}}%
\pgfpathclose%
\pgfusepath{stroke,fill}%
\end{pgfscope}%
\begin{pgfscope}%
\pgfpathrectangle{\pgfqpoint{0.100000in}{0.212622in}}{\pgfqpoint{3.696000in}{3.696000in}}%
\pgfusepath{clip}%
\pgfsetbuttcap%
\pgfsetroundjoin%
\definecolor{currentfill}{rgb}{0.121569,0.466667,0.705882}%
\pgfsetfillcolor{currentfill}%
\pgfsetfillopacity{0.975843}%
\pgfsetlinewidth{1.003750pt}%
\definecolor{currentstroke}{rgb}{0.121569,0.466667,0.705882}%
\pgfsetstrokecolor{currentstroke}%
\pgfsetstrokeopacity{0.975843}%
\pgfsetdash{}{0pt}%
\pgfpathmoveto{\pgfqpoint{2.826963in}{1.057153in}}%
\pgfpathcurveto{\pgfqpoint{2.835199in}{1.057153in}}{\pgfqpoint{2.843100in}{1.060426in}}{\pgfqpoint{2.848923in}{1.066250in}}%
\pgfpathcurveto{\pgfqpoint{2.854747in}{1.072074in}}{\pgfqpoint{2.858020in}{1.079974in}}{\pgfqpoint{2.858020in}{1.088210in}}%
\pgfpathcurveto{\pgfqpoint{2.858020in}{1.096446in}}{\pgfqpoint{2.854747in}{1.104346in}}{\pgfqpoint{2.848923in}{1.110170in}}%
\pgfpathcurveto{\pgfqpoint{2.843100in}{1.115994in}}{\pgfqpoint{2.835199in}{1.119266in}}{\pgfqpoint{2.826963in}{1.119266in}}%
\pgfpathcurveto{\pgfqpoint{2.818727in}{1.119266in}}{\pgfqpoint{2.810827in}{1.115994in}}{\pgfqpoint{2.805003in}{1.110170in}}%
\pgfpathcurveto{\pgfqpoint{2.799179in}{1.104346in}}{\pgfqpoint{2.795907in}{1.096446in}}{\pgfqpoint{2.795907in}{1.088210in}}%
\pgfpathcurveto{\pgfqpoint{2.795907in}{1.079974in}}{\pgfqpoint{2.799179in}{1.072074in}}{\pgfqpoint{2.805003in}{1.066250in}}%
\pgfpathcurveto{\pgfqpoint{2.810827in}{1.060426in}}{\pgfqpoint{2.818727in}{1.057153in}}{\pgfqpoint{2.826963in}{1.057153in}}%
\pgfpathclose%
\pgfusepath{stroke,fill}%
\end{pgfscope}%
\begin{pgfscope}%
\pgfpathrectangle{\pgfqpoint{0.100000in}{0.212622in}}{\pgfqpoint{3.696000in}{3.696000in}}%
\pgfusepath{clip}%
\pgfsetbuttcap%
\pgfsetroundjoin%
\definecolor{currentfill}{rgb}{0.121569,0.466667,0.705882}%
\pgfsetfillcolor{currentfill}%
\pgfsetfillopacity{0.977339}%
\pgfsetlinewidth{1.003750pt}%
\definecolor{currentstroke}{rgb}{0.121569,0.466667,0.705882}%
\pgfsetstrokecolor{currentstroke}%
\pgfsetstrokeopacity{0.977339}%
\pgfsetdash{}{0pt}%
\pgfpathmoveto{\pgfqpoint{2.830801in}{1.055965in}}%
\pgfpathcurveto{\pgfqpoint{2.839038in}{1.055965in}}{\pgfqpoint{2.846938in}{1.059237in}}{\pgfqpoint{2.852762in}{1.065061in}}%
\pgfpathcurveto{\pgfqpoint{2.858585in}{1.070885in}}{\pgfqpoint{2.861858in}{1.078785in}}{\pgfqpoint{2.861858in}{1.087022in}}%
\pgfpathcurveto{\pgfqpoint{2.861858in}{1.095258in}}{\pgfqpoint{2.858585in}{1.103158in}}{\pgfqpoint{2.852762in}{1.108982in}}%
\pgfpathcurveto{\pgfqpoint{2.846938in}{1.114806in}}{\pgfqpoint{2.839038in}{1.118078in}}{\pgfqpoint{2.830801in}{1.118078in}}%
\pgfpathcurveto{\pgfqpoint{2.822565in}{1.118078in}}{\pgfqpoint{2.814665in}{1.114806in}}{\pgfqpoint{2.808841in}{1.108982in}}%
\pgfpathcurveto{\pgfqpoint{2.803017in}{1.103158in}}{\pgfqpoint{2.799745in}{1.095258in}}{\pgfqpoint{2.799745in}{1.087022in}}%
\pgfpathcurveto{\pgfqpoint{2.799745in}{1.078785in}}{\pgfqpoint{2.803017in}{1.070885in}}{\pgfqpoint{2.808841in}{1.065061in}}%
\pgfpathcurveto{\pgfqpoint{2.814665in}{1.059237in}}{\pgfqpoint{2.822565in}{1.055965in}}{\pgfqpoint{2.830801in}{1.055965in}}%
\pgfpathclose%
\pgfusepath{stroke,fill}%
\end{pgfscope}%
\begin{pgfscope}%
\pgfpathrectangle{\pgfqpoint{0.100000in}{0.212622in}}{\pgfqpoint{3.696000in}{3.696000in}}%
\pgfusepath{clip}%
\pgfsetbuttcap%
\pgfsetroundjoin%
\definecolor{currentfill}{rgb}{0.121569,0.466667,0.705882}%
\pgfsetfillcolor{currentfill}%
\pgfsetfillopacity{0.978402}%
\pgfsetlinewidth{1.003750pt}%
\definecolor{currentstroke}{rgb}{0.121569,0.466667,0.705882}%
\pgfsetstrokecolor{currentstroke}%
\pgfsetstrokeopacity{0.978402}%
\pgfsetdash{}{0pt}%
\pgfpathmoveto{\pgfqpoint{2.832705in}{1.055061in}}%
\pgfpathcurveto{\pgfqpoint{2.840942in}{1.055061in}}{\pgfqpoint{2.848842in}{1.058333in}}{\pgfqpoint{2.854666in}{1.064157in}}%
\pgfpathcurveto{\pgfqpoint{2.860490in}{1.069981in}}{\pgfqpoint{2.863762in}{1.077881in}}{\pgfqpoint{2.863762in}{1.086117in}}%
\pgfpathcurveto{\pgfqpoint{2.863762in}{1.094354in}}{\pgfqpoint{2.860490in}{1.102254in}}{\pgfqpoint{2.854666in}{1.108078in}}%
\pgfpathcurveto{\pgfqpoint{2.848842in}{1.113902in}}{\pgfqpoint{2.840942in}{1.117174in}}{\pgfqpoint{2.832705in}{1.117174in}}%
\pgfpathcurveto{\pgfqpoint{2.824469in}{1.117174in}}{\pgfqpoint{2.816569in}{1.113902in}}{\pgfqpoint{2.810745in}{1.108078in}}%
\pgfpathcurveto{\pgfqpoint{2.804921in}{1.102254in}}{\pgfqpoint{2.801649in}{1.094354in}}{\pgfqpoint{2.801649in}{1.086117in}}%
\pgfpathcurveto{\pgfqpoint{2.801649in}{1.077881in}}{\pgfqpoint{2.804921in}{1.069981in}}{\pgfqpoint{2.810745in}{1.064157in}}%
\pgfpathcurveto{\pgfqpoint{2.816569in}{1.058333in}}{\pgfqpoint{2.824469in}{1.055061in}}{\pgfqpoint{2.832705in}{1.055061in}}%
\pgfpathclose%
\pgfusepath{stroke,fill}%
\end{pgfscope}%
\begin{pgfscope}%
\pgfpathrectangle{\pgfqpoint{0.100000in}{0.212622in}}{\pgfqpoint{3.696000in}{3.696000in}}%
\pgfusepath{clip}%
\pgfsetbuttcap%
\pgfsetroundjoin%
\definecolor{currentfill}{rgb}{0.121569,0.466667,0.705882}%
\pgfsetfillcolor{currentfill}%
\pgfsetfillopacity{0.979664}%
\pgfsetlinewidth{1.003750pt}%
\definecolor{currentstroke}{rgb}{0.121569,0.466667,0.705882}%
\pgfsetstrokecolor{currentstroke}%
\pgfsetstrokeopacity{0.979664}%
\pgfsetdash{}{0pt}%
\pgfpathmoveto{\pgfqpoint{2.835182in}{1.053934in}}%
\pgfpathcurveto{\pgfqpoint{2.843418in}{1.053934in}}{\pgfqpoint{2.851319in}{1.057207in}}{\pgfqpoint{2.857142in}{1.063031in}}%
\pgfpathcurveto{\pgfqpoint{2.862966in}{1.068854in}}{\pgfqpoint{2.866239in}{1.076755in}}{\pgfqpoint{2.866239in}{1.084991in}}%
\pgfpathcurveto{\pgfqpoint{2.866239in}{1.093227in}}{\pgfqpoint{2.862966in}{1.101127in}}{\pgfqpoint{2.857142in}{1.106951in}}%
\pgfpathcurveto{\pgfqpoint{2.851319in}{1.112775in}}{\pgfqpoint{2.843418in}{1.116047in}}{\pgfqpoint{2.835182in}{1.116047in}}%
\pgfpathcurveto{\pgfqpoint{2.826946in}{1.116047in}}{\pgfqpoint{2.819046in}{1.112775in}}{\pgfqpoint{2.813222in}{1.106951in}}%
\pgfpathcurveto{\pgfqpoint{2.807398in}{1.101127in}}{\pgfqpoint{2.804126in}{1.093227in}}{\pgfqpoint{2.804126in}{1.084991in}}%
\pgfpathcurveto{\pgfqpoint{2.804126in}{1.076755in}}{\pgfqpoint{2.807398in}{1.068854in}}{\pgfqpoint{2.813222in}{1.063031in}}%
\pgfpathcurveto{\pgfqpoint{2.819046in}{1.057207in}}{\pgfqpoint{2.826946in}{1.053934in}}{\pgfqpoint{2.835182in}{1.053934in}}%
\pgfpathclose%
\pgfusepath{stroke,fill}%
\end{pgfscope}%
\begin{pgfscope}%
\pgfpathrectangle{\pgfqpoint{0.100000in}{0.212622in}}{\pgfqpoint{3.696000in}{3.696000in}}%
\pgfusepath{clip}%
\pgfsetbuttcap%
\pgfsetroundjoin%
\definecolor{currentfill}{rgb}{0.121569,0.466667,0.705882}%
\pgfsetfillcolor{currentfill}%
\pgfsetfillopacity{0.981079}%
\pgfsetlinewidth{1.003750pt}%
\definecolor{currentstroke}{rgb}{0.121569,0.466667,0.705882}%
\pgfsetstrokecolor{currentstroke}%
\pgfsetstrokeopacity{0.981079}%
\pgfsetdash{}{0pt}%
\pgfpathmoveto{\pgfqpoint{2.838515in}{1.052720in}}%
\pgfpathcurveto{\pgfqpoint{2.846751in}{1.052720in}}{\pgfqpoint{2.854651in}{1.055992in}}{\pgfqpoint{2.860475in}{1.061816in}}%
\pgfpathcurveto{\pgfqpoint{2.866299in}{1.067640in}}{\pgfqpoint{2.869571in}{1.075540in}}{\pgfqpoint{2.869571in}{1.083776in}}%
\pgfpathcurveto{\pgfqpoint{2.869571in}{1.092012in}}{\pgfqpoint{2.866299in}{1.099912in}}{\pgfqpoint{2.860475in}{1.105736in}}%
\pgfpathcurveto{\pgfqpoint{2.854651in}{1.111560in}}{\pgfqpoint{2.846751in}{1.114833in}}{\pgfqpoint{2.838515in}{1.114833in}}%
\pgfpathcurveto{\pgfqpoint{2.830278in}{1.114833in}}{\pgfqpoint{2.822378in}{1.111560in}}{\pgfqpoint{2.816554in}{1.105736in}}%
\pgfpathcurveto{\pgfqpoint{2.810730in}{1.099912in}}{\pgfqpoint{2.807458in}{1.092012in}}{\pgfqpoint{2.807458in}{1.083776in}}%
\pgfpathcurveto{\pgfqpoint{2.807458in}{1.075540in}}{\pgfqpoint{2.810730in}{1.067640in}}{\pgfqpoint{2.816554in}{1.061816in}}%
\pgfpathcurveto{\pgfqpoint{2.822378in}{1.055992in}}{\pgfqpoint{2.830278in}{1.052720in}}{\pgfqpoint{2.838515in}{1.052720in}}%
\pgfpathclose%
\pgfusepath{stroke,fill}%
\end{pgfscope}%
\begin{pgfscope}%
\pgfpathrectangle{\pgfqpoint{0.100000in}{0.212622in}}{\pgfqpoint{3.696000in}{3.696000in}}%
\pgfusepath{clip}%
\pgfsetbuttcap%
\pgfsetroundjoin%
\definecolor{currentfill}{rgb}{0.121569,0.466667,0.705882}%
\pgfsetfillcolor{currentfill}%
\pgfsetfillopacity{0.981861}%
\pgfsetlinewidth{1.003750pt}%
\definecolor{currentstroke}{rgb}{0.121569,0.466667,0.705882}%
\pgfsetstrokecolor{currentstroke}%
\pgfsetstrokeopacity{0.981861}%
\pgfsetdash{}{0pt}%
\pgfpathmoveto{\pgfqpoint{2.840353in}{1.052075in}}%
\pgfpathcurveto{\pgfqpoint{2.848589in}{1.052075in}}{\pgfqpoint{2.856489in}{1.055347in}}{\pgfqpoint{2.862313in}{1.061171in}}%
\pgfpathcurveto{\pgfqpoint{2.868137in}{1.066995in}}{\pgfqpoint{2.871409in}{1.074895in}}{\pgfqpoint{2.871409in}{1.083132in}}%
\pgfpathcurveto{\pgfqpoint{2.871409in}{1.091368in}}{\pgfqpoint{2.868137in}{1.099268in}}{\pgfqpoint{2.862313in}{1.105092in}}%
\pgfpathcurveto{\pgfqpoint{2.856489in}{1.110916in}}{\pgfqpoint{2.848589in}{1.114188in}}{\pgfqpoint{2.840353in}{1.114188in}}%
\pgfpathcurveto{\pgfqpoint{2.832117in}{1.114188in}}{\pgfqpoint{2.824217in}{1.110916in}}{\pgfqpoint{2.818393in}{1.105092in}}%
\pgfpathcurveto{\pgfqpoint{2.812569in}{1.099268in}}{\pgfqpoint{2.809296in}{1.091368in}}{\pgfqpoint{2.809296in}{1.083132in}}%
\pgfpathcurveto{\pgfqpoint{2.809296in}{1.074895in}}{\pgfqpoint{2.812569in}{1.066995in}}{\pgfqpoint{2.818393in}{1.061171in}}%
\pgfpathcurveto{\pgfqpoint{2.824217in}{1.055347in}}{\pgfqpoint{2.832117in}{1.052075in}}{\pgfqpoint{2.840353in}{1.052075in}}%
\pgfpathclose%
\pgfusepath{stroke,fill}%
\end{pgfscope}%
\begin{pgfscope}%
\pgfpathrectangle{\pgfqpoint{0.100000in}{0.212622in}}{\pgfqpoint{3.696000in}{3.696000in}}%
\pgfusepath{clip}%
\pgfsetbuttcap%
\pgfsetroundjoin%
\definecolor{currentfill}{rgb}{0.121569,0.466667,0.705882}%
\pgfsetfillcolor{currentfill}%
\pgfsetfillopacity{0.982266}%
\pgfsetlinewidth{1.003750pt}%
\definecolor{currentstroke}{rgb}{0.121569,0.466667,0.705882}%
\pgfsetstrokecolor{currentstroke}%
\pgfsetstrokeopacity{0.982266}%
\pgfsetdash{}{0pt}%
\pgfpathmoveto{\pgfqpoint{2.841386in}{1.051751in}}%
\pgfpathcurveto{\pgfqpoint{2.849623in}{1.051751in}}{\pgfqpoint{2.857523in}{1.055023in}}{\pgfqpoint{2.863347in}{1.060847in}}%
\pgfpathcurveto{\pgfqpoint{2.869171in}{1.066671in}}{\pgfqpoint{2.872443in}{1.074571in}}{\pgfqpoint{2.872443in}{1.082807in}}%
\pgfpathcurveto{\pgfqpoint{2.872443in}{1.091044in}}{\pgfqpoint{2.869171in}{1.098944in}}{\pgfqpoint{2.863347in}{1.104768in}}%
\pgfpathcurveto{\pgfqpoint{2.857523in}{1.110592in}}{\pgfqpoint{2.849623in}{1.113864in}}{\pgfqpoint{2.841386in}{1.113864in}}%
\pgfpathcurveto{\pgfqpoint{2.833150in}{1.113864in}}{\pgfqpoint{2.825250in}{1.110592in}}{\pgfqpoint{2.819426in}{1.104768in}}%
\pgfpathcurveto{\pgfqpoint{2.813602in}{1.098944in}}{\pgfqpoint{2.810330in}{1.091044in}}{\pgfqpoint{2.810330in}{1.082807in}}%
\pgfpathcurveto{\pgfqpoint{2.810330in}{1.074571in}}{\pgfqpoint{2.813602in}{1.066671in}}{\pgfqpoint{2.819426in}{1.060847in}}%
\pgfpathcurveto{\pgfqpoint{2.825250in}{1.055023in}}{\pgfqpoint{2.833150in}{1.051751in}}{\pgfqpoint{2.841386in}{1.051751in}}%
\pgfpathclose%
\pgfusepath{stroke,fill}%
\end{pgfscope}%
\begin{pgfscope}%
\pgfpathrectangle{\pgfqpoint{0.100000in}{0.212622in}}{\pgfqpoint{3.696000in}{3.696000in}}%
\pgfusepath{clip}%
\pgfsetbuttcap%
\pgfsetroundjoin%
\definecolor{currentfill}{rgb}{0.121569,0.466667,0.705882}%
\pgfsetfillcolor{currentfill}%
\pgfsetfillopacity{0.982593}%
\pgfsetlinewidth{1.003750pt}%
\definecolor{currentstroke}{rgb}{0.121569,0.466667,0.705882}%
\pgfsetstrokecolor{currentstroke}%
\pgfsetstrokeopacity{0.982593}%
\pgfsetdash{}{0pt}%
\pgfpathmoveto{\pgfqpoint{2.841864in}{1.051462in}}%
\pgfpathcurveto{\pgfqpoint{2.850101in}{1.051462in}}{\pgfqpoint{2.858001in}{1.054735in}}{\pgfqpoint{2.863825in}{1.060559in}}%
\pgfpathcurveto{\pgfqpoint{2.869649in}{1.066383in}}{\pgfqpoint{2.872921in}{1.074283in}}{\pgfqpoint{2.872921in}{1.082519in}}%
\pgfpathcurveto{\pgfqpoint{2.872921in}{1.090755in}}{\pgfqpoint{2.869649in}{1.098655in}}{\pgfqpoint{2.863825in}{1.104479in}}%
\pgfpathcurveto{\pgfqpoint{2.858001in}{1.110303in}}{\pgfqpoint{2.850101in}{1.113575in}}{\pgfqpoint{2.841864in}{1.113575in}}%
\pgfpathcurveto{\pgfqpoint{2.833628in}{1.113575in}}{\pgfqpoint{2.825728in}{1.110303in}}{\pgfqpoint{2.819904in}{1.104479in}}%
\pgfpathcurveto{\pgfqpoint{2.814080in}{1.098655in}}{\pgfqpoint{2.810808in}{1.090755in}}{\pgfqpoint{2.810808in}{1.082519in}}%
\pgfpathcurveto{\pgfqpoint{2.810808in}{1.074283in}}{\pgfqpoint{2.814080in}{1.066383in}}{\pgfqpoint{2.819904in}{1.060559in}}%
\pgfpathcurveto{\pgfqpoint{2.825728in}{1.054735in}}{\pgfqpoint{2.833628in}{1.051462in}}{\pgfqpoint{2.841864in}{1.051462in}}%
\pgfpathclose%
\pgfusepath{stroke,fill}%
\end{pgfscope}%
\begin{pgfscope}%
\pgfpathrectangle{\pgfqpoint{0.100000in}{0.212622in}}{\pgfqpoint{3.696000in}{3.696000in}}%
\pgfusepath{clip}%
\pgfsetbuttcap%
\pgfsetroundjoin%
\definecolor{currentfill}{rgb}{0.121569,0.466667,0.705882}%
\pgfsetfillcolor{currentfill}%
\pgfsetfillopacity{0.982747}%
\pgfsetlinewidth{1.003750pt}%
\definecolor{currentstroke}{rgb}{0.121569,0.466667,0.705882}%
\pgfsetstrokecolor{currentstroke}%
\pgfsetstrokeopacity{0.982747}%
\pgfsetdash{}{0pt}%
\pgfpathmoveto{\pgfqpoint{2.842151in}{1.051335in}}%
\pgfpathcurveto{\pgfqpoint{2.850387in}{1.051335in}}{\pgfqpoint{2.858287in}{1.054607in}}{\pgfqpoint{2.864111in}{1.060431in}}%
\pgfpathcurveto{\pgfqpoint{2.869935in}{1.066255in}}{\pgfqpoint{2.873207in}{1.074155in}}{\pgfqpoint{2.873207in}{1.082391in}}%
\pgfpathcurveto{\pgfqpoint{2.873207in}{1.090628in}}{\pgfqpoint{2.869935in}{1.098528in}}{\pgfqpoint{2.864111in}{1.104352in}}%
\pgfpathcurveto{\pgfqpoint{2.858287in}{1.110175in}}{\pgfqpoint{2.850387in}{1.113448in}}{\pgfqpoint{2.842151in}{1.113448in}}%
\pgfpathcurveto{\pgfqpoint{2.833915in}{1.113448in}}{\pgfqpoint{2.826015in}{1.110175in}}{\pgfqpoint{2.820191in}{1.104352in}}%
\pgfpathcurveto{\pgfqpoint{2.814367in}{1.098528in}}{\pgfqpoint{2.811094in}{1.090628in}}{\pgfqpoint{2.811094in}{1.082391in}}%
\pgfpathcurveto{\pgfqpoint{2.811094in}{1.074155in}}{\pgfqpoint{2.814367in}{1.066255in}}{\pgfqpoint{2.820191in}{1.060431in}}%
\pgfpathcurveto{\pgfqpoint{2.826015in}{1.054607in}}{\pgfqpoint{2.833915in}{1.051335in}}{\pgfqpoint{2.842151in}{1.051335in}}%
\pgfpathclose%
\pgfusepath{stroke,fill}%
\end{pgfscope}%
\begin{pgfscope}%
\pgfpathrectangle{\pgfqpoint{0.100000in}{0.212622in}}{\pgfqpoint{3.696000in}{3.696000in}}%
\pgfusepath{clip}%
\pgfsetbuttcap%
\pgfsetroundjoin%
\definecolor{currentfill}{rgb}{0.121569,0.466667,0.705882}%
\pgfsetfillcolor{currentfill}%
\pgfsetfillopacity{0.983333}%
\pgfsetlinewidth{1.003750pt}%
\definecolor{currentstroke}{rgb}{0.121569,0.466667,0.705882}%
\pgfsetstrokecolor{currentstroke}%
\pgfsetstrokeopacity{0.983333}%
\pgfsetdash{}{0pt}%
\pgfpathmoveto{\pgfqpoint{2.843348in}{1.050879in}}%
\pgfpathcurveto{\pgfqpoint{2.851585in}{1.050879in}}{\pgfqpoint{2.859485in}{1.054151in}}{\pgfqpoint{2.865309in}{1.059975in}}%
\pgfpathcurveto{\pgfqpoint{2.871133in}{1.065799in}}{\pgfqpoint{2.874405in}{1.073699in}}{\pgfqpoint{2.874405in}{1.081935in}}%
\pgfpathcurveto{\pgfqpoint{2.874405in}{1.090172in}}{\pgfqpoint{2.871133in}{1.098072in}}{\pgfqpoint{2.865309in}{1.103896in}}%
\pgfpathcurveto{\pgfqpoint{2.859485in}{1.109720in}}{\pgfqpoint{2.851585in}{1.112992in}}{\pgfqpoint{2.843348in}{1.112992in}}%
\pgfpathcurveto{\pgfqpoint{2.835112in}{1.112992in}}{\pgfqpoint{2.827212in}{1.109720in}}{\pgfqpoint{2.821388in}{1.103896in}}%
\pgfpathcurveto{\pgfqpoint{2.815564in}{1.098072in}}{\pgfqpoint{2.812292in}{1.090172in}}{\pgfqpoint{2.812292in}{1.081935in}}%
\pgfpathcurveto{\pgfqpoint{2.812292in}{1.073699in}}{\pgfqpoint{2.815564in}{1.065799in}}{\pgfqpoint{2.821388in}{1.059975in}}%
\pgfpathcurveto{\pgfqpoint{2.827212in}{1.054151in}}{\pgfqpoint{2.835112in}{1.050879in}}{\pgfqpoint{2.843348in}{1.050879in}}%
\pgfpathclose%
\pgfusepath{stroke,fill}%
\end{pgfscope}%
\begin{pgfscope}%
\pgfpathrectangle{\pgfqpoint{0.100000in}{0.212622in}}{\pgfqpoint{3.696000in}{3.696000in}}%
\pgfusepath{clip}%
\pgfsetbuttcap%
\pgfsetroundjoin%
\definecolor{currentfill}{rgb}{0.121569,0.466667,0.705882}%
\pgfsetfillcolor{currentfill}%
\pgfsetfillopacity{0.984545}%
\pgfsetlinewidth{1.003750pt}%
\definecolor{currentstroke}{rgb}{0.121569,0.466667,0.705882}%
\pgfsetstrokecolor{currentstroke}%
\pgfsetstrokeopacity{0.984545}%
\pgfsetdash{}{0pt}%
\pgfpathmoveto{\pgfqpoint{2.845731in}{1.049880in}}%
\pgfpathcurveto{\pgfqpoint{2.853968in}{1.049880in}}{\pgfqpoint{2.861868in}{1.053153in}}{\pgfqpoint{2.867692in}{1.058976in}}%
\pgfpathcurveto{\pgfqpoint{2.873516in}{1.064800in}}{\pgfqpoint{2.876788in}{1.072700in}}{\pgfqpoint{2.876788in}{1.080937in}}%
\pgfpathcurveto{\pgfqpoint{2.876788in}{1.089173in}}{\pgfqpoint{2.873516in}{1.097073in}}{\pgfqpoint{2.867692in}{1.102897in}}%
\pgfpathcurveto{\pgfqpoint{2.861868in}{1.108721in}}{\pgfqpoint{2.853968in}{1.111993in}}{\pgfqpoint{2.845731in}{1.111993in}}%
\pgfpathcurveto{\pgfqpoint{2.837495in}{1.111993in}}{\pgfqpoint{2.829595in}{1.108721in}}{\pgfqpoint{2.823771in}{1.102897in}}%
\pgfpathcurveto{\pgfqpoint{2.817947in}{1.097073in}}{\pgfqpoint{2.814675in}{1.089173in}}{\pgfqpoint{2.814675in}{1.080937in}}%
\pgfpathcurveto{\pgfqpoint{2.814675in}{1.072700in}}{\pgfqpoint{2.817947in}{1.064800in}}{\pgfqpoint{2.823771in}{1.058976in}}%
\pgfpathcurveto{\pgfqpoint{2.829595in}{1.053153in}}{\pgfqpoint{2.837495in}{1.049880in}}{\pgfqpoint{2.845731in}{1.049880in}}%
\pgfpathclose%
\pgfusepath{stroke,fill}%
\end{pgfscope}%
\begin{pgfscope}%
\pgfpathrectangle{\pgfqpoint{0.100000in}{0.212622in}}{\pgfqpoint{3.696000in}{3.696000in}}%
\pgfusepath{clip}%
\pgfsetbuttcap%
\pgfsetroundjoin%
\definecolor{currentfill}{rgb}{0.121569,0.466667,0.705882}%
\pgfsetfillcolor{currentfill}%
\pgfsetfillopacity{0.985834}%
\pgfsetlinewidth{1.003750pt}%
\definecolor{currentstroke}{rgb}{0.121569,0.466667,0.705882}%
\pgfsetstrokecolor{currentstroke}%
\pgfsetstrokeopacity{0.985834}%
\pgfsetdash{}{0pt}%
\pgfpathmoveto{\pgfqpoint{2.848945in}{1.048793in}}%
\pgfpathcurveto{\pgfqpoint{2.857181in}{1.048793in}}{\pgfqpoint{2.865081in}{1.052066in}}{\pgfqpoint{2.870905in}{1.057890in}}%
\pgfpathcurveto{\pgfqpoint{2.876729in}{1.063714in}}{\pgfqpoint{2.880001in}{1.071614in}}{\pgfqpoint{2.880001in}{1.079850in}}%
\pgfpathcurveto{\pgfqpoint{2.880001in}{1.088086in}}{\pgfqpoint{2.876729in}{1.095986in}}{\pgfqpoint{2.870905in}{1.101810in}}%
\pgfpathcurveto{\pgfqpoint{2.865081in}{1.107634in}}{\pgfqpoint{2.857181in}{1.110906in}}{\pgfqpoint{2.848945in}{1.110906in}}%
\pgfpathcurveto{\pgfqpoint{2.840708in}{1.110906in}}{\pgfqpoint{2.832808in}{1.107634in}}{\pgfqpoint{2.826984in}{1.101810in}}%
\pgfpathcurveto{\pgfqpoint{2.821160in}{1.095986in}}{\pgfqpoint{2.817888in}{1.088086in}}{\pgfqpoint{2.817888in}{1.079850in}}%
\pgfpathcurveto{\pgfqpoint{2.817888in}{1.071614in}}{\pgfqpoint{2.821160in}{1.063714in}}{\pgfqpoint{2.826984in}{1.057890in}}%
\pgfpathcurveto{\pgfqpoint{2.832808in}{1.052066in}}{\pgfqpoint{2.840708in}{1.048793in}}{\pgfqpoint{2.848945in}{1.048793in}}%
\pgfpathclose%
\pgfusepath{stroke,fill}%
\end{pgfscope}%
\begin{pgfscope}%
\pgfpathrectangle{\pgfqpoint{0.100000in}{0.212622in}}{\pgfqpoint{3.696000in}{3.696000in}}%
\pgfusepath{clip}%
\pgfsetbuttcap%
\pgfsetroundjoin%
\definecolor{currentfill}{rgb}{0.121569,0.466667,0.705882}%
\pgfsetfillcolor{currentfill}%
\pgfsetfillopacity{0.987725}%
\pgfsetlinewidth{1.003750pt}%
\definecolor{currentstroke}{rgb}{0.121569,0.466667,0.705882}%
\pgfsetstrokecolor{currentstroke}%
\pgfsetstrokeopacity{0.987725}%
\pgfsetdash{}{0pt}%
\pgfpathmoveto{\pgfqpoint{2.852659in}{1.047313in}}%
\pgfpathcurveto{\pgfqpoint{2.860896in}{1.047313in}}{\pgfqpoint{2.868796in}{1.050586in}}{\pgfqpoint{2.874620in}{1.056410in}}%
\pgfpathcurveto{\pgfqpoint{2.880444in}{1.062234in}}{\pgfqpoint{2.883716in}{1.070134in}}{\pgfqpoint{2.883716in}{1.078370in}}%
\pgfpathcurveto{\pgfqpoint{2.883716in}{1.086606in}}{\pgfqpoint{2.880444in}{1.094506in}}{\pgfqpoint{2.874620in}{1.100330in}}%
\pgfpathcurveto{\pgfqpoint{2.868796in}{1.106154in}}{\pgfqpoint{2.860896in}{1.109426in}}{\pgfqpoint{2.852659in}{1.109426in}}%
\pgfpathcurveto{\pgfqpoint{2.844423in}{1.109426in}}{\pgfqpoint{2.836523in}{1.106154in}}{\pgfqpoint{2.830699in}{1.100330in}}%
\pgfpathcurveto{\pgfqpoint{2.824875in}{1.094506in}}{\pgfqpoint{2.821603in}{1.086606in}}{\pgfqpoint{2.821603in}{1.078370in}}%
\pgfpathcurveto{\pgfqpoint{2.821603in}{1.070134in}}{\pgfqpoint{2.824875in}{1.062234in}}{\pgfqpoint{2.830699in}{1.056410in}}%
\pgfpathcurveto{\pgfqpoint{2.836523in}{1.050586in}}{\pgfqpoint{2.844423in}{1.047313in}}{\pgfqpoint{2.852659in}{1.047313in}}%
\pgfpathclose%
\pgfusepath{stroke,fill}%
\end{pgfscope}%
\begin{pgfscope}%
\pgfpathrectangle{\pgfqpoint{0.100000in}{0.212622in}}{\pgfqpoint{3.696000in}{3.696000in}}%
\pgfusepath{clip}%
\pgfsetbuttcap%
\pgfsetroundjoin%
\definecolor{currentfill}{rgb}{0.121569,0.466667,0.705882}%
\pgfsetfillcolor{currentfill}%
\pgfsetfillopacity{0.989801}%
\pgfsetlinewidth{1.003750pt}%
\definecolor{currentstroke}{rgb}{0.121569,0.466667,0.705882}%
\pgfsetstrokecolor{currentstroke}%
\pgfsetstrokeopacity{0.989801}%
\pgfsetdash{}{0pt}%
\pgfpathmoveto{\pgfqpoint{2.857708in}{1.045556in}}%
\pgfpathcurveto{\pgfqpoint{2.865945in}{1.045556in}}{\pgfqpoint{2.873845in}{1.048829in}}{\pgfqpoint{2.879669in}{1.054653in}}%
\pgfpathcurveto{\pgfqpoint{2.885492in}{1.060476in}}{\pgfqpoint{2.888765in}{1.068377in}}{\pgfqpoint{2.888765in}{1.076613in}}%
\pgfpathcurveto{\pgfqpoint{2.888765in}{1.084849in}}{\pgfqpoint{2.885492in}{1.092749in}}{\pgfqpoint{2.879669in}{1.098573in}}%
\pgfpathcurveto{\pgfqpoint{2.873845in}{1.104397in}}{\pgfqpoint{2.865945in}{1.107669in}}{\pgfqpoint{2.857708in}{1.107669in}}%
\pgfpathcurveto{\pgfqpoint{2.849472in}{1.107669in}}{\pgfqpoint{2.841572in}{1.104397in}}{\pgfqpoint{2.835748in}{1.098573in}}%
\pgfpathcurveto{\pgfqpoint{2.829924in}{1.092749in}}{\pgfqpoint{2.826652in}{1.084849in}}{\pgfqpoint{2.826652in}{1.076613in}}%
\pgfpathcurveto{\pgfqpoint{2.826652in}{1.068377in}}{\pgfqpoint{2.829924in}{1.060476in}}{\pgfqpoint{2.835748in}{1.054653in}}%
\pgfpathcurveto{\pgfqpoint{2.841572in}{1.048829in}}{\pgfqpoint{2.849472in}{1.045556in}}{\pgfqpoint{2.857708in}{1.045556in}}%
\pgfpathclose%
\pgfusepath{stroke,fill}%
\end{pgfscope}%
\begin{pgfscope}%
\pgfpathrectangle{\pgfqpoint{0.100000in}{0.212622in}}{\pgfqpoint{3.696000in}{3.696000in}}%
\pgfusepath{clip}%
\pgfsetbuttcap%
\pgfsetroundjoin%
\definecolor{currentfill}{rgb}{0.121569,0.466667,0.705882}%
\pgfsetfillcolor{currentfill}%
\pgfsetfillopacity{0.994084}%
\pgfsetlinewidth{1.003750pt}%
\definecolor{currentstroke}{rgb}{0.121569,0.466667,0.705882}%
\pgfsetstrokecolor{currentstroke}%
\pgfsetstrokeopacity{0.994084}%
\pgfsetdash{}{0pt}%
\pgfpathmoveto{\pgfqpoint{2.864464in}{1.041992in}}%
\pgfpathcurveto{\pgfqpoint{2.872700in}{1.041992in}}{\pgfqpoint{2.880600in}{1.045264in}}{\pgfqpoint{2.886424in}{1.051088in}}%
\pgfpathcurveto{\pgfqpoint{2.892248in}{1.056912in}}{\pgfqpoint{2.895520in}{1.064812in}}{\pgfqpoint{2.895520in}{1.073048in}}%
\pgfpathcurveto{\pgfqpoint{2.895520in}{1.081284in}}{\pgfqpoint{2.892248in}{1.089185in}}{\pgfqpoint{2.886424in}{1.095008in}}%
\pgfpathcurveto{\pgfqpoint{2.880600in}{1.100832in}}{\pgfqpoint{2.872700in}{1.104105in}}{\pgfqpoint{2.864464in}{1.104105in}}%
\pgfpathcurveto{\pgfqpoint{2.856227in}{1.104105in}}{\pgfqpoint{2.848327in}{1.100832in}}{\pgfqpoint{2.842504in}{1.095008in}}%
\pgfpathcurveto{\pgfqpoint{2.836680in}{1.089185in}}{\pgfqpoint{2.833407in}{1.081284in}}{\pgfqpoint{2.833407in}{1.073048in}}%
\pgfpathcurveto{\pgfqpoint{2.833407in}{1.064812in}}{\pgfqpoint{2.836680in}{1.056912in}}{\pgfqpoint{2.842504in}{1.051088in}}%
\pgfpathcurveto{\pgfqpoint{2.848327in}{1.045264in}}{\pgfqpoint{2.856227in}{1.041992in}}{\pgfqpoint{2.864464in}{1.041992in}}%
\pgfpathclose%
\pgfusepath{stroke,fill}%
\end{pgfscope}%
\begin{pgfscope}%
\pgfpathrectangle{\pgfqpoint{0.100000in}{0.212622in}}{\pgfqpoint{3.696000in}{3.696000in}}%
\pgfusepath{clip}%
\pgfsetbuttcap%
\pgfsetroundjoin%
\definecolor{currentfill}{rgb}{0.121569,0.466667,0.705882}%
\pgfsetfillcolor{currentfill}%
\pgfsetfillopacity{0.996305}%
\pgfsetlinewidth{1.003750pt}%
\definecolor{currentstroke}{rgb}{0.121569,0.466667,0.705882}%
\pgfsetstrokecolor{currentstroke}%
\pgfsetstrokeopacity{0.996305}%
\pgfsetdash{}{0pt}%
\pgfpathmoveto{\pgfqpoint{2.868256in}{1.040028in}}%
\pgfpathcurveto{\pgfqpoint{2.876492in}{1.040028in}}{\pgfqpoint{2.884392in}{1.043300in}}{\pgfqpoint{2.890216in}{1.049124in}}%
\pgfpathcurveto{\pgfqpoint{2.896040in}{1.054948in}}{\pgfqpoint{2.899312in}{1.062848in}}{\pgfqpoint{2.899312in}{1.071085in}}%
\pgfpathcurveto{\pgfqpoint{2.899312in}{1.079321in}}{\pgfqpoint{2.896040in}{1.087221in}}{\pgfqpoint{2.890216in}{1.093045in}}%
\pgfpathcurveto{\pgfqpoint{2.884392in}{1.098869in}}{\pgfqpoint{2.876492in}{1.102141in}}{\pgfqpoint{2.868256in}{1.102141in}}%
\pgfpathcurveto{\pgfqpoint{2.860020in}{1.102141in}}{\pgfqpoint{2.852120in}{1.098869in}}{\pgfqpoint{2.846296in}{1.093045in}}%
\pgfpathcurveto{\pgfqpoint{2.840472in}{1.087221in}}{\pgfqpoint{2.837199in}{1.079321in}}{\pgfqpoint{2.837199in}{1.071085in}}%
\pgfpathcurveto{\pgfqpoint{2.837199in}{1.062848in}}{\pgfqpoint{2.840472in}{1.054948in}}{\pgfqpoint{2.846296in}{1.049124in}}%
\pgfpathcurveto{\pgfqpoint{2.852120in}{1.043300in}}{\pgfqpoint{2.860020in}{1.040028in}}{\pgfqpoint{2.868256in}{1.040028in}}%
\pgfpathclose%
\pgfusepath{stroke,fill}%
\end{pgfscope}%
\begin{pgfscope}%
\pgfpathrectangle{\pgfqpoint{0.100000in}{0.212622in}}{\pgfqpoint{3.696000in}{3.696000in}}%
\pgfusepath{clip}%
\pgfsetbuttcap%
\pgfsetroundjoin%
\definecolor{currentfill}{rgb}{0.121569,0.466667,0.705882}%
\pgfsetfillcolor{currentfill}%
\pgfsetfillopacity{0.997397}%
\pgfsetlinewidth{1.003750pt}%
\definecolor{currentstroke}{rgb}{0.121569,0.466667,0.705882}%
\pgfsetstrokecolor{currentstroke}%
\pgfsetstrokeopacity{0.997397}%
\pgfsetdash{}{0pt}%
\pgfpathmoveto{\pgfqpoint{2.870461in}{1.039107in}}%
\pgfpathcurveto{\pgfqpoint{2.878697in}{1.039107in}}{\pgfqpoint{2.886597in}{1.042379in}}{\pgfqpoint{2.892421in}{1.048203in}}%
\pgfpathcurveto{\pgfqpoint{2.898245in}{1.054027in}}{\pgfqpoint{2.901517in}{1.061927in}}{\pgfqpoint{2.901517in}{1.070163in}}%
\pgfpathcurveto{\pgfqpoint{2.901517in}{1.078399in}}{\pgfqpoint{2.898245in}{1.086300in}}{\pgfqpoint{2.892421in}{1.092123in}}%
\pgfpathcurveto{\pgfqpoint{2.886597in}{1.097947in}}{\pgfqpoint{2.878697in}{1.101220in}}{\pgfqpoint{2.870461in}{1.101220in}}%
\pgfpathcurveto{\pgfqpoint{2.862224in}{1.101220in}}{\pgfqpoint{2.854324in}{1.097947in}}{\pgfqpoint{2.848501in}{1.092123in}}%
\pgfpathcurveto{\pgfqpoint{2.842677in}{1.086300in}}{\pgfqpoint{2.839404in}{1.078399in}}{\pgfqpoint{2.839404in}{1.070163in}}%
\pgfpathcurveto{\pgfqpoint{2.839404in}{1.061927in}}{\pgfqpoint{2.842677in}{1.054027in}}{\pgfqpoint{2.848501in}{1.048203in}}%
\pgfpathcurveto{\pgfqpoint{2.854324in}{1.042379in}}{\pgfqpoint{2.862224in}{1.039107in}}{\pgfqpoint{2.870461in}{1.039107in}}%
\pgfpathclose%
\pgfusepath{stroke,fill}%
\end{pgfscope}%
\begin{pgfscope}%
\pgfpathrectangle{\pgfqpoint{0.100000in}{0.212622in}}{\pgfqpoint{3.696000in}{3.696000in}}%
\pgfusepath{clip}%
\pgfsetbuttcap%
\pgfsetroundjoin%
\definecolor{currentfill}{rgb}{0.121569,0.466667,0.705882}%
\pgfsetfillcolor{currentfill}%
\pgfsetfillopacity{0.997997}%
\pgfsetlinewidth{1.003750pt}%
\definecolor{currentstroke}{rgb}{0.121569,0.466667,0.705882}%
\pgfsetstrokecolor{currentstroke}%
\pgfsetstrokeopacity{0.997997}%
\pgfsetdash{}{0pt}%
\pgfpathmoveto{\pgfqpoint{2.871680in}{1.038621in}}%
\pgfpathcurveto{\pgfqpoint{2.879917in}{1.038621in}}{\pgfqpoint{2.887817in}{1.041893in}}{\pgfqpoint{2.893641in}{1.047717in}}%
\pgfpathcurveto{\pgfqpoint{2.899464in}{1.053541in}}{\pgfqpoint{2.902737in}{1.061441in}}{\pgfqpoint{2.902737in}{1.069677in}}%
\pgfpathcurveto{\pgfqpoint{2.902737in}{1.077913in}}{\pgfqpoint{2.899464in}{1.085813in}}{\pgfqpoint{2.893641in}{1.091637in}}%
\pgfpathcurveto{\pgfqpoint{2.887817in}{1.097461in}}{\pgfqpoint{2.879917in}{1.100734in}}{\pgfqpoint{2.871680in}{1.100734in}}%
\pgfpathcurveto{\pgfqpoint{2.863444in}{1.100734in}}{\pgfqpoint{2.855544in}{1.097461in}}{\pgfqpoint{2.849720in}{1.091637in}}%
\pgfpathcurveto{\pgfqpoint{2.843896in}{1.085813in}}{\pgfqpoint{2.840624in}{1.077913in}}{\pgfqpoint{2.840624in}{1.069677in}}%
\pgfpathcurveto{\pgfqpoint{2.840624in}{1.061441in}}{\pgfqpoint{2.843896in}{1.053541in}}{\pgfqpoint{2.849720in}{1.047717in}}%
\pgfpathcurveto{\pgfqpoint{2.855544in}{1.041893in}}{\pgfqpoint{2.863444in}{1.038621in}}{\pgfqpoint{2.871680in}{1.038621in}}%
\pgfpathclose%
\pgfusepath{stroke,fill}%
\end{pgfscope}%
\begin{pgfscope}%
\pgfpathrectangle{\pgfqpoint{0.100000in}{0.212622in}}{\pgfqpoint{3.696000in}{3.696000in}}%
\pgfusepath{clip}%
\pgfsetbuttcap%
\pgfsetroundjoin%
\definecolor{currentfill}{rgb}{0.121569,0.466667,0.705882}%
\pgfsetfillcolor{currentfill}%
\pgfsetfillopacity{0.998360}%
\pgfsetlinewidth{1.003750pt}%
\definecolor{currentstroke}{rgb}{0.121569,0.466667,0.705882}%
\pgfsetstrokecolor{currentstroke}%
\pgfsetstrokeopacity{0.998360}%
\pgfsetdash{}{0pt}%
\pgfpathmoveto{\pgfqpoint{2.872324in}{1.038326in}}%
\pgfpathcurveto{\pgfqpoint{2.880561in}{1.038326in}}{\pgfqpoint{2.888461in}{1.041598in}}{\pgfqpoint{2.894285in}{1.047422in}}%
\pgfpathcurveto{\pgfqpoint{2.900109in}{1.053246in}}{\pgfqpoint{2.903381in}{1.061146in}}{\pgfqpoint{2.903381in}{1.069383in}}%
\pgfpathcurveto{\pgfqpoint{2.903381in}{1.077619in}}{\pgfqpoint{2.900109in}{1.085519in}}{\pgfqpoint{2.894285in}{1.091343in}}%
\pgfpathcurveto{\pgfqpoint{2.888461in}{1.097167in}}{\pgfqpoint{2.880561in}{1.100439in}}{\pgfqpoint{2.872324in}{1.100439in}}%
\pgfpathcurveto{\pgfqpoint{2.864088in}{1.100439in}}{\pgfqpoint{2.856188in}{1.097167in}}{\pgfqpoint{2.850364in}{1.091343in}}%
\pgfpathcurveto{\pgfqpoint{2.844540in}{1.085519in}}{\pgfqpoint{2.841268in}{1.077619in}}{\pgfqpoint{2.841268in}{1.069383in}}%
\pgfpathcurveto{\pgfqpoint{2.841268in}{1.061146in}}{\pgfqpoint{2.844540in}{1.053246in}}{\pgfqpoint{2.850364in}{1.047422in}}%
\pgfpathcurveto{\pgfqpoint{2.856188in}{1.041598in}}{\pgfqpoint{2.864088in}{1.038326in}}{\pgfqpoint{2.872324in}{1.038326in}}%
\pgfpathclose%
\pgfusepath{stroke,fill}%
\end{pgfscope}%
\begin{pgfscope}%
\pgfpathrectangle{\pgfqpoint{0.100000in}{0.212622in}}{\pgfqpoint{3.696000in}{3.696000in}}%
\pgfusepath{clip}%
\pgfsetbuttcap%
\pgfsetroundjoin%
\definecolor{currentfill}{rgb}{0.121569,0.466667,0.705882}%
\pgfsetfillcolor{currentfill}%
\pgfsetfillopacity{0.998544}%
\pgfsetlinewidth{1.003750pt}%
\definecolor{currentstroke}{rgb}{0.121569,0.466667,0.705882}%
\pgfsetstrokecolor{currentstroke}%
\pgfsetstrokeopacity{0.998544}%
\pgfsetdash{}{0pt}%
\pgfpathmoveto{\pgfqpoint{2.872691in}{1.038177in}}%
\pgfpathcurveto{\pgfqpoint{2.880928in}{1.038177in}}{\pgfqpoint{2.888828in}{1.041450in}}{\pgfqpoint{2.894652in}{1.047274in}}%
\pgfpathcurveto{\pgfqpoint{2.900476in}{1.053098in}}{\pgfqpoint{2.903748in}{1.060998in}}{\pgfqpoint{2.903748in}{1.069234in}}%
\pgfpathcurveto{\pgfqpoint{2.903748in}{1.077470in}}{\pgfqpoint{2.900476in}{1.085370in}}{\pgfqpoint{2.894652in}{1.091194in}}%
\pgfpathcurveto{\pgfqpoint{2.888828in}{1.097018in}}{\pgfqpoint{2.880928in}{1.100290in}}{\pgfqpoint{2.872691in}{1.100290in}}%
\pgfpathcurveto{\pgfqpoint{2.864455in}{1.100290in}}{\pgfqpoint{2.856555in}{1.097018in}}{\pgfqpoint{2.850731in}{1.091194in}}%
\pgfpathcurveto{\pgfqpoint{2.844907in}{1.085370in}}{\pgfqpoint{2.841635in}{1.077470in}}{\pgfqpoint{2.841635in}{1.069234in}}%
\pgfpathcurveto{\pgfqpoint{2.841635in}{1.060998in}}{\pgfqpoint{2.844907in}{1.053098in}}{\pgfqpoint{2.850731in}{1.047274in}}%
\pgfpathcurveto{\pgfqpoint{2.856555in}{1.041450in}}{\pgfqpoint{2.864455in}{1.038177in}}{\pgfqpoint{2.872691in}{1.038177in}}%
\pgfpathclose%
\pgfusepath{stroke,fill}%
\end{pgfscope}%
\begin{pgfscope}%
\pgfpathrectangle{\pgfqpoint{0.100000in}{0.212622in}}{\pgfqpoint{3.696000in}{3.696000in}}%
\pgfusepath{clip}%
\pgfsetbuttcap%
\pgfsetroundjoin%
\definecolor{currentfill}{rgb}{0.121569,0.466667,0.705882}%
\pgfsetfillcolor{currentfill}%
\pgfsetfillopacity{0.998917}%
\pgfsetlinewidth{1.003750pt}%
\definecolor{currentstroke}{rgb}{0.121569,0.466667,0.705882}%
\pgfsetstrokecolor{currentstroke}%
\pgfsetstrokeopacity{0.998917}%
\pgfsetdash{}{0pt}%
\pgfpathmoveto{\pgfqpoint{2.873507in}{1.037886in}}%
\pgfpathcurveto{\pgfqpoint{2.881743in}{1.037886in}}{\pgfqpoint{2.889643in}{1.041158in}}{\pgfqpoint{2.895467in}{1.046982in}}%
\pgfpathcurveto{\pgfqpoint{2.901291in}{1.052806in}}{\pgfqpoint{2.904564in}{1.060706in}}{\pgfqpoint{2.904564in}{1.068942in}}%
\pgfpathcurveto{\pgfqpoint{2.904564in}{1.077179in}}{\pgfqpoint{2.901291in}{1.085079in}}{\pgfqpoint{2.895467in}{1.090903in}}%
\pgfpathcurveto{\pgfqpoint{2.889643in}{1.096727in}}{\pgfqpoint{2.881743in}{1.099999in}}{\pgfqpoint{2.873507in}{1.099999in}}%
\pgfpathcurveto{\pgfqpoint{2.865271in}{1.099999in}}{\pgfqpoint{2.857371in}{1.096727in}}{\pgfqpoint{2.851547in}{1.090903in}}%
\pgfpathcurveto{\pgfqpoint{2.845723in}{1.085079in}}{\pgfqpoint{2.842451in}{1.077179in}}{\pgfqpoint{2.842451in}{1.068942in}}%
\pgfpathcurveto{\pgfqpoint{2.842451in}{1.060706in}}{\pgfqpoint{2.845723in}{1.052806in}}{\pgfqpoint{2.851547in}{1.046982in}}%
\pgfpathcurveto{\pgfqpoint{2.857371in}{1.041158in}}{\pgfqpoint{2.865271in}{1.037886in}}{\pgfqpoint{2.873507in}{1.037886in}}%
\pgfpathclose%
\pgfusepath{stroke,fill}%
\end{pgfscope}%
\begin{pgfscope}%
\pgfpathrectangle{\pgfqpoint{0.100000in}{0.212622in}}{\pgfqpoint{3.696000in}{3.696000in}}%
\pgfusepath{clip}%
\pgfsetbuttcap%
\pgfsetroundjoin%
\definecolor{currentfill}{rgb}{0.121569,0.466667,0.705882}%
\pgfsetfillcolor{currentfill}%
\pgfsetfillopacity{0.999433}%
\pgfsetlinewidth{1.003750pt}%
\definecolor{currentstroke}{rgb}{0.121569,0.466667,0.705882}%
\pgfsetstrokecolor{currentstroke}%
\pgfsetstrokeopacity{0.999433}%
\pgfsetdash{}{0pt}%
\pgfpathmoveto{\pgfqpoint{2.874849in}{1.037477in}}%
\pgfpathcurveto{\pgfqpoint{2.883085in}{1.037477in}}{\pgfqpoint{2.890985in}{1.040750in}}{\pgfqpoint{2.896809in}{1.046574in}}%
\pgfpathcurveto{\pgfqpoint{2.902633in}{1.052398in}}{\pgfqpoint{2.905905in}{1.060298in}}{\pgfqpoint{2.905905in}{1.068534in}}%
\pgfpathcurveto{\pgfqpoint{2.905905in}{1.076770in}}{\pgfqpoint{2.902633in}{1.084670in}}{\pgfqpoint{2.896809in}{1.090494in}}%
\pgfpathcurveto{\pgfqpoint{2.890985in}{1.096318in}}{\pgfqpoint{2.883085in}{1.099590in}}{\pgfqpoint{2.874849in}{1.099590in}}%
\pgfpathcurveto{\pgfqpoint{2.866613in}{1.099590in}}{\pgfqpoint{2.858713in}{1.096318in}}{\pgfqpoint{2.852889in}{1.090494in}}%
\pgfpathcurveto{\pgfqpoint{2.847065in}{1.084670in}}{\pgfqpoint{2.843792in}{1.076770in}}{\pgfqpoint{2.843792in}{1.068534in}}%
\pgfpathcurveto{\pgfqpoint{2.843792in}{1.060298in}}{\pgfqpoint{2.847065in}{1.052398in}}{\pgfqpoint{2.852889in}{1.046574in}}%
\pgfpathcurveto{\pgfqpoint{2.858713in}{1.040750in}}{\pgfqpoint{2.866613in}{1.037477in}}{\pgfqpoint{2.874849in}{1.037477in}}%
\pgfpathclose%
\pgfusepath{stroke,fill}%
\end{pgfscope}%
\begin{pgfscope}%
\pgfpathrectangle{\pgfqpoint{0.100000in}{0.212622in}}{\pgfqpoint{3.696000in}{3.696000in}}%
\pgfusepath{clip}%
\pgfsetbuttcap%
\pgfsetroundjoin%
\definecolor{currentfill}{rgb}{0.121569,0.466667,0.705882}%
\pgfsetfillcolor{currentfill}%
\pgfsetfillopacity{0.999780}%
\pgfsetlinewidth{1.003750pt}%
\definecolor{currentstroke}{rgb}{0.121569,0.466667,0.705882}%
\pgfsetstrokecolor{currentstroke}%
\pgfsetstrokeopacity{0.999780}%
\pgfsetdash{}{0pt}%
\pgfpathmoveto{\pgfqpoint{2.875517in}{1.037129in}}%
\pgfpathcurveto{\pgfqpoint{2.883753in}{1.037129in}}{\pgfqpoint{2.891653in}{1.040402in}}{\pgfqpoint{2.897477in}{1.046226in}}%
\pgfpathcurveto{\pgfqpoint{2.903301in}{1.052049in}}{\pgfqpoint{2.906573in}{1.059950in}}{\pgfqpoint{2.906573in}{1.068186in}}%
\pgfpathcurveto{\pgfqpoint{2.906573in}{1.076422in}}{\pgfqpoint{2.903301in}{1.084322in}}{\pgfqpoint{2.897477in}{1.090146in}}%
\pgfpathcurveto{\pgfqpoint{2.891653in}{1.095970in}}{\pgfqpoint{2.883753in}{1.099242in}}{\pgfqpoint{2.875517in}{1.099242in}}%
\pgfpathcurveto{\pgfqpoint{2.867280in}{1.099242in}}{\pgfqpoint{2.859380in}{1.095970in}}{\pgfqpoint{2.853556in}{1.090146in}}%
\pgfpathcurveto{\pgfqpoint{2.847732in}{1.084322in}}{\pgfqpoint{2.844460in}{1.076422in}}{\pgfqpoint{2.844460in}{1.068186in}}%
\pgfpathcurveto{\pgfqpoint{2.844460in}{1.059950in}}{\pgfqpoint{2.847732in}{1.052049in}}{\pgfqpoint{2.853556in}{1.046226in}}%
\pgfpathcurveto{\pgfqpoint{2.859380in}{1.040402in}}{\pgfqpoint{2.867280in}{1.037129in}}{\pgfqpoint{2.875517in}{1.037129in}}%
\pgfpathclose%
\pgfusepath{stroke,fill}%
\end{pgfscope}%
\begin{pgfscope}%
\pgfpathrectangle{\pgfqpoint{0.100000in}{0.212622in}}{\pgfqpoint{3.696000in}{3.696000in}}%
\pgfusepath{clip}%
\pgfsetbuttcap%
\pgfsetroundjoin%
\definecolor{currentfill}{rgb}{0.121569,0.466667,0.705882}%
\pgfsetfillcolor{currentfill}%
\pgfsetfillopacity{0.999791}%
\pgfsetlinewidth{1.003750pt}%
\definecolor{currentstroke}{rgb}{0.121569,0.466667,0.705882}%
\pgfsetstrokecolor{currentstroke}%
\pgfsetstrokeopacity{0.999791}%
\pgfsetdash{}{0pt}%
\pgfpathmoveto{\pgfqpoint{2.877219in}{1.037047in}}%
\pgfpathcurveto{\pgfqpoint{2.885455in}{1.037047in}}{\pgfqpoint{2.893355in}{1.040319in}}{\pgfqpoint{2.899179in}{1.046143in}}%
\pgfpathcurveto{\pgfqpoint{2.905003in}{1.051967in}}{\pgfqpoint{2.908276in}{1.059867in}}{\pgfqpoint{2.908276in}{1.068103in}}%
\pgfpathcurveto{\pgfqpoint{2.908276in}{1.076339in}}{\pgfqpoint{2.905003in}{1.084239in}}{\pgfqpoint{2.899179in}{1.090063in}}%
\pgfpathcurveto{\pgfqpoint{2.893355in}{1.095887in}}{\pgfqpoint{2.885455in}{1.099160in}}{\pgfqpoint{2.877219in}{1.099160in}}%
\pgfpathcurveto{\pgfqpoint{2.868983in}{1.099160in}}{\pgfqpoint{2.861083in}{1.095887in}}{\pgfqpoint{2.855259in}{1.090063in}}%
\pgfpathcurveto{\pgfqpoint{2.849435in}{1.084239in}}{\pgfqpoint{2.846163in}{1.076339in}}{\pgfqpoint{2.846163in}{1.068103in}}%
\pgfpathcurveto{\pgfqpoint{2.846163in}{1.059867in}}{\pgfqpoint{2.849435in}{1.051967in}}{\pgfqpoint{2.855259in}{1.046143in}}%
\pgfpathcurveto{\pgfqpoint{2.861083in}{1.040319in}}{\pgfqpoint{2.868983in}{1.037047in}}{\pgfqpoint{2.877219in}{1.037047in}}%
\pgfpathclose%
\pgfusepath{stroke,fill}%
\end{pgfscope}%
\begin{pgfscope}%
\pgfpathrectangle{\pgfqpoint{0.100000in}{0.212622in}}{\pgfqpoint{3.696000in}{3.696000in}}%
\pgfusepath{clip}%
\pgfsetbuttcap%
\pgfsetroundjoin%
\definecolor{currentfill}{rgb}{0.121569,0.466667,0.705882}%
\pgfsetfillcolor{currentfill}%
\pgfsetfillopacity{0.999903}%
\pgfsetlinewidth{1.003750pt}%
\definecolor{currentstroke}{rgb}{0.121569,0.466667,0.705882}%
\pgfsetstrokecolor{currentstroke}%
\pgfsetstrokeopacity{0.999903}%
\pgfsetdash{}{0pt}%
\pgfpathmoveto{\pgfqpoint{2.880599in}{1.036670in}}%
\pgfpathcurveto{\pgfqpoint{2.888835in}{1.036670in}}{\pgfqpoint{2.896735in}{1.039942in}}{\pgfqpoint{2.902559in}{1.045766in}}%
\pgfpathcurveto{\pgfqpoint{2.908383in}{1.051590in}}{\pgfqpoint{2.911655in}{1.059490in}}{\pgfqpoint{2.911655in}{1.067726in}}%
\pgfpathcurveto{\pgfqpoint{2.911655in}{1.075963in}}{\pgfqpoint{2.908383in}{1.083863in}}{\pgfqpoint{2.902559in}{1.089687in}}%
\pgfpathcurveto{\pgfqpoint{2.896735in}{1.095511in}}{\pgfqpoint{2.888835in}{1.098783in}}{\pgfqpoint{2.880599in}{1.098783in}}%
\pgfpathcurveto{\pgfqpoint{2.872363in}{1.098783in}}{\pgfqpoint{2.864463in}{1.095511in}}{\pgfqpoint{2.858639in}{1.089687in}}%
\pgfpathcurveto{\pgfqpoint{2.852815in}{1.083863in}}{\pgfqpoint{2.849542in}{1.075963in}}{\pgfqpoint{2.849542in}{1.067726in}}%
\pgfpathcurveto{\pgfqpoint{2.849542in}{1.059490in}}{\pgfqpoint{2.852815in}{1.051590in}}{\pgfqpoint{2.858639in}{1.045766in}}%
\pgfpathcurveto{\pgfqpoint{2.864463in}{1.039942in}}{\pgfqpoint{2.872363in}{1.036670in}}{\pgfqpoint{2.880599in}{1.036670in}}%
\pgfpathclose%
\pgfusepath{stroke,fill}%
\end{pgfscope}%
\begin{pgfscope}%
\pgfpathrectangle{\pgfqpoint{0.100000in}{0.212622in}}{\pgfqpoint{3.696000in}{3.696000in}}%
\pgfusepath{clip}%
\pgfsetbuttcap%
\pgfsetroundjoin%
\definecolor{currentfill}{rgb}{0.121569,0.466667,0.705882}%
\pgfsetfillcolor{currentfill}%
\pgfsetfillopacity{0.999904}%
\pgfsetlinewidth{1.003750pt}%
\definecolor{currentstroke}{rgb}{0.121569,0.466667,0.705882}%
\pgfsetstrokecolor{currentstroke}%
\pgfsetstrokeopacity{0.999904}%
\pgfsetdash{}{0pt}%
\pgfpathmoveto{\pgfqpoint{2.879668in}{1.036751in}}%
\pgfpathcurveto{\pgfqpoint{2.887904in}{1.036751in}}{\pgfqpoint{2.895805in}{1.040023in}}{\pgfqpoint{2.901628in}{1.045847in}}%
\pgfpathcurveto{\pgfqpoint{2.907452in}{1.051671in}}{\pgfqpoint{2.910725in}{1.059571in}}{\pgfqpoint{2.910725in}{1.067807in}}%
\pgfpathcurveto{\pgfqpoint{2.910725in}{1.076044in}}{\pgfqpoint{2.907452in}{1.083944in}}{\pgfqpoint{2.901628in}{1.089768in}}%
\pgfpathcurveto{\pgfqpoint{2.895805in}{1.095592in}}{\pgfqpoint{2.887904in}{1.098864in}}{\pgfqpoint{2.879668in}{1.098864in}}%
\pgfpathcurveto{\pgfqpoint{2.871432in}{1.098864in}}{\pgfqpoint{2.863532in}{1.095592in}}{\pgfqpoint{2.857708in}{1.089768in}}%
\pgfpathcurveto{\pgfqpoint{2.851884in}{1.083944in}}{\pgfqpoint{2.848612in}{1.076044in}}{\pgfqpoint{2.848612in}{1.067807in}}%
\pgfpathcurveto{\pgfqpoint{2.848612in}{1.059571in}}{\pgfqpoint{2.851884in}{1.051671in}}{\pgfqpoint{2.857708in}{1.045847in}}%
\pgfpathcurveto{\pgfqpoint{2.863532in}{1.040023in}}{\pgfqpoint{2.871432in}{1.036751in}}{\pgfqpoint{2.879668in}{1.036751in}}%
\pgfpathclose%
\pgfusepath{stroke,fill}%
\end{pgfscope}%
\begin{pgfscope}%
\pgfpathrectangle{\pgfqpoint{0.100000in}{0.212622in}}{\pgfqpoint{3.696000in}{3.696000in}}%
\pgfusepath{clip}%
\pgfsetbuttcap%
\pgfsetroundjoin%
\definecolor{currentfill}{rgb}{0.121569,0.466667,0.705882}%
\pgfsetfillcolor{currentfill}%
\pgfsetfillopacity{0.999956}%
\pgfsetlinewidth{1.003750pt}%
\definecolor{currentstroke}{rgb}{0.121569,0.466667,0.705882}%
\pgfsetstrokecolor{currentstroke}%
\pgfsetstrokeopacity{0.999956}%
\pgfsetdash{}{0pt}%
\pgfpathmoveto{\pgfqpoint{2.880227in}{1.036648in}}%
\pgfpathcurveto{\pgfqpoint{2.888463in}{1.036648in}}{\pgfqpoint{2.896363in}{1.039921in}}{\pgfqpoint{2.902187in}{1.045745in}}%
\pgfpathcurveto{\pgfqpoint{2.908011in}{1.051569in}}{\pgfqpoint{2.911283in}{1.059469in}}{\pgfqpoint{2.911283in}{1.067705in}}%
\pgfpathcurveto{\pgfqpoint{2.911283in}{1.075941in}}{\pgfqpoint{2.908011in}{1.083841in}}{\pgfqpoint{2.902187in}{1.089665in}}%
\pgfpathcurveto{\pgfqpoint{2.896363in}{1.095489in}}{\pgfqpoint{2.888463in}{1.098761in}}{\pgfqpoint{2.880227in}{1.098761in}}%
\pgfpathcurveto{\pgfqpoint{2.871991in}{1.098761in}}{\pgfqpoint{2.864091in}{1.095489in}}{\pgfqpoint{2.858267in}{1.089665in}}%
\pgfpathcurveto{\pgfqpoint{2.852443in}{1.083841in}}{\pgfqpoint{2.849170in}{1.075941in}}{\pgfqpoint{2.849170in}{1.067705in}}%
\pgfpathcurveto{\pgfqpoint{2.849170in}{1.059469in}}{\pgfqpoint{2.852443in}{1.051569in}}{\pgfqpoint{2.858267in}{1.045745in}}%
\pgfpathcurveto{\pgfqpoint{2.864091in}{1.039921in}}{\pgfqpoint{2.871991in}{1.036648in}}{\pgfqpoint{2.880227in}{1.036648in}}%
\pgfpathclose%
\pgfusepath{stroke,fill}%
\end{pgfscope}%
\begin{pgfscope}%
\pgfpathrectangle{\pgfqpoint{0.100000in}{0.212622in}}{\pgfqpoint{3.696000in}{3.696000in}}%
\pgfusepath{clip}%
\pgfsetbuttcap%
\pgfsetroundjoin%
\definecolor{currentfill}{rgb}{0.121569,0.466667,0.705882}%
\pgfsetfillcolor{currentfill}%
\pgfsetfillopacity{0.999980}%
\pgfsetlinewidth{1.003750pt}%
\definecolor{currentstroke}{rgb}{0.121569,0.466667,0.705882}%
\pgfsetstrokecolor{currentstroke}%
\pgfsetstrokeopacity{0.999980}%
\pgfsetdash{}{0pt}%
\pgfpathmoveto{\pgfqpoint{2.878507in}{1.036738in}}%
\pgfpathcurveto{\pgfqpoint{2.886743in}{1.036738in}}{\pgfqpoint{2.894643in}{1.040010in}}{\pgfqpoint{2.900467in}{1.045834in}}%
\pgfpathcurveto{\pgfqpoint{2.906291in}{1.051658in}}{\pgfqpoint{2.909563in}{1.059558in}}{\pgfqpoint{2.909563in}{1.067794in}}%
\pgfpathcurveto{\pgfqpoint{2.909563in}{1.076030in}}{\pgfqpoint{2.906291in}{1.083930in}}{\pgfqpoint{2.900467in}{1.089754in}}%
\pgfpathcurveto{\pgfqpoint{2.894643in}{1.095578in}}{\pgfqpoint{2.886743in}{1.098851in}}{\pgfqpoint{2.878507in}{1.098851in}}%
\pgfpathcurveto{\pgfqpoint{2.870270in}{1.098851in}}{\pgfqpoint{2.862370in}{1.095578in}}{\pgfqpoint{2.856546in}{1.089754in}}%
\pgfpathcurveto{\pgfqpoint{2.850722in}{1.083930in}}{\pgfqpoint{2.847450in}{1.076030in}}{\pgfqpoint{2.847450in}{1.067794in}}%
\pgfpathcurveto{\pgfqpoint{2.847450in}{1.059558in}}{\pgfqpoint{2.850722in}{1.051658in}}{\pgfqpoint{2.856546in}{1.045834in}}%
\pgfpathcurveto{\pgfqpoint{2.862370in}{1.040010in}}{\pgfqpoint{2.870270in}{1.036738in}}{\pgfqpoint{2.878507in}{1.036738in}}%
\pgfpathclose%
\pgfusepath{stroke,fill}%
\end{pgfscope}%
\begin{pgfscope}%
\pgfpathrectangle{\pgfqpoint{0.100000in}{0.212622in}}{\pgfqpoint{3.696000in}{3.696000in}}%
\pgfusepath{clip}%
\pgfsetbuttcap%
\pgfsetroundjoin%
\definecolor{currentfill}{rgb}{0.121569,0.466667,0.705882}%
\pgfsetfillcolor{currentfill}%
\pgfsetlinewidth{1.003750pt}%
\definecolor{currentstroke}{rgb}{0.121569,0.466667,0.705882}%
\pgfsetstrokecolor{currentstroke}%
\pgfsetdash{}{0pt}%
\pgfpathmoveto{\pgfqpoint{2.877975in}{1.036746in}}%
\pgfpathcurveto{\pgfqpoint{2.886211in}{1.036746in}}{\pgfqpoint{2.894111in}{1.040018in}}{\pgfqpoint{2.899935in}{1.045842in}}%
\pgfpathcurveto{\pgfqpoint{2.905759in}{1.051666in}}{\pgfqpoint{2.909031in}{1.059566in}}{\pgfqpoint{2.909031in}{1.067803in}}%
\pgfpathcurveto{\pgfqpoint{2.909031in}{1.076039in}}{\pgfqpoint{2.905759in}{1.083939in}}{\pgfqpoint{2.899935in}{1.089763in}}%
\pgfpathcurveto{\pgfqpoint{2.894111in}{1.095587in}}{\pgfqpoint{2.886211in}{1.098859in}}{\pgfqpoint{2.877975in}{1.098859in}}%
\pgfpathcurveto{\pgfqpoint{2.869738in}{1.098859in}}{\pgfqpoint{2.861838in}{1.095587in}}{\pgfqpoint{2.856014in}{1.089763in}}%
\pgfpathcurveto{\pgfqpoint{2.850190in}{1.083939in}}{\pgfqpoint{2.846918in}{1.076039in}}{\pgfqpoint{2.846918in}{1.067803in}}%
\pgfpathcurveto{\pgfqpoint{2.846918in}{1.059566in}}{\pgfqpoint{2.850190in}{1.051666in}}{\pgfqpoint{2.856014in}{1.045842in}}%
\pgfpathcurveto{\pgfqpoint{2.861838in}{1.040018in}}{\pgfqpoint{2.869738in}{1.036746in}}{\pgfqpoint{2.877975in}{1.036746in}}%
\pgfpathclose%
\pgfusepath{stroke,fill}%
\end{pgfscope}%
\begin{pgfscope}%
\definecolor{textcolor}{rgb}{0.000000,0.000000,0.000000}%
\pgfsetstrokecolor{textcolor}%
\pgfsetfillcolor{textcolor}%
\pgftext[x=1.948000in,y=3.991956in,,base]{\color{textcolor}\rmfamily\fontsize{12.000000}{14.400000}\selectfont FAMC}%
\end{pgfscope}%
\begin{pgfscope}%
\pgfsetbuttcap%
\pgfsetmiterjoin%
\definecolor{currentfill}{rgb}{1.000000,1.000000,1.000000}%
\pgfsetfillcolor{currentfill}%
\pgfsetfillopacity{0.800000}%
\pgfsetlinewidth{1.003750pt}%
\definecolor{currentstroke}{rgb}{0.800000,0.800000,0.800000}%
\pgfsetstrokecolor{currentstroke}%
\pgfsetstrokeopacity{0.800000}%
\pgfsetdash{}{0pt}%
\pgfpathmoveto{\pgfqpoint{2.104889in}{3.410289in}}%
\pgfpathlineto{\pgfqpoint{3.698778in}{3.410289in}}%
\pgfpathquadraticcurveto{\pgfqpoint{3.726556in}{3.410289in}}{\pgfqpoint{3.726556in}{3.438067in}}%
\pgfpathlineto{\pgfqpoint{3.726556in}{3.811400in}}%
\pgfpathquadraticcurveto{\pgfqpoint{3.726556in}{3.839178in}}{\pgfqpoint{3.698778in}{3.839178in}}%
\pgfpathlineto{\pgfqpoint{2.104889in}{3.839178in}}%
\pgfpathquadraticcurveto{\pgfqpoint{2.077111in}{3.839178in}}{\pgfqpoint{2.077111in}{3.811400in}}%
\pgfpathlineto{\pgfqpoint{2.077111in}{3.438067in}}%
\pgfpathquadraticcurveto{\pgfqpoint{2.077111in}{3.410289in}}{\pgfqpoint{2.104889in}{3.410289in}}%
\pgfpathclose%
\pgfusepath{stroke,fill}%
\end{pgfscope}%
\begin{pgfscope}%
\pgfsetrectcap%
\pgfsetroundjoin%
\pgfsetlinewidth{1.505625pt}%
\definecolor{currentstroke}{rgb}{0.121569,0.466667,0.705882}%
\pgfsetstrokecolor{currentstroke}%
\pgfsetdash{}{0pt}%
\pgfpathmoveto{\pgfqpoint{2.132667in}{3.735011in}}%
\pgfpathlineto{\pgfqpoint{2.410444in}{3.735011in}}%
\pgfusepath{stroke}%
\end{pgfscope}%
\begin{pgfscope}%
\definecolor{textcolor}{rgb}{0.000000,0.000000,0.000000}%
\pgfsetstrokecolor{textcolor}%
\pgfsetfillcolor{textcolor}%
\pgftext[x=2.521555in,y=3.686400in,left,base]{\color{textcolor}\rmfamily\fontsize{10.000000}{12.000000}\selectfont Ground truth}%
\end{pgfscope}%
\begin{pgfscope}%
\pgfsetbuttcap%
\pgfsetroundjoin%
\definecolor{currentfill}{rgb}{0.121569,0.466667,0.705882}%
\pgfsetfillcolor{currentfill}%
\pgfsetlinewidth{1.003750pt}%
\definecolor{currentstroke}{rgb}{0.121569,0.466667,0.705882}%
\pgfsetstrokecolor{currentstroke}%
\pgfsetdash{}{0pt}%
\pgfsys@defobject{currentmarker}{\pgfqpoint{-0.031056in}{-0.031056in}}{\pgfqpoint{0.031056in}{0.031056in}}{%
\pgfpathmoveto{\pgfqpoint{0.000000in}{-0.031056in}}%
\pgfpathcurveto{\pgfqpoint{0.008236in}{-0.031056in}}{\pgfqpoint{0.016136in}{-0.027784in}}{\pgfqpoint{0.021960in}{-0.021960in}}%
\pgfpathcurveto{\pgfqpoint{0.027784in}{-0.016136in}}{\pgfqpoint{0.031056in}{-0.008236in}}{\pgfqpoint{0.031056in}{0.000000in}}%
\pgfpathcurveto{\pgfqpoint{0.031056in}{0.008236in}}{\pgfqpoint{0.027784in}{0.016136in}}{\pgfqpoint{0.021960in}{0.021960in}}%
\pgfpathcurveto{\pgfqpoint{0.016136in}{0.027784in}}{\pgfqpoint{0.008236in}{0.031056in}}{\pgfqpoint{0.000000in}{0.031056in}}%
\pgfpathcurveto{\pgfqpoint{-0.008236in}{0.031056in}}{\pgfqpoint{-0.016136in}{0.027784in}}{\pgfqpoint{-0.021960in}{0.021960in}}%
\pgfpathcurveto{\pgfqpoint{-0.027784in}{0.016136in}}{\pgfqpoint{-0.031056in}{0.008236in}}{\pgfqpoint{-0.031056in}{0.000000in}}%
\pgfpathcurveto{\pgfqpoint{-0.031056in}{-0.008236in}}{\pgfqpoint{-0.027784in}{-0.016136in}}{\pgfqpoint{-0.021960in}{-0.021960in}}%
\pgfpathcurveto{\pgfqpoint{-0.016136in}{-0.027784in}}{\pgfqpoint{-0.008236in}{-0.031056in}}{\pgfqpoint{0.000000in}{-0.031056in}}%
\pgfpathclose%
\pgfusepath{stroke,fill}%
}%
\begin{pgfscope}%
\pgfsys@transformshift{2.271555in}{3.529248in}%
\pgfsys@useobject{currentmarker}{}%
\end{pgfscope}%
\end{pgfscope}%
\begin{pgfscope}%
\definecolor{textcolor}{rgb}{0.000000,0.000000,0.000000}%
\pgfsetstrokecolor{textcolor}%
\pgfsetfillcolor{textcolor}%
\pgftext[x=2.521555in,y=3.492789in,left,base]{\color{textcolor}\rmfamily\fontsize{10.000000}{12.000000}\selectfont Estimated position}%
\end{pgfscope}%
\end{pgfpicture}%
\makeatother%
\endgroup%
}
%         \caption{ ROLEQ's 3D position estimation had the lowest turn error for the 4-meter line experiment. }
%         \label{fig:line16_2D}
%     \end{subfigure}
%     \begin{subfigure}{0.49\textwidth}
%         \centering
%         \resizebox{1\linewidth}{!}{%% Creator: Matplotlib, PGF backend
%%
%% To include the figure in your LaTeX document, write
%%   \input{<filename>.pgf}
%%
%% Make sure the required packages are loaded in your preamble
%%   \usepackage{pgf}
%%
%% and, on pdftex
%%   \usepackage[utf8]{inputenc}\DeclareUnicodeCharacter{2212}{-}
%%
%% or, on luatex and xetex
%%   \usepackage{unicode-math}
%%
%% Figures using additional raster images can only be included by \input if
%% they are in the same directory as the main LaTeX file. For loading figures
%% from other directories you can use the `import` package
%%   \usepackage{import}
%%
%% and then include the figures with
%%   \import{<path to file>}{<filename>.pgf}
%%
%% Matplotlib used the following preamble
%%   \usepackage{fontspec}
%%
\begingroup%
\makeatletter%
\begin{pgfpicture}%
\pgfpathrectangle{\pgfpointorigin}{\pgfqpoint{4.342355in}{4.008622in}}%
\pgfusepath{use as bounding box, clip}%
\begin{pgfscope}%
\pgfsetbuttcap%
\pgfsetmiterjoin%
\definecolor{currentfill}{rgb}{1.000000,1.000000,1.000000}%
\pgfsetfillcolor{currentfill}%
\pgfsetlinewidth{0.000000pt}%
\definecolor{currentstroke}{rgb}{1.000000,1.000000,1.000000}%
\pgfsetstrokecolor{currentstroke}%
\pgfsetdash{}{0pt}%
\pgfpathmoveto{\pgfqpoint{0.000000in}{0.000000in}}%
\pgfpathlineto{\pgfqpoint{4.342355in}{0.000000in}}%
\pgfpathlineto{\pgfqpoint{4.342355in}{4.008622in}}%
\pgfpathlineto{\pgfqpoint{0.000000in}{4.008622in}}%
\pgfpathclose%
\pgfusepath{fill}%
\end{pgfscope}%
\begin{pgfscope}%
\pgfsetbuttcap%
\pgfsetmiterjoin%
\definecolor{currentfill}{rgb}{1.000000,1.000000,1.000000}%
\pgfsetfillcolor{currentfill}%
\pgfsetlinewidth{0.000000pt}%
\definecolor{currentstroke}{rgb}{0.000000,0.000000,0.000000}%
\pgfsetstrokecolor{currentstroke}%
\pgfsetstrokeopacity{0.000000}%
\pgfsetdash{}{0pt}%
\pgfpathmoveto{\pgfqpoint{0.100000in}{0.212622in}}%
\pgfpathlineto{\pgfqpoint{3.796000in}{0.212622in}}%
\pgfpathlineto{\pgfqpoint{3.796000in}{3.908622in}}%
\pgfpathlineto{\pgfqpoint{0.100000in}{3.908622in}}%
\pgfpathclose%
\pgfusepath{fill}%
\end{pgfscope}%
\begin{pgfscope}%
\pgfsetbuttcap%
\pgfsetmiterjoin%
\definecolor{currentfill}{rgb}{0.950000,0.950000,0.950000}%
\pgfsetfillcolor{currentfill}%
\pgfsetfillopacity{0.500000}%
\pgfsetlinewidth{1.003750pt}%
\definecolor{currentstroke}{rgb}{0.950000,0.950000,0.950000}%
\pgfsetstrokecolor{currentstroke}%
\pgfsetstrokeopacity{0.500000}%
\pgfsetdash{}{0pt}%
\pgfpathmoveto{\pgfqpoint{0.379073in}{1.123938in}}%
\pgfpathlineto{\pgfqpoint{1.599613in}{2.147018in}}%
\pgfpathlineto{\pgfqpoint{1.582647in}{3.622484in}}%
\pgfpathlineto{\pgfqpoint{0.303698in}{2.689165in}}%
\pgfusepath{stroke,fill}%
\end{pgfscope}%
\begin{pgfscope}%
\pgfsetbuttcap%
\pgfsetmiterjoin%
\definecolor{currentfill}{rgb}{0.900000,0.900000,0.900000}%
\pgfsetfillcolor{currentfill}%
\pgfsetfillopacity{0.500000}%
\pgfsetlinewidth{1.003750pt}%
\definecolor{currentstroke}{rgb}{0.900000,0.900000,0.900000}%
\pgfsetstrokecolor{currentstroke}%
\pgfsetstrokeopacity{0.500000}%
\pgfsetdash{}{0pt}%
\pgfpathmoveto{\pgfqpoint{1.599613in}{2.147018in}}%
\pgfpathlineto{\pgfqpoint{3.558144in}{1.577751in}}%
\pgfpathlineto{\pgfqpoint{3.628038in}{3.104037in}}%
\pgfpathlineto{\pgfqpoint{1.582647in}{3.622484in}}%
\pgfusepath{stroke,fill}%
\end{pgfscope}%
\begin{pgfscope}%
\pgfsetbuttcap%
\pgfsetmiterjoin%
\definecolor{currentfill}{rgb}{0.925000,0.925000,0.925000}%
\pgfsetfillcolor{currentfill}%
\pgfsetfillopacity{0.500000}%
\pgfsetlinewidth{1.003750pt}%
\definecolor{currentstroke}{rgb}{0.925000,0.925000,0.925000}%
\pgfsetstrokecolor{currentstroke}%
\pgfsetstrokeopacity{0.500000}%
\pgfsetdash{}{0pt}%
\pgfpathmoveto{\pgfqpoint{0.379073in}{1.123938in}}%
\pgfpathlineto{\pgfqpoint{2.455212in}{0.445871in}}%
\pgfpathlineto{\pgfqpoint{3.558144in}{1.577751in}}%
\pgfpathlineto{\pgfqpoint{1.599613in}{2.147018in}}%
\pgfusepath{stroke,fill}%
\end{pgfscope}%
\begin{pgfscope}%
\pgfsetrectcap%
\pgfsetroundjoin%
\pgfsetlinewidth{0.803000pt}%
\definecolor{currentstroke}{rgb}{0.000000,0.000000,0.000000}%
\pgfsetstrokecolor{currentstroke}%
\pgfsetdash{}{0pt}%
\pgfpathmoveto{\pgfqpoint{0.379073in}{1.123938in}}%
\pgfpathlineto{\pgfqpoint{2.455212in}{0.445871in}}%
\pgfusepath{stroke}%
\end{pgfscope}%
\begin{pgfscope}%
\definecolor{textcolor}{rgb}{0.000000,0.000000,0.000000}%
\pgfsetstrokecolor{textcolor}%
\pgfsetfillcolor{textcolor}%
\pgftext[x=0.730374in, y=0.408886in, left, base,rotate=341.912962]{\color{textcolor}\rmfamily\fontsize{10.000000}{12.000000}\selectfont Position X [\(\displaystyle m\)]}%
\end{pgfscope}%
\begin{pgfscope}%
\pgfsetbuttcap%
\pgfsetroundjoin%
\pgfsetlinewidth{0.803000pt}%
\definecolor{currentstroke}{rgb}{0.690196,0.690196,0.690196}%
\pgfsetstrokecolor{currentstroke}%
\pgfsetdash{}{0pt}%
\pgfpathmoveto{\pgfqpoint{0.504815in}{1.082870in}}%
\pgfpathlineto{\pgfqpoint{1.718725in}{2.112397in}}%
\pgfpathlineto{\pgfqpoint{1.706795in}{3.591016in}}%
\pgfusepath{stroke}%
\end{pgfscope}%
\begin{pgfscope}%
\pgfsetbuttcap%
\pgfsetroundjoin%
\pgfsetlinewidth{0.803000pt}%
\definecolor{currentstroke}{rgb}{0.690196,0.690196,0.690196}%
\pgfsetstrokecolor{currentstroke}%
\pgfsetdash{}{0pt}%
\pgfpathmoveto{\pgfqpoint{0.937302in}{0.941620in}}%
\pgfpathlineto{\pgfqpoint{2.127921in}{1.993460in}}%
\pgfpathlineto{\pgfqpoint{2.133534in}{3.482850in}}%
\pgfusepath{stroke}%
\end{pgfscope}%
\begin{pgfscope}%
\pgfsetbuttcap%
\pgfsetroundjoin%
\pgfsetlinewidth{0.803000pt}%
\definecolor{currentstroke}{rgb}{0.690196,0.690196,0.690196}%
\pgfsetstrokecolor{currentstroke}%
\pgfsetdash{}{0pt}%
\pgfpathmoveto{\pgfqpoint{1.379624in}{0.797158in}}%
\pgfpathlineto{\pgfqpoint{2.545645in}{1.872044in}}%
\pgfpathlineto{\pgfqpoint{2.569556in}{3.372331in}}%
\pgfusepath{stroke}%
\end{pgfscope}%
\begin{pgfscope}%
\pgfsetbuttcap%
\pgfsetroundjoin%
\pgfsetlinewidth{0.803000pt}%
\definecolor{currentstroke}{rgb}{0.690196,0.690196,0.690196}%
\pgfsetstrokecolor{currentstroke}%
\pgfsetdash{}{0pt}%
\pgfpathmoveto{\pgfqpoint{1.832122in}{0.649372in}}%
\pgfpathlineto{\pgfqpoint{2.972166in}{1.748072in}}%
\pgfpathlineto{\pgfqpoint{3.015165in}{3.259382in}}%
\pgfusepath{stroke}%
\end{pgfscope}%
\begin{pgfscope}%
\pgfsetbuttcap%
\pgfsetroundjoin%
\pgfsetlinewidth{0.803000pt}%
\definecolor{currentstroke}{rgb}{0.690196,0.690196,0.690196}%
\pgfsetstrokecolor{currentstroke}%
\pgfsetdash{}{0pt}%
\pgfpathmoveto{\pgfqpoint{2.295152in}{0.498146in}}%
\pgfpathlineto{\pgfqpoint{3.407765in}{1.621460in}}%
\pgfpathlineto{\pgfqpoint{3.470683in}{3.143922in}}%
\pgfusepath{stroke}%
\end{pgfscope}%
\begin{pgfscope}%
\pgfsetrectcap%
\pgfsetroundjoin%
\pgfsetlinewidth{0.803000pt}%
\definecolor{currentstroke}{rgb}{0.000000,0.000000,0.000000}%
\pgfsetstrokecolor{currentstroke}%
\pgfsetdash{}{0pt}%
\pgfpathmoveto{\pgfqpoint{0.515386in}{1.091835in}}%
\pgfpathlineto{\pgfqpoint{0.483629in}{1.064902in}}%
\pgfusepath{stroke}%
\end{pgfscope}%
\begin{pgfscope}%
\definecolor{textcolor}{rgb}{0.000000,0.000000,0.000000}%
\pgfsetstrokecolor{textcolor}%
\pgfsetfillcolor{textcolor}%
\pgftext[x=0.400245in,y=0.864666in,,top]{\color{textcolor}\rmfamily\fontsize{10.000000}{12.000000}\selectfont \(\displaystyle {0}\)}%
\end{pgfscope}%
\begin{pgfscope}%
\pgfsetrectcap%
\pgfsetroundjoin%
\pgfsetlinewidth{0.803000pt}%
\definecolor{currentstroke}{rgb}{0.000000,0.000000,0.000000}%
\pgfsetstrokecolor{currentstroke}%
\pgfsetdash{}{0pt}%
\pgfpathmoveto{\pgfqpoint{0.947679in}{0.950788in}}%
\pgfpathlineto{\pgfqpoint{0.916502in}{0.923245in}}%
\pgfusepath{stroke}%
\end{pgfscope}%
\begin{pgfscope}%
\definecolor{textcolor}{rgb}{0.000000,0.000000,0.000000}%
\pgfsetstrokecolor{textcolor}%
\pgfsetfillcolor{textcolor}%
\pgftext[x=0.833177in,y=0.720422in,,top]{\color{textcolor}\rmfamily\fontsize{10.000000}{12.000000}\selectfont \(\displaystyle {5}\)}%
\end{pgfscope}%
\begin{pgfscope}%
\pgfsetrectcap%
\pgfsetroundjoin%
\pgfsetlinewidth{0.803000pt}%
\definecolor{currentstroke}{rgb}{0.000000,0.000000,0.000000}%
\pgfsetstrokecolor{currentstroke}%
\pgfsetdash{}{0pt}%
\pgfpathmoveto{\pgfqpoint{1.389796in}{0.806535in}}%
\pgfpathlineto{\pgfqpoint{1.359235in}{0.778362in}}%
\pgfusepath{stroke}%
\end{pgfscope}%
\begin{pgfscope}%
\definecolor{textcolor}{rgb}{0.000000,0.000000,0.000000}%
\pgfsetstrokecolor{textcolor}%
\pgfsetfillcolor{textcolor}%
\pgftext[x=1.275993in,y=0.572885in,,top]{\color{textcolor}\rmfamily\fontsize{10.000000}{12.000000}\selectfont \(\displaystyle {10}\)}%
\end{pgfscope}%
\begin{pgfscope}%
\pgfsetrectcap%
\pgfsetroundjoin%
\pgfsetlinewidth{0.803000pt}%
\definecolor{currentstroke}{rgb}{0.000000,0.000000,0.000000}%
\pgfsetstrokecolor{currentstroke}%
\pgfsetdash{}{0pt}%
\pgfpathmoveto{\pgfqpoint{1.842078in}{0.658966in}}%
\pgfpathlineto{\pgfqpoint{1.812168in}{0.630141in}}%
\pgfusepath{stroke}%
\end{pgfscope}%
\begin{pgfscope}%
\definecolor{textcolor}{rgb}{0.000000,0.000000,0.000000}%
\pgfsetstrokecolor{textcolor}%
\pgfsetfillcolor{textcolor}%
\pgftext[x=1.729035in,y=0.421941in,,top]{\color{textcolor}\rmfamily\fontsize{10.000000}{12.000000}\selectfont \(\displaystyle {15}\)}%
\end{pgfscope}%
\begin{pgfscope}%
\pgfsetrectcap%
\pgfsetroundjoin%
\pgfsetlinewidth{0.803000pt}%
\definecolor{currentstroke}{rgb}{0.000000,0.000000,0.000000}%
\pgfsetstrokecolor{currentstroke}%
\pgfsetdash{}{0pt}%
\pgfpathmoveto{\pgfqpoint{2.304877in}{0.507965in}}%
\pgfpathlineto{\pgfqpoint{2.275658in}{0.478465in}}%
\pgfusepath{stroke}%
\end{pgfscope}%
\begin{pgfscope}%
\definecolor{textcolor}{rgb}{0.000000,0.000000,0.000000}%
\pgfsetstrokecolor{textcolor}%
\pgfsetfillcolor{textcolor}%
\pgftext[x=2.192662in,y=0.267471in,,top]{\color{textcolor}\rmfamily\fontsize{10.000000}{12.000000}\selectfont \(\displaystyle {20}\)}%
\end{pgfscope}%
\begin{pgfscope}%
\pgfsetrectcap%
\pgfsetroundjoin%
\pgfsetlinewidth{0.803000pt}%
\definecolor{currentstroke}{rgb}{0.000000,0.000000,0.000000}%
\pgfsetstrokecolor{currentstroke}%
\pgfsetdash{}{0pt}%
\pgfpathmoveto{\pgfqpoint{3.558144in}{1.577751in}}%
\pgfpathlineto{\pgfqpoint{2.455212in}{0.445871in}}%
\pgfusepath{stroke}%
\end{pgfscope}%
\begin{pgfscope}%
\definecolor{textcolor}{rgb}{0.000000,0.000000,0.000000}%
\pgfsetstrokecolor{textcolor}%
\pgfsetfillcolor{textcolor}%
\pgftext[x=3.120747in, y=0.305657in, left, base,rotate=45.742112]{\color{textcolor}\rmfamily\fontsize{10.000000}{12.000000}\selectfont Position Y [\(\displaystyle m\)]}%
\end{pgfscope}%
\begin{pgfscope}%
\pgfsetbuttcap%
\pgfsetroundjoin%
\pgfsetlinewidth{0.803000pt}%
\definecolor{currentstroke}{rgb}{0.690196,0.690196,0.690196}%
\pgfsetstrokecolor{currentstroke}%
\pgfsetdash{}{0pt}%
\pgfpathmoveto{\pgfqpoint{0.526119in}{2.851478in}}%
\pgfpathlineto{\pgfqpoint{0.590672in}{1.301303in}}%
\pgfpathlineto{\pgfqpoint{2.647120in}{0.642816in}}%
\pgfusepath{stroke}%
\end{pgfscope}%
\begin{pgfscope}%
\pgfsetbuttcap%
\pgfsetroundjoin%
\pgfsetlinewidth{0.803000pt}%
\definecolor{currentstroke}{rgb}{0.690196,0.690196,0.690196}%
\pgfsetstrokecolor{currentstroke}%
\pgfsetdash{}{0pt}%
\pgfpathmoveto{\pgfqpoint{0.733340in}{3.002698in}}%
\pgfpathlineto{\pgfqpoint{0.788061in}{1.466759in}}%
\pgfpathlineto{\pgfqpoint{2.825876in}{0.826263in}}%
\pgfusepath{stroke}%
\end{pgfscope}%
\begin{pgfscope}%
\pgfsetbuttcap%
\pgfsetroundjoin%
\pgfsetlinewidth{0.803000pt}%
\definecolor{currentstroke}{rgb}{0.690196,0.690196,0.690196}%
\pgfsetstrokecolor{currentstroke}%
\pgfsetdash{}{0pt}%
\pgfpathmoveto{\pgfqpoint{0.934834in}{3.149740in}}%
\pgfpathlineto{\pgfqpoint{0.980228in}{1.627837in}}%
\pgfpathlineto{\pgfqpoint{2.999658in}{1.004606in}}%
\pgfusepath{stroke}%
\end{pgfscope}%
\begin{pgfscope}%
\pgfsetbuttcap%
\pgfsetroundjoin%
\pgfsetlinewidth{0.803000pt}%
\definecolor{currentstroke}{rgb}{0.690196,0.690196,0.690196}%
\pgfsetstrokecolor{currentstroke}%
\pgfsetdash{}{0pt}%
\pgfpathmoveto{\pgfqpoint{1.130837in}{3.292774in}}%
\pgfpathlineto{\pgfqpoint{1.167378in}{1.784710in}}%
\pgfpathlineto{\pgfqpoint{3.168671in}{1.178055in}}%
\pgfusepath{stroke}%
\end{pgfscope}%
\begin{pgfscope}%
\pgfsetbuttcap%
\pgfsetroundjoin%
\pgfsetlinewidth{0.803000pt}%
\definecolor{currentstroke}{rgb}{0.690196,0.690196,0.690196}%
\pgfsetstrokecolor{currentstroke}%
\pgfsetdash{}{0pt}%
\pgfpathmoveto{\pgfqpoint{1.321569in}{3.431961in}}%
\pgfpathlineto{\pgfqpoint{1.349705in}{1.937540in}}%
\pgfpathlineto{\pgfqpoint{3.333109in}{1.346809in}}%
\pgfusepath{stroke}%
\end{pgfscope}%
\begin{pgfscope}%
\pgfsetbuttcap%
\pgfsetroundjoin%
\pgfsetlinewidth{0.803000pt}%
\definecolor{currentstroke}{rgb}{0.690196,0.690196,0.690196}%
\pgfsetstrokecolor{currentstroke}%
\pgfsetdash{}{0pt}%
\pgfpathmoveto{\pgfqpoint{1.507240in}{3.567456in}}%
\pgfpathlineto{\pgfqpoint{1.527393in}{2.086482in}}%
\pgfpathlineto{\pgfqpoint{3.493154in}{1.511054in}}%
\pgfusepath{stroke}%
\end{pgfscope}%
\begin{pgfscope}%
\pgfsetrectcap%
\pgfsetroundjoin%
\pgfsetlinewidth{0.803000pt}%
\definecolor{currentstroke}{rgb}{0.000000,0.000000,0.000000}%
\pgfsetstrokecolor{currentstroke}%
\pgfsetdash{}{0pt}%
\pgfpathmoveto{\pgfqpoint{2.629798in}{0.648362in}}%
\pgfpathlineto{\pgfqpoint{2.681808in}{0.631708in}}%
\pgfusepath{stroke}%
\end{pgfscope}%
\begin{pgfscope}%
\definecolor{textcolor}{rgb}{0.000000,0.000000,0.000000}%
\pgfsetstrokecolor{textcolor}%
\pgfsetfillcolor{textcolor}%
\pgftext[x=2.824468in,y=0.458138in,,top]{\color{textcolor}\rmfamily\fontsize{10.000000}{12.000000}\selectfont \(\displaystyle {-0.25}\)}%
\end{pgfscope}%
\begin{pgfscope}%
\pgfsetrectcap%
\pgfsetroundjoin%
\pgfsetlinewidth{0.803000pt}%
\definecolor{currentstroke}{rgb}{0.000000,0.000000,0.000000}%
\pgfsetstrokecolor{currentstroke}%
\pgfsetdash{}{0pt}%
\pgfpathmoveto{\pgfqpoint{2.808724in}{0.831655in}}%
\pgfpathlineto{\pgfqpoint{2.860225in}{0.815467in}}%
\pgfusepath{stroke}%
\end{pgfscope}%
\begin{pgfscope}%
\definecolor{textcolor}{rgb}{0.000000,0.000000,0.000000}%
\pgfsetstrokecolor{textcolor}%
\pgfsetfillcolor{textcolor}%
\pgftext[x=3.000825in,y=0.644300in,,top]{\color{textcolor}\rmfamily\fontsize{10.000000}{12.000000}\selectfont \(\displaystyle {-0.20}\)}%
\end{pgfscope}%
\begin{pgfscope}%
\pgfsetrectcap%
\pgfsetroundjoin%
\pgfsetlinewidth{0.803000pt}%
\definecolor{currentstroke}{rgb}{0.000000,0.000000,0.000000}%
\pgfsetstrokecolor{currentstroke}%
\pgfsetdash{}{0pt}%
\pgfpathmoveto{\pgfqpoint{2.982672in}{1.009849in}}%
\pgfpathlineto{\pgfqpoint{3.033673in}{0.994109in}}%
\pgfusepath{stroke}%
\end{pgfscope}%
\begin{pgfscope}%
\definecolor{textcolor}{rgb}{0.000000,0.000000,0.000000}%
\pgfsetstrokecolor{textcolor}%
\pgfsetfillcolor{textcolor}%
\pgftext[x=3.172272in,y=0.825279in,,top]{\color{textcolor}\rmfamily\fontsize{10.000000}{12.000000}\selectfont \(\displaystyle {-0.15}\)}%
\end{pgfscope}%
\begin{pgfscope}%
\pgfsetrectcap%
\pgfsetroundjoin%
\pgfsetlinewidth{0.803000pt}%
\definecolor{currentstroke}{rgb}{0.000000,0.000000,0.000000}%
\pgfsetstrokecolor{currentstroke}%
\pgfsetdash{}{0pt}%
\pgfpathmoveto{\pgfqpoint{3.151849in}{1.183155in}}%
\pgfpathlineto{\pgfqpoint{3.202357in}{1.167844in}}%
\pgfusepath{stroke}%
\end{pgfscope}%
\begin{pgfscope}%
\definecolor{textcolor}{rgb}{0.000000,0.000000,0.000000}%
\pgfsetstrokecolor{textcolor}%
\pgfsetfillcolor{textcolor}%
\pgftext[x=3.339011in,y=1.001289in,,top]{\color{textcolor}\rmfamily\fontsize{10.000000}{12.000000}\selectfont \(\displaystyle {-0.10}\)}%
\end{pgfscope}%
\begin{pgfscope}%
\pgfsetrectcap%
\pgfsetroundjoin%
\pgfsetlinewidth{0.803000pt}%
\definecolor{currentstroke}{rgb}{0.000000,0.000000,0.000000}%
\pgfsetstrokecolor{currentstroke}%
\pgfsetdash{}{0pt}%
\pgfpathmoveto{\pgfqpoint{3.316447in}{1.351771in}}%
\pgfpathlineto{\pgfqpoint{3.366471in}{1.336872in}}%
\pgfusepath{stroke}%
\end{pgfscope}%
\begin{pgfscope}%
\definecolor{textcolor}{rgb}{0.000000,0.000000,0.000000}%
\pgfsetstrokecolor{textcolor}%
\pgfsetfillcolor{textcolor}%
\pgftext[x=3.501234in,y=1.172531in,,top]{\color{textcolor}\rmfamily\fontsize{10.000000}{12.000000}\selectfont \(\displaystyle {-0.05}\)}%
\end{pgfscope}%
\begin{pgfscope}%
\pgfsetrectcap%
\pgfsetroundjoin%
\pgfsetlinewidth{0.803000pt}%
\definecolor{currentstroke}{rgb}{0.000000,0.000000,0.000000}%
\pgfsetstrokecolor{currentstroke}%
\pgfsetdash{}{0pt}%
\pgfpathmoveto{\pgfqpoint{3.476651in}{1.515885in}}%
\pgfpathlineto{\pgfqpoint{3.526198in}{1.501381in}}%
\pgfusepath{stroke}%
\end{pgfscope}%
\begin{pgfscope}%
\definecolor{textcolor}{rgb}{0.000000,0.000000,0.000000}%
\pgfsetstrokecolor{textcolor}%
\pgfsetfillcolor{textcolor}%
\pgftext[x=3.659121in,y=1.339196in,,top]{\color{textcolor}\rmfamily\fontsize{10.000000}{12.000000}\selectfont \(\displaystyle {0.00}\)}%
\end{pgfscope}%
\begin{pgfscope}%
\pgfsetrectcap%
\pgfsetroundjoin%
\pgfsetlinewidth{0.803000pt}%
\definecolor{currentstroke}{rgb}{0.000000,0.000000,0.000000}%
\pgfsetstrokecolor{currentstroke}%
\pgfsetdash{}{0pt}%
\pgfpathmoveto{\pgfqpoint{3.558144in}{1.577751in}}%
\pgfpathlineto{\pgfqpoint{3.628038in}{3.104037in}}%
\pgfusepath{stroke}%
\end{pgfscope}%
\begin{pgfscope}%
\definecolor{textcolor}{rgb}{0.000000,0.000000,0.000000}%
\pgfsetstrokecolor{textcolor}%
\pgfsetfillcolor{textcolor}%
\pgftext[x=4.167903in, y=1.963517in, left, base,rotate=87.378092]{\color{textcolor}\rmfamily\fontsize{10.000000}{12.000000}\selectfont Position Z [\(\displaystyle m\)]}%
\end{pgfscope}%
\begin{pgfscope}%
\pgfsetbuttcap%
\pgfsetroundjoin%
\pgfsetlinewidth{0.803000pt}%
\definecolor{currentstroke}{rgb}{0.690196,0.690196,0.690196}%
\pgfsetstrokecolor{currentstroke}%
\pgfsetdash{}{0pt}%
\pgfpathmoveto{\pgfqpoint{3.562413in}{1.670971in}}%
\pgfpathlineto{\pgfqpoint{1.598575in}{2.237314in}}%
\pgfpathlineto{\pgfqpoint{0.374477in}{1.219385in}}%
\pgfusepath{stroke}%
\end{pgfscope}%
\begin{pgfscope}%
\pgfsetbuttcap%
\pgfsetroundjoin%
\pgfsetlinewidth{0.803000pt}%
\definecolor{currentstroke}{rgb}{0.690196,0.690196,0.690196}%
\pgfsetstrokecolor{currentstroke}%
\pgfsetdash{}{0pt}%
\pgfpathmoveto{\pgfqpoint{3.576199in}{1.972019in}}%
\pgfpathlineto{\pgfqpoint{1.595224in}{2.528756in}}%
\pgfpathlineto{\pgfqpoint{0.359627in}{1.527760in}}%
\pgfusepath{stroke}%
\end{pgfscope}%
\begin{pgfscope}%
\pgfsetbuttcap%
\pgfsetroundjoin%
\pgfsetlinewidth{0.803000pt}%
\definecolor{currentstroke}{rgb}{0.690196,0.690196,0.690196}%
\pgfsetstrokecolor{currentstroke}%
\pgfsetdash{}{0pt}%
\pgfpathmoveto{\pgfqpoint{3.590230in}{2.278419in}}%
\pgfpathlineto{\pgfqpoint{1.591816in}{2.825130in}}%
\pgfpathlineto{\pgfqpoint{0.344502in}{1.841827in}}%
\pgfusepath{stroke}%
\end{pgfscope}%
\begin{pgfscope}%
\pgfsetbuttcap%
\pgfsetroundjoin%
\pgfsetlinewidth{0.803000pt}%
\definecolor{currentstroke}{rgb}{0.690196,0.690196,0.690196}%
\pgfsetstrokecolor{currentstroke}%
\pgfsetdash{}{0pt}%
\pgfpathmoveto{\pgfqpoint{3.604513in}{2.590316in}}%
\pgfpathlineto{\pgfqpoint{1.588349in}{3.126563in}}%
\pgfpathlineto{\pgfqpoint{0.329096in}{2.161748in}}%
\pgfusepath{stroke}%
\end{pgfscope}%
\begin{pgfscope}%
\pgfsetbuttcap%
\pgfsetroundjoin%
\pgfsetlinewidth{0.803000pt}%
\definecolor{currentstroke}{rgb}{0.690196,0.690196,0.690196}%
\pgfsetstrokecolor{currentstroke}%
\pgfsetdash{}{0pt}%
\pgfpathmoveto{\pgfqpoint{3.619054in}{2.907859in}}%
\pgfpathlineto{\pgfqpoint{1.584823in}{3.433186in}}%
\pgfpathlineto{\pgfqpoint{0.313400in}{2.487686in}}%
\pgfusepath{stroke}%
\end{pgfscope}%
\begin{pgfscope}%
\pgfsetrectcap%
\pgfsetroundjoin%
\pgfsetlinewidth{0.803000pt}%
\definecolor{currentstroke}{rgb}{0.000000,0.000000,0.000000}%
\pgfsetstrokecolor{currentstroke}%
\pgfsetdash{}{0pt}%
\pgfpathmoveto{\pgfqpoint{3.545929in}{1.675725in}}%
\pgfpathlineto{\pgfqpoint{3.595421in}{1.661453in}}%
\pgfusepath{stroke}%
\end{pgfscope}%
\begin{pgfscope}%
\definecolor{textcolor}{rgb}{0.000000,0.000000,0.000000}%
\pgfsetstrokecolor{textcolor}%
\pgfsetfillcolor{textcolor}%
\pgftext[x=3.816545in,y=1.706970in,,top]{\color{textcolor}\rmfamily\fontsize{10.000000}{12.000000}\selectfont \(\displaystyle {0.00}\)}%
\end{pgfscope}%
\begin{pgfscope}%
\pgfsetrectcap%
\pgfsetroundjoin%
\pgfsetlinewidth{0.803000pt}%
\definecolor{currentstroke}{rgb}{0.000000,0.000000,0.000000}%
\pgfsetstrokecolor{currentstroke}%
\pgfsetdash{}{0pt}%
\pgfpathmoveto{\pgfqpoint{3.559564in}{1.976694in}}%
\pgfpathlineto{\pgfqpoint{3.609509in}{1.962657in}}%
\pgfusepath{stroke}%
\end{pgfscope}%
\begin{pgfscope}%
\definecolor{textcolor}{rgb}{0.000000,0.000000,0.000000}%
\pgfsetstrokecolor{textcolor}%
\pgfsetfillcolor{textcolor}%
\pgftext[x=3.832522in,y=2.007421in,,top]{\color{textcolor}\rmfamily\fontsize{10.000000}{12.000000}\selectfont \(\displaystyle {0.01}\)}%
\end{pgfscope}%
\begin{pgfscope}%
\pgfsetrectcap%
\pgfsetroundjoin%
\pgfsetlinewidth{0.803000pt}%
\definecolor{currentstroke}{rgb}{0.000000,0.000000,0.000000}%
\pgfsetstrokecolor{currentstroke}%
\pgfsetdash{}{0pt}%
\pgfpathmoveto{\pgfqpoint{3.573442in}{2.283012in}}%
\pgfpathlineto{\pgfqpoint{3.623848in}{2.269222in}}%
\pgfusepath{stroke}%
\end{pgfscope}%
\begin{pgfscope}%
\definecolor{textcolor}{rgb}{0.000000,0.000000,0.000000}%
\pgfsetstrokecolor{textcolor}%
\pgfsetfillcolor{textcolor}%
\pgftext[x=3.848782in,y=2.313197in,,top]{\color{textcolor}\rmfamily\fontsize{10.000000}{12.000000}\selectfont \(\displaystyle {0.02}\)}%
\end{pgfscope}%
\begin{pgfscope}%
\pgfsetrectcap%
\pgfsetroundjoin%
\pgfsetlinewidth{0.803000pt}%
\definecolor{currentstroke}{rgb}{0.000000,0.000000,0.000000}%
\pgfsetstrokecolor{currentstroke}%
\pgfsetdash{}{0pt}%
\pgfpathmoveto{\pgfqpoint{3.587568in}{2.594823in}}%
\pgfpathlineto{\pgfqpoint{3.638444in}{2.581291in}}%
\pgfusepath{stroke}%
\end{pgfscope}%
\begin{pgfscope}%
\definecolor{textcolor}{rgb}{0.000000,0.000000,0.000000}%
\pgfsetstrokecolor{textcolor}%
\pgfsetfillcolor{textcolor}%
\pgftext[x=3.865332in,y=2.624442in,,top]{\color{textcolor}\rmfamily\fontsize{10.000000}{12.000000}\selectfont \(\displaystyle {0.03}\)}%
\end{pgfscope}%
\begin{pgfscope}%
\pgfsetrectcap%
\pgfsetroundjoin%
\pgfsetlinewidth{0.803000pt}%
\definecolor{currentstroke}{rgb}{0.000000,0.000000,0.000000}%
\pgfsetstrokecolor{currentstroke}%
\pgfsetdash{}{0pt}%
\pgfpathmoveto{\pgfqpoint{3.601950in}{2.912276in}}%
\pgfpathlineto{\pgfqpoint{3.653304in}{2.899014in}}%
\pgfusepath{stroke}%
\end{pgfscope}%
\begin{pgfscope}%
\definecolor{textcolor}{rgb}{0.000000,0.000000,0.000000}%
\pgfsetstrokecolor{textcolor}%
\pgfsetfillcolor{textcolor}%
\pgftext[x=3.882181in,y=2.941304in,,top]{\color{textcolor}\rmfamily\fontsize{10.000000}{12.000000}\selectfont \(\displaystyle {0.04}\)}%
\end{pgfscope}%
\begin{pgfscope}%
\pgfpathrectangle{\pgfqpoint{0.100000in}{0.212622in}}{\pgfqpoint{3.696000in}{3.696000in}}%
\pgfusepath{clip}%
\pgfsetrectcap%
\pgfsetroundjoin%
\pgfsetlinewidth{1.505625pt}%
\definecolor{currentstroke}{rgb}{0.121569,0.466667,0.705882}%
\pgfsetstrokecolor{currentstroke}%
\pgfsetdash{}{0pt}%
\pgfpathmoveto{\pgfqpoint{1.645993in}{2.142291in}}%
\pgfpathlineto{\pgfqpoint{2.994319in}{1.750684in}}%
\pgfusepath{stroke}%
\end{pgfscope}%
\begin{pgfscope}%
\pgfpathrectangle{\pgfqpoint{0.100000in}{0.212622in}}{\pgfqpoint{3.696000in}{3.696000in}}%
\pgfusepath{clip}%
\pgfsetbuttcap%
\pgfsetroundjoin%
\definecolor{currentfill}{rgb}{0.121569,0.466667,0.705882}%
\pgfsetfillcolor{currentfill}%
\pgfsetfillopacity{0.300000}%
\pgfsetlinewidth{1.003750pt}%
\definecolor{currentstroke}{rgb}{0.121569,0.466667,0.705882}%
\pgfsetstrokecolor{currentstroke}%
\pgfsetstrokeopacity{0.300000}%
\pgfsetdash{}{0pt}%
\pgfpathmoveto{\pgfqpoint{1.646136in}{2.111348in}}%
\pgfpathcurveto{\pgfqpoint{1.654373in}{2.111348in}}{\pgfqpoint{1.662273in}{2.114620in}}{\pgfqpoint{1.668097in}{2.120444in}}%
\pgfpathcurveto{\pgfqpoint{1.673921in}{2.126268in}}{\pgfqpoint{1.677193in}{2.134168in}}{\pgfqpoint{1.677193in}{2.142404in}}%
\pgfpathcurveto{\pgfqpoint{1.677193in}{2.150640in}}{\pgfqpoint{1.673921in}{2.158540in}}{\pgfqpoint{1.668097in}{2.164364in}}%
\pgfpathcurveto{\pgfqpoint{1.662273in}{2.170188in}}{\pgfqpoint{1.654373in}{2.173461in}}{\pgfqpoint{1.646136in}{2.173461in}}%
\pgfpathcurveto{\pgfqpoint{1.637900in}{2.173461in}}{\pgfqpoint{1.630000in}{2.170188in}}{\pgfqpoint{1.624176in}{2.164364in}}%
\pgfpathcurveto{\pgfqpoint{1.618352in}{2.158540in}}{\pgfqpoint{1.615080in}{2.150640in}}{\pgfqpoint{1.615080in}{2.142404in}}%
\pgfpathcurveto{\pgfqpoint{1.615080in}{2.134168in}}{\pgfqpoint{1.618352in}{2.126268in}}{\pgfqpoint{1.624176in}{2.120444in}}%
\pgfpathcurveto{\pgfqpoint{1.630000in}{2.114620in}}{\pgfqpoint{1.637900in}{2.111348in}}{\pgfqpoint{1.646136in}{2.111348in}}%
\pgfpathclose%
\pgfusepath{stroke,fill}%
\end{pgfscope}%
\begin{pgfscope}%
\pgfpathrectangle{\pgfqpoint{0.100000in}{0.212622in}}{\pgfqpoint{3.696000in}{3.696000in}}%
\pgfusepath{clip}%
\pgfsetbuttcap%
\pgfsetroundjoin%
\definecolor{currentfill}{rgb}{0.121569,0.466667,0.705882}%
\pgfsetfillcolor{currentfill}%
\pgfsetfillopacity{0.300023}%
\pgfsetlinewidth{1.003750pt}%
\definecolor{currentstroke}{rgb}{0.121569,0.466667,0.705882}%
\pgfsetstrokecolor{currentstroke}%
\pgfsetstrokeopacity{0.300023}%
\pgfsetdash{}{0pt}%
\pgfpathmoveto{\pgfqpoint{1.646083in}{2.111303in}}%
\pgfpathcurveto{\pgfqpoint{1.654319in}{2.111303in}}{\pgfqpoint{1.662219in}{2.114575in}}{\pgfqpoint{1.668043in}{2.120399in}}%
\pgfpathcurveto{\pgfqpoint{1.673867in}{2.126223in}}{\pgfqpoint{1.677140in}{2.134123in}}{\pgfqpoint{1.677140in}{2.142359in}}%
\pgfpathcurveto{\pgfqpoint{1.677140in}{2.150596in}}{\pgfqpoint{1.673867in}{2.158496in}}{\pgfqpoint{1.668043in}{2.164320in}}%
\pgfpathcurveto{\pgfqpoint{1.662219in}{2.170144in}}{\pgfqpoint{1.654319in}{2.173416in}}{\pgfqpoint{1.646083in}{2.173416in}}%
\pgfpathcurveto{\pgfqpoint{1.637847in}{2.173416in}}{\pgfqpoint{1.629947in}{2.170144in}}{\pgfqpoint{1.624123in}{2.164320in}}%
\pgfpathcurveto{\pgfqpoint{1.618299in}{2.158496in}}{\pgfqpoint{1.615027in}{2.150596in}}{\pgfqpoint{1.615027in}{2.142359in}}%
\pgfpathcurveto{\pgfqpoint{1.615027in}{2.134123in}}{\pgfqpoint{1.618299in}{2.126223in}}{\pgfqpoint{1.624123in}{2.120399in}}%
\pgfpathcurveto{\pgfqpoint{1.629947in}{2.114575in}}{\pgfqpoint{1.637847in}{2.111303in}}{\pgfqpoint{1.646083in}{2.111303in}}%
\pgfpathclose%
\pgfusepath{stroke,fill}%
\end{pgfscope}%
\begin{pgfscope}%
\pgfpathrectangle{\pgfqpoint{0.100000in}{0.212622in}}{\pgfqpoint{3.696000in}{3.696000in}}%
\pgfusepath{clip}%
\pgfsetbuttcap%
\pgfsetroundjoin%
\definecolor{currentfill}{rgb}{0.121569,0.466667,0.705882}%
\pgfsetfillcolor{currentfill}%
\pgfsetfillopacity{0.300026}%
\pgfsetlinewidth{1.003750pt}%
\definecolor{currentstroke}{rgb}{0.121569,0.466667,0.705882}%
\pgfsetstrokecolor{currentstroke}%
\pgfsetstrokeopacity{0.300026}%
\pgfsetdash{}{0pt}%
\pgfpathmoveto{\pgfqpoint{1.646079in}{2.111303in}}%
\pgfpathcurveto{\pgfqpoint{1.654315in}{2.111303in}}{\pgfqpoint{1.662215in}{2.114575in}}{\pgfqpoint{1.668039in}{2.120399in}}%
\pgfpathcurveto{\pgfqpoint{1.673863in}{2.126223in}}{\pgfqpoint{1.677135in}{2.134123in}}{\pgfqpoint{1.677135in}{2.142359in}}%
\pgfpathcurveto{\pgfqpoint{1.677135in}{2.150595in}}{\pgfqpoint{1.673863in}{2.158495in}}{\pgfqpoint{1.668039in}{2.164319in}}%
\pgfpathcurveto{\pgfqpoint{1.662215in}{2.170143in}}{\pgfqpoint{1.654315in}{2.173416in}}{\pgfqpoint{1.646079in}{2.173416in}}%
\pgfpathcurveto{\pgfqpoint{1.637843in}{2.173416in}}{\pgfqpoint{1.629943in}{2.170143in}}{\pgfqpoint{1.624119in}{2.164319in}}%
\pgfpathcurveto{\pgfqpoint{1.618295in}{2.158495in}}{\pgfqpoint{1.615022in}{2.150595in}}{\pgfqpoint{1.615022in}{2.142359in}}%
\pgfpathcurveto{\pgfqpoint{1.615022in}{2.134123in}}{\pgfqpoint{1.618295in}{2.126223in}}{\pgfqpoint{1.624119in}{2.120399in}}%
\pgfpathcurveto{\pgfqpoint{1.629943in}{2.114575in}}{\pgfqpoint{1.637843in}{2.111303in}}{\pgfqpoint{1.646079in}{2.111303in}}%
\pgfpathclose%
\pgfusepath{stroke,fill}%
\end{pgfscope}%
\begin{pgfscope}%
\pgfpathrectangle{\pgfqpoint{0.100000in}{0.212622in}}{\pgfqpoint{3.696000in}{3.696000in}}%
\pgfusepath{clip}%
\pgfsetbuttcap%
\pgfsetroundjoin%
\definecolor{currentfill}{rgb}{0.121569,0.466667,0.705882}%
\pgfsetfillcolor{currentfill}%
\pgfsetfillopacity{0.300030}%
\pgfsetlinewidth{1.003750pt}%
\definecolor{currentstroke}{rgb}{0.121569,0.466667,0.705882}%
\pgfsetstrokecolor{currentstroke}%
\pgfsetstrokeopacity{0.300030}%
\pgfsetdash{}{0pt}%
\pgfpathmoveto{\pgfqpoint{1.646070in}{2.111293in}}%
\pgfpathcurveto{\pgfqpoint{1.654306in}{2.111293in}}{\pgfqpoint{1.662206in}{2.114566in}}{\pgfqpoint{1.668030in}{2.120390in}}%
\pgfpathcurveto{\pgfqpoint{1.673854in}{2.126213in}}{\pgfqpoint{1.677126in}{2.134114in}}{\pgfqpoint{1.677126in}{2.142350in}}%
\pgfpathcurveto{\pgfqpoint{1.677126in}{2.150586in}}{\pgfqpoint{1.673854in}{2.158486in}}{\pgfqpoint{1.668030in}{2.164310in}}%
\pgfpathcurveto{\pgfqpoint{1.662206in}{2.170134in}}{\pgfqpoint{1.654306in}{2.173406in}}{\pgfqpoint{1.646070in}{2.173406in}}%
\pgfpathcurveto{\pgfqpoint{1.637833in}{2.173406in}}{\pgfqpoint{1.629933in}{2.170134in}}{\pgfqpoint{1.624109in}{2.164310in}}%
\pgfpathcurveto{\pgfqpoint{1.618286in}{2.158486in}}{\pgfqpoint{1.615013in}{2.150586in}}{\pgfqpoint{1.615013in}{2.142350in}}%
\pgfpathcurveto{\pgfqpoint{1.615013in}{2.134114in}}{\pgfqpoint{1.618286in}{2.126213in}}{\pgfqpoint{1.624109in}{2.120390in}}%
\pgfpathcurveto{\pgfqpoint{1.629933in}{2.114566in}}{\pgfqpoint{1.637833in}{2.111293in}}{\pgfqpoint{1.646070in}{2.111293in}}%
\pgfpathclose%
\pgfusepath{stroke,fill}%
\end{pgfscope}%
\begin{pgfscope}%
\pgfpathrectangle{\pgfqpoint{0.100000in}{0.212622in}}{\pgfqpoint{3.696000in}{3.696000in}}%
\pgfusepath{clip}%
\pgfsetbuttcap%
\pgfsetroundjoin%
\definecolor{currentfill}{rgb}{0.121569,0.466667,0.705882}%
\pgfsetfillcolor{currentfill}%
\pgfsetfillopacity{0.300031}%
\pgfsetlinewidth{1.003750pt}%
\definecolor{currentstroke}{rgb}{0.121569,0.466667,0.705882}%
\pgfsetstrokecolor{currentstroke}%
\pgfsetstrokeopacity{0.300031}%
\pgfsetdash{}{0pt}%
\pgfpathmoveto{\pgfqpoint{1.646068in}{2.111293in}}%
\pgfpathcurveto{\pgfqpoint{1.654305in}{2.111293in}}{\pgfqpoint{1.662205in}{2.114565in}}{\pgfqpoint{1.668029in}{2.120389in}}%
\pgfpathcurveto{\pgfqpoint{1.673853in}{2.126213in}}{\pgfqpoint{1.677125in}{2.134113in}}{\pgfqpoint{1.677125in}{2.142349in}}%
\pgfpathcurveto{\pgfqpoint{1.677125in}{2.150586in}}{\pgfqpoint{1.673853in}{2.158486in}}{\pgfqpoint{1.668029in}{2.164310in}}%
\pgfpathcurveto{\pgfqpoint{1.662205in}{2.170133in}}{\pgfqpoint{1.654305in}{2.173406in}}{\pgfqpoint{1.646068in}{2.173406in}}%
\pgfpathcurveto{\pgfqpoint{1.637832in}{2.173406in}}{\pgfqpoint{1.629932in}{2.170133in}}{\pgfqpoint{1.624108in}{2.164310in}}%
\pgfpathcurveto{\pgfqpoint{1.618284in}{2.158486in}}{\pgfqpoint{1.615012in}{2.150586in}}{\pgfqpoint{1.615012in}{2.142349in}}%
\pgfpathcurveto{\pgfqpoint{1.615012in}{2.134113in}}{\pgfqpoint{1.618284in}{2.126213in}}{\pgfqpoint{1.624108in}{2.120389in}}%
\pgfpathcurveto{\pgfqpoint{1.629932in}{2.114565in}}{\pgfqpoint{1.637832in}{2.111293in}}{\pgfqpoint{1.646068in}{2.111293in}}%
\pgfpathclose%
\pgfusepath{stroke,fill}%
\end{pgfscope}%
\begin{pgfscope}%
\pgfpathrectangle{\pgfqpoint{0.100000in}{0.212622in}}{\pgfqpoint{3.696000in}{3.696000in}}%
\pgfusepath{clip}%
\pgfsetbuttcap%
\pgfsetroundjoin%
\definecolor{currentfill}{rgb}{0.121569,0.466667,0.705882}%
\pgfsetfillcolor{currentfill}%
\pgfsetfillopacity{0.300032}%
\pgfsetlinewidth{1.003750pt}%
\definecolor{currentstroke}{rgb}{0.121569,0.466667,0.705882}%
\pgfsetstrokecolor{currentstroke}%
\pgfsetstrokeopacity{0.300032}%
\pgfsetdash{}{0pt}%
\pgfpathmoveto{\pgfqpoint{1.646067in}{2.111291in}}%
\pgfpathcurveto{\pgfqpoint{1.654303in}{2.111291in}}{\pgfqpoint{1.662203in}{2.114563in}}{\pgfqpoint{1.668027in}{2.120387in}}%
\pgfpathcurveto{\pgfqpoint{1.673851in}{2.126211in}}{\pgfqpoint{1.677123in}{2.134111in}}{\pgfqpoint{1.677123in}{2.142348in}}%
\pgfpathcurveto{\pgfqpoint{1.677123in}{2.150584in}}{\pgfqpoint{1.673851in}{2.158484in}}{\pgfqpoint{1.668027in}{2.164308in}}%
\pgfpathcurveto{\pgfqpoint{1.662203in}{2.170132in}}{\pgfqpoint{1.654303in}{2.173404in}}{\pgfqpoint{1.646067in}{2.173404in}}%
\pgfpathcurveto{\pgfqpoint{1.637831in}{2.173404in}}{\pgfqpoint{1.629931in}{2.170132in}}{\pgfqpoint{1.624107in}{2.164308in}}%
\pgfpathcurveto{\pgfqpoint{1.618283in}{2.158484in}}{\pgfqpoint{1.615010in}{2.150584in}}{\pgfqpoint{1.615010in}{2.142348in}}%
\pgfpathcurveto{\pgfqpoint{1.615010in}{2.134111in}}{\pgfqpoint{1.618283in}{2.126211in}}{\pgfqpoint{1.624107in}{2.120387in}}%
\pgfpathcurveto{\pgfqpoint{1.629931in}{2.114563in}}{\pgfqpoint{1.637831in}{2.111291in}}{\pgfqpoint{1.646067in}{2.111291in}}%
\pgfpathclose%
\pgfusepath{stroke,fill}%
\end{pgfscope}%
\begin{pgfscope}%
\pgfpathrectangle{\pgfqpoint{0.100000in}{0.212622in}}{\pgfqpoint{3.696000in}{3.696000in}}%
\pgfusepath{clip}%
\pgfsetbuttcap%
\pgfsetroundjoin%
\definecolor{currentfill}{rgb}{0.121569,0.466667,0.705882}%
\pgfsetfillcolor{currentfill}%
\pgfsetfillopacity{0.300032}%
\pgfsetlinewidth{1.003750pt}%
\definecolor{currentstroke}{rgb}{0.121569,0.466667,0.705882}%
\pgfsetstrokecolor{currentstroke}%
\pgfsetstrokeopacity{0.300032}%
\pgfsetdash{}{0pt}%
\pgfpathmoveto{\pgfqpoint{1.646066in}{2.111291in}}%
\pgfpathcurveto{\pgfqpoint{1.654303in}{2.111291in}}{\pgfqpoint{1.662203in}{2.114563in}}{\pgfqpoint{1.668027in}{2.120387in}}%
\pgfpathcurveto{\pgfqpoint{1.673851in}{2.126211in}}{\pgfqpoint{1.677123in}{2.134111in}}{\pgfqpoint{1.677123in}{2.142347in}}%
\pgfpathcurveto{\pgfqpoint{1.677123in}{2.150584in}}{\pgfqpoint{1.673851in}{2.158484in}}{\pgfqpoint{1.668027in}{2.164308in}}%
\pgfpathcurveto{\pgfqpoint{1.662203in}{2.170132in}}{\pgfqpoint{1.654303in}{2.173404in}}{\pgfqpoint{1.646066in}{2.173404in}}%
\pgfpathcurveto{\pgfqpoint{1.637830in}{2.173404in}}{\pgfqpoint{1.629930in}{2.170132in}}{\pgfqpoint{1.624106in}{2.164308in}}%
\pgfpathcurveto{\pgfqpoint{1.618282in}{2.158484in}}{\pgfqpoint{1.615010in}{2.150584in}}{\pgfqpoint{1.615010in}{2.142347in}}%
\pgfpathcurveto{\pgfqpoint{1.615010in}{2.134111in}}{\pgfqpoint{1.618282in}{2.126211in}}{\pgfqpoint{1.624106in}{2.120387in}}%
\pgfpathcurveto{\pgfqpoint{1.629930in}{2.114563in}}{\pgfqpoint{1.637830in}{2.111291in}}{\pgfqpoint{1.646066in}{2.111291in}}%
\pgfpathclose%
\pgfusepath{stroke,fill}%
\end{pgfscope}%
\begin{pgfscope}%
\pgfpathrectangle{\pgfqpoint{0.100000in}{0.212622in}}{\pgfqpoint{3.696000in}{3.696000in}}%
\pgfusepath{clip}%
\pgfsetbuttcap%
\pgfsetroundjoin%
\definecolor{currentfill}{rgb}{0.121569,0.466667,0.705882}%
\pgfsetfillcolor{currentfill}%
\pgfsetfillopacity{0.300032}%
\pgfsetlinewidth{1.003750pt}%
\definecolor{currentstroke}{rgb}{0.121569,0.466667,0.705882}%
\pgfsetstrokecolor{currentstroke}%
\pgfsetstrokeopacity{0.300032}%
\pgfsetdash{}{0pt}%
\pgfpathmoveto{\pgfqpoint{1.646066in}{2.111291in}}%
\pgfpathcurveto{\pgfqpoint{1.654302in}{2.111291in}}{\pgfqpoint{1.662202in}{2.114563in}}{\pgfqpoint{1.668026in}{2.120387in}}%
\pgfpathcurveto{\pgfqpoint{1.673850in}{2.126211in}}{\pgfqpoint{1.677123in}{2.134111in}}{\pgfqpoint{1.677123in}{2.142347in}}%
\pgfpathcurveto{\pgfqpoint{1.677123in}{2.150583in}}{\pgfqpoint{1.673850in}{2.158483in}}{\pgfqpoint{1.668026in}{2.164307in}}%
\pgfpathcurveto{\pgfqpoint{1.662202in}{2.170131in}}{\pgfqpoint{1.654302in}{2.173404in}}{\pgfqpoint{1.646066in}{2.173404in}}%
\pgfpathcurveto{\pgfqpoint{1.637830in}{2.173404in}}{\pgfqpoint{1.629930in}{2.170131in}}{\pgfqpoint{1.624106in}{2.164307in}}%
\pgfpathcurveto{\pgfqpoint{1.618282in}{2.158483in}}{\pgfqpoint{1.615010in}{2.150583in}}{\pgfqpoint{1.615010in}{2.142347in}}%
\pgfpathcurveto{\pgfqpoint{1.615010in}{2.134111in}}{\pgfqpoint{1.618282in}{2.126211in}}{\pgfqpoint{1.624106in}{2.120387in}}%
\pgfpathcurveto{\pgfqpoint{1.629930in}{2.114563in}}{\pgfqpoint{1.637830in}{2.111291in}}{\pgfqpoint{1.646066in}{2.111291in}}%
\pgfpathclose%
\pgfusepath{stroke,fill}%
\end{pgfscope}%
\begin{pgfscope}%
\pgfpathrectangle{\pgfqpoint{0.100000in}{0.212622in}}{\pgfqpoint{3.696000in}{3.696000in}}%
\pgfusepath{clip}%
\pgfsetbuttcap%
\pgfsetroundjoin%
\definecolor{currentfill}{rgb}{0.121569,0.466667,0.705882}%
\pgfsetfillcolor{currentfill}%
\pgfsetfillopacity{0.300032}%
\pgfsetlinewidth{1.003750pt}%
\definecolor{currentstroke}{rgb}{0.121569,0.466667,0.705882}%
\pgfsetstrokecolor{currentstroke}%
\pgfsetstrokeopacity{0.300032}%
\pgfsetdash{}{0pt}%
\pgfpathmoveto{\pgfqpoint{1.646066in}{2.111290in}}%
\pgfpathcurveto{\pgfqpoint{1.654302in}{2.111290in}}{\pgfqpoint{1.662202in}{2.114563in}}{\pgfqpoint{1.668026in}{2.120387in}}%
\pgfpathcurveto{\pgfqpoint{1.673850in}{2.126211in}}{\pgfqpoint{1.677123in}{2.134111in}}{\pgfqpoint{1.677123in}{2.142347in}}%
\pgfpathcurveto{\pgfqpoint{1.677123in}{2.150583in}}{\pgfqpoint{1.673850in}{2.158483in}}{\pgfqpoint{1.668026in}{2.164307in}}%
\pgfpathcurveto{\pgfqpoint{1.662202in}{2.170131in}}{\pgfqpoint{1.654302in}{2.173403in}}{\pgfqpoint{1.646066in}{2.173403in}}%
\pgfpathcurveto{\pgfqpoint{1.637830in}{2.173403in}}{\pgfqpoint{1.629930in}{2.170131in}}{\pgfqpoint{1.624106in}{2.164307in}}%
\pgfpathcurveto{\pgfqpoint{1.618282in}{2.158483in}}{\pgfqpoint{1.615010in}{2.150583in}}{\pgfqpoint{1.615010in}{2.142347in}}%
\pgfpathcurveto{\pgfqpoint{1.615010in}{2.134111in}}{\pgfqpoint{1.618282in}{2.126211in}}{\pgfqpoint{1.624106in}{2.120387in}}%
\pgfpathcurveto{\pgfqpoint{1.629930in}{2.114563in}}{\pgfqpoint{1.637830in}{2.111290in}}{\pgfqpoint{1.646066in}{2.111290in}}%
\pgfpathclose%
\pgfusepath{stroke,fill}%
\end{pgfscope}%
\begin{pgfscope}%
\pgfpathrectangle{\pgfqpoint{0.100000in}{0.212622in}}{\pgfqpoint{3.696000in}{3.696000in}}%
\pgfusepath{clip}%
\pgfsetbuttcap%
\pgfsetroundjoin%
\definecolor{currentfill}{rgb}{0.121569,0.466667,0.705882}%
\pgfsetfillcolor{currentfill}%
\pgfsetfillopacity{0.300032}%
\pgfsetlinewidth{1.003750pt}%
\definecolor{currentstroke}{rgb}{0.121569,0.466667,0.705882}%
\pgfsetstrokecolor{currentstroke}%
\pgfsetstrokeopacity{0.300032}%
\pgfsetdash{}{0pt}%
\pgfpathmoveto{\pgfqpoint{1.646066in}{2.111290in}}%
\pgfpathcurveto{\pgfqpoint{1.654302in}{2.111290in}}{\pgfqpoint{1.662202in}{2.114563in}}{\pgfqpoint{1.668026in}{2.120387in}}%
\pgfpathcurveto{\pgfqpoint{1.673850in}{2.126211in}}{\pgfqpoint{1.677122in}{2.134111in}}{\pgfqpoint{1.677122in}{2.142347in}}%
\pgfpathcurveto{\pgfqpoint{1.677122in}{2.150583in}}{\pgfqpoint{1.673850in}{2.158483in}}{\pgfqpoint{1.668026in}{2.164307in}}%
\pgfpathcurveto{\pgfqpoint{1.662202in}{2.170131in}}{\pgfqpoint{1.654302in}{2.173403in}}{\pgfqpoint{1.646066in}{2.173403in}}%
\pgfpathcurveto{\pgfqpoint{1.637830in}{2.173403in}}{\pgfqpoint{1.629930in}{2.170131in}}{\pgfqpoint{1.624106in}{2.164307in}}%
\pgfpathcurveto{\pgfqpoint{1.618282in}{2.158483in}}{\pgfqpoint{1.615009in}{2.150583in}}{\pgfqpoint{1.615009in}{2.142347in}}%
\pgfpathcurveto{\pgfqpoint{1.615009in}{2.134111in}}{\pgfqpoint{1.618282in}{2.126211in}}{\pgfqpoint{1.624106in}{2.120387in}}%
\pgfpathcurveto{\pgfqpoint{1.629930in}{2.114563in}}{\pgfqpoint{1.637830in}{2.111290in}}{\pgfqpoint{1.646066in}{2.111290in}}%
\pgfpathclose%
\pgfusepath{stroke,fill}%
\end{pgfscope}%
\begin{pgfscope}%
\pgfpathrectangle{\pgfqpoint{0.100000in}{0.212622in}}{\pgfqpoint{3.696000in}{3.696000in}}%
\pgfusepath{clip}%
\pgfsetbuttcap%
\pgfsetroundjoin%
\definecolor{currentfill}{rgb}{0.121569,0.466667,0.705882}%
\pgfsetfillcolor{currentfill}%
\pgfsetfillopacity{0.300032}%
\pgfsetlinewidth{1.003750pt}%
\definecolor{currentstroke}{rgb}{0.121569,0.466667,0.705882}%
\pgfsetstrokecolor{currentstroke}%
\pgfsetstrokeopacity{0.300032}%
\pgfsetdash{}{0pt}%
\pgfpathmoveto{\pgfqpoint{1.646066in}{2.111290in}}%
\pgfpathcurveto{\pgfqpoint{1.654302in}{2.111290in}}{\pgfqpoint{1.662202in}{2.114563in}}{\pgfqpoint{1.668026in}{2.120387in}}%
\pgfpathcurveto{\pgfqpoint{1.673850in}{2.126210in}}{\pgfqpoint{1.677122in}{2.134111in}}{\pgfqpoint{1.677122in}{2.142347in}}%
\pgfpathcurveto{\pgfqpoint{1.677122in}{2.150583in}}{\pgfqpoint{1.673850in}{2.158483in}}{\pgfqpoint{1.668026in}{2.164307in}}%
\pgfpathcurveto{\pgfqpoint{1.662202in}{2.170131in}}{\pgfqpoint{1.654302in}{2.173403in}}{\pgfqpoint{1.646066in}{2.173403in}}%
\pgfpathcurveto{\pgfqpoint{1.637830in}{2.173403in}}{\pgfqpoint{1.629929in}{2.170131in}}{\pgfqpoint{1.624106in}{2.164307in}}%
\pgfpathcurveto{\pgfqpoint{1.618282in}{2.158483in}}{\pgfqpoint{1.615009in}{2.150583in}}{\pgfqpoint{1.615009in}{2.142347in}}%
\pgfpathcurveto{\pgfqpoint{1.615009in}{2.134111in}}{\pgfqpoint{1.618282in}{2.126210in}}{\pgfqpoint{1.624106in}{2.120387in}}%
\pgfpathcurveto{\pgfqpoint{1.629929in}{2.114563in}}{\pgfqpoint{1.637830in}{2.111290in}}{\pgfqpoint{1.646066in}{2.111290in}}%
\pgfpathclose%
\pgfusepath{stroke,fill}%
\end{pgfscope}%
\begin{pgfscope}%
\pgfpathrectangle{\pgfqpoint{0.100000in}{0.212622in}}{\pgfqpoint{3.696000in}{3.696000in}}%
\pgfusepath{clip}%
\pgfsetbuttcap%
\pgfsetroundjoin%
\definecolor{currentfill}{rgb}{0.121569,0.466667,0.705882}%
\pgfsetfillcolor{currentfill}%
\pgfsetfillopacity{0.300032}%
\pgfsetlinewidth{1.003750pt}%
\definecolor{currentstroke}{rgb}{0.121569,0.466667,0.705882}%
\pgfsetstrokecolor{currentstroke}%
\pgfsetstrokeopacity{0.300032}%
\pgfsetdash{}{0pt}%
\pgfpathmoveto{\pgfqpoint{1.646066in}{2.111290in}}%
\pgfpathcurveto{\pgfqpoint{1.654302in}{2.111290in}}{\pgfqpoint{1.662202in}{2.114563in}}{\pgfqpoint{1.668026in}{2.120387in}}%
\pgfpathcurveto{\pgfqpoint{1.673850in}{2.126210in}}{\pgfqpoint{1.677122in}{2.134111in}}{\pgfqpoint{1.677122in}{2.142347in}}%
\pgfpathcurveto{\pgfqpoint{1.677122in}{2.150583in}}{\pgfqpoint{1.673850in}{2.158483in}}{\pgfqpoint{1.668026in}{2.164307in}}%
\pgfpathcurveto{\pgfqpoint{1.662202in}{2.170131in}}{\pgfqpoint{1.654302in}{2.173403in}}{\pgfqpoint{1.646066in}{2.173403in}}%
\pgfpathcurveto{\pgfqpoint{1.637830in}{2.173403in}}{\pgfqpoint{1.629929in}{2.170131in}}{\pgfqpoint{1.624106in}{2.164307in}}%
\pgfpathcurveto{\pgfqpoint{1.618282in}{2.158483in}}{\pgfqpoint{1.615009in}{2.150583in}}{\pgfqpoint{1.615009in}{2.142347in}}%
\pgfpathcurveto{\pgfqpoint{1.615009in}{2.134111in}}{\pgfqpoint{1.618282in}{2.126210in}}{\pgfqpoint{1.624106in}{2.120387in}}%
\pgfpathcurveto{\pgfqpoint{1.629929in}{2.114563in}}{\pgfqpoint{1.637830in}{2.111290in}}{\pgfqpoint{1.646066in}{2.111290in}}%
\pgfpathclose%
\pgfusepath{stroke,fill}%
\end{pgfscope}%
\begin{pgfscope}%
\pgfpathrectangle{\pgfqpoint{0.100000in}{0.212622in}}{\pgfqpoint{3.696000in}{3.696000in}}%
\pgfusepath{clip}%
\pgfsetbuttcap%
\pgfsetroundjoin%
\definecolor{currentfill}{rgb}{0.121569,0.466667,0.705882}%
\pgfsetfillcolor{currentfill}%
\pgfsetfillopacity{0.300032}%
\pgfsetlinewidth{1.003750pt}%
\definecolor{currentstroke}{rgb}{0.121569,0.466667,0.705882}%
\pgfsetstrokecolor{currentstroke}%
\pgfsetstrokeopacity{0.300032}%
\pgfsetdash{}{0pt}%
\pgfpathmoveto{\pgfqpoint{1.646066in}{2.111290in}}%
\pgfpathcurveto{\pgfqpoint{1.654302in}{2.111290in}}{\pgfqpoint{1.662202in}{2.114563in}}{\pgfqpoint{1.668026in}{2.120387in}}%
\pgfpathcurveto{\pgfqpoint{1.673850in}{2.126210in}}{\pgfqpoint{1.677122in}{2.134110in}}{\pgfqpoint{1.677122in}{2.142347in}}%
\pgfpathcurveto{\pgfqpoint{1.677122in}{2.150583in}}{\pgfqpoint{1.673850in}{2.158483in}}{\pgfqpoint{1.668026in}{2.164307in}}%
\pgfpathcurveto{\pgfqpoint{1.662202in}{2.170131in}}{\pgfqpoint{1.654302in}{2.173403in}}{\pgfqpoint{1.646066in}{2.173403in}}%
\pgfpathcurveto{\pgfqpoint{1.637829in}{2.173403in}}{\pgfqpoint{1.629929in}{2.170131in}}{\pgfqpoint{1.624106in}{2.164307in}}%
\pgfpathcurveto{\pgfqpoint{1.618282in}{2.158483in}}{\pgfqpoint{1.615009in}{2.150583in}}{\pgfqpoint{1.615009in}{2.142347in}}%
\pgfpathcurveto{\pgfqpoint{1.615009in}{2.134110in}}{\pgfqpoint{1.618282in}{2.126210in}}{\pgfqpoint{1.624106in}{2.120387in}}%
\pgfpathcurveto{\pgfqpoint{1.629929in}{2.114563in}}{\pgfqpoint{1.637829in}{2.111290in}}{\pgfqpoint{1.646066in}{2.111290in}}%
\pgfpathclose%
\pgfusepath{stroke,fill}%
\end{pgfscope}%
\begin{pgfscope}%
\pgfpathrectangle{\pgfqpoint{0.100000in}{0.212622in}}{\pgfqpoint{3.696000in}{3.696000in}}%
\pgfusepath{clip}%
\pgfsetbuttcap%
\pgfsetroundjoin%
\definecolor{currentfill}{rgb}{0.121569,0.466667,0.705882}%
\pgfsetfillcolor{currentfill}%
\pgfsetfillopacity{0.300032}%
\pgfsetlinewidth{1.003750pt}%
\definecolor{currentstroke}{rgb}{0.121569,0.466667,0.705882}%
\pgfsetstrokecolor{currentstroke}%
\pgfsetstrokeopacity{0.300032}%
\pgfsetdash{}{0pt}%
\pgfpathmoveto{\pgfqpoint{1.646066in}{2.111290in}}%
\pgfpathcurveto{\pgfqpoint{1.654302in}{2.111290in}}{\pgfqpoint{1.662202in}{2.114563in}}{\pgfqpoint{1.668026in}{2.120386in}}%
\pgfpathcurveto{\pgfqpoint{1.673850in}{2.126210in}}{\pgfqpoint{1.677122in}{2.134110in}}{\pgfqpoint{1.677122in}{2.142347in}}%
\pgfpathcurveto{\pgfqpoint{1.677122in}{2.150583in}}{\pgfqpoint{1.673850in}{2.158483in}}{\pgfqpoint{1.668026in}{2.164307in}}%
\pgfpathcurveto{\pgfqpoint{1.662202in}{2.170131in}}{\pgfqpoint{1.654302in}{2.173403in}}{\pgfqpoint{1.646066in}{2.173403in}}%
\pgfpathcurveto{\pgfqpoint{1.637829in}{2.173403in}}{\pgfqpoint{1.629929in}{2.170131in}}{\pgfqpoint{1.624105in}{2.164307in}}%
\pgfpathcurveto{\pgfqpoint{1.618282in}{2.158483in}}{\pgfqpoint{1.615009in}{2.150583in}}{\pgfqpoint{1.615009in}{2.142347in}}%
\pgfpathcurveto{\pgfqpoint{1.615009in}{2.134110in}}{\pgfqpoint{1.618282in}{2.126210in}}{\pgfqpoint{1.624105in}{2.120386in}}%
\pgfpathcurveto{\pgfqpoint{1.629929in}{2.114563in}}{\pgfqpoint{1.637829in}{2.111290in}}{\pgfqpoint{1.646066in}{2.111290in}}%
\pgfpathclose%
\pgfusepath{stroke,fill}%
\end{pgfscope}%
\begin{pgfscope}%
\pgfpathrectangle{\pgfqpoint{0.100000in}{0.212622in}}{\pgfqpoint{3.696000in}{3.696000in}}%
\pgfusepath{clip}%
\pgfsetbuttcap%
\pgfsetroundjoin%
\definecolor{currentfill}{rgb}{0.121569,0.466667,0.705882}%
\pgfsetfillcolor{currentfill}%
\pgfsetfillopacity{0.300032}%
\pgfsetlinewidth{1.003750pt}%
\definecolor{currentstroke}{rgb}{0.121569,0.466667,0.705882}%
\pgfsetstrokecolor{currentstroke}%
\pgfsetstrokeopacity{0.300032}%
\pgfsetdash{}{0pt}%
\pgfpathmoveto{\pgfqpoint{1.646066in}{2.111290in}}%
\pgfpathcurveto{\pgfqpoint{1.654302in}{2.111290in}}{\pgfqpoint{1.662202in}{2.114563in}}{\pgfqpoint{1.668026in}{2.120386in}}%
\pgfpathcurveto{\pgfqpoint{1.673850in}{2.126210in}}{\pgfqpoint{1.677122in}{2.134110in}}{\pgfqpoint{1.677122in}{2.142347in}}%
\pgfpathcurveto{\pgfqpoint{1.677122in}{2.150583in}}{\pgfqpoint{1.673850in}{2.158483in}}{\pgfqpoint{1.668026in}{2.164307in}}%
\pgfpathcurveto{\pgfqpoint{1.662202in}{2.170131in}}{\pgfqpoint{1.654302in}{2.173403in}}{\pgfqpoint{1.646066in}{2.173403in}}%
\pgfpathcurveto{\pgfqpoint{1.637829in}{2.173403in}}{\pgfqpoint{1.629929in}{2.170131in}}{\pgfqpoint{1.624105in}{2.164307in}}%
\pgfpathcurveto{\pgfqpoint{1.618282in}{2.158483in}}{\pgfqpoint{1.615009in}{2.150583in}}{\pgfqpoint{1.615009in}{2.142347in}}%
\pgfpathcurveto{\pgfqpoint{1.615009in}{2.134110in}}{\pgfqpoint{1.618282in}{2.126210in}}{\pgfqpoint{1.624105in}{2.120386in}}%
\pgfpathcurveto{\pgfqpoint{1.629929in}{2.114563in}}{\pgfqpoint{1.637829in}{2.111290in}}{\pgfqpoint{1.646066in}{2.111290in}}%
\pgfpathclose%
\pgfusepath{stroke,fill}%
\end{pgfscope}%
\begin{pgfscope}%
\pgfpathrectangle{\pgfqpoint{0.100000in}{0.212622in}}{\pgfqpoint{3.696000in}{3.696000in}}%
\pgfusepath{clip}%
\pgfsetbuttcap%
\pgfsetroundjoin%
\definecolor{currentfill}{rgb}{0.121569,0.466667,0.705882}%
\pgfsetfillcolor{currentfill}%
\pgfsetfillopacity{0.300032}%
\pgfsetlinewidth{1.003750pt}%
\definecolor{currentstroke}{rgb}{0.121569,0.466667,0.705882}%
\pgfsetstrokecolor{currentstroke}%
\pgfsetstrokeopacity{0.300032}%
\pgfsetdash{}{0pt}%
\pgfpathmoveto{\pgfqpoint{1.646066in}{2.111290in}}%
\pgfpathcurveto{\pgfqpoint{1.654302in}{2.111290in}}{\pgfqpoint{1.662202in}{2.114563in}}{\pgfqpoint{1.668026in}{2.120386in}}%
\pgfpathcurveto{\pgfqpoint{1.673850in}{2.126210in}}{\pgfqpoint{1.677122in}{2.134110in}}{\pgfqpoint{1.677122in}{2.142347in}}%
\pgfpathcurveto{\pgfqpoint{1.677122in}{2.150583in}}{\pgfqpoint{1.673850in}{2.158483in}}{\pgfqpoint{1.668026in}{2.164307in}}%
\pgfpathcurveto{\pgfqpoint{1.662202in}{2.170131in}}{\pgfqpoint{1.654302in}{2.173403in}}{\pgfqpoint{1.646066in}{2.173403in}}%
\pgfpathcurveto{\pgfqpoint{1.637829in}{2.173403in}}{\pgfqpoint{1.629929in}{2.170131in}}{\pgfqpoint{1.624105in}{2.164307in}}%
\pgfpathcurveto{\pgfqpoint{1.618282in}{2.158483in}}{\pgfqpoint{1.615009in}{2.150583in}}{\pgfqpoint{1.615009in}{2.142347in}}%
\pgfpathcurveto{\pgfqpoint{1.615009in}{2.134110in}}{\pgfqpoint{1.618282in}{2.126210in}}{\pgfqpoint{1.624105in}{2.120386in}}%
\pgfpathcurveto{\pgfqpoint{1.629929in}{2.114563in}}{\pgfqpoint{1.637829in}{2.111290in}}{\pgfqpoint{1.646066in}{2.111290in}}%
\pgfpathclose%
\pgfusepath{stroke,fill}%
\end{pgfscope}%
\begin{pgfscope}%
\pgfpathrectangle{\pgfqpoint{0.100000in}{0.212622in}}{\pgfqpoint{3.696000in}{3.696000in}}%
\pgfusepath{clip}%
\pgfsetbuttcap%
\pgfsetroundjoin%
\definecolor{currentfill}{rgb}{0.121569,0.466667,0.705882}%
\pgfsetfillcolor{currentfill}%
\pgfsetfillopacity{0.300032}%
\pgfsetlinewidth{1.003750pt}%
\definecolor{currentstroke}{rgb}{0.121569,0.466667,0.705882}%
\pgfsetstrokecolor{currentstroke}%
\pgfsetstrokeopacity{0.300032}%
\pgfsetdash{}{0pt}%
\pgfpathmoveto{\pgfqpoint{1.646066in}{2.111290in}}%
\pgfpathcurveto{\pgfqpoint{1.654302in}{2.111290in}}{\pgfqpoint{1.662202in}{2.114563in}}{\pgfqpoint{1.668026in}{2.120386in}}%
\pgfpathcurveto{\pgfqpoint{1.673850in}{2.126210in}}{\pgfqpoint{1.677122in}{2.134110in}}{\pgfqpoint{1.677122in}{2.142347in}}%
\pgfpathcurveto{\pgfqpoint{1.677122in}{2.150583in}}{\pgfqpoint{1.673850in}{2.158483in}}{\pgfqpoint{1.668026in}{2.164307in}}%
\pgfpathcurveto{\pgfqpoint{1.662202in}{2.170131in}}{\pgfqpoint{1.654302in}{2.173403in}}{\pgfqpoint{1.646066in}{2.173403in}}%
\pgfpathcurveto{\pgfqpoint{1.637829in}{2.173403in}}{\pgfqpoint{1.629929in}{2.170131in}}{\pgfqpoint{1.624105in}{2.164307in}}%
\pgfpathcurveto{\pgfqpoint{1.618282in}{2.158483in}}{\pgfqpoint{1.615009in}{2.150583in}}{\pgfqpoint{1.615009in}{2.142347in}}%
\pgfpathcurveto{\pgfqpoint{1.615009in}{2.134110in}}{\pgfqpoint{1.618282in}{2.126210in}}{\pgfqpoint{1.624105in}{2.120386in}}%
\pgfpathcurveto{\pgfqpoint{1.629929in}{2.114563in}}{\pgfqpoint{1.637829in}{2.111290in}}{\pgfqpoint{1.646066in}{2.111290in}}%
\pgfpathclose%
\pgfusepath{stroke,fill}%
\end{pgfscope}%
\begin{pgfscope}%
\pgfpathrectangle{\pgfqpoint{0.100000in}{0.212622in}}{\pgfqpoint{3.696000in}{3.696000in}}%
\pgfusepath{clip}%
\pgfsetbuttcap%
\pgfsetroundjoin%
\definecolor{currentfill}{rgb}{0.121569,0.466667,0.705882}%
\pgfsetfillcolor{currentfill}%
\pgfsetfillopacity{0.300032}%
\pgfsetlinewidth{1.003750pt}%
\definecolor{currentstroke}{rgb}{0.121569,0.466667,0.705882}%
\pgfsetstrokecolor{currentstroke}%
\pgfsetstrokeopacity{0.300032}%
\pgfsetdash{}{0pt}%
\pgfpathmoveto{\pgfqpoint{1.646066in}{2.111290in}}%
\pgfpathcurveto{\pgfqpoint{1.654302in}{2.111290in}}{\pgfqpoint{1.662202in}{2.114563in}}{\pgfqpoint{1.668026in}{2.120386in}}%
\pgfpathcurveto{\pgfqpoint{1.673850in}{2.126210in}}{\pgfqpoint{1.677122in}{2.134110in}}{\pgfqpoint{1.677122in}{2.142347in}}%
\pgfpathcurveto{\pgfqpoint{1.677122in}{2.150583in}}{\pgfqpoint{1.673850in}{2.158483in}}{\pgfqpoint{1.668026in}{2.164307in}}%
\pgfpathcurveto{\pgfqpoint{1.662202in}{2.170131in}}{\pgfqpoint{1.654302in}{2.173403in}}{\pgfqpoint{1.646066in}{2.173403in}}%
\pgfpathcurveto{\pgfqpoint{1.637829in}{2.173403in}}{\pgfqpoint{1.629929in}{2.170131in}}{\pgfqpoint{1.624105in}{2.164307in}}%
\pgfpathcurveto{\pgfqpoint{1.618282in}{2.158483in}}{\pgfqpoint{1.615009in}{2.150583in}}{\pgfqpoint{1.615009in}{2.142347in}}%
\pgfpathcurveto{\pgfqpoint{1.615009in}{2.134110in}}{\pgfqpoint{1.618282in}{2.126210in}}{\pgfqpoint{1.624105in}{2.120386in}}%
\pgfpathcurveto{\pgfqpoint{1.629929in}{2.114563in}}{\pgfqpoint{1.637829in}{2.111290in}}{\pgfqpoint{1.646066in}{2.111290in}}%
\pgfpathclose%
\pgfusepath{stroke,fill}%
\end{pgfscope}%
\begin{pgfscope}%
\pgfpathrectangle{\pgfqpoint{0.100000in}{0.212622in}}{\pgfqpoint{3.696000in}{3.696000in}}%
\pgfusepath{clip}%
\pgfsetbuttcap%
\pgfsetroundjoin%
\definecolor{currentfill}{rgb}{0.121569,0.466667,0.705882}%
\pgfsetfillcolor{currentfill}%
\pgfsetfillopacity{0.300032}%
\pgfsetlinewidth{1.003750pt}%
\definecolor{currentstroke}{rgb}{0.121569,0.466667,0.705882}%
\pgfsetstrokecolor{currentstroke}%
\pgfsetstrokeopacity{0.300032}%
\pgfsetdash{}{0pt}%
\pgfpathmoveto{\pgfqpoint{1.646066in}{2.111290in}}%
\pgfpathcurveto{\pgfqpoint{1.654302in}{2.111290in}}{\pgfqpoint{1.662202in}{2.114563in}}{\pgfqpoint{1.668026in}{2.120386in}}%
\pgfpathcurveto{\pgfqpoint{1.673850in}{2.126210in}}{\pgfqpoint{1.677122in}{2.134110in}}{\pgfqpoint{1.677122in}{2.142347in}}%
\pgfpathcurveto{\pgfqpoint{1.677122in}{2.150583in}}{\pgfqpoint{1.673850in}{2.158483in}}{\pgfqpoint{1.668026in}{2.164307in}}%
\pgfpathcurveto{\pgfqpoint{1.662202in}{2.170131in}}{\pgfqpoint{1.654302in}{2.173403in}}{\pgfqpoint{1.646066in}{2.173403in}}%
\pgfpathcurveto{\pgfqpoint{1.637829in}{2.173403in}}{\pgfqpoint{1.629929in}{2.170131in}}{\pgfqpoint{1.624105in}{2.164307in}}%
\pgfpathcurveto{\pgfqpoint{1.618282in}{2.158483in}}{\pgfqpoint{1.615009in}{2.150583in}}{\pgfqpoint{1.615009in}{2.142347in}}%
\pgfpathcurveto{\pgfqpoint{1.615009in}{2.134110in}}{\pgfqpoint{1.618282in}{2.126210in}}{\pgfqpoint{1.624105in}{2.120386in}}%
\pgfpathcurveto{\pgfqpoint{1.629929in}{2.114563in}}{\pgfqpoint{1.637829in}{2.111290in}}{\pgfqpoint{1.646066in}{2.111290in}}%
\pgfpathclose%
\pgfusepath{stroke,fill}%
\end{pgfscope}%
\begin{pgfscope}%
\pgfpathrectangle{\pgfqpoint{0.100000in}{0.212622in}}{\pgfqpoint{3.696000in}{3.696000in}}%
\pgfusepath{clip}%
\pgfsetbuttcap%
\pgfsetroundjoin%
\definecolor{currentfill}{rgb}{0.121569,0.466667,0.705882}%
\pgfsetfillcolor{currentfill}%
\pgfsetfillopacity{0.300032}%
\pgfsetlinewidth{1.003750pt}%
\definecolor{currentstroke}{rgb}{0.121569,0.466667,0.705882}%
\pgfsetstrokecolor{currentstroke}%
\pgfsetstrokeopacity{0.300032}%
\pgfsetdash{}{0pt}%
\pgfpathmoveto{\pgfqpoint{1.646066in}{2.111290in}}%
\pgfpathcurveto{\pgfqpoint{1.654302in}{2.111290in}}{\pgfqpoint{1.662202in}{2.114563in}}{\pgfqpoint{1.668026in}{2.120386in}}%
\pgfpathcurveto{\pgfqpoint{1.673850in}{2.126210in}}{\pgfqpoint{1.677122in}{2.134110in}}{\pgfqpoint{1.677122in}{2.142347in}}%
\pgfpathcurveto{\pgfqpoint{1.677122in}{2.150583in}}{\pgfqpoint{1.673850in}{2.158483in}}{\pgfqpoint{1.668026in}{2.164307in}}%
\pgfpathcurveto{\pgfqpoint{1.662202in}{2.170131in}}{\pgfqpoint{1.654302in}{2.173403in}}{\pgfqpoint{1.646066in}{2.173403in}}%
\pgfpathcurveto{\pgfqpoint{1.637829in}{2.173403in}}{\pgfqpoint{1.629929in}{2.170131in}}{\pgfqpoint{1.624105in}{2.164307in}}%
\pgfpathcurveto{\pgfqpoint{1.618282in}{2.158483in}}{\pgfqpoint{1.615009in}{2.150583in}}{\pgfqpoint{1.615009in}{2.142347in}}%
\pgfpathcurveto{\pgfqpoint{1.615009in}{2.134110in}}{\pgfqpoint{1.618282in}{2.126210in}}{\pgfqpoint{1.624105in}{2.120386in}}%
\pgfpathcurveto{\pgfqpoint{1.629929in}{2.114563in}}{\pgfqpoint{1.637829in}{2.111290in}}{\pgfqpoint{1.646066in}{2.111290in}}%
\pgfpathclose%
\pgfusepath{stroke,fill}%
\end{pgfscope}%
\begin{pgfscope}%
\pgfpathrectangle{\pgfqpoint{0.100000in}{0.212622in}}{\pgfqpoint{3.696000in}{3.696000in}}%
\pgfusepath{clip}%
\pgfsetbuttcap%
\pgfsetroundjoin%
\definecolor{currentfill}{rgb}{0.121569,0.466667,0.705882}%
\pgfsetfillcolor{currentfill}%
\pgfsetfillopacity{0.300032}%
\pgfsetlinewidth{1.003750pt}%
\definecolor{currentstroke}{rgb}{0.121569,0.466667,0.705882}%
\pgfsetstrokecolor{currentstroke}%
\pgfsetstrokeopacity{0.300032}%
\pgfsetdash{}{0pt}%
\pgfpathmoveto{\pgfqpoint{1.646066in}{2.111290in}}%
\pgfpathcurveto{\pgfqpoint{1.654302in}{2.111290in}}{\pgfqpoint{1.662202in}{2.114563in}}{\pgfqpoint{1.668026in}{2.120386in}}%
\pgfpathcurveto{\pgfqpoint{1.673850in}{2.126210in}}{\pgfqpoint{1.677122in}{2.134110in}}{\pgfqpoint{1.677122in}{2.142347in}}%
\pgfpathcurveto{\pgfqpoint{1.677122in}{2.150583in}}{\pgfqpoint{1.673850in}{2.158483in}}{\pgfqpoint{1.668026in}{2.164307in}}%
\pgfpathcurveto{\pgfqpoint{1.662202in}{2.170131in}}{\pgfqpoint{1.654302in}{2.173403in}}{\pgfqpoint{1.646066in}{2.173403in}}%
\pgfpathcurveto{\pgfqpoint{1.637829in}{2.173403in}}{\pgfqpoint{1.629929in}{2.170131in}}{\pgfqpoint{1.624105in}{2.164307in}}%
\pgfpathcurveto{\pgfqpoint{1.618282in}{2.158483in}}{\pgfqpoint{1.615009in}{2.150583in}}{\pgfqpoint{1.615009in}{2.142347in}}%
\pgfpathcurveto{\pgfqpoint{1.615009in}{2.134110in}}{\pgfqpoint{1.618282in}{2.126210in}}{\pgfqpoint{1.624105in}{2.120386in}}%
\pgfpathcurveto{\pgfqpoint{1.629929in}{2.114563in}}{\pgfqpoint{1.637829in}{2.111290in}}{\pgfqpoint{1.646066in}{2.111290in}}%
\pgfpathclose%
\pgfusepath{stroke,fill}%
\end{pgfscope}%
\begin{pgfscope}%
\pgfpathrectangle{\pgfqpoint{0.100000in}{0.212622in}}{\pgfqpoint{3.696000in}{3.696000in}}%
\pgfusepath{clip}%
\pgfsetbuttcap%
\pgfsetroundjoin%
\definecolor{currentfill}{rgb}{0.121569,0.466667,0.705882}%
\pgfsetfillcolor{currentfill}%
\pgfsetfillopacity{0.300032}%
\pgfsetlinewidth{1.003750pt}%
\definecolor{currentstroke}{rgb}{0.121569,0.466667,0.705882}%
\pgfsetstrokecolor{currentstroke}%
\pgfsetstrokeopacity{0.300032}%
\pgfsetdash{}{0pt}%
\pgfpathmoveto{\pgfqpoint{1.646066in}{2.111290in}}%
\pgfpathcurveto{\pgfqpoint{1.654302in}{2.111290in}}{\pgfqpoint{1.662202in}{2.114563in}}{\pgfqpoint{1.668026in}{2.120386in}}%
\pgfpathcurveto{\pgfqpoint{1.673850in}{2.126210in}}{\pgfqpoint{1.677122in}{2.134110in}}{\pgfqpoint{1.677122in}{2.142347in}}%
\pgfpathcurveto{\pgfqpoint{1.677122in}{2.150583in}}{\pgfqpoint{1.673850in}{2.158483in}}{\pgfqpoint{1.668026in}{2.164307in}}%
\pgfpathcurveto{\pgfqpoint{1.662202in}{2.170131in}}{\pgfqpoint{1.654302in}{2.173403in}}{\pgfqpoint{1.646066in}{2.173403in}}%
\pgfpathcurveto{\pgfqpoint{1.637829in}{2.173403in}}{\pgfqpoint{1.629929in}{2.170131in}}{\pgfqpoint{1.624105in}{2.164307in}}%
\pgfpathcurveto{\pgfqpoint{1.618282in}{2.158483in}}{\pgfqpoint{1.615009in}{2.150583in}}{\pgfqpoint{1.615009in}{2.142347in}}%
\pgfpathcurveto{\pgfqpoint{1.615009in}{2.134110in}}{\pgfqpoint{1.618282in}{2.126210in}}{\pgfqpoint{1.624105in}{2.120386in}}%
\pgfpathcurveto{\pgfqpoint{1.629929in}{2.114563in}}{\pgfqpoint{1.637829in}{2.111290in}}{\pgfqpoint{1.646066in}{2.111290in}}%
\pgfpathclose%
\pgfusepath{stroke,fill}%
\end{pgfscope}%
\begin{pgfscope}%
\pgfpathrectangle{\pgfqpoint{0.100000in}{0.212622in}}{\pgfqpoint{3.696000in}{3.696000in}}%
\pgfusepath{clip}%
\pgfsetbuttcap%
\pgfsetroundjoin%
\definecolor{currentfill}{rgb}{0.121569,0.466667,0.705882}%
\pgfsetfillcolor{currentfill}%
\pgfsetfillopacity{0.300032}%
\pgfsetlinewidth{1.003750pt}%
\definecolor{currentstroke}{rgb}{0.121569,0.466667,0.705882}%
\pgfsetstrokecolor{currentstroke}%
\pgfsetstrokeopacity{0.300032}%
\pgfsetdash{}{0pt}%
\pgfpathmoveto{\pgfqpoint{1.646066in}{2.111290in}}%
\pgfpathcurveto{\pgfqpoint{1.654302in}{2.111290in}}{\pgfqpoint{1.662202in}{2.114563in}}{\pgfqpoint{1.668026in}{2.120386in}}%
\pgfpathcurveto{\pgfqpoint{1.673850in}{2.126210in}}{\pgfqpoint{1.677122in}{2.134110in}}{\pgfqpoint{1.677122in}{2.142347in}}%
\pgfpathcurveto{\pgfqpoint{1.677122in}{2.150583in}}{\pgfqpoint{1.673850in}{2.158483in}}{\pgfqpoint{1.668026in}{2.164307in}}%
\pgfpathcurveto{\pgfqpoint{1.662202in}{2.170131in}}{\pgfqpoint{1.654302in}{2.173403in}}{\pgfqpoint{1.646066in}{2.173403in}}%
\pgfpathcurveto{\pgfqpoint{1.637829in}{2.173403in}}{\pgfqpoint{1.629929in}{2.170131in}}{\pgfqpoint{1.624105in}{2.164307in}}%
\pgfpathcurveto{\pgfqpoint{1.618282in}{2.158483in}}{\pgfqpoint{1.615009in}{2.150583in}}{\pgfqpoint{1.615009in}{2.142347in}}%
\pgfpathcurveto{\pgfqpoint{1.615009in}{2.134110in}}{\pgfqpoint{1.618282in}{2.126210in}}{\pgfqpoint{1.624105in}{2.120386in}}%
\pgfpathcurveto{\pgfqpoint{1.629929in}{2.114563in}}{\pgfqpoint{1.637829in}{2.111290in}}{\pgfqpoint{1.646066in}{2.111290in}}%
\pgfpathclose%
\pgfusepath{stroke,fill}%
\end{pgfscope}%
\begin{pgfscope}%
\pgfpathrectangle{\pgfqpoint{0.100000in}{0.212622in}}{\pgfqpoint{3.696000in}{3.696000in}}%
\pgfusepath{clip}%
\pgfsetbuttcap%
\pgfsetroundjoin%
\definecolor{currentfill}{rgb}{0.121569,0.466667,0.705882}%
\pgfsetfillcolor{currentfill}%
\pgfsetfillopacity{0.300032}%
\pgfsetlinewidth{1.003750pt}%
\definecolor{currentstroke}{rgb}{0.121569,0.466667,0.705882}%
\pgfsetstrokecolor{currentstroke}%
\pgfsetstrokeopacity{0.300032}%
\pgfsetdash{}{0pt}%
\pgfpathmoveto{\pgfqpoint{1.646066in}{2.111290in}}%
\pgfpathcurveto{\pgfqpoint{1.654302in}{2.111290in}}{\pgfqpoint{1.662202in}{2.114563in}}{\pgfqpoint{1.668026in}{2.120386in}}%
\pgfpathcurveto{\pgfqpoint{1.673850in}{2.126210in}}{\pgfqpoint{1.677122in}{2.134110in}}{\pgfqpoint{1.677122in}{2.142347in}}%
\pgfpathcurveto{\pgfqpoint{1.677122in}{2.150583in}}{\pgfqpoint{1.673850in}{2.158483in}}{\pgfqpoint{1.668026in}{2.164307in}}%
\pgfpathcurveto{\pgfqpoint{1.662202in}{2.170131in}}{\pgfqpoint{1.654302in}{2.173403in}}{\pgfqpoint{1.646066in}{2.173403in}}%
\pgfpathcurveto{\pgfqpoint{1.637829in}{2.173403in}}{\pgfqpoint{1.629929in}{2.170131in}}{\pgfqpoint{1.624105in}{2.164307in}}%
\pgfpathcurveto{\pgfqpoint{1.618282in}{2.158483in}}{\pgfqpoint{1.615009in}{2.150583in}}{\pgfqpoint{1.615009in}{2.142347in}}%
\pgfpathcurveto{\pgfqpoint{1.615009in}{2.134110in}}{\pgfqpoint{1.618282in}{2.126210in}}{\pgfqpoint{1.624105in}{2.120386in}}%
\pgfpathcurveto{\pgfqpoint{1.629929in}{2.114563in}}{\pgfqpoint{1.637829in}{2.111290in}}{\pgfqpoint{1.646066in}{2.111290in}}%
\pgfpathclose%
\pgfusepath{stroke,fill}%
\end{pgfscope}%
\begin{pgfscope}%
\pgfpathrectangle{\pgfqpoint{0.100000in}{0.212622in}}{\pgfqpoint{3.696000in}{3.696000in}}%
\pgfusepath{clip}%
\pgfsetbuttcap%
\pgfsetroundjoin%
\definecolor{currentfill}{rgb}{0.121569,0.466667,0.705882}%
\pgfsetfillcolor{currentfill}%
\pgfsetfillopacity{0.300032}%
\pgfsetlinewidth{1.003750pt}%
\definecolor{currentstroke}{rgb}{0.121569,0.466667,0.705882}%
\pgfsetstrokecolor{currentstroke}%
\pgfsetstrokeopacity{0.300032}%
\pgfsetdash{}{0pt}%
\pgfpathmoveto{\pgfqpoint{1.646066in}{2.111290in}}%
\pgfpathcurveto{\pgfqpoint{1.654302in}{2.111290in}}{\pgfqpoint{1.662202in}{2.114563in}}{\pgfqpoint{1.668026in}{2.120386in}}%
\pgfpathcurveto{\pgfqpoint{1.673850in}{2.126210in}}{\pgfqpoint{1.677122in}{2.134110in}}{\pgfqpoint{1.677122in}{2.142347in}}%
\pgfpathcurveto{\pgfqpoint{1.677122in}{2.150583in}}{\pgfqpoint{1.673850in}{2.158483in}}{\pgfqpoint{1.668026in}{2.164307in}}%
\pgfpathcurveto{\pgfqpoint{1.662202in}{2.170131in}}{\pgfqpoint{1.654302in}{2.173403in}}{\pgfqpoint{1.646066in}{2.173403in}}%
\pgfpathcurveto{\pgfqpoint{1.637829in}{2.173403in}}{\pgfqpoint{1.629929in}{2.170131in}}{\pgfqpoint{1.624105in}{2.164307in}}%
\pgfpathcurveto{\pgfqpoint{1.618282in}{2.158483in}}{\pgfqpoint{1.615009in}{2.150583in}}{\pgfqpoint{1.615009in}{2.142347in}}%
\pgfpathcurveto{\pgfqpoint{1.615009in}{2.134110in}}{\pgfqpoint{1.618282in}{2.126210in}}{\pgfqpoint{1.624105in}{2.120386in}}%
\pgfpathcurveto{\pgfqpoint{1.629929in}{2.114563in}}{\pgfqpoint{1.637829in}{2.111290in}}{\pgfqpoint{1.646066in}{2.111290in}}%
\pgfpathclose%
\pgfusepath{stroke,fill}%
\end{pgfscope}%
\begin{pgfscope}%
\pgfpathrectangle{\pgfqpoint{0.100000in}{0.212622in}}{\pgfqpoint{3.696000in}{3.696000in}}%
\pgfusepath{clip}%
\pgfsetbuttcap%
\pgfsetroundjoin%
\definecolor{currentfill}{rgb}{0.121569,0.466667,0.705882}%
\pgfsetfillcolor{currentfill}%
\pgfsetfillopacity{0.300032}%
\pgfsetlinewidth{1.003750pt}%
\definecolor{currentstroke}{rgb}{0.121569,0.466667,0.705882}%
\pgfsetstrokecolor{currentstroke}%
\pgfsetstrokeopacity{0.300032}%
\pgfsetdash{}{0pt}%
\pgfpathmoveto{\pgfqpoint{1.646066in}{2.111290in}}%
\pgfpathcurveto{\pgfqpoint{1.654302in}{2.111290in}}{\pgfqpoint{1.662202in}{2.114563in}}{\pgfqpoint{1.668026in}{2.120386in}}%
\pgfpathcurveto{\pgfqpoint{1.673850in}{2.126210in}}{\pgfqpoint{1.677122in}{2.134110in}}{\pgfqpoint{1.677122in}{2.142347in}}%
\pgfpathcurveto{\pgfqpoint{1.677122in}{2.150583in}}{\pgfqpoint{1.673850in}{2.158483in}}{\pgfqpoint{1.668026in}{2.164307in}}%
\pgfpathcurveto{\pgfqpoint{1.662202in}{2.170131in}}{\pgfqpoint{1.654302in}{2.173403in}}{\pgfqpoint{1.646066in}{2.173403in}}%
\pgfpathcurveto{\pgfqpoint{1.637829in}{2.173403in}}{\pgfqpoint{1.629929in}{2.170131in}}{\pgfqpoint{1.624105in}{2.164307in}}%
\pgfpathcurveto{\pgfqpoint{1.618282in}{2.158483in}}{\pgfqpoint{1.615009in}{2.150583in}}{\pgfqpoint{1.615009in}{2.142347in}}%
\pgfpathcurveto{\pgfqpoint{1.615009in}{2.134110in}}{\pgfqpoint{1.618282in}{2.126210in}}{\pgfqpoint{1.624105in}{2.120386in}}%
\pgfpathcurveto{\pgfqpoint{1.629929in}{2.114563in}}{\pgfqpoint{1.637829in}{2.111290in}}{\pgfqpoint{1.646066in}{2.111290in}}%
\pgfpathclose%
\pgfusepath{stroke,fill}%
\end{pgfscope}%
\begin{pgfscope}%
\pgfpathrectangle{\pgfqpoint{0.100000in}{0.212622in}}{\pgfqpoint{3.696000in}{3.696000in}}%
\pgfusepath{clip}%
\pgfsetbuttcap%
\pgfsetroundjoin%
\definecolor{currentfill}{rgb}{0.121569,0.466667,0.705882}%
\pgfsetfillcolor{currentfill}%
\pgfsetfillopacity{0.300032}%
\pgfsetlinewidth{1.003750pt}%
\definecolor{currentstroke}{rgb}{0.121569,0.466667,0.705882}%
\pgfsetstrokecolor{currentstroke}%
\pgfsetstrokeopacity{0.300032}%
\pgfsetdash{}{0pt}%
\pgfpathmoveto{\pgfqpoint{1.646066in}{2.111290in}}%
\pgfpathcurveto{\pgfqpoint{1.654302in}{2.111290in}}{\pgfqpoint{1.662202in}{2.114563in}}{\pgfqpoint{1.668026in}{2.120386in}}%
\pgfpathcurveto{\pgfqpoint{1.673850in}{2.126210in}}{\pgfqpoint{1.677122in}{2.134110in}}{\pgfqpoint{1.677122in}{2.142347in}}%
\pgfpathcurveto{\pgfqpoint{1.677122in}{2.150583in}}{\pgfqpoint{1.673850in}{2.158483in}}{\pgfqpoint{1.668026in}{2.164307in}}%
\pgfpathcurveto{\pgfqpoint{1.662202in}{2.170131in}}{\pgfqpoint{1.654302in}{2.173403in}}{\pgfqpoint{1.646066in}{2.173403in}}%
\pgfpathcurveto{\pgfqpoint{1.637829in}{2.173403in}}{\pgfqpoint{1.629929in}{2.170131in}}{\pgfqpoint{1.624105in}{2.164307in}}%
\pgfpathcurveto{\pgfqpoint{1.618282in}{2.158483in}}{\pgfqpoint{1.615009in}{2.150583in}}{\pgfqpoint{1.615009in}{2.142347in}}%
\pgfpathcurveto{\pgfqpoint{1.615009in}{2.134110in}}{\pgfqpoint{1.618282in}{2.126210in}}{\pgfqpoint{1.624105in}{2.120386in}}%
\pgfpathcurveto{\pgfqpoint{1.629929in}{2.114563in}}{\pgfqpoint{1.637829in}{2.111290in}}{\pgfqpoint{1.646066in}{2.111290in}}%
\pgfpathclose%
\pgfusepath{stroke,fill}%
\end{pgfscope}%
\begin{pgfscope}%
\pgfpathrectangle{\pgfqpoint{0.100000in}{0.212622in}}{\pgfqpoint{3.696000in}{3.696000in}}%
\pgfusepath{clip}%
\pgfsetbuttcap%
\pgfsetroundjoin%
\definecolor{currentfill}{rgb}{0.121569,0.466667,0.705882}%
\pgfsetfillcolor{currentfill}%
\pgfsetfillopacity{0.300032}%
\pgfsetlinewidth{1.003750pt}%
\definecolor{currentstroke}{rgb}{0.121569,0.466667,0.705882}%
\pgfsetstrokecolor{currentstroke}%
\pgfsetstrokeopacity{0.300032}%
\pgfsetdash{}{0pt}%
\pgfpathmoveto{\pgfqpoint{1.646066in}{2.111290in}}%
\pgfpathcurveto{\pgfqpoint{1.654302in}{2.111290in}}{\pgfqpoint{1.662202in}{2.114563in}}{\pgfqpoint{1.668026in}{2.120386in}}%
\pgfpathcurveto{\pgfqpoint{1.673850in}{2.126210in}}{\pgfqpoint{1.677122in}{2.134110in}}{\pgfqpoint{1.677122in}{2.142347in}}%
\pgfpathcurveto{\pgfqpoint{1.677122in}{2.150583in}}{\pgfqpoint{1.673850in}{2.158483in}}{\pgfqpoint{1.668026in}{2.164307in}}%
\pgfpathcurveto{\pgfqpoint{1.662202in}{2.170131in}}{\pgfqpoint{1.654302in}{2.173403in}}{\pgfqpoint{1.646066in}{2.173403in}}%
\pgfpathcurveto{\pgfqpoint{1.637829in}{2.173403in}}{\pgfqpoint{1.629929in}{2.170131in}}{\pgfqpoint{1.624105in}{2.164307in}}%
\pgfpathcurveto{\pgfqpoint{1.618282in}{2.158483in}}{\pgfqpoint{1.615009in}{2.150583in}}{\pgfqpoint{1.615009in}{2.142347in}}%
\pgfpathcurveto{\pgfqpoint{1.615009in}{2.134110in}}{\pgfqpoint{1.618282in}{2.126210in}}{\pgfqpoint{1.624105in}{2.120386in}}%
\pgfpathcurveto{\pgfqpoint{1.629929in}{2.114563in}}{\pgfqpoint{1.637829in}{2.111290in}}{\pgfqpoint{1.646066in}{2.111290in}}%
\pgfpathclose%
\pgfusepath{stroke,fill}%
\end{pgfscope}%
\begin{pgfscope}%
\pgfpathrectangle{\pgfqpoint{0.100000in}{0.212622in}}{\pgfqpoint{3.696000in}{3.696000in}}%
\pgfusepath{clip}%
\pgfsetbuttcap%
\pgfsetroundjoin%
\definecolor{currentfill}{rgb}{0.121569,0.466667,0.705882}%
\pgfsetfillcolor{currentfill}%
\pgfsetfillopacity{0.300032}%
\pgfsetlinewidth{1.003750pt}%
\definecolor{currentstroke}{rgb}{0.121569,0.466667,0.705882}%
\pgfsetstrokecolor{currentstroke}%
\pgfsetstrokeopacity{0.300032}%
\pgfsetdash{}{0pt}%
\pgfpathmoveto{\pgfqpoint{1.646066in}{2.111290in}}%
\pgfpathcurveto{\pgfqpoint{1.654302in}{2.111290in}}{\pgfqpoint{1.662202in}{2.114563in}}{\pgfqpoint{1.668026in}{2.120386in}}%
\pgfpathcurveto{\pgfqpoint{1.673850in}{2.126210in}}{\pgfqpoint{1.677122in}{2.134110in}}{\pgfqpoint{1.677122in}{2.142347in}}%
\pgfpathcurveto{\pgfqpoint{1.677122in}{2.150583in}}{\pgfqpoint{1.673850in}{2.158483in}}{\pgfqpoint{1.668026in}{2.164307in}}%
\pgfpathcurveto{\pgfqpoint{1.662202in}{2.170131in}}{\pgfqpoint{1.654302in}{2.173403in}}{\pgfqpoint{1.646066in}{2.173403in}}%
\pgfpathcurveto{\pgfqpoint{1.637829in}{2.173403in}}{\pgfqpoint{1.629929in}{2.170131in}}{\pgfqpoint{1.624105in}{2.164307in}}%
\pgfpathcurveto{\pgfqpoint{1.618282in}{2.158483in}}{\pgfqpoint{1.615009in}{2.150583in}}{\pgfqpoint{1.615009in}{2.142347in}}%
\pgfpathcurveto{\pgfqpoint{1.615009in}{2.134110in}}{\pgfqpoint{1.618282in}{2.126210in}}{\pgfqpoint{1.624105in}{2.120386in}}%
\pgfpathcurveto{\pgfqpoint{1.629929in}{2.114563in}}{\pgfqpoint{1.637829in}{2.111290in}}{\pgfqpoint{1.646066in}{2.111290in}}%
\pgfpathclose%
\pgfusepath{stroke,fill}%
\end{pgfscope}%
\begin{pgfscope}%
\pgfpathrectangle{\pgfqpoint{0.100000in}{0.212622in}}{\pgfqpoint{3.696000in}{3.696000in}}%
\pgfusepath{clip}%
\pgfsetbuttcap%
\pgfsetroundjoin%
\definecolor{currentfill}{rgb}{0.121569,0.466667,0.705882}%
\pgfsetfillcolor{currentfill}%
\pgfsetfillopacity{0.300032}%
\pgfsetlinewidth{1.003750pt}%
\definecolor{currentstroke}{rgb}{0.121569,0.466667,0.705882}%
\pgfsetstrokecolor{currentstroke}%
\pgfsetstrokeopacity{0.300032}%
\pgfsetdash{}{0pt}%
\pgfpathmoveto{\pgfqpoint{1.646066in}{2.111290in}}%
\pgfpathcurveto{\pgfqpoint{1.654302in}{2.111290in}}{\pgfqpoint{1.662202in}{2.114563in}}{\pgfqpoint{1.668026in}{2.120386in}}%
\pgfpathcurveto{\pgfqpoint{1.673850in}{2.126210in}}{\pgfqpoint{1.677122in}{2.134110in}}{\pgfqpoint{1.677122in}{2.142347in}}%
\pgfpathcurveto{\pgfqpoint{1.677122in}{2.150583in}}{\pgfqpoint{1.673850in}{2.158483in}}{\pgfqpoint{1.668026in}{2.164307in}}%
\pgfpathcurveto{\pgfqpoint{1.662202in}{2.170131in}}{\pgfqpoint{1.654302in}{2.173403in}}{\pgfqpoint{1.646066in}{2.173403in}}%
\pgfpathcurveto{\pgfqpoint{1.637829in}{2.173403in}}{\pgfqpoint{1.629929in}{2.170131in}}{\pgfqpoint{1.624105in}{2.164307in}}%
\pgfpathcurveto{\pgfqpoint{1.618282in}{2.158483in}}{\pgfqpoint{1.615009in}{2.150583in}}{\pgfqpoint{1.615009in}{2.142347in}}%
\pgfpathcurveto{\pgfqpoint{1.615009in}{2.134110in}}{\pgfqpoint{1.618282in}{2.126210in}}{\pgfqpoint{1.624105in}{2.120386in}}%
\pgfpathcurveto{\pgfqpoint{1.629929in}{2.114563in}}{\pgfqpoint{1.637829in}{2.111290in}}{\pgfqpoint{1.646066in}{2.111290in}}%
\pgfpathclose%
\pgfusepath{stroke,fill}%
\end{pgfscope}%
\begin{pgfscope}%
\pgfpathrectangle{\pgfqpoint{0.100000in}{0.212622in}}{\pgfqpoint{3.696000in}{3.696000in}}%
\pgfusepath{clip}%
\pgfsetbuttcap%
\pgfsetroundjoin%
\definecolor{currentfill}{rgb}{0.121569,0.466667,0.705882}%
\pgfsetfillcolor{currentfill}%
\pgfsetfillopacity{0.300032}%
\pgfsetlinewidth{1.003750pt}%
\definecolor{currentstroke}{rgb}{0.121569,0.466667,0.705882}%
\pgfsetstrokecolor{currentstroke}%
\pgfsetstrokeopacity{0.300032}%
\pgfsetdash{}{0pt}%
\pgfpathmoveto{\pgfqpoint{1.646066in}{2.111290in}}%
\pgfpathcurveto{\pgfqpoint{1.654302in}{2.111290in}}{\pgfqpoint{1.662202in}{2.114563in}}{\pgfqpoint{1.668026in}{2.120386in}}%
\pgfpathcurveto{\pgfqpoint{1.673850in}{2.126210in}}{\pgfqpoint{1.677122in}{2.134110in}}{\pgfqpoint{1.677122in}{2.142347in}}%
\pgfpathcurveto{\pgfqpoint{1.677122in}{2.150583in}}{\pgfqpoint{1.673850in}{2.158483in}}{\pgfqpoint{1.668026in}{2.164307in}}%
\pgfpathcurveto{\pgfqpoint{1.662202in}{2.170131in}}{\pgfqpoint{1.654302in}{2.173403in}}{\pgfqpoint{1.646066in}{2.173403in}}%
\pgfpathcurveto{\pgfqpoint{1.637829in}{2.173403in}}{\pgfqpoint{1.629929in}{2.170131in}}{\pgfqpoint{1.624105in}{2.164307in}}%
\pgfpathcurveto{\pgfqpoint{1.618282in}{2.158483in}}{\pgfqpoint{1.615009in}{2.150583in}}{\pgfqpoint{1.615009in}{2.142347in}}%
\pgfpathcurveto{\pgfqpoint{1.615009in}{2.134110in}}{\pgfqpoint{1.618282in}{2.126210in}}{\pgfqpoint{1.624105in}{2.120386in}}%
\pgfpathcurveto{\pgfqpoint{1.629929in}{2.114563in}}{\pgfqpoint{1.637829in}{2.111290in}}{\pgfqpoint{1.646066in}{2.111290in}}%
\pgfpathclose%
\pgfusepath{stroke,fill}%
\end{pgfscope}%
\begin{pgfscope}%
\pgfpathrectangle{\pgfqpoint{0.100000in}{0.212622in}}{\pgfqpoint{3.696000in}{3.696000in}}%
\pgfusepath{clip}%
\pgfsetbuttcap%
\pgfsetroundjoin%
\definecolor{currentfill}{rgb}{0.121569,0.466667,0.705882}%
\pgfsetfillcolor{currentfill}%
\pgfsetfillopacity{0.300032}%
\pgfsetlinewidth{1.003750pt}%
\definecolor{currentstroke}{rgb}{0.121569,0.466667,0.705882}%
\pgfsetstrokecolor{currentstroke}%
\pgfsetstrokeopacity{0.300032}%
\pgfsetdash{}{0pt}%
\pgfpathmoveto{\pgfqpoint{1.646066in}{2.111290in}}%
\pgfpathcurveto{\pgfqpoint{1.654302in}{2.111290in}}{\pgfqpoint{1.662202in}{2.114563in}}{\pgfqpoint{1.668026in}{2.120386in}}%
\pgfpathcurveto{\pgfqpoint{1.673850in}{2.126210in}}{\pgfqpoint{1.677122in}{2.134110in}}{\pgfqpoint{1.677122in}{2.142347in}}%
\pgfpathcurveto{\pgfqpoint{1.677122in}{2.150583in}}{\pgfqpoint{1.673850in}{2.158483in}}{\pgfqpoint{1.668026in}{2.164307in}}%
\pgfpathcurveto{\pgfqpoint{1.662202in}{2.170131in}}{\pgfqpoint{1.654302in}{2.173403in}}{\pgfqpoint{1.646066in}{2.173403in}}%
\pgfpathcurveto{\pgfqpoint{1.637829in}{2.173403in}}{\pgfqpoint{1.629929in}{2.170131in}}{\pgfqpoint{1.624105in}{2.164307in}}%
\pgfpathcurveto{\pgfqpoint{1.618282in}{2.158483in}}{\pgfqpoint{1.615009in}{2.150583in}}{\pgfqpoint{1.615009in}{2.142347in}}%
\pgfpathcurveto{\pgfqpoint{1.615009in}{2.134110in}}{\pgfqpoint{1.618282in}{2.126210in}}{\pgfqpoint{1.624105in}{2.120386in}}%
\pgfpathcurveto{\pgfqpoint{1.629929in}{2.114563in}}{\pgfqpoint{1.637829in}{2.111290in}}{\pgfqpoint{1.646066in}{2.111290in}}%
\pgfpathclose%
\pgfusepath{stroke,fill}%
\end{pgfscope}%
\begin{pgfscope}%
\pgfpathrectangle{\pgfqpoint{0.100000in}{0.212622in}}{\pgfqpoint{3.696000in}{3.696000in}}%
\pgfusepath{clip}%
\pgfsetbuttcap%
\pgfsetroundjoin%
\definecolor{currentfill}{rgb}{0.121569,0.466667,0.705882}%
\pgfsetfillcolor{currentfill}%
\pgfsetfillopacity{0.300032}%
\pgfsetlinewidth{1.003750pt}%
\definecolor{currentstroke}{rgb}{0.121569,0.466667,0.705882}%
\pgfsetstrokecolor{currentstroke}%
\pgfsetstrokeopacity{0.300032}%
\pgfsetdash{}{0pt}%
\pgfpathmoveto{\pgfqpoint{1.646066in}{2.111290in}}%
\pgfpathcurveto{\pgfqpoint{1.654302in}{2.111290in}}{\pgfqpoint{1.662202in}{2.114563in}}{\pgfqpoint{1.668026in}{2.120386in}}%
\pgfpathcurveto{\pgfqpoint{1.673850in}{2.126210in}}{\pgfqpoint{1.677122in}{2.134110in}}{\pgfqpoint{1.677122in}{2.142347in}}%
\pgfpathcurveto{\pgfqpoint{1.677122in}{2.150583in}}{\pgfqpoint{1.673850in}{2.158483in}}{\pgfqpoint{1.668026in}{2.164307in}}%
\pgfpathcurveto{\pgfqpoint{1.662202in}{2.170131in}}{\pgfqpoint{1.654302in}{2.173403in}}{\pgfqpoint{1.646066in}{2.173403in}}%
\pgfpathcurveto{\pgfqpoint{1.637829in}{2.173403in}}{\pgfqpoint{1.629929in}{2.170131in}}{\pgfqpoint{1.624105in}{2.164307in}}%
\pgfpathcurveto{\pgfqpoint{1.618282in}{2.158483in}}{\pgfqpoint{1.615009in}{2.150583in}}{\pgfqpoint{1.615009in}{2.142347in}}%
\pgfpathcurveto{\pgfqpoint{1.615009in}{2.134110in}}{\pgfqpoint{1.618282in}{2.126210in}}{\pgfqpoint{1.624105in}{2.120386in}}%
\pgfpathcurveto{\pgfqpoint{1.629929in}{2.114563in}}{\pgfqpoint{1.637829in}{2.111290in}}{\pgfqpoint{1.646066in}{2.111290in}}%
\pgfpathclose%
\pgfusepath{stroke,fill}%
\end{pgfscope}%
\begin{pgfscope}%
\pgfpathrectangle{\pgfqpoint{0.100000in}{0.212622in}}{\pgfqpoint{3.696000in}{3.696000in}}%
\pgfusepath{clip}%
\pgfsetbuttcap%
\pgfsetroundjoin%
\definecolor{currentfill}{rgb}{0.121569,0.466667,0.705882}%
\pgfsetfillcolor{currentfill}%
\pgfsetfillopacity{0.300032}%
\pgfsetlinewidth{1.003750pt}%
\definecolor{currentstroke}{rgb}{0.121569,0.466667,0.705882}%
\pgfsetstrokecolor{currentstroke}%
\pgfsetstrokeopacity{0.300032}%
\pgfsetdash{}{0pt}%
\pgfpathmoveto{\pgfqpoint{1.646066in}{2.111290in}}%
\pgfpathcurveto{\pgfqpoint{1.654302in}{2.111290in}}{\pgfqpoint{1.662202in}{2.114563in}}{\pgfqpoint{1.668026in}{2.120386in}}%
\pgfpathcurveto{\pgfqpoint{1.673850in}{2.126210in}}{\pgfqpoint{1.677122in}{2.134110in}}{\pgfqpoint{1.677122in}{2.142347in}}%
\pgfpathcurveto{\pgfqpoint{1.677122in}{2.150583in}}{\pgfqpoint{1.673850in}{2.158483in}}{\pgfqpoint{1.668026in}{2.164307in}}%
\pgfpathcurveto{\pgfqpoint{1.662202in}{2.170131in}}{\pgfqpoint{1.654302in}{2.173403in}}{\pgfqpoint{1.646066in}{2.173403in}}%
\pgfpathcurveto{\pgfqpoint{1.637829in}{2.173403in}}{\pgfqpoint{1.629929in}{2.170131in}}{\pgfqpoint{1.624105in}{2.164307in}}%
\pgfpathcurveto{\pgfqpoint{1.618282in}{2.158483in}}{\pgfqpoint{1.615009in}{2.150583in}}{\pgfqpoint{1.615009in}{2.142347in}}%
\pgfpathcurveto{\pgfqpoint{1.615009in}{2.134110in}}{\pgfqpoint{1.618282in}{2.126210in}}{\pgfqpoint{1.624105in}{2.120386in}}%
\pgfpathcurveto{\pgfqpoint{1.629929in}{2.114563in}}{\pgfqpoint{1.637829in}{2.111290in}}{\pgfqpoint{1.646066in}{2.111290in}}%
\pgfpathclose%
\pgfusepath{stroke,fill}%
\end{pgfscope}%
\begin{pgfscope}%
\pgfpathrectangle{\pgfqpoint{0.100000in}{0.212622in}}{\pgfqpoint{3.696000in}{3.696000in}}%
\pgfusepath{clip}%
\pgfsetbuttcap%
\pgfsetroundjoin%
\definecolor{currentfill}{rgb}{0.121569,0.466667,0.705882}%
\pgfsetfillcolor{currentfill}%
\pgfsetfillopacity{0.300032}%
\pgfsetlinewidth{1.003750pt}%
\definecolor{currentstroke}{rgb}{0.121569,0.466667,0.705882}%
\pgfsetstrokecolor{currentstroke}%
\pgfsetstrokeopacity{0.300032}%
\pgfsetdash{}{0pt}%
\pgfpathmoveto{\pgfqpoint{1.646066in}{2.111290in}}%
\pgfpathcurveto{\pgfqpoint{1.654302in}{2.111290in}}{\pgfqpoint{1.662202in}{2.114563in}}{\pgfqpoint{1.668026in}{2.120386in}}%
\pgfpathcurveto{\pgfqpoint{1.673850in}{2.126210in}}{\pgfqpoint{1.677122in}{2.134110in}}{\pgfqpoint{1.677122in}{2.142347in}}%
\pgfpathcurveto{\pgfqpoint{1.677122in}{2.150583in}}{\pgfqpoint{1.673850in}{2.158483in}}{\pgfqpoint{1.668026in}{2.164307in}}%
\pgfpathcurveto{\pgfqpoint{1.662202in}{2.170131in}}{\pgfqpoint{1.654302in}{2.173403in}}{\pgfqpoint{1.646066in}{2.173403in}}%
\pgfpathcurveto{\pgfqpoint{1.637829in}{2.173403in}}{\pgfqpoint{1.629929in}{2.170131in}}{\pgfqpoint{1.624105in}{2.164307in}}%
\pgfpathcurveto{\pgfqpoint{1.618282in}{2.158483in}}{\pgfqpoint{1.615009in}{2.150583in}}{\pgfqpoint{1.615009in}{2.142347in}}%
\pgfpathcurveto{\pgfqpoint{1.615009in}{2.134110in}}{\pgfqpoint{1.618282in}{2.126210in}}{\pgfqpoint{1.624105in}{2.120386in}}%
\pgfpathcurveto{\pgfqpoint{1.629929in}{2.114563in}}{\pgfqpoint{1.637829in}{2.111290in}}{\pgfqpoint{1.646066in}{2.111290in}}%
\pgfpathclose%
\pgfusepath{stroke,fill}%
\end{pgfscope}%
\begin{pgfscope}%
\pgfpathrectangle{\pgfqpoint{0.100000in}{0.212622in}}{\pgfqpoint{3.696000in}{3.696000in}}%
\pgfusepath{clip}%
\pgfsetbuttcap%
\pgfsetroundjoin%
\definecolor{currentfill}{rgb}{0.121569,0.466667,0.705882}%
\pgfsetfillcolor{currentfill}%
\pgfsetfillopacity{0.300032}%
\pgfsetlinewidth{1.003750pt}%
\definecolor{currentstroke}{rgb}{0.121569,0.466667,0.705882}%
\pgfsetstrokecolor{currentstroke}%
\pgfsetstrokeopacity{0.300032}%
\pgfsetdash{}{0pt}%
\pgfpathmoveto{\pgfqpoint{1.646066in}{2.111290in}}%
\pgfpathcurveto{\pgfqpoint{1.654302in}{2.111290in}}{\pgfqpoint{1.662202in}{2.114563in}}{\pgfqpoint{1.668026in}{2.120386in}}%
\pgfpathcurveto{\pgfqpoint{1.673850in}{2.126210in}}{\pgfqpoint{1.677122in}{2.134110in}}{\pgfqpoint{1.677122in}{2.142347in}}%
\pgfpathcurveto{\pgfqpoint{1.677122in}{2.150583in}}{\pgfqpoint{1.673850in}{2.158483in}}{\pgfqpoint{1.668026in}{2.164307in}}%
\pgfpathcurveto{\pgfqpoint{1.662202in}{2.170131in}}{\pgfqpoint{1.654302in}{2.173403in}}{\pgfqpoint{1.646066in}{2.173403in}}%
\pgfpathcurveto{\pgfqpoint{1.637829in}{2.173403in}}{\pgfqpoint{1.629929in}{2.170131in}}{\pgfqpoint{1.624105in}{2.164307in}}%
\pgfpathcurveto{\pgfqpoint{1.618282in}{2.158483in}}{\pgfqpoint{1.615009in}{2.150583in}}{\pgfqpoint{1.615009in}{2.142347in}}%
\pgfpathcurveto{\pgfqpoint{1.615009in}{2.134110in}}{\pgfqpoint{1.618282in}{2.126210in}}{\pgfqpoint{1.624105in}{2.120386in}}%
\pgfpathcurveto{\pgfqpoint{1.629929in}{2.114563in}}{\pgfqpoint{1.637829in}{2.111290in}}{\pgfqpoint{1.646066in}{2.111290in}}%
\pgfpathclose%
\pgfusepath{stroke,fill}%
\end{pgfscope}%
\begin{pgfscope}%
\pgfpathrectangle{\pgfqpoint{0.100000in}{0.212622in}}{\pgfqpoint{3.696000in}{3.696000in}}%
\pgfusepath{clip}%
\pgfsetbuttcap%
\pgfsetroundjoin%
\definecolor{currentfill}{rgb}{0.121569,0.466667,0.705882}%
\pgfsetfillcolor{currentfill}%
\pgfsetfillopacity{0.300032}%
\pgfsetlinewidth{1.003750pt}%
\definecolor{currentstroke}{rgb}{0.121569,0.466667,0.705882}%
\pgfsetstrokecolor{currentstroke}%
\pgfsetstrokeopacity{0.300032}%
\pgfsetdash{}{0pt}%
\pgfpathmoveto{\pgfqpoint{1.646066in}{2.111290in}}%
\pgfpathcurveto{\pgfqpoint{1.654302in}{2.111290in}}{\pgfqpoint{1.662202in}{2.114563in}}{\pgfqpoint{1.668026in}{2.120386in}}%
\pgfpathcurveto{\pgfqpoint{1.673850in}{2.126210in}}{\pgfqpoint{1.677122in}{2.134110in}}{\pgfqpoint{1.677122in}{2.142347in}}%
\pgfpathcurveto{\pgfqpoint{1.677122in}{2.150583in}}{\pgfqpoint{1.673850in}{2.158483in}}{\pgfqpoint{1.668026in}{2.164307in}}%
\pgfpathcurveto{\pgfqpoint{1.662202in}{2.170131in}}{\pgfqpoint{1.654302in}{2.173403in}}{\pgfqpoint{1.646066in}{2.173403in}}%
\pgfpathcurveto{\pgfqpoint{1.637829in}{2.173403in}}{\pgfqpoint{1.629929in}{2.170131in}}{\pgfqpoint{1.624105in}{2.164307in}}%
\pgfpathcurveto{\pgfqpoint{1.618282in}{2.158483in}}{\pgfqpoint{1.615009in}{2.150583in}}{\pgfqpoint{1.615009in}{2.142347in}}%
\pgfpathcurveto{\pgfqpoint{1.615009in}{2.134110in}}{\pgfqpoint{1.618282in}{2.126210in}}{\pgfqpoint{1.624105in}{2.120386in}}%
\pgfpathcurveto{\pgfqpoint{1.629929in}{2.114563in}}{\pgfqpoint{1.637829in}{2.111290in}}{\pgfqpoint{1.646066in}{2.111290in}}%
\pgfpathclose%
\pgfusepath{stroke,fill}%
\end{pgfscope}%
\begin{pgfscope}%
\pgfpathrectangle{\pgfqpoint{0.100000in}{0.212622in}}{\pgfqpoint{3.696000in}{3.696000in}}%
\pgfusepath{clip}%
\pgfsetbuttcap%
\pgfsetroundjoin%
\definecolor{currentfill}{rgb}{0.121569,0.466667,0.705882}%
\pgfsetfillcolor{currentfill}%
\pgfsetfillopacity{0.300032}%
\pgfsetlinewidth{1.003750pt}%
\definecolor{currentstroke}{rgb}{0.121569,0.466667,0.705882}%
\pgfsetstrokecolor{currentstroke}%
\pgfsetstrokeopacity{0.300032}%
\pgfsetdash{}{0pt}%
\pgfpathmoveto{\pgfqpoint{1.646066in}{2.111290in}}%
\pgfpathcurveto{\pgfqpoint{1.654302in}{2.111290in}}{\pgfqpoint{1.662202in}{2.114563in}}{\pgfqpoint{1.668026in}{2.120386in}}%
\pgfpathcurveto{\pgfqpoint{1.673850in}{2.126210in}}{\pgfqpoint{1.677122in}{2.134110in}}{\pgfqpoint{1.677122in}{2.142347in}}%
\pgfpathcurveto{\pgfqpoint{1.677122in}{2.150583in}}{\pgfqpoint{1.673850in}{2.158483in}}{\pgfqpoint{1.668026in}{2.164307in}}%
\pgfpathcurveto{\pgfqpoint{1.662202in}{2.170131in}}{\pgfqpoint{1.654302in}{2.173403in}}{\pgfqpoint{1.646066in}{2.173403in}}%
\pgfpathcurveto{\pgfqpoint{1.637829in}{2.173403in}}{\pgfqpoint{1.629929in}{2.170131in}}{\pgfqpoint{1.624105in}{2.164307in}}%
\pgfpathcurveto{\pgfqpoint{1.618282in}{2.158483in}}{\pgfqpoint{1.615009in}{2.150583in}}{\pgfqpoint{1.615009in}{2.142347in}}%
\pgfpathcurveto{\pgfqpoint{1.615009in}{2.134110in}}{\pgfqpoint{1.618282in}{2.126210in}}{\pgfqpoint{1.624105in}{2.120386in}}%
\pgfpathcurveto{\pgfqpoint{1.629929in}{2.114563in}}{\pgfqpoint{1.637829in}{2.111290in}}{\pgfqpoint{1.646066in}{2.111290in}}%
\pgfpathclose%
\pgfusepath{stroke,fill}%
\end{pgfscope}%
\begin{pgfscope}%
\pgfpathrectangle{\pgfqpoint{0.100000in}{0.212622in}}{\pgfqpoint{3.696000in}{3.696000in}}%
\pgfusepath{clip}%
\pgfsetbuttcap%
\pgfsetroundjoin%
\definecolor{currentfill}{rgb}{0.121569,0.466667,0.705882}%
\pgfsetfillcolor{currentfill}%
\pgfsetfillopacity{0.300032}%
\pgfsetlinewidth{1.003750pt}%
\definecolor{currentstroke}{rgb}{0.121569,0.466667,0.705882}%
\pgfsetstrokecolor{currentstroke}%
\pgfsetstrokeopacity{0.300032}%
\pgfsetdash{}{0pt}%
\pgfpathmoveto{\pgfqpoint{1.646066in}{2.111290in}}%
\pgfpathcurveto{\pgfqpoint{1.654302in}{2.111290in}}{\pgfqpoint{1.662202in}{2.114563in}}{\pgfqpoint{1.668026in}{2.120386in}}%
\pgfpathcurveto{\pgfqpoint{1.673850in}{2.126210in}}{\pgfqpoint{1.677122in}{2.134110in}}{\pgfqpoint{1.677122in}{2.142347in}}%
\pgfpathcurveto{\pgfqpoint{1.677122in}{2.150583in}}{\pgfqpoint{1.673850in}{2.158483in}}{\pgfqpoint{1.668026in}{2.164307in}}%
\pgfpathcurveto{\pgfqpoint{1.662202in}{2.170131in}}{\pgfqpoint{1.654302in}{2.173403in}}{\pgfqpoint{1.646066in}{2.173403in}}%
\pgfpathcurveto{\pgfqpoint{1.637829in}{2.173403in}}{\pgfqpoint{1.629929in}{2.170131in}}{\pgfqpoint{1.624105in}{2.164307in}}%
\pgfpathcurveto{\pgfqpoint{1.618282in}{2.158483in}}{\pgfqpoint{1.615009in}{2.150583in}}{\pgfqpoint{1.615009in}{2.142347in}}%
\pgfpathcurveto{\pgfqpoint{1.615009in}{2.134110in}}{\pgfqpoint{1.618282in}{2.126210in}}{\pgfqpoint{1.624105in}{2.120386in}}%
\pgfpathcurveto{\pgfqpoint{1.629929in}{2.114563in}}{\pgfqpoint{1.637829in}{2.111290in}}{\pgfqpoint{1.646066in}{2.111290in}}%
\pgfpathclose%
\pgfusepath{stroke,fill}%
\end{pgfscope}%
\begin{pgfscope}%
\pgfpathrectangle{\pgfqpoint{0.100000in}{0.212622in}}{\pgfqpoint{3.696000in}{3.696000in}}%
\pgfusepath{clip}%
\pgfsetbuttcap%
\pgfsetroundjoin%
\definecolor{currentfill}{rgb}{0.121569,0.466667,0.705882}%
\pgfsetfillcolor{currentfill}%
\pgfsetfillopacity{0.300651}%
\pgfsetlinewidth{1.003750pt}%
\definecolor{currentstroke}{rgb}{0.121569,0.466667,0.705882}%
\pgfsetstrokecolor{currentstroke}%
\pgfsetstrokeopacity{0.300651}%
\pgfsetdash{}{0pt}%
\pgfpathmoveto{\pgfqpoint{1.647814in}{2.112464in}}%
\pgfpathcurveto{\pgfqpoint{1.656050in}{2.112464in}}{\pgfqpoint{1.663950in}{2.115736in}}{\pgfqpoint{1.669774in}{2.121560in}}%
\pgfpathcurveto{\pgfqpoint{1.675598in}{2.127384in}}{\pgfqpoint{1.678871in}{2.135284in}}{\pgfqpoint{1.678871in}{2.143520in}}%
\pgfpathcurveto{\pgfqpoint{1.678871in}{2.151757in}}{\pgfqpoint{1.675598in}{2.159657in}}{\pgfqpoint{1.669774in}{2.165481in}}%
\pgfpathcurveto{\pgfqpoint{1.663950in}{2.171305in}}{\pgfqpoint{1.656050in}{2.174577in}}{\pgfqpoint{1.647814in}{2.174577in}}%
\pgfpathcurveto{\pgfqpoint{1.639578in}{2.174577in}}{\pgfqpoint{1.631678in}{2.171305in}}{\pgfqpoint{1.625854in}{2.165481in}}%
\pgfpathcurveto{\pgfqpoint{1.620030in}{2.159657in}}{\pgfqpoint{1.616758in}{2.151757in}}{\pgfqpoint{1.616758in}{2.143520in}}%
\pgfpathcurveto{\pgfqpoint{1.616758in}{2.135284in}}{\pgfqpoint{1.620030in}{2.127384in}}{\pgfqpoint{1.625854in}{2.121560in}}%
\pgfpathcurveto{\pgfqpoint{1.631678in}{2.115736in}}{\pgfqpoint{1.639578in}{2.112464in}}{\pgfqpoint{1.647814in}{2.112464in}}%
\pgfpathclose%
\pgfusepath{stroke,fill}%
\end{pgfscope}%
\begin{pgfscope}%
\pgfpathrectangle{\pgfqpoint{0.100000in}{0.212622in}}{\pgfqpoint{3.696000in}{3.696000in}}%
\pgfusepath{clip}%
\pgfsetbuttcap%
\pgfsetroundjoin%
\definecolor{currentfill}{rgb}{0.121569,0.466667,0.705882}%
\pgfsetfillcolor{currentfill}%
\pgfsetfillopacity{0.300651}%
\pgfsetlinewidth{1.003750pt}%
\definecolor{currentstroke}{rgb}{0.121569,0.466667,0.705882}%
\pgfsetstrokecolor{currentstroke}%
\pgfsetstrokeopacity{0.300651}%
\pgfsetdash{}{0pt}%
\pgfpathmoveto{\pgfqpoint{1.647814in}{2.112464in}}%
\pgfpathcurveto{\pgfqpoint{1.656050in}{2.112464in}}{\pgfqpoint{1.663950in}{2.115736in}}{\pgfqpoint{1.669774in}{2.121560in}}%
\pgfpathcurveto{\pgfqpoint{1.675598in}{2.127384in}}{\pgfqpoint{1.678871in}{2.135284in}}{\pgfqpoint{1.678871in}{2.143520in}}%
\pgfpathcurveto{\pgfqpoint{1.678871in}{2.151757in}}{\pgfqpoint{1.675598in}{2.159657in}}{\pgfqpoint{1.669774in}{2.165481in}}%
\pgfpathcurveto{\pgfqpoint{1.663950in}{2.171305in}}{\pgfqpoint{1.656050in}{2.174577in}}{\pgfqpoint{1.647814in}{2.174577in}}%
\pgfpathcurveto{\pgfqpoint{1.639578in}{2.174577in}}{\pgfqpoint{1.631678in}{2.171305in}}{\pgfqpoint{1.625854in}{2.165481in}}%
\pgfpathcurveto{\pgfqpoint{1.620030in}{2.159657in}}{\pgfqpoint{1.616758in}{2.151757in}}{\pgfqpoint{1.616758in}{2.143520in}}%
\pgfpathcurveto{\pgfqpoint{1.616758in}{2.135284in}}{\pgfqpoint{1.620030in}{2.127384in}}{\pgfqpoint{1.625854in}{2.121560in}}%
\pgfpathcurveto{\pgfqpoint{1.631678in}{2.115736in}}{\pgfqpoint{1.639578in}{2.112464in}}{\pgfqpoint{1.647814in}{2.112464in}}%
\pgfpathclose%
\pgfusepath{stroke,fill}%
\end{pgfscope}%
\begin{pgfscope}%
\pgfpathrectangle{\pgfqpoint{0.100000in}{0.212622in}}{\pgfqpoint{3.696000in}{3.696000in}}%
\pgfusepath{clip}%
\pgfsetbuttcap%
\pgfsetroundjoin%
\definecolor{currentfill}{rgb}{0.121569,0.466667,0.705882}%
\pgfsetfillcolor{currentfill}%
\pgfsetfillopacity{0.300651}%
\pgfsetlinewidth{1.003750pt}%
\definecolor{currentstroke}{rgb}{0.121569,0.466667,0.705882}%
\pgfsetstrokecolor{currentstroke}%
\pgfsetstrokeopacity{0.300651}%
\pgfsetdash{}{0pt}%
\pgfpathmoveto{\pgfqpoint{1.647814in}{2.112464in}}%
\pgfpathcurveto{\pgfqpoint{1.656050in}{2.112464in}}{\pgfqpoint{1.663950in}{2.115736in}}{\pgfqpoint{1.669774in}{2.121560in}}%
\pgfpathcurveto{\pgfqpoint{1.675598in}{2.127384in}}{\pgfqpoint{1.678871in}{2.135284in}}{\pgfqpoint{1.678871in}{2.143520in}}%
\pgfpathcurveto{\pgfqpoint{1.678871in}{2.151757in}}{\pgfqpoint{1.675598in}{2.159657in}}{\pgfqpoint{1.669774in}{2.165481in}}%
\pgfpathcurveto{\pgfqpoint{1.663950in}{2.171305in}}{\pgfqpoint{1.656050in}{2.174577in}}{\pgfqpoint{1.647814in}{2.174577in}}%
\pgfpathcurveto{\pgfqpoint{1.639578in}{2.174577in}}{\pgfqpoint{1.631678in}{2.171305in}}{\pgfqpoint{1.625854in}{2.165481in}}%
\pgfpathcurveto{\pgfqpoint{1.620030in}{2.159657in}}{\pgfqpoint{1.616758in}{2.151757in}}{\pgfqpoint{1.616758in}{2.143520in}}%
\pgfpathcurveto{\pgfqpoint{1.616758in}{2.135284in}}{\pgfqpoint{1.620030in}{2.127384in}}{\pgfqpoint{1.625854in}{2.121560in}}%
\pgfpathcurveto{\pgfqpoint{1.631678in}{2.115736in}}{\pgfqpoint{1.639578in}{2.112464in}}{\pgfqpoint{1.647814in}{2.112464in}}%
\pgfpathclose%
\pgfusepath{stroke,fill}%
\end{pgfscope}%
\begin{pgfscope}%
\pgfpathrectangle{\pgfqpoint{0.100000in}{0.212622in}}{\pgfqpoint{3.696000in}{3.696000in}}%
\pgfusepath{clip}%
\pgfsetbuttcap%
\pgfsetroundjoin%
\definecolor{currentfill}{rgb}{0.121569,0.466667,0.705882}%
\pgfsetfillcolor{currentfill}%
\pgfsetfillopacity{0.300651}%
\pgfsetlinewidth{1.003750pt}%
\definecolor{currentstroke}{rgb}{0.121569,0.466667,0.705882}%
\pgfsetstrokecolor{currentstroke}%
\pgfsetstrokeopacity{0.300651}%
\pgfsetdash{}{0pt}%
\pgfpathmoveto{\pgfqpoint{1.647814in}{2.112464in}}%
\pgfpathcurveto{\pgfqpoint{1.656050in}{2.112464in}}{\pgfqpoint{1.663950in}{2.115736in}}{\pgfqpoint{1.669774in}{2.121560in}}%
\pgfpathcurveto{\pgfqpoint{1.675598in}{2.127384in}}{\pgfqpoint{1.678871in}{2.135284in}}{\pgfqpoint{1.678871in}{2.143520in}}%
\pgfpathcurveto{\pgfqpoint{1.678871in}{2.151757in}}{\pgfqpoint{1.675598in}{2.159657in}}{\pgfqpoint{1.669774in}{2.165481in}}%
\pgfpathcurveto{\pgfqpoint{1.663950in}{2.171305in}}{\pgfqpoint{1.656050in}{2.174577in}}{\pgfqpoint{1.647814in}{2.174577in}}%
\pgfpathcurveto{\pgfqpoint{1.639578in}{2.174577in}}{\pgfqpoint{1.631678in}{2.171305in}}{\pgfqpoint{1.625854in}{2.165481in}}%
\pgfpathcurveto{\pgfqpoint{1.620030in}{2.159657in}}{\pgfqpoint{1.616758in}{2.151757in}}{\pgfqpoint{1.616758in}{2.143520in}}%
\pgfpathcurveto{\pgfqpoint{1.616758in}{2.135284in}}{\pgfqpoint{1.620030in}{2.127384in}}{\pgfqpoint{1.625854in}{2.121560in}}%
\pgfpathcurveto{\pgfqpoint{1.631678in}{2.115736in}}{\pgfqpoint{1.639578in}{2.112464in}}{\pgfqpoint{1.647814in}{2.112464in}}%
\pgfpathclose%
\pgfusepath{stroke,fill}%
\end{pgfscope}%
\begin{pgfscope}%
\pgfpathrectangle{\pgfqpoint{0.100000in}{0.212622in}}{\pgfqpoint{3.696000in}{3.696000in}}%
\pgfusepath{clip}%
\pgfsetbuttcap%
\pgfsetroundjoin%
\definecolor{currentfill}{rgb}{0.121569,0.466667,0.705882}%
\pgfsetfillcolor{currentfill}%
\pgfsetfillopacity{0.300651}%
\pgfsetlinewidth{1.003750pt}%
\definecolor{currentstroke}{rgb}{0.121569,0.466667,0.705882}%
\pgfsetstrokecolor{currentstroke}%
\pgfsetstrokeopacity{0.300651}%
\pgfsetdash{}{0pt}%
\pgfpathmoveto{\pgfqpoint{1.647814in}{2.112464in}}%
\pgfpathcurveto{\pgfqpoint{1.656050in}{2.112464in}}{\pgfqpoint{1.663950in}{2.115736in}}{\pgfqpoint{1.669774in}{2.121560in}}%
\pgfpathcurveto{\pgfqpoint{1.675598in}{2.127384in}}{\pgfqpoint{1.678871in}{2.135284in}}{\pgfqpoint{1.678871in}{2.143520in}}%
\pgfpathcurveto{\pgfqpoint{1.678871in}{2.151757in}}{\pgfqpoint{1.675598in}{2.159657in}}{\pgfqpoint{1.669774in}{2.165481in}}%
\pgfpathcurveto{\pgfqpoint{1.663950in}{2.171305in}}{\pgfqpoint{1.656050in}{2.174577in}}{\pgfqpoint{1.647814in}{2.174577in}}%
\pgfpathcurveto{\pgfqpoint{1.639578in}{2.174577in}}{\pgfqpoint{1.631678in}{2.171305in}}{\pgfqpoint{1.625854in}{2.165481in}}%
\pgfpathcurveto{\pgfqpoint{1.620030in}{2.159657in}}{\pgfqpoint{1.616758in}{2.151757in}}{\pgfqpoint{1.616758in}{2.143520in}}%
\pgfpathcurveto{\pgfqpoint{1.616758in}{2.135284in}}{\pgfqpoint{1.620030in}{2.127384in}}{\pgfqpoint{1.625854in}{2.121560in}}%
\pgfpathcurveto{\pgfqpoint{1.631678in}{2.115736in}}{\pgfqpoint{1.639578in}{2.112464in}}{\pgfqpoint{1.647814in}{2.112464in}}%
\pgfpathclose%
\pgfusepath{stroke,fill}%
\end{pgfscope}%
\begin{pgfscope}%
\pgfpathrectangle{\pgfqpoint{0.100000in}{0.212622in}}{\pgfqpoint{3.696000in}{3.696000in}}%
\pgfusepath{clip}%
\pgfsetbuttcap%
\pgfsetroundjoin%
\definecolor{currentfill}{rgb}{0.121569,0.466667,0.705882}%
\pgfsetfillcolor{currentfill}%
\pgfsetfillopacity{0.300651}%
\pgfsetlinewidth{1.003750pt}%
\definecolor{currentstroke}{rgb}{0.121569,0.466667,0.705882}%
\pgfsetstrokecolor{currentstroke}%
\pgfsetstrokeopacity{0.300651}%
\pgfsetdash{}{0pt}%
\pgfpathmoveto{\pgfqpoint{1.647814in}{2.112464in}}%
\pgfpathcurveto{\pgfqpoint{1.656050in}{2.112464in}}{\pgfqpoint{1.663950in}{2.115736in}}{\pgfqpoint{1.669774in}{2.121560in}}%
\pgfpathcurveto{\pgfqpoint{1.675598in}{2.127384in}}{\pgfqpoint{1.678871in}{2.135284in}}{\pgfqpoint{1.678871in}{2.143520in}}%
\pgfpathcurveto{\pgfqpoint{1.678871in}{2.151757in}}{\pgfqpoint{1.675598in}{2.159657in}}{\pgfqpoint{1.669774in}{2.165481in}}%
\pgfpathcurveto{\pgfqpoint{1.663950in}{2.171305in}}{\pgfqpoint{1.656050in}{2.174577in}}{\pgfqpoint{1.647814in}{2.174577in}}%
\pgfpathcurveto{\pgfqpoint{1.639578in}{2.174577in}}{\pgfqpoint{1.631678in}{2.171305in}}{\pgfqpoint{1.625854in}{2.165481in}}%
\pgfpathcurveto{\pgfqpoint{1.620030in}{2.159657in}}{\pgfqpoint{1.616758in}{2.151757in}}{\pgfqpoint{1.616758in}{2.143520in}}%
\pgfpathcurveto{\pgfqpoint{1.616758in}{2.135284in}}{\pgfqpoint{1.620030in}{2.127384in}}{\pgfqpoint{1.625854in}{2.121560in}}%
\pgfpathcurveto{\pgfqpoint{1.631678in}{2.115736in}}{\pgfqpoint{1.639578in}{2.112464in}}{\pgfqpoint{1.647814in}{2.112464in}}%
\pgfpathclose%
\pgfusepath{stroke,fill}%
\end{pgfscope}%
\begin{pgfscope}%
\pgfpathrectangle{\pgfqpoint{0.100000in}{0.212622in}}{\pgfqpoint{3.696000in}{3.696000in}}%
\pgfusepath{clip}%
\pgfsetbuttcap%
\pgfsetroundjoin%
\definecolor{currentfill}{rgb}{0.121569,0.466667,0.705882}%
\pgfsetfillcolor{currentfill}%
\pgfsetfillopacity{0.300651}%
\pgfsetlinewidth{1.003750pt}%
\definecolor{currentstroke}{rgb}{0.121569,0.466667,0.705882}%
\pgfsetstrokecolor{currentstroke}%
\pgfsetstrokeopacity{0.300651}%
\pgfsetdash{}{0pt}%
\pgfpathmoveto{\pgfqpoint{1.647814in}{2.112464in}}%
\pgfpathcurveto{\pgfqpoint{1.656050in}{2.112464in}}{\pgfqpoint{1.663950in}{2.115736in}}{\pgfqpoint{1.669774in}{2.121560in}}%
\pgfpathcurveto{\pgfqpoint{1.675598in}{2.127384in}}{\pgfqpoint{1.678871in}{2.135284in}}{\pgfqpoint{1.678871in}{2.143520in}}%
\pgfpathcurveto{\pgfqpoint{1.678871in}{2.151757in}}{\pgfqpoint{1.675598in}{2.159657in}}{\pgfqpoint{1.669774in}{2.165481in}}%
\pgfpathcurveto{\pgfqpoint{1.663950in}{2.171305in}}{\pgfqpoint{1.656050in}{2.174577in}}{\pgfqpoint{1.647814in}{2.174577in}}%
\pgfpathcurveto{\pgfqpoint{1.639578in}{2.174577in}}{\pgfqpoint{1.631678in}{2.171305in}}{\pgfqpoint{1.625854in}{2.165481in}}%
\pgfpathcurveto{\pgfqpoint{1.620030in}{2.159657in}}{\pgfqpoint{1.616758in}{2.151757in}}{\pgfqpoint{1.616758in}{2.143520in}}%
\pgfpathcurveto{\pgfqpoint{1.616758in}{2.135284in}}{\pgfqpoint{1.620030in}{2.127384in}}{\pgfqpoint{1.625854in}{2.121560in}}%
\pgfpathcurveto{\pgfqpoint{1.631678in}{2.115736in}}{\pgfqpoint{1.639578in}{2.112464in}}{\pgfqpoint{1.647814in}{2.112464in}}%
\pgfpathclose%
\pgfusepath{stroke,fill}%
\end{pgfscope}%
\begin{pgfscope}%
\pgfpathrectangle{\pgfqpoint{0.100000in}{0.212622in}}{\pgfqpoint{3.696000in}{3.696000in}}%
\pgfusepath{clip}%
\pgfsetbuttcap%
\pgfsetroundjoin%
\definecolor{currentfill}{rgb}{0.121569,0.466667,0.705882}%
\pgfsetfillcolor{currentfill}%
\pgfsetfillopacity{0.300651}%
\pgfsetlinewidth{1.003750pt}%
\definecolor{currentstroke}{rgb}{0.121569,0.466667,0.705882}%
\pgfsetstrokecolor{currentstroke}%
\pgfsetstrokeopacity{0.300651}%
\pgfsetdash{}{0pt}%
\pgfpathmoveto{\pgfqpoint{1.647814in}{2.112464in}}%
\pgfpathcurveto{\pgfqpoint{1.656050in}{2.112464in}}{\pgfqpoint{1.663950in}{2.115736in}}{\pgfqpoint{1.669774in}{2.121560in}}%
\pgfpathcurveto{\pgfqpoint{1.675598in}{2.127384in}}{\pgfqpoint{1.678871in}{2.135284in}}{\pgfqpoint{1.678871in}{2.143520in}}%
\pgfpathcurveto{\pgfqpoint{1.678871in}{2.151757in}}{\pgfqpoint{1.675598in}{2.159657in}}{\pgfqpoint{1.669774in}{2.165481in}}%
\pgfpathcurveto{\pgfqpoint{1.663950in}{2.171305in}}{\pgfqpoint{1.656050in}{2.174577in}}{\pgfqpoint{1.647814in}{2.174577in}}%
\pgfpathcurveto{\pgfqpoint{1.639578in}{2.174577in}}{\pgfqpoint{1.631678in}{2.171305in}}{\pgfqpoint{1.625854in}{2.165481in}}%
\pgfpathcurveto{\pgfqpoint{1.620030in}{2.159657in}}{\pgfqpoint{1.616758in}{2.151757in}}{\pgfqpoint{1.616758in}{2.143520in}}%
\pgfpathcurveto{\pgfqpoint{1.616758in}{2.135284in}}{\pgfqpoint{1.620030in}{2.127384in}}{\pgfqpoint{1.625854in}{2.121560in}}%
\pgfpathcurveto{\pgfqpoint{1.631678in}{2.115736in}}{\pgfqpoint{1.639578in}{2.112464in}}{\pgfqpoint{1.647814in}{2.112464in}}%
\pgfpathclose%
\pgfusepath{stroke,fill}%
\end{pgfscope}%
\begin{pgfscope}%
\pgfpathrectangle{\pgfqpoint{0.100000in}{0.212622in}}{\pgfqpoint{3.696000in}{3.696000in}}%
\pgfusepath{clip}%
\pgfsetbuttcap%
\pgfsetroundjoin%
\definecolor{currentfill}{rgb}{0.121569,0.466667,0.705882}%
\pgfsetfillcolor{currentfill}%
\pgfsetfillopacity{0.300651}%
\pgfsetlinewidth{1.003750pt}%
\definecolor{currentstroke}{rgb}{0.121569,0.466667,0.705882}%
\pgfsetstrokecolor{currentstroke}%
\pgfsetstrokeopacity{0.300651}%
\pgfsetdash{}{0pt}%
\pgfpathmoveto{\pgfqpoint{1.647814in}{2.112464in}}%
\pgfpathcurveto{\pgfqpoint{1.656050in}{2.112464in}}{\pgfqpoint{1.663950in}{2.115736in}}{\pgfqpoint{1.669774in}{2.121560in}}%
\pgfpathcurveto{\pgfqpoint{1.675598in}{2.127384in}}{\pgfqpoint{1.678871in}{2.135284in}}{\pgfqpoint{1.678871in}{2.143520in}}%
\pgfpathcurveto{\pgfqpoint{1.678871in}{2.151757in}}{\pgfqpoint{1.675598in}{2.159657in}}{\pgfqpoint{1.669774in}{2.165481in}}%
\pgfpathcurveto{\pgfqpoint{1.663950in}{2.171305in}}{\pgfqpoint{1.656050in}{2.174577in}}{\pgfqpoint{1.647814in}{2.174577in}}%
\pgfpathcurveto{\pgfqpoint{1.639578in}{2.174577in}}{\pgfqpoint{1.631678in}{2.171305in}}{\pgfqpoint{1.625854in}{2.165481in}}%
\pgfpathcurveto{\pgfqpoint{1.620030in}{2.159657in}}{\pgfqpoint{1.616758in}{2.151757in}}{\pgfqpoint{1.616758in}{2.143520in}}%
\pgfpathcurveto{\pgfqpoint{1.616758in}{2.135284in}}{\pgfqpoint{1.620030in}{2.127384in}}{\pgfqpoint{1.625854in}{2.121560in}}%
\pgfpathcurveto{\pgfqpoint{1.631678in}{2.115736in}}{\pgfqpoint{1.639578in}{2.112464in}}{\pgfqpoint{1.647814in}{2.112464in}}%
\pgfpathclose%
\pgfusepath{stroke,fill}%
\end{pgfscope}%
\begin{pgfscope}%
\pgfpathrectangle{\pgfqpoint{0.100000in}{0.212622in}}{\pgfqpoint{3.696000in}{3.696000in}}%
\pgfusepath{clip}%
\pgfsetbuttcap%
\pgfsetroundjoin%
\definecolor{currentfill}{rgb}{0.121569,0.466667,0.705882}%
\pgfsetfillcolor{currentfill}%
\pgfsetfillopacity{0.300651}%
\pgfsetlinewidth{1.003750pt}%
\definecolor{currentstroke}{rgb}{0.121569,0.466667,0.705882}%
\pgfsetstrokecolor{currentstroke}%
\pgfsetstrokeopacity{0.300651}%
\pgfsetdash{}{0pt}%
\pgfpathmoveto{\pgfqpoint{1.647814in}{2.112464in}}%
\pgfpathcurveto{\pgfqpoint{1.656050in}{2.112464in}}{\pgfqpoint{1.663950in}{2.115736in}}{\pgfqpoint{1.669774in}{2.121560in}}%
\pgfpathcurveto{\pgfqpoint{1.675598in}{2.127384in}}{\pgfqpoint{1.678871in}{2.135284in}}{\pgfqpoint{1.678871in}{2.143520in}}%
\pgfpathcurveto{\pgfqpoint{1.678871in}{2.151757in}}{\pgfqpoint{1.675598in}{2.159657in}}{\pgfqpoint{1.669774in}{2.165481in}}%
\pgfpathcurveto{\pgfqpoint{1.663950in}{2.171305in}}{\pgfqpoint{1.656050in}{2.174577in}}{\pgfqpoint{1.647814in}{2.174577in}}%
\pgfpathcurveto{\pgfqpoint{1.639578in}{2.174577in}}{\pgfqpoint{1.631678in}{2.171305in}}{\pgfqpoint{1.625854in}{2.165481in}}%
\pgfpathcurveto{\pgfqpoint{1.620030in}{2.159657in}}{\pgfqpoint{1.616758in}{2.151757in}}{\pgfqpoint{1.616758in}{2.143520in}}%
\pgfpathcurveto{\pgfqpoint{1.616758in}{2.135284in}}{\pgfqpoint{1.620030in}{2.127384in}}{\pgfqpoint{1.625854in}{2.121560in}}%
\pgfpathcurveto{\pgfqpoint{1.631678in}{2.115736in}}{\pgfqpoint{1.639578in}{2.112464in}}{\pgfqpoint{1.647814in}{2.112464in}}%
\pgfpathclose%
\pgfusepath{stroke,fill}%
\end{pgfscope}%
\begin{pgfscope}%
\pgfpathrectangle{\pgfqpoint{0.100000in}{0.212622in}}{\pgfqpoint{3.696000in}{3.696000in}}%
\pgfusepath{clip}%
\pgfsetbuttcap%
\pgfsetroundjoin%
\definecolor{currentfill}{rgb}{0.121569,0.466667,0.705882}%
\pgfsetfillcolor{currentfill}%
\pgfsetfillopacity{0.300651}%
\pgfsetlinewidth{1.003750pt}%
\definecolor{currentstroke}{rgb}{0.121569,0.466667,0.705882}%
\pgfsetstrokecolor{currentstroke}%
\pgfsetstrokeopacity{0.300651}%
\pgfsetdash{}{0pt}%
\pgfpathmoveto{\pgfqpoint{1.647814in}{2.112464in}}%
\pgfpathcurveto{\pgfqpoint{1.656050in}{2.112464in}}{\pgfqpoint{1.663950in}{2.115736in}}{\pgfqpoint{1.669774in}{2.121560in}}%
\pgfpathcurveto{\pgfqpoint{1.675598in}{2.127384in}}{\pgfqpoint{1.678871in}{2.135284in}}{\pgfqpoint{1.678871in}{2.143520in}}%
\pgfpathcurveto{\pgfqpoint{1.678871in}{2.151757in}}{\pgfqpoint{1.675598in}{2.159657in}}{\pgfqpoint{1.669774in}{2.165481in}}%
\pgfpathcurveto{\pgfqpoint{1.663950in}{2.171305in}}{\pgfqpoint{1.656050in}{2.174577in}}{\pgfqpoint{1.647814in}{2.174577in}}%
\pgfpathcurveto{\pgfqpoint{1.639578in}{2.174577in}}{\pgfqpoint{1.631678in}{2.171305in}}{\pgfqpoint{1.625854in}{2.165481in}}%
\pgfpathcurveto{\pgfqpoint{1.620030in}{2.159657in}}{\pgfqpoint{1.616758in}{2.151757in}}{\pgfqpoint{1.616758in}{2.143520in}}%
\pgfpathcurveto{\pgfqpoint{1.616758in}{2.135284in}}{\pgfqpoint{1.620030in}{2.127384in}}{\pgfqpoint{1.625854in}{2.121560in}}%
\pgfpathcurveto{\pgfqpoint{1.631678in}{2.115736in}}{\pgfqpoint{1.639578in}{2.112464in}}{\pgfqpoint{1.647814in}{2.112464in}}%
\pgfpathclose%
\pgfusepath{stroke,fill}%
\end{pgfscope}%
\begin{pgfscope}%
\pgfpathrectangle{\pgfqpoint{0.100000in}{0.212622in}}{\pgfqpoint{3.696000in}{3.696000in}}%
\pgfusepath{clip}%
\pgfsetbuttcap%
\pgfsetroundjoin%
\definecolor{currentfill}{rgb}{0.121569,0.466667,0.705882}%
\pgfsetfillcolor{currentfill}%
\pgfsetfillopacity{0.300651}%
\pgfsetlinewidth{1.003750pt}%
\definecolor{currentstroke}{rgb}{0.121569,0.466667,0.705882}%
\pgfsetstrokecolor{currentstroke}%
\pgfsetstrokeopacity{0.300651}%
\pgfsetdash{}{0pt}%
\pgfpathmoveto{\pgfqpoint{1.647814in}{2.112464in}}%
\pgfpathcurveto{\pgfqpoint{1.656050in}{2.112464in}}{\pgfqpoint{1.663950in}{2.115736in}}{\pgfqpoint{1.669774in}{2.121560in}}%
\pgfpathcurveto{\pgfqpoint{1.675598in}{2.127384in}}{\pgfqpoint{1.678871in}{2.135284in}}{\pgfqpoint{1.678871in}{2.143520in}}%
\pgfpathcurveto{\pgfqpoint{1.678871in}{2.151757in}}{\pgfqpoint{1.675598in}{2.159657in}}{\pgfqpoint{1.669774in}{2.165481in}}%
\pgfpathcurveto{\pgfqpoint{1.663950in}{2.171305in}}{\pgfqpoint{1.656050in}{2.174577in}}{\pgfqpoint{1.647814in}{2.174577in}}%
\pgfpathcurveto{\pgfqpoint{1.639578in}{2.174577in}}{\pgfqpoint{1.631678in}{2.171305in}}{\pgfqpoint{1.625854in}{2.165481in}}%
\pgfpathcurveto{\pgfqpoint{1.620030in}{2.159657in}}{\pgfqpoint{1.616758in}{2.151757in}}{\pgfqpoint{1.616758in}{2.143520in}}%
\pgfpathcurveto{\pgfqpoint{1.616758in}{2.135284in}}{\pgfqpoint{1.620030in}{2.127384in}}{\pgfqpoint{1.625854in}{2.121560in}}%
\pgfpathcurveto{\pgfqpoint{1.631678in}{2.115736in}}{\pgfqpoint{1.639578in}{2.112464in}}{\pgfqpoint{1.647814in}{2.112464in}}%
\pgfpathclose%
\pgfusepath{stroke,fill}%
\end{pgfscope}%
\begin{pgfscope}%
\pgfpathrectangle{\pgfqpoint{0.100000in}{0.212622in}}{\pgfqpoint{3.696000in}{3.696000in}}%
\pgfusepath{clip}%
\pgfsetbuttcap%
\pgfsetroundjoin%
\definecolor{currentfill}{rgb}{0.121569,0.466667,0.705882}%
\pgfsetfillcolor{currentfill}%
\pgfsetfillopacity{0.300651}%
\pgfsetlinewidth{1.003750pt}%
\definecolor{currentstroke}{rgb}{0.121569,0.466667,0.705882}%
\pgfsetstrokecolor{currentstroke}%
\pgfsetstrokeopacity{0.300651}%
\pgfsetdash{}{0pt}%
\pgfpathmoveto{\pgfqpoint{1.647814in}{2.112464in}}%
\pgfpathcurveto{\pgfqpoint{1.656050in}{2.112464in}}{\pgfqpoint{1.663950in}{2.115736in}}{\pgfqpoint{1.669774in}{2.121560in}}%
\pgfpathcurveto{\pgfqpoint{1.675598in}{2.127384in}}{\pgfqpoint{1.678871in}{2.135284in}}{\pgfqpoint{1.678871in}{2.143520in}}%
\pgfpathcurveto{\pgfqpoint{1.678871in}{2.151757in}}{\pgfqpoint{1.675598in}{2.159657in}}{\pgfqpoint{1.669774in}{2.165481in}}%
\pgfpathcurveto{\pgfqpoint{1.663950in}{2.171305in}}{\pgfqpoint{1.656050in}{2.174577in}}{\pgfqpoint{1.647814in}{2.174577in}}%
\pgfpathcurveto{\pgfqpoint{1.639578in}{2.174577in}}{\pgfqpoint{1.631678in}{2.171305in}}{\pgfqpoint{1.625854in}{2.165481in}}%
\pgfpathcurveto{\pgfqpoint{1.620030in}{2.159657in}}{\pgfqpoint{1.616758in}{2.151757in}}{\pgfqpoint{1.616758in}{2.143520in}}%
\pgfpathcurveto{\pgfqpoint{1.616758in}{2.135284in}}{\pgfqpoint{1.620030in}{2.127384in}}{\pgfqpoint{1.625854in}{2.121560in}}%
\pgfpathcurveto{\pgfqpoint{1.631678in}{2.115736in}}{\pgfqpoint{1.639578in}{2.112464in}}{\pgfqpoint{1.647814in}{2.112464in}}%
\pgfpathclose%
\pgfusepath{stroke,fill}%
\end{pgfscope}%
\begin{pgfscope}%
\pgfpathrectangle{\pgfqpoint{0.100000in}{0.212622in}}{\pgfqpoint{3.696000in}{3.696000in}}%
\pgfusepath{clip}%
\pgfsetbuttcap%
\pgfsetroundjoin%
\definecolor{currentfill}{rgb}{0.121569,0.466667,0.705882}%
\pgfsetfillcolor{currentfill}%
\pgfsetfillopacity{0.300651}%
\pgfsetlinewidth{1.003750pt}%
\definecolor{currentstroke}{rgb}{0.121569,0.466667,0.705882}%
\pgfsetstrokecolor{currentstroke}%
\pgfsetstrokeopacity{0.300651}%
\pgfsetdash{}{0pt}%
\pgfpathmoveto{\pgfqpoint{1.647814in}{2.112464in}}%
\pgfpathcurveto{\pgfqpoint{1.656050in}{2.112464in}}{\pgfqpoint{1.663950in}{2.115736in}}{\pgfqpoint{1.669774in}{2.121560in}}%
\pgfpathcurveto{\pgfqpoint{1.675598in}{2.127384in}}{\pgfqpoint{1.678871in}{2.135284in}}{\pgfqpoint{1.678871in}{2.143520in}}%
\pgfpathcurveto{\pgfqpoint{1.678871in}{2.151757in}}{\pgfqpoint{1.675598in}{2.159657in}}{\pgfqpoint{1.669774in}{2.165481in}}%
\pgfpathcurveto{\pgfqpoint{1.663950in}{2.171305in}}{\pgfqpoint{1.656050in}{2.174577in}}{\pgfqpoint{1.647814in}{2.174577in}}%
\pgfpathcurveto{\pgfqpoint{1.639578in}{2.174577in}}{\pgfqpoint{1.631678in}{2.171305in}}{\pgfqpoint{1.625854in}{2.165481in}}%
\pgfpathcurveto{\pgfqpoint{1.620030in}{2.159657in}}{\pgfqpoint{1.616758in}{2.151757in}}{\pgfqpoint{1.616758in}{2.143520in}}%
\pgfpathcurveto{\pgfqpoint{1.616758in}{2.135284in}}{\pgfqpoint{1.620030in}{2.127384in}}{\pgfqpoint{1.625854in}{2.121560in}}%
\pgfpathcurveto{\pgfqpoint{1.631678in}{2.115736in}}{\pgfqpoint{1.639578in}{2.112464in}}{\pgfqpoint{1.647814in}{2.112464in}}%
\pgfpathclose%
\pgfusepath{stroke,fill}%
\end{pgfscope}%
\begin{pgfscope}%
\pgfpathrectangle{\pgfqpoint{0.100000in}{0.212622in}}{\pgfqpoint{3.696000in}{3.696000in}}%
\pgfusepath{clip}%
\pgfsetbuttcap%
\pgfsetroundjoin%
\definecolor{currentfill}{rgb}{0.121569,0.466667,0.705882}%
\pgfsetfillcolor{currentfill}%
\pgfsetfillopacity{0.300651}%
\pgfsetlinewidth{1.003750pt}%
\definecolor{currentstroke}{rgb}{0.121569,0.466667,0.705882}%
\pgfsetstrokecolor{currentstroke}%
\pgfsetstrokeopacity{0.300651}%
\pgfsetdash{}{0pt}%
\pgfpathmoveto{\pgfqpoint{1.647814in}{2.112464in}}%
\pgfpathcurveto{\pgfqpoint{1.656050in}{2.112464in}}{\pgfqpoint{1.663950in}{2.115736in}}{\pgfqpoint{1.669774in}{2.121560in}}%
\pgfpathcurveto{\pgfqpoint{1.675598in}{2.127384in}}{\pgfqpoint{1.678871in}{2.135284in}}{\pgfqpoint{1.678871in}{2.143520in}}%
\pgfpathcurveto{\pgfqpoint{1.678871in}{2.151757in}}{\pgfqpoint{1.675598in}{2.159657in}}{\pgfqpoint{1.669774in}{2.165481in}}%
\pgfpathcurveto{\pgfqpoint{1.663950in}{2.171305in}}{\pgfqpoint{1.656050in}{2.174577in}}{\pgfqpoint{1.647814in}{2.174577in}}%
\pgfpathcurveto{\pgfqpoint{1.639578in}{2.174577in}}{\pgfqpoint{1.631678in}{2.171305in}}{\pgfqpoint{1.625854in}{2.165481in}}%
\pgfpathcurveto{\pgfqpoint{1.620030in}{2.159657in}}{\pgfqpoint{1.616758in}{2.151757in}}{\pgfqpoint{1.616758in}{2.143520in}}%
\pgfpathcurveto{\pgfqpoint{1.616758in}{2.135284in}}{\pgfqpoint{1.620030in}{2.127384in}}{\pgfqpoint{1.625854in}{2.121560in}}%
\pgfpathcurveto{\pgfqpoint{1.631678in}{2.115736in}}{\pgfqpoint{1.639578in}{2.112464in}}{\pgfqpoint{1.647814in}{2.112464in}}%
\pgfpathclose%
\pgfusepath{stroke,fill}%
\end{pgfscope}%
\begin{pgfscope}%
\pgfpathrectangle{\pgfqpoint{0.100000in}{0.212622in}}{\pgfqpoint{3.696000in}{3.696000in}}%
\pgfusepath{clip}%
\pgfsetbuttcap%
\pgfsetroundjoin%
\definecolor{currentfill}{rgb}{0.121569,0.466667,0.705882}%
\pgfsetfillcolor{currentfill}%
\pgfsetfillopacity{0.300651}%
\pgfsetlinewidth{1.003750pt}%
\definecolor{currentstroke}{rgb}{0.121569,0.466667,0.705882}%
\pgfsetstrokecolor{currentstroke}%
\pgfsetstrokeopacity{0.300651}%
\pgfsetdash{}{0pt}%
\pgfpathmoveto{\pgfqpoint{1.647814in}{2.112464in}}%
\pgfpathcurveto{\pgfqpoint{1.656050in}{2.112464in}}{\pgfqpoint{1.663950in}{2.115736in}}{\pgfqpoint{1.669774in}{2.121560in}}%
\pgfpathcurveto{\pgfqpoint{1.675598in}{2.127384in}}{\pgfqpoint{1.678871in}{2.135284in}}{\pgfqpoint{1.678871in}{2.143520in}}%
\pgfpathcurveto{\pgfqpoint{1.678871in}{2.151757in}}{\pgfqpoint{1.675598in}{2.159657in}}{\pgfqpoint{1.669774in}{2.165481in}}%
\pgfpathcurveto{\pgfqpoint{1.663950in}{2.171305in}}{\pgfqpoint{1.656050in}{2.174577in}}{\pgfqpoint{1.647814in}{2.174577in}}%
\pgfpathcurveto{\pgfqpoint{1.639578in}{2.174577in}}{\pgfqpoint{1.631678in}{2.171305in}}{\pgfqpoint{1.625854in}{2.165481in}}%
\pgfpathcurveto{\pgfqpoint{1.620030in}{2.159657in}}{\pgfqpoint{1.616758in}{2.151757in}}{\pgfqpoint{1.616758in}{2.143520in}}%
\pgfpathcurveto{\pgfqpoint{1.616758in}{2.135284in}}{\pgfqpoint{1.620030in}{2.127384in}}{\pgfqpoint{1.625854in}{2.121560in}}%
\pgfpathcurveto{\pgfqpoint{1.631678in}{2.115736in}}{\pgfqpoint{1.639578in}{2.112464in}}{\pgfqpoint{1.647814in}{2.112464in}}%
\pgfpathclose%
\pgfusepath{stroke,fill}%
\end{pgfscope}%
\begin{pgfscope}%
\pgfpathrectangle{\pgfqpoint{0.100000in}{0.212622in}}{\pgfqpoint{3.696000in}{3.696000in}}%
\pgfusepath{clip}%
\pgfsetbuttcap%
\pgfsetroundjoin%
\definecolor{currentfill}{rgb}{0.121569,0.466667,0.705882}%
\pgfsetfillcolor{currentfill}%
\pgfsetfillopacity{0.300651}%
\pgfsetlinewidth{1.003750pt}%
\definecolor{currentstroke}{rgb}{0.121569,0.466667,0.705882}%
\pgfsetstrokecolor{currentstroke}%
\pgfsetstrokeopacity{0.300651}%
\pgfsetdash{}{0pt}%
\pgfpathmoveto{\pgfqpoint{1.647814in}{2.112464in}}%
\pgfpathcurveto{\pgfqpoint{1.656050in}{2.112464in}}{\pgfqpoint{1.663950in}{2.115736in}}{\pgfqpoint{1.669774in}{2.121560in}}%
\pgfpathcurveto{\pgfqpoint{1.675598in}{2.127384in}}{\pgfqpoint{1.678871in}{2.135284in}}{\pgfqpoint{1.678871in}{2.143520in}}%
\pgfpathcurveto{\pgfqpoint{1.678871in}{2.151757in}}{\pgfqpoint{1.675598in}{2.159657in}}{\pgfqpoint{1.669774in}{2.165481in}}%
\pgfpathcurveto{\pgfqpoint{1.663950in}{2.171305in}}{\pgfqpoint{1.656050in}{2.174577in}}{\pgfqpoint{1.647814in}{2.174577in}}%
\pgfpathcurveto{\pgfqpoint{1.639578in}{2.174577in}}{\pgfqpoint{1.631678in}{2.171305in}}{\pgfqpoint{1.625854in}{2.165481in}}%
\pgfpathcurveto{\pgfqpoint{1.620030in}{2.159657in}}{\pgfqpoint{1.616758in}{2.151757in}}{\pgfqpoint{1.616758in}{2.143520in}}%
\pgfpathcurveto{\pgfqpoint{1.616758in}{2.135284in}}{\pgfqpoint{1.620030in}{2.127384in}}{\pgfqpoint{1.625854in}{2.121560in}}%
\pgfpathcurveto{\pgfqpoint{1.631678in}{2.115736in}}{\pgfqpoint{1.639578in}{2.112464in}}{\pgfqpoint{1.647814in}{2.112464in}}%
\pgfpathclose%
\pgfusepath{stroke,fill}%
\end{pgfscope}%
\begin{pgfscope}%
\pgfpathrectangle{\pgfqpoint{0.100000in}{0.212622in}}{\pgfqpoint{3.696000in}{3.696000in}}%
\pgfusepath{clip}%
\pgfsetbuttcap%
\pgfsetroundjoin%
\definecolor{currentfill}{rgb}{0.121569,0.466667,0.705882}%
\pgfsetfillcolor{currentfill}%
\pgfsetfillopacity{0.300651}%
\pgfsetlinewidth{1.003750pt}%
\definecolor{currentstroke}{rgb}{0.121569,0.466667,0.705882}%
\pgfsetstrokecolor{currentstroke}%
\pgfsetstrokeopacity{0.300651}%
\pgfsetdash{}{0pt}%
\pgfpathmoveto{\pgfqpoint{1.647814in}{2.112464in}}%
\pgfpathcurveto{\pgfqpoint{1.656050in}{2.112464in}}{\pgfqpoint{1.663950in}{2.115736in}}{\pgfqpoint{1.669774in}{2.121560in}}%
\pgfpathcurveto{\pgfqpoint{1.675598in}{2.127384in}}{\pgfqpoint{1.678871in}{2.135284in}}{\pgfqpoint{1.678871in}{2.143520in}}%
\pgfpathcurveto{\pgfqpoint{1.678871in}{2.151757in}}{\pgfqpoint{1.675598in}{2.159657in}}{\pgfqpoint{1.669774in}{2.165481in}}%
\pgfpathcurveto{\pgfqpoint{1.663950in}{2.171305in}}{\pgfqpoint{1.656050in}{2.174577in}}{\pgfqpoint{1.647814in}{2.174577in}}%
\pgfpathcurveto{\pgfqpoint{1.639578in}{2.174577in}}{\pgfqpoint{1.631678in}{2.171305in}}{\pgfqpoint{1.625854in}{2.165481in}}%
\pgfpathcurveto{\pgfqpoint{1.620030in}{2.159657in}}{\pgfqpoint{1.616758in}{2.151757in}}{\pgfqpoint{1.616758in}{2.143520in}}%
\pgfpathcurveto{\pgfqpoint{1.616758in}{2.135284in}}{\pgfqpoint{1.620030in}{2.127384in}}{\pgfqpoint{1.625854in}{2.121560in}}%
\pgfpathcurveto{\pgfqpoint{1.631678in}{2.115736in}}{\pgfqpoint{1.639578in}{2.112464in}}{\pgfqpoint{1.647814in}{2.112464in}}%
\pgfpathclose%
\pgfusepath{stroke,fill}%
\end{pgfscope}%
\begin{pgfscope}%
\pgfpathrectangle{\pgfqpoint{0.100000in}{0.212622in}}{\pgfqpoint{3.696000in}{3.696000in}}%
\pgfusepath{clip}%
\pgfsetbuttcap%
\pgfsetroundjoin%
\definecolor{currentfill}{rgb}{0.121569,0.466667,0.705882}%
\pgfsetfillcolor{currentfill}%
\pgfsetfillopacity{0.300651}%
\pgfsetlinewidth{1.003750pt}%
\definecolor{currentstroke}{rgb}{0.121569,0.466667,0.705882}%
\pgfsetstrokecolor{currentstroke}%
\pgfsetstrokeopacity{0.300651}%
\pgfsetdash{}{0pt}%
\pgfpathmoveto{\pgfqpoint{1.647814in}{2.112464in}}%
\pgfpathcurveto{\pgfqpoint{1.656050in}{2.112464in}}{\pgfqpoint{1.663950in}{2.115736in}}{\pgfqpoint{1.669774in}{2.121560in}}%
\pgfpathcurveto{\pgfqpoint{1.675598in}{2.127384in}}{\pgfqpoint{1.678871in}{2.135284in}}{\pgfqpoint{1.678871in}{2.143520in}}%
\pgfpathcurveto{\pgfqpoint{1.678871in}{2.151757in}}{\pgfqpoint{1.675598in}{2.159657in}}{\pgfqpoint{1.669774in}{2.165481in}}%
\pgfpathcurveto{\pgfqpoint{1.663950in}{2.171305in}}{\pgfqpoint{1.656050in}{2.174577in}}{\pgfqpoint{1.647814in}{2.174577in}}%
\pgfpathcurveto{\pgfqpoint{1.639578in}{2.174577in}}{\pgfqpoint{1.631678in}{2.171305in}}{\pgfqpoint{1.625854in}{2.165481in}}%
\pgfpathcurveto{\pgfqpoint{1.620030in}{2.159657in}}{\pgfqpoint{1.616758in}{2.151757in}}{\pgfqpoint{1.616758in}{2.143520in}}%
\pgfpathcurveto{\pgfqpoint{1.616758in}{2.135284in}}{\pgfqpoint{1.620030in}{2.127384in}}{\pgfqpoint{1.625854in}{2.121560in}}%
\pgfpathcurveto{\pgfqpoint{1.631678in}{2.115736in}}{\pgfqpoint{1.639578in}{2.112464in}}{\pgfqpoint{1.647814in}{2.112464in}}%
\pgfpathclose%
\pgfusepath{stroke,fill}%
\end{pgfscope}%
\begin{pgfscope}%
\pgfpathrectangle{\pgfqpoint{0.100000in}{0.212622in}}{\pgfqpoint{3.696000in}{3.696000in}}%
\pgfusepath{clip}%
\pgfsetbuttcap%
\pgfsetroundjoin%
\definecolor{currentfill}{rgb}{0.121569,0.466667,0.705882}%
\pgfsetfillcolor{currentfill}%
\pgfsetfillopacity{0.300651}%
\pgfsetlinewidth{1.003750pt}%
\definecolor{currentstroke}{rgb}{0.121569,0.466667,0.705882}%
\pgfsetstrokecolor{currentstroke}%
\pgfsetstrokeopacity{0.300651}%
\pgfsetdash{}{0pt}%
\pgfpathmoveto{\pgfqpoint{1.647814in}{2.112464in}}%
\pgfpathcurveto{\pgfqpoint{1.656050in}{2.112464in}}{\pgfqpoint{1.663950in}{2.115736in}}{\pgfqpoint{1.669774in}{2.121560in}}%
\pgfpathcurveto{\pgfqpoint{1.675598in}{2.127384in}}{\pgfqpoint{1.678871in}{2.135284in}}{\pgfqpoint{1.678871in}{2.143520in}}%
\pgfpathcurveto{\pgfqpoint{1.678871in}{2.151757in}}{\pgfqpoint{1.675598in}{2.159657in}}{\pgfqpoint{1.669774in}{2.165481in}}%
\pgfpathcurveto{\pgfqpoint{1.663950in}{2.171305in}}{\pgfqpoint{1.656050in}{2.174577in}}{\pgfqpoint{1.647814in}{2.174577in}}%
\pgfpathcurveto{\pgfqpoint{1.639578in}{2.174577in}}{\pgfqpoint{1.631678in}{2.171305in}}{\pgfqpoint{1.625854in}{2.165481in}}%
\pgfpathcurveto{\pgfqpoint{1.620030in}{2.159657in}}{\pgfqpoint{1.616758in}{2.151757in}}{\pgfqpoint{1.616758in}{2.143520in}}%
\pgfpathcurveto{\pgfqpoint{1.616758in}{2.135284in}}{\pgfqpoint{1.620030in}{2.127384in}}{\pgfqpoint{1.625854in}{2.121560in}}%
\pgfpathcurveto{\pgfqpoint{1.631678in}{2.115736in}}{\pgfqpoint{1.639578in}{2.112464in}}{\pgfqpoint{1.647814in}{2.112464in}}%
\pgfpathclose%
\pgfusepath{stroke,fill}%
\end{pgfscope}%
\begin{pgfscope}%
\pgfpathrectangle{\pgfqpoint{0.100000in}{0.212622in}}{\pgfqpoint{3.696000in}{3.696000in}}%
\pgfusepath{clip}%
\pgfsetbuttcap%
\pgfsetroundjoin%
\definecolor{currentfill}{rgb}{0.121569,0.466667,0.705882}%
\pgfsetfillcolor{currentfill}%
\pgfsetfillopacity{0.300651}%
\pgfsetlinewidth{1.003750pt}%
\definecolor{currentstroke}{rgb}{0.121569,0.466667,0.705882}%
\pgfsetstrokecolor{currentstroke}%
\pgfsetstrokeopacity{0.300651}%
\pgfsetdash{}{0pt}%
\pgfpathmoveto{\pgfqpoint{1.647814in}{2.112464in}}%
\pgfpathcurveto{\pgfqpoint{1.656050in}{2.112464in}}{\pgfqpoint{1.663950in}{2.115736in}}{\pgfqpoint{1.669774in}{2.121560in}}%
\pgfpathcurveto{\pgfqpoint{1.675598in}{2.127384in}}{\pgfqpoint{1.678871in}{2.135284in}}{\pgfqpoint{1.678871in}{2.143520in}}%
\pgfpathcurveto{\pgfqpoint{1.678871in}{2.151757in}}{\pgfqpoint{1.675598in}{2.159657in}}{\pgfqpoint{1.669774in}{2.165481in}}%
\pgfpathcurveto{\pgfqpoint{1.663950in}{2.171305in}}{\pgfqpoint{1.656050in}{2.174577in}}{\pgfqpoint{1.647814in}{2.174577in}}%
\pgfpathcurveto{\pgfqpoint{1.639578in}{2.174577in}}{\pgfqpoint{1.631678in}{2.171305in}}{\pgfqpoint{1.625854in}{2.165481in}}%
\pgfpathcurveto{\pgfqpoint{1.620030in}{2.159657in}}{\pgfqpoint{1.616758in}{2.151757in}}{\pgfqpoint{1.616758in}{2.143520in}}%
\pgfpathcurveto{\pgfqpoint{1.616758in}{2.135284in}}{\pgfqpoint{1.620030in}{2.127384in}}{\pgfqpoint{1.625854in}{2.121560in}}%
\pgfpathcurveto{\pgfqpoint{1.631678in}{2.115736in}}{\pgfqpoint{1.639578in}{2.112464in}}{\pgfqpoint{1.647814in}{2.112464in}}%
\pgfpathclose%
\pgfusepath{stroke,fill}%
\end{pgfscope}%
\begin{pgfscope}%
\pgfpathrectangle{\pgfqpoint{0.100000in}{0.212622in}}{\pgfqpoint{3.696000in}{3.696000in}}%
\pgfusepath{clip}%
\pgfsetbuttcap%
\pgfsetroundjoin%
\definecolor{currentfill}{rgb}{0.121569,0.466667,0.705882}%
\pgfsetfillcolor{currentfill}%
\pgfsetfillopacity{0.300651}%
\pgfsetlinewidth{1.003750pt}%
\definecolor{currentstroke}{rgb}{0.121569,0.466667,0.705882}%
\pgfsetstrokecolor{currentstroke}%
\pgfsetstrokeopacity{0.300651}%
\pgfsetdash{}{0pt}%
\pgfpathmoveto{\pgfqpoint{1.647814in}{2.112464in}}%
\pgfpathcurveto{\pgfqpoint{1.656050in}{2.112464in}}{\pgfqpoint{1.663950in}{2.115736in}}{\pgfqpoint{1.669774in}{2.121560in}}%
\pgfpathcurveto{\pgfqpoint{1.675598in}{2.127384in}}{\pgfqpoint{1.678871in}{2.135284in}}{\pgfqpoint{1.678871in}{2.143520in}}%
\pgfpathcurveto{\pgfqpoint{1.678871in}{2.151757in}}{\pgfqpoint{1.675598in}{2.159657in}}{\pgfqpoint{1.669774in}{2.165481in}}%
\pgfpathcurveto{\pgfqpoint{1.663950in}{2.171305in}}{\pgfqpoint{1.656050in}{2.174577in}}{\pgfqpoint{1.647814in}{2.174577in}}%
\pgfpathcurveto{\pgfqpoint{1.639578in}{2.174577in}}{\pgfqpoint{1.631678in}{2.171305in}}{\pgfqpoint{1.625854in}{2.165481in}}%
\pgfpathcurveto{\pgfqpoint{1.620030in}{2.159657in}}{\pgfqpoint{1.616758in}{2.151757in}}{\pgfqpoint{1.616758in}{2.143520in}}%
\pgfpathcurveto{\pgfqpoint{1.616758in}{2.135284in}}{\pgfqpoint{1.620030in}{2.127384in}}{\pgfqpoint{1.625854in}{2.121560in}}%
\pgfpathcurveto{\pgfqpoint{1.631678in}{2.115736in}}{\pgfqpoint{1.639578in}{2.112464in}}{\pgfqpoint{1.647814in}{2.112464in}}%
\pgfpathclose%
\pgfusepath{stroke,fill}%
\end{pgfscope}%
\begin{pgfscope}%
\pgfpathrectangle{\pgfqpoint{0.100000in}{0.212622in}}{\pgfqpoint{3.696000in}{3.696000in}}%
\pgfusepath{clip}%
\pgfsetbuttcap%
\pgfsetroundjoin%
\definecolor{currentfill}{rgb}{0.121569,0.466667,0.705882}%
\pgfsetfillcolor{currentfill}%
\pgfsetfillopacity{0.300651}%
\pgfsetlinewidth{1.003750pt}%
\definecolor{currentstroke}{rgb}{0.121569,0.466667,0.705882}%
\pgfsetstrokecolor{currentstroke}%
\pgfsetstrokeopacity{0.300651}%
\pgfsetdash{}{0pt}%
\pgfpathmoveto{\pgfqpoint{1.647814in}{2.112464in}}%
\pgfpathcurveto{\pgfqpoint{1.656050in}{2.112464in}}{\pgfqpoint{1.663950in}{2.115736in}}{\pgfqpoint{1.669774in}{2.121560in}}%
\pgfpathcurveto{\pgfqpoint{1.675598in}{2.127384in}}{\pgfqpoint{1.678871in}{2.135284in}}{\pgfqpoint{1.678871in}{2.143520in}}%
\pgfpathcurveto{\pgfqpoint{1.678871in}{2.151757in}}{\pgfqpoint{1.675598in}{2.159657in}}{\pgfqpoint{1.669774in}{2.165481in}}%
\pgfpathcurveto{\pgfqpoint{1.663950in}{2.171305in}}{\pgfqpoint{1.656050in}{2.174577in}}{\pgfqpoint{1.647814in}{2.174577in}}%
\pgfpathcurveto{\pgfqpoint{1.639578in}{2.174577in}}{\pgfqpoint{1.631678in}{2.171305in}}{\pgfqpoint{1.625854in}{2.165481in}}%
\pgfpathcurveto{\pgfqpoint{1.620030in}{2.159657in}}{\pgfqpoint{1.616758in}{2.151757in}}{\pgfqpoint{1.616758in}{2.143520in}}%
\pgfpathcurveto{\pgfqpoint{1.616758in}{2.135284in}}{\pgfqpoint{1.620030in}{2.127384in}}{\pgfqpoint{1.625854in}{2.121560in}}%
\pgfpathcurveto{\pgfqpoint{1.631678in}{2.115736in}}{\pgfqpoint{1.639578in}{2.112464in}}{\pgfqpoint{1.647814in}{2.112464in}}%
\pgfpathclose%
\pgfusepath{stroke,fill}%
\end{pgfscope}%
\begin{pgfscope}%
\pgfpathrectangle{\pgfqpoint{0.100000in}{0.212622in}}{\pgfqpoint{3.696000in}{3.696000in}}%
\pgfusepath{clip}%
\pgfsetbuttcap%
\pgfsetroundjoin%
\definecolor{currentfill}{rgb}{0.121569,0.466667,0.705882}%
\pgfsetfillcolor{currentfill}%
\pgfsetfillopacity{0.300651}%
\pgfsetlinewidth{1.003750pt}%
\definecolor{currentstroke}{rgb}{0.121569,0.466667,0.705882}%
\pgfsetstrokecolor{currentstroke}%
\pgfsetstrokeopacity{0.300651}%
\pgfsetdash{}{0pt}%
\pgfpathmoveto{\pgfqpoint{1.647814in}{2.112464in}}%
\pgfpathcurveto{\pgfqpoint{1.656050in}{2.112464in}}{\pgfqpoint{1.663950in}{2.115736in}}{\pgfqpoint{1.669774in}{2.121560in}}%
\pgfpathcurveto{\pgfqpoint{1.675598in}{2.127384in}}{\pgfqpoint{1.678871in}{2.135284in}}{\pgfqpoint{1.678871in}{2.143520in}}%
\pgfpathcurveto{\pgfqpoint{1.678871in}{2.151757in}}{\pgfqpoint{1.675598in}{2.159657in}}{\pgfqpoint{1.669774in}{2.165481in}}%
\pgfpathcurveto{\pgfqpoint{1.663950in}{2.171305in}}{\pgfqpoint{1.656050in}{2.174577in}}{\pgfqpoint{1.647814in}{2.174577in}}%
\pgfpathcurveto{\pgfqpoint{1.639578in}{2.174577in}}{\pgfqpoint{1.631678in}{2.171305in}}{\pgfqpoint{1.625854in}{2.165481in}}%
\pgfpathcurveto{\pgfqpoint{1.620030in}{2.159657in}}{\pgfqpoint{1.616758in}{2.151757in}}{\pgfqpoint{1.616758in}{2.143520in}}%
\pgfpathcurveto{\pgfqpoint{1.616758in}{2.135284in}}{\pgfqpoint{1.620030in}{2.127384in}}{\pgfqpoint{1.625854in}{2.121560in}}%
\pgfpathcurveto{\pgfqpoint{1.631678in}{2.115736in}}{\pgfqpoint{1.639578in}{2.112464in}}{\pgfqpoint{1.647814in}{2.112464in}}%
\pgfpathclose%
\pgfusepath{stroke,fill}%
\end{pgfscope}%
\begin{pgfscope}%
\pgfpathrectangle{\pgfqpoint{0.100000in}{0.212622in}}{\pgfqpoint{3.696000in}{3.696000in}}%
\pgfusepath{clip}%
\pgfsetbuttcap%
\pgfsetroundjoin%
\definecolor{currentfill}{rgb}{0.121569,0.466667,0.705882}%
\pgfsetfillcolor{currentfill}%
\pgfsetfillopacity{0.300651}%
\pgfsetlinewidth{1.003750pt}%
\definecolor{currentstroke}{rgb}{0.121569,0.466667,0.705882}%
\pgfsetstrokecolor{currentstroke}%
\pgfsetstrokeopacity{0.300651}%
\pgfsetdash{}{0pt}%
\pgfpathmoveto{\pgfqpoint{1.647814in}{2.112464in}}%
\pgfpathcurveto{\pgfqpoint{1.656050in}{2.112464in}}{\pgfqpoint{1.663950in}{2.115736in}}{\pgfqpoint{1.669774in}{2.121560in}}%
\pgfpathcurveto{\pgfqpoint{1.675598in}{2.127384in}}{\pgfqpoint{1.678871in}{2.135284in}}{\pgfqpoint{1.678871in}{2.143520in}}%
\pgfpathcurveto{\pgfqpoint{1.678871in}{2.151757in}}{\pgfqpoint{1.675598in}{2.159657in}}{\pgfqpoint{1.669774in}{2.165481in}}%
\pgfpathcurveto{\pgfqpoint{1.663950in}{2.171305in}}{\pgfqpoint{1.656050in}{2.174577in}}{\pgfqpoint{1.647814in}{2.174577in}}%
\pgfpathcurveto{\pgfqpoint{1.639578in}{2.174577in}}{\pgfqpoint{1.631678in}{2.171305in}}{\pgfqpoint{1.625854in}{2.165481in}}%
\pgfpathcurveto{\pgfqpoint{1.620030in}{2.159657in}}{\pgfqpoint{1.616758in}{2.151757in}}{\pgfqpoint{1.616758in}{2.143520in}}%
\pgfpathcurveto{\pgfqpoint{1.616758in}{2.135284in}}{\pgfqpoint{1.620030in}{2.127384in}}{\pgfqpoint{1.625854in}{2.121560in}}%
\pgfpathcurveto{\pgfqpoint{1.631678in}{2.115736in}}{\pgfqpoint{1.639578in}{2.112464in}}{\pgfqpoint{1.647814in}{2.112464in}}%
\pgfpathclose%
\pgfusepath{stroke,fill}%
\end{pgfscope}%
\begin{pgfscope}%
\pgfpathrectangle{\pgfqpoint{0.100000in}{0.212622in}}{\pgfqpoint{3.696000in}{3.696000in}}%
\pgfusepath{clip}%
\pgfsetbuttcap%
\pgfsetroundjoin%
\definecolor{currentfill}{rgb}{0.121569,0.466667,0.705882}%
\pgfsetfillcolor{currentfill}%
\pgfsetfillopacity{0.300651}%
\pgfsetlinewidth{1.003750pt}%
\definecolor{currentstroke}{rgb}{0.121569,0.466667,0.705882}%
\pgfsetstrokecolor{currentstroke}%
\pgfsetstrokeopacity{0.300651}%
\pgfsetdash{}{0pt}%
\pgfpathmoveto{\pgfqpoint{1.647814in}{2.112464in}}%
\pgfpathcurveto{\pgfqpoint{1.656050in}{2.112464in}}{\pgfqpoint{1.663950in}{2.115736in}}{\pgfqpoint{1.669774in}{2.121560in}}%
\pgfpathcurveto{\pgfqpoint{1.675598in}{2.127384in}}{\pgfqpoint{1.678871in}{2.135284in}}{\pgfqpoint{1.678871in}{2.143520in}}%
\pgfpathcurveto{\pgfqpoint{1.678871in}{2.151757in}}{\pgfqpoint{1.675598in}{2.159657in}}{\pgfqpoint{1.669774in}{2.165481in}}%
\pgfpathcurveto{\pgfqpoint{1.663950in}{2.171305in}}{\pgfqpoint{1.656050in}{2.174577in}}{\pgfqpoint{1.647814in}{2.174577in}}%
\pgfpathcurveto{\pgfqpoint{1.639578in}{2.174577in}}{\pgfqpoint{1.631678in}{2.171305in}}{\pgfqpoint{1.625854in}{2.165481in}}%
\pgfpathcurveto{\pgfqpoint{1.620030in}{2.159657in}}{\pgfqpoint{1.616758in}{2.151757in}}{\pgfqpoint{1.616758in}{2.143520in}}%
\pgfpathcurveto{\pgfqpoint{1.616758in}{2.135284in}}{\pgfqpoint{1.620030in}{2.127384in}}{\pgfqpoint{1.625854in}{2.121560in}}%
\pgfpathcurveto{\pgfqpoint{1.631678in}{2.115736in}}{\pgfqpoint{1.639578in}{2.112464in}}{\pgfqpoint{1.647814in}{2.112464in}}%
\pgfpathclose%
\pgfusepath{stroke,fill}%
\end{pgfscope}%
\begin{pgfscope}%
\pgfpathrectangle{\pgfqpoint{0.100000in}{0.212622in}}{\pgfqpoint{3.696000in}{3.696000in}}%
\pgfusepath{clip}%
\pgfsetbuttcap%
\pgfsetroundjoin%
\definecolor{currentfill}{rgb}{0.121569,0.466667,0.705882}%
\pgfsetfillcolor{currentfill}%
\pgfsetfillopacity{0.300651}%
\pgfsetlinewidth{1.003750pt}%
\definecolor{currentstroke}{rgb}{0.121569,0.466667,0.705882}%
\pgfsetstrokecolor{currentstroke}%
\pgfsetstrokeopacity{0.300651}%
\pgfsetdash{}{0pt}%
\pgfpathmoveto{\pgfqpoint{1.647814in}{2.112464in}}%
\pgfpathcurveto{\pgfqpoint{1.656050in}{2.112464in}}{\pgfqpoint{1.663950in}{2.115736in}}{\pgfqpoint{1.669774in}{2.121560in}}%
\pgfpathcurveto{\pgfqpoint{1.675598in}{2.127384in}}{\pgfqpoint{1.678871in}{2.135284in}}{\pgfqpoint{1.678871in}{2.143520in}}%
\pgfpathcurveto{\pgfqpoint{1.678871in}{2.151757in}}{\pgfqpoint{1.675598in}{2.159657in}}{\pgfqpoint{1.669774in}{2.165481in}}%
\pgfpathcurveto{\pgfqpoint{1.663950in}{2.171305in}}{\pgfqpoint{1.656050in}{2.174577in}}{\pgfqpoint{1.647814in}{2.174577in}}%
\pgfpathcurveto{\pgfqpoint{1.639578in}{2.174577in}}{\pgfqpoint{1.631678in}{2.171305in}}{\pgfqpoint{1.625854in}{2.165481in}}%
\pgfpathcurveto{\pgfqpoint{1.620030in}{2.159657in}}{\pgfqpoint{1.616758in}{2.151757in}}{\pgfqpoint{1.616758in}{2.143520in}}%
\pgfpathcurveto{\pgfqpoint{1.616758in}{2.135284in}}{\pgfqpoint{1.620030in}{2.127384in}}{\pgfqpoint{1.625854in}{2.121560in}}%
\pgfpathcurveto{\pgfqpoint{1.631678in}{2.115736in}}{\pgfqpoint{1.639578in}{2.112464in}}{\pgfqpoint{1.647814in}{2.112464in}}%
\pgfpathclose%
\pgfusepath{stroke,fill}%
\end{pgfscope}%
\begin{pgfscope}%
\pgfpathrectangle{\pgfqpoint{0.100000in}{0.212622in}}{\pgfqpoint{3.696000in}{3.696000in}}%
\pgfusepath{clip}%
\pgfsetbuttcap%
\pgfsetroundjoin%
\definecolor{currentfill}{rgb}{0.121569,0.466667,0.705882}%
\pgfsetfillcolor{currentfill}%
\pgfsetfillopacity{0.300651}%
\pgfsetlinewidth{1.003750pt}%
\definecolor{currentstroke}{rgb}{0.121569,0.466667,0.705882}%
\pgfsetstrokecolor{currentstroke}%
\pgfsetstrokeopacity{0.300651}%
\pgfsetdash{}{0pt}%
\pgfpathmoveto{\pgfqpoint{1.647814in}{2.112464in}}%
\pgfpathcurveto{\pgfqpoint{1.656050in}{2.112464in}}{\pgfqpoint{1.663950in}{2.115736in}}{\pgfqpoint{1.669774in}{2.121560in}}%
\pgfpathcurveto{\pgfqpoint{1.675598in}{2.127384in}}{\pgfqpoint{1.678871in}{2.135284in}}{\pgfqpoint{1.678871in}{2.143520in}}%
\pgfpathcurveto{\pgfqpoint{1.678871in}{2.151757in}}{\pgfqpoint{1.675598in}{2.159657in}}{\pgfqpoint{1.669774in}{2.165481in}}%
\pgfpathcurveto{\pgfqpoint{1.663950in}{2.171305in}}{\pgfqpoint{1.656050in}{2.174577in}}{\pgfqpoint{1.647814in}{2.174577in}}%
\pgfpathcurveto{\pgfqpoint{1.639578in}{2.174577in}}{\pgfqpoint{1.631678in}{2.171305in}}{\pgfqpoint{1.625854in}{2.165481in}}%
\pgfpathcurveto{\pgfqpoint{1.620030in}{2.159657in}}{\pgfqpoint{1.616758in}{2.151757in}}{\pgfqpoint{1.616758in}{2.143520in}}%
\pgfpathcurveto{\pgfqpoint{1.616758in}{2.135284in}}{\pgfqpoint{1.620030in}{2.127384in}}{\pgfqpoint{1.625854in}{2.121560in}}%
\pgfpathcurveto{\pgfqpoint{1.631678in}{2.115736in}}{\pgfqpoint{1.639578in}{2.112464in}}{\pgfqpoint{1.647814in}{2.112464in}}%
\pgfpathclose%
\pgfusepath{stroke,fill}%
\end{pgfscope}%
\begin{pgfscope}%
\pgfpathrectangle{\pgfqpoint{0.100000in}{0.212622in}}{\pgfqpoint{3.696000in}{3.696000in}}%
\pgfusepath{clip}%
\pgfsetbuttcap%
\pgfsetroundjoin%
\definecolor{currentfill}{rgb}{0.121569,0.466667,0.705882}%
\pgfsetfillcolor{currentfill}%
\pgfsetfillopacity{0.300651}%
\pgfsetlinewidth{1.003750pt}%
\definecolor{currentstroke}{rgb}{0.121569,0.466667,0.705882}%
\pgfsetstrokecolor{currentstroke}%
\pgfsetstrokeopacity{0.300651}%
\pgfsetdash{}{0pt}%
\pgfpathmoveto{\pgfqpoint{1.647814in}{2.112464in}}%
\pgfpathcurveto{\pgfqpoint{1.656050in}{2.112464in}}{\pgfqpoint{1.663950in}{2.115736in}}{\pgfqpoint{1.669774in}{2.121560in}}%
\pgfpathcurveto{\pgfqpoint{1.675598in}{2.127384in}}{\pgfqpoint{1.678871in}{2.135284in}}{\pgfqpoint{1.678871in}{2.143520in}}%
\pgfpathcurveto{\pgfqpoint{1.678871in}{2.151757in}}{\pgfqpoint{1.675598in}{2.159657in}}{\pgfqpoint{1.669774in}{2.165481in}}%
\pgfpathcurveto{\pgfqpoint{1.663950in}{2.171305in}}{\pgfqpoint{1.656050in}{2.174577in}}{\pgfqpoint{1.647814in}{2.174577in}}%
\pgfpathcurveto{\pgfqpoint{1.639578in}{2.174577in}}{\pgfqpoint{1.631678in}{2.171305in}}{\pgfqpoint{1.625854in}{2.165481in}}%
\pgfpathcurveto{\pgfqpoint{1.620030in}{2.159657in}}{\pgfqpoint{1.616758in}{2.151757in}}{\pgfqpoint{1.616758in}{2.143520in}}%
\pgfpathcurveto{\pgfqpoint{1.616758in}{2.135284in}}{\pgfqpoint{1.620030in}{2.127384in}}{\pgfqpoint{1.625854in}{2.121560in}}%
\pgfpathcurveto{\pgfqpoint{1.631678in}{2.115736in}}{\pgfqpoint{1.639578in}{2.112464in}}{\pgfqpoint{1.647814in}{2.112464in}}%
\pgfpathclose%
\pgfusepath{stroke,fill}%
\end{pgfscope}%
\begin{pgfscope}%
\pgfpathrectangle{\pgfqpoint{0.100000in}{0.212622in}}{\pgfqpoint{3.696000in}{3.696000in}}%
\pgfusepath{clip}%
\pgfsetbuttcap%
\pgfsetroundjoin%
\definecolor{currentfill}{rgb}{0.121569,0.466667,0.705882}%
\pgfsetfillcolor{currentfill}%
\pgfsetfillopacity{0.300651}%
\pgfsetlinewidth{1.003750pt}%
\definecolor{currentstroke}{rgb}{0.121569,0.466667,0.705882}%
\pgfsetstrokecolor{currentstroke}%
\pgfsetstrokeopacity{0.300651}%
\pgfsetdash{}{0pt}%
\pgfpathmoveto{\pgfqpoint{1.647814in}{2.112464in}}%
\pgfpathcurveto{\pgfqpoint{1.656050in}{2.112464in}}{\pgfqpoint{1.663950in}{2.115736in}}{\pgfqpoint{1.669774in}{2.121560in}}%
\pgfpathcurveto{\pgfqpoint{1.675598in}{2.127384in}}{\pgfqpoint{1.678871in}{2.135284in}}{\pgfqpoint{1.678871in}{2.143520in}}%
\pgfpathcurveto{\pgfqpoint{1.678871in}{2.151757in}}{\pgfqpoint{1.675598in}{2.159657in}}{\pgfqpoint{1.669774in}{2.165481in}}%
\pgfpathcurveto{\pgfqpoint{1.663950in}{2.171305in}}{\pgfqpoint{1.656050in}{2.174577in}}{\pgfqpoint{1.647814in}{2.174577in}}%
\pgfpathcurveto{\pgfqpoint{1.639578in}{2.174577in}}{\pgfqpoint{1.631678in}{2.171305in}}{\pgfqpoint{1.625854in}{2.165481in}}%
\pgfpathcurveto{\pgfqpoint{1.620030in}{2.159657in}}{\pgfqpoint{1.616758in}{2.151757in}}{\pgfqpoint{1.616758in}{2.143520in}}%
\pgfpathcurveto{\pgfqpoint{1.616758in}{2.135284in}}{\pgfqpoint{1.620030in}{2.127384in}}{\pgfqpoint{1.625854in}{2.121560in}}%
\pgfpathcurveto{\pgfqpoint{1.631678in}{2.115736in}}{\pgfqpoint{1.639578in}{2.112464in}}{\pgfqpoint{1.647814in}{2.112464in}}%
\pgfpathclose%
\pgfusepath{stroke,fill}%
\end{pgfscope}%
\begin{pgfscope}%
\pgfpathrectangle{\pgfqpoint{0.100000in}{0.212622in}}{\pgfqpoint{3.696000in}{3.696000in}}%
\pgfusepath{clip}%
\pgfsetbuttcap%
\pgfsetroundjoin%
\definecolor{currentfill}{rgb}{0.121569,0.466667,0.705882}%
\pgfsetfillcolor{currentfill}%
\pgfsetfillopacity{0.300651}%
\pgfsetlinewidth{1.003750pt}%
\definecolor{currentstroke}{rgb}{0.121569,0.466667,0.705882}%
\pgfsetstrokecolor{currentstroke}%
\pgfsetstrokeopacity{0.300651}%
\pgfsetdash{}{0pt}%
\pgfpathmoveto{\pgfqpoint{1.647814in}{2.112464in}}%
\pgfpathcurveto{\pgfqpoint{1.656050in}{2.112464in}}{\pgfqpoint{1.663950in}{2.115736in}}{\pgfqpoint{1.669774in}{2.121560in}}%
\pgfpathcurveto{\pgfqpoint{1.675598in}{2.127384in}}{\pgfqpoint{1.678871in}{2.135284in}}{\pgfqpoint{1.678871in}{2.143520in}}%
\pgfpathcurveto{\pgfqpoint{1.678871in}{2.151757in}}{\pgfqpoint{1.675598in}{2.159657in}}{\pgfqpoint{1.669774in}{2.165481in}}%
\pgfpathcurveto{\pgfqpoint{1.663950in}{2.171305in}}{\pgfqpoint{1.656050in}{2.174577in}}{\pgfqpoint{1.647814in}{2.174577in}}%
\pgfpathcurveto{\pgfqpoint{1.639578in}{2.174577in}}{\pgfqpoint{1.631678in}{2.171305in}}{\pgfqpoint{1.625854in}{2.165481in}}%
\pgfpathcurveto{\pgfqpoint{1.620030in}{2.159657in}}{\pgfqpoint{1.616758in}{2.151757in}}{\pgfqpoint{1.616758in}{2.143520in}}%
\pgfpathcurveto{\pgfqpoint{1.616758in}{2.135284in}}{\pgfqpoint{1.620030in}{2.127384in}}{\pgfqpoint{1.625854in}{2.121560in}}%
\pgfpathcurveto{\pgfqpoint{1.631678in}{2.115736in}}{\pgfqpoint{1.639578in}{2.112464in}}{\pgfqpoint{1.647814in}{2.112464in}}%
\pgfpathclose%
\pgfusepath{stroke,fill}%
\end{pgfscope}%
\begin{pgfscope}%
\pgfpathrectangle{\pgfqpoint{0.100000in}{0.212622in}}{\pgfqpoint{3.696000in}{3.696000in}}%
\pgfusepath{clip}%
\pgfsetbuttcap%
\pgfsetroundjoin%
\definecolor{currentfill}{rgb}{0.121569,0.466667,0.705882}%
\pgfsetfillcolor{currentfill}%
\pgfsetfillopacity{0.300651}%
\pgfsetlinewidth{1.003750pt}%
\definecolor{currentstroke}{rgb}{0.121569,0.466667,0.705882}%
\pgfsetstrokecolor{currentstroke}%
\pgfsetstrokeopacity{0.300651}%
\pgfsetdash{}{0pt}%
\pgfpathmoveto{\pgfqpoint{1.647814in}{2.112464in}}%
\pgfpathcurveto{\pgfqpoint{1.656050in}{2.112464in}}{\pgfqpoint{1.663950in}{2.115736in}}{\pgfqpoint{1.669774in}{2.121560in}}%
\pgfpathcurveto{\pgfqpoint{1.675598in}{2.127384in}}{\pgfqpoint{1.678871in}{2.135284in}}{\pgfqpoint{1.678871in}{2.143520in}}%
\pgfpathcurveto{\pgfqpoint{1.678871in}{2.151757in}}{\pgfqpoint{1.675598in}{2.159657in}}{\pgfqpoint{1.669774in}{2.165481in}}%
\pgfpathcurveto{\pgfqpoint{1.663950in}{2.171305in}}{\pgfqpoint{1.656050in}{2.174577in}}{\pgfqpoint{1.647814in}{2.174577in}}%
\pgfpathcurveto{\pgfqpoint{1.639578in}{2.174577in}}{\pgfqpoint{1.631678in}{2.171305in}}{\pgfqpoint{1.625854in}{2.165481in}}%
\pgfpathcurveto{\pgfqpoint{1.620030in}{2.159657in}}{\pgfqpoint{1.616758in}{2.151757in}}{\pgfqpoint{1.616758in}{2.143520in}}%
\pgfpathcurveto{\pgfqpoint{1.616758in}{2.135284in}}{\pgfqpoint{1.620030in}{2.127384in}}{\pgfqpoint{1.625854in}{2.121560in}}%
\pgfpathcurveto{\pgfqpoint{1.631678in}{2.115736in}}{\pgfqpoint{1.639578in}{2.112464in}}{\pgfqpoint{1.647814in}{2.112464in}}%
\pgfpathclose%
\pgfusepath{stroke,fill}%
\end{pgfscope}%
\begin{pgfscope}%
\pgfpathrectangle{\pgfqpoint{0.100000in}{0.212622in}}{\pgfqpoint{3.696000in}{3.696000in}}%
\pgfusepath{clip}%
\pgfsetbuttcap%
\pgfsetroundjoin%
\definecolor{currentfill}{rgb}{0.121569,0.466667,0.705882}%
\pgfsetfillcolor{currentfill}%
\pgfsetfillopacity{0.300651}%
\pgfsetlinewidth{1.003750pt}%
\definecolor{currentstroke}{rgb}{0.121569,0.466667,0.705882}%
\pgfsetstrokecolor{currentstroke}%
\pgfsetstrokeopacity{0.300651}%
\pgfsetdash{}{0pt}%
\pgfpathmoveto{\pgfqpoint{1.647814in}{2.112464in}}%
\pgfpathcurveto{\pgfqpoint{1.656050in}{2.112464in}}{\pgfqpoint{1.663950in}{2.115736in}}{\pgfqpoint{1.669774in}{2.121560in}}%
\pgfpathcurveto{\pgfqpoint{1.675598in}{2.127384in}}{\pgfqpoint{1.678871in}{2.135284in}}{\pgfqpoint{1.678871in}{2.143520in}}%
\pgfpathcurveto{\pgfqpoint{1.678871in}{2.151757in}}{\pgfqpoint{1.675598in}{2.159657in}}{\pgfqpoint{1.669774in}{2.165481in}}%
\pgfpathcurveto{\pgfqpoint{1.663950in}{2.171305in}}{\pgfqpoint{1.656050in}{2.174577in}}{\pgfqpoint{1.647814in}{2.174577in}}%
\pgfpathcurveto{\pgfqpoint{1.639578in}{2.174577in}}{\pgfqpoint{1.631678in}{2.171305in}}{\pgfqpoint{1.625854in}{2.165481in}}%
\pgfpathcurveto{\pgfqpoint{1.620030in}{2.159657in}}{\pgfqpoint{1.616758in}{2.151757in}}{\pgfqpoint{1.616758in}{2.143520in}}%
\pgfpathcurveto{\pgfqpoint{1.616758in}{2.135284in}}{\pgfqpoint{1.620030in}{2.127384in}}{\pgfqpoint{1.625854in}{2.121560in}}%
\pgfpathcurveto{\pgfqpoint{1.631678in}{2.115736in}}{\pgfqpoint{1.639578in}{2.112464in}}{\pgfqpoint{1.647814in}{2.112464in}}%
\pgfpathclose%
\pgfusepath{stroke,fill}%
\end{pgfscope}%
\begin{pgfscope}%
\pgfpathrectangle{\pgfqpoint{0.100000in}{0.212622in}}{\pgfqpoint{3.696000in}{3.696000in}}%
\pgfusepath{clip}%
\pgfsetbuttcap%
\pgfsetroundjoin%
\definecolor{currentfill}{rgb}{0.121569,0.466667,0.705882}%
\pgfsetfillcolor{currentfill}%
\pgfsetfillopacity{0.300651}%
\pgfsetlinewidth{1.003750pt}%
\definecolor{currentstroke}{rgb}{0.121569,0.466667,0.705882}%
\pgfsetstrokecolor{currentstroke}%
\pgfsetstrokeopacity{0.300651}%
\pgfsetdash{}{0pt}%
\pgfpathmoveto{\pgfqpoint{1.647814in}{2.112464in}}%
\pgfpathcurveto{\pgfqpoint{1.656050in}{2.112464in}}{\pgfqpoint{1.663950in}{2.115736in}}{\pgfqpoint{1.669774in}{2.121560in}}%
\pgfpathcurveto{\pgfqpoint{1.675598in}{2.127384in}}{\pgfqpoint{1.678871in}{2.135284in}}{\pgfqpoint{1.678871in}{2.143520in}}%
\pgfpathcurveto{\pgfqpoint{1.678871in}{2.151757in}}{\pgfqpoint{1.675598in}{2.159657in}}{\pgfqpoint{1.669774in}{2.165481in}}%
\pgfpathcurveto{\pgfqpoint{1.663950in}{2.171305in}}{\pgfqpoint{1.656050in}{2.174577in}}{\pgfqpoint{1.647814in}{2.174577in}}%
\pgfpathcurveto{\pgfqpoint{1.639578in}{2.174577in}}{\pgfqpoint{1.631678in}{2.171305in}}{\pgfqpoint{1.625854in}{2.165481in}}%
\pgfpathcurveto{\pgfqpoint{1.620030in}{2.159657in}}{\pgfqpoint{1.616758in}{2.151757in}}{\pgfqpoint{1.616758in}{2.143520in}}%
\pgfpathcurveto{\pgfqpoint{1.616758in}{2.135284in}}{\pgfqpoint{1.620030in}{2.127384in}}{\pgfqpoint{1.625854in}{2.121560in}}%
\pgfpathcurveto{\pgfqpoint{1.631678in}{2.115736in}}{\pgfqpoint{1.639578in}{2.112464in}}{\pgfqpoint{1.647814in}{2.112464in}}%
\pgfpathclose%
\pgfusepath{stroke,fill}%
\end{pgfscope}%
\begin{pgfscope}%
\pgfpathrectangle{\pgfqpoint{0.100000in}{0.212622in}}{\pgfqpoint{3.696000in}{3.696000in}}%
\pgfusepath{clip}%
\pgfsetbuttcap%
\pgfsetroundjoin%
\definecolor{currentfill}{rgb}{0.121569,0.466667,0.705882}%
\pgfsetfillcolor{currentfill}%
\pgfsetfillopacity{0.300651}%
\pgfsetlinewidth{1.003750pt}%
\definecolor{currentstroke}{rgb}{0.121569,0.466667,0.705882}%
\pgfsetstrokecolor{currentstroke}%
\pgfsetstrokeopacity{0.300651}%
\pgfsetdash{}{0pt}%
\pgfpathmoveto{\pgfqpoint{1.647814in}{2.112464in}}%
\pgfpathcurveto{\pgfqpoint{1.656050in}{2.112464in}}{\pgfqpoint{1.663950in}{2.115736in}}{\pgfqpoint{1.669774in}{2.121560in}}%
\pgfpathcurveto{\pgfqpoint{1.675598in}{2.127384in}}{\pgfqpoint{1.678871in}{2.135284in}}{\pgfqpoint{1.678871in}{2.143520in}}%
\pgfpathcurveto{\pgfqpoint{1.678871in}{2.151757in}}{\pgfqpoint{1.675598in}{2.159657in}}{\pgfqpoint{1.669774in}{2.165481in}}%
\pgfpathcurveto{\pgfqpoint{1.663950in}{2.171305in}}{\pgfqpoint{1.656050in}{2.174577in}}{\pgfqpoint{1.647814in}{2.174577in}}%
\pgfpathcurveto{\pgfqpoint{1.639578in}{2.174577in}}{\pgfqpoint{1.631678in}{2.171305in}}{\pgfqpoint{1.625854in}{2.165481in}}%
\pgfpathcurveto{\pgfqpoint{1.620030in}{2.159657in}}{\pgfqpoint{1.616758in}{2.151757in}}{\pgfqpoint{1.616758in}{2.143520in}}%
\pgfpathcurveto{\pgfqpoint{1.616758in}{2.135284in}}{\pgfqpoint{1.620030in}{2.127384in}}{\pgfqpoint{1.625854in}{2.121560in}}%
\pgfpathcurveto{\pgfqpoint{1.631678in}{2.115736in}}{\pgfqpoint{1.639578in}{2.112464in}}{\pgfqpoint{1.647814in}{2.112464in}}%
\pgfpathclose%
\pgfusepath{stroke,fill}%
\end{pgfscope}%
\begin{pgfscope}%
\pgfpathrectangle{\pgfqpoint{0.100000in}{0.212622in}}{\pgfqpoint{3.696000in}{3.696000in}}%
\pgfusepath{clip}%
\pgfsetbuttcap%
\pgfsetroundjoin%
\definecolor{currentfill}{rgb}{0.121569,0.466667,0.705882}%
\pgfsetfillcolor{currentfill}%
\pgfsetfillopacity{0.300651}%
\pgfsetlinewidth{1.003750pt}%
\definecolor{currentstroke}{rgb}{0.121569,0.466667,0.705882}%
\pgfsetstrokecolor{currentstroke}%
\pgfsetstrokeopacity{0.300651}%
\pgfsetdash{}{0pt}%
\pgfpathmoveto{\pgfqpoint{1.647814in}{2.112464in}}%
\pgfpathcurveto{\pgfqpoint{1.656050in}{2.112464in}}{\pgfqpoint{1.663950in}{2.115736in}}{\pgfqpoint{1.669774in}{2.121560in}}%
\pgfpathcurveto{\pgfqpoint{1.675598in}{2.127384in}}{\pgfqpoint{1.678871in}{2.135284in}}{\pgfqpoint{1.678871in}{2.143520in}}%
\pgfpathcurveto{\pgfqpoint{1.678871in}{2.151757in}}{\pgfqpoint{1.675598in}{2.159657in}}{\pgfqpoint{1.669774in}{2.165481in}}%
\pgfpathcurveto{\pgfqpoint{1.663950in}{2.171305in}}{\pgfqpoint{1.656050in}{2.174577in}}{\pgfqpoint{1.647814in}{2.174577in}}%
\pgfpathcurveto{\pgfqpoint{1.639578in}{2.174577in}}{\pgfqpoint{1.631678in}{2.171305in}}{\pgfqpoint{1.625854in}{2.165481in}}%
\pgfpathcurveto{\pgfqpoint{1.620030in}{2.159657in}}{\pgfqpoint{1.616758in}{2.151757in}}{\pgfqpoint{1.616758in}{2.143520in}}%
\pgfpathcurveto{\pgfqpoint{1.616758in}{2.135284in}}{\pgfqpoint{1.620030in}{2.127384in}}{\pgfqpoint{1.625854in}{2.121560in}}%
\pgfpathcurveto{\pgfqpoint{1.631678in}{2.115736in}}{\pgfqpoint{1.639578in}{2.112464in}}{\pgfqpoint{1.647814in}{2.112464in}}%
\pgfpathclose%
\pgfusepath{stroke,fill}%
\end{pgfscope}%
\begin{pgfscope}%
\pgfpathrectangle{\pgfqpoint{0.100000in}{0.212622in}}{\pgfqpoint{3.696000in}{3.696000in}}%
\pgfusepath{clip}%
\pgfsetbuttcap%
\pgfsetroundjoin%
\definecolor{currentfill}{rgb}{0.121569,0.466667,0.705882}%
\pgfsetfillcolor{currentfill}%
\pgfsetfillopacity{0.300651}%
\pgfsetlinewidth{1.003750pt}%
\definecolor{currentstroke}{rgb}{0.121569,0.466667,0.705882}%
\pgfsetstrokecolor{currentstroke}%
\pgfsetstrokeopacity{0.300651}%
\pgfsetdash{}{0pt}%
\pgfpathmoveto{\pgfqpoint{1.647814in}{2.112464in}}%
\pgfpathcurveto{\pgfqpoint{1.656050in}{2.112464in}}{\pgfqpoint{1.663950in}{2.115736in}}{\pgfqpoint{1.669774in}{2.121560in}}%
\pgfpathcurveto{\pgfqpoint{1.675598in}{2.127384in}}{\pgfqpoint{1.678871in}{2.135284in}}{\pgfqpoint{1.678871in}{2.143520in}}%
\pgfpathcurveto{\pgfqpoint{1.678871in}{2.151757in}}{\pgfqpoint{1.675598in}{2.159657in}}{\pgfqpoint{1.669774in}{2.165481in}}%
\pgfpathcurveto{\pgfqpoint{1.663950in}{2.171305in}}{\pgfqpoint{1.656050in}{2.174577in}}{\pgfqpoint{1.647814in}{2.174577in}}%
\pgfpathcurveto{\pgfqpoint{1.639578in}{2.174577in}}{\pgfqpoint{1.631678in}{2.171305in}}{\pgfqpoint{1.625854in}{2.165481in}}%
\pgfpathcurveto{\pgfqpoint{1.620030in}{2.159657in}}{\pgfqpoint{1.616758in}{2.151757in}}{\pgfqpoint{1.616758in}{2.143520in}}%
\pgfpathcurveto{\pgfqpoint{1.616758in}{2.135284in}}{\pgfqpoint{1.620030in}{2.127384in}}{\pgfqpoint{1.625854in}{2.121560in}}%
\pgfpathcurveto{\pgfqpoint{1.631678in}{2.115736in}}{\pgfqpoint{1.639578in}{2.112464in}}{\pgfqpoint{1.647814in}{2.112464in}}%
\pgfpathclose%
\pgfusepath{stroke,fill}%
\end{pgfscope}%
\begin{pgfscope}%
\pgfpathrectangle{\pgfqpoint{0.100000in}{0.212622in}}{\pgfqpoint{3.696000in}{3.696000in}}%
\pgfusepath{clip}%
\pgfsetbuttcap%
\pgfsetroundjoin%
\definecolor{currentfill}{rgb}{0.121569,0.466667,0.705882}%
\pgfsetfillcolor{currentfill}%
\pgfsetfillopacity{0.300651}%
\pgfsetlinewidth{1.003750pt}%
\definecolor{currentstroke}{rgb}{0.121569,0.466667,0.705882}%
\pgfsetstrokecolor{currentstroke}%
\pgfsetstrokeopacity{0.300651}%
\pgfsetdash{}{0pt}%
\pgfpathmoveto{\pgfqpoint{1.647814in}{2.112464in}}%
\pgfpathcurveto{\pgfqpoint{1.656050in}{2.112464in}}{\pgfqpoint{1.663950in}{2.115736in}}{\pgfqpoint{1.669774in}{2.121560in}}%
\pgfpathcurveto{\pgfqpoint{1.675598in}{2.127384in}}{\pgfqpoint{1.678871in}{2.135284in}}{\pgfqpoint{1.678871in}{2.143520in}}%
\pgfpathcurveto{\pgfqpoint{1.678871in}{2.151757in}}{\pgfqpoint{1.675598in}{2.159657in}}{\pgfqpoint{1.669774in}{2.165481in}}%
\pgfpathcurveto{\pgfqpoint{1.663950in}{2.171305in}}{\pgfqpoint{1.656050in}{2.174577in}}{\pgfqpoint{1.647814in}{2.174577in}}%
\pgfpathcurveto{\pgfqpoint{1.639578in}{2.174577in}}{\pgfqpoint{1.631678in}{2.171305in}}{\pgfqpoint{1.625854in}{2.165481in}}%
\pgfpathcurveto{\pgfqpoint{1.620030in}{2.159657in}}{\pgfqpoint{1.616758in}{2.151757in}}{\pgfqpoint{1.616758in}{2.143520in}}%
\pgfpathcurveto{\pgfqpoint{1.616758in}{2.135284in}}{\pgfqpoint{1.620030in}{2.127384in}}{\pgfqpoint{1.625854in}{2.121560in}}%
\pgfpathcurveto{\pgfqpoint{1.631678in}{2.115736in}}{\pgfqpoint{1.639578in}{2.112464in}}{\pgfqpoint{1.647814in}{2.112464in}}%
\pgfpathclose%
\pgfusepath{stroke,fill}%
\end{pgfscope}%
\begin{pgfscope}%
\pgfpathrectangle{\pgfqpoint{0.100000in}{0.212622in}}{\pgfqpoint{3.696000in}{3.696000in}}%
\pgfusepath{clip}%
\pgfsetbuttcap%
\pgfsetroundjoin%
\definecolor{currentfill}{rgb}{0.121569,0.466667,0.705882}%
\pgfsetfillcolor{currentfill}%
\pgfsetfillopacity{0.300651}%
\pgfsetlinewidth{1.003750pt}%
\definecolor{currentstroke}{rgb}{0.121569,0.466667,0.705882}%
\pgfsetstrokecolor{currentstroke}%
\pgfsetstrokeopacity{0.300651}%
\pgfsetdash{}{0pt}%
\pgfpathmoveto{\pgfqpoint{1.647814in}{2.112464in}}%
\pgfpathcurveto{\pgfqpoint{1.656050in}{2.112464in}}{\pgfqpoint{1.663950in}{2.115736in}}{\pgfqpoint{1.669774in}{2.121560in}}%
\pgfpathcurveto{\pgfqpoint{1.675598in}{2.127384in}}{\pgfqpoint{1.678871in}{2.135284in}}{\pgfqpoint{1.678871in}{2.143520in}}%
\pgfpathcurveto{\pgfqpoint{1.678871in}{2.151757in}}{\pgfqpoint{1.675598in}{2.159657in}}{\pgfqpoint{1.669774in}{2.165481in}}%
\pgfpathcurveto{\pgfqpoint{1.663950in}{2.171305in}}{\pgfqpoint{1.656050in}{2.174577in}}{\pgfqpoint{1.647814in}{2.174577in}}%
\pgfpathcurveto{\pgfqpoint{1.639578in}{2.174577in}}{\pgfqpoint{1.631678in}{2.171305in}}{\pgfqpoint{1.625854in}{2.165481in}}%
\pgfpathcurveto{\pgfqpoint{1.620030in}{2.159657in}}{\pgfqpoint{1.616758in}{2.151757in}}{\pgfqpoint{1.616758in}{2.143520in}}%
\pgfpathcurveto{\pgfqpoint{1.616758in}{2.135284in}}{\pgfqpoint{1.620030in}{2.127384in}}{\pgfqpoint{1.625854in}{2.121560in}}%
\pgfpathcurveto{\pgfqpoint{1.631678in}{2.115736in}}{\pgfqpoint{1.639578in}{2.112464in}}{\pgfqpoint{1.647814in}{2.112464in}}%
\pgfpathclose%
\pgfusepath{stroke,fill}%
\end{pgfscope}%
\begin{pgfscope}%
\pgfpathrectangle{\pgfqpoint{0.100000in}{0.212622in}}{\pgfqpoint{3.696000in}{3.696000in}}%
\pgfusepath{clip}%
\pgfsetbuttcap%
\pgfsetroundjoin%
\definecolor{currentfill}{rgb}{0.121569,0.466667,0.705882}%
\pgfsetfillcolor{currentfill}%
\pgfsetfillopacity{0.300651}%
\pgfsetlinewidth{1.003750pt}%
\definecolor{currentstroke}{rgb}{0.121569,0.466667,0.705882}%
\pgfsetstrokecolor{currentstroke}%
\pgfsetstrokeopacity{0.300651}%
\pgfsetdash{}{0pt}%
\pgfpathmoveto{\pgfqpoint{1.647814in}{2.112464in}}%
\pgfpathcurveto{\pgfqpoint{1.656050in}{2.112464in}}{\pgfqpoint{1.663950in}{2.115736in}}{\pgfqpoint{1.669774in}{2.121560in}}%
\pgfpathcurveto{\pgfqpoint{1.675598in}{2.127384in}}{\pgfqpoint{1.678871in}{2.135284in}}{\pgfqpoint{1.678871in}{2.143520in}}%
\pgfpathcurveto{\pgfqpoint{1.678871in}{2.151757in}}{\pgfqpoint{1.675598in}{2.159657in}}{\pgfqpoint{1.669774in}{2.165481in}}%
\pgfpathcurveto{\pgfqpoint{1.663950in}{2.171305in}}{\pgfqpoint{1.656050in}{2.174577in}}{\pgfqpoint{1.647814in}{2.174577in}}%
\pgfpathcurveto{\pgfqpoint{1.639578in}{2.174577in}}{\pgfqpoint{1.631678in}{2.171305in}}{\pgfqpoint{1.625854in}{2.165481in}}%
\pgfpathcurveto{\pgfqpoint{1.620030in}{2.159657in}}{\pgfqpoint{1.616758in}{2.151757in}}{\pgfqpoint{1.616758in}{2.143520in}}%
\pgfpathcurveto{\pgfqpoint{1.616758in}{2.135284in}}{\pgfqpoint{1.620030in}{2.127384in}}{\pgfqpoint{1.625854in}{2.121560in}}%
\pgfpathcurveto{\pgfqpoint{1.631678in}{2.115736in}}{\pgfqpoint{1.639578in}{2.112464in}}{\pgfqpoint{1.647814in}{2.112464in}}%
\pgfpathclose%
\pgfusepath{stroke,fill}%
\end{pgfscope}%
\begin{pgfscope}%
\pgfpathrectangle{\pgfqpoint{0.100000in}{0.212622in}}{\pgfqpoint{3.696000in}{3.696000in}}%
\pgfusepath{clip}%
\pgfsetbuttcap%
\pgfsetroundjoin%
\definecolor{currentfill}{rgb}{0.121569,0.466667,0.705882}%
\pgfsetfillcolor{currentfill}%
\pgfsetfillopacity{0.300651}%
\pgfsetlinewidth{1.003750pt}%
\definecolor{currentstroke}{rgb}{0.121569,0.466667,0.705882}%
\pgfsetstrokecolor{currentstroke}%
\pgfsetstrokeopacity{0.300651}%
\pgfsetdash{}{0pt}%
\pgfpathmoveto{\pgfqpoint{1.647814in}{2.112464in}}%
\pgfpathcurveto{\pgfqpoint{1.656050in}{2.112464in}}{\pgfqpoint{1.663950in}{2.115736in}}{\pgfqpoint{1.669774in}{2.121560in}}%
\pgfpathcurveto{\pgfqpoint{1.675598in}{2.127384in}}{\pgfqpoint{1.678871in}{2.135284in}}{\pgfqpoint{1.678871in}{2.143520in}}%
\pgfpathcurveto{\pgfqpoint{1.678871in}{2.151757in}}{\pgfqpoint{1.675598in}{2.159657in}}{\pgfqpoint{1.669774in}{2.165481in}}%
\pgfpathcurveto{\pgfqpoint{1.663950in}{2.171305in}}{\pgfqpoint{1.656050in}{2.174577in}}{\pgfqpoint{1.647814in}{2.174577in}}%
\pgfpathcurveto{\pgfqpoint{1.639578in}{2.174577in}}{\pgfqpoint{1.631678in}{2.171305in}}{\pgfqpoint{1.625854in}{2.165481in}}%
\pgfpathcurveto{\pgfqpoint{1.620030in}{2.159657in}}{\pgfqpoint{1.616758in}{2.151757in}}{\pgfqpoint{1.616758in}{2.143520in}}%
\pgfpathcurveto{\pgfqpoint{1.616758in}{2.135284in}}{\pgfqpoint{1.620030in}{2.127384in}}{\pgfqpoint{1.625854in}{2.121560in}}%
\pgfpathcurveto{\pgfqpoint{1.631678in}{2.115736in}}{\pgfqpoint{1.639578in}{2.112464in}}{\pgfqpoint{1.647814in}{2.112464in}}%
\pgfpathclose%
\pgfusepath{stroke,fill}%
\end{pgfscope}%
\begin{pgfscope}%
\pgfpathrectangle{\pgfqpoint{0.100000in}{0.212622in}}{\pgfqpoint{3.696000in}{3.696000in}}%
\pgfusepath{clip}%
\pgfsetbuttcap%
\pgfsetroundjoin%
\definecolor{currentfill}{rgb}{0.121569,0.466667,0.705882}%
\pgfsetfillcolor{currentfill}%
\pgfsetfillopacity{0.300651}%
\pgfsetlinewidth{1.003750pt}%
\definecolor{currentstroke}{rgb}{0.121569,0.466667,0.705882}%
\pgfsetstrokecolor{currentstroke}%
\pgfsetstrokeopacity{0.300651}%
\pgfsetdash{}{0pt}%
\pgfpathmoveto{\pgfqpoint{1.647814in}{2.112464in}}%
\pgfpathcurveto{\pgfqpoint{1.656050in}{2.112464in}}{\pgfqpoint{1.663950in}{2.115736in}}{\pgfqpoint{1.669774in}{2.121560in}}%
\pgfpathcurveto{\pgfqpoint{1.675598in}{2.127384in}}{\pgfqpoint{1.678871in}{2.135284in}}{\pgfqpoint{1.678871in}{2.143520in}}%
\pgfpathcurveto{\pgfqpoint{1.678871in}{2.151757in}}{\pgfqpoint{1.675598in}{2.159657in}}{\pgfqpoint{1.669774in}{2.165481in}}%
\pgfpathcurveto{\pgfqpoint{1.663950in}{2.171305in}}{\pgfqpoint{1.656050in}{2.174577in}}{\pgfqpoint{1.647814in}{2.174577in}}%
\pgfpathcurveto{\pgfqpoint{1.639578in}{2.174577in}}{\pgfqpoint{1.631678in}{2.171305in}}{\pgfqpoint{1.625854in}{2.165481in}}%
\pgfpathcurveto{\pgfqpoint{1.620030in}{2.159657in}}{\pgfqpoint{1.616758in}{2.151757in}}{\pgfqpoint{1.616758in}{2.143520in}}%
\pgfpathcurveto{\pgfqpoint{1.616758in}{2.135284in}}{\pgfqpoint{1.620030in}{2.127384in}}{\pgfqpoint{1.625854in}{2.121560in}}%
\pgfpathcurveto{\pgfqpoint{1.631678in}{2.115736in}}{\pgfqpoint{1.639578in}{2.112464in}}{\pgfqpoint{1.647814in}{2.112464in}}%
\pgfpathclose%
\pgfusepath{stroke,fill}%
\end{pgfscope}%
\begin{pgfscope}%
\pgfpathrectangle{\pgfqpoint{0.100000in}{0.212622in}}{\pgfqpoint{3.696000in}{3.696000in}}%
\pgfusepath{clip}%
\pgfsetbuttcap%
\pgfsetroundjoin%
\definecolor{currentfill}{rgb}{0.121569,0.466667,0.705882}%
\pgfsetfillcolor{currentfill}%
\pgfsetfillopacity{0.300651}%
\pgfsetlinewidth{1.003750pt}%
\definecolor{currentstroke}{rgb}{0.121569,0.466667,0.705882}%
\pgfsetstrokecolor{currentstroke}%
\pgfsetstrokeopacity{0.300651}%
\pgfsetdash{}{0pt}%
\pgfpathmoveto{\pgfqpoint{1.647814in}{2.112464in}}%
\pgfpathcurveto{\pgfqpoint{1.656050in}{2.112464in}}{\pgfqpoint{1.663950in}{2.115736in}}{\pgfqpoint{1.669774in}{2.121560in}}%
\pgfpathcurveto{\pgfqpoint{1.675598in}{2.127384in}}{\pgfqpoint{1.678871in}{2.135284in}}{\pgfqpoint{1.678871in}{2.143520in}}%
\pgfpathcurveto{\pgfqpoint{1.678871in}{2.151757in}}{\pgfqpoint{1.675598in}{2.159657in}}{\pgfqpoint{1.669774in}{2.165481in}}%
\pgfpathcurveto{\pgfqpoint{1.663950in}{2.171305in}}{\pgfqpoint{1.656050in}{2.174577in}}{\pgfqpoint{1.647814in}{2.174577in}}%
\pgfpathcurveto{\pgfqpoint{1.639578in}{2.174577in}}{\pgfqpoint{1.631678in}{2.171305in}}{\pgfqpoint{1.625854in}{2.165481in}}%
\pgfpathcurveto{\pgfqpoint{1.620030in}{2.159657in}}{\pgfqpoint{1.616758in}{2.151757in}}{\pgfqpoint{1.616758in}{2.143520in}}%
\pgfpathcurveto{\pgfqpoint{1.616758in}{2.135284in}}{\pgfqpoint{1.620030in}{2.127384in}}{\pgfqpoint{1.625854in}{2.121560in}}%
\pgfpathcurveto{\pgfqpoint{1.631678in}{2.115736in}}{\pgfqpoint{1.639578in}{2.112464in}}{\pgfqpoint{1.647814in}{2.112464in}}%
\pgfpathclose%
\pgfusepath{stroke,fill}%
\end{pgfscope}%
\begin{pgfscope}%
\pgfpathrectangle{\pgfqpoint{0.100000in}{0.212622in}}{\pgfqpoint{3.696000in}{3.696000in}}%
\pgfusepath{clip}%
\pgfsetbuttcap%
\pgfsetroundjoin%
\definecolor{currentfill}{rgb}{0.121569,0.466667,0.705882}%
\pgfsetfillcolor{currentfill}%
\pgfsetfillopacity{0.300651}%
\pgfsetlinewidth{1.003750pt}%
\definecolor{currentstroke}{rgb}{0.121569,0.466667,0.705882}%
\pgfsetstrokecolor{currentstroke}%
\pgfsetstrokeopacity{0.300651}%
\pgfsetdash{}{0pt}%
\pgfpathmoveto{\pgfqpoint{1.647814in}{2.112464in}}%
\pgfpathcurveto{\pgfqpoint{1.656050in}{2.112464in}}{\pgfqpoint{1.663950in}{2.115736in}}{\pgfqpoint{1.669774in}{2.121560in}}%
\pgfpathcurveto{\pgfqpoint{1.675598in}{2.127384in}}{\pgfqpoint{1.678871in}{2.135284in}}{\pgfqpoint{1.678871in}{2.143520in}}%
\pgfpathcurveto{\pgfqpoint{1.678871in}{2.151757in}}{\pgfqpoint{1.675598in}{2.159657in}}{\pgfqpoint{1.669774in}{2.165481in}}%
\pgfpathcurveto{\pgfqpoint{1.663950in}{2.171305in}}{\pgfqpoint{1.656050in}{2.174577in}}{\pgfqpoint{1.647814in}{2.174577in}}%
\pgfpathcurveto{\pgfqpoint{1.639578in}{2.174577in}}{\pgfqpoint{1.631678in}{2.171305in}}{\pgfqpoint{1.625854in}{2.165481in}}%
\pgfpathcurveto{\pgfqpoint{1.620030in}{2.159657in}}{\pgfqpoint{1.616758in}{2.151757in}}{\pgfqpoint{1.616758in}{2.143520in}}%
\pgfpathcurveto{\pgfqpoint{1.616758in}{2.135284in}}{\pgfqpoint{1.620030in}{2.127384in}}{\pgfqpoint{1.625854in}{2.121560in}}%
\pgfpathcurveto{\pgfqpoint{1.631678in}{2.115736in}}{\pgfqpoint{1.639578in}{2.112464in}}{\pgfqpoint{1.647814in}{2.112464in}}%
\pgfpathclose%
\pgfusepath{stroke,fill}%
\end{pgfscope}%
\begin{pgfscope}%
\pgfpathrectangle{\pgfqpoint{0.100000in}{0.212622in}}{\pgfqpoint{3.696000in}{3.696000in}}%
\pgfusepath{clip}%
\pgfsetbuttcap%
\pgfsetroundjoin%
\definecolor{currentfill}{rgb}{0.121569,0.466667,0.705882}%
\pgfsetfillcolor{currentfill}%
\pgfsetfillopacity{0.300651}%
\pgfsetlinewidth{1.003750pt}%
\definecolor{currentstroke}{rgb}{0.121569,0.466667,0.705882}%
\pgfsetstrokecolor{currentstroke}%
\pgfsetstrokeopacity{0.300651}%
\pgfsetdash{}{0pt}%
\pgfpathmoveto{\pgfqpoint{1.647814in}{2.112464in}}%
\pgfpathcurveto{\pgfqpoint{1.656050in}{2.112464in}}{\pgfqpoint{1.663950in}{2.115736in}}{\pgfqpoint{1.669774in}{2.121560in}}%
\pgfpathcurveto{\pgfqpoint{1.675598in}{2.127384in}}{\pgfqpoint{1.678871in}{2.135284in}}{\pgfqpoint{1.678871in}{2.143520in}}%
\pgfpathcurveto{\pgfqpoint{1.678871in}{2.151757in}}{\pgfqpoint{1.675598in}{2.159657in}}{\pgfqpoint{1.669774in}{2.165481in}}%
\pgfpathcurveto{\pgfqpoint{1.663950in}{2.171305in}}{\pgfqpoint{1.656050in}{2.174577in}}{\pgfqpoint{1.647814in}{2.174577in}}%
\pgfpathcurveto{\pgfqpoint{1.639578in}{2.174577in}}{\pgfqpoint{1.631678in}{2.171305in}}{\pgfqpoint{1.625854in}{2.165481in}}%
\pgfpathcurveto{\pgfqpoint{1.620030in}{2.159657in}}{\pgfqpoint{1.616758in}{2.151757in}}{\pgfqpoint{1.616758in}{2.143520in}}%
\pgfpathcurveto{\pgfqpoint{1.616758in}{2.135284in}}{\pgfqpoint{1.620030in}{2.127384in}}{\pgfqpoint{1.625854in}{2.121560in}}%
\pgfpathcurveto{\pgfqpoint{1.631678in}{2.115736in}}{\pgfqpoint{1.639578in}{2.112464in}}{\pgfqpoint{1.647814in}{2.112464in}}%
\pgfpathclose%
\pgfusepath{stroke,fill}%
\end{pgfscope}%
\begin{pgfscope}%
\pgfpathrectangle{\pgfqpoint{0.100000in}{0.212622in}}{\pgfqpoint{3.696000in}{3.696000in}}%
\pgfusepath{clip}%
\pgfsetbuttcap%
\pgfsetroundjoin%
\definecolor{currentfill}{rgb}{0.121569,0.466667,0.705882}%
\pgfsetfillcolor{currentfill}%
\pgfsetfillopacity{0.300651}%
\pgfsetlinewidth{1.003750pt}%
\definecolor{currentstroke}{rgb}{0.121569,0.466667,0.705882}%
\pgfsetstrokecolor{currentstroke}%
\pgfsetstrokeopacity{0.300651}%
\pgfsetdash{}{0pt}%
\pgfpathmoveto{\pgfqpoint{1.647814in}{2.112464in}}%
\pgfpathcurveto{\pgfqpoint{1.656050in}{2.112464in}}{\pgfqpoint{1.663950in}{2.115736in}}{\pgfqpoint{1.669774in}{2.121560in}}%
\pgfpathcurveto{\pgfqpoint{1.675598in}{2.127384in}}{\pgfqpoint{1.678871in}{2.135284in}}{\pgfqpoint{1.678871in}{2.143520in}}%
\pgfpathcurveto{\pgfqpoint{1.678871in}{2.151757in}}{\pgfqpoint{1.675598in}{2.159657in}}{\pgfqpoint{1.669774in}{2.165481in}}%
\pgfpathcurveto{\pgfqpoint{1.663950in}{2.171305in}}{\pgfqpoint{1.656050in}{2.174577in}}{\pgfqpoint{1.647814in}{2.174577in}}%
\pgfpathcurveto{\pgfqpoint{1.639578in}{2.174577in}}{\pgfqpoint{1.631678in}{2.171305in}}{\pgfqpoint{1.625854in}{2.165481in}}%
\pgfpathcurveto{\pgfqpoint{1.620030in}{2.159657in}}{\pgfqpoint{1.616758in}{2.151757in}}{\pgfqpoint{1.616758in}{2.143520in}}%
\pgfpathcurveto{\pgfqpoint{1.616758in}{2.135284in}}{\pgfqpoint{1.620030in}{2.127384in}}{\pgfqpoint{1.625854in}{2.121560in}}%
\pgfpathcurveto{\pgfqpoint{1.631678in}{2.115736in}}{\pgfqpoint{1.639578in}{2.112464in}}{\pgfqpoint{1.647814in}{2.112464in}}%
\pgfpathclose%
\pgfusepath{stroke,fill}%
\end{pgfscope}%
\begin{pgfscope}%
\pgfpathrectangle{\pgfqpoint{0.100000in}{0.212622in}}{\pgfqpoint{3.696000in}{3.696000in}}%
\pgfusepath{clip}%
\pgfsetbuttcap%
\pgfsetroundjoin%
\definecolor{currentfill}{rgb}{0.121569,0.466667,0.705882}%
\pgfsetfillcolor{currentfill}%
\pgfsetfillopacity{0.300651}%
\pgfsetlinewidth{1.003750pt}%
\definecolor{currentstroke}{rgb}{0.121569,0.466667,0.705882}%
\pgfsetstrokecolor{currentstroke}%
\pgfsetstrokeopacity{0.300651}%
\pgfsetdash{}{0pt}%
\pgfpathmoveto{\pgfqpoint{1.647814in}{2.112464in}}%
\pgfpathcurveto{\pgfqpoint{1.656050in}{2.112464in}}{\pgfqpoint{1.663950in}{2.115736in}}{\pgfqpoint{1.669774in}{2.121560in}}%
\pgfpathcurveto{\pgfqpoint{1.675598in}{2.127384in}}{\pgfqpoint{1.678871in}{2.135284in}}{\pgfqpoint{1.678871in}{2.143520in}}%
\pgfpathcurveto{\pgfqpoint{1.678871in}{2.151757in}}{\pgfqpoint{1.675598in}{2.159657in}}{\pgfqpoint{1.669774in}{2.165481in}}%
\pgfpathcurveto{\pgfqpoint{1.663950in}{2.171305in}}{\pgfqpoint{1.656050in}{2.174577in}}{\pgfqpoint{1.647814in}{2.174577in}}%
\pgfpathcurveto{\pgfqpoint{1.639578in}{2.174577in}}{\pgfqpoint{1.631678in}{2.171305in}}{\pgfqpoint{1.625854in}{2.165481in}}%
\pgfpathcurveto{\pgfqpoint{1.620030in}{2.159657in}}{\pgfqpoint{1.616758in}{2.151757in}}{\pgfqpoint{1.616758in}{2.143520in}}%
\pgfpathcurveto{\pgfqpoint{1.616758in}{2.135284in}}{\pgfqpoint{1.620030in}{2.127384in}}{\pgfqpoint{1.625854in}{2.121560in}}%
\pgfpathcurveto{\pgfqpoint{1.631678in}{2.115736in}}{\pgfqpoint{1.639578in}{2.112464in}}{\pgfqpoint{1.647814in}{2.112464in}}%
\pgfpathclose%
\pgfusepath{stroke,fill}%
\end{pgfscope}%
\begin{pgfscope}%
\pgfpathrectangle{\pgfqpoint{0.100000in}{0.212622in}}{\pgfqpoint{3.696000in}{3.696000in}}%
\pgfusepath{clip}%
\pgfsetbuttcap%
\pgfsetroundjoin%
\definecolor{currentfill}{rgb}{0.121569,0.466667,0.705882}%
\pgfsetfillcolor{currentfill}%
\pgfsetfillopacity{0.300651}%
\pgfsetlinewidth{1.003750pt}%
\definecolor{currentstroke}{rgb}{0.121569,0.466667,0.705882}%
\pgfsetstrokecolor{currentstroke}%
\pgfsetstrokeopacity{0.300651}%
\pgfsetdash{}{0pt}%
\pgfpathmoveto{\pgfqpoint{1.647814in}{2.112464in}}%
\pgfpathcurveto{\pgfqpoint{1.656050in}{2.112464in}}{\pgfqpoint{1.663950in}{2.115736in}}{\pgfqpoint{1.669774in}{2.121560in}}%
\pgfpathcurveto{\pgfqpoint{1.675598in}{2.127384in}}{\pgfqpoint{1.678870in}{2.135284in}}{\pgfqpoint{1.678870in}{2.143520in}}%
\pgfpathcurveto{\pgfqpoint{1.678870in}{2.151757in}}{\pgfqpoint{1.675598in}{2.159657in}}{\pgfqpoint{1.669774in}{2.165480in}}%
\pgfpathcurveto{\pgfqpoint{1.663950in}{2.171304in}}{\pgfqpoint{1.656050in}{2.174577in}}{\pgfqpoint{1.647814in}{2.174577in}}%
\pgfpathcurveto{\pgfqpoint{1.639578in}{2.174577in}}{\pgfqpoint{1.631678in}{2.171304in}}{\pgfqpoint{1.625854in}{2.165480in}}%
\pgfpathcurveto{\pgfqpoint{1.620030in}{2.159657in}}{\pgfqpoint{1.616757in}{2.151757in}}{\pgfqpoint{1.616757in}{2.143520in}}%
\pgfpathcurveto{\pgfqpoint{1.616757in}{2.135284in}}{\pgfqpoint{1.620030in}{2.127384in}}{\pgfqpoint{1.625854in}{2.121560in}}%
\pgfpathcurveto{\pgfqpoint{1.631678in}{2.115736in}}{\pgfqpoint{1.639578in}{2.112464in}}{\pgfqpoint{1.647814in}{2.112464in}}%
\pgfpathclose%
\pgfusepath{stroke,fill}%
\end{pgfscope}%
\begin{pgfscope}%
\pgfpathrectangle{\pgfqpoint{0.100000in}{0.212622in}}{\pgfqpoint{3.696000in}{3.696000in}}%
\pgfusepath{clip}%
\pgfsetbuttcap%
\pgfsetroundjoin%
\definecolor{currentfill}{rgb}{0.121569,0.466667,0.705882}%
\pgfsetfillcolor{currentfill}%
\pgfsetfillopacity{0.300651}%
\pgfsetlinewidth{1.003750pt}%
\definecolor{currentstroke}{rgb}{0.121569,0.466667,0.705882}%
\pgfsetstrokecolor{currentstroke}%
\pgfsetstrokeopacity{0.300651}%
\pgfsetdash{}{0pt}%
\pgfpathmoveto{\pgfqpoint{1.647814in}{2.112464in}}%
\pgfpathcurveto{\pgfqpoint{1.656050in}{2.112464in}}{\pgfqpoint{1.663950in}{2.115736in}}{\pgfqpoint{1.669774in}{2.121560in}}%
\pgfpathcurveto{\pgfqpoint{1.675598in}{2.127384in}}{\pgfqpoint{1.678870in}{2.135284in}}{\pgfqpoint{1.678870in}{2.143520in}}%
\pgfpathcurveto{\pgfqpoint{1.678870in}{2.151756in}}{\pgfqpoint{1.675598in}{2.159657in}}{\pgfqpoint{1.669774in}{2.165480in}}%
\pgfpathcurveto{\pgfqpoint{1.663950in}{2.171304in}}{\pgfqpoint{1.656050in}{2.174577in}}{\pgfqpoint{1.647814in}{2.174577in}}%
\pgfpathcurveto{\pgfqpoint{1.639578in}{2.174577in}}{\pgfqpoint{1.631678in}{2.171304in}}{\pgfqpoint{1.625854in}{2.165480in}}%
\pgfpathcurveto{\pgfqpoint{1.620030in}{2.159657in}}{\pgfqpoint{1.616757in}{2.151756in}}{\pgfqpoint{1.616757in}{2.143520in}}%
\pgfpathcurveto{\pgfqpoint{1.616757in}{2.135284in}}{\pgfqpoint{1.620030in}{2.127384in}}{\pgfqpoint{1.625854in}{2.121560in}}%
\pgfpathcurveto{\pgfqpoint{1.631678in}{2.115736in}}{\pgfqpoint{1.639578in}{2.112464in}}{\pgfqpoint{1.647814in}{2.112464in}}%
\pgfpathclose%
\pgfusepath{stroke,fill}%
\end{pgfscope}%
\begin{pgfscope}%
\pgfpathrectangle{\pgfqpoint{0.100000in}{0.212622in}}{\pgfqpoint{3.696000in}{3.696000in}}%
\pgfusepath{clip}%
\pgfsetbuttcap%
\pgfsetroundjoin%
\definecolor{currentfill}{rgb}{0.121569,0.466667,0.705882}%
\pgfsetfillcolor{currentfill}%
\pgfsetfillopacity{0.300651}%
\pgfsetlinewidth{1.003750pt}%
\definecolor{currentstroke}{rgb}{0.121569,0.466667,0.705882}%
\pgfsetstrokecolor{currentstroke}%
\pgfsetstrokeopacity{0.300651}%
\pgfsetdash{}{0pt}%
\pgfpathmoveto{\pgfqpoint{1.647813in}{2.112463in}}%
\pgfpathcurveto{\pgfqpoint{1.656050in}{2.112463in}}{\pgfqpoint{1.663950in}{2.115736in}}{\pgfqpoint{1.669774in}{2.121560in}}%
\pgfpathcurveto{\pgfqpoint{1.675598in}{2.127384in}}{\pgfqpoint{1.678870in}{2.135284in}}{\pgfqpoint{1.678870in}{2.143520in}}%
\pgfpathcurveto{\pgfqpoint{1.678870in}{2.151756in}}{\pgfqpoint{1.675598in}{2.159656in}}{\pgfqpoint{1.669774in}{2.165480in}}%
\pgfpathcurveto{\pgfqpoint{1.663950in}{2.171304in}}{\pgfqpoint{1.656050in}{2.174576in}}{\pgfqpoint{1.647813in}{2.174576in}}%
\pgfpathcurveto{\pgfqpoint{1.639577in}{2.174576in}}{\pgfqpoint{1.631677in}{2.171304in}}{\pgfqpoint{1.625853in}{2.165480in}}%
\pgfpathcurveto{\pgfqpoint{1.620029in}{2.159656in}}{\pgfqpoint{1.616757in}{2.151756in}}{\pgfqpoint{1.616757in}{2.143520in}}%
\pgfpathcurveto{\pgfqpoint{1.616757in}{2.135284in}}{\pgfqpoint{1.620029in}{2.127384in}}{\pgfqpoint{1.625853in}{2.121560in}}%
\pgfpathcurveto{\pgfqpoint{1.631677in}{2.115736in}}{\pgfqpoint{1.639577in}{2.112463in}}{\pgfqpoint{1.647813in}{2.112463in}}%
\pgfpathclose%
\pgfusepath{stroke,fill}%
\end{pgfscope}%
\begin{pgfscope}%
\pgfpathrectangle{\pgfqpoint{0.100000in}{0.212622in}}{\pgfqpoint{3.696000in}{3.696000in}}%
\pgfusepath{clip}%
\pgfsetbuttcap%
\pgfsetroundjoin%
\definecolor{currentfill}{rgb}{0.121569,0.466667,0.705882}%
\pgfsetfillcolor{currentfill}%
\pgfsetfillopacity{0.300651}%
\pgfsetlinewidth{1.003750pt}%
\definecolor{currentstroke}{rgb}{0.121569,0.466667,0.705882}%
\pgfsetstrokecolor{currentstroke}%
\pgfsetstrokeopacity{0.300651}%
\pgfsetdash{}{0pt}%
\pgfpathmoveto{\pgfqpoint{1.647812in}{2.112462in}}%
\pgfpathcurveto{\pgfqpoint{1.656048in}{2.112462in}}{\pgfqpoint{1.663948in}{2.115735in}}{\pgfqpoint{1.669772in}{2.121559in}}%
\pgfpathcurveto{\pgfqpoint{1.675596in}{2.127383in}}{\pgfqpoint{1.678869in}{2.135283in}}{\pgfqpoint{1.678869in}{2.143519in}}%
\pgfpathcurveto{\pgfqpoint{1.678869in}{2.151755in}}{\pgfqpoint{1.675596in}{2.159655in}}{\pgfqpoint{1.669772in}{2.165479in}}%
\pgfpathcurveto{\pgfqpoint{1.663948in}{2.171303in}}{\pgfqpoint{1.656048in}{2.174575in}}{\pgfqpoint{1.647812in}{2.174575in}}%
\pgfpathcurveto{\pgfqpoint{1.639576in}{2.174575in}}{\pgfqpoint{1.631676in}{2.171303in}}{\pgfqpoint{1.625852in}{2.165479in}}%
\pgfpathcurveto{\pgfqpoint{1.620028in}{2.159655in}}{\pgfqpoint{1.616756in}{2.151755in}}{\pgfqpoint{1.616756in}{2.143519in}}%
\pgfpathcurveto{\pgfqpoint{1.616756in}{2.135283in}}{\pgfqpoint{1.620028in}{2.127383in}}{\pgfqpoint{1.625852in}{2.121559in}}%
\pgfpathcurveto{\pgfqpoint{1.631676in}{2.115735in}}{\pgfqpoint{1.639576in}{2.112462in}}{\pgfqpoint{1.647812in}{2.112462in}}%
\pgfpathclose%
\pgfusepath{stroke,fill}%
\end{pgfscope}%
\begin{pgfscope}%
\pgfpathrectangle{\pgfqpoint{0.100000in}{0.212622in}}{\pgfqpoint{3.696000in}{3.696000in}}%
\pgfusepath{clip}%
\pgfsetbuttcap%
\pgfsetroundjoin%
\definecolor{currentfill}{rgb}{0.121569,0.466667,0.705882}%
\pgfsetfillcolor{currentfill}%
\pgfsetfillopacity{0.300652}%
\pgfsetlinewidth{1.003750pt}%
\definecolor{currentstroke}{rgb}{0.121569,0.466667,0.705882}%
\pgfsetstrokecolor{currentstroke}%
\pgfsetstrokeopacity{0.300652}%
\pgfsetdash{}{0pt}%
\pgfpathmoveto{\pgfqpoint{1.647810in}{2.112461in}}%
\pgfpathcurveto{\pgfqpoint{1.656047in}{2.112461in}}{\pgfqpoint{1.663947in}{2.115733in}}{\pgfqpoint{1.669771in}{2.121557in}}%
\pgfpathcurveto{\pgfqpoint{1.675595in}{2.127381in}}{\pgfqpoint{1.678867in}{2.135281in}}{\pgfqpoint{1.678867in}{2.143518in}}%
\pgfpathcurveto{\pgfqpoint{1.678867in}{2.151754in}}{\pgfqpoint{1.675595in}{2.159654in}}{\pgfqpoint{1.669771in}{2.165478in}}%
\pgfpathcurveto{\pgfqpoint{1.663947in}{2.171302in}}{\pgfqpoint{1.656047in}{2.174574in}}{\pgfqpoint{1.647810in}{2.174574in}}%
\pgfpathcurveto{\pgfqpoint{1.639574in}{2.174574in}}{\pgfqpoint{1.631674in}{2.171302in}}{\pgfqpoint{1.625850in}{2.165478in}}%
\pgfpathcurveto{\pgfqpoint{1.620026in}{2.159654in}}{\pgfqpoint{1.616754in}{2.151754in}}{\pgfqpoint{1.616754in}{2.143518in}}%
\pgfpathcurveto{\pgfqpoint{1.616754in}{2.135281in}}{\pgfqpoint{1.620026in}{2.127381in}}{\pgfqpoint{1.625850in}{2.121557in}}%
\pgfpathcurveto{\pgfqpoint{1.631674in}{2.115733in}}{\pgfqpoint{1.639574in}{2.112461in}}{\pgfqpoint{1.647810in}{2.112461in}}%
\pgfpathclose%
\pgfusepath{stroke,fill}%
\end{pgfscope}%
\begin{pgfscope}%
\pgfpathrectangle{\pgfqpoint{0.100000in}{0.212622in}}{\pgfqpoint{3.696000in}{3.696000in}}%
\pgfusepath{clip}%
\pgfsetbuttcap%
\pgfsetroundjoin%
\definecolor{currentfill}{rgb}{0.121569,0.466667,0.705882}%
\pgfsetfillcolor{currentfill}%
\pgfsetfillopacity{0.300653}%
\pgfsetlinewidth{1.003750pt}%
\definecolor{currentstroke}{rgb}{0.121569,0.466667,0.705882}%
\pgfsetstrokecolor{currentstroke}%
\pgfsetstrokeopacity{0.300653}%
\pgfsetdash{}{0pt}%
\pgfpathmoveto{\pgfqpoint{1.647808in}{2.112459in}}%
\pgfpathcurveto{\pgfqpoint{1.656044in}{2.112459in}}{\pgfqpoint{1.663944in}{2.115731in}}{\pgfqpoint{1.669768in}{2.121555in}}%
\pgfpathcurveto{\pgfqpoint{1.675592in}{2.127379in}}{\pgfqpoint{1.678864in}{2.135279in}}{\pgfqpoint{1.678864in}{2.143515in}}%
\pgfpathcurveto{\pgfqpoint{1.678864in}{2.151751in}}{\pgfqpoint{1.675592in}{2.159652in}}{\pgfqpoint{1.669768in}{2.165475in}}%
\pgfpathcurveto{\pgfqpoint{1.663944in}{2.171299in}}{\pgfqpoint{1.656044in}{2.174572in}}{\pgfqpoint{1.647808in}{2.174572in}}%
\pgfpathcurveto{\pgfqpoint{1.639571in}{2.174572in}}{\pgfqpoint{1.631671in}{2.171299in}}{\pgfqpoint{1.625847in}{2.165475in}}%
\pgfpathcurveto{\pgfqpoint{1.620023in}{2.159652in}}{\pgfqpoint{1.616751in}{2.151751in}}{\pgfqpoint{1.616751in}{2.143515in}}%
\pgfpathcurveto{\pgfqpoint{1.616751in}{2.135279in}}{\pgfqpoint{1.620023in}{2.127379in}}{\pgfqpoint{1.625847in}{2.121555in}}%
\pgfpathcurveto{\pgfqpoint{1.631671in}{2.115731in}}{\pgfqpoint{1.639571in}{2.112459in}}{\pgfqpoint{1.647808in}{2.112459in}}%
\pgfpathclose%
\pgfusepath{stroke,fill}%
\end{pgfscope}%
\begin{pgfscope}%
\pgfpathrectangle{\pgfqpoint{0.100000in}{0.212622in}}{\pgfqpoint{3.696000in}{3.696000in}}%
\pgfusepath{clip}%
\pgfsetbuttcap%
\pgfsetroundjoin%
\definecolor{currentfill}{rgb}{0.121569,0.466667,0.705882}%
\pgfsetfillcolor{currentfill}%
\pgfsetfillopacity{0.300654}%
\pgfsetlinewidth{1.003750pt}%
\definecolor{currentstroke}{rgb}{0.121569,0.466667,0.705882}%
\pgfsetstrokecolor{currentstroke}%
\pgfsetstrokeopacity{0.300654}%
\pgfsetdash{}{0pt}%
\pgfpathmoveto{\pgfqpoint{1.647803in}{2.112456in}}%
\pgfpathcurveto{\pgfqpoint{1.656040in}{2.112456in}}{\pgfqpoint{1.663940in}{2.115729in}}{\pgfqpoint{1.669764in}{2.121553in}}%
\pgfpathcurveto{\pgfqpoint{1.675588in}{2.127377in}}{\pgfqpoint{1.678860in}{2.135277in}}{\pgfqpoint{1.678860in}{2.143513in}}%
\pgfpathcurveto{\pgfqpoint{1.678860in}{2.151749in}}{\pgfqpoint{1.675588in}{2.159649in}}{\pgfqpoint{1.669764in}{2.165473in}}%
\pgfpathcurveto{\pgfqpoint{1.663940in}{2.171297in}}{\pgfqpoint{1.656040in}{2.174569in}}{\pgfqpoint{1.647803in}{2.174569in}}%
\pgfpathcurveto{\pgfqpoint{1.639567in}{2.174569in}}{\pgfqpoint{1.631667in}{2.171297in}}{\pgfqpoint{1.625843in}{2.165473in}}%
\pgfpathcurveto{\pgfqpoint{1.620019in}{2.159649in}}{\pgfqpoint{1.616747in}{2.151749in}}{\pgfqpoint{1.616747in}{2.143513in}}%
\pgfpathcurveto{\pgfqpoint{1.616747in}{2.135277in}}{\pgfqpoint{1.620019in}{2.127377in}}{\pgfqpoint{1.625843in}{2.121553in}}%
\pgfpathcurveto{\pgfqpoint{1.631667in}{2.115729in}}{\pgfqpoint{1.639567in}{2.112456in}}{\pgfqpoint{1.647803in}{2.112456in}}%
\pgfpathclose%
\pgfusepath{stroke,fill}%
\end{pgfscope}%
\begin{pgfscope}%
\pgfpathrectangle{\pgfqpoint{0.100000in}{0.212622in}}{\pgfqpoint{3.696000in}{3.696000in}}%
\pgfusepath{clip}%
\pgfsetbuttcap%
\pgfsetroundjoin%
\definecolor{currentfill}{rgb}{0.121569,0.466667,0.705882}%
\pgfsetfillcolor{currentfill}%
\pgfsetfillopacity{0.300659}%
\pgfsetlinewidth{1.003750pt}%
\definecolor{currentstroke}{rgb}{0.121569,0.466667,0.705882}%
\pgfsetstrokecolor{currentstroke}%
\pgfsetstrokeopacity{0.300659}%
\pgfsetdash{}{0pt}%
\pgfpathmoveto{\pgfqpoint{1.647787in}{2.112442in}}%
\pgfpathcurveto{\pgfqpoint{1.656023in}{2.112442in}}{\pgfqpoint{1.663923in}{2.115714in}}{\pgfqpoint{1.669747in}{2.121538in}}%
\pgfpathcurveto{\pgfqpoint{1.675571in}{2.127362in}}{\pgfqpoint{1.678843in}{2.135262in}}{\pgfqpoint{1.678843in}{2.143498in}}%
\pgfpathcurveto{\pgfqpoint{1.678843in}{2.151734in}}{\pgfqpoint{1.675571in}{2.159634in}}{\pgfqpoint{1.669747in}{2.165458in}}%
\pgfpathcurveto{\pgfqpoint{1.663923in}{2.171282in}}{\pgfqpoint{1.656023in}{2.174555in}}{\pgfqpoint{1.647787in}{2.174555in}}%
\pgfpathcurveto{\pgfqpoint{1.639551in}{2.174555in}}{\pgfqpoint{1.631651in}{2.171282in}}{\pgfqpoint{1.625827in}{2.165458in}}%
\pgfpathcurveto{\pgfqpoint{1.620003in}{2.159634in}}{\pgfqpoint{1.616730in}{2.151734in}}{\pgfqpoint{1.616730in}{2.143498in}}%
\pgfpathcurveto{\pgfqpoint{1.616730in}{2.135262in}}{\pgfqpoint{1.620003in}{2.127362in}}{\pgfqpoint{1.625827in}{2.121538in}}%
\pgfpathcurveto{\pgfqpoint{1.631651in}{2.115714in}}{\pgfqpoint{1.639551in}{2.112442in}}{\pgfqpoint{1.647787in}{2.112442in}}%
\pgfpathclose%
\pgfusepath{stroke,fill}%
\end{pgfscope}%
\begin{pgfscope}%
\pgfpathrectangle{\pgfqpoint{0.100000in}{0.212622in}}{\pgfqpoint{3.696000in}{3.696000in}}%
\pgfusepath{clip}%
\pgfsetbuttcap%
\pgfsetroundjoin%
\definecolor{currentfill}{rgb}{0.121569,0.466667,0.705882}%
\pgfsetfillcolor{currentfill}%
\pgfsetfillopacity{0.300661}%
\pgfsetlinewidth{1.003750pt}%
\definecolor{currentstroke}{rgb}{0.121569,0.466667,0.705882}%
\pgfsetstrokecolor{currentstroke}%
\pgfsetstrokeopacity{0.300661}%
\pgfsetdash{}{0pt}%
\pgfpathmoveto{\pgfqpoint{1.647775in}{2.112436in}}%
\pgfpathcurveto{\pgfqpoint{1.656011in}{2.112436in}}{\pgfqpoint{1.663911in}{2.115708in}}{\pgfqpoint{1.669735in}{2.121532in}}%
\pgfpathcurveto{\pgfqpoint{1.675559in}{2.127356in}}{\pgfqpoint{1.678831in}{2.135256in}}{\pgfqpoint{1.678831in}{2.143492in}}%
\pgfpathcurveto{\pgfqpoint{1.678831in}{2.151729in}}{\pgfqpoint{1.675559in}{2.159629in}}{\pgfqpoint{1.669735in}{2.165453in}}%
\pgfpathcurveto{\pgfqpoint{1.663911in}{2.171277in}}{\pgfqpoint{1.656011in}{2.174549in}}{\pgfqpoint{1.647775in}{2.174549in}}%
\pgfpathcurveto{\pgfqpoint{1.639538in}{2.174549in}}{\pgfqpoint{1.631638in}{2.171277in}}{\pgfqpoint{1.625814in}{2.165453in}}%
\pgfpathcurveto{\pgfqpoint{1.619990in}{2.159629in}}{\pgfqpoint{1.616718in}{2.151729in}}{\pgfqpoint{1.616718in}{2.143492in}}%
\pgfpathcurveto{\pgfqpoint{1.616718in}{2.135256in}}{\pgfqpoint{1.619990in}{2.127356in}}{\pgfqpoint{1.625814in}{2.121532in}}%
\pgfpathcurveto{\pgfqpoint{1.631638in}{2.115708in}}{\pgfqpoint{1.639538in}{2.112436in}}{\pgfqpoint{1.647775in}{2.112436in}}%
\pgfpathclose%
\pgfusepath{stroke,fill}%
\end{pgfscope}%
\begin{pgfscope}%
\pgfpathrectangle{\pgfqpoint{0.100000in}{0.212622in}}{\pgfqpoint{3.696000in}{3.696000in}}%
\pgfusepath{clip}%
\pgfsetbuttcap%
\pgfsetroundjoin%
\definecolor{currentfill}{rgb}{0.121569,0.466667,0.705882}%
\pgfsetfillcolor{currentfill}%
\pgfsetfillopacity{0.300672}%
\pgfsetlinewidth{1.003750pt}%
\definecolor{currentstroke}{rgb}{0.121569,0.466667,0.705882}%
\pgfsetstrokecolor{currentstroke}%
\pgfsetstrokeopacity{0.300672}%
\pgfsetdash{}{0pt}%
\pgfpathmoveto{\pgfqpoint{1.647716in}{2.112391in}}%
\pgfpathcurveto{\pgfqpoint{1.655952in}{2.112391in}}{\pgfqpoint{1.663852in}{2.115664in}}{\pgfqpoint{1.669676in}{2.121487in}}%
\pgfpathcurveto{\pgfqpoint{1.675500in}{2.127311in}}{\pgfqpoint{1.678772in}{2.135211in}}{\pgfqpoint{1.678772in}{2.143448in}}%
\pgfpathcurveto{\pgfqpoint{1.678772in}{2.151684in}}{\pgfqpoint{1.675500in}{2.159584in}}{\pgfqpoint{1.669676in}{2.165408in}}%
\pgfpathcurveto{\pgfqpoint{1.663852in}{2.171232in}}{\pgfqpoint{1.655952in}{2.174504in}}{\pgfqpoint{1.647716in}{2.174504in}}%
\pgfpathcurveto{\pgfqpoint{1.639479in}{2.174504in}}{\pgfqpoint{1.631579in}{2.171232in}}{\pgfqpoint{1.625755in}{2.165408in}}%
\pgfpathcurveto{\pgfqpoint{1.619932in}{2.159584in}}{\pgfqpoint{1.616659in}{2.151684in}}{\pgfqpoint{1.616659in}{2.143448in}}%
\pgfpathcurveto{\pgfqpoint{1.616659in}{2.135211in}}{\pgfqpoint{1.619932in}{2.127311in}}{\pgfqpoint{1.625755in}{2.121487in}}%
\pgfpathcurveto{\pgfqpoint{1.631579in}{2.115664in}}{\pgfqpoint{1.639479in}{2.112391in}}{\pgfqpoint{1.647716in}{2.112391in}}%
\pgfpathclose%
\pgfusepath{stroke,fill}%
\end{pgfscope}%
\begin{pgfscope}%
\pgfpathrectangle{\pgfqpoint{0.100000in}{0.212622in}}{\pgfqpoint{3.696000in}{3.696000in}}%
\pgfusepath{clip}%
\pgfsetbuttcap%
\pgfsetroundjoin%
\definecolor{currentfill}{rgb}{0.121569,0.466667,0.705882}%
\pgfsetfillcolor{currentfill}%
\pgfsetfillopacity{0.300677}%
\pgfsetlinewidth{1.003750pt}%
\definecolor{currentstroke}{rgb}{0.121569,0.466667,0.705882}%
\pgfsetstrokecolor{currentstroke}%
\pgfsetstrokeopacity{0.300677}%
\pgfsetdash{}{0pt}%
\pgfpathmoveto{\pgfqpoint{1.647720in}{2.112384in}}%
\pgfpathcurveto{\pgfqpoint{1.655956in}{2.112384in}}{\pgfqpoint{1.663856in}{2.115657in}}{\pgfqpoint{1.669680in}{2.121481in}}%
\pgfpathcurveto{\pgfqpoint{1.675504in}{2.127304in}}{\pgfqpoint{1.678776in}{2.135204in}}{\pgfqpoint{1.678776in}{2.143441in}}%
\pgfpathcurveto{\pgfqpoint{1.678776in}{2.151677in}}{\pgfqpoint{1.675504in}{2.159577in}}{\pgfqpoint{1.669680in}{2.165401in}}%
\pgfpathcurveto{\pgfqpoint{1.663856in}{2.171225in}}{\pgfqpoint{1.655956in}{2.174497in}}{\pgfqpoint{1.647720in}{2.174497in}}%
\pgfpathcurveto{\pgfqpoint{1.639484in}{2.174497in}}{\pgfqpoint{1.631584in}{2.171225in}}{\pgfqpoint{1.625760in}{2.165401in}}%
\pgfpathcurveto{\pgfqpoint{1.619936in}{2.159577in}}{\pgfqpoint{1.616663in}{2.151677in}}{\pgfqpoint{1.616663in}{2.143441in}}%
\pgfpathcurveto{\pgfqpoint{1.616663in}{2.135204in}}{\pgfqpoint{1.619936in}{2.127304in}}{\pgfqpoint{1.625760in}{2.121481in}}%
\pgfpathcurveto{\pgfqpoint{1.631584in}{2.115657in}}{\pgfqpoint{1.639484in}{2.112384in}}{\pgfqpoint{1.647720in}{2.112384in}}%
\pgfpathclose%
\pgfusepath{stroke,fill}%
\end{pgfscope}%
\begin{pgfscope}%
\pgfpathrectangle{\pgfqpoint{0.100000in}{0.212622in}}{\pgfqpoint{3.696000in}{3.696000in}}%
\pgfusepath{clip}%
\pgfsetbuttcap%
\pgfsetroundjoin%
\definecolor{currentfill}{rgb}{0.121569,0.466667,0.705882}%
\pgfsetfillcolor{currentfill}%
\pgfsetfillopacity{0.300684}%
\pgfsetlinewidth{1.003750pt}%
\definecolor{currentstroke}{rgb}{0.121569,0.466667,0.705882}%
\pgfsetstrokecolor{currentstroke}%
\pgfsetstrokeopacity{0.300684}%
\pgfsetdash{}{0pt}%
\pgfpathmoveto{\pgfqpoint{1.647644in}{2.112331in}}%
\pgfpathcurveto{\pgfqpoint{1.655881in}{2.112331in}}{\pgfqpoint{1.663781in}{2.115603in}}{\pgfqpoint{1.669605in}{2.121427in}}%
\pgfpathcurveto{\pgfqpoint{1.675429in}{2.127251in}}{\pgfqpoint{1.678701in}{2.135151in}}{\pgfqpoint{1.678701in}{2.143388in}}%
\pgfpathcurveto{\pgfqpoint{1.678701in}{2.151624in}}{\pgfqpoint{1.675429in}{2.159524in}}{\pgfqpoint{1.669605in}{2.165348in}}%
\pgfpathcurveto{\pgfqpoint{1.663781in}{2.171172in}}{\pgfqpoint{1.655881in}{2.174444in}}{\pgfqpoint{1.647644in}{2.174444in}}%
\pgfpathcurveto{\pgfqpoint{1.639408in}{2.174444in}}{\pgfqpoint{1.631508in}{2.171172in}}{\pgfqpoint{1.625684in}{2.165348in}}%
\pgfpathcurveto{\pgfqpoint{1.619860in}{2.159524in}}{\pgfqpoint{1.616588in}{2.151624in}}{\pgfqpoint{1.616588in}{2.143388in}}%
\pgfpathcurveto{\pgfqpoint{1.616588in}{2.135151in}}{\pgfqpoint{1.619860in}{2.127251in}}{\pgfqpoint{1.625684in}{2.121427in}}%
\pgfpathcurveto{\pgfqpoint{1.631508in}{2.115603in}}{\pgfqpoint{1.639408in}{2.112331in}}{\pgfqpoint{1.647644in}{2.112331in}}%
\pgfpathclose%
\pgfusepath{stroke,fill}%
\end{pgfscope}%
\begin{pgfscope}%
\pgfpathrectangle{\pgfqpoint{0.100000in}{0.212622in}}{\pgfqpoint{3.696000in}{3.696000in}}%
\pgfusepath{clip}%
\pgfsetbuttcap%
\pgfsetroundjoin%
\definecolor{currentfill}{rgb}{0.121569,0.466667,0.705882}%
\pgfsetfillcolor{currentfill}%
\pgfsetfillopacity{0.300689}%
\pgfsetlinewidth{1.003750pt}%
\definecolor{currentstroke}{rgb}{0.121569,0.466667,0.705882}%
\pgfsetstrokecolor{currentstroke}%
\pgfsetstrokeopacity{0.300689}%
\pgfsetdash{}{0pt}%
\pgfpathmoveto{\pgfqpoint{1.647571in}{2.112305in}}%
\pgfpathcurveto{\pgfqpoint{1.655807in}{2.112305in}}{\pgfqpoint{1.663707in}{2.115577in}}{\pgfqpoint{1.669531in}{2.121401in}}%
\pgfpathcurveto{\pgfqpoint{1.675355in}{2.127225in}}{\pgfqpoint{1.678627in}{2.135125in}}{\pgfqpoint{1.678627in}{2.143361in}}%
\pgfpathcurveto{\pgfqpoint{1.678627in}{2.151598in}}{\pgfqpoint{1.675355in}{2.159498in}}{\pgfqpoint{1.669531in}{2.165322in}}%
\pgfpathcurveto{\pgfqpoint{1.663707in}{2.171146in}}{\pgfqpoint{1.655807in}{2.174418in}}{\pgfqpoint{1.647571in}{2.174418in}}%
\pgfpathcurveto{\pgfqpoint{1.639334in}{2.174418in}}{\pgfqpoint{1.631434in}{2.171146in}}{\pgfqpoint{1.625611in}{2.165322in}}%
\pgfpathcurveto{\pgfqpoint{1.619787in}{2.159498in}}{\pgfqpoint{1.616514in}{2.151598in}}{\pgfqpoint{1.616514in}{2.143361in}}%
\pgfpathcurveto{\pgfqpoint{1.616514in}{2.135125in}}{\pgfqpoint{1.619787in}{2.127225in}}{\pgfqpoint{1.625611in}{2.121401in}}%
\pgfpathcurveto{\pgfqpoint{1.631434in}{2.115577in}}{\pgfqpoint{1.639334in}{2.112305in}}{\pgfqpoint{1.647571in}{2.112305in}}%
\pgfpathclose%
\pgfusepath{stroke,fill}%
\end{pgfscope}%
\begin{pgfscope}%
\pgfpathrectangle{\pgfqpoint{0.100000in}{0.212622in}}{\pgfqpoint{3.696000in}{3.696000in}}%
\pgfusepath{clip}%
\pgfsetbuttcap%
\pgfsetroundjoin%
\definecolor{currentfill}{rgb}{0.121569,0.466667,0.705882}%
\pgfsetfillcolor{currentfill}%
\pgfsetfillopacity{0.300712}%
\pgfsetlinewidth{1.003750pt}%
\definecolor{currentstroke}{rgb}{0.121569,0.466667,0.705882}%
\pgfsetstrokecolor{currentstroke}%
\pgfsetstrokeopacity{0.300712}%
\pgfsetdash{}{0pt}%
\pgfpathmoveto{\pgfqpoint{1.649644in}{2.113535in}}%
\pgfpathcurveto{\pgfqpoint{1.657881in}{2.113535in}}{\pgfqpoint{1.665781in}{2.116807in}}{\pgfqpoint{1.671605in}{2.122631in}}%
\pgfpathcurveto{\pgfqpoint{1.677428in}{2.128455in}}{\pgfqpoint{1.680701in}{2.136355in}}{\pgfqpoint{1.680701in}{2.144591in}}%
\pgfpathcurveto{\pgfqpoint{1.680701in}{2.152828in}}{\pgfqpoint{1.677428in}{2.160728in}}{\pgfqpoint{1.671605in}{2.166551in}}%
\pgfpathcurveto{\pgfqpoint{1.665781in}{2.172375in}}{\pgfqpoint{1.657881in}{2.175648in}}{\pgfqpoint{1.649644in}{2.175648in}}%
\pgfpathcurveto{\pgfqpoint{1.641408in}{2.175648in}}{\pgfqpoint{1.633508in}{2.172375in}}{\pgfqpoint{1.627684in}{2.166551in}}%
\pgfpathcurveto{\pgfqpoint{1.621860in}{2.160728in}}{\pgfqpoint{1.618588in}{2.152828in}}{\pgfqpoint{1.618588in}{2.144591in}}%
\pgfpathcurveto{\pgfqpoint{1.618588in}{2.136355in}}{\pgfqpoint{1.621860in}{2.128455in}}{\pgfqpoint{1.627684in}{2.122631in}}%
\pgfpathcurveto{\pgfqpoint{1.633508in}{2.116807in}}{\pgfqpoint{1.641408in}{2.113535in}}{\pgfqpoint{1.649644in}{2.113535in}}%
\pgfpathclose%
\pgfusepath{stroke,fill}%
\end{pgfscope}%
\begin{pgfscope}%
\pgfpathrectangle{\pgfqpoint{0.100000in}{0.212622in}}{\pgfqpoint{3.696000in}{3.696000in}}%
\pgfusepath{clip}%
\pgfsetbuttcap%
\pgfsetroundjoin%
\definecolor{currentfill}{rgb}{0.121569,0.466667,0.705882}%
\pgfsetfillcolor{currentfill}%
\pgfsetfillopacity{0.300727}%
\pgfsetlinewidth{1.003750pt}%
\definecolor{currentstroke}{rgb}{0.121569,0.466667,0.705882}%
\pgfsetstrokecolor{currentstroke}%
\pgfsetstrokeopacity{0.300727}%
\pgfsetdash{}{0pt}%
\pgfpathmoveto{\pgfqpoint{1.645086in}{2.110574in}}%
\pgfpathcurveto{\pgfqpoint{1.653322in}{2.110574in}}{\pgfqpoint{1.661222in}{2.113846in}}{\pgfqpoint{1.667046in}{2.119670in}}%
\pgfpathcurveto{\pgfqpoint{1.672870in}{2.125494in}}{\pgfqpoint{1.676142in}{2.133394in}}{\pgfqpoint{1.676142in}{2.141630in}}%
\pgfpathcurveto{\pgfqpoint{1.676142in}{2.149867in}}{\pgfqpoint{1.672870in}{2.157767in}}{\pgfqpoint{1.667046in}{2.163591in}}%
\pgfpathcurveto{\pgfqpoint{1.661222in}{2.169415in}}{\pgfqpoint{1.653322in}{2.172687in}}{\pgfqpoint{1.645086in}{2.172687in}}%
\pgfpathcurveto{\pgfqpoint{1.636850in}{2.172687in}}{\pgfqpoint{1.628949in}{2.169415in}}{\pgfqpoint{1.623126in}{2.163591in}}%
\pgfpathcurveto{\pgfqpoint{1.617302in}{2.157767in}}{\pgfqpoint{1.614029in}{2.149867in}}{\pgfqpoint{1.614029in}{2.141630in}}%
\pgfpathcurveto{\pgfqpoint{1.614029in}{2.133394in}}{\pgfqpoint{1.617302in}{2.125494in}}{\pgfqpoint{1.623126in}{2.119670in}}%
\pgfpathcurveto{\pgfqpoint{1.628949in}{2.113846in}}{\pgfqpoint{1.636850in}{2.110574in}}{\pgfqpoint{1.645086in}{2.110574in}}%
\pgfpathclose%
\pgfusepath{stroke,fill}%
\end{pgfscope}%
\begin{pgfscope}%
\pgfpathrectangle{\pgfqpoint{0.100000in}{0.212622in}}{\pgfqpoint{3.696000in}{3.696000in}}%
\pgfusepath{clip}%
\pgfsetbuttcap%
\pgfsetroundjoin%
\definecolor{currentfill}{rgb}{0.121569,0.466667,0.705882}%
\pgfsetfillcolor{currentfill}%
\pgfsetfillopacity{0.300766}%
\pgfsetlinewidth{1.003750pt}%
\definecolor{currentstroke}{rgb}{0.121569,0.466667,0.705882}%
\pgfsetstrokecolor{currentstroke}%
\pgfsetstrokeopacity{0.300766}%
\pgfsetdash{}{0pt}%
\pgfpathmoveto{\pgfqpoint{1.648784in}{2.112925in}}%
\pgfpathcurveto{\pgfqpoint{1.657021in}{2.112925in}}{\pgfqpoint{1.664921in}{2.116197in}}{\pgfqpoint{1.670745in}{2.122021in}}%
\pgfpathcurveto{\pgfqpoint{1.676568in}{2.127845in}}{\pgfqpoint{1.679841in}{2.135745in}}{\pgfqpoint{1.679841in}{2.143981in}}%
\pgfpathcurveto{\pgfqpoint{1.679841in}{2.152217in}}{\pgfqpoint{1.676568in}{2.160117in}}{\pgfqpoint{1.670745in}{2.165941in}}%
\pgfpathcurveto{\pgfqpoint{1.664921in}{2.171765in}}{\pgfqpoint{1.657021in}{2.175038in}}{\pgfqpoint{1.648784in}{2.175038in}}%
\pgfpathcurveto{\pgfqpoint{1.640548in}{2.175038in}}{\pgfqpoint{1.632648in}{2.171765in}}{\pgfqpoint{1.626824in}{2.165941in}}%
\pgfpathcurveto{\pgfqpoint{1.621000in}{2.160117in}}{\pgfqpoint{1.617728in}{2.152217in}}{\pgfqpoint{1.617728in}{2.143981in}}%
\pgfpathcurveto{\pgfqpoint{1.617728in}{2.135745in}}{\pgfqpoint{1.621000in}{2.127845in}}{\pgfqpoint{1.626824in}{2.122021in}}%
\pgfpathcurveto{\pgfqpoint{1.632648in}{2.116197in}}{\pgfqpoint{1.640548in}{2.112925in}}{\pgfqpoint{1.648784in}{2.112925in}}%
\pgfpathclose%
\pgfusepath{stroke,fill}%
\end{pgfscope}%
\begin{pgfscope}%
\pgfpathrectangle{\pgfqpoint{0.100000in}{0.212622in}}{\pgfqpoint{3.696000in}{3.696000in}}%
\pgfusepath{clip}%
\pgfsetbuttcap%
\pgfsetroundjoin%
\definecolor{currentfill}{rgb}{0.121569,0.466667,0.705882}%
\pgfsetfillcolor{currentfill}%
\pgfsetfillopacity{0.300799}%
\pgfsetlinewidth{1.003750pt}%
\definecolor{currentstroke}{rgb}{0.121569,0.466667,0.705882}%
\pgfsetstrokecolor{currentstroke}%
\pgfsetstrokeopacity{0.300799}%
\pgfsetdash{}{0pt}%
\pgfpathmoveto{\pgfqpoint{1.647160in}{2.111960in}}%
\pgfpathcurveto{\pgfqpoint{1.655396in}{2.111960in}}{\pgfqpoint{1.663296in}{2.115232in}}{\pgfqpoint{1.669120in}{2.121056in}}%
\pgfpathcurveto{\pgfqpoint{1.674944in}{2.126880in}}{\pgfqpoint{1.678216in}{2.134780in}}{\pgfqpoint{1.678216in}{2.143016in}}%
\pgfpathcurveto{\pgfqpoint{1.678216in}{2.151253in}}{\pgfqpoint{1.674944in}{2.159153in}}{\pgfqpoint{1.669120in}{2.164977in}}%
\pgfpathcurveto{\pgfqpoint{1.663296in}{2.170801in}}{\pgfqpoint{1.655396in}{2.174073in}}{\pgfqpoint{1.647160in}{2.174073in}}%
\pgfpathcurveto{\pgfqpoint{1.638923in}{2.174073in}}{\pgfqpoint{1.631023in}{2.170801in}}{\pgfqpoint{1.625199in}{2.164977in}}%
\pgfpathcurveto{\pgfqpoint{1.619375in}{2.159153in}}{\pgfqpoint{1.616103in}{2.151253in}}{\pgfqpoint{1.616103in}{2.143016in}}%
\pgfpathcurveto{\pgfqpoint{1.616103in}{2.134780in}}{\pgfqpoint{1.619375in}{2.126880in}}{\pgfqpoint{1.625199in}{2.121056in}}%
\pgfpathcurveto{\pgfqpoint{1.631023in}{2.115232in}}{\pgfqpoint{1.638923in}{2.111960in}}{\pgfqpoint{1.647160in}{2.111960in}}%
\pgfpathclose%
\pgfusepath{stroke,fill}%
\end{pgfscope}%
\begin{pgfscope}%
\pgfpathrectangle{\pgfqpoint{0.100000in}{0.212622in}}{\pgfqpoint{3.696000in}{3.696000in}}%
\pgfusepath{clip}%
\pgfsetbuttcap%
\pgfsetroundjoin%
\definecolor{currentfill}{rgb}{0.121569,0.466667,0.705882}%
\pgfsetfillcolor{currentfill}%
\pgfsetfillopacity{0.300829}%
\pgfsetlinewidth{1.003750pt}%
\definecolor{currentstroke}{rgb}{0.121569,0.466667,0.705882}%
\pgfsetstrokecolor{currentstroke}%
\pgfsetstrokeopacity{0.300829}%
\pgfsetdash{}{0pt}%
\pgfpathmoveto{\pgfqpoint{1.649775in}{2.113659in}}%
\pgfpathcurveto{\pgfqpoint{1.658011in}{2.113659in}}{\pgfqpoint{1.665911in}{2.116931in}}{\pgfqpoint{1.671735in}{2.122755in}}%
\pgfpathcurveto{\pgfqpoint{1.677559in}{2.128579in}}{\pgfqpoint{1.680832in}{2.136479in}}{\pgfqpoint{1.680832in}{2.144715in}}%
\pgfpathcurveto{\pgfqpoint{1.680832in}{2.152951in}}{\pgfqpoint{1.677559in}{2.160852in}}{\pgfqpoint{1.671735in}{2.166675in}}%
\pgfpathcurveto{\pgfqpoint{1.665911in}{2.172499in}}{\pgfqpoint{1.658011in}{2.175772in}}{\pgfqpoint{1.649775in}{2.175772in}}%
\pgfpathcurveto{\pgfqpoint{1.641539in}{2.175772in}}{\pgfqpoint{1.633639in}{2.172499in}}{\pgfqpoint{1.627815in}{2.166675in}}%
\pgfpathcurveto{\pgfqpoint{1.621991in}{2.160852in}}{\pgfqpoint{1.618719in}{2.152951in}}{\pgfqpoint{1.618719in}{2.144715in}}%
\pgfpathcurveto{\pgfqpoint{1.618719in}{2.136479in}}{\pgfqpoint{1.621991in}{2.128579in}}{\pgfqpoint{1.627815in}{2.122755in}}%
\pgfpathcurveto{\pgfqpoint{1.633639in}{2.116931in}}{\pgfqpoint{1.641539in}{2.113659in}}{\pgfqpoint{1.649775in}{2.113659in}}%
\pgfpathclose%
\pgfusepath{stroke,fill}%
\end{pgfscope}%
\begin{pgfscope}%
\pgfpathrectangle{\pgfqpoint{0.100000in}{0.212622in}}{\pgfqpoint{3.696000in}{3.696000in}}%
\pgfusepath{clip}%
\pgfsetbuttcap%
\pgfsetroundjoin%
\definecolor{currentfill}{rgb}{0.121569,0.466667,0.705882}%
\pgfsetfillcolor{currentfill}%
\pgfsetfillopacity{0.300909}%
\pgfsetlinewidth{1.003750pt}%
\definecolor{currentstroke}{rgb}{0.121569,0.466667,0.705882}%
\pgfsetstrokecolor{currentstroke}%
\pgfsetstrokeopacity{0.300909}%
\pgfsetdash{}{0pt}%
\pgfpathmoveto{\pgfqpoint{1.646660in}{2.111594in}}%
\pgfpathcurveto{\pgfqpoint{1.654896in}{2.111594in}}{\pgfqpoint{1.662796in}{2.114866in}}{\pgfqpoint{1.668620in}{2.120690in}}%
\pgfpathcurveto{\pgfqpoint{1.674444in}{2.126514in}}{\pgfqpoint{1.677716in}{2.134414in}}{\pgfqpoint{1.677716in}{2.142651in}}%
\pgfpathcurveto{\pgfqpoint{1.677716in}{2.150887in}}{\pgfqpoint{1.674444in}{2.158787in}}{\pgfqpoint{1.668620in}{2.164611in}}%
\pgfpathcurveto{\pgfqpoint{1.662796in}{2.170435in}}{\pgfqpoint{1.654896in}{2.173707in}}{\pgfqpoint{1.646660in}{2.173707in}}%
\pgfpathcurveto{\pgfqpoint{1.638424in}{2.173707in}}{\pgfqpoint{1.630524in}{2.170435in}}{\pgfqpoint{1.624700in}{2.164611in}}%
\pgfpathcurveto{\pgfqpoint{1.618876in}{2.158787in}}{\pgfqpoint{1.615603in}{2.150887in}}{\pgfqpoint{1.615603in}{2.142651in}}%
\pgfpathcurveto{\pgfqpoint{1.615603in}{2.134414in}}{\pgfqpoint{1.618876in}{2.126514in}}{\pgfqpoint{1.624700in}{2.120690in}}%
\pgfpathcurveto{\pgfqpoint{1.630524in}{2.114866in}}{\pgfqpoint{1.638424in}{2.111594in}}{\pgfqpoint{1.646660in}{2.111594in}}%
\pgfpathclose%
\pgfusepath{stroke,fill}%
\end{pgfscope}%
\begin{pgfscope}%
\pgfpathrectangle{\pgfqpoint{0.100000in}{0.212622in}}{\pgfqpoint{3.696000in}{3.696000in}}%
\pgfusepath{clip}%
\pgfsetbuttcap%
\pgfsetroundjoin%
\definecolor{currentfill}{rgb}{0.121569,0.466667,0.705882}%
\pgfsetfillcolor{currentfill}%
\pgfsetfillopacity{0.301136}%
\pgfsetlinewidth{1.003750pt}%
\definecolor{currentstroke}{rgb}{0.121569,0.466667,0.705882}%
\pgfsetstrokecolor{currentstroke}%
\pgfsetstrokeopacity{0.301136}%
\pgfsetdash{}{0pt}%
\pgfpathmoveto{\pgfqpoint{1.645661in}{2.110805in}}%
\pgfpathcurveto{\pgfqpoint{1.653898in}{2.110805in}}{\pgfqpoint{1.661798in}{2.114078in}}{\pgfqpoint{1.667622in}{2.119902in}}%
\pgfpathcurveto{\pgfqpoint{1.673446in}{2.125725in}}{\pgfqpoint{1.676718in}{2.133626in}}{\pgfqpoint{1.676718in}{2.141862in}}%
\pgfpathcurveto{\pgfqpoint{1.676718in}{2.150098in}}{\pgfqpoint{1.673446in}{2.157998in}}{\pgfqpoint{1.667622in}{2.163822in}}%
\pgfpathcurveto{\pgfqpoint{1.661798in}{2.169646in}}{\pgfqpoint{1.653898in}{2.172918in}}{\pgfqpoint{1.645661in}{2.172918in}}%
\pgfpathcurveto{\pgfqpoint{1.637425in}{2.172918in}}{\pgfqpoint{1.629525in}{2.169646in}}{\pgfqpoint{1.623701in}{2.163822in}}%
\pgfpathcurveto{\pgfqpoint{1.617877in}{2.157998in}}{\pgfqpoint{1.614605in}{2.150098in}}{\pgfqpoint{1.614605in}{2.141862in}}%
\pgfpathcurveto{\pgfqpoint{1.614605in}{2.133626in}}{\pgfqpoint{1.617877in}{2.125725in}}{\pgfqpoint{1.623701in}{2.119902in}}%
\pgfpathcurveto{\pgfqpoint{1.629525in}{2.114078in}}{\pgfqpoint{1.637425in}{2.110805in}}{\pgfqpoint{1.645661in}{2.110805in}}%
\pgfpathclose%
\pgfusepath{stroke,fill}%
\end{pgfscope}%
\begin{pgfscope}%
\pgfpathrectangle{\pgfqpoint{0.100000in}{0.212622in}}{\pgfqpoint{3.696000in}{3.696000in}}%
\pgfusepath{clip}%
\pgfsetbuttcap%
\pgfsetroundjoin%
\definecolor{currentfill}{rgb}{0.121569,0.466667,0.705882}%
\pgfsetfillcolor{currentfill}%
\pgfsetfillopacity{0.301276}%
\pgfsetlinewidth{1.003750pt}%
\definecolor{currentstroke}{rgb}{0.121569,0.466667,0.705882}%
\pgfsetstrokecolor{currentstroke}%
\pgfsetstrokeopacity{0.301276}%
\pgfsetdash{}{0pt}%
\pgfpathmoveto{\pgfqpoint{1.644055in}{2.109588in}}%
\pgfpathcurveto{\pgfqpoint{1.652291in}{2.109588in}}{\pgfqpoint{1.660191in}{2.112861in}}{\pgfqpoint{1.666015in}{2.118684in}}%
\pgfpathcurveto{\pgfqpoint{1.671839in}{2.124508in}}{\pgfqpoint{1.675112in}{2.132408in}}{\pgfqpoint{1.675112in}{2.140645in}}%
\pgfpathcurveto{\pgfqpoint{1.675112in}{2.148881in}}{\pgfqpoint{1.671839in}{2.156781in}}{\pgfqpoint{1.666015in}{2.162605in}}%
\pgfpathcurveto{\pgfqpoint{1.660191in}{2.168429in}}{\pgfqpoint{1.652291in}{2.171701in}}{\pgfqpoint{1.644055in}{2.171701in}}%
\pgfpathcurveto{\pgfqpoint{1.635819in}{2.171701in}}{\pgfqpoint{1.627919in}{2.168429in}}{\pgfqpoint{1.622095in}{2.162605in}}%
\pgfpathcurveto{\pgfqpoint{1.616271in}{2.156781in}}{\pgfqpoint{1.612999in}{2.148881in}}{\pgfqpoint{1.612999in}{2.140645in}}%
\pgfpathcurveto{\pgfqpoint{1.612999in}{2.132408in}}{\pgfqpoint{1.616271in}{2.124508in}}{\pgfqpoint{1.622095in}{2.118684in}}%
\pgfpathcurveto{\pgfqpoint{1.627919in}{2.112861in}}{\pgfqpoint{1.635819in}{2.109588in}}{\pgfqpoint{1.644055in}{2.109588in}}%
\pgfpathclose%
\pgfusepath{stroke,fill}%
\end{pgfscope}%
\begin{pgfscope}%
\pgfpathrectangle{\pgfqpoint{0.100000in}{0.212622in}}{\pgfqpoint{3.696000in}{3.696000in}}%
\pgfusepath{clip}%
\pgfsetbuttcap%
\pgfsetroundjoin%
\definecolor{currentfill}{rgb}{0.121569,0.466667,0.705882}%
\pgfsetfillcolor{currentfill}%
\pgfsetfillopacity{0.301367}%
\pgfsetlinewidth{1.003750pt}%
\definecolor{currentstroke}{rgb}{0.121569,0.466667,0.705882}%
\pgfsetstrokecolor{currentstroke}%
\pgfsetstrokeopacity{0.301367}%
\pgfsetdash{}{0pt}%
\pgfpathmoveto{\pgfqpoint{1.644075in}{2.109677in}}%
\pgfpathcurveto{\pgfqpoint{1.652311in}{2.109677in}}{\pgfqpoint{1.660211in}{2.112949in}}{\pgfqpoint{1.666035in}{2.118773in}}%
\pgfpathcurveto{\pgfqpoint{1.671859in}{2.124597in}}{\pgfqpoint{1.675131in}{2.132497in}}{\pgfqpoint{1.675131in}{2.140733in}}%
\pgfpathcurveto{\pgfqpoint{1.675131in}{2.148970in}}{\pgfqpoint{1.671859in}{2.156870in}}{\pgfqpoint{1.666035in}{2.162694in}}%
\pgfpathcurveto{\pgfqpoint{1.660211in}{2.168518in}}{\pgfqpoint{1.652311in}{2.171790in}}{\pgfqpoint{1.644075in}{2.171790in}}%
\pgfpathcurveto{\pgfqpoint{1.635838in}{2.171790in}}{\pgfqpoint{1.627938in}{2.168518in}}{\pgfqpoint{1.622114in}{2.162694in}}%
\pgfpathcurveto{\pgfqpoint{1.616290in}{2.156870in}}{\pgfqpoint{1.613018in}{2.148970in}}{\pgfqpoint{1.613018in}{2.140733in}}%
\pgfpathcurveto{\pgfqpoint{1.613018in}{2.132497in}}{\pgfqpoint{1.616290in}{2.124597in}}{\pgfqpoint{1.622114in}{2.118773in}}%
\pgfpathcurveto{\pgfqpoint{1.627938in}{2.112949in}}{\pgfqpoint{1.635838in}{2.109677in}}{\pgfqpoint{1.644075in}{2.109677in}}%
\pgfpathclose%
\pgfusepath{stroke,fill}%
\end{pgfscope}%
\begin{pgfscope}%
\pgfpathrectangle{\pgfqpoint{0.100000in}{0.212622in}}{\pgfqpoint{3.696000in}{3.696000in}}%
\pgfusepath{clip}%
\pgfsetbuttcap%
\pgfsetroundjoin%
\definecolor{currentfill}{rgb}{0.121569,0.466667,0.705882}%
\pgfsetfillcolor{currentfill}%
\pgfsetfillopacity{0.301445}%
\pgfsetlinewidth{1.003750pt}%
\definecolor{currentstroke}{rgb}{0.121569,0.466667,0.705882}%
\pgfsetstrokecolor{currentstroke}%
\pgfsetstrokeopacity{0.301445}%
\pgfsetdash{}{0pt}%
\pgfpathmoveto{\pgfqpoint{1.643984in}{2.109553in}}%
\pgfpathcurveto{\pgfqpoint{1.652220in}{2.109553in}}{\pgfqpoint{1.660120in}{2.112826in}}{\pgfqpoint{1.665944in}{2.118649in}}%
\pgfpathcurveto{\pgfqpoint{1.671768in}{2.124473in}}{\pgfqpoint{1.675041in}{2.132373in}}{\pgfqpoint{1.675041in}{2.140610in}}%
\pgfpathcurveto{\pgfqpoint{1.675041in}{2.148846in}}{\pgfqpoint{1.671768in}{2.156746in}}{\pgfqpoint{1.665944in}{2.162570in}}%
\pgfpathcurveto{\pgfqpoint{1.660120in}{2.168394in}}{\pgfqpoint{1.652220in}{2.171666in}}{\pgfqpoint{1.643984in}{2.171666in}}%
\pgfpathcurveto{\pgfqpoint{1.635748in}{2.171666in}}{\pgfqpoint{1.627848in}{2.168394in}}{\pgfqpoint{1.622024in}{2.162570in}}%
\pgfpathcurveto{\pgfqpoint{1.616200in}{2.156746in}}{\pgfqpoint{1.612928in}{2.148846in}}{\pgfqpoint{1.612928in}{2.140610in}}%
\pgfpathcurveto{\pgfqpoint{1.612928in}{2.132373in}}{\pgfqpoint{1.616200in}{2.124473in}}{\pgfqpoint{1.622024in}{2.118649in}}%
\pgfpathcurveto{\pgfqpoint{1.627848in}{2.112826in}}{\pgfqpoint{1.635748in}{2.109553in}}{\pgfqpoint{1.643984in}{2.109553in}}%
\pgfpathclose%
\pgfusepath{stroke,fill}%
\end{pgfscope}%
\begin{pgfscope}%
\pgfpathrectangle{\pgfqpoint{0.100000in}{0.212622in}}{\pgfqpoint{3.696000in}{3.696000in}}%
\pgfusepath{clip}%
\pgfsetbuttcap%
\pgfsetroundjoin%
\definecolor{currentfill}{rgb}{0.121569,0.466667,0.705882}%
\pgfsetfillcolor{currentfill}%
\pgfsetfillopacity{0.301449}%
\pgfsetlinewidth{1.003750pt}%
\definecolor{currentstroke}{rgb}{0.121569,0.466667,0.705882}%
\pgfsetstrokecolor{currentstroke}%
\pgfsetstrokeopacity{0.301449}%
\pgfsetdash{}{0pt}%
\pgfpathmoveto{\pgfqpoint{1.644116in}{2.109663in}}%
\pgfpathcurveto{\pgfqpoint{1.652352in}{2.109663in}}{\pgfqpoint{1.660253in}{2.112935in}}{\pgfqpoint{1.666076in}{2.118759in}}%
\pgfpathcurveto{\pgfqpoint{1.671900in}{2.124583in}}{\pgfqpoint{1.675173in}{2.132483in}}{\pgfqpoint{1.675173in}{2.140719in}}%
\pgfpathcurveto{\pgfqpoint{1.675173in}{2.148956in}}{\pgfqpoint{1.671900in}{2.156856in}}{\pgfqpoint{1.666076in}{2.162680in}}%
\pgfpathcurveto{\pgfqpoint{1.660253in}{2.168503in}}{\pgfqpoint{1.652352in}{2.171776in}}{\pgfqpoint{1.644116in}{2.171776in}}%
\pgfpathcurveto{\pgfqpoint{1.635880in}{2.171776in}}{\pgfqpoint{1.627980in}{2.168503in}}{\pgfqpoint{1.622156in}{2.162680in}}%
\pgfpathcurveto{\pgfqpoint{1.616332in}{2.156856in}}{\pgfqpoint{1.613060in}{2.148956in}}{\pgfqpoint{1.613060in}{2.140719in}}%
\pgfpathcurveto{\pgfqpoint{1.613060in}{2.132483in}}{\pgfqpoint{1.616332in}{2.124583in}}{\pgfqpoint{1.622156in}{2.118759in}}%
\pgfpathcurveto{\pgfqpoint{1.627980in}{2.112935in}}{\pgfqpoint{1.635880in}{2.109663in}}{\pgfqpoint{1.644116in}{2.109663in}}%
\pgfpathclose%
\pgfusepath{stroke,fill}%
\end{pgfscope}%
\begin{pgfscope}%
\pgfpathrectangle{\pgfqpoint{0.100000in}{0.212622in}}{\pgfqpoint{3.696000in}{3.696000in}}%
\pgfusepath{clip}%
\pgfsetbuttcap%
\pgfsetroundjoin%
\definecolor{currentfill}{rgb}{0.121569,0.466667,0.705882}%
\pgfsetfillcolor{currentfill}%
\pgfsetfillopacity{0.301449}%
\pgfsetlinewidth{1.003750pt}%
\definecolor{currentstroke}{rgb}{0.121569,0.466667,0.705882}%
\pgfsetstrokecolor{currentstroke}%
\pgfsetstrokeopacity{0.301449}%
\pgfsetdash{}{0pt}%
\pgfpathmoveto{\pgfqpoint{1.644123in}{2.109667in}}%
\pgfpathcurveto{\pgfqpoint{1.652359in}{2.109667in}}{\pgfqpoint{1.660259in}{2.112939in}}{\pgfqpoint{1.666083in}{2.118763in}}%
\pgfpathcurveto{\pgfqpoint{1.671907in}{2.124587in}}{\pgfqpoint{1.675180in}{2.132487in}}{\pgfqpoint{1.675180in}{2.140724in}}%
\pgfpathcurveto{\pgfqpoint{1.675180in}{2.148960in}}{\pgfqpoint{1.671907in}{2.156860in}}{\pgfqpoint{1.666083in}{2.162684in}}%
\pgfpathcurveto{\pgfqpoint{1.660259in}{2.168508in}}{\pgfqpoint{1.652359in}{2.171780in}}{\pgfqpoint{1.644123in}{2.171780in}}%
\pgfpathcurveto{\pgfqpoint{1.635887in}{2.171780in}}{\pgfqpoint{1.627987in}{2.168508in}}{\pgfqpoint{1.622163in}{2.162684in}}%
\pgfpathcurveto{\pgfqpoint{1.616339in}{2.156860in}}{\pgfqpoint{1.613067in}{2.148960in}}{\pgfqpoint{1.613067in}{2.140724in}}%
\pgfpathcurveto{\pgfqpoint{1.613067in}{2.132487in}}{\pgfqpoint{1.616339in}{2.124587in}}{\pgfqpoint{1.622163in}{2.118763in}}%
\pgfpathcurveto{\pgfqpoint{1.627987in}{2.112939in}}{\pgfqpoint{1.635887in}{2.109667in}}{\pgfqpoint{1.644123in}{2.109667in}}%
\pgfpathclose%
\pgfusepath{stroke,fill}%
\end{pgfscope}%
\begin{pgfscope}%
\pgfpathrectangle{\pgfqpoint{0.100000in}{0.212622in}}{\pgfqpoint{3.696000in}{3.696000in}}%
\pgfusepath{clip}%
\pgfsetbuttcap%
\pgfsetroundjoin%
\definecolor{currentfill}{rgb}{0.121569,0.466667,0.705882}%
\pgfsetfillcolor{currentfill}%
\pgfsetfillopacity{0.301449}%
\pgfsetlinewidth{1.003750pt}%
\definecolor{currentstroke}{rgb}{0.121569,0.466667,0.705882}%
\pgfsetstrokecolor{currentstroke}%
\pgfsetstrokeopacity{0.301449}%
\pgfsetdash{}{0pt}%
\pgfpathmoveto{\pgfqpoint{1.644123in}{2.109667in}}%
\pgfpathcurveto{\pgfqpoint{1.652359in}{2.109667in}}{\pgfqpoint{1.660259in}{2.112939in}}{\pgfqpoint{1.666083in}{2.118763in}}%
\pgfpathcurveto{\pgfqpoint{1.671907in}{2.124587in}}{\pgfqpoint{1.675180in}{2.132487in}}{\pgfqpoint{1.675180in}{2.140724in}}%
\pgfpathcurveto{\pgfqpoint{1.675180in}{2.148960in}}{\pgfqpoint{1.671907in}{2.156860in}}{\pgfqpoint{1.666083in}{2.162684in}}%
\pgfpathcurveto{\pgfqpoint{1.660259in}{2.168508in}}{\pgfqpoint{1.652359in}{2.171780in}}{\pgfqpoint{1.644123in}{2.171780in}}%
\pgfpathcurveto{\pgfqpoint{1.635887in}{2.171780in}}{\pgfqpoint{1.627987in}{2.168508in}}{\pgfqpoint{1.622163in}{2.162684in}}%
\pgfpathcurveto{\pgfqpoint{1.616339in}{2.156860in}}{\pgfqpoint{1.613067in}{2.148960in}}{\pgfqpoint{1.613067in}{2.140724in}}%
\pgfpathcurveto{\pgfqpoint{1.613067in}{2.132487in}}{\pgfqpoint{1.616339in}{2.124587in}}{\pgfqpoint{1.622163in}{2.118763in}}%
\pgfpathcurveto{\pgfqpoint{1.627987in}{2.112939in}}{\pgfqpoint{1.635887in}{2.109667in}}{\pgfqpoint{1.644123in}{2.109667in}}%
\pgfpathclose%
\pgfusepath{stroke,fill}%
\end{pgfscope}%
\begin{pgfscope}%
\pgfpathrectangle{\pgfqpoint{0.100000in}{0.212622in}}{\pgfqpoint{3.696000in}{3.696000in}}%
\pgfusepath{clip}%
\pgfsetbuttcap%
\pgfsetroundjoin%
\definecolor{currentfill}{rgb}{0.121569,0.466667,0.705882}%
\pgfsetfillcolor{currentfill}%
\pgfsetfillopacity{0.301449}%
\pgfsetlinewidth{1.003750pt}%
\definecolor{currentstroke}{rgb}{0.121569,0.466667,0.705882}%
\pgfsetstrokecolor{currentstroke}%
\pgfsetstrokeopacity{0.301449}%
\pgfsetdash{}{0pt}%
\pgfpathmoveto{\pgfqpoint{1.644123in}{2.109667in}}%
\pgfpathcurveto{\pgfqpoint{1.652359in}{2.109667in}}{\pgfqpoint{1.660259in}{2.112939in}}{\pgfqpoint{1.666083in}{2.118763in}}%
\pgfpathcurveto{\pgfqpoint{1.671907in}{2.124587in}}{\pgfqpoint{1.675180in}{2.132487in}}{\pgfqpoint{1.675180in}{2.140724in}}%
\pgfpathcurveto{\pgfqpoint{1.675180in}{2.148960in}}{\pgfqpoint{1.671907in}{2.156860in}}{\pgfqpoint{1.666083in}{2.162684in}}%
\pgfpathcurveto{\pgfqpoint{1.660259in}{2.168508in}}{\pgfqpoint{1.652359in}{2.171780in}}{\pgfqpoint{1.644123in}{2.171780in}}%
\pgfpathcurveto{\pgfqpoint{1.635887in}{2.171780in}}{\pgfqpoint{1.627987in}{2.168508in}}{\pgfqpoint{1.622163in}{2.162684in}}%
\pgfpathcurveto{\pgfqpoint{1.616339in}{2.156860in}}{\pgfqpoint{1.613067in}{2.148960in}}{\pgfqpoint{1.613067in}{2.140724in}}%
\pgfpathcurveto{\pgfqpoint{1.613067in}{2.132487in}}{\pgfqpoint{1.616339in}{2.124587in}}{\pgfqpoint{1.622163in}{2.118763in}}%
\pgfpathcurveto{\pgfqpoint{1.627987in}{2.112939in}}{\pgfqpoint{1.635887in}{2.109667in}}{\pgfqpoint{1.644123in}{2.109667in}}%
\pgfpathclose%
\pgfusepath{stroke,fill}%
\end{pgfscope}%
\begin{pgfscope}%
\pgfpathrectangle{\pgfqpoint{0.100000in}{0.212622in}}{\pgfqpoint{3.696000in}{3.696000in}}%
\pgfusepath{clip}%
\pgfsetbuttcap%
\pgfsetroundjoin%
\definecolor{currentfill}{rgb}{0.121569,0.466667,0.705882}%
\pgfsetfillcolor{currentfill}%
\pgfsetfillopacity{0.301449}%
\pgfsetlinewidth{1.003750pt}%
\definecolor{currentstroke}{rgb}{0.121569,0.466667,0.705882}%
\pgfsetstrokecolor{currentstroke}%
\pgfsetstrokeopacity{0.301449}%
\pgfsetdash{}{0pt}%
\pgfpathmoveto{\pgfqpoint{1.644123in}{2.109667in}}%
\pgfpathcurveto{\pgfqpoint{1.652359in}{2.109667in}}{\pgfqpoint{1.660259in}{2.112939in}}{\pgfqpoint{1.666083in}{2.118763in}}%
\pgfpathcurveto{\pgfqpoint{1.671907in}{2.124587in}}{\pgfqpoint{1.675180in}{2.132487in}}{\pgfqpoint{1.675180in}{2.140724in}}%
\pgfpathcurveto{\pgfqpoint{1.675180in}{2.148960in}}{\pgfqpoint{1.671907in}{2.156860in}}{\pgfqpoint{1.666083in}{2.162684in}}%
\pgfpathcurveto{\pgfqpoint{1.660259in}{2.168508in}}{\pgfqpoint{1.652359in}{2.171780in}}{\pgfqpoint{1.644123in}{2.171780in}}%
\pgfpathcurveto{\pgfqpoint{1.635887in}{2.171780in}}{\pgfqpoint{1.627987in}{2.168508in}}{\pgfqpoint{1.622163in}{2.162684in}}%
\pgfpathcurveto{\pgfqpoint{1.616339in}{2.156860in}}{\pgfqpoint{1.613067in}{2.148960in}}{\pgfqpoint{1.613067in}{2.140724in}}%
\pgfpathcurveto{\pgfqpoint{1.613067in}{2.132487in}}{\pgfqpoint{1.616339in}{2.124587in}}{\pgfqpoint{1.622163in}{2.118763in}}%
\pgfpathcurveto{\pgfqpoint{1.627987in}{2.112939in}}{\pgfqpoint{1.635887in}{2.109667in}}{\pgfqpoint{1.644123in}{2.109667in}}%
\pgfpathclose%
\pgfusepath{stroke,fill}%
\end{pgfscope}%
\begin{pgfscope}%
\pgfpathrectangle{\pgfqpoint{0.100000in}{0.212622in}}{\pgfqpoint{3.696000in}{3.696000in}}%
\pgfusepath{clip}%
\pgfsetbuttcap%
\pgfsetroundjoin%
\definecolor{currentfill}{rgb}{0.121569,0.466667,0.705882}%
\pgfsetfillcolor{currentfill}%
\pgfsetfillopacity{0.301449}%
\pgfsetlinewidth{1.003750pt}%
\definecolor{currentstroke}{rgb}{0.121569,0.466667,0.705882}%
\pgfsetstrokecolor{currentstroke}%
\pgfsetstrokeopacity{0.301449}%
\pgfsetdash{}{0pt}%
\pgfpathmoveto{\pgfqpoint{1.644123in}{2.109667in}}%
\pgfpathcurveto{\pgfqpoint{1.652359in}{2.109667in}}{\pgfqpoint{1.660259in}{2.112939in}}{\pgfqpoint{1.666083in}{2.118763in}}%
\pgfpathcurveto{\pgfqpoint{1.671907in}{2.124587in}}{\pgfqpoint{1.675180in}{2.132487in}}{\pgfqpoint{1.675180in}{2.140724in}}%
\pgfpathcurveto{\pgfqpoint{1.675180in}{2.148960in}}{\pgfqpoint{1.671907in}{2.156860in}}{\pgfqpoint{1.666083in}{2.162684in}}%
\pgfpathcurveto{\pgfqpoint{1.660259in}{2.168508in}}{\pgfqpoint{1.652359in}{2.171780in}}{\pgfqpoint{1.644123in}{2.171780in}}%
\pgfpathcurveto{\pgfqpoint{1.635887in}{2.171780in}}{\pgfqpoint{1.627987in}{2.168508in}}{\pgfqpoint{1.622163in}{2.162684in}}%
\pgfpathcurveto{\pgfqpoint{1.616339in}{2.156860in}}{\pgfqpoint{1.613067in}{2.148960in}}{\pgfqpoint{1.613067in}{2.140724in}}%
\pgfpathcurveto{\pgfqpoint{1.613067in}{2.132487in}}{\pgfqpoint{1.616339in}{2.124587in}}{\pgfqpoint{1.622163in}{2.118763in}}%
\pgfpathcurveto{\pgfqpoint{1.627987in}{2.112939in}}{\pgfqpoint{1.635887in}{2.109667in}}{\pgfqpoint{1.644123in}{2.109667in}}%
\pgfpathclose%
\pgfusepath{stroke,fill}%
\end{pgfscope}%
\begin{pgfscope}%
\pgfpathrectangle{\pgfqpoint{0.100000in}{0.212622in}}{\pgfqpoint{3.696000in}{3.696000in}}%
\pgfusepath{clip}%
\pgfsetbuttcap%
\pgfsetroundjoin%
\definecolor{currentfill}{rgb}{0.121569,0.466667,0.705882}%
\pgfsetfillcolor{currentfill}%
\pgfsetfillopacity{0.301449}%
\pgfsetlinewidth{1.003750pt}%
\definecolor{currentstroke}{rgb}{0.121569,0.466667,0.705882}%
\pgfsetstrokecolor{currentstroke}%
\pgfsetstrokeopacity{0.301449}%
\pgfsetdash{}{0pt}%
\pgfpathmoveto{\pgfqpoint{1.644123in}{2.109667in}}%
\pgfpathcurveto{\pgfqpoint{1.652359in}{2.109667in}}{\pgfqpoint{1.660259in}{2.112939in}}{\pgfqpoint{1.666083in}{2.118763in}}%
\pgfpathcurveto{\pgfqpoint{1.671907in}{2.124587in}}{\pgfqpoint{1.675180in}{2.132487in}}{\pgfqpoint{1.675180in}{2.140724in}}%
\pgfpathcurveto{\pgfqpoint{1.675180in}{2.148960in}}{\pgfqpoint{1.671907in}{2.156860in}}{\pgfqpoint{1.666083in}{2.162684in}}%
\pgfpathcurveto{\pgfqpoint{1.660259in}{2.168508in}}{\pgfqpoint{1.652359in}{2.171780in}}{\pgfqpoint{1.644123in}{2.171780in}}%
\pgfpathcurveto{\pgfqpoint{1.635887in}{2.171780in}}{\pgfqpoint{1.627987in}{2.168508in}}{\pgfqpoint{1.622163in}{2.162684in}}%
\pgfpathcurveto{\pgfqpoint{1.616339in}{2.156860in}}{\pgfqpoint{1.613067in}{2.148960in}}{\pgfqpoint{1.613067in}{2.140724in}}%
\pgfpathcurveto{\pgfqpoint{1.613067in}{2.132487in}}{\pgfqpoint{1.616339in}{2.124587in}}{\pgfqpoint{1.622163in}{2.118763in}}%
\pgfpathcurveto{\pgfqpoint{1.627987in}{2.112939in}}{\pgfqpoint{1.635887in}{2.109667in}}{\pgfqpoint{1.644123in}{2.109667in}}%
\pgfpathclose%
\pgfusepath{stroke,fill}%
\end{pgfscope}%
\begin{pgfscope}%
\pgfpathrectangle{\pgfqpoint{0.100000in}{0.212622in}}{\pgfqpoint{3.696000in}{3.696000in}}%
\pgfusepath{clip}%
\pgfsetbuttcap%
\pgfsetroundjoin%
\definecolor{currentfill}{rgb}{0.121569,0.466667,0.705882}%
\pgfsetfillcolor{currentfill}%
\pgfsetfillopacity{0.301449}%
\pgfsetlinewidth{1.003750pt}%
\definecolor{currentstroke}{rgb}{0.121569,0.466667,0.705882}%
\pgfsetstrokecolor{currentstroke}%
\pgfsetstrokeopacity{0.301449}%
\pgfsetdash{}{0pt}%
\pgfpathmoveto{\pgfqpoint{1.644123in}{2.109667in}}%
\pgfpathcurveto{\pgfqpoint{1.652359in}{2.109667in}}{\pgfqpoint{1.660259in}{2.112939in}}{\pgfqpoint{1.666083in}{2.118763in}}%
\pgfpathcurveto{\pgfqpoint{1.671907in}{2.124587in}}{\pgfqpoint{1.675180in}{2.132487in}}{\pgfqpoint{1.675180in}{2.140724in}}%
\pgfpathcurveto{\pgfqpoint{1.675180in}{2.148960in}}{\pgfqpoint{1.671907in}{2.156860in}}{\pgfqpoint{1.666083in}{2.162684in}}%
\pgfpathcurveto{\pgfqpoint{1.660259in}{2.168508in}}{\pgfqpoint{1.652359in}{2.171780in}}{\pgfqpoint{1.644123in}{2.171780in}}%
\pgfpathcurveto{\pgfqpoint{1.635887in}{2.171780in}}{\pgfqpoint{1.627987in}{2.168508in}}{\pgfqpoint{1.622163in}{2.162684in}}%
\pgfpathcurveto{\pgfqpoint{1.616339in}{2.156860in}}{\pgfqpoint{1.613067in}{2.148960in}}{\pgfqpoint{1.613067in}{2.140724in}}%
\pgfpathcurveto{\pgfqpoint{1.613067in}{2.132487in}}{\pgfqpoint{1.616339in}{2.124587in}}{\pgfqpoint{1.622163in}{2.118763in}}%
\pgfpathcurveto{\pgfqpoint{1.627987in}{2.112939in}}{\pgfqpoint{1.635887in}{2.109667in}}{\pgfqpoint{1.644123in}{2.109667in}}%
\pgfpathclose%
\pgfusepath{stroke,fill}%
\end{pgfscope}%
\begin{pgfscope}%
\pgfpathrectangle{\pgfqpoint{0.100000in}{0.212622in}}{\pgfqpoint{3.696000in}{3.696000in}}%
\pgfusepath{clip}%
\pgfsetbuttcap%
\pgfsetroundjoin%
\definecolor{currentfill}{rgb}{0.121569,0.466667,0.705882}%
\pgfsetfillcolor{currentfill}%
\pgfsetfillopacity{0.301449}%
\pgfsetlinewidth{1.003750pt}%
\definecolor{currentstroke}{rgb}{0.121569,0.466667,0.705882}%
\pgfsetstrokecolor{currentstroke}%
\pgfsetstrokeopacity{0.301449}%
\pgfsetdash{}{0pt}%
\pgfpathmoveto{\pgfqpoint{1.644123in}{2.109667in}}%
\pgfpathcurveto{\pgfqpoint{1.652359in}{2.109667in}}{\pgfqpoint{1.660259in}{2.112939in}}{\pgfqpoint{1.666083in}{2.118763in}}%
\pgfpathcurveto{\pgfqpoint{1.671907in}{2.124587in}}{\pgfqpoint{1.675180in}{2.132487in}}{\pgfqpoint{1.675180in}{2.140724in}}%
\pgfpathcurveto{\pgfqpoint{1.675180in}{2.148960in}}{\pgfqpoint{1.671907in}{2.156860in}}{\pgfqpoint{1.666083in}{2.162684in}}%
\pgfpathcurveto{\pgfqpoint{1.660259in}{2.168508in}}{\pgfqpoint{1.652359in}{2.171780in}}{\pgfqpoint{1.644123in}{2.171780in}}%
\pgfpathcurveto{\pgfqpoint{1.635887in}{2.171780in}}{\pgfqpoint{1.627987in}{2.168508in}}{\pgfqpoint{1.622163in}{2.162684in}}%
\pgfpathcurveto{\pgfqpoint{1.616339in}{2.156860in}}{\pgfqpoint{1.613067in}{2.148960in}}{\pgfqpoint{1.613067in}{2.140724in}}%
\pgfpathcurveto{\pgfqpoint{1.613067in}{2.132487in}}{\pgfqpoint{1.616339in}{2.124587in}}{\pgfqpoint{1.622163in}{2.118763in}}%
\pgfpathcurveto{\pgfqpoint{1.627987in}{2.112939in}}{\pgfqpoint{1.635887in}{2.109667in}}{\pgfqpoint{1.644123in}{2.109667in}}%
\pgfpathclose%
\pgfusepath{stroke,fill}%
\end{pgfscope}%
\begin{pgfscope}%
\pgfpathrectangle{\pgfqpoint{0.100000in}{0.212622in}}{\pgfqpoint{3.696000in}{3.696000in}}%
\pgfusepath{clip}%
\pgfsetbuttcap%
\pgfsetroundjoin%
\definecolor{currentfill}{rgb}{0.121569,0.466667,0.705882}%
\pgfsetfillcolor{currentfill}%
\pgfsetfillopacity{0.301449}%
\pgfsetlinewidth{1.003750pt}%
\definecolor{currentstroke}{rgb}{0.121569,0.466667,0.705882}%
\pgfsetstrokecolor{currentstroke}%
\pgfsetstrokeopacity{0.301449}%
\pgfsetdash{}{0pt}%
\pgfpathmoveto{\pgfqpoint{1.644123in}{2.109667in}}%
\pgfpathcurveto{\pgfqpoint{1.652359in}{2.109667in}}{\pgfqpoint{1.660259in}{2.112939in}}{\pgfqpoint{1.666083in}{2.118763in}}%
\pgfpathcurveto{\pgfqpoint{1.671907in}{2.124587in}}{\pgfqpoint{1.675180in}{2.132487in}}{\pgfqpoint{1.675180in}{2.140724in}}%
\pgfpathcurveto{\pgfqpoint{1.675180in}{2.148960in}}{\pgfqpoint{1.671907in}{2.156860in}}{\pgfqpoint{1.666083in}{2.162684in}}%
\pgfpathcurveto{\pgfqpoint{1.660259in}{2.168508in}}{\pgfqpoint{1.652359in}{2.171780in}}{\pgfqpoint{1.644123in}{2.171780in}}%
\pgfpathcurveto{\pgfqpoint{1.635887in}{2.171780in}}{\pgfqpoint{1.627987in}{2.168508in}}{\pgfqpoint{1.622163in}{2.162684in}}%
\pgfpathcurveto{\pgfqpoint{1.616339in}{2.156860in}}{\pgfqpoint{1.613067in}{2.148960in}}{\pgfqpoint{1.613067in}{2.140724in}}%
\pgfpathcurveto{\pgfqpoint{1.613067in}{2.132487in}}{\pgfqpoint{1.616339in}{2.124587in}}{\pgfqpoint{1.622163in}{2.118763in}}%
\pgfpathcurveto{\pgfqpoint{1.627987in}{2.112939in}}{\pgfqpoint{1.635887in}{2.109667in}}{\pgfqpoint{1.644123in}{2.109667in}}%
\pgfpathclose%
\pgfusepath{stroke,fill}%
\end{pgfscope}%
\begin{pgfscope}%
\pgfpathrectangle{\pgfqpoint{0.100000in}{0.212622in}}{\pgfqpoint{3.696000in}{3.696000in}}%
\pgfusepath{clip}%
\pgfsetbuttcap%
\pgfsetroundjoin%
\definecolor{currentfill}{rgb}{0.121569,0.466667,0.705882}%
\pgfsetfillcolor{currentfill}%
\pgfsetfillopacity{0.301449}%
\pgfsetlinewidth{1.003750pt}%
\definecolor{currentstroke}{rgb}{0.121569,0.466667,0.705882}%
\pgfsetstrokecolor{currentstroke}%
\pgfsetstrokeopacity{0.301449}%
\pgfsetdash{}{0pt}%
\pgfpathmoveto{\pgfqpoint{1.644123in}{2.109667in}}%
\pgfpathcurveto{\pgfqpoint{1.652359in}{2.109667in}}{\pgfqpoint{1.660259in}{2.112939in}}{\pgfqpoint{1.666083in}{2.118763in}}%
\pgfpathcurveto{\pgfqpoint{1.671907in}{2.124587in}}{\pgfqpoint{1.675180in}{2.132487in}}{\pgfqpoint{1.675180in}{2.140724in}}%
\pgfpathcurveto{\pgfqpoint{1.675180in}{2.148960in}}{\pgfqpoint{1.671907in}{2.156860in}}{\pgfqpoint{1.666083in}{2.162684in}}%
\pgfpathcurveto{\pgfqpoint{1.660259in}{2.168508in}}{\pgfqpoint{1.652359in}{2.171780in}}{\pgfqpoint{1.644123in}{2.171780in}}%
\pgfpathcurveto{\pgfqpoint{1.635887in}{2.171780in}}{\pgfqpoint{1.627987in}{2.168508in}}{\pgfqpoint{1.622163in}{2.162684in}}%
\pgfpathcurveto{\pgfqpoint{1.616339in}{2.156860in}}{\pgfqpoint{1.613067in}{2.148960in}}{\pgfqpoint{1.613067in}{2.140724in}}%
\pgfpathcurveto{\pgfqpoint{1.613067in}{2.132487in}}{\pgfqpoint{1.616339in}{2.124587in}}{\pgfqpoint{1.622163in}{2.118763in}}%
\pgfpathcurveto{\pgfqpoint{1.627987in}{2.112939in}}{\pgfqpoint{1.635887in}{2.109667in}}{\pgfqpoint{1.644123in}{2.109667in}}%
\pgfpathclose%
\pgfusepath{stroke,fill}%
\end{pgfscope}%
\begin{pgfscope}%
\pgfpathrectangle{\pgfqpoint{0.100000in}{0.212622in}}{\pgfqpoint{3.696000in}{3.696000in}}%
\pgfusepath{clip}%
\pgfsetbuttcap%
\pgfsetroundjoin%
\definecolor{currentfill}{rgb}{0.121569,0.466667,0.705882}%
\pgfsetfillcolor{currentfill}%
\pgfsetfillopacity{0.301449}%
\pgfsetlinewidth{1.003750pt}%
\definecolor{currentstroke}{rgb}{0.121569,0.466667,0.705882}%
\pgfsetstrokecolor{currentstroke}%
\pgfsetstrokeopacity{0.301449}%
\pgfsetdash{}{0pt}%
\pgfpathmoveto{\pgfqpoint{1.644123in}{2.109667in}}%
\pgfpathcurveto{\pgfqpoint{1.652359in}{2.109667in}}{\pgfqpoint{1.660259in}{2.112939in}}{\pgfqpoint{1.666083in}{2.118763in}}%
\pgfpathcurveto{\pgfqpoint{1.671907in}{2.124587in}}{\pgfqpoint{1.675180in}{2.132487in}}{\pgfqpoint{1.675180in}{2.140724in}}%
\pgfpathcurveto{\pgfqpoint{1.675180in}{2.148960in}}{\pgfqpoint{1.671907in}{2.156860in}}{\pgfqpoint{1.666083in}{2.162684in}}%
\pgfpathcurveto{\pgfqpoint{1.660259in}{2.168508in}}{\pgfqpoint{1.652359in}{2.171780in}}{\pgfqpoint{1.644123in}{2.171780in}}%
\pgfpathcurveto{\pgfqpoint{1.635887in}{2.171780in}}{\pgfqpoint{1.627987in}{2.168508in}}{\pgfqpoint{1.622163in}{2.162684in}}%
\pgfpathcurveto{\pgfqpoint{1.616339in}{2.156860in}}{\pgfqpoint{1.613067in}{2.148960in}}{\pgfqpoint{1.613067in}{2.140724in}}%
\pgfpathcurveto{\pgfqpoint{1.613067in}{2.132487in}}{\pgfqpoint{1.616339in}{2.124587in}}{\pgfqpoint{1.622163in}{2.118763in}}%
\pgfpathcurveto{\pgfqpoint{1.627987in}{2.112939in}}{\pgfqpoint{1.635887in}{2.109667in}}{\pgfqpoint{1.644123in}{2.109667in}}%
\pgfpathclose%
\pgfusepath{stroke,fill}%
\end{pgfscope}%
\begin{pgfscope}%
\pgfpathrectangle{\pgfqpoint{0.100000in}{0.212622in}}{\pgfqpoint{3.696000in}{3.696000in}}%
\pgfusepath{clip}%
\pgfsetbuttcap%
\pgfsetroundjoin%
\definecolor{currentfill}{rgb}{0.121569,0.466667,0.705882}%
\pgfsetfillcolor{currentfill}%
\pgfsetfillopacity{0.301449}%
\pgfsetlinewidth{1.003750pt}%
\definecolor{currentstroke}{rgb}{0.121569,0.466667,0.705882}%
\pgfsetstrokecolor{currentstroke}%
\pgfsetstrokeopacity{0.301449}%
\pgfsetdash{}{0pt}%
\pgfpathmoveto{\pgfqpoint{1.644123in}{2.109667in}}%
\pgfpathcurveto{\pgfqpoint{1.652359in}{2.109667in}}{\pgfqpoint{1.660259in}{2.112939in}}{\pgfqpoint{1.666083in}{2.118763in}}%
\pgfpathcurveto{\pgfqpoint{1.671907in}{2.124587in}}{\pgfqpoint{1.675180in}{2.132487in}}{\pgfqpoint{1.675180in}{2.140724in}}%
\pgfpathcurveto{\pgfqpoint{1.675180in}{2.148960in}}{\pgfqpoint{1.671907in}{2.156860in}}{\pgfqpoint{1.666083in}{2.162684in}}%
\pgfpathcurveto{\pgfqpoint{1.660259in}{2.168508in}}{\pgfqpoint{1.652359in}{2.171780in}}{\pgfqpoint{1.644123in}{2.171780in}}%
\pgfpathcurveto{\pgfqpoint{1.635887in}{2.171780in}}{\pgfqpoint{1.627987in}{2.168508in}}{\pgfqpoint{1.622163in}{2.162684in}}%
\pgfpathcurveto{\pgfqpoint{1.616339in}{2.156860in}}{\pgfqpoint{1.613067in}{2.148960in}}{\pgfqpoint{1.613067in}{2.140724in}}%
\pgfpathcurveto{\pgfqpoint{1.613067in}{2.132487in}}{\pgfqpoint{1.616339in}{2.124587in}}{\pgfqpoint{1.622163in}{2.118763in}}%
\pgfpathcurveto{\pgfqpoint{1.627987in}{2.112939in}}{\pgfqpoint{1.635887in}{2.109667in}}{\pgfqpoint{1.644123in}{2.109667in}}%
\pgfpathclose%
\pgfusepath{stroke,fill}%
\end{pgfscope}%
\begin{pgfscope}%
\pgfpathrectangle{\pgfqpoint{0.100000in}{0.212622in}}{\pgfqpoint{3.696000in}{3.696000in}}%
\pgfusepath{clip}%
\pgfsetbuttcap%
\pgfsetroundjoin%
\definecolor{currentfill}{rgb}{0.121569,0.466667,0.705882}%
\pgfsetfillcolor{currentfill}%
\pgfsetfillopacity{0.301449}%
\pgfsetlinewidth{1.003750pt}%
\definecolor{currentstroke}{rgb}{0.121569,0.466667,0.705882}%
\pgfsetstrokecolor{currentstroke}%
\pgfsetstrokeopacity{0.301449}%
\pgfsetdash{}{0pt}%
\pgfpathmoveto{\pgfqpoint{1.644123in}{2.109667in}}%
\pgfpathcurveto{\pgfqpoint{1.652359in}{2.109667in}}{\pgfqpoint{1.660259in}{2.112939in}}{\pgfqpoint{1.666083in}{2.118763in}}%
\pgfpathcurveto{\pgfqpoint{1.671907in}{2.124587in}}{\pgfqpoint{1.675180in}{2.132487in}}{\pgfqpoint{1.675180in}{2.140724in}}%
\pgfpathcurveto{\pgfqpoint{1.675180in}{2.148960in}}{\pgfqpoint{1.671907in}{2.156860in}}{\pgfqpoint{1.666083in}{2.162684in}}%
\pgfpathcurveto{\pgfqpoint{1.660259in}{2.168508in}}{\pgfqpoint{1.652359in}{2.171780in}}{\pgfqpoint{1.644123in}{2.171780in}}%
\pgfpathcurveto{\pgfqpoint{1.635887in}{2.171780in}}{\pgfqpoint{1.627987in}{2.168508in}}{\pgfqpoint{1.622163in}{2.162684in}}%
\pgfpathcurveto{\pgfqpoint{1.616339in}{2.156860in}}{\pgfqpoint{1.613067in}{2.148960in}}{\pgfqpoint{1.613067in}{2.140724in}}%
\pgfpathcurveto{\pgfqpoint{1.613067in}{2.132487in}}{\pgfqpoint{1.616339in}{2.124587in}}{\pgfqpoint{1.622163in}{2.118763in}}%
\pgfpathcurveto{\pgfqpoint{1.627987in}{2.112939in}}{\pgfqpoint{1.635887in}{2.109667in}}{\pgfqpoint{1.644123in}{2.109667in}}%
\pgfpathclose%
\pgfusepath{stroke,fill}%
\end{pgfscope}%
\begin{pgfscope}%
\pgfpathrectangle{\pgfqpoint{0.100000in}{0.212622in}}{\pgfqpoint{3.696000in}{3.696000in}}%
\pgfusepath{clip}%
\pgfsetbuttcap%
\pgfsetroundjoin%
\definecolor{currentfill}{rgb}{0.121569,0.466667,0.705882}%
\pgfsetfillcolor{currentfill}%
\pgfsetfillopacity{0.301449}%
\pgfsetlinewidth{1.003750pt}%
\definecolor{currentstroke}{rgb}{0.121569,0.466667,0.705882}%
\pgfsetstrokecolor{currentstroke}%
\pgfsetstrokeopacity{0.301449}%
\pgfsetdash{}{0pt}%
\pgfpathmoveto{\pgfqpoint{1.644123in}{2.109667in}}%
\pgfpathcurveto{\pgfqpoint{1.652359in}{2.109667in}}{\pgfqpoint{1.660259in}{2.112939in}}{\pgfqpoint{1.666083in}{2.118763in}}%
\pgfpathcurveto{\pgfqpoint{1.671907in}{2.124587in}}{\pgfqpoint{1.675180in}{2.132487in}}{\pgfqpoint{1.675180in}{2.140724in}}%
\pgfpathcurveto{\pgfqpoint{1.675180in}{2.148960in}}{\pgfqpoint{1.671907in}{2.156860in}}{\pgfqpoint{1.666083in}{2.162684in}}%
\pgfpathcurveto{\pgfqpoint{1.660259in}{2.168508in}}{\pgfqpoint{1.652359in}{2.171780in}}{\pgfqpoint{1.644123in}{2.171780in}}%
\pgfpathcurveto{\pgfqpoint{1.635887in}{2.171780in}}{\pgfqpoint{1.627987in}{2.168508in}}{\pgfqpoint{1.622163in}{2.162684in}}%
\pgfpathcurveto{\pgfqpoint{1.616339in}{2.156860in}}{\pgfqpoint{1.613067in}{2.148960in}}{\pgfqpoint{1.613067in}{2.140724in}}%
\pgfpathcurveto{\pgfqpoint{1.613067in}{2.132487in}}{\pgfqpoint{1.616339in}{2.124587in}}{\pgfqpoint{1.622163in}{2.118763in}}%
\pgfpathcurveto{\pgfqpoint{1.627987in}{2.112939in}}{\pgfqpoint{1.635887in}{2.109667in}}{\pgfqpoint{1.644123in}{2.109667in}}%
\pgfpathclose%
\pgfusepath{stroke,fill}%
\end{pgfscope}%
\begin{pgfscope}%
\pgfpathrectangle{\pgfqpoint{0.100000in}{0.212622in}}{\pgfqpoint{3.696000in}{3.696000in}}%
\pgfusepath{clip}%
\pgfsetbuttcap%
\pgfsetroundjoin%
\definecolor{currentfill}{rgb}{0.121569,0.466667,0.705882}%
\pgfsetfillcolor{currentfill}%
\pgfsetfillopacity{0.301449}%
\pgfsetlinewidth{1.003750pt}%
\definecolor{currentstroke}{rgb}{0.121569,0.466667,0.705882}%
\pgfsetstrokecolor{currentstroke}%
\pgfsetstrokeopacity{0.301449}%
\pgfsetdash{}{0pt}%
\pgfpathmoveto{\pgfqpoint{1.644123in}{2.109667in}}%
\pgfpathcurveto{\pgfqpoint{1.652359in}{2.109667in}}{\pgfqpoint{1.660259in}{2.112939in}}{\pgfqpoint{1.666083in}{2.118763in}}%
\pgfpathcurveto{\pgfqpoint{1.671907in}{2.124587in}}{\pgfqpoint{1.675180in}{2.132487in}}{\pgfqpoint{1.675180in}{2.140724in}}%
\pgfpathcurveto{\pgfqpoint{1.675180in}{2.148960in}}{\pgfqpoint{1.671907in}{2.156860in}}{\pgfqpoint{1.666083in}{2.162684in}}%
\pgfpathcurveto{\pgfqpoint{1.660259in}{2.168508in}}{\pgfqpoint{1.652359in}{2.171780in}}{\pgfqpoint{1.644123in}{2.171780in}}%
\pgfpathcurveto{\pgfqpoint{1.635887in}{2.171780in}}{\pgfqpoint{1.627987in}{2.168508in}}{\pgfqpoint{1.622163in}{2.162684in}}%
\pgfpathcurveto{\pgfqpoint{1.616339in}{2.156860in}}{\pgfqpoint{1.613067in}{2.148960in}}{\pgfqpoint{1.613067in}{2.140724in}}%
\pgfpathcurveto{\pgfqpoint{1.613067in}{2.132487in}}{\pgfqpoint{1.616339in}{2.124587in}}{\pgfqpoint{1.622163in}{2.118763in}}%
\pgfpathcurveto{\pgfqpoint{1.627987in}{2.112939in}}{\pgfqpoint{1.635887in}{2.109667in}}{\pgfqpoint{1.644123in}{2.109667in}}%
\pgfpathclose%
\pgfusepath{stroke,fill}%
\end{pgfscope}%
\begin{pgfscope}%
\pgfpathrectangle{\pgfqpoint{0.100000in}{0.212622in}}{\pgfqpoint{3.696000in}{3.696000in}}%
\pgfusepath{clip}%
\pgfsetbuttcap%
\pgfsetroundjoin%
\definecolor{currentfill}{rgb}{0.121569,0.466667,0.705882}%
\pgfsetfillcolor{currentfill}%
\pgfsetfillopacity{0.301449}%
\pgfsetlinewidth{1.003750pt}%
\definecolor{currentstroke}{rgb}{0.121569,0.466667,0.705882}%
\pgfsetstrokecolor{currentstroke}%
\pgfsetstrokeopacity{0.301449}%
\pgfsetdash{}{0pt}%
\pgfpathmoveto{\pgfqpoint{1.644123in}{2.109667in}}%
\pgfpathcurveto{\pgfqpoint{1.652359in}{2.109667in}}{\pgfqpoint{1.660259in}{2.112939in}}{\pgfqpoint{1.666083in}{2.118763in}}%
\pgfpathcurveto{\pgfqpoint{1.671907in}{2.124587in}}{\pgfqpoint{1.675180in}{2.132487in}}{\pgfqpoint{1.675180in}{2.140724in}}%
\pgfpathcurveto{\pgfqpoint{1.675180in}{2.148960in}}{\pgfqpoint{1.671907in}{2.156860in}}{\pgfqpoint{1.666083in}{2.162684in}}%
\pgfpathcurveto{\pgfqpoint{1.660259in}{2.168508in}}{\pgfqpoint{1.652359in}{2.171780in}}{\pgfqpoint{1.644123in}{2.171780in}}%
\pgfpathcurveto{\pgfqpoint{1.635887in}{2.171780in}}{\pgfqpoint{1.627987in}{2.168508in}}{\pgfqpoint{1.622163in}{2.162684in}}%
\pgfpathcurveto{\pgfqpoint{1.616339in}{2.156860in}}{\pgfqpoint{1.613067in}{2.148960in}}{\pgfqpoint{1.613067in}{2.140724in}}%
\pgfpathcurveto{\pgfqpoint{1.613067in}{2.132487in}}{\pgfqpoint{1.616339in}{2.124587in}}{\pgfqpoint{1.622163in}{2.118763in}}%
\pgfpathcurveto{\pgfqpoint{1.627987in}{2.112939in}}{\pgfqpoint{1.635887in}{2.109667in}}{\pgfqpoint{1.644123in}{2.109667in}}%
\pgfpathclose%
\pgfusepath{stroke,fill}%
\end{pgfscope}%
\begin{pgfscope}%
\pgfpathrectangle{\pgfqpoint{0.100000in}{0.212622in}}{\pgfqpoint{3.696000in}{3.696000in}}%
\pgfusepath{clip}%
\pgfsetbuttcap%
\pgfsetroundjoin%
\definecolor{currentfill}{rgb}{0.121569,0.466667,0.705882}%
\pgfsetfillcolor{currentfill}%
\pgfsetfillopacity{0.301449}%
\pgfsetlinewidth{1.003750pt}%
\definecolor{currentstroke}{rgb}{0.121569,0.466667,0.705882}%
\pgfsetstrokecolor{currentstroke}%
\pgfsetstrokeopacity{0.301449}%
\pgfsetdash{}{0pt}%
\pgfpathmoveto{\pgfqpoint{1.644123in}{2.109667in}}%
\pgfpathcurveto{\pgfqpoint{1.652359in}{2.109667in}}{\pgfqpoint{1.660259in}{2.112939in}}{\pgfqpoint{1.666083in}{2.118763in}}%
\pgfpathcurveto{\pgfqpoint{1.671907in}{2.124587in}}{\pgfqpoint{1.675180in}{2.132487in}}{\pgfqpoint{1.675180in}{2.140724in}}%
\pgfpathcurveto{\pgfqpoint{1.675180in}{2.148960in}}{\pgfqpoint{1.671907in}{2.156860in}}{\pgfqpoint{1.666083in}{2.162684in}}%
\pgfpathcurveto{\pgfqpoint{1.660259in}{2.168508in}}{\pgfqpoint{1.652359in}{2.171780in}}{\pgfqpoint{1.644123in}{2.171780in}}%
\pgfpathcurveto{\pgfqpoint{1.635887in}{2.171780in}}{\pgfqpoint{1.627987in}{2.168508in}}{\pgfqpoint{1.622163in}{2.162684in}}%
\pgfpathcurveto{\pgfqpoint{1.616339in}{2.156860in}}{\pgfqpoint{1.613067in}{2.148960in}}{\pgfqpoint{1.613067in}{2.140724in}}%
\pgfpathcurveto{\pgfqpoint{1.613067in}{2.132487in}}{\pgfqpoint{1.616339in}{2.124587in}}{\pgfqpoint{1.622163in}{2.118763in}}%
\pgfpathcurveto{\pgfqpoint{1.627987in}{2.112939in}}{\pgfqpoint{1.635887in}{2.109667in}}{\pgfqpoint{1.644123in}{2.109667in}}%
\pgfpathclose%
\pgfusepath{stroke,fill}%
\end{pgfscope}%
\begin{pgfscope}%
\pgfpathrectangle{\pgfqpoint{0.100000in}{0.212622in}}{\pgfqpoint{3.696000in}{3.696000in}}%
\pgfusepath{clip}%
\pgfsetbuttcap%
\pgfsetroundjoin%
\definecolor{currentfill}{rgb}{0.121569,0.466667,0.705882}%
\pgfsetfillcolor{currentfill}%
\pgfsetfillopacity{0.301449}%
\pgfsetlinewidth{1.003750pt}%
\definecolor{currentstroke}{rgb}{0.121569,0.466667,0.705882}%
\pgfsetstrokecolor{currentstroke}%
\pgfsetstrokeopacity{0.301449}%
\pgfsetdash{}{0pt}%
\pgfpathmoveto{\pgfqpoint{1.644123in}{2.109667in}}%
\pgfpathcurveto{\pgfqpoint{1.652359in}{2.109667in}}{\pgfqpoint{1.660259in}{2.112939in}}{\pgfqpoint{1.666083in}{2.118763in}}%
\pgfpathcurveto{\pgfqpoint{1.671907in}{2.124587in}}{\pgfqpoint{1.675180in}{2.132487in}}{\pgfqpoint{1.675180in}{2.140724in}}%
\pgfpathcurveto{\pgfqpoint{1.675180in}{2.148960in}}{\pgfqpoint{1.671907in}{2.156860in}}{\pgfqpoint{1.666083in}{2.162684in}}%
\pgfpathcurveto{\pgfqpoint{1.660259in}{2.168508in}}{\pgfqpoint{1.652359in}{2.171780in}}{\pgfqpoint{1.644123in}{2.171780in}}%
\pgfpathcurveto{\pgfqpoint{1.635887in}{2.171780in}}{\pgfqpoint{1.627987in}{2.168508in}}{\pgfqpoint{1.622163in}{2.162684in}}%
\pgfpathcurveto{\pgfqpoint{1.616339in}{2.156860in}}{\pgfqpoint{1.613067in}{2.148960in}}{\pgfqpoint{1.613067in}{2.140724in}}%
\pgfpathcurveto{\pgfqpoint{1.613067in}{2.132487in}}{\pgfqpoint{1.616339in}{2.124587in}}{\pgfqpoint{1.622163in}{2.118763in}}%
\pgfpathcurveto{\pgfqpoint{1.627987in}{2.112939in}}{\pgfqpoint{1.635887in}{2.109667in}}{\pgfqpoint{1.644123in}{2.109667in}}%
\pgfpathclose%
\pgfusepath{stroke,fill}%
\end{pgfscope}%
\begin{pgfscope}%
\pgfpathrectangle{\pgfqpoint{0.100000in}{0.212622in}}{\pgfqpoint{3.696000in}{3.696000in}}%
\pgfusepath{clip}%
\pgfsetbuttcap%
\pgfsetroundjoin%
\definecolor{currentfill}{rgb}{0.121569,0.466667,0.705882}%
\pgfsetfillcolor{currentfill}%
\pgfsetfillopacity{0.301449}%
\pgfsetlinewidth{1.003750pt}%
\definecolor{currentstroke}{rgb}{0.121569,0.466667,0.705882}%
\pgfsetstrokecolor{currentstroke}%
\pgfsetstrokeopacity{0.301449}%
\pgfsetdash{}{0pt}%
\pgfpathmoveto{\pgfqpoint{1.644123in}{2.109667in}}%
\pgfpathcurveto{\pgfqpoint{1.652359in}{2.109667in}}{\pgfqpoint{1.660259in}{2.112939in}}{\pgfqpoint{1.666083in}{2.118763in}}%
\pgfpathcurveto{\pgfqpoint{1.671907in}{2.124587in}}{\pgfqpoint{1.675180in}{2.132487in}}{\pgfqpoint{1.675180in}{2.140724in}}%
\pgfpathcurveto{\pgfqpoint{1.675180in}{2.148960in}}{\pgfqpoint{1.671907in}{2.156860in}}{\pgfqpoint{1.666083in}{2.162684in}}%
\pgfpathcurveto{\pgfqpoint{1.660259in}{2.168508in}}{\pgfqpoint{1.652359in}{2.171780in}}{\pgfqpoint{1.644123in}{2.171780in}}%
\pgfpathcurveto{\pgfqpoint{1.635887in}{2.171780in}}{\pgfqpoint{1.627987in}{2.168508in}}{\pgfqpoint{1.622163in}{2.162684in}}%
\pgfpathcurveto{\pgfqpoint{1.616339in}{2.156860in}}{\pgfqpoint{1.613067in}{2.148960in}}{\pgfqpoint{1.613067in}{2.140724in}}%
\pgfpathcurveto{\pgfqpoint{1.613067in}{2.132487in}}{\pgfqpoint{1.616339in}{2.124587in}}{\pgfqpoint{1.622163in}{2.118763in}}%
\pgfpathcurveto{\pgfqpoint{1.627987in}{2.112939in}}{\pgfqpoint{1.635887in}{2.109667in}}{\pgfqpoint{1.644123in}{2.109667in}}%
\pgfpathclose%
\pgfusepath{stroke,fill}%
\end{pgfscope}%
\begin{pgfscope}%
\pgfpathrectangle{\pgfqpoint{0.100000in}{0.212622in}}{\pgfqpoint{3.696000in}{3.696000in}}%
\pgfusepath{clip}%
\pgfsetbuttcap%
\pgfsetroundjoin%
\definecolor{currentfill}{rgb}{0.121569,0.466667,0.705882}%
\pgfsetfillcolor{currentfill}%
\pgfsetfillopacity{0.301449}%
\pgfsetlinewidth{1.003750pt}%
\definecolor{currentstroke}{rgb}{0.121569,0.466667,0.705882}%
\pgfsetstrokecolor{currentstroke}%
\pgfsetstrokeopacity{0.301449}%
\pgfsetdash{}{0pt}%
\pgfpathmoveto{\pgfqpoint{1.644123in}{2.109667in}}%
\pgfpathcurveto{\pgfqpoint{1.652359in}{2.109667in}}{\pgfqpoint{1.660259in}{2.112939in}}{\pgfqpoint{1.666083in}{2.118763in}}%
\pgfpathcurveto{\pgfqpoint{1.671907in}{2.124587in}}{\pgfqpoint{1.675180in}{2.132487in}}{\pgfqpoint{1.675180in}{2.140724in}}%
\pgfpathcurveto{\pgfqpoint{1.675180in}{2.148960in}}{\pgfqpoint{1.671907in}{2.156860in}}{\pgfqpoint{1.666083in}{2.162684in}}%
\pgfpathcurveto{\pgfqpoint{1.660259in}{2.168508in}}{\pgfqpoint{1.652359in}{2.171780in}}{\pgfqpoint{1.644123in}{2.171780in}}%
\pgfpathcurveto{\pgfqpoint{1.635887in}{2.171780in}}{\pgfqpoint{1.627987in}{2.168508in}}{\pgfqpoint{1.622163in}{2.162684in}}%
\pgfpathcurveto{\pgfqpoint{1.616339in}{2.156860in}}{\pgfqpoint{1.613067in}{2.148960in}}{\pgfqpoint{1.613067in}{2.140724in}}%
\pgfpathcurveto{\pgfqpoint{1.613067in}{2.132487in}}{\pgfqpoint{1.616339in}{2.124587in}}{\pgfqpoint{1.622163in}{2.118763in}}%
\pgfpathcurveto{\pgfqpoint{1.627987in}{2.112939in}}{\pgfqpoint{1.635887in}{2.109667in}}{\pgfqpoint{1.644123in}{2.109667in}}%
\pgfpathclose%
\pgfusepath{stroke,fill}%
\end{pgfscope}%
\begin{pgfscope}%
\pgfpathrectangle{\pgfqpoint{0.100000in}{0.212622in}}{\pgfqpoint{3.696000in}{3.696000in}}%
\pgfusepath{clip}%
\pgfsetbuttcap%
\pgfsetroundjoin%
\definecolor{currentfill}{rgb}{0.121569,0.466667,0.705882}%
\pgfsetfillcolor{currentfill}%
\pgfsetfillopacity{0.301449}%
\pgfsetlinewidth{1.003750pt}%
\definecolor{currentstroke}{rgb}{0.121569,0.466667,0.705882}%
\pgfsetstrokecolor{currentstroke}%
\pgfsetstrokeopacity{0.301449}%
\pgfsetdash{}{0pt}%
\pgfpathmoveto{\pgfqpoint{1.644123in}{2.109667in}}%
\pgfpathcurveto{\pgfqpoint{1.652359in}{2.109667in}}{\pgfqpoint{1.660259in}{2.112939in}}{\pgfqpoint{1.666083in}{2.118763in}}%
\pgfpathcurveto{\pgfqpoint{1.671907in}{2.124587in}}{\pgfqpoint{1.675180in}{2.132487in}}{\pgfqpoint{1.675180in}{2.140724in}}%
\pgfpathcurveto{\pgfqpoint{1.675180in}{2.148960in}}{\pgfqpoint{1.671907in}{2.156860in}}{\pgfqpoint{1.666083in}{2.162684in}}%
\pgfpathcurveto{\pgfqpoint{1.660259in}{2.168508in}}{\pgfqpoint{1.652359in}{2.171780in}}{\pgfqpoint{1.644123in}{2.171780in}}%
\pgfpathcurveto{\pgfqpoint{1.635887in}{2.171780in}}{\pgfqpoint{1.627987in}{2.168508in}}{\pgfqpoint{1.622163in}{2.162684in}}%
\pgfpathcurveto{\pgfqpoint{1.616339in}{2.156860in}}{\pgfqpoint{1.613067in}{2.148960in}}{\pgfqpoint{1.613067in}{2.140724in}}%
\pgfpathcurveto{\pgfqpoint{1.613067in}{2.132487in}}{\pgfqpoint{1.616339in}{2.124587in}}{\pgfqpoint{1.622163in}{2.118763in}}%
\pgfpathcurveto{\pgfqpoint{1.627987in}{2.112939in}}{\pgfqpoint{1.635887in}{2.109667in}}{\pgfqpoint{1.644123in}{2.109667in}}%
\pgfpathclose%
\pgfusepath{stroke,fill}%
\end{pgfscope}%
\begin{pgfscope}%
\pgfpathrectangle{\pgfqpoint{0.100000in}{0.212622in}}{\pgfqpoint{3.696000in}{3.696000in}}%
\pgfusepath{clip}%
\pgfsetbuttcap%
\pgfsetroundjoin%
\definecolor{currentfill}{rgb}{0.121569,0.466667,0.705882}%
\pgfsetfillcolor{currentfill}%
\pgfsetfillopacity{0.301449}%
\pgfsetlinewidth{1.003750pt}%
\definecolor{currentstroke}{rgb}{0.121569,0.466667,0.705882}%
\pgfsetstrokecolor{currentstroke}%
\pgfsetstrokeopacity{0.301449}%
\pgfsetdash{}{0pt}%
\pgfpathmoveto{\pgfqpoint{1.644123in}{2.109667in}}%
\pgfpathcurveto{\pgfqpoint{1.652359in}{2.109667in}}{\pgfqpoint{1.660259in}{2.112939in}}{\pgfqpoint{1.666083in}{2.118763in}}%
\pgfpathcurveto{\pgfqpoint{1.671907in}{2.124587in}}{\pgfqpoint{1.675180in}{2.132487in}}{\pgfqpoint{1.675180in}{2.140724in}}%
\pgfpathcurveto{\pgfqpoint{1.675180in}{2.148960in}}{\pgfqpoint{1.671907in}{2.156860in}}{\pgfqpoint{1.666083in}{2.162684in}}%
\pgfpathcurveto{\pgfqpoint{1.660259in}{2.168508in}}{\pgfqpoint{1.652359in}{2.171780in}}{\pgfqpoint{1.644123in}{2.171780in}}%
\pgfpathcurveto{\pgfqpoint{1.635887in}{2.171780in}}{\pgfqpoint{1.627987in}{2.168508in}}{\pgfqpoint{1.622163in}{2.162684in}}%
\pgfpathcurveto{\pgfqpoint{1.616339in}{2.156860in}}{\pgfqpoint{1.613067in}{2.148960in}}{\pgfqpoint{1.613067in}{2.140724in}}%
\pgfpathcurveto{\pgfqpoint{1.613067in}{2.132487in}}{\pgfqpoint{1.616339in}{2.124587in}}{\pgfqpoint{1.622163in}{2.118763in}}%
\pgfpathcurveto{\pgfqpoint{1.627987in}{2.112939in}}{\pgfqpoint{1.635887in}{2.109667in}}{\pgfqpoint{1.644123in}{2.109667in}}%
\pgfpathclose%
\pgfusepath{stroke,fill}%
\end{pgfscope}%
\begin{pgfscope}%
\pgfpathrectangle{\pgfqpoint{0.100000in}{0.212622in}}{\pgfqpoint{3.696000in}{3.696000in}}%
\pgfusepath{clip}%
\pgfsetbuttcap%
\pgfsetroundjoin%
\definecolor{currentfill}{rgb}{0.121569,0.466667,0.705882}%
\pgfsetfillcolor{currentfill}%
\pgfsetfillopacity{0.301449}%
\pgfsetlinewidth{1.003750pt}%
\definecolor{currentstroke}{rgb}{0.121569,0.466667,0.705882}%
\pgfsetstrokecolor{currentstroke}%
\pgfsetstrokeopacity{0.301449}%
\pgfsetdash{}{0pt}%
\pgfpathmoveto{\pgfqpoint{1.644123in}{2.109667in}}%
\pgfpathcurveto{\pgfqpoint{1.652359in}{2.109667in}}{\pgfqpoint{1.660259in}{2.112939in}}{\pgfqpoint{1.666083in}{2.118763in}}%
\pgfpathcurveto{\pgfqpoint{1.671907in}{2.124587in}}{\pgfqpoint{1.675180in}{2.132487in}}{\pgfqpoint{1.675180in}{2.140724in}}%
\pgfpathcurveto{\pgfqpoint{1.675180in}{2.148960in}}{\pgfqpoint{1.671907in}{2.156860in}}{\pgfqpoint{1.666083in}{2.162684in}}%
\pgfpathcurveto{\pgfqpoint{1.660259in}{2.168508in}}{\pgfqpoint{1.652359in}{2.171780in}}{\pgfqpoint{1.644123in}{2.171780in}}%
\pgfpathcurveto{\pgfqpoint{1.635887in}{2.171780in}}{\pgfqpoint{1.627987in}{2.168508in}}{\pgfqpoint{1.622163in}{2.162684in}}%
\pgfpathcurveto{\pgfqpoint{1.616339in}{2.156860in}}{\pgfqpoint{1.613067in}{2.148960in}}{\pgfqpoint{1.613067in}{2.140724in}}%
\pgfpathcurveto{\pgfqpoint{1.613067in}{2.132487in}}{\pgfqpoint{1.616339in}{2.124587in}}{\pgfqpoint{1.622163in}{2.118763in}}%
\pgfpathcurveto{\pgfqpoint{1.627987in}{2.112939in}}{\pgfqpoint{1.635887in}{2.109667in}}{\pgfqpoint{1.644123in}{2.109667in}}%
\pgfpathclose%
\pgfusepath{stroke,fill}%
\end{pgfscope}%
\begin{pgfscope}%
\pgfpathrectangle{\pgfqpoint{0.100000in}{0.212622in}}{\pgfqpoint{3.696000in}{3.696000in}}%
\pgfusepath{clip}%
\pgfsetbuttcap%
\pgfsetroundjoin%
\definecolor{currentfill}{rgb}{0.121569,0.466667,0.705882}%
\pgfsetfillcolor{currentfill}%
\pgfsetfillopacity{0.301449}%
\pgfsetlinewidth{1.003750pt}%
\definecolor{currentstroke}{rgb}{0.121569,0.466667,0.705882}%
\pgfsetstrokecolor{currentstroke}%
\pgfsetstrokeopacity{0.301449}%
\pgfsetdash{}{0pt}%
\pgfpathmoveto{\pgfqpoint{1.644123in}{2.109667in}}%
\pgfpathcurveto{\pgfqpoint{1.652359in}{2.109667in}}{\pgfqpoint{1.660259in}{2.112939in}}{\pgfqpoint{1.666083in}{2.118763in}}%
\pgfpathcurveto{\pgfqpoint{1.671907in}{2.124587in}}{\pgfqpoint{1.675180in}{2.132487in}}{\pgfqpoint{1.675180in}{2.140724in}}%
\pgfpathcurveto{\pgfqpoint{1.675180in}{2.148960in}}{\pgfqpoint{1.671907in}{2.156860in}}{\pgfqpoint{1.666083in}{2.162684in}}%
\pgfpathcurveto{\pgfqpoint{1.660259in}{2.168508in}}{\pgfqpoint{1.652359in}{2.171780in}}{\pgfqpoint{1.644123in}{2.171780in}}%
\pgfpathcurveto{\pgfqpoint{1.635887in}{2.171780in}}{\pgfqpoint{1.627987in}{2.168508in}}{\pgfqpoint{1.622163in}{2.162684in}}%
\pgfpathcurveto{\pgfqpoint{1.616339in}{2.156860in}}{\pgfqpoint{1.613067in}{2.148960in}}{\pgfqpoint{1.613067in}{2.140724in}}%
\pgfpathcurveto{\pgfqpoint{1.613067in}{2.132487in}}{\pgfqpoint{1.616339in}{2.124587in}}{\pgfqpoint{1.622163in}{2.118763in}}%
\pgfpathcurveto{\pgfqpoint{1.627987in}{2.112939in}}{\pgfqpoint{1.635887in}{2.109667in}}{\pgfqpoint{1.644123in}{2.109667in}}%
\pgfpathclose%
\pgfusepath{stroke,fill}%
\end{pgfscope}%
\begin{pgfscope}%
\pgfpathrectangle{\pgfqpoint{0.100000in}{0.212622in}}{\pgfqpoint{3.696000in}{3.696000in}}%
\pgfusepath{clip}%
\pgfsetbuttcap%
\pgfsetroundjoin%
\definecolor{currentfill}{rgb}{0.121569,0.466667,0.705882}%
\pgfsetfillcolor{currentfill}%
\pgfsetfillopacity{0.301449}%
\pgfsetlinewidth{1.003750pt}%
\definecolor{currentstroke}{rgb}{0.121569,0.466667,0.705882}%
\pgfsetstrokecolor{currentstroke}%
\pgfsetstrokeopacity{0.301449}%
\pgfsetdash{}{0pt}%
\pgfpathmoveto{\pgfqpoint{1.644123in}{2.109667in}}%
\pgfpathcurveto{\pgfqpoint{1.652359in}{2.109667in}}{\pgfqpoint{1.660259in}{2.112939in}}{\pgfqpoint{1.666083in}{2.118763in}}%
\pgfpathcurveto{\pgfqpoint{1.671907in}{2.124587in}}{\pgfqpoint{1.675180in}{2.132487in}}{\pgfqpoint{1.675180in}{2.140724in}}%
\pgfpathcurveto{\pgfqpoint{1.675180in}{2.148960in}}{\pgfqpoint{1.671907in}{2.156860in}}{\pgfqpoint{1.666083in}{2.162684in}}%
\pgfpathcurveto{\pgfqpoint{1.660259in}{2.168508in}}{\pgfqpoint{1.652359in}{2.171780in}}{\pgfqpoint{1.644123in}{2.171780in}}%
\pgfpathcurveto{\pgfqpoint{1.635887in}{2.171780in}}{\pgfqpoint{1.627987in}{2.168508in}}{\pgfqpoint{1.622163in}{2.162684in}}%
\pgfpathcurveto{\pgfqpoint{1.616339in}{2.156860in}}{\pgfqpoint{1.613067in}{2.148960in}}{\pgfqpoint{1.613067in}{2.140724in}}%
\pgfpathcurveto{\pgfqpoint{1.613067in}{2.132487in}}{\pgfqpoint{1.616339in}{2.124587in}}{\pgfqpoint{1.622163in}{2.118763in}}%
\pgfpathcurveto{\pgfqpoint{1.627987in}{2.112939in}}{\pgfqpoint{1.635887in}{2.109667in}}{\pgfqpoint{1.644123in}{2.109667in}}%
\pgfpathclose%
\pgfusepath{stroke,fill}%
\end{pgfscope}%
\begin{pgfscope}%
\pgfpathrectangle{\pgfqpoint{0.100000in}{0.212622in}}{\pgfqpoint{3.696000in}{3.696000in}}%
\pgfusepath{clip}%
\pgfsetbuttcap%
\pgfsetroundjoin%
\definecolor{currentfill}{rgb}{0.121569,0.466667,0.705882}%
\pgfsetfillcolor{currentfill}%
\pgfsetfillopacity{0.301449}%
\pgfsetlinewidth{1.003750pt}%
\definecolor{currentstroke}{rgb}{0.121569,0.466667,0.705882}%
\pgfsetstrokecolor{currentstroke}%
\pgfsetstrokeopacity{0.301449}%
\pgfsetdash{}{0pt}%
\pgfpathmoveto{\pgfqpoint{1.644123in}{2.109667in}}%
\pgfpathcurveto{\pgfqpoint{1.652359in}{2.109667in}}{\pgfqpoint{1.660259in}{2.112939in}}{\pgfqpoint{1.666083in}{2.118763in}}%
\pgfpathcurveto{\pgfqpoint{1.671907in}{2.124587in}}{\pgfqpoint{1.675180in}{2.132487in}}{\pgfqpoint{1.675180in}{2.140724in}}%
\pgfpathcurveto{\pgfqpoint{1.675180in}{2.148960in}}{\pgfqpoint{1.671907in}{2.156860in}}{\pgfqpoint{1.666083in}{2.162684in}}%
\pgfpathcurveto{\pgfqpoint{1.660259in}{2.168508in}}{\pgfqpoint{1.652359in}{2.171780in}}{\pgfqpoint{1.644123in}{2.171780in}}%
\pgfpathcurveto{\pgfqpoint{1.635887in}{2.171780in}}{\pgfqpoint{1.627987in}{2.168508in}}{\pgfqpoint{1.622163in}{2.162684in}}%
\pgfpathcurveto{\pgfqpoint{1.616339in}{2.156860in}}{\pgfqpoint{1.613067in}{2.148960in}}{\pgfqpoint{1.613067in}{2.140724in}}%
\pgfpathcurveto{\pgfqpoint{1.613067in}{2.132487in}}{\pgfqpoint{1.616339in}{2.124587in}}{\pgfqpoint{1.622163in}{2.118763in}}%
\pgfpathcurveto{\pgfqpoint{1.627987in}{2.112939in}}{\pgfqpoint{1.635887in}{2.109667in}}{\pgfqpoint{1.644123in}{2.109667in}}%
\pgfpathclose%
\pgfusepath{stroke,fill}%
\end{pgfscope}%
\begin{pgfscope}%
\pgfpathrectangle{\pgfqpoint{0.100000in}{0.212622in}}{\pgfqpoint{3.696000in}{3.696000in}}%
\pgfusepath{clip}%
\pgfsetbuttcap%
\pgfsetroundjoin%
\definecolor{currentfill}{rgb}{0.121569,0.466667,0.705882}%
\pgfsetfillcolor{currentfill}%
\pgfsetfillopacity{0.301449}%
\pgfsetlinewidth{1.003750pt}%
\definecolor{currentstroke}{rgb}{0.121569,0.466667,0.705882}%
\pgfsetstrokecolor{currentstroke}%
\pgfsetstrokeopacity{0.301449}%
\pgfsetdash{}{0pt}%
\pgfpathmoveto{\pgfqpoint{1.644123in}{2.109667in}}%
\pgfpathcurveto{\pgfqpoint{1.652359in}{2.109667in}}{\pgfqpoint{1.660259in}{2.112939in}}{\pgfqpoint{1.666083in}{2.118763in}}%
\pgfpathcurveto{\pgfqpoint{1.671907in}{2.124587in}}{\pgfqpoint{1.675180in}{2.132487in}}{\pgfqpoint{1.675180in}{2.140724in}}%
\pgfpathcurveto{\pgfqpoint{1.675180in}{2.148960in}}{\pgfqpoint{1.671907in}{2.156860in}}{\pgfqpoint{1.666083in}{2.162684in}}%
\pgfpathcurveto{\pgfqpoint{1.660259in}{2.168508in}}{\pgfqpoint{1.652359in}{2.171780in}}{\pgfqpoint{1.644123in}{2.171780in}}%
\pgfpathcurveto{\pgfqpoint{1.635887in}{2.171780in}}{\pgfqpoint{1.627987in}{2.168508in}}{\pgfqpoint{1.622163in}{2.162684in}}%
\pgfpathcurveto{\pgfqpoint{1.616339in}{2.156860in}}{\pgfqpoint{1.613067in}{2.148960in}}{\pgfqpoint{1.613067in}{2.140724in}}%
\pgfpathcurveto{\pgfqpoint{1.613067in}{2.132487in}}{\pgfqpoint{1.616339in}{2.124587in}}{\pgfqpoint{1.622163in}{2.118763in}}%
\pgfpathcurveto{\pgfqpoint{1.627987in}{2.112939in}}{\pgfqpoint{1.635887in}{2.109667in}}{\pgfqpoint{1.644123in}{2.109667in}}%
\pgfpathclose%
\pgfusepath{stroke,fill}%
\end{pgfscope}%
\begin{pgfscope}%
\pgfpathrectangle{\pgfqpoint{0.100000in}{0.212622in}}{\pgfqpoint{3.696000in}{3.696000in}}%
\pgfusepath{clip}%
\pgfsetbuttcap%
\pgfsetroundjoin%
\definecolor{currentfill}{rgb}{0.121569,0.466667,0.705882}%
\pgfsetfillcolor{currentfill}%
\pgfsetfillopacity{0.301449}%
\pgfsetlinewidth{1.003750pt}%
\definecolor{currentstroke}{rgb}{0.121569,0.466667,0.705882}%
\pgfsetstrokecolor{currentstroke}%
\pgfsetstrokeopacity{0.301449}%
\pgfsetdash{}{0pt}%
\pgfpathmoveto{\pgfqpoint{1.644123in}{2.109667in}}%
\pgfpathcurveto{\pgfqpoint{1.652359in}{2.109667in}}{\pgfqpoint{1.660259in}{2.112939in}}{\pgfqpoint{1.666083in}{2.118763in}}%
\pgfpathcurveto{\pgfqpoint{1.671907in}{2.124587in}}{\pgfqpoint{1.675180in}{2.132487in}}{\pgfqpoint{1.675180in}{2.140724in}}%
\pgfpathcurveto{\pgfqpoint{1.675180in}{2.148960in}}{\pgfqpoint{1.671907in}{2.156860in}}{\pgfqpoint{1.666083in}{2.162684in}}%
\pgfpathcurveto{\pgfqpoint{1.660259in}{2.168508in}}{\pgfqpoint{1.652359in}{2.171780in}}{\pgfqpoint{1.644123in}{2.171780in}}%
\pgfpathcurveto{\pgfqpoint{1.635887in}{2.171780in}}{\pgfqpoint{1.627987in}{2.168508in}}{\pgfqpoint{1.622163in}{2.162684in}}%
\pgfpathcurveto{\pgfqpoint{1.616339in}{2.156860in}}{\pgfqpoint{1.613067in}{2.148960in}}{\pgfqpoint{1.613067in}{2.140724in}}%
\pgfpathcurveto{\pgfqpoint{1.613067in}{2.132487in}}{\pgfqpoint{1.616339in}{2.124587in}}{\pgfqpoint{1.622163in}{2.118763in}}%
\pgfpathcurveto{\pgfqpoint{1.627987in}{2.112939in}}{\pgfqpoint{1.635887in}{2.109667in}}{\pgfqpoint{1.644123in}{2.109667in}}%
\pgfpathclose%
\pgfusepath{stroke,fill}%
\end{pgfscope}%
\begin{pgfscope}%
\pgfpathrectangle{\pgfqpoint{0.100000in}{0.212622in}}{\pgfqpoint{3.696000in}{3.696000in}}%
\pgfusepath{clip}%
\pgfsetbuttcap%
\pgfsetroundjoin%
\definecolor{currentfill}{rgb}{0.121569,0.466667,0.705882}%
\pgfsetfillcolor{currentfill}%
\pgfsetfillopacity{0.301449}%
\pgfsetlinewidth{1.003750pt}%
\definecolor{currentstroke}{rgb}{0.121569,0.466667,0.705882}%
\pgfsetstrokecolor{currentstroke}%
\pgfsetstrokeopacity{0.301449}%
\pgfsetdash{}{0pt}%
\pgfpathmoveto{\pgfqpoint{1.644123in}{2.109667in}}%
\pgfpathcurveto{\pgfqpoint{1.652359in}{2.109667in}}{\pgfqpoint{1.660259in}{2.112939in}}{\pgfqpoint{1.666083in}{2.118763in}}%
\pgfpathcurveto{\pgfqpoint{1.671907in}{2.124587in}}{\pgfqpoint{1.675180in}{2.132487in}}{\pgfqpoint{1.675180in}{2.140724in}}%
\pgfpathcurveto{\pgfqpoint{1.675180in}{2.148960in}}{\pgfqpoint{1.671907in}{2.156860in}}{\pgfqpoint{1.666083in}{2.162684in}}%
\pgfpathcurveto{\pgfqpoint{1.660259in}{2.168508in}}{\pgfqpoint{1.652359in}{2.171780in}}{\pgfqpoint{1.644123in}{2.171780in}}%
\pgfpathcurveto{\pgfqpoint{1.635887in}{2.171780in}}{\pgfqpoint{1.627987in}{2.168508in}}{\pgfqpoint{1.622163in}{2.162684in}}%
\pgfpathcurveto{\pgfqpoint{1.616339in}{2.156860in}}{\pgfqpoint{1.613067in}{2.148960in}}{\pgfqpoint{1.613067in}{2.140724in}}%
\pgfpathcurveto{\pgfqpoint{1.613067in}{2.132487in}}{\pgfqpoint{1.616339in}{2.124587in}}{\pgfqpoint{1.622163in}{2.118763in}}%
\pgfpathcurveto{\pgfqpoint{1.627987in}{2.112939in}}{\pgfqpoint{1.635887in}{2.109667in}}{\pgfqpoint{1.644123in}{2.109667in}}%
\pgfpathclose%
\pgfusepath{stroke,fill}%
\end{pgfscope}%
\begin{pgfscope}%
\pgfpathrectangle{\pgfqpoint{0.100000in}{0.212622in}}{\pgfqpoint{3.696000in}{3.696000in}}%
\pgfusepath{clip}%
\pgfsetbuttcap%
\pgfsetroundjoin%
\definecolor{currentfill}{rgb}{0.121569,0.466667,0.705882}%
\pgfsetfillcolor{currentfill}%
\pgfsetfillopacity{0.301449}%
\pgfsetlinewidth{1.003750pt}%
\definecolor{currentstroke}{rgb}{0.121569,0.466667,0.705882}%
\pgfsetstrokecolor{currentstroke}%
\pgfsetstrokeopacity{0.301449}%
\pgfsetdash{}{0pt}%
\pgfpathmoveto{\pgfqpoint{1.644123in}{2.109667in}}%
\pgfpathcurveto{\pgfqpoint{1.652359in}{2.109667in}}{\pgfqpoint{1.660259in}{2.112939in}}{\pgfqpoint{1.666083in}{2.118763in}}%
\pgfpathcurveto{\pgfqpoint{1.671907in}{2.124587in}}{\pgfqpoint{1.675180in}{2.132487in}}{\pgfqpoint{1.675180in}{2.140724in}}%
\pgfpathcurveto{\pgfqpoint{1.675180in}{2.148960in}}{\pgfqpoint{1.671907in}{2.156860in}}{\pgfqpoint{1.666083in}{2.162684in}}%
\pgfpathcurveto{\pgfqpoint{1.660259in}{2.168508in}}{\pgfqpoint{1.652359in}{2.171780in}}{\pgfqpoint{1.644123in}{2.171780in}}%
\pgfpathcurveto{\pgfqpoint{1.635887in}{2.171780in}}{\pgfqpoint{1.627987in}{2.168508in}}{\pgfqpoint{1.622163in}{2.162684in}}%
\pgfpathcurveto{\pgfqpoint{1.616339in}{2.156860in}}{\pgfqpoint{1.613067in}{2.148960in}}{\pgfqpoint{1.613067in}{2.140724in}}%
\pgfpathcurveto{\pgfqpoint{1.613067in}{2.132487in}}{\pgfqpoint{1.616339in}{2.124587in}}{\pgfqpoint{1.622163in}{2.118763in}}%
\pgfpathcurveto{\pgfqpoint{1.627987in}{2.112939in}}{\pgfqpoint{1.635887in}{2.109667in}}{\pgfqpoint{1.644123in}{2.109667in}}%
\pgfpathclose%
\pgfusepath{stroke,fill}%
\end{pgfscope}%
\begin{pgfscope}%
\pgfpathrectangle{\pgfqpoint{0.100000in}{0.212622in}}{\pgfqpoint{3.696000in}{3.696000in}}%
\pgfusepath{clip}%
\pgfsetbuttcap%
\pgfsetroundjoin%
\definecolor{currentfill}{rgb}{0.121569,0.466667,0.705882}%
\pgfsetfillcolor{currentfill}%
\pgfsetfillopacity{0.301449}%
\pgfsetlinewidth{1.003750pt}%
\definecolor{currentstroke}{rgb}{0.121569,0.466667,0.705882}%
\pgfsetstrokecolor{currentstroke}%
\pgfsetstrokeopacity{0.301449}%
\pgfsetdash{}{0pt}%
\pgfpathmoveto{\pgfqpoint{1.644123in}{2.109667in}}%
\pgfpathcurveto{\pgfqpoint{1.652359in}{2.109667in}}{\pgfqpoint{1.660259in}{2.112939in}}{\pgfqpoint{1.666083in}{2.118763in}}%
\pgfpathcurveto{\pgfqpoint{1.671907in}{2.124587in}}{\pgfqpoint{1.675180in}{2.132487in}}{\pgfqpoint{1.675180in}{2.140724in}}%
\pgfpathcurveto{\pgfqpoint{1.675180in}{2.148960in}}{\pgfqpoint{1.671907in}{2.156860in}}{\pgfqpoint{1.666083in}{2.162684in}}%
\pgfpathcurveto{\pgfqpoint{1.660259in}{2.168508in}}{\pgfqpoint{1.652359in}{2.171780in}}{\pgfqpoint{1.644123in}{2.171780in}}%
\pgfpathcurveto{\pgfqpoint{1.635887in}{2.171780in}}{\pgfqpoint{1.627987in}{2.168508in}}{\pgfqpoint{1.622163in}{2.162684in}}%
\pgfpathcurveto{\pgfqpoint{1.616339in}{2.156860in}}{\pgfqpoint{1.613067in}{2.148960in}}{\pgfqpoint{1.613067in}{2.140724in}}%
\pgfpathcurveto{\pgfqpoint{1.613067in}{2.132487in}}{\pgfqpoint{1.616339in}{2.124587in}}{\pgfqpoint{1.622163in}{2.118763in}}%
\pgfpathcurveto{\pgfqpoint{1.627987in}{2.112939in}}{\pgfqpoint{1.635887in}{2.109667in}}{\pgfqpoint{1.644123in}{2.109667in}}%
\pgfpathclose%
\pgfusepath{stroke,fill}%
\end{pgfscope}%
\begin{pgfscope}%
\pgfpathrectangle{\pgfqpoint{0.100000in}{0.212622in}}{\pgfqpoint{3.696000in}{3.696000in}}%
\pgfusepath{clip}%
\pgfsetbuttcap%
\pgfsetroundjoin%
\definecolor{currentfill}{rgb}{0.121569,0.466667,0.705882}%
\pgfsetfillcolor{currentfill}%
\pgfsetfillopacity{0.301449}%
\pgfsetlinewidth{1.003750pt}%
\definecolor{currentstroke}{rgb}{0.121569,0.466667,0.705882}%
\pgfsetstrokecolor{currentstroke}%
\pgfsetstrokeopacity{0.301449}%
\pgfsetdash{}{0pt}%
\pgfpathmoveto{\pgfqpoint{1.644123in}{2.109667in}}%
\pgfpathcurveto{\pgfqpoint{1.652359in}{2.109667in}}{\pgfqpoint{1.660259in}{2.112939in}}{\pgfqpoint{1.666083in}{2.118763in}}%
\pgfpathcurveto{\pgfqpoint{1.671907in}{2.124587in}}{\pgfqpoint{1.675180in}{2.132487in}}{\pgfqpoint{1.675180in}{2.140724in}}%
\pgfpathcurveto{\pgfqpoint{1.675180in}{2.148960in}}{\pgfqpoint{1.671907in}{2.156860in}}{\pgfqpoint{1.666083in}{2.162684in}}%
\pgfpathcurveto{\pgfqpoint{1.660259in}{2.168508in}}{\pgfqpoint{1.652359in}{2.171780in}}{\pgfqpoint{1.644123in}{2.171780in}}%
\pgfpathcurveto{\pgfqpoint{1.635887in}{2.171780in}}{\pgfqpoint{1.627987in}{2.168508in}}{\pgfqpoint{1.622163in}{2.162684in}}%
\pgfpathcurveto{\pgfqpoint{1.616339in}{2.156860in}}{\pgfqpoint{1.613067in}{2.148960in}}{\pgfqpoint{1.613067in}{2.140724in}}%
\pgfpathcurveto{\pgfqpoint{1.613067in}{2.132487in}}{\pgfqpoint{1.616339in}{2.124587in}}{\pgfqpoint{1.622163in}{2.118763in}}%
\pgfpathcurveto{\pgfqpoint{1.627987in}{2.112939in}}{\pgfqpoint{1.635887in}{2.109667in}}{\pgfqpoint{1.644123in}{2.109667in}}%
\pgfpathclose%
\pgfusepath{stroke,fill}%
\end{pgfscope}%
\begin{pgfscope}%
\pgfpathrectangle{\pgfqpoint{0.100000in}{0.212622in}}{\pgfqpoint{3.696000in}{3.696000in}}%
\pgfusepath{clip}%
\pgfsetbuttcap%
\pgfsetroundjoin%
\definecolor{currentfill}{rgb}{0.121569,0.466667,0.705882}%
\pgfsetfillcolor{currentfill}%
\pgfsetfillopacity{0.301449}%
\pgfsetlinewidth{1.003750pt}%
\definecolor{currentstroke}{rgb}{0.121569,0.466667,0.705882}%
\pgfsetstrokecolor{currentstroke}%
\pgfsetstrokeopacity{0.301449}%
\pgfsetdash{}{0pt}%
\pgfpathmoveto{\pgfqpoint{1.644123in}{2.109667in}}%
\pgfpathcurveto{\pgfqpoint{1.652359in}{2.109667in}}{\pgfqpoint{1.660259in}{2.112939in}}{\pgfqpoint{1.666083in}{2.118763in}}%
\pgfpathcurveto{\pgfqpoint{1.671907in}{2.124587in}}{\pgfqpoint{1.675180in}{2.132487in}}{\pgfqpoint{1.675180in}{2.140724in}}%
\pgfpathcurveto{\pgfqpoint{1.675180in}{2.148960in}}{\pgfqpoint{1.671907in}{2.156860in}}{\pgfqpoint{1.666083in}{2.162684in}}%
\pgfpathcurveto{\pgfqpoint{1.660259in}{2.168508in}}{\pgfqpoint{1.652359in}{2.171780in}}{\pgfqpoint{1.644123in}{2.171780in}}%
\pgfpathcurveto{\pgfqpoint{1.635887in}{2.171780in}}{\pgfqpoint{1.627987in}{2.168508in}}{\pgfqpoint{1.622163in}{2.162684in}}%
\pgfpathcurveto{\pgfqpoint{1.616339in}{2.156860in}}{\pgfqpoint{1.613067in}{2.148960in}}{\pgfqpoint{1.613067in}{2.140724in}}%
\pgfpathcurveto{\pgfqpoint{1.613067in}{2.132487in}}{\pgfqpoint{1.616339in}{2.124587in}}{\pgfqpoint{1.622163in}{2.118763in}}%
\pgfpathcurveto{\pgfqpoint{1.627987in}{2.112939in}}{\pgfqpoint{1.635887in}{2.109667in}}{\pgfqpoint{1.644123in}{2.109667in}}%
\pgfpathclose%
\pgfusepath{stroke,fill}%
\end{pgfscope}%
\begin{pgfscope}%
\pgfpathrectangle{\pgfqpoint{0.100000in}{0.212622in}}{\pgfqpoint{3.696000in}{3.696000in}}%
\pgfusepath{clip}%
\pgfsetbuttcap%
\pgfsetroundjoin%
\definecolor{currentfill}{rgb}{0.121569,0.466667,0.705882}%
\pgfsetfillcolor{currentfill}%
\pgfsetfillopacity{0.301449}%
\pgfsetlinewidth{1.003750pt}%
\definecolor{currentstroke}{rgb}{0.121569,0.466667,0.705882}%
\pgfsetstrokecolor{currentstroke}%
\pgfsetstrokeopacity{0.301449}%
\pgfsetdash{}{0pt}%
\pgfpathmoveto{\pgfqpoint{1.644123in}{2.109667in}}%
\pgfpathcurveto{\pgfqpoint{1.652359in}{2.109667in}}{\pgfqpoint{1.660259in}{2.112939in}}{\pgfqpoint{1.666083in}{2.118763in}}%
\pgfpathcurveto{\pgfqpoint{1.671907in}{2.124587in}}{\pgfqpoint{1.675180in}{2.132487in}}{\pgfqpoint{1.675180in}{2.140724in}}%
\pgfpathcurveto{\pgfqpoint{1.675180in}{2.148960in}}{\pgfqpoint{1.671907in}{2.156860in}}{\pgfqpoint{1.666083in}{2.162684in}}%
\pgfpathcurveto{\pgfqpoint{1.660259in}{2.168508in}}{\pgfqpoint{1.652359in}{2.171780in}}{\pgfqpoint{1.644123in}{2.171780in}}%
\pgfpathcurveto{\pgfqpoint{1.635887in}{2.171780in}}{\pgfqpoint{1.627987in}{2.168508in}}{\pgfqpoint{1.622163in}{2.162684in}}%
\pgfpathcurveto{\pgfqpoint{1.616339in}{2.156860in}}{\pgfqpoint{1.613067in}{2.148960in}}{\pgfqpoint{1.613067in}{2.140724in}}%
\pgfpathcurveto{\pgfqpoint{1.613067in}{2.132487in}}{\pgfqpoint{1.616339in}{2.124587in}}{\pgfqpoint{1.622163in}{2.118763in}}%
\pgfpathcurveto{\pgfqpoint{1.627987in}{2.112939in}}{\pgfqpoint{1.635887in}{2.109667in}}{\pgfqpoint{1.644123in}{2.109667in}}%
\pgfpathclose%
\pgfusepath{stroke,fill}%
\end{pgfscope}%
\begin{pgfscope}%
\pgfpathrectangle{\pgfqpoint{0.100000in}{0.212622in}}{\pgfqpoint{3.696000in}{3.696000in}}%
\pgfusepath{clip}%
\pgfsetbuttcap%
\pgfsetroundjoin%
\definecolor{currentfill}{rgb}{0.121569,0.466667,0.705882}%
\pgfsetfillcolor{currentfill}%
\pgfsetfillopacity{0.301449}%
\pgfsetlinewidth{1.003750pt}%
\definecolor{currentstroke}{rgb}{0.121569,0.466667,0.705882}%
\pgfsetstrokecolor{currentstroke}%
\pgfsetstrokeopacity{0.301449}%
\pgfsetdash{}{0pt}%
\pgfpathmoveto{\pgfqpoint{1.644123in}{2.109667in}}%
\pgfpathcurveto{\pgfqpoint{1.652359in}{2.109667in}}{\pgfqpoint{1.660259in}{2.112939in}}{\pgfqpoint{1.666083in}{2.118763in}}%
\pgfpathcurveto{\pgfqpoint{1.671907in}{2.124587in}}{\pgfqpoint{1.675180in}{2.132487in}}{\pgfqpoint{1.675180in}{2.140724in}}%
\pgfpathcurveto{\pgfqpoint{1.675180in}{2.148960in}}{\pgfqpoint{1.671907in}{2.156860in}}{\pgfqpoint{1.666083in}{2.162684in}}%
\pgfpathcurveto{\pgfqpoint{1.660259in}{2.168508in}}{\pgfqpoint{1.652359in}{2.171780in}}{\pgfqpoint{1.644123in}{2.171780in}}%
\pgfpathcurveto{\pgfqpoint{1.635887in}{2.171780in}}{\pgfqpoint{1.627987in}{2.168508in}}{\pgfqpoint{1.622163in}{2.162684in}}%
\pgfpathcurveto{\pgfqpoint{1.616339in}{2.156860in}}{\pgfqpoint{1.613067in}{2.148960in}}{\pgfqpoint{1.613067in}{2.140724in}}%
\pgfpathcurveto{\pgfqpoint{1.613067in}{2.132487in}}{\pgfqpoint{1.616339in}{2.124587in}}{\pgfqpoint{1.622163in}{2.118763in}}%
\pgfpathcurveto{\pgfqpoint{1.627987in}{2.112939in}}{\pgfqpoint{1.635887in}{2.109667in}}{\pgfqpoint{1.644123in}{2.109667in}}%
\pgfpathclose%
\pgfusepath{stroke,fill}%
\end{pgfscope}%
\begin{pgfscope}%
\pgfpathrectangle{\pgfqpoint{0.100000in}{0.212622in}}{\pgfqpoint{3.696000in}{3.696000in}}%
\pgfusepath{clip}%
\pgfsetbuttcap%
\pgfsetroundjoin%
\definecolor{currentfill}{rgb}{0.121569,0.466667,0.705882}%
\pgfsetfillcolor{currentfill}%
\pgfsetfillopacity{0.301449}%
\pgfsetlinewidth{1.003750pt}%
\definecolor{currentstroke}{rgb}{0.121569,0.466667,0.705882}%
\pgfsetstrokecolor{currentstroke}%
\pgfsetstrokeopacity{0.301449}%
\pgfsetdash{}{0pt}%
\pgfpathmoveto{\pgfqpoint{1.644123in}{2.109667in}}%
\pgfpathcurveto{\pgfqpoint{1.652359in}{2.109667in}}{\pgfqpoint{1.660259in}{2.112939in}}{\pgfqpoint{1.666083in}{2.118763in}}%
\pgfpathcurveto{\pgfqpoint{1.671907in}{2.124587in}}{\pgfqpoint{1.675180in}{2.132487in}}{\pgfqpoint{1.675180in}{2.140724in}}%
\pgfpathcurveto{\pgfqpoint{1.675180in}{2.148960in}}{\pgfqpoint{1.671907in}{2.156860in}}{\pgfqpoint{1.666083in}{2.162684in}}%
\pgfpathcurveto{\pgfqpoint{1.660259in}{2.168508in}}{\pgfqpoint{1.652359in}{2.171780in}}{\pgfqpoint{1.644123in}{2.171780in}}%
\pgfpathcurveto{\pgfqpoint{1.635887in}{2.171780in}}{\pgfqpoint{1.627987in}{2.168508in}}{\pgfqpoint{1.622163in}{2.162684in}}%
\pgfpathcurveto{\pgfqpoint{1.616339in}{2.156860in}}{\pgfqpoint{1.613067in}{2.148960in}}{\pgfqpoint{1.613067in}{2.140724in}}%
\pgfpathcurveto{\pgfqpoint{1.613067in}{2.132487in}}{\pgfqpoint{1.616339in}{2.124587in}}{\pgfqpoint{1.622163in}{2.118763in}}%
\pgfpathcurveto{\pgfqpoint{1.627987in}{2.112939in}}{\pgfqpoint{1.635887in}{2.109667in}}{\pgfqpoint{1.644123in}{2.109667in}}%
\pgfpathclose%
\pgfusepath{stroke,fill}%
\end{pgfscope}%
\begin{pgfscope}%
\pgfpathrectangle{\pgfqpoint{0.100000in}{0.212622in}}{\pgfqpoint{3.696000in}{3.696000in}}%
\pgfusepath{clip}%
\pgfsetbuttcap%
\pgfsetroundjoin%
\definecolor{currentfill}{rgb}{0.121569,0.466667,0.705882}%
\pgfsetfillcolor{currentfill}%
\pgfsetfillopacity{0.301449}%
\pgfsetlinewidth{1.003750pt}%
\definecolor{currentstroke}{rgb}{0.121569,0.466667,0.705882}%
\pgfsetstrokecolor{currentstroke}%
\pgfsetstrokeopacity{0.301449}%
\pgfsetdash{}{0pt}%
\pgfpathmoveto{\pgfqpoint{1.644123in}{2.109667in}}%
\pgfpathcurveto{\pgfqpoint{1.652359in}{2.109667in}}{\pgfqpoint{1.660259in}{2.112939in}}{\pgfqpoint{1.666083in}{2.118763in}}%
\pgfpathcurveto{\pgfqpoint{1.671907in}{2.124587in}}{\pgfqpoint{1.675180in}{2.132487in}}{\pgfqpoint{1.675180in}{2.140724in}}%
\pgfpathcurveto{\pgfqpoint{1.675180in}{2.148960in}}{\pgfqpoint{1.671907in}{2.156860in}}{\pgfqpoint{1.666083in}{2.162684in}}%
\pgfpathcurveto{\pgfqpoint{1.660259in}{2.168508in}}{\pgfqpoint{1.652359in}{2.171780in}}{\pgfqpoint{1.644123in}{2.171780in}}%
\pgfpathcurveto{\pgfqpoint{1.635887in}{2.171780in}}{\pgfqpoint{1.627987in}{2.168508in}}{\pgfqpoint{1.622163in}{2.162684in}}%
\pgfpathcurveto{\pgfqpoint{1.616339in}{2.156860in}}{\pgfqpoint{1.613067in}{2.148960in}}{\pgfqpoint{1.613067in}{2.140724in}}%
\pgfpathcurveto{\pgfqpoint{1.613067in}{2.132487in}}{\pgfqpoint{1.616339in}{2.124587in}}{\pgfqpoint{1.622163in}{2.118763in}}%
\pgfpathcurveto{\pgfqpoint{1.627987in}{2.112939in}}{\pgfqpoint{1.635887in}{2.109667in}}{\pgfqpoint{1.644123in}{2.109667in}}%
\pgfpathclose%
\pgfusepath{stroke,fill}%
\end{pgfscope}%
\begin{pgfscope}%
\pgfpathrectangle{\pgfqpoint{0.100000in}{0.212622in}}{\pgfqpoint{3.696000in}{3.696000in}}%
\pgfusepath{clip}%
\pgfsetbuttcap%
\pgfsetroundjoin%
\definecolor{currentfill}{rgb}{0.121569,0.466667,0.705882}%
\pgfsetfillcolor{currentfill}%
\pgfsetfillopacity{0.301449}%
\pgfsetlinewidth{1.003750pt}%
\definecolor{currentstroke}{rgb}{0.121569,0.466667,0.705882}%
\pgfsetstrokecolor{currentstroke}%
\pgfsetstrokeopacity{0.301449}%
\pgfsetdash{}{0pt}%
\pgfpathmoveto{\pgfqpoint{1.644123in}{2.109667in}}%
\pgfpathcurveto{\pgfqpoint{1.652359in}{2.109667in}}{\pgfqpoint{1.660259in}{2.112939in}}{\pgfqpoint{1.666083in}{2.118763in}}%
\pgfpathcurveto{\pgfqpoint{1.671907in}{2.124587in}}{\pgfqpoint{1.675180in}{2.132487in}}{\pgfqpoint{1.675180in}{2.140724in}}%
\pgfpathcurveto{\pgfqpoint{1.675180in}{2.148960in}}{\pgfqpoint{1.671907in}{2.156860in}}{\pgfqpoint{1.666083in}{2.162684in}}%
\pgfpathcurveto{\pgfqpoint{1.660259in}{2.168508in}}{\pgfqpoint{1.652359in}{2.171780in}}{\pgfqpoint{1.644123in}{2.171780in}}%
\pgfpathcurveto{\pgfqpoint{1.635887in}{2.171780in}}{\pgfqpoint{1.627987in}{2.168508in}}{\pgfqpoint{1.622163in}{2.162684in}}%
\pgfpathcurveto{\pgfqpoint{1.616339in}{2.156860in}}{\pgfqpoint{1.613067in}{2.148960in}}{\pgfqpoint{1.613067in}{2.140724in}}%
\pgfpathcurveto{\pgfqpoint{1.613067in}{2.132487in}}{\pgfqpoint{1.616339in}{2.124587in}}{\pgfqpoint{1.622163in}{2.118763in}}%
\pgfpathcurveto{\pgfqpoint{1.627987in}{2.112939in}}{\pgfqpoint{1.635887in}{2.109667in}}{\pgfqpoint{1.644123in}{2.109667in}}%
\pgfpathclose%
\pgfusepath{stroke,fill}%
\end{pgfscope}%
\begin{pgfscope}%
\pgfpathrectangle{\pgfqpoint{0.100000in}{0.212622in}}{\pgfqpoint{3.696000in}{3.696000in}}%
\pgfusepath{clip}%
\pgfsetbuttcap%
\pgfsetroundjoin%
\definecolor{currentfill}{rgb}{0.121569,0.466667,0.705882}%
\pgfsetfillcolor{currentfill}%
\pgfsetfillopacity{0.301449}%
\pgfsetlinewidth{1.003750pt}%
\definecolor{currentstroke}{rgb}{0.121569,0.466667,0.705882}%
\pgfsetstrokecolor{currentstroke}%
\pgfsetstrokeopacity{0.301449}%
\pgfsetdash{}{0pt}%
\pgfpathmoveto{\pgfqpoint{1.644123in}{2.109667in}}%
\pgfpathcurveto{\pgfqpoint{1.652359in}{2.109667in}}{\pgfqpoint{1.660259in}{2.112939in}}{\pgfqpoint{1.666083in}{2.118763in}}%
\pgfpathcurveto{\pgfqpoint{1.671907in}{2.124587in}}{\pgfqpoint{1.675180in}{2.132487in}}{\pgfqpoint{1.675180in}{2.140724in}}%
\pgfpathcurveto{\pgfqpoint{1.675180in}{2.148960in}}{\pgfqpoint{1.671907in}{2.156860in}}{\pgfqpoint{1.666083in}{2.162684in}}%
\pgfpathcurveto{\pgfqpoint{1.660259in}{2.168508in}}{\pgfqpoint{1.652359in}{2.171780in}}{\pgfqpoint{1.644123in}{2.171780in}}%
\pgfpathcurveto{\pgfqpoint{1.635887in}{2.171780in}}{\pgfqpoint{1.627987in}{2.168508in}}{\pgfqpoint{1.622163in}{2.162684in}}%
\pgfpathcurveto{\pgfqpoint{1.616339in}{2.156860in}}{\pgfqpoint{1.613067in}{2.148960in}}{\pgfqpoint{1.613067in}{2.140724in}}%
\pgfpathcurveto{\pgfqpoint{1.613067in}{2.132487in}}{\pgfqpoint{1.616339in}{2.124587in}}{\pgfqpoint{1.622163in}{2.118763in}}%
\pgfpathcurveto{\pgfqpoint{1.627987in}{2.112939in}}{\pgfqpoint{1.635887in}{2.109667in}}{\pgfqpoint{1.644123in}{2.109667in}}%
\pgfpathclose%
\pgfusepath{stroke,fill}%
\end{pgfscope}%
\begin{pgfscope}%
\pgfpathrectangle{\pgfqpoint{0.100000in}{0.212622in}}{\pgfqpoint{3.696000in}{3.696000in}}%
\pgfusepath{clip}%
\pgfsetbuttcap%
\pgfsetroundjoin%
\definecolor{currentfill}{rgb}{0.121569,0.466667,0.705882}%
\pgfsetfillcolor{currentfill}%
\pgfsetfillopacity{0.301449}%
\pgfsetlinewidth{1.003750pt}%
\definecolor{currentstroke}{rgb}{0.121569,0.466667,0.705882}%
\pgfsetstrokecolor{currentstroke}%
\pgfsetstrokeopacity{0.301449}%
\pgfsetdash{}{0pt}%
\pgfpathmoveto{\pgfqpoint{1.644123in}{2.109667in}}%
\pgfpathcurveto{\pgfqpoint{1.652359in}{2.109667in}}{\pgfqpoint{1.660259in}{2.112939in}}{\pgfqpoint{1.666083in}{2.118763in}}%
\pgfpathcurveto{\pgfqpoint{1.671907in}{2.124587in}}{\pgfqpoint{1.675180in}{2.132487in}}{\pgfqpoint{1.675180in}{2.140724in}}%
\pgfpathcurveto{\pgfqpoint{1.675180in}{2.148960in}}{\pgfqpoint{1.671907in}{2.156860in}}{\pgfqpoint{1.666083in}{2.162684in}}%
\pgfpathcurveto{\pgfqpoint{1.660259in}{2.168508in}}{\pgfqpoint{1.652359in}{2.171780in}}{\pgfqpoint{1.644123in}{2.171780in}}%
\pgfpathcurveto{\pgfqpoint{1.635887in}{2.171780in}}{\pgfqpoint{1.627987in}{2.168508in}}{\pgfqpoint{1.622163in}{2.162684in}}%
\pgfpathcurveto{\pgfqpoint{1.616339in}{2.156860in}}{\pgfqpoint{1.613067in}{2.148960in}}{\pgfqpoint{1.613067in}{2.140724in}}%
\pgfpathcurveto{\pgfqpoint{1.613067in}{2.132487in}}{\pgfqpoint{1.616339in}{2.124587in}}{\pgfqpoint{1.622163in}{2.118763in}}%
\pgfpathcurveto{\pgfqpoint{1.627987in}{2.112939in}}{\pgfqpoint{1.635887in}{2.109667in}}{\pgfqpoint{1.644123in}{2.109667in}}%
\pgfpathclose%
\pgfusepath{stroke,fill}%
\end{pgfscope}%
\begin{pgfscope}%
\pgfpathrectangle{\pgfqpoint{0.100000in}{0.212622in}}{\pgfqpoint{3.696000in}{3.696000in}}%
\pgfusepath{clip}%
\pgfsetbuttcap%
\pgfsetroundjoin%
\definecolor{currentfill}{rgb}{0.121569,0.466667,0.705882}%
\pgfsetfillcolor{currentfill}%
\pgfsetfillopacity{0.301449}%
\pgfsetlinewidth{1.003750pt}%
\definecolor{currentstroke}{rgb}{0.121569,0.466667,0.705882}%
\pgfsetstrokecolor{currentstroke}%
\pgfsetstrokeopacity{0.301449}%
\pgfsetdash{}{0pt}%
\pgfpathmoveto{\pgfqpoint{1.644123in}{2.109667in}}%
\pgfpathcurveto{\pgfqpoint{1.652359in}{2.109667in}}{\pgfqpoint{1.660259in}{2.112939in}}{\pgfqpoint{1.666083in}{2.118763in}}%
\pgfpathcurveto{\pgfqpoint{1.671907in}{2.124587in}}{\pgfqpoint{1.675180in}{2.132487in}}{\pgfqpoint{1.675180in}{2.140724in}}%
\pgfpathcurveto{\pgfqpoint{1.675180in}{2.148960in}}{\pgfqpoint{1.671907in}{2.156860in}}{\pgfqpoint{1.666083in}{2.162684in}}%
\pgfpathcurveto{\pgfqpoint{1.660259in}{2.168508in}}{\pgfqpoint{1.652359in}{2.171780in}}{\pgfqpoint{1.644123in}{2.171780in}}%
\pgfpathcurveto{\pgfqpoint{1.635887in}{2.171780in}}{\pgfqpoint{1.627987in}{2.168508in}}{\pgfqpoint{1.622163in}{2.162684in}}%
\pgfpathcurveto{\pgfqpoint{1.616339in}{2.156860in}}{\pgfqpoint{1.613067in}{2.148960in}}{\pgfqpoint{1.613067in}{2.140724in}}%
\pgfpathcurveto{\pgfqpoint{1.613067in}{2.132487in}}{\pgfqpoint{1.616339in}{2.124587in}}{\pgfqpoint{1.622163in}{2.118763in}}%
\pgfpathcurveto{\pgfqpoint{1.627987in}{2.112939in}}{\pgfqpoint{1.635887in}{2.109667in}}{\pgfqpoint{1.644123in}{2.109667in}}%
\pgfpathclose%
\pgfusepath{stroke,fill}%
\end{pgfscope}%
\begin{pgfscope}%
\pgfpathrectangle{\pgfqpoint{0.100000in}{0.212622in}}{\pgfqpoint{3.696000in}{3.696000in}}%
\pgfusepath{clip}%
\pgfsetbuttcap%
\pgfsetroundjoin%
\definecolor{currentfill}{rgb}{0.121569,0.466667,0.705882}%
\pgfsetfillcolor{currentfill}%
\pgfsetfillopacity{0.301449}%
\pgfsetlinewidth{1.003750pt}%
\definecolor{currentstroke}{rgb}{0.121569,0.466667,0.705882}%
\pgfsetstrokecolor{currentstroke}%
\pgfsetstrokeopacity{0.301449}%
\pgfsetdash{}{0pt}%
\pgfpathmoveto{\pgfqpoint{1.644123in}{2.109667in}}%
\pgfpathcurveto{\pgfqpoint{1.652359in}{2.109667in}}{\pgfqpoint{1.660259in}{2.112939in}}{\pgfqpoint{1.666083in}{2.118763in}}%
\pgfpathcurveto{\pgfqpoint{1.671907in}{2.124587in}}{\pgfqpoint{1.675180in}{2.132487in}}{\pgfqpoint{1.675180in}{2.140724in}}%
\pgfpathcurveto{\pgfqpoint{1.675180in}{2.148960in}}{\pgfqpoint{1.671907in}{2.156860in}}{\pgfqpoint{1.666083in}{2.162684in}}%
\pgfpathcurveto{\pgfqpoint{1.660259in}{2.168508in}}{\pgfqpoint{1.652359in}{2.171780in}}{\pgfqpoint{1.644123in}{2.171780in}}%
\pgfpathcurveto{\pgfqpoint{1.635887in}{2.171780in}}{\pgfqpoint{1.627987in}{2.168508in}}{\pgfqpoint{1.622163in}{2.162684in}}%
\pgfpathcurveto{\pgfqpoint{1.616339in}{2.156860in}}{\pgfqpoint{1.613067in}{2.148960in}}{\pgfqpoint{1.613067in}{2.140724in}}%
\pgfpathcurveto{\pgfqpoint{1.613067in}{2.132487in}}{\pgfqpoint{1.616339in}{2.124587in}}{\pgfqpoint{1.622163in}{2.118763in}}%
\pgfpathcurveto{\pgfqpoint{1.627987in}{2.112939in}}{\pgfqpoint{1.635887in}{2.109667in}}{\pgfqpoint{1.644123in}{2.109667in}}%
\pgfpathclose%
\pgfusepath{stroke,fill}%
\end{pgfscope}%
\begin{pgfscope}%
\pgfpathrectangle{\pgfqpoint{0.100000in}{0.212622in}}{\pgfqpoint{3.696000in}{3.696000in}}%
\pgfusepath{clip}%
\pgfsetbuttcap%
\pgfsetroundjoin%
\definecolor{currentfill}{rgb}{0.121569,0.466667,0.705882}%
\pgfsetfillcolor{currentfill}%
\pgfsetfillopacity{0.301449}%
\pgfsetlinewidth{1.003750pt}%
\definecolor{currentstroke}{rgb}{0.121569,0.466667,0.705882}%
\pgfsetstrokecolor{currentstroke}%
\pgfsetstrokeopacity{0.301449}%
\pgfsetdash{}{0pt}%
\pgfpathmoveto{\pgfqpoint{1.644123in}{2.109667in}}%
\pgfpathcurveto{\pgfqpoint{1.652359in}{2.109667in}}{\pgfqpoint{1.660259in}{2.112939in}}{\pgfqpoint{1.666083in}{2.118763in}}%
\pgfpathcurveto{\pgfqpoint{1.671907in}{2.124587in}}{\pgfqpoint{1.675180in}{2.132487in}}{\pgfqpoint{1.675180in}{2.140724in}}%
\pgfpathcurveto{\pgfqpoint{1.675180in}{2.148960in}}{\pgfqpoint{1.671907in}{2.156860in}}{\pgfqpoint{1.666083in}{2.162684in}}%
\pgfpathcurveto{\pgfqpoint{1.660259in}{2.168508in}}{\pgfqpoint{1.652359in}{2.171780in}}{\pgfqpoint{1.644123in}{2.171780in}}%
\pgfpathcurveto{\pgfqpoint{1.635887in}{2.171780in}}{\pgfqpoint{1.627987in}{2.168508in}}{\pgfqpoint{1.622163in}{2.162684in}}%
\pgfpathcurveto{\pgfqpoint{1.616339in}{2.156860in}}{\pgfqpoint{1.613067in}{2.148960in}}{\pgfqpoint{1.613067in}{2.140724in}}%
\pgfpathcurveto{\pgfqpoint{1.613067in}{2.132487in}}{\pgfqpoint{1.616339in}{2.124587in}}{\pgfqpoint{1.622163in}{2.118763in}}%
\pgfpathcurveto{\pgfqpoint{1.627987in}{2.112939in}}{\pgfqpoint{1.635887in}{2.109667in}}{\pgfqpoint{1.644123in}{2.109667in}}%
\pgfpathclose%
\pgfusepath{stroke,fill}%
\end{pgfscope}%
\begin{pgfscope}%
\pgfpathrectangle{\pgfqpoint{0.100000in}{0.212622in}}{\pgfqpoint{3.696000in}{3.696000in}}%
\pgfusepath{clip}%
\pgfsetbuttcap%
\pgfsetroundjoin%
\definecolor{currentfill}{rgb}{0.121569,0.466667,0.705882}%
\pgfsetfillcolor{currentfill}%
\pgfsetfillopacity{0.301449}%
\pgfsetlinewidth{1.003750pt}%
\definecolor{currentstroke}{rgb}{0.121569,0.466667,0.705882}%
\pgfsetstrokecolor{currentstroke}%
\pgfsetstrokeopacity{0.301449}%
\pgfsetdash{}{0pt}%
\pgfpathmoveto{\pgfqpoint{1.644123in}{2.109667in}}%
\pgfpathcurveto{\pgfqpoint{1.652359in}{2.109667in}}{\pgfqpoint{1.660259in}{2.112939in}}{\pgfqpoint{1.666083in}{2.118763in}}%
\pgfpathcurveto{\pgfqpoint{1.671907in}{2.124587in}}{\pgfqpoint{1.675180in}{2.132487in}}{\pgfqpoint{1.675180in}{2.140724in}}%
\pgfpathcurveto{\pgfqpoint{1.675180in}{2.148960in}}{\pgfqpoint{1.671907in}{2.156860in}}{\pgfqpoint{1.666083in}{2.162684in}}%
\pgfpathcurveto{\pgfqpoint{1.660259in}{2.168508in}}{\pgfqpoint{1.652359in}{2.171780in}}{\pgfqpoint{1.644123in}{2.171780in}}%
\pgfpathcurveto{\pgfqpoint{1.635887in}{2.171780in}}{\pgfqpoint{1.627987in}{2.168508in}}{\pgfqpoint{1.622163in}{2.162684in}}%
\pgfpathcurveto{\pgfqpoint{1.616339in}{2.156860in}}{\pgfqpoint{1.613067in}{2.148960in}}{\pgfqpoint{1.613067in}{2.140724in}}%
\pgfpathcurveto{\pgfqpoint{1.613067in}{2.132487in}}{\pgfqpoint{1.616339in}{2.124587in}}{\pgfqpoint{1.622163in}{2.118763in}}%
\pgfpathcurveto{\pgfqpoint{1.627987in}{2.112939in}}{\pgfqpoint{1.635887in}{2.109667in}}{\pgfqpoint{1.644123in}{2.109667in}}%
\pgfpathclose%
\pgfusepath{stroke,fill}%
\end{pgfscope}%
\begin{pgfscope}%
\pgfpathrectangle{\pgfqpoint{0.100000in}{0.212622in}}{\pgfqpoint{3.696000in}{3.696000in}}%
\pgfusepath{clip}%
\pgfsetbuttcap%
\pgfsetroundjoin%
\definecolor{currentfill}{rgb}{0.121569,0.466667,0.705882}%
\pgfsetfillcolor{currentfill}%
\pgfsetfillopacity{0.301449}%
\pgfsetlinewidth{1.003750pt}%
\definecolor{currentstroke}{rgb}{0.121569,0.466667,0.705882}%
\pgfsetstrokecolor{currentstroke}%
\pgfsetstrokeopacity{0.301449}%
\pgfsetdash{}{0pt}%
\pgfpathmoveto{\pgfqpoint{1.644123in}{2.109667in}}%
\pgfpathcurveto{\pgfqpoint{1.652359in}{2.109667in}}{\pgfqpoint{1.660259in}{2.112939in}}{\pgfqpoint{1.666083in}{2.118763in}}%
\pgfpathcurveto{\pgfqpoint{1.671907in}{2.124587in}}{\pgfqpoint{1.675180in}{2.132487in}}{\pgfqpoint{1.675180in}{2.140724in}}%
\pgfpathcurveto{\pgfqpoint{1.675180in}{2.148960in}}{\pgfqpoint{1.671907in}{2.156860in}}{\pgfqpoint{1.666083in}{2.162684in}}%
\pgfpathcurveto{\pgfqpoint{1.660259in}{2.168508in}}{\pgfqpoint{1.652359in}{2.171780in}}{\pgfqpoint{1.644123in}{2.171780in}}%
\pgfpathcurveto{\pgfqpoint{1.635887in}{2.171780in}}{\pgfqpoint{1.627987in}{2.168508in}}{\pgfqpoint{1.622163in}{2.162684in}}%
\pgfpathcurveto{\pgfqpoint{1.616339in}{2.156860in}}{\pgfqpoint{1.613067in}{2.148960in}}{\pgfqpoint{1.613067in}{2.140724in}}%
\pgfpathcurveto{\pgfqpoint{1.613067in}{2.132487in}}{\pgfqpoint{1.616339in}{2.124587in}}{\pgfqpoint{1.622163in}{2.118763in}}%
\pgfpathcurveto{\pgfqpoint{1.627987in}{2.112939in}}{\pgfqpoint{1.635887in}{2.109667in}}{\pgfqpoint{1.644123in}{2.109667in}}%
\pgfpathclose%
\pgfusepath{stroke,fill}%
\end{pgfscope}%
\begin{pgfscope}%
\pgfpathrectangle{\pgfqpoint{0.100000in}{0.212622in}}{\pgfqpoint{3.696000in}{3.696000in}}%
\pgfusepath{clip}%
\pgfsetbuttcap%
\pgfsetroundjoin%
\definecolor{currentfill}{rgb}{0.121569,0.466667,0.705882}%
\pgfsetfillcolor{currentfill}%
\pgfsetfillopacity{0.301449}%
\pgfsetlinewidth{1.003750pt}%
\definecolor{currentstroke}{rgb}{0.121569,0.466667,0.705882}%
\pgfsetstrokecolor{currentstroke}%
\pgfsetstrokeopacity{0.301449}%
\pgfsetdash{}{0pt}%
\pgfpathmoveto{\pgfqpoint{1.644123in}{2.109667in}}%
\pgfpathcurveto{\pgfqpoint{1.652359in}{2.109667in}}{\pgfqpoint{1.660259in}{2.112939in}}{\pgfqpoint{1.666083in}{2.118763in}}%
\pgfpathcurveto{\pgfqpoint{1.671907in}{2.124587in}}{\pgfqpoint{1.675180in}{2.132487in}}{\pgfqpoint{1.675180in}{2.140724in}}%
\pgfpathcurveto{\pgfqpoint{1.675180in}{2.148960in}}{\pgfqpoint{1.671907in}{2.156860in}}{\pgfqpoint{1.666083in}{2.162684in}}%
\pgfpathcurveto{\pgfqpoint{1.660259in}{2.168508in}}{\pgfqpoint{1.652359in}{2.171780in}}{\pgfqpoint{1.644123in}{2.171780in}}%
\pgfpathcurveto{\pgfqpoint{1.635887in}{2.171780in}}{\pgfqpoint{1.627987in}{2.168508in}}{\pgfqpoint{1.622163in}{2.162684in}}%
\pgfpathcurveto{\pgfqpoint{1.616339in}{2.156860in}}{\pgfqpoint{1.613067in}{2.148960in}}{\pgfqpoint{1.613067in}{2.140724in}}%
\pgfpathcurveto{\pgfqpoint{1.613067in}{2.132487in}}{\pgfqpoint{1.616339in}{2.124587in}}{\pgfqpoint{1.622163in}{2.118763in}}%
\pgfpathcurveto{\pgfqpoint{1.627987in}{2.112939in}}{\pgfqpoint{1.635887in}{2.109667in}}{\pgfqpoint{1.644123in}{2.109667in}}%
\pgfpathclose%
\pgfusepath{stroke,fill}%
\end{pgfscope}%
\begin{pgfscope}%
\pgfpathrectangle{\pgfqpoint{0.100000in}{0.212622in}}{\pgfqpoint{3.696000in}{3.696000in}}%
\pgfusepath{clip}%
\pgfsetbuttcap%
\pgfsetroundjoin%
\definecolor{currentfill}{rgb}{0.121569,0.466667,0.705882}%
\pgfsetfillcolor{currentfill}%
\pgfsetfillopacity{0.301449}%
\pgfsetlinewidth{1.003750pt}%
\definecolor{currentstroke}{rgb}{0.121569,0.466667,0.705882}%
\pgfsetstrokecolor{currentstroke}%
\pgfsetstrokeopacity{0.301449}%
\pgfsetdash{}{0pt}%
\pgfpathmoveto{\pgfqpoint{1.644123in}{2.109667in}}%
\pgfpathcurveto{\pgfqpoint{1.652359in}{2.109667in}}{\pgfqpoint{1.660259in}{2.112939in}}{\pgfqpoint{1.666083in}{2.118763in}}%
\pgfpathcurveto{\pgfqpoint{1.671907in}{2.124587in}}{\pgfqpoint{1.675180in}{2.132487in}}{\pgfqpoint{1.675180in}{2.140724in}}%
\pgfpathcurveto{\pgfqpoint{1.675180in}{2.148960in}}{\pgfqpoint{1.671907in}{2.156860in}}{\pgfqpoint{1.666083in}{2.162684in}}%
\pgfpathcurveto{\pgfqpoint{1.660259in}{2.168508in}}{\pgfqpoint{1.652359in}{2.171780in}}{\pgfqpoint{1.644123in}{2.171780in}}%
\pgfpathcurveto{\pgfqpoint{1.635887in}{2.171780in}}{\pgfqpoint{1.627987in}{2.168508in}}{\pgfqpoint{1.622163in}{2.162684in}}%
\pgfpathcurveto{\pgfqpoint{1.616339in}{2.156860in}}{\pgfqpoint{1.613067in}{2.148960in}}{\pgfqpoint{1.613067in}{2.140724in}}%
\pgfpathcurveto{\pgfqpoint{1.613067in}{2.132487in}}{\pgfqpoint{1.616339in}{2.124587in}}{\pgfqpoint{1.622163in}{2.118763in}}%
\pgfpathcurveto{\pgfqpoint{1.627987in}{2.112939in}}{\pgfqpoint{1.635887in}{2.109667in}}{\pgfqpoint{1.644123in}{2.109667in}}%
\pgfpathclose%
\pgfusepath{stroke,fill}%
\end{pgfscope}%
\begin{pgfscope}%
\pgfpathrectangle{\pgfqpoint{0.100000in}{0.212622in}}{\pgfqpoint{3.696000in}{3.696000in}}%
\pgfusepath{clip}%
\pgfsetbuttcap%
\pgfsetroundjoin%
\definecolor{currentfill}{rgb}{0.121569,0.466667,0.705882}%
\pgfsetfillcolor{currentfill}%
\pgfsetfillopacity{0.301449}%
\pgfsetlinewidth{1.003750pt}%
\definecolor{currentstroke}{rgb}{0.121569,0.466667,0.705882}%
\pgfsetstrokecolor{currentstroke}%
\pgfsetstrokeopacity{0.301449}%
\pgfsetdash{}{0pt}%
\pgfpathmoveto{\pgfqpoint{1.644123in}{2.109667in}}%
\pgfpathcurveto{\pgfqpoint{1.652359in}{2.109667in}}{\pgfqpoint{1.660259in}{2.112939in}}{\pgfqpoint{1.666083in}{2.118763in}}%
\pgfpathcurveto{\pgfqpoint{1.671907in}{2.124587in}}{\pgfqpoint{1.675180in}{2.132487in}}{\pgfqpoint{1.675180in}{2.140724in}}%
\pgfpathcurveto{\pgfqpoint{1.675180in}{2.148960in}}{\pgfqpoint{1.671907in}{2.156860in}}{\pgfqpoint{1.666083in}{2.162684in}}%
\pgfpathcurveto{\pgfqpoint{1.660259in}{2.168508in}}{\pgfqpoint{1.652359in}{2.171780in}}{\pgfqpoint{1.644123in}{2.171780in}}%
\pgfpathcurveto{\pgfqpoint{1.635887in}{2.171780in}}{\pgfqpoint{1.627987in}{2.168508in}}{\pgfqpoint{1.622163in}{2.162684in}}%
\pgfpathcurveto{\pgfqpoint{1.616339in}{2.156860in}}{\pgfqpoint{1.613067in}{2.148960in}}{\pgfqpoint{1.613067in}{2.140724in}}%
\pgfpathcurveto{\pgfqpoint{1.613067in}{2.132487in}}{\pgfqpoint{1.616339in}{2.124587in}}{\pgfqpoint{1.622163in}{2.118763in}}%
\pgfpathcurveto{\pgfqpoint{1.627987in}{2.112939in}}{\pgfqpoint{1.635887in}{2.109667in}}{\pgfqpoint{1.644123in}{2.109667in}}%
\pgfpathclose%
\pgfusepath{stroke,fill}%
\end{pgfscope}%
\begin{pgfscope}%
\pgfpathrectangle{\pgfqpoint{0.100000in}{0.212622in}}{\pgfqpoint{3.696000in}{3.696000in}}%
\pgfusepath{clip}%
\pgfsetbuttcap%
\pgfsetroundjoin%
\definecolor{currentfill}{rgb}{0.121569,0.466667,0.705882}%
\pgfsetfillcolor{currentfill}%
\pgfsetfillopacity{0.301449}%
\pgfsetlinewidth{1.003750pt}%
\definecolor{currentstroke}{rgb}{0.121569,0.466667,0.705882}%
\pgfsetstrokecolor{currentstroke}%
\pgfsetstrokeopacity{0.301449}%
\pgfsetdash{}{0pt}%
\pgfpathmoveto{\pgfqpoint{1.644123in}{2.109667in}}%
\pgfpathcurveto{\pgfqpoint{1.652359in}{2.109667in}}{\pgfqpoint{1.660259in}{2.112939in}}{\pgfqpoint{1.666083in}{2.118763in}}%
\pgfpathcurveto{\pgfqpoint{1.671907in}{2.124587in}}{\pgfqpoint{1.675180in}{2.132487in}}{\pgfqpoint{1.675180in}{2.140724in}}%
\pgfpathcurveto{\pgfqpoint{1.675180in}{2.148960in}}{\pgfqpoint{1.671907in}{2.156860in}}{\pgfqpoint{1.666083in}{2.162684in}}%
\pgfpathcurveto{\pgfqpoint{1.660259in}{2.168508in}}{\pgfqpoint{1.652359in}{2.171780in}}{\pgfqpoint{1.644123in}{2.171780in}}%
\pgfpathcurveto{\pgfqpoint{1.635887in}{2.171780in}}{\pgfqpoint{1.627987in}{2.168508in}}{\pgfqpoint{1.622163in}{2.162684in}}%
\pgfpathcurveto{\pgfqpoint{1.616339in}{2.156860in}}{\pgfqpoint{1.613067in}{2.148960in}}{\pgfqpoint{1.613067in}{2.140724in}}%
\pgfpathcurveto{\pgfqpoint{1.613067in}{2.132487in}}{\pgfqpoint{1.616339in}{2.124587in}}{\pgfqpoint{1.622163in}{2.118763in}}%
\pgfpathcurveto{\pgfqpoint{1.627987in}{2.112939in}}{\pgfqpoint{1.635887in}{2.109667in}}{\pgfqpoint{1.644123in}{2.109667in}}%
\pgfpathclose%
\pgfusepath{stroke,fill}%
\end{pgfscope}%
\begin{pgfscope}%
\pgfpathrectangle{\pgfqpoint{0.100000in}{0.212622in}}{\pgfqpoint{3.696000in}{3.696000in}}%
\pgfusepath{clip}%
\pgfsetbuttcap%
\pgfsetroundjoin%
\definecolor{currentfill}{rgb}{0.121569,0.466667,0.705882}%
\pgfsetfillcolor{currentfill}%
\pgfsetfillopacity{0.301449}%
\pgfsetlinewidth{1.003750pt}%
\definecolor{currentstroke}{rgb}{0.121569,0.466667,0.705882}%
\pgfsetstrokecolor{currentstroke}%
\pgfsetstrokeopacity{0.301449}%
\pgfsetdash{}{0pt}%
\pgfpathmoveto{\pgfqpoint{1.644123in}{2.109667in}}%
\pgfpathcurveto{\pgfqpoint{1.652359in}{2.109667in}}{\pgfqpoint{1.660259in}{2.112939in}}{\pgfqpoint{1.666083in}{2.118763in}}%
\pgfpathcurveto{\pgfqpoint{1.671907in}{2.124587in}}{\pgfqpoint{1.675180in}{2.132487in}}{\pgfqpoint{1.675180in}{2.140724in}}%
\pgfpathcurveto{\pgfqpoint{1.675180in}{2.148960in}}{\pgfqpoint{1.671907in}{2.156860in}}{\pgfqpoint{1.666083in}{2.162684in}}%
\pgfpathcurveto{\pgfqpoint{1.660259in}{2.168508in}}{\pgfqpoint{1.652359in}{2.171780in}}{\pgfqpoint{1.644123in}{2.171780in}}%
\pgfpathcurveto{\pgfqpoint{1.635887in}{2.171780in}}{\pgfqpoint{1.627987in}{2.168508in}}{\pgfqpoint{1.622163in}{2.162684in}}%
\pgfpathcurveto{\pgfqpoint{1.616339in}{2.156860in}}{\pgfqpoint{1.613067in}{2.148960in}}{\pgfqpoint{1.613067in}{2.140724in}}%
\pgfpathcurveto{\pgfqpoint{1.613067in}{2.132487in}}{\pgfqpoint{1.616339in}{2.124587in}}{\pgfqpoint{1.622163in}{2.118763in}}%
\pgfpathcurveto{\pgfqpoint{1.627987in}{2.112939in}}{\pgfqpoint{1.635887in}{2.109667in}}{\pgfqpoint{1.644123in}{2.109667in}}%
\pgfpathclose%
\pgfusepath{stroke,fill}%
\end{pgfscope}%
\begin{pgfscope}%
\pgfpathrectangle{\pgfqpoint{0.100000in}{0.212622in}}{\pgfqpoint{3.696000in}{3.696000in}}%
\pgfusepath{clip}%
\pgfsetbuttcap%
\pgfsetroundjoin%
\definecolor{currentfill}{rgb}{0.121569,0.466667,0.705882}%
\pgfsetfillcolor{currentfill}%
\pgfsetfillopacity{0.301449}%
\pgfsetlinewidth{1.003750pt}%
\definecolor{currentstroke}{rgb}{0.121569,0.466667,0.705882}%
\pgfsetstrokecolor{currentstroke}%
\pgfsetstrokeopacity{0.301449}%
\pgfsetdash{}{0pt}%
\pgfpathmoveto{\pgfqpoint{1.644123in}{2.109667in}}%
\pgfpathcurveto{\pgfqpoint{1.652359in}{2.109667in}}{\pgfqpoint{1.660259in}{2.112939in}}{\pgfqpoint{1.666083in}{2.118763in}}%
\pgfpathcurveto{\pgfqpoint{1.671907in}{2.124587in}}{\pgfqpoint{1.675180in}{2.132487in}}{\pgfqpoint{1.675180in}{2.140724in}}%
\pgfpathcurveto{\pgfqpoint{1.675180in}{2.148960in}}{\pgfqpoint{1.671907in}{2.156860in}}{\pgfqpoint{1.666083in}{2.162684in}}%
\pgfpathcurveto{\pgfqpoint{1.660259in}{2.168508in}}{\pgfqpoint{1.652359in}{2.171780in}}{\pgfqpoint{1.644123in}{2.171780in}}%
\pgfpathcurveto{\pgfqpoint{1.635887in}{2.171780in}}{\pgfqpoint{1.627987in}{2.168508in}}{\pgfqpoint{1.622163in}{2.162684in}}%
\pgfpathcurveto{\pgfqpoint{1.616339in}{2.156860in}}{\pgfqpoint{1.613067in}{2.148960in}}{\pgfqpoint{1.613067in}{2.140724in}}%
\pgfpathcurveto{\pgfqpoint{1.613067in}{2.132487in}}{\pgfqpoint{1.616339in}{2.124587in}}{\pgfqpoint{1.622163in}{2.118763in}}%
\pgfpathcurveto{\pgfqpoint{1.627987in}{2.112939in}}{\pgfqpoint{1.635887in}{2.109667in}}{\pgfqpoint{1.644123in}{2.109667in}}%
\pgfpathclose%
\pgfusepath{stroke,fill}%
\end{pgfscope}%
\begin{pgfscope}%
\pgfpathrectangle{\pgfqpoint{0.100000in}{0.212622in}}{\pgfqpoint{3.696000in}{3.696000in}}%
\pgfusepath{clip}%
\pgfsetbuttcap%
\pgfsetroundjoin%
\definecolor{currentfill}{rgb}{0.121569,0.466667,0.705882}%
\pgfsetfillcolor{currentfill}%
\pgfsetfillopacity{0.301449}%
\pgfsetlinewidth{1.003750pt}%
\definecolor{currentstroke}{rgb}{0.121569,0.466667,0.705882}%
\pgfsetstrokecolor{currentstroke}%
\pgfsetstrokeopacity{0.301449}%
\pgfsetdash{}{0pt}%
\pgfpathmoveto{\pgfqpoint{1.644123in}{2.109667in}}%
\pgfpathcurveto{\pgfqpoint{1.652359in}{2.109667in}}{\pgfqpoint{1.660259in}{2.112939in}}{\pgfqpoint{1.666083in}{2.118763in}}%
\pgfpathcurveto{\pgfqpoint{1.671907in}{2.124587in}}{\pgfqpoint{1.675180in}{2.132487in}}{\pgfqpoint{1.675180in}{2.140724in}}%
\pgfpathcurveto{\pgfqpoint{1.675180in}{2.148960in}}{\pgfqpoint{1.671907in}{2.156860in}}{\pgfqpoint{1.666083in}{2.162684in}}%
\pgfpathcurveto{\pgfqpoint{1.660259in}{2.168508in}}{\pgfqpoint{1.652359in}{2.171780in}}{\pgfqpoint{1.644123in}{2.171780in}}%
\pgfpathcurveto{\pgfqpoint{1.635887in}{2.171780in}}{\pgfqpoint{1.627987in}{2.168508in}}{\pgfqpoint{1.622163in}{2.162684in}}%
\pgfpathcurveto{\pgfqpoint{1.616339in}{2.156860in}}{\pgfqpoint{1.613067in}{2.148960in}}{\pgfqpoint{1.613067in}{2.140724in}}%
\pgfpathcurveto{\pgfqpoint{1.613067in}{2.132487in}}{\pgfqpoint{1.616339in}{2.124587in}}{\pgfqpoint{1.622163in}{2.118763in}}%
\pgfpathcurveto{\pgfqpoint{1.627987in}{2.112939in}}{\pgfqpoint{1.635887in}{2.109667in}}{\pgfqpoint{1.644123in}{2.109667in}}%
\pgfpathclose%
\pgfusepath{stroke,fill}%
\end{pgfscope}%
\begin{pgfscope}%
\pgfpathrectangle{\pgfqpoint{0.100000in}{0.212622in}}{\pgfqpoint{3.696000in}{3.696000in}}%
\pgfusepath{clip}%
\pgfsetbuttcap%
\pgfsetroundjoin%
\definecolor{currentfill}{rgb}{0.121569,0.466667,0.705882}%
\pgfsetfillcolor{currentfill}%
\pgfsetfillopacity{0.301449}%
\pgfsetlinewidth{1.003750pt}%
\definecolor{currentstroke}{rgb}{0.121569,0.466667,0.705882}%
\pgfsetstrokecolor{currentstroke}%
\pgfsetstrokeopacity{0.301449}%
\pgfsetdash{}{0pt}%
\pgfpathmoveto{\pgfqpoint{1.644123in}{2.109667in}}%
\pgfpathcurveto{\pgfqpoint{1.652359in}{2.109667in}}{\pgfqpoint{1.660259in}{2.112939in}}{\pgfqpoint{1.666083in}{2.118763in}}%
\pgfpathcurveto{\pgfqpoint{1.671907in}{2.124587in}}{\pgfqpoint{1.675180in}{2.132487in}}{\pgfqpoint{1.675180in}{2.140724in}}%
\pgfpathcurveto{\pgfqpoint{1.675180in}{2.148960in}}{\pgfqpoint{1.671907in}{2.156860in}}{\pgfqpoint{1.666083in}{2.162684in}}%
\pgfpathcurveto{\pgfqpoint{1.660259in}{2.168508in}}{\pgfqpoint{1.652359in}{2.171780in}}{\pgfqpoint{1.644123in}{2.171780in}}%
\pgfpathcurveto{\pgfqpoint{1.635887in}{2.171780in}}{\pgfqpoint{1.627987in}{2.168508in}}{\pgfqpoint{1.622163in}{2.162684in}}%
\pgfpathcurveto{\pgfqpoint{1.616339in}{2.156860in}}{\pgfqpoint{1.613067in}{2.148960in}}{\pgfqpoint{1.613067in}{2.140724in}}%
\pgfpathcurveto{\pgfqpoint{1.613067in}{2.132487in}}{\pgfqpoint{1.616339in}{2.124587in}}{\pgfqpoint{1.622163in}{2.118763in}}%
\pgfpathcurveto{\pgfqpoint{1.627987in}{2.112939in}}{\pgfqpoint{1.635887in}{2.109667in}}{\pgfqpoint{1.644123in}{2.109667in}}%
\pgfpathclose%
\pgfusepath{stroke,fill}%
\end{pgfscope}%
\begin{pgfscope}%
\pgfpathrectangle{\pgfqpoint{0.100000in}{0.212622in}}{\pgfqpoint{3.696000in}{3.696000in}}%
\pgfusepath{clip}%
\pgfsetbuttcap%
\pgfsetroundjoin%
\definecolor{currentfill}{rgb}{0.121569,0.466667,0.705882}%
\pgfsetfillcolor{currentfill}%
\pgfsetfillopacity{0.301449}%
\pgfsetlinewidth{1.003750pt}%
\definecolor{currentstroke}{rgb}{0.121569,0.466667,0.705882}%
\pgfsetstrokecolor{currentstroke}%
\pgfsetstrokeopacity{0.301449}%
\pgfsetdash{}{0pt}%
\pgfpathmoveto{\pgfqpoint{1.644123in}{2.109667in}}%
\pgfpathcurveto{\pgfqpoint{1.652359in}{2.109667in}}{\pgfqpoint{1.660259in}{2.112939in}}{\pgfqpoint{1.666083in}{2.118763in}}%
\pgfpathcurveto{\pgfqpoint{1.671907in}{2.124587in}}{\pgfqpoint{1.675180in}{2.132487in}}{\pgfqpoint{1.675180in}{2.140724in}}%
\pgfpathcurveto{\pgfqpoint{1.675180in}{2.148960in}}{\pgfqpoint{1.671907in}{2.156860in}}{\pgfqpoint{1.666083in}{2.162684in}}%
\pgfpathcurveto{\pgfqpoint{1.660259in}{2.168508in}}{\pgfqpoint{1.652359in}{2.171780in}}{\pgfqpoint{1.644123in}{2.171780in}}%
\pgfpathcurveto{\pgfqpoint{1.635887in}{2.171780in}}{\pgfqpoint{1.627987in}{2.168508in}}{\pgfqpoint{1.622163in}{2.162684in}}%
\pgfpathcurveto{\pgfqpoint{1.616339in}{2.156860in}}{\pgfqpoint{1.613067in}{2.148960in}}{\pgfqpoint{1.613067in}{2.140724in}}%
\pgfpathcurveto{\pgfqpoint{1.613067in}{2.132487in}}{\pgfqpoint{1.616339in}{2.124587in}}{\pgfqpoint{1.622163in}{2.118763in}}%
\pgfpathcurveto{\pgfqpoint{1.627987in}{2.112939in}}{\pgfqpoint{1.635887in}{2.109667in}}{\pgfqpoint{1.644123in}{2.109667in}}%
\pgfpathclose%
\pgfusepath{stroke,fill}%
\end{pgfscope}%
\begin{pgfscope}%
\pgfpathrectangle{\pgfqpoint{0.100000in}{0.212622in}}{\pgfqpoint{3.696000in}{3.696000in}}%
\pgfusepath{clip}%
\pgfsetbuttcap%
\pgfsetroundjoin%
\definecolor{currentfill}{rgb}{0.121569,0.466667,0.705882}%
\pgfsetfillcolor{currentfill}%
\pgfsetfillopacity{0.301449}%
\pgfsetlinewidth{1.003750pt}%
\definecolor{currentstroke}{rgb}{0.121569,0.466667,0.705882}%
\pgfsetstrokecolor{currentstroke}%
\pgfsetstrokeopacity{0.301449}%
\pgfsetdash{}{0pt}%
\pgfpathmoveto{\pgfqpoint{1.644123in}{2.109667in}}%
\pgfpathcurveto{\pgfqpoint{1.652359in}{2.109667in}}{\pgfqpoint{1.660259in}{2.112939in}}{\pgfqpoint{1.666083in}{2.118763in}}%
\pgfpathcurveto{\pgfqpoint{1.671907in}{2.124587in}}{\pgfqpoint{1.675180in}{2.132487in}}{\pgfqpoint{1.675180in}{2.140724in}}%
\pgfpathcurveto{\pgfqpoint{1.675180in}{2.148960in}}{\pgfqpoint{1.671907in}{2.156860in}}{\pgfqpoint{1.666083in}{2.162684in}}%
\pgfpathcurveto{\pgfqpoint{1.660259in}{2.168508in}}{\pgfqpoint{1.652359in}{2.171780in}}{\pgfqpoint{1.644123in}{2.171780in}}%
\pgfpathcurveto{\pgfqpoint{1.635887in}{2.171780in}}{\pgfqpoint{1.627987in}{2.168508in}}{\pgfqpoint{1.622163in}{2.162684in}}%
\pgfpathcurveto{\pgfqpoint{1.616339in}{2.156860in}}{\pgfqpoint{1.613067in}{2.148960in}}{\pgfqpoint{1.613067in}{2.140724in}}%
\pgfpathcurveto{\pgfqpoint{1.613067in}{2.132487in}}{\pgfqpoint{1.616339in}{2.124587in}}{\pgfqpoint{1.622163in}{2.118763in}}%
\pgfpathcurveto{\pgfqpoint{1.627987in}{2.112939in}}{\pgfqpoint{1.635887in}{2.109667in}}{\pgfqpoint{1.644123in}{2.109667in}}%
\pgfpathclose%
\pgfusepath{stroke,fill}%
\end{pgfscope}%
\begin{pgfscope}%
\pgfpathrectangle{\pgfqpoint{0.100000in}{0.212622in}}{\pgfqpoint{3.696000in}{3.696000in}}%
\pgfusepath{clip}%
\pgfsetbuttcap%
\pgfsetroundjoin%
\definecolor{currentfill}{rgb}{0.121569,0.466667,0.705882}%
\pgfsetfillcolor{currentfill}%
\pgfsetfillopacity{0.301449}%
\pgfsetlinewidth{1.003750pt}%
\definecolor{currentstroke}{rgb}{0.121569,0.466667,0.705882}%
\pgfsetstrokecolor{currentstroke}%
\pgfsetstrokeopacity{0.301449}%
\pgfsetdash{}{0pt}%
\pgfpathmoveto{\pgfqpoint{1.644123in}{2.109667in}}%
\pgfpathcurveto{\pgfqpoint{1.652359in}{2.109667in}}{\pgfqpoint{1.660259in}{2.112939in}}{\pgfqpoint{1.666083in}{2.118763in}}%
\pgfpathcurveto{\pgfqpoint{1.671907in}{2.124587in}}{\pgfqpoint{1.675180in}{2.132487in}}{\pgfqpoint{1.675180in}{2.140724in}}%
\pgfpathcurveto{\pgfqpoint{1.675180in}{2.148960in}}{\pgfqpoint{1.671907in}{2.156860in}}{\pgfqpoint{1.666083in}{2.162684in}}%
\pgfpathcurveto{\pgfqpoint{1.660259in}{2.168508in}}{\pgfqpoint{1.652359in}{2.171780in}}{\pgfqpoint{1.644123in}{2.171780in}}%
\pgfpathcurveto{\pgfqpoint{1.635887in}{2.171780in}}{\pgfqpoint{1.627987in}{2.168508in}}{\pgfqpoint{1.622163in}{2.162684in}}%
\pgfpathcurveto{\pgfqpoint{1.616339in}{2.156860in}}{\pgfqpoint{1.613067in}{2.148960in}}{\pgfqpoint{1.613067in}{2.140724in}}%
\pgfpathcurveto{\pgfqpoint{1.613067in}{2.132487in}}{\pgfqpoint{1.616339in}{2.124587in}}{\pgfqpoint{1.622163in}{2.118763in}}%
\pgfpathcurveto{\pgfqpoint{1.627987in}{2.112939in}}{\pgfqpoint{1.635887in}{2.109667in}}{\pgfqpoint{1.644123in}{2.109667in}}%
\pgfpathclose%
\pgfusepath{stroke,fill}%
\end{pgfscope}%
\begin{pgfscope}%
\pgfpathrectangle{\pgfqpoint{0.100000in}{0.212622in}}{\pgfqpoint{3.696000in}{3.696000in}}%
\pgfusepath{clip}%
\pgfsetbuttcap%
\pgfsetroundjoin%
\definecolor{currentfill}{rgb}{0.121569,0.466667,0.705882}%
\pgfsetfillcolor{currentfill}%
\pgfsetfillopacity{0.301449}%
\pgfsetlinewidth{1.003750pt}%
\definecolor{currentstroke}{rgb}{0.121569,0.466667,0.705882}%
\pgfsetstrokecolor{currentstroke}%
\pgfsetstrokeopacity{0.301449}%
\pgfsetdash{}{0pt}%
\pgfpathmoveto{\pgfqpoint{1.644123in}{2.109667in}}%
\pgfpathcurveto{\pgfqpoint{1.652359in}{2.109667in}}{\pgfqpoint{1.660259in}{2.112939in}}{\pgfqpoint{1.666083in}{2.118763in}}%
\pgfpathcurveto{\pgfqpoint{1.671907in}{2.124587in}}{\pgfqpoint{1.675180in}{2.132487in}}{\pgfqpoint{1.675180in}{2.140724in}}%
\pgfpathcurveto{\pgfqpoint{1.675180in}{2.148960in}}{\pgfqpoint{1.671907in}{2.156860in}}{\pgfqpoint{1.666083in}{2.162684in}}%
\pgfpathcurveto{\pgfqpoint{1.660259in}{2.168508in}}{\pgfqpoint{1.652359in}{2.171780in}}{\pgfqpoint{1.644123in}{2.171780in}}%
\pgfpathcurveto{\pgfqpoint{1.635887in}{2.171780in}}{\pgfqpoint{1.627987in}{2.168508in}}{\pgfqpoint{1.622163in}{2.162684in}}%
\pgfpathcurveto{\pgfqpoint{1.616339in}{2.156860in}}{\pgfqpoint{1.613067in}{2.148960in}}{\pgfqpoint{1.613067in}{2.140724in}}%
\pgfpathcurveto{\pgfqpoint{1.613067in}{2.132487in}}{\pgfqpoint{1.616339in}{2.124587in}}{\pgfqpoint{1.622163in}{2.118763in}}%
\pgfpathcurveto{\pgfqpoint{1.627987in}{2.112939in}}{\pgfqpoint{1.635887in}{2.109667in}}{\pgfqpoint{1.644123in}{2.109667in}}%
\pgfpathclose%
\pgfusepath{stroke,fill}%
\end{pgfscope}%
\begin{pgfscope}%
\pgfpathrectangle{\pgfqpoint{0.100000in}{0.212622in}}{\pgfqpoint{3.696000in}{3.696000in}}%
\pgfusepath{clip}%
\pgfsetbuttcap%
\pgfsetroundjoin%
\definecolor{currentfill}{rgb}{0.121569,0.466667,0.705882}%
\pgfsetfillcolor{currentfill}%
\pgfsetfillopacity{0.301449}%
\pgfsetlinewidth{1.003750pt}%
\definecolor{currentstroke}{rgb}{0.121569,0.466667,0.705882}%
\pgfsetstrokecolor{currentstroke}%
\pgfsetstrokeopacity{0.301449}%
\pgfsetdash{}{0pt}%
\pgfpathmoveto{\pgfqpoint{1.644123in}{2.109667in}}%
\pgfpathcurveto{\pgfqpoint{1.652359in}{2.109667in}}{\pgfqpoint{1.660259in}{2.112939in}}{\pgfqpoint{1.666083in}{2.118763in}}%
\pgfpathcurveto{\pgfqpoint{1.671907in}{2.124587in}}{\pgfqpoint{1.675180in}{2.132487in}}{\pgfqpoint{1.675180in}{2.140724in}}%
\pgfpathcurveto{\pgfqpoint{1.675180in}{2.148960in}}{\pgfqpoint{1.671907in}{2.156860in}}{\pgfqpoint{1.666083in}{2.162684in}}%
\pgfpathcurveto{\pgfqpoint{1.660259in}{2.168508in}}{\pgfqpoint{1.652359in}{2.171780in}}{\pgfqpoint{1.644123in}{2.171780in}}%
\pgfpathcurveto{\pgfqpoint{1.635887in}{2.171780in}}{\pgfqpoint{1.627987in}{2.168508in}}{\pgfqpoint{1.622163in}{2.162684in}}%
\pgfpathcurveto{\pgfqpoint{1.616339in}{2.156860in}}{\pgfqpoint{1.613067in}{2.148960in}}{\pgfqpoint{1.613067in}{2.140724in}}%
\pgfpathcurveto{\pgfqpoint{1.613067in}{2.132487in}}{\pgfqpoint{1.616339in}{2.124587in}}{\pgfqpoint{1.622163in}{2.118763in}}%
\pgfpathcurveto{\pgfqpoint{1.627987in}{2.112939in}}{\pgfqpoint{1.635887in}{2.109667in}}{\pgfqpoint{1.644123in}{2.109667in}}%
\pgfpathclose%
\pgfusepath{stroke,fill}%
\end{pgfscope}%
\begin{pgfscope}%
\pgfpathrectangle{\pgfqpoint{0.100000in}{0.212622in}}{\pgfqpoint{3.696000in}{3.696000in}}%
\pgfusepath{clip}%
\pgfsetbuttcap%
\pgfsetroundjoin%
\definecolor{currentfill}{rgb}{0.121569,0.466667,0.705882}%
\pgfsetfillcolor{currentfill}%
\pgfsetfillopacity{0.301449}%
\pgfsetlinewidth{1.003750pt}%
\definecolor{currentstroke}{rgb}{0.121569,0.466667,0.705882}%
\pgfsetstrokecolor{currentstroke}%
\pgfsetstrokeopacity{0.301449}%
\pgfsetdash{}{0pt}%
\pgfpathmoveto{\pgfqpoint{1.644123in}{2.109667in}}%
\pgfpathcurveto{\pgfqpoint{1.652359in}{2.109667in}}{\pgfqpoint{1.660259in}{2.112939in}}{\pgfqpoint{1.666083in}{2.118763in}}%
\pgfpathcurveto{\pgfqpoint{1.671907in}{2.124587in}}{\pgfqpoint{1.675180in}{2.132487in}}{\pgfqpoint{1.675180in}{2.140724in}}%
\pgfpathcurveto{\pgfqpoint{1.675180in}{2.148960in}}{\pgfqpoint{1.671907in}{2.156860in}}{\pgfqpoint{1.666083in}{2.162684in}}%
\pgfpathcurveto{\pgfqpoint{1.660259in}{2.168508in}}{\pgfqpoint{1.652359in}{2.171780in}}{\pgfqpoint{1.644123in}{2.171780in}}%
\pgfpathcurveto{\pgfqpoint{1.635887in}{2.171780in}}{\pgfqpoint{1.627987in}{2.168508in}}{\pgfqpoint{1.622163in}{2.162684in}}%
\pgfpathcurveto{\pgfqpoint{1.616339in}{2.156860in}}{\pgfqpoint{1.613067in}{2.148960in}}{\pgfqpoint{1.613067in}{2.140724in}}%
\pgfpathcurveto{\pgfqpoint{1.613067in}{2.132487in}}{\pgfqpoint{1.616339in}{2.124587in}}{\pgfqpoint{1.622163in}{2.118763in}}%
\pgfpathcurveto{\pgfqpoint{1.627987in}{2.112939in}}{\pgfqpoint{1.635887in}{2.109667in}}{\pgfqpoint{1.644123in}{2.109667in}}%
\pgfpathclose%
\pgfusepath{stroke,fill}%
\end{pgfscope}%
\begin{pgfscope}%
\pgfpathrectangle{\pgfqpoint{0.100000in}{0.212622in}}{\pgfqpoint{3.696000in}{3.696000in}}%
\pgfusepath{clip}%
\pgfsetbuttcap%
\pgfsetroundjoin%
\definecolor{currentfill}{rgb}{0.121569,0.466667,0.705882}%
\pgfsetfillcolor{currentfill}%
\pgfsetfillopacity{0.301449}%
\pgfsetlinewidth{1.003750pt}%
\definecolor{currentstroke}{rgb}{0.121569,0.466667,0.705882}%
\pgfsetstrokecolor{currentstroke}%
\pgfsetstrokeopacity{0.301449}%
\pgfsetdash{}{0pt}%
\pgfpathmoveto{\pgfqpoint{1.644123in}{2.109667in}}%
\pgfpathcurveto{\pgfqpoint{1.652359in}{2.109667in}}{\pgfqpoint{1.660259in}{2.112939in}}{\pgfqpoint{1.666083in}{2.118763in}}%
\pgfpathcurveto{\pgfqpoint{1.671907in}{2.124587in}}{\pgfqpoint{1.675180in}{2.132487in}}{\pgfqpoint{1.675180in}{2.140724in}}%
\pgfpathcurveto{\pgfqpoint{1.675180in}{2.148960in}}{\pgfqpoint{1.671907in}{2.156860in}}{\pgfqpoint{1.666083in}{2.162684in}}%
\pgfpathcurveto{\pgfqpoint{1.660259in}{2.168508in}}{\pgfqpoint{1.652359in}{2.171780in}}{\pgfqpoint{1.644123in}{2.171780in}}%
\pgfpathcurveto{\pgfqpoint{1.635887in}{2.171780in}}{\pgfqpoint{1.627987in}{2.168508in}}{\pgfqpoint{1.622163in}{2.162684in}}%
\pgfpathcurveto{\pgfqpoint{1.616339in}{2.156860in}}{\pgfqpoint{1.613067in}{2.148960in}}{\pgfqpoint{1.613067in}{2.140724in}}%
\pgfpathcurveto{\pgfqpoint{1.613067in}{2.132487in}}{\pgfqpoint{1.616339in}{2.124587in}}{\pgfqpoint{1.622163in}{2.118763in}}%
\pgfpathcurveto{\pgfqpoint{1.627987in}{2.112939in}}{\pgfqpoint{1.635887in}{2.109667in}}{\pgfqpoint{1.644123in}{2.109667in}}%
\pgfpathclose%
\pgfusepath{stroke,fill}%
\end{pgfscope}%
\begin{pgfscope}%
\pgfpathrectangle{\pgfqpoint{0.100000in}{0.212622in}}{\pgfqpoint{3.696000in}{3.696000in}}%
\pgfusepath{clip}%
\pgfsetbuttcap%
\pgfsetroundjoin%
\definecolor{currentfill}{rgb}{0.121569,0.466667,0.705882}%
\pgfsetfillcolor{currentfill}%
\pgfsetfillopacity{0.301449}%
\pgfsetlinewidth{1.003750pt}%
\definecolor{currentstroke}{rgb}{0.121569,0.466667,0.705882}%
\pgfsetstrokecolor{currentstroke}%
\pgfsetstrokeopacity{0.301449}%
\pgfsetdash{}{0pt}%
\pgfpathmoveto{\pgfqpoint{1.644123in}{2.109667in}}%
\pgfpathcurveto{\pgfqpoint{1.652359in}{2.109667in}}{\pgfqpoint{1.660259in}{2.112939in}}{\pgfqpoint{1.666083in}{2.118763in}}%
\pgfpathcurveto{\pgfqpoint{1.671907in}{2.124587in}}{\pgfqpoint{1.675179in}{2.132487in}}{\pgfqpoint{1.675179in}{2.140724in}}%
\pgfpathcurveto{\pgfqpoint{1.675179in}{2.148960in}}{\pgfqpoint{1.671907in}{2.156860in}}{\pgfqpoint{1.666083in}{2.162684in}}%
\pgfpathcurveto{\pgfqpoint{1.660259in}{2.168508in}}{\pgfqpoint{1.652359in}{2.171780in}}{\pgfqpoint{1.644123in}{2.171780in}}%
\pgfpathcurveto{\pgfqpoint{1.635887in}{2.171780in}}{\pgfqpoint{1.627987in}{2.168508in}}{\pgfqpoint{1.622163in}{2.162684in}}%
\pgfpathcurveto{\pgfqpoint{1.616339in}{2.156860in}}{\pgfqpoint{1.613066in}{2.148960in}}{\pgfqpoint{1.613066in}{2.140724in}}%
\pgfpathcurveto{\pgfqpoint{1.613066in}{2.132487in}}{\pgfqpoint{1.616339in}{2.124587in}}{\pgfqpoint{1.622163in}{2.118763in}}%
\pgfpathcurveto{\pgfqpoint{1.627987in}{2.112939in}}{\pgfqpoint{1.635887in}{2.109667in}}{\pgfqpoint{1.644123in}{2.109667in}}%
\pgfpathclose%
\pgfusepath{stroke,fill}%
\end{pgfscope}%
\begin{pgfscope}%
\pgfpathrectangle{\pgfqpoint{0.100000in}{0.212622in}}{\pgfqpoint{3.696000in}{3.696000in}}%
\pgfusepath{clip}%
\pgfsetbuttcap%
\pgfsetroundjoin%
\definecolor{currentfill}{rgb}{0.121569,0.466667,0.705882}%
\pgfsetfillcolor{currentfill}%
\pgfsetfillopacity{0.301449}%
\pgfsetlinewidth{1.003750pt}%
\definecolor{currentstroke}{rgb}{0.121569,0.466667,0.705882}%
\pgfsetstrokecolor{currentstroke}%
\pgfsetstrokeopacity{0.301449}%
\pgfsetdash{}{0pt}%
\pgfpathmoveto{\pgfqpoint{1.644123in}{2.109667in}}%
\pgfpathcurveto{\pgfqpoint{1.652359in}{2.109667in}}{\pgfqpoint{1.660259in}{2.112939in}}{\pgfqpoint{1.666083in}{2.118763in}}%
\pgfpathcurveto{\pgfqpoint{1.671907in}{2.124587in}}{\pgfqpoint{1.675180in}{2.132487in}}{\pgfqpoint{1.675180in}{2.140724in}}%
\pgfpathcurveto{\pgfqpoint{1.675180in}{2.148960in}}{\pgfqpoint{1.671907in}{2.156860in}}{\pgfqpoint{1.666083in}{2.162684in}}%
\pgfpathcurveto{\pgfqpoint{1.660259in}{2.168508in}}{\pgfqpoint{1.652359in}{2.171780in}}{\pgfqpoint{1.644123in}{2.171780in}}%
\pgfpathcurveto{\pgfqpoint{1.635887in}{2.171780in}}{\pgfqpoint{1.627987in}{2.168508in}}{\pgfqpoint{1.622163in}{2.162684in}}%
\pgfpathcurveto{\pgfqpoint{1.616339in}{2.156860in}}{\pgfqpoint{1.613067in}{2.148960in}}{\pgfqpoint{1.613067in}{2.140724in}}%
\pgfpathcurveto{\pgfqpoint{1.613067in}{2.132487in}}{\pgfqpoint{1.616339in}{2.124587in}}{\pgfqpoint{1.622163in}{2.118763in}}%
\pgfpathcurveto{\pgfqpoint{1.627987in}{2.112939in}}{\pgfqpoint{1.635887in}{2.109667in}}{\pgfqpoint{1.644123in}{2.109667in}}%
\pgfpathclose%
\pgfusepath{stroke,fill}%
\end{pgfscope}%
\begin{pgfscope}%
\pgfpathrectangle{\pgfqpoint{0.100000in}{0.212622in}}{\pgfqpoint{3.696000in}{3.696000in}}%
\pgfusepath{clip}%
\pgfsetbuttcap%
\pgfsetroundjoin%
\definecolor{currentfill}{rgb}{0.121569,0.466667,0.705882}%
\pgfsetfillcolor{currentfill}%
\pgfsetfillopacity{0.301449}%
\pgfsetlinewidth{1.003750pt}%
\definecolor{currentstroke}{rgb}{0.121569,0.466667,0.705882}%
\pgfsetstrokecolor{currentstroke}%
\pgfsetstrokeopacity{0.301449}%
\pgfsetdash{}{0pt}%
\pgfpathmoveto{\pgfqpoint{1.644123in}{2.109667in}}%
\pgfpathcurveto{\pgfqpoint{1.652359in}{2.109667in}}{\pgfqpoint{1.660259in}{2.112939in}}{\pgfqpoint{1.666083in}{2.118763in}}%
\pgfpathcurveto{\pgfqpoint{1.671907in}{2.124587in}}{\pgfqpoint{1.675180in}{2.132487in}}{\pgfqpoint{1.675180in}{2.140724in}}%
\pgfpathcurveto{\pgfqpoint{1.675180in}{2.148960in}}{\pgfqpoint{1.671907in}{2.156860in}}{\pgfqpoint{1.666083in}{2.162684in}}%
\pgfpathcurveto{\pgfqpoint{1.660259in}{2.168508in}}{\pgfqpoint{1.652359in}{2.171780in}}{\pgfqpoint{1.644123in}{2.171780in}}%
\pgfpathcurveto{\pgfqpoint{1.635887in}{2.171780in}}{\pgfqpoint{1.627987in}{2.168508in}}{\pgfqpoint{1.622163in}{2.162684in}}%
\pgfpathcurveto{\pgfqpoint{1.616339in}{2.156860in}}{\pgfqpoint{1.613067in}{2.148960in}}{\pgfqpoint{1.613067in}{2.140724in}}%
\pgfpathcurveto{\pgfqpoint{1.613067in}{2.132487in}}{\pgfqpoint{1.616339in}{2.124587in}}{\pgfqpoint{1.622163in}{2.118763in}}%
\pgfpathcurveto{\pgfqpoint{1.627987in}{2.112939in}}{\pgfqpoint{1.635887in}{2.109667in}}{\pgfqpoint{1.644123in}{2.109667in}}%
\pgfpathclose%
\pgfusepath{stroke,fill}%
\end{pgfscope}%
\begin{pgfscope}%
\pgfpathrectangle{\pgfqpoint{0.100000in}{0.212622in}}{\pgfqpoint{3.696000in}{3.696000in}}%
\pgfusepath{clip}%
\pgfsetbuttcap%
\pgfsetroundjoin%
\definecolor{currentfill}{rgb}{0.121569,0.466667,0.705882}%
\pgfsetfillcolor{currentfill}%
\pgfsetfillopacity{0.301449}%
\pgfsetlinewidth{1.003750pt}%
\definecolor{currentstroke}{rgb}{0.121569,0.466667,0.705882}%
\pgfsetstrokecolor{currentstroke}%
\pgfsetstrokeopacity{0.301449}%
\pgfsetdash{}{0pt}%
\pgfpathmoveto{\pgfqpoint{1.644123in}{2.109667in}}%
\pgfpathcurveto{\pgfqpoint{1.652359in}{2.109667in}}{\pgfqpoint{1.660259in}{2.112939in}}{\pgfqpoint{1.666083in}{2.118763in}}%
\pgfpathcurveto{\pgfqpoint{1.671907in}{2.124587in}}{\pgfqpoint{1.675180in}{2.132487in}}{\pgfqpoint{1.675180in}{2.140724in}}%
\pgfpathcurveto{\pgfqpoint{1.675180in}{2.148960in}}{\pgfqpoint{1.671907in}{2.156860in}}{\pgfqpoint{1.666083in}{2.162684in}}%
\pgfpathcurveto{\pgfqpoint{1.660259in}{2.168508in}}{\pgfqpoint{1.652359in}{2.171780in}}{\pgfqpoint{1.644123in}{2.171780in}}%
\pgfpathcurveto{\pgfqpoint{1.635887in}{2.171780in}}{\pgfqpoint{1.627987in}{2.168508in}}{\pgfqpoint{1.622163in}{2.162684in}}%
\pgfpathcurveto{\pgfqpoint{1.616339in}{2.156860in}}{\pgfqpoint{1.613067in}{2.148960in}}{\pgfqpoint{1.613067in}{2.140724in}}%
\pgfpathcurveto{\pgfqpoint{1.613067in}{2.132487in}}{\pgfqpoint{1.616339in}{2.124587in}}{\pgfqpoint{1.622163in}{2.118763in}}%
\pgfpathcurveto{\pgfqpoint{1.627987in}{2.112939in}}{\pgfqpoint{1.635887in}{2.109667in}}{\pgfqpoint{1.644123in}{2.109667in}}%
\pgfpathclose%
\pgfusepath{stroke,fill}%
\end{pgfscope}%
\begin{pgfscope}%
\pgfpathrectangle{\pgfqpoint{0.100000in}{0.212622in}}{\pgfqpoint{3.696000in}{3.696000in}}%
\pgfusepath{clip}%
\pgfsetbuttcap%
\pgfsetroundjoin%
\definecolor{currentfill}{rgb}{0.121569,0.466667,0.705882}%
\pgfsetfillcolor{currentfill}%
\pgfsetfillopacity{0.301449}%
\pgfsetlinewidth{1.003750pt}%
\definecolor{currentstroke}{rgb}{0.121569,0.466667,0.705882}%
\pgfsetstrokecolor{currentstroke}%
\pgfsetstrokeopacity{0.301449}%
\pgfsetdash{}{0pt}%
\pgfpathmoveto{\pgfqpoint{1.644123in}{2.109667in}}%
\pgfpathcurveto{\pgfqpoint{1.652359in}{2.109667in}}{\pgfqpoint{1.660259in}{2.112939in}}{\pgfqpoint{1.666083in}{2.118763in}}%
\pgfpathcurveto{\pgfqpoint{1.671907in}{2.124587in}}{\pgfqpoint{1.675180in}{2.132487in}}{\pgfqpoint{1.675180in}{2.140724in}}%
\pgfpathcurveto{\pgfqpoint{1.675180in}{2.148960in}}{\pgfqpoint{1.671907in}{2.156860in}}{\pgfqpoint{1.666083in}{2.162684in}}%
\pgfpathcurveto{\pgfqpoint{1.660259in}{2.168508in}}{\pgfqpoint{1.652359in}{2.171780in}}{\pgfqpoint{1.644123in}{2.171780in}}%
\pgfpathcurveto{\pgfqpoint{1.635887in}{2.171780in}}{\pgfqpoint{1.627987in}{2.168508in}}{\pgfqpoint{1.622163in}{2.162684in}}%
\pgfpathcurveto{\pgfqpoint{1.616339in}{2.156860in}}{\pgfqpoint{1.613067in}{2.148960in}}{\pgfqpoint{1.613067in}{2.140724in}}%
\pgfpathcurveto{\pgfqpoint{1.613067in}{2.132487in}}{\pgfqpoint{1.616339in}{2.124587in}}{\pgfqpoint{1.622163in}{2.118763in}}%
\pgfpathcurveto{\pgfqpoint{1.627987in}{2.112939in}}{\pgfqpoint{1.635887in}{2.109667in}}{\pgfqpoint{1.644123in}{2.109667in}}%
\pgfpathclose%
\pgfusepath{stroke,fill}%
\end{pgfscope}%
\begin{pgfscope}%
\pgfpathrectangle{\pgfqpoint{0.100000in}{0.212622in}}{\pgfqpoint{3.696000in}{3.696000in}}%
\pgfusepath{clip}%
\pgfsetbuttcap%
\pgfsetroundjoin%
\definecolor{currentfill}{rgb}{0.121569,0.466667,0.705882}%
\pgfsetfillcolor{currentfill}%
\pgfsetfillopacity{0.301449}%
\pgfsetlinewidth{1.003750pt}%
\definecolor{currentstroke}{rgb}{0.121569,0.466667,0.705882}%
\pgfsetstrokecolor{currentstroke}%
\pgfsetstrokeopacity{0.301449}%
\pgfsetdash{}{0pt}%
\pgfpathmoveto{\pgfqpoint{1.644123in}{2.109667in}}%
\pgfpathcurveto{\pgfqpoint{1.652359in}{2.109667in}}{\pgfqpoint{1.660259in}{2.112939in}}{\pgfqpoint{1.666083in}{2.118763in}}%
\pgfpathcurveto{\pgfqpoint{1.671907in}{2.124587in}}{\pgfqpoint{1.675179in}{2.132487in}}{\pgfqpoint{1.675179in}{2.140724in}}%
\pgfpathcurveto{\pgfqpoint{1.675179in}{2.148960in}}{\pgfqpoint{1.671907in}{2.156860in}}{\pgfqpoint{1.666083in}{2.162684in}}%
\pgfpathcurveto{\pgfqpoint{1.660259in}{2.168508in}}{\pgfqpoint{1.652359in}{2.171780in}}{\pgfqpoint{1.644123in}{2.171780in}}%
\pgfpathcurveto{\pgfqpoint{1.635887in}{2.171780in}}{\pgfqpoint{1.627987in}{2.168508in}}{\pgfqpoint{1.622163in}{2.162684in}}%
\pgfpathcurveto{\pgfqpoint{1.616339in}{2.156860in}}{\pgfqpoint{1.613066in}{2.148960in}}{\pgfqpoint{1.613066in}{2.140724in}}%
\pgfpathcurveto{\pgfqpoint{1.613066in}{2.132487in}}{\pgfqpoint{1.616339in}{2.124587in}}{\pgfqpoint{1.622163in}{2.118763in}}%
\pgfpathcurveto{\pgfqpoint{1.627987in}{2.112939in}}{\pgfqpoint{1.635887in}{2.109667in}}{\pgfqpoint{1.644123in}{2.109667in}}%
\pgfpathclose%
\pgfusepath{stroke,fill}%
\end{pgfscope}%
\begin{pgfscope}%
\pgfpathrectangle{\pgfqpoint{0.100000in}{0.212622in}}{\pgfqpoint{3.696000in}{3.696000in}}%
\pgfusepath{clip}%
\pgfsetbuttcap%
\pgfsetroundjoin%
\definecolor{currentfill}{rgb}{0.121569,0.466667,0.705882}%
\pgfsetfillcolor{currentfill}%
\pgfsetfillopacity{0.301449}%
\pgfsetlinewidth{1.003750pt}%
\definecolor{currentstroke}{rgb}{0.121569,0.466667,0.705882}%
\pgfsetstrokecolor{currentstroke}%
\pgfsetstrokeopacity{0.301449}%
\pgfsetdash{}{0pt}%
\pgfpathmoveto{\pgfqpoint{1.644123in}{2.109667in}}%
\pgfpathcurveto{\pgfqpoint{1.652359in}{2.109667in}}{\pgfqpoint{1.660259in}{2.112939in}}{\pgfqpoint{1.666083in}{2.118763in}}%
\pgfpathcurveto{\pgfqpoint{1.671907in}{2.124587in}}{\pgfqpoint{1.675179in}{2.132487in}}{\pgfqpoint{1.675179in}{2.140723in}}%
\pgfpathcurveto{\pgfqpoint{1.675179in}{2.148960in}}{\pgfqpoint{1.671907in}{2.156860in}}{\pgfqpoint{1.666083in}{2.162684in}}%
\pgfpathcurveto{\pgfqpoint{1.660259in}{2.168508in}}{\pgfqpoint{1.652359in}{2.171780in}}{\pgfqpoint{1.644123in}{2.171780in}}%
\pgfpathcurveto{\pgfqpoint{1.635887in}{2.171780in}}{\pgfqpoint{1.627986in}{2.168508in}}{\pgfqpoint{1.622163in}{2.162684in}}%
\pgfpathcurveto{\pgfqpoint{1.616339in}{2.156860in}}{\pgfqpoint{1.613066in}{2.148960in}}{\pgfqpoint{1.613066in}{2.140723in}}%
\pgfpathcurveto{\pgfqpoint{1.613066in}{2.132487in}}{\pgfqpoint{1.616339in}{2.124587in}}{\pgfqpoint{1.622163in}{2.118763in}}%
\pgfpathcurveto{\pgfqpoint{1.627986in}{2.112939in}}{\pgfqpoint{1.635887in}{2.109667in}}{\pgfqpoint{1.644123in}{2.109667in}}%
\pgfpathclose%
\pgfusepath{stroke,fill}%
\end{pgfscope}%
\begin{pgfscope}%
\pgfpathrectangle{\pgfqpoint{0.100000in}{0.212622in}}{\pgfqpoint{3.696000in}{3.696000in}}%
\pgfusepath{clip}%
\pgfsetbuttcap%
\pgfsetroundjoin%
\definecolor{currentfill}{rgb}{0.121569,0.466667,0.705882}%
\pgfsetfillcolor{currentfill}%
\pgfsetfillopacity{0.301449}%
\pgfsetlinewidth{1.003750pt}%
\definecolor{currentstroke}{rgb}{0.121569,0.466667,0.705882}%
\pgfsetstrokecolor{currentstroke}%
\pgfsetstrokeopacity{0.301449}%
\pgfsetdash{}{0pt}%
\pgfpathmoveto{\pgfqpoint{1.644123in}{2.109667in}}%
\pgfpathcurveto{\pgfqpoint{1.652359in}{2.109667in}}{\pgfqpoint{1.660259in}{2.112939in}}{\pgfqpoint{1.666083in}{2.118763in}}%
\pgfpathcurveto{\pgfqpoint{1.671907in}{2.124587in}}{\pgfqpoint{1.675179in}{2.132487in}}{\pgfqpoint{1.675179in}{2.140723in}}%
\pgfpathcurveto{\pgfqpoint{1.675179in}{2.148960in}}{\pgfqpoint{1.671907in}{2.156860in}}{\pgfqpoint{1.666083in}{2.162684in}}%
\pgfpathcurveto{\pgfqpoint{1.660259in}{2.168508in}}{\pgfqpoint{1.652359in}{2.171780in}}{\pgfqpoint{1.644123in}{2.171780in}}%
\pgfpathcurveto{\pgfqpoint{1.635886in}{2.171780in}}{\pgfqpoint{1.627986in}{2.168508in}}{\pgfqpoint{1.622162in}{2.162684in}}%
\pgfpathcurveto{\pgfqpoint{1.616339in}{2.156860in}}{\pgfqpoint{1.613066in}{2.148960in}}{\pgfqpoint{1.613066in}{2.140723in}}%
\pgfpathcurveto{\pgfqpoint{1.613066in}{2.132487in}}{\pgfqpoint{1.616339in}{2.124587in}}{\pgfqpoint{1.622162in}{2.118763in}}%
\pgfpathcurveto{\pgfqpoint{1.627986in}{2.112939in}}{\pgfqpoint{1.635886in}{2.109667in}}{\pgfqpoint{1.644123in}{2.109667in}}%
\pgfpathclose%
\pgfusepath{stroke,fill}%
\end{pgfscope}%
\begin{pgfscope}%
\pgfpathrectangle{\pgfqpoint{0.100000in}{0.212622in}}{\pgfqpoint{3.696000in}{3.696000in}}%
\pgfusepath{clip}%
\pgfsetbuttcap%
\pgfsetroundjoin%
\definecolor{currentfill}{rgb}{0.121569,0.466667,0.705882}%
\pgfsetfillcolor{currentfill}%
\pgfsetfillopacity{0.301449}%
\pgfsetlinewidth{1.003750pt}%
\definecolor{currentstroke}{rgb}{0.121569,0.466667,0.705882}%
\pgfsetstrokecolor{currentstroke}%
\pgfsetstrokeopacity{0.301449}%
\pgfsetdash{}{0pt}%
\pgfpathmoveto{\pgfqpoint{1.644122in}{2.109666in}}%
\pgfpathcurveto{\pgfqpoint{1.652358in}{2.109666in}}{\pgfqpoint{1.660259in}{2.112939in}}{\pgfqpoint{1.666082in}{2.118763in}}%
\pgfpathcurveto{\pgfqpoint{1.671906in}{2.124587in}}{\pgfqpoint{1.675179in}{2.132487in}}{\pgfqpoint{1.675179in}{2.140723in}}%
\pgfpathcurveto{\pgfqpoint{1.675179in}{2.148959in}}{\pgfqpoint{1.671906in}{2.156859in}}{\pgfqpoint{1.666082in}{2.162683in}}%
\pgfpathcurveto{\pgfqpoint{1.660259in}{2.168507in}}{\pgfqpoint{1.652358in}{2.171779in}}{\pgfqpoint{1.644122in}{2.171779in}}%
\pgfpathcurveto{\pgfqpoint{1.635886in}{2.171779in}}{\pgfqpoint{1.627986in}{2.168507in}}{\pgfqpoint{1.622162in}{2.162683in}}%
\pgfpathcurveto{\pgfqpoint{1.616338in}{2.156859in}}{\pgfqpoint{1.613066in}{2.148959in}}{\pgfqpoint{1.613066in}{2.140723in}}%
\pgfpathcurveto{\pgfqpoint{1.613066in}{2.132487in}}{\pgfqpoint{1.616338in}{2.124587in}}{\pgfqpoint{1.622162in}{2.118763in}}%
\pgfpathcurveto{\pgfqpoint{1.627986in}{2.112939in}}{\pgfqpoint{1.635886in}{2.109666in}}{\pgfqpoint{1.644122in}{2.109666in}}%
\pgfpathclose%
\pgfusepath{stroke,fill}%
\end{pgfscope}%
\begin{pgfscope}%
\pgfpathrectangle{\pgfqpoint{0.100000in}{0.212622in}}{\pgfqpoint{3.696000in}{3.696000in}}%
\pgfusepath{clip}%
\pgfsetbuttcap%
\pgfsetroundjoin%
\definecolor{currentfill}{rgb}{0.121569,0.466667,0.705882}%
\pgfsetfillcolor{currentfill}%
\pgfsetfillopacity{0.301449}%
\pgfsetlinewidth{1.003750pt}%
\definecolor{currentstroke}{rgb}{0.121569,0.466667,0.705882}%
\pgfsetstrokecolor{currentstroke}%
\pgfsetstrokeopacity{0.301449}%
\pgfsetdash{}{0pt}%
\pgfpathmoveto{\pgfqpoint{1.644122in}{2.109666in}}%
\pgfpathcurveto{\pgfqpoint{1.652358in}{2.109666in}}{\pgfqpoint{1.660258in}{2.112938in}}{\pgfqpoint{1.666082in}{2.118762in}}%
\pgfpathcurveto{\pgfqpoint{1.671906in}{2.124586in}}{\pgfqpoint{1.675178in}{2.132486in}}{\pgfqpoint{1.675178in}{2.140723in}}%
\pgfpathcurveto{\pgfqpoint{1.675178in}{2.148959in}}{\pgfqpoint{1.671906in}{2.156859in}}{\pgfqpoint{1.666082in}{2.162683in}}%
\pgfpathcurveto{\pgfqpoint{1.660258in}{2.168507in}}{\pgfqpoint{1.652358in}{2.171779in}}{\pgfqpoint{1.644122in}{2.171779in}}%
\pgfpathcurveto{\pgfqpoint{1.635885in}{2.171779in}}{\pgfqpoint{1.627985in}{2.168507in}}{\pgfqpoint{1.622161in}{2.162683in}}%
\pgfpathcurveto{\pgfqpoint{1.616338in}{2.156859in}}{\pgfqpoint{1.613065in}{2.148959in}}{\pgfqpoint{1.613065in}{2.140723in}}%
\pgfpathcurveto{\pgfqpoint{1.613065in}{2.132486in}}{\pgfqpoint{1.616338in}{2.124586in}}{\pgfqpoint{1.622161in}{2.118762in}}%
\pgfpathcurveto{\pgfqpoint{1.627985in}{2.112938in}}{\pgfqpoint{1.635885in}{2.109666in}}{\pgfqpoint{1.644122in}{2.109666in}}%
\pgfpathclose%
\pgfusepath{stroke,fill}%
\end{pgfscope}%
\begin{pgfscope}%
\pgfpathrectangle{\pgfqpoint{0.100000in}{0.212622in}}{\pgfqpoint{3.696000in}{3.696000in}}%
\pgfusepath{clip}%
\pgfsetbuttcap%
\pgfsetroundjoin%
\definecolor{currentfill}{rgb}{0.121569,0.466667,0.705882}%
\pgfsetfillcolor{currentfill}%
\pgfsetfillopacity{0.301449}%
\pgfsetlinewidth{1.003750pt}%
\definecolor{currentstroke}{rgb}{0.121569,0.466667,0.705882}%
\pgfsetstrokecolor{currentstroke}%
\pgfsetstrokeopacity{0.301449}%
\pgfsetdash{}{0pt}%
\pgfpathmoveto{\pgfqpoint{1.644120in}{2.109665in}}%
\pgfpathcurveto{\pgfqpoint{1.652357in}{2.109665in}}{\pgfqpoint{1.660257in}{2.112937in}}{\pgfqpoint{1.666081in}{2.118761in}}%
\pgfpathcurveto{\pgfqpoint{1.671905in}{2.124585in}}{\pgfqpoint{1.675177in}{2.132485in}}{\pgfqpoint{1.675177in}{2.140722in}}%
\pgfpathcurveto{\pgfqpoint{1.675177in}{2.148958in}}{\pgfqpoint{1.671905in}{2.156858in}}{\pgfqpoint{1.666081in}{2.162682in}}%
\pgfpathcurveto{\pgfqpoint{1.660257in}{2.168506in}}{\pgfqpoint{1.652357in}{2.171778in}}{\pgfqpoint{1.644120in}{2.171778in}}%
\pgfpathcurveto{\pgfqpoint{1.635884in}{2.171778in}}{\pgfqpoint{1.627984in}{2.168506in}}{\pgfqpoint{1.622160in}{2.162682in}}%
\pgfpathcurveto{\pgfqpoint{1.616336in}{2.156858in}}{\pgfqpoint{1.613064in}{2.148958in}}{\pgfqpoint{1.613064in}{2.140722in}}%
\pgfpathcurveto{\pgfqpoint{1.613064in}{2.132485in}}{\pgfqpoint{1.616336in}{2.124585in}}{\pgfqpoint{1.622160in}{2.118761in}}%
\pgfpathcurveto{\pgfqpoint{1.627984in}{2.112937in}}{\pgfqpoint{1.635884in}{2.109665in}}{\pgfqpoint{1.644120in}{2.109665in}}%
\pgfpathclose%
\pgfusepath{stroke,fill}%
\end{pgfscope}%
\begin{pgfscope}%
\pgfpathrectangle{\pgfqpoint{0.100000in}{0.212622in}}{\pgfqpoint{3.696000in}{3.696000in}}%
\pgfusepath{clip}%
\pgfsetbuttcap%
\pgfsetroundjoin%
\definecolor{currentfill}{rgb}{0.121569,0.466667,0.705882}%
\pgfsetfillcolor{currentfill}%
\pgfsetfillopacity{0.301449}%
\pgfsetlinewidth{1.003750pt}%
\definecolor{currentstroke}{rgb}{0.121569,0.466667,0.705882}%
\pgfsetstrokecolor{currentstroke}%
\pgfsetstrokeopacity{0.301449}%
\pgfsetdash{}{0pt}%
\pgfpathmoveto{\pgfqpoint{1.644119in}{2.109665in}}%
\pgfpathcurveto{\pgfqpoint{1.652356in}{2.109665in}}{\pgfqpoint{1.660256in}{2.112937in}}{\pgfqpoint{1.666080in}{2.118761in}}%
\pgfpathcurveto{\pgfqpoint{1.671904in}{2.124585in}}{\pgfqpoint{1.675176in}{2.132485in}}{\pgfqpoint{1.675176in}{2.140721in}}%
\pgfpathcurveto{\pgfqpoint{1.675176in}{2.148958in}}{\pgfqpoint{1.671904in}{2.156858in}}{\pgfqpoint{1.666080in}{2.162681in}}%
\pgfpathcurveto{\pgfqpoint{1.660256in}{2.168505in}}{\pgfqpoint{1.652356in}{2.171778in}}{\pgfqpoint{1.644119in}{2.171778in}}%
\pgfpathcurveto{\pgfqpoint{1.635883in}{2.171778in}}{\pgfqpoint{1.627983in}{2.168505in}}{\pgfqpoint{1.622159in}{2.162681in}}%
\pgfpathcurveto{\pgfqpoint{1.616335in}{2.156858in}}{\pgfqpoint{1.613063in}{2.148958in}}{\pgfqpoint{1.613063in}{2.140721in}}%
\pgfpathcurveto{\pgfqpoint{1.613063in}{2.132485in}}{\pgfqpoint{1.616335in}{2.124585in}}{\pgfqpoint{1.622159in}{2.118761in}}%
\pgfpathcurveto{\pgfqpoint{1.627983in}{2.112937in}}{\pgfqpoint{1.635883in}{2.109665in}}{\pgfqpoint{1.644119in}{2.109665in}}%
\pgfpathclose%
\pgfusepath{stroke,fill}%
\end{pgfscope}%
\begin{pgfscope}%
\pgfpathrectangle{\pgfqpoint{0.100000in}{0.212622in}}{\pgfqpoint{3.696000in}{3.696000in}}%
\pgfusepath{clip}%
\pgfsetbuttcap%
\pgfsetroundjoin%
\definecolor{currentfill}{rgb}{0.121569,0.466667,0.705882}%
\pgfsetfillcolor{currentfill}%
\pgfsetfillopacity{0.301450}%
\pgfsetlinewidth{1.003750pt}%
\definecolor{currentstroke}{rgb}{0.121569,0.466667,0.705882}%
\pgfsetstrokecolor{currentstroke}%
\pgfsetstrokeopacity{0.301450}%
\pgfsetdash{}{0pt}%
\pgfpathmoveto{\pgfqpoint{1.644115in}{2.109660in}}%
\pgfpathcurveto{\pgfqpoint{1.652351in}{2.109660in}}{\pgfqpoint{1.660251in}{2.112932in}}{\pgfqpoint{1.666075in}{2.118756in}}%
\pgfpathcurveto{\pgfqpoint{1.671899in}{2.124580in}}{\pgfqpoint{1.675171in}{2.132480in}}{\pgfqpoint{1.675171in}{2.140716in}}%
\pgfpathcurveto{\pgfqpoint{1.675171in}{2.148953in}}{\pgfqpoint{1.671899in}{2.156853in}}{\pgfqpoint{1.666075in}{2.162677in}}%
\pgfpathcurveto{\pgfqpoint{1.660251in}{2.168501in}}{\pgfqpoint{1.652351in}{2.171773in}}{\pgfqpoint{1.644115in}{2.171773in}}%
\pgfpathcurveto{\pgfqpoint{1.635879in}{2.171773in}}{\pgfqpoint{1.627979in}{2.168501in}}{\pgfqpoint{1.622155in}{2.162677in}}%
\pgfpathcurveto{\pgfqpoint{1.616331in}{2.156853in}}{\pgfqpoint{1.613058in}{2.148953in}}{\pgfqpoint{1.613058in}{2.140716in}}%
\pgfpathcurveto{\pgfqpoint{1.613058in}{2.132480in}}{\pgfqpoint{1.616331in}{2.124580in}}{\pgfqpoint{1.622155in}{2.118756in}}%
\pgfpathcurveto{\pgfqpoint{1.627979in}{2.112932in}}{\pgfqpoint{1.635879in}{2.109660in}}{\pgfqpoint{1.644115in}{2.109660in}}%
\pgfpathclose%
\pgfusepath{stroke,fill}%
\end{pgfscope}%
\begin{pgfscope}%
\pgfpathrectangle{\pgfqpoint{0.100000in}{0.212622in}}{\pgfqpoint{3.696000in}{3.696000in}}%
\pgfusepath{clip}%
\pgfsetbuttcap%
\pgfsetroundjoin%
\definecolor{currentfill}{rgb}{0.121569,0.466667,0.705882}%
\pgfsetfillcolor{currentfill}%
\pgfsetfillopacity{0.301451}%
\pgfsetlinewidth{1.003750pt}%
\definecolor{currentstroke}{rgb}{0.121569,0.466667,0.705882}%
\pgfsetstrokecolor{currentstroke}%
\pgfsetstrokeopacity{0.301451}%
\pgfsetdash{}{0pt}%
\pgfpathmoveto{\pgfqpoint{1.644105in}{2.109653in}}%
\pgfpathcurveto{\pgfqpoint{1.652341in}{2.109653in}}{\pgfqpoint{1.660241in}{2.112926in}}{\pgfqpoint{1.666065in}{2.118750in}}%
\pgfpathcurveto{\pgfqpoint{1.671889in}{2.124573in}}{\pgfqpoint{1.675162in}{2.132474in}}{\pgfqpoint{1.675162in}{2.140710in}}%
\pgfpathcurveto{\pgfqpoint{1.675162in}{2.148946in}}{\pgfqpoint{1.671889in}{2.156846in}}{\pgfqpoint{1.666065in}{2.162670in}}%
\pgfpathcurveto{\pgfqpoint{1.660241in}{2.168494in}}{\pgfqpoint{1.652341in}{2.171766in}}{\pgfqpoint{1.644105in}{2.171766in}}%
\pgfpathcurveto{\pgfqpoint{1.635869in}{2.171766in}}{\pgfqpoint{1.627969in}{2.168494in}}{\pgfqpoint{1.622145in}{2.162670in}}%
\pgfpathcurveto{\pgfqpoint{1.616321in}{2.156846in}}{\pgfqpoint{1.613049in}{2.148946in}}{\pgfqpoint{1.613049in}{2.140710in}}%
\pgfpathcurveto{\pgfqpoint{1.613049in}{2.132474in}}{\pgfqpoint{1.616321in}{2.124573in}}{\pgfqpoint{1.622145in}{2.118750in}}%
\pgfpathcurveto{\pgfqpoint{1.627969in}{2.112926in}}{\pgfqpoint{1.635869in}{2.109653in}}{\pgfqpoint{1.644105in}{2.109653in}}%
\pgfpathclose%
\pgfusepath{stroke,fill}%
\end{pgfscope}%
\begin{pgfscope}%
\pgfpathrectangle{\pgfqpoint{0.100000in}{0.212622in}}{\pgfqpoint{3.696000in}{3.696000in}}%
\pgfusepath{clip}%
\pgfsetbuttcap%
\pgfsetroundjoin%
\definecolor{currentfill}{rgb}{0.121569,0.466667,0.705882}%
\pgfsetfillcolor{currentfill}%
\pgfsetfillopacity{0.301451}%
\pgfsetlinewidth{1.003750pt}%
\definecolor{currentstroke}{rgb}{0.121569,0.466667,0.705882}%
\pgfsetstrokecolor{currentstroke}%
\pgfsetstrokeopacity{0.301451}%
\pgfsetdash{}{0pt}%
\pgfpathmoveto{\pgfqpoint{1.644095in}{2.109648in}}%
\pgfpathcurveto{\pgfqpoint{1.652331in}{2.109648in}}{\pgfqpoint{1.660231in}{2.112921in}}{\pgfqpoint{1.666055in}{2.118745in}}%
\pgfpathcurveto{\pgfqpoint{1.671879in}{2.124569in}}{\pgfqpoint{1.675151in}{2.132469in}}{\pgfqpoint{1.675151in}{2.140705in}}%
\pgfpathcurveto{\pgfqpoint{1.675151in}{2.148941in}}{\pgfqpoint{1.671879in}{2.156841in}}{\pgfqpoint{1.666055in}{2.162665in}}%
\pgfpathcurveto{\pgfqpoint{1.660231in}{2.168489in}}{\pgfqpoint{1.652331in}{2.171761in}}{\pgfqpoint{1.644095in}{2.171761in}}%
\pgfpathcurveto{\pgfqpoint{1.635859in}{2.171761in}}{\pgfqpoint{1.627959in}{2.168489in}}{\pgfqpoint{1.622135in}{2.162665in}}%
\pgfpathcurveto{\pgfqpoint{1.616311in}{2.156841in}}{\pgfqpoint{1.613038in}{2.148941in}}{\pgfqpoint{1.613038in}{2.140705in}}%
\pgfpathcurveto{\pgfqpoint{1.613038in}{2.132469in}}{\pgfqpoint{1.616311in}{2.124569in}}{\pgfqpoint{1.622135in}{2.118745in}}%
\pgfpathcurveto{\pgfqpoint{1.627959in}{2.112921in}}{\pgfqpoint{1.635859in}{2.109648in}}{\pgfqpoint{1.644095in}{2.109648in}}%
\pgfpathclose%
\pgfusepath{stroke,fill}%
\end{pgfscope}%
\begin{pgfscope}%
\pgfpathrectangle{\pgfqpoint{0.100000in}{0.212622in}}{\pgfqpoint{3.696000in}{3.696000in}}%
\pgfusepath{clip}%
\pgfsetbuttcap%
\pgfsetroundjoin%
\definecolor{currentfill}{rgb}{0.121569,0.466667,0.705882}%
\pgfsetfillcolor{currentfill}%
\pgfsetfillopacity{0.301456}%
\pgfsetlinewidth{1.003750pt}%
\definecolor{currentstroke}{rgb}{0.121569,0.466667,0.705882}%
\pgfsetstrokecolor{currentstroke}%
\pgfsetstrokeopacity{0.301456}%
\pgfsetdash{}{0pt}%
\pgfpathmoveto{\pgfqpoint{1.644060in}{2.109619in}}%
\pgfpathcurveto{\pgfqpoint{1.652296in}{2.109619in}}{\pgfqpoint{1.660196in}{2.112891in}}{\pgfqpoint{1.666020in}{2.118715in}}%
\pgfpathcurveto{\pgfqpoint{1.671844in}{2.124539in}}{\pgfqpoint{1.675116in}{2.132439in}}{\pgfqpoint{1.675116in}{2.140675in}}%
\pgfpathcurveto{\pgfqpoint{1.675116in}{2.148912in}}{\pgfqpoint{1.671844in}{2.156812in}}{\pgfqpoint{1.666020in}{2.162636in}}%
\pgfpathcurveto{\pgfqpoint{1.660196in}{2.168460in}}{\pgfqpoint{1.652296in}{2.171732in}}{\pgfqpoint{1.644060in}{2.171732in}}%
\pgfpathcurveto{\pgfqpoint{1.635824in}{2.171732in}}{\pgfqpoint{1.627924in}{2.168460in}}{\pgfqpoint{1.622100in}{2.162636in}}%
\pgfpathcurveto{\pgfqpoint{1.616276in}{2.156812in}}{\pgfqpoint{1.613003in}{2.148912in}}{\pgfqpoint{1.613003in}{2.140675in}}%
\pgfpathcurveto{\pgfqpoint{1.613003in}{2.132439in}}{\pgfqpoint{1.616276in}{2.124539in}}{\pgfqpoint{1.622100in}{2.118715in}}%
\pgfpathcurveto{\pgfqpoint{1.627924in}{2.112891in}}{\pgfqpoint{1.635824in}{2.109619in}}{\pgfqpoint{1.644060in}{2.109619in}}%
\pgfpathclose%
\pgfusepath{stroke,fill}%
\end{pgfscope}%
\begin{pgfscope}%
\pgfpathrectangle{\pgfqpoint{0.100000in}{0.212622in}}{\pgfqpoint{3.696000in}{3.696000in}}%
\pgfusepath{clip}%
\pgfsetbuttcap%
\pgfsetroundjoin%
\definecolor{currentfill}{rgb}{0.121569,0.466667,0.705882}%
\pgfsetfillcolor{currentfill}%
\pgfsetfillopacity{0.301464}%
\pgfsetlinewidth{1.003750pt}%
\definecolor{currentstroke}{rgb}{0.121569,0.466667,0.705882}%
\pgfsetstrokecolor{currentstroke}%
\pgfsetstrokeopacity{0.301464}%
\pgfsetdash{}{0pt}%
\pgfpathmoveto{\pgfqpoint{1.644005in}{2.109576in}}%
\pgfpathcurveto{\pgfqpoint{1.652241in}{2.109576in}}{\pgfqpoint{1.660142in}{2.112848in}}{\pgfqpoint{1.665965in}{2.118672in}}%
\pgfpathcurveto{\pgfqpoint{1.671789in}{2.124496in}}{\pgfqpoint{1.675062in}{2.132396in}}{\pgfqpoint{1.675062in}{2.140632in}}%
\pgfpathcurveto{\pgfqpoint{1.675062in}{2.148868in}}{\pgfqpoint{1.671789in}{2.156768in}}{\pgfqpoint{1.665965in}{2.162592in}}%
\pgfpathcurveto{\pgfqpoint{1.660142in}{2.168416in}}{\pgfqpoint{1.652241in}{2.171689in}}{\pgfqpoint{1.644005in}{2.171689in}}%
\pgfpathcurveto{\pgfqpoint{1.635769in}{2.171689in}}{\pgfqpoint{1.627869in}{2.168416in}}{\pgfqpoint{1.622045in}{2.162592in}}%
\pgfpathcurveto{\pgfqpoint{1.616221in}{2.156768in}}{\pgfqpoint{1.612949in}{2.148868in}}{\pgfqpoint{1.612949in}{2.140632in}}%
\pgfpathcurveto{\pgfqpoint{1.612949in}{2.132396in}}{\pgfqpoint{1.616221in}{2.124496in}}{\pgfqpoint{1.622045in}{2.118672in}}%
\pgfpathcurveto{\pgfqpoint{1.627869in}{2.112848in}}{\pgfqpoint{1.635769in}{2.109576in}}{\pgfqpoint{1.644005in}{2.109576in}}%
\pgfpathclose%
\pgfusepath{stroke,fill}%
\end{pgfscope}%
\begin{pgfscope}%
\pgfpathrectangle{\pgfqpoint{0.100000in}{0.212622in}}{\pgfqpoint{3.696000in}{3.696000in}}%
\pgfusepath{clip}%
\pgfsetbuttcap%
\pgfsetroundjoin%
\definecolor{currentfill}{rgb}{0.121569,0.466667,0.705882}%
\pgfsetfillcolor{currentfill}%
\pgfsetfillopacity{0.302100}%
\pgfsetlinewidth{1.003750pt}%
\definecolor{currentstroke}{rgb}{0.121569,0.466667,0.705882}%
\pgfsetstrokecolor{currentstroke}%
\pgfsetstrokeopacity{0.302100}%
\pgfsetdash{}{0pt}%
\pgfpathmoveto{\pgfqpoint{1.648865in}{2.115035in}}%
\pgfpathcurveto{\pgfqpoint{1.657101in}{2.115035in}}{\pgfqpoint{1.665001in}{2.118308in}}{\pgfqpoint{1.670825in}{2.124132in}}%
\pgfpathcurveto{\pgfqpoint{1.676649in}{2.129956in}}{\pgfqpoint{1.679922in}{2.137856in}}{\pgfqpoint{1.679922in}{2.146092in}}%
\pgfpathcurveto{\pgfqpoint{1.679922in}{2.154328in}}{\pgfqpoint{1.676649in}{2.162228in}}{\pgfqpoint{1.670825in}{2.168052in}}%
\pgfpathcurveto{\pgfqpoint{1.665001in}{2.173876in}}{\pgfqpoint{1.657101in}{2.177148in}}{\pgfqpoint{1.648865in}{2.177148in}}%
\pgfpathcurveto{\pgfqpoint{1.640629in}{2.177148in}}{\pgfqpoint{1.632729in}{2.173876in}}{\pgfqpoint{1.626905in}{2.168052in}}%
\pgfpathcurveto{\pgfqpoint{1.621081in}{2.162228in}}{\pgfqpoint{1.617809in}{2.154328in}}{\pgfqpoint{1.617809in}{2.146092in}}%
\pgfpathcurveto{\pgfqpoint{1.617809in}{2.137856in}}{\pgfqpoint{1.621081in}{2.129956in}}{\pgfqpoint{1.626905in}{2.124132in}}%
\pgfpathcurveto{\pgfqpoint{1.632729in}{2.118308in}}{\pgfqpoint{1.640629in}{2.115035in}}{\pgfqpoint{1.648865in}{2.115035in}}%
\pgfpathclose%
\pgfusepath{stroke,fill}%
\end{pgfscope}%
\begin{pgfscope}%
\pgfpathrectangle{\pgfqpoint{0.100000in}{0.212622in}}{\pgfqpoint{3.696000in}{3.696000in}}%
\pgfusepath{clip}%
\pgfsetbuttcap%
\pgfsetroundjoin%
\definecolor{currentfill}{rgb}{0.121569,0.466667,0.705882}%
\pgfsetfillcolor{currentfill}%
\pgfsetfillopacity{0.304056}%
\pgfsetlinewidth{1.003750pt}%
\definecolor{currentstroke}{rgb}{0.121569,0.466667,0.705882}%
\pgfsetstrokecolor{currentstroke}%
\pgfsetstrokeopacity{0.304056}%
\pgfsetdash{}{0pt}%
\pgfpathmoveto{\pgfqpoint{1.644245in}{2.109937in}}%
\pgfpathcurveto{\pgfqpoint{1.652481in}{2.109937in}}{\pgfqpoint{1.660381in}{2.113210in}}{\pgfqpoint{1.666205in}{2.119034in}}%
\pgfpathcurveto{\pgfqpoint{1.672029in}{2.124857in}}{\pgfqpoint{1.675301in}{2.132757in}}{\pgfqpoint{1.675301in}{2.140994in}}%
\pgfpathcurveto{\pgfqpoint{1.675301in}{2.149230in}}{\pgfqpoint{1.672029in}{2.157130in}}{\pgfqpoint{1.666205in}{2.162954in}}%
\pgfpathcurveto{\pgfqpoint{1.660381in}{2.168778in}}{\pgfqpoint{1.652481in}{2.172050in}}{\pgfqpoint{1.644245in}{2.172050in}}%
\pgfpathcurveto{\pgfqpoint{1.636008in}{2.172050in}}{\pgfqpoint{1.628108in}{2.168778in}}{\pgfqpoint{1.622284in}{2.162954in}}%
\pgfpathcurveto{\pgfqpoint{1.616460in}{2.157130in}}{\pgfqpoint{1.613188in}{2.149230in}}{\pgfqpoint{1.613188in}{2.140994in}}%
\pgfpathcurveto{\pgfqpoint{1.613188in}{2.132757in}}{\pgfqpoint{1.616460in}{2.124857in}}{\pgfqpoint{1.622284in}{2.119034in}}%
\pgfpathcurveto{\pgfqpoint{1.628108in}{2.113210in}}{\pgfqpoint{1.636008in}{2.109937in}}{\pgfqpoint{1.644245in}{2.109937in}}%
\pgfpathclose%
\pgfusepath{stroke,fill}%
\end{pgfscope}%
\begin{pgfscope}%
\pgfpathrectangle{\pgfqpoint{0.100000in}{0.212622in}}{\pgfqpoint{3.696000in}{3.696000in}}%
\pgfusepath{clip}%
\pgfsetbuttcap%
\pgfsetroundjoin%
\definecolor{currentfill}{rgb}{0.121569,0.466667,0.705882}%
\pgfsetfillcolor{currentfill}%
\pgfsetfillopacity{0.304764}%
\pgfsetlinewidth{1.003750pt}%
\definecolor{currentstroke}{rgb}{0.121569,0.466667,0.705882}%
\pgfsetstrokecolor{currentstroke}%
\pgfsetstrokeopacity{0.304764}%
\pgfsetdash{}{0pt}%
\pgfpathmoveto{\pgfqpoint{1.645084in}{2.113164in}}%
\pgfpathcurveto{\pgfqpoint{1.653320in}{2.113164in}}{\pgfqpoint{1.661220in}{2.116436in}}{\pgfqpoint{1.667044in}{2.122260in}}%
\pgfpathcurveto{\pgfqpoint{1.672868in}{2.128084in}}{\pgfqpoint{1.676140in}{2.135984in}}{\pgfqpoint{1.676140in}{2.144220in}}%
\pgfpathcurveto{\pgfqpoint{1.676140in}{2.152457in}}{\pgfqpoint{1.672868in}{2.160357in}}{\pgfqpoint{1.667044in}{2.166181in}}%
\pgfpathcurveto{\pgfqpoint{1.661220in}{2.172005in}}{\pgfqpoint{1.653320in}{2.175277in}}{\pgfqpoint{1.645084in}{2.175277in}}%
\pgfpathcurveto{\pgfqpoint{1.636848in}{2.175277in}}{\pgfqpoint{1.628948in}{2.172005in}}{\pgfqpoint{1.623124in}{2.166181in}}%
\pgfpathcurveto{\pgfqpoint{1.617300in}{2.160357in}}{\pgfqpoint{1.614027in}{2.152457in}}{\pgfqpoint{1.614027in}{2.144220in}}%
\pgfpathcurveto{\pgfqpoint{1.614027in}{2.135984in}}{\pgfqpoint{1.617300in}{2.128084in}}{\pgfqpoint{1.623124in}{2.122260in}}%
\pgfpathcurveto{\pgfqpoint{1.628948in}{2.116436in}}{\pgfqpoint{1.636848in}{2.113164in}}{\pgfqpoint{1.645084in}{2.113164in}}%
\pgfpathclose%
\pgfusepath{stroke,fill}%
\end{pgfscope}%
\begin{pgfscope}%
\pgfpathrectangle{\pgfqpoint{0.100000in}{0.212622in}}{\pgfqpoint{3.696000in}{3.696000in}}%
\pgfusepath{clip}%
\pgfsetbuttcap%
\pgfsetroundjoin%
\definecolor{currentfill}{rgb}{0.121569,0.466667,0.705882}%
\pgfsetfillcolor{currentfill}%
\pgfsetfillopacity{0.309200}%
\pgfsetlinewidth{1.003750pt}%
\definecolor{currentstroke}{rgb}{0.121569,0.466667,0.705882}%
\pgfsetstrokecolor{currentstroke}%
\pgfsetstrokeopacity{0.309200}%
\pgfsetdash{}{0pt}%
\pgfpathmoveto{\pgfqpoint{1.635893in}{2.104212in}}%
\pgfpathcurveto{\pgfqpoint{1.644129in}{2.104212in}}{\pgfqpoint{1.652029in}{2.107484in}}{\pgfqpoint{1.657853in}{2.113308in}}%
\pgfpathcurveto{\pgfqpoint{1.663677in}{2.119132in}}{\pgfqpoint{1.666950in}{2.127032in}}{\pgfqpoint{1.666950in}{2.135268in}}%
\pgfpathcurveto{\pgfqpoint{1.666950in}{2.143504in}}{\pgfqpoint{1.663677in}{2.151405in}}{\pgfqpoint{1.657853in}{2.157228in}}%
\pgfpathcurveto{\pgfqpoint{1.652029in}{2.163052in}}{\pgfqpoint{1.644129in}{2.166325in}}{\pgfqpoint{1.635893in}{2.166325in}}%
\pgfpathcurveto{\pgfqpoint{1.627657in}{2.166325in}}{\pgfqpoint{1.619757in}{2.163052in}}{\pgfqpoint{1.613933in}{2.157228in}}%
\pgfpathcurveto{\pgfqpoint{1.608109in}{2.151405in}}{\pgfqpoint{1.604837in}{2.143504in}}{\pgfqpoint{1.604837in}{2.135268in}}%
\pgfpathcurveto{\pgfqpoint{1.604837in}{2.127032in}}{\pgfqpoint{1.608109in}{2.119132in}}{\pgfqpoint{1.613933in}{2.113308in}}%
\pgfpathcurveto{\pgfqpoint{1.619757in}{2.107484in}}{\pgfqpoint{1.627657in}{2.104212in}}{\pgfqpoint{1.635893in}{2.104212in}}%
\pgfpathclose%
\pgfusepath{stroke,fill}%
\end{pgfscope}%
\begin{pgfscope}%
\pgfpathrectangle{\pgfqpoint{0.100000in}{0.212622in}}{\pgfqpoint{3.696000in}{3.696000in}}%
\pgfusepath{clip}%
\pgfsetbuttcap%
\pgfsetroundjoin%
\definecolor{currentfill}{rgb}{0.121569,0.466667,0.705882}%
\pgfsetfillcolor{currentfill}%
\pgfsetfillopacity{0.310041}%
\pgfsetlinewidth{1.003750pt}%
\definecolor{currentstroke}{rgb}{0.121569,0.466667,0.705882}%
\pgfsetstrokecolor{currentstroke}%
\pgfsetstrokeopacity{0.310041}%
\pgfsetdash{}{0pt}%
\pgfpathmoveto{\pgfqpoint{1.635020in}{2.103100in}}%
\pgfpathcurveto{\pgfqpoint{1.643257in}{2.103100in}}{\pgfqpoint{1.651157in}{2.106372in}}{\pgfqpoint{1.656981in}{2.112196in}}%
\pgfpathcurveto{\pgfqpoint{1.662804in}{2.118020in}}{\pgfqpoint{1.666077in}{2.125920in}}{\pgfqpoint{1.666077in}{2.134157in}}%
\pgfpathcurveto{\pgfqpoint{1.666077in}{2.142393in}}{\pgfqpoint{1.662804in}{2.150293in}}{\pgfqpoint{1.656981in}{2.156117in}}%
\pgfpathcurveto{\pgfqpoint{1.651157in}{2.161941in}}{\pgfqpoint{1.643257in}{2.165213in}}{\pgfqpoint{1.635020in}{2.165213in}}%
\pgfpathcurveto{\pgfqpoint{1.626784in}{2.165213in}}{\pgfqpoint{1.618884in}{2.161941in}}{\pgfqpoint{1.613060in}{2.156117in}}%
\pgfpathcurveto{\pgfqpoint{1.607236in}{2.150293in}}{\pgfqpoint{1.603964in}{2.142393in}}{\pgfqpoint{1.603964in}{2.134157in}}%
\pgfpathcurveto{\pgfqpoint{1.603964in}{2.125920in}}{\pgfqpoint{1.607236in}{2.118020in}}{\pgfqpoint{1.613060in}{2.112196in}}%
\pgfpathcurveto{\pgfqpoint{1.618884in}{2.106372in}}{\pgfqpoint{1.626784in}{2.103100in}}{\pgfqpoint{1.635020in}{2.103100in}}%
\pgfpathclose%
\pgfusepath{stroke,fill}%
\end{pgfscope}%
\begin{pgfscope}%
\pgfpathrectangle{\pgfqpoint{0.100000in}{0.212622in}}{\pgfqpoint{3.696000in}{3.696000in}}%
\pgfusepath{clip}%
\pgfsetbuttcap%
\pgfsetroundjoin%
\definecolor{currentfill}{rgb}{0.121569,0.466667,0.705882}%
\pgfsetfillcolor{currentfill}%
\pgfsetfillopacity{0.313502}%
\pgfsetlinewidth{1.003750pt}%
\definecolor{currentstroke}{rgb}{0.121569,0.466667,0.705882}%
\pgfsetstrokecolor{currentstroke}%
\pgfsetstrokeopacity{0.313502}%
\pgfsetdash{}{0pt}%
\pgfpathmoveto{\pgfqpoint{1.629264in}{2.099304in}}%
\pgfpathcurveto{\pgfqpoint{1.637500in}{2.099304in}}{\pgfqpoint{1.645400in}{2.102576in}}{\pgfqpoint{1.651224in}{2.108400in}}%
\pgfpathcurveto{\pgfqpoint{1.657048in}{2.114224in}}{\pgfqpoint{1.660320in}{2.122124in}}{\pgfqpoint{1.660320in}{2.130360in}}%
\pgfpathcurveto{\pgfqpoint{1.660320in}{2.138596in}}{\pgfqpoint{1.657048in}{2.146496in}}{\pgfqpoint{1.651224in}{2.152320in}}%
\pgfpathcurveto{\pgfqpoint{1.645400in}{2.158144in}}{\pgfqpoint{1.637500in}{2.161417in}}{\pgfqpoint{1.629264in}{2.161417in}}%
\pgfpathcurveto{\pgfqpoint{1.621028in}{2.161417in}}{\pgfqpoint{1.613127in}{2.158144in}}{\pgfqpoint{1.607304in}{2.152320in}}%
\pgfpathcurveto{\pgfqpoint{1.601480in}{2.146496in}}{\pgfqpoint{1.598207in}{2.138596in}}{\pgfqpoint{1.598207in}{2.130360in}}%
\pgfpathcurveto{\pgfqpoint{1.598207in}{2.122124in}}{\pgfqpoint{1.601480in}{2.114224in}}{\pgfqpoint{1.607304in}{2.108400in}}%
\pgfpathcurveto{\pgfqpoint{1.613127in}{2.102576in}}{\pgfqpoint{1.621028in}{2.099304in}}{\pgfqpoint{1.629264in}{2.099304in}}%
\pgfpathclose%
\pgfusepath{stroke,fill}%
\end{pgfscope}%
\begin{pgfscope}%
\pgfpathrectangle{\pgfqpoint{0.100000in}{0.212622in}}{\pgfqpoint{3.696000in}{3.696000in}}%
\pgfusepath{clip}%
\pgfsetbuttcap%
\pgfsetroundjoin%
\definecolor{currentfill}{rgb}{0.121569,0.466667,0.705882}%
\pgfsetfillcolor{currentfill}%
\pgfsetfillopacity{0.318043}%
\pgfsetlinewidth{1.003750pt}%
\definecolor{currentstroke}{rgb}{0.121569,0.466667,0.705882}%
\pgfsetstrokecolor{currentstroke}%
\pgfsetstrokeopacity{0.318043}%
\pgfsetdash{}{0pt}%
\pgfpathmoveto{\pgfqpoint{1.620560in}{2.090294in}}%
\pgfpathcurveto{\pgfqpoint{1.628796in}{2.090294in}}{\pgfqpoint{1.636696in}{2.093567in}}{\pgfqpoint{1.642520in}{2.099391in}}%
\pgfpathcurveto{\pgfqpoint{1.648344in}{2.105215in}}{\pgfqpoint{1.651616in}{2.113115in}}{\pgfqpoint{1.651616in}{2.121351in}}%
\pgfpathcurveto{\pgfqpoint{1.651616in}{2.129587in}}{\pgfqpoint{1.648344in}{2.137487in}}{\pgfqpoint{1.642520in}{2.143311in}}%
\pgfpathcurveto{\pgfqpoint{1.636696in}{2.149135in}}{\pgfqpoint{1.628796in}{2.152407in}}{\pgfqpoint{1.620560in}{2.152407in}}%
\pgfpathcurveto{\pgfqpoint{1.612324in}{2.152407in}}{\pgfqpoint{1.604424in}{2.149135in}}{\pgfqpoint{1.598600in}{2.143311in}}%
\pgfpathcurveto{\pgfqpoint{1.592776in}{2.137487in}}{\pgfqpoint{1.589503in}{2.129587in}}{\pgfqpoint{1.589503in}{2.121351in}}%
\pgfpathcurveto{\pgfqpoint{1.589503in}{2.113115in}}{\pgfqpoint{1.592776in}{2.105215in}}{\pgfqpoint{1.598600in}{2.099391in}}%
\pgfpathcurveto{\pgfqpoint{1.604424in}{2.093567in}}{\pgfqpoint{1.612324in}{2.090294in}}{\pgfqpoint{1.620560in}{2.090294in}}%
\pgfpathclose%
\pgfusepath{stroke,fill}%
\end{pgfscope}%
\begin{pgfscope}%
\pgfpathrectangle{\pgfqpoint{0.100000in}{0.212622in}}{\pgfqpoint{3.696000in}{3.696000in}}%
\pgfusepath{clip}%
\pgfsetbuttcap%
\pgfsetroundjoin%
\definecolor{currentfill}{rgb}{0.121569,0.466667,0.705882}%
\pgfsetfillcolor{currentfill}%
\pgfsetfillopacity{0.318753}%
\pgfsetlinewidth{1.003750pt}%
\definecolor{currentstroke}{rgb}{0.121569,0.466667,0.705882}%
\pgfsetstrokecolor{currentstroke}%
\pgfsetstrokeopacity{0.318753}%
\pgfsetdash{}{0pt}%
\pgfpathmoveto{\pgfqpoint{1.620726in}{2.090593in}}%
\pgfpathcurveto{\pgfqpoint{1.628962in}{2.090593in}}{\pgfqpoint{1.636863in}{2.093865in}}{\pgfqpoint{1.642686in}{2.099689in}}%
\pgfpathcurveto{\pgfqpoint{1.648510in}{2.105513in}}{\pgfqpoint{1.651783in}{2.113413in}}{\pgfqpoint{1.651783in}{2.121649in}}%
\pgfpathcurveto{\pgfqpoint{1.651783in}{2.129886in}}{\pgfqpoint{1.648510in}{2.137786in}}{\pgfqpoint{1.642686in}{2.143610in}}%
\pgfpathcurveto{\pgfqpoint{1.636863in}{2.149434in}}{\pgfqpoint{1.628962in}{2.152706in}}{\pgfqpoint{1.620726in}{2.152706in}}%
\pgfpathcurveto{\pgfqpoint{1.612490in}{2.152706in}}{\pgfqpoint{1.604590in}{2.149434in}}{\pgfqpoint{1.598766in}{2.143610in}}%
\pgfpathcurveto{\pgfqpoint{1.592942in}{2.137786in}}{\pgfqpoint{1.589670in}{2.129886in}}{\pgfqpoint{1.589670in}{2.121649in}}%
\pgfpathcurveto{\pgfqpoint{1.589670in}{2.113413in}}{\pgfqpoint{1.592942in}{2.105513in}}{\pgfqpoint{1.598766in}{2.099689in}}%
\pgfpathcurveto{\pgfqpoint{1.604590in}{2.093865in}}{\pgfqpoint{1.612490in}{2.090593in}}{\pgfqpoint{1.620726in}{2.090593in}}%
\pgfpathclose%
\pgfusepath{stroke,fill}%
\end{pgfscope}%
\begin{pgfscope}%
\pgfpathrectangle{\pgfqpoint{0.100000in}{0.212622in}}{\pgfqpoint{3.696000in}{3.696000in}}%
\pgfusepath{clip}%
\pgfsetbuttcap%
\pgfsetroundjoin%
\definecolor{currentfill}{rgb}{0.121569,0.466667,0.705882}%
\pgfsetfillcolor{currentfill}%
\pgfsetfillopacity{0.321804}%
\pgfsetlinewidth{1.003750pt}%
\definecolor{currentstroke}{rgb}{0.121569,0.466667,0.705882}%
\pgfsetstrokecolor{currentstroke}%
\pgfsetstrokeopacity{0.321804}%
\pgfsetdash{}{0pt}%
\pgfpathmoveto{\pgfqpoint{1.616670in}{2.088607in}}%
\pgfpathcurveto{\pgfqpoint{1.624906in}{2.088607in}}{\pgfqpoint{1.632806in}{2.091880in}}{\pgfqpoint{1.638630in}{2.097704in}}%
\pgfpathcurveto{\pgfqpoint{1.644454in}{2.103527in}}{\pgfqpoint{1.647726in}{2.111428in}}{\pgfqpoint{1.647726in}{2.119664in}}%
\pgfpathcurveto{\pgfqpoint{1.647726in}{2.127900in}}{\pgfqpoint{1.644454in}{2.135800in}}{\pgfqpoint{1.638630in}{2.141624in}}%
\pgfpathcurveto{\pgfqpoint{1.632806in}{2.147448in}}{\pgfqpoint{1.624906in}{2.150720in}}{\pgfqpoint{1.616670in}{2.150720in}}%
\pgfpathcurveto{\pgfqpoint{1.608433in}{2.150720in}}{\pgfqpoint{1.600533in}{2.147448in}}{\pgfqpoint{1.594710in}{2.141624in}}%
\pgfpathcurveto{\pgfqpoint{1.588886in}{2.135800in}}{\pgfqpoint{1.585613in}{2.127900in}}{\pgfqpoint{1.585613in}{2.119664in}}%
\pgfpathcurveto{\pgfqpoint{1.585613in}{2.111428in}}{\pgfqpoint{1.588886in}{2.103527in}}{\pgfqpoint{1.594710in}{2.097704in}}%
\pgfpathcurveto{\pgfqpoint{1.600533in}{2.091880in}}{\pgfqpoint{1.608433in}{2.088607in}}{\pgfqpoint{1.616670in}{2.088607in}}%
\pgfpathclose%
\pgfusepath{stroke,fill}%
\end{pgfscope}%
\begin{pgfscope}%
\pgfpathrectangle{\pgfqpoint{0.100000in}{0.212622in}}{\pgfqpoint{3.696000in}{3.696000in}}%
\pgfusepath{clip}%
\pgfsetbuttcap%
\pgfsetroundjoin%
\definecolor{currentfill}{rgb}{0.121569,0.466667,0.705882}%
\pgfsetfillcolor{currentfill}%
\pgfsetfillopacity{0.327055}%
\pgfsetlinewidth{1.003750pt}%
\definecolor{currentstroke}{rgb}{0.121569,0.466667,0.705882}%
\pgfsetstrokecolor{currentstroke}%
\pgfsetstrokeopacity{0.327055}%
\pgfsetdash{}{0pt}%
\pgfpathmoveto{\pgfqpoint{1.606738in}{2.077301in}}%
\pgfpathcurveto{\pgfqpoint{1.614974in}{2.077301in}}{\pgfqpoint{1.622874in}{2.080573in}}{\pgfqpoint{1.628698in}{2.086397in}}%
\pgfpathcurveto{\pgfqpoint{1.634522in}{2.092221in}}{\pgfqpoint{1.637794in}{2.100121in}}{\pgfqpoint{1.637794in}{2.108358in}}%
\pgfpathcurveto{\pgfqpoint{1.637794in}{2.116594in}}{\pgfqpoint{1.634522in}{2.124494in}}{\pgfqpoint{1.628698in}{2.130318in}}%
\pgfpathcurveto{\pgfqpoint{1.622874in}{2.136142in}}{\pgfqpoint{1.614974in}{2.139414in}}{\pgfqpoint{1.606738in}{2.139414in}}%
\pgfpathcurveto{\pgfqpoint{1.598501in}{2.139414in}}{\pgfqpoint{1.590601in}{2.136142in}}{\pgfqpoint{1.584777in}{2.130318in}}%
\pgfpathcurveto{\pgfqpoint{1.578953in}{2.124494in}}{\pgfqpoint{1.575681in}{2.116594in}}{\pgfqpoint{1.575681in}{2.108358in}}%
\pgfpathcurveto{\pgfqpoint{1.575681in}{2.100121in}}{\pgfqpoint{1.578953in}{2.092221in}}{\pgfqpoint{1.584777in}{2.086397in}}%
\pgfpathcurveto{\pgfqpoint{1.590601in}{2.080573in}}{\pgfqpoint{1.598501in}{2.077301in}}{\pgfqpoint{1.606738in}{2.077301in}}%
\pgfpathclose%
\pgfusepath{stroke,fill}%
\end{pgfscope}%
\begin{pgfscope}%
\pgfpathrectangle{\pgfqpoint{0.100000in}{0.212622in}}{\pgfqpoint{3.696000in}{3.696000in}}%
\pgfusepath{clip}%
\pgfsetbuttcap%
\pgfsetroundjoin%
\definecolor{currentfill}{rgb}{0.121569,0.466667,0.705882}%
\pgfsetfillcolor{currentfill}%
\pgfsetfillopacity{0.329111}%
\pgfsetlinewidth{1.003750pt}%
\definecolor{currentstroke}{rgb}{0.121569,0.466667,0.705882}%
\pgfsetstrokecolor{currentstroke}%
\pgfsetstrokeopacity{0.329111}%
\pgfsetdash{}{0pt}%
\pgfpathmoveto{\pgfqpoint{1.604192in}{2.075634in}}%
\pgfpathcurveto{\pgfqpoint{1.612428in}{2.075634in}}{\pgfqpoint{1.620328in}{2.078906in}}{\pgfqpoint{1.626152in}{2.084730in}}%
\pgfpathcurveto{\pgfqpoint{1.631976in}{2.090554in}}{\pgfqpoint{1.635248in}{2.098454in}}{\pgfqpoint{1.635248in}{2.106690in}}%
\pgfpathcurveto{\pgfqpoint{1.635248in}{2.114926in}}{\pgfqpoint{1.631976in}{2.122826in}}{\pgfqpoint{1.626152in}{2.128650in}}%
\pgfpathcurveto{\pgfqpoint{1.620328in}{2.134474in}}{\pgfqpoint{1.612428in}{2.137746in}}{\pgfqpoint{1.604192in}{2.137746in}}%
\pgfpathcurveto{\pgfqpoint{1.595955in}{2.137746in}}{\pgfqpoint{1.588055in}{2.134474in}}{\pgfqpoint{1.582231in}{2.128650in}}%
\pgfpathcurveto{\pgfqpoint{1.576408in}{2.122826in}}{\pgfqpoint{1.573135in}{2.114926in}}{\pgfqpoint{1.573135in}{2.106690in}}%
\pgfpathcurveto{\pgfqpoint{1.573135in}{2.098454in}}{\pgfqpoint{1.576408in}{2.090554in}}{\pgfqpoint{1.582231in}{2.084730in}}%
\pgfpathcurveto{\pgfqpoint{1.588055in}{2.078906in}}{\pgfqpoint{1.595955in}{2.075634in}}{\pgfqpoint{1.604192in}{2.075634in}}%
\pgfpathclose%
\pgfusepath{stroke,fill}%
\end{pgfscope}%
\begin{pgfscope}%
\pgfpathrectangle{\pgfqpoint{0.100000in}{0.212622in}}{\pgfqpoint{3.696000in}{3.696000in}}%
\pgfusepath{clip}%
\pgfsetbuttcap%
\pgfsetroundjoin%
\definecolor{currentfill}{rgb}{0.121569,0.466667,0.705882}%
\pgfsetfillcolor{currentfill}%
\pgfsetfillopacity{0.334386}%
\pgfsetlinewidth{1.003750pt}%
\definecolor{currentstroke}{rgb}{0.121569,0.466667,0.705882}%
\pgfsetstrokecolor{currentstroke}%
\pgfsetstrokeopacity{0.334386}%
\pgfsetdash{}{0pt}%
\pgfpathmoveto{\pgfqpoint{1.593683in}{2.065847in}}%
\pgfpathcurveto{\pgfqpoint{1.601919in}{2.065847in}}{\pgfqpoint{1.609819in}{2.069119in}}{\pgfqpoint{1.615643in}{2.074943in}}%
\pgfpathcurveto{\pgfqpoint{1.621467in}{2.080767in}}{\pgfqpoint{1.624739in}{2.088667in}}{\pgfqpoint{1.624739in}{2.096903in}}%
\pgfpathcurveto{\pgfqpoint{1.624739in}{2.105140in}}{\pgfqpoint{1.621467in}{2.113040in}}{\pgfqpoint{1.615643in}{2.118864in}}%
\pgfpathcurveto{\pgfqpoint{1.609819in}{2.124688in}}{\pgfqpoint{1.601919in}{2.127960in}}{\pgfqpoint{1.593683in}{2.127960in}}%
\pgfpathcurveto{\pgfqpoint{1.585446in}{2.127960in}}{\pgfqpoint{1.577546in}{2.124688in}}{\pgfqpoint{1.571722in}{2.118864in}}%
\pgfpathcurveto{\pgfqpoint{1.565898in}{2.113040in}}{\pgfqpoint{1.562626in}{2.105140in}}{\pgfqpoint{1.562626in}{2.096903in}}%
\pgfpathcurveto{\pgfqpoint{1.562626in}{2.088667in}}{\pgfqpoint{1.565898in}{2.080767in}}{\pgfqpoint{1.571722in}{2.074943in}}%
\pgfpathcurveto{\pgfqpoint{1.577546in}{2.069119in}}{\pgfqpoint{1.585446in}{2.065847in}}{\pgfqpoint{1.593683in}{2.065847in}}%
\pgfpathclose%
\pgfusepath{stroke,fill}%
\end{pgfscope}%
\begin{pgfscope}%
\pgfpathrectangle{\pgfqpoint{0.100000in}{0.212622in}}{\pgfqpoint{3.696000in}{3.696000in}}%
\pgfusepath{clip}%
\pgfsetbuttcap%
\pgfsetroundjoin%
\definecolor{currentfill}{rgb}{0.121569,0.466667,0.705882}%
\pgfsetfillcolor{currentfill}%
\pgfsetfillopacity{0.343882}%
\pgfsetlinewidth{1.003750pt}%
\definecolor{currentstroke}{rgb}{0.121569,0.466667,0.705882}%
\pgfsetstrokecolor{currentstroke}%
\pgfsetstrokeopacity{0.343882}%
\pgfsetdash{}{0pt}%
\pgfpathmoveto{\pgfqpoint{1.574016in}{2.049415in}}%
\pgfpathcurveto{\pgfqpoint{1.582252in}{2.049415in}}{\pgfqpoint{1.590152in}{2.052687in}}{\pgfqpoint{1.595976in}{2.058511in}}%
\pgfpathcurveto{\pgfqpoint{1.601800in}{2.064335in}}{\pgfqpoint{1.605072in}{2.072235in}}{\pgfqpoint{1.605072in}{2.080472in}}%
\pgfpathcurveto{\pgfqpoint{1.605072in}{2.088708in}}{\pgfqpoint{1.601800in}{2.096608in}}{\pgfqpoint{1.595976in}{2.102432in}}%
\pgfpathcurveto{\pgfqpoint{1.590152in}{2.108256in}}{\pgfqpoint{1.582252in}{2.111528in}}{\pgfqpoint{1.574016in}{2.111528in}}%
\pgfpathcurveto{\pgfqpoint{1.565779in}{2.111528in}}{\pgfqpoint{1.557879in}{2.108256in}}{\pgfqpoint{1.552055in}{2.102432in}}%
\pgfpathcurveto{\pgfqpoint{1.546231in}{2.096608in}}{\pgfqpoint{1.542959in}{2.088708in}}{\pgfqpoint{1.542959in}{2.080472in}}%
\pgfpathcurveto{\pgfqpoint{1.542959in}{2.072235in}}{\pgfqpoint{1.546231in}{2.064335in}}{\pgfqpoint{1.552055in}{2.058511in}}%
\pgfpathcurveto{\pgfqpoint{1.557879in}{2.052687in}}{\pgfqpoint{1.565779in}{2.049415in}}{\pgfqpoint{1.574016in}{2.049415in}}%
\pgfpathclose%
\pgfusepath{stroke,fill}%
\end{pgfscope}%
\begin{pgfscope}%
\pgfpathrectangle{\pgfqpoint{0.100000in}{0.212622in}}{\pgfqpoint{3.696000in}{3.696000in}}%
\pgfusepath{clip}%
\pgfsetbuttcap%
\pgfsetroundjoin%
\definecolor{currentfill}{rgb}{0.121569,0.466667,0.705882}%
\pgfsetfillcolor{currentfill}%
\pgfsetfillopacity{0.347858}%
\pgfsetlinewidth{1.003750pt}%
\definecolor{currentstroke}{rgb}{0.121569,0.466667,0.705882}%
\pgfsetstrokecolor{currentstroke}%
\pgfsetstrokeopacity{0.347858}%
\pgfsetdash{}{0pt}%
\pgfpathmoveto{\pgfqpoint{1.566120in}{2.042151in}}%
\pgfpathcurveto{\pgfqpoint{1.574356in}{2.042151in}}{\pgfqpoint{1.582256in}{2.045423in}}{\pgfqpoint{1.588080in}{2.051247in}}%
\pgfpathcurveto{\pgfqpoint{1.593904in}{2.057071in}}{\pgfqpoint{1.597176in}{2.064971in}}{\pgfqpoint{1.597176in}{2.073207in}}%
\pgfpathcurveto{\pgfqpoint{1.597176in}{2.081444in}}{\pgfqpoint{1.593904in}{2.089344in}}{\pgfqpoint{1.588080in}{2.095168in}}%
\pgfpathcurveto{\pgfqpoint{1.582256in}{2.100992in}}{\pgfqpoint{1.574356in}{2.104264in}}{\pgfqpoint{1.566120in}{2.104264in}}%
\pgfpathcurveto{\pgfqpoint{1.557883in}{2.104264in}}{\pgfqpoint{1.549983in}{2.100992in}}{\pgfqpoint{1.544159in}{2.095168in}}%
\pgfpathcurveto{\pgfqpoint{1.538336in}{2.089344in}}{\pgfqpoint{1.535063in}{2.081444in}}{\pgfqpoint{1.535063in}{2.073207in}}%
\pgfpathcurveto{\pgfqpoint{1.535063in}{2.064971in}}{\pgfqpoint{1.538336in}{2.057071in}}{\pgfqpoint{1.544159in}{2.051247in}}%
\pgfpathcurveto{\pgfqpoint{1.549983in}{2.045423in}}{\pgfqpoint{1.557883in}{2.042151in}}{\pgfqpoint{1.566120in}{2.042151in}}%
\pgfpathclose%
\pgfusepath{stroke,fill}%
\end{pgfscope}%
\begin{pgfscope}%
\pgfpathrectangle{\pgfqpoint{0.100000in}{0.212622in}}{\pgfqpoint{3.696000in}{3.696000in}}%
\pgfusepath{clip}%
\pgfsetbuttcap%
\pgfsetroundjoin%
\definecolor{currentfill}{rgb}{0.121569,0.466667,0.705882}%
\pgfsetfillcolor{currentfill}%
\pgfsetfillopacity{0.356925}%
\pgfsetlinewidth{1.003750pt}%
\definecolor{currentstroke}{rgb}{0.121569,0.466667,0.705882}%
\pgfsetstrokecolor{currentstroke}%
\pgfsetstrokeopacity{0.356925}%
\pgfsetdash{}{0pt}%
\pgfpathmoveto{\pgfqpoint{1.550367in}{2.030823in}}%
\pgfpathcurveto{\pgfqpoint{1.558603in}{2.030823in}}{\pgfqpoint{1.566503in}{2.034095in}}{\pgfqpoint{1.572327in}{2.039919in}}%
\pgfpathcurveto{\pgfqpoint{1.578151in}{2.045743in}}{\pgfqpoint{1.581423in}{2.053643in}}{\pgfqpoint{1.581423in}{2.061879in}}%
\pgfpathcurveto{\pgfqpoint{1.581423in}{2.070116in}}{\pgfqpoint{1.578151in}{2.078016in}}{\pgfqpoint{1.572327in}{2.083840in}}%
\pgfpathcurveto{\pgfqpoint{1.566503in}{2.089664in}}{\pgfqpoint{1.558603in}{2.092936in}}{\pgfqpoint{1.550367in}{2.092936in}}%
\pgfpathcurveto{\pgfqpoint{1.542131in}{2.092936in}}{\pgfqpoint{1.534231in}{2.089664in}}{\pgfqpoint{1.528407in}{2.083840in}}%
\pgfpathcurveto{\pgfqpoint{1.522583in}{2.078016in}}{\pgfqpoint{1.519310in}{2.070116in}}{\pgfqpoint{1.519310in}{2.061879in}}%
\pgfpathcurveto{\pgfqpoint{1.519310in}{2.053643in}}{\pgfqpoint{1.522583in}{2.045743in}}{\pgfqpoint{1.528407in}{2.039919in}}%
\pgfpathcurveto{\pgfqpoint{1.534231in}{2.034095in}}{\pgfqpoint{1.542131in}{2.030823in}}{\pgfqpoint{1.550367in}{2.030823in}}%
\pgfpathclose%
\pgfusepath{stroke,fill}%
\end{pgfscope}%
\begin{pgfscope}%
\pgfpathrectangle{\pgfqpoint{0.100000in}{0.212622in}}{\pgfqpoint{3.696000in}{3.696000in}}%
\pgfusepath{clip}%
\pgfsetbuttcap%
\pgfsetroundjoin%
\definecolor{currentfill}{rgb}{0.121569,0.466667,0.705882}%
\pgfsetfillcolor{currentfill}%
\pgfsetfillopacity{0.371656}%
\pgfsetlinewidth{1.003750pt}%
\definecolor{currentstroke}{rgb}{0.121569,0.466667,0.705882}%
\pgfsetstrokecolor{currentstroke}%
\pgfsetstrokeopacity{0.371656}%
\pgfsetdash{}{0pt}%
\pgfpathmoveto{\pgfqpoint{1.517928in}{1.996131in}}%
\pgfpathcurveto{\pgfqpoint{1.526164in}{1.996131in}}{\pgfqpoint{1.534064in}{1.999404in}}{\pgfqpoint{1.539888in}{2.005228in}}%
\pgfpathcurveto{\pgfqpoint{1.545712in}{2.011051in}}{\pgfqpoint{1.548984in}{2.018952in}}{\pgfqpoint{1.548984in}{2.027188in}}%
\pgfpathcurveto{\pgfqpoint{1.548984in}{2.035424in}}{\pgfqpoint{1.545712in}{2.043324in}}{\pgfqpoint{1.539888in}{2.049148in}}%
\pgfpathcurveto{\pgfqpoint{1.534064in}{2.054972in}}{\pgfqpoint{1.526164in}{2.058244in}}{\pgfqpoint{1.517928in}{2.058244in}}%
\pgfpathcurveto{\pgfqpoint{1.509691in}{2.058244in}}{\pgfqpoint{1.501791in}{2.054972in}}{\pgfqpoint{1.495967in}{2.049148in}}%
\pgfpathcurveto{\pgfqpoint{1.490143in}{2.043324in}}{\pgfqpoint{1.486871in}{2.035424in}}{\pgfqpoint{1.486871in}{2.027188in}}%
\pgfpathcurveto{\pgfqpoint{1.486871in}{2.018952in}}{\pgfqpoint{1.490143in}{2.011051in}}{\pgfqpoint{1.495967in}{2.005228in}}%
\pgfpathcurveto{\pgfqpoint{1.501791in}{1.999404in}}{\pgfqpoint{1.509691in}{1.996131in}}{\pgfqpoint{1.517928in}{1.996131in}}%
\pgfpathclose%
\pgfusepath{stroke,fill}%
\end{pgfscope}%
\begin{pgfscope}%
\pgfpathrectangle{\pgfqpoint{0.100000in}{0.212622in}}{\pgfqpoint{3.696000in}{3.696000in}}%
\pgfusepath{clip}%
\pgfsetbuttcap%
\pgfsetroundjoin%
\definecolor{currentfill}{rgb}{0.121569,0.466667,0.705882}%
\pgfsetfillcolor{currentfill}%
\pgfsetfillopacity{0.371903}%
\pgfsetlinewidth{1.003750pt}%
\definecolor{currentstroke}{rgb}{0.121569,0.466667,0.705882}%
\pgfsetstrokecolor{currentstroke}%
\pgfsetstrokeopacity{0.371903}%
\pgfsetdash{}{0pt}%
\pgfpathmoveto{\pgfqpoint{1.522284in}{2.001698in}}%
\pgfpathcurveto{\pgfqpoint{1.530520in}{2.001698in}}{\pgfqpoint{1.538420in}{2.004970in}}{\pgfqpoint{1.544244in}{2.010794in}}%
\pgfpathcurveto{\pgfqpoint{1.550068in}{2.016618in}}{\pgfqpoint{1.553340in}{2.024518in}}{\pgfqpoint{1.553340in}{2.032755in}}%
\pgfpathcurveto{\pgfqpoint{1.553340in}{2.040991in}}{\pgfqpoint{1.550068in}{2.048891in}}{\pgfqpoint{1.544244in}{2.054715in}}%
\pgfpathcurveto{\pgfqpoint{1.538420in}{2.060539in}}{\pgfqpoint{1.530520in}{2.063811in}}{\pgfqpoint{1.522284in}{2.063811in}}%
\pgfpathcurveto{\pgfqpoint{1.514047in}{2.063811in}}{\pgfqpoint{1.506147in}{2.060539in}}{\pgfqpoint{1.500323in}{2.054715in}}%
\pgfpathcurveto{\pgfqpoint{1.494499in}{2.048891in}}{\pgfqpoint{1.491227in}{2.040991in}}{\pgfqpoint{1.491227in}{2.032755in}}%
\pgfpathcurveto{\pgfqpoint{1.491227in}{2.024518in}}{\pgfqpoint{1.494499in}{2.016618in}}{\pgfqpoint{1.500323in}{2.010794in}}%
\pgfpathcurveto{\pgfqpoint{1.506147in}{2.004970in}}{\pgfqpoint{1.514047in}{2.001698in}}{\pgfqpoint{1.522284in}{2.001698in}}%
\pgfpathclose%
\pgfusepath{stroke,fill}%
\end{pgfscope}%
\begin{pgfscope}%
\pgfpathrectangle{\pgfqpoint{0.100000in}{0.212622in}}{\pgfqpoint{3.696000in}{3.696000in}}%
\pgfusepath{clip}%
\pgfsetbuttcap%
\pgfsetroundjoin%
\definecolor{currentfill}{rgb}{0.121569,0.466667,0.705882}%
\pgfsetfillcolor{currentfill}%
\pgfsetfillopacity{0.375179}%
\pgfsetlinewidth{1.003750pt}%
\definecolor{currentstroke}{rgb}{0.121569,0.466667,0.705882}%
\pgfsetstrokecolor{currentstroke}%
\pgfsetstrokeopacity{0.375179}%
\pgfsetdash{}{0pt}%
\pgfpathmoveto{\pgfqpoint{1.515546in}{1.993958in}}%
\pgfpathcurveto{\pgfqpoint{1.523782in}{1.993958in}}{\pgfqpoint{1.531682in}{1.997231in}}{\pgfqpoint{1.537506in}{2.003055in}}%
\pgfpathcurveto{\pgfqpoint{1.543330in}{2.008879in}}{\pgfqpoint{1.546603in}{2.016779in}}{\pgfqpoint{1.546603in}{2.025015in}}%
\pgfpathcurveto{\pgfqpoint{1.546603in}{2.033251in}}{\pgfqpoint{1.543330in}{2.041151in}}{\pgfqpoint{1.537506in}{2.046975in}}%
\pgfpathcurveto{\pgfqpoint{1.531682in}{2.052799in}}{\pgfqpoint{1.523782in}{2.056071in}}{\pgfqpoint{1.515546in}{2.056071in}}%
\pgfpathcurveto{\pgfqpoint{1.507310in}{2.056071in}}{\pgfqpoint{1.499410in}{2.052799in}}{\pgfqpoint{1.493586in}{2.046975in}}%
\pgfpathcurveto{\pgfqpoint{1.487762in}{2.041151in}}{\pgfqpoint{1.484490in}{2.033251in}}{\pgfqpoint{1.484490in}{2.025015in}}%
\pgfpathcurveto{\pgfqpoint{1.484490in}{2.016779in}}{\pgfqpoint{1.487762in}{2.008879in}}{\pgfqpoint{1.493586in}{2.003055in}}%
\pgfpathcurveto{\pgfqpoint{1.499410in}{1.997231in}}{\pgfqpoint{1.507310in}{1.993958in}}{\pgfqpoint{1.515546in}{1.993958in}}%
\pgfpathclose%
\pgfusepath{stroke,fill}%
\end{pgfscope}%
\begin{pgfscope}%
\pgfpathrectangle{\pgfqpoint{0.100000in}{0.212622in}}{\pgfqpoint{3.696000in}{3.696000in}}%
\pgfusepath{clip}%
\pgfsetbuttcap%
\pgfsetroundjoin%
\definecolor{currentfill}{rgb}{0.121569,0.466667,0.705882}%
\pgfsetfillcolor{currentfill}%
\pgfsetfillopacity{0.378836}%
\pgfsetlinewidth{1.003750pt}%
\definecolor{currentstroke}{rgb}{0.121569,0.466667,0.705882}%
\pgfsetstrokecolor{currentstroke}%
\pgfsetstrokeopacity{0.378836}%
\pgfsetdash{}{0pt}%
\pgfpathmoveto{\pgfqpoint{1.508887in}{1.986586in}}%
\pgfpathcurveto{\pgfqpoint{1.517124in}{1.986586in}}{\pgfqpoint{1.525024in}{1.989858in}}{\pgfqpoint{1.530848in}{1.995682in}}%
\pgfpathcurveto{\pgfqpoint{1.536672in}{2.001506in}}{\pgfqpoint{1.539944in}{2.009406in}}{\pgfqpoint{1.539944in}{2.017643in}}%
\pgfpathcurveto{\pgfqpoint{1.539944in}{2.025879in}}{\pgfqpoint{1.536672in}{2.033779in}}{\pgfqpoint{1.530848in}{2.039603in}}%
\pgfpathcurveto{\pgfqpoint{1.525024in}{2.045427in}}{\pgfqpoint{1.517124in}{2.048699in}}{\pgfqpoint{1.508887in}{2.048699in}}%
\pgfpathcurveto{\pgfqpoint{1.500651in}{2.048699in}}{\pgfqpoint{1.492751in}{2.045427in}}{\pgfqpoint{1.486927in}{2.039603in}}%
\pgfpathcurveto{\pgfqpoint{1.481103in}{2.033779in}}{\pgfqpoint{1.477831in}{2.025879in}}{\pgfqpoint{1.477831in}{2.017643in}}%
\pgfpathcurveto{\pgfqpoint{1.477831in}{2.009406in}}{\pgfqpoint{1.481103in}{2.001506in}}{\pgfqpoint{1.486927in}{1.995682in}}%
\pgfpathcurveto{\pgfqpoint{1.492751in}{1.989858in}}{\pgfqpoint{1.500651in}{1.986586in}}{\pgfqpoint{1.508887in}{1.986586in}}%
\pgfpathclose%
\pgfusepath{stroke,fill}%
\end{pgfscope}%
\begin{pgfscope}%
\pgfpathrectangle{\pgfqpoint{0.100000in}{0.212622in}}{\pgfqpoint{3.696000in}{3.696000in}}%
\pgfusepath{clip}%
\pgfsetbuttcap%
\pgfsetroundjoin%
\definecolor{currentfill}{rgb}{0.121569,0.466667,0.705882}%
\pgfsetfillcolor{currentfill}%
\pgfsetfillopacity{0.383772}%
\pgfsetlinewidth{1.003750pt}%
\definecolor{currentstroke}{rgb}{0.121569,0.466667,0.705882}%
\pgfsetstrokecolor{currentstroke}%
\pgfsetstrokeopacity{0.383772}%
\pgfsetdash{}{0pt}%
\pgfpathmoveto{\pgfqpoint{1.498329in}{1.978121in}}%
\pgfpathcurveto{\pgfqpoint{1.506566in}{1.978121in}}{\pgfqpoint{1.514466in}{1.981393in}}{\pgfqpoint{1.520290in}{1.987217in}}%
\pgfpathcurveto{\pgfqpoint{1.526114in}{1.993041in}}{\pgfqpoint{1.529386in}{2.000941in}}{\pgfqpoint{1.529386in}{2.009178in}}%
\pgfpathcurveto{\pgfqpoint{1.529386in}{2.017414in}}{\pgfqpoint{1.526114in}{2.025314in}}{\pgfqpoint{1.520290in}{2.031138in}}%
\pgfpathcurveto{\pgfqpoint{1.514466in}{2.036962in}}{\pgfqpoint{1.506566in}{2.040234in}}{\pgfqpoint{1.498329in}{2.040234in}}%
\pgfpathcurveto{\pgfqpoint{1.490093in}{2.040234in}}{\pgfqpoint{1.482193in}{2.036962in}}{\pgfqpoint{1.476369in}{2.031138in}}%
\pgfpathcurveto{\pgfqpoint{1.470545in}{2.025314in}}{\pgfqpoint{1.467273in}{2.017414in}}{\pgfqpoint{1.467273in}{2.009178in}}%
\pgfpathcurveto{\pgfqpoint{1.467273in}{2.000941in}}{\pgfqpoint{1.470545in}{1.993041in}}{\pgfqpoint{1.476369in}{1.987217in}}%
\pgfpathcurveto{\pgfqpoint{1.482193in}{1.981393in}}{\pgfqpoint{1.490093in}{1.978121in}}{\pgfqpoint{1.498329in}{1.978121in}}%
\pgfpathclose%
\pgfusepath{stroke,fill}%
\end{pgfscope}%
\begin{pgfscope}%
\pgfpathrectangle{\pgfqpoint{0.100000in}{0.212622in}}{\pgfqpoint{3.696000in}{3.696000in}}%
\pgfusepath{clip}%
\pgfsetbuttcap%
\pgfsetroundjoin%
\definecolor{currentfill}{rgb}{0.121569,0.466667,0.705882}%
\pgfsetfillcolor{currentfill}%
\pgfsetfillopacity{0.386239}%
\pgfsetlinewidth{1.003750pt}%
\definecolor{currentstroke}{rgb}{0.121569,0.466667,0.705882}%
\pgfsetstrokecolor{currentstroke}%
\pgfsetstrokeopacity{0.386239}%
\pgfsetdash{}{0pt}%
\pgfpathmoveto{\pgfqpoint{1.492793in}{1.972757in}}%
\pgfpathcurveto{\pgfqpoint{1.501029in}{1.972757in}}{\pgfqpoint{1.508929in}{1.976029in}}{\pgfqpoint{1.514753in}{1.981853in}}%
\pgfpathcurveto{\pgfqpoint{1.520577in}{1.987677in}}{\pgfqpoint{1.523849in}{1.995577in}}{\pgfqpoint{1.523849in}{2.003813in}}%
\pgfpathcurveto{\pgfqpoint{1.523849in}{2.012049in}}{\pgfqpoint{1.520577in}{2.019949in}}{\pgfqpoint{1.514753in}{2.025773in}}%
\pgfpathcurveto{\pgfqpoint{1.508929in}{2.031597in}}{\pgfqpoint{1.501029in}{2.034870in}}{\pgfqpoint{1.492793in}{2.034870in}}%
\pgfpathcurveto{\pgfqpoint{1.484557in}{2.034870in}}{\pgfqpoint{1.476657in}{2.031597in}}{\pgfqpoint{1.470833in}{2.025773in}}%
\pgfpathcurveto{\pgfqpoint{1.465009in}{2.019949in}}{\pgfqpoint{1.461736in}{2.012049in}}{\pgfqpoint{1.461736in}{2.003813in}}%
\pgfpathcurveto{\pgfqpoint{1.461736in}{1.995577in}}{\pgfqpoint{1.465009in}{1.987677in}}{\pgfqpoint{1.470833in}{1.981853in}}%
\pgfpathcurveto{\pgfqpoint{1.476657in}{1.976029in}}{\pgfqpoint{1.484557in}{1.972757in}}{\pgfqpoint{1.492793in}{1.972757in}}%
\pgfpathclose%
\pgfusepath{stroke,fill}%
\end{pgfscope}%
\begin{pgfscope}%
\pgfpathrectangle{\pgfqpoint{0.100000in}{0.212622in}}{\pgfqpoint{3.696000in}{3.696000in}}%
\pgfusepath{clip}%
\pgfsetbuttcap%
\pgfsetroundjoin%
\definecolor{currentfill}{rgb}{0.121569,0.466667,0.705882}%
\pgfsetfillcolor{currentfill}%
\pgfsetfillopacity{0.386394}%
\pgfsetlinewidth{1.003750pt}%
\definecolor{currentstroke}{rgb}{0.121569,0.466667,0.705882}%
\pgfsetstrokecolor{currentstroke}%
\pgfsetstrokeopacity{0.386394}%
\pgfsetdash{}{0pt}%
\pgfpathmoveto{\pgfqpoint{1.493099in}{1.973414in}}%
\pgfpathcurveto{\pgfqpoint{1.501335in}{1.973414in}}{\pgfqpoint{1.509235in}{1.976686in}}{\pgfqpoint{1.515059in}{1.982510in}}%
\pgfpathcurveto{\pgfqpoint{1.520883in}{1.988334in}}{\pgfqpoint{1.524155in}{1.996234in}}{\pgfqpoint{1.524155in}{2.004470in}}%
\pgfpathcurveto{\pgfqpoint{1.524155in}{2.012707in}}{\pgfqpoint{1.520883in}{2.020607in}}{\pgfqpoint{1.515059in}{2.026431in}}%
\pgfpathcurveto{\pgfqpoint{1.509235in}{2.032255in}}{\pgfqpoint{1.501335in}{2.035527in}}{\pgfqpoint{1.493099in}{2.035527in}}%
\pgfpathcurveto{\pgfqpoint{1.484863in}{2.035527in}}{\pgfqpoint{1.476963in}{2.032255in}}{\pgfqpoint{1.471139in}{2.026431in}}%
\pgfpathcurveto{\pgfqpoint{1.465315in}{2.020607in}}{\pgfqpoint{1.462042in}{2.012707in}}{\pgfqpoint{1.462042in}{2.004470in}}%
\pgfpathcurveto{\pgfqpoint{1.462042in}{1.996234in}}{\pgfqpoint{1.465315in}{1.988334in}}{\pgfqpoint{1.471139in}{1.982510in}}%
\pgfpathcurveto{\pgfqpoint{1.476963in}{1.976686in}}{\pgfqpoint{1.484863in}{1.973414in}}{\pgfqpoint{1.493099in}{1.973414in}}%
\pgfpathclose%
\pgfusepath{stroke,fill}%
\end{pgfscope}%
\begin{pgfscope}%
\pgfpathrectangle{\pgfqpoint{0.100000in}{0.212622in}}{\pgfqpoint{3.696000in}{3.696000in}}%
\pgfusepath{clip}%
\pgfsetbuttcap%
\pgfsetroundjoin%
\definecolor{currentfill}{rgb}{0.121569,0.466667,0.705882}%
\pgfsetfillcolor{currentfill}%
\pgfsetfillopacity{0.390162}%
\pgfsetlinewidth{1.003750pt}%
\definecolor{currentstroke}{rgb}{0.121569,0.466667,0.705882}%
\pgfsetstrokecolor{currentstroke}%
\pgfsetstrokeopacity{0.390162}%
\pgfsetdash{}{0pt}%
\pgfpathmoveto{\pgfqpoint{1.484773in}{1.965342in}}%
\pgfpathcurveto{\pgfqpoint{1.493009in}{1.965342in}}{\pgfqpoint{1.500909in}{1.968614in}}{\pgfqpoint{1.506733in}{1.974438in}}%
\pgfpathcurveto{\pgfqpoint{1.512557in}{1.980262in}}{\pgfqpoint{1.515829in}{1.988162in}}{\pgfqpoint{1.515829in}{1.996398in}}%
\pgfpathcurveto{\pgfqpoint{1.515829in}{2.004634in}}{\pgfqpoint{1.512557in}{2.012534in}}{\pgfqpoint{1.506733in}{2.018358in}}%
\pgfpathcurveto{\pgfqpoint{1.500909in}{2.024182in}}{\pgfqpoint{1.493009in}{2.027455in}}{\pgfqpoint{1.484773in}{2.027455in}}%
\pgfpathcurveto{\pgfqpoint{1.476536in}{2.027455in}}{\pgfqpoint{1.468636in}{2.024182in}}{\pgfqpoint{1.462812in}{2.018358in}}%
\pgfpathcurveto{\pgfqpoint{1.456988in}{2.012534in}}{\pgfqpoint{1.453716in}{2.004634in}}{\pgfqpoint{1.453716in}{1.996398in}}%
\pgfpathcurveto{\pgfqpoint{1.453716in}{1.988162in}}{\pgfqpoint{1.456988in}{1.980262in}}{\pgfqpoint{1.462812in}{1.974438in}}%
\pgfpathcurveto{\pgfqpoint{1.468636in}{1.968614in}}{\pgfqpoint{1.476536in}{1.965342in}}{\pgfqpoint{1.484773in}{1.965342in}}%
\pgfpathclose%
\pgfusepath{stroke,fill}%
\end{pgfscope}%
\begin{pgfscope}%
\pgfpathrectangle{\pgfqpoint{0.100000in}{0.212622in}}{\pgfqpoint{3.696000in}{3.696000in}}%
\pgfusepath{clip}%
\pgfsetbuttcap%
\pgfsetroundjoin%
\definecolor{currentfill}{rgb}{0.121569,0.466667,0.705882}%
\pgfsetfillcolor{currentfill}%
\pgfsetfillopacity{0.391158}%
\pgfsetlinewidth{1.003750pt}%
\definecolor{currentstroke}{rgb}{0.121569,0.466667,0.705882}%
\pgfsetstrokecolor{currentstroke}%
\pgfsetstrokeopacity{0.391158}%
\pgfsetdash{}{0pt}%
\pgfpathmoveto{\pgfqpoint{1.483444in}{1.964985in}}%
\pgfpathcurveto{\pgfqpoint{1.491680in}{1.964985in}}{\pgfqpoint{1.499580in}{1.968257in}}{\pgfqpoint{1.505404in}{1.974081in}}%
\pgfpathcurveto{\pgfqpoint{1.511228in}{1.979905in}}{\pgfqpoint{1.514501in}{1.987805in}}{\pgfqpoint{1.514501in}{1.996041in}}%
\pgfpathcurveto{\pgfqpoint{1.514501in}{2.004278in}}{\pgfqpoint{1.511228in}{2.012178in}}{\pgfqpoint{1.505404in}{2.018002in}}%
\pgfpathcurveto{\pgfqpoint{1.499580in}{2.023826in}}{\pgfqpoint{1.491680in}{2.027098in}}{\pgfqpoint{1.483444in}{2.027098in}}%
\pgfpathcurveto{\pgfqpoint{1.475208in}{2.027098in}}{\pgfqpoint{1.467308in}{2.023826in}}{\pgfqpoint{1.461484in}{2.018002in}}%
\pgfpathcurveto{\pgfqpoint{1.455660in}{2.012178in}}{\pgfqpoint{1.452388in}{2.004278in}}{\pgfqpoint{1.452388in}{1.996041in}}%
\pgfpathcurveto{\pgfqpoint{1.452388in}{1.987805in}}{\pgfqpoint{1.455660in}{1.979905in}}{\pgfqpoint{1.461484in}{1.974081in}}%
\pgfpathcurveto{\pgfqpoint{1.467308in}{1.968257in}}{\pgfqpoint{1.475208in}{1.964985in}}{\pgfqpoint{1.483444in}{1.964985in}}%
\pgfpathclose%
\pgfusepath{stroke,fill}%
\end{pgfscope}%
\begin{pgfscope}%
\pgfpathrectangle{\pgfqpoint{0.100000in}{0.212622in}}{\pgfqpoint{3.696000in}{3.696000in}}%
\pgfusepath{clip}%
\pgfsetbuttcap%
\pgfsetroundjoin%
\definecolor{currentfill}{rgb}{0.121569,0.466667,0.705882}%
\pgfsetfillcolor{currentfill}%
\pgfsetfillopacity{0.391747}%
\pgfsetlinewidth{1.003750pt}%
\definecolor{currentstroke}{rgb}{0.121569,0.466667,0.705882}%
\pgfsetstrokecolor{currentstroke}%
\pgfsetstrokeopacity{0.391747}%
\pgfsetdash{}{0pt}%
\pgfpathmoveto{\pgfqpoint{1.482455in}{1.964169in}}%
\pgfpathcurveto{\pgfqpoint{1.490691in}{1.964169in}}{\pgfqpoint{1.498591in}{1.967441in}}{\pgfqpoint{1.504415in}{1.973265in}}%
\pgfpathcurveto{\pgfqpoint{1.510239in}{1.979089in}}{\pgfqpoint{1.513512in}{1.986989in}}{\pgfqpoint{1.513512in}{1.995226in}}%
\pgfpathcurveto{\pgfqpoint{1.513512in}{2.003462in}}{\pgfqpoint{1.510239in}{2.011362in}}{\pgfqpoint{1.504415in}{2.017186in}}%
\pgfpathcurveto{\pgfqpoint{1.498591in}{2.023010in}}{\pgfqpoint{1.490691in}{2.026282in}}{\pgfqpoint{1.482455in}{2.026282in}}%
\pgfpathcurveto{\pgfqpoint{1.474219in}{2.026282in}}{\pgfqpoint{1.466319in}{2.023010in}}{\pgfqpoint{1.460495in}{2.017186in}}%
\pgfpathcurveto{\pgfqpoint{1.454671in}{2.011362in}}{\pgfqpoint{1.451399in}{2.003462in}}{\pgfqpoint{1.451399in}{1.995226in}}%
\pgfpathcurveto{\pgfqpoint{1.451399in}{1.986989in}}{\pgfqpoint{1.454671in}{1.979089in}}{\pgfqpoint{1.460495in}{1.973265in}}%
\pgfpathcurveto{\pgfqpoint{1.466319in}{1.967441in}}{\pgfqpoint{1.474219in}{1.964169in}}{\pgfqpoint{1.482455in}{1.964169in}}%
\pgfpathclose%
\pgfusepath{stroke,fill}%
\end{pgfscope}%
\begin{pgfscope}%
\pgfpathrectangle{\pgfqpoint{0.100000in}{0.212622in}}{\pgfqpoint{3.696000in}{3.696000in}}%
\pgfusepath{clip}%
\pgfsetbuttcap%
\pgfsetroundjoin%
\definecolor{currentfill}{rgb}{0.121569,0.466667,0.705882}%
\pgfsetfillcolor{currentfill}%
\pgfsetfillopacity{0.392937}%
\pgfsetlinewidth{1.003750pt}%
\definecolor{currentstroke}{rgb}{0.121569,0.466667,0.705882}%
\pgfsetstrokecolor{currentstroke}%
\pgfsetstrokeopacity{0.392937}%
\pgfsetdash{}{0pt}%
\pgfpathmoveto{\pgfqpoint{1.480729in}{1.962016in}}%
\pgfpathcurveto{\pgfqpoint{1.488965in}{1.962016in}}{\pgfqpoint{1.496865in}{1.965288in}}{\pgfqpoint{1.502689in}{1.971112in}}%
\pgfpathcurveto{\pgfqpoint{1.508513in}{1.976936in}}{\pgfqpoint{1.511785in}{1.984836in}}{\pgfqpoint{1.511785in}{1.993072in}}%
\pgfpathcurveto{\pgfqpoint{1.511785in}{2.001308in}}{\pgfqpoint{1.508513in}{2.009209in}}{\pgfqpoint{1.502689in}{2.015032in}}%
\pgfpathcurveto{\pgfqpoint{1.496865in}{2.020856in}}{\pgfqpoint{1.488965in}{2.024129in}}{\pgfqpoint{1.480729in}{2.024129in}}%
\pgfpathcurveto{\pgfqpoint{1.472492in}{2.024129in}}{\pgfqpoint{1.464592in}{2.020856in}}{\pgfqpoint{1.458768in}{2.015032in}}%
\pgfpathcurveto{\pgfqpoint{1.452944in}{2.009209in}}{\pgfqpoint{1.449672in}{2.001308in}}{\pgfqpoint{1.449672in}{1.993072in}}%
\pgfpathcurveto{\pgfqpoint{1.449672in}{1.984836in}}{\pgfqpoint{1.452944in}{1.976936in}}{\pgfqpoint{1.458768in}{1.971112in}}%
\pgfpathcurveto{\pgfqpoint{1.464592in}{1.965288in}}{\pgfqpoint{1.472492in}{1.962016in}}{\pgfqpoint{1.480729in}{1.962016in}}%
\pgfpathclose%
\pgfusepath{stroke,fill}%
\end{pgfscope}%
\begin{pgfscope}%
\pgfpathrectangle{\pgfqpoint{0.100000in}{0.212622in}}{\pgfqpoint{3.696000in}{3.696000in}}%
\pgfusepath{clip}%
\pgfsetbuttcap%
\pgfsetroundjoin%
\definecolor{currentfill}{rgb}{0.121569,0.466667,0.705882}%
\pgfsetfillcolor{currentfill}%
\pgfsetfillopacity{0.393120}%
\pgfsetlinewidth{1.003750pt}%
\definecolor{currentstroke}{rgb}{0.121569,0.466667,0.705882}%
\pgfsetstrokecolor{currentstroke}%
\pgfsetstrokeopacity{0.393120}%
\pgfsetdash{}{0pt}%
\pgfpathmoveto{\pgfqpoint{1.481402in}{1.963329in}}%
\pgfpathcurveto{\pgfqpoint{1.489638in}{1.963329in}}{\pgfqpoint{1.497538in}{1.966601in}}{\pgfqpoint{1.503362in}{1.972425in}}%
\pgfpathcurveto{\pgfqpoint{1.509186in}{1.978249in}}{\pgfqpoint{1.512458in}{1.986149in}}{\pgfqpoint{1.512458in}{1.994386in}}%
\pgfpathcurveto{\pgfqpoint{1.512458in}{2.002622in}}{\pgfqpoint{1.509186in}{2.010522in}}{\pgfqpoint{1.503362in}{2.016346in}}%
\pgfpathcurveto{\pgfqpoint{1.497538in}{2.022170in}}{\pgfqpoint{1.489638in}{2.025442in}}{\pgfqpoint{1.481402in}{2.025442in}}%
\pgfpathcurveto{\pgfqpoint{1.473165in}{2.025442in}}{\pgfqpoint{1.465265in}{2.022170in}}{\pgfqpoint{1.459441in}{2.016346in}}%
\pgfpathcurveto{\pgfqpoint{1.453617in}{2.010522in}}{\pgfqpoint{1.450345in}{2.002622in}}{\pgfqpoint{1.450345in}{1.994386in}}%
\pgfpathcurveto{\pgfqpoint{1.450345in}{1.986149in}}{\pgfqpoint{1.453617in}{1.978249in}}{\pgfqpoint{1.459441in}{1.972425in}}%
\pgfpathcurveto{\pgfqpoint{1.465265in}{1.966601in}}{\pgfqpoint{1.473165in}{1.963329in}}{\pgfqpoint{1.481402in}{1.963329in}}%
\pgfpathclose%
\pgfusepath{stroke,fill}%
\end{pgfscope}%
\begin{pgfscope}%
\pgfpathrectangle{\pgfqpoint{0.100000in}{0.212622in}}{\pgfqpoint{3.696000in}{3.696000in}}%
\pgfusepath{clip}%
\pgfsetbuttcap%
\pgfsetroundjoin%
\definecolor{currentfill}{rgb}{0.121569,0.466667,0.705882}%
\pgfsetfillcolor{currentfill}%
\pgfsetfillopacity{0.393705}%
\pgfsetlinewidth{1.003750pt}%
\definecolor{currentstroke}{rgb}{0.121569,0.466667,0.705882}%
\pgfsetstrokecolor{currentstroke}%
\pgfsetstrokeopacity{0.393705}%
\pgfsetdash{}{0pt}%
\pgfpathmoveto{\pgfqpoint{1.480218in}{1.961888in}}%
\pgfpathcurveto{\pgfqpoint{1.488455in}{1.961888in}}{\pgfqpoint{1.496355in}{1.965160in}}{\pgfqpoint{1.502179in}{1.970984in}}%
\pgfpathcurveto{\pgfqpoint{1.508003in}{1.976808in}}{\pgfqpoint{1.511275in}{1.984708in}}{\pgfqpoint{1.511275in}{1.992944in}}%
\pgfpathcurveto{\pgfqpoint{1.511275in}{2.001180in}}{\pgfqpoint{1.508003in}{2.009080in}}{\pgfqpoint{1.502179in}{2.014904in}}%
\pgfpathcurveto{\pgfqpoint{1.496355in}{2.020728in}}{\pgfqpoint{1.488455in}{2.024001in}}{\pgfqpoint{1.480218in}{2.024001in}}%
\pgfpathcurveto{\pgfqpoint{1.471982in}{2.024001in}}{\pgfqpoint{1.464082in}{2.020728in}}{\pgfqpoint{1.458258in}{2.014904in}}%
\pgfpathcurveto{\pgfqpoint{1.452434in}{2.009080in}}{\pgfqpoint{1.449162in}{2.001180in}}{\pgfqpoint{1.449162in}{1.992944in}}%
\pgfpathcurveto{\pgfqpoint{1.449162in}{1.984708in}}{\pgfqpoint{1.452434in}{1.976808in}}{\pgfqpoint{1.458258in}{1.970984in}}%
\pgfpathcurveto{\pgfqpoint{1.464082in}{1.965160in}}{\pgfqpoint{1.471982in}{1.961888in}}{\pgfqpoint{1.480218in}{1.961888in}}%
\pgfpathclose%
\pgfusepath{stroke,fill}%
\end{pgfscope}%
\begin{pgfscope}%
\pgfpathrectangle{\pgfqpoint{0.100000in}{0.212622in}}{\pgfqpoint{3.696000in}{3.696000in}}%
\pgfusepath{clip}%
\pgfsetbuttcap%
\pgfsetroundjoin%
\definecolor{currentfill}{rgb}{0.121569,0.466667,0.705882}%
\pgfsetfillcolor{currentfill}%
\pgfsetfillopacity{0.393773}%
\pgfsetlinewidth{1.003750pt}%
\definecolor{currentstroke}{rgb}{0.121569,0.466667,0.705882}%
\pgfsetstrokecolor{currentstroke}%
\pgfsetstrokeopacity{0.393773}%
\pgfsetdash{}{0pt}%
\pgfpathmoveto{\pgfqpoint{1.480372in}{1.962193in}}%
\pgfpathcurveto{\pgfqpoint{1.488608in}{1.962193in}}{\pgfqpoint{1.496509in}{1.965466in}}{\pgfqpoint{1.502332in}{1.971289in}}%
\pgfpathcurveto{\pgfqpoint{1.508156in}{1.977113in}}{\pgfqpoint{1.511429in}{1.985013in}}{\pgfqpoint{1.511429in}{1.993250in}}%
\pgfpathcurveto{\pgfqpoint{1.511429in}{2.001486in}}{\pgfqpoint{1.508156in}{2.009386in}}{\pgfqpoint{1.502332in}{2.015210in}}%
\pgfpathcurveto{\pgfqpoint{1.496509in}{2.021034in}}{\pgfqpoint{1.488608in}{2.024306in}}{\pgfqpoint{1.480372in}{2.024306in}}%
\pgfpathcurveto{\pgfqpoint{1.472136in}{2.024306in}}{\pgfqpoint{1.464236in}{2.021034in}}{\pgfqpoint{1.458412in}{2.015210in}}%
\pgfpathcurveto{\pgfqpoint{1.452588in}{2.009386in}}{\pgfqpoint{1.449316in}{2.001486in}}{\pgfqpoint{1.449316in}{1.993250in}}%
\pgfpathcurveto{\pgfqpoint{1.449316in}{1.985013in}}{\pgfqpoint{1.452588in}{1.977113in}}{\pgfqpoint{1.458412in}{1.971289in}}%
\pgfpathcurveto{\pgfqpoint{1.464236in}{1.965466in}}{\pgfqpoint{1.472136in}{1.962193in}}{\pgfqpoint{1.480372in}{1.962193in}}%
\pgfpathclose%
\pgfusepath{stroke,fill}%
\end{pgfscope}%
\begin{pgfscope}%
\pgfpathrectangle{\pgfqpoint{0.100000in}{0.212622in}}{\pgfqpoint{3.696000in}{3.696000in}}%
\pgfusepath{clip}%
\pgfsetbuttcap%
\pgfsetroundjoin%
\definecolor{currentfill}{rgb}{0.121569,0.466667,0.705882}%
\pgfsetfillcolor{currentfill}%
\pgfsetfillopacity{0.400056}%
\pgfsetlinewidth{1.003750pt}%
\definecolor{currentstroke}{rgb}{0.121569,0.466667,0.705882}%
\pgfsetstrokecolor{currentstroke}%
\pgfsetstrokeopacity{0.400056}%
\pgfsetdash{}{0pt}%
\pgfpathmoveto{\pgfqpoint{1.474433in}{1.960947in}}%
\pgfpathcurveto{\pgfqpoint{1.482669in}{1.960947in}}{\pgfqpoint{1.490569in}{1.964220in}}{\pgfqpoint{1.496393in}{1.970043in}}%
\pgfpathcurveto{\pgfqpoint{1.502217in}{1.975867in}}{\pgfqpoint{1.505489in}{1.983767in}}{\pgfqpoint{1.505489in}{1.992004in}}%
\pgfpathcurveto{\pgfqpoint{1.505489in}{2.000240in}}{\pgfqpoint{1.502217in}{2.008140in}}{\pgfqpoint{1.496393in}{2.013964in}}%
\pgfpathcurveto{\pgfqpoint{1.490569in}{2.019788in}}{\pgfqpoint{1.482669in}{2.023060in}}{\pgfqpoint{1.474433in}{2.023060in}}%
\pgfpathcurveto{\pgfqpoint{1.466196in}{2.023060in}}{\pgfqpoint{1.458296in}{2.019788in}}{\pgfqpoint{1.452472in}{2.013964in}}%
\pgfpathcurveto{\pgfqpoint{1.446649in}{2.008140in}}{\pgfqpoint{1.443376in}{2.000240in}}{\pgfqpoint{1.443376in}{1.992004in}}%
\pgfpathcurveto{\pgfqpoint{1.443376in}{1.983767in}}{\pgfqpoint{1.446649in}{1.975867in}}{\pgfqpoint{1.452472in}{1.970043in}}%
\pgfpathcurveto{\pgfqpoint{1.458296in}{1.964220in}}{\pgfqpoint{1.466196in}{1.960947in}}{\pgfqpoint{1.474433in}{1.960947in}}%
\pgfpathclose%
\pgfusepath{stroke,fill}%
\end{pgfscope}%
\begin{pgfscope}%
\pgfpathrectangle{\pgfqpoint{0.100000in}{0.212622in}}{\pgfqpoint{3.696000in}{3.696000in}}%
\pgfusepath{clip}%
\pgfsetbuttcap%
\pgfsetroundjoin%
\definecolor{currentfill}{rgb}{0.121569,0.466667,0.705882}%
\pgfsetfillcolor{currentfill}%
\pgfsetfillopacity{0.401131}%
\pgfsetlinewidth{1.003750pt}%
\definecolor{currentstroke}{rgb}{0.121569,0.466667,0.705882}%
\pgfsetstrokecolor{currentstroke}%
\pgfsetstrokeopacity{0.401131}%
\pgfsetdash{}{0pt}%
\pgfpathmoveto{\pgfqpoint{1.464949in}{1.947601in}}%
\pgfpathcurveto{\pgfqpoint{1.473186in}{1.947601in}}{\pgfqpoint{1.481086in}{1.950873in}}{\pgfqpoint{1.486910in}{1.956697in}}%
\pgfpathcurveto{\pgfqpoint{1.492733in}{1.962521in}}{\pgfqpoint{1.496006in}{1.970421in}}{\pgfqpoint{1.496006in}{1.978657in}}%
\pgfpathcurveto{\pgfqpoint{1.496006in}{1.986893in}}{\pgfqpoint{1.492733in}{1.994793in}}{\pgfqpoint{1.486910in}{2.000617in}}%
\pgfpathcurveto{\pgfqpoint{1.481086in}{2.006441in}}{\pgfqpoint{1.473186in}{2.009714in}}{\pgfqpoint{1.464949in}{2.009714in}}%
\pgfpathcurveto{\pgfqpoint{1.456713in}{2.009714in}}{\pgfqpoint{1.448813in}{2.006441in}}{\pgfqpoint{1.442989in}{2.000617in}}%
\pgfpathcurveto{\pgfqpoint{1.437165in}{1.994793in}}{\pgfqpoint{1.433893in}{1.986893in}}{\pgfqpoint{1.433893in}{1.978657in}}%
\pgfpathcurveto{\pgfqpoint{1.433893in}{1.970421in}}{\pgfqpoint{1.437165in}{1.962521in}}{\pgfqpoint{1.442989in}{1.956697in}}%
\pgfpathcurveto{\pgfqpoint{1.448813in}{1.950873in}}{\pgfqpoint{1.456713in}{1.947601in}}{\pgfqpoint{1.464949in}{1.947601in}}%
\pgfpathclose%
\pgfusepath{stroke,fill}%
\end{pgfscope}%
\begin{pgfscope}%
\pgfpathrectangle{\pgfqpoint{0.100000in}{0.212622in}}{\pgfqpoint{3.696000in}{3.696000in}}%
\pgfusepath{clip}%
\pgfsetbuttcap%
\pgfsetroundjoin%
\definecolor{currentfill}{rgb}{0.121569,0.466667,0.705882}%
\pgfsetfillcolor{currentfill}%
\pgfsetfillopacity{0.404943}%
\pgfsetlinewidth{1.003750pt}%
\definecolor{currentstroke}{rgb}{0.121569,0.466667,0.705882}%
\pgfsetstrokecolor{currentstroke}%
\pgfsetstrokeopacity{0.404943}%
\pgfsetdash{}{0pt}%
\pgfpathmoveto{\pgfqpoint{1.469423in}{1.952945in}}%
\pgfpathcurveto{\pgfqpoint{1.477659in}{1.952945in}}{\pgfqpoint{1.485559in}{1.956217in}}{\pgfqpoint{1.491383in}{1.962041in}}%
\pgfpathcurveto{\pgfqpoint{1.497207in}{1.967865in}}{\pgfqpoint{1.500479in}{1.975765in}}{\pgfqpoint{1.500479in}{1.984001in}}%
\pgfpathcurveto{\pgfqpoint{1.500479in}{1.992238in}}{\pgfqpoint{1.497207in}{2.000138in}}{\pgfqpoint{1.491383in}{2.005962in}}%
\pgfpathcurveto{\pgfqpoint{1.485559in}{2.011785in}}{\pgfqpoint{1.477659in}{2.015058in}}{\pgfqpoint{1.469423in}{2.015058in}}%
\pgfpathcurveto{\pgfqpoint{1.461186in}{2.015058in}}{\pgfqpoint{1.453286in}{2.011785in}}{\pgfqpoint{1.447462in}{2.005962in}}%
\pgfpathcurveto{\pgfqpoint{1.441639in}{2.000138in}}{\pgfqpoint{1.438366in}{1.992238in}}{\pgfqpoint{1.438366in}{1.984001in}}%
\pgfpathcurveto{\pgfqpoint{1.438366in}{1.975765in}}{\pgfqpoint{1.441639in}{1.967865in}}{\pgfqpoint{1.447462in}{1.962041in}}%
\pgfpathcurveto{\pgfqpoint{1.453286in}{1.956217in}}{\pgfqpoint{1.461186in}{1.952945in}}{\pgfqpoint{1.469423in}{1.952945in}}%
\pgfpathclose%
\pgfusepath{stroke,fill}%
\end{pgfscope}%
\begin{pgfscope}%
\pgfpathrectangle{\pgfqpoint{0.100000in}{0.212622in}}{\pgfqpoint{3.696000in}{3.696000in}}%
\pgfusepath{clip}%
\pgfsetbuttcap%
\pgfsetroundjoin%
\definecolor{currentfill}{rgb}{0.121569,0.466667,0.705882}%
\pgfsetfillcolor{currentfill}%
\pgfsetfillopacity{0.405345}%
\pgfsetlinewidth{1.003750pt}%
\definecolor{currentstroke}{rgb}{0.121569,0.466667,0.705882}%
\pgfsetstrokecolor{currentstroke}%
\pgfsetstrokeopacity{0.405345}%
\pgfsetdash{}{0pt}%
\pgfpathmoveto{\pgfqpoint{1.465579in}{1.950438in}}%
\pgfpathcurveto{\pgfqpoint{1.473816in}{1.950438in}}{\pgfqpoint{1.481716in}{1.953710in}}{\pgfqpoint{1.487540in}{1.959534in}}%
\pgfpathcurveto{\pgfqpoint{1.493364in}{1.965358in}}{\pgfqpoint{1.496636in}{1.973258in}}{\pgfqpoint{1.496636in}{1.981494in}}%
\pgfpathcurveto{\pgfqpoint{1.496636in}{1.989731in}}{\pgfqpoint{1.493364in}{1.997631in}}{\pgfqpoint{1.487540in}{2.003455in}}%
\pgfpathcurveto{\pgfqpoint{1.481716in}{2.009278in}}{\pgfqpoint{1.473816in}{2.012551in}}{\pgfqpoint{1.465579in}{2.012551in}}%
\pgfpathcurveto{\pgfqpoint{1.457343in}{2.012551in}}{\pgfqpoint{1.449443in}{2.009278in}}{\pgfqpoint{1.443619in}{2.003455in}}%
\pgfpathcurveto{\pgfqpoint{1.437795in}{1.997631in}}{\pgfqpoint{1.434523in}{1.989731in}}{\pgfqpoint{1.434523in}{1.981494in}}%
\pgfpathcurveto{\pgfqpoint{1.434523in}{1.973258in}}{\pgfqpoint{1.437795in}{1.965358in}}{\pgfqpoint{1.443619in}{1.959534in}}%
\pgfpathcurveto{\pgfqpoint{1.449443in}{1.953710in}}{\pgfqpoint{1.457343in}{1.950438in}}{\pgfqpoint{1.465579in}{1.950438in}}%
\pgfpathclose%
\pgfusepath{stroke,fill}%
\end{pgfscope}%
\begin{pgfscope}%
\pgfpathrectangle{\pgfqpoint{0.100000in}{0.212622in}}{\pgfqpoint{3.696000in}{3.696000in}}%
\pgfusepath{clip}%
\pgfsetbuttcap%
\pgfsetroundjoin%
\definecolor{currentfill}{rgb}{0.121569,0.466667,0.705882}%
\pgfsetfillcolor{currentfill}%
\pgfsetfillopacity{0.409224}%
\pgfsetlinewidth{1.003750pt}%
\definecolor{currentstroke}{rgb}{0.121569,0.466667,0.705882}%
\pgfsetstrokecolor{currentstroke}%
\pgfsetstrokeopacity{0.409224}%
\pgfsetdash{}{0pt}%
\pgfpathmoveto{\pgfqpoint{1.462113in}{1.944829in}}%
\pgfpathcurveto{\pgfqpoint{1.470349in}{1.944829in}}{\pgfqpoint{1.478249in}{1.948101in}}{\pgfqpoint{1.484073in}{1.953925in}}%
\pgfpathcurveto{\pgfqpoint{1.489897in}{1.959749in}}{\pgfqpoint{1.493169in}{1.967649in}}{\pgfqpoint{1.493169in}{1.975886in}}%
\pgfpathcurveto{\pgfqpoint{1.493169in}{1.984122in}}{\pgfqpoint{1.489897in}{1.992022in}}{\pgfqpoint{1.484073in}{1.997846in}}%
\pgfpathcurveto{\pgfqpoint{1.478249in}{2.003670in}}{\pgfqpoint{1.470349in}{2.006942in}}{\pgfqpoint{1.462113in}{2.006942in}}%
\pgfpathcurveto{\pgfqpoint{1.453877in}{2.006942in}}{\pgfqpoint{1.445977in}{2.003670in}}{\pgfqpoint{1.440153in}{1.997846in}}%
\pgfpathcurveto{\pgfqpoint{1.434329in}{1.992022in}}{\pgfqpoint{1.431056in}{1.984122in}}{\pgfqpoint{1.431056in}{1.975886in}}%
\pgfpathcurveto{\pgfqpoint{1.431056in}{1.967649in}}{\pgfqpoint{1.434329in}{1.959749in}}{\pgfqpoint{1.440153in}{1.953925in}}%
\pgfpathcurveto{\pgfqpoint{1.445977in}{1.948101in}}{\pgfqpoint{1.453877in}{1.944829in}}{\pgfqpoint{1.462113in}{1.944829in}}%
\pgfpathclose%
\pgfusepath{stroke,fill}%
\end{pgfscope}%
\begin{pgfscope}%
\pgfpathrectangle{\pgfqpoint{0.100000in}{0.212622in}}{\pgfqpoint{3.696000in}{3.696000in}}%
\pgfusepath{clip}%
\pgfsetbuttcap%
\pgfsetroundjoin%
\definecolor{currentfill}{rgb}{0.121569,0.466667,0.705882}%
\pgfsetfillcolor{currentfill}%
\pgfsetfillopacity{0.411004}%
\pgfsetlinewidth{1.003750pt}%
\definecolor{currentstroke}{rgb}{0.121569,0.466667,0.705882}%
\pgfsetstrokecolor{currentstroke}%
\pgfsetstrokeopacity{0.411004}%
\pgfsetdash{}{0pt}%
\pgfpathmoveto{\pgfqpoint{1.460021in}{1.943232in}}%
\pgfpathcurveto{\pgfqpoint{1.468258in}{1.943232in}}{\pgfqpoint{1.476158in}{1.946504in}}{\pgfqpoint{1.481982in}{1.952328in}}%
\pgfpathcurveto{\pgfqpoint{1.487806in}{1.958152in}}{\pgfqpoint{1.491078in}{1.966052in}}{\pgfqpoint{1.491078in}{1.974288in}}%
\pgfpathcurveto{\pgfqpoint{1.491078in}{1.982524in}}{\pgfqpoint{1.487806in}{1.990424in}}{\pgfqpoint{1.481982in}{1.996248in}}%
\pgfpathcurveto{\pgfqpoint{1.476158in}{2.002072in}}{\pgfqpoint{1.468258in}{2.005345in}}{\pgfqpoint{1.460021in}{2.005345in}}%
\pgfpathcurveto{\pgfqpoint{1.451785in}{2.005345in}}{\pgfqpoint{1.443885in}{2.002072in}}{\pgfqpoint{1.438061in}{1.996248in}}%
\pgfpathcurveto{\pgfqpoint{1.432237in}{1.990424in}}{\pgfqpoint{1.428965in}{1.982524in}}{\pgfqpoint{1.428965in}{1.974288in}}%
\pgfpathcurveto{\pgfqpoint{1.428965in}{1.966052in}}{\pgfqpoint{1.432237in}{1.958152in}}{\pgfqpoint{1.438061in}{1.952328in}}%
\pgfpathcurveto{\pgfqpoint{1.443885in}{1.946504in}}{\pgfqpoint{1.451785in}{1.943232in}}{\pgfqpoint{1.460021in}{1.943232in}}%
\pgfpathclose%
\pgfusepath{stroke,fill}%
\end{pgfscope}%
\begin{pgfscope}%
\pgfpathrectangle{\pgfqpoint{0.100000in}{0.212622in}}{\pgfqpoint{3.696000in}{3.696000in}}%
\pgfusepath{clip}%
\pgfsetbuttcap%
\pgfsetroundjoin%
\definecolor{currentfill}{rgb}{0.121569,0.466667,0.705882}%
\pgfsetfillcolor{currentfill}%
\pgfsetfillopacity{0.414156}%
\pgfsetlinewidth{1.003750pt}%
\definecolor{currentstroke}{rgb}{0.121569,0.466667,0.705882}%
\pgfsetstrokecolor{currentstroke}%
\pgfsetstrokeopacity{0.414156}%
\pgfsetdash{}{0pt}%
\pgfpathmoveto{\pgfqpoint{1.455510in}{1.939296in}}%
\pgfpathcurveto{\pgfqpoint{1.463746in}{1.939296in}}{\pgfqpoint{1.471646in}{1.942569in}}{\pgfqpoint{1.477470in}{1.948393in}}%
\pgfpathcurveto{\pgfqpoint{1.483294in}{1.954217in}}{\pgfqpoint{1.486567in}{1.962117in}}{\pgfqpoint{1.486567in}{1.970353in}}%
\pgfpathcurveto{\pgfqpoint{1.486567in}{1.978589in}}{\pgfqpoint{1.483294in}{1.986489in}}{\pgfqpoint{1.477470in}{1.992313in}}%
\pgfpathcurveto{\pgfqpoint{1.471646in}{1.998137in}}{\pgfqpoint{1.463746in}{2.001409in}}{\pgfqpoint{1.455510in}{2.001409in}}%
\pgfpathcurveto{\pgfqpoint{1.447274in}{2.001409in}}{\pgfqpoint{1.439374in}{1.998137in}}{\pgfqpoint{1.433550in}{1.992313in}}%
\pgfpathcurveto{\pgfqpoint{1.427726in}{1.986489in}}{\pgfqpoint{1.424454in}{1.978589in}}{\pgfqpoint{1.424454in}{1.970353in}}%
\pgfpathcurveto{\pgfqpoint{1.424454in}{1.962117in}}{\pgfqpoint{1.427726in}{1.954217in}}{\pgfqpoint{1.433550in}{1.948393in}}%
\pgfpathcurveto{\pgfqpoint{1.439374in}{1.942569in}}{\pgfqpoint{1.447274in}{1.939296in}}{\pgfqpoint{1.455510in}{1.939296in}}%
\pgfpathclose%
\pgfusepath{stroke,fill}%
\end{pgfscope}%
\begin{pgfscope}%
\pgfpathrectangle{\pgfqpoint{0.100000in}{0.212622in}}{\pgfqpoint{3.696000in}{3.696000in}}%
\pgfusepath{clip}%
\pgfsetbuttcap%
\pgfsetroundjoin%
\definecolor{currentfill}{rgb}{0.121569,0.466667,0.705882}%
\pgfsetfillcolor{currentfill}%
\pgfsetfillopacity{0.415765}%
\pgfsetlinewidth{1.003750pt}%
\definecolor{currentstroke}{rgb}{0.121569,0.466667,0.705882}%
\pgfsetstrokecolor{currentstroke}%
\pgfsetstrokeopacity{0.415765}%
\pgfsetdash{}{0pt}%
\pgfpathmoveto{\pgfqpoint{1.457308in}{1.944584in}}%
\pgfpathcurveto{\pgfqpoint{1.465545in}{1.944584in}}{\pgfqpoint{1.473445in}{1.947856in}}{\pgfqpoint{1.479269in}{1.953680in}}%
\pgfpathcurveto{\pgfqpoint{1.485093in}{1.959504in}}{\pgfqpoint{1.488365in}{1.967404in}}{\pgfqpoint{1.488365in}{1.975641in}}%
\pgfpathcurveto{\pgfqpoint{1.488365in}{1.983877in}}{\pgfqpoint{1.485093in}{1.991777in}}{\pgfqpoint{1.479269in}{1.997601in}}%
\pgfpathcurveto{\pgfqpoint{1.473445in}{2.003425in}}{\pgfqpoint{1.465545in}{2.006697in}}{\pgfqpoint{1.457308in}{2.006697in}}%
\pgfpathcurveto{\pgfqpoint{1.449072in}{2.006697in}}{\pgfqpoint{1.441172in}{2.003425in}}{\pgfqpoint{1.435348in}{1.997601in}}%
\pgfpathcurveto{\pgfqpoint{1.429524in}{1.991777in}}{\pgfqpoint{1.426252in}{1.983877in}}{\pgfqpoint{1.426252in}{1.975641in}}%
\pgfpathcurveto{\pgfqpoint{1.426252in}{1.967404in}}{\pgfqpoint{1.429524in}{1.959504in}}{\pgfqpoint{1.435348in}{1.953680in}}%
\pgfpathcurveto{\pgfqpoint{1.441172in}{1.947856in}}{\pgfqpoint{1.449072in}{1.944584in}}{\pgfqpoint{1.457308in}{1.944584in}}%
\pgfpathclose%
\pgfusepath{stroke,fill}%
\end{pgfscope}%
\begin{pgfscope}%
\pgfpathrectangle{\pgfqpoint{0.100000in}{0.212622in}}{\pgfqpoint{3.696000in}{3.696000in}}%
\pgfusepath{clip}%
\pgfsetbuttcap%
\pgfsetroundjoin%
\definecolor{currentfill}{rgb}{0.121569,0.466667,0.705882}%
\pgfsetfillcolor{currentfill}%
\pgfsetfillopacity{0.417676}%
\pgfsetlinewidth{1.003750pt}%
\definecolor{currentstroke}{rgb}{0.121569,0.466667,0.705882}%
\pgfsetstrokecolor{currentstroke}%
\pgfsetstrokeopacity{0.417676}%
\pgfsetdash{}{0pt}%
\pgfpathmoveto{\pgfqpoint{1.455020in}{1.942913in}}%
\pgfpathcurveto{\pgfqpoint{1.463257in}{1.942913in}}{\pgfqpoint{1.471157in}{1.946185in}}{\pgfqpoint{1.476981in}{1.952009in}}%
\pgfpathcurveto{\pgfqpoint{1.482804in}{1.957833in}}{\pgfqpoint{1.486077in}{1.965733in}}{\pgfqpoint{1.486077in}{1.973969in}}%
\pgfpathcurveto{\pgfqpoint{1.486077in}{1.982205in}}{\pgfqpoint{1.482804in}{1.990105in}}{\pgfqpoint{1.476981in}{1.995929in}}%
\pgfpathcurveto{\pgfqpoint{1.471157in}{2.001753in}}{\pgfqpoint{1.463257in}{2.005026in}}{\pgfqpoint{1.455020in}{2.005026in}}%
\pgfpathcurveto{\pgfqpoint{1.446784in}{2.005026in}}{\pgfqpoint{1.438884in}{2.001753in}}{\pgfqpoint{1.433060in}{1.995929in}}%
\pgfpathcurveto{\pgfqpoint{1.427236in}{1.990105in}}{\pgfqpoint{1.423964in}{1.982205in}}{\pgfqpoint{1.423964in}{1.973969in}}%
\pgfpathcurveto{\pgfqpoint{1.423964in}{1.965733in}}{\pgfqpoint{1.427236in}{1.957833in}}{\pgfqpoint{1.433060in}{1.952009in}}%
\pgfpathcurveto{\pgfqpoint{1.438884in}{1.946185in}}{\pgfqpoint{1.446784in}{1.942913in}}{\pgfqpoint{1.455020in}{1.942913in}}%
\pgfpathclose%
\pgfusepath{stroke,fill}%
\end{pgfscope}%
\begin{pgfscope}%
\pgfpathrectangle{\pgfqpoint{0.100000in}{0.212622in}}{\pgfqpoint{3.696000in}{3.696000in}}%
\pgfusepath{clip}%
\pgfsetbuttcap%
\pgfsetroundjoin%
\definecolor{currentfill}{rgb}{0.121569,0.466667,0.705882}%
\pgfsetfillcolor{currentfill}%
\pgfsetfillopacity{0.419382}%
\pgfsetlinewidth{1.003750pt}%
\definecolor{currentstroke}{rgb}{0.121569,0.466667,0.705882}%
\pgfsetstrokecolor{currentstroke}%
\pgfsetstrokeopacity{0.419382}%
\pgfsetdash{}{0pt}%
\pgfpathmoveto{\pgfqpoint{1.452100in}{1.940575in}}%
\pgfpathcurveto{\pgfqpoint{1.460337in}{1.940575in}}{\pgfqpoint{1.468237in}{1.943847in}}{\pgfqpoint{1.474061in}{1.949671in}}%
\pgfpathcurveto{\pgfqpoint{1.479884in}{1.955495in}}{\pgfqpoint{1.483157in}{1.963395in}}{\pgfqpoint{1.483157in}{1.971631in}}%
\pgfpathcurveto{\pgfqpoint{1.483157in}{1.979867in}}{\pgfqpoint{1.479884in}{1.987767in}}{\pgfqpoint{1.474061in}{1.993591in}}%
\pgfpathcurveto{\pgfqpoint{1.468237in}{1.999415in}}{\pgfqpoint{1.460337in}{2.002688in}}{\pgfqpoint{1.452100in}{2.002688in}}%
\pgfpathcurveto{\pgfqpoint{1.443864in}{2.002688in}}{\pgfqpoint{1.435964in}{1.999415in}}{\pgfqpoint{1.430140in}{1.993591in}}%
\pgfpathcurveto{\pgfqpoint{1.424316in}{1.987767in}}{\pgfqpoint{1.421044in}{1.979867in}}{\pgfqpoint{1.421044in}{1.971631in}}%
\pgfpathcurveto{\pgfqpoint{1.421044in}{1.963395in}}{\pgfqpoint{1.424316in}{1.955495in}}{\pgfqpoint{1.430140in}{1.949671in}}%
\pgfpathcurveto{\pgfqpoint{1.435964in}{1.943847in}}{\pgfqpoint{1.443864in}{1.940575in}}{\pgfqpoint{1.452100in}{1.940575in}}%
\pgfpathclose%
\pgfusepath{stroke,fill}%
\end{pgfscope}%
\begin{pgfscope}%
\pgfpathrectangle{\pgfqpoint{0.100000in}{0.212622in}}{\pgfqpoint{3.696000in}{3.696000in}}%
\pgfusepath{clip}%
\pgfsetbuttcap%
\pgfsetroundjoin%
\definecolor{currentfill}{rgb}{0.121569,0.466667,0.705882}%
\pgfsetfillcolor{currentfill}%
\pgfsetfillopacity{0.420421}%
\pgfsetlinewidth{1.003750pt}%
\definecolor{currentstroke}{rgb}{0.121569,0.466667,0.705882}%
\pgfsetstrokecolor{currentstroke}%
\pgfsetstrokeopacity{0.420421}%
\pgfsetdash{}{0pt}%
\pgfpathmoveto{\pgfqpoint{1.454734in}{1.950404in}}%
\pgfpathcurveto{\pgfqpoint{1.462970in}{1.950404in}}{\pgfqpoint{1.470870in}{1.953676in}}{\pgfqpoint{1.476694in}{1.959500in}}%
\pgfpathcurveto{\pgfqpoint{1.482518in}{1.965324in}}{\pgfqpoint{1.485790in}{1.973224in}}{\pgfqpoint{1.485790in}{1.981460in}}%
\pgfpathcurveto{\pgfqpoint{1.485790in}{1.989697in}}{\pgfqpoint{1.482518in}{1.997597in}}{\pgfqpoint{1.476694in}{2.003421in}}%
\pgfpathcurveto{\pgfqpoint{1.470870in}{2.009244in}}{\pgfqpoint{1.462970in}{2.012517in}}{\pgfqpoint{1.454734in}{2.012517in}}%
\pgfpathcurveto{\pgfqpoint{1.446498in}{2.012517in}}{\pgfqpoint{1.438598in}{2.009244in}}{\pgfqpoint{1.432774in}{2.003421in}}%
\pgfpathcurveto{\pgfqpoint{1.426950in}{1.997597in}}{\pgfqpoint{1.423677in}{1.989697in}}{\pgfqpoint{1.423677in}{1.981460in}}%
\pgfpathcurveto{\pgfqpoint{1.423677in}{1.973224in}}{\pgfqpoint{1.426950in}{1.965324in}}{\pgfqpoint{1.432774in}{1.959500in}}%
\pgfpathcurveto{\pgfqpoint{1.438598in}{1.953676in}}{\pgfqpoint{1.446498in}{1.950404in}}{\pgfqpoint{1.454734in}{1.950404in}}%
\pgfpathclose%
\pgfusepath{stroke,fill}%
\end{pgfscope}%
\begin{pgfscope}%
\pgfpathrectangle{\pgfqpoint{0.100000in}{0.212622in}}{\pgfqpoint{3.696000in}{3.696000in}}%
\pgfusepath{clip}%
\pgfsetbuttcap%
\pgfsetroundjoin%
\definecolor{currentfill}{rgb}{0.121569,0.466667,0.705882}%
\pgfsetfillcolor{currentfill}%
\pgfsetfillopacity{0.424248}%
\pgfsetlinewidth{1.003750pt}%
\definecolor{currentstroke}{rgb}{0.121569,0.466667,0.705882}%
\pgfsetstrokecolor{currentstroke}%
\pgfsetstrokeopacity{0.424248}%
\pgfsetdash{}{0pt}%
\pgfpathmoveto{\pgfqpoint{1.446351in}{1.943357in}}%
\pgfpathcurveto{\pgfqpoint{1.454587in}{1.943357in}}{\pgfqpoint{1.462487in}{1.946629in}}{\pgfqpoint{1.468311in}{1.952453in}}%
\pgfpathcurveto{\pgfqpoint{1.474135in}{1.958277in}}{\pgfqpoint{1.477407in}{1.966177in}}{\pgfqpoint{1.477407in}{1.974413in}}%
\pgfpathcurveto{\pgfqpoint{1.477407in}{1.982650in}}{\pgfqpoint{1.474135in}{1.990550in}}{\pgfqpoint{1.468311in}{1.996374in}}%
\pgfpathcurveto{\pgfqpoint{1.462487in}{2.002198in}}{\pgfqpoint{1.454587in}{2.005470in}}{\pgfqpoint{1.446351in}{2.005470in}}%
\pgfpathcurveto{\pgfqpoint{1.438114in}{2.005470in}}{\pgfqpoint{1.430214in}{2.002198in}}{\pgfqpoint{1.424390in}{1.996374in}}%
\pgfpathcurveto{\pgfqpoint{1.418566in}{1.990550in}}{\pgfqpoint{1.415294in}{1.982650in}}{\pgfqpoint{1.415294in}{1.974413in}}%
\pgfpathcurveto{\pgfqpoint{1.415294in}{1.966177in}}{\pgfqpoint{1.418566in}{1.958277in}}{\pgfqpoint{1.424390in}{1.952453in}}%
\pgfpathcurveto{\pgfqpoint{1.430214in}{1.946629in}}{\pgfqpoint{1.438114in}{1.943357in}}{\pgfqpoint{1.446351in}{1.943357in}}%
\pgfpathclose%
\pgfusepath{stroke,fill}%
\end{pgfscope}%
\begin{pgfscope}%
\pgfpathrectangle{\pgfqpoint{0.100000in}{0.212622in}}{\pgfqpoint{3.696000in}{3.696000in}}%
\pgfusepath{clip}%
\pgfsetbuttcap%
\pgfsetroundjoin%
\definecolor{currentfill}{rgb}{0.121569,0.466667,0.705882}%
\pgfsetfillcolor{currentfill}%
\pgfsetfillopacity{0.424930}%
\pgfsetlinewidth{1.003750pt}%
\definecolor{currentstroke}{rgb}{0.121569,0.466667,0.705882}%
\pgfsetstrokecolor{currentstroke}%
\pgfsetstrokeopacity{0.424930}%
\pgfsetdash{}{0pt}%
\pgfpathmoveto{\pgfqpoint{1.445419in}{1.942768in}}%
\pgfpathcurveto{\pgfqpoint{1.453655in}{1.942768in}}{\pgfqpoint{1.461555in}{1.946041in}}{\pgfqpoint{1.467379in}{1.951864in}}%
\pgfpathcurveto{\pgfqpoint{1.473203in}{1.957688in}}{\pgfqpoint{1.476476in}{1.965588in}}{\pgfqpoint{1.476476in}{1.973825in}}%
\pgfpathcurveto{\pgfqpoint{1.476476in}{1.982061in}}{\pgfqpoint{1.473203in}{1.989961in}}{\pgfqpoint{1.467379in}{1.995785in}}%
\pgfpathcurveto{\pgfqpoint{1.461555in}{2.001609in}}{\pgfqpoint{1.453655in}{2.004881in}}{\pgfqpoint{1.445419in}{2.004881in}}%
\pgfpathcurveto{\pgfqpoint{1.437183in}{2.004881in}}{\pgfqpoint{1.429283in}{2.001609in}}{\pgfqpoint{1.423459in}{1.995785in}}%
\pgfpathcurveto{\pgfqpoint{1.417635in}{1.989961in}}{\pgfqpoint{1.414363in}{1.982061in}}{\pgfqpoint{1.414363in}{1.973825in}}%
\pgfpathcurveto{\pgfqpoint{1.414363in}{1.965588in}}{\pgfqpoint{1.417635in}{1.957688in}}{\pgfqpoint{1.423459in}{1.951864in}}%
\pgfpathcurveto{\pgfqpoint{1.429283in}{1.946041in}}{\pgfqpoint{1.437183in}{1.942768in}}{\pgfqpoint{1.445419in}{1.942768in}}%
\pgfpathclose%
\pgfusepath{stroke,fill}%
\end{pgfscope}%
\begin{pgfscope}%
\pgfpathrectangle{\pgfqpoint{0.100000in}{0.212622in}}{\pgfqpoint{3.696000in}{3.696000in}}%
\pgfusepath{clip}%
\pgfsetbuttcap%
\pgfsetroundjoin%
\definecolor{currentfill}{rgb}{0.121569,0.466667,0.705882}%
\pgfsetfillcolor{currentfill}%
\pgfsetfillopacity{0.427642}%
\pgfsetlinewidth{1.003750pt}%
\definecolor{currentstroke}{rgb}{0.121569,0.466667,0.705882}%
\pgfsetstrokecolor{currentstroke}%
\pgfsetstrokeopacity{0.427642}%
\pgfsetdash{}{0pt}%
\pgfpathmoveto{\pgfqpoint{1.440591in}{1.937011in}}%
\pgfpathcurveto{\pgfqpoint{1.448827in}{1.937011in}}{\pgfqpoint{1.456728in}{1.940283in}}{\pgfqpoint{1.462551in}{1.946107in}}%
\pgfpathcurveto{\pgfqpoint{1.468375in}{1.951931in}}{\pgfqpoint{1.471648in}{1.959831in}}{\pgfqpoint{1.471648in}{1.968067in}}%
\pgfpathcurveto{\pgfqpoint{1.471648in}{1.976304in}}{\pgfqpoint{1.468375in}{1.984204in}}{\pgfqpoint{1.462551in}{1.990027in}}%
\pgfpathcurveto{\pgfqpoint{1.456728in}{1.995851in}}{\pgfqpoint{1.448827in}{1.999124in}}{\pgfqpoint{1.440591in}{1.999124in}}%
\pgfpathcurveto{\pgfqpoint{1.432355in}{1.999124in}}{\pgfqpoint{1.424455in}{1.995851in}}{\pgfqpoint{1.418631in}{1.990027in}}%
\pgfpathcurveto{\pgfqpoint{1.412807in}{1.984204in}}{\pgfqpoint{1.409535in}{1.976304in}}{\pgfqpoint{1.409535in}{1.968067in}}%
\pgfpathcurveto{\pgfqpoint{1.409535in}{1.959831in}}{\pgfqpoint{1.412807in}{1.951931in}}{\pgfqpoint{1.418631in}{1.946107in}}%
\pgfpathcurveto{\pgfqpoint{1.424455in}{1.940283in}}{\pgfqpoint{1.432355in}{1.937011in}}{\pgfqpoint{1.440591in}{1.937011in}}%
\pgfpathclose%
\pgfusepath{stroke,fill}%
\end{pgfscope}%
\begin{pgfscope}%
\pgfpathrectangle{\pgfqpoint{0.100000in}{0.212622in}}{\pgfqpoint{3.696000in}{3.696000in}}%
\pgfusepath{clip}%
\pgfsetbuttcap%
\pgfsetroundjoin%
\definecolor{currentfill}{rgb}{0.121569,0.466667,0.705882}%
\pgfsetfillcolor{currentfill}%
\pgfsetfillopacity{0.428712}%
\pgfsetlinewidth{1.003750pt}%
\definecolor{currentstroke}{rgb}{0.121569,0.466667,0.705882}%
\pgfsetstrokecolor{currentstroke}%
\pgfsetstrokeopacity{0.428712}%
\pgfsetdash{}{0pt}%
\pgfpathmoveto{\pgfqpoint{1.439371in}{1.936024in}}%
\pgfpathcurveto{\pgfqpoint{1.447607in}{1.936024in}}{\pgfqpoint{1.455507in}{1.939296in}}{\pgfqpoint{1.461331in}{1.945120in}}%
\pgfpathcurveto{\pgfqpoint{1.467155in}{1.950944in}}{\pgfqpoint{1.470427in}{1.958844in}}{\pgfqpoint{1.470427in}{1.967081in}}%
\pgfpathcurveto{\pgfqpoint{1.470427in}{1.975317in}}{\pgfqpoint{1.467155in}{1.983217in}}{\pgfqpoint{1.461331in}{1.989041in}}%
\pgfpathcurveto{\pgfqpoint{1.455507in}{1.994865in}}{\pgfqpoint{1.447607in}{1.998137in}}{\pgfqpoint{1.439371in}{1.998137in}}%
\pgfpathcurveto{\pgfqpoint{1.431134in}{1.998137in}}{\pgfqpoint{1.423234in}{1.994865in}}{\pgfqpoint{1.417410in}{1.989041in}}%
\pgfpathcurveto{\pgfqpoint{1.411587in}{1.983217in}}{\pgfqpoint{1.408314in}{1.975317in}}{\pgfqpoint{1.408314in}{1.967081in}}%
\pgfpathcurveto{\pgfqpoint{1.408314in}{1.958844in}}{\pgfqpoint{1.411587in}{1.950944in}}{\pgfqpoint{1.417410in}{1.945120in}}%
\pgfpathcurveto{\pgfqpoint{1.423234in}{1.939296in}}{\pgfqpoint{1.431134in}{1.936024in}}{\pgfqpoint{1.439371in}{1.936024in}}%
\pgfpathclose%
\pgfusepath{stroke,fill}%
\end{pgfscope}%
\begin{pgfscope}%
\pgfpathrectangle{\pgfqpoint{0.100000in}{0.212622in}}{\pgfqpoint{3.696000in}{3.696000in}}%
\pgfusepath{clip}%
\pgfsetbuttcap%
\pgfsetroundjoin%
\definecolor{currentfill}{rgb}{0.121569,0.466667,0.705882}%
\pgfsetfillcolor{currentfill}%
\pgfsetfillopacity{0.429431}%
\pgfsetlinewidth{1.003750pt}%
\definecolor{currentstroke}{rgb}{0.121569,0.466667,0.705882}%
\pgfsetstrokecolor{currentstroke}%
\pgfsetstrokeopacity{0.429431}%
\pgfsetdash{}{0pt}%
\pgfpathmoveto{\pgfqpoint{1.438125in}{1.934357in}}%
\pgfpathcurveto{\pgfqpoint{1.446361in}{1.934357in}}{\pgfqpoint{1.454261in}{1.937629in}}{\pgfqpoint{1.460085in}{1.943453in}}%
\pgfpathcurveto{\pgfqpoint{1.465909in}{1.949277in}}{\pgfqpoint{1.469182in}{1.957177in}}{\pgfqpoint{1.469182in}{1.965413in}}%
\pgfpathcurveto{\pgfqpoint{1.469182in}{1.973649in}}{\pgfqpoint{1.465909in}{1.981549in}}{\pgfqpoint{1.460085in}{1.987373in}}%
\pgfpathcurveto{\pgfqpoint{1.454261in}{1.993197in}}{\pgfqpoint{1.446361in}{1.996470in}}{\pgfqpoint{1.438125in}{1.996470in}}%
\pgfpathcurveto{\pgfqpoint{1.429889in}{1.996470in}}{\pgfqpoint{1.421989in}{1.993197in}}{\pgfqpoint{1.416165in}{1.987373in}}%
\pgfpathcurveto{\pgfqpoint{1.410341in}{1.981549in}}{\pgfqpoint{1.407069in}{1.973649in}}{\pgfqpoint{1.407069in}{1.965413in}}%
\pgfpathcurveto{\pgfqpoint{1.407069in}{1.957177in}}{\pgfqpoint{1.410341in}{1.949277in}}{\pgfqpoint{1.416165in}{1.943453in}}%
\pgfpathcurveto{\pgfqpoint{1.421989in}{1.937629in}}{\pgfqpoint{1.429889in}{1.934357in}}{\pgfqpoint{1.438125in}{1.934357in}}%
\pgfpathclose%
\pgfusepath{stroke,fill}%
\end{pgfscope}%
\begin{pgfscope}%
\pgfpathrectangle{\pgfqpoint{0.100000in}{0.212622in}}{\pgfqpoint{3.696000in}{3.696000in}}%
\pgfusepath{clip}%
\pgfsetbuttcap%
\pgfsetroundjoin%
\definecolor{currentfill}{rgb}{0.121569,0.466667,0.705882}%
\pgfsetfillcolor{currentfill}%
\pgfsetfillopacity{0.430547}%
\pgfsetlinewidth{1.003750pt}%
\definecolor{currentstroke}{rgb}{0.121569,0.466667,0.705882}%
\pgfsetstrokecolor{currentstroke}%
\pgfsetstrokeopacity{0.430547}%
\pgfsetdash{}{0pt}%
\pgfpathmoveto{\pgfqpoint{1.437018in}{1.933097in}}%
\pgfpathcurveto{\pgfqpoint{1.445255in}{1.933097in}}{\pgfqpoint{1.453155in}{1.936369in}}{\pgfqpoint{1.458979in}{1.942193in}}%
\pgfpathcurveto{\pgfqpoint{1.464803in}{1.948017in}}{\pgfqpoint{1.468075in}{1.955917in}}{\pgfqpoint{1.468075in}{1.964154in}}%
\pgfpathcurveto{\pgfqpoint{1.468075in}{1.972390in}}{\pgfqpoint{1.464803in}{1.980290in}}{\pgfqpoint{1.458979in}{1.986114in}}%
\pgfpathcurveto{\pgfqpoint{1.453155in}{1.991938in}}{\pgfqpoint{1.445255in}{1.995210in}}{\pgfqpoint{1.437018in}{1.995210in}}%
\pgfpathcurveto{\pgfqpoint{1.428782in}{1.995210in}}{\pgfqpoint{1.420882in}{1.991938in}}{\pgfqpoint{1.415058in}{1.986114in}}%
\pgfpathcurveto{\pgfqpoint{1.409234in}{1.980290in}}{\pgfqpoint{1.405962in}{1.972390in}}{\pgfqpoint{1.405962in}{1.964154in}}%
\pgfpathcurveto{\pgfqpoint{1.405962in}{1.955917in}}{\pgfqpoint{1.409234in}{1.948017in}}{\pgfqpoint{1.415058in}{1.942193in}}%
\pgfpathcurveto{\pgfqpoint{1.420882in}{1.936369in}}{\pgfqpoint{1.428782in}{1.933097in}}{\pgfqpoint{1.437018in}{1.933097in}}%
\pgfpathclose%
\pgfusepath{stroke,fill}%
\end{pgfscope}%
\begin{pgfscope}%
\pgfpathrectangle{\pgfqpoint{0.100000in}{0.212622in}}{\pgfqpoint{3.696000in}{3.696000in}}%
\pgfusepath{clip}%
\pgfsetbuttcap%
\pgfsetroundjoin%
\definecolor{currentfill}{rgb}{0.121569,0.466667,0.705882}%
\pgfsetfillcolor{currentfill}%
\pgfsetfillopacity{0.431200}%
\pgfsetlinewidth{1.003750pt}%
\definecolor{currentstroke}{rgb}{0.121569,0.466667,0.705882}%
\pgfsetstrokecolor{currentstroke}%
\pgfsetstrokeopacity{0.431200}%
\pgfsetdash{}{0pt}%
\pgfpathmoveto{\pgfqpoint{1.436945in}{1.934510in}}%
\pgfpathcurveto{\pgfqpoint{1.445182in}{1.934510in}}{\pgfqpoint{1.453082in}{1.937783in}}{\pgfqpoint{1.458906in}{1.943607in}}%
\pgfpathcurveto{\pgfqpoint{1.464730in}{1.949431in}}{\pgfqpoint{1.468002in}{1.957331in}}{\pgfqpoint{1.468002in}{1.965567in}}%
\pgfpathcurveto{\pgfqpoint{1.468002in}{1.973803in}}{\pgfqpoint{1.464730in}{1.981703in}}{\pgfqpoint{1.458906in}{1.987527in}}%
\pgfpathcurveto{\pgfqpoint{1.453082in}{1.993351in}}{\pgfqpoint{1.445182in}{1.996623in}}{\pgfqpoint{1.436945in}{1.996623in}}%
\pgfpathcurveto{\pgfqpoint{1.428709in}{1.996623in}}{\pgfqpoint{1.420809in}{1.993351in}}{\pgfqpoint{1.414985in}{1.987527in}}%
\pgfpathcurveto{\pgfqpoint{1.409161in}{1.981703in}}{\pgfqpoint{1.405889in}{1.973803in}}{\pgfqpoint{1.405889in}{1.965567in}}%
\pgfpathcurveto{\pgfqpoint{1.405889in}{1.957331in}}{\pgfqpoint{1.409161in}{1.949431in}}{\pgfqpoint{1.414985in}{1.943607in}}%
\pgfpathcurveto{\pgfqpoint{1.420809in}{1.937783in}}{\pgfqpoint{1.428709in}{1.934510in}}{\pgfqpoint{1.436945in}{1.934510in}}%
\pgfpathclose%
\pgfusepath{stroke,fill}%
\end{pgfscope}%
\begin{pgfscope}%
\pgfpathrectangle{\pgfqpoint{0.100000in}{0.212622in}}{\pgfqpoint{3.696000in}{3.696000in}}%
\pgfusepath{clip}%
\pgfsetbuttcap%
\pgfsetroundjoin%
\definecolor{currentfill}{rgb}{0.121569,0.466667,0.705882}%
\pgfsetfillcolor{currentfill}%
\pgfsetfillopacity{0.432218}%
\pgfsetlinewidth{1.003750pt}%
\definecolor{currentstroke}{rgb}{0.121569,0.466667,0.705882}%
\pgfsetstrokecolor{currentstroke}%
\pgfsetstrokeopacity{0.432218}%
\pgfsetdash{}{0pt}%
\pgfpathmoveto{\pgfqpoint{1.442314in}{1.943801in}}%
\pgfpathcurveto{\pgfqpoint{1.450551in}{1.943801in}}{\pgfqpoint{1.458451in}{1.947073in}}{\pgfqpoint{1.464275in}{1.952897in}}%
\pgfpathcurveto{\pgfqpoint{1.470099in}{1.958721in}}{\pgfqpoint{1.473371in}{1.966621in}}{\pgfqpoint{1.473371in}{1.974858in}}%
\pgfpathcurveto{\pgfqpoint{1.473371in}{1.983094in}}{\pgfqpoint{1.470099in}{1.990994in}}{\pgfqpoint{1.464275in}{1.996818in}}%
\pgfpathcurveto{\pgfqpoint{1.458451in}{2.002642in}}{\pgfqpoint{1.450551in}{2.005914in}}{\pgfqpoint{1.442314in}{2.005914in}}%
\pgfpathcurveto{\pgfqpoint{1.434078in}{2.005914in}}{\pgfqpoint{1.426178in}{2.002642in}}{\pgfqpoint{1.420354in}{1.996818in}}%
\pgfpathcurveto{\pgfqpoint{1.414530in}{1.990994in}}{\pgfqpoint{1.411258in}{1.983094in}}{\pgfqpoint{1.411258in}{1.974858in}}%
\pgfpathcurveto{\pgfqpoint{1.411258in}{1.966621in}}{\pgfqpoint{1.414530in}{1.958721in}}{\pgfqpoint{1.420354in}{1.952897in}}%
\pgfpathcurveto{\pgfqpoint{1.426178in}{1.947073in}}{\pgfqpoint{1.434078in}{1.943801in}}{\pgfqpoint{1.442314in}{1.943801in}}%
\pgfpathclose%
\pgfusepath{stroke,fill}%
\end{pgfscope}%
\begin{pgfscope}%
\pgfpathrectangle{\pgfqpoint{0.100000in}{0.212622in}}{\pgfqpoint{3.696000in}{3.696000in}}%
\pgfusepath{clip}%
\pgfsetbuttcap%
\pgfsetroundjoin%
\definecolor{currentfill}{rgb}{0.121569,0.466667,0.705882}%
\pgfsetfillcolor{currentfill}%
\pgfsetfillopacity{0.433567}%
\pgfsetlinewidth{1.003750pt}%
\definecolor{currentstroke}{rgb}{0.121569,0.466667,0.705882}%
\pgfsetstrokecolor{currentstroke}%
\pgfsetstrokeopacity{0.433567}%
\pgfsetdash{}{0pt}%
\pgfpathmoveto{\pgfqpoint{1.432633in}{1.929225in}}%
\pgfpathcurveto{\pgfqpoint{1.440869in}{1.929225in}}{\pgfqpoint{1.448769in}{1.932497in}}{\pgfqpoint{1.454593in}{1.938321in}}%
\pgfpathcurveto{\pgfqpoint{1.460417in}{1.944145in}}{\pgfqpoint{1.463689in}{1.952045in}}{\pgfqpoint{1.463689in}{1.960281in}}%
\pgfpathcurveto{\pgfqpoint{1.463689in}{1.968518in}}{\pgfqpoint{1.460417in}{1.976418in}}{\pgfqpoint{1.454593in}{1.982242in}}%
\pgfpathcurveto{\pgfqpoint{1.448769in}{1.988066in}}{\pgfqpoint{1.440869in}{1.991338in}}{\pgfqpoint{1.432633in}{1.991338in}}%
\pgfpathcurveto{\pgfqpoint{1.424397in}{1.991338in}}{\pgfqpoint{1.416497in}{1.988066in}}{\pgfqpoint{1.410673in}{1.982242in}}%
\pgfpathcurveto{\pgfqpoint{1.404849in}{1.976418in}}{\pgfqpoint{1.401576in}{1.968518in}}{\pgfqpoint{1.401576in}{1.960281in}}%
\pgfpathcurveto{\pgfqpoint{1.401576in}{1.952045in}}{\pgfqpoint{1.404849in}{1.944145in}}{\pgfqpoint{1.410673in}{1.938321in}}%
\pgfpathcurveto{\pgfqpoint{1.416497in}{1.932497in}}{\pgfqpoint{1.424397in}{1.929225in}}{\pgfqpoint{1.432633in}{1.929225in}}%
\pgfpathclose%
\pgfusepath{stroke,fill}%
\end{pgfscope}%
\begin{pgfscope}%
\pgfpathrectangle{\pgfqpoint{0.100000in}{0.212622in}}{\pgfqpoint{3.696000in}{3.696000in}}%
\pgfusepath{clip}%
\pgfsetbuttcap%
\pgfsetroundjoin%
\definecolor{currentfill}{rgb}{0.121569,0.466667,0.705882}%
\pgfsetfillcolor{currentfill}%
\pgfsetfillopacity{0.436790}%
\pgfsetlinewidth{1.003750pt}%
\definecolor{currentstroke}{rgb}{0.121569,0.466667,0.705882}%
\pgfsetstrokecolor{currentstroke}%
\pgfsetstrokeopacity{0.436790}%
\pgfsetdash{}{0pt}%
\pgfpathmoveto{\pgfqpoint{1.431979in}{1.932679in}}%
\pgfpathcurveto{\pgfqpoint{1.440216in}{1.932679in}}{\pgfqpoint{1.448116in}{1.935952in}}{\pgfqpoint{1.453940in}{1.941776in}}%
\pgfpathcurveto{\pgfqpoint{1.459764in}{1.947599in}}{\pgfqpoint{1.463036in}{1.955499in}}{\pgfqpoint{1.463036in}{1.963736in}}%
\pgfpathcurveto{\pgfqpoint{1.463036in}{1.971972in}}{\pgfqpoint{1.459764in}{1.979872in}}{\pgfqpoint{1.453940in}{1.985696in}}%
\pgfpathcurveto{\pgfqpoint{1.448116in}{1.991520in}}{\pgfqpoint{1.440216in}{1.994792in}}{\pgfqpoint{1.431979in}{1.994792in}}%
\pgfpathcurveto{\pgfqpoint{1.423743in}{1.994792in}}{\pgfqpoint{1.415843in}{1.991520in}}{\pgfqpoint{1.410019in}{1.985696in}}%
\pgfpathcurveto{\pgfqpoint{1.404195in}{1.979872in}}{\pgfqpoint{1.400923in}{1.971972in}}{\pgfqpoint{1.400923in}{1.963736in}}%
\pgfpathcurveto{\pgfqpoint{1.400923in}{1.955499in}}{\pgfqpoint{1.404195in}{1.947599in}}{\pgfqpoint{1.410019in}{1.941776in}}%
\pgfpathcurveto{\pgfqpoint{1.415843in}{1.935952in}}{\pgfqpoint{1.423743in}{1.932679in}}{\pgfqpoint{1.431979in}{1.932679in}}%
\pgfpathclose%
\pgfusepath{stroke,fill}%
\end{pgfscope}%
\begin{pgfscope}%
\pgfpathrectangle{\pgfqpoint{0.100000in}{0.212622in}}{\pgfqpoint{3.696000in}{3.696000in}}%
\pgfusepath{clip}%
\pgfsetbuttcap%
\pgfsetroundjoin%
\definecolor{currentfill}{rgb}{0.121569,0.466667,0.705882}%
\pgfsetfillcolor{currentfill}%
\pgfsetfillopacity{0.437227}%
\pgfsetlinewidth{1.003750pt}%
\definecolor{currentstroke}{rgb}{0.121569,0.466667,0.705882}%
\pgfsetstrokecolor{currentstroke}%
\pgfsetstrokeopacity{0.437227}%
\pgfsetdash{}{0pt}%
\pgfpathmoveto{\pgfqpoint{1.438246in}{1.943868in}}%
\pgfpathcurveto{\pgfqpoint{1.446483in}{1.943868in}}{\pgfqpoint{1.454383in}{1.947140in}}{\pgfqpoint{1.460207in}{1.952964in}}%
\pgfpathcurveto{\pgfqpoint{1.466030in}{1.958788in}}{\pgfqpoint{1.469303in}{1.966688in}}{\pgfqpoint{1.469303in}{1.974924in}}%
\pgfpathcurveto{\pgfqpoint{1.469303in}{1.983160in}}{\pgfqpoint{1.466030in}{1.991060in}}{\pgfqpoint{1.460207in}{1.996884in}}%
\pgfpathcurveto{\pgfqpoint{1.454383in}{2.002708in}}{\pgfqpoint{1.446483in}{2.005981in}}{\pgfqpoint{1.438246in}{2.005981in}}%
\pgfpathcurveto{\pgfqpoint{1.430010in}{2.005981in}}{\pgfqpoint{1.422110in}{2.002708in}}{\pgfqpoint{1.416286in}{1.996884in}}%
\pgfpathcurveto{\pgfqpoint{1.410462in}{1.991060in}}{\pgfqpoint{1.407190in}{1.983160in}}{\pgfqpoint{1.407190in}{1.974924in}}%
\pgfpathcurveto{\pgfqpoint{1.407190in}{1.966688in}}{\pgfqpoint{1.410462in}{1.958788in}}{\pgfqpoint{1.416286in}{1.952964in}}%
\pgfpathcurveto{\pgfqpoint{1.422110in}{1.947140in}}{\pgfqpoint{1.430010in}{1.943868in}}{\pgfqpoint{1.438246in}{1.943868in}}%
\pgfpathclose%
\pgfusepath{stroke,fill}%
\end{pgfscope}%
\begin{pgfscope}%
\pgfpathrectangle{\pgfqpoint{0.100000in}{0.212622in}}{\pgfqpoint{3.696000in}{3.696000in}}%
\pgfusepath{clip}%
\pgfsetbuttcap%
\pgfsetroundjoin%
\definecolor{currentfill}{rgb}{0.121569,0.466667,0.705882}%
\pgfsetfillcolor{currentfill}%
\pgfsetfillopacity{0.440196}%
\pgfsetlinewidth{1.003750pt}%
\definecolor{currentstroke}{rgb}{0.121569,0.466667,0.705882}%
\pgfsetstrokecolor{currentstroke}%
\pgfsetstrokeopacity{0.440196}%
\pgfsetdash{}{0pt}%
\pgfpathmoveto{\pgfqpoint{1.449403in}{1.946712in}}%
\pgfpathcurveto{\pgfqpoint{1.457639in}{1.946712in}}{\pgfqpoint{1.465539in}{1.949984in}}{\pgfqpoint{1.471363in}{1.955808in}}%
\pgfpathcurveto{\pgfqpoint{1.477187in}{1.961632in}}{\pgfqpoint{1.480459in}{1.969532in}}{\pgfqpoint{1.480459in}{1.977768in}}%
\pgfpathcurveto{\pgfqpoint{1.480459in}{1.986005in}}{\pgfqpoint{1.477187in}{1.993905in}}{\pgfqpoint{1.471363in}{1.999729in}}%
\pgfpathcurveto{\pgfqpoint{1.465539in}{2.005553in}}{\pgfqpoint{1.457639in}{2.008825in}}{\pgfqpoint{1.449403in}{2.008825in}}%
\pgfpathcurveto{\pgfqpoint{1.441166in}{2.008825in}}{\pgfqpoint{1.433266in}{2.005553in}}{\pgfqpoint{1.427442in}{1.999729in}}%
\pgfpathcurveto{\pgfqpoint{1.421618in}{1.993905in}}{\pgfqpoint{1.418346in}{1.986005in}}{\pgfqpoint{1.418346in}{1.977768in}}%
\pgfpathcurveto{\pgfqpoint{1.418346in}{1.969532in}}{\pgfqpoint{1.421618in}{1.961632in}}{\pgfqpoint{1.427442in}{1.955808in}}%
\pgfpathcurveto{\pgfqpoint{1.433266in}{1.949984in}}{\pgfqpoint{1.441166in}{1.946712in}}{\pgfqpoint{1.449403in}{1.946712in}}%
\pgfpathclose%
\pgfusepath{stroke,fill}%
\end{pgfscope}%
\begin{pgfscope}%
\pgfpathrectangle{\pgfqpoint{0.100000in}{0.212622in}}{\pgfqpoint{3.696000in}{3.696000in}}%
\pgfusepath{clip}%
\pgfsetbuttcap%
\pgfsetroundjoin%
\definecolor{currentfill}{rgb}{0.121569,0.466667,0.705882}%
\pgfsetfillcolor{currentfill}%
\pgfsetfillopacity{0.440334}%
\pgfsetlinewidth{1.003750pt}%
\definecolor{currentstroke}{rgb}{0.121569,0.466667,0.705882}%
\pgfsetstrokecolor{currentstroke}%
\pgfsetstrokeopacity{0.440334}%
\pgfsetdash{}{0pt}%
\pgfpathmoveto{\pgfqpoint{1.438367in}{1.935731in}}%
\pgfpathcurveto{\pgfqpoint{1.446603in}{1.935731in}}{\pgfqpoint{1.454503in}{1.939003in}}{\pgfqpoint{1.460327in}{1.944827in}}%
\pgfpathcurveto{\pgfqpoint{1.466151in}{1.950651in}}{\pgfqpoint{1.469423in}{1.958551in}}{\pgfqpoint{1.469423in}{1.966787in}}%
\pgfpathcurveto{\pgfqpoint{1.469423in}{1.975023in}}{\pgfqpoint{1.466151in}{1.982923in}}{\pgfqpoint{1.460327in}{1.988747in}}%
\pgfpathcurveto{\pgfqpoint{1.454503in}{1.994571in}}{\pgfqpoint{1.446603in}{1.997844in}}{\pgfqpoint{1.438367in}{1.997844in}}%
\pgfpathcurveto{\pgfqpoint{1.430131in}{1.997844in}}{\pgfqpoint{1.422231in}{1.994571in}}{\pgfqpoint{1.416407in}{1.988747in}}%
\pgfpathcurveto{\pgfqpoint{1.410583in}{1.982923in}}{\pgfqpoint{1.407310in}{1.975023in}}{\pgfqpoint{1.407310in}{1.966787in}}%
\pgfpathcurveto{\pgfqpoint{1.407310in}{1.958551in}}{\pgfqpoint{1.410583in}{1.950651in}}{\pgfqpoint{1.416407in}{1.944827in}}%
\pgfpathcurveto{\pgfqpoint{1.422231in}{1.939003in}}{\pgfqpoint{1.430131in}{1.935731in}}{\pgfqpoint{1.438367in}{1.935731in}}%
\pgfpathclose%
\pgfusepath{stroke,fill}%
\end{pgfscope}%
\begin{pgfscope}%
\pgfpathrectangle{\pgfqpoint{0.100000in}{0.212622in}}{\pgfqpoint{3.696000in}{3.696000in}}%
\pgfusepath{clip}%
\pgfsetbuttcap%
\pgfsetroundjoin%
\definecolor{currentfill}{rgb}{0.121569,0.466667,0.705882}%
\pgfsetfillcolor{currentfill}%
\pgfsetfillopacity{0.442497}%
\pgfsetlinewidth{1.003750pt}%
\definecolor{currentstroke}{rgb}{0.121569,0.466667,0.705882}%
\pgfsetstrokecolor{currentstroke}%
\pgfsetstrokeopacity{0.442497}%
\pgfsetdash{}{0pt}%
\pgfpathmoveto{\pgfqpoint{1.433790in}{1.928433in}}%
\pgfpathcurveto{\pgfqpoint{1.442026in}{1.928433in}}{\pgfqpoint{1.449926in}{1.931706in}}{\pgfqpoint{1.455750in}{1.937530in}}%
\pgfpathcurveto{\pgfqpoint{1.461574in}{1.943354in}}{\pgfqpoint{1.464846in}{1.951254in}}{\pgfqpoint{1.464846in}{1.959490in}}%
\pgfpathcurveto{\pgfqpoint{1.464846in}{1.967726in}}{\pgfqpoint{1.461574in}{1.975626in}}{\pgfqpoint{1.455750in}{1.981450in}}%
\pgfpathcurveto{\pgfqpoint{1.449926in}{1.987274in}}{\pgfqpoint{1.442026in}{1.990546in}}{\pgfqpoint{1.433790in}{1.990546in}}%
\pgfpathcurveto{\pgfqpoint{1.425554in}{1.990546in}}{\pgfqpoint{1.417653in}{1.987274in}}{\pgfqpoint{1.411830in}{1.981450in}}%
\pgfpathcurveto{\pgfqpoint{1.406006in}{1.975626in}}{\pgfqpoint{1.402733in}{1.967726in}}{\pgfqpoint{1.402733in}{1.959490in}}%
\pgfpathcurveto{\pgfqpoint{1.402733in}{1.951254in}}{\pgfqpoint{1.406006in}{1.943354in}}{\pgfqpoint{1.411830in}{1.937530in}}%
\pgfpathcurveto{\pgfqpoint{1.417653in}{1.931706in}}{\pgfqpoint{1.425554in}{1.928433in}}{\pgfqpoint{1.433790in}{1.928433in}}%
\pgfpathclose%
\pgfusepath{stroke,fill}%
\end{pgfscope}%
\begin{pgfscope}%
\pgfpathrectangle{\pgfqpoint{0.100000in}{0.212622in}}{\pgfqpoint{3.696000in}{3.696000in}}%
\pgfusepath{clip}%
\pgfsetbuttcap%
\pgfsetroundjoin%
\definecolor{currentfill}{rgb}{0.121569,0.466667,0.705882}%
\pgfsetfillcolor{currentfill}%
\pgfsetfillopacity{0.444235}%
\pgfsetlinewidth{1.003750pt}%
\definecolor{currentstroke}{rgb}{0.121569,0.466667,0.705882}%
\pgfsetstrokecolor{currentstroke}%
\pgfsetstrokeopacity{0.444235}%
\pgfsetdash{}{0pt}%
\pgfpathmoveto{\pgfqpoint{1.422990in}{1.918400in}}%
\pgfpathcurveto{\pgfqpoint{1.431226in}{1.918400in}}{\pgfqpoint{1.439126in}{1.921672in}}{\pgfqpoint{1.444950in}{1.927496in}}%
\pgfpathcurveto{\pgfqpoint{1.450774in}{1.933320in}}{\pgfqpoint{1.454046in}{1.941220in}}{\pgfqpoint{1.454046in}{1.949456in}}%
\pgfpathcurveto{\pgfqpoint{1.454046in}{1.957693in}}{\pgfqpoint{1.450774in}{1.965593in}}{\pgfqpoint{1.444950in}{1.971417in}}%
\pgfpathcurveto{\pgfqpoint{1.439126in}{1.977241in}}{\pgfqpoint{1.431226in}{1.980513in}}{\pgfqpoint{1.422990in}{1.980513in}}%
\pgfpathcurveto{\pgfqpoint{1.414754in}{1.980513in}}{\pgfqpoint{1.406854in}{1.977241in}}{\pgfqpoint{1.401030in}{1.971417in}}%
\pgfpathcurveto{\pgfqpoint{1.395206in}{1.965593in}}{\pgfqpoint{1.391933in}{1.957693in}}{\pgfqpoint{1.391933in}{1.949456in}}%
\pgfpathcurveto{\pgfqpoint{1.391933in}{1.941220in}}{\pgfqpoint{1.395206in}{1.933320in}}{\pgfqpoint{1.401030in}{1.927496in}}%
\pgfpathcurveto{\pgfqpoint{1.406854in}{1.921672in}}{\pgfqpoint{1.414754in}{1.918400in}}{\pgfqpoint{1.422990in}{1.918400in}}%
\pgfpathclose%
\pgfusepath{stroke,fill}%
\end{pgfscope}%
\begin{pgfscope}%
\pgfpathrectangle{\pgfqpoint{0.100000in}{0.212622in}}{\pgfqpoint{3.696000in}{3.696000in}}%
\pgfusepath{clip}%
\pgfsetbuttcap%
\pgfsetroundjoin%
\definecolor{currentfill}{rgb}{0.121569,0.466667,0.705882}%
\pgfsetfillcolor{currentfill}%
\pgfsetfillopacity{0.449105}%
\pgfsetlinewidth{1.003750pt}%
\definecolor{currentstroke}{rgb}{0.121569,0.466667,0.705882}%
\pgfsetstrokecolor{currentstroke}%
\pgfsetstrokeopacity{0.449105}%
\pgfsetdash{}{0pt}%
\pgfpathmoveto{\pgfqpoint{1.439418in}{1.929490in}}%
\pgfpathcurveto{\pgfqpoint{1.447654in}{1.929490in}}{\pgfqpoint{1.455554in}{1.932762in}}{\pgfqpoint{1.461378in}{1.938586in}}%
\pgfpathcurveto{\pgfqpoint{1.467202in}{1.944410in}}{\pgfqpoint{1.470474in}{1.952310in}}{\pgfqpoint{1.470474in}{1.960546in}}%
\pgfpathcurveto{\pgfqpoint{1.470474in}{1.968783in}}{\pgfqpoint{1.467202in}{1.976683in}}{\pgfqpoint{1.461378in}{1.982507in}}%
\pgfpathcurveto{\pgfqpoint{1.455554in}{1.988331in}}{\pgfqpoint{1.447654in}{1.991603in}}{\pgfqpoint{1.439418in}{1.991603in}}%
\pgfpathcurveto{\pgfqpoint{1.431181in}{1.991603in}}{\pgfqpoint{1.423281in}{1.988331in}}{\pgfqpoint{1.417457in}{1.982507in}}%
\pgfpathcurveto{\pgfqpoint{1.411634in}{1.976683in}}{\pgfqpoint{1.408361in}{1.968783in}}{\pgfqpoint{1.408361in}{1.960546in}}%
\pgfpathcurveto{\pgfqpoint{1.408361in}{1.952310in}}{\pgfqpoint{1.411634in}{1.944410in}}{\pgfqpoint{1.417457in}{1.938586in}}%
\pgfpathcurveto{\pgfqpoint{1.423281in}{1.932762in}}{\pgfqpoint{1.431181in}{1.929490in}}{\pgfqpoint{1.439418in}{1.929490in}}%
\pgfpathclose%
\pgfusepath{stroke,fill}%
\end{pgfscope}%
\begin{pgfscope}%
\pgfpathrectangle{\pgfqpoint{0.100000in}{0.212622in}}{\pgfqpoint{3.696000in}{3.696000in}}%
\pgfusepath{clip}%
\pgfsetbuttcap%
\pgfsetroundjoin%
\definecolor{currentfill}{rgb}{0.121569,0.466667,0.705882}%
\pgfsetfillcolor{currentfill}%
\pgfsetfillopacity{0.450414}%
\pgfsetlinewidth{1.003750pt}%
\definecolor{currentstroke}{rgb}{0.121569,0.466667,0.705882}%
\pgfsetstrokecolor{currentstroke}%
\pgfsetstrokeopacity{0.450414}%
\pgfsetdash{}{0pt}%
\pgfpathmoveto{\pgfqpoint{1.430871in}{1.923133in}}%
\pgfpathcurveto{\pgfqpoint{1.439107in}{1.923133in}}{\pgfqpoint{1.447007in}{1.926405in}}{\pgfqpoint{1.452831in}{1.932229in}}%
\pgfpathcurveto{\pgfqpoint{1.458655in}{1.938053in}}{\pgfqpoint{1.461927in}{1.945953in}}{\pgfqpoint{1.461927in}{1.954190in}}%
\pgfpathcurveto{\pgfqpoint{1.461927in}{1.962426in}}{\pgfqpoint{1.458655in}{1.970326in}}{\pgfqpoint{1.452831in}{1.976150in}}%
\pgfpathcurveto{\pgfqpoint{1.447007in}{1.981974in}}{\pgfqpoint{1.439107in}{1.985246in}}{\pgfqpoint{1.430871in}{1.985246in}}%
\pgfpathcurveto{\pgfqpoint{1.422634in}{1.985246in}}{\pgfqpoint{1.414734in}{1.981974in}}{\pgfqpoint{1.408910in}{1.976150in}}%
\pgfpathcurveto{\pgfqpoint{1.403086in}{1.970326in}}{\pgfqpoint{1.399814in}{1.962426in}}{\pgfqpoint{1.399814in}{1.954190in}}%
\pgfpathcurveto{\pgfqpoint{1.399814in}{1.945953in}}{\pgfqpoint{1.403086in}{1.938053in}}{\pgfqpoint{1.408910in}{1.932229in}}%
\pgfpathcurveto{\pgfqpoint{1.414734in}{1.926405in}}{\pgfqpoint{1.422634in}{1.923133in}}{\pgfqpoint{1.430871in}{1.923133in}}%
\pgfpathclose%
\pgfusepath{stroke,fill}%
\end{pgfscope}%
\begin{pgfscope}%
\pgfpathrectangle{\pgfqpoint{0.100000in}{0.212622in}}{\pgfqpoint{3.696000in}{3.696000in}}%
\pgfusepath{clip}%
\pgfsetbuttcap%
\pgfsetroundjoin%
\definecolor{currentfill}{rgb}{0.121569,0.466667,0.705882}%
\pgfsetfillcolor{currentfill}%
\pgfsetfillopacity{0.456671}%
\pgfsetlinewidth{1.003750pt}%
\definecolor{currentstroke}{rgb}{0.121569,0.466667,0.705882}%
\pgfsetstrokecolor{currentstroke}%
\pgfsetstrokeopacity{0.456671}%
\pgfsetdash{}{0pt}%
\pgfpathmoveto{\pgfqpoint{1.421266in}{1.908491in}}%
\pgfpathcurveto{\pgfqpoint{1.429502in}{1.908491in}}{\pgfqpoint{1.437402in}{1.911763in}}{\pgfqpoint{1.443226in}{1.917587in}}%
\pgfpathcurveto{\pgfqpoint{1.449050in}{1.923411in}}{\pgfqpoint{1.452322in}{1.931311in}}{\pgfqpoint{1.452322in}{1.939548in}}%
\pgfpathcurveto{\pgfqpoint{1.452322in}{1.947784in}}{\pgfqpoint{1.449050in}{1.955684in}}{\pgfqpoint{1.443226in}{1.961508in}}%
\pgfpathcurveto{\pgfqpoint{1.437402in}{1.967332in}}{\pgfqpoint{1.429502in}{1.970604in}}{\pgfqpoint{1.421266in}{1.970604in}}%
\pgfpathcurveto{\pgfqpoint{1.413029in}{1.970604in}}{\pgfqpoint{1.405129in}{1.967332in}}{\pgfqpoint{1.399305in}{1.961508in}}%
\pgfpathcurveto{\pgfqpoint{1.393481in}{1.955684in}}{\pgfqpoint{1.390209in}{1.947784in}}{\pgfqpoint{1.390209in}{1.939548in}}%
\pgfpathcurveto{\pgfqpoint{1.390209in}{1.931311in}}{\pgfqpoint{1.393481in}{1.923411in}}{\pgfqpoint{1.399305in}{1.917587in}}%
\pgfpathcurveto{\pgfqpoint{1.405129in}{1.911763in}}{\pgfqpoint{1.413029in}{1.908491in}}{\pgfqpoint{1.421266in}{1.908491in}}%
\pgfpathclose%
\pgfusepath{stroke,fill}%
\end{pgfscope}%
\begin{pgfscope}%
\pgfpathrectangle{\pgfqpoint{0.100000in}{0.212622in}}{\pgfqpoint{3.696000in}{3.696000in}}%
\pgfusepath{clip}%
\pgfsetbuttcap%
\pgfsetroundjoin%
\definecolor{currentfill}{rgb}{0.121569,0.466667,0.705882}%
\pgfsetfillcolor{currentfill}%
\pgfsetfillopacity{0.457792}%
\pgfsetlinewidth{1.003750pt}%
\definecolor{currentstroke}{rgb}{0.121569,0.466667,0.705882}%
\pgfsetstrokecolor{currentstroke}%
\pgfsetstrokeopacity{0.457792}%
\pgfsetdash{}{0pt}%
\pgfpathmoveto{\pgfqpoint{1.424625in}{1.914728in}}%
\pgfpathcurveto{\pgfqpoint{1.432861in}{1.914728in}}{\pgfqpoint{1.440761in}{1.918000in}}{\pgfqpoint{1.446585in}{1.923824in}}%
\pgfpathcurveto{\pgfqpoint{1.452409in}{1.929648in}}{\pgfqpoint{1.455682in}{1.937548in}}{\pgfqpoint{1.455682in}{1.945785in}}%
\pgfpathcurveto{\pgfqpoint{1.455682in}{1.954021in}}{\pgfqpoint{1.452409in}{1.961921in}}{\pgfqpoint{1.446585in}{1.967745in}}%
\pgfpathcurveto{\pgfqpoint{1.440761in}{1.973569in}}{\pgfqpoint{1.432861in}{1.976841in}}{\pgfqpoint{1.424625in}{1.976841in}}%
\pgfpathcurveto{\pgfqpoint{1.416389in}{1.976841in}}{\pgfqpoint{1.408489in}{1.973569in}}{\pgfqpoint{1.402665in}{1.967745in}}%
\pgfpathcurveto{\pgfqpoint{1.396841in}{1.961921in}}{\pgfqpoint{1.393569in}{1.954021in}}{\pgfqpoint{1.393569in}{1.945785in}}%
\pgfpathcurveto{\pgfqpoint{1.393569in}{1.937548in}}{\pgfqpoint{1.396841in}{1.929648in}}{\pgfqpoint{1.402665in}{1.923824in}}%
\pgfpathcurveto{\pgfqpoint{1.408489in}{1.918000in}}{\pgfqpoint{1.416389in}{1.914728in}}{\pgfqpoint{1.424625in}{1.914728in}}%
\pgfpathclose%
\pgfusepath{stroke,fill}%
\end{pgfscope}%
\begin{pgfscope}%
\pgfpathrectangle{\pgfqpoint{0.100000in}{0.212622in}}{\pgfqpoint{3.696000in}{3.696000in}}%
\pgfusepath{clip}%
\pgfsetbuttcap%
\pgfsetroundjoin%
\definecolor{currentfill}{rgb}{0.121569,0.466667,0.705882}%
\pgfsetfillcolor{currentfill}%
\pgfsetfillopacity{0.467406}%
\pgfsetlinewidth{1.003750pt}%
\definecolor{currentstroke}{rgb}{0.121569,0.466667,0.705882}%
\pgfsetstrokecolor{currentstroke}%
\pgfsetstrokeopacity{0.467406}%
\pgfsetdash{}{0pt}%
\pgfpathmoveto{\pgfqpoint{1.407029in}{1.901027in}}%
\pgfpathcurveto{\pgfqpoint{1.415265in}{1.901027in}}{\pgfqpoint{1.423165in}{1.904299in}}{\pgfqpoint{1.428989in}{1.910123in}}%
\pgfpathcurveto{\pgfqpoint{1.434813in}{1.915947in}}{\pgfqpoint{1.438085in}{1.923847in}}{\pgfqpoint{1.438085in}{1.932083in}}%
\pgfpathcurveto{\pgfqpoint{1.438085in}{1.940320in}}{\pgfqpoint{1.434813in}{1.948220in}}{\pgfqpoint{1.428989in}{1.954044in}}%
\pgfpathcurveto{\pgfqpoint{1.423165in}{1.959868in}}{\pgfqpoint{1.415265in}{1.963140in}}{\pgfqpoint{1.407029in}{1.963140in}}%
\pgfpathcurveto{\pgfqpoint{1.398792in}{1.963140in}}{\pgfqpoint{1.390892in}{1.959868in}}{\pgfqpoint{1.385068in}{1.954044in}}%
\pgfpathcurveto{\pgfqpoint{1.379244in}{1.948220in}}{\pgfqpoint{1.375972in}{1.940320in}}{\pgfqpoint{1.375972in}{1.932083in}}%
\pgfpathcurveto{\pgfqpoint{1.375972in}{1.923847in}}{\pgfqpoint{1.379244in}{1.915947in}}{\pgfqpoint{1.385068in}{1.910123in}}%
\pgfpathcurveto{\pgfqpoint{1.390892in}{1.904299in}}{\pgfqpoint{1.398792in}{1.901027in}}{\pgfqpoint{1.407029in}{1.901027in}}%
\pgfpathclose%
\pgfusepath{stroke,fill}%
\end{pgfscope}%
\begin{pgfscope}%
\pgfpathrectangle{\pgfqpoint{0.100000in}{0.212622in}}{\pgfqpoint{3.696000in}{3.696000in}}%
\pgfusepath{clip}%
\pgfsetbuttcap%
\pgfsetroundjoin%
\definecolor{currentfill}{rgb}{0.121569,0.466667,0.705882}%
\pgfsetfillcolor{currentfill}%
\pgfsetfillopacity{0.474934}%
\pgfsetlinewidth{1.003750pt}%
\definecolor{currentstroke}{rgb}{0.121569,0.466667,0.705882}%
\pgfsetstrokecolor{currentstroke}%
\pgfsetstrokeopacity{0.474934}%
\pgfsetdash{}{0pt}%
\pgfpathmoveto{\pgfqpoint{1.404508in}{1.916709in}}%
\pgfpathcurveto{\pgfqpoint{1.412744in}{1.916709in}}{\pgfqpoint{1.420645in}{1.919981in}}{\pgfqpoint{1.426468in}{1.925805in}}%
\pgfpathcurveto{\pgfqpoint{1.432292in}{1.931629in}}{\pgfqpoint{1.435565in}{1.939529in}}{\pgfqpoint{1.435565in}{1.947765in}}%
\pgfpathcurveto{\pgfqpoint{1.435565in}{1.956001in}}{\pgfqpoint{1.432292in}{1.963901in}}{\pgfqpoint{1.426468in}{1.969725in}}%
\pgfpathcurveto{\pgfqpoint{1.420645in}{1.975549in}}{\pgfqpoint{1.412744in}{1.978822in}}{\pgfqpoint{1.404508in}{1.978822in}}%
\pgfpathcurveto{\pgfqpoint{1.396272in}{1.978822in}}{\pgfqpoint{1.388372in}{1.975549in}}{\pgfqpoint{1.382548in}{1.969725in}}%
\pgfpathcurveto{\pgfqpoint{1.376724in}{1.963901in}}{\pgfqpoint{1.373452in}{1.956001in}}{\pgfqpoint{1.373452in}{1.947765in}}%
\pgfpathcurveto{\pgfqpoint{1.373452in}{1.939529in}}{\pgfqpoint{1.376724in}{1.931629in}}{\pgfqpoint{1.382548in}{1.925805in}}%
\pgfpathcurveto{\pgfqpoint{1.388372in}{1.919981in}}{\pgfqpoint{1.396272in}{1.916709in}}{\pgfqpoint{1.404508in}{1.916709in}}%
\pgfpathclose%
\pgfusepath{stroke,fill}%
\end{pgfscope}%
\begin{pgfscope}%
\pgfpathrectangle{\pgfqpoint{0.100000in}{0.212622in}}{\pgfqpoint{3.696000in}{3.696000in}}%
\pgfusepath{clip}%
\pgfsetbuttcap%
\pgfsetroundjoin%
\definecolor{currentfill}{rgb}{0.121569,0.466667,0.705882}%
\pgfsetfillcolor{currentfill}%
\pgfsetfillopacity{0.512435}%
\pgfsetlinewidth{1.003750pt}%
\definecolor{currentstroke}{rgb}{0.121569,0.466667,0.705882}%
\pgfsetstrokecolor{currentstroke}%
\pgfsetstrokeopacity{0.512435}%
\pgfsetdash{}{0pt}%
\pgfpathmoveto{\pgfqpoint{1.318523in}{1.836646in}}%
\pgfpathcurveto{\pgfqpoint{1.326759in}{1.836646in}}{\pgfqpoint{1.334659in}{1.839918in}}{\pgfqpoint{1.340483in}{1.845742in}}%
\pgfpathcurveto{\pgfqpoint{1.346307in}{1.851566in}}{\pgfqpoint{1.349579in}{1.859466in}}{\pgfqpoint{1.349579in}{1.867702in}}%
\pgfpathcurveto{\pgfqpoint{1.349579in}{1.875939in}}{\pgfqpoint{1.346307in}{1.883839in}}{\pgfqpoint{1.340483in}{1.889663in}}%
\pgfpathcurveto{\pgfqpoint{1.334659in}{1.895486in}}{\pgfqpoint{1.326759in}{1.898759in}}{\pgfqpoint{1.318523in}{1.898759in}}%
\pgfpathcurveto{\pgfqpoint{1.310287in}{1.898759in}}{\pgfqpoint{1.302387in}{1.895486in}}{\pgfqpoint{1.296563in}{1.889663in}}%
\pgfpathcurveto{\pgfqpoint{1.290739in}{1.883839in}}{\pgfqpoint{1.287466in}{1.875939in}}{\pgfqpoint{1.287466in}{1.867702in}}%
\pgfpathcurveto{\pgfqpoint{1.287466in}{1.859466in}}{\pgfqpoint{1.290739in}{1.851566in}}{\pgfqpoint{1.296563in}{1.845742in}}%
\pgfpathcurveto{\pgfqpoint{1.302387in}{1.839918in}}{\pgfqpoint{1.310287in}{1.836646in}}{\pgfqpoint{1.318523in}{1.836646in}}%
\pgfpathclose%
\pgfusepath{stroke,fill}%
\end{pgfscope}%
\begin{pgfscope}%
\pgfpathrectangle{\pgfqpoint{0.100000in}{0.212622in}}{\pgfqpoint{3.696000in}{3.696000in}}%
\pgfusepath{clip}%
\pgfsetbuttcap%
\pgfsetroundjoin%
\definecolor{currentfill}{rgb}{0.121569,0.466667,0.705882}%
\pgfsetfillcolor{currentfill}%
\pgfsetfillopacity{0.517644}%
\pgfsetlinewidth{1.003750pt}%
\definecolor{currentstroke}{rgb}{0.121569,0.466667,0.705882}%
\pgfsetstrokecolor{currentstroke}%
\pgfsetstrokeopacity{0.517644}%
\pgfsetdash{}{0pt}%
\pgfpathmoveto{\pgfqpoint{1.312766in}{1.834521in}}%
\pgfpathcurveto{\pgfqpoint{1.321003in}{1.834521in}}{\pgfqpoint{1.328903in}{1.837794in}}{\pgfqpoint{1.334727in}{1.843618in}}%
\pgfpathcurveto{\pgfqpoint{1.340551in}{1.849442in}}{\pgfqpoint{1.343823in}{1.857342in}}{\pgfqpoint{1.343823in}{1.865578in}}%
\pgfpathcurveto{\pgfqpoint{1.343823in}{1.873814in}}{\pgfqpoint{1.340551in}{1.881714in}}{\pgfqpoint{1.334727in}{1.887538in}}%
\pgfpathcurveto{\pgfqpoint{1.328903in}{1.893362in}}{\pgfqpoint{1.321003in}{1.896634in}}{\pgfqpoint{1.312766in}{1.896634in}}%
\pgfpathcurveto{\pgfqpoint{1.304530in}{1.896634in}}{\pgfqpoint{1.296630in}{1.893362in}}{\pgfqpoint{1.290806in}{1.887538in}}%
\pgfpathcurveto{\pgfqpoint{1.284982in}{1.881714in}}{\pgfqpoint{1.281710in}{1.873814in}}{\pgfqpoint{1.281710in}{1.865578in}}%
\pgfpathcurveto{\pgfqpoint{1.281710in}{1.857342in}}{\pgfqpoint{1.284982in}{1.849442in}}{\pgfqpoint{1.290806in}{1.843618in}}%
\pgfpathcurveto{\pgfqpoint{1.296630in}{1.837794in}}{\pgfqpoint{1.304530in}{1.834521in}}{\pgfqpoint{1.312766in}{1.834521in}}%
\pgfpathclose%
\pgfusepath{stroke,fill}%
\end{pgfscope}%
\begin{pgfscope}%
\pgfpathrectangle{\pgfqpoint{0.100000in}{0.212622in}}{\pgfqpoint{3.696000in}{3.696000in}}%
\pgfusepath{clip}%
\pgfsetbuttcap%
\pgfsetroundjoin%
\definecolor{currentfill}{rgb}{0.121569,0.466667,0.705882}%
\pgfsetfillcolor{currentfill}%
\pgfsetfillopacity{0.531870}%
\pgfsetlinewidth{1.003750pt}%
\definecolor{currentstroke}{rgb}{0.121569,0.466667,0.705882}%
\pgfsetstrokecolor{currentstroke}%
\pgfsetstrokeopacity{0.531870}%
\pgfsetdash{}{0pt}%
\pgfpathmoveto{\pgfqpoint{1.284819in}{1.809282in}}%
\pgfpathcurveto{\pgfqpoint{1.293056in}{1.809282in}}{\pgfqpoint{1.300956in}{1.812554in}}{\pgfqpoint{1.306780in}{1.818378in}}%
\pgfpathcurveto{\pgfqpoint{1.312604in}{1.824202in}}{\pgfqpoint{1.315876in}{1.832102in}}{\pgfqpoint{1.315876in}{1.840338in}}%
\pgfpathcurveto{\pgfqpoint{1.315876in}{1.848575in}}{\pgfqpoint{1.312604in}{1.856475in}}{\pgfqpoint{1.306780in}{1.862299in}}%
\pgfpathcurveto{\pgfqpoint{1.300956in}{1.868123in}}{\pgfqpoint{1.293056in}{1.871395in}}{\pgfqpoint{1.284819in}{1.871395in}}%
\pgfpathcurveto{\pgfqpoint{1.276583in}{1.871395in}}{\pgfqpoint{1.268683in}{1.868123in}}{\pgfqpoint{1.262859in}{1.862299in}}%
\pgfpathcurveto{\pgfqpoint{1.257035in}{1.856475in}}{\pgfqpoint{1.253763in}{1.848575in}}{\pgfqpoint{1.253763in}{1.840338in}}%
\pgfpathcurveto{\pgfqpoint{1.253763in}{1.832102in}}{\pgfqpoint{1.257035in}{1.824202in}}{\pgfqpoint{1.262859in}{1.818378in}}%
\pgfpathcurveto{\pgfqpoint{1.268683in}{1.812554in}}{\pgfqpoint{1.276583in}{1.809282in}}{\pgfqpoint{1.284819in}{1.809282in}}%
\pgfpathclose%
\pgfusepath{stroke,fill}%
\end{pgfscope}%
\begin{pgfscope}%
\pgfpathrectangle{\pgfqpoint{0.100000in}{0.212622in}}{\pgfqpoint{3.696000in}{3.696000in}}%
\pgfusepath{clip}%
\pgfsetbuttcap%
\pgfsetroundjoin%
\definecolor{currentfill}{rgb}{0.121569,0.466667,0.705882}%
\pgfsetfillcolor{currentfill}%
\pgfsetfillopacity{0.539235}%
\pgfsetlinewidth{1.003750pt}%
\definecolor{currentstroke}{rgb}{0.121569,0.466667,0.705882}%
\pgfsetstrokecolor{currentstroke}%
\pgfsetstrokeopacity{0.539235}%
\pgfsetdash{}{0pt}%
\pgfpathmoveto{\pgfqpoint{1.273022in}{1.804666in}}%
\pgfpathcurveto{\pgfqpoint{1.281258in}{1.804666in}}{\pgfqpoint{1.289158in}{1.807939in}}{\pgfqpoint{1.294982in}{1.813762in}}%
\pgfpathcurveto{\pgfqpoint{1.300806in}{1.819586in}}{\pgfqpoint{1.304078in}{1.827486in}}{\pgfqpoint{1.304078in}{1.835723in}}%
\pgfpathcurveto{\pgfqpoint{1.304078in}{1.843959in}}{\pgfqpoint{1.300806in}{1.851859in}}{\pgfqpoint{1.294982in}{1.857683in}}%
\pgfpathcurveto{\pgfqpoint{1.289158in}{1.863507in}}{\pgfqpoint{1.281258in}{1.866779in}}{\pgfqpoint{1.273022in}{1.866779in}}%
\pgfpathcurveto{\pgfqpoint{1.264785in}{1.866779in}}{\pgfqpoint{1.256885in}{1.863507in}}{\pgfqpoint{1.251061in}{1.857683in}}%
\pgfpathcurveto{\pgfqpoint{1.245237in}{1.851859in}}{\pgfqpoint{1.241965in}{1.843959in}}{\pgfqpoint{1.241965in}{1.835723in}}%
\pgfpathcurveto{\pgfqpoint{1.241965in}{1.827486in}}{\pgfqpoint{1.245237in}{1.819586in}}{\pgfqpoint{1.251061in}{1.813762in}}%
\pgfpathcurveto{\pgfqpoint{1.256885in}{1.807939in}}{\pgfqpoint{1.264785in}{1.804666in}}{\pgfqpoint{1.273022in}{1.804666in}}%
\pgfpathclose%
\pgfusepath{stroke,fill}%
\end{pgfscope}%
\begin{pgfscope}%
\pgfpathrectangle{\pgfqpoint{0.100000in}{0.212622in}}{\pgfqpoint{3.696000in}{3.696000in}}%
\pgfusepath{clip}%
\pgfsetbuttcap%
\pgfsetroundjoin%
\definecolor{currentfill}{rgb}{0.121569,0.466667,0.705882}%
\pgfsetfillcolor{currentfill}%
\pgfsetfillopacity{0.543724}%
\pgfsetlinewidth{1.003750pt}%
\definecolor{currentstroke}{rgb}{0.121569,0.466667,0.705882}%
\pgfsetstrokecolor{currentstroke}%
\pgfsetstrokeopacity{0.543724}%
\pgfsetdash{}{0pt}%
\pgfpathmoveto{\pgfqpoint{1.265447in}{1.801236in}}%
\pgfpathcurveto{\pgfqpoint{1.273683in}{1.801236in}}{\pgfqpoint{1.281583in}{1.804508in}}{\pgfqpoint{1.287407in}{1.810332in}}%
\pgfpathcurveto{\pgfqpoint{1.293231in}{1.816156in}}{\pgfqpoint{1.296503in}{1.824056in}}{\pgfqpoint{1.296503in}{1.832293in}}%
\pgfpathcurveto{\pgfqpoint{1.296503in}{1.840529in}}{\pgfqpoint{1.293231in}{1.848429in}}{\pgfqpoint{1.287407in}{1.854253in}}%
\pgfpathcurveto{\pgfqpoint{1.281583in}{1.860077in}}{\pgfqpoint{1.273683in}{1.863349in}}{\pgfqpoint{1.265447in}{1.863349in}}%
\pgfpathcurveto{\pgfqpoint{1.257210in}{1.863349in}}{\pgfqpoint{1.249310in}{1.860077in}}{\pgfqpoint{1.243486in}{1.854253in}}%
\pgfpathcurveto{\pgfqpoint{1.237662in}{1.848429in}}{\pgfqpoint{1.234390in}{1.840529in}}{\pgfqpoint{1.234390in}{1.832293in}}%
\pgfpathcurveto{\pgfqpoint{1.234390in}{1.824056in}}{\pgfqpoint{1.237662in}{1.816156in}}{\pgfqpoint{1.243486in}{1.810332in}}%
\pgfpathcurveto{\pgfqpoint{1.249310in}{1.804508in}}{\pgfqpoint{1.257210in}{1.801236in}}{\pgfqpoint{1.265447in}{1.801236in}}%
\pgfpathclose%
\pgfusepath{stroke,fill}%
\end{pgfscope}%
\begin{pgfscope}%
\pgfpathrectangle{\pgfqpoint{0.100000in}{0.212622in}}{\pgfqpoint{3.696000in}{3.696000in}}%
\pgfusepath{clip}%
\pgfsetbuttcap%
\pgfsetroundjoin%
\definecolor{currentfill}{rgb}{0.121569,0.466667,0.705882}%
\pgfsetfillcolor{currentfill}%
\pgfsetfillopacity{0.553548}%
\pgfsetlinewidth{1.003750pt}%
\definecolor{currentstroke}{rgb}{0.121569,0.466667,0.705882}%
\pgfsetstrokecolor{currentstroke}%
\pgfsetstrokeopacity{0.553548}%
\pgfsetdash{}{0pt}%
\pgfpathmoveto{\pgfqpoint{1.247664in}{1.790147in}}%
\pgfpathcurveto{\pgfqpoint{1.255900in}{1.790147in}}{\pgfqpoint{1.263800in}{1.793420in}}{\pgfqpoint{1.269624in}{1.799243in}}%
\pgfpathcurveto{\pgfqpoint{1.275448in}{1.805067in}}{\pgfqpoint{1.278720in}{1.812967in}}{\pgfqpoint{1.278720in}{1.821204in}}%
\pgfpathcurveto{\pgfqpoint{1.278720in}{1.829440in}}{\pgfqpoint{1.275448in}{1.837340in}}{\pgfqpoint{1.269624in}{1.843164in}}%
\pgfpathcurveto{\pgfqpoint{1.263800in}{1.848988in}}{\pgfqpoint{1.255900in}{1.852260in}}{\pgfqpoint{1.247664in}{1.852260in}}%
\pgfpathcurveto{\pgfqpoint{1.239427in}{1.852260in}}{\pgfqpoint{1.231527in}{1.848988in}}{\pgfqpoint{1.225704in}{1.843164in}}%
\pgfpathcurveto{\pgfqpoint{1.219880in}{1.837340in}}{\pgfqpoint{1.216607in}{1.829440in}}{\pgfqpoint{1.216607in}{1.821204in}}%
\pgfpathcurveto{\pgfqpoint{1.216607in}{1.812967in}}{\pgfqpoint{1.219880in}{1.805067in}}{\pgfqpoint{1.225704in}{1.799243in}}%
\pgfpathcurveto{\pgfqpoint{1.231527in}{1.793420in}}{\pgfqpoint{1.239427in}{1.790147in}}{\pgfqpoint{1.247664in}{1.790147in}}%
\pgfpathclose%
\pgfusepath{stroke,fill}%
\end{pgfscope}%
\begin{pgfscope}%
\pgfpathrectangle{\pgfqpoint{0.100000in}{0.212622in}}{\pgfqpoint{3.696000in}{3.696000in}}%
\pgfusepath{clip}%
\pgfsetbuttcap%
\pgfsetroundjoin%
\definecolor{currentfill}{rgb}{0.121569,0.466667,0.705882}%
\pgfsetfillcolor{currentfill}%
\pgfsetfillopacity{0.556023}%
\pgfsetlinewidth{1.003750pt}%
\definecolor{currentstroke}{rgb}{0.121569,0.466667,0.705882}%
\pgfsetstrokecolor{currentstroke}%
\pgfsetstrokeopacity{0.556023}%
\pgfsetdash{}{0pt}%
\pgfpathmoveto{\pgfqpoint{1.244498in}{1.787729in}}%
\pgfpathcurveto{\pgfqpoint{1.252734in}{1.787729in}}{\pgfqpoint{1.260634in}{1.791002in}}{\pgfqpoint{1.266458in}{1.796825in}}%
\pgfpathcurveto{\pgfqpoint{1.272282in}{1.802649in}}{\pgfqpoint{1.275554in}{1.810549in}}{\pgfqpoint{1.275554in}{1.818786in}}%
\pgfpathcurveto{\pgfqpoint{1.275554in}{1.827022in}}{\pgfqpoint{1.272282in}{1.834922in}}{\pgfqpoint{1.266458in}{1.840746in}}%
\pgfpathcurveto{\pgfqpoint{1.260634in}{1.846570in}}{\pgfqpoint{1.252734in}{1.849842in}}{\pgfqpoint{1.244498in}{1.849842in}}%
\pgfpathcurveto{\pgfqpoint{1.236262in}{1.849842in}}{\pgfqpoint{1.228362in}{1.846570in}}{\pgfqpoint{1.222538in}{1.840746in}}%
\pgfpathcurveto{\pgfqpoint{1.216714in}{1.834922in}}{\pgfqpoint{1.213441in}{1.827022in}}{\pgfqpoint{1.213441in}{1.818786in}}%
\pgfpathcurveto{\pgfqpoint{1.213441in}{1.810549in}}{\pgfqpoint{1.216714in}{1.802649in}}{\pgfqpoint{1.222538in}{1.796825in}}%
\pgfpathcurveto{\pgfqpoint{1.228362in}{1.791002in}}{\pgfqpoint{1.236262in}{1.787729in}}{\pgfqpoint{1.244498in}{1.787729in}}%
\pgfpathclose%
\pgfusepath{stroke,fill}%
\end{pgfscope}%
\begin{pgfscope}%
\pgfpathrectangle{\pgfqpoint{0.100000in}{0.212622in}}{\pgfqpoint{3.696000in}{3.696000in}}%
\pgfusepath{clip}%
\pgfsetbuttcap%
\pgfsetroundjoin%
\definecolor{currentfill}{rgb}{0.121569,0.466667,0.705882}%
\pgfsetfillcolor{currentfill}%
\pgfsetfillopacity{0.558868}%
\pgfsetlinewidth{1.003750pt}%
\definecolor{currentstroke}{rgb}{0.121569,0.466667,0.705882}%
\pgfsetstrokecolor{currentstroke}%
\pgfsetstrokeopacity{0.558868}%
\pgfsetdash{}{0pt}%
\pgfpathmoveto{\pgfqpoint{1.242420in}{1.788727in}}%
\pgfpathcurveto{\pgfqpoint{1.250656in}{1.788727in}}{\pgfqpoint{1.258556in}{1.792000in}}{\pgfqpoint{1.264380in}{1.797824in}}%
\pgfpathcurveto{\pgfqpoint{1.270204in}{1.803648in}}{\pgfqpoint{1.273476in}{1.811548in}}{\pgfqpoint{1.273476in}{1.819784in}}%
\pgfpathcurveto{\pgfqpoint{1.273476in}{1.828020in}}{\pgfqpoint{1.270204in}{1.835920in}}{\pgfqpoint{1.264380in}{1.841744in}}%
\pgfpathcurveto{\pgfqpoint{1.258556in}{1.847568in}}{\pgfqpoint{1.250656in}{1.850840in}}{\pgfqpoint{1.242420in}{1.850840in}}%
\pgfpathcurveto{\pgfqpoint{1.234183in}{1.850840in}}{\pgfqpoint{1.226283in}{1.847568in}}{\pgfqpoint{1.220459in}{1.841744in}}%
\pgfpathcurveto{\pgfqpoint{1.214636in}{1.835920in}}{\pgfqpoint{1.211363in}{1.828020in}}{\pgfqpoint{1.211363in}{1.819784in}}%
\pgfpathcurveto{\pgfqpoint{1.211363in}{1.811548in}}{\pgfqpoint{1.214636in}{1.803648in}}{\pgfqpoint{1.220459in}{1.797824in}}%
\pgfpathcurveto{\pgfqpoint{1.226283in}{1.792000in}}{\pgfqpoint{1.234183in}{1.788727in}}{\pgfqpoint{1.242420in}{1.788727in}}%
\pgfpathclose%
\pgfusepath{stroke,fill}%
\end{pgfscope}%
\begin{pgfscope}%
\pgfpathrectangle{\pgfqpoint{0.100000in}{0.212622in}}{\pgfqpoint{3.696000in}{3.696000in}}%
\pgfusepath{clip}%
\pgfsetbuttcap%
\pgfsetroundjoin%
\definecolor{currentfill}{rgb}{0.121569,0.466667,0.705882}%
\pgfsetfillcolor{currentfill}%
\pgfsetfillopacity{0.561025}%
\pgfsetlinewidth{1.003750pt}%
\definecolor{currentstroke}{rgb}{0.121569,0.466667,0.705882}%
\pgfsetstrokecolor{currentstroke}%
\pgfsetstrokeopacity{0.561025}%
\pgfsetdash{}{0pt}%
\pgfpathmoveto{\pgfqpoint{1.244516in}{1.793960in}}%
\pgfpathcurveto{\pgfqpoint{1.252753in}{1.793960in}}{\pgfqpoint{1.260653in}{1.797232in}}{\pgfqpoint{1.266477in}{1.803056in}}%
\pgfpathcurveto{\pgfqpoint{1.272301in}{1.808880in}}{\pgfqpoint{1.275573in}{1.816780in}}{\pgfqpoint{1.275573in}{1.825017in}}%
\pgfpathcurveto{\pgfqpoint{1.275573in}{1.833253in}}{\pgfqpoint{1.272301in}{1.841153in}}{\pgfqpoint{1.266477in}{1.846977in}}%
\pgfpathcurveto{\pgfqpoint{1.260653in}{1.852801in}}{\pgfqpoint{1.252753in}{1.856073in}}{\pgfqpoint{1.244516in}{1.856073in}}%
\pgfpathcurveto{\pgfqpoint{1.236280in}{1.856073in}}{\pgfqpoint{1.228380in}{1.852801in}}{\pgfqpoint{1.222556in}{1.846977in}}%
\pgfpathcurveto{\pgfqpoint{1.216732in}{1.841153in}}{\pgfqpoint{1.213460in}{1.833253in}}{\pgfqpoint{1.213460in}{1.825017in}}%
\pgfpathcurveto{\pgfqpoint{1.213460in}{1.816780in}}{\pgfqpoint{1.216732in}{1.808880in}}{\pgfqpoint{1.222556in}{1.803056in}}%
\pgfpathcurveto{\pgfqpoint{1.228380in}{1.797232in}}{\pgfqpoint{1.236280in}{1.793960in}}{\pgfqpoint{1.244516in}{1.793960in}}%
\pgfpathclose%
\pgfusepath{stroke,fill}%
\end{pgfscope}%
\begin{pgfscope}%
\pgfpathrectangle{\pgfqpoint{0.100000in}{0.212622in}}{\pgfqpoint{3.696000in}{3.696000in}}%
\pgfusepath{clip}%
\pgfsetbuttcap%
\pgfsetroundjoin%
\definecolor{currentfill}{rgb}{0.121569,0.466667,0.705882}%
\pgfsetfillcolor{currentfill}%
\pgfsetfillopacity{0.561042}%
\pgfsetlinewidth{1.003750pt}%
\definecolor{currentstroke}{rgb}{0.121569,0.466667,0.705882}%
\pgfsetstrokecolor{currentstroke}%
\pgfsetstrokeopacity{0.561042}%
\pgfsetdash{}{0pt}%
\pgfpathmoveto{\pgfqpoint{1.248704in}{1.799711in}}%
\pgfpathcurveto{\pgfqpoint{1.256940in}{1.799711in}}{\pgfqpoint{1.264840in}{1.802983in}}{\pgfqpoint{1.270664in}{1.808807in}}%
\pgfpathcurveto{\pgfqpoint{1.276488in}{1.814631in}}{\pgfqpoint{1.279760in}{1.822531in}}{\pgfqpoint{1.279760in}{1.830767in}}%
\pgfpathcurveto{\pgfqpoint{1.279760in}{1.839003in}}{\pgfqpoint{1.276488in}{1.846903in}}{\pgfqpoint{1.270664in}{1.852727in}}%
\pgfpathcurveto{\pgfqpoint{1.264840in}{1.858551in}}{\pgfqpoint{1.256940in}{1.861824in}}{\pgfqpoint{1.248704in}{1.861824in}}%
\pgfpathcurveto{\pgfqpoint{1.240468in}{1.861824in}}{\pgfqpoint{1.232567in}{1.858551in}}{\pgfqpoint{1.226744in}{1.852727in}}%
\pgfpathcurveto{\pgfqpoint{1.220920in}{1.846903in}}{\pgfqpoint{1.217647in}{1.839003in}}{\pgfqpoint{1.217647in}{1.830767in}}%
\pgfpathcurveto{\pgfqpoint{1.217647in}{1.822531in}}{\pgfqpoint{1.220920in}{1.814631in}}{\pgfqpoint{1.226744in}{1.808807in}}%
\pgfpathcurveto{\pgfqpoint{1.232567in}{1.802983in}}{\pgfqpoint{1.240468in}{1.799711in}}{\pgfqpoint{1.248704in}{1.799711in}}%
\pgfpathclose%
\pgfusepath{stroke,fill}%
\end{pgfscope}%
\begin{pgfscope}%
\pgfpathrectangle{\pgfqpoint{0.100000in}{0.212622in}}{\pgfqpoint{3.696000in}{3.696000in}}%
\pgfusepath{clip}%
\pgfsetbuttcap%
\pgfsetroundjoin%
\definecolor{currentfill}{rgb}{0.121569,0.466667,0.705882}%
\pgfsetfillcolor{currentfill}%
\pgfsetfillopacity{0.562523}%
\pgfsetlinewidth{1.003750pt}%
\definecolor{currentstroke}{rgb}{0.121569,0.466667,0.705882}%
\pgfsetstrokecolor{currentstroke}%
\pgfsetstrokeopacity{0.562523}%
\pgfsetdash{}{0pt}%
\pgfpathmoveto{\pgfqpoint{1.252152in}{1.807401in}}%
\pgfpathcurveto{\pgfqpoint{1.260388in}{1.807401in}}{\pgfqpoint{1.268288in}{1.810673in}}{\pgfqpoint{1.274112in}{1.816497in}}%
\pgfpathcurveto{\pgfqpoint{1.279936in}{1.822321in}}{\pgfqpoint{1.283208in}{1.830221in}}{\pgfqpoint{1.283208in}{1.838457in}}%
\pgfpathcurveto{\pgfqpoint{1.283208in}{1.846693in}}{\pgfqpoint{1.279936in}{1.854594in}}{\pgfqpoint{1.274112in}{1.860417in}}%
\pgfpathcurveto{\pgfqpoint{1.268288in}{1.866241in}}{\pgfqpoint{1.260388in}{1.869514in}}{\pgfqpoint{1.252152in}{1.869514in}}%
\pgfpathcurveto{\pgfqpoint{1.243915in}{1.869514in}}{\pgfqpoint{1.236015in}{1.866241in}}{\pgfqpoint{1.230191in}{1.860417in}}%
\pgfpathcurveto{\pgfqpoint{1.224367in}{1.854594in}}{\pgfqpoint{1.221095in}{1.846693in}}{\pgfqpoint{1.221095in}{1.838457in}}%
\pgfpathcurveto{\pgfqpoint{1.221095in}{1.830221in}}{\pgfqpoint{1.224367in}{1.822321in}}{\pgfqpoint{1.230191in}{1.816497in}}%
\pgfpathcurveto{\pgfqpoint{1.236015in}{1.810673in}}{\pgfqpoint{1.243915in}{1.807401in}}{\pgfqpoint{1.252152in}{1.807401in}}%
\pgfpathclose%
\pgfusepath{stroke,fill}%
\end{pgfscope}%
\begin{pgfscope}%
\pgfpathrectangle{\pgfqpoint{0.100000in}{0.212622in}}{\pgfqpoint{3.696000in}{3.696000in}}%
\pgfusepath{clip}%
\pgfsetbuttcap%
\pgfsetroundjoin%
\definecolor{currentfill}{rgb}{0.121569,0.466667,0.705882}%
\pgfsetfillcolor{currentfill}%
\pgfsetfillopacity{0.566048}%
\pgfsetlinewidth{1.003750pt}%
\definecolor{currentstroke}{rgb}{0.121569,0.466667,0.705882}%
\pgfsetstrokecolor{currentstroke}%
\pgfsetstrokeopacity{0.566048}%
\pgfsetdash{}{0pt}%
\pgfpathmoveto{\pgfqpoint{1.284153in}{1.850471in}}%
\pgfpathcurveto{\pgfqpoint{1.292390in}{1.850471in}}{\pgfqpoint{1.300290in}{1.853743in}}{\pgfqpoint{1.306113in}{1.859567in}}%
\pgfpathcurveto{\pgfqpoint{1.311937in}{1.865391in}}{\pgfqpoint{1.315210in}{1.873291in}}{\pgfqpoint{1.315210in}{1.881528in}}%
\pgfpathcurveto{\pgfqpoint{1.315210in}{1.889764in}}{\pgfqpoint{1.311937in}{1.897664in}}{\pgfqpoint{1.306113in}{1.903488in}}%
\pgfpathcurveto{\pgfqpoint{1.300290in}{1.909312in}}{\pgfqpoint{1.292390in}{1.912584in}}{\pgfqpoint{1.284153in}{1.912584in}}%
\pgfpathcurveto{\pgfqpoint{1.275917in}{1.912584in}}{\pgfqpoint{1.268017in}{1.909312in}}{\pgfqpoint{1.262193in}{1.903488in}}%
\pgfpathcurveto{\pgfqpoint{1.256369in}{1.897664in}}{\pgfqpoint{1.253097in}{1.889764in}}{\pgfqpoint{1.253097in}{1.881528in}}%
\pgfpathcurveto{\pgfqpoint{1.253097in}{1.873291in}}{\pgfqpoint{1.256369in}{1.865391in}}{\pgfqpoint{1.262193in}{1.859567in}}%
\pgfpathcurveto{\pgfqpoint{1.268017in}{1.853743in}}{\pgfqpoint{1.275917in}{1.850471in}}{\pgfqpoint{1.284153in}{1.850471in}}%
\pgfpathclose%
\pgfusepath{stroke,fill}%
\end{pgfscope}%
\begin{pgfscope}%
\pgfpathrectangle{\pgfqpoint{0.100000in}{0.212622in}}{\pgfqpoint{3.696000in}{3.696000in}}%
\pgfusepath{clip}%
\pgfsetbuttcap%
\pgfsetroundjoin%
\definecolor{currentfill}{rgb}{0.121569,0.466667,0.705882}%
\pgfsetfillcolor{currentfill}%
\pgfsetfillopacity{0.569795}%
\pgfsetlinewidth{1.003750pt}%
\definecolor{currentstroke}{rgb}{0.121569,0.466667,0.705882}%
\pgfsetstrokecolor{currentstroke}%
\pgfsetstrokeopacity{0.569795}%
\pgfsetdash{}{0pt}%
\pgfpathmoveto{\pgfqpoint{1.253102in}{1.812541in}}%
\pgfpathcurveto{\pgfqpoint{1.261338in}{1.812541in}}{\pgfqpoint{1.269238in}{1.815813in}}{\pgfqpoint{1.275062in}{1.821637in}}%
\pgfpathcurveto{\pgfqpoint{1.280886in}{1.827461in}}{\pgfqpoint{1.284158in}{1.835361in}}{\pgfqpoint{1.284158in}{1.843597in}}%
\pgfpathcurveto{\pgfqpoint{1.284158in}{1.851834in}}{\pgfqpoint{1.280886in}{1.859734in}}{\pgfqpoint{1.275062in}{1.865558in}}%
\pgfpathcurveto{\pgfqpoint{1.269238in}{1.871382in}}{\pgfqpoint{1.261338in}{1.874654in}}{\pgfqpoint{1.253102in}{1.874654in}}%
\pgfpathcurveto{\pgfqpoint{1.244865in}{1.874654in}}{\pgfqpoint{1.236965in}{1.871382in}}{\pgfqpoint{1.231141in}{1.865558in}}%
\pgfpathcurveto{\pgfqpoint{1.225317in}{1.859734in}}{\pgfqpoint{1.222045in}{1.851834in}}{\pgfqpoint{1.222045in}{1.843597in}}%
\pgfpathcurveto{\pgfqpoint{1.222045in}{1.835361in}}{\pgfqpoint{1.225317in}{1.827461in}}{\pgfqpoint{1.231141in}{1.821637in}}%
\pgfpathcurveto{\pgfqpoint{1.236965in}{1.815813in}}{\pgfqpoint{1.244865in}{1.812541in}}{\pgfqpoint{1.253102in}{1.812541in}}%
\pgfpathclose%
\pgfusepath{stroke,fill}%
\end{pgfscope}%
\begin{pgfscope}%
\pgfpathrectangle{\pgfqpoint{0.100000in}{0.212622in}}{\pgfqpoint{3.696000in}{3.696000in}}%
\pgfusepath{clip}%
\pgfsetbuttcap%
\pgfsetroundjoin%
\definecolor{currentfill}{rgb}{0.121569,0.466667,0.705882}%
\pgfsetfillcolor{currentfill}%
\pgfsetfillopacity{0.570476}%
\pgfsetlinewidth{1.003750pt}%
\definecolor{currentstroke}{rgb}{0.121569,0.466667,0.705882}%
\pgfsetstrokecolor{currentstroke}%
\pgfsetstrokeopacity{0.570476}%
\pgfsetdash{}{0pt}%
\pgfpathmoveto{\pgfqpoint{1.239544in}{1.796789in}}%
\pgfpathcurveto{\pgfqpoint{1.247781in}{1.796789in}}{\pgfqpoint{1.255681in}{1.800062in}}{\pgfqpoint{1.261505in}{1.805886in}}%
\pgfpathcurveto{\pgfqpoint{1.267329in}{1.811710in}}{\pgfqpoint{1.270601in}{1.819610in}}{\pgfqpoint{1.270601in}{1.827846in}}%
\pgfpathcurveto{\pgfqpoint{1.270601in}{1.836082in}}{\pgfqpoint{1.267329in}{1.843982in}}{\pgfqpoint{1.261505in}{1.849806in}}%
\pgfpathcurveto{\pgfqpoint{1.255681in}{1.855630in}}{\pgfqpoint{1.247781in}{1.858902in}}{\pgfqpoint{1.239544in}{1.858902in}}%
\pgfpathcurveto{\pgfqpoint{1.231308in}{1.858902in}}{\pgfqpoint{1.223408in}{1.855630in}}{\pgfqpoint{1.217584in}{1.849806in}}%
\pgfpathcurveto{\pgfqpoint{1.211760in}{1.843982in}}{\pgfqpoint{1.208488in}{1.836082in}}{\pgfqpoint{1.208488in}{1.827846in}}%
\pgfpathcurveto{\pgfqpoint{1.208488in}{1.819610in}}{\pgfqpoint{1.211760in}{1.811710in}}{\pgfqpoint{1.217584in}{1.805886in}}%
\pgfpathcurveto{\pgfqpoint{1.223408in}{1.800062in}}{\pgfqpoint{1.231308in}{1.796789in}}{\pgfqpoint{1.239544in}{1.796789in}}%
\pgfpathclose%
\pgfusepath{stroke,fill}%
\end{pgfscope}%
\begin{pgfscope}%
\pgfpathrectangle{\pgfqpoint{0.100000in}{0.212622in}}{\pgfqpoint{3.696000in}{3.696000in}}%
\pgfusepath{clip}%
\pgfsetbuttcap%
\pgfsetroundjoin%
\definecolor{currentfill}{rgb}{0.121569,0.466667,0.705882}%
\pgfsetfillcolor{currentfill}%
\pgfsetfillopacity{0.571924}%
\pgfsetlinewidth{1.003750pt}%
\definecolor{currentstroke}{rgb}{0.121569,0.466667,0.705882}%
\pgfsetstrokecolor{currentstroke}%
\pgfsetstrokeopacity{0.571924}%
\pgfsetdash{}{0pt}%
\pgfpathmoveto{\pgfqpoint{1.239700in}{1.796362in}}%
\pgfpathcurveto{\pgfqpoint{1.247936in}{1.796362in}}{\pgfqpoint{1.255836in}{1.799634in}}{\pgfqpoint{1.261660in}{1.805458in}}%
\pgfpathcurveto{\pgfqpoint{1.267484in}{1.811282in}}{\pgfqpoint{1.270757in}{1.819182in}}{\pgfqpoint{1.270757in}{1.827418in}}%
\pgfpathcurveto{\pgfqpoint{1.270757in}{1.835654in}}{\pgfqpoint{1.267484in}{1.843554in}}{\pgfqpoint{1.261660in}{1.849378in}}%
\pgfpathcurveto{\pgfqpoint{1.255836in}{1.855202in}}{\pgfqpoint{1.247936in}{1.858475in}}{\pgfqpoint{1.239700in}{1.858475in}}%
\pgfpathcurveto{\pgfqpoint{1.231464in}{1.858475in}}{\pgfqpoint{1.223564in}{1.855202in}}{\pgfqpoint{1.217740in}{1.849378in}}%
\pgfpathcurveto{\pgfqpoint{1.211916in}{1.843554in}}{\pgfqpoint{1.208644in}{1.835654in}}{\pgfqpoint{1.208644in}{1.827418in}}%
\pgfpathcurveto{\pgfqpoint{1.208644in}{1.819182in}}{\pgfqpoint{1.211916in}{1.811282in}}{\pgfqpoint{1.217740in}{1.805458in}}%
\pgfpathcurveto{\pgfqpoint{1.223564in}{1.799634in}}{\pgfqpoint{1.231464in}{1.796362in}}{\pgfqpoint{1.239700in}{1.796362in}}%
\pgfpathclose%
\pgfusepath{stroke,fill}%
\end{pgfscope}%
\begin{pgfscope}%
\pgfpathrectangle{\pgfqpoint{0.100000in}{0.212622in}}{\pgfqpoint{3.696000in}{3.696000in}}%
\pgfusepath{clip}%
\pgfsetbuttcap%
\pgfsetroundjoin%
\definecolor{currentfill}{rgb}{0.121569,0.466667,0.705882}%
\pgfsetfillcolor{currentfill}%
\pgfsetfillopacity{0.576861}%
\pgfsetlinewidth{1.003750pt}%
\definecolor{currentstroke}{rgb}{0.121569,0.466667,0.705882}%
\pgfsetstrokecolor{currentstroke}%
\pgfsetstrokeopacity{0.576861}%
\pgfsetdash{}{0pt}%
\pgfpathmoveto{\pgfqpoint{1.267504in}{1.832352in}}%
\pgfpathcurveto{\pgfqpoint{1.275741in}{1.832352in}}{\pgfqpoint{1.283641in}{1.835624in}}{\pgfqpoint{1.289465in}{1.841448in}}%
\pgfpathcurveto{\pgfqpoint{1.295288in}{1.847272in}}{\pgfqpoint{1.298561in}{1.855172in}}{\pgfqpoint{1.298561in}{1.863409in}}%
\pgfpathcurveto{\pgfqpoint{1.298561in}{1.871645in}}{\pgfqpoint{1.295288in}{1.879545in}}{\pgfqpoint{1.289465in}{1.885369in}}%
\pgfpathcurveto{\pgfqpoint{1.283641in}{1.891193in}}{\pgfqpoint{1.275741in}{1.894465in}}{\pgfqpoint{1.267504in}{1.894465in}}%
\pgfpathcurveto{\pgfqpoint{1.259268in}{1.894465in}}{\pgfqpoint{1.251368in}{1.891193in}}{\pgfqpoint{1.245544in}{1.885369in}}%
\pgfpathcurveto{\pgfqpoint{1.239720in}{1.879545in}}{\pgfqpoint{1.236448in}{1.871645in}}{\pgfqpoint{1.236448in}{1.863409in}}%
\pgfpathcurveto{\pgfqpoint{1.236448in}{1.855172in}}{\pgfqpoint{1.239720in}{1.847272in}}{\pgfqpoint{1.245544in}{1.841448in}}%
\pgfpathcurveto{\pgfqpoint{1.251368in}{1.835624in}}{\pgfqpoint{1.259268in}{1.832352in}}{\pgfqpoint{1.267504in}{1.832352in}}%
\pgfpathclose%
\pgfusepath{stroke,fill}%
\end{pgfscope}%
\begin{pgfscope}%
\pgfpathrectangle{\pgfqpoint{0.100000in}{0.212622in}}{\pgfqpoint{3.696000in}{3.696000in}}%
\pgfusepath{clip}%
\pgfsetbuttcap%
\pgfsetroundjoin%
\definecolor{currentfill}{rgb}{0.121569,0.466667,0.705882}%
\pgfsetfillcolor{currentfill}%
\pgfsetfillopacity{0.577882}%
\pgfsetlinewidth{1.003750pt}%
\definecolor{currentstroke}{rgb}{0.121569,0.466667,0.705882}%
\pgfsetstrokecolor{currentstroke}%
\pgfsetstrokeopacity{0.577882}%
\pgfsetdash{}{0pt}%
\pgfpathmoveto{\pgfqpoint{1.317381in}{1.895583in}}%
\pgfpathcurveto{\pgfqpoint{1.325617in}{1.895583in}}{\pgfqpoint{1.333517in}{1.898856in}}{\pgfqpoint{1.339341in}{1.904680in}}%
\pgfpathcurveto{\pgfqpoint{1.345165in}{1.910504in}}{\pgfqpoint{1.348437in}{1.918404in}}{\pgfqpoint{1.348437in}{1.926640in}}%
\pgfpathcurveto{\pgfqpoint{1.348437in}{1.934876in}}{\pgfqpoint{1.345165in}{1.942776in}}{\pgfqpoint{1.339341in}{1.948600in}}%
\pgfpathcurveto{\pgfqpoint{1.333517in}{1.954424in}}{\pgfqpoint{1.325617in}{1.957696in}}{\pgfqpoint{1.317381in}{1.957696in}}%
\pgfpathcurveto{\pgfqpoint{1.309145in}{1.957696in}}{\pgfqpoint{1.301245in}{1.954424in}}{\pgfqpoint{1.295421in}{1.948600in}}%
\pgfpathcurveto{\pgfqpoint{1.289597in}{1.942776in}}{\pgfqpoint{1.286324in}{1.934876in}}{\pgfqpoint{1.286324in}{1.926640in}}%
\pgfpathcurveto{\pgfqpoint{1.286324in}{1.918404in}}{\pgfqpoint{1.289597in}{1.910504in}}{\pgfqpoint{1.295421in}{1.904680in}}%
\pgfpathcurveto{\pgfqpoint{1.301245in}{1.898856in}}{\pgfqpoint{1.309145in}{1.895583in}}{\pgfqpoint{1.317381in}{1.895583in}}%
\pgfpathclose%
\pgfusepath{stroke,fill}%
\end{pgfscope}%
\begin{pgfscope}%
\pgfpathrectangle{\pgfqpoint{0.100000in}{0.212622in}}{\pgfqpoint{3.696000in}{3.696000in}}%
\pgfusepath{clip}%
\pgfsetbuttcap%
\pgfsetroundjoin%
\definecolor{currentfill}{rgb}{0.121569,0.466667,0.705882}%
\pgfsetfillcolor{currentfill}%
\pgfsetfillopacity{0.578362}%
\pgfsetlinewidth{1.003750pt}%
\definecolor{currentstroke}{rgb}{0.121569,0.466667,0.705882}%
\pgfsetstrokecolor{currentstroke}%
\pgfsetstrokeopacity{0.578362}%
\pgfsetdash{}{0pt}%
\pgfpathmoveto{\pgfqpoint{1.238220in}{1.795772in}}%
\pgfpathcurveto{\pgfqpoint{1.246456in}{1.795772in}}{\pgfqpoint{1.254356in}{1.799044in}}{\pgfqpoint{1.260180in}{1.804868in}}%
\pgfpathcurveto{\pgfqpoint{1.266004in}{1.810692in}}{\pgfqpoint{1.269276in}{1.818592in}}{\pgfqpoint{1.269276in}{1.826828in}}%
\pgfpathcurveto{\pgfqpoint{1.269276in}{1.835064in}}{\pgfqpoint{1.266004in}{1.842965in}}{\pgfqpoint{1.260180in}{1.848788in}}%
\pgfpathcurveto{\pgfqpoint{1.254356in}{1.854612in}}{\pgfqpoint{1.246456in}{1.857885in}}{\pgfqpoint{1.238220in}{1.857885in}}%
\pgfpathcurveto{\pgfqpoint{1.229984in}{1.857885in}}{\pgfqpoint{1.222084in}{1.854612in}}{\pgfqpoint{1.216260in}{1.848788in}}%
\pgfpathcurveto{\pgfqpoint{1.210436in}{1.842965in}}{\pgfqpoint{1.207163in}{1.835064in}}{\pgfqpoint{1.207163in}{1.826828in}}%
\pgfpathcurveto{\pgfqpoint{1.207163in}{1.818592in}}{\pgfqpoint{1.210436in}{1.810692in}}{\pgfqpoint{1.216260in}{1.804868in}}%
\pgfpathcurveto{\pgfqpoint{1.222084in}{1.799044in}}{\pgfqpoint{1.229984in}{1.795772in}}{\pgfqpoint{1.238220in}{1.795772in}}%
\pgfpathclose%
\pgfusepath{stroke,fill}%
\end{pgfscope}%
\begin{pgfscope}%
\pgfpathrectangle{\pgfqpoint{0.100000in}{0.212622in}}{\pgfqpoint{3.696000in}{3.696000in}}%
\pgfusepath{clip}%
\pgfsetbuttcap%
\pgfsetroundjoin%
\definecolor{currentfill}{rgb}{0.121569,0.466667,0.705882}%
\pgfsetfillcolor{currentfill}%
\pgfsetfillopacity{0.581067}%
\pgfsetlinewidth{1.003750pt}%
\definecolor{currentstroke}{rgb}{0.121569,0.466667,0.705882}%
\pgfsetstrokecolor{currentstroke}%
\pgfsetstrokeopacity{0.581067}%
\pgfsetdash{}{0pt}%
\pgfpathmoveto{\pgfqpoint{1.271821in}{1.841163in}}%
\pgfpathcurveto{\pgfqpoint{1.280057in}{1.841163in}}{\pgfqpoint{1.287957in}{1.844436in}}{\pgfqpoint{1.293781in}{1.850260in}}%
\pgfpathcurveto{\pgfqpoint{1.299605in}{1.856084in}}{\pgfqpoint{1.302877in}{1.863984in}}{\pgfqpoint{1.302877in}{1.872220in}}%
\pgfpathcurveto{\pgfqpoint{1.302877in}{1.880456in}}{\pgfqpoint{1.299605in}{1.888356in}}{\pgfqpoint{1.293781in}{1.894180in}}%
\pgfpathcurveto{\pgfqpoint{1.287957in}{1.900004in}}{\pgfqpoint{1.280057in}{1.903276in}}{\pgfqpoint{1.271821in}{1.903276in}}%
\pgfpathcurveto{\pgfqpoint{1.263584in}{1.903276in}}{\pgfqpoint{1.255684in}{1.900004in}}{\pgfqpoint{1.249860in}{1.894180in}}%
\pgfpathcurveto{\pgfqpoint{1.244036in}{1.888356in}}{\pgfqpoint{1.240764in}{1.880456in}}{\pgfqpoint{1.240764in}{1.872220in}}%
\pgfpathcurveto{\pgfqpoint{1.240764in}{1.863984in}}{\pgfqpoint{1.244036in}{1.856084in}}{\pgfqpoint{1.249860in}{1.850260in}}%
\pgfpathcurveto{\pgfqpoint{1.255684in}{1.844436in}}{\pgfqpoint{1.263584in}{1.841163in}}{\pgfqpoint{1.271821in}{1.841163in}}%
\pgfpathclose%
\pgfusepath{stroke,fill}%
\end{pgfscope}%
\begin{pgfscope}%
\pgfpathrectangle{\pgfqpoint{0.100000in}{0.212622in}}{\pgfqpoint{3.696000in}{3.696000in}}%
\pgfusepath{clip}%
\pgfsetbuttcap%
\pgfsetroundjoin%
\definecolor{currentfill}{rgb}{0.121569,0.466667,0.705882}%
\pgfsetfillcolor{currentfill}%
\pgfsetfillopacity{0.585267}%
\pgfsetlinewidth{1.003750pt}%
\definecolor{currentstroke}{rgb}{0.121569,0.466667,0.705882}%
\pgfsetstrokecolor{currentstroke}%
\pgfsetstrokeopacity{0.585267}%
\pgfsetdash{}{0pt}%
\pgfpathmoveto{\pgfqpoint{1.273516in}{1.837887in}}%
\pgfpathcurveto{\pgfqpoint{1.281753in}{1.837887in}}{\pgfqpoint{1.289653in}{1.841159in}}{\pgfqpoint{1.295477in}{1.846983in}}%
\pgfpathcurveto{\pgfqpoint{1.301301in}{1.852807in}}{\pgfqpoint{1.304573in}{1.860707in}}{\pgfqpoint{1.304573in}{1.868944in}}%
\pgfpathcurveto{\pgfqpoint{1.304573in}{1.877180in}}{\pgfqpoint{1.301301in}{1.885080in}}{\pgfqpoint{1.295477in}{1.890904in}}%
\pgfpathcurveto{\pgfqpoint{1.289653in}{1.896728in}}{\pgfqpoint{1.281753in}{1.900000in}}{\pgfqpoint{1.273516in}{1.900000in}}%
\pgfpathcurveto{\pgfqpoint{1.265280in}{1.900000in}}{\pgfqpoint{1.257380in}{1.896728in}}{\pgfqpoint{1.251556in}{1.890904in}}%
\pgfpathcurveto{\pgfqpoint{1.245732in}{1.885080in}}{\pgfqpoint{1.242460in}{1.877180in}}{\pgfqpoint{1.242460in}{1.868944in}}%
\pgfpathcurveto{\pgfqpoint{1.242460in}{1.860707in}}{\pgfqpoint{1.245732in}{1.852807in}}{\pgfqpoint{1.251556in}{1.846983in}}%
\pgfpathcurveto{\pgfqpoint{1.257380in}{1.841159in}}{\pgfqpoint{1.265280in}{1.837887in}}{\pgfqpoint{1.273516in}{1.837887in}}%
\pgfpathclose%
\pgfusepath{stroke,fill}%
\end{pgfscope}%
\begin{pgfscope}%
\pgfpathrectangle{\pgfqpoint{0.100000in}{0.212622in}}{\pgfqpoint{3.696000in}{3.696000in}}%
\pgfusepath{clip}%
\pgfsetbuttcap%
\pgfsetroundjoin%
\definecolor{currentfill}{rgb}{0.121569,0.466667,0.705882}%
\pgfsetfillcolor{currentfill}%
\pgfsetfillopacity{0.589746}%
\pgfsetlinewidth{1.003750pt}%
\definecolor{currentstroke}{rgb}{0.121569,0.466667,0.705882}%
\pgfsetstrokecolor{currentstroke}%
\pgfsetstrokeopacity{0.589746}%
\pgfsetdash{}{0pt}%
\pgfpathmoveto{\pgfqpoint{1.343534in}{1.915797in}}%
\pgfpathcurveto{\pgfqpoint{1.351770in}{1.915797in}}{\pgfqpoint{1.359670in}{1.919069in}}{\pgfqpoint{1.365494in}{1.924893in}}%
\pgfpathcurveto{\pgfqpoint{1.371318in}{1.930717in}}{\pgfqpoint{1.374590in}{1.938617in}}{\pgfqpoint{1.374590in}{1.946854in}}%
\pgfpathcurveto{\pgfqpoint{1.374590in}{1.955090in}}{\pgfqpoint{1.371318in}{1.962990in}}{\pgfqpoint{1.365494in}{1.968814in}}%
\pgfpathcurveto{\pgfqpoint{1.359670in}{1.974638in}}{\pgfqpoint{1.351770in}{1.977910in}}{\pgfqpoint{1.343534in}{1.977910in}}%
\pgfpathcurveto{\pgfqpoint{1.335298in}{1.977910in}}{\pgfqpoint{1.327398in}{1.974638in}}{\pgfqpoint{1.321574in}{1.968814in}}%
\pgfpathcurveto{\pgfqpoint{1.315750in}{1.962990in}}{\pgfqpoint{1.312477in}{1.955090in}}{\pgfqpoint{1.312477in}{1.946854in}}%
\pgfpathcurveto{\pgfqpoint{1.312477in}{1.938617in}}{\pgfqpoint{1.315750in}{1.930717in}}{\pgfqpoint{1.321574in}{1.924893in}}%
\pgfpathcurveto{\pgfqpoint{1.327398in}{1.919069in}}{\pgfqpoint{1.335298in}{1.915797in}}{\pgfqpoint{1.343534in}{1.915797in}}%
\pgfpathclose%
\pgfusepath{stroke,fill}%
\end{pgfscope}%
\begin{pgfscope}%
\pgfpathrectangle{\pgfqpoint{0.100000in}{0.212622in}}{\pgfqpoint{3.696000in}{3.696000in}}%
\pgfusepath{clip}%
\pgfsetbuttcap%
\pgfsetroundjoin%
\definecolor{currentfill}{rgb}{0.121569,0.466667,0.705882}%
\pgfsetfillcolor{currentfill}%
\pgfsetfillopacity{0.592582}%
\pgfsetlinewidth{1.003750pt}%
\definecolor{currentstroke}{rgb}{0.121569,0.466667,0.705882}%
\pgfsetstrokecolor{currentstroke}%
\pgfsetstrokeopacity{0.592582}%
\pgfsetdash{}{0pt}%
\pgfpathmoveto{\pgfqpoint{1.310141in}{1.887148in}}%
\pgfpathcurveto{\pgfqpoint{1.318377in}{1.887148in}}{\pgfqpoint{1.326277in}{1.890420in}}{\pgfqpoint{1.332101in}{1.896244in}}%
\pgfpathcurveto{\pgfqpoint{1.337925in}{1.902068in}}{\pgfqpoint{1.341197in}{1.909968in}}{\pgfqpoint{1.341197in}{1.918205in}}%
\pgfpathcurveto{\pgfqpoint{1.341197in}{1.926441in}}{\pgfqpoint{1.337925in}{1.934341in}}{\pgfqpoint{1.332101in}{1.940165in}}%
\pgfpathcurveto{\pgfqpoint{1.326277in}{1.945989in}}{\pgfqpoint{1.318377in}{1.949261in}}{\pgfqpoint{1.310141in}{1.949261in}}%
\pgfpathcurveto{\pgfqpoint{1.301904in}{1.949261in}}{\pgfqpoint{1.294004in}{1.945989in}}{\pgfqpoint{1.288180in}{1.940165in}}%
\pgfpathcurveto{\pgfqpoint{1.282356in}{1.934341in}}{\pgfqpoint{1.279084in}{1.926441in}}{\pgfqpoint{1.279084in}{1.918205in}}%
\pgfpathcurveto{\pgfqpoint{1.279084in}{1.909968in}}{\pgfqpoint{1.282356in}{1.902068in}}{\pgfqpoint{1.288180in}{1.896244in}}%
\pgfpathcurveto{\pgfqpoint{1.294004in}{1.890420in}}{\pgfqpoint{1.301904in}{1.887148in}}{\pgfqpoint{1.310141in}{1.887148in}}%
\pgfpathclose%
\pgfusepath{stroke,fill}%
\end{pgfscope}%
\begin{pgfscope}%
\pgfpathrectangle{\pgfqpoint{0.100000in}{0.212622in}}{\pgfqpoint{3.696000in}{3.696000in}}%
\pgfusepath{clip}%
\pgfsetbuttcap%
\pgfsetroundjoin%
\definecolor{currentfill}{rgb}{0.121569,0.466667,0.705882}%
\pgfsetfillcolor{currentfill}%
\pgfsetfillopacity{0.593374}%
\pgfsetlinewidth{1.003750pt}%
\definecolor{currentstroke}{rgb}{0.121569,0.466667,0.705882}%
\pgfsetstrokecolor{currentstroke}%
\pgfsetstrokeopacity{0.593374}%
\pgfsetdash{}{0pt}%
\pgfpathmoveto{\pgfqpoint{1.445296in}{2.010664in}}%
\pgfpathcurveto{\pgfqpoint{1.453532in}{2.010664in}}{\pgfqpoint{1.461432in}{2.013937in}}{\pgfqpoint{1.467256in}{2.019761in}}%
\pgfpathcurveto{\pgfqpoint{1.473080in}{2.025584in}}{\pgfqpoint{1.476352in}{2.033484in}}{\pgfqpoint{1.476352in}{2.041721in}}%
\pgfpathcurveto{\pgfqpoint{1.476352in}{2.049957in}}{\pgfqpoint{1.473080in}{2.057857in}}{\pgfqpoint{1.467256in}{2.063681in}}%
\pgfpathcurveto{\pgfqpoint{1.461432in}{2.069505in}}{\pgfqpoint{1.453532in}{2.072777in}}{\pgfqpoint{1.445296in}{2.072777in}}%
\pgfpathcurveto{\pgfqpoint{1.437060in}{2.072777in}}{\pgfqpoint{1.429159in}{2.069505in}}{\pgfqpoint{1.423336in}{2.063681in}}%
\pgfpathcurveto{\pgfqpoint{1.417512in}{2.057857in}}{\pgfqpoint{1.414239in}{2.049957in}}{\pgfqpoint{1.414239in}{2.041721in}}%
\pgfpathcurveto{\pgfqpoint{1.414239in}{2.033484in}}{\pgfqpoint{1.417512in}{2.025584in}}{\pgfqpoint{1.423336in}{2.019761in}}%
\pgfpathcurveto{\pgfqpoint{1.429159in}{2.013937in}}{\pgfqpoint{1.437060in}{2.010664in}}{\pgfqpoint{1.445296in}{2.010664in}}%
\pgfpathclose%
\pgfusepath{stroke,fill}%
\end{pgfscope}%
\begin{pgfscope}%
\pgfpathrectangle{\pgfqpoint{0.100000in}{0.212622in}}{\pgfqpoint{3.696000in}{3.696000in}}%
\pgfusepath{clip}%
\pgfsetbuttcap%
\pgfsetroundjoin%
\definecolor{currentfill}{rgb}{0.121569,0.466667,0.705882}%
\pgfsetfillcolor{currentfill}%
\pgfsetfillopacity{0.593562}%
\pgfsetlinewidth{1.003750pt}%
\definecolor{currentstroke}{rgb}{0.121569,0.466667,0.705882}%
\pgfsetstrokecolor{currentstroke}%
\pgfsetstrokeopacity{0.593562}%
\pgfsetdash{}{0pt}%
\pgfpathmoveto{\pgfqpoint{1.319301in}{1.893341in}}%
\pgfpathcurveto{\pgfqpoint{1.327537in}{1.893341in}}{\pgfqpoint{1.335437in}{1.896613in}}{\pgfqpoint{1.341261in}{1.902437in}}%
\pgfpathcurveto{\pgfqpoint{1.347085in}{1.908261in}}{\pgfqpoint{1.350357in}{1.916161in}}{\pgfqpoint{1.350357in}{1.924397in}}%
\pgfpathcurveto{\pgfqpoint{1.350357in}{1.932633in}}{\pgfqpoint{1.347085in}{1.940534in}}{\pgfqpoint{1.341261in}{1.946357in}}%
\pgfpathcurveto{\pgfqpoint{1.335437in}{1.952181in}}{\pgfqpoint{1.327537in}{1.955454in}}{\pgfqpoint{1.319301in}{1.955454in}}%
\pgfpathcurveto{\pgfqpoint{1.311064in}{1.955454in}}{\pgfqpoint{1.303164in}{1.952181in}}{\pgfqpoint{1.297340in}{1.946357in}}%
\pgfpathcurveto{\pgfqpoint{1.291516in}{1.940534in}}{\pgfqpoint{1.288244in}{1.932633in}}{\pgfqpoint{1.288244in}{1.924397in}}%
\pgfpathcurveto{\pgfqpoint{1.288244in}{1.916161in}}{\pgfqpoint{1.291516in}{1.908261in}}{\pgfqpoint{1.297340in}{1.902437in}}%
\pgfpathcurveto{\pgfqpoint{1.303164in}{1.896613in}}{\pgfqpoint{1.311064in}{1.893341in}}{\pgfqpoint{1.319301in}{1.893341in}}%
\pgfpathclose%
\pgfusepath{stroke,fill}%
\end{pgfscope}%
\begin{pgfscope}%
\pgfpathrectangle{\pgfqpoint{0.100000in}{0.212622in}}{\pgfqpoint{3.696000in}{3.696000in}}%
\pgfusepath{clip}%
\pgfsetbuttcap%
\pgfsetroundjoin%
\definecolor{currentfill}{rgb}{0.121569,0.466667,0.705882}%
\pgfsetfillcolor{currentfill}%
\pgfsetfillopacity{0.594378}%
\pgfsetlinewidth{1.003750pt}%
\definecolor{currentstroke}{rgb}{0.121569,0.466667,0.705882}%
\pgfsetstrokecolor{currentstroke}%
\pgfsetstrokeopacity{0.594378}%
\pgfsetdash{}{0pt}%
\pgfpathmoveto{\pgfqpoint{1.293100in}{1.867698in}}%
\pgfpathcurveto{\pgfqpoint{1.301336in}{1.867698in}}{\pgfqpoint{1.309236in}{1.870971in}}{\pgfqpoint{1.315060in}{1.876795in}}%
\pgfpathcurveto{\pgfqpoint{1.320884in}{1.882619in}}{\pgfqpoint{1.324157in}{1.890519in}}{\pgfqpoint{1.324157in}{1.898755in}}%
\pgfpathcurveto{\pgfqpoint{1.324157in}{1.906991in}}{\pgfqpoint{1.320884in}{1.914891in}}{\pgfqpoint{1.315060in}{1.920715in}}%
\pgfpathcurveto{\pgfqpoint{1.309236in}{1.926539in}}{\pgfqpoint{1.301336in}{1.929811in}}{\pgfqpoint{1.293100in}{1.929811in}}%
\pgfpathcurveto{\pgfqpoint{1.284864in}{1.929811in}}{\pgfqpoint{1.276964in}{1.926539in}}{\pgfqpoint{1.271140in}{1.920715in}}%
\pgfpathcurveto{\pgfqpoint{1.265316in}{1.914891in}}{\pgfqpoint{1.262044in}{1.906991in}}{\pgfqpoint{1.262044in}{1.898755in}}%
\pgfpathcurveto{\pgfqpoint{1.262044in}{1.890519in}}{\pgfqpoint{1.265316in}{1.882619in}}{\pgfqpoint{1.271140in}{1.876795in}}%
\pgfpathcurveto{\pgfqpoint{1.276964in}{1.870971in}}{\pgfqpoint{1.284864in}{1.867698in}}{\pgfqpoint{1.293100in}{1.867698in}}%
\pgfpathclose%
\pgfusepath{stroke,fill}%
\end{pgfscope}%
\begin{pgfscope}%
\pgfpathrectangle{\pgfqpoint{0.100000in}{0.212622in}}{\pgfqpoint{3.696000in}{3.696000in}}%
\pgfusepath{clip}%
\pgfsetbuttcap%
\pgfsetroundjoin%
\definecolor{currentfill}{rgb}{0.121569,0.466667,0.705882}%
\pgfsetfillcolor{currentfill}%
\pgfsetfillopacity{0.594442}%
\pgfsetlinewidth{1.003750pt}%
\definecolor{currentstroke}{rgb}{0.121569,0.466667,0.705882}%
\pgfsetstrokecolor{currentstroke}%
\pgfsetstrokeopacity{0.594442}%
\pgfsetdash{}{0pt}%
\pgfpathmoveto{\pgfqpoint{1.407142in}{1.990957in}}%
\pgfpathcurveto{\pgfqpoint{1.415378in}{1.990957in}}{\pgfqpoint{1.423278in}{1.994229in}}{\pgfqpoint{1.429102in}{2.000053in}}%
\pgfpathcurveto{\pgfqpoint{1.434926in}{2.005877in}}{\pgfqpoint{1.438198in}{2.013777in}}{\pgfqpoint{1.438198in}{2.022013in}}%
\pgfpathcurveto{\pgfqpoint{1.438198in}{2.030250in}}{\pgfqpoint{1.434926in}{2.038150in}}{\pgfqpoint{1.429102in}{2.043974in}}%
\pgfpathcurveto{\pgfqpoint{1.423278in}{2.049798in}}{\pgfqpoint{1.415378in}{2.053070in}}{\pgfqpoint{1.407142in}{2.053070in}}%
\pgfpathcurveto{\pgfqpoint{1.398905in}{2.053070in}}{\pgfqpoint{1.391005in}{2.049798in}}{\pgfqpoint{1.385181in}{2.043974in}}%
\pgfpathcurveto{\pgfqpoint{1.379357in}{2.038150in}}{\pgfqpoint{1.376085in}{2.030250in}}{\pgfqpoint{1.376085in}{2.022013in}}%
\pgfpathcurveto{\pgfqpoint{1.376085in}{2.013777in}}{\pgfqpoint{1.379357in}{2.005877in}}{\pgfqpoint{1.385181in}{2.000053in}}%
\pgfpathcurveto{\pgfqpoint{1.391005in}{1.994229in}}{\pgfqpoint{1.398905in}{1.990957in}}{\pgfqpoint{1.407142in}{1.990957in}}%
\pgfpathclose%
\pgfusepath{stroke,fill}%
\end{pgfscope}%
\begin{pgfscope}%
\pgfpathrectangle{\pgfqpoint{0.100000in}{0.212622in}}{\pgfqpoint{3.696000in}{3.696000in}}%
\pgfusepath{clip}%
\pgfsetbuttcap%
\pgfsetroundjoin%
\definecolor{currentfill}{rgb}{0.121569,0.466667,0.705882}%
\pgfsetfillcolor{currentfill}%
\pgfsetfillopacity{0.594890}%
\pgfsetlinewidth{1.003750pt}%
\definecolor{currentstroke}{rgb}{0.121569,0.466667,0.705882}%
\pgfsetstrokecolor{currentstroke}%
\pgfsetstrokeopacity{0.594890}%
\pgfsetdash{}{0pt}%
\pgfpathmoveto{\pgfqpoint{1.426296in}{1.990858in}}%
\pgfpathcurveto{\pgfqpoint{1.434533in}{1.990858in}}{\pgfqpoint{1.442433in}{1.994130in}}{\pgfqpoint{1.448257in}{1.999954in}}%
\pgfpathcurveto{\pgfqpoint{1.454081in}{2.005778in}}{\pgfqpoint{1.457353in}{2.013678in}}{\pgfqpoint{1.457353in}{2.021914in}}%
\pgfpathcurveto{\pgfqpoint{1.457353in}{2.030150in}}{\pgfqpoint{1.454081in}{2.038050in}}{\pgfqpoint{1.448257in}{2.043874in}}%
\pgfpathcurveto{\pgfqpoint{1.442433in}{2.049698in}}{\pgfqpoint{1.434533in}{2.052971in}}{\pgfqpoint{1.426296in}{2.052971in}}%
\pgfpathcurveto{\pgfqpoint{1.418060in}{2.052971in}}{\pgfqpoint{1.410160in}{2.049698in}}{\pgfqpoint{1.404336in}{2.043874in}}%
\pgfpathcurveto{\pgfqpoint{1.398512in}{2.038050in}}{\pgfqpoint{1.395240in}{2.030150in}}{\pgfqpoint{1.395240in}{2.021914in}}%
\pgfpathcurveto{\pgfqpoint{1.395240in}{2.013678in}}{\pgfqpoint{1.398512in}{2.005778in}}{\pgfqpoint{1.404336in}{1.999954in}}%
\pgfpathcurveto{\pgfqpoint{1.410160in}{1.994130in}}{\pgfqpoint{1.418060in}{1.990858in}}{\pgfqpoint{1.426296in}{1.990858in}}%
\pgfpathclose%
\pgfusepath{stroke,fill}%
\end{pgfscope}%
\begin{pgfscope}%
\pgfpathrectangle{\pgfqpoint{0.100000in}{0.212622in}}{\pgfqpoint{3.696000in}{3.696000in}}%
\pgfusepath{clip}%
\pgfsetbuttcap%
\pgfsetroundjoin%
\definecolor{currentfill}{rgb}{0.121569,0.466667,0.705882}%
\pgfsetfillcolor{currentfill}%
\pgfsetfillopacity{0.595090}%
\pgfsetlinewidth{1.003750pt}%
\definecolor{currentstroke}{rgb}{0.121569,0.466667,0.705882}%
\pgfsetstrokecolor{currentstroke}%
\pgfsetstrokeopacity{0.595090}%
\pgfsetdash{}{0pt}%
\pgfpathmoveto{\pgfqpoint{1.427521in}{1.991339in}}%
\pgfpathcurveto{\pgfqpoint{1.435757in}{1.991339in}}{\pgfqpoint{1.443657in}{1.994612in}}{\pgfqpoint{1.449481in}{2.000435in}}%
\pgfpathcurveto{\pgfqpoint{1.455305in}{2.006259in}}{\pgfqpoint{1.458577in}{2.014159in}}{\pgfqpoint{1.458577in}{2.022396in}}%
\pgfpathcurveto{\pgfqpoint{1.458577in}{2.030632in}}{\pgfqpoint{1.455305in}{2.038532in}}{\pgfqpoint{1.449481in}{2.044356in}}%
\pgfpathcurveto{\pgfqpoint{1.443657in}{2.050180in}}{\pgfqpoint{1.435757in}{2.053452in}}{\pgfqpoint{1.427521in}{2.053452in}}%
\pgfpathcurveto{\pgfqpoint{1.419284in}{2.053452in}}{\pgfqpoint{1.411384in}{2.050180in}}{\pgfqpoint{1.405560in}{2.044356in}}%
\pgfpathcurveto{\pgfqpoint{1.399736in}{2.038532in}}{\pgfqpoint{1.396464in}{2.030632in}}{\pgfqpoint{1.396464in}{2.022396in}}%
\pgfpathcurveto{\pgfqpoint{1.396464in}{2.014159in}}{\pgfqpoint{1.399736in}{2.006259in}}{\pgfqpoint{1.405560in}{2.000435in}}%
\pgfpathcurveto{\pgfqpoint{1.411384in}{1.994612in}}{\pgfqpoint{1.419284in}{1.991339in}}{\pgfqpoint{1.427521in}{1.991339in}}%
\pgfpathclose%
\pgfusepath{stroke,fill}%
\end{pgfscope}%
\begin{pgfscope}%
\pgfpathrectangle{\pgfqpoint{0.100000in}{0.212622in}}{\pgfqpoint{3.696000in}{3.696000in}}%
\pgfusepath{clip}%
\pgfsetbuttcap%
\pgfsetroundjoin%
\definecolor{currentfill}{rgb}{0.121569,0.466667,0.705882}%
\pgfsetfillcolor{currentfill}%
\pgfsetfillopacity{0.595194}%
\pgfsetlinewidth{1.003750pt}%
\definecolor{currentstroke}{rgb}{0.121569,0.466667,0.705882}%
\pgfsetstrokecolor{currentstroke}%
\pgfsetstrokeopacity{0.595194}%
\pgfsetdash{}{0pt}%
\pgfpathmoveto{\pgfqpoint{1.399091in}{1.982701in}}%
\pgfpathcurveto{\pgfqpoint{1.407327in}{1.982701in}}{\pgfqpoint{1.415227in}{1.985973in}}{\pgfqpoint{1.421051in}{1.991797in}}%
\pgfpathcurveto{\pgfqpoint{1.426875in}{1.997621in}}{\pgfqpoint{1.430147in}{2.005521in}}{\pgfqpoint{1.430147in}{2.013757in}}%
\pgfpathcurveto{\pgfqpoint{1.430147in}{2.021993in}}{\pgfqpoint{1.426875in}{2.029893in}}{\pgfqpoint{1.421051in}{2.035717in}}%
\pgfpathcurveto{\pgfqpoint{1.415227in}{2.041541in}}{\pgfqpoint{1.407327in}{2.044814in}}{\pgfqpoint{1.399091in}{2.044814in}}%
\pgfpathcurveto{\pgfqpoint{1.390854in}{2.044814in}}{\pgfqpoint{1.382954in}{2.041541in}}{\pgfqpoint{1.377130in}{2.035717in}}%
\pgfpathcurveto{\pgfqpoint{1.371306in}{2.029893in}}{\pgfqpoint{1.368034in}{2.021993in}}{\pgfqpoint{1.368034in}{2.013757in}}%
\pgfpathcurveto{\pgfqpoint{1.368034in}{2.005521in}}{\pgfqpoint{1.371306in}{1.997621in}}{\pgfqpoint{1.377130in}{1.991797in}}%
\pgfpathcurveto{\pgfqpoint{1.382954in}{1.985973in}}{\pgfqpoint{1.390854in}{1.982701in}}{\pgfqpoint{1.399091in}{1.982701in}}%
\pgfpathclose%
\pgfusepath{stroke,fill}%
\end{pgfscope}%
\begin{pgfscope}%
\pgfpathrectangle{\pgfqpoint{0.100000in}{0.212622in}}{\pgfqpoint{3.696000in}{3.696000in}}%
\pgfusepath{clip}%
\pgfsetbuttcap%
\pgfsetroundjoin%
\definecolor{currentfill}{rgb}{0.121569,0.466667,0.705882}%
\pgfsetfillcolor{currentfill}%
\pgfsetfillopacity{0.597004}%
\pgfsetlinewidth{1.003750pt}%
\definecolor{currentstroke}{rgb}{0.121569,0.466667,0.705882}%
\pgfsetstrokecolor{currentstroke}%
\pgfsetstrokeopacity{0.597004}%
\pgfsetdash{}{0pt}%
\pgfpathmoveto{\pgfqpoint{1.346482in}{1.933395in}}%
\pgfpathcurveto{\pgfqpoint{1.354718in}{1.933395in}}{\pgfqpoint{1.362618in}{1.936667in}}{\pgfqpoint{1.368442in}{1.942491in}}%
\pgfpathcurveto{\pgfqpoint{1.374266in}{1.948315in}}{\pgfqpoint{1.377538in}{1.956215in}}{\pgfqpoint{1.377538in}{1.964451in}}%
\pgfpathcurveto{\pgfqpoint{1.377538in}{1.972688in}}{\pgfqpoint{1.374266in}{1.980588in}}{\pgfqpoint{1.368442in}{1.986412in}}%
\pgfpathcurveto{\pgfqpoint{1.362618in}{1.992236in}}{\pgfqpoint{1.354718in}{1.995508in}}{\pgfqpoint{1.346482in}{1.995508in}}%
\pgfpathcurveto{\pgfqpoint{1.338245in}{1.995508in}}{\pgfqpoint{1.330345in}{1.992236in}}{\pgfqpoint{1.324521in}{1.986412in}}%
\pgfpathcurveto{\pgfqpoint{1.318697in}{1.980588in}}{\pgfqpoint{1.315425in}{1.972688in}}{\pgfqpoint{1.315425in}{1.964451in}}%
\pgfpathcurveto{\pgfqpoint{1.315425in}{1.956215in}}{\pgfqpoint{1.318697in}{1.948315in}}{\pgfqpoint{1.324521in}{1.942491in}}%
\pgfpathcurveto{\pgfqpoint{1.330345in}{1.936667in}}{\pgfqpoint{1.338245in}{1.933395in}}{\pgfqpoint{1.346482in}{1.933395in}}%
\pgfpathclose%
\pgfusepath{stroke,fill}%
\end{pgfscope}%
\begin{pgfscope}%
\pgfpathrectangle{\pgfqpoint{0.100000in}{0.212622in}}{\pgfqpoint{3.696000in}{3.696000in}}%
\pgfusepath{clip}%
\pgfsetbuttcap%
\pgfsetroundjoin%
\definecolor{currentfill}{rgb}{0.121569,0.466667,0.705882}%
\pgfsetfillcolor{currentfill}%
\pgfsetfillopacity{0.597075}%
\pgfsetlinewidth{1.003750pt}%
\definecolor{currentstroke}{rgb}{0.121569,0.466667,0.705882}%
\pgfsetstrokecolor{currentstroke}%
\pgfsetstrokeopacity{0.597075}%
\pgfsetdash{}{0pt}%
\pgfpathmoveto{\pgfqpoint{1.414715in}{1.990483in}}%
\pgfpathcurveto{\pgfqpoint{1.422951in}{1.990483in}}{\pgfqpoint{1.430851in}{1.993755in}}{\pgfqpoint{1.436675in}{1.999579in}}%
\pgfpathcurveto{\pgfqpoint{1.442499in}{2.005403in}}{\pgfqpoint{1.445771in}{2.013303in}}{\pgfqpoint{1.445771in}{2.021539in}}%
\pgfpathcurveto{\pgfqpoint{1.445771in}{2.029775in}}{\pgfqpoint{1.442499in}{2.037676in}}{\pgfqpoint{1.436675in}{2.043499in}}%
\pgfpathcurveto{\pgfqpoint{1.430851in}{2.049323in}}{\pgfqpoint{1.422951in}{2.052596in}}{\pgfqpoint{1.414715in}{2.052596in}}%
\pgfpathcurveto{\pgfqpoint{1.406478in}{2.052596in}}{\pgfqpoint{1.398578in}{2.049323in}}{\pgfqpoint{1.392754in}{2.043499in}}%
\pgfpathcurveto{\pgfqpoint{1.386930in}{2.037676in}}{\pgfqpoint{1.383658in}{2.029775in}}{\pgfqpoint{1.383658in}{2.021539in}}%
\pgfpathcurveto{\pgfqpoint{1.383658in}{2.013303in}}{\pgfqpoint{1.386930in}{2.005403in}}{\pgfqpoint{1.392754in}{1.999579in}}%
\pgfpathcurveto{\pgfqpoint{1.398578in}{1.993755in}}{\pgfqpoint{1.406478in}{1.990483in}}{\pgfqpoint{1.414715in}{1.990483in}}%
\pgfpathclose%
\pgfusepath{stroke,fill}%
\end{pgfscope}%
\begin{pgfscope}%
\pgfpathrectangle{\pgfqpoint{0.100000in}{0.212622in}}{\pgfqpoint{3.696000in}{3.696000in}}%
\pgfusepath{clip}%
\pgfsetbuttcap%
\pgfsetroundjoin%
\definecolor{currentfill}{rgb}{0.121569,0.466667,0.705882}%
\pgfsetfillcolor{currentfill}%
\pgfsetfillopacity{0.597095}%
\pgfsetlinewidth{1.003750pt}%
\definecolor{currentstroke}{rgb}{0.121569,0.466667,0.705882}%
\pgfsetstrokecolor{currentstroke}%
\pgfsetstrokeopacity{0.597095}%
\pgfsetdash{}{0pt}%
\pgfpathmoveto{\pgfqpoint{1.381957in}{1.962467in}}%
\pgfpathcurveto{\pgfqpoint{1.390193in}{1.962467in}}{\pgfqpoint{1.398093in}{1.965740in}}{\pgfqpoint{1.403917in}{1.971563in}}%
\pgfpathcurveto{\pgfqpoint{1.409741in}{1.977387in}}{\pgfqpoint{1.413013in}{1.985287in}}{\pgfqpoint{1.413013in}{1.993524in}}%
\pgfpathcurveto{\pgfqpoint{1.413013in}{2.001760in}}{\pgfqpoint{1.409741in}{2.009660in}}{\pgfqpoint{1.403917in}{2.015484in}}%
\pgfpathcurveto{\pgfqpoint{1.398093in}{2.021308in}}{\pgfqpoint{1.390193in}{2.024580in}}{\pgfqpoint{1.381957in}{2.024580in}}%
\pgfpathcurveto{\pgfqpoint{1.373721in}{2.024580in}}{\pgfqpoint{1.365821in}{2.021308in}}{\pgfqpoint{1.359997in}{2.015484in}}%
\pgfpathcurveto{\pgfqpoint{1.354173in}{2.009660in}}{\pgfqpoint{1.350900in}{2.001760in}}{\pgfqpoint{1.350900in}{1.993524in}}%
\pgfpathcurveto{\pgfqpoint{1.350900in}{1.985287in}}{\pgfqpoint{1.354173in}{1.977387in}}{\pgfqpoint{1.359997in}{1.971563in}}%
\pgfpathcurveto{\pgfqpoint{1.365821in}{1.965740in}}{\pgfqpoint{1.373721in}{1.962467in}}{\pgfqpoint{1.381957in}{1.962467in}}%
\pgfpathclose%
\pgfusepath{stroke,fill}%
\end{pgfscope}%
\begin{pgfscope}%
\pgfpathrectangle{\pgfqpoint{0.100000in}{0.212622in}}{\pgfqpoint{3.696000in}{3.696000in}}%
\pgfusepath{clip}%
\pgfsetbuttcap%
\pgfsetroundjoin%
\definecolor{currentfill}{rgb}{0.121569,0.466667,0.705882}%
\pgfsetfillcolor{currentfill}%
\pgfsetfillopacity{0.597360}%
\pgfsetlinewidth{1.003750pt}%
\definecolor{currentstroke}{rgb}{0.121569,0.466667,0.705882}%
\pgfsetstrokecolor{currentstroke}%
\pgfsetstrokeopacity{0.597360}%
\pgfsetdash{}{0pt}%
\pgfpathmoveto{\pgfqpoint{1.407994in}{1.981991in}}%
\pgfpathcurveto{\pgfqpoint{1.416230in}{1.981991in}}{\pgfqpoint{1.424130in}{1.985263in}}{\pgfqpoint{1.429954in}{1.991087in}}%
\pgfpathcurveto{\pgfqpoint{1.435778in}{1.996911in}}{\pgfqpoint{1.439050in}{2.004811in}}{\pgfqpoint{1.439050in}{2.013047in}}%
\pgfpathcurveto{\pgfqpoint{1.439050in}{2.021283in}}{\pgfqpoint{1.435778in}{2.029183in}}{\pgfqpoint{1.429954in}{2.035007in}}%
\pgfpathcurveto{\pgfqpoint{1.424130in}{2.040831in}}{\pgfqpoint{1.416230in}{2.044104in}}{\pgfqpoint{1.407994in}{2.044104in}}%
\pgfpathcurveto{\pgfqpoint{1.399757in}{2.044104in}}{\pgfqpoint{1.391857in}{2.040831in}}{\pgfqpoint{1.386033in}{2.035007in}}%
\pgfpathcurveto{\pgfqpoint{1.380209in}{2.029183in}}{\pgfqpoint{1.376937in}{2.021283in}}{\pgfqpoint{1.376937in}{2.013047in}}%
\pgfpathcurveto{\pgfqpoint{1.376937in}{2.004811in}}{\pgfqpoint{1.380209in}{1.996911in}}{\pgfqpoint{1.386033in}{1.991087in}}%
\pgfpathcurveto{\pgfqpoint{1.391857in}{1.985263in}}{\pgfqpoint{1.399757in}{1.981991in}}{\pgfqpoint{1.407994in}{1.981991in}}%
\pgfpathclose%
\pgfusepath{stroke,fill}%
\end{pgfscope}%
\begin{pgfscope}%
\pgfpathrectangle{\pgfqpoint{0.100000in}{0.212622in}}{\pgfqpoint{3.696000in}{3.696000in}}%
\pgfusepath{clip}%
\pgfsetbuttcap%
\pgfsetroundjoin%
\definecolor{currentfill}{rgb}{0.121569,0.466667,0.705882}%
\pgfsetfillcolor{currentfill}%
\pgfsetfillopacity{0.597665}%
\pgfsetlinewidth{1.003750pt}%
\definecolor{currentstroke}{rgb}{0.121569,0.466667,0.705882}%
\pgfsetstrokecolor{currentstroke}%
\pgfsetstrokeopacity{0.597665}%
\pgfsetdash{}{0pt}%
\pgfpathmoveto{\pgfqpoint{1.424874in}{1.985792in}}%
\pgfpathcurveto{\pgfqpoint{1.433110in}{1.985792in}}{\pgfqpoint{1.441010in}{1.989065in}}{\pgfqpoint{1.446834in}{1.994888in}}%
\pgfpathcurveto{\pgfqpoint{1.452658in}{2.000712in}}{\pgfqpoint{1.455930in}{2.008612in}}{\pgfqpoint{1.455930in}{2.016849in}}%
\pgfpathcurveto{\pgfqpoint{1.455930in}{2.025085in}}{\pgfqpoint{1.452658in}{2.032985in}}{\pgfqpoint{1.446834in}{2.038809in}}%
\pgfpathcurveto{\pgfqpoint{1.441010in}{2.044633in}}{\pgfqpoint{1.433110in}{2.047905in}}{\pgfqpoint{1.424874in}{2.047905in}}%
\pgfpathcurveto{\pgfqpoint{1.416638in}{2.047905in}}{\pgfqpoint{1.408738in}{2.044633in}}{\pgfqpoint{1.402914in}{2.038809in}}%
\pgfpathcurveto{\pgfqpoint{1.397090in}{2.032985in}}{\pgfqpoint{1.393817in}{2.025085in}}{\pgfqpoint{1.393817in}{2.016849in}}%
\pgfpathcurveto{\pgfqpoint{1.393817in}{2.008612in}}{\pgfqpoint{1.397090in}{2.000712in}}{\pgfqpoint{1.402914in}{1.994888in}}%
\pgfpathcurveto{\pgfqpoint{1.408738in}{1.989065in}}{\pgfqpoint{1.416638in}{1.985792in}}{\pgfqpoint{1.424874in}{1.985792in}}%
\pgfpathclose%
\pgfusepath{stroke,fill}%
\end{pgfscope}%
\begin{pgfscope}%
\pgfpathrectangle{\pgfqpoint{0.100000in}{0.212622in}}{\pgfqpoint{3.696000in}{3.696000in}}%
\pgfusepath{clip}%
\pgfsetbuttcap%
\pgfsetroundjoin%
\definecolor{currentfill}{rgb}{0.121569,0.466667,0.705882}%
\pgfsetfillcolor{currentfill}%
\pgfsetfillopacity{0.597755}%
\pgfsetlinewidth{1.003750pt}%
\definecolor{currentstroke}{rgb}{0.121569,0.466667,0.705882}%
\pgfsetstrokecolor{currentstroke}%
\pgfsetstrokeopacity{0.597755}%
\pgfsetdash{}{0pt}%
\pgfpathmoveto{\pgfqpoint{1.404281in}{1.982560in}}%
\pgfpathcurveto{\pgfqpoint{1.412517in}{1.982560in}}{\pgfqpoint{1.420417in}{1.985832in}}{\pgfqpoint{1.426241in}{1.991656in}}%
\pgfpathcurveto{\pgfqpoint{1.432065in}{1.997480in}}{\pgfqpoint{1.435337in}{2.005380in}}{\pgfqpoint{1.435337in}{2.013616in}}%
\pgfpathcurveto{\pgfqpoint{1.435337in}{2.021853in}}{\pgfqpoint{1.432065in}{2.029753in}}{\pgfqpoint{1.426241in}{2.035577in}}%
\pgfpathcurveto{\pgfqpoint{1.420417in}{2.041401in}}{\pgfqpoint{1.412517in}{2.044673in}}{\pgfqpoint{1.404281in}{2.044673in}}%
\pgfpathcurveto{\pgfqpoint{1.396044in}{2.044673in}}{\pgfqpoint{1.388144in}{2.041401in}}{\pgfqpoint{1.382320in}{2.035577in}}%
\pgfpathcurveto{\pgfqpoint{1.376497in}{2.029753in}}{\pgfqpoint{1.373224in}{2.021853in}}{\pgfqpoint{1.373224in}{2.013616in}}%
\pgfpathcurveto{\pgfqpoint{1.373224in}{2.005380in}}{\pgfqpoint{1.376497in}{1.997480in}}{\pgfqpoint{1.382320in}{1.991656in}}%
\pgfpathcurveto{\pgfqpoint{1.388144in}{1.985832in}}{\pgfqpoint{1.396044in}{1.982560in}}{\pgfqpoint{1.404281in}{1.982560in}}%
\pgfpathclose%
\pgfusepath{stroke,fill}%
\end{pgfscope}%
\begin{pgfscope}%
\pgfpathrectangle{\pgfqpoint{0.100000in}{0.212622in}}{\pgfqpoint{3.696000in}{3.696000in}}%
\pgfusepath{clip}%
\pgfsetbuttcap%
\pgfsetroundjoin%
\definecolor{currentfill}{rgb}{0.121569,0.466667,0.705882}%
\pgfsetfillcolor{currentfill}%
\pgfsetfillopacity{0.599445}%
\pgfsetlinewidth{1.003750pt}%
\definecolor{currentstroke}{rgb}{0.121569,0.466667,0.705882}%
\pgfsetstrokecolor{currentstroke}%
\pgfsetstrokeopacity{0.599445}%
\pgfsetdash{}{0pt}%
\pgfpathmoveto{\pgfqpoint{1.411468in}{1.977789in}}%
\pgfpathcurveto{\pgfqpoint{1.419704in}{1.977789in}}{\pgfqpoint{1.427604in}{1.981062in}}{\pgfqpoint{1.433428in}{1.986886in}}%
\pgfpathcurveto{\pgfqpoint{1.439252in}{1.992710in}}{\pgfqpoint{1.442525in}{2.000610in}}{\pgfqpoint{1.442525in}{2.008846in}}%
\pgfpathcurveto{\pgfqpoint{1.442525in}{2.017082in}}{\pgfqpoint{1.439252in}{2.024982in}}{\pgfqpoint{1.433428in}{2.030806in}}%
\pgfpathcurveto{\pgfqpoint{1.427604in}{2.036630in}}{\pgfqpoint{1.419704in}{2.039902in}}{\pgfqpoint{1.411468in}{2.039902in}}%
\pgfpathcurveto{\pgfqpoint{1.403232in}{2.039902in}}{\pgfqpoint{1.395332in}{2.036630in}}{\pgfqpoint{1.389508in}{2.030806in}}%
\pgfpathcurveto{\pgfqpoint{1.383684in}{2.024982in}}{\pgfqpoint{1.380412in}{2.017082in}}{\pgfqpoint{1.380412in}{2.008846in}}%
\pgfpathcurveto{\pgfqpoint{1.380412in}{2.000610in}}{\pgfqpoint{1.383684in}{1.992710in}}{\pgfqpoint{1.389508in}{1.986886in}}%
\pgfpathcurveto{\pgfqpoint{1.395332in}{1.981062in}}{\pgfqpoint{1.403232in}{1.977789in}}{\pgfqpoint{1.411468in}{1.977789in}}%
\pgfpathclose%
\pgfusepath{stroke,fill}%
\end{pgfscope}%
\begin{pgfscope}%
\pgfpathrectangle{\pgfqpoint{0.100000in}{0.212622in}}{\pgfqpoint{3.696000in}{3.696000in}}%
\pgfusepath{clip}%
\pgfsetbuttcap%
\pgfsetroundjoin%
\definecolor{currentfill}{rgb}{0.121569,0.466667,0.705882}%
\pgfsetfillcolor{currentfill}%
\pgfsetfillopacity{0.601495}%
\pgfsetlinewidth{1.003750pt}%
\definecolor{currentstroke}{rgb}{0.121569,0.466667,0.705882}%
\pgfsetstrokecolor{currentstroke}%
\pgfsetstrokeopacity{0.601495}%
\pgfsetdash{}{0pt}%
\pgfpathmoveto{\pgfqpoint{1.432946in}{1.999532in}}%
\pgfpathcurveto{\pgfqpoint{1.441182in}{1.999532in}}{\pgfqpoint{1.449082in}{2.002805in}}{\pgfqpoint{1.454906in}{2.008629in}}%
\pgfpathcurveto{\pgfqpoint{1.460730in}{2.014453in}}{\pgfqpoint{1.464002in}{2.022353in}}{\pgfqpoint{1.464002in}{2.030589in}}%
\pgfpathcurveto{\pgfqpoint{1.464002in}{2.038825in}}{\pgfqpoint{1.460730in}{2.046725in}}{\pgfqpoint{1.454906in}{2.052549in}}%
\pgfpathcurveto{\pgfqpoint{1.449082in}{2.058373in}}{\pgfqpoint{1.441182in}{2.061645in}}{\pgfqpoint{1.432946in}{2.061645in}}%
\pgfpathcurveto{\pgfqpoint{1.424709in}{2.061645in}}{\pgfqpoint{1.416809in}{2.058373in}}{\pgfqpoint{1.410985in}{2.052549in}}%
\pgfpathcurveto{\pgfqpoint{1.405161in}{2.046725in}}{\pgfqpoint{1.401889in}{2.038825in}}{\pgfqpoint{1.401889in}{2.030589in}}%
\pgfpathcurveto{\pgfqpoint{1.401889in}{2.022353in}}{\pgfqpoint{1.405161in}{2.014453in}}{\pgfqpoint{1.410985in}{2.008629in}}%
\pgfpathcurveto{\pgfqpoint{1.416809in}{2.002805in}}{\pgfqpoint{1.424709in}{1.999532in}}{\pgfqpoint{1.432946in}{1.999532in}}%
\pgfpathclose%
\pgfusepath{stroke,fill}%
\end{pgfscope}%
\begin{pgfscope}%
\pgfpathrectangle{\pgfqpoint{0.100000in}{0.212622in}}{\pgfqpoint{3.696000in}{3.696000in}}%
\pgfusepath{clip}%
\pgfsetbuttcap%
\pgfsetroundjoin%
\definecolor{currentfill}{rgb}{0.121569,0.466667,0.705882}%
\pgfsetfillcolor{currentfill}%
\pgfsetfillopacity{0.603243}%
\pgfsetlinewidth{1.003750pt}%
\definecolor{currentstroke}{rgb}{0.121569,0.466667,0.705882}%
\pgfsetstrokecolor{currentstroke}%
\pgfsetstrokeopacity{0.603243}%
\pgfsetdash{}{0pt}%
\pgfpathmoveto{\pgfqpoint{1.440778in}{2.014052in}}%
\pgfpathcurveto{\pgfqpoint{1.449014in}{2.014052in}}{\pgfqpoint{1.456914in}{2.017325in}}{\pgfqpoint{1.462738in}{2.023149in}}%
\pgfpathcurveto{\pgfqpoint{1.468562in}{2.028973in}}{\pgfqpoint{1.471834in}{2.036873in}}{\pgfqpoint{1.471834in}{2.045109in}}%
\pgfpathcurveto{\pgfqpoint{1.471834in}{2.053345in}}{\pgfqpoint{1.468562in}{2.061245in}}{\pgfqpoint{1.462738in}{2.067069in}}%
\pgfpathcurveto{\pgfqpoint{1.456914in}{2.072893in}}{\pgfqpoint{1.449014in}{2.076165in}}{\pgfqpoint{1.440778in}{2.076165in}}%
\pgfpathcurveto{\pgfqpoint{1.432541in}{2.076165in}}{\pgfqpoint{1.424641in}{2.072893in}}{\pgfqpoint{1.418817in}{2.067069in}}%
\pgfpathcurveto{\pgfqpoint{1.412993in}{2.061245in}}{\pgfqpoint{1.409721in}{2.053345in}}{\pgfqpoint{1.409721in}{2.045109in}}%
\pgfpathcurveto{\pgfqpoint{1.409721in}{2.036873in}}{\pgfqpoint{1.412993in}{2.028973in}}{\pgfqpoint{1.418817in}{2.023149in}}%
\pgfpathcurveto{\pgfqpoint{1.424641in}{2.017325in}}{\pgfqpoint{1.432541in}{2.014052in}}{\pgfqpoint{1.440778in}{2.014052in}}%
\pgfpathclose%
\pgfusepath{stroke,fill}%
\end{pgfscope}%
\begin{pgfscope}%
\pgfpathrectangle{\pgfqpoint{0.100000in}{0.212622in}}{\pgfqpoint{3.696000in}{3.696000in}}%
\pgfusepath{clip}%
\pgfsetbuttcap%
\pgfsetroundjoin%
\definecolor{currentfill}{rgb}{0.121569,0.466667,0.705882}%
\pgfsetfillcolor{currentfill}%
\pgfsetfillopacity{0.610833}%
\pgfsetlinewidth{1.003750pt}%
\definecolor{currentstroke}{rgb}{0.121569,0.466667,0.705882}%
\pgfsetstrokecolor{currentstroke}%
\pgfsetstrokeopacity{0.610833}%
\pgfsetdash{}{0pt}%
\pgfpathmoveto{\pgfqpoint{1.331183in}{1.923882in}}%
\pgfpathcurveto{\pgfqpoint{1.339419in}{1.923882in}}{\pgfqpoint{1.347319in}{1.927154in}}{\pgfqpoint{1.353143in}{1.932978in}}%
\pgfpathcurveto{\pgfqpoint{1.358967in}{1.938802in}}{\pgfqpoint{1.362239in}{1.946702in}}{\pgfqpoint{1.362239in}{1.954938in}}%
\pgfpathcurveto{\pgfqpoint{1.362239in}{1.963175in}}{\pgfqpoint{1.358967in}{1.971075in}}{\pgfqpoint{1.353143in}{1.976899in}}%
\pgfpathcurveto{\pgfqpoint{1.347319in}{1.982723in}}{\pgfqpoint{1.339419in}{1.985995in}}{\pgfqpoint{1.331183in}{1.985995in}}%
\pgfpathcurveto{\pgfqpoint{1.322947in}{1.985995in}}{\pgfqpoint{1.315047in}{1.982723in}}{\pgfqpoint{1.309223in}{1.976899in}}%
\pgfpathcurveto{\pgfqpoint{1.303399in}{1.971075in}}{\pgfqpoint{1.300126in}{1.963175in}}{\pgfqpoint{1.300126in}{1.954938in}}%
\pgfpathcurveto{\pgfqpoint{1.300126in}{1.946702in}}{\pgfqpoint{1.303399in}{1.938802in}}{\pgfqpoint{1.309223in}{1.932978in}}%
\pgfpathcurveto{\pgfqpoint{1.315047in}{1.927154in}}{\pgfqpoint{1.322947in}{1.923882in}}{\pgfqpoint{1.331183in}{1.923882in}}%
\pgfpathclose%
\pgfusepath{stroke,fill}%
\end{pgfscope}%
\begin{pgfscope}%
\pgfpathrectangle{\pgfqpoint{0.100000in}{0.212622in}}{\pgfqpoint{3.696000in}{3.696000in}}%
\pgfusepath{clip}%
\pgfsetbuttcap%
\pgfsetroundjoin%
\definecolor{currentfill}{rgb}{0.121569,0.466667,0.705882}%
\pgfsetfillcolor{currentfill}%
\pgfsetfillopacity{0.611840}%
\pgfsetlinewidth{1.003750pt}%
\definecolor{currentstroke}{rgb}{0.121569,0.466667,0.705882}%
\pgfsetstrokecolor{currentstroke}%
\pgfsetstrokeopacity{0.611840}%
\pgfsetdash{}{0pt}%
\pgfpathmoveto{\pgfqpoint{1.431876in}{2.005161in}}%
\pgfpathcurveto{\pgfqpoint{1.440112in}{2.005161in}}{\pgfqpoint{1.448012in}{2.008433in}}{\pgfqpoint{1.453836in}{2.014257in}}%
\pgfpathcurveto{\pgfqpoint{1.459660in}{2.020081in}}{\pgfqpoint{1.462932in}{2.027981in}}{\pgfqpoint{1.462932in}{2.036217in}}%
\pgfpathcurveto{\pgfqpoint{1.462932in}{2.044453in}}{\pgfqpoint{1.459660in}{2.052353in}}{\pgfqpoint{1.453836in}{2.058177in}}%
\pgfpathcurveto{\pgfqpoint{1.448012in}{2.064001in}}{\pgfqpoint{1.440112in}{2.067274in}}{\pgfqpoint{1.431876in}{2.067274in}}%
\pgfpathcurveto{\pgfqpoint{1.423639in}{2.067274in}}{\pgfqpoint{1.415739in}{2.064001in}}{\pgfqpoint{1.409915in}{2.058177in}}%
\pgfpathcurveto{\pgfqpoint{1.404091in}{2.052353in}}{\pgfqpoint{1.400819in}{2.044453in}}{\pgfqpoint{1.400819in}{2.036217in}}%
\pgfpathcurveto{\pgfqpoint{1.400819in}{2.027981in}}{\pgfqpoint{1.404091in}{2.020081in}}{\pgfqpoint{1.409915in}{2.014257in}}%
\pgfpathcurveto{\pgfqpoint{1.415739in}{2.008433in}}{\pgfqpoint{1.423639in}{2.005161in}}{\pgfqpoint{1.431876in}{2.005161in}}%
\pgfpathclose%
\pgfusepath{stroke,fill}%
\end{pgfscope}%
\begin{pgfscope}%
\pgfpathrectangle{\pgfqpoint{0.100000in}{0.212622in}}{\pgfqpoint{3.696000in}{3.696000in}}%
\pgfusepath{clip}%
\pgfsetbuttcap%
\pgfsetroundjoin%
\definecolor{currentfill}{rgb}{0.121569,0.466667,0.705882}%
\pgfsetfillcolor{currentfill}%
\pgfsetfillopacity{0.619560}%
\pgfsetlinewidth{1.003750pt}%
\definecolor{currentstroke}{rgb}{0.121569,0.466667,0.705882}%
\pgfsetstrokecolor{currentstroke}%
\pgfsetstrokeopacity{0.619560}%
\pgfsetdash{}{0pt}%
\pgfpathmoveto{\pgfqpoint{1.427416in}{2.003035in}}%
\pgfpathcurveto{\pgfqpoint{1.435652in}{2.003035in}}{\pgfqpoint{1.443552in}{2.006308in}}{\pgfqpoint{1.449376in}{2.012132in}}%
\pgfpathcurveto{\pgfqpoint{1.455200in}{2.017956in}}{\pgfqpoint{1.458472in}{2.025856in}}{\pgfqpoint{1.458472in}{2.034092in}}%
\pgfpathcurveto{\pgfqpoint{1.458472in}{2.042328in}}{\pgfqpoint{1.455200in}{2.050228in}}{\pgfqpoint{1.449376in}{2.056052in}}%
\pgfpathcurveto{\pgfqpoint{1.443552in}{2.061876in}}{\pgfqpoint{1.435652in}{2.065148in}}{\pgfqpoint{1.427416in}{2.065148in}}%
\pgfpathcurveto{\pgfqpoint{1.419179in}{2.065148in}}{\pgfqpoint{1.411279in}{2.061876in}}{\pgfqpoint{1.405455in}{2.056052in}}%
\pgfpathcurveto{\pgfqpoint{1.399632in}{2.050228in}}{\pgfqpoint{1.396359in}{2.042328in}}{\pgfqpoint{1.396359in}{2.034092in}}%
\pgfpathcurveto{\pgfqpoint{1.396359in}{2.025856in}}{\pgfqpoint{1.399632in}{2.017956in}}{\pgfqpoint{1.405455in}{2.012132in}}%
\pgfpathcurveto{\pgfqpoint{1.411279in}{2.006308in}}{\pgfqpoint{1.419179in}{2.003035in}}{\pgfqpoint{1.427416in}{2.003035in}}%
\pgfpathclose%
\pgfusepath{stroke,fill}%
\end{pgfscope}%
\begin{pgfscope}%
\pgfpathrectangle{\pgfqpoint{0.100000in}{0.212622in}}{\pgfqpoint{3.696000in}{3.696000in}}%
\pgfusepath{clip}%
\pgfsetbuttcap%
\pgfsetroundjoin%
\definecolor{currentfill}{rgb}{0.121569,0.466667,0.705882}%
\pgfsetfillcolor{currentfill}%
\pgfsetfillopacity{0.622357}%
\pgfsetlinewidth{1.003750pt}%
\definecolor{currentstroke}{rgb}{0.121569,0.466667,0.705882}%
\pgfsetstrokecolor{currentstroke}%
\pgfsetstrokeopacity{0.622357}%
\pgfsetdash{}{0pt}%
\pgfpathmoveto{\pgfqpoint{1.427122in}{2.000043in}}%
\pgfpathcurveto{\pgfqpoint{1.435358in}{2.000043in}}{\pgfqpoint{1.443258in}{2.003315in}}{\pgfqpoint{1.449082in}{2.009139in}}%
\pgfpathcurveto{\pgfqpoint{1.454906in}{2.014963in}}{\pgfqpoint{1.458178in}{2.022863in}}{\pgfqpoint{1.458178in}{2.031099in}}%
\pgfpathcurveto{\pgfqpoint{1.458178in}{2.039335in}}{\pgfqpoint{1.454906in}{2.047235in}}{\pgfqpoint{1.449082in}{2.053059in}}%
\pgfpathcurveto{\pgfqpoint{1.443258in}{2.058883in}}{\pgfqpoint{1.435358in}{2.062156in}}{\pgfqpoint{1.427122in}{2.062156in}}%
\pgfpathcurveto{\pgfqpoint{1.418885in}{2.062156in}}{\pgfqpoint{1.410985in}{2.058883in}}{\pgfqpoint{1.405161in}{2.053059in}}%
\pgfpathcurveto{\pgfqpoint{1.399338in}{2.047235in}}{\pgfqpoint{1.396065in}{2.039335in}}{\pgfqpoint{1.396065in}{2.031099in}}%
\pgfpathcurveto{\pgfqpoint{1.396065in}{2.022863in}}{\pgfqpoint{1.399338in}{2.014963in}}{\pgfqpoint{1.405161in}{2.009139in}}%
\pgfpathcurveto{\pgfqpoint{1.410985in}{2.003315in}}{\pgfqpoint{1.418885in}{2.000043in}}{\pgfqpoint{1.427122in}{2.000043in}}%
\pgfpathclose%
\pgfusepath{stroke,fill}%
\end{pgfscope}%
\begin{pgfscope}%
\pgfpathrectangle{\pgfqpoint{0.100000in}{0.212622in}}{\pgfqpoint{3.696000in}{3.696000in}}%
\pgfusepath{clip}%
\pgfsetbuttcap%
\pgfsetroundjoin%
\definecolor{currentfill}{rgb}{0.121569,0.466667,0.705882}%
\pgfsetfillcolor{currentfill}%
\pgfsetfillopacity{0.623205}%
\pgfsetlinewidth{1.003750pt}%
\definecolor{currentstroke}{rgb}{0.121569,0.466667,0.705882}%
\pgfsetstrokecolor{currentstroke}%
\pgfsetstrokeopacity{0.623205}%
\pgfsetdash{}{0pt}%
\pgfpathmoveto{\pgfqpoint{1.429354in}{2.002272in}}%
\pgfpathcurveto{\pgfqpoint{1.437590in}{2.002272in}}{\pgfqpoint{1.445490in}{2.005544in}}{\pgfqpoint{1.451314in}{2.011368in}}%
\pgfpathcurveto{\pgfqpoint{1.457138in}{2.017192in}}{\pgfqpoint{1.460410in}{2.025092in}}{\pgfqpoint{1.460410in}{2.033328in}}%
\pgfpathcurveto{\pgfqpoint{1.460410in}{2.041564in}}{\pgfqpoint{1.457138in}{2.049465in}}{\pgfqpoint{1.451314in}{2.055288in}}%
\pgfpathcurveto{\pgfqpoint{1.445490in}{2.061112in}}{\pgfqpoint{1.437590in}{2.064385in}}{\pgfqpoint{1.429354in}{2.064385in}}%
\pgfpathcurveto{\pgfqpoint{1.421117in}{2.064385in}}{\pgfqpoint{1.413217in}{2.061112in}}{\pgfqpoint{1.407393in}{2.055288in}}%
\pgfpathcurveto{\pgfqpoint{1.401570in}{2.049465in}}{\pgfqpoint{1.398297in}{2.041564in}}{\pgfqpoint{1.398297in}{2.033328in}}%
\pgfpathcurveto{\pgfqpoint{1.398297in}{2.025092in}}{\pgfqpoint{1.401570in}{2.017192in}}{\pgfqpoint{1.407393in}{2.011368in}}%
\pgfpathcurveto{\pgfqpoint{1.413217in}{2.005544in}}{\pgfqpoint{1.421117in}{2.002272in}}{\pgfqpoint{1.429354in}{2.002272in}}%
\pgfpathclose%
\pgfusepath{stroke,fill}%
\end{pgfscope}%
\begin{pgfscope}%
\pgfpathrectangle{\pgfqpoint{0.100000in}{0.212622in}}{\pgfqpoint{3.696000in}{3.696000in}}%
\pgfusepath{clip}%
\pgfsetbuttcap%
\pgfsetroundjoin%
\definecolor{currentfill}{rgb}{0.121569,0.466667,0.705882}%
\pgfsetfillcolor{currentfill}%
\pgfsetfillopacity{0.624779}%
\pgfsetlinewidth{1.003750pt}%
\definecolor{currentstroke}{rgb}{0.121569,0.466667,0.705882}%
\pgfsetstrokecolor{currentstroke}%
\pgfsetstrokeopacity{0.624779}%
\pgfsetdash{}{0pt}%
\pgfpathmoveto{\pgfqpoint{1.432116in}{2.004912in}}%
\pgfpathcurveto{\pgfqpoint{1.440352in}{2.004912in}}{\pgfqpoint{1.448252in}{2.008185in}}{\pgfqpoint{1.454076in}{2.014009in}}%
\pgfpathcurveto{\pgfqpoint{1.459900in}{2.019833in}}{\pgfqpoint{1.463172in}{2.027733in}}{\pgfqpoint{1.463172in}{2.035969in}}%
\pgfpathcurveto{\pgfqpoint{1.463172in}{2.044205in}}{\pgfqpoint{1.459900in}{2.052105in}}{\pgfqpoint{1.454076in}{2.057929in}}%
\pgfpathcurveto{\pgfqpoint{1.448252in}{2.063753in}}{\pgfqpoint{1.440352in}{2.067025in}}{\pgfqpoint{1.432116in}{2.067025in}}%
\pgfpathcurveto{\pgfqpoint{1.423880in}{2.067025in}}{\pgfqpoint{1.415980in}{2.063753in}}{\pgfqpoint{1.410156in}{2.057929in}}%
\pgfpathcurveto{\pgfqpoint{1.404332in}{2.052105in}}{\pgfqpoint{1.401059in}{2.044205in}}{\pgfqpoint{1.401059in}{2.035969in}}%
\pgfpathcurveto{\pgfqpoint{1.401059in}{2.027733in}}{\pgfqpoint{1.404332in}{2.019833in}}{\pgfqpoint{1.410156in}{2.014009in}}%
\pgfpathcurveto{\pgfqpoint{1.415980in}{2.008185in}}{\pgfqpoint{1.423880in}{2.004912in}}{\pgfqpoint{1.432116in}{2.004912in}}%
\pgfpathclose%
\pgfusepath{stroke,fill}%
\end{pgfscope}%
\begin{pgfscope}%
\pgfpathrectangle{\pgfqpoint{0.100000in}{0.212622in}}{\pgfqpoint{3.696000in}{3.696000in}}%
\pgfusepath{clip}%
\pgfsetbuttcap%
\pgfsetroundjoin%
\definecolor{currentfill}{rgb}{0.121569,0.466667,0.705882}%
\pgfsetfillcolor{currentfill}%
\pgfsetfillopacity{0.628568}%
\pgfsetlinewidth{1.003750pt}%
\definecolor{currentstroke}{rgb}{0.121569,0.466667,0.705882}%
\pgfsetstrokecolor{currentstroke}%
\pgfsetstrokeopacity{0.628568}%
\pgfsetdash{}{0pt}%
\pgfpathmoveto{\pgfqpoint{1.458129in}{2.041724in}}%
\pgfpathcurveto{\pgfqpoint{1.466365in}{2.041724in}}{\pgfqpoint{1.474265in}{2.044996in}}{\pgfqpoint{1.480089in}{2.050820in}}%
\pgfpathcurveto{\pgfqpoint{1.485913in}{2.056644in}}{\pgfqpoint{1.489185in}{2.064544in}}{\pgfqpoint{1.489185in}{2.072780in}}%
\pgfpathcurveto{\pgfqpoint{1.489185in}{2.081017in}}{\pgfqpoint{1.485913in}{2.088917in}}{\pgfqpoint{1.480089in}{2.094741in}}%
\pgfpathcurveto{\pgfqpoint{1.474265in}{2.100565in}}{\pgfqpoint{1.466365in}{2.103837in}}{\pgfqpoint{1.458129in}{2.103837in}}%
\pgfpathcurveto{\pgfqpoint{1.449892in}{2.103837in}}{\pgfqpoint{1.441992in}{2.100565in}}{\pgfqpoint{1.436168in}{2.094741in}}%
\pgfpathcurveto{\pgfqpoint{1.430345in}{2.088917in}}{\pgfqpoint{1.427072in}{2.081017in}}{\pgfqpoint{1.427072in}{2.072780in}}%
\pgfpathcurveto{\pgfqpoint{1.427072in}{2.064544in}}{\pgfqpoint{1.430345in}{2.056644in}}{\pgfqpoint{1.436168in}{2.050820in}}%
\pgfpathcurveto{\pgfqpoint{1.441992in}{2.044996in}}{\pgfqpoint{1.449892in}{2.041724in}}{\pgfqpoint{1.458129in}{2.041724in}}%
\pgfpathclose%
\pgfusepath{stroke,fill}%
\end{pgfscope}%
\begin{pgfscope}%
\pgfpathrectangle{\pgfqpoint{0.100000in}{0.212622in}}{\pgfqpoint{3.696000in}{3.696000in}}%
\pgfusepath{clip}%
\pgfsetbuttcap%
\pgfsetroundjoin%
\definecolor{currentfill}{rgb}{0.121569,0.466667,0.705882}%
\pgfsetfillcolor{currentfill}%
\pgfsetfillopacity{0.628963}%
\pgfsetlinewidth{1.003750pt}%
\definecolor{currentstroke}{rgb}{0.121569,0.466667,0.705882}%
\pgfsetstrokecolor{currentstroke}%
\pgfsetstrokeopacity{0.628963}%
\pgfsetdash{}{0pt}%
\pgfpathmoveto{\pgfqpoint{1.428980in}{2.000418in}}%
\pgfpathcurveto{\pgfqpoint{1.437217in}{2.000418in}}{\pgfqpoint{1.445117in}{2.003690in}}{\pgfqpoint{1.450941in}{2.009514in}}%
\pgfpathcurveto{\pgfqpoint{1.456764in}{2.015338in}}{\pgfqpoint{1.460037in}{2.023238in}}{\pgfqpoint{1.460037in}{2.031474in}}%
\pgfpathcurveto{\pgfqpoint{1.460037in}{2.039710in}}{\pgfqpoint{1.456764in}{2.047610in}}{\pgfqpoint{1.450941in}{2.053434in}}%
\pgfpathcurveto{\pgfqpoint{1.445117in}{2.059258in}}{\pgfqpoint{1.437217in}{2.062531in}}{\pgfqpoint{1.428980in}{2.062531in}}%
\pgfpathcurveto{\pgfqpoint{1.420744in}{2.062531in}}{\pgfqpoint{1.412844in}{2.059258in}}{\pgfqpoint{1.407020in}{2.053434in}}%
\pgfpathcurveto{\pgfqpoint{1.401196in}{2.047610in}}{\pgfqpoint{1.397924in}{2.039710in}}{\pgfqpoint{1.397924in}{2.031474in}}%
\pgfpathcurveto{\pgfqpoint{1.397924in}{2.023238in}}{\pgfqpoint{1.401196in}{2.015338in}}{\pgfqpoint{1.407020in}{2.009514in}}%
\pgfpathcurveto{\pgfqpoint{1.412844in}{2.003690in}}{\pgfqpoint{1.420744in}{2.000418in}}{\pgfqpoint{1.428980in}{2.000418in}}%
\pgfpathclose%
\pgfusepath{stroke,fill}%
\end{pgfscope}%
\begin{pgfscope}%
\pgfpathrectangle{\pgfqpoint{0.100000in}{0.212622in}}{\pgfqpoint{3.696000in}{3.696000in}}%
\pgfusepath{clip}%
\pgfsetbuttcap%
\pgfsetroundjoin%
\definecolor{currentfill}{rgb}{0.121569,0.466667,0.705882}%
\pgfsetfillcolor{currentfill}%
\pgfsetfillopacity{0.631491}%
\pgfsetlinewidth{1.003750pt}%
\definecolor{currentstroke}{rgb}{0.121569,0.466667,0.705882}%
\pgfsetstrokecolor{currentstroke}%
\pgfsetstrokeopacity{0.631491}%
\pgfsetdash{}{0pt}%
\pgfpathmoveto{\pgfqpoint{1.427615in}{2.000623in}}%
\pgfpathcurveto{\pgfqpoint{1.435851in}{2.000623in}}{\pgfqpoint{1.443751in}{2.003895in}}{\pgfqpoint{1.449575in}{2.009719in}}%
\pgfpathcurveto{\pgfqpoint{1.455399in}{2.015543in}}{\pgfqpoint{1.458672in}{2.023443in}}{\pgfqpoint{1.458672in}{2.031679in}}%
\pgfpathcurveto{\pgfqpoint{1.458672in}{2.039915in}}{\pgfqpoint{1.455399in}{2.047815in}}{\pgfqpoint{1.449575in}{2.053639in}}%
\pgfpathcurveto{\pgfqpoint{1.443751in}{2.059463in}}{\pgfqpoint{1.435851in}{2.062736in}}{\pgfqpoint{1.427615in}{2.062736in}}%
\pgfpathcurveto{\pgfqpoint{1.419379in}{2.062736in}}{\pgfqpoint{1.411479in}{2.059463in}}{\pgfqpoint{1.405655in}{2.053639in}}%
\pgfpathcurveto{\pgfqpoint{1.399831in}{2.047815in}}{\pgfqpoint{1.396559in}{2.039915in}}{\pgfqpoint{1.396559in}{2.031679in}}%
\pgfpathcurveto{\pgfqpoint{1.396559in}{2.023443in}}{\pgfqpoint{1.399831in}{2.015543in}}{\pgfqpoint{1.405655in}{2.009719in}}%
\pgfpathcurveto{\pgfqpoint{1.411479in}{2.003895in}}{\pgfqpoint{1.419379in}{2.000623in}}{\pgfqpoint{1.427615in}{2.000623in}}%
\pgfpathclose%
\pgfusepath{stroke,fill}%
\end{pgfscope}%
\begin{pgfscope}%
\pgfpathrectangle{\pgfqpoint{0.100000in}{0.212622in}}{\pgfqpoint{3.696000in}{3.696000in}}%
\pgfusepath{clip}%
\pgfsetbuttcap%
\pgfsetroundjoin%
\definecolor{currentfill}{rgb}{0.121569,0.466667,0.705882}%
\pgfsetfillcolor{currentfill}%
\pgfsetfillopacity{0.639007}%
\pgfsetlinewidth{1.003750pt}%
\definecolor{currentstroke}{rgb}{0.121569,0.466667,0.705882}%
\pgfsetstrokecolor{currentstroke}%
\pgfsetstrokeopacity{0.639007}%
\pgfsetdash{}{0pt}%
\pgfpathmoveto{\pgfqpoint{1.415681in}{1.986437in}}%
\pgfpathcurveto{\pgfqpoint{1.423918in}{1.986437in}}{\pgfqpoint{1.431818in}{1.989709in}}{\pgfqpoint{1.437642in}{1.995533in}}%
\pgfpathcurveto{\pgfqpoint{1.443466in}{2.001357in}}{\pgfqpoint{1.446738in}{2.009257in}}{\pgfqpoint{1.446738in}{2.017494in}}%
\pgfpathcurveto{\pgfqpoint{1.446738in}{2.025730in}}{\pgfqpoint{1.443466in}{2.033630in}}{\pgfqpoint{1.437642in}{2.039454in}}%
\pgfpathcurveto{\pgfqpoint{1.431818in}{2.045278in}}{\pgfqpoint{1.423918in}{2.048550in}}{\pgfqpoint{1.415681in}{2.048550in}}%
\pgfpathcurveto{\pgfqpoint{1.407445in}{2.048550in}}{\pgfqpoint{1.399545in}{2.045278in}}{\pgfqpoint{1.393721in}{2.039454in}}%
\pgfpathcurveto{\pgfqpoint{1.387897in}{2.033630in}}{\pgfqpoint{1.384625in}{2.025730in}}{\pgfqpoint{1.384625in}{2.017494in}}%
\pgfpathcurveto{\pgfqpoint{1.384625in}{2.009257in}}{\pgfqpoint{1.387897in}{2.001357in}}{\pgfqpoint{1.393721in}{1.995533in}}%
\pgfpathcurveto{\pgfqpoint{1.399545in}{1.989709in}}{\pgfqpoint{1.407445in}{1.986437in}}{\pgfqpoint{1.415681in}{1.986437in}}%
\pgfpathclose%
\pgfusepath{stroke,fill}%
\end{pgfscope}%
\begin{pgfscope}%
\pgfpathrectangle{\pgfqpoint{0.100000in}{0.212622in}}{\pgfqpoint{3.696000in}{3.696000in}}%
\pgfusepath{clip}%
\pgfsetbuttcap%
\pgfsetroundjoin%
\definecolor{currentfill}{rgb}{0.121569,0.466667,0.705882}%
\pgfsetfillcolor{currentfill}%
\pgfsetfillopacity{0.651879}%
\pgfsetlinewidth{1.003750pt}%
\definecolor{currentstroke}{rgb}{0.121569,0.466667,0.705882}%
\pgfsetstrokecolor{currentstroke}%
\pgfsetstrokeopacity{0.651879}%
\pgfsetdash{}{0pt}%
\pgfpathmoveto{\pgfqpoint{1.409037in}{1.992931in}}%
\pgfpathcurveto{\pgfqpoint{1.417274in}{1.992931in}}{\pgfqpoint{1.425174in}{1.996203in}}{\pgfqpoint{1.430998in}{2.002027in}}%
\pgfpathcurveto{\pgfqpoint{1.436822in}{2.007851in}}{\pgfqpoint{1.440094in}{2.015751in}}{\pgfqpoint{1.440094in}{2.023987in}}%
\pgfpathcurveto{\pgfqpoint{1.440094in}{2.032224in}}{\pgfqpoint{1.436822in}{2.040124in}}{\pgfqpoint{1.430998in}{2.045948in}}%
\pgfpathcurveto{\pgfqpoint{1.425174in}{2.051771in}}{\pgfqpoint{1.417274in}{2.055044in}}{\pgfqpoint{1.409037in}{2.055044in}}%
\pgfpathcurveto{\pgfqpoint{1.400801in}{2.055044in}}{\pgfqpoint{1.392901in}{2.051771in}}{\pgfqpoint{1.387077in}{2.045948in}}%
\pgfpathcurveto{\pgfqpoint{1.381253in}{2.040124in}}{\pgfqpoint{1.377981in}{2.032224in}}{\pgfqpoint{1.377981in}{2.023987in}}%
\pgfpathcurveto{\pgfqpoint{1.377981in}{2.015751in}}{\pgfqpoint{1.381253in}{2.007851in}}{\pgfqpoint{1.387077in}{2.002027in}}%
\pgfpathcurveto{\pgfqpoint{1.392901in}{1.996203in}}{\pgfqpoint{1.400801in}{1.992931in}}{\pgfqpoint{1.409037in}{1.992931in}}%
\pgfpathclose%
\pgfusepath{stroke,fill}%
\end{pgfscope}%
\begin{pgfscope}%
\pgfpathrectangle{\pgfqpoint{0.100000in}{0.212622in}}{\pgfqpoint{3.696000in}{3.696000in}}%
\pgfusepath{clip}%
\pgfsetbuttcap%
\pgfsetroundjoin%
\definecolor{currentfill}{rgb}{0.121569,0.466667,0.705882}%
\pgfsetfillcolor{currentfill}%
\pgfsetfillopacity{0.652877}%
\pgfsetlinewidth{1.003750pt}%
\definecolor{currentstroke}{rgb}{0.121569,0.466667,0.705882}%
\pgfsetstrokecolor{currentstroke}%
\pgfsetstrokeopacity{0.652877}%
\pgfsetdash{}{0pt}%
\pgfpathmoveto{\pgfqpoint{1.413575in}{1.999708in}}%
\pgfpathcurveto{\pgfqpoint{1.421812in}{1.999708in}}{\pgfqpoint{1.429712in}{2.002980in}}{\pgfqpoint{1.435536in}{2.008804in}}%
\pgfpathcurveto{\pgfqpoint{1.441360in}{2.014628in}}{\pgfqpoint{1.444632in}{2.022528in}}{\pgfqpoint{1.444632in}{2.030764in}}%
\pgfpathcurveto{\pgfqpoint{1.444632in}{2.039001in}}{\pgfqpoint{1.441360in}{2.046901in}}{\pgfqpoint{1.435536in}{2.052725in}}%
\pgfpathcurveto{\pgfqpoint{1.429712in}{2.058549in}}{\pgfqpoint{1.421812in}{2.061821in}}{\pgfqpoint{1.413575in}{2.061821in}}%
\pgfpathcurveto{\pgfqpoint{1.405339in}{2.061821in}}{\pgfqpoint{1.397439in}{2.058549in}}{\pgfqpoint{1.391615in}{2.052725in}}%
\pgfpathcurveto{\pgfqpoint{1.385791in}{2.046901in}}{\pgfqpoint{1.382519in}{2.039001in}}{\pgfqpoint{1.382519in}{2.030764in}}%
\pgfpathcurveto{\pgfqpoint{1.382519in}{2.022528in}}{\pgfqpoint{1.385791in}{2.014628in}}{\pgfqpoint{1.391615in}{2.008804in}}%
\pgfpathcurveto{\pgfqpoint{1.397439in}{2.002980in}}{\pgfqpoint{1.405339in}{1.999708in}}{\pgfqpoint{1.413575in}{1.999708in}}%
\pgfpathclose%
\pgfusepath{stroke,fill}%
\end{pgfscope}%
\begin{pgfscope}%
\pgfpathrectangle{\pgfqpoint{0.100000in}{0.212622in}}{\pgfqpoint{3.696000in}{3.696000in}}%
\pgfusepath{clip}%
\pgfsetbuttcap%
\pgfsetroundjoin%
\definecolor{currentfill}{rgb}{0.121569,0.466667,0.705882}%
\pgfsetfillcolor{currentfill}%
\pgfsetfillopacity{0.669139}%
\pgfsetlinewidth{1.003750pt}%
\definecolor{currentstroke}{rgb}{0.121569,0.466667,0.705882}%
\pgfsetstrokecolor{currentstroke}%
\pgfsetstrokeopacity{0.669139}%
\pgfsetdash{}{0pt}%
\pgfpathmoveto{\pgfqpoint{1.382326in}{1.972002in}}%
\pgfpathcurveto{\pgfqpoint{1.390563in}{1.972002in}}{\pgfqpoint{1.398463in}{1.975275in}}{\pgfqpoint{1.404287in}{1.981099in}}%
\pgfpathcurveto{\pgfqpoint{1.410110in}{1.986923in}}{\pgfqpoint{1.413383in}{1.994823in}}{\pgfqpoint{1.413383in}{2.003059in}}%
\pgfpathcurveto{\pgfqpoint{1.413383in}{2.011295in}}{\pgfqpoint{1.410110in}{2.019195in}}{\pgfqpoint{1.404287in}{2.025019in}}%
\pgfpathcurveto{\pgfqpoint{1.398463in}{2.030843in}}{\pgfqpoint{1.390563in}{2.034115in}}{\pgfqpoint{1.382326in}{2.034115in}}%
\pgfpathcurveto{\pgfqpoint{1.374090in}{2.034115in}}{\pgfqpoint{1.366190in}{2.030843in}}{\pgfqpoint{1.360366in}{2.025019in}}%
\pgfpathcurveto{\pgfqpoint{1.354542in}{2.019195in}}{\pgfqpoint{1.351270in}{2.011295in}}{\pgfqpoint{1.351270in}{2.003059in}}%
\pgfpathcurveto{\pgfqpoint{1.351270in}{1.994823in}}{\pgfqpoint{1.354542in}{1.986923in}}{\pgfqpoint{1.360366in}{1.981099in}}%
\pgfpathcurveto{\pgfqpoint{1.366190in}{1.975275in}}{\pgfqpoint{1.374090in}{1.972002in}}{\pgfqpoint{1.382326in}{1.972002in}}%
\pgfpathclose%
\pgfusepath{stroke,fill}%
\end{pgfscope}%
\begin{pgfscope}%
\pgfpathrectangle{\pgfqpoint{0.100000in}{0.212622in}}{\pgfqpoint{3.696000in}{3.696000in}}%
\pgfusepath{clip}%
\pgfsetbuttcap%
\pgfsetroundjoin%
\definecolor{currentfill}{rgb}{0.121569,0.466667,0.705882}%
\pgfsetfillcolor{currentfill}%
\pgfsetfillopacity{0.675593}%
\pgfsetlinewidth{1.003750pt}%
\definecolor{currentstroke}{rgb}{0.121569,0.466667,0.705882}%
\pgfsetstrokecolor{currentstroke}%
\pgfsetstrokeopacity{0.675593}%
\pgfsetdash{}{0pt}%
\pgfpathmoveto{\pgfqpoint{1.378564in}{1.968914in}}%
\pgfpathcurveto{\pgfqpoint{1.386800in}{1.968914in}}{\pgfqpoint{1.394700in}{1.972186in}}{\pgfqpoint{1.400524in}{1.978010in}}%
\pgfpathcurveto{\pgfqpoint{1.406348in}{1.983834in}}{\pgfqpoint{1.409620in}{1.991734in}}{\pgfqpoint{1.409620in}{1.999970in}}%
\pgfpathcurveto{\pgfqpoint{1.409620in}{2.008206in}}{\pgfqpoint{1.406348in}{2.016106in}}{\pgfqpoint{1.400524in}{2.021930in}}%
\pgfpathcurveto{\pgfqpoint{1.394700in}{2.027754in}}{\pgfqpoint{1.386800in}{2.031027in}}{\pgfqpoint{1.378564in}{2.031027in}}%
\pgfpathcurveto{\pgfqpoint{1.370327in}{2.031027in}}{\pgfqpoint{1.362427in}{2.027754in}}{\pgfqpoint{1.356603in}{2.021930in}}%
\pgfpathcurveto{\pgfqpoint{1.350779in}{2.016106in}}{\pgfqpoint{1.347507in}{2.008206in}}{\pgfqpoint{1.347507in}{1.999970in}}%
\pgfpathcurveto{\pgfqpoint{1.347507in}{1.991734in}}{\pgfqpoint{1.350779in}{1.983834in}}{\pgfqpoint{1.356603in}{1.978010in}}%
\pgfpathcurveto{\pgfqpoint{1.362427in}{1.972186in}}{\pgfqpoint{1.370327in}{1.968914in}}{\pgfqpoint{1.378564in}{1.968914in}}%
\pgfpathclose%
\pgfusepath{stroke,fill}%
\end{pgfscope}%
\begin{pgfscope}%
\pgfpathrectangle{\pgfqpoint{0.100000in}{0.212622in}}{\pgfqpoint{3.696000in}{3.696000in}}%
\pgfusepath{clip}%
\pgfsetbuttcap%
\pgfsetroundjoin%
\definecolor{currentfill}{rgb}{0.121569,0.466667,0.705882}%
\pgfsetfillcolor{currentfill}%
\pgfsetfillopacity{0.682191}%
\pgfsetlinewidth{1.003750pt}%
\definecolor{currentstroke}{rgb}{0.121569,0.466667,0.705882}%
\pgfsetstrokecolor{currentstroke}%
\pgfsetstrokeopacity{0.682191}%
\pgfsetdash{}{0pt}%
\pgfpathmoveto{\pgfqpoint{1.375809in}{1.964993in}}%
\pgfpathcurveto{\pgfqpoint{1.384046in}{1.964993in}}{\pgfqpoint{1.391946in}{1.968265in}}{\pgfqpoint{1.397770in}{1.974089in}}%
\pgfpathcurveto{\pgfqpoint{1.403593in}{1.979913in}}{\pgfqpoint{1.406866in}{1.987813in}}{\pgfqpoint{1.406866in}{1.996049in}}%
\pgfpathcurveto{\pgfqpoint{1.406866in}{2.004285in}}{\pgfqpoint{1.403593in}{2.012185in}}{\pgfqpoint{1.397770in}{2.018009in}}%
\pgfpathcurveto{\pgfqpoint{1.391946in}{2.023833in}}{\pgfqpoint{1.384046in}{2.027106in}}{\pgfqpoint{1.375809in}{2.027106in}}%
\pgfpathcurveto{\pgfqpoint{1.367573in}{2.027106in}}{\pgfqpoint{1.359673in}{2.023833in}}{\pgfqpoint{1.353849in}{2.018009in}}%
\pgfpathcurveto{\pgfqpoint{1.348025in}{2.012185in}}{\pgfqpoint{1.344753in}{2.004285in}}{\pgfqpoint{1.344753in}{1.996049in}}%
\pgfpathcurveto{\pgfqpoint{1.344753in}{1.987813in}}{\pgfqpoint{1.348025in}{1.979913in}}{\pgfqpoint{1.353849in}{1.974089in}}%
\pgfpathcurveto{\pgfqpoint{1.359673in}{1.968265in}}{\pgfqpoint{1.367573in}{1.964993in}}{\pgfqpoint{1.375809in}{1.964993in}}%
\pgfpathclose%
\pgfusepath{stroke,fill}%
\end{pgfscope}%
\begin{pgfscope}%
\pgfpathrectangle{\pgfqpoint{0.100000in}{0.212622in}}{\pgfqpoint{3.696000in}{3.696000in}}%
\pgfusepath{clip}%
\pgfsetbuttcap%
\pgfsetroundjoin%
\definecolor{currentfill}{rgb}{0.121569,0.466667,0.705882}%
\pgfsetfillcolor{currentfill}%
\pgfsetfillopacity{0.685161}%
\pgfsetlinewidth{1.003750pt}%
\definecolor{currentstroke}{rgb}{0.121569,0.466667,0.705882}%
\pgfsetstrokecolor{currentstroke}%
\pgfsetstrokeopacity{0.685161}%
\pgfsetdash{}{0pt}%
\pgfpathmoveto{\pgfqpoint{1.378301in}{1.972485in}}%
\pgfpathcurveto{\pgfqpoint{1.386537in}{1.972485in}}{\pgfqpoint{1.394437in}{1.975757in}}{\pgfqpoint{1.400261in}{1.981581in}}%
\pgfpathcurveto{\pgfqpoint{1.406085in}{1.987405in}}{\pgfqpoint{1.409357in}{1.995305in}}{\pgfqpoint{1.409357in}{2.003541in}}%
\pgfpathcurveto{\pgfqpoint{1.409357in}{2.011777in}}{\pgfqpoint{1.406085in}{2.019677in}}{\pgfqpoint{1.400261in}{2.025501in}}%
\pgfpathcurveto{\pgfqpoint{1.394437in}{2.031325in}}{\pgfqpoint{1.386537in}{2.034598in}}{\pgfqpoint{1.378301in}{2.034598in}}%
\pgfpathcurveto{\pgfqpoint{1.370064in}{2.034598in}}{\pgfqpoint{1.362164in}{2.031325in}}{\pgfqpoint{1.356340in}{2.025501in}}%
\pgfpathcurveto{\pgfqpoint{1.350517in}{2.019677in}}{\pgfqpoint{1.347244in}{2.011777in}}{\pgfqpoint{1.347244in}{2.003541in}}%
\pgfpathcurveto{\pgfqpoint{1.347244in}{1.995305in}}{\pgfqpoint{1.350517in}{1.987405in}}{\pgfqpoint{1.356340in}{1.981581in}}%
\pgfpathcurveto{\pgfqpoint{1.362164in}{1.975757in}}{\pgfqpoint{1.370064in}{1.972485in}}{\pgfqpoint{1.378301in}{1.972485in}}%
\pgfpathclose%
\pgfusepath{stroke,fill}%
\end{pgfscope}%
\begin{pgfscope}%
\pgfpathrectangle{\pgfqpoint{0.100000in}{0.212622in}}{\pgfqpoint{3.696000in}{3.696000in}}%
\pgfusepath{clip}%
\pgfsetbuttcap%
\pgfsetroundjoin%
\definecolor{currentfill}{rgb}{0.121569,0.466667,0.705882}%
\pgfsetfillcolor{currentfill}%
\pgfsetfillopacity{0.696567}%
\pgfsetlinewidth{1.003750pt}%
\definecolor{currentstroke}{rgb}{0.121569,0.466667,0.705882}%
\pgfsetstrokecolor{currentstroke}%
\pgfsetstrokeopacity{0.696567}%
\pgfsetdash{}{0pt}%
\pgfpathmoveto{\pgfqpoint{1.360156in}{1.952534in}}%
\pgfpathcurveto{\pgfqpoint{1.368392in}{1.952534in}}{\pgfqpoint{1.376292in}{1.955806in}}{\pgfqpoint{1.382116in}{1.961630in}}%
\pgfpathcurveto{\pgfqpoint{1.387940in}{1.967454in}}{\pgfqpoint{1.391212in}{1.975354in}}{\pgfqpoint{1.391212in}{1.983590in}}%
\pgfpathcurveto{\pgfqpoint{1.391212in}{1.991827in}}{\pgfqpoint{1.387940in}{1.999727in}}{\pgfqpoint{1.382116in}{2.005551in}}%
\pgfpathcurveto{\pgfqpoint{1.376292in}{2.011375in}}{\pgfqpoint{1.368392in}{2.014647in}}{\pgfqpoint{1.360156in}{2.014647in}}%
\pgfpathcurveto{\pgfqpoint{1.351920in}{2.014647in}}{\pgfqpoint{1.344019in}{2.011375in}}{\pgfqpoint{1.338196in}{2.005551in}}%
\pgfpathcurveto{\pgfqpoint{1.332372in}{1.999727in}}{\pgfqpoint{1.329099in}{1.991827in}}{\pgfqpoint{1.329099in}{1.983590in}}%
\pgfpathcurveto{\pgfqpoint{1.329099in}{1.975354in}}{\pgfqpoint{1.332372in}{1.967454in}}{\pgfqpoint{1.338196in}{1.961630in}}%
\pgfpathcurveto{\pgfqpoint{1.344019in}{1.955806in}}{\pgfqpoint{1.351920in}{1.952534in}}{\pgfqpoint{1.360156in}{1.952534in}}%
\pgfpathclose%
\pgfusepath{stroke,fill}%
\end{pgfscope}%
\begin{pgfscope}%
\pgfpathrectangle{\pgfqpoint{0.100000in}{0.212622in}}{\pgfqpoint{3.696000in}{3.696000in}}%
\pgfusepath{clip}%
\pgfsetbuttcap%
\pgfsetroundjoin%
\definecolor{currentfill}{rgb}{0.121569,0.466667,0.705882}%
\pgfsetfillcolor{currentfill}%
\pgfsetfillopacity{0.700918}%
\pgfsetlinewidth{1.003750pt}%
\definecolor{currentstroke}{rgb}{0.121569,0.466667,0.705882}%
\pgfsetstrokecolor{currentstroke}%
\pgfsetstrokeopacity{0.700918}%
\pgfsetdash{}{0pt}%
\pgfpathmoveto{\pgfqpoint{1.365367in}{1.963117in}}%
\pgfpathcurveto{\pgfqpoint{1.373603in}{1.963117in}}{\pgfqpoint{1.381503in}{1.966390in}}{\pgfqpoint{1.387327in}{1.972213in}}%
\pgfpathcurveto{\pgfqpoint{1.393151in}{1.978037in}}{\pgfqpoint{1.396423in}{1.985937in}}{\pgfqpoint{1.396423in}{1.994174in}}%
\pgfpathcurveto{\pgfqpoint{1.396423in}{2.002410in}}{\pgfqpoint{1.393151in}{2.010310in}}{\pgfqpoint{1.387327in}{2.016134in}}%
\pgfpathcurveto{\pgfqpoint{1.381503in}{2.021958in}}{\pgfqpoint{1.373603in}{2.025230in}}{\pgfqpoint{1.365367in}{2.025230in}}%
\pgfpathcurveto{\pgfqpoint{1.357131in}{2.025230in}}{\pgfqpoint{1.349231in}{2.021958in}}{\pgfqpoint{1.343407in}{2.016134in}}%
\pgfpathcurveto{\pgfqpoint{1.337583in}{2.010310in}}{\pgfqpoint{1.334310in}{2.002410in}}{\pgfqpoint{1.334310in}{1.994174in}}%
\pgfpathcurveto{\pgfqpoint{1.334310in}{1.985937in}}{\pgfqpoint{1.337583in}{1.978037in}}{\pgfqpoint{1.343407in}{1.972213in}}%
\pgfpathcurveto{\pgfqpoint{1.349231in}{1.966390in}}{\pgfqpoint{1.357131in}{1.963117in}}{\pgfqpoint{1.365367in}{1.963117in}}%
\pgfpathclose%
\pgfusepath{stroke,fill}%
\end{pgfscope}%
\begin{pgfscope}%
\pgfpathrectangle{\pgfqpoint{0.100000in}{0.212622in}}{\pgfqpoint{3.696000in}{3.696000in}}%
\pgfusepath{clip}%
\pgfsetbuttcap%
\pgfsetroundjoin%
\definecolor{currentfill}{rgb}{0.121569,0.466667,0.705882}%
\pgfsetfillcolor{currentfill}%
\pgfsetfillopacity{0.703475}%
\pgfsetlinewidth{1.003750pt}%
\definecolor{currentstroke}{rgb}{0.121569,0.466667,0.705882}%
\pgfsetstrokecolor{currentstroke}%
\pgfsetstrokeopacity{0.703475}%
\pgfsetdash{}{0pt}%
\pgfpathmoveto{\pgfqpoint{1.367733in}{1.968229in}}%
\pgfpathcurveto{\pgfqpoint{1.375970in}{1.968229in}}{\pgfqpoint{1.383870in}{1.971501in}}{\pgfqpoint{1.389694in}{1.977325in}}%
\pgfpathcurveto{\pgfqpoint{1.395518in}{1.983149in}}{\pgfqpoint{1.398790in}{1.991049in}}{\pgfqpoint{1.398790in}{1.999285in}}%
\pgfpathcurveto{\pgfqpoint{1.398790in}{2.007521in}}{\pgfqpoint{1.395518in}{2.015421in}}{\pgfqpoint{1.389694in}{2.021245in}}%
\pgfpathcurveto{\pgfqpoint{1.383870in}{2.027069in}}{\pgfqpoint{1.375970in}{2.030342in}}{\pgfqpoint{1.367733in}{2.030342in}}%
\pgfpathcurveto{\pgfqpoint{1.359497in}{2.030342in}}{\pgfqpoint{1.351597in}{2.027069in}}{\pgfqpoint{1.345773in}{2.021245in}}%
\pgfpathcurveto{\pgfqpoint{1.339949in}{2.015421in}}{\pgfqpoint{1.336677in}{2.007521in}}{\pgfqpoint{1.336677in}{1.999285in}}%
\pgfpathcurveto{\pgfqpoint{1.336677in}{1.991049in}}{\pgfqpoint{1.339949in}{1.983149in}}{\pgfqpoint{1.345773in}{1.977325in}}%
\pgfpathcurveto{\pgfqpoint{1.351597in}{1.971501in}}{\pgfqpoint{1.359497in}{1.968229in}}{\pgfqpoint{1.367733in}{1.968229in}}%
\pgfpathclose%
\pgfusepath{stroke,fill}%
\end{pgfscope}%
\begin{pgfscope}%
\pgfpathrectangle{\pgfqpoint{0.100000in}{0.212622in}}{\pgfqpoint{3.696000in}{3.696000in}}%
\pgfusepath{clip}%
\pgfsetbuttcap%
\pgfsetroundjoin%
\definecolor{currentfill}{rgb}{0.121569,0.466667,0.705882}%
\pgfsetfillcolor{currentfill}%
\pgfsetfillopacity{0.713335}%
\pgfsetlinewidth{1.003750pt}%
\definecolor{currentstroke}{rgb}{0.121569,0.466667,0.705882}%
\pgfsetstrokecolor{currentstroke}%
\pgfsetstrokeopacity{0.713335}%
\pgfsetdash{}{0pt}%
\pgfpathmoveto{\pgfqpoint{1.352447in}{1.954227in}}%
\pgfpathcurveto{\pgfqpoint{1.360684in}{1.954227in}}{\pgfqpoint{1.368584in}{1.957499in}}{\pgfqpoint{1.374408in}{1.963323in}}%
\pgfpathcurveto{\pgfqpoint{1.380232in}{1.969147in}}{\pgfqpoint{1.383504in}{1.977047in}}{\pgfqpoint{1.383504in}{1.985284in}}%
\pgfpathcurveto{\pgfqpoint{1.383504in}{1.993520in}}{\pgfqpoint{1.380232in}{2.001420in}}{\pgfqpoint{1.374408in}{2.007244in}}%
\pgfpathcurveto{\pgfqpoint{1.368584in}{2.013068in}}{\pgfqpoint{1.360684in}{2.016340in}}{\pgfqpoint{1.352447in}{2.016340in}}%
\pgfpathcurveto{\pgfqpoint{1.344211in}{2.016340in}}{\pgfqpoint{1.336311in}{2.013068in}}{\pgfqpoint{1.330487in}{2.007244in}}%
\pgfpathcurveto{\pgfqpoint{1.324663in}{2.001420in}}{\pgfqpoint{1.321391in}{1.993520in}}{\pgfqpoint{1.321391in}{1.985284in}}%
\pgfpathcurveto{\pgfqpoint{1.321391in}{1.977047in}}{\pgfqpoint{1.324663in}{1.969147in}}{\pgfqpoint{1.330487in}{1.963323in}}%
\pgfpathcurveto{\pgfqpoint{1.336311in}{1.957499in}}{\pgfqpoint{1.344211in}{1.954227in}}{\pgfqpoint{1.352447in}{1.954227in}}%
\pgfpathclose%
\pgfusepath{stroke,fill}%
\end{pgfscope}%
\begin{pgfscope}%
\pgfpathrectangle{\pgfqpoint{0.100000in}{0.212622in}}{\pgfqpoint{3.696000in}{3.696000in}}%
\pgfusepath{clip}%
\pgfsetbuttcap%
\pgfsetroundjoin%
\definecolor{currentfill}{rgb}{0.121569,0.466667,0.705882}%
\pgfsetfillcolor{currentfill}%
\pgfsetfillopacity{0.725143}%
\pgfsetlinewidth{1.003750pt}%
\definecolor{currentstroke}{rgb}{0.121569,0.466667,0.705882}%
\pgfsetstrokecolor{currentstroke}%
\pgfsetstrokeopacity{0.725143}%
\pgfsetdash{}{0pt}%
\pgfpathmoveto{\pgfqpoint{1.339614in}{1.954525in}}%
\pgfpathcurveto{\pgfqpoint{1.347850in}{1.954525in}}{\pgfqpoint{1.355751in}{1.957797in}}{\pgfqpoint{1.361574in}{1.963621in}}%
\pgfpathcurveto{\pgfqpoint{1.367398in}{1.969445in}}{\pgfqpoint{1.370671in}{1.977345in}}{\pgfqpoint{1.370671in}{1.985581in}}%
\pgfpathcurveto{\pgfqpoint{1.370671in}{1.993818in}}{\pgfqpoint{1.367398in}{2.001718in}}{\pgfqpoint{1.361574in}{2.007542in}}%
\pgfpathcurveto{\pgfqpoint{1.355751in}{2.013366in}}{\pgfqpoint{1.347850in}{2.016638in}}{\pgfqpoint{1.339614in}{2.016638in}}%
\pgfpathcurveto{\pgfqpoint{1.331378in}{2.016638in}}{\pgfqpoint{1.323478in}{2.013366in}}{\pgfqpoint{1.317654in}{2.007542in}}%
\pgfpathcurveto{\pgfqpoint{1.311830in}{2.001718in}}{\pgfqpoint{1.308558in}{1.993818in}}{\pgfqpoint{1.308558in}{1.985581in}}%
\pgfpathcurveto{\pgfqpoint{1.308558in}{1.977345in}}{\pgfqpoint{1.311830in}{1.969445in}}{\pgfqpoint{1.317654in}{1.963621in}}%
\pgfpathcurveto{\pgfqpoint{1.323478in}{1.957797in}}{\pgfqpoint{1.331378in}{1.954525in}}{\pgfqpoint{1.339614in}{1.954525in}}%
\pgfpathclose%
\pgfusepath{stroke,fill}%
\end{pgfscope}%
\begin{pgfscope}%
\pgfpathrectangle{\pgfqpoint{0.100000in}{0.212622in}}{\pgfqpoint{3.696000in}{3.696000in}}%
\pgfusepath{clip}%
\pgfsetbuttcap%
\pgfsetroundjoin%
\definecolor{currentfill}{rgb}{0.121569,0.466667,0.705882}%
\pgfsetfillcolor{currentfill}%
\pgfsetfillopacity{0.738336}%
\pgfsetlinewidth{1.003750pt}%
\definecolor{currentstroke}{rgb}{0.121569,0.466667,0.705882}%
\pgfsetstrokecolor{currentstroke}%
\pgfsetstrokeopacity{0.738336}%
\pgfsetdash{}{0pt}%
\pgfpathmoveto{\pgfqpoint{1.317435in}{1.927039in}}%
\pgfpathcurveto{\pgfqpoint{1.325671in}{1.927039in}}{\pgfqpoint{1.333571in}{1.930311in}}{\pgfqpoint{1.339395in}{1.936135in}}%
\pgfpathcurveto{\pgfqpoint{1.345219in}{1.941959in}}{\pgfqpoint{1.348492in}{1.949859in}}{\pgfqpoint{1.348492in}{1.958095in}}%
\pgfpathcurveto{\pgfqpoint{1.348492in}{1.966332in}}{\pgfqpoint{1.345219in}{1.974232in}}{\pgfqpoint{1.339395in}{1.980056in}}%
\pgfpathcurveto{\pgfqpoint{1.333571in}{1.985880in}}{\pgfqpoint{1.325671in}{1.989152in}}{\pgfqpoint{1.317435in}{1.989152in}}%
\pgfpathcurveto{\pgfqpoint{1.309199in}{1.989152in}}{\pgfqpoint{1.301299in}{1.985880in}}{\pgfqpoint{1.295475in}{1.980056in}}%
\pgfpathcurveto{\pgfqpoint{1.289651in}{1.974232in}}{\pgfqpoint{1.286379in}{1.966332in}}{\pgfqpoint{1.286379in}{1.958095in}}%
\pgfpathcurveto{\pgfqpoint{1.286379in}{1.949859in}}{\pgfqpoint{1.289651in}{1.941959in}}{\pgfqpoint{1.295475in}{1.936135in}}%
\pgfpathcurveto{\pgfqpoint{1.301299in}{1.930311in}}{\pgfqpoint{1.309199in}{1.927039in}}{\pgfqpoint{1.317435in}{1.927039in}}%
\pgfpathclose%
\pgfusepath{stroke,fill}%
\end{pgfscope}%
\begin{pgfscope}%
\pgfpathrectangle{\pgfqpoint{0.100000in}{0.212622in}}{\pgfqpoint{3.696000in}{3.696000in}}%
\pgfusepath{clip}%
\pgfsetbuttcap%
\pgfsetroundjoin%
\definecolor{currentfill}{rgb}{0.121569,0.466667,0.705882}%
\pgfsetfillcolor{currentfill}%
\pgfsetfillopacity{0.740948}%
\pgfsetlinewidth{1.003750pt}%
\definecolor{currentstroke}{rgb}{0.121569,0.466667,0.705882}%
\pgfsetstrokecolor{currentstroke}%
\pgfsetstrokeopacity{0.740948}%
\pgfsetdash{}{0pt}%
\pgfpathmoveto{\pgfqpoint{1.317993in}{1.926666in}}%
\pgfpathcurveto{\pgfqpoint{1.326229in}{1.926666in}}{\pgfqpoint{1.334129in}{1.929938in}}{\pgfqpoint{1.339953in}{1.935762in}}%
\pgfpathcurveto{\pgfqpoint{1.345777in}{1.941586in}}{\pgfqpoint{1.349049in}{1.949486in}}{\pgfqpoint{1.349049in}{1.957722in}}%
\pgfpathcurveto{\pgfqpoint{1.349049in}{1.965958in}}{\pgfqpoint{1.345777in}{1.973858in}}{\pgfqpoint{1.339953in}{1.979682in}}%
\pgfpathcurveto{\pgfqpoint{1.334129in}{1.985506in}}{\pgfqpoint{1.326229in}{1.988779in}}{\pgfqpoint{1.317993in}{1.988779in}}%
\pgfpathcurveto{\pgfqpoint{1.309757in}{1.988779in}}{\pgfqpoint{1.301856in}{1.985506in}}{\pgfqpoint{1.296033in}{1.979682in}}%
\pgfpathcurveto{\pgfqpoint{1.290209in}{1.973858in}}{\pgfqpoint{1.286936in}{1.965958in}}{\pgfqpoint{1.286936in}{1.957722in}}%
\pgfpathcurveto{\pgfqpoint{1.286936in}{1.949486in}}{\pgfqpoint{1.290209in}{1.941586in}}{\pgfqpoint{1.296033in}{1.935762in}}%
\pgfpathcurveto{\pgfqpoint{1.301856in}{1.929938in}}{\pgfqpoint{1.309757in}{1.926666in}}{\pgfqpoint{1.317993in}{1.926666in}}%
\pgfpathclose%
\pgfusepath{stroke,fill}%
\end{pgfscope}%
\begin{pgfscope}%
\pgfpathrectangle{\pgfqpoint{0.100000in}{0.212622in}}{\pgfqpoint{3.696000in}{3.696000in}}%
\pgfusepath{clip}%
\pgfsetbuttcap%
\pgfsetroundjoin%
\definecolor{currentfill}{rgb}{0.121569,0.466667,0.705882}%
\pgfsetfillcolor{currentfill}%
\pgfsetfillopacity{0.744171}%
\pgfsetlinewidth{1.003750pt}%
\definecolor{currentstroke}{rgb}{0.121569,0.466667,0.705882}%
\pgfsetstrokecolor{currentstroke}%
\pgfsetstrokeopacity{0.744171}%
\pgfsetdash{}{0pt}%
\pgfpathmoveto{\pgfqpoint{1.319000in}{1.928898in}}%
\pgfpathcurveto{\pgfqpoint{1.327236in}{1.928898in}}{\pgfqpoint{1.335136in}{1.932171in}}{\pgfqpoint{1.340960in}{1.937995in}}%
\pgfpathcurveto{\pgfqpoint{1.346784in}{1.943818in}}{\pgfqpoint{1.350056in}{1.951719in}}{\pgfqpoint{1.350056in}{1.959955in}}%
\pgfpathcurveto{\pgfqpoint{1.350056in}{1.968191in}}{\pgfqpoint{1.346784in}{1.976091in}}{\pgfqpoint{1.340960in}{1.981915in}}%
\pgfpathcurveto{\pgfqpoint{1.335136in}{1.987739in}}{\pgfqpoint{1.327236in}{1.991011in}}{\pgfqpoint{1.319000in}{1.991011in}}%
\pgfpathcurveto{\pgfqpoint{1.310764in}{1.991011in}}{\pgfqpoint{1.302864in}{1.987739in}}{\pgfqpoint{1.297040in}{1.981915in}}%
\pgfpathcurveto{\pgfqpoint{1.291216in}{1.976091in}}{\pgfqpoint{1.287943in}{1.968191in}}{\pgfqpoint{1.287943in}{1.959955in}}%
\pgfpathcurveto{\pgfqpoint{1.287943in}{1.951719in}}{\pgfqpoint{1.291216in}{1.943818in}}{\pgfqpoint{1.297040in}{1.937995in}}%
\pgfpathcurveto{\pgfqpoint{1.302864in}{1.932171in}}{\pgfqpoint{1.310764in}{1.928898in}}{\pgfqpoint{1.319000in}{1.928898in}}%
\pgfpathclose%
\pgfusepath{stroke,fill}%
\end{pgfscope}%
\begin{pgfscope}%
\pgfpathrectangle{\pgfqpoint{0.100000in}{0.212622in}}{\pgfqpoint{3.696000in}{3.696000in}}%
\pgfusepath{clip}%
\pgfsetbuttcap%
\pgfsetroundjoin%
\definecolor{currentfill}{rgb}{0.121569,0.466667,0.705882}%
\pgfsetfillcolor{currentfill}%
\pgfsetfillopacity{0.754813}%
\pgfsetlinewidth{1.003750pt}%
\definecolor{currentstroke}{rgb}{0.121569,0.466667,0.705882}%
\pgfsetstrokecolor{currentstroke}%
\pgfsetstrokeopacity{0.754813}%
\pgfsetdash{}{0pt}%
\pgfpathmoveto{\pgfqpoint{1.301604in}{1.906364in}}%
\pgfpathcurveto{\pgfqpoint{1.309840in}{1.906364in}}{\pgfqpoint{1.317740in}{1.909636in}}{\pgfqpoint{1.323564in}{1.915460in}}%
\pgfpathcurveto{\pgfqpoint{1.329388in}{1.921284in}}{\pgfqpoint{1.332661in}{1.929184in}}{\pgfqpoint{1.332661in}{1.937421in}}%
\pgfpathcurveto{\pgfqpoint{1.332661in}{1.945657in}}{\pgfqpoint{1.329388in}{1.953557in}}{\pgfqpoint{1.323564in}{1.959381in}}%
\pgfpathcurveto{\pgfqpoint{1.317740in}{1.965205in}}{\pgfqpoint{1.309840in}{1.968477in}}{\pgfqpoint{1.301604in}{1.968477in}}%
\pgfpathcurveto{\pgfqpoint{1.293368in}{1.968477in}}{\pgfqpoint{1.285468in}{1.965205in}}{\pgfqpoint{1.279644in}{1.959381in}}%
\pgfpathcurveto{\pgfqpoint{1.273820in}{1.953557in}}{\pgfqpoint{1.270548in}{1.945657in}}{\pgfqpoint{1.270548in}{1.937421in}}%
\pgfpathcurveto{\pgfqpoint{1.270548in}{1.929184in}}{\pgfqpoint{1.273820in}{1.921284in}}{\pgfqpoint{1.279644in}{1.915460in}}%
\pgfpathcurveto{\pgfqpoint{1.285468in}{1.909636in}}{\pgfqpoint{1.293368in}{1.906364in}}{\pgfqpoint{1.301604in}{1.906364in}}%
\pgfpathclose%
\pgfusepath{stroke,fill}%
\end{pgfscope}%
\begin{pgfscope}%
\pgfpathrectangle{\pgfqpoint{0.100000in}{0.212622in}}{\pgfqpoint{3.696000in}{3.696000in}}%
\pgfusepath{clip}%
\pgfsetbuttcap%
\pgfsetroundjoin%
\definecolor{currentfill}{rgb}{0.121569,0.466667,0.705882}%
\pgfsetfillcolor{currentfill}%
\pgfsetfillopacity{0.765027}%
\pgfsetlinewidth{1.003750pt}%
\definecolor{currentstroke}{rgb}{0.121569,0.466667,0.705882}%
\pgfsetstrokecolor{currentstroke}%
\pgfsetstrokeopacity{0.765027}%
\pgfsetdash{}{0pt}%
\pgfpathmoveto{\pgfqpoint{1.291375in}{1.899175in}}%
\pgfpathcurveto{\pgfqpoint{1.299611in}{1.899175in}}{\pgfqpoint{1.307511in}{1.902447in}}{\pgfqpoint{1.313335in}{1.908271in}}%
\pgfpathcurveto{\pgfqpoint{1.319159in}{1.914095in}}{\pgfqpoint{1.322431in}{1.921995in}}{\pgfqpoint{1.322431in}{1.930231in}}%
\pgfpathcurveto{\pgfqpoint{1.322431in}{1.938468in}}{\pgfqpoint{1.319159in}{1.946368in}}{\pgfqpoint{1.313335in}{1.952192in}}%
\pgfpathcurveto{\pgfqpoint{1.307511in}{1.958016in}}{\pgfqpoint{1.299611in}{1.961288in}}{\pgfqpoint{1.291375in}{1.961288in}}%
\pgfpathcurveto{\pgfqpoint{1.283139in}{1.961288in}}{\pgfqpoint{1.275239in}{1.958016in}}{\pgfqpoint{1.269415in}{1.952192in}}%
\pgfpathcurveto{\pgfqpoint{1.263591in}{1.946368in}}{\pgfqpoint{1.260318in}{1.938468in}}{\pgfqpoint{1.260318in}{1.930231in}}%
\pgfpathcurveto{\pgfqpoint{1.260318in}{1.921995in}}{\pgfqpoint{1.263591in}{1.914095in}}{\pgfqpoint{1.269415in}{1.908271in}}%
\pgfpathcurveto{\pgfqpoint{1.275239in}{1.902447in}}{\pgfqpoint{1.283139in}{1.899175in}}{\pgfqpoint{1.291375in}{1.899175in}}%
\pgfpathclose%
\pgfusepath{stroke,fill}%
\end{pgfscope}%
\begin{pgfscope}%
\pgfpathrectangle{\pgfqpoint{0.100000in}{0.212622in}}{\pgfqpoint{3.696000in}{3.696000in}}%
\pgfusepath{clip}%
\pgfsetbuttcap%
\pgfsetroundjoin%
\definecolor{currentfill}{rgb}{0.121569,0.466667,0.705882}%
\pgfsetfillcolor{currentfill}%
\pgfsetfillopacity{0.772122}%
\pgfsetlinewidth{1.003750pt}%
\definecolor{currentstroke}{rgb}{0.121569,0.466667,0.705882}%
\pgfsetstrokecolor{currentstroke}%
\pgfsetstrokeopacity{0.772122}%
\pgfsetdash{}{0pt}%
\pgfpathmoveto{\pgfqpoint{1.282389in}{1.892836in}}%
\pgfpathcurveto{\pgfqpoint{1.290625in}{1.892836in}}{\pgfqpoint{1.298525in}{1.896109in}}{\pgfqpoint{1.304349in}{1.901933in}}%
\pgfpathcurveto{\pgfqpoint{1.310173in}{1.907757in}}{\pgfqpoint{1.313445in}{1.915657in}}{\pgfqpoint{1.313445in}{1.923893in}}%
\pgfpathcurveto{\pgfqpoint{1.313445in}{1.932129in}}{\pgfqpoint{1.310173in}{1.940029in}}{\pgfqpoint{1.304349in}{1.945853in}}%
\pgfpathcurveto{\pgfqpoint{1.298525in}{1.951677in}}{\pgfqpoint{1.290625in}{1.954949in}}{\pgfqpoint{1.282389in}{1.954949in}}%
\pgfpathcurveto{\pgfqpoint{1.274152in}{1.954949in}}{\pgfqpoint{1.266252in}{1.951677in}}{\pgfqpoint{1.260428in}{1.945853in}}%
\pgfpathcurveto{\pgfqpoint{1.254604in}{1.940029in}}{\pgfqpoint{1.251332in}{1.932129in}}{\pgfqpoint{1.251332in}{1.923893in}}%
\pgfpathcurveto{\pgfqpoint{1.251332in}{1.915657in}}{\pgfqpoint{1.254604in}{1.907757in}}{\pgfqpoint{1.260428in}{1.901933in}}%
\pgfpathcurveto{\pgfqpoint{1.266252in}{1.896109in}}{\pgfqpoint{1.274152in}{1.892836in}}{\pgfqpoint{1.282389in}{1.892836in}}%
\pgfpathclose%
\pgfusepath{stroke,fill}%
\end{pgfscope}%
\begin{pgfscope}%
\pgfpathrectangle{\pgfqpoint{0.100000in}{0.212622in}}{\pgfqpoint{3.696000in}{3.696000in}}%
\pgfusepath{clip}%
\pgfsetbuttcap%
\pgfsetroundjoin%
\definecolor{currentfill}{rgb}{0.121569,0.466667,0.705882}%
\pgfsetfillcolor{currentfill}%
\pgfsetfillopacity{0.777020}%
\pgfsetlinewidth{1.003750pt}%
\definecolor{currentstroke}{rgb}{0.121569,0.466667,0.705882}%
\pgfsetstrokecolor{currentstroke}%
\pgfsetstrokeopacity{0.777020}%
\pgfsetdash{}{0pt}%
\pgfpathmoveto{\pgfqpoint{1.278125in}{1.885872in}}%
\pgfpathcurveto{\pgfqpoint{1.286361in}{1.885872in}}{\pgfqpoint{1.294261in}{1.889144in}}{\pgfqpoint{1.300085in}{1.894968in}}%
\pgfpathcurveto{\pgfqpoint{1.305909in}{1.900792in}}{\pgfqpoint{1.309181in}{1.908692in}}{\pgfqpoint{1.309181in}{1.916928in}}%
\pgfpathcurveto{\pgfqpoint{1.309181in}{1.925164in}}{\pgfqpoint{1.305909in}{1.933064in}}{\pgfqpoint{1.300085in}{1.938888in}}%
\pgfpathcurveto{\pgfqpoint{1.294261in}{1.944712in}}{\pgfqpoint{1.286361in}{1.947985in}}{\pgfqpoint{1.278125in}{1.947985in}}%
\pgfpathcurveto{\pgfqpoint{1.269889in}{1.947985in}}{\pgfqpoint{1.261989in}{1.944712in}}{\pgfqpoint{1.256165in}{1.938888in}}%
\pgfpathcurveto{\pgfqpoint{1.250341in}{1.933064in}}{\pgfqpoint{1.247068in}{1.925164in}}{\pgfqpoint{1.247068in}{1.916928in}}%
\pgfpathcurveto{\pgfqpoint{1.247068in}{1.908692in}}{\pgfqpoint{1.250341in}{1.900792in}}{\pgfqpoint{1.256165in}{1.894968in}}%
\pgfpathcurveto{\pgfqpoint{1.261989in}{1.889144in}}{\pgfqpoint{1.269889in}{1.885872in}}{\pgfqpoint{1.278125in}{1.885872in}}%
\pgfpathclose%
\pgfusepath{stroke,fill}%
\end{pgfscope}%
\begin{pgfscope}%
\pgfpathrectangle{\pgfqpoint{0.100000in}{0.212622in}}{\pgfqpoint{3.696000in}{3.696000in}}%
\pgfusepath{clip}%
\pgfsetbuttcap%
\pgfsetroundjoin%
\definecolor{currentfill}{rgb}{0.121569,0.466667,0.705882}%
\pgfsetfillcolor{currentfill}%
\pgfsetfillopacity{0.782565}%
\pgfsetlinewidth{1.003750pt}%
\definecolor{currentstroke}{rgb}{0.121569,0.466667,0.705882}%
\pgfsetstrokecolor{currentstroke}%
\pgfsetstrokeopacity{0.782565}%
\pgfsetdash{}{0pt}%
\pgfpathmoveto{\pgfqpoint{1.277471in}{1.890431in}}%
\pgfpathcurveto{\pgfqpoint{1.285707in}{1.890431in}}{\pgfqpoint{1.293607in}{1.893704in}}{\pgfqpoint{1.299431in}{1.899528in}}%
\pgfpathcurveto{\pgfqpoint{1.305255in}{1.905351in}}{\pgfqpoint{1.308527in}{1.913252in}}{\pgfqpoint{1.308527in}{1.921488in}}%
\pgfpathcurveto{\pgfqpoint{1.308527in}{1.929724in}}{\pgfqpoint{1.305255in}{1.937624in}}{\pgfqpoint{1.299431in}{1.943448in}}%
\pgfpathcurveto{\pgfqpoint{1.293607in}{1.949272in}}{\pgfqpoint{1.285707in}{1.952544in}}{\pgfqpoint{1.277471in}{1.952544in}}%
\pgfpathcurveto{\pgfqpoint{1.269234in}{1.952544in}}{\pgfqpoint{1.261334in}{1.949272in}}{\pgfqpoint{1.255510in}{1.943448in}}%
\pgfpathcurveto{\pgfqpoint{1.249686in}{1.937624in}}{\pgfqpoint{1.246414in}{1.929724in}}{\pgfqpoint{1.246414in}{1.921488in}}%
\pgfpathcurveto{\pgfqpoint{1.246414in}{1.913252in}}{\pgfqpoint{1.249686in}{1.905351in}}{\pgfqpoint{1.255510in}{1.899528in}}%
\pgfpathcurveto{\pgfqpoint{1.261334in}{1.893704in}}{\pgfqpoint{1.269234in}{1.890431in}}{\pgfqpoint{1.277471in}{1.890431in}}%
\pgfpathclose%
\pgfusepath{stroke,fill}%
\end{pgfscope}%
\begin{pgfscope}%
\pgfpathrectangle{\pgfqpoint{0.100000in}{0.212622in}}{\pgfqpoint{3.696000in}{3.696000in}}%
\pgfusepath{clip}%
\pgfsetbuttcap%
\pgfsetroundjoin%
\definecolor{currentfill}{rgb}{0.121569,0.466667,0.705882}%
\pgfsetfillcolor{currentfill}%
\pgfsetfillopacity{0.792238}%
\pgfsetlinewidth{1.003750pt}%
\definecolor{currentstroke}{rgb}{0.121569,0.466667,0.705882}%
\pgfsetstrokecolor{currentstroke}%
\pgfsetstrokeopacity{0.792238}%
\pgfsetdash{}{0pt}%
\pgfpathmoveto{\pgfqpoint{1.284692in}{1.880617in}}%
\pgfpathcurveto{\pgfqpoint{1.292929in}{1.880617in}}{\pgfqpoint{1.300829in}{1.883889in}}{\pgfqpoint{1.306653in}{1.889713in}}%
\pgfpathcurveto{\pgfqpoint{1.312476in}{1.895537in}}{\pgfqpoint{1.315749in}{1.903437in}}{\pgfqpoint{1.315749in}{1.911674in}}%
\pgfpathcurveto{\pgfqpoint{1.315749in}{1.919910in}}{\pgfqpoint{1.312476in}{1.927810in}}{\pgfqpoint{1.306653in}{1.933634in}}%
\pgfpathcurveto{\pgfqpoint{1.300829in}{1.939458in}}{\pgfqpoint{1.292929in}{1.942730in}}{\pgfqpoint{1.284692in}{1.942730in}}%
\pgfpathcurveto{\pgfqpoint{1.276456in}{1.942730in}}{\pgfqpoint{1.268556in}{1.939458in}}{\pgfqpoint{1.262732in}{1.933634in}}%
\pgfpathcurveto{\pgfqpoint{1.256908in}{1.927810in}}{\pgfqpoint{1.253636in}{1.919910in}}{\pgfqpoint{1.253636in}{1.911674in}}%
\pgfpathcurveto{\pgfqpoint{1.253636in}{1.903437in}}{\pgfqpoint{1.256908in}{1.895537in}}{\pgfqpoint{1.262732in}{1.889713in}}%
\pgfpathcurveto{\pgfqpoint{1.268556in}{1.883889in}}{\pgfqpoint{1.276456in}{1.880617in}}{\pgfqpoint{1.284692in}{1.880617in}}%
\pgfpathclose%
\pgfusepath{stroke,fill}%
\end{pgfscope}%
\begin{pgfscope}%
\pgfpathrectangle{\pgfqpoint{0.100000in}{0.212622in}}{\pgfqpoint{3.696000in}{3.696000in}}%
\pgfusepath{clip}%
\pgfsetbuttcap%
\pgfsetroundjoin%
\definecolor{currentfill}{rgb}{0.121569,0.466667,0.705882}%
\pgfsetfillcolor{currentfill}%
\pgfsetfillopacity{0.795367}%
\pgfsetlinewidth{1.003750pt}%
\definecolor{currentstroke}{rgb}{0.121569,0.466667,0.705882}%
\pgfsetstrokecolor{currentstroke}%
\pgfsetstrokeopacity{0.795367}%
\pgfsetdash{}{0pt}%
\pgfpathmoveto{\pgfqpoint{1.286801in}{1.881943in}}%
\pgfpathcurveto{\pgfqpoint{1.295038in}{1.881943in}}{\pgfqpoint{1.302938in}{1.885215in}}{\pgfqpoint{1.308762in}{1.891039in}}%
\pgfpathcurveto{\pgfqpoint{1.314585in}{1.896863in}}{\pgfqpoint{1.317858in}{1.904763in}}{\pgfqpoint{1.317858in}{1.912999in}}%
\pgfpathcurveto{\pgfqpoint{1.317858in}{1.921236in}}{\pgfqpoint{1.314585in}{1.929136in}}{\pgfqpoint{1.308762in}{1.934960in}}%
\pgfpathcurveto{\pgfqpoint{1.302938in}{1.940784in}}{\pgfqpoint{1.295038in}{1.944056in}}{\pgfqpoint{1.286801in}{1.944056in}}%
\pgfpathcurveto{\pgfqpoint{1.278565in}{1.944056in}}{\pgfqpoint{1.270665in}{1.940784in}}{\pgfqpoint{1.264841in}{1.934960in}}%
\pgfpathcurveto{\pgfqpoint{1.259017in}{1.929136in}}{\pgfqpoint{1.255745in}{1.921236in}}{\pgfqpoint{1.255745in}{1.912999in}}%
\pgfpathcurveto{\pgfqpoint{1.255745in}{1.904763in}}{\pgfqpoint{1.259017in}{1.896863in}}{\pgfqpoint{1.264841in}{1.891039in}}%
\pgfpathcurveto{\pgfqpoint{1.270665in}{1.885215in}}{\pgfqpoint{1.278565in}{1.881943in}}{\pgfqpoint{1.286801in}{1.881943in}}%
\pgfpathclose%
\pgfusepath{stroke,fill}%
\end{pgfscope}%
\begin{pgfscope}%
\pgfpathrectangle{\pgfqpoint{0.100000in}{0.212622in}}{\pgfqpoint{3.696000in}{3.696000in}}%
\pgfusepath{clip}%
\pgfsetbuttcap%
\pgfsetroundjoin%
\definecolor{currentfill}{rgb}{0.121569,0.466667,0.705882}%
\pgfsetfillcolor{currentfill}%
\pgfsetfillopacity{0.796444}%
\pgfsetlinewidth{1.003750pt}%
\definecolor{currentstroke}{rgb}{0.121569,0.466667,0.705882}%
\pgfsetstrokecolor{currentstroke}%
\pgfsetstrokeopacity{0.796444}%
\pgfsetdash{}{0pt}%
\pgfpathmoveto{\pgfqpoint{1.277448in}{1.871130in}}%
\pgfpathcurveto{\pgfqpoint{1.285685in}{1.871130in}}{\pgfqpoint{1.293585in}{1.874402in}}{\pgfqpoint{1.299409in}{1.880226in}}%
\pgfpathcurveto{\pgfqpoint{1.305232in}{1.886050in}}{\pgfqpoint{1.308505in}{1.893950in}}{\pgfqpoint{1.308505in}{1.902186in}}%
\pgfpathcurveto{\pgfqpoint{1.308505in}{1.910422in}}{\pgfqpoint{1.305232in}{1.918322in}}{\pgfqpoint{1.299409in}{1.924146in}}%
\pgfpathcurveto{\pgfqpoint{1.293585in}{1.929970in}}{\pgfqpoint{1.285685in}{1.933243in}}{\pgfqpoint{1.277448in}{1.933243in}}%
\pgfpathcurveto{\pgfqpoint{1.269212in}{1.933243in}}{\pgfqpoint{1.261312in}{1.929970in}}{\pgfqpoint{1.255488in}{1.924146in}}%
\pgfpathcurveto{\pgfqpoint{1.249664in}{1.918322in}}{\pgfqpoint{1.246392in}{1.910422in}}{\pgfqpoint{1.246392in}{1.902186in}}%
\pgfpathcurveto{\pgfqpoint{1.246392in}{1.893950in}}{\pgfqpoint{1.249664in}{1.886050in}}{\pgfqpoint{1.255488in}{1.880226in}}%
\pgfpathcurveto{\pgfqpoint{1.261312in}{1.874402in}}{\pgfqpoint{1.269212in}{1.871130in}}{\pgfqpoint{1.277448in}{1.871130in}}%
\pgfpathclose%
\pgfusepath{stroke,fill}%
\end{pgfscope}%
\begin{pgfscope}%
\pgfpathrectangle{\pgfqpoint{0.100000in}{0.212622in}}{\pgfqpoint{3.696000in}{3.696000in}}%
\pgfusepath{clip}%
\pgfsetbuttcap%
\pgfsetroundjoin%
\definecolor{currentfill}{rgb}{0.121569,0.466667,0.705882}%
\pgfsetfillcolor{currentfill}%
\pgfsetfillopacity{0.797810}%
\pgfsetlinewidth{1.003750pt}%
\definecolor{currentstroke}{rgb}{0.121569,0.466667,0.705882}%
\pgfsetstrokecolor{currentstroke}%
\pgfsetstrokeopacity{0.797810}%
\pgfsetdash{}{0pt}%
\pgfpathmoveto{\pgfqpoint{1.289674in}{1.887032in}}%
\pgfpathcurveto{\pgfqpoint{1.297910in}{1.887032in}}{\pgfqpoint{1.305810in}{1.890305in}}{\pgfqpoint{1.311634in}{1.896129in}}%
\pgfpathcurveto{\pgfqpoint{1.317458in}{1.901953in}}{\pgfqpoint{1.320730in}{1.909853in}}{\pgfqpoint{1.320730in}{1.918089in}}%
\pgfpathcurveto{\pgfqpoint{1.320730in}{1.926325in}}{\pgfqpoint{1.317458in}{1.934225in}}{\pgfqpoint{1.311634in}{1.940049in}}%
\pgfpathcurveto{\pgfqpoint{1.305810in}{1.945873in}}{\pgfqpoint{1.297910in}{1.949145in}}{\pgfqpoint{1.289674in}{1.949145in}}%
\pgfpathcurveto{\pgfqpoint{1.281437in}{1.949145in}}{\pgfqpoint{1.273537in}{1.945873in}}{\pgfqpoint{1.267713in}{1.940049in}}%
\pgfpathcurveto{\pgfqpoint{1.261889in}{1.934225in}}{\pgfqpoint{1.258617in}{1.926325in}}{\pgfqpoint{1.258617in}{1.918089in}}%
\pgfpathcurveto{\pgfqpoint{1.258617in}{1.909853in}}{\pgfqpoint{1.261889in}{1.901953in}}{\pgfqpoint{1.267713in}{1.896129in}}%
\pgfpathcurveto{\pgfqpoint{1.273537in}{1.890305in}}{\pgfqpoint{1.281437in}{1.887032in}}{\pgfqpoint{1.289674in}{1.887032in}}%
\pgfpathclose%
\pgfusepath{stroke,fill}%
\end{pgfscope}%
\begin{pgfscope}%
\pgfpathrectangle{\pgfqpoint{0.100000in}{0.212622in}}{\pgfqpoint{3.696000in}{3.696000in}}%
\pgfusepath{clip}%
\pgfsetbuttcap%
\pgfsetroundjoin%
\definecolor{currentfill}{rgb}{0.121569,0.466667,0.705882}%
\pgfsetfillcolor{currentfill}%
\pgfsetfillopacity{0.800095}%
\pgfsetlinewidth{1.003750pt}%
\definecolor{currentstroke}{rgb}{0.121569,0.466667,0.705882}%
\pgfsetstrokecolor{currentstroke}%
\pgfsetstrokeopacity{0.800095}%
\pgfsetdash{}{0pt}%
\pgfpathmoveto{\pgfqpoint{1.256710in}{1.852384in}}%
\pgfpathcurveto{\pgfqpoint{1.264946in}{1.852384in}}{\pgfqpoint{1.272846in}{1.855657in}}{\pgfqpoint{1.278670in}{1.861481in}}%
\pgfpathcurveto{\pgfqpoint{1.284494in}{1.867305in}}{\pgfqpoint{1.287766in}{1.875205in}}{\pgfqpoint{1.287766in}{1.883441in}}%
\pgfpathcurveto{\pgfqpoint{1.287766in}{1.891677in}}{\pgfqpoint{1.284494in}{1.899577in}}{\pgfqpoint{1.278670in}{1.905401in}}%
\pgfpathcurveto{\pgfqpoint{1.272846in}{1.911225in}}{\pgfqpoint{1.264946in}{1.914497in}}{\pgfqpoint{1.256710in}{1.914497in}}%
\pgfpathcurveto{\pgfqpoint{1.248473in}{1.914497in}}{\pgfqpoint{1.240573in}{1.911225in}}{\pgfqpoint{1.234750in}{1.905401in}}%
\pgfpathcurveto{\pgfqpoint{1.228926in}{1.899577in}}{\pgfqpoint{1.225653in}{1.891677in}}{\pgfqpoint{1.225653in}{1.883441in}}%
\pgfpathcurveto{\pgfqpoint{1.225653in}{1.875205in}}{\pgfqpoint{1.228926in}{1.867305in}}{\pgfqpoint{1.234750in}{1.861481in}}%
\pgfpathcurveto{\pgfqpoint{1.240573in}{1.855657in}}{\pgfqpoint{1.248473in}{1.852384in}}{\pgfqpoint{1.256710in}{1.852384in}}%
\pgfpathclose%
\pgfusepath{stroke,fill}%
\end{pgfscope}%
\begin{pgfscope}%
\pgfpathrectangle{\pgfqpoint{0.100000in}{0.212622in}}{\pgfqpoint{3.696000in}{3.696000in}}%
\pgfusepath{clip}%
\pgfsetbuttcap%
\pgfsetroundjoin%
\definecolor{currentfill}{rgb}{0.121569,0.466667,0.705882}%
\pgfsetfillcolor{currentfill}%
\pgfsetfillopacity{0.801566}%
\pgfsetlinewidth{1.003750pt}%
\definecolor{currentstroke}{rgb}{0.121569,0.466667,0.705882}%
\pgfsetstrokecolor{currentstroke}%
\pgfsetstrokeopacity{0.801566}%
\pgfsetdash{}{0pt}%
\pgfpathmoveto{\pgfqpoint{1.247208in}{1.854226in}}%
\pgfpathcurveto{\pgfqpoint{1.255445in}{1.854226in}}{\pgfqpoint{1.263345in}{1.857499in}}{\pgfqpoint{1.269169in}{1.863323in}}%
\pgfpathcurveto{\pgfqpoint{1.274993in}{1.869147in}}{\pgfqpoint{1.278265in}{1.877047in}}{\pgfqpoint{1.278265in}{1.885283in}}%
\pgfpathcurveto{\pgfqpoint{1.278265in}{1.893519in}}{\pgfqpoint{1.274993in}{1.901419in}}{\pgfqpoint{1.269169in}{1.907243in}}%
\pgfpathcurveto{\pgfqpoint{1.263345in}{1.913067in}}{\pgfqpoint{1.255445in}{1.916339in}}{\pgfqpoint{1.247208in}{1.916339in}}%
\pgfpathcurveto{\pgfqpoint{1.238972in}{1.916339in}}{\pgfqpoint{1.231072in}{1.913067in}}{\pgfqpoint{1.225248in}{1.907243in}}%
\pgfpathcurveto{\pgfqpoint{1.219424in}{1.901419in}}{\pgfqpoint{1.216152in}{1.893519in}}{\pgfqpoint{1.216152in}{1.885283in}}%
\pgfpathcurveto{\pgfqpoint{1.216152in}{1.877047in}}{\pgfqpoint{1.219424in}{1.869147in}}{\pgfqpoint{1.225248in}{1.863323in}}%
\pgfpathcurveto{\pgfqpoint{1.231072in}{1.857499in}}{\pgfqpoint{1.238972in}{1.854226in}}{\pgfqpoint{1.247208in}{1.854226in}}%
\pgfpathclose%
\pgfusepath{stroke,fill}%
\end{pgfscope}%
\begin{pgfscope}%
\pgfpathrectangle{\pgfqpoint{0.100000in}{0.212622in}}{\pgfqpoint{3.696000in}{3.696000in}}%
\pgfusepath{clip}%
\pgfsetbuttcap%
\pgfsetroundjoin%
\definecolor{currentfill}{rgb}{0.121569,0.466667,0.705882}%
\pgfsetfillcolor{currentfill}%
\pgfsetfillopacity{0.801651}%
\pgfsetlinewidth{1.003750pt}%
\definecolor{currentstroke}{rgb}{0.121569,0.466667,0.705882}%
\pgfsetstrokecolor{currentstroke}%
\pgfsetstrokeopacity{0.801651}%
\pgfsetdash{}{0pt}%
\pgfpathmoveto{\pgfqpoint{1.288158in}{1.882870in}}%
\pgfpathcurveto{\pgfqpoint{1.296394in}{1.882870in}}{\pgfqpoint{1.304294in}{1.886142in}}{\pgfqpoint{1.310118in}{1.891966in}}%
\pgfpathcurveto{\pgfqpoint{1.315942in}{1.897790in}}{\pgfqpoint{1.319215in}{1.905690in}}{\pgfqpoint{1.319215in}{1.913927in}}%
\pgfpathcurveto{\pgfqpoint{1.319215in}{1.922163in}}{\pgfqpoint{1.315942in}{1.930063in}}{\pgfqpoint{1.310118in}{1.935887in}}%
\pgfpathcurveto{\pgfqpoint{1.304294in}{1.941711in}}{\pgfqpoint{1.296394in}{1.944983in}}{\pgfqpoint{1.288158in}{1.944983in}}%
\pgfpathcurveto{\pgfqpoint{1.279922in}{1.944983in}}{\pgfqpoint{1.272022in}{1.941711in}}{\pgfqpoint{1.266198in}{1.935887in}}%
\pgfpathcurveto{\pgfqpoint{1.260374in}{1.930063in}}{\pgfqpoint{1.257102in}{1.922163in}}{\pgfqpoint{1.257102in}{1.913927in}}%
\pgfpathcurveto{\pgfqpoint{1.257102in}{1.905690in}}{\pgfqpoint{1.260374in}{1.897790in}}{\pgfqpoint{1.266198in}{1.891966in}}%
\pgfpathcurveto{\pgfqpoint{1.272022in}{1.886142in}}{\pgfqpoint{1.279922in}{1.882870in}}{\pgfqpoint{1.288158in}{1.882870in}}%
\pgfpathclose%
\pgfusepath{stroke,fill}%
\end{pgfscope}%
\begin{pgfscope}%
\pgfpathrectangle{\pgfqpoint{0.100000in}{0.212622in}}{\pgfqpoint{3.696000in}{3.696000in}}%
\pgfusepath{clip}%
\pgfsetbuttcap%
\pgfsetroundjoin%
\definecolor{currentfill}{rgb}{0.121569,0.466667,0.705882}%
\pgfsetfillcolor{currentfill}%
\pgfsetfillopacity{0.802177}%
\pgfsetlinewidth{1.003750pt}%
\definecolor{currentstroke}{rgb}{0.121569,0.466667,0.705882}%
\pgfsetstrokecolor{currentstroke}%
\pgfsetstrokeopacity{0.802177}%
\pgfsetdash{}{0pt}%
\pgfpathmoveto{\pgfqpoint{1.247402in}{1.848736in}}%
\pgfpathcurveto{\pgfqpoint{1.255639in}{1.848736in}}{\pgfqpoint{1.263539in}{1.852008in}}{\pgfqpoint{1.269362in}{1.857832in}}%
\pgfpathcurveto{\pgfqpoint{1.275186in}{1.863656in}}{\pgfqpoint{1.278459in}{1.871556in}}{\pgfqpoint{1.278459in}{1.879792in}}%
\pgfpathcurveto{\pgfqpoint{1.278459in}{1.888028in}}{\pgfqpoint{1.275186in}{1.895929in}}{\pgfqpoint{1.269362in}{1.901752in}}%
\pgfpathcurveto{\pgfqpoint{1.263539in}{1.907576in}}{\pgfqpoint{1.255639in}{1.910849in}}{\pgfqpoint{1.247402in}{1.910849in}}%
\pgfpathcurveto{\pgfqpoint{1.239166in}{1.910849in}}{\pgfqpoint{1.231266in}{1.907576in}}{\pgfqpoint{1.225442in}{1.901752in}}%
\pgfpathcurveto{\pgfqpoint{1.219618in}{1.895929in}}{\pgfqpoint{1.216346in}{1.888028in}}{\pgfqpoint{1.216346in}{1.879792in}}%
\pgfpathcurveto{\pgfqpoint{1.216346in}{1.871556in}}{\pgfqpoint{1.219618in}{1.863656in}}{\pgfqpoint{1.225442in}{1.857832in}}%
\pgfpathcurveto{\pgfqpoint{1.231266in}{1.852008in}}{\pgfqpoint{1.239166in}{1.848736in}}{\pgfqpoint{1.247402in}{1.848736in}}%
\pgfpathclose%
\pgfusepath{stroke,fill}%
\end{pgfscope}%
\begin{pgfscope}%
\pgfpathrectangle{\pgfqpoint{0.100000in}{0.212622in}}{\pgfqpoint{3.696000in}{3.696000in}}%
\pgfusepath{clip}%
\pgfsetbuttcap%
\pgfsetroundjoin%
\definecolor{currentfill}{rgb}{0.121569,0.466667,0.705882}%
\pgfsetfillcolor{currentfill}%
\pgfsetfillopacity{0.803635}%
\pgfsetlinewidth{1.003750pt}%
\definecolor{currentstroke}{rgb}{0.121569,0.466667,0.705882}%
\pgfsetstrokecolor{currentstroke}%
\pgfsetstrokeopacity{0.803635}%
\pgfsetdash{}{0pt}%
\pgfpathmoveto{\pgfqpoint{1.286987in}{1.878593in}}%
\pgfpathcurveto{\pgfqpoint{1.295223in}{1.878593in}}{\pgfqpoint{1.303123in}{1.881865in}}{\pgfqpoint{1.308947in}{1.887689in}}%
\pgfpathcurveto{\pgfqpoint{1.314771in}{1.893513in}}{\pgfqpoint{1.318043in}{1.901413in}}{\pgfqpoint{1.318043in}{1.909649in}}%
\pgfpathcurveto{\pgfqpoint{1.318043in}{1.917885in}}{\pgfqpoint{1.314771in}{1.925785in}}{\pgfqpoint{1.308947in}{1.931609in}}%
\pgfpathcurveto{\pgfqpoint{1.303123in}{1.937433in}}{\pgfqpoint{1.295223in}{1.940706in}}{\pgfqpoint{1.286987in}{1.940706in}}%
\pgfpathcurveto{\pgfqpoint{1.278750in}{1.940706in}}{\pgfqpoint{1.270850in}{1.937433in}}{\pgfqpoint{1.265026in}{1.931609in}}%
\pgfpathcurveto{\pgfqpoint{1.259202in}{1.925785in}}{\pgfqpoint{1.255930in}{1.917885in}}{\pgfqpoint{1.255930in}{1.909649in}}%
\pgfpathcurveto{\pgfqpoint{1.255930in}{1.901413in}}{\pgfqpoint{1.259202in}{1.893513in}}{\pgfqpoint{1.265026in}{1.887689in}}%
\pgfpathcurveto{\pgfqpoint{1.270850in}{1.881865in}}{\pgfqpoint{1.278750in}{1.878593in}}{\pgfqpoint{1.286987in}{1.878593in}}%
\pgfpathclose%
\pgfusepath{stroke,fill}%
\end{pgfscope}%
\begin{pgfscope}%
\pgfpathrectangle{\pgfqpoint{0.100000in}{0.212622in}}{\pgfqpoint{3.696000in}{3.696000in}}%
\pgfusepath{clip}%
\pgfsetbuttcap%
\pgfsetroundjoin%
\definecolor{currentfill}{rgb}{0.121569,0.466667,0.705882}%
\pgfsetfillcolor{currentfill}%
\pgfsetfillopacity{0.804393}%
\pgfsetlinewidth{1.003750pt}%
\definecolor{currentstroke}{rgb}{0.121569,0.466667,0.705882}%
\pgfsetstrokecolor{currentstroke}%
\pgfsetstrokeopacity{0.804393}%
\pgfsetdash{}{0pt}%
\pgfpathmoveto{\pgfqpoint{1.287948in}{1.879713in}}%
\pgfpathcurveto{\pgfqpoint{1.296184in}{1.879713in}}{\pgfqpoint{1.304084in}{1.882986in}}{\pgfqpoint{1.309908in}{1.888810in}}%
\pgfpathcurveto{\pgfqpoint{1.315732in}{1.894633in}}{\pgfqpoint{1.319004in}{1.902534in}}{\pgfqpoint{1.319004in}{1.910770in}}%
\pgfpathcurveto{\pgfqpoint{1.319004in}{1.919006in}}{\pgfqpoint{1.315732in}{1.926906in}}{\pgfqpoint{1.309908in}{1.932730in}}%
\pgfpathcurveto{\pgfqpoint{1.304084in}{1.938554in}}{\pgfqpoint{1.296184in}{1.941826in}}{\pgfqpoint{1.287948in}{1.941826in}}%
\pgfpathcurveto{\pgfqpoint{1.279712in}{1.941826in}}{\pgfqpoint{1.271811in}{1.938554in}}{\pgfqpoint{1.265988in}{1.932730in}}%
\pgfpathcurveto{\pgfqpoint{1.260164in}{1.926906in}}{\pgfqpoint{1.256891in}{1.919006in}}{\pgfqpoint{1.256891in}{1.910770in}}%
\pgfpathcurveto{\pgfqpoint{1.256891in}{1.902534in}}{\pgfqpoint{1.260164in}{1.894633in}}{\pgfqpoint{1.265988in}{1.888810in}}%
\pgfpathcurveto{\pgfqpoint{1.271811in}{1.882986in}}{\pgfqpoint{1.279712in}{1.879713in}}{\pgfqpoint{1.287948in}{1.879713in}}%
\pgfpathclose%
\pgfusepath{stroke,fill}%
\end{pgfscope}%
\begin{pgfscope}%
\pgfpathrectangle{\pgfqpoint{0.100000in}{0.212622in}}{\pgfqpoint{3.696000in}{3.696000in}}%
\pgfusepath{clip}%
\pgfsetbuttcap%
\pgfsetroundjoin%
\definecolor{currentfill}{rgb}{0.121569,0.466667,0.705882}%
\pgfsetfillcolor{currentfill}%
\pgfsetfillopacity{0.804839}%
\pgfsetlinewidth{1.003750pt}%
\definecolor{currentstroke}{rgb}{0.121569,0.466667,0.705882}%
\pgfsetstrokecolor{currentstroke}%
\pgfsetstrokeopacity{0.804839}%
\pgfsetdash{}{0pt}%
\pgfpathmoveto{\pgfqpoint{1.232439in}{1.841745in}}%
\pgfpathcurveto{\pgfqpoint{1.240675in}{1.841745in}}{\pgfqpoint{1.248575in}{1.845017in}}{\pgfqpoint{1.254399in}{1.850841in}}%
\pgfpathcurveto{\pgfqpoint{1.260223in}{1.856665in}}{\pgfqpoint{1.263495in}{1.864565in}}{\pgfqpoint{1.263495in}{1.872801in}}%
\pgfpathcurveto{\pgfqpoint{1.263495in}{1.881038in}}{\pgfqpoint{1.260223in}{1.888938in}}{\pgfqpoint{1.254399in}{1.894762in}}%
\pgfpathcurveto{\pgfqpoint{1.248575in}{1.900586in}}{\pgfqpoint{1.240675in}{1.903858in}}{\pgfqpoint{1.232439in}{1.903858in}}%
\pgfpathcurveto{\pgfqpoint{1.224202in}{1.903858in}}{\pgfqpoint{1.216302in}{1.900586in}}{\pgfqpoint{1.210478in}{1.894762in}}%
\pgfpathcurveto{\pgfqpoint{1.204655in}{1.888938in}}{\pgfqpoint{1.201382in}{1.881038in}}{\pgfqpoint{1.201382in}{1.872801in}}%
\pgfpathcurveto{\pgfqpoint{1.201382in}{1.864565in}}{\pgfqpoint{1.204655in}{1.856665in}}{\pgfqpoint{1.210478in}{1.850841in}}%
\pgfpathcurveto{\pgfqpoint{1.216302in}{1.845017in}}{\pgfqpoint{1.224202in}{1.841745in}}{\pgfqpoint{1.232439in}{1.841745in}}%
\pgfpathclose%
\pgfusepath{stroke,fill}%
\end{pgfscope}%
\begin{pgfscope}%
\pgfpathrectangle{\pgfqpoint{0.100000in}{0.212622in}}{\pgfqpoint{3.696000in}{3.696000in}}%
\pgfusepath{clip}%
\pgfsetbuttcap%
\pgfsetroundjoin%
\definecolor{currentfill}{rgb}{0.121569,0.466667,0.705882}%
\pgfsetfillcolor{currentfill}%
\pgfsetfillopacity{0.808319}%
\pgfsetlinewidth{1.003750pt}%
\definecolor{currentstroke}{rgb}{0.121569,0.466667,0.705882}%
\pgfsetstrokecolor{currentstroke}%
\pgfsetstrokeopacity{0.808319}%
\pgfsetdash{}{0pt}%
\pgfpathmoveto{\pgfqpoint{1.284700in}{1.875240in}}%
\pgfpathcurveto{\pgfqpoint{1.292937in}{1.875240in}}{\pgfqpoint{1.300837in}{1.878512in}}{\pgfqpoint{1.306661in}{1.884336in}}%
\pgfpathcurveto{\pgfqpoint{1.312485in}{1.890160in}}{\pgfqpoint{1.315757in}{1.898060in}}{\pgfqpoint{1.315757in}{1.906297in}}%
\pgfpathcurveto{\pgfqpoint{1.315757in}{1.914533in}}{\pgfqpoint{1.312485in}{1.922433in}}{\pgfqpoint{1.306661in}{1.928257in}}%
\pgfpathcurveto{\pgfqpoint{1.300837in}{1.934081in}}{\pgfqpoint{1.292937in}{1.937353in}}{\pgfqpoint{1.284700in}{1.937353in}}%
\pgfpathcurveto{\pgfqpoint{1.276464in}{1.937353in}}{\pgfqpoint{1.268564in}{1.934081in}}{\pgfqpoint{1.262740in}{1.928257in}}%
\pgfpathcurveto{\pgfqpoint{1.256916in}{1.922433in}}{\pgfqpoint{1.253644in}{1.914533in}}{\pgfqpoint{1.253644in}{1.906297in}}%
\pgfpathcurveto{\pgfqpoint{1.253644in}{1.898060in}}{\pgfqpoint{1.256916in}{1.890160in}}{\pgfqpoint{1.262740in}{1.884336in}}%
\pgfpathcurveto{\pgfqpoint{1.268564in}{1.878512in}}{\pgfqpoint{1.276464in}{1.875240in}}{\pgfqpoint{1.284700in}{1.875240in}}%
\pgfpathclose%
\pgfusepath{stroke,fill}%
\end{pgfscope}%
\begin{pgfscope}%
\pgfpathrectangle{\pgfqpoint{0.100000in}{0.212622in}}{\pgfqpoint{3.696000in}{3.696000in}}%
\pgfusepath{clip}%
\pgfsetbuttcap%
\pgfsetroundjoin%
\definecolor{currentfill}{rgb}{0.121569,0.466667,0.705882}%
\pgfsetfillcolor{currentfill}%
\pgfsetfillopacity{0.809922}%
\pgfsetlinewidth{1.003750pt}%
\definecolor{currentstroke}{rgb}{0.121569,0.466667,0.705882}%
\pgfsetstrokecolor{currentstroke}%
\pgfsetstrokeopacity{0.809922}%
\pgfsetdash{}{0pt}%
\pgfpathmoveto{\pgfqpoint{3.033755in}{2.673330in}}%
\pgfpathcurveto{\pgfqpoint{3.041992in}{2.673330in}}{\pgfqpoint{3.049892in}{2.676602in}}{\pgfqpoint{3.055716in}{2.682426in}}%
\pgfpathcurveto{\pgfqpoint{3.061539in}{2.688250in}}{\pgfqpoint{3.064812in}{2.696150in}}{\pgfqpoint{3.064812in}{2.704386in}}%
\pgfpathcurveto{\pgfqpoint{3.064812in}{2.712623in}}{\pgfqpoint{3.061539in}{2.720523in}}{\pgfqpoint{3.055716in}{2.726347in}}%
\pgfpathcurveto{\pgfqpoint{3.049892in}{2.732170in}}{\pgfqpoint{3.041992in}{2.735443in}}{\pgfqpoint{3.033755in}{2.735443in}}%
\pgfpathcurveto{\pgfqpoint{3.025519in}{2.735443in}}{\pgfqpoint{3.017619in}{2.732170in}}{\pgfqpoint{3.011795in}{2.726347in}}%
\pgfpathcurveto{\pgfqpoint{3.005971in}{2.720523in}}{\pgfqpoint{3.002699in}{2.712623in}}{\pgfqpoint{3.002699in}{2.704386in}}%
\pgfpathcurveto{\pgfqpoint{3.002699in}{2.696150in}}{\pgfqpoint{3.005971in}{2.688250in}}{\pgfqpoint{3.011795in}{2.682426in}}%
\pgfpathcurveto{\pgfqpoint{3.017619in}{2.676602in}}{\pgfqpoint{3.025519in}{2.673330in}}{\pgfqpoint{3.033755in}{2.673330in}}%
\pgfpathclose%
\pgfusepath{stroke,fill}%
\end{pgfscope}%
\begin{pgfscope}%
\pgfpathrectangle{\pgfqpoint{0.100000in}{0.212622in}}{\pgfqpoint{3.696000in}{3.696000in}}%
\pgfusepath{clip}%
\pgfsetbuttcap%
\pgfsetroundjoin%
\definecolor{currentfill}{rgb}{0.121569,0.466667,0.705882}%
\pgfsetfillcolor{currentfill}%
\pgfsetfillopacity{0.812995}%
\pgfsetlinewidth{1.003750pt}%
\definecolor{currentstroke}{rgb}{0.121569,0.466667,0.705882}%
\pgfsetstrokecolor{currentstroke}%
\pgfsetstrokeopacity{0.812995}%
\pgfsetdash{}{0pt}%
\pgfpathmoveto{\pgfqpoint{1.281712in}{1.873513in}}%
\pgfpathcurveto{\pgfqpoint{1.289948in}{1.873513in}}{\pgfqpoint{1.297848in}{1.876785in}}{\pgfqpoint{1.303672in}{1.882609in}}%
\pgfpathcurveto{\pgfqpoint{1.309496in}{1.888433in}}{\pgfqpoint{1.312768in}{1.896333in}}{\pgfqpoint{1.312768in}{1.904569in}}%
\pgfpathcurveto{\pgfqpoint{1.312768in}{1.912806in}}{\pgfqpoint{1.309496in}{1.920706in}}{\pgfqpoint{1.303672in}{1.926530in}}%
\pgfpathcurveto{\pgfqpoint{1.297848in}{1.932354in}}{\pgfqpoint{1.289948in}{1.935626in}}{\pgfqpoint{1.281712in}{1.935626in}}%
\pgfpathcurveto{\pgfqpoint{1.273476in}{1.935626in}}{\pgfqpoint{1.265576in}{1.932354in}}{\pgfqpoint{1.259752in}{1.926530in}}%
\pgfpathcurveto{\pgfqpoint{1.253928in}{1.920706in}}{\pgfqpoint{1.250655in}{1.912806in}}{\pgfqpoint{1.250655in}{1.904569in}}%
\pgfpathcurveto{\pgfqpoint{1.250655in}{1.896333in}}{\pgfqpoint{1.253928in}{1.888433in}}{\pgfqpoint{1.259752in}{1.882609in}}%
\pgfpathcurveto{\pgfqpoint{1.265576in}{1.876785in}}{\pgfqpoint{1.273476in}{1.873513in}}{\pgfqpoint{1.281712in}{1.873513in}}%
\pgfpathclose%
\pgfusepath{stroke,fill}%
\end{pgfscope}%
\begin{pgfscope}%
\pgfpathrectangle{\pgfqpoint{0.100000in}{0.212622in}}{\pgfqpoint{3.696000in}{3.696000in}}%
\pgfusepath{clip}%
\pgfsetbuttcap%
\pgfsetroundjoin%
\definecolor{currentfill}{rgb}{0.121569,0.466667,0.705882}%
\pgfsetfillcolor{currentfill}%
\pgfsetfillopacity{0.816774}%
\pgfsetlinewidth{1.003750pt}%
\definecolor{currentstroke}{rgb}{0.121569,0.466667,0.705882}%
\pgfsetstrokecolor{currentstroke}%
\pgfsetstrokeopacity{0.816774}%
\pgfsetdash{}{0pt}%
\pgfpathmoveto{\pgfqpoint{1.276387in}{1.867354in}}%
\pgfpathcurveto{\pgfqpoint{1.284623in}{1.867354in}}{\pgfqpoint{1.292523in}{1.870626in}}{\pgfqpoint{1.298347in}{1.876450in}}%
\pgfpathcurveto{\pgfqpoint{1.304171in}{1.882274in}}{\pgfqpoint{1.307443in}{1.890174in}}{\pgfqpoint{1.307443in}{1.898410in}}%
\pgfpathcurveto{\pgfqpoint{1.307443in}{1.906647in}}{\pgfqpoint{1.304171in}{1.914547in}}{\pgfqpoint{1.298347in}{1.920371in}}%
\pgfpathcurveto{\pgfqpoint{1.292523in}{1.926195in}}{\pgfqpoint{1.284623in}{1.929467in}}{\pgfqpoint{1.276387in}{1.929467in}}%
\pgfpathcurveto{\pgfqpoint{1.268150in}{1.929467in}}{\pgfqpoint{1.260250in}{1.926195in}}{\pgfqpoint{1.254426in}{1.920371in}}%
\pgfpathcurveto{\pgfqpoint{1.248603in}{1.914547in}}{\pgfqpoint{1.245330in}{1.906647in}}{\pgfqpoint{1.245330in}{1.898410in}}%
\pgfpathcurveto{\pgfqpoint{1.245330in}{1.890174in}}{\pgfqpoint{1.248603in}{1.882274in}}{\pgfqpoint{1.254426in}{1.876450in}}%
\pgfpathcurveto{\pgfqpoint{1.260250in}{1.870626in}}{\pgfqpoint{1.268150in}{1.867354in}}{\pgfqpoint{1.276387in}{1.867354in}}%
\pgfpathclose%
\pgfusepath{stroke,fill}%
\end{pgfscope}%
\begin{pgfscope}%
\pgfpathrectangle{\pgfqpoint{0.100000in}{0.212622in}}{\pgfqpoint{3.696000in}{3.696000in}}%
\pgfusepath{clip}%
\pgfsetbuttcap%
\pgfsetroundjoin%
\definecolor{currentfill}{rgb}{0.121569,0.466667,0.705882}%
\pgfsetfillcolor{currentfill}%
\pgfsetfillopacity{0.821908}%
\pgfsetlinewidth{1.003750pt}%
\definecolor{currentstroke}{rgb}{0.121569,0.466667,0.705882}%
\pgfsetstrokecolor{currentstroke}%
\pgfsetstrokeopacity{0.821908}%
\pgfsetdash{}{0pt}%
\pgfpathmoveto{\pgfqpoint{1.268642in}{1.858704in}}%
\pgfpathcurveto{\pgfqpoint{1.276878in}{1.858704in}}{\pgfqpoint{1.284778in}{1.861976in}}{\pgfqpoint{1.290602in}{1.867800in}}%
\pgfpathcurveto{\pgfqpoint{1.296426in}{1.873624in}}{\pgfqpoint{1.299698in}{1.881524in}}{\pgfqpoint{1.299698in}{1.889760in}}%
\pgfpathcurveto{\pgfqpoint{1.299698in}{1.897996in}}{\pgfqpoint{1.296426in}{1.905896in}}{\pgfqpoint{1.290602in}{1.911720in}}%
\pgfpathcurveto{\pgfqpoint{1.284778in}{1.917544in}}{\pgfqpoint{1.276878in}{1.920817in}}{\pgfqpoint{1.268642in}{1.920817in}}%
\pgfpathcurveto{\pgfqpoint{1.260406in}{1.920817in}}{\pgfqpoint{1.252506in}{1.917544in}}{\pgfqpoint{1.246682in}{1.911720in}}%
\pgfpathcurveto{\pgfqpoint{1.240858in}{1.905896in}}{\pgfqpoint{1.237585in}{1.897996in}}{\pgfqpoint{1.237585in}{1.889760in}}%
\pgfpathcurveto{\pgfqpoint{1.237585in}{1.881524in}}{\pgfqpoint{1.240858in}{1.873624in}}{\pgfqpoint{1.246682in}{1.867800in}}%
\pgfpathcurveto{\pgfqpoint{1.252506in}{1.861976in}}{\pgfqpoint{1.260406in}{1.858704in}}{\pgfqpoint{1.268642in}{1.858704in}}%
\pgfpathclose%
\pgfusepath{stroke,fill}%
\end{pgfscope}%
\begin{pgfscope}%
\pgfpathrectangle{\pgfqpoint{0.100000in}{0.212622in}}{\pgfqpoint{3.696000in}{3.696000in}}%
\pgfusepath{clip}%
\pgfsetbuttcap%
\pgfsetroundjoin%
\definecolor{currentfill}{rgb}{0.121569,0.466667,0.705882}%
\pgfsetfillcolor{currentfill}%
\pgfsetfillopacity{0.822750}%
\pgfsetlinewidth{1.003750pt}%
\definecolor{currentstroke}{rgb}{0.121569,0.466667,0.705882}%
\pgfsetstrokecolor{currentstroke}%
\pgfsetstrokeopacity{0.822750}%
\pgfsetdash{}{0pt}%
\pgfpathmoveto{\pgfqpoint{2.988738in}{2.636693in}}%
\pgfpathcurveto{\pgfqpoint{2.996975in}{2.636693in}}{\pgfqpoint{3.004875in}{2.639965in}}{\pgfqpoint{3.010699in}{2.645789in}}%
\pgfpathcurveto{\pgfqpoint{3.016523in}{2.651613in}}{\pgfqpoint{3.019795in}{2.659513in}}{\pgfqpoint{3.019795in}{2.667749in}}%
\pgfpathcurveto{\pgfqpoint{3.019795in}{2.675985in}}{\pgfqpoint{3.016523in}{2.683886in}}{\pgfqpoint{3.010699in}{2.689709in}}%
\pgfpathcurveto{\pgfqpoint{3.004875in}{2.695533in}}{\pgfqpoint{2.996975in}{2.698806in}}{\pgfqpoint{2.988738in}{2.698806in}}%
\pgfpathcurveto{\pgfqpoint{2.980502in}{2.698806in}}{\pgfqpoint{2.972602in}{2.695533in}}{\pgfqpoint{2.966778in}{2.689709in}}%
\pgfpathcurveto{\pgfqpoint{2.960954in}{2.683886in}}{\pgfqpoint{2.957682in}{2.675985in}}{\pgfqpoint{2.957682in}{2.667749in}}%
\pgfpathcurveto{\pgfqpoint{2.957682in}{2.659513in}}{\pgfqpoint{2.960954in}{2.651613in}}{\pgfqpoint{2.966778in}{2.645789in}}%
\pgfpathcurveto{\pgfqpoint{2.972602in}{2.639965in}}{\pgfqpoint{2.980502in}{2.636693in}}{\pgfqpoint{2.988738in}{2.636693in}}%
\pgfpathclose%
\pgfusepath{stroke,fill}%
\end{pgfscope}%
\begin{pgfscope}%
\pgfpathrectangle{\pgfqpoint{0.100000in}{0.212622in}}{\pgfqpoint{3.696000in}{3.696000in}}%
\pgfusepath{clip}%
\pgfsetbuttcap%
\pgfsetroundjoin%
\definecolor{currentfill}{rgb}{0.121569,0.466667,0.705882}%
\pgfsetfillcolor{currentfill}%
\pgfsetfillopacity{0.826018}%
\pgfsetlinewidth{1.003750pt}%
\definecolor{currentstroke}{rgb}{0.121569,0.466667,0.705882}%
\pgfsetstrokecolor{currentstroke}%
\pgfsetstrokeopacity{0.826018}%
\pgfsetdash{}{0pt}%
\pgfpathmoveto{\pgfqpoint{1.270357in}{1.869936in}}%
\pgfpathcurveto{\pgfqpoint{1.278593in}{1.869936in}}{\pgfqpoint{1.286493in}{1.873209in}}{\pgfqpoint{1.292317in}{1.879032in}}%
\pgfpathcurveto{\pgfqpoint{1.298141in}{1.884856in}}{\pgfqpoint{1.301414in}{1.892756in}}{\pgfqpoint{1.301414in}{1.900993in}}%
\pgfpathcurveto{\pgfqpoint{1.301414in}{1.909229in}}{\pgfqpoint{1.298141in}{1.917129in}}{\pgfqpoint{1.292317in}{1.922953in}}%
\pgfpathcurveto{\pgfqpoint{1.286493in}{1.928777in}}{\pgfqpoint{1.278593in}{1.932049in}}{\pgfqpoint{1.270357in}{1.932049in}}%
\pgfpathcurveto{\pgfqpoint{1.262121in}{1.932049in}}{\pgfqpoint{1.254221in}{1.928777in}}{\pgfqpoint{1.248397in}{1.922953in}}%
\pgfpathcurveto{\pgfqpoint{1.242573in}{1.917129in}}{\pgfqpoint{1.239301in}{1.909229in}}{\pgfqpoint{1.239301in}{1.900993in}}%
\pgfpathcurveto{\pgfqpoint{1.239301in}{1.892756in}}{\pgfqpoint{1.242573in}{1.884856in}}{\pgfqpoint{1.248397in}{1.879032in}}%
\pgfpathcurveto{\pgfqpoint{1.254221in}{1.873209in}}{\pgfqpoint{1.262121in}{1.869936in}}{\pgfqpoint{1.270357in}{1.869936in}}%
\pgfpathclose%
\pgfusepath{stroke,fill}%
\end{pgfscope}%
\begin{pgfscope}%
\pgfpathrectangle{\pgfqpoint{0.100000in}{0.212622in}}{\pgfqpoint{3.696000in}{3.696000in}}%
\pgfusepath{clip}%
\pgfsetbuttcap%
\pgfsetroundjoin%
\definecolor{currentfill}{rgb}{0.121569,0.466667,0.705882}%
\pgfsetfillcolor{currentfill}%
\pgfsetfillopacity{0.827985}%
\pgfsetlinewidth{1.003750pt}%
\definecolor{currentstroke}{rgb}{0.121569,0.466667,0.705882}%
\pgfsetstrokecolor{currentstroke}%
\pgfsetstrokeopacity{0.827985}%
\pgfsetdash{}{0pt}%
\pgfpathmoveto{\pgfqpoint{2.886731in}{2.612984in}}%
\pgfpathcurveto{\pgfqpoint{2.894968in}{2.612984in}}{\pgfqpoint{2.902868in}{2.616257in}}{\pgfqpoint{2.908692in}{2.622080in}}%
\pgfpathcurveto{\pgfqpoint{2.914516in}{2.627904in}}{\pgfqpoint{2.917788in}{2.635804in}}{\pgfqpoint{2.917788in}{2.644041in}}%
\pgfpathcurveto{\pgfqpoint{2.917788in}{2.652277in}}{\pgfqpoint{2.914516in}{2.660177in}}{\pgfqpoint{2.908692in}{2.666001in}}%
\pgfpathcurveto{\pgfqpoint{2.902868in}{2.671825in}}{\pgfqpoint{2.894968in}{2.675097in}}{\pgfqpoint{2.886731in}{2.675097in}}%
\pgfpathcurveto{\pgfqpoint{2.878495in}{2.675097in}}{\pgfqpoint{2.870595in}{2.671825in}}{\pgfqpoint{2.864771in}{2.666001in}}%
\pgfpathcurveto{\pgfqpoint{2.858947in}{2.660177in}}{\pgfqpoint{2.855675in}{2.652277in}}{\pgfqpoint{2.855675in}{2.644041in}}%
\pgfpathcurveto{\pgfqpoint{2.855675in}{2.635804in}}{\pgfqpoint{2.858947in}{2.627904in}}{\pgfqpoint{2.864771in}{2.622080in}}%
\pgfpathcurveto{\pgfqpoint{2.870595in}{2.616257in}}{\pgfqpoint{2.878495in}{2.612984in}}{\pgfqpoint{2.886731in}{2.612984in}}%
\pgfpathclose%
\pgfusepath{stroke,fill}%
\end{pgfscope}%
\begin{pgfscope}%
\pgfpathrectangle{\pgfqpoint{0.100000in}{0.212622in}}{\pgfqpoint{3.696000in}{3.696000in}}%
\pgfusepath{clip}%
\pgfsetbuttcap%
\pgfsetroundjoin%
\definecolor{currentfill}{rgb}{0.121569,0.466667,0.705882}%
\pgfsetfillcolor{currentfill}%
\pgfsetfillopacity{0.831487}%
\pgfsetlinewidth{1.003750pt}%
\definecolor{currentstroke}{rgb}{0.121569,0.466667,0.705882}%
\pgfsetstrokecolor{currentstroke}%
\pgfsetstrokeopacity{0.831487}%
\pgfsetdash{}{0pt}%
\pgfpathmoveto{\pgfqpoint{2.906230in}{2.595345in}}%
\pgfpathcurveto{\pgfqpoint{2.914467in}{2.595345in}}{\pgfqpoint{2.922367in}{2.598617in}}{\pgfqpoint{2.928191in}{2.604441in}}%
\pgfpathcurveto{\pgfqpoint{2.934015in}{2.610265in}}{\pgfqpoint{2.937287in}{2.618165in}}{\pgfqpoint{2.937287in}{2.626401in}}%
\pgfpathcurveto{\pgfqpoint{2.937287in}{2.634637in}}{\pgfqpoint{2.934015in}{2.642538in}}{\pgfqpoint{2.928191in}{2.648361in}}%
\pgfpathcurveto{\pgfqpoint{2.922367in}{2.654185in}}{\pgfqpoint{2.914467in}{2.657458in}}{\pgfqpoint{2.906230in}{2.657458in}}%
\pgfpathcurveto{\pgfqpoint{2.897994in}{2.657458in}}{\pgfqpoint{2.890094in}{2.654185in}}{\pgfqpoint{2.884270in}{2.648361in}}%
\pgfpathcurveto{\pgfqpoint{2.878446in}{2.642538in}}{\pgfqpoint{2.875174in}{2.634637in}}{\pgfqpoint{2.875174in}{2.626401in}}%
\pgfpathcurveto{\pgfqpoint{2.875174in}{2.618165in}}{\pgfqpoint{2.878446in}{2.610265in}}{\pgfqpoint{2.884270in}{2.604441in}}%
\pgfpathcurveto{\pgfqpoint{2.890094in}{2.598617in}}{\pgfqpoint{2.897994in}{2.595345in}}{\pgfqpoint{2.906230in}{2.595345in}}%
\pgfpathclose%
\pgfusepath{stroke,fill}%
\end{pgfscope}%
\begin{pgfscope}%
\pgfpathrectangle{\pgfqpoint{0.100000in}{0.212622in}}{\pgfqpoint{3.696000in}{3.696000in}}%
\pgfusepath{clip}%
\pgfsetbuttcap%
\pgfsetroundjoin%
\definecolor{currentfill}{rgb}{0.121569,0.466667,0.705882}%
\pgfsetfillcolor{currentfill}%
\pgfsetfillopacity{0.833970}%
\pgfsetlinewidth{1.003750pt}%
\definecolor{currentstroke}{rgb}{0.121569,0.466667,0.705882}%
\pgfsetstrokecolor{currentstroke}%
\pgfsetstrokeopacity{0.833970}%
\pgfsetdash{}{0pt}%
\pgfpathmoveto{\pgfqpoint{2.940048in}{2.617886in}}%
\pgfpathcurveto{\pgfqpoint{2.948284in}{2.617886in}}{\pgfqpoint{2.956184in}{2.621158in}}{\pgfqpoint{2.962008in}{2.626982in}}%
\pgfpathcurveto{\pgfqpoint{2.967832in}{2.632806in}}{\pgfqpoint{2.971105in}{2.640706in}}{\pgfqpoint{2.971105in}{2.648942in}}%
\pgfpathcurveto{\pgfqpoint{2.971105in}{2.657179in}}{\pgfqpoint{2.967832in}{2.665079in}}{\pgfqpoint{2.962008in}{2.670903in}}%
\pgfpathcurveto{\pgfqpoint{2.956184in}{2.676727in}}{\pgfqpoint{2.948284in}{2.679999in}}{\pgfqpoint{2.940048in}{2.679999in}}%
\pgfpathcurveto{\pgfqpoint{2.931812in}{2.679999in}}{\pgfqpoint{2.923912in}{2.676727in}}{\pgfqpoint{2.918088in}{2.670903in}}%
\pgfpathcurveto{\pgfqpoint{2.912264in}{2.665079in}}{\pgfqpoint{2.908992in}{2.657179in}}{\pgfqpoint{2.908992in}{2.648942in}}%
\pgfpathcurveto{\pgfqpoint{2.908992in}{2.640706in}}{\pgfqpoint{2.912264in}{2.632806in}}{\pgfqpoint{2.918088in}{2.626982in}}%
\pgfpathcurveto{\pgfqpoint{2.923912in}{2.621158in}}{\pgfqpoint{2.931812in}{2.617886in}}{\pgfqpoint{2.940048in}{2.617886in}}%
\pgfpathclose%
\pgfusepath{stroke,fill}%
\end{pgfscope}%
\begin{pgfscope}%
\pgfpathrectangle{\pgfqpoint{0.100000in}{0.212622in}}{\pgfqpoint{3.696000in}{3.696000in}}%
\pgfusepath{clip}%
\pgfsetbuttcap%
\pgfsetroundjoin%
\definecolor{currentfill}{rgb}{0.121569,0.466667,0.705882}%
\pgfsetfillcolor{currentfill}%
\pgfsetfillopacity{0.837503}%
\pgfsetlinewidth{1.003750pt}%
\definecolor{currentstroke}{rgb}{0.121569,0.466667,0.705882}%
\pgfsetstrokecolor{currentstroke}%
\pgfsetstrokeopacity{0.837503}%
\pgfsetdash{}{0pt}%
\pgfpathmoveto{\pgfqpoint{2.912012in}{2.603385in}}%
\pgfpathcurveto{\pgfqpoint{2.920248in}{2.603385in}}{\pgfqpoint{2.928148in}{2.606658in}}{\pgfqpoint{2.933972in}{2.612482in}}%
\pgfpathcurveto{\pgfqpoint{2.939796in}{2.618306in}}{\pgfqpoint{2.943068in}{2.626206in}}{\pgfqpoint{2.943068in}{2.634442in}}%
\pgfpathcurveto{\pgfqpoint{2.943068in}{2.642678in}}{\pgfqpoint{2.939796in}{2.650578in}}{\pgfqpoint{2.933972in}{2.656402in}}%
\pgfpathcurveto{\pgfqpoint{2.928148in}{2.662226in}}{\pgfqpoint{2.920248in}{2.665498in}}{\pgfqpoint{2.912012in}{2.665498in}}%
\pgfpathcurveto{\pgfqpoint{2.903775in}{2.665498in}}{\pgfqpoint{2.895875in}{2.662226in}}{\pgfqpoint{2.890051in}{2.656402in}}%
\pgfpathcurveto{\pgfqpoint{2.884228in}{2.650578in}}{\pgfqpoint{2.880955in}{2.642678in}}{\pgfqpoint{2.880955in}{2.634442in}}%
\pgfpathcurveto{\pgfqpoint{2.880955in}{2.626206in}}{\pgfqpoint{2.884228in}{2.618306in}}{\pgfqpoint{2.890051in}{2.612482in}}%
\pgfpathcurveto{\pgfqpoint{2.895875in}{2.606658in}}{\pgfqpoint{2.903775in}{2.603385in}}{\pgfqpoint{2.912012in}{2.603385in}}%
\pgfpathclose%
\pgfusepath{stroke,fill}%
\end{pgfscope}%
\begin{pgfscope}%
\pgfpathrectangle{\pgfqpoint{0.100000in}{0.212622in}}{\pgfqpoint{3.696000in}{3.696000in}}%
\pgfusepath{clip}%
\pgfsetbuttcap%
\pgfsetroundjoin%
\definecolor{currentfill}{rgb}{0.121569,0.466667,0.705882}%
\pgfsetfillcolor{currentfill}%
\pgfsetfillopacity{0.839833}%
\pgfsetlinewidth{1.003750pt}%
\definecolor{currentstroke}{rgb}{0.121569,0.466667,0.705882}%
\pgfsetstrokecolor{currentstroke}%
\pgfsetstrokeopacity{0.839833}%
\pgfsetdash{}{0pt}%
\pgfpathmoveto{\pgfqpoint{1.246400in}{1.847644in}}%
\pgfpathcurveto{\pgfqpoint{1.254636in}{1.847644in}}{\pgfqpoint{1.262536in}{1.850916in}}{\pgfqpoint{1.268360in}{1.856740in}}%
\pgfpathcurveto{\pgfqpoint{1.274184in}{1.862564in}}{\pgfqpoint{1.277457in}{1.870464in}}{\pgfqpoint{1.277457in}{1.878701in}}%
\pgfpathcurveto{\pgfqpoint{1.277457in}{1.886937in}}{\pgfqpoint{1.274184in}{1.894837in}}{\pgfqpoint{1.268360in}{1.900661in}}%
\pgfpathcurveto{\pgfqpoint{1.262536in}{1.906485in}}{\pgfqpoint{1.254636in}{1.909757in}}{\pgfqpoint{1.246400in}{1.909757in}}%
\pgfpathcurveto{\pgfqpoint{1.238164in}{1.909757in}}{\pgfqpoint{1.230264in}{1.906485in}}{\pgfqpoint{1.224440in}{1.900661in}}%
\pgfpathcurveto{\pgfqpoint{1.218616in}{1.894837in}}{\pgfqpoint{1.215344in}{1.886937in}}{\pgfqpoint{1.215344in}{1.878701in}}%
\pgfpathcurveto{\pgfqpoint{1.215344in}{1.870464in}}{\pgfqpoint{1.218616in}{1.862564in}}{\pgfqpoint{1.224440in}{1.856740in}}%
\pgfpathcurveto{\pgfqpoint{1.230264in}{1.850916in}}{\pgfqpoint{1.238164in}{1.847644in}}{\pgfqpoint{1.246400in}{1.847644in}}%
\pgfpathclose%
\pgfusepath{stroke,fill}%
\end{pgfscope}%
\begin{pgfscope}%
\pgfpathrectangle{\pgfqpoint{0.100000in}{0.212622in}}{\pgfqpoint{3.696000in}{3.696000in}}%
\pgfusepath{clip}%
\pgfsetbuttcap%
\pgfsetroundjoin%
\definecolor{currentfill}{rgb}{0.121569,0.466667,0.705882}%
\pgfsetfillcolor{currentfill}%
\pgfsetfillopacity{0.840036}%
\pgfsetlinewidth{1.003750pt}%
\definecolor{currentstroke}{rgb}{0.121569,0.466667,0.705882}%
\pgfsetstrokecolor{currentstroke}%
\pgfsetstrokeopacity{0.840036}%
\pgfsetdash{}{0pt}%
\pgfpathmoveto{\pgfqpoint{1.257977in}{1.853875in}}%
\pgfpathcurveto{\pgfqpoint{1.266213in}{1.853875in}}{\pgfqpoint{1.274113in}{1.857147in}}{\pgfqpoint{1.279937in}{1.862971in}}%
\pgfpathcurveto{\pgfqpoint{1.285761in}{1.868795in}}{\pgfqpoint{1.289033in}{1.876695in}}{\pgfqpoint{1.289033in}{1.884932in}}%
\pgfpathcurveto{\pgfqpoint{1.289033in}{1.893168in}}{\pgfqpoint{1.285761in}{1.901068in}}{\pgfqpoint{1.279937in}{1.906892in}}%
\pgfpathcurveto{\pgfqpoint{1.274113in}{1.912716in}}{\pgfqpoint{1.266213in}{1.915988in}}{\pgfqpoint{1.257977in}{1.915988in}}%
\pgfpathcurveto{\pgfqpoint{1.249741in}{1.915988in}}{\pgfqpoint{1.241841in}{1.912716in}}{\pgfqpoint{1.236017in}{1.906892in}}%
\pgfpathcurveto{\pgfqpoint{1.230193in}{1.901068in}}{\pgfqpoint{1.226920in}{1.893168in}}{\pgfqpoint{1.226920in}{1.884932in}}%
\pgfpathcurveto{\pgfqpoint{1.226920in}{1.876695in}}{\pgfqpoint{1.230193in}{1.868795in}}{\pgfqpoint{1.236017in}{1.862971in}}%
\pgfpathcurveto{\pgfqpoint{1.241841in}{1.857147in}}{\pgfqpoint{1.249741in}{1.853875in}}{\pgfqpoint{1.257977in}{1.853875in}}%
\pgfpathclose%
\pgfusepath{stroke,fill}%
\end{pgfscope}%
\begin{pgfscope}%
\pgfpathrectangle{\pgfqpoint{0.100000in}{0.212622in}}{\pgfqpoint{3.696000in}{3.696000in}}%
\pgfusepath{clip}%
\pgfsetbuttcap%
\pgfsetroundjoin%
\definecolor{currentfill}{rgb}{0.121569,0.466667,0.705882}%
\pgfsetfillcolor{currentfill}%
\pgfsetfillopacity{0.840203}%
\pgfsetlinewidth{1.003750pt}%
\definecolor{currentstroke}{rgb}{0.121569,0.466667,0.705882}%
\pgfsetstrokecolor{currentstroke}%
\pgfsetstrokeopacity{0.840203}%
\pgfsetdash{}{0pt}%
\pgfpathmoveto{\pgfqpoint{3.082382in}{2.668001in}}%
\pgfpathcurveto{\pgfqpoint{3.090618in}{2.668001in}}{\pgfqpoint{3.098519in}{2.671274in}}{\pgfqpoint{3.104342in}{2.677098in}}%
\pgfpathcurveto{\pgfqpoint{3.110166in}{2.682922in}}{\pgfqpoint{3.113439in}{2.690822in}}{\pgfqpoint{3.113439in}{2.699058in}}%
\pgfpathcurveto{\pgfqpoint{3.113439in}{2.707294in}}{\pgfqpoint{3.110166in}{2.715194in}}{\pgfqpoint{3.104342in}{2.721018in}}%
\pgfpathcurveto{\pgfqpoint{3.098519in}{2.726842in}}{\pgfqpoint{3.090618in}{2.730114in}}{\pgfqpoint{3.082382in}{2.730114in}}%
\pgfpathcurveto{\pgfqpoint{3.074146in}{2.730114in}}{\pgfqpoint{3.066246in}{2.726842in}}{\pgfqpoint{3.060422in}{2.721018in}}%
\pgfpathcurveto{\pgfqpoint{3.054598in}{2.715194in}}{\pgfqpoint{3.051326in}{2.707294in}}{\pgfqpoint{3.051326in}{2.699058in}}%
\pgfpathcurveto{\pgfqpoint{3.051326in}{2.690822in}}{\pgfqpoint{3.054598in}{2.682922in}}{\pgfqpoint{3.060422in}{2.677098in}}%
\pgfpathcurveto{\pgfqpoint{3.066246in}{2.671274in}}{\pgfqpoint{3.074146in}{2.668001in}}{\pgfqpoint{3.082382in}{2.668001in}}%
\pgfpathclose%
\pgfusepath{stroke,fill}%
\end{pgfscope}%
\begin{pgfscope}%
\pgfpathrectangle{\pgfqpoint{0.100000in}{0.212622in}}{\pgfqpoint{3.696000in}{3.696000in}}%
\pgfusepath{clip}%
\pgfsetbuttcap%
\pgfsetroundjoin%
\definecolor{currentfill}{rgb}{0.121569,0.466667,0.705882}%
\pgfsetfillcolor{currentfill}%
\pgfsetfillopacity{0.840807}%
\pgfsetlinewidth{1.003750pt}%
\definecolor{currentstroke}{rgb}{0.121569,0.466667,0.705882}%
\pgfsetstrokecolor{currentstroke}%
\pgfsetstrokeopacity{0.840807}%
\pgfsetdash{}{0pt}%
\pgfpathmoveto{\pgfqpoint{1.249693in}{1.851017in}}%
\pgfpathcurveto{\pgfqpoint{1.257930in}{1.851017in}}{\pgfqpoint{1.265830in}{1.854289in}}{\pgfqpoint{1.271654in}{1.860113in}}%
\pgfpathcurveto{\pgfqpoint{1.277477in}{1.865937in}}{\pgfqpoint{1.280750in}{1.873837in}}{\pgfqpoint{1.280750in}{1.882073in}}%
\pgfpathcurveto{\pgfqpoint{1.280750in}{1.890310in}}{\pgfqpoint{1.277477in}{1.898210in}}{\pgfqpoint{1.271654in}{1.904034in}}%
\pgfpathcurveto{\pgfqpoint{1.265830in}{1.909858in}}{\pgfqpoint{1.257930in}{1.913130in}}{\pgfqpoint{1.249693in}{1.913130in}}%
\pgfpathcurveto{\pgfqpoint{1.241457in}{1.913130in}}{\pgfqpoint{1.233557in}{1.909858in}}{\pgfqpoint{1.227733in}{1.904034in}}%
\pgfpathcurveto{\pgfqpoint{1.221909in}{1.898210in}}{\pgfqpoint{1.218637in}{1.890310in}}{\pgfqpoint{1.218637in}{1.882073in}}%
\pgfpathcurveto{\pgfqpoint{1.218637in}{1.873837in}}{\pgfqpoint{1.221909in}{1.865937in}}{\pgfqpoint{1.227733in}{1.860113in}}%
\pgfpathcurveto{\pgfqpoint{1.233557in}{1.854289in}}{\pgfqpoint{1.241457in}{1.851017in}}{\pgfqpoint{1.249693in}{1.851017in}}%
\pgfpathclose%
\pgfusepath{stroke,fill}%
\end{pgfscope}%
\begin{pgfscope}%
\pgfpathrectangle{\pgfqpoint{0.100000in}{0.212622in}}{\pgfqpoint{3.696000in}{3.696000in}}%
\pgfusepath{clip}%
\pgfsetbuttcap%
\pgfsetroundjoin%
\definecolor{currentfill}{rgb}{0.121569,0.466667,0.705882}%
\pgfsetfillcolor{currentfill}%
\pgfsetfillopacity{0.841169}%
\pgfsetlinewidth{1.003750pt}%
\definecolor{currentstroke}{rgb}{0.121569,0.466667,0.705882}%
\pgfsetstrokecolor{currentstroke}%
\pgfsetstrokeopacity{0.841169}%
\pgfsetdash{}{0pt}%
\pgfpathmoveto{\pgfqpoint{1.250807in}{1.849379in}}%
\pgfpathcurveto{\pgfqpoint{1.259043in}{1.849379in}}{\pgfqpoint{1.266943in}{1.852651in}}{\pgfqpoint{1.272767in}{1.858475in}}%
\pgfpathcurveto{\pgfqpoint{1.278591in}{1.864299in}}{\pgfqpoint{1.281863in}{1.872199in}}{\pgfqpoint{1.281863in}{1.880435in}}%
\pgfpathcurveto{\pgfqpoint{1.281863in}{1.888672in}}{\pgfqpoint{1.278591in}{1.896572in}}{\pgfqpoint{1.272767in}{1.902396in}}%
\pgfpathcurveto{\pgfqpoint{1.266943in}{1.908220in}}{\pgfqpoint{1.259043in}{1.911492in}}{\pgfqpoint{1.250807in}{1.911492in}}%
\pgfpathcurveto{\pgfqpoint{1.242571in}{1.911492in}}{\pgfqpoint{1.234671in}{1.908220in}}{\pgfqpoint{1.228847in}{1.902396in}}%
\pgfpathcurveto{\pgfqpoint{1.223023in}{1.896572in}}{\pgfqpoint{1.219750in}{1.888672in}}{\pgfqpoint{1.219750in}{1.880435in}}%
\pgfpathcurveto{\pgfqpoint{1.219750in}{1.872199in}}{\pgfqpoint{1.223023in}{1.864299in}}{\pgfqpoint{1.228847in}{1.858475in}}%
\pgfpathcurveto{\pgfqpoint{1.234671in}{1.852651in}}{\pgfqpoint{1.242571in}{1.849379in}}{\pgfqpoint{1.250807in}{1.849379in}}%
\pgfpathclose%
\pgfusepath{stroke,fill}%
\end{pgfscope}%
\begin{pgfscope}%
\pgfpathrectangle{\pgfqpoint{0.100000in}{0.212622in}}{\pgfqpoint{3.696000in}{3.696000in}}%
\pgfusepath{clip}%
\pgfsetbuttcap%
\pgfsetroundjoin%
\definecolor{currentfill}{rgb}{0.121569,0.466667,0.705882}%
\pgfsetfillcolor{currentfill}%
\pgfsetfillopacity{0.841300}%
\pgfsetlinewidth{1.003750pt}%
\definecolor{currentstroke}{rgb}{0.121569,0.466667,0.705882}%
\pgfsetstrokecolor{currentstroke}%
\pgfsetstrokeopacity{0.841300}%
\pgfsetdash{}{0pt}%
\pgfpathmoveto{\pgfqpoint{3.050437in}{2.641975in}}%
\pgfpathcurveto{\pgfqpoint{3.058673in}{2.641975in}}{\pgfqpoint{3.066573in}{2.645248in}}{\pgfqpoint{3.072397in}{2.651072in}}%
\pgfpathcurveto{\pgfqpoint{3.078221in}{2.656895in}}{\pgfqpoint{3.081493in}{2.664796in}}{\pgfqpoint{3.081493in}{2.673032in}}%
\pgfpathcurveto{\pgfqpoint{3.081493in}{2.681268in}}{\pgfqpoint{3.078221in}{2.689168in}}{\pgfqpoint{3.072397in}{2.694992in}}%
\pgfpathcurveto{\pgfqpoint{3.066573in}{2.700816in}}{\pgfqpoint{3.058673in}{2.704088in}}{\pgfqpoint{3.050437in}{2.704088in}}%
\pgfpathcurveto{\pgfqpoint{3.042200in}{2.704088in}}{\pgfqpoint{3.034300in}{2.700816in}}{\pgfqpoint{3.028476in}{2.694992in}}%
\pgfpathcurveto{\pgfqpoint{3.022652in}{2.689168in}}{\pgfqpoint{3.019380in}{2.681268in}}{\pgfqpoint{3.019380in}{2.673032in}}%
\pgfpathcurveto{\pgfqpoint{3.019380in}{2.664796in}}{\pgfqpoint{3.022652in}{2.656895in}}{\pgfqpoint{3.028476in}{2.651072in}}%
\pgfpathcurveto{\pgfqpoint{3.034300in}{2.645248in}}{\pgfqpoint{3.042200in}{2.641975in}}{\pgfqpoint{3.050437in}{2.641975in}}%
\pgfpathclose%
\pgfusepath{stroke,fill}%
\end{pgfscope}%
\begin{pgfscope}%
\pgfpathrectangle{\pgfqpoint{0.100000in}{0.212622in}}{\pgfqpoint{3.696000in}{3.696000in}}%
\pgfusepath{clip}%
\pgfsetbuttcap%
\pgfsetroundjoin%
\definecolor{currentfill}{rgb}{0.121569,0.466667,0.705882}%
\pgfsetfillcolor{currentfill}%
\pgfsetfillopacity{0.841515}%
\pgfsetlinewidth{1.003750pt}%
\definecolor{currentstroke}{rgb}{0.121569,0.466667,0.705882}%
\pgfsetstrokecolor{currentstroke}%
\pgfsetstrokeopacity{0.841515}%
\pgfsetdash{}{0pt}%
\pgfpathmoveto{\pgfqpoint{2.908994in}{2.585486in}}%
\pgfpathcurveto{\pgfqpoint{2.917230in}{2.585486in}}{\pgfqpoint{2.925130in}{2.588759in}}{\pgfqpoint{2.930954in}{2.594583in}}%
\pgfpathcurveto{\pgfqpoint{2.936778in}{2.600407in}}{\pgfqpoint{2.940050in}{2.608307in}}{\pgfqpoint{2.940050in}{2.616543in}}%
\pgfpathcurveto{\pgfqpoint{2.940050in}{2.624779in}}{\pgfqpoint{2.936778in}{2.632679in}}{\pgfqpoint{2.930954in}{2.638503in}}%
\pgfpathcurveto{\pgfqpoint{2.925130in}{2.644327in}}{\pgfqpoint{2.917230in}{2.647599in}}{\pgfqpoint{2.908994in}{2.647599in}}%
\pgfpathcurveto{\pgfqpoint{2.900757in}{2.647599in}}{\pgfqpoint{2.892857in}{2.644327in}}{\pgfqpoint{2.887033in}{2.638503in}}%
\pgfpathcurveto{\pgfqpoint{2.881209in}{2.632679in}}{\pgfqpoint{2.877937in}{2.624779in}}{\pgfqpoint{2.877937in}{2.616543in}}%
\pgfpathcurveto{\pgfqpoint{2.877937in}{2.608307in}}{\pgfqpoint{2.881209in}{2.600407in}}{\pgfqpoint{2.887033in}{2.594583in}}%
\pgfpathcurveto{\pgfqpoint{2.892857in}{2.588759in}}{\pgfqpoint{2.900757in}{2.585486in}}{\pgfqpoint{2.908994in}{2.585486in}}%
\pgfpathclose%
\pgfusepath{stroke,fill}%
\end{pgfscope}%
\begin{pgfscope}%
\pgfpathrectangle{\pgfqpoint{0.100000in}{0.212622in}}{\pgfqpoint{3.696000in}{3.696000in}}%
\pgfusepath{clip}%
\pgfsetbuttcap%
\pgfsetroundjoin%
\definecolor{currentfill}{rgb}{0.121569,0.466667,0.705882}%
\pgfsetfillcolor{currentfill}%
\pgfsetfillopacity{0.842755}%
\pgfsetlinewidth{1.003750pt}%
\definecolor{currentstroke}{rgb}{0.121569,0.466667,0.705882}%
\pgfsetstrokecolor{currentstroke}%
\pgfsetstrokeopacity{0.842755}%
\pgfsetdash{}{0pt}%
\pgfpathmoveto{\pgfqpoint{3.043462in}{2.635256in}}%
\pgfpathcurveto{\pgfqpoint{3.051698in}{2.635256in}}{\pgfqpoint{3.059598in}{2.638529in}}{\pgfqpoint{3.065422in}{2.644353in}}%
\pgfpathcurveto{\pgfqpoint{3.071246in}{2.650177in}}{\pgfqpoint{3.074518in}{2.658077in}}{\pgfqpoint{3.074518in}{2.666313in}}%
\pgfpathcurveto{\pgfqpoint{3.074518in}{2.674549in}}{\pgfqpoint{3.071246in}{2.682449in}}{\pgfqpoint{3.065422in}{2.688273in}}%
\pgfpathcurveto{\pgfqpoint{3.059598in}{2.694097in}}{\pgfqpoint{3.051698in}{2.697369in}}{\pgfqpoint{3.043462in}{2.697369in}}%
\pgfpathcurveto{\pgfqpoint{3.035225in}{2.697369in}}{\pgfqpoint{3.027325in}{2.694097in}}{\pgfqpoint{3.021501in}{2.688273in}}%
\pgfpathcurveto{\pgfqpoint{3.015678in}{2.682449in}}{\pgfqpoint{3.012405in}{2.674549in}}{\pgfqpoint{3.012405in}{2.666313in}}%
\pgfpathcurveto{\pgfqpoint{3.012405in}{2.658077in}}{\pgfqpoint{3.015678in}{2.650177in}}{\pgfqpoint{3.021501in}{2.644353in}}%
\pgfpathcurveto{\pgfqpoint{3.027325in}{2.638529in}}{\pgfqpoint{3.035225in}{2.635256in}}{\pgfqpoint{3.043462in}{2.635256in}}%
\pgfpathclose%
\pgfusepath{stroke,fill}%
\end{pgfscope}%
\begin{pgfscope}%
\pgfpathrectangle{\pgfqpoint{0.100000in}{0.212622in}}{\pgfqpoint{3.696000in}{3.696000in}}%
\pgfusepath{clip}%
\pgfsetbuttcap%
\pgfsetroundjoin%
\definecolor{currentfill}{rgb}{0.121569,0.466667,0.705882}%
\pgfsetfillcolor{currentfill}%
\pgfsetfillopacity{0.842872}%
\pgfsetlinewidth{1.003750pt}%
\definecolor{currentstroke}{rgb}{0.121569,0.466667,0.705882}%
\pgfsetstrokecolor{currentstroke}%
\pgfsetstrokeopacity{0.842872}%
\pgfsetdash{}{0pt}%
\pgfpathmoveto{\pgfqpoint{2.987513in}{2.605517in}}%
\pgfpathcurveto{\pgfqpoint{2.995749in}{2.605517in}}{\pgfqpoint{3.003649in}{2.608789in}}{\pgfqpoint{3.009473in}{2.614613in}}%
\pgfpathcurveto{\pgfqpoint{3.015297in}{2.620437in}}{\pgfqpoint{3.018569in}{2.628337in}}{\pgfqpoint{3.018569in}{2.636573in}}%
\pgfpathcurveto{\pgfqpoint{3.018569in}{2.644809in}}{\pgfqpoint{3.015297in}{2.652709in}}{\pgfqpoint{3.009473in}{2.658533in}}%
\pgfpathcurveto{\pgfqpoint{3.003649in}{2.664357in}}{\pgfqpoint{2.995749in}{2.667630in}}{\pgfqpoint{2.987513in}{2.667630in}}%
\pgfpathcurveto{\pgfqpoint{2.979276in}{2.667630in}}{\pgfqpoint{2.971376in}{2.664357in}}{\pgfqpoint{2.965552in}{2.658533in}}%
\pgfpathcurveto{\pgfqpoint{2.959728in}{2.652709in}}{\pgfqpoint{2.956456in}{2.644809in}}{\pgfqpoint{2.956456in}{2.636573in}}%
\pgfpathcurveto{\pgfqpoint{2.956456in}{2.628337in}}{\pgfqpoint{2.959728in}{2.620437in}}{\pgfqpoint{2.965552in}{2.614613in}}%
\pgfpathcurveto{\pgfqpoint{2.971376in}{2.608789in}}{\pgfqpoint{2.979276in}{2.605517in}}{\pgfqpoint{2.987513in}{2.605517in}}%
\pgfpathclose%
\pgfusepath{stroke,fill}%
\end{pgfscope}%
\begin{pgfscope}%
\pgfpathrectangle{\pgfqpoint{0.100000in}{0.212622in}}{\pgfqpoint{3.696000in}{3.696000in}}%
\pgfusepath{clip}%
\pgfsetbuttcap%
\pgfsetroundjoin%
\definecolor{currentfill}{rgb}{0.121569,0.466667,0.705882}%
\pgfsetfillcolor{currentfill}%
\pgfsetfillopacity{0.843093}%
\pgfsetlinewidth{1.003750pt}%
\definecolor{currentstroke}{rgb}{0.121569,0.466667,0.705882}%
\pgfsetstrokecolor{currentstroke}%
\pgfsetstrokeopacity{0.843093}%
\pgfsetdash{}{0pt}%
\pgfpathmoveto{\pgfqpoint{3.048439in}{2.639242in}}%
\pgfpathcurveto{\pgfqpoint{3.056675in}{2.639242in}}{\pgfqpoint{3.064575in}{2.642515in}}{\pgfqpoint{3.070399in}{2.648338in}}%
\pgfpathcurveto{\pgfqpoint{3.076223in}{2.654162in}}{\pgfqpoint{3.079496in}{2.662062in}}{\pgfqpoint{3.079496in}{2.670299in}}%
\pgfpathcurveto{\pgfqpoint{3.079496in}{2.678535in}}{\pgfqpoint{3.076223in}{2.686435in}}{\pgfqpoint{3.070399in}{2.692259in}}%
\pgfpathcurveto{\pgfqpoint{3.064575in}{2.698083in}}{\pgfqpoint{3.056675in}{2.701355in}}{\pgfqpoint{3.048439in}{2.701355in}}%
\pgfpathcurveto{\pgfqpoint{3.040203in}{2.701355in}}{\pgfqpoint{3.032303in}{2.698083in}}{\pgfqpoint{3.026479in}{2.692259in}}%
\pgfpathcurveto{\pgfqpoint{3.020655in}{2.686435in}}{\pgfqpoint{3.017383in}{2.678535in}}{\pgfqpoint{3.017383in}{2.670299in}}%
\pgfpathcurveto{\pgfqpoint{3.017383in}{2.662062in}}{\pgfqpoint{3.020655in}{2.654162in}}{\pgfqpoint{3.026479in}{2.648338in}}%
\pgfpathcurveto{\pgfqpoint{3.032303in}{2.642515in}}{\pgfqpoint{3.040203in}{2.639242in}}{\pgfqpoint{3.048439in}{2.639242in}}%
\pgfpathclose%
\pgfusepath{stroke,fill}%
\end{pgfscope}%
\begin{pgfscope}%
\pgfpathrectangle{\pgfqpoint{0.100000in}{0.212622in}}{\pgfqpoint{3.696000in}{3.696000in}}%
\pgfusepath{clip}%
\pgfsetbuttcap%
\pgfsetroundjoin%
\definecolor{currentfill}{rgb}{0.121569,0.466667,0.705882}%
\pgfsetfillcolor{currentfill}%
\pgfsetfillopacity{0.843193}%
\pgfsetlinewidth{1.003750pt}%
\definecolor{currentstroke}{rgb}{0.121569,0.466667,0.705882}%
\pgfsetstrokecolor{currentstroke}%
\pgfsetstrokeopacity{0.843193}%
\pgfsetdash{}{0pt}%
\pgfpathmoveto{\pgfqpoint{2.930693in}{2.592384in}}%
\pgfpathcurveto{\pgfqpoint{2.938929in}{2.592384in}}{\pgfqpoint{2.946830in}{2.595657in}}{\pgfqpoint{2.952653in}{2.601481in}}%
\pgfpathcurveto{\pgfqpoint{2.958477in}{2.607304in}}{\pgfqpoint{2.961750in}{2.615205in}}{\pgfqpoint{2.961750in}{2.623441in}}%
\pgfpathcurveto{\pgfqpoint{2.961750in}{2.631677in}}{\pgfqpoint{2.958477in}{2.639577in}}{\pgfqpoint{2.952653in}{2.645401in}}%
\pgfpathcurveto{\pgfqpoint{2.946830in}{2.651225in}}{\pgfqpoint{2.938929in}{2.654497in}}{\pgfqpoint{2.930693in}{2.654497in}}%
\pgfpathcurveto{\pgfqpoint{2.922457in}{2.654497in}}{\pgfqpoint{2.914557in}{2.651225in}}{\pgfqpoint{2.908733in}{2.645401in}}%
\pgfpathcurveto{\pgfqpoint{2.902909in}{2.639577in}}{\pgfqpoint{2.899637in}{2.631677in}}{\pgfqpoint{2.899637in}{2.623441in}}%
\pgfpathcurveto{\pgfqpoint{2.899637in}{2.615205in}}{\pgfqpoint{2.902909in}{2.607304in}}{\pgfqpoint{2.908733in}{2.601481in}}%
\pgfpathcurveto{\pgfqpoint{2.914557in}{2.595657in}}{\pgfqpoint{2.922457in}{2.592384in}}{\pgfqpoint{2.930693in}{2.592384in}}%
\pgfpathclose%
\pgfusepath{stroke,fill}%
\end{pgfscope}%
\begin{pgfscope}%
\pgfpathrectangle{\pgfqpoint{0.100000in}{0.212622in}}{\pgfqpoint{3.696000in}{3.696000in}}%
\pgfusepath{clip}%
\pgfsetbuttcap%
\pgfsetroundjoin%
\definecolor{currentfill}{rgb}{0.121569,0.466667,0.705882}%
\pgfsetfillcolor{currentfill}%
\pgfsetfillopacity{0.844271}%
\pgfsetlinewidth{1.003750pt}%
\definecolor{currentstroke}{rgb}{0.121569,0.466667,0.705882}%
\pgfsetstrokecolor{currentstroke}%
\pgfsetstrokeopacity{0.844271}%
\pgfsetdash{}{0pt}%
\pgfpathmoveto{\pgfqpoint{2.884962in}{2.570209in}}%
\pgfpathcurveto{\pgfqpoint{2.893198in}{2.570209in}}{\pgfqpoint{2.901098in}{2.573482in}}{\pgfqpoint{2.906922in}{2.579306in}}%
\pgfpathcurveto{\pgfqpoint{2.912746in}{2.585130in}}{\pgfqpoint{2.916018in}{2.593030in}}{\pgfqpoint{2.916018in}{2.601266in}}%
\pgfpathcurveto{\pgfqpoint{2.916018in}{2.609502in}}{\pgfqpoint{2.912746in}{2.617402in}}{\pgfqpoint{2.906922in}{2.623226in}}%
\pgfpathcurveto{\pgfqpoint{2.901098in}{2.629050in}}{\pgfqpoint{2.893198in}{2.632322in}}{\pgfqpoint{2.884962in}{2.632322in}}%
\pgfpathcurveto{\pgfqpoint{2.876726in}{2.632322in}}{\pgfqpoint{2.868826in}{2.629050in}}{\pgfqpoint{2.863002in}{2.623226in}}%
\pgfpathcurveto{\pgfqpoint{2.857178in}{2.617402in}}{\pgfqpoint{2.853905in}{2.609502in}}{\pgfqpoint{2.853905in}{2.601266in}}%
\pgfpathcurveto{\pgfqpoint{2.853905in}{2.593030in}}{\pgfqpoint{2.857178in}{2.585130in}}{\pgfqpoint{2.863002in}{2.579306in}}%
\pgfpathcurveto{\pgfqpoint{2.868826in}{2.573482in}}{\pgfqpoint{2.876726in}{2.570209in}}{\pgfqpoint{2.884962in}{2.570209in}}%
\pgfpathclose%
\pgfusepath{stroke,fill}%
\end{pgfscope}%
\begin{pgfscope}%
\pgfpathrectangle{\pgfqpoint{0.100000in}{0.212622in}}{\pgfqpoint{3.696000in}{3.696000in}}%
\pgfusepath{clip}%
\pgfsetbuttcap%
\pgfsetroundjoin%
\definecolor{currentfill}{rgb}{0.121569,0.466667,0.705882}%
\pgfsetfillcolor{currentfill}%
\pgfsetfillopacity{0.844435}%
\pgfsetlinewidth{1.003750pt}%
\definecolor{currentstroke}{rgb}{0.121569,0.466667,0.705882}%
\pgfsetstrokecolor{currentstroke}%
\pgfsetstrokeopacity{0.844435}%
\pgfsetdash{}{0pt}%
\pgfpathmoveto{\pgfqpoint{3.034310in}{2.630162in}}%
\pgfpathcurveto{\pgfqpoint{3.042547in}{2.630162in}}{\pgfqpoint{3.050447in}{2.633434in}}{\pgfqpoint{3.056271in}{2.639258in}}%
\pgfpathcurveto{\pgfqpoint{3.062095in}{2.645082in}}{\pgfqpoint{3.065367in}{2.652982in}}{\pgfqpoint{3.065367in}{2.661218in}}%
\pgfpathcurveto{\pgfqpoint{3.065367in}{2.669455in}}{\pgfqpoint{3.062095in}{2.677355in}}{\pgfqpoint{3.056271in}{2.683179in}}%
\pgfpathcurveto{\pgfqpoint{3.050447in}{2.689003in}}{\pgfqpoint{3.042547in}{2.692275in}}{\pgfqpoint{3.034310in}{2.692275in}}%
\pgfpathcurveto{\pgfqpoint{3.026074in}{2.692275in}}{\pgfqpoint{3.018174in}{2.689003in}}{\pgfqpoint{3.012350in}{2.683179in}}%
\pgfpathcurveto{\pgfqpoint{3.006526in}{2.677355in}}{\pgfqpoint{3.003254in}{2.669455in}}{\pgfqpoint{3.003254in}{2.661218in}}%
\pgfpathcurveto{\pgfqpoint{3.003254in}{2.652982in}}{\pgfqpoint{3.006526in}{2.645082in}}{\pgfqpoint{3.012350in}{2.639258in}}%
\pgfpathcurveto{\pgfqpoint{3.018174in}{2.633434in}}{\pgfqpoint{3.026074in}{2.630162in}}{\pgfqpoint{3.034310in}{2.630162in}}%
\pgfpathclose%
\pgfusepath{stroke,fill}%
\end{pgfscope}%
\begin{pgfscope}%
\pgfpathrectangle{\pgfqpoint{0.100000in}{0.212622in}}{\pgfqpoint{3.696000in}{3.696000in}}%
\pgfusepath{clip}%
\pgfsetbuttcap%
\pgfsetroundjoin%
\definecolor{currentfill}{rgb}{0.121569,0.466667,0.705882}%
\pgfsetfillcolor{currentfill}%
\pgfsetfillopacity{0.845902}%
\pgfsetlinewidth{1.003750pt}%
\definecolor{currentstroke}{rgb}{0.121569,0.466667,0.705882}%
\pgfsetstrokecolor{currentstroke}%
\pgfsetstrokeopacity{0.845902}%
\pgfsetdash{}{0pt}%
\pgfpathmoveto{\pgfqpoint{3.022386in}{2.628824in}}%
\pgfpathcurveto{\pgfqpoint{3.030622in}{2.628824in}}{\pgfqpoint{3.038522in}{2.632097in}}{\pgfqpoint{3.044346in}{2.637921in}}%
\pgfpathcurveto{\pgfqpoint{3.050170in}{2.643744in}}{\pgfqpoint{3.053443in}{2.651644in}}{\pgfqpoint{3.053443in}{2.659881in}}%
\pgfpathcurveto{\pgfqpoint{3.053443in}{2.668117in}}{\pgfqpoint{3.050170in}{2.676017in}}{\pgfqpoint{3.044346in}{2.681841in}}%
\pgfpathcurveto{\pgfqpoint{3.038522in}{2.687665in}}{\pgfqpoint{3.030622in}{2.690937in}}{\pgfqpoint{3.022386in}{2.690937in}}%
\pgfpathcurveto{\pgfqpoint{3.014150in}{2.690937in}}{\pgfqpoint{3.006250in}{2.687665in}}{\pgfqpoint{3.000426in}{2.681841in}}%
\pgfpathcurveto{\pgfqpoint{2.994602in}{2.676017in}}{\pgfqpoint{2.991330in}{2.668117in}}{\pgfqpoint{2.991330in}{2.659881in}}%
\pgfpathcurveto{\pgfqpoint{2.991330in}{2.651644in}}{\pgfqpoint{2.994602in}{2.643744in}}{\pgfqpoint{3.000426in}{2.637921in}}%
\pgfpathcurveto{\pgfqpoint{3.006250in}{2.632097in}}{\pgfqpoint{3.014150in}{2.628824in}}{\pgfqpoint{3.022386in}{2.628824in}}%
\pgfpathclose%
\pgfusepath{stroke,fill}%
\end{pgfscope}%
\begin{pgfscope}%
\pgfpathrectangle{\pgfqpoint{0.100000in}{0.212622in}}{\pgfqpoint{3.696000in}{3.696000in}}%
\pgfusepath{clip}%
\pgfsetbuttcap%
\pgfsetroundjoin%
\definecolor{currentfill}{rgb}{0.121569,0.466667,0.705882}%
\pgfsetfillcolor{currentfill}%
\pgfsetfillopacity{0.845964}%
\pgfsetlinewidth{1.003750pt}%
\definecolor{currentstroke}{rgb}{0.121569,0.466667,0.705882}%
\pgfsetstrokecolor{currentstroke}%
\pgfsetstrokeopacity{0.845964}%
\pgfsetdash{}{0pt}%
\pgfpathmoveto{\pgfqpoint{3.052395in}{2.640468in}}%
\pgfpathcurveto{\pgfqpoint{3.060631in}{2.640468in}}{\pgfqpoint{3.068531in}{2.643741in}}{\pgfqpoint{3.074355in}{2.649565in}}%
\pgfpathcurveto{\pgfqpoint{3.080179in}{2.655388in}}{\pgfqpoint{3.083451in}{2.663288in}}{\pgfqpoint{3.083451in}{2.671525in}}%
\pgfpathcurveto{\pgfqpoint{3.083451in}{2.679761in}}{\pgfqpoint{3.080179in}{2.687661in}}{\pgfqpoint{3.074355in}{2.693485in}}%
\pgfpathcurveto{\pgfqpoint{3.068531in}{2.699309in}}{\pgfqpoint{3.060631in}{2.702581in}}{\pgfqpoint{3.052395in}{2.702581in}}%
\pgfpathcurveto{\pgfqpoint{3.044158in}{2.702581in}}{\pgfqpoint{3.036258in}{2.699309in}}{\pgfqpoint{3.030434in}{2.693485in}}%
\pgfpathcurveto{\pgfqpoint{3.024610in}{2.687661in}}{\pgfqpoint{3.021338in}{2.679761in}}{\pgfqpoint{3.021338in}{2.671525in}}%
\pgfpathcurveto{\pgfqpoint{3.021338in}{2.663288in}}{\pgfqpoint{3.024610in}{2.655388in}}{\pgfqpoint{3.030434in}{2.649565in}}%
\pgfpathcurveto{\pgfqpoint{3.036258in}{2.643741in}}{\pgfqpoint{3.044158in}{2.640468in}}{\pgfqpoint{3.052395in}{2.640468in}}%
\pgfpathclose%
\pgfusepath{stroke,fill}%
\end{pgfscope}%
\begin{pgfscope}%
\pgfpathrectangle{\pgfqpoint{0.100000in}{0.212622in}}{\pgfqpoint{3.696000in}{3.696000in}}%
\pgfusepath{clip}%
\pgfsetbuttcap%
\pgfsetroundjoin%
\definecolor{currentfill}{rgb}{0.121569,0.466667,0.705882}%
\pgfsetfillcolor{currentfill}%
\pgfsetfillopacity{0.846058}%
\pgfsetlinewidth{1.003750pt}%
\definecolor{currentstroke}{rgb}{0.121569,0.466667,0.705882}%
\pgfsetstrokecolor{currentstroke}%
\pgfsetstrokeopacity{0.846058}%
\pgfsetdash{}{0pt}%
\pgfpathmoveto{\pgfqpoint{1.248600in}{1.838836in}}%
\pgfpathcurveto{\pgfqpoint{1.256837in}{1.838836in}}{\pgfqpoint{1.264737in}{1.842109in}}{\pgfqpoint{1.270561in}{1.847933in}}%
\pgfpathcurveto{\pgfqpoint{1.276385in}{1.853757in}}{\pgfqpoint{1.279657in}{1.861657in}}{\pgfqpoint{1.279657in}{1.869893in}}%
\pgfpathcurveto{\pgfqpoint{1.279657in}{1.878129in}}{\pgfqpoint{1.276385in}{1.886029in}}{\pgfqpoint{1.270561in}{1.891853in}}%
\pgfpathcurveto{\pgfqpoint{1.264737in}{1.897677in}}{\pgfqpoint{1.256837in}{1.900949in}}{\pgfqpoint{1.248600in}{1.900949in}}%
\pgfpathcurveto{\pgfqpoint{1.240364in}{1.900949in}}{\pgfqpoint{1.232464in}{1.897677in}}{\pgfqpoint{1.226640in}{1.891853in}}%
\pgfpathcurveto{\pgfqpoint{1.220816in}{1.886029in}}{\pgfqpoint{1.217544in}{1.878129in}}{\pgfqpoint{1.217544in}{1.869893in}}%
\pgfpathcurveto{\pgfqpoint{1.217544in}{1.861657in}}{\pgfqpoint{1.220816in}{1.853757in}}{\pgfqpoint{1.226640in}{1.847933in}}%
\pgfpathcurveto{\pgfqpoint{1.232464in}{1.842109in}}{\pgfqpoint{1.240364in}{1.838836in}}{\pgfqpoint{1.248600in}{1.838836in}}%
\pgfpathclose%
\pgfusepath{stroke,fill}%
\end{pgfscope}%
\begin{pgfscope}%
\pgfpathrectangle{\pgfqpoint{0.100000in}{0.212622in}}{\pgfqpoint{3.696000in}{3.696000in}}%
\pgfusepath{clip}%
\pgfsetbuttcap%
\pgfsetroundjoin%
\definecolor{currentfill}{rgb}{0.121569,0.466667,0.705882}%
\pgfsetfillcolor{currentfill}%
\pgfsetfillopacity{0.846229}%
\pgfsetlinewidth{1.003750pt}%
\definecolor{currentstroke}{rgb}{0.121569,0.466667,0.705882}%
\pgfsetstrokecolor{currentstroke}%
\pgfsetstrokeopacity{0.846229}%
\pgfsetdash{}{0pt}%
\pgfpathmoveto{\pgfqpoint{3.046276in}{2.637170in}}%
\pgfpathcurveto{\pgfqpoint{3.054512in}{2.637170in}}{\pgfqpoint{3.062412in}{2.640442in}}{\pgfqpoint{3.068236in}{2.646266in}}%
\pgfpathcurveto{\pgfqpoint{3.074060in}{2.652090in}}{\pgfqpoint{3.077332in}{2.659990in}}{\pgfqpoint{3.077332in}{2.668226in}}%
\pgfpathcurveto{\pgfqpoint{3.077332in}{2.676462in}}{\pgfqpoint{3.074060in}{2.684362in}}{\pgfqpoint{3.068236in}{2.690186in}}%
\pgfpathcurveto{\pgfqpoint{3.062412in}{2.696010in}}{\pgfqpoint{3.054512in}{2.699283in}}{\pgfqpoint{3.046276in}{2.699283in}}%
\pgfpathcurveto{\pgfqpoint{3.038040in}{2.699283in}}{\pgfqpoint{3.030140in}{2.696010in}}{\pgfqpoint{3.024316in}{2.690186in}}%
\pgfpathcurveto{\pgfqpoint{3.018492in}{2.684362in}}{\pgfqpoint{3.015219in}{2.676462in}}{\pgfqpoint{3.015219in}{2.668226in}}%
\pgfpathcurveto{\pgfqpoint{3.015219in}{2.659990in}}{\pgfqpoint{3.018492in}{2.652090in}}{\pgfqpoint{3.024316in}{2.646266in}}%
\pgfpathcurveto{\pgfqpoint{3.030140in}{2.640442in}}{\pgfqpoint{3.038040in}{2.637170in}}{\pgfqpoint{3.046276in}{2.637170in}}%
\pgfpathclose%
\pgfusepath{stroke,fill}%
\end{pgfscope}%
\begin{pgfscope}%
\pgfpathrectangle{\pgfqpoint{0.100000in}{0.212622in}}{\pgfqpoint{3.696000in}{3.696000in}}%
\pgfusepath{clip}%
\pgfsetbuttcap%
\pgfsetroundjoin%
\definecolor{currentfill}{rgb}{0.121569,0.466667,0.705882}%
\pgfsetfillcolor{currentfill}%
\pgfsetfillopacity{0.847917}%
\pgfsetlinewidth{1.003750pt}%
\definecolor{currentstroke}{rgb}{0.121569,0.466667,0.705882}%
\pgfsetstrokecolor{currentstroke}%
\pgfsetstrokeopacity{0.847917}%
\pgfsetdash{}{0pt}%
\pgfpathmoveto{\pgfqpoint{3.053147in}{2.633166in}}%
\pgfpathcurveto{\pgfqpoint{3.061383in}{2.633166in}}{\pgfqpoint{3.069283in}{2.636438in}}{\pgfqpoint{3.075107in}{2.642262in}}%
\pgfpathcurveto{\pgfqpoint{3.080931in}{2.648086in}}{\pgfqpoint{3.084204in}{2.655986in}}{\pgfqpoint{3.084204in}{2.664222in}}%
\pgfpathcurveto{\pgfqpoint{3.084204in}{2.672458in}}{\pgfqpoint{3.080931in}{2.680358in}}{\pgfqpoint{3.075107in}{2.686182in}}%
\pgfpathcurveto{\pgfqpoint{3.069283in}{2.692006in}}{\pgfqpoint{3.061383in}{2.695279in}}{\pgfqpoint{3.053147in}{2.695279in}}%
\pgfpathcurveto{\pgfqpoint{3.044911in}{2.695279in}}{\pgfqpoint{3.037011in}{2.692006in}}{\pgfqpoint{3.031187in}{2.686182in}}%
\pgfpathcurveto{\pgfqpoint{3.025363in}{2.680358in}}{\pgfqpoint{3.022091in}{2.672458in}}{\pgfqpoint{3.022091in}{2.664222in}}%
\pgfpathcurveto{\pgfqpoint{3.022091in}{2.655986in}}{\pgfqpoint{3.025363in}{2.648086in}}{\pgfqpoint{3.031187in}{2.642262in}}%
\pgfpathcurveto{\pgfqpoint{3.037011in}{2.636438in}}{\pgfqpoint{3.044911in}{2.633166in}}{\pgfqpoint{3.053147in}{2.633166in}}%
\pgfpathclose%
\pgfusepath{stroke,fill}%
\end{pgfscope}%
\begin{pgfscope}%
\pgfpathrectangle{\pgfqpoint{0.100000in}{0.212622in}}{\pgfqpoint{3.696000in}{3.696000in}}%
\pgfusepath{clip}%
\pgfsetbuttcap%
\pgfsetroundjoin%
\definecolor{currentfill}{rgb}{0.121569,0.466667,0.705882}%
\pgfsetfillcolor{currentfill}%
\pgfsetfillopacity{0.849842}%
\pgfsetlinewidth{1.003750pt}%
\definecolor{currentstroke}{rgb}{0.121569,0.466667,0.705882}%
\pgfsetstrokecolor{currentstroke}%
\pgfsetstrokeopacity{0.849842}%
\pgfsetdash{}{0pt}%
\pgfpathmoveto{\pgfqpoint{1.249468in}{1.841646in}}%
\pgfpathcurveto{\pgfqpoint{1.257704in}{1.841646in}}{\pgfqpoint{1.265604in}{1.844918in}}{\pgfqpoint{1.271428in}{1.850742in}}%
\pgfpathcurveto{\pgfqpoint{1.277252in}{1.856566in}}{\pgfqpoint{1.280524in}{1.864466in}}{\pgfqpoint{1.280524in}{1.872702in}}%
\pgfpathcurveto{\pgfqpoint{1.280524in}{1.880939in}}{\pgfqpoint{1.277252in}{1.888839in}}{\pgfqpoint{1.271428in}{1.894663in}}%
\pgfpathcurveto{\pgfqpoint{1.265604in}{1.900487in}}{\pgfqpoint{1.257704in}{1.903759in}}{\pgfqpoint{1.249468in}{1.903759in}}%
\pgfpathcurveto{\pgfqpoint{1.241231in}{1.903759in}}{\pgfqpoint{1.233331in}{1.900487in}}{\pgfqpoint{1.227507in}{1.894663in}}%
\pgfpathcurveto{\pgfqpoint{1.221683in}{1.888839in}}{\pgfqpoint{1.218411in}{1.880939in}}{\pgfqpoint{1.218411in}{1.872702in}}%
\pgfpathcurveto{\pgfqpoint{1.218411in}{1.864466in}}{\pgfqpoint{1.221683in}{1.856566in}}{\pgfqpoint{1.227507in}{1.850742in}}%
\pgfpathcurveto{\pgfqpoint{1.233331in}{1.844918in}}{\pgfqpoint{1.241231in}{1.841646in}}{\pgfqpoint{1.249468in}{1.841646in}}%
\pgfpathclose%
\pgfusepath{stroke,fill}%
\end{pgfscope}%
\begin{pgfscope}%
\pgfpathrectangle{\pgfqpoint{0.100000in}{0.212622in}}{\pgfqpoint{3.696000in}{3.696000in}}%
\pgfusepath{clip}%
\pgfsetbuttcap%
\pgfsetroundjoin%
\definecolor{currentfill}{rgb}{0.121569,0.466667,0.705882}%
\pgfsetfillcolor{currentfill}%
\pgfsetfillopacity{0.850824}%
\pgfsetlinewidth{1.003750pt}%
\definecolor{currentstroke}{rgb}{0.121569,0.466667,0.705882}%
\pgfsetstrokecolor{currentstroke}%
\pgfsetstrokeopacity{0.850824}%
\pgfsetdash{}{0pt}%
\pgfpathmoveto{\pgfqpoint{3.074137in}{2.650199in}}%
\pgfpathcurveto{\pgfqpoint{3.082373in}{2.650199in}}{\pgfqpoint{3.090273in}{2.653472in}}{\pgfqpoint{3.096097in}{2.659296in}}%
\pgfpathcurveto{\pgfqpoint{3.101921in}{2.665120in}}{\pgfqpoint{3.105193in}{2.673020in}}{\pgfqpoint{3.105193in}{2.681256in}}%
\pgfpathcurveto{\pgfqpoint{3.105193in}{2.689492in}}{\pgfqpoint{3.101921in}{2.697392in}}{\pgfqpoint{3.096097in}{2.703216in}}%
\pgfpathcurveto{\pgfqpoint{3.090273in}{2.709040in}}{\pgfqpoint{3.082373in}{2.712312in}}{\pgfqpoint{3.074137in}{2.712312in}}%
\pgfpathcurveto{\pgfqpoint{3.065901in}{2.712312in}}{\pgfqpoint{3.058001in}{2.709040in}}{\pgfqpoint{3.052177in}{2.703216in}}%
\pgfpathcurveto{\pgfqpoint{3.046353in}{2.697392in}}{\pgfqpoint{3.043080in}{2.689492in}}{\pgfqpoint{3.043080in}{2.681256in}}%
\pgfpathcurveto{\pgfqpoint{3.043080in}{2.673020in}}{\pgfqpoint{3.046353in}{2.665120in}}{\pgfqpoint{3.052177in}{2.659296in}}%
\pgfpathcurveto{\pgfqpoint{3.058001in}{2.653472in}}{\pgfqpoint{3.065901in}{2.650199in}}{\pgfqpoint{3.074137in}{2.650199in}}%
\pgfpathclose%
\pgfusepath{stroke,fill}%
\end{pgfscope}%
\begin{pgfscope}%
\pgfpathrectangle{\pgfqpoint{0.100000in}{0.212622in}}{\pgfqpoint{3.696000in}{3.696000in}}%
\pgfusepath{clip}%
\pgfsetbuttcap%
\pgfsetroundjoin%
\definecolor{currentfill}{rgb}{0.121569,0.466667,0.705882}%
\pgfsetfillcolor{currentfill}%
\pgfsetfillopacity{0.850921}%
\pgfsetlinewidth{1.003750pt}%
\definecolor{currentstroke}{rgb}{0.121569,0.466667,0.705882}%
\pgfsetstrokecolor{currentstroke}%
\pgfsetstrokeopacity{0.850921}%
\pgfsetdash{}{0pt}%
\pgfpathmoveto{\pgfqpoint{2.954582in}{2.564574in}}%
\pgfpathcurveto{\pgfqpoint{2.962818in}{2.564574in}}{\pgfqpoint{2.970718in}{2.567846in}}{\pgfqpoint{2.976542in}{2.573670in}}%
\pgfpathcurveto{\pgfqpoint{2.982366in}{2.579494in}}{\pgfqpoint{2.985638in}{2.587394in}}{\pgfqpoint{2.985638in}{2.595631in}}%
\pgfpathcurveto{\pgfqpoint{2.985638in}{2.603867in}}{\pgfqpoint{2.982366in}{2.611767in}}{\pgfqpoint{2.976542in}{2.617591in}}%
\pgfpathcurveto{\pgfqpoint{2.970718in}{2.623415in}}{\pgfqpoint{2.962818in}{2.626687in}}{\pgfqpoint{2.954582in}{2.626687in}}%
\pgfpathcurveto{\pgfqpoint{2.946345in}{2.626687in}}{\pgfqpoint{2.938445in}{2.623415in}}{\pgfqpoint{2.932621in}{2.617591in}}%
\pgfpathcurveto{\pgfqpoint{2.926797in}{2.611767in}}{\pgfqpoint{2.923525in}{2.603867in}}{\pgfqpoint{2.923525in}{2.595631in}}%
\pgfpathcurveto{\pgfqpoint{2.923525in}{2.587394in}}{\pgfqpoint{2.926797in}{2.579494in}}{\pgfqpoint{2.932621in}{2.573670in}}%
\pgfpathcurveto{\pgfqpoint{2.938445in}{2.567846in}}{\pgfqpoint{2.946345in}{2.564574in}}{\pgfqpoint{2.954582in}{2.564574in}}%
\pgfpathclose%
\pgfusepath{stroke,fill}%
\end{pgfscope}%
\begin{pgfscope}%
\pgfpathrectangle{\pgfqpoint{0.100000in}{0.212622in}}{\pgfqpoint{3.696000in}{3.696000in}}%
\pgfusepath{clip}%
\pgfsetbuttcap%
\pgfsetroundjoin%
\definecolor{currentfill}{rgb}{0.121569,0.466667,0.705882}%
\pgfsetfillcolor{currentfill}%
\pgfsetfillopacity{0.851979}%
\pgfsetlinewidth{1.003750pt}%
\definecolor{currentstroke}{rgb}{0.121569,0.466667,0.705882}%
\pgfsetstrokecolor{currentstroke}%
\pgfsetstrokeopacity{0.851979}%
\pgfsetdash{}{0pt}%
\pgfpathmoveto{\pgfqpoint{2.983232in}{2.603711in}}%
\pgfpathcurveto{\pgfqpoint{2.991468in}{2.603711in}}{\pgfqpoint{2.999368in}{2.606983in}}{\pgfqpoint{3.005192in}{2.612807in}}%
\pgfpathcurveto{\pgfqpoint{3.011016in}{2.618631in}}{\pgfqpoint{3.014288in}{2.626531in}}{\pgfqpoint{3.014288in}{2.634767in}}%
\pgfpathcurveto{\pgfqpoint{3.014288in}{2.643004in}}{\pgfqpoint{3.011016in}{2.650904in}}{\pgfqpoint{3.005192in}{2.656728in}}%
\pgfpathcurveto{\pgfqpoint{2.999368in}{2.662552in}}{\pgfqpoint{2.991468in}{2.665824in}}{\pgfqpoint{2.983232in}{2.665824in}}%
\pgfpathcurveto{\pgfqpoint{2.974995in}{2.665824in}}{\pgfqpoint{2.967095in}{2.662552in}}{\pgfqpoint{2.961271in}{2.656728in}}%
\pgfpathcurveto{\pgfqpoint{2.955448in}{2.650904in}}{\pgfqpoint{2.952175in}{2.643004in}}{\pgfqpoint{2.952175in}{2.634767in}}%
\pgfpathcurveto{\pgfqpoint{2.952175in}{2.626531in}}{\pgfqpoint{2.955448in}{2.618631in}}{\pgfqpoint{2.961271in}{2.612807in}}%
\pgfpathcurveto{\pgfqpoint{2.967095in}{2.606983in}}{\pgfqpoint{2.974995in}{2.603711in}}{\pgfqpoint{2.983232in}{2.603711in}}%
\pgfpathclose%
\pgfusepath{stroke,fill}%
\end{pgfscope}%
\begin{pgfscope}%
\pgfpathrectangle{\pgfqpoint{0.100000in}{0.212622in}}{\pgfqpoint{3.696000in}{3.696000in}}%
\pgfusepath{clip}%
\pgfsetbuttcap%
\pgfsetroundjoin%
\definecolor{currentfill}{rgb}{0.121569,0.466667,0.705882}%
\pgfsetfillcolor{currentfill}%
\pgfsetfillopacity{0.852669}%
\pgfsetlinewidth{1.003750pt}%
\definecolor{currentstroke}{rgb}{0.121569,0.466667,0.705882}%
\pgfsetstrokecolor{currentstroke}%
\pgfsetstrokeopacity{0.852669}%
\pgfsetdash{}{0pt}%
\pgfpathmoveto{\pgfqpoint{2.993156in}{2.614054in}}%
\pgfpathcurveto{\pgfqpoint{3.001392in}{2.614054in}}{\pgfqpoint{3.009292in}{2.617326in}}{\pgfqpoint{3.015116in}{2.623150in}}%
\pgfpathcurveto{\pgfqpoint{3.020940in}{2.628974in}}{\pgfqpoint{3.024212in}{2.636874in}}{\pgfqpoint{3.024212in}{2.645110in}}%
\pgfpathcurveto{\pgfqpoint{3.024212in}{2.653347in}}{\pgfqpoint{3.020940in}{2.661247in}}{\pgfqpoint{3.015116in}{2.667071in}}%
\pgfpathcurveto{\pgfqpoint{3.009292in}{2.672894in}}{\pgfqpoint{3.001392in}{2.676167in}}{\pgfqpoint{2.993156in}{2.676167in}}%
\pgfpathcurveto{\pgfqpoint{2.984919in}{2.676167in}}{\pgfqpoint{2.977019in}{2.672894in}}{\pgfqpoint{2.971195in}{2.667071in}}%
\pgfpathcurveto{\pgfqpoint{2.965372in}{2.661247in}}{\pgfqpoint{2.962099in}{2.653347in}}{\pgfqpoint{2.962099in}{2.645110in}}%
\pgfpathcurveto{\pgfqpoint{2.962099in}{2.636874in}}{\pgfqpoint{2.965372in}{2.628974in}}{\pgfqpoint{2.971195in}{2.623150in}}%
\pgfpathcurveto{\pgfqpoint{2.977019in}{2.617326in}}{\pgfqpoint{2.984919in}{2.614054in}}{\pgfqpoint{2.993156in}{2.614054in}}%
\pgfpathclose%
\pgfusepath{stroke,fill}%
\end{pgfscope}%
\begin{pgfscope}%
\pgfpathrectangle{\pgfqpoint{0.100000in}{0.212622in}}{\pgfqpoint{3.696000in}{3.696000in}}%
\pgfusepath{clip}%
\pgfsetbuttcap%
\pgfsetroundjoin%
\definecolor{currentfill}{rgb}{0.121569,0.466667,0.705882}%
\pgfsetfillcolor{currentfill}%
\pgfsetfillopacity{0.854484}%
\pgfsetlinewidth{1.003750pt}%
\definecolor{currentstroke}{rgb}{0.121569,0.466667,0.705882}%
\pgfsetstrokecolor{currentstroke}%
\pgfsetstrokeopacity{0.854484}%
\pgfsetdash{}{0pt}%
\pgfpathmoveto{\pgfqpoint{3.085212in}{2.651936in}}%
\pgfpathcurveto{\pgfqpoint{3.093448in}{2.651936in}}{\pgfqpoint{3.101348in}{2.655208in}}{\pgfqpoint{3.107172in}{2.661032in}}%
\pgfpathcurveto{\pgfqpoint{3.112996in}{2.666856in}}{\pgfqpoint{3.116268in}{2.674756in}}{\pgfqpoint{3.116268in}{2.682992in}}%
\pgfpathcurveto{\pgfqpoint{3.116268in}{2.691228in}}{\pgfqpoint{3.112996in}{2.699128in}}{\pgfqpoint{3.107172in}{2.704952in}}%
\pgfpathcurveto{\pgfqpoint{3.101348in}{2.710776in}}{\pgfqpoint{3.093448in}{2.714049in}}{\pgfqpoint{3.085212in}{2.714049in}}%
\pgfpathcurveto{\pgfqpoint{3.076975in}{2.714049in}}{\pgfqpoint{3.069075in}{2.710776in}}{\pgfqpoint{3.063251in}{2.704952in}}%
\pgfpathcurveto{\pgfqpoint{3.057427in}{2.699128in}}{\pgfqpoint{3.054155in}{2.691228in}}{\pgfqpoint{3.054155in}{2.682992in}}%
\pgfpathcurveto{\pgfqpoint{3.054155in}{2.674756in}}{\pgfqpoint{3.057427in}{2.666856in}}{\pgfqpoint{3.063251in}{2.661032in}}%
\pgfpathcurveto{\pgfqpoint{3.069075in}{2.655208in}}{\pgfqpoint{3.076975in}{2.651936in}}{\pgfqpoint{3.085212in}{2.651936in}}%
\pgfpathclose%
\pgfusepath{stroke,fill}%
\end{pgfscope}%
\begin{pgfscope}%
\pgfpathrectangle{\pgfqpoint{0.100000in}{0.212622in}}{\pgfqpoint{3.696000in}{3.696000in}}%
\pgfusepath{clip}%
\pgfsetbuttcap%
\pgfsetroundjoin%
\definecolor{currentfill}{rgb}{0.121569,0.466667,0.705882}%
\pgfsetfillcolor{currentfill}%
\pgfsetfillopacity{0.854587}%
\pgfsetlinewidth{1.003750pt}%
\definecolor{currentstroke}{rgb}{0.121569,0.466667,0.705882}%
\pgfsetstrokecolor{currentstroke}%
\pgfsetstrokeopacity{0.854587}%
\pgfsetdash{}{0pt}%
\pgfpathmoveto{\pgfqpoint{2.967948in}{2.583863in}}%
\pgfpathcurveto{\pgfqpoint{2.976184in}{2.583863in}}{\pgfqpoint{2.984084in}{2.587135in}}{\pgfqpoint{2.989908in}{2.592959in}}%
\pgfpathcurveto{\pgfqpoint{2.995732in}{2.598783in}}{\pgfqpoint{2.999005in}{2.606683in}}{\pgfqpoint{2.999005in}{2.614919in}}%
\pgfpathcurveto{\pgfqpoint{2.999005in}{2.623156in}}{\pgfqpoint{2.995732in}{2.631056in}}{\pgfqpoint{2.989908in}{2.636880in}}%
\pgfpathcurveto{\pgfqpoint{2.984084in}{2.642703in}}{\pgfqpoint{2.976184in}{2.645976in}}{\pgfqpoint{2.967948in}{2.645976in}}%
\pgfpathcurveto{\pgfqpoint{2.959712in}{2.645976in}}{\pgfqpoint{2.951812in}{2.642703in}}{\pgfqpoint{2.945988in}{2.636880in}}%
\pgfpathcurveto{\pgfqpoint{2.940164in}{2.631056in}}{\pgfqpoint{2.936892in}{2.623156in}}{\pgfqpoint{2.936892in}{2.614919in}}%
\pgfpathcurveto{\pgfqpoint{2.936892in}{2.606683in}}{\pgfqpoint{2.940164in}{2.598783in}}{\pgfqpoint{2.945988in}{2.592959in}}%
\pgfpathcurveto{\pgfqpoint{2.951812in}{2.587135in}}{\pgfqpoint{2.959712in}{2.583863in}}{\pgfqpoint{2.967948in}{2.583863in}}%
\pgfpathclose%
\pgfusepath{stroke,fill}%
\end{pgfscope}%
\begin{pgfscope}%
\pgfpathrectangle{\pgfqpoint{0.100000in}{0.212622in}}{\pgfqpoint{3.696000in}{3.696000in}}%
\pgfusepath{clip}%
\pgfsetbuttcap%
\pgfsetroundjoin%
\definecolor{currentfill}{rgb}{0.121569,0.466667,0.705882}%
\pgfsetfillcolor{currentfill}%
\pgfsetfillopacity{0.854878}%
\pgfsetlinewidth{1.003750pt}%
\definecolor{currentstroke}{rgb}{0.121569,0.466667,0.705882}%
\pgfsetstrokecolor{currentstroke}%
\pgfsetstrokeopacity{0.854878}%
\pgfsetdash{}{0pt}%
\pgfpathmoveto{\pgfqpoint{2.802634in}{2.532641in}}%
\pgfpathcurveto{\pgfqpoint{2.810870in}{2.532641in}}{\pgfqpoint{2.818770in}{2.535913in}}{\pgfqpoint{2.824594in}{2.541737in}}%
\pgfpathcurveto{\pgfqpoint{2.830418in}{2.547561in}}{\pgfqpoint{2.833691in}{2.555461in}}{\pgfqpoint{2.833691in}{2.563697in}}%
\pgfpathcurveto{\pgfqpoint{2.833691in}{2.571933in}}{\pgfqpoint{2.830418in}{2.579834in}}{\pgfqpoint{2.824594in}{2.585657in}}%
\pgfpathcurveto{\pgfqpoint{2.818770in}{2.591481in}}{\pgfqpoint{2.810870in}{2.594754in}}{\pgfqpoint{2.802634in}{2.594754in}}%
\pgfpathcurveto{\pgfqpoint{2.794398in}{2.594754in}}{\pgfqpoint{2.786498in}{2.591481in}}{\pgfqpoint{2.780674in}{2.585657in}}%
\pgfpathcurveto{\pgfqpoint{2.774850in}{2.579834in}}{\pgfqpoint{2.771578in}{2.571933in}}{\pgfqpoint{2.771578in}{2.563697in}}%
\pgfpathcurveto{\pgfqpoint{2.771578in}{2.555461in}}{\pgfqpoint{2.774850in}{2.547561in}}{\pgfqpoint{2.780674in}{2.541737in}}%
\pgfpathcurveto{\pgfqpoint{2.786498in}{2.535913in}}{\pgfqpoint{2.794398in}{2.532641in}}{\pgfqpoint{2.802634in}{2.532641in}}%
\pgfpathclose%
\pgfusepath{stroke,fill}%
\end{pgfscope}%
\begin{pgfscope}%
\pgfpathrectangle{\pgfqpoint{0.100000in}{0.212622in}}{\pgfqpoint{3.696000in}{3.696000in}}%
\pgfusepath{clip}%
\pgfsetbuttcap%
\pgfsetroundjoin%
\definecolor{currentfill}{rgb}{0.121569,0.466667,0.705882}%
\pgfsetfillcolor{currentfill}%
\pgfsetfillopacity{0.855868}%
\pgfsetlinewidth{1.003750pt}%
\definecolor{currentstroke}{rgb}{0.121569,0.466667,0.705882}%
\pgfsetstrokecolor{currentstroke}%
\pgfsetstrokeopacity{0.855868}%
\pgfsetdash{}{0pt}%
\pgfpathmoveto{\pgfqpoint{2.991095in}{2.601419in}}%
\pgfpathcurveto{\pgfqpoint{2.999332in}{2.601419in}}{\pgfqpoint{3.007232in}{2.604692in}}{\pgfqpoint{3.013055in}{2.610516in}}%
\pgfpathcurveto{\pgfqpoint{3.018879in}{2.616340in}}{\pgfqpoint{3.022152in}{2.624240in}}{\pgfqpoint{3.022152in}{2.632476in}}%
\pgfpathcurveto{\pgfqpoint{3.022152in}{2.640712in}}{\pgfqpoint{3.018879in}{2.648612in}}{\pgfqpoint{3.013055in}{2.654436in}}%
\pgfpathcurveto{\pgfqpoint{3.007232in}{2.660260in}}{\pgfqpoint{2.999332in}{2.663532in}}{\pgfqpoint{2.991095in}{2.663532in}}%
\pgfpathcurveto{\pgfqpoint{2.982859in}{2.663532in}}{\pgfqpoint{2.974959in}{2.660260in}}{\pgfqpoint{2.969135in}{2.654436in}}%
\pgfpathcurveto{\pgfqpoint{2.963311in}{2.648612in}}{\pgfqpoint{2.960039in}{2.640712in}}{\pgfqpoint{2.960039in}{2.632476in}}%
\pgfpathcurveto{\pgfqpoint{2.960039in}{2.624240in}}{\pgfqpoint{2.963311in}{2.616340in}}{\pgfqpoint{2.969135in}{2.610516in}}%
\pgfpathcurveto{\pgfqpoint{2.974959in}{2.604692in}}{\pgfqpoint{2.982859in}{2.601419in}}{\pgfqpoint{2.991095in}{2.601419in}}%
\pgfpathclose%
\pgfusepath{stroke,fill}%
\end{pgfscope}%
\begin{pgfscope}%
\pgfpathrectangle{\pgfqpoint{0.100000in}{0.212622in}}{\pgfqpoint{3.696000in}{3.696000in}}%
\pgfusepath{clip}%
\pgfsetbuttcap%
\pgfsetroundjoin%
\definecolor{currentfill}{rgb}{0.121569,0.466667,0.705882}%
\pgfsetfillcolor{currentfill}%
\pgfsetfillopacity{0.856450}%
\pgfsetlinewidth{1.003750pt}%
\definecolor{currentstroke}{rgb}{0.121569,0.466667,0.705882}%
\pgfsetstrokecolor{currentstroke}%
\pgfsetstrokeopacity{0.856450}%
\pgfsetdash{}{0pt}%
\pgfpathmoveto{\pgfqpoint{3.089299in}{2.651924in}}%
\pgfpathcurveto{\pgfqpoint{3.097535in}{2.651924in}}{\pgfqpoint{3.105435in}{2.655196in}}{\pgfqpoint{3.111259in}{2.661020in}}%
\pgfpathcurveto{\pgfqpoint{3.117083in}{2.666844in}}{\pgfqpoint{3.120355in}{2.674744in}}{\pgfqpoint{3.120355in}{2.682980in}}%
\pgfpathcurveto{\pgfqpoint{3.120355in}{2.691217in}}{\pgfqpoint{3.117083in}{2.699117in}}{\pgfqpoint{3.111259in}{2.704941in}}%
\pgfpathcurveto{\pgfqpoint{3.105435in}{2.710765in}}{\pgfqpoint{3.097535in}{2.714037in}}{\pgfqpoint{3.089299in}{2.714037in}}%
\pgfpathcurveto{\pgfqpoint{3.081063in}{2.714037in}}{\pgfqpoint{3.073163in}{2.710765in}}{\pgfqpoint{3.067339in}{2.704941in}}%
\pgfpathcurveto{\pgfqpoint{3.061515in}{2.699117in}}{\pgfqpoint{3.058242in}{2.691217in}}{\pgfqpoint{3.058242in}{2.682980in}}%
\pgfpathcurveto{\pgfqpoint{3.058242in}{2.674744in}}{\pgfqpoint{3.061515in}{2.666844in}}{\pgfqpoint{3.067339in}{2.661020in}}%
\pgfpathcurveto{\pgfqpoint{3.073163in}{2.655196in}}{\pgfqpoint{3.081063in}{2.651924in}}{\pgfqpoint{3.089299in}{2.651924in}}%
\pgfpathclose%
\pgfusepath{stroke,fill}%
\end{pgfscope}%
\begin{pgfscope}%
\pgfpathrectangle{\pgfqpoint{0.100000in}{0.212622in}}{\pgfqpoint{3.696000in}{3.696000in}}%
\pgfusepath{clip}%
\pgfsetbuttcap%
\pgfsetroundjoin%
\definecolor{currentfill}{rgb}{0.121569,0.466667,0.705882}%
\pgfsetfillcolor{currentfill}%
\pgfsetfillopacity{0.856746}%
\pgfsetlinewidth{1.003750pt}%
\definecolor{currentstroke}{rgb}{0.121569,0.466667,0.705882}%
\pgfsetstrokecolor{currentstroke}%
\pgfsetstrokeopacity{0.856746}%
\pgfsetdash{}{0pt}%
\pgfpathmoveto{\pgfqpoint{3.069991in}{2.646194in}}%
\pgfpathcurveto{\pgfqpoint{3.078227in}{2.646194in}}{\pgfqpoint{3.086127in}{2.649467in}}{\pgfqpoint{3.091951in}{2.655291in}}%
\pgfpathcurveto{\pgfqpoint{3.097775in}{2.661115in}}{\pgfqpoint{3.101047in}{2.669015in}}{\pgfqpoint{3.101047in}{2.677251in}}%
\pgfpathcurveto{\pgfqpoint{3.101047in}{2.685487in}}{\pgfqpoint{3.097775in}{2.693387in}}{\pgfqpoint{3.091951in}{2.699211in}}%
\pgfpathcurveto{\pgfqpoint{3.086127in}{2.705035in}}{\pgfqpoint{3.078227in}{2.708307in}}{\pgfqpoint{3.069991in}{2.708307in}}%
\pgfpathcurveto{\pgfqpoint{3.061755in}{2.708307in}}{\pgfqpoint{3.053854in}{2.705035in}}{\pgfqpoint{3.048031in}{2.699211in}}%
\pgfpathcurveto{\pgfqpoint{3.042207in}{2.693387in}}{\pgfqpoint{3.038934in}{2.685487in}}{\pgfqpoint{3.038934in}{2.677251in}}%
\pgfpathcurveto{\pgfqpoint{3.038934in}{2.669015in}}{\pgfqpoint{3.042207in}{2.661115in}}{\pgfqpoint{3.048031in}{2.655291in}}%
\pgfpathcurveto{\pgfqpoint{3.053854in}{2.649467in}}{\pgfqpoint{3.061755in}{2.646194in}}{\pgfqpoint{3.069991in}{2.646194in}}%
\pgfpathclose%
\pgfusepath{stroke,fill}%
\end{pgfscope}%
\begin{pgfscope}%
\pgfpathrectangle{\pgfqpoint{0.100000in}{0.212622in}}{\pgfqpoint{3.696000in}{3.696000in}}%
\pgfusepath{clip}%
\pgfsetbuttcap%
\pgfsetroundjoin%
\definecolor{currentfill}{rgb}{0.121569,0.466667,0.705882}%
\pgfsetfillcolor{currentfill}%
\pgfsetfillopacity{0.856748}%
\pgfsetlinewidth{1.003750pt}%
\definecolor{currentstroke}{rgb}{0.121569,0.466667,0.705882}%
\pgfsetstrokecolor{currentstroke}%
\pgfsetstrokeopacity{0.856748}%
\pgfsetdash{}{0pt}%
\pgfpathmoveto{\pgfqpoint{2.975996in}{2.593567in}}%
\pgfpathcurveto{\pgfqpoint{2.984232in}{2.593567in}}{\pgfqpoint{2.992132in}{2.596839in}}{\pgfqpoint{2.997956in}{2.602663in}}%
\pgfpathcurveto{\pgfqpoint{3.003780in}{2.608487in}}{\pgfqpoint{3.007053in}{2.616387in}}{\pgfqpoint{3.007053in}{2.624623in}}%
\pgfpathcurveto{\pgfqpoint{3.007053in}{2.632860in}}{\pgfqpoint{3.003780in}{2.640760in}}{\pgfqpoint{2.997956in}{2.646584in}}%
\pgfpathcurveto{\pgfqpoint{2.992132in}{2.652408in}}{\pgfqpoint{2.984232in}{2.655680in}}{\pgfqpoint{2.975996in}{2.655680in}}%
\pgfpathcurveto{\pgfqpoint{2.967760in}{2.655680in}}{\pgfqpoint{2.959860in}{2.652408in}}{\pgfqpoint{2.954036in}{2.646584in}}%
\pgfpathcurveto{\pgfqpoint{2.948212in}{2.640760in}}{\pgfqpoint{2.944940in}{2.632860in}}{\pgfqpoint{2.944940in}{2.624623in}}%
\pgfpathcurveto{\pgfqpoint{2.944940in}{2.616387in}}{\pgfqpoint{2.948212in}{2.608487in}}{\pgfqpoint{2.954036in}{2.602663in}}%
\pgfpathcurveto{\pgfqpoint{2.959860in}{2.596839in}}{\pgfqpoint{2.967760in}{2.593567in}}{\pgfqpoint{2.975996in}{2.593567in}}%
\pgfpathclose%
\pgfusepath{stroke,fill}%
\end{pgfscope}%
\begin{pgfscope}%
\pgfpathrectangle{\pgfqpoint{0.100000in}{0.212622in}}{\pgfqpoint{3.696000in}{3.696000in}}%
\pgfusepath{clip}%
\pgfsetbuttcap%
\pgfsetroundjoin%
\definecolor{currentfill}{rgb}{0.121569,0.466667,0.705882}%
\pgfsetfillcolor{currentfill}%
\pgfsetfillopacity{0.857378}%
\pgfsetlinewidth{1.003750pt}%
\definecolor{currentstroke}{rgb}{0.121569,0.466667,0.705882}%
\pgfsetstrokecolor{currentstroke}%
\pgfsetstrokeopacity{0.857378}%
\pgfsetdash{}{0pt}%
\pgfpathmoveto{\pgfqpoint{1.233584in}{1.820556in}}%
\pgfpathcurveto{\pgfqpoint{1.241820in}{1.820556in}}{\pgfqpoint{1.249720in}{1.823829in}}{\pgfqpoint{1.255544in}{1.829652in}}%
\pgfpathcurveto{\pgfqpoint{1.261368in}{1.835476in}}{\pgfqpoint{1.264640in}{1.843376in}}{\pgfqpoint{1.264640in}{1.851613in}}%
\pgfpathcurveto{\pgfqpoint{1.264640in}{1.859849in}}{\pgfqpoint{1.261368in}{1.867749in}}{\pgfqpoint{1.255544in}{1.873573in}}%
\pgfpathcurveto{\pgfqpoint{1.249720in}{1.879397in}}{\pgfqpoint{1.241820in}{1.882669in}}{\pgfqpoint{1.233584in}{1.882669in}}%
\pgfpathcurveto{\pgfqpoint{1.225347in}{1.882669in}}{\pgfqpoint{1.217447in}{1.879397in}}{\pgfqpoint{1.211623in}{1.873573in}}%
\pgfpathcurveto{\pgfqpoint{1.205800in}{1.867749in}}{\pgfqpoint{1.202527in}{1.859849in}}{\pgfqpoint{1.202527in}{1.851613in}}%
\pgfpathcurveto{\pgfqpoint{1.202527in}{1.843376in}}{\pgfqpoint{1.205800in}{1.835476in}}{\pgfqpoint{1.211623in}{1.829652in}}%
\pgfpathcurveto{\pgfqpoint{1.217447in}{1.823829in}}{\pgfqpoint{1.225347in}{1.820556in}}{\pgfqpoint{1.233584in}{1.820556in}}%
\pgfpathclose%
\pgfusepath{stroke,fill}%
\end{pgfscope}%
\begin{pgfscope}%
\pgfpathrectangle{\pgfqpoint{0.100000in}{0.212622in}}{\pgfqpoint{3.696000in}{3.696000in}}%
\pgfusepath{clip}%
\pgfsetbuttcap%
\pgfsetroundjoin%
\definecolor{currentfill}{rgb}{0.121569,0.466667,0.705882}%
\pgfsetfillcolor{currentfill}%
\pgfsetfillopacity{0.857525}%
\pgfsetlinewidth{1.003750pt}%
\definecolor{currentstroke}{rgb}{0.121569,0.466667,0.705882}%
\pgfsetstrokecolor{currentstroke}%
\pgfsetstrokeopacity{0.857525}%
\pgfsetdash{}{0pt}%
\pgfpathmoveto{\pgfqpoint{1.235977in}{1.822571in}}%
\pgfpathcurveto{\pgfqpoint{1.244213in}{1.822571in}}{\pgfqpoint{1.252113in}{1.825843in}}{\pgfqpoint{1.257937in}{1.831667in}}%
\pgfpathcurveto{\pgfqpoint{1.263761in}{1.837491in}}{\pgfqpoint{1.267033in}{1.845391in}}{\pgfqpoint{1.267033in}{1.853628in}}%
\pgfpathcurveto{\pgfqpoint{1.267033in}{1.861864in}}{\pgfqpoint{1.263761in}{1.869764in}}{\pgfqpoint{1.257937in}{1.875588in}}%
\pgfpathcurveto{\pgfqpoint{1.252113in}{1.881412in}}{\pgfqpoint{1.244213in}{1.884684in}}{\pgfqpoint{1.235977in}{1.884684in}}%
\pgfpathcurveto{\pgfqpoint{1.227740in}{1.884684in}}{\pgfqpoint{1.219840in}{1.881412in}}{\pgfqpoint{1.214016in}{1.875588in}}%
\pgfpathcurveto{\pgfqpoint{1.208192in}{1.869764in}}{\pgfqpoint{1.204920in}{1.861864in}}{\pgfqpoint{1.204920in}{1.853628in}}%
\pgfpathcurveto{\pgfqpoint{1.204920in}{1.845391in}}{\pgfqpoint{1.208192in}{1.837491in}}{\pgfqpoint{1.214016in}{1.831667in}}%
\pgfpathcurveto{\pgfqpoint{1.219840in}{1.825843in}}{\pgfqpoint{1.227740in}{1.822571in}}{\pgfqpoint{1.235977in}{1.822571in}}%
\pgfpathclose%
\pgfusepath{stroke,fill}%
\end{pgfscope}%
\begin{pgfscope}%
\pgfpathrectangle{\pgfqpoint{0.100000in}{0.212622in}}{\pgfqpoint{3.696000in}{3.696000in}}%
\pgfusepath{clip}%
\pgfsetbuttcap%
\pgfsetroundjoin%
\definecolor{currentfill}{rgb}{0.121569,0.466667,0.705882}%
\pgfsetfillcolor{currentfill}%
\pgfsetfillopacity{0.858360}%
\pgfsetlinewidth{1.003750pt}%
\definecolor{currentstroke}{rgb}{0.121569,0.466667,0.705882}%
\pgfsetstrokecolor{currentstroke}%
\pgfsetstrokeopacity{0.858360}%
\pgfsetdash{}{0pt}%
\pgfpathmoveto{\pgfqpoint{2.825966in}{2.499010in}}%
\pgfpathcurveto{\pgfqpoint{2.834202in}{2.499010in}}{\pgfqpoint{2.842102in}{2.502283in}}{\pgfqpoint{2.847926in}{2.508107in}}%
\pgfpathcurveto{\pgfqpoint{2.853750in}{2.513931in}}{\pgfqpoint{2.857022in}{2.521831in}}{\pgfqpoint{2.857022in}{2.530067in}}%
\pgfpathcurveto{\pgfqpoint{2.857022in}{2.538303in}}{\pgfqpoint{2.853750in}{2.546203in}}{\pgfqpoint{2.847926in}{2.552027in}}%
\pgfpathcurveto{\pgfqpoint{2.842102in}{2.557851in}}{\pgfqpoint{2.834202in}{2.561123in}}{\pgfqpoint{2.825966in}{2.561123in}}%
\pgfpathcurveto{\pgfqpoint{2.817730in}{2.561123in}}{\pgfqpoint{2.809830in}{2.557851in}}{\pgfqpoint{2.804006in}{2.552027in}}%
\pgfpathcurveto{\pgfqpoint{2.798182in}{2.546203in}}{\pgfqpoint{2.794909in}{2.538303in}}{\pgfqpoint{2.794909in}{2.530067in}}%
\pgfpathcurveto{\pgfqpoint{2.794909in}{2.521831in}}{\pgfqpoint{2.798182in}{2.513931in}}{\pgfqpoint{2.804006in}{2.508107in}}%
\pgfpathcurveto{\pgfqpoint{2.809830in}{2.502283in}}{\pgfqpoint{2.817730in}{2.499010in}}{\pgfqpoint{2.825966in}{2.499010in}}%
\pgfpathclose%
\pgfusepath{stroke,fill}%
\end{pgfscope}%
\begin{pgfscope}%
\pgfpathrectangle{\pgfqpoint{0.100000in}{0.212622in}}{\pgfqpoint{3.696000in}{3.696000in}}%
\pgfusepath{clip}%
\pgfsetbuttcap%
\pgfsetroundjoin%
\definecolor{currentfill}{rgb}{0.121569,0.466667,0.705882}%
\pgfsetfillcolor{currentfill}%
\pgfsetfillopacity{0.858664}%
\pgfsetlinewidth{1.003750pt}%
\definecolor{currentstroke}{rgb}{0.121569,0.466667,0.705882}%
\pgfsetstrokecolor{currentstroke}%
\pgfsetstrokeopacity{0.858664}%
\pgfsetdash{}{0pt}%
\pgfpathmoveto{\pgfqpoint{3.047508in}{2.615598in}}%
\pgfpathcurveto{\pgfqpoint{3.055744in}{2.615598in}}{\pgfqpoint{3.063644in}{2.618870in}}{\pgfqpoint{3.069468in}{2.624694in}}%
\pgfpathcurveto{\pgfqpoint{3.075292in}{2.630518in}}{\pgfqpoint{3.078564in}{2.638418in}}{\pgfqpoint{3.078564in}{2.646654in}}%
\pgfpathcurveto{\pgfqpoint{3.078564in}{2.654891in}}{\pgfqpoint{3.075292in}{2.662791in}}{\pgfqpoint{3.069468in}{2.668615in}}%
\pgfpathcurveto{\pgfqpoint{3.063644in}{2.674439in}}{\pgfqpoint{3.055744in}{2.677711in}}{\pgfqpoint{3.047508in}{2.677711in}}%
\pgfpathcurveto{\pgfqpoint{3.039272in}{2.677711in}}{\pgfqpoint{3.031372in}{2.674439in}}{\pgfqpoint{3.025548in}{2.668615in}}%
\pgfpathcurveto{\pgfqpoint{3.019724in}{2.662791in}}{\pgfqpoint{3.016451in}{2.654891in}}{\pgfqpoint{3.016451in}{2.646654in}}%
\pgfpathcurveto{\pgfqpoint{3.016451in}{2.638418in}}{\pgfqpoint{3.019724in}{2.630518in}}{\pgfqpoint{3.025548in}{2.624694in}}%
\pgfpathcurveto{\pgfqpoint{3.031372in}{2.618870in}}{\pgfqpoint{3.039272in}{2.615598in}}{\pgfqpoint{3.047508in}{2.615598in}}%
\pgfpathclose%
\pgfusepath{stroke,fill}%
\end{pgfscope}%
\begin{pgfscope}%
\pgfpathrectangle{\pgfqpoint{0.100000in}{0.212622in}}{\pgfqpoint{3.696000in}{3.696000in}}%
\pgfusepath{clip}%
\pgfsetbuttcap%
\pgfsetroundjoin%
\definecolor{currentfill}{rgb}{0.121569,0.466667,0.705882}%
\pgfsetfillcolor{currentfill}%
\pgfsetfillopacity{0.858961}%
\pgfsetlinewidth{1.003750pt}%
\definecolor{currentstroke}{rgb}{0.121569,0.466667,0.705882}%
\pgfsetstrokecolor{currentstroke}%
\pgfsetstrokeopacity{0.858961}%
\pgfsetdash{}{0pt}%
\pgfpathmoveto{\pgfqpoint{3.059278in}{2.632984in}}%
\pgfpathcurveto{\pgfqpoint{3.067514in}{2.632984in}}{\pgfqpoint{3.075414in}{2.636257in}}{\pgfqpoint{3.081238in}{2.642081in}}%
\pgfpathcurveto{\pgfqpoint{3.087062in}{2.647905in}}{\pgfqpoint{3.090334in}{2.655805in}}{\pgfqpoint{3.090334in}{2.664041in}}%
\pgfpathcurveto{\pgfqpoint{3.090334in}{2.672277in}}{\pgfqpoint{3.087062in}{2.680177in}}{\pgfqpoint{3.081238in}{2.686001in}}%
\pgfpathcurveto{\pgfqpoint{3.075414in}{2.691825in}}{\pgfqpoint{3.067514in}{2.695097in}}{\pgfqpoint{3.059278in}{2.695097in}}%
\pgfpathcurveto{\pgfqpoint{3.051041in}{2.695097in}}{\pgfqpoint{3.043141in}{2.691825in}}{\pgfqpoint{3.037317in}{2.686001in}}%
\pgfpathcurveto{\pgfqpoint{3.031493in}{2.680177in}}{\pgfqpoint{3.028221in}{2.672277in}}{\pgfqpoint{3.028221in}{2.664041in}}%
\pgfpathcurveto{\pgfqpoint{3.028221in}{2.655805in}}{\pgfqpoint{3.031493in}{2.647905in}}{\pgfqpoint{3.037317in}{2.642081in}}%
\pgfpathcurveto{\pgfqpoint{3.043141in}{2.636257in}}{\pgfqpoint{3.051041in}{2.632984in}}{\pgfqpoint{3.059278in}{2.632984in}}%
\pgfpathclose%
\pgfusepath{stroke,fill}%
\end{pgfscope}%
\begin{pgfscope}%
\pgfpathrectangle{\pgfqpoint{0.100000in}{0.212622in}}{\pgfqpoint{3.696000in}{3.696000in}}%
\pgfusepath{clip}%
\pgfsetbuttcap%
\pgfsetroundjoin%
\definecolor{currentfill}{rgb}{0.121569,0.466667,0.705882}%
\pgfsetfillcolor{currentfill}%
\pgfsetfillopacity{0.860032}%
\pgfsetlinewidth{1.003750pt}%
\definecolor{currentstroke}{rgb}{0.121569,0.466667,0.705882}%
\pgfsetstrokecolor{currentstroke}%
\pgfsetstrokeopacity{0.860032}%
\pgfsetdash{}{0pt}%
\pgfpathmoveto{\pgfqpoint{3.065523in}{2.633653in}}%
\pgfpathcurveto{\pgfqpoint{3.073759in}{2.633653in}}{\pgfqpoint{3.081659in}{2.636925in}}{\pgfqpoint{3.087483in}{2.642749in}}%
\pgfpathcurveto{\pgfqpoint{3.093307in}{2.648573in}}{\pgfqpoint{3.096579in}{2.656473in}}{\pgfqpoint{3.096579in}{2.664709in}}%
\pgfpathcurveto{\pgfqpoint{3.096579in}{2.672945in}}{\pgfqpoint{3.093307in}{2.680845in}}{\pgfqpoint{3.087483in}{2.686669in}}%
\pgfpathcurveto{\pgfqpoint{3.081659in}{2.692493in}}{\pgfqpoint{3.073759in}{2.695766in}}{\pgfqpoint{3.065523in}{2.695766in}}%
\pgfpathcurveto{\pgfqpoint{3.057287in}{2.695766in}}{\pgfqpoint{3.049387in}{2.692493in}}{\pgfqpoint{3.043563in}{2.686669in}}%
\pgfpathcurveto{\pgfqpoint{3.037739in}{2.680845in}}{\pgfqpoint{3.034466in}{2.672945in}}{\pgfqpoint{3.034466in}{2.664709in}}%
\pgfpathcurveto{\pgfqpoint{3.034466in}{2.656473in}}{\pgfqpoint{3.037739in}{2.648573in}}{\pgfqpoint{3.043563in}{2.642749in}}%
\pgfpathcurveto{\pgfqpoint{3.049387in}{2.636925in}}{\pgfqpoint{3.057287in}{2.633653in}}{\pgfqpoint{3.065523in}{2.633653in}}%
\pgfpathclose%
\pgfusepath{stroke,fill}%
\end{pgfscope}%
\begin{pgfscope}%
\pgfpathrectangle{\pgfqpoint{0.100000in}{0.212622in}}{\pgfqpoint{3.696000in}{3.696000in}}%
\pgfusepath{clip}%
\pgfsetbuttcap%
\pgfsetroundjoin%
\definecolor{currentfill}{rgb}{0.121569,0.466667,0.705882}%
\pgfsetfillcolor{currentfill}%
\pgfsetfillopacity{0.860679}%
\pgfsetlinewidth{1.003750pt}%
\definecolor{currentstroke}{rgb}{0.121569,0.466667,0.705882}%
\pgfsetstrokecolor{currentstroke}%
\pgfsetstrokeopacity{0.860679}%
\pgfsetdash{}{0pt}%
\pgfpathmoveto{\pgfqpoint{3.088419in}{2.642562in}}%
\pgfpathcurveto{\pgfqpoint{3.096655in}{2.642562in}}{\pgfqpoint{3.104555in}{2.645835in}}{\pgfqpoint{3.110379in}{2.651659in}}%
\pgfpathcurveto{\pgfqpoint{3.116203in}{2.657483in}}{\pgfqpoint{3.119475in}{2.665383in}}{\pgfqpoint{3.119475in}{2.673619in}}%
\pgfpathcurveto{\pgfqpoint{3.119475in}{2.681855in}}{\pgfqpoint{3.116203in}{2.689755in}}{\pgfqpoint{3.110379in}{2.695579in}}%
\pgfpathcurveto{\pgfqpoint{3.104555in}{2.701403in}}{\pgfqpoint{3.096655in}{2.704675in}}{\pgfqpoint{3.088419in}{2.704675in}}%
\pgfpathcurveto{\pgfqpoint{3.080182in}{2.704675in}}{\pgfqpoint{3.072282in}{2.701403in}}{\pgfqpoint{3.066458in}{2.695579in}}%
\pgfpathcurveto{\pgfqpoint{3.060634in}{2.689755in}}{\pgfqpoint{3.057362in}{2.681855in}}{\pgfqpoint{3.057362in}{2.673619in}}%
\pgfpathcurveto{\pgfqpoint{3.057362in}{2.665383in}}{\pgfqpoint{3.060634in}{2.657483in}}{\pgfqpoint{3.066458in}{2.651659in}}%
\pgfpathcurveto{\pgfqpoint{3.072282in}{2.645835in}}{\pgfqpoint{3.080182in}{2.642562in}}{\pgfqpoint{3.088419in}{2.642562in}}%
\pgfpathclose%
\pgfusepath{stroke,fill}%
\end{pgfscope}%
\begin{pgfscope}%
\pgfpathrectangle{\pgfqpoint{0.100000in}{0.212622in}}{\pgfqpoint{3.696000in}{3.696000in}}%
\pgfusepath{clip}%
\pgfsetbuttcap%
\pgfsetroundjoin%
\definecolor{currentfill}{rgb}{0.121569,0.466667,0.705882}%
\pgfsetfillcolor{currentfill}%
\pgfsetfillopacity{0.861531}%
\pgfsetlinewidth{1.003750pt}%
\definecolor{currentstroke}{rgb}{0.121569,0.466667,0.705882}%
\pgfsetstrokecolor{currentstroke}%
\pgfsetstrokeopacity{0.861531}%
\pgfsetdash{}{0pt}%
\pgfpathmoveto{\pgfqpoint{3.097306in}{2.645479in}}%
\pgfpathcurveto{\pgfqpoint{3.105542in}{2.645479in}}{\pgfqpoint{3.113442in}{2.648751in}}{\pgfqpoint{3.119266in}{2.654575in}}%
\pgfpathcurveto{\pgfqpoint{3.125090in}{2.660399in}}{\pgfqpoint{3.128362in}{2.668299in}}{\pgfqpoint{3.128362in}{2.676536in}}%
\pgfpathcurveto{\pgfqpoint{3.128362in}{2.684772in}}{\pgfqpoint{3.125090in}{2.692672in}}{\pgfqpoint{3.119266in}{2.698496in}}%
\pgfpathcurveto{\pgfqpoint{3.113442in}{2.704320in}}{\pgfqpoint{3.105542in}{2.707592in}}{\pgfqpoint{3.097306in}{2.707592in}}%
\pgfpathcurveto{\pgfqpoint{3.089070in}{2.707592in}}{\pgfqpoint{3.081170in}{2.704320in}}{\pgfqpoint{3.075346in}{2.698496in}}%
\pgfpathcurveto{\pgfqpoint{3.069522in}{2.692672in}}{\pgfqpoint{3.066249in}{2.684772in}}{\pgfqpoint{3.066249in}{2.676536in}}%
\pgfpathcurveto{\pgfqpoint{3.066249in}{2.668299in}}{\pgfqpoint{3.069522in}{2.660399in}}{\pgfqpoint{3.075346in}{2.654575in}}%
\pgfpathcurveto{\pgfqpoint{3.081170in}{2.648751in}}{\pgfqpoint{3.089070in}{2.645479in}}{\pgfqpoint{3.097306in}{2.645479in}}%
\pgfpathclose%
\pgfusepath{stroke,fill}%
\end{pgfscope}%
\begin{pgfscope}%
\pgfpathrectangle{\pgfqpoint{0.100000in}{0.212622in}}{\pgfqpoint{3.696000in}{3.696000in}}%
\pgfusepath{clip}%
\pgfsetbuttcap%
\pgfsetroundjoin%
\definecolor{currentfill}{rgb}{0.121569,0.466667,0.705882}%
\pgfsetfillcolor{currentfill}%
\pgfsetfillopacity{0.861551}%
\pgfsetlinewidth{1.003750pt}%
\definecolor{currentstroke}{rgb}{0.121569,0.466667,0.705882}%
\pgfsetstrokecolor{currentstroke}%
\pgfsetstrokeopacity{0.861551}%
\pgfsetdash{}{0pt}%
\pgfpathmoveto{\pgfqpoint{1.238367in}{1.825613in}}%
\pgfpathcurveto{\pgfqpoint{1.246604in}{1.825613in}}{\pgfqpoint{1.254504in}{1.828886in}}{\pgfqpoint{1.260328in}{1.834710in}}%
\pgfpathcurveto{\pgfqpoint{1.266152in}{1.840534in}}{\pgfqpoint{1.269424in}{1.848434in}}{\pgfqpoint{1.269424in}{1.856670in}}%
\pgfpathcurveto{\pgfqpoint{1.269424in}{1.864906in}}{\pgfqpoint{1.266152in}{1.872806in}}{\pgfqpoint{1.260328in}{1.878630in}}%
\pgfpathcurveto{\pgfqpoint{1.254504in}{1.884454in}}{\pgfqpoint{1.246604in}{1.887726in}}{\pgfqpoint{1.238367in}{1.887726in}}%
\pgfpathcurveto{\pgfqpoint{1.230131in}{1.887726in}}{\pgfqpoint{1.222231in}{1.884454in}}{\pgfqpoint{1.216407in}{1.878630in}}%
\pgfpathcurveto{\pgfqpoint{1.210583in}{1.872806in}}{\pgfqpoint{1.207311in}{1.864906in}}{\pgfqpoint{1.207311in}{1.856670in}}%
\pgfpathcurveto{\pgfqpoint{1.207311in}{1.848434in}}{\pgfqpoint{1.210583in}{1.840534in}}{\pgfqpoint{1.216407in}{1.834710in}}%
\pgfpathcurveto{\pgfqpoint{1.222231in}{1.828886in}}{\pgfqpoint{1.230131in}{1.825613in}}{\pgfqpoint{1.238367in}{1.825613in}}%
\pgfpathclose%
\pgfusepath{stroke,fill}%
\end{pgfscope}%
\begin{pgfscope}%
\pgfpathrectangle{\pgfqpoint{0.100000in}{0.212622in}}{\pgfqpoint{3.696000in}{3.696000in}}%
\pgfusepath{clip}%
\pgfsetbuttcap%
\pgfsetroundjoin%
\definecolor{currentfill}{rgb}{0.121569,0.466667,0.705882}%
\pgfsetfillcolor{currentfill}%
\pgfsetfillopacity{0.862930}%
\pgfsetlinewidth{1.003750pt}%
\definecolor{currentstroke}{rgb}{0.121569,0.466667,0.705882}%
\pgfsetstrokecolor{currentstroke}%
\pgfsetstrokeopacity{0.862930}%
\pgfsetdash{}{0pt}%
\pgfpathmoveto{\pgfqpoint{1.226774in}{1.810545in}}%
\pgfpathcurveto{\pgfqpoint{1.235010in}{1.810545in}}{\pgfqpoint{1.242910in}{1.813818in}}{\pgfqpoint{1.248734in}{1.819642in}}%
\pgfpathcurveto{\pgfqpoint{1.254558in}{1.825466in}}{\pgfqpoint{1.257830in}{1.833366in}}{\pgfqpoint{1.257830in}{1.841602in}}%
\pgfpathcurveto{\pgfqpoint{1.257830in}{1.849838in}}{\pgfqpoint{1.254558in}{1.857738in}}{\pgfqpoint{1.248734in}{1.863562in}}%
\pgfpathcurveto{\pgfqpoint{1.242910in}{1.869386in}}{\pgfqpoint{1.235010in}{1.872658in}}{\pgfqpoint{1.226774in}{1.872658in}}%
\pgfpathcurveto{\pgfqpoint{1.218537in}{1.872658in}}{\pgfqpoint{1.210637in}{1.869386in}}{\pgfqpoint{1.204813in}{1.863562in}}%
\pgfpathcurveto{\pgfqpoint{1.198990in}{1.857738in}}{\pgfqpoint{1.195717in}{1.849838in}}{\pgfqpoint{1.195717in}{1.841602in}}%
\pgfpathcurveto{\pgfqpoint{1.195717in}{1.833366in}}{\pgfqpoint{1.198990in}{1.825466in}}{\pgfqpoint{1.204813in}{1.819642in}}%
\pgfpathcurveto{\pgfqpoint{1.210637in}{1.813818in}}{\pgfqpoint{1.218537in}{1.810545in}}{\pgfqpoint{1.226774in}{1.810545in}}%
\pgfpathclose%
\pgfusepath{stroke,fill}%
\end{pgfscope}%
\begin{pgfscope}%
\pgfpathrectangle{\pgfqpoint{0.100000in}{0.212622in}}{\pgfqpoint{3.696000in}{3.696000in}}%
\pgfusepath{clip}%
\pgfsetbuttcap%
\pgfsetroundjoin%
\definecolor{currentfill}{rgb}{0.121569,0.466667,0.705882}%
\pgfsetfillcolor{currentfill}%
\pgfsetfillopacity{0.863845}%
\pgfsetlinewidth{1.003750pt}%
\definecolor{currentstroke}{rgb}{0.121569,0.466667,0.705882}%
\pgfsetstrokecolor{currentstroke}%
\pgfsetstrokeopacity{0.863845}%
\pgfsetdash{}{0pt}%
\pgfpathmoveto{\pgfqpoint{1.272561in}{1.829978in}}%
\pgfpathcurveto{\pgfqpoint{1.280797in}{1.829978in}}{\pgfqpoint{1.288698in}{1.833250in}}{\pgfqpoint{1.294521in}{1.839074in}}%
\pgfpathcurveto{\pgfqpoint{1.300345in}{1.844898in}}{\pgfqpoint{1.303618in}{1.852798in}}{\pgfqpoint{1.303618in}{1.861035in}}%
\pgfpathcurveto{\pgfqpoint{1.303618in}{1.869271in}}{\pgfqpoint{1.300345in}{1.877171in}}{\pgfqpoint{1.294521in}{1.882995in}}%
\pgfpathcurveto{\pgfqpoint{1.288698in}{1.888819in}}{\pgfqpoint{1.280797in}{1.892091in}}{\pgfqpoint{1.272561in}{1.892091in}}%
\pgfpathcurveto{\pgfqpoint{1.264325in}{1.892091in}}{\pgfqpoint{1.256425in}{1.888819in}}{\pgfqpoint{1.250601in}{1.882995in}}%
\pgfpathcurveto{\pgfqpoint{1.244777in}{1.877171in}}{\pgfqpoint{1.241505in}{1.869271in}}{\pgfqpoint{1.241505in}{1.861035in}}%
\pgfpathcurveto{\pgfqpoint{1.241505in}{1.852798in}}{\pgfqpoint{1.244777in}{1.844898in}}{\pgfqpoint{1.250601in}{1.839074in}}%
\pgfpathcurveto{\pgfqpoint{1.256425in}{1.833250in}}{\pgfqpoint{1.264325in}{1.829978in}}{\pgfqpoint{1.272561in}{1.829978in}}%
\pgfpathclose%
\pgfusepath{stroke,fill}%
\end{pgfscope}%
\begin{pgfscope}%
\pgfpathrectangle{\pgfqpoint{0.100000in}{0.212622in}}{\pgfqpoint{3.696000in}{3.696000in}}%
\pgfusepath{clip}%
\pgfsetbuttcap%
\pgfsetroundjoin%
\definecolor{currentfill}{rgb}{0.121569,0.466667,0.705882}%
\pgfsetfillcolor{currentfill}%
\pgfsetfillopacity{0.864181}%
\pgfsetlinewidth{1.003750pt}%
\definecolor{currentstroke}{rgb}{0.121569,0.466667,0.705882}%
\pgfsetstrokecolor{currentstroke}%
\pgfsetstrokeopacity{0.864181}%
\pgfsetdash{}{0pt}%
\pgfpathmoveto{\pgfqpoint{1.235695in}{1.823593in}}%
\pgfpathcurveto{\pgfqpoint{1.243931in}{1.823593in}}{\pgfqpoint{1.251831in}{1.826865in}}{\pgfqpoint{1.257655in}{1.832689in}}%
\pgfpathcurveto{\pgfqpoint{1.263479in}{1.838513in}}{\pgfqpoint{1.266751in}{1.846413in}}{\pgfqpoint{1.266751in}{1.854649in}}%
\pgfpathcurveto{\pgfqpoint{1.266751in}{1.862885in}}{\pgfqpoint{1.263479in}{1.870785in}}{\pgfqpoint{1.257655in}{1.876609in}}%
\pgfpathcurveto{\pgfqpoint{1.251831in}{1.882433in}}{\pgfqpoint{1.243931in}{1.885706in}}{\pgfqpoint{1.235695in}{1.885706in}}%
\pgfpathcurveto{\pgfqpoint{1.227458in}{1.885706in}}{\pgfqpoint{1.219558in}{1.882433in}}{\pgfqpoint{1.213734in}{1.876609in}}%
\pgfpathcurveto{\pgfqpoint{1.207910in}{1.870785in}}{\pgfqpoint{1.204638in}{1.862885in}}{\pgfqpoint{1.204638in}{1.854649in}}%
\pgfpathcurveto{\pgfqpoint{1.204638in}{1.846413in}}{\pgfqpoint{1.207910in}{1.838513in}}{\pgfqpoint{1.213734in}{1.832689in}}%
\pgfpathcurveto{\pgfqpoint{1.219558in}{1.826865in}}{\pgfqpoint{1.227458in}{1.823593in}}{\pgfqpoint{1.235695in}{1.823593in}}%
\pgfpathclose%
\pgfusepath{stroke,fill}%
\end{pgfscope}%
\begin{pgfscope}%
\pgfpathrectangle{\pgfqpoint{0.100000in}{0.212622in}}{\pgfqpoint{3.696000in}{3.696000in}}%
\pgfusepath{clip}%
\pgfsetbuttcap%
\pgfsetroundjoin%
\definecolor{currentfill}{rgb}{0.121569,0.466667,0.705882}%
\pgfsetfillcolor{currentfill}%
\pgfsetfillopacity{0.865175}%
\pgfsetlinewidth{1.003750pt}%
\definecolor{currentstroke}{rgb}{0.121569,0.466667,0.705882}%
\pgfsetstrokecolor{currentstroke}%
\pgfsetstrokeopacity{0.865175}%
\pgfsetdash{}{0pt}%
\pgfpathmoveto{\pgfqpoint{1.301648in}{1.851478in}}%
\pgfpathcurveto{\pgfqpoint{1.309884in}{1.851478in}}{\pgfqpoint{1.317784in}{1.854750in}}{\pgfqpoint{1.323608in}{1.860574in}}%
\pgfpathcurveto{\pgfqpoint{1.329432in}{1.866398in}}{\pgfqpoint{1.332704in}{1.874298in}}{\pgfqpoint{1.332704in}{1.882534in}}%
\pgfpathcurveto{\pgfqpoint{1.332704in}{1.890771in}}{\pgfqpoint{1.329432in}{1.898671in}}{\pgfqpoint{1.323608in}{1.904495in}}%
\pgfpathcurveto{\pgfqpoint{1.317784in}{1.910318in}}{\pgfqpoint{1.309884in}{1.913591in}}{\pgfqpoint{1.301648in}{1.913591in}}%
\pgfpathcurveto{\pgfqpoint{1.293412in}{1.913591in}}{\pgfqpoint{1.285512in}{1.910318in}}{\pgfqpoint{1.279688in}{1.904495in}}%
\pgfpathcurveto{\pgfqpoint{1.273864in}{1.898671in}}{\pgfqpoint{1.270591in}{1.890771in}}{\pgfqpoint{1.270591in}{1.882534in}}%
\pgfpathcurveto{\pgfqpoint{1.270591in}{1.874298in}}{\pgfqpoint{1.273864in}{1.866398in}}{\pgfqpoint{1.279688in}{1.860574in}}%
\pgfpathcurveto{\pgfqpoint{1.285512in}{1.854750in}}{\pgfqpoint{1.293412in}{1.851478in}}{\pgfqpoint{1.301648in}{1.851478in}}%
\pgfpathclose%
\pgfusepath{stroke,fill}%
\end{pgfscope}%
\begin{pgfscope}%
\pgfpathrectangle{\pgfqpoint{0.100000in}{0.212622in}}{\pgfqpoint{3.696000in}{3.696000in}}%
\pgfusepath{clip}%
\pgfsetbuttcap%
\pgfsetroundjoin%
\definecolor{currentfill}{rgb}{0.121569,0.466667,0.705882}%
\pgfsetfillcolor{currentfill}%
\pgfsetfillopacity{0.868810}%
\pgfsetlinewidth{1.003750pt}%
\definecolor{currentstroke}{rgb}{0.121569,0.466667,0.705882}%
\pgfsetstrokecolor{currentstroke}%
\pgfsetstrokeopacity{0.868810}%
\pgfsetdash{}{0pt}%
\pgfpathmoveto{\pgfqpoint{1.271571in}{1.829156in}}%
\pgfpathcurveto{\pgfqpoint{1.279807in}{1.829156in}}{\pgfqpoint{1.287707in}{1.832428in}}{\pgfqpoint{1.293531in}{1.838252in}}%
\pgfpathcurveto{\pgfqpoint{1.299355in}{1.844076in}}{\pgfqpoint{1.302628in}{1.851976in}}{\pgfqpoint{1.302628in}{1.860212in}}%
\pgfpathcurveto{\pgfqpoint{1.302628in}{1.868448in}}{\pgfqpoint{1.299355in}{1.876348in}}{\pgfqpoint{1.293531in}{1.882172in}}%
\pgfpathcurveto{\pgfqpoint{1.287707in}{1.887996in}}{\pgfqpoint{1.279807in}{1.891269in}}{\pgfqpoint{1.271571in}{1.891269in}}%
\pgfpathcurveto{\pgfqpoint{1.263335in}{1.891269in}}{\pgfqpoint{1.255435in}{1.887996in}}{\pgfqpoint{1.249611in}{1.882172in}}%
\pgfpathcurveto{\pgfqpoint{1.243787in}{1.876348in}}{\pgfqpoint{1.240515in}{1.868448in}}{\pgfqpoint{1.240515in}{1.860212in}}%
\pgfpathcurveto{\pgfqpoint{1.240515in}{1.851976in}}{\pgfqpoint{1.243787in}{1.844076in}}{\pgfqpoint{1.249611in}{1.838252in}}%
\pgfpathcurveto{\pgfqpoint{1.255435in}{1.832428in}}{\pgfqpoint{1.263335in}{1.829156in}}{\pgfqpoint{1.271571in}{1.829156in}}%
\pgfpathclose%
\pgfusepath{stroke,fill}%
\end{pgfscope}%
\begin{pgfscope}%
\pgfpathrectangle{\pgfqpoint{0.100000in}{0.212622in}}{\pgfqpoint{3.696000in}{3.696000in}}%
\pgfusepath{clip}%
\pgfsetbuttcap%
\pgfsetroundjoin%
\definecolor{currentfill}{rgb}{0.121569,0.466667,0.705882}%
\pgfsetfillcolor{currentfill}%
\pgfsetfillopacity{0.869660}%
\pgfsetlinewidth{1.003750pt}%
\definecolor{currentstroke}{rgb}{0.121569,0.466667,0.705882}%
\pgfsetstrokecolor{currentstroke}%
\pgfsetstrokeopacity{0.869660}%
\pgfsetdash{}{0pt}%
\pgfpathmoveto{\pgfqpoint{3.087001in}{2.634851in}}%
\pgfpathcurveto{\pgfqpoint{3.095237in}{2.634851in}}{\pgfqpoint{3.103137in}{2.638123in}}{\pgfqpoint{3.108961in}{2.643947in}}%
\pgfpathcurveto{\pgfqpoint{3.114785in}{2.649771in}}{\pgfqpoint{3.118057in}{2.657671in}}{\pgfqpoint{3.118057in}{2.665908in}}%
\pgfpathcurveto{\pgfqpoint{3.118057in}{2.674144in}}{\pgfqpoint{3.114785in}{2.682044in}}{\pgfqpoint{3.108961in}{2.687868in}}%
\pgfpathcurveto{\pgfqpoint{3.103137in}{2.693692in}}{\pgfqpoint{3.095237in}{2.696964in}}{\pgfqpoint{3.087001in}{2.696964in}}%
\pgfpathcurveto{\pgfqpoint{3.078765in}{2.696964in}}{\pgfqpoint{3.070864in}{2.693692in}}{\pgfqpoint{3.065041in}{2.687868in}}%
\pgfpathcurveto{\pgfqpoint{3.059217in}{2.682044in}}{\pgfqpoint{3.055944in}{2.674144in}}{\pgfqpoint{3.055944in}{2.665908in}}%
\pgfpathcurveto{\pgfqpoint{3.055944in}{2.657671in}}{\pgfqpoint{3.059217in}{2.649771in}}{\pgfqpoint{3.065041in}{2.643947in}}%
\pgfpathcurveto{\pgfqpoint{3.070864in}{2.638123in}}{\pgfqpoint{3.078765in}{2.634851in}}{\pgfqpoint{3.087001in}{2.634851in}}%
\pgfpathclose%
\pgfusepath{stroke,fill}%
\end{pgfscope}%
\begin{pgfscope}%
\pgfpathrectangle{\pgfqpoint{0.100000in}{0.212622in}}{\pgfqpoint{3.696000in}{3.696000in}}%
\pgfusepath{clip}%
\pgfsetbuttcap%
\pgfsetroundjoin%
\definecolor{currentfill}{rgb}{0.121569,0.466667,0.705882}%
\pgfsetfillcolor{currentfill}%
\pgfsetfillopacity{0.871094}%
\pgfsetlinewidth{1.003750pt}%
\definecolor{currentstroke}{rgb}{0.121569,0.466667,0.705882}%
\pgfsetstrokecolor{currentstroke}%
\pgfsetstrokeopacity{0.871094}%
\pgfsetdash{}{0pt}%
\pgfpathmoveto{\pgfqpoint{2.435366in}{2.361062in}}%
\pgfpathcurveto{\pgfqpoint{2.443603in}{2.361062in}}{\pgfqpoint{2.451503in}{2.364335in}}{\pgfqpoint{2.457327in}{2.370158in}}%
\pgfpathcurveto{\pgfqpoint{2.463151in}{2.375982in}}{\pgfqpoint{2.466423in}{2.383882in}}{\pgfqpoint{2.466423in}{2.392119in}}%
\pgfpathcurveto{\pgfqpoint{2.466423in}{2.400355in}}{\pgfqpoint{2.463151in}{2.408255in}}{\pgfqpoint{2.457327in}{2.414079in}}%
\pgfpathcurveto{\pgfqpoint{2.451503in}{2.419903in}}{\pgfqpoint{2.443603in}{2.423175in}}{\pgfqpoint{2.435366in}{2.423175in}}%
\pgfpathcurveto{\pgfqpoint{2.427130in}{2.423175in}}{\pgfqpoint{2.419230in}{2.419903in}}{\pgfqpoint{2.413406in}{2.414079in}}%
\pgfpathcurveto{\pgfqpoint{2.407582in}{2.408255in}}{\pgfqpoint{2.404310in}{2.400355in}}{\pgfqpoint{2.404310in}{2.392119in}}%
\pgfpathcurveto{\pgfqpoint{2.404310in}{2.383882in}}{\pgfqpoint{2.407582in}{2.375982in}}{\pgfqpoint{2.413406in}{2.370158in}}%
\pgfpathcurveto{\pgfqpoint{2.419230in}{2.364335in}}{\pgfqpoint{2.427130in}{2.361062in}}{\pgfqpoint{2.435366in}{2.361062in}}%
\pgfpathclose%
\pgfusepath{stroke,fill}%
\end{pgfscope}%
\begin{pgfscope}%
\pgfpathrectangle{\pgfqpoint{0.100000in}{0.212622in}}{\pgfqpoint{3.696000in}{3.696000in}}%
\pgfusepath{clip}%
\pgfsetbuttcap%
\pgfsetroundjoin%
\definecolor{currentfill}{rgb}{0.121569,0.466667,0.705882}%
\pgfsetfillcolor{currentfill}%
\pgfsetfillopacity{0.871300}%
\pgfsetlinewidth{1.003750pt}%
\definecolor{currentstroke}{rgb}{0.121569,0.466667,0.705882}%
\pgfsetstrokecolor{currentstroke}%
\pgfsetstrokeopacity{0.871300}%
\pgfsetdash{}{0pt}%
\pgfpathmoveto{\pgfqpoint{1.271961in}{1.821976in}}%
\pgfpathcurveto{\pgfqpoint{1.280197in}{1.821976in}}{\pgfqpoint{1.288097in}{1.825248in}}{\pgfqpoint{1.293921in}{1.831072in}}%
\pgfpathcurveto{\pgfqpoint{1.299745in}{1.836896in}}{\pgfqpoint{1.303018in}{1.844796in}}{\pgfqpoint{1.303018in}{1.853033in}}%
\pgfpathcurveto{\pgfqpoint{1.303018in}{1.861269in}}{\pgfqpoint{1.299745in}{1.869169in}}{\pgfqpoint{1.293921in}{1.874993in}}%
\pgfpathcurveto{\pgfqpoint{1.288097in}{1.880817in}}{\pgfqpoint{1.280197in}{1.884089in}}{\pgfqpoint{1.271961in}{1.884089in}}%
\pgfpathcurveto{\pgfqpoint{1.263725in}{1.884089in}}{\pgfqpoint{1.255825in}{1.880817in}}{\pgfqpoint{1.250001in}{1.874993in}}%
\pgfpathcurveto{\pgfqpoint{1.244177in}{1.869169in}}{\pgfqpoint{1.240905in}{1.861269in}}{\pgfqpoint{1.240905in}{1.853033in}}%
\pgfpathcurveto{\pgfqpoint{1.240905in}{1.844796in}}{\pgfqpoint{1.244177in}{1.836896in}}{\pgfqpoint{1.250001in}{1.831072in}}%
\pgfpathcurveto{\pgfqpoint{1.255825in}{1.825248in}}{\pgfqpoint{1.263725in}{1.821976in}}{\pgfqpoint{1.271961in}{1.821976in}}%
\pgfpathclose%
\pgfusepath{stroke,fill}%
\end{pgfscope}%
\begin{pgfscope}%
\pgfpathrectangle{\pgfqpoint{0.100000in}{0.212622in}}{\pgfqpoint{3.696000in}{3.696000in}}%
\pgfusepath{clip}%
\pgfsetbuttcap%
\pgfsetroundjoin%
\definecolor{currentfill}{rgb}{0.121569,0.466667,0.705882}%
\pgfsetfillcolor{currentfill}%
\pgfsetfillopacity{0.871606}%
\pgfsetlinewidth{1.003750pt}%
\definecolor{currentstroke}{rgb}{0.121569,0.466667,0.705882}%
\pgfsetstrokecolor{currentstroke}%
\pgfsetstrokeopacity{0.871606}%
\pgfsetdash{}{0pt}%
\pgfpathmoveto{\pgfqpoint{1.221834in}{1.808053in}}%
\pgfpathcurveto{\pgfqpoint{1.230070in}{1.808053in}}{\pgfqpoint{1.237970in}{1.811325in}}{\pgfqpoint{1.243794in}{1.817149in}}%
\pgfpathcurveto{\pgfqpoint{1.249618in}{1.822973in}}{\pgfqpoint{1.252890in}{1.830873in}}{\pgfqpoint{1.252890in}{1.839109in}}%
\pgfpathcurveto{\pgfqpoint{1.252890in}{1.847345in}}{\pgfqpoint{1.249618in}{1.855246in}}{\pgfqpoint{1.243794in}{1.861069in}}%
\pgfpathcurveto{\pgfqpoint{1.237970in}{1.866893in}}{\pgfqpoint{1.230070in}{1.870166in}}{\pgfqpoint{1.221834in}{1.870166in}}%
\pgfpathcurveto{\pgfqpoint{1.213597in}{1.870166in}}{\pgfqpoint{1.205697in}{1.866893in}}{\pgfqpoint{1.199874in}{1.861069in}}%
\pgfpathcurveto{\pgfqpoint{1.194050in}{1.855246in}}{\pgfqpoint{1.190777in}{1.847345in}}{\pgfqpoint{1.190777in}{1.839109in}}%
\pgfpathcurveto{\pgfqpoint{1.190777in}{1.830873in}}{\pgfqpoint{1.194050in}{1.822973in}}{\pgfqpoint{1.199874in}{1.817149in}}%
\pgfpathcurveto{\pgfqpoint{1.205697in}{1.811325in}}{\pgfqpoint{1.213597in}{1.808053in}}{\pgfqpoint{1.221834in}{1.808053in}}%
\pgfpathclose%
\pgfusepath{stroke,fill}%
\end{pgfscope}%
\begin{pgfscope}%
\pgfpathrectangle{\pgfqpoint{0.100000in}{0.212622in}}{\pgfqpoint{3.696000in}{3.696000in}}%
\pgfusepath{clip}%
\pgfsetbuttcap%
\pgfsetroundjoin%
\definecolor{currentfill}{rgb}{0.121569,0.466667,0.705882}%
\pgfsetfillcolor{currentfill}%
\pgfsetfillopacity{0.872104}%
\pgfsetlinewidth{1.003750pt}%
\definecolor{currentstroke}{rgb}{0.121569,0.466667,0.705882}%
\pgfsetstrokecolor{currentstroke}%
\pgfsetstrokeopacity{0.872104}%
\pgfsetdash{}{0pt}%
\pgfpathmoveto{\pgfqpoint{3.086891in}{2.637534in}}%
\pgfpathcurveto{\pgfqpoint{3.095127in}{2.637534in}}{\pgfqpoint{3.103027in}{2.640806in}}{\pgfqpoint{3.108851in}{2.646630in}}%
\pgfpathcurveto{\pgfqpoint{3.114675in}{2.652454in}}{\pgfqpoint{3.117947in}{2.660354in}}{\pgfqpoint{3.117947in}{2.668591in}}%
\pgfpathcurveto{\pgfqpoint{3.117947in}{2.676827in}}{\pgfqpoint{3.114675in}{2.684727in}}{\pgfqpoint{3.108851in}{2.690551in}}%
\pgfpathcurveto{\pgfqpoint{3.103027in}{2.696375in}}{\pgfqpoint{3.095127in}{2.699647in}}{\pgfqpoint{3.086891in}{2.699647in}}%
\pgfpathcurveto{\pgfqpoint{3.078655in}{2.699647in}}{\pgfqpoint{3.070754in}{2.696375in}}{\pgfqpoint{3.064931in}{2.690551in}}%
\pgfpathcurveto{\pgfqpoint{3.059107in}{2.684727in}}{\pgfqpoint{3.055834in}{2.676827in}}{\pgfqpoint{3.055834in}{2.668591in}}%
\pgfpathcurveto{\pgfqpoint{3.055834in}{2.660354in}}{\pgfqpoint{3.059107in}{2.652454in}}{\pgfqpoint{3.064931in}{2.646630in}}%
\pgfpathcurveto{\pgfqpoint{3.070754in}{2.640806in}}{\pgfqpoint{3.078655in}{2.637534in}}{\pgfqpoint{3.086891in}{2.637534in}}%
\pgfpathclose%
\pgfusepath{stroke,fill}%
\end{pgfscope}%
\begin{pgfscope}%
\pgfpathrectangle{\pgfqpoint{0.100000in}{0.212622in}}{\pgfqpoint{3.696000in}{3.696000in}}%
\pgfusepath{clip}%
\pgfsetbuttcap%
\pgfsetroundjoin%
\definecolor{currentfill}{rgb}{0.121569,0.466667,0.705882}%
\pgfsetfillcolor{currentfill}%
\pgfsetfillopacity{0.873190}%
\pgfsetlinewidth{1.003750pt}%
\definecolor{currentstroke}{rgb}{0.121569,0.466667,0.705882}%
\pgfsetstrokecolor{currentstroke}%
\pgfsetstrokeopacity{0.873190}%
\pgfsetdash{}{0pt}%
\pgfpathmoveto{\pgfqpoint{1.263868in}{1.814116in}}%
\pgfpathcurveto{\pgfqpoint{1.272104in}{1.814116in}}{\pgfqpoint{1.280004in}{1.817388in}}{\pgfqpoint{1.285828in}{1.823212in}}%
\pgfpathcurveto{\pgfqpoint{1.291652in}{1.829036in}}{\pgfqpoint{1.294924in}{1.836936in}}{\pgfqpoint{1.294924in}{1.845172in}}%
\pgfpathcurveto{\pgfqpoint{1.294924in}{1.853409in}}{\pgfqpoint{1.291652in}{1.861309in}}{\pgfqpoint{1.285828in}{1.867133in}}%
\pgfpathcurveto{\pgfqpoint{1.280004in}{1.872957in}}{\pgfqpoint{1.272104in}{1.876229in}}{\pgfqpoint{1.263868in}{1.876229in}}%
\pgfpathcurveto{\pgfqpoint{1.255632in}{1.876229in}}{\pgfqpoint{1.247732in}{1.872957in}}{\pgfqpoint{1.241908in}{1.867133in}}%
\pgfpathcurveto{\pgfqpoint{1.236084in}{1.861309in}}{\pgfqpoint{1.232811in}{1.853409in}}{\pgfqpoint{1.232811in}{1.845172in}}%
\pgfpathcurveto{\pgfqpoint{1.232811in}{1.836936in}}{\pgfqpoint{1.236084in}{1.829036in}}{\pgfqpoint{1.241908in}{1.823212in}}%
\pgfpathcurveto{\pgfqpoint{1.247732in}{1.817388in}}{\pgfqpoint{1.255632in}{1.814116in}}{\pgfqpoint{1.263868in}{1.814116in}}%
\pgfpathclose%
\pgfusepath{stroke,fill}%
\end{pgfscope}%
\begin{pgfscope}%
\pgfpathrectangle{\pgfqpoint{0.100000in}{0.212622in}}{\pgfqpoint{3.696000in}{3.696000in}}%
\pgfusepath{clip}%
\pgfsetbuttcap%
\pgfsetroundjoin%
\definecolor{currentfill}{rgb}{0.121569,0.466667,0.705882}%
\pgfsetfillcolor{currentfill}%
\pgfsetfillopacity{0.873469}%
\pgfsetlinewidth{1.003750pt}%
\definecolor{currentstroke}{rgb}{0.121569,0.466667,0.705882}%
\pgfsetstrokecolor{currentstroke}%
\pgfsetstrokeopacity{0.873469}%
\pgfsetdash{}{0pt}%
\pgfpathmoveto{\pgfqpoint{2.448931in}{2.365640in}}%
\pgfpathcurveto{\pgfqpoint{2.457167in}{2.365640in}}{\pgfqpoint{2.465067in}{2.368913in}}{\pgfqpoint{2.470891in}{2.374737in}}%
\pgfpathcurveto{\pgfqpoint{2.476715in}{2.380560in}}{\pgfqpoint{2.479987in}{2.388460in}}{\pgfqpoint{2.479987in}{2.396697in}}%
\pgfpathcurveto{\pgfqpoint{2.479987in}{2.404933in}}{\pgfqpoint{2.476715in}{2.412833in}}{\pgfqpoint{2.470891in}{2.418657in}}%
\pgfpathcurveto{\pgfqpoint{2.465067in}{2.424481in}}{\pgfqpoint{2.457167in}{2.427753in}}{\pgfqpoint{2.448931in}{2.427753in}}%
\pgfpathcurveto{\pgfqpoint{2.440695in}{2.427753in}}{\pgfqpoint{2.432795in}{2.424481in}}{\pgfqpoint{2.426971in}{2.418657in}}%
\pgfpathcurveto{\pgfqpoint{2.421147in}{2.412833in}}{\pgfqpoint{2.417874in}{2.404933in}}{\pgfqpoint{2.417874in}{2.396697in}}%
\pgfpathcurveto{\pgfqpoint{2.417874in}{2.388460in}}{\pgfqpoint{2.421147in}{2.380560in}}{\pgfqpoint{2.426971in}{2.374737in}}%
\pgfpathcurveto{\pgfqpoint{2.432795in}{2.368913in}}{\pgfqpoint{2.440695in}{2.365640in}}{\pgfqpoint{2.448931in}{2.365640in}}%
\pgfpathclose%
\pgfusepath{stroke,fill}%
\end{pgfscope}%
\begin{pgfscope}%
\pgfpathrectangle{\pgfqpoint{0.100000in}{0.212622in}}{\pgfqpoint{3.696000in}{3.696000in}}%
\pgfusepath{clip}%
\pgfsetbuttcap%
\pgfsetroundjoin%
\definecolor{currentfill}{rgb}{0.121569,0.466667,0.705882}%
\pgfsetfillcolor{currentfill}%
\pgfsetfillopacity{0.873893}%
\pgfsetlinewidth{1.003750pt}%
\definecolor{currentstroke}{rgb}{0.121569,0.466667,0.705882}%
\pgfsetstrokecolor{currentstroke}%
\pgfsetstrokeopacity{0.873893}%
\pgfsetdash{}{0pt}%
\pgfpathmoveto{\pgfqpoint{1.238128in}{1.814465in}}%
\pgfpathcurveto{\pgfqpoint{1.246364in}{1.814465in}}{\pgfqpoint{1.254264in}{1.817738in}}{\pgfqpoint{1.260088in}{1.823562in}}%
\pgfpathcurveto{\pgfqpoint{1.265912in}{1.829385in}}{\pgfqpoint{1.269184in}{1.837286in}}{\pgfqpoint{1.269184in}{1.845522in}}%
\pgfpathcurveto{\pgfqpoint{1.269184in}{1.853758in}}{\pgfqpoint{1.265912in}{1.861658in}}{\pgfqpoint{1.260088in}{1.867482in}}%
\pgfpathcurveto{\pgfqpoint{1.254264in}{1.873306in}}{\pgfqpoint{1.246364in}{1.876578in}}{\pgfqpoint{1.238128in}{1.876578in}}%
\pgfpathcurveto{\pgfqpoint{1.229892in}{1.876578in}}{\pgfqpoint{1.221991in}{1.873306in}}{\pgfqpoint{1.216168in}{1.867482in}}%
\pgfpathcurveto{\pgfqpoint{1.210344in}{1.861658in}}{\pgfqpoint{1.207071in}{1.853758in}}{\pgfqpoint{1.207071in}{1.845522in}}%
\pgfpathcurveto{\pgfqpoint{1.207071in}{1.837286in}}{\pgfqpoint{1.210344in}{1.829385in}}{\pgfqpoint{1.216168in}{1.823562in}}%
\pgfpathcurveto{\pgfqpoint{1.221991in}{1.817738in}}{\pgfqpoint{1.229892in}{1.814465in}}{\pgfqpoint{1.238128in}{1.814465in}}%
\pgfpathclose%
\pgfusepath{stroke,fill}%
\end{pgfscope}%
\begin{pgfscope}%
\pgfpathrectangle{\pgfqpoint{0.100000in}{0.212622in}}{\pgfqpoint{3.696000in}{3.696000in}}%
\pgfusepath{clip}%
\pgfsetbuttcap%
\pgfsetroundjoin%
\definecolor{currentfill}{rgb}{0.121569,0.466667,0.705882}%
\pgfsetfillcolor{currentfill}%
\pgfsetfillopacity{0.874201}%
\pgfsetlinewidth{1.003750pt}%
\definecolor{currentstroke}{rgb}{0.121569,0.466667,0.705882}%
\pgfsetstrokecolor{currentstroke}%
\pgfsetstrokeopacity{0.874201}%
\pgfsetdash{}{0pt}%
\pgfpathmoveto{\pgfqpoint{2.421323in}{2.346409in}}%
\pgfpathcurveto{\pgfqpoint{2.429559in}{2.346409in}}{\pgfqpoint{2.437459in}{2.349681in}}{\pgfqpoint{2.443283in}{2.355505in}}%
\pgfpathcurveto{\pgfqpoint{2.449107in}{2.361329in}}{\pgfqpoint{2.452379in}{2.369229in}}{\pgfqpoint{2.452379in}{2.377465in}}%
\pgfpathcurveto{\pgfqpoint{2.452379in}{2.385702in}}{\pgfqpoint{2.449107in}{2.393602in}}{\pgfqpoint{2.443283in}{2.399426in}}%
\pgfpathcurveto{\pgfqpoint{2.437459in}{2.405250in}}{\pgfqpoint{2.429559in}{2.408522in}}{\pgfqpoint{2.421323in}{2.408522in}}%
\pgfpathcurveto{\pgfqpoint{2.413086in}{2.408522in}}{\pgfqpoint{2.405186in}{2.405250in}}{\pgfqpoint{2.399362in}{2.399426in}}%
\pgfpathcurveto{\pgfqpoint{2.393539in}{2.393602in}}{\pgfqpoint{2.390266in}{2.385702in}}{\pgfqpoint{2.390266in}{2.377465in}}%
\pgfpathcurveto{\pgfqpoint{2.390266in}{2.369229in}}{\pgfqpoint{2.393539in}{2.361329in}}{\pgfqpoint{2.399362in}{2.355505in}}%
\pgfpathcurveto{\pgfqpoint{2.405186in}{2.349681in}}{\pgfqpoint{2.413086in}{2.346409in}}{\pgfqpoint{2.421323in}{2.346409in}}%
\pgfpathclose%
\pgfusepath{stroke,fill}%
\end{pgfscope}%
\begin{pgfscope}%
\pgfpathrectangle{\pgfqpoint{0.100000in}{0.212622in}}{\pgfqpoint{3.696000in}{3.696000in}}%
\pgfusepath{clip}%
\pgfsetbuttcap%
\pgfsetroundjoin%
\definecolor{currentfill}{rgb}{0.121569,0.466667,0.705882}%
\pgfsetfillcolor{currentfill}%
\pgfsetfillopacity{0.874396}%
\pgfsetlinewidth{1.003750pt}%
\definecolor{currentstroke}{rgb}{0.121569,0.466667,0.705882}%
\pgfsetstrokecolor{currentstroke}%
\pgfsetstrokeopacity{0.874396}%
\pgfsetdash{}{0pt}%
\pgfpathmoveto{\pgfqpoint{1.234299in}{1.811438in}}%
\pgfpathcurveto{\pgfqpoint{1.242535in}{1.811438in}}{\pgfqpoint{1.250435in}{1.814711in}}{\pgfqpoint{1.256259in}{1.820534in}}%
\pgfpathcurveto{\pgfqpoint{1.262083in}{1.826358in}}{\pgfqpoint{1.265356in}{1.834258in}}{\pgfqpoint{1.265356in}{1.842495in}}%
\pgfpathcurveto{\pgfqpoint{1.265356in}{1.850731in}}{\pgfqpoint{1.262083in}{1.858631in}}{\pgfqpoint{1.256259in}{1.864455in}}%
\pgfpathcurveto{\pgfqpoint{1.250435in}{1.870279in}}{\pgfqpoint{1.242535in}{1.873551in}}{\pgfqpoint{1.234299in}{1.873551in}}%
\pgfpathcurveto{\pgfqpoint{1.226063in}{1.873551in}}{\pgfqpoint{1.218163in}{1.870279in}}{\pgfqpoint{1.212339in}{1.864455in}}%
\pgfpathcurveto{\pgfqpoint{1.206515in}{1.858631in}}{\pgfqpoint{1.203243in}{1.850731in}}{\pgfqpoint{1.203243in}{1.842495in}}%
\pgfpathcurveto{\pgfqpoint{1.203243in}{1.834258in}}{\pgfqpoint{1.206515in}{1.826358in}}{\pgfqpoint{1.212339in}{1.820534in}}%
\pgfpathcurveto{\pgfqpoint{1.218163in}{1.814711in}}{\pgfqpoint{1.226063in}{1.811438in}}{\pgfqpoint{1.234299in}{1.811438in}}%
\pgfpathclose%
\pgfusepath{stroke,fill}%
\end{pgfscope}%
\begin{pgfscope}%
\pgfpathrectangle{\pgfqpoint{0.100000in}{0.212622in}}{\pgfqpoint{3.696000in}{3.696000in}}%
\pgfusepath{clip}%
\pgfsetbuttcap%
\pgfsetroundjoin%
\definecolor{currentfill}{rgb}{0.121569,0.466667,0.705882}%
\pgfsetfillcolor{currentfill}%
\pgfsetfillopacity{0.874412}%
\pgfsetlinewidth{1.003750pt}%
\definecolor{currentstroke}{rgb}{0.121569,0.466667,0.705882}%
\pgfsetstrokecolor{currentstroke}%
\pgfsetstrokeopacity{0.874412}%
\pgfsetdash{}{0pt}%
\pgfpathmoveto{\pgfqpoint{1.223589in}{1.810872in}}%
\pgfpathcurveto{\pgfqpoint{1.231825in}{1.810872in}}{\pgfqpoint{1.239725in}{1.814145in}}{\pgfqpoint{1.245549in}{1.819969in}}%
\pgfpathcurveto{\pgfqpoint{1.251373in}{1.825793in}}{\pgfqpoint{1.254645in}{1.833693in}}{\pgfqpoint{1.254645in}{1.841929in}}%
\pgfpathcurveto{\pgfqpoint{1.254645in}{1.850165in}}{\pgfqpoint{1.251373in}{1.858065in}}{\pgfqpoint{1.245549in}{1.863889in}}%
\pgfpathcurveto{\pgfqpoint{1.239725in}{1.869713in}}{\pgfqpoint{1.231825in}{1.872985in}}{\pgfqpoint{1.223589in}{1.872985in}}%
\pgfpathcurveto{\pgfqpoint{1.215352in}{1.872985in}}{\pgfqpoint{1.207452in}{1.869713in}}{\pgfqpoint{1.201628in}{1.863889in}}%
\pgfpathcurveto{\pgfqpoint{1.195805in}{1.858065in}}{\pgfqpoint{1.192532in}{1.850165in}}{\pgfqpoint{1.192532in}{1.841929in}}%
\pgfpathcurveto{\pgfqpoint{1.192532in}{1.833693in}}{\pgfqpoint{1.195805in}{1.825793in}}{\pgfqpoint{1.201628in}{1.819969in}}%
\pgfpathcurveto{\pgfqpoint{1.207452in}{1.814145in}}{\pgfqpoint{1.215352in}{1.810872in}}{\pgfqpoint{1.223589in}{1.810872in}}%
\pgfpathclose%
\pgfusepath{stroke,fill}%
\end{pgfscope}%
\begin{pgfscope}%
\pgfpathrectangle{\pgfqpoint{0.100000in}{0.212622in}}{\pgfqpoint{3.696000in}{3.696000in}}%
\pgfusepath{clip}%
\pgfsetbuttcap%
\pgfsetroundjoin%
\definecolor{currentfill}{rgb}{0.121569,0.466667,0.705882}%
\pgfsetfillcolor{currentfill}%
\pgfsetfillopacity{0.874553}%
\pgfsetlinewidth{1.003750pt}%
\definecolor{currentstroke}{rgb}{0.121569,0.466667,0.705882}%
\pgfsetstrokecolor{currentstroke}%
\pgfsetstrokeopacity{0.874553}%
\pgfsetdash{}{0pt}%
\pgfpathmoveto{\pgfqpoint{1.265805in}{1.806261in}}%
\pgfpathcurveto{\pgfqpoint{1.274042in}{1.806261in}}{\pgfqpoint{1.281942in}{1.809534in}}{\pgfqpoint{1.287766in}{1.815357in}}%
\pgfpathcurveto{\pgfqpoint{1.293590in}{1.821181in}}{\pgfqpoint{1.296862in}{1.829081in}}{\pgfqpoint{1.296862in}{1.837318in}}%
\pgfpathcurveto{\pgfqpoint{1.296862in}{1.845554in}}{\pgfqpoint{1.293590in}{1.853454in}}{\pgfqpoint{1.287766in}{1.859278in}}%
\pgfpathcurveto{\pgfqpoint{1.281942in}{1.865102in}}{\pgfqpoint{1.274042in}{1.868374in}}{\pgfqpoint{1.265805in}{1.868374in}}%
\pgfpathcurveto{\pgfqpoint{1.257569in}{1.868374in}}{\pgfqpoint{1.249669in}{1.865102in}}{\pgfqpoint{1.243845in}{1.859278in}}%
\pgfpathcurveto{\pgfqpoint{1.238021in}{1.853454in}}{\pgfqpoint{1.234749in}{1.845554in}}{\pgfqpoint{1.234749in}{1.837318in}}%
\pgfpathcurveto{\pgfqpoint{1.234749in}{1.829081in}}{\pgfqpoint{1.238021in}{1.821181in}}{\pgfqpoint{1.243845in}{1.815357in}}%
\pgfpathcurveto{\pgfqpoint{1.249669in}{1.809534in}}{\pgfqpoint{1.257569in}{1.806261in}}{\pgfqpoint{1.265805in}{1.806261in}}%
\pgfpathclose%
\pgfusepath{stroke,fill}%
\end{pgfscope}%
\begin{pgfscope}%
\pgfpathrectangle{\pgfqpoint{0.100000in}{0.212622in}}{\pgfqpoint{3.696000in}{3.696000in}}%
\pgfusepath{clip}%
\pgfsetbuttcap%
\pgfsetroundjoin%
\definecolor{currentfill}{rgb}{0.121569,0.466667,0.705882}%
\pgfsetfillcolor{currentfill}%
\pgfsetfillopacity{0.874731}%
\pgfsetlinewidth{1.003750pt}%
\definecolor{currentstroke}{rgb}{0.121569,0.466667,0.705882}%
\pgfsetstrokecolor{currentstroke}%
\pgfsetstrokeopacity{0.874731}%
\pgfsetdash{}{0pt}%
\pgfpathmoveto{\pgfqpoint{2.747857in}{2.512696in}}%
\pgfpathcurveto{\pgfqpoint{2.756093in}{2.512696in}}{\pgfqpoint{2.763994in}{2.515968in}}{\pgfqpoint{2.769817in}{2.521792in}}%
\pgfpathcurveto{\pgfqpoint{2.775641in}{2.527616in}}{\pgfqpoint{2.778914in}{2.535516in}}{\pgfqpoint{2.778914in}{2.543752in}}%
\pgfpathcurveto{\pgfqpoint{2.778914in}{2.551989in}}{\pgfqpoint{2.775641in}{2.559889in}}{\pgfqpoint{2.769817in}{2.565713in}}%
\pgfpathcurveto{\pgfqpoint{2.763994in}{2.571537in}}{\pgfqpoint{2.756093in}{2.574809in}}{\pgfqpoint{2.747857in}{2.574809in}}%
\pgfpathcurveto{\pgfqpoint{2.739621in}{2.574809in}}{\pgfqpoint{2.731721in}{2.571537in}}{\pgfqpoint{2.725897in}{2.565713in}}%
\pgfpathcurveto{\pgfqpoint{2.720073in}{2.559889in}}{\pgfqpoint{2.716801in}{2.551989in}}{\pgfqpoint{2.716801in}{2.543752in}}%
\pgfpathcurveto{\pgfqpoint{2.716801in}{2.535516in}}{\pgfqpoint{2.720073in}{2.527616in}}{\pgfqpoint{2.725897in}{2.521792in}}%
\pgfpathcurveto{\pgfqpoint{2.731721in}{2.515968in}}{\pgfqpoint{2.739621in}{2.512696in}}{\pgfqpoint{2.747857in}{2.512696in}}%
\pgfpathclose%
\pgfusepath{stroke,fill}%
\end{pgfscope}%
\begin{pgfscope}%
\pgfpathrectangle{\pgfqpoint{0.100000in}{0.212622in}}{\pgfqpoint{3.696000in}{3.696000in}}%
\pgfusepath{clip}%
\pgfsetbuttcap%
\pgfsetroundjoin%
\definecolor{currentfill}{rgb}{0.121569,0.466667,0.705882}%
\pgfsetfillcolor{currentfill}%
\pgfsetfillopacity{0.874744}%
\pgfsetlinewidth{1.003750pt}%
\definecolor{currentstroke}{rgb}{0.121569,0.466667,0.705882}%
\pgfsetstrokecolor{currentstroke}%
\pgfsetstrokeopacity{0.874744}%
\pgfsetdash{}{0pt}%
\pgfpathmoveto{\pgfqpoint{2.502540in}{2.392274in}}%
\pgfpathcurveto{\pgfqpoint{2.510776in}{2.392274in}}{\pgfqpoint{2.518676in}{2.395547in}}{\pgfqpoint{2.524500in}{2.401371in}}%
\pgfpathcurveto{\pgfqpoint{2.530324in}{2.407195in}}{\pgfqpoint{2.533596in}{2.415095in}}{\pgfqpoint{2.533596in}{2.423331in}}%
\pgfpathcurveto{\pgfqpoint{2.533596in}{2.431567in}}{\pgfqpoint{2.530324in}{2.439467in}}{\pgfqpoint{2.524500in}{2.445291in}}%
\pgfpathcurveto{\pgfqpoint{2.518676in}{2.451115in}}{\pgfqpoint{2.510776in}{2.454387in}}{\pgfqpoint{2.502540in}{2.454387in}}%
\pgfpathcurveto{\pgfqpoint{2.494303in}{2.454387in}}{\pgfqpoint{2.486403in}{2.451115in}}{\pgfqpoint{2.480579in}{2.445291in}}%
\pgfpathcurveto{\pgfqpoint{2.474755in}{2.439467in}}{\pgfqpoint{2.471483in}{2.431567in}}{\pgfqpoint{2.471483in}{2.423331in}}%
\pgfpathcurveto{\pgfqpoint{2.471483in}{2.415095in}}{\pgfqpoint{2.474755in}{2.407195in}}{\pgfqpoint{2.480579in}{2.401371in}}%
\pgfpathcurveto{\pgfqpoint{2.486403in}{2.395547in}}{\pgfqpoint{2.494303in}{2.392274in}}{\pgfqpoint{2.502540in}{2.392274in}}%
\pgfpathclose%
\pgfusepath{stroke,fill}%
\end{pgfscope}%
\begin{pgfscope}%
\pgfpathrectangle{\pgfqpoint{0.100000in}{0.212622in}}{\pgfqpoint{3.696000in}{3.696000in}}%
\pgfusepath{clip}%
\pgfsetbuttcap%
\pgfsetroundjoin%
\definecolor{currentfill}{rgb}{0.121569,0.466667,0.705882}%
\pgfsetfillcolor{currentfill}%
\pgfsetfillopacity{0.875461}%
\pgfsetlinewidth{1.003750pt}%
\definecolor{currentstroke}{rgb}{0.121569,0.466667,0.705882}%
\pgfsetstrokecolor{currentstroke}%
\pgfsetstrokeopacity{0.875461}%
\pgfsetdash{}{0pt}%
\pgfpathmoveto{\pgfqpoint{1.236241in}{1.801072in}}%
\pgfpathcurveto{\pgfqpoint{1.244478in}{1.801072in}}{\pgfqpoint{1.252378in}{1.804344in}}{\pgfqpoint{1.258202in}{1.810168in}}%
\pgfpathcurveto{\pgfqpoint{1.264026in}{1.815992in}}{\pgfqpoint{1.267298in}{1.823892in}}{\pgfqpoint{1.267298in}{1.832128in}}%
\pgfpathcurveto{\pgfqpoint{1.267298in}{1.840364in}}{\pgfqpoint{1.264026in}{1.848264in}}{\pgfqpoint{1.258202in}{1.854088in}}%
\pgfpathcurveto{\pgfqpoint{1.252378in}{1.859912in}}{\pgfqpoint{1.244478in}{1.863185in}}{\pgfqpoint{1.236241in}{1.863185in}}%
\pgfpathcurveto{\pgfqpoint{1.228005in}{1.863185in}}{\pgfqpoint{1.220105in}{1.859912in}}{\pgfqpoint{1.214281in}{1.854088in}}%
\pgfpathcurveto{\pgfqpoint{1.208457in}{1.848264in}}{\pgfqpoint{1.205185in}{1.840364in}}{\pgfqpoint{1.205185in}{1.832128in}}%
\pgfpathcurveto{\pgfqpoint{1.205185in}{1.823892in}}{\pgfqpoint{1.208457in}{1.815992in}}{\pgfqpoint{1.214281in}{1.810168in}}%
\pgfpathcurveto{\pgfqpoint{1.220105in}{1.804344in}}{\pgfqpoint{1.228005in}{1.801072in}}{\pgfqpoint{1.236241in}{1.801072in}}%
\pgfpathclose%
\pgfusepath{stroke,fill}%
\end{pgfscope}%
\begin{pgfscope}%
\pgfpathrectangle{\pgfqpoint{0.100000in}{0.212622in}}{\pgfqpoint{3.696000in}{3.696000in}}%
\pgfusepath{clip}%
\pgfsetbuttcap%
\pgfsetroundjoin%
\definecolor{currentfill}{rgb}{0.121569,0.466667,0.705882}%
\pgfsetfillcolor{currentfill}%
\pgfsetfillopacity{0.875980}%
\pgfsetlinewidth{1.003750pt}%
\definecolor{currentstroke}{rgb}{0.121569,0.466667,0.705882}%
\pgfsetstrokecolor{currentstroke}%
\pgfsetstrokeopacity{0.875980}%
\pgfsetdash{}{0pt}%
\pgfpathmoveto{\pgfqpoint{2.408709in}{2.342960in}}%
\pgfpathcurveto{\pgfqpoint{2.416945in}{2.342960in}}{\pgfqpoint{2.424845in}{2.346232in}}{\pgfqpoint{2.430669in}{2.352056in}}%
\pgfpathcurveto{\pgfqpoint{2.436493in}{2.357880in}}{\pgfqpoint{2.439765in}{2.365780in}}{\pgfqpoint{2.439765in}{2.374016in}}%
\pgfpathcurveto{\pgfqpoint{2.439765in}{2.382252in}}{\pgfqpoint{2.436493in}{2.390152in}}{\pgfqpoint{2.430669in}{2.395976in}}%
\pgfpathcurveto{\pgfqpoint{2.424845in}{2.401800in}}{\pgfqpoint{2.416945in}{2.405073in}}{\pgfqpoint{2.408709in}{2.405073in}}%
\pgfpathcurveto{\pgfqpoint{2.400473in}{2.405073in}}{\pgfqpoint{2.392573in}{2.401800in}}{\pgfqpoint{2.386749in}{2.395976in}}%
\pgfpathcurveto{\pgfqpoint{2.380925in}{2.390152in}}{\pgfqpoint{2.377652in}{2.382252in}}{\pgfqpoint{2.377652in}{2.374016in}}%
\pgfpathcurveto{\pgfqpoint{2.377652in}{2.365780in}}{\pgfqpoint{2.380925in}{2.357880in}}{\pgfqpoint{2.386749in}{2.352056in}}%
\pgfpathcurveto{\pgfqpoint{2.392573in}{2.346232in}}{\pgfqpoint{2.400473in}{2.342960in}}{\pgfqpoint{2.408709in}{2.342960in}}%
\pgfpathclose%
\pgfusepath{stroke,fill}%
\end{pgfscope}%
\begin{pgfscope}%
\pgfpathrectangle{\pgfqpoint{0.100000in}{0.212622in}}{\pgfqpoint{3.696000in}{3.696000in}}%
\pgfusepath{clip}%
\pgfsetbuttcap%
\pgfsetroundjoin%
\definecolor{currentfill}{rgb}{0.121569,0.466667,0.705882}%
\pgfsetfillcolor{currentfill}%
\pgfsetfillopacity{0.877216}%
\pgfsetlinewidth{1.003750pt}%
\definecolor{currentstroke}{rgb}{0.121569,0.466667,0.705882}%
\pgfsetstrokecolor{currentstroke}%
\pgfsetstrokeopacity{0.877216}%
\pgfsetdash{}{0pt}%
\pgfpathmoveto{\pgfqpoint{2.471797in}{2.364852in}}%
\pgfpathcurveto{\pgfqpoint{2.480033in}{2.364852in}}{\pgfqpoint{2.487933in}{2.368124in}}{\pgfqpoint{2.493757in}{2.373948in}}%
\pgfpathcurveto{\pgfqpoint{2.499581in}{2.379772in}}{\pgfqpoint{2.502854in}{2.387672in}}{\pgfqpoint{2.502854in}{2.395908in}}%
\pgfpathcurveto{\pgfqpoint{2.502854in}{2.404145in}}{\pgfqpoint{2.499581in}{2.412045in}}{\pgfqpoint{2.493757in}{2.417869in}}%
\pgfpathcurveto{\pgfqpoint{2.487933in}{2.423692in}}{\pgfqpoint{2.480033in}{2.426965in}}{\pgfqpoint{2.471797in}{2.426965in}}%
\pgfpathcurveto{\pgfqpoint{2.463561in}{2.426965in}}{\pgfqpoint{2.455661in}{2.423692in}}{\pgfqpoint{2.449837in}{2.417869in}}%
\pgfpathcurveto{\pgfqpoint{2.444013in}{2.412045in}}{\pgfqpoint{2.440741in}{2.404145in}}{\pgfqpoint{2.440741in}{2.395908in}}%
\pgfpathcurveto{\pgfqpoint{2.440741in}{2.387672in}}{\pgfqpoint{2.444013in}{2.379772in}}{\pgfqpoint{2.449837in}{2.373948in}}%
\pgfpathcurveto{\pgfqpoint{2.455661in}{2.368124in}}{\pgfqpoint{2.463561in}{2.364852in}}{\pgfqpoint{2.471797in}{2.364852in}}%
\pgfpathclose%
\pgfusepath{stroke,fill}%
\end{pgfscope}%
\begin{pgfscope}%
\pgfpathrectangle{\pgfqpoint{0.100000in}{0.212622in}}{\pgfqpoint{3.696000in}{3.696000in}}%
\pgfusepath{clip}%
\pgfsetbuttcap%
\pgfsetroundjoin%
\definecolor{currentfill}{rgb}{0.121569,0.466667,0.705882}%
\pgfsetfillcolor{currentfill}%
\pgfsetfillopacity{0.877380}%
\pgfsetlinewidth{1.003750pt}%
\definecolor{currentstroke}{rgb}{0.121569,0.466667,0.705882}%
\pgfsetstrokecolor{currentstroke}%
\pgfsetstrokeopacity{0.877380}%
\pgfsetdash{}{0pt}%
\pgfpathmoveto{\pgfqpoint{3.080219in}{2.629762in}}%
\pgfpathcurveto{\pgfqpoint{3.088455in}{2.629762in}}{\pgfqpoint{3.096355in}{2.633034in}}{\pgfqpoint{3.102179in}{2.638858in}}%
\pgfpathcurveto{\pgfqpoint{3.108003in}{2.644682in}}{\pgfqpoint{3.111275in}{2.652582in}}{\pgfqpoint{3.111275in}{2.660818in}}%
\pgfpathcurveto{\pgfqpoint{3.111275in}{2.669054in}}{\pgfqpoint{3.108003in}{2.676954in}}{\pgfqpoint{3.102179in}{2.682778in}}%
\pgfpathcurveto{\pgfqpoint{3.096355in}{2.688602in}}{\pgfqpoint{3.088455in}{2.691875in}}{\pgfqpoint{3.080219in}{2.691875in}}%
\pgfpathcurveto{\pgfqpoint{3.071983in}{2.691875in}}{\pgfqpoint{3.064083in}{2.688602in}}{\pgfqpoint{3.058259in}{2.682778in}}%
\pgfpathcurveto{\pgfqpoint{3.052435in}{2.676954in}}{\pgfqpoint{3.049162in}{2.669054in}}{\pgfqpoint{3.049162in}{2.660818in}}%
\pgfpathcurveto{\pgfqpoint{3.049162in}{2.652582in}}{\pgfqpoint{3.052435in}{2.644682in}}{\pgfqpoint{3.058259in}{2.638858in}}%
\pgfpathcurveto{\pgfqpoint{3.064083in}{2.633034in}}{\pgfqpoint{3.071983in}{2.629762in}}{\pgfqpoint{3.080219in}{2.629762in}}%
\pgfpathclose%
\pgfusepath{stroke,fill}%
\end{pgfscope}%
\begin{pgfscope}%
\pgfpathrectangle{\pgfqpoint{0.100000in}{0.212622in}}{\pgfqpoint{3.696000in}{3.696000in}}%
\pgfusepath{clip}%
\pgfsetbuttcap%
\pgfsetroundjoin%
\definecolor{currentfill}{rgb}{0.121569,0.466667,0.705882}%
\pgfsetfillcolor{currentfill}%
\pgfsetfillopacity{0.877441}%
\pgfsetlinewidth{1.003750pt}%
\definecolor{currentstroke}{rgb}{0.121569,0.466667,0.705882}%
\pgfsetstrokecolor{currentstroke}%
\pgfsetstrokeopacity{0.877441}%
\pgfsetdash{}{0pt}%
\pgfpathmoveto{\pgfqpoint{2.428201in}{2.352463in}}%
\pgfpathcurveto{\pgfqpoint{2.436437in}{2.352463in}}{\pgfqpoint{2.444337in}{2.355736in}}{\pgfqpoint{2.450161in}{2.361560in}}%
\pgfpathcurveto{\pgfqpoint{2.455985in}{2.367384in}}{\pgfqpoint{2.459258in}{2.375284in}}{\pgfqpoint{2.459258in}{2.383520in}}%
\pgfpathcurveto{\pgfqpoint{2.459258in}{2.391756in}}{\pgfqpoint{2.455985in}{2.399656in}}{\pgfqpoint{2.450161in}{2.405480in}}%
\pgfpathcurveto{\pgfqpoint{2.444337in}{2.411304in}}{\pgfqpoint{2.436437in}{2.414576in}}{\pgfqpoint{2.428201in}{2.414576in}}%
\pgfpathcurveto{\pgfqpoint{2.419965in}{2.414576in}}{\pgfqpoint{2.412065in}{2.411304in}}{\pgfqpoint{2.406241in}{2.405480in}}%
\pgfpathcurveto{\pgfqpoint{2.400417in}{2.399656in}}{\pgfqpoint{2.397145in}{2.391756in}}{\pgfqpoint{2.397145in}{2.383520in}}%
\pgfpathcurveto{\pgfqpoint{2.397145in}{2.375284in}}{\pgfqpoint{2.400417in}{2.367384in}}{\pgfqpoint{2.406241in}{2.361560in}}%
\pgfpathcurveto{\pgfqpoint{2.412065in}{2.355736in}}{\pgfqpoint{2.419965in}{2.352463in}}{\pgfqpoint{2.428201in}{2.352463in}}%
\pgfpathclose%
\pgfusepath{stroke,fill}%
\end{pgfscope}%
\begin{pgfscope}%
\pgfpathrectangle{\pgfqpoint{0.100000in}{0.212622in}}{\pgfqpoint{3.696000in}{3.696000in}}%
\pgfusepath{clip}%
\pgfsetbuttcap%
\pgfsetroundjoin%
\definecolor{currentfill}{rgb}{0.121569,0.466667,0.705882}%
\pgfsetfillcolor{currentfill}%
\pgfsetfillopacity{0.877764}%
\pgfsetlinewidth{1.003750pt}%
\definecolor{currentstroke}{rgb}{0.121569,0.466667,0.705882}%
\pgfsetstrokecolor{currentstroke}%
\pgfsetstrokeopacity{0.877764}%
\pgfsetdash{}{0pt}%
\pgfpathmoveto{\pgfqpoint{1.221672in}{1.801476in}}%
\pgfpathcurveto{\pgfqpoint{1.229908in}{1.801476in}}{\pgfqpoint{1.237808in}{1.804749in}}{\pgfqpoint{1.243632in}{1.810573in}}%
\pgfpathcurveto{\pgfqpoint{1.249456in}{1.816396in}}{\pgfqpoint{1.252728in}{1.824296in}}{\pgfqpoint{1.252728in}{1.832533in}}%
\pgfpathcurveto{\pgfqpoint{1.252728in}{1.840769in}}{\pgfqpoint{1.249456in}{1.848669in}}{\pgfqpoint{1.243632in}{1.854493in}}%
\pgfpathcurveto{\pgfqpoint{1.237808in}{1.860317in}}{\pgfqpoint{1.229908in}{1.863589in}}{\pgfqpoint{1.221672in}{1.863589in}}%
\pgfpathcurveto{\pgfqpoint{1.213435in}{1.863589in}}{\pgfqpoint{1.205535in}{1.860317in}}{\pgfqpoint{1.199711in}{1.854493in}}%
\pgfpathcurveto{\pgfqpoint{1.193888in}{1.848669in}}{\pgfqpoint{1.190615in}{1.840769in}}{\pgfqpoint{1.190615in}{1.832533in}}%
\pgfpathcurveto{\pgfqpoint{1.190615in}{1.824296in}}{\pgfqpoint{1.193888in}{1.816396in}}{\pgfqpoint{1.199711in}{1.810573in}}%
\pgfpathcurveto{\pgfqpoint{1.205535in}{1.804749in}}{\pgfqpoint{1.213435in}{1.801476in}}{\pgfqpoint{1.221672in}{1.801476in}}%
\pgfpathclose%
\pgfusepath{stroke,fill}%
\end{pgfscope}%
\begin{pgfscope}%
\pgfpathrectangle{\pgfqpoint{0.100000in}{0.212622in}}{\pgfqpoint{3.696000in}{3.696000in}}%
\pgfusepath{clip}%
\pgfsetbuttcap%
\pgfsetroundjoin%
\definecolor{currentfill}{rgb}{0.121569,0.466667,0.705882}%
\pgfsetfillcolor{currentfill}%
\pgfsetfillopacity{0.878368}%
\pgfsetlinewidth{1.003750pt}%
\definecolor{currentstroke}{rgb}{0.121569,0.466667,0.705882}%
\pgfsetstrokecolor{currentstroke}%
\pgfsetstrokeopacity{0.878368}%
\pgfsetdash{}{0pt}%
\pgfpathmoveto{\pgfqpoint{2.453779in}{2.361696in}}%
\pgfpathcurveto{\pgfqpoint{2.462015in}{2.361696in}}{\pgfqpoint{2.469915in}{2.364969in}}{\pgfqpoint{2.475739in}{2.370793in}}%
\pgfpathcurveto{\pgfqpoint{2.481563in}{2.376617in}}{\pgfqpoint{2.484836in}{2.384517in}}{\pgfqpoint{2.484836in}{2.392753in}}%
\pgfpathcurveto{\pgfqpoint{2.484836in}{2.400989in}}{\pgfqpoint{2.481563in}{2.408889in}}{\pgfqpoint{2.475739in}{2.414713in}}%
\pgfpathcurveto{\pgfqpoint{2.469915in}{2.420537in}}{\pgfqpoint{2.462015in}{2.423809in}}{\pgfqpoint{2.453779in}{2.423809in}}%
\pgfpathcurveto{\pgfqpoint{2.445543in}{2.423809in}}{\pgfqpoint{2.437643in}{2.420537in}}{\pgfqpoint{2.431819in}{2.414713in}}%
\pgfpathcurveto{\pgfqpoint{2.425995in}{2.408889in}}{\pgfqpoint{2.422723in}{2.400989in}}{\pgfqpoint{2.422723in}{2.392753in}}%
\pgfpathcurveto{\pgfqpoint{2.422723in}{2.384517in}}{\pgfqpoint{2.425995in}{2.376617in}}{\pgfqpoint{2.431819in}{2.370793in}}%
\pgfpathcurveto{\pgfqpoint{2.437643in}{2.364969in}}{\pgfqpoint{2.445543in}{2.361696in}}{\pgfqpoint{2.453779in}{2.361696in}}%
\pgfpathclose%
\pgfusepath{stroke,fill}%
\end{pgfscope}%
\begin{pgfscope}%
\pgfpathrectangle{\pgfqpoint{0.100000in}{0.212622in}}{\pgfqpoint{3.696000in}{3.696000in}}%
\pgfusepath{clip}%
\pgfsetbuttcap%
\pgfsetroundjoin%
\definecolor{currentfill}{rgb}{0.121569,0.466667,0.705882}%
\pgfsetfillcolor{currentfill}%
\pgfsetfillopacity{0.878846}%
\pgfsetlinewidth{1.003750pt}%
\definecolor{currentstroke}{rgb}{0.121569,0.466667,0.705882}%
\pgfsetstrokecolor{currentstroke}%
\pgfsetstrokeopacity{0.878846}%
\pgfsetdash{}{0pt}%
\pgfpathmoveto{\pgfqpoint{2.431144in}{2.346923in}}%
\pgfpathcurveto{\pgfqpoint{2.439381in}{2.346923in}}{\pgfqpoint{2.447281in}{2.350196in}}{\pgfqpoint{2.453105in}{2.356020in}}%
\pgfpathcurveto{\pgfqpoint{2.458929in}{2.361844in}}{\pgfqpoint{2.462201in}{2.369744in}}{\pgfqpoint{2.462201in}{2.377980in}}%
\pgfpathcurveto{\pgfqpoint{2.462201in}{2.386216in}}{\pgfqpoint{2.458929in}{2.394116in}}{\pgfqpoint{2.453105in}{2.399940in}}%
\pgfpathcurveto{\pgfqpoint{2.447281in}{2.405764in}}{\pgfqpoint{2.439381in}{2.409036in}}{\pgfqpoint{2.431144in}{2.409036in}}%
\pgfpathcurveto{\pgfqpoint{2.422908in}{2.409036in}}{\pgfqpoint{2.415008in}{2.405764in}}{\pgfqpoint{2.409184in}{2.399940in}}%
\pgfpathcurveto{\pgfqpoint{2.403360in}{2.394116in}}{\pgfqpoint{2.400088in}{2.386216in}}{\pgfqpoint{2.400088in}{2.377980in}}%
\pgfpathcurveto{\pgfqpoint{2.400088in}{2.369744in}}{\pgfqpoint{2.403360in}{2.361844in}}{\pgfqpoint{2.409184in}{2.356020in}}%
\pgfpathcurveto{\pgfqpoint{2.415008in}{2.350196in}}{\pgfqpoint{2.422908in}{2.346923in}}{\pgfqpoint{2.431144in}{2.346923in}}%
\pgfpathclose%
\pgfusepath{stroke,fill}%
\end{pgfscope}%
\begin{pgfscope}%
\pgfpathrectangle{\pgfqpoint{0.100000in}{0.212622in}}{\pgfqpoint{3.696000in}{3.696000in}}%
\pgfusepath{clip}%
\pgfsetbuttcap%
\pgfsetroundjoin%
\definecolor{currentfill}{rgb}{0.121569,0.466667,0.705882}%
\pgfsetfillcolor{currentfill}%
\pgfsetfillopacity{0.879341}%
\pgfsetlinewidth{1.003750pt}%
\definecolor{currentstroke}{rgb}{0.121569,0.466667,0.705882}%
\pgfsetstrokecolor{currentstroke}%
\pgfsetstrokeopacity{0.879341}%
\pgfsetdash{}{0pt}%
\pgfpathmoveto{\pgfqpoint{2.472492in}{2.363074in}}%
\pgfpathcurveto{\pgfqpoint{2.480728in}{2.363074in}}{\pgfqpoint{2.488628in}{2.366346in}}{\pgfqpoint{2.494452in}{2.372170in}}%
\pgfpathcurveto{\pgfqpoint{2.500276in}{2.377994in}}{\pgfqpoint{2.503548in}{2.385894in}}{\pgfqpoint{2.503548in}{2.394130in}}%
\pgfpathcurveto{\pgfqpoint{2.503548in}{2.402366in}}{\pgfqpoint{2.500276in}{2.410266in}}{\pgfqpoint{2.494452in}{2.416090in}}%
\pgfpathcurveto{\pgfqpoint{2.488628in}{2.421914in}}{\pgfqpoint{2.480728in}{2.425187in}}{\pgfqpoint{2.472492in}{2.425187in}}%
\pgfpathcurveto{\pgfqpoint{2.464256in}{2.425187in}}{\pgfqpoint{2.456356in}{2.421914in}}{\pgfqpoint{2.450532in}{2.416090in}}%
\pgfpathcurveto{\pgfqpoint{2.444708in}{2.410266in}}{\pgfqpoint{2.441435in}{2.402366in}}{\pgfqpoint{2.441435in}{2.394130in}}%
\pgfpathcurveto{\pgfqpoint{2.441435in}{2.385894in}}{\pgfqpoint{2.444708in}{2.377994in}}{\pgfqpoint{2.450532in}{2.372170in}}%
\pgfpathcurveto{\pgfqpoint{2.456356in}{2.366346in}}{\pgfqpoint{2.464256in}{2.363074in}}{\pgfqpoint{2.472492in}{2.363074in}}%
\pgfpathclose%
\pgfusepath{stroke,fill}%
\end{pgfscope}%
\begin{pgfscope}%
\pgfpathrectangle{\pgfqpoint{0.100000in}{0.212622in}}{\pgfqpoint{3.696000in}{3.696000in}}%
\pgfusepath{clip}%
\pgfsetbuttcap%
\pgfsetroundjoin%
\definecolor{currentfill}{rgb}{0.121569,0.466667,0.705882}%
\pgfsetfillcolor{currentfill}%
\pgfsetfillopacity{0.879514}%
\pgfsetlinewidth{1.003750pt}%
\definecolor{currentstroke}{rgb}{0.121569,0.466667,0.705882}%
\pgfsetstrokecolor{currentstroke}%
\pgfsetstrokeopacity{0.879514}%
\pgfsetdash{}{0pt}%
\pgfpathmoveto{\pgfqpoint{2.475134in}{2.363606in}}%
\pgfpathcurveto{\pgfqpoint{2.483371in}{2.363606in}}{\pgfqpoint{2.491271in}{2.366878in}}{\pgfqpoint{2.497094in}{2.372702in}}%
\pgfpathcurveto{\pgfqpoint{2.502918in}{2.378526in}}{\pgfqpoint{2.506191in}{2.386426in}}{\pgfqpoint{2.506191in}{2.394662in}}%
\pgfpathcurveto{\pgfqpoint{2.506191in}{2.402898in}}{\pgfqpoint{2.502918in}{2.410798in}}{\pgfqpoint{2.497094in}{2.416622in}}%
\pgfpathcurveto{\pgfqpoint{2.491271in}{2.422446in}}{\pgfqpoint{2.483371in}{2.425719in}}{\pgfqpoint{2.475134in}{2.425719in}}%
\pgfpathcurveto{\pgfqpoint{2.466898in}{2.425719in}}{\pgfqpoint{2.458998in}{2.422446in}}{\pgfqpoint{2.453174in}{2.416622in}}%
\pgfpathcurveto{\pgfqpoint{2.447350in}{2.410798in}}{\pgfqpoint{2.444078in}{2.402898in}}{\pgfqpoint{2.444078in}{2.394662in}}%
\pgfpathcurveto{\pgfqpoint{2.444078in}{2.386426in}}{\pgfqpoint{2.447350in}{2.378526in}}{\pgfqpoint{2.453174in}{2.372702in}}%
\pgfpathcurveto{\pgfqpoint{2.458998in}{2.366878in}}{\pgfqpoint{2.466898in}{2.363606in}}{\pgfqpoint{2.475134in}{2.363606in}}%
\pgfpathclose%
\pgfusepath{stroke,fill}%
\end{pgfscope}%
\begin{pgfscope}%
\pgfpathrectangle{\pgfqpoint{0.100000in}{0.212622in}}{\pgfqpoint{3.696000in}{3.696000in}}%
\pgfusepath{clip}%
\pgfsetbuttcap%
\pgfsetroundjoin%
\definecolor{currentfill}{rgb}{0.121569,0.466667,0.705882}%
\pgfsetfillcolor{currentfill}%
\pgfsetfillopacity{0.879985}%
\pgfsetlinewidth{1.003750pt}%
\definecolor{currentstroke}{rgb}{0.121569,0.466667,0.705882}%
\pgfsetstrokecolor{currentstroke}%
\pgfsetstrokeopacity{0.879985}%
\pgfsetdash{}{0pt}%
\pgfpathmoveto{\pgfqpoint{3.080852in}{2.629748in}}%
\pgfpathcurveto{\pgfqpoint{3.089088in}{2.629748in}}{\pgfqpoint{3.096988in}{2.633020in}}{\pgfqpoint{3.102812in}{2.638844in}}%
\pgfpathcurveto{\pgfqpoint{3.108636in}{2.644668in}}{\pgfqpoint{3.111908in}{2.652568in}}{\pgfqpoint{3.111908in}{2.660804in}}%
\pgfpathcurveto{\pgfqpoint{3.111908in}{2.669040in}}{\pgfqpoint{3.108636in}{2.676941in}}{\pgfqpoint{3.102812in}{2.682764in}}%
\pgfpathcurveto{\pgfqpoint{3.096988in}{2.688588in}}{\pgfqpoint{3.089088in}{2.691861in}}{\pgfqpoint{3.080852in}{2.691861in}}%
\pgfpathcurveto{\pgfqpoint{3.072615in}{2.691861in}}{\pgfqpoint{3.064715in}{2.688588in}}{\pgfqpoint{3.058891in}{2.682764in}}%
\pgfpathcurveto{\pgfqpoint{3.053067in}{2.676941in}}{\pgfqpoint{3.049795in}{2.669040in}}{\pgfqpoint{3.049795in}{2.660804in}}%
\pgfpathcurveto{\pgfqpoint{3.049795in}{2.652568in}}{\pgfqpoint{3.053067in}{2.644668in}}{\pgfqpoint{3.058891in}{2.638844in}}%
\pgfpathcurveto{\pgfqpoint{3.064715in}{2.633020in}}{\pgfqpoint{3.072615in}{2.629748in}}{\pgfqpoint{3.080852in}{2.629748in}}%
\pgfpathclose%
\pgfusepath{stroke,fill}%
\end{pgfscope}%
\begin{pgfscope}%
\pgfpathrectangle{\pgfqpoint{0.100000in}{0.212622in}}{\pgfqpoint{3.696000in}{3.696000in}}%
\pgfusepath{clip}%
\pgfsetbuttcap%
\pgfsetroundjoin%
\definecolor{currentfill}{rgb}{0.121569,0.466667,0.705882}%
\pgfsetfillcolor{currentfill}%
\pgfsetfillopacity{0.880034}%
\pgfsetlinewidth{1.003750pt}%
\definecolor{currentstroke}{rgb}{0.121569,0.466667,0.705882}%
\pgfsetstrokecolor{currentstroke}%
\pgfsetstrokeopacity{0.880034}%
\pgfsetdash{}{0pt}%
\pgfpathmoveto{\pgfqpoint{2.469353in}{2.361745in}}%
\pgfpathcurveto{\pgfqpoint{2.477589in}{2.361745in}}{\pgfqpoint{2.485489in}{2.365017in}}{\pgfqpoint{2.491313in}{2.370841in}}%
\pgfpathcurveto{\pgfqpoint{2.497137in}{2.376665in}}{\pgfqpoint{2.500409in}{2.384565in}}{\pgfqpoint{2.500409in}{2.392801in}}%
\pgfpathcurveto{\pgfqpoint{2.500409in}{2.401037in}}{\pgfqpoint{2.497137in}{2.408937in}}{\pgfqpoint{2.491313in}{2.414761in}}%
\pgfpathcurveto{\pgfqpoint{2.485489in}{2.420585in}}{\pgfqpoint{2.477589in}{2.423858in}}{\pgfqpoint{2.469353in}{2.423858in}}%
\pgfpathcurveto{\pgfqpoint{2.461116in}{2.423858in}}{\pgfqpoint{2.453216in}{2.420585in}}{\pgfqpoint{2.447392in}{2.414761in}}%
\pgfpathcurveto{\pgfqpoint{2.441568in}{2.408937in}}{\pgfqpoint{2.438296in}{2.401037in}}{\pgfqpoint{2.438296in}{2.392801in}}%
\pgfpathcurveto{\pgfqpoint{2.438296in}{2.384565in}}{\pgfqpoint{2.441568in}{2.376665in}}{\pgfqpoint{2.447392in}{2.370841in}}%
\pgfpathcurveto{\pgfqpoint{2.453216in}{2.365017in}}{\pgfqpoint{2.461116in}{2.361745in}}{\pgfqpoint{2.469353in}{2.361745in}}%
\pgfpathclose%
\pgfusepath{stroke,fill}%
\end{pgfscope}%
\begin{pgfscope}%
\pgfpathrectangle{\pgfqpoint{0.100000in}{0.212622in}}{\pgfqpoint{3.696000in}{3.696000in}}%
\pgfusepath{clip}%
\pgfsetbuttcap%
\pgfsetroundjoin%
\definecolor{currentfill}{rgb}{0.121569,0.466667,0.705882}%
\pgfsetfillcolor{currentfill}%
\pgfsetfillopacity{0.880265}%
\pgfsetlinewidth{1.003750pt}%
\definecolor{currentstroke}{rgb}{0.121569,0.466667,0.705882}%
\pgfsetstrokecolor{currentstroke}%
\pgfsetstrokeopacity{0.880265}%
\pgfsetdash{}{0pt}%
\pgfpathmoveto{\pgfqpoint{2.477955in}{2.362014in}}%
\pgfpathcurveto{\pgfqpoint{2.486191in}{2.362014in}}{\pgfqpoint{2.494091in}{2.365286in}}{\pgfqpoint{2.499915in}{2.371110in}}%
\pgfpathcurveto{\pgfqpoint{2.505739in}{2.376934in}}{\pgfqpoint{2.509011in}{2.384834in}}{\pgfqpoint{2.509011in}{2.393071in}}%
\pgfpathcurveto{\pgfqpoint{2.509011in}{2.401307in}}{\pgfqpoint{2.505739in}{2.409207in}}{\pgfqpoint{2.499915in}{2.415031in}}%
\pgfpathcurveto{\pgfqpoint{2.494091in}{2.420855in}}{\pgfqpoint{2.486191in}{2.424127in}}{\pgfqpoint{2.477955in}{2.424127in}}%
\pgfpathcurveto{\pgfqpoint{2.469718in}{2.424127in}}{\pgfqpoint{2.461818in}{2.420855in}}{\pgfqpoint{2.455994in}{2.415031in}}%
\pgfpathcurveto{\pgfqpoint{2.450170in}{2.409207in}}{\pgfqpoint{2.446898in}{2.401307in}}{\pgfqpoint{2.446898in}{2.393071in}}%
\pgfpathcurveto{\pgfqpoint{2.446898in}{2.384834in}}{\pgfqpoint{2.450170in}{2.376934in}}{\pgfqpoint{2.455994in}{2.371110in}}%
\pgfpathcurveto{\pgfqpoint{2.461818in}{2.365286in}}{\pgfqpoint{2.469718in}{2.362014in}}{\pgfqpoint{2.477955in}{2.362014in}}%
\pgfpathclose%
\pgfusepath{stroke,fill}%
\end{pgfscope}%
\begin{pgfscope}%
\pgfpathrectangle{\pgfqpoint{0.100000in}{0.212622in}}{\pgfqpoint{3.696000in}{3.696000in}}%
\pgfusepath{clip}%
\pgfsetbuttcap%
\pgfsetroundjoin%
\definecolor{currentfill}{rgb}{0.121569,0.466667,0.705882}%
\pgfsetfillcolor{currentfill}%
\pgfsetfillopacity{0.881352}%
\pgfsetlinewidth{1.003750pt}%
\definecolor{currentstroke}{rgb}{0.121569,0.466667,0.705882}%
\pgfsetstrokecolor{currentstroke}%
\pgfsetstrokeopacity{0.881352}%
\pgfsetdash{}{0pt}%
\pgfpathmoveto{\pgfqpoint{3.080557in}{2.627162in}}%
\pgfpathcurveto{\pgfqpoint{3.088794in}{2.627162in}}{\pgfqpoint{3.096694in}{2.630434in}}{\pgfqpoint{3.102518in}{2.636258in}}%
\pgfpathcurveto{\pgfqpoint{3.108341in}{2.642082in}}{\pgfqpoint{3.111614in}{2.649982in}}{\pgfqpoint{3.111614in}{2.658218in}}%
\pgfpathcurveto{\pgfqpoint{3.111614in}{2.666454in}}{\pgfqpoint{3.108341in}{2.674354in}}{\pgfqpoint{3.102518in}{2.680178in}}%
\pgfpathcurveto{\pgfqpoint{3.096694in}{2.686002in}}{\pgfqpoint{3.088794in}{2.689275in}}{\pgfqpoint{3.080557in}{2.689275in}}%
\pgfpathcurveto{\pgfqpoint{3.072321in}{2.689275in}}{\pgfqpoint{3.064421in}{2.686002in}}{\pgfqpoint{3.058597in}{2.680178in}}%
\pgfpathcurveto{\pgfqpoint{3.052773in}{2.674354in}}{\pgfqpoint{3.049501in}{2.666454in}}{\pgfqpoint{3.049501in}{2.658218in}}%
\pgfpathcurveto{\pgfqpoint{3.049501in}{2.649982in}}{\pgfqpoint{3.052773in}{2.642082in}}{\pgfqpoint{3.058597in}{2.636258in}}%
\pgfpathcurveto{\pgfqpoint{3.064421in}{2.630434in}}{\pgfqpoint{3.072321in}{2.627162in}}{\pgfqpoint{3.080557in}{2.627162in}}%
\pgfpathclose%
\pgfusepath{stroke,fill}%
\end{pgfscope}%
\begin{pgfscope}%
\pgfpathrectangle{\pgfqpoint{0.100000in}{0.212622in}}{\pgfqpoint{3.696000in}{3.696000in}}%
\pgfusepath{clip}%
\pgfsetbuttcap%
\pgfsetroundjoin%
\definecolor{currentfill}{rgb}{0.121569,0.466667,0.705882}%
\pgfsetfillcolor{currentfill}%
\pgfsetfillopacity{0.882240}%
\pgfsetlinewidth{1.003750pt}%
\definecolor{currentstroke}{rgb}{0.121569,0.466667,0.705882}%
\pgfsetstrokecolor{currentstroke}%
\pgfsetstrokeopacity{0.882240}%
\pgfsetdash{}{0pt}%
\pgfpathmoveto{\pgfqpoint{2.387930in}{2.333667in}}%
\pgfpathcurveto{\pgfqpoint{2.396166in}{2.333667in}}{\pgfqpoint{2.404066in}{2.336939in}}{\pgfqpoint{2.409890in}{2.342763in}}%
\pgfpathcurveto{\pgfqpoint{2.415714in}{2.348587in}}{\pgfqpoint{2.418987in}{2.356487in}}{\pgfqpoint{2.418987in}{2.364723in}}%
\pgfpathcurveto{\pgfqpoint{2.418987in}{2.372960in}}{\pgfqpoint{2.415714in}{2.380860in}}{\pgfqpoint{2.409890in}{2.386684in}}%
\pgfpathcurveto{\pgfqpoint{2.404066in}{2.392507in}}{\pgfqpoint{2.396166in}{2.395780in}}{\pgfqpoint{2.387930in}{2.395780in}}%
\pgfpathcurveto{\pgfqpoint{2.379694in}{2.395780in}}{\pgfqpoint{2.371794in}{2.392507in}}{\pgfqpoint{2.365970in}{2.386684in}}%
\pgfpathcurveto{\pgfqpoint{2.360146in}{2.380860in}}{\pgfqpoint{2.356874in}{2.372960in}}{\pgfqpoint{2.356874in}{2.364723in}}%
\pgfpathcurveto{\pgfqpoint{2.356874in}{2.356487in}}{\pgfqpoint{2.360146in}{2.348587in}}{\pgfqpoint{2.365970in}{2.342763in}}%
\pgfpathcurveto{\pgfqpoint{2.371794in}{2.336939in}}{\pgfqpoint{2.379694in}{2.333667in}}{\pgfqpoint{2.387930in}{2.333667in}}%
\pgfpathclose%
\pgfusepath{stroke,fill}%
\end{pgfscope}%
\begin{pgfscope}%
\pgfpathrectangle{\pgfqpoint{0.100000in}{0.212622in}}{\pgfqpoint{3.696000in}{3.696000in}}%
\pgfusepath{clip}%
\pgfsetbuttcap%
\pgfsetroundjoin%
\definecolor{currentfill}{rgb}{0.121569,0.466667,0.705882}%
\pgfsetfillcolor{currentfill}%
\pgfsetfillopacity{0.882976}%
\pgfsetlinewidth{1.003750pt}%
\definecolor{currentstroke}{rgb}{0.121569,0.466667,0.705882}%
\pgfsetstrokecolor{currentstroke}%
\pgfsetstrokeopacity{0.882976}%
\pgfsetdash{}{0pt}%
\pgfpathmoveto{\pgfqpoint{2.432523in}{2.343003in}}%
\pgfpathcurveto{\pgfqpoint{2.440760in}{2.343003in}}{\pgfqpoint{2.448660in}{2.346275in}}{\pgfqpoint{2.454484in}{2.352099in}}%
\pgfpathcurveto{\pgfqpoint{2.460308in}{2.357923in}}{\pgfqpoint{2.463580in}{2.365823in}}{\pgfqpoint{2.463580in}{2.374060in}}%
\pgfpathcurveto{\pgfqpoint{2.463580in}{2.382296in}}{\pgfqpoint{2.460308in}{2.390196in}}{\pgfqpoint{2.454484in}{2.396020in}}%
\pgfpathcurveto{\pgfqpoint{2.448660in}{2.401844in}}{\pgfqpoint{2.440760in}{2.405116in}}{\pgfqpoint{2.432523in}{2.405116in}}%
\pgfpathcurveto{\pgfqpoint{2.424287in}{2.405116in}}{\pgfqpoint{2.416387in}{2.401844in}}{\pgfqpoint{2.410563in}{2.396020in}}%
\pgfpathcurveto{\pgfqpoint{2.404739in}{2.390196in}}{\pgfqpoint{2.401467in}{2.382296in}}{\pgfqpoint{2.401467in}{2.374060in}}%
\pgfpathcurveto{\pgfqpoint{2.401467in}{2.365823in}}{\pgfqpoint{2.404739in}{2.357923in}}{\pgfqpoint{2.410563in}{2.352099in}}%
\pgfpathcurveto{\pgfqpoint{2.416387in}{2.346275in}}{\pgfqpoint{2.424287in}{2.343003in}}{\pgfqpoint{2.432523in}{2.343003in}}%
\pgfpathclose%
\pgfusepath{stroke,fill}%
\end{pgfscope}%
\begin{pgfscope}%
\pgfpathrectangle{\pgfqpoint{0.100000in}{0.212622in}}{\pgfqpoint{3.696000in}{3.696000in}}%
\pgfusepath{clip}%
\pgfsetbuttcap%
\pgfsetroundjoin%
\definecolor{currentfill}{rgb}{0.121569,0.466667,0.705882}%
\pgfsetfillcolor{currentfill}%
\pgfsetfillopacity{0.883363}%
\pgfsetlinewidth{1.003750pt}%
\definecolor{currentstroke}{rgb}{0.121569,0.466667,0.705882}%
\pgfsetstrokecolor{currentstroke}%
\pgfsetstrokeopacity{0.883363}%
\pgfsetdash{}{0pt}%
\pgfpathmoveto{\pgfqpoint{3.078709in}{2.626526in}}%
\pgfpathcurveto{\pgfqpoint{3.086946in}{2.626526in}}{\pgfqpoint{3.094846in}{2.629798in}}{\pgfqpoint{3.100670in}{2.635622in}}%
\pgfpathcurveto{\pgfqpoint{3.106494in}{2.641446in}}{\pgfqpoint{3.109766in}{2.649346in}}{\pgfqpoint{3.109766in}{2.657582in}}%
\pgfpathcurveto{\pgfqpoint{3.109766in}{2.665819in}}{\pgfqpoint{3.106494in}{2.673719in}}{\pgfqpoint{3.100670in}{2.679543in}}%
\pgfpathcurveto{\pgfqpoint{3.094846in}{2.685367in}}{\pgfqpoint{3.086946in}{2.688639in}}{\pgfqpoint{3.078709in}{2.688639in}}%
\pgfpathcurveto{\pgfqpoint{3.070473in}{2.688639in}}{\pgfqpoint{3.062573in}{2.685367in}}{\pgfqpoint{3.056749in}{2.679543in}}%
\pgfpathcurveto{\pgfqpoint{3.050925in}{2.673719in}}{\pgfqpoint{3.047653in}{2.665819in}}{\pgfqpoint{3.047653in}{2.657582in}}%
\pgfpathcurveto{\pgfqpoint{3.047653in}{2.649346in}}{\pgfqpoint{3.050925in}{2.641446in}}{\pgfqpoint{3.056749in}{2.635622in}}%
\pgfpathcurveto{\pgfqpoint{3.062573in}{2.629798in}}{\pgfqpoint{3.070473in}{2.626526in}}{\pgfqpoint{3.078709in}{2.626526in}}%
\pgfpathclose%
\pgfusepath{stroke,fill}%
\end{pgfscope}%
\begin{pgfscope}%
\pgfpathrectangle{\pgfqpoint{0.100000in}{0.212622in}}{\pgfqpoint{3.696000in}{3.696000in}}%
\pgfusepath{clip}%
\pgfsetbuttcap%
\pgfsetroundjoin%
\definecolor{currentfill}{rgb}{0.121569,0.466667,0.705882}%
\pgfsetfillcolor{currentfill}%
\pgfsetfillopacity{0.883505}%
\pgfsetlinewidth{1.003750pt}%
\definecolor{currentstroke}{rgb}{0.121569,0.466667,0.705882}%
\pgfsetstrokecolor{currentstroke}%
\pgfsetstrokeopacity{0.883505}%
\pgfsetdash{}{0pt}%
\pgfpathmoveto{\pgfqpoint{2.509194in}{2.399841in}}%
\pgfpathcurveto{\pgfqpoint{2.517431in}{2.399841in}}{\pgfqpoint{2.525331in}{2.403113in}}{\pgfqpoint{2.531155in}{2.408937in}}%
\pgfpathcurveto{\pgfqpoint{2.536979in}{2.414761in}}{\pgfqpoint{2.540251in}{2.422661in}}{\pgfqpoint{2.540251in}{2.430897in}}%
\pgfpathcurveto{\pgfqpoint{2.540251in}{2.439134in}}{\pgfqpoint{2.536979in}{2.447034in}}{\pgfqpoint{2.531155in}{2.452858in}}%
\pgfpathcurveto{\pgfqpoint{2.525331in}{2.458682in}}{\pgfqpoint{2.517431in}{2.461954in}}{\pgfqpoint{2.509194in}{2.461954in}}%
\pgfpathcurveto{\pgfqpoint{2.500958in}{2.461954in}}{\pgfqpoint{2.493058in}{2.458682in}}{\pgfqpoint{2.487234in}{2.452858in}}%
\pgfpathcurveto{\pgfqpoint{2.481410in}{2.447034in}}{\pgfqpoint{2.478138in}{2.439134in}}{\pgfqpoint{2.478138in}{2.430897in}}%
\pgfpathcurveto{\pgfqpoint{2.478138in}{2.422661in}}{\pgfqpoint{2.481410in}{2.414761in}}{\pgfqpoint{2.487234in}{2.408937in}}%
\pgfpathcurveto{\pgfqpoint{2.493058in}{2.403113in}}{\pgfqpoint{2.500958in}{2.399841in}}{\pgfqpoint{2.509194in}{2.399841in}}%
\pgfpathclose%
\pgfusepath{stroke,fill}%
\end{pgfscope}%
\begin{pgfscope}%
\pgfpathrectangle{\pgfqpoint{0.100000in}{0.212622in}}{\pgfqpoint{3.696000in}{3.696000in}}%
\pgfusepath{clip}%
\pgfsetbuttcap%
\pgfsetroundjoin%
\definecolor{currentfill}{rgb}{0.121569,0.466667,0.705882}%
\pgfsetfillcolor{currentfill}%
\pgfsetfillopacity{0.883976}%
\pgfsetlinewidth{1.003750pt}%
\definecolor{currentstroke}{rgb}{0.121569,0.466667,0.705882}%
\pgfsetstrokecolor{currentstroke}%
\pgfsetstrokeopacity{0.883976}%
\pgfsetdash{}{0pt}%
\pgfpathmoveto{\pgfqpoint{2.448858in}{2.341733in}}%
\pgfpathcurveto{\pgfqpoint{2.457094in}{2.341733in}}{\pgfqpoint{2.464995in}{2.345006in}}{\pgfqpoint{2.470818in}{2.350829in}}%
\pgfpathcurveto{\pgfqpoint{2.476642in}{2.356653in}}{\pgfqpoint{2.479915in}{2.364553in}}{\pgfqpoint{2.479915in}{2.372790in}}%
\pgfpathcurveto{\pgfqpoint{2.479915in}{2.381026in}}{\pgfqpoint{2.476642in}{2.388926in}}{\pgfqpoint{2.470818in}{2.394750in}}%
\pgfpathcurveto{\pgfqpoint{2.464995in}{2.400574in}}{\pgfqpoint{2.457094in}{2.403846in}}{\pgfqpoint{2.448858in}{2.403846in}}%
\pgfpathcurveto{\pgfqpoint{2.440622in}{2.403846in}}{\pgfqpoint{2.432722in}{2.400574in}}{\pgfqpoint{2.426898in}{2.394750in}}%
\pgfpathcurveto{\pgfqpoint{2.421074in}{2.388926in}}{\pgfqpoint{2.417802in}{2.381026in}}{\pgfqpoint{2.417802in}{2.372790in}}%
\pgfpathcurveto{\pgfqpoint{2.417802in}{2.364553in}}{\pgfqpoint{2.421074in}{2.356653in}}{\pgfqpoint{2.426898in}{2.350829in}}%
\pgfpathcurveto{\pgfqpoint{2.432722in}{2.345006in}}{\pgfqpoint{2.440622in}{2.341733in}}{\pgfqpoint{2.448858in}{2.341733in}}%
\pgfpathclose%
\pgfusepath{stroke,fill}%
\end{pgfscope}%
\begin{pgfscope}%
\pgfpathrectangle{\pgfqpoint{0.100000in}{0.212622in}}{\pgfqpoint{3.696000in}{3.696000in}}%
\pgfusepath{clip}%
\pgfsetbuttcap%
\pgfsetroundjoin%
\definecolor{currentfill}{rgb}{0.121569,0.466667,0.705882}%
\pgfsetfillcolor{currentfill}%
\pgfsetfillopacity{0.885661}%
\pgfsetlinewidth{1.003750pt}%
\definecolor{currentstroke}{rgb}{0.121569,0.466667,0.705882}%
\pgfsetstrokecolor{currentstroke}%
\pgfsetstrokeopacity{0.885661}%
\pgfsetdash{}{0pt}%
\pgfpathmoveto{\pgfqpoint{3.074690in}{2.622089in}}%
\pgfpathcurveto{\pgfqpoint{3.082926in}{2.622089in}}{\pgfqpoint{3.090826in}{2.625361in}}{\pgfqpoint{3.096650in}{2.631185in}}%
\pgfpathcurveto{\pgfqpoint{3.102474in}{2.637009in}}{\pgfqpoint{3.105746in}{2.644909in}}{\pgfqpoint{3.105746in}{2.653145in}}%
\pgfpathcurveto{\pgfqpoint{3.105746in}{2.661381in}}{\pgfqpoint{3.102474in}{2.669281in}}{\pgfqpoint{3.096650in}{2.675105in}}%
\pgfpathcurveto{\pgfqpoint{3.090826in}{2.680929in}}{\pgfqpoint{3.082926in}{2.684202in}}{\pgfqpoint{3.074690in}{2.684202in}}%
\pgfpathcurveto{\pgfqpoint{3.066453in}{2.684202in}}{\pgfqpoint{3.058553in}{2.680929in}}{\pgfqpoint{3.052729in}{2.675105in}}%
\pgfpathcurveto{\pgfqpoint{3.046906in}{2.669281in}}{\pgfqpoint{3.043633in}{2.661381in}}{\pgfqpoint{3.043633in}{2.653145in}}%
\pgfpathcurveto{\pgfqpoint{3.043633in}{2.644909in}}{\pgfqpoint{3.046906in}{2.637009in}}{\pgfqpoint{3.052729in}{2.631185in}}%
\pgfpathcurveto{\pgfqpoint{3.058553in}{2.625361in}}{\pgfqpoint{3.066453in}{2.622089in}}{\pgfqpoint{3.074690in}{2.622089in}}%
\pgfpathclose%
\pgfusepath{stroke,fill}%
\end{pgfscope}%
\begin{pgfscope}%
\pgfpathrectangle{\pgfqpoint{0.100000in}{0.212622in}}{\pgfqpoint{3.696000in}{3.696000in}}%
\pgfusepath{clip}%
\pgfsetbuttcap%
\pgfsetroundjoin%
\definecolor{currentfill}{rgb}{0.121569,0.466667,0.705882}%
\pgfsetfillcolor{currentfill}%
\pgfsetfillopacity{0.886138}%
\pgfsetlinewidth{1.003750pt}%
\definecolor{currentstroke}{rgb}{0.121569,0.466667,0.705882}%
\pgfsetstrokecolor{currentstroke}%
\pgfsetstrokeopacity{0.886138}%
\pgfsetdash{}{0pt}%
\pgfpathmoveto{\pgfqpoint{3.074196in}{2.621425in}}%
\pgfpathcurveto{\pgfqpoint{3.082432in}{2.621425in}}{\pgfqpoint{3.090332in}{2.624697in}}{\pgfqpoint{3.096156in}{2.630521in}}%
\pgfpathcurveto{\pgfqpoint{3.101980in}{2.636345in}}{\pgfqpoint{3.105252in}{2.644245in}}{\pgfqpoint{3.105252in}{2.652481in}}%
\pgfpathcurveto{\pgfqpoint{3.105252in}{2.660717in}}{\pgfqpoint{3.101980in}{2.668617in}}{\pgfqpoint{3.096156in}{2.674441in}}%
\pgfpathcurveto{\pgfqpoint{3.090332in}{2.680265in}}{\pgfqpoint{3.082432in}{2.683537in}}{\pgfqpoint{3.074196in}{2.683537in}}%
\pgfpathcurveto{\pgfqpoint{3.065959in}{2.683537in}}{\pgfqpoint{3.058059in}{2.680265in}}{\pgfqpoint{3.052235in}{2.674441in}}%
\pgfpathcurveto{\pgfqpoint{3.046412in}{2.668617in}}{\pgfqpoint{3.043139in}{2.660717in}}{\pgfqpoint{3.043139in}{2.652481in}}%
\pgfpathcurveto{\pgfqpoint{3.043139in}{2.644245in}}{\pgfqpoint{3.046412in}{2.636345in}}{\pgfqpoint{3.052235in}{2.630521in}}%
\pgfpathcurveto{\pgfqpoint{3.058059in}{2.624697in}}{\pgfqpoint{3.065959in}{2.621425in}}{\pgfqpoint{3.074196in}{2.621425in}}%
\pgfpathclose%
\pgfusepath{stroke,fill}%
\end{pgfscope}%
\begin{pgfscope}%
\pgfpathrectangle{\pgfqpoint{0.100000in}{0.212622in}}{\pgfqpoint{3.696000in}{3.696000in}}%
\pgfusepath{clip}%
\pgfsetbuttcap%
\pgfsetroundjoin%
\definecolor{currentfill}{rgb}{0.121569,0.466667,0.705882}%
\pgfsetfillcolor{currentfill}%
\pgfsetfillopacity{0.886840}%
\pgfsetlinewidth{1.003750pt}%
\definecolor{currentstroke}{rgb}{0.121569,0.466667,0.705882}%
\pgfsetstrokecolor{currentstroke}%
\pgfsetstrokeopacity{0.886840}%
\pgfsetdash{}{0pt}%
\pgfpathmoveto{\pgfqpoint{1.269897in}{1.816441in}}%
\pgfpathcurveto{\pgfqpoint{1.278134in}{1.816441in}}{\pgfqpoint{1.286034in}{1.819713in}}{\pgfqpoint{1.291858in}{1.825537in}}%
\pgfpathcurveto{\pgfqpoint{1.297682in}{1.831361in}}{\pgfqpoint{1.300954in}{1.839261in}}{\pgfqpoint{1.300954in}{1.847497in}}%
\pgfpathcurveto{\pgfqpoint{1.300954in}{1.855734in}}{\pgfqpoint{1.297682in}{1.863634in}}{\pgfqpoint{1.291858in}{1.869458in}}%
\pgfpathcurveto{\pgfqpoint{1.286034in}{1.875282in}}{\pgfqpoint{1.278134in}{1.878554in}}{\pgfqpoint{1.269897in}{1.878554in}}%
\pgfpathcurveto{\pgfqpoint{1.261661in}{1.878554in}}{\pgfqpoint{1.253761in}{1.875282in}}{\pgfqpoint{1.247937in}{1.869458in}}%
\pgfpathcurveto{\pgfqpoint{1.242113in}{1.863634in}}{\pgfqpoint{1.238841in}{1.855734in}}{\pgfqpoint{1.238841in}{1.847497in}}%
\pgfpathcurveto{\pgfqpoint{1.238841in}{1.839261in}}{\pgfqpoint{1.242113in}{1.831361in}}{\pgfqpoint{1.247937in}{1.825537in}}%
\pgfpathcurveto{\pgfqpoint{1.253761in}{1.819713in}}{\pgfqpoint{1.261661in}{1.816441in}}{\pgfqpoint{1.269897in}{1.816441in}}%
\pgfpathclose%
\pgfusepath{stroke,fill}%
\end{pgfscope}%
\begin{pgfscope}%
\pgfpathrectangle{\pgfqpoint{0.100000in}{0.212622in}}{\pgfqpoint{3.696000in}{3.696000in}}%
\pgfusepath{clip}%
\pgfsetbuttcap%
\pgfsetroundjoin%
\definecolor{currentfill}{rgb}{0.121569,0.466667,0.705882}%
\pgfsetfillcolor{currentfill}%
\pgfsetfillopacity{0.887586}%
\pgfsetlinewidth{1.003750pt}%
\definecolor{currentstroke}{rgb}{0.121569,0.466667,0.705882}%
\pgfsetstrokecolor{currentstroke}%
\pgfsetstrokeopacity{0.887586}%
\pgfsetdash{}{0pt}%
\pgfpathmoveto{\pgfqpoint{1.260441in}{1.810394in}}%
\pgfpathcurveto{\pgfqpoint{1.268677in}{1.810394in}}{\pgfqpoint{1.276577in}{1.813666in}}{\pgfqpoint{1.282401in}{1.819490in}}%
\pgfpathcurveto{\pgfqpoint{1.288225in}{1.825314in}}{\pgfqpoint{1.291498in}{1.833214in}}{\pgfqpoint{1.291498in}{1.841450in}}%
\pgfpathcurveto{\pgfqpoint{1.291498in}{1.849687in}}{\pgfqpoint{1.288225in}{1.857587in}}{\pgfqpoint{1.282401in}{1.863411in}}%
\pgfpathcurveto{\pgfqpoint{1.276577in}{1.869234in}}{\pgfqpoint{1.268677in}{1.872507in}}{\pgfqpoint{1.260441in}{1.872507in}}%
\pgfpathcurveto{\pgfqpoint{1.252205in}{1.872507in}}{\pgfqpoint{1.244305in}{1.869234in}}{\pgfqpoint{1.238481in}{1.863411in}}%
\pgfpathcurveto{\pgfqpoint{1.232657in}{1.857587in}}{\pgfqpoint{1.229385in}{1.849687in}}{\pgfqpoint{1.229385in}{1.841450in}}%
\pgfpathcurveto{\pgfqpoint{1.229385in}{1.833214in}}{\pgfqpoint{1.232657in}{1.825314in}}{\pgfqpoint{1.238481in}{1.819490in}}%
\pgfpathcurveto{\pgfqpoint{1.244305in}{1.813666in}}{\pgfqpoint{1.252205in}{1.810394in}}{\pgfqpoint{1.260441in}{1.810394in}}%
\pgfpathclose%
\pgfusepath{stroke,fill}%
\end{pgfscope}%
\begin{pgfscope}%
\pgfpathrectangle{\pgfqpoint{0.100000in}{0.212622in}}{\pgfqpoint{3.696000in}{3.696000in}}%
\pgfusepath{clip}%
\pgfsetbuttcap%
\pgfsetroundjoin%
\definecolor{currentfill}{rgb}{0.121569,0.466667,0.705882}%
\pgfsetfillcolor{currentfill}%
\pgfsetfillopacity{0.888095}%
\pgfsetlinewidth{1.003750pt}%
\definecolor{currentstroke}{rgb}{0.121569,0.466667,0.705882}%
\pgfsetstrokecolor{currentstroke}%
\pgfsetstrokeopacity{0.888095}%
\pgfsetdash{}{0pt}%
\pgfpathmoveto{\pgfqpoint{2.368343in}{2.319283in}}%
\pgfpathcurveto{\pgfqpoint{2.376579in}{2.319283in}}{\pgfqpoint{2.384479in}{2.322555in}}{\pgfqpoint{2.390303in}{2.328379in}}%
\pgfpathcurveto{\pgfqpoint{2.396127in}{2.334203in}}{\pgfqpoint{2.399399in}{2.342103in}}{\pgfqpoint{2.399399in}{2.350339in}}%
\pgfpathcurveto{\pgfqpoint{2.399399in}{2.358575in}}{\pgfqpoint{2.396127in}{2.366475in}}{\pgfqpoint{2.390303in}{2.372299in}}%
\pgfpathcurveto{\pgfqpoint{2.384479in}{2.378123in}}{\pgfqpoint{2.376579in}{2.381396in}}{\pgfqpoint{2.368343in}{2.381396in}}%
\pgfpathcurveto{\pgfqpoint{2.360106in}{2.381396in}}{\pgfqpoint{2.352206in}{2.378123in}}{\pgfqpoint{2.346382in}{2.372299in}}%
\pgfpathcurveto{\pgfqpoint{2.340559in}{2.366475in}}{\pgfqpoint{2.337286in}{2.358575in}}{\pgfqpoint{2.337286in}{2.350339in}}%
\pgfpathcurveto{\pgfqpoint{2.337286in}{2.342103in}}{\pgfqpoint{2.340559in}{2.334203in}}{\pgfqpoint{2.346382in}{2.328379in}}%
\pgfpathcurveto{\pgfqpoint{2.352206in}{2.322555in}}{\pgfqpoint{2.360106in}{2.319283in}}{\pgfqpoint{2.368343in}{2.319283in}}%
\pgfpathclose%
\pgfusepath{stroke,fill}%
\end{pgfscope}%
\begin{pgfscope}%
\pgfpathrectangle{\pgfqpoint{0.100000in}{0.212622in}}{\pgfqpoint{3.696000in}{3.696000in}}%
\pgfusepath{clip}%
\pgfsetbuttcap%
\pgfsetroundjoin%
\definecolor{currentfill}{rgb}{0.121569,0.466667,0.705882}%
\pgfsetfillcolor{currentfill}%
\pgfsetfillopacity{0.888142}%
\pgfsetlinewidth{1.003750pt}%
\definecolor{currentstroke}{rgb}{0.121569,0.466667,0.705882}%
\pgfsetstrokecolor{currentstroke}%
\pgfsetstrokeopacity{0.888142}%
\pgfsetdash{}{0pt}%
\pgfpathmoveto{\pgfqpoint{2.365453in}{2.325336in}}%
\pgfpathcurveto{\pgfqpoint{2.373689in}{2.325336in}}{\pgfqpoint{2.381589in}{2.328609in}}{\pgfqpoint{2.387413in}{2.334433in}}%
\pgfpathcurveto{\pgfqpoint{2.393237in}{2.340257in}}{\pgfqpoint{2.396509in}{2.348157in}}{\pgfqpoint{2.396509in}{2.356393in}}%
\pgfpathcurveto{\pgfqpoint{2.396509in}{2.364629in}}{\pgfqpoint{2.393237in}{2.372529in}}{\pgfqpoint{2.387413in}{2.378353in}}%
\pgfpathcurveto{\pgfqpoint{2.381589in}{2.384177in}}{\pgfqpoint{2.373689in}{2.387449in}}{\pgfqpoint{2.365453in}{2.387449in}}%
\pgfpathcurveto{\pgfqpoint{2.357217in}{2.387449in}}{\pgfqpoint{2.349316in}{2.384177in}}{\pgfqpoint{2.343493in}{2.378353in}}%
\pgfpathcurveto{\pgfqpoint{2.337669in}{2.372529in}}{\pgfqpoint{2.334396in}{2.364629in}}{\pgfqpoint{2.334396in}{2.356393in}}%
\pgfpathcurveto{\pgfqpoint{2.334396in}{2.348157in}}{\pgfqpoint{2.337669in}{2.340257in}}{\pgfqpoint{2.343493in}{2.334433in}}%
\pgfpathcurveto{\pgfqpoint{2.349316in}{2.328609in}}{\pgfqpoint{2.357217in}{2.325336in}}{\pgfqpoint{2.365453in}{2.325336in}}%
\pgfpathclose%
\pgfusepath{stroke,fill}%
\end{pgfscope}%
\begin{pgfscope}%
\pgfpathrectangle{\pgfqpoint{0.100000in}{0.212622in}}{\pgfqpoint{3.696000in}{3.696000in}}%
\pgfusepath{clip}%
\pgfsetbuttcap%
\pgfsetroundjoin%
\definecolor{currentfill}{rgb}{0.121569,0.466667,0.705882}%
\pgfsetfillcolor{currentfill}%
\pgfsetfillopacity{0.888239}%
\pgfsetlinewidth{1.003750pt}%
\definecolor{currentstroke}{rgb}{0.121569,0.466667,0.705882}%
\pgfsetstrokecolor{currentstroke}%
\pgfsetstrokeopacity{0.888239}%
\pgfsetdash{}{0pt}%
\pgfpathmoveto{\pgfqpoint{3.071960in}{2.618842in}}%
\pgfpathcurveto{\pgfqpoint{3.080196in}{2.618842in}}{\pgfqpoint{3.088096in}{2.622114in}}{\pgfqpoint{3.093920in}{2.627938in}}%
\pgfpathcurveto{\pgfqpoint{3.099744in}{2.633762in}}{\pgfqpoint{3.103017in}{2.641662in}}{\pgfqpoint{3.103017in}{2.649898in}}%
\pgfpathcurveto{\pgfqpoint{3.103017in}{2.658135in}}{\pgfqpoint{3.099744in}{2.666035in}}{\pgfqpoint{3.093920in}{2.671859in}}%
\pgfpathcurveto{\pgfqpoint{3.088096in}{2.677683in}}{\pgfqpoint{3.080196in}{2.680955in}}{\pgfqpoint{3.071960in}{2.680955in}}%
\pgfpathcurveto{\pgfqpoint{3.063724in}{2.680955in}}{\pgfqpoint{3.055824in}{2.677683in}}{\pgfqpoint{3.050000in}{2.671859in}}%
\pgfpathcurveto{\pgfqpoint{3.044176in}{2.666035in}}{\pgfqpoint{3.040904in}{2.658135in}}{\pgfqpoint{3.040904in}{2.649898in}}%
\pgfpathcurveto{\pgfqpoint{3.040904in}{2.641662in}}{\pgfqpoint{3.044176in}{2.633762in}}{\pgfqpoint{3.050000in}{2.627938in}}%
\pgfpathcurveto{\pgfqpoint{3.055824in}{2.622114in}}{\pgfqpoint{3.063724in}{2.618842in}}{\pgfqpoint{3.071960in}{2.618842in}}%
\pgfpathclose%
\pgfusepath{stroke,fill}%
\end{pgfscope}%
\begin{pgfscope}%
\pgfpathrectangle{\pgfqpoint{0.100000in}{0.212622in}}{\pgfqpoint{3.696000in}{3.696000in}}%
\pgfusepath{clip}%
\pgfsetbuttcap%
\pgfsetroundjoin%
\definecolor{currentfill}{rgb}{0.121569,0.466667,0.705882}%
\pgfsetfillcolor{currentfill}%
\pgfsetfillopacity{0.888762}%
\pgfsetlinewidth{1.003750pt}%
\definecolor{currentstroke}{rgb}{0.121569,0.466667,0.705882}%
\pgfsetstrokecolor{currentstroke}%
\pgfsetstrokeopacity{0.888762}%
\pgfsetdash{}{0pt}%
\pgfpathmoveto{\pgfqpoint{1.259707in}{1.807797in}}%
\pgfpathcurveto{\pgfqpoint{1.267943in}{1.807797in}}{\pgfqpoint{1.275843in}{1.811070in}}{\pgfqpoint{1.281667in}{1.816894in}}%
\pgfpathcurveto{\pgfqpoint{1.287491in}{1.822718in}}{\pgfqpoint{1.290763in}{1.830618in}}{\pgfqpoint{1.290763in}{1.838854in}}%
\pgfpathcurveto{\pgfqpoint{1.290763in}{1.847090in}}{\pgfqpoint{1.287491in}{1.854990in}}{\pgfqpoint{1.281667in}{1.860814in}}%
\pgfpathcurveto{\pgfqpoint{1.275843in}{1.866638in}}{\pgfqpoint{1.267943in}{1.869910in}}{\pgfqpoint{1.259707in}{1.869910in}}%
\pgfpathcurveto{\pgfqpoint{1.251471in}{1.869910in}}{\pgfqpoint{1.243571in}{1.866638in}}{\pgfqpoint{1.237747in}{1.860814in}}%
\pgfpathcurveto{\pgfqpoint{1.231923in}{1.854990in}}{\pgfqpoint{1.228650in}{1.847090in}}{\pgfqpoint{1.228650in}{1.838854in}}%
\pgfpathcurveto{\pgfqpoint{1.228650in}{1.830618in}}{\pgfqpoint{1.231923in}{1.822718in}}{\pgfqpoint{1.237747in}{1.816894in}}%
\pgfpathcurveto{\pgfqpoint{1.243571in}{1.811070in}}{\pgfqpoint{1.251471in}{1.807797in}}{\pgfqpoint{1.259707in}{1.807797in}}%
\pgfpathclose%
\pgfusepath{stroke,fill}%
\end{pgfscope}%
\begin{pgfscope}%
\pgfpathrectangle{\pgfqpoint{0.100000in}{0.212622in}}{\pgfqpoint{3.696000in}{3.696000in}}%
\pgfusepath{clip}%
\pgfsetbuttcap%
\pgfsetroundjoin%
\definecolor{currentfill}{rgb}{0.121569,0.466667,0.705882}%
\pgfsetfillcolor{currentfill}%
\pgfsetfillopacity{0.888997}%
\pgfsetlinewidth{1.003750pt}%
\definecolor{currentstroke}{rgb}{0.121569,0.466667,0.705882}%
\pgfsetstrokecolor{currentstroke}%
\pgfsetstrokeopacity{0.888997}%
\pgfsetdash{}{0pt}%
\pgfpathmoveto{\pgfqpoint{1.250700in}{1.798628in}}%
\pgfpathcurveto{\pgfqpoint{1.258936in}{1.798628in}}{\pgfqpoint{1.266836in}{1.801901in}}{\pgfqpoint{1.272660in}{1.807725in}}%
\pgfpathcurveto{\pgfqpoint{1.278484in}{1.813548in}}{\pgfqpoint{1.281757in}{1.821449in}}{\pgfqpoint{1.281757in}{1.829685in}}%
\pgfpathcurveto{\pgfqpoint{1.281757in}{1.837921in}}{\pgfqpoint{1.278484in}{1.845821in}}{\pgfqpoint{1.272660in}{1.851645in}}%
\pgfpathcurveto{\pgfqpoint{1.266836in}{1.857469in}}{\pgfqpoint{1.258936in}{1.860741in}}{\pgfqpoint{1.250700in}{1.860741in}}%
\pgfpathcurveto{\pgfqpoint{1.242464in}{1.860741in}}{\pgfqpoint{1.234564in}{1.857469in}}{\pgfqpoint{1.228740in}{1.851645in}}%
\pgfpathcurveto{\pgfqpoint{1.222916in}{1.845821in}}{\pgfqpoint{1.219644in}{1.837921in}}{\pgfqpoint{1.219644in}{1.829685in}}%
\pgfpathcurveto{\pgfqpoint{1.219644in}{1.821449in}}{\pgfqpoint{1.222916in}{1.813548in}}{\pgfqpoint{1.228740in}{1.807725in}}%
\pgfpathcurveto{\pgfqpoint{1.234564in}{1.801901in}}{\pgfqpoint{1.242464in}{1.798628in}}{\pgfqpoint{1.250700in}{1.798628in}}%
\pgfpathclose%
\pgfusepath{stroke,fill}%
\end{pgfscope}%
\begin{pgfscope}%
\pgfpathrectangle{\pgfqpoint{0.100000in}{0.212622in}}{\pgfqpoint{3.696000in}{3.696000in}}%
\pgfusepath{clip}%
\pgfsetbuttcap%
\pgfsetroundjoin%
\definecolor{currentfill}{rgb}{0.121569,0.466667,0.705882}%
\pgfsetfillcolor{currentfill}%
\pgfsetfillopacity{0.889351}%
\pgfsetlinewidth{1.003750pt}%
\definecolor{currentstroke}{rgb}{0.121569,0.466667,0.705882}%
\pgfsetstrokecolor{currentstroke}%
\pgfsetstrokeopacity{0.889351}%
\pgfsetdash{}{0pt}%
\pgfpathmoveto{\pgfqpoint{1.289313in}{1.817613in}}%
\pgfpathcurveto{\pgfqpoint{1.297549in}{1.817613in}}{\pgfqpoint{1.305449in}{1.820885in}}{\pgfqpoint{1.311273in}{1.826709in}}%
\pgfpathcurveto{\pgfqpoint{1.317097in}{1.832533in}}{\pgfqpoint{1.320369in}{1.840433in}}{\pgfqpoint{1.320369in}{1.848669in}}%
\pgfpathcurveto{\pgfqpoint{1.320369in}{1.856905in}}{\pgfqpoint{1.317097in}{1.864805in}}{\pgfqpoint{1.311273in}{1.870629in}}%
\pgfpathcurveto{\pgfqpoint{1.305449in}{1.876453in}}{\pgfqpoint{1.297549in}{1.879726in}}{\pgfqpoint{1.289313in}{1.879726in}}%
\pgfpathcurveto{\pgfqpoint{1.281076in}{1.879726in}}{\pgfqpoint{1.273176in}{1.876453in}}{\pgfqpoint{1.267352in}{1.870629in}}%
\pgfpathcurveto{\pgfqpoint{1.261528in}{1.864805in}}{\pgfqpoint{1.258256in}{1.856905in}}{\pgfqpoint{1.258256in}{1.848669in}}%
\pgfpathcurveto{\pgfqpoint{1.258256in}{1.840433in}}{\pgfqpoint{1.261528in}{1.832533in}}{\pgfqpoint{1.267352in}{1.826709in}}%
\pgfpathcurveto{\pgfqpoint{1.273176in}{1.820885in}}{\pgfqpoint{1.281076in}{1.817613in}}{\pgfqpoint{1.289313in}{1.817613in}}%
\pgfpathclose%
\pgfusepath{stroke,fill}%
\end{pgfscope}%
\begin{pgfscope}%
\pgfpathrectangle{\pgfqpoint{0.100000in}{0.212622in}}{\pgfqpoint{3.696000in}{3.696000in}}%
\pgfusepath{clip}%
\pgfsetbuttcap%
\pgfsetroundjoin%
\definecolor{currentfill}{rgb}{0.121569,0.466667,0.705882}%
\pgfsetfillcolor{currentfill}%
\pgfsetfillopacity{0.889591}%
\pgfsetlinewidth{1.003750pt}%
\definecolor{currentstroke}{rgb}{0.121569,0.466667,0.705882}%
\pgfsetstrokecolor{currentstroke}%
\pgfsetstrokeopacity{0.889591}%
\pgfsetdash{}{0pt}%
\pgfpathmoveto{\pgfqpoint{1.801509in}{2.116109in}}%
\pgfpathcurveto{\pgfqpoint{1.809745in}{2.116109in}}{\pgfqpoint{1.817645in}{2.119381in}}{\pgfqpoint{1.823469in}{2.125205in}}%
\pgfpathcurveto{\pgfqpoint{1.829293in}{2.131029in}}{\pgfqpoint{1.832565in}{2.138929in}}{\pgfqpoint{1.832565in}{2.147166in}}%
\pgfpathcurveto{\pgfqpoint{1.832565in}{2.155402in}}{\pgfqpoint{1.829293in}{2.163302in}}{\pgfqpoint{1.823469in}{2.169126in}}%
\pgfpathcurveto{\pgfqpoint{1.817645in}{2.174950in}}{\pgfqpoint{1.809745in}{2.178222in}}{\pgfqpoint{1.801509in}{2.178222in}}%
\pgfpathcurveto{\pgfqpoint{1.793273in}{2.178222in}}{\pgfqpoint{1.785372in}{2.174950in}}{\pgfqpoint{1.779549in}{2.169126in}}%
\pgfpathcurveto{\pgfqpoint{1.773725in}{2.163302in}}{\pgfqpoint{1.770452in}{2.155402in}}{\pgfqpoint{1.770452in}{2.147166in}}%
\pgfpathcurveto{\pgfqpoint{1.770452in}{2.138929in}}{\pgfqpoint{1.773725in}{2.131029in}}{\pgfqpoint{1.779549in}{2.125205in}}%
\pgfpathcurveto{\pgfqpoint{1.785372in}{2.119381in}}{\pgfqpoint{1.793273in}{2.116109in}}{\pgfqpoint{1.801509in}{2.116109in}}%
\pgfpathclose%
\pgfusepath{stroke,fill}%
\end{pgfscope}%
\begin{pgfscope}%
\pgfpathrectangle{\pgfqpoint{0.100000in}{0.212622in}}{\pgfqpoint{3.696000in}{3.696000in}}%
\pgfusepath{clip}%
\pgfsetbuttcap%
\pgfsetroundjoin%
\definecolor{currentfill}{rgb}{0.121569,0.466667,0.705882}%
\pgfsetfillcolor{currentfill}%
\pgfsetfillopacity{0.890392}%
\pgfsetlinewidth{1.003750pt}%
\definecolor{currentstroke}{rgb}{0.121569,0.466667,0.705882}%
\pgfsetstrokecolor{currentstroke}%
\pgfsetstrokeopacity{0.890392}%
\pgfsetdash{}{0pt}%
\pgfpathmoveto{\pgfqpoint{2.352857in}{2.312825in}}%
\pgfpathcurveto{\pgfqpoint{2.361093in}{2.312825in}}{\pgfqpoint{2.368993in}{2.316097in}}{\pgfqpoint{2.374817in}{2.321921in}}%
\pgfpathcurveto{\pgfqpoint{2.380641in}{2.327745in}}{\pgfqpoint{2.383913in}{2.335645in}}{\pgfqpoint{2.383913in}{2.343881in}}%
\pgfpathcurveto{\pgfqpoint{2.383913in}{2.352118in}}{\pgfqpoint{2.380641in}{2.360018in}}{\pgfqpoint{2.374817in}{2.365842in}}%
\pgfpathcurveto{\pgfqpoint{2.368993in}{2.371665in}}{\pgfqpoint{2.361093in}{2.374938in}}{\pgfqpoint{2.352857in}{2.374938in}}%
\pgfpathcurveto{\pgfqpoint{2.344621in}{2.374938in}}{\pgfqpoint{2.336721in}{2.371665in}}{\pgfqpoint{2.330897in}{2.365842in}}%
\pgfpathcurveto{\pgfqpoint{2.325073in}{2.360018in}}{\pgfqpoint{2.321800in}{2.352118in}}{\pgfqpoint{2.321800in}{2.343881in}}%
\pgfpathcurveto{\pgfqpoint{2.321800in}{2.335645in}}{\pgfqpoint{2.325073in}{2.327745in}}{\pgfqpoint{2.330897in}{2.321921in}}%
\pgfpathcurveto{\pgfqpoint{2.336721in}{2.316097in}}{\pgfqpoint{2.344621in}{2.312825in}}{\pgfqpoint{2.352857in}{2.312825in}}%
\pgfpathclose%
\pgfusepath{stroke,fill}%
\end{pgfscope}%
\begin{pgfscope}%
\pgfpathrectangle{\pgfqpoint{0.100000in}{0.212622in}}{\pgfqpoint{3.696000in}{3.696000in}}%
\pgfusepath{clip}%
\pgfsetbuttcap%
\pgfsetroundjoin%
\definecolor{currentfill}{rgb}{0.121569,0.466667,0.705882}%
\pgfsetfillcolor{currentfill}%
\pgfsetfillopacity{0.890780}%
\pgfsetlinewidth{1.003750pt}%
\definecolor{currentstroke}{rgb}{0.121569,0.466667,0.705882}%
\pgfsetstrokecolor{currentstroke}%
\pgfsetstrokeopacity{0.890780}%
\pgfsetdash{}{0pt}%
\pgfpathmoveto{\pgfqpoint{3.071435in}{2.618449in}}%
\pgfpathcurveto{\pgfqpoint{3.079672in}{2.618449in}}{\pgfqpoint{3.087572in}{2.621721in}}{\pgfqpoint{3.093396in}{2.627545in}}%
\pgfpathcurveto{\pgfqpoint{3.099220in}{2.633369in}}{\pgfqpoint{3.102492in}{2.641269in}}{\pgfqpoint{3.102492in}{2.649505in}}%
\pgfpathcurveto{\pgfqpoint{3.102492in}{2.657742in}}{\pgfqpoint{3.099220in}{2.665642in}}{\pgfqpoint{3.093396in}{2.671466in}}%
\pgfpathcurveto{\pgfqpoint{3.087572in}{2.677290in}}{\pgfqpoint{3.079672in}{2.680562in}}{\pgfqpoint{3.071435in}{2.680562in}}%
\pgfpathcurveto{\pgfqpoint{3.063199in}{2.680562in}}{\pgfqpoint{3.055299in}{2.677290in}}{\pgfqpoint{3.049475in}{2.671466in}}%
\pgfpathcurveto{\pgfqpoint{3.043651in}{2.665642in}}{\pgfqpoint{3.040379in}{2.657742in}}{\pgfqpoint{3.040379in}{2.649505in}}%
\pgfpathcurveto{\pgfqpoint{3.040379in}{2.641269in}}{\pgfqpoint{3.043651in}{2.633369in}}{\pgfqpoint{3.049475in}{2.627545in}}%
\pgfpathcurveto{\pgfqpoint{3.055299in}{2.621721in}}{\pgfqpoint{3.063199in}{2.618449in}}{\pgfqpoint{3.071435in}{2.618449in}}%
\pgfpathclose%
\pgfusepath{stroke,fill}%
\end{pgfscope}%
\begin{pgfscope}%
\pgfpathrectangle{\pgfqpoint{0.100000in}{0.212622in}}{\pgfqpoint{3.696000in}{3.696000in}}%
\pgfusepath{clip}%
\pgfsetbuttcap%
\pgfsetroundjoin%
\definecolor{currentfill}{rgb}{0.121569,0.466667,0.705882}%
\pgfsetfillcolor{currentfill}%
\pgfsetfillopacity{0.890972}%
\pgfsetlinewidth{1.003750pt}%
\definecolor{currentstroke}{rgb}{0.121569,0.466667,0.705882}%
\pgfsetstrokecolor{currentstroke}%
\pgfsetstrokeopacity{0.890972}%
\pgfsetdash{}{0pt}%
\pgfpathmoveto{\pgfqpoint{2.700416in}{2.496158in}}%
\pgfpathcurveto{\pgfqpoint{2.708652in}{2.496158in}}{\pgfqpoint{2.716552in}{2.499431in}}{\pgfqpoint{2.722376in}{2.505254in}}%
\pgfpathcurveto{\pgfqpoint{2.728200in}{2.511078in}}{\pgfqpoint{2.731472in}{2.518978in}}{\pgfqpoint{2.731472in}{2.527215in}}%
\pgfpathcurveto{\pgfqpoint{2.731472in}{2.535451in}}{\pgfqpoint{2.728200in}{2.543351in}}{\pgfqpoint{2.722376in}{2.549175in}}%
\pgfpathcurveto{\pgfqpoint{2.716552in}{2.554999in}}{\pgfqpoint{2.708652in}{2.558271in}}{\pgfqpoint{2.700416in}{2.558271in}}%
\pgfpathcurveto{\pgfqpoint{2.692179in}{2.558271in}}{\pgfqpoint{2.684279in}{2.554999in}}{\pgfqpoint{2.678455in}{2.549175in}}%
\pgfpathcurveto{\pgfqpoint{2.672631in}{2.543351in}}{\pgfqpoint{2.669359in}{2.535451in}}{\pgfqpoint{2.669359in}{2.527215in}}%
\pgfpathcurveto{\pgfqpoint{2.669359in}{2.518978in}}{\pgfqpoint{2.672631in}{2.511078in}}{\pgfqpoint{2.678455in}{2.505254in}}%
\pgfpathcurveto{\pgfqpoint{2.684279in}{2.499431in}}{\pgfqpoint{2.692179in}{2.496158in}}{\pgfqpoint{2.700416in}{2.496158in}}%
\pgfpathclose%
\pgfusepath{stroke,fill}%
\end{pgfscope}%
\begin{pgfscope}%
\pgfpathrectangle{\pgfqpoint{0.100000in}{0.212622in}}{\pgfqpoint{3.696000in}{3.696000in}}%
\pgfusepath{clip}%
\pgfsetbuttcap%
\pgfsetroundjoin%
\definecolor{currentfill}{rgb}{0.121569,0.466667,0.705882}%
\pgfsetfillcolor{currentfill}%
\pgfsetfillopacity{0.891044}%
\pgfsetlinewidth{1.003750pt}%
\definecolor{currentstroke}{rgb}{0.121569,0.466667,0.705882}%
\pgfsetstrokecolor{currentstroke}%
\pgfsetstrokeopacity{0.891044}%
\pgfsetdash{}{0pt}%
\pgfpathmoveto{\pgfqpoint{1.802295in}{2.115894in}}%
\pgfpathcurveto{\pgfqpoint{1.810531in}{2.115894in}}{\pgfqpoint{1.818431in}{2.119166in}}{\pgfqpoint{1.824255in}{2.124990in}}%
\pgfpathcurveto{\pgfqpoint{1.830079in}{2.130814in}}{\pgfqpoint{1.833352in}{2.138714in}}{\pgfqpoint{1.833352in}{2.146950in}}%
\pgfpathcurveto{\pgfqpoint{1.833352in}{2.155187in}}{\pgfqpoint{1.830079in}{2.163087in}}{\pgfqpoint{1.824255in}{2.168911in}}%
\pgfpathcurveto{\pgfqpoint{1.818431in}{2.174735in}}{\pgfqpoint{1.810531in}{2.178007in}}{\pgfqpoint{1.802295in}{2.178007in}}%
\pgfpathcurveto{\pgfqpoint{1.794059in}{2.178007in}}{\pgfqpoint{1.786159in}{2.174735in}}{\pgfqpoint{1.780335in}{2.168911in}}%
\pgfpathcurveto{\pgfqpoint{1.774511in}{2.163087in}}{\pgfqpoint{1.771239in}{2.155187in}}{\pgfqpoint{1.771239in}{2.146950in}}%
\pgfpathcurveto{\pgfqpoint{1.771239in}{2.138714in}}{\pgfqpoint{1.774511in}{2.130814in}}{\pgfqpoint{1.780335in}{2.124990in}}%
\pgfpathcurveto{\pgfqpoint{1.786159in}{2.119166in}}{\pgfqpoint{1.794059in}{2.115894in}}{\pgfqpoint{1.802295in}{2.115894in}}%
\pgfpathclose%
\pgfusepath{stroke,fill}%
\end{pgfscope}%
\begin{pgfscope}%
\pgfpathrectangle{\pgfqpoint{0.100000in}{0.212622in}}{\pgfqpoint{3.696000in}{3.696000in}}%
\pgfusepath{clip}%
\pgfsetbuttcap%
\pgfsetroundjoin%
\definecolor{currentfill}{rgb}{0.121569,0.466667,0.705882}%
\pgfsetfillcolor{currentfill}%
\pgfsetfillopacity{0.891133}%
\pgfsetlinewidth{1.003750pt}%
\definecolor{currentstroke}{rgb}{0.121569,0.466667,0.705882}%
\pgfsetstrokecolor{currentstroke}%
\pgfsetstrokeopacity{0.891133}%
\pgfsetdash{}{0pt}%
\pgfpathmoveto{\pgfqpoint{1.268634in}{1.815539in}}%
\pgfpathcurveto{\pgfqpoint{1.276870in}{1.815539in}}{\pgfqpoint{1.284770in}{1.818811in}}{\pgfqpoint{1.290594in}{1.824635in}}%
\pgfpathcurveto{\pgfqpoint{1.296418in}{1.830459in}}{\pgfqpoint{1.299691in}{1.838359in}}{\pgfqpoint{1.299691in}{1.846595in}}%
\pgfpathcurveto{\pgfqpoint{1.299691in}{1.854832in}}{\pgfqpoint{1.296418in}{1.862732in}}{\pgfqpoint{1.290594in}{1.868556in}}%
\pgfpathcurveto{\pgfqpoint{1.284770in}{1.874379in}}{\pgfqpoint{1.276870in}{1.877652in}}{\pgfqpoint{1.268634in}{1.877652in}}%
\pgfpathcurveto{\pgfqpoint{1.260398in}{1.877652in}}{\pgfqpoint{1.252498in}{1.874379in}}{\pgfqpoint{1.246674in}{1.868556in}}%
\pgfpathcurveto{\pgfqpoint{1.240850in}{1.862732in}}{\pgfqpoint{1.237578in}{1.854832in}}{\pgfqpoint{1.237578in}{1.846595in}}%
\pgfpathcurveto{\pgfqpoint{1.237578in}{1.838359in}}{\pgfqpoint{1.240850in}{1.830459in}}{\pgfqpoint{1.246674in}{1.824635in}}%
\pgfpathcurveto{\pgfqpoint{1.252498in}{1.818811in}}{\pgfqpoint{1.260398in}{1.815539in}}{\pgfqpoint{1.268634in}{1.815539in}}%
\pgfpathclose%
\pgfusepath{stroke,fill}%
\end{pgfscope}%
\begin{pgfscope}%
\pgfpathrectangle{\pgfqpoint{0.100000in}{0.212622in}}{\pgfqpoint{3.696000in}{3.696000in}}%
\pgfusepath{clip}%
\pgfsetbuttcap%
\pgfsetroundjoin%
\definecolor{currentfill}{rgb}{0.121569,0.466667,0.705882}%
\pgfsetfillcolor{currentfill}%
\pgfsetfillopacity{0.891159}%
\pgfsetlinewidth{1.003750pt}%
\definecolor{currentstroke}{rgb}{0.121569,0.466667,0.705882}%
\pgfsetstrokecolor{currentstroke}%
\pgfsetstrokeopacity{0.891159}%
\pgfsetdash{}{0pt}%
\pgfpathmoveto{\pgfqpoint{2.516936in}{2.396790in}}%
\pgfpathcurveto{\pgfqpoint{2.525172in}{2.396790in}}{\pgfqpoint{2.533072in}{2.400062in}}{\pgfqpoint{2.538896in}{2.405886in}}%
\pgfpathcurveto{\pgfqpoint{2.544720in}{2.411710in}}{\pgfqpoint{2.547992in}{2.419610in}}{\pgfqpoint{2.547992in}{2.427846in}}%
\pgfpathcurveto{\pgfqpoint{2.547992in}{2.436083in}}{\pgfqpoint{2.544720in}{2.443983in}}{\pgfqpoint{2.538896in}{2.449807in}}%
\pgfpathcurveto{\pgfqpoint{2.533072in}{2.455630in}}{\pgfqpoint{2.525172in}{2.458903in}}{\pgfqpoint{2.516936in}{2.458903in}}%
\pgfpathcurveto{\pgfqpoint{2.508699in}{2.458903in}}{\pgfqpoint{2.500799in}{2.455630in}}{\pgfqpoint{2.494975in}{2.449807in}}%
\pgfpathcurveto{\pgfqpoint{2.489151in}{2.443983in}}{\pgfqpoint{2.485879in}{2.436083in}}{\pgfqpoint{2.485879in}{2.427846in}}%
\pgfpathcurveto{\pgfqpoint{2.485879in}{2.419610in}}{\pgfqpoint{2.489151in}{2.411710in}}{\pgfqpoint{2.494975in}{2.405886in}}%
\pgfpathcurveto{\pgfqpoint{2.500799in}{2.400062in}}{\pgfqpoint{2.508699in}{2.396790in}}{\pgfqpoint{2.516936in}{2.396790in}}%
\pgfpathclose%
\pgfusepath{stroke,fill}%
\end{pgfscope}%
\begin{pgfscope}%
\pgfpathrectangle{\pgfqpoint{0.100000in}{0.212622in}}{\pgfqpoint{3.696000in}{3.696000in}}%
\pgfusepath{clip}%
\pgfsetbuttcap%
\pgfsetroundjoin%
\definecolor{currentfill}{rgb}{0.121569,0.466667,0.705882}%
\pgfsetfillcolor{currentfill}%
\pgfsetfillopacity{0.892399}%
\pgfsetlinewidth{1.003750pt}%
\definecolor{currentstroke}{rgb}{0.121569,0.466667,0.705882}%
\pgfsetstrokecolor{currentstroke}%
\pgfsetstrokeopacity{0.892399}%
\pgfsetdash{}{0pt}%
\pgfpathmoveto{\pgfqpoint{1.288626in}{1.816818in}}%
\pgfpathcurveto{\pgfqpoint{1.296863in}{1.816818in}}{\pgfqpoint{1.304763in}{1.820090in}}{\pgfqpoint{1.310587in}{1.825914in}}%
\pgfpathcurveto{\pgfqpoint{1.316410in}{1.831738in}}{\pgfqpoint{1.319683in}{1.839638in}}{\pgfqpoint{1.319683in}{1.847874in}}%
\pgfpathcurveto{\pgfqpoint{1.319683in}{1.856111in}}{\pgfqpoint{1.316410in}{1.864011in}}{\pgfqpoint{1.310587in}{1.869835in}}%
\pgfpathcurveto{\pgfqpoint{1.304763in}{1.875659in}}{\pgfqpoint{1.296863in}{1.878931in}}{\pgfqpoint{1.288626in}{1.878931in}}%
\pgfpathcurveto{\pgfqpoint{1.280390in}{1.878931in}}{\pgfqpoint{1.272490in}{1.875659in}}{\pgfqpoint{1.266666in}{1.869835in}}%
\pgfpathcurveto{\pgfqpoint{1.260842in}{1.864011in}}{\pgfqpoint{1.257570in}{1.856111in}}{\pgfqpoint{1.257570in}{1.847874in}}%
\pgfpathcurveto{\pgfqpoint{1.257570in}{1.839638in}}{\pgfqpoint{1.260842in}{1.831738in}}{\pgfqpoint{1.266666in}{1.825914in}}%
\pgfpathcurveto{\pgfqpoint{1.272490in}{1.820090in}}{\pgfqpoint{1.280390in}{1.816818in}}{\pgfqpoint{1.288626in}{1.816818in}}%
\pgfpathclose%
\pgfusepath{stroke,fill}%
\end{pgfscope}%
\begin{pgfscope}%
\pgfpathrectangle{\pgfqpoint{0.100000in}{0.212622in}}{\pgfqpoint{3.696000in}{3.696000in}}%
\pgfusepath{clip}%
\pgfsetbuttcap%
\pgfsetroundjoin%
\definecolor{currentfill}{rgb}{0.121569,0.466667,0.705882}%
\pgfsetfillcolor{currentfill}%
\pgfsetfillopacity{0.892478}%
\pgfsetlinewidth{1.003750pt}%
\definecolor{currentstroke}{rgb}{0.121569,0.466667,0.705882}%
\pgfsetstrokecolor{currentstroke}%
\pgfsetstrokeopacity{0.892478}%
\pgfsetdash{}{0pt}%
\pgfpathmoveto{\pgfqpoint{1.804538in}{2.115305in}}%
\pgfpathcurveto{\pgfqpoint{1.812774in}{2.115305in}}{\pgfqpoint{1.820675in}{2.118578in}}{\pgfqpoint{1.826498in}{2.124402in}}%
\pgfpathcurveto{\pgfqpoint{1.832322in}{2.130226in}}{\pgfqpoint{1.835595in}{2.138126in}}{\pgfqpoint{1.835595in}{2.146362in}}%
\pgfpathcurveto{\pgfqpoint{1.835595in}{2.154598in}}{\pgfqpoint{1.832322in}{2.162498in}}{\pgfqpoint{1.826498in}{2.168322in}}%
\pgfpathcurveto{\pgfqpoint{1.820675in}{2.174146in}}{\pgfqpoint{1.812774in}{2.177418in}}{\pgfqpoint{1.804538in}{2.177418in}}%
\pgfpathcurveto{\pgfqpoint{1.796302in}{2.177418in}}{\pgfqpoint{1.788402in}{2.174146in}}{\pgfqpoint{1.782578in}{2.168322in}}%
\pgfpathcurveto{\pgfqpoint{1.776754in}{2.162498in}}{\pgfqpoint{1.773482in}{2.154598in}}{\pgfqpoint{1.773482in}{2.146362in}}%
\pgfpathcurveto{\pgfqpoint{1.773482in}{2.138126in}}{\pgfqpoint{1.776754in}{2.130226in}}{\pgfqpoint{1.782578in}{2.124402in}}%
\pgfpathcurveto{\pgfqpoint{1.788402in}{2.118578in}}{\pgfqpoint{1.796302in}{2.115305in}}{\pgfqpoint{1.804538in}{2.115305in}}%
\pgfpathclose%
\pgfusepath{stroke,fill}%
\end{pgfscope}%
\begin{pgfscope}%
\pgfpathrectangle{\pgfqpoint{0.100000in}{0.212622in}}{\pgfqpoint{3.696000in}{3.696000in}}%
\pgfusepath{clip}%
\pgfsetbuttcap%
\pgfsetroundjoin%
\definecolor{currentfill}{rgb}{0.121569,0.466667,0.705882}%
\pgfsetfillcolor{currentfill}%
\pgfsetfillopacity{0.892991}%
\pgfsetlinewidth{1.003750pt}%
\definecolor{currentstroke}{rgb}{0.121569,0.466667,0.705882}%
\pgfsetstrokecolor{currentstroke}%
\pgfsetstrokeopacity{0.892991}%
\pgfsetdash{}{0pt}%
\pgfpathmoveto{\pgfqpoint{1.786973in}{2.106748in}}%
\pgfpathcurveto{\pgfqpoint{1.795209in}{2.106748in}}{\pgfqpoint{1.803109in}{2.110020in}}{\pgfqpoint{1.808933in}{2.115844in}}%
\pgfpathcurveto{\pgfqpoint{1.814757in}{2.121668in}}{\pgfqpoint{1.818029in}{2.129568in}}{\pgfqpoint{1.818029in}{2.137804in}}%
\pgfpathcurveto{\pgfqpoint{1.818029in}{2.146041in}}{\pgfqpoint{1.814757in}{2.153941in}}{\pgfqpoint{1.808933in}{2.159765in}}%
\pgfpathcurveto{\pgfqpoint{1.803109in}{2.165589in}}{\pgfqpoint{1.795209in}{2.168861in}}{\pgfqpoint{1.786973in}{2.168861in}}%
\pgfpathcurveto{\pgfqpoint{1.778737in}{2.168861in}}{\pgfqpoint{1.770837in}{2.165589in}}{\pgfqpoint{1.765013in}{2.159765in}}%
\pgfpathcurveto{\pgfqpoint{1.759189in}{2.153941in}}{\pgfqpoint{1.755916in}{2.146041in}}{\pgfqpoint{1.755916in}{2.137804in}}%
\pgfpathcurveto{\pgfqpoint{1.755916in}{2.129568in}}{\pgfqpoint{1.759189in}{2.121668in}}{\pgfqpoint{1.765013in}{2.115844in}}%
\pgfpathcurveto{\pgfqpoint{1.770837in}{2.110020in}}{\pgfqpoint{1.778737in}{2.106748in}}{\pgfqpoint{1.786973in}{2.106748in}}%
\pgfpathclose%
\pgfusepath{stroke,fill}%
\end{pgfscope}%
\begin{pgfscope}%
\pgfpathrectangle{\pgfqpoint{0.100000in}{0.212622in}}{\pgfqpoint{3.696000in}{3.696000in}}%
\pgfusepath{clip}%
\pgfsetbuttcap%
\pgfsetroundjoin%
\definecolor{currentfill}{rgb}{0.121569,0.466667,0.705882}%
\pgfsetfillcolor{currentfill}%
\pgfsetfillopacity{0.893132}%
\pgfsetlinewidth{1.003750pt}%
\definecolor{currentstroke}{rgb}{0.121569,0.466667,0.705882}%
\pgfsetstrokecolor{currentstroke}%
\pgfsetstrokeopacity{0.893132}%
\pgfsetdash{}{0pt}%
\pgfpathmoveto{\pgfqpoint{2.348585in}{2.306500in}}%
\pgfpathcurveto{\pgfqpoint{2.356821in}{2.306500in}}{\pgfqpoint{2.364721in}{2.309773in}}{\pgfqpoint{2.370545in}{2.315597in}}%
\pgfpathcurveto{\pgfqpoint{2.376369in}{2.321421in}}{\pgfqpoint{2.379641in}{2.329321in}}{\pgfqpoint{2.379641in}{2.337557in}}%
\pgfpathcurveto{\pgfqpoint{2.379641in}{2.345793in}}{\pgfqpoint{2.376369in}{2.353693in}}{\pgfqpoint{2.370545in}{2.359517in}}%
\pgfpathcurveto{\pgfqpoint{2.364721in}{2.365341in}}{\pgfqpoint{2.356821in}{2.368613in}}{\pgfqpoint{2.348585in}{2.368613in}}%
\pgfpathcurveto{\pgfqpoint{2.340348in}{2.368613in}}{\pgfqpoint{2.332448in}{2.365341in}}{\pgfqpoint{2.326624in}{2.359517in}}%
\pgfpathcurveto{\pgfqpoint{2.320800in}{2.353693in}}{\pgfqpoint{2.317528in}{2.345793in}}{\pgfqpoint{2.317528in}{2.337557in}}%
\pgfpathcurveto{\pgfqpoint{2.317528in}{2.329321in}}{\pgfqpoint{2.320800in}{2.321421in}}{\pgfqpoint{2.326624in}{2.315597in}}%
\pgfpathcurveto{\pgfqpoint{2.332448in}{2.309773in}}{\pgfqpoint{2.340348in}{2.306500in}}{\pgfqpoint{2.348585in}{2.306500in}}%
\pgfpathclose%
\pgfusepath{stroke,fill}%
\end{pgfscope}%
\begin{pgfscope}%
\pgfpathrectangle{\pgfqpoint{0.100000in}{0.212622in}}{\pgfqpoint{3.696000in}{3.696000in}}%
\pgfusepath{clip}%
\pgfsetbuttcap%
\pgfsetroundjoin%
\definecolor{currentfill}{rgb}{0.121569,0.466667,0.705882}%
\pgfsetfillcolor{currentfill}%
\pgfsetfillopacity{0.893184}%
\pgfsetlinewidth{1.003750pt}%
\definecolor{currentstroke}{rgb}{0.121569,0.466667,0.705882}%
\pgfsetstrokecolor{currentstroke}%
\pgfsetstrokeopacity{0.893184}%
\pgfsetdash{}{0pt}%
\pgfpathmoveto{\pgfqpoint{2.502398in}{2.382366in}}%
\pgfpathcurveto{\pgfqpoint{2.510635in}{2.382366in}}{\pgfqpoint{2.518535in}{2.385638in}}{\pgfqpoint{2.524359in}{2.391462in}}%
\pgfpathcurveto{\pgfqpoint{2.530183in}{2.397286in}}{\pgfqpoint{2.533455in}{2.405186in}}{\pgfqpoint{2.533455in}{2.413422in}}%
\pgfpathcurveto{\pgfqpoint{2.533455in}{2.421658in}}{\pgfqpoint{2.530183in}{2.429558in}}{\pgfqpoint{2.524359in}{2.435382in}}%
\pgfpathcurveto{\pgfqpoint{2.518535in}{2.441206in}}{\pgfqpoint{2.510635in}{2.444479in}}{\pgfqpoint{2.502398in}{2.444479in}}%
\pgfpathcurveto{\pgfqpoint{2.494162in}{2.444479in}}{\pgfqpoint{2.486262in}{2.441206in}}{\pgfqpoint{2.480438in}{2.435382in}}%
\pgfpathcurveto{\pgfqpoint{2.474614in}{2.429558in}}{\pgfqpoint{2.471342in}{2.421658in}}{\pgfqpoint{2.471342in}{2.413422in}}%
\pgfpathcurveto{\pgfqpoint{2.471342in}{2.405186in}}{\pgfqpoint{2.474614in}{2.397286in}}{\pgfqpoint{2.480438in}{2.391462in}}%
\pgfpathcurveto{\pgfqpoint{2.486262in}{2.385638in}}{\pgfqpoint{2.494162in}{2.382366in}}{\pgfqpoint{2.502398in}{2.382366in}}%
\pgfpathclose%
\pgfusepath{stroke,fill}%
\end{pgfscope}%
\begin{pgfscope}%
\pgfpathrectangle{\pgfqpoint{0.100000in}{0.212622in}}{\pgfqpoint{3.696000in}{3.696000in}}%
\pgfusepath{clip}%
\pgfsetbuttcap%
\pgfsetroundjoin%
\definecolor{currentfill}{rgb}{0.121569,0.466667,0.705882}%
\pgfsetfillcolor{currentfill}%
\pgfsetfillopacity{0.893294}%
\pgfsetlinewidth{1.003750pt}%
\definecolor{currentstroke}{rgb}{0.121569,0.466667,0.705882}%
\pgfsetstrokecolor{currentstroke}%
\pgfsetstrokeopacity{0.893294}%
\pgfsetdash{}{0pt}%
\pgfpathmoveto{\pgfqpoint{2.492082in}{2.375990in}}%
\pgfpathcurveto{\pgfqpoint{2.500319in}{2.375990in}}{\pgfqpoint{2.508219in}{2.379263in}}{\pgfqpoint{2.514043in}{2.385087in}}%
\pgfpathcurveto{\pgfqpoint{2.519867in}{2.390911in}}{\pgfqpoint{2.523139in}{2.398811in}}{\pgfqpoint{2.523139in}{2.407047in}}%
\pgfpathcurveto{\pgfqpoint{2.523139in}{2.415283in}}{\pgfqpoint{2.519867in}{2.423183in}}{\pgfqpoint{2.514043in}{2.429007in}}%
\pgfpathcurveto{\pgfqpoint{2.508219in}{2.434831in}}{\pgfqpoint{2.500319in}{2.438103in}}{\pgfqpoint{2.492082in}{2.438103in}}%
\pgfpathcurveto{\pgfqpoint{2.483846in}{2.438103in}}{\pgfqpoint{2.475946in}{2.434831in}}{\pgfqpoint{2.470122in}{2.429007in}}%
\pgfpathcurveto{\pgfqpoint{2.464298in}{2.423183in}}{\pgfqpoint{2.461026in}{2.415283in}}{\pgfqpoint{2.461026in}{2.407047in}}%
\pgfpathcurveto{\pgfqpoint{2.461026in}{2.398811in}}{\pgfqpoint{2.464298in}{2.390911in}}{\pgfqpoint{2.470122in}{2.385087in}}%
\pgfpathcurveto{\pgfqpoint{2.475946in}{2.379263in}}{\pgfqpoint{2.483846in}{2.375990in}}{\pgfqpoint{2.492082in}{2.375990in}}%
\pgfpathclose%
\pgfusepath{stroke,fill}%
\end{pgfscope}%
\begin{pgfscope}%
\pgfpathrectangle{\pgfqpoint{0.100000in}{0.212622in}}{\pgfqpoint{3.696000in}{3.696000in}}%
\pgfusepath{clip}%
\pgfsetbuttcap%
\pgfsetroundjoin%
\definecolor{currentfill}{rgb}{0.121569,0.466667,0.705882}%
\pgfsetfillcolor{currentfill}%
\pgfsetfillopacity{0.893591}%
\pgfsetlinewidth{1.003750pt}%
\definecolor{currentstroke}{rgb}{0.121569,0.466667,0.705882}%
\pgfsetstrokecolor{currentstroke}%
\pgfsetstrokeopacity{0.893591}%
\pgfsetdash{}{0pt}%
\pgfpathmoveto{\pgfqpoint{1.811593in}{2.118488in}}%
\pgfpathcurveto{\pgfqpoint{1.819829in}{2.118488in}}{\pgfqpoint{1.827729in}{2.121760in}}{\pgfqpoint{1.833553in}{2.127584in}}%
\pgfpathcurveto{\pgfqpoint{1.839377in}{2.133408in}}{\pgfqpoint{1.842649in}{2.141308in}}{\pgfqpoint{1.842649in}{2.149544in}}%
\pgfpathcurveto{\pgfqpoint{1.842649in}{2.157780in}}{\pgfqpoint{1.839377in}{2.165680in}}{\pgfqpoint{1.833553in}{2.171504in}}%
\pgfpathcurveto{\pgfqpoint{1.827729in}{2.177328in}}{\pgfqpoint{1.819829in}{2.180601in}}{\pgfqpoint{1.811593in}{2.180601in}}%
\pgfpathcurveto{\pgfqpoint{1.803357in}{2.180601in}}{\pgfqpoint{1.795457in}{2.177328in}}{\pgfqpoint{1.789633in}{2.171504in}}%
\pgfpathcurveto{\pgfqpoint{1.783809in}{2.165680in}}{\pgfqpoint{1.780536in}{2.157780in}}{\pgfqpoint{1.780536in}{2.149544in}}%
\pgfpathcurveto{\pgfqpoint{1.780536in}{2.141308in}}{\pgfqpoint{1.783809in}{2.133408in}}{\pgfqpoint{1.789633in}{2.127584in}}%
\pgfpathcurveto{\pgfqpoint{1.795457in}{2.121760in}}{\pgfqpoint{1.803357in}{2.118488in}}{\pgfqpoint{1.811593in}{2.118488in}}%
\pgfpathclose%
\pgfusepath{stroke,fill}%
\end{pgfscope}%
\begin{pgfscope}%
\pgfpathrectangle{\pgfqpoint{0.100000in}{0.212622in}}{\pgfqpoint{3.696000in}{3.696000in}}%
\pgfusepath{clip}%
\pgfsetbuttcap%
\pgfsetroundjoin%
\definecolor{currentfill}{rgb}{0.121569,0.466667,0.705882}%
\pgfsetfillcolor{currentfill}%
\pgfsetfillopacity{0.893941}%
\pgfsetlinewidth{1.003750pt}%
\definecolor{currentstroke}{rgb}{0.121569,0.466667,0.705882}%
\pgfsetstrokecolor{currentstroke}%
\pgfsetstrokeopacity{0.893941}%
\pgfsetdash{}{0pt}%
\pgfpathmoveto{\pgfqpoint{2.534340in}{2.401460in}}%
\pgfpathcurveto{\pgfqpoint{2.542576in}{2.401460in}}{\pgfqpoint{2.550476in}{2.404732in}}{\pgfqpoint{2.556300in}{2.410556in}}%
\pgfpathcurveto{\pgfqpoint{2.562124in}{2.416380in}}{\pgfqpoint{2.565397in}{2.424280in}}{\pgfqpoint{2.565397in}{2.432516in}}%
\pgfpathcurveto{\pgfqpoint{2.565397in}{2.440753in}}{\pgfqpoint{2.562124in}{2.448653in}}{\pgfqpoint{2.556300in}{2.454477in}}%
\pgfpathcurveto{\pgfqpoint{2.550476in}{2.460301in}}{\pgfqpoint{2.542576in}{2.463573in}}{\pgfqpoint{2.534340in}{2.463573in}}%
\pgfpathcurveto{\pgfqpoint{2.526104in}{2.463573in}}{\pgfqpoint{2.518204in}{2.460301in}}{\pgfqpoint{2.512380in}{2.454477in}}%
\pgfpathcurveto{\pgfqpoint{2.506556in}{2.448653in}}{\pgfqpoint{2.503284in}{2.440753in}}{\pgfqpoint{2.503284in}{2.432516in}}%
\pgfpathcurveto{\pgfqpoint{2.503284in}{2.424280in}}{\pgfqpoint{2.506556in}{2.416380in}}{\pgfqpoint{2.512380in}{2.410556in}}%
\pgfpathcurveto{\pgfqpoint{2.518204in}{2.404732in}}{\pgfqpoint{2.526104in}{2.401460in}}{\pgfqpoint{2.534340in}{2.401460in}}%
\pgfpathclose%
\pgfusepath{stroke,fill}%
\end{pgfscope}%
\begin{pgfscope}%
\pgfpathrectangle{\pgfqpoint{0.100000in}{0.212622in}}{\pgfqpoint{3.696000in}{3.696000in}}%
\pgfusepath{clip}%
\pgfsetbuttcap%
\pgfsetroundjoin%
\definecolor{currentfill}{rgb}{0.121569,0.466667,0.705882}%
\pgfsetfillcolor{currentfill}%
\pgfsetfillopacity{0.894382}%
\pgfsetlinewidth{1.003750pt}%
\definecolor{currentstroke}{rgb}{0.121569,0.466667,0.705882}%
\pgfsetstrokecolor{currentstroke}%
\pgfsetstrokeopacity{0.894382}%
\pgfsetdash{}{0pt}%
\pgfpathmoveto{\pgfqpoint{2.495466in}{2.377525in}}%
\pgfpathcurveto{\pgfqpoint{2.503703in}{2.377525in}}{\pgfqpoint{2.511603in}{2.380797in}}{\pgfqpoint{2.517427in}{2.386621in}}%
\pgfpathcurveto{\pgfqpoint{2.523251in}{2.392445in}}{\pgfqpoint{2.526523in}{2.400345in}}{\pgfqpoint{2.526523in}{2.408581in}}%
\pgfpathcurveto{\pgfqpoint{2.526523in}{2.416818in}}{\pgfqpoint{2.523251in}{2.424718in}}{\pgfqpoint{2.517427in}{2.430542in}}%
\pgfpathcurveto{\pgfqpoint{2.511603in}{2.436365in}}{\pgfqpoint{2.503703in}{2.439638in}}{\pgfqpoint{2.495466in}{2.439638in}}%
\pgfpathcurveto{\pgfqpoint{2.487230in}{2.439638in}}{\pgfqpoint{2.479330in}{2.436365in}}{\pgfqpoint{2.473506in}{2.430542in}}%
\pgfpathcurveto{\pgfqpoint{2.467682in}{2.424718in}}{\pgfqpoint{2.464410in}{2.416818in}}{\pgfqpoint{2.464410in}{2.408581in}}%
\pgfpathcurveto{\pgfqpoint{2.464410in}{2.400345in}}{\pgfqpoint{2.467682in}{2.392445in}}{\pgfqpoint{2.473506in}{2.386621in}}%
\pgfpathcurveto{\pgfqpoint{2.479330in}{2.380797in}}{\pgfqpoint{2.487230in}{2.377525in}}{\pgfqpoint{2.495466in}{2.377525in}}%
\pgfpathclose%
\pgfusepath{stroke,fill}%
\end{pgfscope}%
\begin{pgfscope}%
\pgfpathrectangle{\pgfqpoint{0.100000in}{0.212622in}}{\pgfqpoint{3.696000in}{3.696000in}}%
\pgfusepath{clip}%
\pgfsetbuttcap%
\pgfsetroundjoin%
\definecolor{currentfill}{rgb}{0.121569,0.466667,0.705882}%
\pgfsetfillcolor{currentfill}%
\pgfsetfillopacity{0.894488}%
\pgfsetlinewidth{1.003750pt}%
\definecolor{currentstroke}{rgb}{0.121569,0.466667,0.705882}%
\pgfsetstrokecolor{currentstroke}%
\pgfsetstrokeopacity{0.894488}%
\pgfsetdash{}{0pt}%
\pgfpathmoveto{\pgfqpoint{2.493500in}{2.377057in}}%
\pgfpathcurveto{\pgfqpoint{2.501736in}{2.377057in}}{\pgfqpoint{2.509636in}{2.380329in}}{\pgfqpoint{2.515460in}{2.386153in}}%
\pgfpathcurveto{\pgfqpoint{2.521284in}{2.391977in}}{\pgfqpoint{2.524556in}{2.399877in}}{\pgfqpoint{2.524556in}{2.408113in}}%
\pgfpathcurveto{\pgfqpoint{2.524556in}{2.416350in}}{\pgfqpoint{2.521284in}{2.424250in}}{\pgfqpoint{2.515460in}{2.430074in}}%
\pgfpathcurveto{\pgfqpoint{2.509636in}{2.435897in}}{\pgfqpoint{2.501736in}{2.439170in}}{\pgfqpoint{2.493500in}{2.439170in}}%
\pgfpathcurveto{\pgfqpoint{2.485263in}{2.439170in}}{\pgfqpoint{2.477363in}{2.435897in}}{\pgfqpoint{2.471539in}{2.430074in}}%
\pgfpathcurveto{\pgfqpoint{2.465715in}{2.424250in}}{\pgfqpoint{2.462443in}{2.416350in}}{\pgfqpoint{2.462443in}{2.408113in}}%
\pgfpathcurveto{\pgfqpoint{2.462443in}{2.399877in}}{\pgfqpoint{2.465715in}{2.391977in}}{\pgfqpoint{2.471539in}{2.386153in}}%
\pgfpathcurveto{\pgfqpoint{2.477363in}{2.380329in}}{\pgfqpoint{2.485263in}{2.377057in}}{\pgfqpoint{2.493500in}{2.377057in}}%
\pgfpathclose%
\pgfusepath{stroke,fill}%
\end{pgfscope}%
\begin{pgfscope}%
\pgfpathrectangle{\pgfqpoint{0.100000in}{0.212622in}}{\pgfqpoint{3.696000in}{3.696000in}}%
\pgfusepath{clip}%
\pgfsetbuttcap%
\pgfsetroundjoin%
\definecolor{currentfill}{rgb}{0.121569,0.466667,0.705882}%
\pgfsetfillcolor{currentfill}%
\pgfsetfillopacity{0.894761}%
\pgfsetlinewidth{1.003750pt}%
\definecolor{currentstroke}{rgb}{0.121569,0.466667,0.705882}%
\pgfsetstrokecolor{currentstroke}%
\pgfsetstrokeopacity{0.894761}%
\pgfsetdash{}{0pt}%
\pgfpathmoveto{\pgfqpoint{2.465036in}{2.343649in}}%
\pgfpathcurveto{\pgfqpoint{2.473272in}{2.343649in}}{\pgfqpoint{2.481172in}{2.346922in}}{\pgfqpoint{2.486996in}{2.352746in}}%
\pgfpathcurveto{\pgfqpoint{2.492820in}{2.358570in}}{\pgfqpoint{2.496093in}{2.366470in}}{\pgfqpoint{2.496093in}{2.374706in}}%
\pgfpathcurveto{\pgfqpoint{2.496093in}{2.382942in}}{\pgfqpoint{2.492820in}{2.390842in}}{\pgfqpoint{2.486996in}{2.396666in}}%
\pgfpathcurveto{\pgfqpoint{2.481172in}{2.402490in}}{\pgfqpoint{2.473272in}{2.405762in}}{\pgfqpoint{2.465036in}{2.405762in}}%
\pgfpathcurveto{\pgfqpoint{2.456800in}{2.405762in}}{\pgfqpoint{2.448900in}{2.402490in}}{\pgfqpoint{2.443076in}{2.396666in}}%
\pgfpathcurveto{\pgfqpoint{2.437252in}{2.390842in}}{\pgfqpoint{2.433980in}{2.382942in}}{\pgfqpoint{2.433980in}{2.374706in}}%
\pgfpathcurveto{\pgfqpoint{2.433980in}{2.366470in}}{\pgfqpoint{2.437252in}{2.358570in}}{\pgfqpoint{2.443076in}{2.352746in}}%
\pgfpathcurveto{\pgfqpoint{2.448900in}{2.346922in}}{\pgfqpoint{2.456800in}{2.343649in}}{\pgfqpoint{2.465036in}{2.343649in}}%
\pgfpathclose%
\pgfusepath{stroke,fill}%
\end{pgfscope}%
\begin{pgfscope}%
\pgfpathrectangle{\pgfqpoint{0.100000in}{0.212622in}}{\pgfqpoint{3.696000in}{3.696000in}}%
\pgfusepath{clip}%
\pgfsetbuttcap%
\pgfsetroundjoin%
\definecolor{currentfill}{rgb}{0.121569,0.466667,0.705882}%
\pgfsetfillcolor{currentfill}%
\pgfsetfillopacity{0.895181}%
\pgfsetlinewidth{1.003750pt}%
\definecolor{currentstroke}{rgb}{0.121569,0.466667,0.705882}%
\pgfsetstrokecolor{currentstroke}%
\pgfsetstrokeopacity{0.895181}%
\pgfsetdash{}{0pt}%
\pgfpathmoveto{\pgfqpoint{1.768318in}{2.093149in}}%
\pgfpathcurveto{\pgfqpoint{1.776555in}{2.093149in}}{\pgfqpoint{1.784455in}{2.096422in}}{\pgfqpoint{1.790279in}{2.102246in}}%
\pgfpathcurveto{\pgfqpoint{1.796103in}{2.108070in}}{\pgfqpoint{1.799375in}{2.115970in}}{\pgfqpoint{1.799375in}{2.124206in}}%
\pgfpathcurveto{\pgfqpoint{1.799375in}{2.132442in}}{\pgfqpoint{1.796103in}{2.140342in}}{\pgfqpoint{1.790279in}{2.146166in}}%
\pgfpathcurveto{\pgfqpoint{1.784455in}{2.151990in}}{\pgfqpoint{1.776555in}{2.155262in}}{\pgfqpoint{1.768318in}{2.155262in}}%
\pgfpathcurveto{\pgfqpoint{1.760082in}{2.155262in}}{\pgfqpoint{1.752182in}{2.151990in}}{\pgfqpoint{1.746358in}{2.146166in}}%
\pgfpathcurveto{\pgfqpoint{1.740534in}{2.140342in}}{\pgfqpoint{1.737262in}{2.132442in}}{\pgfqpoint{1.737262in}{2.124206in}}%
\pgfpathcurveto{\pgfqpoint{1.737262in}{2.115970in}}{\pgfqpoint{1.740534in}{2.108070in}}{\pgfqpoint{1.746358in}{2.102246in}}%
\pgfpathcurveto{\pgfqpoint{1.752182in}{2.096422in}}{\pgfqpoint{1.760082in}{2.093149in}}{\pgfqpoint{1.768318in}{2.093149in}}%
\pgfpathclose%
\pgfusepath{stroke,fill}%
\end{pgfscope}%
\begin{pgfscope}%
\pgfpathrectangle{\pgfqpoint{0.100000in}{0.212622in}}{\pgfqpoint{3.696000in}{3.696000in}}%
\pgfusepath{clip}%
\pgfsetbuttcap%
\pgfsetroundjoin%
\definecolor{currentfill}{rgb}{0.121569,0.466667,0.705882}%
\pgfsetfillcolor{currentfill}%
\pgfsetfillopacity{0.895536}%
\pgfsetlinewidth{1.003750pt}%
\definecolor{currentstroke}{rgb}{0.121569,0.466667,0.705882}%
\pgfsetstrokecolor{currentstroke}%
\pgfsetstrokeopacity{0.895536}%
\pgfsetdash{}{0pt}%
\pgfpathmoveto{\pgfqpoint{2.617929in}{2.461330in}}%
\pgfpathcurveto{\pgfqpoint{2.626165in}{2.461330in}}{\pgfqpoint{2.634065in}{2.464603in}}{\pgfqpoint{2.639889in}{2.470427in}}%
\pgfpathcurveto{\pgfqpoint{2.645713in}{2.476250in}}{\pgfqpoint{2.648985in}{2.484150in}}{\pgfqpoint{2.648985in}{2.492387in}}%
\pgfpathcurveto{\pgfqpoint{2.648985in}{2.500623in}}{\pgfqpoint{2.645713in}{2.508523in}}{\pgfqpoint{2.639889in}{2.514347in}}%
\pgfpathcurveto{\pgfqpoint{2.634065in}{2.520171in}}{\pgfqpoint{2.626165in}{2.523443in}}{\pgfqpoint{2.617929in}{2.523443in}}%
\pgfpathcurveto{\pgfqpoint{2.609693in}{2.523443in}}{\pgfqpoint{2.601792in}{2.520171in}}{\pgfqpoint{2.595969in}{2.514347in}}%
\pgfpathcurveto{\pgfqpoint{2.590145in}{2.508523in}}{\pgfqpoint{2.586872in}{2.500623in}}{\pgfqpoint{2.586872in}{2.492387in}}%
\pgfpathcurveto{\pgfqpoint{2.586872in}{2.484150in}}{\pgfqpoint{2.590145in}{2.476250in}}{\pgfqpoint{2.595969in}{2.470427in}}%
\pgfpathcurveto{\pgfqpoint{2.601792in}{2.464603in}}{\pgfqpoint{2.609693in}{2.461330in}}{\pgfqpoint{2.617929in}{2.461330in}}%
\pgfpathclose%
\pgfusepath{stroke,fill}%
\end{pgfscope}%
\begin{pgfscope}%
\pgfpathrectangle{\pgfqpoint{0.100000in}{0.212622in}}{\pgfqpoint{3.696000in}{3.696000in}}%
\pgfusepath{clip}%
\pgfsetbuttcap%
\pgfsetroundjoin%
\definecolor{currentfill}{rgb}{0.121569,0.466667,0.705882}%
\pgfsetfillcolor{currentfill}%
\pgfsetfillopacity{0.895981}%
\pgfsetlinewidth{1.003750pt}%
\definecolor{currentstroke}{rgb}{0.121569,0.466667,0.705882}%
\pgfsetstrokecolor{currentstroke}%
\pgfsetstrokeopacity{0.895981}%
\pgfsetdash{}{0pt}%
\pgfpathmoveto{\pgfqpoint{2.335702in}{2.297314in}}%
\pgfpathcurveto{\pgfqpoint{2.343938in}{2.297314in}}{\pgfqpoint{2.351838in}{2.300586in}}{\pgfqpoint{2.357662in}{2.306410in}}%
\pgfpathcurveto{\pgfqpoint{2.363486in}{2.312234in}}{\pgfqpoint{2.366758in}{2.320134in}}{\pgfqpoint{2.366758in}{2.328371in}}%
\pgfpathcurveto{\pgfqpoint{2.366758in}{2.336607in}}{\pgfqpoint{2.363486in}{2.344507in}}{\pgfqpoint{2.357662in}{2.350331in}}%
\pgfpathcurveto{\pgfqpoint{2.351838in}{2.356155in}}{\pgfqpoint{2.343938in}{2.359427in}}{\pgfqpoint{2.335702in}{2.359427in}}%
\pgfpathcurveto{\pgfqpoint{2.327465in}{2.359427in}}{\pgfqpoint{2.319565in}{2.356155in}}{\pgfqpoint{2.313741in}{2.350331in}}%
\pgfpathcurveto{\pgfqpoint{2.307917in}{2.344507in}}{\pgfqpoint{2.304645in}{2.336607in}}{\pgfqpoint{2.304645in}{2.328371in}}%
\pgfpathcurveto{\pgfqpoint{2.304645in}{2.320134in}}{\pgfqpoint{2.307917in}{2.312234in}}{\pgfqpoint{2.313741in}{2.306410in}}%
\pgfpathcurveto{\pgfqpoint{2.319565in}{2.300586in}}{\pgfqpoint{2.327465in}{2.297314in}}{\pgfqpoint{2.335702in}{2.297314in}}%
\pgfpathclose%
\pgfusepath{stroke,fill}%
\end{pgfscope}%
\begin{pgfscope}%
\pgfpathrectangle{\pgfqpoint{0.100000in}{0.212622in}}{\pgfqpoint{3.696000in}{3.696000in}}%
\pgfusepath{clip}%
\pgfsetbuttcap%
\pgfsetroundjoin%
\definecolor{currentfill}{rgb}{0.121569,0.466667,0.705882}%
\pgfsetfillcolor{currentfill}%
\pgfsetfillopacity{0.897153}%
\pgfsetlinewidth{1.003750pt}%
\definecolor{currentstroke}{rgb}{0.121569,0.466667,0.705882}%
\pgfsetstrokecolor{currentstroke}%
\pgfsetstrokeopacity{0.897153}%
\pgfsetdash{}{0pt}%
\pgfpathmoveto{\pgfqpoint{2.579042in}{2.423953in}}%
\pgfpathcurveto{\pgfqpoint{2.587278in}{2.423953in}}{\pgfqpoint{2.595178in}{2.427226in}}{\pgfqpoint{2.601002in}{2.433050in}}%
\pgfpathcurveto{\pgfqpoint{2.606826in}{2.438874in}}{\pgfqpoint{2.610098in}{2.446774in}}{\pgfqpoint{2.610098in}{2.455010in}}%
\pgfpathcurveto{\pgfqpoint{2.610098in}{2.463246in}}{\pgfqpoint{2.606826in}{2.471146in}}{\pgfqpoint{2.601002in}{2.476970in}}%
\pgfpathcurveto{\pgfqpoint{2.595178in}{2.482794in}}{\pgfqpoint{2.587278in}{2.486066in}}{\pgfqpoint{2.579042in}{2.486066in}}%
\pgfpathcurveto{\pgfqpoint{2.570805in}{2.486066in}}{\pgfqpoint{2.562905in}{2.482794in}}{\pgfqpoint{2.557081in}{2.476970in}}%
\pgfpathcurveto{\pgfqpoint{2.551257in}{2.471146in}}{\pgfqpoint{2.547985in}{2.463246in}}{\pgfqpoint{2.547985in}{2.455010in}}%
\pgfpathcurveto{\pgfqpoint{2.547985in}{2.446774in}}{\pgfqpoint{2.551257in}{2.438874in}}{\pgfqpoint{2.557081in}{2.433050in}}%
\pgfpathcurveto{\pgfqpoint{2.562905in}{2.427226in}}{\pgfqpoint{2.570805in}{2.423953in}}{\pgfqpoint{2.579042in}{2.423953in}}%
\pgfpathclose%
\pgfusepath{stroke,fill}%
\end{pgfscope}%
\begin{pgfscope}%
\pgfpathrectangle{\pgfqpoint{0.100000in}{0.212622in}}{\pgfqpoint{3.696000in}{3.696000in}}%
\pgfusepath{clip}%
\pgfsetbuttcap%
\pgfsetroundjoin%
\definecolor{currentfill}{rgb}{0.121569,0.466667,0.705882}%
\pgfsetfillcolor{currentfill}%
\pgfsetfillopacity{0.897571}%
\pgfsetlinewidth{1.003750pt}%
\definecolor{currentstroke}{rgb}{0.121569,0.466667,0.705882}%
\pgfsetstrokecolor{currentstroke}%
\pgfsetstrokeopacity{0.897571}%
\pgfsetdash{}{0pt}%
\pgfpathmoveto{\pgfqpoint{1.597838in}{1.999768in}}%
\pgfpathcurveto{\pgfqpoint{1.606074in}{1.999768in}}{\pgfqpoint{1.613974in}{2.003041in}}{\pgfqpoint{1.619798in}{2.008864in}}%
\pgfpathcurveto{\pgfqpoint{1.625622in}{2.014688in}}{\pgfqpoint{1.628894in}{2.022588in}}{\pgfqpoint{1.628894in}{2.030825in}}%
\pgfpathcurveto{\pgfqpoint{1.628894in}{2.039061in}}{\pgfqpoint{1.625622in}{2.046961in}}{\pgfqpoint{1.619798in}{2.052785in}}%
\pgfpathcurveto{\pgfqpoint{1.613974in}{2.058609in}}{\pgfqpoint{1.606074in}{2.061881in}}{\pgfqpoint{1.597838in}{2.061881in}}%
\pgfpathcurveto{\pgfqpoint{1.589601in}{2.061881in}}{\pgfqpoint{1.581701in}{2.058609in}}{\pgfqpoint{1.575877in}{2.052785in}}%
\pgfpathcurveto{\pgfqpoint{1.570053in}{2.046961in}}{\pgfqpoint{1.566781in}{2.039061in}}{\pgfqpoint{1.566781in}{2.030825in}}%
\pgfpathcurveto{\pgfqpoint{1.566781in}{2.022588in}}{\pgfqpoint{1.570053in}{2.014688in}}{\pgfqpoint{1.575877in}{2.008864in}}%
\pgfpathcurveto{\pgfqpoint{1.581701in}{2.003041in}}{\pgfqpoint{1.589601in}{1.999768in}}{\pgfqpoint{1.597838in}{1.999768in}}%
\pgfpathclose%
\pgfusepath{stroke,fill}%
\end{pgfscope}%
\begin{pgfscope}%
\pgfpathrectangle{\pgfqpoint{0.100000in}{0.212622in}}{\pgfqpoint{3.696000in}{3.696000in}}%
\pgfusepath{clip}%
\pgfsetbuttcap%
\pgfsetroundjoin%
\definecolor{currentfill}{rgb}{0.121569,0.466667,0.705882}%
\pgfsetfillcolor{currentfill}%
\pgfsetfillopacity{0.898607}%
\pgfsetlinewidth{1.003750pt}%
\definecolor{currentstroke}{rgb}{0.121569,0.466667,0.705882}%
\pgfsetstrokecolor{currentstroke}%
\pgfsetstrokeopacity{0.898607}%
\pgfsetdash{}{0pt}%
\pgfpathmoveto{\pgfqpoint{2.324853in}{2.297462in}}%
\pgfpathcurveto{\pgfqpoint{2.333089in}{2.297462in}}{\pgfqpoint{2.340989in}{2.300734in}}{\pgfqpoint{2.346813in}{2.306558in}}%
\pgfpathcurveto{\pgfqpoint{2.352637in}{2.312382in}}{\pgfqpoint{2.355909in}{2.320282in}}{\pgfqpoint{2.355909in}{2.328518in}}%
\pgfpathcurveto{\pgfqpoint{2.355909in}{2.336754in}}{\pgfqpoint{2.352637in}{2.344654in}}{\pgfqpoint{2.346813in}{2.350478in}}%
\pgfpathcurveto{\pgfqpoint{2.340989in}{2.356302in}}{\pgfqpoint{2.333089in}{2.359575in}}{\pgfqpoint{2.324853in}{2.359575in}}%
\pgfpathcurveto{\pgfqpoint{2.316617in}{2.359575in}}{\pgfqpoint{2.308717in}{2.356302in}}{\pgfqpoint{2.302893in}{2.350478in}}%
\pgfpathcurveto{\pgfqpoint{2.297069in}{2.344654in}}{\pgfqpoint{2.293796in}{2.336754in}}{\pgfqpoint{2.293796in}{2.328518in}}%
\pgfpathcurveto{\pgfqpoint{2.293796in}{2.320282in}}{\pgfqpoint{2.297069in}{2.312382in}}{\pgfqpoint{2.302893in}{2.306558in}}%
\pgfpathcurveto{\pgfqpoint{2.308717in}{2.300734in}}{\pgfqpoint{2.316617in}{2.297462in}}{\pgfqpoint{2.324853in}{2.297462in}}%
\pgfpathclose%
\pgfusepath{stroke,fill}%
\end{pgfscope}%
\begin{pgfscope}%
\pgfpathrectangle{\pgfqpoint{0.100000in}{0.212622in}}{\pgfqpoint{3.696000in}{3.696000in}}%
\pgfusepath{clip}%
\pgfsetbuttcap%
\pgfsetroundjoin%
\definecolor{currentfill}{rgb}{0.121569,0.466667,0.705882}%
\pgfsetfillcolor{currentfill}%
\pgfsetfillopacity{0.899357}%
\pgfsetlinewidth{1.003750pt}%
\definecolor{currentstroke}{rgb}{0.121569,0.466667,0.705882}%
\pgfsetstrokecolor{currentstroke}%
\pgfsetstrokeopacity{0.899357}%
\pgfsetdash{}{0pt}%
\pgfpathmoveto{\pgfqpoint{1.251920in}{1.780702in}}%
\pgfpathcurveto{\pgfqpoint{1.260156in}{1.780702in}}{\pgfqpoint{1.268056in}{1.783974in}}{\pgfqpoint{1.273880in}{1.789798in}}%
\pgfpathcurveto{\pgfqpoint{1.279704in}{1.795622in}}{\pgfqpoint{1.282977in}{1.803522in}}{\pgfqpoint{1.282977in}{1.811758in}}%
\pgfpathcurveto{\pgfqpoint{1.282977in}{1.819995in}}{\pgfqpoint{1.279704in}{1.827895in}}{\pgfqpoint{1.273880in}{1.833719in}}%
\pgfpathcurveto{\pgfqpoint{1.268056in}{1.839542in}}{\pgfqpoint{1.260156in}{1.842815in}}{\pgfqpoint{1.251920in}{1.842815in}}%
\pgfpathcurveto{\pgfqpoint{1.243684in}{1.842815in}}{\pgfqpoint{1.235784in}{1.839542in}}{\pgfqpoint{1.229960in}{1.833719in}}%
\pgfpathcurveto{\pgfqpoint{1.224136in}{1.827895in}}{\pgfqpoint{1.220864in}{1.819995in}}{\pgfqpoint{1.220864in}{1.811758in}}%
\pgfpathcurveto{\pgfqpoint{1.220864in}{1.803522in}}{\pgfqpoint{1.224136in}{1.795622in}}{\pgfqpoint{1.229960in}{1.789798in}}%
\pgfpathcurveto{\pgfqpoint{1.235784in}{1.783974in}}{\pgfqpoint{1.243684in}{1.780702in}}{\pgfqpoint{1.251920in}{1.780702in}}%
\pgfpathclose%
\pgfusepath{stroke,fill}%
\end{pgfscope}%
\begin{pgfscope}%
\pgfpathrectangle{\pgfqpoint{0.100000in}{0.212622in}}{\pgfqpoint{3.696000in}{3.696000in}}%
\pgfusepath{clip}%
\pgfsetbuttcap%
\pgfsetroundjoin%
\definecolor{currentfill}{rgb}{0.121569,0.466667,0.705882}%
\pgfsetfillcolor{currentfill}%
\pgfsetfillopacity{0.899478}%
\pgfsetlinewidth{1.003750pt}%
\definecolor{currentstroke}{rgb}{0.121569,0.466667,0.705882}%
\pgfsetstrokecolor{currentstroke}%
\pgfsetstrokeopacity{0.899478}%
\pgfsetdash{}{0pt}%
\pgfpathmoveto{\pgfqpoint{3.056773in}{2.598334in}}%
\pgfpathcurveto{\pgfqpoint{3.065009in}{2.598334in}}{\pgfqpoint{3.072909in}{2.601606in}}{\pgfqpoint{3.078733in}{2.607430in}}%
\pgfpathcurveto{\pgfqpoint{3.084557in}{2.613254in}}{\pgfqpoint{3.087829in}{2.621154in}}{\pgfqpoint{3.087829in}{2.629391in}}%
\pgfpathcurveto{\pgfqpoint{3.087829in}{2.637627in}}{\pgfqpoint{3.084557in}{2.645527in}}{\pgfqpoint{3.078733in}{2.651351in}}%
\pgfpathcurveto{\pgfqpoint{3.072909in}{2.657175in}}{\pgfqpoint{3.065009in}{2.660447in}}{\pgfqpoint{3.056773in}{2.660447in}}%
\pgfpathcurveto{\pgfqpoint{3.048536in}{2.660447in}}{\pgfqpoint{3.040636in}{2.657175in}}{\pgfqpoint{3.034812in}{2.651351in}}%
\pgfpathcurveto{\pgfqpoint{3.028988in}{2.645527in}}{\pgfqpoint{3.025716in}{2.637627in}}{\pgfqpoint{3.025716in}{2.629391in}}%
\pgfpathcurveto{\pgfqpoint{3.025716in}{2.621154in}}{\pgfqpoint{3.028988in}{2.613254in}}{\pgfqpoint{3.034812in}{2.607430in}}%
\pgfpathcurveto{\pgfqpoint{3.040636in}{2.601606in}}{\pgfqpoint{3.048536in}{2.598334in}}{\pgfqpoint{3.056773in}{2.598334in}}%
\pgfpathclose%
\pgfusepath{stroke,fill}%
\end{pgfscope}%
\begin{pgfscope}%
\pgfpathrectangle{\pgfqpoint{0.100000in}{0.212622in}}{\pgfqpoint{3.696000in}{3.696000in}}%
\pgfusepath{clip}%
\pgfsetbuttcap%
\pgfsetroundjoin%
\definecolor{currentfill}{rgb}{0.121569,0.466667,0.705882}%
\pgfsetfillcolor{currentfill}%
\pgfsetfillopacity{0.899700}%
\pgfsetlinewidth{1.003750pt}%
\definecolor{currentstroke}{rgb}{0.121569,0.466667,0.705882}%
\pgfsetstrokecolor{currentstroke}%
\pgfsetstrokeopacity{0.899700}%
\pgfsetdash{}{0pt}%
\pgfpathmoveto{\pgfqpoint{2.530952in}{2.389802in}}%
\pgfpathcurveto{\pgfqpoint{2.539189in}{2.389802in}}{\pgfqpoint{2.547089in}{2.393074in}}{\pgfqpoint{2.552913in}{2.398898in}}%
\pgfpathcurveto{\pgfqpoint{2.558737in}{2.404722in}}{\pgfqpoint{2.562009in}{2.412622in}}{\pgfqpoint{2.562009in}{2.420858in}}%
\pgfpathcurveto{\pgfqpoint{2.562009in}{2.429094in}}{\pgfqpoint{2.558737in}{2.436994in}}{\pgfqpoint{2.552913in}{2.442818in}}%
\pgfpathcurveto{\pgfqpoint{2.547089in}{2.448642in}}{\pgfqpoint{2.539189in}{2.451915in}}{\pgfqpoint{2.530952in}{2.451915in}}%
\pgfpathcurveto{\pgfqpoint{2.522716in}{2.451915in}}{\pgfqpoint{2.514816in}{2.448642in}}{\pgfqpoint{2.508992in}{2.442818in}}%
\pgfpathcurveto{\pgfqpoint{2.503168in}{2.436994in}}{\pgfqpoint{2.499896in}{2.429094in}}{\pgfqpoint{2.499896in}{2.420858in}}%
\pgfpathcurveto{\pgfqpoint{2.499896in}{2.412622in}}{\pgfqpoint{2.503168in}{2.404722in}}{\pgfqpoint{2.508992in}{2.398898in}}%
\pgfpathcurveto{\pgfqpoint{2.514816in}{2.393074in}}{\pgfqpoint{2.522716in}{2.389802in}}{\pgfqpoint{2.530952in}{2.389802in}}%
\pgfpathclose%
\pgfusepath{stroke,fill}%
\end{pgfscope}%
\begin{pgfscope}%
\pgfpathrectangle{\pgfqpoint{0.100000in}{0.212622in}}{\pgfqpoint{3.696000in}{3.696000in}}%
\pgfusepath{clip}%
\pgfsetbuttcap%
\pgfsetroundjoin%
\definecolor{currentfill}{rgb}{0.121569,0.466667,0.705882}%
\pgfsetfillcolor{currentfill}%
\pgfsetfillopacity{0.899828}%
\pgfsetlinewidth{1.003750pt}%
\definecolor{currentstroke}{rgb}{0.121569,0.466667,0.705882}%
\pgfsetstrokecolor{currentstroke}%
\pgfsetstrokeopacity{0.899828}%
\pgfsetdash{}{0pt}%
\pgfpathmoveto{\pgfqpoint{1.279505in}{1.809793in}}%
\pgfpathcurveto{\pgfqpoint{1.287742in}{1.809793in}}{\pgfqpoint{1.295642in}{1.813065in}}{\pgfqpoint{1.301466in}{1.818889in}}%
\pgfpathcurveto{\pgfqpoint{1.307289in}{1.824713in}}{\pgfqpoint{1.310562in}{1.832613in}}{\pgfqpoint{1.310562in}{1.840849in}}%
\pgfpathcurveto{\pgfqpoint{1.310562in}{1.849085in}}{\pgfqpoint{1.307289in}{1.856985in}}{\pgfqpoint{1.301466in}{1.862809in}}%
\pgfpathcurveto{\pgfqpoint{1.295642in}{1.868633in}}{\pgfqpoint{1.287742in}{1.871906in}}{\pgfqpoint{1.279505in}{1.871906in}}%
\pgfpathcurveto{\pgfqpoint{1.271269in}{1.871906in}}{\pgfqpoint{1.263369in}{1.868633in}}{\pgfqpoint{1.257545in}{1.862809in}}%
\pgfpathcurveto{\pgfqpoint{1.251721in}{1.856985in}}{\pgfqpoint{1.248449in}{1.849085in}}{\pgfqpoint{1.248449in}{1.840849in}}%
\pgfpathcurveto{\pgfqpoint{1.248449in}{1.832613in}}{\pgfqpoint{1.251721in}{1.824713in}}{\pgfqpoint{1.257545in}{1.818889in}}%
\pgfpathcurveto{\pgfqpoint{1.263369in}{1.813065in}}{\pgfqpoint{1.271269in}{1.809793in}}{\pgfqpoint{1.279505in}{1.809793in}}%
\pgfpathclose%
\pgfusepath{stroke,fill}%
\end{pgfscope}%
\begin{pgfscope}%
\pgfpathrectangle{\pgfqpoint{0.100000in}{0.212622in}}{\pgfqpoint{3.696000in}{3.696000in}}%
\pgfusepath{clip}%
\pgfsetbuttcap%
\pgfsetroundjoin%
\definecolor{currentfill}{rgb}{0.121569,0.466667,0.705882}%
\pgfsetfillcolor{currentfill}%
\pgfsetfillopacity{0.900947}%
\pgfsetlinewidth{1.003750pt}%
\definecolor{currentstroke}{rgb}{0.121569,0.466667,0.705882}%
\pgfsetstrokecolor{currentstroke}%
\pgfsetstrokeopacity{0.900947}%
\pgfsetdash{}{0pt}%
\pgfpathmoveto{\pgfqpoint{2.502929in}{2.368702in}}%
\pgfpathcurveto{\pgfqpoint{2.511165in}{2.368702in}}{\pgfqpoint{2.519065in}{2.371975in}}{\pgfqpoint{2.524889in}{2.377798in}}%
\pgfpathcurveto{\pgfqpoint{2.530713in}{2.383622in}}{\pgfqpoint{2.533986in}{2.391522in}}{\pgfqpoint{2.533986in}{2.399759in}}%
\pgfpathcurveto{\pgfqpoint{2.533986in}{2.407995in}}{\pgfqpoint{2.530713in}{2.415895in}}{\pgfqpoint{2.524889in}{2.421719in}}%
\pgfpathcurveto{\pgfqpoint{2.519065in}{2.427543in}}{\pgfqpoint{2.511165in}{2.430815in}}{\pgfqpoint{2.502929in}{2.430815in}}%
\pgfpathcurveto{\pgfqpoint{2.494693in}{2.430815in}}{\pgfqpoint{2.486793in}{2.427543in}}{\pgfqpoint{2.480969in}{2.421719in}}%
\pgfpathcurveto{\pgfqpoint{2.475145in}{2.415895in}}{\pgfqpoint{2.471873in}{2.407995in}}{\pgfqpoint{2.471873in}{2.399759in}}%
\pgfpathcurveto{\pgfqpoint{2.471873in}{2.391522in}}{\pgfqpoint{2.475145in}{2.383622in}}{\pgfqpoint{2.480969in}{2.377798in}}%
\pgfpathcurveto{\pgfqpoint{2.486793in}{2.371975in}}{\pgfqpoint{2.494693in}{2.368702in}}{\pgfqpoint{2.502929in}{2.368702in}}%
\pgfpathclose%
\pgfusepath{stroke,fill}%
\end{pgfscope}%
\begin{pgfscope}%
\pgfpathrectangle{\pgfqpoint{0.100000in}{0.212622in}}{\pgfqpoint{3.696000in}{3.696000in}}%
\pgfusepath{clip}%
\pgfsetbuttcap%
\pgfsetroundjoin%
\definecolor{currentfill}{rgb}{0.121569,0.466667,0.705882}%
\pgfsetfillcolor{currentfill}%
\pgfsetfillopacity{0.901752}%
\pgfsetlinewidth{1.003750pt}%
\definecolor{currentstroke}{rgb}{0.121569,0.466667,0.705882}%
\pgfsetstrokecolor{currentstroke}%
\pgfsetstrokeopacity{0.901752}%
\pgfsetdash{}{0pt}%
\pgfpathmoveto{\pgfqpoint{2.553004in}{2.394247in}}%
\pgfpathcurveto{\pgfqpoint{2.561240in}{2.394247in}}{\pgfqpoint{2.569140in}{2.397519in}}{\pgfqpoint{2.574964in}{2.403343in}}%
\pgfpathcurveto{\pgfqpoint{2.580788in}{2.409167in}}{\pgfqpoint{2.584060in}{2.417067in}}{\pgfqpoint{2.584060in}{2.425303in}}%
\pgfpathcurveto{\pgfqpoint{2.584060in}{2.433539in}}{\pgfqpoint{2.580788in}{2.441439in}}{\pgfqpoint{2.574964in}{2.447263in}}%
\pgfpathcurveto{\pgfqpoint{2.569140in}{2.453087in}}{\pgfqpoint{2.561240in}{2.456360in}}{\pgfqpoint{2.553004in}{2.456360in}}%
\pgfpathcurveto{\pgfqpoint{2.544768in}{2.456360in}}{\pgfqpoint{2.536868in}{2.453087in}}{\pgfqpoint{2.531044in}{2.447263in}}%
\pgfpathcurveto{\pgfqpoint{2.525220in}{2.441439in}}{\pgfqpoint{2.521947in}{2.433539in}}{\pgfqpoint{2.521947in}{2.425303in}}%
\pgfpathcurveto{\pgfqpoint{2.521947in}{2.417067in}}{\pgfqpoint{2.525220in}{2.409167in}}{\pgfqpoint{2.531044in}{2.403343in}}%
\pgfpathcurveto{\pgfqpoint{2.536868in}{2.397519in}}{\pgfqpoint{2.544768in}{2.394247in}}{\pgfqpoint{2.553004in}{2.394247in}}%
\pgfpathclose%
\pgfusepath{stroke,fill}%
\end{pgfscope}%
\begin{pgfscope}%
\pgfpathrectangle{\pgfqpoint{0.100000in}{0.212622in}}{\pgfqpoint{3.696000in}{3.696000in}}%
\pgfusepath{clip}%
\pgfsetbuttcap%
\pgfsetroundjoin%
\definecolor{currentfill}{rgb}{0.121569,0.466667,0.705882}%
\pgfsetfillcolor{currentfill}%
\pgfsetfillopacity{0.902209}%
\pgfsetlinewidth{1.003750pt}%
\definecolor{currentstroke}{rgb}{0.121569,0.466667,0.705882}%
\pgfsetstrokecolor{currentstroke}%
\pgfsetstrokeopacity{0.902209}%
\pgfsetdash{}{0pt}%
\pgfpathmoveto{\pgfqpoint{1.554795in}{1.987518in}}%
\pgfpathcurveto{\pgfqpoint{1.563031in}{1.987518in}}{\pgfqpoint{1.570931in}{1.990791in}}{\pgfqpoint{1.576755in}{1.996615in}}%
\pgfpathcurveto{\pgfqpoint{1.582579in}{2.002438in}}{\pgfqpoint{1.585851in}{2.010338in}}{\pgfqpoint{1.585851in}{2.018575in}}%
\pgfpathcurveto{\pgfqpoint{1.585851in}{2.026811in}}{\pgfqpoint{1.582579in}{2.034711in}}{\pgfqpoint{1.576755in}{2.040535in}}%
\pgfpathcurveto{\pgfqpoint{1.570931in}{2.046359in}}{\pgfqpoint{1.563031in}{2.049631in}}{\pgfqpoint{1.554795in}{2.049631in}}%
\pgfpathcurveto{\pgfqpoint{1.546559in}{2.049631in}}{\pgfqpoint{1.538659in}{2.046359in}}{\pgfqpoint{1.532835in}{2.040535in}}%
\pgfpathcurveto{\pgfqpoint{1.527011in}{2.034711in}}{\pgfqpoint{1.523738in}{2.026811in}}{\pgfqpoint{1.523738in}{2.018575in}}%
\pgfpathcurveto{\pgfqpoint{1.523738in}{2.010338in}}{\pgfqpoint{1.527011in}{2.002438in}}{\pgfqpoint{1.532835in}{1.996615in}}%
\pgfpathcurveto{\pgfqpoint{1.538659in}{1.990791in}}{\pgfqpoint{1.546559in}{1.987518in}}{\pgfqpoint{1.554795in}{1.987518in}}%
\pgfpathclose%
\pgfusepath{stroke,fill}%
\end{pgfscope}%
\begin{pgfscope}%
\pgfpathrectangle{\pgfqpoint{0.100000in}{0.212622in}}{\pgfqpoint{3.696000in}{3.696000in}}%
\pgfusepath{clip}%
\pgfsetbuttcap%
\pgfsetroundjoin%
\definecolor{currentfill}{rgb}{0.121569,0.466667,0.705882}%
\pgfsetfillcolor{currentfill}%
\pgfsetfillopacity{0.902430}%
\pgfsetlinewidth{1.003750pt}%
\definecolor{currentstroke}{rgb}{0.121569,0.466667,0.705882}%
\pgfsetstrokecolor{currentstroke}%
\pgfsetstrokeopacity{0.902430}%
\pgfsetdash{}{0pt}%
\pgfpathmoveto{\pgfqpoint{1.610000in}{2.002685in}}%
\pgfpathcurveto{\pgfqpoint{1.618236in}{2.002685in}}{\pgfqpoint{1.626136in}{2.005958in}}{\pgfqpoint{1.631960in}{2.011782in}}%
\pgfpathcurveto{\pgfqpoint{1.637784in}{2.017606in}}{\pgfqpoint{1.641056in}{2.025506in}}{\pgfqpoint{1.641056in}{2.033742in}}%
\pgfpathcurveto{\pgfqpoint{1.641056in}{2.041978in}}{\pgfqpoint{1.637784in}{2.049878in}}{\pgfqpoint{1.631960in}{2.055702in}}%
\pgfpathcurveto{\pgfqpoint{1.626136in}{2.061526in}}{\pgfqpoint{1.618236in}{2.064798in}}{\pgfqpoint{1.610000in}{2.064798in}}%
\pgfpathcurveto{\pgfqpoint{1.601763in}{2.064798in}}{\pgfqpoint{1.593863in}{2.061526in}}{\pgfqpoint{1.588040in}{2.055702in}}%
\pgfpathcurveto{\pgfqpoint{1.582216in}{2.049878in}}{\pgfqpoint{1.578943in}{2.041978in}}{\pgfqpoint{1.578943in}{2.033742in}}%
\pgfpathcurveto{\pgfqpoint{1.578943in}{2.025506in}}{\pgfqpoint{1.582216in}{2.017606in}}{\pgfqpoint{1.588040in}{2.011782in}}%
\pgfpathcurveto{\pgfqpoint{1.593863in}{2.005958in}}{\pgfqpoint{1.601763in}{2.002685in}}{\pgfqpoint{1.610000in}{2.002685in}}%
\pgfpathclose%
\pgfusepath{stroke,fill}%
\end{pgfscope}%
\begin{pgfscope}%
\pgfpathrectangle{\pgfqpoint{0.100000in}{0.212622in}}{\pgfqpoint{3.696000in}{3.696000in}}%
\pgfusepath{clip}%
\pgfsetbuttcap%
\pgfsetroundjoin%
\definecolor{currentfill}{rgb}{0.121569,0.466667,0.705882}%
\pgfsetfillcolor{currentfill}%
\pgfsetfillopacity{0.902452}%
\pgfsetlinewidth{1.003750pt}%
\definecolor{currentstroke}{rgb}{0.121569,0.466667,0.705882}%
\pgfsetstrokecolor{currentstroke}%
\pgfsetstrokeopacity{0.902452}%
\pgfsetdash{}{0pt}%
\pgfpathmoveto{\pgfqpoint{2.537846in}{2.396827in}}%
\pgfpathcurveto{\pgfqpoint{2.546082in}{2.396827in}}{\pgfqpoint{2.553982in}{2.400099in}}{\pgfqpoint{2.559806in}{2.405923in}}%
\pgfpathcurveto{\pgfqpoint{2.565630in}{2.411747in}}{\pgfqpoint{2.568903in}{2.419647in}}{\pgfqpoint{2.568903in}{2.427883in}}%
\pgfpathcurveto{\pgfqpoint{2.568903in}{2.436119in}}{\pgfqpoint{2.565630in}{2.444020in}}{\pgfqpoint{2.559806in}{2.449843in}}%
\pgfpathcurveto{\pgfqpoint{2.553982in}{2.455667in}}{\pgfqpoint{2.546082in}{2.458940in}}{\pgfqpoint{2.537846in}{2.458940in}}%
\pgfpathcurveto{\pgfqpoint{2.529610in}{2.458940in}}{\pgfqpoint{2.521710in}{2.455667in}}{\pgfqpoint{2.515886in}{2.449843in}}%
\pgfpathcurveto{\pgfqpoint{2.510062in}{2.444020in}}{\pgfqpoint{2.506790in}{2.436119in}}{\pgfqpoint{2.506790in}{2.427883in}}%
\pgfpathcurveto{\pgfqpoint{2.506790in}{2.419647in}}{\pgfqpoint{2.510062in}{2.411747in}}{\pgfqpoint{2.515886in}{2.405923in}}%
\pgfpathcurveto{\pgfqpoint{2.521710in}{2.400099in}}{\pgfqpoint{2.529610in}{2.396827in}}{\pgfqpoint{2.537846in}{2.396827in}}%
\pgfpathclose%
\pgfusepath{stroke,fill}%
\end{pgfscope}%
\begin{pgfscope}%
\pgfpathrectangle{\pgfqpoint{0.100000in}{0.212622in}}{\pgfqpoint{3.696000in}{3.696000in}}%
\pgfusepath{clip}%
\pgfsetbuttcap%
\pgfsetroundjoin%
\definecolor{currentfill}{rgb}{0.121569,0.466667,0.705882}%
\pgfsetfillcolor{currentfill}%
\pgfsetfillopacity{0.902526}%
\pgfsetlinewidth{1.003750pt}%
\definecolor{currentstroke}{rgb}{0.121569,0.466667,0.705882}%
\pgfsetstrokecolor{currentstroke}%
\pgfsetstrokeopacity{0.902526}%
\pgfsetdash{}{0pt}%
\pgfpathmoveto{\pgfqpoint{1.569995in}{1.976697in}}%
\pgfpathcurveto{\pgfqpoint{1.578231in}{1.976697in}}{\pgfqpoint{1.586131in}{1.979969in}}{\pgfqpoint{1.591955in}{1.985793in}}%
\pgfpathcurveto{\pgfqpoint{1.597779in}{1.991617in}}{\pgfqpoint{1.601052in}{1.999517in}}{\pgfqpoint{1.601052in}{2.007754in}}%
\pgfpathcurveto{\pgfqpoint{1.601052in}{2.015990in}}{\pgfqpoint{1.597779in}{2.023890in}}{\pgfqpoint{1.591955in}{2.029714in}}%
\pgfpathcurveto{\pgfqpoint{1.586131in}{2.035538in}}{\pgfqpoint{1.578231in}{2.038810in}}{\pgfqpoint{1.569995in}{2.038810in}}%
\pgfpathcurveto{\pgfqpoint{1.561759in}{2.038810in}}{\pgfqpoint{1.553859in}{2.035538in}}{\pgfqpoint{1.548035in}{2.029714in}}%
\pgfpathcurveto{\pgfqpoint{1.542211in}{2.023890in}}{\pgfqpoint{1.538939in}{2.015990in}}{\pgfqpoint{1.538939in}{2.007754in}}%
\pgfpathcurveto{\pgfqpoint{1.538939in}{1.999517in}}{\pgfqpoint{1.542211in}{1.991617in}}{\pgfqpoint{1.548035in}{1.985793in}}%
\pgfpathcurveto{\pgfqpoint{1.553859in}{1.979969in}}{\pgfqpoint{1.561759in}{1.976697in}}{\pgfqpoint{1.569995in}{1.976697in}}%
\pgfpathclose%
\pgfusepath{stroke,fill}%
\end{pgfscope}%
\begin{pgfscope}%
\pgfpathrectangle{\pgfqpoint{0.100000in}{0.212622in}}{\pgfqpoint{3.696000in}{3.696000in}}%
\pgfusepath{clip}%
\pgfsetbuttcap%
\pgfsetroundjoin%
\definecolor{currentfill}{rgb}{0.121569,0.466667,0.705882}%
\pgfsetfillcolor{currentfill}%
\pgfsetfillopacity{0.902870}%
\pgfsetlinewidth{1.003750pt}%
\definecolor{currentstroke}{rgb}{0.121569,0.466667,0.705882}%
\pgfsetstrokecolor{currentstroke}%
\pgfsetstrokeopacity{0.902870}%
\pgfsetdash{}{0pt}%
\pgfpathmoveto{\pgfqpoint{2.518615in}{2.375825in}}%
\pgfpathcurveto{\pgfqpoint{2.526851in}{2.375825in}}{\pgfqpoint{2.534751in}{2.379097in}}{\pgfqpoint{2.540575in}{2.384921in}}%
\pgfpathcurveto{\pgfqpoint{2.546399in}{2.390745in}}{\pgfqpoint{2.549671in}{2.398645in}}{\pgfqpoint{2.549671in}{2.406881in}}%
\pgfpathcurveto{\pgfqpoint{2.549671in}{2.415118in}}{\pgfqpoint{2.546399in}{2.423018in}}{\pgfqpoint{2.540575in}{2.428842in}}%
\pgfpathcurveto{\pgfqpoint{2.534751in}{2.434666in}}{\pgfqpoint{2.526851in}{2.437938in}}{\pgfqpoint{2.518615in}{2.437938in}}%
\pgfpathcurveto{\pgfqpoint{2.510378in}{2.437938in}}{\pgfqpoint{2.502478in}{2.434666in}}{\pgfqpoint{2.496654in}{2.428842in}}%
\pgfpathcurveto{\pgfqpoint{2.490830in}{2.423018in}}{\pgfqpoint{2.487558in}{2.415118in}}{\pgfqpoint{2.487558in}{2.406881in}}%
\pgfpathcurveto{\pgfqpoint{2.487558in}{2.398645in}}{\pgfqpoint{2.490830in}{2.390745in}}{\pgfqpoint{2.496654in}{2.384921in}}%
\pgfpathcurveto{\pgfqpoint{2.502478in}{2.379097in}}{\pgfqpoint{2.510378in}{2.375825in}}{\pgfqpoint{2.518615in}{2.375825in}}%
\pgfpathclose%
\pgfusepath{stroke,fill}%
\end{pgfscope}%
\begin{pgfscope}%
\pgfpathrectangle{\pgfqpoint{0.100000in}{0.212622in}}{\pgfqpoint{3.696000in}{3.696000in}}%
\pgfusepath{clip}%
\pgfsetbuttcap%
\pgfsetroundjoin%
\definecolor{currentfill}{rgb}{0.121569,0.466667,0.705882}%
\pgfsetfillcolor{currentfill}%
\pgfsetfillopacity{0.903067}%
\pgfsetlinewidth{1.003750pt}%
\definecolor{currentstroke}{rgb}{0.121569,0.466667,0.705882}%
\pgfsetstrokecolor{currentstroke}%
\pgfsetstrokeopacity{0.903067}%
\pgfsetdash{}{0pt}%
\pgfpathmoveto{\pgfqpoint{3.058254in}{2.603761in}}%
\pgfpathcurveto{\pgfqpoint{3.066490in}{2.603761in}}{\pgfqpoint{3.074391in}{2.607034in}}{\pgfqpoint{3.080214in}{2.612857in}}%
\pgfpathcurveto{\pgfqpoint{3.086038in}{2.618681in}}{\pgfqpoint{3.089311in}{2.626581in}}{\pgfqpoint{3.089311in}{2.634818in}}%
\pgfpathcurveto{\pgfqpoint{3.089311in}{2.643054in}}{\pgfqpoint{3.086038in}{2.650954in}}{\pgfqpoint{3.080214in}{2.656778in}}%
\pgfpathcurveto{\pgfqpoint{3.074391in}{2.662602in}}{\pgfqpoint{3.066490in}{2.665874in}}{\pgfqpoint{3.058254in}{2.665874in}}%
\pgfpathcurveto{\pgfqpoint{3.050018in}{2.665874in}}{\pgfqpoint{3.042118in}{2.662602in}}{\pgfqpoint{3.036294in}{2.656778in}}%
\pgfpathcurveto{\pgfqpoint{3.030470in}{2.650954in}}{\pgfqpoint{3.027198in}{2.643054in}}{\pgfqpoint{3.027198in}{2.634818in}}%
\pgfpathcurveto{\pgfqpoint{3.027198in}{2.626581in}}{\pgfqpoint{3.030470in}{2.618681in}}{\pgfqpoint{3.036294in}{2.612857in}}%
\pgfpathcurveto{\pgfqpoint{3.042118in}{2.607034in}}{\pgfqpoint{3.050018in}{2.603761in}}{\pgfqpoint{3.058254in}{2.603761in}}%
\pgfpathclose%
\pgfusepath{stroke,fill}%
\end{pgfscope}%
\begin{pgfscope}%
\pgfpathrectangle{\pgfqpoint{0.100000in}{0.212622in}}{\pgfqpoint{3.696000in}{3.696000in}}%
\pgfusepath{clip}%
\pgfsetbuttcap%
\pgfsetroundjoin%
\definecolor{currentfill}{rgb}{0.121569,0.466667,0.705882}%
\pgfsetfillcolor{currentfill}%
\pgfsetfillopacity{0.903515}%
\pgfsetlinewidth{1.003750pt}%
\definecolor{currentstroke}{rgb}{0.121569,0.466667,0.705882}%
\pgfsetstrokecolor{currentstroke}%
\pgfsetstrokeopacity{0.903515}%
\pgfsetdash{}{0pt}%
\pgfpathmoveto{\pgfqpoint{1.611337in}{2.002712in}}%
\pgfpathcurveto{\pgfqpoint{1.619573in}{2.002712in}}{\pgfqpoint{1.627473in}{2.005984in}}{\pgfqpoint{1.633297in}{2.011808in}}%
\pgfpathcurveto{\pgfqpoint{1.639121in}{2.017632in}}{\pgfqpoint{1.642394in}{2.025532in}}{\pgfqpoint{1.642394in}{2.033768in}}%
\pgfpathcurveto{\pgfqpoint{1.642394in}{2.042005in}}{\pgfqpoint{1.639121in}{2.049905in}}{\pgfqpoint{1.633297in}{2.055729in}}%
\pgfpathcurveto{\pgfqpoint{1.627473in}{2.061553in}}{\pgfqpoint{1.619573in}{2.064825in}}{\pgfqpoint{1.611337in}{2.064825in}}%
\pgfpathcurveto{\pgfqpoint{1.603101in}{2.064825in}}{\pgfqpoint{1.595201in}{2.061553in}}{\pgfqpoint{1.589377in}{2.055729in}}%
\pgfpathcurveto{\pgfqpoint{1.583553in}{2.049905in}}{\pgfqpoint{1.580281in}{2.042005in}}{\pgfqpoint{1.580281in}{2.033768in}}%
\pgfpathcurveto{\pgfqpoint{1.580281in}{2.025532in}}{\pgfqpoint{1.583553in}{2.017632in}}{\pgfqpoint{1.589377in}{2.011808in}}%
\pgfpathcurveto{\pgfqpoint{1.595201in}{2.005984in}}{\pgfqpoint{1.603101in}{2.002712in}}{\pgfqpoint{1.611337in}{2.002712in}}%
\pgfpathclose%
\pgfusepath{stroke,fill}%
\end{pgfscope}%
\begin{pgfscope}%
\pgfpathrectangle{\pgfqpoint{0.100000in}{0.212622in}}{\pgfqpoint{3.696000in}{3.696000in}}%
\pgfusepath{clip}%
\pgfsetbuttcap%
\pgfsetroundjoin%
\definecolor{currentfill}{rgb}{0.121569,0.466667,0.705882}%
\pgfsetfillcolor{currentfill}%
\pgfsetfillopacity{0.904023}%
\pgfsetlinewidth{1.003750pt}%
\definecolor{currentstroke}{rgb}{0.121569,0.466667,0.705882}%
\pgfsetstrokecolor{currentstroke}%
\pgfsetstrokeopacity{0.904023}%
\pgfsetdash{}{0pt}%
\pgfpathmoveto{\pgfqpoint{1.280698in}{1.810469in}}%
\pgfpathcurveto{\pgfqpoint{1.288935in}{1.810469in}}{\pgfqpoint{1.296835in}{1.813742in}}{\pgfqpoint{1.302659in}{1.819566in}}%
\pgfpathcurveto{\pgfqpoint{1.308482in}{1.825390in}}{\pgfqpoint{1.311755in}{1.833290in}}{\pgfqpoint{1.311755in}{1.841526in}}%
\pgfpathcurveto{\pgfqpoint{1.311755in}{1.849762in}}{\pgfqpoint{1.308482in}{1.857662in}}{\pgfqpoint{1.302659in}{1.863486in}}%
\pgfpathcurveto{\pgfqpoint{1.296835in}{1.869310in}}{\pgfqpoint{1.288935in}{1.872582in}}{\pgfqpoint{1.280698in}{1.872582in}}%
\pgfpathcurveto{\pgfqpoint{1.272462in}{1.872582in}}{\pgfqpoint{1.264562in}{1.869310in}}{\pgfqpoint{1.258738in}{1.863486in}}%
\pgfpathcurveto{\pgfqpoint{1.252914in}{1.857662in}}{\pgfqpoint{1.249642in}{1.849762in}}{\pgfqpoint{1.249642in}{1.841526in}}%
\pgfpathcurveto{\pgfqpoint{1.249642in}{1.833290in}}{\pgfqpoint{1.252914in}{1.825390in}}{\pgfqpoint{1.258738in}{1.819566in}}%
\pgfpathcurveto{\pgfqpoint{1.264562in}{1.813742in}}{\pgfqpoint{1.272462in}{1.810469in}}{\pgfqpoint{1.280698in}{1.810469in}}%
\pgfpathclose%
\pgfusepath{stroke,fill}%
\end{pgfscope}%
\begin{pgfscope}%
\pgfpathrectangle{\pgfqpoint{0.100000in}{0.212622in}}{\pgfqpoint{3.696000in}{3.696000in}}%
\pgfusepath{clip}%
\pgfsetbuttcap%
\pgfsetroundjoin%
\definecolor{currentfill}{rgb}{0.121569,0.466667,0.705882}%
\pgfsetfillcolor{currentfill}%
\pgfsetfillopacity{0.904159}%
\pgfsetlinewidth{1.003750pt}%
\definecolor{currentstroke}{rgb}{0.121569,0.466667,0.705882}%
\pgfsetstrokecolor{currentstroke}%
\pgfsetstrokeopacity{0.904159}%
\pgfsetdash{}{0pt}%
\pgfpathmoveto{\pgfqpoint{1.593796in}{1.994062in}}%
\pgfpathcurveto{\pgfqpoint{1.602033in}{1.994062in}}{\pgfqpoint{1.609933in}{1.997335in}}{\pgfqpoint{1.615757in}{2.003159in}}%
\pgfpathcurveto{\pgfqpoint{1.621581in}{2.008983in}}{\pgfqpoint{1.624853in}{2.016883in}}{\pgfqpoint{1.624853in}{2.025119in}}%
\pgfpathcurveto{\pgfqpoint{1.624853in}{2.033355in}}{\pgfqpoint{1.621581in}{2.041255in}}{\pgfqpoint{1.615757in}{2.047079in}}%
\pgfpathcurveto{\pgfqpoint{1.609933in}{2.052903in}}{\pgfqpoint{1.602033in}{2.056175in}}{\pgfqpoint{1.593796in}{2.056175in}}%
\pgfpathcurveto{\pgfqpoint{1.585560in}{2.056175in}}{\pgfqpoint{1.577660in}{2.052903in}}{\pgfqpoint{1.571836in}{2.047079in}}%
\pgfpathcurveto{\pgfqpoint{1.566012in}{2.041255in}}{\pgfqpoint{1.562740in}{2.033355in}}{\pgfqpoint{1.562740in}{2.025119in}}%
\pgfpathcurveto{\pgfqpoint{1.562740in}{2.016883in}}{\pgfqpoint{1.566012in}{2.008983in}}{\pgfqpoint{1.571836in}{2.003159in}}%
\pgfpathcurveto{\pgfqpoint{1.577660in}{1.997335in}}{\pgfqpoint{1.585560in}{1.994062in}}{\pgfqpoint{1.593796in}{1.994062in}}%
\pgfpathclose%
\pgfusepath{stroke,fill}%
\end{pgfscope}%
\begin{pgfscope}%
\pgfpathrectangle{\pgfqpoint{0.100000in}{0.212622in}}{\pgfqpoint{3.696000in}{3.696000in}}%
\pgfusepath{clip}%
\pgfsetbuttcap%
\pgfsetroundjoin%
\definecolor{currentfill}{rgb}{0.121569,0.466667,0.705882}%
\pgfsetfillcolor{currentfill}%
\pgfsetfillopacity{0.904440}%
\pgfsetlinewidth{1.003750pt}%
\definecolor{currentstroke}{rgb}{0.121569,0.466667,0.705882}%
\pgfsetstrokecolor{currentstroke}%
\pgfsetstrokeopacity{0.904440}%
\pgfsetdash{}{0pt}%
\pgfpathmoveto{\pgfqpoint{2.619400in}{2.461172in}}%
\pgfpathcurveto{\pgfqpoint{2.627636in}{2.461172in}}{\pgfqpoint{2.635536in}{2.464444in}}{\pgfqpoint{2.641360in}{2.470268in}}%
\pgfpathcurveto{\pgfqpoint{2.647184in}{2.476092in}}{\pgfqpoint{2.650456in}{2.483992in}}{\pgfqpoint{2.650456in}{2.492228in}}%
\pgfpathcurveto{\pgfqpoint{2.650456in}{2.500464in}}{\pgfqpoint{2.647184in}{2.508364in}}{\pgfqpoint{2.641360in}{2.514188in}}%
\pgfpathcurveto{\pgfqpoint{2.635536in}{2.520012in}}{\pgfqpoint{2.627636in}{2.523285in}}{\pgfqpoint{2.619400in}{2.523285in}}%
\pgfpathcurveto{\pgfqpoint{2.611163in}{2.523285in}}{\pgfqpoint{2.603263in}{2.520012in}}{\pgfqpoint{2.597439in}{2.514188in}}%
\pgfpathcurveto{\pgfqpoint{2.591616in}{2.508364in}}{\pgfqpoint{2.588343in}{2.500464in}}{\pgfqpoint{2.588343in}{2.492228in}}%
\pgfpathcurveto{\pgfqpoint{2.588343in}{2.483992in}}{\pgfqpoint{2.591616in}{2.476092in}}{\pgfqpoint{2.597439in}{2.470268in}}%
\pgfpathcurveto{\pgfqpoint{2.603263in}{2.464444in}}{\pgfqpoint{2.611163in}{2.461172in}}{\pgfqpoint{2.619400in}{2.461172in}}%
\pgfpathclose%
\pgfusepath{stroke,fill}%
\end{pgfscope}%
\begin{pgfscope}%
\pgfpathrectangle{\pgfqpoint{0.100000in}{0.212622in}}{\pgfqpoint{3.696000in}{3.696000in}}%
\pgfusepath{clip}%
\pgfsetbuttcap%
\pgfsetroundjoin%
\definecolor{currentfill}{rgb}{0.121569,0.466667,0.705882}%
\pgfsetfillcolor{currentfill}%
\pgfsetfillopacity{0.904440}%
\pgfsetlinewidth{1.003750pt}%
\definecolor{currentstroke}{rgb}{0.121569,0.466667,0.705882}%
\pgfsetstrokecolor{currentstroke}%
\pgfsetstrokeopacity{0.904440}%
\pgfsetdash{}{0pt}%
\pgfpathmoveto{\pgfqpoint{2.525094in}{2.382238in}}%
\pgfpathcurveto{\pgfqpoint{2.533331in}{2.382238in}}{\pgfqpoint{2.541231in}{2.385511in}}{\pgfqpoint{2.547055in}{2.391335in}}%
\pgfpathcurveto{\pgfqpoint{2.552879in}{2.397159in}}{\pgfqpoint{2.556151in}{2.405059in}}{\pgfqpoint{2.556151in}{2.413295in}}%
\pgfpathcurveto{\pgfqpoint{2.556151in}{2.421531in}}{\pgfqpoint{2.552879in}{2.429431in}}{\pgfqpoint{2.547055in}{2.435255in}}%
\pgfpathcurveto{\pgfqpoint{2.541231in}{2.441079in}}{\pgfqpoint{2.533331in}{2.444351in}}{\pgfqpoint{2.525094in}{2.444351in}}%
\pgfpathcurveto{\pgfqpoint{2.516858in}{2.444351in}}{\pgfqpoint{2.508958in}{2.441079in}}{\pgfqpoint{2.503134in}{2.435255in}}%
\pgfpathcurveto{\pgfqpoint{2.497310in}{2.429431in}}{\pgfqpoint{2.494038in}{2.421531in}}{\pgfqpoint{2.494038in}{2.413295in}}%
\pgfpathcurveto{\pgfqpoint{2.494038in}{2.405059in}}{\pgfqpoint{2.497310in}{2.397159in}}{\pgfqpoint{2.503134in}{2.391335in}}%
\pgfpathcurveto{\pgfqpoint{2.508958in}{2.385511in}}{\pgfqpoint{2.516858in}{2.382238in}}{\pgfqpoint{2.525094in}{2.382238in}}%
\pgfpathclose%
\pgfusepath{stroke,fill}%
\end{pgfscope}%
\begin{pgfscope}%
\pgfpathrectangle{\pgfqpoint{0.100000in}{0.212622in}}{\pgfqpoint{3.696000in}{3.696000in}}%
\pgfusepath{clip}%
\pgfsetbuttcap%
\pgfsetroundjoin%
\definecolor{currentfill}{rgb}{0.121569,0.466667,0.705882}%
\pgfsetfillcolor{currentfill}%
\pgfsetfillopacity{0.904521}%
\pgfsetlinewidth{1.003750pt}%
\definecolor{currentstroke}{rgb}{0.121569,0.466667,0.705882}%
\pgfsetstrokecolor{currentstroke}%
\pgfsetstrokeopacity{0.904521}%
\pgfsetdash{}{0pt}%
\pgfpathmoveto{\pgfqpoint{2.538482in}{2.387524in}}%
\pgfpathcurveto{\pgfqpoint{2.546718in}{2.387524in}}{\pgfqpoint{2.554618in}{2.390797in}}{\pgfqpoint{2.560442in}{2.396621in}}%
\pgfpathcurveto{\pgfqpoint{2.566266in}{2.402445in}}{\pgfqpoint{2.569538in}{2.410345in}}{\pgfqpoint{2.569538in}{2.418581in}}%
\pgfpathcurveto{\pgfqpoint{2.569538in}{2.426817in}}{\pgfqpoint{2.566266in}{2.434717in}}{\pgfqpoint{2.560442in}{2.440541in}}%
\pgfpathcurveto{\pgfqpoint{2.554618in}{2.446365in}}{\pgfqpoint{2.546718in}{2.449637in}}{\pgfqpoint{2.538482in}{2.449637in}}%
\pgfpathcurveto{\pgfqpoint{2.530246in}{2.449637in}}{\pgfqpoint{2.522345in}{2.446365in}}{\pgfqpoint{2.516522in}{2.440541in}}%
\pgfpathcurveto{\pgfqpoint{2.510698in}{2.434717in}}{\pgfqpoint{2.507425in}{2.426817in}}{\pgfqpoint{2.507425in}{2.418581in}}%
\pgfpathcurveto{\pgfqpoint{2.507425in}{2.410345in}}{\pgfqpoint{2.510698in}{2.402445in}}{\pgfqpoint{2.516522in}{2.396621in}}%
\pgfpathcurveto{\pgfqpoint{2.522345in}{2.390797in}}{\pgfqpoint{2.530246in}{2.387524in}}{\pgfqpoint{2.538482in}{2.387524in}}%
\pgfpathclose%
\pgfusepath{stroke,fill}%
\end{pgfscope}%
\begin{pgfscope}%
\pgfpathrectangle{\pgfqpoint{0.100000in}{0.212622in}}{\pgfqpoint{3.696000in}{3.696000in}}%
\pgfusepath{clip}%
\pgfsetbuttcap%
\pgfsetroundjoin%
\definecolor{currentfill}{rgb}{0.121569,0.466667,0.705882}%
\pgfsetfillcolor{currentfill}%
\pgfsetfillopacity{0.904689}%
\pgfsetlinewidth{1.003750pt}%
\definecolor{currentstroke}{rgb}{0.121569,0.466667,0.705882}%
\pgfsetstrokecolor{currentstroke}%
\pgfsetstrokeopacity{0.904689}%
\pgfsetdash{}{0pt}%
\pgfpathmoveto{\pgfqpoint{1.810601in}{2.111260in}}%
\pgfpathcurveto{\pgfqpoint{1.818837in}{2.111260in}}{\pgfqpoint{1.826737in}{2.114532in}}{\pgfqpoint{1.832561in}{2.120356in}}%
\pgfpathcurveto{\pgfqpoint{1.838385in}{2.126180in}}{\pgfqpoint{1.841657in}{2.134080in}}{\pgfqpoint{1.841657in}{2.142317in}}%
\pgfpathcurveto{\pgfqpoint{1.841657in}{2.150553in}}{\pgfqpoint{1.838385in}{2.158453in}}{\pgfqpoint{1.832561in}{2.164277in}}%
\pgfpathcurveto{\pgfqpoint{1.826737in}{2.170101in}}{\pgfqpoint{1.818837in}{2.173373in}}{\pgfqpoint{1.810601in}{2.173373in}}%
\pgfpathcurveto{\pgfqpoint{1.802365in}{2.173373in}}{\pgfqpoint{1.794465in}{2.170101in}}{\pgfqpoint{1.788641in}{2.164277in}}%
\pgfpathcurveto{\pgfqpoint{1.782817in}{2.158453in}}{\pgfqpoint{1.779544in}{2.150553in}}{\pgfqpoint{1.779544in}{2.142317in}}%
\pgfpathcurveto{\pgfqpoint{1.779544in}{2.134080in}}{\pgfqpoint{1.782817in}{2.126180in}}{\pgfqpoint{1.788641in}{2.120356in}}%
\pgfpathcurveto{\pgfqpoint{1.794465in}{2.114532in}}{\pgfqpoint{1.802365in}{2.111260in}}{\pgfqpoint{1.810601in}{2.111260in}}%
\pgfpathclose%
\pgfusepath{stroke,fill}%
\end{pgfscope}%
\begin{pgfscope}%
\pgfpathrectangle{\pgfqpoint{0.100000in}{0.212622in}}{\pgfqpoint{3.696000in}{3.696000in}}%
\pgfusepath{clip}%
\pgfsetbuttcap%
\pgfsetroundjoin%
\definecolor{currentfill}{rgb}{0.121569,0.466667,0.705882}%
\pgfsetfillcolor{currentfill}%
\pgfsetfillopacity{0.906334}%
\pgfsetlinewidth{1.003750pt}%
\definecolor{currentstroke}{rgb}{0.121569,0.466667,0.705882}%
\pgfsetstrokecolor{currentstroke}%
\pgfsetstrokeopacity{0.906334}%
\pgfsetdash{}{0pt}%
\pgfpathmoveto{\pgfqpoint{1.642444in}{2.030385in}}%
\pgfpathcurveto{\pgfqpoint{1.650680in}{2.030385in}}{\pgfqpoint{1.658580in}{2.033657in}}{\pgfqpoint{1.664404in}{2.039481in}}%
\pgfpathcurveto{\pgfqpoint{1.670228in}{2.045305in}}{\pgfqpoint{1.673500in}{2.053205in}}{\pgfqpoint{1.673500in}{2.061441in}}%
\pgfpathcurveto{\pgfqpoint{1.673500in}{2.069677in}}{\pgfqpoint{1.670228in}{2.077577in}}{\pgfqpoint{1.664404in}{2.083401in}}%
\pgfpathcurveto{\pgfqpoint{1.658580in}{2.089225in}}{\pgfqpoint{1.650680in}{2.092498in}}{\pgfqpoint{1.642444in}{2.092498in}}%
\pgfpathcurveto{\pgfqpoint{1.634207in}{2.092498in}}{\pgfqpoint{1.626307in}{2.089225in}}{\pgfqpoint{1.620483in}{2.083401in}}%
\pgfpathcurveto{\pgfqpoint{1.614659in}{2.077577in}}{\pgfqpoint{1.611387in}{2.069677in}}{\pgfqpoint{1.611387in}{2.061441in}}%
\pgfpathcurveto{\pgfqpoint{1.611387in}{2.053205in}}{\pgfqpoint{1.614659in}{2.045305in}}{\pgfqpoint{1.620483in}{2.039481in}}%
\pgfpathcurveto{\pgfqpoint{1.626307in}{2.033657in}}{\pgfqpoint{1.634207in}{2.030385in}}{\pgfqpoint{1.642444in}{2.030385in}}%
\pgfpathclose%
\pgfusepath{stroke,fill}%
\end{pgfscope}%
\begin{pgfscope}%
\pgfpathrectangle{\pgfqpoint{0.100000in}{0.212622in}}{\pgfqpoint{3.696000in}{3.696000in}}%
\pgfusepath{clip}%
\pgfsetbuttcap%
\pgfsetroundjoin%
\definecolor{currentfill}{rgb}{0.121569,0.466667,0.705882}%
\pgfsetfillcolor{currentfill}%
\pgfsetfillopacity{0.906445}%
\pgfsetlinewidth{1.003750pt}%
\definecolor{currentstroke}{rgb}{0.121569,0.466667,0.705882}%
\pgfsetstrokecolor{currentstroke}%
\pgfsetstrokeopacity{0.906445}%
\pgfsetdash{}{0pt}%
\pgfpathmoveto{\pgfqpoint{1.567110in}{1.971376in}}%
\pgfpathcurveto{\pgfqpoint{1.575347in}{1.971376in}}{\pgfqpoint{1.583247in}{1.974649in}}{\pgfqpoint{1.589071in}{1.980473in}}%
\pgfpathcurveto{\pgfqpoint{1.594894in}{1.986297in}}{\pgfqpoint{1.598167in}{1.994197in}}{\pgfqpoint{1.598167in}{2.002433in}}%
\pgfpathcurveto{\pgfqpoint{1.598167in}{2.010669in}}{\pgfqpoint{1.594894in}{2.018569in}}{\pgfqpoint{1.589071in}{2.024393in}}%
\pgfpathcurveto{\pgfqpoint{1.583247in}{2.030217in}}{\pgfqpoint{1.575347in}{2.033489in}}{\pgfqpoint{1.567110in}{2.033489in}}%
\pgfpathcurveto{\pgfqpoint{1.558874in}{2.033489in}}{\pgfqpoint{1.550974in}{2.030217in}}{\pgfqpoint{1.545150in}{2.024393in}}%
\pgfpathcurveto{\pgfqpoint{1.539326in}{2.018569in}}{\pgfqpoint{1.536054in}{2.010669in}}{\pgfqpoint{1.536054in}{2.002433in}}%
\pgfpathcurveto{\pgfqpoint{1.536054in}{1.994197in}}{\pgfqpoint{1.539326in}{1.986297in}}{\pgfqpoint{1.545150in}{1.980473in}}%
\pgfpathcurveto{\pgfqpoint{1.550974in}{1.974649in}}{\pgfqpoint{1.558874in}{1.971376in}}{\pgfqpoint{1.567110in}{1.971376in}}%
\pgfpathclose%
\pgfusepath{stroke,fill}%
\end{pgfscope}%
\begin{pgfscope}%
\pgfpathrectangle{\pgfqpoint{0.100000in}{0.212622in}}{\pgfqpoint{3.696000in}{3.696000in}}%
\pgfusepath{clip}%
\pgfsetbuttcap%
\pgfsetroundjoin%
\definecolor{currentfill}{rgb}{0.121569,0.466667,0.705882}%
\pgfsetfillcolor{currentfill}%
\pgfsetfillopacity{0.906560}%
\pgfsetlinewidth{1.003750pt}%
\definecolor{currentstroke}{rgb}{0.121569,0.466667,0.705882}%
\pgfsetstrokecolor{currentstroke}%
\pgfsetstrokeopacity{0.906560}%
\pgfsetdash{}{0pt}%
\pgfpathmoveto{\pgfqpoint{1.617987in}{2.006263in}}%
\pgfpathcurveto{\pgfqpoint{1.626224in}{2.006263in}}{\pgfqpoint{1.634124in}{2.009535in}}{\pgfqpoint{1.639948in}{2.015359in}}%
\pgfpathcurveto{\pgfqpoint{1.645772in}{2.021183in}}{\pgfqpoint{1.649044in}{2.029083in}}{\pgfqpoint{1.649044in}{2.037320in}}%
\pgfpathcurveto{\pgfqpoint{1.649044in}{2.045556in}}{\pgfqpoint{1.645772in}{2.053456in}}{\pgfqpoint{1.639948in}{2.059280in}}%
\pgfpathcurveto{\pgfqpoint{1.634124in}{2.065104in}}{\pgfqpoint{1.626224in}{2.068376in}}{\pgfqpoint{1.617987in}{2.068376in}}%
\pgfpathcurveto{\pgfqpoint{1.609751in}{2.068376in}}{\pgfqpoint{1.601851in}{2.065104in}}{\pgfqpoint{1.596027in}{2.059280in}}%
\pgfpathcurveto{\pgfqpoint{1.590203in}{2.053456in}}{\pgfqpoint{1.586931in}{2.045556in}}{\pgfqpoint{1.586931in}{2.037320in}}%
\pgfpathcurveto{\pgfqpoint{1.586931in}{2.029083in}}{\pgfqpoint{1.590203in}{2.021183in}}{\pgfqpoint{1.596027in}{2.015359in}}%
\pgfpathcurveto{\pgfqpoint{1.601851in}{2.009535in}}{\pgfqpoint{1.609751in}{2.006263in}}{\pgfqpoint{1.617987in}{2.006263in}}%
\pgfpathclose%
\pgfusepath{stroke,fill}%
\end{pgfscope}%
\begin{pgfscope}%
\pgfpathrectangle{\pgfqpoint{0.100000in}{0.212622in}}{\pgfqpoint{3.696000in}{3.696000in}}%
\pgfusepath{clip}%
\pgfsetbuttcap%
\pgfsetroundjoin%
\definecolor{currentfill}{rgb}{0.121569,0.466667,0.705882}%
\pgfsetfillcolor{currentfill}%
\pgfsetfillopacity{0.906880}%
\pgfsetlinewidth{1.003750pt}%
\definecolor{currentstroke}{rgb}{0.121569,0.466667,0.705882}%
\pgfsetstrokecolor{currentstroke}%
\pgfsetstrokeopacity{0.906880}%
\pgfsetdash{}{0pt}%
\pgfpathmoveto{\pgfqpoint{1.608800in}{1.998328in}}%
\pgfpathcurveto{\pgfqpoint{1.617036in}{1.998328in}}{\pgfqpoint{1.624936in}{2.001600in}}{\pgfqpoint{1.630760in}{2.007424in}}%
\pgfpathcurveto{\pgfqpoint{1.636584in}{2.013248in}}{\pgfqpoint{1.639857in}{2.021148in}}{\pgfqpoint{1.639857in}{2.029384in}}%
\pgfpathcurveto{\pgfqpoint{1.639857in}{2.037620in}}{\pgfqpoint{1.636584in}{2.045520in}}{\pgfqpoint{1.630760in}{2.051344in}}%
\pgfpathcurveto{\pgfqpoint{1.624936in}{2.057168in}}{\pgfqpoint{1.617036in}{2.060441in}}{\pgfqpoint{1.608800in}{2.060441in}}%
\pgfpathcurveto{\pgfqpoint{1.600564in}{2.060441in}}{\pgfqpoint{1.592664in}{2.057168in}}{\pgfqpoint{1.586840in}{2.051344in}}%
\pgfpathcurveto{\pgfqpoint{1.581016in}{2.045520in}}{\pgfqpoint{1.577744in}{2.037620in}}{\pgfqpoint{1.577744in}{2.029384in}}%
\pgfpathcurveto{\pgfqpoint{1.577744in}{2.021148in}}{\pgfqpoint{1.581016in}{2.013248in}}{\pgfqpoint{1.586840in}{2.007424in}}%
\pgfpathcurveto{\pgfqpoint{1.592664in}{2.001600in}}{\pgfqpoint{1.600564in}{1.998328in}}{\pgfqpoint{1.608800in}{1.998328in}}%
\pgfpathclose%
\pgfusepath{stroke,fill}%
\end{pgfscope}%
\begin{pgfscope}%
\pgfpathrectangle{\pgfqpoint{0.100000in}{0.212622in}}{\pgfqpoint{3.696000in}{3.696000in}}%
\pgfusepath{clip}%
\pgfsetbuttcap%
\pgfsetroundjoin%
\definecolor{currentfill}{rgb}{0.121569,0.466667,0.705882}%
\pgfsetfillcolor{currentfill}%
\pgfsetfillopacity{0.907241}%
\pgfsetlinewidth{1.003750pt}%
\definecolor{currentstroke}{rgb}{0.121569,0.466667,0.705882}%
\pgfsetstrokecolor{currentstroke}%
\pgfsetstrokeopacity{0.907241}%
\pgfsetdash{}{0pt}%
\pgfpathmoveto{\pgfqpoint{2.299268in}{2.278074in}}%
\pgfpathcurveto{\pgfqpoint{2.307504in}{2.278074in}}{\pgfqpoint{2.315404in}{2.281347in}}{\pgfqpoint{2.321228in}{2.287170in}}%
\pgfpathcurveto{\pgfqpoint{2.327052in}{2.292994in}}{\pgfqpoint{2.330324in}{2.300894in}}{\pgfqpoint{2.330324in}{2.309131in}}%
\pgfpathcurveto{\pgfqpoint{2.330324in}{2.317367in}}{\pgfqpoint{2.327052in}{2.325267in}}{\pgfqpoint{2.321228in}{2.331091in}}%
\pgfpathcurveto{\pgfqpoint{2.315404in}{2.336915in}}{\pgfqpoint{2.307504in}{2.340187in}}{\pgfqpoint{2.299268in}{2.340187in}}%
\pgfpathcurveto{\pgfqpoint{2.291031in}{2.340187in}}{\pgfqpoint{2.283131in}{2.336915in}}{\pgfqpoint{2.277307in}{2.331091in}}%
\pgfpathcurveto{\pgfqpoint{2.271484in}{2.325267in}}{\pgfqpoint{2.268211in}{2.317367in}}{\pgfqpoint{2.268211in}{2.309131in}}%
\pgfpathcurveto{\pgfqpoint{2.268211in}{2.300894in}}{\pgfqpoint{2.271484in}{2.292994in}}{\pgfqpoint{2.277307in}{2.287170in}}%
\pgfpathcurveto{\pgfqpoint{2.283131in}{2.281347in}}{\pgfqpoint{2.291031in}{2.278074in}}{\pgfqpoint{2.299268in}{2.278074in}}%
\pgfpathclose%
\pgfusepath{stroke,fill}%
\end{pgfscope}%
\begin{pgfscope}%
\pgfpathrectangle{\pgfqpoint{0.100000in}{0.212622in}}{\pgfqpoint{3.696000in}{3.696000in}}%
\pgfusepath{clip}%
\pgfsetbuttcap%
\pgfsetroundjoin%
\definecolor{currentfill}{rgb}{0.121569,0.466667,0.705882}%
\pgfsetfillcolor{currentfill}%
\pgfsetfillopacity{0.907471}%
\pgfsetlinewidth{1.003750pt}%
\definecolor{currentstroke}{rgb}{0.121569,0.466667,0.705882}%
\pgfsetstrokecolor{currentstroke}%
\pgfsetstrokeopacity{0.907471}%
\pgfsetdash{}{0pt}%
\pgfpathmoveto{\pgfqpoint{1.532559in}{1.967583in}}%
\pgfpathcurveto{\pgfqpoint{1.540795in}{1.967583in}}{\pgfqpoint{1.548695in}{1.970855in}}{\pgfqpoint{1.554519in}{1.976679in}}%
\pgfpathcurveto{\pgfqpoint{1.560343in}{1.982503in}}{\pgfqpoint{1.563616in}{1.990403in}}{\pgfqpoint{1.563616in}{1.998639in}}%
\pgfpathcurveto{\pgfqpoint{1.563616in}{2.006875in}}{\pgfqpoint{1.560343in}{2.014775in}}{\pgfqpoint{1.554519in}{2.020599in}}%
\pgfpathcurveto{\pgfqpoint{1.548695in}{2.026423in}}{\pgfqpoint{1.540795in}{2.029696in}}{\pgfqpoint{1.532559in}{2.029696in}}%
\pgfpathcurveto{\pgfqpoint{1.524323in}{2.029696in}}{\pgfqpoint{1.516423in}{2.026423in}}{\pgfqpoint{1.510599in}{2.020599in}}%
\pgfpathcurveto{\pgfqpoint{1.504775in}{2.014775in}}{\pgfqpoint{1.501503in}{2.006875in}}{\pgfqpoint{1.501503in}{1.998639in}}%
\pgfpathcurveto{\pgfqpoint{1.501503in}{1.990403in}}{\pgfqpoint{1.504775in}{1.982503in}}{\pgfqpoint{1.510599in}{1.976679in}}%
\pgfpathcurveto{\pgfqpoint{1.516423in}{1.970855in}}{\pgfqpoint{1.524323in}{1.967583in}}{\pgfqpoint{1.532559in}{1.967583in}}%
\pgfpathclose%
\pgfusepath{stroke,fill}%
\end{pgfscope}%
\begin{pgfscope}%
\pgfpathrectangle{\pgfqpoint{0.100000in}{0.212622in}}{\pgfqpoint{3.696000in}{3.696000in}}%
\pgfusepath{clip}%
\pgfsetbuttcap%
\pgfsetroundjoin%
\definecolor{currentfill}{rgb}{0.121569,0.466667,0.705882}%
\pgfsetfillcolor{currentfill}%
\pgfsetfillopacity{0.907575}%
\pgfsetlinewidth{1.003750pt}%
\definecolor{currentstroke}{rgb}{0.121569,0.466667,0.705882}%
\pgfsetstrokecolor{currentstroke}%
\pgfsetstrokeopacity{0.907575}%
\pgfsetdash{}{0pt}%
\pgfpathmoveto{\pgfqpoint{2.565372in}{2.392469in}}%
\pgfpathcurveto{\pgfqpoint{2.573608in}{2.392469in}}{\pgfqpoint{2.581508in}{2.395741in}}{\pgfqpoint{2.587332in}{2.401565in}}%
\pgfpathcurveto{\pgfqpoint{2.593156in}{2.407389in}}{\pgfqpoint{2.596428in}{2.415289in}}{\pgfqpoint{2.596428in}{2.423526in}}%
\pgfpathcurveto{\pgfqpoint{2.596428in}{2.431762in}}{\pgfqpoint{2.593156in}{2.439662in}}{\pgfqpoint{2.587332in}{2.445486in}}%
\pgfpathcurveto{\pgfqpoint{2.581508in}{2.451310in}}{\pgfqpoint{2.573608in}{2.454582in}}{\pgfqpoint{2.565372in}{2.454582in}}%
\pgfpathcurveto{\pgfqpoint{2.557135in}{2.454582in}}{\pgfqpoint{2.549235in}{2.451310in}}{\pgfqpoint{2.543411in}{2.445486in}}%
\pgfpathcurveto{\pgfqpoint{2.537587in}{2.439662in}}{\pgfqpoint{2.534315in}{2.431762in}}{\pgfqpoint{2.534315in}{2.423526in}}%
\pgfpathcurveto{\pgfqpoint{2.534315in}{2.415289in}}{\pgfqpoint{2.537587in}{2.407389in}}{\pgfqpoint{2.543411in}{2.401565in}}%
\pgfpathcurveto{\pgfqpoint{2.549235in}{2.395741in}}{\pgfqpoint{2.557135in}{2.392469in}}{\pgfqpoint{2.565372in}{2.392469in}}%
\pgfpathclose%
\pgfusepath{stroke,fill}%
\end{pgfscope}%
\begin{pgfscope}%
\pgfpathrectangle{\pgfqpoint{0.100000in}{0.212622in}}{\pgfqpoint{3.696000in}{3.696000in}}%
\pgfusepath{clip}%
\pgfsetbuttcap%
\pgfsetroundjoin%
\definecolor{currentfill}{rgb}{0.121569,0.466667,0.705882}%
\pgfsetfillcolor{currentfill}%
\pgfsetfillopacity{0.909473}%
\pgfsetlinewidth{1.003750pt}%
\definecolor{currentstroke}{rgb}{0.121569,0.466667,0.705882}%
\pgfsetstrokecolor{currentstroke}%
\pgfsetstrokeopacity{0.909473}%
\pgfsetdash{}{0pt}%
\pgfpathmoveto{\pgfqpoint{1.584887in}{1.977893in}}%
\pgfpathcurveto{\pgfqpoint{1.593123in}{1.977893in}}{\pgfqpoint{1.601023in}{1.981165in}}{\pgfqpoint{1.606847in}{1.986989in}}%
\pgfpathcurveto{\pgfqpoint{1.612671in}{1.992813in}}{\pgfqpoint{1.615943in}{2.000713in}}{\pgfqpoint{1.615943in}{2.008949in}}%
\pgfpathcurveto{\pgfqpoint{1.615943in}{2.017185in}}{\pgfqpoint{1.612671in}{2.025085in}}{\pgfqpoint{1.606847in}{2.030909in}}%
\pgfpathcurveto{\pgfqpoint{1.601023in}{2.036733in}}{\pgfqpoint{1.593123in}{2.040006in}}{\pgfqpoint{1.584887in}{2.040006in}}%
\pgfpathcurveto{\pgfqpoint{1.576651in}{2.040006in}}{\pgfqpoint{1.568751in}{2.036733in}}{\pgfqpoint{1.562927in}{2.030909in}}%
\pgfpathcurveto{\pgfqpoint{1.557103in}{2.025085in}}{\pgfqpoint{1.553830in}{2.017185in}}{\pgfqpoint{1.553830in}{2.008949in}}%
\pgfpathcurveto{\pgfqpoint{1.553830in}{2.000713in}}{\pgfqpoint{1.557103in}{1.992813in}}{\pgfqpoint{1.562927in}{1.986989in}}%
\pgfpathcurveto{\pgfqpoint{1.568751in}{1.981165in}}{\pgfqpoint{1.576651in}{1.977893in}}{\pgfqpoint{1.584887in}{1.977893in}}%
\pgfpathclose%
\pgfusepath{stroke,fill}%
\end{pgfscope}%
\begin{pgfscope}%
\pgfpathrectangle{\pgfqpoint{0.100000in}{0.212622in}}{\pgfqpoint{3.696000in}{3.696000in}}%
\pgfusepath{clip}%
\pgfsetbuttcap%
\pgfsetroundjoin%
\definecolor{currentfill}{rgb}{0.121569,0.466667,0.705882}%
\pgfsetfillcolor{currentfill}%
\pgfsetfillopacity{0.910557}%
\pgfsetlinewidth{1.003750pt}%
\definecolor{currentstroke}{rgb}{0.121569,0.466667,0.705882}%
\pgfsetstrokecolor{currentstroke}%
\pgfsetstrokeopacity{0.910557}%
\pgfsetdash{}{0pt}%
\pgfpathmoveto{\pgfqpoint{1.778829in}{2.072004in}}%
\pgfpathcurveto{\pgfqpoint{1.787065in}{2.072004in}}{\pgfqpoint{1.794966in}{2.075276in}}{\pgfqpoint{1.800789in}{2.081100in}}%
\pgfpathcurveto{\pgfqpoint{1.806613in}{2.086924in}}{\pgfqpoint{1.809886in}{2.094824in}}{\pgfqpoint{1.809886in}{2.103060in}}%
\pgfpathcurveto{\pgfqpoint{1.809886in}{2.111296in}}{\pgfqpoint{1.806613in}{2.119196in}}{\pgfqpoint{1.800789in}{2.125020in}}%
\pgfpathcurveto{\pgfqpoint{1.794966in}{2.130844in}}{\pgfqpoint{1.787065in}{2.134117in}}{\pgfqpoint{1.778829in}{2.134117in}}%
\pgfpathcurveto{\pgfqpoint{1.770593in}{2.134117in}}{\pgfqpoint{1.762693in}{2.130844in}}{\pgfqpoint{1.756869in}{2.125020in}}%
\pgfpathcurveto{\pgfqpoint{1.751045in}{2.119196in}}{\pgfqpoint{1.747773in}{2.111296in}}{\pgfqpoint{1.747773in}{2.103060in}}%
\pgfpathcurveto{\pgfqpoint{1.747773in}{2.094824in}}{\pgfqpoint{1.751045in}{2.086924in}}{\pgfqpoint{1.756869in}{2.081100in}}%
\pgfpathcurveto{\pgfqpoint{1.762693in}{2.075276in}}{\pgfqpoint{1.770593in}{2.072004in}}{\pgfqpoint{1.778829in}{2.072004in}}%
\pgfpathclose%
\pgfusepath{stroke,fill}%
\end{pgfscope}%
\begin{pgfscope}%
\pgfpathrectangle{\pgfqpoint{0.100000in}{0.212622in}}{\pgfqpoint{3.696000in}{3.696000in}}%
\pgfusepath{clip}%
\pgfsetbuttcap%
\pgfsetroundjoin%
\definecolor{currentfill}{rgb}{0.121569,0.466667,0.705882}%
\pgfsetfillcolor{currentfill}%
\pgfsetfillopacity{0.910649}%
\pgfsetlinewidth{1.003750pt}%
\definecolor{currentstroke}{rgb}{0.121569,0.466667,0.705882}%
\pgfsetstrokecolor{currentstroke}%
\pgfsetstrokeopacity{0.910649}%
\pgfsetdash{}{0pt}%
\pgfpathmoveto{\pgfqpoint{1.724908in}{2.069887in}}%
\pgfpathcurveto{\pgfqpoint{1.733144in}{2.069887in}}{\pgfqpoint{1.741044in}{2.073160in}}{\pgfqpoint{1.746868in}{2.078983in}}%
\pgfpathcurveto{\pgfqpoint{1.752692in}{2.084807in}}{\pgfqpoint{1.755964in}{2.092707in}}{\pgfqpoint{1.755964in}{2.100944in}}%
\pgfpathcurveto{\pgfqpoint{1.755964in}{2.109180in}}{\pgfqpoint{1.752692in}{2.117080in}}{\pgfqpoint{1.746868in}{2.122904in}}%
\pgfpathcurveto{\pgfqpoint{1.741044in}{2.128728in}}{\pgfqpoint{1.733144in}{2.132000in}}{\pgfqpoint{1.724908in}{2.132000in}}%
\pgfpathcurveto{\pgfqpoint{1.716672in}{2.132000in}}{\pgfqpoint{1.708772in}{2.128728in}}{\pgfqpoint{1.702948in}{2.122904in}}%
\pgfpathcurveto{\pgfqpoint{1.697124in}{2.117080in}}{\pgfqpoint{1.693851in}{2.109180in}}{\pgfqpoint{1.693851in}{2.100944in}}%
\pgfpathcurveto{\pgfqpoint{1.693851in}{2.092707in}}{\pgfqpoint{1.697124in}{2.084807in}}{\pgfqpoint{1.702948in}{2.078983in}}%
\pgfpathcurveto{\pgfqpoint{1.708772in}{2.073160in}}{\pgfqpoint{1.716672in}{2.069887in}}{\pgfqpoint{1.724908in}{2.069887in}}%
\pgfpathclose%
\pgfusepath{stroke,fill}%
\end{pgfscope}%
\begin{pgfscope}%
\pgfpathrectangle{\pgfqpoint{0.100000in}{0.212622in}}{\pgfqpoint{3.696000in}{3.696000in}}%
\pgfusepath{clip}%
\pgfsetbuttcap%
\pgfsetroundjoin%
\definecolor{currentfill}{rgb}{0.121569,0.466667,0.705882}%
\pgfsetfillcolor{currentfill}%
\pgfsetfillopacity{0.911356}%
\pgfsetlinewidth{1.003750pt}%
\definecolor{currentstroke}{rgb}{0.121569,0.466667,0.705882}%
\pgfsetstrokecolor{currentstroke}%
\pgfsetstrokeopacity{0.911356}%
\pgfsetdash{}{0pt}%
\pgfpathmoveto{\pgfqpoint{1.495046in}{1.953769in}}%
\pgfpathcurveto{\pgfqpoint{1.503283in}{1.953769in}}{\pgfqpoint{1.511183in}{1.957042in}}{\pgfqpoint{1.517007in}{1.962866in}}%
\pgfpathcurveto{\pgfqpoint{1.522831in}{1.968690in}}{\pgfqpoint{1.526103in}{1.976590in}}{\pgfqpoint{1.526103in}{1.984826in}}%
\pgfpathcurveto{\pgfqpoint{1.526103in}{1.993062in}}{\pgfqpoint{1.522831in}{2.000962in}}{\pgfqpoint{1.517007in}{2.006786in}}%
\pgfpathcurveto{\pgfqpoint{1.511183in}{2.012610in}}{\pgfqpoint{1.503283in}{2.015882in}}{\pgfqpoint{1.495046in}{2.015882in}}%
\pgfpathcurveto{\pgfqpoint{1.486810in}{2.015882in}}{\pgfqpoint{1.478910in}{2.012610in}}{\pgfqpoint{1.473086in}{2.006786in}}%
\pgfpathcurveto{\pgfqpoint{1.467262in}{2.000962in}}{\pgfqpoint{1.463990in}{1.993062in}}{\pgfqpoint{1.463990in}{1.984826in}}%
\pgfpathcurveto{\pgfqpoint{1.463990in}{1.976590in}}{\pgfqpoint{1.467262in}{1.968690in}}{\pgfqpoint{1.473086in}{1.962866in}}%
\pgfpathcurveto{\pgfqpoint{1.478910in}{1.957042in}}{\pgfqpoint{1.486810in}{1.953769in}}{\pgfqpoint{1.495046in}{1.953769in}}%
\pgfpathclose%
\pgfusepath{stroke,fill}%
\end{pgfscope}%
\begin{pgfscope}%
\pgfpathrectangle{\pgfqpoint{0.100000in}{0.212622in}}{\pgfqpoint{3.696000in}{3.696000in}}%
\pgfusepath{clip}%
\pgfsetbuttcap%
\pgfsetroundjoin%
\definecolor{currentfill}{rgb}{0.121569,0.466667,0.705882}%
\pgfsetfillcolor{currentfill}%
\pgfsetfillopacity{0.911516}%
\pgfsetlinewidth{1.003750pt}%
\definecolor{currentstroke}{rgb}{0.121569,0.466667,0.705882}%
\pgfsetstrokecolor{currentstroke}%
\pgfsetstrokeopacity{0.911516}%
\pgfsetdash{}{0pt}%
\pgfpathmoveto{\pgfqpoint{1.697065in}{2.059842in}}%
\pgfpathcurveto{\pgfqpoint{1.705301in}{2.059842in}}{\pgfqpoint{1.713201in}{2.063114in}}{\pgfqpoint{1.719025in}{2.068938in}}%
\pgfpathcurveto{\pgfqpoint{1.724849in}{2.074762in}}{\pgfqpoint{1.728122in}{2.082662in}}{\pgfqpoint{1.728122in}{2.090898in}}%
\pgfpathcurveto{\pgfqpoint{1.728122in}{2.099134in}}{\pgfqpoint{1.724849in}{2.107034in}}{\pgfqpoint{1.719025in}{2.112858in}}%
\pgfpathcurveto{\pgfqpoint{1.713201in}{2.118682in}}{\pgfqpoint{1.705301in}{2.121955in}}{\pgfqpoint{1.697065in}{2.121955in}}%
\pgfpathcurveto{\pgfqpoint{1.688829in}{2.121955in}}{\pgfqpoint{1.680929in}{2.118682in}}{\pgfqpoint{1.675105in}{2.112858in}}%
\pgfpathcurveto{\pgfqpoint{1.669281in}{2.107034in}}{\pgfqpoint{1.666009in}{2.099134in}}{\pgfqpoint{1.666009in}{2.090898in}}%
\pgfpathcurveto{\pgfqpoint{1.666009in}{2.082662in}}{\pgfqpoint{1.669281in}{2.074762in}}{\pgfqpoint{1.675105in}{2.068938in}}%
\pgfpathcurveto{\pgfqpoint{1.680929in}{2.063114in}}{\pgfqpoint{1.688829in}{2.059842in}}{\pgfqpoint{1.697065in}{2.059842in}}%
\pgfpathclose%
\pgfusepath{stroke,fill}%
\end{pgfscope}%
\begin{pgfscope}%
\pgfpathrectangle{\pgfqpoint{0.100000in}{0.212622in}}{\pgfqpoint{3.696000in}{3.696000in}}%
\pgfusepath{clip}%
\pgfsetbuttcap%
\pgfsetroundjoin%
\definecolor{currentfill}{rgb}{0.121569,0.466667,0.705882}%
\pgfsetfillcolor{currentfill}%
\pgfsetfillopacity{0.911607}%
\pgfsetlinewidth{1.003750pt}%
\definecolor{currentstroke}{rgb}{0.121569,0.466667,0.705882}%
\pgfsetstrokecolor{currentstroke}%
\pgfsetstrokeopacity{0.911607}%
\pgfsetdash{}{0pt}%
\pgfpathmoveto{\pgfqpoint{1.688092in}{2.055984in}}%
\pgfpathcurveto{\pgfqpoint{1.696328in}{2.055984in}}{\pgfqpoint{1.704228in}{2.059256in}}{\pgfqpoint{1.710052in}{2.065080in}}%
\pgfpathcurveto{\pgfqpoint{1.715876in}{2.070904in}}{\pgfqpoint{1.719148in}{2.078804in}}{\pgfqpoint{1.719148in}{2.087040in}}%
\pgfpathcurveto{\pgfqpoint{1.719148in}{2.095277in}}{\pgfqpoint{1.715876in}{2.103177in}}{\pgfqpoint{1.710052in}{2.109001in}}%
\pgfpathcurveto{\pgfqpoint{1.704228in}{2.114825in}}{\pgfqpoint{1.696328in}{2.118097in}}{\pgfqpoint{1.688092in}{2.118097in}}%
\pgfpathcurveto{\pgfqpoint{1.679855in}{2.118097in}}{\pgfqpoint{1.671955in}{2.114825in}}{\pgfqpoint{1.666131in}{2.109001in}}%
\pgfpathcurveto{\pgfqpoint{1.660307in}{2.103177in}}{\pgfqpoint{1.657035in}{2.095277in}}{\pgfqpoint{1.657035in}{2.087040in}}%
\pgfpathcurveto{\pgfqpoint{1.657035in}{2.078804in}}{\pgfqpoint{1.660307in}{2.070904in}}{\pgfqpoint{1.666131in}{2.065080in}}%
\pgfpathcurveto{\pgfqpoint{1.671955in}{2.059256in}}{\pgfqpoint{1.679855in}{2.055984in}}{\pgfqpoint{1.688092in}{2.055984in}}%
\pgfpathclose%
\pgfusepath{stroke,fill}%
\end{pgfscope}%
\begin{pgfscope}%
\pgfpathrectangle{\pgfqpoint{0.100000in}{0.212622in}}{\pgfqpoint{3.696000in}{3.696000in}}%
\pgfusepath{clip}%
\pgfsetbuttcap%
\pgfsetroundjoin%
\definecolor{currentfill}{rgb}{0.121569,0.466667,0.705882}%
\pgfsetfillcolor{currentfill}%
\pgfsetfillopacity{0.912124}%
\pgfsetlinewidth{1.003750pt}%
\definecolor{currentstroke}{rgb}{0.121569,0.466667,0.705882}%
\pgfsetstrokecolor{currentstroke}%
\pgfsetstrokeopacity{0.912124}%
\pgfsetdash{}{0pt}%
\pgfpathmoveto{\pgfqpoint{1.809084in}{2.105312in}}%
\pgfpathcurveto{\pgfqpoint{1.817320in}{2.105312in}}{\pgfqpoint{1.825220in}{2.108584in}}{\pgfqpoint{1.831044in}{2.114408in}}%
\pgfpathcurveto{\pgfqpoint{1.836868in}{2.120232in}}{\pgfqpoint{1.840141in}{2.128132in}}{\pgfqpoint{1.840141in}{2.136369in}}%
\pgfpathcurveto{\pgfqpoint{1.840141in}{2.144605in}}{\pgfqpoint{1.836868in}{2.152505in}}{\pgfqpoint{1.831044in}{2.158329in}}%
\pgfpathcurveto{\pgfqpoint{1.825220in}{2.164153in}}{\pgfqpoint{1.817320in}{2.167425in}}{\pgfqpoint{1.809084in}{2.167425in}}%
\pgfpathcurveto{\pgfqpoint{1.800848in}{2.167425in}}{\pgfqpoint{1.792948in}{2.164153in}}{\pgfqpoint{1.787124in}{2.158329in}}%
\pgfpathcurveto{\pgfqpoint{1.781300in}{2.152505in}}{\pgfqpoint{1.778028in}{2.144605in}}{\pgfqpoint{1.778028in}{2.136369in}}%
\pgfpathcurveto{\pgfqpoint{1.778028in}{2.128132in}}{\pgfqpoint{1.781300in}{2.120232in}}{\pgfqpoint{1.787124in}{2.114408in}}%
\pgfpathcurveto{\pgfqpoint{1.792948in}{2.108584in}}{\pgfqpoint{1.800848in}{2.105312in}}{\pgfqpoint{1.809084in}{2.105312in}}%
\pgfpathclose%
\pgfusepath{stroke,fill}%
\end{pgfscope}%
\begin{pgfscope}%
\pgfpathrectangle{\pgfqpoint{0.100000in}{0.212622in}}{\pgfqpoint{3.696000in}{3.696000in}}%
\pgfusepath{clip}%
\pgfsetbuttcap%
\pgfsetroundjoin%
\definecolor{currentfill}{rgb}{0.121569,0.466667,0.705882}%
\pgfsetfillcolor{currentfill}%
\pgfsetfillopacity{0.912825}%
\pgfsetlinewidth{1.003750pt}%
\definecolor{currentstroke}{rgb}{0.121569,0.466667,0.705882}%
\pgfsetstrokecolor{currentstroke}%
\pgfsetstrokeopacity{0.912825}%
\pgfsetdash{}{0pt}%
\pgfpathmoveto{\pgfqpoint{1.511361in}{1.957083in}}%
\pgfpathcurveto{\pgfqpoint{1.519598in}{1.957083in}}{\pgfqpoint{1.527498in}{1.960355in}}{\pgfqpoint{1.533322in}{1.966179in}}%
\pgfpathcurveto{\pgfqpoint{1.539146in}{1.972003in}}{\pgfqpoint{1.542418in}{1.979903in}}{\pgfqpoint{1.542418in}{1.988139in}}%
\pgfpathcurveto{\pgfqpoint{1.542418in}{1.996375in}}{\pgfqpoint{1.539146in}{2.004275in}}{\pgfqpoint{1.533322in}{2.010099in}}%
\pgfpathcurveto{\pgfqpoint{1.527498in}{2.015923in}}{\pgfqpoint{1.519598in}{2.019196in}}{\pgfqpoint{1.511361in}{2.019196in}}%
\pgfpathcurveto{\pgfqpoint{1.503125in}{2.019196in}}{\pgfqpoint{1.495225in}{2.015923in}}{\pgfqpoint{1.489401in}{2.010099in}}%
\pgfpathcurveto{\pgfqpoint{1.483577in}{2.004275in}}{\pgfqpoint{1.480305in}{1.996375in}}{\pgfqpoint{1.480305in}{1.988139in}}%
\pgfpathcurveto{\pgfqpoint{1.480305in}{1.979903in}}{\pgfqpoint{1.483577in}{1.972003in}}{\pgfqpoint{1.489401in}{1.966179in}}%
\pgfpathcurveto{\pgfqpoint{1.495225in}{1.960355in}}{\pgfqpoint{1.503125in}{1.957083in}}{\pgfqpoint{1.511361in}{1.957083in}}%
\pgfpathclose%
\pgfusepath{stroke,fill}%
\end{pgfscope}%
\begin{pgfscope}%
\pgfpathrectangle{\pgfqpoint{0.100000in}{0.212622in}}{\pgfqpoint{3.696000in}{3.696000in}}%
\pgfusepath{clip}%
\pgfsetbuttcap%
\pgfsetroundjoin%
\definecolor{currentfill}{rgb}{0.121569,0.466667,0.705882}%
\pgfsetfillcolor{currentfill}%
\pgfsetfillopacity{0.912858}%
\pgfsetlinewidth{1.003750pt}%
\definecolor{currentstroke}{rgb}{0.121569,0.466667,0.705882}%
\pgfsetstrokecolor{currentstroke}%
\pgfsetstrokeopacity{0.912858}%
\pgfsetdash{}{0pt}%
\pgfpathmoveto{\pgfqpoint{2.587912in}{2.426543in}}%
\pgfpathcurveto{\pgfqpoint{2.596148in}{2.426543in}}{\pgfqpoint{2.604048in}{2.429816in}}{\pgfqpoint{2.609872in}{2.435640in}}%
\pgfpathcurveto{\pgfqpoint{2.615696in}{2.441464in}}{\pgfqpoint{2.618968in}{2.449364in}}{\pgfqpoint{2.618968in}{2.457600in}}%
\pgfpathcurveto{\pgfqpoint{2.618968in}{2.465836in}}{\pgfqpoint{2.615696in}{2.473736in}}{\pgfqpoint{2.609872in}{2.479560in}}%
\pgfpathcurveto{\pgfqpoint{2.604048in}{2.485384in}}{\pgfqpoint{2.596148in}{2.488656in}}{\pgfqpoint{2.587912in}{2.488656in}}%
\pgfpathcurveto{\pgfqpoint{2.579675in}{2.488656in}}{\pgfqpoint{2.571775in}{2.485384in}}{\pgfqpoint{2.565951in}{2.479560in}}%
\pgfpathcurveto{\pgfqpoint{2.560127in}{2.473736in}}{\pgfqpoint{2.556855in}{2.465836in}}{\pgfqpoint{2.556855in}{2.457600in}}%
\pgfpathcurveto{\pgfqpoint{2.556855in}{2.449364in}}{\pgfqpoint{2.560127in}{2.441464in}}{\pgfqpoint{2.565951in}{2.435640in}}%
\pgfpathcurveto{\pgfqpoint{2.571775in}{2.429816in}}{\pgfqpoint{2.579675in}{2.426543in}}{\pgfqpoint{2.587912in}{2.426543in}}%
\pgfpathclose%
\pgfusepath{stroke,fill}%
\end{pgfscope}%
\begin{pgfscope}%
\pgfpathrectangle{\pgfqpoint{0.100000in}{0.212622in}}{\pgfqpoint{3.696000in}{3.696000in}}%
\pgfusepath{clip}%
\pgfsetbuttcap%
\pgfsetroundjoin%
\definecolor{currentfill}{rgb}{0.121569,0.466667,0.705882}%
\pgfsetfillcolor{currentfill}%
\pgfsetfillopacity{0.912900}%
\pgfsetlinewidth{1.003750pt}%
\definecolor{currentstroke}{rgb}{0.121569,0.466667,0.705882}%
\pgfsetstrokecolor{currentstroke}%
\pgfsetstrokeopacity{0.912900}%
\pgfsetdash{}{0pt}%
\pgfpathmoveto{\pgfqpoint{1.797597in}{2.094585in}}%
\pgfpathcurveto{\pgfqpoint{1.805833in}{2.094585in}}{\pgfqpoint{1.813733in}{2.097857in}}{\pgfqpoint{1.819557in}{2.103681in}}%
\pgfpathcurveto{\pgfqpoint{1.825381in}{2.109505in}}{\pgfqpoint{1.828654in}{2.117405in}}{\pgfqpoint{1.828654in}{2.125641in}}%
\pgfpathcurveto{\pgfqpoint{1.828654in}{2.133878in}}{\pgfqpoint{1.825381in}{2.141778in}}{\pgfqpoint{1.819557in}{2.147602in}}%
\pgfpathcurveto{\pgfqpoint{1.813733in}{2.153425in}}{\pgfqpoint{1.805833in}{2.156698in}}{\pgfqpoint{1.797597in}{2.156698in}}%
\pgfpathcurveto{\pgfqpoint{1.789361in}{2.156698in}}{\pgfqpoint{1.781461in}{2.153425in}}{\pgfqpoint{1.775637in}{2.147602in}}%
\pgfpathcurveto{\pgfqpoint{1.769813in}{2.141778in}}{\pgfqpoint{1.766541in}{2.133878in}}{\pgfqpoint{1.766541in}{2.125641in}}%
\pgfpathcurveto{\pgfqpoint{1.766541in}{2.117405in}}{\pgfqpoint{1.769813in}{2.109505in}}{\pgfqpoint{1.775637in}{2.103681in}}%
\pgfpathcurveto{\pgfqpoint{1.781461in}{2.097857in}}{\pgfqpoint{1.789361in}{2.094585in}}{\pgfqpoint{1.797597in}{2.094585in}}%
\pgfpathclose%
\pgfusepath{stroke,fill}%
\end{pgfscope}%
\begin{pgfscope}%
\pgfpathrectangle{\pgfqpoint{0.100000in}{0.212622in}}{\pgfqpoint{3.696000in}{3.696000in}}%
\pgfusepath{clip}%
\pgfsetbuttcap%
\pgfsetroundjoin%
\definecolor{currentfill}{rgb}{0.121569,0.466667,0.705882}%
\pgfsetfillcolor{currentfill}%
\pgfsetfillopacity{0.914238}%
\pgfsetlinewidth{1.003750pt}%
\definecolor{currentstroke}{rgb}{0.121569,0.466667,0.705882}%
\pgfsetstrokecolor{currentstroke}%
\pgfsetstrokeopacity{0.914238}%
\pgfsetdash{}{0pt}%
\pgfpathmoveto{\pgfqpoint{2.279257in}{2.266015in}}%
\pgfpathcurveto{\pgfqpoint{2.287493in}{2.266015in}}{\pgfqpoint{2.295393in}{2.269288in}}{\pgfqpoint{2.301217in}{2.275112in}}%
\pgfpathcurveto{\pgfqpoint{2.307041in}{2.280936in}}{\pgfqpoint{2.310313in}{2.288836in}}{\pgfqpoint{2.310313in}{2.297072in}}%
\pgfpathcurveto{\pgfqpoint{2.310313in}{2.305308in}}{\pgfqpoint{2.307041in}{2.313208in}}{\pgfqpoint{2.301217in}{2.319032in}}%
\pgfpathcurveto{\pgfqpoint{2.295393in}{2.324856in}}{\pgfqpoint{2.287493in}{2.328128in}}{\pgfqpoint{2.279257in}{2.328128in}}%
\pgfpathcurveto{\pgfqpoint{2.271021in}{2.328128in}}{\pgfqpoint{2.263120in}{2.324856in}}{\pgfqpoint{2.257297in}{2.319032in}}%
\pgfpathcurveto{\pgfqpoint{2.251473in}{2.313208in}}{\pgfqpoint{2.248200in}{2.305308in}}{\pgfqpoint{2.248200in}{2.297072in}}%
\pgfpathcurveto{\pgfqpoint{2.248200in}{2.288836in}}{\pgfqpoint{2.251473in}{2.280936in}}{\pgfqpoint{2.257297in}{2.275112in}}%
\pgfpathcurveto{\pgfqpoint{2.263120in}{2.269288in}}{\pgfqpoint{2.271021in}{2.266015in}}{\pgfqpoint{2.279257in}{2.266015in}}%
\pgfpathclose%
\pgfusepath{stroke,fill}%
\end{pgfscope}%
\begin{pgfscope}%
\pgfpathrectangle{\pgfqpoint{0.100000in}{0.212622in}}{\pgfqpoint{3.696000in}{3.696000in}}%
\pgfusepath{clip}%
\pgfsetbuttcap%
\pgfsetroundjoin%
\definecolor{currentfill}{rgb}{0.121569,0.466667,0.705882}%
\pgfsetfillcolor{currentfill}%
\pgfsetfillopacity{0.914510}%
\pgfsetlinewidth{1.003750pt}%
\definecolor{currentstroke}{rgb}{0.121569,0.466667,0.705882}%
\pgfsetstrokecolor{currentstroke}%
\pgfsetstrokeopacity{0.914510}%
\pgfsetdash{}{0pt}%
\pgfpathmoveto{\pgfqpoint{2.631639in}{2.440884in}}%
\pgfpathcurveto{\pgfqpoint{2.639876in}{2.440884in}}{\pgfqpoint{2.647776in}{2.444157in}}{\pgfqpoint{2.653600in}{2.449981in}}%
\pgfpathcurveto{\pgfqpoint{2.659424in}{2.455804in}}{\pgfqpoint{2.662696in}{2.463705in}}{\pgfqpoint{2.662696in}{2.471941in}}%
\pgfpathcurveto{\pgfqpoint{2.662696in}{2.480177in}}{\pgfqpoint{2.659424in}{2.488077in}}{\pgfqpoint{2.653600in}{2.493901in}}%
\pgfpathcurveto{\pgfqpoint{2.647776in}{2.499725in}}{\pgfqpoint{2.639876in}{2.502997in}}{\pgfqpoint{2.631639in}{2.502997in}}%
\pgfpathcurveto{\pgfqpoint{2.623403in}{2.502997in}}{\pgfqpoint{2.615503in}{2.499725in}}{\pgfqpoint{2.609679in}{2.493901in}}%
\pgfpathcurveto{\pgfqpoint{2.603855in}{2.488077in}}{\pgfqpoint{2.600583in}{2.480177in}}{\pgfqpoint{2.600583in}{2.471941in}}%
\pgfpathcurveto{\pgfqpoint{2.600583in}{2.463705in}}{\pgfqpoint{2.603855in}{2.455804in}}{\pgfqpoint{2.609679in}{2.449981in}}%
\pgfpathcurveto{\pgfqpoint{2.615503in}{2.444157in}}{\pgfqpoint{2.623403in}{2.440884in}}{\pgfqpoint{2.631639in}{2.440884in}}%
\pgfpathclose%
\pgfusepath{stroke,fill}%
\end{pgfscope}%
\begin{pgfscope}%
\pgfpathrectangle{\pgfqpoint{0.100000in}{0.212622in}}{\pgfqpoint{3.696000in}{3.696000in}}%
\pgfusepath{clip}%
\pgfsetbuttcap%
\pgfsetroundjoin%
\definecolor{currentfill}{rgb}{0.121569,0.466667,0.705882}%
\pgfsetfillcolor{currentfill}%
\pgfsetfillopacity{0.914745}%
\pgfsetlinewidth{1.003750pt}%
\definecolor{currentstroke}{rgb}{0.121569,0.466667,0.705882}%
\pgfsetstrokecolor{currentstroke}%
\pgfsetstrokeopacity{0.914745}%
\pgfsetdash{}{0pt}%
\pgfpathmoveto{\pgfqpoint{1.696796in}{2.063390in}}%
\pgfpathcurveto{\pgfqpoint{1.705033in}{2.063390in}}{\pgfqpoint{1.712933in}{2.066662in}}{\pgfqpoint{1.718756in}{2.072486in}}%
\pgfpathcurveto{\pgfqpoint{1.724580in}{2.078310in}}{\pgfqpoint{1.727853in}{2.086210in}}{\pgfqpoint{1.727853in}{2.094446in}}%
\pgfpathcurveto{\pgfqpoint{1.727853in}{2.102683in}}{\pgfqpoint{1.724580in}{2.110583in}}{\pgfqpoint{1.718756in}{2.116406in}}%
\pgfpathcurveto{\pgfqpoint{1.712933in}{2.122230in}}{\pgfqpoint{1.705033in}{2.125503in}}{\pgfqpoint{1.696796in}{2.125503in}}%
\pgfpathcurveto{\pgfqpoint{1.688560in}{2.125503in}}{\pgfqpoint{1.680660in}{2.122230in}}{\pgfqpoint{1.674836in}{2.116406in}}%
\pgfpathcurveto{\pgfqpoint{1.669012in}{2.110583in}}{\pgfqpoint{1.665740in}{2.102683in}}{\pgfqpoint{1.665740in}{2.094446in}}%
\pgfpathcurveto{\pgfqpoint{1.665740in}{2.086210in}}{\pgfqpoint{1.669012in}{2.078310in}}{\pgfqpoint{1.674836in}{2.072486in}}%
\pgfpathcurveto{\pgfqpoint{1.680660in}{2.066662in}}{\pgfqpoint{1.688560in}{2.063390in}}{\pgfqpoint{1.696796in}{2.063390in}}%
\pgfpathclose%
\pgfusepath{stroke,fill}%
\end{pgfscope}%
\begin{pgfscope}%
\pgfpathrectangle{\pgfqpoint{0.100000in}{0.212622in}}{\pgfqpoint{3.696000in}{3.696000in}}%
\pgfusepath{clip}%
\pgfsetbuttcap%
\pgfsetroundjoin%
\definecolor{currentfill}{rgb}{0.121569,0.466667,0.705882}%
\pgfsetfillcolor{currentfill}%
\pgfsetfillopacity{0.915086}%
\pgfsetlinewidth{1.003750pt}%
\definecolor{currentstroke}{rgb}{0.121569,0.466667,0.705882}%
\pgfsetstrokecolor{currentstroke}%
\pgfsetstrokeopacity{0.915086}%
\pgfsetdash{}{0pt}%
\pgfpathmoveto{\pgfqpoint{1.810440in}{2.099587in}}%
\pgfpathcurveto{\pgfqpoint{1.818677in}{2.099587in}}{\pgfqpoint{1.826577in}{2.102859in}}{\pgfqpoint{1.832401in}{2.108683in}}%
\pgfpathcurveto{\pgfqpoint{1.838224in}{2.114507in}}{\pgfqpoint{1.841497in}{2.122407in}}{\pgfqpoint{1.841497in}{2.130643in}}%
\pgfpathcurveto{\pgfqpoint{1.841497in}{2.138879in}}{\pgfqpoint{1.838224in}{2.146779in}}{\pgfqpoint{1.832401in}{2.152603in}}%
\pgfpathcurveto{\pgfqpoint{1.826577in}{2.158427in}}{\pgfqpoint{1.818677in}{2.161700in}}{\pgfqpoint{1.810440in}{2.161700in}}%
\pgfpathcurveto{\pgfqpoint{1.802204in}{2.161700in}}{\pgfqpoint{1.794304in}{2.158427in}}{\pgfqpoint{1.788480in}{2.152603in}}%
\pgfpathcurveto{\pgfqpoint{1.782656in}{2.146779in}}{\pgfqpoint{1.779384in}{2.138879in}}{\pgfqpoint{1.779384in}{2.130643in}}%
\pgfpathcurveto{\pgfqpoint{1.779384in}{2.122407in}}{\pgfqpoint{1.782656in}{2.114507in}}{\pgfqpoint{1.788480in}{2.108683in}}%
\pgfpathcurveto{\pgfqpoint{1.794304in}{2.102859in}}{\pgfqpoint{1.802204in}{2.099587in}}{\pgfqpoint{1.810440in}{2.099587in}}%
\pgfpathclose%
\pgfusepath{stroke,fill}%
\end{pgfscope}%
\begin{pgfscope}%
\pgfpathrectangle{\pgfqpoint{0.100000in}{0.212622in}}{\pgfqpoint{3.696000in}{3.696000in}}%
\pgfusepath{clip}%
\pgfsetbuttcap%
\pgfsetroundjoin%
\definecolor{currentfill}{rgb}{0.121569,0.466667,0.705882}%
\pgfsetfillcolor{currentfill}%
\pgfsetfillopacity{0.915414}%
\pgfsetlinewidth{1.003750pt}%
\definecolor{currentstroke}{rgb}{0.121569,0.466667,0.705882}%
\pgfsetstrokecolor{currentstroke}%
\pgfsetstrokeopacity{0.915414}%
\pgfsetdash{}{0pt}%
\pgfpathmoveto{\pgfqpoint{1.508447in}{1.950350in}}%
\pgfpathcurveto{\pgfqpoint{1.516683in}{1.950350in}}{\pgfqpoint{1.524583in}{1.953623in}}{\pgfqpoint{1.530407in}{1.959446in}}%
\pgfpathcurveto{\pgfqpoint{1.536231in}{1.965270in}}{\pgfqpoint{1.539504in}{1.973170in}}{\pgfqpoint{1.539504in}{1.981407in}}%
\pgfpathcurveto{\pgfqpoint{1.539504in}{1.989643in}}{\pgfqpoint{1.536231in}{1.997543in}}{\pgfqpoint{1.530407in}{2.003367in}}%
\pgfpathcurveto{\pgfqpoint{1.524583in}{2.009191in}}{\pgfqpoint{1.516683in}{2.012463in}}{\pgfqpoint{1.508447in}{2.012463in}}%
\pgfpathcurveto{\pgfqpoint{1.500211in}{2.012463in}}{\pgfqpoint{1.492311in}{2.009191in}}{\pgfqpoint{1.486487in}{2.003367in}}%
\pgfpathcurveto{\pgfqpoint{1.480663in}{1.997543in}}{\pgfqpoint{1.477391in}{1.989643in}}{\pgfqpoint{1.477391in}{1.981407in}}%
\pgfpathcurveto{\pgfqpoint{1.477391in}{1.973170in}}{\pgfqpoint{1.480663in}{1.965270in}}{\pgfqpoint{1.486487in}{1.959446in}}%
\pgfpathcurveto{\pgfqpoint{1.492311in}{1.953623in}}{\pgfqpoint{1.500211in}{1.950350in}}{\pgfqpoint{1.508447in}{1.950350in}}%
\pgfpathclose%
\pgfusepath{stroke,fill}%
\end{pgfscope}%
\begin{pgfscope}%
\pgfpathrectangle{\pgfqpoint{0.100000in}{0.212622in}}{\pgfqpoint{3.696000in}{3.696000in}}%
\pgfusepath{clip}%
\pgfsetbuttcap%
\pgfsetroundjoin%
\definecolor{currentfill}{rgb}{0.121569,0.466667,0.705882}%
\pgfsetfillcolor{currentfill}%
\pgfsetfillopacity{0.915768}%
\pgfsetlinewidth{1.003750pt}%
\definecolor{currentstroke}{rgb}{0.121569,0.466667,0.705882}%
\pgfsetstrokecolor{currentstroke}%
\pgfsetstrokeopacity{0.915768}%
\pgfsetdash{}{0pt}%
\pgfpathmoveto{\pgfqpoint{1.817256in}{2.094660in}}%
\pgfpathcurveto{\pgfqpoint{1.825493in}{2.094660in}}{\pgfqpoint{1.833393in}{2.097932in}}{\pgfqpoint{1.839217in}{2.103756in}}%
\pgfpathcurveto{\pgfqpoint{1.845040in}{2.109580in}}{\pgfqpoint{1.848313in}{2.117480in}}{\pgfqpoint{1.848313in}{2.125716in}}%
\pgfpathcurveto{\pgfqpoint{1.848313in}{2.133953in}}{\pgfqpoint{1.845040in}{2.141853in}}{\pgfqpoint{1.839217in}{2.147677in}}%
\pgfpathcurveto{\pgfqpoint{1.833393in}{2.153501in}}{\pgfqpoint{1.825493in}{2.156773in}}{\pgfqpoint{1.817256in}{2.156773in}}%
\pgfpathcurveto{\pgfqpoint{1.809020in}{2.156773in}}{\pgfqpoint{1.801120in}{2.153501in}}{\pgfqpoint{1.795296in}{2.147677in}}%
\pgfpathcurveto{\pgfqpoint{1.789472in}{2.141853in}}{\pgfqpoint{1.786200in}{2.133953in}}{\pgfqpoint{1.786200in}{2.125716in}}%
\pgfpathcurveto{\pgfqpoint{1.786200in}{2.117480in}}{\pgfqpoint{1.789472in}{2.109580in}}{\pgfqpoint{1.795296in}{2.103756in}}%
\pgfpathcurveto{\pgfqpoint{1.801120in}{2.097932in}}{\pgfqpoint{1.809020in}{2.094660in}}{\pgfqpoint{1.817256in}{2.094660in}}%
\pgfpathclose%
\pgfusepath{stroke,fill}%
\end{pgfscope}%
\begin{pgfscope}%
\pgfpathrectangle{\pgfqpoint{0.100000in}{0.212622in}}{\pgfqpoint{3.696000in}{3.696000in}}%
\pgfusepath{clip}%
\pgfsetbuttcap%
\pgfsetroundjoin%
\definecolor{currentfill}{rgb}{0.121569,0.466667,0.705882}%
\pgfsetfillcolor{currentfill}%
\pgfsetfillopacity{0.916277}%
\pgfsetlinewidth{1.003750pt}%
\definecolor{currentstroke}{rgb}{0.121569,0.466667,0.705882}%
\pgfsetstrokecolor{currentstroke}%
\pgfsetstrokeopacity{0.916277}%
\pgfsetdash{}{0pt}%
\pgfpathmoveto{\pgfqpoint{1.261837in}{1.784544in}}%
\pgfpathcurveto{\pgfqpoint{1.270073in}{1.784544in}}{\pgfqpoint{1.277973in}{1.787816in}}{\pgfqpoint{1.283797in}{1.793640in}}%
\pgfpathcurveto{\pgfqpoint{1.289621in}{1.799464in}}{\pgfqpoint{1.292893in}{1.807364in}}{\pgfqpoint{1.292893in}{1.815600in}}%
\pgfpathcurveto{\pgfqpoint{1.292893in}{1.823837in}}{\pgfqpoint{1.289621in}{1.831737in}}{\pgfqpoint{1.283797in}{1.837561in}}%
\pgfpathcurveto{\pgfqpoint{1.277973in}{1.843385in}}{\pgfqpoint{1.270073in}{1.846657in}}{\pgfqpoint{1.261837in}{1.846657in}}%
\pgfpathcurveto{\pgfqpoint{1.253601in}{1.846657in}}{\pgfqpoint{1.245700in}{1.843385in}}{\pgfqpoint{1.239877in}{1.837561in}}%
\pgfpathcurveto{\pgfqpoint{1.234053in}{1.831737in}}{\pgfqpoint{1.230780in}{1.823837in}}{\pgfqpoint{1.230780in}{1.815600in}}%
\pgfpathcurveto{\pgfqpoint{1.230780in}{1.807364in}}{\pgfqpoint{1.234053in}{1.799464in}}{\pgfqpoint{1.239877in}{1.793640in}}%
\pgfpathcurveto{\pgfqpoint{1.245700in}{1.787816in}}{\pgfqpoint{1.253601in}{1.784544in}}{\pgfqpoint{1.261837in}{1.784544in}}%
\pgfpathclose%
\pgfusepath{stroke,fill}%
\end{pgfscope}%
\begin{pgfscope}%
\pgfpathrectangle{\pgfqpoint{0.100000in}{0.212622in}}{\pgfqpoint{3.696000in}{3.696000in}}%
\pgfusepath{clip}%
\pgfsetbuttcap%
\pgfsetroundjoin%
\definecolor{currentfill}{rgb}{0.121569,0.466667,0.705882}%
\pgfsetfillcolor{currentfill}%
\pgfsetfillopacity{0.917285}%
\pgfsetlinewidth{1.003750pt}%
\definecolor{currentstroke}{rgb}{0.121569,0.466667,0.705882}%
\pgfsetstrokecolor{currentstroke}%
\pgfsetstrokeopacity{0.917285}%
\pgfsetdash{}{0pt}%
\pgfpathmoveto{\pgfqpoint{2.601652in}{2.437378in}}%
\pgfpathcurveto{\pgfqpoint{2.609888in}{2.437378in}}{\pgfqpoint{2.617788in}{2.440651in}}{\pgfqpoint{2.623612in}{2.446475in}}%
\pgfpathcurveto{\pgfqpoint{2.629436in}{2.452299in}}{\pgfqpoint{2.632708in}{2.460199in}}{\pgfqpoint{2.632708in}{2.468435in}}%
\pgfpathcurveto{\pgfqpoint{2.632708in}{2.476671in}}{\pgfqpoint{2.629436in}{2.484571in}}{\pgfqpoint{2.623612in}{2.490395in}}%
\pgfpathcurveto{\pgfqpoint{2.617788in}{2.496219in}}{\pgfqpoint{2.609888in}{2.499491in}}{\pgfqpoint{2.601652in}{2.499491in}}%
\pgfpathcurveto{\pgfqpoint{2.593415in}{2.499491in}}{\pgfqpoint{2.585515in}{2.496219in}}{\pgfqpoint{2.579692in}{2.490395in}}%
\pgfpathcurveto{\pgfqpoint{2.573868in}{2.484571in}}{\pgfqpoint{2.570595in}{2.476671in}}{\pgfqpoint{2.570595in}{2.468435in}}%
\pgfpathcurveto{\pgfqpoint{2.570595in}{2.460199in}}{\pgfqpoint{2.573868in}{2.452299in}}{\pgfqpoint{2.579692in}{2.446475in}}%
\pgfpathcurveto{\pgfqpoint{2.585515in}{2.440651in}}{\pgfqpoint{2.593415in}{2.437378in}}{\pgfqpoint{2.601652in}{2.437378in}}%
\pgfpathclose%
\pgfusepath{stroke,fill}%
\end{pgfscope}%
\begin{pgfscope}%
\pgfpathrectangle{\pgfqpoint{0.100000in}{0.212622in}}{\pgfqpoint{3.696000in}{3.696000in}}%
\pgfusepath{clip}%
\pgfsetbuttcap%
\pgfsetroundjoin%
\definecolor{currentfill}{rgb}{0.121569,0.466667,0.705882}%
\pgfsetfillcolor{currentfill}%
\pgfsetfillopacity{0.917479}%
\pgfsetlinewidth{1.003750pt}%
\definecolor{currentstroke}{rgb}{0.121569,0.466667,0.705882}%
\pgfsetstrokecolor{currentstroke}%
\pgfsetstrokeopacity{0.917479}%
\pgfsetdash{}{0pt}%
\pgfpathmoveto{\pgfqpoint{1.597966in}{1.974189in}}%
\pgfpathcurveto{\pgfqpoint{1.606202in}{1.974189in}}{\pgfqpoint{1.614102in}{1.977462in}}{\pgfqpoint{1.619926in}{1.983286in}}%
\pgfpathcurveto{\pgfqpoint{1.625750in}{1.989110in}}{\pgfqpoint{1.629022in}{1.997010in}}{\pgfqpoint{1.629022in}{2.005246in}}%
\pgfpathcurveto{\pgfqpoint{1.629022in}{2.013482in}}{\pgfqpoint{1.625750in}{2.021382in}}{\pgfqpoint{1.619926in}{2.027206in}}%
\pgfpathcurveto{\pgfqpoint{1.614102in}{2.033030in}}{\pgfqpoint{1.606202in}{2.036302in}}{\pgfqpoint{1.597966in}{2.036302in}}%
\pgfpathcurveto{\pgfqpoint{1.589729in}{2.036302in}}{\pgfqpoint{1.581829in}{2.033030in}}{\pgfqpoint{1.576005in}{2.027206in}}%
\pgfpathcurveto{\pgfqpoint{1.570181in}{2.021382in}}{\pgfqpoint{1.566909in}{2.013482in}}{\pgfqpoint{1.566909in}{2.005246in}}%
\pgfpathcurveto{\pgfqpoint{1.566909in}{1.997010in}}{\pgfqpoint{1.570181in}{1.989110in}}{\pgfqpoint{1.576005in}{1.983286in}}%
\pgfpathcurveto{\pgfqpoint{1.581829in}{1.977462in}}{\pgfqpoint{1.589729in}{1.974189in}}{\pgfqpoint{1.597966in}{1.974189in}}%
\pgfpathclose%
\pgfusepath{stroke,fill}%
\end{pgfscope}%
\begin{pgfscope}%
\pgfpathrectangle{\pgfqpoint{0.100000in}{0.212622in}}{\pgfqpoint{3.696000in}{3.696000in}}%
\pgfusepath{clip}%
\pgfsetbuttcap%
\pgfsetroundjoin%
\definecolor{currentfill}{rgb}{0.121569,0.466667,0.705882}%
\pgfsetfillcolor{currentfill}%
\pgfsetfillopacity{0.918229}%
\pgfsetlinewidth{1.003750pt}%
\definecolor{currentstroke}{rgb}{0.121569,0.466667,0.705882}%
\pgfsetstrokecolor{currentstroke}%
\pgfsetstrokeopacity{0.918229}%
\pgfsetdash{}{0pt}%
\pgfpathmoveto{\pgfqpoint{2.267234in}{2.258293in}}%
\pgfpathcurveto{\pgfqpoint{2.275470in}{2.258293in}}{\pgfqpoint{2.283370in}{2.261565in}}{\pgfqpoint{2.289194in}{2.267389in}}%
\pgfpathcurveto{\pgfqpoint{2.295018in}{2.273213in}}{\pgfqpoint{2.298290in}{2.281113in}}{\pgfqpoint{2.298290in}{2.289349in}}%
\pgfpathcurveto{\pgfqpoint{2.298290in}{2.297586in}}{\pgfqpoint{2.295018in}{2.305486in}}{\pgfqpoint{2.289194in}{2.311310in}}%
\pgfpathcurveto{\pgfqpoint{2.283370in}{2.317134in}}{\pgfqpoint{2.275470in}{2.320406in}}{\pgfqpoint{2.267234in}{2.320406in}}%
\pgfpathcurveto{\pgfqpoint{2.258997in}{2.320406in}}{\pgfqpoint{2.251097in}{2.317134in}}{\pgfqpoint{2.245273in}{2.311310in}}%
\pgfpathcurveto{\pgfqpoint{2.239449in}{2.305486in}}{\pgfqpoint{2.236177in}{2.297586in}}{\pgfqpoint{2.236177in}{2.289349in}}%
\pgfpathcurveto{\pgfqpoint{2.236177in}{2.281113in}}{\pgfqpoint{2.239449in}{2.273213in}}{\pgfqpoint{2.245273in}{2.267389in}}%
\pgfpathcurveto{\pgfqpoint{2.251097in}{2.261565in}}{\pgfqpoint{2.258997in}{2.258293in}}{\pgfqpoint{2.267234in}{2.258293in}}%
\pgfpathclose%
\pgfusepath{stroke,fill}%
\end{pgfscope}%
\begin{pgfscope}%
\pgfpathrectangle{\pgfqpoint{0.100000in}{0.212622in}}{\pgfqpoint{3.696000in}{3.696000in}}%
\pgfusepath{clip}%
\pgfsetbuttcap%
\pgfsetroundjoin%
\definecolor{currentfill}{rgb}{0.121569,0.466667,0.705882}%
\pgfsetfillcolor{currentfill}%
\pgfsetfillopacity{0.918308}%
\pgfsetlinewidth{1.003750pt}%
\definecolor{currentstroke}{rgb}{0.121569,0.466667,0.705882}%
\pgfsetstrokecolor{currentstroke}%
\pgfsetstrokeopacity{0.918308}%
\pgfsetdash{}{0pt}%
\pgfpathmoveto{\pgfqpoint{1.518943in}{1.928591in}}%
\pgfpathcurveto{\pgfqpoint{1.527179in}{1.928591in}}{\pgfqpoint{1.535079in}{1.931864in}}{\pgfqpoint{1.540903in}{1.937687in}}%
\pgfpathcurveto{\pgfqpoint{1.546727in}{1.943511in}}{\pgfqpoint{1.550000in}{1.951411in}}{\pgfqpoint{1.550000in}{1.959648in}}%
\pgfpathcurveto{\pgfqpoint{1.550000in}{1.967884in}}{\pgfqpoint{1.546727in}{1.975784in}}{\pgfqpoint{1.540903in}{1.981608in}}%
\pgfpathcurveto{\pgfqpoint{1.535079in}{1.987432in}}{\pgfqpoint{1.527179in}{1.990704in}}{\pgfqpoint{1.518943in}{1.990704in}}%
\pgfpathcurveto{\pgfqpoint{1.510707in}{1.990704in}}{\pgfqpoint{1.502807in}{1.987432in}}{\pgfqpoint{1.496983in}{1.981608in}}%
\pgfpathcurveto{\pgfqpoint{1.491159in}{1.975784in}}{\pgfqpoint{1.487887in}{1.967884in}}{\pgfqpoint{1.487887in}{1.959648in}}%
\pgfpathcurveto{\pgfqpoint{1.487887in}{1.951411in}}{\pgfqpoint{1.491159in}{1.943511in}}{\pgfqpoint{1.496983in}{1.937687in}}%
\pgfpathcurveto{\pgfqpoint{1.502807in}{1.931864in}}{\pgfqpoint{1.510707in}{1.928591in}}{\pgfqpoint{1.518943in}{1.928591in}}%
\pgfpathclose%
\pgfusepath{stroke,fill}%
\end{pgfscope}%
\begin{pgfscope}%
\pgfpathrectangle{\pgfqpoint{0.100000in}{0.212622in}}{\pgfqpoint{3.696000in}{3.696000in}}%
\pgfusepath{clip}%
\pgfsetbuttcap%
\pgfsetroundjoin%
\definecolor{currentfill}{rgb}{0.121569,0.466667,0.705882}%
\pgfsetfillcolor{currentfill}%
\pgfsetfillopacity{0.918878}%
\pgfsetlinewidth{1.003750pt}%
\definecolor{currentstroke}{rgb}{0.121569,0.466667,0.705882}%
\pgfsetstrokecolor{currentstroke}%
\pgfsetstrokeopacity{0.918878}%
\pgfsetdash{}{0pt}%
\pgfpathmoveto{\pgfqpoint{2.614300in}{2.452021in}}%
\pgfpathcurveto{\pgfqpoint{2.622536in}{2.452021in}}{\pgfqpoint{2.630436in}{2.455293in}}{\pgfqpoint{2.636260in}{2.461117in}}%
\pgfpathcurveto{\pgfqpoint{2.642084in}{2.466941in}}{\pgfqpoint{2.645356in}{2.474841in}}{\pgfqpoint{2.645356in}{2.483078in}}%
\pgfpathcurveto{\pgfqpoint{2.645356in}{2.491314in}}{\pgfqpoint{2.642084in}{2.499214in}}{\pgfqpoint{2.636260in}{2.505038in}}%
\pgfpathcurveto{\pgfqpoint{2.630436in}{2.510862in}}{\pgfqpoint{2.622536in}{2.514134in}}{\pgfqpoint{2.614300in}{2.514134in}}%
\pgfpathcurveto{\pgfqpoint{2.606063in}{2.514134in}}{\pgfqpoint{2.598163in}{2.510862in}}{\pgfqpoint{2.592340in}{2.505038in}}%
\pgfpathcurveto{\pgfqpoint{2.586516in}{2.499214in}}{\pgfqpoint{2.583243in}{2.491314in}}{\pgfqpoint{2.583243in}{2.483078in}}%
\pgfpathcurveto{\pgfqpoint{2.583243in}{2.474841in}}{\pgfqpoint{2.586516in}{2.466941in}}{\pgfqpoint{2.592340in}{2.461117in}}%
\pgfpathcurveto{\pgfqpoint{2.598163in}{2.455293in}}{\pgfqpoint{2.606063in}{2.452021in}}{\pgfqpoint{2.614300in}{2.452021in}}%
\pgfpathclose%
\pgfusepath{stroke,fill}%
\end{pgfscope}%
\begin{pgfscope}%
\pgfpathrectangle{\pgfqpoint{0.100000in}{0.212622in}}{\pgfqpoint{3.696000in}{3.696000in}}%
\pgfusepath{clip}%
\pgfsetbuttcap%
\pgfsetroundjoin%
\definecolor{currentfill}{rgb}{0.121569,0.466667,0.705882}%
\pgfsetfillcolor{currentfill}%
\pgfsetfillopacity{0.920282}%
\pgfsetlinewidth{1.003750pt}%
\definecolor{currentstroke}{rgb}{0.121569,0.466667,0.705882}%
\pgfsetstrokecolor{currentstroke}%
\pgfsetstrokeopacity{0.920282}%
\pgfsetdash{}{0pt}%
\pgfpathmoveto{\pgfqpoint{1.696874in}{2.057571in}}%
\pgfpathcurveto{\pgfqpoint{1.705110in}{2.057571in}}{\pgfqpoint{1.713010in}{2.060843in}}{\pgfqpoint{1.718834in}{2.066667in}}%
\pgfpathcurveto{\pgfqpoint{1.724658in}{2.072491in}}{\pgfqpoint{1.727930in}{2.080391in}}{\pgfqpoint{1.727930in}{2.088628in}}%
\pgfpathcurveto{\pgfqpoint{1.727930in}{2.096864in}}{\pgfqpoint{1.724658in}{2.104764in}}{\pgfqpoint{1.718834in}{2.110588in}}%
\pgfpathcurveto{\pgfqpoint{1.713010in}{2.116412in}}{\pgfqpoint{1.705110in}{2.119684in}}{\pgfqpoint{1.696874in}{2.119684in}}%
\pgfpathcurveto{\pgfqpoint{1.688638in}{2.119684in}}{\pgfqpoint{1.680738in}{2.116412in}}{\pgfqpoint{1.674914in}{2.110588in}}%
\pgfpathcurveto{\pgfqpoint{1.669090in}{2.104764in}}{\pgfqpoint{1.665817in}{2.096864in}}{\pgfqpoint{1.665817in}{2.088628in}}%
\pgfpathcurveto{\pgfqpoint{1.665817in}{2.080391in}}{\pgfqpoint{1.669090in}{2.072491in}}{\pgfqpoint{1.674914in}{2.066667in}}%
\pgfpathcurveto{\pgfqpoint{1.680738in}{2.060843in}}{\pgfqpoint{1.688638in}{2.057571in}}{\pgfqpoint{1.696874in}{2.057571in}}%
\pgfpathclose%
\pgfusepath{stroke,fill}%
\end{pgfscope}%
\begin{pgfscope}%
\pgfpathrectangle{\pgfqpoint{0.100000in}{0.212622in}}{\pgfqpoint{3.696000in}{3.696000in}}%
\pgfusepath{clip}%
\pgfsetbuttcap%
\pgfsetroundjoin%
\definecolor{currentfill}{rgb}{0.121569,0.466667,0.705882}%
\pgfsetfillcolor{currentfill}%
\pgfsetfillopacity{0.920659}%
\pgfsetlinewidth{1.003750pt}%
\definecolor{currentstroke}{rgb}{0.121569,0.466667,0.705882}%
\pgfsetstrokecolor{currentstroke}%
\pgfsetstrokeopacity{0.920659}%
\pgfsetdash{}{0pt}%
\pgfpathmoveto{\pgfqpoint{1.688522in}{2.052130in}}%
\pgfpathcurveto{\pgfqpoint{1.696758in}{2.052130in}}{\pgfqpoint{1.704658in}{2.055403in}}{\pgfqpoint{1.710482in}{2.061227in}}%
\pgfpathcurveto{\pgfqpoint{1.716306in}{2.067051in}}{\pgfqpoint{1.719578in}{2.074951in}}{\pgfqpoint{1.719578in}{2.083187in}}%
\pgfpathcurveto{\pgfqpoint{1.719578in}{2.091423in}}{\pgfqpoint{1.716306in}{2.099323in}}{\pgfqpoint{1.710482in}{2.105147in}}%
\pgfpathcurveto{\pgfqpoint{1.704658in}{2.110971in}}{\pgfqpoint{1.696758in}{2.114243in}}{\pgfqpoint{1.688522in}{2.114243in}}%
\pgfpathcurveto{\pgfqpoint{1.680285in}{2.114243in}}{\pgfqpoint{1.672385in}{2.110971in}}{\pgfqpoint{1.666561in}{2.105147in}}%
\pgfpathcurveto{\pgfqpoint{1.660737in}{2.099323in}}{\pgfqpoint{1.657465in}{2.091423in}}{\pgfqpoint{1.657465in}{2.083187in}}%
\pgfpathcurveto{\pgfqpoint{1.657465in}{2.074951in}}{\pgfqpoint{1.660737in}{2.067051in}}{\pgfqpoint{1.666561in}{2.061227in}}%
\pgfpathcurveto{\pgfqpoint{1.672385in}{2.055403in}}{\pgfqpoint{1.680285in}{2.052130in}}{\pgfqpoint{1.688522in}{2.052130in}}%
\pgfpathclose%
\pgfusepath{stroke,fill}%
\end{pgfscope}%
\begin{pgfscope}%
\pgfpathrectangle{\pgfqpoint{0.100000in}{0.212622in}}{\pgfqpoint{3.696000in}{3.696000in}}%
\pgfusepath{clip}%
\pgfsetbuttcap%
\pgfsetroundjoin%
\definecolor{currentfill}{rgb}{0.121569,0.466667,0.705882}%
\pgfsetfillcolor{currentfill}%
\pgfsetfillopacity{0.920775}%
\pgfsetlinewidth{1.003750pt}%
\definecolor{currentstroke}{rgb}{0.121569,0.466667,0.705882}%
\pgfsetstrokecolor{currentstroke}%
\pgfsetstrokeopacity{0.920775}%
\pgfsetdash{}{0pt}%
\pgfpathmoveto{\pgfqpoint{1.815768in}{2.081824in}}%
\pgfpathcurveto{\pgfqpoint{1.824004in}{2.081824in}}{\pgfqpoint{1.831904in}{2.085096in}}{\pgfqpoint{1.837728in}{2.090920in}}%
\pgfpathcurveto{\pgfqpoint{1.843552in}{2.096744in}}{\pgfqpoint{1.846824in}{2.104644in}}{\pgfqpoint{1.846824in}{2.112880in}}%
\pgfpathcurveto{\pgfqpoint{1.846824in}{2.121116in}}{\pgfqpoint{1.843552in}{2.129016in}}{\pgfqpoint{1.837728in}{2.134840in}}%
\pgfpathcurveto{\pgfqpoint{1.831904in}{2.140664in}}{\pgfqpoint{1.824004in}{2.143937in}}{\pgfqpoint{1.815768in}{2.143937in}}%
\pgfpathcurveto{\pgfqpoint{1.807532in}{2.143937in}}{\pgfqpoint{1.799632in}{2.140664in}}{\pgfqpoint{1.793808in}{2.134840in}}%
\pgfpathcurveto{\pgfqpoint{1.787984in}{2.129016in}}{\pgfqpoint{1.784711in}{2.121116in}}{\pgfqpoint{1.784711in}{2.112880in}}%
\pgfpathcurveto{\pgfqpoint{1.784711in}{2.104644in}}{\pgfqpoint{1.787984in}{2.096744in}}{\pgfqpoint{1.793808in}{2.090920in}}%
\pgfpathcurveto{\pgfqpoint{1.799632in}{2.085096in}}{\pgfqpoint{1.807532in}{2.081824in}}{\pgfqpoint{1.815768in}{2.081824in}}%
\pgfpathclose%
\pgfusepath{stroke,fill}%
\end{pgfscope}%
\begin{pgfscope}%
\pgfpathrectangle{\pgfqpoint{0.100000in}{0.212622in}}{\pgfqpoint{3.696000in}{3.696000in}}%
\pgfusepath{clip}%
\pgfsetbuttcap%
\pgfsetroundjoin%
\definecolor{currentfill}{rgb}{0.121569,0.466667,0.705882}%
\pgfsetfillcolor{currentfill}%
\pgfsetfillopacity{0.922937}%
\pgfsetlinewidth{1.003750pt}%
\definecolor{currentstroke}{rgb}{0.121569,0.466667,0.705882}%
\pgfsetstrokecolor{currentstroke}%
\pgfsetstrokeopacity{0.922937}%
\pgfsetdash{}{0pt}%
\pgfpathmoveto{\pgfqpoint{2.250395in}{2.247374in}}%
\pgfpathcurveto{\pgfqpoint{2.258632in}{2.247374in}}{\pgfqpoint{2.266532in}{2.250647in}}{\pgfqpoint{2.272356in}{2.256471in}}%
\pgfpathcurveto{\pgfqpoint{2.278180in}{2.262294in}}{\pgfqpoint{2.281452in}{2.270194in}}{\pgfqpoint{2.281452in}{2.278431in}}%
\pgfpathcurveto{\pgfqpoint{2.281452in}{2.286667in}}{\pgfqpoint{2.278180in}{2.294567in}}{\pgfqpoint{2.272356in}{2.300391in}}%
\pgfpathcurveto{\pgfqpoint{2.266532in}{2.306215in}}{\pgfqpoint{2.258632in}{2.309487in}}{\pgfqpoint{2.250395in}{2.309487in}}%
\pgfpathcurveto{\pgfqpoint{2.242159in}{2.309487in}}{\pgfqpoint{2.234259in}{2.306215in}}{\pgfqpoint{2.228435in}{2.300391in}}%
\pgfpathcurveto{\pgfqpoint{2.222611in}{2.294567in}}{\pgfqpoint{2.219339in}{2.286667in}}{\pgfqpoint{2.219339in}{2.278431in}}%
\pgfpathcurveto{\pgfqpoint{2.219339in}{2.270194in}}{\pgfqpoint{2.222611in}{2.262294in}}{\pgfqpoint{2.228435in}{2.256471in}}%
\pgfpathcurveto{\pgfqpoint{2.234259in}{2.250647in}}{\pgfqpoint{2.242159in}{2.247374in}}{\pgfqpoint{2.250395in}{2.247374in}}%
\pgfpathclose%
\pgfusepath{stroke,fill}%
\end{pgfscope}%
\begin{pgfscope}%
\pgfpathrectangle{\pgfqpoint{0.100000in}{0.212622in}}{\pgfqpoint{3.696000in}{3.696000in}}%
\pgfusepath{clip}%
\pgfsetbuttcap%
\pgfsetroundjoin%
\definecolor{currentfill}{rgb}{0.121569,0.466667,0.705882}%
\pgfsetfillcolor{currentfill}%
\pgfsetfillopacity{0.923374}%
\pgfsetlinewidth{1.003750pt}%
\definecolor{currentstroke}{rgb}{0.121569,0.466667,0.705882}%
\pgfsetstrokecolor{currentstroke}%
\pgfsetstrokeopacity{0.923374}%
\pgfsetdash{}{0pt}%
\pgfpathmoveto{\pgfqpoint{2.244308in}{2.244902in}}%
\pgfpathcurveto{\pgfqpoint{2.252544in}{2.244902in}}{\pgfqpoint{2.260444in}{2.248174in}}{\pgfqpoint{2.266268in}{2.253998in}}%
\pgfpathcurveto{\pgfqpoint{2.272092in}{2.259822in}}{\pgfqpoint{2.275364in}{2.267722in}}{\pgfqpoint{2.275364in}{2.275958in}}%
\pgfpathcurveto{\pgfqpoint{2.275364in}{2.284194in}}{\pgfqpoint{2.272092in}{2.292094in}}{\pgfqpoint{2.266268in}{2.297918in}}%
\pgfpathcurveto{\pgfqpoint{2.260444in}{2.303742in}}{\pgfqpoint{2.252544in}{2.307015in}}{\pgfqpoint{2.244308in}{2.307015in}}%
\pgfpathcurveto{\pgfqpoint{2.236072in}{2.307015in}}{\pgfqpoint{2.228171in}{2.303742in}}{\pgfqpoint{2.222348in}{2.297918in}}%
\pgfpathcurveto{\pgfqpoint{2.216524in}{2.292094in}}{\pgfqpoint{2.213251in}{2.284194in}}{\pgfqpoint{2.213251in}{2.275958in}}%
\pgfpathcurveto{\pgfqpoint{2.213251in}{2.267722in}}{\pgfqpoint{2.216524in}{2.259822in}}{\pgfqpoint{2.222348in}{2.253998in}}%
\pgfpathcurveto{\pgfqpoint{2.228171in}{2.248174in}}{\pgfqpoint{2.236072in}{2.244902in}}{\pgfqpoint{2.244308in}{2.244902in}}%
\pgfpathclose%
\pgfusepath{stroke,fill}%
\end{pgfscope}%
\begin{pgfscope}%
\pgfpathrectangle{\pgfqpoint{0.100000in}{0.212622in}}{\pgfqpoint{3.696000in}{3.696000in}}%
\pgfusepath{clip}%
\pgfsetbuttcap%
\pgfsetroundjoin%
\definecolor{currentfill}{rgb}{0.121569,0.466667,0.705882}%
\pgfsetfillcolor{currentfill}%
\pgfsetfillopacity{0.924488}%
\pgfsetlinewidth{1.003750pt}%
\definecolor{currentstroke}{rgb}{0.121569,0.466667,0.705882}%
\pgfsetstrokecolor{currentstroke}%
\pgfsetstrokeopacity{0.924488}%
\pgfsetdash{}{0pt}%
\pgfpathmoveto{\pgfqpoint{1.445461in}{1.897464in}}%
\pgfpathcurveto{\pgfqpoint{1.453697in}{1.897464in}}{\pgfqpoint{1.461597in}{1.900737in}}{\pgfqpoint{1.467421in}{1.906560in}}%
\pgfpathcurveto{\pgfqpoint{1.473245in}{1.912384in}}{\pgfqpoint{1.476517in}{1.920284in}}{\pgfqpoint{1.476517in}{1.928521in}}%
\pgfpathcurveto{\pgfqpoint{1.476517in}{1.936757in}}{\pgfqpoint{1.473245in}{1.944657in}}{\pgfqpoint{1.467421in}{1.950481in}}%
\pgfpathcurveto{\pgfqpoint{1.461597in}{1.956305in}}{\pgfqpoint{1.453697in}{1.959577in}}{\pgfqpoint{1.445461in}{1.959577in}}%
\pgfpathcurveto{\pgfqpoint{1.437224in}{1.959577in}}{\pgfqpoint{1.429324in}{1.956305in}}{\pgfqpoint{1.423501in}{1.950481in}}%
\pgfpathcurveto{\pgfqpoint{1.417677in}{1.944657in}}{\pgfqpoint{1.414404in}{1.936757in}}{\pgfqpoint{1.414404in}{1.928521in}}%
\pgfpathcurveto{\pgfqpoint{1.414404in}{1.920284in}}{\pgfqpoint{1.417677in}{1.912384in}}{\pgfqpoint{1.423501in}{1.906560in}}%
\pgfpathcurveto{\pgfqpoint{1.429324in}{1.900737in}}{\pgfqpoint{1.437224in}{1.897464in}}{\pgfqpoint{1.445461in}{1.897464in}}%
\pgfpathclose%
\pgfusepath{stroke,fill}%
\end{pgfscope}%
\begin{pgfscope}%
\pgfpathrectangle{\pgfqpoint{0.100000in}{0.212622in}}{\pgfqpoint{3.696000in}{3.696000in}}%
\pgfusepath{clip}%
\pgfsetbuttcap%
\pgfsetroundjoin%
\definecolor{currentfill}{rgb}{0.121569,0.466667,0.705882}%
\pgfsetfillcolor{currentfill}%
\pgfsetfillopacity{0.925172}%
\pgfsetlinewidth{1.003750pt}%
\definecolor{currentstroke}{rgb}{0.121569,0.466667,0.705882}%
\pgfsetstrokecolor{currentstroke}%
\pgfsetstrokeopacity{0.925172}%
\pgfsetdash{}{0pt}%
\pgfpathmoveto{\pgfqpoint{1.629966in}{2.027874in}}%
\pgfpathcurveto{\pgfqpoint{1.638203in}{2.027874in}}{\pgfqpoint{1.646103in}{2.031146in}}{\pgfqpoint{1.651926in}{2.036970in}}%
\pgfpathcurveto{\pgfqpoint{1.657750in}{2.042794in}}{\pgfqpoint{1.661023in}{2.050694in}}{\pgfqpoint{1.661023in}{2.058930in}}%
\pgfpathcurveto{\pgfqpoint{1.661023in}{2.067167in}}{\pgfqpoint{1.657750in}{2.075067in}}{\pgfqpoint{1.651926in}{2.080891in}}%
\pgfpathcurveto{\pgfqpoint{1.646103in}{2.086714in}}{\pgfqpoint{1.638203in}{2.089987in}}{\pgfqpoint{1.629966in}{2.089987in}}%
\pgfpathcurveto{\pgfqpoint{1.621730in}{2.089987in}}{\pgfqpoint{1.613830in}{2.086714in}}{\pgfqpoint{1.608006in}{2.080891in}}%
\pgfpathcurveto{\pgfqpoint{1.602182in}{2.075067in}}{\pgfqpoint{1.598910in}{2.067167in}}{\pgfqpoint{1.598910in}{2.058930in}}%
\pgfpathcurveto{\pgfqpoint{1.598910in}{2.050694in}}{\pgfqpoint{1.602182in}{2.042794in}}{\pgfqpoint{1.608006in}{2.036970in}}%
\pgfpathcurveto{\pgfqpoint{1.613830in}{2.031146in}}{\pgfqpoint{1.621730in}{2.027874in}}{\pgfqpoint{1.629966in}{2.027874in}}%
\pgfpathclose%
\pgfusepath{stroke,fill}%
\end{pgfscope}%
\begin{pgfscope}%
\pgfpathrectangle{\pgfqpoint{0.100000in}{0.212622in}}{\pgfqpoint{3.696000in}{3.696000in}}%
\pgfusepath{clip}%
\pgfsetbuttcap%
\pgfsetroundjoin%
\definecolor{currentfill}{rgb}{0.121569,0.466667,0.705882}%
\pgfsetfillcolor{currentfill}%
\pgfsetfillopacity{0.925283}%
\pgfsetlinewidth{1.003750pt}%
\definecolor{currentstroke}{rgb}{0.121569,0.466667,0.705882}%
\pgfsetstrokecolor{currentstroke}%
\pgfsetstrokeopacity{0.925283}%
\pgfsetdash{}{0pt}%
\pgfpathmoveto{\pgfqpoint{2.235553in}{2.237348in}}%
\pgfpathcurveto{\pgfqpoint{2.243789in}{2.237348in}}{\pgfqpoint{2.251689in}{2.240621in}}{\pgfqpoint{2.257513in}{2.246445in}}%
\pgfpathcurveto{\pgfqpoint{2.263337in}{2.252269in}}{\pgfqpoint{2.266609in}{2.260169in}}{\pgfqpoint{2.266609in}{2.268405in}}%
\pgfpathcurveto{\pgfqpoint{2.266609in}{2.276641in}}{\pgfqpoint{2.263337in}{2.284541in}}{\pgfqpoint{2.257513in}{2.290365in}}%
\pgfpathcurveto{\pgfqpoint{2.251689in}{2.296189in}}{\pgfqpoint{2.243789in}{2.299461in}}{\pgfqpoint{2.235553in}{2.299461in}}%
\pgfpathcurveto{\pgfqpoint{2.227316in}{2.299461in}}{\pgfqpoint{2.219416in}{2.296189in}}{\pgfqpoint{2.213592in}{2.290365in}}%
\pgfpathcurveto{\pgfqpoint{2.207769in}{2.284541in}}{\pgfqpoint{2.204496in}{2.276641in}}{\pgfqpoint{2.204496in}{2.268405in}}%
\pgfpathcurveto{\pgfqpoint{2.204496in}{2.260169in}}{\pgfqpoint{2.207769in}{2.252269in}}{\pgfqpoint{2.213592in}{2.246445in}}%
\pgfpathcurveto{\pgfqpoint{2.219416in}{2.240621in}}{\pgfqpoint{2.227316in}{2.237348in}}{\pgfqpoint{2.235553in}{2.237348in}}%
\pgfpathclose%
\pgfusepath{stroke,fill}%
\end{pgfscope}%
\begin{pgfscope}%
\pgfpathrectangle{\pgfqpoint{0.100000in}{0.212622in}}{\pgfqpoint{3.696000in}{3.696000in}}%
\pgfusepath{clip}%
\pgfsetbuttcap%
\pgfsetroundjoin%
\definecolor{currentfill}{rgb}{0.121569,0.466667,0.705882}%
\pgfsetfillcolor{currentfill}%
\pgfsetfillopacity{0.925302}%
\pgfsetlinewidth{1.003750pt}%
\definecolor{currentstroke}{rgb}{0.121569,0.466667,0.705882}%
\pgfsetstrokecolor{currentstroke}%
\pgfsetstrokeopacity{0.925302}%
\pgfsetdash{}{0pt}%
\pgfpathmoveto{\pgfqpoint{2.236664in}{2.237907in}}%
\pgfpathcurveto{\pgfqpoint{2.244900in}{2.237907in}}{\pgfqpoint{2.252800in}{2.241179in}}{\pgfqpoint{2.258624in}{2.247003in}}%
\pgfpathcurveto{\pgfqpoint{2.264448in}{2.252827in}}{\pgfqpoint{2.267720in}{2.260727in}}{\pgfqpoint{2.267720in}{2.268963in}}%
\pgfpathcurveto{\pgfqpoint{2.267720in}{2.277200in}}{\pgfqpoint{2.264448in}{2.285100in}}{\pgfqpoint{2.258624in}{2.290924in}}%
\pgfpathcurveto{\pgfqpoint{2.252800in}{2.296748in}}{\pgfqpoint{2.244900in}{2.300020in}}{\pgfqpoint{2.236664in}{2.300020in}}%
\pgfpathcurveto{\pgfqpoint{2.228427in}{2.300020in}}{\pgfqpoint{2.220527in}{2.296748in}}{\pgfqpoint{2.214703in}{2.290924in}}%
\pgfpathcurveto{\pgfqpoint{2.208879in}{2.285100in}}{\pgfqpoint{2.205607in}{2.277200in}}{\pgfqpoint{2.205607in}{2.268963in}}%
\pgfpathcurveto{\pgfqpoint{2.205607in}{2.260727in}}{\pgfqpoint{2.208879in}{2.252827in}}{\pgfqpoint{2.214703in}{2.247003in}}%
\pgfpathcurveto{\pgfqpoint{2.220527in}{2.241179in}}{\pgfqpoint{2.228427in}{2.237907in}}{\pgfqpoint{2.236664in}{2.237907in}}%
\pgfpathclose%
\pgfusepath{stroke,fill}%
\end{pgfscope}%
\begin{pgfscope}%
\pgfpathrectangle{\pgfqpoint{0.100000in}{0.212622in}}{\pgfqpoint{3.696000in}{3.696000in}}%
\pgfusepath{clip}%
\pgfsetbuttcap%
\pgfsetroundjoin%
\definecolor{currentfill}{rgb}{0.121569,0.466667,0.705882}%
\pgfsetfillcolor{currentfill}%
\pgfsetfillopacity{0.925935}%
\pgfsetlinewidth{1.003750pt}%
\definecolor{currentstroke}{rgb}{0.121569,0.466667,0.705882}%
\pgfsetstrokecolor{currentstroke}%
\pgfsetstrokeopacity{0.925935}%
\pgfsetdash{}{0pt}%
\pgfpathmoveto{\pgfqpoint{3.018341in}{2.562030in}}%
\pgfpathcurveto{\pgfqpoint{3.026577in}{2.562030in}}{\pgfqpoint{3.034477in}{2.565303in}}{\pgfqpoint{3.040301in}{2.571126in}}%
\pgfpathcurveto{\pgfqpoint{3.046125in}{2.576950in}}{\pgfqpoint{3.049397in}{2.584850in}}{\pgfqpoint{3.049397in}{2.593087in}}%
\pgfpathcurveto{\pgfqpoint{3.049397in}{2.601323in}}{\pgfqpoint{3.046125in}{2.609223in}}{\pgfqpoint{3.040301in}{2.615047in}}%
\pgfpathcurveto{\pgfqpoint{3.034477in}{2.620871in}}{\pgfqpoint{3.026577in}{2.624143in}}{\pgfqpoint{3.018341in}{2.624143in}}%
\pgfpathcurveto{\pgfqpoint{3.010105in}{2.624143in}}{\pgfqpoint{3.002205in}{2.620871in}}{\pgfqpoint{2.996381in}{2.615047in}}%
\pgfpathcurveto{\pgfqpoint{2.990557in}{2.609223in}}{\pgfqpoint{2.987284in}{2.601323in}}{\pgfqpoint{2.987284in}{2.593087in}}%
\pgfpathcurveto{\pgfqpoint{2.987284in}{2.584850in}}{\pgfqpoint{2.990557in}{2.576950in}}{\pgfqpoint{2.996381in}{2.571126in}}%
\pgfpathcurveto{\pgfqpoint{3.002205in}{2.565303in}}{\pgfqpoint{3.010105in}{2.562030in}}{\pgfqpoint{3.018341in}{2.562030in}}%
\pgfpathclose%
\pgfusepath{stroke,fill}%
\end{pgfscope}%
\begin{pgfscope}%
\pgfpathrectangle{\pgfqpoint{0.100000in}{0.212622in}}{\pgfqpoint{3.696000in}{3.696000in}}%
\pgfusepath{clip}%
\pgfsetbuttcap%
\pgfsetroundjoin%
\definecolor{currentfill}{rgb}{0.121569,0.466667,0.705882}%
\pgfsetfillcolor{currentfill}%
\pgfsetfillopacity{0.926158}%
\pgfsetlinewidth{1.003750pt}%
\definecolor{currentstroke}{rgb}{0.121569,0.466667,0.705882}%
\pgfsetstrokecolor{currentstroke}%
\pgfsetstrokeopacity{0.926158}%
\pgfsetdash{}{0pt}%
\pgfpathmoveto{\pgfqpoint{2.231764in}{2.235333in}}%
\pgfpathcurveto{\pgfqpoint{2.240000in}{2.235333in}}{\pgfqpoint{2.247900in}{2.238605in}}{\pgfqpoint{2.253724in}{2.244429in}}%
\pgfpathcurveto{\pgfqpoint{2.259548in}{2.250253in}}{\pgfqpoint{2.262820in}{2.258153in}}{\pgfqpoint{2.262820in}{2.266389in}}%
\pgfpathcurveto{\pgfqpoint{2.262820in}{2.274625in}}{\pgfqpoint{2.259548in}{2.282525in}}{\pgfqpoint{2.253724in}{2.288349in}}%
\pgfpathcurveto{\pgfqpoint{2.247900in}{2.294173in}}{\pgfqpoint{2.240000in}{2.297446in}}{\pgfqpoint{2.231764in}{2.297446in}}%
\pgfpathcurveto{\pgfqpoint{2.223527in}{2.297446in}}{\pgfqpoint{2.215627in}{2.294173in}}{\pgfqpoint{2.209803in}{2.288349in}}%
\pgfpathcurveto{\pgfqpoint{2.203979in}{2.282525in}}{\pgfqpoint{2.200707in}{2.274625in}}{\pgfqpoint{2.200707in}{2.266389in}}%
\pgfpathcurveto{\pgfqpoint{2.200707in}{2.258153in}}{\pgfqpoint{2.203979in}{2.250253in}}{\pgfqpoint{2.209803in}{2.244429in}}%
\pgfpathcurveto{\pgfqpoint{2.215627in}{2.238605in}}{\pgfqpoint{2.223527in}{2.235333in}}{\pgfqpoint{2.231764in}{2.235333in}}%
\pgfpathclose%
\pgfusepath{stroke,fill}%
\end{pgfscope}%
\begin{pgfscope}%
\pgfpathrectangle{\pgfqpoint{0.100000in}{0.212622in}}{\pgfqpoint{3.696000in}{3.696000in}}%
\pgfusepath{clip}%
\pgfsetbuttcap%
\pgfsetroundjoin%
\definecolor{currentfill}{rgb}{0.121569,0.466667,0.705882}%
\pgfsetfillcolor{currentfill}%
\pgfsetfillopacity{0.926241}%
\pgfsetlinewidth{1.003750pt}%
\definecolor{currentstroke}{rgb}{0.121569,0.466667,0.705882}%
\pgfsetstrokecolor{currentstroke}%
\pgfsetstrokeopacity{0.926241}%
\pgfsetdash{}{0pt}%
\pgfpathmoveto{\pgfqpoint{1.254956in}{1.780337in}}%
\pgfpathcurveto{\pgfqpoint{1.263192in}{1.780337in}}{\pgfqpoint{1.271092in}{1.783609in}}{\pgfqpoint{1.276916in}{1.789433in}}%
\pgfpathcurveto{\pgfqpoint{1.282740in}{1.795257in}}{\pgfqpoint{1.286012in}{1.803157in}}{\pgfqpoint{1.286012in}{1.811393in}}%
\pgfpathcurveto{\pgfqpoint{1.286012in}{1.819629in}}{\pgfqpoint{1.282740in}{1.827529in}}{\pgfqpoint{1.276916in}{1.833353in}}%
\pgfpathcurveto{\pgfqpoint{1.271092in}{1.839177in}}{\pgfqpoint{1.263192in}{1.842450in}}{\pgfqpoint{1.254956in}{1.842450in}}%
\pgfpathcurveto{\pgfqpoint{1.246719in}{1.842450in}}{\pgfqpoint{1.238819in}{1.839177in}}{\pgfqpoint{1.232995in}{1.833353in}}%
\pgfpathcurveto{\pgfqpoint{1.227171in}{1.827529in}}{\pgfqpoint{1.223899in}{1.819629in}}{\pgfqpoint{1.223899in}{1.811393in}}%
\pgfpathcurveto{\pgfqpoint{1.223899in}{1.803157in}}{\pgfqpoint{1.227171in}{1.795257in}}{\pgfqpoint{1.232995in}{1.789433in}}%
\pgfpathcurveto{\pgfqpoint{1.238819in}{1.783609in}}{\pgfqpoint{1.246719in}{1.780337in}}{\pgfqpoint{1.254956in}{1.780337in}}%
\pgfpathclose%
\pgfusepath{stroke,fill}%
\end{pgfscope}%
\begin{pgfscope}%
\pgfpathrectangle{\pgfqpoint{0.100000in}{0.212622in}}{\pgfqpoint{3.696000in}{3.696000in}}%
\pgfusepath{clip}%
\pgfsetbuttcap%
\pgfsetroundjoin%
\definecolor{currentfill}{rgb}{0.121569,0.466667,0.705882}%
\pgfsetfillcolor{currentfill}%
\pgfsetfillopacity{0.927182}%
\pgfsetlinewidth{1.003750pt}%
\definecolor{currentstroke}{rgb}{0.121569,0.466667,0.705882}%
\pgfsetstrokecolor{currentstroke}%
\pgfsetstrokeopacity{0.927182}%
\pgfsetdash{}{0pt}%
\pgfpathmoveto{\pgfqpoint{2.225665in}{2.231211in}}%
\pgfpathcurveto{\pgfqpoint{2.233901in}{2.231211in}}{\pgfqpoint{2.241801in}{2.234483in}}{\pgfqpoint{2.247625in}{2.240307in}}%
\pgfpathcurveto{\pgfqpoint{2.253449in}{2.246131in}}{\pgfqpoint{2.256721in}{2.254031in}}{\pgfqpoint{2.256721in}{2.262267in}}%
\pgfpathcurveto{\pgfqpoint{2.256721in}{2.270503in}}{\pgfqpoint{2.253449in}{2.278403in}}{\pgfqpoint{2.247625in}{2.284227in}}%
\pgfpathcurveto{\pgfqpoint{2.241801in}{2.290051in}}{\pgfqpoint{2.233901in}{2.293324in}}{\pgfqpoint{2.225665in}{2.293324in}}%
\pgfpathcurveto{\pgfqpoint{2.217428in}{2.293324in}}{\pgfqpoint{2.209528in}{2.290051in}}{\pgfqpoint{2.203704in}{2.284227in}}%
\pgfpathcurveto{\pgfqpoint{2.197881in}{2.278403in}}{\pgfqpoint{2.194608in}{2.270503in}}{\pgfqpoint{2.194608in}{2.262267in}}%
\pgfpathcurveto{\pgfqpoint{2.194608in}{2.254031in}}{\pgfqpoint{2.197881in}{2.246131in}}{\pgfqpoint{2.203704in}{2.240307in}}%
\pgfpathcurveto{\pgfqpoint{2.209528in}{2.234483in}}{\pgfqpoint{2.217428in}{2.231211in}}{\pgfqpoint{2.225665in}{2.231211in}}%
\pgfpathclose%
\pgfusepath{stroke,fill}%
\end{pgfscope}%
\begin{pgfscope}%
\pgfpathrectangle{\pgfqpoint{0.100000in}{0.212622in}}{\pgfqpoint{3.696000in}{3.696000in}}%
\pgfusepath{clip}%
\pgfsetbuttcap%
\pgfsetroundjoin%
\definecolor{currentfill}{rgb}{0.121569,0.466667,0.705882}%
\pgfsetfillcolor{currentfill}%
\pgfsetfillopacity{0.927368}%
\pgfsetlinewidth{1.003750pt}%
\definecolor{currentstroke}{rgb}{0.121569,0.466667,0.705882}%
\pgfsetstrokecolor{currentstroke}%
\pgfsetstrokeopacity{0.927368}%
\pgfsetdash{}{0pt}%
\pgfpathmoveto{\pgfqpoint{1.816583in}{2.083892in}}%
\pgfpathcurveto{\pgfqpoint{1.824820in}{2.083892in}}{\pgfqpoint{1.832720in}{2.087165in}}{\pgfqpoint{1.838544in}{2.092989in}}%
\pgfpathcurveto{\pgfqpoint{1.844367in}{2.098813in}}{\pgfqpoint{1.847640in}{2.106713in}}{\pgfqpoint{1.847640in}{2.114949in}}%
\pgfpathcurveto{\pgfqpoint{1.847640in}{2.123185in}}{\pgfqpoint{1.844367in}{2.131085in}}{\pgfqpoint{1.838544in}{2.136909in}}%
\pgfpathcurveto{\pgfqpoint{1.832720in}{2.142733in}}{\pgfqpoint{1.824820in}{2.146005in}}{\pgfqpoint{1.816583in}{2.146005in}}%
\pgfpathcurveto{\pgfqpoint{1.808347in}{2.146005in}}{\pgfqpoint{1.800447in}{2.142733in}}{\pgfqpoint{1.794623in}{2.136909in}}%
\pgfpathcurveto{\pgfqpoint{1.788799in}{2.131085in}}{\pgfqpoint{1.785527in}{2.123185in}}{\pgfqpoint{1.785527in}{2.114949in}}%
\pgfpathcurveto{\pgfqpoint{1.785527in}{2.106713in}}{\pgfqpoint{1.788799in}{2.098813in}}{\pgfqpoint{1.794623in}{2.092989in}}%
\pgfpathcurveto{\pgfqpoint{1.800447in}{2.087165in}}{\pgfqpoint{1.808347in}{2.083892in}}{\pgfqpoint{1.816583in}{2.083892in}}%
\pgfpathclose%
\pgfusepath{stroke,fill}%
\end{pgfscope}%
\begin{pgfscope}%
\pgfpathrectangle{\pgfqpoint{0.100000in}{0.212622in}}{\pgfqpoint{3.696000in}{3.696000in}}%
\pgfusepath{clip}%
\pgfsetbuttcap%
\pgfsetroundjoin%
\definecolor{currentfill}{rgb}{0.121569,0.466667,0.705882}%
\pgfsetfillcolor{currentfill}%
\pgfsetfillopacity{0.927389}%
\pgfsetlinewidth{1.003750pt}%
\definecolor{currentstroke}{rgb}{0.121569,0.466667,0.705882}%
\pgfsetstrokecolor{currentstroke}%
\pgfsetstrokeopacity{0.927389}%
\pgfsetdash{}{0pt}%
\pgfpathmoveto{\pgfqpoint{1.466332in}{1.910658in}}%
\pgfpathcurveto{\pgfqpoint{1.474568in}{1.910658in}}{\pgfqpoint{1.482468in}{1.913930in}}{\pgfqpoint{1.488292in}{1.919754in}}%
\pgfpathcurveto{\pgfqpoint{1.494116in}{1.925578in}}{\pgfqpoint{1.497388in}{1.933478in}}{\pgfqpoint{1.497388in}{1.941715in}}%
\pgfpathcurveto{\pgfqpoint{1.497388in}{1.949951in}}{\pgfqpoint{1.494116in}{1.957851in}}{\pgfqpoint{1.488292in}{1.963675in}}%
\pgfpathcurveto{\pgfqpoint{1.482468in}{1.969499in}}{\pgfqpoint{1.474568in}{1.972771in}}{\pgfqpoint{1.466332in}{1.972771in}}%
\pgfpathcurveto{\pgfqpoint{1.458096in}{1.972771in}}{\pgfqpoint{1.450196in}{1.969499in}}{\pgfqpoint{1.444372in}{1.963675in}}%
\pgfpathcurveto{\pgfqpoint{1.438548in}{1.957851in}}{\pgfqpoint{1.435275in}{1.949951in}}{\pgfqpoint{1.435275in}{1.941715in}}%
\pgfpathcurveto{\pgfqpoint{1.435275in}{1.933478in}}{\pgfqpoint{1.438548in}{1.925578in}}{\pgfqpoint{1.444372in}{1.919754in}}%
\pgfpathcurveto{\pgfqpoint{1.450196in}{1.913930in}}{\pgfqpoint{1.458096in}{1.910658in}}{\pgfqpoint{1.466332in}{1.910658in}}%
\pgfpathclose%
\pgfusepath{stroke,fill}%
\end{pgfscope}%
\begin{pgfscope}%
\pgfpathrectangle{\pgfqpoint{0.100000in}{0.212622in}}{\pgfqpoint{3.696000in}{3.696000in}}%
\pgfusepath{clip}%
\pgfsetbuttcap%
\pgfsetroundjoin%
\definecolor{currentfill}{rgb}{0.121569,0.466667,0.705882}%
\pgfsetfillcolor{currentfill}%
\pgfsetfillopacity{0.928970}%
\pgfsetlinewidth{1.003750pt}%
\definecolor{currentstroke}{rgb}{0.121569,0.466667,0.705882}%
\pgfsetstrokecolor{currentstroke}%
\pgfsetstrokeopacity{0.928970}%
\pgfsetdash{}{0pt}%
\pgfpathmoveto{\pgfqpoint{1.827235in}{2.082467in}}%
\pgfpathcurveto{\pgfqpoint{1.835471in}{2.082467in}}{\pgfqpoint{1.843371in}{2.085739in}}{\pgfqpoint{1.849195in}{2.091563in}}%
\pgfpathcurveto{\pgfqpoint{1.855019in}{2.097387in}}{\pgfqpoint{1.858291in}{2.105287in}}{\pgfqpoint{1.858291in}{2.113524in}}%
\pgfpathcurveto{\pgfqpoint{1.858291in}{2.121760in}}{\pgfqpoint{1.855019in}{2.129660in}}{\pgfqpoint{1.849195in}{2.135484in}}%
\pgfpathcurveto{\pgfqpoint{1.843371in}{2.141308in}}{\pgfqpoint{1.835471in}{2.144580in}}{\pgfqpoint{1.827235in}{2.144580in}}%
\pgfpathcurveto{\pgfqpoint{1.818999in}{2.144580in}}{\pgfqpoint{1.811099in}{2.141308in}}{\pgfqpoint{1.805275in}{2.135484in}}%
\pgfpathcurveto{\pgfqpoint{1.799451in}{2.129660in}}{\pgfqpoint{1.796178in}{2.121760in}}{\pgfqpoint{1.796178in}{2.113524in}}%
\pgfpathcurveto{\pgfqpoint{1.796178in}{2.105287in}}{\pgfqpoint{1.799451in}{2.097387in}}{\pgfqpoint{1.805275in}{2.091563in}}%
\pgfpathcurveto{\pgfqpoint{1.811099in}{2.085739in}}{\pgfqpoint{1.818999in}{2.082467in}}{\pgfqpoint{1.827235in}{2.082467in}}%
\pgfpathclose%
\pgfusepath{stroke,fill}%
\end{pgfscope}%
\begin{pgfscope}%
\pgfpathrectangle{\pgfqpoint{0.100000in}{0.212622in}}{\pgfqpoint{3.696000in}{3.696000in}}%
\pgfusepath{clip}%
\pgfsetbuttcap%
\pgfsetroundjoin%
\definecolor{currentfill}{rgb}{0.121569,0.466667,0.705882}%
\pgfsetfillcolor{currentfill}%
\pgfsetfillopacity{0.929801}%
\pgfsetlinewidth{1.003750pt}%
\definecolor{currentstroke}{rgb}{0.121569,0.466667,0.705882}%
\pgfsetstrokecolor{currentstroke}%
\pgfsetstrokeopacity{0.929801}%
\pgfsetdash{}{0pt}%
\pgfpathmoveto{\pgfqpoint{2.214114in}{2.226399in}}%
\pgfpathcurveto{\pgfqpoint{2.222350in}{2.226399in}}{\pgfqpoint{2.230250in}{2.229671in}}{\pgfqpoint{2.236074in}{2.235495in}}%
\pgfpathcurveto{\pgfqpoint{2.241898in}{2.241319in}}{\pgfqpoint{2.245171in}{2.249219in}}{\pgfqpoint{2.245171in}{2.257455in}}%
\pgfpathcurveto{\pgfqpoint{2.245171in}{2.265691in}}{\pgfqpoint{2.241898in}{2.273591in}}{\pgfqpoint{2.236074in}{2.279415in}}%
\pgfpathcurveto{\pgfqpoint{2.230250in}{2.285239in}}{\pgfqpoint{2.222350in}{2.288512in}}{\pgfqpoint{2.214114in}{2.288512in}}%
\pgfpathcurveto{\pgfqpoint{2.205878in}{2.288512in}}{\pgfqpoint{2.197978in}{2.285239in}}{\pgfqpoint{2.192154in}{2.279415in}}%
\pgfpathcurveto{\pgfqpoint{2.186330in}{2.273591in}}{\pgfqpoint{2.183058in}{2.265691in}}{\pgfqpoint{2.183058in}{2.257455in}}%
\pgfpathcurveto{\pgfqpoint{2.183058in}{2.249219in}}{\pgfqpoint{2.186330in}{2.241319in}}{\pgfqpoint{2.192154in}{2.235495in}}%
\pgfpathcurveto{\pgfqpoint{2.197978in}{2.229671in}}{\pgfqpoint{2.205878in}{2.226399in}}{\pgfqpoint{2.214114in}{2.226399in}}%
\pgfpathclose%
\pgfusepath{stroke,fill}%
\end{pgfscope}%
\begin{pgfscope}%
\pgfpathrectangle{\pgfqpoint{0.100000in}{0.212622in}}{\pgfqpoint{3.696000in}{3.696000in}}%
\pgfusepath{clip}%
\pgfsetbuttcap%
\pgfsetroundjoin%
\definecolor{currentfill}{rgb}{0.121569,0.466667,0.705882}%
\pgfsetfillcolor{currentfill}%
\pgfsetfillopacity{0.930035}%
\pgfsetlinewidth{1.003750pt}%
\definecolor{currentstroke}{rgb}{0.121569,0.466667,0.705882}%
\pgfsetstrokecolor{currentstroke}%
\pgfsetstrokeopacity{0.930035}%
\pgfsetdash{}{0pt}%
\pgfpathmoveto{\pgfqpoint{1.286489in}{1.787618in}}%
\pgfpathcurveto{\pgfqpoint{1.294725in}{1.787618in}}{\pgfqpoint{1.302625in}{1.790890in}}{\pgfqpoint{1.308449in}{1.796714in}}%
\pgfpathcurveto{\pgfqpoint{1.314273in}{1.802538in}}{\pgfqpoint{1.317545in}{1.810438in}}{\pgfqpoint{1.317545in}{1.818675in}}%
\pgfpathcurveto{\pgfqpoint{1.317545in}{1.826911in}}{\pgfqpoint{1.314273in}{1.834811in}}{\pgfqpoint{1.308449in}{1.840635in}}%
\pgfpathcurveto{\pgfqpoint{1.302625in}{1.846459in}}{\pgfqpoint{1.294725in}{1.849731in}}{\pgfqpoint{1.286489in}{1.849731in}}%
\pgfpathcurveto{\pgfqpoint{1.278252in}{1.849731in}}{\pgfqpoint{1.270352in}{1.846459in}}{\pgfqpoint{1.264528in}{1.840635in}}%
\pgfpathcurveto{\pgfqpoint{1.258704in}{1.834811in}}{\pgfqpoint{1.255432in}{1.826911in}}{\pgfqpoint{1.255432in}{1.818675in}}%
\pgfpathcurveto{\pgfqpoint{1.255432in}{1.810438in}}{\pgfqpoint{1.258704in}{1.802538in}}{\pgfqpoint{1.264528in}{1.796714in}}%
\pgfpathcurveto{\pgfqpoint{1.270352in}{1.790890in}}{\pgfqpoint{1.278252in}{1.787618in}}{\pgfqpoint{1.286489in}{1.787618in}}%
\pgfpathclose%
\pgfusepath{stroke,fill}%
\end{pgfscope}%
\begin{pgfscope}%
\pgfpathrectangle{\pgfqpoint{0.100000in}{0.212622in}}{\pgfqpoint{3.696000in}{3.696000in}}%
\pgfusepath{clip}%
\pgfsetbuttcap%
\pgfsetroundjoin%
\definecolor{currentfill}{rgb}{0.121569,0.466667,0.705882}%
\pgfsetfillcolor{currentfill}%
\pgfsetfillopacity{0.932024}%
\pgfsetlinewidth{1.003750pt}%
\definecolor{currentstroke}{rgb}{0.121569,0.466667,0.705882}%
\pgfsetstrokecolor{currentstroke}%
\pgfsetstrokeopacity{0.932024}%
\pgfsetdash{}{0pt}%
\pgfpathmoveto{\pgfqpoint{1.832571in}{2.088577in}}%
\pgfpathcurveto{\pgfqpoint{1.840807in}{2.088577in}}{\pgfqpoint{1.848707in}{2.091850in}}{\pgfqpoint{1.854531in}{2.097673in}}%
\pgfpathcurveto{\pgfqpoint{1.860355in}{2.103497in}}{\pgfqpoint{1.863627in}{2.111397in}}{\pgfqpoint{1.863627in}{2.119634in}}%
\pgfpathcurveto{\pgfqpoint{1.863627in}{2.127870in}}{\pgfqpoint{1.860355in}{2.135770in}}{\pgfqpoint{1.854531in}{2.141594in}}%
\pgfpathcurveto{\pgfqpoint{1.848707in}{2.147418in}}{\pgfqpoint{1.840807in}{2.150690in}}{\pgfqpoint{1.832571in}{2.150690in}}%
\pgfpathcurveto{\pgfqpoint{1.824335in}{2.150690in}}{\pgfqpoint{1.816435in}{2.147418in}}{\pgfqpoint{1.810611in}{2.141594in}}%
\pgfpathcurveto{\pgfqpoint{1.804787in}{2.135770in}}{\pgfqpoint{1.801515in}{2.127870in}}{\pgfqpoint{1.801515in}{2.119634in}}%
\pgfpathcurveto{\pgfqpoint{1.801515in}{2.111397in}}{\pgfqpoint{1.804787in}{2.103497in}}{\pgfqpoint{1.810611in}{2.097673in}}%
\pgfpathcurveto{\pgfqpoint{1.816435in}{2.091850in}}{\pgfqpoint{1.824335in}{2.088577in}}{\pgfqpoint{1.832571in}{2.088577in}}%
\pgfpathclose%
\pgfusepath{stroke,fill}%
\end{pgfscope}%
\begin{pgfscope}%
\pgfpathrectangle{\pgfqpoint{0.100000in}{0.212622in}}{\pgfqpoint{3.696000in}{3.696000in}}%
\pgfusepath{clip}%
\pgfsetbuttcap%
\pgfsetroundjoin%
\definecolor{currentfill}{rgb}{0.121569,0.466667,0.705882}%
\pgfsetfillcolor{currentfill}%
\pgfsetfillopacity{0.932278}%
\pgfsetlinewidth{1.003750pt}%
\definecolor{currentstroke}{rgb}{0.121569,0.466667,0.705882}%
\pgfsetstrokecolor{currentstroke}%
\pgfsetstrokeopacity{0.932278}%
\pgfsetdash{}{0pt}%
\pgfpathmoveto{\pgfqpoint{1.287110in}{1.790332in}}%
\pgfpathcurveto{\pgfqpoint{1.295347in}{1.790332in}}{\pgfqpoint{1.303247in}{1.793604in}}{\pgfqpoint{1.309071in}{1.799428in}}%
\pgfpathcurveto{\pgfqpoint{1.314894in}{1.805252in}}{\pgfqpoint{1.318167in}{1.813152in}}{\pgfqpoint{1.318167in}{1.821388in}}%
\pgfpathcurveto{\pgfqpoint{1.318167in}{1.829624in}}{\pgfqpoint{1.314894in}{1.837524in}}{\pgfqpoint{1.309071in}{1.843348in}}%
\pgfpathcurveto{\pgfqpoint{1.303247in}{1.849172in}}{\pgfqpoint{1.295347in}{1.852445in}}{\pgfqpoint{1.287110in}{1.852445in}}%
\pgfpathcurveto{\pgfqpoint{1.278874in}{1.852445in}}{\pgfqpoint{1.270974in}{1.849172in}}{\pgfqpoint{1.265150in}{1.843348in}}%
\pgfpathcurveto{\pgfqpoint{1.259326in}{1.837524in}}{\pgfqpoint{1.256054in}{1.829624in}}{\pgfqpoint{1.256054in}{1.821388in}}%
\pgfpathcurveto{\pgfqpoint{1.256054in}{1.813152in}}{\pgfqpoint{1.259326in}{1.805252in}}{\pgfqpoint{1.265150in}{1.799428in}}%
\pgfpathcurveto{\pgfqpoint{1.270974in}{1.793604in}}{\pgfqpoint{1.278874in}{1.790332in}}{\pgfqpoint{1.287110in}{1.790332in}}%
\pgfpathclose%
\pgfusepath{stroke,fill}%
\end{pgfscope}%
\begin{pgfscope}%
\pgfpathrectangle{\pgfqpoint{0.100000in}{0.212622in}}{\pgfqpoint{3.696000in}{3.696000in}}%
\pgfusepath{clip}%
\pgfsetbuttcap%
\pgfsetroundjoin%
\definecolor{currentfill}{rgb}{0.121569,0.466667,0.705882}%
\pgfsetfillcolor{currentfill}%
\pgfsetfillopacity{0.932471}%
\pgfsetlinewidth{1.003750pt}%
\definecolor{currentstroke}{rgb}{0.121569,0.466667,0.705882}%
\pgfsetstrokecolor{currentstroke}%
\pgfsetstrokeopacity{0.932471}%
\pgfsetdash{}{0pt}%
\pgfpathmoveto{\pgfqpoint{1.272767in}{1.778825in}}%
\pgfpathcurveto{\pgfqpoint{1.281003in}{1.778825in}}{\pgfqpoint{1.288903in}{1.782097in}}{\pgfqpoint{1.294727in}{1.787921in}}%
\pgfpathcurveto{\pgfqpoint{1.300551in}{1.793745in}}{\pgfqpoint{1.303823in}{1.801645in}}{\pgfqpoint{1.303823in}{1.809882in}}%
\pgfpathcurveto{\pgfqpoint{1.303823in}{1.818118in}}{\pgfqpoint{1.300551in}{1.826018in}}{\pgfqpoint{1.294727in}{1.831842in}}%
\pgfpathcurveto{\pgfqpoint{1.288903in}{1.837666in}}{\pgfqpoint{1.281003in}{1.840938in}}{\pgfqpoint{1.272767in}{1.840938in}}%
\pgfpathcurveto{\pgfqpoint{1.264530in}{1.840938in}}{\pgfqpoint{1.256630in}{1.837666in}}{\pgfqpoint{1.250806in}{1.831842in}}%
\pgfpathcurveto{\pgfqpoint{1.244983in}{1.826018in}}{\pgfqpoint{1.241710in}{1.818118in}}{\pgfqpoint{1.241710in}{1.809882in}}%
\pgfpathcurveto{\pgfqpoint{1.241710in}{1.801645in}}{\pgfqpoint{1.244983in}{1.793745in}}{\pgfqpoint{1.250806in}{1.787921in}}%
\pgfpathcurveto{\pgfqpoint{1.256630in}{1.782097in}}{\pgfqpoint{1.264530in}{1.778825in}}{\pgfqpoint{1.272767in}{1.778825in}}%
\pgfpathclose%
\pgfusepath{stroke,fill}%
\end{pgfscope}%
\begin{pgfscope}%
\pgfpathrectangle{\pgfqpoint{0.100000in}{0.212622in}}{\pgfqpoint{3.696000in}{3.696000in}}%
\pgfusepath{clip}%
\pgfsetbuttcap%
\pgfsetroundjoin%
\definecolor{currentfill}{rgb}{0.121569,0.466667,0.705882}%
\pgfsetfillcolor{currentfill}%
\pgfsetfillopacity{0.932968}%
\pgfsetlinewidth{1.003750pt}%
\definecolor{currentstroke}{rgb}{0.121569,0.466667,0.705882}%
\pgfsetstrokecolor{currentstroke}%
\pgfsetstrokeopacity{0.932968}%
\pgfsetdash{}{0pt}%
\pgfpathmoveto{\pgfqpoint{1.810265in}{2.073129in}}%
\pgfpathcurveto{\pgfqpoint{1.818501in}{2.073129in}}{\pgfqpoint{1.826401in}{2.076401in}}{\pgfqpoint{1.832225in}{2.082225in}}%
\pgfpathcurveto{\pgfqpoint{1.838049in}{2.088049in}}{\pgfqpoint{1.841321in}{2.095949in}}{\pgfqpoint{1.841321in}{2.104185in}}%
\pgfpathcurveto{\pgfqpoint{1.841321in}{2.112421in}}{\pgfqpoint{1.838049in}{2.120321in}}{\pgfqpoint{1.832225in}{2.126145in}}%
\pgfpathcurveto{\pgfqpoint{1.826401in}{2.131969in}}{\pgfqpoint{1.818501in}{2.135242in}}{\pgfqpoint{1.810265in}{2.135242in}}%
\pgfpathcurveto{\pgfqpoint{1.802028in}{2.135242in}}{\pgfqpoint{1.794128in}{2.131969in}}{\pgfqpoint{1.788304in}{2.126145in}}%
\pgfpathcurveto{\pgfqpoint{1.782481in}{2.120321in}}{\pgfqpoint{1.779208in}{2.112421in}}{\pgfqpoint{1.779208in}{2.104185in}}%
\pgfpathcurveto{\pgfqpoint{1.779208in}{2.095949in}}{\pgfqpoint{1.782481in}{2.088049in}}{\pgfqpoint{1.788304in}{2.082225in}}%
\pgfpathcurveto{\pgfqpoint{1.794128in}{2.076401in}}{\pgfqpoint{1.802028in}{2.073129in}}{\pgfqpoint{1.810265in}{2.073129in}}%
\pgfpathclose%
\pgfusepath{stroke,fill}%
\end{pgfscope}%
\begin{pgfscope}%
\pgfpathrectangle{\pgfqpoint{0.100000in}{0.212622in}}{\pgfqpoint{3.696000in}{3.696000in}}%
\pgfusepath{clip}%
\pgfsetbuttcap%
\pgfsetroundjoin%
\definecolor{currentfill}{rgb}{0.121569,0.466667,0.705882}%
\pgfsetfillcolor{currentfill}%
\pgfsetfillopacity{0.933380}%
\pgfsetlinewidth{1.003750pt}%
\definecolor{currentstroke}{rgb}{0.121569,0.466667,0.705882}%
\pgfsetstrokecolor{currentstroke}%
\pgfsetstrokeopacity{0.933380}%
\pgfsetdash{}{0pt}%
\pgfpathmoveto{\pgfqpoint{1.259781in}{1.772590in}}%
\pgfpathcurveto{\pgfqpoint{1.268018in}{1.772590in}}{\pgfqpoint{1.275918in}{1.775863in}}{\pgfqpoint{1.281742in}{1.781687in}}%
\pgfpathcurveto{\pgfqpoint{1.287566in}{1.787510in}}{\pgfqpoint{1.290838in}{1.795410in}}{\pgfqpoint{1.290838in}{1.803647in}}%
\pgfpathcurveto{\pgfqpoint{1.290838in}{1.811883in}}{\pgfqpoint{1.287566in}{1.819783in}}{\pgfqpoint{1.281742in}{1.825607in}}%
\pgfpathcurveto{\pgfqpoint{1.275918in}{1.831431in}}{\pgfqpoint{1.268018in}{1.834703in}}{\pgfqpoint{1.259781in}{1.834703in}}%
\pgfpathcurveto{\pgfqpoint{1.251545in}{1.834703in}}{\pgfqpoint{1.243645in}{1.831431in}}{\pgfqpoint{1.237821in}{1.825607in}}%
\pgfpathcurveto{\pgfqpoint{1.231997in}{1.819783in}}{\pgfqpoint{1.228725in}{1.811883in}}{\pgfqpoint{1.228725in}{1.803647in}}%
\pgfpathcurveto{\pgfqpoint{1.228725in}{1.795410in}}{\pgfqpoint{1.231997in}{1.787510in}}{\pgfqpoint{1.237821in}{1.781687in}}%
\pgfpathcurveto{\pgfqpoint{1.243645in}{1.775863in}}{\pgfqpoint{1.251545in}{1.772590in}}{\pgfqpoint{1.259781in}{1.772590in}}%
\pgfpathclose%
\pgfusepath{stroke,fill}%
\end{pgfscope}%
\begin{pgfscope}%
\pgfpathrectangle{\pgfqpoint{0.100000in}{0.212622in}}{\pgfqpoint{3.696000in}{3.696000in}}%
\pgfusepath{clip}%
\pgfsetbuttcap%
\pgfsetroundjoin%
\definecolor{currentfill}{rgb}{0.121569,0.466667,0.705882}%
\pgfsetfillcolor{currentfill}%
\pgfsetfillopacity{0.934014}%
\pgfsetlinewidth{1.003750pt}%
\definecolor{currentstroke}{rgb}{0.121569,0.466667,0.705882}%
\pgfsetstrokecolor{currentstroke}%
\pgfsetstrokeopacity{0.934014}%
\pgfsetdash{}{0pt}%
\pgfpathmoveto{\pgfqpoint{1.251033in}{1.770372in}}%
\pgfpathcurveto{\pgfqpoint{1.259269in}{1.770372in}}{\pgfqpoint{1.267169in}{1.773645in}}{\pgfqpoint{1.272993in}{1.779469in}}%
\pgfpathcurveto{\pgfqpoint{1.278817in}{1.785293in}}{\pgfqpoint{1.282090in}{1.793193in}}{\pgfqpoint{1.282090in}{1.801429in}}%
\pgfpathcurveto{\pgfqpoint{1.282090in}{1.809665in}}{\pgfqpoint{1.278817in}{1.817565in}}{\pgfqpoint{1.272993in}{1.823389in}}%
\pgfpathcurveto{\pgfqpoint{1.267169in}{1.829213in}}{\pgfqpoint{1.259269in}{1.832485in}}{\pgfqpoint{1.251033in}{1.832485in}}%
\pgfpathcurveto{\pgfqpoint{1.242797in}{1.832485in}}{\pgfqpoint{1.234897in}{1.829213in}}{\pgfqpoint{1.229073in}{1.823389in}}%
\pgfpathcurveto{\pgfqpoint{1.223249in}{1.817565in}}{\pgfqpoint{1.219977in}{1.809665in}}{\pgfqpoint{1.219977in}{1.801429in}}%
\pgfpathcurveto{\pgfqpoint{1.219977in}{1.793193in}}{\pgfqpoint{1.223249in}{1.785293in}}{\pgfqpoint{1.229073in}{1.779469in}}%
\pgfpathcurveto{\pgfqpoint{1.234897in}{1.773645in}}{\pgfqpoint{1.242797in}{1.770372in}}{\pgfqpoint{1.251033in}{1.770372in}}%
\pgfpathclose%
\pgfusepath{stroke,fill}%
\end{pgfscope}%
\begin{pgfscope}%
\pgfpathrectangle{\pgfqpoint{0.100000in}{0.212622in}}{\pgfqpoint{3.696000in}{3.696000in}}%
\pgfusepath{clip}%
\pgfsetbuttcap%
\pgfsetroundjoin%
\definecolor{currentfill}{rgb}{0.121569,0.466667,0.705882}%
\pgfsetfillcolor{currentfill}%
\pgfsetfillopacity{0.934637}%
\pgfsetlinewidth{1.003750pt}%
\definecolor{currentstroke}{rgb}{0.121569,0.466667,0.705882}%
\pgfsetstrokecolor{currentstroke}%
\pgfsetstrokeopacity{0.934637}%
\pgfsetdash{}{0pt}%
\pgfpathmoveto{\pgfqpoint{1.294995in}{1.795617in}}%
\pgfpathcurveto{\pgfqpoint{1.303231in}{1.795617in}}{\pgfqpoint{1.311131in}{1.798889in}}{\pgfqpoint{1.316955in}{1.804713in}}%
\pgfpathcurveto{\pgfqpoint{1.322779in}{1.810537in}}{\pgfqpoint{1.326051in}{1.818437in}}{\pgfqpoint{1.326051in}{1.826674in}}%
\pgfpathcurveto{\pgfqpoint{1.326051in}{1.834910in}}{\pgfqpoint{1.322779in}{1.842810in}}{\pgfqpoint{1.316955in}{1.848634in}}%
\pgfpathcurveto{\pgfqpoint{1.311131in}{1.854458in}}{\pgfqpoint{1.303231in}{1.857730in}}{\pgfqpoint{1.294995in}{1.857730in}}%
\pgfpathcurveto{\pgfqpoint{1.286758in}{1.857730in}}{\pgfqpoint{1.278858in}{1.854458in}}{\pgfqpoint{1.273034in}{1.848634in}}%
\pgfpathcurveto{\pgfqpoint{1.267211in}{1.842810in}}{\pgfqpoint{1.263938in}{1.834910in}}{\pgfqpoint{1.263938in}{1.826674in}}%
\pgfpathcurveto{\pgfqpoint{1.263938in}{1.818437in}}{\pgfqpoint{1.267211in}{1.810537in}}{\pgfqpoint{1.273034in}{1.804713in}}%
\pgfpathcurveto{\pgfqpoint{1.278858in}{1.798889in}}{\pgfqpoint{1.286758in}{1.795617in}}{\pgfqpoint{1.294995in}{1.795617in}}%
\pgfpathclose%
\pgfusepath{stroke,fill}%
\end{pgfscope}%
\begin{pgfscope}%
\pgfpathrectangle{\pgfqpoint{0.100000in}{0.212622in}}{\pgfqpoint{3.696000in}{3.696000in}}%
\pgfusepath{clip}%
\pgfsetbuttcap%
\pgfsetroundjoin%
\definecolor{currentfill}{rgb}{0.121569,0.466667,0.705882}%
\pgfsetfillcolor{currentfill}%
\pgfsetfillopacity{0.935541}%
\pgfsetlinewidth{1.003750pt}%
\definecolor{currentstroke}{rgb}{0.121569,0.466667,0.705882}%
\pgfsetstrokecolor{currentstroke}%
\pgfsetstrokeopacity{0.935541}%
\pgfsetdash{}{0pt}%
\pgfpathmoveto{\pgfqpoint{2.188227in}{2.206271in}}%
\pgfpathcurveto{\pgfqpoint{2.196463in}{2.206271in}}{\pgfqpoint{2.204363in}{2.209543in}}{\pgfqpoint{2.210187in}{2.215367in}}%
\pgfpathcurveto{\pgfqpoint{2.216011in}{2.221191in}}{\pgfqpoint{2.219284in}{2.229091in}}{\pgfqpoint{2.219284in}{2.237328in}}%
\pgfpathcurveto{\pgfqpoint{2.219284in}{2.245564in}}{\pgfqpoint{2.216011in}{2.253464in}}{\pgfqpoint{2.210187in}{2.259288in}}%
\pgfpathcurveto{\pgfqpoint{2.204363in}{2.265112in}}{\pgfqpoint{2.196463in}{2.268384in}}{\pgfqpoint{2.188227in}{2.268384in}}%
\pgfpathcurveto{\pgfqpoint{2.179991in}{2.268384in}}{\pgfqpoint{2.172091in}{2.265112in}}{\pgfqpoint{2.166267in}{2.259288in}}%
\pgfpathcurveto{\pgfqpoint{2.160443in}{2.253464in}}{\pgfqpoint{2.157171in}{2.245564in}}{\pgfqpoint{2.157171in}{2.237328in}}%
\pgfpathcurveto{\pgfqpoint{2.157171in}{2.229091in}}{\pgfqpoint{2.160443in}{2.221191in}}{\pgfqpoint{2.166267in}{2.215367in}}%
\pgfpathcurveto{\pgfqpoint{2.172091in}{2.209543in}}{\pgfqpoint{2.179991in}{2.206271in}}{\pgfqpoint{2.188227in}{2.206271in}}%
\pgfpathclose%
\pgfusepath{stroke,fill}%
\end{pgfscope}%
\begin{pgfscope}%
\pgfpathrectangle{\pgfqpoint{0.100000in}{0.212622in}}{\pgfqpoint{3.696000in}{3.696000in}}%
\pgfusepath{clip}%
\pgfsetbuttcap%
\pgfsetroundjoin%
\definecolor{currentfill}{rgb}{0.121569,0.466667,0.705882}%
\pgfsetfillcolor{currentfill}%
\pgfsetfillopacity{0.935764}%
\pgfsetlinewidth{1.003750pt}%
\definecolor{currentstroke}{rgb}{0.121569,0.466667,0.705882}%
\pgfsetstrokecolor{currentstroke}%
\pgfsetstrokeopacity{0.935764}%
\pgfsetdash{}{0pt}%
\pgfpathmoveto{\pgfqpoint{1.401582in}{1.867898in}}%
\pgfpathcurveto{\pgfqpoint{1.409818in}{1.867898in}}{\pgfqpoint{1.417718in}{1.871170in}}{\pgfqpoint{1.423542in}{1.876994in}}%
\pgfpathcurveto{\pgfqpoint{1.429366in}{1.882818in}}{\pgfqpoint{1.432638in}{1.890718in}}{\pgfqpoint{1.432638in}{1.898954in}}%
\pgfpathcurveto{\pgfqpoint{1.432638in}{1.907190in}}{\pgfqpoint{1.429366in}{1.915090in}}{\pgfqpoint{1.423542in}{1.920914in}}%
\pgfpathcurveto{\pgfqpoint{1.417718in}{1.926738in}}{\pgfqpoint{1.409818in}{1.930011in}}{\pgfqpoint{1.401582in}{1.930011in}}%
\pgfpathcurveto{\pgfqpoint{1.393345in}{1.930011in}}{\pgfqpoint{1.385445in}{1.926738in}}{\pgfqpoint{1.379621in}{1.920914in}}%
\pgfpathcurveto{\pgfqpoint{1.373797in}{1.915090in}}{\pgfqpoint{1.370525in}{1.907190in}}{\pgfqpoint{1.370525in}{1.898954in}}%
\pgfpathcurveto{\pgfqpoint{1.370525in}{1.890718in}}{\pgfqpoint{1.373797in}{1.882818in}}{\pgfqpoint{1.379621in}{1.876994in}}%
\pgfpathcurveto{\pgfqpoint{1.385445in}{1.871170in}}{\pgfqpoint{1.393345in}{1.867898in}}{\pgfqpoint{1.401582in}{1.867898in}}%
\pgfpathclose%
\pgfusepath{stroke,fill}%
\end{pgfscope}%
\begin{pgfscope}%
\pgfpathrectangle{\pgfqpoint{0.100000in}{0.212622in}}{\pgfqpoint{3.696000in}{3.696000in}}%
\pgfusepath{clip}%
\pgfsetbuttcap%
\pgfsetroundjoin%
\definecolor{currentfill}{rgb}{0.121569,0.466667,0.705882}%
\pgfsetfillcolor{currentfill}%
\pgfsetfillopacity{0.935982}%
\pgfsetlinewidth{1.003750pt}%
\definecolor{currentstroke}{rgb}{0.121569,0.466667,0.705882}%
\pgfsetstrokecolor{currentstroke}%
\pgfsetstrokeopacity{0.935982}%
\pgfsetdash{}{0pt}%
\pgfpathmoveto{\pgfqpoint{1.584641in}{1.973705in}}%
\pgfpathcurveto{\pgfqpoint{1.592877in}{1.973705in}}{\pgfqpoint{1.600777in}{1.976977in}}{\pgfqpoint{1.606601in}{1.982801in}}%
\pgfpathcurveto{\pgfqpoint{1.612425in}{1.988625in}}{\pgfqpoint{1.615697in}{1.996525in}}{\pgfqpoint{1.615697in}{2.004761in}}%
\pgfpathcurveto{\pgfqpoint{1.615697in}{2.012997in}}{\pgfqpoint{1.612425in}{2.020897in}}{\pgfqpoint{1.606601in}{2.026721in}}%
\pgfpathcurveto{\pgfqpoint{1.600777in}{2.032545in}}{\pgfqpoint{1.592877in}{2.035818in}}{\pgfqpoint{1.584641in}{2.035818in}}%
\pgfpathcurveto{\pgfqpoint{1.576405in}{2.035818in}}{\pgfqpoint{1.568504in}{2.032545in}}{\pgfqpoint{1.562681in}{2.026721in}}%
\pgfpathcurveto{\pgfqpoint{1.556857in}{2.020897in}}{\pgfqpoint{1.553584in}{2.012997in}}{\pgfqpoint{1.553584in}{2.004761in}}%
\pgfpathcurveto{\pgfqpoint{1.553584in}{1.996525in}}{\pgfqpoint{1.556857in}{1.988625in}}{\pgfqpoint{1.562681in}{1.982801in}}%
\pgfpathcurveto{\pgfqpoint{1.568504in}{1.976977in}}{\pgfqpoint{1.576405in}{1.973705in}}{\pgfqpoint{1.584641in}{1.973705in}}%
\pgfpathclose%
\pgfusepath{stroke,fill}%
\end{pgfscope}%
\begin{pgfscope}%
\pgfpathrectangle{\pgfqpoint{0.100000in}{0.212622in}}{\pgfqpoint{3.696000in}{3.696000in}}%
\pgfusepath{clip}%
\pgfsetbuttcap%
\pgfsetroundjoin%
\definecolor{currentfill}{rgb}{0.121569,0.466667,0.705882}%
\pgfsetfillcolor{currentfill}%
\pgfsetfillopacity{0.936729}%
\pgfsetlinewidth{1.003750pt}%
\definecolor{currentstroke}{rgb}{0.121569,0.466667,0.705882}%
\pgfsetstrokecolor{currentstroke}%
\pgfsetstrokeopacity{0.936729}%
\pgfsetdash{}{0pt}%
\pgfpathmoveto{\pgfqpoint{1.256214in}{1.767406in}}%
\pgfpathcurveto{\pgfqpoint{1.264450in}{1.767406in}}{\pgfqpoint{1.272350in}{1.770678in}}{\pgfqpoint{1.278174in}{1.776502in}}%
\pgfpathcurveto{\pgfqpoint{1.283998in}{1.782326in}}{\pgfqpoint{1.287270in}{1.790226in}}{\pgfqpoint{1.287270in}{1.798462in}}%
\pgfpathcurveto{\pgfqpoint{1.287270in}{1.806698in}}{\pgfqpoint{1.283998in}{1.814598in}}{\pgfqpoint{1.278174in}{1.820422in}}%
\pgfpathcurveto{\pgfqpoint{1.272350in}{1.826246in}}{\pgfqpoint{1.264450in}{1.829519in}}{\pgfqpoint{1.256214in}{1.829519in}}%
\pgfpathcurveto{\pgfqpoint{1.247977in}{1.829519in}}{\pgfqpoint{1.240077in}{1.826246in}}{\pgfqpoint{1.234253in}{1.820422in}}%
\pgfpathcurveto{\pgfqpoint{1.228429in}{1.814598in}}{\pgfqpoint{1.225157in}{1.806698in}}{\pgfqpoint{1.225157in}{1.798462in}}%
\pgfpathcurveto{\pgfqpoint{1.225157in}{1.790226in}}{\pgfqpoint{1.228429in}{1.782326in}}{\pgfqpoint{1.234253in}{1.776502in}}%
\pgfpathcurveto{\pgfqpoint{1.240077in}{1.770678in}}{\pgfqpoint{1.247977in}{1.767406in}}{\pgfqpoint{1.256214in}{1.767406in}}%
\pgfpathclose%
\pgfusepath{stroke,fill}%
\end{pgfscope}%
\begin{pgfscope}%
\pgfpathrectangle{\pgfqpoint{0.100000in}{0.212622in}}{\pgfqpoint{3.696000in}{3.696000in}}%
\pgfusepath{clip}%
\pgfsetbuttcap%
\pgfsetroundjoin%
\definecolor{currentfill}{rgb}{0.121569,0.466667,0.705882}%
\pgfsetfillcolor{currentfill}%
\pgfsetfillopacity{0.937473}%
\pgfsetlinewidth{1.003750pt}%
\definecolor{currentstroke}{rgb}{0.121569,0.466667,0.705882}%
\pgfsetstrokecolor{currentstroke}%
\pgfsetstrokeopacity{0.937473}%
\pgfsetdash{}{0pt}%
\pgfpathmoveto{\pgfqpoint{1.607779in}{1.979465in}}%
\pgfpathcurveto{\pgfqpoint{1.616015in}{1.979465in}}{\pgfqpoint{1.623915in}{1.982738in}}{\pgfqpoint{1.629739in}{1.988562in}}%
\pgfpathcurveto{\pgfqpoint{1.635563in}{1.994386in}}{\pgfqpoint{1.638835in}{2.002286in}}{\pgfqpoint{1.638835in}{2.010522in}}%
\pgfpathcurveto{\pgfqpoint{1.638835in}{2.018758in}}{\pgfqpoint{1.635563in}{2.026658in}}{\pgfqpoint{1.629739in}{2.032482in}}%
\pgfpathcurveto{\pgfqpoint{1.623915in}{2.038306in}}{\pgfqpoint{1.616015in}{2.041578in}}{\pgfqpoint{1.607779in}{2.041578in}}%
\pgfpathcurveto{\pgfqpoint{1.599543in}{2.041578in}}{\pgfqpoint{1.591643in}{2.038306in}}{\pgfqpoint{1.585819in}{2.032482in}}%
\pgfpathcurveto{\pgfqpoint{1.579995in}{2.026658in}}{\pgfqpoint{1.576722in}{2.018758in}}{\pgfqpoint{1.576722in}{2.010522in}}%
\pgfpathcurveto{\pgfqpoint{1.576722in}{2.002286in}}{\pgfqpoint{1.579995in}{1.994386in}}{\pgfqpoint{1.585819in}{1.988562in}}%
\pgfpathcurveto{\pgfqpoint{1.591643in}{1.982738in}}{\pgfqpoint{1.599543in}{1.979465in}}{\pgfqpoint{1.607779in}{1.979465in}}%
\pgfpathclose%
\pgfusepath{stroke,fill}%
\end{pgfscope}%
\begin{pgfscope}%
\pgfpathrectangle{\pgfqpoint{0.100000in}{0.212622in}}{\pgfqpoint{3.696000in}{3.696000in}}%
\pgfusepath{clip}%
\pgfsetbuttcap%
\pgfsetroundjoin%
\definecolor{currentfill}{rgb}{0.121569,0.466667,0.705882}%
\pgfsetfillcolor{currentfill}%
\pgfsetfillopacity{0.938144}%
\pgfsetlinewidth{1.003750pt}%
\definecolor{currentstroke}{rgb}{0.121569,0.466667,0.705882}%
\pgfsetstrokecolor{currentstroke}%
\pgfsetstrokeopacity{0.938144}%
\pgfsetdash{}{0pt}%
\pgfpathmoveto{\pgfqpoint{1.278178in}{1.779763in}}%
\pgfpathcurveto{\pgfqpoint{1.286414in}{1.779763in}}{\pgfqpoint{1.294314in}{1.783035in}}{\pgfqpoint{1.300138in}{1.788859in}}%
\pgfpathcurveto{\pgfqpoint{1.305962in}{1.794683in}}{\pgfqpoint{1.309234in}{1.802583in}}{\pgfqpoint{1.309234in}{1.810819in}}%
\pgfpathcurveto{\pgfqpoint{1.309234in}{1.819056in}}{\pgfqpoint{1.305962in}{1.826956in}}{\pgfqpoint{1.300138in}{1.832780in}}%
\pgfpathcurveto{\pgfqpoint{1.294314in}{1.838604in}}{\pgfqpoint{1.286414in}{1.841876in}}{\pgfqpoint{1.278178in}{1.841876in}}%
\pgfpathcurveto{\pgfqpoint{1.269941in}{1.841876in}}{\pgfqpoint{1.262041in}{1.838604in}}{\pgfqpoint{1.256217in}{1.832780in}}%
\pgfpathcurveto{\pgfqpoint{1.250394in}{1.826956in}}{\pgfqpoint{1.247121in}{1.819056in}}{\pgfqpoint{1.247121in}{1.810819in}}%
\pgfpathcurveto{\pgfqpoint{1.247121in}{1.802583in}}{\pgfqpoint{1.250394in}{1.794683in}}{\pgfqpoint{1.256217in}{1.788859in}}%
\pgfpathcurveto{\pgfqpoint{1.262041in}{1.783035in}}{\pgfqpoint{1.269941in}{1.779763in}}{\pgfqpoint{1.278178in}{1.779763in}}%
\pgfpathclose%
\pgfusepath{stroke,fill}%
\end{pgfscope}%
\begin{pgfscope}%
\pgfpathrectangle{\pgfqpoint{0.100000in}{0.212622in}}{\pgfqpoint{3.696000in}{3.696000in}}%
\pgfusepath{clip}%
\pgfsetbuttcap%
\pgfsetroundjoin%
\definecolor{currentfill}{rgb}{0.121569,0.466667,0.705882}%
\pgfsetfillcolor{currentfill}%
\pgfsetfillopacity{0.941201}%
\pgfsetlinewidth{1.003750pt}%
\definecolor{currentstroke}{rgb}{0.121569,0.466667,0.705882}%
\pgfsetstrokecolor{currentstroke}%
\pgfsetstrokeopacity{0.941201}%
\pgfsetdash{}{0pt}%
\pgfpathmoveto{\pgfqpoint{3.038951in}{2.604249in}}%
\pgfpathcurveto{\pgfqpoint{3.047187in}{2.604249in}}{\pgfqpoint{3.055087in}{2.607522in}}{\pgfqpoint{3.060911in}{2.613346in}}%
\pgfpathcurveto{\pgfqpoint{3.066735in}{2.619169in}}{\pgfqpoint{3.070008in}{2.627069in}}{\pgfqpoint{3.070008in}{2.635306in}}%
\pgfpathcurveto{\pgfqpoint{3.070008in}{2.643542in}}{\pgfqpoint{3.066735in}{2.651442in}}{\pgfqpoint{3.060911in}{2.657266in}}%
\pgfpathcurveto{\pgfqpoint{3.055087in}{2.663090in}}{\pgfqpoint{3.047187in}{2.666362in}}{\pgfqpoint{3.038951in}{2.666362in}}%
\pgfpathcurveto{\pgfqpoint{3.030715in}{2.666362in}}{\pgfqpoint{3.022815in}{2.663090in}}{\pgfqpoint{3.016991in}{2.657266in}}%
\pgfpathcurveto{\pgfqpoint{3.011167in}{2.651442in}}{\pgfqpoint{3.007895in}{2.643542in}}{\pgfqpoint{3.007895in}{2.635306in}}%
\pgfpathcurveto{\pgfqpoint{3.007895in}{2.627069in}}{\pgfqpoint{3.011167in}{2.619169in}}{\pgfqpoint{3.016991in}{2.613346in}}%
\pgfpathcurveto{\pgfqpoint{3.022815in}{2.607522in}}{\pgfqpoint{3.030715in}{2.604249in}}{\pgfqpoint{3.038951in}{2.604249in}}%
\pgfpathclose%
\pgfusepath{stroke,fill}%
\end{pgfscope}%
\begin{pgfscope}%
\pgfpathrectangle{\pgfqpoint{0.100000in}{0.212622in}}{\pgfqpoint{3.696000in}{3.696000in}}%
\pgfusepath{clip}%
\pgfsetbuttcap%
\pgfsetroundjoin%
\definecolor{currentfill}{rgb}{0.121569,0.466667,0.705882}%
\pgfsetfillcolor{currentfill}%
\pgfsetfillopacity{0.942051}%
\pgfsetlinewidth{1.003750pt}%
\definecolor{currentstroke}{rgb}{0.121569,0.466667,0.705882}%
\pgfsetstrokecolor{currentstroke}%
\pgfsetstrokeopacity{0.942051}%
\pgfsetdash{}{0pt}%
\pgfpathmoveto{\pgfqpoint{1.392071in}{1.849612in}}%
\pgfpathcurveto{\pgfqpoint{1.400307in}{1.849612in}}{\pgfqpoint{1.408207in}{1.852885in}}{\pgfqpoint{1.414031in}{1.858709in}}%
\pgfpathcurveto{\pgfqpoint{1.419855in}{1.864533in}}{\pgfqpoint{1.423127in}{1.872433in}}{\pgfqpoint{1.423127in}{1.880669in}}%
\pgfpathcurveto{\pgfqpoint{1.423127in}{1.888905in}}{\pgfqpoint{1.419855in}{1.896805in}}{\pgfqpoint{1.414031in}{1.902629in}}%
\pgfpathcurveto{\pgfqpoint{1.408207in}{1.908453in}}{\pgfqpoint{1.400307in}{1.911725in}}{\pgfqpoint{1.392071in}{1.911725in}}%
\pgfpathcurveto{\pgfqpoint{1.383835in}{1.911725in}}{\pgfqpoint{1.375935in}{1.908453in}}{\pgfqpoint{1.370111in}{1.902629in}}%
\pgfpathcurveto{\pgfqpoint{1.364287in}{1.896805in}}{\pgfqpoint{1.361014in}{1.888905in}}{\pgfqpoint{1.361014in}{1.880669in}}%
\pgfpathcurveto{\pgfqpoint{1.361014in}{1.872433in}}{\pgfqpoint{1.364287in}{1.864533in}}{\pgfqpoint{1.370111in}{1.858709in}}%
\pgfpathcurveto{\pgfqpoint{1.375935in}{1.852885in}}{\pgfqpoint{1.383835in}{1.849612in}}{\pgfqpoint{1.392071in}{1.849612in}}%
\pgfpathclose%
\pgfusepath{stroke,fill}%
\end{pgfscope}%
\begin{pgfscope}%
\pgfpathrectangle{\pgfqpoint{0.100000in}{0.212622in}}{\pgfqpoint{3.696000in}{3.696000in}}%
\pgfusepath{clip}%
\pgfsetbuttcap%
\pgfsetroundjoin%
\definecolor{currentfill}{rgb}{0.121569,0.466667,0.705882}%
\pgfsetfillcolor{currentfill}%
\pgfsetfillopacity{0.942684}%
\pgfsetlinewidth{1.003750pt}%
\definecolor{currentstroke}{rgb}{0.121569,0.466667,0.705882}%
\pgfsetstrokecolor{currentstroke}%
\pgfsetstrokeopacity{0.942684}%
\pgfsetdash{}{0pt}%
\pgfpathmoveto{\pgfqpoint{1.822039in}{2.078471in}}%
\pgfpathcurveto{\pgfqpoint{1.830275in}{2.078471in}}{\pgfqpoint{1.838175in}{2.081743in}}{\pgfqpoint{1.843999in}{2.087567in}}%
\pgfpathcurveto{\pgfqpoint{1.849823in}{2.093391in}}{\pgfqpoint{1.853095in}{2.101291in}}{\pgfqpoint{1.853095in}{2.109527in}}%
\pgfpathcurveto{\pgfqpoint{1.853095in}{2.117763in}}{\pgfqpoint{1.849823in}{2.125664in}}{\pgfqpoint{1.843999in}{2.131487in}}%
\pgfpathcurveto{\pgfqpoint{1.838175in}{2.137311in}}{\pgfqpoint{1.830275in}{2.140584in}}{\pgfqpoint{1.822039in}{2.140584in}}%
\pgfpathcurveto{\pgfqpoint{1.813802in}{2.140584in}}{\pgfqpoint{1.805902in}{2.137311in}}{\pgfqpoint{1.800078in}{2.131487in}}%
\pgfpathcurveto{\pgfqpoint{1.794255in}{2.125664in}}{\pgfqpoint{1.790982in}{2.117763in}}{\pgfqpoint{1.790982in}{2.109527in}}%
\pgfpathcurveto{\pgfqpoint{1.790982in}{2.101291in}}{\pgfqpoint{1.794255in}{2.093391in}}{\pgfqpoint{1.800078in}{2.087567in}}%
\pgfpathcurveto{\pgfqpoint{1.805902in}{2.081743in}}{\pgfqpoint{1.813802in}{2.078471in}}{\pgfqpoint{1.822039in}{2.078471in}}%
\pgfpathclose%
\pgfusepath{stroke,fill}%
\end{pgfscope}%
\begin{pgfscope}%
\pgfpathrectangle{\pgfqpoint{0.100000in}{0.212622in}}{\pgfqpoint{3.696000in}{3.696000in}}%
\pgfusepath{clip}%
\pgfsetbuttcap%
\pgfsetroundjoin%
\definecolor{currentfill}{rgb}{0.121569,0.466667,0.705882}%
\pgfsetfillcolor{currentfill}%
\pgfsetfillopacity{0.942935}%
\pgfsetlinewidth{1.003750pt}%
\definecolor{currentstroke}{rgb}{0.121569,0.466667,0.705882}%
\pgfsetstrokecolor{currentstroke}%
\pgfsetstrokeopacity{0.942935}%
\pgfsetdash{}{0pt}%
\pgfpathmoveto{\pgfqpoint{3.035310in}{2.601216in}}%
\pgfpathcurveto{\pgfqpoint{3.043546in}{2.601216in}}{\pgfqpoint{3.051446in}{2.604489in}}{\pgfqpoint{3.057270in}{2.610313in}}%
\pgfpathcurveto{\pgfqpoint{3.063094in}{2.616136in}}{\pgfqpoint{3.066367in}{2.624037in}}{\pgfqpoint{3.066367in}{2.632273in}}%
\pgfpathcurveto{\pgfqpoint{3.066367in}{2.640509in}}{\pgfqpoint{3.063094in}{2.648409in}}{\pgfqpoint{3.057270in}{2.654233in}}%
\pgfpathcurveto{\pgfqpoint{3.051446in}{2.660057in}}{\pgfqpoint{3.043546in}{2.663329in}}{\pgfqpoint{3.035310in}{2.663329in}}%
\pgfpathcurveto{\pgfqpoint{3.027074in}{2.663329in}}{\pgfqpoint{3.019174in}{2.660057in}}{\pgfqpoint{3.013350in}{2.654233in}}%
\pgfpathcurveto{\pgfqpoint{3.007526in}{2.648409in}}{\pgfqpoint{3.004254in}{2.640509in}}{\pgfqpoint{3.004254in}{2.632273in}}%
\pgfpathcurveto{\pgfqpoint{3.004254in}{2.624037in}}{\pgfqpoint{3.007526in}{2.616136in}}{\pgfqpoint{3.013350in}{2.610313in}}%
\pgfpathcurveto{\pgfqpoint{3.019174in}{2.604489in}}{\pgfqpoint{3.027074in}{2.601216in}}{\pgfqpoint{3.035310in}{2.601216in}}%
\pgfpathclose%
\pgfusepath{stroke,fill}%
\end{pgfscope}%
\begin{pgfscope}%
\pgfpathrectangle{\pgfqpoint{0.100000in}{0.212622in}}{\pgfqpoint{3.696000in}{3.696000in}}%
\pgfusepath{clip}%
\pgfsetbuttcap%
\pgfsetroundjoin%
\definecolor{currentfill}{rgb}{0.121569,0.466667,0.705882}%
\pgfsetfillcolor{currentfill}%
\pgfsetfillopacity{0.943375}%
\pgfsetlinewidth{1.003750pt}%
\definecolor{currentstroke}{rgb}{0.121569,0.466667,0.705882}%
\pgfsetstrokecolor{currentstroke}%
\pgfsetstrokeopacity{0.943375}%
\pgfsetdash{}{0pt}%
\pgfpathmoveto{\pgfqpoint{1.298430in}{1.798303in}}%
\pgfpathcurveto{\pgfqpoint{1.306666in}{1.798303in}}{\pgfqpoint{1.314566in}{1.801576in}}{\pgfqpoint{1.320390in}{1.807400in}}%
\pgfpathcurveto{\pgfqpoint{1.326214in}{1.813224in}}{\pgfqpoint{1.329486in}{1.821124in}}{\pgfqpoint{1.329486in}{1.829360in}}%
\pgfpathcurveto{\pgfqpoint{1.329486in}{1.837596in}}{\pgfqpoint{1.326214in}{1.845496in}}{\pgfqpoint{1.320390in}{1.851320in}}%
\pgfpathcurveto{\pgfqpoint{1.314566in}{1.857144in}}{\pgfqpoint{1.306666in}{1.860416in}}{\pgfqpoint{1.298430in}{1.860416in}}%
\pgfpathcurveto{\pgfqpoint{1.290193in}{1.860416in}}{\pgfqpoint{1.282293in}{1.857144in}}{\pgfqpoint{1.276469in}{1.851320in}}%
\pgfpathcurveto{\pgfqpoint{1.270646in}{1.845496in}}{\pgfqpoint{1.267373in}{1.837596in}}{\pgfqpoint{1.267373in}{1.829360in}}%
\pgfpathcurveto{\pgfqpoint{1.267373in}{1.821124in}}{\pgfqpoint{1.270646in}{1.813224in}}{\pgfqpoint{1.276469in}{1.807400in}}%
\pgfpathcurveto{\pgfqpoint{1.282293in}{1.801576in}}{\pgfqpoint{1.290193in}{1.798303in}}{\pgfqpoint{1.298430in}{1.798303in}}%
\pgfpathclose%
\pgfusepath{stroke,fill}%
\end{pgfscope}%
\begin{pgfscope}%
\pgfpathrectangle{\pgfqpoint{0.100000in}{0.212622in}}{\pgfqpoint{3.696000in}{3.696000in}}%
\pgfusepath{clip}%
\pgfsetbuttcap%
\pgfsetroundjoin%
\definecolor{currentfill}{rgb}{0.121569,0.466667,0.705882}%
\pgfsetfillcolor{currentfill}%
\pgfsetfillopacity{0.943676}%
\pgfsetlinewidth{1.003750pt}%
\definecolor{currentstroke}{rgb}{0.121569,0.466667,0.705882}%
\pgfsetstrokecolor{currentstroke}%
\pgfsetstrokeopacity{0.943676}%
\pgfsetdash{}{0pt}%
\pgfpathmoveto{\pgfqpoint{2.161705in}{2.193492in}}%
\pgfpathcurveto{\pgfqpoint{2.169941in}{2.193492in}}{\pgfqpoint{2.177841in}{2.196764in}}{\pgfqpoint{2.183665in}{2.202588in}}%
\pgfpathcurveto{\pgfqpoint{2.189489in}{2.208412in}}{\pgfqpoint{2.192762in}{2.216312in}}{\pgfqpoint{2.192762in}{2.224549in}}%
\pgfpathcurveto{\pgfqpoint{2.192762in}{2.232785in}}{\pgfqpoint{2.189489in}{2.240685in}}{\pgfqpoint{2.183665in}{2.246509in}}%
\pgfpathcurveto{\pgfqpoint{2.177841in}{2.252333in}}{\pgfqpoint{2.169941in}{2.255605in}}{\pgfqpoint{2.161705in}{2.255605in}}%
\pgfpathcurveto{\pgfqpoint{2.153469in}{2.255605in}}{\pgfqpoint{2.145569in}{2.252333in}}{\pgfqpoint{2.139745in}{2.246509in}}%
\pgfpathcurveto{\pgfqpoint{2.133921in}{2.240685in}}{\pgfqpoint{2.130649in}{2.232785in}}{\pgfqpoint{2.130649in}{2.224549in}}%
\pgfpathcurveto{\pgfqpoint{2.130649in}{2.216312in}}{\pgfqpoint{2.133921in}{2.208412in}}{\pgfqpoint{2.139745in}{2.202588in}}%
\pgfpathcurveto{\pgfqpoint{2.145569in}{2.196764in}}{\pgfqpoint{2.153469in}{2.193492in}}{\pgfqpoint{2.161705in}{2.193492in}}%
\pgfpathclose%
\pgfusepath{stroke,fill}%
\end{pgfscope}%
\begin{pgfscope}%
\pgfpathrectangle{\pgfqpoint{0.100000in}{0.212622in}}{\pgfqpoint{3.696000in}{3.696000in}}%
\pgfusepath{clip}%
\pgfsetbuttcap%
\pgfsetroundjoin%
\definecolor{currentfill}{rgb}{0.121569,0.466667,0.705882}%
\pgfsetfillcolor{currentfill}%
\pgfsetfillopacity{0.946091}%
\pgfsetlinewidth{1.003750pt}%
\definecolor{currentstroke}{rgb}{0.121569,0.466667,0.705882}%
\pgfsetstrokecolor{currentstroke}%
\pgfsetstrokeopacity{0.946091}%
\pgfsetdash{}{0pt}%
\pgfpathmoveto{\pgfqpoint{3.028685in}{2.595697in}}%
\pgfpathcurveto{\pgfqpoint{3.036921in}{2.595697in}}{\pgfqpoint{3.044821in}{2.598970in}}{\pgfqpoint{3.050645in}{2.604794in}}%
\pgfpathcurveto{\pgfqpoint{3.056469in}{2.610617in}}{\pgfqpoint{3.059741in}{2.618517in}}{\pgfqpoint{3.059741in}{2.626754in}}%
\pgfpathcurveto{\pgfqpoint{3.059741in}{2.634990in}}{\pgfqpoint{3.056469in}{2.642890in}}{\pgfqpoint{3.050645in}{2.648714in}}%
\pgfpathcurveto{\pgfqpoint{3.044821in}{2.654538in}}{\pgfqpoint{3.036921in}{2.657810in}}{\pgfqpoint{3.028685in}{2.657810in}}%
\pgfpathcurveto{\pgfqpoint{3.020449in}{2.657810in}}{\pgfqpoint{3.012549in}{2.654538in}}{\pgfqpoint{3.006725in}{2.648714in}}%
\pgfpathcurveto{\pgfqpoint{3.000901in}{2.642890in}}{\pgfqpoint{2.997628in}{2.634990in}}{\pgfqpoint{2.997628in}{2.626754in}}%
\pgfpathcurveto{\pgfqpoint{2.997628in}{2.618517in}}{\pgfqpoint{3.000901in}{2.610617in}}{\pgfqpoint{3.006725in}{2.604794in}}%
\pgfpathcurveto{\pgfqpoint{3.012549in}{2.598970in}}{\pgfqpoint{3.020449in}{2.595697in}}{\pgfqpoint{3.028685in}{2.595697in}}%
\pgfpathclose%
\pgfusepath{stroke,fill}%
\end{pgfscope}%
\begin{pgfscope}%
\pgfpathrectangle{\pgfqpoint{0.100000in}{0.212622in}}{\pgfqpoint{3.696000in}{3.696000in}}%
\pgfusepath{clip}%
\pgfsetbuttcap%
\pgfsetroundjoin%
\definecolor{currentfill}{rgb}{0.121569,0.466667,0.705882}%
\pgfsetfillcolor{currentfill}%
\pgfsetfillopacity{0.947066}%
\pgfsetlinewidth{1.003750pt}%
\definecolor{currentstroke}{rgb}{0.121569,0.466667,0.705882}%
\pgfsetstrokecolor{currentstroke}%
\pgfsetstrokeopacity{0.947066}%
\pgfsetdash{}{0pt}%
\pgfpathmoveto{\pgfqpoint{1.271955in}{1.763142in}}%
\pgfpathcurveto{\pgfqpoint{1.280191in}{1.763142in}}{\pgfqpoint{1.288091in}{1.766414in}}{\pgfqpoint{1.293915in}{1.772238in}}%
\pgfpathcurveto{\pgfqpoint{1.299739in}{1.778062in}}{\pgfqpoint{1.303011in}{1.785962in}}{\pgfqpoint{1.303011in}{1.794198in}}%
\pgfpathcurveto{\pgfqpoint{1.303011in}{1.802434in}}{\pgfqpoint{1.299739in}{1.810335in}}{\pgfqpoint{1.293915in}{1.816158in}}%
\pgfpathcurveto{\pgfqpoint{1.288091in}{1.821982in}}{\pgfqpoint{1.280191in}{1.825255in}}{\pgfqpoint{1.271955in}{1.825255in}}%
\pgfpathcurveto{\pgfqpoint{1.263718in}{1.825255in}}{\pgfqpoint{1.255818in}{1.821982in}}{\pgfqpoint{1.249994in}{1.816158in}}%
\pgfpathcurveto{\pgfqpoint{1.244170in}{1.810335in}}{\pgfqpoint{1.240898in}{1.802434in}}{\pgfqpoint{1.240898in}{1.794198in}}%
\pgfpathcurveto{\pgfqpoint{1.240898in}{1.785962in}}{\pgfqpoint{1.244170in}{1.778062in}}{\pgfqpoint{1.249994in}{1.772238in}}%
\pgfpathcurveto{\pgfqpoint{1.255818in}{1.766414in}}{\pgfqpoint{1.263718in}{1.763142in}}{\pgfqpoint{1.271955in}{1.763142in}}%
\pgfpathclose%
\pgfusepath{stroke,fill}%
\end{pgfscope}%
\begin{pgfscope}%
\pgfpathrectangle{\pgfqpoint{0.100000in}{0.212622in}}{\pgfqpoint{3.696000in}{3.696000in}}%
\pgfusepath{clip}%
\pgfsetbuttcap%
\pgfsetroundjoin%
\definecolor{currentfill}{rgb}{0.121569,0.466667,0.705882}%
\pgfsetfillcolor{currentfill}%
\pgfsetfillopacity{0.950057}%
\pgfsetlinewidth{1.003750pt}%
\definecolor{currentstroke}{rgb}{0.121569,0.466667,0.705882}%
\pgfsetstrokecolor{currentstroke}%
\pgfsetstrokeopacity{0.950057}%
\pgfsetdash{}{0pt}%
\pgfpathmoveto{\pgfqpoint{2.132230in}{2.174769in}}%
\pgfpathcurveto{\pgfqpoint{2.140467in}{2.174769in}}{\pgfqpoint{2.148367in}{2.178041in}}{\pgfqpoint{2.154191in}{2.183865in}}%
\pgfpathcurveto{\pgfqpoint{2.160015in}{2.189689in}}{\pgfqpoint{2.163287in}{2.197589in}}{\pgfqpoint{2.163287in}{2.205825in}}%
\pgfpathcurveto{\pgfqpoint{2.163287in}{2.214061in}}{\pgfqpoint{2.160015in}{2.221961in}}{\pgfqpoint{2.154191in}{2.227785in}}%
\pgfpathcurveto{\pgfqpoint{2.148367in}{2.233609in}}{\pgfqpoint{2.140467in}{2.236882in}}{\pgfqpoint{2.132230in}{2.236882in}}%
\pgfpathcurveto{\pgfqpoint{2.123994in}{2.236882in}}{\pgfqpoint{2.116094in}{2.233609in}}{\pgfqpoint{2.110270in}{2.227785in}}%
\pgfpathcurveto{\pgfqpoint{2.104446in}{2.221961in}}{\pgfqpoint{2.101174in}{2.214061in}}{\pgfqpoint{2.101174in}{2.205825in}}%
\pgfpathcurveto{\pgfqpoint{2.101174in}{2.197589in}}{\pgfqpoint{2.104446in}{2.189689in}}{\pgfqpoint{2.110270in}{2.183865in}}%
\pgfpathcurveto{\pgfqpoint{2.116094in}{2.178041in}}{\pgfqpoint{2.123994in}{2.174769in}}{\pgfqpoint{2.132230in}{2.174769in}}%
\pgfpathclose%
\pgfusepath{stroke,fill}%
\end{pgfscope}%
\begin{pgfscope}%
\pgfpathrectangle{\pgfqpoint{0.100000in}{0.212622in}}{\pgfqpoint{3.696000in}{3.696000in}}%
\pgfusepath{clip}%
\pgfsetbuttcap%
\pgfsetroundjoin%
\definecolor{currentfill}{rgb}{0.121569,0.466667,0.705882}%
\pgfsetfillcolor{currentfill}%
\pgfsetfillopacity{0.950214}%
\pgfsetlinewidth{1.003750pt}%
\definecolor{currentstroke}{rgb}{0.121569,0.466667,0.705882}%
\pgfsetstrokecolor{currentstroke}%
\pgfsetstrokeopacity{0.950214}%
\pgfsetdash{}{0pt}%
\pgfpathmoveto{\pgfqpoint{2.137580in}{2.174328in}}%
\pgfpathcurveto{\pgfqpoint{2.145817in}{2.174328in}}{\pgfqpoint{2.153717in}{2.177600in}}{\pgfqpoint{2.159541in}{2.183424in}}%
\pgfpathcurveto{\pgfqpoint{2.165365in}{2.189248in}}{\pgfqpoint{2.168637in}{2.197148in}}{\pgfqpoint{2.168637in}{2.205384in}}%
\pgfpathcurveto{\pgfqpoint{2.168637in}{2.213621in}}{\pgfqpoint{2.165365in}{2.221521in}}{\pgfqpoint{2.159541in}{2.227345in}}%
\pgfpathcurveto{\pgfqpoint{2.153717in}{2.233169in}}{\pgfqpoint{2.145817in}{2.236441in}}{\pgfqpoint{2.137580in}{2.236441in}}%
\pgfpathcurveto{\pgfqpoint{2.129344in}{2.236441in}}{\pgfqpoint{2.121444in}{2.233169in}}{\pgfqpoint{2.115620in}{2.227345in}}%
\pgfpathcurveto{\pgfqpoint{2.109796in}{2.221521in}}{\pgfqpoint{2.106524in}{2.213621in}}{\pgfqpoint{2.106524in}{2.205384in}}%
\pgfpathcurveto{\pgfqpoint{2.106524in}{2.197148in}}{\pgfqpoint{2.109796in}{2.189248in}}{\pgfqpoint{2.115620in}{2.183424in}}%
\pgfpathcurveto{\pgfqpoint{2.121444in}{2.177600in}}{\pgfqpoint{2.129344in}{2.174328in}}{\pgfqpoint{2.137580in}{2.174328in}}%
\pgfpathclose%
\pgfusepath{stroke,fill}%
\end{pgfscope}%
\begin{pgfscope}%
\pgfpathrectangle{\pgfqpoint{0.100000in}{0.212622in}}{\pgfqpoint{3.696000in}{3.696000in}}%
\pgfusepath{clip}%
\pgfsetbuttcap%
\pgfsetroundjoin%
\definecolor{currentfill}{rgb}{0.121569,0.466667,0.705882}%
\pgfsetfillcolor{currentfill}%
\pgfsetfillopacity{0.951297}%
\pgfsetlinewidth{1.003750pt}%
\definecolor{currentstroke}{rgb}{0.121569,0.466667,0.705882}%
\pgfsetstrokecolor{currentstroke}%
\pgfsetstrokeopacity{0.951297}%
\pgfsetdash{}{0pt}%
\pgfpathmoveto{\pgfqpoint{2.117663in}{2.163315in}}%
\pgfpathcurveto{\pgfqpoint{2.125899in}{2.163315in}}{\pgfqpoint{2.133799in}{2.166587in}}{\pgfqpoint{2.139623in}{2.172411in}}%
\pgfpathcurveto{\pgfqpoint{2.145447in}{2.178235in}}{\pgfqpoint{2.148719in}{2.186135in}}{\pgfqpoint{2.148719in}{2.194371in}}%
\pgfpathcurveto{\pgfqpoint{2.148719in}{2.202608in}}{\pgfqpoint{2.145447in}{2.210508in}}{\pgfqpoint{2.139623in}{2.216332in}}%
\pgfpathcurveto{\pgfqpoint{2.133799in}{2.222156in}}{\pgfqpoint{2.125899in}{2.225428in}}{\pgfqpoint{2.117663in}{2.225428in}}%
\pgfpathcurveto{\pgfqpoint{2.109426in}{2.225428in}}{\pgfqpoint{2.101526in}{2.222156in}}{\pgfqpoint{2.095702in}{2.216332in}}%
\pgfpathcurveto{\pgfqpoint{2.089878in}{2.210508in}}{\pgfqpoint{2.086606in}{2.202608in}}{\pgfqpoint{2.086606in}{2.194371in}}%
\pgfpathcurveto{\pgfqpoint{2.086606in}{2.186135in}}{\pgfqpoint{2.089878in}{2.178235in}}{\pgfqpoint{2.095702in}{2.172411in}}%
\pgfpathcurveto{\pgfqpoint{2.101526in}{2.166587in}}{\pgfqpoint{2.109426in}{2.163315in}}{\pgfqpoint{2.117663in}{2.163315in}}%
\pgfpathclose%
\pgfusepath{stroke,fill}%
\end{pgfscope}%
\begin{pgfscope}%
\pgfpathrectangle{\pgfqpoint{0.100000in}{0.212622in}}{\pgfqpoint{3.696000in}{3.696000in}}%
\pgfusepath{clip}%
\pgfsetbuttcap%
\pgfsetroundjoin%
\definecolor{currentfill}{rgb}{0.121569,0.466667,0.705882}%
\pgfsetfillcolor{currentfill}%
\pgfsetfillopacity{0.951351}%
\pgfsetlinewidth{1.003750pt}%
\definecolor{currentstroke}{rgb}{0.121569,0.466667,0.705882}%
\pgfsetstrokecolor{currentstroke}%
\pgfsetstrokeopacity{0.951351}%
\pgfsetdash{}{0pt}%
\pgfpathmoveto{\pgfqpoint{1.340174in}{1.830876in}}%
\pgfpathcurveto{\pgfqpoint{1.348410in}{1.830876in}}{\pgfqpoint{1.356310in}{1.834148in}}{\pgfqpoint{1.362134in}{1.839972in}}%
\pgfpathcurveto{\pgfqpoint{1.367958in}{1.845796in}}{\pgfqpoint{1.371230in}{1.853696in}}{\pgfqpoint{1.371230in}{1.861932in}}%
\pgfpathcurveto{\pgfqpoint{1.371230in}{1.870168in}}{\pgfqpoint{1.367958in}{1.878068in}}{\pgfqpoint{1.362134in}{1.883892in}}%
\pgfpathcurveto{\pgfqpoint{1.356310in}{1.889716in}}{\pgfqpoint{1.348410in}{1.892989in}}{\pgfqpoint{1.340174in}{1.892989in}}%
\pgfpathcurveto{\pgfqpoint{1.331938in}{1.892989in}}{\pgfqpoint{1.324037in}{1.889716in}}{\pgfqpoint{1.318214in}{1.883892in}}%
\pgfpathcurveto{\pgfqpoint{1.312390in}{1.878068in}}{\pgfqpoint{1.309117in}{1.870168in}}{\pgfqpoint{1.309117in}{1.861932in}}%
\pgfpathcurveto{\pgfqpoint{1.309117in}{1.853696in}}{\pgfqpoint{1.312390in}{1.845796in}}{\pgfqpoint{1.318214in}{1.839972in}}%
\pgfpathcurveto{\pgfqpoint{1.324037in}{1.834148in}}{\pgfqpoint{1.331938in}{1.830876in}}{\pgfqpoint{1.340174in}{1.830876in}}%
\pgfpathclose%
\pgfusepath{stroke,fill}%
\end{pgfscope}%
\begin{pgfscope}%
\pgfpathrectangle{\pgfqpoint{0.100000in}{0.212622in}}{\pgfqpoint{3.696000in}{3.696000in}}%
\pgfusepath{clip}%
\pgfsetbuttcap%
\pgfsetroundjoin%
\definecolor{currentfill}{rgb}{0.121569,0.466667,0.705882}%
\pgfsetfillcolor{currentfill}%
\pgfsetfillopacity{0.952139}%
\pgfsetlinewidth{1.003750pt}%
\definecolor{currentstroke}{rgb}{0.121569,0.466667,0.705882}%
\pgfsetstrokecolor{currentstroke}%
\pgfsetstrokeopacity{0.952139}%
\pgfsetdash{}{0pt}%
\pgfpathmoveto{\pgfqpoint{3.015367in}{2.582743in}}%
\pgfpathcurveto{\pgfqpoint{3.023603in}{2.582743in}}{\pgfqpoint{3.031503in}{2.586015in}}{\pgfqpoint{3.037327in}{2.591839in}}%
\pgfpathcurveto{\pgfqpoint{3.043151in}{2.597663in}}{\pgfqpoint{3.046423in}{2.605563in}}{\pgfqpoint{3.046423in}{2.613800in}}%
\pgfpathcurveto{\pgfqpoint{3.046423in}{2.622036in}}{\pgfqpoint{3.043151in}{2.629936in}}{\pgfqpoint{3.037327in}{2.635760in}}%
\pgfpathcurveto{\pgfqpoint{3.031503in}{2.641584in}}{\pgfqpoint{3.023603in}{2.644856in}}{\pgfqpoint{3.015367in}{2.644856in}}%
\pgfpathcurveto{\pgfqpoint{3.007130in}{2.644856in}}{\pgfqpoint{2.999230in}{2.641584in}}{\pgfqpoint{2.993406in}{2.635760in}}%
\pgfpathcurveto{\pgfqpoint{2.987582in}{2.629936in}}{\pgfqpoint{2.984310in}{2.622036in}}{\pgfqpoint{2.984310in}{2.613800in}}%
\pgfpathcurveto{\pgfqpoint{2.984310in}{2.605563in}}{\pgfqpoint{2.987582in}{2.597663in}}{\pgfqpoint{2.993406in}{2.591839in}}%
\pgfpathcurveto{\pgfqpoint{2.999230in}{2.586015in}}{\pgfqpoint{3.007130in}{2.582743in}}{\pgfqpoint{3.015367in}{2.582743in}}%
\pgfpathclose%
\pgfusepath{stroke,fill}%
\end{pgfscope}%
\begin{pgfscope}%
\pgfpathrectangle{\pgfqpoint{0.100000in}{0.212622in}}{\pgfqpoint{3.696000in}{3.696000in}}%
\pgfusepath{clip}%
\pgfsetbuttcap%
\pgfsetroundjoin%
\definecolor{currentfill}{rgb}{0.121569,0.466667,0.705882}%
\pgfsetfillcolor{currentfill}%
\pgfsetfillopacity{0.953525}%
\pgfsetlinewidth{1.003750pt}%
\definecolor{currentstroke}{rgb}{0.121569,0.466667,0.705882}%
\pgfsetstrokecolor{currentstroke}%
\pgfsetstrokeopacity{0.953525}%
\pgfsetdash{}{0pt}%
\pgfpathmoveto{\pgfqpoint{2.091390in}{2.158528in}}%
\pgfpathcurveto{\pgfqpoint{2.099626in}{2.158528in}}{\pgfqpoint{2.107526in}{2.161800in}}{\pgfqpoint{2.113350in}{2.167624in}}%
\pgfpathcurveto{\pgfqpoint{2.119174in}{2.173448in}}{\pgfqpoint{2.122447in}{2.181348in}}{\pgfqpoint{2.122447in}{2.189585in}}%
\pgfpathcurveto{\pgfqpoint{2.122447in}{2.197821in}}{\pgfqpoint{2.119174in}{2.205721in}}{\pgfqpoint{2.113350in}{2.211545in}}%
\pgfpathcurveto{\pgfqpoint{2.107526in}{2.217369in}}{\pgfqpoint{2.099626in}{2.220641in}}{\pgfqpoint{2.091390in}{2.220641in}}%
\pgfpathcurveto{\pgfqpoint{2.083154in}{2.220641in}}{\pgfqpoint{2.075254in}{2.217369in}}{\pgfqpoint{2.069430in}{2.211545in}}%
\pgfpathcurveto{\pgfqpoint{2.063606in}{2.205721in}}{\pgfqpoint{2.060334in}{2.197821in}}{\pgfqpoint{2.060334in}{2.189585in}}%
\pgfpathcurveto{\pgfqpoint{2.060334in}{2.181348in}}{\pgfqpoint{2.063606in}{2.173448in}}{\pgfqpoint{2.069430in}{2.167624in}}%
\pgfpathcurveto{\pgfqpoint{2.075254in}{2.161800in}}{\pgfqpoint{2.083154in}{2.158528in}}{\pgfqpoint{2.091390in}{2.158528in}}%
\pgfpathclose%
\pgfusepath{stroke,fill}%
\end{pgfscope}%
\begin{pgfscope}%
\pgfpathrectangle{\pgfqpoint{0.100000in}{0.212622in}}{\pgfqpoint{3.696000in}{3.696000in}}%
\pgfusepath{clip}%
\pgfsetbuttcap%
\pgfsetroundjoin%
\definecolor{currentfill}{rgb}{0.121569,0.466667,0.705882}%
\pgfsetfillcolor{currentfill}%
\pgfsetfillopacity{0.954707}%
\pgfsetlinewidth{1.003750pt}%
\definecolor{currentstroke}{rgb}{0.121569,0.466667,0.705882}%
\pgfsetstrokecolor{currentstroke}%
\pgfsetstrokeopacity{0.954707}%
\pgfsetdash{}{0pt}%
\pgfpathmoveto{\pgfqpoint{2.978573in}{2.538937in}}%
\pgfpathcurveto{\pgfqpoint{2.986809in}{2.538937in}}{\pgfqpoint{2.994709in}{2.542210in}}{\pgfqpoint{3.000533in}{2.548033in}}%
\pgfpathcurveto{\pgfqpoint{3.006357in}{2.553857in}}{\pgfqpoint{3.009629in}{2.561757in}}{\pgfqpoint{3.009629in}{2.569994in}}%
\pgfpathcurveto{\pgfqpoint{3.009629in}{2.578230in}}{\pgfqpoint{3.006357in}{2.586130in}}{\pgfqpoint{3.000533in}{2.591954in}}%
\pgfpathcurveto{\pgfqpoint{2.994709in}{2.597778in}}{\pgfqpoint{2.986809in}{2.601050in}}{\pgfqpoint{2.978573in}{2.601050in}}%
\pgfpathcurveto{\pgfqpoint{2.970336in}{2.601050in}}{\pgfqpoint{2.962436in}{2.597778in}}{\pgfqpoint{2.956612in}{2.591954in}}%
\pgfpathcurveto{\pgfqpoint{2.950788in}{2.586130in}}{\pgfqpoint{2.947516in}{2.578230in}}{\pgfqpoint{2.947516in}{2.569994in}}%
\pgfpathcurveto{\pgfqpoint{2.947516in}{2.561757in}}{\pgfqpoint{2.950788in}{2.553857in}}{\pgfqpoint{2.956612in}{2.548033in}}%
\pgfpathcurveto{\pgfqpoint{2.962436in}{2.542210in}}{\pgfqpoint{2.970336in}{2.538937in}}{\pgfqpoint{2.978573in}{2.538937in}}%
\pgfpathclose%
\pgfusepath{stroke,fill}%
\end{pgfscope}%
\begin{pgfscope}%
\pgfpathrectangle{\pgfqpoint{0.100000in}{0.212622in}}{\pgfqpoint{3.696000in}{3.696000in}}%
\pgfusepath{clip}%
\pgfsetbuttcap%
\pgfsetroundjoin%
\definecolor{currentfill}{rgb}{0.121569,0.466667,0.705882}%
\pgfsetfillcolor{currentfill}%
\pgfsetfillopacity{0.954770}%
\pgfsetlinewidth{1.003750pt}%
\definecolor{currentstroke}{rgb}{0.121569,0.466667,0.705882}%
\pgfsetstrokecolor{currentstroke}%
\pgfsetstrokeopacity{0.954770}%
\pgfsetdash{}{0pt}%
\pgfpathmoveto{\pgfqpoint{2.006429in}{2.124414in}}%
\pgfpathcurveto{\pgfqpoint{2.014665in}{2.124414in}}{\pgfqpoint{2.022565in}{2.127686in}}{\pgfqpoint{2.028389in}{2.133510in}}%
\pgfpathcurveto{\pgfqpoint{2.034213in}{2.139334in}}{\pgfqpoint{2.037485in}{2.147234in}}{\pgfqpoint{2.037485in}{2.155470in}}%
\pgfpathcurveto{\pgfqpoint{2.037485in}{2.163706in}}{\pgfqpoint{2.034213in}{2.171606in}}{\pgfqpoint{2.028389in}{2.177430in}}%
\pgfpathcurveto{\pgfqpoint{2.022565in}{2.183254in}}{\pgfqpoint{2.014665in}{2.186527in}}{\pgfqpoint{2.006429in}{2.186527in}}%
\pgfpathcurveto{\pgfqpoint{1.998192in}{2.186527in}}{\pgfqpoint{1.990292in}{2.183254in}}{\pgfqpoint{1.984468in}{2.177430in}}%
\pgfpathcurveto{\pgfqpoint{1.978644in}{2.171606in}}{\pgfqpoint{1.975372in}{2.163706in}}{\pgfqpoint{1.975372in}{2.155470in}}%
\pgfpathcurveto{\pgfqpoint{1.975372in}{2.147234in}}{\pgfqpoint{1.978644in}{2.139334in}}{\pgfqpoint{1.984468in}{2.133510in}}%
\pgfpathcurveto{\pgfqpoint{1.990292in}{2.127686in}}{\pgfqpoint{1.998192in}{2.124414in}}{\pgfqpoint{2.006429in}{2.124414in}}%
\pgfpathclose%
\pgfusepath{stroke,fill}%
\end{pgfscope}%
\begin{pgfscope}%
\pgfpathrectangle{\pgfqpoint{0.100000in}{0.212622in}}{\pgfqpoint{3.696000in}{3.696000in}}%
\pgfusepath{clip}%
\pgfsetbuttcap%
\pgfsetroundjoin%
\definecolor{currentfill}{rgb}{0.121569,0.466667,0.705882}%
\pgfsetfillcolor{currentfill}%
\pgfsetfillopacity{0.955381}%
\pgfsetlinewidth{1.003750pt}%
\definecolor{currentstroke}{rgb}{0.121569,0.466667,0.705882}%
\pgfsetstrokecolor{currentstroke}%
\pgfsetstrokeopacity{0.955381}%
\pgfsetdash{}{0pt}%
\pgfpathmoveto{\pgfqpoint{2.113632in}{2.158088in}}%
\pgfpathcurveto{\pgfqpoint{2.121868in}{2.158088in}}{\pgfqpoint{2.129768in}{2.161360in}}{\pgfqpoint{2.135592in}{2.167184in}}%
\pgfpathcurveto{\pgfqpoint{2.141416in}{2.173008in}}{\pgfqpoint{2.144689in}{2.180908in}}{\pgfqpoint{2.144689in}{2.189144in}}%
\pgfpathcurveto{\pgfqpoint{2.144689in}{2.197381in}}{\pgfqpoint{2.141416in}{2.205281in}}{\pgfqpoint{2.135592in}{2.211105in}}%
\pgfpathcurveto{\pgfqpoint{2.129768in}{2.216929in}}{\pgfqpoint{2.121868in}{2.220201in}}{\pgfqpoint{2.113632in}{2.220201in}}%
\pgfpathcurveto{\pgfqpoint{2.105396in}{2.220201in}}{\pgfqpoint{2.097496in}{2.216929in}}{\pgfqpoint{2.091672in}{2.211105in}}%
\pgfpathcurveto{\pgfqpoint{2.085848in}{2.205281in}}{\pgfqpoint{2.082576in}{2.197381in}}{\pgfqpoint{2.082576in}{2.189144in}}%
\pgfpathcurveto{\pgfqpoint{2.082576in}{2.180908in}}{\pgfqpoint{2.085848in}{2.173008in}}{\pgfqpoint{2.091672in}{2.167184in}}%
\pgfpathcurveto{\pgfqpoint{2.097496in}{2.161360in}}{\pgfqpoint{2.105396in}{2.158088in}}{\pgfqpoint{2.113632in}{2.158088in}}%
\pgfpathclose%
\pgfusepath{stroke,fill}%
\end{pgfscope}%
\begin{pgfscope}%
\pgfpathrectangle{\pgfqpoint{0.100000in}{0.212622in}}{\pgfqpoint{3.696000in}{3.696000in}}%
\pgfusepath{clip}%
\pgfsetbuttcap%
\pgfsetroundjoin%
\definecolor{currentfill}{rgb}{0.121569,0.466667,0.705882}%
\pgfsetfillcolor{currentfill}%
\pgfsetfillopacity{0.956690}%
\pgfsetlinewidth{1.003750pt}%
\definecolor{currentstroke}{rgb}{0.121569,0.466667,0.705882}%
\pgfsetstrokecolor{currentstroke}%
\pgfsetstrokeopacity{0.956690}%
\pgfsetdash{}{0pt}%
\pgfpathmoveto{\pgfqpoint{1.803378in}{2.054162in}}%
\pgfpathcurveto{\pgfqpoint{1.811615in}{2.054162in}}{\pgfqpoint{1.819515in}{2.057434in}}{\pgfqpoint{1.825339in}{2.063258in}}%
\pgfpathcurveto{\pgfqpoint{1.831162in}{2.069082in}}{\pgfqpoint{1.834435in}{2.076982in}}{\pgfqpoint{1.834435in}{2.085218in}}%
\pgfpathcurveto{\pgfqpoint{1.834435in}{2.093455in}}{\pgfqpoint{1.831162in}{2.101355in}}{\pgfqpoint{1.825339in}{2.107178in}}%
\pgfpathcurveto{\pgfqpoint{1.819515in}{2.113002in}}{\pgfqpoint{1.811615in}{2.116275in}}{\pgfqpoint{1.803378in}{2.116275in}}%
\pgfpathcurveto{\pgfqpoint{1.795142in}{2.116275in}}{\pgfqpoint{1.787242in}{2.113002in}}{\pgfqpoint{1.781418in}{2.107178in}}%
\pgfpathcurveto{\pgfqpoint{1.775594in}{2.101355in}}{\pgfqpoint{1.772322in}{2.093455in}}{\pgfqpoint{1.772322in}{2.085218in}}%
\pgfpathcurveto{\pgfqpoint{1.772322in}{2.076982in}}{\pgfqpoint{1.775594in}{2.069082in}}{\pgfqpoint{1.781418in}{2.063258in}}%
\pgfpathcurveto{\pgfqpoint{1.787242in}{2.057434in}}{\pgfqpoint{1.795142in}{2.054162in}}{\pgfqpoint{1.803378in}{2.054162in}}%
\pgfpathclose%
\pgfusepath{stroke,fill}%
\end{pgfscope}%
\begin{pgfscope}%
\pgfpathrectangle{\pgfqpoint{0.100000in}{0.212622in}}{\pgfqpoint{3.696000in}{3.696000in}}%
\pgfusepath{clip}%
\pgfsetbuttcap%
\pgfsetroundjoin%
\definecolor{currentfill}{rgb}{0.121569,0.466667,0.705882}%
\pgfsetfillcolor{currentfill}%
\pgfsetfillopacity{0.958427}%
\pgfsetlinewidth{1.003750pt}%
\definecolor{currentstroke}{rgb}{0.121569,0.466667,0.705882}%
\pgfsetstrokecolor{currentstroke}%
\pgfsetstrokeopacity{0.958427}%
\pgfsetdash{}{0pt}%
\pgfpathmoveto{\pgfqpoint{3.000322in}{2.567157in}}%
\pgfpathcurveto{\pgfqpoint{3.008559in}{2.567157in}}{\pgfqpoint{3.016459in}{2.570429in}}{\pgfqpoint{3.022283in}{2.576253in}}%
\pgfpathcurveto{\pgfqpoint{3.028106in}{2.582077in}}{\pgfqpoint{3.031379in}{2.589977in}}{\pgfqpoint{3.031379in}{2.598214in}}%
\pgfpathcurveto{\pgfqpoint{3.031379in}{2.606450in}}{\pgfqpoint{3.028106in}{2.614350in}}{\pgfqpoint{3.022283in}{2.620174in}}%
\pgfpathcurveto{\pgfqpoint{3.016459in}{2.625998in}}{\pgfqpoint{3.008559in}{2.629270in}}{\pgfqpoint{3.000322in}{2.629270in}}%
\pgfpathcurveto{\pgfqpoint{2.992086in}{2.629270in}}{\pgfqpoint{2.984186in}{2.625998in}}{\pgfqpoint{2.978362in}{2.620174in}}%
\pgfpathcurveto{\pgfqpoint{2.972538in}{2.614350in}}{\pgfqpoint{2.969266in}{2.606450in}}{\pgfqpoint{2.969266in}{2.598214in}}%
\pgfpathcurveto{\pgfqpoint{2.969266in}{2.589977in}}{\pgfqpoint{2.972538in}{2.582077in}}{\pgfqpoint{2.978362in}{2.576253in}}%
\pgfpathcurveto{\pgfqpoint{2.984186in}{2.570429in}}{\pgfqpoint{2.992086in}{2.567157in}}{\pgfqpoint{3.000322in}{2.567157in}}%
\pgfpathclose%
\pgfusepath{stroke,fill}%
\end{pgfscope}%
\begin{pgfscope}%
\pgfpathrectangle{\pgfqpoint{0.100000in}{0.212622in}}{\pgfqpoint{3.696000in}{3.696000in}}%
\pgfusepath{clip}%
\pgfsetbuttcap%
\pgfsetroundjoin%
\definecolor{currentfill}{rgb}{0.121569,0.466667,0.705882}%
\pgfsetfillcolor{currentfill}%
\pgfsetfillopacity{0.958953}%
\pgfsetlinewidth{1.003750pt}%
\definecolor{currentstroke}{rgb}{0.121569,0.466667,0.705882}%
\pgfsetstrokecolor{currentstroke}%
\pgfsetstrokeopacity{0.958953}%
\pgfsetdash{}{0pt}%
\pgfpathmoveto{\pgfqpoint{2.011735in}{2.126330in}}%
\pgfpathcurveto{\pgfqpoint{2.019971in}{2.126330in}}{\pgfqpoint{2.027871in}{2.129603in}}{\pgfqpoint{2.033695in}{2.135427in}}%
\pgfpathcurveto{\pgfqpoint{2.039519in}{2.141251in}}{\pgfqpoint{2.042791in}{2.149151in}}{\pgfqpoint{2.042791in}{2.157387in}}%
\pgfpathcurveto{\pgfqpoint{2.042791in}{2.165623in}}{\pgfqpoint{2.039519in}{2.173523in}}{\pgfqpoint{2.033695in}{2.179347in}}%
\pgfpathcurveto{\pgfqpoint{2.027871in}{2.185171in}}{\pgfqpoint{2.019971in}{2.188443in}}{\pgfqpoint{2.011735in}{2.188443in}}%
\pgfpathcurveto{\pgfqpoint{2.003499in}{2.188443in}}{\pgfqpoint{1.995598in}{2.185171in}}{\pgfqpoint{1.989775in}{2.179347in}}%
\pgfpathcurveto{\pgfqpoint{1.983951in}{2.173523in}}{\pgfqpoint{1.980678in}{2.165623in}}{\pgfqpoint{1.980678in}{2.157387in}}%
\pgfpathcurveto{\pgfqpoint{1.980678in}{2.149151in}}{\pgfqpoint{1.983951in}{2.141251in}}{\pgfqpoint{1.989775in}{2.135427in}}%
\pgfpathcurveto{\pgfqpoint{1.995598in}{2.129603in}}{\pgfqpoint{2.003499in}{2.126330in}}{\pgfqpoint{2.011735in}{2.126330in}}%
\pgfpathclose%
\pgfusepath{stroke,fill}%
\end{pgfscope}%
\begin{pgfscope}%
\pgfpathrectangle{\pgfqpoint{0.100000in}{0.212622in}}{\pgfqpoint{3.696000in}{3.696000in}}%
\pgfusepath{clip}%
\pgfsetbuttcap%
\pgfsetroundjoin%
\definecolor{currentfill}{rgb}{0.121569,0.466667,0.705882}%
\pgfsetfillcolor{currentfill}%
\pgfsetfillopacity{0.959271}%
\pgfsetlinewidth{1.003750pt}%
\definecolor{currentstroke}{rgb}{0.121569,0.466667,0.705882}%
\pgfsetstrokecolor{currentstroke}%
\pgfsetstrokeopacity{0.959271}%
\pgfsetdash{}{0pt}%
\pgfpathmoveto{\pgfqpoint{1.817590in}{2.073812in}}%
\pgfpathcurveto{\pgfqpoint{1.825826in}{2.073812in}}{\pgfqpoint{1.833726in}{2.077085in}}{\pgfqpoint{1.839550in}{2.082909in}}%
\pgfpathcurveto{\pgfqpoint{1.845374in}{2.088733in}}{\pgfqpoint{1.848646in}{2.096633in}}{\pgfqpoint{1.848646in}{2.104869in}}%
\pgfpathcurveto{\pgfqpoint{1.848646in}{2.113105in}}{\pgfqpoint{1.845374in}{2.121005in}}{\pgfqpoint{1.839550in}{2.126829in}}%
\pgfpathcurveto{\pgfqpoint{1.833726in}{2.132653in}}{\pgfqpoint{1.825826in}{2.135925in}}{\pgfqpoint{1.817590in}{2.135925in}}%
\pgfpathcurveto{\pgfqpoint{1.809354in}{2.135925in}}{\pgfqpoint{1.801454in}{2.132653in}}{\pgfqpoint{1.795630in}{2.126829in}}%
\pgfpathcurveto{\pgfqpoint{1.789806in}{2.121005in}}{\pgfqpoint{1.786533in}{2.113105in}}{\pgfqpoint{1.786533in}{2.104869in}}%
\pgfpathcurveto{\pgfqpoint{1.786533in}{2.096633in}}{\pgfqpoint{1.789806in}{2.088733in}}{\pgfqpoint{1.795630in}{2.082909in}}%
\pgfpathcurveto{\pgfqpoint{1.801454in}{2.077085in}}{\pgfqpoint{1.809354in}{2.073812in}}{\pgfqpoint{1.817590in}{2.073812in}}%
\pgfpathclose%
\pgfusepath{stroke,fill}%
\end{pgfscope}%
\begin{pgfscope}%
\pgfpathrectangle{\pgfqpoint{0.100000in}{0.212622in}}{\pgfqpoint{3.696000in}{3.696000in}}%
\pgfusepath{clip}%
\pgfsetbuttcap%
\pgfsetroundjoin%
\definecolor{currentfill}{rgb}{0.121569,0.466667,0.705882}%
\pgfsetfillcolor{currentfill}%
\pgfsetfillopacity{0.959447}%
\pgfsetlinewidth{1.003750pt}%
\definecolor{currentstroke}{rgb}{0.121569,0.466667,0.705882}%
\pgfsetstrokecolor{currentstroke}%
\pgfsetstrokeopacity{0.959447}%
\pgfsetdash{}{0pt}%
\pgfpathmoveto{\pgfqpoint{1.983534in}{2.107286in}}%
\pgfpathcurveto{\pgfqpoint{1.991770in}{2.107286in}}{\pgfqpoint{1.999670in}{2.110558in}}{\pgfqpoint{2.005494in}{2.116382in}}%
\pgfpathcurveto{\pgfqpoint{2.011318in}{2.122206in}}{\pgfqpoint{2.014590in}{2.130106in}}{\pgfqpoint{2.014590in}{2.138342in}}%
\pgfpathcurveto{\pgfqpoint{2.014590in}{2.146578in}}{\pgfqpoint{2.011318in}{2.154478in}}{\pgfqpoint{2.005494in}{2.160302in}}%
\pgfpathcurveto{\pgfqpoint{1.999670in}{2.166126in}}{\pgfqpoint{1.991770in}{2.169399in}}{\pgfqpoint{1.983534in}{2.169399in}}%
\pgfpathcurveto{\pgfqpoint{1.975298in}{2.169399in}}{\pgfqpoint{1.967398in}{2.166126in}}{\pgfqpoint{1.961574in}{2.160302in}}%
\pgfpathcurveto{\pgfqpoint{1.955750in}{2.154478in}}{\pgfqpoint{1.952477in}{2.146578in}}{\pgfqpoint{1.952477in}{2.138342in}}%
\pgfpathcurveto{\pgfqpoint{1.952477in}{2.130106in}}{\pgfqpoint{1.955750in}{2.122206in}}{\pgfqpoint{1.961574in}{2.116382in}}%
\pgfpathcurveto{\pgfqpoint{1.967398in}{2.110558in}}{\pgfqpoint{1.975298in}{2.107286in}}{\pgfqpoint{1.983534in}{2.107286in}}%
\pgfpathclose%
\pgfusepath{stroke,fill}%
\end{pgfscope}%
\begin{pgfscope}%
\pgfpathrectangle{\pgfqpoint{0.100000in}{0.212622in}}{\pgfqpoint{3.696000in}{3.696000in}}%
\pgfusepath{clip}%
\pgfsetbuttcap%
\pgfsetroundjoin%
\definecolor{currentfill}{rgb}{0.121569,0.466667,0.705882}%
\pgfsetfillcolor{currentfill}%
\pgfsetfillopacity{0.959587}%
\pgfsetlinewidth{1.003750pt}%
\definecolor{currentstroke}{rgb}{0.121569,0.466667,0.705882}%
\pgfsetstrokecolor{currentstroke}%
\pgfsetstrokeopacity{0.959587}%
\pgfsetdash{}{0pt}%
\pgfpathmoveto{\pgfqpoint{1.273918in}{1.775146in}}%
\pgfpathcurveto{\pgfqpoint{1.282154in}{1.775146in}}{\pgfqpoint{1.290054in}{1.778419in}}{\pgfqpoint{1.295878in}{1.784243in}}%
\pgfpathcurveto{\pgfqpoint{1.301702in}{1.790066in}}{\pgfqpoint{1.304974in}{1.797967in}}{\pgfqpoint{1.304974in}{1.806203in}}%
\pgfpathcurveto{\pgfqpoint{1.304974in}{1.814439in}}{\pgfqpoint{1.301702in}{1.822339in}}{\pgfqpoint{1.295878in}{1.828163in}}%
\pgfpathcurveto{\pgfqpoint{1.290054in}{1.833987in}}{\pgfqpoint{1.282154in}{1.837259in}}{\pgfqpoint{1.273918in}{1.837259in}}%
\pgfpathcurveto{\pgfqpoint{1.265681in}{1.837259in}}{\pgfqpoint{1.257781in}{1.833987in}}{\pgfqpoint{1.251957in}{1.828163in}}%
\pgfpathcurveto{\pgfqpoint{1.246134in}{1.822339in}}{\pgfqpoint{1.242861in}{1.814439in}}{\pgfqpoint{1.242861in}{1.806203in}}%
\pgfpathcurveto{\pgfqpoint{1.242861in}{1.797967in}}{\pgfqpoint{1.246134in}{1.790066in}}{\pgfqpoint{1.251957in}{1.784243in}}%
\pgfpathcurveto{\pgfqpoint{1.257781in}{1.778419in}}{\pgfqpoint{1.265681in}{1.775146in}}{\pgfqpoint{1.273918in}{1.775146in}}%
\pgfpathclose%
\pgfusepath{stroke,fill}%
\end{pgfscope}%
\begin{pgfscope}%
\pgfpathrectangle{\pgfqpoint{0.100000in}{0.212622in}}{\pgfqpoint{3.696000in}{3.696000in}}%
\pgfusepath{clip}%
\pgfsetbuttcap%
\pgfsetroundjoin%
\definecolor{currentfill}{rgb}{0.121569,0.466667,0.705882}%
\pgfsetfillcolor{currentfill}%
\pgfsetfillopacity{0.961950}%
\pgfsetlinewidth{1.003750pt}%
\definecolor{currentstroke}{rgb}{0.121569,0.466667,0.705882}%
\pgfsetstrokecolor{currentstroke}%
\pgfsetstrokeopacity{0.961950}%
\pgfsetdash{}{0pt}%
\pgfpathmoveto{\pgfqpoint{1.828078in}{2.072021in}}%
\pgfpathcurveto{\pgfqpoint{1.836314in}{2.072021in}}{\pgfqpoint{1.844214in}{2.075293in}}{\pgfqpoint{1.850038in}{2.081117in}}%
\pgfpathcurveto{\pgfqpoint{1.855862in}{2.086941in}}{\pgfqpoint{1.859134in}{2.094841in}}{\pgfqpoint{1.859134in}{2.103077in}}%
\pgfpathcurveto{\pgfqpoint{1.859134in}{2.111314in}}{\pgfqpoint{1.855862in}{2.119214in}}{\pgfqpoint{1.850038in}{2.125038in}}%
\pgfpathcurveto{\pgfqpoint{1.844214in}{2.130861in}}{\pgfqpoint{1.836314in}{2.134134in}}{\pgfqpoint{1.828078in}{2.134134in}}%
\pgfpathcurveto{\pgfqpoint{1.819841in}{2.134134in}}{\pgfqpoint{1.811941in}{2.130861in}}{\pgfqpoint{1.806117in}{2.125038in}}%
\pgfpathcurveto{\pgfqpoint{1.800293in}{2.119214in}}{\pgfqpoint{1.797021in}{2.111314in}}{\pgfqpoint{1.797021in}{2.103077in}}%
\pgfpathcurveto{\pgfqpoint{1.797021in}{2.094841in}}{\pgfqpoint{1.800293in}{2.086941in}}{\pgfqpoint{1.806117in}{2.081117in}}%
\pgfpathcurveto{\pgfqpoint{1.811941in}{2.075293in}}{\pgfqpoint{1.819841in}{2.072021in}}{\pgfqpoint{1.828078in}{2.072021in}}%
\pgfpathclose%
\pgfusepath{stroke,fill}%
\end{pgfscope}%
\begin{pgfscope}%
\pgfpathrectangle{\pgfqpoint{0.100000in}{0.212622in}}{\pgfqpoint{3.696000in}{3.696000in}}%
\pgfusepath{clip}%
\pgfsetbuttcap%
\pgfsetroundjoin%
\definecolor{currentfill}{rgb}{0.121569,0.466667,0.705882}%
\pgfsetfillcolor{currentfill}%
\pgfsetfillopacity{0.962683}%
\pgfsetlinewidth{1.003750pt}%
\definecolor{currentstroke}{rgb}{0.121569,0.466667,0.705882}%
\pgfsetstrokecolor{currentstroke}%
\pgfsetstrokeopacity{0.962683}%
\pgfsetdash{}{0pt}%
\pgfpathmoveto{\pgfqpoint{2.027606in}{2.135630in}}%
\pgfpathcurveto{\pgfqpoint{2.035842in}{2.135630in}}{\pgfqpoint{2.043742in}{2.138902in}}{\pgfqpoint{2.049566in}{2.144726in}}%
\pgfpathcurveto{\pgfqpoint{2.055390in}{2.150550in}}{\pgfqpoint{2.058662in}{2.158450in}}{\pgfqpoint{2.058662in}{2.166687in}}%
\pgfpathcurveto{\pgfqpoint{2.058662in}{2.174923in}}{\pgfqpoint{2.055390in}{2.182823in}}{\pgfqpoint{2.049566in}{2.188647in}}%
\pgfpathcurveto{\pgfqpoint{2.043742in}{2.194471in}}{\pgfqpoint{2.035842in}{2.197743in}}{\pgfqpoint{2.027606in}{2.197743in}}%
\pgfpathcurveto{\pgfqpoint{2.019369in}{2.197743in}}{\pgfqpoint{2.011469in}{2.194471in}}{\pgfqpoint{2.005645in}{2.188647in}}%
\pgfpathcurveto{\pgfqpoint{1.999821in}{2.182823in}}{\pgfqpoint{1.996549in}{2.174923in}}{\pgfqpoint{1.996549in}{2.166687in}}%
\pgfpathcurveto{\pgfqpoint{1.996549in}{2.158450in}}{\pgfqpoint{1.999821in}{2.150550in}}{\pgfqpoint{2.005645in}{2.144726in}}%
\pgfpathcurveto{\pgfqpoint{2.011469in}{2.138902in}}{\pgfqpoint{2.019369in}{2.135630in}}{\pgfqpoint{2.027606in}{2.135630in}}%
\pgfpathclose%
\pgfusepath{stroke,fill}%
\end{pgfscope}%
\begin{pgfscope}%
\pgfpathrectangle{\pgfqpoint{0.100000in}{0.212622in}}{\pgfqpoint{3.696000in}{3.696000in}}%
\pgfusepath{clip}%
\pgfsetbuttcap%
\pgfsetroundjoin%
\definecolor{currentfill}{rgb}{0.121569,0.466667,0.705882}%
\pgfsetfillcolor{currentfill}%
\pgfsetfillopacity{0.962786}%
\pgfsetlinewidth{1.003750pt}%
\definecolor{currentstroke}{rgb}{0.121569,0.466667,0.705882}%
\pgfsetstrokecolor{currentstroke}%
\pgfsetstrokeopacity{0.962786}%
\pgfsetdash{}{0pt}%
\pgfpathmoveto{\pgfqpoint{2.990520in}{2.557495in}}%
\pgfpathcurveto{\pgfqpoint{2.998756in}{2.557495in}}{\pgfqpoint{3.006656in}{2.560767in}}{\pgfqpoint{3.012480in}{2.566591in}}%
\pgfpathcurveto{\pgfqpoint{3.018304in}{2.572415in}}{\pgfqpoint{3.021576in}{2.580315in}}{\pgfqpoint{3.021576in}{2.588552in}}%
\pgfpathcurveto{\pgfqpoint{3.021576in}{2.596788in}}{\pgfqpoint{3.018304in}{2.604688in}}{\pgfqpoint{3.012480in}{2.610512in}}%
\pgfpathcurveto{\pgfqpoint{3.006656in}{2.616336in}}{\pgfqpoint{2.998756in}{2.619608in}}{\pgfqpoint{2.990520in}{2.619608in}}%
\pgfpathcurveto{\pgfqpoint{2.982283in}{2.619608in}}{\pgfqpoint{2.974383in}{2.616336in}}{\pgfqpoint{2.968559in}{2.610512in}}%
\pgfpathcurveto{\pgfqpoint{2.962735in}{2.604688in}}{\pgfqpoint{2.959463in}{2.596788in}}{\pgfqpoint{2.959463in}{2.588552in}}%
\pgfpathcurveto{\pgfqpoint{2.959463in}{2.580315in}}{\pgfqpoint{2.962735in}{2.572415in}}{\pgfqpoint{2.968559in}{2.566591in}}%
\pgfpathcurveto{\pgfqpoint{2.974383in}{2.560767in}}{\pgfqpoint{2.982283in}{2.557495in}}{\pgfqpoint{2.990520in}{2.557495in}}%
\pgfpathclose%
\pgfusepath{stroke,fill}%
\end{pgfscope}%
\begin{pgfscope}%
\pgfpathrectangle{\pgfqpoint{0.100000in}{0.212622in}}{\pgfqpoint{3.696000in}{3.696000in}}%
\pgfusepath{clip}%
\pgfsetbuttcap%
\pgfsetroundjoin%
\definecolor{currentfill}{rgb}{0.121569,0.466667,0.705882}%
\pgfsetfillcolor{currentfill}%
\pgfsetfillopacity{0.963494}%
\pgfsetlinewidth{1.003750pt}%
\definecolor{currentstroke}{rgb}{0.121569,0.466667,0.705882}%
\pgfsetstrokecolor{currentstroke}%
\pgfsetstrokeopacity{0.963494}%
\pgfsetdash{}{0pt}%
\pgfpathmoveto{\pgfqpoint{2.076506in}{2.115887in}}%
\pgfpathcurveto{\pgfqpoint{2.084742in}{2.115887in}}{\pgfqpoint{2.092642in}{2.119160in}}{\pgfqpoint{2.098466in}{2.124984in}}%
\pgfpathcurveto{\pgfqpoint{2.104290in}{2.130808in}}{\pgfqpoint{2.107562in}{2.138708in}}{\pgfqpoint{2.107562in}{2.146944in}}%
\pgfpathcurveto{\pgfqpoint{2.107562in}{2.155180in}}{\pgfqpoint{2.104290in}{2.163080in}}{\pgfqpoint{2.098466in}{2.168904in}}%
\pgfpathcurveto{\pgfqpoint{2.092642in}{2.174728in}}{\pgfqpoint{2.084742in}{2.178000in}}{\pgfqpoint{2.076506in}{2.178000in}}%
\pgfpathcurveto{\pgfqpoint{2.068269in}{2.178000in}}{\pgfqpoint{2.060369in}{2.174728in}}{\pgfqpoint{2.054545in}{2.168904in}}%
\pgfpathcurveto{\pgfqpoint{2.048721in}{2.163080in}}{\pgfqpoint{2.045449in}{2.155180in}}{\pgfqpoint{2.045449in}{2.146944in}}%
\pgfpathcurveto{\pgfqpoint{2.045449in}{2.138708in}}{\pgfqpoint{2.048721in}{2.130808in}}{\pgfqpoint{2.054545in}{2.124984in}}%
\pgfpathcurveto{\pgfqpoint{2.060369in}{2.119160in}}{\pgfqpoint{2.068269in}{2.115887in}}{\pgfqpoint{2.076506in}{2.115887in}}%
\pgfpathclose%
\pgfusepath{stroke,fill}%
\end{pgfscope}%
\begin{pgfscope}%
\pgfpathrectangle{\pgfqpoint{0.100000in}{0.212622in}}{\pgfqpoint{3.696000in}{3.696000in}}%
\pgfusepath{clip}%
\pgfsetbuttcap%
\pgfsetroundjoin%
\definecolor{currentfill}{rgb}{0.121569,0.466667,0.705882}%
\pgfsetfillcolor{currentfill}%
\pgfsetfillopacity{0.964724}%
\pgfsetlinewidth{1.003750pt}%
\definecolor{currentstroke}{rgb}{0.121569,0.466667,0.705882}%
\pgfsetstrokecolor{currentstroke}%
\pgfsetstrokeopacity{0.964724}%
\pgfsetdash{}{0pt}%
\pgfpathmoveto{\pgfqpoint{1.291142in}{1.776419in}}%
\pgfpathcurveto{\pgfqpoint{1.299379in}{1.776419in}}{\pgfqpoint{1.307279in}{1.779691in}}{\pgfqpoint{1.313103in}{1.785515in}}%
\pgfpathcurveto{\pgfqpoint{1.318927in}{1.791339in}}{\pgfqpoint{1.322199in}{1.799239in}}{\pgfqpoint{1.322199in}{1.807475in}}%
\pgfpathcurveto{\pgfqpoint{1.322199in}{1.815712in}}{\pgfqpoint{1.318927in}{1.823612in}}{\pgfqpoint{1.313103in}{1.829436in}}%
\pgfpathcurveto{\pgfqpoint{1.307279in}{1.835260in}}{\pgfqpoint{1.299379in}{1.838532in}}{\pgfqpoint{1.291142in}{1.838532in}}%
\pgfpathcurveto{\pgfqpoint{1.282906in}{1.838532in}}{\pgfqpoint{1.275006in}{1.835260in}}{\pgfqpoint{1.269182in}{1.829436in}}%
\pgfpathcurveto{\pgfqpoint{1.263358in}{1.823612in}}{\pgfqpoint{1.260086in}{1.815712in}}{\pgfqpoint{1.260086in}{1.807475in}}%
\pgfpathcurveto{\pgfqpoint{1.260086in}{1.799239in}}{\pgfqpoint{1.263358in}{1.791339in}}{\pgfqpoint{1.269182in}{1.785515in}}%
\pgfpathcurveto{\pgfqpoint{1.275006in}{1.779691in}}{\pgfqpoint{1.282906in}{1.776419in}}{\pgfqpoint{1.291142in}{1.776419in}}%
\pgfpathclose%
\pgfusepath{stroke,fill}%
\end{pgfscope}%
\begin{pgfscope}%
\pgfpathrectangle{\pgfqpoint{0.100000in}{0.212622in}}{\pgfqpoint{3.696000in}{3.696000in}}%
\pgfusepath{clip}%
\pgfsetbuttcap%
\pgfsetroundjoin%
\definecolor{currentfill}{rgb}{0.121569,0.466667,0.705882}%
\pgfsetfillcolor{currentfill}%
\pgfsetfillopacity{0.964817}%
\pgfsetlinewidth{1.003750pt}%
\definecolor{currentstroke}{rgb}{0.121569,0.466667,0.705882}%
\pgfsetstrokecolor{currentstroke}%
\pgfsetstrokeopacity{0.964817}%
\pgfsetdash{}{0pt}%
\pgfpathmoveto{\pgfqpoint{1.279530in}{1.786955in}}%
\pgfpathcurveto{\pgfqpoint{1.287766in}{1.786955in}}{\pgfqpoint{1.295666in}{1.790227in}}{\pgfqpoint{1.301490in}{1.796051in}}%
\pgfpathcurveto{\pgfqpoint{1.307314in}{1.801875in}}{\pgfqpoint{1.310586in}{1.809775in}}{\pgfqpoint{1.310586in}{1.818012in}}%
\pgfpathcurveto{\pgfqpoint{1.310586in}{1.826248in}}{\pgfqpoint{1.307314in}{1.834148in}}{\pgfqpoint{1.301490in}{1.839972in}}%
\pgfpathcurveto{\pgfqpoint{1.295666in}{1.845796in}}{\pgfqpoint{1.287766in}{1.849068in}}{\pgfqpoint{1.279530in}{1.849068in}}%
\pgfpathcurveto{\pgfqpoint{1.271294in}{1.849068in}}{\pgfqpoint{1.263394in}{1.845796in}}{\pgfqpoint{1.257570in}{1.839972in}}%
\pgfpathcurveto{\pgfqpoint{1.251746in}{1.834148in}}{\pgfqpoint{1.248473in}{1.826248in}}{\pgfqpoint{1.248473in}{1.818012in}}%
\pgfpathcurveto{\pgfqpoint{1.248473in}{1.809775in}}{\pgfqpoint{1.251746in}{1.801875in}}{\pgfqpoint{1.257570in}{1.796051in}}%
\pgfpathcurveto{\pgfqpoint{1.263394in}{1.790227in}}{\pgfqpoint{1.271294in}{1.786955in}}{\pgfqpoint{1.279530in}{1.786955in}}%
\pgfpathclose%
\pgfusepath{stroke,fill}%
\end{pgfscope}%
\begin{pgfscope}%
\pgfpathrectangle{\pgfqpoint{0.100000in}{0.212622in}}{\pgfqpoint{3.696000in}{3.696000in}}%
\pgfusepath{clip}%
\pgfsetbuttcap%
\pgfsetroundjoin%
\definecolor{currentfill}{rgb}{0.121569,0.466667,0.705882}%
\pgfsetfillcolor{currentfill}%
\pgfsetfillopacity{0.965091}%
\pgfsetlinewidth{1.003750pt}%
\definecolor{currentstroke}{rgb}{0.121569,0.466667,0.705882}%
\pgfsetstrokecolor{currentstroke}%
\pgfsetstrokeopacity{0.965091}%
\pgfsetdash{}{0pt}%
\pgfpathmoveto{\pgfqpoint{2.964475in}{2.524424in}}%
\pgfpathcurveto{\pgfqpoint{2.972711in}{2.524424in}}{\pgfqpoint{2.980611in}{2.527697in}}{\pgfqpoint{2.986435in}{2.533521in}}%
\pgfpathcurveto{\pgfqpoint{2.992259in}{2.539345in}}{\pgfqpoint{2.995531in}{2.547245in}}{\pgfqpoint{2.995531in}{2.555481in}}%
\pgfpathcurveto{\pgfqpoint{2.995531in}{2.563717in}}{\pgfqpoint{2.992259in}{2.571617in}}{\pgfqpoint{2.986435in}{2.577441in}}%
\pgfpathcurveto{\pgfqpoint{2.980611in}{2.583265in}}{\pgfqpoint{2.972711in}{2.586537in}}{\pgfqpoint{2.964475in}{2.586537in}}%
\pgfpathcurveto{\pgfqpoint{2.956239in}{2.586537in}}{\pgfqpoint{2.948339in}{2.583265in}}{\pgfqpoint{2.942515in}{2.577441in}}%
\pgfpathcurveto{\pgfqpoint{2.936691in}{2.571617in}}{\pgfqpoint{2.933418in}{2.563717in}}{\pgfqpoint{2.933418in}{2.555481in}}%
\pgfpathcurveto{\pgfqpoint{2.933418in}{2.547245in}}{\pgfqpoint{2.936691in}{2.539345in}}{\pgfqpoint{2.942515in}{2.533521in}}%
\pgfpathcurveto{\pgfqpoint{2.948339in}{2.527697in}}{\pgfqpoint{2.956239in}{2.524424in}}{\pgfqpoint{2.964475in}{2.524424in}}%
\pgfpathclose%
\pgfusepath{stroke,fill}%
\end{pgfscope}%
\begin{pgfscope}%
\pgfpathrectangle{\pgfqpoint{0.100000in}{0.212622in}}{\pgfqpoint{3.696000in}{3.696000in}}%
\pgfusepath{clip}%
\pgfsetbuttcap%
\pgfsetroundjoin%
\definecolor{currentfill}{rgb}{0.121569,0.466667,0.705882}%
\pgfsetfillcolor{currentfill}%
\pgfsetfillopacity{0.965333}%
\pgfsetlinewidth{1.003750pt}%
\definecolor{currentstroke}{rgb}{0.121569,0.466667,0.705882}%
\pgfsetstrokecolor{currentstroke}%
\pgfsetstrokeopacity{0.965333}%
\pgfsetdash{}{0pt}%
\pgfpathmoveto{\pgfqpoint{1.811268in}{2.062658in}}%
\pgfpathcurveto{\pgfqpoint{1.819505in}{2.062658in}}{\pgfqpoint{1.827405in}{2.065931in}}{\pgfqpoint{1.833229in}{2.071755in}}%
\pgfpathcurveto{\pgfqpoint{1.839053in}{2.077579in}}{\pgfqpoint{1.842325in}{2.085479in}}{\pgfqpoint{1.842325in}{2.093715in}}%
\pgfpathcurveto{\pgfqpoint{1.842325in}{2.101951in}}{\pgfqpoint{1.839053in}{2.109851in}}{\pgfqpoint{1.833229in}{2.115675in}}%
\pgfpathcurveto{\pgfqpoint{1.827405in}{2.121499in}}{\pgfqpoint{1.819505in}{2.124771in}}{\pgfqpoint{1.811268in}{2.124771in}}%
\pgfpathcurveto{\pgfqpoint{1.803032in}{2.124771in}}{\pgfqpoint{1.795132in}{2.121499in}}{\pgfqpoint{1.789308in}{2.115675in}}%
\pgfpathcurveto{\pgfqpoint{1.783484in}{2.109851in}}{\pgfqpoint{1.780212in}{2.101951in}}{\pgfqpoint{1.780212in}{2.093715in}}%
\pgfpathcurveto{\pgfqpoint{1.780212in}{2.085479in}}{\pgfqpoint{1.783484in}{2.077579in}}{\pgfqpoint{1.789308in}{2.071755in}}%
\pgfpathcurveto{\pgfqpoint{1.795132in}{2.065931in}}{\pgfqpoint{1.803032in}{2.062658in}}{\pgfqpoint{1.811268in}{2.062658in}}%
\pgfpathclose%
\pgfusepath{stroke,fill}%
\end{pgfscope}%
\begin{pgfscope}%
\pgfpathrectangle{\pgfqpoint{0.100000in}{0.212622in}}{\pgfqpoint{3.696000in}{3.696000in}}%
\pgfusepath{clip}%
\pgfsetbuttcap%
\pgfsetroundjoin%
\definecolor{currentfill}{rgb}{0.121569,0.466667,0.705882}%
\pgfsetfillcolor{currentfill}%
\pgfsetfillopacity{0.965946}%
\pgfsetlinewidth{1.003750pt}%
\definecolor{currentstroke}{rgb}{0.121569,0.466667,0.705882}%
\pgfsetstrokecolor{currentstroke}%
\pgfsetstrokeopacity{0.965946}%
\pgfsetdash{}{0pt}%
\pgfpathmoveto{\pgfqpoint{1.284717in}{1.790083in}}%
\pgfpathcurveto{\pgfqpoint{1.292954in}{1.790083in}}{\pgfqpoint{1.300854in}{1.793355in}}{\pgfqpoint{1.306678in}{1.799179in}}%
\pgfpathcurveto{\pgfqpoint{1.312502in}{1.805003in}}{\pgfqpoint{1.315774in}{1.812903in}}{\pgfqpoint{1.315774in}{1.821140in}}%
\pgfpathcurveto{\pgfqpoint{1.315774in}{1.829376in}}{\pgfqpoint{1.312502in}{1.837276in}}{\pgfqpoint{1.306678in}{1.843100in}}%
\pgfpathcurveto{\pgfqpoint{1.300854in}{1.848924in}}{\pgfqpoint{1.292954in}{1.852196in}}{\pgfqpoint{1.284717in}{1.852196in}}%
\pgfpathcurveto{\pgfqpoint{1.276481in}{1.852196in}}{\pgfqpoint{1.268581in}{1.848924in}}{\pgfqpoint{1.262757in}{1.843100in}}%
\pgfpathcurveto{\pgfqpoint{1.256933in}{1.837276in}}{\pgfqpoint{1.253661in}{1.829376in}}{\pgfqpoint{1.253661in}{1.821140in}}%
\pgfpathcurveto{\pgfqpoint{1.253661in}{1.812903in}}{\pgfqpoint{1.256933in}{1.805003in}}{\pgfqpoint{1.262757in}{1.799179in}}%
\pgfpathcurveto{\pgfqpoint{1.268581in}{1.793355in}}{\pgfqpoint{1.276481in}{1.790083in}}{\pgfqpoint{1.284717in}{1.790083in}}%
\pgfpathclose%
\pgfusepath{stroke,fill}%
\end{pgfscope}%
\begin{pgfscope}%
\pgfpathrectangle{\pgfqpoint{0.100000in}{0.212622in}}{\pgfqpoint{3.696000in}{3.696000in}}%
\pgfusepath{clip}%
\pgfsetbuttcap%
\pgfsetroundjoin%
\definecolor{currentfill}{rgb}{0.121569,0.466667,0.705882}%
\pgfsetfillcolor{currentfill}%
\pgfsetfillopacity{0.966041}%
\pgfsetlinewidth{1.003750pt}%
\definecolor{currentstroke}{rgb}{0.121569,0.466667,0.705882}%
\pgfsetstrokecolor{currentstroke}%
\pgfsetstrokeopacity{0.966041}%
\pgfsetdash{}{0pt}%
\pgfpathmoveto{\pgfqpoint{2.009930in}{2.117326in}}%
\pgfpathcurveto{\pgfqpoint{2.018166in}{2.117326in}}{\pgfqpoint{2.026066in}{2.120598in}}{\pgfqpoint{2.031890in}{2.126422in}}%
\pgfpathcurveto{\pgfqpoint{2.037714in}{2.132246in}}{\pgfqpoint{2.040986in}{2.140146in}}{\pgfqpoint{2.040986in}{2.148382in}}%
\pgfpathcurveto{\pgfqpoint{2.040986in}{2.156619in}}{\pgfqpoint{2.037714in}{2.164519in}}{\pgfqpoint{2.031890in}{2.170343in}}%
\pgfpathcurveto{\pgfqpoint{2.026066in}{2.176166in}}{\pgfqpoint{2.018166in}{2.179439in}}{\pgfqpoint{2.009930in}{2.179439in}}%
\pgfpathcurveto{\pgfqpoint{2.001694in}{2.179439in}}{\pgfqpoint{1.993793in}{2.176166in}}{\pgfqpoint{1.987970in}{2.170343in}}%
\pgfpathcurveto{\pgfqpoint{1.982146in}{2.164519in}}{\pgfqpoint{1.978873in}{2.156619in}}{\pgfqpoint{1.978873in}{2.148382in}}%
\pgfpathcurveto{\pgfqpoint{1.978873in}{2.140146in}}{\pgfqpoint{1.982146in}{2.132246in}}{\pgfqpoint{1.987970in}{2.126422in}}%
\pgfpathcurveto{\pgfqpoint{1.993793in}{2.120598in}}{\pgfqpoint{2.001694in}{2.117326in}}{\pgfqpoint{2.009930in}{2.117326in}}%
\pgfpathclose%
\pgfusepath{stroke,fill}%
\end{pgfscope}%
\begin{pgfscope}%
\pgfpathrectangle{\pgfqpoint{0.100000in}{0.212622in}}{\pgfqpoint{3.696000in}{3.696000in}}%
\pgfusepath{clip}%
\pgfsetbuttcap%
\pgfsetroundjoin%
\definecolor{currentfill}{rgb}{0.121569,0.466667,0.705882}%
\pgfsetfillcolor{currentfill}%
\pgfsetfillopacity{0.966149}%
\pgfsetlinewidth{1.003750pt}%
\definecolor{currentstroke}{rgb}{0.121569,0.466667,0.705882}%
\pgfsetstrokecolor{currentstroke}%
\pgfsetstrokeopacity{0.966149}%
\pgfsetdash{}{0pt}%
\pgfpathmoveto{\pgfqpoint{1.830990in}{2.070211in}}%
\pgfpathcurveto{\pgfqpoint{1.839227in}{2.070211in}}{\pgfqpoint{1.847127in}{2.073483in}}{\pgfqpoint{1.852951in}{2.079307in}}%
\pgfpathcurveto{\pgfqpoint{1.858775in}{2.085131in}}{\pgfqpoint{1.862047in}{2.093031in}}{\pgfqpoint{1.862047in}{2.101268in}}%
\pgfpathcurveto{\pgfqpoint{1.862047in}{2.109504in}}{\pgfqpoint{1.858775in}{2.117404in}}{\pgfqpoint{1.852951in}{2.123228in}}%
\pgfpathcurveto{\pgfqpoint{1.847127in}{2.129052in}}{\pgfqpoint{1.839227in}{2.132324in}}{\pgfqpoint{1.830990in}{2.132324in}}%
\pgfpathcurveto{\pgfqpoint{1.822754in}{2.132324in}}{\pgfqpoint{1.814854in}{2.129052in}}{\pgfqpoint{1.809030in}{2.123228in}}%
\pgfpathcurveto{\pgfqpoint{1.803206in}{2.117404in}}{\pgfqpoint{1.799934in}{2.109504in}}{\pgfqpoint{1.799934in}{2.101268in}}%
\pgfpathcurveto{\pgfqpoint{1.799934in}{2.093031in}}{\pgfqpoint{1.803206in}{2.085131in}}{\pgfqpoint{1.809030in}{2.079307in}}%
\pgfpathcurveto{\pgfqpoint{1.814854in}{2.073483in}}{\pgfqpoint{1.822754in}{2.070211in}}{\pgfqpoint{1.830990in}{2.070211in}}%
\pgfpathclose%
\pgfusepath{stroke,fill}%
\end{pgfscope}%
\begin{pgfscope}%
\pgfpathrectangle{\pgfqpoint{0.100000in}{0.212622in}}{\pgfqpoint{3.696000in}{3.696000in}}%
\pgfusepath{clip}%
\pgfsetbuttcap%
\pgfsetroundjoin%
\definecolor{currentfill}{rgb}{0.121569,0.466667,0.705882}%
\pgfsetfillcolor{currentfill}%
\pgfsetfillopacity{0.966285}%
\pgfsetlinewidth{1.003750pt}%
\definecolor{currentstroke}{rgb}{0.121569,0.466667,0.705882}%
\pgfsetstrokecolor{currentstroke}%
\pgfsetstrokeopacity{0.966285}%
\pgfsetdash{}{0pt}%
\pgfpathmoveto{\pgfqpoint{1.963551in}{2.107822in}}%
\pgfpathcurveto{\pgfqpoint{1.971787in}{2.107822in}}{\pgfqpoint{1.979687in}{2.111094in}}{\pgfqpoint{1.985511in}{2.116918in}}%
\pgfpathcurveto{\pgfqpoint{1.991335in}{2.122742in}}{\pgfqpoint{1.994607in}{2.130642in}}{\pgfqpoint{1.994607in}{2.138878in}}%
\pgfpathcurveto{\pgfqpoint{1.994607in}{2.147115in}}{\pgfqpoint{1.991335in}{2.155015in}}{\pgfqpoint{1.985511in}{2.160839in}}%
\pgfpathcurveto{\pgfqpoint{1.979687in}{2.166663in}}{\pgfqpoint{1.971787in}{2.169935in}}{\pgfqpoint{1.963551in}{2.169935in}}%
\pgfpathcurveto{\pgfqpoint{1.955314in}{2.169935in}}{\pgfqpoint{1.947414in}{2.166663in}}{\pgfqpoint{1.941590in}{2.160839in}}%
\pgfpathcurveto{\pgfqpoint{1.935767in}{2.155015in}}{\pgfqpoint{1.932494in}{2.147115in}}{\pgfqpoint{1.932494in}{2.138878in}}%
\pgfpathcurveto{\pgfqpoint{1.932494in}{2.130642in}}{\pgfqpoint{1.935767in}{2.122742in}}{\pgfqpoint{1.941590in}{2.116918in}}%
\pgfpathcurveto{\pgfqpoint{1.947414in}{2.111094in}}{\pgfqpoint{1.955314in}{2.107822in}}{\pgfqpoint{1.963551in}{2.107822in}}%
\pgfpathclose%
\pgfusepath{stroke,fill}%
\end{pgfscope}%
\begin{pgfscope}%
\pgfpathrectangle{\pgfqpoint{0.100000in}{0.212622in}}{\pgfqpoint{3.696000in}{3.696000in}}%
\pgfusepath{clip}%
\pgfsetbuttcap%
\pgfsetroundjoin%
\definecolor{currentfill}{rgb}{0.121569,0.466667,0.705882}%
\pgfsetfillcolor{currentfill}%
\pgfsetfillopacity{0.966349}%
\pgfsetlinewidth{1.003750pt}%
\definecolor{currentstroke}{rgb}{0.121569,0.466667,0.705882}%
\pgfsetstrokecolor{currentstroke}%
\pgfsetstrokeopacity{0.966349}%
\pgfsetdash{}{0pt}%
\pgfpathmoveto{\pgfqpoint{2.982037in}{2.550057in}}%
\pgfpathcurveto{\pgfqpoint{2.990273in}{2.550057in}}{\pgfqpoint{2.998173in}{2.553329in}}{\pgfqpoint{3.003997in}{2.559153in}}%
\pgfpathcurveto{\pgfqpoint{3.009821in}{2.564977in}}{\pgfqpoint{3.013093in}{2.572877in}}{\pgfqpoint{3.013093in}{2.581113in}}%
\pgfpathcurveto{\pgfqpoint{3.013093in}{2.589350in}}{\pgfqpoint{3.009821in}{2.597250in}}{\pgfqpoint{3.003997in}{2.603074in}}%
\pgfpathcurveto{\pgfqpoint{2.998173in}{2.608898in}}{\pgfqpoint{2.990273in}{2.612170in}}{\pgfqpoint{2.982037in}{2.612170in}}%
\pgfpathcurveto{\pgfqpoint{2.973800in}{2.612170in}}{\pgfqpoint{2.965900in}{2.608898in}}{\pgfqpoint{2.960076in}{2.603074in}}%
\pgfpathcurveto{\pgfqpoint{2.954253in}{2.597250in}}{\pgfqpoint{2.950980in}{2.589350in}}{\pgfqpoint{2.950980in}{2.581113in}}%
\pgfpathcurveto{\pgfqpoint{2.950980in}{2.572877in}}{\pgfqpoint{2.954253in}{2.564977in}}{\pgfqpoint{2.960076in}{2.559153in}}%
\pgfpathcurveto{\pgfqpoint{2.965900in}{2.553329in}}{\pgfqpoint{2.973800in}{2.550057in}}{\pgfqpoint{2.982037in}{2.550057in}}%
\pgfpathclose%
\pgfusepath{stroke,fill}%
\end{pgfscope}%
\begin{pgfscope}%
\pgfpathrectangle{\pgfqpoint{0.100000in}{0.212622in}}{\pgfqpoint{3.696000in}{3.696000in}}%
\pgfusepath{clip}%
\pgfsetbuttcap%
\pgfsetroundjoin%
\definecolor{currentfill}{rgb}{0.121569,0.466667,0.705882}%
\pgfsetfillcolor{currentfill}%
\pgfsetfillopacity{0.967271}%
\pgfsetlinewidth{1.003750pt}%
\definecolor{currentstroke}{rgb}{0.121569,0.466667,0.705882}%
\pgfsetstrokecolor{currentstroke}%
\pgfsetstrokeopacity{0.967271}%
\pgfsetdash{}{0pt}%
\pgfpathmoveto{\pgfqpoint{1.957422in}{2.116972in}}%
\pgfpathcurveto{\pgfqpoint{1.965658in}{2.116972in}}{\pgfqpoint{1.973558in}{2.120244in}}{\pgfqpoint{1.979382in}{2.126068in}}%
\pgfpathcurveto{\pgfqpoint{1.985206in}{2.131892in}}{\pgfqpoint{1.988478in}{2.139792in}}{\pgfqpoint{1.988478in}{2.148028in}}%
\pgfpathcurveto{\pgfqpoint{1.988478in}{2.156265in}}{\pgfqpoint{1.985206in}{2.164165in}}{\pgfqpoint{1.979382in}{2.169989in}}%
\pgfpathcurveto{\pgfqpoint{1.973558in}{2.175812in}}{\pgfqpoint{1.965658in}{2.179085in}}{\pgfqpoint{1.957422in}{2.179085in}}%
\pgfpathcurveto{\pgfqpoint{1.949185in}{2.179085in}}{\pgfqpoint{1.941285in}{2.175812in}}{\pgfqpoint{1.935461in}{2.169989in}}%
\pgfpathcurveto{\pgfqpoint{1.929637in}{2.164165in}}{\pgfqpoint{1.926365in}{2.156265in}}{\pgfqpoint{1.926365in}{2.148028in}}%
\pgfpathcurveto{\pgfqpoint{1.926365in}{2.139792in}}{\pgfqpoint{1.929637in}{2.131892in}}{\pgfqpoint{1.935461in}{2.126068in}}%
\pgfpathcurveto{\pgfqpoint{1.941285in}{2.120244in}}{\pgfqpoint{1.949185in}{2.116972in}}{\pgfqpoint{1.957422in}{2.116972in}}%
\pgfpathclose%
\pgfusepath{stroke,fill}%
\end{pgfscope}%
\begin{pgfscope}%
\pgfpathrectangle{\pgfqpoint{0.100000in}{0.212622in}}{\pgfqpoint{3.696000in}{3.696000in}}%
\pgfusepath{clip}%
\pgfsetbuttcap%
\pgfsetroundjoin%
\definecolor{currentfill}{rgb}{0.121569,0.466667,0.705882}%
\pgfsetfillcolor{currentfill}%
\pgfsetfillopacity{0.967394}%
\pgfsetlinewidth{1.003750pt}%
\definecolor{currentstroke}{rgb}{0.121569,0.466667,0.705882}%
\pgfsetstrokecolor{currentstroke}%
\pgfsetstrokeopacity{0.967394}%
\pgfsetdash{}{0pt}%
\pgfpathmoveto{\pgfqpoint{2.046544in}{2.127197in}}%
\pgfpathcurveto{\pgfqpoint{2.054780in}{2.127197in}}{\pgfqpoint{2.062680in}{2.130470in}}{\pgfqpoint{2.068504in}{2.136293in}}%
\pgfpathcurveto{\pgfqpoint{2.074328in}{2.142117in}}{\pgfqpoint{2.077600in}{2.150017in}}{\pgfqpoint{2.077600in}{2.158254in}}%
\pgfpathcurveto{\pgfqpoint{2.077600in}{2.166490in}}{\pgfqpoint{2.074328in}{2.174390in}}{\pgfqpoint{2.068504in}{2.180214in}}%
\pgfpathcurveto{\pgfqpoint{2.062680in}{2.186038in}}{\pgfqpoint{2.054780in}{2.189310in}}{\pgfqpoint{2.046544in}{2.189310in}}%
\pgfpathcurveto{\pgfqpoint{2.038307in}{2.189310in}}{\pgfqpoint{2.030407in}{2.186038in}}{\pgfqpoint{2.024584in}{2.180214in}}%
\pgfpathcurveto{\pgfqpoint{2.018760in}{2.174390in}}{\pgfqpoint{2.015487in}{2.166490in}}{\pgfqpoint{2.015487in}{2.158254in}}%
\pgfpathcurveto{\pgfqpoint{2.015487in}{2.150017in}}{\pgfqpoint{2.018760in}{2.142117in}}{\pgfqpoint{2.024584in}{2.136293in}}%
\pgfpathcurveto{\pgfqpoint{2.030407in}{2.130470in}}{\pgfqpoint{2.038307in}{2.127197in}}{\pgfqpoint{2.046544in}{2.127197in}}%
\pgfpathclose%
\pgfusepath{stroke,fill}%
\end{pgfscope}%
\begin{pgfscope}%
\pgfpathrectangle{\pgfqpoint{0.100000in}{0.212622in}}{\pgfqpoint{3.696000in}{3.696000in}}%
\pgfusepath{clip}%
\pgfsetbuttcap%
\pgfsetroundjoin%
\definecolor{currentfill}{rgb}{0.121569,0.466667,0.705882}%
\pgfsetfillcolor{currentfill}%
\pgfsetfillopacity{0.968886}%
\pgfsetlinewidth{1.003750pt}%
\definecolor{currentstroke}{rgb}{0.121569,0.466667,0.705882}%
\pgfsetstrokecolor{currentstroke}%
\pgfsetstrokeopacity{0.968886}%
\pgfsetdash{}{0pt}%
\pgfpathmoveto{\pgfqpoint{1.303297in}{1.769372in}}%
\pgfpathcurveto{\pgfqpoint{1.311534in}{1.769372in}}{\pgfqpoint{1.319434in}{1.772644in}}{\pgfqpoint{1.325258in}{1.778468in}}%
\pgfpathcurveto{\pgfqpoint{1.331082in}{1.784292in}}{\pgfqpoint{1.334354in}{1.792192in}}{\pgfqpoint{1.334354in}{1.800428in}}%
\pgfpathcurveto{\pgfqpoint{1.334354in}{1.808665in}}{\pgfqpoint{1.331082in}{1.816565in}}{\pgfqpoint{1.325258in}{1.822389in}}%
\pgfpathcurveto{\pgfqpoint{1.319434in}{1.828212in}}{\pgfqpoint{1.311534in}{1.831485in}}{\pgfqpoint{1.303297in}{1.831485in}}%
\pgfpathcurveto{\pgfqpoint{1.295061in}{1.831485in}}{\pgfqpoint{1.287161in}{1.828212in}}{\pgfqpoint{1.281337in}{1.822389in}}%
\pgfpathcurveto{\pgfqpoint{1.275513in}{1.816565in}}{\pgfqpoint{1.272241in}{1.808665in}}{\pgfqpoint{1.272241in}{1.800428in}}%
\pgfpathcurveto{\pgfqpoint{1.272241in}{1.792192in}}{\pgfqpoint{1.275513in}{1.784292in}}{\pgfqpoint{1.281337in}{1.778468in}}%
\pgfpathcurveto{\pgfqpoint{1.287161in}{1.772644in}}{\pgfqpoint{1.295061in}{1.769372in}}{\pgfqpoint{1.303297in}{1.769372in}}%
\pgfpathclose%
\pgfusepath{stroke,fill}%
\end{pgfscope}%
\begin{pgfscope}%
\pgfpathrectangle{\pgfqpoint{0.100000in}{0.212622in}}{\pgfqpoint{3.696000in}{3.696000in}}%
\pgfusepath{clip}%
\pgfsetbuttcap%
\pgfsetroundjoin%
\definecolor{currentfill}{rgb}{0.121569,0.466667,0.705882}%
\pgfsetfillcolor{currentfill}%
\pgfsetfillopacity{0.970375}%
\pgfsetlinewidth{1.003750pt}%
\definecolor{currentstroke}{rgb}{0.121569,0.466667,0.705882}%
\pgfsetstrokecolor{currentstroke}%
\pgfsetstrokeopacity{0.970375}%
\pgfsetdash{}{0pt}%
\pgfpathmoveto{\pgfqpoint{2.957334in}{2.516294in}}%
\pgfpathcurveto{\pgfqpoint{2.965570in}{2.516294in}}{\pgfqpoint{2.973470in}{2.519567in}}{\pgfqpoint{2.979294in}{2.525391in}}%
\pgfpathcurveto{\pgfqpoint{2.985118in}{2.531214in}}{\pgfqpoint{2.988390in}{2.539115in}}{\pgfqpoint{2.988390in}{2.547351in}}%
\pgfpathcurveto{\pgfqpoint{2.988390in}{2.555587in}}{\pgfqpoint{2.985118in}{2.563487in}}{\pgfqpoint{2.979294in}{2.569311in}}%
\pgfpathcurveto{\pgfqpoint{2.973470in}{2.575135in}}{\pgfqpoint{2.965570in}{2.578407in}}{\pgfqpoint{2.957334in}{2.578407in}}%
\pgfpathcurveto{\pgfqpoint{2.949097in}{2.578407in}}{\pgfqpoint{2.941197in}{2.575135in}}{\pgfqpoint{2.935373in}{2.569311in}}%
\pgfpathcurveto{\pgfqpoint{2.929550in}{2.563487in}}{\pgfqpoint{2.926277in}{2.555587in}}{\pgfqpoint{2.926277in}{2.547351in}}%
\pgfpathcurveto{\pgfqpoint{2.926277in}{2.539115in}}{\pgfqpoint{2.929550in}{2.531214in}}{\pgfqpoint{2.935373in}{2.525391in}}%
\pgfpathcurveto{\pgfqpoint{2.941197in}{2.519567in}}{\pgfqpoint{2.949097in}{2.516294in}}{\pgfqpoint{2.957334in}{2.516294in}}%
\pgfpathclose%
\pgfusepath{stroke,fill}%
\end{pgfscope}%
\begin{pgfscope}%
\pgfpathrectangle{\pgfqpoint{0.100000in}{0.212622in}}{\pgfqpoint{3.696000in}{3.696000in}}%
\pgfusepath{clip}%
\pgfsetbuttcap%
\pgfsetroundjoin%
\definecolor{currentfill}{rgb}{0.121569,0.466667,0.705882}%
\pgfsetfillcolor{currentfill}%
\pgfsetfillopacity{0.971014}%
\pgfsetlinewidth{1.003750pt}%
\definecolor{currentstroke}{rgb}{0.121569,0.466667,0.705882}%
\pgfsetstrokecolor{currentstroke}%
\pgfsetstrokeopacity{0.971014}%
\pgfsetdash{}{0pt}%
\pgfpathmoveto{\pgfqpoint{2.017502in}{2.120059in}}%
\pgfpathcurveto{\pgfqpoint{2.025738in}{2.120059in}}{\pgfqpoint{2.033638in}{2.123331in}}{\pgfqpoint{2.039462in}{2.129155in}}%
\pgfpathcurveto{\pgfqpoint{2.045286in}{2.134979in}}{\pgfqpoint{2.048558in}{2.142879in}}{\pgfqpoint{2.048558in}{2.151115in}}%
\pgfpathcurveto{\pgfqpoint{2.048558in}{2.159352in}}{\pgfqpoint{2.045286in}{2.167252in}}{\pgfqpoint{2.039462in}{2.173076in}}%
\pgfpathcurveto{\pgfqpoint{2.033638in}{2.178900in}}{\pgfqpoint{2.025738in}{2.182172in}}{\pgfqpoint{2.017502in}{2.182172in}}%
\pgfpathcurveto{\pgfqpoint{2.009265in}{2.182172in}}{\pgfqpoint{2.001365in}{2.178900in}}{\pgfqpoint{1.995541in}{2.173076in}}%
\pgfpathcurveto{\pgfqpoint{1.989718in}{2.167252in}}{\pgfqpoint{1.986445in}{2.159352in}}{\pgfqpoint{1.986445in}{2.151115in}}%
\pgfpathcurveto{\pgfqpoint{1.986445in}{2.142879in}}{\pgfqpoint{1.989718in}{2.134979in}}{\pgfqpoint{1.995541in}{2.129155in}}%
\pgfpathcurveto{\pgfqpoint{2.001365in}{2.123331in}}{\pgfqpoint{2.009265in}{2.120059in}}{\pgfqpoint{2.017502in}{2.120059in}}%
\pgfpathclose%
\pgfusepath{stroke,fill}%
\end{pgfscope}%
\begin{pgfscope}%
\pgfpathrectangle{\pgfqpoint{0.100000in}{0.212622in}}{\pgfqpoint{3.696000in}{3.696000in}}%
\pgfusepath{clip}%
\pgfsetbuttcap%
\pgfsetroundjoin%
\definecolor{currentfill}{rgb}{0.121569,0.466667,0.705882}%
\pgfsetfillcolor{currentfill}%
\pgfsetfillopacity{0.972939}%
\pgfsetlinewidth{1.003750pt}%
\definecolor{currentstroke}{rgb}{0.121569,0.466667,0.705882}%
\pgfsetstrokecolor{currentstroke}%
\pgfsetstrokeopacity{0.972939}%
\pgfsetdash{}{0pt}%
\pgfpathmoveto{\pgfqpoint{2.963310in}{2.525559in}}%
\pgfpathcurveto{\pgfqpoint{2.971546in}{2.525559in}}{\pgfqpoint{2.979446in}{2.528831in}}{\pgfqpoint{2.985270in}{2.534655in}}%
\pgfpathcurveto{\pgfqpoint{2.991094in}{2.540479in}}{\pgfqpoint{2.994366in}{2.548379in}}{\pgfqpoint{2.994366in}{2.556616in}}%
\pgfpathcurveto{\pgfqpoint{2.994366in}{2.564852in}}{\pgfqpoint{2.991094in}{2.572752in}}{\pgfqpoint{2.985270in}{2.578576in}}%
\pgfpathcurveto{\pgfqpoint{2.979446in}{2.584400in}}{\pgfqpoint{2.971546in}{2.587672in}}{\pgfqpoint{2.963310in}{2.587672in}}%
\pgfpathcurveto{\pgfqpoint{2.955073in}{2.587672in}}{\pgfqpoint{2.947173in}{2.584400in}}{\pgfqpoint{2.941349in}{2.578576in}}%
\pgfpathcurveto{\pgfqpoint{2.935525in}{2.572752in}}{\pgfqpoint{2.932253in}{2.564852in}}{\pgfqpoint{2.932253in}{2.556616in}}%
\pgfpathcurveto{\pgfqpoint{2.932253in}{2.548379in}}{\pgfqpoint{2.935525in}{2.540479in}}{\pgfqpoint{2.941349in}{2.534655in}}%
\pgfpathcurveto{\pgfqpoint{2.947173in}{2.528831in}}{\pgfqpoint{2.955073in}{2.525559in}}{\pgfqpoint{2.963310in}{2.525559in}}%
\pgfpathclose%
\pgfusepath{stroke,fill}%
\end{pgfscope}%
\begin{pgfscope}%
\pgfpathrectangle{\pgfqpoint{0.100000in}{0.212622in}}{\pgfqpoint{3.696000in}{3.696000in}}%
\pgfusepath{clip}%
\pgfsetbuttcap%
\pgfsetroundjoin%
\definecolor{currentfill}{rgb}{0.121569,0.466667,0.705882}%
\pgfsetfillcolor{currentfill}%
\pgfsetfillopacity{0.973194}%
\pgfsetlinewidth{1.003750pt}%
\definecolor{currentstroke}{rgb}{0.121569,0.466667,0.705882}%
\pgfsetstrokecolor{currentstroke}%
\pgfsetstrokeopacity{0.973194}%
\pgfsetdash{}{0pt}%
\pgfpathmoveto{\pgfqpoint{2.024009in}{2.121005in}}%
\pgfpathcurveto{\pgfqpoint{2.032246in}{2.121005in}}{\pgfqpoint{2.040146in}{2.124277in}}{\pgfqpoint{2.045970in}{2.130101in}}%
\pgfpathcurveto{\pgfqpoint{2.051794in}{2.135925in}}{\pgfqpoint{2.055066in}{2.143825in}}{\pgfqpoint{2.055066in}{2.152061in}}%
\pgfpathcurveto{\pgfqpoint{2.055066in}{2.160298in}}{\pgfqpoint{2.051794in}{2.168198in}}{\pgfqpoint{2.045970in}{2.174022in}}%
\pgfpathcurveto{\pgfqpoint{2.040146in}{2.179846in}}{\pgfqpoint{2.032246in}{2.183118in}}{\pgfqpoint{2.024009in}{2.183118in}}%
\pgfpathcurveto{\pgfqpoint{2.015773in}{2.183118in}}{\pgfqpoint{2.007873in}{2.179846in}}{\pgfqpoint{2.002049in}{2.174022in}}%
\pgfpathcurveto{\pgfqpoint{1.996225in}{2.168198in}}{\pgfqpoint{1.992953in}{2.160298in}}{\pgfqpoint{1.992953in}{2.152061in}}%
\pgfpathcurveto{\pgfqpoint{1.992953in}{2.143825in}}{\pgfqpoint{1.996225in}{2.135925in}}{\pgfqpoint{2.002049in}{2.130101in}}%
\pgfpathcurveto{\pgfqpoint{2.007873in}{2.124277in}}{\pgfqpoint{2.015773in}{2.121005in}}{\pgfqpoint{2.024009in}{2.121005in}}%
\pgfpathclose%
\pgfusepath{stroke,fill}%
\end{pgfscope}%
\begin{pgfscope}%
\pgfpathrectangle{\pgfqpoint{0.100000in}{0.212622in}}{\pgfqpoint{3.696000in}{3.696000in}}%
\pgfusepath{clip}%
\pgfsetbuttcap%
\pgfsetroundjoin%
\definecolor{currentfill}{rgb}{0.121569,0.466667,0.705882}%
\pgfsetfillcolor{currentfill}%
\pgfsetfillopacity{0.974098}%
\pgfsetlinewidth{1.003750pt}%
\definecolor{currentstroke}{rgb}{0.121569,0.466667,0.705882}%
\pgfsetstrokecolor{currentstroke}%
\pgfsetstrokeopacity{0.974098}%
\pgfsetdash{}{0pt}%
\pgfpathmoveto{\pgfqpoint{2.951636in}{2.509974in}}%
\pgfpathcurveto{\pgfqpoint{2.959872in}{2.509974in}}{\pgfqpoint{2.967772in}{2.513246in}}{\pgfqpoint{2.973596in}{2.519070in}}%
\pgfpathcurveto{\pgfqpoint{2.979420in}{2.524894in}}{\pgfqpoint{2.982692in}{2.532794in}}{\pgfqpoint{2.982692in}{2.541031in}}%
\pgfpathcurveto{\pgfqpoint{2.982692in}{2.549267in}}{\pgfqpoint{2.979420in}{2.557167in}}{\pgfqpoint{2.973596in}{2.562991in}}%
\pgfpathcurveto{\pgfqpoint{2.967772in}{2.568815in}}{\pgfqpoint{2.959872in}{2.572087in}}{\pgfqpoint{2.951636in}{2.572087in}}%
\pgfpathcurveto{\pgfqpoint{2.943399in}{2.572087in}}{\pgfqpoint{2.935499in}{2.568815in}}{\pgfqpoint{2.929675in}{2.562991in}}%
\pgfpathcurveto{\pgfqpoint{2.923851in}{2.557167in}}{\pgfqpoint{2.920579in}{2.549267in}}{\pgfqpoint{2.920579in}{2.541031in}}%
\pgfpathcurveto{\pgfqpoint{2.920579in}{2.532794in}}{\pgfqpoint{2.923851in}{2.524894in}}{\pgfqpoint{2.929675in}{2.519070in}}%
\pgfpathcurveto{\pgfqpoint{2.935499in}{2.513246in}}{\pgfqpoint{2.943399in}{2.509974in}}{\pgfqpoint{2.951636in}{2.509974in}}%
\pgfpathclose%
\pgfusepath{stroke,fill}%
\end{pgfscope}%
\begin{pgfscope}%
\pgfpathrectangle{\pgfqpoint{0.100000in}{0.212622in}}{\pgfqpoint{3.696000in}{3.696000in}}%
\pgfusepath{clip}%
\pgfsetbuttcap%
\pgfsetroundjoin%
\definecolor{currentfill}{rgb}{0.121569,0.466667,0.705882}%
\pgfsetfillcolor{currentfill}%
\pgfsetfillopacity{0.975143}%
\pgfsetlinewidth{1.003750pt}%
\definecolor{currentstroke}{rgb}{0.121569,0.466667,0.705882}%
\pgfsetstrokecolor{currentstroke}%
\pgfsetstrokeopacity{0.975143}%
\pgfsetdash{}{0pt}%
\pgfpathmoveto{\pgfqpoint{1.860055in}{2.088721in}}%
\pgfpathcurveto{\pgfqpoint{1.868291in}{2.088721in}}{\pgfqpoint{1.876191in}{2.091993in}}{\pgfqpoint{1.882015in}{2.097817in}}%
\pgfpathcurveto{\pgfqpoint{1.887839in}{2.103641in}}{\pgfqpoint{1.891111in}{2.111541in}}{\pgfqpoint{1.891111in}{2.119777in}}%
\pgfpathcurveto{\pgfqpoint{1.891111in}{2.128013in}}{\pgfqpoint{1.887839in}{2.135913in}}{\pgfqpoint{1.882015in}{2.141737in}}%
\pgfpathcurveto{\pgfqpoint{1.876191in}{2.147561in}}{\pgfqpoint{1.868291in}{2.150834in}}{\pgfqpoint{1.860055in}{2.150834in}}%
\pgfpathcurveto{\pgfqpoint{1.851818in}{2.150834in}}{\pgfqpoint{1.843918in}{2.147561in}}{\pgfqpoint{1.838094in}{2.141737in}}%
\pgfpathcurveto{\pgfqpoint{1.832270in}{2.135913in}}{\pgfqpoint{1.828998in}{2.128013in}}{\pgfqpoint{1.828998in}{2.119777in}}%
\pgfpathcurveto{\pgfqpoint{1.828998in}{2.111541in}}{\pgfqpoint{1.832270in}{2.103641in}}{\pgfqpoint{1.838094in}{2.097817in}}%
\pgfpathcurveto{\pgfqpoint{1.843918in}{2.091993in}}{\pgfqpoint{1.851818in}{2.088721in}}{\pgfqpoint{1.860055in}{2.088721in}}%
\pgfpathclose%
\pgfusepath{stroke,fill}%
\end{pgfscope}%
\begin{pgfscope}%
\pgfpathrectangle{\pgfqpoint{0.100000in}{0.212622in}}{\pgfqpoint{3.696000in}{3.696000in}}%
\pgfusepath{clip}%
\pgfsetbuttcap%
\pgfsetroundjoin%
\definecolor{currentfill}{rgb}{0.121569,0.466667,0.705882}%
\pgfsetfillcolor{currentfill}%
\pgfsetfillopacity{0.975268}%
\pgfsetlinewidth{1.003750pt}%
\definecolor{currentstroke}{rgb}{0.121569,0.466667,0.705882}%
\pgfsetstrokecolor{currentstroke}%
\pgfsetstrokeopacity{0.975268}%
\pgfsetdash{}{0pt}%
\pgfpathmoveto{\pgfqpoint{2.950465in}{2.508687in}}%
\pgfpathcurveto{\pgfqpoint{2.958701in}{2.508687in}}{\pgfqpoint{2.966601in}{2.511959in}}{\pgfqpoint{2.972425in}{2.517783in}}%
\pgfpathcurveto{\pgfqpoint{2.978249in}{2.523607in}}{\pgfqpoint{2.981522in}{2.531507in}}{\pgfqpoint{2.981522in}{2.539744in}}%
\pgfpathcurveto{\pgfqpoint{2.981522in}{2.547980in}}{\pgfqpoint{2.978249in}{2.555880in}}{\pgfqpoint{2.972425in}{2.561704in}}%
\pgfpathcurveto{\pgfqpoint{2.966601in}{2.567528in}}{\pgfqpoint{2.958701in}{2.570800in}}{\pgfqpoint{2.950465in}{2.570800in}}%
\pgfpathcurveto{\pgfqpoint{2.942229in}{2.570800in}}{\pgfqpoint{2.934329in}{2.567528in}}{\pgfqpoint{2.928505in}{2.561704in}}%
\pgfpathcurveto{\pgfqpoint{2.922681in}{2.555880in}}{\pgfqpoint{2.919409in}{2.547980in}}{\pgfqpoint{2.919409in}{2.539744in}}%
\pgfpathcurveto{\pgfqpoint{2.919409in}{2.531507in}}{\pgfqpoint{2.922681in}{2.523607in}}{\pgfqpoint{2.928505in}{2.517783in}}%
\pgfpathcurveto{\pgfqpoint{2.934329in}{2.511959in}}{\pgfqpoint{2.942229in}{2.508687in}}{\pgfqpoint{2.950465in}{2.508687in}}%
\pgfpathclose%
\pgfusepath{stroke,fill}%
\end{pgfscope}%
\begin{pgfscope}%
\pgfpathrectangle{\pgfqpoint{0.100000in}{0.212622in}}{\pgfqpoint{3.696000in}{3.696000in}}%
\pgfusepath{clip}%
\pgfsetbuttcap%
\pgfsetroundjoin%
\definecolor{currentfill}{rgb}{0.121569,0.466667,0.705882}%
\pgfsetfillcolor{currentfill}%
\pgfsetfillopacity{0.975428}%
\pgfsetlinewidth{1.003750pt}%
\definecolor{currentstroke}{rgb}{0.121569,0.466667,0.705882}%
\pgfsetstrokecolor{currentstroke}%
\pgfsetstrokeopacity{0.975428}%
\pgfsetdash{}{0pt}%
\pgfpathmoveto{\pgfqpoint{1.926836in}{2.088885in}}%
\pgfpathcurveto{\pgfqpoint{1.935073in}{2.088885in}}{\pgfqpoint{1.942973in}{2.092157in}}{\pgfqpoint{1.948797in}{2.097981in}}%
\pgfpathcurveto{\pgfqpoint{1.954621in}{2.103805in}}{\pgfqpoint{1.957893in}{2.111705in}}{\pgfqpoint{1.957893in}{2.119941in}}%
\pgfpathcurveto{\pgfqpoint{1.957893in}{2.128178in}}{\pgfqpoint{1.954621in}{2.136078in}}{\pgfqpoint{1.948797in}{2.141902in}}%
\pgfpathcurveto{\pgfqpoint{1.942973in}{2.147726in}}{\pgfqpoint{1.935073in}{2.150998in}}{\pgfqpoint{1.926836in}{2.150998in}}%
\pgfpathcurveto{\pgfqpoint{1.918600in}{2.150998in}}{\pgfqpoint{1.910700in}{2.147726in}}{\pgfqpoint{1.904876in}{2.141902in}}%
\pgfpathcurveto{\pgfqpoint{1.899052in}{2.136078in}}{\pgfqpoint{1.895780in}{2.128178in}}{\pgfqpoint{1.895780in}{2.119941in}}%
\pgfpathcurveto{\pgfqpoint{1.895780in}{2.111705in}}{\pgfqpoint{1.899052in}{2.103805in}}{\pgfqpoint{1.904876in}{2.097981in}}%
\pgfpathcurveto{\pgfqpoint{1.910700in}{2.092157in}}{\pgfqpoint{1.918600in}{2.088885in}}{\pgfqpoint{1.926836in}{2.088885in}}%
\pgfpathclose%
\pgfusepath{stroke,fill}%
\end{pgfscope}%
\begin{pgfscope}%
\pgfpathrectangle{\pgfqpoint{0.100000in}{0.212622in}}{\pgfqpoint{3.696000in}{3.696000in}}%
\pgfusepath{clip}%
\pgfsetbuttcap%
\pgfsetroundjoin%
\definecolor{currentfill}{rgb}{0.121569,0.466667,0.705882}%
\pgfsetfillcolor{currentfill}%
\pgfsetfillopacity{0.975659}%
\pgfsetlinewidth{1.003750pt}%
\definecolor{currentstroke}{rgb}{0.121569,0.466667,0.705882}%
\pgfsetstrokecolor{currentstroke}%
\pgfsetstrokeopacity{0.975659}%
\pgfsetdash{}{0pt}%
\pgfpathmoveto{\pgfqpoint{2.951747in}{2.509347in}}%
\pgfpathcurveto{\pgfqpoint{2.959983in}{2.509347in}}{\pgfqpoint{2.967883in}{2.512619in}}{\pgfqpoint{2.973707in}{2.518443in}}%
\pgfpathcurveto{\pgfqpoint{2.979531in}{2.524267in}}{\pgfqpoint{2.982804in}{2.532167in}}{\pgfqpoint{2.982804in}{2.540404in}}%
\pgfpathcurveto{\pgfqpoint{2.982804in}{2.548640in}}{\pgfqpoint{2.979531in}{2.556540in}}{\pgfqpoint{2.973707in}{2.562364in}}%
\pgfpathcurveto{\pgfqpoint{2.967883in}{2.568188in}}{\pgfqpoint{2.959983in}{2.571460in}}{\pgfqpoint{2.951747in}{2.571460in}}%
\pgfpathcurveto{\pgfqpoint{2.943511in}{2.571460in}}{\pgfqpoint{2.935611in}{2.568188in}}{\pgfqpoint{2.929787in}{2.562364in}}%
\pgfpathcurveto{\pgfqpoint{2.923963in}{2.556540in}}{\pgfqpoint{2.920691in}{2.548640in}}{\pgfqpoint{2.920691in}{2.540404in}}%
\pgfpathcurveto{\pgfqpoint{2.920691in}{2.532167in}}{\pgfqpoint{2.923963in}{2.524267in}}{\pgfqpoint{2.929787in}{2.518443in}}%
\pgfpathcurveto{\pgfqpoint{2.935611in}{2.512619in}}{\pgfqpoint{2.943511in}{2.509347in}}{\pgfqpoint{2.951747in}{2.509347in}}%
\pgfpathclose%
\pgfusepath{stroke,fill}%
\end{pgfscope}%
\begin{pgfscope}%
\pgfpathrectangle{\pgfqpoint{0.100000in}{0.212622in}}{\pgfqpoint{3.696000in}{3.696000in}}%
\pgfusepath{clip}%
\pgfsetbuttcap%
\pgfsetroundjoin%
\definecolor{currentfill}{rgb}{0.121569,0.466667,0.705882}%
\pgfsetfillcolor{currentfill}%
\pgfsetfillopacity{0.975953}%
\pgfsetlinewidth{1.003750pt}%
\definecolor{currentstroke}{rgb}{0.121569,0.466667,0.705882}%
\pgfsetstrokecolor{currentstroke}%
\pgfsetstrokeopacity{0.975953}%
\pgfsetdash{}{0pt}%
\pgfpathmoveto{\pgfqpoint{1.870263in}{2.088177in}}%
\pgfpathcurveto{\pgfqpoint{1.878500in}{2.088177in}}{\pgfqpoint{1.886400in}{2.091449in}}{\pgfqpoint{1.892224in}{2.097273in}}%
\pgfpathcurveto{\pgfqpoint{1.898048in}{2.103097in}}{\pgfqpoint{1.901320in}{2.110997in}}{\pgfqpoint{1.901320in}{2.119233in}}%
\pgfpathcurveto{\pgfqpoint{1.901320in}{2.127470in}}{\pgfqpoint{1.898048in}{2.135370in}}{\pgfqpoint{1.892224in}{2.141194in}}%
\pgfpathcurveto{\pgfqpoint{1.886400in}{2.147017in}}{\pgfqpoint{1.878500in}{2.150290in}}{\pgfqpoint{1.870263in}{2.150290in}}%
\pgfpathcurveto{\pgfqpoint{1.862027in}{2.150290in}}{\pgfqpoint{1.854127in}{2.147017in}}{\pgfqpoint{1.848303in}{2.141194in}}%
\pgfpathcurveto{\pgfqpoint{1.842479in}{2.135370in}}{\pgfqpoint{1.839207in}{2.127470in}}{\pgfqpoint{1.839207in}{2.119233in}}%
\pgfpathcurveto{\pgfqpoint{1.839207in}{2.110997in}}{\pgfqpoint{1.842479in}{2.103097in}}{\pgfqpoint{1.848303in}{2.097273in}}%
\pgfpathcurveto{\pgfqpoint{1.854127in}{2.091449in}}{\pgfqpoint{1.862027in}{2.088177in}}{\pgfqpoint{1.870263in}{2.088177in}}%
\pgfpathclose%
\pgfusepath{stroke,fill}%
\end{pgfscope}%
\begin{pgfscope}%
\pgfpathrectangle{\pgfqpoint{0.100000in}{0.212622in}}{\pgfqpoint{3.696000in}{3.696000in}}%
\pgfusepath{clip}%
\pgfsetbuttcap%
\pgfsetroundjoin%
\definecolor{currentfill}{rgb}{0.121569,0.466667,0.705882}%
\pgfsetfillcolor{currentfill}%
\pgfsetfillopacity{0.976187}%
\pgfsetlinewidth{1.003750pt}%
\definecolor{currentstroke}{rgb}{0.121569,0.466667,0.705882}%
\pgfsetstrokecolor{currentstroke}%
\pgfsetstrokeopacity{0.976187}%
\pgfsetdash{}{0pt}%
\pgfpathmoveto{\pgfqpoint{2.949051in}{2.506755in}}%
\pgfpathcurveto{\pgfqpoint{2.957287in}{2.506755in}}{\pgfqpoint{2.965187in}{2.510027in}}{\pgfqpoint{2.971011in}{2.515851in}}%
\pgfpathcurveto{\pgfqpoint{2.976835in}{2.521675in}}{\pgfqpoint{2.980107in}{2.529575in}}{\pgfqpoint{2.980107in}{2.537811in}}%
\pgfpathcurveto{\pgfqpoint{2.980107in}{2.546047in}}{\pgfqpoint{2.976835in}{2.553947in}}{\pgfqpoint{2.971011in}{2.559771in}}%
\pgfpathcurveto{\pgfqpoint{2.965187in}{2.565595in}}{\pgfqpoint{2.957287in}{2.568868in}}{\pgfqpoint{2.949051in}{2.568868in}}%
\pgfpathcurveto{\pgfqpoint{2.940814in}{2.568868in}}{\pgfqpoint{2.932914in}{2.565595in}}{\pgfqpoint{2.927091in}{2.559771in}}%
\pgfpathcurveto{\pgfqpoint{2.921267in}{2.553947in}}{\pgfqpoint{2.917994in}{2.546047in}}{\pgfqpoint{2.917994in}{2.537811in}}%
\pgfpathcurveto{\pgfqpoint{2.917994in}{2.529575in}}{\pgfqpoint{2.921267in}{2.521675in}}{\pgfqpoint{2.927091in}{2.515851in}}%
\pgfpathcurveto{\pgfqpoint{2.932914in}{2.510027in}}{\pgfqpoint{2.940814in}{2.506755in}}{\pgfqpoint{2.949051in}{2.506755in}}%
\pgfpathclose%
\pgfusepath{stroke,fill}%
\end{pgfscope}%
\begin{pgfscope}%
\pgfpathrectangle{\pgfqpoint{0.100000in}{0.212622in}}{\pgfqpoint{3.696000in}{3.696000in}}%
\pgfusepath{clip}%
\pgfsetbuttcap%
\pgfsetroundjoin%
\definecolor{currentfill}{rgb}{0.121569,0.466667,0.705882}%
\pgfsetfillcolor{currentfill}%
\pgfsetfillopacity{0.976531}%
\pgfsetlinewidth{1.003750pt}%
\definecolor{currentstroke}{rgb}{0.121569,0.466667,0.705882}%
\pgfsetstrokecolor{currentstroke}%
\pgfsetstrokeopacity{0.976531}%
\pgfsetdash{}{0pt}%
\pgfpathmoveto{\pgfqpoint{2.954099in}{2.515744in}}%
\pgfpathcurveto{\pgfqpoint{2.962335in}{2.515744in}}{\pgfqpoint{2.970235in}{2.519017in}}{\pgfqpoint{2.976059in}{2.524841in}}%
\pgfpathcurveto{\pgfqpoint{2.981883in}{2.530665in}}{\pgfqpoint{2.985155in}{2.538565in}}{\pgfqpoint{2.985155in}{2.546801in}}%
\pgfpathcurveto{\pgfqpoint{2.985155in}{2.555037in}}{\pgfqpoint{2.981883in}{2.562937in}}{\pgfqpoint{2.976059in}{2.568761in}}%
\pgfpathcurveto{\pgfqpoint{2.970235in}{2.574585in}}{\pgfqpoint{2.962335in}{2.577857in}}{\pgfqpoint{2.954099in}{2.577857in}}%
\pgfpathcurveto{\pgfqpoint{2.945863in}{2.577857in}}{\pgfqpoint{2.937962in}{2.574585in}}{\pgfqpoint{2.932139in}{2.568761in}}%
\pgfpathcurveto{\pgfqpoint{2.926315in}{2.562937in}}{\pgfqpoint{2.923042in}{2.555037in}}{\pgfqpoint{2.923042in}{2.546801in}}%
\pgfpathcurveto{\pgfqpoint{2.923042in}{2.538565in}}{\pgfqpoint{2.926315in}{2.530665in}}{\pgfqpoint{2.932139in}{2.524841in}}%
\pgfpathcurveto{\pgfqpoint{2.937962in}{2.519017in}}{\pgfqpoint{2.945863in}{2.515744in}}{\pgfqpoint{2.954099in}{2.515744in}}%
\pgfpathclose%
\pgfusepath{stroke,fill}%
\end{pgfscope}%
\begin{pgfscope}%
\pgfpathrectangle{\pgfqpoint{0.100000in}{0.212622in}}{\pgfqpoint{3.696000in}{3.696000in}}%
\pgfusepath{clip}%
\pgfsetbuttcap%
\pgfsetroundjoin%
\definecolor{currentfill}{rgb}{0.121569,0.466667,0.705882}%
\pgfsetfillcolor{currentfill}%
\pgfsetfillopacity{0.977362}%
\pgfsetlinewidth{1.003750pt}%
\definecolor{currentstroke}{rgb}{0.121569,0.466667,0.705882}%
\pgfsetstrokecolor{currentstroke}%
\pgfsetstrokeopacity{0.977362}%
\pgfsetdash{}{0pt}%
\pgfpathmoveto{\pgfqpoint{1.819162in}{2.052733in}}%
\pgfpathcurveto{\pgfqpoint{1.827398in}{2.052733in}}{\pgfqpoint{1.835298in}{2.056005in}}{\pgfqpoint{1.841122in}{2.061829in}}%
\pgfpathcurveto{\pgfqpoint{1.846946in}{2.067653in}}{\pgfqpoint{1.850219in}{2.075553in}}{\pgfqpoint{1.850219in}{2.083789in}}%
\pgfpathcurveto{\pgfqpoint{1.850219in}{2.092025in}}{\pgfqpoint{1.846946in}{2.099925in}}{\pgfqpoint{1.841122in}{2.105749in}}%
\pgfpathcurveto{\pgfqpoint{1.835298in}{2.111573in}}{\pgfqpoint{1.827398in}{2.114846in}}{\pgfqpoint{1.819162in}{2.114846in}}%
\pgfpathcurveto{\pgfqpoint{1.810926in}{2.114846in}}{\pgfqpoint{1.803026in}{2.111573in}}{\pgfqpoint{1.797202in}{2.105749in}}%
\pgfpathcurveto{\pgfqpoint{1.791378in}{2.099925in}}{\pgfqpoint{1.788106in}{2.092025in}}{\pgfqpoint{1.788106in}{2.083789in}}%
\pgfpathcurveto{\pgfqpoint{1.788106in}{2.075553in}}{\pgfqpoint{1.791378in}{2.067653in}}{\pgfqpoint{1.797202in}{2.061829in}}%
\pgfpathcurveto{\pgfqpoint{1.803026in}{2.056005in}}{\pgfqpoint{1.810926in}{2.052733in}}{\pgfqpoint{1.819162in}{2.052733in}}%
\pgfpathclose%
\pgfusepath{stroke,fill}%
\end{pgfscope}%
\begin{pgfscope}%
\pgfpathrectangle{\pgfqpoint{0.100000in}{0.212622in}}{\pgfqpoint{3.696000in}{3.696000in}}%
\pgfusepath{clip}%
\pgfsetbuttcap%
\pgfsetroundjoin%
\definecolor{currentfill}{rgb}{0.121569,0.466667,0.705882}%
\pgfsetfillcolor{currentfill}%
\pgfsetfillopacity{0.977367}%
\pgfsetlinewidth{1.003750pt}%
\definecolor{currentstroke}{rgb}{0.121569,0.466667,0.705882}%
\pgfsetstrokecolor{currentstroke}%
\pgfsetstrokeopacity{0.977367}%
\pgfsetdash{}{0pt}%
\pgfpathmoveto{\pgfqpoint{1.926500in}{2.085532in}}%
\pgfpathcurveto{\pgfqpoint{1.934737in}{2.085532in}}{\pgfqpoint{1.942637in}{2.088804in}}{\pgfqpoint{1.948460in}{2.094628in}}%
\pgfpathcurveto{\pgfqpoint{1.954284in}{2.100452in}}{\pgfqpoint{1.957557in}{2.108352in}}{\pgfqpoint{1.957557in}{2.116588in}}%
\pgfpathcurveto{\pgfqpoint{1.957557in}{2.124825in}}{\pgfqpoint{1.954284in}{2.132725in}}{\pgfqpoint{1.948460in}{2.138548in}}%
\pgfpathcurveto{\pgfqpoint{1.942637in}{2.144372in}}{\pgfqpoint{1.934737in}{2.147645in}}{\pgfqpoint{1.926500in}{2.147645in}}%
\pgfpathcurveto{\pgfqpoint{1.918264in}{2.147645in}}{\pgfqpoint{1.910364in}{2.144372in}}{\pgfqpoint{1.904540in}{2.138548in}}%
\pgfpathcurveto{\pgfqpoint{1.898716in}{2.132725in}}{\pgfqpoint{1.895444in}{2.124825in}}{\pgfqpoint{1.895444in}{2.116588in}}%
\pgfpathcurveto{\pgfqpoint{1.895444in}{2.108352in}}{\pgfqpoint{1.898716in}{2.100452in}}{\pgfqpoint{1.904540in}{2.094628in}}%
\pgfpathcurveto{\pgfqpoint{1.910364in}{2.088804in}}{\pgfqpoint{1.918264in}{2.085532in}}{\pgfqpoint{1.926500in}{2.085532in}}%
\pgfpathclose%
\pgfusepath{stroke,fill}%
\end{pgfscope}%
\begin{pgfscope}%
\pgfpathrectangle{\pgfqpoint{0.100000in}{0.212622in}}{\pgfqpoint{3.696000in}{3.696000in}}%
\pgfusepath{clip}%
\pgfsetbuttcap%
\pgfsetroundjoin%
\definecolor{currentfill}{rgb}{0.121569,0.466667,0.705882}%
\pgfsetfillcolor{currentfill}%
\pgfsetfillopacity{0.978797}%
\pgfsetlinewidth{1.003750pt}%
\definecolor{currentstroke}{rgb}{0.121569,0.466667,0.705882}%
\pgfsetstrokecolor{currentstroke}%
\pgfsetstrokeopacity{0.978797}%
\pgfsetdash{}{0pt}%
\pgfpathmoveto{\pgfqpoint{1.856735in}{2.072128in}}%
\pgfpathcurveto{\pgfqpoint{1.864971in}{2.072128in}}{\pgfqpoint{1.872871in}{2.075400in}}{\pgfqpoint{1.878695in}{2.081224in}}%
\pgfpathcurveto{\pgfqpoint{1.884519in}{2.087048in}}{\pgfqpoint{1.887791in}{2.094948in}}{\pgfqpoint{1.887791in}{2.103185in}}%
\pgfpathcurveto{\pgfqpoint{1.887791in}{2.111421in}}{\pgfqpoint{1.884519in}{2.119321in}}{\pgfqpoint{1.878695in}{2.125145in}}%
\pgfpathcurveto{\pgfqpoint{1.872871in}{2.130969in}}{\pgfqpoint{1.864971in}{2.134241in}}{\pgfqpoint{1.856735in}{2.134241in}}%
\pgfpathcurveto{\pgfqpoint{1.848499in}{2.134241in}}{\pgfqpoint{1.840599in}{2.130969in}}{\pgfqpoint{1.834775in}{2.125145in}}%
\pgfpathcurveto{\pgfqpoint{1.828951in}{2.119321in}}{\pgfqpoint{1.825678in}{2.111421in}}{\pgfqpoint{1.825678in}{2.103185in}}%
\pgfpathcurveto{\pgfqpoint{1.825678in}{2.094948in}}{\pgfqpoint{1.828951in}{2.087048in}}{\pgfqpoint{1.834775in}{2.081224in}}%
\pgfpathcurveto{\pgfqpoint{1.840599in}{2.075400in}}{\pgfqpoint{1.848499in}{2.072128in}}{\pgfqpoint{1.856735in}{2.072128in}}%
\pgfpathclose%
\pgfusepath{stroke,fill}%
\end{pgfscope}%
\begin{pgfscope}%
\pgfpathrectangle{\pgfqpoint{0.100000in}{0.212622in}}{\pgfqpoint{3.696000in}{3.696000in}}%
\pgfusepath{clip}%
\pgfsetbuttcap%
\pgfsetroundjoin%
\definecolor{currentfill}{rgb}{0.121569,0.466667,0.705882}%
\pgfsetfillcolor{currentfill}%
\pgfsetfillopacity{0.983602}%
\pgfsetlinewidth{1.003750pt}%
\definecolor{currentstroke}{rgb}{0.121569,0.466667,0.705882}%
\pgfsetstrokecolor{currentstroke}%
\pgfsetstrokeopacity{0.983602}%
\pgfsetdash{}{0pt}%
\pgfpathmoveto{\pgfqpoint{1.899301in}{2.064587in}}%
\pgfpathcurveto{\pgfqpoint{1.907537in}{2.064587in}}{\pgfqpoint{1.915437in}{2.067860in}}{\pgfqpoint{1.921261in}{2.073684in}}%
\pgfpathcurveto{\pgfqpoint{1.927085in}{2.079507in}}{\pgfqpoint{1.930358in}{2.087408in}}{\pgfqpoint{1.930358in}{2.095644in}}%
\pgfpathcurveto{\pgfqpoint{1.930358in}{2.103880in}}{\pgfqpoint{1.927085in}{2.111780in}}{\pgfqpoint{1.921261in}{2.117604in}}%
\pgfpathcurveto{\pgfqpoint{1.915437in}{2.123428in}}{\pgfqpoint{1.907537in}{2.126700in}}{\pgfqpoint{1.899301in}{2.126700in}}%
\pgfpathcurveto{\pgfqpoint{1.891065in}{2.126700in}}{\pgfqpoint{1.883165in}{2.123428in}}{\pgfqpoint{1.877341in}{2.117604in}}%
\pgfpathcurveto{\pgfqpoint{1.871517in}{2.111780in}}{\pgfqpoint{1.868245in}{2.103880in}}{\pgfqpoint{1.868245in}{2.095644in}}%
\pgfpathcurveto{\pgfqpoint{1.868245in}{2.087408in}}{\pgfqpoint{1.871517in}{2.079507in}}{\pgfqpoint{1.877341in}{2.073684in}}%
\pgfpathcurveto{\pgfqpoint{1.883165in}{2.067860in}}{\pgfqpoint{1.891065in}{2.064587in}}{\pgfqpoint{1.899301in}{2.064587in}}%
\pgfpathclose%
\pgfusepath{stroke,fill}%
\end{pgfscope}%
\begin{pgfscope}%
\pgfpathrectangle{\pgfqpoint{0.100000in}{0.212622in}}{\pgfqpoint{3.696000in}{3.696000in}}%
\pgfusepath{clip}%
\pgfsetbuttcap%
\pgfsetroundjoin%
\definecolor{currentfill}{rgb}{0.121569,0.466667,0.705882}%
\pgfsetfillcolor{currentfill}%
\pgfsetfillopacity{0.984795}%
\pgfsetlinewidth{1.003750pt}%
\definecolor{currentstroke}{rgb}{0.121569,0.466667,0.705882}%
\pgfsetstrokecolor{currentstroke}%
\pgfsetstrokeopacity{0.984795}%
\pgfsetdash{}{0pt}%
\pgfpathmoveto{\pgfqpoint{1.816717in}{2.042029in}}%
\pgfpathcurveto{\pgfqpoint{1.824953in}{2.042029in}}{\pgfqpoint{1.832853in}{2.045301in}}{\pgfqpoint{1.838677in}{2.051125in}}%
\pgfpathcurveto{\pgfqpoint{1.844501in}{2.056949in}}{\pgfqpoint{1.847773in}{2.064849in}}{\pgfqpoint{1.847773in}{2.073085in}}%
\pgfpathcurveto{\pgfqpoint{1.847773in}{2.081322in}}{\pgfqpoint{1.844501in}{2.089222in}}{\pgfqpoint{1.838677in}{2.095045in}}%
\pgfpathcurveto{\pgfqpoint{1.832853in}{2.100869in}}{\pgfqpoint{1.824953in}{2.104142in}}{\pgfqpoint{1.816717in}{2.104142in}}%
\pgfpathcurveto{\pgfqpoint{1.808480in}{2.104142in}}{\pgfqpoint{1.800580in}{2.100869in}}{\pgfqpoint{1.794756in}{2.095045in}}%
\pgfpathcurveto{\pgfqpoint{1.788932in}{2.089222in}}{\pgfqpoint{1.785660in}{2.081322in}}{\pgfqpoint{1.785660in}{2.073085in}}%
\pgfpathcurveto{\pgfqpoint{1.785660in}{2.064849in}}{\pgfqpoint{1.788932in}{2.056949in}}{\pgfqpoint{1.794756in}{2.051125in}}%
\pgfpathcurveto{\pgfqpoint{1.800580in}{2.045301in}}{\pgfqpoint{1.808480in}{2.042029in}}{\pgfqpoint{1.816717in}{2.042029in}}%
\pgfpathclose%
\pgfusepath{stroke,fill}%
\end{pgfscope}%
\begin{pgfscope}%
\pgfpathrectangle{\pgfqpoint{0.100000in}{0.212622in}}{\pgfqpoint{3.696000in}{3.696000in}}%
\pgfusepath{clip}%
\pgfsetbuttcap%
\pgfsetroundjoin%
\definecolor{currentfill}{rgb}{0.121569,0.466667,0.705882}%
\pgfsetfillcolor{currentfill}%
\pgfsetfillopacity{0.988793}%
\pgfsetlinewidth{1.003750pt}%
\definecolor{currentstroke}{rgb}{0.121569,0.466667,0.705882}%
\pgfsetstrokecolor{currentstroke}%
\pgfsetstrokeopacity{0.988793}%
\pgfsetdash{}{0pt}%
\pgfpathmoveto{\pgfqpoint{1.856826in}{2.068472in}}%
\pgfpathcurveto{\pgfqpoint{1.865062in}{2.068472in}}{\pgfqpoint{1.872962in}{2.071744in}}{\pgfqpoint{1.878786in}{2.077568in}}%
\pgfpathcurveto{\pgfqpoint{1.884610in}{2.083392in}}{\pgfqpoint{1.887883in}{2.091292in}}{\pgfqpoint{1.887883in}{2.099528in}}%
\pgfpathcurveto{\pgfqpoint{1.887883in}{2.107764in}}{\pgfqpoint{1.884610in}{2.115665in}}{\pgfqpoint{1.878786in}{2.121488in}}%
\pgfpathcurveto{\pgfqpoint{1.872962in}{2.127312in}}{\pgfqpoint{1.865062in}{2.130585in}}{\pgfqpoint{1.856826in}{2.130585in}}%
\pgfpathcurveto{\pgfqpoint{1.848590in}{2.130585in}}{\pgfqpoint{1.840690in}{2.127312in}}{\pgfqpoint{1.834866in}{2.121488in}}%
\pgfpathcurveto{\pgfqpoint{1.829042in}{2.115665in}}{\pgfqpoint{1.825770in}{2.107764in}}{\pgfqpoint{1.825770in}{2.099528in}}%
\pgfpathcurveto{\pgfqpoint{1.825770in}{2.091292in}}{\pgfqpoint{1.829042in}{2.083392in}}{\pgfqpoint{1.834866in}{2.077568in}}%
\pgfpathcurveto{\pgfqpoint{1.840690in}{2.071744in}}{\pgfqpoint{1.848590in}{2.068472in}}{\pgfqpoint{1.856826in}{2.068472in}}%
\pgfpathclose%
\pgfusepath{stroke,fill}%
\end{pgfscope}%
\begin{pgfscope}%
\pgfpathrectangle{\pgfqpoint{0.100000in}{0.212622in}}{\pgfqpoint{3.696000in}{3.696000in}}%
\pgfusepath{clip}%
\pgfsetbuttcap%
\pgfsetroundjoin%
\definecolor{currentfill}{rgb}{0.121569,0.466667,0.705882}%
\pgfsetfillcolor{currentfill}%
\pgfsetfillopacity{0.988996}%
\pgfsetlinewidth{1.003750pt}%
\definecolor{currentstroke}{rgb}{0.121569,0.466667,0.705882}%
\pgfsetstrokecolor{currentstroke}%
\pgfsetstrokeopacity{0.988996}%
\pgfsetdash{}{0pt}%
\pgfpathmoveto{\pgfqpoint{1.844910in}{2.056610in}}%
\pgfpathcurveto{\pgfqpoint{1.853147in}{2.056610in}}{\pgfqpoint{1.861047in}{2.059882in}}{\pgfqpoint{1.866871in}{2.065706in}}%
\pgfpathcurveto{\pgfqpoint{1.872694in}{2.071530in}}{\pgfqpoint{1.875967in}{2.079430in}}{\pgfqpoint{1.875967in}{2.087667in}}%
\pgfpathcurveto{\pgfqpoint{1.875967in}{2.095903in}}{\pgfqpoint{1.872694in}{2.103803in}}{\pgfqpoint{1.866871in}{2.109627in}}%
\pgfpathcurveto{\pgfqpoint{1.861047in}{2.115451in}}{\pgfqpoint{1.853147in}{2.118723in}}{\pgfqpoint{1.844910in}{2.118723in}}%
\pgfpathcurveto{\pgfqpoint{1.836674in}{2.118723in}}{\pgfqpoint{1.828774in}{2.115451in}}{\pgfqpoint{1.822950in}{2.109627in}}%
\pgfpathcurveto{\pgfqpoint{1.817126in}{2.103803in}}{\pgfqpoint{1.813854in}{2.095903in}}{\pgfqpoint{1.813854in}{2.087667in}}%
\pgfpathcurveto{\pgfqpoint{1.813854in}{2.079430in}}{\pgfqpoint{1.817126in}{2.071530in}}{\pgfqpoint{1.822950in}{2.065706in}}%
\pgfpathcurveto{\pgfqpoint{1.828774in}{2.059882in}}{\pgfqpoint{1.836674in}{2.056610in}}{\pgfqpoint{1.844910in}{2.056610in}}%
\pgfpathclose%
\pgfusepath{stroke,fill}%
\end{pgfscope}%
\begin{pgfscope}%
\pgfpathrectangle{\pgfqpoint{0.100000in}{0.212622in}}{\pgfqpoint{3.696000in}{3.696000in}}%
\pgfusepath{clip}%
\pgfsetbuttcap%
\pgfsetroundjoin%
\definecolor{currentfill}{rgb}{0.121569,0.466667,0.705882}%
\pgfsetfillcolor{currentfill}%
\pgfsetfillopacity{0.996073}%
\pgfsetlinewidth{1.003750pt}%
\definecolor{currentstroke}{rgb}{0.121569,0.466667,0.705882}%
\pgfsetstrokecolor{currentstroke}%
\pgfsetstrokeopacity{0.996073}%
\pgfsetdash{}{0pt}%
\pgfpathmoveto{\pgfqpoint{1.857446in}{2.035338in}}%
\pgfpathcurveto{\pgfqpoint{1.865683in}{2.035338in}}{\pgfqpoint{1.873583in}{2.038610in}}{\pgfqpoint{1.879407in}{2.044434in}}%
\pgfpathcurveto{\pgfqpoint{1.885231in}{2.050258in}}{\pgfqpoint{1.888503in}{2.058158in}}{\pgfqpoint{1.888503in}{2.066394in}}%
\pgfpathcurveto{\pgfqpoint{1.888503in}{2.074631in}}{\pgfqpoint{1.885231in}{2.082531in}}{\pgfqpoint{1.879407in}{2.088355in}}%
\pgfpathcurveto{\pgfqpoint{1.873583in}{2.094179in}}{\pgfqpoint{1.865683in}{2.097451in}}{\pgfqpoint{1.857446in}{2.097451in}}%
\pgfpathcurveto{\pgfqpoint{1.849210in}{2.097451in}}{\pgfqpoint{1.841310in}{2.094179in}}{\pgfqpoint{1.835486in}{2.088355in}}%
\pgfpathcurveto{\pgfqpoint{1.829662in}{2.082531in}}{\pgfqpoint{1.826390in}{2.074631in}}{\pgfqpoint{1.826390in}{2.066394in}}%
\pgfpathcurveto{\pgfqpoint{1.826390in}{2.058158in}}{\pgfqpoint{1.829662in}{2.050258in}}{\pgfqpoint{1.835486in}{2.044434in}}%
\pgfpathcurveto{\pgfqpoint{1.841310in}{2.038610in}}{\pgfqpoint{1.849210in}{2.035338in}}{\pgfqpoint{1.857446in}{2.035338in}}%
\pgfpathclose%
\pgfusepath{stroke,fill}%
\end{pgfscope}%
\begin{pgfscope}%
\pgfpathrectangle{\pgfqpoint{0.100000in}{0.212622in}}{\pgfqpoint{3.696000in}{3.696000in}}%
\pgfusepath{clip}%
\pgfsetbuttcap%
\pgfsetroundjoin%
\definecolor{currentfill}{rgb}{0.121569,0.466667,0.705882}%
\pgfsetfillcolor{currentfill}%
\pgfsetlinewidth{1.003750pt}%
\definecolor{currentstroke}{rgb}{0.121569,0.466667,0.705882}%
\pgfsetstrokecolor{currentstroke}%
\pgfsetdash{}{0pt}%
\pgfpathmoveto{\pgfqpoint{1.841454in}{2.044049in}}%
\pgfpathcurveto{\pgfqpoint{1.849690in}{2.044049in}}{\pgfqpoint{1.857591in}{2.047321in}}{\pgfqpoint{1.863414in}{2.053145in}}%
\pgfpathcurveto{\pgfqpoint{1.869238in}{2.058969in}}{\pgfqpoint{1.872511in}{2.066869in}}{\pgfqpoint{1.872511in}{2.075105in}}%
\pgfpathcurveto{\pgfqpoint{1.872511in}{2.083342in}}{\pgfqpoint{1.869238in}{2.091242in}}{\pgfqpoint{1.863414in}{2.097066in}}%
\pgfpathcurveto{\pgfqpoint{1.857591in}{2.102889in}}{\pgfqpoint{1.849690in}{2.106162in}}{\pgfqpoint{1.841454in}{2.106162in}}%
\pgfpathcurveto{\pgfqpoint{1.833218in}{2.106162in}}{\pgfqpoint{1.825318in}{2.102889in}}{\pgfqpoint{1.819494in}{2.097066in}}%
\pgfpathcurveto{\pgfqpoint{1.813670in}{2.091242in}}{\pgfqpoint{1.810398in}{2.083342in}}{\pgfqpoint{1.810398in}{2.075105in}}%
\pgfpathcurveto{\pgfqpoint{1.810398in}{2.066869in}}{\pgfqpoint{1.813670in}{2.058969in}}{\pgfqpoint{1.819494in}{2.053145in}}%
\pgfpathcurveto{\pgfqpoint{1.825318in}{2.047321in}}{\pgfqpoint{1.833218in}{2.044049in}}{\pgfqpoint{1.841454in}{2.044049in}}%
\pgfpathclose%
\pgfusepath{stroke,fill}%
\end{pgfscope}%
\begin{pgfscope}%
\pgfsetbuttcap%
\pgfsetmiterjoin%
\definecolor{currentfill}{rgb}{1.000000,1.000000,1.000000}%
\pgfsetfillcolor{currentfill}%
\pgfsetfillopacity{0.800000}%
\pgfsetlinewidth{1.003750pt}%
\definecolor{currentstroke}{rgb}{0.800000,0.800000,0.800000}%
\pgfsetstrokecolor{currentstroke}%
\pgfsetstrokeopacity{0.800000}%
\pgfsetdash{}{0pt}%
\pgfpathmoveto{\pgfqpoint{2.104889in}{3.410289in}}%
\pgfpathlineto{\pgfqpoint{3.698778in}{3.410289in}}%
\pgfpathquadraticcurveto{\pgfqpoint{3.726556in}{3.410289in}}{\pgfqpoint{3.726556in}{3.438067in}}%
\pgfpathlineto{\pgfqpoint{3.726556in}{3.811400in}}%
\pgfpathquadraticcurveto{\pgfqpoint{3.726556in}{3.839178in}}{\pgfqpoint{3.698778in}{3.839178in}}%
\pgfpathlineto{\pgfqpoint{2.104889in}{3.839178in}}%
\pgfpathquadraticcurveto{\pgfqpoint{2.077111in}{3.839178in}}{\pgfqpoint{2.077111in}{3.811400in}}%
\pgfpathlineto{\pgfqpoint{2.077111in}{3.438067in}}%
\pgfpathquadraticcurveto{\pgfqpoint{2.077111in}{3.410289in}}{\pgfqpoint{2.104889in}{3.410289in}}%
\pgfpathclose%
\pgfusepath{stroke,fill}%
\end{pgfscope}%
\begin{pgfscope}%
\pgfsetrectcap%
\pgfsetroundjoin%
\pgfsetlinewidth{1.505625pt}%
\definecolor{currentstroke}{rgb}{0.121569,0.466667,0.705882}%
\pgfsetstrokecolor{currentstroke}%
\pgfsetdash{}{0pt}%
\pgfpathmoveto{\pgfqpoint{2.132667in}{3.735011in}}%
\pgfpathlineto{\pgfqpoint{2.410444in}{3.735011in}}%
\pgfusepath{stroke}%
\end{pgfscope}%
\begin{pgfscope}%
\definecolor{textcolor}{rgb}{0.000000,0.000000,0.000000}%
\pgfsetstrokecolor{textcolor}%
\pgfsetfillcolor{textcolor}%
\pgftext[x=2.521555in,y=3.686400in,left,base]{\color{textcolor}\rmfamily\fontsize{10.000000}{12.000000}\selectfont Ground truth}%
\end{pgfscope}%
\begin{pgfscope}%
\pgfsetbuttcap%
\pgfsetroundjoin%
\definecolor{currentfill}{rgb}{0.121569,0.466667,0.705882}%
\pgfsetfillcolor{currentfill}%
\pgfsetlinewidth{1.003750pt}%
\definecolor{currentstroke}{rgb}{0.121569,0.466667,0.705882}%
\pgfsetstrokecolor{currentstroke}%
\pgfsetdash{}{0pt}%
\pgfsys@defobject{currentmarker}{\pgfqpoint{-0.031056in}{-0.031056in}}{\pgfqpoint{0.031056in}{0.031056in}}{%
\pgfpathmoveto{\pgfqpoint{0.000000in}{-0.031056in}}%
\pgfpathcurveto{\pgfqpoint{0.008236in}{-0.031056in}}{\pgfqpoint{0.016136in}{-0.027784in}}{\pgfqpoint{0.021960in}{-0.021960in}}%
\pgfpathcurveto{\pgfqpoint{0.027784in}{-0.016136in}}{\pgfqpoint{0.031056in}{-0.008236in}}{\pgfqpoint{0.031056in}{0.000000in}}%
\pgfpathcurveto{\pgfqpoint{0.031056in}{0.008236in}}{\pgfqpoint{0.027784in}{0.016136in}}{\pgfqpoint{0.021960in}{0.021960in}}%
\pgfpathcurveto{\pgfqpoint{0.016136in}{0.027784in}}{\pgfqpoint{0.008236in}{0.031056in}}{\pgfqpoint{0.000000in}{0.031056in}}%
\pgfpathcurveto{\pgfqpoint{-0.008236in}{0.031056in}}{\pgfqpoint{-0.016136in}{0.027784in}}{\pgfqpoint{-0.021960in}{0.021960in}}%
\pgfpathcurveto{\pgfqpoint{-0.027784in}{0.016136in}}{\pgfqpoint{-0.031056in}{0.008236in}}{\pgfqpoint{-0.031056in}{0.000000in}}%
\pgfpathcurveto{\pgfqpoint{-0.031056in}{-0.008236in}}{\pgfqpoint{-0.027784in}{-0.016136in}}{\pgfqpoint{-0.021960in}{-0.021960in}}%
\pgfpathcurveto{\pgfqpoint{-0.016136in}{-0.027784in}}{\pgfqpoint{-0.008236in}{-0.031056in}}{\pgfqpoint{0.000000in}{-0.031056in}}%
\pgfpathclose%
\pgfusepath{stroke,fill}%
}%
\begin{pgfscope}%
\pgfsys@transformshift{2.271555in}{3.529248in}%
\pgfsys@useobject{currentmarker}{}%
\end{pgfscope}%
\end{pgfscope}%
\begin{pgfscope}%
\definecolor{textcolor}{rgb}{0.000000,0.000000,0.000000}%
\pgfsetstrokecolor{textcolor}%
\pgfsetfillcolor{textcolor}%
\pgftext[x=2.521555in,y=3.492789in,left,base]{\color{textcolor}\rmfamily\fontsize{10.000000}{12.000000}\selectfont Estimated position}%
\end{pgfscope}%
\end{pgfpicture}%
\makeatother%
\endgroup%
}
%         \caption{ROLEQ's 3D position estimation had the lowest turn error for the 4-meter line experiment.}
%         \label{fig:line16_3D}
%     \end{subfigure}
%     \caption{Position estimation by the best performing algorithms in the 4-meter line experiment.}
%     \label{fig:line16}
% \end{figure}

% \subsubsection{28 meter}

% For the 28-meter line experiment, the OLEQ algorithm which had the lowest displacement error with an average of 0.16 meters (16\% of error margin), and ROLEQ with an average of 0.24 meters of turn error (7\% of error margin).

% \begin{figure}[!h]
%     \centering
%     \begin{table}[H]
    \begin{center}
    \resizebox{1\linewidth}{!}{

        \begin{tabular}[t]{lcccc}
            \hline
            Algorithm                   & Displacement Error[$m$] & Displacement Error[\%]      & Turn Error[$m$]  & Turn Error[\%]             \\
            \hline 
            AngularRate            & 1.29  & 4.61 & 6.08 & 21.73              \\            AQUA            & 8.66  & 30.93 & 14.77 & 52.75              \\            Complementary            & 0.52  & 1.85 & 4.31 & 15.41              \\            Davenport            & 0.73  & 2.60 & 5.47 & 19.53              \\            EKF            & 0.76  & 2.73 & 4.35 & 15.55              \\            FAMC            & 0.50  & 1.80 & 4.33 & 15.45              \\            FLAE            & 0.73  & 2.59 & 5.47 & 19.53              \\            Fourati            & 1.07  & 3.81 & 6.21 & 22.18              \\            Madgwick            & 0.55  & 1.96 & 4.29 & 15.32              \\            Mahony            & 0.53  & 1.88 & 4.19 & 14.98              \\            OLEQ            & 0.69  & 2.47 & 5.35 & 19.11              \\            QUEST            & 3.08  & 11.01 & 11.72 & 41.87              \\            ROLEQ            & 0.83  & 2.95 & 5.39 & 19.24              \\            SAAM            & 0.51  & 1.81 & 4.23 & 15.09              \\            Tilt            & 0.51  & 1.81 & 4.23 & 15.09              \\
            \hline
            Average & 1.40 & 4.99 & 6.03 & 21.52
        \end{tabular}
        }
        \caption{Accelerometer Specifications. }
        \label{tab:accelerometer_specification}
    \end{center}
\end{table}
% \end{figure}

% \begin{figure}[!h]
%     \centering
%     \begin{subfigure}{0.49\textwidth}
%         \centering
%         \resizebox{1\linewidth}{!}{%% Creator: Matplotlib, PGF backend
%%
%% To include the figure in your LaTeX document, write
%%   \input{<filename>.pgf}
%%
%% Make sure the required packages are loaded in your preamble
%%   \usepackage{pgf}
%%
%% and, on pdftex
%%   \usepackage[utf8]{inputenc}\DeclareUnicodeCharacter{2212}{-}
%%
%% or, on luatex and xetex
%%   \usepackage{unicode-math}
%%
%% Figures using additional raster images can only be included by \input if
%% they are in the same directory as the main LaTeX file. For loading figures
%% from other directories you can use the `import` package
%%   \usepackage{import}
%%
%% and then include the figures with
%%   \import{<path to file>}{<filename>.pgf}
%%
%% Matplotlib used the following preamble
%%   \usepackage{fontspec}
%%
\begingroup%
\makeatletter%
\begin{pgfpicture}%
\pgfpathrectangle{\pgfpointorigin}{\pgfqpoint{5.698611in}{4.311000in}}%
\pgfusepath{use as bounding box, clip}%
\begin{pgfscope}%
\pgfsetbuttcap%
\pgfsetmiterjoin%
\definecolor{currentfill}{rgb}{1.000000,1.000000,1.000000}%
\pgfsetfillcolor{currentfill}%
\pgfsetlinewidth{0.000000pt}%
\definecolor{currentstroke}{rgb}{1.000000,1.000000,1.000000}%
\pgfsetstrokecolor{currentstroke}%
\pgfsetdash{}{0pt}%
\pgfpathmoveto{\pgfqpoint{0.000000in}{0.000000in}}%
\pgfpathlineto{\pgfqpoint{5.698611in}{0.000000in}}%
\pgfpathlineto{\pgfqpoint{5.698611in}{4.311000in}}%
\pgfpathlineto{\pgfqpoint{0.000000in}{4.311000in}}%
\pgfpathclose%
\pgfusepath{fill}%
\end{pgfscope}%
\begin{pgfscope}%
\pgfsetbuttcap%
\pgfsetmiterjoin%
\definecolor{currentfill}{rgb}{1.000000,1.000000,1.000000}%
\pgfsetfillcolor{currentfill}%
\pgfsetlinewidth{0.000000pt}%
\definecolor{currentstroke}{rgb}{0.000000,0.000000,0.000000}%
\pgfsetstrokecolor{currentstroke}%
\pgfsetstrokeopacity{0.000000}%
\pgfsetdash{}{0pt}%
\pgfpathmoveto{\pgfqpoint{0.638611in}{0.515000in}}%
\pgfpathlineto{\pgfqpoint{5.598611in}{0.515000in}}%
\pgfpathlineto{\pgfqpoint{5.598611in}{4.211000in}}%
\pgfpathlineto{\pgfqpoint{0.638611in}{4.211000in}}%
\pgfpathclose%
\pgfusepath{fill}%
\end{pgfscope}%
\begin{pgfscope}%
\pgfpathrectangle{\pgfqpoint{0.638611in}{0.515000in}}{\pgfqpoint{4.960000in}{3.696000in}}%
\pgfusepath{clip}%
\pgfsetbuttcap%
\pgfsetroundjoin%
\definecolor{currentfill}{rgb}{0.121569,0.466667,0.705882}%
\pgfsetfillcolor{currentfill}%
\pgfsetlinewidth{1.003750pt}%
\definecolor{currentstroke}{rgb}{0.121569,0.466667,0.705882}%
\pgfsetstrokecolor{currentstroke}%
\pgfsetdash{}{0pt}%
\pgfsys@defobject{currentmarker}{\pgfqpoint{-0.041667in}{-0.041667in}}{\pgfqpoint{0.041667in}{0.041667in}}{%
\pgfpathmoveto{\pgfqpoint{0.000000in}{-0.041667in}}%
\pgfpathcurveto{\pgfqpoint{0.011050in}{-0.041667in}}{\pgfqpoint{0.021649in}{-0.037276in}}{\pgfqpoint{0.029463in}{-0.029463in}}%
\pgfpathcurveto{\pgfqpoint{0.037276in}{-0.021649in}}{\pgfqpoint{0.041667in}{-0.011050in}}{\pgfqpoint{0.041667in}{0.000000in}}%
\pgfpathcurveto{\pgfqpoint{0.041667in}{0.011050in}}{\pgfqpoint{0.037276in}{0.021649in}}{\pgfqpoint{0.029463in}{0.029463in}}%
\pgfpathcurveto{\pgfqpoint{0.021649in}{0.037276in}}{\pgfqpoint{0.011050in}{0.041667in}}{\pgfqpoint{0.000000in}{0.041667in}}%
\pgfpathcurveto{\pgfqpoint{-0.011050in}{0.041667in}}{\pgfqpoint{-0.021649in}{0.037276in}}{\pgfqpoint{-0.029463in}{0.029463in}}%
\pgfpathcurveto{\pgfqpoint{-0.037276in}{0.021649in}}{\pgfqpoint{-0.041667in}{0.011050in}}{\pgfqpoint{-0.041667in}{0.000000in}}%
\pgfpathcurveto{\pgfqpoint{-0.041667in}{-0.011050in}}{\pgfqpoint{-0.037276in}{-0.021649in}}{\pgfqpoint{-0.029463in}{-0.029463in}}%
\pgfpathcurveto{\pgfqpoint{-0.021649in}{-0.037276in}}{\pgfqpoint{-0.011050in}{-0.041667in}}{\pgfqpoint{0.000000in}{-0.041667in}}%
\pgfpathclose%
\pgfusepath{stroke,fill}%
}%
\begin{pgfscope}%
\pgfsys@transformshift{0.864072in}{2.390003in}%
\pgfsys@useobject{currentmarker}{}%
\end{pgfscope}%
\begin{pgfscope}%
\pgfsys@transformshift{0.864076in}{2.390000in}%
\pgfsys@useobject{currentmarker}{}%
\end{pgfscope}%
\begin{pgfscope}%
\pgfsys@transformshift{0.864079in}{2.390000in}%
\pgfsys@useobject{currentmarker}{}%
\end{pgfscope}%
\begin{pgfscope}%
\pgfsys@transformshift{0.864080in}{2.390000in}%
\pgfsys@useobject{currentmarker}{}%
\end{pgfscope}%
\begin{pgfscope}%
\pgfsys@transformshift{0.864081in}{2.390000in}%
\pgfsys@useobject{currentmarker}{}%
\end{pgfscope}%
\begin{pgfscope}%
\pgfsys@transformshift{0.864081in}{2.390000in}%
\pgfsys@useobject{currentmarker}{}%
\end{pgfscope}%
\begin{pgfscope}%
\pgfsys@transformshift{0.864082in}{2.390000in}%
\pgfsys@useobject{currentmarker}{}%
\end{pgfscope}%
\begin{pgfscope}%
\pgfsys@transformshift{0.864815in}{2.390024in}%
\pgfsys@useobject{currentmarker}{}%
\end{pgfscope}%
\begin{pgfscope}%
\pgfsys@transformshift{0.866170in}{2.390066in}%
\pgfsys@useobject{currentmarker}{}%
\end{pgfscope}%
\begin{pgfscope}%
\pgfsys@transformshift{0.866915in}{2.390090in}%
\pgfsys@useobject{currentmarker}{}%
\end{pgfscope}%
\begin{pgfscope}%
\pgfsys@transformshift{0.867325in}{2.390110in}%
\pgfsys@useobject{currentmarker}{}%
\end{pgfscope}%
\begin{pgfscope}%
\pgfsys@transformshift{0.867551in}{2.390116in}%
\pgfsys@useobject{currentmarker}{}%
\end{pgfscope}%
\begin{pgfscope}%
\pgfsys@transformshift{0.867674in}{2.390125in}%
\pgfsys@useobject{currentmarker}{}%
\end{pgfscope}%
\begin{pgfscope}%
\pgfsys@transformshift{0.867742in}{2.390128in}%
\pgfsys@useobject{currentmarker}{}%
\end{pgfscope}%
\begin{pgfscope}%
\pgfsys@transformshift{0.867780in}{2.390131in}%
\pgfsys@useobject{currentmarker}{}%
\end{pgfscope}%
\begin{pgfscope}%
\pgfsys@transformshift{0.867800in}{2.390133in}%
\pgfsys@useobject{currentmarker}{}%
\end{pgfscope}%
\begin{pgfscope}%
\pgfsys@transformshift{0.867812in}{2.390133in}%
\pgfsys@useobject{currentmarker}{}%
\end{pgfscope}%
\begin{pgfscope}%
\pgfsys@transformshift{0.867818in}{2.390134in}%
\pgfsys@useobject{currentmarker}{}%
\end{pgfscope}%
\begin{pgfscope}%
\pgfsys@transformshift{0.867821in}{2.390134in}%
\pgfsys@useobject{currentmarker}{}%
\end{pgfscope}%
\begin{pgfscope}%
\pgfsys@transformshift{0.867823in}{2.390134in}%
\pgfsys@useobject{currentmarker}{}%
\end{pgfscope}%
\begin{pgfscope}%
\pgfsys@transformshift{0.867824in}{2.390134in}%
\pgfsys@useobject{currentmarker}{}%
\end{pgfscope}%
\begin{pgfscope}%
\pgfsys@transformshift{0.867825in}{2.390134in}%
\pgfsys@useobject{currentmarker}{}%
\end{pgfscope}%
\begin{pgfscope}%
\pgfsys@transformshift{0.867825in}{2.390134in}%
\pgfsys@useobject{currentmarker}{}%
\end{pgfscope}%
\begin{pgfscope}%
\pgfsys@transformshift{0.867825in}{2.390134in}%
\pgfsys@useobject{currentmarker}{}%
\end{pgfscope}%
\begin{pgfscope}%
\pgfsys@transformshift{0.867825in}{2.390134in}%
\pgfsys@useobject{currentmarker}{}%
\end{pgfscope}%
\begin{pgfscope}%
\pgfsys@transformshift{0.867825in}{2.390134in}%
\pgfsys@useobject{currentmarker}{}%
\end{pgfscope}%
\begin{pgfscope}%
\pgfsys@transformshift{0.867825in}{2.390134in}%
\pgfsys@useobject{currentmarker}{}%
\end{pgfscope}%
\begin{pgfscope}%
\pgfsys@transformshift{0.867825in}{2.390134in}%
\pgfsys@useobject{currentmarker}{}%
\end{pgfscope}%
\begin{pgfscope}%
\pgfsys@transformshift{0.867825in}{2.390134in}%
\pgfsys@useobject{currentmarker}{}%
\end{pgfscope}%
\begin{pgfscope}%
\pgfsys@transformshift{0.867825in}{2.390134in}%
\pgfsys@useobject{currentmarker}{}%
\end{pgfscope}%
\begin{pgfscope}%
\pgfsys@transformshift{0.867826in}{2.390134in}%
\pgfsys@useobject{currentmarker}{}%
\end{pgfscope}%
\begin{pgfscope}%
\pgfsys@transformshift{0.867826in}{2.390134in}%
\pgfsys@useobject{currentmarker}{}%
\end{pgfscope}%
\begin{pgfscope}%
\pgfsys@transformshift{0.867826in}{2.390134in}%
\pgfsys@useobject{currentmarker}{}%
\end{pgfscope}%
\begin{pgfscope}%
\pgfsys@transformshift{0.867826in}{2.390134in}%
\pgfsys@useobject{currentmarker}{}%
\end{pgfscope}%
\begin{pgfscope}%
\pgfsys@transformshift{0.867826in}{2.390134in}%
\pgfsys@useobject{currentmarker}{}%
\end{pgfscope}%
\begin{pgfscope}%
\pgfsys@transformshift{0.867826in}{2.390134in}%
\pgfsys@useobject{currentmarker}{}%
\end{pgfscope}%
\begin{pgfscope}%
\pgfsys@transformshift{0.867826in}{2.390134in}%
\pgfsys@useobject{currentmarker}{}%
\end{pgfscope}%
\begin{pgfscope}%
\pgfsys@transformshift{0.867826in}{2.390134in}%
\pgfsys@useobject{currentmarker}{}%
\end{pgfscope}%
\begin{pgfscope}%
\pgfsys@transformshift{0.867826in}{2.390134in}%
\pgfsys@useobject{currentmarker}{}%
\end{pgfscope}%
\begin{pgfscope}%
\pgfsys@transformshift{0.867826in}{2.390134in}%
\pgfsys@useobject{currentmarker}{}%
\end{pgfscope}%
\begin{pgfscope}%
\pgfsys@transformshift{0.867826in}{2.390134in}%
\pgfsys@useobject{currentmarker}{}%
\end{pgfscope}%
\begin{pgfscope}%
\pgfsys@transformshift{0.867826in}{2.390134in}%
\pgfsys@useobject{currentmarker}{}%
\end{pgfscope}%
\begin{pgfscope}%
\pgfsys@transformshift{0.867826in}{2.390134in}%
\pgfsys@useobject{currentmarker}{}%
\end{pgfscope}%
\begin{pgfscope}%
\pgfsys@transformshift{0.867826in}{2.390134in}%
\pgfsys@useobject{currentmarker}{}%
\end{pgfscope}%
\begin{pgfscope}%
\pgfsys@transformshift{0.867826in}{2.390134in}%
\pgfsys@useobject{currentmarker}{}%
\end{pgfscope}%
\begin{pgfscope}%
\pgfsys@transformshift{0.867826in}{2.390134in}%
\pgfsys@useobject{currentmarker}{}%
\end{pgfscope}%
\begin{pgfscope}%
\pgfsys@transformshift{0.867826in}{2.390134in}%
\pgfsys@useobject{currentmarker}{}%
\end{pgfscope}%
\begin{pgfscope}%
\pgfsys@transformshift{0.867826in}{2.390134in}%
\pgfsys@useobject{currentmarker}{}%
\end{pgfscope}%
\begin{pgfscope}%
\pgfsys@transformshift{0.867826in}{2.390134in}%
\pgfsys@useobject{currentmarker}{}%
\end{pgfscope}%
\begin{pgfscope}%
\pgfsys@transformshift{0.867826in}{2.390134in}%
\pgfsys@useobject{currentmarker}{}%
\end{pgfscope}%
\begin{pgfscope}%
\pgfsys@transformshift{0.867826in}{2.390134in}%
\pgfsys@useobject{currentmarker}{}%
\end{pgfscope}%
\begin{pgfscope}%
\pgfsys@transformshift{0.867826in}{2.390134in}%
\pgfsys@useobject{currentmarker}{}%
\end{pgfscope}%
\begin{pgfscope}%
\pgfsys@transformshift{0.867826in}{2.390134in}%
\pgfsys@useobject{currentmarker}{}%
\end{pgfscope}%
\begin{pgfscope}%
\pgfsys@transformshift{0.867826in}{2.390134in}%
\pgfsys@useobject{currentmarker}{}%
\end{pgfscope}%
\begin{pgfscope}%
\pgfsys@transformshift{0.867826in}{2.390134in}%
\pgfsys@useobject{currentmarker}{}%
\end{pgfscope}%
\begin{pgfscope}%
\pgfsys@transformshift{0.867826in}{2.390134in}%
\pgfsys@useobject{currentmarker}{}%
\end{pgfscope}%
\begin{pgfscope}%
\pgfsys@transformshift{0.867826in}{2.390134in}%
\pgfsys@useobject{currentmarker}{}%
\end{pgfscope}%
\begin{pgfscope}%
\pgfsys@transformshift{0.867826in}{2.390134in}%
\pgfsys@useobject{currentmarker}{}%
\end{pgfscope}%
\begin{pgfscope}%
\pgfsys@transformshift{0.867826in}{2.390134in}%
\pgfsys@useobject{currentmarker}{}%
\end{pgfscope}%
\begin{pgfscope}%
\pgfsys@transformshift{0.867826in}{2.390134in}%
\pgfsys@useobject{currentmarker}{}%
\end{pgfscope}%
\begin{pgfscope}%
\pgfsys@transformshift{0.867826in}{2.390134in}%
\pgfsys@useobject{currentmarker}{}%
\end{pgfscope}%
\begin{pgfscope}%
\pgfsys@transformshift{0.867826in}{2.390134in}%
\pgfsys@useobject{currentmarker}{}%
\end{pgfscope}%
\begin{pgfscope}%
\pgfsys@transformshift{0.867826in}{2.390134in}%
\pgfsys@useobject{currentmarker}{}%
\end{pgfscope}%
\begin{pgfscope}%
\pgfsys@transformshift{0.867826in}{2.390134in}%
\pgfsys@useobject{currentmarker}{}%
\end{pgfscope}%
\begin{pgfscope}%
\pgfsys@transformshift{0.867826in}{2.390134in}%
\pgfsys@useobject{currentmarker}{}%
\end{pgfscope}%
\begin{pgfscope}%
\pgfsys@transformshift{0.867826in}{2.390134in}%
\pgfsys@useobject{currentmarker}{}%
\end{pgfscope}%
\begin{pgfscope}%
\pgfsys@transformshift{0.867826in}{2.390134in}%
\pgfsys@useobject{currentmarker}{}%
\end{pgfscope}%
\begin{pgfscope}%
\pgfsys@transformshift{0.867826in}{2.390134in}%
\pgfsys@useobject{currentmarker}{}%
\end{pgfscope}%
\begin{pgfscope}%
\pgfsys@transformshift{0.867826in}{2.390134in}%
\pgfsys@useobject{currentmarker}{}%
\end{pgfscope}%
\begin{pgfscope}%
\pgfsys@transformshift{0.867826in}{2.390134in}%
\pgfsys@useobject{currentmarker}{}%
\end{pgfscope}%
\begin{pgfscope}%
\pgfsys@transformshift{0.867826in}{2.390134in}%
\pgfsys@useobject{currentmarker}{}%
\end{pgfscope}%
\begin{pgfscope}%
\pgfsys@transformshift{0.867826in}{2.390134in}%
\pgfsys@useobject{currentmarker}{}%
\end{pgfscope}%
\begin{pgfscope}%
\pgfsys@transformshift{0.867826in}{2.390134in}%
\pgfsys@useobject{currentmarker}{}%
\end{pgfscope}%
\begin{pgfscope}%
\pgfsys@transformshift{0.867826in}{2.390134in}%
\pgfsys@useobject{currentmarker}{}%
\end{pgfscope}%
\begin{pgfscope}%
\pgfsys@transformshift{0.867826in}{2.390134in}%
\pgfsys@useobject{currentmarker}{}%
\end{pgfscope}%
\begin{pgfscope}%
\pgfsys@transformshift{0.867826in}{2.390134in}%
\pgfsys@useobject{currentmarker}{}%
\end{pgfscope}%
\begin{pgfscope}%
\pgfsys@transformshift{0.867826in}{2.390134in}%
\pgfsys@useobject{currentmarker}{}%
\end{pgfscope}%
\begin{pgfscope}%
\pgfsys@transformshift{0.867826in}{2.390134in}%
\pgfsys@useobject{currentmarker}{}%
\end{pgfscope}%
\begin{pgfscope}%
\pgfsys@transformshift{0.867826in}{2.390134in}%
\pgfsys@useobject{currentmarker}{}%
\end{pgfscope}%
\begin{pgfscope}%
\pgfsys@transformshift{0.867826in}{2.390134in}%
\pgfsys@useobject{currentmarker}{}%
\end{pgfscope}%
\begin{pgfscope}%
\pgfsys@transformshift{0.867826in}{2.390134in}%
\pgfsys@useobject{currentmarker}{}%
\end{pgfscope}%
\begin{pgfscope}%
\pgfsys@transformshift{0.867826in}{2.390134in}%
\pgfsys@useobject{currentmarker}{}%
\end{pgfscope}%
\begin{pgfscope}%
\pgfsys@transformshift{0.867826in}{2.390134in}%
\pgfsys@useobject{currentmarker}{}%
\end{pgfscope}%
\begin{pgfscope}%
\pgfsys@transformshift{0.867826in}{2.390134in}%
\pgfsys@useobject{currentmarker}{}%
\end{pgfscope}%
\begin{pgfscope}%
\pgfsys@transformshift{0.867826in}{2.390134in}%
\pgfsys@useobject{currentmarker}{}%
\end{pgfscope}%
\begin{pgfscope}%
\pgfsys@transformshift{0.867826in}{2.390134in}%
\pgfsys@useobject{currentmarker}{}%
\end{pgfscope}%
\begin{pgfscope}%
\pgfsys@transformshift{0.867826in}{2.390134in}%
\pgfsys@useobject{currentmarker}{}%
\end{pgfscope}%
\begin{pgfscope}%
\pgfsys@transformshift{0.867826in}{2.390134in}%
\pgfsys@useobject{currentmarker}{}%
\end{pgfscope}%
\begin{pgfscope}%
\pgfsys@transformshift{0.867826in}{2.390134in}%
\pgfsys@useobject{currentmarker}{}%
\end{pgfscope}%
\begin{pgfscope}%
\pgfsys@transformshift{0.867826in}{2.390134in}%
\pgfsys@useobject{currentmarker}{}%
\end{pgfscope}%
\begin{pgfscope}%
\pgfsys@transformshift{0.867826in}{2.390134in}%
\pgfsys@useobject{currentmarker}{}%
\end{pgfscope}%
\begin{pgfscope}%
\pgfsys@transformshift{0.867826in}{2.390134in}%
\pgfsys@useobject{currentmarker}{}%
\end{pgfscope}%
\begin{pgfscope}%
\pgfsys@transformshift{0.867826in}{2.390134in}%
\pgfsys@useobject{currentmarker}{}%
\end{pgfscope}%
\begin{pgfscope}%
\pgfsys@transformshift{0.867826in}{2.390134in}%
\pgfsys@useobject{currentmarker}{}%
\end{pgfscope}%
\begin{pgfscope}%
\pgfsys@transformshift{0.867826in}{2.390134in}%
\pgfsys@useobject{currentmarker}{}%
\end{pgfscope}%
\begin{pgfscope}%
\pgfsys@transformshift{0.867826in}{2.390134in}%
\pgfsys@useobject{currentmarker}{}%
\end{pgfscope}%
\begin{pgfscope}%
\pgfsys@transformshift{0.867826in}{2.390134in}%
\pgfsys@useobject{currentmarker}{}%
\end{pgfscope}%
\begin{pgfscope}%
\pgfsys@transformshift{0.867826in}{2.390134in}%
\pgfsys@useobject{currentmarker}{}%
\end{pgfscope}%
\begin{pgfscope}%
\pgfsys@transformshift{0.867826in}{2.390134in}%
\pgfsys@useobject{currentmarker}{}%
\end{pgfscope}%
\begin{pgfscope}%
\pgfsys@transformshift{0.867826in}{2.390134in}%
\pgfsys@useobject{currentmarker}{}%
\end{pgfscope}%
\begin{pgfscope}%
\pgfsys@transformshift{0.867826in}{2.390134in}%
\pgfsys@useobject{currentmarker}{}%
\end{pgfscope}%
\begin{pgfscope}%
\pgfsys@transformshift{0.867826in}{2.390134in}%
\pgfsys@useobject{currentmarker}{}%
\end{pgfscope}%
\begin{pgfscope}%
\pgfsys@transformshift{0.867826in}{2.390134in}%
\pgfsys@useobject{currentmarker}{}%
\end{pgfscope}%
\begin{pgfscope}%
\pgfsys@transformshift{0.867826in}{2.390134in}%
\pgfsys@useobject{currentmarker}{}%
\end{pgfscope}%
\begin{pgfscope}%
\pgfsys@transformshift{0.867826in}{2.390134in}%
\pgfsys@useobject{currentmarker}{}%
\end{pgfscope}%
\begin{pgfscope}%
\pgfsys@transformshift{0.867826in}{2.390134in}%
\pgfsys@useobject{currentmarker}{}%
\end{pgfscope}%
\begin{pgfscope}%
\pgfsys@transformshift{0.867826in}{2.390134in}%
\pgfsys@useobject{currentmarker}{}%
\end{pgfscope}%
\begin{pgfscope}%
\pgfsys@transformshift{0.867826in}{2.390134in}%
\pgfsys@useobject{currentmarker}{}%
\end{pgfscope}%
\begin{pgfscope}%
\pgfsys@transformshift{0.867826in}{2.390134in}%
\pgfsys@useobject{currentmarker}{}%
\end{pgfscope}%
\begin{pgfscope}%
\pgfsys@transformshift{0.867826in}{2.390134in}%
\pgfsys@useobject{currentmarker}{}%
\end{pgfscope}%
\begin{pgfscope}%
\pgfsys@transformshift{0.867826in}{2.390134in}%
\pgfsys@useobject{currentmarker}{}%
\end{pgfscope}%
\begin{pgfscope}%
\pgfsys@transformshift{0.867826in}{2.390134in}%
\pgfsys@useobject{currentmarker}{}%
\end{pgfscope}%
\begin{pgfscope}%
\pgfsys@transformshift{0.867826in}{2.390134in}%
\pgfsys@useobject{currentmarker}{}%
\end{pgfscope}%
\begin{pgfscope}%
\pgfsys@transformshift{0.867826in}{2.390134in}%
\pgfsys@useobject{currentmarker}{}%
\end{pgfscope}%
\begin{pgfscope}%
\pgfsys@transformshift{0.867826in}{2.390134in}%
\pgfsys@useobject{currentmarker}{}%
\end{pgfscope}%
\begin{pgfscope}%
\pgfsys@transformshift{0.867826in}{2.390134in}%
\pgfsys@useobject{currentmarker}{}%
\end{pgfscope}%
\begin{pgfscope}%
\pgfsys@transformshift{0.867826in}{2.390134in}%
\pgfsys@useobject{currentmarker}{}%
\end{pgfscope}%
\begin{pgfscope}%
\pgfsys@transformshift{0.867826in}{2.390134in}%
\pgfsys@useobject{currentmarker}{}%
\end{pgfscope}%
\begin{pgfscope}%
\pgfsys@transformshift{0.868561in}{2.390275in}%
\pgfsys@useobject{currentmarker}{}%
\end{pgfscope}%
\begin{pgfscope}%
\pgfsys@transformshift{0.868973in}{2.390251in}%
\pgfsys@useobject{currentmarker}{}%
\end{pgfscope}%
\begin{pgfscope}%
\pgfsys@transformshift{0.869196in}{2.390289in}%
\pgfsys@useobject{currentmarker}{}%
\end{pgfscope}%
\begin{pgfscope}%
\pgfsys@transformshift{0.869320in}{2.390280in}%
\pgfsys@useobject{currentmarker}{}%
\end{pgfscope}%
\begin{pgfscope}%
\pgfsys@transformshift{0.869388in}{2.390292in}%
\pgfsys@useobject{currentmarker}{}%
\end{pgfscope}%
\begin{pgfscope}%
\pgfsys@transformshift{0.869426in}{2.390291in}%
\pgfsys@useobject{currentmarker}{}%
\end{pgfscope}%
\begin{pgfscope}%
\pgfsys@transformshift{0.869446in}{2.390293in}%
\pgfsys@useobject{currentmarker}{}%
\end{pgfscope}%
\begin{pgfscope}%
\pgfsys@transformshift{0.869458in}{2.390293in}%
\pgfsys@useobject{currentmarker}{}%
\end{pgfscope}%
\begin{pgfscope}%
\pgfsys@transformshift{0.869464in}{2.390294in}%
\pgfsys@useobject{currentmarker}{}%
\end{pgfscope}%
\begin{pgfscope}%
\pgfsys@transformshift{0.869467in}{2.390294in}%
\pgfsys@useobject{currentmarker}{}%
\end{pgfscope}%
\begin{pgfscope}%
\pgfsys@transformshift{0.869469in}{2.390294in}%
\pgfsys@useobject{currentmarker}{}%
\end{pgfscope}%
\begin{pgfscope}%
\pgfsys@transformshift{0.869470in}{2.390294in}%
\pgfsys@useobject{currentmarker}{}%
\end{pgfscope}%
\begin{pgfscope}%
\pgfsys@transformshift{0.869471in}{2.390294in}%
\pgfsys@useobject{currentmarker}{}%
\end{pgfscope}%
\begin{pgfscope}%
\pgfsys@transformshift{0.869471in}{2.390294in}%
\pgfsys@useobject{currentmarker}{}%
\end{pgfscope}%
\begin{pgfscope}%
\pgfsys@transformshift{0.869471in}{2.390294in}%
\pgfsys@useobject{currentmarker}{}%
\end{pgfscope}%
\begin{pgfscope}%
\pgfsys@transformshift{0.869471in}{2.390294in}%
\pgfsys@useobject{currentmarker}{}%
\end{pgfscope}%
\begin{pgfscope}%
\pgfsys@transformshift{0.869471in}{2.390294in}%
\pgfsys@useobject{currentmarker}{}%
\end{pgfscope}%
\begin{pgfscope}%
\pgfsys@transformshift{0.869471in}{2.390294in}%
\pgfsys@useobject{currentmarker}{}%
\end{pgfscope}%
\begin{pgfscope}%
\pgfsys@transformshift{0.869471in}{2.390294in}%
\pgfsys@useobject{currentmarker}{}%
\end{pgfscope}%
\begin{pgfscope}%
\pgfsys@transformshift{0.869471in}{2.390294in}%
\pgfsys@useobject{currentmarker}{}%
\end{pgfscope}%
\begin{pgfscope}%
\pgfsys@transformshift{0.869471in}{2.390294in}%
\pgfsys@useobject{currentmarker}{}%
\end{pgfscope}%
\begin{pgfscope}%
\pgfsys@transformshift{0.869471in}{2.390294in}%
\pgfsys@useobject{currentmarker}{}%
\end{pgfscope}%
\begin{pgfscope}%
\pgfsys@transformshift{0.869471in}{2.390294in}%
\pgfsys@useobject{currentmarker}{}%
\end{pgfscope}%
\begin{pgfscope}%
\pgfsys@transformshift{0.869471in}{2.390294in}%
\pgfsys@useobject{currentmarker}{}%
\end{pgfscope}%
\begin{pgfscope}%
\pgfsys@transformshift{0.869471in}{2.390294in}%
\pgfsys@useobject{currentmarker}{}%
\end{pgfscope}%
\begin{pgfscope}%
\pgfsys@transformshift{0.869471in}{2.390294in}%
\pgfsys@useobject{currentmarker}{}%
\end{pgfscope}%
\begin{pgfscope}%
\pgfsys@transformshift{0.869471in}{2.390294in}%
\pgfsys@useobject{currentmarker}{}%
\end{pgfscope}%
\begin{pgfscope}%
\pgfsys@transformshift{0.869471in}{2.390294in}%
\pgfsys@useobject{currentmarker}{}%
\end{pgfscope}%
\begin{pgfscope}%
\pgfsys@transformshift{0.869471in}{2.390294in}%
\pgfsys@useobject{currentmarker}{}%
\end{pgfscope}%
\begin{pgfscope}%
\pgfsys@transformshift{0.869471in}{2.390294in}%
\pgfsys@useobject{currentmarker}{}%
\end{pgfscope}%
\begin{pgfscope}%
\pgfsys@transformshift{0.869471in}{2.390294in}%
\pgfsys@useobject{currentmarker}{}%
\end{pgfscope}%
\begin{pgfscope}%
\pgfsys@transformshift{0.869471in}{2.390294in}%
\pgfsys@useobject{currentmarker}{}%
\end{pgfscope}%
\begin{pgfscope}%
\pgfsys@transformshift{0.869471in}{2.390294in}%
\pgfsys@useobject{currentmarker}{}%
\end{pgfscope}%
\begin{pgfscope}%
\pgfsys@transformshift{0.869471in}{2.390294in}%
\pgfsys@useobject{currentmarker}{}%
\end{pgfscope}%
\begin{pgfscope}%
\pgfsys@transformshift{0.869471in}{2.390294in}%
\pgfsys@useobject{currentmarker}{}%
\end{pgfscope}%
\begin{pgfscope}%
\pgfsys@transformshift{0.869471in}{2.390294in}%
\pgfsys@useobject{currentmarker}{}%
\end{pgfscope}%
\begin{pgfscope}%
\pgfsys@transformshift{0.869471in}{2.390294in}%
\pgfsys@useobject{currentmarker}{}%
\end{pgfscope}%
\begin{pgfscope}%
\pgfsys@transformshift{0.869471in}{2.390294in}%
\pgfsys@useobject{currentmarker}{}%
\end{pgfscope}%
\begin{pgfscope}%
\pgfsys@transformshift{0.869471in}{2.390294in}%
\pgfsys@useobject{currentmarker}{}%
\end{pgfscope}%
\begin{pgfscope}%
\pgfsys@transformshift{0.869471in}{2.390294in}%
\pgfsys@useobject{currentmarker}{}%
\end{pgfscope}%
\begin{pgfscope}%
\pgfsys@transformshift{0.869471in}{2.390294in}%
\pgfsys@useobject{currentmarker}{}%
\end{pgfscope}%
\begin{pgfscope}%
\pgfsys@transformshift{0.869471in}{2.390294in}%
\pgfsys@useobject{currentmarker}{}%
\end{pgfscope}%
\begin{pgfscope}%
\pgfsys@transformshift{0.869471in}{2.390294in}%
\pgfsys@useobject{currentmarker}{}%
\end{pgfscope}%
\begin{pgfscope}%
\pgfsys@transformshift{0.869471in}{2.390294in}%
\pgfsys@useobject{currentmarker}{}%
\end{pgfscope}%
\begin{pgfscope}%
\pgfsys@transformshift{0.869471in}{2.390294in}%
\pgfsys@useobject{currentmarker}{}%
\end{pgfscope}%
\begin{pgfscope}%
\pgfsys@transformshift{0.869471in}{2.390294in}%
\pgfsys@useobject{currentmarker}{}%
\end{pgfscope}%
\begin{pgfscope}%
\pgfsys@transformshift{0.869471in}{2.390294in}%
\pgfsys@useobject{currentmarker}{}%
\end{pgfscope}%
\begin{pgfscope}%
\pgfsys@transformshift{0.869471in}{2.390294in}%
\pgfsys@useobject{currentmarker}{}%
\end{pgfscope}%
\begin{pgfscope}%
\pgfsys@transformshift{0.869471in}{2.390294in}%
\pgfsys@useobject{currentmarker}{}%
\end{pgfscope}%
\begin{pgfscope}%
\pgfsys@transformshift{0.869471in}{2.390294in}%
\pgfsys@useobject{currentmarker}{}%
\end{pgfscope}%
\begin{pgfscope}%
\pgfsys@transformshift{0.869471in}{2.390294in}%
\pgfsys@useobject{currentmarker}{}%
\end{pgfscope}%
\begin{pgfscope}%
\pgfsys@transformshift{0.869471in}{2.390294in}%
\pgfsys@useobject{currentmarker}{}%
\end{pgfscope}%
\begin{pgfscope}%
\pgfsys@transformshift{0.869471in}{2.390294in}%
\pgfsys@useobject{currentmarker}{}%
\end{pgfscope}%
\begin{pgfscope}%
\pgfsys@transformshift{0.869471in}{2.390294in}%
\pgfsys@useobject{currentmarker}{}%
\end{pgfscope}%
\begin{pgfscope}%
\pgfsys@transformshift{0.869471in}{2.390294in}%
\pgfsys@useobject{currentmarker}{}%
\end{pgfscope}%
\begin{pgfscope}%
\pgfsys@transformshift{0.869471in}{2.390294in}%
\pgfsys@useobject{currentmarker}{}%
\end{pgfscope}%
\begin{pgfscope}%
\pgfsys@transformshift{0.869471in}{2.390294in}%
\pgfsys@useobject{currentmarker}{}%
\end{pgfscope}%
\begin{pgfscope}%
\pgfsys@transformshift{0.869471in}{2.390294in}%
\pgfsys@useobject{currentmarker}{}%
\end{pgfscope}%
\begin{pgfscope}%
\pgfsys@transformshift{0.869471in}{2.390294in}%
\pgfsys@useobject{currentmarker}{}%
\end{pgfscope}%
\begin{pgfscope}%
\pgfsys@transformshift{0.869471in}{2.390294in}%
\pgfsys@useobject{currentmarker}{}%
\end{pgfscope}%
\begin{pgfscope}%
\pgfsys@transformshift{0.869471in}{2.390294in}%
\pgfsys@useobject{currentmarker}{}%
\end{pgfscope}%
\begin{pgfscope}%
\pgfsys@transformshift{0.869471in}{2.390294in}%
\pgfsys@useobject{currentmarker}{}%
\end{pgfscope}%
\begin{pgfscope}%
\pgfsys@transformshift{0.869471in}{2.390294in}%
\pgfsys@useobject{currentmarker}{}%
\end{pgfscope}%
\begin{pgfscope}%
\pgfsys@transformshift{0.869471in}{2.390294in}%
\pgfsys@useobject{currentmarker}{}%
\end{pgfscope}%
\begin{pgfscope}%
\pgfsys@transformshift{0.869471in}{2.390294in}%
\pgfsys@useobject{currentmarker}{}%
\end{pgfscope}%
\begin{pgfscope}%
\pgfsys@transformshift{0.869471in}{2.390294in}%
\pgfsys@useobject{currentmarker}{}%
\end{pgfscope}%
\begin{pgfscope}%
\pgfsys@transformshift{0.869471in}{2.390294in}%
\pgfsys@useobject{currentmarker}{}%
\end{pgfscope}%
\begin{pgfscope}%
\pgfsys@transformshift{0.869471in}{2.390294in}%
\pgfsys@useobject{currentmarker}{}%
\end{pgfscope}%
\begin{pgfscope}%
\pgfsys@transformshift{0.869471in}{2.390294in}%
\pgfsys@useobject{currentmarker}{}%
\end{pgfscope}%
\begin{pgfscope}%
\pgfsys@transformshift{0.869471in}{2.390294in}%
\pgfsys@useobject{currentmarker}{}%
\end{pgfscope}%
\begin{pgfscope}%
\pgfsys@transformshift{0.869471in}{2.390294in}%
\pgfsys@useobject{currentmarker}{}%
\end{pgfscope}%
\begin{pgfscope}%
\pgfsys@transformshift{0.869471in}{2.390294in}%
\pgfsys@useobject{currentmarker}{}%
\end{pgfscope}%
\begin{pgfscope}%
\pgfsys@transformshift{0.869471in}{2.390294in}%
\pgfsys@useobject{currentmarker}{}%
\end{pgfscope}%
\begin{pgfscope}%
\pgfsys@transformshift{0.869471in}{2.390294in}%
\pgfsys@useobject{currentmarker}{}%
\end{pgfscope}%
\begin{pgfscope}%
\pgfsys@transformshift{0.869471in}{2.390294in}%
\pgfsys@useobject{currentmarker}{}%
\end{pgfscope}%
\begin{pgfscope}%
\pgfsys@transformshift{0.869471in}{2.390294in}%
\pgfsys@useobject{currentmarker}{}%
\end{pgfscope}%
\begin{pgfscope}%
\pgfsys@transformshift{0.869471in}{2.390294in}%
\pgfsys@useobject{currentmarker}{}%
\end{pgfscope}%
\begin{pgfscope}%
\pgfsys@transformshift{0.869471in}{2.390294in}%
\pgfsys@useobject{currentmarker}{}%
\end{pgfscope}%
\begin{pgfscope}%
\pgfsys@transformshift{0.869471in}{2.390294in}%
\pgfsys@useobject{currentmarker}{}%
\end{pgfscope}%
\begin{pgfscope}%
\pgfsys@transformshift{0.869471in}{2.390294in}%
\pgfsys@useobject{currentmarker}{}%
\end{pgfscope}%
\begin{pgfscope}%
\pgfsys@transformshift{0.869471in}{2.390294in}%
\pgfsys@useobject{currentmarker}{}%
\end{pgfscope}%
\begin{pgfscope}%
\pgfsys@transformshift{0.869471in}{2.390294in}%
\pgfsys@useobject{currentmarker}{}%
\end{pgfscope}%
\begin{pgfscope}%
\pgfsys@transformshift{0.869471in}{2.390294in}%
\pgfsys@useobject{currentmarker}{}%
\end{pgfscope}%
\begin{pgfscope}%
\pgfsys@transformshift{0.869471in}{2.390294in}%
\pgfsys@useobject{currentmarker}{}%
\end{pgfscope}%
\begin{pgfscope}%
\pgfsys@transformshift{0.869471in}{2.390294in}%
\pgfsys@useobject{currentmarker}{}%
\end{pgfscope}%
\begin{pgfscope}%
\pgfsys@transformshift{0.869471in}{2.390294in}%
\pgfsys@useobject{currentmarker}{}%
\end{pgfscope}%
\begin{pgfscope}%
\pgfsys@transformshift{0.869471in}{2.390294in}%
\pgfsys@useobject{currentmarker}{}%
\end{pgfscope}%
\begin{pgfscope}%
\pgfsys@transformshift{0.869471in}{2.390294in}%
\pgfsys@useobject{currentmarker}{}%
\end{pgfscope}%
\begin{pgfscope}%
\pgfsys@transformshift{0.869471in}{2.390294in}%
\pgfsys@useobject{currentmarker}{}%
\end{pgfscope}%
\begin{pgfscope}%
\pgfsys@transformshift{0.869471in}{2.390294in}%
\pgfsys@useobject{currentmarker}{}%
\end{pgfscope}%
\begin{pgfscope}%
\pgfsys@transformshift{0.869471in}{2.390294in}%
\pgfsys@useobject{currentmarker}{}%
\end{pgfscope}%
\begin{pgfscope}%
\pgfsys@transformshift{0.869471in}{2.390294in}%
\pgfsys@useobject{currentmarker}{}%
\end{pgfscope}%
\begin{pgfscope}%
\pgfsys@transformshift{0.869471in}{2.390294in}%
\pgfsys@useobject{currentmarker}{}%
\end{pgfscope}%
\begin{pgfscope}%
\pgfsys@transformshift{0.869471in}{2.390294in}%
\pgfsys@useobject{currentmarker}{}%
\end{pgfscope}%
\begin{pgfscope}%
\pgfsys@transformshift{0.869471in}{2.390294in}%
\pgfsys@useobject{currentmarker}{}%
\end{pgfscope}%
\begin{pgfscope}%
\pgfsys@transformshift{0.869471in}{2.390294in}%
\pgfsys@useobject{currentmarker}{}%
\end{pgfscope}%
\begin{pgfscope}%
\pgfsys@transformshift{0.869471in}{2.390294in}%
\pgfsys@useobject{currentmarker}{}%
\end{pgfscope}%
\begin{pgfscope}%
\pgfsys@transformshift{0.869471in}{2.390294in}%
\pgfsys@useobject{currentmarker}{}%
\end{pgfscope}%
\begin{pgfscope}%
\pgfsys@transformshift{0.869471in}{2.390294in}%
\pgfsys@useobject{currentmarker}{}%
\end{pgfscope}%
\begin{pgfscope}%
\pgfsys@transformshift{0.869471in}{2.390294in}%
\pgfsys@useobject{currentmarker}{}%
\end{pgfscope}%
\begin{pgfscope}%
\pgfsys@transformshift{0.869471in}{2.390294in}%
\pgfsys@useobject{currentmarker}{}%
\end{pgfscope}%
\begin{pgfscope}%
\pgfsys@transformshift{0.869471in}{2.390294in}%
\pgfsys@useobject{currentmarker}{}%
\end{pgfscope}%
\begin{pgfscope}%
\pgfsys@transformshift{0.869471in}{2.390294in}%
\pgfsys@useobject{currentmarker}{}%
\end{pgfscope}%
\begin{pgfscope}%
\pgfsys@transformshift{0.869471in}{2.390294in}%
\pgfsys@useobject{currentmarker}{}%
\end{pgfscope}%
\begin{pgfscope}%
\pgfsys@transformshift{0.869471in}{2.390294in}%
\pgfsys@useobject{currentmarker}{}%
\end{pgfscope}%
\begin{pgfscope}%
\pgfsys@transformshift{0.869471in}{2.390294in}%
\pgfsys@useobject{currentmarker}{}%
\end{pgfscope}%
\begin{pgfscope}%
\pgfsys@transformshift{0.869471in}{2.390294in}%
\pgfsys@useobject{currentmarker}{}%
\end{pgfscope}%
\begin{pgfscope}%
\pgfsys@transformshift{0.869471in}{2.390294in}%
\pgfsys@useobject{currentmarker}{}%
\end{pgfscope}%
\begin{pgfscope}%
\pgfsys@transformshift{0.869471in}{2.390294in}%
\pgfsys@useobject{currentmarker}{}%
\end{pgfscope}%
\begin{pgfscope}%
\pgfsys@transformshift{0.869471in}{2.390294in}%
\pgfsys@useobject{currentmarker}{}%
\end{pgfscope}%
\begin{pgfscope}%
\pgfsys@transformshift{0.869471in}{2.390294in}%
\pgfsys@useobject{currentmarker}{}%
\end{pgfscope}%
\begin{pgfscope}%
\pgfsys@transformshift{0.869471in}{2.390294in}%
\pgfsys@useobject{currentmarker}{}%
\end{pgfscope}%
\begin{pgfscope}%
\pgfsys@transformshift{0.869471in}{2.390294in}%
\pgfsys@useobject{currentmarker}{}%
\end{pgfscope}%
\begin{pgfscope}%
\pgfsys@transformshift{0.869471in}{2.390294in}%
\pgfsys@useobject{currentmarker}{}%
\end{pgfscope}%
\begin{pgfscope}%
\pgfsys@transformshift{0.869471in}{2.390294in}%
\pgfsys@useobject{currentmarker}{}%
\end{pgfscope}%
\begin{pgfscope}%
\pgfsys@transformshift{0.869471in}{2.390294in}%
\pgfsys@useobject{currentmarker}{}%
\end{pgfscope}%
\begin{pgfscope}%
\pgfsys@transformshift{0.869471in}{2.390294in}%
\pgfsys@useobject{currentmarker}{}%
\end{pgfscope}%
\begin{pgfscope}%
\pgfsys@transformshift{0.869471in}{2.390294in}%
\pgfsys@useobject{currentmarker}{}%
\end{pgfscope}%
\begin{pgfscope}%
\pgfsys@transformshift{0.869471in}{2.390294in}%
\pgfsys@useobject{currentmarker}{}%
\end{pgfscope}%
\begin{pgfscope}%
\pgfsys@transformshift{0.869471in}{2.390294in}%
\pgfsys@useobject{currentmarker}{}%
\end{pgfscope}%
\begin{pgfscope}%
\pgfsys@transformshift{0.869471in}{2.390294in}%
\pgfsys@useobject{currentmarker}{}%
\end{pgfscope}%
\begin{pgfscope}%
\pgfsys@transformshift{0.869471in}{2.390294in}%
\pgfsys@useobject{currentmarker}{}%
\end{pgfscope}%
\begin{pgfscope}%
\pgfsys@transformshift{0.869471in}{2.390294in}%
\pgfsys@useobject{currentmarker}{}%
\end{pgfscope}%
\begin{pgfscope}%
\pgfsys@transformshift{0.876765in}{2.388591in}%
\pgfsys@useobject{currentmarker}{}%
\end{pgfscope}%
\begin{pgfscope}%
\pgfsys@transformshift{0.884756in}{2.385879in}%
\pgfsys@useobject{currentmarker}{}%
\end{pgfscope}%
\begin{pgfscope}%
\pgfsys@transformshift{0.889302in}{2.384946in}%
\pgfsys@useobject{currentmarker}{}%
\end{pgfscope}%
\begin{pgfscope}%
\pgfsys@transformshift{0.894592in}{2.384904in}%
\pgfsys@useobject{currentmarker}{}%
\end{pgfscope}%
\begin{pgfscope}%
\pgfsys@transformshift{0.903626in}{2.382738in}%
\pgfsys@useobject{currentmarker}{}%
\end{pgfscope}%
\begin{pgfscope}%
\pgfsys@transformshift{0.908729in}{2.382493in}%
\pgfsys@useobject{currentmarker}{}%
\end{pgfscope}%
\begin{pgfscope}%
\pgfsys@transformshift{0.915577in}{2.381887in}%
\pgfsys@useobject{currentmarker}{}%
\end{pgfscope}%
\begin{pgfscope}%
\pgfsys@transformshift{0.923450in}{2.379980in}%
\pgfsys@useobject{currentmarker}{}%
\end{pgfscope}%
\begin{pgfscope}%
\pgfsys@transformshift{0.927900in}{2.379760in}%
\pgfsys@useobject{currentmarker}{}%
\end{pgfscope}%
\begin{pgfscope}%
\pgfsys@transformshift{0.933447in}{2.378642in}%
\pgfsys@useobject{currentmarker}{}%
\end{pgfscope}%
\begin{pgfscope}%
\pgfsys@transformshift{0.939869in}{2.378071in}%
\pgfsys@useobject{currentmarker}{}%
\end{pgfscope}%
\begin{pgfscope}%
\pgfsys@transformshift{0.943409in}{2.378289in}%
\pgfsys@useobject{currentmarker}{}%
\end{pgfscope}%
\begin{pgfscope}%
\pgfsys@transformshift{0.948553in}{2.378709in}%
\pgfsys@useobject{currentmarker}{}%
\end{pgfscope}%
\begin{pgfscope}%
\pgfsys@transformshift{0.954553in}{2.378865in}%
\pgfsys@useobject{currentmarker}{}%
\end{pgfscope}%
\begin{pgfscope}%
\pgfsys@transformshift{0.957827in}{2.379284in}%
\pgfsys@useobject{currentmarker}{}%
\end{pgfscope}%
\begin{pgfscope}%
\pgfsys@transformshift{0.962456in}{2.378980in}%
\pgfsys@useobject{currentmarker}{}%
\end{pgfscope}%
\begin{pgfscope}%
\pgfsys@transformshift{0.969335in}{2.379043in}%
\pgfsys@useobject{currentmarker}{}%
\end{pgfscope}%
\begin{pgfscope}%
\pgfsys@transformshift{0.973119in}{2.378985in}%
\pgfsys@useobject{currentmarker}{}%
\end{pgfscope}%
\begin{pgfscope}%
\pgfsys@transformshift{0.978142in}{2.378290in}%
\pgfsys@useobject{currentmarker}{}%
\end{pgfscope}%
\begin{pgfscope}%
\pgfsys@transformshift{0.984956in}{2.378337in}%
\pgfsys@useobject{currentmarker}{}%
\end{pgfscope}%
\begin{pgfscope}%
\pgfsys@transformshift{0.993550in}{2.377378in}%
\pgfsys@useobject{currentmarker}{}%
\end{pgfscope}%
\begin{pgfscope}%
\pgfsys@transformshift{0.998305in}{2.377447in}%
\pgfsys@useobject{currentmarker}{}%
\end{pgfscope}%
\begin{pgfscope}%
\pgfsys@transformshift{1.000921in}{2.377490in}%
\pgfsys@useobject{currentmarker}{}%
\end{pgfscope}%
\begin{pgfscope}%
\pgfsys@transformshift{1.005189in}{2.377399in}%
\pgfsys@useobject{currentmarker}{}%
\end{pgfscope}%
\begin{pgfscope}%
\pgfsys@transformshift{1.011066in}{2.377579in}%
\pgfsys@useobject{currentmarker}{}%
\end{pgfscope}%
\begin{pgfscope}%
\pgfsys@transformshift{1.017715in}{2.377276in}%
\pgfsys@useobject{currentmarker}{}%
\end{pgfscope}%
\begin{pgfscope}%
\pgfsys@transformshift{1.025396in}{2.378151in}%
\pgfsys@useobject{currentmarker}{}%
\end{pgfscope}%
\begin{pgfscope}%
\pgfsys@transformshift{1.035132in}{2.379404in}%
\pgfsys@useobject{currentmarker}{}%
\end{pgfscope}%
\begin{pgfscope}%
\pgfsys@transformshift{1.045826in}{2.378320in}%
\pgfsys@useobject{currentmarker}{}%
\end{pgfscope}%
\begin{pgfscope}%
\pgfsys@transformshift{1.051737in}{2.378446in}%
\pgfsys@useobject{currentmarker}{}%
\end{pgfscope}%
\begin{pgfscope}%
\pgfsys@transformshift{1.054984in}{2.378610in}%
\pgfsys@useobject{currentmarker}{}%
\end{pgfscope}%
\begin{pgfscope}%
\pgfsys@transformshift{1.060036in}{2.378518in}%
\pgfsys@useobject{currentmarker}{}%
\end{pgfscope}%
\begin{pgfscope}%
\pgfsys@transformshift{1.066468in}{2.379107in}%
\pgfsys@useobject{currentmarker}{}%
\end{pgfscope}%
\begin{pgfscope}%
\pgfsys@transformshift{1.070009in}{2.378826in}%
\pgfsys@useobject{currentmarker}{}%
\end{pgfscope}%
\begin{pgfscope}%
\pgfsys@transformshift{1.074468in}{2.379056in}%
\pgfsys@useobject{currentmarker}{}%
\end{pgfscope}%
\begin{pgfscope}%
\pgfsys@transformshift{1.081170in}{2.379095in}%
\pgfsys@useobject{currentmarker}{}%
\end{pgfscope}%
\begin{pgfscope}%
\pgfsys@transformshift{1.088870in}{2.378596in}%
\pgfsys@useobject{currentmarker}{}%
\end{pgfscope}%
\begin{pgfscope}%
\pgfsys@transformshift{1.093110in}{2.378423in}%
\pgfsys@useobject{currentmarker}{}%
\end{pgfscope}%
\begin{pgfscope}%
\pgfsys@transformshift{1.095443in}{2.378485in}%
\pgfsys@useobject{currentmarker}{}%
\end{pgfscope}%
\begin{pgfscope}%
\pgfsys@transformshift{1.100298in}{2.377927in}%
\pgfsys@useobject{currentmarker}{}%
\end{pgfscope}%
\begin{pgfscope}%
\pgfsys@transformshift{1.106899in}{2.378475in}%
\pgfsys@useobject{currentmarker}{}%
\end{pgfscope}%
\begin{pgfscope}%
\pgfsys@transformshift{1.114358in}{2.377823in}%
\pgfsys@useobject{currentmarker}{}%
\end{pgfscope}%
\begin{pgfscope}%
\pgfsys@transformshift{1.118466in}{2.377542in}%
\pgfsys@useobject{currentmarker}{}%
\end{pgfscope}%
\begin{pgfscope}%
\pgfsys@transformshift{1.123760in}{2.377466in}%
\pgfsys@useobject{currentmarker}{}%
\end{pgfscope}%
\begin{pgfscope}%
\pgfsys@transformshift{1.131959in}{2.376928in}%
\pgfsys@useobject{currentmarker}{}%
\end{pgfscope}%
\begin{pgfscope}%
\pgfsys@transformshift{1.142477in}{2.377373in}%
\pgfsys@useobject{currentmarker}{}%
\end{pgfscope}%
\begin{pgfscope}%
\pgfsys@transformshift{1.153715in}{2.376141in}%
\pgfsys@useobject{currentmarker}{}%
\end{pgfscope}%
\begin{pgfscope}%
\pgfsys@transformshift{1.165687in}{2.375477in}%
\pgfsys@useobject{currentmarker}{}%
\end{pgfscope}%
\begin{pgfscope}%
\pgfsys@transformshift{1.178960in}{2.375138in}%
\pgfsys@useobject{currentmarker}{}%
\end{pgfscope}%
\begin{pgfscope}%
\pgfsys@transformshift{1.194579in}{2.373985in}%
\pgfsys@useobject{currentmarker}{}%
\end{pgfscope}%
\begin{pgfscope}%
\pgfsys@transformshift{1.211351in}{2.373981in}%
\pgfsys@useobject{currentmarker}{}%
\end{pgfscope}%
\begin{pgfscope}%
\pgfsys@transformshift{1.220562in}{2.373482in}%
\pgfsys@useobject{currentmarker}{}%
\end{pgfscope}%
\begin{pgfscope}%
\pgfsys@transformshift{1.230498in}{2.373713in}%
\pgfsys@useobject{currentmarker}{}%
\end{pgfscope}%
\begin{pgfscope}%
\pgfsys@transformshift{1.235949in}{2.373309in}%
\pgfsys@useobject{currentmarker}{}%
\end{pgfscope}%
\begin{pgfscope}%
\pgfsys@transformshift{1.243074in}{2.373668in}%
\pgfsys@useobject{currentmarker}{}%
\end{pgfscope}%
\begin{pgfscope}%
\pgfsys@transformshift{1.251713in}{2.373833in}%
\pgfsys@useobject{currentmarker}{}%
\end{pgfscope}%
\begin{pgfscope}%
\pgfsys@transformshift{1.261941in}{2.373424in}%
\pgfsys@useobject{currentmarker}{}%
\end{pgfscope}%
\begin{pgfscope}%
\pgfsys@transformshift{1.272720in}{2.374876in}%
\pgfsys@useobject{currentmarker}{}%
\end{pgfscope}%
\begin{pgfscope}%
\pgfsys@transformshift{1.284232in}{2.374431in}%
\pgfsys@useobject{currentmarker}{}%
\end{pgfscope}%
\begin{pgfscope}%
\pgfsys@transformshift{1.296712in}{2.375734in}%
\pgfsys@useobject{currentmarker}{}%
\end{pgfscope}%
\begin{pgfscope}%
\pgfsys@transformshift{1.311305in}{2.376830in}%
\pgfsys@useobject{currentmarker}{}%
\end{pgfscope}%
\begin{pgfscope}%
\pgfsys@transformshift{1.327413in}{2.376246in}%
\pgfsys@useobject{currentmarker}{}%
\end{pgfscope}%
\begin{pgfscope}%
\pgfsys@transformshift{1.343980in}{2.379302in}%
\pgfsys@useobject{currentmarker}{}%
\end{pgfscope}%
\begin{pgfscope}%
\pgfsys@transformshift{1.361447in}{2.376493in}%
\pgfsys@useobject{currentmarker}{}%
\end{pgfscope}%
\begin{pgfscope}%
\pgfsys@transformshift{1.380062in}{2.379493in}%
\pgfsys@useobject{currentmarker}{}%
\end{pgfscope}%
\begin{pgfscope}%
\pgfsys@transformshift{1.400499in}{2.379333in}%
\pgfsys@useobject{currentmarker}{}%
\end{pgfscope}%
\begin{pgfscope}%
\pgfsys@transformshift{1.422365in}{2.380881in}%
\pgfsys@useobject{currentmarker}{}%
\end{pgfscope}%
\begin{pgfscope}%
\pgfsys@transformshift{1.445697in}{2.385098in}%
\pgfsys@useobject{currentmarker}{}%
\end{pgfscope}%
\begin{pgfscope}%
\pgfsys@transformshift{1.470372in}{2.383694in}%
\pgfsys@useobject{currentmarker}{}%
\end{pgfscope}%
\begin{pgfscope}%
\pgfsys@transformshift{1.483809in}{2.385751in}%
\pgfsys@useobject{currentmarker}{}%
\end{pgfscope}%
\begin{pgfscope}%
\pgfsys@transformshift{1.491191in}{2.384566in}%
\pgfsys@useobject{currentmarker}{}%
\end{pgfscope}%
\begin{pgfscope}%
\pgfsys@transformshift{1.500082in}{2.385807in}%
\pgfsys@useobject{currentmarker}{}%
\end{pgfscope}%
\begin{pgfscope}%
\pgfsys@transformshift{1.510394in}{2.385774in}%
\pgfsys@useobject{currentmarker}{}%
\end{pgfscope}%
\begin{pgfscope}%
\pgfsys@transformshift{1.522692in}{2.387116in}%
\pgfsys@useobject{currentmarker}{}%
\end{pgfscope}%
\begin{pgfscope}%
\pgfsys@transformshift{1.536630in}{2.386919in}%
\pgfsys@useobject{currentmarker}{}%
\end{pgfscope}%
\begin{pgfscope}%
\pgfsys@transformshift{1.551068in}{2.388851in}%
\pgfsys@useobject{currentmarker}{}%
\end{pgfscope}%
\begin{pgfscope}%
\pgfsys@transformshift{1.566236in}{2.386925in}%
\pgfsys@useobject{currentmarker}{}%
\end{pgfscope}%
\begin{pgfscope}%
\pgfsys@transformshift{1.582518in}{2.386982in}%
\pgfsys@useobject{currentmarker}{}%
\end{pgfscope}%
\begin{pgfscope}%
\pgfsys@transformshift{1.600908in}{2.387444in}%
\pgfsys@useobject{currentmarker}{}%
\end{pgfscope}%
\begin{pgfscope}%
\pgfsys@transformshift{1.620268in}{2.387579in}%
\pgfsys@useobject{currentmarker}{}%
\end{pgfscope}%
\begin{pgfscope}%
\pgfsys@transformshift{1.640413in}{2.386407in}%
\pgfsys@useobject{currentmarker}{}%
\end{pgfscope}%
\begin{pgfscope}%
\pgfsys@transformshift{1.651479in}{2.387263in}%
\pgfsys@useobject{currentmarker}{}%
\end{pgfscope}%
\begin{pgfscope}%
\pgfsys@transformshift{1.657583in}{2.387243in}%
\pgfsys@useobject{currentmarker}{}%
\end{pgfscope}%
\begin{pgfscope}%
\pgfsys@transformshift{1.666350in}{2.387753in}%
\pgfsys@useobject{currentmarker}{}%
\end{pgfscope}%
\begin{pgfscope}%
\pgfsys@transformshift{1.676781in}{2.387961in}%
\pgfsys@useobject{currentmarker}{}%
\end{pgfscope}%
\begin{pgfscope}%
\pgfsys@transformshift{1.688375in}{2.388820in}%
\pgfsys@useobject{currentmarker}{}%
\end{pgfscope}%
\begin{pgfscope}%
\pgfsys@transformshift{1.694731in}{2.389520in}%
\pgfsys@useobject{currentmarker}{}%
\end{pgfscope}%
\begin{pgfscope}%
\pgfsys@transformshift{1.698248in}{2.389497in}%
\pgfsys@useobject{currentmarker}{}%
\end{pgfscope}%
\begin{pgfscope}%
\pgfsys@transformshift{1.703623in}{2.389736in}%
\pgfsys@useobject{currentmarker}{}%
\end{pgfscope}%
\begin{pgfscope}%
\pgfsys@transformshift{1.709909in}{2.390081in}%
\pgfsys@useobject{currentmarker}{}%
\end{pgfscope}%
\begin{pgfscope}%
\pgfsys@transformshift{1.717544in}{2.389980in}%
\pgfsys@useobject{currentmarker}{}%
\end{pgfscope}%
\begin{pgfscope}%
\pgfsys@transformshift{1.721738in}{2.390189in}%
\pgfsys@useobject{currentmarker}{}%
\end{pgfscope}%
\begin{pgfscope}%
\pgfsys@transformshift{1.726714in}{2.389748in}%
\pgfsys@useobject{currentmarker}{}%
\end{pgfscope}%
\begin{pgfscope}%
\pgfsys@transformshift{1.729460in}{2.389847in}%
\pgfsys@useobject{currentmarker}{}%
\end{pgfscope}%
\begin{pgfscope}%
\pgfsys@transformshift{1.733082in}{2.389789in}%
\pgfsys@useobject{currentmarker}{}%
\end{pgfscope}%
\begin{pgfscope}%
\pgfsys@transformshift{1.737903in}{2.390425in}%
\pgfsys@useobject{currentmarker}{}%
\end{pgfscope}%
\begin{pgfscope}%
\pgfsys@transformshift{1.744399in}{2.391114in}%
\pgfsys@useobject{currentmarker}{}%
\end{pgfscope}%
\begin{pgfscope}%
\pgfsys@transformshift{1.752023in}{2.391231in}%
\pgfsys@useobject{currentmarker}{}%
\end{pgfscope}%
\begin{pgfscope}%
\pgfsys@transformshift{1.756217in}{2.391295in}%
\pgfsys@useobject{currentmarker}{}%
\end{pgfscope}%
\begin{pgfscope}%
\pgfsys@transformshift{1.758521in}{2.391402in}%
\pgfsys@useobject{currentmarker}{}%
\end{pgfscope}%
\begin{pgfscope}%
\pgfsys@transformshift{1.762138in}{2.391189in}%
\pgfsys@useobject{currentmarker}{}%
\end{pgfscope}%
\begin{pgfscope}%
\pgfsys@transformshift{1.767592in}{2.391709in}%
\pgfsys@useobject{currentmarker}{}%
\end{pgfscope}%
\begin{pgfscope}%
\pgfsys@transformshift{1.775770in}{2.389907in}%
\pgfsys@useobject{currentmarker}{}%
\end{pgfscope}%
\begin{pgfscope}%
\pgfsys@transformshift{1.784813in}{2.391317in}%
\pgfsys@useobject{currentmarker}{}%
\end{pgfscope}%
\begin{pgfscope}%
\pgfsys@transformshift{1.789675in}{2.390017in}%
\pgfsys@useobject{currentmarker}{}%
\end{pgfscope}%
\begin{pgfscope}%
\pgfsys@transformshift{1.792391in}{2.390556in}%
\pgfsys@useobject{currentmarker}{}%
\end{pgfscope}%
\begin{pgfscope}%
\pgfsys@transformshift{1.796372in}{2.390274in}%
\pgfsys@useobject{currentmarker}{}%
\end{pgfscope}%
\begin{pgfscope}%
\pgfsys@transformshift{1.802135in}{2.391460in}%
\pgfsys@useobject{currentmarker}{}%
\end{pgfscope}%
\begin{pgfscope}%
\pgfsys@transformshift{1.809830in}{2.391487in}%
\pgfsys@useobject{currentmarker}{}%
\end{pgfscope}%
\begin{pgfscope}%
\pgfsys@transformshift{1.818664in}{2.392707in}%
\pgfsys@useobject{currentmarker}{}%
\end{pgfscope}%
\begin{pgfscope}%
\pgfsys@transformshift{1.823569in}{2.392739in}%
\pgfsys@useobject{currentmarker}{}%
\end{pgfscope}%
\begin{pgfscope}%
\pgfsys@transformshift{1.829404in}{2.391396in}%
\pgfsys@useobject{currentmarker}{}%
\end{pgfscope}%
\begin{pgfscope}%
\pgfsys@transformshift{1.832694in}{2.391245in}%
\pgfsys@useobject{currentmarker}{}%
\end{pgfscope}%
\begin{pgfscope}%
\pgfsys@transformshift{1.836791in}{2.390789in}%
\pgfsys@useobject{currentmarker}{}%
\end{pgfscope}%
\begin{pgfscope}%
\pgfsys@transformshift{1.839056in}{2.390693in}%
\pgfsys@useobject{currentmarker}{}%
\end{pgfscope}%
\begin{pgfscope}%
\pgfsys@transformshift{1.840277in}{2.390441in}%
\pgfsys@useobject{currentmarker}{}%
\end{pgfscope}%
\begin{pgfscope}%
\pgfsys@transformshift{1.842329in}{2.390458in}%
\pgfsys@useobject{currentmarker}{}%
\end{pgfscope}%
\begin{pgfscope}%
\pgfsys@transformshift{1.845301in}{2.390185in}%
\pgfsys@useobject{currentmarker}{}%
\end{pgfscope}%
\begin{pgfscope}%
\pgfsys@transformshift{1.849646in}{2.390482in}%
\pgfsys@useobject{currentmarker}{}%
\end{pgfscope}%
\begin{pgfscope}%
\pgfsys@transformshift{1.855231in}{2.390369in}%
\pgfsys@useobject{currentmarker}{}%
\end{pgfscope}%
\begin{pgfscope}%
\pgfsys@transformshift{1.862575in}{2.390340in}%
\pgfsys@useobject{currentmarker}{}%
\end{pgfscope}%
\begin{pgfscope}%
\pgfsys@transformshift{1.872169in}{2.390585in}%
\pgfsys@useobject{currentmarker}{}%
\end{pgfscope}%
\begin{pgfscope}%
\pgfsys@transformshift{1.882636in}{2.390679in}%
\pgfsys@useobject{currentmarker}{}%
\end{pgfscope}%
\begin{pgfscope}%
\pgfsys@transformshift{1.894111in}{2.390850in}%
\pgfsys@useobject{currentmarker}{}%
\end{pgfscope}%
\begin{pgfscope}%
\pgfsys@transformshift{1.906344in}{2.389594in}%
\pgfsys@useobject{currentmarker}{}%
\end{pgfscope}%
\begin{pgfscope}%
\pgfsys@transformshift{1.913107in}{2.389516in}%
\pgfsys@useobject{currentmarker}{}%
\end{pgfscope}%
\begin{pgfscope}%
\pgfsys@transformshift{1.920636in}{2.389452in}%
\pgfsys@useobject{currentmarker}{}%
\end{pgfscope}%
\begin{pgfscope}%
\pgfsys@transformshift{1.929176in}{2.389083in}%
\pgfsys@useobject{currentmarker}{}%
\end{pgfscope}%
\begin{pgfscope}%
\pgfsys@transformshift{1.938993in}{2.388369in}%
\pgfsys@useobject{currentmarker}{}%
\end{pgfscope}%
\begin{pgfscope}%
\pgfsys@transformshift{1.951289in}{2.388531in}%
\pgfsys@useobject{currentmarker}{}%
\end{pgfscope}%
\begin{pgfscope}%
\pgfsys@transformshift{1.967272in}{2.389636in}%
\pgfsys@useobject{currentmarker}{}%
\end{pgfscope}%
\begin{pgfscope}%
\pgfsys@transformshift{1.986184in}{2.386936in}%
\pgfsys@useobject{currentmarker}{}%
\end{pgfscope}%
\begin{pgfscope}%
\pgfsys@transformshift{2.007162in}{2.387643in}%
\pgfsys@useobject{currentmarker}{}%
\end{pgfscope}%
\begin{pgfscope}%
\pgfsys@transformshift{2.028881in}{2.387008in}%
\pgfsys@useobject{currentmarker}{}%
\end{pgfscope}%
\begin{pgfscope}%
\pgfsys@transformshift{2.040828in}{2.386768in}%
\pgfsys@useobject{currentmarker}{}%
\end{pgfscope}%
\begin{pgfscope}%
\pgfsys@transformshift{2.053681in}{2.385226in}%
\pgfsys@useobject{currentmarker}{}%
\end{pgfscope}%
\begin{pgfscope}%
\pgfsys@transformshift{2.060767in}{2.384534in}%
\pgfsys@useobject{currentmarker}{}%
\end{pgfscope}%
\begin{pgfscope}%
\pgfsys@transformshift{2.064659in}{2.384102in}%
\pgfsys@useobject{currentmarker}{}%
\end{pgfscope}%
\begin{pgfscope}%
\pgfsys@transformshift{2.069597in}{2.383887in}%
\pgfsys@useobject{currentmarker}{}%
\end{pgfscope}%
\begin{pgfscope}%
\pgfsys@transformshift{2.075697in}{2.384020in}%
\pgfsys@useobject{currentmarker}{}%
\end{pgfscope}%
\begin{pgfscope}%
\pgfsys@transformshift{2.082731in}{2.383451in}%
\pgfsys@useobject{currentmarker}{}%
\end{pgfscope}%
\begin{pgfscope}%
\pgfsys@transformshift{2.091090in}{2.383625in}%
\pgfsys@useobject{currentmarker}{}%
\end{pgfscope}%
\begin{pgfscope}%
\pgfsys@transformshift{2.100687in}{2.383685in}%
\pgfsys@useobject{currentmarker}{}%
\end{pgfscope}%
\begin{pgfscope}%
\pgfsys@transformshift{2.111972in}{2.384911in}%
\pgfsys@useobject{currentmarker}{}%
\end{pgfscope}%
\begin{pgfscope}%
\pgfsys@transformshift{2.124734in}{2.383445in}%
\pgfsys@useobject{currentmarker}{}%
\end{pgfscope}%
\begin{pgfscope}%
\pgfsys@transformshift{2.138936in}{2.382486in}%
\pgfsys@useobject{currentmarker}{}%
\end{pgfscope}%
\begin{pgfscope}%
\pgfsys@transformshift{2.154046in}{2.383197in}%
\pgfsys@useobject{currentmarker}{}%
\end{pgfscope}%
\begin{pgfscope}%
\pgfsys@transformshift{2.162348in}{2.383748in}%
\pgfsys@useobject{currentmarker}{}%
\end{pgfscope}%
\begin{pgfscope}%
\pgfsys@transformshift{2.171315in}{2.383039in}%
\pgfsys@useobject{currentmarker}{}%
\end{pgfscope}%
\begin{pgfscope}%
\pgfsys@transformshift{2.176202in}{2.383805in}%
\pgfsys@useobject{currentmarker}{}%
\end{pgfscope}%
\begin{pgfscope}%
\pgfsys@transformshift{2.181812in}{2.383628in}%
\pgfsys@useobject{currentmarker}{}%
\end{pgfscope}%
\begin{pgfscope}%
\pgfsys@transformshift{2.184840in}{2.384233in}%
\pgfsys@useobject{currentmarker}{}%
\end{pgfscope}%
\begin{pgfscope}%
\pgfsys@transformshift{2.188927in}{2.384723in}%
\pgfsys@useobject{currentmarker}{}%
\end{pgfscope}%
\begin{pgfscope}%
\pgfsys@transformshift{2.194538in}{2.385779in}%
\pgfsys@useobject{currentmarker}{}%
\end{pgfscope}%
\begin{pgfscope}%
\pgfsys@transformshift{2.202132in}{2.386329in}%
\pgfsys@useobject{currentmarker}{}%
\end{pgfscope}%
\begin{pgfscope}%
\pgfsys@transformshift{2.210958in}{2.388414in}%
\pgfsys@useobject{currentmarker}{}%
\end{pgfscope}%
\begin{pgfscope}%
\pgfsys@transformshift{2.222320in}{2.391539in}%
\pgfsys@useobject{currentmarker}{}%
\end{pgfscope}%
\begin{pgfscope}%
\pgfsys@transformshift{2.235110in}{2.391228in}%
\pgfsys@useobject{currentmarker}{}%
\end{pgfscope}%
\begin{pgfscope}%
\pgfsys@transformshift{2.248546in}{2.394762in}%
\pgfsys@useobject{currentmarker}{}%
\end{pgfscope}%
\begin{pgfscope}%
\pgfsys@transformshift{2.256076in}{2.393469in}%
\pgfsys@useobject{currentmarker}{}%
\end{pgfscope}%
\begin{pgfscope}%
\pgfsys@transformshift{2.260213in}{2.394211in}%
\pgfsys@useobject{currentmarker}{}%
\end{pgfscope}%
\begin{pgfscope}%
\pgfsys@transformshift{2.265011in}{2.393305in}%
\pgfsys@useobject{currentmarker}{}%
\end{pgfscope}%
\begin{pgfscope}%
\pgfsys@transformshift{2.271160in}{2.394315in}%
\pgfsys@useobject{currentmarker}{}%
\end{pgfscope}%
\begin{pgfscope}%
\pgfsys@transformshift{2.278296in}{2.394358in}%
\pgfsys@useobject{currentmarker}{}%
\end{pgfscope}%
\begin{pgfscope}%
\pgfsys@transformshift{2.287293in}{2.395276in}%
\pgfsys@useobject{currentmarker}{}%
\end{pgfscope}%
\begin{pgfscope}%
\pgfsys@transformshift{2.299439in}{2.395282in}%
\pgfsys@useobject{currentmarker}{}%
\end{pgfscope}%
\begin{pgfscope}%
\pgfsys@transformshift{2.312613in}{2.396213in}%
\pgfsys@useobject{currentmarker}{}%
\end{pgfscope}%
\begin{pgfscope}%
\pgfsys@transformshift{2.319876in}{2.396312in}%
\pgfsys@useobject{currentmarker}{}%
\end{pgfscope}%
\begin{pgfscope}%
\pgfsys@transformshift{2.328128in}{2.394952in}%
\pgfsys@useobject{currentmarker}{}%
\end{pgfscope}%
\begin{pgfscope}%
\pgfsys@transformshift{2.332722in}{2.395184in}%
\pgfsys@useobject{currentmarker}{}%
\end{pgfscope}%
\begin{pgfscope}%
\pgfsys@transformshift{2.335252in}{2.395117in}%
\pgfsys@useobject{currentmarker}{}%
\end{pgfscope}%
\begin{pgfscope}%
\pgfsys@transformshift{2.338659in}{2.395009in}%
\pgfsys@useobject{currentmarker}{}%
\end{pgfscope}%
\begin{pgfscope}%
\pgfsys@transformshift{2.340534in}{2.395015in}%
\pgfsys@useobject{currentmarker}{}%
\end{pgfscope}%
\begin{pgfscope}%
\pgfsys@transformshift{2.343641in}{2.394828in}%
\pgfsys@useobject{currentmarker}{}%
\end{pgfscope}%
\begin{pgfscope}%
\pgfsys@transformshift{2.348891in}{2.394711in}%
\pgfsys@useobject{currentmarker}{}%
\end{pgfscope}%
\begin{pgfscope}%
\pgfsys@transformshift{2.356180in}{2.393979in}%
\pgfsys@useobject{currentmarker}{}%
\end{pgfscope}%
\begin{pgfscope}%
\pgfsys@transformshift{2.366015in}{2.394187in}%
\pgfsys@useobject{currentmarker}{}%
\end{pgfscope}%
\begin{pgfscope}%
\pgfsys@transformshift{2.378549in}{2.393096in}%
\pgfsys@useobject{currentmarker}{}%
\end{pgfscope}%
\begin{pgfscope}%
\pgfsys@transformshift{2.393228in}{2.394123in}%
\pgfsys@useobject{currentmarker}{}%
\end{pgfscope}%
\begin{pgfscope}%
\pgfsys@transformshift{2.408453in}{2.391359in}%
\pgfsys@useobject{currentmarker}{}%
\end{pgfscope}%
\begin{pgfscope}%
\pgfsys@transformshift{2.425226in}{2.393387in}%
\pgfsys@useobject{currentmarker}{}%
\end{pgfscope}%
\begin{pgfscope}%
\pgfsys@transformshift{2.442313in}{2.389020in}%
\pgfsys@useobject{currentmarker}{}%
\end{pgfscope}%
\begin{pgfscope}%
\pgfsys@transformshift{2.451966in}{2.389965in}%
\pgfsys@useobject{currentmarker}{}%
\end{pgfscope}%
\begin{pgfscope}%
\pgfsys@transformshift{2.462937in}{2.388059in}%
\pgfsys@useobject{currentmarker}{}%
\end{pgfscope}%
\begin{pgfscope}%
\pgfsys@transformshift{2.469061in}{2.388044in}%
\pgfsys@useobject{currentmarker}{}%
\end{pgfscope}%
\begin{pgfscope}%
\pgfsys@transformshift{2.472425in}{2.387858in}%
\pgfsys@useobject{currentmarker}{}%
\end{pgfscope}%
\begin{pgfscope}%
\pgfsys@transformshift{2.476851in}{2.387436in}%
\pgfsys@useobject{currentmarker}{}%
\end{pgfscope}%
\begin{pgfscope}%
\pgfsys@transformshift{2.479295in}{2.387507in}%
\pgfsys@useobject{currentmarker}{}%
\end{pgfscope}%
\begin{pgfscope}%
\pgfsys@transformshift{2.483303in}{2.387443in}%
\pgfsys@useobject{currentmarker}{}%
\end{pgfscope}%
\begin{pgfscope}%
\pgfsys@transformshift{2.489011in}{2.387051in}%
\pgfsys@useobject{currentmarker}{}%
\end{pgfscope}%
\begin{pgfscope}%
\pgfsys@transformshift{2.496771in}{2.386878in}%
\pgfsys@useobject{currentmarker}{}%
\end{pgfscope}%
\begin{pgfscope}%
\pgfsys@transformshift{2.506683in}{2.386892in}%
\pgfsys@useobject{currentmarker}{}%
\end{pgfscope}%
\begin{pgfscope}%
\pgfsys@transformshift{2.518744in}{2.384800in}%
\pgfsys@useobject{currentmarker}{}%
\end{pgfscope}%
\begin{pgfscope}%
\pgfsys@transformshift{2.532840in}{2.383839in}%
\pgfsys@useobject{currentmarker}{}%
\end{pgfscope}%
\begin{pgfscope}%
\pgfsys@transformshift{2.548201in}{2.381594in}%
\pgfsys@useobject{currentmarker}{}%
\end{pgfscope}%
\begin{pgfscope}%
\pgfsys@transformshift{2.566419in}{2.379648in}%
\pgfsys@useobject{currentmarker}{}%
\end{pgfscope}%
\begin{pgfscope}%
\pgfsys@transformshift{2.586336in}{2.377188in}%
\pgfsys@useobject{currentmarker}{}%
\end{pgfscope}%
\begin{pgfscope}%
\pgfsys@transformshift{2.608855in}{2.374127in}%
\pgfsys@useobject{currentmarker}{}%
\end{pgfscope}%
\begin{pgfscope}%
\pgfsys@transformshift{2.635156in}{2.375854in}%
\pgfsys@useobject{currentmarker}{}%
\end{pgfscope}%
\begin{pgfscope}%
\pgfsys@transformshift{2.664994in}{2.375866in}%
\pgfsys@useobject{currentmarker}{}%
\end{pgfscope}%
\begin{pgfscope}%
\pgfsys@transformshift{2.698038in}{2.374286in}%
\pgfsys@useobject{currentmarker}{}%
\end{pgfscope}%
\begin{pgfscope}%
\pgfsys@transformshift{2.733860in}{2.373884in}%
\pgfsys@useobject{currentmarker}{}%
\end{pgfscope}%
\begin{pgfscope}%
\pgfsys@transformshift{2.770827in}{2.370817in}%
\pgfsys@useobject{currentmarker}{}%
\end{pgfscope}%
\begin{pgfscope}%
\pgfsys@transformshift{2.810117in}{2.371584in}%
\pgfsys@useobject{currentmarker}{}%
\end{pgfscope}%
\begin{pgfscope}%
\pgfsys@transformshift{2.850722in}{2.370269in}%
\pgfsys@useobject{currentmarker}{}%
\end{pgfscope}%
\begin{pgfscope}%
\pgfsys@transformshift{2.892565in}{2.367797in}%
\pgfsys@useobject{currentmarker}{}%
\end{pgfscope}%
\begin{pgfscope}%
\pgfsys@transformshift{2.935198in}{2.366686in}%
\pgfsys@useobject{currentmarker}{}%
\end{pgfscope}%
\begin{pgfscope}%
\pgfsys@transformshift{2.978467in}{2.365169in}%
\pgfsys@useobject{currentmarker}{}%
\end{pgfscope}%
\begin{pgfscope}%
\pgfsys@transformshift{3.022730in}{2.359694in}%
\pgfsys@useobject{currentmarker}{}%
\end{pgfscope}%
\begin{pgfscope}%
\pgfsys@transformshift{3.047259in}{2.359923in}%
\pgfsys@useobject{currentmarker}{}%
\end{pgfscope}%
\begin{pgfscope}%
\pgfsys@transformshift{3.071926in}{2.354169in}%
\pgfsys@useobject{currentmarker}{}%
\end{pgfscope}%
\begin{pgfscope}%
\pgfsys@transformshift{3.085841in}{2.354828in}%
\pgfsys@useobject{currentmarker}{}%
\end{pgfscope}%
\begin{pgfscope}%
\pgfsys@transformshift{3.093467in}{2.354078in}%
\pgfsys@useobject{currentmarker}{}%
\end{pgfscope}%
\begin{pgfscope}%
\pgfsys@transformshift{3.101880in}{2.352993in}%
\pgfsys@useobject{currentmarker}{}%
\end{pgfscope}%
\begin{pgfscope}%
\pgfsys@transformshift{3.111369in}{2.352791in}%
\pgfsys@useobject{currentmarker}{}%
\end{pgfscope}%
\begin{pgfscope}%
\pgfsys@transformshift{3.122161in}{2.352867in}%
\pgfsys@useobject{currentmarker}{}%
\end{pgfscope}%
\begin{pgfscope}%
\pgfsys@transformshift{3.134396in}{2.351963in}%
\pgfsys@useobject{currentmarker}{}%
\end{pgfscope}%
\begin{pgfscope}%
\pgfsys@transformshift{3.147646in}{2.351042in}%
\pgfsys@useobject{currentmarker}{}%
\end{pgfscope}%
\begin{pgfscope}%
\pgfsys@transformshift{3.162499in}{2.350703in}%
\pgfsys@useobject{currentmarker}{}%
\end{pgfscope}%
\begin{pgfscope}%
\pgfsys@transformshift{3.180016in}{2.351681in}%
\pgfsys@useobject{currentmarker}{}%
\end{pgfscope}%
\begin{pgfscope}%
\pgfsys@transformshift{3.199114in}{2.352471in}%
\pgfsys@useobject{currentmarker}{}%
\end{pgfscope}%
\begin{pgfscope}%
\pgfsys@transformshift{3.219194in}{2.351913in}%
\pgfsys@useobject{currentmarker}{}%
\end{pgfscope}%
\begin{pgfscope}%
\pgfsys@transformshift{3.240220in}{2.353175in}%
\pgfsys@useobject{currentmarker}{}%
\end{pgfscope}%
\begin{pgfscope}%
\pgfsys@transformshift{3.262149in}{2.351948in}%
\pgfsys@useobject{currentmarker}{}%
\end{pgfscope}%
\begin{pgfscope}%
\pgfsys@transformshift{3.284975in}{2.352991in}%
\pgfsys@useobject{currentmarker}{}%
\end{pgfscope}%
\begin{pgfscope}%
\pgfsys@transformshift{3.308782in}{2.350312in}%
\pgfsys@useobject{currentmarker}{}%
\end{pgfscope}%
\begin{pgfscope}%
\pgfsys@transformshift{3.321785in}{2.352443in}%
\pgfsys@useobject{currentmarker}{}%
\end{pgfscope}%
\begin{pgfscope}%
\pgfsys@transformshift{3.335540in}{2.350458in}%
\pgfsys@useobject{currentmarker}{}%
\end{pgfscope}%
\begin{pgfscope}%
\pgfsys@transformshift{3.343157in}{2.351093in}%
\pgfsys@useobject{currentmarker}{}%
\end{pgfscope}%
\begin{pgfscope}%
\pgfsys@transformshift{3.347344in}{2.350715in}%
\pgfsys@useobject{currentmarker}{}%
\end{pgfscope}%
\begin{pgfscope}%
\pgfsys@transformshift{3.352601in}{2.351138in}%
\pgfsys@useobject{currentmarker}{}%
\end{pgfscope}%
\begin{pgfscope}%
\pgfsys@transformshift{3.359187in}{2.351063in}%
\pgfsys@useobject{currentmarker}{}%
\end{pgfscope}%
\begin{pgfscope}%
\pgfsys@transformshift{3.362797in}{2.351375in}%
\pgfsys@useobject{currentmarker}{}%
\end{pgfscope}%
\begin{pgfscope}%
\pgfsys@transformshift{3.369087in}{2.350329in}%
\pgfsys@useobject{currentmarker}{}%
\end{pgfscope}%
\begin{pgfscope}%
\pgfsys@transformshift{3.376976in}{2.350796in}%
\pgfsys@useobject{currentmarker}{}%
\end{pgfscope}%
\begin{pgfscope}%
\pgfsys@transformshift{3.385700in}{2.349905in}%
\pgfsys@useobject{currentmarker}{}%
\end{pgfscope}%
\begin{pgfscope}%
\pgfsys@transformshift{3.395661in}{2.349996in}%
\pgfsys@useobject{currentmarker}{}%
\end{pgfscope}%
\begin{pgfscope}%
\pgfsys@transformshift{3.406591in}{2.349382in}%
\pgfsys@useobject{currentmarker}{}%
\end{pgfscope}%
\begin{pgfscope}%
\pgfsys@transformshift{3.412597in}{2.349802in}%
\pgfsys@useobject{currentmarker}{}%
\end{pgfscope}%
\begin{pgfscope}%
\pgfsys@transformshift{3.420315in}{2.348792in}%
\pgfsys@useobject{currentmarker}{}%
\end{pgfscope}%
\begin{pgfscope}%
\pgfsys@transformshift{3.429159in}{2.349208in}%
\pgfsys@useobject{currentmarker}{}%
\end{pgfscope}%
\begin{pgfscope}%
\pgfsys@transformshift{3.439335in}{2.348321in}%
\pgfsys@useobject{currentmarker}{}%
\end{pgfscope}%
\begin{pgfscope}%
\pgfsys@transformshift{3.451228in}{2.348368in}%
\pgfsys@useobject{currentmarker}{}%
\end{pgfscope}%
\begin{pgfscope}%
\pgfsys@transformshift{3.464589in}{2.348168in}%
\pgfsys@useobject{currentmarker}{}%
\end{pgfscope}%
\begin{pgfscope}%
\pgfsys@transformshift{3.478784in}{2.348950in}%
\pgfsys@useobject{currentmarker}{}%
\end{pgfscope}%
\begin{pgfscope}%
\pgfsys@transformshift{3.494816in}{2.346899in}%
\pgfsys@useobject{currentmarker}{}%
\end{pgfscope}%
\begin{pgfscope}%
\pgfsys@transformshift{3.511950in}{2.348568in}%
\pgfsys@useobject{currentmarker}{}%
\end{pgfscope}%
\begin{pgfscope}%
\pgfsys@transformshift{3.530164in}{2.347051in}%
\pgfsys@useobject{currentmarker}{}%
\end{pgfscope}%
\begin{pgfscope}%
\pgfsys@transformshift{3.549410in}{2.347310in}%
\pgfsys@useobject{currentmarker}{}%
\end{pgfscope}%
\begin{pgfscope}%
\pgfsys@transformshift{3.569523in}{2.349173in}%
\pgfsys@useobject{currentmarker}{}%
\end{pgfscope}%
\begin{pgfscope}%
\pgfsys@transformshift{3.590400in}{2.350564in}%
\pgfsys@useobject{currentmarker}{}%
\end{pgfscope}%
\begin{pgfscope}%
\pgfsys@transformshift{3.612889in}{2.351512in}%
\pgfsys@useobject{currentmarker}{}%
\end{pgfscope}%
\begin{pgfscope}%
\pgfsys@transformshift{3.625260in}{2.352001in}%
\pgfsys@useobject{currentmarker}{}%
\end{pgfscope}%
\begin{pgfscope}%
\pgfsys@transformshift{3.638149in}{2.349679in}%
\pgfsys@useobject{currentmarker}{}%
\end{pgfscope}%
\begin{pgfscope}%
\pgfsys@transformshift{3.645330in}{2.350234in}%
\pgfsys@useobject{currentmarker}{}%
\end{pgfscope}%
\begin{pgfscope}%
\pgfsys@transformshift{3.653100in}{2.348232in}%
\pgfsys@useobject{currentmarker}{}%
\end{pgfscope}%
\begin{pgfscope}%
\pgfsys@transformshift{3.657511in}{2.348347in}%
\pgfsys@useobject{currentmarker}{}%
\end{pgfscope}%
\begin{pgfscope}%
\pgfsys@transformshift{3.659890in}{2.347866in}%
\pgfsys@useobject{currentmarker}{}%
\end{pgfscope}%
\begin{pgfscope}%
\pgfsys@transformshift{3.663019in}{2.348124in}%
\pgfsys@useobject{currentmarker}{}%
\end{pgfscope}%
\begin{pgfscope}%
\pgfsys@transformshift{3.664731in}{2.347901in}%
\pgfsys@useobject{currentmarker}{}%
\end{pgfscope}%
\begin{pgfscope}%
\pgfsys@transformshift{3.667520in}{2.347928in}%
\pgfsys@useobject{currentmarker}{}%
\end{pgfscope}%
\begin{pgfscope}%
\pgfsys@transformshift{3.671177in}{2.347822in}%
\pgfsys@useobject{currentmarker}{}%
\end{pgfscope}%
\begin{pgfscope}%
\pgfsys@transformshift{3.675516in}{2.347434in}%
\pgfsys@useobject{currentmarker}{}%
\end{pgfscope}%
\begin{pgfscope}%
\pgfsys@transformshift{3.681044in}{2.348056in}%
\pgfsys@useobject{currentmarker}{}%
\end{pgfscope}%
\begin{pgfscope}%
\pgfsys@transformshift{3.687901in}{2.347263in}%
\pgfsys@useobject{currentmarker}{}%
\end{pgfscope}%
\begin{pgfscope}%
\pgfsys@transformshift{3.696959in}{2.347755in}%
\pgfsys@useobject{currentmarker}{}%
\end{pgfscope}%
\begin{pgfscope}%
\pgfsys@transformshift{3.707670in}{2.346850in}%
\pgfsys@useobject{currentmarker}{}%
\end{pgfscope}%
\begin{pgfscope}%
\pgfsys@transformshift{3.720057in}{2.346766in}%
\pgfsys@useobject{currentmarker}{}%
\end{pgfscope}%
\begin{pgfscope}%
\pgfsys@transformshift{3.734252in}{2.345531in}%
\pgfsys@useobject{currentmarker}{}%
\end{pgfscope}%
\begin{pgfscope}%
\pgfsys@transformshift{3.751020in}{2.345465in}%
\pgfsys@useobject{currentmarker}{}%
\end{pgfscope}%
\begin{pgfscope}%
\pgfsys@transformshift{3.769182in}{2.343294in}%
\pgfsys@useobject{currentmarker}{}%
\end{pgfscope}%
\begin{pgfscope}%
\pgfsys@transformshift{3.789618in}{2.342846in}%
\pgfsys@useobject{currentmarker}{}%
\end{pgfscope}%
\begin{pgfscope}%
\pgfsys@transformshift{3.812291in}{2.343018in}%
\pgfsys@useobject{currentmarker}{}%
\end{pgfscope}%
\begin{pgfscope}%
\pgfsys@transformshift{3.837809in}{2.342063in}%
\pgfsys@useobject{currentmarker}{}%
\end{pgfscope}%
\begin{pgfscope}%
\pgfsys@transformshift{3.864977in}{2.343300in}%
\pgfsys@useobject{currentmarker}{}%
\end{pgfscope}%
\begin{pgfscope}%
\pgfsys@transformshift{3.893633in}{2.343666in}%
\pgfsys@useobject{currentmarker}{}%
\end{pgfscope}%
\begin{pgfscope}%
\pgfsys@transformshift{3.923671in}{2.345492in}%
\pgfsys@useobject{currentmarker}{}%
\end{pgfscope}%
\begin{pgfscope}%
\pgfsys@transformshift{3.955551in}{2.344930in}%
\pgfsys@useobject{currentmarker}{}%
\end{pgfscope}%
\begin{pgfscope}%
\pgfsys@transformshift{3.988758in}{2.346013in}%
\pgfsys@useobject{currentmarker}{}%
\end{pgfscope}%
\begin{pgfscope}%
\pgfsys@transformshift{4.023462in}{2.343250in}%
\pgfsys@useobject{currentmarker}{}%
\end{pgfscope}%
\begin{pgfscope}%
\pgfsys@transformshift{4.060031in}{2.344794in}%
\pgfsys@useobject{currentmarker}{}%
\end{pgfscope}%
\begin{pgfscope}%
\pgfsys@transformshift{4.098185in}{2.343716in}%
\pgfsys@useobject{currentmarker}{}%
\end{pgfscope}%
\begin{pgfscope}%
\pgfsys@transformshift{4.137630in}{2.345799in}%
\pgfsys@useobject{currentmarker}{}%
\end{pgfscope}%
\begin{pgfscope}%
\pgfsys@transformshift{4.178453in}{2.347534in}%
\pgfsys@useobject{currentmarker}{}%
\end{pgfscope}%
\begin{pgfscope}%
\pgfsys@transformshift{4.220980in}{2.346837in}%
\pgfsys@useobject{currentmarker}{}%
\end{pgfscope}%
\begin{pgfscope}%
\pgfsys@transformshift{4.264466in}{2.349966in}%
\pgfsys@useobject{currentmarker}{}%
\end{pgfscope}%
\begin{pgfscope}%
\pgfsys@transformshift{4.308745in}{2.345755in}%
\pgfsys@useobject{currentmarker}{}%
\end{pgfscope}%
\begin{pgfscope}%
\pgfsys@transformshift{4.332960in}{2.349231in}%
\pgfsys@useobject{currentmarker}{}%
\end{pgfscope}%
\begin{pgfscope}%
\pgfsys@transformshift{4.358025in}{2.346947in}%
\pgfsys@useobject{currentmarker}{}%
\end{pgfscope}%
\begin{pgfscope}%
\pgfsys@transformshift{4.371580in}{2.349757in}%
\pgfsys@useobject{currentmarker}{}%
\end{pgfscope}%
\begin{pgfscope}%
\pgfsys@transformshift{4.379189in}{2.349493in}%
\pgfsys@useobject{currentmarker}{}%
\end{pgfscope}%
\begin{pgfscope}%
\pgfsys@transformshift{4.387319in}{2.350854in}%
\pgfsys@useobject{currentmarker}{}%
\end{pgfscope}%
\begin{pgfscope}%
\pgfsys@transformshift{4.391852in}{2.350909in}%
\pgfsys@useobject{currentmarker}{}%
\end{pgfscope}%
\begin{pgfscope}%
\pgfsys@transformshift{4.397551in}{2.351011in}%
\pgfsys@useobject{currentmarker}{}%
\end{pgfscope}%
\begin{pgfscope}%
\pgfsys@transformshift{4.400683in}{2.351144in}%
\pgfsys@useobject{currentmarker}{}%
\end{pgfscope}%
\begin{pgfscope}%
\pgfsys@transformshift{4.404832in}{2.351243in}%
\pgfsys@useobject{currentmarker}{}%
\end{pgfscope}%
\begin{pgfscope}%
\pgfsys@transformshift{4.407098in}{2.351514in}%
\pgfsys@useobject{currentmarker}{}%
\end{pgfscope}%
\begin{pgfscope}%
\pgfsys@transformshift{4.411359in}{2.350636in}%
\pgfsys@useobject{currentmarker}{}%
\end{pgfscope}%
\begin{pgfscope}%
\pgfsys@transformshift{4.417370in}{2.351056in}%
\pgfsys@useobject{currentmarker}{}%
\end{pgfscope}%
\begin{pgfscope}%
\pgfsys@transformshift{4.424537in}{2.350103in}%
\pgfsys@useobject{currentmarker}{}%
\end{pgfscope}%
\begin{pgfscope}%
\pgfsys@transformshift{4.432390in}{2.350504in}%
\pgfsys@useobject{currentmarker}{}%
\end{pgfscope}%
\begin{pgfscope}%
\pgfsys@transformshift{4.442857in}{2.348947in}%
\pgfsys@useobject{currentmarker}{}%
\end{pgfscope}%
\begin{pgfscope}%
\pgfsys@transformshift{4.455470in}{2.349062in}%
\pgfsys@useobject{currentmarker}{}%
\end{pgfscope}%
\begin{pgfscope}%
\pgfsys@transformshift{4.469717in}{2.346698in}%
\pgfsys@useobject{currentmarker}{}%
\end{pgfscope}%
\begin{pgfscope}%
\pgfsys@transformshift{4.486482in}{2.347546in}%
\pgfsys@useobject{currentmarker}{}%
\end{pgfscope}%
\begin{pgfscope}%
\pgfsys@transformshift{4.504672in}{2.346597in}%
\pgfsys@useobject{currentmarker}{}%
\end{pgfscope}%
\begin{pgfscope}%
\pgfsys@transformshift{4.523994in}{2.346299in}%
\pgfsys@useobject{currentmarker}{}%
\end{pgfscope}%
\begin{pgfscope}%
\pgfsys@transformshift{4.547183in}{2.345321in}%
\pgfsys@useobject{currentmarker}{}%
\end{pgfscope}%
\begin{pgfscope}%
\pgfsys@transformshift{4.572588in}{2.344944in}%
\pgfsys@useobject{currentmarker}{}%
\end{pgfscope}%
\begin{pgfscope}%
\pgfsys@transformshift{4.600595in}{2.345682in}%
\pgfsys@useobject{currentmarker}{}%
\end{pgfscope}%
\begin{pgfscope}%
\pgfsys@transformshift{4.630965in}{2.346591in}%
\pgfsys@useobject{currentmarker}{}%
\end{pgfscope}%
\begin{pgfscope}%
\pgfsys@transformshift{4.663093in}{2.345144in}%
\pgfsys@useobject{currentmarker}{}%
\end{pgfscope}%
\begin{pgfscope}%
\pgfsys@transformshift{4.698210in}{2.346472in}%
\pgfsys@useobject{currentmarker}{}%
\end{pgfscope}%
\begin{pgfscope}%
\pgfsys@transformshift{4.735232in}{2.344265in}%
\pgfsys@useobject{currentmarker}{}%
\end{pgfscope}%
\begin{pgfscope}%
\pgfsys@transformshift{4.774584in}{2.346413in}%
\pgfsys@useobject{currentmarker}{}%
\end{pgfscope}%
\begin{pgfscope}%
\pgfsys@transformshift{4.815519in}{2.341666in}%
\pgfsys@useobject{currentmarker}{}%
\end{pgfscope}%
\begin{pgfscope}%
\pgfsys@transformshift{4.858979in}{2.344520in}%
\pgfsys@useobject{currentmarker}{}%
\end{pgfscope}%
\begin{pgfscope}%
\pgfsys@transformshift{4.903392in}{2.340251in}%
\pgfsys@useobject{currentmarker}{}%
\end{pgfscope}%
\begin{pgfscope}%
\pgfsys@transformshift{4.949234in}{2.341940in}%
\pgfsys@useobject{currentmarker}{}%
\end{pgfscope}%
\begin{pgfscope}%
\pgfsys@transformshift{4.995802in}{2.337136in}%
\pgfsys@useobject{currentmarker}{}%
\end{pgfscope}%
\begin{pgfscope}%
\pgfsys@transformshift{5.044023in}{2.339920in}%
\pgfsys@useobject{currentmarker}{}%
\end{pgfscope}%
\begin{pgfscope}%
\pgfsys@transformshift{5.093680in}{2.336423in}%
\pgfsys@useobject{currentmarker}{}%
\end{pgfscope}%
\begin{pgfscope}%
\pgfsys@transformshift{5.144027in}{2.332948in}%
\pgfsys@useobject{currentmarker}{}%
\end{pgfscope}%
\begin{pgfscope}%
\pgfsys@transformshift{5.195812in}{2.331479in}%
\pgfsys@useobject{currentmarker}{}%
\end{pgfscope}%
\begin{pgfscope}%
\pgfsys@transformshift{5.224304in}{2.331784in}%
\pgfsys@useobject{currentmarker}{}%
\end{pgfscope}%
\begin{pgfscope}%
\pgfsys@transformshift{5.254026in}{2.331547in}%
\pgfsys@useobject{currentmarker}{}%
\end{pgfscope}%
\begin{pgfscope}%
\pgfsys@transformshift{5.270267in}{2.329687in}%
\pgfsys@useobject{currentmarker}{}%
\end{pgfscope}%
\begin{pgfscope}%
\pgfsys@transformshift{5.279249in}{2.330086in}%
\pgfsys@useobject{currentmarker}{}%
\end{pgfscope}%
\begin{pgfscope}%
\pgfsys@transformshift{5.289503in}{2.330058in}%
\pgfsys@useobject{currentmarker}{}%
\end{pgfscope}%
\begin{pgfscope}%
\pgfsys@transformshift{5.300751in}{2.331587in}%
\pgfsys@useobject{currentmarker}{}%
\end{pgfscope}%
\begin{pgfscope}%
\pgfsys@transformshift{5.306983in}{2.331951in}%
\pgfsys@useobject{currentmarker}{}%
\end{pgfscope}%
\begin{pgfscope}%
\pgfsys@transformshift{5.310384in}{2.332424in}%
\pgfsys@useobject{currentmarker}{}%
\end{pgfscope}%
\begin{pgfscope}%
\pgfsys@transformshift{5.312141in}{2.333117in}%
\pgfsys@useobject{currentmarker}{}%
\end{pgfscope}%
\begin{pgfscope}%
\pgfsys@transformshift{5.316423in}{2.338131in}%
\pgfsys@useobject{currentmarker}{}%
\end{pgfscope}%
\begin{pgfscope}%
\pgfsys@transformshift{5.322956in}{2.343077in}%
\pgfsys@useobject{currentmarker}{}%
\end{pgfscope}%
\begin{pgfscope}%
\pgfsys@transformshift{5.328873in}{2.352365in}%
\pgfsys@useobject{currentmarker}{}%
\end{pgfscope}%
\begin{pgfscope}%
\pgfsys@transformshift{5.343086in}{2.358210in}%
\pgfsys@useobject{currentmarker}{}%
\end{pgfscope}%
\begin{pgfscope}%
\pgfsys@transformshift{5.349861in}{2.363263in}%
\pgfsys@useobject{currentmarker}{}%
\end{pgfscope}%
\begin{pgfscope}%
\pgfsys@transformshift{5.353934in}{2.365505in}%
\pgfsys@useobject{currentmarker}{}%
\end{pgfscope}%
\begin{pgfscope}%
\pgfsys@transformshift{5.358738in}{2.368935in}%
\pgfsys@useobject{currentmarker}{}%
\end{pgfscope}%
\begin{pgfscope}%
\pgfsys@transformshift{5.364649in}{2.372362in}%
\pgfsys@useobject{currentmarker}{}%
\end{pgfscope}%
\begin{pgfscope}%
\pgfsys@transformshift{5.367860in}{2.374315in}%
\pgfsys@useobject{currentmarker}{}%
\end{pgfscope}%
\begin{pgfscope}%
\pgfsys@transformshift{5.369704in}{2.375249in}%
\pgfsys@useobject{currentmarker}{}%
\end{pgfscope}%
\begin{pgfscope}%
\pgfsys@transformshift{5.371931in}{2.376943in}%
\pgfsys@useobject{currentmarker}{}%
\end{pgfscope}%
\begin{pgfscope}%
\pgfsys@transformshift{5.373157in}{2.377875in}%
\pgfsys@useobject{currentmarker}{}%
\end{pgfscope}%
\end{pgfscope}%
\begin{pgfscope}%
\pgfsetbuttcap%
\pgfsetroundjoin%
\definecolor{currentfill}{rgb}{0.000000,0.000000,0.000000}%
\pgfsetfillcolor{currentfill}%
\pgfsetlinewidth{0.803000pt}%
\definecolor{currentstroke}{rgb}{0.000000,0.000000,0.000000}%
\pgfsetstrokecolor{currentstroke}%
\pgfsetdash{}{0pt}%
\pgfsys@defobject{currentmarker}{\pgfqpoint{0.000000in}{-0.048611in}}{\pgfqpoint{0.000000in}{0.000000in}}{%
\pgfpathmoveto{\pgfqpoint{0.000000in}{0.000000in}}%
\pgfpathlineto{\pgfqpoint{0.000000in}{-0.048611in}}%
\pgfusepath{stroke,fill}%
}%
\begin{pgfscope}%
\pgfsys@transformshift{0.864066in}{0.515000in}%
\pgfsys@useobject{currentmarker}{}%
\end{pgfscope}%
\end{pgfscope}%
\begin{pgfscope}%
\definecolor{textcolor}{rgb}{0.000000,0.000000,0.000000}%
\pgfsetstrokecolor{textcolor}%
\pgfsetfillcolor{textcolor}%
\pgftext[x=0.864066in,y=0.417777in,,top]{\color{textcolor}\rmfamily\fontsize{10.000000}{12.000000}\selectfont \(\displaystyle {0}\)}%
\end{pgfscope}%
\begin{pgfscope}%
\pgfsetbuttcap%
\pgfsetroundjoin%
\definecolor{currentfill}{rgb}{0.000000,0.000000,0.000000}%
\pgfsetfillcolor{currentfill}%
\pgfsetlinewidth{0.803000pt}%
\definecolor{currentstroke}{rgb}{0.000000,0.000000,0.000000}%
\pgfsetstrokecolor{currentstroke}%
\pgfsetdash{}{0pt}%
\pgfsys@defobject{currentmarker}{\pgfqpoint{0.000000in}{-0.048611in}}{\pgfqpoint{0.000000in}{0.000000in}}{%
\pgfpathmoveto{\pgfqpoint{0.000000in}{0.000000in}}%
\pgfpathlineto{\pgfqpoint{0.000000in}{-0.048611in}}%
\pgfusepath{stroke,fill}%
}%
\begin{pgfscope}%
\pgfsys@transformshift{1.483661in}{0.515000in}%
\pgfsys@useobject{currentmarker}{}%
\end{pgfscope}%
\end{pgfscope}%
\begin{pgfscope}%
\definecolor{textcolor}{rgb}{0.000000,0.000000,0.000000}%
\pgfsetstrokecolor{textcolor}%
\pgfsetfillcolor{textcolor}%
\pgftext[x=1.483661in,y=0.417777in,,top]{\color{textcolor}\rmfamily\fontsize{10.000000}{12.000000}\selectfont \(\displaystyle {5}\)}%
\end{pgfscope}%
\begin{pgfscope}%
\pgfsetbuttcap%
\pgfsetroundjoin%
\definecolor{currentfill}{rgb}{0.000000,0.000000,0.000000}%
\pgfsetfillcolor{currentfill}%
\pgfsetlinewidth{0.803000pt}%
\definecolor{currentstroke}{rgb}{0.000000,0.000000,0.000000}%
\pgfsetstrokecolor{currentstroke}%
\pgfsetdash{}{0pt}%
\pgfsys@defobject{currentmarker}{\pgfqpoint{0.000000in}{-0.048611in}}{\pgfqpoint{0.000000in}{0.000000in}}{%
\pgfpathmoveto{\pgfqpoint{0.000000in}{0.000000in}}%
\pgfpathlineto{\pgfqpoint{0.000000in}{-0.048611in}}%
\pgfusepath{stroke,fill}%
}%
\begin{pgfscope}%
\pgfsys@transformshift{2.103257in}{0.515000in}%
\pgfsys@useobject{currentmarker}{}%
\end{pgfscope}%
\end{pgfscope}%
\begin{pgfscope}%
\definecolor{textcolor}{rgb}{0.000000,0.000000,0.000000}%
\pgfsetstrokecolor{textcolor}%
\pgfsetfillcolor{textcolor}%
\pgftext[x=2.103257in,y=0.417777in,,top]{\color{textcolor}\rmfamily\fontsize{10.000000}{12.000000}\selectfont \(\displaystyle {10}\)}%
\end{pgfscope}%
\begin{pgfscope}%
\pgfsetbuttcap%
\pgfsetroundjoin%
\definecolor{currentfill}{rgb}{0.000000,0.000000,0.000000}%
\pgfsetfillcolor{currentfill}%
\pgfsetlinewidth{0.803000pt}%
\definecolor{currentstroke}{rgb}{0.000000,0.000000,0.000000}%
\pgfsetstrokecolor{currentstroke}%
\pgfsetdash{}{0pt}%
\pgfsys@defobject{currentmarker}{\pgfqpoint{0.000000in}{-0.048611in}}{\pgfqpoint{0.000000in}{0.000000in}}{%
\pgfpathmoveto{\pgfqpoint{0.000000in}{0.000000in}}%
\pgfpathlineto{\pgfqpoint{0.000000in}{-0.048611in}}%
\pgfusepath{stroke,fill}%
}%
\begin{pgfscope}%
\pgfsys@transformshift{2.722853in}{0.515000in}%
\pgfsys@useobject{currentmarker}{}%
\end{pgfscope}%
\end{pgfscope}%
\begin{pgfscope}%
\definecolor{textcolor}{rgb}{0.000000,0.000000,0.000000}%
\pgfsetstrokecolor{textcolor}%
\pgfsetfillcolor{textcolor}%
\pgftext[x=2.722853in,y=0.417777in,,top]{\color{textcolor}\rmfamily\fontsize{10.000000}{12.000000}\selectfont \(\displaystyle {15}\)}%
\end{pgfscope}%
\begin{pgfscope}%
\pgfsetbuttcap%
\pgfsetroundjoin%
\definecolor{currentfill}{rgb}{0.000000,0.000000,0.000000}%
\pgfsetfillcolor{currentfill}%
\pgfsetlinewidth{0.803000pt}%
\definecolor{currentstroke}{rgb}{0.000000,0.000000,0.000000}%
\pgfsetstrokecolor{currentstroke}%
\pgfsetdash{}{0pt}%
\pgfsys@defobject{currentmarker}{\pgfqpoint{0.000000in}{-0.048611in}}{\pgfqpoint{0.000000in}{0.000000in}}{%
\pgfpathmoveto{\pgfqpoint{0.000000in}{0.000000in}}%
\pgfpathlineto{\pgfqpoint{0.000000in}{-0.048611in}}%
\pgfusepath{stroke,fill}%
}%
\begin{pgfscope}%
\pgfsys@transformshift{3.342448in}{0.515000in}%
\pgfsys@useobject{currentmarker}{}%
\end{pgfscope}%
\end{pgfscope}%
\begin{pgfscope}%
\definecolor{textcolor}{rgb}{0.000000,0.000000,0.000000}%
\pgfsetstrokecolor{textcolor}%
\pgfsetfillcolor{textcolor}%
\pgftext[x=3.342448in,y=0.417777in,,top]{\color{textcolor}\rmfamily\fontsize{10.000000}{12.000000}\selectfont \(\displaystyle {20}\)}%
\end{pgfscope}%
\begin{pgfscope}%
\pgfsetbuttcap%
\pgfsetroundjoin%
\definecolor{currentfill}{rgb}{0.000000,0.000000,0.000000}%
\pgfsetfillcolor{currentfill}%
\pgfsetlinewidth{0.803000pt}%
\definecolor{currentstroke}{rgb}{0.000000,0.000000,0.000000}%
\pgfsetstrokecolor{currentstroke}%
\pgfsetdash{}{0pt}%
\pgfsys@defobject{currentmarker}{\pgfqpoint{0.000000in}{-0.048611in}}{\pgfqpoint{0.000000in}{0.000000in}}{%
\pgfpathmoveto{\pgfqpoint{0.000000in}{0.000000in}}%
\pgfpathlineto{\pgfqpoint{0.000000in}{-0.048611in}}%
\pgfusepath{stroke,fill}%
}%
\begin{pgfscope}%
\pgfsys@transformshift{3.962044in}{0.515000in}%
\pgfsys@useobject{currentmarker}{}%
\end{pgfscope}%
\end{pgfscope}%
\begin{pgfscope}%
\definecolor{textcolor}{rgb}{0.000000,0.000000,0.000000}%
\pgfsetstrokecolor{textcolor}%
\pgfsetfillcolor{textcolor}%
\pgftext[x=3.962044in,y=0.417777in,,top]{\color{textcolor}\rmfamily\fontsize{10.000000}{12.000000}\selectfont \(\displaystyle {25}\)}%
\end{pgfscope}%
\begin{pgfscope}%
\pgfsetbuttcap%
\pgfsetroundjoin%
\definecolor{currentfill}{rgb}{0.000000,0.000000,0.000000}%
\pgfsetfillcolor{currentfill}%
\pgfsetlinewidth{0.803000pt}%
\definecolor{currentstroke}{rgb}{0.000000,0.000000,0.000000}%
\pgfsetstrokecolor{currentstroke}%
\pgfsetdash{}{0pt}%
\pgfsys@defobject{currentmarker}{\pgfqpoint{0.000000in}{-0.048611in}}{\pgfqpoint{0.000000in}{0.000000in}}{%
\pgfpathmoveto{\pgfqpoint{0.000000in}{0.000000in}}%
\pgfpathlineto{\pgfqpoint{0.000000in}{-0.048611in}}%
\pgfusepath{stroke,fill}%
}%
\begin{pgfscope}%
\pgfsys@transformshift{4.581640in}{0.515000in}%
\pgfsys@useobject{currentmarker}{}%
\end{pgfscope}%
\end{pgfscope}%
\begin{pgfscope}%
\definecolor{textcolor}{rgb}{0.000000,0.000000,0.000000}%
\pgfsetstrokecolor{textcolor}%
\pgfsetfillcolor{textcolor}%
\pgftext[x=4.581640in,y=0.417777in,,top]{\color{textcolor}\rmfamily\fontsize{10.000000}{12.000000}\selectfont \(\displaystyle {30}\)}%
\end{pgfscope}%
\begin{pgfscope}%
\pgfsetbuttcap%
\pgfsetroundjoin%
\definecolor{currentfill}{rgb}{0.000000,0.000000,0.000000}%
\pgfsetfillcolor{currentfill}%
\pgfsetlinewidth{0.803000pt}%
\definecolor{currentstroke}{rgb}{0.000000,0.000000,0.000000}%
\pgfsetstrokecolor{currentstroke}%
\pgfsetdash{}{0pt}%
\pgfsys@defobject{currentmarker}{\pgfqpoint{0.000000in}{-0.048611in}}{\pgfqpoint{0.000000in}{0.000000in}}{%
\pgfpathmoveto{\pgfqpoint{0.000000in}{0.000000in}}%
\pgfpathlineto{\pgfqpoint{0.000000in}{-0.048611in}}%
\pgfusepath{stroke,fill}%
}%
\begin{pgfscope}%
\pgfsys@transformshift{5.201235in}{0.515000in}%
\pgfsys@useobject{currentmarker}{}%
\end{pgfscope}%
\end{pgfscope}%
\begin{pgfscope}%
\definecolor{textcolor}{rgb}{0.000000,0.000000,0.000000}%
\pgfsetstrokecolor{textcolor}%
\pgfsetfillcolor{textcolor}%
\pgftext[x=5.201235in,y=0.417777in,,top]{\color{textcolor}\rmfamily\fontsize{10.000000}{12.000000}\selectfont \(\displaystyle {35}\)}%
\end{pgfscope}%
\begin{pgfscope}%
\definecolor{textcolor}{rgb}{0.000000,0.000000,0.000000}%
\pgfsetstrokecolor{textcolor}%
\pgfsetfillcolor{textcolor}%
\pgftext[x=3.118611in,y=0.238889in,,top]{\color{textcolor}\rmfamily\fontsize{10.000000}{12.000000}\selectfont Position X [\(\displaystyle m\)]}%
\end{pgfscope}%
\begin{pgfscope}%
\pgfsetbuttcap%
\pgfsetroundjoin%
\definecolor{currentfill}{rgb}{0.000000,0.000000,0.000000}%
\pgfsetfillcolor{currentfill}%
\pgfsetlinewidth{0.803000pt}%
\definecolor{currentstroke}{rgb}{0.000000,0.000000,0.000000}%
\pgfsetstrokecolor{currentstroke}%
\pgfsetdash{}{0pt}%
\pgfsys@defobject{currentmarker}{\pgfqpoint{-0.048611in}{0.000000in}}{\pgfqpoint{-0.000000in}{0.000000in}}{%
\pgfpathmoveto{\pgfqpoint{-0.000000in}{0.000000in}}%
\pgfpathlineto{\pgfqpoint{-0.048611in}{0.000000in}}%
\pgfusepath{stroke,fill}%
}%
\begin{pgfscope}%
\pgfsys@transformshift{0.638611in}{0.531211in}%
\pgfsys@useobject{currentmarker}{}%
\end{pgfscope}%
\end{pgfscope}%
\begin{pgfscope}%
\definecolor{textcolor}{rgb}{0.000000,0.000000,0.000000}%
\pgfsetstrokecolor{textcolor}%
\pgfsetfillcolor{textcolor}%
\pgftext[x=0.294444in, y=0.483016in, left, base]{\color{textcolor}\rmfamily\fontsize{10.000000}{12.000000}\selectfont \(\displaystyle {−15}\)}%
\end{pgfscope}%
\begin{pgfscope}%
\pgfsetbuttcap%
\pgfsetroundjoin%
\definecolor{currentfill}{rgb}{0.000000,0.000000,0.000000}%
\pgfsetfillcolor{currentfill}%
\pgfsetlinewidth{0.803000pt}%
\definecolor{currentstroke}{rgb}{0.000000,0.000000,0.000000}%
\pgfsetstrokecolor{currentstroke}%
\pgfsetdash{}{0pt}%
\pgfsys@defobject{currentmarker}{\pgfqpoint{-0.048611in}{0.000000in}}{\pgfqpoint{-0.000000in}{0.000000in}}{%
\pgfpathmoveto{\pgfqpoint{-0.000000in}{0.000000in}}%
\pgfpathlineto{\pgfqpoint{-0.048611in}{0.000000in}}%
\pgfusepath{stroke,fill}%
}%
\begin{pgfscope}%
\pgfsys@transformshift{0.638611in}{1.150806in}%
\pgfsys@useobject{currentmarker}{}%
\end{pgfscope}%
\end{pgfscope}%
\begin{pgfscope}%
\definecolor{textcolor}{rgb}{0.000000,0.000000,0.000000}%
\pgfsetstrokecolor{textcolor}%
\pgfsetfillcolor{textcolor}%
\pgftext[x=0.294444in, y=1.102612in, left, base]{\color{textcolor}\rmfamily\fontsize{10.000000}{12.000000}\selectfont \(\displaystyle {−10}\)}%
\end{pgfscope}%
\begin{pgfscope}%
\pgfsetbuttcap%
\pgfsetroundjoin%
\definecolor{currentfill}{rgb}{0.000000,0.000000,0.000000}%
\pgfsetfillcolor{currentfill}%
\pgfsetlinewidth{0.803000pt}%
\definecolor{currentstroke}{rgb}{0.000000,0.000000,0.000000}%
\pgfsetstrokecolor{currentstroke}%
\pgfsetdash{}{0pt}%
\pgfsys@defobject{currentmarker}{\pgfqpoint{-0.048611in}{0.000000in}}{\pgfqpoint{-0.000000in}{0.000000in}}{%
\pgfpathmoveto{\pgfqpoint{-0.000000in}{0.000000in}}%
\pgfpathlineto{\pgfqpoint{-0.048611in}{0.000000in}}%
\pgfusepath{stroke,fill}%
}%
\begin{pgfscope}%
\pgfsys@transformshift{0.638611in}{1.770402in}%
\pgfsys@useobject{currentmarker}{}%
\end{pgfscope}%
\end{pgfscope}%
\begin{pgfscope}%
\definecolor{textcolor}{rgb}{0.000000,0.000000,0.000000}%
\pgfsetstrokecolor{textcolor}%
\pgfsetfillcolor{textcolor}%
\pgftext[x=0.363889in, y=1.722208in, left, base]{\color{textcolor}\rmfamily\fontsize{10.000000}{12.000000}\selectfont \(\displaystyle {−5}\)}%
\end{pgfscope}%
\begin{pgfscope}%
\pgfsetbuttcap%
\pgfsetroundjoin%
\definecolor{currentfill}{rgb}{0.000000,0.000000,0.000000}%
\pgfsetfillcolor{currentfill}%
\pgfsetlinewidth{0.803000pt}%
\definecolor{currentstroke}{rgb}{0.000000,0.000000,0.000000}%
\pgfsetstrokecolor{currentstroke}%
\pgfsetdash{}{0pt}%
\pgfsys@defobject{currentmarker}{\pgfqpoint{-0.048611in}{0.000000in}}{\pgfqpoint{-0.000000in}{0.000000in}}{%
\pgfpathmoveto{\pgfqpoint{-0.000000in}{0.000000in}}%
\pgfpathlineto{\pgfqpoint{-0.048611in}{0.000000in}}%
\pgfusepath{stroke,fill}%
}%
\begin{pgfscope}%
\pgfsys@transformshift{0.638611in}{2.389998in}%
\pgfsys@useobject{currentmarker}{}%
\end{pgfscope}%
\end{pgfscope}%
\begin{pgfscope}%
\definecolor{textcolor}{rgb}{0.000000,0.000000,0.000000}%
\pgfsetstrokecolor{textcolor}%
\pgfsetfillcolor{textcolor}%
\pgftext[x=0.471944in, y=2.341803in, left, base]{\color{textcolor}\rmfamily\fontsize{10.000000}{12.000000}\selectfont \(\displaystyle {0}\)}%
\end{pgfscope}%
\begin{pgfscope}%
\pgfsetbuttcap%
\pgfsetroundjoin%
\definecolor{currentfill}{rgb}{0.000000,0.000000,0.000000}%
\pgfsetfillcolor{currentfill}%
\pgfsetlinewidth{0.803000pt}%
\definecolor{currentstroke}{rgb}{0.000000,0.000000,0.000000}%
\pgfsetstrokecolor{currentstroke}%
\pgfsetdash{}{0pt}%
\pgfsys@defobject{currentmarker}{\pgfqpoint{-0.048611in}{0.000000in}}{\pgfqpoint{-0.000000in}{0.000000in}}{%
\pgfpathmoveto{\pgfqpoint{-0.000000in}{0.000000in}}%
\pgfpathlineto{\pgfqpoint{-0.048611in}{0.000000in}}%
\pgfusepath{stroke,fill}%
}%
\begin{pgfscope}%
\pgfsys@transformshift{0.638611in}{3.009593in}%
\pgfsys@useobject{currentmarker}{}%
\end{pgfscope}%
\end{pgfscope}%
\begin{pgfscope}%
\definecolor{textcolor}{rgb}{0.000000,0.000000,0.000000}%
\pgfsetstrokecolor{textcolor}%
\pgfsetfillcolor{textcolor}%
\pgftext[x=0.471944in, y=2.961399in, left, base]{\color{textcolor}\rmfamily\fontsize{10.000000}{12.000000}\selectfont \(\displaystyle {5}\)}%
\end{pgfscope}%
\begin{pgfscope}%
\pgfsetbuttcap%
\pgfsetroundjoin%
\definecolor{currentfill}{rgb}{0.000000,0.000000,0.000000}%
\pgfsetfillcolor{currentfill}%
\pgfsetlinewidth{0.803000pt}%
\definecolor{currentstroke}{rgb}{0.000000,0.000000,0.000000}%
\pgfsetstrokecolor{currentstroke}%
\pgfsetdash{}{0pt}%
\pgfsys@defobject{currentmarker}{\pgfqpoint{-0.048611in}{0.000000in}}{\pgfqpoint{-0.000000in}{0.000000in}}{%
\pgfpathmoveto{\pgfqpoint{-0.000000in}{0.000000in}}%
\pgfpathlineto{\pgfqpoint{-0.048611in}{0.000000in}}%
\pgfusepath{stroke,fill}%
}%
\begin{pgfscope}%
\pgfsys@transformshift{0.638611in}{3.629189in}%
\pgfsys@useobject{currentmarker}{}%
\end{pgfscope}%
\end{pgfscope}%
\begin{pgfscope}%
\definecolor{textcolor}{rgb}{0.000000,0.000000,0.000000}%
\pgfsetstrokecolor{textcolor}%
\pgfsetfillcolor{textcolor}%
\pgftext[x=0.402500in, y=3.580995in, left, base]{\color{textcolor}\rmfamily\fontsize{10.000000}{12.000000}\selectfont \(\displaystyle {10}\)}%
\end{pgfscope}%
\begin{pgfscope}%
\definecolor{textcolor}{rgb}{0.000000,0.000000,0.000000}%
\pgfsetstrokecolor{textcolor}%
\pgfsetfillcolor{textcolor}%
\pgftext[x=0.238889in,y=2.363000in,,bottom,rotate=90.000000]{\color{textcolor}\rmfamily\fontsize{10.000000}{12.000000}\selectfont Position Y [\(\displaystyle m\)]}%
\end{pgfscope}%
\begin{pgfscope}%
\pgfpathrectangle{\pgfqpoint{0.638611in}{0.515000in}}{\pgfqpoint{4.960000in}{3.696000in}}%
\pgfusepath{clip}%
\pgfsetrectcap%
\pgfsetroundjoin%
\pgfsetlinewidth{1.505625pt}%
\definecolor{currentstroke}{rgb}{0.121569,0.466667,0.705882}%
\pgfsetstrokecolor{currentstroke}%
\pgfsetdash{}{0pt}%
\pgfpathmoveto{\pgfqpoint{0.864066in}{2.389998in}}%
\pgfpathlineto{\pgfqpoint{4.333801in}{2.389998in}}%
\pgfusepath{stroke}%
\end{pgfscope}%
\begin{pgfscope}%
\pgfsetrectcap%
\pgfsetmiterjoin%
\pgfsetlinewidth{0.803000pt}%
\definecolor{currentstroke}{rgb}{0.000000,0.000000,0.000000}%
\pgfsetstrokecolor{currentstroke}%
\pgfsetdash{}{0pt}%
\pgfpathmoveto{\pgfqpoint{0.638611in}{0.515000in}}%
\pgfpathlineto{\pgfqpoint{0.638611in}{4.211000in}}%
\pgfusepath{stroke}%
\end{pgfscope}%
\begin{pgfscope}%
\pgfsetrectcap%
\pgfsetmiterjoin%
\pgfsetlinewidth{0.803000pt}%
\definecolor{currentstroke}{rgb}{0.000000,0.000000,0.000000}%
\pgfsetstrokecolor{currentstroke}%
\pgfsetdash{}{0pt}%
\pgfpathmoveto{\pgfqpoint{5.598611in}{0.515000in}}%
\pgfpathlineto{\pgfqpoint{5.598611in}{4.211000in}}%
\pgfusepath{stroke}%
\end{pgfscope}%
\begin{pgfscope}%
\pgfsetrectcap%
\pgfsetmiterjoin%
\pgfsetlinewidth{0.803000pt}%
\definecolor{currentstroke}{rgb}{0.000000,0.000000,0.000000}%
\pgfsetstrokecolor{currentstroke}%
\pgfsetdash{}{0pt}%
\pgfpathmoveto{\pgfqpoint{0.638611in}{0.515000in}}%
\pgfpathlineto{\pgfqpoint{5.598611in}{0.515000in}}%
\pgfusepath{stroke}%
\end{pgfscope}%
\begin{pgfscope}%
\pgfsetrectcap%
\pgfsetmiterjoin%
\pgfsetlinewidth{0.803000pt}%
\definecolor{currentstroke}{rgb}{0.000000,0.000000,0.000000}%
\pgfsetstrokecolor{currentstroke}%
\pgfsetdash{}{0pt}%
\pgfpathmoveto{\pgfqpoint{0.638611in}{4.211000in}}%
\pgfpathlineto{\pgfqpoint{5.598611in}{4.211000in}}%
\pgfusepath{stroke}%
\end{pgfscope}%
\begin{pgfscope}%
\pgfsetbuttcap%
\pgfsetmiterjoin%
\definecolor{currentfill}{rgb}{1.000000,1.000000,1.000000}%
\pgfsetfillcolor{currentfill}%
\pgfsetfillopacity{0.800000}%
\pgfsetlinewidth{1.003750pt}%
\definecolor{currentstroke}{rgb}{0.800000,0.800000,0.800000}%
\pgfsetstrokecolor{currentstroke}%
\pgfsetstrokeopacity{0.800000}%
\pgfsetdash{}{0pt}%
\pgfpathmoveto{\pgfqpoint{3.907500in}{3.712667in}}%
\pgfpathlineto{\pgfqpoint{5.501389in}{3.712667in}}%
\pgfpathquadraticcurveto{\pgfqpoint{5.529167in}{3.712667in}}{\pgfqpoint{5.529167in}{3.740444in}}%
\pgfpathlineto{\pgfqpoint{5.529167in}{4.113777in}}%
\pgfpathquadraticcurveto{\pgfqpoint{5.529167in}{4.141555in}}{\pgfqpoint{5.501389in}{4.141555in}}%
\pgfpathlineto{\pgfqpoint{3.907500in}{4.141555in}}%
\pgfpathquadraticcurveto{\pgfqpoint{3.879722in}{4.141555in}}{\pgfqpoint{3.879722in}{4.113777in}}%
\pgfpathlineto{\pgfqpoint{3.879722in}{3.740444in}}%
\pgfpathquadraticcurveto{\pgfqpoint{3.879722in}{3.712667in}}{\pgfqpoint{3.907500in}{3.712667in}}%
\pgfpathclose%
\pgfusepath{stroke,fill}%
\end{pgfscope}%
\begin{pgfscope}%
\pgfsetrectcap%
\pgfsetroundjoin%
\pgfsetlinewidth{1.505625pt}%
\definecolor{currentstroke}{rgb}{0.121569,0.466667,0.705882}%
\pgfsetstrokecolor{currentstroke}%
\pgfsetdash{}{0pt}%
\pgfpathmoveto{\pgfqpoint{3.935278in}{4.037388in}}%
\pgfpathlineto{\pgfqpoint{4.213056in}{4.037388in}}%
\pgfusepath{stroke}%
\end{pgfscope}%
\begin{pgfscope}%
\definecolor{textcolor}{rgb}{0.000000,0.000000,0.000000}%
\pgfsetstrokecolor{textcolor}%
\pgfsetfillcolor{textcolor}%
\pgftext[x=4.324167in,y=3.988777in,left,base]{\color{textcolor}\rmfamily\fontsize{10.000000}{12.000000}\selectfont Ground truth}%
\end{pgfscope}%
\begin{pgfscope}%
\pgfsetbuttcap%
\pgfsetroundjoin%
\definecolor{currentfill}{rgb}{0.121569,0.466667,0.705882}%
\pgfsetfillcolor{currentfill}%
\pgfsetlinewidth{1.003750pt}%
\definecolor{currentstroke}{rgb}{0.121569,0.466667,0.705882}%
\pgfsetstrokecolor{currentstroke}%
\pgfsetdash{}{0pt}%
\pgfsys@defobject{currentmarker}{\pgfqpoint{-0.041667in}{-0.041667in}}{\pgfqpoint{0.041667in}{0.041667in}}{%
\pgfpathmoveto{\pgfqpoint{0.000000in}{-0.041667in}}%
\pgfpathcurveto{\pgfqpoint{0.011050in}{-0.041667in}}{\pgfqpoint{0.021649in}{-0.037276in}}{\pgfqpoint{0.029463in}{-0.029463in}}%
\pgfpathcurveto{\pgfqpoint{0.037276in}{-0.021649in}}{\pgfqpoint{0.041667in}{-0.011050in}}{\pgfqpoint{0.041667in}{0.000000in}}%
\pgfpathcurveto{\pgfqpoint{0.041667in}{0.011050in}}{\pgfqpoint{0.037276in}{0.021649in}}{\pgfqpoint{0.029463in}{0.029463in}}%
\pgfpathcurveto{\pgfqpoint{0.021649in}{0.037276in}}{\pgfqpoint{0.011050in}{0.041667in}}{\pgfqpoint{0.000000in}{0.041667in}}%
\pgfpathcurveto{\pgfqpoint{-0.011050in}{0.041667in}}{\pgfqpoint{-0.021649in}{0.037276in}}{\pgfqpoint{-0.029463in}{0.029463in}}%
\pgfpathcurveto{\pgfqpoint{-0.037276in}{0.021649in}}{\pgfqpoint{-0.041667in}{0.011050in}}{\pgfqpoint{-0.041667in}{0.000000in}}%
\pgfpathcurveto{\pgfqpoint{-0.041667in}{-0.011050in}}{\pgfqpoint{-0.037276in}{-0.021649in}}{\pgfqpoint{-0.029463in}{-0.029463in}}%
\pgfpathcurveto{\pgfqpoint{-0.021649in}{-0.037276in}}{\pgfqpoint{-0.011050in}{-0.041667in}}{\pgfqpoint{0.000000in}{-0.041667in}}%
\pgfpathclose%
\pgfusepath{stroke,fill}%
}%
\begin{pgfscope}%
\pgfsys@transformshift{4.074167in}{3.831625in}%
\pgfsys@useobject{currentmarker}{}%
\end{pgfscope}%
\end{pgfscope}%
\begin{pgfscope}%
\definecolor{textcolor}{rgb}{0.000000,0.000000,0.000000}%
\pgfsetstrokecolor{textcolor}%
\pgfsetfillcolor{textcolor}%
\pgftext[x=4.324167in,y=3.795166in,left,base]{\color{textcolor}\rmfamily\fontsize{10.000000}{12.000000}\selectfont Estimated position}%
\end{pgfscope}%
\end{pgfpicture}%
\makeatother%
\endgroup%
}
%         \caption{ROLEQ's 3D position estimation had the lowest turn error for the 4-meter line experiment.}
%         \label{fig:line28_2D}
%     \end{subfigure}
%     \begin{subfigure}{0.49\textwidth}
%         \centering
%         \resizebox{1\linewidth}{!}{%% Creator: Matplotlib, PGF backend
%%
%% To include the figure in your LaTeX document, write
%%   \input{<filename>.pgf}
%%
%% Make sure the required packages are loaded in your preamble
%%   \usepackage{pgf}
%%
%% and, on pdftex
%%   \usepackage[utf8]{inputenc}\DeclareUnicodeCharacter{2212}{-}
%%
%% or, on luatex and xetex
%%   \usepackage{unicode-math}
%%
%% Figures using additional raster images can only be included by \input if
%% they are in the same directory as the main LaTeX file. For loading figures
%% from other directories you can use the `import` package
%%   \usepackage{import}
%%
%% and then include the figures with
%%   \import{<path to file>}{<filename>.pgf}
%%
%% Matplotlib used the following preamble
%%   \usepackage{fontspec}
%%
\begingroup%
\makeatletter%
\begin{pgfpicture}%
\pgfpathrectangle{\pgfpointorigin}{\pgfqpoint{4.342355in}{4.207622in}}%
\pgfusepath{use as bounding box, clip}%
\begin{pgfscope}%
\pgfsetbuttcap%
\pgfsetmiterjoin%
\definecolor{currentfill}{rgb}{1.000000,1.000000,1.000000}%
\pgfsetfillcolor{currentfill}%
\pgfsetlinewidth{0.000000pt}%
\definecolor{currentstroke}{rgb}{1.000000,1.000000,1.000000}%
\pgfsetstrokecolor{currentstroke}%
\pgfsetdash{}{0pt}%
\pgfpathmoveto{\pgfqpoint{0.000000in}{0.000000in}}%
\pgfpathlineto{\pgfqpoint{4.342355in}{0.000000in}}%
\pgfpathlineto{\pgfqpoint{4.342355in}{4.207622in}}%
\pgfpathlineto{\pgfqpoint{0.000000in}{4.207622in}}%
\pgfpathclose%
\pgfusepath{fill}%
\end{pgfscope}%
\begin{pgfscope}%
\pgfsetbuttcap%
\pgfsetmiterjoin%
\definecolor{currentfill}{rgb}{1.000000,1.000000,1.000000}%
\pgfsetfillcolor{currentfill}%
\pgfsetlinewidth{0.000000pt}%
\definecolor{currentstroke}{rgb}{0.000000,0.000000,0.000000}%
\pgfsetstrokecolor{currentstroke}%
\pgfsetstrokeopacity{0.000000}%
\pgfsetdash{}{0pt}%
\pgfpathmoveto{\pgfqpoint{0.100000in}{0.212622in}}%
\pgfpathlineto{\pgfqpoint{3.796000in}{0.212622in}}%
\pgfpathlineto{\pgfqpoint{3.796000in}{3.908622in}}%
\pgfpathlineto{\pgfqpoint{0.100000in}{3.908622in}}%
\pgfpathclose%
\pgfusepath{fill}%
\end{pgfscope}%
\begin{pgfscope}%
\pgfsetbuttcap%
\pgfsetmiterjoin%
\definecolor{currentfill}{rgb}{0.950000,0.950000,0.950000}%
\pgfsetfillcolor{currentfill}%
\pgfsetfillopacity{0.500000}%
\pgfsetlinewidth{1.003750pt}%
\definecolor{currentstroke}{rgb}{0.950000,0.950000,0.950000}%
\pgfsetstrokecolor{currentstroke}%
\pgfsetstrokeopacity{0.500000}%
\pgfsetdash{}{0pt}%
\pgfpathmoveto{\pgfqpoint{0.379073in}{1.123938in}}%
\pgfpathlineto{\pgfqpoint{1.599613in}{2.147018in}}%
\pgfpathlineto{\pgfqpoint{1.582647in}{3.622484in}}%
\pgfpathlineto{\pgfqpoint{0.303698in}{2.689165in}}%
\pgfusepath{stroke,fill}%
\end{pgfscope}%
\begin{pgfscope}%
\pgfsetbuttcap%
\pgfsetmiterjoin%
\definecolor{currentfill}{rgb}{0.900000,0.900000,0.900000}%
\pgfsetfillcolor{currentfill}%
\pgfsetfillopacity{0.500000}%
\pgfsetlinewidth{1.003750pt}%
\definecolor{currentstroke}{rgb}{0.900000,0.900000,0.900000}%
\pgfsetstrokecolor{currentstroke}%
\pgfsetstrokeopacity{0.500000}%
\pgfsetdash{}{0pt}%
\pgfpathmoveto{\pgfqpoint{1.599613in}{2.147018in}}%
\pgfpathlineto{\pgfqpoint{3.558144in}{1.577751in}}%
\pgfpathlineto{\pgfqpoint{3.628038in}{3.104037in}}%
\pgfpathlineto{\pgfqpoint{1.582647in}{3.622484in}}%
\pgfusepath{stroke,fill}%
\end{pgfscope}%
\begin{pgfscope}%
\pgfsetbuttcap%
\pgfsetmiterjoin%
\definecolor{currentfill}{rgb}{0.925000,0.925000,0.925000}%
\pgfsetfillcolor{currentfill}%
\pgfsetfillopacity{0.500000}%
\pgfsetlinewidth{1.003750pt}%
\definecolor{currentstroke}{rgb}{0.925000,0.925000,0.925000}%
\pgfsetstrokecolor{currentstroke}%
\pgfsetstrokeopacity{0.500000}%
\pgfsetdash{}{0pt}%
\pgfpathmoveto{\pgfqpoint{0.379073in}{1.123938in}}%
\pgfpathlineto{\pgfqpoint{2.455212in}{0.445871in}}%
\pgfpathlineto{\pgfqpoint{3.558144in}{1.577751in}}%
\pgfpathlineto{\pgfqpoint{1.599613in}{2.147018in}}%
\pgfusepath{stroke,fill}%
\end{pgfscope}%
\begin{pgfscope}%
\pgfsetrectcap%
\pgfsetroundjoin%
\pgfsetlinewidth{0.803000pt}%
\definecolor{currentstroke}{rgb}{0.000000,0.000000,0.000000}%
\pgfsetstrokecolor{currentstroke}%
\pgfsetdash{}{0pt}%
\pgfpathmoveto{\pgfqpoint{0.379073in}{1.123938in}}%
\pgfpathlineto{\pgfqpoint{2.455212in}{0.445871in}}%
\pgfusepath{stroke}%
\end{pgfscope}%
\begin{pgfscope}%
\definecolor{textcolor}{rgb}{0.000000,0.000000,0.000000}%
\pgfsetstrokecolor{textcolor}%
\pgfsetfillcolor{textcolor}%
\pgftext[x=0.730374in, y=0.408886in, left, base,rotate=341.912962]{\color{textcolor}\rmfamily\fontsize{10.000000}{12.000000}\selectfont Position X [\(\displaystyle m\)]}%
\end{pgfscope}%
\begin{pgfscope}%
\pgfsetbuttcap%
\pgfsetroundjoin%
\pgfsetlinewidth{0.803000pt}%
\definecolor{currentstroke}{rgb}{0.690196,0.690196,0.690196}%
\pgfsetstrokecolor{currentstroke}%
\pgfsetdash{}{0pt}%
\pgfpathmoveto{\pgfqpoint{0.504815in}{1.082870in}}%
\pgfpathlineto{\pgfqpoint{1.718725in}{2.112397in}}%
\pgfpathlineto{\pgfqpoint{1.706795in}{3.591016in}}%
\pgfusepath{stroke}%
\end{pgfscope}%
\begin{pgfscope}%
\pgfsetbuttcap%
\pgfsetroundjoin%
\pgfsetlinewidth{0.803000pt}%
\definecolor{currentstroke}{rgb}{0.690196,0.690196,0.690196}%
\pgfsetstrokecolor{currentstroke}%
\pgfsetdash{}{0pt}%
\pgfpathmoveto{\pgfqpoint{0.985429in}{0.925902in}}%
\pgfpathlineto{\pgfqpoint{2.173410in}{1.980238in}}%
\pgfpathlineto{\pgfqpoint{2.180997in}{3.470819in}}%
\pgfusepath{stroke}%
\end{pgfscope}%
\begin{pgfscope}%
\pgfsetbuttcap%
\pgfsetroundjoin%
\pgfsetlinewidth{0.803000pt}%
\definecolor{currentstroke}{rgb}{0.690196,0.690196,0.690196}%
\pgfsetstrokecolor{currentstroke}%
\pgfsetdash{}{0pt}%
\pgfpathmoveto{\pgfqpoint{1.478205in}{0.764961in}}%
\pgfpathlineto{\pgfqpoint{2.638636in}{1.845015in}}%
\pgfpathlineto{\pgfqpoint{2.666674in}{3.347714in}}%
\pgfusepath{stroke}%
\end{pgfscope}%
\begin{pgfscope}%
\pgfsetbuttcap%
\pgfsetroundjoin%
\pgfsetlinewidth{0.803000pt}%
\definecolor{currentstroke}{rgb}{0.690196,0.690196,0.690196}%
\pgfsetstrokecolor{currentstroke}%
\pgfsetdash{}{0pt}%
\pgfpathmoveto{\pgfqpoint{1.983611in}{0.599896in}}%
\pgfpathlineto{\pgfqpoint{3.114775in}{1.706621in}}%
\pgfpathlineto{\pgfqpoint{3.164248in}{3.221594in}}%
\pgfusepath{stroke}%
\end{pgfscope}%
\begin{pgfscope}%
\pgfsetrectcap%
\pgfsetroundjoin%
\pgfsetlinewidth{0.803000pt}%
\definecolor{currentstroke}{rgb}{0.000000,0.000000,0.000000}%
\pgfsetstrokecolor{currentstroke}%
\pgfsetdash{}{0pt}%
\pgfpathmoveto{\pgfqpoint{0.515386in}{1.091835in}}%
\pgfpathlineto{\pgfqpoint{0.483629in}{1.064902in}}%
\pgfusepath{stroke}%
\end{pgfscope}%
\begin{pgfscope}%
\definecolor{textcolor}{rgb}{0.000000,0.000000,0.000000}%
\pgfsetstrokecolor{textcolor}%
\pgfsetfillcolor{textcolor}%
\pgftext[x=0.400245in,y=0.864666in,,top]{\color{textcolor}\rmfamily\fontsize{10.000000}{12.000000}\selectfont \(\displaystyle {0}\)}%
\end{pgfscope}%
\begin{pgfscope}%
\pgfsetrectcap%
\pgfsetroundjoin%
\pgfsetlinewidth{0.803000pt}%
\definecolor{currentstroke}{rgb}{0.000000,0.000000,0.000000}%
\pgfsetstrokecolor{currentstroke}%
\pgfsetdash{}{0pt}%
\pgfpathmoveto{\pgfqpoint{0.995784in}{0.935092in}}%
\pgfpathlineto{\pgfqpoint{0.964673in}{0.907481in}}%
\pgfusepath{stroke}%
\end{pgfscope}%
\begin{pgfscope}%
\definecolor{textcolor}{rgb}{0.000000,0.000000,0.000000}%
\pgfsetstrokecolor{textcolor}%
\pgfsetfillcolor{textcolor}%
\pgftext[x=0.881356in,y=0.704370in,,top]{\color{textcolor}\rmfamily\fontsize{10.000000}{12.000000}\selectfont \(\displaystyle {10}\)}%
\end{pgfscope}%
\begin{pgfscope}%
\pgfsetrectcap%
\pgfsetroundjoin%
\pgfsetlinewidth{0.803000pt}%
\definecolor{currentstroke}{rgb}{0.000000,0.000000,0.000000}%
\pgfsetstrokecolor{currentstroke}%
\pgfsetdash{}{0pt}%
\pgfpathmoveto{\pgfqpoint{1.488331in}{0.774386in}}%
\pgfpathlineto{\pgfqpoint{1.457909in}{0.746071in}}%
\pgfusepath{stroke}%
\end{pgfscope}%
\begin{pgfscope}%
\definecolor{textcolor}{rgb}{0.000000,0.000000,0.000000}%
\pgfsetstrokecolor{textcolor}%
\pgfsetfillcolor{textcolor}%
\pgftext[x=1.374689in,y=0.540002in,,top]{\color{textcolor}\rmfamily\fontsize{10.000000}{12.000000}\selectfont \(\displaystyle {20}\)}%
\end{pgfscope}%
\begin{pgfscope}%
\pgfsetrectcap%
\pgfsetroundjoin%
\pgfsetlinewidth{0.803000pt}%
\definecolor{currentstroke}{rgb}{0.000000,0.000000,0.000000}%
\pgfsetstrokecolor{currentstroke}%
\pgfsetdash{}{0pt}%
\pgfpathmoveto{\pgfqpoint{1.993492in}{0.609563in}}%
\pgfpathlineto{\pgfqpoint{1.963806in}{0.580518in}}%
\pgfusepath{stroke}%
\end{pgfscope}%
\begin{pgfscope}%
\definecolor{textcolor}{rgb}{0.000000,0.000000,0.000000}%
\pgfsetstrokecolor{textcolor}%
\pgfsetfillcolor{textcolor}%
\pgftext[x=1.880715in,y=0.371405in,,top]{\color{textcolor}\rmfamily\fontsize{10.000000}{12.000000}\selectfont \(\displaystyle {30}\)}%
\end{pgfscope}%
\begin{pgfscope}%
\pgfsetrectcap%
\pgfsetroundjoin%
\pgfsetlinewidth{0.803000pt}%
\definecolor{currentstroke}{rgb}{0.000000,0.000000,0.000000}%
\pgfsetstrokecolor{currentstroke}%
\pgfsetdash{}{0pt}%
\pgfpathmoveto{\pgfqpoint{3.558144in}{1.577751in}}%
\pgfpathlineto{\pgfqpoint{2.455212in}{0.445871in}}%
\pgfusepath{stroke}%
\end{pgfscope}%
\begin{pgfscope}%
\definecolor{textcolor}{rgb}{0.000000,0.000000,0.000000}%
\pgfsetstrokecolor{textcolor}%
\pgfsetfillcolor{textcolor}%
\pgftext[x=3.120747in, y=0.305657in, left, base,rotate=45.742112]{\color{textcolor}\rmfamily\fontsize{10.000000}{12.000000}\selectfont Position Y [\(\displaystyle m\)]}%
\end{pgfscope}%
\begin{pgfscope}%
\pgfsetbuttcap%
\pgfsetroundjoin%
\pgfsetlinewidth{0.803000pt}%
\definecolor{currentstroke}{rgb}{0.690196,0.690196,0.690196}%
\pgfsetstrokecolor{currentstroke}%
\pgfsetdash{}{0pt}%
\pgfpathmoveto{\pgfqpoint{0.571795in}{2.884810in}}%
\pgfpathlineto{\pgfqpoint{0.634160in}{1.337756in}}%
\pgfpathlineto{\pgfqpoint{2.686525in}{0.683255in}}%
\pgfusepath{stroke}%
\end{pgfscope}%
\begin{pgfscope}%
\pgfsetbuttcap%
\pgfsetroundjoin%
\pgfsetlinewidth{0.803000pt}%
\definecolor{currentstroke}{rgb}{0.690196,0.690196,0.690196}%
\pgfsetstrokecolor{currentstroke}%
\pgfsetdash{}{0pt}%
\pgfpathmoveto{\pgfqpoint{0.814474in}{3.061906in}}%
\pgfpathlineto{\pgfqpoint{0.865411in}{1.531595in}}%
\pgfpathlineto{\pgfqpoint{2.895855in}{0.898079in}}%
\pgfusepath{stroke}%
\end{pgfscope}%
\begin{pgfscope}%
\pgfsetbuttcap%
\pgfsetroundjoin%
\pgfsetlinewidth{0.803000pt}%
\definecolor{currentstroke}{rgb}{0.690196,0.690196,0.690196}%
\pgfsetstrokecolor{currentstroke}%
\pgfsetdash{}{0pt}%
\pgfpathmoveto{\pgfqpoint{1.049293in}{3.233266in}}%
\pgfpathlineto{\pgfqpoint{1.089490in}{1.719423in}}%
\pgfpathlineto{\pgfqpoint{3.098360in}{1.105898in}}%
\pgfusepath{stroke}%
\end{pgfscope}%
\begin{pgfscope}%
\pgfsetbuttcap%
\pgfsetroundjoin%
\pgfsetlinewidth{0.803000pt}%
\definecolor{currentstroke}{rgb}{0.690196,0.690196,0.690196}%
\pgfsetstrokecolor{currentstroke}%
\pgfsetdash{}{0pt}%
\pgfpathmoveto{\pgfqpoint{1.276629in}{3.399166in}}%
\pgfpathlineto{\pgfqpoint{1.306727in}{1.901515in}}%
\pgfpathlineto{\pgfqpoint{3.294367in}{1.307050in}}%
\pgfusepath{stroke}%
\end{pgfscope}%
\begin{pgfscope}%
\pgfsetbuttcap%
\pgfsetroundjoin%
\pgfsetlinewidth{0.803000pt}%
\definecolor{currentstroke}{rgb}{0.690196,0.690196,0.690196}%
\pgfsetstrokecolor{currentstroke}%
\pgfsetdash{}{0pt}%
\pgfpathmoveto{\pgfqpoint{1.496834in}{3.559862in}}%
\pgfpathlineto{\pgfqpoint{1.517429in}{2.078130in}}%
\pgfpathlineto{\pgfqpoint{3.484185in}{1.501850in}}%
\pgfusepath{stroke}%
\end{pgfscope}%
\begin{pgfscope}%
\pgfsetrectcap%
\pgfsetroundjoin%
\pgfsetlinewidth{0.803000pt}%
\definecolor{currentstroke}{rgb}{0.000000,0.000000,0.000000}%
\pgfsetstrokecolor{currentstroke}%
\pgfsetdash{}{0pt}%
\pgfpathmoveto{\pgfqpoint{2.669241in}{0.688767in}}%
\pgfpathlineto{\pgfqpoint{2.721139in}{0.672217in}}%
\pgfusepath{stroke}%
\end{pgfscope}%
\begin{pgfscope}%
\definecolor{textcolor}{rgb}{0.000000,0.000000,0.000000}%
\pgfsetstrokecolor{textcolor}%
\pgfsetfillcolor{textcolor}%
\pgftext[x=2.863345in,y=0.499176in,,top]{\color{textcolor}\rmfamily\fontsize{10.000000}{12.000000}\selectfont \(\displaystyle {-0.2}\)}%
\end{pgfscope}%
\begin{pgfscope}%
\pgfsetrectcap%
\pgfsetroundjoin%
\pgfsetlinewidth{0.803000pt}%
\definecolor{currentstroke}{rgb}{0.000000,0.000000,0.000000}%
\pgfsetstrokecolor{currentstroke}%
\pgfsetdash{}{0pt}%
\pgfpathmoveto{\pgfqpoint{2.878769in}{0.903410in}}%
\pgfpathlineto{\pgfqpoint{2.930070in}{0.887404in}}%
\pgfusepath{stroke}%
\end{pgfscope}%
\begin{pgfscope}%
\definecolor{textcolor}{rgb}{0.000000,0.000000,0.000000}%
\pgfsetstrokecolor{textcolor}%
\pgfsetfillcolor{textcolor}%
\pgftext[x=3.069864in,y=0.717178in,,top]{\color{textcolor}\rmfamily\fontsize{10.000000}{12.000000}\selectfont \(\displaystyle {-0.1}\)}%
\end{pgfscope}%
\begin{pgfscope}%
\pgfsetrectcap%
\pgfsetroundjoin%
\pgfsetlinewidth{0.803000pt}%
\definecolor{currentstroke}{rgb}{0.000000,0.000000,0.000000}%
\pgfsetstrokecolor{currentstroke}%
\pgfsetdash{}{0pt}%
\pgfpathmoveto{\pgfqpoint{3.081469in}{1.111057in}}%
\pgfpathlineto{\pgfqpoint{3.132183in}{1.095568in}}%
\pgfusepath{stroke}%
\end{pgfscope}%
\begin{pgfscope}%
\definecolor{textcolor}{rgb}{0.000000,0.000000,0.000000}%
\pgfsetstrokecolor{textcolor}%
\pgfsetfillcolor{textcolor}%
\pgftext[x=3.269646in,y=0.928067in,,top]{\color{textcolor}\rmfamily\fontsize{10.000000}{12.000000}\selectfont \(\displaystyle {0.0}\)}%
\end{pgfscope}%
\begin{pgfscope}%
\pgfsetrectcap%
\pgfsetroundjoin%
\pgfsetlinewidth{0.803000pt}%
\definecolor{currentstroke}{rgb}{0.000000,0.000000,0.000000}%
\pgfsetstrokecolor{currentstroke}%
\pgfsetdash{}{0pt}%
\pgfpathmoveto{\pgfqpoint{3.277668in}{1.312044in}}%
\pgfpathlineto{\pgfqpoint{3.327806in}{1.297049in}}%
\pgfusepath{stroke}%
\end{pgfscope}%
\begin{pgfscope}%
\definecolor{textcolor}{rgb}{0.000000,0.000000,0.000000}%
\pgfsetstrokecolor{textcolor}%
\pgfsetfillcolor{textcolor}%
\pgftext[x=3.463014in,y=1.132186in,,top]{\color{textcolor}\rmfamily\fontsize{10.000000}{12.000000}\selectfont \(\displaystyle {0.1}\)}%
\end{pgfscope}%
\begin{pgfscope}%
\pgfsetrectcap%
\pgfsetroundjoin%
\pgfsetlinewidth{0.803000pt}%
\definecolor{currentstroke}{rgb}{0.000000,0.000000,0.000000}%
\pgfsetstrokecolor{currentstroke}%
\pgfsetdash{}{0pt}%
\pgfpathmoveto{\pgfqpoint{3.467673in}{1.506688in}}%
\pgfpathlineto{\pgfqpoint{3.517247in}{1.492162in}}%
\pgfusepath{stroke}%
\end{pgfscope}%
\begin{pgfscope}%
\definecolor{textcolor}{rgb}{0.000000,0.000000,0.000000}%
\pgfsetstrokecolor{textcolor}%
\pgfsetfillcolor{textcolor}%
\pgftext[x=3.650273in,y=1.329856in,,top]{\color{textcolor}\rmfamily\fontsize{10.000000}{12.000000}\selectfont \(\displaystyle {0.2}\)}%
\end{pgfscope}%
\begin{pgfscope}%
\pgfsetrectcap%
\pgfsetroundjoin%
\pgfsetlinewidth{0.803000pt}%
\definecolor{currentstroke}{rgb}{0.000000,0.000000,0.000000}%
\pgfsetstrokecolor{currentstroke}%
\pgfsetdash{}{0pt}%
\pgfpathmoveto{\pgfqpoint{3.558144in}{1.577751in}}%
\pgfpathlineto{\pgfqpoint{3.628038in}{3.104037in}}%
\pgfusepath{stroke}%
\end{pgfscope}%
\begin{pgfscope}%
\definecolor{textcolor}{rgb}{0.000000,0.000000,0.000000}%
\pgfsetstrokecolor{textcolor}%
\pgfsetfillcolor{textcolor}%
\pgftext[x=4.167903in, y=1.963517in, left, base,rotate=87.378092]{\color{textcolor}\rmfamily\fontsize{10.000000}{12.000000}\selectfont Position Z [\(\displaystyle m\)]}%
\end{pgfscope}%
\begin{pgfscope}%
\pgfsetbuttcap%
\pgfsetroundjoin%
\pgfsetlinewidth{0.803000pt}%
\definecolor{currentstroke}{rgb}{0.690196,0.690196,0.690196}%
\pgfsetstrokecolor{currentstroke}%
\pgfsetdash{}{0pt}%
\pgfpathmoveto{\pgfqpoint{3.562413in}{1.670968in}}%
\pgfpathlineto{\pgfqpoint{1.598575in}{2.237310in}}%
\pgfpathlineto{\pgfqpoint{0.374477in}{1.219382in}}%
\pgfusepath{stroke}%
\end{pgfscope}%
\begin{pgfscope}%
\pgfsetbuttcap%
\pgfsetroundjoin%
\pgfsetlinewidth{0.803000pt}%
\definecolor{currentstroke}{rgb}{0.690196,0.690196,0.690196}%
\pgfsetstrokecolor{currentstroke}%
\pgfsetdash{}{0pt}%
\pgfpathmoveto{\pgfqpoint{3.577953in}{2.010320in}}%
\pgfpathlineto{\pgfqpoint{1.594798in}{2.565817in}}%
\pgfpathlineto{\pgfqpoint{0.357737in}{1.567008in}}%
\pgfusepath{stroke}%
\end{pgfscope}%
\begin{pgfscope}%
\pgfsetbuttcap%
\pgfsetroundjoin%
\pgfsetlinewidth{0.803000pt}%
\definecolor{currentstroke}{rgb}{0.690196,0.690196,0.690196}%
\pgfsetstrokecolor{currentstroke}%
\pgfsetdash{}{0pt}%
\pgfpathmoveto{\pgfqpoint{3.593805in}{2.356481in}}%
\pgfpathlineto{\pgfqpoint{1.590948in}{2.900598in}}%
\pgfpathlineto{\pgfqpoint{0.340648in}{1.921877in}}%
\pgfusepath{stroke}%
\end{pgfscope}%
\begin{pgfscope}%
\pgfsetbuttcap%
\pgfsetroundjoin%
\pgfsetlinewidth{0.803000pt}%
\definecolor{currentstroke}{rgb}{0.690196,0.690196,0.690196}%
\pgfsetstrokecolor{currentstroke}%
\pgfsetdash{}{0pt}%
\pgfpathmoveto{\pgfqpoint{3.609978in}{2.709659in}}%
\pgfpathlineto{\pgfqpoint{1.587024in}{3.241834in}}%
\pgfpathlineto{\pgfqpoint{0.323198in}{2.284219in}}%
\pgfusepath{stroke}%
\end{pgfscope}%
\begin{pgfscope}%
\pgfsetbuttcap%
\pgfsetroundjoin%
\pgfsetlinewidth{0.803000pt}%
\definecolor{currentstroke}{rgb}{0.690196,0.690196,0.690196}%
\pgfsetstrokecolor{currentstroke}%
\pgfsetdash{}{0pt}%
\pgfpathmoveto{\pgfqpoint{3.626482in}{3.070068in}}%
\pgfpathlineto{\pgfqpoint{1.583024in}{3.589714in}}%
\pgfpathlineto{\pgfqpoint{0.305378in}{2.654272in}}%
\pgfusepath{stroke}%
\end{pgfscope}%
\begin{pgfscope}%
\pgfsetrectcap%
\pgfsetroundjoin%
\pgfsetlinewidth{0.803000pt}%
\definecolor{currentstroke}{rgb}{0.000000,0.000000,0.000000}%
\pgfsetstrokecolor{currentstroke}%
\pgfsetdash{}{0pt}%
\pgfpathmoveto{\pgfqpoint{3.545929in}{1.675722in}}%
\pgfpathlineto{\pgfqpoint{3.595421in}{1.661449in}}%
\pgfusepath{stroke}%
\end{pgfscope}%
\begin{pgfscope}%
\definecolor{textcolor}{rgb}{0.000000,0.000000,0.000000}%
\pgfsetstrokecolor{textcolor}%
\pgfsetfillcolor{textcolor}%
\pgftext[x=3.816545in,y=1.706967in,,top]{\color{textcolor}\rmfamily\fontsize{10.000000}{12.000000}\selectfont \(\displaystyle {0}\)}%
\end{pgfscope}%
\begin{pgfscope}%
\pgfsetrectcap%
\pgfsetroundjoin%
\pgfsetlinewidth{0.803000pt}%
\definecolor{currentstroke}{rgb}{0.000000,0.000000,0.000000}%
\pgfsetstrokecolor{currentstroke}%
\pgfsetdash{}{0pt}%
\pgfpathmoveto{\pgfqpoint{3.561299in}{2.014985in}}%
\pgfpathlineto{\pgfqpoint{3.611301in}{2.000979in}}%
\pgfusepath{stroke}%
\end{pgfscope}%
\begin{pgfscope}%
\definecolor{textcolor}{rgb}{0.000000,0.000000,0.000000}%
\pgfsetstrokecolor{textcolor}%
\pgfsetfillcolor{textcolor}%
\pgftext[x=3.834554in,y=2.045645in,,top]{\color{textcolor}\rmfamily\fontsize{10.000000}{12.000000}\selectfont \(\displaystyle {1}\)}%
\end{pgfscope}%
\begin{pgfscope}%
\pgfsetrectcap%
\pgfsetroundjoin%
\pgfsetlinewidth{0.803000pt}%
\definecolor{currentstroke}{rgb}{0.000000,0.000000,0.000000}%
\pgfsetstrokecolor{currentstroke}%
\pgfsetdash{}{0pt}%
\pgfpathmoveto{\pgfqpoint{3.576977in}{2.361053in}}%
\pgfpathlineto{\pgfqpoint{3.627501in}{2.347327in}}%
\pgfusepath{stroke}%
\end{pgfscope}%
\begin{pgfscope}%
\definecolor{textcolor}{rgb}{0.000000,0.000000,0.000000}%
\pgfsetstrokecolor{textcolor}%
\pgfsetfillcolor{textcolor}%
\pgftext[x=3.852924in,y=2.391098in,,top]{\color{textcolor}\rmfamily\fontsize{10.000000}{12.000000}\selectfont \(\displaystyle {2}\)}%
\end{pgfscope}%
\begin{pgfscope}%
\pgfsetrectcap%
\pgfsetroundjoin%
\pgfsetlinewidth{0.803000pt}%
\definecolor{currentstroke}{rgb}{0.000000,0.000000,0.000000}%
\pgfsetstrokecolor{currentstroke}%
\pgfsetdash{}{0pt}%
\pgfpathmoveto{\pgfqpoint{3.592973in}{2.714132in}}%
\pgfpathlineto{\pgfqpoint{3.644029in}{2.700701in}}%
\pgfusepath{stroke}%
\end{pgfscope}%
\begin{pgfscope}%
\definecolor{textcolor}{rgb}{0.000000,0.000000,0.000000}%
\pgfsetstrokecolor{textcolor}%
\pgfsetfillcolor{textcolor}%
\pgftext[x=3.871665in,y=2.743532in,,top]{\color{textcolor}\rmfamily\fontsize{10.000000}{12.000000}\selectfont \(\displaystyle {3}\)}%
\end{pgfscope}%
\begin{pgfscope}%
\pgfsetrectcap%
\pgfsetroundjoin%
\pgfsetlinewidth{0.803000pt}%
\definecolor{currentstroke}{rgb}{0.000000,0.000000,0.000000}%
\pgfsetstrokecolor{currentstroke}%
\pgfsetdash{}{0pt}%
\pgfpathmoveto{\pgfqpoint{3.609297in}{3.074439in}}%
\pgfpathlineto{\pgfqpoint{3.660895in}{3.061317in}}%
\pgfusepath{stroke}%
\end{pgfscope}%
\begin{pgfscope}%
\definecolor{textcolor}{rgb}{0.000000,0.000000,0.000000}%
\pgfsetstrokecolor{textcolor}%
\pgfsetfillcolor{textcolor}%
\pgftext[x=3.890788in,y=3.103159in,,top]{\color{textcolor}\rmfamily\fontsize{10.000000}{12.000000}\selectfont \(\displaystyle {4}\)}%
\end{pgfscope}%
\begin{pgfscope}%
\pgfpathrectangle{\pgfqpoint{0.100000in}{0.212622in}}{\pgfqpoint{3.696000in}{3.696000in}}%
\pgfusepath{clip}%
\pgfsetrectcap%
\pgfsetroundjoin%
\pgfsetlinewidth{1.505625pt}%
\definecolor{currentstroke}{rgb}{0.121569,0.466667,0.705882}%
\pgfsetstrokecolor{currentstroke}%
\pgfsetdash{}{0pt}%
\pgfpathmoveto{\pgfqpoint{1.209319in}{1.774867in}}%
\pgfpathlineto{\pgfqpoint{2.545798in}{1.369836in}}%
\pgfusepath{stroke}%
\end{pgfscope}%
\begin{pgfscope}%
\pgfpathrectangle{\pgfqpoint{0.100000in}{0.212622in}}{\pgfqpoint{3.696000in}{3.696000in}}%
\pgfusepath{clip}%
\pgfsetrectcap%
\pgfsetroundjoin%
\pgfsetlinewidth{1.505625pt}%
\definecolor{currentstroke}{rgb}{1.000000,0.000000,0.000000}%
\pgfsetstrokecolor{currentstroke}%
\pgfsetdash{}{0pt}%
\pgfpathmoveto{\pgfqpoint{1.209240in}{1.774803in}}%
\pgfpathlineto{\pgfqpoint{1.209319in}{1.774867in}}%
\pgfusepath{stroke}%
\end{pgfscope}%
\begin{pgfscope}%
\pgfpathrectangle{\pgfqpoint{0.100000in}{0.212622in}}{\pgfqpoint{3.696000in}{3.696000in}}%
\pgfusepath{clip}%
\pgfsetrectcap%
\pgfsetroundjoin%
\pgfsetlinewidth{1.505625pt}%
\definecolor{currentstroke}{rgb}{1.000000,0.000000,0.000000}%
\pgfsetstrokecolor{currentstroke}%
\pgfsetdash{}{0pt}%
\pgfpathmoveto{\pgfqpoint{2.411274in}{2.072746in}}%
\pgfpathlineto{\pgfqpoint{2.545798in}{1.369836in}}%
\pgfusepath{stroke}%
\end{pgfscope}%
\begin{pgfscope}%
\pgfpathrectangle{\pgfqpoint{0.100000in}{0.212622in}}{\pgfqpoint{3.696000in}{3.696000in}}%
\pgfusepath{clip}%
\pgfsetbuttcap%
\pgfsetroundjoin%
\definecolor{currentfill}{rgb}{0.121569,0.466667,0.705882}%
\pgfsetfillcolor{currentfill}%
\pgfsetfillopacity{0.300000}%
\pgfsetlinewidth{1.003750pt}%
\definecolor{currentstroke}{rgb}{0.121569,0.466667,0.705882}%
\pgfsetstrokecolor{currentstroke}%
\pgfsetstrokeopacity{0.300000}%
\pgfsetdash{}{0pt}%
\pgfpathmoveto{\pgfqpoint{1.458454in}{1.942141in}}%
\pgfpathcurveto{\pgfqpoint{1.466690in}{1.942141in}}{\pgfqpoint{1.474590in}{1.945414in}}{\pgfqpoint{1.480414in}{1.951238in}}%
\pgfpathcurveto{\pgfqpoint{1.486238in}{1.957061in}}{\pgfqpoint{1.489510in}{1.964962in}}{\pgfqpoint{1.489510in}{1.973198in}}%
\pgfpathcurveto{\pgfqpoint{1.489510in}{1.981434in}}{\pgfqpoint{1.486238in}{1.989334in}}{\pgfqpoint{1.480414in}{1.995158in}}%
\pgfpathcurveto{\pgfqpoint{1.474590in}{2.000982in}}{\pgfqpoint{1.466690in}{2.004254in}}{\pgfqpoint{1.458454in}{2.004254in}}%
\pgfpathcurveto{\pgfqpoint{1.450217in}{2.004254in}}{\pgfqpoint{1.442317in}{2.000982in}}{\pgfqpoint{1.436493in}{1.995158in}}%
\pgfpathcurveto{\pgfqpoint{1.430669in}{1.989334in}}{\pgfqpoint{1.427397in}{1.981434in}}{\pgfqpoint{1.427397in}{1.973198in}}%
\pgfpathcurveto{\pgfqpoint{1.427397in}{1.964962in}}{\pgfqpoint{1.430669in}{1.957061in}}{\pgfqpoint{1.436493in}{1.951238in}}%
\pgfpathcurveto{\pgfqpoint{1.442317in}{1.945414in}}{\pgfqpoint{1.450217in}{1.942141in}}{\pgfqpoint{1.458454in}{1.942141in}}%
\pgfpathclose%
\pgfusepath{stroke,fill}%
\end{pgfscope}%
\begin{pgfscope}%
\pgfpathrectangle{\pgfqpoint{0.100000in}{0.212622in}}{\pgfqpoint{3.696000in}{3.696000in}}%
\pgfusepath{clip}%
\pgfsetbuttcap%
\pgfsetroundjoin%
\definecolor{currentfill}{rgb}{0.121569,0.466667,0.705882}%
\pgfsetfillcolor{currentfill}%
\pgfsetfillopacity{0.300742}%
\pgfsetlinewidth{1.003750pt}%
\definecolor{currentstroke}{rgb}{0.121569,0.466667,0.705882}%
\pgfsetstrokecolor{currentstroke}%
\pgfsetstrokeopacity{0.300742}%
\pgfsetdash{}{0pt}%
\pgfpathmoveto{\pgfqpoint{1.457712in}{1.939467in}}%
\pgfpathcurveto{\pgfqpoint{1.465948in}{1.939467in}}{\pgfqpoint{1.473849in}{1.942739in}}{\pgfqpoint{1.479672in}{1.948563in}}%
\pgfpathcurveto{\pgfqpoint{1.485496in}{1.954387in}}{\pgfqpoint{1.488769in}{1.962287in}}{\pgfqpoint{1.488769in}{1.970524in}}%
\pgfpathcurveto{\pgfqpoint{1.488769in}{1.978760in}}{\pgfqpoint{1.485496in}{1.986660in}}{\pgfqpoint{1.479672in}{1.992484in}}%
\pgfpathcurveto{\pgfqpoint{1.473849in}{1.998308in}}{\pgfqpoint{1.465948in}{2.001580in}}{\pgfqpoint{1.457712in}{2.001580in}}%
\pgfpathcurveto{\pgfqpoint{1.449476in}{2.001580in}}{\pgfqpoint{1.441576in}{1.998308in}}{\pgfqpoint{1.435752in}{1.992484in}}%
\pgfpathcurveto{\pgfqpoint{1.429928in}{1.986660in}}{\pgfqpoint{1.426656in}{1.978760in}}{\pgfqpoint{1.426656in}{1.970524in}}%
\pgfpathcurveto{\pgfqpoint{1.426656in}{1.962287in}}{\pgfqpoint{1.429928in}{1.954387in}}{\pgfqpoint{1.435752in}{1.948563in}}%
\pgfpathcurveto{\pgfqpoint{1.441576in}{1.942739in}}{\pgfqpoint{1.449476in}{1.939467in}}{\pgfqpoint{1.457712in}{1.939467in}}%
\pgfpathclose%
\pgfusepath{stroke,fill}%
\end{pgfscope}%
\begin{pgfscope}%
\pgfpathrectangle{\pgfqpoint{0.100000in}{0.212622in}}{\pgfqpoint{3.696000in}{3.696000in}}%
\pgfusepath{clip}%
\pgfsetbuttcap%
\pgfsetroundjoin%
\definecolor{currentfill}{rgb}{0.121569,0.466667,0.705882}%
\pgfsetfillcolor{currentfill}%
\pgfsetfillopacity{0.301977}%
\pgfsetlinewidth{1.003750pt}%
\definecolor{currentstroke}{rgb}{0.121569,0.466667,0.705882}%
\pgfsetstrokecolor{currentstroke}%
\pgfsetstrokeopacity{0.301977}%
\pgfsetdash{}{0pt}%
\pgfpathmoveto{\pgfqpoint{1.456940in}{1.937888in}}%
\pgfpathcurveto{\pgfqpoint{1.465177in}{1.937888in}}{\pgfqpoint{1.473077in}{1.941160in}}{\pgfqpoint{1.478901in}{1.946984in}}%
\pgfpathcurveto{\pgfqpoint{1.484725in}{1.952808in}}{\pgfqpoint{1.487997in}{1.960708in}}{\pgfqpoint{1.487997in}{1.968945in}}%
\pgfpathcurveto{\pgfqpoint{1.487997in}{1.977181in}}{\pgfqpoint{1.484725in}{1.985081in}}{\pgfqpoint{1.478901in}{1.990905in}}%
\pgfpathcurveto{\pgfqpoint{1.473077in}{1.996729in}}{\pgfqpoint{1.465177in}{2.000001in}}{\pgfqpoint{1.456940in}{2.000001in}}%
\pgfpathcurveto{\pgfqpoint{1.448704in}{2.000001in}}{\pgfqpoint{1.440804in}{1.996729in}}{\pgfqpoint{1.434980in}{1.990905in}}%
\pgfpathcurveto{\pgfqpoint{1.429156in}{1.985081in}}{\pgfqpoint{1.425884in}{1.977181in}}{\pgfqpoint{1.425884in}{1.968945in}}%
\pgfpathcurveto{\pgfqpoint{1.425884in}{1.960708in}}{\pgfqpoint{1.429156in}{1.952808in}}{\pgfqpoint{1.434980in}{1.946984in}}%
\pgfpathcurveto{\pgfqpoint{1.440804in}{1.941160in}}{\pgfqpoint{1.448704in}{1.937888in}}{\pgfqpoint{1.456940in}{1.937888in}}%
\pgfpathclose%
\pgfusepath{stroke,fill}%
\end{pgfscope}%
\begin{pgfscope}%
\pgfpathrectangle{\pgfqpoint{0.100000in}{0.212622in}}{\pgfqpoint{3.696000in}{3.696000in}}%
\pgfusepath{clip}%
\pgfsetbuttcap%
\pgfsetroundjoin%
\definecolor{currentfill}{rgb}{0.121569,0.466667,0.705882}%
\pgfsetfillcolor{currentfill}%
\pgfsetfillopacity{0.304134}%
\pgfsetlinewidth{1.003750pt}%
\definecolor{currentstroke}{rgb}{0.121569,0.466667,0.705882}%
\pgfsetstrokecolor{currentstroke}%
\pgfsetstrokeopacity{0.304134}%
\pgfsetdash{}{0pt}%
\pgfpathmoveto{\pgfqpoint{1.441622in}{1.928456in}}%
\pgfpathcurveto{\pgfqpoint{1.449858in}{1.928456in}}{\pgfqpoint{1.457758in}{1.931728in}}{\pgfqpoint{1.463582in}{1.937552in}}%
\pgfpathcurveto{\pgfqpoint{1.469406in}{1.943376in}}{\pgfqpoint{1.472678in}{1.951276in}}{\pgfqpoint{1.472678in}{1.959512in}}%
\pgfpathcurveto{\pgfqpoint{1.472678in}{1.967749in}}{\pgfqpoint{1.469406in}{1.975649in}}{\pgfqpoint{1.463582in}{1.981473in}}%
\pgfpathcurveto{\pgfqpoint{1.457758in}{1.987297in}}{\pgfqpoint{1.449858in}{1.990569in}}{\pgfqpoint{1.441622in}{1.990569in}}%
\pgfpathcurveto{\pgfqpoint{1.433385in}{1.990569in}}{\pgfqpoint{1.425485in}{1.987297in}}{\pgfqpoint{1.419661in}{1.981473in}}%
\pgfpathcurveto{\pgfqpoint{1.413837in}{1.975649in}}{\pgfqpoint{1.410565in}{1.967749in}}{\pgfqpoint{1.410565in}{1.959512in}}%
\pgfpathcurveto{\pgfqpoint{1.410565in}{1.951276in}}{\pgfqpoint{1.413837in}{1.943376in}}{\pgfqpoint{1.419661in}{1.937552in}}%
\pgfpathcurveto{\pgfqpoint{1.425485in}{1.931728in}}{\pgfqpoint{1.433385in}{1.928456in}}{\pgfqpoint{1.441622in}{1.928456in}}%
\pgfpathclose%
\pgfusepath{stroke,fill}%
\end{pgfscope}%
\begin{pgfscope}%
\pgfpathrectangle{\pgfqpoint{0.100000in}{0.212622in}}{\pgfqpoint{3.696000in}{3.696000in}}%
\pgfusepath{clip}%
\pgfsetbuttcap%
\pgfsetroundjoin%
\definecolor{currentfill}{rgb}{0.121569,0.466667,0.705882}%
\pgfsetfillcolor{currentfill}%
\pgfsetfillopacity{0.306264}%
\pgfsetlinewidth{1.003750pt}%
\definecolor{currentstroke}{rgb}{0.121569,0.466667,0.705882}%
\pgfsetstrokecolor{currentstroke}%
\pgfsetstrokeopacity{0.306264}%
\pgfsetdash{}{0pt}%
\pgfpathmoveto{\pgfqpoint{1.447495in}{1.929618in}}%
\pgfpathcurveto{\pgfqpoint{1.455731in}{1.929618in}}{\pgfqpoint{1.463631in}{1.932891in}}{\pgfqpoint{1.469455in}{1.938715in}}%
\pgfpathcurveto{\pgfqpoint{1.475279in}{1.944539in}}{\pgfqpoint{1.478552in}{1.952439in}}{\pgfqpoint{1.478552in}{1.960675in}}%
\pgfpathcurveto{\pgfqpoint{1.478552in}{1.968911in}}{\pgfqpoint{1.475279in}{1.976811in}}{\pgfqpoint{1.469455in}{1.982635in}}%
\pgfpathcurveto{\pgfqpoint{1.463631in}{1.988459in}}{\pgfqpoint{1.455731in}{1.991731in}}{\pgfqpoint{1.447495in}{1.991731in}}%
\pgfpathcurveto{\pgfqpoint{1.439259in}{1.991731in}}{\pgfqpoint{1.431359in}{1.988459in}}{\pgfqpoint{1.425535in}{1.982635in}}%
\pgfpathcurveto{\pgfqpoint{1.419711in}{1.976811in}}{\pgfqpoint{1.416439in}{1.968911in}}{\pgfqpoint{1.416439in}{1.960675in}}%
\pgfpathcurveto{\pgfqpoint{1.416439in}{1.952439in}}{\pgfqpoint{1.419711in}{1.944539in}}{\pgfqpoint{1.425535in}{1.938715in}}%
\pgfpathcurveto{\pgfqpoint{1.431359in}{1.932891in}}{\pgfqpoint{1.439259in}{1.929618in}}{\pgfqpoint{1.447495in}{1.929618in}}%
\pgfpathclose%
\pgfusepath{stroke,fill}%
\end{pgfscope}%
\begin{pgfscope}%
\pgfpathrectangle{\pgfqpoint{0.100000in}{0.212622in}}{\pgfqpoint{3.696000in}{3.696000in}}%
\pgfusepath{clip}%
\pgfsetbuttcap%
\pgfsetroundjoin%
\definecolor{currentfill}{rgb}{0.121569,0.466667,0.705882}%
\pgfsetfillcolor{currentfill}%
\pgfsetfillopacity{0.307208}%
\pgfsetlinewidth{1.003750pt}%
\definecolor{currentstroke}{rgb}{0.121569,0.466667,0.705882}%
\pgfsetstrokecolor{currentstroke}%
\pgfsetstrokeopacity{0.307208}%
\pgfsetdash{}{0pt}%
\pgfpathmoveto{\pgfqpoint{1.472338in}{1.947377in}}%
\pgfpathcurveto{\pgfqpoint{1.480575in}{1.947377in}}{\pgfqpoint{1.488475in}{1.950649in}}{\pgfqpoint{1.494299in}{1.956473in}}%
\pgfpathcurveto{\pgfqpoint{1.500123in}{1.962297in}}{\pgfqpoint{1.503395in}{1.970197in}}{\pgfqpoint{1.503395in}{1.978434in}}%
\pgfpathcurveto{\pgfqpoint{1.503395in}{1.986670in}}{\pgfqpoint{1.500123in}{1.994570in}}{\pgfqpoint{1.494299in}{2.000394in}}%
\pgfpathcurveto{\pgfqpoint{1.488475in}{2.006218in}}{\pgfqpoint{1.480575in}{2.009490in}}{\pgfqpoint{1.472338in}{2.009490in}}%
\pgfpathcurveto{\pgfqpoint{1.464102in}{2.009490in}}{\pgfqpoint{1.456202in}{2.006218in}}{\pgfqpoint{1.450378in}{2.000394in}}%
\pgfpathcurveto{\pgfqpoint{1.444554in}{1.994570in}}{\pgfqpoint{1.441282in}{1.986670in}}{\pgfqpoint{1.441282in}{1.978434in}}%
\pgfpathcurveto{\pgfqpoint{1.441282in}{1.970197in}}{\pgfqpoint{1.444554in}{1.962297in}}{\pgfqpoint{1.450378in}{1.956473in}}%
\pgfpathcurveto{\pgfqpoint{1.456202in}{1.950649in}}{\pgfqpoint{1.464102in}{1.947377in}}{\pgfqpoint{1.472338in}{1.947377in}}%
\pgfpathclose%
\pgfusepath{stroke,fill}%
\end{pgfscope}%
\begin{pgfscope}%
\pgfpathrectangle{\pgfqpoint{0.100000in}{0.212622in}}{\pgfqpoint{3.696000in}{3.696000in}}%
\pgfusepath{clip}%
\pgfsetbuttcap%
\pgfsetroundjoin%
\definecolor{currentfill}{rgb}{0.121569,0.466667,0.705882}%
\pgfsetfillcolor{currentfill}%
\pgfsetfillopacity{0.307474}%
\pgfsetlinewidth{1.003750pt}%
\definecolor{currentstroke}{rgb}{0.121569,0.466667,0.705882}%
\pgfsetstrokecolor{currentstroke}%
\pgfsetstrokeopacity{0.307474}%
\pgfsetdash{}{0pt}%
\pgfpathmoveto{\pgfqpoint{1.427765in}{1.917302in}}%
\pgfpathcurveto{\pgfqpoint{1.436001in}{1.917302in}}{\pgfqpoint{1.443901in}{1.920575in}}{\pgfqpoint{1.449725in}{1.926399in}}%
\pgfpathcurveto{\pgfqpoint{1.455549in}{1.932222in}}{\pgfqpoint{1.458821in}{1.940123in}}{\pgfqpoint{1.458821in}{1.948359in}}%
\pgfpathcurveto{\pgfqpoint{1.458821in}{1.956595in}}{\pgfqpoint{1.455549in}{1.964495in}}{\pgfqpoint{1.449725in}{1.970319in}}%
\pgfpathcurveto{\pgfqpoint{1.443901in}{1.976143in}}{\pgfqpoint{1.436001in}{1.979415in}}{\pgfqpoint{1.427765in}{1.979415in}}%
\pgfpathcurveto{\pgfqpoint{1.419528in}{1.979415in}}{\pgfqpoint{1.411628in}{1.976143in}}{\pgfqpoint{1.405804in}{1.970319in}}%
\pgfpathcurveto{\pgfqpoint{1.399980in}{1.964495in}}{\pgfqpoint{1.396708in}{1.956595in}}{\pgfqpoint{1.396708in}{1.948359in}}%
\pgfpathcurveto{\pgfqpoint{1.396708in}{1.940123in}}{\pgfqpoint{1.399980in}{1.932222in}}{\pgfqpoint{1.405804in}{1.926399in}}%
\pgfpathcurveto{\pgfqpoint{1.411628in}{1.920575in}}{\pgfqpoint{1.419528in}{1.917302in}}{\pgfqpoint{1.427765in}{1.917302in}}%
\pgfpathclose%
\pgfusepath{stroke,fill}%
\end{pgfscope}%
\begin{pgfscope}%
\pgfpathrectangle{\pgfqpoint{0.100000in}{0.212622in}}{\pgfqpoint{3.696000in}{3.696000in}}%
\pgfusepath{clip}%
\pgfsetbuttcap%
\pgfsetroundjoin%
\definecolor{currentfill}{rgb}{0.121569,0.466667,0.705882}%
\pgfsetfillcolor{currentfill}%
\pgfsetfillopacity{0.307722}%
\pgfsetlinewidth{1.003750pt}%
\definecolor{currentstroke}{rgb}{0.121569,0.466667,0.705882}%
\pgfsetstrokecolor{currentstroke}%
\pgfsetstrokeopacity{0.307722}%
\pgfsetdash{}{0pt}%
\pgfpathmoveto{\pgfqpoint{1.467484in}{1.942955in}}%
\pgfpathcurveto{\pgfqpoint{1.475720in}{1.942955in}}{\pgfqpoint{1.483620in}{1.946227in}}{\pgfqpoint{1.489444in}{1.952051in}}%
\pgfpathcurveto{\pgfqpoint{1.495268in}{1.957875in}}{\pgfqpoint{1.498540in}{1.965775in}}{\pgfqpoint{1.498540in}{1.974011in}}%
\pgfpathcurveto{\pgfqpoint{1.498540in}{1.982248in}}{\pgfqpoint{1.495268in}{1.990148in}}{\pgfqpoint{1.489444in}{1.995971in}}%
\pgfpathcurveto{\pgfqpoint{1.483620in}{2.001795in}}{\pgfqpoint{1.475720in}{2.005068in}}{\pgfqpoint{1.467484in}{2.005068in}}%
\pgfpathcurveto{\pgfqpoint{1.459247in}{2.005068in}}{\pgfqpoint{1.451347in}{2.001795in}}{\pgfqpoint{1.445523in}{1.995971in}}%
\pgfpathcurveto{\pgfqpoint{1.439700in}{1.990148in}}{\pgfqpoint{1.436427in}{1.982248in}}{\pgfqpoint{1.436427in}{1.974011in}}%
\pgfpathcurveto{\pgfqpoint{1.436427in}{1.965775in}}{\pgfqpoint{1.439700in}{1.957875in}}{\pgfqpoint{1.445523in}{1.952051in}}%
\pgfpathcurveto{\pgfqpoint{1.451347in}{1.946227in}}{\pgfqpoint{1.459247in}{1.942955in}}{\pgfqpoint{1.467484in}{1.942955in}}%
\pgfpathclose%
\pgfusepath{stroke,fill}%
\end{pgfscope}%
\begin{pgfscope}%
\pgfpathrectangle{\pgfqpoint{0.100000in}{0.212622in}}{\pgfqpoint{3.696000in}{3.696000in}}%
\pgfusepath{clip}%
\pgfsetbuttcap%
\pgfsetroundjoin%
\definecolor{currentfill}{rgb}{0.121569,0.466667,0.705882}%
\pgfsetfillcolor{currentfill}%
\pgfsetfillopacity{0.307984}%
\pgfsetlinewidth{1.003750pt}%
\definecolor{currentstroke}{rgb}{0.121569,0.466667,0.705882}%
\pgfsetstrokecolor{currentstroke}%
\pgfsetstrokeopacity{0.307984}%
\pgfsetdash{}{0pt}%
\pgfpathmoveto{\pgfqpoint{1.443770in}{1.924966in}}%
\pgfpathcurveto{\pgfqpoint{1.452006in}{1.924966in}}{\pgfqpoint{1.459906in}{1.928238in}}{\pgfqpoint{1.465730in}{1.934062in}}%
\pgfpathcurveto{\pgfqpoint{1.471554in}{1.939886in}}{\pgfqpoint{1.474826in}{1.947786in}}{\pgfqpoint{1.474826in}{1.956022in}}%
\pgfpathcurveto{\pgfqpoint{1.474826in}{1.964259in}}{\pgfqpoint{1.471554in}{1.972159in}}{\pgfqpoint{1.465730in}{1.977983in}}%
\pgfpathcurveto{\pgfqpoint{1.459906in}{1.983806in}}{\pgfqpoint{1.452006in}{1.987079in}}{\pgfqpoint{1.443770in}{1.987079in}}%
\pgfpathcurveto{\pgfqpoint{1.435533in}{1.987079in}}{\pgfqpoint{1.427633in}{1.983806in}}{\pgfqpoint{1.421809in}{1.977983in}}%
\pgfpathcurveto{\pgfqpoint{1.415985in}{1.972159in}}{\pgfqpoint{1.412713in}{1.964259in}}{\pgfqpoint{1.412713in}{1.956022in}}%
\pgfpathcurveto{\pgfqpoint{1.412713in}{1.947786in}}{\pgfqpoint{1.415985in}{1.939886in}}{\pgfqpoint{1.421809in}{1.934062in}}%
\pgfpathcurveto{\pgfqpoint{1.427633in}{1.928238in}}{\pgfqpoint{1.435533in}{1.924966in}}{\pgfqpoint{1.443770in}{1.924966in}}%
\pgfpathclose%
\pgfusepath{stroke,fill}%
\end{pgfscope}%
\begin{pgfscope}%
\pgfpathrectangle{\pgfqpoint{0.100000in}{0.212622in}}{\pgfqpoint{3.696000in}{3.696000in}}%
\pgfusepath{clip}%
\pgfsetbuttcap%
\pgfsetroundjoin%
\definecolor{currentfill}{rgb}{0.121569,0.466667,0.705882}%
\pgfsetfillcolor{currentfill}%
\pgfsetfillopacity{0.308370}%
\pgfsetlinewidth{1.003750pt}%
\definecolor{currentstroke}{rgb}{0.121569,0.466667,0.705882}%
\pgfsetstrokecolor{currentstroke}%
\pgfsetstrokeopacity{0.308370}%
\pgfsetdash{}{0pt}%
\pgfpathmoveto{\pgfqpoint{1.476248in}{1.948235in}}%
\pgfpathcurveto{\pgfqpoint{1.484484in}{1.948235in}}{\pgfqpoint{1.492384in}{1.951507in}}{\pgfqpoint{1.498208in}{1.957331in}}%
\pgfpathcurveto{\pgfqpoint{1.504032in}{1.963155in}}{\pgfqpoint{1.507304in}{1.971055in}}{\pgfqpoint{1.507304in}{1.979292in}}%
\pgfpathcurveto{\pgfqpoint{1.507304in}{1.987528in}}{\pgfqpoint{1.504032in}{1.995428in}}{\pgfqpoint{1.498208in}{2.001252in}}%
\pgfpathcurveto{\pgfqpoint{1.492384in}{2.007076in}}{\pgfqpoint{1.484484in}{2.010348in}}{\pgfqpoint{1.476248in}{2.010348in}}%
\pgfpathcurveto{\pgfqpoint{1.468011in}{2.010348in}}{\pgfqpoint{1.460111in}{2.007076in}}{\pgfqpoint{1.454287in}{2.001252in}}%
\pgfpathcurveto{\pgfqpoint{1.448463in}{1.995428in}}{\pgfqpoint{1.445191in}{1.987528in}}{\pgfqpoint{1.445191in}{1.979292in}}%
\pgfpathcurveto{\pgfqpoint{1.445191in}{1.971055in}}{\pgfqpoint{1.448463in}{1.963155in}}{\pgfqpoint{1.454287in}{1.957331in}}%
\pgfpathcurveto{\pgfqpoint{1.460111in}{1.951507in}}{\pgfqpoint{1.468011in}{1.948235in}}{\pgfqpoint{1.476248in}{1.948235in}}%
\pgfpathclose%
\pgfusepath{stroke,fill}%
\end{pgfscope}%
\begin{pgfscope}%
\pgfpathrectangle{\pgfqpoint{0.100000in}{0.212622in}}{\pgfqpoint{3.696000in}{3.696000in}}%
\pgfusepath{clip}%
\pgfsetbuttcap%
\pgfsetroundjoin%
\definecolor{currentfill}{rgb}{0.121569,0.466667,0.705882}%
\pgfsetfillcolor{currentfill}%
\pgfsetfillopacity{0.308636}%
\pgfsetlinewidth{1.003750pt}%
\definecolor{currentstroke}{rgb}{0.121569,0.466667,0.705882}%
\pgfsetstrokecolor{currentstroke}%
\pgfsetstrokeopacity{0.308636}%
\pgfsetdash{}{0pt}%
\pgfpathmoveto{\pgfqpoint{1.469287in}{1.943852in}}%
\pgfpathcurveto{\pgfqpoint{1.477523in}{1.943852in}}{\pgfqpoint{1.485423in}{1.947125in}}{\pgfqpoint{1.491247in}{1.952949in}}%
\pgfpathcurveto{\pgfqpoint{1.497071in}{1.958773in}}{\pgfqpoint{1.500343in}{1.966673in}}{\pgfqpoint{1.500343in}{1.974909in}}%
\pgfpathcurveto{\pgfqpoint{1.500343in}{1.983145in}}{\pgfqpoint{1.497071in}{1.991045in}}{\pgfqpoint{1.491247in}{1.996869in}}%
\pgfpathcurveto{\pgfqpoint{1.485423in}{2.002693in}}{\pgfqpoint{1.477523in}{2.005965in}}{\pgfqpoint{1.469287in}{2.005965in}}%
\pgfpathcurveto{\pgfqpoint{1.461050in}{2.005965in}}{\pgfqpoint{1.453150in}{2.002693in}}{\pgfqpoint{1.447326in}{1.996869in}}%
\pgfpathcurveto{\pgfqpoint{1.441502in}{1.991045in}}{\pgfqpoint{1.438230in}{1.983145in}}{\pgfqpoint{1.438230in}{1.974909in}}%
\pgfpathcurveto{\pgfqpoint{1.438230in}{1.966673in}}{\pgfqpoint{1.441502in}{1.958773in}}{\pgfqpoint{1.447326in}{1.952949in}}%
\pgfpathcurveto{\pgfqpoint{1.453150in}{1.947125in}}{\pgfqpoint{1.461050in}{1.943852in}}{\pgfqpoint{1.469287in}{1.943852in}}%
\pgfpathclose%
\pgfusepath{stroke,fill}%
\end{pgfscope}%
\begin{pgfscope}%
\pgfpathrectangle{\pgfqpoint{0.100000in}{0.212622in}}{\pgfqpoint{3.696000in}{3.696000in}}%
\pgfusepath{clip}%
\pgfsetbuttcap%
\pgfsetroundjoin%
\definecolor{currentfill}{rgb}{0.121569,0.466667,0.705882}%
\pgfsetfillcolor{currentfill}%
\pgfsetfillopacity{0.309357}%
\pgfsetlinewidth{1.003750pt}%
\definecolor{currentstroke}{rgb}{0.121569,0.466667,0.705882}%
\pgfsetstrokecolor{currentstroke}%
\pgfsetstrokeopacity{0.309357}%
\pgfsetdash{}{0pt}%
\pgfpathmoveto{\pgfqpoint{1.469474in}{1.942939in}}%
\pgfpathcurveto{\pgfqpoint{1.477710in}{1.942939in}}{\pgfqpoint{1.485610in}{1.946211in}}{\pgfqpoint{1.491434in}{1.952035in}}%
\pgfpathcurveto{\pgfqpoint{1.497258in}{1.957859in}}{\pgfqpoint{1.500530in}{1.965759in}}{\pgfqpoint{1.500530in}{1.973995in}}%
\pgfpathcurveto{\pgfqpoint{1.500530in}{1.982232in}}{\pgfqpoint{1.497258in}{1.990132in}}{\pgfqpoint{1.491434in}{1.995956in}}%
\pgfpathcurveto{\pgfqpoint{1.485610in}{2.001780in}}{\pgfqpoint{1.477710in}{2.005052in}}{\pgfqpoint{1.469474in}{2.005052in}}%
\pgfpathcurveto{\pgfqpoint{1.461238in}{2.005052in}}{\pgfqpoint{1.453338in}{2.001780in}}{\pgfqpoint{1.447514in}{1.995956in}}%
\pgfpathcurveto{\pgfqpoint{1.441690in}{1.990132in}}{\pgfqpoint{1.438417in}{1.982232in}}{\pgfqpoint{1.438417in}{1.973995in}}%
\pgfpathcurveto{\pgfqpoint{1.438417in}{1.965759in}}{\pgfqpoint{1.441690in}{1.957859in}}{\pgfqpoint{1.447514in}{1.952035in}}%
\pgfpathcurveto{\pgfqpoint{1.453338in}{1.946211in}}{\pgfqpoint{1.461238in}{1.942939in}}{\pgfqpoint{1.469474in}{1.942939in}}%
\pgfpathclose%
\pgfusepath{stroke,fill}%
\end{pgfscope}%
\begin{pgfscope}%
\pgfpathrectangle{\pgfqpoint{0.100000in}{0.212622in}}{\pgfqpoint{3.696000in}{3.696000in}}%
\pgfusepath{clip}%
\pgfsetbuttcap%
\pgfsetroundjoin%
\definecolor{currentfill}{rgb}{0.121569,0.466667,0.705882}%
\pgfsetfillcolor{currentfill}%
\pgfsetfillopacity{0.309511}%
\pgfsetlinewidth{1.003750pt}%
\definecolor{currentstroke}{rgb}{0.121569,0.466667,0.705882}%
\pgfsetstrokecolor{currentstroke}%
\pgfsetstrokeopacity{0.309511}%
\pgfsetdash{}{0pt}%
\pgfpathmoveto{\pgfqpoint{1.455749in}{1.931753in}}%
\pgfpathcurveto{\pgfqpoint{1.463985in}{1.931753in}}{\pgfqpoint{1.471885in}{1.935025in}}{\pgfqpoint{1.477709in}{1.940849in}}%
\pgfpathcurveto{\pgfqpoint{1.483533in}{1.946673in}}{\pgfqpoint{1.486806in}{1.954573in}}{\pgfqpoint{1.486806in}{1.962810in}}%
\pgfpathcurveto{\pgfqpoint{1.486806in}{1.971046in}}{\pgfqpoint{1.483533in}{1.978946in}}{\pgfqpoint{1.477709in}{1.984770in}}%
\pgfpathcurveto{\pgfqpoint{1.471885in}{1.990594in}}{\pgfqpoint{1.463985in}{1.993866in}}{\pgfqpoint{1.455749in}{1.993866in}}%
\pgfpathcurveto{\pgfqpoint{1.447513in}{1.993866in}}{\pgfqpoint{1.439613in}{1.990594in}}{\pgfqpoint{1.433789in}{1.984770in}}%
\pgfpathcurveto{\pgfqpoint{1.427965in}{1.978946in}}{\pgfqpoint{1.424693in}{1.971046in}}{\pgfqpoint{1.424693in}{1.962810in}}%
\pgfpathcurveto{\pgfqpoint{1.424693in}{1.954573in}}{\pgfqpoint{1.427965in}{1.946673in}}{\pgfqpoint{1.433789in}{1.940849in}}%
\pgfpathcurveto{\pgfqpoint{1.439613in}{1.935025in}}{\pgfqpoint{1.447513in}{1.931753in}}{\pgfqpoint{1.455749in}{1.931753in}}%
\pgfpathclose%
\pgfusepath{stroke,fill}%
\end{pgfscope}%
\begin{pgfscope}%
\pgfpathrectangle{\pgfqpoint{0.100000in}{0.212622in}}{\pgfqpoint{3.696000in}{3.696000in}}%
\pgfusepath{clip}%
\pgfsetbuttcap%
\pgfsetroundjoin%
\definecolor{currentfill}{rgb}{0.121569,0.466667,0.705882}%
\pgfsetfillcolor{currentfill}%
\pgfsetfillopacity{0.309558}%
\pgfsetlinewidth{1.003750pt}%
\definecolor{currentstroke}{rgb}{0.121569,0.466667,0.705882}%
\pgfsetstrokecolor{currentstroke}%
\pgfsetstrokeopacity{0.309558}%
\pgfsetdash{}{0pt}%
\pgfpathmoveto{\pgfqpoint{1.418963in}{1.911427in}}%
\pgfpathcurveto{\pgfqpoint{1.427199in}{1.911427in}}{\pgfqpoint{1.435099in}{1.914700in}}{\pgfqpoint{1.440923in}{1.920524in}}%
\pgfpathcurveto{\pgfqpoint{1.446747in}{1.926348in}}{\pgfqpoint{1.450019in}{1.934248in}}{\pgfqpoint{1.450019in}{1.942484in}}%
\pgfpathcurveto{\pgfqpoint{1.450019in}{1.950720in}}{\pgfqpoint{1.446747in}{1.958620in}}{\pgfqpoint{1.440923in}{1.964444in}}%
\pgfpathcurveto{\pgfqpoint{1.435099in}{1.970268in}}{\pgfqpoint{1.427199in}{1.973540in}}{\pgfqpoint{1.418963in}{1.973540in}}%
\pgfpathcurveto{\pgfqpoint{1.410727in}{1.973540in}}{\pgfqpoint{1.402827in}{1.970268in}}{\pgfqpoint{1.397003in}{1.964444in}}%
\pgfpathcurveto{\pgfqpoint{1.391179in}{1.958620in}}{\pgfqpoint{1.387906in}{1.950720in}}{\pgfqpoint{1.387906in}{1.942484in}}%
\pgfpathcurveto{\pgfqpoint{1.387906in}{1.934248in}}{\pgfqpoint{1.391179in}{1.926348in}}{\pgfqpoint{1.397003in}{1.920524in}}%
\pgfpathcurveto{\pgfqpoint{1.402827in}{1.914700in}}{\pgfqpoint{1.410727in}{1.911427in}}{\pgfqpoint{1.418963in}{1.911427in}}%
\pgfpathclose%
\pgfusepath{stroke,fill}%
\end{pgfscope}%
\begin{pgfscope}%
\pgfpathrectangle{\pgfqpoint{0.100000in}{0.212622in}}{\pgfqpoint{3.696000in}{3.696000in}}%
\pgfusepath{clip}%
\pgfsetbuttcap%
\pgfsetroundjoin%
\definecolor{currentfill}{rgb}{0.121569,0.466667,0.705882}%
\pgfsetfillcolor{currentfill}%
\pgfsetfillopacity{0.310128}%
\pgfsetlinewidth{1.003750pt}%
\definecolor{currentstroke}{rgb}{0.121569,0.466667,0.705882}%
\pgfsetstrokecolor{currentstroke}%
\pgfsetstrokeopacity{0.310128}%
\pgfsetdash{}{0pt}%
\pgfpathmoveto{\pgfqpoint{1.439881in}{1.920862in}}%
\pgfpathcurveto{\pgfqpoint{1.448118in}{1.920862in}}{\pgfqpoint{1.456018in}{1.924134in}}{\pgfqpoint{1.461842in}{1.929958in}}%
\pgfpathcurveto{\pgfqpoint{1.467665in}{1.935782in}}{\pgfqpoint{1.470938in}{1.943682in}}{\pgfqpoint{1.470938in}{1.951918in}}%
\pgfpathcurveto{\pgfqpoint{1.470938in}{1.960155in}}{\pgfqpoint{1.467665in}{1.968055in}}{\pgfqpoint{1.461842in}{1.973879in}}%
\pgfpathcurveto{\pgfqpoint{1.456018in}{1.979703in}}{\pgfqpoint{1.448118in}{1.982975in}}{\pgfqpoint{1.439881in}{1.982975in}}%
\pgfpathcurveto{\pgfqpoint{1.431645in}{1.982975in}}{\pgfqpoint{1.423745in}{1.979703in}}{\pgfqpoint{1.417921in}{1.973879in}}%
\pgfpathcurveto{\pgfqpoint{1.412097in}{1.968055in}}{\pgfqpoint{1.408825in}{1.960155in}}{\pgfqpoint{1.408825in}{1.951918in}}%
\pgfpathcurveto{\pgfqpoint{1.408825in}{1.943682in}}{\pgfqpoint{1.412097in}{1.935782in}}{\pgfqpoint{1.417921in}{1.929958in}}%
\pgfpathcurveto{\pgfqpoint{1.423745in}{1.924134in}}{\pgfqpoint{1.431645in}{1.920862in}}{\pgfqpoint{1.439881in}{1.920862in}}%
\pgfpathclose%
\pgfusepath{stroke,fill}%
\end{pgfscope}%
\begin{pgfscope}%
\pgfpathrectangle{\pgfqpoint{0.100000in}{0.212622in}}{\pgfqpoint{3.696000in}{3.696000in}}%
\pgfusepath{clip}%
\pgfsetbuttcap%
\pgfsetroundjoin%
\definecolor{currentfill}{rgb}{0.121569,0.466667,0.705882}%
\pgfsetfillcolor{currentfill}%
\pgfsetfillopacity{0.310549}%
\pgfsetlinewidth{1.003750pt}%
\definecolor{currentstroke}{rgb}{0.121569,0.466667,0.705882}%
\pgfsetstrokecolor{currentstroke}%
\pgfsetstrokeopacity{0.310549}%
\pgfsetdash{}{0pt}%
\pgfpathmoveto{\pgfqpoint{1.448547in}{1.927019in}}%
\pgfpathcurveto{\pgfqpoint{1.456784in}{1.927019in}}{\pgfqpoint{1.464684in}{1.930291in}}{\pgfqpoint{1.470507in}{1.936115in}}%
\pgfpathcurveto{\pgfqpoint{1.476331in}{1.941939in}}{\pgfqpoint{1.479604in}{1.949839in}}{\pgfqpoint{1.479604in}{1.958075in}}%
\pgfpathcurveto{\pgfqpoint{1.479604in}{1.966312in}}{\pgfqpoint{1.476331in}{1.974212in}}{\pgfqpoint{1.470507in}{1.980036in}}%
\pgfpathcurveto{\pgfqpoint{1.464684in}{1.985859in}}{\pgfqpoint{1.456784in}{1.989132in}}{\pgfqpoint{1.448547in}{1.989132in}}%
\pgfpathcurveto{\pgfqpoint{1.440311in}{1.989132in}}{\pgfqpoint{1.432411in}{1.985859in}}{\pgfqpoint{1.426587in}{1.980036in}}%
\pgfpathcurveto{\pgfqpoint{1.420763in}{1.974212in}}{\pgfqpoint{1.417491in}{1.966312in}}{\pgfqpoint{1.417491in}{1.958075in}}%
\pgfpathcurveto{\pgfqpoint{1.417491in}{1.949839in}}{\pgfqpoint{1.420763in}{1.941939in}}{\pgfqpoint{1.426587in}{1.936115in}}%
\pgfpathcurveto{\pgfqpoint{1.432411in}{1.930291in}}{\pgfqpoint{1.440311in}{1.927019in}}{\pgfqpoint{1.448547in}{1.927019in}}%
\pgfpathclose%
\pgfusepath{stroke,fill}%
\end{pgfscope}%
\begin{pgfscope}%
\pgfpathrectangle{\pgfqpoint{0.100000in}{0.212622in}}{\pgfqpoint{3.696000in}{3.696000in}}%
\pgfusepath{clip}%
\pgfsetbuttcap%
\pgfsetroundjoin%
\definecolor{currentfill}{rgb}{0.121569,0.466667,0.705882}%
\pgfsetfillcolor{currentfill}%
\pgfsetfillopacity{0.310920}%
\pgfsetlinewidth{1.003750pt}%
\definecolor{currentstroke}{rgb}{0.121569,0.466667,0.705882}%
\pgfsetstrokecolor{currentstroke}%
\pgfsetstrokeopacity{0.310920}%
\pgfsetdash{}{0pt}%
\pgfpathmoveto{\pgfqpoint{1.472498in}{1.944354in}}%
\pgfpathcurveto{\pgfqpoint{1.480735in}{1.944354in}}{\pgfqpoint{1.488635in}{1.947627in}}{\pgfqpoint{1.494459in}{1.953450in}}%
\pgfpathcurveto{\pgfqpoint{1.500283in}{1.959274in}}{\pgfqpoint{1.503555in}{1.967174in}}{\pgfqpoint{1.503555in}{1.975411in}}%
\pgfpathcurveto{\pgfqpoint{1.503555in}{1.983647in}}{\pgfqpoint{1.500283in}{1.991547in}}{\pgfqpoint{1.494459in}{1.997371in}}%
\pgfpathcurveto{\pgfqpoint{1.488635in}{2.003195in}}{\pgfqpoint{1.480735in}{2.006467in}}{\pgfqpoint{1.472498in}{2.006467in}}%
\pgfpathcurveto{\pgfqpoint{1.464262in}{2.006467in}}{\pgfqpoint{1.456362in}{2.003195in}}{\pgfqpoint{1.450538in}{1.997371in}}%
\pgfpathcurveto{\pgfqpoint{1.444714in}{1.991547in}}{\pgfqpoint{1.441442in}{1.983647in}}{\pgfqpoint{1.441442in}{1.975411in}}%
\pgfpathcurveto{\pgfqpoint{1.441442in}{1.967174in}}{\pgfqpoint{1.444714in}{1.959274in}}{\pgfqpoint{1.450538in}{1.953450in}}%
\pgfpathcurveto{\pgfqpoint{1.456362in}{1.947627in}}{\pgfqpoint{1.464262in}{1.944354in}}{\pgfqpoint{1.472498in}{1.944354in}}%
\pgfpathclose%
\pgfusepath{stroke,fill}%
\end{pgfscope}%
\begin{pgfscope}%
\pgfpathrectangle{\pgfqpoint{0.100000in}{0.212622in}}{\pgfqpoint{3.696000in}{3.696000in}}%
\pgfusepath{clip}%
\pgfsetbuttcap%
\pgfsetroundjoin%
\definecolor{currentfill}{rgb}{0.121569,0.466667,0.705882}%
\pgfsetfillcolor{currentfill}%
\pgfsetfillopacity{0.311047}%
\pgfsetlinewidth{1.003750pt}%
\definecolor{currentstroke}{rgb}{0.121569,0.466667,0.705882}%
\pgfsetstrokecolor{currentstroke}%
\pgfsetstrokeopacity{0.311047}%
\pgfsetdash{}{0pt}%
\pgfpathmoveto{\pgfqpoint{1.441312in}{1.921333in}}%
\pgfpathcurveto{\pgfqpoint{1.449548in}{1.921333in}}{\pgfqpoint{1.457448in}{1.924606in}}{\pgfqpoint{1.463272in}{1.930430in}}%
\pgfpathcurveto{\pgfqpoint{1.469096in}{1.936254in}}{\pgfqpoint{1.472368in}{1.944154in}}{\pgfqpoint{1.472368in}{1.952390in}}%
\pgfpathcurveto{\pgfqpoint{1.472368in}{1.960626in}}{\pgfqpoint{1.469096in}{1.968526in}}{\pgfqpoint{1.463272in}{1.974350in}}%
\pgfpathcurveto{\pgfqpoint{1.457448in}{1.980174in}}{\pgfqpoint{1.449548in}{1.983446in}}{\pgfqpoint{1.441312in}{1.983446in}}%
\pgfpathcurveto{\pgfqpoint{1.433075in}{1.983446in}}{\pgfqpoint{1.425175in}{1.980174in}}{\pgfqpoint{1.419351in}{1.974350in}}%
\pgfpathcurveto{\pgfqpoint{1.413527in}{1.968526in}}{\pgfqpoint{1.410255in}{1.960626in}}{\pgfqpoint{1.410255in}{1.952390in}}%
\pgfpathcurveto{\pgfqpoint{1.410255in}{1.944154in}}{\pgfqpoint{1.413527in}{1.936254in}}{\pgfqpoint{1.419351in}{1.930430in}}%
\pgfpathcurveto{\pgfqpoint{1.425175in}{1.924606in}}{\pgfqpoint{1.433075in}{1.921333in}}{\pgfqpoint{1.441312in}{1.921333in}}%
\pgfpathclose%
\pgfusepath{stroke,fill}%
\end{pgfscope}%
\begin{pgfscope}%
\pgfpathrectangle{\pgfqpoint{0.100000in}{0.212622in}}{\pgfqpoint{3.696000in}{3.696000in}}%
\pgfusepath{clip}%
\pgfsetbuttcap%
\pgfsetroundjoin%
\definecolor{currentfill}{rgb}{0.121569,0.466667,0.705882}%
\pgfsetfillcolor{currentfill}%
\pgfsetfillopacity{0.311054}%
\pgfsetlinewidth{1.003750pt}%
\definecolor{currentstroke}{rgb}{0.121569,0.466667,0.705882}%
\pgfsetstrokecolor{currentstroke}%
\pgfsetstrokeopacity{0.311054}%
\pgfsetdash{}{0pt}%
\pgfpathmoveto{\pgfqpoint{1.443448in}{1.922227in}}%
\pgfpathcurveto{\pgfqpoint{1.451684in}{1.922227in}}{\pgfqpoint{1.459584in}{1.925500in}}{\pgfqpoint{1.465408in}{1.931324in}}%
\pgfpathcurveto{\pgfqpoint{1.471232in}{1.937148in}}{\pgfqpoint{1.474504in}{1.945048in}}{\pgfqpoint{1.474504in}{1.953284in}}%
\pgfpathcurveto{\pgfqpoint{1.474504in}{1.961520in}}{\pgfqpoint{1.471232in}{1.969420in}}{\pgfqpoint{1.465408in}{1.975244in}}%
\pgfpathcurveto{\pgfqpoint{1.459584in}{1.981068in}}{\pgfqpoint{1.451684in}{1.984340in}}{\pgfqpoint{1.443448in}{1.984340in}}%
\pgfpathcurveto{\pgfqpoint{1.435211in}{1.984340in}}{\pgfqpoint{1.427311in}{1.981068in}}{\pgfqpoint{1.421487in}{1.975244in}}%
\pgfpathcurveto{\pgfqpoint{1.415664in}{1.969420in}}{\pgfqpoint{1.412391in}{1.961520in}}{\pgfqpoint{1.412391in}{1.953284in}}%
\pgfpathcurveto{\pgfqpoint{1.412391in}{1.945048in}}{\pgfqpoint{1.415664in}{1.937148in}}{\pgfqpoint{1.421487in}{1.931324in}}%
\pgfpathcurveto{\pgfqpoint{1.427311in}{1.925500in}}{\pgfqpoint{1.435211in}{1.922227in}}{\pgfqpoint{1.443448in}{1.922227in}}%
\pgfpathclose%
\pgfusepath{stroke,fill}%
\end{pgfscope}%
\begin{pgfscope}%
\pgfpathrectangle{\pgfqpoint{0.100000in}{0.212622in}}{\pgfqpoint{3.696000in}{3.696000in}}%
\pgfusepath{clip}%
\pgfsetbuttcap%
\pgfsetroundjoin%
\definecolor{currentfill}{rgb}{0.121569,0.466667,0.705882}%
\pgfsetfillcolor{currentfill}%
\pgfsetfillopacity{0.311196}%
\pgfsetlinewidth{1.003750pt}%
\definecolor{currentstroke}{rgb}{0.121569,0.466667,0.705882}%
\pgfsetstrokecolor{currentstroke}%
\pgfsetstrokeopacity{0.311196}%
\pgfsetdash{}{0pt}%
\pgfpathmoveto{\pgfqpoint{1.476085in}{1.945606in}}%
\pgfpathcurveto{\pgfqpoint{1.484322in}{1.945606in}}{\pgfqpoint{1.492222in}{1.948879in}}{\pgfqpoint{1.498046in}{1.954703in}}%
\pgfpathcurveto{\pgfqpoint{1.503870in}{1.960526in}}{\pgfqpoint{1.507142in}{1.968427in}}{\pgfqpoint{1.507142in}{1.976663in}}%
\pgfpathcurveto{\pgfqpoint{1.507142in}{1.984899in}}{\pgfqpoint{1.503870in}{1.992799in}}{\pgfqpoint{1.498046in}{1.998623in}}%
\pgfpathcurveto{\pgfqpoint{1.492222in}{2.004447in}}{\pgfqpoint{1.484322in}{2.007719in}}{\pgfqpoint{1.476085in}{2.007719in}}%
\pgfpathcurveto{\pgfqpoint{1.467849in}{2.007719in}}{\pgfqpoint{1.459949in}{2.004447in}}{\pgfqpoint{1.454125in}{1.998623in}}%
\pgfpathcurveto{\pgfqpoint{1.448301in}{1.992799in}}{\pgfqpoint{1.445029in}{1.984899in}}{\pgfqpoint{1.445029in}{1.976663in}}%
\pgfpathcurveto{\pgfqpoint{1.445029in}{1.968427in}}{\pgfqpoint{1.448301in}{1.960526in}}{\pgfqpoint{1.454125in}{1.954703in}}%
\pgfpathcurveto{\pgfqpoint{1.459949in}{1.948879in}}{\pgfqpoint{1.467849in}{1.945606in}}{\pgfqpoint{1.476085in}{1.945606in}}%
\pgfpathclose%
\pgfusepath{stroke,fill}%
\end{pgfscope}%
\begin{pgfscope}%
\pgfpathrectangle{\pgfqpoint{0.100000in}{0.212622in}}{\pgfqpoint{3.696000in}{3.696000in}}%
\pgfusepath{clip}%
\pgfsetbuttcap%
\pgfsetroundjoin%
\definecolor{currentfill}{rgb}{0.121569,0.466667,0.705882}%
\pgfsetfillcolor{currentfill}%
\pgfsetfillopacity{0.318320}%
\pgfsetlinewidth{1.003750pt}%
\definecolor{currentstroke}{rgb}{0.121569,0.466667,0.705882}%
\pgfsetstrokecolor{currentstroke}%
\pgfsetstrokeopacity{0.318320}%
\pgfsetdash{}{0pt}%
\pgfpathmoveto{\pgfqpoint{1.386012in}{1.882192in}}%
\pgfpathcurveto{\pgfqpoint{1.394249in}{1.882192in}}{\pgfqpoint{1.402149in}{1.885464in}}{\pgfqpoint{1.407973in}{1.891288in}}%
\pgfpathcurveto{\pgfqpoint{1.413797in}{1.897112in}}{\pgfqpoint{1.417069in}{1.905012in}}{\pgfqpoint{1.417069in}{1.913249in}}%
\pgfpathcurveto{\pgfqpoint{1.417069in}{1.921485in}}{\pgfqpoint{1.413797in}{1.929385in}}{\pgfqpoint{1.407973in}{1.935209in}}%
\pgfpathcurveto{\pgfqpoint{1.402149in}{1.941033in}}{\pgfqpoint{1.394249in}{1.944305in}}{\pgfqpoint{1.386012in}{1.944305in}}%
\pgfpathcurveto{\pgfqpoint{1.377776in}{1.944305in}}{\pgfqpoint{1.369876in}{1.941033in}}{\pgfqpoint{1.364052in}{1.935209in}}%
\pgfpathcurveto{\pgfqpoint{1.358228in}{1.929385in}}{\pgfqpoint{1.354956in}{1.921485in}}{\pgfqpoint{1.354956in}{1.913249in}}%
\pgfpathcurveto{\pgfqpoint{1.354956in}{1.905012in}}{\pgfqpoint{1.358228in}{1.897112in}}{\pgfqpoint{1.364052in}{1.891288in}}%
\pgfpathcurveto{\pgfqpoint{1.369876in}{1.885464in}}{\pgfqpoint{1.377776in}{1.882192in}}{\pgfqpoint{1.386012in}{1.882192in}}%
\pgfpathclose%
\pgfusepath{stroke,fill}%
\end{pgfscope}%
\begin{pgfscope}%
\pgfpathrectangle{\pgfqpoint{0.100000in}{0.212622in}}{\pgfqpoint{3.696000in}{3.696000in}}%
\pgfusepath{clip}%
\pgfsetbuttcap%
\pgfsetroundjoin%
\definecolor{currentfill}{rgb}{0.121569,0.466667,0.705882}%
\pgfsetfillcolor{currentfill}%
\pgfsetfillopacity{0.318737}%
\pgfsetlinewidth{1.003750pt}%
\definecolor{currentstroke}{rgb}{0.121569,0.466667,0.705882}%
\pgfsetstrokecolor{currentstroke}%
\pgfsetstrokeopacity{0.318737}%
\pgfsetdash{}{0pt}%
\pgfpathmoveto{\pgfqpoint{1.470235in}{1.934910in}}%
\pgfpathcurveto{\pgfqpoint{1.478471in}{1.934910in}}{\pgfqpoint{1.486371in}{1.938183in}}{\pgfqpoint{1.492195in}{1.944007in}}%
\pgfpathcurveto{\pgfqpoint{1.498019in}{1.949831in}}{\pgfqpoint{1.501291in}{1.957731in}}{\pgfqpoint{1.501291in}{1.965967in}}%
\pgfpathcurveto{\pgfqpoint{1.501291in}{1.974203in}}{\pgfqpoint{1.498019in}{1.982103in}}{\pgfqpoint{1.492195in}{1.987927in}}%
\pgfpathcurveto{\pgfqpoint{1.486371in}{1.993751in}}{\pgfqpoint{1.478471in}{1.997023in}}{\pgfqpoint{1.470235in}{1.997023in}}%
\pgfpathcurveto{\pgfqpoint{1.461998in}{1.997023in}}{\pgfqpoint{1.454098in}{1.993751in}}{\pgfqpoint{1.448274in}{1.987927in}}%
\pgfpathcurveto{\pgfqpoint{1.442451in}{1.982103in}}{\pgfqpoint{1.439178in}{1.974203in}}{\pgfqpoint{1.439178in}{1.965967in}}%
\pgfpathcurveto{\pgfqpoint{1.439178in}{1.957731in}}{\pgfqpoint{1.442451in}{1.949831in}}{\pgfqpoint{1.448274in}{1.944007in}}%
\pgfpathcurveto{\pgfqpoint{1.454098in}{1.938183in}}{\pgfqpoint{1.461998in}{1.934910in}}{\pgfqpoint{1.470235in}{1.934910in}}%
\pgfpathclose%
\pgfusepath{stroke,fill}%
\end{pgfscope}%
\begin{pgfscope}%
\pgfpathrectangle{\pgfqpoint{0.100000in}{0.212622in}}{\pgfqpoint{3.696000in}{3.696000in}}%
\pgfusepath{clip}%
\pgfsetbuttcap%
\pgfsetroundjoin%
\definecolor{currentfill}{rgb}{0.121569,0.466667,0.705882}%
\pgfsetfillcolor{currentfill}%
\pgfsetfillopacity{0.319040}%
\pgfsetlinewidth{1.003750pt}%
\definecolor{currentstroke}{rgb}{0.121569,0.466667,0.705882}%
\pgfsetstrokecolor{currentstroke}%
\pgfsetstrokeopacity{0.319040}%
\pgfsetdash{}{0pt}%
\pgfpathmoveto{\pgfqpoint{1.467304in}{1.932850in}}%
\pgfpathcurveto{\pgfqpoint{1.475540in}{1.932850in}}{\pgfqpoint{1.483440in}{1.936122in}}{\pgfqpoint{1.489264in}{1.941946in}}%
\pgfpathcurveto{\pgfqpoint{1.495088in}{1.947770in}}{\pgfqpoint{1.498361in}{1.955670in}}{\pgfqpoint{1.498361in}{1.963906in}}%
\pgfpathcurveto{\pgfqpoint{1.498361in}{1.972143in}}{\pgfqpoint{1.495088in}{1.980043in}}{\pgfqpoint{1.489264in}{1.985867in}}%
\pgfpathcurveto{\pgfqpoint{1.483440in}{1.991691in}}{\pgfqpoint{1.475540in}{1.994963in}}{\pgfqpoint{1.467304in}{1.994963in}}%
\pgfpathcurveto{\pgfqpoint{1.459068in}{1.994963in}}{\pgfqpoint{1.451168in}{1.991691in}}{\pgfqpoint{1.445344in}{1.985867in}}%
\pgfpathcurveto{\pgfqpoint{1.439520in}{1.980043in}}{\pgfqpoint{1.436248in}{1.972143in}}{\pgfqpoint{1.436248in}{1.963906in}}%
\pgfpathcurveto{\pgfqpoint{1.436248in}{1.955670in}}{\pgfqpoint{1.439520in}{1.947770in}}{\pgfqpoint{1.445344in}{1.941946in}}%
\pgfpathcurveto{\pgfqpoint{1.451168in}{1.936122in}}{\pgfqpoint{1.459068in}{1.932850in}}{\pgfqpoint{1.467304in}{1.932850in}}%
\pgfpathclose%
\pgfusepath{stroke,fill}%
\end{pgfscope}%
\begin{pgfscope}%
\pgfpathrectangle{\pgfqpoint{0.100000in}{0.212622in}}{\pgfqpoint{3.696000in}{3.696000in}}%
\pgfusepath{clip}%
\pgfsetbuttcap%
\pgfsetroundjoin%
\definecolor{currentfill}{rgb}{0.121569,0.466667,0.705882}%
\pgfsetfillcolor{currentfill}%
\pgfsetfillopacity{0.319665}%
\pgfsetlinewidth{1.003750pt}%
\definecolor{currentstroke}{rgb}{0.121569,0.466667,0.705882}%
\pgfsetstrokecolor{currentstroke}%
\pgfsetstrokeopacity{0.319665}%
\pgfsetdash{}{0pt}%
\pgfpathmoveto{\pgfqpoint{1.379158in}{1.881311in}}%
\pgfpathcurveto{\pgfqpoint{1.387394in}{1.881311in}}{\pgfqpoint{1.395294in}{1.884584in}}{\pgfqpoint{1.401118in}{1.890407in}}%
\pgfpathcurveto{\pgfqpoint{1.406942in}{1.896231in}}{\pgfqpoint{1.410214in}{1.904131in}}{\pgfqpoint{1.410214in}{1.912368in}}%
\pgfpathcurveto{\pgfqpoint{1.410214in}{1.920604in}}{\pgfqpoint{1.406942in}{1.928504in}}{\pgfqpoint{1.401118in}{1.934328in}}%
\pgfpathcurveto{\pgfqpoint{1.395294in}{1.940152in}}{\pgfqpoint{1.387394in}{1.943424in}}{\pgfqpoint{1.379158in}{1.943424in}}%
\pgfpathcurveto{\pgfqpoint{1.370921in}{1.943424in}}{\pgfqpoint{1.363021in}{1.940152in}}{\pgfqpoint{1.357197in}{1.934328in}}%
\pgfpathcurveto{\pgfqpoint{1.351373in}{1.928504in}}{\pgfqpoint{1.348101in}{1.920604in}}{\pgfqpoint{1.348101in}{1.912368in}}%
\pgfpathcurveto{\pgfqpoint{1.348101in}{1.904131in}}{\pgfqpoint{1.351373in}{1.896231in}}{\pgfqpoint{1.357197in}{1.890407in}}%
\pgfpathcurveto{\pgfqpoint{1.363021in}{1.884584in}}{\pgfqpoint{1.370921in}{1.881311in}}{\pgfqpoint{1.379158in}{1.881311in}}%
\pgfpathclose%
\pgfusepath{stroke,fill}%
\end{pgfscope}%
\begin{pgfscope}%
\pgfpathrectangle{\pgfqpoint{0.100000in}{0.212622in}}{\pgfqpoint{3.696000in}{3.696000in}}%
\pgfusepath{clip}%
\pgfsetbuttcap%
\pgfsetroundjoin%
\definecolor{currentfill}{rgb}{0.121569,0.466667,0.705882}%
\pgfsetfillcolor{currentfill}%
\pgfsetfillopacity{0.320373}%
\pgfsetlinewidth{1.003750pt}%
\definecolor{currentstroke}{rgb}{0.121569,0.466667,0.705882}%
\pgfsetstrokecolor{currentstroke}%
\pgfsetstrokeopacity{0.320373}%
\pgfsetdash{}{0pt}%
\pgfpathmoveto{\pgfqpoint{1.459262in}{1.926005in}}%
\pgfpathcurveto{\pgfqpoint{1.467498in}{1.926005in}}{\pgfqpoint{1.475398in}{1.929277in}}{\pgfqpoint{1.481222in}{1.935101in}}%
\pgfpathcurveto{\pgfqpoint{1.487046in}{1.940925in}}{\pgfqpoint{1.490318in}{1.948825in}}{\pgfqpoint{1.490318in}{1.957061in}}%
\pgfpathcurveto{\pgfqpoint{1.490318in}{1.965297in}}{\pgfqpoint{1.487046in}{1.973197in}}{\pgfqpoint{1.481222in}{1.979021in}}%
\pgfpathcurveto{\pgfqpoint{1.475398in}{1.984845in}}{\pgfqpoint{1.467498in}{1.988118in}}{\pgfqpoint{1.459262in}{1.988118in}}%
\pgfpathcurveto{\pgfqpoint{1.451025in}{1.988118in}}{\pgfqpoint{1.443125in}{1.984845in}}{\pgfqpoint{1.437301in}{1.979021in}}%
\pgfpathcurveto{\pgfqpoint{1.431477in}{1.973197in}}{\pgfqpoint{1.428205in}{1.965297in}}{\pgfqpoint{1.428205in}{1.957061in}}%
\pgfpathcurveto{\pgfqpoint{1.428205in}{1.948825in}}{\pgfqpoint{1.431477in}{1.940925in}}{\pgfqpoint{1.437301in}{1.935101in}}%
\pgfpathcurveto{\pgfqpoint{1.443125in}{1.929277in}}{\pgfqpoint{1.451025in}{1.926005in}}{\pgfqpoint{1.459262in}{1.926005in}}%
\pgfpathclose%
\pgfusepath{stroke,fill}%
\end{pgfscope}%
\begin{pgfscope}%
\pgfpathrectangle{\pgfqpoint{0.100000in}{0.212622in}}{\pgfqpoint{3.696000in}{3.696000in}}%
\pgfusepath{clip}%
\pgfsetbuttcap%
\pgfsetroundjoin%
\definecolor{currentfill}{rgb}{0.121569,0.466667,0.705882}%
\pgfsetfillcolor{currentfill}%
\pgfsetfillopacity{0.320411}%
\pgfsetlinewidth{1.003750pt}%
\definecolor{currentstroke}{rgb}{0.121569,0.466667,0.705882}%
\pgfsetstrokecolor{currentstroke}%
\pgfsetstrokeopacity{0.320411}%
\pgfsetdash{}{0pt}%
\pgfpathmoveto{\pgfqpoint{1.451822in}{1.919215in}}%
\pgfpathcurveto{\pgfqpoint{1.460058in}{1.919215in}}{\pgfqpoint{1.467959in}{1.922488in}}{\pgfqpoint{1.473782in}{1.928312in}}%
\pgfpathcurveto{\pgfqpoint{1.479606in}{1.934136in}}{\pgfqpoint{1.482879in}{1.942036in}}{\pgfqpoint{1.482879in}{1.950272in}}%
\pgfpathcurveto{\pgfqpoint{1.482879in}{1.958508in}}{\pgfqpoint{1.479606in}{1.966408in}}{\pgfqpoint{1.473782in}{1.972232in}}%
\pgfpathcurveto{\pgfqpoint{1.467959in}{1.978056in}}{\pgfqpoint{1.460058in}{1.981328in}}{\pgfqpoint{1.451822in}{1.981328in}}%
\pgfpathcurveto{\pgfqpoint{1.443586in}{1.981328in}}{\pgfqpoint{1.435686in}{1.978056in}}{\pgfqpoint{1.429862in}{1.972232in}}%
\pgfpathcurveto{\pgfqpoint{1.424038in}{1.966408in}}{\pgfqpoint{1.420766in}{1.958508in}}{\pgfqpoint{1.420766in}{1.950272in}}%
\pgfpathcurveto{\pgfqpoint{1.420766in}{1.942036in}}{\pgfqpoint{1.424038in}{1.934136in}}{\pgfqpoint{1.429862in}{1.928312in}}%
\pgfpathcurveto{\pgfqpoint{1.435686in}{1.922488in}}{\pgfqpoint{1.443586in}{1.919215in}}{\pgfqpoint{1.451822in}{1.919215in}}%
\pgfpathclose%
\pgfusepath{stroke,fill}%
\end{pgfscope}%
\begin{pgfscope}%
\pgfpathrectangle{\pgfqpoint{0.100000in}{0.212622in}}{\pgfqpoint{3.696000in}{3.696000in}}%
\pgfusepath{clip}%
\pgfsetbuttcap%
\pgfsetroundjoin%
\definecolor{currentfill}{rgb}{0.121569,0.466667,0.705882}%
\pgfsetfillcolor{currentfill}%
\pgfsetfillopacity{0.320876}%
\pgfsetlinewidth{1.003750pt}%
\definecolor{currentstroke}{rgb}{0.121569,0.466667,0.705882}%
\pgfsetstrokecolor{currentstroke}%
\pgfsetstrokeopacity{0.320876}%
\pgfsetdash{}{0pt}%
\pgfpathmoveto{\pgfqpoint{1.554203in}{2.000225in}}%
\pgfpathcurveto{\pgfqpoint{1.562439in}{2.000225in}}{\pgfqpoint{1.570339in}{2.003497in}}{\pgfqpoint{1.576163in}{2.009321in}}%
\pgfpathcurveto{\pgfqpoint{1.581987in}{2.015145in}}{\pgfqpoint{1.585259in}{2.023045in}}{\pgfqpoint{1.585259in}{2.031282in}}%
\pgfpathcurveto{\pgfqpoint{1.585259in}{2.039518in}}{\pgfqpoint{1.581987in}{2.047418in}}{\pgfqpoint{1.576163in}{2.053242in}}%
\pgfpathcurveto{\pgfqpoint{1.570339in}{2.059066in}}{\pgfqpoint{1.562439in}{2.062338in}}{\pgfqpoint{1.554203in}{2.062338in}}%
\pgfpathcurveto{\pgfqpoint{1.545966in}{2.062338in}}{\pgfqpoint{1.538066in}{2.059066in}}{\pgfqpoint{1.532243in}{2.053242in}}%
\pgfpathcurveto{\pgfqpoint{1.526419in}{2.047418in}}{\pgfqpoint{1.523146in}{2.039518in}}{\pgfqpoint{1.523146in}{2.031282in}}%
\pgfpathcurveto{\pgfqpoint{1.523146in}{2.023045in}}{\pgfqpoint{1.526419in}{2.015145in}}{\pgfqpoint{1.532243in}{2.009321in}}%
\pgfpathcurveto{\pgfqpoint{1.538066in}{2.003497in}}{\pgfqpoint{1.545966in}{2.000225in}}{\pgfqpoint{1.554203in}{2.000225in}}%
\pgfpathclose%
\pgfusepath{stroke,fill}%
\end{pgfscope}%
\begin{pgfscope}%
\pgfpathrectangle{\pgfqpoint{0.100000in}{0.212622in}}{\pgfqpoint{3.696000in}{3.696000in}}%
\pgfusepath{clip}%
\pgfsetbuttcap%
\pgfsetroundjoin%
\definecolor{currentfill}{rgb}{0.121569,0.466667,0.705882}%
\pgfsetfillcolor{currentfill}%
\pgfsetfillopacity{0.320916}%
\pgfsetlinewidth{1.003750pt}%
\definecolor{currentstroke}{rgb}{0.121569,0.466667,0.705882}%
\pgfsetstrokecolor{currentstroke}%
\pgfsetstrokeopacity{0.320916}%
\pgfsetdash{}{0pt}%
\pgfpathmoveto{\pgfqpoint{1.591235in}{2.029532in}}%
\pgfpathcurveto{\pgfqpoint{1.599471in}{2.029532in}}{\pgfqpoint{1.607371in}{2.032804in}}{\pgfqpoint{1.613195in}{2.038628in}}%
\pgfpathcurveto{\pgfqpoint{1.619019in}{2.044452in}}{\pgfqpoint{1.622292in}{2.052352in}}{\pgfqpoint{1.622292in}{2.060589in}}%
\pgfpathcurveto{\pgfqpoint{1.622292in}{2.068825in}}{\pgfqpoint{1.619019in}{2.076725in}}{\pgfqpoint{1.613195in}{2.082549in}}%
\pgfpathcurveto{\pgfqpoint{1.607371in}{2.088373in}}{\pgfqpoint{1.599471in}{2.091645in}}{\pgfqpoint{1.591235in}{2.091645in}}%
\pgfpathcurveto{\pgfqpoint{1.582999in}{2.091645in}}{\pgfqpoint{1.575099in}{2.088373in}}{\pgfqpoint{1.569275in}{2.082549in}}%
\pgfpathcurveto{\pgfqpoint{1.563451in}{2.076725in}}{\pgfqpoint{1.560179in}{2.068825in}}{\pgfqpoint{1.560179in}{2.060589in}}%
\pgfpathcurveto{\pgfqpoint{1.560179in}{2.052352in}}{\pgfqpoint{1.563451in}{2.044452in}}{\pgfqpoint{1.569275in}{2.038628in}}%
\pgfpathcurveto{\pgfqpoint{1.575099in}{2.032804in}}{\pgfqpoint{1.582999in}{2.029532in}}{\pgfqpoint{1.591235in}{2.029532in}}%
\pgfpathclose%
\pgfusepath{stroke,fill}%
\end{pgfscope}%
\begin{pgfscope}%
\pgfpathrectangle{\pgfqpoint{0.100000in}{0.212622in}}{\pgfqpoint{3.696000in}{3.696000in}}%
\pgfusepath{clip}%
\pgfsetbuttcap%
\pgfsetroundjoin%
\definecolor{currentfill}{rgb}{0.121569,0.466667,0.705882}%
\pgfsetfillcolor{currentfill}%
\pgfsetfillopacity{0.321087}%
\pgfsetlinewidth{1.003750pt}%
\definecolor{currentstroke}{rgb}{0.121569,0.466667,0.705882}%
\pgfsetstrokecolor{currentstroke}%
\pgfsetstrokeopacity{0.321087}%
\pgfsetdash{}{0pt}%
\pgfpathmoveto{\pgfqpoint{1.564759in}{2.009337in}}%
\pgfpathcurveto{\pgfqpoint{1.572995in}{2.009337in}}{\pgfqpoint{1.580895in}{2.012609in}}{\pgfqpoint{1.586719in}{2.018433in}}%
\pgfpathcurveto{\pgfqpoint{1.592543in}{2.024257in}}{\pgfqpoint{1.595815in}{2.032157in}}{\pgfqpoint{1.595815in}{2.040394in}}%
\pgfpathcurveto{\pgfqpoint{1.595815in}{2.048630in}}{\pgfqpoint{1.592543in}{2.056530in}}{\pgfqpoint{1.586719in}{2.062354in}}%
\pgfpathcurveto{\pgfqpoint{1.580895in}{2.068178in}}{\pgfqpoint{1.572995in}{2.071450in}}{\pgfqpoint{1.564759in}{2.071450in}}%
\pgfpathcurveto{\pgfqpoint{1.556522in}{2.071450in}}{\pgfqpoint{1.548622in}{2.068178in}}{\pgfqpoint{1.542798in}{2.062354in}}%
\pgfpathcurveto{\pgfqpoint{1.536975in}{2.056530in}}{\pgfqpoint{1.533702in}{2.048630in}}{\pgfqpoint{1.533702in}{2.040394in}}%
\pgfpathcurveto{\pgfqpoint{1.533702in}{2.032157in}}{\pgfqpoint{1.536975in}{2.024257in}}{\pgfqpoint{1.542798in}{2.018433in}}%
\pgfpathcurveto{\pgfqpoint{1.548622in}{2.012609in}}{\pgfqpoint{1.556522in}{2.009337in}}{\pgfqpoint{1.564759in}{2.009337in}}%
\pgfpathclose%
\pgfusepath{stroke,fill}%
\end{pgfscope}%
\begin{pgfscope}%
\pgfpathrectangle{\pgfqpoint{0.100000in}{0.212622in}}{\pgfqpoint{3.696000in}{3.696000in}}%
\pgfusepath{clip}%
\pgfsetbuttcap%
\pgfsetroundjoin%
\definecolor{currentfill}{rgb}{0.121569,0.466667,0.705882}%
\pgfsetfillcolor{currentfill}%
\pgfsetfillopacity{0.321300}%
\pgfsetlinewidth{1.003750pt}%
\definecolor{currentstroke}{rgb}{0.121569,0.466667,0.705882}%
\pgfsetstrokecolor{currentstroke}%
\pgfsetstrokeopacity{0.321300}%
\pgfsetdash{}{0pt}%
\pgfpathmoveto{\pgfqpoint{1.465491in}{1.930411in}}%
\pgfpathcurveto{\pgfqpoint{1.473727in}{1.930411in}}{\pgfqpoint{1.481628in}{1.933683in}}{\pgfqpoint{1.487451in}{1.939507in}}%
\pgfpathcurveto{\pgfqpoint{1.493275in}{1.945331in}}{\pgfqpoint{1.496548in}{1.953231in}}{\pgfqpoint{1.496548in}{1.961468in}}%
\pgfpathcurveto{\pgfqpoint{1.496548in}{1.969704in}}{\pgfqpoint{1.493275in}{1.977604in}}{\pgfqpoint{1.487451in}{1.983428in}}%
\pgfpathcurveto{\pgfqpoint{1.481628in}{1.989252in}}{\pgfqpoint{1.473727in}{1.992524in}}{\pgfqpoint{1.465491in}{1.992524in}}%
\pgfpathcurveto{\pgfqpoint{1.457255in}{1.992524in}}{\pgfqpoint{1.449355in}{1.989252in}}{\pgfqpoint{1.443531in}{1.983428in}}%
\pgfpathcurveto{\pgfqpoint{1.437707in}{1.977604in}}{\pgfqpoint{1.434435in}{1.969704in}}{\pgfqpoint{1.434435in}{1.961468in}}%
\pgfpathcurveto{\pgfqpoint{1.434435in}{1.953231in}}{\pgfqpoint{1.437707in}{1.945331in}}{\pgfqpoint{1.443531in}{1.939507in}}%
\pgfpathcurveto{\pgfqpoint{1.449355in}{1.933683in}}{\pgfqpoint{1.457255in}{1.930411in}}{\pgfqpoint{1.465491in}{1.930411in}}%
\pgfpathclose%
\pgfusepath{stroke,fill}%
\end{pgfscope}%
\begin{pgfscope}%
\pgfpathrectangle{\pgfqpoint{0.100000in}{0.212622in}}{\pgfqpoint{3.696000in}{3.696000in}}%
\pgfusepath{clip}%
\pgfsetbuttcap%
\pgfsetroundjoin%
\definecolor{currentfill}{rgb}{0.121569,0.466667,0.705882}%
\pgfsetfillcolor{currentfill}%
\pgfsetfillopacity{0.321593}%
\pgfsetlinewidth{1.003750pt}%
\definecolor{currentstroke}{rgb}{0.121569,0.466667,0.705882}%
\pgfsetstrokecolor{currentstroke}%
\pgfsetstrokeopacity{0.321593}%
\pgfsetdash{}{0pt}%
\pgfpathmoveto{\pgfqpoint{1.583760in}{2.024059in}}%
\pgfpathcurveto{\pgfqpoint{1.591996in}{2.024059in}}{\pgfqpoint{1.599896in}{2.027331in}}{\pgfqpoint{1.605720in}{2.033155in}}%
\pgfpathcurveto{\pgfqpoint{1.611544in}{2.038979in}}{\pgfqpoint{1.614817in}{2.046879in}}{\pgfqpoint{1.614817in}{2.055116in}}%
\pgfpathcurveto{\pgfqpoint{1.614817in}{2.063352in}}{\pgfqpoint{1.611544in}{2.071252in}}{\pgfqpoint{1.605720in}{2.077076in}}%
\pgfpathcurveto{\pgfqpoint{1.599896in}{2.082900in}}{\pgfqpoint{1.591996in}{2.086172in}}{\pgfqpoint{1.583760in}{2.086172in}}%
\pgfpathcurveto{\pgfqpoint{1.575524in}{2.086172in}}{\pgfqpoint{1.567624in}{2.082900in}}{\pgfqpoint{1.561800in}{2.077076in}}%
\pgfpathcurveto{\pgfqpoint{1.555976in}{2.071252in}}{\pgfqpoint{1.552704in}{2.063352in}}{\pgfqpoint{1.552704in}{2.055116in}}%
\pgfpathcurveto{\pgfqpoint{1.552704in}{2.046879in}}{\pgfqpoint{1.555976in}{2.038979in}}{\pgfqpoint{1.561800in}{2.033155in}}%
\pgfpathcurveto{\pgfqpoint{1.567624in}{2.027331in}}{\pgfqpoint{1.575524in}{2.024059in}}{\pgfqpoint{1.583760in}{2.024059in}}%
\pgfpathclose%
\pgfusepath{stroke,fill}%
\end{pgfscope}%
\begin{pgfscope}%
\pgfpathrectangle{\pgfqpoint{0.100000in}{0.212622in}}{\pgfqpoint{3.696000in}{3.696000in}}%
\pgfusepath{clip}%
\pgfsetbuttcap%
\pgfsetroundjoin%
\definecolor{currentfill}{rgb}{0.121569,0.466667,0.705882}%
\pgfsetfillcolor{currentfill}%
\pgfsetfillopacity{0.321604}%
\pgfsetlinewidth{1.003750pt}%
\definecolor{currentstroke}{rgb}{0.121569,0.466667,0.705882}%
\pgfsetstrokecolor{currentstroke}%
\pgfsetstrokeopacity{0.321604}%
\pgfsetdash{}{0pt}%
\pgfpathmoveto{\pgfqpoint{1.575548in}{2.017596in}}%
\pgfpathcurveto{\pgfqpoint{1.583785in}{2.017596in}}{\pgfqpoint{1.591685in}{2.020868in}}{\pgfqpoint{1.597509in}{2.026692in}}%
\pgfpathcurveto{\pgfqpoint{1.603333in}{2.032516in}}{\pgfqpoint{1.606605in}{2.040416in}}{\pgfqpoint{1.606605in}{2.048652in}}%
\pgfpathcurveto{\pgfqpoint{1.606605in}{2.056889in}}{\pgfqpoint{1.603333in}{2.064789in}}{\pgfqpoint{1.597509in}{2.070613in}}%
\pgfpathcurveto{\pgfqpoint{1.591685in}{2.076437in}}{\pgfqpoint{1.583785in}{2.079709in}}{\pgfqpoint{1.575548in}{2.079709in}}%
\pgfpathcurveto{\pgfqpoint{1.567312in}{2.079709in}}{\pgfqpoint{1.559412in}{2.076437in}}{\pgfqpoint{1.553588in}{2.070613in}}%
\pgfpathcurveto{\pgfqpoint{1.547764in}{2.064789in}}{\pgfqpoint{1.544492in}{2.056889in}}{\pgfqpoint{1.544492in}{2.048652in}}%
\pgfpathcurveto{\pgfqpoint{1.544492in}{2.040416in}}{\pgfqpoint{1.547764in}{2.032516in}}{\pgfqpoint{1.553588in}{2.026692in}}%
\pgfpathcurveto{\pgfqpoint{1.559412in}{2.020868in}}{\pgfqpoint{1.567312in}{2.017596in}}{\pgfqpoint{1.575548in}{2.017596in}}%
\pgfpathclose%
\pgfusepath{stroke,fill}%
\end{pgfscope}%
\begin{pgfscope}%
\pgfpathrectangle{\pgfqpoint{0.100000in}{0.212622in}}{\pgfqpoint{3.696000in}{3.696000in}}%
\pgfusepath{clip}%
\pgfsetbuttcap%
\pgfsetroundjoin%
\definecolor{currentfill}{rgb}{0.121569,0.466667,0.705882}%
\pgfsetfillcolor{currentfill}%
\pgfsetfillopacity{0.322066}%
\pgfsetlinewidth{1.003750pt}%
\definecolor{currentstroke}{rgb}{0.121569,0.466667,0.705882}%
\pgfsetstrokecolor{currentstroke}%
\pgfsetstrokeopacity{0.322066}%
\pgfsetdash{}{0pt}%
\pgfpathmoveto{\pgfqpoint{1.476696in}{1.938610in}}%
\pgfpathcurveto{\pgfqpoint{1.484932in}{1.938610in}}{\pgfqpoint{1.492832in}{1.941882in}}{\pgfqpoint{1.498656in}{1.947706in}}%
\pgfpathcurveto{\pgfqpoint{1.504480in}{1.953530in}}{\pgfqpoint{1.507752in}{1.961430in}}{\pgfqpoint{1.507752in}{1.969666in}}%
\pgfpathcurveto{\pgfqpoint{1.507752in}{1.977902in}}{\pgfqpoint{1.504480in}{1.985802in}}{\pgfqpoint{1.498656in}{1.991626in}}%
\pgfpathcurveto{\pgfqpoint{1.492832in}{1.997450in}}{\pgfqpoint{1.484932in}{2.000723in}}{\pgfqpoint{1.476696in}{2.000723in}}%
\pgfpathcurveto{\pgfqpoint{1.468459in}{2.000723in}}{\pgfqpoint{1.460559in}{1.997450in}}{\pgfqpoint{1.454735in}{1.991626in}}%
\pgfpathcurveto{\pgfqpoint{1.448911in}{1.985802in}}{\pgfqpoint{1.445639in}{1.977902in}}{\pgfqpoint{1.445639in}{1.969666in}}%
\pgfpathcurveto{\pgfqpoint{1.445639in}{1.961430in}}{\pgfqpoint{1.448911in}{1.953530in}}{\pgfqpoint{1.454735in}{1.947706in}}%
\pgfpathcurveto{\pgfqpoint{1.460559in}{1.941882in}}{\pgfqpoint{1.468459in}{1.938610in}}{\pgfqpoint{1.476696in}{1.938610in}}%
\pgfpathclose%
\pgfusepath{stroke,fill}%
\end{pgfscope}%
\begin{pgfscope}%
\pgfpathrectangle{\pgfqpoint{0.100000in}{0.212622in}}{\pgfqpoint{3.696000in}{3.696000in}}%
\pgfusepath{clip}%
\pgfsetbuttcap%
\pgfsetroundjoin%
\definecolor{currentfill}{rgb}{0.121569,0.466667,0.705882}%
\pgfsetfillcolor{currentfill}%
\pgfsetfillopacity{0.322100}%
\pgfsetlinewidth{1.003750pt}%
\definecolor{currentstroke}{rgb}{0.121569,0.466667,0.705882}%
\pgfsetstrokecolor{currentstroke}%
\pgfsetstrokeopacity{0.322100}%
\pgfsetdash{}{0pt}%
\pgfpathmoveto{\pgfqpoint{1.489505in}{1.949927in}}%
\pgfpathcurveto{\pgfqpoint{1.497741in}{1.949927in}}{\pgfqpoint{1.505641in}{1.953199in}}{\pgfqpoint{1.511465in}{1.959023in}}%
\pgfpathcurveto{\pgfqpoint{1.517289in}{1.964847in}}{\pgfqpoint{1.520561in}{1.972747in}}{\pgfqpoint{1.520561in}{1.980983in}}%
\pgfpathcurveto{\pgfqpoint{1.520561in}{1.989220in}}{\pgfqpoint{1.517289in}{1.997120in}}{\pgfqpoint{1.511465in}{2.002944in}}%
\pgfpathcurveto{\pgfqpoint{1.505641in}{2.008768in}}{\pgfqpoint{1.497741in}{2.012040in}}{\pgfqpoint{1.489505in}{2.012040in}}%
\pgfpathcurveto{\pgfqpoint{1.481269in}{2.012040in}}{\pgfqpoint{1.473368in}{2.008768in}}{\pgfqpoint{1.467545in}{2.002944in}}%
\pgfpathcurveto{\pgfqpoint{1.461721in}{1.997120in}}{\pgfqpoint{1.458448in}{1.989220in}}{\pgfqpoint{1.458448in}{1.980983in}}%
\pgfpathcurveto{\pgfqpoint{1.458448in}{1.972747in}}{\pgfqpoint{1.461721in}{1.964847in}}{\pgfqpoint{1.467545in}{1.959023in}}%
\pgfpathcurveto{\pgfqpoint{1.473368in}{1.953199in}}{\pgfqpoint{1.481269in}{1.949927in}}{\pgfqpoint{1.489505in}{1.949927in}}%
\pgfpathclose%
\pgfusepath{stroke,fill}%
\end{pgfscope}%
\begin{pgfscope}%
\pgfpathrectangle{\pgfqpoint{0.100000in}{0.212622in}}{\pgfqpoint{3.696000in}{3.696000in}}%
\pgfusepath{clip}%
\pgfsetbuttcap%
\pgfsetroundjoin%
\definecolor{currentfill}{rgb}{0.121569,0.466667,0.705882}%
\pgfsetfillcolor{currentfill}%
\pgfsetfillopacity{0.322250}%
\pgfsetlinewidth{1.003750pt}%
\definecolor{currentstroke}{rgb}{0.121569,0.466667,0.705882}%
\pgfsetstrokecolor{currentstroke}%
\pgfsetstrokeopacity{0.322250}%
\pgfsetdash{}{0pt}%
\pgfpathmoveto{\pgfqpoint{1.484067in}{1.944845in}}%
\pgfpathcurveto{\pgfqpoint{1.492303in}{1.944845in}}{\pgfqpoint{1.500203in}{1.948118in}}{\pgfqpoint{1.506027in}{1.953942in}}%
\pgfpathcurveto{\pgfqpoint{1.511851in}{1.959766in}}{\pgfqpoint{1.515123in}{1.967666in}}{\pgfqpoint{1.515123in}{1.975902in}}%
\pgfpathcurveto{\pgfqpoint{1.515123in}{1.984138in}}{\pgfqpoint{1.511851in}{1.992038in}}{\pgfqpoint{1.506027in}{1.997862in}}%
\pgfpathcurveto{\pgfqpoint{1.500203in}{2.003686in}}{\pgfqpoint{1.492303in}{2.006958in}}{\pgfqpoint{1.484067in}{2.006958in}}%
\pgfpathcurveto{\pgfqpoint{1.475830in}{2.006958in}}{\pgfqpoint{1.467930in}{2.003686in}}{\pgfqpoint{1.462106in}{1.997862in}}%
\pgfpathcurveto{\pgfqpoint{1.456283in}{1.992038in}}{\pgfqpoint{1.453010in}{1.984138in}}{\pgfqpoint{1.453010in}{1.975902in}}%
\pgfpathcurveto{\pgfqpoint{1.453010in}{1.967666in}}{\pgfqpoint{1.456283in}{1.959766in}}{\pgfqpoint{1.462106in}{1.953942in}}%
\pgfpathcurveto{\pgfqpoint{1.467930in}{1.948118in}}{\pgfqpoint{1.475830in}{1.944845in}}{\pgfqpoint{1.484067in}{1.944845in}}%
\pgfpathclose%
\pgfusepath{stroke,fill}%
\end{pgfscope}%
\begin{pgfscope}%
\pgfpathrectangle{\pgfqpoint{0.100000in}{0.212622in}}{\pgfqpoint{3.696000in}{3.696000in}}%
\pgfusepath{clip}%
\pgfsetbuttcap%
\pgfsetroundjoin%
\definecolor{currentfill}{rgb}{0.121569,0.466667,0.705882}%
\pgfsetfillcolor{currentfill}%
\pgfsetfillopacity{0.322337}%
\pgfsetlinewidth{1.003750pt}%
\definecolor{currentstroke}{rgb}{0.121569,0.466667,0.705882}%
\pgfsetstrokecolor{currentstroke}%
\pgfsetstrokeopacity{0.322337}%
\pgfsetdash{}{0pt}%
\pgfpathmoveto{\pgfqpoint{1.577019in}{2.019052in}}%
\pgfpathcurveto{\pgfqpoint{1.585255in}{2.019052in}}{\pgfqpoint{1.593155in}{2.022325in}}{\pgfqpoint{1.598979in}{2.028149in}}%
\pgfpathcurveto{\pgfqpoint{1.604803in}{2.033973in}}{\pgfqpoint{1.608075in}{2.041873in}}{\pgfqpoint{1.608075in}{2.050109in}}%
\pgfpathcurveto{\pgfqpoint{1.608075in}{2.058345in}}{\pgfqpoint{1.604803in}{2.066245in}}{\pgfqpoint{1.598979in}{2.072069in}}%
\pgfpathcurveto{\pgfqpoint{1.593155in}{2.077893in}}{\pgfqpoint{1.585255in}{2.081165in}}{\pgfqpoint{1.577019in}{2.081165in}}%
\pgfpathcurveto{\pgfqpoint{1.568783in}{2.081165in}}{\pgfqpoint{1.560883in}{2.077893in}}{\pgfqpoint{1.555059in}{2.072069in}}%
\pgfpathcurveto{\pgfqpoint{1.549235in}{2.066245in}}{\pgfqpoint{1.545962in}{2.058345in}}{\pgfqpoint{1.545962in}{2.050109in}}%
\pgfpathcurveto{\pgfqpoint{1.545962in}{2.041873in}}{\pgfqpoint{1.549235in}{2.033973in}}{\pgfqpoint{1.555059in}{2.028149in}}%
\pgfpathcurveto{\pgfqpoint{1.560883in}{2.022325in}}{\pgfqpoint{1.568783in}{2.019052in}}{\pgfqpoint{1.577019in}{2.019052in}}%
\pgfpathclose%
\pgfusepath{stroke,fill}%
\end{pgfscope}%
\begin{pgfscope}%
\pgfpathrectangle{\pgfqpoint{0.100000in}{0.212622in}}{\pgfqpoint{3.696000in}{3.696000in}}%
\pgfusepath{clip}%
\pgfsetbuttcap%
\pgfsetroundjoin%
\definecolor{currentfill}{rgb}{0.121569,0.466667,0.705882}%
\pgfsetfillcolor{currentfill}%
\pgfsetfillopacity{0.322347}%
\pgfsetlinewidth{1.003750pt}%
\definecolor{currentstroke}{rgb}{0.121569,0.466667,0.705882}%
\pgfsetstrokecolor{currentstroke}%
\pgfsetstrokeopacity{0.322347}%
\pgfsetdash{}{0pt}%
\pgfpathmoveto{\pgfqpoint{1.485114in}{1.945199in}}%
\pgfpathcurveto{\pgfqpoint{1.493350in}{1.945199in}}{\pgfqpoint{1.501250in}{1.948471in}}{\pgfqpoint{1.507074in}{1.954295in}}%
\pgfpathcurveto{\pgfqpoint{1.512898in}{1.960119in}}{\pgfqpoint{1.516170in}{1.968019in}}{\pgfqpoint{1.516170in}{1.976255in}}%
\pgfpathcurveto{\pgfqpoint{1.516170in}{1.984491in}}{\pgfqpoint{1.512898in}{1.992392in}}{\pgfqpoint{1.507074in}{1.998215in}}%
\pgfpathcurveto{\pgfqpoint{1.501250in}{2.004039in}}{\pgfqpoint{1.493350in}{2.007312in}}{\pgfqpoint{1.485114in}{2.007312in}}%
\pgfpathcurveto{\pgfqpoint{1.476877in}{2.007312in}}{\pgfqpoint{1.468977in}{2.004039in}}{\pgfqpoint{1.463153in}{1.998215in}}%
\pgfpathcurveto{\pgfqpoint{1.457329in}{1.992392in}}{\pgfqpoint{1.454057in}{1.984491in}}{\pgfqpoint{1.454057in}{1.976255in}}%
\pgfpathcurveto{\pgfqpoint{1.454057in}{1.968019in}}{\pgfqpoint{1.457329in}{1.960119in}}{\pgfqpoint{1.463153in}{1.954295in}}%
\pgfpathcurveto{\pgfqpoint{1.468977in}{1.948471in}}{\pgfqpoint{1.476877in}{1.945199in}}{\pgfqpoint{1.485114in}{1.945199in}}%
\pgfpathclose%
\pgfusepath{stroke,fill}%
\end{pgfscope}%
\begin{pgfscope}%
\pgfpathrectangle{\pgfqpoint{0.100000in}{0.212622in}}{\pgfqpoint{3.696000in}{3.696000in}}%
\pgfusepath{clip}%
\pgfsetbuttcap%
\pgfsetroundjoin%
\definecolor{currentfill}{rgb}{0.121569,0.466667,0.705882}%
\pgfsetfillcolor{currentfill}%
\pgfsetfillopacity{0.322645}%
\pgfsetlinewidth{1.003750pt}%
\definecolor{currentstroke}{rgb}{0.121569,0.466667,0.705882}%
\pgfsetstrokecolor{currentstroke}%
\pgfsetstrokeopacity{0.322645}%
\pgfsetdash{}{0pt}%
\pgfpathmoveto{\pgfqpoint{1.467634in}{1.931291in}}%
\pgfpathcurveto{\pgfqpoint{1.475870in}{1.931291in}}{\pgfqpoint{1.483770in}{1.934564in}}{\pgfqpoint{1.489594in}{1.940388in}}%
\pgfpathcurveto{\pgfqpoint{1.495418in}{1.946212in}}{\pgfqpoint{1.498690in}{1.954112in}}{\pgfqpoint{1.498690in}{1.962348in}}%
\pgfpathcurveto{\pgfqpoint{1.498690in}{1.970584in}}{\pgfqpoint{1.495418in}{1.978484in}}{\pgfqpoint{1.489594in}{1.984308in}}%
\pgfpathcurveto{\pgfqpoint{1.483770in}{1.990132in}}{\pgfqpoint{1.475870in}{1.993404in}}{\pgfqpoint{1.467634in}{1.993404in}}%
\pgfpathcurveto{\pgfqpoint{1.459398in}{1.993404in}}{\pgfqpoint{1.451498in}{1.990132in}}{\pgfqpoint{1.445674in}{1.984308in}}%
\pgfpathcurveto{\pgfqpoint{1.439850in}{1.978484in}}{\pgfqpoint{1.436577in}{1.970584in}}{\pgfqpoint{1.436577in}{1.962348in}}%
\pgfpathcurveto{\pgfqpoint{1.436577in}{1.954112in}}{\pgfqpoint{1.439850in}{1.946212in}}{\pgfqpoint{1.445674in}{1.940388in}}%
\pgfpathcurveto{\pgfqpoint{1.451498in}{1.934564in}}{\pgfqpoint{1.459398in}{1.931291in}}{\pgfqpoint{1.467634in}{1.931291in}}%
\pgfpathclose%
\pgfusepath{stroke,fill}%
\end{pgfscope}%
\begin{pgfscope}%
\pgfpathrectangle{\pgfqpoint{0.100000in}{0.212622in}}{\pgfqpoint{3.696000in}{3.696000in}}%
\pgfusepath{clip}%
\pgfsetbuttcap%
\pgfsetroundjoin%
\definecolor{currentfill}{rgb}{0.121569,0.466667,0.705882}%
\pgfsetfillcolor{currentfill}%
\pgfsetfillopacity{0.322706}%
\pgfsetlinewidth{1.003750pt}%
\definecolor{currentstroke}{rgb}{0.121569,0.466667,0.705882}%
\pgfsetstrokecolor{currentstroke}%
\pgfsetstrokeopacity{0.322706}%
\pgfsetdash{}{0pt}%
\pgfpathmoveto{\pgfqpoint{1.470432in}{1.933712in}}%
\pgfpathcurveto{\pgfqpoint{1.478668in}{1.933712in}}{\pgfqpoint{1.486568in}{1.936984in}}{\pgfqpoint{1.492392in}{1.942808in}}%
\pgfpathcurveto{\pgfqpoint{1.498216in}{1.948632in}}{\pgfqpoint{1.501488in}{1.956532in}}{\pgfqpoint{1.501488in}{1.964768in}}%
\pgfpathcurveto{\pgfqpoint{1.501488in}{1.973005in}}{\pgfqpoint{1.498216in}{1.980905in}}{\pgfqpoint{1.492392in}{1.986729in}}%
\pgfpathcurveto{\pgfqpoint{1.486568in}{1.992552in}}{\pgfqpoint{1.478668in}{1.995825in}}{\pgfqpoint{1.470432in}{1.995825in}}%
\pgfpathcurveto{\pgfqpoint{1.462196in}{1.995825in}}{\pgfqpoint{1.454296in}{1.992552in}}{\pgfqpoint{1.448472in}{1.986729in}}%
\pgfpathcurveto{\pgfqpoint{1.442648in}{1.980905in}}{\pgfqpoint{1.439375in}{1.973005in}}{\pgfqpoint{1.439375in}{1.964768in}}%
\pgfpathcurveto{\pgfqpoint{1.439375in}{1.956532in}}{\pgfqpoint{1.442648in}{1.948632in}}{\pgfqpoint{1.448472in}{1.942808in}}%
\pgfpathcurveto{\pgfqpoint{1.454296in}{1.936984in}}{\pgfqpoint{1.462196in}{1.933712in}}{\pgfqpoint{1.470432in}{1.933712in}}%
\pgfpathclose%
\pgfusepath{stroke,fill}%
\end{pgfscope}%
\begin{pgfscope}%
\pgfpathrectangle{\pgfqpoint{0.100000in}{0.212622in}}{\pgfqpoint{3.696000in}{3.696000in}}%
\pgfusepath{clip}%
\pgfsetbuttcap%
\pgfsetroundjoin%
\definecolor{currentfill}{rgb}{0.121569,0.466667,0.705882}%
\pgfsetfillcolor{currentfill}%
\pgfsetfillopacity{0.322777}%
\pgfsetlinewidth{1.003750pt}%
\definecolor{currentstroke}{rgb}{0.121569,0.466667,0.705882}%
\pgfsetstrokecolor{currentstroke}%
\pgfsetstrokeopacity{0.322777}%
\pgfsetdash{}{0pt}%
\pgfpathmoveto{\pgfqpoint{1.464701in}{1.928366in}}%
\pgfpathcurveto{\pgfqpoint{1.472937in}{1.928366in}}{\pgfqpoint{1.480837in}{1.931638in}}{\pgfqpoint{1.486661in}{1.937462in}}%
\pgfpathcurveto{\pgfqpoint{1.492485in}{1.943286in}}{\pgfqpoint{1.495757in}{1.951186in}}{\pgfqpoint{1.495757in}{1.959422in}}%
\pgfpathcurveto{\pgfqpoint{1.495757in}{1.967659in}}{\pgfqpoint{1.492485in}{1.975559in}}{\pgfqpoint{1.486661in}{1.981382in}}%
\pgfpathcurveto{\pgfqpoint{1.480837in}{1.987206in}}{\pgfqpoint{1.472937in}{1.990479in}}{\pgfqpoint{1.464701in}{1.990479in}}%
\pgfpathcurveto{\pgfqpoint{1.456465in}{1.990479in}}{\pgfqpoint{1.448565in}{1.987206in}}{\pgfqpoint{1.442741in}{1.981382in}}%
\pgfpathcurveto{\pgfqpoint{1.436917in}{1.975559in}}{\pgfqpoint{1.433644in}{1.967659in}}{\pgfqpoint{1.433644in}{1.959422in}}%
\pgfpathcurveto{\pgfqpoint{1.433644in}{1.951186in}}{\pgfqpoint{1.436917in}{1.943286in}}{\pgfqpoint{1.442741in}{1.937462in}}%
\pgfpathcurveto{\pgfqpoint{1.448565in}{1.931638in}}{\pgfqpoint{1.456465in}{1.928366in}}{\pgfqpoint{1.464701in}{1.928366in}}%
\pgfpathclose%
\pgfusepath{stroke,fill}%
\end{pgfscope}%
\begin{pgfscope}%
\pgfpathrectangle{\pgfqpoint{0.100000in}{0.212622in}}{\pgfqpoint{3.696000in}{3.696000in}}%
\pgfusepath{clip}%
\pgfsetbuttcap%
\pgfsetroundjoin%
\definecolor{currentfill}{rgb}{0.121569,0.466667,0.705882}%
\pgfsetfillcolor{currentfill}%
\pgfsetfillopacity{0.323117}%
\pgfsetlinewidth{1.003750pt}%
\definecolor{currentstroke}{rgb}{0.121569,0.466667,0.705882}%
\pgfsetstrokecolor{currentstroke}%
\pgfsetstrokeopacity{0.323117}%
\pgfsetdash{}{0pt}%
\pgfpathmoveto{\pgfqpoint{1.478724in}{1.940690in}}%
\pgfpathcurveto{\pgfqpoint{1.486960in}{1.940690in}}{\pgfqpoint{1.494860in}{1.943962in}}{\pgfqpoint{1.500684in}{1.949786in}}%
\pgfpathcurveto{\pgfqpoint{1.506508in}{1.955610in}}{\pgfqpoint{1.509780in}{1.963510in}}{\pgfqpoint{1.509780in}{1.971746in}}%
\pgfpathcurveto{\pgfqpoint{1.509780in}{1.979982in}}{\pgfqpoint{1.506508in}{1.987882in}}{\pgfqpoint{1.500684in}{1.993706in}}%
\pgfpathcurveto{\pgfqpoint{1.494860in}{1.999530in}}{\pgfqpoint{1.486960in}{2.002803in}}{\pgfqpoint{1.478724in}{2.002803in}}%
\pgfpathcurveto{\pgfqpoint{1.470488in}{2.002803in}}{\pgfqpoint{1.462588in}{1.999530in}}{\pgfqpoint{1.456764in}{1.993706in}}%
\pgfpathcurveto{\pgfqpoint{1.450940in}{1.987882in}}{\pgfqpoint{1.447667in}{1.979982in}}{\pgfqpoint{1.447667in}{1.971746in}}%
\pgfpathcurveto{\pgfqpoint{1.447667in}{1.963510in}}{\pgfqpoint{1.450940in}{1.955610in}}{\pgfqpoint{1.456764in}{1.949786in}}%
\pgfpathcurveto{\pgfqpoint{1.462588in}{1.943962in}}{\pgfqpoint{1.470488in}{1.940690in}}{\pgfqpoint{1.478724in}{1.940690in}}%
\pgfpathclose%
\pgfusepath{stroke,fill}%
\end{pgfscope}%
\begin{pgfscope}%
\pgfpathrectangle{\pgfqpoint{0.100000in}{0.212622in}}{\pgfqpoint{3.696000in}{3.696000in}}%
\pgfusepath{clip}%
\pgfsetbuttcap%
\pgfsetroundjoin%
\definecolor{currentfill}{rgb}{0.121569,0.466667,0.705882}%
\pgfsetfillcolor{currentfill}%
\pgfsetfillopacity{0.323872}%
\pgfsetlinewidth{1.003750pt}%
\definecolor{currentstroke}{rgb}{0.121569,0.466667,0.705882}%
\pgfsetstrokecolor{currentstroke}%
\pgfsetstrokeopacity{0.323872}%
\pgfsetdash{}{0pt}%
\pgfpathmoveto{\pgfqpoint{1.561852in}{2.005553in}}%
\pgfpathcurveto{\pgfqpoint{1.570088in}{2.005553in}}{\pgfqpoint{1.577988in}{2.008825in}}{\pgfqpoint{1.583812in}{2.014649in}}%
\pgfpathcurveto{\pgfqpoint{1.589636in}{2.020473in}}{\pgfqpoint{1.592908in}{2.028373in}}{\pgfqpoint{1.592908in}{2.036609in}}%
\pgfpathcurveto{\pgfqpoint{1.592908in}{2.044846in}}{\pgfqpoint{1.589636in}{2.052746in}}{\pgfqpoint{1.583812in}{2.058570in}}%
\pgfpathcurveto{\pgfqpoint{1.577988in}{2.064394in}}{\pgfqpoint{1.570088in}{2.067666in}}{\pgfqpoint{1.561852in}{2.067666in}}%
\pgfpathcurveto{\pgfqpoint{1.553615in}{2.067666in}}{\pgfqpoint{1.545715in}{2.064394in}}{\pgfqpoint{1.539891in}{2.058570in}}%
\pgfpathcurveto{\pgfqpoint{1.534067in}{2.052746in}}{\pgfqpoint{1.530795in}{2.044846in}}{\pgfqpoint{1.530795in}{2.036609in}}%
\pgfpathcurveto{\pgfqpoint{1.530795in}{2.028373in}}{\pgfqpoint{1.534067in}{2.020473in}}{\pgfqpoint{1.539891in}{2.014649in}}%
\pgfpathcurveto{\pgfqpoint{1.545715in}{2.008825in}}{\pgfqpoint{1.553615in}{2.005553in}}{\pgfqpoint{1.561852in}{2.005553in}}%
\pgfpathclose%
\pgfusepath{stroke,fill}%
\end{pgfscope}%
\begin{pgfscope}%
\pgfpathrectangle{\pgfqpoint{0.100000in}{0.212622in}}{\pgfqpoint{3.696000in}{3.696000in}}%
\pgfusepath{clip}%
\pgfsetbuttcap%
\pgfsetroundjoin%
\definecolor{currentfill}{rgb}{0.121569,0.466667,0.705882}%
\pgfsetfillcolor{currentfill}%
\pgfsetfillopacity{0.324093}%
\pgfsetlinewidth{1.003750pt}%
\definecolor{currentstroke}{rgb}{0.121569,0.466667,0.705882}%
\pgfsetstrokecolor{currentstroke}%
\pgfsetstrokeopacity{0.324093}%
\pgfsetdash{}{0pt}%
\pgfpathmoveto{\pgfqpoint{1.362350in}{1.867835in}}%
\pgfpathcurveto{\pgfqpoint{1.370586in}{1.867835in}}{\pgfqpoint{1.378486in}{1.871108in}}{\pgfqpoint{1.384310in}{1.876932in}}%
\pgfpathcurveto{\pgfqpoint{1.390134in}{1.882756in}}{\pgfqpoint{1.393406in}{1.890656in}}{\pgfqpoint{1.393406in}{1.898892in}}%
\pgfpathcurveto{\pgfqpoint{1.393406in}{1.907128in}}{\pgfqpoint{1.390134in}{1.915028in}}{\pgfqpoint{1.384310in}{1.920852in}}%
\pgfpathcurveto{\pgfqpoint{1.378486in}{1.926676in}}{\pgfqpoint{1.370586in}{1.929948in}}{\pgfqpoint{1.362350in}{1.929948in}}%
\pgfpathcurveto{\pgfqpoint{1.354114in}{1.929948in}}{\pgfqpoint{1.346213in}{1.926676in}}{\pgfqpoint{1.340390in}{1.920852in}}%
\pgfpathcurveto{\pgfqpoint{1.334566in}{1.915028in}}{\pgfqpoint{1.331293in}{1.907128in}}{\pgfqpoint{1.331293in}{1.898892in}}%
\pgfpathcurveto{\pgfqpoint{1.331293in}{1.890656in}}{\pgfqpoint{1.334566in}{1.882756in}}{\pgfqpoint{1.340390in}{1.876932in}}%
\pgfpathcurveto{\pgfqpoint{1.346213in}{1.871108in}}{\pgfqpoint{1.354114in}{1.867835in}}{\pgfqpoint{1.362350in}{1.867835in}}%
\pgfpathclose%
\pgfusepath{stroke,fill}%
\end{pgfscope}%
\begin{pgfscope}%
\pgfpathrectangle{\pgfqpoint{0.100000in}{0.212622in}}{\pgfqpoint{3.696000in}{3.696000in}}%
\pgfusepath{clip}%
\pgfsetbuttcap%
\pgfsetroundjoin%
\definecolor{currentfill}{rgb}{0.121569,0.466667,0.705882}%
\pgfsetfillcolor{currentfill}%
\pgfsetfillopacity{0.324376}%
\pgfsetlinewidth{1.003750pt}%
\definecolor{currentstroke}{rgb}{0.121569,0.466667,0.705882}%
\pgfsetstrokecolor{currentstroke}%
\pgfsetstrokeopacity{0.324376}%
\pgfsetdash{}{0pt}%
\pgfpathmoveto{\pgfqpoint{1.592309in}{2.024230in}}%
\pgfpathcurveto{\pgfqpoint{1.600545in}{2.024230in}}{\pgfqpoint{1.608445in}{2.027502in}}{\pgfqpoint{1.614269in}{2.033326in}}%
\pgfpathcurveto{\pgfqpoint{1.620093in}{2.039150in}}{\pgfqpoint{1.623365in}{2.047050in}}{\pgfqpoint{1.623365in}{2.055287in}}%
\pgfpathcurveto{\pgfqpoint{1.623365in}{2.063523in}}{\pgfqpoint{1.620093in}{2.071423in}}{\pgfqpoint{1.614269in}{2.077247in}}%
\pgfpathcurveto{\pgfqpoint{1.608445in}{2.083071in}}{\pgfqpoint{1.600545in}{2.086343in}}{\pgfqpoint{1.592309in}{2.086343in}}%
\pgfpathcurveto{\pgfqpoint{1.584073in}{2.086343in}}{\pgfqpoint{1.576172in}{2.083071in}}{\pgfqpoint{1.570349in}{2.077247in}}%
\pgfpathcurveto{\pgfqpoint{1.564525in}{2.071423in}}{\pgfqpoint{1.561252in}{2.063523in}}{\pgfqpoint{1.561252in}{2.055287in}}%
\pgfpathcurveto{\pgfqpoint{1.561252in}{2.047050in}}{\pgfqpoint{1.564525in}{2.039150in}}{\pgfqpoint{1.570349in}{2.033326in}}%
\pgfpathcurveto{\pgfqpoint{1.576172in}{2.027502in}}{\pgfqpoint{1.584073in}{2.024230in}}{\pgfqpoint{1.592309in}{2.024230in}}%
\pgfpathclose%
\pgfusepath{stroke,fill}%
\end{pgfscope}%
\begin{pgfscope}%
\pgfpathrectangle{\pgfqpoint{0.100000in}{0.212622in}}{\pgfqpoint{3.696000in}{3.696000in}}%
\pgfusepath{clip}%
\pgfsetbuttcap%
\pgfsetroundjoin%
\definecolor{currentfill}{rgb}{0.121569,0.466667,0.705882}%
\pgfsetfillcolor{currentfill}%
\pgfsetfillopacity{0.325043}%
\pgfsetlinewidth{1.003750pt}%
\definecolor{currentstroke}{rgb}{0.121569,0.466667,0.705882}%
\pgfsetstrokecolor{currentstroke}%
\pgfsetstrokeopacity{0.325043}%
\pgfsetdash{}{0pt}%
\pgfpathmoveto{\pgfqpoint{1.584576in}{2.021689in}}%
\pgfpathcurveto{\pgfqpoint{1.592812in}{2.021689in}}{\pgfqpoint{1.600712in}{2.024961in}}{\pgfqpoint{1.606536in}{2.030785in}}%
\pgfpathcurveto{\pgfqpoint{1.612360in}{2.036609in}}{\pgfqpoint{1.615632in}{2.044509in}}{\pgfqpoint{1.615632in}{2.052745in}}%
\pgfpathcurveto{\pgfqpoint{1.615632in}{2.060981in}}{\pgfqpoint{1.612360in}{2.068881in}}{\pgfqpoint{1.606536in}{2.074705in}}%
\pgfpathcurveto{\pgfqpoint{1.600712in}{2.080529in}}{\pgfqpoint{1.592812in}{2.083802in}}{\pgfqpoint{1.584576in}{2.083802in}}%
\pgfpathcurveto{\pgfqpoint{1.576339in}{2.083802in}}{\pgfqpoint{1.568439in}{2.080529in}}{\pgfqpoint{1.562615in}{2.074705in}}%
\pgfpathcurveto{\pgfqpoint{1.556791in}{2.068881in}}{\pgfqpoint{1.553519in}{2.060981in}}{\pgfqpoint{1.553519in}{2.052745in}}%
\pgfpathcurveto{\pgfqpoint{1.553519in}{2.044509in}}{\pgfqpoint{1.556791in}{2.036609in}}{\pgfqpoint{1.562615in}{2.030785in}}%
\pgfpathcurveto{\pgfqpoint{1.568439in}{2.024961in}}{\pgfqpoint{1.576339in}{2.021689in}}{\pgfqpoint{1.584576in}{2.021689in}}%
\pgfpathclose%
\pgfusepath{stroke,fill}%
\end{pgfscope}%
\begin{pgfscope}%
\pgfpathrectangle{\pgfqpoint{0.100000in}{0.212622in}}{\pgfqpoint{3.696000in}{3.696000in}}%
\pgfusepath{clip}%
\pgfsetbuttcap%
\pgfsetroundjoin%
\definecolor{currentfill}{rgb}{0.121569,0.466667,0.705882}%
\pgfsetfillcolor{currentfill}%
\pgfsetfillopacity{0.325794}%
\pgfsetlinewidth{1.003750pt}%
\definecolor{currentstroke}{rgb}{0.121569,0.466667,0.705882}%
\pgfsetstrokecolor{currentstroke}%
\pgfsetstrokeopacity{0.325794}%
\pgfsetdash{}{0pt}%
\pgfpathmoveto{\pgfqpoint{1.595943in}{2.028106in}}%
\pgfpathcurveto{\pgfqpoint{1.604179in}{2.028106in}}{\pgfqpoint{1.612079in}{2.031379in}}{\pgfqpoint{1.617903in}{2.037203in}}%
\pgfpathcurveto{\pgfqpoint{1.623727in}{2.043026in}}{\pgfqpoint{1.627000in}{2.050927in}}{\pgfqpoint{1.627000in}{2.059163in}}%
\pgfpathcurveto{\pgfqpoint{1.627000in}{2.067399in}}{\pgfqpoint{1.623727in}{2.075299in}}{\pgfqpoint{1.617903in}{2.081123in}}%
\pgfpathcurveto{\pgfqpoint{1.612079in}{2.086947in}}{\pgfqpoint{1.604179in}{2.090219in}}{\pgfqpoint{1.595943in}{2.090219in}}%
\pgfpathcurveto{\pgfqpoint{1.587707in}{2.090219in}}{\pgfqpoint{1.579807in}{2.086947in}}{\pgfqpoint{1.573983in}{2.081123in}}%
\pgfpathcurveto{\pgfqpoint{1.568159in}{2.075299in}}{\pgfqpoint{1.564887in}{2.067399in}}{\pgfqpoint{1.564887in}{2.059163in}}%
\pgfpathcurveto{\pgfqpoint{1.564887in}{2.050927in}}{\pgfqpoint{1.568159in}{2.043026in}}{\pgfqpoint{1.573983in}{2.037203in}}%
\pgfpathcurveto{\pgfqpoint{1.579807in}{2.031379in}}{\pgfqpoint{1.587707in}{2.028106in}}{\pgfqpoint{1.595943in}{2.028106in}}%
\pgfpathclose%
\pgfusepath{stroke,fill}%
\end{pgfscope}%
\begin{pgfscope}%
\pgfpathrectangle{\pgfqpoint{0.100000in}{0.212622in}}{\pgfqpoint{3.696000in}{3.696000in}}%
\pgfusepath{clip}%
\pgfsetbuttcap%
\pgfsetroundjoin%
\definecolor{currentfill}{rgb}{0.121569,0.466667,0.705882}%
\pgfsetfillcolor{currentfill}%
\pgfsetfillopacity{0.325945}%
\pgfsetlinewidth{1.003750pt}%
\definecolor{currentstroke}{rgb}{0.121569,0.466667,0.705882}%
\pgfsetstrokecolor{currentstroke}%
\pgfsetstrokeopacity{0.325945}%
\pgfsetdash{}{0pt}%
\pgfpathmoveto{\pgfqpoint{1.520763in}{1.975281in}}%
\pgfpathcurveto{\pgfqpoint{1.529000in}{1.975281in}}{\pgfqpoint{1.536900in}{1.978553in}}{\pgfqpoint{1.542724in}{1.984377in}}%
\pgfpathcurveto{\pgfqpoint{1.548548in}{1.990201in}}{\pgfqpoint{1.551820in}{1.998101in}}{\pgfqpoint{1.551820in}{2.006337in}}%
\pgfpathcurveto{\pgfqpoint{1.551820in}{2.014574in}}{\pgfqpoint{1.548548in}{2.022474in}}{\pgfqpoint{1.542724in}{2.028297in}}%
\pgfpathcurveto{\pgfqpoint{1.536900in}{2.034121in}}{\pgfqpoint{1.529000in}{2.037394in}}{\pgfqpoint{1.520763in}{2.037394in}}%
\pgfpathcurveto{\pgfqpoint{1.512527in}{2.037394in}}{\pgfqpoint{1.504627in}{2.034121in}}{\pgfqpoint{1.498803in}{2.028297in}}%
\pgfpathcurveto{\pgfqpoint{1.492979in}{2.022474in}}{\pgfqpoint{1.489707in}{2.014574in}}{\pgfqpoint{1.489707in}{2.006337in}}%
\pgfpathcurveto{\pgfqpoint{1.489707in}{1.998101in}}{\pgfqpoint{1.492979in}{1.990201in}}{\pgfqpoint{1.498803in}{1.984377in}}%
\pgfpathcurveto{\pgfqpoint{1.504627in}{1.978553in}}{\pgfqpoint{1.512527in}{1.975281in}}{\pgfqpoint{1.520763in}{1.975281in}}%
\pgfpathclose%
\pgfusepath{stroke,fill}%
\end{pgfscope}%
\begin{pgfscope}%
\pgfpathrectangle{\pgfqpoint{0.100000in}{0.212622in}}{\pgfqpoint{3.696000in}{3.696000in}}%
\pgfusepath{clip}%
\pgfsetbuttcap%
\pgfsetroundjoin%
\definecolor{currentfill}{rgb}{0.121569,0.466667,0.705882}%
\pgfsetfillcolor{currentfill}%
\pgfsetfillopacity{0.326374}%
\pgfsetlinewidth{1.003750pt}%
\definecolor{currentstroke}{rgb}{0.121569,0.466667,0.705882}%
\pgfsetstrokecolor{currentstroke}%
\pgfsetstrokeopacity{0.326374}%
\pgfsetdash{}{0pt}%
\pgfpathmoveto{\pgfqpoint{1.526177in}{1.974376in}}%
\pgfpathcurveto{\pgfqpoint{1.534414in}{1.974376in}}{\pgfqpoint{1.542314in}{1.977648in}}{\pgfqpoint{1.548138in}{1.983472in}}%
\pgfpathcurveto{\pgfqpoint{1.553962in}{1.989296in}}{\pgfqpoint{1.557234in}{1.997196in}}{\pgfqpoint{1.557234in}{2.005432in}}%
\pgfpathcurveto{\pgfqpoint{1.557234in}{2.013669in}}{\pgfqpoint{1.553962in}{2.021569in}}{\pgfqpoint{1.548138in}{2.027392in}}%
\pgfpathcurveto{\pgfqpoint{1.542314in}{2.033216in}}{\pgfqpoint{1.534414in}{2.036489in}}{\pgfqpoint{1.526177in}{2.036489in}}%
\pgfpathcurveto{\pgfqpoint{1.517941in}{2.036489in}}{\pgfqpoint{1.510041in}{2.033216in}}{\pgfqpoint{1.504217in}{2.027392in}}%
\pgfpathcurveto{\pgfqpoint{1.498393in}{2.021569in}}{\pgfqpoint{1.495121in}{2.013669in}}{\pgfqpoint{1.495121in}{2.005432in}}%
\pgfpathcurveto{\pgfqpoint{1.495121in}{1.997196in}}{\pgfqpoint{1.498393in}{1.989296in}}{\pgfqpoint{1.504217in}{1.983472in}}%
\pgfpathcurveto{\pgfqpoint{1.510041in}{1.977648in}}{\pgfqpoint{1.517941in}{1.974376in}}{\pgfqpoint{1.526177in}{1.974376in}}%
\pgfpathclose%
\pgfusepath{stroke,fill}%
\end{pgfscope}%
\begin{pgfscope}%
\pgfpathrectangle{\pgfqpoint{0.100000in}{0.212622in}}{\pgfqpoint{3.696000in}{3.696000in}}%
\pgfusepath{clip}%
\pgfsetbuttcap%
\pgfsetroundjoin%
\definecolor{currentfill}{rgb}{0.121569,0.466667,0.705882}%
\pgfsetfillcolor{currentfill}%
\pgfsetfillopacity{0.326786}%
\pgfsetlinewidth{1.003750pt}%
\definecolor{currentstroke}{rgb}{0.121569,0.466667,0.705882}%
\pgfsetstrokecolor{currentstroke}%
\pgfsetstrokeopacity{0.326786}%
\pgfsetdash{}{0pt}%
\pgfpathmoveto{\pgfqpoint{1.478838in}{1.939269in}}%
\pgfpathcurveto{\pgfqpoint{1.487074in}{1.939269in}}{\pgfqpoint{1.494974in}{1.942541in}}{\pgfqpoint{1.500798in}{1.948365in}}%
\pgfpathcurveto{\pgfqpoint{1.506622in}{1.954189in}}{\pgfqpoint{1.509894in}{1.962089in}}{\pgfqpoint{1.509894in}{1.970326in}}%
\pgfpathcurveto{\pgfqpoint{1.509894in}{1.978562in}}{\pgfqpoint{1.506622in}{1.986462in}}{\pgfqpoint{1.500798in}{1.992286in}}%
\pgfpathcurveto{\pgfqpoint{1.494974in}{1.998110in}}{\pgfqpoint{1.487074in}{2.001382in}}{\pgfqpoint{1.478838in}{2.001382in}}%
\pgfpathcurveto{\pgfqpoint{1.470601in}{2.001382in}}{\pgfqpoint{1.462701in}{1.998110in}}{\pgfqpoint{1.456877in}{1.992286in}}%
\pgfpathcurveto{\pgfqpoint{1.451053in}{1.986462in}}{\pgfqpoint{1.447781in}{1.978562in}}{\pgfqpoint{1.447781in}{1.970326in}}%
\pgfpathcurveto{\pgfqpoint{1.447781in}{1.962089in}}{\pgfqpoint{1.451053in}{1.954189in}}{\pgfqpoint{1.456877in}{1.948365in}}%
\pgfpathcurveto{\pgfqpoint{1.462701in}{1.942541in}}{\pgfqpoint{1.470601in}{1.939269in}}{\pgfqpoint{1.478838in}{1.939269in}}%
\pgfpathclose%
\pgfusepath{stroke,fill}%
\end{pgfscope}%
\begin{pgfscope}%
\pgfpathrectangle{\pgfqpoint{0.100000in}{0.212622in}}{\pgfqpoint{3.696000in}{3.696000in}}%
\pgfusepath{clip}%
\pgfsetbuttcap%
\pgfsetroundjoin%
\definecolor{currentfill}{rgb}{0.121569,0.466667,0.705882}%
\pgfsetfillcolor{currentfill}%
\pgfsetfillopacity{0.327182}%
\pgfsetlinewidth{1.003750pt}%
\definecolor{currentstroke}{rgb}{0.121569,0.466667,0.705882}%
\pgfsetstrokecolor{currentstroke}%
\pgfsetstrokeopacity{0.327182}%
\pgfsetdash{}{0pt}%
\pgfpathmoveto{\pgfqpoint{1.508509in}{1.963614in}}%
\pgfpathcurveto{\pgfqpoint{1.516745in}{1.963614in}}{\pgfqpoint{1.524645in}{1.966886in}}{\pgfqpoint{1.530469in}{1.972710in}}%
\pgfpathcurveto{\pgfqpoint{1.536293in}{1.978534in}}{\pgfqpoint{1.539565in}{1.986434in}}{\pgfqpoint{1.539565in}{1.994671in}}%
\pgfpathcurveto{\pgfqpoint{1.539565in}{2.002907in}}{\pgfqpoint{1.536293in}{2.010807in}}{\pgfqpoint{1.530469in}{2.016631in}}%
\pgfpathcurveto{\pgfqpoint{1.524645in}{2.022455in}}{\pgfqpoint{1.516745in}{2.025727in}}{\pgfqpoint{1.508509in}{2.025727in}}%
\pgfpathcurveto{\pgfqpoint{1.500272in}{2.025727in}}{\pgfqpoint{1.492372in}{2.022455in}}{\pgfqpoint{1.486548in}{2.016631in}}%
\pgfpathcurveto{\pgfqpoint{1.480724in}{2.010807in}}{\pgfqpoint{1.477452in}{2.002907in}}{\pgfqpoint{1.477452in}{1.994671in}}%
\pgfpathcurveto{\pgfqpoint{1.477452in}{1.986434in}}{\pgfqpoint{1.480724in}{1.978534in}}{\pgfqpoint{1.486548in}{1.972710in}}%
\pgfpathcurveto{\pgfqpoint{1.492372in}{1.966886in}}{\pgfqpoint{1.500272in}{1.963614in}}{\pgfqpoint{1.508509in}{1.963614in}}%
\pgfpathclose%
\pgfusepath{stroke,fill}%
\end{pgfscope}%
\begin{pgfscope}%
\pgfpathrectangle{\pgfqpoint{0.100000in}{0.212622in}}{\pgfqpoint{3.696000in}{3.696000in}}%
\pgfusepath{clip}%
\pgfsetbuttcap%
\pgfsetroundjoin%
\definecolor{currentfill}{rgb}{0.121569,0.466667,0.705882}%
\pgfsetfillcolor{currentfill}%
\pgfsetfillopacity{0.328025}%
\pgfsetlinewidth{1.003750pt}%
\definecolor{currentstroke}{rgb}{0.121569,0.466667,0.705882}%
\pgfsetstrokecolor{currentstroke}%
\pgfsetstrokeopacity{0.328025}%
\pgfsetdash{}{0pt}%
\pgfpathmoveto{\pgfqpoint{1.491401in}{1.949081in}}%
\pgfpathcurveto{\pgfqpoint{1.499638in}{1.949081in}}{\pgfqpoint{1.507538in}{1.952353in}}{\pgfqpoint{1.513362in}{1.958177in}}%
\pgfpathcurveto{\pgfqpoint{1.519186in}{1.964001in}}{\pgfqpoint{1.522458in}{1.971901in}}{\pgfqpoint{1.522458in}{1.980137in}}%
\pgfpathcurveto{\pgfqpoint{1.522458in}{1.988374in}}{\pgfqpoint{1.519186in}{1.996274in}}{\pgfqpoint{1.513362in}{2.002098in}}%
\pgfpathcurveto{\pgfqpoint{1.507538in}{2.007921in}}{\pgfqpoint{1.499638in}{2.011194in}}{\pgfqpoint{1.491401in}{2.011194in}}%
\pgfpathcurveto{\pgfqpoint{1.483165in}{2.011194in}}{\pgfqpoint{1.475265in}{2.007921in}}{\pgfqpoint{1.469441in}{2.002098in}}%
\pgfpathcurveto{\pgfqpoint{1.463617in}{1.996274in}}{\pgfqpoint{1.460345in}{1.988374in}}{\pgfqpoint{1.460345in}{1.980137in}}%
\pgfpathcurveto{\pgfqpoint{1.460345in}{1.971901in}}{\pgfqpoint{1.463617in}{1.964001in}}{\pgfqpoint{1.469441in}{1.958177in}}%
\pgfpathcurveto{\pgfqpoint{1.475265in}{1.952353in}}{\pgfqpoint{1.483165in}{1.949081in}}{\pgfqpoint{1.491401in}{1.949081in}}%
\pgfpathclose%
\pgfusepath{stroke,fill}%
\end{pgfscope}%
\begin{pgfscope}%
\pgfpathrectangle{\pgfqpoint{0.100000in}{0.212622in}}{\pgfqpoint{3.696000in}{3.696000in}}%
\pgfusepath{clip}%
\pgfsetbuttcap%
\pgfsetroundjoin%
\definecolor{currentfill}{rgb}{0.121569,0.466667,0.705882}%
\pgfsetfillcolor{currentfill}%
\pgfsetfillopacity{0.328039}%
\pgfsetlinewidth{1.003750pt}%
\definecolor{currentstroke}{rgb}{0.121569,0.466667,0.705882}%
\pgfsetstrokecolor{currentstroke}%
\pgfsetstrokeopacity{0.328039}%
\pgfsetdash{}{0pt}%
\pgfpathmoveto{\pgfqpoint{1.485972in}{1.944684in}}%
\pgfpathcurveto{\pgfqpoint{1.494208in}{1.944684in}}{\pgfqpoint{1.502108in}{1.947956in}}{\pgfqpoint{1.507932in}{1.953780in}}%
\pgfpathcurveto{\pgfqpoint{1.513756in}{1.959604in}}{\pgfqpoint{1.517029in}{1.967504in}}{\pgfqpoint{1.517029in}{1.975741in}}%
\pgfpathcurveto{\pgfqpoint{1.517029in}{1.983977in}}{\pgfqpoint{1.513756in}{1.991877in}}{\pgfqpoint{1.507932in}{1.997701in}}%
\pgfpathcurveto{\pgfqpoint{1.502108in}{2.003525in}}{\pgfqpoint{1.494208in}{2.006797in}}{\pgfqpoint{1.485972in}{2.006797in}}%
\pgfpathcurveto{\pgfqpoint{1.477736in}{2.006797in}}{\pgfqpoint{1.469836in}{2.003525in}}{\pgfqpoint{1.464012in}{1.997701in}}%
\pgfpathcurveto{\pgfqpoint{1.458188in}{1.991877in}}{\pgfqpoint{1.454916in}{1.983977in}}{\pgfqpoint{1.454916in}{1.975741in}}%
\pgfpathcurveto{\pgfqpoint{1.454916in}{1.967504in}}{\pgfqpoint{1.458188in}{1.959604in}}{\pgfqpoint{1.464012in}{1.953780in}}%
\pgfpathcurveto{\pgfqpoint{1.469836in}{1.947956in}}{\pgfqpoint{1.477736in}{1.944684in}}{\pgfqpoint{1.485972in}{1.944684in}}%
\pgfpathclose%
\pgfusepath{stroke,fill}%
\end{pgfscope}%
\begin{pgfscope}%
\pgfpathrectangle{\pgfqpoint{0.100000in}{0.212622in}}{\pgfqpoint{3.696000in}{3.696000in}}%
\pgfusepath{clip}%
\pgfsetbuttcap%
\pgfsetroundjoin%
\definecolor{currentfill}{rgb}{0.121569,0.466667,0.705882}%
\pgfsetfillcolor{currentfill}%
\pgfsetfillopacity{0.329219}%
\pgfsetlinewidth{1.003750pt}%
\definecolor{currentstroke}{rgb}{0.121569,0.466667,0.705882}%
\pgfsetstrokecolor{currentstroke}%
\pgfsetstrokeopacity{0.329219}%
\pgfsetdash{}{0pt}%
\pgfpathmoveto{\pgfqpoint{1.476292in}{1.937018in}}%
\pgfpathcurveto{\pgfqpoint{1.484528in}{1.937018in}}{\pgfqpoint{1.492428in}{1.940290in}}{\pgfqpoint{1.498252in}{1.946114in}}%
\pgfpathcurveto{\pgfqpoint{1.504076in}{1.951938in}}{\pgfqpoint{1.507348in}{1.959838in}}{\pgfqpoint{1.507348in}{1.968074in}}%
\pgfpathcurveto{\pgfqpoint{1.507348in}{1.976310in}}{\pgfqpoint{1.504076in}{1.984211in}}{\pgfqpoint{1.498252in}{1.990034in}}%
\pgfpathcurveto{\pgfqpoint{1.492428in}{1.995858in}}{\pgfqpoint{1.484528in}{1.999131in}}{\pgfqpoint{1.476292in}{1.999131in}}%
\pgfpathcurveto{\pgfqpoint{1.468056in}{1.999131in}}{\pgfqpoint{1.460155in}{1.995858in}}{\pgfqpoint{1.454332in}{1.990034in}}%
\pgfpathcurveto{\pgfqpoint{1.448508in}{1.984211in}}{\pgfqpoint{1.445235in}{1.976310in}}{\pgfqpoint{1.445235in}{1.968074in}}%
\pgfpathcurveto{\pgfqpoint{1.445235in}{1.959838in}}{\pgfqpoint{1.448508in}{1.951938in}}{\pgfqpoint{1.454332in}{1.946114in}}%
\pgfpathcurveto{\pgfqpoint{1.460155in}{1.940290in}}{\pgfqpoint{1.468056in}{1.937018in}}{\pgfqpoint{1.476292in}{1.937018in}}%
\pgfpathclose%
\pgfusepath{stroke,fill}%
\end{pgfscope}%
\begin{pgfscope}%
\pgfpathrectangle{\pgfqpoint{0.100000in}{0.212622in}}{\pgfqpoint{3.696000in}{3.696000in}}%
\pgfusepath{clip}%
\pgfsetbuttcap%
\pgfsetroundjoin%
\definecolor{currentfill}{rgb}{0.121569,0.466667,0.705882}%
\pgfsetfillcolor{currentfill}%
\pgfsetfillopacity{0.329344}%
\pgfsetlinewidth{1.003750pt}%
\definecolor{currentstroke}{rgb}{0.121569,0.466667,0.705882}%
\pgfsetstrokecolor{currentstroke}%
\pgfsetstrokeopacity{0.329344}%
\pgfsetdash{}{0pt}%
\pgfpathmoveto{\pgfqpoint{1.478828in}{1.939640in}}%
\pgfpathcurveto{\pgfqpoint{1.487064in}{1.939640in}}{\pgfqpoint{1.494964in}{1.942912in}}{\pgfqpoint{1.500788in}{1.948736in}}%
\pgfpathcurveto{\pgfqpoint{1.506612in}{1.954560in}}{\pgfqpoint{1.509884in}{1.962460in}}{\pgfqpoint{1.509884in}{1.970696in}}%
\pgfpathcurveto{\pgfqpoint{1.509884in}{1.978932in}}{\pgfqpoint{1.506612in}{1.986833in}}{\pgfqpoint{1.500788in}{1.992656in}}%
\pgfpathcurveto{\pgfqpoint{1.494964in}{1.998480in}}{\pgfqpoint{1.487064in}{2.001753in}}{\pgfqpoint{1.478828in}{2.001753in}}%
\pgfpathcurveto{\pgfqpoint{1.470591in}{2.001753in}}{\pgfqpoint{1.462691in}{1.998480in}}{\pgfqpoint{1.456867in}{1.992656in}}%
\pgfpathcurveto{\pgfqpoint{1.451043in}{1.986833in}}{\pgfqpoint{1.447771in}{1.978932in}}{\pgfqpoint{1.447771in}{1.970696in}}%
\pgfpathcurveto{\pgfqpoint{1.447771in}{1.962460in}}{\pgfqpoint{1.451043in}{1.954560in}}{\pgfqpoint{1.456867in}{1.948736in}}%
\pgfpathcurveto{\pgfqpoint{1.462691in}{1.942912in}}{\pgfqpoint{1.470591in}{1.939640in}}{\pgfqpoint{1.478828in}{1.939640in}}%
\pgfpathclose%
\pgfusepath{stroke,fill}%
\end{pgfscope}%
\begin{pgfscope}%
\pgfpathrectangle{\pgfqpoint{0.100000in}{0.212622in}}{\pgfqpoint{3.696000in}{3.696000in}}%
\pgfusepath{clip}%
\pgfsetbuttcap%
\pgfsetroundjoin%
\definecolor{currentfill}{rgb}{0.121569,0.466667,0.705882}%
\pgfsetfillcolor{currentfill}%
\pgfsetfillopacity{0.329702}%
\pgfsetlinewidth{1.003750pt}%
\definecolor{currentstroke}{rgb}{0.121569,0.466667,0.705882}%
\pgfsetstrokecolor{currentstroke}%
\pgfsetstrokeopacity{0.329702}%
\pgfsetdash{}{0pt}%
\pgfpathmoveto{\pgfqpoint{1.492306in}{1.947372in}}%
\pgfpathcurveto{\pgfqpoint{1.500542in}{1.947372in}}{\pgfqpoint{1.508442in}{1.950644in}}{\pgfqpoint{1.514266in}{1.956468in}}%
\pgfpathcurveto{\pgfqpoint{1.520090in}{1.962292in}}{\pgfqpoint{1.523362in}{1.970192in}}{\pgfqpoint{1.523362in}{1.978428in}}%
\pgfpathcurveto{\pgfqpoint{1.523362in}{1.986664in}}{\pgfqpoint{1.520090in}{1.994564in}}{\pgfqpoint{1.514266in}{2.000388in}}%
\pgfpathcurveto{\pgfqpoint{1.508442in}{2.006212in}}{\pgfqpoint{1.500542in}{2.009485in}}{\pgfqpoint{1.492306in}{2.009485in}}%
\pgfpathcurveto{\pgfqpoint{1.484070in}{2.009485in}}{\pgfqpoint{1.476170in}{2.006212in}}{\pgfqpoint{1.470346in}{2.000388in}}%
\pgfpathcurveto{\pgfqpoint{1.464522in}{1.994564in}}{\pgfqpoint{1.461249in}{1.986664in}}{\pgfqpoint{1.461249in}{1.978428in}}%
\pgfpathcurveto{\pgfqpoint{1.461249in}{1.970192in}}{\pgfqpoint{1.464522in}{1.962292in}}{\pgfqpoint{1.470346in}{1.956468in}}%
\pgfpathcurveto{\pgfqpoint{1.476170in}{1.950644in}}{\pgfqpoint{1.484070in}{1.947372in}}{\pgfqpoint{1.492306in}{1.947372in}}%
\pgfpathclose%
\pgfusepath{stroke,fill}%
\end{pgfscope}%
\begin{pgfscope}%
\pgfpathrectangle{\pgfqpoint{0.100000in}{0.212622in}}{\pgfqpoint{3.696000in}{3.696000in}}%
\pgfusepath{clip}%
\pgfsetbuttcap%
\pgfsetroundjoin%
\definecolor{currentfill}{rgb}{0.121569,0.466667,0.705882}%
\pgfsetfillcolor{currentfill}%
\pgfsetfillopacity{0.329955}%
\pgfsetlinewidth{1.003750pt}%
\definecolor{currentstroke}{rgb}{0.121569,0.466667,0.705882}%
\pgfsetstrokecolor{currentstroke}%
\pgfsetstrokeopacity{0.329955}%
\pgfsetdash{}{0pt}%
\pgfpathmoveto{\pgfqpoint{1.591166in}{2.022945in}}%
\pgfpathcurveto{\pgfqpoint{1.599403in}{2.022945in}}{\pgfqpoint{1.607303in}{2.026217in}}{\pgfqpoint{1.613127in}{2.032041in}}%
\pgfpathcurveto{\pgfqpoint{1.618951in}{2.037865in}}{\pgfqpoint{1.622223in}{2.045765in}}{\pgfqpoint{1.622223in}{2.054001in}}%
\pgfpathcurveto{\pgfqpoint{1.622223in}{2.062238in}}{\pgfqpoint{1.618951in}{2.070138in}}{\pgfqpoint{1.613127in}{2.075962in}}%
\pgfpathcurveto{\pgfqpoint{1.607303in}{2.081785in}}{\pgfqpoint{1.599403in}{2.085058in}}{\pgfqpoint{1.591166in}{2.085058in}}%
\pgfpathcurveto{\pgfqpoint{1.582930in}{2.085058in}}{\pgfqpoint{1.575030in}{2.081785in}}{\pgfqpoint{1.569206in}{2.075962in}}%
\pgfpathcurveto{\pgfqpoint{1.563382in}{2.070138in}}{\pgfqpoint{1.560110in}{2.062238in}}{\pgfqpoint{1.560110in}{2.054001in}}%
\pgfpathcurveto{\pgfqpoint{1.560110in}{2.045765in}}{\pgfqpoint{1.563382in}{2.037865in}}{\pgfqpoint{1.569206in}{2.032041in}}%
\pgfpathcurveto{\pgfqpoint{1.575030in}{2.026217in}}{\pgfqpoint{1.582930in}{2.022945in}}{\pgfqpoint{1.591166in}{2.022945in}}%
\pgfpathclose%
\pgfusepath{stroke,fill}%
\end{pgfscope}%
\begin{pgfscope}%
\pgfpathrectangle{\pgfqpoint{0.100000in}{0.212622in}}{\pgfqpoint{3.696000in}{3.696000in}}%
\pgfusepath{clip}%
\pgfsetbuttcap%
\pgfsetroundjoin%
\definecolor{currentfill}{rgb}{0.121569,0.466667,0.705882}%
\pgfsetfillcolor{currentfill}%
\pgfsetfillopacity{0.329978}%
\pgfsetlinewidth{1.003750pt}%
\definecolor{currentstroke}{rgb}{0.121569,0.466667,0.705882}%
\pgfsetstrokecolor{currentstroke}%
\pgfsetstrokeopacity{0.329978}%
\pgfsetdash{}{0pt}%
\pgfpathmoveto{\pgfqpoint{1.338250in}{1.848245in}}%
\pgfpathcurveto{\pgfqpoint{1.346486in}{1.848245in}}{\pgfqpoint{1.354386in}{1.851517in}}{\pgfqpoint{1.360210in}{1.857341in}}%
\pgfpathcurveto{\pgfqpoint{1.366034in}{1.863165in}}{\pgfqpoint{1.369306in}{1.871065in}}{\pgfqpoint{1.369306in}{1.879301in}}%
\pgfpathcurveto{\pgfqpoint{1.369306in}{1.887537in}}{\pgfqpoint{1.366034in}{1.895438in}}{\pgfqpoint{1.360210in}{1.901261in}}%
\pgfpathcurveto{\pgfqpoint{1.354386in}{1.907085in}}{\pgfqpoint{1.346486in}{1.910358in}}{\pgfqpoint{1.338250in}{1.910358in}}%
\pgfpathcurveto{\pgfqpoint{1.330014in}{1.910358in}}{\pgfqpoint{1.322114in}{1.907085in}}{\pgfqpoint{1.316290in}{1.901261in}}%
\pgfpathcurveto{\pgfqpoint{1.310466in}{1.895438in}}{\pgfqpoint{1.307193in}{1.887537in}}{\pgfqpoint{1.307193in}{1.879301in}}%
\pgfpathcurveto{\pgfqpoint{1.307193in}{1.871065in}}{\pgfqpoint{1.310466in}{1.863165in}}{\pgfqpoint{1.316290in}{1.857341in}}%
\pgfpathcurveto{\pgfqpoint{1.322114in}{1.851517in}}{\pgfqpoint{1.330014in}{1.848245in}}{\pgfqpoint{1.338250in}{1.848245in}}%
\pgfpathclose%
\pgfusepath{stroke,fill}%
\end{pgfscope}%
\begin{pgfscope}%
\pgfpathrectangle{\pgfqpoint{0.100000in}{0.212622in}}{\pgfqpoint{3.696000in}{3.696000in}}%
\pgfusepath{clip}%
\pgfsetbuttcap%
\pgfsetroundjoin%
\definecolor{currentfill}{rgb}{0.121569,0.466667,0.705882}%
\pgfsetfillcolor{currentfill}%
\pgfsetfillopacity{0.332896}%
\pgfsetlinewidth{1.003750pt}%
\definecolor{currentstroke}{rgb}{0.121569,0.466667,0.705882}%
\pgfsetstrokecolor{currentstroke}%
\pgfsetstrokeopacity{0.332896}%
\pgfsetdash{}{0pt}%
\pgfpathmoveto{\pgfqpoint{1.326429in}{1.840193in}}%
\pgfpathcurveto{\pgfqpoint{1.334665in}{1.840193in}}{\pgfqpoint{1.342565in}{1.843466in}}{\pgfqpoint{1.348389in}{1.849290in}}%
\pgfpathcurveto{\pgfqpoint{1.354213in}{1.855113in}}{\pgfqpoint{1.357485in}{1.863014in}}{\pgfqpoint{1.357485in}{1.871250in}}%
\pgfpathcurveto{\pgfqpoint{1.357485in}{1.879486in}}{\pgfqpoint{1.354213in}{1.887386in}}{\pgfqpoint{1.348389in}{1.893210in}}%
\pgfpathcurveto{\pgfqpoint{1.342565in}{1.899034in}}{\pgfqpoint{1.334665in}{1.902306in}}{\pgfqpoint{1.326429in}{1.902306in}}%
\pgfpathcurveto{\pgfqpoint{1.318192in}{1.902306in}}{\pgfqpoint{1.310292in}{1.899034in}}{\pgfqpoint{1.304468in}{1.893210in}}%
\pgfpathcurveto{\pgfqpoint{1.298644in}{1.887386in}}{\pgfqpoint{1.295372in}{1.879486in}}{\pgfqpoint{1.295372in}{1.871250in}}%
\pgfpathcurveto{\pgfqpoint{1.295372in}{1.863014in}}{\pgfqpoint{1.298644in}{1.855113in}}{\pgfqpoint{1.304468in}{1.849290in}}%
\pgfpathcurveto{\pgfqpoint{1.310292in}{1.843466in}}{\pgfqpoint{1.318192in}{1.840193in}}{\pgfqpoint{1.326429in}{1.840193in}}%
\pgfpathclose%
\pgfusepath{stroke,fill}%
\end{pgfscope}%
\begin{pgfscope}%
\pgfpathrectangle{\pgfqpoint{0.100000in}{0.212622in}}{\pgfqpoint{3.696000in}{3.696000in}}%
\pgfusepath{clip}%
\pgfsetbuttcap%
\pgfsetroundjoin%
\definecolor{currentfill}{rgb}{0.121569,0.466667,0.705882}%
\pgfsetfillcolor{currentfill}%
\pgfsetfillopacity{0.333344}%
\pgfsetlinewidth{1.003750pt}%
\definecolor{currentstroke}{rgb}{0.121569,0.466667,0.705882}%
\pgfsetstrokecolor{currentstroke}%
\pgfsetstrokeopacity{0.333344}%
\pgfsetdash{}{0pt}%
\pgfpathmoveto{\pgfqpoint{1.589490in}{2.018687in}}%
\pgfpathcurveto{\pgfqpoint{1.597726in}{2.018687in}}{\pgfqpoint{1.605626in}{2.021959in}}{\pgfqpoint{1.611450in}{2.027783in}}%
\pgfpathcurveto{\pgfqpoint{1.617274in}{2.033607in}}{\pgfqpoint{1.620546in}{2.041507in}}{\pgfqpoint{1.620546in}{2.049744in}}%
\pgfpathcurveto{\pgfqpoint{1.620546in}{2.057980in}}{\pgfqpoint{1.617274in}{2.065880in}}{\pgfqpoint{1.611450in}{2.071704in}}%
\pgfpathcurveto{\pgfqpoint{1.605626in}{2.077528in}}{\pgfqpoint{1.597726in}{2.080800in}}{\pgfqpoint{1.589490in}{2.080800in}}%
\pgfpathcurveto{\pgfqpoint{1.581254in}{2.080800in}}{\pgfqpoint{1.573354in}{2.077528in}}{\pgfqpoint{1.567530in}{2.071704in}}%
\pgfpathcurveto{\pgfqpoint{1.561706in}{2.065880in}}{\pgfqpoint{1.558433in}{2.057980in}}{\pgfqpoint{1.558433in}{2.049744in}}%
\pgfpathcurveto{\pgfqpoint{1.558433in}{2.041507in}}{\pgfqpoint{1.561706in}{2.033607in}}{\pgfqpoint{1.567530in}{2.027783in}}%
\pgfpathcurveto{\pgfqpoint{1.573354in}{2.021959in}}{\pgfqpoint{1.581254in}{2.018687in}}{\pgfqpoint{1.589490in}{2.018687in}}%
\pgfpathclose%
\pgfusepath{stroke,fill}%
\end{pgfscope}%
\begin{pgfscope}%
\pgfpathrectangle{\pgfqpoint{0.100000in}{0.212622in}}{\pgfqpoint{3.696000in}{3.696000in}}%
\pgfusepath{clip}%
\pgfsetbuttcap%
\pgfsetroundjoin%
\definecolor{currentfill}{rgb}{0.121569,0.466667,0.705882}%
\pgfsetfillcolor{currentfill}%
\pgfsetfillopacity{0.334773}%
\pgfsetlinewidth{1.003750pt}%
\definecolor{currentstroke}{rgb}{0.121569,0.466667,0.705882}%
\pgfsetstrokecolor{currentstroke}%
\pgfsetstrokeopacity{0.334773}%
\pgfsetdash{}{0pt}%
\pgfpathmoveto{\pgfqpoint{1.597206in}{2.028711in}}%
\pgfpathcurveto{\pgfqpoint{1.605442in}{2.028711in}}{\pgfqpoint{1.613342in}{2.031983in}}{\pgfqpoint{1.619166in}{2.037807in}}%
\pgfpathcurveto{\pgfqpoint{1.624990in}{2.043631in}}{\pgfqpoint{1.628262in}{2.051531in}}{\pgfqpoint{1.628262in}{2.059767in}}%
\pgfpathcurveto{\pgfqpoint{1.628262in}{2.068003in}}{\pgfqpoint{1.624990in}{2.075903in}}{\pgfqpoint{1.619166in}{2.081727in}}%
\pgfpathcurveto{\pgfqpoint{1.613342in}{2.087551in}}{\pgfqpoint{1.605442in}{2.090824in}}{\pgfqpoint{1.597206in}{2.090824in}}%
\pgfpathcurveto{\pgfqpoint{1.588970in}{2.090824in}}{\pgfqpoint{1.581070in}{2.087551in}}{\pgfqpoint{1.575246in}{2.081727in}}%
\pgfpathcurveto{\pgfqpoint{1.569422in}{2.075903in}}{\pgfqpoint{1.566149in}{2.068003in}}{\pgfqpoint{1.566149in}{2.059767in}}%
\pgfpathcurveto{\pgfqpoint{1.566149in}{2.051531in}}{\pgfqpoint{1.569422in}{2.043631in}}{\pgfqpoint{1.575246in}{2.037807in}}%
\pgfpathcurveto{\pgfqpoint{1.581070in}{2.031983in}}{\pgfqpoint{1.588970in}{2.028711in}}{\pgfqpoint{1.597206in}{2.028711in}}%
\pgfpathclose%
\pgfusepath{stroke,fill}%
\end{pgfscope}%
\begin{pgfscope}%
\pgfpathrectangle{\pgfqpoint{0.100000in}{0.212622in}}{\pgfqpoint{3.696000in}{3.696000in}}%
\pgfusepath{clip}%
\pgfsetbuttcap%
\pgfsetroundjoin%
\definecolor{currentfill}{rgb}{0.121569,0.466667,0.705882}%
\pgfsetfillcolor{currentfill}%
\pgfsetfillopacity{0.340393}%
\pgfsetlinewidth{1.003750pt}%
\definecolor{currentstroke}{rgb}{0.121569,0.466667,0.705882}%
\pgfsetstrokecolor{currentstroke}%
\pgfsetstrokeopacity{0.340393}%
\pgfsetdash{}{0pt}%
\pgfpathmoveto{\pgfqpoint{1.300041in}{1.817492in}}%
\pgfpathcurveto{\pgfqpoint{1.308277in}{1.817492in}}{\pgfqpoint{1.316177in}{1.820764in}}{\pgfqpoint{1.322001in}{1.826588in}}%
\pgfpathcurveto{\pgfqpoint{1.327825in}{1.832412in}}{\pgfqpoint{1.331097in}{1.840312in}}{\pgfqpoint{1.331097in}{1.848548in}}%
\pgfpathcurveto{\pgfqpoint{1.331097in}{1.856784in}}{\pgfqpoint{1.327825in}{1.864685in}}{\pgfqpoint{1.322001in}{1.870508in}}%
\pgfpathcurveto{\pgfqpoint{1.316177in}{1.876332in}}{\pgfqpoint{1.308277in}{1.879605in}}{\pgfqpoint{1.300041in}{1.879605in}}%
\pgfpathcurveto{\pgfqpoint{1.291805in}{1.879605in}}{\pgfqpoint{1.283905in}{1.876332in}}{\pgfqpoint{1.278081in}{1.870508in}}%
\pgfpathcurveto{\pgfqpoint{1.272257in}{1.864685in}}{\pgfqpoint{1.268984in}{1.856784in}}{\pgfqpoint{1.268984in}{1.848548in}}%
\pgfpathcurveto{\pgfqpoint{1.268984in}{1.840312in}}{\pgfqpoint{1.272257in}{1.832412in}}{\pgfqpoint{1.278081in}{1.826588in}}%
\pgfpathcurveto{\pgfqpoint{1.283905in}{1.820764in}}{\pgfqpoint{1.291805in}{1.817492in}}{\pgfqpoint{1.300041in}{1.817492in}}%
\pgfpathclose%
\pgfusepath{stroke,fill}%
\end{pgfscope}%
\begin{pgfscope}%
\pgfpathrectangle{\pgfqpoint{0.100000in}{0.212622in}}{\pgfqpoint{3.696000in}{3.696000in}}%
\pgfusepath{clip}%
\pgfsetbuttcap%
\pgfsetroundjoin%
\definecolor{currentfill}{rgb}{0.121569,0.466667,0.705882}%
\pgfsetfillcolor{currentfill}%
\pgfsetfillopacity{0.343837}%
\pgfsetlinewidth{1.003750pt}%
\definecolor{currentstroke}{rgb}{0.121569,0.466667,0.705882}%
\pgfsetstrokecolor{currentstroke}%
\pgfsetstrokeopacity{0.343837}%
\pgfsetdash{}{0pt}%
\pgfpathmoveto{\pgfqpoint{1.579095in}{2.008860in}}%
\pgfpathcurveto{\pgfqpoint{1.587331in}{2.008860in}}{\pgfqpoint{1.595231in}{2.012132in}}{\pgfqpoint{1.601055in}{2.017956in}}%
\pgfpathcurveto{\pgfqpoint{1.606879in}{2.023780in}}{\pgfqpoint{1.610151in}{2.031680in}}{\pgfqpoint{1.610151in}{2.039916in}}%
\pgfpathcurveto{\pgfqpoint{1.610151in}{2.048153in}}{\pgfqpoint{1.606879in}{2.056053in}}{\pgfqpoint{1.601055in}{2.061877in}}%
\pgfpathcurveto{\pgfqpoint{1.595231in}{2.067701in}}{\pgfqpoint{1.587331in}{2.070973in}}{\pgfqpoint{1.579095in}{2.070973in}}%
\pgfpathcurveto{\pgfqpoint{1.570859in}{2.070973in}}{\pgfqpoint{1.562958in}{2.067701in}}{\pgfqpoint{1.557135in}{2.061877in}}%
\pgfpathcurveto{\pgfqpoint{1.551311in}{2.056053in}}{\pgfqpoint{1.548038in}{2.048153in}}{\pgfqpoint{1.548038in}{2.039916in}}%
\pgfpathcurveto{\pgfqpoint{1.548038in}{2.031680in}}{\pgfqpoint{1.551311in}{2.023780in}}{\pgfqpoint{1.557135in}{2.017956in}}%
\pgfpathcurveto{\pgfqpoint{1.562958in}{2.012132in}}{\pgfqpoint{1.570859in}{2.008860in}}{\pgfqpoint{1.579095in}{2.008860in}}%
\pgfpathclose%
\pgfusepath{stroke,fill}%
\end{pgfscope}%
\begin{pgfscope}%
\pgfpathrectangle{\pgfqpoint{0.100000in}{0.212622in}}{\pgfqpoint{3.696000in}{3.696000in}}%
\pgfusepath{clip}%
\pgfsetbuttcap%
\pgfsetroundjoin%
\definecolor{currentfill}{rgb}{0.121569,0.466667,0.705882}%
\pgfsetfillcolor{currentfill}%
\pgfsetfillopacity{0.345434}%
\pgfsetlinewidth{1.003750pt}%
\definecolor{currentstroke}{rgb}{0.121569,0.466667,0.705882}%
\pgfsetstrokecolor{currentstroke}%
\pgfsetstrokeopacity{0.345434}%
\pgfsetdash{}{0pt}%
\pgfpathmoveto{\pgfqpoint{1.599717in}{2.029245in}}%
\pgfpathcurveto{\pgfqpoint{1.607953in}{2.029245in}}{\pgfqpoint{1.615853in}{2.032517in}}{\pgfqpoint{1.621677in}{2.038341in}}%
\pgfpathcurveto{\pgfqpoint{1.627501in}{2.044165in}}{\pgfqpoint{1.630773in}{2.052065in}}{\pgfqpoint{1.630773in}{2.060302in}}%
\pgfpathcurveto{\pgfqpoint{1.630773in}{2.068538in}}{\pgfqpoint{1.627501in}{2.076438in}}{\pgfqpoint{1.621677in}{2.082262in}}%
\pgfpathcurveto{\pgfqpoint{1.615853in}{2.088086in}}{\pgfqpoint{1.607953in}{2.091358in}}{\pgfqpoint{1.599717in}{2.091358in}}%
\pgfpathcurveto{\pgfqpoint{1.591480in}{2.091358in}}{\pgfqpoint{1.583580in}{2.088086in}}{\pgfqpoint{1.577756in}{2.082262in}}%
\pgfpathcurveto{\pgfqpoint{1.571932in}{2.076438in}}{\pgfqpoint{1.568660in}{2.068538in}}{\pgfqpoint{1.568660in}{2.060302in}}%
\pgfpathcurveto{\pgfqpoint{1.568660in}{2.052065in}}{\pgfqpoint{1.571932in}{2.044165in}}{\pgfqpoint{1.577756in}{2.038341in}}%
\pgfpathcurveto{\pgfqpoint{1.583580in}{2.032517in}}{\pgfqpoint{1.591480in}{2.029245in}}{\pgfqpoint{1.599717in}{2.029245in}}%
\pgfpathclose%
\pgfusepath{stroke,fill}%
\end{pgfscope}%
\begin{pgfscope}%
\pgfpathrectangle{\pgfqpoint{0.100000in}{0.212622in}}{\pgfqpoint{3.696000in}{3.696000in}}%
\pgfusepath{clip}%
\pgfsetbuttcap%
\pgfsetroundjoin%
\definecolor{currentfill}{rgb}{0.121569,0.466667,0.705882}%
\pgfsetfillcolor{currentfill}%
\pgfsetfillopacity{0.347379}%
\pgfsetlinewidth{1.003750pt}%
\definecolor{currentstroke}{rgb}{0.121569,0.466667,0.705882}%
\pgfsetstrokecolor{currentstroke}%
\pgfsetstrokeopacity{0.347379}%
\pgfsetdash{}{0pt}%
\pgfpathmoveto{\pgfqpoint{1.606745in}{2.033322in}}%
\pgfpathcurveto{\pgfqpoint{1.614982in}{2.033322in}}{\pgfqpoint{1.622882in}{2.036594in}}{\pgfqpoint{1.628706in}{2.042418in}}%
\pgfpathcurveto{\pgfqpoint{1.634529in}{2.048242in}}{\pgfqpoint{1.637802in}{2.056142in}}{\pgfqpoint{1.637802in}{2.064378in}}%
\pgfpathcurveto{\pgfqpoint{1.637802in}{2.072614in}}{\pgfqpoint{1.634529in}{2.080514in}}{\pgfqpoint{1.628706in}{2.086338in}}%
\pgfpathcurveto{\pgfqpoint{1.622882in}{2.092162in}}{\pgfqpoint{1.614982in}{2.095435in}}{\pgfqpoint{1.606745in}{2.095435in}}%
\pgfpathcurveto{\pgfqpoint{1.598509in}{2.095435in}}{\pgfqpoint{1.590609in}{2.092162in}}{\pgfqpoint{1.584785in}{2.086338in}}%
\pgfpathcurveto{\pgfqpoint{1.578961in}{2.080514in}}{\pgfqpoint{1.575689in}{2.072614in}}{\pgfqpoint{1.575689in}{2.064378in}}%
\pgfpathcurveto{\pgfqpoint{1.575689in}{2.056142in}}{\pgfqpoint{1.578961in}{2.048242in}}{\pgfqpoint{1.584785in}{2.042418in}}%
\pgfpathcurveto{\pgfqpoint{1.590609in}{2.036594in}}{\pgfqpoint{1.598509in}{2.033322in}}{\pgfqpoint{1.606745in}{2.033322in}}%
\pgfpathclose%
\pgfusepath{stroke,fill}%
\end{pgfscope}%
\begin{pgfscope}%
\pgfpathrectangle{\pgfqpoint{0.100000in}{0.212622in}}{\pgfqpoint{3.696000in}{3.696000in}}%
\pgfusepath{clip}%
\pgfsetbuttcap%
\pgfsetroundjoin%
\definecolor{currentfill}{rgb}{0.121569,0.466667,0.705882}%
\pgfsetfillcolor{currentfill}%
\pgfsetfillopacity{0.350054}%
\pgfsetlinewidth{1.003750pt}%
\definecolor{currentstroke}{rgb}{0.121569,0.466667,0.705882}%
\pgfsetstrokecolor{currentstroke}%
\pgfsetstrokeopacity{0.350054}%
\pgfsetdash{}{0pt}%
\pgfpathmoveto{\pgfqpoint{1.572724in}{2.005370in}}%
\pgfpathcurveto{\pgfqpoint{1.580960in}{2.005370in}}{\pgfqpoint{1.588860in}{2.008643in}}{\pgfqpoint{1.594684in}{2.014467in}}%
\pgfpathcurveto{\pgfqpoint{1.600508in}{2.020290in}}{\pgfqpoint{1.603781in}{2.028190in}}{\pgfqpoint{1.603781in}{2.036427in}}%
\pgfpathcurveto{\pgfqpoint{1.603781in}{2.044663in}}{\pgfqpoint{1.600508in}{2.052563in}}{\pgfqpoint{1.594684in}{2.058387in}}%
\pgfpathcurveto{\pgfqpoint{1.588860in}{2.064211in}}{\pgfqpoint{1.580960in}{2.067483in}}{\pgfqpoint{1.572724in}{2.067483in}}%
\pgfpathcurveto{\pgfqpoint{1.564488in}{2.067483in}}{\pgfqpoint{1.556588in}{2.064211in}}{\pgfqpoint{1.550764in}{2.058387in}}%
\pgfpathcurveto{\pgfqpoint{1.544940in}{2.052563in}}{\pgfqpoint{1.541668in}{2.044663in}}{\pgfqpoint{1.541668in}{2.036427in}}%
\pgfpathcurveto{\pgfqpoint{1.541668in}{2.028190in}}{\pgfqpoint{1.544940in}{2.020290in}}{\pgfqpoint{1.550764in}{2.014467in}}%
\pgfpathcurveto{\pgfqpoint{1.556588in}{2.008643in}}{\pgfqpoint{1.564488in}{2.005370in}}{\pgfqpoint{1.572724in}{2.005370in}}%
\pgfpathclose%
\pgfusepath{stroke,fill}%
\end{pgfscope}%
\begin{pgfscope}%
\pgfpathrectangle{\pgfqpoint{0.100000in}{0.212622in}}{\pgfqpoint{3.696000in}{3.696000in}}%
\pgfusepath{clip}%
\pgfsetbuttcap%
\pgfsetroundjoin%
\definecolor{currentfill}{rgb}{0.121569,0.466667,0.705882}%
\pgfsetfillcolor{currentfill}%
\pgfsetfillopacity{0.356019}%
\pgfsetlinewidth{1.003750pt}%
\definecolor{currentstroke}{rgb}{0.121569,0.466667,0.705882}%
\pgfsetstrokecolor{currentstroke}%
\pgfsetstrokeopacity{0.356019}%
\pgfsetdash{}{0pt}%
\pgfpathmoveto{\pgfqpoint{1.592582in}{2.014519in}}%
\pgfpathcurveto{\pgfqpoint{1.600819in}{2.014519in}}{\pgfqpoint{1.608719in}{2.017791in}}{\pgfqpoint{1.614543in}{2.023615in}}%
\pgfpathcurveto{\pgfqpoint{1.620366in}{2.029439in}}{\pgfqpoint{1.623639in}{2.037339in}}{\pgfqpoint{1.623639in}{2.045576in}}%
\pgfpathcurveto{\pgfqpoint{1.623639in}{2.053812in}}{\pgfqpoint{1.620366in}{2.061712in}}{\pgfqpoint{1.614543in}{2.067536in}}%
\pgfpathcurveto{\pgfqpoint{1.608719in}{2.073360in}}{\pgfqpoint{1.600819in}{2.076632in}}{\pgfqpoint{1.592582in}{2.076632in}}%
\pgfpathcurveto{\pgfqpoint{1.584346in}{2.076632in}}{\pgfqpoint{1.576446in}{2.073360in}}{\pgfqpoint{1.570622in}{2.067536in}}%
\pgfpathcurveto{\pgfqpoint{1.564798in}{2.061712in}}{\pgfqpoint{1.561526in}{2.053812in}}{\pgfqpoint{1.561526in}{2.045576in}}%
\pgfpathcurveto{\pgfqpoint{1.561526in}{2.037339in}}{\pgfqpoint{1.564798in}{2.029439in}}{\pgfqpoint{1.570622in}{2.023615in}}%
\pgfpathcurveto{\pgfqpoint{1.576446in}{2.017791in}}{\pgfqpoint{1.584346in}{2.014519in}}{\pgfqpoint{1.592582in}{2.014519in}}%
\pgfpathclose%
\pgfusepath{stroke,fill}%
\end{pgfscope}%
\begin{pgfscope}%
\pgfpathrectangle{\pgfqpoint{0.100000in}{0.212622in}}{\pgfqpoint{3.696000in}{3.696000in}}%
\pgfusepath{clip}%
\pgfsetbuttcap%
\pgfsetroundjoin%
\definecolor{currentfill}{rgb}{0.121569,0.466667,0.705882}%
\pgfsetfillcolor{currentfill}%
\pgfsetfillopacity{0.356028}%
\pgfsetlinewidth{1.003750pt}%
\definecolor{currentstroke}{rgb}{0.121569,0.466667,0.705882}%
\pgfsetstrokecolor{currentstroke}%
\pgfsetstrokeopacity{0.356028}%
\pgfsetdash{}{0pt}%
\pgfpathmoveto{\pgfqpoint{1.246152in}{1.771927in}}%
\pgfpathcurveto{\pgfqpoint{1.254388in}{1.771927in}}{\pgfqpoint{1.262288in}{1.775199in}}{\pgfqpoint{1.268112in}{1.781023in}}%
\pgfpathcurveto{\pgfqpoint{1.273936in}{1.786847in}}{\pgfqpoint{1.277208in}{1.794747in}}{\pgfqpoint{1.277208in}{1.802983in}}%
\pgfpathcurveto{\pgfqpoint{1.277208in}{1.811220in}}{\pgfqpoint{1.273936in}{1.819120in}}{\pgfqpoint{1.268112in}{1.824944in}}%
\pgfpathcurveto{\pgfqpoint{1.262288in}{1.830768in}}{\pgfqpoint{1.254388in}{1.834040in}}{\pgfqpoint{1.246152in}{1.834040in}}%
\pgfpathcurveto{\pgfqpoint{1.237916in}{1.834040in}}{\pgfqpoint{1.230016in}{1.830768in}}{\pgfqpoint{1.224192in}{1.824944in}}%
\pgfpathcurveto{\pgfqpoint{1.218368in}{1.819120in}}{\pgfqpoint{1.215095in}{1.811220in}}{\pgfqpoint{1.215095in}{1.802983in}}%
\pgfpathcurveto{\pgfqpoint{1.215095in}{1.794747in}}{\pgfqpoint{1.218368in}{1.786847in}}{\pgfqpoint{1.224192in}{1.781023in}}%
\pgfpathcurveto{\pgfqpoint{1.230016in}{1.775199in}}{\pgfqpoint{1.237916in}{1.771927in}}{\pgfqpoint{1.246152in}{1.771927in}}%
\pgfpathclose%
\pgfusepath{stroke,fill}%
\end{pgfscope}%
\begin{pgfscope}%
\pgfpathrectangle{\pgfqpoint{0.100000in}{0.212622in}}{\pgfqpoint{3.696000in}{3.696000in}}%
\pgfusepath{clip}%
\pgfsetbuttcap%
\pgfsetroundjoin%
\definecolor{currentfill}{rgb}{0.121569,0.466667,0.705882}%
\pgfsetfillcolor{currentfill}%
\pgfsetfillopacity{0.365566}%
\pgfsetlinewidth{1.003750pt}%
\definecolor{currentstroke}{rgb}{0.121569,0.466667,0.705882}%
\pgfsetstrokecolor{currentstroke}%
\pgfsetstrokeopacity{0.365566}%
\pgfsetdash{}{0pt}%
\pgfpathmoveto{\pgfqpoint{1.209240in}{1.743746in}}%
\pgfpathcurveto{\pgfqpoint{1.217476in}{1.743746in}}{\pgfqpoint{1.225376in}{1.747018in}}{\pgfqpoint{1.231200in}{1.752842in}}%
\pgfpathcurveto{\pgfqpoint{1.237024in}{1.758666in}}{\pgfqpoint{1.240296in}{1.766566in}}{\pgfqpoint{1.240296in}{1.774803in}}%
\pgfpathcurveto{\pgfqpoint{1.240296in}{1.783039in}}{\pgfqpoint{1.237024in}{1.790939in}}{\pgfqpoint{1.231200in}{1.796763in}}%
\pgfpathcurveto{\pgfqpoint{1.225376in}{1.802587in}}{\pgfqpoint{1.217476in}{1.805859in}}{\pgfqpoint{1.209240in}{1.805859in}}%
\pgfpathcurveto{\pgfqpoint{1.201003in}{1.805859in}}{\pgfqpoint{1.193103in}{1.802587in}}{\pgfqpoint{1.187279in}{1.796763in}}%
\pgfpathcurveto{\pgfqpoint{1.181456in}{1.790939in}}{\pgfqpoint{1.178183in}{1.783039in}}{\pgfqpoint{1.178183in}{1.774803in}}%
\pgfpathcurveto{\pgfqpoint{1.178183in}{1.766566in}}{\pgfqpoint{1.181456in}{1.758666in}}{\pgfqpoint{1.187279in}{1.752842in}}%
\pgfpathcurveto{\pgfqpoint{1.193103in}{1.747018in}}{\pgfqpoint{1.201003in}{1.743746in}}{\pgfqpoint{1.209240in}{1.743746in}}%
\pgfpathclose%
\pgfusepath{stroke,fill}%
\end{pgfscope}%
\begin{pgfscope}%
\pgfpathrectangle{\pgfqpoint{0.100000in}{0.212622in}}{\pgfqpoint{3.696000in}{3.696000in}}%
\pgfusepath{clip}%
\pgfsetbuttcap%
\pgfsetroundjoin%
\definecolor{currentfill}{rgb}{0.121569,0.466667,0.705882}%
\pgfsetfillcolor{currentfill}%
\pgfsetfillopacity{0.365567}%
\pgfsetlinewidth{1.003750pt}%
\definecolor{currentstroke}{rgb}{0.121569,0.466667,0.705882}%
\pgfsetstrokecolor{currentstroke}%
\pgfsetstrokeopacity{0.365567}%
\pgfsetdash{}{0pt}%
\pgfpathmoveto{\pgfqpoint{1.209238in}{1.743743in}}%
\pgfpathcurveto{\pgfqpoint{1.217474in}{1.743743in}}{\pgfqpoint{1.225374in}{1.747016in}}{\pgfqpoint{1.231198in}{1.752840in}}%
\pgfpathcurveto{\pgfqpoint{1.237022in}{1.758664in}}{\pgfqpoint{1.240294in}{1.766564in}}{\pgfqpoint{1.240294in}{1.774800in}}%
\pgfpathcurveto{\pgfqpoint{1.240294in}{1.783036in}}{\pgfqpoint{1.237022in}{1.790936in}}{\pgfqpoint{1.231198in}{1.796760in}}%
\pgfpathcurveto{\pgfqpoint{1.225374in}{1.802584in}}{\pgfqpoint{1.217474in}{1.805856in}}{\pgfqpoint{1.209238in}{1.805856in}}%
\pgfpathcurveto{\pgfqpoint{1.201001in}{1.805856in}}{\pgfqpoint{1.193101in}{1.802584in}}{\pgfqpoint{1.187277in}{1.796760in}}%
\pgfpathcurveto{\pgfqpoint{1.181453in}{1.790936in}}{\pgfqpoint{1.178181in}{1.783036in}}{\pgfqpoint{1.178181in}{1.774800in}}%
\pgfpathcurveto{\pgfqpoint{1.178181in}{1.766564in}}{\pgfqpoint{1.181453in}{1.758664in}}{\pgfqpoint{1.187277in}{1.752840in}}%
\pgfpathcurveto{\pgfqpoint{1.193101in}{1.747016in}}{\pgfqpoint{1.201001in}{1.743743in}}{\pgfqpoint{1.209238in}{1.743743in}}%
\pgfpathclose%
\pgfusepath{stroke,fill}%
\end{pgfscope}%
\begin{pgfscope}%
\pgfpathrectangle{\pgfqpoint{0.100000in}{0.212622in}}{\pgfqpoint{3.696000in}{3.696000in}}%
\pgfusepath{clip}%
\pgfsetbuttcap%
\pgfsetroundjoin%
\definecolor{currentfill}{rgb}{0.121569,0.466667,0.705882}%
\pgfsetfillcolor{currentfill}%
\pgfsetfillopacity{0.365568}%
\pgfsetlinewidth{1.003750pt}%
\definecolor{currentstroke}{rgb}{0.121569,0.466667,0.705882}%
\pgfsetstrokecolor{currentstroke}%
\pgfsetstrokeopacity{0.365568}%
\pgfsetdash{}{0pt}%
\pgfpathmoveto{\pgfqpoint{1.209238in}{1.743743in}}%
\pgfpathcurveto{\pgfqpoint{1.217474in}{1.743743in}}{\pgfqpoint{1.225374in}{1.747015in}}{\pgfqpoint{1.231198in}{1.752839in}}%
\pgfpathcurveto{\pgfqpoint{1.237022in}{1.758663in}}{\pgfqpoint{1.240294in}{1.766563in}}{\pgfqpoint{1.240294in}{1.774800in}}%
\pgfpathcurveto{\pgfqpoint{1.240294in}{1.783036in}}{\pgfqpoint{1.237022in}{1.790936in}}{\pgfqpoint{1.231198in}{1.796760in}}%
\pgfpathcurveto{\pgfqpoint{1.225374in}{1.802584in}}{\pgfqpoint{1.217474in}{1.805856in}}{\pgfqpoint{1.209238in}{1.805856in}}%
\pgfpathcurveto{\pgfqpoint{1.201002in}{1.805856in}}{\pgfqpoint{1.193102in}{1.802584in}}{\pgfqpoint{1.187278in}{1.796760in}}%
\pgfpathcurveto{\pgfqpoint{1.181454in}{1.790936in}}{\pgfqpoint{1.178181in}{1.783036in}}{\pgfqpoint{1.178181in}{1.774800in}}%
\pgfpathcurveto{\pgfqpoint{1.178181in}{1.766563in}}{\pgfqpoint{1.181454in}{1.758663in}}{\pgfqpoint{1.187278in}{1.752839in}}%
\pgfpathcurveto{\pgfqpoint{1.193102in}{1.747015in}}{\pgfqpoint{1.201002in}{1.743743in}}{\pgfqpoint{1.209238in}{1.743743in}}%
\pgfpathclose%
\pgfusepath{stroke,fill}%
\end{pgfscope}%
\begin{pgfscope}%
\pgfpathrectangle{\pgfqpoint{0.100000in}{0.212622in}}{\pgfqpoint{3.696000in}{3.696000in}}%
\pgfusepath{clip}%
\pgfsetbuttcap%
\pgfsetroundjoin%
\definecolor{currentfill}{rgb}{0.121569,0.466667,0.705882}%
\pgfsetfillcolor{currentfill}%
\pgfsetfillopacity{0.365568}%
\pgfsetlinewidth{1.003750pt}%
\definecolor{currentstroke}{rgb}{0.121569,0.466667,0.705882}%
\pgfsetstrokecolor{currentstroke}%
\pgfsetstrokeopacity{0.365568}%
\pgfsetdash{}{0pt}%
\pgfpathmoveto{\pgfqpoint{1.209238in}{1.743743in}}%
\pgfpathcurveto{\pgfqpoint{1.217474in}{1.743743in}}{\pgfqpoint{1.225374in}{1.747015in}}{\pgfqpoint{1.231198in}{1.752839in}}%
\pgfpathcurveto{\pgfqpoint{1.237022in}{1.758663in}}{\pgfqpoint{1.240294in}{1.766563in}}{\pgfqpoint{1.240294in}{1.774799in}}%
\pgfpathcurveto{\pgfqpoint{1.240294in}{1.783035in}}{\pgfqpoint{1.237022in}{1.790935in}}{\pgfqpoint{1.231198in}{1.796759in}}%
\pgfpathcurveto{\pgfqpoint{1.225374in}{1.802583in}}{\pgfqpoint{1.217474in}{1.805856in}}{\pgfqpoint{1.209238in}{1.805856in}}%
\pgfpathcurveto{\pgfqpoint{1.201001in}{1.805856in}}{\pgfqpoint{1.193101in}{1.802583in}}{\pgfqpoint{1.187277in}{1.796759in}}%
\pgfpathcurveto{\pgfqpoint{1.181453in}{1.790935in}}{\pgfqpoint{1.178181in}{1.783035in}}{\pgfqpoint{1.178181in}{1.774799in}}%
\pgfpathcurveto{\pgfqpoint{1.178181in}{1.766563in}}{\pgfqpoint{1.181453in}{1.758663in}}{\pgfqpoint{1.187277in}{1.752839in}}%
\pgfpathcurveto{\pgfqpoint{1.193101in}{1.747015in}}{\pgfqpoint{1.201001in}{1.743743in}}{\pgfqpoint{1.209238in}{1.743743in}}%
\pgfpathclose%
\pgfusepath{stroke,fill}%
\end{pgfscope}%
\begin{pgfscope}%
\pgfpathrectangle{\pgfqpoint{0.100000in}{0.212622in}}{\pgfqpoint{3.696000in}{3.696000in}}%
\pgfusepath{clip}%
\pgfsetbuttcap%
\pgfsetroundjoin%
\definecolor{currentfill}{rgb}{0.121569,0.466667,0.705882}%
\pgfsetfillcolor{currentfill}%
\pgfsetfillopacity{0.365569}%
\pgfsetlinewidth{1.003750pt}%
\definecolor{currentstroke}{rgb}{0.121569,0.466667,0.705882}%
\pgfsetstrokecolor{currentstroke}%
\pgfsetstrokeopacity{0.365569}%
\pgfsetdash{}{0pt}%
\pgfpathmoveto{\pgfqpoint{1.209237in}{1.743742in}}%
\pgfpathcurveto{\pgfqpoint{1.217473in}{1.743742in}}{\pgfqpoint{1.225373in}{1.747014in}}{\pgfqpoint{1.231197in}{1.752838in}}%
\pgfpathcurveto{\pgfqpoint{1.237021in}{1.758662in}}{\pgfqpoint{1.240293in}{1.766562in}}{\pgfqpoint{1.240293in}{1.774798in}}%
\pgfpathcurveto{\pgfqpoint{1.240293in}{1.783034in}}{\pgfqpoint{1.237021in}{1.790934in}}{\pgfqpoint{1.231197in}{1.796758in}}%
\pgfpathcurveto{\pgfqpoint{1.225373in}{1.802582in}}{\pgfqpoint{1.217473in}{1.805855in}}{\pgfqpoint{1.209237in}{1.805855in}}%
\pgfpathcurveto{\pgfqpoint{1.201000in}{1.805855in}}{\pgfqpoint{1.193100in}{1.802582in}}{\pgfqpoint{1.187276in}{1.796758in}}%
\pgfpathcurveto{\pgfqpoint{1.181453in}{1.790934in}}{\pgfqpoint{1.178180in}{1.783034in}}{\pgfqpoint{1.178180in}{1.774798in}}%
\pgfpathcurveto{\pgfqpoint{1.178180in}{1.766562in}}{\pgfqpoint{1.181453in}{1.758662in}}{\pgfqpoint{1.187276in}{1.752838in}}%
\pgfpathcurveto{\pgfqpoint{1.193100in}{1.747014in}}{\pgfqpoint{1.201000in}{1.743742in}}{\pgfqpoint{1.209237in}{1.743742in}}%
\pgfpathclose%
\pgfusepath{stroke,fill}%
\end{pgfscope}%
\begin{pgfscope}%
\pgfpathrectangle{\pgfqpoint{0.100000in}{0.212622in}}{\pgfqpoint{3.696000in}{3.696000in}}%
\pgfusepath{clip}%
\pgfsetbuttcap%
\pgfsetroundjoin%
\definecolor{currentfill}{rgb}{0.121569,0.466667,0.705882}%
\pgfsetfillcolor{currentfill}%
\pgfsetfillopacity{0.365569}%
\pgfsetlinewidth{1.003750pt}%
\definecolor{currentstroke}{rgb}{0.121569,0.466667,0.705882}%
\pgfsetstrokecolor{currentstroke}%
\pgfsetstrokeopacity{0.365569}%
\pgfsetdash{}{0pt}%
\pgfpathmoveto{\pgfqpoint{1.209237in}{1.743742in}}%
\pgfpathcurveto{\pgfqpoint{1.217473in}{1.743742in}}{\pgfqpoint{1.225373in}{1.747014in}}{\pgfqpoint{1.231197in}{1.752838in}}%
\pgfpathcurveto{\pgfqpoint{1.237021in}{1.758662in}}{\pgfqpoint{1.240293in}{1.766562in}}{\pgfqpoint{1.240293in}{1.774798in}}%
\pgfpathcurveto{\pgfqpoint{1.240293in}{1.783034in}}{\pgfqpoint{1.237021in}{1.790935in}}{\pgfqpoint{1.231197in}{1.796758in}}%
\pgfpathcurveto{\pgfqpoint{1.225373in}{1.802582in}}{\pgfqpoint{1.217473in}{1.805855in}}{\pgfqpoint{1.209237in}{1.805855in}}%
\pgfpathcurveto{\pgfqpoint{1.201001in}{1.805855in}}{\pgfqpoint{1.193101in}{1.802582in}}{\pgfqpoint{1.187277in}{1.796758in}}%
\pgfpathcurveto{\pgfqpoint{1.181453in}{1.790935in}}{\pgfqpoint{1.178180in}{1.783034in}}{\pgfqpoint{1.178180in}{1.774798in}}%
\pgfpathcurveto{\pgfqpoint{1.178180in}{1.766562in}}{\pgfqpoint{1.181453in}{1.758662in}}{\pgfqpoint{1.187277in}{1.752838in}}%
\pgfpathcurveto{\pgfqpoint{1.193101in}{1.747014in}}{\pgfqpoint{1.201001in}{1.743742in}}{\pgfqpoint{1.209237in}{1.743742in}}%
\pgfpathclose%
\pgfusepath{stroke,fill}%
\end{pgfscope}%
\begin{pgfscope}%
\pgfpathrectangle{\pgfqpoint{0.100000in}{0.212622in}}{\pgfqpoint{3.696000in}{3.696000in}}%
\pgfusepath{clip}%
\pgfsetbuttcap%
\pgfsetroundjoin%
\definecolor{currentfill}{rgb}{0.121569,0.466667,0.705882}%
\pgfsetfillcolor{currentfill}%
\pgfsetfillopacity{0.365569}%
\pgfsetlinewidth{1.003750pt}%
\definecolor{currentstroke}{rgb}{0.121569,0.466667,0.705882}%
\pgfsetstrokecolor{currentstroke}%
\pgfsetstrokeopacity{0.365569}%
\pgfsetdash{}{0pt}%
\pgfpathmoveto{\pgfqpoint{1.209237in}{1.743742in}}%
\pgfpathcurveto{\pgfqpoint{1.217473in}{1.743742in}}{\pgfqpoint{1.225373in}{1.747014in}}{\pgfqpoint{1.231197in}{1.752838in}}%
\pgfpathcurveto{\pgfqpoint{1.237021in}{1.758662in}}{\pgfqpoint{1.240293in}{1.766562in}}{\pgfqpoint{1.240293in}{1.774798in}}%
\pgfpathcurveto{\pgfqpoint{1.240293in}{1.783034in}}{\pgfqpoint{1.237021in}{1.790934in}}{\pgfqpoint{1.231197in}{1.796758in}}%
\pgfpathcurveto{\pgfqpoint{1.225373in}{1.802582in}}{\pgfqpoint{1.217473in}{1.805855in}}{\pgfqpoint{1.209237in}{1.805855in}}%
\pgfpathcurveto{\pgfqpoint{1.201000in}{1.805855in}}{\pgfqpoint{1.193100in}{1.802582in}}{\pgfqpoint{1.187277in}{1.796758in}}%
\pgfpathcurveto{\pgfqpoint{1.181453in}{1.790934in}}{\pgfqpoint{1.178180in}{1.783034in}}{\pgfqpoint{1.178180in}{1.774798in}}%
\pgfpathcurveto{\pgfqpoint{1.178180in}{1.766562in}}{\pgfqpoint{1.181453in}{1.758662in}}{\pgfqpoint{1.187277in}{1.752838in}}%
\pgfpathcurveto{\pgfqpoint{1.193100in}{1.747014in}}{\pgfqpoint{1.201000in}{1.743742in}}{\pgfqpoint{1.209237in}{1.743742in}}%
\pgfpathclose%
\pgfusepath{stroke,fill}%
\end{pgfscope}%
\begin{pgfscope}%
\pgfpathrectangle{\pgfqpoint{0.100000in}{0.212622in}}{\pgfqpoint{3.696000in}{3.696000in}}%
\pgfusepath{clip}%
\pgfsetbuttcap%
\pgfsetroundjoin%
\definecolor{currentfill}{rgb}{0.121569,0.466667,0.705882}%
\pgfsetfillcolor{currentfill}%
\pgfsetfillopacity{0.366194}%
\pgfsetlinewidth{1.003750pt}%
\definecolor{currentstroke}{rgb}{0.121569,0.466667,0.705882}%
\pgfsetstrokecolor{currentstroke}%
\pgfsetstrokeopacity{0.366194}%
\pgfsetdash{}{0pt}%
\pgfpathmoveto{\pgfqpoint{1.207722in}{1.742309in}}%
\pgfpathcurveto{\pgfqpoint{1.215958in}{1.742309in}}{\pgfqpoint{1.223859in}{1.745581in}}{\pgfqpoint{1.229682in}{1.751405in}}%
\pgfpathcurveto{\pgfqpoint{1.235506in}{1.757229in}}{\pgfqpoint{1.238779in}{1.765129in}}{\pgfqpoint{1.238779in}{1.773366in}}%
\pgfpathcurveto{\pgfqpoint{1.238779in}{1.781602in}}{\pgfqpoint{1.235506in}{1.789502in}}{\pgfqpoint{1.229682in}{1.795326in}}%
\pgfpathcurveto{\pgfqpoint{1.223859in}{1.801150in}}{\pgfqpoint{1.215958in}{1.804422in}}{\pgfqpoint{1.207722in}{1.804422in}}%
\pgfpathcurveto{\pgfqpoint{1.199486in}{1.804422in}}{\pgfqpoint{1.191586in}{1.801150in}}{\pgfqpoint{1.185762in}{1.795326in}}%
\pgfpathcurveto{\pgfqpoint{1.179938in}{1.789502in}}{\pgfqpoint{1.176666in}{1.781602in}}{\pgfqpoint{1.176666in}{1.773366in}}%
\pgfpathcurveto{\pgfqpoint{1.176666in}{1.765129in}}{\pgfqpoint{1.179938in}{1.757229in}}{\pgfqpoint{1.185762in}{1.751405in}}%
\pgfpathcurveto{\pgfqpoint{1.191586in}{1.745581in}}{\pgfqpoint{1.199486in}{1.742309in}}{\pgfqpoint{1.207722in}{1.742309in}}%
\pgfpathclose%
\pgfusepath{stroke,fill}%
\end{pgfscope}%
\begin{pgfscope}%
\pgfpathrectangle{\pgfqpoint{0.100000in}{0.212622in}}{\pgfqpoint{3.696000in}{3.696000in}}%
\pgfusepath{clip}%
\pgfsetbuttcap%
\pgfsetroundjoin%
\definecolor{currentfill}{rgb}{0.121569,0.466667,0.705882}%
\pgfsetfillcolor{currentfill}%
\pgfsetfillopacity{0.367062}%
\pgfsetlinewidth{1.003750pt}%
\definecolor{currentstroke}{rgb}{0.121569,0.466667,0.705882}%
\pgfsetstrokecolor{currentstroke}%
\pgfsetstrokeopacity{0.367062}%
\pgfsetdash{}{0pt}%
\pgfpathmoveto{\pgfqpoint{1.206258in}{1.740640in}}%
\pgfpathcurveto{\pgfqpoint{1.214495in}{1.740640in}}{\pgfqpoint{1.222395in}{1.743912in}}{\pgfqpoint{1.228219in}{1.749736in}}%
\pgfpathcurveto{\pgfqpoint{1.234043in}{1.755560in}}{\pgfqpoint{1.237315in}{1.763460in}}{\pgfqpoint{1.237315in}{1.771697in}}%
\pgfpathcurveto{\pgfqpoint{1.237315in}{1.779933in}}{\pgfqpoint{1.234043in}{1.787833in}}{\pgfqpoint{1.228219in}{1.793657in}}%
\pgfpathcurveto{\pgfqpoint{1.222395in}{1.799481in}}{\pgfqpoint{1.214495in}{1.802753in}}{\pgfqpoint{1.206258in}{1.802753in}}%
\pgfpathcurveto{\pgfqpoint{1.198022in}{1.802753in}}{\pgfqpoint{1.190122in}{1.799481in}}{\pgfqpoint{1.184298in}{1.793657in}}%
\pgfpathcurveto{\pgfqpoint{1.178474in}{1.787833in}}{\pgfqpoint{1.175202in}{1.779933in}}{\pgfqpoint{1.175202in}{1.771697in}}%
\pgfpathcurveto{\pgfqpoint{1.175202in}{1.763460in}}{\pgfqpoint{1.178474in}{1.755560in}}{\pgfqpoint{1.184298in}{1.749736in}}%
\pgfpathcurveto{\pgfqpoint{1.190122in}{1.743912in}}{\pgfqpoint{1.198022in}{1.740640in}}{\pgfqpoint{1.206258in}{1.740640in}}%
\pgfpathclose%
\pgfusepath{stroke,fill}%
\end{pgfscope}%
\begin{pgfscope}%
\pgfpathrectangle{\pgfqpoint{0.100000in}{0.212622in}}{\pgfqpoint{3.696000in}{3.696000in}}%
\pgfusepath{clip}%
\pgfsetbuttcap%
\pgfsetroundjoin%
\definecolor{currentfill}{rgb}{0.121569,0.466667,0.705882}%
\pgfsetfillcolor{currentfill}%
\pgfsetfillopacity{0.367155}%
\pgfsetlinewidth{1.003750pt}%
\definecolor{currentstroke}{rgb}{0.121569,0.466667,0.705882}%
\pgfsetstrokecolor{currentstroke}%
\pgfsetstrokeopacity{0.367155}%
\pgfsetdash{}{0pt}%
\pgfpathmoveto{\pgfqpoint{1.568297in}{1.977212in}}%
\pgfpathcurveto{\pgfqpoint{1.576534in}{1.977212in}}{\pgfqpoint{1.584434in}{1.980485in}}{\pgfqpoint{1.590258in}{1.986309in}}%
\pgfpathcurveto{\pgfqpoint{1.596082in}{1.992133in}}{\pgfqpoint{1.599354in}{2.000033in}}{\pgfqpoint{1.599354in}{2.008269in}}%
\pgfpathcurveto{\pgfqpoint{1.599354in}{2.016505in}}{\pgfqpoint{1.596082in}{2.024405in}}{\pgfqpoint{1.590258in}{2.030229in}}%
\pgfpathcurveto{\pgfqpoint{1.584434in}{2.036053in}}{\pgfqpoint{1.576534in}{2.039325in}}{\pgfqpoint{1.568297in}{2.039325in}}%
\pgfpathcurveto{\pgfqpoint{1.560061in}{2.039325in}}{\pgfqpoint{1.552161in}{2.036053in}}{\pgfqpoint{1.546337in}{2.030229in}}%
\pgfpathcurveto{\pgfqpoint{1.540513in}{2.024405in}}{\pgfqpoint{1.537241in}{2.016505in}}{\pgfqpoint{1.537241in}{2.008269in}}%
\pgfpathcurveto{\pgfqpoint{1.537241in}{2.000033in}}{\pgfqpoint{1.540513in}{1.992133in}}{\pgfqpoint{1.546337in}{1.986309in}}%
\pgfpathcurveto{\pgfqpoint{1.552161in}{1.980485in}}{\pgfqpoint{1.560061in}{1.977212in}}{\pgfqpoint{1.568297in}{1.977212in}}%
\pgfpathclose%
\pgfusepath{stroke,fill}%
\end{pgfscope}%
\begin{pgfscope}%
\pgfpathrectangle{\pgfqpoint{0.100000in}{0.212622in}}{\pgfqpoint{3.696000in}{3.696000in}}%
\pgfusepath{clip}%
\pgfsetbuttcap%
\pgfsetroundjoin%
\definecolor{currentfill}{rgb}{0.121569,0.466667,0.705882}%
\pgfsetfillcolor{currentfill}%
\pgfsetfillopacity{0.367174}%
\pgfsetlinewidth{1.003750pt}%
\definecolor{currentstroke}{rgb}{0.121569,0.466667,0.705882}%
\pgfsetstrokecolor{currentstroke}%
\pgfsetstrokeopacity{0.367174}%
\pgfsetdash{}{0pt}%
\pgfpathmoveto{\pgfqpoint{1.205469in}{1.740130in}}%
\pgfpathcurveto{\pgfqpoint{1.213706in}{1.740130in}}{\pgfqpoint{1.221606in}{1.743402in}}{\pgfqpoint{1.227430in}{1.749226in}}%
\pgfpathcurveto{\pgfqpoint{1.233254in}{1.755050in}}{\pgfqpoint{1.236526in}{1.762950in}}{\pgfqpoint{1.236526in}{1.771186in}}%
\pgfpathcurveto{\pgfqpoint{1.236526in}{1.779423in}}{\pgfqpoint{1.233254in}{1.787323in}}{\pgfqpoint{1.227430in}{1.793147in}}%
\pgfpathcurveto{\pgfqpoint{1.221606in}{1.798970in}}{\pgfqpoint{1.213706in}{1.802243in}}{\pgfqpoint{1.205469in}{1.802243in}}%
\pgfpathcurveto{\pgfqpoint{1.197233in}{1.802243in}}{\pgfqpoint{1.189333in}{1.798970in}}{\pgfqpoint{1.183509in}{1.793147in}}%
\pgfpathcurveto{\pgfqpoint{1.177685in}{1.787323in}}{\pgfqpoint{1.174413in}{1.779423in}}{\pgfqpoint{1.174413in}{1.771186in}}%
\pgfpathcurveto{\pgfqpoint{1.174413in}{1.762950in}}{\pgfqpoint{1.177685in}{1.755050in}}{\pgfqpoint{1.183509in}{1.749226in}}%
\pgfpathcurveto{\pgfqpoint{1.189333in}{1.743402in}}{\pgfqpoint{1.197233in}{1.740130in}}{\pgfqpoint{1.205469in}{1.740130in}}%
\pgfpathclose%
\pgfusepath{stroke,fill}%
\end{pgfscope}%
\begin{pgfscope}%
\pgfpathrectangle{\pgfqpoint{0.100000in}{0.212622in}}{\pgfqpoint{3.696000in}{3.696000in}}%
\pgfusepath{clip}%
\pgfsetbuttcap%
\pgfsetroundjoin%
\definecolor{currentfill}{rgb}{0.121569,0.466667,0.705882}%
\pgfsetfillcolor{currentfill}%
\pgfsetfillopacity{0.367227}%
\pgfsetlinewidth{1.003750pt}%
\definecolor{currentstroke}{rgb}{0.121569,0.466667,0.705882}%
\pgfsetstrokecolor{currentstroke}%
\pgfsetstrokeopacity{0.367227}%
\pgfsetdash{}{0pt}%
\pgfpathmoveto{\pgfqpoint{1.205976in}{1.740289in}}%
\pgfpathcurveto{\pgfqpoint{1.214213in}{1.740289in}}{\pgfqpoint{1.222113in}{1.743561in}}{\pgfqpoint{1.227937in}{1.749385in}}%
\pgfpathcurveto{\pgfqpoint{1.233761in}{1.755209in}}{\pgfqpoint{1.237033in}{1.763109in}}{\pgfqpoint{1.237033in}{1.771345in}}%
\pgfpathcurveto{\pgfqpoint{1.237033in}{1.779581in}}{\pgfqpoint{1.233761in}{1.787481in}}{\pgfqpoint{1.227937in}{1.793305in}}%
\pgfpathcurveto{\pgfqpoint{1.222113in}{1.799129in}}{\pgfqpoint{1.214213in}{1.802402in}}{\pgfqpoint{1.205976in}{1.802402in}}%
\pgfpathcurveto{\pgfqpoint{1.197740in}{1.802402in}}{\pgfqpoint{1.189840in}{1.799129in}}{\pgfqpoint{1.184016in}{1.793305in}}%
\pgfpathcurveto{\pgfqpoint{1.178192in}{1.787481in}}{\pgfqpoint{1.174920in}{1.779581in}}{\pgfqpoint{1.174920in}{1.771345in}}%
\pgfpathcurveto{\pgfqpoint{1.174920in}{1.763109in}}{\pgfqpoint{1.178192in}{1.755209in}}{\pgfqpoint{1.184016in}{1.749385in}}%
\pgfpathcurveto{\pgfqpoint{1.189840in}{1.743561in}}{\pgfqpoint{1.197740in}{1.740289in}}{\pgfqpoint{1.205976in}{1.740289in}}%
\pgfpathclose%
\pgfusepath{stroke,fill}%
\end{pgfscope}%
\begin{pgfscope}%
\pgfpathrectangle{\pgfqpoint{0.100000in}{0.212622in}}{\pgfqpoint{3.696000in}{3.696000in}}%
\pgfusepath{clip}%
\pgfsetbuttcap%
\pgfsetroundjoin%
\definecolor{currentfill}{rgb}{0.121569,0.466667,0.705882}%
\pgfsetfillcolor{currentfill}%
\pgfsetfillopacity{0.367272}%
\pgfsetlinewidth{1.003750pt}%
\definecolor{currentstroke}{rgb}{0.121569,0.466667,0.705882}%
\pgfsetstrokecolor{currentstroke}%
\pgfsetstrokeopacity{0.367272}%
\pgfsetdash{}{0pt}%
\pgfpathmoveto{\pgfqpoint{1.205972in}{1.740240in}}%
\pgfpathcurveto{\pgfqpoint{1.214208in}{1.740240in}}{\pgfqpoint{1.222108in}{1.743513in}}{\pgfqpoint{1.227932in}{1.749337in}}%
\pgfpathcurveto{\pgfqpoint{1.233756in}{1.755161in}}{\pgfqpoint{1.237029in}{1.763061in}}{\pgfqpoint{1.237029in}{1.771297in}}%
\pgfpathcurveto{\pgfqpoint{1.237029in}{1.779533in}}{\pgfqpoint{1.233756in}{1.787433in}}{\pgfqpoint{1.227932in}{1.793257in}}%
\pgfpathcurveto{\pgfqpoint{1.222108in}{1.799081in}}{\pgfqpoint{1.214208in}{1.802353in}}{\pgfqpoint{1.205972in}{1.802353in}}%
\pgfpathcurveto{\pgfqpoint{1.197736in}{1.802353in}}{\pgfqpoint{1.189836in}{1.799081in}}{\pgfqpoint{1.184012in}{1.793257in}}%
\pgfpathcurveto{\pgfqpoint{1.178188in}{1.787433in}}{\pgfqpoint{1.174916in}{1.779533in}}{\pgfqpoint{1.174916in}{1.771297in}}%
\pgfpathcurveto{\pgfqpoint{1.174916in}{1.763061in}}{\pgfqpoint{1.178188in}{1.755161in}}{\pgfqpoint{1.184012in}{1.749337in}}%
\pgfpathcurveto{\pgfqpoint{1.189836in}{1.743513in}}{\pgfqpoint{1.197736in}{1.740240in}}{\pgfqpoint{1.205972in}{1.740240in}}%
\pgfpathclose%
\pgfusepath{stroke,fill}%
\end{pgfscope}%
\begin{pgfscope}%
\pgfpathrectangle{\pgfqpoint{0.100000in}{0.212622in}}{\pgfqpoint{3.696000in}{3.696000in}}%
\pgfusepath{clip}%
\pgfsetbuttcap%
\pgfsetroundjoin%
\definecolor{currentfill}{rgb}{0.121569,0.466667,0.705882}%
\pgfsetfillcolor{currentfill}%
\pgfsetfillopacity{0.367296}%
\pgfsetlinewidth{1.003750pt}%
\definecolor{currentstroke}{rgb}{0.121569,0.466667,0.705882}%
\pgfsetstrokecolor{currentstroke}%
\pgfsetstrokeopacity{0.367296}%
\pgfsetdash{}{0pt}%
\pgfpathmoveto{\pgfqpoint{1.205974in}{1.740228in}}%
\pgfpathcurveto{\pgfqpoint{1.214210in}{1.740228in}}{\pgfqpoint{1.222110in}{1.743500in}}{\pgfqpoint{1.227934in}{1.749324in}}%
\pgfpathcurveto{\pgfqpoint{1.233758in}{1.755148in}}{\pgfqpoint{1.237030in}{1.763048in}}{\pgfqpoint{1.237030in}{1.771284in}}%
\pgfpathcurveto{\pgfqpoint{1.237030in}{1.779521in}}{\pgfqpoint{1.233758in}{1.787421in}}{\pgfqpoint{1.227934in}{1.793245in}}%
\pgfpathcurveto{\pgfqpoint{1.222110in}{1.799068in}}{\pgfqpoint{1.214210in}{1.802341in}}{\pgfqpoint{1.205974in}{1.802341in}}%
\pgfpathcurveto{\pgfqpoint{1.197738in}{1.802341in}}{\pgfqpoint{1.189838in}{1.799068in}}{\pgfqpoint{1.184014in}{1.793245in}}%
\pgfpathcurveto{\pgfqpoint{1.178190in}{1.787421in}}{\pgfqpoint{1.174917in}{1.779521in}}{\pgfqpoint{1.174917in}{1.771284in}}%
\pgfpathcurveto{\pgfqpoint{1.174917in}{1.763048in}}{\pgfqpoint{1.178190in}{1.755148in}}{\pgfqpoint{1.184014in}{1.749324in}}%
\pgfpathcurveto{\pgfqpoint{1.189838in}{1.743500in}}{\pgfqpoint{1.197738in}{1.740228in}}{\pgfqpoint{1.205974in}{1.740228in}}%
\pgfpathclose%
\pgfusepath{stroke,fill}%
\end{pgfscope}%
\begin{pgfscope}%
\pgfpathrectangle{\pgfqpoint{0.100000in}{0.212622in}}{\pgfqpoint{3.696000in}{3.696000in}}%
\pgfusepath{clip}%
\pgfsetbuttcap%
\pgfsetroundjoin%
\definecolor{currentfill}{rgb}{0.121569,0.466667,0.705882}%
\pgfsetfillcolor{currentfill}%
\pgfsetfillopacity{0.367334}%
\pgfsetlinewidth{1.003750pt}%
\definecolor{currentstroke}{rgb}{0.121569,0.466667,0.705882}%
\pgfsetstrokecolor{currentstroke}%
\pgfsetstrokeopacity{0.367334}%
\pgfsetdash{}{0pt}%
\pgfpathmoveto{\pgfqpoint{1.205896in}{1.740148in}}%
\pgfpathcurveto{\pgfqpoint{1.214132in}{1.740148in}}{\pgfqpoint{1.222032in}{1.743420in}}{\pgfqpoint{1.227856in}{1.749244in}}%
\pgfpathcurveto{\pgfqpoint{1.233680in}{1.755068in}}{\pgfqpoint{1.236952in}{1.762968in}}{\pgfqpoint{1.236952in}{1.771205in}}%
\pgfpathcurveto{\pgfqpoint{1.236952in}{1.779441in}}{\pgfqpoint{1.233680in}{1.787341in}}{\pgfqpoint{1.227856in}{1.793165in}}%
\pgfpathcurveto{\pgfqpoint{1.222032in}{1.798989in}}{\pgfqpoint{1.214132in}{1.802261in}}{\pgfqpoint{1.205896in}{1.802261in}}%
\pgfpathcurveto{\pgfqpoint{1.197660in}{1.802261in}}{\pgfqpoint{1.189760in}{1.798989in}}{\pgfqpoint{1.183936in}{1.793165in}}%
\pgfpathcurveto{\pgfqpoint{1.178112in}{1.787341in}}{\pgfqpoint{1.174839in}{1.779441in}}{\pgfqpoint{1.174839in}{1.771205in}}%
\pgfpathcurveto{\pgfqpoint{1.174839in}{1.762968in}}{\pgfqpoint{1.178112in}{1.755068in}}{\pgfqpoint{1.183936in}{1.749244in}}%
\pgfpathcurveto{\pgfqpoint{1.189760in}{1.743420in}}{\pgfqpoint{1.197660in}{1.740148in}}{\pgfqpoint{1.205896in}{1.740148in}}%
\pgfpathclose%
\pgfusepath{stroke,fill}%
\end{pgfscope}%
\begin{pgfscope}%
\pgfpathrectangle{\pgfqpoint{0.100000in}{0.212622in}}{\pgfqpoint{3.696000in}{3.696000in}}%
\pgfusepath{clip}%
\pgfsetbuttcap%
\pgfsetroundjoin%
\definecolor{currentfill}{rgb}{0.121569,0.466667,0.705882}%
\pgfsetfillcolor{currentfill}%
\pgfsetfillopacity{0.367349}%
\pgfsetlinewidth{1.003750pt}%
\definecolor{currentstroke}{rgb}{0.121569,0.466667,0.705882}%
\pgfsetstrokecolor{currentstroke}%
\pgfsetstrokeopacity{0.367349}%
\pgfsetdash{}{0pt}%
\pgfpathmoveto{\pgfqpoint{1.205870in}{1.740119in}}%
\pgfpathcurveto{\pgfqpoint{1.214107in}{1.740119in}}{\pgfqpoint{1.222007in}{1.743391in}}{\pgfqpoint{1.227831in}{1.749215in}}%
\pgfpathcurveto{\pgfqpoint{1.233654in}{1.755039in}}{\pgfqpoint{1.236927in}{1.762939in}}{\pgfqpoint{1.236927in}{1.771175in}}%
\pgfpathcurveto{\pgfqpoint{1.236927in}{1.779411in}}{\pgfqpoint{1.233654in}{1.787311in}}{\pgfqpoint{1.227831in}{1.793135in}}%
\pgfpathcurveto{\pgfqpoint{1.222007in}{1.798959in}}{\pgfqpoint{1.214107in}{1.802232in}}{\pgfqpoint{1.205870in}{1.802232in}}%
\pgfpathcurveto{\pgfqpoint{1.197634in}{1.802232in}}{\pgfqpoint{1.189734in}{1.798959in}}{\pgfqpoint{1.183910in}{1.793135in}}%
\pgfpathcurveto{\pgfqpoint{1.178086in}{1.787311in}}{\pgfqpoint{1.174814in}{1.779411in}}{\pgfqpoint{1.174814in}{1.771175in}}%
\pgfpathcurveto{\pgfqpoint{1.174814in}{1.762939in}}{\pgfqpoint{1.178086in}{1.755039in}}{\pgfqpoint{1.183910in}{1.749215in}}%
\pgfpathcurveto{\pgfqpoint{1.189734in}{1.743391in}}{\pgfqpoint{1.197634in}{1.740119in}}{\pgfqpoint{1.205870in}{1.740119in}}%
\pgfpathclose%
\pgfusepath{stroke,fill}%
\end{pgfscope}%
\begin{pgfscope}%
\pgfpathrectangle{\pgfqpoint{0.100000in}{0.212622in}}{\pgfqpoint{3.696000in}{3.696000in}}%
\pgfusepath{clip}%
\pgfsetbuttcap%
\pgfsetroundjoin%
\definecolor{currentfill}{rgb}{0.121569,0.466667,0.705882}%
\pgfsetfillcolor{currentfill}%
\pgfsetfillopacity{0.367358}%
\pgfsetlinewidth{1.003750pt}%
\definecolor{currentstroke}{rgb}{0.121569,0.466667,0.705882}%
\pgfsetstrokecolor{currentstroke}%
\pgfsetstrokeopacity{0.367358}%
\pgfsetdash{}{0pt}%
\pgfpathmoveto{\pgfqpoint{1.205856in}{1.740103in}}%
\pgfpathcurveto{\pgfqpoint{1.214093in}{1.740103in}}{\pgfqpoint{1.221993in}{1.743375in}}{\pgfqpoint{1.227817in}{1.749199in}}%
\pgfpathcurveto{\pgfqpoint{1.233641in}{1.755023in}}{\pgfqpoint{1.236913in}{1.762923in}}{\pgfqpoint{1.236913in}{1.771159in}}%
\pgfpathcurveto{\pgfqpoint{1.236913in}{1.779395in}}{\pgfqpoint{1.233641in}{1.787295in}}{\pgfqpoint{1.227817in}{1.793119in}}%
\pgfpathcurveto{\pgfqpoint{1.221993in}{1.798943in}}{\pgfqpoint{1.214093in}{1.802216in}}{\pgfqpoint{1.205856in}{1.802216in}}%
\pgfpathcurveto{\pgfqpoint{1.197620in}{1.802216in}}{\pgfqpoint{1.189720in}{1.798943in}}{\pgfqpoint{1.183896in}{1.793119in}}%
\pgfpathcurveto{\pgfqpoint{1.178072in}{1.787295in}}{\pgfqpoint{1.174800in}{1.779395in}}{\pgfqpoint{1.174800in}{1.771159in}}%
\pgfpathcurveto{\pgfqpoint{1.174800in}{1.762923in}}{\pgfqpoint{1.178072in}{1.755023in}}{\pgfqpoint{1.183896in}{1.749199in}}%
\pgfpathcurveto{\pgfqpoint{1.189720in}{1.743375in}}{\pgfqpoint{1.197620in}{1.740103in}}{\pgfqpoint{1.205856in}{1.740103in}}%
\pgfpathclose%
\pgfusepath{stroke,fill}%
\end{pgfscope}%
\begin{pgfscope}%
\pgfpathrectangle{\pgfqpoint{0.100000in}{0.212622in}}{\pgfqpoint{3.696000in}{3.696000in}}%
\pgfusepath{clip}%
\pgfsetbuttcap%
\pgfsetroundjoin%
\definecolor{currentfill}{rgb}{0.121569,0.466667,0.705882}%
\pgfsetfillcolor{currentfill}%
\pgfsetfillopacity{0.367367}%
\pgfsetlinewidth{1.003750pt}%
\definecolor{currentstroke}{rgb}{0.121569,0.466667,0.705882}%
\pgfsetstrokecolor{currentstroke}%
\pgfsetstrokeopacity{0.367367}%
\pgfsetdash{}{0pt}%
\pgfpathmoveto{\pgfqpoint{1.205834in}{1.740082in}}%
\pgfpathcurveto{\pgfqpoint{1.214071in}{1.740082in}}{\pgfqpoint{1.221971in}{1.743354in}}{\pgfqpoint{1.227795in}{1.749178in}}%
\pgfpathcurveto{\pgfqpoint{1.233619in}{1.755002in}}{\pgfqpoint{1.236891in}{1.762902in}}{\pgfqpoint{1.236891in}{1.771138in}}%
\pgfpathcurveto{\pgfqpoint{1.236891in}{1.779375in}}{\pgfqpoint{1.233619in}{1.787275in}}{\pgfqpoint{1.227795in}{1.793099in}}%
\pgfpathcurveto{\pgfqpoint{1.221971in}{1.798923in}}{\pgfqpoint{1.214071in}{1.802195in}}{\pgfqpoint{1.205834in}{1.802195in}}%
\pgfpathcurveto{\pgfqpoint{1.197598in}{1.802195in}}{\pgfqpoint{1.189698in}{1.798923in}}{\pgfqpoint{1.183874in}{1.793099in}}%
\pgfpathcurveto{\pgfqpoint{1.178050in}{1.787275in}}{\pgfqpoint{1.174778in}{1.779375in}}{\pgfqpoint{1.174778in}{1.771138in}}%
\pgfpathcurveto{\pgfqpoint{1.174778in}{1.762902in}}{\pgfqpoint{1.178050in}{1.755002in}}{\pgfqpoint{1.183874in}{1.749178in}}%
\pgfpathcurveto{\pgfqpoint{1.189698in}{1.743354in}}{\pgfqpoint{1.197598in}{1.740082in}}{\pgfqpoint{1.205834in}{1.740082in}}%
\pgfpathclose%
\pgfusepath{stroke,fill}%
\end{pgfscope}%
\begin{pgfscope}%
\pgfpathrectangle{\pgfqpoint{0.100000in}{0.212622in}}{\pgfqpoint{3.696000in}{3.696000in}}%
\pgfusepath{clip}%
\pgfsetbuttcap%
\pgfsetroundjoin%
\definecolor{currentfill}{rgb}{0.121569,0.466667,0.705882}%
\pgfsetfillcolor{currentfill}%
\pgfsetfillopacity{0.367368}%
\pgfsetlinewidth{1.003750pt}%
\definecolor{currentstroke}{rgb}{0.121569,0.466667,0.705882}%
\pgfsetstrokecolor{currentstroke}%
\pgfsetstrokeopacity{0.367368}%
\pgfsetdash{}{0pt}%
\pgfpathmoveto{\pgfqpoint{1.205837in}{1.740083in}}%
\pgfpathcurveto{\pgfqpoint{1.214073in}{1.740083in}}{\pgfqpoint{1.221973in}{1.743355in}}{\pgfqpoint{1.227797in}{1.749179in}}%
\pgfpathcurveto{\pgfqpoint{1.233621in}{1.755003in}}{\pgfqpoint{1.236893in}{1.762903in}}{\pgfqpoint{1.236893in}{1.771139in}}%
\pgfpathcurveto{\pgfqpoint{1.236893in}{1.779375in}}{\pgfqpoint{1.233621in}{1.787275in}}{\pgfqpoint{1.227797in}{1.793099in}}%
\pgfpathcurveto{\pgfqpoint{1.221973in}{1.798923in}}{\pgfqpoint{1.214073in}{1.802196in}}{\pgfqpoint{1.205837in}{1.802196in}}%
\pgfpathcurveto{\pgfqpoint{1.197600in}{1.802196in}}{\pgfqpoint{1.189700in}{1.798923in}}{\pgfqpoint{1.183876in}{1.793099in}}%
\pgfpathcurveto{\pgfqpoint{1.178053in}{1.787275in}}{\pgfqpoint{1.174780in}{1.779375in}}{\pgfqpoint{1.174780in}{1.771139in}}%
\pgfpathcurveto{\pgfqpoint{1.174780in}{1.762903in}}{\pgfqpoint{1.178053in}{1.755003in}}{\pgfqpoint{1.183876in}{1.749179in}}%
\pgfpathcurveto{\pgfqpoint{1.189700in}{1.743355in}}{\pgfqpoint{1.197600in}{1.740083in}}{\pgfqpoint{1.205837in}{1.740083in}}%
\pgfpathclose%
\pgfusepath{stroke,fill}%
\end{pgfscope}%
\begin{pgfscope}%
\pgfpathrectangle{\pgfqpoint{0.100000in}{0.212622in}}{\pgfqpoint{3.696000in}{3.696000in}}%
\pgfusepath{clip}%
\pgfsetbuttcap%
\pgfsetroundjoin%
\definecolor{currentfill}{rgb}{0.121569,0.466667,0.705882}%
\pgfsetfillcolor{currentfill}%
\pgfsetfillopacity{0.367369}%
\pgfsetlinewidth{1.003750pt}%
\definecolor{currentstroke}{rgb}{0.121569,0.466667,0.705882}%
\pgfsetstrokecolor{currentstroke}%
\pgfsetstrokeopacity{0.367369}%
\pgfsetdash{}{0pt}%
\pgfpathmoveto{\pgfqpoint{1.205833in}{1.740079in}}%
\pgfpathcurveto{\pgfqpoint{1.214070in}{1.740079in}}{\pgfqpoint{1.221970in}{1.743351in}}{\pgfqpoint{1.227794in}{1.749175in}}%
\pgfpathcurveto{\pgfqpoint{1.233618in}{1.754999in}}{\pgfqpoint{1.236890in}{1.762899in}}{\pgfqpoint{1.236890in}{1.771136in}}%
\pgfpathcurveto{\pgfqpoint{1.236890in}{1.779372in}}{\pgfqpoint{1.233618in}{1.787272in}}{\pgfqpoint{1.227794in}{1.793096in}}%
\pgfpathcurveto{\pgfqpoint{1.221970in}{1.798920in}}{\pgfqpoint{1.214070in}{1.802192in}}{\pgfqpoint{1.205833in}{1.802192in}}%
\pgfpathcurveto{\pgfqpoint{1.197597in}{1.802192in}}{\pgfqpoint{1.189697in}{1.798920in}}{\pgfqpoint{1.183873in}{1.793096in}}%
\pgfpathcurveto{\pgfqpoint{1.178049in}{1.787272in}}{\pgfqpoint{1.174777in}{1.779372in}}{\pgfqpoint{1.174777in}{1.771136in}}%
\pgfpathcurveto{\pgfqpoint{1.174777in}{1.762899in}}{\pgfqpoint{1.178049in}{1.754999in}}{\pgfqpoint{1.183873in}{1.749175in}}%
\pgfpathcurveto{\pgfqpoint{1.189697in}{1.743351in}}{\pgfqpoint{1.197597in}{1.740079in}}{\pgfqpoint{1.205833in}{1.740079in}}%
\pgfpathclose%
\pgfusepath{stroke,fill}%
\end{pgfscope}%
\begin{pgfscope}%
\pgfpathrectangle{\pgfqpoint{0.100000in}{0.212622in}}{\pgfqpoint{3.696000in}{3.696000in}}%
\pgfusepath{clip}%
\pgfsetbuttcap%
\pgfsetroundjoin%
\definecolor{currentfill}{rgb}{0.121569,0.466667,0.705882}%
\pgfsetfillcolor{currentfill}%
\pgfsetfillopacity{0.367370}%
\pgfsetlinewidth{1.003750pt}%
\definecolor{currentstroke}{rgb}{0.121569,0.466667,0.705882}%
\pgfsetstrokecolor{currentstroke}%
\pgfsetstrokeopacity{0.367370}%
\pgfsetdash{}{0pt}%
\pgfpathmoveto{\pgfqpoint{1.205831in}{1.740077in}}%
\pgfpathcurveto{\pgfqpoint{1.214067in}{1.740077in}}{\pgfqpoint{1.221967in}{1.743349in}}{\pgfqpoint{1.227791in}{1.749173in}}%
\pgfpathcurveto{\pgfqpoint{1.233615in}{1.754997in}}{\pgfqpoint{1.236887in}{1.762897in}}{\pgfqpoint{1.236887in}{1.771133in}}%
\pgfpathcurveto{\pgfqpoint{1.236887in}{1.779369in}}{\pgfqpoint{1.233615in}{1.787269in}}{\pgfqpoint{1.227791in}{1.793093in}}%
\pgfpathcurveto{\pgfqpoint{1.221967in}{1.798917in}}{\pgfqpoint{1.214067in}{1.802190in}}{\pgfqpoint{1.205831in}{1.802190in}}%
\pgfpathcurveto{\pgfqpoint{1.197595in}{1.802190in}}{\pgfqpoint{1.189695in}{1.798917in}}{\pgfqpoint{1.183871in}{1.793093in}}%
\pgfpathcurveto{\pgfqpoint{1.178047in}{1.787269in}}{\pgfqpoint{1.174774in}{1.779369in}}{\pgfqpoint{1.174774in}{1.771133in}}%
\pgfpathcurveto{\pgfqpoint{1.174774in}{1.762897in}}{\pgfqpoint{1.178047in}{1.754997in}}{\pgfqpoint{1.183871in}{1.749173in}}%
\pgfpathcurveto{\pgfqpoint{1.189695in}{1.743349in}}{\pgfqpoint{1.197595in}{1.740077in}}{\pgfqpoint{1.205831in}{1.740077in}}%
\pgfpathclose%
\pgfusepath{stroke,fill}%
\end{pgfscope}%
\begin{pgfscope}%
\pgfpathrectangle{\pgfqpoint{0.100000in}{0.212622in}}{\pgfqpoint{3.696000in}{3.696000in}}%
\pgfusepath{clip}%
\pgfsetbuttcap%
\pgfsetroundjoin%
\definecolor{currentfill}{rgb}{0.121569,0.466667,0.705882}%
\pgfsetfillcolor{currentfill}%
\pgfsetfillopacity{0.367370}%
\pgfsetlinewidth{1.003750pt}%
\definecolor{currentstroke}{rgb}{0.121569,0.466667,0.705882}%
\pgfsetstrokecolor{currentstroke}%
\pgfsetstrokeopacity{0.367370}%
\pgfsetdash{}{0pt}%
\pgfpathmoveto{\pgfqpoint{1.205831in}{1.740077in}}%
\pgfpathcurveto{\pgfqpoint{1.214068in}{1.740077in}}{\pgfqpoint{1.221968in}{1.743349in}}{\pgfqpoint{1.227792in}{1.749173in}}%
\pgfpathcurveto{\pgfqpoint{1.233616in}{1.754997in}}{\pgfqpoint{1.236888in}{1.762897in}}{\pgfqpoint{1.236888in}{1.771133in}}%
\pgfpathcurveto{\pgfqpoint{1.236888in}{1.779370in}}{\pgfqpoint{1.233616in}{1.787270in}}{\pgfqpoint{1.227792in}{1.793094in}}%
\pgfpathcurveto{\pgfqpoint{1.221968in}{1.798917in}}{\pgfqpoint{1.214068in}{1.802190in}}{\pgfqpoint{1.205831in}{1.802190in}}%
\pgfpathcurveto{\pgfqpoint{1.197595in}{1.802190in}}{\pgfqpoint{1.189695in}{1.798917in}}{\pgfqpoint{1.183871in}{1.793094in}}%
\pgfpathcurveto{\pgfqpoint{1.178047in}{1.787270in}}{\pgfqpoint{1.174775in}{1.779370in}}{\pgfqpoint{1.174775in}{1.771133in}}%
\pgfpathcurveto{\pgfqpoint{1.174775in}{1.762897in}}{\pgfqpoint{1.178047in}{1.754997in}}{\pgfqpoint{1.183871in}{1.749173in}}%
\pgfpathcurveto{\pgfqpoint{1.189695in}{1.743349in}}{\pgfqpoint{1.197595in}{1.740077in}}{\pgfqpoint{1.205831in}{1.740077in}}%
\pgfpathclose%
\pgfusepath{stroke,fill}%
\end{pgfscope}%
\begin{pgfscope}%
\pgfpathrectangle{\pgfqpoint{0.100000in}{0.212622in}}{\pgfqpoint{3.696000in}{3.696000in}}%
\pgfusepath{clip}%
\pgfsetbuttcap%
\pgfsetroundjoin%
\definecolor{currentfill}{rgb}{0.121569,0.466667,0.705882}%
\pgfsetfillcolor{currentfill}%
\pgfsetfillopacity{0.367371}%
\pgfsetlinewidth{1.003750pt}%
\definecolor{currentstroke}{rgb}{0.121569,0.466667,0.705882}%
\pgfsetstrokecolor{currentstroke}%
\pgfsetstrokeopacity{0.367371}%
\pgfsetdash{}{0pt}%
\pgfpathmoveto{\pgfqpoint{1.205831in}{1.740076in}}%
\pgfpathcurveto{\pgfqpoint{1.214067in}{1.740076in}}{\pgfqpoint{1.221967in}{1.743348in}}{\pgfqpoint{1.227791in}{1.749172in}}%
\pgfpathcurveto{\pgfqpoint{1.233615in}{1.754996in}}{\pgfqpoint{1.236887in}{1.762896in}}{\pgfqpoint{1.236887in}{1.771133in}}%
\pgfpathcurveto{\pgfqpoint{1.236887in}{1.779369in}}{\pgfqpoint{1.233615in}{1.787269in}}{\pgfqpoint{1.227791in}{1.793093in}}%
\pgfpathcurveto{\pgfqpoint{1.221967in}{1.798917in}}{\pgfqpoint{1.214067in}{1.802189in}}{\pgfqpoint{1.205831in}{1.802189in}}%
\pgfpathcurveto{\pgfqpoint{1.197595in}{1.802189in}}{\pgfqpoint{1.189695in}{1.798917in}}{\pgfqpoint{1.183871in}{1.793093in}}%
\pgfpathcurveto{\pgfqpoint{1.178047in}{1.787269in}}{\pgfqpoint{1.174774in}{1.779369in}}{\pgfqpoint{1.174774in}{1.771133in}}%
\pgfpathcurveto{\pgfqpoint{1.174774in}{1.762896in}}{\pgfqpoint{1.178047in}{1.754996in}}{\pgfqpoint{1.183871in}{1.749172in}}%
\pgfpathcurveto{\pgfqpoint{1.189695in}{1.743348in}}{\pgfqpoint{1.197595in}{1.740076in}}{\pgfqpoint{1.205831in}{1.740076in}}%
\pgfpathclose%
\pgfusepath{stroke,fill}%
\end{pgfscope}%
\begin{pgfscope}%
\pgfpathrectangle{\pgfqpoint{0.100000in}{0.212622in}}{\pgfqpoint{3.696000in}{3.696000in}}%
\pgfusepath{clip}%
\pgfsetbuttcap%
\pgfsetroundjoin%
\definecolor{currentfill}{rgb}{0.121569,0.466667,0.705882}%
\pgfsetfillcolor{currentfill}%
\pgfsetfillopacity{0.367371}%
\pgfsetlinewidth{1.003750pt}%
\definecolor{currentstroke}{rgb}{0.121569,0.466667,0.705882}%
\pgfsetstrokecolor{currentstroke}%
\pgfsetstrokeopacity{0.367371}%
\pgfsetdash{}{0pt}%
\pgfpathmoveto{\pgfqpoint{1.205830in}{1.740076in}}%
\pgfpathcurveto{\pgfqpoint{1.214067in}{1.740076in}}{\pgfqpoint{1.221967in}{1.743348in}}{\pgfqpoint{1.227791in}{1.749172in}}%
\pgfpathcurveto{\pgfqpoint{1.233615in}{1.754996in}}{\pgfqpoint{1.236887in}{1.762896in}}{\pgfqpoint{1.236887in}{1.771132in}}%
\pgfpathcurveto{\pgfqpoint{1.236887in}{1.779369in}}{\pgfqpoint{1.233615in}{1.787269in}}{\pgfqpoint{1.227791in}{1.793093in}}%
\pgfpathcurveto{\pgfqpoint{1.221967in}{1.798916in}}{\pgfqpoint{1.214067in}{1.802189in}}{\pgfqpoint{1.205830in}{1.802189in}}%
\pgfpathcurveto{\pgfqpoint{1.197594in}{1.802189in}}{\pgfqpoint{1.189694in}{1.798916in}}{\pgfqpoint{1.183870in}{1.793093in}}%
\pgfpathcurveto{\pgfqpoint{1.178046in}{1.787269in}}{\pgfqpoint{1.174774in}{1.779369in}}{\pgfqpoint{1.174774in}{1.771132in}}%
\pgfpathcurveto{\pgfqpoint{1.174774in}{1.762896in}}{\pgfqpoint{1.178046in}{1.754996in}}{\pgfqpoint{1.183870in}{1.749172in}}%
\pgfpathcurveto{\pgfqpoint{1.189694in}{1.743348in}}{\pgfqpoint{1.197594in}{1.740076in}}{\pgfqpoint{1.205830in}{1.740076in}}%
\pgfpathclose%
\pgfusepath{stroke,fill}%
\end{pgfscope}%
\begin{pgfscope}%
\pgfpathrectangle{\pgfqpoint{0.100000in}{0.212622in}}{\pgfqpoint{3.696000in}{3.696000in}}%
\pgfusepath{clip}%
\pgfsetbuttcap%
\pgfsetroundjoin%
\definecolor{currentfill}{rgb}{0.121569,0.466667,0.705882}%
\pgfsetfillcolor{currentfill}%
\pgfsetfillopacity{0.367371}%
\pgfsetlinewidth{1.003750pt}%
\definecolor{currentstroke}{rgb}{0.121569,0.466667,0.705882}%
\pgfsetstrokecolor{currentstroke}%
\pgfsetstrokeopacity{0.367371}%
\pgfsetdash{}{0pt}%
\pgfpathmoveto{\pgfqpoint{1.205830in}{1.740076in}}%
\pgfpathcurveto{\pgfqpoint{1.214067in}{1.740076in}}{\pgfqpoint{1.221967in}{1.743348in}}{\pgfqpoint{1.227791in}{1.749172in}}%
\pgfpathcurveto{\pgfqpoint{1.233614in}{1.754996in}}{\pgfqpoint{1.236887in}{1.762896in}}{\pgfqpoint{1.236887in}{1.771132in}}%
\pgfpathcurveto{\pgfqpoint{1.236887in}{1.779368in}}{\pgfqpoint{1.233614in}{1.787268in}}{\pgfqpoint{1.227791in}{1.793092in}}%
\pgfpathcurveto{\pgfqpoint{1.221967in}{1.798916in}}{\pgfqpoint{1.214067in}{1.802189in}}{\pgfqpoint{1.205830in}{1.802189in}}%
\pgfpathcurveto{\pgfqpoint{1.197594in}{1.802189in}}{\pgfqpoint{1.189694in}{1.798916in}}{\pgfqpoint{1.183870in}{1.793092in}}%
\pgfpathcurveto{\pgfqpoint{1.178046in}{1.787268in}}{\pgfqpoint{1.174774in}{1.779368in}}{\pgfqpoint{1.174774in}{1.771132in}}%
\pgfpathcurveto{\pgfqpoint{1.174774in}{1.762896in}}{\pgfqpoint{1.178046in}{1.754996in}}{\pgfqpoint{1.183870in}{1.749172in}}%
\pgfpathcurveto{\pgfqpoint{1.189694in}{1.743348in}}{\pgfqpoint{1.197594in}{1.740076in}}{\pgfqpoint{1.205830in}{1.740076in}}%
\pgfpathclose%
\pgfusepath{stroke,fill}%
\end{pgfscope}%
\begin{pgfscope}%
\pgfpathrectangle{\pgfqpoint{0.100000in}{0.212622in}}{\pgfqpoint{3.696000in}{3.696000in}}%
\pgfusepath{clip}%
\pgfsetbuttcap%
\pgfsetroundjoin%
\definecolor{currentfill}{rgb}{0.121569,0.466667,0.705882}%
\pgfsetfillcolor{currentfill}%
\pgfsetfillopacity{0.367371}%
\pgfsetlinewidth{1.003750pt}%
\definecolor{currentstroke}{rgb}{0.121569,0.466667,0.705882}%
\pgfsetstrokecolor{currentstroke}%
\pgfsetstrokeopacity{0.367371}%
\pgfsetdash{}{0pt}%
\pgfpathmoveto{\pgfqpoint{1.205830in}{1.740075in}}%
\pgfpathcurveto{\pgfqpoint{1.214066in}{1.740075in}}{\pgfqpoint{1.221966in}{1.743348in}}{\pgfqpoint{1.227790in}{1.749172in}}%
\pgfpathcurveto{\pgfqpoint{1.233614in}{1.754996in}}{\pgfqpoint{1.236887in}{1.762896in}}{\pgfqpoint{1.236887in}{1.771132in}}%
\pgfpathcurveto{\pgfqpoint{1.236887in}{1.779368in}}{\pgfqpoint{1.233614in}{1.787268in}}{\pgfqpoint{1.227790in}{1.793092in}}%
\pgfpathcurveto{\pgfqpoint{1.221966in}{1.798916in}}{\pgfqpoint{1.214066in}{1.802188in}}{\pgfqpoint{1.205830in}{1.802188in}}%
\pgfpathcurveto{\pgfqpoint{1.197594in}{1.802188in}}{\pgfqpoint{1.189694in}{1.798916in}}{\pgfqpoint{1.183870in}{1.793092in}}%
\pgfpathcurveto{\pgfqpoint{1.178046in}{1.787268in}}{\pgfqpoint{1.174774in}{1.779368in}}{\pgfqpoint{1.174774in}{1.771132in}}%
\pgfpathcurveto{\pgfqpoint{1.174774in}{1.762896in}}{\pgfqpoint{1.178046in}{1.754996in}}{\pgfqpoint{1.183870in}{1.749172in}}%
\pgfpathcurveto{\pgfqpoint{1.189694in}{1.743348in}}{\pgfqpoint{1.197594in}{1.740075in}}{\pgfqpoint{1.205830in}{1.740075in}}%
\pgfpathclose%
\pgfusepath{stroke,fill}%
\end{pgfscope}%
\begin{pgfscope}%
\pgfpathrectangle{\pgfqpoint{0.100000in}{0.212622in}}{\pgfqpoint{3.696000in}{3.696000in}}%
\pgfusepath{clip}%
\pgfsetbuttcap%
\pgfsetroundjoin%
\definecolor{currentfill}{rgb}{0.121569,0.466667,0.705882}%
\pgfsetfillcolor{currentfill}%
\pgfsetfillopacity{0.367371}%
\pgfsetlinewidth{1.003750pt}%
\definecolor{currentstroke}{rgb}{0.121569,0.466667,0.705882}%
\pgfsetstrokecolor{currentstroke}%
\pgfsetstrokeopacity{0.367371}%
\pgfsetdash{}{0pt}%
\pgfpathmoveto{\pgfqpoint{1.205830in}{1.740075in}}%
\pgfpathcurveto{\pgfqpoint{1.214066in}{1.740075in}}{\pgfqpoint{1.221966in}{1.743348in}}{\pgfqpoint{1.227790in}{1.749172in}}%
\pgfpathcurveto{\pgfqpoint{1.233614in}{1.754996in}}{\pgfqpoint{1.236887in}{1.762896in}}{\pgfqpoint{1.236887in}{1.771132in}}%
\pgfpathcurveto{\pgfqpoint{1.236887in}{1.779368in}}{\pgfqpoint{1.233614in}{1.787268in}}{\pgfqpoint{1.227790in}{1.793092in}}%
\pgfpathcurveto{\pgfqpoint{1.221966in}{1.798916in}}{\pgfqpoint{1.214066in}{1.802188in}}{\pgfqpoint{1.205830in}{1.802188in}}%
\pgfpathcurveto{\pgfqpoint{1.197594in}{1.802188in}}{\pgfqpoint{1.189694in}{1.798916in}}{\pgfqpoint{1.183870in}{1.793092in}}%
\pgfpathcurveto{\pgfqpoint{1.178046in}{1.787268in}}{\pgfqpoint{1.174774in}{1.779368in}}{\pgfqpoint{1.174774in}{1.771132in}}%
\pgfpathcurveto{\pgfqpoint{1.174774in}{1.762896in}}{\pgfqpoint{1.178046in}{1.754996in}}{\pgfqpoint{1.183870in}{1.749172in}}%
\pgfpathcurveto{\pgfqpoint{1.189694in}{1.743348in}}{\pgfqpoint{1.197594in}{1.740075in}}{\pgfqpoint{1.205830in}{1.740075in}}%
\pgfpathclose%
\pgfusepath{stroke,fill}%
\end{pgfscope}%
\begin{pgfscope}%
\pgfpathrectangle{\pgfqpoint{0.100000in}{0.212622in}}{\pgfqpoint{3.696000in}{3.696000in}}%
\pgfusepath{clip}%
\pgfsetbuttcap%
\pgfsetroundjoin%
\definecolor{currentfill}{rgb}{0.121569,0.466667,0.705882}%
\pgfsetfillcolor{currentfill}%
\pgfsetfillopacity{0.367371}%
\pgfsetlinewidth{1.003750pt}%
\definecolor{currentstroke}{rgb}{0.121569,0.466667,0.705882}%
\pgfsetstrokecolor{currentstroke}%
\pgfsetstrokeopacity{0.367371}%
\pgfsetdash{}{0pt}%
\pgfpathmoveto{\pgfqpoint{1.205830in}{1.740075in}}%
\pgfpathcurveto{\pgfqpoint{1.214066in}{1.740075in}}{\pgfqpoint{1.221966in}{1.743348in}}{\pgfqpoint{1.227790in}{1.749172in}}%
\pgfpathcurveto{\pgfqpoint{1.233614in}{1.754995in}}{\pgfqpoint{1.236887in}{1.762896in}}{\pgfqpoint{1.236887in}{1.771132in}}%
\pgfpathcurveto{\pgfqpoint{1.236887in}{1.779368in}}{\pgfqpoint{1.233614in}{1.787268in}}{\pgfqpoint{1.227790in}{1.793092in}}%
\pgfpathcurveto{\pgfqpoint{1.221966in}{1.798916in}}{\pgfqpoint{1.214066in}{1.802188in}}{\pgfqpoint{1.205830in}{1.802188in}}%
\pgfpathcurveto{\pgfqpoint{1.197594in}{1.802188in}}{\pgfqpoint{1.189694in}{1.798916in}}{\pgfqpoint{1.183870in}{1.793092in}}%
\pgfpathcurveto{\pgfqpoint{1.178046in}{1.787268in}}{\pgfqpoint{1.174774in}{1.779368in}}{\pgfqpoint{1.174774in}{1.771132in}}%
\pgfpathcurveto{\pgfqpoint{1.174774in}{1.762896in}}{\pgfqpoint{1.178046in}{1.754995in}}{\pgfqpoint{1.183870in}{1.749172in}}%
\pgfpathcurveto{\pgfqpoint{1.189694in}{1.743348in}}{\pgfqpoint{1.197594in}{1.740075in}}{\pgfqpoint{1.205830in}{1.740075in}}%
\pgfpathclose%
\pgfusepath{stroke,fill}%
\end{pgfscope}%
\begin{pgfscope}%
\pgfpathrectangle{\pgfqpoint{0.100000in}{0.212622in}}{\pgfqpoint{3.696000in}{3.696000in}}%
\pgfusepath{clip}%
\pgfsetbuttcap%
\pgfsetroundjoin%
\definecolor{currentfill}{rgb}{0.121569,0.466667,0.705882}%
\pgfsetfillcolor{currentfill}%
\pgfsetfillopacity{0.367371}%
\pgfsetlinewidth{1.003750pt}%
\definecolor{currentstroke}{rgb}{0.121569,0.466667,0.705882}%
\pgfsetstrokecolor{currentstroke}%
\pgfsetstrokeopacity{0.367371}%
\pgfsetdash{}{0pt}%
\pgfpathmoveto{\pgfqpoint{1.205830in}{1.740075in}}%
\pgfpathcurveto{\pgfqpoint{1.214066in}{1.740075in}}{\pgfqpoint{1.221966in}{1.743348in}}{\pgfqpoint{1.227790in}{1.749172in}}%
\pgfpathcurveto{\pgfqpoint{1.233614in}{1.754995in}}{\pgfqpoint{1.236887in}{1.762896in}}{\pgfqpoint{1.236887in}{1.771132in}}%
\pgfpathcurveto{\pgfqpoint{1.236887in}{1.779368in}}{\pgfqpoint{1.233614in}{1.787268in}}{\pgfqpoint{1.227790in}{1.793092in}}%
\pgfpathcurveto{\pgfqpoint{1.221966in}{1.798916in}}{\pgfqpoint{1.214066in}{1.802188in}}{\pgfqpoint{1.205830in}{1.802188in}}%
\pgfpathcurveto{\pgfqpoint{1.197594in}{1.802188in}}{\pgfqpoint{1.189694in}{1.798916in}}{\pgfqpoint{1.183870in}{1.793092in}}%
\pgfpathcurveto{\pgfqpoint{1.178046in}{1.787268in}}{\pgfqpoint{1.174774in}{1.779368in}}{\pgfqpoint{1.174774in}{1.771132in}}%
\pgfpathcurveto{\pgfqpoint{1.174774in}{1.762896in}}{\pgfqpoint{1.178046in}{1.754995in}}{\pgfqpoint{1.183870in}{1.749172in}}%
\pgfpathcurveto{\pgfqpoint{1.189694in}{1.743348in}}{\pgfqpoint{1.197594in}{1.740075in}}{\pgfqpoint{1.205830in}{1.740075in}}%
\pgfpathclose%
\pgfusepath{stroke,fill}%
\end{pgfscope}%
\begin{pgfscope}%
\pgfpathrectangle{\pgfqpoint{0.100000in}{0.212622in}}{\pgfqpoint{3.696000in}{3.696000in}}%
\pgfusepath{clip}%
\pgfsetbuttcap%
\pgfsetroundjoin%
\definecolor{currentfill}{rgb}{0.121569,0.466667,0.705882}%
\pgfsetfillcolor{currentfill}%
\pgfsetfillopacity{0.367371}%
\pgfsetlinewidth{1.003750pt}%
\definecolor{currentstroke}{rgb}{0.121569,0.466667,0.705882}%
\pgfsetstrokecolor{currentstroke}%
\pgfsetstrokeopacity{0.367371}%
\pgfsetdash{}{0pt}%
\pgfpathmoveto{\pgfqpoint{1.205830in}{1.740075in}}%
\pgfpathcurveto{\pgfqpoint{1.214066in}{1.740075in}}{\pgfqpoint{1.221966in}{1.743348in}}{\pgfqpoint{1.227790in}{1.749172in}}%
\pgfpathcurveto{\pgfqpoint{1.233614in}{1.754995in}}{\pgfqpoint{1.236887in}{1.762896in}}{\pgfqpoint{1.236887in}{1.771132in}}%
\pgfpathcurveto{\pgfqpoint{1.236887in}{1.779368in}}{\pgfqpoint{1.233614in}{1.787268in}}{\pgfqpoint{1.227790in}{1.793092in}}%
\pgfpathcurveto{\pgfqpoint{1.221966in}{1.798916in}}{\pgfqpoint{1.214066in}{1.802188in}}{\pgfqpoint{1.205830in}{1.802188in}}%
\pgfpathcurveto{\pgfqpoint{1.197594in}{1.802188in}}{\pgfqpoint{1.189694in}{1.798916in}}{\pgfqpoint{1.183870in}{1.793092in}}%
\pgfpathcurveto{\pgfqpoint{1.178046in}{1.787268in}}{\pgfqpoint{1.174774in}{1.779368in}}{\pgfqpoint{1.174774in}{1.771132in}}%
\pgfpathcurveto{\pgfqpoint{1.174774in}{1.762896in}}{\pgfqpoint{1.178046in}{1.754995in}}{\pgfqpoint{1.183870in}{1.749172in}}%
\pgfpathcurveto{\pgfqpoint{1.189694in}{1.743348in}}{\pgfqpoint{1.197594in}{1.740075in}}{\pgfqpoint{1.205830in}{1.740075in}}%
\pgfpathclose%
\pgfusepath{stroke,fill}%
\end{pgfscope}%
\begin{pgfscope}%
\pgfpathrectangle{\pgfqpoint{0.100000in}{0.212622in}}{\pgfqpoint{3.696000in}{3.696000in}}%
\pgfusepath{clip}%
\pgfsetbuttcap%
\pgfsetroundjoin%
\definecolor{currentfill}{rgb}{0.121569,0.466667,0.705882}%
\pgfsetfillcolor{currentfill}%
\pgfsetfillopacity{0.367371}%
\pgfsetlinewidth{1.003750pt}%
\definecolor{currentstroke}{rgb}{0.121569,0.466667,0.705882}%
\pgfsetstrokecolor{currentstroke}%
\pgfsetstrokeopacity{0.367371}%
\pgfsetdash{}{0pt}%
\pgfpathmoveto{\pgfqpoint{1.205830in}{1.740075in}}%
\pgfpathcurveto{\pgfqpoint{1.214066in}{1.740075in}}{\pgfqpoint{1.221966in}{1.743348in}}{\pgfqpoint{1.227790in}{1.749172in}}%
\pgfpathcurveto{\pgfqpoint{1.233614in}{1.754995in}}{\pgfqpoint{1.236887in}{1.762896in}}{\pgfqpoint{1.236887in}{1.771132in}}%
\pgfpathcurveto{\pgfqpoint{1.236887in}{1.779368in}}{\pgfqpoint{1.233614in}{1.787268in}}{\pgfqpoint{1.227790in}{1.793092in}}%
\pgfpathcurveto{\pgfqpoint{1.221966in}{1.798916in}}{\pgfqpoint{1.214066in}{1.802188in}}{\pgfqpoint{1.205830in}{1.802188in}}%
\pgfpathcurveto{\pgfqpoint{1.197594in}{1.802188in}}{\pgfqpoint{1.189694in}{1.798916in}}{\pgfqpoint{1.183870in}{1.793092in}}%
\pgfpathcurveto{\pgfqpoint{1.178046in}{1.787268in}}{\pgfqpoint{1.174774in}{1.779368in}}{\pgfqpoint{1.174774in}{1.771132in}}%
\pgfpathcurveto{\pgfqpoint{1.174774in}{1.762896in}}{\pgfqpoint{1.178046in}{1.754995in}}{\pgfqpoint{1.183870in}{1.749172in}}%
\pgfpathcurveto{\pgfqpoint{1.189694in}{1.743348in}}{\pgfqpoint{1.197594in}{1.740075in}}{\pgfqpoint{1.205830in}{1.740075in}}%
\pgfpathclose%
\pgfusepath{stroke,fill}%
\end{pgfscope}%
\begin{pgfscope}%
\pgfpathrectangle{\pgfqpoint{0.100000in}{0.212622in}}{\pgfqpoint{3.696000in}{3.696000in}}%
\pgfusepath{clip}%
\pgfsetbuttcap%
\pgfsetroundjoin%
\definecolor{currentfill}{rgb}{0.121569,0.466667,0.705882}%
\pgfsetfillcolor{currentfill}%
\pgfsetfillopacity{0.367371}%
\pgfsetlinewidth{1.003750pt}%
\definecolor{currentstroke}{rgb}{0.121569,0.466667,0.705882}%
\pgfsetstrokecolor{currentstroke}%
\pgfsetstrokeopacity{0.367371}%
\pgfsetdash{}{0pt}%
\pgfpathmoveto{\pgfqpoint{1.205830in}{1.740075in}}%
\pgfpathcurveto{\pgfqpoint{1.214066in}{1.740075in}}{\pgfqpoint{1.221966in}{1.743348in}}{\pgfqpoint{1.227790in}{1.749172in}}%
\pgfpathcurveto{\pgfqpoint{1.233614in}{1.754995in}}{\pgfqpoint{1.236887in}{1.762896in}}{\pgfqpoint{1.236887in}{1.771132in}}%
\pgfpathcurveto{\pgfqpoint{1.236887in}{1.779368in}}{\pgfqpoint{1.233614in}{1.787268in}}{\pgfqpoint{1.227790in}{1.793092in}}%
\pgfpathcurveto{\pgfqpoint{1.221966in}{1.798916in}}{\pgfqpoint{1.214066in}{1.802188in}}{\pgfqpoint{1.205830in}{1.802188in}}%
\pgfpathcurveto{\pgfqpoint{1.197594in}{1.802188in}}{\pgfqpoint{1.189694in}{1.798916in}}{\pgfqpoint{1.183870in}{1.793092in}}%
\pgfpathcurveto{\pgfqpoint{1.178046in}{1.787268in}}{\pgfqpoint{1.174774in}{1.779368in}}{\pgfqpoint{1.174774in}{1.771132in}}%
\pgfpathcurveto{\pgfqpoint{1.174774in}{1.762896in}}{\pgfqpoint{1.178046in}{1.754995in}}{\pgfqpoint{1.183870in}{1.749172in}}%
\pgfpathcurveto{\pgfqpoint{1.189694in}{1.743348in}}{\pgfqpoint{1.197594in}{1.740075in}}{\pgfqpoint{1.205830in}{1.740075in}}%
\pgfpathclose%
\pgfusepath{stroke,fill}%
\end{pgfscope}%
\begin{pgfscope}%
\pgfpathrectangle{\pgfqpoint{0.100000in}{0.212622in}}{\pgfqpoint{3.696000in}{3.696000in}}%
\pgfusepath{clip}%
\pgfsetbuttcap%
\pgfsetroundjoin%
\definecolor{currentfill}{rgb}{0.121569,0.466667,0.705882}%
\pgfsetfillcolor{currentfill}%
\pgfsetfillopacity{0.367371}%
\pgfsetlinewidth{1.003750pt}%
\definecolor{currentstroke}{rgb}{0.121569,0.466667,0.705882}%
\pgfsetstrokecolor{currentstroke}%
\pgfsetstrokeopacity{0.367371}%
\pgfsetdash{}{0pt}%
\pgfpathmoveto{\pgfqpoint{1.205830in}{1.740075in}}%
\pgfpathcurveto{\pgfqpoint{1.214066in}{1.740075in}}{\pgfqpoint{1.221966in}{1.743348in}}{\pgfqpoint{1.227790in}{1.749172in}}%
\pgfpathcurveto{\pgfqpoint{1.233614in}{1.754995in}}{\pgfqpoint{1.236887in}{1.762896in}}{\pgfqpoint{1.236887in}{1.771132in}}%
\pgfpathcurveto{\pgfqpoint{1.236887in}{1.779368in}}{\pgfqpoint{1.233614in}{1.787268in}}{\pgfqpoint{1.227790in}{1.793092in}}%
\pgfpathcurveto{\pgfqpoint{1.221966in}{1.798916in}}{\pgfqpoint{1.214066in}{1.802188in}}{\pgfqpoint{1.205830in}{1.802188in}}%
\pgfpathcurveto{\pgfqpoint{1.197594in}{1.802188in}}{\pgfqpoint{1.189694in}{1.798916in}}{\pgfqpoint{1.183870in}{1.793092in}}%
\pgfpathcurveto{\pgfqpoint{1.178046in}{1.787268in}}{\pgfqpoint{1.174774in}{1.779368in}}{\pgfqpoint{1.174774in}{1.771132in}}%
\pgfpathcurveto{\pgfqpoint{1.174774in}{1.762896in}}{\pgfqpoint{1.178046in}{1.754995in}}{\pgfqpoint{1.183870in}{1.749172in}}%
\pgfpathcurveto{\pgfqpoint{1.189694in}{1.743348in}}{\pgfqpoint{1.197594in}{1.740075in}}{\pgfqpoint{1.205830in}{1.740075in}}%
\pgfpathclose%
\pgfusepath{stroke,fill}%
\end{pgfscope}%
\begin{pgfscope}%
\pgfpathrectangle{\pgfqpoint{0.100000in}{0.212622in}}{\pgfqpoint{3.696000in}{3.696000in}}%
\pgfusepath{clip}%
\pgfsetbuttcap%
\pgfsetroundjoin%
\definecolor{currentfill}{rgb}{0.121569,0.466667,0.705882}%
\pgfsetfillcolor{currentfill}%
\pgfsetfillopacity{0.367371}%
\pgfsetlinewidth{1.003750pt}%
\definecolor{currentstroke}{rgb}{0.121569,0.466667,0.705882}%
\pgfsetstrokecolor{currentstroke}%
\pgfsetstrokeopacity{0.367371}%
\pgfsetdash{}{0pt}%
\pgfpathmoveto{\pgfqpoint{1.205830in}{1.740075in}}%
\pgfpathcurveto{\pgfqpoint{1.214066in}{1.740075in}}{\pgfqpoint{1.221966in}{1.743348in}}{\pgfqpoint{1.227790in}{1.749172in}}%
\pgfpathcurveto{\pgfqpoint{1.233614in}{1.754995in}}{\pgfqpoint{1.236887in}{1.762896in}}{\pgfqpoint{1.236887in}{1.771132in}}%
\pgfpathcurveto{\pgfqpoint{1.236887in}{1.779368in}}{\pgfqpoint{1.233614in}{1.787268in}}{\pgfqpoint{1.227790in}{1.793092in}}%
\pgfpathcurveto{\pgfqpoint{1.221966in}{1.798916in}}{\pgfqpoint{1.214066in}{1.802188in}}{\pgfqpoint{1.205830in}{1.802188in}}%
\pgfpathcurveto{\pgfqpoint{1.197594in}{1.802188in}}{\pgfqpoint{1.189694in}{1.798916in}}{\pgfqpoint{1.183870in}{1.793092in}}%
\pgfpathcurveto{\pgfqpoint{1.178046in}{1.787268in}}{\pgfqpoint{1.174774in}{1.779368in}}{\pgfqpoint{1.174774in}{1.771132in}}%
\pgfpathcurveto{\pgfqpoint{1.174774in}{1.762896in}}{\pgfqpoint{1.178046in}{1.754995in}}{\pgfqpoint{1.183870in}{1.749172in}}%
\pgfpathcurveto{\pgfqpoint{1.189694in}{1.743348in}}{\pgfqpoint{1.197594in}{1.740075in}}{\pgfqpoint{1.205830in}{1.740075in}}%
\pgfpathclose%
\pgfusepath{stroke,fill}%
\end{pgfscope}%
\begin{pgfscope}%
\pgfpathrectangle{\pgfqpoint{0.100000in}{0.212622in}}{\pgfqpoint{3.696000in}{3.696000in}}%
\pgfusepath{clip}%
\pgfsetbuttcap%
\pgfsetroundjoin%
\definecolor{currentfill}{rgb}{0.121569,0.466667,0.705882}%
\pgfsetfillcolor{currentfill}%
\pgfsetfillopacity{0.367371}%
\pgfsetlinewidth{1.003750pt}%
\definecolor{currentstroke}{rgb}{0.121569,0.466667,0.705882}%
\pgfsetstrokecolor{currentstroke}%
\pgfsetstrokeopacity{0.367371}%
\pgfsetdash{}{0pt}%
\pgfpathmoveto{\pgfqpoint{1.205830in}{1.740075in}}%
\pgfpathcurveto{\pgfqpoint{1.214066in}{1.740075in}}{\pgfqpoint{1.221966in}{1.743348in}}{\pgfqpoint{1.227790in}{1.749172in}}%
\pgfpathcurveto{\pgfqpoint{1.233614in}{1.754995in}}{\pgfqpoint{1.236887in}{1.762896in}}{\pgfqpoint{1.236887in}{1.771132in}}%
\pgfpathcurveto{\pgfqpoint{1.236887in}{1.779368in}}{\pgfqpoint{1.233614in}{1.787268in}}{\pgfqpoint{1.227790in}{1.793092in}}%
\pgfpathcurveto{\pgfqpoint{1.221966in}{1.798916in}}{\pgfqpoint{1.214066in}{1.802188in}}{\pgfqpoint{1.205830in}{1.802188in}}%
\pgfpathcurveto{\pgfqpoint{1.197594in}{1.802188in}}{\pgfqpoint{1.189694in}{1.798916in}}{\pgfqpoint{1.183870in}{1.793092in}}%
\pgfpathcurveto{\pgfqpoint{1.178046in}{1.787268in}}{\pgfqpoint{1.174774in}{1.779368in}}{\pgfqpoint{1.174774in}{1.771132in}}%
\pgfpathcurveto{\pgfqpoint{1.174774in}{1.762896in}}{\pgfqpoint{1.178046in}{1.754995in}}{\pgfqpoint{1.183870in}{1.749172in}}%
\pgfpathcurveto{\pgfqpoint{1.189694in}{1.743348in}}{\pgfqpoint{1.197594in}{1.740075in}}{\pgfqpoint{1.205830in}{1.740075in}}%
\pgfpathclose%
\pgfusepath{stroke,fill}%
\end{pgfscope}%
\begin{pgfscope}%
\pgfpathrectangle{\pgfqpoint{0.100000in}{0.212622in}}{\pgfqpoint{3.696000in}{3.696000in}}%
\pgfusepath{clip}%
\pgfsetbuttcap%
\pgfsetroundjoin%
\definecolor{currentfill}{rgb}{0.121569,0.466667,0.705882}%
\pgfsetfillcolor{currentfill}%
\pgfsetfillopacity{0.367371}%
\pgfsetlinewidth{1.003750pt}%
\definecolor{currentstroke}{rgb}{0.121569,0.466667,0.705882}%
\pgfsetstrokecolor{currentstroke}%
\pgfsetstrokeopacity{0.367371}%
\pgfsetdash{}{0pt}%
\pgfpathmoveto{\pgfqpoint{1.205830in}{1.740075in}}%
\pgfpathcurveto{\pgfqpoint{1.214066in}{1.740075in}}{\pgfqpoint{1.221966in}{1.743348in}}{\pgfqpoint{1.227790in}{1.749172in}}%
\pgfpathcurveto{\pgfqpoint{1.233614in}{1.754995in}}{\pgfqpoint{1.236887in}{1.762896in}}{\pgfqpoint{1.236887in}{1.771132in}}%
\pgfpathcurveto{\pgfqpoint{1.236887in}{1.779368in}}{\pgfqpoint{1.233614in}{1.787268in}}{\pgfqpoint{1.227790in}{1.793092in}}%
\pgfpathcurveto{\pgfqpoint{1.221966in}{1.798916in}}{\pgfqpoint{1.214066in}{1.802188in}}{\pgfqpoint{1.205830in}{1.802188in}}%
\pgfpathcurveto{\pgfqpoint{1.197594in}{1.802188in}}{\pgfqpoint{1.189694in}{1.798916in}}{\pgfqpoint{1.183870in}{1.793092in}}%
\pgfpathcurveto{\pgfqpoint{1.178046in}{1.787268in}}{\pgfqpoint{1.174774in}{1.779368in}}{\pgfqpoint{1.174774in}{1.771132in}}%
\pgfpathcurveto{\pgfqpoint{1.174774in}{1.762896in}}{\pgfqpoint{1.178046in}{1.754995in}}{\pgfqpoint{1.183870in}{1.749172in}}%
\pgfpathcurveto{\pgfqpoint{1.189694in}{1.743348in}}{\pgfqpoint{1.197594in}{1.740075in}}{\pgfqpoint{1.205830in}{1.740075in}}%
\pgfpathclose%
\pgfusepath{stroke,fill}%
\end{pgfscope}%
\begin{pgfscope}%
\pgfpathrectangle{\pgfqpoint{0.100000in}{0.212622in}}{\pgfqpoint{3.696000in}{3.696000in}}%
\pgfusepath{clip}%
\pgfsetbuttcap%
\pgfsetroundjoin%
\definecolor{currentfill}{rgb}{0.121569,0.466667,0.705882}%
\pgfsetfillcolor{currentfill}%
\pgfsetfillopacity{0.367371}%
\pgfsetlinewidth{1.003750pt}%
\definecolor{currentstroke}{rgb}{0.121569,0.466667,0.705882}%
\pgfsetstrokecolor{currentstroke}%
\pgfsetstrokeopacity{0.367371}%
\pgfsetdash{}{0pt}%
\pgfpathmoveto{\pgfqpoint{1.205830in}{1.740075in}}%
\pgfpathcurveto{\pgfqpoint{1.214066in}{1.740075in}}{\pgfqpoint{1.221966in}{1.743348in}}{\pgfqpoint{1.227790in}{1.749172in}}%
\pgfpathcurveto{\pgfqpoint{1.233614in}{1.754995in}}{\pgfqpoint{1.236887in}{1.762896in}}{\pgfqpoint{1.236887in}{1.771132in}}%
\pgfpathcurveto{\pgfqpoint{1.236887in}{1.779368in}}{\pgfqpoint{1.233614in}{1.787268in}}{\pgfqpoint{1.227790in}{1.793092in}}%
\pgfpathcurveto{\pgfqpoint{1.221966in}{1.798916in}}{\pgfqpoint{1.214066in}{1.802188in}}{\pgfqpoint{1.205830in}{1.802188in}}%
\pgfpathcurveto{\pgfqpoint{1.197594in}{1.802188in}}{\pgfqpoint{1.189694in}{1.798916in}}{\pgfqpoint{1.183870in}{1.793092in}}%
\pgfpathcurveto{\pgfqpoint{1.178046in}{1.787268in}}{\pgfqpoint{1.174774in}{1.779368in}}{\pgfqpoint{1.174774in}{1.771132in}}%
\pgfpathcurveto{\pgfqpoint{1.174774in}{1.762896in}}{\pgfqpoint{1.178046in}{1.754995in}}{\pgfqpoint{1.183870in}{1.749172in}}%
\pgfpathcurveto{\pgfqpoint{1.189694in}{1.743348in}}{\pgfqpoint{1.197594in}{1.740075in}}{\pgfqpoint{1.205830in}{1.740075in}}%
\pgfpathclose%
\pgfusepath{stroke,fill}%
\end{pgfscope}%
\begin{pgfscope}%
\pgfpathrectangle{\pgfqpoint{0.100000in}{0.212622in}}{\pgfqpoint{3.696000in}{3.696000in}}%
\pgfusepath{clip}%
\pgfsetbuttcap%
\pgfsetroundjoin%
\definecolor{currentfill}{rgb}{0.121569,0.466667,0.705882}%
\pgfsetfillcolor{currentfill}%
\pgfsetfillopacity{0.367371}%
\pgfsetlinewidth{1.003750pt}%
\definecolor{currentstroke}{rgb}{0.121569,0.466667,0.705882}%
\pgfsetstrokecolor{currentstroke}%
\pgfsetstrokeopacity{0.367371}%
\pgfsetdash{}{0pt}%
\pgfpathmoveto{\pgfqpoint{1.205830in}{1.740075in}}%
\pgfpathcurveto{\pgfqpoint{1.214066in}{1.740075in}}{\pgfqpoint{1.221966in}{1.743348in}}{\pgfqpoint{1.227790in}{1.749172in}}%
\pgfpathcurveto{\pgfqpoint{1.233614in}{1.754995in}}{\pgfqpoint{1.236887in}{1.762896in}}{\pgfqpoint{1.236887in}{1.771132in}}%
\pgfpathcurveto{\pgfqpoint{1.236887in}{1.779368in}}{\pgfqpoint{1.233614in}{1.787268in}}{\pgfqpoint{1.227790in}{1.793092in}}%
\pgfpathcurveto{\pgfqpoint{1.221966in}{1.798916in}}{\pgfqpoint{1.214066in}{1.802188in}}{\pgfqpoint{1.205830in}{1.802188in}}%
\pgfpathcurveto{\pgfqpoint{1.197594in}{1.802188in}}{\pgfqpoint{1.189694in}{1.798916in}}{\pgfqpoint{1.183870in}{1.793092in}}%
\pgfpathcurveto{\pgfqpoint{1.178046in}{1.787268in}}{\pgfqpoint{1.174774in}{1.779368in}}{\pgfqpoint{1.174774in}{1.771132in}}%
\pgfpathcurveto{\pgfqpoint{1.174774in}{1.762896in}}{\pgfqpoint{1.178046in}{1.754995in}}{\pgfqpoint{1.183870in}{1.749172in}}%
\pgfpathcurveto{\pgfqpoint{1.189694in}{1.743348in}}{\pgfqpoint{1.197594in}{1.740075in}}{\pgfqpoint{1.205830in}{1.740075in}}%
\pgfpathclose%
\pgfusepath{stroke,fill}%
\end{pgfscope}%
\begin{pgfscope}%
\pgfpathrectangle{\pgfqpoint{0.100000in}{0.212622in}}{\pgfqpoint{3.696000in}{3.696000in}}%
\pgfusepath{clip}%
\pgfsetbuttcap%
\pgfsetroundjoin%
\definecolor{currentfill}{rgb}{0.121569,0.466667,0.705882}%
\pgfsetfillcolor{currentfill}%
\pgfsetfillopacity{0.367371}%
\pgfsetlinewidth{1.003750pt}%
\definecolor{currentstroke}{rgb}{0.121569,0.466667,0.705882}%
\pgfsetstrokecolor{currentstroke}%
\pgfsetstrokeopacity{0.367371}%
\pgfsetdash{}{0pt}%
\pgfpathmoveto{\pgfqpoint{1.205830in}{1.740075in}}%
\pgfpathcurveto{\pgfqpoint{1.214066in}{1.740075in}}{\pgfqpoint{1.221966in}{1.743348in}}{\pgfqpoint{1.227790in}{1.749172in}}%
\pgfpathcurveto{\pgfqpoint{1.233614in}{1.754995in}}{\pgfqpoint{1.236887in}{1.762896in}}{\pgfqpoint{1.236887in}{1.771132in}}%
\pgfpathcurveto{\pgfqpoint{1.236887in}{1.779368in}}{\pgfqpoint{1.233614in}{1.787268in}}{\pgfqpoint{1.227790in}{1.793092in}}%
\pgfpathcurveto{\pgfqpoint{1.221966in}{1.798916in}}{\pgfqpoint{1.214066in}{1.802188in}}{\pgfqpoint{1.205830in}{1.802188in}}%
\pgfpathcurveto{\pgfqpoint{1.197594in}{1.802188in}}{\pgfqpoint{1.189694in}{1.798916in}}{\pgfqpoint{1.183870in}{1.793092in}}%
\pgfpathcurveto{\pgfqpoint{1.178046in}{1.787268in}}{\pgfqpoint{1.174774in}{1.779368in}}{\pgfqpoint{1.174774in}{1.771132in}}%
\pgfpathcurveto{\pgfqpoint{1.174774in}{1.762896in}}{\pgfqpoint{1.178046in}{1.754995in}}{\pgfqpoint{1.183870in}{1.749172in}}%
\pgfpathcurveto{\pgfqpoint{1.189694in}{1.743348in}}{\pgfqpoint{1.197594in}{1.740075in}}{\pgfqpoint{1.205830in}{1.740075in}}%
\pgfpathclose%
\pgfusepath{stroke,fill}%
\end{pgfscope}%
\begin{pgfscope}%
\pgfpathrectangle{\pgfqpoint{0.100000in}{0.212622in}}{\pgfqpoint{3.696000in}{3.696000in}}%
\pgfusepath{clip}%
\pgfsetbuttcap%
\pgfsetroundjoin%
\definecolor{currentfill}{rgb}{0.121569,0.466667,0.705882}%
\pgfsetfillcolor{currentfill}%
\pgfsetfillopacity{0.367371}%
\pgfsetlinewidth{1.003750pt}%
\definecolor{currentstroke}{rgb}{0.121569,0.466667,0.705882}%
\pgfsetstrokecolor{currentstroke}%
\pgfsetstrokeopacity{0.367371}%
\pgfsetdash{}{0pt}%
\pgfpathmoveto{\pgfqpoint{1.205830in}{1.740075in}}%
\pgfpathcurveto{\pgfqpoint{1.214066in}{1.740075in}}{\pgfqpoint{1.221966in}{1.743348in}}{\pgfqpoint{1.227790in}{1.749172in}}%
\pgfpathcurveto{\pgfqpoint{1.233614in}{1.754995in}}{\pgfqpoint{1.236887in}{1.762896in}}{\pgfqpoint{1.236887in}{1.771132in}}%
\pgfpathcurveto{\pgfqpoint{1.236887in}{1.779368in}}{\pgfqpoint{1.233614in}{1.787268in}}{\pgfqpoint{1.227790in}{1.793092in}}%
\pgfpathcurveto{\pgfqpoint{1.221966in}{1.798916in}}{\pgfqpoint{1.214066in}{1.802188in}}{\pgfqpoint{1.205830in}{1.802188in}}%
\pgfpathcurveto{\pgfqpoint{1.197594in}{1.802188in}}{\pgfqpoint{1.189694in}{1.798916in}}{\pgfqpoint{1.183870in}{1.793092in}}%
\pgfpathcurveto{\pgfqpoint{1.178046in}{1.787268in}}{\pgfqpoint{1.174774in}{1.779368in}}{\pgfqpoint{1.174774in}{1.771132in}}%
\pgfpathcurveto{\pgfqpoint{1.174774in}{1.762896in}}{\pgfqpoint{1.178046in}{1.754995in}}{\pgfqpoint{1.183870in}{1.749172in}}%
\pgfpathcurveto{\pgfqpoint{1.189694in}{1.743348in}}{\pgfqpoint{1.197594in}{1.740075in}}{\pgfqpoint{1.205830in}{1.740075in}}%
\pgfpathclose%
\pgfusepath{stroke,fill}%
\end{pgfscope}%
\begin{pgfscope}%
\pgfpathrectangle{\pgfqpoint{0.100000in}{0.212622in}}{\pgfqpoint{3.696000in}{3.696000in}}%
\pgfusepath{clip}%
\pgfsetbuttcap%
\pgfsetroundjoin%
\definecolor{currentfill}{rgb}{0.121569,0.466667,0.705882}%
\pgfsetfillcolor{currentfill}%
\pgfsetfillopacity{0.367371}%
\pgfsetlinewidth{1.003750pt}%
\definecolor{currentstroke}{rgb}{0.121569,0.466667,0.705882}%
\pgfsetstrokecolor{currentstroke}%
\pgfsetstrokeopacity{0.367371}%
\pgfsetdash{}{0pt}%
\pgfpathmoveto{\pgfqpoint{1.205830in}{1.740075in}}%
\pgfpathcurveto{\pgfqpoint{1.214066in}{1.740075in}}{\pgfqpoint{1.221966in}{1.743348in}}{\pgfqpoint{1.227790in}{1.749172in}}%
\pgfpathcurveto{\pgfqpoint{1.233614in}{1.754995in}}{\pgfqpoint{1.236887in}{1.762896in}}{\pgfqpoint{1.236887in}{1.771132in}}%
\pgfpathcurveto{\pgfqpoint{1.236887in}{1.779368in}}{\pgfqpoint{1.233614in}{1.787268in}}{\pgfqpoint{1.227790in}{1.793092in}}%
\pgfpathcurveto{\pgfqpoint{1.221966in}{1.798916in}}{\pgfqpoint{1.214066in}{1.802188in}}{\pgfqpoint{1.205830in}{1.802188in}}%
\pgfpathcurveto{\pgfqpoint{1.197594in}{1.802188in}}{\pgfqpoint{1.189694in}{1.798916in}}{\pgfqpoint{1.183870in}{1.793092in}}%
\pgfpathcurveto{\pgfqpoint{1.178046in}{1.787268in}}{\pgfqpoint{1.174774in}{1.779368in}}{\pgfqpoint{1.174774in}{1.771132in}}%
\pgfpathcurveto{\pgfqpoint{1.174774in}{1.762896in}}{\pgfqpoint{1.178046in}{1.754995in}}{\pgfqpoint{1.183870in}{1.749172in}}%
\pgfpathcurveto{\pgfqpoint{1.189694in}{1.743348in}}{\pgfqpoint{1.197594in}{1.740075in}}{\pgfqpoint{1.205830in}{1.740075in}}%
\pgfpathclose%
\pgfusepath{stroke,fill}%
\end{pgfscope}%
\begin{pgfscope}%
\pgfpathrectangle{\pgfqpoint{0.100000in}{0.212622in}}{\pgfqpoint{3.696000in}{3.696000in}}%
\pgfusepath{clip}%
\pgfsetbuttcap%
\pgfsetroundjoin%
\definecolor{currentfill}{rgb}{0.121569,0.466667,0.705882}%
\pgfsetfillcolor{currentfill}%
\pgfsetfillopacity{0.367371}%
\pgfsetlinewidth{1.003750pt}%
\definecolor{currentstroke}{rgb}{0.121569,0.466667,0.705882}%
\pgfsetstrokecolor{currentstroke}%
\pgfsetstrokeopacity{0.367371}%
\pgfsetdash{}{0pt}%
\pgfpathmoveto{\pgfqpoint{1.205830in}{1.740075in}}%
\pgfpathcurveto{\pgfqpoint{1.214066in}{1.740075in}}{\pgfqpoint{1.221966in}{1.743348in}}{\pgfqpoint{1.227790in}{1.749172in}}%
\pgfpathcurveto{\pgfqpoint{1.233614in}{1.754995in}}{\pgfqpoint{1.236887in}{1.762896in}}{\pgfqpoint{1.236887in}{1.771132in}}%
\pgfpathcurveto{\pgfqpoint{1.236887in}{1.779368in}}{\pgfqpoint{1.233614in}{1.787268in}}{\pgfqpoint{1.227790in}{1.793092in}}%
\pgfpathcurveto{\pgfqpoint{1.221966in}{1.798916in}}{\pgfqpoint{1.214066in}{1.802188in}}{\pgfqpoint{1.205830in}{1.802188in}}%
\pgfpathcurveto{\pgfqpoint{1.197594in}{1.802188in}}{\pgfqpoint{1.189694in}{1.798916in}}{\pgfqpoint{1.183870in}{1.793092in}}%
\pgfpathcurveto{\pgfqpoint{1.178046in}{1.787268in}}{\pgfqpoint{1.174774in}{1.779368in}}{\pgfqpoint{1.174774in}{1.771132in}}%
\pgfpathcurveto{\pgfqpoint{1.174774in}{1.762896in}}{\pgfqpoint{1.178046in}{1.754995in}}{\pgfqpoint{1.183870in}{1.749172in}}%
\pgfpathcurveto{\pgfqpoint{1.189694in}{1.743348in}}{\pgfqpoint{1.197594in}{1.740075in}}{\pgfqpoint{1.205830in}{1.740075in}}%
\pgfpathclose%
\pgfusepath{stroke,fill}%
\end{pgfscope}%
\begin{pgfscope}%
\pgfpathrectangle{\pgfqpoint{0.100000in}{0.212622in}}{\pgfqpoint{3.696000in}{3.696000in}}%
\pgfusepath{clip}%
\pgfsetbuttcap%
\pgfsetroundjoin%
\definecolor{currentfill}{rgb}{0.121569,0.466667,0.705882}%
\pgfsetfillcolor{currentfill}%
\pgfsetfillopacity{0.367371}%
\pgfsetlinewidth{1.003750pt}%
\definecolor{currentstroke}{rgb}{0.121569,0.466667,0.705882}%
\pgfsetstrokecolor{currentstroke}%
\pgfsetstrokeopacity{0.367371}%
\pgfsetdash{}{0pt}%
\pgfpathmoveto{\pgfqpoint{1.205830in}{1.740075in}}%
\pgfpathcurveto{\pgfqpoint{1.214066in}{1.740075in}}{\pgfqpoint{1.221966in}{1.743348in}}{\pgfqpoint{1.227790in}{1.749172in}}%
\pgfpathcurveto{\pgfqpoint{1.233614in}{1.754995in}}{\pgfqpoint{1.236887in}{1.762896in}}{\pgfqpoint{1.236887in}{1.771132in}}%
\pgfpathcurveto{\pgfqpoint{1.236887in}{1.779368in}}{\pgfqpoint{1.233614in}{1.787268in}}{\pgfqpoint{1.227790in}{1.793092in}}%
\pgfpathcurveto{\pgfqpoint{1.221966in}{1.798916in}}{\pgfqpoint{1.214066in}{1.802188in}}{\pgfqpoint{1.205830in}{1.802188in}}%
\pgfpathcurveto{\pgfqpoint{1.197594in}{1.802188in}}{\pgfqpoint{1.189694in}{1.798916in}}{\pgfqpoint{1.183870in}{1.793092in}}%
\pgfpathcurveto{\pgfqpoint{1.178046in}{1.787268in}}{\pgfqpoint{1.174774in}{1.779368in}}{\pgfqpoint{1.174774in}{1.771132in}}%
\pgfpathcurveto{\pgfqpoint{1.174774in}{1.762896in}}{\pgfqpoint{1.178046in}{1.754995in}}{\pgfqpoint{1.183870in}{1.749172in}}%
\pgfpathcurveto{\pgfqpoint{1.189694in}{1.743348in}}{\pgfqpoint{1.197594in}{1.740075in}}{\pgfqpoint{1.205830in}{1.740075in}}%
\pgfpathclose%
\pgfusepath{stroke,fill}%
\end{pgfscope}%
\begin{pgfscope}%
\pgfpathrectangle{\pgfqpoint{0.100000in}{0.212622in}}{\pgfqpoint{3.696000in}{3.696000in}}%
\pgfusepath{clip}%
\pgfsetbuttcap%
\pgfsetroundjoin%
\definecolor{currentfill}{rgb}{0.121569,0.466667,0.705882}%
\pgfsetfillcolor{currentfill}%
\pgfsetfillopacity{0.367371}%
\pgfsetlinewidth{1.003750pt}%
\definecolor{currentstroke}{rgb}{0.121569,0.466667,0.705882}%
\pgfsetstrokecolor{currentstroke}%
\pgfsetstrokeopacity{0.367371}%
\pgfsetdash{}{0pt}%
\pgfpathmoveto{\pgfqpoint{1.205830in}{1.740075in}}%
\pgfpathcurveto{\pgfqpoint{1.214066in}{1.740075in}}{\pgfqpoint{1.221966in}{1.743348in}}{\pgfqpoint{1.227790in}{1.749172in}}%
\pgfpathcurveto{\pgfqpoint{1.233614in}{1.754995in}}{\pgfqpoint{1.236887in}{1.762896in}}{\pgfqpoint{1.236887in}{1.771132in}}%
\pgfpathcurveto{\pgfqpoint{1.236887in}{1.779368in}}{\pgfqpoint{1.233614in}{1.787268in}}{\pgfqpoint{1.227790in}{1.793092in}}%
\pgfpathcurveto{\pgfqpoint{1.221966in}{1.798916in}}{\pgfqpoint{1.214066in}{1.802188in}}{\pgfqpoint{1.205830in}{1.802188in}}%
\pgfpathcurveto{\pgfqpoint{1.197594in}{1.802188in}}{\pgfqpoint{1.189694in}{1.798916in}}{\pgfqpoint{1.183870in}{1.793092in}}%
\pgfpathcurveto{\pgfqpoint{1.178046in}{1.787268in}}{\pgfqpoint{1.174774in}{1.779368in}}{\pgfqpoint{1.174774in}{1.771132in}}%
\pgfpathcurveto{\pgfqpoint{1.174774in}{1.762896in}}{\pgfqpoint{1.178046in}{1.754995in}}{\pgfqpoint{1.183870in}{1.749172in}}%
\pgfpathcurveto{\pgfqpoint{1.189694in}{1.743348in}}{\pgfqpoint{1.197594in}{1.740075in}}{\pgfqpoint{1.205830in}{1.740075in}}%
\pgfpathclose%
\pgfusepath{stroke,fill}%
\end{pgfscope}%
\begin{pgfscope}%
\pgfpathrectangle{\pgfqpoint{0.100000in}{0.212622in}}{\pgfqpoint{3.696000in}{3.696000in}}%
\pgfusepath{clip}%
\pgfsetbuttcap%
\pgfsetroundjoin%
\definecolor{currentfill}{rgb}{0.121569,0.466667,0.705882}%
\pgfsetfillcolor{currentfill}%
\pgfsetfillopacity{0.367371}%
\pgfsetlinewidth{1.003750pt}%
\definecolor{currentstroke}{rgb}{0.121569,0.466667,0.705882}%
\pgfsetstrokecolor{currentstroke}%
\pgfsetstrokeopacity{0.367371}%
\pgfsetdash{}{0pt}%
\pgfpathmoveto{\pgfqpoint{1.205830in}{1.740075in}}%
\pgfpathcurveto{\pgfqpoint{1.214066in}{1.740075in}}{\pgfqpoint{1.221966in}{1.743348in}}{\pgfqpoint{1.227790in}{1.749172in}}%
\pgfpathcurveto{\pgfqpoint{1.233614in}{1.754995in}}{\pgfqpoint{1.236887in}{1.762896in}}{\pgfqpoint{1.236887in}{1.771132in}}%
\pgfpathcurveto{\pgfqpoint{1.236887in}{1.779368in}}{\pgfqpoint{1.233614in}{1.787268in}}{\pgfqpoint{1.227790in}{1.793092in}}%
\pgfpathcurveto{\pgfqpoint{1.221966in}{1.798916in}}{\pgfqpoint{1.214066in}{1.802188in}}{\pgfqpoint{1.205830in}{1.802188in}}%
\pgfpathcurveto{\pgfqpoint{1.197594in}{1.802188in}}{\pgfqpoint{1.189694in}{1.798916in}}{\pgfqpoint{1.183870in}{1.793092in}}%
\pgfpathcurveto{\pgfqpoint{1.178046in}{1.787268in}}{\pgfqpoint{1.174774in}{1.779368in}}{\pgfqpoint{1.174774in}{1.771132in}}%
\pgfpathcurveto{\pgfqpoint{1.174774in}{1.762896in}}{\pgfqpoint{1.178046in}{1.754995in}}{\pgfqpoint{1.183870in}{1.749172in}}%
\pgfpathcurveto{\pgfqpoint{1.189694in}{1.743348in}}{\pgfqpoint{1.197594in}{1.740075in}}{\pgfqpoint{1.205830in}{1.740075in}}%
\pgfpathclose%
\pgfusepath{stroke,fill}%
\end{pgfscope}%
\begin{pgfscope}%
\pgfpathrectangle{\pgfqpoint{0.100000in}{0.212622in}}{\pgfqpoint{3.696000in}{3.696000in}}%
\pgfusepath{clip}%
\pgfsetbuttcap%
\pgfsetroundjoin%
\definecolor{currentfill}{rgb}{0.121569,0.466667,0.705882}%
\pgfsetfillcolor{currentfill}%
\pgfsetfillopacity{0.367371}%
\pgfsetlinewidth{1.003750pt}%
\definecolor{currentstroke}{rgb}{0.121569,0.466667,0.705882}%
\pgfsetstrokecolor{currentstroke}%
\pgfsetstrokeopacity{0.367371}%
\pgfsetdash{}{0pt}%
\pgfpathmoveto{\pgfqpoint{1.205830in}{1.740075in}}%
\pgfpathcurveto{\pgfqpoint{1.214066in}{1.740075in}}{\pgfqpoint{1.221966in}{1.743348in}}{\pgfqpoint{1.227790in}{1.749172in}}%
\pgfpathcurveto{\pgfqpoint{1.233614in}{1.754995in}}{\pgfqpoint{1.236887in}{1.762896in}}{\pgfqpoint{1.236887in}{1.771132in}}%
\pgfpathcurveto{\pgfqpoint{1.236887in}{1.779368in}}{\pgfqpoint{1.233614in}{1.787268in}}{\pgfqpoint{1.227790in}{1.793092in}}%
\pgfpathcurveto{\pgfqpoint{1.221966in}{1.798916in}}{\pgfqpoint{1.214066in}{1.802188in}}{\pgfqpoint{1.205830in}{1.802188in}}%
\pgfpathcurveto{\pgfqpoint{1.197594in}{1.802188in}}{\pgfqpoint{1.189694in}{1.798916in}}{\pgfqpoint{1.183870in}{1.793092in}}%
\pgfpathcurveto{\pgfqpoint{1.178046in}{1.787268in}}{\pgfqpoint{1.174774in}{1.779368in}}{\pgfqpoint{1.174774in}{1.771132in}}%
\pgfpathcurveto{\pgfqpoint{1.174774in}{1.762896in}}{\pgfqpoint{1.178046in}{1.754995in}}{\pgfqpoint{1.183870in}{1.749172in}}%
\pgfpathcurveto{\pgfqpoint{1.189694in}{1.743348in}}{\pgfqpoint{1.197594in}{1.740075in}}{\pgfqpoint{1.205830in}{1.740075in}}%
\pgfpathclose%
\pgfusepath{stroke,fill}%
\end{pgfscope}%
\begin{pgfscope}%
\pgfpathrectangle{\pgfqpoint{0.100000in}{0.212622in}}{\pgfqpoint{3.696000in}{3.696000in}}%
\pgfusepath{clip}%
\pgfsetbuttcap%
\pgfsetroundjoin%
\definecolor{currentfill}{rgb}{0.121569,0.466667,0.705882}%
\pgfsetfillcolor{currentfill}%
\pgfsetfillopacity{0.367371}%
\pgfsetlinewidth{1.003750pt}%
\definecolor{currentstroke}{rgb}{0.121569,0.466667,0.705882}%
\pgfsetstrokecolor{currentstroke}%
\pgfsetstrokeopacity{0.367371}%
\pgfsetdash{}{0pt}%
\pgfpathmoveto{\pgfqpoint{1.205830in}{1.740075in}}%
\pgfpathcurveto{\pgfqpoint{1.214066in}{1.740075in}}{\pgfqpoint{1.221966in}{1.743348in}}{\pgfqpoint{1.227790in}{1.749172in}}%
\pgfpathcurveto{\pgfqpoint{1.233614in}{1.754995in}}{\pgfqpoint{1.236887in}{1.762896in}}{\pgfqpoint{1.236887in}{1.771132in}}%
\pgfpathcurveto{\pgfqpoint{1.236887in}{1.779368in}}{\pgfqpoint{1.233614in}{1.787268in}}{\pgfqpoint{1.227790in}{1.793092in}}%
\pgfpathcurveto{\pgfqpoint{1.221966in}{1.798916in}}{\pgfqpoint{1.214066in}{1.802188in}}{\pgfqpoint{1.205830in}{1.802188in}}%
\pgfpathcurveto{\pgfqpoint{1.197594in}{1.802188in}}{\pgfqpoint{1.189694in}{1.798916in}}{\pgfqpoint{1.183870in}{1.793092in}}%
\pgfpathcurveto{\pgfqpoint{1.178046in}{1.787268in}}{\pgfqpoint{1.174774in}{1.779368in}}{\pgfqpoint{1.174774in}{1.771132in}}%
\pgfpathcurveto{\pgfqpoint{1.174774in}{1.762896in}}{\pgfqpoint{1.178046in}{1.754995in}}{\pgfqpoint{1.183870in}{1.749172in}}%
\pgfpathcurveto{\pgfqpoint{1.189694in}{1.743348in}}{\pgfqpoint{1.197594in}{1.740075in}}{\pgfqpoint{1.205830in}{1.740075in}}%
\pgfpathclose%
\pgfusepath{stroke,fill}%
\end{pgfscope}%
\begin{pgfscope}%
\pgfpathrectangle{\pgfqpoint{0.100000in}{0.212622in}}{\pgfqpoint{3.696000in}{3.696000in}}%
\pgfusepath{clip}%
\pgfsetbuttcap%
\pgfsetroundjoin%
\definecolor{currentfill}{rgb}{0.121569,0.466667,0.705882}%
\pgfsetfillcolor{currentfill}%
\pgfsetfillopacity{0.367371}%
\pgfsetlinewidth{1.003750pt}%
\definecolor{currentstroke}{rgb}{0.121569,0.466667,0.705882}%
\pgfsetstrokecolor{currentstroke}%
\pgfsetstrokeopacity{0.367371}%
\pgfsetdash{}{0pt}%
\pgfpathmoveto{\pgfqpoint{1.205830in}{1.740075in}}%
\pgfpathcurveto{\pgfqpoint{1.214066in}{1.740075in}}{\pgfqpoint{1.221966in}{1.743348in}}{\pgfqpoint{1.227790in}{1.749172in}}%
\pgfpathcurveto{\pgfqpoint{1.233614in}{1.754995in}}{\pgfqpoint{1.236887in}{1.762896in}}{\pgfqpoint{1.236887in}{1.771132in}}%
\pgfpathcurveto{\pgfqpoint{1.236887in}{1.779368in}}{\pgfqpoint{1.233614in}{1.787268in}}{\pgfqpoint{1.227790in}{1.793092in}}%
\pgfpathcurveto{\pgfqpoint{1.221966in}{1.798916in}}{\pgfqpoint{1.214066in}{1.802188in}}{\pgfqpoint{1.205830in}{1.802188in}}%
\pgfpathcurveto{\pgfqpoint{1.197594in}{1.802188in}}{\pgfqpoint{1.189694in}{1.798916in}}{\pgfqpoint{1.183870in}{1.793092in}}%
\pgfpathcurveto{\pgfqpoint{1.178046in}{1.787268in}}{\pgfqpoint{1.174774in}{1.779368in}}{\pgfqpoint{1.174774in}{1.771132in}}%
\pgfpathcurveto{\pgfqpoint{1.174774in}{1.762896in}}{\pgfqpoint{1.178046in}{1.754995in}}{\pgfqpoint{1.183870in}{1.749172in}}%
\pgfpathcurveto{\pgfqpoint{1.189694in}{1.743348in}}{\pgfqpoint{1.197594in}{1.740075in}}{\pgfqpoint{1.205830in}{1.740075in}}%
\pgfpathclose%
\pgfusepath{stroke,fill}%
\end{pgfscope}%
\begin{pgfscope}%
\pgfpathrectangle{\pgfqpoint{0.100000in}{0.212622in}}{\pgfqpoint{3.696000in}{3.696000in}}%
\pgfusepath{clip}%
\pgfsetbuttcap%
\pgfsetroundjoin%
\definecolor{currentfill}{rgb}{0.121569,0.466667,0.705882}%
\pgfsetfillcolor{currentfill}%
\pgfsetfillopacity{0.367371}%
\pgfsetlinewidth{1.003750pt}%
\definecolor{currentstroke}{rgb}{0.121569,0.466667,0.705882}%
\pgfsetstrokecolor{currentstroke}%
\pgfsetstrokeopacity{0.367371}%
\pgfsetdash{}{0pt}%
\pgfpathmoveto{\pgfqpoint{1.205830in}{1.740075in}}%
\pgfpathcurveto{\pgfqpoint{1.214066in}{1.740075in}}{\pgfqpoint{1.221966in}{1.743348in}}{\pgfqpoint{1.227790in}{1.749172in}}%
\pgfpathcurveto{\pgfqpoint{1.233614in}{1.754995in}}{\pgfqpoint{1.236887in}{1.762896in}}{\pgfqpoint{1.236887in}{1.771132in}}%
\pgfpathcurveto{\pgfqpoint{1.236887in}{1.779368in}}{\pgfqpoint{1.233614in}{1.787268in}}{\pgfqpoint{1.227790in}{1.793092in}}%
\pgfpathcurveto{\pgfqpoint{1.221966in}{1.798916in}}{\pgfqpoint{1.214066in}{1.802188in}}{\pgfqpoint{1.205830in}{1.802188in}}%
\pgfpathcurveto{\pgfqpoint{1.197594in}{1.802188in}}{\pgfqpoint{1.189694in}{1.798916in}}{\pgfqpoint{1.183870in}{1.793092in}}%
\pgfpathcurveto{\pgfqpoint{1.178046in}{1.787268in}}{\pgfqpoint{1.174774in}{1.779368in}}{\pgfqpoint{1.174774in}{1.771132in}}%
\pgfpathcurveto{\pgfqpoint{1.174774in}{1.762896in}}{\pgfqpoint{1.178046in}{1.754995in}}{\pgfqpoint{1.183870in}{1.749172in}}%
\pgfpathcurveto{\pgfqpoint{1.189694in}{1.743348in}}{\pgfqpoint{1.197594in}{1.740075in}}{\pgfqpoint{1.205830in}{1.740075in}}%
\pgfpathclose%
\pgfusepath{stroke,fill}%
\end{pgfscope}%
\begin{pgfscope}%
\pgfpathrectangle{\pgfqpoint{0.100000in}{0.212622in}}{\pgfqpoint{3.696000in}{3.696000in}}%
\pgfusepath{clip}%
\pgfsetbuttcap%
\pgfsetroundjoin%
\definecolor{currentfill}{rgb}{0.121569,0.466667,0.705882}%
\pgfsetfillcolor{currentfill}%
\pgfsetfillopacity{0.367371}%
\pgfsetlinewidth{1.003750pt}%
\definecolor{currentstroke}{rgb}{0.121569,0.466667,0.705882}%
\pgfsetstrokecolor{currentstroke}%
\pgfsetstrokeopacity{0.367371}%
\pgfsetdash{}{0pt}%
\pgfpathmoveto{\pgfqpoint{1.205830in}{1.740075in}}%
\pgfpathcurveto{\pgfqpoint{1.214066in}{1.740075in}}{\pgfqpoint{1.221966in}{1.743348in}}{\pgfqpoint{1.227790in}{1.749172in}}%
\pgfpathcurveto{\pgfqpoint{1.233614in}{1.754995in}}{\pgfqpoint{1.236887in}{1.762896in}}{\pgfqpoint{1.236887in}{1.771132in}}%
\pgfpathcurveto{\pgfqpoint{1.236887in}{1.779368in}}{\pgfqpoint{1.233614in}{1.787268in}}{\pgfqpoint{1.227790in}{1.793092in}}%
\pgfpathcurveto{\pgfqpoint{1.221966in}{1.798916in}}{\pgfqpoint{1.214066in}{1.802188in}}{\pgfqpoint{1.205830in}{1.802188in}}%
\pgfpathcurveto{\pgfqpoint{1.197594in}{1.802188in}}{\pgfqpoint{1.189694in}{1.798916in}}{\pgfqpoint{1.183870in}{1.793092in}}%
\pgfpathcurveto{\pgfqpoint{1.178046in}{1.787268in}}{\pgfqpoint{1.174774in}{1.779368in}}{\pgfqpoint{1.174774in}{1.771132in}}%
\pgfpathcurveto{\pgfqpoint{1.174774in}{1.762896in}}{\pgfqpoint{1.178046in}{1.754995in}}{\pgfqpoint{1.183870in}{1.749172in}}%
\pgfpathcurveto{\pgfqpoint{1.189694in}{1.743348in}}{\pgfqpoint{1.197594in}{1.740075in}}{\pgfqpoint{1.205830in}{1.740075in}}%
\pgfpathclose%
\pgfusepath{stroke,fill}%
\end{pgfscope}%
\begin{pgfscope}%
\pgfpathrectangle{\pgfqpoint{0.100000in}{0.212622in}}{\pgfqpoint{3.696000in}{3.696000in}}%
\pgfusepath{clip}%
\pgfsetbuttcap%
\pgfsetroundjoin%
\definecolor{currentfill}{rgb}{0.121569,0.466667,0.705882}%
\pgfsetfillcolor{currentfill}%
\pgfsetfillopacity{0.367371}%
\pgfsetlinewidth{1.003750pt}%
\definecolor{currentstroke}{rgb}{0.121569,0.466667,0.705882}%
\pgfsetstrokecolor{currentstroke}%
\pgfsetstrokeopacity{0.367371}%
\pgfsetdash{}{0pt}%
\pgfpathmoveto{\pgfqpoint{1.205830in}{1.740075in}}%
\pgfpathcurveto{\pgfqpoint{1.214066in}{1.740075in}}{\pgfqpoint{1.221966in}{1.743348in}}{\pgfqpoint{1.227790in}{1.749172in}}%
\pgfpathcurveto{\pgfqpoint{1.233614in}{1.754995in}}{\pgfqpoint{1.236887in}{1.762896in}}{\pgfqpoint{1.236887in}{1.771132in}}%
\pgfpathcurveto{\pgfqpoint{1.236887in}{1.779368in}}{\pgfqpoint{1.233614in}{1.787268in}}{\pgfqpoint{1.227790in}{1.793092in}}%
\pgfpathcurveto{\pgfqpoint{1.221966in}{1.798916in}}{\pgfqpoint{1.214066in}{1.802188in}}{\pgfqpoint{1.205830in}{1.802188in}}%
\pgfpathcurveto{\pgfqpoint{1.197594in}{1.802188in}}{\pgfqpoint{1.189694in}{1.798916in}}{\pgfqpoint{1.183870in}{1.793092in}}%
\pgfpathcurveto{\pgfqpoint{1.178046in}{1.787268in}}{\pgfqpoint{1.174774in}{1.779368in}}{\pgfqpoint{1.174774in}{1.771132in}}%
\pgfpathcurveto{\pgfqpoint{1.174774in}{1.762896in}}{\pgfqpoint{1.178046in}{1.754995in}}{\pgfqpoint{1.183870in}{1.749172in}}%
\pgfpathcurveto{\pgfqpoint{1.189694in}{1.743348in}}{\pgfqpoint{1.197594in}{1.740075in}}{\pgfqpoint{1.205830in}{1.740075in}}%
\pgfpathclose%
\pgfusepath{stroke,fill}%
\end{pgfscope}%
\begin{pgfscope}%
\pgfpathrectangle{\pgfqpoint{0.100000in}{0.212622in}}{\pgfqpoint{3.696000in}{3.696000in}}%
\pgfusepath{clip}%
\pgfsetbuttcap%
\pgfsetroundjoin%
\definecolor{currentfill}{rgb}{0.121569,0.466667,0.705882}%
\pgfsetfillcolor{currentfill}%
\pgfsetfillopacity{0.367371}%
\pgfsetlinewidth{1.003750pt}%
\definecolor{currentstroke}{rgb}{0.121569,0.466667,0.705882}%
\pgfsetstrokecolor{currentstroke}%
\pgfsetstrokeopacity{0.367371}%
\pgfsetdash{}{0pt}%
\pgfpathmoveto{\pgfqpoint{1.205830in}{1.740075in}}%
\pgfpathcurveto{\pgfqpoint{1.214066in}{1.740075in}}{\pgfqpoint{1.221966in}{1.743348in}}{\pgfqpoint{1.227790in}{1.749172in}}%
\pgfpathcurveto{\pgfqpoint{1.233614in}{1.754995in}}{\pgfqpoint{1.236887in}{1.762896in}}{\pgfqpoint{1.236887in}{1.771132in}}%
\pgfpathcurveto{\pgfqpoint{1.236887in}{1.779368in}}{\pgfqpoint{1.233614in}{1.787268in}}{\pgfqpoint{1.227790in}{1.793092in}}%
\pgfpathcurveto{\pgfqpoint{1.221966in}{1.798916in}}{\pgfqpoint{1.214066in}{1.802188in}}{\pgfqpoint{1.205830in}{1.802188in}}%
\pgfpathcurveto{\pgfqpoint{1.197594in}{1.802188in}}{\pgfqpoint{1.189694in}{1.798916in}}{\pgfqpoint{1.183870in}{1.793092in}}%
\pgfpathcurveto{\pgfqpoint{1.178046in}{1.787268in}}{\pgfqpoint{1.174774in}{1.779368in}}{\pgfqpoint{1.174774in}{1.771132in}}%
\pgfpathcurveto{\pgfqpoint{1.174774in}{1.762896in}}{\pgfqpoint{1.178046in}{1.754995in}}{\pgfqpoint{1.183870in}{1.749172in}}%
\pgfpathcurveto{\pgfqpoint{1.189694in}{1.743348in}}{\pgfqpoint{1.197594in}{1.740075in}}{\pgfqpoint{1.205830in}{1.740075in}}%
\pgfpathclose%
\pgfusepath{stroke,fill}%
\end{pgfscope}%
\begin{pgfscope}%
\pgfpathrectangle{\pgfqpoint{0.100000in}{0.212622in}}{\pgfqpoint{3.696000in}{3.696000in}}%
\pgfusepath{clip}%
\pgfsetbuttcap%
\pgfsetroundjoin%
\definecolor{currentfill}{rgb}{0.121569,0.466667,0.705882}%
\pgfsetfillcolor{currentfill}%
\pgfsetfillopacity{0.367371}%
\pgfsetlinewidth{1.003750pt}%
\definecolor{currentstroke}{rgb}{0.121569,0.466667,0.705882}%
\pgfsetstrokecolor{currentstroke}%
\pgfsetstrokeopacity{0.367371}%
\pgfsetdash{}{0pt}%
\pgfpathmoveto{\pgfqpoint{1.205830in}{1.740075in}}%
\pgfpathcurveto{\pgfqpoint{1.214066in}{1.740075in}}{\pgfqpoint{1.221966in}{1.743348in}}{\pgfqpoint{1.227790in}{1.749172in}}%
\pgfpathcurveto{\pgfqpoint{1.233614in}{1.754995in}}{\pgfqpoint{1.236887in}{1.762896in}}{\pgfqpoint{1.236887in}{1.771132in}}%
\pgfpathcurveto{\pgfqpoint{1.236887in}{1.779368in}}{\pgfqpoint{1.233614in}{1.787268in}}{\pgfqpoint{1.227790in}{1.793092in}}%
\pgfpathcurveto{\pgfqpoint{1.221966in}{1.798916in}}{\pgfqpoint{1.214066in}{1.802188in}}{\pgfqpoint{1.205830in}{1.802188in}}%
\pgfpathcurveto{\pgfqpoint{1.197594in}{1.802188in}}{\pgfqpoint{1.189694in}{1.798916in}}{\pgfqpoint{1.183870in}{1.793092in}}%
\pgfpathcurveto{\pgfqpoint{1.178046in}{1.787268in}}{\pgfqpoint{1.174774in}{1.779368in}}{\pgfqpoint{1.174774in}{1.771132in}}%
\pgfpathcurveto{\pgfqpoint{1.174774in}{1.762896in}}{\pgfqpoint{1.178046in}{1.754995in}}{\pgfqpoint{1.183870in}{1.749172in}}%
\pgfpathcurveto{\pgfqpoint{1.189694in}{1.743348in}}{\pgfqpoint{1.197594in}{1.740075in}}{\pgfqpoint{1.205830in}{1.740075in}}%
\pgfpathclose%
\pgfusepath{stroke,fill}%
\end{pgfscope}%
\begin{pgfscope}%
\pgfpathrectangle{\pgfqpoint{0.100000in}{0.212622in}}{\pgfqpoint{3.696000in}{3.696000in}}%
\pgfusepath{clip}%
\pgfsetbuttcap%
\pgfsetroundjoin%
\definecolor{currentfill}{rgb}{0.121569,0.466667,0.705882}%
\pgfsetfillcolor{currentfill}%
\pgfsetfillopacity{0.367371}%
\pgfsetlinewidth{1.003750pt}%
\definecolor{currentstroke}{rgb}{0.121569,0.466667,0.705882}%
\pgfsetstrokecolor{currentstroke}%
\pgfsetstrokeopacity{0.367371}%
\pgfsetdash{}{0pt}%
\pgfpathmoveto{\pgfqpoint{1.205830in}{1.740075in}}%
\pgfpathcurveto{\pgfqpoint{1.214066in}{1.740075in}}{\pgfqpoint{1.221966in}{1.743348in}}{\pgfqpoint{1.227790in}{1.749172in}}%
\pgfpathcurveto{\pgfqpoint{1.233614in}{1.754995in}}{\pgfqpoint{1.236887in}{1.762896in}}{\pgfqpoint{1.236887in}{1.771132in}}%
\pgfpathcurveto{\pgfqpoint{1.236887in}{1.779368in}}{\pgfqpoint{1.233614in}{1.787268in}}{\pgfqpoint{1.227790in}{1.793092in}}%
\pgfpathcurveto{\pgfqpoint{1.221966in}{1.798916in}}{\pgfqpoint{1.214066in}{1.802188in}}{\pgfqpoint{1.205830in}{1.802188in}}%
\pgfpathcurveto{\pgfqpoint{1.197594in}{1.802188in}}{\pgfqpoint{1.189694in}{1.798916in}}{\pgfqpoint{1.183870in}{1.793092in}}%
\pgfpathcurveto{\pgfqpoint{1.178046in}{1.787268in}}{\pgfqpoint{1.174774in}{1.779368in}}{\pgfqpoint{1.174774in}{1.771132in}}%
\pgfpathcurveto{\pgfqpoint{1.174774in}{1.762896in}}{\pgfqpoint{1.178046in}{1.754995in}}{\pgfqpoint{1.183870in}{1.749172in}}%
\pgfpathcurveto{\pgfqpoint{1.189694in}{1.743348in}}{\pgfqpoint{1.197594in}{1.740075in}}{\pgfqpoint{1.205830in}{1.740075in}}%
\pgfpathclose%
\pgfusepath{stroke,fill}%
\end{pgfscope}%
\begin{pgfscope}%
\pgfpathrectangle{\pgfqpoint{0.100000in}{0.212622in}}{\pgfqpoint{3.696000in}{3.696000in}}%
\pgfusepath{clip}%
\pgfsetbuttcap%
\pgfsetroundjoin%
\definecolor{currentfill}{rgb}{0.121569,0.466667,0.705882}%
\pgfsetfillcolor{currentfill}%
\pgfsetfillopacity{0.367371}%
\pgfsetlinewidth{1.003750pt}%
\definecolor{currentstroke}{rgb}{0.121569,0.466667,0.705882}%
\pgfsetstrokecolor{currentstroke}%
\pgfsetstrokeopacity{0.367371}%
\pgfsetdash{}{0pt}%
\pgfpathmoveto{\pgfqpoint{1.205830in}{1.740075in}}%
\pgfpathcurveto{\pgfqpoint{1.214066in}{1.740075in}}{\pgfqpoint{1.221966in}{1.743348in}}{\pgfqpoint{1.227790in}{1.749172in}}%
\pgfpathcurveto{\pgfqpoint{1.233614in}{1.754995in}}{\pgfqpoint{1.236887in}{1.762896in}}{\pgfqpoint{1.236887in}{1.771132in}}%
\pgfpathcurveto{\pgfqpoint{1.236887in}{1.779368in}}{\pgfqpoint{1.233614in}{1.787268in}}{\pgfqpoint{1.227790in}{1.793092in}}%
\pgfpathcurveto{\pgfqpoint{1.221966in}{1.798916in}}{\pgfqpoint{1.214066in}{1.802188in}}{\pgfqpoint{1.205830in}{1.802188in}}%
\pgfpathcurveto{\pgfqpoint{1.197594in}{1.802188in}}{\pgfqpoint{1.189694in}{1.798916in}}{\pgfqpoint{1.183870in}{1.793092in}}%
\pgfpathcurveto{\pgfqpoint{1.178046in}{1.787268in}}{\pgfqpoint{1.174774in}{1.779368in}}{\pgfqpoint{1.174774in}{1.771132in}}%
\pgfpathcurveto{\pgfqpoint{1.174774in}{1.762896in}}{\pgfqpoint{1.178046in}{1.754995in}}{\pgfqpoint{1.183870in}{1.749172in}}%
\pgfpathcurveto{\pgfqpoint{1.189694in}{1.743348in}}{\pgfqpoint{1.197594in}{1.740075in}}{\pgfqpoint{1.205830in}{1.740075in}}%
\pgfpathclose%
\pgfusepath{stroke,fill}%
\end{pgfscope}%
\begin{pgfscope}%
\pgfpathrectangle{\pgfqpoint{0.100000in}{0.212622in}}{\pgfqpoint{3.696000in}{3.696000in}}%
\pgfusepath{clip}%
\pgfsetbuttcap%
\pgfsetroundjoin%
\definecolor{currentfill}{rgb}{0.121569,0.466667,0.705882}%
\pgfsetfillcolor{currentfill}%
\pgfsetfillopacity{0.367371}%
\pgfsetlinewidth{1.003750pt}%
\definecolor{currentstroke}{rgb}{0.121569,0.466667,0.705882}%
\pgfsetstrokecolor{currentstroke}%
\pgfsetstrokeopacity{0.367371}%
\pgfsetdash{}{0pt}%
\pgfpathmoveto{\pgfqpoint{1.205830in}{1.740075in}}%
\pgfpathcurveto{\pgfqpoint{1.214066in}{1.740075in}}{\pgfqpoint{1.221966in}{1.743348in}}{\pgfqpoint{1.227790in}{1.749172in}}%
\pgfpathcurveto{\pgfqpoint{1.233614in}{1.754995in}}{\pgfqpoint{1.236887in}{1.762896in}}{\pgfqpoint{1.236887in}{1.771132in}}%
\pgfpathcurveto{\pgfqpoint{1.236887in}{1.779368in}}{\pgfqpoint{1.233614in}{1.787268in}}{\pgfqpoint{1.227790in}{1.793092in}}%
\pgfpathcurveto{\pgfqpoint{1.221966in}{1.798916in}}{\pgfqpoint{1.214066in}{1.802188in}}{\pgfqpoint{1.205830in}{1.802188in}}%
\pgfpathcurveto{\pgfqpoint{1.197594in}{1.802188in}}{\pgfqpoint{1.189694in}{1.798916in}}{\pgfqpoint{1.183870in}{1.793092in}}%
\pgfpathcurveto{\pgfqpoint{1.178046in}{1.787268in}}{\pgfqpoint{1.174774in}{1.779368in}}{\pgfqpoint{1.174774in}{1.771132in}}%
\pgfpathcurveto{\pgfqpoint{1.174774in}{1.762896in}}{\pgfqpoint{1.178046in}{1.754995in}}{\pgfqpoint{1.183870in}{1.749172in}}%
\pgfpathcurveto{\pgfqpoint{1.189694in}{1.743348in}}{\pgfqpoint{1.197594in}{1.740075in}}{\pgfqpoint{1.205830in}{1.740075in}}%
\pgfpathclose%
\pgfusepath{stroke,fill}%
\end{pgfscope}%
\begin{pgfscope}%
\pgfpathrectangle{\pgfqpoint{0.100000in}{0.212622in}}{\pgfqpoint{3.696000in}{3.696000in}}%
\pgfusepath{clip}%
\pgfsetbuttcap%
\pgfsetroundjoin%
\definecolor{currentfill}{rgb}{0.121569,0.466667,0.705882}%
\pgfsetfillcolor{currentfill}%
\pgfsetfillopacity{0.367371}%
\pgfsetlinewidth{1.003750pt}%
\definecolor{currentstroke}{rgb}{0.121569,0.466667,0.705882}%
\pgfsetstrokecolor{currentstroke}%
\pgfsetstrokeopacity{0.367371}%
\pgfsetdash{}{0pt}%
\pgfpathmoveto{\pgfqpoint{1.205830in}{1.740075in}}%
\pgfpathcurveto{\pgfqpoint{1.214066in}{1.740075in}}{\pgfqpoint{1.221966in}{1.743348in}}{\pgfqpoint{1.227790in}{1.749172in}}%
\pgfpathcurveto{\pgfqpoint{1.233614in}{1.754995in}}{\pgfqpoint{1.236887in}{1.762896in}}{\pgfqpoint{1.236887in}{1.771132in}}%
\pgfpathcurveto{\pgfqpoint{1.236887in}{1.779368in}}{\pgfqpoint{1.233614in}{1.787268in}}{\pgfqpoint{1.227790in}{1.793092in}}%
\pgfpathcurveto{\pgfqpoint{1.221966in}{1.798916in}}{\pgfqpoint{1.214066in}{1.802188in}}{\pgfqpoint{1.205830in}{1.802188in}}%
\pgfpathcurveto{\pgfqpoint{1.197594in}{1.802188in}}{\pgfqpoint{1.189694in}{1.798916in}}{\pgfqpoint{1.183870in}{1.793092in}}%
\pgfpathcurveto{\pgfqpoint{1.178046in}{1.787268in}}{\pgfqpoint{1.174774in}{1.779368in}}{\pgfqpoint{1.174774in}{1.771132in}}%
\pgfpathcurveto{\pgfqpoint{1.174774in}{1.762896in}}{\pgfqpoint{1.178046in}{1.754995in}}{\pgfqpoint{1.183870in}{1.749172in}}%
\pgfpathcurveto{\pgfqpoint{1.189694in}{1.743348in}}{\pgfqpoint{1.197594in}{1.740075in}}{\pgfqpoint{1.205830in}{1.740075in}}%
\pgfpathclose%
\pgfusepath{stroke,fill}%
\end{pgfscope}%
\begin{pgfscope}%
\pgfpathrectangle{\pgfqpoint{0.100000in}{0.212622in}}{\pgfqpoint{3.696000in}{3.696000in}}%
\pgfusepath{clip}%
\pgfsetbuttcap%
\pgfsetroundjoin%
\definecolor{currentfill}{rgb}{0.121569,0.466667,0.705882}%
\pgfsetfillcolor{currentfill}%
\pgfsetfillopacity{0.367371}%
\pgfsetlinewidth{1.003750pt}%
\definecolor{currentstroke}{rgb}{0.121569,0.466667,0.705882}%
\pgfsetstrokecolor{currentstroke}%
\pgfsetstrokeopacity{0.367371}%
\pgfsetdash{}{0pt}%
\pgfpathmoveto{\pgfqpoint{1.205830in}{1.740075in}}%
\pgfpathcurveto{\pgfqpoint{1.214066in}{1.740075in}}{\pgfqpoint{1.221966in}{1.743348in}}{\pgfqpoint{1.227790in}{1.749172in}}%
\pgfpathcurveto{\pgfqpoint{1.233614in}{1.754995in}}{\pgfqpoint{1.236887in}{1.762896in}}{\pgfqpoint{1.236887in}{1.771132in}}%
\pgfpathcurveto{\pgfqpoint{1.236887in}{1.779368in}}{\pgfqpoint{1.233614in}{1.787268in}}{\pgfqpoint{1.227790in}{1.793092in}}%
\pgfpathcurveto{\pgfqpoint{1.221966in}{1.798916in}}{\pgfqpoint{1.214066in}{1.802188in}}{\pgfqpoint{1.205830in}{1.802188in}}%
\pgfpathcurveto{\pgfqpoint{1.197594in}{1.802188in}}{\pgfqpoint{1.189694in}{1.798916in}}{\pgfqpoint{1.183870in}{1.793092in}}%
\pgfpathcurveto{\pgfqpoint{1.178046in}{1.787268in}}{\pgfqpoint{1.174774in}{1.779368in}}{\pgfqpoint{1.174774in}{1.771132in}}%
\pgfpathcurveto{\pgfqpoint{1.174774in}{1.762896in}}{\pgfqpoint{1.178046in}{1.754995in}}{\pgfqpoint{1.183870in}{1.749172in}}%
\pgfpathcurveto{\pgfqpoint{1.189694in}{1.743348in}}{\pgfqpoint{1.197594in}{1.740075in}}{\pgfqpoint{1.205830in}{1.740075in}}%
\pgfpathclose%
\pgfusepath{stroke,fill}%
\end{pgfscope}%
\begin{pgfscope}%
\pgfpathrectangle{\pgfqpoint{0.100000in}{0.212622in}}{\pgfqpoint{3.696000in}{3.696000in}}%
\pgfusepath{clip}%
\pgfsetbuttcap%
\pgfsetroundjoin%
\definecolor{currentfill}{rgb}{0.121569,0.466667,0.705882}%
\pgfsetfillcolor{currentfill}%
\pgfsetfillopacity{0.367371}%
\pgfsetlinewidth{1.003750pt}%
\definecolor{currentstroke}{rgb}{0.121569,0.466667,0.705882}%
\pgfsetstrokecolor{currentstroke}%
\pgfsetstrokeopacity{0.367371}%
\pgfsetdash{}{0pt}%
\pgfpathmoveto{\pgfqpoint{1.205830in}{1.740075in}}%
\pgfpathcurveto{\pgfqpoint{1.214066in}{1.740075in}}{\pgfqpoint{1.221966in}{1.743348in}}{\pgfqpoint{1.227790in}{1.749172in}}%
\pgfpathcurveto{\pgfqpoint{1.233614in}{1.754995in}}{\pgfqpoint{1.236887in}{1.762896in}}{\pgfqpoint{1.236887in}{1.771132in}}%
\pgfpathcurveto{\pgfqpoint{1.236887in}{1.779368in}}{\pgfqpoint{1.233614in}{1.787268in}}{\pgfqpoint{1.227790in}{1.793092in}}%
\pgfpathcurveto{\pgfqpoint{1.221966in}{1.798916in}}{\pgfqpoint{1.214066in}{1.802188in}}{\pgfqpoint{1.205830in}{1.802188in}}%
\pgfpathcurveto{\pgfqpoint{1.197594in}{1.802188in}}{\pgfqpoint{1.189694in}{1.798916in}}{\pgfqpoint{1.183870in}{1.793092in}}%
\pgfpathcurveto{\pgfqpoint{1.178046in}{1.787268in}}{\pgfqpoint{1.174774in}{1.779368in}}{\pgfqpoint{1.174774in}{1.771132in}}%
\pgfpathcurveto{\pgfqpoint{1.174774in}{1.762896in}}{\pgfqpoint{1.178046in}{1.754995in}}{\pgfqpoint{1.183870in}{1.749172in}}%
\pgfpathcurveto{\pgfqpoint{1.189694in}{1.743348in}}{\pgfqpoint{1.197594in}{1.740075in}}{\pgfqpoint{1.205830in}{1.740075in}}%
\pgfpathclose%
\pgfusepath{stroke,fill}%
\end{pgfscope}%
\begin{pgfscope}%
\pgfpathrectangle{\pgfqpoint{0.100000in}{0.212622in}}{\pgfqpoint{3.696000in}{3.696000in}}%
\pgfusepath{clip}%
\pgfsetbuttcap%
\pgfsetroundjoin%
\definecolor{currentfill}{rgb}{0.121569,0.466667,0.705882}%
\pgfsetfillcolor{currentfill}%
\pgfsetfillopacity{0.367371}%
\pgfsetlinewidth{1.003750pt}%
\definecolor{currentstroke}{rgb}{0.121569,0.466667,0.705882}%
\pgfsetstrokecolor{currentstroke}%
\pgfsetstrokeopacity{0.367371}%
\pgfsetdash{}{0pt}%
\pgfpathmoveto{\pgfqpoint{1.205830in}{1.740075in}}%
\pgfpathcurveto{\pgfqpoint{1.214066in}{1.740075in}}{\pgfqpoint{1.221966in}{1.743348in}}{\pgfqpoint{1.227790in}{1.749172in}}%
\pgfpathcurveto{\pgfqpoint{1.233614in}{1.754995in}}{\pgfqpoint{1.236887in}{1.762896in}}{\pgfqpoint{1.236887in}{1.771132in}}%
\pgfpathcurveto{\pgfqpoint{1.236887in}{1.779368in}}{\pgfqpoint{1.233614in}{1.787268in}}{\pgfqpoint{1.227790in}{1.793092in}}%
\pgfpathcurveto{\pgfqpoint{1.221966in}{1.798916in}}{\pgfqpoint{1.214066in}{1.802188in}}{\pgfqpoint{1.205830in}{1.802188in}}%
\pgfpathcurveto{\pgfqpoint{1.197594in}{1.802188in}}{\pgfqpoint{1.189694in}{1.798916in}}{\pgfqpoint{1.183870in}{1.793092in}}%
\pgfpathcurveto{\pgfqpoint{1.178046in}{1.787268in}}{\pgfqpoint{1.174774in}{1.779368in}}{\pgfqpoint{1.174774in}{1.771132in}}%
\pgfpathcurveto{\pgfqpoint{1.174774in}{1.762896in}}{\pgfqpoint{1.178046in}{1.754995in}}{\pgfqpoint{1.183870in}{1.749172in}}%
\pgfpathcurveto{\pgfqpoint{1.189694in}{1.743348in}}{\pgfqpoint{1.197594in}{1.740075in}}{\pgfqpoint{1.205830in}{1.740075in}}%
\pgfpathclose%
\pgfusepath{stroke,fill}%
\end{pgfscope}%
\begin{pgfscope}%
\pgfpathrectangle{\pgfqpoint{0.100000in}{0.212622in}}{\pgfqpoint{3.696000in}{3.696000in}}%
\pgfusepath{clip}%
\pgfsetbuttcap%
\pgfsetroundjoin%
\definecolor{currentfill}{rgb}{0.121569,0.466667,0.705882}%
\pgfsetfillcolor{currentfill}%
\pgfsetfillopacity{0.367371}%
\pgfsetlinewidth{1.003750pt}%
\definecolor{currentstroke}{rgb}{0.121569,0.466667,0.705882}%
\pgfsetstrokecolor{currentstroke}%
\pgfsetstrokeopacity{0.367371}%
\pgfsetdash{}{0pt}%
\pgfpathmoveto{\pgfqpoint{1.205830in}{1.740075in}}%
\pgfpathcurveto{\pgfqpoint{1.214066in}{1.740075in}}{\pgfqpoint{1.221966in}{1.743348in}}{\pgfqpoint{1.227790in}{1.749172in}}%
\pgfpathcurveto{\pgfqpoint{1.233614in}{1.754995in}}{\pgfqpoint{1.236887in}{1.762896in}}{\pgfqpoint{1.236887in}{1.771132in}}%
\pgfpathcurveto{\pgfqpoint{1.236887in}{1.779368in}}{\pgfqpoint{1.233614in}{1.787268in}}{\pgfqpoint{1.227790in}{1.793092in}}%
\pgfpathcurveto{\pgfqpoint{1.221966in}{1.798916in}}{\pgfqpoint{1.214066in}{1.802188in}}{\pgfqpoint{1.205830in}{1.802188in}}%
\pgfpathcurveto{\pgfqpoint{1.197594in}{1.802188in}}{\pgfqpoint{1.189694in}{1.798916in}}{\pgfqpoint{1.183870in}{1.793092in}}%
\pgfpathcurveto{\pgfqpoint{1.178046in}{1.787268in}}{\pgfqpoint{1.174774in}{1.779368in}}{\pgfqpoint{1.174774in}{1.771132in}}%
\pgfpathcurveto{\pgfqpoint{1.174774in}{1.762896in}}{\pgfqpoint{1.178046in}{1.754995in}}{\pgfqpoint{1.183870in}{1.749172in}}%
\pgfpathcurveto{\pgfqpoint{1.189694in}{1.743348in}}{\pgfqpoint{1.197594in}{1.740075in}}{\pgfqpoint{1.205830in}{1.740075in}}%
\pgfpathclose%
\pgfusepath{stroke,fill}%
\end{pgfscope}%
\begin{pgfscope}%
\pgfpathrectangle{\pgfqpoint{0.100000in}{0.212622in}}{\pgfqpoint{3.696000in}{3.696000in}}%
\pgfusepath{clip}%
\pgfsetbuttcap%
\pgfsetroundjoin%
\definecolor{currentfill}{rgb}{0.121569,0.466667,0.705882}%
\pgfsetfillcolor{currentfill}%
\pgfsetfillopacity{0.367371}%
\pgfsetlinewidth{1.003750pt}%
\definecolor{currentstroke}{rgb}{0.121569,0.466667,0.705882}%
\pgfsetstrokecolor{currentstroke}%
\pgfsetstrokeopacity{0.367371}%
\pgfsetdash{}{0pt}%
\pgfpathmoveto{\pgfqpoint{1.205830in}{1.740075in}}%
\pgfpathcurveto{\pgfqpoint{1.214066in}{1.740075in}}{\pgfqpoint{1.221966in}{1.743348in}}{\pgfqpoint{1.227790in}{1.749172in}}%
\pgfpathcurveto{\pgfqpoint{1.233614in}{1.754995in}}{\pgfqpoint{1.236887in}{1.762896in}}{\pgfqpoint{1.236887in}{1.771132in}}%
\pgfpathcurveto{\pgfqpoint{1.236887in}{1.779368in}}{\pgfqpoint{1.233614in}{1.787268in}}{\pgfqpoint{1.227790in}{1.793092in}}%
\pgfpathcurveto{\pgfqpoint{1.221966in}{1.798916in}}{\pgfqpoint{1.214066in}{1.802188in}}{\pgfqpoint{1.205830in}{1.802188in}}%
\pgfpathcurveto{\pgfqpoint{1.197594in}{1.802188in}}{\pgfqpoint{1.189694in}{1.798916in}}{\pgfqpoint{1.183870in}{1.793092in}}%
\pgfpathcurveto{\pgfqpoint{1.178046in}{1.787268in}}{\pgfqpoint{1.174774in}{1.779368in}}{\pgfqpoint{1.174774in}{1.771132in}}%
\pgfpathcurveto{\pgfqpoint{1.174774in}{1.762896in}}{\pgfqpoint{1.178046in}{1.754995in}}{\pgfqpoint{1.183870in}{1.749172in}}%
\pgfpathcurveto{\pgfqpoint{1.189694in}{1.743348in}}{\pgfqpoint{1.197594in}{1.740075in}}{\pgfqpoint{1.205830in}{1.740075in}}%
\pgfpathclose%
\pgfusepath{stroke,fill}%
\end{pgfscope}%
\begin{pgfscope}%
\pgfpathrectangle{\pgfqpoint{0.100000in}{0.212622in}}{\pgfqpoint{3.696000in}{3.696000in}}%
\pgfusepath{clip}%
\pgfsetbuttcap%
\pgfsetroundjoin%
\definecolor{currentfill}{rgb}{0.121569,0.466667,0.705882}%
\pgfsetfillcolor{currentfill}%
\pgfsetfillopacity{0.367371}%
\pgfsetlinewidth{1.003750pt}%
\definecolor{currentstroke}{rgb}{0.121569,0.466667,0.705882}%
\pgfsetstrokecolor{currentstroke}%
\pgfsetstrokeopacity{0.367371}%
\pgfsetdash{}{0pt}%
\pgfpathmoveto{\pgfqpoint{1.205830in}{1.740075in}}%
\pgfpathcurveto{\pgfqpoint{1.214066in}{1.740075in}}{\pgfqpoint{1.221966in}{1.743348in}}{\pgfqpoint{1.227790in}{1.749172in}}%
\pgfpathcurveto{\pgfqpoint{1.233614in}{1.754995in}}{\pgfqpoint{1.236887in}{1.762896in}}{\pgfqpoint{1.236887in}{1.771132in}}%
\pgfpathcurveto{\pgfqpoint{1.236887in}{1.779368in}}{\pgfqpoint{1.233614in}{1.787268in}}{\pgfqpoint{1.227790in}{1.793092in}}%
\pgfpathcurveto{\pgfqpoint{1.221966in}{1.798916in}}{\pgfqpoint{1.214066in}{1.802188in}}{\pgfqpoint{1.205830in}{1.802188in}}%
\pgfpathcurveto{\pgfqpoint{1.197594in}{1.802188in}}{\pgfqpoint{1.189694in}{1.798916in}}{\pgfqpoint{1.183870in}{1.793092in}}%
\pgfpathcurveto{\pgfqpoint{1.178046in}{1.787268in}}{\pgfqpoint{1.174774in}{1.779368in}}{\pgfqpoint{1.174774in}{1.771132in}}%
\pgfpathcurveto{\pgfqpoint{1.174774in}{1.762896in}}{\pgfqpoint{1.178046in}{1.754995in}}{\pgfqpoint{1.183870in}{1.749172in}}%
\pgfpathcurveto{\pgfqpoint{1.189694in}{1.743348in}}{\pgfqpoint{1.197594in}{1.740075in}}{\pgfqpoint{1.205830in}{1.740075in}}%
\pgfpathclose%
\pgfusepath{stroke,fill}%
\end{pgfscope}%
\begin{pgfscope}%
\pgfpathrectangle{\pgfqpoint{0.100000in}{0.212622in}}{\pgfqpoint{3.696000in}{3.696000in}}%
\pgfusepath{clip}%
\pgfsetbuttcap%
\pgfsetroundjoin%
\definecolor{currentfill}{rgb}{0.121569,0.466667,0.705882}%
\pgfsetfillcolor{currentfill}%
\pgfsetfillopacity{0.367371}%
\pgfsetlinewidth{1.003750pt}%
\definecolor{currentstroke}{rgb}{0.121569,0.466667,0.705882}%
\pgfsetstrokecolor{currentstroke}%
\pgfsetstrokeopacity{0.367371}%
\pgfsetdash{}{0pt}%
\pgfpathmoveto{\pgfqpoint{1.205830in}{1.740075in}}%
\pgfpathcurveto{\pgfqpoint{1.214066in}{1.740075in}}{\pgfqpoint{1.221966in}{1.743348in}}{\pgfqpoint{1.227790in}{1.749172in}}%
\pgfpathcurveto{\pgfqpoint{1.233614in}{1.754995in}}{\pgfqpoint{1.236887in}{1.762896in}}{\pgfqpoint{1.236887in}{1.771132in}}%
\pgfpathcurveto{\pgfqpoint{1.236887in}{1.779368in}}{\pgfqpoint{1.233614in}{1.787268in}}{\pgfqpoint{1.227790in}{1.793092in}}%
\pgfpathcurveto{\pgfqpoint{1.221966in}{1.798916in}}{\pgfqpoint{1.214066in}{1.802188in}}{\pgfqpoint{1.205830in}{1.802188in}}%
\pgfpathcurveto{\pgfqpoint{1.197594in}{1.802188in}}{\pgfqpoint{1.189694in}{1.798916in}}{\pgfqpoint{1.183870in}{1.793092in}}%
\pgfpathcurveto{\pgfqpoint{1.178046in}{1.787268in}}{\pgfqpoint{1.174774in}{1.779368in}}{\pgfqpoint{1.174774in}{1.771132in}}%
\pgfpathcurveto{\pgfqpoint{1.174774in}{1.762896in}}{\pgfqpoint{1.178046in}{1.754995in}}{\pgfqpoint{1.183870in}{1.749172in}}%
\pgfpathcurveto{\pgfqpoint{1.189694in}{1.743348in}}{\pgfqpoint{1.197594in}{1.740075in}}{\pgfqpoint{1.205830in}{1.740075in}}%
\pgfpathclose%
\pgfusepath{stroke,fill}%
\end{pgfscope}%
\begin{pgfscope}%
\pgfpathrectangle{\pgfqpoint{0.100000in}{0.212622in}}{\pgfqpoint{3.696000in}{3.696000in}}%
\pgfusepath{clip}%
\pgfsetbuttcap%
\pgfsetroundjoin%
\definecolor{currentfill}{rgb}{0.121569,0.466667,0.705882}%
\pgfsetfillcolor{currentfill}%
\pgfsetfillopacity{0.367371}%
\pgfsetlinewidth{1.003750pt}%
\definecolor{currentstroke}{rgb}{0.121569,0.466667,0.705882}%
\pgfsetstrokecolor{currentstroke}%
\pgfsetstrokeopacity{0.367371}%
\pgfsetdash{}{0pt}%
\pgfpathmoveto{\pgfqpoint{1.205830in}{1.740075in}}%
\pgfpathcurveto{\pgfqpoint{1.214066in}{1.740075in}}{\pgfqpoint{1.221966in}{1.743348in}}{\pgfqpoint{1.227790in}{1.749172in}}%
\pgfpathcurveto{\pgfqpoint{1.233614in}{1.754995in}}{\pgfqpoint{1.236887in}{1.762896in}}{\pgfqpoint{1.236887in}{1.771132in}}%
\pgfpathcurveto{\pgfqpoint{1.236887in}{1.779368in}}{\pgfqpoint{1.233614in}{1.787268in}}{\pgfqpoint{1.227790in}{1.793092in}}%
\pgfpathcurveto{\pgfqpoint{1.221966in}{1.798916in}}{\pgfqpoint{1.214066in}{1.802188in}}{\pgfqpoint{1.205830in}{1.802188in}}%
\pgfpathcurveto{\pgfqpoint{1.197594in}{1.802188in}}{\pgfqpoint{1.189694in}{1.798916in}}{\pgfqpoint{1.183870in}{1.793092in}}%
\pgfpathcurveto{\pgfqpoint{1.178046in}{1.787268in}}{\pgfqpoint{1.174774in}{1.779368in}}{\pgfqpoint{1.174774in}{1.771132in}}%
\pgfpathcurveto{\pgfqpoint{1.174774in}{1.762896in}}{\pgfqpoint{1.178046in}{1.754995in}}{\pgfqpoint{1.183870in}{1.749172in}}%
\pgfpathcurveto{\pgfqpoint{1.189694in}{1.743348in}}{\pgfqpoint{1.197594in}{1.740075in}}{\pgfqpoint{1.205830in}{1.740075in}}%
\pgfpathclose%
\pgfusepath{stroke,fill}%
\end{pgfscope}%
\begin{pgfscope}%
\pgfpathrectangle{\pgfqpoint{0.100000in}{0.212622in}}{\pgfqpoint{3.696000in}{3.696000in}}%
\pgfusepath{clip}%
\pgfsetbuttcap%
\pgfsetroundjoin%
\definecolor{currentfill}{rgb}{0.121569,0.466667,0.705882}%
\pgfsetfillcolor{currentfill}%
\pgfsetfillopacity{0.367371}%
\pgfsetlinewidth{1.003750pt}%
\definecolor{currentstroke}{rgb}{0.121569,0.466667,0.705882}%
\pgfsetstrokecolor{currentstroke}%
\pgfsetstrokeopacity{0.367371}%
\pgfsetdash{}{0pt}%
\pgfpathmoveto{\pgfqpoint{1.205830in}{1.740075in}}%
\pgfpathcurveto{\pgfqpoint{1.214066in}{1.740075in}}{\pgfqpoint{1.221966in}{1.743348in}}{\pgfqpoint{1.227790in}{1.749172in}}%
\pgfpathcurveto{\pgfqpoint{1.233614in}{1.754995in}}{\pgfqpoint{1.236887in}{1.762896in}}{\pgfqpoint{1.236887in}{1.771132in}}%
\pgfpathcurveto{\pgfqpoint{1.236887in}{1.779368in}}{\pgfqpoint{1.233614in}{1.787268in}}{\pgfqpoint{1.227790in}{1.793092in}}%
\pgfpathcurveto{\pgfqpoint{1.221966in}{1.798916in}}{\pgfqpoint{1.214066in}{1.802188in}}{\pgfqpoint{1.205830in}{1.802188in}}%
\pgfpathcurveto{\pgfqpoint{1.197594in}{1.802188in}}{\pgfqpoint{1.189694in}{1.798916in}}{\pgfqpoint{1.183870in}{1.793092in}}%
\pgfpathcurveto{\pgfqpoint{1.178046in}{1.787268in}}{\pgfqpoint{1.174774in}{1.779368in}}{\pgfqpoint{1.174774in}{1.771132in}}%
\pgfpathcurveto{\pgfqpoint{1.174774in}{1.762896in}}{\pgfqpoint{1.178046in}{1.754995in}}{\pgfqpoint{1.183870in}{1.749172in}}%
\pgfpathcurveto{\pgfqpoint{1.189694in}{1.743348in}}{\pgfqpoint{1.197594in}{1.740075in}}{\pgfqpoint{1.205830in}{1.740075in}}%
\pgfpathclose%
\pgfusepath{stroke,fill}%
\end{pgfscope}%
\begin{pgfscope}%
\pgfpathrectangle{\pgfqpoint{0.100000in}{0.212622in}}{\pgfqpoint{3.696000in}{3.696000in}}%
\pgfusepath{clip}%
\pgfsetbuttcap%
\pgfsetroundjoin%
\definecolor{currentfill}{rgb}{0.121569,0.466667,0.705882}%
\pgfsetfillcolor{currentfill}%
\pgfsetfillopacity{0.367371}%
\pgfsetlinewidth{1.003750pt}%
\definecolor{currentstroke}{rgb}{0.121569,0.466667,0.705882}%
\pgfsetstrokecolor{currentstroke}%
\pgfsetstrokeopacity{0.367371}%
\pgfsetdash{}{0pt}%
\pgfpathmoveto{\pgfqpoint{1.205830in}{1.740075in}}%
\pgfpathcurveto{\pgfqpoint{1.214066in}{1.740075in}}{\pgfqpoint{1.221966in}{1.743348in}}{\pgfqpoint{1.227790in}{1.749172in}}%
\pgfpathcurveto{\pgfqpoint{1.233614in}{1.754995in}}{\pgfqpoint{1.236887in}{1.762896in}}{\pgfqpoint{1.236887in}{1.771132in}}%
\pgfpathcurveto{\pgfqpoint{1.236887in}{1.779368in}}{\pgfqpoint{1.233614in}{1.787268in}}{\pgfqpoint{1.227790in}{1.793092in}}%
\pgfpathcurveto{\pgfqpoint{1.221966in}{1.798916in}}{\pgfqpoint{1.214066in}{1.802188in}}{\pgfqpoint{1.205830in}{1.802188in}}%
\pgfpathcurveto{\pgfqpoint{1.197594in}{1.802188in}}{\pgfqpoint{1.189694in}{1.798916in}}{\pgfqpoint{1.183870in}{1.793092in}}%
\pgfpathcurveto{\pgfqpoint{1.178046in}{1.787268in}}{\pgfqpoint{1.174774in}{1.779368in}}{\pgfqpoint{1.174774in}{1.771132in}}%
\pgfpathcurveto{\pgfqpoint{1.174774in}{1.762896in}}{\pgfqpoint{1.178046in}{1.754995in}}{\pgfqpoint{1.183870in}{1.749172in}}%
\pgfpathcurveto{\pgfqpoint{1.189694in}{1.743348in}}{\pgfqpoint{1.197594in}{1.740075in}}{\pgfqpoint{1.205830in}{1.740075in}}%
\pgfpathclose%
\pgfusepath{stroke,fill}%
\end{pgfscope}%
\begin{pgfscope}%
\pgfpathrectangle{\pgfqpoint{0.100000in}{0.212622in}}{\pgfqpoint{3.696000in}{3.696000in}}%
\pgfusepath{clip}%
\pgfsetbuttcap%
\pgfsetroundjoin%
\definecolor{currentfill}{rgb}{0.121569,0.466667,0.705882}%
\pgfsetfillcolor{currentfill}%
\pgfsetfillopacity{0.367371}%
\pgfsetlinewidth{1.003750pt}%
\definecolor{currentstroke}{rgb}{0.121569,0.466667,0.705882}%
\pgfsetstrokecolor{currentstroke}%
\pgfsetstrokeopacity{0.367371}%
\pgfsetdash{}{0pt}%
\pgfpathmoveto{\pgfqpoint{1.205830in}{1.740075in}}%
\pgfpathcurveto{\pgfqpoint{1.214066in}{1.740075in}}{\pgfqpoint{1.221966in}{1.743348in}}{\pgfqpoint{1.227790in}{1.749172in}}%
\pgfpathcurveto{\pgfqpoint{1.233614in}{1.754995in}}{\pgfqpoint{1.236887in}{1.762896in}}{\pgfqpoint{1.236887in}{1.771132in}}%
\pgfpathcurveto{\pgfqpoint{1.236887in}{1.779368in}}{\pgfqpoint{1.233614in}{1.787268in}}{\pgfqpoint{1.227790in}{1.793092in}}%
\pgfpathcurveto{\pgfqpoint{1.221966in}{1.798916in}}{\pgfqpoint{1.214066in}{1.802188in}}{\pgfqpoint{1.205830in}{1.802188in}}%
\pgfpathcurveto{\pgfqpoint{1.197594in}{1.802188in}}{\pgfqpoint{1.189694in}{1.798916in}}{\pgfqpoint{1.183870in}{1.793092in}}%
\pgfpathcurveto{\pgfqpoint{1.178046in}{1.787268in}}{\pgfqpoint{1.174774in}{1.779368in}}{\pgfqpoint{1.174774in}{1.771132in}}%
\pgfpathcurveto{\pgfqpoint{1.174774in}{1.762896in}}{\pgfqpoint{1.178046in}{1.754995in}}{\pgfqpoint{1.183870in}{1.749172in}}%
\pgfpathcurveto{\pgfqpoint{1.189694in}{1.743348in}}{\pgfqpoint{1.197594in}{1.740075in}}{\pgfqpoint{1.205830in}{1.740075in}}%
\pgfpathclose%
\pgfusepath{stroke,fill}%
\end{pgfscope}%
\begin{pgfscope}%
\pgfpathrectangle{\pgfqpoint{0.100000in}{0.212622in}}{\pgfqpoint{3.696000in}{3.696000in}}%
\pgfusepath{clip}%
\pgfsetbuttcap%
\pgfsetroundjoin%
\definecolor{currentfill}{rgb}{0.121569,0.466667,0.705882}%
\pgfsetfillcolor{currentfill}%
\pgfsetfillopacity{0.367371}%
\pgfsetlinewidth{1.003750pt}%
\definecolor{currentstroke}{rgb}{0.121569,0.466667,0.705882}%
\pgfsetstrokecolor{currentstroke}%
\pgfsetstrokeopacity{0.367371}%
\pgfsetdash{}{0pt}%
\pgfpathmoveto{\pgfqpoint{1.205830in}{1.740075in}}%
\pgfpathcurveto{\pgfqpoint{1.214066in}{1.740075in}}{\pgfqpoint{1.221966in}{1.743348in}}{\pgfqpoint{1.227790in}{1.749172in}}%
\pgfpathcurveto{\pgfqpoint{1.233614in}{1.754995in}}{\pgfqpoint{1.236887in}{1.762896in}}{\pgfqpoint{1.236887in}{1.771132in}}%
\pgfpathcurveto{\pgfqpoint{1.236887in}{1.779368in}}{\pgfqpoint{1.233614in}{1.787268in}}{\pgfqpoint{1.227790in}{1.793092in}}%
\pgfpathcurveto{\pgfqpoint{1.221966in}{1.798916in}}{\pgfqpoint{1.214066in}{1.802188in}}{\pgfqpoint{1.205830in}{1.802188in}}%
\pgfpathcurveto{\pgfqpoint{1.197594in}{1.802188in}}{\pgfqpoint{1.189694in}{1.798916in}}{\pgfqpoint{1.183870in}{1.793092in}}%
\pgfpathcurveto{\pgfqpoint{1.178046in}{1.787268in}}{\pgfqpoint{1.174774in}{1.779368in}}{\pgfqpoint{1.174774in}{1.771132in}}%
\pgfpathcurveto{\pgfqpoint{1.174774in}{1.762896in}}{\pgfqpoint{1.178046in}{1.754995in}}{\pgfqpoint{1.183870in}{1.749172in}}%
\pgfpathcurveto{\pgfqpoint{1.189694in}{1.743348in}}{\pgfqpoint{1.197594in}{1.740075in}}{\pgfqpoint{1.205830in}{1.740075in}}%
\pgfpathclose%
\pgfusepath{stroke,fill}%
\end{pgfscope}%
\begin{pgfscope}%
\pgfpathrectangle{\pgfqpoint{0.100000in}{0.212622in}}{\pgfqpoint{3.696000in}{3.696000in}}%
\pgfusepath{clip}%
\pgfsetbuttcap%
\pgfsetroundjoin%
\definecolor{currentfill}{rgb}{0.121569,0.466667,0.705882}%
\pgfsetfillcolor{currentfill}%
\pgfsetfillopacity{0.367371}%
\pgfsetlinewidth{1.003750pt}%
\definecolor{currentstroke}{rgb}{0.121569,0.466667,0.705882}%
\pgfsetstrokecolor{currentstroke}%
\pgfsetstrokeopacity{0.367371}%
\pgfsetdash{}{0pt}%
\pgfpathmoveto{\pgfqpoint{1.205830in}{1.740075in}}%
\pgfpathcurveto{\pgfqpoint{1.214066in}{1.740075in}}{\pgfqpoint{1.221966in}{1.743348in}}{\pgfqpoint{1.227790in}{1.749172in}}%
\pgfpathcurveto{\pgfqpoint{1.233614in}{1.754995in}}{\pgfqpoint{1.236887in}{1.762896in}}{\pgfqpoint{1.236887in}{1.771132in}}%
\pgfpathcurveto{\pgfqpoint{1.236887in}{1.779368in}}{\pgfqpoint{1.233614in}{1.787268in}}{\pgfqpoint{1.227790in}{1.793092in}}%
\pgfpathcurveto{\pgfqpoint{1.221966in}{1.798916in}}{\pgfqpoint{1.214066in}{1.802188in}}{\pgfqpoint{1.205830in}{1.802188in}}%
\pgfpathcurveto{\pgfqpoint{1.197594in}{1.802188in}}{\pgfqpoint{1.189694in}{1.798916in}}{\pgfqpoint{1.183870in}{1.793092in}}%
\pgfpathcurveto{\pgfqpoint{1.178046in}{1.787268in}}{\pgfqpoint{1.174774in}{1.779368in}}{\pgfqpoint{1.174774in}{1.771132in}}%
\pgfpathcurveto{\pgfqpoint{1.174774in}{1.762896in}}{\pgfqpoint{1.178046in}{1.754995in}}{\pgfqpoint{1.183870in}{1.749172in}}%
\pgfpathcurveto{\pgfqpoint{1.189694in}{1.743348in}}{\pgfqpoint{1.197594in}{1.740075in}}{\pgfqpoint{1.205830in}{1.740075in}}%
\pgfpathclose%
\pgfusepath{stroke,fill}%
\end{pgfscope}%
\begin{pgfscope}%
\pgfpathrectangle{\pgfqpoint{0.100000in}{0.212622in}}{\pgfqpoint{3.696000in}{3.696000in}}%
\pgfusepath{clip}%
\pgfsetbuttcap%
\pgfsetroundjoin%
\definecolor{currentfill}{rgb}{0.121569,0.466667,0.705882}%
\pgfsetfillcolor{currentfill}%
\pgfsetfillopacity{0.367371}%
\pgfsetlinewidth{1.003750pt}%
\definecolor{currentstroke}{rgb}{0.121569,0.466667,0.705882}%
\pgfsetstrokecolor{currentstroke}%
\pgfsetstrokeopacity{0.367371}%
\pgfsetdash{}{0pt}%
\pgfpathmoveto{\pgfqpoint{1.205830in}{1.740075in}}%
\pgfpathcurveto{\pgfqpoint{1.214066in}{1.740075in}}{\pgfqpoint{1.221966in}{1.743348in}}{\pgfqpoint{1.227790in}{1.749172in}}%
\pgfpathcurveto{\pgfqpoint{1.233614in}{1.754995in}}{\pgfqpoint{1.236887in}{1.762896in}}{\pgfqpoint{1.236887in}{1.771132in}}%
\pgfpathcurveto{\pgfqpoint{1.236887in}{1.779368in}}{\pgfqpoint{1.233614in}{1.787268in}}{\pgfqpoint{1.227790in}{1.793092in}}%
\pgfpathcurveto{\pgfqpoint{1.221966in}{1.798916in}}{\pgfqpoint{1.214066in}{1.802188in}}{\pgfqpoint{1.205830in}{1.802188in}}%
\pgfpathcurveto{\pgfqpoint{1.197594in}{1.802188in}}{\pgfqpoint{1.189694in}{1.798916in}}{\pgfqpoint{1.183870in}{1.793092in}}%
\pgfpathcurveto{\pgfqpoint{1.178046in}{1.787268in}}{\pgfqpoint{1.174774in}{1.779368in}}{\pgfqpoint{1.174774in}{1.771132in}}%
\pgfpathcurveto{\pgfqpoint{1.174774in}{1.762896in}}{\pgfqpoint{1.178046in}{1.754995in}}{\pgfqpoint{1.183870in}{1.749172in}}%
\pgfpathcurveto{\pgfqpoint{1.189694in}{1.743348in}}{\pgfqpoint{1.197594in}{1.740075in}}{\pgfqpoint{1.205830in}{1.740075in}}%
\pgfpathclose%
\pgfusepath{stroke,fill}%
\end{pgfscope}%
\begin{pgfscope}%
\pgfpathrectangle{\pgfqpoint{0.100000in}{0.212622in}}{\pgfqpoint{3.696000in}{3.696000in}}%
\pgfusepath{clip}%
\pgfsetbuttcap%
\pgfsetroundjoin%
\definecolor{currentfill}{rgb}{0.121569,0.466667,0.705882}%
\pgfsetfillcolor{currentfill}%
\pgfsetfillopacity{0.367371}%
\pgfsetlinewidth{1.003750pt}%
\definecolor{currentstroke}{rgb}{0.121569,0.466667,0.705882}%
\pgfsetstrokecolor{currentstroke}%
\pgfsetstrokeopacity{0.367371}%
\pgfsetdash{}{0pt}%
\pgfpathmoveto{\pgfqpoint{1.205830in}{1.740075in}}%
\pgfpathcurveto{\pgfqpoint{1.214066in}{1.740075in}}{\pgfqpoint{1.221966in}{1.743348in}}{\pgfqpoint{1.227790in}{1.749172in}}%
\pgfpathcurveto{\pgfqpoint{1.233614in}{1.754995in}}{\pgfqpoint{1.236887in}{1.762896in}}{\pgfqpoint{1.236887in}{1.771132in}}%
\pgfpathcurveto{\pgfqpoint{1.236887in}{1.779368in}}{\pgfqpoint{1.233614in}{1.787268in}}{\pgfqpoint{1.227790in}{1.793092in}}%
\pgfpathcurveto{\pgfqpoint{1.221966in}{1.798916in}}{\pgfqpoint{1.214066in}{1.802188in}}{\pgfqpoint{1.205830in}{1.802188in}}%
\pgfpathcurveto{\pgfqpoint{1.197594in}{1.802188in}}{\pgfqpoint{1.189694in}{1.798916in}}{\pgfqpoint{1.183870in}{1.793092in}}%
\pgfpathcurveto{\pgfqpoint{1.178046in}{1.787268in}}{\pgfqpoint{1.174774in}{1.779368in}}{\pgfqpoint{1.174774in}{1.771132in}}%
\pgfpathcurveto{\pgfqpoint{1.174774in}{1.762896in}}{\pgfqpoint{1.178046in}{1.754995in}}{\pgfqpoint{1.183870in}{1.749172in}}%
\pgfpathcurveto{\pgfqpoint{1.189694in}{1.743348in}}{\pgfqpoint{1.197594in}{1.740075in}}{\pgfqpoint{1.205830in}{1.740075in}}%
\pgfpathclose%
\pgfusepath{stroke,fill}%
\end{pgfscope}%
\begin{pgfscope}%
\pgfpathrectangle{\pgfqpoint{0.100000in}{0.212622in}}{\pgfqpoint{3.696000in}{3.696000in}}%
\pgfusepath{clip}%
\pgfsetbuttcap%
\pgfsetroundjoin%
\definecolor{currentfill}{rgb}{0.121569,0.466667,0.705882}%
\pgfsetfillcolor{currentfill}%
\pgfsetfillopacity{0.367371}%
\pgfsetlinewidth{1.003750pt}%
\definecolor{currentstroke}{rgb}{0.121569,0.466667,0.705882}%
\pgfsetstrokecolor{currentstroke}%
\pgfsetstrokeopacity{0.367371}%
\pgfsetdash{}{0pt}%
\pgfpathmoveto{\pgfqpoint{1.205830in}{1.740075in}}%
\pgfpathcurveto{\pgfqpoint{1.214066in}{1.740075in}}{\pgfqpoint{1.221966in}{1.743348in}}{\pgfqpoint{1.227790in}{1.749172in}}%
\pgfpathcurveto{\pgfqpoint{1.233614in}{1.754995in}}{\pgfqpoint{1.236887in}{1.762896in}}{\pgfqpoint{1.236887in}{1.771132in}}%
\pgfpathcurveto{\pgfqpoint{1.236887in}{1.779368in}}{\pgfqpoint{1.233614in}{1.787268in}}{\pgfqpoint{1.227790in}{1.793092in}}%
\pgfpathcurveto{\pgfqpoint{1.221966in}{1.798916in}}{\pgfqpoint{1.214066in}{1.802188in}}{\pgfqpoint{1.205830in}{1.802188in}}%
\pgfpathcurveto{\pgfqpoint{1.197594in}{1.802188in}}{\pgfqpoint{1.189694in}{1.798916in}}{\pgfqpoint{1.183870in}{1.793092in}}%
\pgfpathcurveto{\pgfqpoint{1.178046in}{1.787268in}}{\pgfqpoint{1.174774in}{1.779368in}}{\pgfqpoint{1.174774in}{1.771132in}}%
\pgfpathcurveto{\pgfqpoint{1.174774in}{1.762896in}}{\pgfqpoint{1.178046in}{1.754995in}}{\pgfqpoint{1.183870in}{1.749172in}}%
\pgfpathcurveto{\pgfqpoint{1.189694in}{1.743348in}}{\pgfqpoint{1.197594in}{1.740075in}}{\pgfqpoint{1.205830in}{1.740075in}}%
\pgfpathclose%
\pgfusepath{stroke,fill}%
\end{pgfscope}%
\begin{pgfscope}%
\pgfpathrectangle{\pgfqpoint{0.100000in}{0.212622in}}{\pgfqpoint{3.696000in}{3.696000in}}%
\pgfusepath{clip}%
\pgfsetbuttcap%
\pgfsetroundjoin%
\definecolor{currentfill}{rgb}{0.121569,0.466667,0.705882}%
\pgfsetfillcolor{currentfill}%
\pgfsetfillopacity{0.367371}%
\pgfsetlinewidth{1.003750pt}%
\definecolor{currentstroke}{rgb}{0.121569,0.466667,0.705882}%
\pgfsetstrokecolor{currentstroke}%
\pgfsetstrokeopacity{0.367371}%
\pgfsetdash{}{0pt}%
\pgfpathmoveto{\pgfqpoint{1.205830in}{1.740075in}}%
\pgfpathcurveto{\pgfqpoint{1.214066in}{1.740075in}}{\pgfqpoint{1.221966in}{1.743348in}}{\pgfqpoint{1.227790in}{1.749172in}}%
\pgfpathcurveto{\pgfqpoint{1.233614in}{1.754995in}}{\pgfqpoint{1.236887in}{1.762896in}}{\pgfqpoint{1.236887in}{1.771132in}}%
\pgfpathcurveto{\pgfqpoint{1.236887in}{1.779368in}}{\pgfqpoint{1.233614in}{1.787268in}}{\pgfqpoint{1.227790in}{1.793092in}}%
\pgfpathcurveto{\pgfqpoint{1.221966in}{1.798916in}}{\pgfqpoint{1.214066in}{1.802188in}}{\pgfqpoint{1.205830in}{1.802188in}}%
\pgfpathcurveto{\pgfqpoint{1.197594in}{1.802188in}}{\pgfqpoint{1.189694in}{1.798916in}}{\pgfqpoint{1.183870in}{1.793092in}}%
\pgfpathcurveto{\pgfqpoint{1.178046in}{1.787268in}}{\pgfqpoint{1.174774in}{1.779368in}}{\pgfqpoint{1.174774in}{1.771132in}}%
\pgfpathcurveto{\pgfqpoint{1.174774in}{1.762896in}}{\pgfqpoint{1.178046in}{1.754995in}}{\pgfqpoint{1.183870in}{1.749172in}}%
\pgfpathcurveto{\pgfqpoint{1.189694in}{1.743348in}}{\pgfqpoint{1.197594in}{1.740075in}}{\pgfqpoint{1.205830in}{1.740075in}}%
\pgfpathclose%
\pgfusepath{stroke,fill}%
\end{pgfscope}%
\begin{pgfscope}%
\pgfpathrectangle{\pgfqpoint{0.100000in}{0.212622in}}{\pgfqpoint{3.696000in}{3.696000in}}%
\pgfusepath{clip}%
\pgfsetbuttcap%
\pgfsetroundjoin%
\definecolor{currentfill}{rgb}{0.121569,0.466667,0.705882}%
\pgfsetfillcolor{currentfill}%
\pgfsetfillopacity{0.367371}%
\pgfsetlinewidth{1.003750pt}%
\definecolor{currentstroke}{rgb}{0.121569,0.466667,0.705882}%
\pgfsetstrokecolor{currentstroke}%
\pgfsetstrokeopacity{0.367371}%
\pgfsetdash{}{0pt}%
\pgfpathmoveto{\pgfqpoint{1.205830in}{1.740075in}}%
\pgfpathcurveto{\pgfqpoint{1.214066in}{1.740075in}}{\pgfqpoint{1.221966in}{1.743348in}}{\pgfqpoint{1.227790in}{1.749172in}}%
\pgfpathcurveto{\pgfqpoint{1.233614in}{1.754995in}}{\pgfqpoint{1.236887in}{1.762896in}}{\pgfqpoint{1.236887in}{1.771132in}}%
\pgfpathcurveto{\pgfqpoint{1.236887in}{1.779368in}}{\pgfqpoint{1.233614in}{1.787268in}}{\pgfqpoint{1.227790in}{1.793092in}}%
\pgfpathcurveto{\pgfqpoint{1.221966in}{1.798916in}}{\pgfqpoint{1.214066in}{1.802188in}}{\pgfqpoint{1.205830in}{1.802188in}}%
\pgfpathcurveto{\pgfqpoint{1.197594in}{1.802188in}}{\pgfqpoint{1.189694in}{1.798916in}}{\pgfqpoint{1.183870in}{1.793092in}}%
\pgfpathcurveto{\pgfqpoint{1.178046in}{1.787268in}}{\pgfqpoint{1.174774in}{1.779368in}}{\pgfqpoint{1.174774in}{1.771132in}}%
\pgfpathcurveto{\pgfqpoint{1.174774in}{1.762896in}}{\pgfqpoint{1.178046in}{1.754995in}}{\pgfqpoint{1.183870in}{1.749172in}}%
\pgfpathcurveto{\pgfqpoint{1.189694in}{1.743348in}}{\pgfqpoint{1.197594in}{1.740075in}}{\pgfqpoint{1.205830in}{1.740075in}}%
\pgfpathclose%
\pgfusepath{stroke,fill}%
\end{pgfscope}%
\begin{pgfscope}%
\pgfpathrectangle{\pgfqpoint{0.100000in}{0.212622in}}{\pgfqpoint{3.696000in}{3.696000in}}%
\pgfusepath{clip}%
\pgfsetbuttcap%
\pgfsetroundjoin%
\definecolor{currentfill}{rgb}{0.121569,0.466667,0.705882}%
\pgfsetfillcolor{currentfill}%
\pgfsetfillopacity{0.367371}%
\pgfsetlinewidth{1.003750pt}%
\definecolor{currentstroke}{rgb}{0.121569,0.466667,0.705882}%
\pgfsetstrokecolor{currentstroke}%
\pgfsetstrokeopacity{0.367371}%
\pgfsetdash{}{0pt}%
\pgfpathmoveto{\pgfqpoint{1.205830in}{1.740075in}}%
\pgfpathcurveto{\pgfqpoint{1.214066in}{1.740075in}}{\pgfqpoint{1.221966in}{1.743348in}}{\pgfqpoint{1.227790in}{1.749172in}}%
\pgfpathcurveto{\pgfqpoint{1.233614in}{1.754995in}}{\pgfqpoint{1.236887in}{1.762896in}}{\pgfqpoint{1.236887in}{1.771132in}}%
\pgfpathcurveto{\pgfqpoint{1.236887in}{1.779368in}}{\pgfqpoint{1.233614in}{1.787268in}}{\pgfqpoint{1.227790in}{1.793092in}}%
\pgfpathcurveto{\pgfqpoint{1.221966in}{1.798916in}}{\pgfqpoint{1.214066in}{1.802188in}}{\pgfqpoint{1.205830in}{1.802188in}}%
\pgfpathcurveto{\pgfqpoint{1.197594in}{1.802188in}}{\pgfqpoint{1.189694in}{1.798916in}}{\pgfqpoint{1.183870in}{1.793092in}}%
\pgfpathcurveto{\pgfqpoint{1.178046in}{1.787268in}}{\pgfqpoint{1.174774in}{1.779368in}}{\pgfqpoint{1.174774in}{1.771132in}}%
\pgfpathcurveto{\pgfqpoint{1.174774in}{1.762896in}}{\pgfqpoint{1.178046in}{1.754995in}}{\pgfqpoint{1.183870in}{1.749172in}}%
\pgfpathcurveto{\pgfqpoint{1.189694in}{1.743348in}}{\pgfqpoint{1.197594in}{1.740075in}}{\pgfqpoint{1.205830in}{1.740075in}}%
\pgfpathclose%
\pgfusepath{stroke,fill}%
\end{pgfscope}%
\begin{pgfscope}%
\pgfpathrectangle{\pgfqpoint{0.100000in}{0.212622in}}{\pgfqpoint{3.696000in}{3.696000in}}%
\pgfusepath{clip}%
\pgfsetbuttcap%
\pgfsetroundjoin%
\definecolor{currentfill}{rgb}{0.121569,0.466667,0.705882}%
\pgfsetfillcolor{currentfill}%
\pgfsetfillopacity{0.367371}%
\pgfsetlinewidth{1.003750pt}%
\definecolor{currentstroke}{rgb}{0.121569,0.466667,0.705882}%
\pgfsetstrokecolor{currentstroke}%
\pgfsetstrokeopacity{0.367371}%
\pgfsetdash{}{0pt}%
\pgfpathmoveto{\pgfqpoint{1.205830in}{1.740075in}}%
\pgfpathcurveto{\pgfqpoint{1.214066in}{1.740075in}}{\pgfqpoint{1.221966in}{1.743348in}}{\pgfqpoint{1.227790in}{1.749172in}}%
\pgfpathcurveto{\pgfqpoint{1.233614in}{1.754995in}}{\pgfqpoint{1.236887in}{1.762896in}}{\pgfqpoint{1.236887in}{1.771132in}}%
\pgfpathcurveto{\pgfqpoint{1.236887in}{1.779368in}}{\pgfqpoint{1.233614in}{1.787268in}}{\pgfqpoint{1.227790in}{1.793092in}}%
\pgfpathcurveto{\pgfqpoint{1.221966in}{1.798916in}}{\pgfqpoint{1.214066in}{1.802188in}}{\pgfqpoint{1.205830in}{1.802188in}}%
\pgfpathcurveto{\pgfqpoint{1.197594in}{1.802188in}}{\pgfqpoint{1.189694in}{1.798916in}}{\pgfqpoint{1.183870in}{1.793092in}}%
\pgfpathcurveto{\pgfqpoint{1.178046in}{1.787268in}}{\pgfqpoint{1.174774in}{1.779368in}}{\pgfqpoint{1.174774in}{1.771132in}}%
\pgfpathcurveto{\pgfqpoint{1.174774in}{1.762896in}}{\pgfqpoint{1.178046in}{1.754995in}}{\pgfqpoint{1.183870in}{1.749172in}}%
\pgfpathcurveto{\pgfqpoint{1.189694in}{1.743348in}}{\pgfqpoint{1.197594in}{1.740075in}}{\pgfqpoint{1.205830in}{1.740075in}}%
\pgfpathclose%
\pgfusepath{stroke,fill}%
\end{pgfscope}%
\begin{pgfscope}%
\pgfpathrectangle{\pgfqpoint{0.100000in}{0.212622in}}{\pgfqpoint{3.696000in}{3.696000in}}%
\pgfusepath{clip}%
\pgfsetbuttcap%
\pgfsetroundjoin%
\definecolor{currentfill}{rgb}{0.121569,0.466667,0.705882}%
\pgfsetfillcolor{currentfill}%
\pgfsetfillopacity{0.367371}%
\pgfsetlinewidth{1.003750pt}%
\definecolor{currentstroke}{rgb}{0.121569,0.466667,0.705882}%
\pgfsetstrokecolor{currentstroke}%
\pgfsetstrokeopacity{0.367371}%
\pgfsetdash{}{0pt}%
\pgfpathmoveto{\pgfqpoint{1.205830in}{1.740075in}}%
\pgfpathcurveto{\pgfqpoint{1.214066in}{1.740075in}}{\pgfqpoint{1.221966in}{1.743348in}}{\pgfqpoint{1.227790in}{1.749172in}}%
\pgfpathcurveto{\pgfqpoint{1.233614in}{1.754995in}}{\pgfqpoint{1.236887in}{1.762896in}}{\pgfqpoint{1.236887in}{1.771132in}}%
\pgfpathcurveto{\pgfqpoint{1.236887in}{1.779368in}}{\pgfqpoint{1.233614in}{1.787268in}}{\pgfqpoint{1.227790in}{1.793092in}}%
\pgfpathcurveto{\pgfqpoint{1.221966in}{1.798916in}}{\pgfqpoint{1.214066in}{1.802188in}}{\pgfqpoint{1.205830in}{1.802188in}}%
\pgfpathcurveto{\pgfqpoint{1.197594in}{1.802188in}}{\pgfqpoint{1.189694in}{1.798916in}}{\pgfqpoint{1.183870in}{1.793092in}}%
\pgfpathcurveto{\pgfqpoint{1.178046in}{1.787268in}}{\pgfqpoint{1.174774in}{1.779368in}}{\pgfqpoint{1.174774in}{1.771132in}}%
\pgfpathcurveto{\pgfqpoint{1.174774in}{1.762896in}}{\pgfqpoint{1.178046in}{1.754995in}}{\pgfqpoint{1.183870in}{1.749172in}}%
\pgfpathcurveto{\pgfqpoint{1.189694in}{1.743348in}}{\pgfqpoint{1.197594in}{1.740075in}}{\pgfqpoint{1.205830in}{1.740075in}}%
\pgfpathclose%
\pgfusepath{stroke,fill}%
\end{pgfscope}%
\begin{pgfscope}%
\pgfpathrectangle{\pgfqpoint{0.100000in}{0.212622in}}{\pgfqpoint{3.696000in}{3.696000in}}%
\pgfusepath{clip}%
\pgfsetbuttcap%
\pgfsetroundjoin%
\definecolor{currentfill}{rgb}{0.121569,0.466667,0.705882}%
\pgfsetfillcolor{currentfill}%
\pgfsetfillopacity{0.367371}%
\pgfsetlinewidth{1.003750pt}%
\definecolor{currentstroke}{rgb}{0.121569,0.466667,0.705882}%
\pgfsetstrokecolor{currentstroke}%
\pgfsetstrokeopacity{0.367371}%
\pgfsetdash{}{0pt}%
\pgfpathmoveto{\pgfqpoint{1.205830in}{1.740075in}}%
\pgfpathcurveto{\pgfqpoint{1.214066in}{1.740075in}}{\pgfqpoint{1.221966in}{1.743348in}}{\pgfqpoint{1.227790in}{1.749172in}}%
\pgfpathcurveto{\pgfqpoint{1.233614in}{1.754995in}}{\pgfqpoint{1.236887in}{1.762896in}}{\pgfqpoint{1.236887in}{1.771132in}}%
\pgfpathcurveto{\pgfqpoint{1.236887in}{1.779368in}}{\pgfqpoint{1.233614in}{1.787268in}}{\pgfqpoint{1.227790in}{1.793092in}}%
\pgfpathcurveto{\pgfqpoint{1.221966in}{1.798916in}}{\pgfqpoint{1.214066in}{1.802188in}}{\pgfqpoint{1.205830in}{1.802188in}}%
\pgfpathcurveto{\pgfqpoint{1.197594in}{1.802188in}}{\pgfqpoint{1.189694in}{1.798916in}}{\pgfqpoint{1.183870in}{1.793092in}}%
\pgfpathcurveto{\pgfqpoint{1.178046in}{1.787268in}}{\pgfqpoint{1.174774in}{1.779368in}}{\pgfqpoint{1.174774in}{1.771132in}}%
\pgfpathcurveto{\pgfqpoint{1.174774in}{1.762896in}}{\pgfqpoint{1.178046in}{1.754995in}}{\pgfqpoint{1.183870in}{1.749172in}}%
\pgfpathcurveto{\pgfqpoint{1.189694in}{1.743348in}}{\pgfqpoint{1.197594in}{1.740075in}}{\pgfqpoint{1.205830in}{1.740075in}}%
\pgfpathclose%
\pgfusepath{stroke,fill}%
\end{pgfscope}%
\begin{pgfscope}%
\pgfpathrectangle{\pgfqpoint{0.100000in}{0.212622in}}{\pgfqpoint{3.696000in}{3.696000in}}%
\pgfusepath{clip}%
\pgfsetbuttcap%
\pgfsetroundjoin%
\definecolor{currentfill}{rgb}{0.121569,0.466667,0.705882}%
\pgfsetfillcolor{currentfill}%
\pgfsetfillopacity{0.367371}%
\pgfsetlinewidth{1.003750pt}%
\definecolor{currentstroke}{rgb}{0.121569,0.466667,0.705882}%
\pgfsetstrokecolor{currentstroke}%
\pgfsetstrokeopacity{0.367371}%
\pgfsetdash{}{0pt}%
\pgfpathmoveto{\pgfqpoint{1.205830in}{1.740075in}}%
\pgfpathcurveto{\pgfqpoint{1.214066in}{1.740075in}}{\pgfqpoint{1.221966in}{1.743348in}}{\pgfqpoint{1.227790in}{1.749172in}}%
\pgfpathcurveto{\pgfqpoint{1.233614in}{1.754995in}}{\pgfqpoint{1.236887in}{1.762896in}}{\pgfqpoint{1.236887in}{1.771132in}}%
\pgfpathcurveto{\pgfqpoint{1.236887in}{1.779368in}}{\pgfqpoint{1.233614in}{1.787268in}}{\pgfqpoint{1.227790in}{1.793092in}}%
\pgfpathcurveto{\pgfqpoint{1.221966in}{1.798916in}}{\pgfqpoint{1.214066in}{1.802188in}}{\pgfqpoint{1.205830in}{1.802188in}}%
\pgfpathcurveto{\pgfqpoint{1.197594in}{1.802188in}}{\pgfqpoint{1.189694in}{1.798916in}}{\pgfqpoint{1.183870in}{1.793092in}}%
\pgfpathcurveto{\pgfqpoint{1.178046in}{1.787268in}}{\pgfqpoint{1.174774in}{1.779368in}}{\pgfqpoint{1.174774in}{1.771132in}}%
\pgfpathcurveto{\pgfqpoint{1.174774in}{1.762896in}}{\pgfqpoint{1.178046in}{1.754995in}}{\pgfqpoint{1.183870in}{1.749172in}}%
\pgfpathcurveto{\pgfqpoint{1.189694in}{1.743348in}}{\pgfqpoint{1.197594in}{1.740075in}}{\pgfqpoint{1.205830in}{1.740075in}}%
\pgfpathclose%
\pgfusepath{stroke,fill}%
\end{pgfscope}%
\begin{pgfscope}%
\pgfpathrectangle{\pgfqpoint{0.100000in}{0.212622in}}{\pgfqpoint{3.696000in}{3.696000in}}%
\pgfusepath{clip}%
\pgfsetbuttcap%
\pgfsetroundjoin%
\definecolor{currentfill}{rgb}{0.121569,0.466667,0.705882}%
\pgfsetfillcolor{currentfill}%
\pgfsetfillopacity{0.367371}%
\pgfsetlinewidth{1.003750pt}%
\definecolor{currentstroke}{rgb}{0.121569,0.466667,0.705882}%
\pgfsetstrokecolor{currentstroke}%
\pgfsetstrokeopacity{0.367371}%
\pgfsetdash{}{0pt}%
\pgfpathmoveto{\pgfqpoint{1.205830in}{1.740075in}}%
\pgfpathcurveto{\pgfqpoint{1.214066in}{1.740075in}}{\pgfqpoint{1.221966in}{1.743348in}}{\pgfqpoint{1.227790in}{1.749172in}}%
\pgfpathcurveto{\pgfqpoint{1.233614in}{1.754995in}}{\pgfqpoint{1.236887in}{1.762896in}}{\pgfqpoint{1.236887in}{1.771132in}}%
\pgfpathcurveto{\pgfqpoint{1.236887in}{1.779368in}}{\pgfqpoint{1.233614in}{1.787268in}}{\pgfqpoint{1.227790in}{1.793092in}}%
\pgfpathcurveto{\pgfqpoint{1.221966in}{1.798916in}}{\pgfqpoint{1.214066in}{1.802188in}}{\pgfqpoint{1.205830in}{1.802188in}}%
\pgfpathcurveto{\pgfqpoint{1.197594in}{1.802188in}}{\pgfqpoint{1.189694in}{1.798916in}}{\pgfqpoint{1.183870in}{1.793092in}}%
\pgfpathcurveto{\pgfqpoint{1.178046in}{1.787268in}}{\pgfqpoint{1.174774in}{1.779368in}}{\pgfqpoint{1.174774in}{1.771132in}}%
\pgfpathcurveto{\pgfqpoint{1.174774in}{1.762896in}}{\pgfqpoint{1.178046in}{1.754995in}}{\pgfqpoint{1.183870in}{1.749172in}}%
\pgfpathcurveto{\pgfqpoint{1.189694in}{1.743348in}}{\pgfqpoint{1.197594in}{1.740075in}}{\pgfqpoint{1.205830in}{1.740075in}}%
\pgfpathclose%
\pgfusepath{stroke,fill}%
\end{pgfscope}%
\begin{pgfscope}%
\pgfpathrectangle{\pgfqpoint{0.100000in}{0.212622in}}{\pgfqpoint{3.696000in}{3.696000in}}%
\pgfusepath{clip}%
\pgfsetbuttcap%
\pgfsetroundjoin%
\definecolor{currentfill}{rgb}{0.121569,0.466667,0.705882}%
\pgfsetfillcolor{currentfill}%
\pgfsetfillopacity{0.367371}%
\pgfsetlinewidth{1.003750pt}%
\definecolor{currentstroke}{rgb}{0.121569,0.466667,0.705882}%
\pgfsetstrokecolor{currentstroke}%
\pgfsetstrokeopacity{0.367371}%
\pgfsetdash{}{0pt}%
\pgfpathmoveto{\pgfqpoint{1.205830in}{1.740075in}}%
\pgfpathcurveto{\pgfqpoint{1.214066in}{1.740075in}}{\pgfqpoint{1.221966in}{1.743348in}}{\pgfqpoint{1.227790in}{1.749172in}}%
\pgfpathcurveto{\pgfqpoint{1.233614in}{1.754995in}}{\pgfqpoint{1.236887in}{1.762896in}}{\pgfqpoint{1.236887in}{1.771132in}}%
\pgfpathcurveto{\pgfqpoint{1.236887in}{1.779368in}}{\pgfqpoint{1.233614in}{1.787268in}}{\pgfqpoint{1.227790in}{1.793092in}}%
\pgfpathcurveto{\pgfqpoint{1.221966in}{1.798916in}}{\pgfqpoint{1.214066in}{1.802188in}}{\pgfqpoint{1.205830in}{1.802188in}}%
\pgfpathcurveto{\pgfqpoint{1.197594in}{1.802188in}}{\pgfqpoint{1.189694in}{1.798916in}}{\pgfqpoint{1.183870in}{1.793092in}}%
\pgfpathcurveto{\pgfqpoint{1.178046in}{1.787268in}}{\pgfqpoint{1.174774in}{1.779368in}}{\pgfqpoint{1.174774in}{1.771132in}}%
\pgfpathcurveto{\pgfqpoint{1.174774in}{1.762896in}}{\pgfqpoint{1.178046in}{1.754995in}}{\pgfqpoint{1.183870in}{1.749172in}}%
\pgfpathcurveto{\pgfqpoint{1.189694in}{1.743348in}}{\pgfqpoint{1.197594in}{1.740075in}}{\pgfqpoint{1.205830in}{1.740075in}}%
\pgfpathclose%
\pgfusepath{stroke,fill}%
\end{pgfscope}%
\begin{pgfscope}%
\pgfpathrectangle{\pgfqpoint{0.100000in}{0.212622in}}{\pgfqpoint{3.696000in}{3.696000in}}%
\pgfusepath{clip}%
\pgfsetbuttcap%
\pgfsetroundjoin%
\definecolor{currentfill}{rgb}{0.121569,0.466667,0.705882}%
\pgfsetfillcolor{currentfill}%
\pgfsetfillopacity{0.367371}%
\pgfsetlinewidth{1.003750pt}%
\definecolor{currentstroke}{rgb}{0.121569,0.466667,0.705882}%
\pgfsetstrokecolor{currentstroke}%
\pgfsetstrokeopacity{0.367371}%
\pgfsetdash{}{0pt}%
\pgfpathmoveto{\pgfqpoint{1.205830in}{1.740075in}}%
\pgfpathcurveto{\pgfqpoint{1.214066in}{1.740075in}}{\pgfqpoint{1.221966in}{1.743348in}}{\pgfqpoint{1.227790in}{1.749172in}}%
\pgfpathcurveto{\pgfqpoint{1.233614in}{1.754995in}}{\pgfqpoint{1.236887in}{1.762896in}}{\pgfqpoint{1.236887in}{1.771132in}}%
\pgfpathcurveto{\pgfqpoint{1.236887in}{1.779368in}}{\pgfqpoint{1.233614in}{1.787268in}}{\pgfqpoint{1.227790in}{1.793092in}}%
\pgfpathcurveto{\pgfqpoint{1.221966in}{1.798916in}}{\pgfqpoint{1.214066in}{1.802188in}}{\pgfqpoint{1.205830in}{1.802188in}}%
\pgfpathcurveto{\pgfqpoint{1.197594in}{1.802188in}}{\pgfqpoint{1.189694in}{1.798916in}}{\pgfqpoint{1.183870in}{1.793092in}}%
\pgfpathcurveto{\pgfqpoint{1.178046in}{1.787268in}}{\pgfqpoint{1.174774in}{1.779368in}}{\pgfqpoint{1.174774in}{1.771132in}}%
\pgfpathcurveto{\pgfqpoint{1.174774in}{1.762896in}}{\pgfqpoint{1.178046in}{1.754995in}}{\pgfqpoint{1.183870in}{1.749172in}}%
\pgfpathcurveto{\pgfqpoint{1.189694in}{1.743348in}}{\pgfqpoint{1.197594in}{1.740075in}}{\pgfqpoint{1.205830in}{1.740075in}}%
\pgfpathclose%
\pgfusepath{stroke,fill}%
\end{pgfscope}%
\begin{pgfscope}%
\pgfpathrectangle{\pgfqpoint{0.100000in}{0.212622in}}{\pgfqpoint{3.696000in}{3.696000in}}%
\pgfusepath{clip}%
\pgfsetbuttcap%
\pgfsetroundjoin%
\definecolor{currentfill}{rgb}{0.121569,0.466667,0.705882}%
\pgfsetfillcolor{currentfill}%
\pgfsetfillopacity{0.367371}%
\pgfsetlinewidth{1.003750pt}%
\definecolor{currentstroke}{rgb}{0.121569,0.466667,0.705882}%
\pgfsetstrokecolor{currentstroke}%
\pgfsetstrokeopacity{0.367371}%
\pgfsetdash{}{0pt}%
\pgfpathmoveto{\pgfqpoint{1.205830in}{1.740075in}}%
\pgfpathcurveto{\pgfqpoint{1.214066in}{1.740075in}}{\pgfqpoint{1.221966in}{1.743348in}}{\pgfqpoint{1.227790in}{1.749172in}}%
\pgfpathcurveto{\pgfqpoint{1.233614in}{1.754995in}}{\pgfqpoint{1.236887in}{1.762896in}}{\pgfqpoint{1.236887in}{1.771132in}}%
\pgfpathcurveto{\pgfqpoint{1.236887in}{1.779368in}}{\pgfqpoint{1.233614in}{1.787268in}}{\pgfqpoint{1.227790in}{1.793092in}}%
\pgfpathcurveto{\pgfqpoint{1.221966in}{1.798916in}}{\pgfqpoint{1.214066in}{1.802188in}}{\pgfqpoint{1.205830in}{1.802188in}}%
\pgfpathcurveto{\pgfqpoint{1.197594in}{1.802188in}}{\pgfqpoint{1.189694in}{1.798916in}}{\pgfqpoint{1.183870in}{1.793092in}}%
\pgfpathcurveto{\pgfqpoint{1.178046in}{1.787268in}}{\pgfqpoint{1.174774in}{1.779368in}}{\pgfqpoint{1.174774in}{1.771132in}}%
\pgfpathcurveto{\pgfqpoint{1.174774in}{1.762896in}}{\pgfqpoint{1.178046in}{1.754995in}}{\pgfqpoint{1.183870in}{1.749172in}}%
\pgfpathcurveto{\pgfqpoint{1.189694in}{1.743348in}}{\pgfqpoint{1.197594in}{1.740075in}}{\pgfqpoint{1.205830in}{1.740075in}}%
\pgfpathclose%
\pgfusepath{stroke,fill}%
\end{pgfscope}%
\begin{pgfscope}%
\pgfpathrectangle{\pgfqpoint{0.100000in}{0.212622in}}{\pgfqpoint{3.696000in}{3.696000in}}%
\pgfusepath{clip}%
\pgfsetbuttcap%
\pgfsetroundjoin%
\definecolor{currentfill}{rgb}{0.121569,0.466667,0.705882}%
\pgfsetfillcolor{currentfill}%
\pgfsetfillopacity{0.367371}%
\pgfsetlinewidth{1.003750pt}%
\definecolor{currentstroke}{rgb}{0.121569,0.466667,0.705882}%
\pgfsetstrokecolor{currentstroke}%
\pgfsetstrokeopacity{0.367371}%
\pgfsetdash{}{0pt}%
\pgfpathmoveto{\pgfqpoint{1.205830in}{1.740075in}}%
\pgfpathcurveto{\pgfqpoint{1.214066in}{1.740075in}}{\pgfqpoint{1.221966in}{1.743348in}}{\pgfqpoint{1.227790in}{1.749172in}}%
\pgfpathcurveto{\pgfqpoint{1.233614in}{1.754995in}}{\pgfqpoint{1.236887in}{1.762896in}}{\pgfqpoint{1.236887in}{1.771132in}}%
\pgfpathcurveto{\pgfqpoint{1.236887in}{1.779368in}}{\pgfqpoint{1.233614in}{1.787268in}}{\pgfqpoint{1.227790in}{1.793092in}}%
\pgfpathcurveto{\pgfqpoint{1.221966in}{1.798916in}}{\pgfqpoint{1.214066in}{1.802188in}}{\pgfqpoint{1.205830in}{1.802188in}}%
\pgfpathcurveto{\pgfqpoint{1.197594in}{1.802188in}}{\pgfqpoint{1.189694in}{1.798916in}}{\pgfqpoint{1.183870in}{1.793092in}}%
\pgfpathcurveto{\pgfqpoint{1.178046in}{1.787268in}}{\pgfqpoint{1.174774in}{1.779368in}}{\pgfqpoint{1.174774in}{1.771132in}}%
\pgfpathcurveto{\pgfqpoint{1.174774in}{1.762896in}}{\pgfqpoint{1.178046in}{1.754995in}}{\pgfqpoint{1.183870in}{1.749172in}}%
\pgfpathcurveto{\pgfqpoint{1.189694in}{1.743348in}}{\pgfqpoint{1.197594in}{1.740075in}}{\pgfqpoint{1.205830in}{1.740075in}}%
\pgfpathclose%
\pgfusepath{stroke,fill}%
\end{pgfscope}%
\begin{pgfscope}%
\pgfpathrectangle{\pgfqpoint{0.100000in}{0.212622in}}{\pgfqpoint{3.696000in}{3.696000in}}%
\pgfusepath{clip}%
\pgfsetbuttcap%
\pgfsetroundjoin%
\definecolor{currentfill}{rgb}{0.121569,0.466667,0.705882}%
\pgfsetfillcolor{currentfill}%
\pgfsetfillopacity{0.367371}%
\pgfsetlinewidth{1.003750pt}%
\definecolor{currentstroke}{rgb}{0.121569,0.466667,0.705882}%
\pgfsetstrokecolor{currentstroke}%
\pgfsetstrokeopacity{0.367371}%
\pgfsetdash{}{0pt}%
\pgfpathmoveto{\pgfqpoint{1.205830in}{1.740075in}}%
\pgfpathcurveto{\pgfqpoint{1.214066in}{1.740075in}}{\pgfqpoint{1.221966in}{1.743348in}}{\pgfqpoint{1.227790in}{1.749172in}}%
\pgfpathcurveto{\pgfqpoint{1.233614in}{1.754995in}}{\pgfqpoint{1.236887in}{1.762896in}}{\pgfqpoint{1.236887in}{1.771132in}}%
\pgfpathcurveto{\pgfqpoint{1.236887in}{1.779368in}}{\pgfqpoint{1.233614in}{1.787268in}}{\pgfqpoint{1.227790in}{1.793092in}}%
\pgfpathcurveto{\pgfqpoint{1.221966in}{1.798916in}}{\pgfqpoint{1.214066in}{1.802188in}}{\pgfqpoint{1.205830in}{1.802188in}}%
\pgfpathcurveto{\pgfqpoint{1.197594in}{1.802188in}}{\pgfqpoint{1.189694in}{1.798916in}}{\pgfqpoint{1.183870in}{1.793092in}}%
\pgfpathcurveto{\pgfqpoint{1.178046in}{1.787268in}}{\pgfqpoint{1.174774in}{1.779368in}}{\pgfqpoint{1.174774in}{1.771132in}}%
\pgfpathcurveto{\pgfqpoint{1.174774in}{1.762896in}}{\pgfqpoint{1.178046in}{1.754995in}}{\pgfqpoint{1.183870in}{1.749172in}}%
\pgfpathcurveto{\pgfqpoint{1.189694in}{1.743348in}}{\pgfqpoint{1.197594in}{1.740075in}}{\pgfqpoint{1.205830in}{1.740075in}}%
\pgfpathclose%
\pgfusepath{stroke,fill}%
\end{pgfscope}%
\begin{pgfscope}%
\pgfpathrectangle{\pgfqpoint{0.100000in}{0.212622in}}{\pgfqpoint{3.696000in}{3.696000in}}%
\pgfusepath{clip}%
\pgfsetbuttcap%
\pgfsetroundjoin%
\definecolor{currentfill}{rgb}{0.121569,0.466667,0.705882}%
\pgfsetfillcolor{currentfill}%
\pgfsetfillopacity{0.367371}%
\pgfsetlinewidth{1.003750pt}%
\definecolor{currentstroke}{rgb}{0.121569,0.466667,0.705882}%
\pgfsetstrokecolor{currentstroke}%
\pgfsetstrokeopacity{0.367371}%
\pgfsetdash{}{0pt}%
\pgfpathmoveto{\pgfqpoint{1.205830in}{1.740075in}}%
\pgfpathcurveto{\pgfqpoint{1.214066in}{1.740075in}}{\pgfqpoint{1.221966in}{1.743348in}}{\pgfqpoint{1.227790in}{1.749172in}}%
\pgfpathcurveto{\pgfqpoint{1.233614in}{1.754995in}}{\pgfqpoint{1.236887in}{1.762896in}}{\pgfqpoint{1.236887in}{1.771132in}}%
\pgfpathcurveto{\pgfqpoint{1.236887in}{1.779368in}}{\pgfqpoint{1.233614in}{1.787268in}}{\pgfqpoint{1.227790in}{1.793092in}}%
\pgfpathcurveto{\pgfqpoint{1.221966in}{1.798916in}}{\pgfqpoint{1.214066in}{1.802188in}}{\pgfqpoint{1.205830in}{1.802188in}}%
\pgfpathcurveto{\pgfqpoint{1.197594in}{1.802188in}}{\pgfqpoint{1.189694in}{1.798916in}}{\pgfqpoint{1.183870in}{1.793092in}}%
\pgfpathcurveto{\pgfqpoint{1.178046in}{1.787268in}}{\pgfqpoint{1.174774in}{1.779368in}}{\pgfqpoint{1.174774in}{1.771132in}}%
\pgfpathcurveto{\pgfqpoint{1.174774in}{1.762896in}}{\pgfqpoint{1.178046in}{1.754995in}}{\pgfqpoint{1.183870in}{1.749172in}}%
\pgfpathcurveto{\pgfqpoint{1.189694in}{1.743348in}}{\pgfqpoint{1.197594in}{1.740075in}}{\pgfqpoint{1.205830in}{1.740075in}}%
\pgfpathclose%
\pgfusepath{stroke,fill}%
\end{pgfscope}%
\begin{pgfscope}%
\pgfpathrectangle{\pgfqpoint{0.100000in}{0.212622in}}{\pgfqpoint{3.696000in}{3.696000in}}%
\pgfusepath{clip}%
\pgfsetbuttcap%
\pgfsetroundjoin%
\definecolor{currentfill}{rgb}{0.121569,0.466667,0.705882}%
\pgfsetfillcolor{currentfill}%
\pgfsetfillopacity{0.367371}%
\pgfsetlinewidth{1.003750pt}%
\definecolor{currentstroke}{rgb}{0.121569,0.466667,0.705882}%
\pgfsetstrokecolor{currentstroke}%
\pgfsetstrokeopacity{0.367371}%
\pgfsetdash{}{0pt}%
\pgfpathmoveto{\pgfqpoint{1.205830in}{1.740075in}}%
\pgfpathcurveto{\pgfqpoint{1.214066in}{1.740075in}}{\pgfqpoint{1.221966in}{1.743348in}}{\pgfqpoint{1.227790in}{1.749172in}}%
\pgfpathcurveto{\pgfqpoint{1.233614in}{1.754995in}}{\pgfqpoint{1.236887in}{1.762896in}}{\pgfqpoint{1.236887in}{1.771132in}}%
\pgfpathcurveto{\pgfqpoint{1.236887in}{1.779368in}}{\pgfqpoint{1.233614in}{1.787268in}}{\pgfqpoint{1.227790in}{1.793092in}}%
\pgfpathcurveto{\pgfqpoint{1.221966in}{1.798916in}}{\pgfqpoint{1.214066in}{1.802188in}}{\pgfqpoint{1.205830in}{1.802188in}}%
\pgfpathcurveto{\pgfqpoint{1.197594in}{1.802188in}}{\pgfqpoint{1.189694in}{1.798916in}}{\pgfqpoint{1.183870in}{1.793092in}}%
\pgfpathcurveto{\pgfqpoint{1.178046in}{1.787268in}}{\pgfqpoint{1.174774in}{1.779368in}}{\pgfqpoint{1.174774in}{1.771132in}}%
\pgfpathcurveto{\pgfqpoint{1.174774in}{1.762896in}}{\pgfqpoint{1.178046in}{1.754995in}}{\pgfqpoint{1.183870in}{1.749172in}}%
\pgfpathcurveto{\pgfqpoint{1.189694in}{1.743348in}}{\pgfqpoint{1.197594in}{1.740075in}}{\pgfqpoint{1.205830in}{1.740075in}}%
\pgfpathclose%
\pgfusepath{stroke,fill}%
\end{pgfscope}%
\begin{pgfscope}%
\pgfpathrectangle{\pgfqpoint{0.100000in}{0.212622in}}{\pgfqpoint{3.696000in}{3.696000in}}%
\pgfusepath{clip}%
\pgfsetbuttcap%
\pgfsetroundjoin%
\definecolor{currentfill}{rgb}{0.121569,0.466667,0.705882}%
\pgfsetfillcolor{currentfill}%
\pgfsetfillopacity{0.367371}%
\pgfsetlinewidth{1.003750pt}%
\definecolor{currentstroke}{rgb}{0.121569,0.466667,0.705882}%
\pgfsetstrokecolor{currentstroke}%
\pgfsetstrokeopacity{0.367371}%
\pgfsetdash{}{0pt}%
\pgfpathmoveto{\pgfqpoint{1.205830in}{1.740075in}}%
\pgfpathcurveto{\pgfqpoint{1.214066in}{1.740075in}}{\pgfqpoint{1.221966in}{1.743348in}}{\pgfqpoint{1.227790in}{1.749172in}}%
\pgfpathcurveto{\pgfqpoint{1.233614in}{1.754995in}}{\pgfqpoint{1.236887in}{1.762896in}}{\pgfqpoint{1.236887in}{1.771132in}}%
\pgfpathcurveto{\pgfqpoint{1.236887in}{1.779368in}}{\pgfqpoint{1.233614in}{1.787268in}}{\pgfqpoint{1.227790in}{1.793092in}}%
\pgfpathcurveto{\pgfqpoint{1.221966in}{1.798916in}}{\pgfqpoint{1.214066in}{1.802188in}}{\pgfqpoint{1.205830in}{1.802188in}}%
\pgfpathcurveto{\pgfqpoint{1.197594in}{1.802188in}}{\pgfqpoint{1.189694in}{1.798916in}}{\pgfqpoint{1.183870in}{1.793092in}}%
\pgfpathcurveto{\pgfqpoint{1.178046in}{1.787268in}}{\pgfqpoint{1.174774in}{1.779368in}}{\pgfqpoint{1.174774in}{1.771132in}}%
\pgfpathcurveto{\pgfqpoint{1.174774in}{1.762896in}}{\pgfqpoint{1.178046in}{1.754995in}}{\pgfqpoint{1.183870in}{1.749172in}}%
\pgfpathcurveto{\pgfqpoint{1.189694in}{1.743348in}}{\pgfqpoint{1.197594in}{1.740075in}}{\pgfqpoint{1.205830in}{1.740075in}}%
\pgfpathclose%
\pgfusepath{stroke,fill}%
\end{pgfscope}%
\begin{pgfscope}%
\pgfpathrectangle{\pgfqpoint{0.100000in}{0.212622in}}{\pgfqpoint{3.696000in}{3.696000in}}%
\pgfusepath{clip}%
\pgfsetbuttcap%
\pgfsetroundjoin%
\definecolor{currentfill}{rgb}{0.121569,0.466667,0.705882}%
\pgfsetfillcolor{currentfill}%
\pgfsetfillopacity{0.367371}%
\pgfsetlinewidth{1.003750pt}%
\definecolor{currentstroke}{rgb}{0.121569,0.466667,0.705882}%
\pgfsetstrokecolor{currentstroke}%
\pgfsetstrokeopacity{0.367371}%
\pgfsetdash{}{0pt}%
\pgfpathmoveto{\pgfqpoint{1.205830in}{1.740075in}}%
\pgfpathcurveto{\pgfqpoint{1.214066in}{1.740075in}}{\pgfqpoint{1.221966in}{1.743348in}}{\pgfqpoint{1.227790in}{1.749172in}}%
\pgfpathcurveto{\pgfqpoint{1.233614in}{1.754995in}}{\pgfqpoint{1.236887in}{1.762896in}}{\pgfqpoint{1.236887in}{1.771132in}}%
\pgfpathcurveto{\pgfqpoint{1.236887in}{1.779368in}}{\pgfqpoint{1.233614in}{1.787268in}}{\pgfqpoint{1.227790in}{1.793092in}}%
\pgfpathcurveto{\pgfqpoint{1.221966in}{1.798916in}}{\pgfqpoint{1.214066in}{1.802188in}}{\pgfqpoint{1.205830in}{1.802188in}}%
\pgfpathcurveto{\pgfqpoint{1.197594in}{1.802188in}}{\pgfqpoint{1.189694in}{1.798916in}}{\pgfqpoint{1.183870in}{1.793092in}}%
\pgfpathcurveto{\pgfqpoint{1.178046in}{1.787268in}}{\pgfqpoint{1.174774in}{1.779368in}}{\pgfqpoint{1.174774in}{1.771132in}}%
\pgfpathcurveto{\pgfqpoint{1.174774in}{1.762896in}}{\pgfqpoint{1.178046in}{1.754995in}}{\pgfqpoint{1.183870in}{1.749172in}}%
\pgfpathcurveto{\pgfqpoint{1.189694in}{1.743348in}}{\pgfqpoint{1.197594in}{1.740075in}}{\pgfqpoint{1.205830in}{1.740075in}}%
\pgfpathclose%
\pgfusepath{stroke,fill}%
\end{pgfscope}%
\begin{pgfscope}%
\pgfpathrectangle{\pgfqpoint{0.100000in}{0.212622in}}{\pgfqpoint{3.696000in}{3.696000in}}%
\pgfusepath{clip}%
\pgfsetbuttcap%
\pgfsetroundjoin%
\definecolor{currentfill}{rgb}{0.121569,0.466667,0.705882}%
\pgfsetfillcolor{currentfill}%
\pgfsetfillopacity{0.367371}%
\pgfsetlinewidth{1.003750pt}%
\definecolor{currentstroke}{rgb}{0.121569,0.466667,0.705882}%
\pgfsetstrokecolor{currentstroke}%
\pgfsetstrokeopacity{0.367371}%
\pgfsetdash{}{0pt}%
\pgfpathmoveto{\pgfqpoint{1.205830in}{1.740075in}}%
\pgfpathcurveto{\pgfqpoint{1.214066in}{1.740075in}}{\pgfqpoint{1.221966in}{1.743348in}}{\pgfqpoint{1.227790in}{1.749172in}}%
\pgfpathcurveto{\pgfqpoint{1.233614in}{1.754995in}}{\pgfqpoint{1.236887in}{1.762896in}}{\pgfqpoint{1.236887in}{1.771132in}}%
\pgfpathcurveto{\pgfqpoint{1.236887in}{1.779368in}}{\pgfqpoint{1.233614in}{1.787268in}}{\pgfqpoint{1.227790in}{1.793092in}}%
\pgfpathcurveto{\pgfqpoint{1.221966in}{1.798916in}}{\pgfqpoint{1.214066in}{1.802188in}}{\pgfqpoint{1.205830in}{1.802188in}}%
\pgfpathcurveto{\pgfqpoint{1.197594in}{1.802188in}}{\pgfqpoint{1.189694in}{1.798916in}}{\pgfqpoint{1.183870in}{1.793092in}}%
\pgfpathcurveto{\pgfqpoint{1.178046in}{1.787268in}}{\pgfqpoint{1.174774in}{1.779368in}}{\pgfqpoint{1.174774in}{1.771132in}}%
\pgfpathcurveto{\pgfqpoint{1.174774in}{1.762896in}}{\pgfqpoint{1.178046in}{1.754995in}}{\pgfqpoint{1.183870in}{1.749172in}}%
\pgfpathcurveto{\pgfqpoint{1.189694in}{1.743348in}}{\pgfqpoint{1.197594in}{1.740075in}}{\pgfqpoint{1.205830in}{1.740075in}}%
\pgfpathclose%
\pgfusepath{stroke,fill}%
\end{pgfscope}%
\begin{pgfscope}%
\pgfpathrectangle{\pgfqpoint{0.100000in}{0.212622in}}{\pgfqpoint{3.696000in}{3.696000in}}%
\pgfusepath{clip}%
\pgfsetbuttcap%
\pgfsetroundjoin%
\definecolor{currentfill}{rgb}{0.121569,0.466667,0.705882}%
\pgfsetfillcolor{currentfill}%
\pgfsetfillopacity{0.367371}%
\pgfsetlinewidth{1.003750pt}%
\definecolor{currentstroke}{rgb}{0.121569,0.466667,0.705882}%
\pgfsetstrokecolor{currentstroke}%
\pgfsetstrokeopacity{0.367371}%
\pgfsetdash{}{0pt}%
\pgfpathmoveto{\pgfqpoint{1.205830in}{1.740075in}}%
\pgfpathcurveto{\pgfqpoint{1.214066in}{1.740075in}}{\pgfqpoint{1.221966in}{1.743348in}}{\pgfqpoint{1.227790in}{1.749172in}}%
\pgfpathcurveto{\pgfqpoint{1.233614in}{1.754995in}}{\pgfqpoint{1.236887in}{1.762896in}}{\pgfqpoint{1.236887in}{1.771132in}}%
\pgfpathcurveto{\pgfqpoint{1.236887in}{1.779368in}}{\pgfqpoint{1.233614in}{1.787268in}}{\pgfqpoint{1.227790in}{1.793092in}}%
\pgfpathcurveto{\pgfqpoint{1.221966in}{1.798916in}}{\pgfqpoint{1.214066in}{1.802188in}}{\pgfqpoint{1.205830in}{1.802188in}}%
\pgfpathcurveto{\pgfqpoint{1.197594in}{1.802188in}}{\pgfqpoint{1.189694in}{1.798916in}}{\pgfqpoint{1.183870in}{1.793092in}}%
\pgfpathcurveto{\pgfqpoint{1.178046in}{1.787268in}}{\pgfqpoint{1.174774in}{1.779368in}}{\pgfqpoint{1.174774in}{1.771132in}}%
\pgfpathcurveto{\pgfqpoint{1.174774in}{1.762896in}}{\pgfqpoint{1.178046in}{1.754995in}}{\pgfqpoint{1.183870in}{1.749172in}}%
\pgfpathcurveto{\pgfqpoint{1.189694in}{1.743348in}}{\pgfqpoint{1.197594in}{1.740075in}}{\pgfqpoint{1.205830in}{1.740075in}}%
\pgfpathclose%
\pgfusepath{stroke,fill}%
\end{pgfscope}%
\begin{pgfscope}%
\pgfpathrectangle{\pgfqpoint{0.100000in}{0.212622in}}{\pgfqpoint{3.696000in}{3.696000in}}%
\pgfusepath{clip}%
\pgfsetbuttcap%
\pgfsetroundjoin%
\definecolor{currentfill}{rgb}{0.121569,0.466667,0.705882}%
\pgfsetfillcolor{currentfill}%
\pgfsetfillopacity{0.367371}%
\pgfsetlinewidth{1.003750pt}%
\definecolor{currentstroke}{rgb}{0.121569,0.466667,0.705882}%
\pgfsetstrokecolor{currentstroke}%
\pgfsetstrokeopacity{0.367371}%
\pgfsetdash{}{0pt}%
\pgfpathmoveto{\pgfqpoint{1.205830in}{1.740075in}}%
\pgfpathcurveto{\pgfqpoint{1.214066in}{1.740075in}}{\pgfqpoint{1.221966in}{1.743348in}}{\pgfqpoint{1.227790in}{1.749172in}}%
\pgfpathcurveto{\pgfqpoint{1.233614in}{1.754995in}}{\pgfqpoint{1.236887in}{1.762896in}}{\pgfqpoint{1.236887in}{1.771132in}}%
\pgfpathcurveto{\pgfqpoint{1.236887in}{1.779368in}}{\pgfqpoint{1.233614in}{1.787268in}}{\pgfqpoint{1.227790in}{1.793092in}}%
\pgfpathcurveto{\pgfqpoint{1.221966in}{1.798916in}}{\pgfqpoint{1.214066in}{1.802188in}}{\pgfqpoint{1.205830in}{1.802188in}}%
\pgfpathcurveto{\pgfqpoint{1.197594in}{1.802188in}}{\pgfqpoint{1.189694in}{1.798916in}}{\pgfqpoint{1.183870in}{1.793092in}}%
\pgfpathcurveto{\pgfqpoint{1.178046in}{1.787268in}}{\pgfqpoint{1.174774in}{1.779368in}}{\pgfqpoint{1.174774in}{1.771132in}}%
\pgfpathcurveto{\pgfqpoint{1.174774in}{1.762896in}}{\pgfqpoint{1.178046in}{1.754995in}}{\pgfqpoint{1.183870in}{1.749172in}}%
\pgfpathcurveto{\pgfqpoint{1.189694in}{1.743348in}}{\pgfqpoint{1.197594in}{1.740075in}}{\pgfqpoint{1.205830in}{1.740075in}}%
\pgfpathclose%
\pgfusepath{stroke,fill}%
\end{pgfscope}%
\begin{pgfscope}%
\pgfpathrectangle{\pgfqpoint{0.100000in}{0.212622in}}{\pgfqpoint{3.696000in}{3.696000in}}%
\pgfusepath{clip}%
\pgfsetbuttcap%
\pgfsetroundjoin%
\definecolor{currentfill}{rgb}{0.121569,0.466667,0.705882}%
\pgfsetfillcolor{currentfill}%
\pgfsetfillopacity{0.367371}%
\pgfsetlinewidth{1.003750pt}%
\definecolor{currentstroke}{rgb}{0.121569,0.466667,0.705882}%
\pgfsetstrokecolor{currentstroke}%
\pgfsetstrokeopacity{0.367371}%
\pgfsetdash{}{0pt}%
\pgfpathmoveto{\pgfqpoint{1.205830in}{1.740075in}}%
\pgfpathcurveto{\pgfqpoint{1.214066in}{1.740075in}}{\pgfqpoint{1.221966in}{1.743348in}}{\pgfqpoint{1.227790in}{1.749172in}}%
\pgfpathcurveto{\pgfqpoint{1.233614in}{1.754995in}}{\pgfqpoint{1.236887in}{1.762896in}}{\pgfqpoint{1.236887in}{1.771132in}}%
\pgfpathcurveto{\pgfqpoint{1.236887in}{1.779368in}}{\pgfqpoint{1.233614in}{1.787268in}}{\pgfqpoint{1.227790in}{1.793092in}}%
\pgfpathcurveto{\pgfqpoint{1.221966in}{1.798916in}}{\pgfqpoint{1.214066in}{1.802188in}}{\pgfqpoint{1.205830in}{1.802188in}}%
\pgfpathcurveto{\pgfqpoint{1.197594in}{1.802188in}}{\pgfqpoint{1.189694in}{1.798916in}}{\pgfqpoint{1.183870in}{1.793092in}}%
\pgfpathcurveto{\pgfqpoint{1.178046in}{1.787268in}}{\pgfqpoint{1.174774in}{1.779368in}}{\pgfqpoint{1.174774in}{1.771132in}}%
\pgfpathcurveto{\pgfqpoint{1.174774in}{1.762896in}}{\pgfqpoint{1.178046in}{1.754995in}}{\pgfqpoint{1.183870in}{1.749172in}}%
\pgfpathcurveto{\pgfqpoint{1.189694in}{1.743348in}}{\pgfqpoint{1.197594in}{1.740075in}}{\pgfqpoint{1.205830in}{1.740075in}}%
\pgfpathclose%
\pgfusepath{stroke,fill}%
\end{pgfscope}%
\begin{pgfscope}%
\pgfpathrectangle{\pgfqpoint{0.100000in}{0.212622in}}{\pgfqpoint{3.696000in}{3.696000in}}%
\pgfusepath{clip}%
\pgfsetbuttcap%
\pgfsetroundjoin%
\definecolor{currentfill}{rgb}{0.121569,0.466667,0.705882}%
\pgfsetfillcolor{currentfill}%
\pgfsetfillopacity{0.367371}%
\pgfsetlinewidth{1.003750pt}%
\definecolor{currentstroke}{rgb}{0.121569,0.466667,0.705882}%
\pgfsetstrokecolor{currentstroke}%
\pgfsetstrokeopacity{0.367371}%
\pgfsetdash{}{0pt}%
\pgfpathmoveto{\pgfqpoint{1.205830in}{1.740075in}}%
\pgfpathcurveto{\pgfqpoint{1.214066in}{1.740075in}}{\pgfqpoint{1.221966in}{1.743348in}}{\pgfqpoint{1.227790in}{1.749172in}}%
\pgfpathcurveto{\pgfqpoint{1.233614in}{1.754995in}}{\pgfqpoint{1.236887in}{1.762896in}}{\pgfqpoint{1.236887in}{1.771132in}}%
\pgfpathcurveto{\pgfqpoint{1.236887in}{1.779368in}}{\pgfqpoint{1.233614in}{1.787268in}}{\pgfqpoint{1.227790in}{1.793092in}}%
\pgfpathcurveto{\pgfqpoint{1.221966in}{1.798916in}}{\pgfqpoint{1.214066in}{1.802188in}}{\pgfqpoint{1.205830in}{1.802188in}}%
\pgfpathcurveto{\pgfqpoint{1.197594in}{1.802188in}}{\pgfqpoint{1.189694in}{1.798916in}}{\pgfqpoint{1.183870in}{1.793092in}}%
\pgfpathcurveto{\pgfqpoint{1.178046in}{1.787268in}}{\pgfqpoint{1.174774in}{1.779368in}}{\pgfqpoint{1.174774in}{1.771132in}}%
\pgfpathcurveto{\pgfqpoint{1.174774in}{1.762896in}}{\pgfqpoint{1.178046in}{1.754995in}}{\pgfqpoint{1.183870in}{1.749172in}}%
\pgfpathcurveto{\pgfqpoint{1.189694in}{1.743348in}}{\pgfqpoint{1.197594in}{1.740075in}}{\pgfqpoint{1.205830in}{1.740075in}}%
\pgfpathclose%
\pgfusepath{stroke,fill}%
\end{pgfscope}%
\begin{pgfscope}%
\pgfpathrectangle{\pgfqpoint{0.100000in}{0.212622in}}{\pgfqpoint{3.696000in}{3.696000in}}%
\pgfusepath{clip}%
\pgfsetbuttcap%
\pgfsetroundjoin%
\definecolor{currentfill}{rgb}{0.121569,0.466667,0.705882}%
\pgfsetfillcolor{currentfill}%
\pgfsetfillopacity{0.367371}%
\pgfsetlinewidth{1.003750pt}%
\definecolor{currentstroke}{rgb}{0.121569,0.466667,0.705882}%
\pgfsetstrokecolor{currentstroke}%
\pgfsetstrokeopacity{0.367371}%
\pgfsetdash{}{0pt}%
\pgfpathmoveto{\pgfqpoint{1.205830in}{1.740075in}}%
\pgfpathcurveto{\pgfqpoint{1.214066in}{1.740075in}}{\pgfqpoint{1.221966in}{1.743348in}}{\pgfqpoint{1.227790in}{1.749172in}}%
\pgfpathcurveto{\pgfqpoint{1.233614in}{1.754995in}}{\pgfqpoint{1.236887in}{1.762896in}}{\pgfqpoint{1.236887in}{1.771132in}}%
\pgfpathcurveto{\pgfqpoint{1.236887in}{1.779368in}}{\pgfqpoint{1.233614in}{1.787268in}}{\pgfqpoint{1.227790in}{1.793092in}}%
\pgfpathcurveto{\pgfqpoint{1.221966in}{1.798916in}}{\pgfqpoint{1.214066in}{1.802188in}}{\pgfqpoint{1.205830in}{1.802188in}}%
\pgfpathcurveto{\pgfqpoint{1.197594in}{1.802188in}}{\pgfqpoint{1.189694in}{1.798916in}}{\pgfqpoint{1.183870in}{1.793092in}}%
\pgfpathcurveto{\pgfqpoint{1.178046in}{1.787268in}}{\pgfqpoint{1.174774in}{1.779368in}}{\pgfqpoint{1.174774in}{1.771132in}}%
\pgfpathcurveto{\pgfqpoint{1.174774in}{1.762896in}}{\pgfqpoint{1.178046in}{1.754995in}}{\pgfqpoint{1.183870in}{1.749172in}}%
\pgfpathcurveto{\pgfqpoint{1.189694in}{1.743348in}}{\pgfqpoint{1.197594in}{1.740075in}}{\pgfqpoint{1.205830in}{1.740075in}}%
\pgfpathclose%
\pgfusepath{stroke,fill}%
\end{pgfscope}%
\begin{pgfscope}%
\pgfpathrectangle{\pgfqpoint{0.100000in}{0.212622in}}{\pgfqpoint{3.696000in}{3.696000in}}%
\pgfusepath{clip}%
\pgfsetbuttcap%
\pgfsetroundjoin%
\definecolor{currentfill}{rgb}{0.121569,0.466667,0.705882}%
\pgfsetfillcolor{currentfill}%
\pgfsetfillopacity{0.367371}%
\pgfsetlinewidth{1.003750pt}%
\definecolor{currentstroke}{rgb}{0.121569,0.466667,0.705882}%
\pgfsetstrokecolor{currentstroke}%
\pgfsetstrokeopacity{0.367371}%
\pgfsetdash{}{0pt}%
\pgfpathmoveto{\pgfqpoint{1.205830in}{1.740075in}}%
\pgfpathcurveto{\pgfqpoint{1.214066in}{1.740075in}}{\pgfqpoint{1.221966in}{1.743348in}}{\pgfqpoint{1.227790in}{1.749172in}}%
\pgfpathcurveto{\pgfqpoint{1.233614in}{1.754995in}}{\pgfqpoint{1.236887in}{1.762896in}}{\pgfqpoint{1.236887in}{1.771132in}}%
\pgfpathcurveto{\pgfqpoint{1.236887in}{1.779368in}}{\pgfqpoint{1.233614in}{1.787268in}}{\pgfqpoint{1.227790in}{1.793092in}}%
\pgfpathcurveto{\pgfqpoint{1.221966in}{1.798916in}}{\pgfqpoint{1.214066in}{1.802188in}}{\pgfqpoint{1.205830in}{1.802188in}}%
\pgfpathcurveto{\pgfqpoint{1.197594in}{1.802188in}}{\pgfqpoint{1.189694in}{1.798916in}}{\pgfqpoint{1.183870in}{1.793092in}}%
\pgfpathcurveto{\pgfqpoint{1.178046in}{1.787268in}}{\pgfqpoint{1.174774in}{1.779368in}}{\pgfqpoint{1.174774in}{1.771132in}}%
\pgfpathcurveto{\pgfqpoint{1.174774in}{1.762896in}}{\pgfqpoint{1.178046in}{1.754995in}}{\pgfqpoint{1.183870in}{1.749172in}}%
\pgfpathcurveto{\pgfqpoint{1.189694in}{1.743348in}}{\pgfqpoint{1.197594in}{1.740075in}}{\pgfqpoint{1.205830in}{1.740075in}}%
\pgfpathclose%
\pgfusepath{stroke,fill}%
\end{pgfscope}%
\begin{pgfscope}%
\pgfpathrectangle{\pgfqpoint{0.100000in}{0.212622in}}{\pgfqpoint{3.696000in}{3.696000in}}%
\pgfusepath{clip}%
\pgfsetbuttcap%
\pgfsetroundjoin%
\definecolor{currentfill}{rgb}{0.121569,0.466667,0.705882}%
\pgfsetfillcolor{currentfill}%
\pgfsetfillopacity{0.367371}%
\pgfsetlinewidth{1.003750pt}%
\definecolor{currentstroke}{rgb}{0.121569,0.466667,0.705882}%
\pgfsetstrokecolor{currentstroke}%
\pgfsetstrokeopacity{0.367371}%
\pgfsetdash{}{0pt}%
\pgfpathmoveto{\pgfqpoint{1.205830in}{1.740075in}}%
\pgfpathcurveto{\pgfqpoint{1.214066in}{1.740075in}}{\pgfqpoint{1.221966in}{1.743348in}}{\pgfqpoint{1.227790in}{1.749172in}}%
\pgfpathcurveto{\pgfqpoint{1.233614in}{1.754995in}}{\pgfqpoint{1.236887in}{1.762896in}}{\pgfqpoint{1.236887in}{1.771132in}}%
\pgfpathcurveto{\pgfqpoint{1.236887in}{1.779368in}}{\pgfqpoint{1.233614in}{1.787268in}}{\pgfqpoint{1.227790in}{1.793092in}}%
\pgfpathcurveto{\pgfqpoint{1.221966in}{1.798916in}}{\pgfqpoint{1.214066in}{1.802188in}}{\pgfqpoint{1.205830in}{1.802188in}}%
\pgfpathcurveto{\pgfqpoint{1.197594in}{1.802188in}}{\pgfqpoint{1.189694in}{1.798916in}}{\pgfqpoint{1.183870in}{1.793092in}}%
\pgfpathcurveto{\pgfqpoint{1.178046in}{1.787268in}}{\pgfqpoint{1.174774in}{1.779368in}}{\pgfqpoint{1.174774in}{1.771132in}}%
\pgfpathcurveto{\pgfqpoint{1.174774in}{1.762896in}}{\pgfqpoint{1.178046in}{1.754995in}}{\pgfqpoint{1.183870in}{1.749172in}}%
\pgfpathcurveto{\pgfqpoint{1.189694in}{1.743348in}}{\pgfqpoint{1.197594in}{1.740075in}}{\pgfqpoint{1.205830in}{1.740075in}}%
\pgfpathclose%
\pgfusepath{stroke,fill}%
\end{pgfscope}%
\begin{pgfscope}%
\pgfpathrectangle{\pgfqpoint{0.100000in}{0.212622in}}{\pgfqpoint{3.696000in}{3.696000in}}%
\pgfusepath{clip}%
\pgfsetbuttcap%
\pgfsetroundjoin%
\definecolor{currentfill}{rgb}{0.121569,0.466667,0.705882}%
\pgfsetfillcolor{currentfill}%
\pgfsetfillopacity{0.367371}%
\pgfsetlinewidth{1.003750pt}%
\definecolor{currentstroke}{rgb}{0.121569,0.466667,0.705882}%
\pgfsetstrokecolor{currentstroke}%
\pgfsetstrokeopacity{0.367371}%
\pgfsetdash{}{0pt}%
\pgfpathmoveto{\pgfqpoint{1.205830in}{1.740075in}}%
\pgfpathcurveto{\pgfqpoint{1.214066in}{1.740075in}}{\pgfqpoint{1.221966in}{1.743348in}}{\pgfqpoint{1.227790in}{1.749172in}}%
\pgfpathcurveto{\pgfqpoint{1.233614in}{1.754995in}}{\pgfqpoint{1.236887in}{1.762896in}}{\pgfqpoint{1.236887in}{1.771132in}}%
\pgfpathcurveto{\pgfqpoint{1.236887in}{1.779368in}}{\pgfqpoint{1.233614in}{1.787268in}}{\pgfqpoint{1.227790in}{1.793092in}}%
\pgfpathcurveto{\pgfqpoint{1.221966in}{1.798916in}}{\pgfqpoint{1.214066in}{1.802188in}}{\pgfqpoint{1.205830in}{1.802188in}}%
\pgfpathcurveto{\pgfqpoint{1.197594in}{1.802188in}}{\pgfqpoint{1.189694in}{1.798916in}}{\pgfqpoint{1.183870in}{1.793092in}}%
\pgfpathcurveto{\pgfqpoint{1.178046in}{1.787268in}}{\pgfqpoint{1.174774in}{1.779368in}}{\pgfqpoint{1.174774in}{1.771132in}}%
\pgfpathcurveto{\pgfqpoint{1.174774in}{1.762896in}}{\pgfqpoint{1.178046in}{1.754995in}}{\pgfqpoint{1.183870in}{1.749172in}}%
\pgfpathcurveto{\pgfqpoint{1.189694in}{1.743348in}}{\pgfqpoint{1.197594in}{1.740075in}}{\pgfqpoint{1.205830in}{1.740075in}}%
\pgfpathclose%
\pgfusepath{stroke,fill}%
\end{pgfscope}%
\begin{pgfscope}%
\pgfpathrectangle{\pgfqpoint{0.100000in}{0.212622in}}{\pgfqpoint{3.696000in}{3.696000in}}%
\pgfusepath{clip}%
\pgfsetbuttcap%
\pgfsetroundjoin%
\definecolor{currentfill}{rgb}{0.121569,0.466667,0.705882}%
\pgfsetfillcolor{currentfill}%
\pgfsetfillopacity{0.367371}%
\pgfsetlinewidth{1.003750pt}%
\definecolor{currentstroke}{rgb}{0.121569,0.466667,0.705882}%
\pgfsetstrokecolor{currentstroke}%
\pgfsetstrokeopacity{0.367371}%
\pgfsetdash{}{0pt}%
\pgfpathmoveto{\pgfqpoint{1.205830in}{1.740075in}}%
\pgfpathcurveto{\pgfqpoint{1.214066in}{1.740075in}}{\pgfqpoint{1.221966in}{1.743348in}}{\pgfqpoint{1.227790in}{1.749172in}}%
\pgfpathcurveto{\pgfqpoint{1.233614in}{1.754995in}}{\pgfqpoint{1.236887in}{1.762896in}}{\pgfqpoint{1.236887in}{1.771132in}}%
\pgfpathcurveto{\pgfqpoint{1.236887in}{1.779368in}}{\pgfqpoint{1.233614in}{1.787268in}}{\pgfqpoint{1.227790in}{1.793092in}}%
\pgfpathcurveto{\pgfqpoint{1.221966in}{1.798916in}}{\pgfqpoint{1.214066in}{1.802188in}}{\pgfqpoint{1.205830in}{1.802188in}}%
\pgfpathcurveto{\pgfqpoint{1.197594in}{1.802188in}}{\pgfqpoint{1.189694in}{1.798916in}}{\pgfqpoint{1.183870in}{1.793092in}}%
\pgfpathcurveto{\pgfqpoint{1.178046in}{1.787268in}}{\pgfqpoint{1.174774in}{1.779368in}}{\pgfqpoint{1.174774in}{1.771132in}}%
\pgfpathcurveto{\pgfqpoint{1.174774in}{1.762896in}}{\pgfqpoint{1.178046in}{1.754995in}}{\pgfqpoint{1.183870in}{1.749172in}}%
\pgfpathcurveto{\pgfqpoint{1.189694in}{1.743348in}}{\pgfqpoint{1.197594in}{1.740075in}}{\pgfqpoint{1.205830in}{1.740075in}}%
\pgfpathclose%
\pgfusepath{stroke,fill}%
\end{pgfscope}%
\begin{pgfscope}%
\pgfpathrectangle{\pgfqpoint{0.100000in}{0.212622in}}{\pgfqpoint{3.696000in}{3.696000in}}%
\pgfusepath{clip}%
\pgfsetbuttcap%
\pgfsetroundjoin%
\definecolor{currentfill}{rgb}{0.121569,0.466667,0.705882}%
\pgfsetfillcolor{currentfill}%
\pgfsetfillopacity{0.367371}%
\pgfsetlinewidth{1.003750pt}%
\definecolor{currentstroke}{rgb}{0.121569,0.466667,0.705882}%
\pgfsetstrokecolor{currentstroke}%
\pgfsetstrokeopacity{0.367371}%
\pgfsetdash{}{0pt}%
\pgfpathmoveto{\pgfqpoint{1.205830in}{1.740075in}}%
\pgfpathcurveto{\pgfqpoint{1.214066in}{1.740075in}}{\pgfqpoint{1.221966in}{1.743348in}}{\pgfqpoint{1.227790in}{1.749172in}}%
\pgfpathcurveto{\pgfqpoint{1.233614in}{1.754995in}}{\pgfqpoint{1.236887in}{1.762896in}}{\pgfqpoint{1.236887in}{1.771132in}}%
\pgfpathcurveto{\pgfqpoint{1.236887in}{1.779368in}}{\pgfqpoint{1.233614in}{1.787268in}}{\pgfqpoint{1.227790in}{1.793092in}}%
\pgfpathcurveto{\pgfqpoint{1.221966in}{1.798916in}}{\pgfqpoint{1.214066in}{1.802188in}}{\pgfqpoint{1.205830in}{1.802188in}}%
\pgfpathcurveto{\pgfqpoint{1.197594in}{1.802188in}}{\pgfqpoint{1.189694in}{1.798916in}}{\pgfqpoint{1.183870in}{1.793092in}}%
\pgfpathcurveto{\pgfqpoint{1.178046in}{1.787268in}}{\pgfqpoint{1.174774in}{1.779368in}}{\pgfqpoint{1.174774in}{1.771132in}}%
\pgfpathcurveto{\pgfqpoint{1.174774in}{1.762896in}}{\pgfqpoint{1.178046in}{1.754995in}}{\pgfqpoint{1.183870in}{1.749172in}}%
\pgfpathcurveto{\pgfqpoint{1.189694in}{1.743348in}}{\pgfqpoint{1.197594in}{1.740075in}}{\pgfqpoint{1.205830in}{1.740075in}}%
\pgfpathclose%
\pgfusepath{stroke,fill}%
\end{pgfscope}%
\begin{pgfscope}%
\pgfpathrectangle{\pgfqpoint{0.100000in}{0.212622in}}{\pgfqpoint{3.696000in}{3.696000in}}%
\pgfusepath{clip}%
\pgfsetbuttcap%
\pgfsetroundjoin%
\definecolor{currentfill}{rgb}{0.121569,0.466667,0.705882}%
\pgfsetfillcolor{currentfill}%
\pgfsetfillopacity{0.367371}%
\pgfsetlinewidth{1.003750pt}%
\definecolor{currentstroke}{rgb}{0.121569,0.466667,0.705882}%
\pgfsetstrokecolor{currentstroke}%
\pgfsetstrokeopacity{0.367371}%
\pgfsetdash{}{0pt}%
\pgfpathmoveto{\pgfqpoint{1.205830in}{1.740075in}}%
\pgfpathcurveto{\pgfqpoint{1.214066in}{1.740075in}}{\pgfqpoint{1.221966in}{1.743348in}}{\pgfqpoint{1.227790in}{1.749172in}}%
\pgfpathcurveto{\pgfqpoint{1.233614in}{1.754995in}}{\pgfqpoint{1.236887in}{1.762896in}}{\pgfqpoint{1.236887in}{1.771132in}}%
\pgfpathcurveto{\pgfqpoint{1.236887in}{1.779368in}}{\pgfqpoint{1.233614in}{1.787268in}}{\pgfqpoint{1.227790in}{1.793092in}}%
\pgfpathcurveto{\pgfqpoint{1.221966in}{1.798916in}}{\pgfqpoint{1.214066in}{1.802188in}}{\pgfqpoint{1.205830in}{1.802188in}}%
\pgfpathcurveto{\pgfqpoint{1.197594in}{1.802188in}}{\pgfqpoint{1.189694in}{1.798916in}}{\pgfqpoint{1.183870in}{1.793092in}}%
\pgfpathcurveto{\pgfqpoint{1.178046in}{1.787268in}}{\pgfqpoint{1.174774in}{1.779368in}}{\pgfqpoint{1.174774in}{1.771132in}}%
\pgfpathcurveto{\pgfqpoint{1.174774in}{1.762896in}}{\pgfqpoint{1.178046in}{1.754995in}}{\pgfqpoint{1.183870in}{1.749172in}}%
\pgfpathcurveto{\pgfqpoint{1.189694in}{1.743348in}}{\pgfqpoint{1.197594in}{1.740075in}}{\pgfqpoint{1.205830in}{1.740075in}}%
\pgfpathclose%
\pgfusepath{stroke,fill}%
\end{pgfscope}%
\begin{pgfscope}%
\pgfpathrectangle{\pgfqpoint{0.100000in}{0.212622in}}{\pgfqpoint{3.696000in}{3.696000in}}%
\pgfusepath{clip}%
\pgfsetbuttcap%
\pgfsetroundjoin%
\definecolor{currentfill}{rgb}{0.121569,0.466667,0.705882}%
\pgfsetfillcolor{currentfill}%
\pgfsetfillopacity{0.367371}%
\pgfsetlinewidth{1.003750pt}%
\definecolor{currentstroke}{rgb}{0.121569,0.466667,0.705882}%
\pgfsetstrokecolor{currentstroke}%
\pgfsetstrokeopacity{0.367371}%
\pgfsetdash{}{0pt}%
\pgfpathmoveto{\pgfqpoint{1.205830in}{1.740075in}}%
\pgfpathcurveto{\pgfqpoint{1.214066in}{1.740075in}}{\pgfqpoint{1.221966in}{1.743348in}}{\pgfqpoint{1.227790in}{1.749172in}}%
\pgfpathcurveto{\pgfqpoint{1.233614in}{1.754995in}}{\pgfqpoint{1.236887in}{1.762896in}}{\pgfqpoint{1.236887in}{1.771132in}}%
\pgfpathcurveto{\pgfqpoint{1.236887in}{1.779368in}}{\pgfqpoint{1.233614in}{1.787268in}}{\pgfqpoint{1.227790in}{1.793092in}}%
\pgfpathcurveto{\pgfqpoint{1.221966in}{1.798916in}}{\pgfqpoint{1.214066in}{1.802188in}}{\pgfqpoint{1.205830in}{1.802188in}}%
\pgfpathcurveto{\pgfqpoint{1.197594in}{1.802188in}}{\pgfqpoint{1.189694in}{1.798916in}}{\pgfqpoint{1.183870in}{1.793092in}}%
\pgfpathcurveto{\pgfqpoint{1.178046in}{1.787268in}}{\pgfqpoint{1.174774in}{1.779368in}}{\pgfqpoint{1.174774in}{1.771132in}}%
\pgfpathcurveto{\pgfqpoint{1.174774in}{1.762896in}}{\pgfqpoint{1.178046in}{1.754995in}}{\pgfqpoint{1.183870in}{1.749172in}}%
\pgfpathcurveto{\pgfqpoint{1.189694in}{1.743348in}}{\pgfqpoint{1.197594in}{1.740075in}}{\pgfqpoint{1.205830in}{1.740075in}}%
\pgfpathclose%
\pgfusepath{stroke,fill}%
\end{pgfscope}%
\begin{pgfscope}%
\pgfpathrectangle{\pgfqpoint{0.100000in}{0.212622in}}{\pgfqpoint{3.696000in}{3.696000in}}%
\pgfusepath{clip}%
\pgfsetbuttcap%
\pgfsetroundjoin%
\definecolor{currentfill}{rgb}{0.121569,0.466667,0.705882}%
\pgfsetfillcolor{currentfill}%
\pgfsetfillopacity{0.367371}%
\pgfsetlinewidth{1.003750pt}%
\definecolor{currentstroke}{rgb}{0.121569,0.466667,0.705882}%
\pgfsetstrokecolor{currentstroke}%
\pgfsetstrokeopacity{0.367371}%
\pgfsetdash{}{0pt}%
\pgfpathmoveto{\pgfqpoint{1.205830in}{1.740075in}}%
\pgfpathcurveto{\pgfqpoint{1.214066in}{1.740075in}}{\pgfqpoint{1.221966in}{1.743348in}}{\pgfqpoint{1.227790in}{1.749172in}}%
\pgfpathcurveto{\pgfqpoint{1.233614in}{1.754995in}}{\pgfqpoint{1.236887in}{1.762896in}}{\pgfqpoint{1.236887in}{1.771132in}}%
\pgfpathcurveto{\pgfqpoint{1.236887in}{1.779368in}}{\pgfqpoint{1.233614in}{1.787268in}}{\pgfqpoint{1.227790in}{1.793092in}}%
\pgfpathcurveto{\pgfqpoint{1.221966in}{1.798916in}}{\pgfqpoint{1.214066in}{1.802188in}}{\pgfqpoint{1.205830in}{1.802188in}}%
\pgfpathcurveto{\pgfqpoint{1.197594in}{1.802188in}}{\pgfqpoint{1.189694in}{1.798916in}}{\pgfqpoint{1.183870in}{1.793092in}}%
\pgfpathcurveto{\pgfqpoint{1.178046in}{1.787268in}}{\pgfqpoint{1.174774in}{1.779368in}}{\pgfqpoint{1.174774in}{1.771132in}}%
\pgfpathcurveto{\pgfqpoint{1.174774in}{1.762896in}}{\pgfqpoint{1.178046in}{1.754995in}}{\pgfqpoint{1.183870in}{1.749172in}}%
\pgfpathcurveto{\pgfqpoint{1.189694in}{1.743348in}}{\pgfqpoint{1.197594in}{1.740075in}}{\pgfqpoint{1.205830in}{1.740075in}}%
\pgfpathclose%
\pgfusepath{stroke,fill}%
\end{pgfscope}%
\begin{pgfscope}%
\pgfpathrectangle{\pgfqpoint{0.100000in}{0.212622in}}{\pgfqpoint{3.696000in}{3.696000in}}%
\pgfusepath{clip}%
\pgfsetbuttcap%
\pgfsetroundjoin%
\definecolor{currentfill}{rgb}{0.121569,0.466667,0.705882}%
\pgfsetfillcolor{currentfill}%
\pgfsetfillopacity{0.367371}%
\pgfsetlinewidth{1.003750pt}%
\definecolor{currentstroke}{rgb}{0.121569,0.466667,0.705882}%
\pgfsetstrokecolor{currentstroke}%
\pgfsetstrokeopacity{0.367371}%
\pgfsetdash{}{0pt}%
\pgfpathmoveto{\pgfqpoint{1.205830in}{1.740075in}}%
\pgfpathcurveto{\pgfqpoint{1.214066in}{1.740075in}}{\pgfqpoint{1.221966in}{1.743348in}}{\pgfqpoint{1.227790in}{1.749172in}}%
\pgfpathcurveto{\pgfqpoint{1.233614in}{1.754995in}}{\pgfqpoint{1.236887in}{1.762896in}}{\pgfqpoint{1.236887in}{1.771132in}}%
\pgfpathcurveto{\pgfqpoint{1.236887in}{1.779368in}}{\pgfqpoint{1.233614in}{1.787268in}}{\pgfqpoint{1.227790in}{1.793092in}}%
\pgfpathcurveto{\pgfqpoint{1.221966in}{1.798916in}}{\pgfqpoint{1.214066in}{1.802188in}}{\pgfqpoint{1.205830in}{1.802188in}}%
\pgfpathcurveto{\pgfqpoint{1.197594in}{1.802188in}}{\pgfqpoint{1.189694in}{1.798916in}}{\pgfqpoint{1.183870in}{1.793092in}}%
\pgfpathcurveto{\pgfqpoint{1.178046in}{1.787268in}}{\pgfqpoint{1.174774in}{1.779368in}}{\pgfqpoint{1.174774in}{1.771132in}}%
\pgfpathcurveto{\pgfqpoint{1.174774in}{1.762896in}}{\pgfqpoint{1.178046in}{1.754995in}}{\pgfqpoint{1.183870in}{1.749172in}}%
\pgfpathcurveto{\pgfqpoint{1.189694in}{1.743348in}}{\pgfqpoint{1.197594in}{1.740075in}}{\pgfqpoint{1.205830in}{1.740075in}}%
\pgfpathclose%
\pgfusepath{stroke,fill}%
\end{pgfscope}%
\begin{pgfscope}%
\pgfpathrectangle{\pgfqpoint{0.100000in}{0.212622in}}{\pgfqpoint{3.696000in}{3.696000in}}%
\pgfusepath{clip}%
\pgfsetbuttcap%
\pgfsetroundjoin%
\definecolor{currentfill}{rgb}{0.121569,0.466667,0.705882}%
\pgfsetfillcolor{currentfill}%
\pgfsetfillopacity{0.367371}%
\pgfsetlinewidth{1.003750pt}%
\definecolor{currentstroke}{rgb}{0.121569,0.466667,0.705882}%
\pgfsetstrokecolor{currentstroke}%
\pgfsetstrokeopacity{0.367371}%
\pgfsetdash{}{0pt}%
\pgfpathmoveto{\pgfqpoint{1.205830in}{1.740075in}}%
\pgfpathcurveto{\pgfqpoint{1.214066in}{1.740075in}}{\pgfqpoint{1.221966in}{1.743348in}}{\pgfqpoint{1.227790in}{1.749172in}}%
\pgfpathcurveto{\pgfqpoint{1.233614in}{1.754995in}}{\pgfqpoint{1.236887in}{1.762896in}}{\pgfqpoint{1.236887in}{1.771132in}}%
\pgfpathcurveto{\pgfqpoint{1.236887in}{1.779368in}}{\pgfqpoint{1.233614in}{1.787268in}}{\pgfqpoint{1.227790in}{1.793092in}}%
\pgfpathcurveto{\pgfqpoint{1.221966in}{1.798916in}}{\pgfqpoint{1.214066in}{1.802188in}}{\pgfqpoint{1.205830in}{1.802188in}}%
\pgfpathcurveto{\pgfqpoint{1.197594in}{1.802188in}}{\pgfqpoint{1.189694in}{1.798916in}}{\pgfqpoint{1.183870in}{1.793092in}}%
\pgfpathcurveto{\pgfqpoint{1.178046in}{1.787268in}}{\pgfqpoint{1.174774in}{1.779368in}}{\pgfqpoint{1.174774in}{1.771132in}}%
\pgfpathcurveto{\pgfqpoint{1.174774in}{1.762896in}}{\pgfqpoint{1.178046in}{1.754995in}}{\pgfqpoint{1.183870in}{1.749172in}}%
\pgfpathcurveto{\pgfqpoint{1.189694in}{1.743348in}}{\pgfqpoint{1.197594in}{1.740075in}}{\pgfqpoint{1.205830in}{1.740075in}}%
\pgfpathclose%
\pgfusepath{stroke,fill}%
\end{pgfscope}%
\begin{pgfscope}%
\pgfpathrectangle{\pgfqpoint{0.100000in}{0.212622in}}{\pgfqpoint{3.696000in}{3.696000in}}%
\pgfusepath{clip}%
\pgfsetbuttcap%
\pgfsetroundjoin%
\definecolor{currentfill}{rgb}{0.121569,0.466667,0.705882}%
\pgfsetfillcolor{currentfill}%
\pgfsetfillopacity{0.367371}%
\pgfsetlinewidth{1.003750pt}%
\definecolor{currentstroke}{rgb}{0.121569,0.466667,0.705882}%
\pgfsetstrokecolor{currentstroke}%
\pgfsetstrokeopacity{0.367371}%
\pgfsetdash{}{0pt}%
\pgfpathmoveto{\pgfqpoint{1.205830in}{1.740075in}}%
\pgfpathcurveto{\pgfqpoint{1.214066in}{1.740075in}}{\pgfqpoint{1.221966in}{1.743348in}}{\pgfqpoint{1.227790in}{1.749172in}}%
\pgfpathcurveto{\pgfqpoint{1.233614in}{1.754995in}}{\pgfqpoint{1.236887in}{1.762896in}}{\pgfqpoint{1.236887in}{1.771132in}}%
\pgfpathcurveto{\pgfqpoint{1.236887in}{1.779368in}}{\pgfqpoint{1.233614in}{1.787268in}}{\pgfqpoint{1.227790in}{1.793092in}}%
\pgfpathcurveto{\pgfqpoint{1.221966in}{1.798916in}}{\pgfqpoint{1.214066in}{1.802188in}}{\pgfqpoint{1.205830in}{1.802188in}}%
\pgfpathcurveto{\pgfqpoint{1.197594in}{1.802188in}}{\pgfqpoint{1.189694in}{1.798916in}}{\pgfqpoint{1.183870in}{1.793092in}}%
\pgfpathcurveto{\pgfqpoint{1.178046in}{1.787268in}}{\pgfqpoint{1.174774in}{1.779368in}}{\pgfqpoint{1.174774in}{1.771132in}}%
\pgfpathcurveto{\pgfqpoint{1.174774in}{1.762896in}}{\pgfqpoint{1.178046in}{1.754995in}}{\pgfqpoint{1.183870in}{1.749172in}}%
\pgfpathcurveto{\pgfqpoint{1.189694in}{1.743348in}}{\pgfqpoint{1.197594in}{1.740075in}}{\pgfqpoint{1.205830in}{1.740075in}}%
\pgfpathclose%
\pgfusepath{stroke,fill}%
\end{pgfscope}%
\begin{pgfscope}%
\pgfpathrectangle{\pgfqpoint{0.100000in}{0.212622in}}{\pgfqpoint{3.696000in}{3.696000in}}%
\pgfusepath{clip}%
\pgfsetbuttcap%
\pgfsetroundjoin%
\definecolor{currentfill}{rgb}{0.121569,0.466667,0.705882}%
\pgfsetfillcolor{currentfill}%
\pgfsetfillopacity{0.367371}%
\pgfsetlinewidth{1.003750pt}%
\definecolor{currentstroke}{rgb}{0.121569,0.466667,0.705882}%
\pgfsetstrokecolor{currentstroke}%
\pgfsetstrokeopacity{0.367371}%
\pgfsetdash{}{0pt}%
\pgfpathmoveto{\pgfqpoint{1.205830in}{1.740075in}}%
\pgfpathcurveto{\pgfqpoint{1.214066in}{1.740075in}}{\pgfqpoint{1.221966in}{1.743348in}}{\pgfqpoint{1.227790in}{1.749172in}}%
\pgfpathcurveto{\pgfqpoint{1.233614in}{1.754995in}}{\pgfqpoint{1.236887in}{1.762896in}}{\pgfqpoint{1.236887in}{1.771132in}}%
\pgfpathcurveto{\pgfqpoint{1.236887in}{1.779368in}}{\pgfqpoint{1.233614in}{1.787268in}}{\pgfqpoint{1.227790in}{1.793092in}}%
\pgfpathcurveto{\pgfqpoint{1.221966in}{1.798916in}}{\pgfqpoint{1.214066in}{1.802188in}}{\pgfqpoint{1.205830in}{1.802188in}}%
\pgfpathcurveto{\pgfqpoint{1.197594in}{1.802188in}}{\pgfqpoint{1.189694in}{1.798916in}}{\pgfqpoint{1.183870in}{1.793092in}}%
\pgfpathcurveto{\pgfqpoint{1.178046in}{1.787268in}}{\pgfqpoint{1.174774in}{1.779368in}}{\pgfqpoint{1.174774in}{1.771132in}}%
\pgfpathcurveto{\pgfqpoint{1.174774in}{1.762896in}}{\pgfqpoint{1.178046in}{1.754995in}}{\pgfqpoint{1.183870in}{1.749172in}}%
\pgfpathcurveto{\pgfqpoint{1.189694in}{1.743348in}}{\pgfqpoint{1.197594in}{1.740075in}}{\pgfqpoint{1.205830in}{1.740075in}}%
\pgfpathclose%
\pgfusepath{stroke,fill}%
\end{pgfscope}%
\begin{pgfscope}%
\pgfpathrectangle{\pgfqpoint{0.100000in}{0.212622in}}{\pgfqpoint{3.696000in}{3.696000in}}%
\pgfusepath{clip}%
\pgfsetbuttcap%
\pgfsetroundjoin%
\definecolor{currentfill}{rgb}{0.121569,0.466667,0.705882}%
\pgfsetfillcolor{currentfill}%
\pgfsetfillopacity{0.367371}%
\pgfsetlinewidth{1.003750pt}%
\definecolor{currentstroke}{rgb}{0.121569,0.466667,0.705882}%
\pgfsetstrokecolor{currentstroke}%
\pgfsetstrokeopacity{0.367371}%
\pgfsetdash{}{0pt}%
\pgfpathmoveto{\pgfqpoint{1.205830in}{1.740075in}}%
\pgfpathcurveto{\pgfqpoint{1.214066in}{1.740075in}}{\pgfqpoint{1.221966in}{1.743348in}}{\pgfqpoint{1.227790in}{1.749172in}}%
\pgfpathcurveto{\pgfqpoint{1.233614in}{1.754995in}}{\pgfqpoint{1.236887in}{1.762896in}}{\pgfqpoint{1.236887in}{1.771132in}}%
\pgfpathcurveto{\pgfqpoint{1.236887in}{1.779368in}}{\pgfqpoint{1.233614in}{1.787268in}}{\pgfqpoint{1.227790in}{1.793092in}}%
\pgfpathcurveto{\pgfqpoint{1.221966in}{1.798916in}}{\pgfqpoint{1.214066in}{1.802188in}}{\pgfqpoint{1.205830in}{1.802188in}}%
\pgfpathcurveto{\pgfqpoint{1.197594in}{1.802188in}}{\pgfqpoint{1.189694in}{1.798916in}}{\pgfqpoint{1.183870in}{1.793092in}}%
\pgfpathcurveto{\pgfqpoint{1.178046in}{1.787268in}}{\pgfqpoint{1.174774in}{1.779368in}}{\pgfqpoint{1.174774in}{1.771132in}}%
\pgfpathcurveto{\pgfqpoint{1.174774in}{1.762896in}}{\pgfqpoint{1.178046in}{1.754995in}}{\pgfqpoint{1.183870in}{1.749172in}}%
\pgfpathcurveto{\pgfqpoint{1.189694in}{1.743348in}}{\pgfqpoint{1.197594in}{1.740075in}}{\pgfqpoint{1.205830in}{1.740075in}}%
\pgfpathclose%
\pgfusepath{stroke,fill}%
\end{pgfscope}%
\begin{pgfscope}%
\pgfpathrectangle{\pgfqpoint{0.100000in}{0.212622in}}{\pgfqpoint{3.696000in}{3.696000in}}%
\pgfusepath{clip}%
\pgfsetbuttcap%
\pgfsetroundjoin%
\definecolor{currentfill}{rgb}{0.121569,0.466667,0.705882}%
\pgfsetfillcolor{currentfill}%
\pgfsetfillopacity{0.367371}%
\pgfsetlinewidth{1.003750pt}%
\definecolor{currentstroke}{rgb}{0.121569,0.466667,0.705882}%
\pgfsetstrokecolor{currentstroke}%
\pgfsetstrokeopacity{0.367371}%
\pgfsetdash{}{0pt}%
\pgfpathmoveto{\pgfqpoint{1.205830in}{1.740075in}}%
\pgfpathcurveto{\pgfqpoint{1.214066in}{1.740075in}}{\pgfqpoint{1.221966in}{1.743348in}}{\pgfqpoint{1.227790in}{1.749172in}}%
\pgfpathcurveto{\pgfqpoint{1.233614in}{1.754995in}}{\pgfqpoint{1.236887in}{1.762896in}}{\pgfqpoint{1.236887in}{1.771132in}}%
\pgfpathcurveto{\pgfqpoint{1.236887in}{1.779368in}}{\pgfqpoint{1.233614in}{1.787268in}}{\pgfqpoint{1.227790in}{1.793092in}}%
\pgfpathcurveto{\pgfqpoint{1.221966in}{1.798916in}}{\pgfqpoint{1.214066in}{1.802188in}}{\pgfqpoint{1.205830in}{1.802188in}}%
\pgfpathcurveto{\pgfqpoint{1.197594in}{1.802188in}}{\pgfqpoint{1.189694in}{1.798916in}}{\pgfqpoint{1.183870in}{1.793092in}}%
\pgfpathcurveto{\pgfqpoint{1.178046in}{1.787268in}}{\pgfqpoint{1.174774in}{1.779368in}}{\pgfqpoint{1.174774in}{1.771132in}}%
\pgfpathcurveto{\pgfqpoint{1.174774in}{1.762896in}}{\pgfqpoint{1.178046in}{1.754995in}}{\pgfqpoint{1.183870in}{1.749172in}}%
\pgfpathcurveto{\pgfqpoint{1.189694in}{1.743348in}}{\pgfqpoint{1.197594in}{1.740075in}}{\pgfqpoint{1.205830in}{1.740075in}}%
\pgfpathclose%
\pgfusepath{stroke,fill}%
\end{pgfscope}%
\begin{pgfscope}%
\pgfpathrectangle{\pgfqpoint{0.100000in}{0.212622in}}{\pgfqpoint{3.696000in}{3.696000in}}%
\pgfusepath{clip}%
\pgfsetbuttcap%
\pgfsetroundjoin%
\definecolor{currentfill}{rgb}{0.121569,0.466667,0.705882}%
\pgfsetfillcolor{currentfill}%
\pgfsetfillopacity{0.367371}%
\pgfsetlinewidth{1.003750pt}%
\definecolor{currentstroke}{rgb}{0.121569,0.466667,0.705882}%
\pgfsetstrokecolor{currentstroke}%
\pgfsetstrokeopacity{0.367371}%
\pgfsetdash{}{0pt}%
\pgfpathmoveto{\pgfqpoint{1.205830in}{1.740075in}}%
\pgfpathcurveto{\pgfqpoint{1.214066in}{1.740075in}}{\pgfqpoint{1.221966in}{1.743348in}}{\pgfqpoint{1.227790in}{1.749172in}}%
\pgfpathcurveto{\pgfqpoint{1.233614in}{1.754995in}}{\pgfqpoint{1.236887in}{1.762896in}}{\pgfqpoint{1.236887in}{1.771132in}}%
\pgfpathcurveto{\pgfqpoint{1.236887in}{1.779368in}}{\pgfqpoint{1.233614in}{1.787268in}}{\pgfqpoint{1.227790in}{1.793092in}}%
\pgfpathcurveto{\pgfqpoint{1.221966in}{1.798916in}}{\pgfqpoint{1.214066in}{1.802188in}}{\pgfqpoint{1.205830in}{1.802188in}}%
\pgfpathcurveto{\pgfqpoint{1.197594in}{1.802188in}}{\pgfqpoint{1.189694in}{1.798916in}}{\pgfqpoint{1.183870in}{1.793092in}}%
\pgfpathcurveto{\pgfqpoint{1.178046in}{1.787268in}}{\pgfqpoint{1.174774in}{1.779368in}}{\pgfqpoint{1.174774in}{1.771132in}}%
\pgfpathcurveto{\pgfqpoint{1.174774in}{1.762896in}}{\pgfqpoint{1.178046in}{1.754995in}}{\pgfqpoint{1.183870in}{1.749172in}}%
\pgfpathcurveto{\pgfqpoint{1.189694in}{1.743348in}}{\pgfqpoint{1.197594in}{1.740075in}}{\pgfqpoint{1.205830in}{1.740075in}}%
\pgfpathclose%
\pgfusepath{stroke,fill}%
\end{pgfscope}%
\begin{pgfscope}%
\pgfpathrectangle{\pgfqpoint{0.100000in}{0.212622in}}{\pgfqpoint{3.696000in}{3.696000in}}%
\pgfusepath{clip}%
\pgfsetbuttcap%
\pgfsetroundjoin%
\definecolor{currentfill}{rgb}{0.121569,0.466667,0.705882}%
\pgfsetfillcolor{currentfill}%
\pgfsetfillopacity{0.367371}%
\pgfsetlinewidth{1.003750pt}%
\definecolor{currentstroke}{rgb}{0.121569,0.466667,0.705882}%
\pgfsetstrokecolor{currentstroke}%
\pgfsetstrokeopacity{0.367371}%
\pgfsetdash{}{0pt}%
\pgfpathmoveto{\pgfqpoint{1.205830in}{1.740075in}}%
\pgfpathcurveto{\pgfqpoint{1.214066in}{1.740075in}}{\pgfqpoint{1.221966in}{1.743348in}}{\pgfqpoint{1.227790in}{1.749172in}}%
\pgfpathcurveto{\pgfqpoint{1.233614in}{1.754995in}}{\pgfqpoint{1.236887in}{1.762896in}}{\pgfqpoint{1.236887in}{1.771132in}}%
\pgfpathcurveto{\pgfqpoint{1.236887in}{1.779368in}}{\pgfqpoint{1.233614in}{1.787268in}}{\pgfqpoint{1.227790in}{1.793092in}}%
\pgfpathcurveto{\pgfqpoint{1.221966in}{1.798916in}}{\pgfqpoint{1.214066in}{1.802188in}}{\pgfqpoint{1.205830in}{1.802188in}}%
\pgfpathcurveto{\pgfqpoint{1.197594in}{1.802188in}}{\pgfqpoint{1.189694in}{1.798916in}}{\pgfqpoint{1.183870in}{1.793092in}}%
\pgfpathcurveto{\pgfqpoint{1.178046in}{1.787268in}}{\pgfqpoint{1.174774in}{1.779368in}}{\pgfqpoint{1.174774in}{1.771132in}}%
\pgfpathcurveto{\pgfqpoint{1.174774in}{1.762896in}}{\pgfqpoint{1.178046in}{1.754995in}}{\pgfqpoint{1.183870in}{1.749172in}}%
\pgfpathcurveto{\pgfqpoint{1.189694in}{1.743348in}}{\pgfqpoint{1.197594in}{1.740075in}}{\pgfqpoint{1.205830in}{1.740075in}}%
\pgfpathclose%
\pgfusepath{stroke,fill}%
\end{pgfscope}%
\begin{pgfscope}%
\pgfpathrectangle{\pgfqpoint{0.100000in}{0.212622in}}{\pgfqpoint{3.696000in}{3.696000in}}%
\pgfusepath{clip}%
\pgfsetbuttcap%
\pgfsetroundjoin%
\definecolor{currentfill}{rgb}{0.121569,0.466667,0.705882}%
\pgfsetfillcolor{currentfill}%
\pgfsetfillopacity{0.367371}%
\pgfsetlinewidth{1.003750pt}%
\definecolor{currentstroke}{rgb}{0.121569,0.466667,0.705882}%
\pgfsetstrokecolor{currentstroke}%
\pgfsetstrokeopacity{0.367371}%
\pgfsetdash{}{0pt}%
\pgfpathmoveto{\pgfqpoint{1.205830in}{1.740075in}}%
\pgfpathcurveto{\pgfqpoint{1.214066in}{1.740075in}}{\pgfqpoint{1.221966in}{1.743348in}}{\pgfqpoint{1.227790in}{1.749172in}}%
\pgfpathcurveto{\pgfqpoint{1.233614in}{1.754995in}}{\pgfqpoint{1.236887in}{1.762896in}}{\pgfqpoint{1.236887in}{1.771132in}}%
\pgfpathcurveto{\pgfqpoint{1.236887in}{1.779368in}}{\pgfqpoint{1.233614in}{1.787268in}}{\pgfqpoint{1.227790in}{1.793092in}}%
\pgfpathcurveto{\pgfqpoint{1.221966in}{1.798916in}}{\pgfqpoint{1.214066in}{1.802188in}}{\pgfqpoint{1.205830in}{1.802188in}}%
\pgfpathcurveto{\pgfqpoint{1.197594in}{1.802188in}}{\pgfqpoint{1.189694in}{1.798916in}}{\pgfqpoint{1.183870in}{1.793092in}}%
\pgfpathcurveto{\pgfqpoint{1.178046in}{1.787268in}}{\pgfqpoint{1.174774in}{1.779368in}}{\pgfqpoint{1.174774in}{1.771132in}}%
\pgfpathcurveto{\pgfqpoint{1.174774in}{1.762896in}}{\pgfqpoint{1.178046in}{1.754995in}}{\pgfqpoint{1.183870in}{1.749172in}}%
\pgfpathcurveto{\pgfqpoint{1.189694in}{1.743348in}}{\pgfqpoint{1.197594in}{1.740075in}}{\pgfqpoint{1.205830in}{1.740075in}}%
\pgfpathclose%
\pgfusepath{stroke,fill}%
\end{pgfscope}%
\begin{pgfscope}%
\pgfpathrectangle{\pgfqpoint{0.100000in}{0.212622in}}{\pgfqpoint{3.696000in}{3.696000in}}%
\pgfusepath{clip}%
\pgfsetbuttcap%
\pgfsetroundjoin%
\definecolor{currentfill}{rgb}{0.121569,0.466667,0.705882}%
\pgfsetfillcolor{currentfill}%
\pgfsetfillopacity{0.367371}%
\pgfsetlinewidth{1.003750pt}%
\definecolor{currentstroke}{rgb}{0.121569,0.466667,0.705882}%
\pgfsetstrokecolor{currentstroke}%
\pgfsetstrokeopacity{0.367371}%
\pgfsetdash{}{0pt}%
\pgfpathmoveto{\pgfqpoint{1.205830in}{1.740075in}}%
\pgfpathcurveto{\pgfqpoint{1.214066in}{1.740075in}}{\pgfqpoint{1.221966in}{1.743348in}}{\pgfqpoint{1.227790in}{1.749172in}}%
\pgfpathcurveto{\pgfqpoint{1.233614in}{1.754995in}}{\pgfqpoint{1.236887in}{1.762896in}}{\pgfqpoint{1.236887in}{1.771132in}}%
\pgfpathcurveto{\pgfqpoint{1.236887in}{1.779368in}}{\pgfqpoint{1.233614in}{1.787268in}}{\pgfqpoint{1.227790in}{1.793092in}}%
\pgfpathcurveto{\pgfqpoint{1.221966in}{1.798916in}}{\pgfqpoint{1.214066in}{1.802188in}}{\pgfqpoint{1.205830in}{1.802188in}}%
\pgfpathcurveto{\pgfqpoint{1.197594in}{1.802188in}}{\pgfqpoint{1.189694in}{1.798916in}}{\pgfqpoint{1.183870in}{1.793092in}}%
\pgfpathcurveto{\pgfqpoint{1.178046in}{1.787268in}}{\pgfqpoint{1.174774in}{1.779368in}}{\pgfqpoint{1.174774in}{1.771132in}}%
\pgfpathcurveto{\pgfqpoint{1.174774in}{1.762896in}}{\pgfqpoint{1.178046in}{1.754995in}}{\pgfqpoint{1.183870in}{1.749172in}}%
\pgfpathcurveto{\pgfqpoint{1.189694in}{1.743348in}}{\pgfqpoint{1.197594in}{1.740075in}}{\pgfqpoint{1.205830in}{1.740075in}}%
\pgfpathclose%
\pgfusepath{stroke,fill}%
\end{pgfscope}%
\begin{pgfscope}%
\pgfpathrectangle{\pgfqpoint{0.100000in}{0.212622in}}{\pgfqpoint{3.696000in}{3.696000in}}%
\pgfusepath{clip}%
\pgfsetbuttcap%
\pgfsetroundjoin%
\definecolor{currentfill}{rgb}{0.121569,0.466667,0.705882}%
\pgfsetfillcolor{currentfill}%
\pgfsetfillopacity{0.367371}%
\pgfsetlinewidth{1.003750pt}%
\definecolor{currentstroke}{rgb}{0.121569,0.466667,0.705882}%
\pgfsetstrokecolor{currentstroke}%
\pgfsetstrokeopacity{0.367371}%
\pgfsetdash{}{0pt}%
\pgfpathmoveto{\pgfqpoint{1.205830in}{1.740075in}}%
\pgfpathcurveto{\pgfqpoint{1.214066in}{1.740075in}}{\pgfqpoint{1.221966in}{1.743348in}}{\pgfqpoint{1.227790in}{1.749172in}}%
\pgfpathcurveto{\pgfqpoint{1.233614in}{1.754995in}}{\pgfqpoint{1.236887in}{1.762896in}}{\pgfqpoint{1.236887in}{1.771132in}}%
\pgfpathcurveto{\pgfqpoint{1.236887in}{1.779368in}}{\pgfqpoint{1.233614in}{1.787268in}}{\pgfqpoint{1.227790in}{1.793092in}}%
\pgfpathcurveto{\pgfqpoint{1.221966in}{1.798916in}}{\pgfqpoint{1.214066in}{1.802188in}}{\pgfqpoint{1.205830in}{1.802188in}}%
\pgfpathcurveto{\pgfqpoint{1.197594in}{1.802188in}}{\pgfqpoint{1.189694in}{1.798916in}}{\pgfqpoint{1.183870in}{1.793092in}}%
\pgfpathcurveto{\pgfqpoint{1.178046in}{1.787268in}}{\pgfqpoint{1.174774in}{1.779368in}}{\pgfqpoint{1.174774in}{1.771132in}}%
\pgfpathcurveto{\pgfqpoint{1.174774in}{1.762896in}}{\pgfqpoint{1.178046in}{1.754995in}}{\pgfqpoint{1.183870in}{1.749172in}}%
\pgfpathcurveto{\pgfqpoint{1.189694in}{1.743348in}}{\pgfqpoint{1.197594in}{1.740075in}}{\pgfqpoint{1.205830in}{1.740075in}}%
\pgfpathclose%
\pgfusepath{stroke,fill}%
\end{pgfscope}%
\begin{pgfscope}%
\pgfpathrectangle{\pgfqpoint{0.100000in}{0.212622in}}{\pgfqpoint{3.696000in}{3.696000in}}%
\pgfusepath{clip}%
\pgfsetbuttcap%
\pgfsetroundjoin%
\definecolor{currentfill}{rgb}{0.121569,0.466667,0.705882}%
\pgfsetfillcolor{currentfill}%
\pgfsetfillopacity{0.367371}%
\pgfsetlinewidth{1.003750pt}%
\definecolor{currentstroke}{rgb}{0.121569,0.466667,0.705882}%
\pgfsetstrokecolor{currentstroke}%
\pgfsetstrokeopacity{0.367371}%
\pgfsetdash{}{0pt}%
\pgfpathmoveto{\pgfqpoint{1.205830in}{1.740075in}}%
\pgfpathcurveto{\pgfqpoint{1.214066in}{1.740075in}}{\pgfqpoint{1.221966in}{1.743348in}}{\pgfqpoint{1.227790in}{1.749172in}}%
\pgfpathcurveto{\pgfqpoint{1.233614in}{1.754995in}}{\pgfqpoint{1.236887in}{1.762896in}}{\pgfqpoint{1.236887in}{1.771132in}}%
\pgfpathcurveto{\pgfqpoint{1.236887in}{1.779368in}}{\pgfqpoint{1.233614in}{1.787268in}}{\pgfqpoint{1.227790in}{1.793092in}}%
\pgfpathcurveto{\pgfqpoint{1.221966in}{1.798916in}}{\pgfqpoint{1.214066in}{1.802188in}}{\pgfqpoint{1.205830in}{1.802188in}}%
\pgfpathcurveto{\pgfqpoint{1.197594in}{1.802188in}}{\pgfqpoint{1.189694in}{1.798916in}}{\pgfqpoint{1.183870in}{1.793092in}}%
\pgfpathcurveto{\pgfqpoint{1.178046in}{1.787268in}}{\pgfqpoint{1.174774in}{1.779368in}}{\pgfqpoint{1.174774in}{1.771132in}}%
\pgfpathcurveto{\pgfqpoint{1.174774in}{1.762896in}}{\pgfqpoint{1.178046in}{1.754995in}}{\pgfqpoint{1.183870in}{1.749172in}}%
\pgfpathcurveto{\pgfqpoint{1.189694in}{1.743348in}}{\pgfqpoint{1.197594in}{1.740075in}}{\pgfqpoint{1.205830in}{1.740075in}}%
\pgfpathclose%
\pgfusepath{stroke,fill}%
\end{pgfscope}%
\begin{pgfscope}%
\pgfpathrectangle{\pgfqpoint{0.100000in}{0.212622in}}{\pgfqpoint{3.696000in}{3.696000in}}%
\pgfusepath{clip}%
\pgfsetbuttcap%
\pgfsetroundjoin%
\definecolor{currentfill}{rgb}{0.121569,0.466667,0.705882}%
\pgfsetfillcolor{currentfill}%
\pgfsetfillopacity{0.367371}%
\pgfsetlinewidth{1.003750pt}%
\definecolor{currentstroke}{rgb}{0.121569,0.466667,0.705882}%
\pgfsetstrokecolor{currentstroke}%
\pgfsetstrokeopacity{0.367371}%
\pgfsetdash{}{0pt}%
\pgfpathmoveto{\pgfqpoint{1.205830in}{1.740075in}}%
\pgfpathcurveto{\pgfqpoint{1.214066in}{1.740075in}}{\pgfqpoint{1.221966in}{1.743348in}}{\pgfqpoint{1.227790in}{1.749172in}}%
\pgfpathcurveto{\pgfqpoint{1.233614in}{1.754995in}}{\pgfqpoint{1.236887in}{1.762896in}}{\pgfqpoint{1.236887in}{1.771132in}}%
\pgfpathcurveto{\pgfqpoint{1.236887in}{1.779368in}}{\pgfqpoint{1.233614in}{1.787268in}}{\pgfqpoint{1.227790in}{1.793092in}}%
\pgfpathcurveto{\pgfqpoint{1.221966in}{1.798916in}}{\pgfqpoint{1.214066in}{1.802188in}}{\pgfqpoint{1.205830in}{1.802188in}}%
\pgfpathcurveto{\pgfqpoint{1.197594in}{1.802188in}}{\pgfqpoint{1.189694in}{1.798916in}}{\pgfqpoint{1.183870in}{1.793092in}}%
\pgfpathcurveto{\pgfqpoint{1.178046in}{1.787268in}}{\pgfqpoint{1.174774in}{1.779368in}}{\pgfqpoint{1.174774in}{1.771132in}}%
\pgfpathcurveto{\pgfqpoint{1.174774in}{1.762896in}}{\pgfqpoint{1.178046in}{1.754995in}}{\pgfqpoint{1.183870in}{1.749172in}}%
\pgfpathcurveto{\pgfqpoint{1.189694in}{1.743348in}}{\pgfqpoint{1.197594in}{1.740075in}}{\pgfqpoint{1.205830in}{1.740075in}}%
\pgfpathclose%
\pgfusepath{stroke,fill}%
\end{pgfscope}%
\begin{pgfscope}%
\pgfpathrectangle{\pgfqpoint{0.100000in}{0.212622in}}{\pgfqpoint{3.696000in}{3.696000in}}%
\pgfusepath{clip}%
\pgfsetbuttcap%
\pgfsetroundjoin%
\definecolor{currentfill}{rgb}{0.121569,0.466667,0.705882}%
\pgfsetfillcolor{currentfill}%
\pgfsetfillopacity{0.367371}%
\pgfsetlinewidth{1.003750pt}%
\definecolor{currentstroke}{rgb}{0.121569,0.466667,0.705882}%
\pgfsetstrokecolor{currentstroke}%
\pgfsetstrokeopacity{0.367371}%
\pgfsetdash{}{0pt}%
\pgfpathmoveto{\pgfqpoint{1.205830in}{1.740075in}}%
\pgfpathcurveto{\pgfqpoint{1.214066in}{1.740075in}}{\pgfqpoint{1.221966in}{1.743348in}}{\pgfqpoint{1.227790in}{1.749172in}}%
\pgfpathcurveto{\pgfqpoint{1.233614in}{1.754995in}}{\pgfqpoint{1.236887in}{1.762896in}}{\pgfqpoint{1.236887in}{1.771132in}}%
\pgfpathcurveto{\pgfqpoint{1.236887in}{1.779368in}}{\pgfqpoint{1.233614in}{1.787268in}}{\pgfqpoint{1.227790in}{1.793092in}}%
\pgfpathcurveto{\pgfqpoint{1.221966in}{1.798916in}}{\pgfqpoint{1.214066in}{1.802188in}}{\pgfqpoint{1.205830in}{1.802188in}}%
\pgfpathcurveto{\pgfqpoint{1.197594in}{1.802188in}}{\pgfqpoint{1.189694in}{1.798916in}}{\pgfqpoint{1.183870in}{1.793092in}}%
\pgfpathcurveto{\pgfqpoint{1.178046in}{1.787268in}}{\pgfqpoint{1.174774in}{1.779368in}}{\pgfqpoint{1.174774in}{1.771132in}}%
\pgfpathcurveto{\pgfqpoint{1.174774in}{1.762896in}}{\pgfqpoint{1.178046in}{1.754995in}}{\pgfqpoint{1.183870in}{1.749172in}}%
\pgfpathcurveto{\pgfqpoint{1.189694in}{1.743348in}}{\pgfqpoint{1.197594in}{1.740075in}}{\pgfqpoint{1.205830in}{1.740075in}}%
\pgfpathclose%
\pgfusepath{stroke,fill}%
\end{pgfscope}%
\begin{pgfscope}%
\pgfpathrectangle{\pgfqpoint{0.100000in}{0.212622in}}{\pgfqpoint{3.696000in}{3.696000in}}%
\pgfusepath{clip}%
\pgfsetbuttcap%
\pgfsetroundjoin%
\definecolor{currentfill}{rgb}{0.121569,0.466667,0.705882}%
\pgfsetfillcolor{currentfill}%
\pgfsetfillopacity{0.367371}%
\pgfsetlinewidth{1.003750pt}%
\definecolor{currentstroke}{rgb}{0.121569,0.466667,0.705882}%
\pgfsetstrokecolor{currentstroke}%
\pgfsetstrokeopacity{0.367371}%
\pgfsetdash{}{0pt}%
\pgfpathmoveto{\pgfqpoint{1.205830in}{1.740075in}}%
\pgfpathcurveto{\pgfqpoint{1.214066in}{1.740075in}}{\pgfqpoint{1.221966in}{1.743348in}}{\pgfqpoint{1.227790in}{1.749172in}}%
\pgfpathcurveto{\pgfqpoint{1.233614in}{1.754995in}}{\pgfqpoint{1.236887in}{1.762896in}}{\pgfqpoint{1.236887in}{1.771132in}}%
\pgfpathcurveto{\pgfqpoint{1.236887in}{1.779368in}}{\pgfqpoint{1.233614in}{1.787268in}}{\pgfqpoint{1.227790in}{1.793092in}}%
\pgfpathcurveto{\pgfqpoint{1.221966in}{1.798916in}}{\pgfqpoint{1.214066in}{1.802188in}}{\pgfqpoint{1.205830in}{1.802188in}}%
\pgfpathcurveto{\pgfqpoint{1.197594in}{1.802188in}}{\pgfqpoint{1.189694in}{1.798916in}}{\pgfqpoint{1.183870in}{1.793092in}}%
\pgfpathcurveto{\pgfqpoint{1.178046in}{1.787268in}}{\pgfqpoint{1.174774in}{1.779368in}}{\pgfqpoint{1.174774in}{1.771132in}}%
\pgfpathcurveto{\pgfqpoint{1.174774in}{1.762896in}}{\pgfqpoint{1.178046in}{1.754995in}}{\pgfqpoint{1.183870in}{1.749172in}}%
\pgfpathcurveto{\pgfqpoint{1.189694in}{1.743348in}}{\pgfqpoint{1.197594in}{1.740075in}}{\pgfqpoint{1.205830in}{1.740075in}}%
\pgfpathclose%
\pgfusepath{stroke,fill}%
\end{pgfscope}%
\begin{pgfscope}%
\pgfpathrectangle{\pgfqpoint{0.100000in}{0.212622in}}{\pgfqpoint{3.696000in}{3.696000in}}%
\pgfusepath{clip}%
\pgfsetbuttcap%
\pgfsetroundjoin%
\definecolor{currentfill}{rgb}{0.121569,0.466667,0.705882}%
\pgfsetfillcolor{currentfill}%
\pgfsetfillopacity{0.367663}%
\pgfsetlinewidth{1.003750pt}%
\definecolor{currentstroke}{rgb}{0.121569,0.466667,0.705882}%
\pgfsetstrokecolor{currentstroke}%
\pgfsetstrokeopacity{0.367663}%
\pgfsetdash{}{0pt}%
\pgfpathmoveto{\pgfqpoint{1.205363in}{1.739522in}}%
\pgfpathcurveto{\pgfqpoint{1.213600in}{1.739522in}}{\pgfqpoint{1.221500in}{1.742795in}}{\pgfqpoint{1.227324in}{1.748619in}}%
\pgfpathcurveto{\pgfqpoint{1.233147in}{1.754442in}}{\pgfqpoint{1.236420in}{1.762343in}}{\pgfqpoint{1.236420in}{1.770579in}}%
\pgfpathcurveto{\pgfqpoint{1.236420in}{1.778815in}}{\pgfqpoint{1.233147in}{1.786715in}}{\pgfqpoint{1.227324in}{1.792539in}}%
\pgfpathcurveto{\pgfqpoint{1.221500in}{1.798363in}}{\pgfqpoint{1.213600in}{1.801635in}}{\pgfqpoint{1.205363in}{1.801635in}}%
\pgfpathcurveto{\pgfqpoint{1.197127in}{1.801635in}}{\pgfqpoint{1.189227in}{1.798363in}}{\pgfqpoint{1.183403in}{1.792539in}}%
\pgfpathcurveto{\pgfqpoint{1.177579in}{1.786715in}}{\pgfqpoint{1.174307in}{1.778815in}}{\pgfqpoint{1.174307in}{1.770579in}}%
\pgfpathcurveto{\pgfqpoint{1.174307in}{1.762343in}}{\pgfqpoint{1.177579in}{1.754442in}}{\pgfqpoint{1.183403in}{1.748619in}}%
\pgfpathcurveto{\pgfqpoint{1.189227in}{1.742795in}}{\pgfqpoint{1.197127in}{1.739522in}}{\pgfqpoint{1.205363in}{1.739522in}}%
\pgfpathclose%
\pgfusepath{stroke,fill}%
\end{pgfscope}%
\begin{pgfscope}%
\pgfpathrectangle{\pgfqpoint{0.100000in}{0.212622in}}{\pgfqpoint{3.696000in}{3.696000in}}%
\pgfusepath{clip}%
\pgfsetbuttcap%
\pgfsetroundjoin%
\definecolor{currentfill}{rgb}{0.121569,0.466667,0.705882}%
\pgfsetfillcolor{currentfill}%
\pgfsetfillopacity{0.367938}%
\pgfsetlinewidth{1.003750pt}%
\definecolor{currentstroke}{rgb}{0.121569,0.466667,0.705882}%
\pgfsetstrokecolor{currentstroke}%
\pgfsetstrokeopacity{0.367938}%
\pgfsetdash{}{0pt}%
\pgfpathmoveto{\pgfqpoint{1.204752in}{1.738921in}}%
\pgfpathcurveto{\pgfqpoint{1.212988in}{1.738921in}}{\pgfqpoint{1.220888in}{1.742194in}}{\pgfqpoint{1.226712in}{1.748017in}}%
\pgfpathcurveto{\pgfqpoint{1.232536in}{1.753841in}}{\pgfqpoint{1.235808in}{1.761741in}}{\pgfqpoint{1.235808in}{1.769978in}}%
\pgfpathcurveto{\pgfqpoint{1.235808in}{1.778214in}}{\pgfqpoint{1.232536in}{1.786114in}}{\pgfqpoint{1.226712in}{1.791938in}}%
\pgfpathcurveto{\pgfqpoint{1.220888in}{1.797762in}}{\pgfqpoint{1.212988in}{1.801034in}}{\pgfqpoint{1.204752in}{1.801034in}}%
\pgfpathcurveto{\pgfqpoint{1.196515in}{1.801034in}}{\pgfqpoint{1.188615in}{1.797762in}}{\pgfqpoint{1.182791in}{1.791938in}}%
\pgfpathcurveto{\pgfqpoint{1.176968in}{1.786114in}}{\pgfqpoint{1.173695in}{1.778214in}}{\pgfqpoint{1.173695in}{1.769978in}}%
\pgfpathcurveto{\pgfqpoint{1.173695in}{1.761741in}}{\pgfqpoint{1.176968in}{1.753841in}}{\pgfqpoint{1.182791in}{1.748017in}}%
\pgfpathcurveto{\pgfqpoint{1.188615in}{1.742194in}}{\pgfqpoint{1.196515in}{1.738921in}}{\pgfqpoint{1.204752in}{1.738921in}}%
\pgfpathclose%
\pgfusepath{stroke,fill}%
\end{pgfscope}%
\begin{pgfscope}%
\pgfpathrectangle{\pgfqpoint{0.100000in}{0.212622in}}{\pgfqpoint{3.696000in}{3.696000in}}%
\pgfusepath{clip}%
\pgfsetbuttcap%
\pgfsetroundjoin%
\definecolor{currentfill}{rgb}{0.121569,0.466667,0.705882}%
\pgfsetfillcolor{currentfill}%
\pgfsetfillopacity{0.368002}%
\pgfsetlinewidth{1.003750pt}%
\definecolor{currentstroke}{rgb}{0.121569,0.466667,0.705882}%
\pgfsetstrokecolor{currentstroke}%
\pgfsetstrokeopacity{0.368002}%
\pgfsetdash{}{0pt}%
\pgfpathmoveto{\pgfqpoint{1.204684in}{1.738817in}}%
\pgfpathcurveto{\pgfqpoint{1.212921in}{1.738817in}}{\pgfqpoint{1.220821in}{1.742089in}}{\pgfqpoint{1.226645in}{1.747913in}}%
\pgfpathcurveto{\pgfqpoint{1.232468in}{1.753737in}}{\pgfqpoint{1.235741in}{1.761637in}}{\pgfqpoint{1.235741in}{1.769874in}}%
\pgfpathcurveto{\pgfqpoint{1.235741in}{1.778110in}}{\pgfqpoint{1.232468in}{1.786010in}}{\pgfqpoint{1.226645in}{1.791834in}}%
\pgfpathcurveto{\pgfqpoint{1.220821in}{1.797658in}}{\pgfqpoint{1.212921in}{1.800930in}}{\pgfqpoint{1.204684in}{1.800930in}}%
\pgfpathcurveto{\pgfqpoint{1.196448in}{1.800930in}}{\pgfqpoint{1.188548in}{1.797658in}}{\pgfqpoint{1.182724in}{1.791834in}}%
\pgfpathcurveto{\pgfqpoint{1.176900in}{1.786010in}}{\pgfqpoint{1.173628in}{1.778110in}}{\pgfqpoint{1.173628in}{1.769874in}}%
\pgfpathcurveto{\pgfqpoint{1.173628in}{1.761637in}}{\pgfqpoint{1.176900in}{1.753737in}}{\pgfqpoint{1.182724in}{1.747913in}}%
\pgfpathcurveto{\pgfqpoint{1.188548in}{1.742089in}}{\pgfqpoint{1.196448in}{1.738817in}}{\pgfqpoint{1.204684in}{1.738817in}}%
\pgfpathclose%
\pgfusepath{stroke,fill}%
\end{pgfscope}%
\begin{pgfscope}%
\pgfpathrectangle{\pgfqpoint{0.100000in}{0.212622in}}{\pgfqpoint{3.696000in}{3.696000in}}%
\pgfusepath{clip}%
\pgfsetbuttcap%
\pgfsetroundjoin%
\definecolor{currentfill}{rgb}{0.121569,0.466667,0.705882}%
\pgfsetfillcolor{currentfill}%
\pgfsetfillopacity{0.368051}%
\pgfsetlinewidth{1.003750pt}%
\definecolor{currentstroke}{rgb}{0.121569,0.466667,0.705882}%
\pgfsetstrokecolor{currentstroke}%
\pgfsetstrokeopacity{0.368051}%
\pgfsetdash{}{0pt}%
\pgfpathmoveto{\pgfqpoint{1.204606in}{1.738724in}}%
\pgfpathcurveto{\pgfqpoint{1.212842in}{1.738724in}}{\pgfqpoint{1.220742in}{1.741997in}}{\pgfqpoint{1.226566in}{1.747821in}}%
\pgfpathcurveto{\pgfqpoint{1.232390in}{1.753645in}}{\pgfqpoint{1.235662in}{1.761545in}}{\pgfqpoint{1.235662in}{1.769781in}}%
\pgfpathcurveto{\pgfqpoint{1.235662in}{1.778017in}}{\pgfqpoint{1.232390in}{1.785917in}}{\pgfqpoint{1.226566in}{1.791741in}}%
\pgfpathcurveto{\pgfqpoint{1.220742in}{1.797565in}}{\pgfqpoint{1.212842in}{1.800837in}}{\pgfqpoint{1.204606in}{1.800837in}}%
\pgfpathcurveto{\pgfqpoint{1.196370in}{1.800837in}}{\pgfqpoint{1.188470in}{1.797565in}}{\pgfqpoint{1.182646in}{1.791741in}}%
\pgfpathcurveto{\pgfqpoint{1.176822in}{1.785917in}}{\pgfqpoint{1.173549in}{1.778017in}}{\pgfqpoint{1.173549in}{1.769781in}}%
\pgfpathcurveto{\pgfqpoint{1.173549in}{1.761545in}}{\pgfqpoint{1.176822in}{1.753645in}}{\pgfqpoint{1.182646in}{1.747821in}}%
\pgfpathcurveto{\pgfqpoint{1.188470in}{1.741997in}}{\pgfqpoint{1.196370in}{1.738724in}}{\pgfqpoint{1.204606in}{1.738724in}}%
\pgfpathclose%
\pgfusepath{stroke,fill}%
\end{pgfscope}%
\begin{pgfscope}%
\pgfpathrectangle{\pgfqpoint{0.100000in}{0.212622in}}{\pgfqpoint{3.696000in}{3.696000in}}%
\pgfusepath{clip}%
\pgfsetbuttcap%
\pgfsetroundjoin%
\definecolor{currentfill}{rgb}{0.121569,0.466667,0.705882}%
\pgfsetfillcolor{currentfill}%
\pgfsetfillopacity{0.368069}%
\pgfsetlinewidth{1.003750pt}%
\definecolor{currentstroke}{rgb}{0.121569,0.466667,0.705882}%
\pgfsetstrokecolor{currentstroke}%
\pgfsetstrokeopacity{0.368069}%
\pgfsetdash{}{0pt}%
\pgfpathmoveto{\pgfqpoint{1.204593in}{1.738700in}}%
\pgfpathcurveto{\pgfqpoint{1.212829in}{1.738700in}}{\pgfqpoint{1.220729in}{1.741972in}}{\pgfqpoint{1.226553in}{1.747796in}}%
\pgfpathcurveto{\pgfqpoint{1.232377in}{1.753620in}}{\pgfqpoint{1.235650in}{1.761520in}}{\pgfqpoint{1.235650in}{1.769756in}}%
\pgfpathcurveto{\pgfqpoint{1.235650in}{1.777993in}}{\pgfqpoint{1.232377in}{1.785893in}}{\pgfqpoint{1.226553in}{1.791717in}}%
\pgfpathcurveto{\pgfqpoint{1.220729in}{1.797541in}}{\pgfqpoint{1.212829in}{1.800813in}}{\pgfqpoint{1.204593in}{1.800813in}}%
\pgfpathcurveto{\pgfqpoint{1.196357in}{1.800813in}}{\pgfqpoint{1.188457in}{1.797541in}}{\pgfqpoint{1.182633in}{1.791717in}}%
\pgfpathcurveto{\pgfqpoint{1.176809in}{1.785893in}}{\pgfqpoint{1.173537in}{1.777993in}}{\pgfqpoint{1.173537in}{1.769756in}}%
\pgfpathcurveto{\pgfqpoint{1.173537in}{1.761520in}}{\pgfqpoint{1.176809in}{1.753620in}}{\pgfqpoint{1.182633in}{1.747796in}}%
\pgfpathcurveto{\pgfqpoint{1.188457in}{1.741972in}}{\pgfqpoint{1.196357in}{1.738700in}}{\pgfqpoint{1.204593in}{1.738700in}}%
\pgfpathclose%
\pgfusepath{stroke,fill}%
\end{pgfscope}%
\begin{pgfscope}%
\pgfpathrectangle{\pgfqpoint{0.100000in}{0.212622in}}{\pgfqpoint{3.696000in}{3.696000in}}%
\pgfusepath{clip}%
\pgfsetbuttcap%
\pgfsetroundjoin%
\definecolor{currentfill}{rgb}{0.121569,0.466667,0.705882}%
\pgfsetfillcolor{currentfill}%
\pgfsetfillopacity{0.368085}%
\pgfsetlinewidth{1.003750pt}%
\definecolor{currentstroke}{rgb}{0.121569,0.466667,0.705882}%
\pgfsetstrokecolor{currentstroke}%
\pgfsetstrokeopacity{0.368085}%
\pgfsetdash{}{0pt}%
\pgfpathmoveto{\pgfqpoint{1.204564in}{1.738667in}}%
\pgfpathcurveto{\pgfqpoint{1.212800in}{1.738667in}}{\pgfqpoint{1.220700in}{1.741940in}}{\pgfqpoint{1.226524in}{1.747764in}}%
\pgfpathcurveto{\pgfqpoint{1.232348in}{1.753588in}}{\pgfqpoint{1.235620in}{1.761488in}}{\pgfqpoint{1.235620in}{1.769724in}}%
\pgfpathcurveto{\pgfqpoint{1.235620in}{1.777960in}}{\pgfqpoint{1.232348in}{1.785860in}}{\pgfqpoint{1.226524in}{1.791684in}}%
\pgfpathcurveto{\pgfqpoint{1.220700in}{1.797508in}}{\pgfqpoint{1.212800in}{1.800780in}}{\pgfqpoint{1.204564in}{1.800780in}}%
\pgfpathcurveto{\pgfqpoint{1.196328in}{1.800780in}}{\pgfqpoint{1.188428in}{1.797508in}}{\pgfqpoint{1.182604in}{1.791684in}}%
\pgfpathcurveto{\pgfqpoint{1.176780in}{1.785860in}}{\pgfqpoint{1.173507in}{1.777960in}}{\pgfqpoint{1.173507in}{1.769724in}}%
\pgfpathcurveto{\pgfqpoint{1.173507in}{1.761488in}}{\pgfqpoint{1.176780in}{1.753588in}}{\pgfqpoint{1.182604in}{1.747764in}}%
\pgfpathcurveto{\pgfqpoint{1.188428in}{1.741940in}}{\pgfqpoint{1.196328in}{1.738667in}}{\pgfqpoint{1.204564in}{1.738667in}}%
\pgfpathclose%
\pgfusepath{stroke,fill}%
\end{pgfscope}%
\begin{pgfscope}%
\pgfpathrectangle{\pgfqpoint{0.100000in}{0.212622in}}{\pgfqpoint{3.696000in}{3.696000in}}%
\pgfusepath{clip}%
\pgfsetbuttcap%
\pgfsetroundjoin%
\definecolor{currentfill}{rgb}{0.121569,0.466667,0.705882}%
\pgfsetfillcolor{currentfill}%
\pgfsetfillopacity{0.368096}%
\pgfsetlinewidth{1.003750pt}%
\definecolor{currentstroke}{rgb}{0.121569,0.466667,0.705882}%
\pgfsetstrokecolor{currentstroke}%
\pgfsetstrokeopacity{0.368096}%
\pgfsetdash{}{0pt}%
\pgfpathmoveto{\pgfqpoint{1.204542in}{1.738644in}}%
\pgfpathcurveto{\pgfqpoint{1.212778in}{1.738644in}}{\pgfqpoint{1.220678in}{1.741917in}}{\pgfqpoint{1.226502in}{1.747741in}}%
\pgfpathcurveto{\pgfqpoint{1.232326in}{1.753564in}}{\pgfqpoint{1.235598in}{1.761465in}}{\pgfqpoint{1.235598in}{1.769701in}}%
\pgfpathcurveto{\pgfqpoint{1.235598in}{1.777937in}}{\pgfqpoint{1.232326in}{1.785837in}}{\pgfqpoint{1.226502in}{1.791661in}}%
\pgfpathcurveto{\pgfqpoint{1.220678in}{1.797485in}}{\pgfqpoint{1.212778in}{1.800757in}}{\pgfqpoint{1.204542in}{1.800757in}}%
\pgfpathcurveto{\pgfqpoint{1.196305in}{1.800757in}}{\pgfqpoint{1.188405in}{1.797485in}}{\pgfqpoint{1.182581in}{1.791661in}}%
\pgfpathcurveto{\pgfqpoint{1.176757in}{1.785837in}}{\pgfqpoint{1.173485in}{1.777937in}}{\pgfqpoint{1.173485in}{1.769701in}}%
\pgfpathcurveto{\pgfqpoint{1.173485in}{1.761465in}}{\pgfqpoint{1.176757in}{1.753564in}}{\pgfqpoint{1.182581in}{1.747741in}}%
\pgfpathcurveto{\pgfqpoint{1.188405in}{1.741917in}}{\pgfqpoint{1.196305in}{1.738644in}}{\pgfqpoint{1.204542in}{1.738644in}}%
\pgfpathclose%
\pgfusepath{stroke,fill}%
\end{pgfscope}%
\begin{pgfscope}%
\pgfpathrectangle{\pgfqpoint{0.100000in}{0.212622in}}{\pgfqpoint{3.696000in}{3.696000in}}%
\pgfusepath{clip}%
\pgfsetbuttcap%
\pgfsetroundjoin%
\definecolor{currentfill}{rgb}{0.121569,0.466667,0.705882}%
\pgfsetfillcolor{currentfill}%
\pgfsetfillopacity{0.368099}%
\pgfsetlinewidth{1.003750pt}%
\definecolor{currentstroke}{rgb}{0.121569,0.466667,0.705882}%
\pgfsetstrokecolor{currentstroke}%
\pgfsetstrokeopacity{0.368099}%
\pgfsetdash{}{0pt}%
\pgfpathmoveto{\pgfqpoint{1.204540in}{1.738641in}}%
\pgfpathcurveto{\pgfqpoint{1.212777in}{1.738641in}}{\pgfqpoint{1.220677in}{1.741913in}}{\pgfqpoint{1.226501in}{1.747737in}}%
\pgfpathcurveto{\pgfqpoint{1.232325in}{1.753561in}}{\pgfqpoint{1.235597in}{1.761461in}}{\pgfqpoint{1.235597in}{1.769698in}}%
\pgfpathcurveto{\pgfqpoint{1.235597in}{1.777934in}}{\pgfqpoint{1.232325in}{1.785834in}}{\pgfqpoint{1.226501in}{1.791658in}}%
\pgfpathcurveto{\pgfqpoint{1.220677in}{1.797482in}}{\pgfqpoint{1.212777in}{1.800754in}}{\pgfqpoint{1.204540in}{1.800754in}}%
\pgfpathcurveto{\pgfqpoint{1.196304in}{1.800754in}}{\pgfqpoint{1.188404in}{1.797482in}}{\pgfqpoint{1.182580in}{1.791658in}}%
\pgfpathcurveto{\pgfqpoint{1.176756in}{1.785834in}}{\pgfqpoint{1.173484in}{1.777934in}}{\pgfqpoint{1.173484in}{1.769698in}}%
\pgfpathcurveto{\pgfqpoint{1.173484in}{1.761461in}}{\pgfqpoint{1.176756in}{1.753561in}}{\pgfqpoint{1.182580in}{1.747737in}}%
\pgfpathcurveto{\pgfqpoint{1.188404in}{1.741913in}}{\pgfqpoint{1.196304in}{1.738641in}}{\pgfqpoint{1.204540in}{1.738641in}}%
\pgfpathclose%
\pgfusepath{stroke,fill}%
\end{pgfscope}%
\begin{pgfscope}%
\pgfpathrectangle{\pgfqpoint{0.100000in}{0.212622in}}{\pgfqpoint{3.696000in}{3.696000in}}%
\pgfusepath{clip}%
\pgfsetbuttcap%
\pgfsetroundjoin%
\definecolor{currentfill}{rgb}{0.121569,0.466667,0.705882}%
\pgfsetfillcolor{currentfill}%
\pgfsetfillopacity{0.368101}%
\pgfsetlinewidth{1.003750pt}%
\definecolor{currentstroke}{rgb}{0.121569,0.466667,0.705882}%
\pgfsetstrokecolor{currentstroke}%
\pgfsetstrokeopacity{0.368101}%
\pgfsetdash{}{0pt}%
\pgfpathmoveto{\pgfqpoint{1.204536in}{1.738636in}}%
\pgfpathcurveto{\pgfqpoint{1.212773in}{1.738636in}}{\pgfqpoint{1.220673in}{1.741909in}}{\pgfqpoint{1.226497in}{1.747733in}}%
\pgfpathcurveto{\pgfqpoint{1.232321in}{1.753557in}}{\pgfqpoint{1.235593in}{1.761457in}}{\pgfqpoint{1.235593in}{1.769693in}}%
\pgfpathcurveto{\pgfqpoint{1.235593in}{1.777929in}}{\pgfqpoint{1.232321in}{1.785829in}}{\pgfqpoint{1.226497in}{1.791653in}}%
\pgfpathcurveto{\pgfqpoint{1.220673in}{1.797477in}}{\pgfqpoint{1.212773in}{1.800749in}}{\pgfqpoint{1.204536in}{1.800749in}}%
\pgfpathcurveto{\pgfqpoint{1.196300in}{1.800749in}}{\pgfqpoint{1.188400in}{1.797477in}}{\pgfqpoint{1.182576in}{1.791653in}}%
\pgfpathcurveto{\pgfqpoint{1.176752in}{1.785829in}}{\pgfqpoint{1.173480in}{1.777929in}}{\pgfqpoint{1.173480in}{1.769693in}}%
\pgfpathcurveto{\pgfqpoint{1.173480in}{1.761457in}}{\pgfqpoint{1.176752in}{1.753557in}}{\pgfqpoint{1.182576in}{1.747733in}}%
\pgfpathcurveto{\pgfqpoint{1.188400in}{1.741909in}}{\pgfqpoint{1.196300in}{1.738636in}}{\pgfqpoint{1.204536in}{1.738636in}}%
\pgfpathclose%
\pgfusepath{stroke,fill}%
\end{pgfscope}%
\begin{pgfscope}%
\pgfpathrectangle{\pgfqpoint{0.100000in}{0.212622in}}{\pgfqpoint{3.696000in}{3.696000in}}%
\pgfusepath{clip}%
\pgfsetbuttcap%
\pgfsetroundjoin%
\definecolor{currentfill}{rgb}{0.121569,0.466667,0.705882}%
\pgfsetfillcolor{currentfill}%
\pgfsetfillopacity{0.368103}%
\pgfsetlinewidth{1.003750pt}%
\definecolor{currentstroke}{rgb}{0.121569,0.466667,0.705882}%
\pgfsetstrokecolor{currentstroke}%
\pgfsetstrokeopacity{0.368103}%
\pgfsetdash{}{0pt}%
\pgfpathmoveto{\pgfqpoint{1.204533in}{1.738633in}}%
\pgfpathcurveto{\pgfqpoint{1.212770in}{1.738633in}}{\pgfqpoint{1.220670in}{1.741905in}}{\pgfqpoint{1.226494in}{1.747729in}}%
\pgfpathcurveto{\pgfqpoint{1.232317in}{1.753553in}}{\pgfqpoint{1.235590in}{1.761453in}}{\pgfqpoint{1.235590in}{1.769689in}}%
\pgfpathcurveto{\pgfqpoint{1.235590in}{1.777926in}}{\pgfqpoint{1.232317in}{1.785826in}}{\pgfqpoint{1.226494in}{1.791650in}}%
\pgfpathcurveto{\pgfqpoint{1.220670in}{1.797474in}}{\pgfqpoint{1.212770in}{1.800746in}}{\pgfqpoint{1.204533in}{1.800746in}}%
\pgfpathcurveto{\pgfqpoint{1.196297in}{1.800746in}}{\pgfqpoint{1.188397in}{1.797474in}}{\pgfqpoint{1.182573in}{1.791650in}}%
\pgfpathcurveto{\pgfqpoint{1.176749in}{1.785826in}}{\pgfqpoint{1.173477in}{1.777926in}}{\pgfqpoint{1.173477in}{1.769689in}}%
\pgfpathcurveto{\pgfqpoint{1.173477in}{1.761453in}}{\pgfqpoint{1.176749in}{1.753553in}}{\pgfqpoint{1.182573in}{1.747729in}}%
\pgfpathcurveto{\pgfqpoint{1.188397in}{1.741905in}}{\pgfqpoint{1.196297in}{1.738633in}}{\pgfqpoint{1.204533in}{1.738633in}}%
\pgfpathclose%
\pgfusepath{stroke,fill}%
\end{pgfscope}%
\begin{pgfscope}%
\pgfpathrectangle{\pgfqpoint{0.100000in}{0.212622in}}{\pgfqpoint{3.696000in}{3.696000in}}%
\pgfusepath{clip}%
\pgfsetbuttcap%
\pgfsetroundjoin%
\definecolor{currentfill}{rgb}{0.121569,0.466667,0.705882}%
\pgfsetfillcolor{currentfill}%
\pgfsetfillopacity{0.368103}%
\pgfsetlinewidth{1.003750pt}%
\definecolor{currentstroke}{rgb}{0.121569,0.466667,0.705882}%
\pgfsetstrokecolor{currentstroke}%
\pgfsetstrokeopacity{0.368103}%
\pgfsetdash{}{0pt}%
\pgfpathmoveto{\pgfqpoint{1.204532in}{1.738632in}}%
\pgfpathcurveto{\pgfqpoint{1.212769in}{1.738632in}}{\pgfqpoint{1.220669in}{1.741904in}}{\pgfqpoint{1.226493in}{1.747728in}}%
\pgfpathcurveto{\pgfqpoint{1.232317in}{1.753552in}}{\pgfqpoint{1.235589in}{1.761452in}}{\pgfqpoint{1.235589in}{1.769688in}}%
\pgfpathcurveto{\pgfqpoint{1.235589in}{1.777925in}}{\pgfqpoint{1.232317in}{1.785825in}}{\pgfqpoint{1.226493in}{1.791649in}}%
\pgfpathcurveto{\pgfqpoint{1.220669in}{1.797472in}}{\pgfqpoint{1.212769in}{1.800745in}}{\pgfqpoint{1.204532in}{1.800745in}}%
\pgfpathcurveto{\pgfqpoint{1.196296in}{1.800745in}}{\pgfqpoint{1.188396in}{1.797472in}}{\pgfqpoint{1.182572in}{1.791649in}}%
\pgfpathcurveto{\pgfqpoint{1.176748in}{1.785825in}}{\pgfqpoint{1.173476in}{1.777925in}}{\pgfqpoint{1.173476in}{1.769688in}}%
\pgfpathcurveto{\pgfqpoint{1.173476in}{1.761452in}}{\pgfqpoint{1.176748in}{1.753552in}}{\pgfqpoint{1.182572in}{1.747728in}}%
\pgfpathcurveto{\pgfqpoint{1.188396in}{1.741904in}}{\pgfqpoint{1.196296in}{1.738632in}}{\pgfqpoint{1.204532in}{1.738632in}}%
\pgfpathclose%
\pgfusepath{stroke,fill}%
\end{pgfscope}%
\begin{pgfscope}%
\pgfpathrectangle{\pgfqpoint{0.100000in}{0.212622in}}{\pgfqpoint{3.696000in}{3.696000in}}%
\pgfusepath{clip}%
\pgfsetbuttcap%
\pgfsetroundjoin%
\definecolor{currentfill}{rgb}{0.121569,0.466667,0.705882}%
\pgfsetfillcolor{currentfill}%
\pgfsetfillopacity{0.368104}%
\pgfsetlinewidth{1.003750pt}%
\definecolor{currentstroke}{rgb}{0.121569,0.466667,0.705882}%
\pgfsetstrokecolor{currentstroke}%
\pgfsetstrokeopacity{0.368104}%
\pgfsetdash{}{0pt}%
\pgfpathmoveto{\pgfqpoint{1.204530in}{1.738630in}}%
\pgfpathcurveto{\pgfqpoint{1.212767in}{1.738630in}}{\pgfqpoint{1.220667in}{1.741902in}}{\pgfqpoint{1.226491in}{1.747726in}}%
\pgfpathcurveto{\pgfqpoint{1.232315in}{1.753550in}}{\pgfqpoint{1.235587in}{1.761450in}}{\pgfqpoint{1.235587in}{1.769686in}}%
\pgfpathcurveto{\pgfqpoint{1.235587in}{1.777923in}}{\pgfqpoint{1.232315in}{1.785823in}}{\pgfqpoint{1.226491in}{1.791647in}}%
\pgfpathcurveto{\pgfqpoint{1.220667in}{1.797471in}}{\pgfqpoint{1.212767in}{1.800743in}}{\pgfqpoint{1.204530in}{1.800743in}}%
\pgfpathcurveto{\pgfqpoint{1.196294in}{1.800743in}}{\pgfqpoint{1.188394in}{1.797471in}}{\pgfqpoint{1.182570in}{1.791647in}}%
\pgfpathcurveto{\pgfqpoint{1.176746in}{1.785823in}}{\pgfqpoint{1.173474in}{1.777923in}}{\pgfqpoint{1.173474in}{1.769686in}}%
\pgfpathcurveto{\pgfqpoint{1.173474in}{1.761450in}}{\pgfqpoint{1.176746in}{1.753550in}}{\pgfqpoint{1.182570in}{1.747726in}}%
\pgfpathcurveto{\pgfqpoint{1.188394in}{1.741902in}}{\pgfqpoint{1.196294in}{1.738630in}}{\pgfqpoint{1.204530in}{1.738630in}}%
\pgfpathclose%
\pgfusepath{stroke,fill}%
\end{pgfscope}%
\begin{pgfscope}%
\pgfpathrectangle{\pgfqpoint{0.100000in}{0.212622in}}{\pgfqpoint{3.696000in}{3.696000in}}%
\pgfusepath{clip}%
\pgfsetbuttcap%
\pgfsetroundjoin%
\definecolor{currentfill}{rgb}{0.121569,0.466667,0.705882}%
\pgfsetfillcolor{currentfill}%
\pgfsetfillopacity{0.368105}%
\pgfsetlinewidth{1.003750pt}%
\definecolor{currentstroke}{rgb}{0.121569,0.466667,0.705882}%
\pgfsetstrokecolor{currentstroke}%
\pgfsetstrokeopacity{0.368105}%
\pgfsetdash{}{0pt}%
\pgfpathmoveto{\pgfqpoint{1.204529in}{1.738629in}}%
\pgfpathcurveto{\pgfqpoint{1.212766in}{1.738629in}}{\pgfqpoint{1.220666in}{1.741901in}}{\pgfqpoint{1.226490in}{1.747725in}}%
\pgfpathcurveto{\pgfqpoint{1.232314in}{1.753549in}}{\pgfqpoint{1.235586in}{1.761449in}}{\pgfqpoint{1.235586in}{1.769685in}}%
\pgfpathcurveto{\pgfqpoint{1.235586in}{1.777922in}}{\pgfqpoint{1.232314in}{1.785822in}}{\pgfqpoint{1.226490in}{1.791646in}}%
\pgfpathcurveto{\pgfqpoint{1.220666in}{1.797470in}}{\pgfqpoint{1.212766in}{1.800742in}}{\pgfqpoint{1.204529in}{1.800742in}}%
\pgfpathcurveto{\pgfqpoint{1.196293in}{1.800742in}}{\pgfqpoint{1.188393in}{1.797470in}}{\pgfqpoint{1.182569in}{1.791646in}}%
\pgfpathcurveto{\pgfqpoint{1.176745in}{1.785822in}}{\pgfqpoint{1.173473in}{1.777922in}}{\pgfqpoint{1.173473in}{1.769685in}}%
\pgfpathcurveto{\pgfqpoint{1.173473in}{1.761449in}}{\pgfqpoint{1.176745in}{1.753549in}}{\pgfqpoint{1.182569in}{1.747725in}}%
\pgfpathcurveto{\pgfqpoint{1.188393in}{1.741901in}}{\pgfqpoint{1.196293in}{1.738629in}}{\pgfqpoint{1.204529in}{1.738629in}}%
\pgfpathclose%
\pgfusepath{stroke,fill}%
\end{pgfscope}%
\begin{pgfscope}%
\pgfpathrectangle{\pgfqpoint{0.100000in}{0.212622in}}{\pgfqpoint{3.696000in}{3.696000in}}%
\pgfusepath{clip}%
\pgfsetbuttcap%
\pgfsetroundjoin%
\definecolor{currentfill}{rgb}{0.121569,0.466667,0.705882}%
\pgfsetfillcolor{currentfill}%
\pgfsetfillopacity{0.368105}%
\pgfsetlinewidth{1.003750pt}%
\definecolor{currentstroke}{rgb}{0.121569,0.466667,0.705882}%
\pgfsetstrokecolor{currentstroke}%
\pgfsetstrokeopacity{0.368105}%
\pgfsetdash{}{0pt}%
\pgfpathmoveto{\pgfqpoint{1.204529in}{1.738629in}}%
\pgfpathcurveto{\pgfqpoint{1.212765in}{1.738629in}}{\pgfqpoint{1.220665in}{1.741901in}}{\pgfqpoint{1.226489in}{1.747725in}}%
\pgfpathcurveto{\pgfqpoint{1.232313in}{1.753549in}}{\pgfqpoint{1.235585in}{1.761449in}}{\pgfqpoint{1.235585in}{1.769685in}}%
\pgfpathcurveto{\pgfqpoint{1.235585in}{1.777921in}}{\pgfqpoint{1.232313in}{1.785821in}}{\pgfqpoint{1.226489in}{1.791645in}}%
\pgfpathcurveto{\pgfqpoint{1.220665in}{1.797469in}}{\pgfqpoint{1.212765in}{1.800742in}}{\pgfqpoint{1.204529in}{1.800742in}}%
\pgfpathcurveto{\pgfqpoint{1.196293in}{1.800742in}}{\pgfqpoint{1.188393in}{1.797469in}}{\pgfqpoint{1.182569in}{1.791645in}}%
\pgfpathcurveto{\pgfqpoint{1.176745in}{1.785821in}}{\pgfqpoint{1.173472in}{1.777921in}}{\pgfqpoint{1.173472in}{1.769685in}}%
\pgfpathcurveto{\pgfqpoint{1.173472in}{1.761449in}}{\pgfqpoint{1.176745in}{1.753549in}}{\pgfqpoint{1.182569in}{1.747725in}}%
\pgfpathcurveto{\pgfqpoint{1.188393in}{1.741901in}}{\pgfqpoint{1.196293in}{1.738629in}}{\pgfqpoint{1.204529in}{1.738629in}}%
\pgfpathclose%
\pgfusepath{stroke,fill}%
\end{pgfscope}%
\begin{pgfscope}%
\pgfpathrectangle{\pgfqpoint{0.100000in}{0.212622in}}{\pgfqpoint{3.696000in}{3.696000in}}%
\pgfusepath{clip}%
\pgfsetbuttcap%
\pgfsetroundjoin%
\definecolor{currentfill}{rgb}{0.121569,0.466667,0.705882}%
\pgfsetfillcolor{currentfill}%
\pgfsetfillopacity{0.368105}%
\pgfsetlinewidth{1.003750pt}%
\definecolor{currentstroke}{rgb}{0.121569,0.466667,0.705882}%
\pgfsetstrokecolor{currentstroke}%
\pgfsetstrokeopacity{0.368105}%
\pgfsetdash{}{0pt}%
\pgfpathmoveto{\pgfqpoint{1.204529in}{1.738628in}}%
\pgfpathcurveto{\pgfqpoint{1.212765in}{1.738628in}}{\pgfqpoint{1.220665in}{1.741901in}}{\pgfqpoint{1.226489in}{1.747725in}}%
\pgfpathcurveto{\pgfqpoint{1.232313in}{1.753548in}}{\pgfqpoint{1.235585in}{1.761449in}}{\pgfqpoint{1.235585in}{1.769685in}}%
\pgfpathcurveto{\pgfqpoint{1.235585in}{1.777921in}}{\pgfqpoint{1.232313in}{1.785821in}}{\pgfqpoint{1.226489in}{1.791645in}}%
\pgfpathcurveto{\pgfqpoint{1.220665in}{1.797469in}}{\pgfqpoint{1.212765in}{1.800741in}}{\pgfqpoint{1.204529in}{1.800741in}}%
\pgfpathcurveto{\pgfqpoint{1.196292in}{1.800741in}}{\pgfqpoint{1.188392in}{1.797469in}}{\pgfqpoint{1.182568in}{1.791645in}}%
\pgfpathcurveto{\pgfqpoint{1.176745in}{1.785821in}}{\pgfqpoint{1.173472in}{1.777921in}}{\pgfqpoint{1.173472in}{1.769685in}}%
\pgfpathcurveto{\pgfqpoint{1.173472in}{1.761449in}}{\pgfqpoint{1.176745in}{1.753548in}}{\pgfqpoint{1.182568in}{1.747725in}}%
\pgfpathcurveto{\pgfqpoint{1.188392in}{1.741901in}}{\pgfqpoint{1.196292in}{1.738628in}}{\pgfqpoint{1.204529in}{1.738628in}}%
\pgfpathclose%
\pgfusepath{stroke,fill}%
\end{pgfscope}%
\begin{pgfscope}%
\pgfpathrectangle{\pgfqpoint{0.100000in}{0.212622in}}{\pgfqpoint{3.696000in}{3.696000in}}%
\pgfusepath{clip}%
\pgfsetbuttcap%
\pgfsetroundjoin%
\definecolor{currentfill}{rgb}{0.121569,0.466667,0.705882}%
\pgfsetfillcolor{currentfill}%
\pgfsetfillopacity{0.368105}%
\pgfsetlinewidth{1.003750pt}%
\definecolor{currentstroke}{rgb}{0.121569,0.466667,0.705882}%
\pgfsetstrokecolor{currentstroke}%
\pgfsetstrokeopacity{0.368105}%
\pgfsetdash{}{0pt}%
\pgfpathmoveto{\pgfqpoint{1.204529in}{1.738628in}}%
\pgfpathcurveto{\pgfqpoint{1.212765in}{1.738628in}}{\pgfqpoint{1.220665in}{1.741901in}}{\pgfqpoint{1.226489in}{1.747724in}}%
\pgfpathcurveto{\pgfqpoint{1.232313in}{1.753548in}}{\pgfqpoint{1.235585in}{1.761448in}}{\pgfqpoint{1.235585in}{1.769685in}}%
\pgfpathcurveto{\pgfqpoint{1.235585in}{1.777921in}}{\pgfqpoint{1.232313in}{1.785821in}}{\pgfqpoint{1.226489in}{1.791645in}}%
\pgfpathcurveto{\pgfqpoint{1.220665in}{1.797469in}}{\pgfqpoint{1.212765in}{1.800741in}}{\pgfqpoint{1.204529in}{1.800741in}}%
\pgfpathcurveto{\pgfqpoint{1.196292in}{1.800741in}}{\pgfqpoint{1.188392in}{1.797469in}}{\pgfqpoint{1.182568in}{1.791645in}}%
\pgfpathcurveto{\pgfqpoint{1.176744in}{1.785821in}}{\pgfqpoint{1.173472in}{1.777921in}}{\pgfqpoint{1.173472in}{1.769685in}}%
\pgfpathcurveto{\pgfqpoint{1.173472in}{1.761448in}}{\pgfqpoint{1.176744in}{1.753548in}}{\pgfqpoint{1.182568in}{1.747724in}}%
\pgfpathcurveto{\pgfqpoint{1.188392in}{1.741901in}}{\pgfqpoint{1.196292in}{1.738628in}}{\pgfqpoint{1.204529in}{1.738628in}}%
\pgfpathclose%
\pgfusepath{stroke,fill}%
\end{pgfscope}%
\begin{pgfscope}%
\pgfpathrectangle{\pgfqpoint{0.100000in}{0.212622in}}{\pgfqpoint{3.696000in}{3.696000in}}%
\pgfusepath{clip}%
\pgfsetbuttcap%
\pgfsetroundjoin%
\definecolor{currentfill}{rgb}{0.121569,0.466667,0.705882}%
\pgfsetfillcolor{currentfill}%
\pgfsetfillopacity{0.368105}%
\pgfsetlinewidth{1.003750pt}%
\definecolor{currentstroke}{rgb}{0.121569,0.466667,0.705882}%
\pgfsetstrokecolor{currentstroke}%
\pgfsetstrokeopacity{0.368105}%
\pgfsetdash{}{0pt}%
\pgfpathmoveto{\pgfqpoint{1.204529in}{1.738628in}}%
\pgfpathcurveto{\pgfqpoint{1.212765in}{1.738628in}}{\pgfqpoint{1.220665in}{1.741901in}}{\pgfqpoint{1.226489in}{1.747724in}}%
\pgfpathcurveto{\pgfqpoint{1.232313in}{1.753548in}}{\pgfqpoint{1.235585in}{1.761448in}}{\pgfqpoint{1.235585in}{1.769685in}}%
\pgfpathcurveto{\pgfqpoint{1.235585in}{1.777921in}}{\pgfqpoint{1.232313in}{1.785821in}}{\pgfqpoint{1.226489in}{1.791645in}}%
\pgfpathcurveto{\pgfqpoint{1.220665in}{1.797469in}}{\pgfqpoint{1.212765in}{1.800741in}}{\pgfqpoint{1.204529in}{1.800741in}}%
\pgfpathcurveto{\pgfqpoint{1.196292in}{1.800741in}}{\pgfqpoint{1.188392in}{1.797469in}}{\pgfqpoint{1.182568in}{1.791645in}}%
\pgfpathcurveto{\pgfqpoint{1.176744in}{1.785821in}}{\pgfqpoint{1.173472in}{1.777921in}}{\pgfqpoint{1.173472in}{1.769685in}}%
\pgfpathcurveto{\pgfqpoint{1.173472in}{1.761448in}}{\pgfqpoint{1.176744in}{1.753548in}}{\pgfqpoint{1.182568in}{1.747724in}}%
\pgfpathcurveto{\pgfqpoint{1.188392in}{1.741901in}}{\pgfqpoint{1.196292in}{1.738628in}}{\pgfqpoint{1.204529in}{1.738628in}}%
\pgfpathclose%
\pgfusepath{stroke,fill}%
\end{pgfscope}%
\begin{pgfscope}%
\pgfpathrectangle{\pgfqpoint{0.100000in}{0.212622in}}{\pgfqpoint{3.696000in}{3.696000in}}%
\pgfusepath{clip}%
\pgfsetbuttcap%
\pgfsetroundjoin%
\definecolor{currentfill}{rgb}{0.121569,0.466667,0.705882}%
\pgfsetfillcolor{currentfill}%
\pgfsetfillopacity{0.368105}%
\pgfsetlinewidth{1.003750pt}%
\definecolor{currentstroke}{rgb}{0.121569,0.466667,0.705882}%
\pgfsetstrokecolor{currentstroke}%
\pgfsetstrokeopacity{0.368105}%
\pgfsetdash{}{0pt}%
\pgfpathmoveto{\pgfqpoint{1.204529in}{1.738628in}}%
\pgfpathcurveto{\pgfqpoint{1.212765in}{1.738628in}}{\pgfqpoint{1.220665in}{1.741901in}}{\pgfqpoint{1.226489in}{1.747724in}}%
\pgfpathcurveto{\pgfqpoint{1.232313in}{1.753548in}}{\pgfqpoint{1.235585in}{1.761448in}}{\pgfqpoint{1.235585in}{1.769685in}}%
\pgfpathcurveto{\pgfqpoint{1.235585in}{1.777921in}}{\pgfqpoint{1.232313in}{1.785821in}}{\pgfqpoint{1.226489in}{1.791645in}}%
\pgfpathcurveto{\pgfqpoint{1.220665in}{1.797469in}}{\pgfqpoint{1.212765in}{1.800741in}}{\pgfqpoint{1.204529in}{1.800741in}}%
\pgfpathcurveto{\pgfqpoint{1.196292in}{1.800741in}}{\pgfqpoint{1.188392in}{1.797469in}}{\pgfqpoint{1.182568in}{1.791645in}}%
\pgfpathcurveto{\pgfqpoint{1.176744in}{1.785821in}}{\pgfqpoint{1.173472in}{1.777921in}}{\pgfqpoint{1.173472in}{1.769685in}}%
\pgfpathcurveto{\pgfqpoint{1.173472in}{1.761448in}}{\pgfqpoint{1.176744in}{1.753548in}}{\pgfqpoint{1.182568in}{1.747724in}}%
\pgfpathcurveto{\pgfqpoint{1.188392in}{1.741901in}}{\pgfqpoint{1.196292in}{1.738628in}}{\pgfqpoint{1.204529in}{1.738628in}}%
\pgfpathclose%
\pgfusepath{stroke,fill}%
\end{pgfscope}%
\begin{pgfscope}%
\pgfpathrectangle{\pgfqpoint{0.100000in}{0.212622in}}{\pgfqpoint{3.696000in}{3.696000in}}%
\pgfusepath{clip}%
\pgfsetbuttcap%
\pgfsetroundjoin%
\definecolor{currentfill}{rgb}{0.121569,0.466667,0.705882}%
\pgfsetfillcolor{currentfill}%
\pgfsetfillopacity{0.368105}%
\pgfsetlinewidth{1.003750pt}%
\definecolor{currentstroke}{rgb}{0.121569,0.466667,0.705882}%
\pgfsetstrokecolor{currentstroke}%
\pgfsetstrokeopacity{0.368105}%
\pgfsetdash{}{0pt}%
\pgfpathmoveto{\pgfqpoint{1.204529in}{1.738628in}}%
\pgfpathcurveto{\pgfqpoint{1.212765in}{1.738628in}}{\pgfqpoint{1.220665in}{1.741901in}}{\pgfqpoint{1.226489in}{1.747724in}}%
\pgfpathcurveto{\pgfqpoint{1.232313in}{1.753548in}}{\pgfqpoint{1.235585in}{1.761448in}}{\pgfqpoint{1.235585in}{1.769685in}}%
\pgfpathcurveto{\pgfqpoint{1.235585in}{1.777921in}}{\pgfqpoint{1.232313in}{1.785821in}}{\pgfqpoint{1.226489in}{1.791645in}}%
\pgfpathcurveto{\pgfqpoint{1.220665in}{1.797469in}}{\pgfqpoint{1.212765in}{1.800741in}}{\pgfqpoint{1.204529in}{1.800741in}}%
\pgfpathcurveto{\pgfqpoint{1.196292in}{1.800741in}}{\pgfqpoint{1.188392in}{1.797469in}}{\pgfqpoint{1.182568in}{1.791645in}}%
\pgfpathcurveto{\pgfqpoint{1.176744in}{1.785821in}}{\pgfqpoint{1.173472in}{1.777921in}}{\pgfqpoint{1.173472in}{1.769685in}}%
\pgfpathcurveto{\pgfqpoint{1.173472in}{1.761448in}}{\pgfqpoint{1.176744in}{1.753548in}}{\pgfqpoint{1.182568in}{1.747724in}}%
\pgfpathcurveto{\pgfqpoint{1.188392in}{1.741901in}}{\pgfqpoint{1.196292in}{1.738628in}}{\pgfqpoint{1.204529in}{1.738628in}}%
\pgfpathclose%
\pgfusepath{stroke,fill}%
\end{pgfscope}%
\begin{pgfscope}%
\pgfpathrectangle{\pgfqpoint{0.100000in}{0.212622in}}{\pgfqpoint{3.696000in}{3.696000in}}%
\pgfusepath{clip}%
\pgfsetbuttcap%
\pgfsetroundjoin%
\definecolor{currentfill}{rgb}{0.121569,0.466667,0.705882}%
\pgfsetfillcolor{currentfill}%
\pgfsetfillopacity{0.368105}%
\pgfsetlinewidth{1.003750pt}%
\definecolor{currentstroke}{rgb}{0.121569,0.466667,0.705882}%
\pgfsetstrokecolor{currentstroke}%
\pgfsetstrokeopacity{0.368105}%
\pgfsetdash{}{0pt}%
\pgfpathmoveto{\pgfqpoint{1.204529in}{1.738628in}}%
\pgfpathcurveto{\pgfqpoint{1.212765in}{1.738628in}}{\pgfqpoint{1.220665in}{1.741901in}}{\pgfqpoint{1.226489in}{1.747724in}}%
\pgfpathcurveto{\pgfqpoint{1.232313in}{1.753548in}}{\pgfqpoint{1.235585in}{1.761448in}}{\pgfqpoint{1.235585in}{1.769685in}}%
\pgfpathcurveto{\pgfqpoint{1.235585in}{1.777921in}}{\pgfqpoint{1.232313in}{1.785821in}}{\pgfqpoint{1.226489in}{1.791645in}}%
\pgfpathcurveto{\pgfqpoint{1.220665in}{1.797469in}}{\pgfqpoint{1.212765in}{1.800741in}}{\pgfqpoint{1.204529in}{1.800741in}}%
\pgfpathcurveto{\pgfqpoint{1.196292in}{1.800741in}}{\pgfqpoint{1.188392in}{1.797469in}}{\pgfqpoint{1.182568in}{1.791645in}}%
\pgfpathcurveto{\pgfqpoint{1.176744in}{1.785821in}}{\pgfqpoint{1.173472in}{1.777921in}}{\pgfqpoint{1.173472in}{1.769685in}}%
\pgfpathcurveto{\pgfqpoint{1.173472in}{1.761448in}}{\pgfqpoint{1.176744in}{1.753548in}}{\pgfqpoint{1.182568in}{1.747724in}}%
\pgfpathcurveto{\pgfqpoint{1.188392in}{1.741901in}}{\pgfqpoint{1.196292in}{1.738628in}}{\pgfqpoint{1.204529in}{1.738628in}}%
\pgfpathclose%
\pgfusepath{stroke,fill}%
\end{pgfscope}%
\begin{pgfscope}%
\pgfpathrectangle{\pgfqpoint{0.100000in}{0.212622in}}{\pgfqpoint{3.696000in}{3.696000in}}%
\pgfusepath{clip}%
\pgfsetbuttcap%
\pgfsetroundjoin%
\definecolor{currentfill}{rgb}{0.121569,0.466667,0.705882}%
\pgfsetfillcolor{currentfill}%
\pgfsetfillopacity{0.368105}%
\pgfsetlinewidth{1.003750pt}%
\definecolor{currentstroke}{rgb}{0.121569,0.466667,0.705882}%
\pgfsetstrokecolor{currentstroke}%
\pgfsetstrokeopacity{0.368105}%
\pgfsetdash{}{0pt}%
\pgfpathmoveto{\pgfqpoint{1.204529in}{1.738628in}}%
\pgfpathcurveto{\pgfqpoint{1.212765in}{1.738628in}}{\pgfqpoint{1.220665in}{1.741901in}}{\pgfqpoint{1.226489in}{1.747724in}}%
\pgfpathcurveto{\pgfqpoint{1.232313in}{1.753548in}}{\pgfqpoint{1.235585in}{1.761448in}}{\pgfqpoint{1.235585in}{1.769685in}}%
\pgfpathcurveto{\pgfqpoint{1.235585in}{1.777921in}}{\pgfqpoint{1.232313in}{1.785821in}}{\pgfqpoint{1.226489in}{1.791645in}}%
\pgfpathcurveto{\pgfqpoint{1.220665in}{1.797469in}}{\pgfqpoint{1.212765in}{1.800741in}}{\pgfqpoint{1.204529in}{1.800741in}}%
\pgfpathcurveto{\pgfqpoint{1.196292in}{1.800741in}}{\pgfqpoint{1.188392in}{1.797469in}}{\pgfqpoint{1.182568in}{1.791645in}}%
\pgfpathcurveto{\pgfqpoint{1.176744in}{1.785821in}}{\pgfqpoint{1.173472in}{1.777921in}}{\pgfqpoint{1.173472in}{1.769685in}}%
\pgfpathcurveto{\pgfqpoint{1.173472in}{1.761448in}}{\pgfqpoint{1.176744in}{1.753548in}}{\pgfqpoint{1.182568in}{1.747724in}}%
\pgfpathcurveto{\pgfqpoint{1.188392in}{1.741901in}}{\pgfqpoint{1.196292in}{1.738628in}}{\pgfqpoint{1.204529in}{1.738628in}}%
\pgfpathclose%
\pgfusepath{stroke,fill}%
\end{pgfscope}%
\begin{pgfscope}%
\pgfpathrectangle{\pgfqpoint{0.100000in}{0.212622in}}{\pgfqpoint{3.696000in}{3.696000in}}%
\pgfusepath{clip}%
\pgfsetbuttcap%
\pgfsetroundjoin%
\definecolor{currentfill}{rgb}{0.121569,0.466667,0.705882}%
\pgfsetfillcolor{currentfill}%
\pgfsetfillopacity{0.368105}%
\pgfsetlinewidth{1.003750pt}%
\definecolor{currentstroke}{rgb}{0.121569,0.466667,0.705882}%
\pgfsetstrokecolor{currentstroke}%
\pgfsetstrokeopacity{0.368105}%
\pgfsetdash{}{0pt}%
\pgfpathmoveto{\pgfqpoint{1.204529in}{1.738628in}}%
\pgfpathcurveto{\pgfqpoint{1.212765in}{1.738628in}}{\pgfqpoint{1.220665in}{1.741901in}}{\pgfqpoint{1.226489in}{1.747724in}}%
\pgfpathcurveto{\pgfqpoint{1.232313in}{1.753548in}}{\pgfqpoint{1.235585in}{1.761448in}}{\pgfqpoint{1.235585in}{1.769685in}}%
\pgfpathcurveto{\pgfqpoint{1.235585in}{1.777921in}}{\pgfqpoint{1.232313in}{1.785821in}}{\pgfqpoint{1.226489in}{1.791645in}}%
\pgfpathcurveto{\pgfqpoint{1.220665in}{1.797469in}}{\pgfqpoint{1.212765in}{1.800741in}}{\pgfqpoint{1.204529in}{1.800741in}}%
\pgfpathcurveto{\pgfqpoint{1.196292in}{1.800741in}}{\pgfqpoint{1.188392in}{1.797469in}}{\pgfqpoint{1.182568in}{1.791645in}}%
\pgfpathcurveto{\pgfqpoint{1.176744in}{1.785821in}}{\pgfqpoint{1.173472in}{1.777921in}}{\pgfqpoint{1.173472in}{1.769685in}}%
\pgfpathcurveto{\pgfqpoint{1.173472in}{1.761448in}}{\pgfqpoint{1.176744in}{1.753548in}}{\pgfqpoint{1.182568in}{1.747724in}}%
\pgfpathcurveto{\pgfqpoint{1.188392in}{1.741901in}}{\pgfqpoint{1.196292in}{1.738628in}}{\pgfqpoint{1.204529in}{1.738628in}}%
\pgfpathclose%
\pgfusepath{stroke,fill}%
\end{pgfscope}%
\begin{pgfscope}%
\pgfpathrectangle{\pgfqpoint{0.100000in}{0.212622in}}{\pgfqpoint{3.696000in}{3.696000in}}%
\pgfusepath{clip}%
\pgfsetbuttcap%
\pgfsetroundjoin%
\definecolor{currentfill}{rgb}{0.121569,0.466667,0.705882}%
\pgfsetfillcolor{currentfill}%
\pgfsetfillopacity{0.368105}%
\pgfsetlinewidth{1.003750pt}%
\definecolor{currentstroke}{rgb}{0.121569,0.466667,0.705882}%
\pgfsetstrokecolor{currentstroke}%
\pgfsetstrokeopacity{0.368105}%
\pgfsetdash{}{0pt}%
\pgfpathmoveto{\pgfqpoint{1.204529in}{1.738628in}}%
\pgfpathcurveto{\pgfqpoint{1.212765in}{1.738628in}}{\pgfqpoint{1.220665in}{1.741901in}}{\pgfqpoint{1.226489in}{1.747724in}}%
\pgfpathcurveto{\pgfqpoint{1.232313in}{1.753548in}}{\pgfqpoint{1.235585in}{1.761448in}}{\pgfqpoint{1.235585in}{1.769685in}}%
\pgfpathcurveto{\pgfqpoint{1.235585in}{1.777921in}}{\pgfqpoint{1.232313in}{1.785821in}}{\pgfqpoint{1.226489in}{1.791645in}}%
\pgfpathcurveto{\pgfqpoint{1.220665in}{1.797469in}}{\pgfqpoint{1.212765in}{1.800741in}}{\pgfqpoint{1.204529in}{1.800741in}}%
\pgfpathcurveto{\pgfqpoint{1.196292in}{1.800741in}}{\pgfqpoint{1.188392in}{1.797469in}}{\pgfqpoint{1.182568in}{1.791645in}}%
\pgfpathcurveto{\pgfqpoint{1.176744in}{1.785821in}}{\pgfqpoint{1.173472in}{1.777921in}}{\pgfqpoint{1.173472in}{1.769685in}}%
\pgfpathcurveto{\pgfqpoint{1.173472in}{1.761448in}}{\pgfqpoint{1.176744in}{1.753548in}}{\pgfqpoint{1.182568in}{1.747724in}}%
\pgfpathcurveto{\pgfqpoint{1.188392in}{1.741901in}}{\pgfqpoint{1.196292in}{1.738628in}}{\pgfqpoint{1.204529in}{1.738628in}}%
\pgfpathclose%
\pgfusepath{stroke,fill}%
\end{pgfscope}%
\begin{pgfscope}%
\pgfpathrectangle{\pgfqpoint{0.100000in}{0.212622in}}{\pgfqpoint{3.696000in}{3.696000in}}%
\pgfusepath{clip}%
\pgfsetbuttcap%
\pgfsetroundjoin%
\definecolor{currentfill}{rgb}{0.121569,0.466667,0.705882}%
\pgfsetfillcolor{currentfill}%
\pgfsetfillopacity{0.368105}%
\pgfsetlinewidth{1.003750pt}%
\definecolor{currentstroke}{rgb}{0.121569,0.466667,0.705882}%
\pgfsetstrokecolor{currentstroke}%
\pgfsetstrokeopacity{0.368105}%
\pgfsetdash{}{0pt}%
\pgfpathmoveto{\pgfqpoint{1.204529in}{1.738628in}}%
\pgfpathcurveto{\pgfqpoint{1.212765in}{1.738628in}}{\pgfqpoint{1.220665in}{1.741901in}}{\pgfqpoint{1.226489in}{1.747724in}}%
\pgfpathcurveto{\pgfqpoint{1.232313in}{1.753548in}}{\pgfqpoint{1.235585in}{1.761448in}}{\pgfqpoint{1.235585in}{1.769685in}}%
\pgfpathcurveto{\pgfqpoint{1.235585in}{1.777921in}}{\pgfqpoint{1.232313in}{1.785821in}}{\pgfqpoint{1.226489in}{1.791645in}}%
\pgfpathcurveto{\pgfqpoint{1.220665in}{1.797469in}}{\pgfqpoint{1.212765in}{1.800741in}}{\pgfqpoint{1.204529in}{1.800741in}}%
\pgfpathcurveto{\pgfqpoint{1.196292in}{1.800741in}}{\pgfqpoint{1.188392in}{1.797469in}}{\pgfqpoint{1.182568in}{1.791645in}}%
\pgfpathcurveto{\pgfqpoint{1.176744in}{1.785821in}}{\pgfqpoint{1.173472in}{1.777921in}}{\pgfqpoint{1.173472in}{1.769685in}}%
\pgfpathcurveto{\pgfqpoint{1.173472in}{1.761448in}}{\pgfqpoint{1.176744in}{1.753548in}}{\pgfqpoint{1.182568in}{1.747724in}}%
\pgfpathcurveto{\pgfqpoint{1.188392in}{1.741901in}}{\pgfqpoint{1.196292in}{1.738628in}}{\pgfqpoint{1.204529in}{1.738628in}}%
\pgfpathclose%
\pgfusepath{stroke,fill}%
\end{pgfscope}%
\begin{pgfscope}%
\pgfpathrectangle{\pgfqpoint{0.100000in}{0.212622in}}{\pgfqpoint{3.696000in}{3.696000in}}%
\pgfusepath{clip}%
\pgfsetbuttcap%
\pgfsetroundjoin%
\definecolor{currentfill}{rgb}{0.121569,0.466667,0.705882}%
\pgfsetfillcolor{currentfill}%
\pgfsetfillopacity{0.368105}%
\pgfsetlinewidth{1.003750pt}%
\definecolor{currentstroke}{rgb}{0.121569,0.466667,0.705882}%
\pgfsetstrokecolor{currentstroke}%
\pgfsetstrokeopacity{0.368105}%
\pgfsetdash{}{0pt}%
\pgfpathmoveto{\pgfqpoint{1.204529in}{1.738628in}}%
\pgfpathcurveto{\pgfqpoint{1.212765in}{1.738628in}}{\pgfqpoint{1.220665in}{1.741901in}}{\pgfqpoint{1.226489in}{1.747724in}}%
\pgfpathcurveto{\pgfqpoint{1.232313in}{1.753548in}}{\pgfqpoint{1.235585in}{1.761448in}}{\pgfqpoint{1.235585in}{1.769685in}}%
\pgfpathcurveto{\pgfqpoint{1.235585in}{1.777921in}}{\pgfqpoint{1.232313in}{1.785821in}}{\pgfqpoint{1.226489in}{1.791645in}}%
\pgfpathcurveto{\pgfqpoint{1.220665in}{1.797469in}}{\pgfqpoint{1.212765in}{1.800741in}}{\pgfqpoint{1.204529in}{1.800741in}}%
\pgfpathcurveto{\pgfqpoint{1.196292in}{1.800741in}}{\pgfqpoint{1.188392in}{1.797469in}}{\pgfqpoint{1.182568in}{1.791645in}}%
\pgfpathcurveto{\pgfqpoint{1.176744in}{1.785821in}}{\pgfqpoint{1.173472in}{1.777921in}}{\pgfqpoint{1.173472in}{1.769685in}}%
\pgfpathcurveto{\pgfqpoint{1.173472in}{1.761448in}}{\pgfqpoint{1.176744in}{1.753548in}}{\pgfqpoint{1.182568in}{1.747724in}}%
\pgfpathcurveto{\pgfqpoint{1.188392in}{1.741901in}}{\pgfqpoint{1.196292in}{1.738628in}}{\pgfqpoint{1.204529in}{1.738628in}}%
\pgfpathclose%
\pgfusepath{stroke,fill}%
\end{pgfscope}%
\begin{pgfscope}%
\pgfpathrectangle{\pgfqpoint{0.100000in}{0.212622in}}{\pgfqpoint{3.696000in}{3.696000in}}%
\pgfusepath{clip}%
\pgfsetbuttcap%
\pgfsetroundjoin%
\definecolor{currentfill}{rgb}{0.121569,0.466667,0.705882}%
\pgfsetfillcolor{currentfill}%
\pgfsetfillopacity{0.368105}%
\pgfsetlinewidth{1.003750pt}%
\definecolor{currentstroke}{rgb}{0.121569,0.466667,0.705882}%
\pgfsetstrokecolor{currentstroke}%
\pgfsetstrokeopacity{0.368105}%
\pgfsetdash{}{0pt}%
\pgfpathmoveto{\pgfqpoint{1.204529in}{1.738628in}}%
\pgfpathcurveto{\pgfqpoint{1.212765in}{1.738628in}}{\pgfqpoint{1.220665in}{1.741901in}}{\pgfqpoint{1.226489in}{1.747724in}}%
\pgfpathcurveto{\pgfqpoint{1.232313in}{1.753548in}}{\pgfqpoint{1.235585in}{1.761448in}}{\pgfqpoint{1.235585in}{1.769685in}}%
\pgfpathcurveto{\pgfqpoint{1.235585in}{1.777921in}}{\pgfqpoint{1.232313in}{1.785821in}}{\pgfqpoint{1.226489in}{1.791645in}}%
\pgfpathcurveto{\pgfqpoint{1.220665in}{1.797469in}}{\pgfqpoint{1.212765in}{1.800741in}}{\pgfqpoint{1.204529in}{1.800741in}}%
\pgfpathcurveto{\pgfqpoint{1.196292in}{1.800741in}}{\pgfqpoint{1.188392in}{1.797469in}}{\pgfqpoint{1.182568in}{1.791645in}}%
\pgfpathcurveto{\pgfqpoint{1.176744in}{1.785821in}}{\pgfqpoint{1.173472in}{1.777921in}}{\pgfqpoint{1.173472in}{1.769685in}}%
\pgfpathcurveto{\pgfqpoint{1.173472in}{1.761448in}}{\pgfqpoint{1.176744in}{1.753548in}}{\pgfqpoint{1.182568in}{1.747724in}}%
\pgfpathcurveto{\pgfqpoint{1.188392in}{1.741901in}}{\pgfqpoint{1.196292in}{1.738628in}}{\pgfqpoint{1.204529in}{1.738628in}}%
\pgfpathclose%
\pgfusepath{stroke,fill}%
\end{pgfscope}%
\begin{pgfscope}%
\pgfpathrectangle{\pgfqpoint{0.100000in}{0.212622in}}{\pgfqpoint{3.696000in}{3.696000in}}%
\pgfusepath{clip}%
\pgfsetbuttcap%
\pgfsetroundjoin%
\definecolor{currentfill}{rgb}{0.121569,0.466667,0.705882}%
\pgfsetfillcolor{currentfill}%
\pgfsetfillopacity{0.368105}%
\pgfsetlinewidth{1.003750pt}%
\definecolor{currentstroke}{rgb}{0.121569,0.466667,0.705882}%
\pgfsetstrokecolor{currentstroke}%
\pgfsetstrokeopacity{0.368105}%
\pgfsetdash{}{0pt}%
\pgfpathmoveto{\pgfqpoint{1.204529in}{1.738628in}}%
\pgfpathcurveto{\pgfqpoint{1.212765in}{1.738628in}}{\pgfqpoint{1.220665in}{1.741901in}}{\pgfqpoint{1.226489in}{1.747724in}}%
\pgfpathcurveto{\pgfqpoint{1.232313in}{1.753548in}}{\pgfqpoint{1.235585in}{1.761448in}}{\pgfqpoint{1.235585in}{1.769685in}}%
\pgfpathcurveto{\pgfqpoint{1.235585in}{1.777921in}}{\pgfqpoint{1.232313in}{1.785821in}}{\pgfqpoint{1.226489in}{1.791645in}}%
\pgfpathcurveto{\pgfqpoint{1.220665in}{1.797469in}}{\pgfqpoint{1.212765in}{1.800741in}}{\pgfqpoint{1.204529in}{1.800741in}}%
\pgfpathcurveto{\pgfqpoint{1.196292in}{1.800741in}}{\pgfqpoint{1.188392in}{1.797469in}}{\pgfqpoint{1.182568in}{1.791645in}}%
\pgfpathcurveto{\pgfqpoint{1.176744in}{1.785821in}}{\pgfqpoint{1.173472in}{1.777921in}}{\pgfqpoint{1.173472in}{1.769685in}}%
\pgfpathcurveto{\pgfqpoint{1.173472in}{1.761448in}}{\pgfqpoint{1.176744in}{1.753548in}}{\pgfqpoint{1.182568in}{1.747724in}}%
\pgfpathcurveto{\pgfqpoint{1.188392in}{1.741901in}}{\pgfqpoint{1.196292in}{1.738628in}}{\pgfqpoint{1.204529in}{1.738628in}}%
\pgfpathclose%
\pgfusepath{stroke,fill}%
\end{pgfscope}%
\begin{pgfscope}%
\pgfpathrectangle{\pgfqpoint{0.100000in}{0.212622in}}{\pgfqpoint{3.696000in}{3.696000in}}%
\pgfusepath{clip}%
\pgfsetbuttcap%
\pgfsetroundjoin%
\definecolor{currentfill}{rgb}{0.121569,0.466667,0.705882}%
\pgfsetfillcolor{currentfill}%
\pgfsetfillopacity{0.368105}%
\pgfsetlinewidth{1.003750pt}%
\definecolor{currentstroke}{rgb}{0.121569,0.466667,0.705882}%
\pgfsetstrokecolor{currentstroke}%
\pgfsetstrokeopacity{0.368105}%
\pgfsetdash{}{0pt}%
\pgfpathmoveto{\pgfqpoint{1.204529in}{1.738628in}}%
\pgfpathcurveto{\pgfqpoint{1.212765in}{1.738628in}}{\pgfqpoint{1.220665in}{1.741901in}}{\pgfqpoint{1.226489in}{1.747724in}}%
\pgfpathcurveto{\pgfqpoint{1.232313in}{1.753548in}}{\pgfqpoint{1.235585in}{1.761448in}}{\pgfqpoint{1.235585in}{1.769685in}}%
\pgfpathcurveto{\pgfqpoint{1.235585in}{1.777921in}}{\pgfqpoint{1.232313in}{1.785821in}}{\pgfqpoint{1.226489in}{1.791645in}}%
\pgfpathcurveto{\pgfqpoint{1.220665in}{1.797469in}}{\pgfqpoint{1.212765in}{1.800741in}}{\pgfqpoint{1.204529in}{1.800741in}}%
\pgfpathcurveto{\pgfqpoint{1.196292in}{1.800741in}}{\pgfqpoint{1.188392in}{1.797469in}}{\pgfqpoint{1.182568in}{1.791645in}}%
\pgfpathcurveto{\pgfqpoint{1.176744in}{1.785821in}}{\pgfqpoint{1.173472in}{1.777921in}}{\pgfqpoint{1.173472in}{1.769685in}}%
\pgfpathcurveto{\pgfqpoint{1.173472in}{1.761448in}}{\pgfqpoint{1.176744in}{1.753548in}}{\pgfqpoint{1.182568in}{1.747724in}}%
\pgfpathcurveto{\pgfqpoint{1.188392in}{1.741901in}}{\pgfqpoint{1.196292in}{1.738628in}}{\pgfqpoint{1.204529in}{1.738628in}}%
\pgfpathclose%
\pgfusepath{stroke,fill}%
\end{pgfscope}%
\begin{pgfscope}%
\pgfpathrectangle{\pgfqpoint{0.100000in}{0.212622in}}{\pgfqpoint{3.696000in}{3.696000in}}%
\pgfusepath{clip}%
\pgfsetbuttcap%
\pgfsetroundjoin%
\definecolor{currentfill}{rgb}{0.121569,0.466667,0.705882}%
\pgfsetfillcolor{currentfill}%
\pgfsetfillopacity{0.368105}%
\pgfsetlinewidth{1.003750pt}%
\definecolor{currentstroke}{rgb}{0.121569,0.466667,0.705882}%
\pgfsetstrokecolor{currentstroke}%
\pgfsetstrokeopacity{0.368105}%
\pgfsetdash{}{0pt}%
\pgfpathmoveto{\pgfqpoint{1.204529in}{1.738628in}}%
\pgfpathcurveto{\pgfqpoint{1.212765in}{1.738628in}}{\pgfqpoint{1.220665in}{1.741901in}}{\pgfqpoint{1.226489in}{1.747724in}}%
\pgfpathcurveto{\pgfqpoint{1.232313in}{1.753548in}}{\pgfqpoint{1.235585in}{1.761448in}}{\pgfqpoint{1.235585in}{1.769685in}}%
\pgfpathcurveto{\pgfqpoint{1.235585in}{1.777921in}}{\pgfqpoint{1.232313in}{1.785821in}}{\pgfqpoint{1.226489in}{1.791645in}}%
\pgfpathcurveto{\pgfqpoint{1.220665in}{1.797469in}}{\pgfqpoint{1.212765in}{1.800741in}}{\pgfqpoint{1.204529in}{1.800741in}}%
\pgfpathcurveto{\pgfqpoint{1.196292in}{1.800741in}}{\pgfqpoint{1.188392in}{1.797469in}}{\pgfqpoint{1.182568in}{1.791645in}}%
\pgfpathcurveto{\pgfqpoint{1.176744in}{1.785821in}}{\pgfqpoint{1.173472in}{1.777921in}}{\pgfqpoint{1.173472in}{1.769685in}}%
\pgfpathcurveto{\pgfqpoint{1.173472in}{1.761448in}}{\pgfqpoint{1.176744in}{1.753548in}}{\pgfqpoint{1.182568in}{1.747724in}}%
\pgfpathcurveto{\pgfqpoint{1.188392in}{1.741901in}}{\pgfqpoint{1.196292in}{1.738628in}}{\pgfqpoint{1.204529in}{1.738628in}}%
\pgfpathclose%
\pgfusepath{stroke,fill}%
\end{pgfscope}%
\begin{pgfscope}%
\pgfpathrectangle{\pgfqpoint{0.100000in}{0.212622in}}{\pgfqpoint{3.696000in}{3.696000in}}%
\pgfusepath{clip}%
\pgfsetbuttcap%
\pgfsetroundjoin%
\definecolor{currentfill}{rgb}{0.121569,0.466667,0.705882}%
\pgfsetfillcolor{currentfill}%
\pgfsetfillopacity{0.368105}%
\pgfsetlinewidth{1.003750pt}%
\definecolor{currentstroke}{rgb}{0.121569,0.466667,0.705882}%
\pgfsetstrokecolor{currentstroke}%
\pgfsetstrokeopacity{0.368105}%
\pgfsetdash{}{0pt}%
\pgfpathmoveto{\pgfqpoint{1.204529in}{1.738628in}}%
\pgfpathcurveto{\pgfqpoint{1.212765in}{1.738628in}}{\pgfqpoint{1.220665in}{1.741901in}}{\pgfqpoint{1.226489in}{1.747724in}}%
\pgfpathcurveto{\pgfqpoint{1.232313in}{1.753548in}}{\pgfqpoint{1.235585in}{1.761448in}}{\pgfqpoint{1.235585in}{1.769685in}}%
\pgfpathcurveto{\pgfqpoint{1.235585in}{1.777921in}}{\pgfqpoint{1.232313in}{1.785821in}}{\pgfqpoint{1.226489in}{1.791645in}}%
\pgfpathcurveto{\pgfqpoint{1.220665in}{1.797469in}}{\pgfqpoint{1.212765in}{1.800741in}}{\pgfqpoint{1.204529in}{1.800741in}}%
\pgfpathcurveto{\pgfqpoint{1.196292in}{1.800741in}}{\pgfqpoint{1.188392in}{1.797469in}}{\pgfqpoint{1.182568in}{1.791645in}}%
\pgfpathcurveto{\pgfqpoint{1.176744in}{1.785821in}}{\pgfqpoint{1.173472in}{1.777921in}}{\pgfqpoint{1.173472in}{1.769685in}}%
\pgfpathcurveto{\pgfqpoint{1.173472in}{1.761448in}}{\pgfqpoint{1.176744in}{1.753548in}}{\pgfqpoint{1.182568in}{1.747724in}}%
\pgfpathcurveto{\pgfqpoint{1.188392in}{1.741901in}}{\pgfqpoint{1.196292in}{1.738628in}}{\pgfqpoint{1.204529in}{1.738628in}}%
\pgfpathclose%
\pgfusepath{stroke,fill}%
\end{pgfscope}%
\begin{pgfscope}%
\pgfpathrectangle{\pgfqpoint{0.100000in}{0.212622in}}{\pgfqpoint{3.696000in}{3.696000in}}%
\pgfusepath{clip}%
\pgfsetbuttcap%
\pgfsetroundjoin%
\definecolor{currentfill}{rgb}{0.121569,0.466667,0.705882}%
\pgfsetfillcolor{currentfill}%
\pgfsetfillopacity{0.368105}%
\pgfsetlinewidth{1.003750pt}%
\definecolor{currentstroke}{rgb}{0.121569,0.466667,0.705882}%
\pgfsetstrokecolor{currentstroke}%
\pgfsetstrokeopacity{0.368105}%
\pgfsetdash{}{0pt}%
\pgfpathmoveto{\pgfqpoint{1.204529in}{1.738628in}}%
\pgfpathcurveto{\pgfqpoint{1.212765in}{1.738628in}}{\pgfqpoint{1.220665in}{1.741901in}}{\pgfqpoint{1.226489in}{1.747724in}}%
\pgfpathcurveto{\pgfqpoint{1.232313in}{1.753548in}}{\pgfqpoint{1.235585in}{1.761448in}}{\pgfqpoint{1.235585in}{1.769685in}}%
\pgfpathcurveto{\pgfqpoint{1.235585in}{1.777921in}}{\pgfqpoint{1.232313in}{1.785821in}}{\pgfqpoint{1.226489in}{1.791645in}}%
\pgfpathcurveto{\pgfqpoint{1.220665in}{1.797469in}}{\pgfqpoint{1.212765in}{1.800741in}}{\pgfqpoint{1.204529in}{1.800741in}}%
\pgfpathcurveto{\pgfqpoint{1.196292in}{1.800741in}}{\pgfqpoint{1.188392in}{1.797469in}}{\pgfqpoint{1.182568in}{1.791645in}}%
\pgfpathcurveto{\pgfqpoint{1.176744in}{1.785821in}}{\pgfqpoint{1.173472in}{1.777921in}}{\pgfqpoint{1.173472in}{1.769685in}}%
\pgfpathcurveto{\pgfqpoint{1.173472in}{1.761448in}}{\pgfqpoint{1.176744in}{1.753548in}}{\pgfqpoint{1.182568in}{1.747724in}}%
\pgfpathcurveto{\pgfqpoint{1.188392in}{1.741901in}}{\pgfqpoint{1.196292in}{1.738628in}}{\pgfqpoint{1.204529in}{1.738628in}}%
\pgfpathclose%
\pgfusepath{stroke,fill}%
\end{pgfscope}%
\begin{pgfscope}%
\pgfpathrectangle{\pgfqpoint{0.100000in}{0.212622in}}{\pgfqpoint{3.696000in}{3.696000in}}%
\pgfusepath{clip}%
\pgfsetbuttcap%
\pgfsetroundjoin%
\definecolor{currentfill}{rgb}{0.121569,0.466667,0.705882}%
\pgfsetfillcolor{currentfill}%
\pgfsetfillopacity{0.368105}%
\pgfsetlinewidth{1.003750pt}%
\definecolor{currentstroke}{rgb}{0.121569,0.466667,0.705882}%
\pgfsetstrokecolor{currentstroke}%
\pgfsetstrokeopacity{0.368105}%
\pgfsetdash{}{0pt}%
\pgfpathmoveto{\pgfqpoint{1.204529in}{1.738628in}}%
\pgfpathcurveto{\pgfqpoint{1.212765in}{1.738628in}}{\pgfqpoint{1.220665in}{1.741901in}}{\pgfqpoint{1.226489in}{1.747724in}}%
\pgfpathcurveto{\pgfqpoint{1.232313in}{1.753548in}}{\pgfqpoint{1.235585in}{1.761448in}}{\pgfqpoint{1.235585in}{1.769685in}}%
\pgfpathcurveto{\pgfqpoint{1.235585in}{1.777921in}}{\pgfqpoint{1.232313in}{1.785821in}}{\pgfqpoint{1.226489in}{1.791645in}}%
\pgfpathcurveto{\pgfqpoint{1.220665in}{1.797469in}}{\pgfqpoint{1.212765in}{1.800741in}}{\pgfqpoint{1.204529in}{1.800741in}}%
\pgfpathcurveto{\pgfqpoint{1.196292in}{1.800741in}}{\pgfqpoint{1.188392in}{1.797469in}}{\pgfqpoint{1.182568in}{1.791645in}}%
\pgfpathcurveto{\pgfqpoint{1.176744in}{1.785821in}}{\pgfqpoint{1.173472in}{1.777921in}}{\pgfqpoint{1.173472in}{1.769685in}}%
\pgfpathcurveto{\pgfqpoint{1.173472in}{1.761448in}}{\pgfqpoint{1.176744in}{1.753548in}}{\pgfqpoint{1.182568in}{1.747724in}}%
\pgfpathcurveto{\pgfqpoint{1.188392in}{1.741901in}}{\pgfqpoint{1.196292in}{1.738628in}}{\pgfqpoint{1.204529in}{1.738628in}}%
\pgfpathclose%
\pgfusepath{stroke,fill}%
\end{pgfscope}%
\begin{pgfscope}%
\pgfpathrectangle{\pgfqpoint{0.100000in}{0.212622in}}{\pgfqpoint{3.696000in}{3.696000in}}%
\pgfusepath{clip}%
\pgfsetbuttcap%
\pgfsetroundjoin%
\definecolor{currentfill}{rgb}{0.121569,0.466667,0.705882}%
\pgfsetfillcolor{currentfill}%
\pgfsetfillopacity{0.368105}%
\pgfsetlinewidth{1.003750pt}%
\definecolor{currentstroke}{rgb}{0.121569,0.466667,0.705882}%
\pgfsetstrokecolor{currentstroke}%
\pgfsetstrokeopacity{0.368105}%
\pgfsetdash{}{0pt}%
\pgfpathmoveto{\pgfqpoint{1.204529in}{1.738628in}}%
\pgfpathcurveto{\pgfqpoint{1.212765in}{1.738628in}}{\pgfqpoint{1.220665in}{1.741901in}}{\pgfqpoint{1.226489in}{1.747724in}}%
\pgfpathcurveto{\pgfqpoint{1.232313in}{1.753548in}}{\pgfqpoint{1.235585in}{1.761448in}}{\pgfqpoint{1.235585in}{1.769685in}}%
\pgfpathcurveto{\pgfqpoint{1.235585in}{1.777921in}}{\pgfqpoint{1.232313in}{1.785821in}}{\pgfqpoint{1.226489in}{1.791645in}}%
\pgfpathcurveto{\pgfqpoint{1.220665in}{1.797469in}}{\pgfqpoint{1.212765in}{1.800741in}}{\pgfqpoint{1.204529in}{1.800741in}}%
\pgfpathcurveto{\pgfqpoint{1.196292in}{1.800741in}}{\pgfqpoint{1.188392in}{1.797469in}}{\pgfqpoint{1.182568in}{1.791645in}}%
\pgfpathcurveto{\pgfqpoint{1.176744in}{1.785821in}}{\pgfqpoint{1.173472in}{1.777921in}}{\pgfqpoint{1.173472in}{1.769685in}}%
\pgfpathcurveto{\pgfqpoint{1.173472in}{1.761448in}}{\pgfqpoint{1.176744in}{1.753548in}}{\pgfqpoint{1.182568in}{1.747724in}}%
\pgfpathcurveto{\pgfqpoint{1.188392in}{1.741901in}}{\pgfqpoint{1.196292in}{1.738628in}}{\pgfqpoint{1.204529in}{1.738628in}}%
\pgfpathclose%
\pgfusepath{stroke,fill}%
\end{pgfscope}%
\begin{pgfscope}%
\pgfpathrectangle{\pgfqpoint{0.100000in}{0.212622in}}{\pgfqpoint{3.696000in}{3.696000in}}%
\pgfusepath{clip}%
\pgfsetbuttcap%
\pgfsetroundjoin%
\definecolor{currentfill}{rgb}{0.121569,0.466667,0.705882}%
\pgfsetfillcolor{currentfill}%
\pgfsetfillopacity{0.368105}%
\pgfsetlinewidth{1.003750pt}%
\definecolor{currentstroke}{rgb}{0.121569,0.466667,0.705882}%
\pgfsetstrokecolor{currentstroke}%
\pgfsetstrokeopacity{0.368105}%
\pgfsetdash{}{0pt}%
\pgfpathmoveto{\pgfqpoint{1.204529in}{1.738628in}}%
\pgfpathcurveto{\pgfqpoint{1.212765in}{1.738628in}}{\pgfqpoint{1.220665in}{1.741901in}}{\pgfqpoint{1.226489in}{1.747724in}}%
\pgfpathcurveto{\pgfqpoint{1.232313in}{1.753548in}}{\pgfqpoint{1.235585in}{1.761448in}}{\pgfqpoint{1.235585in}{1.769685in}}%
\pgfpathcurveto{\pgfqpoint{1.235585in}{1.777921in}}{\pgfqpoint{1.232313in}{1.785821in}}{\pgfqpoint{1.226489in}{1.791645in}}%
\pgfpathcurveto{\pgfqpoint{1.220665in}{1.797469in}}{\pgfqpoint{1.212765in}{1.800741in}}{\pgfqpoint{1.204529in}{1.800741in}}%
\pgfpathcurveto{\pgfqpoint{1.196292in}{1.800741in}}{\pgfqpoint{1.188392in}{1.797469in}}{\pgfqpoint{1.182568in}{1.791645in}}%
\pgfpathcurveto{\pgfqpoint{1.176744in}{1.785821in}}{\pgfqpoint{1.173472in}{1.777921in}}{\pgfqpoint{1.173472in}{1.769685in}}%
\pgfpathcurveto{\pgfqpoint{1.173472in}{1.761448in}}{\pgfqpoint{1.176744in}{1.753548in}}{\pgfqpoint{1.182568in}{1.747724in}}%
\pgfpathcurveto{\pgfqpoint{1.188392in}{1.741901in}}{\pgfqpoint{1.196292in}{1.738628in}}{\pgfqpoint{1.204529in}{1.738628in}}%
\pgfpathclose%
\pgfusepath{stroke,fill}%
\end{pgfscope}%
\begin{pgfscope}%
\pgfpathrectangle{\pgfqpoint{0.100000in}{0.212622in}}{\pgfqpoint{3.696000in}{3.696000in}}%
\pgfusepath{clip}%
\pgfsetbuttcap%
\pgfsetroundjoin%
\definecolor{currentfill}{rgb}{0.121569,0.466667,0.705882}%
\pgfsetfillcolor{currentfill}%
\pgfsetfillopacity{0.368105}%
\pgfsetlinewidth{1.003750pt}%
\definecolor{currentstroke}{rgb}{0.121569,0.466667,0.705882}%
\pgfsetstrokecolor{currentstroke}%
\pgfsetstrokeopacity{0.368105}%
\pgfsetdash{}{0pt}%
\pgfpathmoveto{\pgfqpoint{1.204529in}{1.738628in}}%
\pgfpathcurveto{\pgfqpoint{1.212765in}{1.738628in}}{\pgfqpoint{1.220665in}{1.741901in}}{\pgfqpoint{1.226489in}{1.747724in}}%
\pgfpathcurveto{\pgfqpoint{1.232313in}{1.753548in}}{\pgfqpoint{1.235585in}{1.761448in}}{\pgfqpoint{1.235585in}{1.769685in}}%
\pgfpathcurveto{\pgfqpoint{1.235585in}{1.777921in}}{\pgfqpoint{1.232313in}{1.785821in}}{\pgfqpoint{1.226489in}{1.791645in}}%
\pgfpathcurveto{\pgfqpoint{1.220665in}{1.797469in}}{\pgfqpoint{1.212765in}{1.800741in}}{\pgfqpoint{1.204529in}{1.800741in}}%
\pgfpathcurveto{\pgfqpoint{1.196292in}{1.800741in}}{\pgfqpoint{1.188392in}{1.797469in}}{\pgfqpoint{1.182568in}{1.791645in}}%
\pgfpathcurveto{\pgfqpoint{1.176744in}{1.785821in}}{\pgfqpoint{1.173472in}{1.777921in}}{\pgfqpoint{1.173472in}{1.769685in}}%
\pgfpathcurveto{\pgfqpoint{1.173472in}{1.761448in}}{\pgfqpoint{1.176744in}{1.753548in}}{\pgfqpoint{1.182568in}{1.747724in}}%
\pgfpathcurveto{\pgfqpoint{1.188392in}{1.741901in}}{\pgfqpoint{1.196292in}{1.738628in}}{\pgfqpoint{1.204529in}{1.738628in}}%
\pgfpathclose%
\pgfusepath{stroke,fill}%
\end{pgfscope}%
\begin{pgfscope}%
\pgfpathrectangle{\pgfqpoint{0.100000in}{0.212622in}}{\pgfqpoint{3.696000in}{3.696000in}}%
\pgfusepath{clip}%
\pgfsetbuttcap%
\pgfsetroundjoin%
\definecolor{currentfill}{rgb}{0.121569,0.466667,0.705882}%
\pgfsetfillcolor{currentfill}%
\pgfsetfillopacity{0.368105}%
\pgfsetlinewidth{1.003750pt}%
\definecolor{currentstroke}{rgb}{0.121569,0.466667,0.705882}%
\pgfsetstrokecolor{currentstroke}%
\pgfsetstrokeopacity{0.368105}%
\pgfsetdash{}{0pt}%
\pgfpathmoveto{\pgfqpoint{1.204529in}{1.738628in}}%
\pgfpathcurveto{\pgfqpoint{1.212765in}{1.738628in}}{\pgfqpoint{1.220665in}{1.741901in}}{\pgfqpoint{1.226489in}{1.747724in}}%
\pgfpathcurveto{\pgfqpoint{1.232313in}{1.753548in}}{\pgfqpoint{1.235585in}{1.761448in}}{\pgfqpoint{1.235585in}{1.769685in}}%
\pgfpathcurveto{\pgfqpoint{1.235585in}{1.777921in}}{\pgfqpoint{1.232313in}{1.785821in}}{\pgfqpoint{1.226489in}{1.791645in}}%
\pgfpathcurveto{\pgfqpoint{1.220665in}{1.797469in}}{\pgfqpoint{1.212765in}{1.800741in}}{\pgfqpoint{1.204529in}{1.800741in}}%
\pgfpathcurveto{\pgfqpoint{1.196292in}{1.800741in}}{\pgfqpoint{1.188392in}{1.797469in}}{\pgfqpoint{1.182568in}{1.791645in}}%
\pgfpathcurveto{\pgfqpoint{1.176744in}{1.785821in}}{\pgfqpoint{1.173472in}{1.777921in}}{\pgfqpoint{1.173472in}{1.769685in}}%
\pgfpathcurveto{\pgfqpoint{1.173472in}{1.761448in}}{\pgfqpoint{1.176744in}{1.753548in}}{\pgfqpoint{1.182568in}{1.747724in}}%
\pgfpathcurveto{\pgfqpoint{1.188392in}{1.741901in}}{\pgfqpoint{1.196292in}{1.738628in}}{\pgfqpoint{1.204529in}{1.738628in}}%
\pgfpathclose%
\pgfusepath{stroke,fill}%
\end{pgfscope}%
\begin{pgfscope}%
\pgfpathrectangle{\pgfqpoint{0.100000in}{0.212622in}}{\pgfqpoint{3.696000in}{3.696000in}}%
\pgfusepath{clip}%
\pgfsetbuttcap%
\pgfsetroundjoin%
\definecolor{currentfill}{rgb}{0.121569,0.466667,0.705882}%
\pgfsetfillcolor{currentfill}%
\pgfsetfillopacity{0.368105}%
\pgfsetlinewidth{1.003750pt}%
\definecolor{currentstroke}{rgb}{0.121569,0.466667,0.705882}%
\pgfsetstrokecolor{currentstroke}%
\pgfsetstrokeopacity{0.368105}%
\pgfsetdash{}{0pt}%
\pgfpathmoveto{\pgfqpoint{1.204529in}{1.738628in}}%
\pgfpathcurveto{\pgfqpoint{1.212765in}{1.738628in}}{\pgfqpoint{1.220665in}{1.741901in}}{\pgfqpoint{1.226489in}{1.747724in}}%
\pgfpathcurveto{\pgfqpoint{1.232313in}{1.753548in}}{\pgfqpoint{1.235585in}{1.761448in}}{\pgfqpoint{1.235585in}{1.769685in}}%
\pgfpathcurveto{\pgfqpoint{1.235585in}{1.777921in}}{\pgfqpoint{1.232313in}{1.785821in}}{\pgfqpoint{1.226489in}{1.791645in}}%
\pgfpathcurveto{\pgfqpoint{1.220665in}{1.797469in}}{\pgfqpoint{1.212765in}{1.800741in}}{\pgfqpoint{1.204529in}{1.800741in}}%
\pgfpathcurveto{\pgfqpoint{1.196292in}{1.800741in}}{\pgfqpoint{1.188392in}{1.797469in}}{\pgfqpoint{1.182568in}{1.791645in}}%
\pgfpathcurveto{\pgfqpoint{1.176744in}{1.785821in}}{\pgfqpoint{1.173472in}{1.777921in}}{\pgfqpoint{1.173472in}{1.769685in}}%
\pgfpathcurveto{\pgfqpoint{1.173472in}{1.761448in}}{\pgfqpoint{1.176744in}{1.753548in}}{\pgfqpoint{1.182568in}{1.747724in}}%
\pgfpathcurveto{\pgfqpoint{1.188392in}{1.741901in}}{\pgfqpoint{1.196292in}{1.738628in}}{\pgfqpoint{1.204529in}{1.738628in}}%
\pgfpathclose%
\pgfusepath{stroke,fill}%
\end{pgfscope}%
\begin{pgfscope}%
\pgfpathrectangle{\pgfqpoint{0.100000in}{0.212622in}}{\pgfqpoint{3.696000in}{3.696000in}}%
\pgfusepath{clip}%
\pgfsetbuttcap%
\pgfsetroundjoin%
\definecolor{currentfill}{rgb}{0.121569,0.466667,0.705882}%
\pgfsetfillcolor{currentfill}%
\pgfsetfillopacity{0.368105}%
\pgfsetlinewidth{1.003750pt}%
\definecolor{currentstroke}{rgb}{0.121569,0.466667,0.705882}%
\pgfsetstrokecolor{currentstroke}%
\pgfsetstrokeopacity{0.368105}%
\pgfsetdash{}{0pt}%
\pgfpathmoveto{\pgfqpoint{1.204529in}{1.738628in}}%
\pgfpathcurveto{\pgfqpoint{1.212765in}{1.738628in}}{\pgfqpoint{1.220665in}{1.741901in}}{\pgfqpoint{1.226489in}{1.747724in}}%
\pgfpathcurveto{\pgfqpoint{1.232313in}{1.753548in}}{\pgfqpoint{1.235585in}{1.761448in}}{\pgfqpoint{1.235585in}{1.769685in}}%
\pgfpathcurveto{\pgfqpoint{1.235585in}{1.777921in}}{\pgfqpoint{1.232313in}{1.785821in}}{\pgfqpoint{1.226489in}{1.791645in}}%
\pgfpathcurveto{\pgfqpoint{1.220665in}{1.797469in}}{\pgfqpoint{1.212765in}{1.800741in}}{\pgfqpoint{1.204529in}{1.800741in}}%
\pgfpathcurveto{\pgfqpoint{1.196292in}{1.800741in}}{\pgfqpoint{1.188392in}{1.797469in}}{\pgfqpoint{1.182568in}{1.791645in}}%
\pgfpathcurveto{\pgfqpoint{1.176744in}{1.785821in}}{\pgfqpoint{1.173472in}{1.777921in}}{\pgfqpoint{1.173472in}{1.769685in}}%
\pgfpathcurveto{\pgfqpoint{1.173472in}{1.761448in}}{\pgfqpoint{1.176744in}{1.753548in}}{\pgfqpoint{1.182568in}{1.747724in}}%
\pgfpathcurveto{\pgfqpoint{1.188392in}{1.741901in}}{\pgfqpoint{1.196292in}{1.738628in}}{\pgfqpoint{1.204529in}{1.738628in}}%
\pgfpathclose%
\pgfusepath{stroke,fill}%
\end{pgfscope}%
\begin{pgfscope}%
\pgfpathrectangle{\pgfqpoint{0.100000in}{0.212622in}}{\pgfqpoint{3.696000in}{3.696000in}}%
\pgfusepath{clip}%
\pgfsetbuttcap%
\pgfsetroundjoin%
\definecolor{currentfill}{rgb}{0.121569,0.466667,0.705882}%
\pgfsetfillcolor{currentfill}%
\pgfsetfillopacity{0.368105}%
\pgfsetlinewidth{1.003750pt}%
\definecolor{currentstroke}{rgb}{0.121569,0.466667,0.705882}%
\pgfsetstrokecolor{currentstroke}%
\pgfsetstrokeopacity{0.368105}%
\pgfsetdash{}{0pt}%
\pgfpathmoveto{\pgfqpoint{1.204529in}{1.738628in}}%
\pgfpathcurveto{\pgfqpoint{1.212765in}{1.738628in}}{\pgfqpoint{1.220665in}{1.741901in}}{\pgfqpoint{1.226489in}{1.747724in}}%
\pgfpathcurveto{\pgfqpoint{1.232313in}{1.753548in}}{\pgfqpoint{1.235585in}{1.761448in}}{\pgfqpoint{1.235585in}{1.769685in}}%
\pgfpathcurveto{\pgfqpoint{1.235585in}{1.777921in}}{\pgfqpoint{1.232313in}{1.785821in}}{\pgfqpoint{1.226489in}{1.791645in}}%
\pgfpathcurveto{\pgfqpoint{1.220665in}{1.797469in}}{\pgfqpoint{1.212765in}{1.800741in}}{\pgfqpoint{1.204529in}{1.800741in}}%
\pgfpathcurveto{\pgfqpoint{1.196292in}{1.800741in}}{\pgfqpoint{1.188392in}{1.797469in}}{\pgfqpoint{1.182568in}{1.791645in}}%
\pgfpathcurveto{\pgfqpoint{1.176744in}{1.785821in}}{\pgfqpoint{1.173472in}{1.777921in}}{\pgfqpoint{1.173472in}{1.769685in}}%
\pgfpathcurveto{\pgfqpoint{1.173472in}{1.761448in}}{\pgfqpoint{1.176744in}{1.753548in}}{\pgfqpoint{1.182568in}{1.747724in}}%
\pgfpathcurveto{\pgfqpoint{1.188392in}{1.741901in}}{\pgfqpoint{1.196292in}{1.738628in}}{\pgfqpoint{1.204529in}{1.738628in}}%
\pgfpathclose%
\pgfusepath{stroke,fill}%
\end{pgfscope}%
\begin{pgfscope}%
\pgfpathrectangle{\pgfqpoint{0.100000in}{0.212622in}}{\pgfqpoint{3.696000in}{3.696000in}}%
\pgfusepath{clip}%
\pgfsetbuttcap%
\pgfsetroundjoin%
\definecolor{currentfill}{rgb}{0.121569,0.466667,0.705882}%
\pgfsetfillcolor{currentfill}%
\pgfsetfillopacity{0.368105}%
\pgfsetlinewidth{1.003750pt}%
\definecolor{currentstroke}{rgb}{0.121569,0.466667,0.705882}%
\pgfsetstrokecolor{currentstroke}%
\pgfsetstrokeopacity{0.368105}%
\pgfsetdash{}{0pt}%
\pgfpathmoveto{\pgfqpoint{1.204529in}{1.738628in}}%
\pgfpathcurveto{\pgfqpoint{1.212765in}{1.738628in}}{\pgfqpoint{1.220665in}{1.741901in}}{\pgfqpoint{1.226489in}{1.747724in}}%
\pgfpathcurveto{\pgfqpoint{1.232313in}{1.753548in}}{\pgfqpoint{1.235585in}{1.761448in}}{\pgfqpoint{1.235585in}{1.769685in}}%
\pgfpathcurveto{\pgfqpoint{1.235585in}{1.777921in}}{\pgfqpoint{1.232313in}{1.785821in}}{\pgfqpoint{1.226489in}{1.791645in}}%
\pgfpathcurveto{\pgfqpoint{1.220665in}{1.797469in}}{\pgfqpoint{1.212765in}{1.800741in}}{\pgfqpoint{1.204529in}{1.800741in}}%
\pgfpathcurveto{\pgfqpoint{1.196292in}{1.800741in}}{\pgfqpoint{1.188392in}{1.797469in}}{\pgfqpoint{1.182568in}{1.791645in}}%
\pgfpathcurveto{\pgfqpoint{1.176744in}{1.785821in}}{\pgfqpoint{1.173472in}{1.777921in}}{\pgfqpoint{1.173472in}{1.769685in}}%
\pgfpathcurveto{\pgfqpoint{1.173472in}{1.761448in}}{\pgfqpoint{1.176744in}{1.753548in}}{\pgfqpoint{1.182568in}{1.747724in}}%
\pgfpathcurveto{\pgfqpoint{1.188392in}{1.741901in}}{\pgfqpoint{1.196292in}{1.738628in}}{\pgfqpoint{1.204529in}{1.738628in}}%
\pgfpathclose%
\pgfusepath{stroke,fill}%
\end{pgfscope}%
\begin{pgfscope}%
\pgfpathrectangle{\pgfqpoint{0.100000in}{0.212622in}}{\pgfqpoint{3.696000in}{3.696000in}}%
\pgfusepath{clip}%
\pgfsetbuttcap%
\pgfsetroundjoin%
\definecolor{currentfill}{rgb}{0.121569,0.466667,0.705882}%
\pgfsetfillcolor{currentfill}%
\pgfsetfillopacity{0.368105}%
\pgfsetlinewidth{1.003750pt}%
\definecolor{currentstroke}{rgb}{0.121569,0.466667,0.705882}%
\pgfsetstrokecolor{currentstroke}%
\pgfsetstrokeopacity{0.368105}%
\pgfsetdash{}{0pt}%
\pgfpathmoveto{\pgfqpoint{1.204529in}{1.738628in}}%
\pgfpathcurveto{\pgfqpoint{1.212765in}{1.738628in}}{\pgfqpoint{1.220665in}{1.741901in}}{\pgfqpoint{1.226489in}{1.747724in}}%
\pgfpathcurveto{\pgfqpoint{1.232313in}{1.753548in}}{\pgfqpoint{1.235585in}{1.761448in}}{\pgfqpoint{1.235585in}{1.769685in}}%
\pgfpathcurveto{\pgfqpoint{1.235585in}{1.777921in}}{\pgfqpoint{1.232313in}{1.785821in}}{\pgfqpoint{1.226489in}{1.791645in}}%
\pgfpathcurveto{\pgfqpoint{1.220665in}{1.797469in}}{\pgfqpoint{1.212765in}{1.800741in}}{\pgfqpoint{1.204529in}{1.800741in}}%
\pgfpathcurveto{\pgfqpoint{1.196292in}{1.800741in}}{\pgfqpoint{1.188392in}{1.797469in}}{\pgfqpoint{1.182568in}{1.791645in}}%
\pgfpathcurveto{\pgfqpoint{1.176744in}{1.785821in}}{\pgfqpoint{1.173472in}{1.777921in}}{\pgfqpoint{1.173472in}{1.769685in}}%
\pgfpathcurveto{\pgfqpoint{1.173472in}{1.761448in}}{\pgfqpoint{1.176744in}{1.753548in}}{\pgfqpoint{1.182568in}{1.747724in}}%
\pgfpathcurveto{\pgfqpoint{1.188392in}{1.741901in}}{\pgfqpoint{1.196292in}{1.738628in}}{\pgfqpoint{1.204529in}{1.738628in}}%
\pgfpathclose%
\pgfusepath{stroke,fill}%
\end{pgfscope}%
\begin{pgfscope}%
\pgfpathrectangle{\pgfqpoint{0.100000in}{0.212622in}}{\pgfqpoint{3.696000in}{3.696000in}}%
\pgfusepath{clip}%
\pgfsetbuttcap%
\pgfsetroundjoin%
\definecolor{currentfill}{rgb}{0.121569,0.466667,0.705882}%
\pgfsetfillcolor{currentfill}%
\pgfsetfillopacity{0.368105}%
\pgfsetlinewidth{1.003750pt}%
\definecolor{currentstroke}{rgb}{0.121569,0.466667,0.705882}%
\pgfsetstrokecolor{currentstroke}%
\pgfsetstrokeopacity{0.368105}%
\pgfsetdash{}{0pt}%
\pgfpathmoveto{\pgfqpoint{1.204529in}{1.738628in}}%
\pgfpathcurveto{\pgfqpoint{1.212765in}{1.738628in}}{\pgfqpoint{1.220665in}{1.741901in}}{\pgfqpoint{1.226489in}{1.747724in}}%
\pgfpathcurveto{\pgfqpoint{1.232313in}{1.753548in}}{\pgfqpoint{1.235585in}{1.761448in}}{\pgfqpoint{1.235585in}{1.769685in}}%
\pgfpathcurveto{\pgfqpoint{1.235585in}{1.777921in}}{\pgfqpoint{1.232313in}{1.785821in}}{\pgfqpoint{1.226489in}{1.791645in}}%
\pgfpathcurveto{\pgfqpoint{1.220665in}{1.797469in}}{\pgfqpoint{1.212765in}{1.800741in}}{\pgfqpoint{1.204529in}{1.800741in}}%
\pgfpathcurveto{\pgfqpoint{1.196292in}{1.800741in}}{\pgfqpoint{1.188392in}{1.797469in}}{\pgfqpoint{1.182568in}{1.791645in}}%
\pgfpathcurveto{\pgfqpoint{1.176744in}{1.785821in}}{\pgfqpoint{1.173472in}{1.777921in}}{\pgfqpoint{1.173472in}{1.769685in}}%
\pgfpathcurveto{\pgfqpoint{1.173472in}{1.761448in}}{\pgfqpoint{1.176744in}{1.753548in}}{\pgfqpoint{1.182568in}{1.747724in}}%
\pgfpathcurveto{\pgfqpoint{1.188392in}{1.741901in}}{\pgfqpoint{1.196292in}{1.738628in}}{\pgfqpoint{1.204529in}{1.738628in}}%
\pgfpathclose%
\pgfusepath{stroke,fill}%
\end{pgfscope}%
\begin{pgfscope}%
\pgfpathrectangle{\pgfqpoint{0.100000in}{0.212622in}}{\pgfqpoint{3.696000in}{3.696000in}}%
\pgfusepath{clip}%
\pgfsetbuttcap%
\pgfsetroundjoin%
\definecolor{currentfill}{rgb}{0.121569,0.466667,0.705882}%
\pgfsetfillcolor{currentfill}%
\pgfsetfillopacity{0.368105}%
\pgfsetlinewidth{1.003750pt}%
\definecolor{currentstroke}{rgb}{0.121569,0.466667,0.705882}%
\pgfsetstrokecolor{currentstroke}%
\pgfsetstrokeopacity{0.368105}%
\pgfsetdash{}{0pt}%
\pgfpathmoveto{\pgfqpoint{1.204529in}{1.738628in}}%
\pgfpathcurveto{\pgfqpoint{1.212765in}{1.738628in}}{\pgfqpoint{1.220665in}{1.741901in}}{\pgfqpoint{1.226489in}{1.747724in}}%
\pgfpathcurveto{\pgfqpoint{1.232313in}{1.753548in}}{\pgfqpoint{1.235585in}{1.761448in}}{\pgfqpoint{1.235585in}{1.769685in}}%
\pgfpathcurveto{\pgfqpoint{1.235585in}{1.777921in}}{\pgfqpoint{1.232313in}{1.785821in}}{\pgfqpoint{1.226489in}{1.791645in}}%
\pgfpathcurveto{\pgfqpoint{1.220665in}{1.797469in}}{\pgfqpoint{1.212765in}{1.800741in}}{\pgfqpoint{1.204529in}{1.800741in}}%
\pgfpathcurveto{\pgfqpoint{1.196292in}{1.800741in}}{\pgfqpoint{1.188392in}{1.797469in}}{\pgfqpoint{1.182568in}{1.791645in}}%
\pgfpathcurveto{\pgfqpoint{1.176744in}{1.785821in}}{\pgfqpoint{1.173472in}{1.777921in}}{\pgfqpoint{1.173472in}{1.769685in}}%
\pgfpathcurveto{\pgfqpoint{1.173472in}{1.761448in}}{\pgfqpoint{1.176744in}{1.753548in}}{\pgfqpoint{1.182568in}{1.747724in}}%
\pgfpathcurveto{\pgfqpoint{1.188392in}{1.741901in}}{\pgfqpoint{1.196292in}{1.738628in}}{\pgfqpoint{1.204529in}{1.738628in}}%
\pgfpathclose%
\pgfusepath{stroke,fill}%
\end{pgfscope}%
\begin{pgfscope}%
\pgfpathrectangle{\pgfqpoint{0.100000in}{0.212622in}}{\pgfqpoint{3.696000in}{3.696000in}}%
\pgfusepath{clip}%
\pgfsetbuttcap%
\pgfsetroundjoin%
\definecolor{currentfill}{rgb}{0.121569,0.466667,0.705882}%
\pgfsetfillcolor{currentfill}%
\pgfsetfillopacity{0.368105}%
\pgfsetlinewidth{1.003750pt}%
\definecolor{currentstroke}{rgb}{0.121569,0.466667,0.705882}%
\pgfsetstrokecolor{currentstroke}%
\pgfsetstrokeopacity{0.368105}%
\pgfsetdash{}{0pt}%
\pgfpathmoveto{\pgfqpoint{1.204529in}{1.738628in}}%
\pgfpathcurveto{\pgfqpoint{1.212765in}{1.738628in}}{\pgfqpoint{1.220665in}{1.741901in}}{\pgfqpoint{1.226489in}{1.747724in}}%
\pgfpathcurveto{\pgfqpoint{1.232313in}{1.753548in}}{\pgfqpoint{1.235585in}{1.761448in}}{\pgfqpoint{1.235585in}{1.769685in}}%
\pgfpathcurveto{\pgfqpoint{1.235585in}{1.777921in}}{\pgfqpoint{1.232313in}{1.785821in}}{\pgfqpoint{1.226489in}{1.791645in}}%
\pgfpathcurveto{\pgfqpoint{1.220665in}{1.797469in}}{\pgfqpoint{1.212765in}{1.800741in}}{\pgfqpoint{1.204529in}{1.800741in}}%
\pgfpathcurveto{\pgfqpoint{1.196292in}{1.800741in}}{\pgfqpoint{1.188392in}{1.797469in}}{\pgfqpoint{1.182568in}{1.791645in}}%
\pgfpathcurveto{\pgfqpoint{1.176744in}{1.785821in}}{\pgfqpoint{1.173472in}{1.777921in}}{\pgfqpoint{1.173472in}{1.769685in}}%
\pgfpathcurveto{\pgfqpoint{1.173472in}{1.761448in}}{\pgfqpoint{1.176744in}{1.753548in}}{\pgfqpoint{1.182568in}{1.747724in}}%
\pgfpathcurveto{\pgfqpoint{1.188392in}{1.741901in}}{\pgfqpoint{1.196292in}{1.738628in}}{\pgfqpoint{1.204529in}{1.738628in}}%
\pgfpathclose%
\pgfusepath{stroke,fill}%
\end{pgfscope}%
\begin{pgfscope}%
\pgfpathrectangle{\pgfqpoint{0.100000in}{0.212622in}}{\pgfqpoint{3.696000in}{3.696000in}}%
\pgfusepath{clip}%
\pgfsetbuttcap%
\pgfsetroundjoin%
\definecolor{currentfill}{rgb}{0.121569,0.466667,0.705882}%
\pgfsetfillcolor{currentfill}%
\pgfsetfillopacity{0.368105}%
\pgfsetlinewidth{1.003750pt}%
\definecolor{currentstroke}{rgb}{0.121569,0.466667,0.705882}%
\pgfsetstrokecolor{currentstroke}%
\pgfsetstrokeopacity{0.368105}%
\pgfsetdash{}{0pt}%
\pgfpathmoveto{\pgfqpoint{1.204529in}{1.738628in}}%
\pgfpathcurveto{\pgfqpoint{1.212765in}{1.738628in}}{\pgfqpoint{1.220665in}{1.741901in}}{\pgfqpoint{1.226489in}{1.747724in}}%
\pgfpathcurveto{\pgfqpoint{1.232313in}{1.753548in}}{\pgfqpoint{1.235585in}{1.761448in}}{\pgfqpoint{1.235585in}{1.769685in}}%
\pgfpathcurveto{\pgfqpoint{1.235585in}{1.777921in}}{\pgfqpoint{1.232313in}{1.785821in}}{\pgfqpoint{1.226489in}{1.791645in}}%
\pgfpathcurveto{\pgfqpoint{1.220665in}{1.797469in}}{\pgfqpoint{1.212765in}{1.800741in}}{\pgfqpoint{1.204529in}{1.800741in}}%
\pgfpathcurveto{\pgfqpoint{1.196292in}{1.800741in}}{\pgfqpoint{1.188392in}{1.797469in}}{\pgfqpoint{1.182568in}{1.791645in}}%
\pgfpathcurveto{\pgfqpoint{1.176744in}{1.785821in}}{\pgfqpoint{1.173472in}{1.777921in}}{\pgfqpoint{1.173472in}{1.769685in}}%
\pgfpathcurveto{\pgfqpoint{1.173472in}{1.761448in}}{\pgfqpoint{1.176744in}{1.753548in}}{\pgfqpoint{1.182568in}{1.747724in}}%
\pgfpathcurveto{\pgfqpoint{1.188392in}{1.741901in}}{\pgfqpoint{1.196292in}{1.738628in}}{\pgfqpoint{1.204529in}{1.738628in}}%
\pgfpathclose%
\pgfusepath{stroke,fill}%
\end{pgfscope}%
\begin{pgfscope}%
\pgfpathrectangle{\pgfqpoint{0.100000in}{0.212622in}}{\pgfqpoint{3.696000in}{3.696000in}}%
\pgfusepath{clip}%
\pgfsetbuttcap%
\pgfsetroundjoin%
\definecolor{currentfill}{rgb}{0.121569,0.466667,0.705882}%
\pgfsetfillcolor{currentfill}%
\pgfsetfillopacity{0.368105}%
\pgfsetlinewidth{1.003750pt}%
\definecolor{currentstroke}{rgb}{0.121569,0.466667,0.705882}%
\pgfsetstrokecolor{currentstroke}%
\pgfsetstrokeopacity{0.368105}%
\pgfsetdash{}{0pt}%
\pgfpathmoveto{\pgfqpoint{1.204529in}{1.738628in}}%
\pgfpathcurveto{\pgfqpoint{1.212765in}{1.738628in}}{\pgfqpoint{1.220665in}{1.741901in}}{\pgfqpoint{1.226489in}{1.747724in}}%
\pgfpathcurveto{\pgfqpoint{1.232313in}{1.753548in}}{\pgfqpoint{1.235585in}{1.761448in}}{\pgfqpoint{1.235585in}{1.769685in}}%
\pgfpathcurveto{\pgfqpoint{1.235585in}{1.777921in}}{\pgfqpoint{1.232313in}{1.785821in}}{\pgfqpoint{1.226489in}{1.791645in}}%
\pgfpathcurveto{\pgfqpoint{1.220665in}{1.797469in}}{\pgfqpoint{1.212765in}{1.800741in}}{\pgfqpoint{1.204529in}{1.800741in}}%
\pgfpathcurveto{\pgfqpoint{1.196292in}{1.800741in}}{\pgfqpoint{1.188392in}{1.797469in}}{\pgfqpoint{1.182568in}{1.791645in}}%
\pgfpathcurveto{\pgfqpoint{1.176744in}{1.785821in}}{\pgfqpoint{1.173472in}{1.777921in}}{\pgfqpoint{1.173472in}{1.769685in}}%
\pgfpathcurveto{\pgfqpoint{1.173472in}{1.761448in}}{\pgfqpoint{1.176744in}{1.753548in}}{\pgfqpoint{1.182568in}{1.747724in}}%
\pgfpathcurveto{\pgfqpoint{1.188392in}{1.741901in}}{\pgfqpoint{1.196292in}{1.738628in}}{\pgfqpoint{1.204529in}{1.738628in}}%
\pgfpathclose%
\pgfusepath{stroke,fill}%
\end{pgfscope}%
\begin{pgfscope}%
\pgfpathrectangle{\pgfqpoint{0.100000in}{0.212622in}}{\pgfqpoint{3.696000in}{3.696000in}}%
\pgfusepath{clip}%
\pgfsetbuttcap%
\pgfsetroundjoin%
\definecolor{currentfill}{rgb}{0.121569,0.466667,0.705882}%
\pgfsetfillcolor{currentfill}%
\pgfsetfillopacity{0.368105}%
\pgfsetlinewidth{1.003750pt}%
\definecolor{currentstroke}{rgb}{0.121569,0.466667,0.705882}%
\pgfsetstrokecolor{currentstroke}%
\pgfsetstrokeopacity{0.368105}%
\pgfsetdash{}{0pt}%
\pgfpathmoveto{\pgfqpoint{1.204529in}{1.738628in}}%
\pgfpathcurveto{\pgfqpoint{1.212765in}{1.738628in}}{\pgfqpoint{1.220665in}{1.741901in}}{\pgfqpoint{1.226489in}{1.747724in}}%
\pgfpathcurveto{\pgfqpoint{1.232313in}{1.753548in}}{\pgfqpoint{1.235585in}{1.761448in}}{\pgfqpoint{1.235585in}{1.769685in}}%
\pgfpathcurveto{\pgfqpoint{1.235585in}{1.777921in}}{\pgfqpoint{1.232313in}{1.785821in}}{\pgfqpoint{1.226489in}{1.791645in}}%
\pgfpathcurveto{\pgfqpoint{1.220665in}{1.797469in}}{\pgfqpoint{1.212765in}{1.800741in}}{\pgfqpoint{1.204529in}{1.800741in}}%
\pgfpathcurveto{\pgfqpoint{1.196292in}{1.800741in}}{\pgfqpoint{1.188392in}{1.797469in}}{\pgfqpoint{1.182568in}{1.791645in}}%
\pgfpathcurveto{\pgfqpoint{1.176744in}{1.785821in}}{\pgfqpoint{1.173472in}{1.777921in}}{\pgfqpoint{1.173472in}{1.769685in}}%
\pgfpathcurveto{\pgfqpoint{1.173472in}{1.761448in}}{\pgfqpoint{1.176744in}{1.753548in}}{\pgfqpoint{1.182568in}{1.747724in}}%
\pgfpathcurveto{\pgfqpoint{1.188392in}{1.741901in}}{\pgfqpoint{1.196292in}{1.738628in}}{\pgfqpoint{1.204529in}{1.738628in}}%
\pgfpathclose%
\pgfusepath{stroke,fill}%
\end{pgfscope}%
\begin{pgfscope}%
\pgfpathrectangle{\pgfqpoint{0.100000in}{0.212622in}}{\pgfqpoint{3.696000in}{3.696000in}}%
\pgfusepath{clip}%
\pgfsetbuttcap%
\pgfsetroundjoin%
\definecolor{currentfill}{rgb}{0.121569,0.466667,0.705882}%
\pgfsetfillcolor{currentfill}%
\pgfsetfillopacity{0.368105}%
\pgfsetlinewidth{1.003750pt}%
\definecolor{currentstroke}{rgb}{0.121569,0.466667,0.705882}%
\pgfsetstrokecolor{currentstroke}%
\pgfsetstrokeopacity{0.368105}%
\pgfsetdash{}{0pt}%
\pgfpathmoveto{\pgfqpoint{1.204529in}{1.738628in}}%
\pgfpathcurveto{\pgfqpoint{1.212765in}{1.738628in}}{\pgfqpoint{1.220665in}{1.741901in}}{\pgfqpoint{1.226489in}{1.747724in}}%
\pgfpathcurveto{\pgfqpoint{1.232313in}{1.753548in}}{\pgfqpoint{1.235585in}{1.761448in}}{\pgfqpoint{1.235585in}{1.769685in}}%
\pgfpathcurveto{\pgfqpoint{1.235585in}{1.777921in}}{\pgfqpoint{1.232313in}{1.785821in}}{\pgfqpoint{1.226489in}{1.791645in}}%
\pgfpathcurveto{\pgfqpoint{1.220665in}{1.797469in}}{\pgfqpoint{1.212765in}{1.800741in}}{\pgfqpoint{1.204529in}{1.800741in}}%
\pgfpathcurveto{\pgfqpoint{1.196292in}{1.800741in}}{\pgfqpoint{1.188392in}{1.797469in}}{\pgfqpoint{1.182568in}{1.791645in}}%
\pgfpathcurveto{\pgfqpoint{1.176744in}{1.785821in}}{\pgfqpoint{1.173472in}{1.777921in}}{\pgfqpoint{1.173472in}{1.769685in}}%
\pgfpathcurveto{\pgfqpoint{1.173472in}{1.761448in}}{\pgfqpoint{1.176744in}{1.753548in}}{\pgfqpoint{1.182568in}{1.747724in}}%
\pgfpathcurveto{\pgfqpoint{1.188392in}{1.741901in}}{\pgfqpoint{1.196292in}{1.738628in}}{\pgfqpoint{1.204529in}{1.738628in}}%
\pgfpathclose%
\pgfusepath{stroke,fill}%
\end{pgfscope}%
\begin{pgfscope}%
\pgfpathrectangle{\pgfqpoint{0.100000in}{0.212622in}}{\pgfqpoint{3.696000in}{3.696000in}}%
\pgfusepath{clip}%
\pgfsetbuttcap%
\pgfsetroundjoin%
\definecolor{currentfill}{rgb}{0.121569,0.466667,0.705882}%
\pgfsetfillcolor{currentfill}%
\pgfsetfillopacity{0.368105}%
\pgfsetlinewidth{1.003750pt}%
\definecolor{currentstroke}{rgb}{0.121569,0.466667,0.705882}%
\pgfsetstrokecolor{currentstroke}%
\pgfsetstrokeopacity{0.368105}%
\pgfsetdash{}{0pt}%
\pgfpathmoveto{\pgfqpoint{1.204529in}{1.738628in}}%
\pgfpathcurveto{\pgfqpoint{1.212765in}{1.738628in}}{\pgfqpoint{1.220665in}{1.741901in}}{\pgfqpoint{1.226489in}{1.747724in}}%
\pgfpathcurveto{\pgfqpoint{1.232313in}{1.753548in}}{\pgfqpoint{1.235585in}{1.761448in}}{\pgfqpoint{1.235585in}{1.769685in}}%
\pgfpathcurveto{\pgfqpoint{1.235585in}{1.777921in}}{\pgfqpoint{1.232313in}{1.785821in}}{\pgfqpoint{1.226489in}{1.791645in}}%
\pgfpathcurveto{\pgfqpoint{1.220665in}{1.797469in}}{\pgfqpoint{1.212765in}{1.800741in}}{\pgfqpoint{1.204529in}{1.800741in}}%
\pgfpathcurveto{\pgfqpoint{1.196292in}{1.800741in}}{\pgfqpoint{1.188392in}{1.797469in}}{\pgfqpoint{1.182568in}{1.791645in}}%
\pgfpathcurveto{\pgfqpoint{1.176744in}{1.785821in}}{\pgfqpoint{1.173472in}{1.777921in}}{\pgfqpoint{1.173472in}{1.769685in}}%
\pgfpathcurveto{\pgfqpoint{1.173472in}{1.761448in}}{\pgfqpoint{1.176744in}{1.753548in}}{\pgfqpoint{1.182568in}{1.747724in}}%
\pgfpathcurveto{\pgfqpoint{1.188392in}{1.741901in}}{\pgfqpoint{1.196292in}{1.738628in}}{\pgfqpoint{1.204529in}{1.738628in}}%
\pgfpathclose%
\pgfusepath{stroke,fill}%
\end{pgfscope}%
\begin{pgfscope}%
\pgfpathrectangle{\pgfqpoint{0.100000in}{0.212622in}}{\pgfqpoint{3.696000in}{3.696000in}}%
\pgfusepath{clip}%
\pgfsetbuttcap%
\pgfsetroundjoin%
\definecolor{currentfill}{rgb}{0.121569,0.466667,0.705882}%
\pgfsetfillcolor{currentfill}%
\pgfsetfillopacity{0.368105}%
\pgfsetlinewidth{1.003750pt}%
\definecolor{currentstroke}{rgb}{0.121569,0.466667,0.705882}%
\pgfsetstrokecolor{currentstroke}%
\pgfsetstrokeopacity{0.368105}%
\pgfsetdash{}{0pt}%
\pgfpathmoveto{\pgfqpoint{1.204529in}{1.738628in}}%
\pgfpathcurveto{\pgfqpoint{1.212765in}{1.738628in}}{\pgfqpoint{1.220665in}{1.741901in}}{\pgfqpoint{1.226489in}{1.747724in}}%
\pgfpathcurveto{\pgfqpoint{1.232313in}{1.753548in}}{\pgfqpoint{1.235585in}{1.761448in}}{\pgfqpoint{1.235585in}{1.769685in}}%
\pgfpathcurveto{\pgfqpoint{1.235585in}{1.777921in}}{\pgfqpoint{1.232313in}{1.785821in}}{\pgfqpoint{1.226489in}{1.791645in}}%
\pgfpathcurveto{\pgfqpoint{1.220665in}{1.797469in}}{\pgfqpoint{1.212765in}{1.800741in}}{\pgfqpoint{1.204529in}{1.800741in}}%
\pgfpathcurveto{\pgfqpoint{1.196292in}{1.800741in}}{\pgfqpoint{1.188392in}{1.797469in}}{\pgfqpoint{1.182568in}{1.791645in}}%
\pgfpathcurveto{\pgfqpoint{1.176744in}{1.785821in}}{\pgfqpoint{1.173472in}{1.777921in}}{\pgfqpoint{1.173472in}{1.769685in}}%
\pgfpathcurveto{\pgfqpoint{1.173472in}{1.761448in}}{\pgfqpoint{1.176744in}{1.753548in}}{\pgfqpoint{1.182568in}{1.747724in}}%
\pgfpathcurveto{\pgfqpoint{1.188392in}{1.741901in}}{\pgfqpoint{1.196292in}{1.738628in}}{\pgfqpoint{1.204529in}{1.738628in}}%
\pgfpathclose%
\pgfusepath{stroke,fill}%
\end{pgfscope}%
\begin{pgfscope}%
\pgfpathrectangle{\pgfqpoint{0.100000in}{0.212622in}}{\pgfqpoint{3.696000in}{3.696000in}}%
\pgfusepath{clip}%
\pgfsetbuttcap%
\pgfsetroundjoin%
\definecolor{currentfill}{rgb}{0.121569,0.466667,0.705882}%
\pgfsetfillcolor{currentfill}%
\pgfsetfillopacity{0.368105}%
\pgfsetlinewidth{1.003750pt}%
\definecolor{currentstroke}{rgb}{0.121569,0.466667,0.705882}%
\pgfsetstrokecolor{currentstroke}%
\pgfsetstrokeopacity{0.368105}%
\pgfsetdash{}{0pt}%
\pgfpathmoveto{\pgfqpoint{1.204529in}{1.738628in}}%
\pgfpathcurveto{\pgfqpoint{1.212765in}{1.738628in}}{\pgfqpoint{1.220665in}{1.741901in}}{\pgfqpoint{1.226489in}{1.747724in}}%
\pgfpathcurveto{\pgfqpoint{1.232313in}{1.753548in}}{\pgfqpoint{1.235585in}{1.761448in}}{\pgfqpoint{1.235585in}{1.769685in}}%
\pgfpathcurveto{\pgfqpoint{1.235585in}{1.777921in}}{\pgfqpoint{1.232313in}{1.785821in}}{\pgfqpoint{1.226489in}{1.791645in}}%
\pgfpathcurveto{\pgfqpoint{1.220665in}{1.797469in}}{\pgfqpoint{1.212765in}{1.800741in}}{\pgfqpoint{1.204529in}{1.800741in}}%
\pgfpathcurveto{\pgfqpoint{1.196292in}{1.800741in}}{\pgfqpoint{1.188392in}{1.797469in}}{\pgfqpoint{1.182568in}{1.791645in}}%
\pgfpathcurveto{\pgfqpoint{1.176744in}{1.785821in}}{\pgfqpoint{1.173472in}{1.777921in}}{\pgfqpoint{1.173472in}{1.769685in}}%
\pgfpathcurveto{\pgfqpoint{1.173472in}{1.761448in}}{\pgfqpoint{1.176744in}{1.753548in}}{\pgfqpoint{1.182568in}{1.747724in}}%
\pgfpathcurveto{\pgfqpoint{1.188392in}{1.741901in}}{\pgfqpoint{1.196292in}{1.738628in}}{\pgfqpoint{1.204529in}{1.738628in}}%
\pgfpathclose%
\pgfusepath{stroke,fill}%
\end{pgfscope}%
\begin{pgfscope}%
\pgfpathrectangle{\pgfqpoint{0.100000in}{0.212622in}}{\pgfqpoint{3.696000in}{3.696000in}}%
\pgfusepath{clip}%
\pgfsetbuttcap%
\pgfsetroundjoin%
\definecolor{currentfill}{rgb}{0.121569,0.466667,0.705882}%
\pgfsetfillcolor{currentfill}%
\pgfsetfillopacity{0.368105}%
\pgfsetlinewidth{1.003750pt}%
\definecolor{currentstroke}{rgb}{0.121569,0.466667,0.705882}%
\pgfsetstrokecolor{currentstroke}%
\pgfsetstrokeopacity{0.368105}%
\pgfsetdash{}{0pt}%
\pgfpathmoveto{\pgfqpoint{1.204529in}{1.738628in}}%
\pgfpathcurveto{\pgfqpoint{1.212765in}{1.738628in}}{\pgfqpoint{1.220665in}{1.741901in}}{\pgfqpoint{1.226489in}{1.747724in}}%
\pgfpathcurveto{\pgfqpoint{1.232313in}{1.753548in}}{\pgfqpoint{1.235585in}{1.761448in}}{\pgfqpoint{1.235585in}{1.769685in}}%
\pgfpathcurveto{\pgfqpoint{1.235585in}{1.777921in}}{\pgfqpoint{1.232313in}{1.785821in}}{\pgfqpoint{1.226489in}{1.791645in}}%
\pgfpathcurveto{\pgfqpoint{1.220665in}{1.797469in}}{\pgfqpoint{1.212765in}{1.800741in}}{\pgfqpoint{1.204529in}{1.800741in}}%
\pgfpathcurveto{\pgfqpoint{1.196292in}{1.800741in}}{\pgfqpoint{1.188392in}{1.797469in}}{\pgfqpoint{1.182568in}{1.791645in}}%
\pgfpathcurveto{\pgfqpoint{1.176744in}{1.785821in}}{\pgfqpoint{1.173472in}{1.777921in}}{\pgfqpoint{1.173472in}{1.769685in}}%
\pgfpathcurveto{\pgfqpoint{1.173472in}{1.761448in}}{\pgfqpoint{1.176744in}{1.753548in}}{\pgfqpoint{1.182568in}{1.747724in}}%
\pgfpathcurveto{\pgfqpoint{1.188392in}{1.741901in}}{\pgfqpoint{1.196292in}{1.738628in}}{\pgfqpoint{1.204529in}{1.738628in}}%
\pgfpathclose%
\pgfusepath{stroke,fill}%
\end{pgfscope}%
\begin{pgfscope}%
\pgfpathrectangle{\pgfqpoint{0.100000in}{0.212622in}}{\pgfqpoint{3.696000in}{3.696000in}}%
\pgfusepath{clip}%
\pgfsetbuttcap%
\pgfsetroundjoin%
\definecolor{currentfill}{rgb}{0.121569,0.466667,0.705882}%
\pgfsetfillcolor{currentfill}%
\pgfsetfillopacity{0.368105}%
\pgfsetlinewidth{1.003750pt}%
\definecolor{currentstroke}{rgb}{0.121569,0.466667,0.705882}%
\pgfsetstrokecolor{currentstroke}%
\pgfsetstrokeopacity{0.368105}%
\pgfsetdash{}{0pt}%
\pgfpathmoveto{\pgfqpoint{1.204529in}{1.738628in}}%
\pgfpathcurveto{\pgfqpoint{1.212765in}{1.738628in}}{\pgfqpoint{1.220665in}{1.741901in}}{\pgfqpoint{1.226489in}{1.747724in}}%
\pgfpathcurveto{\pgfqpoint{1.232313in}{1.753548in}}{\pgfqpoint{1.235585in}{1.761448in}}{\pgfqpoint{1.235585in}{1.769685in}}%
\pgfpathcurveto{\pgfqpoint{1.235585in}{1.777921in}}{\pgfqpoint{1.232313in}{1.785821in}}{\pgfqpoint{1.226489in}{1.791645in}}%
\pgfpathcurveto{\pgfqpoint{1.220665in}{1.797469in}}{\pgfqpoint{1.212765in}{1.800741in}}{\pgfqpoint{1.204529in}{1.800741in}}%
\pgfpathcurveto{\pgfqpoint{1.196292in}{1.800741in}}{\pgfqpoint{1.188392in}{1.797469in}}{\pgfqpoint{1.182568in}{1.791645in}}%
\pgfpathcurveto{\pgfqpoint{1.176744in}{1.785821in}}{\pgfqpoint{1.173472in}{1.777921in}}{\pgfqpoint{1.173472in}{1.769685in}}%
\pgfpathcurveto{\pgfqpoint{1.173472in}{1.761448in}}{\pgfqpoint{1.176744in}{1.753548in}}{\pgfqpoint{1.182568in}{1.747724in}}%
\pgfpathcurveto{\pgfqpoint{1.188392in}{1.741901in}}{\pgfqpoint{1.196292in}{1.738628in}}{\pgfqpoint{1.204529in}{1.738628in}}%
\pgfpathclose%
\pgfusepath{stroke,fill}%
\end{pgfscope}%
\begin{pgfscope}%
\pgfpathrectangle{\pgfqpoint{0.100000in}{0.212622in}}{\pgfqpoint{3.696000in}{3.696000in}}%
\pgfusepath{clip}%
\pgfsetbuttcap%
\pgfsetroundjoin%
\definecolor{currentfill}{rgb}{0.121569,0.466667,0.705882}%
\pgfsetfillcolor{currentfill}%
\pgfsetfillopacity{0.368105}%
\pgfsetlinewidth{1.003750pt}%
\definecolor{currentstroke}{rgb}{0.121569,0.466667,0.705882}%
\pgfsetstrokecolor{currentstroke}%
\pgfsetstrokeopacity{0.368105}%
\pgfsetdash{}{0pt}%
\pgfpathmoveto{\pgfqpoint{1.204529in}{1.738628in}}%
\pgfpathcurveto{\pgfqpoint{1.212765in}{1.738628in}}{\pgfqpoint{1.220665in}{1.741901in}}{\pgfqpoint{1.226489in}{1.747724in}}%
\pgfpathcurveto{\pgfqpoint{1.232313in}{1.753548in}}{\pgfqpoint{1.235585in}{1.761448in}}{\pgfqpoint{1.235585in}{1.769685in}}%
\pgfpathcurveto{\pgfqpoint{1.235585in}{1.777921in}}{\pgfqpoint{1.232313in}{1.785821in}}{\pgfqpoint{1.226489in}{1.791645in}}%
\pgfpathcurveto{\pgfqpoint{1.220665in}{1.797469in}}{\pgfqpoint{1.212765in}{1.800741in}}{\pgfqpoint{1.204529in}{1.800741in}}%
\pgfpathcurveto{\pgfqpoint{1.196292in}{1.800741in}}{\pgfqpoint{1.188392in}{1.797469in}}{\pgfqpoint{1.182568in}{1.791645in}}%
\pgfpathcurveto{\pgfqpoint{1.176744in}{1.785821in}}{\pgfqpoint{1.173472in}{1.777921in}}{\pgfqpoint{1.173472in}{1.769685in}}%
\pgfpathcurveto{\pgfqpoint{1.173472in}{1.761448in}}{\pgfqpoint{1.176744in}{1.753548in}}{\pgfqpoint{1.182568in}{1.747724in}}%
\pgfpathcurveto{\pgfqpoint{1.188392in}{1.741901in}}{\pgfqpoint{1.196292in}{1.738628in}}{\pgfqpoint{1.204529in}{1.738628in}}%
\pgfpathclose%
\pgfusepath{stroke,fill}%
\end{pgfscope}%
\begin{pgfscope}%
\pgfpathrectangle{\pgfqpoint{0.100000in}{0.212622in}}{\pgfqpoint{3.696000in}{3.696000in}}%
\pgfusepath{clip}%
\pgfsetbuttcap%
\pgfsetroundjoin%
\definecolor{currentfill}{rgb}{0.121569,0.466667,0.705882}%
\pgfsetfillcolor{currentfill}%
\pgfsetfillopacity{0.368105}%
\pgfsetlinewidth{1.003750pt}%
\definecolor{currentstroke}{rgb}{0.121569,0.466667,0.705882}%
\pgfsetstrokecolor{currentstroke}%
\pgfsetstrokeopacity{0.368105}%
\pgfsetdash{}{0pt}%
\pgfpathmoveto{\pgfqpoint{1.204529in}{1.738628in}}%
\pgfpathcurveto{\pgfqpoint{1.212765in}{1.738628in}}{\pgfqpoint{1.220665in}{1.741901in}}{\pgfqpoint{1.226489in}{1.747724in}}%
\pgfpathcurveto{\pgfqpoint{1.232313in}{1.753548in}}{\pgfqpoint{1.235585in}{1.761448in}}{\pgfqpoint{1.235585in}{1.769685in}}%
\pgfpathcurveto{\pgfqpoint{1.235585in}{1.777921in}}{\pgfqpoint{1.232313in}{1.785821in}}{\pgfqpoint{1.226489in}{1.791645in}}%
\pgfpathcurveto{\pgfqpoint{1.220665in}{1.797469in}}{\pgfqpoint{1.212765in}{1.800741in}}{\pgfqpoint{1.204529in}{1.800741in}}%
\pgfpathcurveto{\pgfqpoint{1.196292in}{1.800741in}}{\pgfqpoint{1.188392in}{1.797469in}}{\pgfqpoint{1.182568in}{1.791645in}}%
\pgfpathcurveto{\pgfqpoint{1.176744in}{1.785821in}}{\pgfqpoint{1.173472in}{1.777921in}}{\pgfqpoint{1.173472in}{1.769685in}}%
\pgfpathcurveto{\pgfqpoint{1.173472in}{1.761448in}}{\pgfqpoint{1.176744in}{1.753548in}}{\pgfqpoint{1.182568in}{1.747724in}}%
\pgfpathcurveto{\pgfqpoint{1.188392in}{1.741901in}}{\pgfqpoint{1.196292in}{1.738628in}}{\pgfqpoint{1.204529in}{1.738628in}}%
\pgfpathclose%
\pgfusepath{stroke,fill}%
\end{pgfscope}%
\begin{pgfscope}%
\pgfpathrectangle{\pgfqpoint{0.100000in}{0.212622in}}{\pgfqpoint{3.696000in}{3.696000in}}%
\pgfusepath{clip}%
\pgfsetbuttcap%
\pgfsetroundjoin%
\definecolor{currentfill}{rgb}{0.121569,0.466667,0.705882}%
\pgfsetfillcolor{currentfill}%
\pgfsetfillopacity{0.368105}%
\pgfsetlinewidth{1.003750pt}%
\definecolor{currentstroke}{rgb}{0.121569,0.466667,0.705882}%
\pgfsetstrokecolor{currentstroke}%
\pgfsetstrokeopacity{0.368105}%
\pgfsetdash{}{0pt}%
\pgfpathmoveto{\pgfqpoint{1.204529in}{1.738628in}}%
\pgfpathcurveto{\pgfqpoint{1.212765in}{1.738628in}}{\pgfqpoint{1.220665in}{1.741901in}}{\pgfqpoint{1.226489in}{1.747724in}}%
\pgfpathcurveto{\pgfqpoint{1.232313in}{1.753548in}}{\pgfqpoint{1.235585in}{1.761448in}}{\pgfqpoint{1.235585in}{1.769685in}}%
\pgfpathcurveto{\pgfqpoint{1.235585in}{1.777921in}}{\pgfqpoint{1.232313in}{1.785821in}}{\pgfqpoint{1.226489in}{1.791645in}}%
\pgfpathcurveto{\pgfqpoint{1.220665in}{1.797469in}}{\pgfqpoint{1.212765in}{1.800741in}}{\pgfqpoint{1.204529in}{1.800741in}}%
\pgfpathcurveto{\pgfqpoint{1.196292in}{1.800741in}}{\pgfqpoint{1.188392in}{1.797469in}}{\pgfqpoint{1.182568in}{1.791645in}}%
\pgfpathcurveto{\pgfqpoint{1.176744in}{1.785821in}}{\pgfqpoint{1.173472in}{1.777921in}}{\pgfqpoint{1.173472in}{1.769685in}}%
\pgfpathcurveto{\pgfqpoint{1.173472in}{1.761448in}}{\pgfqpoint{1.176744in}{1.753548in}}{\pgfqpoint{1.182568in}{1.747724in}}%
\pgfpathcurveto{\pgfqpoint{1.188392in}{1.741901in}}{\pgfqpoint{1.196292in}{1.738628in}}{\pgfqpoint{1.204529in}{1.738628in}}%
\pgfpathclose%
\pgfusepath{stroke,fill}%
\end{pgfscope}%
\begin{pgfscope}%
\pgfpathrectangle{\pgfqpoint{0.100000in}{0.212622in}}{\pgfqpoint{3.696000in}{3.696000in}}%
\pgfusepath{clip}%
\pgfsetbuttcap%
\pgfsetroundjoin%
\definecolor{currentfill}{rgb}{0.121569,0.466667,0.705882}%
\pgfsetfillcolor{currentfill}%
\pgfsetfillopacity{0.368105}%
\pgfsetlinewidth{1.003750pt}%
\definecolor{currentstroke}{rgb}{0.121569,0.466667,0.705882}%
\pgfsetstrokecolor{currentstroke}%
\pgfsetstrokeopacity{0.368105}%
\pgfsetdash{}{0pt}%
\pgfpathmoveto{\pgfqpoint{1.204529in}{1.738628in}}%
\pgfpathcurveto{\pgfqpoint{1.212765in}{1.738628in}}{\pgfqpoint{1.220665in}{1.741901in}}{\pgfqpoint{1.226489in}{1.747724in}}%
\pgfpathcurveto{\pgfqpoint{1.232313in}{1.753548in}}{\pgfqpoint{1.235585in}{1.761448in}}{\pgfqpoint{1.235585in}{1.769685in}}%
\pgfpathcurveto{\pgfqpoint{1.235585in}{1.777921in}}{\pgfqpoint{1.232313in}{1.785821in}}{\pgfqpoint{1.226489in}{1.791645in}}%
\pgfpathcurveto{\pgfqpoint{1.220665in}{1.797469in}}{\pgfqpoint{1.212765in}{1.800741in}}{\pgfqpoint{1.204529in}{1.800741in}}%
\pgfpathcurveto{\pgfqpoint{1.196292in}{1.800741in}}{\pgfqpoint{1.188392in}{1.797469in}}{\pgfqpoint{1.182568in}{1.791645in}}%
\pgfpathcurveto{\pgfqpoint{1.176744in}{1.785821in}}{\pgfqpoint{1.173472in}{1.777921in}}{\pgfqpoint{1.173472in}{1.769685in}}%
\pgfpathcurveto{\pgfqpoint{1.173472in}{1.761448in}}{\pgfqpoint{1.176744in}{1.753548in}}{\pgfqpoint{1.182568in}{1.747724in}}%
\pgfpathcurveto{\pgfqpoint{1.188392in}{1.741901in}}{\pgfqpoint{1.196292in}{1.738628in}}{\pgfqpoint{1.204529in}{1.738628in}}%
\pgfpathclose%
\pgfusepath{stroke,fill}%
\end{pgfscope}%
\begin{pgfscope}%
\pgfpathrectangle{\pgfqpoint{0.100000in}{0.212622in}}{\pgfqpoint{3.696000in}{3.696000in}}%
\pgfusepath{clip}%
\pgfsetbuttcap%
\pgfsetroundjoin%
\definecolor{currentfill}{rgb}{0.121569,0.466667,0.705882}%
\pgfsetfillcolor{currentfill}%
\pgfsetfillopacity{0.368105}%
\pgfsetlinewidth{1.003750pt}%
\definecolor{currentstroke}{rgb}{0.121569,0.466667,0.705882}%
\pgfsetstrokecolor{currentstroke}%
\pgfsetstrokeopacity{0.368105}%
\pgfsetdash{}{0pt}%
\pgfpathmoveto{\pgfqpoint{1.204529in}{1.738628in}}%
\pgfpathcurveto{\pgfqpoint{1.212765in}{1.738628in}}{\pgfqpoint{1.220665in}{1.741901in}}{\pgfqpoint{1.226489in}{1.747724in}}%
\pgfpathcurveto{\pgfqpoint{1.232313in}{1.753548in}}{\pgfqpoint{1.235585in}{1.761448in}}{\pgfqpoint{1.235585in}{1.769685in}}%
\pgfpathcurveto{\pgfqpoint{1.235585in}{1.777921in}}{\pgfqpoint{1.232313in}{1.785821in}}{\pgfqpoint{1.226489in}{1.791645in}}%
\pgfpathcurveto{\pgfqpoint{1.220665in}{1.797469in}}{\pgfqpoint{1.212765in}{1.800741in}}{\pgfqpoint{1.204529in}{1.800741in}}%
\pgfpathcurveto{\pgfqpoint{1.196292in}{1.800741in}}{\pgfqpoint{1.188392in}{1.797469in}}{\pgfqpoint{1.182568in}{1.791645in}}%
\pgfpathcurveto{\pgfqpoint{1.176744in}{1.785821in}}{\pgfqpoint{1.173472in}{1.777921in}}{\pgfqpoint{1.173472in}{1.769685in}}%
\pgfpathcurveto{\pgfqpoint{1.173472in}{1.761448in}}{\pgfqpoint{1.176744in}{1.753548in}}{\pgfqpoint{1.182568in}{1.747724in}}%
\pgfpathcurveto{\pgfqpoint{1.188392in}{1.741901in}}{\pgfqpoint{1.196292in}{1.738628in}}{\pgfqpoint{1.204529in}{1.738628in}}%
\pgfpathclose%
\pgfusepath{stroke,fill}%
\end{pgfscope}%
\begin{pgfscope}%
\pgfpathrectangle{\pgfqpoint{0.100000in}{0.212622in}}{\pgfqpoint{3.696000in}{3.696000in}}%
\pgfusepath{clip}%
\pgfsetbuttcap%
\pgfsetroundjoin%
\definecolor{currentfill}{rgb}{0.121569,0.466667,0.705882}%
\pgfsetfillcolor{currentfill}%
\pgfsetfillopacity{0.368105}%
\pgfsetlinewidth{1.003750pt}%
\definecolor{currentstroke}{rgb}{0.121569,0.466667,0.705882}%
\pgfsetstrokecolor{currentstroke}%
\pgfsetstrokeopacity{0.368105}%
\pgfsetdash{}{0pt}%
\pgfpathmoveto{\pgfqpoint{1.204529in}{1.738628in}}%
\pgfpathcurveto{\pgfqpoint{1.212765in}{1.738628in}}{\pgfqpoint{1.220665in}{1.741901in}}{\pgfqpoint{1.226489in}{1.747724in}}%
\pgfpathcurveto{\pgfqpoint{1.232313in}{1.753548in}}{\pgfqpoint{1.235585in}{1.761448in}}{\pgfqpoint{1.235585in}{1.769685in}}%
\pgfpathcurveto{\pgfqpoint{1.235585in}{1.777921in}}{\pgfqpoint{1.232313in}{1.785821in}}{\pgfqpoint{1.226489in}{1.791645in}}%
\pgfpathcurveto{\pgfqpoint{1.220665in}{1.797469in}}{\pgfqpoint{1.212765in}{1.800741in}}{\pgfqpoint{1.204529in}{1.800741in}}%
\pgfpathcurveto{\pgfqpoint{1.196292in}{1.800741in}}{\pgfqpoint{1.188392in}{1.797469in}}{\pgfqpoint{1.182568in}{1.791645in}}%
\pgfpathcurveto{\pgfqpoint{1.176744in}{1.785821in}}{\pgfqpoint{1.173472in}{1.777921in}}{\pgfqpoint{1.173472in}{1.769685in}}%
\pgfpathcurveto{\pgfqpoint{1.173472in}{1.761448in}}{\pgfqpoint{1.176744in}{1.753548in}}{\pgfqpoint{1.182568in}{1.747724in}}%
\pgfpathcurveto{\pgfqpoint{1.188392in}{1.741901in}}{\pgfqpoint{1.196292in}{1.738628in}}{\pgfqpoint{1.204529in}{1.738628in}}%
\pgfpathclose%
\pgfusepath{stroke,fill}%
\end{pgfscope}%
\begin{pgfscope}%
\pgfpathrectangle{\pgfqpoint{0.100000in}{0.212622in}}{\pgfqpoint{3.696000in}{3.696000in}}%
\pgfusepath{clip}%
\pgfsetbuttcap%
\pgfsetroundjoin%
\definecolor{currentfill}{rgb}{0.121569,0.466667,0.705882}%
\pgfsetfillcolor{currentfill}%
\pgfsetfillopacity{0.368105}%
\pgfsetlinewidth{1.003750pt}%
\definecolor{currentstroke}{rgb}{0.121569,0.466667,0.705882}%
\pgfsetstrokecolor{currentstroke}%
\pgfsetstrokeopacity{0.368105}%
\pgfsetdash{}{0pt}%
\pgfpathmoveto{\pgfqpoint{1.204529in}{1.738628in}}%
\pgfpathcurveto{\pgfqpoint{1.212765in}{1.738628in}}{\pgfqpoint{1.220665in}{1.741901in}}{\pgfqpoint{1.226489in}{1.747724in}}%
\pgfpathcurveto{\pgfqpoint{1.232313in}{1.753548in}}{\pgfqpoint{1.235585in}{1.761448in}}{\pgfqpoint{1.235585in}{1.769685in}}%
\pgfpathcurveto{\pgfqpoint{1.235585in}{1.777921in}}{\pgfqpoint{1.232313in}{1.785821in}}{\pgfqpoint{1.226489in}{1.791645in}}%
\pgfpathcurveto{\pgfqpoint{1.220665in}{1.797469in}}{\pgfqpoint{1.212765in}{1.800741in}}{\pgfqpoint{1.204529in}{1.800741in}}%
\pgfpathcurveto{\pgfqpoint{1.196292in}{1.800741in}}{\pgfqpoint{1.188392in}{1.797469in}}{\pgfqpoint{1.182568in}{1.791645in}}%
\pgfpathcurveto{\pgfqpoint{1.176744in}{1.785821in}}{\pgfqpoint{1.173472in}{1.777921in}}{\pgfqpoint{1.173472in}{1.769685in}}%
\pgfpathcurveto{\pgfqpoint{1.173472in}{1.761448in}}{\pgfqpoint{1.176744in}{1.753548in}}{\pgfqpoint{1.182568in}{1.747724in}}%
\pgfpathcurveto{\pgfqpoint{1.188392in}{1.741901in}}{\pgfqpoint{1.196292in}{1.738628in}}{\pgfqpoint{1.204529in}{1.738628in}}%
\pgfpathclose%
\pgfusepath{stroke,fill}%
\end{pgfscope}%
\begin{pgfscope}%
\pgfpathrectangle{\pgfqpoint{0.100000in}{0.212622in}}{\pgfqpoint{3.696000in}{3.696000in}}%
\pgfusepath{clip}%
\pgfsetbuttcap%
\pgfsetroundjoin%
\definecolor{currentfill}{rgb}{0.121569,0.466667,0.705882}%
\pgfsetfillcolor{currentfill}%
\pgfsetfillopacity{0.368105}%
\pgfsetlinewidth{1.003750pt}%
\definecolor{currentstroke}{rgb}{0.121569,0.466667,0.705882}%
\pgfsetstrokecolor{currentstroke}%
\pgfsetstrokeopacity{0.368105}%
\pgfsetdash{}{0pt}%
\pgfpathmoveto{\pgfqpoint{1.204529in}{1.738628in}}%
\pgfpathcurveto{\pgfqpoint{1.212765in}{1.738628in}}{\pgfqpoint{1.220665in}{1.741901in}}{\pgfqpoint{1.226489in}{1.747724in}}%
\pgfpathcurveto{\pgfqpoint{1.232313in}{1.753548in}}{\pgfqpoint{1.235585in}{1.761448in}}{\pgfqpoint{1.235585in}{1.769685in}}%
\pgfpathcurveto{\pgfqpoint{1.235585in}{1.777921in}}{\pgfqpoint{1.232313in}{1.785821in}}{\pgfqpoint{1.226489in}{1.791645in}}%
\pgfpathcurveto{\pgfqpoint{1.220665in}{1.797469in}}{\pgfqpoint{1.212765in}{1.800741in}}{\pgfqpoint{1.204529in}{1.800741in}}%
\pgfpathcurveto{\pgfqpoint{1.196292in}{1.800741in}}{\pgfqpoint{1.188392in}{1.797469in}}{\pgfqpoint{1.182568in}{1.791645in}}%
\pgfpathcurveto{\pgfqpoint{1.176744in}{1.785821in}}{\pgfqpoint{1.173472in}{1.777921in}}{\pgfqpoint{1.173472in}{1.769685in}}%
\pgfpathcurveto{\pgfqpoint{1.173472in}{1.761448in}}{\pgfqpoint{1.176744in}{1.753548in}}{\pgfqpoint{1.182568in}{1.747724in}}%
\pgfpathcurveto{\pgfqpoint{1.188392in}{1.741901in}}{\pgfqpoint{1.196292in}{1.738628in}}{\pgfqpoint{1.204529in}{1.738628in}}%
\pgfpathclose%
\pgfusepath{stroke,fill}%
\end{pgfscope}%
\begin{pgfscope}%
\pgfpathrectangle{\pgfqpoint{0.100000in}{0.212622in}}{\pgfqpoint{3.696000in}{3.696000in}}%
\pgfusepath{clip}%
\pgfsetbuttcap%
\pgfsetroundjoin%
\definecolor{currentfill}{rgb}{0.121569,0.466667,0.705882}%
\pgfsetfillcolor{currentfill}%
\pgfsetfillopacity{0.368105}%
\pgfsetlinewidth{1.003750pt}%
\definecolor{currentstroke}{rgb}{0.121569,0.466667,0.705882}%
\pgfsetstrokecolor{currentstroke}%
\pgfsetstrokeopacity{0.368105}%
\pgfsetdash{}{0pt}%
\pgfpathmoveto{\pgfqpoint{1.204529in}{1.738628in}}%
\pgfpathcurveto{\pgfqpoint{1.212765in}{1.738628in}}{\pgfqpoint{1.220665in}{1.741901in}}{\pgfqpoint{1.226489in}{1.747724in}}%
\pgfpathcurveto{\pgfqpoint{1.232313in}{1.753548in}}{\pgfqpoint{1.235585in}{1.761448in}}{\pgfqpoint{1.235585in}{1.769685in}}%
\pgfpathcurveto{\pgfqpoint{1.235585in}{1.777921in}}{\pgfqpoint{1.232313in}{1.785821in}}{\pgfqpoint{1.226489in}{1.791645in}}%
\pgfpathcurveto{\pgfqpoint{1.220665in}{1.797469in}}{\pgfqpoint{1.212765in}{1.800741in}}{\pgfqpoint{1.204529in}{1.800741in}}%
\pgfpathcurveto{\pgfqpoint{1.196292in}{1.800741in}}{\pgfqpoint{1.188392in}{1.797469in}}{\pgfqpoint{1.182568in}{1.791645in}}%
\pgfpathcurveto{\pgfqpoint{1.176744in}{1.785821in}}{\pgfqpoint{1.173472in}{1.777921in}}{\pgfqpoint{1.173472in}{1.769685in}}%
\pgfpathcurveto{\pgfqpoint{1.173472in}{1.761448in}}{\pgfqpoint{1.176744in}{1.753548in}}{\pgfqpoint{1.182568in}{1.747724in}}%
\pgfpathcurveto{\pgfqpoint{1.188392in}{1.741901in}}{\pgfqpoint{1.196292in}{1.738628in}}{\pgfqpoint{1.204529in}{1.738628in}}%
\pgfpathclose%
\pgfusepath{stroke,fill}%
\end{pgfscope}%
\begin{pgfscope}%
\pgfpathrectangle{\pgfqpoint{0.100000in}{0.212622in}}{\pgfqpoint{3.696000in}{3.696000in}}%
\pgfusepath{clip}%
\pgfsetbuttcap%
\pgfsetroundjoin%
\definecolor{currentfill}{rgb}{0.121569,0.466667,0.705882}%
\pgfsetfillcolor{currentfill}%
\pgfsetfillopacity{0.368105}%
\pgfsetlinewidth{1.003750pt}%
\definecolor{currentstroke}{rgb}{0.121569,0.466667,0.705882}%
\pgfsetstrokecolor{currentstroke}%
\pgfsetstrokeopacity{0.368105}%
\pgfsetdash{}{0pt}%
\pgfpathmoveto{\pgfqpoint{1.204529in}{1.738628in}}%
\pgfpathcurveto{\pgfqpoint{1.212765in}{1.738628in}}{\pgfqpoint{1.220665in}{1.741901in}}{\pgfqpoint{1.226489in}{1.747724in}}%
\pgfpathcurveto{\pgfqpoint{1.232313in}{1.753548in}}{\pgfqpoint{1.235585in}{1.761448in}}{\pgfqpoint{1.235585in}{1.769685in}}%
\pgfpathcurveto{\pgfqpoint{1.235585in}{1.777921in}}{\pgfqpoint{1.232313in}{1.785821in}}{\pgfqpoint{1.226489in}{1.791645in}}%
\pgfpathcurveto{\pgfqpoint{1.220665in}{1.797469in}}{\pgfqpoint{1.212765in}{1.800741in}}{\pgfqpoint{1.204529in}{1.800741in}}%
\pgfpathcurveto{\pgfqpoint{1.196292in}{1.800741in}}{\pgfqpoint{1.188392in}{1.797469in}}{\pgfqpoint{1.182568in}{1.791645in}}%
\pgfpathcurveto{\pgfqpoint{1.176744in}{1.785821in}}{\pgfqpoint{1.173472in}{1.777921in}}{\pgfqpoint{1.173472in}{1.769685in}}%
\pgfpathcurveto{\pgfqpoint{1.173472in}{1.761448in}}{\pgfqpoint{1.176744in}{1.753548in}}{\pgfqpoint{1.182568in}{1.747724in}}%
\pgfpathcurveto{\pgfqpoint{1.188392in}{1.741901in}}{\pgfqpoint{1.196292in}{1.738628in}}{\pgfqpoint{1.204529in}{1.738628in}}%
\pgfpathclose%
\pgfusepath{stroke,fill}%
\end{pgfscope}%
\begin{pgfscope}%
\pgfpathrectangle{\pgfqpoint{0.100000in}{0.212622in}}{\pgfqpoint{3.696000in}{3.696000in}}%
\pgfusepath{clip}%
\pgfsetbuttcap%
\pgfsetroundjoin%
\definecolor{currentfill}{rgb}{0.121569,0.466667,0.705882}%
\pgfsetfillcolor{currentfill}%
\pgfsetfillopacity{0.368105}%
\pgfsetlinewidth{1.003750pt}%
\definecolor{currentstroke}{rgb}{0.121569,0.466667,0.705882}%
\pgfsetstrokecolor{currentstroke}%
\pgfsetstrokeopacity{0.368105}%
\pgfsetdash{}{0pt}%
\pgfpathmoveto{\pgfqpoint{1.204529in}{1.738628in}}%
\pgfpathcurveto{\pgfqpoint{1.212765in}{1.738628in}}{\pgfqpoint{1.220665in}{1.741901in}}{\pgfqpoint{1.226489in}{1.747724in}}%
\pgfpathcurveto{\pgfqpoint{1.232313in}{1.753548in}}{\pgfqpoint{1.235585in}{1.761448in}}{\pgfqpoint{1.235585in}{1.769685in}}%
\pgfpathcurveto{\pgfqpoint{1.235585in}{1.777921in}}{\pgfqpoint{1.232313in}{1.785821in}}{\pgfqpoint{1.226489in}{1.791645in}}%
\pgfpathcurveto{\pgfqpoint{1.220665in}{1.797469in}}{\pgfqpoint{1.212765in}{1.800741in}}{\pgfqpoint{1.204529in}{1.800741in}}%
\pgfpathcurveto{\pgfqpoint{1.196292in}{1.800741in}}{\pgfqpoint{1.188392in}{1.797469in}}{\pgfqpoint{1.182568in}{1.791645in}}%
\pgfpathcurveto{\pgfqpoint{1.176744in}{1.785821in}}{\pgfqpoint{1.173472in}{1.777921in}}{\pgfqpoint{1.173472in}{1.769685in}}%
\pgfpathcurveto{\pgfqpoint{1.173472in}{1.761448in}}{\pgfqpoint{1.176744in}{1.753548in}}{\pgfqpoint{1.182568in}{1.747724in}}%
\pgfpathcurveto{\pgfqpoint{1.188392in}{1.741901in}}{\pgfqpoint{1.196292in}{1.738628in}}{\pgfqpoint{1.204529in}{1.738628in}}%
\pgfpathclose%
\pgfusepath{stroke,fill}%
\end{pgfscope}%
\begin{pgfscope}%
\pgfpathrectangle{\pgfqpoint{0.100000in}{0.212622in}}{\pgfqpoint{3.696000in}{3.696000in}}%
\pgfusepath{clip}%
\pgfsetbuttcap%
\pgfsetroundjoin%
\definecolor{currentfill}{rgb}{0.121569,0.466667,0.705882}%
\pgfsetfillcolor{currentfill}%
\pgfsetfillopacity{0.368105}%
\pgfsetlinewidth{1.003750pt}%
\definecolor{currentstroke}{rgb}{0.121569,0.466667,0.705882}%
\pgfsetstrokecolor{currentstroke}%
\pgfsetstrokeopacity{0.368105}%
\pgfsetdash{}{0pt}%
\pgfpathmoveto{\pgfqpoint{1.204529in}{1.738628in}}%
\pgfpathcurveto{\pgfqpoint{1.212765in}{1.738628in}}{\pgfqpoint{1.220665in}{1.741901in}}{\pgfqpoint{1.226489in}{1.747724in}}%
\pgfpathcurveto{\pgfqpoint{1.232313in}{1.753548in}}{\pgfqpoint{1.235585in}{1.761448in}}{\pgfqpoint{1.235585in}{1.769685in}}%
\pgfpathcurveto{\pgfqpoint{1.235585in}{1.777921in}}{\pgfqpoint{1.232313in}{1.785821in}}{\pgfqpoint{1.226489in}{1.791645in}}%
\pgfpathcurveto{\pgfqpoint{1.220665in}{1.797469in}}{\pgfqpoint{1.212765in}{1.800741in}}{\pgfqpoint{1.204529in}{1.800741in}}%
\pgfpathcurveto{\pgfqpoint{1.196292in}{1.800741in}}{\pgfqpoint{1.188392in}{1.797469in}}{\pgfqpoint{1.182568in}{1.791645in}}%
\pgfpathcurveto{\pgfqpoint{1.176744in}{1.785821in}}{\pgfqpoint{1.173472in}{1.777921in}}{\pgfqpoint{1.173472in}{1.769685in}}%
\pgfpathcurveto{\pgfqpoint{1.173472in}{1.761448in}}{\pgfqpoint{1.176744in}{1.753548in}}{\pgfqpoint{1.182568in}{1.747724in}}%
\pgfpathcurveto{\pgfqpoint{1.188392in}{1.741901in}}{\pgfqpoint{1.196292in}{1.738628in}}{\pgfqpoint{1.204529in}{1.738628in}}%
\pgfpathclose%
\pgfusepath{stroke,fill}%
\end{pgfscope}%
\begin{pgfscope}%
\pgfpathrectangle{\pgfqpoint{0.100000in}{0.212622in}}{\pgfqpoint{3.696000in}{3.696000in}}%
\pgfusepath{clip}%
\pgfsetbuttcap%
\pgfsetroundjoin%
\definecolor{currentfill}{rgb}{0.121569,0.466667,0.705882}%
\pgfsetfillcolor{currentfill}%
\pgfsetfillopacity{0.368105}%
\pgfsetlinewidth{1.003750pt}%
\definecolor{currentstroke}{rgb}{0.121569,0.466667,0.705882}%
\pgfsetstrokecolor{currentstroke}%
\pgfsetstrokeopacity{0.368105}%
\pgfsetdash{}{0pt}%
\pgfpathmoveto{\pgfqpoint{1.204529in}{1.738628in}}%
\pgfpathcurveto{\pgfqpoint{1.212765in}{1.738628in}}{\pgfqpoint{1.220665in}{1.741901in}}{\pgfqpoint{1.226489in}{1.747724in}}%
\pgfpathcurveto{\pgfqpoint{1.232313in}{1.753548in}}{\pgfqpoint{1.235585in}{1.761448in}}{\pgfqpoint{1.235585in}{1.769685in}}%
\pgfpathcurveto{\pgfqpoint{1.235585in}{1.777921in}}{\pgfqpoint{1.232313in}{1.785821in}}{\pgfqpoint{1.226489in}{1.791645in}}%
\pgfpathcurveto{\pgfqpoint{1.220665in}{1.797469in}}{\pgfqpoint{1.212765in}{1.800741in}}{\pgfqpoint{1.204529in}{1.800741in}}%
\pgfpathcurveto{\pgfqpoint{1.196292in}{1.800741in}}{\pgfqpoint{1.188392in}{1.797469in}}{\pgfqpoint{1.182568in}{1.791645in}}%
\pgfpathcurveto{\pgfqpoint{1.176744in}{1.785821in}}{\pgfqpoint{1.173472in}{1.777921in}}{\pgfqpoint{1.173472in}{1.769685in}}%
\pgfpathcurveto{\pgfqpoint{1.173472in}{1.761448in}}{\pgfqpoint{1.176744in}{1.753548in}}{\pgfqpoint{1.182568in}{1.747724in}}%
\pgfpathcurveto{\pgfqpoint{1.188392in}{1.741901in}}{\pgfqpoint{1.196292in}{1.738628in}}{\pgfqpoint{1.204529in}{1.738628in}}%
\pgfpathclose%
\pgfusepath{stroke,fill}%
\end{pgfscope}%
\begin{pgfscope}%
\pgfpathrectangle{\pgfqpoint{0.100000in}{0.212622in}}{\pgfqpoint{3.696000in}{3.696000in}}%
\pgfusepath{clip}%
\pgfsetbuttcap%
\pgfsetroundjoin%
\definecolor{currentfill}{rgb}{0.121569,0.466667,0.705882}%
\pgfsetfillcolor{currentfill}%
\pgfsetfillopacity{0.368105}%
\pgfsetlinewidth{1.003750pt}%
\definecolor{currentstroke}{rgb}{0.121569,0.466667,0.705882}%
\pgfsetstrokecolor{currentstroke}%
\pgfsetstrokeopacity{0.368105}%
\pgfsetdash{}{0pt}%
\pgfpathmoveto{\pgfqpoint{1.204529in}{1.738628in}}%
\pgfpathcurveto{\pgfqpoint{1.212765in}{1.738628in}}{\pgfqpoint{1.220665in}{1.741901in}}{\pgfqpoint{1.226489in}{1.747724in}}%
\pgfpathcurveto{\pgfqpoint{1.232313in}{1.753548in}}{\pgfqpoint{1.235585in}{1.761448in}}{\pgfqpoint{1.235585in}{1.769685in}}%
\pgfpathcurveto{\pgfqpoint{1.235585in}{1.777921in}}{\pgfqpoint{1.232313in}{1.785821in}}{\pgfqpoint{1.226489in}{1.791645in}}%
\pgfpathcurveto{\pgfqpoint{1.220665in}{1.797469in}}{\pgfqpoint{1.212765in}{1.800741in}}{\pgfqpoint{1.204529in}{1.800741in}}%
\pgfpathcurveto{\pgfqpoint{1.196292in}{1.800741in}}{\pgfqpoint{1.188392in}{1.797469in}}{\pgfqpoint{1.182568in}{1.791645in}}%
\pgfpathcurveto{\pgfqpoint{1.176744in}{1.785821in}}{\pgfqpoint{1.173472in}{1.777921in}}{\pgfqpoint{1.173472in}{1.769685in}}%
\pgfpathcurveto{\pgfqpoint{1.173472in}{1.761448in}}{\pgfqpoint{1.176744in}{1.753548in}}{\pgfqpoint{1.182568in}{1.747724in}}%
\pgfpathcurveto{\pgfqpoint{1.188392in}{1.741901in}}{\pgfqpoint{1.196292in}{1.738628in}}{\pgfqpoint{1.204529in}{1.738628in}}%
\pgfpathclose%
\pgfusepath{stroke,fill}%
\end{pgfscope}%
\begin{pgfscope}%
\pgfpathrectangle{\pgfqpoint{0.100000in}{0.212622in}}{\pgfqpoint{3.696000in}{3.696000in}}%
\pgfusepath{clip}%
\pgfsetbuttcap%
\pgfsetroundjoin%
\definecolor{currentfill}{rgb}{0.121569,0.466667,0.705882}%
\pgfsetfillcolor{currentfill}%
\pgfsetfillopacity{0.368105}%
\pgfsetlinewidth{1.003750pt}%
\definecolor{currentstroke}{rgb}{0.121569,0.466667,0.705882}%
\pgfsetstrokecolor{currentstroke}%
\pgfsetstrokeopacity{0.368105}%
\pgfsetdash{}{0pt}%
\pgfpathmoveto{\pgfqpoint{1.204529in}{1.738628in}}%
\pgfpathcurveto{\pgfqpoint{1.212765in}{1.738628in}}{\pgfqpoint{1.220665in}{1.741901in}}{\pgfqpoint{1.226489in}{1.747724in}}%
\pgfpathcurveto{\pgfqpoint{1.232313in}{1.753548in}}{\pgfqpoint{1.235585in}{1.761448in}}{\pgfqpoint{1.235585in}{1.769685in}}%
\pgfpathcurveto{\pgfqpoint{1.235585in}{1.777921in}}{\pgfqpoint{1.232313in}{1.785821in}}{\pgfqpoint{1.226489in}{1.791645in}}%
\pgfpathcurveto{\pgfqpoint{1.220665in}{1.797469in}}{\pgfqpoint{1.212765in}{1.800741in}}{\pgfqpoint{1.204529in}{1.800741in}}%
\pgfpathcurveto{\pgfqpoint{1.196292in}{1.800741in}}{\pgfqpoint{1.188392in}{1.797469in}}{\pgfqpoint{1.182568in}{1.791645in}}%
\pgfpathcurveto{\pgfqpoint{1.176744in}{1.785821in}}{\pgfqpoint{1.173472in}{1.777921in}}{\pgfqpoint{1.173472in}{1.769685in}}%
\pgfpathcurveto{\pgfqpoint{1.173472in}{1.761448in}}{\pgfqpoint{1.176744in}{1.753548in}}{\pgfqpoint{1.182568in}{1.747724in}}%
\pgfpathcurveto{\pgfqpoint{1.188392in}{1.741901in}}{\pgfqpoint{1.196292in}{1.738628in}}{\pgfqpoint{1.204529in}{1.738628in}}%
\pgfpathclose%
\pgfusepath{stroke,fill}%
\end{pgfscope}%
\begin{pgfscope}%
\pgfpathrectangle{\pgfqpoint{0.100000in}{0.212622in}}{\pgfqpoint{3.696000in}{3.696000in}}%
\pgfusepath{clip}%
\pgfsetbuttcap%
\pgfsetroundjoin%
\definecolor{currentfill}{rgb}{0.121569,0.466667,0.705882}%
\pgfsetfillcolor{currentfill}%
\pgfsetfillopacity{0.368105}%
\pgfsetlinewidth{1.003750pt}%
\definecolor{currentstroke}{rgb}{0.121569,0.466667,0.705882}%
\pgfsetstrokecolor{currentstroke}%
\pgfsetstrokeopacity{0.368105}%
\pgfsetdash{}{0pt}%
\pgfpathmoveto{\pgfqpoint{1.204529in}{1.738628in}}%
\pgfpathcurveto{\pgfqpoint{1.212765in}{1.738628in}}{\pgfqpoint{1.220665in}{1.741901in}}{\pgfqpoint{1.226489in}{1.747724in}}%
\pgfpathcurveto{\pgfqpoint{1.232313in}{1.753548in}}{\pgfqpoint{1.235585in}{1.761448in}}{\pgfqpoint{1.235585in}{1.769685in}}%
\pgfpathcurveto{\pgfqpoint{1.235585in}{1.777921in}}{\pgfqpoint{1.232313in}{1.785821in}}{\pgfqpoint{1.226489in}{1.791645in}}%
\pgfpathcurveto{\pgfqpoint{1.220665in}{1.797469in}}{\pgfqpoint{1.212765in}{1.800741in}}{\pgfqpoint{1.204529in}{1.800741in}}%
\pgfpathcurveto{\pgfqpoint{1.196292in}{1.800741in}}{\pgfqpoint{1.188392in}{1.797469in}}{\pgfqpoint{1.182568in}{1.791645in}}%
\pgfpathcurveto{\pgfqpoint{1.176744in}{1.785821in}}{\pgfqpoint{1.173472in}{1.777921in}}{\pgfqpoint{1.173472in}{1.769685in}}%
\pgfpathcurveto{\pgfqpoint{1.173472in}{1.761448in}}{\pgfqpoint{1.176744in}{1.753548in}}{\pgfqpoint{1.182568in}{1.747724in}}%
\pgfpathcurveto{\pgfqpoint{1.188392in}{1.741901in}}{\pgfqpoint{1.196292in}{1.738628in}}{\pgfqpoint{1.204529in}{1.738628in}}%
\pgfpathclose%
\pgfusepath{stroke,fill}%
\end{pgfscope}%
\begin{pgfscope}%
\pgfpathrectangle{\pgfqpoint{0.100000in}{0.212622in}}{\pgfqpoint{3.696000in}{3.696000in}}%
\pgfusepath{clip}%
\pgfsetbuttcap%
\pgfsetroundjoin%
\definecolor{currentfill}{rgb}{0.121569,0.466667,0.705882}%
\pgfsetfillcolor{currentfill}%
\pgfsetfillopacity{0.368105}%
\pgfsetlinewidth{1.003750pt}%
\definecolor{currentstroke}{rgb}{0.121569,0.466667,0.705882}%
\pgfsetstrokecolor{currentstroke}%
\pgfsetstrokeopacity{0.368105}%
\pgfsetdash{}{0pt}%
\pgfpathmoveto{\pgfqpoint{1.204529in}{1.738628in}}%
\pgfpathcurveto{\pgfqpoint{1.212765in}{1.738628in}}{\pgfqpoint{1.220665in}{1.741901in}}{\pgfqpoint{1.226489in}{1.747724in}}%
\pgfpathcurveto{\pgfqpoint{1.232313in}{1.753548in}}{\pgfqpoint{1.235585in}{1.761448in}}{\pgfqpoint{1.235585in}{1.769685in}}%
\pgfpathcurveto{\pgfqpoint{1.235585in}{1.777921in}}{\pgfqpoint{1.232313in}{1.785821in}}{\pgfqpoint{1.226489in}{1.791645in}}%
\pgfpathcurveto{\pgfqpoint{1.220665in}{1.797469in}}{\pgfqpoint{1.212765in}{1.800741in}}{\pgfqpoint{1.204529in}{1.800741in}}%
\pgfpathcurveto{\pgfqpoint{1.196292in}{1.800741in}}{\pgfqpoint{1.188392in}{1.797469in}}{\pgfqpoint{1.182568in}{1.791645in}}%
\pgfpathcurveto{\pgfqpoint{1.176744in}{1.785821in}}{\pgfqpoint{1.173472in}{1.777921in}}{\pgfqpoint{1.173472in}{1.769685in}}%
\pgfpathcurveto{\pgfqpoint{1.173472in}{1.761448in}}{\pgfqpoint{1.176744in}{1.753548in}}{\pgfqpoint{1.182568in}{1.747724in}}%
\pgfpathcurveto{\pgfqpoint{1.188392in}{1.741901in}}{\pgfqpoint{1.196292in}{1.738628in}}{\pgfqpoint{1.204529in}{1.738628in}}%
\pgfpathclose%
\pgfusepath{stroke,fill}%
\end{pgfscope}%
\begin{pgfscope}%
\pgfpathrectangle{\pgfqpoint{0.100000in}{0.212622in}}{\pgfqpoint{3.696000in}{3.696000in}}%
\pgfusepath{clip}%
\pgfsetbuttcap%
\pgfsetroundjoin%
\definecolor{currentfill}{rgb}{0.121569,0.466667,0.705882}%
\pgfsetfillcolor{currentfill}%
\pgfsetfillopacity{0.368105}%
\pgfsetlinewidth{1.003750pt}%
\definecolor{currentstroke}{rgb}{0.121569,0.466667,0.705882}%
\pgfsetstrokecolor{currentstroke}%
\pgfsetstrokeopacity{0.368105}%
\pgfsetdash{}{0pt}%
\pgfpathmoveto{\pgfqpoint{1.204529in}{1.738628in}}%
\pgfpathcurveto{\pgfqpoint{1.212765in}{1.738628in}}{\pgfqpoint{1.220665in}{1.741901in}}{\pgfqpoint{1.226489in}{1.747724in}}%
\pgfpathcurveto{\pgfqpoint{1.232313in}{1.753548in}}{\pgfqpoint{1.235585in}{1.761448in}}{\pgfqpoint{1.235585in}{1.769685in}}%
\pgfpathcurveto{\pgfqpoint{1.235585in}{1.777921in}}{\pgfqpoint{1.232313in}{1.785821in}}{\pgfqpoint{1.226489in}{1.791645in}}%
\pgfpathcurveto{\pgfqpoint{1.220665in}{1.797469in}}{\pgfqpoint{1.212765in}{1.800741in}}{\pgfqpoint{1.204529in}{1.800741in}}%
\pgfpathcurveto{\pgfqpoint{1.196292in}{1.800741in}}{\pgfqpoint{1.188392in}{1.797469in}}{\pgfqpoint{1.182568in}{1.791645in}}%
\pgfpathcurveto{\pgfqpoint{1.176744in}{1.785821in}}{\pgfqpoint{1.173472in}{1.777921in}}{\pgfqpoint{1.173472in}{1.769685in}}%
\pgfpathcurveto{\pgfqpoint{1.173472in}{1.761448in}}{\pgfqpoint{1.176744in}{1.753548in}}{\pgfqpoint{1.182568in}{1.747724in}}%
\pgfpathcurveto{\pgfqpoint{1.188392in}{1.741901in}}{\pgfqpoint{1.196292in}{1.738628in}}{\pgfqpoint{1.204529in}{1.738628in}}%
\pgfpathclose%
\pgfusepath{stroke,fill}%
\end{pgfscope}%
\begin{pgfscope}%
\pgfpathrectangle{\pgfqpoint{0.100000in}{0.212622in}}{\pgfqpoint{3.696000in}{3.696000in}}%
\pgfusepath{clip}%
\pgfsetbuttcap%
\pgfsetroundjoin%
\definecolor{currentfill}{rgb}{0.121569,0.466667,0.705882}%
\pgfsetfillcolor{currentfill}%
\pgfsetfillopacity{0.368105}%
\pgfsetlinewidth{1.003750pt}%
\definecolor{currentstroke}{rgb}{0.121569,0.466667,0.705882}%
\pgfsetstrokecolor{currentstroke}%
\pgfsetstrokeopacity{0.368105}%
\pgfsetdash{}{0pt}%
\pgfpathmoveto{\pgfqpoint{1.204529in}{1.738628in}}%
\pgfpathcurveto{\pgfqpoint{1.212765in}{1.738628in}}{\pgfqpoint{1.220665in}{1.741901in}}{\pgfqpoint{1.226489in}{1.747724in}}%
\pgfpathcurveto{\pgfqpoint{1.232313in}{1.753548in}}{\pgfqpoint{1.235585in}{1.761448in}}{\pgfqpoint{1.235585in}{1.769685in}}%
\pgfpathcurveto{\pgfqpoint{1.235585in}{1.777921in}}{\pgfqpoint{1.232313in}{1.785821in}}{\pgfqpoint{1.226489in}{1.791645in}}%
\pgfpathcurveto{\pgfqpoint{1.220665in}{1.797469in}}{\pgfqpoint{1.212765in}{1.800741in}}{\pgfqpoint{1.204529in}{1.800741in}}%
\pgfpathcurveto{\pgfqpoint{1.196292in}{1.800741in}}{\pgfqpoint{1.188392in}{1.797469in}}{\pgfqpoint{1.182568in}{1.791645in}}%
\pgfpathcurveto{\pgfqpoint{1.176744in}{1.785821in}}{\pgfqpoint{1.173472in}{1.777921in}}{\pgfqpoint{1.173472in}{1.769685in}}%
\pgfpathcurveto{\pgfqpoint{1.173472in}{1.761448in}}{\pgfqpoint{1.176744in}{1.753548in}}{\pgfqpoint{1.182568in}{1.747724in}}%
\pgfpathcurveto{\pgfqpoint{1.188392in}{1.741901in}}{\pgfqpoint{1.196292in}{1.738628in}}{\pgfqpoint{1.204529in}{1.738628in}}%
\pgfpathclose%
\pgfusepath{stroke,fill}%
\end{pgfscope}%
\begin{pgfscope}%
\pgfpathrectangle{\pgfqpoint{0.100000in}{0.212622in}}{\pgfqpoint{3.696000in}{3.696000in}}%
\pgfusepath{clip}%
\pgfsetbuttcap%
\pgfsetroundjoin%
\definecolor{currentfill}{rgb}{0.121569,0.466667,0.705882}%
\pgfsetfillcolor{currentfill}%
\pgfsetfillopacity{0.368105}%
\pgfsetlinewidth{1.003750pt}%
\definecolor{currentstroke}{rgb}{0.121569,0.466667,0.705882}%
\pgfsetstrokecolor{currentstroke}%
\pgfsetstrokeopacity{0.368105}%
\pgfsetdash{}{0pt}%
\pgfpathmoveto{\pgfqpoint{1.204529in}{1.738628in}}%
\pgfpathcurveto{\pgfqpoint{1.212765in}{1.738628in}}{\pgfqpoint{1.220665in}{1.741901in}}{\pgfqpoint{1.226489in}{1.747724in}}%
\pgfpathcurveto{\pgfqpoint{1.232313in}{1.753548in}}{\pgfqpoint{1.235585in}{1.761448in}}{\pgfqpoint{1.235585in}{1.769685in}}%
\pgfpathcurveto{\pgfqpoint{1.235585in}{1.777921in}}{\pgfqpoint{1.232313in}{1.785821in}}{\pgfqpoint{1.226489in}{1.791645in}}%
\pgfpathcurveto{\pgfqpoint{1.220665in}{1.797469in}}{\pgfqpoint{1.212765in}{1.800741in}}{\pgfqpoint{1.204529in}{1.800741in}}%
\pgfpathcurveto{\pgfqpoint{1.196292in}{1.800741in}}{\pgfqpoint{1.188392in}{1.797469in}}{\pgfqpoint{1.182568in}{1.791645in}}%
\pgfpathcurveto{\pgfqpoint{1.176744in}{1.785821in}}{\pgfqpoint{1.173472in}{1.777921in}}{\pgfqpoint{1.173472in}{1.769685in}}%
\pgfpathcurveto{\pgfqpoint{1.173472in}{1.761448in}}{\pgfqpoint{1.176744in}{1.753548in}}{\pgfqpoint{1.182568in}{1.747724in}}%
\pgfpathcurveto{\pgfqpoint{1.188392in}{1.741901in}}{\pgfqpoint{1.196292in}{1.738628in}}{\pgfqpoint{1.204529in}{1.738628in}}%
\pgfpathclose%
\pgfusepath{stroke,fill}%
\end{pgfscope}%
\begin{pgfscope}%
\pgfpathrectangle{\pgfqpoint{0.100000in}{0.212622in}}{\pgfqpoint{3.696000in}{3.696000in}}%
\pgfusepath{clip}%
\pgfsetbuttcap%
\pgfsetroundjoin%
\definecolor{currentfill}{rgb}{0.121569,0.466667,0.705882}%
\pgfsetfillcolor{currentfill}%
\pgfsetfillopacity{0.368105}%
\pgfsetlinewidth{1.003750pt}%
\definecolor{currentstroke}{rgb}{0.121569,0.466667,0.705882}%
\pgfsetstrokecolor{currentstroke}%
\pgfsetstrokeopacity{0.368105}%
\pgfsetdash{}{0pt}%
\pgfpathmoveto{\pgfqpoint{1.204529in}{1.738628in}}%
\pgfpathcurveto{\pgfqpoint{1.212765in}{1.738628in}}{\pgfqpoint{1.220665in}{1.741901in}}{\pgfqpoint{1.226489in}{1.747724in}}%
\pgfpathcurveto{\pgfqpoint{1.232313in}{1.753548in}}{\pgfqpoint{1.235585in}{1.761448in}}{\pgfqpoint{1.235585in}{1.769685in}}%
\pgfpathcurveto{\pgfqpoint{1.235585in}{1.777921in}}{\pgfqpoint{1.232313in}{1.785821in}}{\pgfqpoint{1.226489in}{1.791645in}}%
\pgfpathcurveto{\pgfqpoint{1.220665in}{1.797469in}}{\pgfqpoint{1.212765in}{1.800741in}}{\pgfqpoint{1.204529in}{1.800741in}}%
\pgfpathcurveto{\pgfqpoint{1.196292in}{1.800741in}}{\pgfqpoint{1.188392in}{1.797469in}}{\pgfqpoint{1.182568in}{1.791645in}}%
\pgfpathcurveto{\pgfqpoint{1.176744in}{1.785821in}}{\pgfqpoint{1.173472in}{1.777921in}}{\pgfqpoint{1.173472in}{1.769685in}}%
\pgfpathcurveto{\pgfqpoint{1.173472in}{1.761448in}}{\pgfqpoint{1.176744in}{1.753548in}}{\pgfqpoint{1.182568in}{1.747724in}}%
\pgfpathcurveto{\pgfqpoint{1.188392in}{1.741901in}}{\pgfqpoint{1.196292in}{1.738628in}}{\pgfqpoint{1.204529in}{1.738628in}}%
\pgfpathclose%
\pgfusepath{stroke,fill}%
\end{pgfscope}%
\begin{pgfscope}%
\pgfpathrectangle{\pgfqpoint{0.100000in}{0.212622in}}{\pgfqpoint{3.696000in}{3.696000in}}%
\pgfusepath{clip}%
\pgfsetbuttcap%
\pgfsetroundjoin%
\definecolor{currentfill}{rgb}{0.121569,0.466667,0.705882}%
\pgfsetfillcolor{currentfill}%
\pgfsetfillopacity{0.368105}%
\pgfsetlinewidth{1.003750pt}%
\definecolor{currentstroke}{rgb}{0.121569,0.466667,0.705882}%
\pgfsetstrokecolor{currentstroke}%
\pgfsetstrokeopacity{0.368105}%
\pgfsetdash{}{0pt}%
\pgfpathmoveto{\pgfqpoint{1.204529in}{1.738628in}}%
\pgfpathcurveto{\pgfqpoint{1.212765in}{1.738628in}}{\pgfqpoint{1.220665in}{1.741901in}}{\pgfqpoint{1.226489in}{1.747724in}}%
\pgfpathcurveto{\pgfqpoint{1.232313in}{1.753548in}}{\pgfqpoint{1.235585in}{1.761448in}}{\pgfqpoint{1.235585in}{1.769685in}}%
\pgfpathcurveto{\pgfqpoint{1.235585in}{1.777921in}}{\pgfqpoint{1.232313in}{1.785821in}}{\pgfqpoint{1.226489in}{1.791645in}}%
\pgfpathcurveto{\pgfqpoint{1.220665in}{1.797469in}}{\pgfqpoint{1.212765in}{1.800741in}}{\pgfqpoint{1.204529in}{1.800741in}}%
\pgfpathcurveto{\pgfqpoint{1.196292in}{1.800741in}}{\pgfqpoint{1.188392in}{1.797469in}}{\pgfqpoint{1.182568in}{1.791645in}}%
\pgfpathcurveto{\pgfqpoint{1.176744in}{1.785821in}}{\pgfqpoint{1.173472in}{1.777921in}}{\pgfqpoint{1.173472in}{1.769685in}}%
\pgfpathcurveto{\pgfqpoint{1.173472in}{1.761448in}}{\pgfqpoint{1.176744in}{1.753548in}}{\pgfqpoint{1.182568in}{1.747724in}}%
\pgfpathcurveto{\pgfqpoint{1.188392in}{1.741901in}}{\pgfqpoint{1.196292in}{1.738628in}}{\pgfqpoint{1.204529in}{1.738628in}}%
\pgfpathclose%
\pgfusepath{stroke,fill}%
\end{pgfscope}%
\begin{pgfscope}%
\pgfpathrectangle{\pgfqpoint{0.100000in}{0.212622in}}{\pgfqpoint{3.696000in}{3.696000in}}%
\pgfusepath{clip}%
\pgfsetbuttcap%
\pgfsetroundjoin%
\definecolor{currentfill}{rgb}{0.121569,0.466667,0.705882}%
\pgfsetfillcolor{currentfill}%
\pgfsetfillopacity{0.368105}%
\pgfsetlinewidth{1.003750pt}%
\definecolor{currentstroke}{rgb}{0.121569,0.466667,0.705882}%
\pgfsetstrokecolor{currentstroke}%
\pgfsetstrokeopacity{0.368105}%
\pgfsetdash{}{0pt}%
\pgfpathmoveto{\pgfqpoint{1.204529in}{1.738628in}}%
\pgfpathcurveto{\pgfqpoint{1.212765in}{1.738628in}}{\pgfqpoint{1.220665in}{1.741901in}}{\pgfqpoint{1.226489in}{1.747724in}}%
\pgfpathcurveto{\pgfqpoint{1.232313in}{1.753548in}}{\pgfqpoint{1.235585in}{1.761448in}}{\pgfqpoint{1.235585in}{1.769685in}}%
\pgfpathcurveto{\pgfqpoint{1.235585in}{1.777921in}}{\pgfqpoint{1.232313in}{1.785821in}}{\pgfqpoint{1.226489in}{1.791645in}}%
\pgfpathcurveto{\pgfqpoint{1.220665in}{1.797469in}}{\pgfqpoint{1.212765in}{1.800741in}}{\pgfqpoint{1.204529in}{1.800741in}}%
\pgfpathcurveto{\pgfqpoint{1.196292in}{1.800741in}}{\pgfqpoint{1.188392in}{1.797469in}}{\pgfqpoint{1.182568in}{1.791645in}}%
\pgfpathcurveto{\pgfqpoint{1.176744in}{1.785821in}}{\pgfqpoint{1.173472in}{1.777921in}}{\pgfqpoint{1.173472in}{1.769685in}}%
\pgfpathcurveto{\pgfqpoint{1.173472in}{1.761448in}}{\pgfqpoint{1.176744in}{1.753548in}}{\pgfqpoint{1.182568in}{1.747724in}}%
\pgfpathcurveto{\pgfqpoint{1.188392in}{1.741901in}}{\pgfqpoint{1.196292in}{1.738628in}}{\pgfqpoint{1.204529in}{1.738628in}}%
\pgfpathclose%
\pgfusepath{stroke,fill}%
\end{pgfscope}%
\begin{pgfscope}%
\pgfpathrectangle{\pgfqpoint{0.100000in}{0.212622in}}{\pgfqpoint{3.696000in}{3.696000in}}%
\pgfusepath{clip}%
\pgfsetbuttcap%
\pgfsetroundjoin%
\definecolor{currentfill}{rgb}{0.121569,0.466667,0.705882}%
\pgfsetfillcolor{currentfill}%
\pgfsetfillopacity{0.368105}%
\pgfsetlinewidth{1.003750pt}%
\definecolor{currentstroke}{rgb}{0.121569,0.466667,0.705882}%
\pgfsetstrokecolor{currentstroke}%
\pgfsetstrokeopacity{0.368105}%
\pgfsetdash{}{0pt}%
\pgfpathmoveto{\pgfqpoint{1.204529in}{1.738628in}}%
\pgfpathcurveto{\pgfqpoint{1.212765in}{1.738628in}}{\pgfqpoint{1.220665in}{1.741901in}}{\pgfqpoint{1.226489in}{1.747724in}}%
\pgfpathcurveto{\pgfqpoint{1.232313in}{1.753548in}}{\pgfqpoint{1.235585in}{1.761448in}}{\pgfqpoint{1.235585in}{1.769685in}}%
\pgfpathcurveto{\pgfqpoint{1.235585in}{1.777921in}}{\pgfqpoint{1.232313in}{1.785821in}}{\pgfqpoint{1.226489in}{1.791645in}}%
\pgfpathcurveto{\pgfqpoint{1.220665in}{1.797469in}}{\pgfqpoint{1.212765in}{1.800741in}}{\pgfqpoint{1.204529in}{1.800741in}}%
\pgfpathcurveto{\pgfqpoint{1.196292in}{1.800741in}}{\pgfqpoint{1.188392in}{1.797469in}}{\pgfqpoint{1.182568in}{1.791645in}}%
\pgfpathcurveto{\pgfqpoint{1.176744in}{1.785821in}}{\pgfqpoint{1.173472in}{1.777921in}}{\pgfqpoint{1.173472in}{1.769685in}}%
\pgfpathcurveto{\pgfqpoint{1.173472in}{1.761448in}}{\pgfqpoint{1.176744in}{1.753548in}}{\pgfqpoint{1.182568in}{1.747724in}}%
\pgfpathcurveto{\pgfqpoint{1.188392in}{1.741901in}}{\pgfqpoint{1.196292in}{1.738628in}}{\pgfqpoint{1.204529in}{1.738628in}}%
\pgfpathclose%
\pgfusepath{stroke,fill}%
\end{pgfscope}%
\begin{pgfscope}%
\pgfpathrectangle{\pgfqpoint{0.100000in}{0.212622in}}{\pgfqpoint{3.696000in}{3.696000in}}%
\pgfusepath{clip}%
\pgfsetbuttcap%
\pgfsetroundjoin%
\definecolor{currentfill}{rgb}{0.121569,0.466667,0.705882}%
\pgfsetfillcolor{currentfill}%
\pgfsetfillopacity{0.368105}%
\pgfsetlinewidth{1.003750pt}%
\definecolor{currentstroke}{rgb}{0.121569,0.466667,0.705882}%
\pgfsetstrokecolor{currentstroke}%
\pgfsetstrokeopacity{0.368105}%
\pgfsetdash{}{0pt}%
\pgfpathmoveto{\pgfqpoint{1.204529in}{1.738628in}}%
\pgfpathcurveto{\pgfqpoint{1.212765in}{1.738628in}}{\pgfqpoint{1.220665in}{1.741901in}}{\pgfqpoint{1.226489in}{1.747724in}}%
\pgfpathcurveto{\pgfqpoint{1.232313in}{1.753548in}}{\pgfqpoint{1.235585in}{1.761448in}}{\pgfqpoint{1.235585in}{1.769685in}}%
\pgfpathcurveto{\pgfqpoint{1.235585in}{1.777921in}}{\pgfqpoint{1.232313in}{1.785821in}}{\pgfqpoint{1.226489in}{1.791645in}}%
\pgfpathcurveto{\pgfqpoint{1.220665in}{1.797469in}}{\pgfqpoint{1.212765in}{1.800741in}}{\pgfqpoint{1.204529in}{1.800741in}}%
\pgfpathcurveto{\pgfqpoint{1.196292in}{1.800741in}}{\pgfqpoint{1.188392in}{1.797469in}}{\pgfqpoint{1.182568in}{1.791645in}}%
\pgfpathcurveto{\pgfqpoint{1.176744in}{1.785821in}}{\pgfqpoint{1.173472in}{1.777921in}}{\pgfqpoint{1.173472in}{1.769685in}}%
\pgfpathcurveto{\pgfqpoint{1.173472in}{1.761448in}}{\pgfqpoint{1.176744in}{1.753548in}}{\pgfqpoint{1.182568in}{1.747724in}}%
\pgfpathcurveto{\pgfqpoint{1.188392in}{1.741901in}}{\pgfqpoint{1.196292in}{1.738628in}}{\pgfqpoint{1.204529in}{1.738628in}}%
\pgfpathclose%
\pgfusepath{stroke,fill}%
\end{pgfscope}%
\begin{pgfscope}%
\pgfpathrectangle{\pgfqpoint{0.100000in}{0.212622in}}{\pgfqpoint{3.696000in}{3.696000in}}%
\pgfusepath{clip}%
\pgfsetbuttcap%
\pgfsetroundjoin%
\definecolor{currentfill}{rgb}{0.121569,0.466667,0.705882}%
\pgfsetfillcolor{currentfill}%
\pgfsetfillopacity{0.368105}%
\pgfsetlinewidth{1.003750pt}%
\definecolor{currentstroke}{rgb}{0.121569,0.466667,0.705882}%
\pgfsetstrokecolor{currentstroke}%
\pgfsetstrokeopacity{0.368105}%
\pgfsetdash{}{0pt}%
\pgfpathmoveto{\pgfqpoint{1.204529in}{1.738628in}}%
\pgfpathcurveto{\pgfqpoint{1.212765in}{1.738628in}}{\pgfqpoint{1.220665in}{1.741901in}}{\pgfqpoint{1.226489in}{1.747724in}}%
\pgfpathcurveto{\pgfqpoint{1.232313in}{1.753548in}}{\pgfqpoint{1.235585in}{1.761448in}}{\pgfqpoint{1.235585in}{1.769685in}}%
\pgfpathcurveto{\pgfqpoint{1.235585in}{1.777921in}}{\pgfqpoint{1.232313in}{1.785821in}}{\pgfqpoint{1.226489in}{1.791645in}}%
\pgfpathcurveto{\pgfqpoint{1.220665in}{1.797469in}}{\pgfqpoint{1.212765in}{1.800741in}}{\pgfqpoint{1.204529in}{1.800741in}}%
\pgfpathcurveto{\pgfqpoint{1.196292in}{1.800741in}}{\pgfqpoint{1.188392in}{1.797469in}}{\pgfqpoint{1.182568in}{1.791645in}}%
\pgfpathcurveto{\pgfqpoint{1.176744in}{1.785821in}}{\pgfqpoint{1.173472in}{1.777921in}}{\pgfqpoint{1.173472in}{1.769685in}}%
\pgfpathcurveto{\pgfqpoint{1.173472in}{1.761448in}}{\pgfqpoint{1.176744in}{1.753548in}}{\pgfqpoint{1.182568in}{1.747724in}}%
\pgfpathcurveto{\pgfqpoint{1.188392in}{1.741901in}}{\pgfqpoint{1.196292in}{1.738628in}}{\pgfqpoint{1.204529in}{1.738628in}}%
\pgfpathclose%
\pgfusepath{stroke,fill}%
\end{pgfscope}%
\begin{pgfscope}%
\pgfpathrectangle{\pgfqpoint{0.100000in}{0.212622in}}{\pgfqpoint{3.696000in}{3.696000in}}%
\pgfusepath{clip}%
\pgfsetbuttcap%
\pgfsetroundjoin%
\definecolor{currentfill}{rgb}{0.121569,0.466667,0.705882}%
\pgfsetfillcolor{currentfill}%
\pgfsetfillopacity{0.368105}%
\pgfsetlinewidth{1.003750pt}%
\definecolor{currentstroke}{rgb}{0.121569,0.466667,0.705882}%
\pgfsetstrokecolor{currentstroke}%
\pgfsetstrokeopacity{0.368105}%
\pgfsetdash{}{0pt}%
\pgfpathmoveto{\pgfqpoint{1.204529in}{1.738628in}}%
\pgfpathcurveto{\pgfqpoint{1.212765in}{1.738628in}}{\pgfqpoint{1.220665in}{1.741901in}}{\pgfqpoint{1.226489in}{1.747724in}}%
\pgfpathcurveto{\pgfqpoint{1.232313in}{1.753548in}}{\pgfqpoint{1.235585in}{1.761448in}}{\pgfqpoint{1.235585in}{1.769685in}}%
\pgfpathcurveto{\pgfqpoint{1.235585in}{1.777921in}}{\pgfqpoint{1.232313in}{1.785821in}}{\pgfqpoint{1.226489in}{1.791645in}}%
\pgfpathcurveto{\pgfqpoint{1.220665in}{1.797469in}}{\pgfqpoint{1.212765in}{1.800741in}}{\pgfqpoint{1.204529in}{1.800741in}}%
\pgfpathcurveto{\pgfqpoint{1.196292in}{1.800741in}}{\pgfqpoint{1.188392in}{1.797469in}}{\pgfqpoint{1.182568in}{1.791645in}}%
\pgfpathcurveto{\pgfqpoint{1.176744in}{1.785821in}}{\pgfqpoint{1.173472in}{1.777921in}}{\pgfqpoint{1.173472in}{1.769685in}}%
\pgfpathcurveto{\pgfqpoint{1.173472in}{1.761448in}}{\pgfqpoint{1.176744in}{1.753548in}}{\pgfqpoint{1.182568in}{1.747724in}}%
\pgfpathcurveto{\pgfqpoint{1.188392in}{1.741901in}}{\pgfqpoint{1.196292in}{1.738628in}}{\pgfqpoint{1.204529in}{1.738628in}}%
\pgfpathclose%
\pgfusepath{stroke,fill}%
\end{pgfscope}%
\begin{pgfscope}%
\pgfpathrectangle{\pgfqpoint{0.100000in}{0.212622in}}{\pgfqpoint{3.696000in}{3.696000in}}%
\pgfusepath{clip}%
\pgfsetbuttcap%
\pgfsetroundjoin%
\definecolor{currentfill}{rgb}{0.121569,0.466667,0.705882}%
\pgfsetfillcolor{currentfill}%
\pgfsetfillopacity{0.368105}%
\pgfsetlinewidth{1.003750pt}%
\definecolor{currentstroke}{rgb}{0.121569,0.466667,0.705882}%
\pgfsetstrokecolor{currentstroke}%
\pgfsetstrokeopacity{0.368105}%
\pgfsetdash{}{0pt}%
\pgfpathmoveto{\pgfqpoint{1.204529in}{1.738628in}}%
\pgfpathcurveto{\pgfqpoint{1.212765in}{1.738628in}}{\pgfqpoint{1.220665in}{1.741901in}}{\pgfqpoint{1.226489in}{1.747724in}}%
\pgfpathcurveto{\pgfqpoint{1.232313in}{1.753548in}}{\pgfqpoint{1.235585in}{1.761448in}}{\pgfqpoint{1.235585in}{1.769685in}}%
\pgfpathcurveto{\pgfqpoint{1.235585in}{1.777921in}}{\pgfqpoint{1.232313in}{1.785821in}}{\pgfqpoint{1.226489in}{1.791645in}}%
\pgfpathcurveto{\pgfqpoint{1.220665in}{1.797469in}}{\pgfqpoint{1.212765in}{1.800741in}}{\pgfqpoint{1.204529in}{1.800741in}}%
\pgfpathcurveto{\pgfqpoint{1.196292in}{1.800741in}}{\pgfqpoint{1.188392in}{1.797469in}}{\pgfqpoint{1.182568in}{1.791645in}}%
\pgfpathcurveto{\pgfqpoint{1.176744in}{1.785821in}}{\pgfqpoint{1.173472in}{1.777921in}}{\pgfqpoint{1.173472in}{1.769685in}}%
\pgfpathcurveto{\pgfqpoint{1.173472in}{1.761448in}}{\pgfqpoint{1.176744in}{1.753548in}}{\pgfqpoint{1.182568in}{1.747724in}}%
\pgfpathcurveto{\pgfqpoint{1.188392in}{1.741901in}}{\pgfqpoint{1.196292in}{1.738628in}}{\pgfqpoint{1.204529in}{1.738628in}}%
\pgfpathclose%
\pgfusepath{stroke,fill}%
\end{pgfscope}%
\begin{pgfscope}%
\pgfpathrectangle{\pgfqpoint{0.100000in}{0.212622in}}{\pgfqpoint{3.696000in}{3.696000in}}%
\pgfusepath{clip}%
\pgfsetbuttcap%
\pgfsetroundjoin%
\definecolor{currentfill}{rgb}{0.121569,0.466667,0.705882}%
\pgfsetfillcolor{currentfill}%
\pgfsetfillopacity{0.368105}%
\pgfsetlinewidth{1.003750pt}%
\definecolor{currentstroke}{rgb}{0.121569,0.466667,0.705882}%
\pgfsetstrokecolor{currentstroke}%
\pgfsetstrokeopacity{0.368105}%
\pgfsetdash{}{0pt}%
\pgfpathmoveto{\pgfqpoint{1.204529in}{1.738628in}}%
\pgfpathcurveto{\pgfqpoint{1.212765in}{1.738628in}}{\pgfqpoint{1.220665in}{1.741901in}}{\pgfqpoint{1.226489in}{1.747724in}}%
\pgfpathcurveto{\pgfqpoint{1.232313in}{1.753548in}}{\pgfqpoint{1.235585in}{1.761448in}}{\pgfqpoint{1.235585in}{1.769685in}}%
\pgfpathcurveto{\pgfqpoint{1.235585in}{1.777921in}}{\pgfqpoint{1.232313in}{1.785821in}}{\pgfqpoint{1.226489in}{1.791645in}}%
\pgfpathcurveto{\pgfqpoint{1.220665in}{1.797469in}}{\pgfqpoint{1.212765in}{1.800741in}}{\pgfqpoint{1.204529in}{1.800741in}}%
\pgfpathcurveto{\pgfqpoint{1.196292in}{1.800741in}}{\pgfqpoint{1.188392in}{1.797469in}}{\pgfqpoint{1.182568in}{1.791645in}}%
\pgfpathcurveto{\pgfqpoint{1.176744in}{1.785821in}}{\pgfqpoint{1.173472in}{1.777921in}}{\pgfqpoint{1.173472in}{1.769685in}}%
\pgfpathcurveto{\pgfqpoint{1.173472in}{1.761448in}}{\pgfqpoint{1.176744in}{1.753548in}}{\pgfqpoint{1.182568in}{1.747724in}}%
\pgfpathcurveto{\pgfqpoint{1.188392in}{1.741901in}}{\pgfqpoint{1.196292in}{1.738628in}}{\pgfqpoint{1.204529in}{1.738628in}}%
\pgfpathclose%
\pgfusepath{stroke,fill}%
\end{pgfscope}%
\begin{pgfscope}%
\pgfpathrectangle{\pgfqpoint{0.100000in}{0.212622in}}{\pgfqpoint{3.696000in}{3.696000in}}%
\pgfusepath{clip}%
\pgfsetbuttcap%
\pgfsetroundjoin%
\definecolor{currentfill}{rgb}{0.121569,0.466667,0.705882}%
\pgfsetfillcolor{currentfill}%
\pgfsetfillopacity{0.368105}%
\pgfsetlinewidth{1.003750pt}%
\definecolor{currentstroke}{rgb}{0.121569,0.466667,0.705882}%
\pgfsetstrokecolor{currentstroke}%
\pgfsetstrokeopacity{0.368105}%
\pgfsetdash{}{0pt}%
\pgfpathmoveto{\pgfqpoint{1.204529in}{1.738628in}}%
\pgfpathcurveto{\pgfqpoint{1.212765in}{1.738628in}}{\pgfqpoint{1.220665in}{1.741901in}}{\pgfqpoint{1.226489in}{1.747724in}}%
\pgfpathcurveto{\pgfqpoint{1.232313in}{1.753548in}}{\pgfqpoint{1.235585in}{1.761448in}}{\pgfqpoint{1.235585in}{1.769685in}}%
\pgfpathcurveto{\pgfqpoint{1.235585in}{1.777921in}}{\pgfqpoint{1.232313in}{1.785821in}}{\pgfqpoint{1.226489in}{1.791645in}}%
\pgfpathcurveto{\pgfqpoint{1.220665in}{1.797469in}}{\pgfqpoint{1.212765in}{1.800741in}}{\pgfqpoint{1.204529in}{1.800741in}}%
\pgfpathcurveto{\pgfqpoint{1.196292in}{1.800741in}}{\pgfqpoint{1.188392in}{1.797469in}}{\pgfqpoint{1.182568in}{1.791645in}}%
\pgfpathcurveto{\pgfqpoint{1.176744in}{1.785821in}}{\pgfqpoint{1.173472in}{1.777921in}}{\pgfqpoint{1.173472in}{1.769685in}}%
\pgfpathcurveto{\pgfqpoint{1.173472in}{1.761448in}}{\pgfqpoint{1.176744in}{1.753548in}}{\pgfqpoint{1.182568in}{1.747724in}}%
\pgfpathcurveto{\pgfqpoint{1.188392in}{1.741901in}}{\pgfqpoint{1.196292in}{1.738628in}}{\pgfqpoint{1.204529in}{1.738628in}}%
\pgfpathclose%
\pgfusepath{stroke,fill}%
\end{pgfscope}%
\begin{pgfscope}%
\pgfpathrectangle{\pgfqpoint{0.100000in}{0.212622in}}{\pgfqpoint{3.696000in}{3.696000in}}%
\pgfusepath{clip}%
\pgfsetbuttcap%
\pgfsetroundjoin%
\definecolor{currentfill}{rgb}{0.121569,0.466667,0.705882}%
\pgfsetfillcolor{currentfill}%
\pgfsetfillopacity{0.368105}%
\pgfsetlinewidth{1.003750pt}%
\definecolor{currentstroke}{rgb}{0.121569,0.466667,0.705882}%
\pgfsetstrokecolor{currentstroke}%
\pgfsetstrokeopacity{0.368105}%
\pgfsetdash{}{0pt}%
\pgfpathmoveto{\pgfqpoint{1.204529in}{1.738628in}}%
\pgfpathcurveto{\pgfqpoint{1.212765in}{1.738628in}}{\pgfqpoint{1.220665in}{1.741901in}}{\pgfqpoint{1.226489in}{1.747724in}}%
\pgfpathcurveto{\pgfqpoint{1.232313in}{1.753548in}}{\pgfqpoint{1.235585in}{1.761448in}}{\pgfqpoint{1.235585in}{1.769685in}}%
\pgfpathcurveto{\pgfqpoint{1.235585in}{1.777921in}}{\pgfqpoint{1.232313in}{1.785821in}}{\pgfqpoint{1.226489in}{1.791645in}}%
\pgfpathcurveto{\pgfqpoint{1.220665in}{1.797469in}}{\pgfqpoint{1.212765in}{1.800741in}}{\pgfqpoint{1.204529in}{1.800741in}}%
\pgfpathcurveto{\pgfqpoint{1.196292in}{1.800741in}}{\pgfqpoint{1.188392in}{1.797469in}}{\pgfqpoint{1.182568in}{1.791645in}}%
\pgfpathcurveto{\pgfqpoint{1.176744in}{1.785821in}}{\pgfqpoint{1.173472in}{1.777921in}}{\pgfqpoint{1.173472in}{1.769685in}}%
\pgfpathcurveto{\pgfqpoint{1.173472in}{1.761448in}}{\pgfqpoint{1.176744in}{1.753548in}}{\pgfqpoint{1.182568in}{1.747724in}}%
\pgfpathcurveto{\pgfqpoint{1.188392in}{1.741901in}}{\pgfqpoint{1.196292in}{1.738628in}}{\pgfqpoint{1.204529in}{1.738628in}}%
\pgfpathclose%
\pgfusepath{stroke,fill}%
\end{pgfscope}%
\begin{pgfscope}%
\pgfpathrectangle{\pgfqpoint{0.100000in}{0.212622in}}{\pgfqpoint{3.696000in}{3.696000in}}%
\pgfusepath{clip}%
\pgfsetbuttcap%
\pgfsetroundjoin%
\definecolor{currentfill}{rgb}{0.121569,0.466667,0.705882}%
\pgfsetfillcolor{currentfill}%
\pgfsetfillopacity{0.368105}%
\pgfsetlinewidth{1.003750pt}%
\definecolor{currentstroke}{rgb}{0.121569,0.466667,0.705882}%
\pgfsetstrokecolor{currentstroke}%
\pgfsetstrokeopacity{0.368105}%
\pgfsetdash{}{0pt}%
\pgfpathmoveto{\pgfqpoint{1.204529in}{1.738628in}}%
\pgfpathcurveto{\pgfqpoint{1.212765in}{1.738628in}}{\pgfqpoint{1.220665in}{1.741901in}}{\pgfqpoint{1.226489in}{1.747724in}}%
\pgfpathcurveto{\pgfqpoint{1.232313in}{1.753548in}}{\pgfqpoint{1.235585in}{1.761448in}}{\pgfqpoint{1.235585in}{1.769685in}}%
\pgfpathcurveto{\pgfqpoint{1.235585in}{1.777921in}}{\pgfqpoint{1.232313in}{1.785821in}}{\pgfqpoint{1.226489in}{1.791645in}}%
\pgfpathcurveto{\pgfqpoint{1.220665in}{1.797469in}}{\pgfqpoint{1.212765in}{1.800741in}}{\pgfqpoint{1.204529in}{1.800741in}}%
\pgfpathcurveto{\pgfqpoint{1.196292in}{1.800741in}}{\pgfqpoint{1.188392in}{1.797469in}}{\pgfqpoint{1.182568in}{1.791645in}}%
\pgfpathcurveto{\pgfqpoint{1.176744in}{1.785821in}}{\pgfqpoint{1.173472in}{1.777921in}}{\pgfqpoint{1.173472in}{1.769685in}}%
\pgfpathcurveto{\pgfqpoint{1.173472in}{1.761448in}}{\pgfqpoint{1.176744in}{1.753548in}}{\pgfqpoint{1.182568in}{1.747724in}}%
\pgfpathcurveto{\pgfqpoint{1.188392in}{1.741901in}}{\pgfqpoint{1.196292in}{1.738628in}}{\pgfqpoint{1.204529in}{1.738628in}}%
\pgfpathclose%
\pgfusepath{stroke,fill}%
\end{pgfscope}%
\begin{pgfscope}%
\pgfpathrectangle{\pgfqpoint{0.100000in}{0.212622in}}{\pgfqpoint{3.696000in}{3.696000in}}%
\pgfusepath{clip}%
\pgfsetbuttcap%
\pgfsetroundjoin%
\definecolor{currentfill}{rgb}{0.121569,0.466667,0.705882}%
\pgfsetfillcolor{currentfill}%
\pgfsetfillopacity{0.368105}%
\pgfsetlinewidth{1.003750pt}%
\definecolor{currentstroke}{rgb}{0.121569,0.466667,0.705882}%
\pgfsetstrokecolor{currentstroke}%
\pgfsetstrokeopacity{0.368105}%
\pgfsetdash{}{0pt}%
\pgfpathmoveto{\pgfqpoint{1.204529in}{1.738628in}}%
\pgfpathcurveto{\pgfqpoint{1.212765in}{1.738628in}}{\pgfqpoint{1.220665in}{1.741901in}}{\pgfqpoint{1.226489in}{1.747724in}}%
\pgfpathcurveto{\pgfqpoint{1.232313in}{1.753548in}}{\pgfqpoint{1.235585in}{1.761448in}}{\pgfqpoint{1.235585in}{1.769685in}}%
\pgfpathcurveto{\pgfqpoint{1.235585in}{1.777921in}}{\pgfqpoint{1.232313in}{1.785821in}}{\pgfqpoint{1.226489in}{1.791645in}}%
\pgfpathcurveto{\pgfqpoint{1.220665in}{1.797469in}}{\pgfqpoint{1.212765in}{1.800741in}}{\pgfqpoint{1.204529in}{1.800741in}}%
\pgfpathcurveto{\pgfqpoint{1.196292in}{1.800741in}}{\pgfqpoint{1.188392in}{1.797469in}}{\pgfqpoint{1.182568in}{1.791645in}}%
\pgfpathcurveto{\pgfqpoint{1.176744in}{1.785821in}}{\pgfqpoint{1.173472in}{1.777921in}}{\pgfqpoint{1.173472in}{1.769685in}}%
\pgfpathcurveto{\pgfqpoint{1.173472in}{1.761448in}}{\pgfqpoint{1.176744in}{1.753548in}}{\pgfqpoint{1.182568in}{1.747724in}}%
\pgfpathcurveto{\pgfqpoint{1.188392in}{1.741901in}}{\pgfqpoint{1.196292in}{1.738628in}}{\pgfqpoint{1.204529in}{1.738628in}}%
\pgfpathclose%
\pgfusepath{stroke,fill}%
\end{pgfscope}%
\begin{pgfscope}%
\pgfpathrectangle{\pgfqpoint{0.100000in}{0.212622in}}{\pgfqpoint{3.696000in}{3.696000in}}%
\pgfusepath{clip}%
\pgfsetbuttcap%
\pgfsetroundjoin%
\definecolor{currentfill}{rgb}{0.121569,0.466667,0.705882}%
\pgfsetfillcolor{currentfill}%
\pgfsetfillopacity{0.368105}%
\pgfsetlinewidth{1.003750pt}%
\definecolor{currentstroke}{rgb}{0.121569,0.466667,0.705882}%
\pgfsetstrokecolor{currentstroke}%
\pgfsetstrokeopacity{0.368105}%
\pgfsetdash{}{0pt}%
\pgfpathmoveto{\pgfqpoint{1.204529in}{1.738628in}}%
\pgfpathcurveto{\pgfqpoint{1.212765in}{1.738628in}}{\pgfqpoint{1.220665in}{1.741901in}}{\pgfqpoint{1.226489in}{1.747724in}}%
\pgfpathcurveto{\pgfqpoint{1.232313in}{1.753548in}}{\pgfqpoint{1.235585in}{1.761448in}}{\pgfqpoint{1.235585in}{1.769685in}}%
\pgfpathcurveto{\pgfqpoint{1.235585in}{1.777921in}}{\pgfqpoint{1.232313in}{1.785821in}}{\pgfqpoint{1.226489in}{1.791645in}}%
\pgfpathcurveto{\pgfqpoint{1.220665in}{1.797469in}}{\pgfqpoint{1.212765in}{1.800741in}}{\pgfqpoint{1.204529in}{1.800741in}}%
\pgfpathcurveto{\pgfqpoint{1.196292in}{1.800741in}}{\pgfqpoint{1.188392in}{1.797469in}}{\pgfqpoint{1.182568in}{1.791645in}}%
\pgfpathcurveto{\pgfqpoint{1.176744in}{1.785821in}}{\pgfqpoint{1.173472in}{1.777921in}}{\pgfqpoint{1.173472in}{1.769685in}}%
\pgfpathcurveto{\pgfqpoint{1.173472in}{1.761448in}}{\pgfqpoint{1.176744in}{1.753548in}}{\pgfqpoint{1.182568in}{1.747724in}}%
\pgfpathcurveto{\pgfqpoint{1.188392in}{1.741901in}}{\pgfqpoint{1.196292in}{1.738628in}}{\pgfqpoint{1.204529in}{1.738628in}}%
\pgfpathclose%
\pgfusepath{stroke,fill}%
\end{pgfscope}%
\begin{pgfscope}%
\pgfpathrectangle{\pgfqpoint{0.100000in}{0.212622in}}{\pgfqpoint{3.696000in}{3.696000in}}%
\pgfusepath{clip}%
\pgfsetbuttcap%
\pgfsetroundjoin%
\definecolor{currentfill}{rgb}{0.121569,0.466667,0.705882}%
\pgfsetfillcolor{currentfill}%
\pgfsetfillopacity{0.368105}%
\pgfsetlinewidth{1.003750pt}%
\definecolor{currentstroke}{rgb}{0.121569,0.466667,0.705882}%
\pgfsetstrokecolor{currentstroke}%
\pgfsetstrokeopacity{0.368105}%
\pgfsetdash{}{0pt}%
\pgfpathmoveto{\pgfqpoint{1.204529in}{1.738628in}}%
\pgfpathcurveto{\pgfqpoint{1.212765in}{1.738628in}}{\pgfqpoint{1.220665in}{1.741901in}}{\pgfqpoint{1.226489in}{1.747724in}}%
\pgfpathcurveto{\pgfqpoint{1.232313in}{1.753548in}}{\pgfqpoint{1.235585in}{1.761448in}}{\pgfqpoint{1.235585in}{1.769685in}}%
\pgfpathcurveto{\pgfqpoint{1.235585in}{1.777921in}}{\pgfqpoint{1.232313in}{1.785821in}}{\pgfqpoint{1.226489in}{1.791645in}}%
\pgfpathcurveto{\pgfqpoint{1.220665in}{1.797469in}}{\pgfqpoint{1.212765in}{1.800741in}}{\pgfqpoint{1.204529in}{1.800741in}}%
\pgfpathcurveto{\pgfqpoint{1.196292in}{1.800741in}}{\pgfqpoint{1.188392in}{1.797469in}}{\pgfqpoint{1.182568in}{1.791645in}}%
\pgfpathcurveto{\pgfqpoint{1.176744in}{1.785821in}}{\pgfqpoint{1.173472in}{1.777921in}}{\pgfqpoint{1.173472in}{1.769685in}}%
\pgfpathcurveto{\pgfqpoint{1.173472in}{1.761448in}}{\pgfqpoint{1.176744in}{1.753548in}}{\pgfqpoint{1.182568in}{1.747724in}}%
\pgfpathcurveto{\pgfqpoint{1.188392in}{1.741901in}}{\pgfqpoint{1.196292in}{1.738628in}}{\pgfqpoint{1.204529in}{1.738628in}}%
\pgfpathclose%
\pgfusepath{stroke,fill}%
\end{pgfscope}%
\begin{pgfscope}%
\pgfpathrectangle{\pgfqpoint{0.100000in}{0.212622in}}{\pgfqpoint{3.696000in}{3.696000in}}%
\pgfusepath{clip}%
\pgfsetbuttcap%
\pgfsetroundjoin%
\definecolor{currentfill}{rgb}{0.121569,0.466667,0.705882}%
\pgfsetfillcolor{currentfill}%
\pgfsetfillopacity{0.368105}%
\pgfsetlinewidth{1.003750pt}%
\definecolor{currentstroke}{rgb}{0.121569,0.466667,0.705882}%
\pgfsetstrokecolor{currentstroke}%
\pgfsetstrokeopacity{0.368105}%
\pgfsetdash{}{0pt}%
\pgfpathmoveto{\pgfqpoint{1.204529in}{1.738628in}}%
\pgfpathcurveto{\pgfqpoint{1.212765in}{1.738628in}}{\pgfqpoint{1.220665in}{1.741901in}}{\pgfqpoint{1.226489in}{1.747724in}}%
\pgfpathcurveto{\pgfqpoint{1.232313in}{1.753548in}}{\pgfqpoint{1.235585in}{1.761448in}}{\pgfqpoint{1.235585in}{1.769685in}}%
\pgfpathcurveto{\pgfqpoint{1.235585in}{1.777921in}}{\pgfqpoint{1.232313in}{1.785821in}}{\pgfqpoint{1.226489in}{1.791645in}}%
\pgfpathcurveto{\pgfqpoint{1.220665in}{1.797469in}}{\pgfqpoint{1.212765in}{1.800741in}}{\pgfqpoint{1.204529in}{1.800741in}}%
\pgfpathcurveto{\pgfqpoint{1.196292in}{1.800741in}}{\pgfqpoint{1.188392in}{1.797469in}}{\pgfqpoint{1.182568in}{1.791645in}}%
\pgfpathcurveto{\pgfqpoint{1.176744in}{1.785821in}}{\pgfqpoint{1.173472in}{1.777921in}}{\pgfqpoint{1.173472in}{1.769685in}}%
\pgfpathcurveto{\pgfqpoint{1.173472in}{1.761448in}}{\pgfqpoint{1.176744in}{1.753548in}}{\pgfqpoint{1.182568in}{1.747724in}}%
\pgfpathcurveto{\pgfqpoint{1.188392in}{1.741901in}}{\pgfqpoint{1.196292in}{1.738628in}}{\pgfqpoint{1.204529in}{1.738628in}}%
\pgfpathclose%
\pgfusepath{stroke,fill}%
\end{pgfscope}%
\begin{pgfscope}%
\pgfpathrectangle{\pgfqpoint{0.100000in}{0.212622in}}{\pgfqpoint{3.696000in}{3.696000in}}%
\pgfusepath{clip}%
\pgfsetbuttcap%
\pgfsetroundjoin%
\definecolor{currentfill}{rgb}{0.121569,0.466667,0.705882}%
\pgfsetfillcolor{currentfill}%
\pgfsetfillopacity{0.368105}%
\pgfsetlinewidth{1.003750pt}%
\definecolor{currentstroke}{rgb}{0.121569,0.466667,0.705882}%
\pgfsetstrokecolor{currentstroke}%
\pgfsetstrokeopacity{0.368105}%
\pgfsetdash{}{0pt}%
\pgfpathmoveto{\pgfqpoint{1.204529in}{1.738628in}}%
\pgfpathcurveto{\pgfqpoint{1.212765in}{1.738628in}}{\pgfqpoint{1.220665in}{1.741901in}}{\pgfqpoint{1.226489in}{1.747724in}}%
\pgfpathcurveto{\pgfqpoint{1.232313in}{1.753548in}}{\pgfqpoint{1.235585in}{1.761448in}}{\pgfqpoint{1.235585in}{1.769685in}}%
\pgfpathcurveto{\pgfqpoint{1.235585in}{1.777921in}}{\pgfqpoint{1.232313in}{1.785821in}}{\pgfqpoint{1.226489in}{1.791645in}}%
\pgfpathcurveto{\pgfqpoint{1.220665in}{1.797469in}}{\pgfqpoint{1.212765in}{1.800741in}}{\pgfqpoint{1.204529in}{1.800741in}}%
\pgfpathcurveto{\pgfqpoint{1.196292in}{1.800741in}}{\pgfqpoint{1.188392in}{1.797469in}}{\pgfqpoint{1.182568in}{1.791645in}}%
\pgfpathcurveto{\pgfqpoint{1.176744in}{1.785821in}}{\pgfqpoint{1.173472in}{1.777921in}}{\pgfqpoint{1.173472in}{1.769685in}}%
\pgfpathcurveto{\pgfqpoint{1.173472in}{1.761448in}}{\pgfqpoint{1.176744in}{1.753548in}}{\pgfqpoint{1.182568in}{1.747724in}}%
\pgfpathcurveto{\pgfqpoint{1.188392in}{1.741901in}}{\pgfqpoint{1.196292in}{1.738628in}}{\pgfqpoint{1.204529in}{1.738628in}}%
\pgfpathclose%
\pgfusepath{stroke,fill}%
\end{pgfscope}%
\begin{pgfscope}%
\pgfpathrectangle{\pgfqpoint{0.100000in}{0.212622in}}{\pgfqpoint{3.696000in}{3.696000in}}%
\pgfusepath{clip}%
\pgfsetbuttcap%
\pgfsetroundjoin%
\definecolor{currentfill}{rgb}{0.121569,0.466667,0.705882}%
\pgfsetfillcolor{currentfill}%
\pgfsetfillopacity{0.368105}%
\pgfsetlinewidth{1.003750pt}%
\definecolor{currentstroke}{rgb}{0.121569,0.466667,0.705882}%
\pgfsetstrokecolor{currentstroke}%
\pgfsetstrokeopacity{0.368105}%
\pgfsetdash{}{0pt}%
\pgfpathmoveto{\pgfqpoint{1.204529in}{1.738628in}}%
\pgfpathcurveto{\pgfqpoint{1.212765in}{1.738628in}}{\pgfqpoint{1.220665in}{1.741901in}}{\pgfqpoint{1.226489in}{1.747724in}}%
\pgfpathcurveto{\pgfqpoint{1.232313in}{1.753548in}}{\pgfqpoint{1.235585in}{1.761448in}}{\pgfqpoint{1.235585in}{1.769685in}}%
\pgfpathcurveto{\pgfqpoint{1.235585in}{1.777921in}}{\pgfqpoint{1.232313in}{1.785821in}}{\pgfqpoint{1.226489in}{1.791645in}}%
\pgfpathcurveto{\pgfqpoint{1.220665in}{1.797469in}}{\pgfqpoint{1.212765in}{1.800741in}}{\pgfqpoint{1.204529in}{1.800741in}}%
\pgfpathcurveto{\pgfqpoint{1.196292in}{1.800741in}}{\pgfqpoint{1.188392in}{1.797469in}}{\pgfqpoint{1.182568in}{1.791645in}}%
\pgfpathcurveto{\pgfqpoint{1.176744in}{1.785821in}}{\pgfqpoint{1.173472in}{1.777921in}}{\pgfqpoint{1.173472in}{1.769685in}}%
\pgfpathcurveto{\pgfqpoint{1.173472in}{1.761448in}}{\pgfqpoint{1.176744in}{1.753548in}}{\pgfqpoint{1.182568in}{1.747724in}}%
\pgfpathcurveto{\pgfqpoint{1.188392in}{1.741901in}}{\pgfqpoint{1.196292in}{1.738628in}}{\pgfqpoint{1.204529in}{1.738628in}}%
\pgfpathclose%
\pgfusepath{stroke,fill}%
\end{pgfscope}%
\begin{pgfscope}%
\pgfpathrectangle{\pgfqpoint{0.100000in}{0.212622in}}{\pgfqpoint{3.696000in}{3.696000in}}%
\pgfusepath{clip}%
\pgfsetbuttcap%
\pgfsetroundjoin%
\definecolor{currentfill}{rgb}{0.121569,0.466667,0.705882}%
\pgfsetfillcolor{currentfill}%
\pgfsetfillopacity{0.368105}%
\pgfsetlinewidth{1.003750pt}%
\definecolor{currentstroke}{rgb}{0.121569,0.466667,0.705882}%
\pgfsetstrokecolor{currentstroke}%
\pgfsetstrokeopacity{0.368105}%
\pgfsetdash{}{0pt}%
\pgfpathmoveto{\pgfqpoint{1.204529in}{1.738628in}}%
\pgfpathcurveto{\pgfqpoint{1.212765in}{1.738628in}}{\pgfqpoint{1.220665in}{1.741901in}}{\pgfqpoint{1.226489in}{1.747724in}}%
\pgfpathcurveto{\pgfqpoint{1.232313in}{1.753548in}}{\pgfqpoint{1.235585in}{1.761448in}}{\pgfqpoint{1.235585in}{1.769685in}}%
\pgfpathcurveto{\pgfqpoint{1.235585in}{1.777921in}}{\pgfqpoint{1.232313in}{1.785821in}}{\pgfqpoint{1.226489in}{1.791645in}}%
\pgfpathcurveto{\pgfqpoint{1.220665in}{1.797469in}}{\pgfqpoint{1.212765in}{1.800741in}}{\pgfqpoint{1.204529in}{1.800741in}}%
\pgfpathcurveto{\pgfqpoint{1.196292in}{1.800741in}}{\pgfqpoint{1.188392in}{1.797469in}}{\pgfqpoint{1.182568in}{1.791645in}}%
\pgfpathcurveto{\pgfqpoint{1.176744in}{1.785821in}}{\pgfqpoint{1.173472in}{1.777921in}}{\pgfqpoint{1.173472in}{1.769685in}}%
\pgfpathcurveto{\pgfqpoint{1.173472in}{1.761448in}}{\pgfqpoint{1.176744in}{1.753548in}}{\pgfqpoint{1.182568in}{1.747724in}}%
\pgfpathcurveto{\pgfqpoint{1.188392in}{1.741901in}}{\pgfqpoint{1.196292in}{1.738628in}}{\pgfqpoint{1.204529in}{1.738628in}}%
\pgfpathclose%
\pgfusepath{stroke,fill}%
\end{pgfscope}%
\begin{pgfscope}%
\pgfpathrectangle{\pgfqpoint{0.100000in}{0.212622in}}{\pgfqpoint{3.696000in}{3.696000in}}%
\pgfusepath{clip}%
\pgfsetbuttcap%
\pgfsetroundjoin%
\definecolor{currentfill}{rgb}{0.121569,0.466667,0.705882}%
\pgfsetfillcolor{currentfill}%
\pgfsetfillopacity{0.368105}%
\pgfsetlinewidth{1.003750pt}%
\definecolor{currentstroke}{rgb}{0.121569,0.466667,0.705882}%
\pgfsetstrokecolor{currentstroke}%
\pgfsetstrokeopacity{0.368105}%
\pgfsetdash{}{0pt}%
\pgfpathmoveto{\pgfqpoint{1.204529in}{1.738628in}}%
\pgfpathcurveto{\pgfqpoint{1.212765in}{1.738628in}}{\pgfqpoint{1.220665in}{1.741901in}}{\pgfqpoint{1.226489in}{1.747724in}}%
\pgfpathcurveto{\pgfqpoint{1.232313in}{1.753548in}}{\pgfqpoint{1.235585in}{1.761448in}}{\pgfqpoint{1.235585in}{1.769685in}}%
\pgfpathcurveto{\pgfqpoint{1.235585in}{1.777921in}}{\pgfqpoint{1.232313in}{1.785821in}}{\pgfqpoint{1.226489in}{1.791645in}}%
\pgfpathcurveto{\pgfqpoint{1.220665in}{1.797469in}}{\pgfqpoint{1.212765in}{1.800741in}}{\pgfqpoint{1.204529in}{1.800741in}}%
\pgfpathcurveto{\pgfqpoint{1.196292in}{1.800741in}}{\pgfqpoint{1.188392in}{1.797469in}}{\pgfqpoint{1.182568in}{1.791645in}}%
\pgfpathcurveto{\pgfqpoint{1.176744in}{1.785821in}}{\pgfqpoint{1.173472in}{1.777921in}}{\pgfqpoint{1.173472in}{1.769685in}}%
\pgfpathcurveto{\pgfqpoint{1.173472in}{1.761448in}}{\pgfqpoint{1.176744in}{1.753548in}}{\pgfqpoint{1.182568in}{1.747724in}}%
\pgfpathcurveto{\pgfqpoint{1.188392in}{1.741901in}}{\pgfqpoint{1.196292in}{1.738628in}}{\pgfqpoint{1.204529in}{1.738628in}}%
\pgfpathclose%
\pgfusepath{stroke,fill}%
\end{pgfscope}%
\begin{pgfscope}%
\pgfpathrectangle{\pgfqpoint{0.100000in}{0.212622in}}{\pgfqpoint{3.696000in}{3.696000in}}%
\pgfusepath{clip}%
\pgfsetbuttcap%
\pgfsetroundjoin%
\definecolor{currentfill}{rgb}{0.121569,0.466667,0.705882}%
\pgfsetfillcolor{currentfill}%
\pgfsetfillopacity{0.368105}%
\pgfsetlinewidth{1.003750pt}%
\definecolor{currentstroke}{rgb}{0.121569,0.466667,0.705882}%
\pgfsetstrokecolor{currentstroke}%
\pgfsetstrokeopacity{0.368105}%
\pgfsetdash{}{0pt}%
\pgfpathmoveto{\pgfqpoint{1.204529in}{1.738628in}}%
\pgfpathcurveto{\pgfqpoint{1.212765in}{1.738628in}}{\pgfqpoint{1.220665in}{1.741901in}}{\pgfqpoint{1.226489in}{1.747724in}}%
\pgfpathcurveto{\pgfqpoint{1.232313in}{1.753548in}}{\pgfqpoint{1.235585in}{1.761448in}}{\pgfqpoint{1.235585in}{1.769685in}}%
\pgfpathcurveto{\pgfqpoint{1.235585in}{1.777921in}}{\pgfqpoint{1.232313in}{1.785821in}}{\pgfqpoint{1.226489in}{1.791645in}}%
\pgfpathcurveto{\pgfqpoint{1.220665in}{1.797469in}}{\pgfqpoint{1.212765in}{1.800741in}}{\pgfqpoint{1.204529in}{1.800741in}}%
\pgfpathcurveto{\pgfqpoint{1.196292in}{1.800741in}}{\pgfqpoint{1.188392in}{1.797469in}}{\pgfqpoint{1.182568in}{1.791645in}}%
\pgfpathcurveto{\pgfqpoint{1.176744in}{1.785821in}}{\pgfqpoint{1.173472in}{1.777921in}}{\pgfqpoint{1.173472in}{1.769685in}}%
\pgfpathcurveto{\pgfqpoint{1.173472in}{1.761448in}}{\pgfqpoint{1.176744in}{1.753548in}}{\pgfqpoint{1.182568in}{1.747724in}}%
\pgfpathcurveto{\pgfqpoint{1.188392in}{1.741901in}}{\pgfqpoint{1.196292in}{1.738628in}}{\pgfqpoint{1.204529in}{1.738628in}}%
\pgfpathclose%
\pgfusepath{stroke,fill}%
\end{pgfscope}%
\begin{pgfscope}%
\pgfpathrectangle{\pgfqpoint{0.100000in}{0.212622in}}{\pgfqpoint{3.696000in}{3.696000in}}%
\pgfusepath{clip}%
\pgfsetbuttcap%
\pgfsetroundjoin%
\definecolor{currentfill}{rgb}{0.121569,0.466667,0.705882}%
\pgfsetfillcolor{currentfill}%
\pgfsetfillopacity{0.368105}%
\pgfsetlinewidth{1.003750pt}%
\definecolor{currentstroke}{rgb}{0.121569,0.466667,0.705882}%
\pgfsetstrokecolor{currentstroke}%
\pgfsetstrokeopacity{0.368105}%
\pgfsetdash{}{0pt}%
\pgfpathmoveto{\pgfqpoint{1.204529in}{1.738628in}}%
\pgfpathcurveto{\pgfqpoint{1.212765in}{1.738628in}}{\pgfqpoint{1.220665in}{1.741901in}}{\pgfqpoint{1.226489in}{1.747724in}}%
\pgfpathcurveto{\pgfqpoint{1.232313in}{1.753548in}}{\pgfqpoint{1.235585in}{1.761448in}}{\pgfqpoint{1.235585in}{1.769685in}}%
\pgfpathcurveto{\pgfqpoint{1.235585in}{1.777921in}}{\pgfqpoint{1.232313in}{1.785821in}}{\pgfqpoint{1.226489in}{1.791645in}}%
\pgfpathcurveto{\pgfqpoint{1.220665in}{1.797469in}}{\pgfqpoint{1.212765in}{1.800741in}}{\pgfqpoint{1.204529in}{1.800741in}}%
\pgfpathcurveto{\pgfqpoint{1.196292in}{1.800741in}}{\pgfqpoint{1.188392in}{1.797469in}}{\pgfqpoint{1.182568in}{1.791645in}}%
\pgfpathcurveto{\pgfqpoint{1.176744in}{1.785821in}}{\pgfqpoint{1.173472in}{1.777921in}}{\pgfqpoint{1.173472in}{1.769685in}}%
\pgfpathcurveto{\pgfqpoint{1.173472in}{1.761448in}}{\pgfqpoint{1.176744in}{1.753548in}}{\pgfqpoint{1.182568in}{1.747724in}}%
\pgfpathcurveto{\pgfqpoint{1.188392in}{1.741901in}}{\pgfqpoint{1.196292in}{1.738628in}}{\pgfqpoint{1.204529in}{1.738628in}}%
\pgfpathclose%
\pgfusepath{stroke,fill}%
\end{pgfscope}%
\begin{pgfscope}%
\pgfpathrectangle{\pgfqpoint{0.100000in}{0.212622in}}{\pgfqpoint{3.696000in}{3.696000in}}%
\pgfusepath{clip}%
\pgfsetbuttcap%
\pgfsetroundjoin%
\definecolor{currentfill}{rgb}{0.121569,0.466667,0.705882}%
\pgfsetfillcolor{currentfill}%
\pgfsetfillopacity{0.368105}%
\pgfsetlinewidth{1.003750pt}%
\definecolor{currentstroke}{rgb}{0.121569,0.466667,0.705882}%
\pgfsetstrokecolor{currentstroke}%
\pgfsetstrokeopacity{0.368105}%
\pgfsetdash{}{0pt}%
\pgfpathmoveto{\pgfqpoint{1.204529in}{1.738628in}}%
\pgfpathcurveto{\pgfqpoint{1.212765in}{1.738628in}}{\pgfqpoint{1.220665in}{1.741901in}}{\pgfqpoint{1.226489in}{1.747724in}}%
\pgfpathcurveto{\pgfqpoint{1.232313in}{1.753548in}}{\pgfqpoint{1.235585in}{1.761448in}}{\pgfqpoint{1.235585in}{1.769685in}}%
\pgfpathcurveto{\pgfqpoint{1.235585in}{1.777921in}}{\pgfqpoint{1.232313in}{1.785821in}}{\pgfqpoint{1.226489in}{1.791645in}}%
\pgfpathcurveto{\pgfqpoint{1.220665in}{1.797469in}}{\pgfqpoint{1.212765in}{1.800741in}}{\pgfqpoint{1.204529in}{1.800741in}}%
\pgfpathcurveto{\pgfqpoint{1.196292in}{1.800741in}}{\pgfqpoint{1.188392in}{1.797469in}}{\pgfqpoint{1.182568in}{1.791645in}}%
\pgfpathcurveto{\pgfqpoint{1.176744in}{1.785821in}}{\pgfqpoint{1.173472in}{1.777921in}}{\pgfqpoint{1.173472in}{1.769685in}}%
\pgfpathcurveto{\pgfqpoint{1.173472in}{1.761448in}}{\pgfqpoint{1.176744in}{1.753548in}}{\pgfqpoint{1.182568in}{1.747724in}}%
\pgfpathcurveto{\pgfqpoint{1.188392in}{1.741901in}}{\pgfqpoint{1.196292in}{1.738628in}}{\pgfqpoint{1.204529in}{1.738628in}}%
\pgfpathclose%
\pgfusepath{stroke,fill}%
\end{pgfscope}%
\begin{pgfscope}%
\pgfpathrectangle{\pgfqpoint{0.100000in}{0.212622in}}{\pgfqpoint{3.696000in}{3.696000in}}%
\pgfusepath{clip}%
\pgfsetbuttcap%
\pgfsetroundjoin%
\definecolor{currentfill}{rgb}{0.121569,0.466667,0.705882}%
\pgfsetfillcolor{currentfill}%
\pgfsetfillopacity{0.368105}%
\pgfsetlinewidth{1.003750pt}%
\definecolor{currentstroke}{rgb}{0.121569,0.466667,0.705882}%
\pgfsetstrokecolor{currentstroke}%
\pgfsetstrokeopacity{0.368105}%
\pgfsetdash{}{0pt}%
\pgfpathmoveto{\pgfqpoint{1.204529in}{1.738628in}}%
\pgfpathcurveto{\pgfqpoint{1.212765in}{1.738628in}}{\pgfqpoint{1.220665in}{1.741901in}}{\pgfqpoint{1.226489in}{1.747724in}}%
\pgfpathcurveto{\pgfqpoint{1.232313in}{1.753548in}}{\pgfqpoint{1.235585in}{1.761448in}}{\pgfqpoint{1.235585in}{1.769685in}}%
\pgfpathcurveto{\pgfqpoint{1.235585in}{1.777921in}}{\pgfqpoint{1.232313in}{1.785821in}}{\pgfqpoint{1.226489in}{1.791645in}}%
\pgfpathcurveto{\pgfqpoint{1.220665in}{1.797469in}}{\pgfqpoint{1.212765in}{1.800741in}}{\pgfqpoint{1.204529in}{1.800741in}}%
\pgfpathcurveto{\pgfqpoint{1.196292in}{1.800741in}}{\pgfqpoint{1.188392in}{1.797469in}}{\pgfqpoint{1.182568in}{1.791645in}}%
\pgfpathcurveto{\pgfqpoint{1.176744in}{1.785821in}}{\pgfqpoint{1.173472in}{1.777921in}}{\pgfqpoint{1.173472in}{1.769685in}}%
\pgfpathcurveto{\pgfqpoint{1.173472in}{1.761448in}}{\pgfqpoint{1.176744in}{1.753548in}}{\pgfqpoint{1.182568in}{1.747724in}}%
\pgfpathcurveto{\pgfqpoint{1.188392in}{1.741901in}}{\pgfqpoint{1.196292in}{1.738628in}}{\pgfqpoint{1.204529in}{1.738628in}}%
\pgfpathclose%
\pgfusepath{stroke,fill}%
\end{pgfscope}%
\begin{pgfscope}%
\pgfpathrectangle{\pgfqpoint{0.100000in}{0.212622in}}{\pgfqpoint{3.696000in}{3.696000in}}%
\pgfusepath{clip}%
\pgfsetbuttcap%
\pgfsetroundjoin%
\definecolor{currentfill}{rgb}{0.121569,0.466667,0.705882}%
\pgfsetfillcolor{currentfill}%
\pgfsetfillopacity{0.368105}%
\pgfsetlinewidth{1.003750pt}%
\definecolor{currentstroke}{rgb}{0.121569,0.466667,0.705882}%
\pgfsetstrokecolor{currentstroke}%
\pgfsetstrokeopacity{0.368105}%
\pgfsetdash{}{0pt}%
\pgfpathmoveto{\pgfqpoint{1.204529in}{1.738628in}}%
\pgfpathcurveto{\pgfqpoint{1.212765in}{1.738628in}}{\pgfqpoint{1.220665in}{1.741901in}}{\pgfqpoint{1.226489in}{1.747724in}}%
\pgfpathcurveto{\pgfqpoint{1.232313in}{1.753548in}}{\pgfqpoint{1.235585in}{1.761448in}}{\pgfqpoint{1.235585in}{1.769685in}}%
\pgfpathcurveto{\pgfqpoint{1.235585in}{1.777921in}}{\pgfqpoint{1.232313in}{1.785821in}}{\pgfqpoint{1.226489in}{1.791645in}}%
\pgfpathcurveto{\pgfqpoint{1.220665in}{1.797469in}}{\pgfqpoint{1.212765in}{1.800741in}}{\pgfqpoint{1.204529in}{1.800741in}}%
\pgfpathcurveto{\pgfqpoint{1.196292in}{1.800741in}}{\pgfqpoint{1.188392in}{1.797469in}}{\pgfqpoint{1.182568in}{1.791645in}}%
\pgfpathcurveto{\pgfqpoint{1.176744in}{1.785821in}}{\pgfqpoint{1.173472in}{1.777921in}}{\pgfqpoint{1.173472in}{1.769685in}}%
\pgfpathcurveto{\pgfqpoint{1.173472in}{1.761448in}}{\pgfqpoint{1.176744in}{1.753548in}}{\pgfqpoint{1.182568in}{1.747724in}}%
\pgfpathcurveto{\pgfqpoint{1.188392in}{1.741901in}}{\pgfqpoint{1.196292in}{1.738628in}}{\pgfqpoint{1.204529in}{1.738628in}}%
\pgfpathclose%
\pgfusepath{stroke,fill}%
\end{pgfscope}%
\begin{pgfscope}%
\pgfpathrectangle{\pgfqpoint{0.100000in}{0.212622in}}{\pgfqpoint{3.696000in}{3.696000in}}%
\pgfusepath{clip}%
\pgfsetbuttcap%
\pgfsetroundjoin%
\definecolor{currentfill}{rgb}{0.121569,0.466667,0.705882}%
\pgfsetfillcolor{currentfill}%
\pgfsetfillopacity{0.368105}%
\pgfsetlinewidth{1.003750pt}%
\definecolor{currentstroke}{rgb}{0.121569,0.466667,0.705882}%
\pgfsetstrokecolor{currentstroke}%
\pgfsetstrokeopacity{0.368105}%
\pgfsetdash{}{0pt}%
\pgfpathmoveto{\pgfqpoint{1.204529in}{1.738628in}}%
\pgfpathcurveto{\pgfqpoint{1.212765in}{1.738628in}}{\pgfqpoint{1.220665in}{1.741901in}}{\pgfqpoint{1.226489in}{1.747724in}}%
\pgfpathcurveto{\pgfqpoint{1.232313in}{1.753548in}}{\pgfqpoint{1.235585in}{1.761448in}}{\pgfqpoint{1.235585in}{1.769685in}}%
\pgfpathcurveto{\pgfqpoint{1.235585in}{1.777921in}}{\pgfqpoint{1.232313in}{1.785821in}}{\pgfqpoint{1.226489in}{1.791645in}}%
\pgfpathcurveto{\pgfqpoint{1.220665in}{1.797469in}}{\pgfqpoint{1.212765in}{1.800741in}}{\pgfqpoint{1.204529in}{1.800741in}}%
\pgfpathcurveto{\pgfqpoint{1.196292in}{1.800741in}}{\pgfqpoint{1.188392in}{1.797469in}}{\pgfqpoint{1.182568in}{1.791645in}}%
\pgfpathcurveto{\pgfqpoint{1.176744in}{1.785821in}}{\pgfqpoint{1.173472in}{1.777921in}}{\pgfqpoint{1.173472in}{1.769685in}}%
\pgfpathcurveto{\pgfqpoint{1.173472in}{1.761448in}}{\pgfqpoint{1.176744in}{1.753548in}}{\pgfqpoint{1.182568in}{1.747724in}}%
\pgfpathcurveto{\pgfqpoint{1.188392in}{1.741901in}}{\pgfqpoint{1.196292in}{1.738628in}}{\pgfqpoint{1.204529in}{1.738628in}}%
\pgfpathclose%
\pgfusepath{stroke,fill}%
\end{pgfscope}%
\begin{pgfscope}%
\pgfpathrectangle{\pgfqpoint{0.100000in}{0.212622in}}{\pgfqpoint{3.696000in}{3.696000in}}%
\pgfusepath{clip}%
\pgfsetbuttcap%
\pgfsetroundjoin%
\definecolor{currentfill}{rgb}{0.121569,0.466667,0.705882}%
\pgfsetfillcolor{currentfill}%
\pgfsetfillopacity{0.368105}%
\pgfsetlinewidth{1.003750pt}%
\definecolor{currentstroke}{rgb}{0.121569,0.466667,0.705882}%
\pgfsetstrokecolor{currentstroke}%
\pgfsetstrokeopacity{0.368105}%
\pgfsetdash{}{0pt}%
\pgfpathmoveto{\pgfqpoint{1.204529in}{1.738628in}}%
\pgfpathcurveto{\pgfqpoint{1.212765in}{1.738628in}}{\pgfqpoint{1.220665in}{1.741901in}}{\pgfqpoint{1.226489in}{1.747724in}}%
\pgfpathcurveto{\pgfqpoint{1.232313in}{1.753548in}}{\pgfqpoint{1.235585in}{1.761448in}}{\pgfqpoint{1.235585in}{1.769685in}}%
\pgfpathcurveto{\pgfqpoint{1.235585in}{1.777921in}}{\pgfqpoint{1.232313in}{1.785821in}}{\pgfqpoint{1.226489in}{1.791645in}}%
\pgfpathcurveto{\pgfqpoint{1.220665in}{1.797469in}}{\pgfqpoint{1.212765in}{1.800741in}}{\pgfqpoint{1.204529in}{1.800741in}}%
\pgfpathcurveto{\pgfqpoint{1.196292in}{1.800741in}}{\pgfqpoint{1.188392in}{1.797469in}}{\pgfqpoint{1.182568in}{1.791645in}}%
\pgfpathcurveto{\pgfqpoint{1.176744in}{1.785821in}}{\pgfqpoint{1.173472in}{1.777921in}}{\pgfqpoint{1.173472in}{1.769685in}}%
\pgfpathcurveto{\pgfqpoint{1.173472in}{1.761448in}}{\pgfqpoint{1.176744in}{1.753548in}}{\pgfqpoint{1.182568in}{1.747724in}}%
\pgfpathcurveto{\pgfqpoint{1.188392in}{1.741901in}}{\pgfqpoint{1.196292in}{1.738628in}}{\pgfqpoint{1.204529in}{1.738628in}}%
\pgfpathclose%
\pgfusepath{stroke,fill}%
\end{pgfscope}%
\begin{pgfscope}%
\pgfpathrectangle{\pgfqpoint{0.100000in}{0.212622in}}{\pgfqpoint{3.696000in}{3.696000in}}%
\pgfusepath{clip}%
\pgfsetbuttcap%
\pgfsetroundjoin%
\definecolor{currentfill}{rgb}{0.121569,0.466667,0.705882}%
\pgfsetfillcolor{currentfill}%
\pgfsetfillopacity{0.368105}%
\pgfsetlinewidth{1.003750pt}%
\definecolor{currentstroke}{rgb}{0.121569,0.466667,0.705882}%
\pgfsetstrokecolor{currentstroke}%
\pgfsetstrokeopacity{0.368105}%
\pgfsetdash{}{0pt}%
\pgfpathmoveto{\pgfqpoint{1.204529in}{1.738628in}}%
\pgfpathcurveto{\pgfqpoint{1.212765in}{1.738628in}}{\pgfqpoint{1.220665in}{1.741901in}}{\pgfqpoint{1.226489in}{1.747724in}}%
\pgfpathcurveto{\pgfqpoint{1.232313in}{1.753548in}}{\pgfqpoint{1.235585in}{1.761448in}}{\pgfqpoint{1.235585in}{1.769685in}}%
\pgfpathcurveto{\pgfqpoint{1.235585in}{1.777921in}}{\pgfqpoint{1.232313in}{1.785821in}}{\pgfqpoint{1.226489in}{1.791645in}}%
\pgfpathcurveto{\pgfqpoint{1.220665in}{1.797469in}}{\pgfqpoint{1.212765in}{1.800741in}}{\pgfqpoint{1.204529in}{1.800741in}}%
\pgfpathcurveto{\pgfqpoint{1.196292in}{1.800741in}}{\pgfqpoint{1.188392in}{1.797469in}}{\pgfqpoint{1.182568in}{1.791645in}}%
\pgfpathcurveto{\pgfqpoint{1.176744in}{1.785821in}}{\pgfqpoint{1.173472in}{1.777921in}}{\pgfqpoint{1.173472in}{1.769685in}}%
\pgfpathcurveto{\pgfqpoint{1.173472in}{1.761448in}}{\pgfqpoint{1.176744in}{1.753548in}}{\pgfqpoint{1.182568in}{1.747724in}}%
\pgfpathcurveto{\pgfqpoint{1.188392in}{1.741901in}}{\pgfqpoint{1.196292in}{1.738628in}}{\pgfqpoint{1.204529in}{1.738628in}}%
\pgfpathclose%
\pgfusepath{stroke,fill}%
\end{pgfscope}%
\begin{pgfscope}%
\pgfpathrectangle{\pgfqpoint{0.100000in}{0.212622in}}{\pgfqpoint{3.696000in}{3.696000in}}%
\pgfusepath{clip}%
\pgfsetbuttcap%
\pgfsetroundjoin%
\definecolor{currentfill}{rgb}{0.121569,0.466667,0.705882}%
\pgfsetfillcolor{currentfill}%
\pgfsetfillopacity{0.368105}%
\pgfsetlinewidth{1.003750pt}%
\definecolor{currentstroke}{rgb}{0.121569,0.466667,0.705882}%
\pgfsetstrokecolor{currentstroke}%
\pgfsetstrokeopacity{0.368105}%
\pgfsetdash{}{0pt}%
\pgfpathmoveto{\pgfqpoint{1.204529in}{1.738628in}}%
\pgfpathcurveto{\pgfqpoint{1.212765in}{1.738628in}}{\pgfqpoint{1.220665in}{1.741901in}}{\pgfqpoint{1.226489in}{1.747724in}}%
\pgfpathcurveto{\pgfqpoint{1.232313in}{1.753548in}}{\pgfqpoint{1.235585in}{1.761448in}}{\pgfqpoint{1.235585in}{1.769685in}}%
\pgfpathcurveto{\pgfqpoint{1.235585in}{1.777921in}}{\pgfqpoint{1.232313in}{1.785821in}}{\pgfqpoint{1.226489in}{1.791645in}}%
\pgfpathcurveto{\pgfqpoint{1.220665in}{1.797469in}}{\pgfqpoint{1.212765in}{1.800741in}}{\pgfqpoint{1.204529in}{1.800741in}}%
\pgfpathcurveto{\pgfqpoint{1.196292in}{1.800741in}}{\pgfqpoint{1.188392in}{1.797469in}}{\pgfqpoint{1.182568in}{1.791645in}}%
\pgfpathcurveto{\pgfqpoint{1.176744in}{1.785821in}}{\pgfqpoint{1.173472in}{1.777921in}}{\pgfqpoint{1.173472in}{1.769685in}}%
\pgfpathcurveto{\pgfqpoint{1.173472in}{1.761448in}}{\pgfqpoint{1.176744in}{1.753548in}}{\pgfqpoint{1.182568in}{1.747724in}}%
\pgfpathcurveto{\pgfqpoint{1.188392in}{1.741901in}}{\pgfqpoint{1.196292in}{1.738628in}}{\pgfqpoint{1.204529in}{1.738628in}}%
\pgfpathclose%
\pgfusepath{stroke,fill}%
\end{pgfscope}%
\begin{pgfscope}%
\pgfpathrectangle{\pgfqpoint{0.100000in}{0.212622in}}{\pgfqpoint{3.696000in}{3.696000in}}%
\pgfusepath{clip}%
\pgfsetbuttcap%
\pgfsetroundjoin%
\definecolor{currentfill}{rgb}{0.121569,0.466667,0.705882}%
\pgfsetfillcolor{currentfill}%
\pgfsetfillopacity{0.368105}%
\pgfsetlinewidth{1.003750pt}%
\definecolor{currentstroke}{rgb}{0.121569,0.466667,0.705882}%
\pgfsetstrokecolor{currentstroke}%
\pgfsetstrokeopacity{0.368105}%
\pgfsetdash{}{0pt}%
\pgfpathmoveto{\pgfqpoint{1.204529in}{1.738628in}}%
\pgfpathcurveto{\pgfqpoint{1.212765in}{1.738628in}}{\pgfqpoint{1.220665in}{1.741901in}}{\pgfqpoint{1.226489in}{1.747724in}}%
\pgfpathcurveto{\pgfqpoint{1.232313in}{1.753548in}}{\pgfqpoint{1.235585in}{1.761448in}}{\pgfqpoint{1.235585in}{1.769685in}}%
\pgfpathcurveto{\pgfqpoint{1.235585in}{1.777921in}}{\pgfqpoint{1.232313in}{1.785821in}}{\pgfqpoint{1.226489in}{1.791645in}}%
\pgfpathcurveto{\pgfqpoint{1.220665in}{1.797469in}}{\pgfqpoint{1.212765in}{1.800741in}}{\pgfqpoint{1.204529in}{1.800741in}}%
\pgfpathcurveto{\pgfqpoint{1.196292in}{1.800741in}}{\pgfqpoint{1.188392in}{1.797469in}}{\pgfqpoint{1.182568in}{1.791645in}}%
\pgfpathcurveto{\pgfqpoint{1.176744in}{1.785821in}}{\pgfqpoint{1.173472in}{1.777921in}}{\pgfqpoint{1.173472in}{1.769685in}}%
\pgfpathcurveto{\pgfqpoint{1.173472in}{1.761448in}}{\pgfqpoint{1.176744in}{1.753548in}}{\pgfqpoint{1.182568in}{1.747724in}}%
\pgfpathcurveto{\pgfqpoint{1.188392in}{1.741901in}}{\pgfqpoint{1.196292in}{1.738628in}}{\pgfqpoint{1.204529in}{1.738628in}}%
\pgfpathclose%
\pgfusepath{stroke,fill}%
\end{pgfscope}%
\begin{pgfscope}%
\pgfpathrectangle{\pgfqpoint{0.100000in}{0.212622in}}{\pgfqpoint{3.696000in}{3.696000in}}%
\pgfusepath{clip}%
\pgfsetbuttcap%
\pgfsetroundjoin%
\definecolor{currentfill}{rgb}{0.121569,0.466667,0.705882}%
\pgfsetfillcolor{currentfill}%
\pgfsetfillopacity{0.368105}%
\pgfsetlinewidth{1.003750pt}%
\definecolor{currentstroke}{rgb}{0.121569,0.466667,0.705882}%
\pgfsetstrokecolor{currentstroke}%
\pgfsetstrokeopacity{0.368105}%
\pgfsetdash{}{0pt}%
\pgfpathmoveto{\pgfqpoint{1.204529in}{1.738628in}}%
\pgfpathcurveto{\pgfqpoint{1.212765in}{1.738628in}}{\pgfqpoint{1.220665in}{1.741901in}}{\pgfqpoint{1.226489in}{1.747724in}}%
\pgfpathcurveto{\pgfqpoint{1.232313in}{1.753548in}}{\pgfqpoint{1.235585in}{1.761448in}}{\pgfqpoint{1.235585in}{1.769685in}}%
\pgfpathcurveto{\pgfqpoint{1.235585in}{1.777921in}}{\pgfqpoint{1.232313in}{1.785821in}}{\pgfqpoint{1.226489in}{1.791645in}}%
\pgfpathcurveto{\pgfqpoint{1.220665in}{1.797469in}}{\pgfqpoint{1.212765in}{1.800741in}}{\pgfqpoint{1.204529in}{1.800741in}}%
\pgfpathcurveto{\pgfqpoint{1.196292in}{1.800741in}}{\pgfqpoint{1.188392in}{1.797469in}}{\pgfqpoint{1.182568in}{1.791645in}}%
\pgfpathcurveto{\pgfqpoint{1.176744in}{1.785821in}}{\pgfqpoint{1.173472in}{1.777921in}}{\pgfqpoint{1.173472in}{1.769685in}}%
\pgfpathcurveto{\pgfqpoint{1.173472in}{1.761448in}}{\pgfqpoint{1.176744in}{1.753548in}}{\pgfqpoint{1.182568in}{1.747724in}}%
\pgfpathcurveto{\pgfqpoint{1.188392in}{1.741901in}}{\pgfqpoint{1.196292in}{1.738628in}}{\pgfqpoint{1.204529in}{1.738628in}}%
\pgfpathclose%
\pgfusepath{stroke,fill}%
\end{pgfscope}%
\begin{pgfscope}%
\pgfpathrectangle{\pgfqpoint{0.100000in}{0.212622in}}{\pgfqpoint{3.696000in}{3.696000in}}%
\pgfusepath{clip}%
\pgfsetbuttcap%
\pgfsetroundjoin%
\definecolor{currentfill}{rgb}{0.121569,0.466667,0.705882}%
\pgfsetfillcolor{currentfill}%
\pgfsetfillopacity{0.368105}%
\pgfsetlinewidth{1.003750pt}%
\definecolor{currentstroke}{rgb}{0.121569,0.466667,0.705882}%
\pgfsetstrokecolor{currentstroke}%
\pgfsetstrokeopacity{0.368105}%
\pgfsetdash{}{0pt}%
\pgfpathmoveto{\pgfqpoint{1.204529in}{1.738628in}}%
\pgfpathcurveto{\pgfqpoint{1.212765in}{1.738628in}}{\pgfqpoint{1.220665in}{1.741901in}}{\pgfqpoint{1.226489in}{1.747724in}}%
\pgfpathcurveto{\pgfqpoint{1.232313in}{1.753548in}}{\pgfqpoint{1.235585in}{1.761448in}}{\pgfqpoint{1.235585in}{1.769685in}}%
\pgfpathcurveto{\pgfqpoint{1.235585in}{1.777921in}}{\pgfqpoint{1.232313in}{1.785821in}}{\pgfqpoint{1.226489in}{1.791645in}}%
\pgfpathcurveto{\pgfqpoint{1.220665in}{1.797469in}}{\pgfqpoint{1.212765in}{1.800741in}}{\pgfqpoint{1.204529in}{1.800741in}}%
\pgfpathcurveto{\pgfqpoint{1.196292in}{1.800741in}}{\pgfqpoint{1.188392in}{1.797469in}}{\pgfqpoint{1.182568in}{1.791645in}}%
\pgfpathcurveto{\pgfqpoint{1.176744in}{1.785821in}}{\pgfqpoint{1.173472in}{1.777921in}}{\pgfqpoint{1.173472in}{1.769685in}}%
\pgfpathcurveto{\pgfqpoint{1.173472in}{1.761448in}}{\pgfqpoint{1.176744in}{1.753548in}}{\pgfqpoint{1.182568in}{1.747724in}}%
\pgfpathcurveto{\pgfqpoint{1.188392in}{1.741901in}}{\pgfqpoint{1.196292in}{1.738628in}}{\pgfqpoint{1.204529in}{1.738628in}}%
\pgfpathclose%
\pgfusepath{stroke,fill}%
\end{pgfscope}%
\begin{pgfscope}%
\pgfpathrectangle{\pgfqpoint{0.100000in}{0.212622in}}{\pgfqpoint{3.696000in}{3.696000in}}%
\pgfusepath{clip}%
\pgfsetbuttcap%
\pgfsetroundjoin%
\definecolor{currentfill}{rgb}{0.121569,0.466667,0.705882}%
\pgfsetfillcolor{currentfill}%
\pgfsetfillopacity{0.368105}%
\pgfsetlinewidth{1.003750pt}%
\definecolor{currentstroke}{rgb}{0.121569,0.466667,0.705882}%
\pgfsetstrokecolor{currentstroke}%
\pgfsetstrokeopacity{0.368105}%
\pgfsetdash{}{0pt}%
\pgfpathmoveto{\pgfqpoint{1.204529in}{1.738628in}}%
\pgfpathcurveto{\pgfqpoint{1.212765in}{1.738628in}}{\pgfqpoint{1.220665in}{1.741901in}}{\pgfqpoint{1.226489in}{1.747724in}}%
\pgfpathcurveto{\pgfqpoint{1.232313in}{1.753548in}}{\pgfqpoint{1.235585in}{1.761448in}}{\pgfqpoint{1.235585in}{1.769685in}}%
\pgfpathcurveto{\pgfqpoint{1.235585in}{1.777921in}}{\pgfqpoint{1.232313in}{1.785821in}}{\pgfqpoint{1.226489in}{1.791645in}}%
\pgfpathcurveto{\pgfqpoint{1.220665in}{1.797469in}}{\pgfqpoint{1.212765in}{1.800741in}}{\pgfqpoint{1.204529in}{1.800741in}}%
\pgfpathcurveto{\pgfqpoint{1.196292in}{1.800741in}}{\pgfqpoint{1.188392in}{1.797469in}}{\pgfqpoint{1.182568in}{1.791645in}}%
\pgfpathcurveto{\pgfqpoint{1.176744in}{1.785821in}}{\pgfqpoint{1.173472in}{1.777921in}}{\pgfqpoint{1.173472in}{1.769685in}}%
\pgfpathcurveto{\pgfqpoint{1.173472in}{1.761448in}}{\pgfqpoint{1.176744in}{1.753548in}}{\pgfqpoint{1.182568in}{1.747724in}}%
\pgfpathcurveto{\pgfqpoint{1.188392in}{1.741901in}}{\pgfqpoint{1.196292in}{1.738628in}}{\pgfqpoint{1.204529in}{1.738628in}}%
\pgfpathclose%
\pgfusepath{stroke,fill}%
\end{pgfscope}%
\begin{pgfscope}%
\pgfpathrectangle{\pgfqpoint{0.100000in}{0.212622in}}{\pgfqpoint{3.696000in}{3.696000in}}%
\pgfusepath{clip}%
\pgfsetbuttcap%
\pgfsetroundjoin%
\definecolor{currentfill}{rgb}{0.121569,0.466667,0.705882}%
\pgfsetfillcolor{currentfill}%
\pgfsetfillopacity{0.368105}%
\pgfsetlinewidth{1.003750pt}%
\definecolor{currentstroke}{rgb}{0.121569,0.466667,0.705882}%
\pgfsetstrokecolor{currentstroke}%
\pgfsetstrokeopacity{0.368105}%
\pgfsetdash{}{0pt}%
\pgfpathmoveto{\pgfqpoint{1.204529in}{1.738628in}}%
\pgfpathcurveto{\pgfqpoint{1.212765in}{1.738628in}}{\pgfqpoint{1.220665in}{1.741901in}}{\pgfqpoint{1.226489in}{1.747724in}}%
\pgfpathcurveto{\pgfqpoint{1.232313in}{1.753548in}}{\pgfqpoint{1.235585in}{1.761448in}}{\pgfqpoint{1.235585in}{1.769685in}}%
\pgfpathcurveto{\pgfqpoint{1.235585in}{1.777921in}}{\pgfqpoint{1.232313in}{1.785821in}}{\pgfqpoint{1.226489in}{1.791645in}}%
\pgfpathcurveto{\pgfqpoint{1.220665in}{1.797469in}}{\pgfqpoint{1.212765in}{1.800741in}}{\pgfqpoint{1.204529in}{1.800741in}}%
\pgfpathcurveto{\pgfqpoint{1.196292in}{1.800741in}}{\pgfqpoint{1.188392in}{1.797469in}}{\pgfqpoint{1.182568in}{1.791645in}}%
\pgfpathcurveto{\pgfqpoint{1.176744in}{1.785821in}}{\pgfqpoint{1.173472in}{1.777921in}}{\pgfqpoint{1.173472in}{1.769685in}}%
\pgfpathcurveto{\pgfqpoint{1.173472in}{1.761448in}}{\pgfqpoint{1.176744in}{1.753548in}}{\pgfqpoint{1.182568in}{1.747724in}}%
\pgfpathcurveto{\pgfqpoint{1.188392in}{1.741901in}}{\pgfqpoint{1.196292in}{1.738628in}}{\pgfqpoint{1.204529in}{1.738628in}}%
\pgfpathclose%
\pgfusepath{stroke,fill}%
\end{pgfscope}%
\begin{pgfscope}%
\pgfpathrectangle{\pgfqpoint{0.100000in}{0.212622in}}{\pgfqpoint{3.696000in}{3.696000in}}%
\pgfusepath{clip}%
\pgfsetbuttcap%
\pgfsetroundjoin%
\definecolor{currentfill}{rgb}{0.121569,0.466667,0.705882}%
\pgfsetfillcolor{currentfill}%
\pgfsetfillopacity{0.368105}%
\pgfsetlinewidth{1.003750pt}%
\definecolor{currentstroke}{rgb}{0.121569,0.466667,0.705882}%
\pgfsetstrokecolor{currentstroke}%
\pgfsetstrokeopacity{0.368105}%
\pgfsetdash{}{0pt}%
\pgfpathmoveto{\pgfqpoint{1.204529in}{1.738628in}}%
\pgfpathcurveto{\pgfqpoint{1.212765in}{1.738628in}}{\pgfqpoint{1.220665in}{1.741901in}}{\pgfqpoint{1.226489in}{1.747724in}}%
\pgfpathcurveto{\pgfqpoint{1.232313in}{1.753548in}}{\pgfqpoint{1.235585in}{1.761448in}}{\pgfqpoint{1.235585in}{1.769685in}}%
\pgfpathcurveto{\pgfqpoint{1.235585in}{1.777921in}}{\pgfqpoint{1.232313in}{1.785821in}}{\pgfqpoint{1.226489in}{1.791645in}}%
\pgfpathcurveto{\pgfqpoint{1.220665in}{1.797469in}}{\pgfqpoint{1.212765in}{1.800741in}}{\pgfqpoint{1.204529in}{1.800741in}}%
\pgfpathcurveto{\pgfqpoint{1.196292in}{1.800741in}}{\pgfqpoint{1.188392in}{1.797469in}}{\pgfqpoint{1.182568in}{1.791645in}}%
\pgfpathcurveto{\pgfqpoint{1.176744in}{1.785821in}}{\pgfqpoint{1.173472in}{1.777921in}}{\pgfqpoint{1.173472in}{1.769685in}}%
\pgfpathcurveto{\pgfqpoint{1.173472in}{1.761448in}}{\pgfqpoint{1.176744in}{1.753548in}}{\pgfqpoint{1.182568in}{1.747724in}}%
\pgfpathcurveto{\pgfqpoint{1.188392in}{1.741901in}}{\pgfqpoint{1.196292in}{1.738628in}}{\pgfqpoint{1.204529in}{1.738628in}}%
\pgfpathclose%
\pgfusepath{stroke,fill}%
\end{pgfscope}%
\begin{pgfscope}%
\pgfpathrectangle{\pgfqpoint{0.100000in}{0.212622in}}{\pgfqpoint{3.696000in}{3.696000in}}%
\pgfusepath{clip}%
\pgfsetbuttcap%
\pgfsetroundjoin%
\definecolor{currentfill}{rgb}{0.121569,0.466667,0.705882}%
\pgfsetfillcolor{currentfill}%
\pgfsetfillopacity{0.368105}%
\pgfsetlinewidth{1.003750pt}%
\definecolor{currentstroke}{rgb}{0.121569,0.466667,0.705882}%
\pgfsetstrokecolor{currentstroke}%
\pgfsetstrokeopacity{0.368105}%
\pgfsetdash{}{0pt}%
\pgfpathmoveto{\pgfqpoint{1.204529in}{1.738628in}}%
\pgfpathcurveto{\pgfqpoint{1.212765in}{1.738628in}}{\pgfqpoint{1.220665in}{1.741901in}}{\pgfqpoint{1.226489in}{1.747724in}}%
\pgfpathcurveto{\pgfqpoint{1.232313in}{1.753548in}}{\pgfqpoint{1.235585in}{1.761448in}}{\pgfqpoint{1.235585in}{1.769685in}}%
\pgfpathcurveto{\pgfqpoint{1.235585in}{1.777921in}}{\pgfqpoint{1.232313in}{1.785821in}}{\pgfqpoint{1.226489in}{1.791645in}}%
\pgfpathcurveto{\pgfqpoint{1.220665in}{1.797469in}}{\pgfqpoint{1.212765in}{1.800741in}}{\pgfqpoint{1.204529in}{1.800741in}}%
\pgfpathcurveto{\pgfqpoint{1.196292in}{1.800741in}}{\pgfqpoint{1.188392in}{1.797469in}}{\pgfqpoint{1.182568in}{1.791645in}}%
\pgfpathcurveto{\pgfqpoint{1.176744in}{1.785821in}}{\pgfqpoint{1.173472in}{1.777921in}}{\pgfqpoint{1.173472in}{1.769685in}}%
\pgfpathcurveto{\pgfqpoint{1.173472in}{1.761448in}}{\pgfqpoint{1.176744in}{1.753548in}}{\pgfqpoint{1.182568in}{1.747724in}}%
\pgfpathcurveto{\pgfqpoint{1.188392in}{1.741901in}}{\pgfqpoint{1.196292in}{1.738628in}}{\pgfqpoint{1.204529in}{1.738628in}}%
\pgfpathclose%
\pgfusepath{stroke,fill}%
\end{pgfscope}%
\begin{pgfscope}%
\pgfpathrectangle{\pgfqpoint{0.100000in}{0.212622in}}{\pgfqpoint{3.696000in}{3.696000in}}%
\pgfusepath{clip}%
\pgfsetbuttcap%
\pgfsetroundjoin%
\definecolor{currentfill}{rgb}{0.121569,0.466667,0.705882}%
\pgfsetfillcolor{currentfill}%
\pgfsetfillopacity{0.368105}%
\pgfsetlinewidth{1.003750pt}%
\definecolor{currentstroke}{rgb}{0.121569,0.466667,0.705882}%
\pgfsetstrokecolor{currentstroke}%
\pgfsetstrokeopacity{0.368105}%
\pgfsetdash{}{0pt}%
\pgfpathmoveto{\pgfqpoint{1.204529in}{1.738628in}}%
\pgfpathcurveto{\pgfqpoint{1.212765in}{1.738628in}}{\pgfqpoint{1.220665in}{1.741901in}}{\pgfqpoint{1.226489in}{1.747724in}}%
\pgfpathcurveto{\pgfqpoint{1.232313in}{1.753548in}}{\pgfqpoint{1.235585in}{1.761448in}}{\pgfqpoint{1.235585in}{1.769685in}}%
\pgfpathcurveto{\pgfqpoint{1.235585in}{1.777921in}}{\pgfqpoint{1.232313in}{1.785821in}}{\pgfqpoint{1.226489in}{1.791645in}}%
\pgfpathcurveto{\pgfqpoint{1.220665in}{1.797469in}}{\pgfqpoint{1.212765in}{1.800741in}}{\pgfqpoint{1.204529in}{1.800741in}}%
\pgfpathcurveto{\pgfqpoint{1.196292in}{1.800741in}}{\pgfqpoint{1.188392in}{1.797469in}}{\pgfqpoint{1.182568in}{1.791645in}}%
\pgfpathcurveto{\pgfqpoint{1.176744in}{1.785821in}}{\pgfqpoint{1.173472in}{1.777921in}}{\pgfqpoint{1.173472in}{1.769685in}}%
\pgfpathcurveto{\pgfqpoint{1.173472in}{1.761448in}}{\pgfqpoint{1.176744in}{1.753548in}}{\pgfqpoint{1.182568in}{1.747724in}}%
\pgfpathcurveto{\pgfqpoint{1.188392in}{1.741901in}}{\pgfqpoint{1.196292in}{1.738628in}}{\pgfqpoint{1.204529in}{1.738628in}}%
\pgfpathclose%
\pgfusepath{stroke,fill}%
\end{pgfscope}%
\begin{pgfscope}%
\pgfpathrectangle{\pgfqpoint{0.100000in}{0.212622in}}{\pgfqpoint{3.696000in}{3.696000in}}%
\pgfusepath{clip}%
\pgfsetbuttcap%
\pgfsetroundjoin%
\definecolor{currentfill}{rgb}{0.121569,0.466667,0.705882}%
\pgfsetfillcolor{currentfill}%
\pgfsetfillopacity{0.368105}%
\pgfsetlinewidth{1.003750pt}%
\definecolor{currentstroke}{rgb}{0.121569,0.466667,0.705882}%
\pgfsetstrokecolor{currentstroke}%
\pgfsetstrokeopacity{0.368105}%
\pgfsetdash{}{0pt}%
\pgfpathmoveto{\pgfqpoint{1.204529in}{1.738628in}}%
\pgfpathcurveto{\pgfqpoint{1.212765in}{1.738628in}}{\pgfqpoint{1.220665in}{1.741901in}}{\pgfqpoint{1.226489in}{1.747724in}}%
\pgfpathcurveto{\pgfqpoint{1.232313in}{1.753548in}}{\pgfqpoint{1.235585in}{1.761448in}}{\pgfqpoint{1.235585in}{1.769685in}}%
\pgfpathcurveto{\pgfqpoint{1.235585in}{1.777921in}}{\pgfqpoint{1.232313in}{1.785821in}}{\pgfqpoint{1.226489in}{1.791645in}}%
\pgfpathcurveto{\pgfqpoint{1.220665in}{1.797469in}}{\pgfqpoint{1.212765in}{1.800741in}}{\pgfqpoint{1.204529in}{1.800741in}}%
\pgfpathcurveto{\pgfqpoint{1.196292in}{1.800741in}}{\pgfqpoint{1.188392in}{1.797469in}}{\pgfqpoint{1.182568in}{1.791645in}}%
\pgfpathcurveto{\pgfqpoint{1.176744in}{1.785821in}}{\pgfqpoint{1.173472in}{1.777921in}}{\pgfqpoint{1.173472in}{1.769685in}}%
\pgfpathcurveto{\pgfqpoint{1.173472in}{1.761448in}}{\pgfqpoint{1.176744in}{1.753548in}}{\pgfqpoint{1.182568in}{1.747724in}}%
\pgfpathcurveto{\pgfqpoint{1.188392in}{1.741901in}}{\pgfqpoint{1.196292in}{1.738628in}}{\pgfqpoint{1.204529in}{1.738628in}}%
\pgfpathclose%
\pgfusepath{stroke,fill}%
\end{pgfscope}%
\begin{pgfscope}%
\pgfpathrectangle{\pgfqpoint{0.100000in}{0.212622in}}{\pgfqpoint{3.696000in}{3.696000in}}%
\pgfusepath{clip}%
\pgfsetbuttcap%
\pgfsetroundjoin%
\definecolor{currentfill}{rgb}{0.121569,0.466667,0.705882}%
\pgfsetfillcolor{currentfill}%
\pgfsetfillopacity{0.368105}%
\pgfsetlinewidth{1.003750pt}%
\definecolor{currentstroke}{rgb}{0.121569,0.466667,0.705882}%
\pgfsetstrokecolor{currentstroke}%
\pgfsetstrokeopacity{0.368105}%
\pgfsetdash{}{0pt}%
\pgfpathmoveto{\pgfqpoint{1.204529in}{1.738628in}}%
\pgfpathcurveto{\pgfqpoint{1.212765in}{1.738628in}}{\pgfqpoint{1.220665in}{1.741901in}}{\pgfqpoint{1.226489in}{1.747724in}}%
\pgfpathcurveto{\pgfqpoint{1.232313in}{1.753548in}}{\pgfqpoint{1.235585in}{1.761448in}}{\pgfqpoint{1.235585in}{1.769685in}}%
\pgfpathcurveto{\pgfqpoint{1.235585in}{1.777921in}}{\pgfqpoint{1.232313in}{1.785821in}}{\pgfqpoint{1.226489in}{1.791645in}}%
\pgfpathcurveto{\pgfqpoint{1.220665in}{1.797469in}}{\pgfqpoint{1.212765in}{1.800741in}}{\pgfqpoint{1.204529in}{1.800741in}}%
\pgfpathcurveto{\pgfqpoint{1.196292in}{1.800741in}}{\pgfqpoint{1.188392in}{1.797469in}}{\pgfqpoint{1.182568in}{1.791645in}}%
\pgfpathcurveto{\pgfqpoint{1.176744in}{1.785821in}}{\pgfqpoint{1.173472in}{1.777921in}}{\pgfqpoint{1.173472in}{1.769685in}}%
\pgfpathcurveto{\pgfqpoint{1.173472in}{1.761448in}}{\pgfqpoint{1.176744in}{1.753548in}}{\pgfqpoint{1.182568in}{1.747724in}}%
\pgfpathcurveto{\pgfqpoint{1.188392in}{1.741901in}}{\pgfqpoint{1.196292in}{1.738628in}}{\pgfqpoint{1.204529in}{1.738628in}}%
\pgfpathclose%
\pgfusepath{stroke,fill}%
\end{pgfscope}%
\begin{pgfscope}%
\pgfpathrectangle{\pgfqpoint{0.100000in}{0.212622in}}{\pgfqpoint{3.696000in}{3.696000in}}%
\pgfusepath{clip}%
\pgfsetbuttcap%
\pgfsetroundjoin%
\definecolor{currentfill}{rgb}{0.121569,0.466667,0.705882}%
\pgfsetfillcolor{currentfill}%
\pgfsetfillopacity{0.368105}%
\pgfsetlinewidth{1.003750pt}%
\definecolor{currentstroke}{rgb}{0.121569,0.466667,0.705882}%
\pgfsetstrokecolor{currentstroke}%
\pgfsetstrokeopacity{0.368105}%
\pgfsetdash{}{0pt}%
\pgfpathmoveto{\pgfqpoint{1.204529in}{1.738628in}}%
\pgfpathcurveto{\pgfqpoint{1.212765in}{1.738628in}}{\pgfqpoint{1.220665in}{1.741901in}}{\pgfqpoint{1.226489in}{1.747724in}}%
\pgfpathcurveto{\pgfqpoint{1.232313in}{1.753548in}}{\pgfqpoint{1.235585in}{1.761448in}}{\pgfqpoint{1.235585in}{1.769685in}}%
\pgfpathcurveto{\pgfqpoint{1.235585in}{1.777921in}}{\pgfqpoint{1.232313in}{1.785821in}}{\pgfqpoint{1.226489in}{1.791645in}}%
\pgfpathcurveto{\pgfqpoint{1.220665in}{1.797469in}}{\pgfqpoint{1.212765in}{1.800741in}}{\pgfqpoint{1.204529in}{1.800741in}}%
\pgfpathcurveto{\pgfqpoint{1.196292in}{1.800741in}}{\pgfqpoint{1.188392in}{1.797469in}}{\pgfqpoint{1.182568in}{1.791645in}}%
\pgfpathcurveto{\pgfqpoint{1.176744in}{1.785821in}}{\pgfqpoint{1.173472in}{1.777921in}}{\pgfqpoint{1.173472in}{1.769685in}}%
\pgfpathcurveto{\pgfqpoint{1.173472in}{1.761448in}}{\pgfqpoint{1.176744in}{1.753548in}}{\pgfqpoint{1.182568in}{1.747724in}}%
\pgfpathcurveto{\pgfqpoint{1.188392in}{1.741901in}}{\pgfqpoint{1.196292in}{1.738628in}}{\pgfqpoint{1.204529in}{1.738628in}}%
\pgfpathclose%
\pgfusepath{stroke,fill}%
\end{pgfscope}%
\begin{pgfscope}%
\pgfpathrectangle{\pgfqpoint{0.100000in}{0.212622in}}{\pgfqpoint{3.696000in}{3.696000in}}%
\pgfusepath{clip}%
\pgfsetbuttcap%
\pgfsetroundjoin%
\definecolor{currentfill}{rgb}{0.121569,0.466667,0.705882}%
\pgfsetfillcolor{currentfill}%
\pgfsetfillopacity{0.368105}%
\pgfsetlinewidth{1.003750pt}%
\definecolor{currentstroke}{rgb}{0.121569,0.466667,0.705882}%
\pgfsetstrokecolor{currentstroke}%
\pgfsetstrokeopacity{0.368105}%
\pgfsetdash{}{0pt}%
\pgfpathmoveto{\pgfqpoint{1.204529in}{1.738628in}}%
\pgfpathcurveto{\pgfqpoint{1.212765in}{1.738628in}}{\pgfqpoint{1.220665in}{1.741901in}}{\pgfqpoint{1.226489in}{1.747724in}}%
\pgfpathcurveto{\pgfqpoint{1.232313in}{1.753548in}}{\pgfqpoint{1.235585in}{1.761448in}}{\pgfqpoint{1.235585in}{1.769685in}}%
\pgfpathcurveto{\pgfqpoint{1.235585in}{1.777921in}}{\pgfqpoint{1.232313in}{1.785821in}}{\pgfqpoint{1.226489in}{1.791645in}}%
\pgfpathcurveto{\pgfqpoint{1.220665in}{1.797469in}}{\pgfqpoint{1.212765in}{1.800741in}}{\pgfqpoint{1.204529in}{1.800741in}}%
\pgfpathcurveto{\pgfqpoint{1.196292in}{1.800741in}}{\pgfqpoint{1.188392in}{1.797469in}}{\pgfqpoint{1.182568in}{1.791645in}}%
\pgfpathcurveto{\pgfqpoint{1.176744in}{1.785821in}}{\pgfqpoint{1.173472in}{1.777921in}}{\pgfqpoint{1.173472in}{1.769685in}}%
\pgfpathcurveto{\pgfqpoint{1.173472in}{1.761448in}}{\pgfqpoint{1.176744in}{1.753548in}}{\pgfqpoint{1.182568in}{1.747724in}}%
\pgfpathcurveto{\pgfqpoint{1.188392in}{1.741901in}}{\pgfqpoint{1.196292in}{1.738628in}}{\pgfqpoint{1.204529in}{1.738628in}}%
\pgfpathclose%
\pgfusepath{stroke,fill}%
\end{pgfscope}%
\begin{pgfscope}%
\pgfpathrectangle{\pgfqpoint{0.100000in}{0.212622in}}{\pgfqpoint{3.696000in}{3.696000in}}%
\pgfusepath{clip}%
\pgfsetbuttcap%
\pgfsetroundjoin%
\definecolor{currentfill}{rgb}{0.121569,0.466667,0.705882}%
\pgfsetfillcolor{currentfill}%
\pgfsetfillopacity{0.368105}%
\pgfsetlinewidth{1.003750pt}%
\definecolor{currentstroke}{rgb}{0.121569,0.466667,0.705882}%
\pgfsetstrokecolor{currentstroke}%
\pgfsetstrokeopacity{0.368105}%
\pgfsetdash{}{0pt}%
\pgfpathmoveto{\pgfqpoint{1.204529in}{1.738628in}}%
\pgfpathcurveto{\pgfqpoint{1.212765in}{1.738628in}}{\pgfqpoint{1.220665in}{1.741901in}}{\pgfqpoint{1.226489in}{1.747724in}}%
\pgfpathcurveto{\pgfqpoint{1.232313in}{1.753548in}}{\pgfqpoint{1.235585in}{1.761448in}}{\pgfqpoint{1.235585in}{1.769685in}}%
\pgfpathcurveto{\pgfqpoint{1.235585in}{1.777921in}}{\pgfqpoint{1.232313in}{1.785821in}}{\pgfqpoint{1.226489in}{1.791645in}}%
\pgfpathcurveto{\pgfqpoint{1.220665in}{1.797469in}}{\pgfqpoint{1.212765in}{1.800741in}}{\pgfqpoint{1.204529in}{1.800741in}}%
\pgfpathcurveto{\pgfqpoint{1.196292in}{1.800741in}}{\pgfqpoint{1.188392in}{1.797469in}}{\pgfqpoint{1.182568in}{1.791645in}}%
\pgfpathcurveto{\pgfqpoint{1.176744in}{1.785821in}}{\pgfqpoint{1.173472in}{1.777921in}}{\pgfqpoint{1.173472in}{1.769685in}}%
\pgfpathcurveto{\pgfqpoint{1.173472in}{1.761448in}}{\pgfqpoint{1.176744in}{1.753548in}}{\pgfqpoint{1.182568in}{1.747724in}}%
\pgfpathcurveto{\pgfqpoint{1.188392in}{1.741901in}}{\pgfqpoint{1.196292in}{1.738628in}}{\pgfqpoint{1.204529in}{1.738628in}}%
\pgfpathclose%
\pgfusepath{stroke,fill}%
\end{pgfscope}%
\begin{pgfscope}%
\pgfpathrectangle{\pgfqpoint{0.100000in}{0.212622in}}{\pgfqpoint{3.696000in}{3.696000in}}%
\pgfusepath{clip}%
\pgfsetbuttcap%
\pgfsetroundjoin%
\definecolor{currentfill}{rgb}{0.121569,0.466667,0.705882}%
\pgfsetfillcolor{currentfill}%
\pgfsetfillopacity{0.368105}%
\pgfsetlinewidth{1.003750pt}%
\definecolor{currentstroke}{rgb}{0.121569,0.466667,0.705882}%
\pgfsetstrokecolor{currentstroke}%
\pgfsetstrokeopacity{0.368105}%
\pgfsetdash{}{0pt}%
\pgfpathmoveto{\pgfqpoint{1.204529in}{1.738628in}}%
\pgfpathcurveto{\pgfqpoint{1.212765in}{1.738628in}}{\pgfqpoint{1.220665in}{1.741901in}}{\pgfqpoint{1.226489in}{1.747724in}}%
\pgfpathcurveto{\pgfqpoint{1.232313in}{1.753548in}}{\pgfqpoint{1.235585in}{1.761448in}}{\pgfqpoint{1.235585in}{1.769685in}}%
\pgfpathcurveto{\pgfqpoint{1.235585in}{1.777921in}}{\pgfqpoint{1.232313in}{1.785821in}}{\pgfqpoint{1.226489in}{1.791645in}}%
\pgfpathcurveto{\pgfqpoint{1.220665in}{1.797469in}}{\pgfqpoint{1.212765in}{1.800741in}}{\pgfqpoint{1.204529in}{1.800741in}}%
\pgfpathcurveto{\pgfqpoint{1.196292in}{1.800741in}}{\pgfqpoint{1.188392in}{1.797469in}}{\pgfqpoint{1.182568in}{1.791645in}}%
\pgfpathcurveto{\pgfqpoint{1.176744in}{1.785821in}}{\pgfqpoint{1.173472in}{1.777921in}}{\pgfqpoint{1.173472in}{1.769685in}}%
\pgfpathcurveto{\pgfqpoint{1.173472in}{1.761448in}}{\pgfqpoint{1.176744in}{1.753548in}}{\pgfqpoint{1.182568in}{1.747724in}}%
\pgfpathcurveto{\pgfqpoint{1.188392in}{1.741901in}}{\pgfqpoint{1.196292in}{1.738628in}}{\pgfqpoint{1.204529in}{1.738628in}}%
\pgfpathclose%
\pgfusepath{stroke,fill}%
\end{pgfscope}%
\begin{pgfscope}%
\pgfpathrectangle{\pgfqpoint{0.100000in}{0.212622in}}{\pgfqpoint{3.696000in}{3.696000in}}%
\pgfusepath{clip}%
\pgfsetbuttcap%
\pgfsetroundjoin%
\definecolor{currentfill}{rgb}{0.121569,0.466667,0.705882}%
\pgfsetfillcolor{currentfill}%
\pgfsetfillopacity{0.368105}%
\pgfsetlinewidth{1.003750pt}%
\definecolor{currentstroke}{rgb}{0.121569,0.466667,0.705882}%
\pgfsetstrokecolor{currentstroke}%
\pgfsetstrokeopacity{0.368105}%
\pgfsetdash{}{0pt}%
\pgfpathmoveto{\pgfqpoint{1.204529in}{1.738628in}}%
\pgfpathcurveto{\pgfqpoint{1.212765in}{1.738628in}}{\pgfqpoint{1.220665in}{1.741901in}}{\pgfqpoint{1.226489in}{1.747724in}}%
\pgfpathcurveto{\pgfqpoint{1.232313in}{1.753548in}}{\pgfqpoint{1.235585in}{1.761448in}}{\pgfqpoint{1.235585in}{1.769685in}}%
\pgfpathcurveto{\pgfqpoint{1.235585in}{1.777921in}}{\pgfqpoint{1.232313in}{1.785821in}}{\pgfqpoint{1.226489in}{1.791645in}}%
\pgfpathcurveto{\pgfqpoint{1.220665in}{1.797469in}}{\pgfqpoint{1.212765in}{1.800741in}}{\pgfqpoint{1.204529in}{1.800741in}}%
\pgfpathcurveto{\pgfqpoint{1.196292in}{1.800741in}}{\pgfqpoint{1.188392in}{1.797469in}}{\pgfqpoint{1.182568in}{1.791645in}}%
\pgfpathcurveto{\pgfqpoint{1.176744in}{1.785821in}}{\pgfqpoint{1.173472in}{1.777921in}}{\pgfqpoint{1.173472in}{1.769685in}}%
\pgfpathcurveto{\pgfqpoint{1.173472in}{1.761448in}}{\pgfqpoint{1.176744in}{1.753548in}}{\pgfqpoint{1.182568in}{1.747724in}}%
\pgfpathcurveto{\pgfqpoint{1.188392in}{1.741901in}}{\pgfqpoint{1.196292in}{1.738628in}}{\pgfqpoint{1.204529in}{1.738628in}}%
\pgfpathclose%
\pgfusepath{stroke,fill}%
\end{pgfscope}%
\begin{pgfscope}%
\pgfpathrectangle{\pgfqpoint{0.100000in}{0.212622in}}{\pgfqpoint{3.696000in}{3.696000in}}%
\pgfusepath{clip}%
\pgfsetbuttcap%
\pgfsetroundjoin%
\definecolor{currentfill}{rgb}{0.121569,0.466667,0.705882}%
\pgfsetfillcolor{currentfill}%
\pgfsetfillopacity{0.368105}%
\pgfsetlinewidth{1.003750pt}%
\definecolor{currentstroke}{rgb}{0.121569,0.466667,0.705882}%
\pgfsetstrokecolor{currentstroke}%
\pgfsetstrokeopacity{0.368105}%
\pgfsetdash{}{0pt}%
\pgfpathmoveto{\pgfqpoint{1.204529in}{1.738628in}}%
\pgfpathcurveto{\pgfqpoint{1.212765in}{1.738628in}}{\pgfqpoint{1.220665in}{1.741901in}}{\pgfqpoint{1.226489in}{1.747724in}}%
\pgfpathcurveto{\pgfqpoint{1.232313in}{1.753548in}}{\pgfqpoint{1.235585in}{1.761448in}}{\pgfqpoint{1.235585in}{1.769685in}}%
\pgfpathcurveto{\pgfqpoint{1.235585in}{1.777921in}}{\pgfqpoint{1.232313in}{1.785821in}}{\pgfqpoint{1.226489in}{1.791645in}}%
\pgfpathcurveto{\pgfqpoint{1.220665in}{1.797469in}}{\pgfqpoint{1.212765in}{1.800741in}}{\pgfqpoint{1.204529in}{1.800741in}}%
\pgfpathcurveto{\pgfqpoint{1.196292in}{1.800741in}}{\pgfqpoint{1.188392in}{1.797469in}}{\pgfqpoint{1.182568in}{1.791645in}}%
\pgfpathcurveto{\pgfqpoint{1.176744in}{1.785821in}}{\pgfqpoint{1.173472in}{1.777921in}}{\pgfqpoint{1.173472in}{1.769685in}}%
\pgfpathcurveto{\pgfqpoint{1.173472in}{1.761448in}}{\pgfqpoint{1.176744in}{1.753548in}}{\pgfqpoint{1.182568in}{1.747724in}}%
\pgfpathcurveto{\pgfqpoint{1.188392in}{1.741901in}}{\pgfqpoint{1.196292in}{1.738628in}}{\pgfqpoint{1.204529in}{1.738628in}}%
\pgfpathclose%
\pgfusepath{stroke,fill}%
\end{pgfscope}%
\begin{pgfscope}%
\pgfpathrectangle{\pgfqpoint{0.100000in}{0.212622in}}{\pgfqpoint{3.696000in}{3.696000in}}%
\pgfusepath{clip}%
\pgfsetbuttcap%
\pgfsetroundjoin%
\definecolor{currentfill}{rgb}{0.121569,0.466667,0.705882}%
\pgfsetfillcolor{currentfill}%
\pgfsetfillopacity{0.368105}%
\pgfsetlinewidth{1.003750pt}%
\definecolor{currentstroke}{rgb}{0.121569,0.466667,0.705882}%
\pgfsetstrokecolor{currentstroke}%
\pgfsetstrokeopacity{0.368105}%
\pgfsetdash{}{0pt}%
\pgfpathmoveto{\pgfqpoint{1.204529in}{1.738628in}}%
\pgfpathcurveto{\pgfqpoint{1.212765in}{1.738628in}}{\pgfqpoint{1.220665in}{1.741901in}}{\pgfqpoint{1.226489in}{1.747724in}}%
\pgfpathcurveto{\pgfqpoint{1.232313in}{1.753548in}}{\pgfqpoint{1.235585in}{1.761448in}}{\pgfqpoint{1.235585in}{1.769685in}}%
\pgfpathcurveto{\pgfqpoint{1.235585in}{1.777921in}}{\pgfqpoint{1.232313in}{1.785821in}}{\pgfqpoint{1.226489in}{1.791645in}}%
\pgfpathcurveto{\pgfqpoint{1.220665in}{1.797469in}}{\pgfqpoint{1.212765in}{1.800741in}}{\pgfqpoint{1.204529in}{1.800741in}}%
\pgfpathcurveto{\pgfqpoint{1.196292in}{1.800741in}}{\pgfqpoint{1.188392in}{1.797469in}}{\pgfqpoint{1.182568in}{1.791645in}}%
\pgfpathcurveto{\pgfqpoint{1.176744in}{1.785821in}}{\pgfqpoint{1.173472in}{1.777921in}}{\pgfqpoint{1.173472in}{1.769685in}}%
\pgfpathcurveto{\pgfqpoint{1.173472in}{1.761448in}}{\pgfqpoint{1.176744in}{1.753548in}}{\pgfqpoint{1.182568in}{1.747724in}}%
\pgfpathcurveto{\pgfqpoint{1.188392in}{1.741901in}}{\pgfqpoint{1.196292in}{1.738628in}}{\pgfqpoint{1.204529in}{1.738628in}}%
\pgfpathclose%
\pgfusepath{stroke,fill}%
\end{pgfscope}%
\begin{pgfscope}%
\pgfpathrectangle{\pgfqpoint{0.100000in}{0.212622in}}{\pgfqpoint{3.696000in}{3.696000in}}%
\pgfusepath{clip}%
\pgfsetbuttcap%
\pgfsetroundjoin%
\definecolor{currentfill}{rgb}{0.121569,0.466667,0.705882}%
\pgfsetfillcolor{currentfill}%
\pgfsetfillopacity{0.368105}%
\pgfsetlinewidth{1.003750pt}%
\definecolor{currentstroke}{rgb}{0.121569,0.466667,0.705882}%
\pgfsetstrokecolor{currentstroke}%
\pgfsetstrokeopacity{0.368105}%
\pgfsetdash{}{0pt}%
\pgfpathmoveto{\pgfqpoint{1.204529in}{1.738628in}}%
\pgfpathcurveto{\pgfqpoint{1.212765in}{1.738628in}}{\pgfqpoint{1.220665in}{1.741901in}}{\pgfqpoint{1.226489in}{1.747724in}}%
\pgfpathcurveto{\pgfqpoint{1.232313in}{1.753548in}}{\pgfqpoint{1.235585in}{1.761448in}}{\pgfqpoint{1.235585in}{1.769685in}}%
\pgfpathcurveto{\pgfqpoint{1.235585in}{1.777921in}}{\pgfqpoint{1.232313in}{1.785821in}}{\pgfqpoint{1.226489in}{1.791645in}}%
\pgfpathcurveto{\pgfqpoint{1.220665in}{1.797469in}}{\pgfqpoint{1.212765in}{1.800741in}}{\pgfqpoint{1.204529in}{1.800741in}}%
\pgfpathcurveto{\pgfqpoint{1.196292in}{1.800741in}}{\pgfqpoint{1.188392in}{1.797469in}}{\pgfqpoint{1.182568in}{1.791645in}}%
\pgfpathcurveto{\pgfqpoint{1.176744in}{1.785821in}}{\pgfqpoint{1.173472in}{1.777921in}}{\pgfqpoint{1.173472in}{1.769685in}}%
\pgfpathcurveto{\pgfqpoint{1.173472in}{1.761448in}}{\pgfqpoint{1.176744in}{1.753548in}}{\pgfqpoint{1.182568in}{1.747724in}}%
\pgfpathcurveto{\pgfqpoint{1.188392in}{1.741901in}}{\pgfqpoint{1.196292in}{1.738628in}}{\pgfqpoint{1.204529in}{1.738628in}}%
\pgfpathclose%
\pgfusepath{stroke,fill}%
\end{pgfscope}%
\begin{pgfscope}%
\pgfpathrectangle{\pgfqpoint{0.100000in}{0.212622in}}{\pgfqpoint{3.696000in}{3.696000in}}%
\pgfusepath{clip}%
\pgfsetbuttcap%
\pgfsetroundjoin%
\definecolor{currentfill}{rgb}{0.121569,0.466667,0.705882}%
\pgfsetfillcolor{currentfill}%
\pgfsetfillopacity{0.368105}%
\pgfsetlinewidth{1.003750pt}%
\definecolor{currentstroke}{rgb}{0.121569,0.466667,0.705882}%
\pgfsetstrokecolor{currentstroke}%
\pgfsetstrokeopacity{0.368105}%
\pgfsetdash{}{0pt}%
\pgfpathmoveto{\pgfqpoint{1.204529in}{1.738628in}}%
\pgfpathcurveto{\pgfqpoint{1.212765in}{1.738628in}}{\pgfqpoint{1.220665in}{1.741901in}}{\pgfqpoint{1.226489in}{1.747724in}}%
\pgfpathcurveto{\pgfqpoint{1.232313in}{1.753548in}}{\pgfqpoint{1.235585in}{1.761448in}}{\pgfqpoint{1.235585in}{1.769685in}}%
\pgfpathcurveto{\pgfqpoint{1.235585in}{1.777921in}}{\pgfqpoint{1.232313in}{1.785821in}}{\pgfqpoint{1.226489in}{1.791645in}}%
\pgfpathcurveto{\pgfqpoint{1.220665in}{1.797469in}}{\pgfqpoint{1.212765in}{1.800741in}}{\pgfqpoint{1.204529in}{1.800741in}}%
\pgfpathcurveto{\pgfqpoint{1.196292in}{1.800741in}}{\pgfqpoint{1.188392in}{1.797469in}}{\pgfqpoint{1.182568in}{1.791645in}}%
\pgfpathcurveto{\pgfqpoint{1.176744in}{1.785821in}}{\pgfqpoint{1.173472in}{1.777921in}}{\pgfqpoint{1.173472in}{1.769685in}}%
\pgfpathcurveto{\pgfqpoint{1.173472in}{1.761448in}}{\pgfqpoint{1.176744in}{1.753548in}}{\pgfqpoint{1.182568in}{1.747724in}}%
\pgfpathcurveto{\pgfqpoint{1.188392in}{1.741901in}}{\pgfqpoint{1.196292in}{1.738628in}}{\pgfqpoint{1.204529in}{1.738628in}}%
\pgfpathclose%
\pgfusepath{stroke,fill}%
\end{pgfscope}%
\begin{pgfscope}%
\pgfpathrectangle{\pgfqpoint{0.100000in}{0.212622in}}{\pgfqpoint{3.696000in}{3.696000in}}%
\pgfusepath{clip}%
\pgfsetbuttcap%
\pgfsetroundjoin%
\definecolor{currentfill}{rgb}{0.121569,0.466667,0.705882}%
\pgfsetfillcolor{currentfill}%
\pgfsetfillopacity{0.368105}%
\pgfsetlinewidth{1.003750pt}%
\definecolor{currentstroke}{rgb}{0.121569,0.466667,0.705882}%
\pgfsetstrokecolor{currentstroke}%
\pgfsetstrokeopacity{0.368105}%
\pgfsetdash{}{0pt}%
\pgfpathmoveto{\pgfqpoint{1.204529in}{1.738628in}}%
\pgfpathcurveto{\pgfqpoint{1.212765in}{1.738628in}}{\pgfqpoint{1.220665in}{1.741901in}}{\pgfqpoint{1.226489in}{1.747724in}}%
\pgfpathcurveto{\pgfqpoint{1.232313in}{1.753548in}}{\pgfqpoint{1.235585in}{1.761448in}}{\pgfqpoint{1.235585in}{1.769685in}}%
\pgfpathcurveto{\pgfqpoint{1.235585in}{1.777921in}}{\pgfqpoint{1.232313in}{1.785821in}}{\pgfqpoint{1.226489in}{1.791645in}}%
\pgfpathcurveto{\pgfqpoint{1.220665in}{1.797469in}}{\pgfqpoint{1.212765in}{1.800741in}}{\pgfqpoint{1.204529in}{1.800741in}}%
\pgfpathcurveto{\pgfqpoint{1.196292in}{1.800741in}}{\pgfqpoint{1.188392in}{1.797469in}}{\pgfqpoint{1.182568in}{1.791645in}}%
\pgfpathcurveto{\pgfqpoint{1.176744in}{1.785821in}}{\pgfqpoint{1.173472in}{1.777921in}}{\pgfqpoint{1.173472in}{1.769685in}}%
\pgfpathcurveto{\pgfqpoint{1.173472in}{1.761448in}}{\pgfqpoint{1.176744in}{1.753548in}}{\pgfqpoint{1.182568in}{1.747724in}}%
\pgfpathcurveto{\pgfqpoint{1.188392in}{1.741901in}}{\pgfqpoint{1.196292in}{1.738628in}}{\pgfqpoint{1.204529in}{1.738628in}}%
\pgfpathclose%
\pgfusepath{stroke,fill}%
\end{pgfscope}%
\begin{pgfscope}%
\pgfpathrectangle{\pgfqpoint{0.100000in}{0.212622in}}{\pgfqpoint{3.696000in}{3.696000in}}%
\pgfusepath{clip}%
\pgfsetbuttcap%
\pgfsetroundjoin%
\definecolor{currentfill}{rgb}{0.121569,0.466667,0.705882}%
\pgfsetfillcolor{currentfill}%
\pgfsetfillopacity{0.368105}%
\pgfsetlinewidth{1.003750pt}%
\definecolor{currentstroke}{rgb}{0.121569,0.466667,0.705882}%
\pgfsetstrokecolor{currentstroke}%
\pgfsetstrokeopacity{0.368105}%
\pgfsetdash{}{0pt}%
\pgfpathmoveto{\pgfqpoint{1.204529in}{1.738628in}}%
\pgfpathcurveto{\pgfqpoint{1.212765in}{1.738628in}}{\pgfqpoint{1.220665in}{1.741901in}}{\pgfqpoint{1.226489in}{1.747724in}}%
\pgfpathcurveto{\pgfqpoint{1.232313in}{1.753548in}}{\pgfqpoint{1.235585in}{1.761448in}}{\pgfqpoint{1.235585in}{1.769685in}}%
\pgfpathcurveto{\pgfqpoint{1.235585in}{1.777921in}}{\pgfqpoint{1.232313in}{1.785821in}}{\pgfqpoint{1.226489in}{1.791645in}}%
\pgfpathcurveto{\pgfqpoint{1.220665in}{1.797469in}}{\pgfqpoint{1.212765in}{1.800741in}}{\pgfqpoint{1.204529in}{1.800741in}}%
\pgfpathcurveto{\pgfqpoint{1.196292in}{1.800741in}}{\pgfqpoint{1.188392in}{1.797469in}}{\pgfqpoint{1.182568in}{1.791645in}}%
\pgfpathcurveto{\pgfqpoint{1.176744in}{1.785821in}}{\pgfqpoint{1.173472in}{1.777921in}}{\pgfqpoint{1.173472in}{1.769685in}}%
\pgfpathcurveto{\pgfqpoint{1.173472in}{1.761448in}}{\pgfqpoint{1.176744in}{1.753548in}}{\pgfqpoint{1.182568in}{1.747724in}}%
\pgfpathcurveto{\pgfqpoint{1.188392in}{1.741901in}}{\pgfqpoint{1.196292in}{1.738628in}}{\pgfqpoint{1.204529in}{1.738628in}}%
\pgfpathclose%
\pgfusepath{stroke,fill}%
\end{pgfscope}%
\begin{pgfscope}%
\pgfpathrectangle{\pgfqpoint{0.100000in}{0.212622in}}{\pgfqpoint{3.696000in}{3.696000in}}%
\pgfusepath{clip}%
\pgfsetbuttcap%
\pgfsetroundjoin%
\definecolor{currentfill}{rgb}{0.121569,0.466667,0.705882}%
\pgfsetfillcolor{currentfill}%
\pgfsetfillopacity{0.379746}%
\pgfsetlinewidth{1.003750pt}%
\definecolor{currentstroke}{rgb}{0.121569,0.466667,0.705882}%
\pgfsetstrokecolor{currentstroke}%
\pgfsetstrokeopacity{0.379746}%
\pgfsetdash{}{0pt}%
\pgfpathmoveto{\pgfqpoint{1.545345in}{1.954034in}}%
\pgfpathcurveto{\pgfqpoint{1.553581in}{1.954034in}}{\pgfqpoint{1.561481in}{1.957306in}}{\pgfqpoint{1.567305in}{1.963130in}}%
\pgfpathcurveto{\pgfqpoint{1.573129in}{1.968954in}}{\pgfqpoint{1.576401in}{1.976854in}}{\pgfqpoint{1.576401in}{1.985091in}}%
\pgfpathcurveto{\pgfqpoint{1.576401in}{1.993327in}}{\pgfqpoint{1.573129in}{2.001227in}}{\pgfqpoint{1.567305in}{2.007051in}}%
\pgfpathcurveto{\pgfqpoint{1.561481in}{2.012875in}}{\pgfqpoint{1.553581in}{2.016147in}}{\pgfqpoint{1.545345in}{2.016147in}}%
\pgfpathcurveto{\pgfqpoint{1.537108in}{2.016147in}}{\pgfqpoint{1.529208in}{2.012875in}}{\pgfqpoint{1.523384in}{2.007051in}}%
\pgfpathcurveto{\pgfqpoint{1.517560in}{2.001227in}}{\pgfqpoint{1.514288in}{1.993327in}}{\pgfqpoint{1.514288in}{1.985091in}}%
\pgfpathcurveto{\pgfqpoint{1.514288in}{1.976854in}}{\pgfqpoint{1.517560in}{1.968954in}}{\pgfqpoint{1.523384in}{1.963130in}}%
\pgfpathcurveto{\pgfqpoint{1.529208in}{1.957306in}}{\pgfqpoint{1.537108in}{1.954034in}}{\pgfqpoint{1.545345in}{1.954034in}}%
\pgfpathclose%
\pgfusepath{stroke,fill}%
\end{pgfscope}%
\begin{pgfscope}%
\pgfpathrectangle{\pgfqpoint{0.100000in}{0.212622in}}{\pgfqpoint{3.696000in}{3.696000in}}%
\pgfusepath{clip}%
\pgfsetbuttcap%
\pgfsetroundjoin%
\definecolor{currentfill}{rgb}{0.121569,0.466667,0.705882}%
\pgfsetfillcolor{currentfill}%
\pgfsetfillopacity{0.383553}%
\pgfsetlinewidth{1.003750pt}%
\definecolor{currentstroke}{rgb}{0.121569,0.466667,0.705882}%
\pgfsetstrokecolor{currentstroke}%
\pgfsetstrokeopacity{0.383553}%
\pgfsetdash{}{0pt}%
\pgfpathmoveto{\pgfqpoint{1.542196in}{1.949309in}}%
\pgfpathcurveto{\pgfqpoint{1.550432in}{1.949309in}}{\pgfqpoint{1.558332in}{1.952582in}}{\pgfqpoint{1.564156in}{1.958406in}}%
\pgfpathcurveto{\pgfqpoint{1.569980in}{1.964229in}}{\pgfqpoint{1.573252in}{1.972129in}}{\pgfqpoint{1.573252in}{1.980366in}}%
\pgfpathcurveto{\pgfqpoint{1.573252in}{1.988602in}}{\pgfqpoint{1.569980in}{1.996502in}}{\pgfqpoint{1.564156in}{2.002326in}}%
\pgfpathcurveto{\pgfqpoint{1.558332in}{2.008150in}}{\pgfqpoint{1.550432in}{2.011422in}}{\pgfqpoint{1.542196in}{2.011422in}}%
\pgfpathcurveto{\pgfqpoint{1.533959in}{2.011422in}}{\pgfqpoint{1.526059in}{2.008150in}}{\pgfqpoint{1.520235in}{2.002326in}}%
\pgfpathcurveto{\pgfqpoint{1.514411in}{1.996502in}}{\pgfqpoint{1.511139in}{1.988602in}}{\pgfqpoint{1.511139in}{1.980366in}}%
\pgfpathcurveto{\pgfqpoint{1.511139in}{1.972129in}}{\pgfqpoint{1.514411in}{1.964229in}}{\pgfqpoint{1.520235in}{1.958406in}}%
\pgfpathcurveto{\pgfqpoint{1.526059in}{1.952582in}}{\pgfqpoint{1.533959in}{1.949309in}}{\pgfqpoint{1.542196in}{1.949309in}}%
\pgfpathclose%
\pgfusepath{stroke,fill}%
\end{pgfscope}%
\begin{pgfscope}%
\pgfpathrectangle{\pgfqpoint{0.100000in}{0.212622in}}{\pgfqpoint{3.696000in}{3.696000in}}%
\pgfusepath{clip}%
\pgfsetbuttcap%
\pgfsetroundjoin%
\definecolor{currentfill}{rgb}{0.121569,0.466667,0.705882}%
\pgfsetfillcolor{currentfill}%
\pgfsetfillopacity{0.384006}%
\pgfsetlinewidth{1.003750pt}%
\definecolor{currentstroke}{rgb}{0.121569,0.466667,0.705882}%
\pgfsetstrokecolor{currentstroke}%
\pgfsetstrokeopacity{0.384006}%
\pgfsetdash{}{0pt}%
\pgfpathmoveto{\pgfqpoint{1.551926in}{1.957734in}}%
\pgfpathcurveto{\pgfqpoint{1.560162in}{1.957734in}}{\pgfqpoint{1.568062in}{1.961006in}}{\pgfqpoint{1.573886in}{1.966830in}}%
\pgfpathcurveto{\pgfqpoint{1.579710in}{1.972654in}}{\pgfqpoint{1.582983in}{1.980554in}}{\pgfqpoint{1.582983in}{1.988790in}}%
\pgfpathcurveto{\pgfqpoint{1.582983in}{1.997027in}}{\pgfqpoint{1.579710in}{2.004927in}}{\pgfqpoint{1.573886in}{2.010751in}}%
\pgfpathcurveto{\pgfqpoint{1.568062in}{2.016574in}}{\pgfqpoint{1.560162in}{2.019847in}}{\pgfqpoint{1.551926in}{2.019847in}}%
\pgfpathcurveto{\pgfqpoint{1.543690in}{2.019847in}}{\pgfqpoint{1.535790in}{2.016574in}}{\pgfqpoint{1.529966in}{2.010751in}}%
\pgfpathcurveto{\pgfqpoint{1.524142in}{2.004927in}}{\pgfqpoint{1.520870in}{1.997027in}}{\pgfqpoint{1.520870in}{1.988790in}}%
\pgfpathcurveto{\pgfqpoint{1.520870in}{1.980554in}}{\pgfqpoint{1.524142in}{1.972654in}}{\pgfqpoint{1.529966in}{1.966830in}}%
\pgfpathcurveto{\pgfqpoint{1.535790in}{1.961006in}}{\pgfqpoint{1.543690in}{1.957734in}}{\pgfqpoint{1.551926in}{1.957734in}}%
\pgfpathclose%
\pgfusepath{stroke,fill}%
\end{pgfscope}%
\begin{pgfscope}%
\pgfpathrectangle{\pgfqpoint{0.100000in}{0.212622in}}{\pgfqpoint{3.696000in}{3.696000in}}%
\pgfusepath{clip}%
\pgfsetbuttcap%
\pgfsetroundjoin%
\definecolor{currentfill}{rgb}{0.121569,0.466667,0.705882}%
\pgfsetfillcolor{currentfill}%
\pgfsetfillopacity{0.388081}%
\pgfsetlinewidth{1.003750pt}%
\definecolor{currentstroke}{rgb}{0.121569,0.466667,0.705882}%
\pgfsetstrokecolor{currentstroke}%
\pgfsetstrokeopacity{0.388081}%
\pgfsetdash{}{0pt}%
\pgfpathmoveto{\pgfqpoint{1.533253in}{1.940993in}}%
\pgfpathcurveto{\pgfqpoint{1.541490in}{1.940993in}}{\pgfqpoint{1.549390in}{1.944265in}}{\pgfqpoint{1.555214in}{1.950089in}}%
\pgfpathcurveto{\pgfqpoint{1.561037in}{1.955913in}}{\pgfqpoint{1.564310in}{1.963813in}}{\pgfqpoint{1.564310in}{1.972050in}}%
\pgfpathcurveto{\pgfqpoint{1.564310in}{1.980286in}}{\pgfqpoint{1.561037in}{1.988186in}}{\pgfqpoint{1.555214in}{1.994010in}}%
\pgfpathcurveto{\pgfqpoint{1.549390in}{1.999834in}}{\pgfqpoint{1.541490in}{2.003106in}}{\pgfqpoint{1.533253in}{2.003106in}}%
\pgfpathcurveto{\pgfqpoint{1.525017in}{2.003106in}}{\pgfqpoint{1.517117in}{1.999834in}}{\pgfqpoint{1.511293in}{1.994010in}}%
\pgfpathcurveto{\pgfqpoint{1.505469in}{1.988186in}}{\pgfqpoint{1.502197in}{1.980286in}}{\pgfqpoint{1.502197in}{1.972050in}}%
\pgfpathcurveto{\pgfqpoint{1.502197in}{1.963813in}}{\pgfqpoint{1.505469in}{1.955913in}}{\pgfqpoint{1.511293in}{1.950089in}}%
\pgfpathcurveto{\pgfqpoint{1.517117in}{1.944265in}}{\pgfqpoint{1.525017in}{1.940993in}}{\pgfqpoint{1.533253in}{1.940993in}}%
\pgfpathclose%
\pgfusepath{stroke,fill}%
\end{pgfscope}%
\begin{pgfscope}%
\pgfpathrectangle{\pgfqpoint{0.100000in}{0.212622in}}{\pgfqpoint{3.696000in}{3.696000in}}%
\pgfusepath{clip}%
\pgfsetbuttcap%
\pgfsetroundjoin%
\definecolor{currentfill}{rgb}{0.121569,0.466667,0.705882}%
\pgfsetfillcolor{currentfill}%
\pgfsetfillopacity{0.388842}%
\pgfsetlinewidth{1.003750pt}%
\definecolor{currentstroke}{rgb}{0.121569,0.466667,0.705882}%
\pgfsetstrokecolor{currentstroke}%
\pgfsetstrokeopacity{0.388842}%
\pgfsetdash{}{0pt}%
\pgfpathmoveto{\pgfqpoint{1.543358in}{1.948107in}}%
\pgfpathcurveto{\pgfqpoint{1.551594in}{1.948107in}}{\pgfqpoint{1.559494in}{1.951379in}}{\pgfqpoint{1.565318in}{1.957203in}}%
\pgfpathcurveto{\pgfqpoint{1.571142in}{1.963027in}}{\pgfqpoint{1.574415in}{1.970927in}}{\pgfqpoint{1.574415in}{1.979164in}}%
\pgfpathcurveto{\pgfqpoint{1.574415in}{1.987400in}}{\pgfqpoint{1.571142in}{1.995300in}}{\pgfqpoint{1.565318in}{2.001124in}}%
\pgfpathcurveto{\pgfqpoint{1.559494in}{2.006948in}}{\pgfqpoint{1.551594in}{2.010220in}}{\pgfqpoint{1.543358in}{2.010220in}}%
\pgfpathcurveto{\pgfqpoint{1.535122in}{2.010220in}}{\pgfqpoint{1.527222in}{2.006948in}}{\pgfqpoint{1.521398in}{2.001124in}}%
\pgfpathcurveto{\pgfqpoint{1.515574in}{1.995300in}}{\pgfqpoint{1.512302in}{1.987400in}}{\pgfqpoint{1.512302in}{1.979164in}}%
\pgfpathcurveto{\pgfqpoint{1.512302in}{1.970927in}}{\pgfqpoint{1.515574in}{1.963027in}}{\pgfqpoint{1.521398in}{1.957203in}}%
\pgfpathcurveto{\pgfqpoint{1.527222in}{1.951379in}}{\pgfqpoint{1.535122in}{1.948107in}}{\pgfqpoint{1.543358in}{1.948107in}}%
\pgfpathclose%
\pgfusepath{stroke,fill}%
\end{pgfscope}%
\begin{pgfscope}%
\pgfpathrectangle{\pgfqpoint{0.100000in}{0.212622in}}{\pgfqpoint{3.696000in}{3.696000in}}%
\pgfusepath{clip}%
\pgfsetbuttcap%
\pgfsetroundjoin%
\definecolor{currentfill}{rgb}{0.121569,0.466667,0.705882}%
\pgfsetfillcolor{currentfill}%
\pgfsetfillopacity{0.390987}%
\pgfsetlinewidth{1.003750pt}%
\definecolor{currentstroke}{rgb}{0.121569,0.466667,0.705882}%
\pgfsetstrokecolor{currentstroke}%
\pgfsetstrokeopacity{0.390987}%
\pgfsetdash{}{0pt}%
\pgfpathmoveto{\pgfqpoint{1.544050in}{1.945613in}}%
\pgfpathcurveto{\pgfqpoint{1.552286in}{1.945613in}}{\pgfqpoint{1.560186in}{1.948885in}}{\pgfqpoint{1.566010in}{1.954709in}}%
\pgfpathcurveto{\pgfqpoint{1.571834in}{1.960533in}}{\pgfqpoint{1.575107in}{1.968433in}}{\pgfqpoint{1.575107in}{1.976669in}}%
\pgfpathcurveto{\pgfqpoint{1.575107in}{1.984906in}}{\pgfqpoint{1.571834in}{1.992806in}}{\pgfqpoint{1.566010in}{1.998630in}}%
\pgfpathcurveto{\pgfqpoint{1.560186in}{2.004454in}}{\pgfqpoint{1.552286in}{2.007726in}}{\pgfqpoint{1.544050in}{2.007726in}}%
\pgfpathcurveto{\pgfqpoint{1.535814in}{2.007726in}}{\pgfqpoint{1.527914in}{2.004454in}}{\pgfqpoint{1.522090in}{1.998630in}}%
\pgfpathcurveto{\pgfqpoint{1.516266in}{1.992806in}}{\pgfqpoint{1.512994in}{1.984906in}}{\pgfqpoint{1.512994in}{1.976669in}}%
\pgfpathcurveto{\pgfqpoint{1.512994in}{1.968433in}}{\pgfqpoint{1.516266in}{1.960533in}}{\pgfqpoint{1.522090in}{1.954709in}}%
\pgfpathcurveto{\pgfqpoint{1.527914in}{1.948885in}}{\pgfqpoint{1.535814in}{1.945613in}}{\pgfqpoint{1.544050in}{1.945613in}}%
\pgfpathclose%
\pgfusepath{stroke,fill}%
\end{pgfscope}%
\begin{pgfscope}%
\pgfpathrectangle{\pgfqpoint{0.100000in}{0.212622in}}{\pgfqpoint{3.696000in}{3.696000in}}%
\pgfusepath{clip}%
\pgfsetbuttcap%
\pgfsetroundjoin%
\definecolor{currentfill}{rgb}{0.121569,0.466667,0.705882}%
\pgfsetfillcolor{currentfill}%
\pgfsetfillopacity{0.393738}%
\pgfsetlinewidth{1.003750pt}%
\definecolor{currentstroke}{rgb}{0.121569,0.466667,0.705882}%
\pgfsetstrokecolor{currentstroke}%
\pgfsetstrokeopacity{0.393738}%
\pgfsetdash{}{0pt}%
\pgfpathmoveto{\pgfqpoint{1.544697in}{1.944946in}}%
\pgfpathcurveto{\pgfqpoint{1.552933in}{1.944946in}}{\pgfqpoint{1.560833in}{1.948218in}}{\pgfqpoint{1.566657in}{1.954042in}}%
\pgfpathcurveto{\pgfqpoint{1.572481in}{1.959866in}}{\pgfqpoint{1.575753in}{1.967766in}}{\pgfqpoint{1.575753in}{1.976002in}}%
\pgfpathcurveto{\pgfqpoint{1.575753in}{1.984239in}}{\pgfqpoint{1.572481in}{1.992139in}}{\pgfqpoint{1.566657in}{1.997963in}}%
\pgfpathcurveto{\pgfqpoint{1.560833in}{2.003787in}}{\pgfqpoint{1.552933in}{2.007059in}}{\pgfqpoint{1.544697in}{2.007059in}}%
\pgfpathcurveto{\pgfqpoint{1.536460in}{2.007059in}}{\pgfqpoint{1.528560in}{2.003787in}}{\pgfqpoint{1.522736in}{1.997963in}}%
\pgfpathcurveto{\pgfqpoint{1.516913in}{1.992139in}}{\pgfqpoint{1.513640in}{1.984239in}}{\pgfqpoint{1.513640in}{1.976002in}}%
\pgfpathcurveto{\pgfqpoint{1.513640in}{1.967766in}}{\pgfqpoint{1.516913in}{1.959866in}}{\pgfqpoint{1.522736in}{1.954042in}}%
\pgfpathcurveto{\pgfqpoint{1.528560in}{1.948218in}}{\pgfqpoint{1.536460in}{1.944946in}}{\pgfqpoint{1.544697in}{1.944946in}}%
\pgfpathclose%
\pgfusepath{stroke,fill}%
\end{pgfscope}%
\begin{pgfscope}%
\pgfpathrectangle{\pgfqpoint{0.100000in}{0.212622in}}{\pgfqpoint{3.696000in}{3.696000in}}%
\pgfusepath{clip}%
\pgfsetbuttcap%
\pgfsetroundjoin%
\definecolor{currentfill}{rgb}{0.121569,0.466667,0.705882}%
\pgfsetfillcolor{currentfill}%
\pgfsetfillopacity{0.395146}%
\pgfsetlinewidth{1.003750pt}%
\definecolor{currentstroke}{rgb}{0.121569,0.466667,0.705882}%
\pgfsetstrokecolor{currentstroke}%
\pgfsetstrokeopacity{0.395146}%
\pgfsetdash{}{0pt}%
\pgfpathmoveto{\pgfqpoint{1.549203in}{1.945931in}}%
\pgfpathcurveto{\pgfqpoint{1.557440in}{1.945931in}}{\pgfqpoint{1.565340in}{1.949204in}}{\pgfqpoint{1.571164in}{1.955028in}}%
\pgfpathcurveto{\pgfqpoint{1.576988in}{1.960851in}}{\pgfqpoint{1.580260in}{1.968752in}}{\pgfqpoint{1.580260in}{1.976988in}}%
\pgfpathcurveto{\pgfqpoint{1.580260in}{1.985224in}}{\pgfqpoint{1.576988in}{1.993124in}}{\pgfqpoint{1.571164in}{1.998948in}}%
\pgfpathcurveto{\pgfqpoint{1.565340in}{2.004772in}}{\pgfqpoint{1.557440in}{2.008044in}}{\pgfqpoint{1.549203in}{2.008044in}}%
\pgfpathcurveto{\pgfqpoint{1.540967in}{2.008044in}}{\pgfqpoint{1.533067in}{2.004772in}}{\pgfqpoint{1.527243in}{1.998948in}}%
\pgfpathcurveto{\pgfqpoint{1.521419in}{1.993124in}}{\pgfqpoint{1.518147in}{1.985224in}}{\pgfqpoint{1.518147in}{1.976988in}}%
\pgfpathcurveto{\pgfqpoint{1.518147in}{1.968752in}}{\pgfqpoint{1.521419in}{1.960851in}}{\pgfqpoint{1.527243in}{1.955028in}}%
\pgfpathcurveto{\pgfqpoint{1.533067in}{1.949204in}}{\pgfqpoint{1.540967in}{1.945931in}}{\pgfqpoint{1.549203in}{1.945931in}}%
\pgfpathclose%
\pgfusepath{stroke,fill}%
\end{pgfscope}%
\begin{pgfscope}%
\pgfpathrectangle{\pgfqpoint{0.100000in}{0.212622in}}{\pgfqpoint{3.696000in}{3.696000in}}%
\pgfusepath{clip}%
\pgfsetbuttcap%
\pgfsetroundjoin%
\definecolor{currentfill}{rgb}{0.121569,0.466667,0.705882}%
\pgfsetfillcolor{currentfill}%
\pgfsetfillopacity{0.400691}%
\pgfsetlinewidth{1.003750pt}%
\definecolor{currentstroke}{rgb}{0.121569,0.466667,0.705882}%
\pgfsetstrokecolor{currentstroke}%
\pgfsetstrokeopacity{0.400691}%
\pgfsetdash{}{0pt}%
\pgfpathmoveto{\pgfqpoint{1.543503in}{1.942855in}}%
\pgfpathcurveto{\pgfqpoint{1.551739in}{1.942855in}}{\pgfqpoint{1.559639in}{1.946127in}}{\pgfqpoint{1.565463in}{1.951951in}}%
\pgfpathcurveto{\pgfqpoint{1.571287in}{1.957775in}}{\pgfqpoint{1.574559in}{1.965675in}}{\pgfqpoint{1.574559in}{1.973911in}}%
\pgfpathcurveto{\pgfqpoint{1.574559in}{1.982148in}}{\pgfqpoint{1.571287in}{1.990048in}}{\pgfqpoint{1.565463in}{1.995872in}}%
\pgfpathcurveto{\pgfqpoint{1.559639in}{2.001696in}}{\pgfqpoint{1.551739in}{2.004968in}}{\pgfqpoint{1.543503in}{2.004968in}}%
\pgfpathcurveto{\pgfqpoint{1.535266in}{2.004968in}}{\pgfqpoint{1.527366in}{2.001696in}}{\pgfqpoint{1.521542in}{1.995872in}}%
\pgfpathcurveto{\pgfqpoint{1.515718in}{1.990048in}}{\pgfqpoint{1.512446in}{1.982148in}}{\pgfqpoint{1.512446in}{1.973911in}}%
\pgfpathcurveto{\pgfqpoint{1.512446in}{1.965675in}}{\pgfqpoint{1.515718in}{1.957775in}}{\pgfqpoint{1.521542in}{1.951951in}}%
\pgfpathcurveto{\pgfqpoint{1.527366in}{1.946127in}}{\pgfqpoint{1.535266in}{1.942855in}}{\pgfqpoint{1.543503in}{1.942855in}}%
\pgfpathclose%
\pgfusepath{stroke,fill}%
\end{pgfscope}%
\begin{pgfscope}%
\pgfpathrectangle{\pgfqpoint{0.100000in}{0.212622in}}{\pgfqpoint{3.696000in}{3.696000in}}%
\pgfusepath{clip}%
\pgfsetbuttcap%
\pgfsetroundjoin%
\definecolor{currentfill}{rgb}{0.121569,0.466667,0.705882}%
\pgfsetfillcolor{currentfill}%
\pgfsetfillopacity{0.402804}%
\pgfsetlinewidth{1.003750pt}%
\definecolor{currentstroke}{rgb}{0.121569,0.466667,0.705882}%
\pgfsetstrokecolor{currentstroke}%
\pgfsetstrokeopacity{0.402804}%
\pgfsetdash{}{0pt}%
\pgfpathmoveto{\pgfqpoint{2.514954in}{2.493407in}}%
\pgfpathcurveto{\pgfqpoint{2.523191in}{2.493407in}}{\pgfqpoint{2.531091in}{2.496679in}}{\pgfqpoint{2.536915in}{2.502503in}}%
\pgfpathcurveto{\pgfqpoint{2.542739in}{2.508327in}}{\pgfqpoint{2.546011in}{2.516227in}}{\pgfqpoint{2.546011in}{2.524463in}}%
\pgfpathcurveto{\pgfqpoint{2.546011in}{2.532699in}}{\pgfqpoint{2.542739in}{2.540600in}}{\pgfqpoint{2.536915in}{2.546423in}}%
\pgfpathcurveto{\pgfqpoint{2.531091in}{2.552247in}}{\pgfqpoint{2.523191in}{2.555520in}}{\pgfqpoint{2.514954in}{2.555520in}}%
\pgfpathcurveto{\pgfqpoint{2.506718in}{2.555520in}}{\pgfqpoint{2.498818in}{2.552247in}}{\pgfqpoint{2.492994in}{2.546423in}}%
\pgfpathcurveto{\pgfqpoint{2.487170in}{2.540600in}}{\pgfqpoint{2.483898in}{2.532699in}}{\pgfqpoint{2.483898in}{2.524463in}}%
\pgfpathcurveto{\pgfqpoint{2.483898in}{2.516227in}}{\pgfqpoint{2.487170in}{2.508327in}}{\pgfqpoint{2.492994in}{2.502503in}}%
\pgfpathcurveto{\pgfqpoint{2.498818in}{2.496679in}}{\pgfqpoint{2.506718in}{2.493407in}}{\pgfqpoint{2.514954in}{2.493407in}}%
\pgfpathclose%
\pgfusepath{stroke,fill}%
\end{pgfscope}%
\begin{pgfscope}%
\pgfpathrectangle{\pgfqpoint{0.100000in}{0.212622in}}{\pgfqpoint{3.696000in}{3.696000in}}%
\pgfusepath{clip}%
\pgfsetbuttcap%
\pgfsetroundjoin%
\definecolor{currentfill}{rgb}{0.121569,0.466667,0.705882}%
\pgfsetfillcolor{currentfill}%
\pgfsetfillopacity{0.403689}%
\pgfsetlinewidth{1.003750pt}%
\definecolor{currentstroke}{rgb}{0.121569,0.466667,0.705882}%
\pgfsetstrokecolor{currentstroke}%
\pgfsetstrokeopacity{0.403689}%
\pgfsetdash{}{0pt}%
\pgfpathmoveto{\pgfqpoint{1.554890in}{1.944600in}}%
\pgfpathcurveto{\pgfqpoint{1.563126in}{1.944600in}}{\pgfqpoint{1.571026in}{1.947873in}}{\pgfqpoint{1.576850in}{1.953697in}}%
\pgfpathcurveto{\pgfqpoint{1.582674in}{1.959520in}}{\pgfqpoint{1.585947in}{1.967421in}}{\pgfqpoint{1.585947in}{1.975657in}}%
\pgfpathcurveto{\pgfqpoint{1.585947in}{1.983893in}}{\pgfqpoint{1.582674in}{1.991793in}}{\pgfqpoint{1.576850in}{1.997617in}}%
\pgfpathcurveto{\pgfqpoint{1.571026in}{2.003441in}}{\pgfqpoint{1.563126in}{2.006713in}}{\pgfqpoint{1.554890in}{2.006713in}}%
\pgfpathcurveto{\pgfqpoint{1.546654in}{2.006713in}}{\pgfqpoint{1.538754in}{2.003441in}}{\pgfqpoint{1.532930in}{1.997617in}}%
\pgfpathcurveto{\pgfqpoint{1.527106in}{1.991793in}}{\pgfqpoint{1.523834in}{1.983893in}}{\pgfqpoint{1.523834in}{1.975657in}}%
\pgfpathcurveto{\pgfqpoint{1.523834in}{1.967421in}}{\pgfqpoint{1.527106in}{1.959520in}}{\pgfqpoint{1.532930in}{1.953697in}}%
\pgfpathcurveto{\pgfqpoint{1.538754in}{1.947873in}}{\pgfqpoint{1.546654in}{1.944600in}}{\pgfqpoint{1.554890in}{1.944600in}}%
\pgfpathclose%
\pgfusepath{stroke,fill}%
\end{pgfscope}%
\begin{pgfscope}%
\pgfpathrectangle{\pgfqpoint{0.100000in}{0.212622in}}{\pgfqpoint{3.696000in}{3.696000in}}%
\pgfusepath{clip}%
\pgfsetbuttcap%
\pgfsetroundjoin%
\definecolor{currentfill}{rgb}{0.121569,0.466667,0.705882}%
\pgfsetfillcolor{currentfill}%
\pgfsetfillopacity{0.405548}%
\pgfsetlinewidth{1.003750pt}%
\definecolor{currentstroke}{rgb}{0.121569,0.466667,0.705882}%
\pgfsetstrokecolor{currentstroke}%
\pgfsetstrokeopacity{0.405548}%
\pgfsetdash{}{0pt}%
\pgfpathmoveto{\pgfqpoint{2.517855in}{2.492304in}}%
\pgfpathcurveto{\pgfqpoint{2.526091in}{2.492304in}}{\pgfqpoint{2.533991in}{2.495576in}}{\pgfqpoint{2.539815in}{2.501400in}}%
\pgfpathcurveto{\pgfqpoint{2.545639in}{2.507224in}}{\pgfqpoint{2.548911in}{2.515124in}}{\pgfqpoint{2.548911in}{2.523361in}}%
\pgfpathcurveto{\pgfqpoint{2.548911in}{2.531597in}}{\pgfqpoint{2.545639in}{2.539497in}}{\pgfqpoint{2.539815in}{2.545321in}}%
\pgfpathcurveto{\pgfqpoint{2.533991in}{2.551145in}}{\pgfqpoint{2.526091in}{2.554417in}}{\pgfqpoint{2.517855in}{2.554417in}}%
\pgfpathcurveto{\pgfqpoint{2.509619in}{2.554417in}}{\pgfqpoint{2.501719in}{2.551145in}}{\pgfqpoint{2.495895in}{2.545321in}}%
\pgfpathcurveto{\pgfqpoint{2.490071in}{2.539497in}}{\pgfqpoint{2.486798in}{2.531597in}}{\pgfqpoint{2.486798in}{2.523361in}}%
\pgfpathcurveto{\pgfqpoint{2.486798in}{2.515124in}}{\pgfqpoint{2.490071in}{2.507224in}}{\pgfqpoint{2.495895in}{2.501400in}}%
\pgfpathcurveto{\pgfqpoint{2.501719in}{2.495576in}}{\pgfqpoint{2.509619in}{2.492304in}}{\pgfqpoint{2.517855in}{2.492304in}}%
\pgfpathclose%
\pgfusepath{stroke,fill}%
\end{pgfscope}%
\begin{pgfscope}%
\pgfpathrectangle{\pgfqpoint{0.100000in}{0.212622in}}{\pgfqpoint{3.696000in}{3.696000in}}%
\pgfusepath{clip}%
\pgfsetbuttcap%
\pgfsetroundjoin%
\definecolor{currentfill}{rgb}{0.121569,0.466667,0.705882}%
\pgfsetfillcolor{currentfill}%
\pgfsetfillopacity{0.406791}%
\pgfsetlinewidth{1.003750pt}%
\definecolor{currentstroke}{rgb}{0.121569,0.466667,0.705882}%
\pgfsetstrokecolor{currentstroke}%
\pgfsetstrokeopacity{0.406791}%
\pgfsetdash{}{0pt}%
\pgfpathmoveto{\pgfqpoint{1.534743in}{1.931743in}}%
\pgfpathcurveto{\pgfqpoint{1.542979in}{1.931743in}}{\pgfqpoint{1.550879in}{1.935016in}}{\pgfqpoint{1.556703in}{1.940839in}}%
\pgfpathcurveto{\pgfqpoint{1.562527in}{1.946663in}}{\pgfqpoint{1.565799in}{1.954563in}}{\pgfqpoint{1.565799in}{1.962800in}}%
\pgfpathcurveto{\pgfqpoint{1.565799in}{1.971036in}}{\pgfqpoint{1.562527in}{1.978936in}}{\pgfqpoint{1.556703in}{1.984760in}}%
\pgfpathcurveto{\pgfqpoint{1.550879in}{1.990584in}}{\pgfqpoint{1.542979in}{1.993856in}}{\pgfqpoint{1.534743in}{1.993856in}}%
\pgfpathcurveto{\pgfqpoint{1.526506in}{1.993856in}}{\pgfqpoint{1.518606in}{1.990584in}}{\pgfqpoint{1.512782in}{1.984760in}}%
\pgfpathcurveto{\pgfqpoint{1.506958in}{1.978936in}}{\pgfqpoint{1.503686in}{1.971036in}}{\pgfqpoint{1.503686in}{1.962800in}}%
\pgfpathcurveto{\pgfqpoint{1.503686in}{1.954563in}}{\pgfqpoint{1.506958in}{1.946663in}}{\pgfqpoint{1.512782in}{1.940839in}}%
\pgfpathcurveto{\pgfqpoint{1.518606in}{1.935016in}}{\pgfqpoint{1.526506in}{1.931743in}}{\pgfqpoint{1.534743in}{1.931743in}}%
\pgfpathclose%
\pgfusepath{stroke,fill}%
\end{pgfscope}%
\begin{pgfscope}%
\pgfpathrectangle{\pgfqpoint{0.100000in}{0.212622in}}{\pgfqpoint{3.696000in}{3.696000in}}%
\pgfusepath{clip}%
\pgfsetbuttcap%
\pgfsetroundjoin%
\definecolor{currentfill}{rgb}{0.121569,0.466667,0.705882}%
\pgfsetfillcolor{currentfill}%
\pgfsetfillopacity{0.409771}%
\pgfsetlinewidth{1.003750pt}%
\definecolor{currentstroke}{rgb}{0.121569,0.466667,0.705882}%
\pgfsetstrokecolor{currentstroke}%
\pgfsetstrokeopacity{0.409771}%
\pgfsetdash{}{0pt}%
\pgfpathmoveto{\pgfqpoint{2.485470in}{2.468012in}}%
\pgfpathcurveto{\pgfqpoint{2.493706in}{2.468012in}}{\pgfqpoint{2.501606in}{2.471285in}}{\pgfqpoint{2.507430in}{2.477108in}}%
\pgfpathcurveto{\pgfqpoint{2.513254in}{2.482932in}}{\pgfqpoint{2.516526in}{2.490832in}}{\pgfqpoint{2.516526in}{2.499069in}}%
\pgfpathcurveto{\pgfqpoint{2.516526in}{2.507305in}}{\pgfqpoint{2.513254in}{2.515205in}}{\pgfqpoint{2.507430in}{2.521029in}}%
\pgfpathcurveto{\pgfqpoint{2.501606in}{2.526853in}}{\pgfqpoint{2.493706in}{2.530125in}}{\pgfqpoint{2.485470in}{2.530125in}}%
\pgfpathcurveto{\pgfqpoint{2.477233in}{2.530125in}}{\pgfqpoint{2.469333in}{2.526853in}}{\pgfqpoint{2.463509in}{2.521029in}}%
\pgfpathcurveto{\pgfqpoint{2.457685in}{2.515205in}}{\pgfqpoint{2.454413in}{2.507305in}}{\pgfqpoint{2.454413in}{2.499069in}}%
\pgfpathcurveto{\pgfqpoint{2.454413in}{2.490832in}}{\pgfqpoint{2.457685in}{2.482932in}}{\pgfqpoint{2.463509in}{2.477108in}}%
\pgfpathcurveto{\pgfqpoint{2.469333in}{2.471285in}}{\pgfqpoint{2.477233in}{2.468012in}}{\pgfqpoint{2.485470in}{2.468012in}}%
\pgfpathclose%
\pgfusepath{stroke,fill}%
\end{pgfscope}%
\begin{pgfscope}%
\pgfpathrectangle{\pgfqpoint{0.100000in}{0.212622in}}{\pgfqpoint{3.696000in}{3.696000in}}%
\pgfusepath{clip}%
\pgfsetbuttcap%
\pgfsetroundjoin%
\definecolor{currentfill}{rgb}{0.121569,0.466667,0.705882}%
\pgfsetfillcolor{currentfill}%
\pgfsetfillopacity{0.411447}%
\pgfsetlinewidth{1.003750pt}%
\definecolor{currentstroke}{rgb}{0.121569,0.466667,0.705882}%
\pgfsetstrokecolor{currentstroke}%
\pgfsetstrokeopacity{0.411447}%
\pgfsetdash{}{0pt}%
\pgfpathmoveto{\pgfqpoint{2.471421in}{2.455046in}}%
\pgfpathcurveto{\pgfqpoint{2.479658in}{2.455046in}}{\pgfqpoint{2.487558in}{2.458319in}}{\pgfqpoint{2.493382in}{2.464143in}}%
\pgfpathcurveto{\pgfqpoint{2.499205in}{2.469967in}}{\pgfqpoint{2.502478in}{2.477867in}}{\pgfqpoint{2.502478in}{2.486103in}}%
\pgfpathcurveto{\pgfqpoint{2.502478in}{2.494339in}}{\pgfqpoint{2.499205in}{2.502239in}}{\pgfqpoint{2.493382in}{2.508063in}}%
\pgfpathcurveto{\pgfqpoint{2.487558in}{2.513887in}}{\pgfqpoint{2.479658in}{2.517159in}}{\pgfqpoint{2.471421in}{2.517159in}}%
\pgfpathcurveto{\pgfqpoint{2.463185in}{2.517159in}}{\pgfqpoint{2.455285in}{2.513887in}}{\pgfqpoint{2.449461in}{2.508063in}}%
\pgfpathcurveto{\pgfqpoint{2.443637in}{2.502239in}}{\pgfqpoint{2.440365in}{2.494339in}}{\pgfqpoint{2.440365in}{2.486103in}}%
\pgfpathcurveto{\pgfqpoint{2.440365in}{2.477867in}}{\pgfqpoint{2.443637in}{2.469967in}}{\pgfqpoint{2.449461in}{2.464143in}}%
\pgfpathcurveto{\pgfqpoint{2.455285in}{2.458319in}}{\pgfqpoint{2.463185in}{2.455046in}}{\pgfqpoint{2.471421in}{2.455046in}}%
\pgfpathclose%
\pgfusepath{stroke,fill}%
\end{pgfscope}%
\begin{pgfscope}%
\pgfpathrectangle{\pgfqpoint{0.100000in}{0.212622in}}{\pgfqpoint{3.696000in}{3.696000in}}%
\pgfusepath{clip}%
\pgfsetbuttcap%
\pgfsetroundjoin%
\definecolor{currentfill}{rgb}{0.121569,0.466667,0.705882}%
\pgfsetfillcolor{currentfill}%
\pgfsetfillopacity{0.412066}%
\pgfsetlinewidth{1.003750pt}%
\definecolor{currentstroke}{rgb}{0.121569,0.466667,0.705882}%
\pgfsetstrokecolor{currentstroke}%
\pgfsetstrokeopacity{0.412066}%
\pgfsetdash{}{0pt}%
\pgfpathmoveto{\pgfqpoint{1.562335in}{1.946604in}}%
\pgfpathcurveto{\pgfqpoint{1.570571in}{1.946604in}}{\pgfqpoint{1.578471in}{1.949877in}}{\pgfqpoint{1.584295in}{1.955701in}}%
\pgfpathcurveto{\pgfqpoint{1.590119in}{1.961525in}}{\pgfqpoint{1.593392in}{1.969425in}}{\pgfqpoint{1.593392in}{1.977661in}}%
\pgfpathcurveto{\pgfqpoint{1.593392in}{1.985897in}}{\pgfqpoint{1.590119in}{1.993797in}}{\pgfqpoint{1.584295in}{1.999621in}}%
\pgfpathcurveto{\pgfqpoint{1.578471in}{2.005445in}}{\pgfqpoint{1.570571in}{2.008717in}}{\pgfqpoint{1.562335in}{2.008717in}}%
\pgfpathcurveto{\pgfqpoint{1.554099in}{2.008717in}}{\pgfqpoint{1.546199in}{2.005445in}}{\pgfqpoint{1.540375in}{1.999621in}}%
\pgfpathcurveto{\pgfqpoint{1.534551in}{1.993797in}}{\pgfqpoint{1.531279in}{1.985897in}}{\pgfqpoint{1.531279in}{1.977661in}}%
\pgfpathcurveto{\pgfqpoint{1.531279in}{1.969425in}}{\pgfqpoint{1.534551in}{1.961525in}}{\pgfqpoint{1.540375in}{1.955701in}}%
\pgfpathcurveto{\pgfqpoint{1.546199in}{1.949877in}}{\pgfqpoint{1.554099in}{1.946604in}}{\pgfqpoint{1.562335in}{1.946604in}}%
\pgfpathclose%
\pgfusepath{stroke,fill}%
\end{pgfscope}%
\begin{pgfscope}%
\pgfpathrectangle{\pgfqpoint{0.100000in}{0.212622in}}{\pgfqpoint{3.696000in}{3.696000in}}%
\pgfusepath{clip}%
\pgfsetbuttcap%
\pgfsetroundjoin%
\definecolor{currentfill}{rgb}{0.121569,0.466667,0.705882}%
\pgfsetfillcolor{currentfill}%
\pgfsetfillopacity{0.412247}%
\pgfsetlinewidth{1.003750pt}%
\definecolor{currentstroke}{rgb}{0.121569,0.466667,0.705882}%
\pgfsetstrokecolor{currentstroke}%
\pgfsetstrokeopacity{0.412247}%
\pgfsetdash{}{0pt}%
\pgfpathmoveto{\pgfqpoint{2.510862in}{2.482779in}}%
\pgfpathcurveto{\pgfqpoint{2.519098in}{2.482779in}}{\pgfqpoint{2.526998in}{2.486051in}}{\pgfqpoint{2.532822in}{2.491875in}}%
\pgfpathcurveto{\pgfqpoint{2.538646in}{2.497699in}}{\pgfqpoint{2.541918in}{2.505599in}}{\pgfqpoint{2.541918in}{2.513836in}}%
\pgfpathcurveto{\pgfqpoint{2.541918in}{2.522072in}}{\pgfqpoint{2.538646in}{2.529972in}}{\pgfqpoint{2.532822in}{2.535796in}}%
\pgfpathcurveto{\pgfqpoint{2.526998in}{2.541620in}}{\pgfqpoint{2.519098in}{2.544892in}}{\pgfqpoint{2.510862in}{2.544892in}}%
\pgfpathcurveto{\pgfqpoint{2.502625in}{2.544892in}}{\pgfqpoint{2.494725in}{2.541620in}}{\pgfqpoint{2.488901in}{2.535796in}}%
\pgfpathcurveto{\pgfqpoint{2.483077in}{2.529972in}}{\pgfqpoint{2.479805in}{2.522072in}}{\pgfqpoint{2.479805in}{2.513836in}}%
\pgfpathcurveto{\pgfqpoint{2.479805in}{2.505599in}}{\pgfqpoint{2.483077in}{2.497699in}}{\pgfqpoint{2.488901in}{2.491875in}}%
\pgfpathcurveto{\pgfqpoint{2.494725in}{2.486051in}}{\pgfqpoint{2.502625in}{2.482779in}}{\pgfqpoint{2.510862in}{2.482779in}}%
\pgfpathclose%
\pgfusepath{stroke,fill}%
\end{pgfscope}%
\begin{pgfscope}%
\pgfpathrectangle{\pgfqpoint{0.100000in}{0.212622in}}{\pgfqpoint{3.696000in}{3.696000in}}%
\pgfusepath{clip}%
\pgfsetbuttcap%
\pgfsetroundjoin%
\definecolor{currentfill}{rgb}{0.121569,0.466667,0.705882}%
\pgfsetfillcolor{currentfill}%
\pgfsetfillopacity{0.413935}%
\pgfsetlinewidth{1.003750pt}%
\definecolor{currentstroke}{rgb}{0.121569,0.466667,0.705882}%
\pgfsetstrokecolor{currentstroke}%
\pgfsetstrokeopacity{0.413935}%
\pgfsetdash{}{0pt}%
\pgfpathmoveto{\pgfqpoint{1.549741in}{1.937298in}}%
\pgfpathcurveto{\pgfqpoint{1.557977in}{1.937298in}}{\pgfqpoint{1.565877in}{1.940571in}}{\pgfqpoint{1.571701in}{1.946395in}}%
\pgfpathcurveto{\pgfqpoint{1.577525in}{1.952219in}}{\pgfqpoint{1.580797in}{1.960119in}}{\pgfqpoint{1.580797in}{1.968355in}}%
\pgfpathcurveto{\pgfqpoint{1.580797in}{1.976591in}}{\pgfqpoint{1.577525in}{1.984491in}}{\pgfqpoint{1.571701in}{1.990315in}}%
\pgfpathcurveto{\pgfqpoint{1.565877in}{1.996139in}}{\pgfqpoint{1.557977in}{1.999411in}}{\pgfqpoint{1.549741in}{1.999411in}}%
\pgfpathcurveto{\pgfqpoint{1.541504in}{1.999411in}}{\pgfqpoint{1.533604in}{1.996139in}}{\pgfqpoint{1.527780in}{1.990315in}}%
\pgfpathcurveto{\pgfqpoint{1.521957in}{1.984491in}}{\pgfqpoint{1.518684in}{1.976591in}}{\pgfqpoint{1.518684in}{1.968355in}}%
\pgfpathcurveto{\pgfqpoint{1.518684in}{1.960119in}}{\pgfqpoint{1.521957in}{1.952219in}}{\pgfqpoint{1.527780in}{1.946395in}}%
\pgfpathcurveto{\pgfqpoint{1.533604in}{1.940571in}}{\pgfqpoint{1.541504in}{1.937298in}}{\pgfqpoint{1.549741in}{1.937298in}}%
\pgfpathclose%
\pgfusepath{stroke,fill}%
\end{pgfscope}%
\begin{pgfscope}%
\pgfpathrectangle{\pgfqpoint{0.100000in}{0.212622in}}{\pgfqpoint{3.696000in}{3.696000in}}%
\pgfusepath{clip}%
\pgfsetbuttcap%
\pgfsetroundjoin%
\definecolor{currentfill}{rgb}{0.121569,0.466667,0.705882}%
\pgfsetfillcolor{currentfill}%
\pgfsetfillopacity{0.414443}%
\pgfsetlinewidth{1.003750pt}%
\definecolor{currentstroke}{rgb}{0.121569,0.466667,0.705882}%
\pgfsetstrokecolor{currentstroke}%
\pgfsetstrokeopacity{0.414443}%
\pgfsetdash{}{0pt}%
\pgfpathmoveto{\pgfqpoint{2.453795in}{2.437474in}}%
\pgfpathcurveto{\pgfqpoint{2.462032in}{2.437474in}}{\pgfqpoint{2.469932in}{2.440747in}}{\pgfqpoint{2.475756in}{2.446571in}}%
\pgfpathcurveto{\pgfqpoint{2.481579in}{2.452395in}}{\pgfqpoint{2.484852in}{2.460295in}}{\pgfqpoint{2.484852in}{2.468531in}}%
\pgfpathcurveto{\pgfqpoint{2.484852in}{2.476767in}}{\pgfqpoint{2.481579in}{2.484667in}}{\pgfqpoint{2.475756in}{2.490491in}}%
\pgfpathcurveto{\pgfqpoint{2.469932in}{2.496315in}}{\pgfqpoint{2.462032in}{2.499587in}}{\pgfqpoint{2.453795in}{2.499587in}}%
\pgfpathcurveto{\pgfqpoint{2.445559in}{2.499587in}}{\pgfqpoint{2.437659in}{2.496315in}}{\pgfqpoint{2.431835in}{2.490491in}}%
\pgfpathcurveto{\pgfqpoint{2.426011in}{2.484667in}}{\pgfqpoint{2.422739in}{2.476767in}}{\pgfqpoint{2.422739in}{2.468531in}}%
\pgfpathcurveto{\pgfqpoint{2.422739in}{2.460295in}}{\pgfqpoint{2.426011in}{2.452395in}}{\pgfqpoint{2.431835in}{2.446571in}}%
\pgfpathcurveto{\pgfqpoint{2.437659in}{2.440747in}}{\pgfqpoint{2.445559in}{2.437474in}}{\pgfqpoint{2.453795in}{2.437474in}}%
\pgfpathclose%
\pgfusepath{stroke,fill}%
\end{pgfscope}%
\begin{pgfscope}%
\pgfpathrectangle{\pgfqpoint{0.100000in}{0.212622in}}{\pgfqpoint{3.696000in}{3.696000in}}%
\pgfusepath{clip}%
\pgfsetbuttcap%
\pgfsetroundjoin%
\definecolor{currentfill}{rgb}{0.121569,0.466667,0.705882}%
\pgfsetfillcolor{currentfill}%
\pgfsetfillopacity{0.414471}%
\pgfsetlinewidth{1.003750pt}%
\definecolor{currentstroke}{rgb}{0.121569,0.466667,0.705882}%
\pgfsetstrokecolor{currentstroke}%
\pgfsetstrokeopacity{0.414471}%
\pgfsetdash{}{0pt}%
\pgfpathmoveto{\pgfqpoint{1.558910in}{1.942787in}}%
\pgfpathcurveto{\pgfqpoint{1.567146in}{1.942787in}}{\pgfqpoint{1.575047in}{1.946059in}}{\pgfqpoint{1.580870in}{1.951883in}}%
\pgfpathcurveto{\pgfqpoint{1.586694in}{1.957707in}}{\pgfqpoint{1.589967in}{1.965607in}}{\pgfqpoint{1.589967in}{1.973843in}}%
\pgfpathcurveto{\pgfqpoint{1.589967in}{1.982080in}}{\pgfqpoint{1.586694in}{1.989980in}}{\pgfqpoint{1.580870in}{1.995804in}}%
\pgfpathcurveto{\pgfqpoint{1.575047in}{2.001628in}}{\pgfqpoint{1.567146in}{2.004900in}}{\pgfqpoint{1.558910in}{2.004900in}}%
\pgfpathcurveto{\pgfqpoint{1.550674in}{2.004900in}}{\pgfqpoint{1.542774in}{2.001628in}}{\pgfqpoint{1.536950in}{1.995804in}}%
\pgfpathcurveto{\pgfqpoint{1.531126in}{1.989980in}}{\pgfqpoint{1.527854in}{1.982080in}}{\pgfqpoint{1.527854in}{1.973843in}}%
\pgfpathcurveto{\pgfqpoint{1.527854in}{1.965607in}}{\pgfqpoint{1.531126in}{1.957707in}}{\pgfqpoint{1.536950in}{1.951883in}}%
\pgfpathcurveto{\pgfqpoint{1.542774in}{1.946059in}}{\pgfqpoint{1.550674in}{1.942787in}}{\pgfqpoint{1.558910in}{1.942787in}}%
\pgfpathclose%
\pgfusepath{stroke,fill}%
\end{pgfscope}%
\begin{pgfscope}%
\pgfpathrectangle{\pgfqpoint{0.100000in}{0.212622in}}{\pgfqpoint{3.696000in}{3.696000in}}%
\pgfusepath{clip}%
\pgfsetbuttcap%
\pgfsetroundjoin%
\definecolor{currentfill}{rgb}{0.121569,0.466667,0.705882}%
\pgfsetfillcolor{currentfill}%
\pgfsetfillopacity{0.414856}%
\pgfsetlinewidth{1.003750pt}%
\definecolor{currentstroke}{rgb}{0.121569,0.466667,0.705882}%
\pgfsetstrokecolor{currentstroke}%
\pgfsetstrokeopacity{0.414856}%
\pgfsetdash{}{0pt}%
\pgfpathmoveto{\pgfqpoint{1.533582in}{1.924340in}}%
\pgfpathcurveto{\pgfqpoint{1.541818in}{1.924340in}}{\pgfqpoint{1.549718in}{1.927613in}}{\pgfqpoint{1.555542in}{1.933437in}}%
\pgfpathcurveto{\pgfqpoint{1.561366in}{1.939261in}}{\pgfqpoint{1.564638in}{1.947161in}}{\pgfqpoint{1.564638in}{1.955397in}}%
\pgfpathcurveto{\pgfqpoint{1.564638in}{1.963633in}}{\pgfqpoint{1.561366in}{1.971533in}}{\pgfqpoint{1.555542in}{1.977357in}}%
\pgfpathcurveto{\pgfqpoint{1.549718in}{1.983181in}}{\pgfqpoint{1.541818in}{1.986453in}}{\pgfqpoint{1.533582in}{1.986453in}}%
\pgfpathcurveto{\pgfqpoint{1.525345in}{1.986453in}}{\pgfqpoint{1.517445in}{1.983181in}}{\pgfqpoint{1.511621in}{1.977357in}}%
\pgfpathcurveto{\pgfqpoint{1.505797in}{1.971533in}}{\pgfqpoint{1.502525in}{1.963633in}}{\pgfqpoint{1.502525in}{1.955397in}}%
\pgfpathcurveto{\pgfqpoint{1.502525in}{1.947161in}}{\pgfqpoint{1.505797in}{1.939261in}}{\pgfqpoint{1.511621in}{1.933437in}}%
\pgfpathcurveto{\pgfqpoint{1.517445in}{1.927613in}}{\pgfqpoint{1.525345in}{1.924340in}}{\pgfqpoint{1.533582in}{1.924340in}}%
\pgfpathclose%
\pgfusepath{stroke,fill}%
\end{pgfscope}%
\begin{pgfscope}%
\pgfpathrectangle{\pgfqpoint{0.100000in}{0.212622in}}{\pgfqpoint{3.696000in}{3.696000in}}%
\pgfusepath{clip}%
\pgfsetbuttcap%
\pgfsetroundjoin%
\definecolor{currentfill}{rgb}{0.121569,0.466667,0.705882}%
\pgfsetfillcolor{currentfill}%
\pgfsetfillopacity{0.415666}%
\pgfsetlinewidth{1.003750pt}%
\definecolor{currentstroke}{rgb}{0.121569,0.466667,0.705882}%
\pgfsetstrokecolor{currentstroke}%
\pgfsetstrokeopacity{0.415666}%
\pgfsetdash{}{0pt}%
\pgfpathmoveto{\pgfqpoint{1.560947in}{1.943625in}}%
\pgfpathcurveto{\pgfqpoint{1.569183in}{1.943625in}}{\pgfqpoint{1.577084in}{1.946897in}}{\pgfqpoint{1.582907in}{1.952721in}}%
\pgfpathcurveto{\pgfqpoint{1.588731in}{1.958545in}}{\pgfqpoint{1.592004in}{1.966445in}}{\pgfqpoint{1.592004in}{1.974682in}}%
\pgfpathcurveto{\pgfqpoint{1.592004in}{1.982918in}}{\pgfqpoint{1.588731in}{1.990818in}}{\pgfqpoint{1.582907in}{1.996642in}}%
\pgfpathcurveto{\pgfqpoint{1.577084in}{2.002466in}}{\pgfqpoint{1.569183in}{2.005738in}}{\pgfqpoint{1.560947in}{2.005738in}}%
\pgfpathcurveto{\pgfqpoint{1.552711in}{2.005738in}}{\pgfqpoint{1.544811in}{2.002466in}}{\pgfqpoint{1.538987in}{1.996642in}}%
\pgfpathcurveto{\pgfqpoint{1.533163in}{1.990818in}}{\pgfqpoint{1.529891in}{1.982918in}}{\pgfqpoint{1.529891in}{1.974682in}}%
\pgfpathcurveto{\pgfqpoint{1.529891in}{1.966445in}}{\pgfqpoint{1.533163in}{1.958545in}}{\pgfqpoint{1.538987in}{1.952721in}}%
\pgfpathcurveto{\pgfqpoint{1.544811in}{1.946897in}}{\pgfqpoint{1.552711in}{1.943625in}}{\pgfqpoint{1.560947in}{1.943625in}}%
\pgfpathclose%
\pgfusepath{stroke,fill}%
\end{pgfscope}%
\begin{pgfscope}%
\pgfpathrectangle{\pgfqpoint{0.100000in}{0.212622in}}{\pgfqpoint{3.696000in}{3.696000in}}%
\pgfusepath{clip}%
\pgfsetbuttcap%
\pgfsetroundjoin%
\definecolor{currentfill}{rgb}{0.121569,0.466667,0.705882}%
\pgfsetfillcolor{currentfill}%
\pgfsetfillopacity{0.415844}%
\pgfsetlinewidth{1.003750pt}%
\definecolor{currentstroke}{rgb}{0.121569,0.466667,0.705882}%
\pgfsetstrokecolor{currentstroke}%
\pgfsetstrokeopacity{0.415844}%
\pgfsetdash{}{0pt}%
\pgfpathmoveto{\pgfqpoint{2.443485in}{2.431090in}}%
\pgfpathcurveto{\pgfqpoint{2.451721in}{2.431090in}}{\pgfqpoint{2.459622in}{2.434362in}}{\pgfqpoint{2.465445in}{2.440186in}}%
\pgfpathcurveto{\pgfqpoint{2.471269in}{2.446010in}}{\pgfqpoint{2.474542in}{2.453910in}}{\pgfqpoint{2.474542in}{2.462147in}}%
\pgfpathcurveto{\pgfqpoint{2.474542in}{2.470383in}}{\pgfqpoint{2.471269in}{2.478283in}}{\pgfqpoint{2.465445in}{2.484107in}}%
\pgfpathcurveto{\pgfqpoint{2.459622in}{2.489931in}}{\pgfqpoint{2.451721in}{2.493203in}}{\pgfqpoint{2.443485in}{2.493203in}}%
\pgfpathcurveto{\pgfqpoint{2.435249in}{2.493203in}}{\pgfqpoint{2.427349in}{2.489931in}}{\pgfqpoint{2.421525in}{2.484107in}}%
\pgfpathcurveto{\pgfqpoint{2.415701in}{2.478283in}}{\pgfqpoint{2.412429in}{2.470383in}}{\pgfqpoint{2.412429in}{2.462147in}}%
\pgfpathcurveto{\pgfqpoint{2.412429in}{2.453910in}}{\pgfqpoint{2.415701in}{2.446010in}}{\pgfqpoint{2.421525in}{2.440186in}}%
\pgfpathcurveto{\pgfqpoint{2.427349in}{2.434362in}}{\pgfqpoint{2.435249in}{2.431090in}}{\pgfqpoint{2.443485in}{2.431090in}}%
\pgfpathclose%
\pgfusepath{stroke,fill}%
\end{pgfscope}%
\begin{pgfscope}%
\pgfpathrectangle{\pgfqpoint{0.100000in}{0.212622in}}{\pgfqpoint{3.696000in}{3.696000in}}%
\pgfusepath{clip}%
\pgfsetbuttcap%
\pgfsetroundjoin%
\definecolor{currentfill}{rgb}{0.121569,0.466667,0.705882}%
\pgfsetfillcolor{currentfill}%
\pgfsetfillopacity{0.415857}%
\pgfsetlinewidth{1.003750pt}%
\definecolor{currentstroke}{rgb}{0.121569,0.466667,0.705882}%
\pgfsetstrokecolor{currentstroke}%
\pgfsetstrokeopacity{0.415857}%
\pgfsetdash{}{0pt}%
\pgfpathmoveto{\pgfqpoint{2.525145in}{2.485477in}}%
\pgfpathcurveto{\pgfqpoint{2.533382in}{2.485477in}}{\pgfqpoint{2.541282in}{2.488750in}}{\pgfqpoint{2.547106in}{2.494574in}}%
\pgfpathcurveto{\pgfqpoint{2.552930in}{2.500398in}}{\pgfqpoint{2.556202in}{2.508298in}}{\pgfqpoint{2.556202in}{2.516534in}}%
\pgfpathcurveto{\pgfqpoint{2.556202in}{2.524770in}}{\pgfqpoint{2.552930in}{2.532670in}}{\pgfqpoint{2.547106in}{2.538494in}}%
\pgfpathcurveto{\pgfqpoint{2.541282in}{2.544318in}}{\pgfqpoint{2.533382in}{2.547590in}}{\pgfqpoint{2.525145in}{2.547590in}}%
\pgfpathcurveto{\pgfqpoint{2.516909in}{2.547590in}}{\pgfqpoint{2.509009in}{2.544318in}}{\pgfqpoint{2.503185in}{2.538494in}}%
\pgfpathcurveto{\pgfqpoint{2.497361in}{2.532670in}}{\pgfqpoint{2.494089in}{2.524770in}}{\pgfqpoint{2.494089in}{2.516534in}}%
\pgfpathcurveto{\pgfqpoint{2.494089in}{2.508298in}}{\pgfqpoint{2.497361in}{2.500398in}}{\pgfqpoint{2.503185in}{2.494574in}}%
\pgfpathcurveto{\pgfqpoint{2.509009in}{2.488750in}}{\pgfqpoint{2.516909in}{2.485477in}}{\pgfqpoint{2.525145in}{2.485477in}}%
\pgfpathclose%
\pgfusepath{stroke,fill}%
\end{pgfscope}%
\begin{pgfscope}%
\pgfpathrectangle{\pgfqpoint{0.100000in}{0.212622in}}{\pgfqpoint{3.696000in}{3.696000in}}%
\pgfusepath{clip}%
\pgfsetbuttcap%
\pgfsetroundjoin%
\definecolor{currentfill}{rgb}{0.121569,0.466667,0.705882}%
\pgfsetfillcolor{currentfill}%
\pgfsetfillopacity{0.418920}%
\pgfsetlinewidth{1.003750pt}%
\definecolor{currentstroke}{rgb}{0.121569,0.466667,0.705882}%
\pgfsetstrokecolor{currentstroke}%
\pgfsetstrokeopacity{0.418920}%
\pgfsetdash{}{0pt}%
\pgfpathmoveto{\pgfqpoint{2.429503in}{2.421846in}}%
\pgfpathcurveto{\pgfqpoint{2.437739in}{2.421846in}}{\pgfqpoint{2.445639in}{2.425119in}}{\pgfqpoint{2.451463in}{2.430943in}}%
\pgfpathcurveto{\pgfqpoint{2.457287in}{2.436766in}}{\pgfqpoint{2.460559in}{2.444666in}}{\pgfqpoint{2.460559in}{2.452903in}}%
\pgfpathcurveto{\pgfqpoint{2.460559in}{2.461139in}}{\pgfqpoint{2.457287in}{2.469039in}}{\pgfqpoint{2.451463in}{2.474863in}}%
\pgfpathcurveto{\pgfqpoint{2.445639in}{2.480687in}}{\pgfqpoint{2.437739in}{2.483959in}}{\pgfqpoint{2.429503in}{2.483959in}}%
\pgfpathcurveto{\pgfqpoint{2.421266in}{2.483959in}}{\pgfqpoint{2.413366in}{2.480687in}}{\pgfqpoint{2.407542in}{2.474863in}}%
\pgfpathcurveto{\pgfqpoint{2.401718in}{2.469039in}}{\pgfqpoint{2.398446in}{2.461139in}}{\pgfqpoint{2.398446in}{2.452903in}}%
\pgfpathcurveto{\pgfqpoint{2.398446in}{2.444666in}}{\pgfqpoint{2.401718in}{2.436766in}}{\pgfqpoint{2.407542in}{2.430943in}}%
\pgfpathcurveto{\pgfqpoint{2.413366in}{2.425119in}}{\pgfqpoint{2.421266in}{2.421846in}}{\pgfqpoint{2.429503in}{2.421846in}}%
\pgfpathclose%
\pgfusepath{stroke,fill}%
\end{pgfscope}%
\begin{pgfscope}%
\pgfpathrectangle{\pgfqpoint{0.100000in}{0.212622in}}{\pgfqpoint{3.696000in}{3.696000in}}%
\pgfusepath{clip}%
\pgfsetbuttcap%
\pgfsetroundjoin%
\definecolor{currentfill}{rgb}{0.121569,0.466667,0.705882}%
\pgfsetfillcolor{currentfill}%
\pgfsetfillopacity{0.419032}%
\pgfsetlinewidth{1.003750pt}%
\definecolor{currentstroke}{rgb}{0.121569,0.466667,0.705882}%
\pgfsetstrokecolor{currentstroke}%
\pgfsetstrokeopacity{0.419032}%
\pgfsetdash{}{0pt}%
\pgfpathmoveto{\pgfqpoint{2.422529in}{2.413561in}}%
\pgfpathcurveto{\pgfqpoint{2.430766in}{2.413561in}}{\pgfqpoint{2.438666in}{2.416833in}}{\pgfqpoint{2.444490in}{2.422657in}}%
\pgfpathcurveto{\pgfqpoint{2.450314in}{2.428481in}}{\pgfqpoint{2.453586in}{2.436381in}}{\pgfqpoint{2.453586in}{2.444617in}}%
\pgfpathcurveto{\pgfqpoint{2.453586in}{2.452853in}}{\pgfqpoint{2.450314in}{2.460754in}}{\pgfqpoint{2.444490in}{2.466577in}}%
\pgfpathcurveto{\pgfqpoint{2.438666in}{2.472401in}}{\pgfqpoint{2.430766in}{2.475674in}}{\pgfqpoint{2.422529in}{2.475674in}}%
\pgfpathcurveto{\pgfqpoint{2.414293in}{2.475674in}}{\pgfqpoint{2.406393in}{2.472401in}}{\pgfqpoint{2.400569in}{2.466577in}}%
\pgfpathcurveto{\pgfqpoint{2.394745in}{2.460754in}}{\pgfqpoint{2.391473in}{2.452853in}}{\pgfqpoint{2.391473in}{2.444617in}}%
\pgfpathcurveto{\pgfqpoint{2.391473in}{2.436381in}}{\pgfqpoint{2.394745in}{2.428481in}}{\pgfqpoint{2.400569in}{2.422657in}}%
\pgfpathcurveto{\pgfqpoint{2.406393in}{2.416833in}}{\pgfqpoint{2.414293in}{2.413561in}}{\pgfqpoint{2.422529in}{2.413561in}}%
\pgfpathclose%
\pgfusepath{stroke,fill}%
\end{pgfscope}%
\begin{pgfscope}%
\pgfpathrectangle{\pgfqpoint{0.100000in}{0.212622in}}{\pgfqpoint{3.696000in}{3.696000in}}%
\pgfusepath{clip}%
\pgfsetbuttcap%
\pgfsetroundjoin%
\definecolor{currentfill}{rgb}{0.121569,0.466667,0.705882}%
\pgfsetfillcolor{currentfill}%
\pgfsetfillopacity{0.420049}%
\pgfsetlinewidth{1.003750pt}%
\definecolor{currentstroke}{rgb}{0.121569,0.466667,0.705882}%
\pgfsetstrokecolor{currentstroke}%
\pgfsetstrokeopacity{0.420049}%
\pgfsetdash{}{0pt}%
\pgfpathmoveto{\pgfqpoint{2.500489in}{2.467720in}}%
\pgfpathcurveto{\pgfqpoint{2.508725in}{2.467720in}}{\pgfqpoint{2.516626in}{2.470992in}}{\pgfqpoint{2.522449in}{2.476816in}}%
\pgfpathcurveto{\pgfqpoint{2.528273in}{2.482640in}}{\pgfqpoint{2.531546in}{2.490540in}}{\pgfqpoint{2.531546in}{2.498776in}}%
\pgfpathcurveto{\pgfqpoint{2.531546in}{2.507013in}}{\pgfqpoint{2.528273in}{2.514913in}}{\pgfqpoint{2.522449in}{2.520737in}}%
\pgfpathcurveto{\pgfqpoint{2.516626in}{2.526560in}}{\pgfqpoint{2.508725in}{2.529833in}}{\pgfqpoint{2.500489in}{2.529833in}}%
\pgfpathcurveto{\pgfqpoint{2.492253in}{2.529833in}}{\pgfqpoint{2.484353in}{2.526560in}}{\pgfqpoint{2.478529in}{2.520737in}}%
\pgfpathcurveto{\pgfqpoint{2.472705in}{2.514913in}}{\pgfqpoint{2.469433in}{2.507013in}}{\pgfqpoint{2.469433in}{2.498776in}}%
\pgfpathcurveto{\pgfqpoint{2.469433in}{2.490540in}}{\pgfqpoint{2.472705in}{2.482640in}}{\pgfqpoint{2.478529in}{2.476816in}}%
\pgfpathcurveto{\pgfqpoint{2.484353in}{2.470992in}}{\pgfqpoint{2.492253in}{2.467720in}}{\pgfqpoint{2.500489in}{2.467720in}}%
\pgfpathclose%
\pgfusepath{stroke,fill}%
\end{pgfscope}%
\begin{pgfscope}%
\pgfpathrectangle{\pgfqpoint{0.100000in}{0.212622in}}{\pgfqpoint{3.696000in}{3.696000in}}%
\pgfusepath{clip}%
\pgfsetbuttcap%
\pgfsetroundjoin%
\definecolor{currentfill}{rgb}{0.121569,0.466667,0.705882}%
\pgfsetfillcolor{currentfill}%
\pgfsetfillopacity{0.420369}%
\pgfsetlinewidth{1.003750pt}%
\definecolor{currentstroke}{rgb}{0.121569,0.466667,0.705882}%
\pgfsetstrokecolor{currentstroke}%
\pgfsetstrokeopacity{0.420369}%
\pgfsetdash{}{0pt}%
\pgfpathmoveto{\pgfqpoint{2.413080in}{2.404158in}}%
\pgfpathcurveto{\pgfqpoint{2.421316in}{2.404158in}}{\pgfqpoint{2.429216in}{2.407430in}}{\pgfqpoint{2.435040in}{2.413254in}}%
\pgfpathcurveto{\pgfqpoint{2.440864in}{2.419078in}}{\pgfqpoint{2.444136in}{2.426978in}}{\pgfqpoint{2.444136in}{2.435215in}}%
\pgfpathcurveto{\pgfqpoint{2.444136in}{2.443451in}}{\pgfqpoint{2.440864in}{2.451351in}}{\pgfqpoint{2.435040in}{2.457175in}}%
\pgfpathcurveto{\pgfqpoint{2.429216in}{2.462999in}}{\pgfqpoint{2.421316in}{2.466271in}}{\pgfqpoint{2.413080in}{2.466271in}}%
\pgfpathcurveto{\pgfqpoint{2.404843in}{2.466271in}}{\pgfqpoint{2.396943in}{2.462999in}}{\pgfqpoint{2.391120in}{2.457175in}}%
\pgfpathcurveto{\pgfqpoint{2.385296in}{2.451351in}}{\pgfqpoint{2.382023in}{2.443451in}}{\pgfqpoint{2.382023in}{2.435215in}}%
\pgfpathcurveto{\pgfqpoint{2.382023in}{2.426978in}}{\pgfqpoint{2.385296in}{2.419078in}}{\pgfqpoint{2.391120in}{2.413254in}}%
\pgfpathcurveto{\pgfqpoint{2.396943in}{2.407430in}}{\pgfqpoint{2.404843in}{2.404158in}}{\pgfqpoint{2.413080in}{2.404158in}}%
\pgfpathclose%
\pgfusepath{stroke,fill}%
\end{pgfscope}%
\begin{pgfscope}%
\pgfpathrectangle{\pgfqpoint{0.100000in}{0.212622in}}{\pgfqpoint{3.696000in}{3.696000in}}%
\pgfusepath{clip}%
\pgfsetbuttcap%
\pgfsetroundjoin%
\definecolor{currentfill}{rgb}{0.121569,0.466667,0.705882}%
\pgfsetfillcolor{currentfill}%
\pgfsetfillopacity{0.422069}%
\pgfsetlinewidth{1.003750pt}%
\definecolor{currentstroke}{rgb}{0.121569,0.466667,0.705882}%
\pgfsetstrokecolor{currentstroke}%
\pgfsetstrokeopacity{0.422069}%
\pgfsetdash{}{0pt}%
\pgfpathmoveto{\pgfqpoint{1.547468in}{1.928650in}}%
\pgfpathcurveto{\pgfqpoint{1.555704in}{1.928650in}}{\pgfqpoint{1.563605in}{1.931922in}}{\pgfqpoint{1.569428in}{1.937746in}}%
\pgfpathcurveto{\pgfqpoint{1.575252in}{1.943570in}}{\pgfqpoint{1.578525in}{1.951470in}}{\pgfqpoint{1.578525in}{1.959707in}}%
\pgfpathcurveto{\pgfqpoint{1.578525in}{1.967943in}}{\pgfqpoint{1.575252in}{1.975843in}}{\pgfqpoint{1.569428in}{1.981667in}}%
\pgfpathcurveto{\pgfqpoint{1.563605in}{1.987491in}}{\pgfqpoint{1.555704in}{1.990763in}}{\pgfqpoint{1.547468in}{1.990763in}}%
\pgfpathcurveto{\pgfqpoint{1.539232in}{1.990763in}}{\pgfqpoint{1.531332in}{1.987491in}}{\pgfqpoint{1.525508in}{1.981667in}}%
\pgfpathcurveto{\pgfqpoint{1.519684in}{1.975843in}}{\pgfqpoint{1.516412in}{1.967943in}}{\pgfqpoint{1.516412in}{1.959707in}}%
\pgfpathcurveto{\pgfqpoint{1.516412in}{1.951470in}}{\pgfqpoint{1.519684in}{1.943570in}}{\pgfqpoint{1.525508in}{1.937746in}}%
\pgfpathcurveto{\pgfqpoint{1.531332in}{1.931922in}}{\pgfqpoint{1.539232in}{1.928650in}}{\pgfqpoint{1.547468in}{1.928650in}}%
\pgfpathclose%
\pgfusepath{stroke,fill}%
\end{pgfscope}%
\begin{pgfscope}%
\pgfpathrectangle{\pgfqpoint{0.100000in}{0.212622in}}{\pgfqpoint{3.696000in}{3.696000in}}%
\pgfusepath{clip}%
\pgfsetbuttcap%
\pgfsetroundjoin%
\definecolor{currentfill}{rgb}{0.121569,0.466667,0.705882}%
\pgfsetfillcolor{currentfill}%
\pgfsetfillopacity{0.423215}%
\pgfsetlinewidth{1.003750pt}%
\definecolor{currentstroke}{rgb}{0.121569,0.466667,0.705882}%
\pgfsetstrokecolor{currentstroke}%
\pgfsetstrokeopacity{0.423215}%
\pgfsetdash{}{0pt}%
\pgfpathmoveto{\pgfqpoint{2.519236in}{2.479172in}}%
\pgfpathcurveto{\pgfqpoint{2.527473in}{2.479172in}}{\pgfqpoint{2.535373in}{2.482445in}}{\pgfqpoint{2.541197in}{2.488269in}}%
\pgfpathcurveto{\pgfqpoint{2.547021in}{2.494093in}}{\pgfqpoint{2.550293in}{2.501993in}}{\pgfqpoint{2.550293in}{2.510229in}}%
\pgfpathcurveto{\pgfqpoint{2.550293in}{2.518465in}}{\pgfqpoint{2.547021in}{2.526365in}}{\pgfqpoint{2.541197in}{2.532189in}}%
\pgfpathcurveto{\pgfqpoint{2.535373in}{2.538013in}}{\pgfqpoint{2.527473in}{2.541285in}}{\pgfqpoint{2.519236in}{2.541285in}}%
\pgfpathcurveto{\pgfqpoint{2.511000in}{2.541285in}}{\pgfqpoint{2.503100in}{2.538013in}}{\pgfqpoint{2.497276in}{2.532189in}}%
\pgfpathcurveto{\pgfqpoint{2.491452in}{2.526365in}}{\pgfqpoint{2.488180in}{2.518465in}}{\pgfqpoint{2.488180in}{2.510229in}}%
\pgfpathcurveto{\pgfqpoint{2.488180in}{2.501993in}}{\pgfqpoint{2.491452in}{2.494093in}}{\pgfqpoint{2.497276in}{2.488269in}}%
\pgfpathcurveto{\pgfqpoint{2.503100in}{2.482445in}}{\pgfqpoint{2.511000in}{2.479172in}}{\pgfqpoint{2.519236in}{2.479172in}}%
\pgfpathclose%
\pgfusepath{stroke,fill}%
\end{pgfscope}%
\begin{pgfscope}%
\pgfpathrectangle{\pgfqpoint{0.100000in}{0.212622in}}{\pgfqpoint{3.696000in}{3.696000in}}%
\pgfusepath{clip}%
\pgfsetbuttcap%
\pgfsetroundjoin%
\definecolor{currentfill}{rgb}{0.121569,0.466667,0.705882}%
\pgfsetfillcolor{currentfill}%
\pgfsetfillopacity{0.423996}%
\pgfsetlinewidth{1.003750pt}%
\definecolor{currentstroke}{rgb}{0.121569,0.466667,0.705882}%
\pgfsetstrokecolor{currentstroke}%
\pgfsetstrokeopacity{0.423996}%
\pgfsetdash{}{0pt}%
\pgfpathmoveto{\pgfqpoint{1.547761in}{1.924090in}}%
\pgfpathcurveto{\pgfqpoint{1.555997in}{1.924090in}}{\pgfqpoint{1.563897in}{1.927362in}}{\pgfqpoint{1.569721in}{1.933186in}}%
\pgfpathcurveto{\pgfqpoint{1.575545in}{1.939010in}}{\pgfqpoint{1.578818in}{1.946910in}}{\pgfqpoint{1.578818in}{1.955146in}}%
\pgfpathcurveto{\pgfqpoint{1.578818in}{1.963383in}}{\pgfqpoint{1.575545in}{1.971283in}}{\pgfqpoint{1.569721in}{1.977106in}}%
\pgfpathcurveto{\pgfqpoint{1.563897in}{1.982930in}}{\pgfqpoint{1.555997in}{1.986203in}}{\pgfqpoint{1.547761in}{1.986203in}}%
\pgfpathcurveto{\pgfqpoint{1.539525in}{1.986203in}}{\pgfqpoint{1.531625in}{1.982930in}}{\pgfqpoint{1.525801in}{1.977106in}}%
\pgfpathcurveto{\pgfqpoint{1.519977in}{1.971283in}}{\pgfqpoint{1.516705in}{1.963383in}}{\pgfqpoint{1.516705in}{1.955146in}}%
\pgfpathcurveto{\pgfqpoint{1.516705in}{1.946910in}}{\pgfqpoint{1.519977in}{1.939010in}}{\pgfqpoint{1.525801in}{1.933186in}}%
\pgfpathcurveto{\pgfqpoint{1.531625in}{1.927362in}}{\pgfqpoint{1.539525in}{1.924090in}}{\pgfqpoint{1.547761in}{1.924090in}}%
\pgfpathclose%
\pgfusepath{stroke,fill}%
\end{pgfscope}%
\begin{pgfscope}%
\pgfpathrectangle{\pgfqpoint{0.100000in}{0.212622in}}{\pgfqpoint{3.696000in}{3.696000in}}%
\pgfusepath{clip}%
\pgfsetbuttcap%
\pgfsetroundjoin%
\definecolor{currentfill}{rgb}{0.121569,0.466667,0.705882}%
\pgfsetfillcolor{currentfill}%
\pgfsetfillopacity{0.426334}%
\pgfsetlinewidth{1.003750pt}%
\definecolor{currentstroke}{rgb}{0.121569,0.466667,0.705882}%
\pgfsetstrokecolor{currentstroke}%
\pgfsetstrokeopacity{0.426334}%
\pgfsetdash{}{0pt}%
\pgfpathmoveto{\pgfqpoint{1.544229in}{1.918796in}}%
\pgfpathcurveto{\pgfqpoint{1.552465in}{1.918796in}}{\pgfqpoint{1.560365in}{1.922069in}}{\pgfqpoint{1.566189in}{1.927893in}}%
\pgfpathcurveto{\pgfqpoint{1.572013in}{1.933716in}}{\pgfqpoint{1.575286in}{1.941616in}}{\pgfqpoint{1.575286in}{1.949853in}}%
\pgfpathcurveto{\pgfqpoint{1.575286in}{1.958089in}}{\pgfqpoint{1.572013in}{1.965989in}}{\pgfqpoint{1.566189in}{1.971813in}}%
\pgfpathcurveto{\pgfqpoint{1.560365in}{1.977637in}}{\pgfqpoint{1.552465in}{1.980909in}}{\pgfqpoint{1.544229in}{1.980909in}}%
\pgfpathcurveto{\pgfqpoint{1.535993in}{1.980909in}}{\pgfqpoint{1.528093in}{1.977637in}}{\pgfqpoint{1.522269in}{1.971813in}}%
\pgfpathcurveto{\pgfqpoint{1.516445in}{1.965989in}}{\pgfqpoint{1.513173in}{1.958089in}}{\pgfqpoint{1.513173in}{1.949853in}}%
\pgfpathcurveto{\pgfqpoint{1.513173in}{1.941616in}}{\pgfqpoint{1.516445in}{1.933716in}}{\pgfqpoint{1.522269in}{1.927893in}}%
\pgfpathcurveto{\pgfqpoint{1.528093in}{1.922069in}}{\pgfqpoint{1.535993in}{1.918796in}}{\pgfqpoint{1.544229in}{1.918796in}}%
\pgfpathclose%
\pgfusepath{stroke,fill}%
\end{pgfscope}%
\begin{pgfscope}%
\pgfpathrectangle{\pgfqpoint{0.100000in}{0.212622in}}{\pgfqpoint{3.696000in}{3.696000in}}%
\pgfusepath{clip}%
\pgfsetbuttcap%
\pgfsetroundjoin%
\definecolor{currentfill}{rgb}{0.121569,0.466667,0.705882}%
\pgfsetfillcolor{currentfill}%
\pgfsetfillopacity{0.426864}%
\pgfsetlinewidth{1.003750pt}%
\definecolor{currentstroke}{rgb}{0.121569,0.466667,0.705882}%
\pgfsetstrokecolor{currentstroke}%
\pgfsetstrokeopacity{0.426864}%
\pgfsetdash{}{0pt}%
\pgfpathmoveto{\pgfqpoint{2.385604in}{2.380958in}}%
\pgfpathcurveto{\pgfqpoint{2.393840in}{2.380958in}}{\pgfqpoint{2.401740in}{2.384230in}}{\pgfqpoint{2.407564in}{2.390054in}}%
\pgfpathcurveto{\pgfqpoint{2.413388in}{2.395878in}}{\pgfqpoint{2.416660in}{2.403778in}}{\pgfqpoint{2.416660in}{2.412014in}}%
\pgfpathcurveto{\pgfqpoint{2.416660in}{2.420251in}}{\pgfqpoint{2.413388in}{2.428151in}}{\pgfqpoint{2.407564in}{2.433975in}}%
\pgfpathcurveto{\pgfqpoint{2.401740in}{2.439799in}}{\pgfqpoint{2.393840in}{2.443071in}}{\pgfqpoint{2.385604in}{2.443071in}}%
\pgfpathcurveto{\pgfqpoint{2.377368in}{2.443071in}}{\pgfqpoint{2.369468in}{2.439799in}}{\pgfqpoint{2.363644in}{2.433975in}}%
\pgfpathcurveto{\pgfqpoint{2.357820in}{2.428151in}}{\pgfqpoint{2.354547in}{2.420251in}}{\pgfqpoint{2.354547in}{2.412014in}}%
\pgfpathcurveto{\pgfqpoint{2.354547in}{2.403778in}}{\pgfqpoint{2.357820in}{2.395878in}}{\pgfqpoint{2.363644in}{2.390054in}}%
\pgfpathcurveto{\pgfqpoint{2.369468in}{2.384230in}}{\pgfqpoint{2.377368in}{2.380958in}}{\pgfqpoint{2.385604in}{2.380958in}}%
\pgfpathclose%
\pgfusepath{stroke,fill}%
\end{pgfscope}%
\begin{pgfscope}%
\pgfpathrectangle{\pgfqpoint{0.100000in}{0.212622in}}{\pgfqpoint{3.696000in}{3.696000in}}%
\pgfusepath{clip}%
\pgfsetbuttcap%
\pgfsetroundjoin%
\definecolor{currentfill}{rgb}{0.121569,0.466667,0.705882}%
\pgfsetfillcolor{currentfill}%
\pgfsetfillopacity{0.427027}%
\pgfsetlinewidth{1.003750pt}%
\definecolor{currentstroke}{rgb}{0.121569,0.466667,0.705882}%
\pgfsetstrokecolor{currentstroke}%
\pgfsetstrokeopacity{0.427027}%
\pgfsetdash{}{0pt}%
\pgfpathmoveto{\pgfqpoint{1.544557in}{1.919104in}}%
\pgfpathcurveto{\pgfqpoint{1.552793in}{1.919104in}}{\pgfqpoint{1.560693in}{1.922376in}}{\pgfqpoint{1.566517in}{1.928200in}}%
\pgfpathcurveto{\pgfqpoint{1.572341in}{1.934024in}}{\pgfqpoint{1.575613in}{1.941924in}}{\pgfqpoint{1.575613in}{1.950161in}}%
\pgfpathcurveto{\pgfqpoint{1.575613in}{1.958397in}}{\pgfqpoint{1.572341in}{1.966297in}}{\pgfqpoint{1.566517in}{1.972121in}}%
\pgfpathcurveto{\pgfqpoint{1.560693in}{1.977945in}}{\pgfqpoint{1.552793in}{1.981217in}}{\pgfqpoint{1.544557in}{1.981217in}}%
\pgfpathcurveto{\pgfqpoint{1.536320in}{1.981217in}}{\pgfqpoint{1.528420in}{1.977945in}}{\pgfqpoint{1.522596in}{1.972121in}}%
\pgfpathcurveto{\pgfqpoint{1.516773in}{1.966297in}}{\pgfqpoint{1.513500in}{1.958397in}}{\pgfqpoint{1.513500in}{1.950161in}}%
\pgfpathcurveto{\pgfqpoint{1.513500in}{1.941924in}}{\pgfqpoint{1.516773in}{1.934024in}}{\pgfqpoint{1.522596in}{1.928200in}}%
\pgfpathcurveto{\pgfqpoint{1.528420in}{1.922376in}}{\pgfqpoint{1.536320in}{1.919104in}}{\pgfqpoint{1.544557in}{1.919104in}}%
\pgfpathclose%
\pgfusepath{stroke,fill}%
\end{pgfscope}%
\begin{pgfscope}%
\pgfpathrectangle{\pgfqpoint{0.100000in}{0.212622in}}{\pgfqpoint{3.696000in}{3.696000in}}%
\pgfusepath{clip}%
\pgfsetbuttcap%
\pgfsetroundjoin%
\definecolor{currentfill}{rgb}{0.121569,0.466667,0.705882}%
\pgfsetfillcolor{currentfill}%
\pgfsetfillopacity{0.427307}%
\pgfsetlinewidth{1.003750pt}%
\definecolor{currentstroke}{rgb}{0.121569,0.466667,0.705882}%
\pgfsetstrokecolor{currentstroke}%
\pgfsetstrokeopacity{0.427307}%
\pgfsetdash{}{0pt}%
\pgfpathmoveto{\pgfqpoint{1.547328in}{1.921321in}}%
\pgfpathcurveto{\pgfqpoint{1.555564in}{1.921321in}}{\pgfqpoint{1.563464in}{1.924593in}}{\pgfqpoint{1.569288in}{1.930417in}}%
\pgfpathcurveto{\pgfqpoint{1.575112in}{1.936241in}}{\pgfqpoint{1.578384in}{1.944141in}}{\pgfqpoint{1.578384in}{1.952378in}}%
\pgfpathcurveto{\pgfqpoint{1.578384in}{1.960614in}}{\pgfqpoint{1.575112in}{1.968514in}}{\pgfqpoint{1.569288in}{1.974338in}}%
\pgfpathcurveto{\pgfqpoint{1.563464in}{1.980162in}}{\pgfqpoint{1.555564in}{1.983434in}}{\pgfqpoint{1.547328in}{1.983434in}}%
\pgfpathcurveto{\pgfqpoint{1.539092in}{1.983434in}}{\pgfqpoint{1.531192in}{1.980162in}}{\pgfqpoint{1.525368in}{1.974338in}}%
\pgfpathcurveto{\pgfqpoint{1.519544in}{1.968514in}}{\pgfqpoint{1.516271in}{1.960614in}}{\pgfqpoint{1.516271in}{1.952378in}}%
\pgfpathcurveto{\pgfqpoint{1.516271in}{1.944141in}}{\pgfqpoint{1.519544in}{1.936241in}}{\pgfqpoint{1.525368in}{1.930417in}}%
\pgfpathcurveto{\pgfqpoint{1.531192in}{1.924593in}}{\pgfqpoint{1.539092in}{1.921321in}}{\pgfqpoint{1.547328in}{1.921321in}}%
\pgfpathclose%
\pgfusepath{stroke,fill}%
\end{pgfscope}%
\begin{pgfscope}%
\pgfpathrectangle{\pgfqpoint{0.100000in}{0.212622in}}{\pgfqpoint{3.696000in}{3.696000in}}%
\pgfusepath{clip}%
\pgfsetbuttcap%
\pgfsetroundjoin%
\definecolor{currentfill}{rgb}{0.121569,0.466667,0.705882}%
\pgfsetfillcolor{currentfill}%
\pgfsetfillopacity{0.427444}%
\pgfsetlinewidth{1.003750pt}%
\definecolor{currentstroke}{rgb}{0.121569,0.466667,0.705882}%
\pgfsetstrokecolor{currentstroke}%
\pgfsetstrokeopacity{0.427444}%
\pgfsetdash{}{0pt}%
\pgfpathmoveto{\pgfqpoint{2.546070in}{2.495960in}}%
\pgfpathcurveto{\pgfqpoint{2.554307in}{2.495960in}}{\pgfqpoint{2.562207in}{2.499233in}}{\pgfqpoint{2.568031in}{2.505057in}}%
\pgfpathcurveto{\pgfqpoint{2.573855in}{2.510881in}}{\pgfqpoint{2.577127in}{2.518781in}}{\pgfqpoint{2.577127in}{2.527017in}}%
\pgfpathcurveto{\pgfqpoint{2.577127in}{2.535253in}}{\pgfqpoint{2.573855in}{2.543153in}}{\pgfqpoint{2.568031in}{2.548977in}}%
\pgfpathcurveto{\pgfqpoint{2.562207in}{2.554801in}}{\pgfqpoint{2.554307in}{2.558073in}}{\pgfqpoint{2.546070in}{2.558073in}}%
\pgfpathcurveto{\pgfqpoint{2.537834in}{2.558073in}}{\pgfqpoint{2.529934in}{2.554801in}}{\pgfqpoint{2.524110in}{2.548977in}}%
\pgfpathcurveto{\pgfqpoint{2.518286in}{2.543153in}}{\pgfqpoint{2.515014in}{2.535253in}}{\pgfqpoint{2.515014in}{2.527017in}}%
\pgfpathcurveto{\pgfqpoint{2.515014in}{2.518781in}}{\pgfqpoint{2.518286in}{2.510881in}}{\pgfqpoint{2.524110in}{2.505057in}}%
\pgfpathcurveto{\pgfqpoint{2.529934in}{2.499233in}}{\pgfqpoint{2.537834in}{2.495960in}}{\pgfqpoint{2.546070in}{2.495960in}}%
\pgfpathclose%
\pgfusepath{stroke,fill}%
\end{pgfscope}%
\begin{pgfscope}%
\pgfpathrectangle{\pgfqpoint{0.100000in}{0.212622in}}{\pgfqpoint{3.696000in}{3.696000in}}%
\pgfusepath{clip}%
\pgfsetbuttcap%
\pgfsetroundjoin%
\definecolor{currentfill}{rgb}{0.121569,0.466667,0.705882}%
\pgfsetfillcolor{currentfill}%
\pgfsetfillopacity{0.427647}%
\pgfsetlinewidth{1.003750pt}%
\definecolor{currentstroke}{rgb}{0.121569,0.466667,0.705882}%
\pgfsetstrokecolor{currentstroke}%
\pgfsetstrokeopacity{0.427647}%
\pgfsetdash{}{0pt}%
\pgfpathmoveto{\pgfqpoint{2.520498in}{2.474885in}}%
\pgfpathcurveto{\pgfqpoint{2.528735in}{2.474885in}}{\pgfqpoint{2.536635in}{2.478158in}}{\pgfqpoint{2.542459in}{2.483981in}}%
\pgfpathcurveto{\pgfqpoint{2.548283in}{2.489805in}}{\pgfqpoint{2.551555in}{2.497705in}}{\pgfqpoint{2.551555in}{2.505942in}}%
\pgfpathcurveto{\pgfqpoint{2.551555in}{2.514178in}}{\pgfqpoint{2.548283in}{2.522078in}}{\pgfqpoint{2.542459in}{2.527902in}}%
\pgfpathcurveto{\pgfqpoint{2.536635in}{2.533726in}}{\pgfqpoint{2.528735in}{2.536998in}}{\pgfqpoint{2.520498in}{2.536998in}}%
\pgfpathcurveto{\pgfqpoint{2.512262in}{2.536998in}}{\pgfqpoint{2.504362in}{2.533726in}}{\pgfqpoint{2.498538in}{2.527902in}}%
\pgfpathcurveto{\pgfqpoint{2.492714in}{2.522078in}}{\pgfqpoint{2.489442in}{2.514178in}}{\pgfqpoint{2.489442in}{2.505942in}}%
\pgfpathcurveto{\pgfqpoint{2.489442in}{2.497705in}}{\pgfqpoint{2.492714in}{2.489805in}}{\pgfqpoint{2.498538in}{2.483981in}}%
\pgfpathcurveto{\pgfqpoint{2.504362in}{2.478158in}}{\pgfqpoint{2.512262in}{2.474885in}}{\pgfqpoint{2.520498in}{2.474885in}}%
\pgfpathclose%
\pgfusepath{stroke,fill}%
\end{pgfscope}%
\begin{pgfscope}%
\pgfpathrectangle{\pgfqpoint{0.100000in}{0.212622in}}{\pgfqpoint{3.696000in}{3.696000in}}%
\pgfusepath{clip}%
\pgfsetbuttcap%
\pgfsetroundjoin%
\definecolor{currentfill}{rgb}{0.121569,0.466667,0.705882}%
\pgfsetfillcolor{currentfill}%
\pgfsetfillopacity{0.428520}%
\pgfsetlinewidth{1.003750pt}%
\definecolor{currentstroke}{rgb}{0.121569,0.466667,0.705882}%
\pgfsetstrokecolor{currentstroke}%
\pgfsetstrokeopacity{0.428520}%
\pgfsetdash{}{0pt}%
\pgfpathmoveto{\pgfqpoint{2.534524in}{2.487141in}}%
\pgfpathcurveto{\pgfqpoint{2.542760in}{2.487141in}}{\pgfqpoint{2.550660in}{2.490413in}}{\pgfqpoint{2.556484in}{2.496237in}}%
\pgfpathcurveto{\pgfqpoint{2.562308in}{2.502061in}}{\pgfqpoint{2.565581in}{2.509961in}}{\pgfqpoint{2.565581in}{2.518197in}}%
\pgfpathcurveto{\pgfqpoint{2.565581in}{2.526434in}}{\pgfqpoint{2.562308in}{2.534334in}}{\pgfqpoint{2.556484in}{2.540158in}}%
\pgfpathcurveto{\pgfqpoint{2.550660in}{2.545982in}}{\pgfqpoint{2.542760in}{2.549254in}}{\pgfqpoint{2.534524in}{2.549254in}}%
\pgfpathcurveto{\pgfqpoint{2.526288in}{2.549254in}}{\pgfqpoint{2.518388in}{2.545982in}}{\pgfqpoint{2.512564in}{2.540158in}}%
\pgfpathcurveto{\pgfqpoint{2.506740in}{2.534334in}}{\pgfqpoint{2.503468in}{2.526434in}}{\pgfqpoint{2.503468in}{2.518197in}}%
\pgfpathcurveto{\pgfqpoint{2.503468in}{2.509961in}}{\pgfqpoint{2.506740in}{2.502061in}}{\pgfqpoint{2.512564in}{2.496237in}}%
\pgfpathcurveto{\pgfqpoint{2.518388in}{2.490413in}}{\pgfqpoint{2.526288in}{2.487141in}}{\pgfqpoint{2.534524in}{2.487141in}}%
\pgfpathclose%
\pgfusepath{stroke,fill}%
\end{pgfscope}%
\begin{pgfscope}%
\pgfpathrectangle{\pgfqpoint{0.100000in}{0.212622in}}{\pgfqpoint{3.696000in}{3.696000in}}%
\pgfusepath{clip}%
\pgfsetbuttcap%
\pgfsetroundjoin%
\definecolor{currentfill}{rgb}{0.121569,0.466667,0.705882}%
\pgfsetfillcolor{currentfill}%
\pgfsetfillopacity{0.430445}%
\pgfsetlinewidth{1.003750pt}%
\definecolor{currentstroke}{rgb}{0.121569,0.466667,0.705882}%
\pgfsetstrokecolor{currentstroke}%
\pgfsetstrokeopacity{0.430445}%
\pgfsetdash{}{0pt}%
\pgfpathmoveto{\pgfqpoint{1.541285in}{1.913754in}}%
\pgfpathcurveto{\pgfqpoint{1.549521in}{1.913754in}}{\pgfqpoint{1.557421in}{1.917026in}}{\pgfqpoint{1.563245in}{1.922850in}}%
\pgfpathcurveto{\pgfqpoint{1.569069in}{1.928674in}}{\pgfqpoint{1.572342in}{1.936574in}}{\pgfqpoint{1.572342in}{1.944810in}}%
\pgfpathcurveto{\pgfqpoint{1.572342in}{1.953046in}}{\pgfqpoint{1.569069in}{1.960947in}}{\pgfqpoint{1.563245in}{1.966770in}}%
\pgfpathcurveto{\pgfqpoint{1.557421in}{1.972594in}}{\pgfqpoint{1.549521in}{1.975867in}}{\pgfqpoint{1.541285in}{1.975867in}}%
\pgfpathcurveto{\pgfqpoint{1.533049in}{1.975867in}}{\pgfqpoint{1.525149in}{1.972594in}}{\pgfqpoint{1.519325in}{1.966770in}}%
\pgfpathcurveto{\pgfqpoint{1.513501in}{1.960947in}}{\pgfqpoint{1.510229in}{1.953046in}}{\pgfqpoint{1.510229in}{1.944810in}}%
\pgfpathcurveto{\pgfqpoint{1.510229in}{1.936574in}}{\pgfqpoint{1.513501in}{1.928674in}}{\pgfqpoint{1.519325in}{1.922850in}}%
\pgfpathcurveto{\pgfqpoint{1.525149in}{1.917026in}}{\pgfqpoint{1.533049in}{1.913754in}}{\pgfqpoint{1.541285in}{1.913754in}}%
\pgfpathclose%
\pgfusepath{stroke,fill}%
\end{pgfscope}%
\begin{pgfscope}%
\pgfpathrectangle{\pgfqpoint{0.100000in}{0.212622in}}{\pgfqpoint{3.696000in}{3.696000in}}%
\pgfusepath{clip}%
\pgfsetbuttcap%
\pgfsetroundjoin%
\definecolor{currentfill}{rgb}{0.121569,0.466667,0.705882}%
\pgfsetfillcolor{currentfill}%
\pgfsetfillopacity{0.431792}%
\pgfsetlinewidth{1.003750pt}%
\definecolor{currentstroke}{rgb}{0.121569,0.466667,0.705882}%
\pgfsetstrokecolor{currentstroke}%
\pgfsetstrokeopacity{0.431792}%
\pgfsetdash{}{0pt}%
\pgfpathmoveto{\pgfqpoint{1.550220in}{1.920388in}}%
\pgfpathcurveto{\pgfqpoint{1.558456in}{1.920388in}}{\pgfqpoint{1.566356in}{1.923660in}}{\pgfqpoint{1.572180in}{1.929484in}}%
\pgfpathcurveto{\pgfqpoint{1.578004in}{1.935308in}}{\pgfqpoint{1.581276in}{1.943208in}}{\pgfqpoint{1.581276in}{1.951445in}}%
\pgfpathcurveto{\pgfqpoint{1.581276in}{1.959681in}}{\pgfqpoint{1.578004in}{1.967581in}}{\pgfqpoint{1.572180in}{1.973405in}}%
\pgfpathcurveto{\pgfqpoint{1.566356in}{1.979229in}}{\pgfqpoint{1.558456in}{1.982501in}}{\pgfqpoint{1.550220in}{1.982501in}}%
\pgfpathcurveto{\pgfqpoint{1.541984in}{1.982501in}}{\pgfqpoint{1.534084in}{1.979229in}}{\pgfqpoint{1.528260in}{1.973405in}}%
\pgfpathcurveto{\pgfqpoint{1.522436in}{1.967581in}}{\pgfqpoint{1.519163in}{1.959681in}}{\pgfqpoint{1.519163in}{1.951445in}}%
\pgfpathcurveto{\pgfqpoint{1.519163in}{1.943208in}}{\pgfqpoint{1.522436in}{1.935308in}}{\pgfqpoint{1.528260in}{1.929484in}}%
\pgfpathcurveto{\pgfqpoint{1.534084in}{1.923660in}}{\pgfqpoint{1.541984in}{1.920388in}}{\pgfqpoint{1.550220in}{1.920388in}}%
\pgfpathclose%
\pgfusepath{stroke,fill}%
\end{pgfscope}%
\begin{pgfscope}%
\pgfpathrectangle{\pgfqpoint{0.100000in}{0.212622in}}{\pgfqpoint{3.696000in}{3.696000in}}%
\pgfusepath{clip}%
\pgfsetbuttcap%
\pgfsetroundjoin%
\definecolor{currentfill}{rgb}{0.121569,0.466667,0.705882}%
\pgfsetfillcolor{currentfill}%
\pgfsetfillopacity{0.431888}%
\pgfsetlinewidth{1.003750pt}%
\definecolor{currentstroke}{rgb}{0.121569,0.466667,0.705882}%
\pgfsetstrokecolor{currentstroke}%
\pgfsetstrokeopacity{0.431888}%
\pgfsetdash{}{0pt}%
\pgfpathmoveto{\pgfqpoint{1.541636in}{1.912655in}}%
\pgfpathcurveto{\pgfqpoint{1.549872in}{1.912655in}}{\pgfqpoint{1.557772in}{1.915928in}}{\pgfqpoint{1.563596in}{1.921752in}}%
\pgfpathcurveto{\pgfqpoint{1.569420in}{1.927576in}}{\pgfqpoint{1.572692in}{1.935476in}}{\pgfqpoint{1.572692in}{1.943712in}}%
\pgfpathcurveto{\pgfqpoint{1.572692in}{1.951948in}}{\pgfqpoint{1.569420in}{1.959848in}}{\pgfqpoint{1.563596in}{1.965672in}}%
\pgfpathcurveto{\pgfqpoint{1.557772in}{1.971496in}}{\pgfqpoint{1.549872in}{1.974768in}}{\pgfqpoint{1.541636in}{1.974768in}}%
\pgfpathcurveto{\pgfqpoint{1.533400in}{1.974768in}}{\pgfqpoint{1.525500in}{1.971496in}}{\pgfqpoint{1.519676in}{1.965672in}}%
\pgfpathcurveto{\pgfqpoint{1.513852in}{1.959848in}}{\pgfqpoint{1.510579in}{1.951948in}}{\pgfqpoint{1.510579in}{1.943712in}}%
\pgfpathcurveto{\pgfqpoint{1.510579in}{1.935476in}}{\pgfqpoint{1.513852in}{1.927576in}}{\pgfqpoint{1.519676in}{1.921752in}}%
\pgfpathcurveto{\pgfqpoint{1.525500in}{1.915928in}}{\pgfqpoint{1.533400in}{1.912655in}}{\pgfqpoint{1.541636in}{1.912655in}}%
\pgfpathclose%
\pgfusepath{stroke,fill}%
\end{pgfscope}%
\begin{pgfscope}%
\pgfpathrectangle{\pgfqpoint{0.100000in}{0.212622in}}{\pgfqpoint{3.696000in}{3.696000in}}%
\pgfusepath{clip}%
\pgfsetbuttcap%
\pgfsetroundjoin%
\definecolor{currentfill}{rgb}{0.121569,0.466667,0.705882}%
\pgfsetfillcolor{currentfill}%
\pgfsetfillopacity{0.432312}%
\pgfsetlinewidth{1.003750pt}%
\definecolor{currentstroke}{rgb}{0.121569,0.466667,0.705882}%
\pgfsetstrokecolor{currentstroke}%
\pgfsetstrokeopacity{0.432312}%
\pgfsetdash{}{0pt}%
\pgfpathmoveto{\pgfqpoint{1.546644in}{1.916947in}}%
\pgfpathcurveto{\pgfqpoint{1.554881in}{1.916947in}}{\pgfqpoint{1.562781in}{1.920219in}}{\pgfqpoint{1.568605in}{1.926043in}}%
\pgfpathcurveto{\pgfqpoint{1.574429in}{1.931867in}}{\pgfqpoint{1.577701in}{1.939767in}}{\pgfqpoint{1.577701in}{1.948003in}}%
\pgfpathcurveto{\pgfqpoint{1.577701in}{1.956239in}}{\pgfqpoint{1.574429in}{1.964139in}}{\pgfqpoint{1.568605in}{1.969963in}}%
\pgfpathcurveto{\pgfqpoint{1.562781in}{1.975787in}}{\pgfqpoint{1.554881in}{1.979060in}}{\pgfqpoint{1.546644in}{1.979060in}}%
\pgfpathcurveto{\pgfqpoint{1.538408in}{1.979060in}}{\pgfqpoint{1.530508in}{1.975787in}}{\pgfqpoint{1.524684in}{1.969963in}}%
\pgfpathcurveto{\pgfqpoint{1.518860in}{1.964139in}}{\pgfqpoint{1.515588in}{1.956239in}}{\pgfqpoint{1.515588in}{1.948003in}}%
\pgfpathcurveto{\pgfqpoint{1.515588in}{1.939767in}}{\pgfqpoint{1.518860in}{1.931867in}}{\pgfqpoint{1.524684in}{1.926043in}}%
\pgfpathcurveto{\pgfqpoint{1.530508in}{1.920219in}}{\pgfqpoint{1.538408in}{1.916947in}}{\pgfqpoint{1.546644in}{1.916947in}}%
\pgfpathclose%
\pgfusepath{stroke,fill}%
\end{pgfscope}%
\begin{pgfscope}%
\pgfpathrectangle{\pgfqpoint{0.100000in}{0.212622in}}{\pgfqpoint{3.696000in}{3.696000in}}%
\pgfusepath{clip}%
\pgfsetbuttcap%
\pgfsetroundjoin%
\definecolor{currentfill}{rgb}{0.121569,0.466667,0.705882}%
\pgfsetfillcolor{currentfill}%
\pgfsetfillopacity{0.432486}%
\pgfsetlinewidth{1.003750pt}%
\definecolor{currentstroke}{rgb}{0.121569,0.466667,0.705882}%
\pgfsetstrokecolor{currentstroke}%
\pgfsetstrokeopacity{0.432486}%
\pgfsetdash{}{0pt}%
\pgfpathmoveto{\pgfqpoint{1.542488in}{1.912803in}}%
\pgfpathcurveto{\pgfqpoint{1.550724in}{1.912803in}}{\pgfqpoint{1.558624in}{1.916075in}}{\pgfqpoint{1.564448in}{1.921899in}}%
\pgfpathcurveto{\pgfqpoint{1.570272in}{1.927723in}}{\pgfqpoint{1.573544in}{1.935623in}}{\pgfqpoint{1.573544in}{1.943859in}}%
\pgfpathcurveto{\pgfqpoint{1.573544in}{1.952096in}}{\pgfqpoint{1.570272in}{1.959996in}}{\pgfqpoint{1.564448in}{1.965820in}}%
\pgfpathcurveto{\pgfqpoint{1.558624in}{1.971644in}}{\pgfqpoint{1.550724in}{1.974916in}}{\pgfqpoint{1.542488in}{1.974916in}}%
\pgfpathcurveto{\pgfqpoint{1.534252in}{1.974916in}}{\pgfqpoint{1.526352in}{1.971644in}}{\pgfqpoint{1.520528in}{1.965820in}}%
\pgfpathcurveto{\pgfqpoint{1.514704in}{1.959996in}}{\pgfqpoint{1.511431in}{1.952096in}}{\pgfqpoint{1.511431in}{1.943859in}}%
\pgfpathcurveto{\pgfqpoint{1.511431in}{1.935623in}}{\pgfqpoint{1.514704in}{1.927723in}}{\pgfqpoint{1.520528in}{1.921899in}}%
\pgfpathcurveto{\pgfqpoint{1.526352in}{1.916075in}}{\pgfqpoint{1.534252in}{1.912803in}}{\pgfqpoint{1.542488in}{1.912803in}}%
\pgfpathclose%
\pgfusepath{stroke,fill}%
\end{pgfscope}%
\begin{pgfscope}%
\pgfpathrectangle{\pgfqpoint{0.100000in}{0.212622in}}{\pgfqpoint{3.696000in}{3.696000in}}%
\pgfusepath{clip}%
\pgfsetbuttcap%
\pgfsetroundjoin%
\definecolor{currentfill}{rgb}{0.121569,0.466667,0.705882}%
\pgfsetfillcolor{currentfill}%
\pgfsetfillopacity{0.433742}%
\pgfsetlinewidth{1.003750pt}%
\definecolor{currentstroke}{rgb}{0.121569,0.466667,0.705882}%
\pgfsetstrokecolor{currentstroke}%
\pgfsetstrokeopacity{0.433742}%
\pgfsetdash{}{0pt}%
\pgfpathmoveto{\pgfqpoint{1.549246in}{1.917150in}}%
\pgfpathcurveto{\pgfqpoint{1.557482in}{1.917150in}}{\pgfqpoint{1.565382in}{1.920422in}}{\pgfqpoint{1.571206in}{1.926246in}}%
\pgfpathcurveto{\pgfqpoint{1.577030in}{1.932070in}}{\pgfqpoint{1.580303in}{1.939970in}}{\pgfqpoint{1.580303in}{1.948206in}}%
\pgfpathcurveto{\pgfqpoint{1.580303in}{1.956442in}}{\pgfqpoint{1.577030in}{1.964342in}}{\pgfqpoint{1.571206in}{1.970166in}}%
\pgfpathcurveto{\pgfqpoint{1.565382in}{1.975990in}}{\pgfqpoint{1.557482in}{1.979263in}}{\pgfqpoint{1.549246in}{1.979263in}}%
\pgfpathcurveto{\pgfqpoint{1.541010in}{1.979263in}}{\pgfqpoint{1.533110in}{1.975990in}}{\pgfqpoint{1.527286in}{1.970166in}}%
\pgfpathcurveto{\pgfqpoint{1.521462in}{1.964342in}}{\pgfqpoint{1.518190in}{1.956442in}}{\pgfqpoint{1.518190in}{1.948206in}}%
\pgfpathcurveto{\pgfqpoint{1.518190in}{1.939970in}}{\pgfqpoint{1.521462in}{1.932070in}}{\pgfqpoint{1.527286in}{1.926246in}}%
\pgfpathcurveto{\pgfqpoint{1.533110in}{1.920422in}}{\pgfqpoint{1.541010in}{1.917150in}}{\pgfqpoint{1.549246in}{1.917150in}}%
\pgfpathclose%
\pgfusepath{stroke,fill}%
\end{pgfscope}%
\begin{pgfscope}%
\pgfpathrectangle{\pgfqpoint{0.100000in}{0.212622in}}{\pgfqpoint{3.696000in}{3.696000in}}%
\pgfusepath{clip}%
\pgfsetbuttcap%
\pgfsetroundjoin%
\definecolor{currentfill}{rgb}{0.121569,0.466667,0.705882}%
\pgfsetfillcolor{currentfill}%
\pgfsetfillopacity{0.433802}%
\pgfsetlinewidth{1.003750pt}%
\definecolor{currentstroke}{rgb}{0.121569,0.466667,0.705882}%
\pgfsetstrokecolor{currentstroke}%
\pgfsetstrokeopacity{0.433802}%
\pgfsetdash{}{0pt}%
\pgfpathmoveto{\pgfqpoint{1.546221in}{1.915804in}}%
\pgfpathcurveto{\pgfqpoint{1.554457in}{1.915804in}}{\pgfqpoint{1.562358in}{1.919076in}}{\pgfqpoint{1.568181in}{1.924900in}}%
\pgfpathcurveto{\pgfqpoint{1.574005in}{1.930724in}}{\pgfqpoint{1.577278in}{1.938624in}}{\pgfqpoint{1.577278in}{1.946860in}}%
\pgfpathcurveto{\pgfqpoint{1.577278in}{1.955097in}}{\pgfqpoint{1.574005in}{1.962997in}}{\pgfqpoint{1.568181in}{1.968821in}}%
\pgfpathcurveto{\pgfqpoint{1.562358in}{1.974645in}}{\pgfqpoint{1.554457in}{1.977917in}}{\pgfqpoint{1.546221in}{1.977917in}}%
\pgfpathcurveto{\pgfqpoint{1.537985in}{1.977917in}}{\pgfqpoint{1.530085in}{1.974645in}}{\pgfqpoint{1.524261in}{1.968821in}}%
\pgfpathcurveto{\pgfqpoint{1.518437in}{1.962997in}}{\pgfqpoint{1.515165in}{1.955097in}}{\pgfqpoint{1.515165in}{1.946860in}}%
\pgfpathcurveto{\pgfqpoint{1.515165in}{1.938624in}}{\pgfqpoint{1.518437in}{1.930724in}}{\pgfqpoint{1.524261in}{1.924900in}}%
\pgfpathcurveto{\pgfqpoint{1.530085in}{1.919076in}}{\pgfqpoint{1.537985in}{1.915804in}}{\pgfqpoint{1.546221in}{1.915804in}}%
\pgfpathclose%
\pgfusepath{stroke,fill}%
\end{pgfscope}%
\begin{pgfscope}%
\pgfpathrectangle{\pgfqpoint{0.100000in}{0.212622in}}{\pgfqpoint{3.696000in}{3.696000in}}%
\pgfusepath{clip}%
\pgfsetbuttcap%
\pgfsetroundjoin%
\definecolor{currentfill}{rgb}{0.121569,0.466667,0.705882}%
\pgfsetfillcolor{currentfill}%
\pgfsetfillopacity{0.435797}%
\pgfsetlinewidth{1.003750pt}%
\definecolor{currentstroke}{rgb}{0.121569,0.466667,0.705882}%
\pgfsetstrokecolor{currentstroke}%
\pgfsetstrokeopacity{0.435797}%
\pgfsetdash{}{0pt}%
\pgfpathmoveto{\pgfqpoint{1.710263in}{2.009521in}}%
\pgfpathcurveto{\pgfqpoint{1.718499in}{2.009521in}}{\pgfqpoint{1.726399in}{2.012793in}}{\pgfqpoint{1.732223in}{2.018617in}}%
\pgfpathcurveto{\pgfqpoint{1.738047in}{2.024441in}}{\pgfqpoint{1.741320in}{2.032341in}}{\pgfqpoint{1.741320in}{2.040577in}}%
\pgfpathcurveto{\pgfqpoint{1.741320in}{2.048813in}}{\pgfqpoint{1.738047in}{2.056713in}}{\pgfqpoint{1.732223in}{2.062537in}}%
\pgfpathcurveto{\pgfqpoint{1.726399in}{2.068361in}}{\pgfqpoint{1.718499in}{2.071634in}}{\pgfqpoint{1.710263in}{2.071634in}}%
\pgfpathcurveto{\pgfqpoint{1.702027in}{2.071634in}}{\pgfqpoint{1.694127in}{2.068361in}}{\pgfqpoint{1.688303in}{2.062537in}}%
\pgfpathcurveto{\pgfqpoint{1.682479in}{2.056713in}}{\pgfqpoint{1.679207in}{2.048813in}}{\pgfqpoint{1.679207in}{2.040577in}}%
\pgfpathcurveto{\pgfqpoint{1.679207in}{2.032341in}}{\pgfqpoint{1.682479in}{2.024441in}}{\pgfqpoint{1.688303in}{2.018617in}}%
\pgfpathcurveto{\pgfqpoint{1.694127in}{2.012793in}}{\pgfqpoint{1.702027in}{2.009521in}}{\pgfqpoint{1.710263in}{2.009521in}}%
\pgfpathclose%
\pgfusepath{stroke,fill}%
\end{pgfscope}%
\begin{pgfscope}%
\pgfpathrectangle{\pgfqpoint{0.100000in}{0.212622in}}{\pgfqpoint{3.696000in}{3.696000in}}%
\pgfusepath{clip}%
\pgfsetbuttcap%
\pgfsetroundjoin%
\definecolor{currentfill}{rgb}{0.121569,0.466667,0.705882}%
\pgfsetfillcolor{currentfill}%
\pgfsetfillopacity{0.436541}%
\pgfsetlinewidth{1.003750pt}%
\definecolor{currentstroke}{rgb}{0.121569,0.466667,0.705882}%
\pgfsetstrokecolor{currentstroke}%
\pgfsetstrokeopacity{0.436541}%
\pgfsetdash{}{0pt}%
\pgfpathmoveto{\pgfqpoint{1.673332in}{1.987245in}}%
\pgfpathcurveto{\pgfqpoint{1.681569in}{1.987245in}}{\pgfqpoint{1.689469in}{1.990517in}}{\pgfqpoint{1.695293in}{1.996341in}}%
\pgfpathcurveto{\pgfqpoint{1.701116in}{2.002165in}}{\pgfqpoint{1.704389in}{2.010065in}}{\pgfqpoint{1.704389in}{2.018302in}}%
\pgfpathcurveto{\pgfqpoint{1.704389in}{2.026538in}}{\pgfqpoint{1.701116in}{2.034438in}}{\pgfqpoint{1.695293in}{2.040262in}}%
\pgfpathcurveto{\pgfqpoint{1.689469in}{2.046086in}}{\pgfqpoint{1.681569in}{2.049358in}}{\pgfqpoint{1.673332in}{2.049358in}}%
\pgfpathcurveto{\pgfqpoint{1.665096in}{2.049358in}}{\pgfqpoint{1.657196in}{2.046086in}}{\pgfqpoint{1.651372in}{2.040262in}}%
\pgfpathcurveto{\pgfqpoint{1.645548in}{2.034438in}}{\pgfqpoint{1.642276in}{2.026538in}}{\pgfqpoint{1.642276in}{2.018302in}}%
\pgfpathcurveto{\pgfqpoint{1.642276in}{2.010065in}}{\pgfqpoint{1.645548in}{2.002165in}}{\pgfqpoint{1.651372in}{1.996341in}}%
\pgfpathcurveto{\pgfqpoint{1.657196in}{1.990517in}}{\pgfqpoint{1.665096in}{1.987245in}}{\pgfqpoint{1.673332in}{1.987245in}}%
\pgfpathclose%
\pgfusepath{stroke,fill}%
\end{pgfscope}%
\begin{pgfscope}%
\pgfpathrectangle{\pgfqpoint{0.100000in}{0.212622in}}{\pgfqpoint{3.696000in}{3.696000in}}%
\pgfusepath{clip}%
\pgfsetbuttcap%
\pgfsetroundjoin%
\definecolor{currentfill}{rgb}{0.121569,0.466667,0.705882}%
\pgfsetfillcolor{currentfill}%
\pgfsetfillopacity{0.436735}%
\pgfsetlinewidth{1.003750pt}%
\definecolor{currentstroke}{rgb}{0.121569,0.466667,0.705882}%
\pgfsetstrokecolor{currentstroke}%
\pgfsetstrokeopacity{0.436735}%
\pgfsetdash{}{0pt}%
\pgfpathmoveto{\pgfqpoint{2.337619in}{2.331659in}}%
\pgfpathcurveto{\pgfqpoint{2.345855in}{2.331659in}}{\pgfqpoint{2.353755in}{2.334931in}}{\pgfqpoint{2.359579in}{2.340755in}}%
\pgfpathcurveto{\pgfqpoint{2.365403in}{2.346579in}}{\pgfqpoint{2.368675in}{2.354479in}}{\pgfqpoint{2.368675in}{2.362716in}}%
\pgfpathcurveto{\pgfqpoint{2.368675in}{2.370952in}}{\pgfqpoint{2.365403in}{2.378852in}}{\pgfqpoint{2.359579in}{2.384676in}}%
\pgfpathcurveto{\pgfqpoint{2.353755in}{2.390500in}}{\pgfqpoint{2.345855in}{2.393772in}}{\pgfqpoint{2.337619in}{2.393772in}}%
\pgfpathcurveto{\pgfqpoint{2.329382in}{2.393772in}}{\pgfqpoint{2.321482in}{2.390500in}}{\pgfqpoint{2.315658in}{2.384676in}}%
\pgfpathcurveto{\pgfqpoint{2.309834in}{2.378852in}}{\pgfqpoint{2.306562in}{2.370952in}}{\pgfqpoint{2.306562in}{2.362716in}}%
\pgfpathcurveto{\pgfqpoint{2.306562in}{2.354479in}}{\pgfqpoint{2.309834in}{2.346579in}}{\pgfqpoint{2.315658in}{2.340755in}}%
\pgfpathcurveto{\pgfqpoint{2.321482in}{2.334931in}}{\pgfqpoint{2.329382in}{2.331659in}}{\pgfqpoint{2.337619in}{2.331659in}}%
\pgfpathclose%
\pgfusepath{stroke,fill}%
\end{pgfscope}%
\begin{pgfscope}%
\pgfpathrectangle{\pgfqpoint{0.100000in}{0.212622in}}{\pgfqpoint{3.696000in}{3.696000in}}%
\pgfusepath{clip}%
\pgfsetbuttcap%
\pgfsetroundjoin%
\definecolor{currentfill}{rgb}{0.121569,0.466667,0.705882}%
\pgfsetfillcolor{currentfill}%
\pgfsetfillopacity{0.437730}%
\pgfsetlinewidth{1.003750pt}%
\definecolor{currentstroke}{rgb}{0.121569,0.466667,0.705882}%
\pgfsetstrokecolor{currentstroke}%
\pgfsetstrokeopacity{0.437730}%
\pgfsetdash{}{0pt}%
\pgfpathmoveto{\pgfqpoint{1.745626in}{2.035608in}}%
\pgfpathcurveto{\pgfqpoint{1.753862in}{2.035608in}}{\pgfqpoint{1.761762in}{2.038880in}}{\pgfqpoint{1.767586in}{2.044704in}}%
\pgfpathcurveto{\pgfqpoint{1.773410in}{2.050528in}}{\pgfqpoint{1.776682in}{2.058428in}}{\pgfqpoint{1.776682in}{2.066664in}}%
\pgfpathcurveto{\pgfqpoint{1.776682in}{2.074901in}}{\pgfqpoint{1.773410in}{2.082801in}}{\pgfqpoint{1.767586in}{2.088625in}}%
\pgfpathcurveto{\pgfqpoint{1.761762in}{2.094449in}}{\pgfqpoint{1.753862in}{2.097721in}}{\pgfqpoint{1.745626in}{2.097721in}}%
\pgfpathcurveto{\pgfqpoint{1.737389in}{2.097721in}}{\pgfqpoint{1.729489in}{2.094449in}}{\pgfqpoint{1.723665in}{2.088625in}}%
\pgfpathcurveto{\pgfqpoint{1.717841in}{2.082801in}}{\pgfqpoint{1.714569in}{2.074901in}}{\pgfqpoint{1.714569in}{2.066664in}}%
\pgfpathcurveto{\pgfqpoint{1.714569in}{2.058428in}}{\pgfqpoint{1.717841in}{2.050528in}}{\pgfqpoint{1.723665in}{2.044704in}}%
\pgfpathcurveto{\pgfqpoint{1.729489in}{2.038880in}}{\pgfqpoint{1.737389in}{2.035608in}}{\pgfqpoint{1.745626in}{2.035608in}}%
\pgfpathclose%
\pgfusepath{stroke,fill}%
\end{pgfscope}%
\begin{pgfscope}%
\pgfpathrectangle{\pgfqpoint{0.100000in}{0.212622in}}{\pgfqpoint{3.696000in}{3.696000in}}%
\pgfusepath{clip}%
\pgfsetbuttcap%
\pgfsetroundjoin%
\definecolor{currentfill}{rgb}{0.121569,0.466667,0.705882}%
\pgfsetfillcolor{currentfill}%
\pgfsetfillopacity{0.438355}%
\pgfsetlinewidth{1.003750pt}%
\definecolor{currentstroke}{rgb}{0.121569,0.466667,0.705882}%
\pgfsetstrokecolor{currentstroke}%
\pgfsetstrokeopacity{0.438355}%
\pgfsetdash{}{0pt}%
\pgfpathmoveto{\pgfqpoint{1.645139in}{1.971633in}}%
\pgfpathcurveto{\pgfqpoint{1.653375in}{1.971633in}}{\pgfqpoint{1.661275in}{1.974905in}}{\pgfqpoint{1.667099in}{1.980729in}}%
\pgfpathcurveto{\pgfqpoint{1.672923in}{1.986553in}}{\pgfqpoint{1.676196in}{1.994453in}}{\pgfqpoint{1.676196in}{2.002689in}}%
\pgfpathcurveto{\pgfqpoint{1.676196in}{2.010926in}}{\pgfqpoint{1.672923in}{2.018826in}}{\pgfqpoint{1.667099in}{2.024650in}}%
\pgfpathcurveto{\pgfqpoint{1.661275in}{2.030474in}}{\pgfqpoint{1.653375in}{2.033746in}}{\pgfqpoint{1.645139in}{2.033746in}}%
\pgfpathcurveto{\pgfqpoint{1.636903in}{2.033746in}}{\pgfqpoint{1.629003in}{2.030474in}}{\pgfqpoint{1.623179in}{2.024650in}}%
\pgfpathcurveto{\pgfqpoint{1.617355in}{2.018826in}}{\pgfqpoint{1.614083in}{2.010926in}}{\pgfqpoint{1.614083in}{2.002689in}}%
\pgfpathcurveto{\pgfqpoint{1.614083in}{1.994453in}}{\pgfqpoint{1.617355in}{1.986553in}}{\pgfqpoint{1.623179in}{1.980729in}}%
\pgfpathcurveto{\pgfqpoint{1.629003in}{1.974905in}}{\pgfqpoint{1.636903in}{1.971633in}}{\pgfqpoint{1.645139in}{1.971633in}}%
\pgfpathclose%
\pgfusepath{stroke,fill}%
\end{pgfscope}%
\begin{pgfscope}%
\pgfpathrectangle{\pgfqpoint{0.100000in}{0.212622in}}{\pgfqpoint{3.696000in}{3.696000in}}%
\pgfusepath{clip}%
\pgfsetbuttcap%
\pgfsetroundjoin%
\definecolor{currentfill}{rgb}{0.121569,0.466667,0.705882}%
\pgfsetfillcolor{currentfill}%
\pgfsetfillopacity{0.438698}%
\pgfsetlinewidth{1.003750pt}%
\definecolor{currentstroke}{rgb}{0.121569,0.466667,0.705882}%
\pgfsetstrokecolor{currentstroke}%
\pgfsetstrokeopacity{0.438698}%
\pgfsetdash{}{0pt}%
\pgfpathmoveto{\pgfqpoint{1.614769in}{1.959124in}}%
\pgfpathcurveto{\pgfqpoint{1.623006in}{1.959124in}}{\pgfqpoint{1.630906in}{1.962397in}}{\pgfqpoint{1.636730in}{1.968221in}}%
\pgfpathcurveto{\pgfqpoint{1.642554in}{1.974045in}}{\pgfqpoint{1.645826in}{1.981945in}}{\pgfqpoint{1.645826in}{1.990181in}}%
\pgfpathcurveto{\pgfqpoint{1.645826in}{1.998417in}}{\pgfqpoint{1.642554in}{2.006317in}}{\pgfqpoint{1.636730in}{2.012141in}}%
\pgfpathcurveto{\pgfqpoint{1.630906in}{2.017965in}}{\pgfqpoint{1.623006in}{2.021237in}}{\pgfqpoint{1.614769in}{2.021237in}}%
\pgfpathcurveto{\pgfqpoint{1.606533in}{2.021237in}}{\pgfqpoint{1.598633in}{2.017965in}}{\pgfqpoint{1.592809in}{2.012141in}}%
\pgfpathcurveto{\pgfqpoint{1.586985in}{2.006317in}}{\pgfqpoint{1.583713in}{1.998417in}}{\pgfqpoint{1.583713in}{1.990181in}}%
\pgfpathcurveto{\pgfqpoint{1.583713in}{1.981945in}}{\pgfqpoint{1.586985in}{1.974045in}}{\pgfqpoint{1.592809in}{1.968221in}}%
\pgfpathcurveto{\pgfqpoint{1.598633in}{1.962397in}}{\pgfqpoint{1.606533in}{1.959124in}}{\pgfqpoint{1.614769in}{1.959124in}}%
\pgfpathclose%
\pgfusepath{stroke,fill}%
\end{pgfscope}%
\begin{pgfscope}%
\pgfpathrectangle{\pgfqpoint{0.100000in}{0.212622in}}{\pgfqpoint{3.696000in}{3.696000in}}%
\pgfusepath{clip}%
\pgfsetbuttcap%
\pgfsetroundjoin%
\definecolor{currentfill}{rgb}{0.121569,0.466667,0.705882}%
\pgfsetfillcolor{currentfill}%
\pgfsetfillopacity{0.438867}%
\pgfsetlinewidth{1.003750pt}%
\definecolor{currentstroke}{rgb}{0.121569,0.466667,0.705882}%
\pgfsetstrokecolor{currentstroke}%
\pgfsetstrokeopacity{0.438867}%
\pgfsetdash{}{0pt}%
\pgfpathmoveto{\pgfqpoint{1.609272in}{1.954491in}}%
\pgfpathcurveto{\pgfqpoint{1.617509in}{1.954491in}}{\pgfqpoint{1.625409in}{1.957763in}}{\pgfqpoint{1.631233in}{1.963587in}}%
\pgfpathcurveto{\pgfqpoint{1.637057in}{1.969411in}}{\pgfqpoint{1.640329in}{1.977311in}}{\pgfqpoint{1.640329in}{1.985547in}}%
\pgfpathcurveto{\pgfqpoint{1.640329in}{1.993784in}}{\pgfqpoint{1.637057in}{2.001684in}}{\pgfqpoint{1.631233in}{2.007508in}}%
\pgfpathcurveto{\pgfqpoint{1.625409in}{2.013331in}}{\pgfqpoint{1.617509in}{2.016604in}}{\pgfqpoint{1.609272in}{2.016604in}}%
\pgfpathcurveto{\pgfqpoint{1.601036in}{2.016604in}}{\pgfqpoint{1.593136in}{2.013331in}}{\pgfqpoint{1.587312in}{2.007508in}}%
\pgfpathcurveto{\pgfqpoint{1.581488in}{2.001684in}}{\pgfqpoint{1.578216in}{1.993784in}}{\pgfqpoint{1.578216in}{1.985547in}}%
\pgfpathcurveto{\pgfqpoint{1.578216in}{1.977311in}}{\pgfqpoint{1.581488in}{1.969411in}}{\pgfqpoint{1.587312in}{1.963587in}}%
\pgfpathcurveto{\pgfqpoint{1.593136in}{1.957763in}}{\pgfqpoint{1.601036in}{1.954491in}}{\pgfqpoint{1.609272in}{1.954491in}}%
\pgfpathclose%
\pgfusepath{stroke,fill}%
\end{pgfscope}%
\begin{pgfscope}%
\pgfpathrectangle{\pgfqpoint{0.100000in}{0.212622in}}{\pgfqpoint{3.696000in}{3.696000in}}%
\pgfusepath{clip}%
\pgfsetbuttcap%
\pgfsetroundjoin%
\definecolor{currentfill}{rgb}{0.121569,0.466667,0.705882}%
\pgfsetfillcolor{currentfill}%
\pgfsetfillopacity{0.438940}%
\pgfsetlinewidth{1.003750pt}%
\definecolor{currentstroke}{rgb}{0.121569,0.466667,0.705882}%
\pgfsetstrokecolor{currentstroke}%
\pgfsetstrokeopacity{0.438940}%
\pgfsetdash{}{0pt}%
\pgfpathmoveto{\pgfqpoint{2.523616in}{2.476277in}}%
\pgfpathcurveto{\pgfqpoint{2.531852in}{2.476277in}}{\pgfqpoint{2.539752in}{2.479549in}}{\pgfqpoint{2.545576in}{2.485373in}}%
\pgfpathcurveto{\pgfqpoint{2.551400in}{2.491197in}}{\pgfqpoint{2.554672in}{2.499097in}}{\pgfqpoint{2.554672in}{2.507333in}}%
\pgfpathcurveto{\pgfqpoint{2.554672in}{2.515570in}}{\pgfqpoint{2.551400in}{2.523470in}}{\pgfqpoint{2.545576in}{2.529293in}}%
\pgfpathcurveto{\pgfqpoint{2.539752in}{2.535117in}}{\pgfqpoint{2.531852in}{2.538390in}}{\pgfqpoint{2.523616in}{2.538390in}}%
\pgfpathcurveto{\pgfqpoint{2.515379in}{2.538390in}}{\pgfqpoint{2.507479in}{2.535117in}}{\pgfqpoint{2.501655in}{2.529293in}}%
\pgfpathcurveto{\pgfqpoint{2.495832in}{2.523470in}}{\pgfqpoint{2.492559in}{2.515570in}}{\pgfqpoint{2.492559in}{2.507333in}}%
\pgfpathcurveto{\pgfqpoint{2.492559in}{2.499097in}}{\pgfqpoint{2.495832in}{2.491197in}}{\pgfqpoint{2.501655in}{2.485373in}}%
\pgfpathcurveto{\pgfqpoint{2.507479in}{2.479549in}}{\pgfqpoint{2.515379in}{2.476277in}}{\pgfqpoint{2.523616in}{2.476277in}}%
\pgfpathclose%
\pgfusepath{stroke,fill}%
\end{pgfscope}%
\begin{pgfscope}%
\pgfpathrectangle{\pgfqpoint{0.100000in}{0.212622in}}{\pgfqpoint{3.696000in}{3.696000in}}%
\pgfusepath{clip}%
\pgfsetbuttcap%
\pgfsetroundjoin%
\definecolor{currentfill}{rgb}{0.121569,0.466667,0.705882}%
\pgfsetfillcolor{currentfill}%
\pgfsetfillopacity{0.439164}%
\pgfsetlinewidth{1.003750pt}%
\definecolor{currentstroke}{rgb}{0.121569,0.466667,0.705882}%
\pgfsetstrokecolor{currentstroke}%
\pgfsetstrokeopacity{0.439164}%
\pgfsetdash{}{0pt}%
\pgfpathmoveto{\pgfqpoint{1.638492in}{1.966920in}}%
\pgfpathcurveto{\pgfqpoint{1.646728in}{1.966920in}}{\pgfqpoint{1.654629in}{1.970192in}}{\pgfqpoint{1.660452in}{1.976016in}}%
\pgfpathcurveto{\pgfqpoint{1.666276in}{1.981840in}}{\pgfqpoint{1.669549in}{1.989740in}}{\pgfqpoint{1.669549in}{1.997977in}}%
\pgfpathcurveto{\pgfqpoint{1.669549in}{2.006213in}}{\pgfqpoint{1.666276in}{2.014113in}}{\pgfqpoint{1.660452in}{2.019937in}}%
\pgfpathcurveto{\pgfqpoint{1.654629in}{2.025761in}}{\pgfqpoint{1.646728in}{2.029033in}}{\pgfqpoint{1.638492in}{2.029033in}}%
\pgfpathcurveto{\pgfqpoint{1.630256in}{2.029033in}}{\pgfqpoint{1.622356in}{2.025761in}}{\pgfqpoint{1.616532in}{2.019937in}}%
\pgfpathcurveto{\pgfqpoint{1.610708in}{2.014113in}}{\pgfqpoint{1.607436in}{2.006213in}}{\pgfqpoint{1.607436in}{1.997977in}}%
\pgfpathcurveto{\pgfqpoint{1.607436in}{1.989740in}}{\pgfqpoint{1.610708in}{1.981840in}}{\pgfqpoint{1.616532in}{1.976016in}}%
\pgfpathcurveto{\pgfqpoint{1.622356in}{1.970192in}}{\pgfqpoint{1.630256in}{1.966920in}}{\pgfqpoint{1.638492in}{1.966920in}}%
\pgfpathclose%
\pgfusepath{stroke,fill}%
\end{pgfscope}%
\begin{pgfscope}%
\pgfpathrectangle{\pgfqpoint{0.100000in}{0.212622in}}{\pgfqpoint{3.696000in}{3.696000in}}%
\pgfusepath{clip}%
\pgfsetbuttcap%
\pgfsetroundjoin%
\definecolor{currentfill}{rgb}{0.121569,0.466667,0.705882}%
\pgfsetfillcolor{currentfill}%
\pgfsetfillopacity{0.439167}%
\pgfsetlinewidth{1.003750pt}%
\definecolor{currentstroke}{rgb}{0.121569,0.466667,0.705882}%
\pgfsetstrokecolor{currentstroke}%
\pgfsetstrokeopacity{0.439167}%
\pgfsetdash{}{0pt}%
\pgfpathmoveto{\pgfqpoint{1.737740in}{2.028127in}}%
\pgfpathcurveto{\pgfqpoint{1.745977in}{2.028127in}}{\pgfqpoint{1.753877in}{2.031399in}}{\pgfqpoint{1.759701in}{2.037223in}}%
\pgfpathcurveto{\pgfqpoint{1.765524in}{2.043047in}}{\pgfqpoint{1.768797in}{2.050947in}}{\pgfqpoint{1.768797in}{2.059183in}}%
\pgfpathcurveto{\pgfqpoint{1.768797in}{2.067419in}}{\pgfqpoint{1.765524in}{2.075320in}}{\pgfqpoint{1.759701in}{2.081143in}}%
\pgfpathcurveto{\pgfqpoint{1.753877in}{2.086967in}}{\pgfqpoint{1.745977in}{2.090240in}}{\pgfqpoint{1.737740in}{2.090240in}}%
\pgfpathcurveto{\pgfqpoint{1.729504in}{2.090240in}}{\pgfqpoint{1.721604in}{2.086967in}}{\pgfqpoint{1.715780in}{2.081143in}}%
\pgfpathcurveto{\pgfqpoint{1.709956in}{2.075320in}}{\pgfqpoint{1.706684in}{2.067419in}}{\pgfqpoint{1.706684in}{2.059183in}}%
\pgfpathcurveto{\pgfqpoint{1.706684in}{2.050947in}}{\pgfqpoint{1.709956in}{2.043047in}}{\pgfqpoint{1.715780in}{2.037223in}}%
\pgfpathcurveto{\pgfqpoint{1.721604in}{2.031399in}}{\pgfqpoint{1.729504in}{2.028127in}}{\pgfqpoint{1.737740in}{2.028127in}}%
\pgfpathclose%
\pgfusepath{stroke,fill}%
\end{pgfscope}%
\begin{pgfscope}%
\pgfpathrectangle{\pgfqpoint{0.100000in}{0.212622in}}{\pgfqpoint{3.696000in}{3.696000in}}%
\pgfusepath{clip}%
\pgfsetbuttcap%
\pgfsetroundjoin%
\definecolor{currentfill}{rgb}{0.121569,0.466667,0.705882}%
\pgfsetfillcolor{currentfill}%
\pgfsetfillopacity{0.439232}%
\pgfsetlinewidth{1.003750pt}%
\definecolor{currentstroke}{rgb}{0.121569,0.466667,0.705882}%
\pgfsetstrokecolor{currentstroke}%
\pgfsetstrokeopacity{0.439232}%
\pgfsetdash{}{0pt}%
\pgfpathmoveto{\pgfqpoint{1.651507in}{1.971871in}}%
\pgfpathcurveto{\pgfqpoint{1.659744in}{1.971871in}}{\pgfqpoint{1.667644in}{1.975143in}}{\pgfqpoint{1.673468in}{1.980967in}}%
\pgfpathcurveto{\pgfqpoint{1.679291in}{1.986791in}}{\pgfqpoint{1.682564in}{1.994691in}}{\pgfqpoint{1.682564in}{2.002927in}}%
\pgfpathcurveto{\pgfqpoint{1.682564in}{2.011163in}}{\pgfqpoint{1.679291in}{2.019063in}}{\pgfqpoint{1.673468in}{2.024887in}}%
\pgfpathcurveto{\pgfqpoint{1.667644in}{2.030711in}}{\pgfqpoint{1.659744in}{2.033984in}}{\pgfqpoint{1.651507in}{2.033984in}}%
\pgfpathcurveto{\pgfqpoint{1.643271in}{2.033984in}}{\pgfqpoint{1.635371in}{2.030711in}}{\pgfqpoint{1.629547in}{2.024887in}}%
\pgfpathcurveto{\pgfqpoint{1.623723in}{2.019063in}}{\pgfqpoint{1.620451in}{2.011163in}}{\pgfqpoint{1.620451in}{2.002927in}}%
\pgfpathcurveto{\pgfqpoint{1.620451in}{1.994691in}}{\pgfqpoint{1.623723in}{1.986791in}}{\pgfqpoint{1.629547in}{1.980967in}}%
\pgfpathcurveto{\pgfqpoint{1.635371in}{1.975143in}}{\pgfqpoint{1.643271in}{1.971871in}}{\pgfqpoint{1.651507in}{1.971871in}}%
\pgfpathclose%
\pgfusepath{stroke,fill}%
\end{pgfscope}%
\begin{pgfscope}%
\pgfpathrectangle{\pgfqpoint{0.100000in}{0.212622in}}{\pgfqpoint{3.696000in}{3.696000in}}%
\pgfusepath{clip}%
\pgfsetbuttcap%
\pgfsetroundjoin%
\definecolor{currentfill}{rgb}{0.121569,0.466667,0.705882}%
\pgfsetfillcolor{currentfill}%
\pgfsetfillopacity{0.439623}%
\pgfsetlinewidth{1.003750pt}%
\definecolor{currentstroke}{rgb}{0.121569,0.466667,0.705882}%
\pgfsetstrokecolor{currentstroke}%
\pgfsetstrokeopacity{0.439623}%
\pgfsetdash{}{0pt}%
\pgfpathmoveto{\pgfqpoint{1.752584in}{2.035898in}}%
\pgfpathcurveto{\pgfqpoint{1.760820in}{2.035898in}}{\pgfqpoint{1.768720in}{2.039170in}}{\pgfqpoint{1.774544in}{2.044994in}}%
\pgfpathcurveto{\pgfqpoint{1.780368in}{2.050818in}}{\pgfqpoint{1.783640in}{2.058718in}}{\pgfqpoint{1.783640in}{2.066955in}}%
\pgfpathcurveto{\pgfqpoint{1.783640in}{2.075191in}}{\pgfqpoint{1.780368in}{2.083091in}}{\pgfqpoint{1.774544in}{2.088915in}}%
\pgfpathcurveto{\pgfqpoint{1.768720in}{2.094739in}}{\pgfqpoint{1.760820in}{2.098011in}}{\pgfqpoint{1.752584in}{2.098011in}}%
\pgfpathcurveto{\pgfqpoint{1.744348in}{2.098011in}}{\pgfqpoint{1.736448in}{2.094739in}}{\pgfqpoint{1.730624in}{2.088915in}}%
\pgfpathcurveto{\pgfqpoint{1.724800in}{2.083091in}}{\pgfqpoint{1.721527in}{2.075191in}}{\pgfqpoint{1.721527in}{2.066955in}}%
\pgfpathcurveto{\pgfqpoint{1.721527in}{2.058718in}}{\pgfqpoint{1.724800in}{2.050818in}}{\pgfqpoint{1.730624in}{2.044994in}}%
\pgfpathcurveto{\pgfqpoint{1.736448in}{2.039170in}}{\pgfqpoint{1.744348in}{2.035898in}}{\pgfqpoint{1.752584in}{2.035898in}}%
\pgfpathclose%
\pgfusepath{stroke,fill}%
\end{pgfscope}%
\begin{pgfscope}%
\pgfpathrectangle{\pgfqpoint{0.100000in}{0.212622in}}{\pgfqpoint{3.696000in}{3.696000in}}%
\pgfusepath{clip}%
\pgfsetbuttcap%
\pgfsetroundjoin%
\definecolor{currentfill}{rgb}{0.121569,0.466667,0.705882}%
\pgfsetfillcolor{currentfill}%
\pgfsetfillopacity{0.440045}%
\pgfsetlinewidth{1.003750pt}%
\definecolor{currentstroke}{rgb}{0.121569,0.466667,0.705882}%
\pgfsetstrokecolor{currentstroke}%
\pgfsetstrokeopacity{0.440045}%
\pgfsetdash{}{0pt}%
\pgfpathmoveto{\pgfqpoint{1.742175in}{2.030568in}}%
\pgfpathcurveto{\pgfqpoint{1.750411in}{2.030568in}}{\pgfqpoint{1.758311in}{2.033841in}}{\pgfqpoint{1.764135in}{2.039665in}}%
\pgfpathcurveto{\pgfqpoint{1.769959in}{2.045488in}}{\pgfqpoint{1.773232in}{2.053388in}}{\pgfqpoint{1.773232in}{2.061625in}}%
\pgfpathcurveto{\pgfqpoint{1.773232in}{2.069861in}}{\pgfqpoint{1.769959in}{2.077761in}}{\pgfqpoint{1.764135in}{2.083585in}}%
\pgfpathcurveto{\pgfqpoint{1.758311in}{2.089409in}}{\pgfqpoint{1.750411in}{2.092681in}}{\pgfqpoint{1.742175in}{2.092681in}}%
\pgfpathcurveto{\pgfqpoint{1.733939in}{2.092681in}}{\pgfqpoint{1.726039in}{2.089409in}}{\pgfqpoint{1.720215in}{2.083585in}}%
\pgfpathcurveto{\pgfqpoint{1.714391in}{2.077761in}}{\pgfqpoint{1.711119in}{2.069861in}}{\pgfqpoint{1.711119in}{2.061625in}}%
\pgfpathcurveto{\pgfqpoint{1.711119in}{2.053388in}}{\pgfqpoint{1.714391in}{2.045488in}}{\pgfqpoint{1.720215in}{2.039665in}}%
\pgfpathcurveto{\pgfqpoint{1.726039in}{2.033841in}}{\pgfqpoint{1.733939in}{2.030568in}}{\pgfqpoint{1.742175in}{2.030568in}}%
\pgfpathclose%
\pgfusepath{stroke,fill}%
\end{pgfscope}%
\begin{pgfscope}%
\pgfpathrectangle{\pgfqpoint{0.100000in}{0.212622in}}{\pgfqpoint{3.696000in}{3.696000in}}%
\pgfusepath{clip}%
\pgfsetbuttcap%
\pgfsetroundjoin%
\definecolor{currentfill}{rgb}{0.121569,0.466667,0.705882}%
\pgfsetfillcolor{currentfill}%
\pgfsetfillopacity{0.440085}%
\pgfsetlinewidth{1.003750pt}%
\definecolor{currentstroke}{rgb}{0.121569,0.466667,0.705882}%
\pgfsetstrokecolor{currentstroke}%
\pgfsetstrokeopacity{0.440085}%
\pgfsetdash{}{0pt}%
\pgfpathmoveto{\pgfqpoint{1.533650in}{1.901143in}}%
\pgfpathcurveto{\pgfqpoint{1.541886in}{1.901143in}}{\pgfqpoint{1.549786in}{1.904416in}}{\pgfqpoint{1.555610in}{1.910240in}}%
\pgfpathcurveto{\pgfqpoint{1.561434in}{1.916063in}}{\pgfqpoint{1.564706in}{1.923963in}}{\pgfqpoint{1.564706in}{1.932200in}}%
\pgfpathcurveto{\pgfqpoint{1.564706in}{1.940436in}}{\pgfqpoint{1.561434in}{1.948336in}}{\pgfqpoint{1.555610in}{1.954160in}}%
\pgfpathcurveto{\pgfqpoint{1.549786in}{1.959984in}}{\pgfqpoint{1.541886in}{1.963256in}}{\pgfqpoint{1.533650in}{1.963256in}}%
\pgfpathcurveto{\pgfqpoint{1.525414in}{1.963256in}}{\pgfqpoint{1.517514in}{1.959984in}}{\pgfqpoint{1.511690in}{1.954160in}}%
\pgfpathcurveto{\pgfqpoint{1.505866in}{1.948336in}}{\pgfqpoint{1.502593in}{1.940436in}}{\pgfqpoint{1.502593in}{1.932200in}}%
\pgfpathcurveto{\pgfqpoint{1.502593in}{1.923963in}}{\pgfqpoint{1.505866in}{1.916063in}}{\pgfqpoint{1.511690in}{1.910240in}}%
\pgfpathcurveto{\pgfqpoint{1.517514in}{1.904416in}}{\pgfqpoint{1.525414in}{1.901143in}}{\pgfqpoint{1.533650in}{1.901143in}}%
\pgfpathclose%
\pgfusepath{stroke,fill}%
\end{pgfscope}%
\begin{pgfscope}%
\pgfpathrectangle{\pgfqpoint{0.100000in}{0.212622in}}{\pgfqpoint{3.696000in}{3.696000in}}%
\pgfusepath{clip}%
\pgfsetbuttcap%
\pgfsetroundjoin%
\definecolor{currentfill}{rgb}{0.121569,0.466667,0.705882}%
\pgfsetfillcolor{currentfill}%
\pgfsetfillopacity{0.440220}%
\pgfsetlinewidth{1.003750pt}%
\definecolor{currentstroke}{rgb}{0.121569,0.466667,0.705882}%
\pgfsetstrokecolor{currentstroke}%
\pgfsetstrokeopacity{0.440220}%
\pgfsetdash{}{0pt}%
\pgfpathmoveto{\pgfqpoint{1.670597in}{1.977628in}}%
\pgfpathcurveto{\pgfqpoint{1.678833in}{1.977628in}}{\pgfqpoint{1.686734in}{1.980900in}}{\pgfqpoint{1.692557in}{1.986724in}}%
\pgfpathcurveto{\pgfqpoint{1.698381in}{1.992548in}}{\pgfqpoint{1.701654in}{2.000448in}}{\pgfqpoint{1.701654in}{2.008684in}}%
\pgfpathcurveto{\pgfqpoint{1.701654in}{2.016921in}}{\pgfqpoint{1.698381in}{2.024821in}}{\pgfqpoint{1.692557in}{2.030644in}}%
\pgfpathcurveto{\pgfqpoint{1.686734in}{2.036468in}}{\pgfqpoint{1.678833in}{2.039741in}}{\pgfqpoint{1.670597in}{2.039741in}}%
\pgfpathcurveto{\pgfqpoint{1.662361in}{2.039741in}}{\pgfqpoint{1.654461in}{2.036468in}}{\pgfqpoint{1.648637in}{2.030644in}}%
\pgfpathcurveto{\pgfqpoint{1.642813in}{2.024821in}}{\pgfqpoint{1.639541in}{2.016921in}}{\pgfqpoint{1.639541in}{2.008684in}}%
\pgfpathcurveto{\pgfqpoint{1.639541in}{2.000448in}}{\pgfqpoint{1.642813in}{1.992548in}}{\pgfqpoint{1.648637in}{1.986724in}}%
\pgfpathcurveto{\pgfqpoint{1.654461in}{1.980900in}}{\pgfqpoint{1.662361in}{1.977628in}}{\pgfqpoint{1.670597in}{1.977628in}}%
\pgfpathclose%
\pgfusepath{stroke,fill}%
\end{pgfscope}%
\begin{pgfscope}%
\pgfpathrectangle{\pgfqpoint{0.100000in}{0.212622in}}{\pgfqpoint{3.696000in}{3.696000in}}%
\pgfusepath{clip}%
\pgfsetbuttcap%
\pgfsetroundjoin%
\definecolor{currentfill}{rgb}{0.121569,0.466667,0.705882}%
\pgfsetfillcolor{currentfill}%
\pgfsetfillopacity{0.440284}%
\pgfsetlinewidth{1.003750pt}%
\definecolor{currentstroke}{rgb}{0.121569,0.466667,0.705882}%
\pgfsetstrokecolor{currentstroke}%
\pgfsetstrokeopacity{0.440284}%
\pgfsetdash{}{0pt}%
\pgfpathmoveto{\pgfqpoint{1.654705in}{1.973526in}}%
\pgfpathcurveto{\pgfqpoint{1.662941in}{1.973526in}}{\pgfqpoint{1.670841in}{1.976798in}}{\pgfqpoint{1.676665in}{1.982622in}}%
\pgfpathcurveto{\pgfqpoint{1.682489in}{1.988446in}}{\pgfqpoint{1.685761in}{1.996346in}}{\pgfqpoint{1.685761in}{2.004582in}}%
\pgfpathcurveto{\pgfqpoint{1.685761in}{2.012819in}}{\pgfqpoint{1.682489in}{2.020719in}}{\pgfqpoint{1.676665in}{2.026543in}}%
\pgfpathcurveto{\pgfqpoint{1.670841in}{2.032367in}}{\pgfqpoint{1.662941in}{2.035639in}}{\pgfqpoint{1.654705in}{2.035639in}}%
\pgfpathcurveto{\pgfqpoint{1.646468in}{2.035639in}}{\pgfqpoint{1.638568in}{2.032367in}}{\pgfqpoint{1.632745in}{2.026543in}}%
\pgfpathcurveto{\pgfqpoint{1.626921in}{2.020719in}}{\pgfqpoint{1.623648in}{2.012819in}}{\pgfqpoint{1.623648in}{2.004582in}}%
\pgfpathcurveto{\pgfqpoint{1.623648in}{1.996346in}}{\pgfqpoint{1.626921in}{1.988446in}}{\pgfqpoint{1.632745in}{1.982622in}}%
\pgfpathcurveto{\pgfqpoint{1.638568in}{1.976798in}}{\pgfqpoint{1.646468in}{1.973526in}}{\pgfqpoint{1.654705in}{1.973526in}}%
\pgfpathclose%
\pgfusepath{stroke,fill}%
\end{pgfscope}%
\begin{pgfscope}%
\pgfpathrectangle{\pgfqpoint{0.100000in}{0.212622in}}{\pgfqpoint{3.696000in}{3.696000in}}%
\pgfusepath{clip}%
\pgfsetbuttcap%
\pgfsetroundjoin%
\definecolor{currentfill}{rgb}{0.121569,0.466667,0.705882}%
\pgfsetfillcolor{currentfill}%
\pgfsetfillopacity{0.440568}%
\pgfsetlinewidth{1.003750pt}%
\definecolor{currentstroke}{rgb}{0.121569,0.466667,0.705882}%
\pgfsetstrokecolor{currentstroke}%
\pgfsetstrokeopacity{0.440568}%
\pgfsetdash{}{0pt}%
\pgfpathmoveto{\pgfqpoint{1.537494in}{1.904623in}}%
\pgfpathcurveto{\pgfqpoint{1.545731in}{1.904623in}}{\pgfqpoint{1.553631in}{1.907895in}}{\pgfqpoint{1.559455in}{1.913719in}}%
\pgfpathcurveto{\pgfqpoint{1.565278in}{1.919543in}}{\pgfqpoint{1.568551in}{1.927443in}}{\pgfqpoint{1.568551in}{1.935679in}}%
\pgfpathcurveto{\pgfqpoint{1.568551in}{1.943916in}}{\pgfqpoint{1.565278in}{1.951816in}}{\pgfqpoint{1.559455in}{1.957640in}}%
\pgfpathcurveto{\pgfqpoint{1.553631in}{1.963464in}}{\pgfqpoint{1.545731in}{1.966736in}}{\pgfqpoint{1.537494in}{1.966736in}}%
\pgfpathcurveto{\pgfqpoint{1.529258in}{1.966736in}}{\pgfqpoint{1.521358in}{1.963464in}}{\pgfqpoint{1.515534in}{1.957640in}}%
\pgfpathcurveto{\pgfqpoint{1.509710in}{1.951816in}}{\pgfqpoint{1.506438in}{1.943916in}}{\pgfqpoint{1.506438in}{1.935679in}}%
\pgfpathcurveto{\pgfqpoint{1.506438in}{1.927443in}}{\pgfqpoint{1.509710in}{1.919543in}}{\pgfqpoint{1.515534in}{1.913719in}}%
\pgfpathcurveto{\pgfqpoint{1.521358in}{1.907895in}}{\pgfqpoint{1.529258in}{1.904623in}}{\pgfqpoint{1.537494in}{1.904623in}}%
\pgfpathclose%
\pgfusepath{stroke,fill}%
\end{pgfscope}%
\begin{pgfscope}%
\pgfpathrectangle{\pgfqpoint{0.100000in}{0.212622in}}{\pgfqpoint{3.696000in}{3.696000in}}%
\pgfusepath{clip}%
\pgfsetbuttcap%
\pgfsetroundjoin%
\definecolor{currentfill}{rgb}{0.121569,0.466667,0.705882}%
\pgfsetfillcolor{currentfill}%
\pgfsetfillopacity{0.440805}%
\pgfsetlinewidth{1.003750pt}%
\definecolor{currentstroke}{rgb}{0.121569,0.466667,0.705882}%
\pgfsetstrokecolor{currentstroke}%
\pgfsetstrokeopacity{0.440805}%
\pgfsetdash{}{0pt}%
\pgfpathmoveto{\pgfqpoint{1.744217in}{2.030888in}}%
\pgfpathcurveto{\pgfqpoint{1.752454in}{2.030888in}}{\pgfqpoint{1.760354in}{2.034160in}}{\pgfqpoint{1.766178in}{2.039984in}}%
\pgfpathcurveto{\pgfqpoint{1.772002in}{2.045808in}}{\pgfqpoint{1.775274in}{2.053708in}}{\pgfqpoint{1.775274in}{2.061945in}}%
\pgfpathcurveto{\pgfqpoint{1.775274in}{2.070181in}}{\pgfqpoint{1.772002in}{2.078081in}}{\pgfqpoint{1.766178in}{2.083905in}}%
\pgfpathcurveto{\pgfqpoint{1.760354in}{2.089729in}}{\pgfqpoint{1.752454in}{2.093001in}}{\pgfqpoint{1.744217in}{2.093001in}}%
\pgfpathcurveto{\pgfqpoint{1.735981in}{2.093001in}}{\pgfqpoint{1.728081in}{2.089729in}}{\pgfqpoint{1.722257in}{2.083905in}}%
\pgfpathcurveto{\pgfqpoint{1.716433in}{2.078081in}}{\pgfqpoint{1.713161in}{2.070181in}}{\pgfqpoint{1.713161in}{2.061945in}}%
\pgfpathcurveto{\pgfqpoint{1.713161in}{2.053708in}}{\pgfqpoint{1.716433in}{2.045808in}}{\pgfqpoint{1.722257in}{2.039984in}}%
\pgfpathcurveto{\pgfqpoint{1.728081in}{2.034160in}}{\pgfqpoint{1.735981in}{2.030888in}}{\pgfqpoint{1.744217in}{2.030888in}}%
\pgfpathclose%
\pgfusepath{stroke,fill}%
\end{pgfscope}%
\begin{pgfscope}%
\pgfpathrectangle{\pgfqpoint{0.100000in}{0.212622in}}{\pgfqpoint{3.696000in}{3.696000in}}%
\pgfusepath{clip}%
\pgfsetbuttcap%
\pgfsetroundjoin%
\definecolor{currentfill}{rgb}{0.121569,0.466667,0.705882}%
\pgfsetfillcolor{currentfill}%
\pgfsetfillopacity{0.440824}%
\pgfsetlinewidth{1.003750pt}%
\definecolor{currentstroke}{rgb}{0.121569,0.466667,0.705882}%
\pgfsetstrokecolor{currentstroke}%
\pgfsetstrokeopacity{0.440824}%
\pgfsetdash{}{0pt}%
\pgfpathmoveto{\pgfqpoint{2.523450in}{2.475614in}}%
\pgfpathcurveto{\pgfqpoint{2.531687in}{2.475614in}}{\pgfqpoint{2.539587in}{2.478886in}}{\pgfqpoint{2.545411in}{2.484710in}}%
\pgfpathcurveto{\pgfqpoint{2.551235in}{2.490534in}}{\pgfqpoint{2.554507in}{2.498434in}}{\pgfqpoint{2.554507in}{2.506670in}}%
\pgfpathcurveto{\pgfqpoint{2.554507in}{2.514907in}}{\pgfqpoint{2.551235in}{2.522807in}}{\pgfqpoint{2.545411in}{2.528631in}}%
\pgfpathcurveto{\pgfqpoint{2.539587in}{2.534455in}}{\pgfqpoint{2.531687in}{2.537727in}}{\pgfqpoint{2.523450in}{2.537727in}}%
\pgfpathcurveto{\pgfqpoint{2.515214in}{2.537727in}}{\pgfqpoint{2.507314in}{2.534455in}}{\pgfqpoint{2.501490in}{2.528631in}}%
\pgfpathcurveto{\pgfqpoint{2.495666in}{2.522807in}}{\pgfqpoint{2.492394in}{2.514907in}}{\pgfqpoint{2.492394in}{2.506670in}}%
\pgfpathcurveto{\pgfqpoint{2.492394in}{2.498434in}}{\pgfqpoint{2.495666in}{2.490534in}}{\pgfqpoint{2.501490in}{2.484710in}}%
\pgfpathcurveto{\pgfqpoint{2.507314in}{2.478886in}}{\pgfqpoint{2.515214in}{2.475614in}}{\pgfqpoint{2.523450in}{2.475614in}}%
\pgfpathclose%
\pgfusepath{stroke,fill}%
\end{pgfscope}%
\begin{pgfscope}%
\pgfpathrectangle{\pgfqpoint{0.100000in}{0.212622in}}{\pgfqpoint{3.696000in}{3.696000in}}%
\pgfusepath{clip}%
\pgfsetbuttcap%
\pgfsetroundjoin%
\definecolor{currentfill}{rgb}{0.121569,0.466667,0.705882}%
\pgfsetfillcolor{currentfill}%
\pgfsetfillopacity{0.440908}%
\pgfsetlinewidth{1.003750pt}%
\definecolor{currentstroke}{rgb}{0.121569,0.466667,0.705882}%
\pgfsetstrokecolor{currentstroke}%
\pgfsetstrokeopacity{0.440908}%
\pgfsetdash{}{0pt}%
\pgfpathmoveto{\pgfqpoint{1.599753in}{1.947291in}}%
\pgfpathcurveto{\pgfqpoint{1.607989in}{1.947291in}}{\pgfqpoint{1.615889in}{1.950564in}}{\pgfqpoint{1.621713in}{1.956388in}}%
\pgfpathcurveto{\pgfqpoint{1.627537in}{1.962212in}}{\pgfqpoint{1.630809in}{1.970112in}}{\pgfqpoint{1.630809in}{1.978348in}}%
\pgfpathcurveto{\pgfqpoint{1.630809in}{1.986584in}}{\pgfqpoint{1.627537in}{1.994484in}}{\pgfqpoint{1.621713in}{2.000308in}}%
\pgfpathcurveto{\pgfqpoint{1.615889in}{2.006132in}}{\pgfqpoint{1.607989in}{2.009404in}}{\pgfqpoint{1.599753in}{2.009404in}}%
\pgfpathcurveto{\pgfqpoint{1.591516in}{2.009404in}}{\pgfqpoint{1.583616in}{2.006132in}}{\pgfqpoint{1.577792in}{2.000308in}}%
\pgfpathcurveto{\pgfqpoint{1.571968in}{1.994484in}}{\pgfqpoint{1.568696in}{1.986584in}}{\pgfqpoint{1.568696in}{1.978348in}}%
\pgfpathcurveto{\pgfqpoint{1.568696in}{1.970112in}}{\pgfqpoint{1.571968in}{1.962212in}}{\pgfqpoint{1.577792in}{1.956388in}}%
\pgfpathcurveto{\pgfqpoint{1.583616in}{1.950564in}}{\pgfqpoint{1.591516in}{1.947291in}}{\pgfqpoint{1.599753in}{1.947291in}}%
\pgfpathclose%
\pgfusepath{stroke,fill}%
\end{pgfscope}%
\begin{pgfscope}%
\pgfpathrectangle{\pgfqpoint{0.100000in}{0.212622in}}{\pgfqpoint{3.696000in}{3.696000in}}%
\pgfusepath{clip}%
\pgfsetbuttcap%
\pgfsetroundjoin%
\definecolor{currentfill}{rgb}{0.121569,0.466667,0.705882}%
\pgfsetfillcolor{currentfill}%
\pgfsetfillopacity{0.440934}%
\pgfsetlinewidth{1.003750pt}%
\definecolor{currentstroke}{rgb}{0.121569,0.466667,0.705882}%
\pgfsetstrokecolor{currentstroke}%
\pgfsetstrokeopacity{0.440934}%
\pgfsetdash{}{0pt}%
\pgfpathmoveto{\pgfqpoint{1.754483in}{2.035406in}}%
\pgfpathcurveto{\pgfqpoint{1.762719in}{2.035406in}}{\pgfqpoint{1.770619in}{2.038678in}}{\pgfqpoint{1.776443in}{2.044502in}}%
\pgfpathcurveto{\pgfqpoint{1.782267in}{2.050326in}}{\pgfqpoint{1.785539in}{2.058226in}}{\pgfqpoint{1.785539in}{2.066463in}}%
\pgfpathcurveto{\pgfqpoint{1.785539in}{2.074699in}}{\pgfqpoint{1.782267in}{2.082599in}}{\pgfqpoint{1.776443in}{2.088423in}}%
\pgfpathcurveto{\pgfqpoint{1.770619in}{2.094247in}}{\pgfqpoint{1.762719in}{2.097519in}}{\pgfqpoint{1.754483in}{2.097519in}}%
\pgfpathcurveto{\pgfqpoint{1.746246in}{2.097519in}}{\pgfqpoint{1.738346in}{2.094247in}}{\pgfqpoint{1.732522in}{2.088423in}}%
\pgfpathcurveto{\pgfqpoint{1.726699in}{2.082599in}}{\pgfqpoint{1.723426in}{2.074699in}}{\pgfqpoint{1.723426in}{2.066463in}}%
\pgfpathcurveto{\pgfqpoint{1.723426in}{2.058226in}}{\pgfqpoint{1.726699in}{2.050326in}}{\pgfqpoint{1.732522in}{2.044502in}}%
\pgfpathcurveto{\pgfqpoint{1.738346in}{2.038678in}}{\pgfqpoint{1.746246in}{2.035406in}}{\pgfqpoint{1.754483in}{2.035406in}}%
\pgfpathclose%
\pgfusepath{stroke,fill}%
\end{pgfscope}%
\begin{pgfscope}%
\pgfpathrectangle{\pgfqpoint{0.100000in}{0.212622in}}{\pgfqpoint{3.696000in}{3.696000in}}%
\pgfusepath{clip}%
\pgfsetbuttcap%
\pgfsetroundjoin%
\definecolor{currentfill}{rgb}{0.121569,0.466667,0.705882}%
\pgfsetfillcolor{currentfill}%
\pgfsetfillopacity{0.441289}%
\pgfsetlinewidth{1.003750pt}%
\definecolor{currentstroke}{rgb}{0.121569,0.466667,0.705882}%
\pgfsetstrokecolor{currentstroke}%
\pgfsetstrokeopacity{0.441289}%
\pgfsetdash{}{0pt}%
\pgfpathmoveto{\pgfqpoint{1.594755in}{1.946035in}}%
\pgfpathcurveto{\pgfqpoint{1.602991in}{1.946035in}}{\pgfqpoint{1.610891in}{1.949307in}}{\pgfqpoint{1.616715in}{1.955131in}}%
\pgfpathcurveto{\pgfqpoint{1.622539in}{1.960955in}}{\pgfqpoint{1.625812in}{1.968855in}}{\pgfqpoint{1.625812in}{1.977091in}}%
\pgfpathcurveto{\pgfqpoint{1.625812in}{1.985328in}}{\pgfqpoint{1.622539in}{1.993228in}}{\pgfqpoint{1.616715in}{1.999052in}}%
\pgfpathcurveto{\pgfqpoint{1.610891in}{2.004876in}}{\pgfqpoint{1.602991in}{2.008148in}}{\pgfqpoint{1.594755in}{2.008148in}}%
\pgfpathcurveto{\pgfqpoint{1.586519in}{2.008148in}}{\pgfqpoint{1.578619in}{2.004876in}}{\pgfqpoint{1.572795in}{1.999052in}}%
\pgfpathcurveto{\pgfqpoint{1.566971in}{1.993228in}}{\pgfqpoint{1.563699in}{1.985328in}}{\pgfqpoint{1.563699in}{1.977091in}}%
\pgfpathcurveto{\pgfqpoint{1.563699in}{1.968855in}}{\pgfqpoint{1.566971in}{1.960955in}}{\pgfqpoint{1.572795in}{1.955131in}}%
\pgfpathcurveto{\pgfqpoint{1.578619in}{1.949307in}}{\pgfqpoint{1.586519in}{1.946035in}}{\pgfqpoint{1.594755in}{1.946035in}}%
\pgfpathclose%
\pgfusepath{stroke,fill}%
\end{pgfscope}%
\begin{pgfscope}%
\pgfpathrectangle{\pgfqpoint{0.100000in}{0.212622in}}{\pgfqpoint{3.696000in}{3.696000in}}%
\pgfusepath{clip}%
\pgfsetbuttcap%
\pgfsetroundjoin%
\definecolor{currentfill}{rgb}{0.121569,0.466667,0.705882}%
\pgfsetfillcolor{currentfill}%
\pgfsetfillopacity{0.441446}%
\pgfsetlinewidth{1.003750pt}%
\definecolor{currentstroke}{rgb}{0.121569,0.466667,0.705882}%
\pgfsetstrokecolor{currentstroke}%
\pgfsetstrokeopacity{0.441446}%
\pgfsetdash{}{0pt}%
\pgfpathmoveto{\pgfqpoint{1.595960in}{1.946100in}}%
\pgfpathcurveto{\pgfqpoint{1.604197in}{1.946100in}}{\pgfqpoint{1.612097in}{1.949373in}}{\pgfqpoint{1.617921in}{1.955197in}}%
\pgfpathcurveto{\pgfqpoint{1.623745in}{1.961020in}}{\pgfqpoint{1.627017in}{1.968921in}}{\pgfqpoint{1.627017in}{1.977157in}}%
\pgfpathcurveto{\pgfqpoint{1.627017in}{1.985393in}}{\pgfqpoint{1.623745in}{1.993293in}}{\pgfqpoint{1.617921in}{1.999117in}}%
\pgfpathcurveto{\pgfqpoint{1.612097in}{2.004941in}}{\pgfqpoint{1.604197in}{2.008213in}}{\pgfqpoint{1.595960in}{2.008213in}}%
\pgfpathcurveto{\pgfqpoint{1.587724in}{2.008213in}}{\pgfqpoint{1.579824in}{2.004941in}}{\pgfqpoint{1.574000in}{1.999117in}}%
\pgfpathcurveto{\pgfqpoint{1.568176in}{1.993293in}}{\pgfqpoint{1.564904in}{1.985393in}}{\pgfqpoint{1.564904in}{1.977157in}}%
\pgfpathcurveto{\pgfqpoint{1.564904in}{1.968921in}}{\pgfqpoint{1.568176in}{1.961020in}}{\pgfqpoint{1.574000in}{1.955197in}}%
\pgfpathcurveto{\pgfqpoint{1.579824in}{1.949373in}}{\pgfqpoint{1.587724in}{1.946100in}}{\pgfqpoint{1.595960in}{1.946100in}}%
\pgfpathclose%
\pgfusepath{stroke,fill}%
\end{pgfscope}%
\begin{pgfscope}%
\pgfpathrectangle{\pgfqpoint{0.100000in}{0.212622in}}{\pgfqpoint{3.696000in}{3.696000in}}%
\pgfusepath{clip}%
\pgfsetbuttcap%
\pgfsetroundjoin%
\definecolor{currentfill}{rgb}{0.121569,0.466667,0.705882}%
\pgfsetfillcolor{currentfill}%
\pgfsetfillopacity{0.441636}%
\pgfsetlinewidth{1.003750pt}%
\definecolor{currentstroke}{rgb}{0.121569,0.466667,0.705882}%
\pgfsetstrokecolor{currentstroke}%
\pgfsetstrokeopacity{0.441636}%
\pgfsetdash{}{0pt}%
\pgfpathmoveto{\pgfqpoint{1.638775in}{1.961578in}}%
\pgfpathcurveto{\pgfqpoint{1.647011in}{1.961578in}}{\pgfqpoint{1.654911in}{1.964850in}}{\pgfqpoint{1.660735in}{1.970674in}}%
\pgfpathcurveto{\pgfqpoint{1.666559in}{1.976498in}}{\pgfqpoint{1.669832in}{1.984398in}}{\pgfqpoint{1.669832in}{1.992634in}}%
\pgfpathcurveto{\pgfqpoint{1.669832in}{2.000871in}}{\pgfqpoint{1.666559in}{2.008771in}}{\pgfqpoint{1.660735in}{2.014595in}}%
\pgfpathcurveto{\pgfqpoint{1.654911in}{2.020418in}}{\pgfqpoint{1.647011in}{2.023691in}}{\pgfqpoint{1.638775in}{2.023691in}}%
\pgfpathcurveto{\pgfqpoint{1.630539in}{2.023691in}}{\pgfqpoint{1.622639in}{2.020418in}}{\pgfqpoint{1.616815in}{2.014595in}}%
\pgfpathcurveto{\pgfqpoint{1.610991in}{2.008771in}}{\pgfqpoint{1.607719in}{2.000871in}}{\pgfqpoint{1.607719in}{1.992634in}}%
\pgfpathcurveto{\pgfqpoint{1.607719in}{1.984398in}}{\pgfqpoint{1.610991in}{1.976498in}}{\pgfqpoint{1.616815in}{1.970674in}}%
\pgfpathcurveto{\pgfqpoint{1.622639in}{1.964850in}}{\pgfqpoint{1.630539in}{1.961578in}}{\pgfqpoint{1.638775in}{1.961578in}}%
\pgfpathclose%
\pgfusepath{stroke,fill}%
\end{pgfscope}%
\begin{pgfscope}%
\pgfpathrectangle{\pgfqpoint{0.100000in}{0.212622in}}{\pgfqpoint{3.696000in}{3.696000in}}%
\pgfusepath{clip}%
\pgfsetbuttcap%
\pgfsetroundjoin%
\definecolor{currentfill}{rgb}{0.121569,0.466667,0.705882}%
\pgfsetfillcolor{currentfill}%
\pgfsetfillopacity{0.441863}%
\pgfsetlinewidth{1.003750pt}%
\definecolor{currentstroke}{rgb}{0.121569,0.466667,0.705882}%
\pgfsetstrokecolor{currentstroke}%
\pgfsetstrokeopacity{0.441863}%
\pgfsetdash{}{0pt}%
\pgfpathmoveto{\pgfqpoint{1.679035in}{1.985447in}}%
\pgfpathcurveto{\pgfqpoint{1.687271in}{1.985447in}}{\pgfqpoint{1.695172in}{1.988719in}}{\pgfqpoint{1.700995in}{1.994543in}}%
\pgfpathcurveto{\pgfqpoint{1.706819in}{2.000367in}}{\pgfqpoint{1.710092in}{2.008267in}}{\pgfqpoint{1.710092in}{2.016503in}}%
\pgfpathcurveto{\pgfqpoint{1.710092in}{2.024739in}}{\pgfqpoint{1.706819in}{2.032639in}}{\pgfqpoint{1.700995in}{2.038463in}}%
\pgfpathcurveto{\pgfqpoint{1.695172in}{2.044287in}}{\pgfqpoint{1.687271in}{2.047560in}}{\pgfqpoint{1.679035in}{2.047560in}}%
\pgfpathcurveto{\pgfqpoint{1.670799in}{2.047560in}}{\pgfqpoint{1.662899in}{2.044287in}}{\pgfqpoint{1.657075in}{2.038463in}}%
\pgfpathcurveto{\pgfqpoint{1.651251in}{2.032639in}}{\pgfqpoint{1.647979in}{2.024739in}}{\pgfqpoint{1.647979in}{2.016503in}}%
\pgfpathcurveto{\pgfqpoint{1.647979in}{2.008267in}}{\pgfqpoint{1.651251in}{2.000367in}}{\pgfqpoint{1.657075in}{1.994543in}}%
\pgfpathcurveto{\pgfqpoint{1.662899in}{1.988719in}}{\pgfqpoint{1.670799in}{1.985447in}}{\pgfqpoint{1.679035in}{1.985447in}}%
\pgfpathclose%
\pgfusepath{stroke,fill}%
\end{pgfscope}%
\begin{pgfscope}%
\pgfpathrectangle{\pgfqpoint{0.100000in}{0.212622in}}{\pgfqpoint{3.696000in}{3.696000in}}%
\pgfusepath{clip}%
\pgfsetbuttcap%
\pgfsetroundjoin%
\definecolor{currentfill}{rgb}{0.121569,0.466667,0.705882}%
\pgfsetfillcolor{currentfill}%
\pgfsetfillopacity{0.442146}%
\pgfsetlinewidth{1.003750pt}%
\definecolor{currentstroke}{rgb}{0.121569,0.466667,0.705882}%
\pgfsetstrokecolor{currentstroke}%
\pgfsetstrokeopacity{0.442146}%
\pgfsetdash{}{0pt}%
\pgfpathmoveto{\pgfqpoint{1.725940in}{2.017030in}}%
\pgfpathcurveto{\pgfqpoint{1.734176in}{2.017030in}}{\pgfqpoint{1.742076in}{2.020303in}}{\pgfqpoint{1.747900in}{2.026127in}}%
\pgfpathcurveto{\pgfqpoint{1.753724in}{2.031950in}}{\pgfqpoint{1.756996in}{2.039850in}}{\pgfqpoint{1.756996in}{2.048087in}}%
\pgfpathcurveto{\pgfqpoint{1.756996in}{2.056323in}}{\pgfqpoint{1.753724in}{2.064223in}}{\pgfqpoint{1.747900in}{2.070047in}}%
\pgfpathcurveto{\pgfqpoint{1.742076in}{2.075871in}}{\pgfqpoint{1.734176in}{2.079143in}}{\pgfqpoint{1.725940in}{2.079143in}}%
\pgfpathcurveto{\pgfqpoint{1.717703in}{2.079143in}}{\pgfqpoint{1.709803in}{2.075871in}}{\pgfqpoint{1.703979in}{2.070047in}}%
\pgfpathcurveto{\pgfqpoint{1.698155in}{2.064223in}}{\pgfqpoint{1.694883in}{2.056323in}}{\pgfqpoint{1.694883in}{2.048087in}}%
\pgfpathcurveto{\pgfqpoint{1.694883in}{2.039850in}}{\pgfqpoint{1.698155in}{2.031950in}}{\pgfqpoint{1.703979in}{2.026127in}}%
\pgfpathcurveto{\pgfqpoint{1.709803in}{2.020303in}}{\pgfqpoint{1.717703in}{2.017030in}}{\pgfqpoint{1.725940in}{2.017030in}}%
\pgfpathclose%
\pgfusepath{stroke,fill}%
\end{pgfscope}%
\begin{pgfscope}%
\pgfpathrectangle{\pgfqpoint{0.100000in}{0.212622in}}{\pgfqpoint{3.696000in}{3.696000in}}%
\pgfusepath{clip}%
\pgfsetbuttcap%
\pgfsetroundjoin%
\definecolor{currentfill}{rgb}{0.121569,0.466667,0.705882}%
\pgfsetfillcolor{currentfill}%
\pgfsetfillopacity{0.442282}%
\pgfsetlinewidth{1.003750pt}%
\definecolor{currentstroke}{rgb}{0.121569,0.466667,0.705882}%
\pgfsetstrokecolor{currentstroke}%
\pgfsetstrokeopacity{0.442282}%
\pgfsetdash{}{0pt}%
\pgfpathmoveto{\pgfqpoint{2.522316in}{2.474668in}}%
\pgfpathcurveto{\pgfqpoint{2.530552in}{2.474668in}}{\pgfqpoint{2.538452in}{2.477940in}}{\pgfqpoint{2.544276in}{2.483764in}}%
\pgfpathcurveto{\pgfqpoint{2.550100in}{2.489588in}}{\pgfqpoint{2.553373in}{2.497488in}}{\pgfqpoint{2.553373in}{2.505724in}}%
\pgfpathcurveto{\pgfqpoint{2.553373in}{2.513961in}}{\pgfqpoint{2.550100in}{2.521861in}}{\pgfqpoint{2.544276in}{2.527685in}}%
\pgfpathcurveto{\pgfqpoint{2.538452in}{2.533509in}}{\pgfqpoint{2.530552in}{2.536781in}}{\pgfqpoint{2.522316in}{2.536781in}}%
\pgfpathcurveto{\pgfqpoint{2.514080in}{2.536781in}}{\pgfqpoint{2.506180in}{2.533509in}}{\pgfqpoint{2.500356in}{2.527685in}}%
\pgfpathcurveto{\pgfqpoint{2.494532in}{2.521861in}}{\pgfqpoint{2.491260in}{2.513961in}}{\pgfqpoint{2.491260in}{2.505724in}}%
\pgfpathcurveto{\pgfqpoint{2.491260in}{2.497488in}}{\pgfqpoint{2.494532in}{2.489588in}}{\pgfqpoint{2.500356in}{2.483764in}}%
\pgfpathcurveto{\pgfqpoint{2.506180in}{2.477940in}}{\pgfqpoint{2.514080in}{2.474668in}}{\pgfqpoint{2.522316in}{2.474668in}}%
\pgfpathclose%
\pgfusepath{stroke,fill}%
\end{pgfscope}%
\begin{pgfscope}%
\pgfpathrectangle{\pgfqpoint{0.100000in}{0.212622in}}{\pgfqpoint{3.696000in}{3.696000in}}%
\pgfusepath{clip}%
\pgfsetbuttcap%
\pgfsetroundjoin%
\definecolor{currentfill}{rgb}{0.121569,0.466667,0.705882}%
\pgfsetfillcolor{currentfill}%
\pgfsetfillopacity{0.442601}%
\pgfsetlinewidth{1.003750pt}%
\definecolor{currentstroke}{rgb}{0.121569,0.466667,0.705882}%
\pgfsetstrokecolor{currentstroke}%
\pgfsetstrokeopacity{0.442601}%
\pgfsetdash{}{0pt}%
\pgfpathmoveto{\pgfqpoint{1.534068in}{1.901150in}}%
\pgfpathcurveto{\pgfqpoint{1.542305in}{1.901150in}}{\pgfqpoint{1.550205in}{1.904422in}}{\pgfqpoint{1.556029in}{1.910246in}}%
\pgfpathcurveto{\pgfqpoint{1.561853in}{1.916070in}}{\pgfqpoint{1.565125in}{1.923970in}}{\pgfqpoint{1.565125in}{1.932206in}}%
\pgfpathcurveto{\pgfqpoint{1.565125in}{1.940443in}}{\pgfqpoint{1.561853in}{1.948343in}}{\pgfqpoint{1.556029in}{1.954167in}}%
\pgfpathcurveto{\pgfqpoint{1.550205in}{1.959991in}}{\pgfqpoint{1.542305in}{1.963263in}}{\pgfqpoint{1.534068in}{1.963263in}}%
\pgfpathcurveto{\pgfqpoint{1.525832in}{1.963263in}}{\pgfqpoint{1.517932in}{1.959991in}}{\pgfqpoint{1.512108in}{1.954167in}}%
\pgfpathcurveto{\pgfqpoint{1.506284in}{1.948343in}}{\pgfqpoint{1.503012in}{1.940443in}}{\pgfqpoint{1.503012in}{1.932206in}}%
\pgfpathcurveto{\pgfqpoint{1.503012in}{1.923970in}}{\pgfqpoint{1.506284in}{1.916070in}}{\pgfqpoint{1.512108in}{1.910246in}}%
\pgfpathcurveto{\pgfqpoint{1.517932in}{1.904422in}}{\pgfqpoint{1.525832in}{1.901150in}}{\pgfqpoint{1.534068in}{1.901150in}}%
\pgfpathclose%
\pgfusepath{stroke,fill}%
\end{pgfscope}%
\begin{pgfscope}%
\pgfpathrectangle{\pgfqpoint{0.100000in}{0.212622in}}{\pgfqpoint{3.696000in}{3.696000in}}%
\pgfusepath{clip}%
\pgfsetbuttcap%
\pgfsetroundjoin%
\definecolor{currentfill}{rgb}{0.121569,0.466667,0.705882}%
\pgfsetfillcolor{currentfill}%
\pgfsetfillopacity{0.442910}%
\pgfsetlinewidth{1.003750pt}%
\definecolor{currentstroke}{rgb}{0.121569,0.466667,0.705882}%
\pgfsetstrokecolor{currentstroke}%
\pgfsetstrokeopacity{0.442910}%
\pgfsetdash{}{0pt}%
\pgfpathmoveto{\pgfqpoint{1.588537in}{1.940697in}}%
\pgfpathcurveto{\pgfqpoint{1.596773in}{1.940697in}}{\pgfqpoint{1.604673in}{1.943970in}}{\pgfqpoint{1.610497in}{1.949794in}}%
\pgfpathcurveto{\pgfqpoint{1.616321in}{1.955618in}}{\pgfqpoint{1.619593in}{1.963518in}}{\pgfqpoint{1.619593in}{1.971754in}}%
\pgfpathcurveto{\pgfqpoint{1.619593in}{1.979990in}}{\pgfqpoint{1.616321in}{1.987890in}}{\pgfqpoint{1.610497in}{1.993714in}}%
\pgfpathcurveto{\pgfqpoint{1.604673in}{1.999538in}}{\pgfqpoint{1.596773in}{2.002810in}}{\pgfqpoint{1.588537in}{2.002810in}}%
\pgfpathcurveto{\pgfqpoint{1.580300in}{2.002810in}}{\pgfqpoint{1.572400in}{1.999538in}}{\pgfqpoint{1.566576in}{1.993714in}}%
\pgfpathcurveto{\pgfqpoint{1.560752in}{1.987890in}}{\pgfqpoint{1.557480in}{1.979990in}}{\pgfqpoint{1.557480in}{1.971754in}}%
\pgfpathcurveto{\pgfqpoint{1.557480in}{1.963518in}}{\pgfqpoint{1.560752in}{1.955618in}}{\pgfqpoint{1.566576in}{1.949794in}}%
\pgfpathcurveto{\pgfqpoint{1.572400in}{1.943970in}}{\pgfqpoint{1.580300in}{1.940697in}}{\pgfqpoint{1.588537in}{1.940697in}}%
\pgfpathclose%
\pgfusepath{stroke,fill}%
\end{pgfscope}%
\begin{pgfscope}%
\pgfpathrectangle{\pgfqpoint{0.100000in}{0.212622in}}{\pgfqpoint{3.696000in}{3.696000in}}%
\pgfusepath{clip}%
\pgfsetbuttcap%
\pgfsetroundjoin%
\definecolor{currentfill}{rgb}{0.121569,0.466667,0.705882}%
\pgfsetfillcolor{currentfill}%
\pgfsetfillopacity{0.443186}%
\pgfsetlinewidth{1.003750pt}%
\definecolor{currentstroke}{rgb}{0.121569,0.466667,0.705882}%
\pgfsetstrokecolor{currentstroke}%
\pgfsetstrokeopacity{0.443186}%
\pgfsetdash{}{0pt}%
\pgfpathmoveto{\pgfqpoint{1.533616in}{1.900031in}}%
\pgfpathcurveto{\pgfqpoint{1.541852in}{1.900031in}}{\pgfqpoint{1.549752in}{1.903304in}}{\pgfqpoint{1.555576in}{1.909128in}}%
\pgfpathcurveto{\pgfqpoint{1.561400in}{1.914952in}}{\pgfqpoint{1.564673in}{1.922852in}}{\pgfqpoint{1.564673in}{1.931088in}}%
\pgfpathcurveto{\pgfqpoint{1.564673in}{1.939324in}}{\pgfqpoint{1.561400in}{1.947224in}}{\pgfqpoint{1.555576in}{1.953048in}}%
\pgfpathcurveto{\pgfqpoint{1.549752in}{1.958872in}}{\pgfqpoint{1.541852in}{1.962144in}}{\pgfqpoint{1.533616in}{1.962144in}}%
\pgfpathcurveto{\pgfqpoint{1.525380in}{1.962144in}}{\pgfqpoint{1.517480in}{1.958872in}}{\pgfqpoint{1.511656in}{1.953048in}}%
\pgfpathcurveto{\pgfqpoint{1.505832in}{1.947224in}}{\pgfqpoint{1.502560in}{1.939324in}}{\pgfqpoint{1.502560in}{1.931088in}}%
\pgfpathcurveto{\pgfqpoint{1.502560in}{1.922852in}}{\pgfqpoint{1.505832in}{1.914952in}}{\pgfqpoint{1.511656in}{1.909128in}}%
\pgfpathcurveto{\pgfqpoint{1.517480in}{1.903304in}}{\pgfqpoint{1.525380in}{1.900031in}}{\pgfqpoint{1.533616in}{1.900031in}}%
\pgfpathclose%
\pgfusepath{stroke,fill}%
\end{pgfscope}%
\begin{pgfscope}%
\pgfpathrectangle{\pgfqpoint{0.100000in}{0.212622in}}{\pgfqpoint{3.696000in}{3.696000in}}%
\pgfusepath{clip}%
\pgfsetbuttcap%
\pgfsetroundjoin%
\definecolor{currentfill}{rgb}{0.121569,0.466667,0.705882}%
\pgfsetfillcolor{currentfill}%
\pgfsetfillopacity{0.443256}%
\pgfsetlinewidth{1.003750pt}%
\definecolor{currentstroke}{rgb}{0.121569,0.466667,0.705882}%
\pgfsetstrokecolor{currentstroke}%
\pgfsetstrokeopacity{0.443256}%
\pgfsetdash{}{0pt}%
\pgfpathmoveto{\pgfqpoint{1.586624in}{1.938984in}}%
\pgfpathcurveto{\pgfqpoint{1.594861in}{1.938984in}}{\pgfqpoint{1.602761in}{1.942256in}}{\pgfqpoint{1.608585in}{1.948080in}}%
\pgfpathcurveto{\pgfqpoint{1.614409in}{1.953904in}}{\pgfqpoint{1.617681in}{1.961804in}}{\pgfqpoint{1.617681in}{1.970040in}}%
\pgfpathcurveto{\pgfqpoint{1.617681in}{1.978277in}}{\pgfqpoint{1.614409in}{1.986177in}}{\pgfqpoint{1.608585in}{1.992001in}}%
\pgfpathcurveto{\pgfqpoint{1.602761in}{1.997825in}}{\pgfqpoint{1.594861in}{2.001097in}}{\pgfqpoint{1.586624in}{2.001097in}}%
\pgfpathcurveto{\pgfqpoint{1.578388in}{2.001097in}}{\pgfqpoint{1.570488in}{1.997825in}}{\pgfqpoint{1.564664in}{1.992001in}}%
\pgfpathcurveto{\pgfqpoint{1.558840in}{1.986177in}}{\pgfqpoint{1.555568in}{1.978277in}}{\pgfqpoint{1.555568in}{1.970040in}}%
\pgfpathcurveto{\pgfqpoint{1.555568in}{1.961804in}}{\pgfqpoint{1.558840in}{1.953904in}}{\pgfqpoint{1.564664in}{1.948080in}}%
\pgfpathcurveto{\pgfqpoint{1.570488in}{1.942256in}}{\pgfqpoint{1.578388in}{1.938984in}}{\pgfqpoint{1.586624in}{1.938984in}}%
\pgfpathclose%
\pgfusepath{stroke,fill}%
\end{pgfscope}%
\begin{pgfscope}%
\pgfpathrectangle{\pgfqpoint{0.100000in}{0.212622in}}{\pgfqpoint{3.696000in}{3.696000in}}%
\pgfusepath{clip}%
\pgfsetbuttcap%
\pgfsetroundjoin%
\definecolor{currentfill}{rgb}{0.121569,0.466667,0.705882}%
\pgfsetfillcolor{currentfill}%
\pgfsetfillopacity{0.443266}%
\pgfsetlinewidth{1.003750pt}%
\definecolor{currentstroke}{rgb}{0.121569,0.466667,0.705882}%
\pgfsetstrokecolor{currentstroke}%
\pgfsetstrokeopacity{0.443266}%
\pgfsetdash{}{0pt}%
\pgfpathmoveto{\pgfqpoint{1.556968in}{1.920142in}}%
\pgfpathcurveto{\pgfqpoint{1.565204in}{1.920142in}}{\pgfqpoint{1.573104in}{1.923414in}}{\pgfqpoint{1.578928in}{1.929238in}}%
\pgfpathcurveto{\pgfqpoint{1.584752in}{1.935062in}}{\pgfqpoint{1.588024in}{1.942962in}}{\pgfqpoint{1.588024in}{1.951198in}}%
\pgfpathcurveto{\pgfqpoint{1.588024in}{1.959435in}}{\pgfqpoint{1.584752in}{1.967335in}}{\pgfqpoint{1.578928in}{1.973159in}}%
\pgfpathcurveto{\pgfqpoint{1.573104in}{1.978983in}}{\pgfqpoint{1.565204in}{1.982255in}}{\pgfqpoint{1.556968in}{1.982255in}}%
\pgfpathcurveto{\pgfqpoint{1.548732in}{1.982255in}}{\pgfqpoint{1.540832in}{1.978983in}}{\pgfqpoint{1.535008in}{1.973159in}}%
\pgfpathcurveto{\pgfqpoint{1.529184in}{1.967335in}}{\pgfqpoint{1.525911in}{1.959435in}}{\pgfqpoint{1.525911in}{1.951198in}}%
\pgfpathcurveto{\pgfqpoint{1.525911in}{1.942962in}}{\pgfqpoint{1.529184in}{1.935062in}}{\pgfqpoint{1.535008in}{1.929238in}}%
\pgfpathcurveto{\pgfqpoint{1.540832in}{1.923414in}}{\pgfqpoint{1.548732in}{1.920142in}}{\pgfqpoint{1.556968in}{1.920142in}}%
\pgfpathclose%
\pgfusepath{stroke,fill}%
\end{pgfscope}%
\begin{pgfscope}%
\pgfpathrectangle{\pgfqpoint{0.100000in}{0.212622in}}{\pgfqpoint{3.696000in}{3.696000in}}%
\pgfusepath{clip}%
\pgfsetbuttcap%
\pgfsetroundjoin%
\definecolor{currentfill}{rgb}{0.121569,0.466667,0.705882}%
\pgfsetfillcolor{currentfill}%
\pgfsetfillopacity{0.443308}%
\pgfsetlinewidth{1.003750pt}%
\definecolor{currentstroke}{rgb}{0.121569,0.466667,0.705882}%
\pgfsetstrokecolor{currentstroke}%
\pgfsetstrokeopacity{0.443308}%
\pgfsetdash{}{0pt}%
\pgfpathmoveto{\pgfqpoint{1.605709in}{1.946429in}}%
\pgfpathcurveto{\pgfqpoint{1.613946in}{1.946429in}}{\pgfqpoint{1.621846in}{1.949702in}}{\pgfqpoint{1.627670in}{1.955526in}}%
\pgfpathcurveto{\pgfqpoint{1.633494in}{1.961350in}}{\pgfqpoint{1.636766in}{1.969250in}}{\pgfqpoint{1.636766in}{1.977486in}}%
\pgfpathcurveto{\pgfqpoint{1.636766in}{1.985722in}}{\pgfqpoint{1.633494in}{1.993622in}}{\pgfqpoint{1.627670in}{1.999446in}}%
\pgfpathcurveto{\pgfqpoint{1.621846in}{2.005270in}}{\pgfqpoint{1.613946in}{2.008542in}}{\pgfqpoint{1.605709in}{2.008542in}}%
\pgfpathcurveto{\pgfqpoint{1.597473in}{2.008542in}}{\pgfqpoint{1.589573in}{2.005270in}}{\pgfqpoint{1.583749in}{1.999446in}}%
\pgfpathcurveto{\pgfqpoint{1.577925in}{1.993622in}}{\pgfqpoint{1.574653in}{1.985722in}}{\pgfqpoint{1.574653in}{1.977486in}}%
\pgfpathcurveto{\pgfqpoint{1.574653in}{1.969250in}}{\pgfqpoint{1.577925in}{1.961350in}}{\pgfqpoint{1.583749in}{1.955526in}}%
\pgfpathcurveto{\pgfqpoint{1.589573in}{1.949702in}}{\pgfqpoint{1.597473in}{1.946429in}}{\pgfqpoint{1.605709in}{1.946429in}}%
\pgfpathclose%
\pgfusepath{stroke,fill}%
\end{pgfscope}%
\begin{pgfscope}%
\pgfpathrectangle{\pgfqpoint{0.100000in}{0.212622in}}{\pgfqpoint{3.696000in}{3.696000in}}%
\pgfusepath{clip}%
\pgfsetbuttcap%
\pgfsetroundjoin%
\definecolor{currentfill}{rgb}{0.121569,0.466667,0.705882}%
\pgfsetfillcolor{currentfill}%
\pgfsetfillopacity{0.443350}%
\pgfsetlinewidth{1.003750pt}%
\definecolor{currentstroke}{rgb}{0.121569,0.466667,0.705882}%
\pgfsetstrokecolor{currentstroke}%
\pgfsetstrokeopacity{0.443350}%
\pgfsetdash{}{0pt}%
\pgfpathmoveto{\pgfqpoint{1.702409in}{2.001741in}}%
\pgfpathcurveto{\pgfqpoint{1.710645in}{2.001741in}}{\pgfqpoint{1.718545in}{2.005013in}}{\pgfqpoint{1.724369in}{2.010837in}}%
\pgfpathcurveto{\pgfqpoint{1.730193in}{2.016661in}}{\pgfqpoint{1.733465in}{2.024561in}}{\pgfqpoint{1.733465in}{2.032797in}}%
\pgfpathcurveto{\pgfqpoint{1.733465in}{2.041033in}}{\pgfqpoint{1.730193in}{2.048934in}}{\pgfqpoint{1.724369in}{2.054757in}}%
\pgfpathcurveto{\pgfqpoint{1.718545in}{2.060581in}}{\pgfqpoint{1.710645in}{2.063854in}}{\pgfqpoint{1.702409in}{2.063854in}}%
\pgfpathcurveto{\pgfqpoint{1.694172in}{2.063854in}}{\pgfqpoint{1.686272in}{2.060581in}}{\pgfqpoint{1.680448in}{2.054757in}}%
\pgfpathcurveto{\pgfqpoint{1.674625in}{2.048934in}}{\pgfqpoint{1.671352in}{2.041033in}}{\pgfqpoint{1.671352in}{2.032797in}}%
\pgfpathcurveto{\pgfqpoint{1.671352in}{2.024561in}}{\pgfqpoint{1.674625in}{2.016661in}}{\pgfqpoint{1.680448in}{2.010837in}}%
\pgfpathcurveto{\pgfqpoint{1.686272in}{2.005013in}}{\pgfqpoint{1.694172in}{2.001741in}}{\pgfqpoint{1.702409in}{2.001741in}}%
\pgfpathclose%
\pgfusepath{stroke,fill}%
\end{pgfscope}%
\begin{pgfscope}%
\pgfpathrectangle{\pgfqpoint{0.100000in}{0.212622in}}{\pgfqpoint{3.696000in}{3.696000in}}%
\pgfusepath{clip}%
\pgfsetbuttcap%
\pgfsetroundjoin%
\definecolor{currentfill}{rgb}{0.121569,0.466667,0.705882}%
\pgfsetfillcolor{currentfill}%
\pgfsetfillopacity{0.443465}%
\pgfsetlinewidth{1.003750pt}%
\definecolor{currentstroke}{rgb}{0.121569,0.466667,0.705882}%
\pgfsetstrokecolor{currentstroke}%
\pgfsetstrokeopacity{0.443465}%
\pgfsetdash{}{0pt}%
\pgfpathmoveto{\pgfqpoint{1.754292in}{2.033895in}}%
\pgfpathcurveto{\pgfqpoint{1.762529in}{2.033895in}}{\pgfqpoint{1.770429in}{2.037167in}}{\pgfqpoint{1.776253in}{2.042991in}}%
\pgfpathcurveto{\pgfqpoint{1.782077in}{2.048815in}}{\pgfqpoint{1.785349in}{2.056715in}}{\pgfqpoint{1.785349in}{2.064951in}}%
\pgfpathcurveto{\pgfqpoint{1.785349in}{2.073187in}}{\pgfqpoint{1.782077in}{2.081088in}}{\pgfqpoint{1.776253in}{2.086911in}}%
\pgfpathcurveto{\pgfqpoint{1.770429in}{2.092735in}}{\pgfqpoint{1.762529in}{2.096008in}}{\pgfqpoint{1.754292in}{2.096008in}}%
\pgfpathcurveto{\pgfqpoint{1.746056in}{2.096008in}}{\pgfqpoint{1.738156in}{2.092735in}}{\pgfqpoint{1.732332in}{2.086911in}}%
\pgfpathcurveto{\pgfqpoint{1.726508in}{2.081088in}}{\pgfqpoint{1.723236in}{2.073187in}}{\pgfqpoint{1.723236in}{2.064951in}}%
\pgfpathcurveto{\pgfqpoint{1.723236in}{2.056715in}}{\pgfqpoint{1.726508in}{2.048815in}}{\pgfqpoint{1.732332in}{2.042991in}}%
\pgfpathcurveto{\pgfqpoint{1.738156in}{2.037167in}}{\pgfqpoint{1.746056in}{2.033895in}}{\pgfqpoint{1.754292in}{2.033895in}}%
\pgfpathclose%
\pgfusepath{stroke,fill}%
\end{pgfscope}%
\begin{pgfscope}%
\pgfpathrectangle{\pgfqpoint{0.100000in}{0.212622in}}{\pgfqpoint{3.696000in}{3.696000in}}%
\pgfusepath{clip}%
\pgfsetbuttcap%
\pgfsetroundjoin%
\definecolor{currentfill}{rgb}{0.121569,0.466667,0.705882}%
\pgfsetfillcolor{currentfill}%
\pgfsetfillopacity{0.443489}%
\pgfsetlinewidth{1.003750pt}%
\definecolor{currentstroke}{rgb}{0.121569,0.466667,0.705882}%
\pgfsetstrokecolor{currentstroke}%
\pgfsetstrokeopacity{0.443489}%
\pgfsetdash{}{0pt}%
\pgfpathmoveto{\pgfqpoint{1.535159in}{1.901270in}}%
\pgfpathcurveto{\pgfqpoint{1.543395in}{1.901270in}}{\pgfqpoint{1.551295in}{1.904542in}}{\pgfqpoint{1.557119in}{1.910366in}}%
\pgfpathcurveto{\pgfqpoint{1.562943in}{1.916190in}}{\pgfqpoint{1.566216in}{1.924090in}}{\pgfqpoint{1.566216in}{1.932326in}}%
\pgfpathcurveto{\pgfqpoint{1.566216in}{1.940562in}}{\pgfqpoint{1.562943in}{1.948462in}}{\pgfqpoint{1.557119in}{1.954286in}}%
\pgfpathcurveto{\pgfqpoint{1.551295in}{1.960110in}}{\pgfqpoint{1.543395in}{1.963383in}}{\pgfqpoint{1.535159in}{1.963383in}}%
\pgfpathcurveto{\pgfqpoint{1.526923in}{1.963383in}}{\pgfqpoint{1.519023in}{1.960110in}}{\pgfqpoint{1.513199in}{1.954286in}}%
\pgfpathcurveto{\pgfqpoint{1.507375in}{1.948462in}}{\pgfqpoint{1.504103in}{1.940562in}}{\pgfqpoint{1.504103in}{1.932326in}}%
\pgfpathcurveto{\pgfqpoint{1.504103in}{1.924090in}}{\pgfqpoint{1.507375in}{1.916190in}}{\pgfqpoint{1.513199in}{1.910366in}}%
\pgfpathcurveto{\pgfqpoint{1.519023in}{1.904542in}}{\pgfqpoint{1.526923in}{1.901270in}}{\pgfqpoint{1.535159in}{1.901270in}}%
\pgfpathclose%
\pgfusepath{stroke,fill}%
\end{pgfscope}%
\begin{pgfscope}%
\pgfpathrectangle{\pgfqpoint{0.100000in}{0.212622in}}{\pgfqpoint{3.696000in}{3.696000in}}%
\pgfusepath{clip}%
\pgfsetbuttcap%
\pgfsetroundjoin%
\definecolor{currentfill}{rgb}{0.121569,0.466667,0.705882}%
\pgfsetfillcolor{currentfill}%
\pgfsetfillopacity{0.443970}%
\pgfsetlinewidth{1.003750pt}%
\definecolor{currentstroke}{rgb}{0.121569,0.466667,0.705882}%
\pgfsetstrokecolor{currentstroke}%
\pgfsetstrokeopacity{0.443970}%
\pgfsetdash{}{0pt}%
\pgfpathmoveto{\pgfqpoint{1.582911in}{1.935705in}}%
\pgfpathcurveto{\pgfqpoint{1.591147in}{1.935705in}}{\pgfqpoint{1.599048in}{1.938977in}}{\pgfqpoint{1.604871in}{1.944801in}}%
\pgfpathcurveto{\pgfqpoint{1.610695in}{1.950625in}}{\pgfqpoint{1.613968in}{1.958525in}}{\pgfqpoint{1.613968in}{1.966761in}}%
\pgfpathcurveto{\pgfqpoint{1.613968in}{1.974998in}}{\pgfqpoint{1.610695in}{1.982898in}}{\pgfqpoint{1.604871in}{1.988722in}}%
\pgfpathcurveto{\pgfqpoint{1.599048in}{1.994546in}}{\pgfqpoint{1.591147in}{1.997818in}}{\pgfqpoint{1.582911in}{1.997818in}}%
\pgfpathcurveto{\pgfqpoint{1.574675in}{1.997818in}}{\pgfqpoint{1.566775in}{1.994546in}}{\pgfqpoint{1.560951in}{1.988722in}}%
\pgfpathcurveto{\pgfqpoint{1.555127in}{1.982898in}}{\pgfqpoint{1.551855in}{1.974998in}}{\pgfqpoint{1.551855in}{1.966761in}}%
\pgfpathcurveto{\pgfqpoint{1.551855in}{1.958525in}}{\pgfqpoint{1.555127in}{1.950625in}}{\pgfqpoint{1.560951in}{1.944801in}}%
\pgfpathcurveto{\pgfqpoint{1.566775in}{1.938977in}}{\pgfqpoint{1.574675in}{1.935705in}}{\pgfqpoint{1.582911in}{1.935705in}}%
\pgfpathclose%
\pgfusepath{stroke,fill}%
\end{pgfscope}%
\begin{pgfscope}%
\pgfpathrectangle{\pgfqpoint{0.100000in}{0.212622in}}{\pgfqpoint{3.696000in}{3.696000in}}%
\pgfusepath{clip}%
\pgfsetbuttcap%
\pgfsetroundjoin%
\definecolor{currentfill}{rgb}{0.121569,0.466667,0.705882}%
\pgfsetfillcolor{currentfill}%
\pgfsetfillopacity{0.444025}%
\pgfsetlinewidth{1.003750pt}%
\definecolor{currentstroke}{rgb}{0.121569,0.466667,0.705882}%
\pgfsetstrokecolor{currentstroke}%
\pgfsetstrokeopacity{0.444025}%
\pgfsetdash{}{0pt}%
\pgfpathmoveto{\pgfqpoint{1.537217in}{1.902798in}}%
\pgfpathcurveto{\pgfqpoint{1.545453in}{1.902798in}}{\pgfqpoint{1.553353in}{1.906071in}}{\pgfqpoint{1.559177in}{1.911895in}}%
\pgfpathcurveto{\pgfqpoint{1.565001in}{1.917719in}}{\pgfqpoint{1.568273in}{1.925619in}}{\pgfqpoint{1.568273in}{1.933855in}}%
\pgfpathcurveto{\pgfqpoint{1.568273in}{1.942091in}}{\pgfqpoint{1.565001in}{1.949991in}}{\pgfqpoint{1.559177in}{1.955815in}}%
\pgfpathcurveto{\pgfqpoint{1.553353in}{1.961639in}}{\pgfqpoint{1.545453in}{1.964911in}}{\pgfqpoint{1.537217in}{1.964911in}}%
\pgfpathcurveto{\pgfqpoint{1.528981in}{1.964911in}}{\pgfqpoint{1.521081in}{1.961639in}}{\pgfqpoint{1.515257in}{1.955815in}}%
\pgfpathcurveto{\pgfqpoint{1.509433in}{1.949991in}}{\pgfqpoint{1.506160in}{1.942091in}}{\pgfqpoint{1.506160in}{1.933855in}}%
\pgfpathcurveto{\pgfqpoint{1.506160in}{1.925619in}}{\pgfqpoint{1.509433in}{1.917719in}}{\pgfqpoint{1.515257in}{1.911895in}}%
\pgfpathcurveto{\pgfqpoint{1.521081in}{1.906071in}}{\pgfqpoint{1.528981in}{1.902798in}}{\pgfqpoint{1.537217in}{1.902798in}}%
\pgfpathclose%
\pgfusepath{stroke,fill}%
\end{pgfscope}%
\begin{pgfscope}%
\pgfpathrectangle{\pgfqpoint{0.100000in}{0.212622in}}{\pgfqpoint{3.696000in}{3.696000in}}%
\pgfusepath{clip}%
\pgfsetbuttcap%
\pgfsetroundjoin%
\definecolor{currentfill}{rgb}{0.121569,0.466667,0.705882}%
\pgfsetfillcolor{currentfill}%
\pgfsetfillopacity{0.444597}%
\pgfsetlinewidth{1.003750pt}%
\definecolor{currentstroke}{rgb}{0.121569,0.466667,0.705882}%
\pgfsetstrokecolor{currentstroke}%
\pgfsetstrokeopacity{0.444597}%
\pgfsetdash{}{0pt}%
\pgfpathmoveto{\pgfqpoint{1.551123in}{1.915177in}}%
\pgfpathcurveto{\pgfqpoint{1.559359in}{1.915177in}}{\pgfqpoint{1.567259in}{1.918449in}}{\pgfqpoint{1.573083in}{1.924273in}}%
\pgfpathcurveto{\pgfqpoint{1.578907in}{1.930097in}}{\pgfqpoint{1.582179in}{1.937997in}}{\pgfqpoint{1.582179in}{1.946233in}}%
\pgfpathcurveto{\pgfqpoint{1.582179in}{1.954470in}}{\pgfqpoint{1.578907in}{1.962370in}}{\pgfqpoint{1.573083in}{1.968194in}}%
\pgfpathcurveto{\pgfqpoint{1.567259in}{1.974018in}}{\pgfqpoint{1.559359in}{1.977290in}}{\pgfqpoint{1.551123in}{1.977290in}}%
\pgfpathcurveto{\pgfqpoint{1.542887in}{1.977290in}}{\pgfqpoint{1.534987in}{1.974018in}}{\pgfqpoint{1.529163in}{1.968194in}}%
\pgfpathcurveto{\pgfqpoint{1.523339in}{1.962370in}}{\pgfqpoint{1.520066in}{1.954470in}}{\pgfqpoint{1.520066in}{1.946233in}}%
\pgfpathcurveto{\pgfqpoint{1.520066in}{1.937997in}}{\pgfqpoint{1.523339in}{1.930097in}}{\pgfqpoint{1.529163in}{1.924273in}}%
\pgfpathcurveto{\pgfqpoint{1.534987in}{1.918449in}}{\pgfqpoint{1.542887in}{1.915177in}}{\pgfqpoint{1.551123in}{1.915177in}}%
\pgfpathclose%
\pgfusepath{stroke,fill}%
\end{pgfscope}%
\begin{pgfscope}%
\pgfpathrectangle{\pgfqpoint{0.100000in}{0.212622in}}{\pgfqpoint{3.696000in}{3.696000in}}%
\pgfusepath{clip}%
\pgfsetbuttcap%
\pgfsetroundjoin%
\definecolor{currentfill}{rgb}{0.121569,0.466667,0.705882}%
\pgfsetfillcolor{currentfill}%
\pgfsetfillopacity{0.444621}%
\pgfsetlinewidth{1.003750pt}%
\definecolor{currentstroke}{rgb}{0.121569,0.466667,0.705882}%
\pgfsetstrokecolor{currentstroke}%
\pgfsetstrokeopacity{0.444621}%
\pgfsetdash{}{0pt}%
\pgfpathmoveto{\pgfqpoint{1.547286in}{1.911054in}}%
\pgfpathcurveto{\pgfqpoint{1.555522in}{1.911054in}}{\pgfqpoint{1.563423in}{1.914327in}}{\pgfqpoint{1.569246in}{1.920151in}}%
\pgfpathcurveto{\pgfqpoint{1.575070in}{1.925974in}}{\pgfqpoint{1.578343in}{1.933875in}}{\pgfqpoint{1.578343in}{1.942111in}}%
\pgfpathcurveto{\pgfqpoint{1.578343in}{1.950347in}}{\pgfqpoint{1.575070in}{1.958247in}}{\pgfqpoint{1.569246in}{1.964071in}}%
\pgfpathcurveto{\pgfqpoint{1.563423in}{1.969895in}}{\pgfqpoint{1.555522in}{1.973167in}}{\pgfqpoint{1.547286in}{1.973167in}}%
\pgfpathcurveto{\pgfqpoint{1.539050in}{1.973167in}}{\pgfqpoint{1.531150in}{1.969895in}}{\pgfqpoint{1.525326in}{1.964071in}}%
\pgfpathcurveto{\pgfqpoint{1.519502in}{1.958247in}}{\pgfqpoint{1.516230in}{1.950347in}}{\pgfqpoint{1.516230in}{1.942111in}}%
\pgfpathcurveto{\pgfqpoint{1.516230in}{1.933875in}}{\pgfqpoint{1.519502in}{1.925974in}}{\pgfqpoint{1.525326in}{1.920151in}}%
\pgfpathcurveto{\pgfqpoint{1.531150in}{1.914327in}}{\pgfqpoint{1.539050in}{1.911054in}}{\pgfqpoint{1.547286in}{1.911054in}}%
\pgfpathclose%
\pgfusepath{stroke,fill}%
\end{pgfscope}%
\begin{pgfscope}%
\pgfpathrectangle{\pgfqpoint{0.100000in}{0.212622in}}{\pgfqpoint{3.696000in}{3.696000in}}%
\pgfusepath{clip}%
\pgfsetbuttcap%
\pgfsetroundjoin%
\definecolor{currentfill}{rgb}{0.121569,0.466667,0.705882}%
\pgfsetfillcolor{currentfill}%
\pgfsetfillopacity{0.444765}%
\pgfsetlinewidth{1.003750pt}%
\definecolor{currentstroke}{rgb}{0.121569,0.466667,0.705882}%
\pgfsetstrokecolor{currentstroke}%
\pgfsetstrokeopacity{0.444765}%
\pgfsetdash{}{0pt}%
\pgfpathmoveto{\pgfqpoint{1.555006in}{1.917895in}}%
\pgfpathcurveto{\pgfqpoint{1.563242in}{1.917895in}}{\pgfqpoint{1.571142in}{1.921167in}}{\pgfqpoint{1.576966in}{1.926991in}}%
\pgfpathcurveto{\pgfqpoint{1.582790in}{1.932815in}}{\pgfqpoint{1.586062in}{1.940715in}}{\pgfqpoint{1.586062in}{1.948951in}}%
\pgfpathcurveto{\pgfqpoint{1.586062in}{1.957187in}}{\pgfqpoint{1.582790in}{1.965088in}}{\pgfqpoint{1.576966in}{1.970911in}}%
\pgfpathcurveto{\pgfqpoint{1.571142in}{1.976735in}}{\pgfqpoint{1.563242in}{1.980008in}}{\pgfqpoint{1.555006in}{1.980008in}}%
\pgfpathcurveto{\pgfqpoint{1.546770in}{1.980008in}}{\pgfqpoint{1.538869in}{1.976735in}}{\pgfqpoint{1.533046in}{1.970911in}}%
\pgfpathcurveto{\pgfqpoint{1.527222in}{1.965088in}}{\pgfqpoint{1.523949in}{1.957187in}}{\pgfqpoint{1.523949in}{1.948951in}}%
\pgfpathcurveto{\pgfqpoint{1.523949in}{1.940715in}}{\pgfqpoint{1.527222in}{1.932815in}}{\pgfqpoint{1.533046in}{1.926991in}}%
\pgfpathcurveto{\pgfqpoint{1.538869in}{1.921167in}}{\pgfqpoint{1.546770in}{1.917895in}}{\pgfqpoint{1.555006in}{1.917895in}}%
\pgfpathclose%
\pgfusepath{stroke,fill}%
\end{pgfscope}%
\begin{pgfscope}%
\pgfpathrectangle{\pgfqpoint{0.100000in}{0.212622in}}{\pgfqpoint{3.696000in}{3.696000in}}%
\pgfusepath{clip}%
\pgfsetbuttcap%
\pgfsetroundjoin%
\definecolor{currentfill}{rgb}{0.121569,0.466667,0.705882}%
\pgfsetfillcolor{currentfill}%
\pgfsetfillopacity{0.445086}%
\pgfsetlinewidth{1.003750pt}%
\definecolor{currentstroke}{rgb}{0.121569,0.466667,0.705882}%
\pgfsetstrokecolor{currentstroke}%
\pgfsetstrokeopacity{0.445086}%
\pgfsetdash{}{0pt}%
\pgfpathmoveto{\pgfqpoint{1.605622in}{1.941226in}}%
\pgfpathcurveto{\pgfqpoint{1.613858in}{1.941226in}}{\pgfqpoint{1.621758in}{1.944498in}}{\pgfqpoint{1.627582in}{1.950322in}}%
\pgfpathcurveto{\pgfqpoint{1.633406in}{1.956146in}}{\pgfqpoint{1.636678in}{1.964046in}}{\pgfqpoint{1.636678in}{1.972282in}}%
\pgfpathcurveto{\pgfqpoint{1.636678in}{1.980518in}}{\pgfqpoint{1.633406in}{1.988418in}}{\pgfqpoint{1.627582in}{1.994242in}}%
\pgfpathcurveto{\pgfqpoint{1.621758in}{2.000066in}}{\pgfqpoint{1.613858in}{2.003339in}}{\pgfqpoint{1.605622in}{2.003339in}}%
\pgfpathcurveto{\pgfqpoint{1.597386in}{2.003339in}}{\pgfqpoint{1.589486in}{2.000066in}}{\pgfqpoint{1.583662in}{1.994242in}}%
\pgfpathcurveto{\pgfqpoint{1.577838in}{1.988418in}}{\pgfqpoint{1.574565in}{1.980518in}}{\pgfqpoint{1.574565in}{1.972282in}}%
\pgfpathcurveto{\pgfqpoint{1.574565in}{1.964046in}}{\pgfqpoint{1.577838in}{1.956146in}}{\pgfqpoint{1.583662in}{1.950322in}}%
\pgfpathcurveto{\pgfqpoint{1.589486in}{1.944498in}}{\pgfqpoint{1.597386in}{1.941226in}}{\pgfqpoint{1.605622in}{1.941226in}}%
\pgfpathclose%
\pgfusepath{stroke,fill}%
\end{pgfscope}%
\begin{pgfscope}%
\pgfpathrectangle{\pgfqpoint{0.100000in}{0.212622in}}{\pgfqpoint{3.696000in}{3.696000in}}%
\pgfusepath{clip}%
\pgfsetbuttcap%
\pgfsetroundjoin%
\definecolor{currentfill}{rgb}{0.121569,0.466667,0.705882}%
\pgfsetfillcolor{currentfill}%
\pgfsetfillopacity{0.445176}%
\pgfsetlinewidth{1.003750pt}%
\definecolor{currentstroke}{rgb}{0.121569,0.466667,0.705882}%
\pgfsetstrokecolor{currentstroke}%
\pgfsetstrokeopacity{0.445176}%
\pgfsetdash{}{0pt}%
\pgfpathmoveto{\pgfqpoint{1.576752in}{1.931083in}}%
\pgfpathcurveto{\pgfqpoint{1.584988in}{1.931083in}}{\pgfqpoint{1.592888in}{1.934356in}}{\pgfqpoint{1.598712in}{1.940180in}}%
\pgfpathcurveto{\pgfqpoint{1.604536in}{1.946004in}}{\pgfqpoint{1.607808in}{1.953904in}}{\pgfqpoint{1.607808in}{1.962140in}}%
\pgfpathcurveto{\pgfqpoint{1.607808in}{1.970376in}}{\pgfqpoint{1.604536in}{1.978276in}}{\pgfqpoint{1.598712in}{1.984100in}}%
\pgfpathcurveto{\pgfqpoint{1.592888in}{1.989924in}}{\pgfqpoint{1.584988in}{1.993196in}}{\pgfqpoint{1.576752in}{1.993196in}}%
\pgfpathcurveto{\pgfqpoint{1.568515in}{1.993196in}}{\pgfqpoint{1.560615in}{1.989924in}}{\pgfqpoint{1.554791in}{1.984100in}}%
\pgfpathcurveto{\pgfqpoint{1.548967in}{1.978276in}}{\pgfqpoint{1.545695in}{1.970376in}}{\pgfqpoint{1.545695in}{1.962140in}}%
\pgfpathcurveto{\pgfqpoint{1.545695in}{1.953904in}}{\pgfqpoint{1.548967in}{1.946004in}}{\pgfqpoint{1.554791in}{1.940180in}}%
\pgfpathcurveto{\pgfqpoint{1.560615in}{1.934356in}}{\pgfqpoint{1.568515in}{1.931083in}}{\pgfqpoint{1.576752in}{1.931083in}}%
\pgfpathclose%
\pgfusepath{stroke,fill}%
\end{pgfscope}%
\begin{pgfscope}%
\pgfpathrectangle{\pgfqpoint{0.100000in}{0.212622in}}{\pgfqpoint{3.696000in}{3.696000in}}%
\pgfusepath{clip}%
\pgfsetbuttcap%
\pgfsetroundjoin%
\definecolor{currentfill}{rgb}{0.121569,0.466667,0.705882}%
\pgfsetfillcolor{currentfill}%
\pgfsetfillopacity{0.445311}%
\pgfsetlinewidth{1.003750pt}%
\definecolor{currentstroke}{rgb}{0.121569,0.466667,0.705882}%
\pgfsetstrokecolor{currentstroke}%
\pgfsetstrokeopacity{0.445311}%
\pgfsetdash{}{0pt}%
\pgfpathmoveto{\pgfqpoint{2.298723in}{2.303224in}}%
\pgfpathcurveto{\pgfqpoint{2.306960in}{2.303224in}}{\pgfqpoint{2.314860in}{2.306496in}}{\pgfqpoint{2.320684in}{2.312320in}}%
\pgfpathcurveto{\pgfqpoint{2.326508in}{2.318144in}}{\pgfqpoint{2.329780in}{2.326044in}}{\pgfqpoint{2.329780in}{2.334280in}}%
\pgfpathcurveto{\pgfqpoint{2.329780in}{2.342516in}}{\pgfqpoint{2.326508in}{2.350416in}}{\pgfqpoint{2.320684in}{2.356240in}}%
\pgfpathcurveto{\pgfqpoint{2.314860in}{2.362064in}}{\pgfqpoint{2.306960in}{2.365337in}}{\pgfqpoint{2.298723in}{2.365337in}}%
\pgfpathcurveto{\pgfqpoint{2.290487in}{2.365337in}}{\pgfqpoint{2.282587in}{2.362064in}}{\pgfqpoint{2.276763in}{2.356240in}}%
\pgfpathcurveto{\pgfqpoint{2.270939in}{2.350416in}}{\pgfqpoint{2.267667in}{2.342516in}}{\pgfqpoint{2.267667in}{2.334280in}}%
\pgfpathcurveto{\pgfqpoint{2.267667in}{2.326044in}}{\pgfqpoint{2.270939in}{2.318144in}}{\pgfqpoint{2.276763in}{2.312320in}}%
\pgfpathcurveto{\pgfqpoint{2.282587in}{2.306496in}}{\pgfqpoint{2.290487in}{2.303224in}}{\pgfqpoint{2.298723in}{2.303224in}}%
\pgfpathclose%
\pgfusepath{stroke,fill}%
\end{pgfscope}%
\begin{pgfscope}%
\pgfpathrectangle{\pgfqpoint{0.100000in}{0.212622in}}{\pgfqpoint{3.696000in}{3.696000in}}%
\pgfusepath{clip}%
\pgfsetbuttcap%
\pgfsetroundjoin%
\definecolor{currentfill}{rgb}{0.121569,0.466667,0.705882}%
\pgfsetfillcolor{currentfill}%
\pgfsetfillopacity{0.445680}%
\pgfsetlinewidth{1.003750pt}%
\definecolor{currentstroke}{rgb}{0.121569,0.466667,0.705882}%
\pgfsetstrokecolor{currentstroke}%
\pgfsetstrokeopacity{0.445680}%
\pgfsetdash{}{0pt}%
\pgfpathmoveto{\pgfqpoint{1.610236in}{1.941321in}}%
\pgfpathcurveto{\pgfqpoint{1.618472in}{1.941321in}}{\pgfqpoint{1.626372in}{1.944594in}}{\pgfqpoint{1.632196in}{1.950418in}}%
\pgfpathcurveto{\pgfqpoint{1.638020in}{1.956242in}}{\pgfqpoint{1.641293in}{1.964142in}}{\pgfqpoint{1.641293in}{1.972378in}}%
\pgfpathcurveto{\pgfqpoint{1.641293in}{1.980614in}}{\pgfqpoint{1.638020in}{1.988514in}}{\pgfqpoint{1.632196in}{1.994338in}}%
\pgfpathcurveto{\pgfqpoint{1.626372in}{2.000162in}}{\pgfqpoint{1.618472in}{2.003434in}}{\pgfqpoint{1.610236in}{2.003434in}}%
\pgfpathcurveto{\pgfqpoint{1.602000in}{2.003434in}}{\pgfqpoint{1.594100in}{2.000162in}}{\pgfqpoint{1.588276in}{1.994338in}}%
\pgfpathcurveto{\pgfqpoint{1.582452in}{1.988514in}}{\pgfqpoint{1.579180in}{1.980614in}}{\pgfqpoint{1.579180in}{1.972378in}}%
\pgfpathcurveto{\pgfqpoint{1.579180in}{1.964142in}}{\pgfqpoint{1.582452in}{1.956242in}}{\pgfqpoint{1.588276in}{1.950418in}}%
\pgfpathcurveto{\pgfqpoint{1.594100in}{1.944594in}}{\pgfqpoint{1.602000in}{1.941321in}}{\pgfqpoint{1.610236in}{1.941321in}}%
\pgfpathclose%
\pgfusepath{stroke,fill}%
\end{pgfscope}%
\begin{pgfscope}%
\pgfpathrectangle{\pgfqpoint{0.100000in}{0.212622in}}{\pgfqpoint{3.696000in}{3.696000in}}%
\pgfusepath{clip}%
\pgfsetbuttcap%
\pgfsetroundjoin%
\definecolor{currentfill}{rgb}{0.121569,0.466667,0.705882}%
\pgfsetfillcolor{currentfill}%
\pgfsetfillopacity{0.445884}%
\pgfsetlinewidth{1.003750pt}%
\definecolor{currentstroke}{rgb}{0.121569,0.466667,0.705882}%
\pgfsetstrokecolor{currentstroke}%
\pgfsetstrokeopacity{0.445884}%
\pgfsetdash{}{0pt}%
\pgfpathmoveto{\pgfqpoint{1.555052in}{1.916263in}}%
\pgfpathcurveto{\pgfqpoint{1.563288in}{1.916263in}}{\pgfqpoint{1.571188in}{1.919536in}}{\pgfqpoint{1.577012in}{1.925360in}}%
\pgfpathcurveto{\pgfqpoint{1.582836in}{1.931184in}}{\pgfqpoint{1.586109in}{1.939084in}}{\pgfqpoint{1.586109in}{1.947320in}}%
\pgfpathcurveto{\pgfqpoint{1.586109in}{1.955556in}}{\pgfqpoint{1.582836in}{1.963456in}}{\pgfqpoint{1.577012in}{1.969280in}}%
\pgfpathcurveto{\pgfqpoint{1.571188in}{1.975104in}}{\pgfqpoint{1.563288in}{1.978376in}}{\pgfqpoint{1.555052in}{1.978376in}}%
\pgfpathcurveto{\pgfqpoint{1.546816in}{1.978376in}}{\pgfqpoint{1.538916in}{1.975104in}}{\pgfqpoint{1.533092in}{1.969280in}}%
\pgfpathcurveto{\pgfqpoint{1.527268in}{1.963456in}}{\pgfqpoint{1.523996in}{1.955556in}}{\pgfqpoint{1.523996in}{1.947320in}}%
\pgfpathcurveto{\pgfqpoint{1.523996in}{1.939084in}}{\pgfqpoint{1.527268in}{1.931184in}}{\pgfqpoint{1.533092in}{1.925360in}}%
\pgfpathcurveto{\pgfqpoint{1.538916in}{1.919536in}}{\pgfqpoint{1.546816in}{1.916263in}}{\pgfqpoint{1.555052in}{1.916263in}}%
\pgfpathclose%
\pgfusepath{stroke,fill}%
\end{pgfscope}%
\begin{pgfscope}%
\pgfpathrectangle{\pgfqpoint{0.100000in}{0.212622in}}{\pgfqpoint{3.696000in}{3.696000in}}%
\pgfusepath{clip}%
\pgfsetbuttcap%
\pgfsetroundjoin%
\definecolor{currentfill}{rgb}{0.121569,0.466667,0.705882}%
\pgfsetfillcolor{currentfill}%
\pgfsetfillopacity{0.446556}%
\pgfsetlinewidth{1.003750pt}%
\definecolor{currentstroke}{rgb}{0.121569,0.466667,0.705882}%
\pgfsetstrokecolor{currentstroke}%
\pgfsetstrokeopacity{0.446556}%
\pgfsetdash{}{0pt}%
\pgfpathmoveto{\pgfqpoint{2.513146in}{2.464259in}}%
\pgfpathcurveto{\pgfqpoint{2.521382in}{2.464259in}}{\pgfqpoint{2.529282in}{2.467531in}}{\pgfqpoint{2.535106in}{2.473355in}}%
\pgfpathcurveto{\pgfqpoint{2.540930in}{2.479179in}}{\pgfqpoint{2.544202in}{2.487079in}}{\pgfqpoint{2.544202in}{2.495315in}}%
\pgfpathcurveto{\pgfqpoint{2.544202in}{2.503551in}}{\pgfqpoint{2.540930in}{2.511451in}}{\pgfqpoint{2.535106in}{2.517275in}}%
\pgfpathcurveto{\pgfqpoint{2.529282in}{2.523099in}}{\pgfqpoint{2.521382in}{2.526372in}}{\pgfqpoint{2.513146in}{2.526372in}}%
\pgfpathcurveto{\pgfqpoint{2.504909in}{2.526372in}}{\pgfqpoint{2.497009in}{2.523099in}}{\pgfqpoint{2.491185in}{2.517275in}}%
\pgfpathcurveto{\pgfqpoint{2.485361in}{2.511451in}}{\pgfqpoint{2.482089in}{2.503551in}}{\pgfqpoint{2.482089in}{2.495315in}}%
\pgfpathcurveto{\pgfqpoint{2.482089in}{2.487079in}}{\pgfqpoint{2.485361in}{2.479179in}}{\pgfqpoint{2.491185in}{2.473355in}}%
\pgfpathcurveto{\pgfqpoint{2.497009in}{2.467531in}}{\pgfqpoint{2.504909in}{2.464259in}}{\pgfqpoint{2.513146in}{2.464259in}}%
\pgfpathclose%
\pgfusepath{stroke,fill}%
\end{pgfscope}%
\begin{pgfscope}%
\pgfpathrectangle{\pgfqpoint{0.100000in}{0.212622in}}{\pgfqpoint{3.696000in}{3.696000in}}%
\pgfusepath{clip}%
\pgfsetbuttcap%
\pgfsetroundjoin%
\definecolor{currentfill}{rgb}{0.121569,0.466667,0.705882}%
\pgfsetfillcolor{currentfill}%
\pgfsetfillopacity{0.446593}%
\pgfsetlinewidth{1.003750pt}%
\definecolor{currentstroke}{rgb}{0.121569,0.466667,0.705882}%
\pgfsetstrokecolor{currentstroke}%
\pgfsetstrokeopacity{0.446593}%
\pgfsetdash{}{0pt}%
\pgfpathmoveto{\pgfqpoint{1.708065in}{2.001896in}}%
\pgfpathcurveto{\pgfqpoint{1.716301in}{2.001896in}}{\pgfqpoint{1.724202in}{2.005168in}}{\pgfqpoint{1.730025in}{2.010992in}}%
\pgfpathcurveto{\pgfqpoint{1.735849in}{2.016816in}}{\pgfqpoint{1.739122in}{2.024716in}}{\pgfqpoint{1.739122in}{2.032953in}}%
\pgfpathcurveto{\pgfqpoint{1.739122in}{2.041189in}}{\pgfqpoint{1.735849in}{2.049089in}}{\pgfqpoint{1.730025in}{2.054913in}}%
\pgfpathcurveto{\pgfqpoint{1.724202in}{2.060737in}}{\pgfqpoint{1.716301in}{2.064009in}}{\pgfqpoint{1.708065in}{2.064009in}}%
\pgfpathcurveto{\pgfqpoint{1.699829in}{2.064009in}}{\pgfqpoint{1.691929in}{2.060737in}}{\pgfqpoint{1.686105in}{2.054913in}}%
\pgfpathcurveto{\pgfqpoint{1.680281in}{2.049089in}}{\pgfqpoint{1.677009in}{2.041189in}}{\pgfqpoint{1.677009in}{2.032953in}}%
\pgfpathcurveto{\pgfqpoint{1.677009in}{2.024716in}}{\pgfqpoint{1.680281in}{2.016816in}}{\pgfqpoint{1.686105in}{2.010992in}}%
\pgfpathcurveto{\pgfqpoint{1.691929in}{2.005168in}}{\pgfqpoint{1.699829in}{2.001896in}}{\pgfqpoint{1.708065in}{2.001896in}}%
\pgfpathclose%
\pgfusepath{stroke,fill}%
\end{pgfscope}%
\begin{pgfscope}%
\pgfpathrectangle{\pgfqpoint{0.100000in}{0.212622in}}{\pgfqpoint{3.696000in}{3.696000in}}%
\pgfusepath{clip}%
\pgfsetbuttcap%
\pgfsetroundjoin%
\definecolor{currentfill}{rgb}{0.121569,0.466667,0.705882}%
\pgfsetfillcolor{currentfill}%
\pgfsetfillopacity{0.446726}%
\pgfsetlinewidth{1.003750pt}%
\definecolor{currentstroke}{rgb}{0.121569,0.466667,0.705882}%
\pgfsetstrokecolor{currentstroke}%
\pgfsetstrokeopacity{0.446726}%
\pgfsetdash{}{0pt}%
\pgfpathmoveto{\pgfqpoint{1.535779in}{1.903922in}}%
\pgfpathcurveto{\pgfqpoint{1.544015in}{1.903922in}}{\pgfqpoint{1.551915in}{1.907194in}}{\pgfqpoint{1.557739in}{1.913018in}}%
\pgfpathcurveto{\pgfqpoint{1.563563in}{1.918842in}}{\pgfqpoint{1.566835in}{1.926742in}}{\pgfqpoint{1.566835in}{1.934978in}}%
\pgfpathcurveto{\pgfqpoint{1.566835in}{1.943214in}}{\pgfqpoint{1.563563in}{1.951115in}}{\pgfqpoint{1.557739in}{1.956938in}}%
\pgfpathcurveto{\pgfqpoint{1.551915in}{1.962762in}}{\pgfqpoint{1.544015in}{1.966035in}}{\pgfqpoint{1.535779in}{1.966035in}}%
\pgfpathcurveto{\pgfqpoint{1.527543in}{1.966035in}}{\pgfqpoint{1.519643in}{1.962762in}}{\pgfqpoint{1.513819in}{1.956938in}}%
\pgfpathcurveto{\pgfqpoint{1.507995in}{1.951115in}}{\pgfqpoint{1.504722in}{1.943214in}}{\pgfqpoint{1.504722in}{1.934978in}}%
\pgfpathcurveto{\pgfqpoint{1.504722in}{1.926742in}}{\pgfqpoint{1.507995in}{1.918842in}}{\pgfqpoint{1.513819in}{1.913018in}}%
\pgfpathcurveto{\pgfqpoint{1.519643in}{1.907194in}}{\pgfqpoint{1.527543in}{1.903922in}}{\pgfqpoint{1.535779in}{1.903922in}}%
\pgfpathclose%
\pgfusepath{stroke,fill}%
\end{pgfscope}%
\begin{pgfscope}%
\pgfpathrectangle{\pgfqpoint{0.100000in}{0.212622in}}{\pgfqpoint{3.696000in}{3.696000in}}%
\pgfusepath{clip}%
\pgfsetbuttcap%
\pgfsetroundjoin%
\definecolor{currentfill}{rgb}{0.121569,0.466667,0.705882}%
\pgfsetfillcolor{currentfill}%
\pgfsetfillopacity{0.447235}%
\pgfsetlinewidth{1.003750pt}%
\definecolor{currentstroke}{rgb}{0.121569,0.466667,0.705882}%
\pgfsetstrokecolor{currentstroke}%
\pgfsetstrokeopacity{0.447235}%
\pgfsetdash{}{0pt}%
\pgfpathmoveto{\pgfqpoint{1.568690in}{1.924998in}}%
\pgfpathcurveto{\pgfqpoint{1.576926in}{1.924998in}}{\pgfqpoint{1.584826in}{1.928270in}}{\pgfqpoint{1.590650in}{1.934094in}}%
\pgfpathcurveto{\pgfqpoint{1.596474in}{1.939918in}}{\pgfqpoint{1.599746in}{1.947818in}}{\pgfqpoint{1.599746in}{1.956054in}}%
\pgfpathcurveto{\pgfqpoint{1.599746in}{1.964291in}}{\pgfqpoint{1.596474in}{1.972191in}}{\pgfqpoint{1.590650in}{1.978015in}}%
\pgfpathcurveto{\pgfqpoint{1.584826in}{1.983839in}}{\pgfqpoint{1.576926in}{1.987111in}}{\pgfqpoint{1.568690in}{1.987111in}}%
\pgfpathcurveto{\pgfqpoint{1.560454in}{1.987111in}}{\pgfqpoint{1.552553in}{1.983839in}}{\pgfqpoint{1.546730in}{1.978015in}}%
\pgfpathcurveto{\pgfqpoint{1.540906in}{1.972191in}}{\pgfqpoint{1.537633in}{1.964291in}}{\pgfqpoint{1.537633in}{1.956054in}}%
\pgfpathcurveto{\pgfqpoint{1.537633in}{1.947818in}}{\pgfqpoint{1.540906in}{1.939918in}}{\pgfqpoint{1.546730in}{1.934094in}}%
\pgfpathcurveto{\pgfqpoint{1.552553in}{1.928270in}}{\pgfqpoint{1.560454in}{1.924998in}}{\pgfqpoint{1.568690in}{1.924998in}}%
\pgfpathclose%
\pgfusepath{stroke,fill}%
\end{pgfscope}%
\begin{pgfscope}%
\pgfpathrectangle{\pgfqpoint{0.100000in}{0.212622in}}{\pgfqpoint{3.696000in}{3.696000in}}%
\pgfusepath{clip}%
\pgfsetbuttcap%
\pgfsetroundjoin%
\definecolor{currentfill}{rgb}{0.121569,0.466667,0.705882}%
\pgfsetfillcolor{currentfill}%
\pgfsetfillopacity{0.447427}%
\pgfsetlinewidth{1.003750pt}%
\definecolor{currentstroke}{rgb}{0.121569,0.466667,0.705882}%
\pgfsetstrokecolor{currentstroke}%
\pgfsetstrokeopacity{0.447427}%
\pgfsetdash{}{0pt}%
\pgfpathmoveto{\pgfqpoint{1.761984in}{2.042653in}}%
\pgfpathcurveto{\pgfqpoint{1.770220in}{2.042653in}}{\pgfqpoint{1.778120in}{2.045926in}}{\pgfqpoint{1.783944in}{2.051749in}}%
\pgfpathcurveto{\pgfqpoint{1.789768in}{2.057573in}}{\pgfqpoint{1.793040in}{2.065473in}}{\pgfqpoint{1.793040in}{2.073710in}}%
\pgfpathcurveto{\pgfqpoint{1.793040in}{2.081946in}}{\pgfqpoint{1.789768in}{2.089846in}}{\pgfqpoint{1.783944in}{2.095670in}}%
\pgfpathcurveto{\pgfqpoint{1.778120in}{2.101494in}}{\pgfqpoint{1.770220in}{2.104766in}}{\pgfqpoint{1.761984in}{2.104766in}}%
\pgfpathcurveto{\pgfqpoint{1.753747in}{2.104766in}}{\pgfqpoint{1.745847in}{2.101494in}}{\pgfqpoint{1.740023in}{2.095670in}}%
\pgfpathcurveto{\pgfqpoint{1.734199in}{2.089846in}}{\pgfqpoint{1.730927in}{2.081946in}}{\pgfqpoint{1.730927in}{2.073710in}}%
\pgfpathcurveto{\pgfqpoint{1.730927in}{2.065473in}}{\pgfqpoint{1.734199in}{2.057573in}}{\pgfqpoint{1.740023in}{2.051749in}}%
\pgfpathcurveto{\pgfqpoint{1.745847in}{2.045926in}}{\pgfqpoint{1.753747in}{2.042653in}}{\pgfqpoint{1.761984in}{2.042653in}}%
\pgfpathclose%
\pgfusepath{stroke,fill}%
\end{pgfscope}%
\begin{pgfscope}%
\pgfpathrectangle{\pgfqpoint{0.100000in}{0.212622in}}{\pgfqpoint{3.696000in}{3.696000in}}%
\pgfusepath{clip}%
\pgfsetbuttcap%
\pgfsetroundjoin%
\definecolor{currentfill}{rgb}{0.121569,0.466667,0.705882}%
\pgfsetfillcolor{currentfill}%
\pgfsetfillopacity{0.448147}%
\pgfsetlinewidth{1.003750pt}%
\definecolor{currentstroke}{rgb}{0.121569,0.466667,0.705882}%
\pgfsetstrokecolor{currentstroke}%
\pgfsetstrokeopacity{0.448147}%
\pgfsetdash{}{0pt}%
\pgfpathmoveto{\pgfqpoint{1.553122in}{1.913442in}}%
\pgfpathcurveto{\pgfqpoint{1.561358in}{1.913442in}}{\pgfqpoint{1.569258in}{1.916714in}}{\pgfqpoint{1.575082in}{1.922538in}}%
\pgfpathcurveto{\pgfqpoint{1.580906in}{1.928362in}}{\pgfqpoint{1.584178in}{1.936262in}}{\pgfqpoint{1.584178in}{1.944498in}}%
\pgfpathcurveto{\pgfqpoint{1.584178in}{1.952735in}}{\pgfqpoint{1.580906in}{1.960635in}}{\pgfqpoint{1.575082in}{1.966459in}}%
\pgfpathcurveto{\pgfqpoint{1.569258in}{1.972283in}}{\pgfqpoint{1.561358in}{1.975555in}}{\pgfqpoint{1.553122in}{1.975555in}}%
\pgfpathcurveto{\pgfqpoint{1.544885in}{1.975555in}}{\pgfqpoint{1.536985in}{1.972283in}}{\pgfqpoint{1.531161in}{1.966459in}}%
\pgfpathcurveto{\pgfqpoint{1.525338in}{1.960635in}}{\pgfqpoint{1.522065in}{1.952735in}}{\pgfqpoint{1.522065in}{1.944498in}}%
\pgfpathcurveto{\pgfqpoint{1.522065in}{1.936262in}}{\pgfqpoint{1.525338in}{1.928362in}}{\pgfqpoint{1.531161in}{1.922538in}}%
\pgfpathcurveto{\pgfqpoint{1.536985in}{1.916714in}}{\pgfqpoint{1.544885in}{1.913442in}}{\pgfqpoint{1.553122in}{1.913442in}}%
\pgfpathclose%
\pgfusepath{stroke,fill}%
\end{pgfscope}%
\begin{pgfscope}%
\pgfpathrectangle{\pgfqpoint{0.100000in}{0.212622in}}{\pgfqpoint{3.696000in}{3.696000in}}%
\pgfusepath{clip}%
\pgfsetbuttcap%
\pgfsetroundjoin%
\definecolor{currentfill}{rgb}{0.121569,0.466667,0.705882}%
\pgfsetfillcolor{currentfill}%
\pgfsetfillopacity{0.448568}%
\pgfsetlinewidth{1.003750pt}%
\definecolor{currentstroke}{rgb}{0.121569,0.466667,0.705882}%
\pgfsetstrokecolor{currentstroke}%
\pgfsetstrokeopacity{0.448568}%
\pgfsetdash{}{0pt}%
\pgfpathmoveto{\pgfqpoint{2.511461in}{2.461114in}}%
\pgfpathcurveto{\pgfqpoint{2.519697in}{2.461114in}}{\pgfqpoint{2.527597in}{2.464386in}}{\pgfqpoint{2.533421in}{2.470210in}}%
\pgfpathcurveto{\pgfqpoint{2.539245in}{2.476034in}}{\pgfqpoint{2.542517in}{2.483934in}}{\pgfqpoint{2.542517in}{2.492170in}}%
\pgfpathcurveto{\pgfqpoint{2.542517in}{2.500407in}}{\pgfqpoint{2.539245in}{2.508307in}}{\pgfqpoint{2.533421in}{2.514131in}}%
\pgfpathcurveto{\pgfqpoint{2.527597in}{2.519955in}}{\pgfqpoint{2.519697in}{2.523227in}}{\pgfqpoint{2.511461in}{2.523227in}}%
\pgfpathcurveto{\pgfqpoint{2.503224in}{2.523227in}}{\pgfqpoint{2.495324in}{2.519955in}}{\pgfqpoint{2.489500in}{2.514131in}}%
\pgfpathcurveto{\pgfqpoint{2.483677in}{2.508307in}}{\pgfqpoint{2.480404in}{2.500407in}}{\pgfqpoint{2.480404in}{2.492170in}}%
\pgfpathcurveto{\pgfqpoint{2.480404in}{2.483934in}}{\pgfqpoint{2.483677in}{2.476034in}}{\pgfqpoint{2.489500in}{2.470210in}}%
\pgfpathcurveto{\pgfqpoint{2.495324in}{2.464386in}}{\pgfqpoint{2.503224in}{2.461114in}}{\pgfqpoint{2.511461in}{2.461114in}}%
\pgfpathclose%
\pgfusepath{stroke,fill}%
\end{pgfscope}%
\begin{pgfscope}%
\pgfpathrectangle{\pgfqpoint{0.100000in}{0.212622in}}{\pgfqpoint{3.696000in}{3.696000in}}%
\pgfusepath{clip}%
\pgfsetbuttcap%
\pgfsetroundjoin%
\definecolor{currentfill}{rgb}{0.121569,0.466667,0.705882}%
\pgfsetfillcolor{currentfill}%
\pgfsetfillopacity{0.449306}%
\pgfsetlinewidth{1.003750pt}%
\definecolor{currentstroke}{rgb}{0.121569,0.466667,0.705882}%
\pgfsetstrokecolor{currentstroke}%
\pgfsetstrokeopacity{0.449306}%
\pgfsetdash{}{0pt}%
\pgfpathmoveto{\pgfqpoint{1.691964in}{1.990083in}}%
\pgfpathcurveto{\pgfqpoint{1.700200in}{1.990083in}}{\pgfqpoint{1.708100in}{1.993356in}}{\pgfqpoint{1.713924in}{1.999180in}}%
\pgfpathcurveto{\pgfqpoint{1.719748in}{2.005003in}}{\pgfqpoint{1.723021in}{2.012904in}}{\pgfqpoint{1.723021in}{2.021140in}}%
\pgfpathcurveto{\pgfqpoint{1.723021in}{2.029376in}}{\pgfqpoint{1.719748in}{2.037276in}}{\pgfqpoint{1.713924in}{2.043100in}}%
\pgfpathcurveto{\pgfqpoint{1.708100in}{2.048924in}}{\pgfqpoint{1.700200in}{2.052196in}}{\pgfqpoint{1.691964in}{2.052196in}}%
\pgfpathcurveto{\pgfqpoint{1.683728in}{2.052196in}}{\pgfqpoint{1.675828in}{2.048924in}}{\pgfqpoint{1.670004in}{2.043100in}}%
\pgfpathcurveto{\pgfqpoint{1.664180in}{2.037276in}}{\pgfqpoint{1.660908in}{2.029376in}}{\pgfqpoint{1.660908in}{2.021140in}}%
\pgfpathcurveto{\pgfqpoint{1.660908in}{2.012904in}}{\pgfqpoint{1.664180in}{2.005003in}}{\pgfqpoint{1.670004in}{1.999180in}}%
\pgfpathcurveto{\pgfqpoint{1.675828in}{1.993356in}}{\pgfqpoint{1.683728in}{1.990083in}}{\pgfqpoint{1.691964in}{1.990083in}}%
\pgfpathclose%
\pgfusepath{stroke,fill}%
\end{pgfscope}%
\begin{pgfscope}%
\pgfpathrectangle{\pgfqpoint{0.100000in}{0.212622in}}{\pgfqpoint{3.696000in}{3.696000in}}%
\pgfusepath{clip}%
\pgfsetbuttcap%
\pgfsetroundjoin%
\definecolor{currentfill}{rgb}{0.121569,0.466667,0.705882}%
\pgfsetfillcolor{currentfill}%
\pgfsetfillopacity{0.450294}%
\pgfsetlinewidth{1.003750pt}%
\definecolor{currentstroke}{rgb}{0.121569,0.466667,0.705882}%
\pgfsetstrokecolor{currentstroke}%
\pgfsetstrokeopacity{0.450294}%
\pgfsetdash{}{0pt}%
\pgfpathmoveto{\pgfqpoint{1.551489in}{1.908554in}}%
\pgfpathcurveto{\pgfqpoint{1.559725in}{1.908554in}}{\pgfqpoint{1.567626in}{1.911826in}}{\pgfqpoint{1.573449in}{1.917650in}}%
\pgfpathcurveto{\pgfqpoint{1.579273in}{1.923474in}}{\pgfqpoint{1.582546in}{1.931374in}}{\pgfqpoint{1.582546in}{1.939611in}}%
\pgfpathcurveto{\pgfqpoint{1.582546in}{1.947847in}}{\pgfqpoint{1.579273in}{1.955747in}}{\pgfqpoint{1.573449in}{1.961571in}}%
\pgfpathcurveto{\pgfqpoint{1.567626in}{1.967395in}}{\pgfqpoint{1.559725in}{1.970667in}}{\pgfqpoint{1.551489in}{1.970667in}}%
\pgfpathcurveto{\pgfqpoint{1.543253in}{1.970667in}}{\pgfqpoint{1.535353in}{1.967395in}}{\pgfqpoint{1.529529in}{1.961571in}}%
\pgfpathcurveto{\pgfqpoint{1.523705in}{1.955747in}}{\pgfqpoint{1.520433in}{1.947847in}}{\pgfqpoint{1.520433in}{1.939611in}}%
\pgfpathcurveto{\pgfqpoint{1.520433in}{1.931374in}}{\pgfqpoint{1.523705in}{1.923474in}}{\pgfqpoint{1.529529in}{1.917650in}}%
\pgfpathcurveto{\pgfqpoint{1.535353in}{1.911826in}}{\pgfqpoint{1.543253in}{1.908554in}}{\pgfqpoint{1.551489in}{1.908554in}}%
\pgfpathclose%
\pgfusepath{stroke,fill}%
\end{pgfscope}%
\begin{pgfscope}%
\pgfpathrectangle{\pgfqpoint{0.100000in}{0.212622in}}{\pgfqpoint{3.696000in}{3.696000in}}%
\pgfusepath{clip}%
\pgfsetbuttcap%
\pgfsetroundjoin%
\definecolor{currentfill}{rgb}{0.121569,0.466667,0.705882}%
\pgfsetfillcolor{currentfill}%
\pgfsetfillopacity{0.450737}%
\pgfsetlinewidth{1.003750pt}%
\definecolor{currentstroke}{rgb}{0.121569,0.466667,0.705882}%
\pgfsetstrokecolor{currentstroke}%
\pgfsetstrokeopacity{0.450737}%
\pgfsetdash{}{0pt}%
\pgfpathmoveto{\pgfqpoint{2.517116in}{2.464304in}}%
\pgfpathcurveto{\pgfqpoint{2.525353in}{2.464304in}}{\pgfqpoint{2.533253in}{2.467577in}}{\pgfqpoint{2.539077in}{2.473401in}}%
\pgfpathcurveto{\pgfqpoint{2.544901in}{2.479225in}}{\pgfqpoint{2.548173in}{2.487125in}}{\pgfqpoint{2.548173in}{2.495361in}}%
\pgfpathcurveto{\pgfqpoint{2.548173in}{2.503597in}}{\pgfqpoint{2.544901in}{2.511497in}}{\pgfqpoint{2.539077in}{2.517321in}}%
\pgfpathcurveto{\pgfqpoint{2.533253in}{2.523145in}}{\pgfqpoint{2.525353in}{2.526417in}}{\pgfqpoint{2.517116in}{2.526417in}}%
\pgfpathcurveto{\pgfqpoint{2.508880in}{2.526417in}}{\pgfqpoint{2.500980in}{2.523145in}}{\pgfqpoint{2.495156in}{2.517321in}}%
\pgfpathcurveto{\pgfqpoint{2.489332in}{2.511497in}}{\pgfqpoint{2.486060in}{2.503597in}}{\pgfqpoint{2.486060in}{2.495361in}}%
\pgfpathcurveto{\pgfqpoint{2.486060in}{2.487125in}}{\pgfqpoint{2.489332in}{2.479225in}}{\pgfqpoint{2.495156in}{2.473401in}}%
\pgfpathcurveto{\pgfqpoint{2.500980in}{2.467577in}}{\pgfqpoint{2.508880in}{2.464304in}}{\pgfqpoint{2.517116in}{2.464304in}}%
\pgfpathclose%
\pgfusepath{stroke,fill}%
\end{pgfscope}%
\begin{pgfscope}%
\pgfpathrectangle{\pgfqpoint{0.100000in}{0.212622in}}{\pgfqpoint{3.696000in}{3.696000in}}%
\pgfusepath{clip}%
\pgfsetbuttcap%
\pgfsetroundjoin%
\definecolor{currentfill}{rgb}{0.121569,0.466667,0.705882}%
\pgfsetfillcolor{currentfill}%
\pgfsetfillopacity{0.450857}%
\pgfsetlinewidth{1.003750pt}%
\definecolor{currentstroke}{rgb}{0.121569,0.466667,0.705882}%
\pgfsetstrokecolor{currentstroke}%
\pgfsetstrokeopacity{0.450857}%
\pgfsetdash{}{0pt}%
\pgfpathmoveto{\pgfqpoint{2.507130in}{2.456095in}}%
\pgfpathcurveto{\pgfqpoint{2.515366in}{2.456095in}}{\pgfqpoint{2.523266in}{2.459367in}}{\pgfqpoint{2.529090in}{2.465191in}}%
\pgfpathcurveto{\pgfqpoint{2.534914in}{2.471015in}}{\pgfqpoint{2.538186in}{2.478915in}}{\pgfqpoint{2.538186in}{2.487151in}}%
\pgfpathcurveto{\pgfqpoint{2.538186in}{2.495387in}}{\pgfqpoint{2.534914in}{2.503287in}}{\pgfqpoint{2.529090in}{2.509111in}}%
\pgfpathcurveto{\pgfqpoint{2.523266in}{2.514935in}}{\pgfqpoint{2.515366in}{2.518208in}}{\pgfqpoint{2.507130in}{2.518208in}}%
\pgfpathcurveto{\pgfqpoint{2.498893in}{2.518208in}}{\pgfqpoint{2.490993in}{2.514935in}}{\pgfqpoint{2.485169in}{2.509111in}}%
\pgfpathcurveto{\pgfqpoint{2.479345in}{2.503287in}}{\pgfqpoint{2.476073in}{2.495387in}}{\pgfqpoint{2.476073in}{2.487151in}}%
\pgfpathcurveto{\pgfqpoint{2.476073in}{2.478915in}}{\pgfqpoint{2.479345in}{2.471015in}}{\pgfqpoint{2.485169in}{2.465191in}}%
\pgfpathcurveto{\pgfqpoint{2.490993in}{2.459367in}}{\pgfqpoint{2.498893in}{2.456095in}}{\pgfqpoint{2.507130in}{2.456095in}}%
\pgfpathclose%
\pgfusepath{stroke,fill}%
\end{pgfscope}%
\begin{pgfscope}%
\pgfpathrectangle{\pgfqpoint{0.100000in}{0.212622in}}{\pgfqpoint{3.696000in}{3.696000in}}%
\pgfusepath{clip}%
\pgfsetbuttcap%
\pgfsetroundjoin%
\definecolor{currentfill}{rgb}{0.121569,0.466667,0.705882}%
\pgfsetfillcolor{currentfill}%
\pgfsetfillopacity{0.451196}%
\pgfsetlinewidth{1.003750pt}%
\definecolor{currentstroke}{rgb}{0.121569,0.466667,0.705882}%
\pgfsetstrokecolor{currentstroke}%
\pgfsetstrokeopacity{0.451196}%
\pgfsetdash{}{0pt}%
\pgfpathmoveto{\pgfqpoint{1.551888in}{1.908181in}}%
\pgfpathcurveto{\pgfqpoint{1.560124in}{1.908181in}}{\pgfqpoint{1.568024in}{1.911454in}}{\pgfqpoint{1.573848in}{1.917278in}}%
\pgfpathcurveto{\pgfqpoint{1.579672in}{1.923102in}}{\pgfqpoint{1.582944in}{1.931002in}}{\pgfqpoint{1.582944in}{1.939238in}}%
\pgfpathcurveto{\pgfqpoint{1.582944in}{1.947474in}}{\pgfqpoint{1.579672in}{1.955374in}}{\pgfqpoint{1.573848in}{1.961198in}}%
\pgfpathcurveto{\pgfqpoint{1.568024in}{1.967022in}}{\pgfqpoint{1.560124in}{1.970294in}}{\pgfqpoint{1.551888in}{1.970294in}}%
\pgfpathcurveto{\pgfqpoint{1.543651in}{1.970294in}}{\pgfqpoint{1.535751in}{1.967022in}}{\pgfqpoint{1.529927in}{1.961198in}}%
\pgfpathcurveto{\pgfqpoint{1.524104in}{1.955374in}}{\pgfqpoint{1.520831in}{1.947474in}}{\pgfqpoint{1.520831in}{1.939238in}}%
\pgfpathcurveto{\pgfqpoint{1.520831in}{1.931002in}}{\pgfqpoint{1.524104in}{1.923102in}}{\pgfqpoint{1.529927in}{1.917278in}}%
\pgfpathcurveto{\pgfqpoint{1.535751in}{1.911454in}}{\pgfqpoint{1.543651in}{1.908181in}}{\pgfqpoint{1.551888in}{1.908181in}}%
\pgfpathclose%
\pgfusepath{stroke,fill}%
\end{pgfscope}%
\begin{pgfscope}%
\pgfpathrectangle{\pgfqpoint{0.100000in}{0.212622in}}{\pgfqpoint{3.696000in}{3.696000in}}%
\pgfusepath{clip}%
\pgfsetbuttcap%
\pgfsetroundjoin%
\definecolor{currentfill}{rgb}{0.121569,0.466667,0.705882}%
\pgfsetfillcolor{currentfill}%
\pgfsetfillopacity{0.451361}%
\pgfsetlinewidth{1.003750pt}%
\definecolor{currentstroke}{rgb}{0.121569,0.466667,0.705882}%
\pgfsetstrokecolor{currentstroke}%
\pgfsetstrokeopacity{0.451361}%
\pgfsetdash{}{0pt}%
\pgfpathmoveto{\pgfqpoint{1.739998in}{2.021995in}}%
\pgfpathcurveto{\pgfqpoint{1.748234in}{2.021995in}}{\pgfqpoint{1.756134in}{2.025268in}}{\pgfqpoint{1.761958in}{2.031092in}}%
\pgfpathcurveto{\pgfqpoint{1.767782in}{2.036916in}}{\pgfqpoint{1.771055in}{2.044816in}}{\pgfqpoint{1.771055in}{2.053052in}}%
\pgfpathcurveto{\pgfqpoint{1.771055in}{2.061288in}}{\pgfqpoint{1.767782in}{2.069188in}}{\pgfqpoint{1.761958in}{2.075012in}}%
\pgfpathcurveto{\pgfqpoint{1.756134in}{2.080836in}}{\pgfqpoint{1.748234in}{2.084108in}}{\pgfqpoint{1.739998in}{2.084108in}}%
\pgfpathcurveto{\pgfqpoint{1.731762in}{2.084108in}}{\pgfqpoint{1.723862in}{2.080836in}}{\pgfqpoint{1.718038in}{2.075012in}}%
\pgfpathcurveto{\pgfqpoint{1.712214in}{2.069188in}}{\pgfqpoint{1.708942in}{2.061288in}}{\pgfqpoint{1.708942in}{2.053052in}}%
\pgfpathcurveto{\pgfqpoint{1.708942in}{2.044816in}}{\pgfqpoint{1.712214in}{2.036916in}}{\pgfqpoint{1.718038in}{2.031092in}}%
\pgfpathcurveto{\pgfqpoint{1.723862in}{2.025268in}}{\pgfqpoint{1.731762in}{2.021995in}}{\pgfqpoint{1.739998in}{2.021995in}}%
\pgfpathclose%
\pgfusepath{stroke,fill}%
\end{pgfscope}%
\begin{pgfscope}%
\pgfpathrectangle{\pgfqpoint{0.100000in}{0.212622in}}{\pgfqpoint{3.696000in}{3.696000in}}%
\pgfusepath{clip}%
\pgfsetbuttcap%
\pgfsetroundjoin%
\definecolor{currentfill}{rgb}{0.121569,0.466667,0.705882}%
\pgfsetfillcolor{currentfill}%
\pgfsetfillopacity{0.451423}%
\pgfsetlinewidth{1.003750pt}%
\definecolor{currentstroke}{rgb}{0.121569,0.466667,0.705882}%
\pgfsetstrokecolor{currentstroke}%
\pgfsetstrokeopacity{0.451423}%
\pgfsetdash{}{0pt}%
\pgfpathmoveto{\pgfqpoint{2.510204in}{2.459408in}}%
\pgfpathcurveto{\pgfqpoint{2.518441in}{2.459408in}}{\pgfqpoint{2.526341in}{2.462680in}}{\pgfqpoint{2.532165in}{2.468504in}}%
\pgfpathcurveto{\pgfqpoint{2.537989in}{2.474328in}}{\pgfqpoint{2.541261in}{2.482228in}}{\pgfqpoint{2.541261in}{2.490464in}}%
\pgfpathcurveto{\pgfqpoint{2.541261in}{2.498701in}}{\pgfqpoint{2.537989in}{2.506601in}}{\pgfqpoint{2.532165in}{2.512425in}}%
\pgfpathcurveto{\pgfqpoint{2.526341in}{2.518249in}}{\pgfqpoint{2.518441in}{2.521521in}}{\pgfqpoint{2.510204in}{2.521521in}}%
\pgfpathcurveto{\pgfqpoint{2.501968in}{2.521521in}}{\pgfqpoint{2.494068in}{2.518249in}}{\pgfqpoint{2.488244in}{2.512425in}}%
\pgfpathcurveto{\pgfqpoint{2.482420in}{2.506601in}}{\pgfqpoint{2.479148in}{2.498701in}}{\pgfqpoint{2.479148in}{2.490464in}}%
\pgfpathcurveto{\pgfqpoint{2.479148in}{2.482228in}}{\pgfqpoint{2.482420in}{2.474328in}}{\pgfqpoint{2.488244in}{2.468504in}}%
\pgfpathcurveto{\pgfqpoint{2.494068in}{2.462680in}}{\pgfqpoint{2.501968in}{2.459408in}}{\pgfqpoint{2.510204in}{2.459408in}}%
\pgfpathclose%
\pgfusepath{stroke,fill}%
\end{pgfscope}%
\begin{pgfscope}%
\pgfpathrectangle{\pgfqpoint{0.100000in}{0.212622in}}{\pgfqpoint{3.696000in}{3.696000in}}%
\pgfusepath{clip}%
\pgfsetbuttcap%
\pgfsetroundjoin%
\definecolor{currentfill}{rgb}{0.121569,0.466667,0.705882}%
\pgfsetfillcolor{currentfill}%
\pgfsetfillopacity{0.453232}%
\pgfsetlinewidth{1.003750pt}%
\definecolor{currentstroke}{rgb}{0.121569,0.466667,0.705882}%
\pgfsetstrokecolor{currentstroke}%
\pgfsetstrokeopacity{0.453232}%
\pgfsetdash{}{0pt}%
\pgfpathmoveto{\pgfqpoint{1.752701in}{2.027898in}}%
\pgfpathcurveto{\pgfqpoint{1.760937in}{2.027898in}}{\pgfqpoint{1.768837in}{2.031171in}}{\pgfqpoint{1.774661in}{2.036995in}}%
\pgfpathcurveto{\pgfqpoint{1.780485in}{2.042819in}}{\pgfqpoint{1.783757in}{2.050719in}}{\pgfqpoint{1.783757in}{2.058955in}}%
\pgfpathcurveto{\pgfqpoint{1.783757in}{2.067191in}}{\pgfqpoint{1.780485in}{2.075091in}}{\pgfqpoint{1.774661in}{2.080915in}}%
\pgfpathcurveto{\pgfqpoint{1.768837in}{2.086739in}}{\pgfqpoint{1.760937in}{2.090011in}}{\pgfqpoint{1.752701in}{2.090011in}}%
\pgfpathcurveto{\pgfqpoint{1.744464in}{2.090011in}}{\pgfqpoint{1.736564in}{2.086739in}}{\pgfqpoint{1.730740in}{2.080915in}}%
\pgfpathcurveto{\pgfqpoint{1.724916in}{2.075091in}}{\pgfqpoint{1.721644in}{2.067191in}}{\pgfqpoint{1.721644in}{2.058955in}}%
\pgfpathcurveto{\pgfqpoint{1.721644in}{2.050719in}}{\pgfqpoint{1.724916in}{2.042819in}}{\pgfqpoint{1.730740in}{2.036995in}}%
\pgfpathcurveto{\pgfqpoint{1.736564in}{2.031171in}}{\pgfqpoint{1.744464in}{2.027898in}}{\pgfqpoint{1.752701in}{2.027898in}}%
\pgfpathclose%
\pgfusepath{stroke,fill}%
\end{pgfscope}%
\begin{pgfscope}%
\pgfpathrectangle{\pgfqpoint{0.100000in}{0.212622in}}{\pgfqpoint{3.696000in}{3.696000in}}%
\pgfusepath{clip}%
\pgfsetbuttcap%
\pgfsetroundjoin%
\definecolor{currentfill}{rgb}{0.121569,0.466667,0.705882}%
\pgfsetfillcolor{currentfill}%
\pgfsetfillopacity{0.453675}%
\pgfsetlinewidth{1.003750pt}%
\definecolor{currentstroke}{rgb}{0.121569,0.466667,0.705882}%
\pgfsetstrokecolor{currentstroke}%
\pgfsetstrokeopacity{0.453675}%
\pgfsetdash{}{0pt}%
\pgfpathmoveto{\pgfqpoint{1.756336in}{2.029000in}}%
\pgfpathcurveto{\pgfqpoint{1.764573in}{2.029000in}}{\pgfqpoint{1.772473in}{2.032272in}}{\pgfqpoint{1.778297in}{2.038096in}}%
\pgfpathcurveto{\pgfqpoint{1.784121in}{2.043920in}}{\pgfqpoint{1.787393in}{2.051820in}}{\pgfqpoint{1.787393in}{2.060056in}}%
\pgfpathcurveto{\pgfqpoint{1.787393in}{2.068292in}}{\pgfqpoint{1.784121in}{2.076193in}}{\pgfqpoint{1.778297in}{2.082016in}}%
\pgfpathcurveto{\pgfqpoint{1.772473in}{2.087840in}}{\pgfqpoint{1.764573in}{2.091113in}}{\pgfqpoint{1.756336in}{2.091113in}}%
\pgfpathcurveto{\pgfqpoint{1.748100in}{2.091113in}}{\pgfqpoint{1.740200in}{2.087840in}}{\pgfqpoint{1.734376in}{2.082016in}}%
\pgfpathcurveto{\pgfqpoint{1.728552in}{2.076193in}}{\pgfqpoint{1.725280in}{2.068292in}}{\pgfqpoint{1.725280in}{2.060056in}}%
\pgfpathcurveto{\pgfqpoint{1.725280in}{2.051820in}}{\pgfqpoint{1.728552in}{2.043920in}}{\pgfqpoint{1.734376in}{2.038096in}}%
\pgfpathcurveto{\pgfqpoint{1.740200in}{2.032272in}}{\pgfqpoint{1.748100in}{2.029000in}}{\pgfqpoint{1.756336in}{2.029000in}}%
\pgfpathclose%
\pgfusepath{stroke,fill}%
\end{pgfscope}%
\begin{pgfscope}%
\pgfpathrectangle{\pgfqpoint{0.100000in}{0.212622in}}{\pgfqpoint{3.696000in}{3.696000in}}%
\pgfusepath{clip}%
\pgfsetbuttcap%
\pgfsetroundjoin%
\definecolor{currentfill}{rgb}{0.121569,0.466667,0.705882}%
\pgfsetfillcolor{currentfill}%
\pgfsetfillopacity{0.457383}%
\pgfsetlinewidth{1.003750pt}%
\definecolor{currentstroke}{rgb}{0.121569,0.466667,0.705882}%
\pgfsetstrokecolor{currentstroke}%
\pgfsetstrokeopacity{0.457383}%
\pgfsetdash{}{0pt}%
\pgfpathmoveto{\pgfqpoint{2.233944in}{2.246758in}}%
\pgfpathcurveto{\pgfqpoint{2.242181in}{2.246758in}}{\pgfqpoint{2.250081in}{2.250030in}}{\pgfqpoint{2.255905in}{2.255854in}}%
\pgfpathcurveto{\pgfqpoint{2.261729in}{2.261678in}}{\pgfqpoint{2.265001in}{2.269578in}}{\pgfqpoint{2.265001in}{2.277814in}}%
\pgfpathcurveto{\pgfqpoint{2.265001in}{2.286051in}}{\pgfqpoint{2.261729in}{2.293951in}}{\pgfqpoint{2.255905in}{2.299774in}}%
\pgfpathcurveto{\pgfqpoint{2.250081in}{2.305598in}}{\pgfqpoint{2.242181in}{2.308871in}}{\pgfqpoint{2.233944in}{2.308871in}}%
\pgfpathcurveto{\pgfqpoint{2.225708in}{2.308871in}}{\pgfqpoint{2.217808in}{2.305598in}}{\pgfqpoint{2.211984in}{2.299774in}}%
\pgfpathcurveto{\pgfqpoint{2.206160in}{2.293951in}}{\pgfqpoint{2.202888in}{2.286051in}}{\pgfqpoint{2.202888in}{2.277814in}}%
\pgfpathcurveto{\pgfqpoint{2.202888in}{2.269578in}}{\pgfqpoint{2.206160in}{2.261678in}}{\pgfqpoint{2.211984in}{2.255854in}}%
\pgfpathcurveto{\pgfqpoint{2.217808in}{2.250030in}}{\pgfqpoint{2.225708in}{2.246758in}}{\pgfqpoint{2.233944in}{2.246758in}}%
\pgfpathclose%
\pgfusepath{stroke,fill}%
\end{pgfscope}%
\begin{pgfscope}%
\pgfpathrectangle{\pgfqpoint{0.100000in}{0.212622in}}{\pgfqpoint{3.696000in}{3.696000in}}%
\pgfusepath{clip}%
\pgfsetbuttcap%
\pgfsetroundjoin%
\definecolor{currentfill}{rgb}{0.121569,0.466667,0.705882}%
\pgfsetfillcolor{currentfill}%
\pgfsetfillopacity{0.457640}%
\pgfsetlinewidth{1.003750pt}%
\definecolor{currentstroke}{rgb}{0.121569,0.466667,0.705882}%
\pgfsetstrokecolor{currentstroke}%
\pgfsetstrokeopacity{0.457640}%
\pgfsetdash{}{0pt}%
\pgfpathmoveto{\pgfqpoint{2.509547in}{2.454174in}}%
\pgfpathcurveto{\pgfqpoint{2.517784in}{2.454174in}}{\pgfqpoint{2.525684in}{2.457446in}}{\pgfqpoint{2.531508in}{2.463270in}}%
\pgfpathcurveto{\pgfqpoint{2.537332in}{2.469094in}}{\pgfqpoint{2.540604in}{2.476994in}}{\pgfqpoint{2.540604in}{2.485230in}}%
\pgfpathcurveto{\pgfqpoint{2.540604in}{2.493467in}}{\pgfqpoint{2.537332in}{2.501367in}}{\pgfqpoint{2.531508in}{2.507191in}}%
\pgfpathcurveto{\pgfqpoint{2.525684in}{2.513015in}}{\pgfqpoint{2.517784in}{2.516287in}}{\pgfqpoint{2.509547in}{2.516287in}}%
\pgfpathcurveto{\pgfqpoint{2.501311in}{2.516287in}}{\pgfqpoint{2.493411in}{2.513015in}}{\pgfqpoint{2.487587in}{2.507191in}}%
\pgfpathcurveto{\pgfqpoint{2.481763in}{2.501367in}}{\pgfqpoint{2.478491in}{2.493467in}}{\pgfqpoint{2.478491in}{2.485230in}}%
\pgfpathcurveto{\pgfqpoint{2.478491in}{2.476994in}}{\pgfqpoint{2.481763in}{2.469094in}}{\pgfqpoint{2.487587in}{2.463270in}}%
\pgfpathcurveto{\pgfqpoint{2.493411in}{2.457446in}}{\pgfqpoint{2.501311in}{2.454174in}}{\pgfqpoint{2.509547in}{2.454174in}}%
\pgfpathclose%
\pgfusepath{stroke,fill}%
\end{pgfscope}%
\begin{pgfscope}%
\pgfpathrectangle{\pgfqpoint{0.100000in}{0.212622in}}{\pgfqpoint{3.696000in}{3.696000in}}%
\pgfusepath{clip}%
\pgfsetbuttcap%
\pgfsetroundjoin%
\definecolor{currentfill}{rgb}{0.121569,0.466667,0.705882}%
\pgfsetfillcolor{currentfill}%
\pgfsetfillopacity{0.457738}%
\pgfsetlinewidth{1.003750pt}%
\definecolor{currentstroke}{rgb}{0.121569,0.466667,0.705882}%
\pgfsetstrokecolor{currentstroke}%
\pgfsetstrokeopacity{0.457738}%
\pgfsetdash{}{0pt}%
\pgfpathmoveto{\pgfqpoint{2.502935in}{2.449805in}}%
\pgfpathcurveto{\pgfqpoint{2.511172in}{2.449805in}}{\pgfqpoint{2.519072in}{2.453077in}}{\pgfqpoint{2.524896in}{2.458901in}}%
\pgfpathcurveto{\pgfqpoint{2.530720in}{2.464725in}}{\pgfqpoint{2.533992in}{2.472625in}}{\pgfqpoint{2.533992in}{2.480862in}}%
\pgfpathcurveto{\pgfqpoint{2.533992in}{2.489098in}}{\pgfqpoint{2.530720in}{2.496998in}}{\pgfqpoint{2.524896in}{2.502822in}}%
\pgfpathcurveto{\pgfqpoint{2.519072in}{2.508646in}}{\pgfqpoint{2.511172in}{2.511918in}}{\pgfqpoint{2.502935in}{2.511918in}}%
\pgfpathcurveto{\pgfqpoint{2.494699in}{2.511918in}}{\pgfqpoint{2.486799in}{2.508646in}}{\pgfqpoint{2.480975in}{2.502822in}}%
\pgfpathcurveto{\pgfqpoint{2.475151in}{2.496998in}}{\pgfqpoint{2.471879in}{2.489098in}}{\pgfqpoint{2.471879in}{2.480862in}}%
\pgfpathcurveto{\pgfqpoint{2.471879in}{2.472625in}}{\pgfqpoint{2.475151in}{2.464725in}}{\pgfqpoint{2.480975in}{2.458901in}}%
\pgfpathcurveto{\pgfqpoint{2.486799in}{2.453077in}}{\pgfqpoint{2.494699in}{2.449805in}}{\pgfqpoint{2.502935in}{2.449805in}}%
\pgfpathclose%
\pgfusepath{stroke,fill}%
\end{pgfscope}%
\begin{pgfscope}%
\pgfpathrectangle{\pgfqpoint{0.100000in}{0.212622in}}{\pgfqpoint{3.696000in}{3.696000in}}%
\pgfusepath{clip}%
\pgfsetbuttcap%
\pgfsetroundjoin%
\definecolor{currentfill}{rgb}{0.121569,0.466667,0.705882}%
\pgfsetfillcolor{currentfill}%
\pgfsetfillopacity{0.458053}%
\pgfsetlinewidth{1.003750pt}%
\definecolor{currentstroke}{rgb}{0.121569,0.466667,0.705882}%
\pgfsetstrokecolor{currentstroke}%
\pgfsetstrokeopacity{0.458053}%
\pgfsetdash{}{0pt}%
\pgfpathmoveto{\pgfqpoint{2.515561in}{2.458365in}}%
\pgfpathcurveto{\pgfqpoint{2.523797in}{2.458365in}}{\pgfqpoint{2.531697in}{2.461637in}}{\pgfqpoint{2.537521in}{2.467461in}}%
\pgfpathcurveto{\pgfqpoint{2.543345in}{2.473285in}}{\pgfqpoint{2.546617in}{2.481185in}}{\pgfqpoint{2.546617in}{2.489421in}}%
\pgfpathcurveto{\pgfqpoint{2.546617in}{2.497658in}}{\pgfqpoint{2.543345in}{2.505558in}}{\pgfqpoint{2.537521in}{2.511382in}}%
\pgfpathcurveto{\pgfqpoint{2.531697in}{2.517206in}}{\pgfqpoint{2.523797in}{2.520478in}}{\pgfqpoint{2.515561in}{2.520478in}}%
\pgfpathcurveto{\pgfqpoint{2.507324in}{2.520478in}}{\pgfqpoint{2.499424in}{2.517206in}}{\pgfqpoint{2.493600in}{2.511382in}}%
\pgfpathcurveto{\pgfqpoint{2.487776in}{2.505558in}}{\pgfqpoint{2.484504in}{2.497658in}}{\pgfqpoint{2.484504in}{2.489421in}}%
\pgfpathcurveto{\pgfqpoint{2.484504in}{2.481185in}}{\pgfqpoint{2.487776in}{2.473285in}}{\pgfqpoint{2.493600in}{2.467461in}}%
\pgfpathcurveto{\pgfqpoint{2.499424in}{2.461637in}}{\pgfqpoint{2.507324in}{2.458365in}}{\pgfqpoint{2.515561in}{2.458365in}}%
\pgfpathclose%
\pgfusepath{stroke,fill}%
\end{pgfscope}%
\begin{pgfscope}%
\pgfpathrectangle{\pgfqpoint{0.100000in}{0.212622in}}{\pgfqpoint{3.696000in}{3.696000in}}%
\pgfusepath{clip}%
\pgfsetbuttcap%
\pgfsetroundjoin%
\definecolor{currentfill}{rgb}{0.121569,0.466667,0.705882}%
\pgfsetfillcolor{currentfill}%
\pgfsetfillopacity{0.458911}%
\pgfsetlinewidth{1.003750pt}%
\definecolor{currentstroke}{rgb}{0.121569,0.466667,0.705882}%
\pgfsetstrokecolor{currentstroke}%
\pgfsetstrokeopacity{0.458911}%
\pgfsetdash{}{0pt}%
\pgfpathmoveto{\pgfqpoint{1.746741in}{2.019720in}}%
\pgfpathcurveto{\pgfqpoint{1.754977in}{2.019720in}}{\pgfqpoint{1.762877in}{2.022992in}}{\pgfqpoint{1.768701in}{2.028816in}}%
\pgfpathcurveto{\pgfqpoint{1.774525in}{2.034640in}}{\pgfqpoint{1.777797in}{2.042540in}}{\pgfqpoint{1.777797in}{2.050776in}}%
\pgfpathcurveto{\pgfqpoint{1.777797in}{2.059012in}}{\pgfqpoint{1.774525in}{2.066912in}}{\pgfqpoint{1.768701in}{2.072736in}}%
\pgfpathcurveto{\pgfqpoint{1.762877in}{2.078560in}}{\pgfqpoint{1.754977in}{2.081833in}}{\pgfqpoint{1.746741in}{2.081833in}}%
\pgfpathcurveto{\pgfqpoint{1.738504in}{2.081833in}}{\pgfqpoint{1.730604in}{2.078560in}}{\pgfqpoint{1.724780in}{2.072736in}}%
\pgfpathcurveto{\pgfqpoint{1.718956in}{2.066912in}}{\pgfqpoint{1.715684in}{2.059012in}}{\pgfqpoint{1.715684in}{2.050776in}}%
\pgfpathcurveto{\pgfqpoint{1.715684in}{2.042540in}}{\pgfqpoint{1.718956in}{2.034640in}}{\pgfqpoint{1.724780in}{2.028816in}}%
\pgfpathcurveto{\pgfqpoint{1.730604in}{2.022992in}}{\pgfqpoint{1.738504in}{2.019720in}}{\pgfqpoint{1.746741in}{2.019720in}}%
\pgfpathclose%
\pgfusepath{stroke,fill}%
\end{pgfscope}%
\begin{pgfscope}%
\pgfpathrectangle{\pgfqpoint{0.100000in}{0.212622in}}{\pgfqpoint{3.696000in}{3.696000in}}%
\pgfusepath{clip}%
\pgfsetbuttcap%
\pgfsetroundjoin%
\definecolor{currentfill}{rgb}{0.121569,0.466667,0.705882}%
\pgfsetfillcolor{currentfill}%
\pgfsetfillopacity{0.460147}%
\pgfsetlinewidth{1.003750pt}%
\definecolor{currentstroke}{rgb}{0.121569,0.466667,0.705882}%
\pgfsetstrokecolor{currentstroke}%
\pgfsetstrokeopacity{0.460147}%
\pgfsetdash{}{0pt}%
\pgfpathmoveto{\pgfqpoint{2.518833in}{2.459990in}}%
\pgfpathcurveto{\pgfqpoint{2.527069in}{2.459990in}}{\pgfqpoint{2.534969in}{2.463262in}}{\pgfqpoint{2.540793in}{2.469086in}}%
\pgfpathcurveto{\pgfqpoint{2.546617in}{2.474910in}}{\pgfqpoint{2.549889in}{2.482810in}}{\pgfqpoint{2.549889in}{2.491046in}}%
\pgfpathcurveto{\pgfqpoint{2.549889in}{2.499282in}}{\pgfqpoint{2.546617in}{2.507182in}}{\pgfqpoint{2.540793in}{2.513006in}}%
\pgfpathcurveto{\pgfqpoint{2.534969in}{2.518830in}}{\pgfqpoint{2.527069in}{2.522103in}}{\pgfqpoint{2.518833in}{2.522103in}}%
\pgfpathcurveto{\pgfqpoint{2.510597in}{2.522103in}}{\pgfqpoint{2.502697in}{2.518830in}}{\pgfqpoint{2.496873in}{2.513006in}}%
\pgfpathcurveto{\pgfqpoint{2.491049in}{2.507182in}}{\pgfqpoint{2.487776in}{2.499282in}}{\pgfqpoint{2.487776in}{2.491046in}}%
\pgfpathcurveto{\pgfqpoint{2.487776in}{2.482810in}}{\pgfqpoint{2.491049in}{2.474910in}}{\pgfqpoint{2.496873in}{2.469086in}}%
\pgfpathcurveto{\pgfqpoint{2.502697in}{2.463262in}}{\pgfqpoint{2.510597in}{2.459990in}}{\pgfqpoint{2.518833in}{2.459990in}}%
\pgfpathclose%
\pgfusepath{stroke,fill}%
\end{pgfscope}%
\begin{pgfscope}%
\pgfpathrectangle{\pgfqpoint{0.100000in}{0.212622in}}{\pgfqpoint{3.696000in}{3.696000in}}%
\pgfusepath{clip}%
\pgfsetbuttcap%
\pgfsetroundjoin%
\definecolor{currentfill}{rgb}{0.121569,0.466667,0.705882}%
\pgfsetfillcolor{currentfill}%
\pgfsetfillopacity{0.460160}%
\pgfsetlinewidth{1.003750pt}%
\definecolor{currentstroke}{rgb}{0.121569,0.466667,0.705882}%
\pgfsetstrokecolor{currentstroke}%
\pgfsetstrokeopacity{0.460160}%
\pgfsetdash{}{0pt}%
\pgfpathmoveto{\pgfqpoint{1.745605in}{2.016467in}}%
\pgfpathcurveto{\pgfqpoint{1.753841in}{2.016467in}}{\pgfqpoint{1.761741in}{2.019740in}}{\pgfqpoint{1.767565in}{2.025564in}}%
\pgfpathcurveto{\pgfqpoint{1.773389in}{2.031387in}}{\pgfqpoint{1.776661in}{2.039287in}}{\pgfqpoint{1.776661in}{2.047524in}}%
\pgfpathcurveto{\pgfqpoint{1.776661in}{2.055760in}}{\pgfqpoint{1.773389in}{2.063660in}}{\pgfqpoint{1.767565in}{2.069484in}}%
\pgfpathcurveto{\pgfqpoint{1.761741in}{2.075308in}}{\pgfqpoint{1.753841in}{2.078580in}}{\pgfqpoint{1.745605in}{2.078580in}}%
\pgfpathcurveto{\pgfqpoint{1.737368in}{2.078580in}}{\pgfqpoint{1.729468in}{2.075308in}}{\pgfqpoint{1.723644in}{2.069484in}}%
\pgfpathcurveto{\pgfqpoint{1.717821in}{2.063660in}}{\pgfqpoint{1.714548in}{2.055760in}}{\pgfqpoint{1.714548in}{2.047524in}}%
\pgfpathcurveto{\pgfqpoint{1.714548in}{2.039287in}}{\pgfqpoint{1.717821in}{2.031387in}}{\pgfqpoint{1.723644in}{2.025564in}}%
\pgfpathcurveto{\pgfqpoint{1.729468in}{2.019740in}}{\pgfqpoint{1.737368in}{2.016467in}}{\pgfqpoint{1.745605in}{2.016467in}}%
\pgfpathclose%
\pgfusepath{stroke,fill}%
\end{pgfscope}%
\begin{pgfscope}%
\pgfpathrectangle{\pgfqpoint{0.100000in}{0.212622in}}{\pgfqpoint{3.696000in}{3.696000in}}%
\pgfusepath{clip}%
\pgfsetbuttcap%
\pgfsetroundjoin%
\definecolor{currentfill}{rgb}{0.121569,0.466667,0.705882}%
\pgfsetfillcolor{currentfill}%
\pgfsetfillopacity{0.460592}%
\pgfsetlinewidth{1.003750pt}%
\definecolor{currentstroke}{rgb}{0.121569,0.466667,0.705882}%
\pgfsetstrokecolor{currentstroke}%
\pgfsetstrokeopacity{0.460592}%
\pgfsetdash{}{0pt}%
\pgfpathmoveto{\pgfqpoint{2.511900in}{2.453091in}}%
\pgfpathcurveto{\pgfqpoint{2.520136in}{2.453091in}}{\pgfqpoint{2.528036in}{2.456364in}}{\pgfqpoint{2.533860in}{2.462188in}}%
\pgfpathcurveto{\pgfqpoint{2.539684in}{2.468011in}}{\pgfqpoint{2.542956in}{2.475912in}}{\pgfqpoint{2.542956in}{2.484148in}}%
\pgfpathcurveto{\pgfqpoint{2.542956in}{2.492384in}}{\pgfqpoint{2.539684in}{2.500284in}}{\pgfqpoint{2.533860in}{2.506108in}}%
\pgfpathcurveto{\pgfqpoint{2.528036in}{2.511932in}}{\pgfqpoint{2.520136in}{2.515204in}}{\pgfqpoint{2.511900in}{2.515204in}}%
\pgfpathcurveto{\pgfqpoint{2.503664in}{2.515204in}}{\pgfqpoint{2.495764in}{2.511932in}}{\pgfqpoint{2.489940in}{2.506108in}}%
\pgfpathcurveto{\pgfqpoint{2.484116in}{2.500284in}}{\pgfqpoint{2.480843in}{2.492384in}}{\pgfqpoint{2.480843in}{2.484148in}}%
\pgfpathcurveto{\pgfqpoint{2.480843in}{2.475912in}}{\pgfqpoint{2.484116in}{2.468011in}}{\pgfqpoint{2.489940in}{2.462188in}}%
\pgfpathcurveto{\pgfqpoint{2.495764in}{2.456364in}}{\pgfqpoint{2.503664in}{2.453091in}}{\pgfqpoint{2.511900in}{2.453091in}}%
\pgfpathclose%
\pgfusepath{stroke,fill}%
\end{pgfscope}%
\begin{pgfscope}%
\pgfpathrectangle{\pgfqpoint{0.100000in}{0.212622in}}{\pgfqpoint{3.696000in}{3.696000in}}%
\pgfusepath{clip}%
\pgfsetbuttcap%
\pgfsetroundjoin%
\definecolor{currentfill}{rgb}{0.121569,0.466667,0.705882}%
\pgfsetfillcolor{currentfill}%
\pgfsetfillopacity{0.462577}%
\pgfsetlinewidth{1.003750pt}%
\definecolor{currentstroke}{rgb}{0.121569,0.466667,0.705882}%
\pgfsetstrokecolor{currentstroke}%
\pgfsetstrokeopacity{0.462577}%
\pgfsetdash{}{0pt}%
\pgfpathmoveto{\pgfqpoint{1.742078in}{2.012671in}}%
\pgfpathcurveto{\pgfqpoint{1.750315in}{2.012671in}}{\pgfqpoint{1.758215in}{2.015943in}}{\pgfqpoint{1.764039in}{2.021767in}}%
\pgfpathcurveto{\pgfqpoint{1.769863in}{2.027591in}}{\pgfqpoint{1.773135in}{2.035491in}}{\pgfqpoint{1.773135in}{2.043727in}}%
\pgfpathcurveto{\pgfqpoint{1.773135in}{2.051964in}}{\pgfqpoint{1.769863in}{2.059864in}}{\pgfqpoint{1.764039in}{2.065688in}}%
\pgfpathcurveto{\pgfqpoint{1.758215in}{2.071512in}}{\pgfqpoint{1.750315in}{2.074784in}}{\pgfqpoint{1.742078in}{2.074784in}}%
\pgfpathcurveto{\pgfqpoint{1.733842in}{2.074784in}}{\pgfqpoint{1.725942in}{2.071512in}}{\pgfqpoint{1.720118in}{2.065688in}}%
\pgfpathcurveto{\pgfqpoint{1.714294in}{2.059864in}}{\pgfqpoint{1.711022in}{2.051964in}}{\pgfqpoint{1.711022in}{2.043727in}}%
\pgfpathcurveto{\pgfqpoint{1.711022in}{2.035491in}}{\pgfqpoint{1.714294in}{2.027591in}}{\pgfqpoint{1.720118in}{2.021767in}}%
\pgfpathcurveto{\pgfqpoint{1.725942in}{2.015943in}}{\pgfqpoint{1.733842in}{2.012671in}}{\pgfqpoint{1.742078in}{2.012671in}}%
\pgfpathclose%
\pgfusepath{stroke,fill}%
\end{pgfscope}%
\begin{pgfscope}%
\pgfpathrectangle{\pgfqpoint{0.100000in}{0.212622in}}{\pgfqpoint{3.696000in}{3.696000in}}%
\pgfusepath{clip}%
\pgfsetbuttcap%
\pgfsetroundjoin%
\definecolor{currentfill}{rgb}{0.121569,0.466667,0.705882}%
\pgfsetfillcolor{currentfill}%
\pgfsetfillopacity{0.462686}%
\pgfsetlinewidth{1.003750pt}%
\definecolor{currentstroke}{rgb}{0.121569,0.466667,0.705882}%
\pgfsetstrokecolor{currentstroke}%
\pgfsetstrokeopacity{0.462686}%
\pgfsetdash{}{0pt}%
\pgfpathmoveto{\pgfqpoint{2.517294in}{2.456962in}}%
\pgfpathcurveto{\pgfqpoint{2.525530in}{2.456962in}}{\pgfqpoint{2.533430in}{2.460235in}}{\pgfqpoint{2.539254in}{2.466059in}}%
\pgfpathcurveto{\pgfqpoint{2.545078in}{2.471883in}}{\pgfqpoint{2.548350in}{2.479783in}}{\pgfqpoint{2.548350in}{2.488019in}}%
\pgfpathcurveto{\pgfqpoint{2.548350in}{2.496255in}}{\pgfqpoint{2.545078in}{2.504155in}}{\pgfqpoint{2.539254in}{2.509979in}}%
\pgfpathcurveto{\pgfqpoint{2.533430in}{2.515803in}}{\pgfqpoint{2.525530in}{2.519075in}}{\pgfqpoint{2.517294in}{2.519075in}}%
\pgfpathcurveto{\pgfqpoint{2.509057in}{2.519075in}}{\pgfqpoint{2.501157in}{2.515803in}}{\pgfqpoint{2.495333in}{2.509979in}}%
\pgfpathcurveto{\pgfqpoint{2.489510in}{2.504155in}}{\pgfqpoint{2.486237in}{2.496255in}}{\pgfqpoint{2.486237in}{2.488019in}}%
\pgfpathcurveto{\pgfqpoint{2.486237in}{2.479783in}}{\pgfqpoint{2.489510in}{2.471883in}}{\pgfqpoint{2.495333in}{2.466059in}}%
\pgfpathcurveto{\pgfqpoint{2.501157in}{2.460235in}}{\pgfqpoint{2.509057in}{2.456962in}}{\pgfqpoint{2.517294in}{2.456962in}}%
\pgfpathclose%
\pgfusepath{stroke,fill}%
\end{pgfscope}%
\begin{pgfscope}%
\pgfpathrectangle{\pgfqpoint{0.100000in}{0.212622in}}{\pgfqpoint{3.696000in}{3.696000in}}%
\pgfusepath{clip}%
\pgfsetbuttcap%
\pgfsetroundjoin%
\definecolor{currentfill}{rgb}{0.121569,0.466667,0.705882}%
\pgfsetfillcolor{currentfill}%
\pgfsetfillopacity{0.465523}%
\pgfsetlinewidth{1.003750pt}%
\definecolor{currentstroke}{rgb}{0.121569,0.466667,0.705882}%
\pgfsetstrokecolor{currentstroke}%
\pgfsetstrokeopacity{0.465523}%
\pgfsetdash{}{0pt}%
\pgfpathmoveto{\pgfqpoint{1.735038in}{2.005304in}}%
\pgfpathcurveto{\pgfqpoint{1.743274in}{2.005304in}}{\pgfqpoint{1.751174in}{2.008576in}}{\pgfqpoint{1.756998in}{2.014400in}}%
\pgfpathcurveto{\pgfqpoint{1.762822in}{2.020224in}}{\pgfqpoint{1.766095in}{2.028124in}}{\pgfqpoint{1.766095in}{2.036360in}}%
\pgfpathcurveto{\pgfqpoint{1.766095in}{2.044597in}}{\pgfqpoint{1.762822in}{2.052497in}}{\pgfqpoint{1.756998in}{2.058321in}}%
\pgfpathcurveto{\pgfqpoint{1.751174in}{2.064145in}}{\pgfqpoint{1.743274in}{2.067417in}}{\pgfqpoint{1.735038in}{2.067417in}}%
\pgfpathcurveto{\pgfqpoint{1.726802in}{2.067417in}}{\pgfqpoint{1.718902in}{2.064145in}}{\pgfqpoint{1.713078in}{2.058321in}}%
\pgfpathcurveto{\pgfqpoint{1.707254in}{2.052497in}}{\pgfqpoint{1.703982in}{2.044597in}}{\pgfqpoint{1.703982in}{2.036360in}}%
\pgfpathcurveto{\pgfqpoint{1.703982in}{2.028124in}}{\pgfqpoint{1.707254in}{2.020224in}}{\pgfqpoint{1.713078in}{2.014400in}}%
\pgfpathcurveto{\pgfqpoint{1.718902in}{2.008576in}}{\pgfqpoint{1.726802in}{2.005304in}}{\pgfqpoint{1.735038in}{2.005304in}}%
\pgfpathclose%
\pgfusepath{stroke,fill}%
\end{pgfscope}%
\begin{pgfscope}%
\pgfpathrectangle{\pgfqpoint{0.100000in}{0.212622in}}{\pgfqpoint{3.696000in}{3.696000in}}%
\pgfusepath{clip}%
\pgfsetbuttcap%
\pgfsetroundjoin%
\definecolor{currentfill}{rgb}{0.121569,0.466667,0.705882}%
\pgfsetfillcolor{currentfill}%
\pgfsetfillopacity{0.467076}%
\pgfsetlinewidth{1.003750pt}%
\definecolor{currentstroke}{rgb}{0.121569,0.466667,0.705882}%
\pgfsetstrokecolor{currentstroke}%
\pgfsetstrokeopacity{0.467076}%
\pgfsetdash{}{0pt}%
\pgfpathmoveto{\pgfqpoint{2.511497in}{2.449981in}}%
\pgfpathcurveto{\pgfqpoint{2.519733in}{2.449981in}}{\pgfqpoint{2.527633in}{2.453253in}}{\pgfqpoint{2.533457in}{2.459077in}}%
\pgfpathcurveto{\pgfqpoint{2.539281in}{2.464901in}}{\pgfqpoint{2.542553in}{2.472801in}}{\pgfqpoint{2.542553in}{2.481037in}}%
\pgfpathcurveto{\pgfqpoint{2.542553in}{2.489274in}}{\pgfqpoint{2.539281in}{2.497174in}}{\pgfqpoint{2.533457in}{2.502997in}}%
\pgfpathcurveto{\pgfqpoint{2.527633in}{2.508821in}}{\pgfqpoint{2.519733in}{2.512094in}}{\pgfqpoint{2.511497in}{2.512094in}}%
\pgfpathcurveto{\pgfqpoint{2.503261in}{2.512094in}}{\pgfqpoint{2.495361in}{2.508821in}}{\pgfqpoint{2.489537in}{2.502997in}}%
\pgfpathcurveto{\pgfqpoint{2.483713in}{2.497174in}}{\pgfqpoint{2.480440in}{2.489274in}}{\pgfqpoint{2.480440in}{2.481037in}}%
\pgfpathcurveto{\pgfqpoint{2.480440in}{2.472801in}}{\pgfqpoint{2.483713in}{2.464901in}}{\pgfqpoint{2.489537in}{2.459077in}}%
\pgfpathcurveto{\pgfqpoint{2.495361in}{2.453253in}}{\pgfqpoint{2.503261in}{2.449981in}}{\pgfqpoint{2.511497in}{2.449981in}}%
\pgfpathclose%
\pgfusepath{stroke,fill}%
\end{pgfscope}%
\begin{pgfscope}%
\pgfpathrectangle{\pgfqpoint{0.100000in}{0.212622in}}{\pgfqpoint{3.696000in}{3.696000in}}%
\pgfusepath{clip}%
\pgfsetbuttcap%
\pgfsetroundjoin%
\definecolor{currentfill}{rgb}{0.121569,0.466667,0.705882}%
\pgfsetfillcolor{currentfill}%
\pgfsetfillopacity{0.467493}%
\pgfsetlinewidth{1.003750pt}%
\definecolor{currentstroke}{rgb}{0.121569,0.466667,0.705882}%
\pgfsetstrokecolor{currentstroke}%
\pgfsetstrokeopacity{0.467493}%
\pgfsetdash{}{0pt}%
\pgfpathmoveto{\pgfqpoint{1.731405in}{2.000392in}}%
\pgfpathcurveto{\pgfqpoint{1.739641in}{2.000392in}}{\pgfqpoint{1.747541in}{2.003664in}}{\pgfqpoint{1.753365in}{2.009488in}}%
\pgfpathcurveto{\pgfqpoint{1.759189in}{2.015312in}}{\pgfqpoint{1.762461in}{2.023212in}}{\pgfqpoint{1.762461in}{2.031448in}}%
\pgfpathcurveto{\pgfqpoint{1.762461in}{2.039684in}}{\pgfqpoint{1.759189in}{2.047584in}}{\pgfqpoint{1.753365in}{2.053408in}}%
\pgfpathcurveto{\pgfqpoint{1.747541in}{2.059232in}}{\pgfqpoint{1.739641in}{2.062505in}}{\pgfqpoint{1.731405in}{2.062505in}}%
\pgfpathcurveto{\pgfqpoint{1.723168in}{2.062505in}}{\pgfqpoint{1.715268in}{2.059232in}}{\pgfqpoint{1.709444in}{2.053408in}}%
\pgfpathcurveto{\pgfqpoint{1.703621in}{2.047584in}}{\pgfqpoint{1.700348in}{2.039684in}}{\pgfqpoint{1.700348in}{2.031448in}}%
\pgfpathcurveto{\pgfqpoint{1.700348in}{2.023212in}}{\pgfqpoint{1.703621in}{2.015312in}}{\pgfqpoint{1.709444in}{2.009488in}}%
\pgfpathcurveto{\pgfqpoint{1.715268in}{2.003664in}}{\pgfqpoint{1.723168in}{2.000392in}}{\pgfqpoint{1.731405in}{2.000392in}}%
\pgfpathclose%
\pgfusepath{stroke,fill}%
\end{pgfscope}%
\begin{pgfscope}%
\pgfpathrectangle{\pgfqpoint{0.100000in}{0.212622in}}{\pgfqpoint{3.696000in}{3.696000in}}%
\pgfusepath{clip}%
\pgfsetbuttcap%
\pgfsetroundjoin%
\definecolor{currentfill}{rgb}{0.121569,0.466667,0.705882}%
\pgfsetfillcolor{currentfill}%
\pgfsetfillopacity{0.468910}%
\pgfsetlinewidth{1.003750pt}%
\definecolor{currentstroke}{rgb}{0.121569,0.466667,0.705882}%
\pgfsetstrokecolor{currentstroke}%
\pgfsetstrokeopacity{0.468910}%
\pgfsetdash{}{0pt}%
\pgfpathmoveto{\pgfqpoint{2.521254in}{2.451570in}}%
\pgfpathcurveto{\pgfqpoint{2.529491in}{2.451570in}}{\pgfqpoint{2.537391in}{2.454842in}}{\pgfqpoint{2.543215in}{2.460666in}}%
\pgfpathcurveto{\pgfqpoint{2.549039in}{2.466490in}}{\pgfqpoint{2.552311in}{2.474390in}}{\pgfqpoint{2.552311in}{2.482626in}}%
\pgfpathcurveto{\pgfqpoint{2.552311in}{2.490862in}}{\pgfqpoint{2.549039in}{2.498762in}}{\pgfqpoint{2.543215in}{2.504586in}}%
\pgfpathcurveto{\pgfqpoint{2.537391in}{2.510410in}}{\pgfqpoint{2.529491in}{2.513683in}}{\pgfqpoint{2.521254in}{2.513683in}}%
\pgfpathcurveto{\pgfqpoint{2.513018in}{2.513683in}}{\pgfqpoint{2.505118in}{2.510410in}}{\pgfqpoint{2.499294in}{2.504586in}}%
\pgfpathcurveto{\pgfqpoint{2.493470in}{2.498762in}}{\pgfqpoint{2.490198in}{2.490862in}}{\pgfqpoint{2.490198in}{2.482626in}}%
\pgfpathcurveto{\pgfqpoint{2.490198in}{2.474390in}}{\pgfqpoint{2.493470in}{2.466490in}}{\pgfqpoint{2.499294in}{2.460666in}}%
\pgfpathcurveto{\pgfqpoint{2.505118in}{2.454842in}}{\pgfqpoint{2.513018in}{2.451570in}}{\pgfqpoint{2.521254in}{2.451570in}}%
\pgfpathclose%
\pgfusepath{stroke,fill}%
\end{pgfscope}%
\begin{pgfscope}%
\pgfpathrectangle{\pgfqpoint{0.100000in}{0.212622in}}{\pgfqpoint{3.696000in}{3.696000in}}%
\pgfusepath{clip}%
\pgfsetbuttcap%
\pgfsetroundjoin%
\definecolor{currentfill}{rgb}{0.121569,0.466667,0.705882}%
\pgfsetfillcolor{currentfill}%
\pgfsetfillopacity{0.469278}%
\pgfsetlinewidth{1.003750pt}%
\definecolor{currentstroke}{rgb}{0.121569,0.466667,0.705882}%
\pgfsetstrokecolor{currentstroke}%
\pgfsetstrokeopacity{0.469278}%
\pgfsetdash{}{0pt}%
\pgfpathmoveto{\pgfqpoint{2.512071in}{2.446366in}}%
\pgfpathcurveto{\pgfqpoint{2.520307in}{2.446366in}}{\pgfqpoint{2.528207in}{2.449639in}}{\pgfqpoint{2.534031in}{2.455462in}}%
\pgfpathcurveto{\pgfqpoint{2.539855in}{2.461286in}}{\pgfqpoint{2.543128in}{2.469186in}}{\pgfqpoint{2.543128in}{2.477423in}}%
\pgfpathcurveto{\pgfqpoint{2.543128in}{2.485659in}}{\pgfqpoint{2.539855in}{2.493559in}}{\pgfqpoint{2.534031in}{2.499383in}}%
\pgfpathcurveto{\pgfqpoint{2.528207in}{2.505207in}}{\pgfqpoint{2.520307in}{2.508479in}}{\pgfqpoint{2.512071in}{2.508479in}}%
\pgfpathcurveto{\pgfqpoint{2.503835in}{2.508479in}}{\pgfqpoint{2.495935in}{2.505207in}}{\pgfqpoint{2.490111in}{2.499383in}}%
\pgfpathcurveto{\pgfqpoint{2.484287in}{2.493559in}}{\pgfqpoint{2.481015in}{2.485659in}}{\pgfqpoint{2.481015in}{2.477423in}}%
\pgfpathcurveto{\pgfqpoint{2.481015in}{2.469186in}}{\pgfqpoint{2.484287in}{2.461286in}}{\pgfqpoint{2.490111in}{2.455462in}}%
\pgfpathcurveto{\pgfqpoint{2.495935in}{2.449639in}}{\pgfqpoint{2.503835in}{2.446366in}}{\pgfqpoint{2.512071in}{2.446366in}}%
\pgfpathclose%
\pgfusepath{stroke,fill}%
\end{pgfscope}%
\begin{pgfscope}%
\pgfpathrectangle{\pgfqpoint{0.100000in}{0.212622in}}{\pgfqpoint{3.696000in}{3.696000in}}%
\pgfusepath{clip}%
\pgfsetbuttcap%
\pgfsetroundjoin%
\definecolor{currentfill}{rgb}{0.121569,0.466667,0.705882}%
\pgfsetfillcolor{currentfill}%
\pgfsetfillopacity{0.471348}%
\pgfsetlinewidth{1.003750pt}%
\definecolor{currentstroke}{rgb}{0.121569,0.466667,0.705882}%
\pgfsetstrokecolor{currentstroke}%
\pgfsetstrokeopacity{0.471348}%
\pgfsetdash{}{0pt}%
\pgfpathmoveto{\pgfqpoint{2.079017in}{2.184894in}}%
\pgfpathcurveto{\pgfqpoint{2.087253in}{2.184894in}}{\pgfqpoint{2.095153in}{2.188167in}}{\pgfqpoint{2.100977in}{2.193991in}}%
\pgfpathcurveto{\pgfqpoint{2.106801in}{2.199815in}}{\pgfqpoint{2.110074in}{2.207715in}}{\pgfqpoint{2.110074in}{2.215951in}}%
\pgfpathcurveto{\pgfqpoint{2.110074in}{2.224187in}}{\pgfqpoint{2.106801in}{2.232087in}}{\pgfqpoint{2.100977in}{2.237911in}}%
\pgfpathcurveto{\pgfqpoint{2.095153in}{2.243735in}}{\pgfqpoint{2.087253in}{2.247007in}}{\pgfqpoint{2.079017in}{2.247007in}}%
\pgfpathcurveto{\pgfqpoint{2.070781in}{2.247007in}}{\pgfqpoint{2.062881in}{2.243735in}}{\pgfqpoint{2.057057in}{2.237911in}}%
\pgfpathcurveto{\pgfqpoint{2.051233in}{2.232087in}}{\pgfqpoint{2.047961in}{2.224187in}}{\pgfqpoint{2.047961in}{2.215951in}}%
\pgfpathcurveto{\pgfqpoint{2.047961in}{2.207715in}}{\pgfqpoint{2.051233in}{2.199815in}}{\pgfqpoint{2.057057in}{2.193991in}}%
\pgfpathcurveto{\pgfqpoint{2.062881in}{2.188167in}}{\pgfqpoint{2.070781in}{2.184894in}}{\pgfqpoint{2.079017in}{2.184894in}}%
\pgfpathclose%
\pgfusepath{stroke,fill}%
\end{pgfscope}%
\begin{pgfscope}%
\pgfpathrectangle{\pgfqpoint{0.100000in}{0.212622in}}{\pgfqpoint{3.696000in}{3.696000in}}%
\pgfusepath{clip}%
\pgfsetbuttcap%
\pgfsetroundjoin%
\definecolor{currentfill}{rgb}{0.121569,0.466667,0.705882}%
\pgfsetfillcolor{currentfill}%
\pgfsetfillopacity{0.472934}%
\pgfsetlinewidth{1.003750pt}%
\definecolor{currentstroke}{rgb}{0.121569,0.466667,0.705882}%
\pgfsetstrokecolor{currentstroke}%
\pgfsetstrokeopacity{0.472934}%
\pgfsetdash{}{0pt}%
\pgfpathmoveto{\pgfqpoint{2.518229in}{2.444294in}}%
\pgfpathcurveto{\pgfqpoint{2.526465in}{2.444294in}}{\pgfqpoint{2.534365in}{2.447566in}}{\pgfqpoint{2.540189in}{2.453390in}}%
\pgfpathcurveto{\pgfqpoint{2.546013in}{2.459214in}}{\pgfqpoint{2.549286in}{2.467114in}}{\pgfqpoint{2.549286in}{2.475350in}}%
\pgfpathcurveto{\pgfqpoint{2.549286in}{2.483587in}}{\pgfqpoint{2.546013in}{2.491487in}}{\pgfqpoint{2.540189in}{2.497311in}}%
\pgfpathcurveto{\pgfqpoint{2.534365in}{2.503134in}}{\pgfqpoint{2.526465in}{2.506407in}}{\pgfqpoint{2.518229in}{2.506407in}}%
\pgfpathcurveto{\pgfqpoint{2.509993in}{2.506407in}}{\pgfqpoint{2.502093in}{2.503134in}}{\pgfqpoint{2.496269in}{2.497311in}}%
\pgfpathcurveto{\pgfqpoint{2.490445in}{2.491487in}}{\pgfqpoint{2.487173in}{2.483587in}}{\pgfqpoint{2.487173in}{2.475350in}}%
\pgfpathcurveto{\pgfqpoint{2.487173in}{2.467114in}}{\pgfqpoint{2.490445in}{2.459214in}}{\pgfqpoint{2.496269in}{2.453390in}}%
\pgfpathcurveto{\pgfqpoint{2.502093in}{2.447566in}}{\pgfqpoint{2.509993in}{2.444294in}}{\pgfqpoint{2.518229in}{2.444294in}}%
\pgfpathclose%
\pgfusepath{stroke,fill}%
\end{pgfscope}%
\begin{pgfscope}%
\pgfpathrectangle{\pgfqpoint{0.100000in}{0.212622in}}{\pgfqpoint{3.696000in}{3.696000in}}%
\pgfusepath{clip}%
\pgfsetbuttcap%
\pgfsetroundjoin%
\definecolor{currentfill}{rgb}{0.121569,0.466667,0.705882}%
\pgfsetfillcolor{currentfill}%
\pgfsetfillopacity{0.473585}%
\pgfsetlinewidth{1.003750pt}%
\definecolor{currentstroke}{rgb}{0.121569,0.466667,0.705882}%
\pgfsetstrokecolor{currentstroke}%
\pgfsetstrokeopacity{0.473585}%
\pgfsetdash{}{0pt}%
\pgfpathmoveto{\pgfqpoint{1.715709in}{1.982418in}}%
\pgfpathcurveto{\pgfqpoint{1.723945in}{1.982418in}}{\pgfqpoint{1.731845in}{1.985690in}}{\pgfqpoint{1.737669in}{1.991514in}}%
\pgfpathcurveto{\pgfqpoint{1.743493in}{1.997338in}}{\pgfqpoint{1.746765in}{2.005238in}}{\pgfqpoint{1.746765in}{2.013474in}}%
\pgfpathcurveto{\pgfqpoint{1.746765in}{2.021710in}}{\pgfqpoint{1.743493in}{2.029610in}}{\pgfqpoint{1.737669in}{2.035434in}}%
\pgfpathcurveto{\pgfqpoint{1.731845in}{2.041258in}}{\pgfqpoint{1.723945in}{2.044531in}}{\pgfqpoint{1.715709in}{2.044531in}}%
\pgfpathcurveto{\pgfqpoint{1.707472in}{2.044531in}}{\pgfqpoint{1.699572in}{2.041258in}}{\pgfqpoint{1.693748in}{2.035434in}}%
\pgfpathcurveto{\pgfqpoint{1.687924in}{2.029610in}}{\pgfqpoint{1.684652in}{2.021710in}}{\pgfqpoint{1.684652in}{2.013474in}}%
\pgfpathcurveto{\pgfqpoint{1.684652in}{2.005238in}}{\pgfqpoint{1.687924in}{1.997338in}}{\pgfqpoint{1.693748in}{1.991514in}}%
\pgfpathcurveto{\pgfqpoint{1.699572in}{1.985690in}}{\pgfqpoint{1.707472in}{1.982418in}}{\pgfqpoint{1.715709in}{1.982418in}}%
\pgfpathclose%
\pgfusepath{stroke,fill}%
\end{pgfscope}%
\begin{pgfscope}%
\pgfpathrectangle{\pgfqpoint{0.100000in}{0.212622in}}{\pgfqpoint{3.696000in}{3.696000in}}%
\pgfusepath{clip}%
\pgfsetbuttcap%
\pgfsetroundjoin%
\definecolor{currentfill}{rgb}{0.121569,0.466667,0.705882}%
\pgfsetfillcolor{currentfill}%
\pgfsetfillopacity{0.475247}%
\pgfsetlinewidth{1.003750pt}%
\definecolor{currentstroke}{rgb}{0.121569,0.466667,0.705882}%
\pgfsetstrokecolor{currentstroke}%
\pgfsetstrokeopacity{0.475247}%
\pgfsetdash{}{0pt}%
\pgfpathmoveto{\pgfqpoint{2.043806in}{2.158429in}}%
\pgfpathcurveto{\pgfqpoint{2.052042in}{2.158429in}}{\pgfqpoint{2.059942in}{2.161701in}}{\pgfqpoint{2.065766in}{2.167525in}}%
\pgfpathcurveto{\pgfqpoint{2.071590in}{2.173349in}}{\pgfqpoint{2.074862in}{2.181249in}}{\pgfqpoint{2.074862in}{2.189485in}}%
\pgfpathcurveto{\pgfqpoint{2.074862in}{2.197721in}}{\pgfqpoint{2.071590in}{2.205621in}}{\pgfqpoint{2.065766in}{2.211445in}}%
\pgfpathcurveto{\pgfqpoint{2.059942in}{2.217269in}}{\pgfqpoint{2.052042in}{2.220542in}}{\pgfqpoint{2.043806in}{2.220542in}}%
\pgfpathcurveto{\pgfqpoint{2.035569in}{2.220542in}}{\pgfqpoint{2.027669in}{2.217269in}}{\pgfqpoint{2.021845in}{2.211445in}}%
\pgfpathcurveto{\pgfqpoint{2.016021in}{2.205621in}}{\pgfqpoint{2.012749in}{2.197721in}}{\pgfqpoint{2.012749in}{2.189485in}}%
\pgfpathcurveto{\pgfqpoint{2.012749in}{2.181249in}}{\pgfqpoint{2.016021in}{2.173349in}}{\pgfqpoint{2.021845in}{2.167525in}}%
\pgfpathcurveto{\pgfqpoint{2.027669in}{2.161701in}}{\pgfqpoint{2.035569in}{2.158429in}}{\pgfqpoint{2.043806in}{2.158429in}}%
\pgfpathclose%
\pgfusepath{stroke,fill}%
\end{pgfscope}%
\begin{pgfscope}%
\pgfpathrectangle{\pgfqpoint{0.100000in}{0.212622in}}{\pgfqpoint{3.696000in}{3.696000in}}%
\pgfusepath{clip}%
\pgfsetbuttcap%
\pgfsetroundjoin%
\definecolor{currentfill}{rgb}{0.121569,0.466667,0.705882}%
\pgfsetfillcolor{currentfill}%
\pgfsetfillopacity{0.475302}%
\pgfsetlinewidth{1.003750pt}%
\definecolor{currentstroke}{rgb}{0.121569,0.466667,0.705882}%
\pgfsetstrokecolor{currentstroke}%
\pgfsetstrokeopacity{0.475302}%
\pgfsetdash{}{0pt}%
\pgfpathmoveto{\pgfqpoint{2.532659in}{2.451176in}}%
\pgfpathcurveto{\pgfqpoint{2.540895in}{2.451176in}}{\pgfqpoint{2.548795in}{2.454449in}}{\pgfqpoint{2.554619in}{2.460273in}}%
\pgfpathcurveto{\pgfqpoint{2.560443in}{2.466097in}}{\pgfqpoint{2.563715in}{2.473997in}}{\pgfqpoint{2.563715in}{2.482233in}}%
\pgfpathcurveto{\pgfqpoint{2.563715in}{2.490469in}}{\pgfqpoint{2.560443in}{2.498369in}}{\pgfqpoint{2.554619in}{2.504193in}}%
\pgfpathcurveto{\pgfqpoint{2.548795in}{2.510017in}}{\pgfqpoint{2.540895in}{2.513289in}}{\pgfqpoint{2.532659in}{2.513289in}}%
\pgfpathcurveto{\pgfqpoint{2.524422in}{2.513289in}}{\pgfqpoint{2.516522in}{2.510017in}}{\pgfqpoint{2.510698in}{2.504193in}}%
\pgfpathcurveto{\pgfqpoint{2.504874in}{2.498369in}}{\pgfqpoint{2.501602in}{2.490469in}}{\pgfqpoint{2.501602in}{2.482233in}}%
\pgfpathcurveto{\pgfqpoint{2.501602in}{2.473997in}}{\pgfqpoint{2.504874in}{2.466097in}}{\pgfqpoint{2.510698in}{2.460273in}}%
\pgfpathcurveto{\pgfqpoint{2.516522in}{2.454449in}}{\pgfqpoint{2.524422in}{2.451176in}}{\pgfqpoint{2.532659in}{2.451176in}}%
\pgfpathclose%
\pgfusepath{stroke,fill}%
\end{pgfscope}%
\begin{pgfscope}%
\pgfpathrectangle{\pgfqpoint{0.100000in}{0.212622in}}{\pgfqpoint{3.696000in}{3.696000in}}%
\pgfusepath{clip}%
\pgfsetbuttcap%
\pgfsetroundjoin%
\definecolor{currentfill}{rgb}{0.121569,0.466667,0.705882}%
\pgfsetfillcolor{currentfill}%
\pgfsetfillopacity{0.475513}%
\pgfsetlinewidth{1.003750pt}%
\definecolor{currentstroke}{rgb}{0.121569,0.466667,0.705882}%
\pgfsetstrokecolor{currentstroke}%
\pgfsetstrokeopacity{0.475513}%
\pgfsetdash{}{0pt}%
\pgfpathmoveto{\pgfqpoint{2.156195in}{2.187814in}}%
\pgfpathcurveto{\pgfqpoint{2.164432in}{2.187814in}}{\pgfqpoint{2.172332in}{2.191086in}}{\pgfqpoint{2.178156in}{2.196910in}}%
\pgfpathcurveto{\pgfqpoint{2.183980in}{2.202734in}}{\pgfqpoint{2.187252in}{2.210634in}}{\pgfqpoint{2.187252in}{2.218870in}}%
\pgfpathcurveto{\pgfqpoint{2.187252in}{2.227107in}}{\pgfqpoint{2.183980in}{2.235007in}}{\pgfqpoint{2.178156in}{2.240831in}}%
\pgfpathcurveto{\pgfqpoint{2.172332in}{2.246654in}}{\pgfqpoint{2.164432in}{2.249927in}}{\pgfqpoint{2.156195in}{2.249927in}}%
\pgfpathcurveto{\pgfqpoint{2.147959in}{2.249927in}}{\pgfqpoint{2.140059in}{2.246654in}}{\pgfqpoint{2.134235in}{2.240831in}}%
\pgfpathcurveto{\pgfqpoint{2.128411in}{2.235007in}}{\pgfqpoint{2.125139in}{2.227107in}}{\pgfqpoint{2.125139in}{2.218870in}}%
\pgfpathcurveto{\pgfqpoint{2.125139in}{2.210634in}}{\pgfqpoint{2.128411in}{2.202734in}}{\pgfqpoint{2.134235in}{2.196910in}}%
\pgfpathcurveto{\pgfqpoint{2.140059in}{2.191086in}}{\pgfqpoint{2.147959in}{2.187814in}}{\pgfqpoint{2.156195in}{2.187814in}}%
\pgfpathclose%
\pgfusepath{stroke,fill}%
\end{pgfscope}%
\begin{pgfscope}%
\pgfpathrectangle{\pgfqpoint{0.100000in}{0.212622in}}{\pgfqpoint{3.696000in}{3.696000in}}%
\pgfusepath{clip}%
\pgfsetbuttcap%
\pgfsetroundjoin%
\definecolor{currentfill}{rgb}{0.121569,0.466667,0.705882}%
\pgfsetfillcolor{currentfill}%
\pgfsetfillopacity{0.475658}%
\pgfsetlinewidth{1.003750pt}%
\definecolor{currentstroke}{rgb}{0.121569,0.466667,0.705882}%
\pgfsetstrokecolor{currentstroke}%
\pgfsetstrokeopacity{0.475658}%
\pgfsetdash{}{0pt}%
\pgfpathmoveto{\pgfqpoint{2.521858in}{2.447239in}}%
\pgfpathcurveto{\pgfqpoint{2.530095in}{2.447239in}}{\pgfqpoint{2.537995in}{2.450512in}}{\pgfqpoint{2.543819in}{2.456336in}}%
\pgfpathcurveto{\pgfqpoint{2.549643in}{2.462159in}}{\pgfqpoint{2.552915in}{2.470060in}}{\pgfqpoint{2.552915in}{2.478296in}}%
\pgfpathcurveto{\pgfqpoint{2.552915in}{2.486532in}}{\pgfqpoint{2.549643in}{2.494432in}}{\pgfqpoint{2.543819in}{2.500256in}}%
\pgfpathcurveto{\pgfqpoint{2.537995in}{2.506080in}}{\pgfqpoint{2.530095in}{2.509352in}}{\pgfqpoint{2.521858in}{2.509352in}}%
\pgfpathcurveto{\pgfqpoint{2.513622in}{2.509352in}}{\pgfqpoint{2.505722in}{2.506080in}}{\pgfqpoint{2.499898in}{2.500256in}}%
\pgfpathcurveto{\pgfqpoint{2.494074in}{2.494432in}}{\pgfqpoint{2.490802in}{2.486532in}}{\pgfqpoint{2.490802in}{2.478296in}}%
\pgfpathcurveto{\pgfqpoint{2.490802in}{2.470060in}}{\pgfqpoint{2.494074in}{2.462159in}}{\pgfqpoint{2.499898in}{2.456336in}}%
\pgfpathcurveto{\pgfqpoint{2.505722in}{2.450512in}}{\pgfqpoint{2.513622in}{2.447239in}}{\pgfqpoint{2.521858in}{2.447239in}}%
\pgfpathclose%
\pgfusepath{stroke,fill}%
\end{pgfscope}%
\begin{pgfscope}%
\pgfpathrectangle{\pgfqpoint{0.100000in}{0.212622in}}{\pgfqpoint{3.696000in}{3.696000in}}%
\pgfusepath{clip}%
\pgfsetbuttcap%
\pgfsetroundjoin%
\definecolor{currentfill}{rgb}{0.121569,0.466667,0.705882}%
\pgfsetfillcolor{currentfill}%
\pgfsetfillopacity{0.477966}%
\pgfsetlinewidth{1.003750pt}%
\definecolor{currentstroke}{rgb}{0.121569,0.466667,0.705882}%
\pgfsetstrokecolor{currentstroke}%
\pgfsetstrokeopacity{0.477966}%
\pgfsetdash{}{0pt}%
\pgfpathmoveto{\pgfqpoint{1.706923in}{1.971773in}}%
\pgfpathcurveto{\pgfqpoint{1.715159in}{1.971773in}}{\pgfqpoint{1.723059in}{1.975046in}}{\pgfqpoint{1.728883in}{1.980870in}}%
\pgfpathcurveto{\pgfqpoint{1.734707in}{1.986694in}}{\pgfqpoint{1.737979in}{1.994594in}}{\pgfqpoint{1.737979in}{2.002830in}}%
\pgfpathcurveto{\pgfqpoint{1.737979in}{2.011066in}}{\pgfqpoint{1.734707in}{2.018966in}}{\pgfqpoint{1.728883in}{2.024790in}}%
\pgfpathcurveto{\pgfqpoint{1.723059in}{2.030614in}}{\pgfqpoint{1.715159in}{2.033886in}}{\pgfqpoint{1.706923in}{2.033886in}}%
\pgfpathcurveto{\pgfqpoint{1.698687in}{2.033886in}}{\pgfqpoint{1.690786in}{2.030614in}}{\pgfqpoint{1.684963in}{2.024790in}}%
\pgfpathcurveto{\pgfqpoint{1.679139in}{2.018966in}}{\pgfqpoint{1.675866in}{2.011066in}}{\pgfqpoint{1.675866in}{2.002830in}}%
\pgfpathcurveto{\pgfqpoint{1.675866in}{1.994594in}}{\pgfqpoint{1.679139in}{1.986694in}}{\pgfqpoint{1.684963in}{1.980870in}}%
\pgfpathcurveto{\pgfqpoint{1.690786in}{1.975046in}}{\pgfqpoint{1.698687in}{1.971773in}}{\pgfqpoint{1.706923in}{1.971773in}}%
\pgfpathclose%
\pgfusepath{stroke,fill}%
\end{pgfscope}%
\begin{pgfscope}%
\pgfpathrectangle{\pgfqpoint{0.100000in}{0.212622in}}{\pgfqpoint{3.696000in}{3.696000in}}%
\pgfusepath{clip}%
\pgfsetbuttcap%
\pgfsetroundjoin%
\definecolor{currentfill}{rgb}{0.121569,0.466667,0.705882}%
\pgfsetfillcolor{currentfill}%
\pgfsetfillopacity{0.478253}%
\pgfsetlinewidth{1.003750pt}%
\definecolor{currentstroke}{rgb}{0.121569,0.466667,0.705882}%
\pgfsetstrokecolor{currentstroke}%
\pgfsetstrokeopacity{0.478253}%
\pgfsetdash{}{0pt}%
\pgfpathmoveto{\pgfqpoint{2.078904in}{2.164987in}}%
\pgfpathcurveto{\pgfqpoint{2.087140in}{2.164987in}}{\pgfqpoint{2.095040in}{2.168260in}}{\pgfqpoint{2.100864in}{2.174084in}}%
\pgfpathcurveto{\pgfqpoint{2.106688in}{2.179908in}}{\pgfqpoint{2.109960in}{2.187808in}}{\pgfqpoint{2.109960in}{2.196044in}}%
\pgfpathcurveto{\pgfqpoint{2.109960in}{2.204280in}}{\pgfqpoint{2.106688in}{2.212180in}}{\pgfqpoint{2.100864in}{2.218004in}}%
\pgfpathcurveto{\pgfqpoint{2.095040in}{2.223828in}}{\pgfqpoint{2.087140in}{2.227100in}}{\pgfqpoint{2.078904in}{2.227100in}}%
\pgfpathcurveto{\pgfqpoint{2.070668in}{2.227100in}}{\pgfqpoint{2.062768in}{2.223828in}}{\pgfqpoint{2.056944in}{2.218004in}}%
\pgfpathcurveto{\pgfqpoint{2.051120in}{2.212180in}}{\pgfqpoint{2.047847in}{2.204280in}}{\pgfqpoint{2.047847in}{2.196044in}}%
\pgfpathcurveto{\pgfqpoint{2.047847in}{2.187808in}}{\pgfqpoint{2.051120in}{2.179908in}}{\pgfqpoint{2.056944in}{2.174084in}}%
\pgfpathcurveto{\pgfqpoint{2.062768in}{2.168260in}}{\pgfqpoint{2.070668in}{2.164987in}}{\pgfqpoint{2.078904in}{2.164987in}}%
\pgfpathclose%
\pgfusepath{stroke,fill}%
\end{pgfscope}%
\begin{pgfscope}%
\pgfpathrectangle{\pgfqpoint{0.100000in}{0.212622in}}{\pgfqpoint{3.696000in}{3.696000in}}%
\pgfusepath{clip}%
\pgfsetbuttcap%
\pgfsetroundjoin%
\definecolor{currentfill}{rgb}{0.121569,0.466667,0.705882}%
\pgfsetfillcolor{currentfill}%
\pgfsetfillopacity{0.479253}%
\pgfsetlinewidth{1.003750pt}%
\definecolor{currentstroke}{rgb}{0.121569,0.466667,0.705882}%
\pgfsetstrokecolor{currentstroke}%
\pgfsetstrokeopacity{0.479253}%
\pgfsetdash{}{0pt}%
\pgfpathmoveto{\pgfqpoint{2.121691in}{2.172447in}}%
\pgfpathcurveto{\pgfqpoint{2.129927in}{2.172447in}}{\pgfqpoint{2.137827in}{2.175719in}}{\pgfqpoint{2.143651in}{2.181543in}}%
\pgfpathcurveto{\pgfqpoint{2.149475in}{2.187367in}}{\pgfqpoint{2.152747in}{2.195267in}}{\pgfqpoint{2.152747in}{2.203503in}}%
\pgfpathcurveto{\pgfqpoint{2.152747in}{2.211739in}}{\pgfqpoint{2.149475in}{2.219640in}}{\pgfqpoint{2.143651in}{2.225463in}}%
\pgfpathcurveto{\pgfqpoint{2.137827in}{2.231287in}}{\pgfqpoint{2.129927in}{2.234560in}}{\pgfqpoint{2.121691in}{2.234560in}}%
\pgfpathcurveto{\pgfqpoint{2.113454in}{2.234560in}}{\pgfqpoint{2.105554in}{2.231287in}}{\pgfqpoint{2.099730in}{2.225463in}}%
\pgfpathcurveto{\pgfqpoint{2.093906in}{2.219640in}}{\pgfqpoint{2.090634in}{2.211739in}}{\pgfqpoint{2.090634in}{2.203503in}}%
\pgfpathcurveto{\pgfqpoint{2.090634in}{2.195267in}}{\pgfqpoint{2.093906in}{2.187367in}}{\pgfqpoint{2.099730in}{2.181543in}}%
\pgfpathcurveto{\pgfqpoint{2.105554in}{2.175719in}}{\pgfqpoint{2.113454in}{2.172447in}}{\pgfqpoint{2.121691in}{2.172447in}}%
\pgfpathclose%
\pgfusepath{stroke,fill}%
\end{pgfscope}%
\begin{pgfscope}%
\pgfpathrectangle{\pgfqpoint{0.100000in}{0.212622in}}{\pgfqpoint{3.696000in}{3.696000in}}%
\pgfusepath{clip}%
\pgfsetbuttcap%
\pgfsetroundjoin%
\definecolor{currentfill}{rgb}{0.121569,0.466667,0.705882}%
\pgfsetfillcolor{currentfill}%
\pgfsetfillopacity{0.481149}%
\pgfsetlinewidth{1.003750pt}%
\definecolor{currentstroke}{rgb}{0.121569,0.466667,0.705882}%
\pgfsetstrokecolor{currentstroke}%
\pgfsetstrokeopacity{0.481149}%
\pgfsetdash{}{0pt}%
\pgfpathmoveto{\pgfqpoint{1.702043in}{1.962848in}}%
\pgfpathcurveto{\pgfqpoint{1.710279in}{1.962848in}}{\pgfqpoint{1.718179in}{1.966121in}}{\pgfqpoint{1.724003in}{1.971945in}}%
\pgfpathcurveto{\pgfqpoint{1.729827in}{1.977769in}}{\pgfqpoint{1.733099in}{1.985669in}}{\pgfqpoint{1.733099in}{1.993905in}}%
\pgfpathcurveto{\pgfqpoint{1.733099in}{2.002141in}}{\pgfqpoint{1.729827in}{2.010041in}}{\pgfqpoint{1.724003in}{2.015865in}}%
\pgfpathcurveto{\pgfqpoint{1.718179in}{2.021689in}}{\pgfqpoint{1.710279in}{2.024961in}}{\pgfqpoint{1.702043in}{2.024961in}}%
\pgfpathcurveto{\pgfqpoint{1.693806in}{2.024961in}}{\pgfqpoint{1.685906in}{2.021689in}}{\pgfqpoint{1.680082in}{2.015865in}}%
\pgfpathcurveto{\pgfqpoint{1.674258in}{2.010041in}}{\pgfqpoint{1.670986in}{2.002141in}}{\pgfqpoint{1.670986in}{1.993905in}}%
\pgfpathcurveto{\pgfqpoint{1.670986in}{1.985669in}}{\pgfqpoint{1.674258in}{1.977769in}}{\pgfqpoint{1.680082in}{1.971945in}}%
\pgfpathcurveto{\pgfqpoint{1.685906in}{1.966121in}}{\pgfqpoint{1.693806in}{1.962848in}}{\pgfqpoint{1.702043in}{1.962848in}}%
\pgfpathclose%
\pgfusepath{stroke,fill}%
\end{pgfscope}%
\begin{pgfscope}%
\pgfpathrectangle{\pgfqpoint{0.100000in}{0.212622in}}{\pgfqpoint{3.696000in}{3.696000in}}%
\pgfusepath{clip}%
\pgfsetbuttcap%
\pgfsetroundjoin%
\definecolor{currentfill}{rgb}{0.121569,0.466667,0.705882}%
\pgfsetfillcolor{currentfill}%
\pgfsetfillopacity{0.481312}%
\pgfsetlinewidth{1.003750pt}%
\definecolor{currentstroke}{rgb}{0.121569,0.466667,0.705882}%
\pgfsetstrokecolor{currentstroke}%
\pgfsetstrokeopacity{0.481312}%
\pgfsetdash{}{0pt}%
\pgfpathmoveto{\pgfqpoint{1.971571in}{2.130726in}}%
\pgfpathcurveto{\pgfqpoint{1.979808in}{2.130726in}}{\pgfqpoint{1.987708in}{2.133998in}}{\pgfqpoint{1.993532in}{2.139822in}}%
\pgfpathcurveto{\pgfqpoint{1.999356in}{2.145646in}}{\pgfqpoint{2.002628in}{2.153546in}}{\pgfqpoint{2.002628in}{2.161782in}}%
\pgfpathcurveto{\pgfqpoint{2.002628in}{2.170018in}}{\pgfqpoint{1.999356in}{2.177918in}}{\pgfqpoint{1.993532in}{2.183742in}}%
\pgfpathcurveto{\pgfqpoint{1.987708in}{2.189566in}}{\pgfqpoint{1.979808in}{2.192839in}}{\pgfqpoint{1.971571in}{2.192839in}}%
\pgfpathcurveto{\pgfqpoint{1.963335in}{2.192839in}}{\pgfqpoint{1.955435in}{2.189566in}}{\pgfqpoint{1.949611in}{2.183742in}}%
\pgfpathcurveto{\pgfqpoint{1.943787in}{2.177918in}}{\pgfqpoint{1.940515in}{2.170018in}}{\pgfqpoint{1.940515in}{2.161782in}}%
\pgfpathcurveto{\pgfqpoint{1.940515in}{2.153546in}}{\pgfqpoint{1.943787in}{2.145646in}}{\pgfqpoint{1.949611in}{2.139822in}}%
\pgfpathcurveto{\pgfqpoint{1.955435in}{2.133998in}}{\pgfqpoint{1.963335in}{2.130726in}}{\pgfqpoint{1.971571in}{2.130726in}}%
\pgfpathclose%
\pgfusepath{stroke,fill}%
\end{pgfscope}%
\begin{pgfscope}%
\pgfpathrectangle{\pgfqpoint{0.100000in}{0.212622in}}{\pgfqpoint{3.696000in}{3.696000in}}%
\pgfusepath{clip}%
\pgfsetbuttcap%
\pgfsetroundjoin%
\definecolor{currentfill}{rgb}{0.121569,0.466667,0.705882}%
\pgfsetfillcolor{currentfill}%
\pgfsetfillopacity{0.481864}%
\pgfsetlinewidth{1.003750pt}%
\definecolor{currentstroke}{rgb}{0.121569,0.466667,0.705882}%
\pgfsetstrokecolor{currentstroke}%
\pgfsetstrokeopacity{0.481864}%
\pgfsetdash{}{0pt}%
\pgfpathmoveto{\pgfqpoint{2.528259in}{2.449787in}}%
\pgfpathcurveto{\pgfqpoint{2.536495in}{2.449787in}}{\pgfqpoint{2.544395in}{2.453060in}}{\pgfqpoint{2.550219in}{2.458884in}}%
\pgfpathcurveto{\pgfqpoint{2.556043in}{2.464708in}}{\pgfqpoint{2.559315in}{2.472608in}}{\pgfqpoint{2.559315in}{2.480844in}}%
\pgfpathcurveto{\pgfqpoint{2.559315in}{2.489080in}}{\pgfqpoint{2.556043in}{2.496980in}}{\pgfqpoint{2.550219in}{2.502804in}}%
\pgfpathcurveto{\pgfqpoint{2.544395in}{2.508628in}}{\pgfqpoint{2.536495in}{2.511900in}}{\pgfqpoint{2.528259in}{2.511900in}}%
\pgfpathcurveto{\pgfqpoint{2.520023in}{2.511900in}}{\pgfqpoint{2.512122in}{2.508628in}}{\pgfqpoint{2.506299in}{2.502804in}}%
\pgfpathcurveto{\pgfqpoint{2.500475in}{2.496980in}}{\pgfqpoint{2.497202in}{2.489080in}}{\pgfqpoint{2.497202in}{2.480844in}}%
\pgfpathcurveto{\pgfqpoint{2.497202in}{2.472608in}}{\pgfqpoint{2.500475in}{2.464708in}}{\pgfqpoint{2.506299in}{2.458884in}}%
\pgfpathcurveto{\pgfqpoint{2.512122in}{2.453060in}}{\pgfqpoint{2.520023in}{2.449787in}}{\pgfqpoint{2.528259in}{2.449787in}}%
\pgfpathclose%
\pgfusepath{stroke,fill}%
\end{pgfscope}%
\begin{pgfscope}%
\pgfpathrectangle{\pgfqpoint{0.100000in}{0.212622in}}{\pgfqpoint{3.696000in}{3.696000in}}%
\pgfusepath{clip}%
\pgfsetbuttcap%
\pgfsetroundjoin%
\definecolor{currentfill}{rgb}{0.121569,0.466667,0.705882}%
\pgfsetfillcolor{currentfill}%
\pgfsetfillopacity{0.488598}%
\pgfsetlinewidth{1.003750pt}%
\definecolor{currentstroke}{rgb}{0.121569,0.466667,0.705882}%
\pgfsetstrokecolor{currentstroke}%
\pgfsetstrokeopacity{0.488598}%
\pgfsetdash{}{0pt}%
\pgfpathmoveto{\pgfqpoint{1.966813in}{2.114834in}}%
\pgfpathcurveto{\pgfqpoint{1.975049in}{2.114834in}}{\pgfqpoint{1.982949in}{2.118106in}}{\pgfqpoint{1.988773in}{2.123930in}}%
\pgfpathcurveto{\pgfqpoint{1.994597in}{2.129754in}}{\pgfqpoint{1.997869in}{2.137654in}}{\pgfqpoint{1.997869in}{2.145890in}}%
\pgfpathcurveto{\pgfqpoint{1.997869in}{2.154126in}}{\pgfqpoint{1.994597in}{2.162027in}}{\pgfqpoint{1.988773in}{2.167850in}}%
\pgfpathcurveto{\pgfqpoint{1.982949in}{2.173674in}}{\pgfqpoint{1.975049in}{2.176947in}}{\pgfqpoint{1.966813in}{2.176947in}}%
\pgfpathcurveto{\pgfqpoint{1.958577in}{2.176947in}}{\pgfqpoint{1.950677in}{2.173674in}}{\pgfqpoint{1.944853in}{2.167850in}}%
\pgfpathcurveto{\pgfqpoint{1.939029in}{2.162027in}}{\pgfqpoint{1.935756in}{2.154126in}}{\pgfqpoint{1.935756in}{2.145890in}}%
\pgfpathcurveto{\pgfqpoint{1.935756in}{2.137654in}}{\pgfqpoint{1.939029in}{2.129754in}}{\pgfqpoint{1.944853in}{2.123930in}}%
\pgfpathcurveto{\pgfqpoint{1.950677in}{2.118106in}}{\pgfqpoint{1.958577in}{2.114834in}}{\pgfqpoint{1.966813in}{2.114834in}}%
\pgfpathclose%
\pgfusepath{stroke,fill}%
\end{pgfscope}%
\begin{pgfscope}%
\pgfpathrectangle{\pgfqpoint{0.100000in}{0.212622in}}{\pgfqpoint{3.696000in}{3.696000in}}%
\pgfusepath{clip}%
\pgfsetbuttcap%
\pgfsetroundjoin%
\definecolor{currentfill}{rgb}{0.121569,0.466667,0.705882}%
\pgfsetfillcolor{currentfill}%
\pgfsetfillopacity{0.488875}%
\pgfsetlinewidth{1.003750pt}%
\definecolor{currentstroke}{rgb}{0.121569,0.466667,0.705882}%
\pgfsetstrokecolor{currentstroke}%
\pgfsetstrokeopacity{0.488875}%
\pgfsetdash{}{0pt}%
\pgfpathmoveto{\pgfqpoint{1.984702in}{2.119228in}}%
\pgfpathcurveto{\pgfqpoint{1.992938in}{2.119228in}}{\pgfqpoint{2.000838in}{2.122501in}}{\pgfqpoint{2.006662in}{2.128325in}}%
\pgfpathcurveto{\pgfqpoint{2.012486in}{2.134149in}}{\pgfqpoint{2.015758in}{2.142049in}}{\pgfqpoint{2.015758in}{2.150285in}}%
\pgfpathcurveto{\pgfqpoint{2.015758in}{2.158521in}}{\pgfqpoint{2.012486in}{2.166421in}}{\pgfqpoint{2.006662in}{2.172245in}}%
\pgfpathcurveto{\pgfqpoint{2.000838in}{2.178069in}}{\pgfqpoint{1.992938in}{2.181341in}}{\pgfqpoint{1.984702in}{2.181341in}}%
\pgfpathcurveto{\pgfqpoint{1.976466in}{2.181341in}}{\pgfqpoint{1.968566in}{2.178069in}}{\pgfqpoint{1.962742in}{2.172245in}}%
\pgfpathcurveto{\pgfqpoint{1.956918in}{2.166421in}}{\pgfqpoint{1.953645in}{2.158521in}}{\pgfqpoint{1.953645in}{2.150285in}}%
\pgfpathcurveto{\pgfqpoint{1.953645in}{2.142049in}}{\pgfqpoint{1.956918in}{2.134149in}}{\pgfqpoint{1.962742in}{2.128325in}}%
\pgfpathcurveto{\pgfqpoint{1.968566in}{2.122501in}}{\pgfqpoint{1.976466in}{2.119228in}}{\pgfqpoint{1.984702in}{2.119228in}}%
\pgfpathclose%
\pgfusepath{stroke,fill}%
\end{pgfscope}%
\begin{pgfscope}%
\pgfpathrectangle{\pgfqpoint{0.100000in}{0.212622in}}{\pgfqpoint{3.696000in}{3.696000in}}%
\pgfusepath{clip}%
\pgfsetbuttcap%
\pgfsetroundjoin%
\definecolor{currentfill}{rgb}{0.121569,0.466667,0.705882}%
\pgfsetfillcolor{currentfill}%
\pgfsetfillopacity{0.491180}%
\pgfsetlinewidth{1.003750pt}%
\definecolor{currentstroke}{rgb}{0.121569,0.466667,0.705882}%
\pgfsetstrokecolor{currentstroke}%
\pgfsetstrokeopacity{0.491180}%
\pgfsetdash{}{0pt}%
\pgfpathmoveto{\pgfqpoint{1.928761in}{2.103450in}}%
\pgfpathcurveto{\pgfqpoint{1.936998in}{2.103450in}}{\pgfqpoint{1.944898in}{2.106722in}}{\pgfqpoint{1.950722in}{2.112546in}}%
\pgfpathcurveto{\pgfqpoint{1.956546in}{2.118370in}}{\pgfqpoint{1.959818in}{2.126270in}}{\pgfqpoint{1.959818in}{2.134506in}}%
\pgfpathcurveto{\pgfqpoint{1.959818in}{2.142743in}}{\pgfqpoint{1.956546in}{2.150643in}}{\pgfqpoint{1.950722in}{2.156467in}}%
\pgfpathcurveto{\pgfqpoint{1.944898in}{2.162291in}}{\pgfqpoint{1.936998in}{2.165563in}}{\pgfqpoint{1.928761in}{2.165563in}}%
\pgfpathcurveto{\pgfqpoint{1.920525in}{2.165563in}}{\pgfqpoint{1.912625in}{2.162291in}}{\pgfqpoint{1.906801in}{2.156467in}}%
\pgfpathcurveto{\pgfqpoint{1.900977in}{2.150643in}}{\pgfqpoint{1.897705in}{2.142743in}}{\pgfqpoint{1.897705in}{2.134506in}}%
\pgfpathcurveto{\pgfqpoint{1.897705in}{2.126270in}}{\pgfqpoint{1.900977in}{2.118370in}}{\pgfqpoint{1.906801in}{2.112546in}}%
\pgfpathcurveto{\pgfqpoint{1.912625in}{2.106722in}}{\pgfqpoint{1.920525in}{2.103450in}}{\pgfqpoint{1.928761in}{2.103450in}}%
\pgfpathclose%
\pgfusepath{stroke,fill}%
\end{pgfscope}%
\begin{pgfscope}%
\pgfpathrectangle{\pgfqpoint{0.100000in}{0.212622in}}{\pgfqpoint{3.696000in}{3.696000in}}%
\pgfusepath{clip}%
\pgfsetbuttcap%
\pgfsetroundjoin%
\definecolor{currentfill}{rgb}{0.121569,0.466667,0.705882}%
\pgfsetfillcolor{currentfill}%
\pgfsetfillopacity{0.492171}%
\pgfsetlinewidth{1.003750pt}%
\definecolor{currentstroke}{rgb}{0.121569,0.466667,0.705882}%
\pgfsetstrokecolor{currentstroke}%
\pgfsetstrokeopacity{0.492171}%
\pgfsetdash{}{0pt}%
\pgfpathmoveto{\pgfqpoint{2.512764in}{2.434792in}}%
\pgfpathcurveto{\pgfqpoint{2.521000in}{2.434792in}}{\pgfqpoint{2.528900in}{2.438064in}}{\pgfqpoint{2.534724in}{2.443888in}}%
\pgfpathcurveto{\pgfqpoint{2.540548in}{2.449712in}}{\pgfqpoint{2.543820in}{2.457612in}}{\pgfqpoint{2.543820in}{2.465849in}}%
\pgfpathcurveto{\pgfqpoint{2.543820in}{2.474085in}}{\pgfqpoint{2.540548in}{2.481985in}}{\pgfqpoint{2.534724in}{2.487809in}}%
\pgfpathcurveto{\pgfqpoint{2.528900in}{2.493633in}}{\pgfqpoint{2.521000in}{2.496905in}}{\pgfqpoint{2.512764in}{2.496905in}}%
\pgfpathcurveto{\pgfqpoint{2.504528in}{2.496905in}}{\pgfqpoint{2.496627in}{2.493633in}}{\pgfqpoint{2.490804in}{2.487809in}}%
\pgfpathcurveto{\pgfqpoint{2.484980in}{2.481985in}}{\pgfqpoint{2.481707in}{2.474085in}}{\pgfqpoint{2.481707in}{2.465849in}}%
\pgfpathcurveto{\pgfqpoint{2.481707in}{2.457612in}}{\pgfqpoint{2.484980in}{2.449712in}}{\pgfqpoint{2.490804in}{2.443888in}}%
\pgfpathcurveto{\pgfqpoint{2.496627in}{2.438064in}}{\pgfqpoint{2.504528in}{2.434792in}}{\pgfqpoint{2.512764in}{2.434792in}}%
\pgfpathclose%
\pgfusepath{stroke,fill}%
\end{pgfscope}%
\begin{pgfscope}%
\pgfpathrectangle{\pgfqpoint{0.100000in}{0.212622in}}{\pgfqpoint{3.696000in}{3.696000in}}%
\pgfusepath{clip}%
\pgfsetbuttcap%
\pgfsetroundjoin%
\definecolor{currentfill}{rgb}{0.121569,0.466667,0.705882}%
\pgfsetfillcolor{currentfill}%
\pgfsetfillopacity{0.494318}%
\pgfsetlinewidth{1.003750pt}%
\definecolor{currentstroke}{rgb}{0.121569,0.466667,0.705882}%
\pgfsetstrokecolor{currentstroke}%
\pgfsetstrokeopacity{0.494318}%
\pgfsetdash{}{0pt}%
\pgfpathmoveto{\pgfqpoint{2.052285in}{2.118339in}}%
\pgfpathcurveto{\pgfqpoint{2.060521in}{2.118339in}}{\pgfqpoint{2.068421in}{2.121612in}}{\pgfqpoint{2.074245in}{2.127436in}}%
\pgfpathcurveto{\pgfqpoint{2.080069in}{2.133259in}}{\pgfqpoint{2.083342in}{2.141160in}}{\pgfqpoint{2.083342in}{2.149396in}}%
\pgfpathcurveto{\pgfqpoint{2.083342in}{2.157632in}}{\pgfqpoint{2.080069in}{2.165532in}}{\pgfqpoint{2.074245in}{2.171356in}}%
\pgfpathcurveto{\pgfqpoint{2.068421in}{2.177180in}}{\pgfqpoint{2.060521in}{2.180452in}}{\pgfqpoint{2.052285in}{2.180452in}}%
\pgfpathcurveto{\pgfqpoint{2.044049in}{2.180452in}}{\pgfqpoint{2.036149in}{2.177180in}}{\pgfqpoint{2.030325in}{2.171356in}}%
\pgfpathcurveto{\pgfqpoint{2.024501in}{2.165532in}}{\pgfqpoint{2.021229in}{2.157632in}}{\pgfqpoint{2.021229in}{2.149396in}}%
\pgfpathcurveto{\pgfqpoint{2.021229in}{2.141160in}}{\pgfqpoint{2.024501in}{2.133259in}}{\pgfqpoint{2.030325in}{2.127436in}}%
\pgfpathcurveto{\pgfqpoint{2.036149in}{2.121612in}}{\pgfqpoint{2.044049in}{2.118339in}}{\pgfqpoint{2.052285in}{2.118339in}}%
\pgfpathclose%
\pgfusepath{stroke,fill}%
\end{pgfscope}%
\begin{pgfscope}%
\pgfpathrectangle{\pgfqpoint{0.100000in}{0.212622in}}{\pgfqpoint{3.696000in}{3.696000in}}%
\pgfusepath{clip}%
\pgfsetbuttcap%
\pgfsetroundjoin%
\definecolor{currentfill}{rgb}{0.121569,0.466667,0.705882}%
\pgfsetfillcolor{currentfill}%
\pgfsetfillopacity{0.499647}%
\pgfsetlinewidth{1.003750pt}%
\definecolor{currentstroke}{rgb}{0.121569,0.466667,0.705882}%
\pgfsetstrokecolor{currentstroke}%
\pgfsetstrokeopacity{0.499647}%
\pgfsetdash{}{0pt}%
\pgfpathmoveto{\pgfqpoint{2.502490in}{2.415880in}}%
\pgfpathcurveto{\pgfqpoint{2.510726in}{2.415880in}}{\pgfqpoint{2.518626in}{2.419153in}}{\pgfqpoint{2.524450in}{2.424976in}}%
\pgfpathcurveto{\pgfqpoint{2.530274in}{2.430800in}}{\pgfqpoint{2.533546in}{2.438700in}}{\pgfqpoint{2.533546in}{2.446937in}}%
\pgfpathcurveto{\pgfqpoint{2.533546in}{2.455173in}}{\pgfqpoint{2.530274in}{2.463073in}}{\pgfqpoint{2.524450in}{2.468897in}}%
\pgfpathcurveto{\pgfqpoint{2.518626in}{2.474721in}}{\pgfqpoint{2.510726in}{2.477993in}}{\pgfqpoint{2.502490in}{2.477993in}}%
\pgfpathcurveto{\pgfqpoint{2.494254in}{2.477993in}}{\pgfqpoint{2.486354in}{2.474721in}}{\pgfqpoint{2.480530in}{2.468897in}}%
\pgfpathcurveto{\pgfqpoint{2.474706in}{2.463073in}}{\pgfqpoint{2.471433in}{2.455173in}}{\pgfqpoint{2.471433in}{2.446937in}}%
\pgfpathcurveto{\pgfqpoint{2.471433in}{2.438700in}}{\pgfqpoint{2.474706in}{2.430800in}}{\pgfqpoint{2.480530in}{2.424976in}}%
\pgfpathcurveto{\pgfqpoint{2.486354in}{2.419153in}}{\pgfqpoint{2.494254in}{2.415880in}}{\pgfqpoint{2.502490in}{2.415880in}}%
\pgfpathclose%
\pgfusepath{stroke,fill}%
\end{pgfscope}%
\begin{pgfscope}%
\pgfpathrectangle{\pgfqpoint{0.100000in}{0.212622in}}{\pgfqpoint{3.696000in}{3.696000in}}%
\pgfusepath{clip}%
\pgfsetbuttcap%
\pgfsetroundjoin%
\definecolor{currentfill}{rgb}{0.121569,0.466667,0.705882}%
\pgfsetfillcolor{currentfill}%
\pgfsetfillopacity{0.502029}%
\pgfsetlinewidth{1.003750pt}%
\definecolor{currentstroke}{rgb}{0.121569,0.466667,0.705882}%
\pgfsetstrokecolor{currentstroke}%
\pgfsetstrokeopacity{0.502029}%
\pgfsetdash{}{0pt}%
\pgfpathmoveto{\pgfqpoint{1.883533in}{2.068720in}}%
\pgfpathcurveto{\pgfqpoint{1.891769in}{2.068720in}}{\pgfqpoint{1.899669in}{2.071993in}}{\pgfqpoint{1.905493in}{2.077817in}}%
\pgfpathcurveto{\pgfqpoint{1.911317in}{2.083641in}}{\pgfqpoint{1.914589in}{2.091541in}}{\pgfqpoint{1.914589in}{2.099777in}}%
\pgfpathcurveto{\pgfqpoint{1.914589in}{2.108013in}}{\pgfqpoint{1.911317in}{2.115913in}}{\pgfqpoint{1.905493in}{2.121737in}}%
\pgfpathcurveto{\pgfqpoint{1.899669in}{2.127561in}}{\pgfqpoint{1.891769in}{2.130833in}}{\pgfqpoint{1.883533in}{2.130833in}}%
\pgfpathcurveto{\pgfqpoint{1.875296in}{2.130833in}}{\pgfqpoint{1.867396in}{2.127561in}}{\pgfqpoint{1.861572in}{2.121737in}}%
\pgfpathcurveto{\pgfqpoint{1.855748in}{2.115913in}}{\pgfqpoint{1.852476in}{2.108013in}}{\pgfqpoint{1.852476in}{2.099777in}}%
\pgfpathcurveto{\pgfqpoint{1.852476in}{2.091541in}}{\pgfqpoint{1.855748in}{2.083641in}}{\pgfqpoint{1.861572in}{2.077817in}}%
\pgfpathcurveto{\pgfqpoint{1.867396in}{2.071993in}}{\pgfqpoint{1.875296in}{2.068720in}}{\pgfqpoint{1.883533in}{2.068720in}}%
\pgfpathclose%
\pgfusepath{stroke,fill}%
\end{pgfscope}%
\begin{pgfscope}%
\pgfpathrectangle{\pgfqpoint{0.100000in}{0.212622in}}{\pgfqpoint{3.696000in}{3.696000in}}%
\pgfusepath{clip}%
\pgfsetbuttcap%
\pgfsetroundjoin%
\definecolor{currentfill}{rgb}{0.121569,0.466667,0.705882}%
\pgfsetfillcolor{currentfill}%
\pgfsetfillopacity{0.506045}%
\pgfsetlinewidth{1.003750pt}%
\definecolor{currentstroke}{rgb}{0.121569,0.466667,0.705882}%
\pgfsetstrokecolor{currentstroke}%
\pgfsetstrokeopacity{0.506045}%
\pgfsetdash{}{0pt}%
\pgfpathmoveto{\pgfqpoint{1.632229in}{1.890891in}}%
\pgfpathcurveto{\pgfqpoint{1.640466in}{1.890891in}}{\pgfqpoint{1.648366in}{1.894163in}}{\pgfqpoint{1.654190in}{1.899987in}}%
\pgfpathcurveto{\pgfqpoint{1.660014in}{1.905811in}}{\pgfqpoint{1.663286in}{1.913711in}}{\pgfqpoint{1.663286in}{1.921947in}}%
\pgfpathcurveto{\pgfqpoint{1.663286in}{1.930183in}}{\pgfqpoint{1.660014in}{1.938083in}}{\pgfqpoint{1.654190in}{1.943907in}}%
\pgfpathcurveto{\pgfqpoint{1.648366in}{1.949731in}}{\pgfqpoint{1.640466in}{1.953004in}}{\pgfqpoint{1.632229in}{1.953004in}}%
\pgfpathcurveto{\pgfqpoint{1.623993in}{1.953004in}}{\pgfqpoint{1.616093in}{1.949731in}}{\pgfqpoint{1.610269in}{1.943907in}}%
\pgfpathcurveto{\pgfqpoint{1.604445in}{1.938083in}}{\pgfqpoint{1.601173in}{1.930183in}}{\pgfqpoint{1.601173in}{1.921947in}}%
\pgfpathcurveto{\pgfqpoint{1.601173in}{1.913711in}}{\pgfqpoint{1.604445in}{1.905811in}}{\pgfqpoint{1.610269in}{1.899987in}}%
\pgfpathcurveto{\pgfqpoint{1.616093in}{1.894163in}}{\pgfqpoint{1.623993in}{1.890891in}}{\pgfqpoint{1.632229in}{1.890891in}}%
\pgfpathclose%
\pgfusepath{stroke,fill}%
\end{pgfscope}%
\begin{pgfscope}%
\pgfpathrectangle{\pgfqpoint{0.100000in}{0.212622in}}{\pgfqpoint{3.696000in}{3.696000in}}%
\pgfusepath{clip}%
\pgfsetbuttcap%
\pgfsetroundjoin%
\definecolor{currentfill}{rgb}{0.121569,0.466667,0.705882}%
\pgfsetfillcolor{currentfill}%
\pgfsetfillopacity{0.508717}%
\pgfsetlinewidth{1.003750pt}%
\definecolor{currentstroke}{rgb}{0.121569,0.466667,0.705882}%
\pgfsetstrokecolor{currentstroke}%
\pgfsetstrokeopacity{0.508717}%
\pgfsetdash{}{0pt}%
\pgfpathmoveto{\pgfqpoint{1.850257in}{2.039701in}}%
\pgfpathcurveto{\pgfqpoint{1.858494in}{2.039701in}}{\pgfqpoint{1.866394in}{2.042973in}}{\pgfqpoint{1.872218in}{2.048797in}}%
\pgfpathcurveto{\pgfqpoint{1.878041in}{2.054621in}}{\pgfqpoint{1.881314in}{2.062521in}}{\pgfqpoint{1.881314in}{2.070757in}}%
\pgfpathcurveto{\pgfqpoint{1.881314in}{2.078993in}}{\pgfqpoint{1.878041in}{2.086893in}}{\pgfqpoint{1.872218in}{2.092717in}}%
\pgfpathcurveto{\pgfqpoint{1.866394in}{2.098541in}}{\pgfqpoint{1.858494in}{2.101814in}}{\pgfqpoint{1.850257in}{2.101814in}}%
\pgfpathcurveto{\pgfqpoint{1.842021in}{2.101814in}}{\pgfqpoint{1.834121in}{2.098541in}}{\pgfqpoint{1.828297in}{2.092717in}}%
\pgfpathcurveto{\pgfqpoint{1.822473in}{2.086893in}}{\pgfqpoint{1.819201in}{2.078993in}}{\pgfqpoint{1.819201in}{2.070757in}}%
\pgfpathcurveto{\pgfqpoint{1.819201in}{2.062521in}}{\pgfqpoint{1.822473in}{2.054621in}}{\pgfqpoint{1.828297in}{2.048797in}}%
\pgfpathcurveto{\pgfqpoint{1.834121in}{2.042973in}}{\pgfqpoint{1.842021in}{2.039701in}}{\pgfqpoint{1.850257in}{2.039701in}}%
\pgfpathclose%
\pgfusepath{stroke,fill}%
\end{pgfscope}%
\begin{pgfscope}%
\pgfpathrectangle{\pgfqpoint{0.100000in}{0.212622in}}{\pgfqpoint{3.696000in}{3.696000in}}%
\pgfusepath{clip}%
\pgfsetbuttcap%
\pgfsetroundjoin%
\definecolor{currentfill}{rgb}{0.121569,0.466667,0.705882}%
\pgfsetfillcolor{currentfill}%
\pgfsetfillopacity{0.510369}%
\pgfsetlinewidth{1.003750pt}%
\definecolor{currentstroke}{rgb}{0.121569,0.466667,0.705882}%
\pgfsetstrokecolor{currentstroke}%
\pgfsetstrokeopacity{0.510369}%
\pgfsetdash{}{0pt}%
\pgfpathmoveto{\pgfqpoint{2.622304in}{2.507236in}}%
\pgfpathcurveto{\pgfqpoint{2.630540in}{2.507236in}}{\pgfqpoint{2.638440in}{2.510508in}}{\pgfqpoint{2.644264in}{2.516332in}}%
\pgfpathcurveto{\pgfqpoint{2.650088in}{2.522156in}}{\pgfqpoint{2.653360in}{2.530056in}}{\pgfqpoint{2.653360in}{2.538293in}}%
\pgfpathcurveto{\pgfqpoint{2.653360in}{2.546529in}}{\pgfqpoint{2.650088in}{2.554429in}}{\pgfqpoint{2.644264in}{2.560253in}}%
\pgfpathcurveto{\pgfqpoint{2.638440in}{2.566077in}}{\pgfqpoint{2.630540in}{2.569349in}}{\pgfqpoint{2.622304in}{2.569349in}}%
\pgfpathcurveto{\pgfqpoint{2.614067in}{2.569349in}}{\pgfqpoint{2.606167in}{2.566077in}}{\pgfqpoint{2.600343in}{2.560253in}}%
\pgfpathcurveto{\pgfqpoint{2.594520in}{2.554429in}}{\pgfqpoint{2.591247in}{2.546529in}}{\pgfqpoint{2.591247in}{2.538293in}}%
\pgfpathcurveto{\pgfqpoint{2.591247in}{2.530056in}}{\pgfqpoint{2.594520in}{2.522156in}}{\pgfqpoint{2.600343in}{2.516332in}}%
\pgfpathcurveto{\pgfqpoint{2.606167in}{2.510508in}}{\pgfqpoint{2.614067in}{2.507236in}}{\pgfqpoint{2.622304in}{2.507236in}}%
\pgfpathclose%
\pgfusepath{stroke,fill}%
\end{pgfscope}%
\begin{pgfscope}%
\pgfpathrectangle{\pgfqpoint{0.100000in}{0.212622in}}{\pgfqpoint{3.696000in}{3.696000in}}%
\pgfusepath{clip}%
\pgfsetbuttcap%
\pgfsetroundjoin%
\definecolor{currentfill}{rgb}{0.121569,0.466667,0.705882}%
\pgfsetfillcolor{currentfill}%
\pgfsetfillopacity{0.510761}%
\pgfsetlinewidth{1.003750pt}%
\definecolor{currentstroke}{rgb}{0.121569,0.466667,0.705882}%
\pgfsetstrokecolor{currentstroke}%
\pgfsetstrokeopacity{0.510761}%
\pgfsetdash{}{0pt}%
\pgfpathmoveto{\pgfqpoint{2.636652in}{2.518628in}}%
\pgfpathcurveto{\pgfqpoint{2.644888in}{2.518628in}}{\pgfqpoint{2.652788in}{2.521900in}}{\pgfqpoint{2.658612in}{2.527724in}}%
\pgfpathcurveto{\pgfqpoint{2.664436in}{2.533548in}}{\pgfqpoint{2.667708in}{2.541448in}}{\pgfqpoint{2.667708in}{2.549684in}}%
\pgfpathcurveto{\pgfqpoint{2.667708in}{2.557921in}}{\pgfqpoint{2.664436in}{2.565821in}}{\pgfqpoint{2.658612in}{2.571645in}}%
\pgfpathcurveto{\pgfqpoint{2.652788in}{2.577469in}}{\pgfqpoint{2.644888in}{2.580741in}}{\pgfqpoint{2.636652in}{2.580741in}}%
\pgfpathcurveto{\pgfqpoint{2.628416in}{2.580741in}}{\pgfqpoint{2.620516in}{2.577469in}}{\pgfqpoint{2.614692in}{2.571645in}}%
\pgfpathcurveto{\pgfqpoint{2.608868in}{2.565821in}}{\pgfqpoint{2.605595in}{2.557921in}}{\pgfqpoint{2.605595in}{2.549684in}}%
\pgfpathcurveto{\pgfqpoint{2.605595in}{2.541448in}}{\pgfqpoint{2.608868in}{2.533548in}}{\pgfqpoint{2.614692in}{2.527724in}}%
\pgfpathcurveto{\pgfqpoint{2.620516in}{2.521900in}}{\pgfqpoint{2.628416in}{2.518628in}}{\pgfqpoint{2.636652in}{2.518628in}}%
\pgfpathclose%
\pgfusepath{stroke,fill}%
\end{pgfscope}%
\begin{pgfscope}%
\pgfpathrectangle{\pgfqpoint{0.100000in}{0.212622in}}{\pgfqpoint{3.696000in}{3.696000in}}%
\pgfusepath{clip}%
\pgfsetbuttcap%
\pgfsetroundjoin%
\definecolor{currentfill}{rgb}{0.121569,0.466667,0.705882}%
\pgfsetfillcolor{currentfill}%
\pgfsetfillopacity{0.510792}%
\pgfsetlinewidth{1.003750pt}%
\definecolor{currentstroke}{rgb}{0.121569,0.466667,0.705882}%
\pgfsetstrokecolor{currentstroke}%
\pgfsetstrokeopacity{0.510792}%
\pgfsetdash{}{0pt}%
\pgfpathmoveto{\pgfqpoint{2.484330in}{2.394293in}}%
\pgfpathcurveto{\pgfqpoint{2.492566in}{2.394293in}}{\pgfqpoint{2.500466in}{2.397565in}}{\pgfqpoint{2.506290in}{2.403389in}}%
\pgfpathcurveto{\pgfqpoint{2.512114in}{2.409213in}}{\pgfqpoint{2.515386in}{2.417113in}}{\pgfqpoint{2.515386in}{2.425349in}}%
\pgfpathcurveto{\pgfqpoint{2.515386in}{2.433586in}}{\pgfqpoint{2.512114in}{2.441486in}}{\pgfqpoint{2.506290in}{2.447310in}}%
\pgfpathcurveto{\pgfqpoint{2.500466in}{2.453134in}}{\pgfqpoint{2.492566in}{2.456406in}}{\pgfqpoint{2.484330in}{2.456406in}}%
\pgfpathcurveto{\pgfqpoint{2.476093in}{2.456406in}}{\pgfqpoint{2.468193in}{2.453134in}}{\pgfqpoint{2.462369in}{2.447310in}}%
\pgfpathcurveto{\pgfqpoint{2.456546in}{2.441486in}}{\pgfqpoint{2.453273in}{2.433586in}}{\pgfqpoint{2.453273in}{2.425349in}}%
\pgfpathcurveto{\pgfqpoint{2.453273in}{2.417113in}}{\pgfqpoint{2.456546in}{2.409213in}}{\pgfqpoint{2.462369in}{2.403389in}}%
\pgfpathcurveto{\pgfqpoint{2.468193in}{2.397565in}}{\pgfqpoint{2.476093in}{2.394293in}}{\pgfqpoint{2.484330in}{2.394293in}}%
\pgfpathclose%
\pgfusepath{stroke,fill}%
\end{pgfscope}%
\begin{pgfscope}%
\pgfpathrectangle{\pgfqpoint{0.100000in}{0.212622in}}{\pgfqpoint{3.696000in}{3.696000in}}%
\pgfusepath{clip}%
\pgfsetbuttcap%
\pgfsetroundjoin%
\definecolor{currentfill}{rgb}{0.121569,0.466667,0.705882}%
\pgfsetfillcolor{currentfill}%
\pgfsetfillopacity{0.511231}%
\pgfsetlinewidth{1.003750pt}%
\definecolor{currentstroke}{rgb}{0.121569,0.466667,0.705882}%
\pgfsetstrokecolor{currentstroke}%
\pgfsetstrokeopacity{0.511231}%
\pgfsetdash{}{0pt}%
\pgfpathmoveto{\pgfqpoint{1.624551in}{1.880669in}}%
\pgfpathcurveto{\pgfqpoint{1.632788in}{1.880669in}}{\pgfqpoint{1.640688in}{1.883942in}}{\pgfqpoint{1.646512in}{1.889766in}}%
\pgfpathcurveto{\pgfqpoint{1.652336in}{1.895590in}}{\pgfqpoint{1.655608in}{1.903490in}}{\pgfqpoint{1.655608in}{1.911726in}}%
\pgfpathcurveto{\pgfqpoint{1.655608in}{1.919962in}}{\pgfqpoint{1.652336in}{1.927862in}}{\pgfqpoint{1.646512in}{1.933686in}}%
\pgfpathcurveto{\pgfqpoint{1.640688in}{1.939510in}}{\pgfqpoint{1.632788in}{1.942782in}}{\pgfqpoint{1.624551in}{1.942782in}}%
\pgfpathcurveto{\pgfqpoint{1.616315in}{1.942782in}}{\pgfqpoint{1.608415in}{1.939510in}}{\pgfqpoint{1.602591in}{1.933686in}}%
\pgfpathcurveto{\pgfqpoint{1.596767in}{1.927862in}}{\pgfqpoint{1.593495in}{1.919962in}}{\pgfqpoint{1.593495in}{1.911726in}}%
\pgfpathcurveto{\pgfqpoint{1.593495in}{1.903490in}}{\pgfqpoint{1.596767in}{1.895590in}}{\pgfqpoint{1.602591in}{1.889766in}}%
\pgfpathcurveto{\pgfqpoint{1.608415in}{1.883942in}}{\pgfqpoint{1.616315in}{1.880669in}}{\pgfqpoint{1.624551in}{1.880669in}}%
\pgfpathclose%
\pgfusepath{stroke,fill}%
\end{pgfscope}%
\begin{pgfscope}%
\pgfpathrectangle{\pgfqpoint{0.100000in}{0.212622in}}{\pgfqpoint{3.696000in}{3.696000in}}%
\pgfusepath{clip}%
\pgfsetbuttcap%
\pgfsetroundjoin%
\definecolor{currentfill}{rgb}{0.121569,0.466667,0.705882}%
\pgfsetfillcolor{currentfill}%
\pgfsetfillopacity{0.513230}%
\pgfsetlinewidth{1.003750pt}%
\definecolor{currentstroke}{rgb}{0.121569,0.466667,0.705882}%
\pgfsetstrokecolor{currentstroke}%
\pgfsetstrokeopacity{0.513230}%
\pgfsetdash{}{0pt}%
\pgfpathmoveto{\pgfqpoint{2.536840in}{2.438363in}}%
\pgfpathcurveto{\pgfqpoint{2.545076in}{2.438363in}}{\pgfqpoint{2.552976in}{2.441635in}}{\pgfqpoint{2.558800in}{2.447459in}}%
\pgfpathcurveto{\pgfqpoint{2.564624in}{2.453283in}}{\pgfqpoint{2.567896in}{2.461183in}}{\pgfqpoint{2.567896in}{2.469420in}}%
\pgfpathcurveto{\pgfqpoint{2.567896in}{2.477656in}}{\pgfqpoint{2.564624in}{2.485556in}}{\pgfqpoint{2.558800in}{2.491380in}}%
\pgfpathcurveto{\pgfqpoint{2.552976in}{2.497204in}}{\pgfqpoint{2.545076in}{2.500476in}}{\pgfqpoint{2.536840in}{2.500476in}}%
\pgfpathcurveto{\pgfqpoint{2.528603in}{2.500476in}}{\pgfqpoint{2.520703in}{2.497204in}}{\pgfqpoint{2.514879in}{2.491380in}}%
\pgfpathcurveto{\pgfqpoint{2.509055in}{2.485556in}}{\pgfqpoint{2.505783in}{2.477656in}}{\pgfqpoint{2.505783in}{2.469420in}}%
\pgfpathcurveto{\pgfqpoint{2.505783in}{2.461183in}}{\pgfqpoint{2.509055in}{2.453283in}}{\pgfqpoint{2.514879in}{2.447459in}}%
\pgfpathcurveto{\pgfqpoint{2.520703in}{2.441635in}}{\pgfqpoint{2.528603in}{2.438363in}}{\pgfqpoint{2.536840in}{2.438363in}}%
\pgfpathclose%
\pgfusepath{stroke,fill}%
\end{pgfscope}%
\begin{pgfscope}%
\pgfpathrectangle{\pgfqpoint{0.100000in}{0.212622in}}{\pgfqpoint{3.696000in}{3.696000in}}%
\pgfusepath{clip}%
\pgfsetbuttcap%
\pgfsetroundjoin%
\definecolor{currentfill}{rgb}{0.121569,0.466667,0.705882}%
\pgfsetfillcolor{currentfill}%
\pgfsetfillopacity{0.513701}%
\pgfsetlinewidth{1.003750pt}%
\definecolor{currentstroke}{rgb}{0.121569,0.466667,0.705882}%
\pgfsetstrokecolor{currentstroke}%
\pgfsetstrokeopacity{0.513701}%
\pgfsetdash{}{0pt}%
\pgfpathmoveto{\pgfqpoint{2.538645in}{2.437864in}}%
\pgfpathcurveto{\pgfqpoint{2.546882in}{2.437864in}}{\pgfqpoint{2.554782in}{2.441136in}}{\pgfqpoint{2.560606in}{2.446960in}}%
\pgfpathcurveto{\pgfqpoint{2.566430in}{2.452784in}}{\pgfqpoint{2.569702in}{2.460684in}}{\pgfqpoint{2.569702in}{2.468920in}}%
\pgfpathcurveto{\pgfqpoint{2.569702in}{2.477157in}}{\pgfqpoint{2.566430in}{2.485057in}}{\pgfqpoint{2.560606in}{2.490881in}}%
\pgfpathcurveto{\pgfqpoint{2.554782in}{2.496705in}}{\pgfqpoint{2.546882in}{2.499977in}}{\pgfqpoint{2.538645in}{2.499977in}}%
\pgfpathcurveto{\pgfqpoint{2.530409in}{2.499977in}}{\pgfqpoint{2.522509in}{2.496705in}}{\pgfqpoint{2.516685in}{2.490881in}}%
\pgfpathcurveto{\pgfqpoint{2.510861in}{2.485057in}}{\pgfqpoint{2.507589in}{2.477157in}}{\pgfqpoint{2.507589in}{2.468920in}}%
\pgfpathcurveto{\pgfqpoint{2.507589in}{2.460684in}}{\pgfqpoint{2.510861in}{2.452784in}}{\pgfqpoint{2.516685in}{2.446960in}}%
\pgfpathcurveto{\pgfqpoint{2.522509in}{2.441136in}}{\pgfqpoint{2.530409in}{2.437864in}}{\pgfqpoint{2.538645in}{2.437864in}}%
\pgfpathclose%
\pgfusepath{stroke,fill}%
\end{pgfscope}%
\begin{pgfscope}%
\pgfpathrectangle{\pgfqpoint{0.100000in}{0.212622in}}{\pgfqpoint{3.696000in}{3.696000in}}%
\pgfusepath{clip}%
\pgfsetbuttcap%
\pgfsetroundjoin%
\definecolor{currentfill}{rgb}{0.121569,0.466667,0.705882}%
\pgfsetfillcolor{currentfill}%
\pgfsetfillopacity{0.514559}%
\pgfsetlinewidth{1.003750pt}%
\definecolor{currentstroke}{rgb}{0.121569,0.466667,0.705882}%
\pgfsetstrokecolor{currentstroke}%
\pgfsetstrokeopacity{0.514559}%
\pgfsetdash{}{0pt}%
\pgfpathmoveto{\pgfqpoint{2.530063in}{2.432141in}}%
\pgfpathcurveto{\pgfqpoint{2.538299in}{2.432141in}}{\pgfqpoint{2.546199in}{2.435413in}}{\pgfqpoint{2.552023in}{2.441237in}}%
\pgfpathcurveto{\pgfqpoint{2.557847in}{2.447061in}}{\pgfqpoint{2.561119in}{2.454961in}}{\pgfqpoint{2.561119in}{2.463197in}}%
\pgfpathcurveto{\pgfqpoint{2.561119in}{2.471433in}}{\pgfqpoint{2.557847in}{2.479333in}}{\pgfqpoint{2.552023in}{2.485157in}}%
\pgfpathcurveto{\pgfqpoint{2.546199in}{2.490981in}}{\pgfqpoint{2.538299in}{2.494254in}}{\pgfqpoint{2.530063in}{2.494254in}}%
\pgfpathcurveto{\pgfqpoint{2.521826in}{2.494254in}}{\pgfqpoint{2.513926in}{2.490981in}}{\pgfqpoint{2.508102in}{2.485157in}}%
\pgfpathcurveto{\pgfqpoint{2.502278in}{2.479333in}}{\pgfqpoint{2.499006in}{2.471433in}}{\pgfqpoint{2.499006in}{2.463197in}}%
\pgfpathcurveto{\pgfqpoint{2.499006in}{2.454961in}}{\pgfqpoint{2.502278in}{2.447061in}}{\pgfqpoint{2.508102in}{2.441237in}}%
\pgfpathcurveto{\pgfqpoint{2.513926in}{2.435413in}}{\pgfqpoint{2.521826in}{2.432141in}}{\pgfqpoint{2.530063in}{2.432141in}}%
\pgfpathclose%
\pgfusepath{stroke,fill}%
\end{pgfscope}%
\begin{pgfscope}%
\pgfpathrectangle{\pgfqpoint{0.100000in}{0.212622in}}{\pgfqpoint{3.696000in}{3.696000in}}%
\pgfusepath{clip}%
\pgfsetbuttcap%
\pgfsetroundjoin%
\definecolor{currentfill}{rgb}{0.121569,0.466667,0.705882}%
\pgfsetfillcolor{currentfill}%
\pgfsetfillopacity{0.514573}%
\pgfsetlinewidth{1.003750pt}%
\definecolor{currentstroke}{rgb}{0.121569,0.466667,0.705882}%
\pgfsetstrokecolor{currentstroke}%
\pgfsetstrokeopacity{0.514573}%
\pgfsetdash{}{0pt}%
\pgfpathmoveto{\pgfqpoint{2.528008in}{2.431754in}}%
\pgfpathcurveto{\pgfqpoint{2.536244in}{2.431754in}}{\pgfqpoint{2.544144in}{2.435026in}}{\pgfqpoint{2.549968in}{2.440850in}}%
\pgfpathcurveto{\pgfqpoint{2.555792in}{2.446674in}}{\pgfqpoint{2.559064in}{2.454574in}}{\pgfqpoint{2.559064in}{2.462810in}}%
\pgfpathcurveto{\pgfqpoint{2.559064in}{2.471046in}}{\pgfqpoint{2.555792in}{2.478946in}}{\pgfqpoint{2.549968in}{2.484770in}}%
\pgfpathcurveto{\pgfqpoint{2.544144in}{2.490594in}}{\pgfqpoint{2.536244in}{2.493867in}}{\pgfqpoint{2.528008in}{2.493867in}}%
\pgfpathcurveto{\pgfqpoint{2.519772in}{2.493867in}}{\pgfqpoint{2.511871in}{2.490594in}}{\pgfqpoint{2.506048in}{2.484770in}}%
\pgfpathcurveto{\pgfqpoint{2.500224in}{2.478946in}}{\pgfqpoint{2.496951in}{2.471046in}}{\pgfqpoint{2.496951in}{2.462810in}}%
\pgfpathcurveto{\pgfqpoint{2.496951in}{2.454574in}}{\pgfqpoint{2.500224in}{2.446674in}}{\pgfqpoint{2.506048in}{2.440850in}}%
\pgfpathcurveto{\pgfqpoint{2.511871in}{2.435026in}}{\pgfqpoint{2.519772in}{2.431754in}}{\pgfqpoint{2.528008in}{2.431754in}}%
\pgfpathclose%
\pgfusepath{stroke,fill}%
\end{pgfscope}%
\begin{pgfscope}%
\pgfpathrectangle{\pgfqpoint{0.100000in}{0.212622in}}{\pgfqpoint{3.696000in}{3.696000in}}%
\pgfusepath{clip}%
\pgfsetbuttcap%
\pgfsetroundjoin%
\definecolor{currentfill}{rgb}{0.121569,0.466667,0.705882}%
\pgfsetfillcolor{currentfill}%
\pgfsetfillopacity{0.515335}%
\pgfsetlinewidth{1.003750pt}%
\definecolor{currentstroke}{rgb}{0.121569,0.466667,0.705882}%
\pgfsetstrokecolor{currentstroke}%
\pgfsetstrokeopacity{0.515335}%
\pgfsetdash{}{0pt}%
\pgfpathmoveto{\pgfqpoint{2.522815in}{2.427319in}}%
\pgfpathcurveto{\pgfqpoint{2.531052in}{2.427319in}}{\pgfqpoint{2.538952in}{2.430592in}}{\pgfqpoint{2.544776in}{2.436416in}}%
\pgfpathcurveto{\pgfqpoint{2.550600in}{2.442239in}}{\pgfqpoint{2.553872in}{2.450140in}}{\pgfqpoint{2.553872in}{2.458376in}}%
\pgfpathcurveto{\pgfqpoint{2.553872in}{2.466612in}}{\pgfqpoint{2.550600in}{2.474512in}}{\pgfqpoint{2.544776in}{2.480336in}}%
\pgfpathcurveto{\pgfqpoint{2.538952in}{2.486160in}}{\pgfqpoint{2.531052in}{2.489432in}}{\pgfqpoint{2.522815in}{2.489432in}}%
\pgfpathcurveto{\pgfqpoint{2.514579in}{2.489432in}}{\pgfqpoint{2.506679in}{2.486160in}}{\pgfqpoint{2.500855in}{2.480336in}}%
\pgfpathcurveto{\pgfqpoint{2.495031in}{2.474512in}}{\pgfqpoint{2.491759in}{2.466612in}}{\pgfqpoint{2.491759in}{2.458376in}}%
\pgfpathcurveto{\pgfqpoint{2.491759in}{2.450140in}}{\pgfqpoint{2.495031in}{2.442239in}}{\pgfqpoint{2.500855in}{2.436416in}}%
\pgfpathcurveto{\pgfqpoint{2.506679in}{2.430592in}}{\pgfqpoint{2.514579in}{2.427319in}}{\pgfqpoint{2.522815in}{2.427319in}}%
\pgfpathclose%
\pgfusepath{stroke,fill}%
\end{pgfscope}%
\begin{pgfscope}%
\pgfpathrectangle{\pgfqpoint{0.100000in}{0.212622in}}{\pgfqpoint{3.696000in}{3.696000in}}%
\pgfusepath{clip}%
\pgfsetbuttcap%
\pgfsetroundjoin%
\definecolor{currentfill}{rgb}{0.121569,0.466667,0.705882}%
\pgfsetfillcolor{currentfill}%
\pgfsetfillopacity{0.515394}%
\pgfsetlinewidth{1.003750pt}%
\definecolor{currentstroke}{rgb}{0.121569,0.466667,0.705882}%
\pgfsetstrokecolor{currentstroke}%
\pgfsetstrokeopacity{0.515394}%
\pgfsetdash{}{0pt}%
\pgfpathmoveto{\pgfqpoint{2.523796in}{2.428002in}}%
\pgfpathcurveto{\pgfqpoint{2.532032in}{2.428002in}}{\pgfqpoint{2.539932in}{2.431274in}}{\pgfqpoint{2.545756in}{2.437098in}}%
\pgfpathcurveto{\pgfqpoint{2.551580in}{2.442922in}}{\pgfqpoint{2.554852in}{2.450822in}}{\pgfqpoint{2.554852in}{2.459058in}}%
\pgfpathcurveto{\pgfqpoint{2.554852in}{2.467295in}}{\pgfqpoint{2.551580in}{2.475195in}}{\pgfqpoint{2.545756in}{2.481019in}}%
\pgfpathcurveto{\pgfqpoint{2.539932in}{2.486843in}}{\pgfqpoint{2.532032in}{2.490115in}}{\pgfqpoint{2.523796in}{2.490115in}}%
\pgfpathcurveto{\pgfqpoint{2.515559in}{2.490115in}}{\pgfqpoint{2.507659in}{2.486843in}}{\pgfqpoint{2.501835in}{2.481019in}}%
\pgfpathcurveto{\pgfqpoint{2.496011in}{2.475195in}}{\pgfqpoint{2.492739in}{2.467295in}}{\pgfqpoint{2.492739in}{2.459058in}}%
\pgfpathcurveto{\pgfqpoint{2.492739in}{2.450822in}}{\pgfqpoint{2.496011in}{2.442922in}}{\pgfqpoint{2.501835in}{2.437098in}}%
\pgfpathcurveto{\pgfqpoint{2.507659in}{2.431274in}}{\pgfqpoint{2.515559in}{2.428002in}}{\pgfqpoint{2.523796in}{2.428002in}}%
\pgfpathclose%
\pgfusepath{stroke,fill}%
\end{pgfscope}%
\begin{pgfscope}%
\pgfpathrectangle{\pgfqpoint{0.100000in}{0.212622in}}{\pgfqpoint{3.696000in}{3.696000in}}%
\pgfusepath{clip}%
\pgfsetbuttcap%
\pgfsetroundjoin%
\definecolor{currentfill}{rgb}{0.121569,0.466667,0.705882}%
\pgfsetfillcolor{currentfill}%
\pgfsetfillopacity{0.515727}%
\pgfsetlinewidth{1.003750pt}%
\definecolor{currentstroke}{rgb}{0.121569,0.466667,0.705882}%
\pgfsetstrokecolor{currentstroke}%
\pgfsetstrokeopacity{0.515727}%
\pgfsetdash{}{0pt}%
\pgfpathmoveto{\pgfqpoint{2.519604in}{2.424654in}}%
\pgfpathcurveto{\pgfqpoint{2.527840in}{2.424654in}}{\pgfqpoint{2.535740in}{2.427927in}}{\pgfqpoint{2.541564in}{2.433751in}}%
\pgfpathcurveto{\pgfqpoint{2.547388in}{2.439575in}}{\pgfqpoint{2.550660in}{2.447475in}}{\pgfqpoint{2.550660in}{2.455711in}}%
\pgfpathcurveto{\pgfqpoint{2.550660in}{2.463947in}}{\pgfqpoint{2.547388in}{2.471847in}}{\pgfqpoint{2.541564in}{2.477671in}}%
\pgfpathcurveto{\pgfqpoint{2.535740in}{2.483495in}}{\pgfqpoint{2.527840in}{2.486767in}}{\pgfqpoint{2.519604in}{2.486767in}}%
\pgfpathcurveto{\pgfqpoint{2.511368in}{2.486767in}}{\pgfqpoint{2.503468in}{2.483495in}}{\pgfqpoint{2.497644in}{2.477671in}}%
\pgfpathcurveto{\pgfqpoint{2.491820in}{2.471847in}}{\pgfqpoint{2.488547in}{2.463947in}}{\pgfqpoint{2.488547in}{2.455711in}}%
\pgfpathcurveto{\pgfqpoint{2.488547in}{2.447475in}}{\pgfqpoint{2.491820in}{2.439575in}}{\pgfqpoint{2.497644in}{2.433751in}}%
\pgfpathcurveto{\pgfqpoint{2.503468in}{2.427927in}}{\pgfqpoint{2.511368in}{2.424654in}}{\pgfqpoint{2.519604in}{2.424654in}}%
\pgfpathclose%
\pgfusepath{stroke,fill}%
\end{pgfscope}%
\begin{pgfscope}%
\pgfpathrectangle{\pgfqpoint{0.100000in}{0.212622in}}{\pgfqpoint{3.696000in}{3.696000in}}%
\pgfusepath{clip}%
\pgfsetbuttcap%
\pgfsetroundjoin%
\definecolor{currentfill}{rgb}{0.121569,0.466667,0.705882}%
\pgfsetfillcolor{currentfill}%
\pgfsetfillopacity{0.516299}%
\pgfsetlinewidth{1.003750pt}%
\definecolor{currentstroke}{rgb}{0.121569,0.466667,0.705882}%
\pgfsetstrokecolor{currentstroke}%
\pgfsetstrokeopacity{0.516299}%
\pgfsetdash{}{0pt}%
\pgfpathmoveto{\pgfqpoint{2.536352in}{2.435738in}}%
\pgfpathcurveto{\pgfqpoint{2.544589in}{2.435738in}}{\pgfqpoint{2.552489in}{2.439010in}}{\pgfqpoint{2.558313in}{2.444834in}}%
\pgfpathcurveto{\pgfqpoint{2.564136in}{2.450658in}}{\pgfqpoint{2.567409in}{2.458558in}}{\pgfqpoint{2.567409in}{2.466794in}}%
\pgfpathcurveto{\pgfqpoint{2.567409in}{2.475031in}}{\pgfqpoint{2.564136in}{2.482931in}}{\pgfqpoint{2.558313in}{2.488755in}}%
\pgfpathcurveto{\pgfqpoint{2.552489in}{2.494578in}}{\pgfqpoint{2.544589in}{2.497851in}}{\pgfqpoint{2.536352in}{2.497851in}}%
\pgfpathcurveto{\pgfqpoint{2.528116in}{2.497851in}}{\pgfqpoint{2.520216in}{2.494578in}}{\pgfqpoint{2.514392in}{2.488755in}}%
\pgfpathcurveto{\pgfqpoint{2.508568in}{2.482931in}}{\pgfqpoint{2.505296in}{2.475031in}}{\pgfqpoint{2.505296in}{2.466794in}}%
\pgfpathcurveto{\pgfqpoint{2.505296in}{2.458558in}}{\pgfqpoint{2.508568in}{2.450658in}}{\pgfqpoint{2.514392in}{2.444834in}}%
\pgfpathcurveto{\pgfqpoint{2.520216in}{2.439010in}}{\pgfqpoint{2.528116in}{2.435738in}}{\pgfqpoint{2.536352in}{2.435738in}}%
\pgfpathclose%
\pgfusepath{stroke,fill}%
\end{pgfscope}%
\begin{pgfscope}%
\pgfpathrectangle{\pgfqpoint{0.100000in}{0.212622in}}{\pgfqpoint{3.696000in}{3.696000in}}%
\pgfusepath{clip}%
\pgfsetbuttcap%
\pgfsetroundjoin%
\definecolor{currentfill}{rgb}{0.121569,0.466667,0.705882}%
\pgfsetfillcolor{currentfill}%
\pgfsetfillopacity{0.516554}%
\pgfsetlinewidth{1.003750pt}%
\definecolor{currentstroke}{rgb}{0.121569,0.466667,0.705882}%
\pgfsetstrokecolor{currentstroke}%
\pgfsetstrokeopacity{0.516554}%
\pgfsetdash{}{0pt}%
\pgfpathmoveto{\pgfqpoint{2.515444in}{2.420472in}}%
\pgfpathcurveto{\pgfqpoint{2.523681in}{2.420472in}}{\pgfqpoint{2.531581in}{2.423745in}}{\pgfqpoint{2.537405in}{2.429569in}}%
\pgfpathcurveto{\pgfqpoint{2.543229in}{2.435393in}}{\pgfqpoint{2.546501in}{2.443293in}}{\pgfqpoint{2.546501in}{2.451529in}}%
\pgfpathcurveto{\pgfqpoint{2.546501in}{2.459765in}}{\pgfqpoint{2.543229in}{2.467665in}}{\pgfqpoint{2.537405in}{2.473489in}}%
\pgfpathcurveto{\pgfqpoint{2.531581in}{2.479313in}}{\pgfqpoint{2.523681in}{2.482585in}}{\pgfqpoint{2.515444in}{2.482585in}}%
\pgfpathcurveto{\pgfqpoint{2.507208in}{2.482585in}}{\pgfqpoint{2.499308in}{2.479313in}}{\pgfqpoint{2.493484in}{2.473489in}}%
\pgfpathcurveto{\pgfqpoint{2.487660in}{2.467665in}}{\pgfqpoint{2.484388in}{2.459765in}}{\pgfqpoint{2.484388in}{2.451529in}}%
\pgfpathcurveto{\pgfqpoint{2.484388in}{2.443293in}}{\pgfqpoint{2.487660in}{2.435393in}}{\pgfqpoint{2.493484in}{2.429569in}}%
\pgfpathcurveto{\pgfqpoint{2.499308in}{2.423745in}}{\pgfqpoint{2.507208in}{2.420472in}}{\pgfqpoint{2.515444in}{2.420472in}}%
\pgfpathclose%
\pgfusepath{stroke,fill}%
\end{pgfscope}%
\begin{pgfscope}%
\pgfpathrectangle{\pgfqpoint{0.100000in}{0.212622in}}{\pgfqpoint{3.696000in}{3.696000in}}%
\pgfusepath{clip}%
\pgfsetbuttcap%
\pgfsetroundjoin%
\definecolor{currentfill}{rgb}{0.121569,0.466667,0.705882}%
\pgfsetfillcolor{currentfill}%
\pgfsetfillopacity{0.517040}%
\pgfsetlinewidth{1.003750pt}%
\definecolor{currentstroke}{rgb}{0.121569,0.466667,0.705882}%
\pgfsetstrokecolor{currentstroke}%
\pgfsetstrokeopacity{0.517040}%
\pgfsetdash{}{0pt}%
\pgfpathmoveto{\pgfqpoint{2.547463in}{2.443533in}}%
\pgfpathcurveto{\pgfqpoint{2.555699in}{2.443533in}}{\pgfqpoint{2.563599in}{2.446805in}}{\pgfqpoint{2.569423in}{2.452629in}}%
\pgfpathcurveto{\pgfqpoint{2.575247in}{2.458453in}}{\pgfqpoint{2.578519in}{2.466353in}}{\pgfqpoint{2.578519in}{2.474589in}}%
\pgfpathcurveto{\pgfqpoint{2.578519in}{2.482826in}}{\pgfqpoint{2.575247in}{2.490726in}}{\pgfqpoint{2.569423in}{2.496550in}}%
\pgfpathcurveto{\pgfqpoint{2.563599in}{2.502374in}}{\pgfqpoint{2.555699in}{2.505646in}}{\pgfqpoint{2.547463in}{2.505646in}}%
\pgfpathcurveto{\pgfqpoint{2.539226in}{2.505646in}}{\pgfqpoint{2.531326in}{2.502374in}}{\pgfqpoint{2.525502in}{2.496550in}}%
\pgfpathcurveto{\pgfqpoint{2.519678in}{2.490726in}}{\pgfqpoint{2.516406in}{2.482826in}}{\pgfqpoint{2.516406in}{2.474589in}}%
\pgfpathcurveto{\pgfqpoint{2.516406in}{2.466353in}}{\pgfqpoint{2.519678in}{2.458453in}}{\pgfqpoint{2.525502in}{2.452629in}}%
\pgfpathcurveto{\pgfqpoint{2.531326in}{2.446805in}}{\pgfqpoint{2.539226in}{2.443533in}}{\pgfqpoint{2.547463in}{2.443533in}}%
\pgfpathclose%
\pgfusepath{stroke,fill}%
\end{pgfscope}%
\begin{pgfscope}%
\pgfpathrectangle{\pgfqpoint{0.100000in}{0.212622in}}{\pgfqpoint{3.696000in}{3.696000in}}%
\pgfusepath{clip}%
\pgfsetbuttcap%
\pgfsetroundjoin%
\definecolor{currentfill}{rgb}{0.121569,0.466667,0.705882}%
\pgfsetfillcolor{currentfill}%
\pgfsetfillopacity{0.517503}%
\pgfsetlinewidth{1.003750pt}%
\definecolor{currentstroke}{rgb}{0.121569,0.466667,0.705882}%
\pgfsetstrokecolor{currentstroke}%
\pgfsetstrokeopacity{0.517503}%
\pgfsetdash{}{0pt}%
\pgfpathmoveto{\pgfqpoint{2.653573in}{2.528878in}}%
\pgfpathcurveto{\pgfqpoint{2.661809in}{2.528878in}}{\pgfqpoint{2.669709in}{2.532150in}}{\pgfqpoint{2.675533in}{2.537974in}}%
\pgfpathcurveto{\pgfqpoint{2.681357in}{2.543798in}}{\pgfqpoint{2.684629in}{2.551698in}}{\pgfqpoint{2.684629in}{2.559934in}}%
\pgfpathcurveto{\pgfqpoint{2.684629in}{2.568171in}}{\pgfqpoint{2.681357in}{2.576071in}}{\pgfqpoint{2.675533in}{2.581895in}}%
\pgfpathcurveto{\pgfqpoint{2.669709in}{2.587718in}}{\pgfqpoint{2.661809in}{2.590991in}}{\pgfqpoint{2.653573in}{2.590991in}}%
\pgfpathcurveto{\pgfqpoint{2.645337in}{2.590991in}}{\pgfqpoint{2.637437in}{2.587718in}}{\pgfqpoint{2.631613in}{2.581895in}}%
\pgfpathcurveto{\pgfqpoint{2.625789in}{2.576071in}}{\pgfqpoint{2.622516in}{2.568171in}}{\pgfqpoint{2.622516in}{2.559934in}}%
\pgfpathcurveto{\pgfqpoint{2.622516in}{2.551698in}}{\pgfqpoint{2.625789in}{2.543798in}}{\pgfqpoint{2.631613in}{2.537974in}}%
\pgfpathcurveto{\pgfqpoint{2.637437in}{2.532150in}}{\pgfqpoint{2.645337in}{2.528878in}}{\pgfqpoint{2.653573in}{2.528878in}}%
\pgfpathclose%
\pgfusepath{stroke,fill}%
\end{pgfscope}%
\begin{pgfscope}%
\pgfpathrectangle{\pgfqpoint{0.100000in}{0.212622in}}{\pgfqpoint{3.696000in}{3.696000in}}%
\pgfusepath{clip}%
\pgfsetbuttcap%
\pgfsetroundjoin%
\definecolor{currentfill}{rgb}{0.121569,0.466667,0.705882}%
\pgfsetfillcolor{currentfill}%
\pgfsetfillopacity{0.517589}%
\pgfsetlinewidth{1.003750pt}%
\definecolor{currentstroke}{rgb}{0.121569,0.466667,0.705882}%
\pgfsetstrokecolor{currentstroke}%
\pgfsetstrokeopacity{0.517589}%
\pgfsetdash{}{0pt}%
\pgfpathmoveto{\pgfqpoint{2.509415in}{2.414890in}}%
\pgfpathcurveto{\pgfqpoint{2.517651in}{2.414890in}}{\pgfqpoint{2.525551in}{2.418163in}}{\pgfqpoint{2.531375in}{2.423987in}}%
\pgfpathcurveto{\pgfqpoint{2.537199in}{2.429811in}}{\pgfqpoint{2.540472in}{2.437711in}}{\pgfqpoint{2.540472in}{2.445947in}}%
\pgfpathcurveto{\pgfqpoint{2.540472in}{2.454183in}}{\pgfqpoint{2.537199in}{2.462083in}}{\pgfqpoint{2.531375in}{2.467907in}}%
\pgfpathcurveto{\pgfqpoint{2.525551in}{2.473731in}}{\pgfqpoint{2.517651in}{2.477003in}}{\pgfqpoint{2.509415in}{2.477003in}}%
\pgfpathcurveto{\pgfqpoint{2.501179in}{2.477003in}}{\pgfqpoint{2.493279in}{2.473731in}}{\pgfqpoint{2.487455in}{2.467907in}}%
\pgfpathcurveto{\pgfqpoint{2.481631in}{2.462083in}}{\pgfqpoint{2.478359in}{2.454183in}}{\pgfqpoint{2.478359in}{2.445947in}}%
\pgfpathcurveto{\pgfqpoint{2.478359in}{2.437711in}}{\pgfqpoint{2.481631in}{2.429811in}}{\pgfqpoint{2.487455in}{2.423987in}}%
\pgfpathcurveto{\pgfqpoint{2.493279in}{2.418163in}}{\pgfqpoint{2.501179in}{2.414890in}}{\pgfqpoint{2.509415in}{2.414890in}}%
\pgfpathclose%
\pgfusepath{stroke,fill}%
\end{pgfscope}%
\begin{pgfscope}%
\pgfpathrectangle{\pgfqpoint{0.100000in}{0.212622in}}{\pgfqpoint{3.696000in}{3.696000in}}%
\pgfusepath{clip}%
\pgfsetbuttcap%
\pgfsetroundjoin%
\definecolor{currentfill}{rgb}{0.121569,0.466667,0.705882}%
\pgfsetfillcolor{currentfill}%
\pgfsetfillopacity{0.518420}%
\pgfsetlinewidth{1.003750pt}%
\definecolor{currentstroke}{rgb}{0.121569,0.466667,0.705882}%
\pgfsetstrokecolor{currentstroke}%
\pgfsetstrokeopacity{0.518420}%
\pgfsetdash{}{0pt}%
\pgfpathmoveto{\pgfqpoint{2.553279in}{2.450936in}}%
\pgfpathcurveto{\pgfqpoint{2.561516in}{2.450936in}}{\pgfqpoint{2.569416in}{2.454208in}}{\pgfqpoint{2.575240in}{2.460032in}}%
\pgfpathcurveto{\pgfqpoint{2.581063in}{2.465856in}}{\pgfqpoint{2.584336in}{2.473756in}}{\pgfqpoint{2.584336in}{2.481993in}}%
\pgfpathcurveto{\pgfqpoint{2.584336in}{2.490229in}}{\pgfqpoint{2.581063in}{2.498129in}}{\pgfqpoint{2.575240in}{2.503953in}}%
\pgfpathcurveto{\pgfqpoint{2.569416in}{2.509777in}}{\pgfqpoint{2.561516in}{2.513049in}}{\pgfqpoint{2.553279in}{2.513049in}}%
\pgfpathcurveto{\pgfqpoint{2.545043in}{2.513049in}}{\pgfqpoint{2.537143in}{2.509777in}}{\pgfqpoint{2.531319in}{2.503953in}}%
\pgfpathcurveto{\pgfqpoint{2.525495in}{2.498129in}}{\pgfqpoint{2.522223in}{2.490229in}}{\pgfqpoint{2.522223in}{2.481993in}}%
\pgfpathcurveto{\pgfqpoint{2.522223in}{2.473756in}}{\pgfqpoint{2.525495in}{2.465856in}}{\pgfqpoint{2.531319in}{2.460032in}}%
\pgfpathcurveto{\pgfqpoint{2.537143in}{2.454208in}}{\pgfqpoint{2.545043in}{2.450936in}}{\pgfqpoint{2.553279in}{2.450936in}}%
\pgfpathclose%
\pgfusepath{stroke,fill}%
\end{pgfscope}%
\begin{pgfscope}%
\pgfpathrectangle{\pgfqpoint{0.100000in}{0.212622in}}{\pgfqpoint{3.696000in}{3.696000in}}%
\pgfusepath{clip}%
\pgfsetbuttcap%
\pgfsetroundjoin%
\definecolor{currentfill}{rgb}{0.121569,0.466667,0.705882}%
\pgfsetfillcolor{currentfill}%
\pgfsetfillopacity{0.518514}%
\pgfsetlinewidth{1.003750pt}%
\definecolor{currentstroke}{rgb}{0.121569,0.466667,0.705882}%
\pgfsetstrokecolor{currentstroke}%
\pgfsetstrokeopacity{0.518514}%
\pgfsetdash{}{0pt}%
\pgfpathmoveto{\pgfqpoint{2.535348in}{2.431437in}}%
\pgfpathcurveto{\pgfqpoint{2.543584in}{2.431437in}}{\pgfqpoint{2.551485in}{2.434709in}}{\pgfqpoint{2.557308in}{2.440533in}}%
\pgfpathcurveto{\pgfqpoint{2.563132in}{2.446357in}}{\pgfqpoint{2.566405in}{2.454257in}}{\pgfqpoint{2.566405in}{2.462494in}}%
\pgfpathcurveto{\pgfqpoint{2.566405in}{2.470730in}}{\pgfqpoint{2.563132in}{2.478630in}}{\pgfqpoint{2.557308in}{2.484454in}}%
\pgfpathcurveto{\pgfqpoint{2.551485in}{2.490278in}}{\pgfqpoint{2.543584in}{2.493550in}}{\pgfqpoint{2.535348in}{2.493550in}}%
\pgfpathcurveto{\pgfqpoint{2.527112in}{2.493550in}}{\pgfqpoint{2.519212in}{2.490278in}}{\pgfqpoint{2.513388in}{2.484454in}}%
\pgfpathcurveto{\pgfqpoint{2.507564in}{2.478630in}}{\pgfqpoint{2.504292in}{2.470730in}}{\pgfqpoint{2.504292in}{2.462494in}}%
\pgfpathcurveto{\pgfqpoint{2.504292in}{2.454257in}}{\pgfqpoint{2.507564in}{2.446357in}}{\pgfqpoint{2.513388in}{2.440533in}}%
\pgfpathcurveto{\pgfqpoint{2.519212in}{2.434709in}}{\pgfqpoint{2.527112in}{2.431437in}}{\pgfqpoint{2.535348in}{2.431437in}}%
\pgfpathclose%
\pgfusepath{stroke,fill}%
\end{pgfscope}%
\begin{pgfscope}%
\pgfpathrectangle{\pgfqpoint{0.100000in}{0.212622in}}{\pgfqpoint{3.696000in}{3.696000in}}%
\pgfusepath{clip}%
\pgfsetbuttcap%
\pgfsetroundjoin%
\definecolor{currentfill}{rgb}{0.121569,0.466667,0.705882}%
\pgfsetfillcolor{currentfill}%
\pgfsetfillopacity{0.518611}%
\pgfsetlinewidth{1.003750pt}%
\definecolor{currentstroke}{rgb}{0.121569,0.466667,0.705882}%
\pgfsetstrokecolor{currentstroke}%
\pgfsetstrokeopacity{0.518611}%
\pgfsetdash{}{0pt}%
\pgfpathmoveto{\pgfqpoint{2.585813in}{2.475578in}}%
\pgfpathcurveto{\pgfqpoint{2.594049in}{2.475578in}}{\pgfqpoint{2.601949in}{2.478850in}}{\pgfqpoint{2.607773in}{2.484674in}}%
\pgfpathcurveto{\pgfqpoint{2.613597in}{2.490498in}}{\pgfqpoint{2.616869in}{2.498398in}}{\pgfqpoint{2.616869in}{2.506635in}}%
\pgfpathcurveto{\pgfqpoint{2.616869in}{2.514871in}}{\pgfqpoint{2.613597in}{2.522771in}}{\pgfqpoint{2.607773in}{2.528595in}}%
\pgfpathcurveto{\pgfqpoint{2.601949in}{2.534419in}}{\pgfqpoint{2.594049in}{2.537691in}}{\pgfqpoint{2.585813in}{2.537691in}}%
\pgfpathcurveto{\pgfqpoint{2.577577in}{2.537691in}}{\pgfqpoint{2.569676in}{2.534419in}}{\pgfqpoint{2.563853in}{2.528595in}}%
\pgfpathcurveto{\pgfqpoint{2.558029in}{2.522771in}}{\pgfqpoint{2.554756in}{2.514871in}}{\pgfqpoint{2.554756in}{2.506635in}}%
\pgfpathcurveto{\pgfqpoint{2.554756in}{2.498398in}}{\pgfqpoint{2.558029in}{2.490498in}}{\pgfqpoint{2.563853in}{2.484674in}}%
\pgfpathcurveto{\pgfqpoint{2.569676in}{2.478850in}}{\pgfqpoint{2.577577in}{2.475578in}}{\pgfqpoint{2.585813in}{2.475578in}}%
\pgfpathclose%
\pgfusepath{stroke,fill}%
\end{pgfscope}%
\begin{pgfscope}%
\pgfpathrectangle{\pgfqpoint{0.100000in}{0.212622in}}{\pgfqpoint{3.696000in}{3.696000in}}%
\pgfusepath{clip}%
\pgfsetbuttcap%
\pgfsetroundjoin%
\definecolor{currentfill}{rgb}{0.121569,0.466667,0.705882}%
\pgfsetfillcolor{currentfill}%
\pgfsetfillopacity{0.518718}%
\pgfsetlinewidth{1.003750pt}%
\definecolor{currentstroke}{rgb}{0.121569,0.466667,0.705882}%
\pgfsetstrokecolor{currentstroke}%
\pgfsetstrokeopacity{0.518718}%
\pgfsetdash{}{0pt}%
\pgfpathmoveto{\pgfqpoint{2.827502in}{2.645131in}}%
\pgfpathcurveto{\pgfqpoint{2.835738in}{2.645131in}}{\pgfqpoint{2.843638in}{2.648403in}}{\pgfqpoint{2.849462in}{2.654227in}}%
\pgfpathcurveto{\pgfqpoint{2.855286in}{2.660051in}}{\pgfqpoint{2.858559in}{2.667951in}}{\pgfqpoint{2.858559in}{2.676187in}}%
\pgfpathcurveto{\pgfqpoint{2.858559in}{2.684423in}}{\pgfqpoint{2.855286in}{2.692323in}}{\pgfqpoint{2.849462in}{2.698147in}}%
\pgfpathcurveto{\pgfqpoint{2.843638in}{2.703971in}}{\pgfqpoint{2.835738in}{2.707244in}}{\pgfqpoint{2.827502in}{2.707244in}}%
\pgfpathcurveto{\pgfqpoint{2.819266in}{2.707244in}}{\pgfqpoint{2.811366in}{2.703971in}}{\pgfqpoint{2.805542in}{2.698147in}}%
\pgfpathcurveto{\pgfqpoint{2.799718in}{2.692323in}}{\pgfqpoint{2.796446in}{2.684423in}}{\pgfqpoint{2.796446in}{2.676187in}}%
\pgfpathcurveto{\pgfqpoint{2.796446in}{2.667951in}}{\pgfqpoint{2.799718in}{2.660051in}}{\pgfqpoint{2.805542in}{2.654227in}}%
\pgfpathcurveto{\pgfqpoint{2.811366in}{2.648403in}}{\pgfqpoint{2.819266in}{2.645131in}}{\pgfqpoint{2.827502in}{2.645131in}}%
\pgfpathclose%
\pgfusepath{stroke,fill}%
\end{pgfscope}%
\begin{pgfscope}%
\pgfpathrectangle{\pgfqpoint{0.100000in}{0.212622in}}{\pgfqpoint{3.696000in}{3.696000in}}%
\pgfusepath{clip}%
\pgfsetbuttcap%
\pgfsetroundjoin%
\definecolor{currentfill}{rgb}{0.121569,0.466667,0.705882}%
\pgfsetfillcolor{currentfill}%
\pgfsetfillopacity{0.518767}%
\pgfsetlinewidth{1.003750pt}%
\definecolor{currentstroke}{rgb}{0.121569,0.466667,0.705882}%
\pgfsetstrokecolor{currentstroke}%
\pgfsetstrokeopacity{0.518767}%
\pgfsetdash{}{0pt}%
\pgfpathmoveto{\pgfqpoint{2.499388in}{2.403016in}}%
\pgfpathcurveto{\pgfqpoint{2.507625in}{2.403016in}}{\pgfqpoint{2.515525in}{2.406288in}}{\pgfqpoint{2.521349in}{2.412112in}}%
\pgfpathcurveto{\pgfqpoint{2.527173in}{2.417936in}}{\pgfqpoint{2.530445in}{2.425836in}}{\pgfqpoint{2.530445in}{2.434072in}}%
\pgfpathcurveto{\pgfqpoint{2.530445in}{2.442309in}}{\pgfqpoint{2.527173in}{2.450209in}}{\pgfqpoint{2.521349in}{2.456033in}}%
\pgfpathcurveto{\pgfqpoint{2.515525in}{2.461857in}}{\pgfqpoint{2.507625in}{2.465129in}}{\pgfqpoint{2.499388in}{2.465129in}}%
\pgfpathcurveto{\pgfqpoint{2.491152in}{2.465129in}}{\pgfqpoint{2.483252in}{2.461857in}}{\pgfqpoint{2.477428in}{2.456033in}}%
\pgfpathcurveto{\pgfqpoint{2.471604in}{2.450209in}}{\pgfqpoint{2.468332in}{2.442309in}}{\pgfqpoint{2.468332in}{2.434072in}}%
\pgfpathcurveto{\pgfqpoint{2.468332in}{2.425836in}}{\pgfqpoint{2.471604in}{2.417936in}}{\pgfqpoint{2.477428in}{2.412112in}}%
\pgfpathcurveto{\pgfqpoint{2.483252in}{2.406288in}}{\pgfqpoint{2.491152in}{2.403016in}}{\pgfqpoint{2.499388in}{2.403016in}}%
\pgfpathclose%
\pgfusepath{stroke,fill}%
\end{pgfscope}%
\begin{pgfscope}%
\pgfpathrectangle{\pgfqpoint{0.100000in}{0.212622in}}{\pgfqpoint{3.696000in}{3.696000in}}%
\pgfusepath{clip}%
\pgfsetbuttcap%
\pgfsetroundjoin%
\definecolor{currentfill}{rgb}{0.121569,0.466667,0.705882}%
\pgfsetfillcolor{currentfill}%
\pgfsetfillopacity{0.519186}%
\pgfsetlinewidth{1.003750pt}%
\definecolor{currentstroke}{rgb}{0.121569,0.466667,0.705882}%
\pgfsetstrokecolor{currentstroke}%
\pgfsetstrokeopacity{0.519186}%
\pgfsetdash{}{0pt}%
\pgfpathmoveto{\pgfqpoint{2.571613in}{2.463601in}}%
\pgfpathcurveto{\pgfqpoint{2.579849in}{2.463601in}}{\pgfqpoint{2.587749in}{2.466873in}}{\pgfqpoint{2.593573in}{2.472697in}}%
\pgfpathcurveto{\pgfqpoint{2.599397in}{2.478521in}}{\pgfqpoint{2.602669in}{2.486421in}}{\pgfqpoint{2.602669in}{2.494657in}}%
\pgfpathcurveto{\pgfqpoint{2.602669in}{2.502893in}}{\pgfqpoint{2.599397in}{2.510793in}}{\pgfqpoint{2.593573in}{2.516617in}}%
\pgfpathcurveto{\pgfqpoint{2.587749in}{2.522441in}}{\pgfqpoint{2.579849in}{2.525714in}}{\pgfqpoint{2.571613in}{2.525714in}}%
\pgfpathcurveto{\pgfqpoint{2.563376in}{2.525714in}}{\pgfqpoint{2.555476in}{2.522441in}}{\pgfqpoint{2.549652in}{2.516617in}}%
\pgfpathcurveto{\pgfqpoint{2.543829in}{2.510793in}}{\pgfqpoint{2.540556in}{2.502893in}}{\pgfqpoint{2.540556in}{2.494657in}}%
\pgfpathcurveto{\pgfqpoint{2.540556in}{2.486421in}}{\pgfqpoint{2.543829in}{2.478521in}}{\pgfqpoint{2.549652in}{2.472697in}}%
\pgfpathcurveto{\pgfqpoint{2.555476in}{2.466873in}}{\pgfqpoint{2.563376in}{2.463601in}}{\pgfqpoint{2.571613in}{2.463601in}}%
\pgfpathclose%
\pgfusepath{stroke,fill}%
\end{pgfscope}%
\begin{pgfscope}%
\pgfpathrectangle{\pgfqpoint{0.100000in}{0.212622in}}{\pgfqpoint{3.696000in}{3.696000in}}%
\pgfusepath{clip}%
\pgfsetbuttcap%
\pgfsetroundjoin%
\definecolor{currentfill}{rgb}{0.121569,0.466667,0.705882}%
\pgfsetfillcolor{currentfill}%
\pgfsetfillopacity{0.519713}%
\pgfsetlinewidth{1.003750pt}%
\definecolor{currentstroke}{rgb}{0.121569,0.466667,0.705882}%
\pgfsetstrokecolor{currentstroke}%
\pgfsetstrokeopacity{0.519713}%
\pgfsetdash{}{0pt}%
\pgfpathmoveto{\pgfqpoint{2.629786in}{2.511236in}}%
\pgfpathcurveto{\pgfqpoint{2.638022in}{2.511236in}}{\pgfqpoint{2.645922in}{2.514508in}}{\pgfqpoint{2.651746in}{2.520332in}}%
\pgfpathcurveto{\pgfqpoint{2.657570in}{2.526156in}}{\pgfqpoint{2.660842in}{2.534056in}}{\pgfqpoint{2.660842in}{2.542292in}}%
\pgfpathcurveto{\pgfqpoint{2.660842in}{2.550529in}}{\pgfqpoint{2.657570in}{2.558429in}}{\pgfqpoint{2.651746in}{2.564253in}}%
\pgfpathcurveto{\pgfqpoint{2.645922in}{2.570076in}}{\pgfqpoint{2.638022in}{2.573349in}}{\pgfqpoint{2.629786in}{2.573349in}}%
\pgfpathcurveto{\pgfqpoint{2.621549in}{2.573349in}}{\pgfqpoint{2.613649in}{2.570076in}}{\pgfqpoint{2.607825in}{2.564253in}}%
\pgfpathcurveto{\pgfqpoint{2.602002in}{2.558429in}}{\pgfqpoint{2.598729in}{2.550529in}}{\pgfqpoint{2.598729in}{2.542292in}}%
\pgfpathcurveto{\pgfqpoint{2.598729in}{2.534056in}}{\pgfqpoint{2.602002in}{2.526156in}}{\pgfqpoint{2.607825in}{2.520332in}}%
\pgfpathcurveto{\pgfqpoint{2.613649in}{2.514508in}}{\pgfqpoint{2.621549in}{2.511236in}}{\pgfqpoint{2.629786in}{2.511236in}}%
\pgfpathclose%
\pgfusepath{stroke,fill}%
\end{pgfscope}%
\begin{pgfscope}%
\pgfpathrectangle{\pgfqpoint{0.100000in}{0.212622in}}{\pgfqpoint{3.696000in}{3.696000in}}%
\pgfusepath{clip}%
\pgfsetbuttcap%
\pgfsetroundjoin%
\definecolor{currentfill}{rgb}{0.121569,0.466667,0.705882}%
\pgfsetfillcolor{currentfill}%
\pgfsetfillopacity{0.519898}%
\pgfsetlinewidth{1.003750pt}%
\definecolor{currentstroke}{rgb}{0.121569,0.466667,0.705882}%
\pgfsetstrokecolor{currentstroke}%
\pgfsetstrokeopacity{0.519898}%
\pgfsetdash{}{0pt}%
\pgfpathmoveto{\pgfqpoint{2.472892in}{2.378443in}}%
\pgfpathcurveto{\pgfqpoint{2.481128in}{2.378443in}}{\pgfqpoint{2.489028in}{2.381715in}}{\pgfqpoint{2.494852in}{2.387539in}}%
\pgfpathcurveto{\pgfqpoint{2.500676in}{2.393363in}}{\pgfqpoint{2.503948in}{2.401263in}}{\pgfqpoint{2.503948in}{2.409499in}}%
\pgfpathcurveto{\pgfqpoint{2.503948in}{2.417735in}}{\pgfqpoint{2.500676in}{2.425636in}}{\pgfqpoint{2.494852in}{2.431459in}}%
\pgfpathcurveto{\pgfqpoint{2.489028in}{2.437283in}}{\pgfqpoint{2.481128in}{2.440556in}}{\pgfqpoint{2.472892in}{2.440556in}}%
\pgfpathcurveto{\pgfqpoint{2.464656in}{2.440556in}}{\pgfqpoint{2.456756in}{2.437283in}}{\pgfqpoint{2.450932in}{2.431459in}}%
\pgfpathcurveto{\pgfqpoint{2.445108in}{2.425636in}}{\pgfqpoint{2.441835in}{2.417735in}}{\pgfqpoint{2.441835in}{2.409499in}}%
\pgfpathcurveto{\pgfqpoint{2.441835in}{2.401263in}}{\pgfqpoint{2.445108in}{2.393363in}}{\pgfqpoint{2.450932in}{2.387539in}}%
\pgfpathcurveto{\pgfqpoint{2.456756in}{2.381715in}}{\pgfqpoint{2.464656in}{2.378443in}}{\pgfqpoint{2.472892in}{2.378443in}}%
\pgfpathclose%
\pgfusepath{stroke,fill}%
\end{pgfscope}%
\begin{pgfscope}%
\pgfpathrectangle{\pgfqpoint{0.100000in}{0.212622in}}{\pgfqpoint{3.696000in}{3.696000in}}%
\pgfusepath{clip}%
\pgfsetbuttcap%
\pgfsetroundjoin%
\definecolor{currentfill}{rgb}{0.121569,0.466667,0.705882}%
\pgfsetfillcolor{currentfill}%
\pgfsetfillopacity{0.520776}%
\pgfsetlinewidth{1.003750pt}%
\definecolor{currentstroke}{rgb}{0.121569,0.466667,0.705882}%
\pgfsetstrokecolor{currentstroke}%
\pgfsetstrokeopacity{0.520776}%
\pgfsetdash{}{0pt}%
\pgfpathmoveto{\pgfqpoint{1.802726in}{1.998306in}}%
\pgfpathcurveto{\pgfqpoint{1.810963in}{1.998306in}}{\pgfqpoint{1.818863in}{2.001579in}}{\pgfqpoint{1.824687in}{2.007402in}}%
\pgfpathcurveto{\pgfqpoint{1.830511in}{2.013226in}}{\pgfqpoint{1.833783in}{2.021126in}}{\pgfqpoint{1.833783in}{2.029363in}}%
\pgfpathcurveto{\pgfqpoint{1.833783in}{2.037599in}}{\pgfqpoint{1.830511in}{2.045499in}}{\pgfqpoint{1.824687in}{2.051323in}}%
\pgfpathcurveto{\pgfqpoint{1.818863in}{2.057147in}}{\pgfqpoint{1.810963in}{2.060419in}}{\pgfqpoint{1.802726in}{2.060419in}}%
\pgfpathcurveto{\pgfqpoint{1.794490in}{2.060419in}}{\pgfqpoint{1.786590in}{2.057147in}}{\pgfqpoint{1.780766in}{2.051323in}}%
\pgfpathcurveto{\pgfqpoint{1.774942in}{2.045499in}}{\pgfqpoint{1.771670in}{2.037599in}}{\pgfqpoint{1.771670in}{2.029363in}}%
\pgfpathcurveto{\pgfqpoint{1.771670in}{2.021126in}}{\pgfqpoint{1.774942in}{2.013226in}}{\pgfqpoint{1.780766in}{2.007402in}}%
\pgfpathcurveto{\pgfqpoint{1.786590in}{2.001579in}}{\pgfqpoint{1.794490in}{1.998306in}}{\pgfqpoint{1.802726in}{1.998306in}}%
\pgfpathclose%
\pgfusepath{stroke,fill}%
\end{pgfscope}%
\begin{pgfscope}%
\pgfpathrectangle{\pgfqpoint{0.100000in}{0.212622in}}{\pgfqpoint{3.696000in}{3.696000in}}%
\pgfusepath{clip}%
\pgfsetbuttcap%
\pgfsetroundjoin%
\definecolor{currentfill}{rgb}{0.121569,0.466667,0.705882}%
\pgfsetfillcolor{currentfill}%
\pgfsetfillopacity{0.521006}%
\pgfsetlinewidth{1.003750pt}%
\definecolor{currentstroke}{rgb}{0.121569,0.466667,0.705882}%
\pgfsetstrokecolor{currentstroke}%
\pgfsetstrokeopacity{0.521006}%
\pgfsetdash{}{0pt}%
\pgfpathmoveto{\pgfqpoint{2.488545in}{2.394325in}}%
\pgfpathcurveto{\pgfqpoint{2.496781in}{2.394325in}}{\pgfqpoint{2.504681in}{2.397597in}}{\pgfqpoint{2.510505in}{2.403421in}}%
\pgfpathcurveto{\pgfqpoint{2.516329in}{2.409245in}}{\pgfqpoint{2.519602in}{2.417145in}}{\pgfqpoint{2.519602in}{2.425382in}}%
\pgfpathcurveto{\pgfqpoint{2.519602in}{2.433618in}}{\pgfqpoint{2.516329in}{2.441518in}}{\pgfqpoint{2.510505in}{2.447342in}}%
\pgfpathcurveto{\pgfqpoint{2.504681in}{2.453166in}}{\pgfqpoint{2.496781in}{2.456438in}}{\pgfqpoint{2.488545in}{2.456438in}}%
\pgfpathcurveto{\pgfqpoint{2.480309in}{2.456438in}}{\pgfqpoint{2.472409in}{2.453166in}}{\pgfqpoint{2.466585in}{2.447342in}}%
\pgfpathcurveto{\pgfqpoint{2.460761in}{2.441518in}}{\pgfqpoint{2.457489in}{2.433618in}}{\pgfqpoint{2.457489in}{2.425382in}}%
\pgfpathcurveto{\pgfqpoint{2.457489in}{2.417145in}}{\pgfqpoint{2.460761in}{2.409245in}}{\pgfqpoint{2.466585in}{2.403421in}}%
\pgfpathcurveto{\pgfqpoint{2.472409in}{2.397597in}}{\pgfqpoint{2.480309in}{2.394325in}}{\pgfqpoint{2.488545in}{2.394325in}}%
\pgfpathclose%
\pgfusepath{stroke,fill}%
\end{pgfscope}%
\begin{pgfscope}%
\pgfpathrectangle{\pgfqpoint{0.100000in}{0.212622in}}{\pgfqpoint{3.696000in}{3.696000in}}%
\pgfusepath{clip}%
\pgfsetbuttcap%
\pgfsetroundjoin%
\definecolor{currentfill}{rgb}{0.121569,0.466667,0.705882}%
\pgfsetfillcolor{currentfill}%
\pgfsetfillopacity{0.522258}%
\pgfsetlinewidth{1.003750pt}%
\definecolor{currentstroke}{rgb}{0.121569,0.466667,0.705882}%
\pgfsetstrokecolor{currentstroke}%
\pgfsetstrokeopacity{0.522258}%
\pgfsetdash{}{0pt}%
\pgfpathmoveto{\pgfqpoint{1.787623in}{1.985258in}}%
\pgfpathcurveto{\pgfqpoint{1.795859in}{1.985258in}}{\pgfqpoint{1.803759in}{1.988530in}}{\pgfqpoint{1.809583in}{1.994354in}}%
\pgfpathcurveto{\pgfqpoint{1.815407in}{2.000178in}}{\pgfqpoint{1.818679in}{2.008078in}}{\pgfqpoint{1.818679in}{2.016314in}}%
\pgfpathcurveto{\pgfqpoint{1.818679in}{2.024551in}}{\pgfqpoint{1.815407in}{2.032451in}}{\pgfqpoint{1.809583in}{2.038275in}}%
\pgfpathcurveto{\pgfqpoint{1.803759in}{2.044099in}}{\pgfqpoint{1.795859in}{2.047371in}}{\pgfqpoint{1.787623in}{2.047371in}}%
\pgfpathcurveto{\pgfqpoint{1.779386in}{2.047371in}}{\pgfqpoint{1.771486in}{2.044099in}}{\pgfqpoint{1.765662in}{2.038275in}}%
\pgfpathcurveto{\pgfqpoint{1.759838in}{2.032451in}}{\pgfqpoint{1.756566in}{2.024551in}}{\pgfqpoint{1.756566in}{2.016314in}}%
\pgfpathcurveto{\pgfqpoint{1.756566in}{2.008078in}}{\pgfqpoint{1.759838in}{2.000178in}}{\pgfqpoint{1.765662in}{1.994354in}}%
\pgfpathcurveto{\pgfqpoint{1.771486in}{1.988530in}}{\pgfqpoint{1.779386in}{1.985258in}}{\pgfqpoint{1.787623in}{1.985258in}}%
\pgfpathclose%
\pgfusepath{stroke,fill}%
\end{pgfscope}%
\begin{pgfscope}%
\pgfpathrectangle{\pgfqpoint{0.100000in}{0.212622in}}{\pgfqpoint{3.696000in}{3.696000in}}%
\pgfusepath{clip}%
\pgfsetbuttcap%
\pgfsetroundjoin%
\definecolor{currentfill}{rgb}{0.121569,0.466667,0.705882}%
\pgfsetfillcolor{currentfill}%
\pgfsetfillopacity{0.522400}%
\pgfsetlinewidth{1.003750pt}%
\definecolor{currentstroke}{rgb}{0.121569,0.466667,0.705882}%
\pgfsetstrokecolor{currentstroke}%
\pgfsetstrokeopacity{0.522400}%
\pgfsetdash{}{0pt}%
\pgfpathmoveto{\pgfqpoint{2.552171in}{2.449255in}}%
\pgfpathcurveto{\pgfqpoint{2.560407in}{2.449255in}}{\pgfqpoint{2.568307in}{2.452527in}}{\pgfqpoint{2.574131in}{2.458351in}}%
\pgfpathcurveto{\pgfqpoint{2.579955in}{2.464175in}}{\pgfqpoint{2.583228in}{2.472075in}}{\pgfqpoint{2.583228in}{2.480311in}}%
\pgfpathcurveto{\pgfqpoint{2.583228in}{2.488547in}}{\pgfqpoint{2.579955in}{2.496447in}}{\pgfqpoint{2.574131in}{2.502271in}}%
\pgfpathcurveto{\pgfqpoint{2.568307in}{2.508095in}}{\pgfqpoint{2.560407in}{2.511368in}}{\pgfqpoint{2.552171in}{2.511368in}}%
\pgfpathcurveto{\pgfqpoint{2.543935in}{2.511368in}}{\pgfqpoint{2.536035in}{2.508095in}}{\pgfqpoint{2.530211in}{2.502271in}}%
\pgfpathcurveto{\pgfqpoint{2.524387in}{2.496447in}}{\pgfqpoint{2.521115in}{2.488547in}}{\pgfqpoint{2.521115in}{2.480311in}}%
\pgfpathcurveto{\pgfqpoint{2.521115in}{2.472075in}}{\pgfqpoint{2.524387in}{2.464175in}}{\pgfqpoint{2.530211in}{2.458351in}}%
\pgfpathcurveto{\pgfqpoint{2.536035in}{2.452527in}}{\pgfqpoint{2.543935in}{2.449255in}}{\pgfqpoint{2.552171in}{2.449255in}}%
\pgfpathclose%
\pgfusepath{stroke,fill}%
\end{pgfscope}%
\begin{pgfscope}%
\pgfpathrectangle{\pgfqpoint{0.100000in}{0.212622in}}{\pgfqpoint{3.696000in}{3.696000in}}%
\pgfusepath{clip}%
\pgfsetbuttcap%
\pgfsetroundjoin%
\definecolor{currentfill}{rgb}{0.121569,0.466667,0.705882}%
\pgfsetfillcolor{currentfill}%
\pgfsetfillopacity{0.523253}%
\pgfsetlinewidth{1.003750pt}%
\definecolor{currentstroke}{rgb}{0.121569,0.466667,0.705882}%
\pgfsetstrokecolor{currentstroke}%
\pgfsetstrokeopacity{0.523253}%
\pgfsetdash{}{0pt}%
\pgfpathmoveto{\pgfqpoint{2.786630in}{2.606414in}}%
\pgfpathcurveto{\pgfqpoint{2.794866in}{2.606414in}}{\pgfqpoint{2.802766in}{2.609686in}}{\pgfqpoint{2.808590in}{2.615510in}}%
\pgfpathcurveto{\pgfqpoint{2.814414in}{2.621334in}}{\pgfqpoint{2.817687in}{2.629234in}}{\pgfqpoint{2.817687in}{2.637470in}}%
\pgfpathcurveto{\pgfqpoint{2.817687in}{2.645707in}}{\pgfqpoint{2.814414in}{2.653607in}}{\pgfqpoint{2.808590in}{2.659431in}}%
\pgfpathcurveto{\pgfqpoint{2.802766in}{2.665255in}}{\pgfqpoint{2.794866in}{2.668527in}}{\pgfqpoint{2.786630in}{2.668527in}}%
\pgfpathcurveto{\pgfqpoint{2.778394in}{2.668527in}}{\pgfqpoint{2.770494in}{2.665255in}}{\pgfqpoint{2.764670in}{2.659431in}}%
\pgfpathcurveto{\pgfqpoint{2.758846in}{2.653607in}}{\pgfqpoint{2.755574in}{2.645707in}}{\pgfqpoint{2.755574in}{2.637470in}}%
\pgfpathcurveto{\pgfqpoint{2.755574in}{2.629234in}}{\pgfqpoint{2.758846in}{2.621334in}}{\pgfqpoint{2.764670in}{2.615510in}}%
\pgfpathcurveto{\pgfqpoint{2.770494in}{2.609686in}}{\pgfqpoint{2.778394in}{2.606414in}}{\pgfqpoint{2.786630in}{2.606414in}}%
\pgfpathclose%
\pgfusepath{stroke,fill}%
\end{pgfscope}%
\begin{pgfscope}%
\pgfpathrectangle{\pgfqpoint{0.100000in}{0.212622in}}{\pgfqpoint{3.696000in}{3.696000in}}%
\pgfusepath{clip}%
\pgfsetbuttcap%
\pgfsetroundjoin%
\definecolor{currentfill}{rgb}{0.121569,0.466667,0.705882}%
\pgfsetfillcolor{currentfill}%
\pgfsetfillopacity{0.523519}%
\pgfsetlinewidth{1.003750pt}%
\definecolor{currentstroke}{rgb}{0.121569,0.466667,0.705882}%
\pgfsetstrokecolor{currentstroke}%
\pgfsetstrokeopacity{0.523519}%
\pgfsetdash{}{0pt}%
\pgfpathmoveto{\pgfqpoint{1.617209in}{1.869220in}}%
\pgfpathcurveto{\pgfqpoint{1.625445in}{1.869220in}}{\pgfqpoint{1.633345in}{1.872493in}}{\pgfqpoint{1.639169in}{1.878316in}}%
\pgfpathcurveto{\pgfqpoint{1.644993in}{1.884140in}}{\pgfqpoint{1.648266in}{1.892040in}}{\pgfqpoint{1.648266in}{1.900277in}}%
\pgfpathcurveto{\pgfqpoint{1.648266in}{1.908513in}}{\pgfqpoint{1.644993in}{1.916413in}}{\pgfqpoint{1.639169in}{1.922237in}}%
\pgfpathcurveto{\pgfqpoint{1.633345in}{1.928061in}}{\pgfqpoint{1.625445in}{1.931333in}}{\pgfqpoint{1.617209in}{1.931333in}}%
\pgfpathcurveto{\pgfqpoint{1.608973in}{1.931333in}}{\pgfqpoint{1.601073in}{1.928061in}}{\pgfqpoint{1.595249in}{1.922237in}}%
\pgfpathcurveto{\pgfqpoint{1.589425in}{1.916413in}}{\pgfqpoint{1.586153in}{1.908513in}}{\pgfqpoint{1.586153in}{1.900277in}}%
\pgfpathcurveto{\pgfqpoint{1.586153in}{1.892040in}}{\pgfqpoint{1.589425in}{1.884140in}}{\pgfqpoint{1.595249in}{1.878316in}}%
\pgfpathcurveto{\pgfqpoint{1.601073in}{1.872493in}}{\pgfqpoint{1.608973in}{1.869220in}}{\pgfqpoint{1.617209in}{1.869220in}}%
\pgfpathclose%
\pgfusepath{stroke,fill}%
\end{pgfscope}%
\begin{pgfscope}%
\pgfpathrectangle{\pgfqpoint{0.100000in}{0.212622in}}{\pgfqpoint{3.696000in}{3.696000in}}%
\pgfusepath{clip}%
\pgfsetbuttcap%
\pgfsetroundjoin%
\definecolor{currentfill}{rgb}{0.121569,0.466667,0.705882}%
\pgfsetfillcolor{currentfill}%
\pgfsetfillopacity{0.523677}%
\pgfsetlinewidth{1.003750pt}%
\definecolor{currentstroke}{rgb}{0.121569,0.466667,0.705882}%
\pgfsetstrokecolor{currentstroke}%
\pgfsetstrokeopacity{0.523677}%
\pgfsetdash{}{0pt}%
\pgfpathmoveto{\pgfqpoint{2.470626in}{2.374885in}}%
\pgfpathcurveto{\pgfqpoint{2.478863in}{2.374885in}}{\pgfqpoint{2.486763in}{2.378158in}}{\pgfqpoint{2.492587in}{2.383981in}}%
\pgfpathcurveto{\pgfqpoint{2.498411in}{2.389805in}}{\pgfqpoint{2.501683in}{2.397705in}}{\pgfqpoint{2.501683in}{2.405942in}}%
\pgfpathcurveto{\pgfqpoint{2.501683in}{2.414178in}}{\pgfqpoint{2.498411in}{2.422078in}}{\pgfqpoint{2.492587in}{2.427902in}}%
\pgfpathcurveto{\pgfqpoint{2.486763in}{2.433726in}}{\pgfqpoint{2.478863in}{2.436998in}}{\pgfqpoint{2.470626in}{2.436998in}}%
\pgfpathcurveto{\pgfqpoint{2.462390in}{2.436998in}}{\pgfqpoint{2.454490in}{2.433726in}}{\pgfqpoint{2.448666in}{2.427902in}}%
\pgfpathcurveto{\pgfqpoint{2.442842in}{2.422078in}}{\pgfqpoint{2.439570in}{2.414178in}}{\pgfqpoint{2.439570in}{2.405942in}}%
\pgfpathcurveto{\pgfqpoint{2.439570in}{2.397705in}}{\pgfqpoint{2.442842in}{2.389805in}}{\pgfqpoint{2.448666in}{2.383981in}}%
\pgfpathcurveto{\pgfqpoint{2.454490in}{2.378158in}}{\pgfqpoint{2.462390in}{2.374885in}}{\pgfqpoint{2.470626in}{2.374885in}}%
\pgfpathclose%
\pgfusepath{stroke,fill}%
\end{pgfscope}%
\begin{pgfscope}%
\pgfpathrectangle{\pgfqpoint{0.100000in}{0.212622in}}{\pgfqpoint{3.696000in}{3.696000in}}%
\pgfusepath{clip}%
\pgfsetbuttcap%
\pgfsetroundjoin%
\definecolor{currentfill}{rgb}{0.121569,0.466667,0.705882}%
\pgfsetfillcolor{currentfill}%
\pgfsetfillopacity{0.523765}%
\pgfsetlinewidth{1.003750pt}%
\definecolor{currentstroke}{rgb}{0.121569,0.466667,0.705882}%
\pgfsetstrokecolor{currentstroke}%
\pgfsetstrokeopacity{0.523765}%
\pgfsetdash{}{0pt}%
\pgfpathmoveto{\pgfqpoint{1.594413in}{1.846744in}}%
\pgfpathcurveto{\pgfqpoint{1.602650in}{1.846744in}}{\pgfqpoint{1.610550in}{1.850016in}}{\pgfqpoint{1.616374in}{1.855840in}}%
\pgfpathcurveto{\pgfqpoint{1.622198in}{1.861664in}}{\pgfqpoint{1.625470in}{1.869564in}}{\pgfqpoint{1.625470in}{1.877800in}}%
\pgfpathcurveto{\pgfqpoint{1.625470in}{1.886036in}}{\pgfqpoint{1.622198in}{1.893936in}}{\pgfqpoint{1.616374in}{1.899760in}}%
\pgfpathcurveto{\pgfqpoint{1.610550in}{1.905584in}}{\pgfqpoint{1.602650in}{1.908857in}}{\pgfqpoint{1.594413in}{1.908857in}}%
\pgfpathcurveto{\pgfqpoint{1.586177in}{1.908857in}}{\pgfqpoint{1.578277in}{1.905584in}}{\pgfqpoint{1.572453in}{1.899760in}}%
\pgfpathcurveto{\pgfqpoint{1.566629in}{1.893936in}}{\pgfqpoint{1.563357in}{1.886036in}}{\pgfqpoint{1.563357in}{1.877800in}}%
\pgfpathcurveto{\pgfqpoint{1.563357in}{1.869564in}}{\pgfqpoint{1.566629in}{1.861664in}}{\pgfqpoint{1.572453in}{1.855840in}}%
\pgfpathcurveto{\pgfqpoint{1.578277in}{1.850016in}}{\pgfqpoint{1.586177in}{1.846744in}}{\pgfqpoint{1.594413in}{1.846744in}}%
\pgfpathclose%
\pgfusepath{stroke,fill}%
\end{pgfscope}%
\begin{pgfscope}%
\pgfpathrectangle{\pgfqpoint{0.100000in}{0.212622in}}{\pgfqpoint{3.696000in}{3.696000in}}%
\pgfusepath{clip}%
\pgfsetbuttcap%
\pgfsetroundjoin%
\definecolor{currentfill}{rgb}{0.121569,0.466667,0.705882}%
\pgfsetfillcolor{currentfill}%
\pgfsetfillopacity{0.523831}%
\pgfsetlinewidth{1.003750pt}%
\definecolor{currentstroke}{rgb}{0.121569,0.466667,0.705882}%
\pgfsetstrokecolor{currentstroke}%
\pgfsetstrokeopacity{0.523831}%
\pgfsetdash{}{0pt}%
\pgfpathmoveto{\pgfqpoint{1.610913in}{1.862738in}}%
\pgfpathcurveto{\pgfqpoint{1.619150in}{1.862738in}}{\pgfqpoint{1.627050in}{1.866011in}}{\pgfqpoint{1.632874in}{1.871834in}}%
\pgfpathcurveto{\pgfqpoint{1.638697in}{1.877658in}}{\pgfqpoint{1.641970in}{1.885558in}}{\pgfqpoint{1.641970in}{1.893795in}}%
\pgfpathcurveto{\pgfqpoint{1.641970in}{1.902031in}}{\pgfqpoint{1.638697in}{1.909931in}}{\pgfqpoint{1.632874in}{1.915755in}}%
\pgfpathcurveto{\pgfqpoint{1.627050in}{1.921579in}}{\pgfqpoint{1.619150in}{1.924851in}}{\pgfqpoint{1.610913in}{1.924851in}}%
\pgfpathcurveto{\pgfqpoint{1.602677in}{1.924851in}}{\pgfqpoint{1.594777in}{1.921579in}}{\pgfqpoint{1.588953in}{1.915755in}}%
\pgfpathcurveto{\pgfqpoint{1.583129in}{1.909931in}}{\pgfqpoint{1.579857in}{1.902031in}}{\pgfqpoint{1.579857in}{1.893795in}}%
\pgfpathcurveto{\pgfqpoint{1.579857in}{1.885558in}}{\pgfqpoint{1.583129in}{1.877658in}}{\pgfqpoint{1.588953in}{1.871834in}}%
\pgfpathcurveto{\pgfqpoint{1.594777in}{1.866011in}}{\pgfqpoint{1.602677in}{1.862738in}}{\pgfqpoint{1.610913in}{1.862738in}}%
\pgfpathclose%
\pgfusepath{stroke,fill}%
\end{pgfscope}%
\begin{pgfscope}%
\pgfpathrectangle{\pgfqpoint{0.100000in}{0.212622in}}{\pgfqpoint{3.696000in}{3.696000in}}%
\pgfusepath{clip}%
\pgfsetbuttcap%
\pgfsetroundjoin%
\definecolor{currentfill}{rgb}{0.121569,0.466667,0.705882}%
\pgfsetfillcolor{currentfill}%
\pgfsetfillopacity{0.524747}%
\pgfsetlinewidth{1.003750pt}%
\definecolor{currentstroke}{rgb}{0.121569,0.466667,0.705882}%
\pgfsetstrokecolor{currentstroke}%
\pgfsetstrokeopacity{0.524747}%
\pgfsetdash{}{0pt}%
\pgfpathmoveto{\pgfqpoint{2.652162in}{2.522540in}}%
\pgfpathcurveto{\pgfqpoint{2.660399in}{2.522540in}}{\pgfqpoint{2.668299in}{2.525812in}}{\pgfqpoint{2.674123in}{2.531636in}}%
\pgfpathcurveto{\pgfqpoint{2.679947in}{2.537460in}}{\pgfqpoint{2.683219in}{2.545360in}}{\pgfqpoint{2.683219in}{2.553597in}}%
\pgfpathcurveto{\pgfqpoint{2.683219in}{2.561833in}}{\pgfqpoint{2.679947in}{2.569733in}}{\pgfqpoint{2.674123in}{2.575557in}}%
\pgfpathcurveto{\pgfqpoint{2.668299in}{2.581381in}}{\pgfqpoint{2.660399in}{2.584653in}}{\pgfqpoint{2.652162in}{2.584653in}}%
\pgfpathcurveto{\pgfqpoint{2.643926in}{2.584653in}}{\pgfqpoint{2.636026in}{2.581381in}}{\pgfqpoint{2.630202in}{2.575557in}}%
\pgfpathcurveto{\pgfqpoint{2.624378in}{2.569733in}}{\pgfqpoint{2.621106in}{2.561833in}}{\pgfqpoint{2.621106in}{2.553597in}}%
\pgfpathcurveto{\pgfqpoint{2.621106in}{2.545360in}}{\pgfqpoint{2.624378in}{2.537460in}}{\pgfqpoint{2.630202in}{2.531636in}}%
\pgfpathcurveto{\pgfqpoint{2.636026in}{2.525812in}}{\pgfqpoint{2.643926in}{2.522540in}}{\pgfqpoint{2.652162in}{2.522540in}}%
\pgfpathclose%
\pgfusepath{stroke,fill}%
\end{pgfscope}%
\begin{pgfscope}%
\pgfpathrectangle{\pgfqpoint{0.100000in}{0.212622in}}{\pgfqpoint{3.696000in}{3.696000in}}%
\pgfusepath{clip}%
\pgfsetbuttcap%
\pgfsetroundjoin%
\definecolor{currentfill}{rgb}{0.121569,0.466667,0.705882}%
\pgfsetfillcolor{currentfill}%
\pgfsetfillopacity{0.525086}%
\pgfsetlinewidth{1.003750pt}%
\definecolor{currentstroke}{rgb}{0.121569,0.466667,0.705882}%
\pgfsetstrokecolor{currentstroke}%
\pgfsetstrokeopacity{0.525086}%
\pgfsetdash{}{0pt}%
\pgfpathmoveto{\pgfqpoint{1.596142in}{1.848908in}}%
\pgfpathcurveto{\pgfqpoint{1.604379in}{1.848908in}}{\pgfqpoint{1.612279in}{1.852180in}}{\pgfqpoint{1.618103in}{1.858004in}}%
\pgfpathcurveto{\pgfqpoint{1.623927in}{1.863828in}}{\pgfqpoint{1.627199in}{1.871728in}}{\pgfqpoint{1.627199in}{1.879964in}}%
\pgfpathcurveto{\pgfqpoint{1.627199in}{1.888200in}}{\pgfqpoint{1.623927in}{1.896100in}}{\pgfqpoint{1.618103in}{1.901924in}}%
\pgfpathcurveto{\pgfqpoint{1.612279in}{1.907748in}}{\pgfqpoint{1.604379in}{1.911021in}}{\pgfqpoint{1.596142in}{1.911021in}}%
\pgfpathcurveto{\pgfqpoint{1.587906in}{1.911021in}}{\pgfqpoint{1.580006in}{1.907748in}}{\pgfqpoint{1.574182in}{1.901924in}}%
\pgfpathcurveto{\pgfqpoint{1.568358in}{1.896100in}}{\pgfqpoint{1.565086in}{1.888200in}}{\pgfqpoint{1.565086in}{1.879964in}}%
\pgfpathcurveto{\pgfqpoint{1.565086in}{1.871728in}}{\pgfqpoint{1.568358in}{1.863828in}}{\pgfqpoint{1.574182in}{1.858004in}}%
\pgfpathcurveto{\pgfqpoint{1.580006in}{1.852180in}}{\pgfqpoint{1.587906in}{1.848908in}}{\pgfqpoint{1.596142in}{1.848908in}}%
\pgfpathclose%
\pgfusepath{stroke,fill}%
\end{pgfscope}%
\begin{pgfscope}%
\pgfpathrectangle{\pgfqpoint{0.100000in}{0.212622in}}{\pgfqpoint{3.696000in}{3.696000in}}%
\pgfusepath{clip}%
\pgfsetbuttcap%
\pgfsetroundjoin%
\definecolor{currentfill}{rgb}{0.121569,0.466667,0.705882}%
\pgfsetfillcolor{currentfill}%
\pgfsetfillopacity{0.525316}%
\pgfsetlinewidth{1.003750pt}%
\definecolor{currentstroke}{rgb}{0.121569,0.466667,0.705882}%
\pgfsetstrokecolor{currentstroke}%
\pgfsetstrokeopacity{0.525316}%
\pgfsetdash{}{0pt}%
\pgfpathmoveto{\pgfqpoint{1.601960in}{1.854306in}}%
\pgfpathcurveto{\pgfqpoint{1.610196in}{1.854306in}}{\pgfqpoint{1.618096in}{1.857579in}}{\pgfqpoint{1.623920in}{1.863402in}}%
\pgfpathcurveto{\pgfqpoint{1.629744in}{1.869226in}}{\pgfqpoint{1.633016in}{1.877126in}}{\pgfqpoint{1.633016in}{1.885363in}}%
\pgfpathcurveto{\pgfqpoint{1.633016in}{1.893599in}}{\pgfqpoint{1.629744in}{1.901499in}}{\pgfqpoint{1.623920in}{1.907323in}}%
\pgfpathcurveto{\pgfqpoint{1.618096in}{1.913147in}}{\pgfqpoint{1.610196in}{1.916419in}}{\pgfqpoint{1.601960in}{1.916419in}}%
\pgfpathcurveto{\pgfqpoint{1.593724in}{1.916419in}}{\pgfqpoint{1.585823in}{1.913147in}}{\pgfqpoint{1.580000in}{1.907323in}}%
\pgfpathcurveto{\pgfqpoint{1.574176in}{1.901499in}}{\pgfqpoint{1.570903in}{1.893599in}}{\pgfqpoint{1.570903in}{1.885363in}}%
\pgfpathcurveto{\pgfqpoint{1.570903in}{1.877126in}}{\pgfqpoint{1.574176in}{1.869226in}}{\pgfqpoint{1.580000in}{1.863402in}}%
\pgfpathcurveto{\pgfqpoint{1.585823in}{1.857579in}}{\pgfqpoint{1.593724in}{1.854306in}}{\pgfqpoint{1.601960in}{1.854306in}}%
\pgfpathclose%
\pgfusepath{stroke,fill}%
\end{pgfscope}%
\begin{pgfscope}%
\pgfpathrectangle{\pgfqpoint{0.100000in}{0.212622in}}{\pgfqpoint{3.696000in}{3.696000in}}%
\pgfusepath{clip}%
\pgfsetbuttcap%
\pgfsetroundjoin%
\definecolor{currentfill}{rgb}{0.121569,0.466667,0.705882}%
\pgfsetfillcolor{currentfill}%
\pgfsetfillopacity{0.525715}%
\pgfsetlinewidth{1.003750pt}%
\definecolor{currentstroke}{rgb}{0.121569,0.466667,0.705882}%
\pgfsetstrokecolor{currentstroke}%
\pgfsetstrokeopacity{0.525715}%
\pgfsetdash{}{0pt}%
\pgfpathmoveto{\pgfqpoint{1.596843in}{1.848480in}}%
\pgfpathcurveto{\pgfqpoint{1.605079in}{1.848480in}}{\pgfqpoint{1.612979in}{1.851753in}}{\pgfqpoint{1.618803in}{1.857577in}}%
\pgfpathcurveto{\pgfqpoint{1.624627in}{1.863400in}}{\pgfqpoint{1.627899in}{1.871301in}}{\pgfqpoint{1.627899in}{1.879537in}}%
\pgfpathcurveto{\pgfqpoint{1.627899in}{1.887773in}}{\pgfqpoint{1.624627in}{1.895673in}}{\pgfqpoint{1.618803in}{1.901497in}}%
\pgfpathcurveto{\pgfqpoint{1.612979in}{1.907321in}}{\pgfqpoint{1.605079in}{1.910593in}}{\pgfqpoint{1.596843in}{1.910593in}}%
\pgfpathcurveto{\pgfqpoint{1.588606in}{1.910593in}}{\pgfqpoint{1.580706in}{1.907321in}}{\pgfqpoint{1.574882in}{1.901497in}}%
\pgfpathcurveto{\pgfqpoint{1.569058in}{1.895673in}}{\pgfqpoint{1.565786in}{1.887773in}}{\pgfqpoint{1.565786in}{1.879537in}}%
\pgfpathcurveto{\pgfqpoint{1.565786in}{1.871301in}}{\pgfqpoint{1.569058in}{1.863400in}}{\pgfqpoint{1.574882in}{1.857577in}}%
\pgfpathcurveto{\pgfqpoint{1.580706in}{1.851753in}}{\pgfqpoint{1.588606in}{1.848480in}}{\pgfqpoint{1.596843in}{1.848480in}}%
\pgfpathclose%
\pgfusepath{stroke,fill}%
\end{pgfscope}%
\begin{pgfscope}%
\pgfpathrectangle{\pgfqpoint{0.100000in}{0.212622in}}{\pgfqpoint{3.696000in}{3.696000in}}%
\pgfusepath{clip}%
\pgfsetbuttcap%
\pgfsetroundjoin%
\definecolor{currentfill}{rgb}{0.121569,0.466667,0.705882}%
\pgfsetfillcolor{currentfill}%
\pgfsetfillopacity{0.526086}%
\pgfsetlinewidth{1.003750pt}%
\definecolor{currentstroke}{rgb}{0.121569,0.466667,0.705882}%
\pgfsetstrokecolor{currentstroke}%
\pgfsetstrokeopacity{0.526086}%
\pgfsetdash{}{0pt}%
\pgfpathmoveto{\pgfqpoint{2.753833in}{2.581659in}}%
\pgfpathcurveto{\pgfqpoint{2.762069in}{2.581659in}}{\pgfqpoint{2.769969in}{2.584931in}}{\pgfqpoint{2.775793in}{2.590755in}}%
\pgfpathcurveto{\pgfqpoint{2.781617in}{2.596579in}}{\pgfqpoint{2.784889in}{2.604479in}}{\pgfqpoint{2.784889in}{2.612716in}}%
\pgfpathcurveto{\pgfqpoint{2.784889in}{2.620952in}}{\pgfqpoint{2.781617in}{2.628852in}}{\pgfqpoint{2.775793in}{2.634676in}}%
\pgfpathcurveto{\pgfqpoint{2.769969in}{2.640500in}}{\pgfqpoint{2.762069in}{2.643772in}}{\pgfqpoint{2.753833in}{2.643772in}}%
\pgfpathcurveto{\pgfqpoint{2.745596in}{2.643772in}}{\pgfqpoint{2.737696in}{2.640500in}}{\pgfqpoint{2.731872in}{2.634676in}}%
\pgfpathcurveto{\pgfqpoint{2.726048in}{2.628852in}}{\pgfqpoint{2.722776in}{2.620952in}}{\pgfqpoint{2.722776in}{2.612716in}}%
\pgfpathcurveto{\pgfqpoint{2.722776in}{2.604479in}}{\pgfqpoint{2.726048in}{2.596579in}}{\pgfqpoint{2.731872in}{2.590755in}}%
\pgfpathcurveto{\pgfqpoint{2.737696in}{2.584931in}}{\pgfqpoint{2.745596in}{2.581659in}}{\pgfqpoint{2.753833in}{2.581659in}}%
\pgfpathclose%
\pgfusepath{stroke,fill}%
\end{pgfscope}%
\begin{pgfscope}%
\pgfpathrectangle{\pgfqpoint{0.100000in}{0.212622in}}{\pgfqpoint{3.696000in}{3.696000in}}%
\pgfusepath{clip}%
\pgfsetbuttcap%
\pgfsetroundjoin%
\definecolor{currentfill}{rgb}{0.121569,0.466667,0.705882}%
\pgfsetfillcolor{currentfill}%
\pgfsetfillopacity{0.528425}%
\pgfsetlinewidth{1.003750pt}%
\definecolor{currentstroke}{rgb}{0.121569,0.466667,0.705882}%
\pgfsetstrokecolor{currentstroke}%
\pgfsetstrokeopacity{0.528425}%
\pgfsetdash{}{0pt}%
\pgfpathmoveto{\pgfqpoint{1.666620in}{1.895033in}}%
\pgfpathcurveto{\pgfqpoint{1.674857in}{1.895033in}}{\pgfqpoint{1.682757in}{1.898305in}}{\pgfqpoint{1.688581in}{1.904129in}}%
\pgfpathcurveto{\pgfqpoint{1.694405in}{1.909953in}}{\pgfqpoint{1.697677in}{1.917853in}}{\pgfqpoint{1.697677in}{1.926089in}}%
\pgfpathcurveto{\pgfqpoint{1.697677in}{1.934325in}}{\pgfqpoint{1.694405in}{1.942225in}}{\pgfqpoint{1.688581in}{1.948049in}}%
\pgfpathcurveto{\pgfqpoint{1.682757in}{1.953873in}}{\pgfqpoint{1.674857in}{1.957146in}}{\pgfqpoint{1.666620in}{1.957146in}}%
\pgfpathcurveto{\pgfqpoint{1.658384in}{1.957146in}}{\pgfqpoint{1.650484in}{1.953873in}}{\pgfqpoint{1.644660in}{1.948049in}}%
\pgfpathcurveto{\pgfqpoint{1.638836in}{1.942225in}}{\pgfqpoint{1.635564in}{1.934325in}}{\pgfqpoint{1.635564in}{1.926089in}}%
\pgfpathcurveto{\pgfqpoint{1.635564in}{1.917853in}}{\pgfqpoint{1.638836in}{1.909953in}}{\pgfqpoint{1.644660in}{1.904129in}}%
\pgfpathcurveto{\pgfqpoint{1.650484in}{1.898305in}}{\pgfqpoint{1.658384in}{1.895033in}}{\pgfqpoint{1.666620in}{1.895033in}}%
\pgfpathclose%
\pgfusepath{stroke,fill}%
\end{pgfscope}%
\begin{pgfscope}%
\pgfpathrectangle{\pgfqpoint{0.100000in}{0.212622in}}{\pgfqpoint{3.696000in}{3.696000in}}%
\pgfusepath{clip}%
\pgfsetbuttcap%
\pgfsetroundjoin%
\definecolor{currentfill}{rgb}{0.121569,0.466667,0.705882}%
\pgfsetfillcolor{currentfill}%
\pgfsetfillopacity{0.528688}%
\pgfsetlinewidth{1.003750pt}%
\definecolor{currentstroke}{rgb}{0.121569,0.466667,0.705882}%
\pgfsetstrokecolor{currentstroke}%
\pgfsetstrokeopacity{0.528688}%
\pgfsetdash{}{0pt}%
\pgfpathmoveto{\pgfqpoint{1.652754in}{1.889735in}}%
\pgfpathcurveto{\pgfqpoint{1.660990in}{1.889735in}}{\pgfqpoint{1.668890in}{1.893007in}}{\pgfqpoint{1.674714in}{1.898831in}}%
\pgfpathcurveto{\pgfqpoint{1.680538in}{1.904655in}}{\pgfqpoint{1.683811in}{1.912555in}}{\pgfqpoint{1.683811in}{1.920791in}}%
\pgfpathcurveto{\pgfqpoint{1.683811in}{1.929028in}}{\pgfqpoint{1.680538in}{1.936928in}}{\pgfqpoint{1.674714in}{1.942752in}}%
\pgfpathcurveto{\pgfqpoint{1.668890in}{1.948576in}}{\pgfqpoint{1.660990in}{1.951848in}}{\pgfqpoint{1.652754in}{1.951848in}}%
\pgfpathcurveto{\pgfqpoint{1.644518in}{1.951848in}}{\pgfqpoint{1.636618in}{1.948576in}}{\pgfqpoint{1.630794in}{1.942752in}}%
\pgfpathcurveto{\pgfqpoint{1.624970in}{1.936928in}}{\pgfqpoint{1.621698in}{1.929028in}}{\pgfqpoint{1.621698in}{1.920791in}}%
\pgfpathcurveto{\pgfqpoint{1.621698in}{1.912555in}}{\pgfqpoint{1.624970in}{1.904655in}}{\pgfqpoint{1.630794in}{1.898831in}}%
\pgfpathcurveto{\pgfqpoint{1.636618in}{1.893007in}}{\pgfqpoint{1.644518in}{1.889735in}}{\pgfqpoint{1.652754in}{1.889735in}}%
\pgfpathclose%
\pgfusepath{stroke,fill}%
\end{pgfscope}%
\begin{pgfscope}%
\pgfpathrectangle{\pgfqpoint{0.100000in}{0.212622in}}{\pgfqpoint{3.696000in}{3.696000in}}%
\pgfusepath{clip}%
\pgfsetbuttcap%
\pgfsetroundjoin%
\definecolor{currentfill}{rgb}{0.121569,0.466667,0.705882}%
\pgfsetfillcolor{currentfill}%
\pgfsetfillopacity{0.528720}%
\pgfsetlinewidth{1.003750pt}%
\definecolor{currentstroke}{rgb}{0.121569,0.466667,0.705882}%
\pgfsetstrokecolor{currentstroke}%
\pgfsetstrokeopacity{0.528720}%
\pgfsetdash{}{0pt}%
\pgfpathmoveto{\pgfqpoint{1.657506in}{1.888555in}}%
\pgfpathcurveto{\pgfqpoint{1.665742in}{1.888555in}}{\pgfqpoint{1.673642in}{1.891828in}}{\pgfqpoint{1.679466in}{1.897651in}}%
\pgfpathcurveto{\pgfqpoint{1.685290in}{1.903475in}}{\pgfqpoint{1.688562in}{1.911375in}}{\pgfqpoint{1.688562in}{1.919612in}}%
\pgfpathcurveto{\pgfqpoint{1.688562in}{1.927848in}}{\pgfqpoint{1.685290in}{1.935748in}}{\pgfqpoint{1.679466in}{1.941572in}}%
\pgfpathcurveto{\pgfqpoint{1.673642in}{1.947396in}}{\pgfqpoint{1.665742in}{1.950668in}}{\pgfqpoint{1.657506in}{1.950668in}}%
\pgfpathcurveto{\pgfqpoint{1.649269in}{1.950668in}}{\pgfqpoint{1.641369in}{1.947396in}}{\pgfqpoint{1.635545in}{1.941572in}}%
\pgfpathcurveto{\pgfqpoint{1.629722in}{1.935748in}}{\pgfqpoint{1.626449in}{1.927848in}}{\pgfqpoint{1.626449in}{1.919612in}}%
\pgfpathcurveto{\pgfqpoint{1.626449in}{1.911375in}}{\pgfqpoint{1.629722in}{1.903475in}}{\pgfqpoint{1.635545in}{1.897651in}}%
\pgfpathcurveto{\pgfqpoint{1.641369in}{1.891828in}}{\pgfqpoint{1.649269in}{1.888555in}}{\pgfqpoint{1.657506in}{1.888555in}}%
\pgfpathclose%
\pgfusepath{stroke,fill}%
\end{pgfscope}%
\begin{pgfscope}%
\pgfpathrectangle{\pgfqpoint{0.100000in}{0.212622in}}{\pgfqpoint{3.696000in}{3.696000in}}%
\pgfusepath{clip}%
\pgfsetbuttcap%
\pgfsetroundjoin%
\definecolor{currentfill}{rgb}{0.121569,0.466667,0.705882}%
\pgfsetfillcolor{currentfill}%
\pgfsetfillopacity{0.528872}%
\pgfsetlinewidth{1.003750pt}%
\definecolor{currentstroke}{rgb}{0.121569,0.466667,0.705882}%
\pgfsetstrokecolor{currentstroke}%
\pgfsetstrokeopacity{0.528872}%
\pgfsetdash{}{0pt}%
\pgfpathmoveto{\pgfqpoint{1.636980in}{1.879205in}}%
\pgfpathcurveto{\pgfqpoint{1.645216in}{1.879205in}}{\pgfqpoint{1.653116in}{1.882477in}}{\pgfqpoint{1.658940in}{1.888301in}}%
\pgfpathcurveto{\pgfqpoint{1.664764in}{1.894125in}}{\pgfqpoint{1.668036in}{1.902025in}}{\pgfqpoint{1.668036in}{1.910261in}}%
\pgfpathcurveto{\pgfqpoint{1.668036in}{1.918498in}}{\pgfqpoint{1.664764in}{1.926398in}}{\pgfqpoint{1.658940in}{1.932222in}}%
\pgfpathcurveto{\pgfqpoint{1.653116in}{1.938045in}}{\pgfqpoint{1.645216in}{1.941318in}}{\pgfqpoint{1.636980in}{1.941318in}}%
\pgfpathcurveto{\pgfqpoint{1.628744in}{1.941318in}}{\pgfqpoint{1.620844in}{1.938045in}}{\pgfqpoint{1.615020in}{1.932222in}}%
\pgfpathcurveto{\pgfqpoint{1.609196in}{1.926398in}}{\pgfqpoint{1.605923in}{1.918498in}}{\pgfqpoint{1.605923in}{1.910261in}}%
\pgfpathcurveto{\pgfqpoint{1.605923in}{1.902025in}}{\pgfqpoint{1.609196in}{1.894125in}}{\pgfqpoint{1.615020in}{1.888301in}}%
\pgfpathcurveto{\pgfqpoint{1.620844in}{1.882477in}}{\pgfqpoint{1.628744in}{1.879205in}}{\pgfqpoint{1.636980in}{1.879205in}}%
\pgfpathclose%
\pgfusepath{stroke,fill}%
\end{pgfscope}%
\begin{pgfscope}%
\pgfpathrectangle{\pgfqpoint{0.100000in}{0.212622in}}{\pgfqpoint{3.696000in}{3.696000in}}%
\pgfusepath{clip}%
\pgfsetbuttcap%
\pgfsetroundjoin%
\definecolor{currentfill}{rgb}{0.121569,0.466667,0.705882}%
\pgfsetfillcolor{currentfill}%
\pgfsetfillopacity{0.529588}%
\pgfsetlinewidth{1.003750pt}%
\definecolor{currentstroke}{rgb}{0.121569,0.466667,0.705882}%
\pgfsetstrokecolor{currentstroke}%
\pgfsetstrokeopacity{0.529588}%
\pgfsetdash{}{0pt}%
\pgfpathmoveto{\pgfqpoint{1.645334in}{1.884159in}}%
\pgfpathcurveto{\pgfqpoint{1.653570in}{1.884159in}}{\pgfqpoint{1.661470in}{1.887431in}}{\pgfqpoint{1.667294in}{1.893255in}}%
\pgfpathcurveto{\pgfqpoint{1.673118in}{1.899079in}}{\pgfqpoint{1.676390in}{1.906979in}}{\pgfqpoint{1.676390in}{1.915215in}}%
\pgfpathcurveto{\pgfqpoint{1.676390in}{1.923451in}}{\pgfqpoint{1.673118in}{1.931351in}}{\pgfqpoint{1.667294in}{1.937175in}}%
\pgfpathcurveto{\pgfqpoint{1.661470in}{1.942999in}}{\pgfqpoint{1.653570in}{1.946272in}}{\pgfqpoint{1.645334in}{1.946272in}}%
\pgfpathcurveto{\pgfqpoint{1.637098in}{1.946272in}}{\pgfqpoint{1.629198in}{1.942999in}}{\pgfqpoint{1.623374in}{1.937175in}}%
\pgfpathcurveto{\pgfqpoint{1.617550in}{1.931351in}}{\pgfqpoint{1.614277in}{1.923451in}}{\pgfqpoint{1.614277in}{1.915215in}}%
\pgfpathcurveto{\pgfqpoint{1.614277in}{1.906979in}}{\pgfqpoint{1.617550in}{1.899079in}}{\pgfqpoint{1.623374in}{1.893255in}}%
\pgfpathcurveto{\pgfqpoint{1.629198in}{1.887431in}}{\pgfqpoint{1.637098in}{1.884159in}}{\pgfqpoint{1.645334in}{1.884159in}}%
\pgfpathclose%
\pgfusepath{stroke,fill}%
\end{pgfscope}%
\begin{pgfscope}%
\pgfpathrectangle{\pgfqpoint{0.100000in}{0.212622in}}{\pgfqpoint{3.696000in}{3.696000in}}%
\pgfusepath{clip}%
\pgfsetbuttcap%
\pgfsetroundjoin%
\definecolor{currentfill}{rgb}{0.121569,0.466667,0.705882}%
\pgfsetfillcolor{currentfill}%
\pgfsetfillopacity{0.529621}%
\pgfsetlinewidth{1.003750pt}%
\definecolor{currentstroke}{rgb}{0.121569,0.466667,0.705882}%
\pgfsetstrokecolor{currentstroke}%
\pgfsetstrokeopacity{0.529621}%
\pgfsetdash{}{0pt}%
\pgfpathmoveto{\pgfqpoint{1.636326in}{1.878014in}}%
\pgfpathcurveto{\pgfqpoint{1.644562in}{1.878014in}}{\pgfqpoint{1.652462in}{1.881287in}}{\pgfqpoint{1.658286in}{1.887111in}}%
\pgfpathcurveto{\pgfqpoint{1.664110in}{1.892935in}}{\pgfqpoint{1.667382in}{1.900835in}}{\pgfqpoint{1.667382in}{1.909071in}}%
\pgfpathcurveto{\pgfqpoint{1.667382in}{1.917307in}}{\pgfqpoint{1.664110in}{1.925207in}}{\pgfqpoint{1.658286in}{1.931031in}}%
\pgfpathcurveto{\pgfqpoint{1.652462in}{1.936855in}}{\pgfqpoint{1.644562in}{1.940127in}}{\pgfqpoint{1.636326in}{1.940127in}}%
\pgfpathcurveto{\pgfqpoint{1.628089in}{1.940127in}}{\pgfqpoint{1.620189in}{1.936855in}}{\pgfqpoint{1.614365in}{1.931031in}}%
\pgfpathcurveto{\pgfqpoint{1.608541in}{1.925207in}}{\pgfqpoint{1.605269in}{1.917307in}}{\pgfqpoint{1.605269in}{1.909071in}}%
\pgfpathcurveto{\pgfqpoint{1.605269in}{1.900835in}}{\pgfqpoint{1.608541in}{1.892935in}}{\pgfqpoint{1.614365in}{1.887111in}}%
\pgfpathcurveto{\pgfqpoint{1.620189in}{1.881287in}}{\pgfqpoint{1.628089in}{1.878014in}}{\pgfqpoint{1.636326in}{1.878014in}}%
\pgfpathclose%
\pgfusepath{stroke,fill}%
\end{pgfscope}%
\begin{pgfscope}%
\pgfpathrectangle{\pgfqpoint{0.100000in}{0.212622in}}{\pgfqpoint{3.696000in}{3.696000in}}%
\pgfusepath{clip}%
\pgfsetbuttcap%
\pgfsetroundjoin%
\definecolor{currentfill}{rgb}{0.121569,0.466667,0.705882}%
\pgfsetfillcolor{currentfill}%
\pgfsetfillopacity{0.529654}%
\pgfsetlinewidth{1.003750pt}%
\definecolor{currentstroke}{rgb}{0.121569,0.466667,0.705882}%
\pgfsetstrokecolor{currentstroke}%
\pgfsetstrokeopacity{0.529654}%
\pgfsetdash{}{0pt}%
\pgfpathmoveto{\pgfqpoint{1.604072in}{1.853965in}}%
\pgfpathcurveto{\pgfqpoint{1.612308in}{1.853965in}}{\pgfqpoint{1.620208in}{1.857237in}}{\pgfqpoint{1.626032in}{1.863061in}}%
\pgfpathcurveto{\pgfqpoint{1.631856in}{1.868885in}}{\pgfqpoint{1.635128in}{1.876785in}}{\pgfqpoint{1.635128in}{1.885021in}}%
\pgfpathcurveto{\pgfqpoint{1.635128in}{1.893257in}}{\pgfqpoint{1.631856in}{1.901157in}}{\pgfqpoint{1.626032in}{1.906981in}}%
\pgfpathcurveto{\pgfqpoint{1.620208in}{1.912805in}}{\pgfqpoint{1.612308in}{1.916078in}}{\pgfqpoint{1.604072in}{1.916078in}}%
\pgfpathcurveto{\pgfqpoint{1.595835in}{1.916078in}}{\pgfqpoint{1.587935in}{1.912805in}}{\pgfqpoint{1.582111in}{1.906981in}}%
\pgfpathcurveto{\pgfqpoint{1.576287in}{1.901157in}}{\pgfqpoint{1.573015in}{1.893257in}}{\pgfqpoint{1.573015in}{1.885021in}}%
\pgfpathcurveto{\pgfqpoint{1.573015in}{1.876785in}}{\pgfqpoint{1.576287in}{1.868885in}}{\pgfqpoint{1.582111in}{1.863061in}}%
\pgfpathcurveto{\pgfqpoint{1.587935in}{1.857237in}}{\pgfqpoint{1.595835in}{1.853965in}}{\pgfqpoint{1.604072in}{1.853965in}}%
\pgfpathclose%
\pgfusepath{stroke,fill}%
\end{pgfscope}%
\begin{pgfscope}%
\pgfpathrectangle{\pgfqpoint{0.100000in}{0.212622in}}{\pgfqpoint{3.696000in}{3.696000in}}%
\pgfusepath{clip}%
\pgfsetbuttcap%
\pgfsetroundjoin%
\definecolor{currentfill}{rgb}{0.121569,0.466667,0.705882}%
\pgfsetfillcolor{currentfill}%
\pgfsetfillopacity{0.529706}%
\pgfsetlinewidth{1.003750pt}%
\definecolor{currentstroke}{rgb}{0.121569,0.466667,0.705882}%
\pgfsetstrokecolor{currentstroke}%
\pgfsetstrokeopacity{0.529706}%
\pgfsetdash{}{0pt}%
\pgfpathmoveto{\pgfqpoint{1.639654in}{1.880576in}}%
\pgfpathcurveto{\pgfqpoint{1.647890in}{1.880576in}}{\pgfqpoint{1.655790in}{1.883848in}}{\pgfqpoint{1.661614in}{1.889672in}}%
\pgfpathcurveto{\pgfqpoint{1.667438in}{1.895496in}}{\pgfqpoint{1.670710in}{1.903396in}}{\pgfqpoint{1.670710in}{1.911633in}}%
\pgfpathcurveto{\pgfqpoint{1.670710in}{1.919869in}}{\pgfqpoint{1.667438in}{1.927769in}}{\pgfqpoint{1.661614in}{1.933593in}}%
\pgfpathcurveto{\pgfqpoint{1.655790in}{1.939417in}}{\pgfqpoint{1.647890in}{1.942689in}}{\pgfqpoint{1.639654in}{1.942689in}}%
\pgfpathcurveto{\pgfqpoint{1.631417in}{1.942689in}}{\pgfqpoint{1.623517in}{1.939417in}}{\pgfqpoint{1.617693in}{1.933593in}}%
\pgfpathcurveto{\pgfqpoint{1.611869in}{1.927769in}}{\pgfqpoint{1.608597in}{1.919869in}}{\pgfqpoint{1.608597in}{1.911633in}}%
\pgfpathcurveto{\pgfqpoint{1.608597in}{1.903396in}}{\pgfqpoint{1.611869in}{1.895496in}}{\pgfqpoint{1.617693in}{1.889672in}}%
\pgfpathcurveto{\pgfqpoint{1.623517in}{1.883848in}}{\pgfqpoint{1.631417in}{1.880576in}}{\pgfqpoint{1.639654in}{1.880576in}}%
\pgfpathclose%
\pgfusepath{stroke,fill}%
\end{pgfscope}%
\begin{pgfscope}%
\pgfpathrectangle{\pgfqpoint{0.100000in}{0.212622in}}{\pgfqpoint{3.696000in}{3.696000in}}%
\pgfusepath{clip}%
\pgfsetbuttcap%
\pgfsetroundjoin%
\definecolor{currentfill}{rgb}{0.121569,0.466667,0.705882}%
\pgfsetfillcolor{currentfill}%
\pgfsetfillopacity{0.529950}%
\pgfsetlinewidth{1.003750pt}%
\definecolor{currentstroke}{rgb}{0.121569,0.466667,0.705882}%
\pgfsetstrokecolor{currentstroke}%
\pgfsetstrokeopacity{0.529950}%
\pgfsetdash{}{0pt}%
\pgfpathmoveto{\pgfqpoint{1.637726in}{1.879154in}}%
\pgfpathcurveto{\pgfqpoint{1.645962in}{1.879154in}}{\pgfqpoint{1.653862in}{1.882427in}}{\pgfqpoint{1.659686in}{1.888251in}}%
\pgfpathcurveto{\pgfqpoint{1.665510in}{1.894074in}}{\pgfqpoint{1.668782in}{1.901975in}}{\pgfqpoint{1.668782in}{1.910211in}}%
\pgfpathcurveto{\pgfqpoint{1.668782in}{1.918447in}}{\pgfqpoint{1.665510in}{1.926347in}}{\pgfqpoint{1.659686in}{1.932171in}}%
\pgfpathcurveto{\pgfqpoint{1.653862in}{1.937995in}}{\pgfqpoint{1.645962in}{1.941267in}}{\pgfqpoint{1.637726in}{1.941267in}}%
\pgfpathcurveto{\pgfqpoint{1.629490in}{1.941267in}}{\pgfqpoint{1.621590in}{1.937995in}}{\pgfqpoint{1.615766in}{1.932171in}}%
\pgfpathcurveto{\pgfqpoint{1.609942in}{1.926347in}}{\pgfqpoint{1.606669in}{1.918447in}}{\pgfqpoint{1.606669in}{1.910211in}}%
\pgfpathcurveto{\pgfqpoint{1.606669in}{1.901975in}}{\pgfqpoint{1.609942in}{1.894074in}}{\pgfqpoint{1.615766in}{1.888251in}}%
\pgfpathcurveto{\pgfqpoint{1.621590in}{1.882427in}}{\pgfqpoint{1.629490in}{1.879154in}}{\pgfqpoint{1.637726in}{1.879154in}}%
\pgfpathclose%
\pgfusepath{stroke,fill}%
\end{pgfscope}%
\begin{pgfscope}%
\pgfpathrectangle{\pgfqpoint{0.100000in}{0.212622in}}{\pgfqpoint{3.696000in}{3.696000in}}%
\pgfusepath{clip}%
\pgfsetbuttcap%
\pgfsetroundjoin%
\definecolor{currentfill}{rgb}{0.121569,0.466667,0.705882}%
\pgfsetfillcolor{currentfill}%
\pgfsetfillopacity{0.530030}%
\pgfsetlinewidth{1.003750pt}%
\definecolor{currentstroke}{rgb}{0.121569,0.466667,0.705882}%
\pgfsetstrokecolor{currentstroke}%
\pgfsetstrokeopacity{0.530030}%
\pgfsetdash{}{0pt}%
\pgfpathmoveto{\pgfqpoint{1.640741in}{1.881137in}}%
\pgfpathcurveto{\pgfqpoint{1.648977in}{1.881137in}}{\pgfqpoint{1.656877in}{1.884409in}}{\pgfqpoint{1.662701in}{1.890233in}}%
\pgfpathcurveto{\pgfqpoint{1.668525in}{1.896057in}}{\pgfqpoint{1.671798in}{1.903957in}}{\pgfqpoint{1.671798in}{1.912193in}}%
\pgfpathcurveto{\pgfqpoint{1.671798in}{1.920430in}}{\pgfqpoint{1.668525in}{1.928330in}}{\pgfqpoint{1.662701in}{1.934154in}}%
\pgfpathcurveto{\pgfqpoint{1.656877in}{1.939978in}}{\pgfqpoint{1.648977in}{1.943250in}}{\pgfqpoint{1.640741in}{1.943250in}}%
\pgfpathcurveto{\pgfqpoint{1.632505in}{1.943250in}}{\pgfqpoint{1.624605in}{1.939978in}}{\pgfqpoint{1.618781in}{1.934154in}}%
\pgfpathcurveto{\pgfqpoint{1.612957in}{1.928330in}}{\pgfqpoint{1.609685in}{1.920430in}}{\pgfqpoint{1.609685in}{1.912193in}}%
\pgfpathcurveto{\pgfqpoint{1.609685in}{1.903957in}}{\pgfqpoint{1.612957in}{1.896057in}}{\pgfqpoint{1.618781in}{1.890233in}}%
\pgfpathcurveto{\pgfqpoint{1.624605in}{1.884409in}}{\pgfqpoint{1.632505in}{1.881137in}}{\pgfqpoint{1.640741in}{1.881137in}}%
\pgfpathclose%
\pgfusepath{stroke,fill}%
\end{pgfscope}%
\begin{pgfscope}%
\pgfpathrectangle{\pgfqpoint{0.100000in}{0.212622in}}{\pgfqpoint{3.696000in}{3.696000in}}%
\pgfusepath{clip}%
\pgfsetbuttcap%
\pgfsetroundjoin%
\definecolor{currentfill}{rgb}{0.121569,0.466667,0.705882}%
\pgfsetfillcolor{currentfill}%
\pgfsetfillopacity{0.530773}%
\pgfsetlinewidth{1.003750pt}%
\definecolor{currentstroke}{rgb}{0.121569,0.466667,0.705882}%
\pgfsetstrokecolor{currentstroke}%
\pgfsetstrokeopacity{0.530773}%
\pgfsetdash{}{0pt}%
\pgfpathmoveto{\pgfqpoint{1.608455in}{1.855557in}}%
\pgfpathcurveto{\pgfqpoint{1.616692in}{1.855557in}}{\pgfqpoint{1.624592in}{1.858829in}}{\pgfqpoint{1.630416in}{1.864653in}}%
\pgfpathcurveto{\pgfqpoint{1.636240in}{1.870477in}}{\pgfqpoint{1.639512in}{1.878377in}}{\pgfqpoint{1.639512in}{1.886613in}}%
\pgfpathcurveto{\pgfqpoint{1.639512in}{1.894850in}}{\pgfqpoint{1.636240in}{1.902750in}}{\pgfqpoint{1.630416in}{1.908574in}}%
\pgfpathcurveto{\pgfqpoint{1.624592in}{1.914398in}}{\pgfqpoint{1.616692in}{1.917670in}}{\pgfqpoint{1.608455in}{1.917670in}}%
\pgfpathcurveto{\pgfqpoint{1.600219in}{1.917670in}}{\pgfqpoint{1.592319in}{1.914398in}}{\pgfqpoint{1.586495in}{1.908574in}}%
\pgfpathcurveto{\pgfqpoint{1.580671in}{1.902750in}}{\pgfqpoint{1.577399in}{1.894850in}}{\pgfqpoint{1.577399in}{1.886613in}}%
\pgfpathcurveto{\pgfqpoint{1.577399in}{1.878377in}}{\pgfqpoint{1.580671in}{1.870477in}}{\pgfqpoint{1.586495in}{1.864653in}}%
\pgfpathcurveto{\pgfqpoint{1.592319in}{1.858829in}}{\pgfqpoint{1.600219in}{1.855557in}}{\pgfqpoint{1.608455in}{1.855557in}}%
\pgfpathclose%
\pgfusepath{stroke,fill}%
\end{pgfscope}%
\begin{pgfscope}%
\pgfpathrectangle{\pgfqpoint{0.100000in}{0.212622in}}{\pgfqpoint{3.696000in}{3.696000in}}%
\pgfusepath{clip}%
\pgfsetbuttcap%
\pgfsetroundjoin%
\definecolor{currentfill}{rgb}{0.121569,0.466667,0.705882}%
\pgfsetfillcolor{currentfill}%
\pgfsetfillopacity{0.531049}%
\pgfsetlinewidth{1.003750pt}%
\definecolor{currentstroke}{rgb}{0.121569,0.466667,0.705882}%
\pgfsetstrokecolor{currentstroke}%
\pgfsetstrokeopacity{0.531049}%
\pgfsetdash{}{0pt}%
\pgfpathmoveto{\pgfqpoint{1.667033in}{1.889538in}}%
\pgfpathcurveto{\pgfqpoint{1.675270in}{1.889538in}}{\pgfqpoint{1.683170in}{1.892811in}}{\pgfqpoint{1.688994in}{1.898635in}}%
\pgfpathcurveto{\pgfqpoint{1.694817in}{1.904459in}}{\pgfqpoint{1.698090in}{1.912359in}}{\pgfqpoint{1.698090in}{1.920595in}}%
\pgfpathcurveto{\pgfqpoint{1.698090in}{1.928831in}}{\pgfqpoint{1.694817in}{1.936731in}}{\pgfqpoint{1.688994in}{1.942555in}}%
\pgfpathcurveto{\pgfqpoint{1.683170in}{1.948379in}}{\pgfqpoint{1.675270in}{1.951651in}}{\pgfqpoint{1.667033in}{1.951651in}}%
\pgfpathcurveto{\pgfqpoint{1.658797in}{1.951651in}}{\pgfqpoint{1.650897in}{1.948379in}}{\pgfqpoint{1.645073in}{1.942555in}}%
\pgfpathcurveto{\pgfqpoint{1.639249in}{1.936731in}}{\pgfqpoint{1.635977in}{1.928831in}}{\pgfqpoint{1.635977in}{1.920595in}}%
\pgfpathcurveto{\pgfqpoint{1.635977in}{1.912359in}}{\pgfqpoint{1.639249in}{1.904459in}}{\pgfqpoint{1.645073in}{1.898635in}}%
\pgfpathcurveto{\pgfqpoint{1.650897in}{1.892811in}}{\pgfqpoint{1.658797in}{1.889538in}}{\pgfqpoint{1.667033in}{1.889538in}}%
\pgfpathclose%
\pgfusepath{stroke,fill}%
\end{pgfscope}%
\begin{pgfscope}%
\pgfpathrectangle{\pgfqpoint{0.100000in}{0.212622in}}{\pgfqpoint{3.696000in}{3.696000in}}%
\pgfusepath{clip}%
\pgfsetbuttcap%
\pgfsetroundjoin%
\definecolor{currentfill}{rgb}{0.121569,0.466667,0.705882}%
\pgfsetfillcolor{currentfill}%
\pgfsetfillopacity{0.531160}%
\pgfsetlinewidth{1.003750pt}%
\definecolor{currentstroke}{rgb}{0.121569,0.466667,0.705882}%
\pgfsetstrokecolor{currentstroke}%
\pgfsetstrokeopacity{0.531160}%
\pgfsetdash{}{0pt}%
\pgfpathmoveto{\pgfqpoint{2.653074in}{2.515680in}}%
\pgfpathcurveto{\pgfqpoint{2.661310in}{2.515680in}}{\pgfqpoint{2.669210in}{2.518953in}}{\pgfqpoint{2.675034in}{2.524777in}}%
\pgfpathcurveto{\pgfqpoint{2.680858in}{2.530600in}}{\pgfqpoint{2.684130in}{2.538501in}}{\pgfqpoint{2.684130in}{2.546737in}}%
\pgfpathcurveto{\pgfqpoint{2.684130in}{2.554973in}}{\pgfqpoint{2.680858in}{2.562873in}}{\pgfqpoint{2.675034in}{2.568697in}}%
\pgfpathcurveto{\pgfqpoint{2.669210in}{2.574521in}}{\pgfqpoint{2.661310in}{2.577793in}}{\pgfqpoint{2.653074in}{2.577793in}}%
\pgfpathcurveto{\pgfqpoint{2.644837in}{2.577793in}}{\pgfqpoint{2.636937in}{2.574521in}}{\pgfqpoint{2.631113in}{2.568697in}}%
\pgfpathcurveto{\pgfqpoint{2.625289in}{2.562873in}}{\pgfqpoint{2.622017in}{2.554973in}}{\pgfqpoint{2.622017in}{2.546737in}}%
\pgfpathcurveto{\pgfqpoint{2.622017in}{2.538501in}}{\pgfqpoint{2.625289in}{2.530600in}}{\pgfqpoint{2.631113in}{2.524777in}}%
\pgfpathcurveto{\pgfqpoint{2.636937in}{2.518953in}}{\pgfqpoint{2.644837in}{2.515680in}}{\pgfqpoint{2.653074in}{2.515680in}}%
\pgfpathclose%
\pgfusepath{stroke,fill}%
\end{pgfscope}%
\begin{pgfscope}%
\pgfpathrectangle{\pgfqpoint{0.100000in}{0.212622in}}{\pgfqpoint{3.696000in}{3.696000in}}%
\pgfusepath{clip}%
\pgfsetbuttcap%
\pgfsetroundjoin%
\definecolor{currentfill}{rgb}{0.121569,0.466667,0.705882}%
\pgfsetfillcolor{currentfill}%
\pgfsetfillopacity{0.531168}%
\pgfsetlinewidth{1.003750pt}%
\definecolor{currentstroke}{rgb}{0.121569,0.466667,0.705882}%
\pgfsetstrokecolor{currentstroke}%
\pgfsetstrokeopacity{0.531168}%
\pgfsetdash{}{0pt}%
\pgfpathmoveto{\pgfqpoint{1.615531in}{1.859046in}}%
\pgfpathcurveto{\pgfqpoint{1.623767in}{1.859046in}}{\pgfqpoint{1.631667in}{1.862318in}}{\pgfqpoint{1.637491in}{1.868142in}}%
\pgfpathcurveto{\pgfqpoint{1.643315in}{1.873966in}}{\pgfqpoint{1.646587in}{1.881866in}}{\pgfqpoint{1.646587in}{1.890102in}}%
\pgfpathcurveto{\pgfqpoint{1.646587in}{1.898338in}}{\pgfqpoint{1.643315in}{1.906238in}}{\pgfqpoint{1.637491in}{1.912062in}}%
\pgfpathcurveto{\pgfqpoint{1.631667in}{1.917886in}}{\pgfqpoint{1.623767in}{1.921159in}}{\pgfqpoint{1.615531in}{1.921159in}}%
\pgfpathcurveto{\pgfqpoint{1.607294in}{1.921159in}}{\pgfqpoint{1.599394in}{1.917886in}}{\pgfqpoint{1.593570in}{1.912062in}}%
\pgfpathcurveto{\pgfqpoint{1.587746in}{1.906238in}}{\pgfqpoint{1.584474in}{1.898338in}}{\pgfqpoint{1.584474in}{1.890102in}}%
\pgfpathcurveto{\pgfqpoint{1.584474in}{1.881866in}}{\pgfqpoint{1.587746in}{1.873966in}}{\pgfqpoint{1.593570in}{1.868142in}}%
\pgfpathcurveto{\pgfqpoint{1.599394in}{1.862318in}}{\pgfqpoint{1.607294in}{1.859046in}}{\pgfqpoint{1.615531in}{1.859046in}}%
\pgfpathclose%
\pgfusepath{stroke,fill}%
\end{pgfscope}%
\begin{pgfscope}%
\pgfpathrectangle{\pgfqpoint{0.100000in}{0.212622in}}{\pgfqpoint{3.696000in}{3.696000in}}%
\pgfusepath{clip}%
\pgfsetbuttcap%
\pgfsetroundjoin%
\definecolor{currentfill}{rgb}{0.121569,0.466667,0.705882}%
\pgfsetfillcolor{currentfill}%
\pgfsetfillopacity{0.531251}%
\pgfsetlinewidth{1.003750pt}%
\definecolor{currentstroke}{rgb}{0.121569,0.466667,0.705882}%
\pgfsetstrokecolor{currentstroke}%
\pgfsetstrokeopacity{0.531251}%
\pgfsetdash{}{0pt}%
\pgfpathmoveto{\pgfqpoint{1.627043in}{1.871132in}}%
\pgfpathcurveto{\pgfqpoint{1.635279in}{1.871132in}}{\pgfqpoint{1.643179in}{1.874404in}}{\pgfqpoint{1.649003in}{1.880228in}}%
\pgfpathcurveto{\pgfqpoint{1.654827in}{1.886052in}}{\pgfqpoint{1.658099in}{1.893952in}}{\pgfqpoint{1.658099in}{1.902188in}}%
\pgfpathcurveto{\pgfqpoint{1.658099in}{1.910424in}}{\pgfqpoint{1.654827in}{1.918324in}}{\pgfqpoint{1.649003in}{1.924148in}}%
\pgfpathcurveto{\pgfqpoint{1.643179in}{1.929972in}}{\pgfqpoint{1.635279in}{1.933245in}}{\pgfqpoint{1.627043in}{1.933245in}}%
\pgfpathcurveto{\pgfqpoint{1.618806in}{1.933245in}}{\pgfqpoint{1.610906in}{1.929972in}}{\pgfqpoint{1.605082in}{1.924148in}}%
\pgfpathcurveto{\pgfqpoint{1.599258in}{1.918324in}}{\pgfqpoint{1.595986in}{1.910424in}}{\pgfqpoint{1.595986in}{1.902188in}}%
\pgfpathcurveto{\pgfqpoint{1.595986in}{1.893952in}}{\pgfqpoint{1.599258in}{1.886052in}}{\pgfqpoint{1.605082in}{1.880228in}}%
\pgfpathcurveto{\pgfqpoint{1.610906in}{1.874404in}}{\pgfqpoint{1.618806in}{1.871132in}}{\pgfqpoint{1.627043in}{1.871132in}}%
\pgfpathclose%
\pgfusepath{stroke,fill}%
\end{pgfscope}%
\begin{pgfscope}%
\pgfpathrectangle{\pgfqpoint{0.100000in}{0.212622in}}{\pgfqpoint{3.696000in}{3.696000in}}%
\pgfusepath{clip}%
\pgfsetbuttcap%
\pgfsetroundjoin%
\definecolor{currentfill}{rgb}{0.121569,0.466667,0.705882}%
\pgfsetfillcolor{currentfill}%
\pgfsetfillopacity{0.531399}%
\pgfsetlinewidth{1.003750pt}%
\definecolor{currentstroke}{rgb}{0.121569,0.466667,0.705882}%
\pgfsetstrokecolor{currentstroke}%
\pgfsetstrokeopacity{0.531399}%
\pgfsetdash{}{0pt}%
\pgfpathmoveto{\pgfqpoint{1.619961in}{1.862935in}}%
\pgfpathcurveto{\pgfqpoint{1.628198in}{1.862935in}}{\pgfqpoint{1.636098in}{1.866207in}}{\pgfqpoint{1.641922in}{1.872031in}}%
\pgfpathcurveto{\pgfqpoint{1.647746in}{1.877855in}}{\pgfqpoint{1.651018in}{1.885755in}}{\pgfqpoint{1.651018in}{1.893991in}}%
\pgfpathcurveto{\pgfqpoint{1.651018in}{1.902228in}}{\pgfqpoint{1.647746in}{1.910128in}}{\pgfqpoint{1.641922in}{1.915952in}}%
\pgfpathcurveto{\pgfqpoint{1.636098in}{1.921776in}}{\pgfqpoint{1.628198in}{1.925048in}}{\pgfqpoint{1.619961in}{1.925048in}}%
\pgfpathcurveto{\pgfqpoint{1.611725in}{1.925048in}}{\pgfqpoint{1.603825in}{1.921776in}}{\pgfqpoint{1.598001in}{1.915952in}}%
\pgfpathcurveto{\pgfqpoint{1.592177in}{1.910128in}}{\pgfqpoint{1.588905in}{1.902228in}}{\pgfqpoint{1.588905in}{1.893991in}}%
\pgfpathcurveto{\pgfqpoint{1.588905in}{1.885755in}}{\pgfqpoint{1.592177in}{1.877855in}}{\pgfqpoint{1.598001in}{1.872031in}}%
\pgfpathcurveto{\pgfqpoint{1.603825in}{1.866207in}}{\pgfqpoint{1.611725in}{1.862935in}}{\pgfqpoint{1.619961in}{1.862935in}}%
\pgfpathclose%
\pgfusepath{stroke,fill}%
\end{pgfscope}%
\begin{pgfscope}%
\pgfpathrectangle{\pgfqpoint{0.100000in}{0.212622in}}{\pgfqpoint{3.696000in}{3.696000in}}%
\pgfusepath{clip}%
\pgfsetbuttcap%
\pgfsetroundjoin%
\definecolor{currentfill}{rgb}{0.121569,0.466667,0.705882}%
\pgfsetfillcolor{currentfill}%
\pgfsetfillopacity{0.531987}%
\pgfsetlinewidth{1.003750pt}%
\definecolor{currentstroke}{rgb}{0.121569,0.466667,0.705882}%
\pgfsetstrokecolor{currentstroke}%
\pgfsetstrokeopacity{0.531987}%
\pgfsetdash{}{0pt}%
\pgfpathmoveto{\pgfqpoint{1.749377in}{1.951215in}}%
\pgfpathcurveto{\pgfqpoint{1.757614in}{1.951215in}}{\pgfqpoint{1.765514in}{1.954487in}}{\pgfqpoint{1.771338in}{1.960311in}}%
\pgfpathcurveto{\pgfqpoint{1.777161in}{1.966135in}}{\pgfqpoint{1.780434in}{1.974035in}}{\pgfqpoint{1.780434in}{1.982271in}}%
\pgfpathcurveto{\pgfqpoint{1.780434in}{1.990508in}}{\pgfqpoint{1.777161in}{1.998408in}}{\pgfqpoint{1.771338in}{2.004232in}}%
\pgfpathcurveto{\pgfqpoint{1.765514in}{2.010056in}}{\pgfqpoint{1.757614in}{2.013328in}}{\pgfqpoint{1.749377in}{2.013328in}}%
\pgfpathcurveto{\pgfqpoint{1.741141in}{2.013328in}}{\pgfqpoint{1.733241in}{2.010056in}}{\pgfqpoint{1.727417in}{2.004232in}}%
\pgfpathcurveto{\pgfqpoint{1.721593in}{1.998408in}}{\pgfqpoint{1.718321in}{1.990508in}}{\pgfqpoint{1.718321in}{1.982271in}}%
\pgfpathcurveto{\pgfqpoint{1.718321in}{1.974035in}}{\pgfqpoint{1.721593in}{1.966135in}}{\pgfqpoint{1.727417in}{1.960311in}}%
\pgfpathcurveto{\pgfqpoint{1.733241in}{1.954487in}}{\pgfqpoint{1.741141in}{1.951215in}}{\pgfqpoint{1.749377in}{1.951215in}}%
\pgfpathclose%
\pgfusepath{stroke,fill}%
\end{pgfscope}%
\begin{pgfscope}%
\pgfpathrectangle{\pgfqpoint{0.100000in}{0.212622in}}{\pgfqpoint{3.696000in}{3.696000in}}%
\pgfusepath{clip}%
\pgfsetbuttcap%
\pgfsetroundjoin%
\definecolor{currentfill}{rgb}{0.121569,0.466667,0.705882}%
\pgfsetfillcolor{currentfill}%
\pgfsetfillopacity{0.532120}%
\pgfsetlinewidth{1.003750pt}%
\definecolor{currentstroke}{rgb}{0.121569,0.466667,0.705882}%
\pgfsetstrokecolor{currentstroke}%
\pgfsetstrokeopacity{0.532120}%
\pgfsetdash{}{0pt}%
\pgfpathmoveto{\pgfqpoint{1.675291in}{1.898082in}}%
\pgfpathcurveto{\pgfqpoint{1.683528in}{1.898082in}}{\pgfqpoint{1.691428in}{1.901354in}}{\pgfqpoint{1.697252in}{1.907178in}}%
\pgfpathcurveto{\pgfqpoint{1.703076in}{1.913002in}}{\pgfqpoint{1.706348in}{1.920902in}}{\pgfqpoint{1.706348in}{1.929139in}}%
\pgfpathcurveto{\pgfqpoint{1.706348in}{1.937375in}}{\pgfqpoint{1.703076in}{1.945275in}}{\pgfqpoint{1.697252in}{1.951099in}}%
\pgfpathcurveto{\pgfqpoint{1.691428in}{1.956923in}}{\pgfqpoint{1.683528in}{1.960195in}}{\pgfqpoint{1.675291in}{1.960195in}}%
\pgfpathcurveto{\pgfqpoint{1.667055in}{1.960195in}}{\pgfqpoint{1.659155in}{1.956923in}}{\pgfqpoint{1.653331in}{1.951099in}}%
\pgfpathcurveto{\pgfqpoint{1.647507in}{1.945275in}}{\pgfqpoint{1.644235in}{1.937375in}}{\pgfqpoint{1.644235in}{1.929139in}}%
\pgfpathcurveto{\pgfqpoint{1.644235in}{1.920902in}}{\pgfqpoint{1.647507in}{1.913002in}}{\pgfqpoint{1.653331in}{1.907178in}}%
\pgfpathcurveto{\pgfqpoint{1.659155in}{1.901354in}}{\pgfqpoint{1.667055in}{1.898082in}}{\pgfqpoint{1.675291in}{1.898082in}}%
\pgfpathclose%
\pgfusepath{stroke,fill}%
\end{pgfscope}%
\begin{pgfscope}%
\pgfpathrectangle{\pgfqpoint{0.100000in}{0.212622in}}{\pgfqpoint{3.696000in}{3.696000in}}%
\pgfusepath{clip}%
\pgfsetbuttcap%
\pgfsetroundjoin%
\definecolor{currentfill}{rgb}{0.121569,0.466667,0.705882}%
\pgfsetfillcolor{currentfill}%
\pgfsetfillopacity{0.535270}%
\pgfsetlinewidth{1.003750pt}%
\definecolor{currentstroke}{rgb}{0.121569,0.466667,0.705882}%
\pgfsetstrokecolor{currentstroke}%
\pgfsetstrokeopacity{0.535270}%
\pgfsetdash{}{0pt}%
\pgfpathmoveto{\pgfqpoint{1.674546in}{1.888520in}}%
\pgfpathcurveto{\pgfqpoint{1.682782in}{1.888520in}}{\pgfqpoint{1.690682in}{1.891792in}}{\pgfqpoint{1.696506in}{1.897616in}}%
\pgfpathcurveto{\pgfqpoint{1.702330in}{1.903440in}}{\pgfqpoint{1.705602in}{1.911340in}}{\pgfqpoint{1.705602in}{1.919576in}}%
\pgfpathcurveto{\pgfqpoint{1.705602in}{1.927813in}}{\pgfqpoint{1.702330in}{1.935713in}}{\pgfqpoint{1.696506in}{1.941537in}}%
\pgfpathcurveto{\pgfqpoint{1.690682in}{1.947361in}}{\pgfqpoint{1.682782in}{1.950633in}}{\pgfqpoint{1.674546in}{1.950633in}}%
\pgfpathcurveto{\pgfqpoint{1.666309in}{1.950633in}}{\pgfqpoint{1.658409in}{1.947361in}}{\pgfqpoint{1.652585in}{1.941537in}}%
\pgfpathcurveto{\pgfqpoint{1.646762in}{1.935713in}}{\pgfqpoint{1.643489in}{1.927813in}}{\pgfqpoint{1.643489in}{1.919576in}}%
\pgfpathcurveto{\pgfqpoint{1.643489in}{1.911340in}}{\pgfqpoint{1.646762in}{1.903440in}}{\pgfqpoint{1.652585in}{1.897616in}}%
\pgfpathcurveto{\pgfqpoint{1.658409in}{1.891792in}}{\pgfqpoint{1.666309in}{1.888520in}}{\pgfqpoint{1.674546in}{1.888520in}}%
\pgfpathclose%
\pgfusepath{stroke,fill}%
\end{pgfscope}%
\begin{pgfscope}%
\pgfpathrectangle{\pgfqpoint{0.100000in}{0.212622in}}{\pgfqpoint{3.696000in}{3.696000in}}%
\pgfusepath{clip}%
\pgfsetbuttcap%
\pgfsetroundjoin%
\definecolor{currentfill}{rgb}{0.121569,0.466667,0.705882}%
\pgfsetfillcolor{currentfill}%
\pgfsetfillopacity{0.535915}%
\pgfsetlinewidth{1.003750pt}%
\definecolor{currentstroke}{rgb}{0.121569,0.466667,0.705882}%
\pgfsetstrokecolor{currentstroke}%
\pgfsetstrokeopacity{0.535915}%
\pgfsetdash{}{0pt}%
\pgfpathmoveto{\pgfqpoint{2.680690in}{2.526044in}}%
\pgfpathcurveto{\pgfqpoint{2.688926in}{2.526044in}}{\pgfqpoint{2.696826in}{2.529316in}}{\pgfqpoint{2.702650in}{2.535140in}}%
\pgfpathcurveto{\pgfqpoint{2.708474in}{2.540964in}}{\pgfqpoint{2.711747in}{2.548864in}}{\pgfqpoint{2.711747in}{2.557101in}}%
\pgfpathcurveto{\pgfqpoint{2.711747in}{2.565337in}}{\pgfqpoint{2.708474in}{2.573237in}}{\pgfqpoint{2.702650in}{2.579061in}}%
\pgfpathcurveto{\pgfqpoint{2.696826in}{2.584885in}}{\pgfqpoint{2.688926in}{2.588157in}}{\pgfqpoint{2.680690in}{2.588157in}}%
\pgfpathcurveto{\pgfqpoint{2.672454in}{2.588157in}}{\pgfqpoint{2.664554in}{2.584885in}}{\pgfqpoint{2.658730in}{2.579061in}}%
\pgfpathcurveto{\pgfqpoint{2.652906in}{2.573237in}}{\pgfqpoint{2.649634in}{2.565337in}}{\pgfqpoint{2.649634in}{2.557101in}}%
\pgfpathcurveto{\pgfqpoint{2.649634in}{2.548864in}}{\pgfqpoint{2.652906in}{2.540964in}}{\pgfqpoint{2.658730in}{2.535140in}}%
\pgfpathcurveto{\pgfqpoint{2.664554in}{2.529316in}}{\pgfqpoint{2.672454in}{2.526044in}}{\pgfqpoint{2.680690in}{2.526044in}}%
\pgfpathclose%
\pgfusepath{stroke,fill}%
\end{pgfscope}%
\begin{pgfscope}%
\pgfpathrectangle{\pgfqpoint{0.100000in}{0.212622in}}{\pgfqpoint{3.696000in}{3.696000in}}%
\pgfusepath{clip}%
\pgfsetbuttcap%
\pgfsetroundjoin%
\definecolor{currentfill}{rgb}{0.121569,0.466667,0.705882}%
\pgfsetfillcolor{currentfill}%
\pgfsetfillopacity{0.535923}%
\pgfsetlinewidth{1.003750pt}%
\definecolor{currentstroke}{rgb}{0.121569,0.466667,0.705882}%
\pgfsetstrokecolor{currentstroke}%
\pgfsetstrokeopacity{0.535923}%
\pgfsetdash{}{0pt}%
\pgfpathmoveto{\pgfqpoint{1.709651in}{1.919656in}}%
\pgfpathcurveto{\pgfqpoint{1.717887in}{1.919656in}}{\pgfqpoint{1.725787in}{1.922928in}}{\pgfqpoint{1.731611in}{1.928752in}}%
\pgfpathcurveto{\pgfqpoint{1.737435in}{1.934576in}}{\pgfqpoint{1.740708in}{1.942476in}}{\pgfqpoint{1.740708in}{1.950712in}}%
\pgfpathcurveto{\pgfqpoint{1.740708in}{1.958948in}}{\pgfqpoint{1.737435in}{1.966848in}}{\pgfqpoint{1.731611in}{1.972672in}}%
\pgfpathcurveto{\pgfqpoint{1.725787in}{1.978496in}}{\pgfqpoint{1.717887in}{1.981769in}}{\pgfqpoint{1.709651in}{1.981769in}}%
\pgfpathcurveto{\pgfqpoint{1.701415in}{1.981769in}}{\pgfqpoint{1.693515in}{1.978496in}}{\pgfqpoint{1.687691in}{1.972672in}}%
\pgfpathcurveto{\pgfqpoint{1.681867in}{1.966848in}}{\pgfqpoint{1.678595in}{1.958948in}}{\pgfqpoint{1.678595in}{1.950712in}}%
\pgfpathcurveto{\pgfqpoint{1.678595in}{1.942476in}}{\pgfqpoint{1.681867in}{1.934576in}}{\pgfqpoint{1.687691in}{1.928752in}}%
\pgfpathcurveto{\pgfqpoint{1.693515in}{1.922928in}}{\pgfqpoint{1.701415in}{1.919656in}}{\pgfqpoint{1.709651in}{1.919656in}}%
\pgfpathclose%
\pgfusepath{stroke,fill}%
\end{pgfscope}%
\begin{pgfscope}%
\pgfpathrectangle{\pgfqpoint{0.100000in}{0.212622in}}{\pgfqpoint{3.696000in}{3.696000in}}%
\pgfusepath{clip}%
\pgfsetbuttcap%
\pgfsetroundjoin%
\definecolor{currentfill}{rgb}{0.121569,0.466667,0.705882}%
\pgfsetfillcolor{currentfill}%
\pgfsetfillopacity{0.536077}%
\pgfsetlinewidth{1.003750pt}%
\definecolor{currentstroke}{rgb}{0.121569,0.466667,0.705882}%
\pgfsetstrokecolor{currentstroke}%
\pgfsetstrokeopacity{0.536077}%
\pgfsetdash{}{0pt}%
\pgfpathmoveto{\pgfqpoint{1.723187in}{1.931430in}}%
\pgfpathcurveto{\pgfqpoint{1.731423in}{1.931430in}}{\pgfqpoint{1.739323in}{1.934702in}}{\pgfqpoint{1.745147in}{1.940526in}}%
\pgfpathcurveto{\pgfqpoint{1.750971in}{1.946350in}}{\pgfqpoint{1.754243in}{1.954250in}}{\pgfqpoint{1.754243in}{1.962486in}}%
\pgfpathcurveto{\pgfqpoint{1.754243in}{1.970722in}}{\pgfqpoint{1.750971in}{1.978622in}}{\pgfqpoint{1.745147in}{1.984446in}}%
\pgfpathcurveto{\pgfqpoint{1.739323in}{1.990270in}}{\pgfqpoint{1.731423in}{1.993543in}}{\pgfqpoint{1.723187in}{1.993543in}}%
\pgfpathcurveto{\pgfqpoint{1.714950in}{1.993543in}}{\pgfqpoint{1.707050in}{1.990270in}}{\pgfqpoint{1.701226in}{1.984446in}}%
\pgfpathcurveto{\pgfqpoint{1.695403in}{1.978622in}}{\pgfqpoint{1.692130in}{1.970722in}}{\pgfqpoint{1.692130in}{1.962486in}}%
\pgfpathcurveto{\pgfqpoint{1.692130in}{1.954250in}}{\pgfqpoint{1.695403in}{1.946350in}}{\pgfqpoint{1.701226in}{1.940526in}}%
\pgfpathcurveto{\pgfqpoint{1.707050in}{1.934702in}}{\pgfqpoint{1.714950in}{1.931430in}}{\pgfqpoint{1.723187in}{1.931430in}}%
\pgfpathclose%
\pgfusepath{stroke,fill}%
\end{pgfscope}%
\begin{pgfscope}%
\pgfpathrectangle{\pgfqpoint{0.100000in}{0.212622in}}{\pgfqpoint{3.696000in}{3.696000in}}%
\pgfusepath{clip}%
\pgfsetbuttcap%
\pgfsetroundjoin%
\definecolor{currentfill}{rgb}{0.121569,0.466667,0.705882}%
\pgfsetfillcolor{currentfill}%
\pgfsetfillopacity{0.536216}%
\pgfsetlinewidth{1.003750pt}%
\definecolor{currentstroke}{rgb}{0.121569,0.466667,0.705882}%
\pgfsetstrokecolor{currentstroke}%
\pgfsetstrokeopacity{0.536216}%
\pgfsetdash{}{0pt}%
\pgfpathmoveto{\pgfqpoint{1.713026in}{1.921805in}}%
\pgfpathcurveto{\pgfqpoint{1.721262in}{1.921805in}}{\pgfqpoint{1.729162in}{1.925077in}}{\pgfqpoint{1.734986in}{1.930901in}}%
\pgfpathcurveto{\pgfqpoint{1.740810in}{1.936725in}}{\pgfqpoint{1.744082in}{1.944625in}}{\pgfqpoint{1.744082in}{1.952861in}}%
\pgfpathcurveto{\pgfqpoint{1.744082in}{1.961097in}}{\pgfqpoint{1.740810in}{1.968997in}}{\pgfqpoint{1.734986in}{1.974821in}}%
\pgfpathcurveto{\pgfqpoint{1.729162in}{1.980645in}}{\pgfqpoint{1.721262in}{1.983918in}}{\pgfqpoint{1.713026in}{1.983918in}}%
\pgfpathcurveto{\pgfqpoint{1.704789in}{1.983918in}}{\pgfqpoint{1.696889in}{1.980645in}}{\pgfqpoint{1.691065in}{1.974821in}}%
\pgfpathcurveto{\pgfqpoint{1.685241in}{1.968997in}}{\pgfqpoint{1.681969in}{1.961097in}}{\pgfqpoint{1.681969in}{1.952861in}}%
\pgfpathcurveto{\pgfqpoint{1.681969in}{1.944625in}}{\pgfqpoint{1.685241in}{1.936725in}}{\pgfqpoint{1.691065in}{1.930901in}}%
\pgfpathcurveto{\pgfqpoint{1.696889in}{1.925077in}}{\pgfqpoint{1.704789in}{1.921805in}}{\pgfqpoint{1.713026in}{1.921805in}}%
\pgfpathclose%
\pgfusepath{stroke,fill}%
\end{pgfscope}%
\begin{pgfscope}%
\pgfpathrectangle{\pgfqpoint{0.100000in}{0.212622in}}{\pgfqpoint{3.696000in}{3.696000in}}%
\pgfusepath{clip}%
\pgfsetbuttcap%
\pgfsetroundjoin%
\definecolor{currentfill}{rgb}{0.121569,0.466667,0.705882}%
\pgfsetfillcolor{currentfill}%
\pgfsetfillopacity{0.536257}%
\pgfsetlinewidth{1.003750pt}%
\definecolor{currentstroke}{rgb}{0.121569,0.466667,0.705882}%
\pgfsetstrokecolor{currentstroke}%
\pgfsetstrokeopacity{0.536257}%
\pgfsetdash{}{0pt}%
\pgfpathmoveto{\pgfqpoint{1.705051in}{1.914318in}}%
\pgfpathcurveto{\pgfqpoint{1.713287in}{1.914318in}}{\pgfqpoint{1.721187in}{1.917590in}}{\pgfqpoint{1.727011in}{1.923414in}}%
\pgfpathcurveto{\pgfqpoint{1.732835in}{1.929238in}}{\pgfqpoint{1.736107in}{1.937138in}}{\pgfqpoint{1.736107in}{1.945374in}}%
\pgfpathcurveto{\pgfqpoint{1.736107in}{1.953611in}}{\pgfqpoint{1.732835in}{1.961511in}}{\pgfqpoint{1.727011in}{1.967335in}}%
\pgfpathcurveto{\pgfqpoint{1.721187in}{1.973158in}}{\pgfqpoint{1.713287in}{1.976431in}}{\pgfqpoint{1.705051in}{1.976431in}}%
\pgfpathcurveto{\pgfqpoint{1.696814in}{1.976431in}}{\pgfqpoint{1.688914in}{1.973158in}}{\pgfqpoint{1.683090in}{1.967335in}}%
\pgfpathcurveto{\pgfqpoint{1.677266in}{1.961511in}}{\pgfqpoint{1.673994in}{1.953611in}}{\pgfqpoint{1.673994in}{1.945374in}}%
\pgfpathcurveto{\pgfqpoint{1.673994in}{1.937138in}}{\pgfqpoint{1.677266in}{1.929238in}}{\pgfqpoint{1.683090in}{1.923414in}}%
\pgfpathcurveto{\pgfqpoint{1.688914in}{1.917590in}}{\pgfqpoint{1.696814in}{1.914318in}}{\pgfqpoint{1.705051in}{1.914318in}}%
\pgfpathclose%
\pgfusepath{stroke,fill}%
\end{pgfscope}%
\begin{pgfscope}%
\pgfpathrectangle{\pgfqpoint{0.100000in}{0.212622in}}{\pgfqpoint{3.696000in}{3.696000in}}%
\pgfusepath{clip}%
\pgfsetbuttcap%
\pgfsetroundjoin%
\definecolor{currentfill}{rgb}{0.121569,0.466667,0.705882}%
\pgfsetfillcolor{currentfill}%
\pgfsetfillopacity{0.536263}%
\pgfsetlinewidth{1.003750pt}%
\definecolor{currentstroke}{rgb}{0.121569,0.466667,0.705882}%
\pgfsetstrokecolor{currentstroke}%
\pgfsetstrokeopacity{0.536263}%
\pgfsetdash{}{0pt}%
\pgfpathmoveto{\pgfqpoint{1.698232in}{1.908013in}}%
\pgfpathcurveto{\pgfqpoint{1.706468in}{1.908013in}}{\pgfqpoint{1.714368in}{1.911285in}}{\pgfqpoint{1.720192in}{1.917109in}}%
\pgfpathcurveto{\pgfqpoint{1.726016in}{1.922933in}}{\pgfqpoint{1.729288in}{1.930833in}}{\pgfqpoint{1.729288in}{1.939069in}}%
\pgfpathcurveto{\pgfqpoint{1.729288in}{1.947306in}}{\pgfqpoint{1.726016in}{1.955206in}}{\pgfqpoint{1.720192in}{1.961030in}}%
\pgfpathcurveto{\pgfqpoint{1.714368in}{1.966854in}}{\pgfqpoint{1.706468in}{1.970126in}}{\pgfqpoint{1.698232in}{1.970126in}}%
\pgfpathcurveto{\pgfqpoint{1.689995in}{1.970126in}}{\pgfqpoint{1.682095in}{1.966854in}}{\pgfqpoint{1.676271in}{1.961030in}}%
\pgfpathcurveto{\pgfqpoint{1.670447in}{1.955206in}}{\pgfqpoint{1.667175in}{1.947306in}}{\pgfqpoint{1.667175in}{1.939069in}}%
\pgfpathcurveto{\pgfqpoint{1.667175in}{1.930833in}}{\pgfqpoint{1.670447in}{1.922933in}}{\pgfqpoint{1.676271in}{1.917109in}}%
\pgfpathcurveto{\pgfqpoint{1.682095in}{1.911285in}}{\pgfqpoint{1.689995in}{1.908013in}}{\pgfqpoint{1.698232in}{1.908013in}}%
\pgfpathclose%
\pgfusepath{stroke,fill}%
\end{pgfscope}%
\begin{pgfscope}%
\pgfpathrectangle{\pgfqpoint{0.100000in}{0.212622in}}{\pgfqpoint{3.696000in}{3.696000in}}%
\pgfusepath{clip}%
\pgfsetbuttcap%
\pgfsetroundjoin%
\definecolor{currentfill}{rgb}{0.121569,0.466667,0.705882}%
\pgfsetfillcolor{currentfill}%
\pgfsetfillopacity{0.536511}%
\pgfsetlinewidth{1.003750pt}%
\definecolor{currentstroke}{rgb}{0.121569,0.466667,0.705882}%
\pgfsetstrokecolor{currentstroke}%
\pgfsetstrokeopacity{0.536511}%
\pgfsetdash{}{0pt}%
\pgfpathmoveto{\pgfqpoint{1.683866in}{1.898046in}}%
\pgfpathcurveto{\pgfqpoint{1.692102in}{1.898046in}}{\pgfqpoint{1.700002in}{1.901319in}}{\pgfqpoint{1.705826in}{1.907143in}}%
\pgfpathcurveto{\pgfqpoint{1.711650in}{1.912967in}}{\pgfqpoint{1.714922in}{1.920867in}}{\pgfqpoint{1.714922in}{1.929103in}}%
\pgfpathcurveto{\pgfqpoint{1.714922in}{1.937339in}}{\pgfqpoint{1.711650in}{1.945239in}}{\pgfqpoint{1.705826in}{1.951063in}}%
\pgfpathcurveto{\pgfqpoint{1.700002in}{1.956887in}}{\pgfqpoint{1.692102in}{1.960159in}}{\pgfqpoint{1.683866in}{1.960159in}}%
\pgfpathcurveto{\pgfqpoint{1.675629in}{1.960159in}}{\pgfqpoint{1.667729in}{1.956887in}}{\pgfqpoint{1.661905in}{1.951063in}}%
\pgfpathcurveto{\pgfqpoint{1.656081in}{1.945239in}}{\pgfqpoint{1.652809in}{1.937339in}}{\pgfqpoint{1.652809in}{1.929103in}}%
\pgfpathcurveto{\pgfqpoint{1.652809in}{1.920867in}}{\pgfqpoint{1.656081in}{1.912967in}}{\pgfqpoint{1.661905in}{1.907143in}}%
\pgfpathcurveto{\pgfqpoint{1.667729in}{1.901319in}}{\pgfqpoint{1.675629in}{1.898046in}}{\pgfqpoint{1.683866in}{1.898046in}}%
\pgfpathclose%
\pgfusepath{stroke,fill}%
\end{pgfscope}%
\begin{pgfscope}%
\pgfpathrectangle{\pgfqpoint{0.100000in}{0.212622in}}{\pgfqpoint{3.696000in}{3.696000in}}%
\pgfusepath{clip}%
\pgfsetbuttcap%
\pgfsetroundjoin%
\definecolor{currentfill}{rgb}{0.121569,0.466667,0.705882}%
\pgfsetfillcolor{currentfill}%
\pgfsetfillopacity{0.536688}%
\pgfsetlinewidth{1.003750pt}%
\definecolor{currentstroke}{rgb}{0.121569,0.466667,0.705882}%
\pgfsetstrokecolor{currentstroke}%
\pgfsetstrokeopacity{0.536688}%
\pgfsetdash{}{0pt}%
\pgfpathmoveto{\pgfqpoint{1.708853in}{1.918194in}}%
\pgfpathcurveto{\pgfqpoint{1.717089in}{1.918194in}}{\pgfqpoint{1.724989in}{1.921466in}}{\pgfqpoint{1.730813in}{1.927290in}}%
\pgfpathcurveto{\pgfqpoint{1.736637in}{1.933114in}}{\pgfqpoint{1.739909in}{1.941014in}}{\pgfqpoint{1.739909in}{1.949250in}}%
\pgfpathcurveto{\pgfqpoint{1.739909in}{1.957487in}}{\pgfqpoint{1.736637in}{1.965387in}}{\pgfqpoint{1.730813in}{1.971211in}}%
\pgfpathcurveto{\pgfqpoint{1.724989in}{1.977035in}}{\pgfqpoint{1.717089in}{1.980307in}}{\pgfqpoint{1.708853in}{1.980307in}}%
\pgfpathcurveto{\pgfqpoint{1.700617in}{1.980307in}}{\pgfqpoint{1.692717in}{1.977035in}}{\pgfqpoint{1.686893in}{1.971211in}}%
\pgfpathcurveto{\pgfqpoint{1.681069in}{1.965387in}}{\pgfqpoint{1.677796in}{1.957487in}}{\pgfqpoint{1.677796in}{1.949250in}}%
\pgfpathcurveto{\pgfqpoint{1.677796in}{1.941014in}}{\pgfqpoint{1.681069in}{1.933114in}}{\pgfqpoint{1.686893in}{1.927290in}}%
\pgfpathcurveto{\pgfqpoint{1.692717in}{1.921466in}}{\pgfqpoint{1.700617in}{1.918194in}}{\pgfqpoint{1.708853in}{1.918194in}}%
\pgfpathclose%
\pgfusepath{stroke,fill}%
\end{pgfscope}%
\begin{pgfscope}%
\pgfpathrectangle{\pgfqpoint{0.100000in}{0.212622in}}{\pgfqpoint{3.696000in}{3.696000in}}%
\pgfusepath{clip}%
\pgfsetbuttcap%
\pgfsetroundjoin%
\definecolor{currentfill}{rgb}{0.121569,0.466667,0.705882}%
\pgfsetfillcolor{currentfill}%
\pgfsetfillopacity{0.537000}%
\pgfsetlinewidth{1.003750pt}%
\definecolor{currentstroke}{rgb}{0.121569,0.466667,0.705882}%
\pgfsetstrokecolor{currentstroke}%
\pgfsetstrokeopacity{0.537000}%
\pgfsetdash{}{0pt}%
\pgfpathmoveto{\pgfqpoint{1.687765in}{1.897577in}}%
\pgfpathcurveto{\pgfqpoint{1.696001in}{1.897577in}}{\pgfqpoint{1.703901in}{1.900849in}}{\pgfqpoint{1.709725in}{1.906673in}}%
\pgfpathcurveto{\pgfqpoint{1.715549in}{1.912497in}}{\pgfqpoint{1.718821in}{1.920397in}}{\pgfqpoint{1.718821in}{1.928633in}}%
\pgfpathcurveto{\pgfqpoint{1.718821in}{1.936870in}}{\pgfqpoint{1.715549in}{1.944770in}}{\pgfqpoint{1.709725in}{1.950594in}}%
\pgfpathcurveto{\pgfqpoint{1.703901in}{1.956417in}}{\pgfqpoint{1.696001in}{1.959690in}}{\pgfqpoint{1.687765in}{1.959690in}}%
\pgfpathcurveto{\pgfqpoint{1.679528in}{1.959690in}}{\pgfqpoint{1.671628in}{1.956417in}}{\pgfqpoint{1.665805in}{1.950594in}}%
\pgfpathcurveto{\pgfqpoint{1.659981in}{1.944770in}}{\pgfqpoint{1.656708in}{1.936870in}}{\pgfqpoint{1.656708in}{1.928633in}}%
\pgfpathcurveto{\pgfqpoint{1.656708in}{1.920397in}}{\pgfqpoint{1.659981in}{1.912497in}}{\pgfqpoint{1.665805in}{1.906673in}}%
\pgfpathcurveto{\pgfqpoint{1.671628in}{1.900849in}}{\pgfqpoint{1.679528in}{1.897577in}}{\pgfqpoint{1.687765in}{1.897577in}}%
\pgfpathclose%
\pgfusepath{stroke,fill}%
\end{pgfscope}%
\begin{pgfscope}%
\pgfpathrectangle{\pgfqpoint{0.100000in}{0.212622in}}{\pgfqpoint{3.696000in}{3.696000in}}%
\pgfusepath{clip}%
\pgfsetbuttcap%
\pgfsetroundjoin%
\definecolor{currentfill}{rgb}{0.121569,0.466667,0.705882}%
\pgfsetfillcolor{currentfill}%
\pgfsetfillopacity{0.537271}%
\pgfsetlinewidth{1.003750pt}%
\definecolor{currentstroke}{rgb}{0.121569,0.466667,0.705882}%
\pgfsetstrokecolor{currentstroke}%
\pgfsetstrokeopacity{0.537271}%
\pgfsetdash{}{0pt}%
\pgfpathmoveto{\pgfqpoint{1.699716in}{1.909589in}}%
\pgfpathcurveto{\pgfqpoint{1.707953in}{1.909589in}}{\pgfqpoint{1.715853in}{1.912861in}}{\pgfqpoint{1.721677in}{1.918685in}}%
\pgfpathcurveto{\pgfqpoint{1.727501in}{1.924509in}}{\pgfqpoint{1.730773in}{1.932409in}}{\pgfqpoint{1.730773in}{1.940645in}}%
\pgfpathcurveto{\pgfqpoint{1.730773in}{1.948882in}}{\pgfqpoint{1.727501in}{1.956782in}}{\pgfqpoint{1.721677in}{1.962606in}}%
\pgfpathcurveto{\pgfqpoint{1.715853in}{1.968430in}}{\pgfqpoint{1.707953in}{1.971702in}}{\pgfqpoint{1.699716in}{1.971702in}}%
\pgfpathcurveto{\pgfqpoint{1.691480in}{1.971702in}}{\pgfqpoint{1.683580in}{1.968430in}}{\pgfqpoint{1.677756in}{1.962606in}}%
\pgfpathcurveto{\pgfqpoint{1.671932in}{1.956782in}}{\pgfqpoint{1.668660in}{1.948882in}}{\pgfqpoint{1.668660in}{1.940645in}}%
\pgfpathcurveto{\pgfqpoint{1.668660in}{1.932409in}}{\pgfqpoint{1.671932in}{1.924509in}}{\pgfqpoint{1.677756in}{1.918685in}}%
\pgfpathcurveto{\pgfqpoint{1.683580in}{1.912861in}}{\pgfqpoint{1.691480in}{1.909589in}}{\pgfqpoint{1.699716in}{1.909589in}}%
\pgfpathclose%
\pgfusepath{stroke,fill}%
\end{pgfscope}%
\begin{pgfscope}%
\pgfpathrectangle{\pgfqpoint{0.100000in}{0.212622in}}{\pgfqpoint{3.696000in}{3.696000in}}%
\pgfusepath{clip}%
\pgfsetbuttcap%
\pgfsetroundjoin%
\definecolor{currentfill}{rgb}{0.121569,0.466667,0.705882}%
\pgfsetfillcolor{currentfill}%
\pgfsetfillopacity{0.537327}%
\pgfsetlinewidth{1.003750pt}%
\definecolor{currentstroke}{rgb}{0.121569,0.466667,0.705882}%
\pgfsetstrokecolor{currentstroke}%
\pgfsetstrokeopacity{0.537327}%
\pgfsetdash{}{0pt}%
\pgfpathmoveto{\pgfqpoint{1.724914in}{1.928752in}}%
\pgfpathcurveto{\pgfqpoint{1.733151in}{1.928752in}}{\pgfqpoint{1.741051in}{1.932024in}}{\pgfqpoint{1.746875in}{1.937848in}}%
\pgfpathcurveto{\pgfqpoint{1.752698in}{1.943672in}}{\pgfqpoint{1.755971in}{1.951572in}}{\pgfqpoint{1.755971in}{1.959809in}}%
\pgfpathcurveto{\pgfqpoint{1.755971in}{1.968045in}}{\pgfqpoint{1.752698in}{1.975945in}}{\pgfqpoint{1.746875in}{1.981769in}}%
\pgfpathcurveto{\pgfqpoint{1.741051in}{1.987593in}}{\pgfqpoint{1.733151in}{1.990865in}}{\pgfqpoint{1.724914in}{1.990865in}}%
\pgfpathcurveto{\pgfqpoint{1.716678in}{1.990865in}}{\pgfqpoint{1.708778in}{1.987593in}}{\pgfqpoint{1.702954in}{1.981769in}}%
\pgfpathcurveto{\pgfqpoint{1.697130in}{1.975945in}}{\pgfqpoint{1.693858in}{1.968045in}}{\pgfqpoint{1.693858in}{1.959809in}}%
\pgfpathcurveto{\pgfqpoint{1.693858in}{1.951572in}}{\pgfqpoint{1.697130in}{1.943672in}}{\pgfqpoint{1.702954in}{1.937848in}}%
\pgfpathcurveto{\pgfqpoint{1.708778in}{1.932024in}}{\pgfqpoint{1.716678in}{1.928752in}}{\pgfqpoint{1.724914in}{1.928752in}}%
\pgfpathclose%
\pgfusepath{stroke,fill}%
\end{pgfscope}%
\begin{pgfscope}%
\pgfpathrectangle{\pgfqpoint{0.100000in}{0.212622in}}{\pgfqpoint{3.696000in}{3.696000in}}%
\pgfusepath{clip}%
\pgfsetbuttcap%
\pgfsetroundjoin%
\definecolor{currentfill}{rgb}{0.121569,0.466667,0.705882}%
\pgfsetfillcolor{currentfill}%
\pgfsetfillopacity{0.537647}%
\pgfsetlinewidth{1.003750pt}%
\definecolor{currentstroke}{rgb}{0.121569,0.466667,0.705882}%
\pgfsetstrokecolor{currentstroke}%
\pgfsetstrokeopacity{0.537647}%
\pgfsetdash{}{0pt}%
\pgfpathmoveto{\pgfqpoint{1.713313in}{1.923140in}}%
\pgfpathcurveto{\pgfqpoint{1.721549in}{1.923140in}}{\pgfqpoint{1.729449in}{1.926412in}}{\pgfqpoint{1.735273in}{1.932236in}}%
\pgfpathcurveto{\pgfqpoint{1.741097in}{1.938060in}}{\pgfqpoint{1.744369in}{1.945960in}}{\pgfqpoint{1.744369in}{1.954196in}}%
\pgfpathcurveto{\pgfqpoint{1.744369in}{1.962433in}}{\pgfqpoint{1.741097in}{1.970333in}}{\pgfqpoint{1.735273in}{1.976157in}}%
\pgfpathcurveto{\pgfqpoint{1.729449in}{1.981981in}}{\pgfqpoint{1.721549in}{1.985253in}}{\pgfqpoint{1.713313in}{1.985253in}}%
\pgfpathcurveto{\pgfqpoint{1.705076in}{1.985253in}}{\pgfqpoint{1.697176in}{1.981981in}}{\pgfqpoint{1.691353in}{1.976157in}}%
\pgfpathcurveto{\pgfqpoint{1.685529in}{1.970333in}}{\pgfqpoint{1.682256in}{1.962433in}}{\pgfqpoint{1.682256in}{1.954196in}}%
\pgfpathcurveto{\pgfqpoint{1.682256in}{1.945960in}}{\pgfqpoint{1.685529in}{1.938060in}}{\pgfqpoint{1.691353in}{1.932236in}}%
\pgfpathcurveto{\pgfqpoint{1.697176in}{1.926412in}}{\pgfqpoint{1.705076in}{1.923140in}}{\pgfqpoint{1.713313in}{1.923140in}}%
\pgfpathclose%
\pgfusepath{stroke,fill}%
\end{pgfscope}%
\begin{pgfscope}%
\pgfpathrectangle{\pgfqpoint{0.100000in}{0.212622in}}{\pgfqpoint{3.696000in}{3.696000in}}%
\pgfusepath{clip}%
\pgfsetbuttcap%
\pgfsetroundjoin%
\definecolor{currentfill}{rgb}{0.121569,0.466667,0.705882}%
\pgfsetfillcolor{currentfill}%
\pgfsetfillopacity{0.538968}%
\pgfsetlinewidth{1.003750pt}%
\definecolor{currentstroke}{rgb}{0.121569,0.466667,0.705882}%
\pgfsetstrokecolor{currentstroke}%
\pgfsetstrokeopacity{0.538968}%
\pgfsetdash{}{0pt}%
\pgfpathmoveto{\pgfqpoint{2.695467in}{2.534979in}}%
\pgfpathcurveto{\pgfqpoint{2.703704in}{2.534979in}}{\pgfqpoint{2.711604in}{2.538251in}}{\pgfqpoint{2.717428in}{2.544075in}}%
\pgfpathcurveto{\pgfqpoint{2.723252in}{2.549899in}}{\pgfqpoint{2.726524in}{2.557799in}}{\pgfqpoint{2.726524in}{2.566035in}}%
\pgfpathcurveto{\pgfqpoint{2.726524in}{2.574272in}}{\pgfqpoint{2.723252in}{2.582172in}}{\pgfqpoint{2.717428in}{2.587995in}}%
\pgfpathcurveto{\pgfqpoint{2.711604in}{2.593819in}}{\pgfqpoint{2.703704in}{2.597092in}}{\pgfqpoint{2.695467in}{2.597092in}}%
\pgfpathcurveto{\pgfqpoint{2.687231in}{2.597092in}}{\pgfqpoint{2.679331in}{2.593819in}}{\pgfqpoint{2.673507in}{2.587995in}}%
\pgfpathcurveto{\pgfqpoint{2.667683in}{2.582172in}}{\pgfqpoint{2.664411in}{2.574272in}}{\pgfqpoint{2.664411in}{2.566035in}}%
\pgfpathcurveto{\pgfqpoint{2.664411in}{2.557799in}}{\pgfqpoint{2.667683in}{2.549899in}}{\pgfqpoint{2.673507in}{2.544075in}}%
\pgfpathcurveto{\pgfqpoint{2.679331in}{2.538251in}}{\pgfqpoint{2.687231in}{2.534979in}}{\pgfqpoint{2.695467in}{2.534979in}}%
\pgfpathclose%
\pgfusepath{stroke,fill}%
\end{pgfscope}%
\begin{pgfscope}%
\pgfpathrectangle{\pgfqpoint{0.100000in}{0.212622in}}{\pgfqpoint{3.696000in}{3.696000in}}%
\pgfusepath{clip}%
\pgfsetbuttcap%
\pgfsetroundjoin%
\definecolor{currentfill}{rgb}{0.121569,0.466667,0.705882}%
\pgfsetfillcolor{currentfill}%
\pgfsetfillopacity{0.540765}%
\pgfsetlinewidth{1.003750pt}%
\definecolor{currentstroke}{rgb}{0.121569,0.466667,0.705882}%
\pgfsetstrokecolor{currentstroke}%
\pgfsetstrokeopacity{0.540765}%
\pgfsetdash{}{0pt}%
\pgfpathmoveto{\pgfqpoint{2.792712in}{2.601999in}}%
\pgfpathcurveto{\pgfqpoint{2.800948in}{2.601999in}}{\pgfqpoint{2.808848in}{2.605271in}}{\pgfqpoint{2.814672in}{2.611095in}}%
\pgfpathcurveto{\pgfqpoint{2.820496in}{2.616919in}}{\pgfqpoint{2.823769in}{2.624819in}}{\pgfqpoint{2.823769in}{2.633056in}}%
\pgfpathcurveto{\pgfqpoint{2.823769in}{2.641292in}}{\pgfqpoint{2.820496in}{2.649192in}}{\pgfqpoint{2.814672in}{2.655016in}}%
\pgfpathcurveto{\pgfqpoint{2.808848in}{2.660840in}}{\pgfqpoint{2.800948in}{2.664112in}}{\pgfqpoint{2.792712in}{2.664112in}}%
\pgfpathcurveto{\pgfqpoint{2.784476in}{2.664112in}}{\pgfqpoint{2.776576in}{2.660840in}}{\pgfqpoint{2.770752in}{2.655016in}}%
\pgfpathcurveto{\pgfqpoint{2.764928in}{2.649192in}}{\pgfqpoint{2.761656in}{2.641292in}}{\pgfqpoint{2.761656in}{2.633056in}}%
\pgfpathcurveto{\pgfqpoint{2.761656in}{2.624819in}}{\pgfqpoint{2.764928in}{2.616919in}}{\pgfqpoint{2.770752in}{2.611095in}}%
\pgfpathcurveto{\pgfqpoint{2.776576in}{2.605271in}}{\pgfqpoint{2.784476in}{2.601999in}}{\pgfqpoint{2.792712in}{2.601999in}}%
\pgfpathclose%
\pgfusepath{stroke,fill}%
\end{pgfscope}%
\begin{pgfscope}%
\pgfpathrectangle{\pgfqpoint{0.100000in}{0.212622in}}{\pgfqpoint{3.696000in}{3.696000in}}%
\pgfusepath{clip}%
\pgfsetbuttcap%
\pgfsetroundjoin%
\definecolor{currentfill}{rgb}{0.121569,0.466667,0.705882}%
\pgfsetfillcolor{currentfill}%
\pgfsetfillopacity{0.543329}%
\pgfsetlinewidth{1.003750pt}%
\definecolor{currentstroke}{rgb}{0.121569,0.466667,0.705882}%
\pgfsetstrokecolor{currentstroke}%
\pgfsetstrokeopacity{0.543329}%
\pgfsetdash{}{0pt}%
\pgfpathmoveto{\pgfqpoint{2.639137in}{2.494037in}}%
\pgfpathcurveto{\pgfqpoint{2.647374in}{2.494037in}}{\pgfqpoint{2.655274in}{2.497310in}}{\pgfqpoint{2.661098in}{2.503134in}}%
\pgfpathcurveto{\pgfqpoint{2.666921in}{2.508958in}}{\pgfqpoint{2.670194in}{2.516858in}}{\pgfqpoint{2.670194in}{2.525094in}}%
\pgfpathcurveto{\pgfqpoint{2.670194in}{2.533330in}}{\pgfqpoint{2.666921in}{2.541230in}}{\pgfqpoint{2.661098in}{2.547054in}}%
\pgfpathcurveto{\pgfqpoint{2.655274in}{2.552878in}}{\pgfqpoint{2.647374in}{2.556150in}}{\pgfqpoint{2.639137in}{2.556150in}}%
\pgfpathcurveto{\pgfqpoint{2.630901in}{2.556150in}}{\pgfqpoint{2.623001in}{2.552878in}}{\pgfqpoint{2.617177in}{2.547054in}}%
\pgfpathcurveto{\pgfqpoint{2.611353in}{2.541230in}}{\pgfqpoint{2.608081in}{2.533330in}}{\pgfqpoint{2.608081in}{2.525094in}}%
\pgfpathcurveto{\pgfqpoint{2.608081in}{2.516858in}}{\pgfqpoint{2.611353in}{2.508958in}}{\pgfqpoint{2.617177in}{2.503134in}}%
\pgfpathcurveto{\pgfqpoint{2.623001in}{2.497310in}}{\pgfqpoint{2.630901in}{2.494037in}}{\pgfqpoint{2.639137in}{2.494037in}}%
\pgfpathclose%
\pgfusepath{stroke,fill}%
\end{pgfscope}%
\begin{pgfscope}%
\pgfpathrectangle{\pgfqpoint{0.100000in}{0.212622in}}{\pgfqpoint{3.696000in}{3.696000in}}%
\pgfusepath{clip}%
\pgfsetbuttcap%
\pgfsetroundjoin%
\definecolor{currentfill}{rgb}{0.121569,0.466667,0.705882}%
\pgfsetfillcolor{currentfill}%
\pgfsetfillopacity{0.545291}%
\pgfsetlinewidth{1.003750pt}%
\definecolor{currentstroke}{rgb}{0.121569,0.466667,0.705882}%
\pgfsetstrokecolor{currentstroke}%
\pgfsetstrokeopacity{0.545291}%
\pgfsetdash{}{0pt}%
\pgfpathmoveto{\pgfqpoint{2.809991in}{2.615317in}}%
\pgfpathcurveto{\pgfqpoint{2.818228in}{2.615317in}}{\pgfqpoint{2.826128in}{2.618589in}}{\pgfqpoint{2.831952in}{2.624413in}}%
\pgfpathcurveto{\pgfqpoint{2.837775in}{2.630237in}}{\pgfqpoint{2.841048in}{2.638137in}}{\pgfqpoint{2.841048in}{2.646373in}}%
\pgfpathcurveto{\pgfqpoint{2.841048in}{2.654609in}}{\pgfqpoint{2.837775in}{2.662510in}}{\pgfqpoint{2.831952in}{2.668333in}}%
\pgfpathcurveto{\pgfqpoint{2.826128in}{2.674157in}}{\pgfqpoint{2.818228in}{2.677430in}}{\pgfqpoint{2.809991in}{2.677430in}}%
\pgfpathcurveto{\pgfqpoint{2.801755in}{2.677430in}}{\pgfqpoint{2.793855in}{2.674157in}}{\pgfqpoint{2.788031in}{2.668333in}}%
\pgfpathcurveto{\pgfqpoint{2.782207in}{2.662510in}}{\pgfqpoint{2.778935in}{2.654609in}}{\pgfqpoint{2.778935in}{2.646373in}}%
\pgfpathcurveto{\pgfqpoint{2.778935in}{2.638137in}}{\pgfqpoint{2.782207in}{2.630237in}}{\pgfqpoint{2.788031in}{2.624413in}}%
\pgfpathcurveto{\pgfqpoint{2.793855in}{2.618589in}}{\pgfqpoint{2.801755in}{2.615317in}}{\pgfqpoint{2.809991in}{2.615317in}}%
\pgfpathclose%
\pgfusepath{stroke,fill}%
\end{pgfscope}%
\begin{pgfscope}%
\pgfpathrectangle{\pgfqpoint{0.100000in}{0.212622in}}{\pgfqpoint{3.696000in}{3.696000in}}%
\pgfusepath{clip}%
\pgfsetbuttcap%
\pgfsetroundjoin%
\definecolor{currentfill}{rgb}{0.121569,0.466667,0.705882}%
\pgfsetfillcolor{currentfill}%
\pgfsetfillopacity{0.558438}%
\pgfsetlinewidth{1.003750pt}%
\definecolor{currentstroke}{rgb}{0.121569,0.466667,0.705882}%
\pgfsetstrokecolor{currentstroke}%
\pgfsetstrokeopacity{0.558438}%
\pgfsetdash{}{0pt}%
\pgfpathmoveto{\pgfqpoint{2.835956in}{2.634198in}}%
\pgfpathcurveto{\pgfqpoint{2.844192in}{2.634198in}}{\pgfqpoint{2.852092in}{2.637471in}}{\pgfqpoint{2.857916in}{2.643295in}}%
\pgfpathcurveto{\pgfqpoint{2.863740in}{2.649119in}}{\pgfqpoint{2.867012in}{2.657019in}}{\pgfqpoint{2.867012in}{2.665255in}}%
\pgfpathcurveto{\pgfqpoint{2.867012in}{2.673491in}}{\pgfqpoint{2.863740in}{2.681391in}}{\pgfqpoint{2.857916in}{2.687215in}}%
\pgfpathcurveto{\pgfqpoint{2.852092in}{2.693039in}}{\pgfqpoint{2.844192in}{2.696311in}}{\pgfqpoint{2.835956in}{2.696311in}}%
\pgfpathcurveto{\pgfqpoint{2.827720in}{2.696311in}}{\pgfqpoint{2.819820in}{2.693039in}}{\pgfqpoint{2.813996in}{2.687215in}}%
\pgfpathcurveto{\pgfqpoint{2.808172in}{2.681391in}}{\pgfqpoint{2.804899in}{2.673491in}}{\pgfqpoint{2.804899in}{2.665255in}}%
\pgfpathcurveto{\pgfqpoint{2.804899in}{2.657019in}}{\pgfqpoint{2.808172in}{2.649119in}}{\pgfqpoint{2.813996in}{2.643295in}}%
\pgfpathcurveto{\pgfqpoint{2.819820in}{2.637471in}}{\pgfqpoint{2.827720in}{2.634198in}}{\pgfqpoint{2.835956in}{2.634198in}}%
\pgfpathclose%
\pgfusepath{stroke,fill}%
\end{pgfscope}%
\begin{pgfscope}%
\pgfpathrectangle{\pgfqpoint{0.100000in}{0.212622in}}{\pgfqpoint{3.696000in}{3.696000in}}%
\pgfusepath{clip}%
\pgfsetbuttcap%
\pgfsetroundjoin%
\definecolor{currentfill}{rgb}{0.121569,0.466667,0.705882}%
\pgfsetfillcolor{currentfill}%
\pgfsetfillopacity{0.559217}%
\pgfsetlinewidth{1.003750pt}%
\definecolor{currentstroke}{rgb}{0.121569,0.466667,0.705882}%
\pgfsetstrokecolor{currentstroke}%
\pgfsetstrokeopacity{0.559217}%
\pgfsetdash{}{0pt}%
\pgfpathmoveto{\pgfqpoint{2.800185in}{2.596960in}}%
\pgfpathcurveto{\pgfqpoint{2.808421in}{2.596960in}}{\pgfqpoint{2.816321in}{2.600232in}}{\pgfqpoint{2.822145in}{2.606056in}}%
\pgfpathcurveto{\pgfqpoint{2.827969in}{2.611880in}}{\pgfqpoint{2.831241in}{2.619780in}}{\pgfqpoint{2.831241in}{2.628016in}}%
\pgfpathcurveto{\pgfqpoint{2.831241in}{2.636252in}}{\pgfqpoint{2.827969in}{2.644152in}}{\pgfqpoint{2.822145in}{2.649976in}}%
\pgfpathcurveto{\pgfqpoint{2.816321in}{2.655800in}}{\pgfqpoint{2.808421in}{2.659073in}}{\pgfqpoint{2.800185in}{2.659073in}}%
\pgfpathcurveto{\pgfqpoint{2.791949in}{2.659073in}}{\pgfqpoint{2.784048in}{2.655800in}}{\pgfqpoint{2.778225in}{2.649976in}}%
\pgfpathcurveto{\pgfqpoint{2.772401in}{2.644152in}}{\pgfqpoint{2.769128in}{2.636252in}}{\pgfqpoint{2.769128in}{2.628016in}}%
\pgfpathcurveto{\pgfqpoint{2.769128in}{2.619780in}}{\pgfqpoint{2.772401in}{2.611880in}}{\pgfqpoint{2.778225in}{2.606056in}}%
\pgfpathcurveto{\pgfqpoint{2.784048in}{2.600232in}}{\pgfqpoint{2.791949in}{2.596960in}}{\pgfqpoint{2.800185in}{2.596960in}}%
\pgfpathclose%
\pgfusepath{stroke,fill}%
\end{pgfscope}%
\begin{pgfscope}%
\pgfpathrectangle{\pgfqpoint{0.100000in}{0.212622in}}{\pgfqpoint{3.696000in}{3.696000in}}%
\pgfusepath{clip}%
\pgfsetbuttcap%
\pgfsetroundjoin%
\definecolor{currentfill}{rgb}{0.121569,0.466667,0.705882}%
\pgfsetfillcolor{currentfill}%
\pgfsetfillopacity{0.565637}%
\pgfsetlinewidth{1.003750pt}%
\definecolor{currentstroke}{rgb}{0.121569,0.466667,0.705882}%
\pgfsetstrokecolor{currentstroke}%
\pgfsetstrokeopacity{0.565637}%
\pgfsetdash{}{0pt}%
\pgfpathmoveto{\pgfqpoint{2.830694in}{2.620222in}}%
\pgfpathcurveto{\pgfqpoint{2.838930in}{2.620222in}}{\pgfqpoint{2.846830in}{2.623494in}}{\pgfqpoint{2.852654in}{2.629318in}}%
\pgfpathcurveto{\pgfqpoint{2.858478in}{2.635142in}}{\pgfqpoint{2.861750in}{2.643042in}}{\pgfqpoint{2.861750in}{2.651278in}}%
\pgfpathcurveto{\pgfqpoint{2.861750in}{2.659514in}}{\pgfqpoint{2.858478in}{2.667414in}}{\pgfqpoint{2.852654in}{2.673238in}}%
\pgfpathcurveto{\pgfqpoint{2.846830in}{2.679062in}}{\pgfqpoint{2.838930in}{2.682335in}}{\pgfqpoint{2.830694in}{2.682335in}}%
\pgfpathcurveto{\pgfqpoint{2.822458in}{2.682335in}}{\pgfqpoint{2.814557in}{2.679062in}}{\pgfqpoint{2.808734in}{2.673238in}}%
\pgfpathcurveto{\pgfqpoint{2.802910in}{2.667414in}}{\pgfqpoint{2.799637in}{2.659514in}}{\pgfqpoint{2.799637in}{2.651278in}}%
\pgfpathcurveto{\pgfqpoint{2.799637in}{2.643042in}}{\pgfqpoint{2.802910in}{2.635142in}}{\pgfqpoint{2.808734in}{2.629318in}}%
\pgfpathcurveto{\pgfqpoint{2.814557in}{2.623494in}}{\pgfqpoint{2.822458in}{2.620222in}}{\pgfqpoint{2.830694in}{2.620222in}}%
\pgfpathclose%
\pgfusepath{stroke,fill}%
\end{pgfscope}%
\begin{pgfscope}%
\pgfpathrectangle{\pgfqpoint{0.100000in}{0.212622in}}{\pgfqpoint{3.696000in}{3.696000in}}%
\pgfusepath{clip}%
\pgfsetbuttcap%
\pgfsetroundjoin%
\definecolor{currentfill}{rgb}{0.121569,0.466667,0.705882}%
\pgfsetfillcolor{currentfill}%
\pgfsetfillopacity{0.565933}%
\pgfsetlinewidth{1.003750pt}%
\definecolor{currentstroke}{rgb}{0.121569,0.466667,0.705882}%
\pgfsetstrokecolor{currentstroke}%
\pgfsetstrokeopacity{0.565933}%
\pgfsetdash{}{0pt}%
\pgfpathmoveto{\pgfqpoint{2.847908in}{2.635751in}}%
\pgfpathcurveto{\pgfqpoint{2.856144in}{2.635751in}}{\pgfqpoint{2.864044in}{2.639023in}}{\pgfqpoint{2.869868in}{2.644847in}}%
\pgfpathcurveto{\pgfqpoint{2.875692in}{2.650671in}}{\pgfqpoint{2.878964in}{2.658571in}}{\pgfqpoint{2.878964in}{2.666807in}}%
\pgfpathcurveto{\pgfqpoint{2.878964in}{2.675043in}}{\pgfqpoint{2.875692in}{2.682944in}}{\pgfqpoint{2.869868in}{2.688767in}}%
\pgfpathcurveto{\pgfqpoint{2.864044in}{2.694591in}}{\pgfqpoint{2.856144in}{2.697864in}}{\pgfqpoint{2.847908in}{2.697864in}}%
\pgfpathcurveto{\pgfqpoint{2.839671in}{2.697864in}}{\pgfqpoint{2.831771in}{2.694591in}}{\pgfqpoint{2.825947in}{2.688767in}}%
\pgfpathcurveto{\pgfqpoint{2.820124in}{2.682944in}}{\pgfqpoint{2.816851in}{2.675043in}}{\pgfqpoint{2.816851in}{2.666807in}}%
\pgfpathcurveto{\pgfqpoint{2.816851in}{2.658571in}}{\pgfqpoint{2.820124in}{2.650671in}}{\pgfqpoint{2.825947in}{2.644847in}}%
\pgfpathcurveto{\pgfqpoint{2.831771in}{2.639023in}}{\pgfqpoint{2.839671in}{2.635751in}}{\pgfqpoint{2.847908in}{2.635751in}}%
\pgfpathclose%
\pgfusepath{stroke,fill}%
\end{pgfscope}%
\begin{pgfscope}%
\pgfpathrectangle{\pgfqpoint{0.100000in}{0.212622in}}{\pgfqpoint{3.696000in}{3.696000in}}%
\pgfusepath{clip}%
\pgfsetbuttcap%
\pgfsetroundjoin%
\definecolor{currentfill}{rgb}{0.121569,0.466667,0.705882}%
\pgfsetfillcolor{currentfill}%
\pgfsetfillopacity{0.567903}%
\pgfsetlinewidth{1.003750pt}%
\definecolor{currentstroke}{rgb}{0.121569,0.466667,0.705882}%
\pgfsetstrokecolor{currentstroke}%
\pgfsetstrokeopacity{0.567903}%
\pgfsetdash{}{0pt}%
\pgfpathmoveto{\pgfqpoint{2.851833in}{2.637349in}}%
\pgfpathcurveto{\pgfqpoint{2.860069in}{2.637349in}}{\pgfqpoint{2.867969in}{2.640621in}}{\pgfqpoint{2.873793in}{2.646445in}}%
\pgfpathcurveto{\pgfqpoint{2.879617in}{2.652269in}}{\pgfqpoint{2.882889in}{2.660169in}}{\pgfqpoint{2.882889in}{2.668406in}}%
\pgfpathcurveto{\pgfqpoint{2.882889in}{2.676642in}}{\pgfqpoint{2.879617in}{2.684542in}}{\pgfqpoint{2.873793in}{2.690366in}}%
\pgfpathcurveto{\pgfqpoint{2.867969in}{2.696190in}}{\pgfqpoint{2.860069in}{2.699462in}}{\pgfqpoint{2.851833in}{2.699462in}}%
\pgfpathcurveto{\pgfqpoint{2.843596in}{2.699462in}}{\pgfqpoint{2.835696in}{2.696190in}}{\pgfqpoint{2.829872in}{2.690366in}}%
\pgfpathcurveto{\pgfqpoint{2.824048in}{2.684542in}}{\pgfqpoint{2.820776in}{2.676642in}}{\pgfqpoint{2.820776in}{2.668406in}}%
\pgfpathcurveto{\pgfqpoint{2.820776in}{2.660169in}}{\pgfqpoint{2.824048in}{2.652269in}}{\pgfqpoint{2.829872in}{2.646445in}}%
\pgfpathcurveto{\pgfqpoint{2.835696in}{2.640621in}}{\pgfqpoint{2.843596in}{2.637349in}}{\pgfqpoint{2.851833in}{2.637349in}}%
\pgfpathclose%
\pgfusepath{stroke,fill}%
\end{pgfscope}%
\begin{pgfscope}%
\pgfpathrectangle{\pgfqpoint{0.100000in}{0.212622in}}{\pgfqpoint{3.696000in}{3.696000in}}%
\pgfusepath{clip}%
\pgfsetbuttcap%
\pgfsetroundjoin%
\definecolor{currentfill}{rgb}{0.121569,0.466667,0.705882}%
\pgfsetfillcolor{currentfill}%
\pgfsetfillopacity{0.571926}%
\pgfsetlinewidth{1.003750pt}%
\definecolor{currentstroke}{rgb}{0.121569,0.466667,0.705882}%
\pgfsetstrokecolor{currentstroke}%
\pgfsetstrokeopacity{0.571926}%
\pgfsetdash{}{0pt}%
\pgfpathmoveto{\pgfqpoint{2.846451in}{2.632076in}}%
\pgfpathcurveto{\pgfqpoint{2.854687in}{2.632076in}}{\pgfqpoint{2.862587in}{2.635348in}}{\pgfqpoint{2.868411in}{2.641172in}}%
\pgfpathcurveto{\pgfqpoint{2.874235in}{2.646996in}}{\pgfqpoint{2.877507in}{2.654896in}}{\pgfqpoint{2.877507in}{2.663132in}}%
\pgfpathcurveto{\pgfqpoint{2.877507in}{2.671369in}}{\pgfqpoint{2.874235in}{2.679269in}}{\pgfqpoint{2.868411in}{2.685093in}}%
\pgfpathcurveto{\pgfqpoint{2.862587in}{2.690916in}}{\pgfqpoint{2.854687in}{2.694189in}}{\pgfqpoint{2.846451in}{2.694189in}}%
\pgfpathcurveto{\pgfqpoint{2.838214in}{2.694189in}}{\pgfqpoint{2.830314in}{2.690916in}}{\pgfqpoint{2.824490in}{2.685093in}}%
\pgfpathcurveto{\pgfqpoint{2.818667in}{2.679269in}}{\pgfqpoint{2.815394in}{2.671369in}}{\pgfqpoint{2.815394in}{2.663132in}}%
\pgfpathcurveto{\pgfqpoint{2.815394in}{2.654896in}}{\pgfqpoint{2.818667in}{2.646996in}}{\pgfqpoint{2.824490in}{2.641172in}}%
\pgfpathcurveto{\pgfqpoint{2.830314in}{2.635348in}}{\pgfqpoint{2.838214in}{2.632076in}}{\pgfqpoint{2.846451in}{2.632076in}}%
\pgfpathclose%
\pgfusepath{stroke,fill}%
\end{pgfscope}%
\begin{pgfscope}%
\pgfpathrectangle{\pgfqpoint{0.100000in}{0.212622in}}{\pgfqpoint{3.696000in}{3.696000in}}%
\pgfusepath{clip}%
\pgfsetbuttcap%
\pgfsetroundjoin%
\definecolor{currentfill}{rgb}{0.121569,0.466667,0.705882}%
\pgfsetfillcolor{currentfill}%
\pgfsetfillopacity{0.573633}%
\pgfsetlinewidth{1.003750pt}%
\definecolor{currentstroke}{rgb}{0.121569,0.466667,0.705882}%
\pgfsetstrokecolor{currentstroke}%
\pgfsetstrokeopacity{0.573633}%
\pgfsetdash{}{0pt}%
\pgfpathmoveto{\pgfqpoint{2.846864in}{2.629959in}}%
\pgfpathcurveto{\pgfqpoint{2.855100in}{2.629959in}}{\pgfqpoint{2.863000in}{2.633232in}}{\pgfqpoint{2.868824in}{2.639056in}}%
\pgfpathcurveto{\pgfqpoint{2.874648in}{2.644880in}}{\pgfqpoint{2.877920in}{2.652780in}}{\pgfqpoint{2.877920in}{2.661016in}}%
\pgfpathcurveto{\pgfqpoint{2.877920in}{2.669252in}}{\pgfqpoint{2.874648in}{2.677152in}}{\pgfqpoint{2.868824in}{2.682976in}}%
\pgfpathcurveto{\pgfqpoint{2.863000in}{2.688800in}}{\pgfqpoint{2.855100in}{2.692072in}}{\pgfqpoint{2.846864in}{2.692072in}}%
\pgfpathcurveto{\pgfqpoint{2.838627in}{2.692072in}}{\pgfqpoint{2.830727in}{2.688800in}}{\pgfqpoint{2.824903in}{2.682976in}}%
\pgfpathcurveto{\pgfqpoint{2.819079in}{2.677152in}}{\pgfqpoint{2.815807in}{2.669252in}}{\pgfqpoint{2.815807in}{2.661016in}}%
\pgfpathcurveto{\pgfqpoint{2.815807in}{2.652780in}}{\pgfqpoint{2.819079in}{2.644880in}}{\pgfqpoint{2.824903in}{2.639056in}}%
\pgfpathcurveto{\pgfqpoint{2.830727in}{2.633232in}}{\pgfqpoint{2.838627in}{2.629959in}}{\pgfqpoint{2.846864in}{2.629959in}}%
\pgfpathclose%
\pgfusepath{stroke,fill}%
\end{pgfscope}%
\begin{pgfscope}%
\pgfpathrectangle{\pgfqpoint{0.100000in}{0.212622in}}{\pgfqpoint{3.696000in}{3.696000in}}%
\pgfusepath{clip}%
\pgfsetbuttcap%
\pgfsetroundjoin%
\definecolor{currentfill}{rgb}{0.121569,0.466667,0.705882}%
\pgfsetfillcolor{currentfill}%
\pgfsetfillopacity{0.574672}%
\pgfsetlinewidth{1.003750pt}%
\definecolor{currentstroke}{rgb}{0.121569,0.466667,0.705882}%
\pgfsetstrokecolor{currentstroke}%
\pgfsetstrokeopacity{0.574672}%
\pgfsetdash{}{0pt}%
\pgfpathmoveto{\pgfqpoint{2.847113in}{2.629550in}}%
\pgfpathcurveto{\pgfqpoint{2.855350in}{2.629550in}}{\pgfqpoint{2.863250in}{2.632822in}}{\pgfqpoint{2.869074in}{2.638646in}}%
\pgfpathcurveto{\pgfqpoint{2.874898in}{2.644470in}}{\pgfqpoint{2.878170in}{2.652370in}}{\pgfqpoint{2.878170in}{2.660607in}}%
\pgfpathcurveto{\pgfqpoint{2.878170in}{2.668843in}}{\pgfqpoint{2.874898in}{2.676743in}}{\pgfqpoint{2.869074in}{2.682567in}}%
\pgfpathcurveto{\pgfqpoint{2.863250in}{2.688391in}}{\pgfqpoint{2.855350in}{2.691663in}}{\pgfqpoint{2.847113in}{2.691663in}}%
\pgfpathcurveto{\pgfqpoint{2.838877in}{2.691663in}}{\pgfqpoint{2.830977in}{2.688391in}}{\pgfqpoint{2.825153in}{2.682567in}}%
\pgfpathcurveto{\pgfqpoint{2.819329in}{2.676743in}}{\pgfqpoint{2.816057in}{2.668843in}}{\pgfqpoint{2.816057in}{2.660607in}}%
\pgfpathcurveto{\pgfqpoint{2.816057in}{2.652370in}}{\pgfqpoint{2.819329in}{2.644470in}}{\pgfqpoint{2.825153in}{2.638646in}}%
\pgfpathcurveto{\pgfqpoint{2.830977in}{2.632822in}}{\pgfqpoint{2.838877in}{2.629550in}}{\pgfqpoint{2.847113in}{2.629550in}}%
\pgfpathclose%
\pgfusepath{stroke,fill}%
\end{pgfscope}%
\begin{pgfscope}%
\pgfpathrectangle{\pgfqpoint{0.100000in}{0.212622in}}{\pgfqpoint{3.696000in}{3.696000in}}%
\pgfusepath{clip}%
\pgfsetbuttcap%
\pgfsetroundjoin%
\definecolor{currentfill}{rgb}{0.121569,0.466667,0.705882}%
\pgfsetfillcolor{currentfill}%
\pgfsetfillopacity{0.575170}%
\pgfsetlinewidth{1.003750pt}%
\definecolor{currentstroke}{rgb}{0.121569,0.466667,0.705882}%
\pgfsetstrokecolor{currentstroke}%
\pgfsetstrokeopacity{0.575170}%
\pgfsetdash{}{0pt}%
\pgfpathmoveto{\pgfqpoint{2.849638in}{2.631206in}}%
\pgfpathcurveto{\pgfqpoint{2.857874in}{2.631206in}}{\pgfqpoint{2.865774in}{2.634478in}}{\pgfqpoint{2.871598in}{2.640302in}}%
\pgfpathcurveto{\pgfqpoint{2.877422in}{2.646126in}}{\pgfqpoint{2.880694in}{2.654026in}}{\pgfqpoint{2.880694in}{2.662262in}}%
\pgfpathcurveto{\pgfqpoint{2.880694in}{2.670499in}}{\pgfqpoint{2.877422in}{2.678399in}}{\pgfqpoint{2.871598in}{2.684223in}}%
\pgfpathcurveto{\pgfqpoint{2.865774in}{2.690046in}}{\pgfqpoint{2.857874in}{2.693319in}}{\pgfqpoint{2.849638in}{2.693319in}}%
\pgfpathcurveto{\pgfqpoint{2.841401in}{2.693319in}}{\pgfqpoint{2.833501in}{2.690046in}}{\pgfqpoint{2.827677in}{2.684223in}}%
\pgfpathcurveto{\pgfqpoint{2.821853in}{2.678399in}}{\pgfqpoint{2.818581in}{2.670499in}}{\pgfqpoint{2.818581in}{2.662262in}}%
\pgfpathcurveto{\pgfqpoint{2.818581in}{2.654026in}}{\pgfqpoint{2.821853in}{2.646126in}}{\pgfqpoint{2.827677in}{2.640302in}}%
\pgfpathcurveto{\pgfqpoint{2.833501in}{2.634478in}}{\pgfqpoint{2.841401in}{2.631206in}}{\pgfqpoint{2.849638in}{2.631206in}}%
\pgfpathclose%
\pgfusepath{stroke,fill}%
\end{pgfscope}%
\begin{pgfscope}%
\pgfpathrectangle{\pgfqpoint{0.100000in}{0.212622in}}{\pgfqpoint{3.696000in}{3.696000in}}%
\pgfusepath{clip}%
\pgfsetbuttcap%
\pgfsetroundjoin%
\definecolor{currentfill}{rgb}{0.121569,0.466667,0.705882}%
\pgfsetfillcolor{currentfill}%
\pgfsetfillopacity{0.575488}%
\pgfsetlinewidth{1.003750pt}%
\definecolor{currentstroke}{rgb}{0.121569,0.466667,0.705882}%
\pgfsetstrokecolor{currentstroke}%
\pgfsetstrokeopacity{0.575488}%
\pgfsetdash{}{0pt}%
\pgfpathmoveto{\pgfqpoint{2.850661in}{2.631148in}}%
\pgfpathcurveto{\pgfqpoint{2.858897in}{2.631148in}}{\pgfqpoint{2.866797in}{2.634421in}}{\pgfqpoint{2.872621in}{2.640245in}}%
\pgfpathcurveto{\pgfqpoint{2.878445in}{2.646069in}}{\pgfqpoint{2.881718in}{2.653969in}}{\pgfqpoint{2.881718in}{2.662205in}}%
\pgfpathcurveto{\pgfqpoint{2.881718in}{2.670441in}}{\pgfqpoint{2.878445in}{2.678341in}}{\pgfqpoint{2.872621in}{2.684165in}}%
\pgfpathcurveto{\pgfqpoint{2.866797in}{2.689989in}}{\pgfqpoint{2.858897in}{2.693261in}}{\pgfqpoint{2.850661in}{2.693261in}}%
\pgfpathcurveto{\pgfqpoint{2.842425in}{2.693261in}}{\pgfqpoint{2.834525in}{2.689989in}}{\pgfqpoint{2.828701in}{2.684165in}}%
\pgfpathcurveto{\pgfqpoint{2.822877in}{2.678341in}}{\pgfqpoint{2.819605in}{2.670441in}}{\pgfqpoint{2.819605in}{2.662205in}}%
\pgfpathcurveto{\pgfqpoint{2.819605in}{2.653969in}}{\pgfqpoint{2.822877in}{2.646069in}}{\pgfqpoint{2.828701in}{2.640245in}}%
\pgfpathcurveto{\pgfqpoint{2.834525in}{2.634421in}}{\pgfqpoint{2.842425in}{2.631148in}}{\pgfqpoint{2.850661in}{2.631148in}}%
\pgfpathclose%
\pgfusepath{stroke,fill}%
\end{pgfscope}%
\begin{pgfscope}%
\pgfpathrectangle{\pgfqpoint{0.100000in}{0.212622in}}{\pgfqpoint{3.696000in}{3.696000in}}%
\pgfusepath{clip}%
\pgfsetbuttcap%
\pgfsetroundjoin%
\definecolor{currentfill}{rgb}{0.121569,0.466667,0.705882}%
\pgfsetfillcolor{currentfill}%
\pgfsetfillopacity{0.576324}%
\pgfsetlinewidth{1.003750pt}%
\definecolor{currentstroke}{rgb}{0.121569,0.466667,0.705882}%
\pgfsetstrokecolor{currentstroke}%
\pgfsetstrokeopacity{0.576324}%
\pgfsetdash{}{0pt}%
\pgfpathmoveto{\pgfqpoint{2.855912in}{2.635252in}}%
\pgfpathcurveto{\pgfqpoint{2.864149in}{2.635252in}}{\pgfqpoint{2.872049in}{2.638524in}}{\pgfqpoint{2.877873in}{2.644348in}}%
\pgfpathcurveto{\pgfqpoint{2.883697in}{2.650172in}}{\pgfqpoint{2.886969in}{2.658072in}}{\pgfqpoint{2.886969in}{2.666308in}}%
\pgfpathcurveto{\pgfqpoint{2.886969in}{2.674545in}}{\pgfqpoint{2.883697in}{2.682445in}}{\pgfqpoint{2.877873in}{2.688269in}}%
\pgfpathcurveto{\pgfqpoint{2.872049in}{2.694092in}}{\pgfqpoint{2.864149in}{2.697365in}}{\pgfqpoint{2.855912in}{2.697365in}}%
\pgfpathcurveto{\pgfqpoint{2.847676in}{2.697365in}}{\pgfqpoint{2.839776in}{2.694092in}}{\pgfqpoint{2.833952in}{2.688269in}}%
\pgfpathcurveto{\pgfqpoint{2.828128in}{2.682445in}}{\pgfqpoint{2.824856in}{2.674545in}}{\pgfqpoint{2.824856in}{2.666308in}}%
\pgfpathcurveto{\pgfqpoint{2.824856in}{2.658072in}}{\pgfqpoint{2.828128in}{2.650172in}}{\pgfqpoint{2.833952in}{2.644348in}}%
\pgfpathcurveto{\pgfqpoint{2.839776in}{2.638524in}}{\pgfqpoint{2.847676in}{2.635252in}}{\pgfqpoint{2.855912in}{2.635252in}}%
\pgfpathclose%
\pgfusepath{stroke,fill}%
\end{pgfscope}%
\begin{pgfscope}%
\pgfpathrectangle{\pgfqpoint{0.100000in}{0.212622in}}{\pgfqpoint{3.696000in}{3.696000in}}%
\pgfusepath{clip}%
\pgfsetbuttcap%
\pgfsetroundjoin%
\definecolor{currentfill}{rgb}{0.121569,0.466667,0.705882}%
\pgfsetfillcolor{currentfill}%
\pgfsetfillopacity{0.576384}%
\pgfsetlinewidth{1.003750pt}%
\definecolor{currentstroke}{rgb}{0.121569,0.466667,0.705882}%
\pgfsetstrokecolor{currentstroke}%
\pgfsetstrokeopacity{0.576384}%
\pgfsetdash{}{0pt}%
\pgfpathmoveto{\pgfqpoint{2.850917in}{2.630493in}}%
\pgfpathcurveto{\pgfqpoint{2.859154in}{2.630493in}}{\pgfqpoint{2.867054in}{2.633765in}}{\pgfqpoint{2.872878in}{2.639589in}}%
\pgfpathcurveto{\pgfqpoint{2.878702in}{2.645413in}}{\pgfqpoint{2.881974in}{2.653313in}}{\pgfqpoint{2.881974in}{2.661549in}}%
\pgfpathcurveto{\pgfqpoint{2.881974in}{2.669785in}}{\pgfqpoint{2.878702in}{2.677686in}}{\pgfqpoint{2.872878in}{2.683509in}}%
\pgfpathcurveto{\pgfqpoint{2.867054in}{2.689333in}}{\pgfqpoint{2.859154in}{2.692606in}}{\pgfqpoint{2.850917in}{2.692606in}}%
\pgfpathcurveto{\pgfqpoint{2.842681in}{2.692606in}}{\pgfqpoint{2.834781in}{2.689333in}}{\pgfqpoint{2.828957in}{2.683509in}}%
\pgfpathcurveto{\pgfqpoint{2.823133in}{2.677686in}}{\pgfqpoint{2.819861in}{2.669785in}}{\pgfqpoint{2.819861in}{2.661549in}}%
\pgfpathcurveto{\pgfqpoint{2.819861in}{2.653313in}}{\pgfqpoint{2.823133in}{2.645413in}}{\pgfqpoint{2.828957in}{2.639589in}}%
\pgfpathcurveto{\pgfqpoint{2.834781in}{2.633765in}}{\pgfqpoint{2.842681in}{2.630493in}}{\pgfqpoint{2.850917in}{2.630493in}}%
\pgfpathclose%
\pgfusepath{stroke,fill}%
\end{pgfscope}%
\begin{pgfscope}%
\pgfpathrectangle{\pgfqpoint{0.100000in}{0.212622in}}{\pgfqpoint{3.696000in}{3.696000in}}%
\pgfusepath{clip}%
\pgfsetbuttcap%
\pgfsetroundjoin%
\definecolor{currentfill}{rgb}{0.121569,0.466667,0.705882}%
\pgfsetfillcolor{currentfill}%
\pgfsetfillopacity{0.577667}%
\pgfsetlinewidth{1.003750pt}%
\definecolor{currentstroke}{rgb}{0.121569,0.466667,0.705882}%
\pgfsetstrokecolor{currentstroke}%
\pgfsetstrokeopacity{0.577667}%
\pgfsetdash{}{0pt}%
\pgfpathmoveto{\pgfqpoint{2.855903in}{2.633247in}}%
\pgfpathcurveto{\pgfqpoint{2.864139in}{2.633247in}}{\pgfqpoint{2.872039in}{2.636519in}}{\pgfqpoint{2.877863in}{2.642343in}}%
\pgfpathcurveto{\pgfqpoint{2.883687in}{2.648167in}}{\pgfqpoint{2.886960in}{2.656067in}}{\pgfqpoint{2.886960in}{2.664303in}}%
\pgfpathcurveto{\pgfqpoint{2.886960in}{2.672540in}}{\pgfqpoint{2.883687in}{2.680440in}}{\pgfqpoint{2.877863in}{2.686264in}}%
\pgfpathcurveto{\pgfqpoint{2.872039in}{2.692087in}}{\pgfqpoint{2.864139in}{2.695360in}}{\pgfqpoint{2.855903in}{2.695360in}}%
\pgfpathcurveto{\pgfqpoint{2.847667in}{2.695360in}}{\pgfqpoint{2.839767in}{2.692087in}}{\pgfqpoint{2.833943in}{2.686264in}}%
\pgfpathcurveto{\pgfqpoint{2.828119in}{2.680440in}}{\pgfqpoint{2.824847in}{2.672540in}}{\pgfqpoint{2.824847in}{2.664303in}}%
\pgfpathcurveto{\pgfqpoint{2.824847in}{2.656067in}}{\pgfqpoint{2.828119in}{2.648167in}}{\pgfqpoint{2.833943in}{2.642343in}}%
\pgfpathcurveto{\pgfqpoint{2.839767in}{2.636519in}}{\pgfqpoint{2.847667in}{2.633247in}}{\pgfqpoint{2.855903in}{2.633247in}}%
\pgfpathclose%
\pgfusepath{stroke,fill}%
\end{pgfscope}%
\begin{pgfscope}%
\pgfpathrectangle{\pgfqpoint{0.100000in}{0.212622in}}{\pgfqpoint{3.696000in}{3.696000in}}%
\pgfusepath{clip}%
\pgfsetbuttcap%
\pgfsetroundjoin%
\definecolor{currentfill}{rgb}{0.121569,0.466667,0.705882}%
\pgfsetfillcolor{currentfill}%
\pgfsetfillopacity{0.577801}%
\pgfsetlinewidth{1.003750pt}%
\definecolor{currentstroke}{rgb}{0.121569,0.466667,0.705882}%
\pgfsetstrokecolor{currentstroke}%
\pgfsetstrokeopacity{0.577801}%
\pgfsetdash{}{0pt}%
\pgfpathmoveto{\pgfqpoint{2.848352in}{2.627029in}}%
\pgfpathcurveto{\pgfqpoint{2.856588in}{2.627029in}}{\pgfqpoint{2.864488in}{2.630301in}}{\pgfqpoint{2.870312in}{2.636125in}}%
\pgfpathcurveto{\pgfqpoint{2.876136in}{2.641949in}}{\pgfqpoint{2.879409in}{2.649849in}}{\pgfqpoint{2.879409in}{2.658085in}}%
\pgfpathcurveto{\pgfqpoint{2.879409in}{2.666322in}}{\pgfqpoint{2.876136in}{2.674222in}}{\pgfqpoint{2.870312in}{2.680046in}}%
\pgfpathcurveto{\pgfqpoint{2.864488in}{2.685870in}}{\pgfqpoint{2.856588in}{2.689142in}}{\pgfqpoint{2.848352in}{2.689142in}}%
\pgfpathcurveto{\pgfqpoint{2.840116in}{2.689142in}}{\pgfqpoint{2.832216in}{2.685870in}}{\pgfqpoint{2.826392in}{2.680046in}}%
\pgfpathcurveto{\pgfqpoint{2.820568in}{2.674222in}}{\pgfqpoint{2.817296in}{2.666322in}}{\pgfqpoint{2.817296in}{2.658085in}}%
\pgfpathcurveto{\pgfqpoint{2.817296in}{2.649849in}}{\pgfqpoint{2.820568in}{2.641949in}}{\pgfqpoint{2.826392in}{2.636125in}}%
\pgfpathcurveto{\pgfqpoint{2.832216in}{2.630301in}}{\pgfqpoint{2.840116in}{2.627029in}}{\pgfqpoint{2.848352in}{2.627029in}}%
\pgfpathclose%
\pgfusepath{stroke,fill}%
\end{pgfscope}%
\begin{pgfscope}%
\pgfpathrectangle{\pgfqpoint{0.100000in}{0.212622in}}{\pgfqpoint{3.696000in}{3.696000in}}%
\pgfusepath{clip}%
\pgfsetbuttcap%
\pgfsetroundjoin%
\definecolor{currentfill}{rgb}{0.121569,0.466667,0.705882}%
\pgfsetfillcolor{currentfill}%
\pgfsetfillopacity{0.578669}%
\pgfsetlinewidth{1.003750pt}%
\definecolor{currentstroke}{rgb}{0.121569,0.466667,0.705882}%
\pgfsetstrokecolor{currentstroke}%
\pgfsetstrokeopacity{0.578669}%
\pgfsetdash{}{0pt}%
\pgfpathmoveto{\pgfqpoint{2.887367in}{2.652222in}}%
\pgfpathcurveto{\pgfqpoint{2.895603in}{2.652222in}}{\pgfqpoint{2.903503in}{2.655494in}}{\pgfqpoint{2.909327in}{2.661318in}}%
\pgfpathcurveto{\pgfqpoint{2.915151in}{2.667142in}}{\pgfqpoint{2.918423in}{2.675042in}}{\pgfqpoint{2.918423in}{2.683278in}}%
\pgfpathcurveto{\pgfqpoint{2.918423in}{2.691515in}}{\pgfqpoint{2.915151in}{2.699415in}}{\pgfqpoint{2.909327in}{2.705239in}}%
\pgfpathcurveto{\pgfqpoint{2.903503in}{2.711063in}}{\pgfqpoint{2.895603in}{2.714335in}}{\pgfqpoint{2.887367in}{2.714335in}}%
\pgfpathcurveto{\pgfqpoint{2.879130in}{2.714335in}}{\pgfqpoint{2.871230in}{2.711063in}}{\pgfqpoint{2.865406in}{2.705239in}}%
\pgfpathcurveto{\pgfqpoint{2.859582in}{2.699415in}}{\pgfqpoint{2.856310in}{2.691515in}}{\pgfqpoint{2.856310in}{2.683278in}}%
\pgfpathcurveto{\pgfqpoint{2.856310in}{2.675042in}}{\pgfqpoint{2.859582in}{2.667142in}}{\pgfqpoint{2.865406in}{2.661318in}}%
\pgfpathcurveto{\pgfqpoint{2.871230in}{2.655494in}}{\pgfqpoint{2.879130in}{2.652222in}}{\pgfqpoint{2.887367in}{2.652222in}}%
\pgfpathclose%
\pgfusepath{stroke,fill}%
\end{pgfscope}%
\begin{pgfscope}%
\pgfpathrectangle{\pgfqpoint{0.100000in}{0.212622in}}{\pgfqpoint{3.696000in}{3.696000in}}%
\pgfusepath{clip}%
\pgfsetbuttcap%
\pgfsetroundjoin%
\definecolor{currentfill}{rgb}{0.121569,0.466667,0.705882}%
\pgfsetfillcolor{currentfill}%
\pgfsetfillopacity{0.579565}%
\pgfsetlinewidth{1.003750pt}%
\definecolor{currentstroke}{rgb}{0.121569,0.466667,0.705882}%
\pgfsetstrokecolor{currentstroke}%
\pgfsetstrokeopacity{0.579565}%
\pgfsetdash{}{0pt}%
\pgfpathmoveto{\pgfqpoint{2.855403in}{2.631085in}}%
\pgfpathcurveto{\pgfqpoint{2.863639in}{2.631085in}}{\pgfqpoint{2.871539in}{2.634357in}}{\pgfqpoint{2.877363in}{2.640181in}}%
\pgfpathcurveto{\pgfqpoint{2.883187in}{2.646005in}}{\pgfqpoint{2.886459in}{2.653905in}}{\pgfqpoint{2.886459in}{2.662142in}}%
\pgfpathcurveto{\pgfqpoint{2.886459in}{2.670378in}}{\pgfqpoint{2.883187in}{2.678278in}}{\pgfqpoint{2.877363in}{2.684102in}}%
\pgfpathcurveto{\pgfqpoint{2.871539in}{2.689926in}}{\pgfqpoint{2.863639in}{2.693198in}}{\pgfqpoint{2.855403in}{2.693198in}}%
\pgfpathcurveto{\pgfqpoint{2.847167in}{2.693198in}}{\pgfqpoint{2.839267in}{2.689926in}}{\pgfqpoint{2.833443in}{2.684102in}}%
\pgfpathcurveto{\pgfqpoint{2.827619in}{2.678278in}}{\pgfqpoint{2.824346in}{2.670378in}}{\pgfqpoint{2.824346in}{2.662142in}}%
\pgfpathcurveto{\pgfqpoint{2.824346in}{2.653905in}}{\pgfqpoint{2.827619in}{2.646005in}}{\pgfqpoint{2.833443in}{2.640181in}}%
\pgfpathcurveto{\pgfqpoint{2.839267in}{2.634357in}}{\pgfqpoint{2.847167in}{2.631085in}}{\pgfqpoint{2.855403in}{2.631085in}}%
\pgfpathclose%
\pgfusepath{stroke,fill}%
\end{pgfscope}%
\begin{pgfscope}%
\pgfpathrectangle{\pgfqpoint{0.100000in}{0.212622in}}{\pgfqpoint{3.696000in}{3.696000in}}%
\pgfusepath{clip}%
\pgfsetbuttcap%
\pgfsetroundjoin%
\definecolor{currentfill}{rgb}{0.121569,0.466667,0.705882}%
\pgfsetfillcolor{currentfill}%
\pgfsetfillopacity{0.581662}%
\pgfsetlinewidth{1.003750pt}%
\definecolor{currentstroke}{rgb}{0.121569,0.466667,0.705882}%
\pgfsetstrokecolor{currentstroke}%
\pgfsetstrokeopacity{0.581662}%
\pgfsetdash{}{0pt}%
\pgfpathmoveto{\pgfqpoint{2.868450in}{2.633408in}}%
\pgfpathcurveto{\pgfqpoint{2.876687in}{2.633408in}}{\pgfqpoint{2.884587in}{2.636681in}}{\pgfqpoint{2.890411in}{2.642505in}}%
\pgfpathcurveto{\pgfqpoint{2.896235in}{2.648329in}}{\pgfqpoint{2.899507in}{2.656229in}}{\pgfqpoint{2.899507in}{2.664465in}}%
\pgfpathcurveto{\pgfqpoint{2.899507in}{2.672701in}}{\pgfqpoint{2.896235in}{2.680601in}}{\pgfqpoint{2.890411in}{2.686425in}}%
\pgfpathcurveto{\pgfqpoint{2.884587in}{2.692249in}}{\pgfqpoint{2.876687in}{2.695521in}}{\pgfqpoint{2.868450in}{2.695521in}}%
\pgfpathcurveto{\pgfqpoint{2.860214in}{2.695521in}}{\pgfqpoint{2.852314in}{2.692249in}}{\pgfqpoint{2.846490in}{2.686425in}}%
\pgfpathcurveto{\pgfqpoint{2.840666in}{2.680601in}}{\pgfqpoint{2.837394in}{2.672701in}}{\pgfqpoint{2.837394in}{2.664465in}}%
\pgfpathcurveto{\pgfqpoint{2.837394in}{2.656229in}}{\pgfqpoint{2.840666in}{2.648329in}}{\pgfqpoint{2.846490in}{2.642505in}}%
\pgfpathcurveto{\pgfqpoint{2.852314in}{2.636681in}}{\pgfqpoint{2.860214in}{2.633408in}}{\pgfqpoint{2.868450in}{2.633408in}}%
\pgfpathclose%
\pgfusepath{stroke,fill}%
\end{pgfscope}%
\begin{pgfscope}%
\pgfpathrectangle{\pgfqpoint{0.100000in}{0.212622in}}{\pgfqpoint{3.696000in}{3.696000in}}%
\pgfusepath{clip}%
\pgfsetbuttcap%
\pgfsetroundjoin%
\definecolor{currentfill}{rgb}{0.121569,0.466667,0.705882}%
\pgfsetfillcolor{currentfill}%
\pgfsetfillopacity{0.581899}%
\pgfsetlinewidth{1.003750pt}%
\definecolor{currentstroke}{rgb}{0.121569,0.466667,0.705882}%
\pgfsetstrokecolor{currentstroke}%
\pgfsetstrokeopacity{0.581899}%
\pgfsetdash{}{0pt}%
\pgfpathmoveto{\pgfqpoint{2.853519in}{2.625901in}}%
\pgfpathcurveto{\pgfqpoint{2.861755in}{2.625901in}}{\pgfqpoint{2.869655in}{2.629173in}}{\pgfqpoint{2.875479in}{2.634997in}}%
\pgfpathcurveto{\pgfqpoint{2.881303in}{2.640821in}}{\pgfqpoint{2.884575in}{2.648721in}}{\pgfqpoint{2.884575in}{2.656958in}}%
\pgfpathcurveto{\pgfqpoint{2.884575in}{2.665194in}}{\pgfqpoint{2.881303in}{2.673094in}}{\pgfqpoint{2.875479in}{2.678918in}}%
\pgfpathcurveto{\pgfqpoint{2.869655in}{2.684742in}}{\pgfqpoint{2.861755in}{2.688014in}}{\pgfqpoint{2.853519in}{2.688014in}}%
\pgfpathcurveto{\pgfqpoint{2.845283in}{2.688014in}}{\pgfqpoint{2.837382in}{2.684742in}}{\pgfqpoint{2.831559in}{2.678918in}}%
\pgfpathcurveto{\pgfqpoint{2.825735in}{2.673094in}}{\pgfqpoint{2.822462in}{2.665194in}}{\pgfqpoint{2.822462in}{2.656958in}}%
\pgfpathcurveto{\pgfqpoint{2.822462in}{2.648721in}}{\pgfqpoint{2.825735in}{2.640821in}}{\pgfqpoint{2.831559in}{2.634997in}}%
\pgfpathcurveto{\pgfqpoint{2.837382in}{2.629173in}}{\pgfqpoint{2.845283in}{2.625901in}}{\pgfqpoint{2.853519in}{2.625901in}}%
\pgfpathclose%
\pgfusepath{stroke,fill}%
\end{pgfscope}%
\begin{pgfscope}%
\pgfpathrectangle{\pgfqpoint{0.100000in}{0.212622in}}{\pgfqpoint{3.696000in}{3.696000in}}%
\pgfusepath{clip}%
\pgfsetbuttcap%
\pgfsetroundjoin%
\definecolor{currentfill}{rgb}{0.121569,0.466667,0.705882}%
\pgfsetfillcolor{currentfill}%
\pgfsetfillopacity{0.582847}%
\pgfsetlinewidth{1.003750pt}%
\definecolor{currentstroke}{rgb}{0.121569,0.466667,0.705882}%
\pgfsetstrokecolor{currentstroke}%
\pgfsetstrokeopacity{0.582847}%
\pgfsetdash{}{0pt}%
\pgfpathmoveto{\pgfqpoint{2.885258in}{2.642262in}}%
\pgfpathcurveto{\pgfqpoint{2.893494in}{2.642262in}}{\pgfqpoint{2.901394in}{2.645534in}}{\pgfqpoint{2.907218in}{2.651358in}}%
\pgfpathcurveto{\pgfqpoint{2.913042in}{2.657182in}}{\pgfqpoint{2.916314in}{2.665082in}}{\pgfqpoint{2.916314in}{2.673318in}}%
\pgfpathcurveto{\pgfqpoint{2.916314in}{2.681554in}}{\pgfqpoint{2.913042in}{2.689455in}}{\pgfqpoint{2.907218in}{2.695278in}}%
\pgfpathcurveto{\pgfqpoint{2.901394in}{2.701102in}}{\pgfqpoint{2.893494in}{2.704375in}}{\pgfqpoint{2.885258in}{2.704375in}}%
\pgfpathcurveto{\pgfqpoint{2.877021in}{2.704375in}}{\pgfqpoint{2.869121in}{2.701102in}}{\pgfqpoint{2.863297in}{2.695278in}}%
\pgfpathcurveto{\pgfqpoint{2.857473in}{2.689455in}}{\pgfqpoint{2.854201in}{2.681554in}}{\pgfqpoint{2.854201in}{2.673318in}}%
\pgfpathcurveto{\pgfqpoint{2.854201in}{2.665082in}}{\pgfqpoint{2.857473in}{2.657182in}}{\pgfqpoint{2.863297in}{2.651358in}}%
\pgfpathcurveto{\pgfqpoint{2.869121in}{2.645534in}}{\pgfqpoint{2.877021in}{2.642262in}}{\pgfqpoint{2.885258in}{2.642262in}}%
\pgfpathclose%
\pgfusepath{stroke,fill}%
\end{pgfscope}%
\begin{pgfscope}%
\pgfpathrectangle{\pgfqpoint{0.100000in}{0.212622in}}{\pgfqpoint{3.696000in}{3.696000in}}%
\pgfusepath{clip}%
\pgfsetbuttcap%
\pgfsetroundjoin%
\definecolor{currentfill}{rgb}{0.121569,0.466667,0.705882}%
\pgfsetfillcolor{currentfill}%
\pgfsetfillopacity{0.584331}%
\pgfsetlinewidth{1.003750pt}%
\definecolor{currentstroke}{rgb}{0.121569,0.466667,0.705882}%
\pgfsetstrokecolor{currentstroke}%
\pgfsetstrokeopacity{0.584331}%
\pgfsetdash{}{0pt}%
\pgfpathmoveto{\pgfqpoint{2.854040in}{2.624693in}}%
\pgfpathcurveto{\pgfqpoint{2.862276in}{2.624693in}}{\pgfqpoint{2.870176in}{2.627966in}}{\pgfqpoint{2.876000in}{2.633790in}}%
\pgfpathcurveto{\pgfqpoint{2.881824in}{2.639614in}}{\pgfqpoint{2.885097in}{2.647514in}}{\pgfqpoint{2.885097in}{2.655750in}}%
\pgfpathcurveto{\pgfqpoint{2.885097in}{2.663986in}}{\pgfqpoint{2.881824in}{2.671886in}}{\pgfqpoint{2.876000in}{2.677710in}}%
\pgfpathcurveto{\pgfqpoint{2.870176in}{2.683534in}}{\pgfqpoint{2.862276in}{2.686806in}}{\pgfqpoint{2.854040in}{2.686806in}}%
\pgfpathcurveto{\pgfqpoint{2.845804in}{2.686806in}}{\pgfqpoint{2.837904in}{2.683534in}}{\pgfqpoint{2.832080in}{2.677710in}}%
\pgfpathcurveto{\pgfqpoint{2.826256in}{2.671886in}}{\pgfqpoint{2.822984in}{2.663986in}}{\pgfqpoint{2.822984in}{2.655750in}}%
\pgfpathcurveto{\pgfqpoint{2.822984in}{2.647514in}}{\pgfqpoint{2.826256in}{2.639614in}}{\pgfqpoint{2.832080in}{2.633790in}}%
\pgfpathcurveto{\pgfqpoint{2.837904in}{2.627966in}}{\pgfqpoint{2.845804in}{2.624693in}}{\pgfqpoint{2.854040in}{2.624693in}}%
\pgfpathclose%
\pgfusepath{stroke,fill}%
\end{pgfscope}%
\begin{pgfscope}%
\pgfpathrectangle{\pgfqpoint{0.100000in}{0.212622in}}{\pgfqpoint{3.696000in}{3.696000in}}%
\pgfusepath{clip}%
\pgfsetbuttcap%
\pgfsetroundjoin%
\definecolor{currentfill}{rgb}{0.121569,0.466667,0.705882}%
\pgfsetfillcolor{currentfill}%
\pgfsetfillopacity{0.588634}%
\pgfsetlinewidth{1.003750pt}%
\definecolor{currentstroke}{rgb}{0.121569,0.466667,0.705882}%
\pgfsetstrokecolor{currentstroke}%
\pgfsetstrokeopacity{0.588634}%
\pgfsetdash{}{0pt}%
\pgfpathmoveto{\pgfqpoint{2.881526in}{2.634840in}}%
\pgfpathcurveto{\pgfqpoint{2.889762in}{2.634840in}}{\pgfqpoint{2.897662in}{2.638112in}}{\pgfqpoint{2.903486in}{2.643936in}}%
\pgfpathcurveto{\pgfqpoint{2.909310in}{2.649760in}}{\pgfqpoint{2.912583in}{2.657660in}}{\pgfqpoint{2.912583in}{2.665897in}}%
\pgfpathcurveto{\pgfqpoint{2.912583in}{2.674133in}}{\pgfqpoint{2.909310in}{2.682033in}}{\pgfqpoint{2.903486in}{2.687857in}}%
\pgfpathcurveto{\pgfqpoint{2.897662in}{2.693681in}}{\pgfqpoint{2.889762in}{2.696953in}}{\pgfqpoint{2.881526in}{2.696953in}}%
\pgfpathcurveto{\pgfqpoint{2.873290in}{2.696953in}}{\pgfqpoint{2.865390in}{2.693681in}}{\pgfqpoint{2.859566in}{2.687857in}}%
\pgfpathcurveto{\pgfqpoint{2.853742in}{2.682033in}}{\pgfqpoint{2.850470in}{2.674133in}}{\pgfqpoint{2.850470in}{2.665897in}}%
\pgfpathcurveto{\pgfqpoint{2.850470in}{2.657660in}}{\pgfqpoint{2.853742in}{2.649760in}}{\pgfqpoint{2.859566in}{2.643936in}}%
\pgfpathcurveto{\pgfqpoint{2.865390in}{2.638112in}}{\pgfqpoint{2.873290in}{2.634840in}}{\pgfqpoint{2.881526in}{2.634840in}}%
\pgfpathclose%
\pgfusepath{stroke,fill}%
\end{pgfscope}%
\begin{pgfscope}%
\pgfpathrectangle{\pgfqpoint{0.100000in}{0.212622in}}{\pgfqpoint{3.696000in}{3.696000in}}%
\pgfusepath{clip}%
\pgfsetbuttcap%
\pgfsetroundjoin%
\definecolor{currentfill}{rgb}{0.121569,0.466667,0.705882}%
\pgfsetfillcolor{currentfill}%
\pgfsetfillopacity{0.593320}%
\pgfsetlinewidth{1.003750pt}%
\definecolor{currentstroke}{rgb}{0.121569,0.466667,0.705882}%
\pgfsetstrokecolor{currentstroke}%
\pgfsetstrokeopacity{0.593320}%
\pgfsetdash{}{0pt}%
\pgfpathmoveto{\pgfqpoint{2.898013in}{2.645225in}}%
\pgfpathcurveto{\pgfqpoint{2.906249in}{2.645225in}}{\pgfqpoint{2.914149in}{2.648497in}}{\pgfqpoint{2.919973in}{2.654321in}}%
\pgfpathcurveto{\pgfqpoint{2.925797in}{2.660145in}}{\pgfqpoint{2.929069in}{2.668045in}}{\pgfqpoint{2.929069in}{2.676281in}}%
\pgfpathcurveto{\pgfqpoint{2.929069in}{2.684518in}}{\pgfqpoint{2.925797in}{2.692418in}}{\pgfqpoint{2.919973in}{2.698242in}}%
\pgfpathcurveto{\pgfqpoint{2.914149in}{2.704065in}}{\pgfqpoint{2.906249in}{2.707338in}}{\pgfqpoint{2.898013in}{2.707338in}}%
\pgfpathcurveto{\pgfqpoint{2.889776in}{2.707338in}}{\pgfqpoint{2.881876in}{2.704065in}}{\pgfqpoint{2.876052in}{2.698242in}}%
\pgfpathcurveto{\pgfqpoint{2.870228in}{2.692418in}}{\pgfqpoint{2.866956in}{2.684518in}}{\pgfqpoint{2.866956in}{2.676281in}}%
\pgfpathcurveto{\pgfqpoint{2.866956in}{2.668045in}}{\pgfqpoint{2.870228in}{2.660145in}}{\pgfqpoint{2.876052in}{2.654321in}}%
\pgfpathcurveto{\pgfqpoint{2.881876in}{2.648497in}}{\pgfqpoint{2.889776in}{2.645225in}}{\pgfqpoint{2.898013in}{2.645225in}}%
\pgfpathclose%
\pgfusepath{stroke,fill}%
\end{pgfscope}%
\begin{pgfscope}%
\pgfpathrectangle{\pgfqpoint{0.100000in}{0.212622in}}{\pgfqpoint{3.696000in}{3.696000in}}%
\pgfusepath{clip}%
\pgfsetbuttcap%
\pgfsetroundjoin%
\definecolor{currentfill}{rgb}{0.121569,0.466667,0.705882}%
\pgfsetfillcolor{currentfill}%
\pgfsetfillopacity{0.596931}%
\pgfsetlinewidth{1.003750pt}%
\definecolor{currentstroke}{rgb}{0.121569,0.466667,0.705882}%
\pgfsetstrokecolor{currentstroke}%
\pgfsetstrokeopacity{0.596931}%
\pgfsetdash{}{0pt}%
\pgfpathmoveto{\pgfqpoint{2.905275in}{2.647787in}}%
\pgfpathcurveto{\pgfqpoint{2.913511in}{2.647787in}}{\pgfqpoint{2.921411in}{2.651060in}}{\pgfqpoint{2.927235in}{2.656884in}}%
\pgfpathcurveto{\pgfqpoint{2.933059in}{2.662708in}}{\pgfqpoint{2.936331in}{2.670608in}}{\pgfqpoint{2.936331in}{2.678844in}}%
\pgfpathcurveto{\pgfqpoint{2.936331in}{2.687080in}}{\pgfqpoint{2.933059in}{2.694980in}}{\pgfqpoint{2.927235in}{2.700804in}}%
\pgfpathcurveto{\pgfqpoint{2.921411in}{2.706628in}}{\pgfqpoint{2.913511in}{2.709900in}}{\pgfqpoint{2.905275in}{2.709900in}}%
\pgfpathcurveto{\pgfqpoint{2.897038in}{2.709900in}}{\pgfqpoint{2.889138in}{2.706628in}}{\pgfqpoint{2.883314in}{2.700804in}}%
\pgfpathcurveto{\pgfqpoint{2.877490in}{2.694980in}}{\pgfqpoint{2.874218in}{2.687080in}}{\pgfqpoint{2.874218in}{2.678844in}}%
\pgfpathcurveto{\pgfqpoint{2.874218in}{2.670608in}}{\pgfqpoint{2.877490in}{2.662708in}}{\pgfqpoint{2.883314in}{2.656884in}}%
\pgfpathcurveto{\pgfqpoint{2.889138in}{2.651060in}}{\pgfqpoint{2.897038in}{2.647787in}}{\pgfqpoint{2.905275in}{2.647787in}}%
\pgfpathclose%
\pgfusepath{stroke,fill}%
\end{pgfscope}%
\begin{pgfscope}%
\pgfpathrectangle{\pgfqpoint{0.100000in}{0.212622in}}{\pgfqpoint{3.696000in}{3.696000in}}%
\pgfusepath{clip}%
\pgfsetbuttcap%
\pgfsetroundjoin%
\definecolor{currentfill}{rgb}{0.121569,0.466667,0.705882}%
\pgfsetfillcolor{currentfill}%
\pgfsetfillopacity{0.597366}%
\pgfsetlinewidth{1.003750pt}%
\definecolor{currentstroke}{rgb}{0.121569,0.466667,0.705882}%
\pgfsetstrokecolor{currentstroke}%
\pgfsetstrokeopacity{0.597366}%
\pgfsetdash{}{0pt}%
\pgfpathmoveto{\pgfqpoint{2.870976in}{2.621527in}}%
\pgfpathcurveto{\pgfqpoint{2.879212in}{2.621527in}}{\pgfqpoint{2.887112in}{2.624799in}}{\pgfqpoint{2.892936in}{2.630623in}}%
\pgfpathcurveto{\pgfqpoint{2.898760in}{2.636447in}}{\pgfqpoint{2.902033in}{2.644347in}}{\pgfqpoint{2.902033in}{2.652584in}}%
\pgfpathcurveto{\pgfqpoint{2.902033in}{2.660820in}}{\pgfqpoint{2.898760in}{2.668720in}}{\pgfqpoint{2.892936in}{2.674544in}}%
\pgfpathcurveto{\pgfqpoint{2.887112in}{2.680368in}}{\pgfqpoint{2.879212in}{2.683640in}}{\pgfqpoint{2.870976in}{2.683640in}}%
\pgfpathcurveto{\pgfqpoint{2.862740in}{2.683640in}}{\pgfqpoint{2.854840in}{2.680368in}}{\pgfqpoint{2.849016in}{2.674544in}}%
\pgfpathcurveto{\pgfqpoint{2.843192in}{2.668720in}}{\pgfqpoint{2.839920in}{2.660820in}}{\pgfqpoint{2.839920in}{2.652584in}}%
\pgfpathcurveto{\pgfqpoint{2.839920in}{2.644347in}}{\pgfqpoint{2.843192in}{2.636447in}}{\pgfqpoint{2.849016in}{2.630623in}}%
\pgfpathcurveto{\pgfqpoint{2.854840in}{2.624799in}}{\pgfqpoint{2.862740in}{2.621527in}}{\pgfqpoint{2.870976in}{2.621527in}}%
\pgfpathclose%
\pgfusepath{stroke,fill}%
\end{pgfscope}%
\begin{pgfscope}%
\pgfpathrectangle{\pgfqpoint{0.100000in}{0.212622in}}{\pgfqpoint{3.696000in}{3.696000in}}%
\pgfusepath{clip}%
\pgfsetbuttcap%
\pgfsetroundjoin%
\definecolor{currentfill}{rgb}{0.121569,0.466667,0.705882}%
\pgfsetfillcolor{currentfill}%
\pgfsetfillopacity{0.598399}%
\pgfsetlinewidth{1.003750pt}%
\definecolor{currentstroke}{rgb}{0.121569,0.466667,0.705882}%
\pgfsetstrokecolor{currentstroke}%
\pgfsetstrokeopacity{0.598399}%
\pgfsetdash{}{0pt}%
\pgfpathmoveto{\pgfqpoint{2.919973in}{2.656859in}}%
\pgfpathcurveto{\pgfqpoint{2.928210in}{2.656859in}}{\pgfqpoint{2.936110in}{2.660132in}}{\pgfqpoint{2.941934in}{2.665956in}}%
\pgfpathcurveto{\pgfqpoint{2.947758in}{2.671780in}}{\pgfqpoint{2.951030in}{2.679680in}}{\pgfqpoint{2.951030in}{2.687916in}}%
\pgfpathcurveto{\pgfqpoint{2.951030in}{2.696152in}}{\pgfqpoint{2.947758in}{2.704052in}}{\pgfqpoint{2.941934in}{2.709876in}}%
\pgfpathcurveto{\pgfqpoint{2.936110in}{2.715700in}}{\pgfqpoint{2.928210in}{2.718972in}}{\pgfqpoint{2.919973in}{2.718972in}}%
\pgfpathcurveto{\pgfqpoint{2.911737in}{2.718972in}}{\pgfqpoint{2.903837in}{2.715700in}}{\pgfqpoint{2.898013in}{2.709876in}}%
\pgfpathcurveto{\pgfqpoint{2.892189in}{2.704052in}}{\pgfqpoint{2.888917in}{2.696152in}}{\pgfqpoint{2.888917in}{2.687916in}}%
\pgfpathcurveto{\pgfqpoint{2.888917in}{2.679680in}}{\pgfqpoint{2.892189in}{2.671780in}}{\pgfqpoint{2.898013in}{2.665956in}}%
\pgfpathcurveto{\pgfqpoint{2.903837in}{2.660132in}}{\pgfqpoint{2.911737in}{2.656859in}}{\pgfqpoint{2.919973in}{2.656859in}}%
\pgfpathclose%
\pgfusepath{stroke,fill}%
\end{pgfscope}%
\begin{pgfscope}%
\pgfpathrectangle{\pgfqpoint{0.100000in}{0.212622in}}{\pgfqpoint{3.696000in}{3.696000in}}%
\pgfusepath{clip}%
\pgfsetbuttcap%
\pgfsetroundjoin%
\definecolor{currentfill}{rgb}{0.121569,0.466667,0.705882}%
\pgfsetfillcolor{currentfill}%
\pgfsetfillopacity{0.608983}%
\pgfsetlinewidth{1.003750pt}%
\definecolor{currentstroke}{rgb}{0.121569,0.466667,0.705882}%
\pgfsetstrokecolor{currentstroke}%
\pgfsetstrokeopacity{0.608983}%
\pgfsetdash{}{0pt}%
\pgfpathmoveto{\pgfqpoint{2.934958in}{2.664156in}}%
\pgfpathcurveto{\pgfqpoint{2.943194in}{2.664156in}}{\pgfqpoint{2.951094in}{2.667428in}}{\pgfqpoint{2.956918in}{2.673252in}}%
\pgfpathcurveto{\pgfqpoint{2.962742in}{2.679076in}}{\pgfqpoint{2.966014in}{2.686976in}}{\pgfqpoint{2.966014in}{2.695212in}}%
\pgfpathcurveto{\pgfqpoint{2.966014in}{2.703449in}}{\pgfqpoint{2.962742in}{2.711349in}}{\pgfqpoint{2.956918in}{2.717173in}}%
\pgfpathcurveto{\pgfqpoint{2.951094in}{2.722996in}}{\pgfqpoint{2.943194in}{2.726269in}}{\pgfqpoint{2.934958in}{2.726269in}}%
\pgfpathcurveto{\pgfqpoint{2.926721in}{2.726269in}}{\pgfqpoint{2.918821in}{2.722996in}}{\pgfqpoint{2.912997in}{2.717173in}}%
\pgfpathcurveto{\pgfqpoint{2.907173in}{2.711349in}}{\pgfqpoint{2.903901in}{2.703449in}}{\pgfqpoint{2.903901in}{2.695212in}}%
\pgfpathcurveto{\pgfqpoint{2.903901in}{2.686976in}}{\pgfqpoint{2.907173in}{2.679076in}}{\pgfqpoint{2.912997in}{2.673252in}}%
\pgfpathcurveto{\pgfqpoint{2.918821in}{2.667428in}}{\pgfqpoint{2.926721in}{2.664156in}}{\pgfqpoint{2.934958in}{2.664156in}}%
\pgfpathclose%
\pgfusepath{stroke,fill}%
\end{pgfscope}%
\begin{pgfscope}%
\pgfpathrectangle{\pgfqpoint{0.100000in}{0.212622in}}{\pgfqpoint{3.696000in}{3.696000in}}%
\pgfusepath{clip}%
\pgfsetbuttcap%
\pgfsetroundjoin%
\definecolor{currentfill}{rgb}{0.121569,0.466667,0.705882}%
\pgfsetfillcolor{currentfill}%
\pgfsetfillopacity{0.612192}%
\pgfsetlinewidth{1.003750pt}%
\definecolor{currentstroke}{rgb}{0.121569,0.466667,0.705882}%
\pgfsetstrokecolor{currentstroke}%
\pgfsetstrokeopacity{0.612192}%
\pgfsetdash{}{0pt}%
\pgfpathmoveto{\pgfqpoint{2.902982in}{2.634274in}}%
\pgfpathcurveto{\pgfqpoint{2.911218in}{2.634274in}}{\pgfqpoint{2.919118in}{2.637546in}}{\pgfqpoint{2.924942in}{2.643370in}}%
\pgfpathcurveto{\pgfqpoint{2.930766in}{2.649194in}}{\pgfqpoint{2.934038in}{2.657094in}}{\pgfqpoint{2.934038in}{2.665330in}}%
\pgfpathcurveto{\pgfqpoint{2.934038in}{2.673567in}}{\pgfqpoint{2.930766in}{2.681467in}}{\pgfqpoint{2.924942in}{2.687291in}}%
\pgfpathcurveto{\pgfqpoint{2.919118in}{2.693115in}}{\pgfqpoint{2.911218in}{2.696387in}}{\pgfqpoint{2.902982in}{2.696387in}}%
\pgfpathcurveto{\pgfqpoint{2.894745in}{2.696387in}}{\pgfqpoint{2.886845in}{2.693115in}}{\pgfqpoint{2.881021in}{2.687291in}}%
\pgfpathcurveto{\pgfqpoint{2.875198in}{2.681467in}}{\pgfqpoint{2.871925in}{2.673567in}}{\pgfqpoint{2.871925in}{2.665330in}}%
\pgfpathcurveto{\pgfqpoint{2.871925in}{2.657094in}}{\pgfqpoint{2.875198in}{2.649194in}}{\pgfqpoint{2.881021in}{2.643370in}}%
\pgfpathcurveto{\pgfqpoint{2.886845in}{2.637546in}}{\pgfqpoint{2.894745in}{2.634274in}}{\pgfqpoint{2.902982in}{2.634274in}}%
\pgfpathclose%
\pgfusepath{stroke,fill}%
\end{pgfscope}%
\begin{pgfscope}%
\pgfpathrectangle{\pgfqpoint{0.100000in}{0.212622in}}{\pgfqpoint{3.696000in}{3.696000in}}%
\pgfusepath{clip}%
\pgfsetbuttcap%
\pgfsetroundjoin%
\definecolor{currentfill}{rgb}{0.121569,0.466667,0.705882}%
\pgfsetfillcolor{currentfill}%
\pgfsetfillopacity{0.620520}%
\pgfsetlinewidth{1.003750pt}%
\definecolor{currentstroke}{rgb}{0.121569,0.466667,0.705882}%
\pgfsetstrokecolor{currentstroke}%
\pgfsetstrokeopacity{0.620520}%
\pgfsetdash{}{0pt}%
\pgfpathmoveto{\pgfqpoint{2.927187in}{2.649148in}}%
\pgfpathcurveto{\pgfqpoint{2.935423in}{2.649148in}}{\pgfqpoint{2.943323in}{2.652421in}}{\pgfqpoint{2.949147in}{2.658245in}}%
\pgfpathcurveto{\pgfqpoint{2.954971in}{2.664068in}}{\pgfqpoint{2.958243in}{2.671969in}}{\pgfqpoint{2.958243in}{2.680205in}}%
\pgfpathcurveto{\pgfqpoint{2.958243in}{2.688441in}}{\pgfqpoint{2.954971in}{2.696341in}}{\pgfqpoint{2.949147in}{2.702165in}}%
\pgfpathcurveto{\pgfqpoint{2.943323in}{2.707989in}}{\pgfqpoint{2.935423in}{2.711261in}}{\pgfqpoint{2.927187in}{2.711261in}}%
\pgfpathcurveto{\pgfqpoint{2.918950in}{2.711261in}}{\pgfqpoint{2.911050in}{2.707989in}}{\pgfqpoint{2.905226in}{2.702165in}}%
\pgfpathcurveto{\pgfqpoint{2.899402in}{2.696341in}}{\pgfqpoint{2.896130in}{2.688441in}}{\pgfqpoint{2.896130in}{2.680205in}}%
\pgfpathcurveto{\pgfqpoint{2.896130in}{2.671969in}}{\pgfqpoint{2.899402in}{2.664068in}}{\pgfqpoint{2.905226in}{2.658245in}}%
\pgfpathcurveto{\pgfqpoint{2.911050in}{2.652421in}}{\pgfqpoint{2.918950in}{2.649148in}}{\pgfqpoint{2.927187in}{2.649148in}}%
\pgfpathclose%
\pgfusepath{stroke,fill}%
\end{pgfscope}%
\begin{pgfscope}%
\pgfpathrectangle{\pgfqpoint{0.100000in}{0.212622in}}{\pgfqpoint{3.696000in}{3.696000in}}%
\pgfusepath{clip}%
\pgfsetbuttcap%
\pgfsetroundjoin%
\definecolor{currentfill}{rgb}{0.121569,0.466667,0.705882}%
\pgfsetfillcolor{currentfill}%
\pgfsetfillopacity{0.621461}%
\pgfsetlinewidth{1.003750pt}%
\definecolor{currentstroke}{rgb}{0.121569,0.466667,0.705882}%
\pgfsetstrokecolor{currentstroke}%
\pgfsetstrokeopacity{0.621461}%
\pgfsetdash{}{0pt}%
\pgfpathmoveto{\pgfqpoint{2.951266in}{2.670080in}}%
\pgfpathcurveto{\pgfqpoint{2.959502in}{2.670080in}}{\pgfqpoint{2.967402in}{2.673352in}}{\pgfqpoint{2.973226in}{2.679176in}}%
\pgfpathcurveto{\pgfqpoint{2.979050in}{2.685000in}}{\pgfqpoint{2.982323in}{2.692900in}}{\pgfqpoint{2.982323in}{2.701137in}}%
\pgfpathcurveto{\pgfqpoint{2.982323in}{2.709373in}}{\pgfqpoint{2.979050in}{2.717273in}}{\pgfqpoint{2.973226in}{2.723097in}}%
\pgfpathcurveto{\pgfqpoint{2.967402in}{2.728921in}}{\pgfqpoint{2.959502in}{2.732193in}}{\pgfqpoint{2.951266in}{2.732193in}}%
\pgfpathcurveto{\pgfqpoint{2.943030in}{2.732193in}}{\pgfqpoint{2.935130in}{2.728921in}}{\pgfqpoint{2.929306in}{2.723097in}}%
\pgfpathcurveto{\pgfqpoint{2.923482in}{2.717273in}}{\pgfqpoint{2.920210in}{2.709373in}}{\pgfqpoint{2.920210in}{2.701137in}}%
\pgfpathcurveto{\pgfqpoint{2.920210in}{2.692900in}}{\pgfqpoint{2.923482in}{2.685000in}}{\pgfqpoint{2.929306in}{2.679176in}}%
\pgfpathcurveto{\pgfqpoint{2.935130in}{2.673352in}}{\pgfqpoint{2.943030in}{2.670080in}}{\pgfqpoint{2.951266in}{2.670080in}}%
\pgfpathclose%
\pgfusepath{stroke,fill}%
\end{pgfscope}%
\begin{pgfscope}%
\pgfpathrectangle{\pgfqpoint{0.100000in}{0.212622in}}{\pgfqpoint{3.696000in}{3.696000in}}%
\pgfusepath{clip}%
\pgfsetbuttcap%
\pgfsetroundjoin%
\definecolor{currentfill}{rgb}{0.121569,0.466667,0.705882}%
\pgfsetfillcolor{currentfill}%
\pgfsetfillopacity{0.636519}%
\pgfsetlinewidth{1.003750pt}%
\definecolor{currentstroke}{rgb}{0.121569,0.466667,0.705882}%
\pgfsetstrokecolor{currentstroke}%
\pgfsetstrokeopacity{0.636519}%
\pgfsetdash{}{0pt}%
\pgfpathmoveto{\pgfqpoint{2.966396in}{2.672794in}}%
\pgfpathcurveto{\pgfqpoint{2.974632in}{2.672794in}}{\pgfqpoint{2.982532in}{2.676066in}}{\pgfqpoint{2.988356in}{2.681890in}}%
\pgfpathcurveto{\pgfqpoint{2.994180in}{2.687714in}}{\pgfqpoint{2.997452in}{2.695614in}}{\pgfqpoint{2.997452in}{2.703850in}}%
\pgfpathcurveto{\pgfqpoint{2.997452in}{2.712087in}}{\pgfqpoint{2.994180in}{2.719987in}}{\pgfqpoint{2.988356in}{2.725811in}}%
\pgfpathcurveto{\pgfqpoint{2.982532in}{2.731634in}}{\pgfqpoint{2.974632in}{2.734907in}}{\pgfqpoint{2.966396in}{2.734907in}}%
\pgfpathcurveto{\pgfqpoint{2.958159in}{2.734907in}}{\pgfqpoint{2.950259in}{2.731634in}}{\pgfqpoint{2.944435in}{2.725811in}}%
\pgfpathcurveto{\pgfqpoint{2.938612in}{2.719987in}}{\pgfqpoint{2.935339in}{2.712087in}}{\pgfqpoint{2.935339in}{2.703850in}}%
\pgfpathcurveto{\pgfqpoint{2.935339in}{2.695614in}}{\pgfqpoint{2.938612in}{2.687714in}}{\pgfqpoint{2.944435in}{2.681890in}}%
\pgfpathcurveto{\pgfqpoint{2.950259in}{2.676066in}}{\pgfqpoint{2.958159in}{2.672794in}}{\pgfqpoint{2.966396in}{2.672794in}}%
\pgfpathclose%
\pgfusepath{stroke,fill}%
\end{pgfscope}%
\begin{pgfscope}%
\pgfpathrectangle{\pgfqpoint{0.100000in}{0.212622in}}{\pgfqpoint{3.696000in}{3.696000in}}%
\pgfusepath{clip}%
\pgfsetbuttcap%
\pgfsetroundjoin%
\definecolor{currentfill}{rgb}{0.121569,0.466667,0.705882}%
\pgfsetfillcolor{currentfill}%
\pgfsetfillopacity{0.638850}%
\pgfsetlinewidth{1.003750pt}%
\definecolor{currentstroke}{rgb}{0.121569,0.466667,0.705882}%
\pgfsetstrokecolor{currentstroke}%
\pgfsetstrokeopacity{0.638850}%
\pgfsetdash{}{0pt}%
\pgfpathmoveto{\pgfqpoint{2.930199in}{2.639474in}}%
\pgfpathcurveto{\pgfqpoint{2.938435in}{2.639474in}}{\pgfqpoint{2.946335in}{2.642746in}}{\pgfqpoint{2.952159in}{2.648570in}}%
\pgfpathcurveto{\pgfqpoint{2.957983in}{2.654394in}}{\pgfqpoint{2.961255in}{2.662294in}}{\pgfqpoint{2.961255in}{2.670530in}}%
\pgfpathcurveto{\pgfqpoint{2.961255in}{2.678767in}}{\pgfqpoint{2.957983in}{2.686667in}}{\pgfqpoint{2.952159in}{2.692491in}}%
\pgfpathcurveto{\pgfqpoint{2.946335in}{2.698314in}}{\pgfqpoint{2.938435in}{2.701587in}}{\pgfqpoint{2.930199in}{2.701587in}}%
\pgfpathcurveto{\pgfqpoint{2.921963in}{2.701587in}}{\pgfqpoint{2.914062in}{2.698314in}}{\pgfqpoint{2.908239in}{2.692491in}}%
\pgfpathcurveto{\pgfqpoint{2.902415in}{2.686667in}}{\pgfqpoint{2.899142in}{2.678767in}}{\pgfqpoint{2.899142in}{2.670530in}}%
\pgfpathcurveto{\pgfqpoint{2.899142in}{2.662294in}}{\pgfqpoint{2.902415in}{2.654394in}}{\pgfqpoint{2.908239in}{2.648570in}}%
\pgfpathcurveto{\pgfqpoint{2.914062in}{2.642746in}}{\pgfqpoint{2.921963in}{2.639474in}}{\pgfqpoint{2.930199in}{2.639474in}}%
\pgfpathclose%
\pgfusepath{stroke,fill}%
\end{pgfscope}%
\begin{pgfscope}%
\pgfpathrectangle{\pgfqpoint{0.100000in}{0.212622in}}{\pgfqpoint{3.696000in}{3.696000in}}%
\pgfusepath{clip}%
\pgfsetbuttcap%
\pgfsetroundjoin%
\definecolor{currentfill}{rgb}{0.121569,0.466667,0.705882}%
\pgfsetfillcolor{currentfill}%
\pgfsetfillopacity{0.649920}%
\pgfsetlinewidth{1.003750pt}%
\definecolor{currentstroke}{rgb}{0.121569,0.466667,0.705882}%
\pgfsetstrokecolor{currentstroke}%
\pgfsetstrokeopacity{0.649920}%
\pgfsetdash{}{0pt}%
\pgfpathmoveto{\pgfqpoint{2.957668in}{2.650049in}}%
\pgfpathcurveto{\pgfqpoint{2.965904in}{2.650049in}}{\pgfqpoint{2.973804in}{2.653322in}}{\pgfqpoint{2.979628in}{2.659146in}}%
\pgfpathcurveto{\pgfqpoint{2.985452in}{2.664969in}}{\pgfqpoint{2.988724in}{2.672869in}}{\pgfqpoint{2.988724in}{2.681106in}}%
\pgfpathcurveto{\pgfqpoint{2.988724in}{2.689342in}}{\pgfqpoint{2.985452in}{2.697242in}}{\pgfqpoint{2.979628in}{2.703066in}}%
\pgfpathcurveto{\pgfqpoint{2.973804in}{2.708890in}}{\pgfqpoint{2.965904in}{2.712162in}}{\pgfqpoint{2.957668in}{2.712162in}}%
\pgfpathcurveto{\pgfqpoint{2.949431in}{2.712162in}}{\pgfqpoint{2.941531in}{2.708890in}}{\pgfqpoint{2.935707in}{2.703066in}}%
\pgfpathcurveto{\pgfqpoint{2.929883in}{2.697242in}}{\pgfqpoint{2.926611in}{2.689342in}}{\pgfqpoint{2.926611in}{2.681106in}}%
\pgfpathcurveto{\pgfqpoint{2.926611in}{2.672869in}}{\pgfqpoint{2.929883in}{2.664969in}}{\pgfqpoint{2.935707in}{2.659146in}}%
\pgfpathcurveto{\pgfqpoint{2.941531in}{2.653322in}}{\pgfqpoint{2.949431in}{2.650049in}}{\pgfqpoint{2.957668in}{2.650049in}}%
\pgfpathclose%
\pgfusepath{stroke,fill}%
\end{pgfscope}%
\begin{pgfscope}%
\pgfpathrectangle{\pgfqpoint{0.100000in}{0.212622in}}{\pgfqpoint{3.696000in}{3.696000in}}%
\pgfusepath{clip}%
\pgfsetbuttcap%
\pgfsetroundjoin%
\definecolor{currentfill}{rgb}{0.121569,0.466667,0.705882}%
\pgfsetfillcolor{currentfill}%
\pgfsetfillopacity{0.650008}%
\pgfsetlinewidth{1.003750pt}%
\definecolor{currentstroke}{rgb}{0.121569,0.466667,0.705882}%
\pgfsetstrokecolor{currentstroke}%
\pgfsetstrokeopacity{0.650008}%
\pgfsetdash{}{0pt}%
\pgfpathmoveto{\pgfqpoint{3.194686in}{2.789385in}}%
\pgfpathcurveto{\pgfqpoint{3.202922in}{2.789385in}}{\pgfqpoint{3.210822in}{2.792657in}}{\pgfqpoint{3.216646in}{2.798481in}}%
\pgfpathcurveto{\pgfqpoint{3.222470in}{2.804305in}}{\pgfqpoint{3.225742in}{2.812205in}}{\pgfqpoint{3.225742in}{2.820441in}}%
\pgfpathcurveto{\pgfqpoint{3.225742in}{2.828678in}}{\pgfqpoint{3.222470in}{2.836578in}}{\pgfqpoint{3.216646in}{2.842402in}}%
\pgfpathcurveto{\pgfqpoint{3.210822in}{2.848226in}}{\pgfqpoint{3.202922in}{2.851498in}}{\pgfqpoint{3.194686in}{2.851498in}}%
\pgfpathcurveto{\pgfqpoint{3.186450in}{2.851498in}}{\pgfqpoint{3.178550in}{2.848226in}}{\pgfqpoint{3.172726in}{2.842402in}}%
\pgfpathcurveto{\pgfqpoint{3.166902in}{2.836578in}}{\pgfqpoint{3.163629in}{2.828678in}}{\pgfqpoint{3.163629in}{2.820441in}}%
\pgfpathcurveto{\pgfqpoint{3.163629in}{2.812205in}}{\pgfqpoint{3.166902in}{2.804305in}}{\pgfqpoint{3.172726in}{2.798481in}}%
\pgfpathcurveto{\pgfqpoint{3.178550in}{2.792657in}}{\pgfqpoint{3.186450in}{2.789385in}}{\pgfqpoint{3.194686in}{2.789385in}}%
\pgfpathclose%
\pgfusepath{stroke,fill}%
\end{pgfscope}%
\begin{pgfscope}%
\pgfpathrectangle{\pgfqpoint{0.100000in}{0.212622in}}{\pgfqpoint{3.696000in}{3.696000in}}%
\pgfusepath{clip}%
\pgfsetbuttcap%
\pgfsetroundjoin%
\definecolor{currentfill}{rgb}{0.121569,0.466667,0.705882}%
\pgfsetfillcolor{currentfill}%
\pgfsetfillopacity{0.652222}%
\pgfsetlinewidth{1.003750pt}%
\definecolor{currentstroke}{rgb}{0.121569,0.466667,0.705882}%
\pgfsetstrokecolor{currentstroke}%
\pgfsetstrokeopacity{0.652222}%
\pgfsetdash{}{0pt}%
\pgfpathmoveto{\pgfqpoint{3.225686in}{2.808496in}}%
\pgfpathcurveto{\pgfqpoint{3.233923in}{2.808496in}}{\pgfqpoint{3.241823in}{2.811768in}}{\pgfqpoint{3.247647in}{2.817592in}}%
\pgfpathcurveto{\pgfqpoint{3.253471in}{2.823416in}}{\pgfqpoint{3.256743in}{2.831316in}}{\pgfqpoint{3.256743in}{2.839552in}}%
\pgfpathcurveto{\pgfqpoint{3.256743in}{2.847789in}}{\pgfqpoint{3.253471in}{2.855689in}}{\pgfqpoint{3.247647in}{2.861513in}}%
\pgfpathcurveto{\pgfqpoint{3.241823in}{2.867337in}}{\pgfqpoint{3.233923in}{2.870609in}}{\pgfqpoint{3.225686in}{2.870609in}}%
\pgfpathcurveto{\pgfqpoint{3.217450in}{2.870609in}}{\pgfqpoint{3.209550in}{2.867337in}}{\pgfqpoint{3.203726in}{2.861513in}}%
\pgfpathcurveto{\pgfqpoint{3.197902in}{2.855689in}}{\pgfqpoint{3.194630in}{2.847789in}}{\pgfqpoint{3.194630in}{2.839552in}}%
\pgfpathcurveto{\pgfqpoint{3.194630in}{2.831316in}}{\pgfqpoint{3.197902in}{2.823416in}}{\pgfqpoint{3.203726in}{2.817592in}}%
\pgfpathcurveto{\pgfqpoint{3.209550in}{2.811768in}}{\pgfqpoint{3.217450in}{2.808496in}}{\pgfqpoint{3.225686in}{2.808496in}}%
\pgfpathclose%
\pgfusepath{stroke,fill}%
\end{pgfscope}%
\begin{pgfscope}%
\pgfpathrectangle{\pgfqpoint{0.100000in}{0.212622in}}{\pgfqpoint{3.696000in}{3.696000in}}%
\pgfusepath{clip}%
\pgfsetbuttcap%
\pgfsetroundjoin%
\definecolor{currentfill}{rgb}{0.121569,0.466667,0.705882}%
\pgfsetfillcolor{currentfill}%
\pgfsetfillopacity{0.654789}%
\pgfsetlinewidth{1.003750pt}%
\definecolor{currentstroke}{rgb}{0.121569,0.466667,0.705882}%
\pgfsetstrokecolor{currentstroke}%
\pgfsetstrokeopacity{0.654789}%
\pgfsetdash{}{0pt}%
\pgfpathmoveto{\pgfqpoint{3.203946in}{2.796857in}}%
\pgfpathcurveto{\pgfqpoint{3.212183in}{2.796857in}}{\pgfqpoint{3.220083in}{2.800129in}}{\pgfqpoint{3.225907in}{2.805953in}}%
\pgfpathcurveto{\pgfqpoint{3.231730in}{2.811777in}}{\pgfqpoint{3.235003in}{2.819677in}}{\pgfqpoint{3.235003in}{2.827913in}}%
\pgfpathcurveto{\pgfqpoint{3.235003in}{2.836150in}}{\pgfqpoint{3.231730in}{2.844050in}}{\pgfqpoint{3.225907in}{2.849874in}}%
\pgfpathcurveto{\pgfqpoint{3.220083in}{2.855697in}}{\pgfqpoint{3.212183in}{2.858970in}}{\pgfqpoint{3.203946in}{2.858970in}}%
\pgfpathcurveto{\pgfqpoint{3.195710in}{2.858970in}}{\pgfqpoint{3.187810in}{2.855697in}}{\pgfqpoint{3.181986in}{2.849874in}}%
\pgfpathcurveto{\pgfqpoint{3.176162in}{2.844050in}}{\pgfqpoint{3.172890in}{2.836150in}}{\pgfqpoint{3.172890in}{2.827913in}}%
\pgfpathcurveto{\pgfqpoint{3.172890in}{2.819677in}}{\pgfqpoint{3.176162in}{2.811777in}}{\pgfqpoint{3.181986in}{2.805953in}}%
\pgfpathcurveto{\pgfqpoint{3.187810in}{2.800129in}}{\pgfqpoint{3.195710in}{2.796857in}}{\pgfqpoint{3.203946in}{2.796857in}}%
\pgfpathclose%
\pgfusepath{stroke,fill}%
\end{pgfscope}%
\begin{pgfscope}%
\pgfpathrectangle{\pgfqpoint{0.100000in}{0.212622in}}{\pgfqpoint{3.696000in}{3.696000in}}%
\pgfusepath{clip}%
\pgfsetbuttcap%
\pgfsetroundjoin%
\definecolor{currentfill}{rgb}{0.121569,0.466667,0.705882}%
\pgfsetfillcolor{currentfill}%
\pgfsetfillopacity{0.655251}%
\pgfsetlinewidth{1.003750pt}%
\definecolor{currentstroke}{rgb}{0.121569,0.466667,0.705882}%
\pgfsetstrokecolor{currentstroke}%
\pgfsetstrokeopacity{0.655251}%
\pgfsetdash{}{0pt}%
\pgfpathmoveto{\pgfqpoint{3.224557in}{2.804950in}}%
\pgfpathcurveto{\pgfqpoint{3.232794in}{2.804950in}}{\pgfqpoint{3.240694in}{2.808222in}}{\pgfqpoint{3.246518in}{2.814046in}}%
\pgfpathcurveto{\pgfqpoint{3.252342in}{2.819870in}}{\pgfqpoint{3.255614in}{2.827770in}}{\pgfqpoint{3.255614in}{2.836007in}}%
\pgfpathcurveto{\pgfqpoint{3.255614in}{2.844243in}}{\pgfqpoint{3.252342in}{2.852143in}}{\pgfqpoint{3.246518in}{2.857967in}}%
\pgfpathcurveto{\pgfqpoint{3.240694in}{2.863791in}}{\pgfqpoint{3.232794in}{2.867063in}}{\pgfqpoint{3.224557in}{2.867063in}}%
\pgfpathcurveto{\pgfqpoint{3.216321in}{2.867063in}}{\pgfqpoint{3.208421in}{2.863791in}}{\pgfqpoint{3.202597in}{2.857967in}}%
\pgfpathcurveto{\pgfqpoint{3.196773in}{2.852143in}}{\pgfqpoint{3.193501in}{2.844243in}}{\pgfqpoint{3.193501in}{2.836007in}}%
\pgfpathcurveto{\pgfqpoint{3.193501in}{2.827770in}}{\pgfqpoint{3.196773in}{2.819870in}}{\pgfqpoint{3.202597in}{2.814046in}}%
\pgfpathcurveto{\pgfqpoint{3.208421in}{2.808222in}}{\pgfqpoint{3.216321in}{2.804950in}}{\pgfqpoint{3.224557in}{2.804950in}}%
\pgfpathclose%
\pgfusepath{stroke,fill}%
\end{pgfscope}%
\begin{pgfscope}%
\pgfpathrectangle{\pgfqpoint{0.100000in}{0.212622in}}{\pgfqpoint{3.696000in}{3.696000in}}%
\pgfusepath{clip}%
\pgfsetbuttcap%
\pgfsetroundjoin%
\definecolor{currentfill}{rgb}{0.121569,0.466667,0.705882}%
\pgfsetfillcolor{currentfill}%
\pgfsetfillopacity{0.656449}%
\pgfsetlinewidth{1.003750pt}%
\definecolor{currentstroke}{rgb}{0.121569,0.466667,0.705882}%
\pgfsetstrokecolor{currentstroke}%
\pgfsetstrokeopacity{0.656449}%
\pgfsetdash{}{0pt}%
\pgfpathmoveto{\pgfqpoint{3.159030in}{2.760606in}}%
\pgfpathcurveto{\pgfqpoint{3.167267in}{2.760606in}}{\pgfqpoint{3.175167in}{2.763878in}}{\pgfqpoint{3.180991in}{2.769702in}}%
\pgfpathcurveto{\pgfqpoint{3.186815in}{2.775526in}}{\pgfqpoint{3.190087in}{2.783426in}}{\pgfqpoint{3.190087in}{2.791662in}}%
\pgfpathcurveto{\pgfqpoint{3.190087in}{2.799899in}}{\pgfqpoint{3.186815in}{2.807799in}}{\pgfqpoint{3.180991in}{2.813623in}}%
\pgfpathcurveto{\pgfqpoint{3.175167in}{2.819447in}}{\pgfqpoint{3.167267in}{2.822719in}}{\pgfqpoint{3.159030in}{2.822719in}}%
\pgfpathcurveto{\pgfqpoint{3.150794in}{2.822719in}}{\pgfqpoint{3.142894in}{2.819447in}}{\pgfqpoint{3.137070in}{2.813623in}}%
\pgfpathcurveto{\pgfqpoint{3.131246in}{2.807799in}}{\pgfqpoint{3.127974in}{2.799899in}}{\pgfqpoint{3.127974in}{2.791662in}}%
\pgfpathcurveto{\pgfqpoint{3.127974in}{2.783426in}}{\pgfqpoint{3.131246in}{2.775526in}}{\pgfqpoint{3.137070in}{2.769702in}}%
\pgfpathcurveto{\pgfqpoint{3.142894in}{2.763878in}}{\pgfqpoint{3.150794in}{2.760606in}}{\pgfqpoint{3.159030in}{2.760606in}}%
\pgfpathclose%
\pgfusepath{stroke,fill}%
\end{pgfscope}%
\begin{pgfscope}%
\pgfpathrectangle{\pgfqpoint{0.100000in}{0.212622in}}{\pgfqpoint{3.696000in}{3.696000in}}%
\pgfusepath{clip}%
\pgfsetbuttcap%
\pgfsetroundjoin%
\definecolor{currentfill}{rgb}{0.121569,0.466667,0.705882}%
\pgfsetfillcolor{currentfill}%
\pgfsetfillopacity{0.659030}%
\pgfsetlinewidth{1.003750pt}%
\definecolor{currentstroke}{rgb}{0.121569,0.466667,0.705882}%
\pgfsetstrokecolor{currentstroke}%
\pgfsetstrokeopacity{0.659030}%
\pgfsetdash{}{0pt}%
\pgfpathmoveto{\pgfqpoint{3.203133in}{2.790839in}}%
\pgfpathcurveto{\pgfqpoint{3.211369in}{2.790839in}}{\pgfqpoint{3.219269in}{2.794112in}}{\pgfqpoint{3.225093in}{2.799936in}}%
\pgfpathcurveto{\pgfqpoint{3.230917in}{2.805760in}}{\pgfqpoint{3.234189in}{2.813660in}}{\pgfqpoint{3.234189in}{2.821896in}}%
\pgfpathcurveto{\pgfqpoint{3.234189in}{2.830132in}}{\pgfqpoint{3.230917in}{2.838032in}}{\pgfqpoint{3.225093in}{2.843856in}}%
\pgfpathcurveto{\pgfqpoint{3.219269in}{2.849680in}}{\pgfqpoint{3.211369in}{2.852952in}}{\pgfqpoint{3.203133in}{2.852952in}}%
\pgfpathcurveto{\pgfqpoint{3.194896in}{2.852952in}}{\pgfqpoint{3.186996in}{2.849680in}}{\pgfqpoint{3.181172in}{2.843856in}}%
\pgfpathcurveto{\pgfqpoint{3.175348in}{2.838032in}}{\pgfqpoint{3.172076in}{2.830132in}}{\pgfqpoint{3.172076in}{2.821896in}}%
\pgfpathcurveto{\pgfqpoint{3.172076in}{2.813660in}}{\pgfqpoint{3.175348in}{2.805760in}}{\pgfqpoint{3.181172in}{2.799936in}}%
\pgfpathcurveto{\pgfqpoint{3.186996in}{2.794112in}}{\pgfqpoint{3.194896in}{2.790839in}}{\pgfqpoint{3.203133in}{2.790839in}}%
\pgfpathclose%
\pgfusepath{stroke,fill}%
\end{pgfscope}%
\begin{pgfscope}%
\pgfpathrectangle{\pgfqpoint{0.100000in}{0.212622in}}{\pgfqpoint{3.696000in}{3.696000in}}%
\pgfusepath{clip}%
\pgfsetbuttcap%
\pgfsetroundjoin%
\definecolor{currentfill}{rgb}{0.121569,0.466667,0.705882}%
\pgfsetfillcolor{currentfill}%
\pgfsetfillopacity{0.660875}%
\pgfsetlinewidth{1.003750pt}%
\definecolor{currentstroke}{rgb}{0.121569,0.466667,0.705882}%
\pgfsetstrokecolor{currentstroke}%
\pgfsetstrokeopacity{0.660875}%
\pgfsetdash{}{0pt}%
\pgfpathmoveto{\pgfqpoint{2.958224in}{2.640895in}}%
\pgfpathcurveto{\pgfqpoint{2.966460in}{2.640895in}}{\pgfqpoint{2.974360in}{2.644168in}}{\pgfqpoint{2.980184in}{2.649992in}}%
\pgfpathcurveto{\pgfqpoint{2.986008in}{2.655816in}}{\pgfqpoint{2.989280in}{2.663716in}}{\pgfqpoint{2.989280in}{2.671952in}}%
\pgfpathcurveto{\pgfqpoint{2.989280in}{2.680188in}}{\pgfqpoint{2.986008in}{2.688088in}}{\pgfqpoint{2.980184in}{2.693912in}}%
\pgfpathcurveto{\pgfqpoint{2.974360in}{2.699736in}}{\pgfqpoint{2.966460in}{2.703008in}}{\pgfqpoint{2.958224in}{2.703008in}}%
\pgfpathcurveto{\pgfqpoint{2.949987in}{2.703008in}}{\pgfqpoint{2.942087in}{2.699736in}}{\pgfqpoint{2.936263in}{2.693912in}}%
\pgfpathcurveto{\pgfqpoint{2.930439in}{2.688088in}}{\pgfqpoint{2.927167in}{2.680188in}}{\pgfqpoint{2.927167in}{2.671952in}}%
\pgfpathcurveto{\pgfqpoint{2.927167in}{2.663716in}}{\pgfqpoint{2.930439in}{2.655816in}}{\pgfqpoint{2.936263in}{2.649992in}}%
\pgfpathcurveto{\pgfqpoint{2.942087in}{2.644168in}}{\pgfqpoint{2.949987in}{2.640895in}}{\pgfqpoint{2.958224in}{2.640895in}}%
\pgfpathclose%
\pgfusepath{stroke,fill}%
\end{pgfscope}%
\begin{pgfscope}%
\pgfpathrectangle{\pgfqpoint{0.100000in}{0.212622in}}{\pgfqpoint{3.696000in}{3.696000in}}%
\pgfusepath{clip}%
\pgfsetbuttcap%
\pgfsetroundjoin%
\definecolor{currentfill}{rgb}{0.121569,0.466667,0.705882}%
\pgfsetfillcolor{currentfill}%
\pgfsetfillopacity{0.661623}%
\pgfsetlinewidth{1.003750pt}%
\definecolor{currentstroke}{rgb}{0.121569,0.466667,0.705882}%
\pgfsetstrokecolor{currentstroke}%
\pgfsetstrokeopacity{0.661623}%
\pgfsetdash{}{0pt}%
\pgfpathmoveto{\pgfqpoint{3.018398in}{2.681012in}}%
\pgfpathcurveto{\pgfqpoint{3.026635in}{2.681012in}}{\pgfqpoint{3.034535in}{2.684285in}}{\pgfqpoint{3.040359in}{2.690108in}}%
\pgfpathcurveto{\pgfqpoint{3.046182in}{2.695932in}}{\pgfqpoint{3.049455in}{2.703832in}}{\pgfqpoint{3.049455in}{2.712069in}}%
\pgfpathcurveto{\pgfqpoint{3.049455in}{2.720305in}}{\pgfqpoint{3.046182in}{2.728205in}}{\pgfqpoint{3.040359in}{2.734029in}}%
\pgfpathcurveto{\pgfqpoint{3.034535in}{2.739853in}}{\pgfqpoint{3.026635in}{2.743125in}}{\pgfqpoint{3.018398in}{2.743125in}}%
\pgfpathcurveto{\pgfqpoint{3.010162in}{2.743125in}}{\pgfqpoint{3.002262in}{2.739853in}}{\pgfqpoint{2.996438in}{2.734029in}}%
\pgfpathcurveto{\pgfqpoint{2.990614in}{2.728205in}}{\pgfqpoint{2.987342in}{2.720305in}}{\pgfqpoint{2.987342in}{2.712069in}}%
\pgfpathcurveto{\pgfqpoint{2.987342in}{2.703832in}}{\pgfqpoint{2.990614in}{2.695932in}}{\pgfqpoint{2.996438in}{2.690108in}}%
\pgfpathcurveto{\pgfqpoint{3.002262in}{2.684285in}}{\pgfqpoint{3.010162in}{2.681012in}}{\pgfqpoint{3.018398in}{2.681012in}}%
\pgfpathclose%
\pgfusepath{stroke,fill}%
\end{pgfscope}%
\begin{pgfscope}%
\pgfpathrectangle{\pgfqpoint{0.100000in}{0.212622in}}{\pgfqpoint{3.696000in}{3.696000in}}%
\pgfusepath{clip}%
\pgfsetbuttcap%
\pgfsetroundjoin%
\definecolor{currentfill}{rgb}{0.121569,0.466667,0.705882}%
\pgfsetfillcolor{currentfill}%
\pgfsetfillopacity{0.662664}%
\pgfsetlinewidth{1.003750pt}%
\definecolor{currentstroke}{rgb}{0.121569,0.466667,0.705882}%
\pgfsetstrokecolor{currentstroke}%
\pgfsetstrokeopacity{0.662664}%
\pgfsetdash{}{0pt}%
\pgfpathmoveto{\pgfqpoint{3.211227in}{2.785325in}}%
\pgfpathcurveto{\pgfqpoint{3.219463in}{2.785325in}}{\pgfqpoint{3.227363in}{2.788597in}}{\pgfqpoint{3.233187in}{2.794421in}}%
\pgfpathcurveto{\pgfqpoint{3.239011in}{2.800245in}}{\pgfqpoint{3.242283in}{2.808145in}}{\pgfqpoint{3.242283in}{2.816381in}}%
\pgfpathcurveto{\pgfqpoint{3.242283in}{2.824618in}}{\pgfqpoint{3.239011in}{2.832518in}}{\pgfqpoint{3.233187in}{2.838341in}}%
\pgfpathcurveto{\pgfqpoint{3.227363in}{2.844165in}}{\pgfqpoint{3.219463in}{2.847438in}}{\pgfqpoint{3.211227in}{2.847438in}}%
\pgfpathcurveto{\pgfqpoint{3.202990in}{2.847438in}}{\pgfqpoint{3.195090in}{2.844165in}}{\pgfqpoint{3.189266in}{2.838341in}}%
\pgfpathcurveto{\pgfqpoint{3.183442in}{2.832518in}}{\pgfqpoint{3.180170in}{2.824618in}}{\pgfqpoint{3.180170in}{2.816381in}}%
\pgfpathcurveto{\pgfqpoint{3.180170in}{2.808145in}}{\pgfqpoint{3.183442in}{2.800245in}}{\pgfqpoint{3.189266in}{2.794421in}}%
\pgfpathcurveto{\pgfqpoint{3.195090in}{2.788597in}}{\pgfqpoint{3.202990in}{2.785325in}}{\pgfqpoint{3.211227in}{2.785325in}}%
\pgfpathclose%
\pgfusepath{stroke,fill}%
\end{pgfscope}%
\begin{pgfscope}%
\pgfpathrectangle{\pgfqpoint{0.100000in}{0.212622in}}{\pgfqpoint{3.696000in}{3.696000in}}%
\pgfusepath{clip}%
\pgfsetbuttcap%
\pgfsetroundjoin%
\definecolor{currentfill}{rgb}{0.121569,0.466667,0.705882}%
\pgfsetfillcolor{currentfill}%
\pgfsetfillopacity{0.663110}%
\pgfsetlinewidth{1.003750pt}%
\definecolor{currentstroke}{rgb}{0.121569,0.466667,0.705882}%
\pgfsetstrokecolor{currentstroke}%
\pgfsetstrokeopacity{0.663110}%
\pgfsetdash{}{0pt}%
\pgfpathmoveto{\pgfqpoint{2.982340in}{2.653360in}}%
\pgfpathcurveto{\pgfqpoint{2.990576in}{2.653360in}}{\pgfqpoint{2.998476in}{2.656632in}}{\pgfqpoint{3.004300in}{2.662456in}}%
\pgfpathcurveto{\pgfqpoint{3.010124in}{2.668280in}}{\pgfqpoint{3.013396in}{2.676180in}}{\pgfqpoint{3.013396in}{2.684417in}}%
\pgfpathcurveto{\pgfqpoint{3.013396in}{2.692653in}}{\pgfqpoint{3.010124in}{2.700553in}}{\pgfqpoint{3.004300in}{2.706377in}}%
\pgfpathcurveto{\pgfqpoint{2.998476in}{2.712201in}}{\pgfqpoint{2.990576in}{2.715473in}}{\pgfqpoint{2.982340in}{2.715473in}}%
\pgfpathcurveto{\pgfqpoint{2.974103in}{2.715473in}}{\pgfqpoint{2.966203in}{2.712201in}}{\pgfqpoint{2.960379in}{2.706377in}}%
\pgfpathcurveto{\pgfqpoint{2.954555in}{2.700553in}}{\pgfqpoint{2.951283in}{2.692653in}}{\pgfqpoint{2.951283in}{2.684417in}}%
\pgfpathcurveto{\pgfqpoint{2.951283in}{2.676180in}}{\pgfqpoint{2.954555in}{2.668280in}}{\pgfqpoint{2.960379in}{2.662456in}}%
\pgfpathcurveto{\pgfqpoint{2.966203in}{2.656632in}}{\pgfqpoint{2.974103in}{2.653360in}}{\pgfqpoint{2.982340in}{2.653360in}}%
\pgfpathclose%
\pgfusepath{stroke,fill}%
\end{pgfscope}%
\begin{pgfscope}%
\pgfpathrectangle{\pgfqpoint{0.100000in}{0.212622in}}{\pgfqpoint{3.696000in}{3.696000in}}%
\pgfusepath{clip}%
\pgfsetbuttcap%
\pgfsetroundjoin%
\definecolor{currentfill}{rgb}{0.121569,0.466667,0.705882}%
\pgfsetfillcolor{currentfill}%
\pgfsetfillopacity{0.664885}%
\pgfsetlinewidth{1.003750pt}%
\definecolor{currentstroke}{rgb}{0.121569,0.466667,0.705882}%
\pgfsetstrokecolor{currentstroke}%
\pgfsetstrokeopacity{0.664885}%
\pgfsetdash{}{0pt}%
\pgfpathmoveto{\pgfqpoint{3.209106in}{2.780589in}}%
\pgfpathcurveto{\pgfqpoint{3.217343in}{2.780589in}}{\pgfqpoint{3.225243in}{2.783861in}}{\pgfqpoint{3.231067in}{2.789685in}}%
\pgfpathcurveto{\pgfqpoint{3.236891in}{2.795509in}}{\pgfqpoint{3.240163in}{2.803409in}}{\pgfqpoint{3.240163in}{2.811645in}}%
\pgfpathcurveto{\pgfqpoint{3.240163in}{2.819882in}}{\pgfqpoint{3.236891in}{2.827782in}}{\pgfqpoint{3.231067in}{2.833606in}}%
\pgfpathcurveto{\pgfqpoint{3.225243in}{2.839430in}}{\pgfqpoint{3.217343in}{2.842702in}}{\pgfqpoint{3.209106in}{2.842702in}}%
\pgfpathcurveto{\pgfqpoint{3.200870in}{2.842702in}}{\pgfqpoint{3.192970in}{2.839430in}}{\pgfqpoint{3.187146in}{2.833606in}}%
\pgfpathcurveto{\pgfqpoint{3.181322in}{2.827782in}}{\pgfqpoint{3.178050in}{2.819882in}}{\pgfqpoint{3.178050in}{2.811645in}}%
\pgfpathcurveto{\pgfqpoint{3.178050in}{2.803409in}}{\pgfqpoint{3.181322in}{2.795509in}}{\pgfqpoint{3.187146in}{2.789685in}}%
\pgfpathcurveto{\pgfqpoint{3.192970in}{2.783861in}}{\pgfqpoint{3.200870in}{2.780589in}}{\pgfqpoint{3.209106in}{2.780589in}}%
\pgfpathclose%
\pgfusepath{stroke,fill}%
\end{pgfscope}%
\begin{pgfscope}%
\pgfpathrectangle{\pgfqpoint{0.100000in}{0.212622in}}{\pgfqpoint{3.696000in}{3.696000in}}%
\pgfusepath{clip}%
\pgfsetbuttcap%
\pgfsetroundjoin%
\definecolor{currentfill}{rgb}{0.121569,0.466667,0.705882}%
\pgfsetfillcolor{currentfill}%
\pgfsetfillopacity{0.668539}%
\pgfsetlinewidth{1.003750pt}%
\definecolor{currentstroke}{rgb}{0.121569,0.466667,0.705882}%
\pgfsetstrokecolor{currentstroke}%
\pgfsetstrokeopacity{0.668539}%
\pgfsetdash{}{0pt}%
\pgfpathmoveto{\pgfqpoint{3.201984in}{2.773042in}}%
\pgfpathcurveto{\pgfqpoint{3.210221in}{2.773042in}}{\pgfqpoint{3.218121in}{2.776315in}}{\pgfqpoint{3.223945in}{2.782139in}}%
\pgfpathcurveto{\pgfqpoint{3.229769in}{2.787963in}}{\pgfqpoint{3.233041in}{2.795863in}}{\pgfqpoint{3.233041in}{2.804099in}}%
\pgfpathcurveto{\pgfqpoint{3.233041in}{2.812335in}}{\pgfqpoint{3.229769in}{2.820235in}}{\pgfqpoint{3.223945in}{2.826059in}}%
\pgfpathcurveto{\pgfqpoint{3.218121in}{2.831883in}}{\pgfqpoint{3.210221in}{2.835155in}}{\pgfqpoint{3.201984in}{2.835155in}}%
\pgfpathcurveto{\pgfqpoint{3.193748in}{2.835155in}}{\pgfqpoint{3.185848in}{2.831883in}}{\pgfqpoint{3.180024in}{2.826059in}}%
\pgfpathcurveto{\pgfqpoint{3.174200in}{2.820235in}}{\pgfqpoint{3.170928in}{2.812335in}}{\pgfqpoint{3.170928in}{2.804099in}}%
\pgfpathcurveto{\pgfqpoint{3.170928in}{2.795863in}}{\pgfqpoint{3.174200in}{2.787963in}}{\pgfqpoint{3.180024in}{2.782139in}}%
\pgfpathcurveto{\pgfqpoint{3.185848in}{2.776315in}}{\pgfqpoint{3.193748in}{2.773042in}}{\pgfqpoint{3.201984in}{2.773042in}}%
\pgfpathclose%
\pgfusepath{stroke,fill}%
\end{pgfscope}%
\begin{pgfscope}%
\pgfpathrectangle{\pgfqpoint{0.100000in}{0.212622in}}{\pgfqpoint{3.696000in}{3.696000in}}%
\pgfusepath{clip}%
\pgfsetbuttcap%
\pgfsetroundjoin%
\definecolor{currentfill}{rgb}{0.121569,0.466667,0.705882}%
\pgfsetfillcolor{currentfill}%
\pgfsetfillopacity{0.670280}%
\pgfsetlinewidth{1.003750pt}%
\definecolor{currentstroke}{rgb}{0.121569,0.466667,0.705882}%
\pgfsetstrokecolor{currentstroke}%
\pgfsetstrokeopacity{0.670280}%
\pgfsetdash{}{0pt}%
\pgfpathmoveto{\pgfqpoint{3.198667in}{2.769229in}}%
\pgfpathcurveto{\pgfqpoint{3.206903in}{2.769229in}}{\pgfqpoint{3.214803in}{2.772502in}}{\pgfqpoint{3.220627in}{2.778326in}}%
\pgfpathcurveto{\pgfqpoint{3.226451in}{2.784150in}}{\pgfqpoint{3.229723in}{2.792050in}}{\pgfqpoint{3.229723in}{2.800286in}}%
\pgfpathcurveto{\pgfqpoint{3.229723in}{2.808522in}}{\pgfqpoint{3.226451in}{2.816422in}}{\pgfqpoint{3.220627in}{2.822246in}}%
\pgfpathcurveto{\pgfqpoint{3.214803in}{2.828070in}}{\pgfqpoint{3.206903in}{2.831342in}}{\pgfqpoint{3.198667in}{2.831342in}}%
\pgfpathcurveto{\pgfqpoint{3.190430in}{2.831342in}}{\pgfqpoint{3.182530in}{2.828070in}}{\pgfqpoint{3.176706in}{2.822246in}}%
\pgfpathcurveto{\pgfqpoint{3.170882in}{2.816422in}}{\pgfqpoint{3.167610in}{2.808522in}}{\pgfqpoint{3.167610in}{2.800286in}}%
\pgfpathcurveto{\pgfqpoint{3.167610in}{2.792050in}}{\pgfqpoint{3.170882in}{2.784150in}}{\pgfqpoint{3.176706in}{2.778326in}}%
\pgfpathcurveto{\pgfqpoint{3.182530in}{2.772502in}}{\pgfqpoint{3.190430in}{2.769229in}}{\pgfqpoint{3.198667in}{2.769229in}}%
\pgfpathclose%
\pgfusepath{stroke,fill}%
\end{pgfscope}%
\begin{pgfscope}%
\pgfpathrectangle{\pgfqpoint{0.100000in}{0.212622in}}{\pgfqpoint{3.696000in}{3.696000in}}%
\pgfusepath{clip}%
\pgfsetbuttcap%
\pgfsetroundjoin%
\definecolor{currentfill}{rgb}{0.121569,0.466667,0.705882}%
\pgfsetfillcolor{currentfill}%
\pgfsetfillopacity{0.673797}%
\pgfsetlinewidth{1.003750pt}%
\definecolor{currentstroke}{rgb}{0.121569,0.466667,0.705882}%
\pgfsetstrokecolor{currentstroke}%
\pgfsetstrokeopacity{0.673797}%
\pgfsetdash{}{0pt}%
\pgfpathmoveto{\pgfqpoint{3.014159in}{2.655536in}}%
\pgfpathcurveto{\pgfqpoint{3.022396in}{2.655536in}}{\pgfqpoint{3.030296in}{2.658809in}}{\pgfqpoint{3.036120in}{2.664633in}}%
\pgfpathcurveto{\pgfqpoint{3.041943in}{2.670456in}}{\pgfqpoint{3.045216in}{2.678356in}}{\pgfqpoint{3.045216in}{2.686593in}}%
\pgfpathcurveto{\pgfqpoint{3.045216in}{2.694829in}}{\pgfqpoint{3.041943in}{2.702729in}}{\pgfqpoint{3.036120in}{2.708553in}}%
\pgfpathcurveto{\pgfqpoint{3.030296in}{2.714377in}}{\pgfqpoint{3.022396in}{2.717649in}}{\pgfqpoint{3.014159in}{2.717649in}}%
\pgfpathcurveto{\pgfqpoint{3.005923in}{2.717649in}}{\pgfqpoint{2.998023in}{2.714377in}}{\pgfqpoint{2.992199in}{2.708553in}}%
\pgfpathcurveto{\pgfqpoint{2.986375in}{2.702729in}}{\pgfqpoint{2.983103in}{2.694829in}}{\pgfqpoint{2.983103in}{2.686593in}}%
\pgfpathcurveto{\pgfqpoint{2.983103in}{2.678356in}}{\pgfqpoint{2.986375in}{2.670456in}}{\pgfqpoint{2.992199in}{2.664633in}}%
\pgfpathcurveto{\pgfqpoint{2.998023in}{2.658809in}}{\pgfqpoint{3.005923in}{2.655536in}}{\pgfqpoint{3.014159in}{2.655536in}}%
\pgfpathclose%
\pgfusepath{stroke,fill}%
\end{pgfscope}%
\begin{pgfscope}%
\pgfpathrectangle{\pgfqpoint{0.100000in}{0.212622in}}{\pgfqpoint{3.696000in}{3.696000in}}%
\pgfusepath{clip}%
\pgfsetbuttcap%
\pgfsetroundjoin%
\definecolor{currentfill}{rgb}{0.121569,0.466667,0.705882}%
\pgfsetfillcolor{currentfill}%
\pgfsetfillopacity{0.674662}%
\pgfsetlinewidth{1.003750pt}%
\definecolor{currentstroke}{rgb}{0.121569,0.466667,0.705882}%
\pgfsetstrokecolor{currentstroke}%
\pgfsetstrokeopacity{0.674662}%
\pgfsetdash{}{0pt}%
\pgfpathmoveto{\pgfqpoint{3.045203in}{2.675420in}}%
\pgfpathcurveto{\pgfqpoint{3.053439in}{2.675420in}}{\pgfqpoint{3.061339in}{2.678692in}}{\pgfqpoint{3.067163in}{2.684516in}}%
\pgfpathcurveto{\pgfqpoint{3.072987in}{2.690340in}}{\pgfqpoint{3.076260in}{2.698240in}}{\pgfqpoint{3.076260in}{2.706476in}}%
\pgfpathcurveto{\pgfqpoint{3.076260in}{2.714712in}}{\pgfqpoint{3.072987in}{2.722612in}}{\pgfqpoint{3.067163in}{2.728436in}}%
\pgfpathcurveto{\pgfqpoint{3.061339in}{2.734260in}}{\pgfqpoint{3.053439in}{2.737533in}}{\pgfqpoint{3.045203in}{2.737533in}}%
\pgfpathcurveto{\pgfqpoint{3.036967in}{2.737533in}}{\pgfqpoint{3.029067in}{2.734260in}}{\pgfqpoint{3.023243in}{2.728436in}}%
\pgfpathcurveto{\pgfqpoint{3.017419in}{2.722612in}}{\pgfqpoint{3.014147in}{2.714712in}}{\pgfqpoint{3.014147in}{2.706476in}}%
\pgfpathcurveto{\pgfqpoint{3.014147in}{2.698240in}}{\pgfqpoint{3.017419in}{2.690340in}}{\pgfqpoint{3.023243in}{2.684516in}}%
\pgfpathcurveto{\pgfqpoint{3.029067in}{2.678692in}}{\pgfqpoint{3.036967in}{2.675420in}}{\pgfqpoint{3.045203in}{2.675420in}}%
\pgfpathclose%
\pgfusepath{stroke,fill}%
\end{pgfscope}%
\begin{pgfscope}%
\pgfpathrectangle{\pgfqpoint{0.100000in}{0.212622in}}{\pgfqpoint{3.696000in}{3.696000in}}%
\pgfusepath{clip}%
\pgfsetbuttcap%
\pgfsetroundjoin%
\definecolor{currentfill}{rgb}{0.121569,0.466667,0.705882}%
\pgfsetfillcolor{currentfill}%
\pgfsetfillopacity{0.681858}%
\pgfsetlinewidth{1.003750pt}%
\definecolor{currentstroke}{rgb}{0.121569,0.466667,0.705882}%
\pgfsetstrokecolor{currentstroke}%
\pgfsetstrokeopacity{0.681858}%
\pgfsetdash{}{0pt}%
\pgfpathmoveto{\pgfqpoint{3.061093in}{2.682554in}}%
\pgfpathcurveto{\pgfqpoint{3.069329in}{2.682554in}}{\pgfqpoint{3.077230in}{2.685826in}}{\pgfqpoint{3.083053in}{2.691650in}}%
\pgfpathcurveto{\pgfqpoint{3.088877in}{2.697474in}}{\pgfqpoint{3.092150in}{2.705374in}}{\pgfqpoint{3.092150in}{2.713610in}}%
\pgfpathcurveto{\pgfqpoint{3.092150in}{2.721847in}}{\pgfqpoint{3.088877in}{2.729747in}}{\pgfqpoint{3.083053in}{2.735571in}}%
\pgfpathcurveto{\pgfqpoint{3.077230in}{2.741395in}}{\pgfqpoint{3.069329in}{2.744667in}}{\pgfqpoint{3.061093in}{2.744667in}}%
\pgfpathcurveto{\pgfqpoint{3.052857in}{2.744667in}}{\pgfqpoint{3.044957in}{2.741395in}}{\pgfqpoint{3.039133in}{2.735571in}}%
\pgfpathcurveto{\pgfqpoint{3.033309in}{2.729747in}}{\pgfqpoint{3.030037in}{2.721847in}}{\pgfqpoint{3.030037in}{2.713610in}}%
\pgfpathcurveto{\pgfqpoint{3.030037in}{2.705374in}}{\pgfqpoint{3.033309in}{2.697474in}}{\pgfqpoint{3.039133in}{2.691650in}}%
\pgfpathcurveto{\pgfqpoint{3.044957in}{2.685826in}}{\pgfqpoint{3.052857in}{2.682554in}}{\pgfqpoint{3.061093in}{2.682554in}}%
\pgfpathclose%
\pgfusepath{stroke,fill}%
\end{pgfscope}%
\begin{pgfscope}%
\pgfpathrectangle{\pgfqpoint{0.100000in}{0.212622in}}{\pgfqpoint{3.696000in}{3.696000in}}%
\pgfusepath{clip}%
\pgfsetbuttcap%
\pgfsetroundjoin%
\definecolor{currentfill}{rgb}{0.121569,0.466667,0.705882}%
\pgfsetfillcolor{currentfill}%
\pgfsetfillopacity{0.687827}%
\pgfsetlinewidth{1.003750pt}%
\definecolor{currentstroke}{rgb}{0.121569,0.466667,0.705882}%
\pgfsetstrokecolor{currentstroke}%
\pgfsetstrokeopacity{0.687827}%
\pgfsetdash{}{0pt}%
\pgfpathmoveto{\pgfqpoint{3.151357in}{2.707029in}}%
\pgfpathcurveto{\pgfqpoint{3.159593in}{2.707029in}}{\pgfqpoint{3.167493in}{2.710301in}}{\pgfqpoint{3.173317in}{2.716125in}}%
\pgfpathcurveto{\pgfqpoint{3.179141in}{2.721949in}}{\pgfqpoint{3.182413in}{2.729849in}}{\pgfqpoint{3.182413in}{2.738086in}}%
\pgfpathcurveto{\pgfqpoint{3.182413in}{2.746322in}}{\pgfqpoint{3.179141in}{2.754222in}}{\pgfqpoint{3.173317in}{2.760046in}}%
\pgfpathcurveto{\pgfqpoint{3.167493in}{2.765870in}}{\pgfqpoint{3.159593in}{2.769142in}}{\pgfqpoint{3.151357in}{2.769142in}}%
\pgfpathcurveto{\pgfqpoint{3.143121in}{2.769142in}}{\pgfqpoint{3.135220in}{2.765870in}}{\pgfqpoint{3.129397in}{2.760046in}}%
\pgfpathcurveto{\pgfqpoint{3.123573in}{2.754222in}}{\pgfqpoint{3.120300in}{2.746322in}}{\pgfqpoint{3.120300in}{2.738086in}}%
\pgfpathcurveto{\pgfqpoint{3.120300in}{2.729849in}}{\pgfqpoint{3.123573in}{2.721949in}}{\pgfqpoint{3.129397in}{2.716125in}}%
\pgfpathcurveto{\pgfqpoint{3.135220in}{2.710301in}}{\pgfqpoint{3.143121in}{2.707029in}}{\pgfqpoint{3.151357in}{2.707029in}}%
\pgfpathclose%
\pgfusepath{stroke,fill}%
\end{pgfscope}%
\begin{pgfscope}%
\pgfpathrectangle{\pgfqpoint{0.100000in}{0.212622in}}{\pgfqpoint{3.696000in}{3.696000in}}%
\pgfusepath{clip}%
\pgfsetbuttcap%
\pgfsetroundjoin%
\definecolor{currentfill}{rgb}{0.121569,0.466667,0.705882}%
\pgfsetfillcolor{currentfill}%
\pgfsetfillopacity{0.721569}%
\pgfsetlinewidth{1.003750pt}%
\definecolor{currentstroke}{rgb}{0.121569,0.466667,0.705882}%
\pgfsetstrokecolor{currentstroke}%
\pgfsetstrokeopacity{0.721569}%
\pgfsetdash{}{0pt}%
\pgfpathmoveto{\pgfqpoint{3.071317in}{2.634159in}}%
\pgfpathcurveto{\pgfqpoint{3.079553in}{2.634159in}}{\pgfqpoint{3.087453in}{2.637431in}}{\pgfqpoint{3.093277in}{2.643255in}}%
\pgfpathcurveto{\pgfqpoint{3.099101in}{2.649079in}}{\pgfqpoint{3.102374in}{2.656979in}}{\pgfqpoint{3.102374in}{2.665215in}}%
\pgfpathcurveto{\pgfqpoint{3.102374in}{2.673451in}}{\pgfqpoint{3.099101in}{2.681351in}}{\pgfqpoint{3.093277in}{2.687175in}}%
\pgfpathcurveto{\pgfqpoint{3.087453in}{2.692999in}}{\pgfqpoint{3.079553in}{2.696272in}}{\pgfqpoint{3.071317in}{2.696272in}}%
\pgfpathcurveto{\pgfqpoint{3.063081in}{2.696272in}}{\pgfqpoint{3.055181in}{2.692999in}}{\pgfqpoint{3.049357in}{2.687175in}}%
\pgfpathcurveto{\pgfqpoint{3.043533in}{2.681351in}}{\pgfqpoint{3.040261in}{2.673451in}}{\pgfqpoint{3.040261in}{2.665215in}}%
\pgfpathcurveto{\pgfqpoint{3.040261in}{2.656979in}}{\pgfqpoint{3.043533in}{2.649079in}}{\pgfqpoint{3.049357in}{2.643255in}}%
\pgfpathcurveto{\pgfqpoint{3.055181in}{2.637431in}}{\pgfqpoint{3.063081in}{2.634159in}}{\pgfqpoint{3.071317in}{2.634159in}}%
\pgfpathclose%
\pgfusepath{stroke,fill}%
\end{pgfscope}%
\begin{pgfscope}%
\pgfpathrectangle{\pgfqpoint{0.100000in}{0.212622in}}{\pgfqpoint{3.696000in}{3.696000in}}%
\pgfusepath{clip}%
\pgfsetbuttcap%
\pgfsetroundjoin%
\definecolor{currentfill}{rgb}{0.121569,0.466667,0.705882}%
\pgfsetfillcolor{currentfill}%
\pgfsetfillopacity{0.790275}%
\pgfsetlinewidth{1.003750pt}%
\definecolor{currentstroke}{rgb}{0.121569,0.466667,0.705882}%
\pgfsetstrokecolor{currentstroke}%
\pgfsetstrokeopacity{0.790275}%
\pgfsetdash{}{0pt}%
\pgfpathmoveto{\pgfqpoint{2.902061in}{2.480633in}}%
\pgfpathcurveto{\pgfqpoint{2.910298in}{2.480633in}}{\pgfqpoint{2.918198in}{2.483905in}}{\pgfqpoint{2.924022in}{2.489729in}}%
\pgfpathcurveto{\pgfqpoint{2.929846in}{2.495553in}}{\pgfqpoint{2.933118in}{2.503453in}}{\pgfqpoint{2.933118in}{2.511689in}}%
\pgfpathcurveto{\pgfqpoint{2.933118in}{2.519925in}}{\pgfqpoint{2.929846in}{2.527825in}}{\pgfqpoint{2.924022in}{2.533649in}}%
\pgfpathcurveto{\pgfqpoint{2.918198in}{2.539473in}}{\pgfqpoint{2.910298in}{2.542746in}}{\pgfqpoint{2.902061in}{2.542746in}}%
\pgfpathcurveto{\pgfqpoint{2.893825in}{2.542746in}}{\pgfqpoint{2.885925in}{2.539473in}}{\pgfqpoint{2.880101in}{2.533649in}}%
\pgfpathcurveto{\pgfqpoint{2.874277in}{2.527825in}}{\pgfqpoint{2.871005in}{2.519925in}}{\pgfqpoint{2.871005in}{2.511689in}}%
\pgfpathcurveto{\pgfqpoint{2.871005in}{2.503453in}}{\pgfqpoint{2.874277in}{2.495553in}}{\pgfqpoint{2.880101in}{2.489729in}}%
\pgfpathcurveto{\pgfqpoint{2.885925in}{2.483905in}}{\pgfqpoint{2.893825in}{2.480633in}}{\pgfqpoint{2.902061in}{2.480633in}}%
\pgfpathclose%
\pgfusepath{stroke,fill}%
\end{pgfscope}%
\begin{pgfscope}%
\pgfpathrectangle{\pgfqpoint{0.100000in}{0.212622in}}{\pgfqpoint{3.696000in}{3.696000in}}%
\pgfusepath{clip}%
\pgfsetbuttcap%
\pgfsetroundjoin%
\definecolor{currentfill}{rgb}{0.121569,0.466667,0.705882}%
\pgfsetfillcolor{currentfill}%
\pgfsetfillopacity{0.863463}%
\pgfsetlinewidth{1.003750pt}%
\definecolor{currentstroke}{rgb}{0.121569,0.466667,0.705882}%
\pgfsetstrokecolor{currentstroke}%
\pgfsetstrokeopacity{0.863463}%
\pgfsetdash{}{0pt}%
\pgfpathmoveto{\pgfqpoint{2.726305in}{2.319431in}}%
\pgfpathcurveto{\pgfqpoint{2.734541in}{2.319431in}}{\pgfqpoint{2.742441in}{2.322703in}}{\pgfqpoint{2.748265in}{2.328527in}}%
\pgfpathcurveto{\pgfqpoint{2.754089in}{2.334351in}}{\pgfqpoint{2.757361in}{2.342251in}}{\pgfqpoint{2.757361in}{2.350487in}}%
\pgfpathcurveto{\pgfqpoint{2.757361in}{2.358724in}}{\pgfqpoint{2.754089in}{2.366624in}}{\pgfqpoint{2.748265in}{2.372448in}}%
\pgfpathcurveto{\pgfqpoint{2.742441in}{2.378272in}}{\pgfqpoint{2.734541in}{2.381544in}}{\pgfqpoint{2.726305in}{2.381544in}}%
\pgfpathcurveto{\pgfqpoint{2.718069in}{2.381544in}}{\pgfqpoint{2.710169in}{2.378272in}}{\pgfqpoint{2.704345in}{2.372448in}}%
\pgfpathcurveto{\pgfqpoint{2.698521in}{2.366624in}}{\pgfqpoint{2.695248in}{2.358724in}}{\pgfqpoint{2.695248in}{2.350487in}}%
\pgfpathcurveto{\pgfqpoint{2.695248in}{2.342251in}}{\pgfqpoint{2.698521in}{2.334351in}}{\pgfqpoint{2.704345in}{2.328527in}}%
\pgfpathcurveto{\pgfqpoint{2.710169in}{2.322703in}}{\pgfqpoint{2.718069in}{2.319431in}}{\pgfqpoint{2.726305in}{2.319431in}}%
\pgfpathclose%
\pgfusepath{stroke,fill}%
\end{pgfscope}%
\begin{pgfscope}%
\pgfpathrectangle{\pgfqpoint{0.100000in}{0.212622in}}{\pgfqpoint{3.696000in}{3.696000in}}%
\pgfusepath{clip}%
\pgfsetbuttcap%
\pgfsetroundjoin%
\definecolor{currentfill}{rgb}{0.121569,0.466667,0.705882}%
\pgfsetfillcolor{currentfill}%
\pgfsetfillopacity{0.890826}%
\pgfsetlinewidth{1.003750pt}%
\definecolor{currentstroke}{rgb}{0.121569,0.466667,0.705882}%
\pgfsetstrokecolor{currentstroke}%
\pgfsetstrokeopacity{0.890826}%
\pgfsetdash{}{0pt}%
\pgfpathmoveto{\pgfqpoint{2.664638in}{2.266384in}}%
\pgfpathcurveto{\pgfqpoint{2.672875in}{2.266384in}}{\pgfqpoint{2.680775in}{2.269656in}}{\pgfqpoint{2.686599in}{2.275480in}}%
\pgfpathcurveto{\pgfqpoint{2.692422in}{2.281304in}}{\pgfqpoint{2.695695in}{2.289204in}}{\pgfqpoint{2.695695in}{2.297440in}}%
\pgfpathcurveto{\pgfqpoint{2.695695in}{2.305677in}}{\pgfqpoint{2.692422in}{2.313577in}}{\pgfqpoint{2.686599in}{2.319401in}}%
\pgfpathcurveto{\pgfqpoint{2.680775in}{2.325224in}}{\pgfqpoint{2.672875in}{2.328497in}}{\pgfqpoint{2.664638in}{2.328497in}}%
\pgfpathcurveto{\pgfqpoint{2.656402in}{2.328497in}}{\pgfqpoint{2.648502in}{2.325224in}}{\pgfqpoint{2.642678in}{2.319401in}}%
\pgfpathcurveto{\pgfqpoint{2.636854in}{2.313577in}}{\pgfqpoint{2.633582in}{2.305677in}}{\pgfqpoint{2.633582in}{2.297440in}}%
\pgfpathcurveto{\pgfqpoint{2.633582in}{2.289204in}}{\pgfqpoint{2.636854in}{2.281304in}}{\pgfqpoint{2.642678in}{2.275480in}}%
\pgfpathcurveto{\pgfqpoint{2.648502in}{2.269656in}}{\pgfqpoint{2.656402in}{2.266384in}}{\pgfqpoint{2.664638in}{2.266384in}}%
\pgfpathclose%
\pgfusepath{stroke,fill}%
\end{pgfscope}%
\begin{pgfscope}%
\pgfpathrectangle{\pgfqpoint{0.100000in}{0.212622in}}{\pgfqpoint{3.696000in}{3.696000in}}%
\pgfusepath{clip}%
\pgfsetbuttcap%
\pgfsetroundjoin%
\definecolor{currentfill}{rgb}{0.121569,0.466667,0.705882}%
\pgfsetfillcolor{currentfill}%
\pgfsetfillopacity{0.908692}%
\pgfsetlinewidth{1.003750pt}%
\definecolor{currentstroke}{rgb}{0.121569,0.466667,0.705882}%
\pgfsetstrokecolor{currentstroke}%
\pgfsetstrokeopacity{0.908692}%
\pgfsetdash{}{0pt}%
\pgfpathmoveto{\pgfqpoint{2.623050in}{2.229046in}}%
\pgfpathcurveto{\pgfqpoint{2.631287in}{2.229046in}}{\pgfqpoint{2.639187in}{2.232318in}}{\pgfqpoint{2.645011in}{2.238142in}}%
\pgfpathcurveto{\pgfqpoint{2.650835in}{2.243966in}}{\pgfqpoint{2.654107in}{2.251866in}}{\pgfqpoint{2.654107in}{2.260102in}}%
\pgfpathcurveto{\pgfqpoint{2.654107in}{2.268338in}}{\pgfqpoint{2.650835in}{2.276238in}}{\pgfqpoint{2.645011in}{2.282062in}}%
\pgfpathcurveto{\pgfqpoint{2.639187in}{2.287886in}}{\pgfqpoint{2.631287in}{2.291159in}}{\pgfqpoint{2.623050in}{2.291159in}}%
\pgfpathcurveto{\pgfqpoint{2.614814in}{2.291159in}}{\pgfqpoint{2.606914in}{2.287886in}}{\pgfqpoint{2.601090in}{2.282062in}}%
\pgfpathcurveto{\pgfqpoint{2.595266in}{2.276238in}}{\pgfqpoint{2.591994in}{2.268338in}}{\pgfqpoint{2.591994in}{2.260102in}}%
\pgfpathcurveto{\pgfqpoint{2.591994in}{2.251866in}}{\pgfqpoint{2.595266in}{2.243966in}}{\pgfqpoint{2.601090in}{2.238142in}}%
\pgfpathcurveto{\pgfqpoint{2.606914in}{2.232318in}}{\pgfqpoint{2.614814in}{2.229046in}}{\pgfqpoint{2.623050in}{2.229046in}}%
\pgfpathclose%
\pgfusepath{stroke,fill}%
\end{pgfscope}%
\begin{pgfscope}%
\pgfpathrectangle{\pgfqpoint{0.100000in}{0.212622in}}{\pgfqpoint{3.696000in}{3.696000in}}%
\pgfusepath{clip}%
\pgfsetbuttcap%
\pgfsetroundjoin%
\definecolor{currentfill}{rgb}{0.121569,0.466667,0.705882}%
\pgfsetfillcolor{currentfill}%
\pgfsetfillopacity{0.933057}%
\pgfsetlinewidth{1.003750pt}%
\definecolor{currentstroke}{rgb}{0.121569,0.466667,0.705882}%
\pgfsetstrokecolor{currentstroke}%
\pgfsetstrokeopacity{0.933057}%
\pgfsetdash{}{0pt}%
\pgfpathmoveto{\pgfqpoint{2.566204in}{2.178569in}}%
\pgfpathcurveto{\pgfqpoint{2.574440in}{2.178569in}}{\pgfqpoint{2.582340in}{2.181842in}}{\pgfqpoint{2.588164in}{2.187665in}}%
\pgfpathcurveto{\pgfqpoint{2.593988in}{2.193489in}}{\pgfqpoint{2.597260in}{2.201389in}}{\pgfqpoint{2.597260in}{2.209626in}}%
\pgfpathcurveto{\pgfqpoint{2.597260in}{2.217862in}}{\pgfqpoint{2.593988in}{2.225762in}}{\pgfqpoint{2.588164in}{2.231586in}}%
\pgfpathcurveto{\pgfqpoint{2.582340in}{2.237410in}}{\pgfqpoint{2.574440in}{2.240682in}}{\pgfqpoint{2.566204in}{2.240682in}}%
\pgfpathcurveto{\pgfqpoint{2.557967in}{2.240682in}}{\pgfqpoint{2.550067in}{2.237410in}}{\pgfqpoint{2.544243in}{2.231586in}}%
\pgfpathcurveto{\pgfqpoint{2.538419in}{2.225762in}}{\pgfqpoint{2.535147in}{2.217862in}}{\pgfqpoint{2.535147in}{2.209626in}}%
\pgfpathcurveto{\pgfqpoint{2.535147in}{2.201389in}}{\pgfqpoint{2.538419in}{2.193489in}}{\pgfqpoint{2.544243in}{2.187665in}}%
\pgfpathcurveto{\pgfqpoint{2.550067in}{2.181842in}}{\pgfqpoint{2.557967in}{2.178569in}}{\pgfqpoint{2.566204in}{2.178569in}}%
\pgfpathclose%
\pgfusepath{stroke,fill}%
\end{pgfscope}%
\begin{pgfscope}%
\pgfpathrectangle{\pgfqpoint{0.100000in}{0.212622in}}{\pgfqpoint{3.696000in}{3.696000in}}%
\pgfusepath{clip}%
\pgfsetbuttcap%
\pgfsetroundjoin%
\definecolor{currentfill}{rgb}{0.121569,0.466667,0.705882}%
\pgfsetfillcolor{currentfill}%
\pgfsetfillopacity{0.960198}%
\pgfsetlinewidth{1.003750pt}%
\definecolor{currentstroke}{rgb}{0.121569,0.466667,0.705882}%
\pgfsetstrokecolor{currentstroke}%
\pgfsetstrokeopacity{0.960198}%
\pgfsetdash{}{0pt}%
\pgfpathmoveto{\pgfqpoint{2.503184in}{2.122622in}}%
\pgfpathcurveto{\pgfqpoint{2.511420in}{2.122622in}}{\pgfqpoint{2.519320in}{2.125894in}}{\pgfqpoint{2.525144in}{2.131718in}}%
\pgfpathcurveto{\pgfqpoint{2.530968in}{2.137542in}}{\pgfqpoint{2.534240in}{2.145442in}}{\pgfqpoint{2.534240in}{2.153678in}}%
\pgfpathcurveto{\pgfqpoint{2.534240in}{2.161914in}}{\pgfqpoint{2.530968in}{2.169814in}}{\pgfqpoint{2.525144in}{2.175638in}}%
\pgfpathcurveto{\pgfqpoint{2.519320in}{2.181462in}}{\pgfqpoint{2.511420in}{2.184735in}}{\pgfqpoint{2.503184in}{2.184735in}}%
\pgfpathcurveto{\pgfqpoint{2.494948in}{2.184735in}}{\pgfqpoint{2.487048in}{2.181462in}}{\pgfqpoint{2.481224in}{2.175638in}}%
\pgfpathcurveto{\pgfqpoint{2.475400in}{2.169814in}}{\pgfqpoint{2.472127in}{2.161914in}}{\pgfqpoint{2.472127in}{2.153678in}}%
\pgfpathcurveto{\pgfqpoint{2.472127in}{2.145442in}}{\pgfqpoint{2.475400in}{2.137542in}}{\pgfqpoint{2.481224in}{2.131718in}}%
\pgfpathcurveto{\pgfqpoint{2.487048in}{2.125894in}}{\pgfqpoint{2.494948in}{2.122622in}}{\pgfqpoint{2.503184in}{2.122622in}}%
\pgfpathclose%
\pgfusepath{stroke,fill}%
\end{pgfscope}%
\begin{pgfscope}%
\pgfpathrectangle{\pgfqpoint{0.100000in}{0.212622in}}{\pgfqpoint{3.696000in}{3.696000in}}%
\pgfusepath{clip}%
\pgfsetbuttcap%
\pgfsetroundjoin%
\definecolor{currentfill}{rgb}{0.121569,0.466667,0.705882}%
\pgfsetfillcolor{currentfill}%
\pgfsetfillopacity{0.975536}%
\pgfsetlinewidth{1.003750pt}%
\definecolor{currentstroke}{rgb}{0.121569,0.466667,0.705882}%
\pgfsetstrokecolor{currentstroke}%
\pgfsetstrokeopacity{0.975536}%
\pgfsetdash{}{0pt}%
\pgfpathmoveto{\pgfqpoint{2.467540in}{2.091070in}}%
\pgfpathcurveto{\pgfqpoint{2.475776in}{2.091070in}}{\pgfqpoint{2.483676in}{2.094342in}}{\pgfqpoint{2.489500in}{2.100166in}}%
\pgfpathcurveto{\pgfqpoint{2.495324in}{2.105990in}}{\pgfqpoint{2.498596in}{2.113890in}}{\pgfqpoint{2.498596in}{2.122126in}}%
\pgfpathcurveto{\pgfqpoint{2.498596in}{2.130362in}}{\pgfqpoint{2.495324in}{2.138262in}}{\pgfqpoint{2.489500in}{2.144086in}}%
\pgfpathcurveto{\pgfqpoint{2.483676in}{2.149910in}}{\pgfqpoint{2.475776in}{2.153183in}}{\pgfqpoint{2.467540in}{2.153183in}}%
\pgfpathcurveto{\pgfqpoint{2.459303in}{2.153183in}}{\pgfqpoint{2.451403in}{2.149910in}}{\pgfqpoint{2.445579in}{2.144086in}}%
\pgfpathcurveto{\pgfqpoint{2.439755in}{2.138262in}}{\pgfqpoint{2.436483in}{2.130362in}}{\pgfqpoint{2.436483in}{2.122126in}}%
\pgfpathcurveto{\pgfqpoint{2.436483in}{2.113890in}}{\pgfqpoint{2.439755in}{2.105990in}}{\pgfqpoint{2.445579in}{2.100166in}}%
\pgfpathcurveto{\pgfqpoint{2.451403in}{2.094342in}}{\pgfqpoint{2.459303in}{2.091070in}}{\pgfqpoint{2.467540in}{2.091070in}}%
\pgfpathclose%
\pgfusepath{stroke,fill}%
\end{pgfscope}%
\begin{pgfscope}%
\pgfpathrectangle{\pgfqpoint{0.100000in}{0.212622in}}{\pgfqpoint{3.696000in}{3.696000in}}%
\pgfusepath{clip}%
\pgfsetbuttcap%
\pgfsetroundjoin%
\definecolor{currentfill}{rgb}{0.121569,0.466667,0.705882}%
\pgfsetfillcolor{currentfill}%
\pgfsetfillopacity{0.983211}%
\pgfsetlinewidth{1.003750pt}%
\definecolor{currentstroke}{rgb}{0.121569,0.466667,0.705882}%
\pgfsetstrokecolor{currentstroke}%
\pgfsetstrokeopacity{0.983211}%
\pgfsetdash{}{0pt}%
\pgfpathmoveto{\pgfqpoint{2.449923in}{2.075578in}}%
\pgfpathcurveto{\pgfqpoint{2.458159in}{2.075578in}}{\pgfqpoint{2.466060in}{2.078850in}}{\pgfqpoint{2.471883in}{2.084674in}}%
\pgfpathcurveto{\pgfqpoint{2.477707in}{2.090498in}}{\pgfqpoint{2.480980in}{2.098398in}}{\pgfqpoint{2.480980in}{2.106634in}}%
\pgfpathcurveto{\pgfqpoint{2.480980in}{2.114871in}}{\pgfqpoint{2.477707in}{2.122771in}}{\pgfqpoint{2.471883in}{2.128595in}}%
\pgfpathcurveto{\pgfqpoint{2.466060in}{2.134418in}}{\pgfqpoint{2.458159in}{2.137691in}}{\pgfqpoint{2.449923in}{2.137691in}}%
\pgfpathcurveto{\pgfqpoint{2.441687in}{2.137691in}}{\pgfqpoint{2.433787in}{2.134418in}}{\pgfqpoint{2.427963in}{2.128595in}}%
\pgfpathcurveto{\pgfqpoint{2.422139in}{2.122771in}}{\pgfqpoint{2.418867in}{2.114871in}}{\pgfqpoint{2.418867in}{2.106634in}}%
\pgfpathcurveto{\pgfqpoint{2.418867in}{2.098398in}}{\pgfqpoint{2.422139in}{2.090498in}}{\pgfqpoint{2.427963in}{2.084674in}}%
\pgfpathcurveto{\pgfqpoint{2.433787in}{2.078850in}}{\pgfqpoint{2.441687in}{2.075578in}}{\pgfqpoint{2.449923in}{2.075578in}}%
\pgfpathclose%
\pgfusepath{stroke,fill}%
\end{pgfscope}%
\begin{pgfscope}%
\pgfpathrectangle{\pgfqpoint{0.100000in}{0.212622in}}{\pgfqpoint{3.696000in}{3.696000in}}%
\pgfusepath{clip}%
\pgfsetbuttcap%
\pgfsetroundjoin%
\definecolor{currentfill}{rgb}{0.121569,0.466667,0.705882}%
\pgfsetfillcolor{currentfill}%
\pgfsetfillopacity{0.993444}%
\pgfsetlinewidth{1.003750pt}%
\definecolor{currentstroke}{rgb}{0.121569,0.466667,0.705882}%
\pgfsetstrokecolor{currentstroke}%
\pgfsetstrokeopacity{0.993444}%
\pgfsetdash{}{0pt}%
\pgfpathmoveto{\pgfqpoint{2.426399in}{2.054687in}}%
\pgfpathcurveto{\pgfqpoint{2.434635in}{2.054687in}}{\pgfqpoint{2.442535in}{2.057960in}}{\pgfqpoint{2.448359in}{2.063784in}}%
\pgfpathcurveto{\pgfqpoint{2.454183in}{2.069608in}}{\pgfqpoint{2.457455in}{2.077508in}}{\pgfqpoint{2.457455in}{2.085744in}}%
\pgfpathcurveto{\pgfqpoint{2.457455in}{2.093980in}}{\pgfqpoint{2.454183in}{2.101880in}}{\pgfqpoint{2.448359in}{2.107704in}}%
\pgfpathcurveto{\pgfqpoint{2.442535in}{2.113528in}}{\pgfqpoint{2.434635in}{2.116800in}}{\pgfqpoint{2.426399in}{2.116800in}}%
\pgfpathcurveto{\pgfqpoint{2.418163in}{2.116800in}}{\pgfqpoint{2.410263in}{2.113528in}}{\pgfqpoint{2.404439in}{2.107704in}}%
\pgfpathcurveto{\pgfqpoint{2.398615in}{2.101880in}}{\pgfqpoint{2.395342in}{2.093980in}}{\pgfqpoint{2.395342in}{2.085744in}}%
\pgfpathcurveto{\pgfqpoint{2.395342in}{2.077508in}}{\pgfqpoint{2.398615in}{2.069608in}}{\pgfqpoint{2.404439in}{2.063784in}}%
\pgfpathcurveto{\pgfqpoint{2.410263in}{2.057960in}}{\pgfqpoint{2.418163in}{2.054687in}}{\pgfqpoint{2.426399in}{2.054687in}}%
\pgfpathclose%
\pgfusepath{stroke,fill}%
\end{pgfscope}%
\begin{pgfscope}%
\pgfpathrectangle{\pgfqpoint{0.100000in}{0.212622in}}{\pgfqpoint{3.696000in}{3.696000in}}%
\pgfusepath{clip}%
\pgfsetbuttcap%
\pgfsetroundjoin%
\definecolor{currentfill}{rgb}{0.121569,0.466667,0.705882}%
\pgfsetfillcolor{currentfill}%
\pgfsetlinewidth{1.003750pt}%
\definecolor{currentstroke}{rgb}{0.121569,0.466667,0.705882}%
\pgfsetstrokecolor{currentstroke}%
\pgfsetdash{}{0pt}%
\pgfpathmoveto{\pgfqpoint{2.411274in}{2.041690in}}%
\pgfpathcurveto{\pgfqpoint{2.419510in}{2.041690in}}{\pgfqpoint{2.427410in}{2.044962in}}{\pgfqpoint{2.433234in}{2.050786in}}%
\pgfpathcurveto{\pgfqpoint{2.439058in}{2.056610in}}{\pgfqpoint{2.442330in}{2.064510in}}{\pgfqpoint{2.442330in}{2.072746in}}%
\pgfpathcurveto{\pgfqpoint{2.442330in}{2.080983in}}{\pgfqpoint{2.439058in}{2.088883in}}{\pgfqpoint{2.433234in}{2.094707in}}%
\pgfpathcurveto{\pgfqpoint{2.427410in}{2.100530in}}{\pgfqpoint{2.419510in}{2.103803in}}{\pgfqpoint{2.411274in}{2.103803in}}%
\pgfpathcurveto{\pgfqpoint{2.403038in}{2.103803in}}{\pgfqpoint{2.395138in}{2.100530in}}{\pgfqpoint{2.389314in}{2.094707in}}%
\pgfpathcurveto{\pgfqpoint{2.383490in}{2.088883in}}{\pgfqpoint{2.380217in}{2.080983in}}{\pgfqpoint{2.380217in}{2.072746in}}%
\pgfpathcurveto{\pgfqpoint{2.380217in}{2.064510in}}{\pgfqpoint{2.383490in}{2.056610in}}{\pgfqpoint{2.389314in}{2.050786in}}%
\pgfpathcurveto{\pgfqpoint{2.395138in}{2.044962in}}{\pgfqpoint{2.403038in}{2.041690in}}{\pgfqpoint{2.411274in}{2.041690in}}%
\pgfpathclose%
\pgfusepath{stroke,fill}%
\end{pgfscope}%
\begin{pgfscope}%
\definecolor{textcolor}{rgb}{0.000000,0.000000,0.000000}%
\pgfsetstrokecolor{textcolor}%
\pgfsetfillcolor{textcolor}%
\pgftext[x=1.948000in,y=3.991956in,,base]{\color{textcolor}\rmfamily\fontsize{12.000000}{14.400000}\selectfont SAAM}%
\end{pgfscope}%
\begin{pgfscope}%
\pgfpathrectangle{\pgfqpoint{0.100000in}{0.212622in}}{\pgfqpoint{3.696000in}{3.696000in}}%
\pgfusepath{clip}%
\pgfsetbuttcap%
\pgfsetroundjoin%
\definecolor{currentfill}{rgb}{1.000000,0.498039,0.054902}%
\pgfsetfillcolor{currentfill}%
\pgfsetfillopacity{0.300000}%
\pgfsetlinewidth{1.003750pt}%
\definecolor{currentstroke}{rgb}{1.000000,0.498039,0.054902}%
\pgfsetstrokecolor{currentstroke}%
\pgfsetstrokeopacity{0.300000}%
\pgfsetdash{}{0pt}%
\pgfpathmoveto{\pgfqpoint{1.209240in}{1.743746in}}%
\pgfpathcurveto{\pgfqpoint{1.217476in}{1.743746in}}{\pgfqpoint{1.225376in}{1.747018in}}{\pgfqpoint{1.231200in}{1.752842in}}%
\pgfpathcurveto{\pgfqpoint{1.237024in}{1.758666in}}{\pgfqpoint{1.240296in}{1.766566in}}{\pgfqpoint{1.240296in}{1.774803in}}%
\pgfpathcurveto{\pgfqpoint{1.240296in}{1.783039in}}{\pgfqpoint{1.237024in}{1.790939in}}{\pgfqpoint{1.231200in}{1.796763in}}%
\pgfpathcurveto{\pgfqpoint{1.225376in}{1.802587in}}{\pgfqpoint{1.217476in}{1.805859in}}{\pgfqpoint{1.209240in}{1.805859in}}%
\pgfpathcurveto{\pgfqpoint{1.201003in}{1.805859in}}{\pgfqpoint{1.193103in}{1.802587in}}{\pgfqpoint{1.187279in}{1.796763in}}%
\pgfpathcurveto{\pgfqpoint{1.181456in}{1.790939in}}{\pgfqpoint{1.178183in}{1.783039in}}{\pgfqpoint{1.178183in}{1.774803in}}%
\pgfpathcurveto{\pgfqpoint{1.178183in}{1.766566in}}{\pgfqpoint{1.181456in}{1.758666in}}{\pgfqpoint{1.187279in}{1.752842in}}%
\pgfpathcurveto{\pgfqpoint{1.193103in}{1.747018in}}{\pgfqpoint{1.201003in}{1.743746in}}{\pgfqpoint{1.209240in}{1.743746in}}%
\pgfpathclose%
\pgfusepath{stroke,fill}%
\end{pgfscope}%
\begin{pgfscope}%
\pgfpathrectangle{\pgfqpoint{0.100000in}{0.212622in}}{\pgfqpoint{3.696000in}{3.696000in}}%
\pgfusepath{clip}%
\pgfsetbuttcap%
\pgfsetroundjoin%
\definecolor{currentfill}{rgb}{1.000000,0.498039,0.054902}%
\pgfsetfillcolor{currentfill}%
\pgfsetlinewidth{1.003750pt}%
\definecolor{currentstroke}{rgb}{1.000000,0.498039,0.054902}%
\pgfsetstrokecolor{currentstroke}%
\pgfsetdash{}{0pt}%
\pgfpathmoveto{\pgfqpoint{2.411274in}{2.041690in}}%
\pgfpathcurveto{\pgfqpoint{2.419510in}{2.041690in}}{\pgfqpoint{2.427410in}{2.044962in}}{\pgfqpoint{2.433234in}{2.050786in}}%
\pgfpathcurveto{\pgfqpoint{2.439058in}{2.056610in}}{\pgfqpoint{2.442330in}{2.064510in}}{\pgfqpoint{2.442330in}{2.072746in}}%
\pgfpathcurveto{\pgfqpoint{2.442330in}{2.080983in}}{\pgfqpoint{2.439058in}{2.088883in}}{\pgfqpoint{2.433234in}{2.094707in}}%
\pgfpathcurveto{\pgfqpoint{2.427410in}{2.100530in}}{\pgfqpoint{2.419510in}{2.103803in}}{\pgfqpoint{2.411274in}{2.103803in}}%
\pgfpathcurveto{\pgfqpoint{2.403038in}{2.103803in}}{\pgfqpoint{2.395138in}{2.100530in}}{\pgfqpoint{2.389314in}{2.094707in}}%
\pgfpathcurveto{\pgfqpoint{2.383490in}{2.088883in}}{\pgfqpoint{2.380217in}{2.080983in}}{\pgfqpoint{2.380217in}{2.072746in}}%
\pgfpathcurveto{\pgfqpoint{2.380217in}{2.064510in}}{\pgfqpoint{2.383490in}{2.056610in}}{\pgfqpoint{2.389314in}{2.050786in}}%
\pgfpathcurveto{\pgfqpoint{2.395138in}{2.044962in}}{\pgfqpoint{2.403038in}{2.041690in}}{\pgfqpoint{2.411274in}{2.041690in}}%
\pgfpathclose%
\pgfusepath{stroke,fill}%
\end{pgfscope}%
\begin{pgfscope}%
\pgfsetbuttcap%
\pgfsetmiterjoin%
\definecolor{currentfill}{rgb}{1.000000,1.000000,1.000000}%
\pgfsetfillcolor{currentfill}%
\pgfsetfillopacity{0.800000}%
\pgfsetlinewidth{1.003750pt}%
\definecolor{currentstroke}{rgb}{0.800000,0.800000,0.800000}%
\pgfsetstrokecolor{currentstroke}%
\pgfsetstrokeopacity{0.800000}%
\pgfsetdash{}{0pt}%
\pgfpathmoveto{\pgfqpoint{2.104889in}{3.216678in}}%
\pgfpathlineto{\pgfqpoint{3.698778in}{3.216678in}}%
\pgfpathquadraticcurveto{\pgfqpoint{3.726556in}{3.216678in}}{\pgfqpoint{3.726556in}{3.244456in}}%
\pgfpathlineto{\pgfqpoint{3.726556in}{3.811400in}}%
\pgfpathquadraticcurveto{\pgfqpoint{3.726556in}{3.839178in}}{\pgfqpoint{3.698778in}{3.839178in}}%
\pgfpathlineto{\pgfqpoint{2.104889in}{3.839178in}}%
\pgfpathquadraticcurveto{\pgfqpoint{2.077111in}{3.839178in}}{\pgfqpoint{2.077111in}{3.811400in}}%
\pgfpathlineto{\pgfqpoint{2.077111in}{3.244456in}}%
\pgfpathquadraticcurveto{\pgfqpoint{2.077111in}{3.216678in}}{\pgfqpoint{2.104889in}{3.216678in}}%
\pgfpathclose%
\pgfusepath{stroke,fill}%
\end{pgfscope}%
\begin{pgfscope}%
\pgfsetrectcap%
\pgfsetroundjoin%
\pgfsetlinewidth{1.505625pt}%
\definecolor{currentstroke}{rgb}{0.121569,0.466667,0.705882}%
\pgfsetstrokecolor{currentstroke}%
\pgfsetdash{}{0pt}%
\pgfpathmoveto{\pgfqpoint{2.132667in}{3.735011in}}%
\pgfpathlineto{\pgfqpoint{2.410444in}{3.735011in}}%
\pgfusepath{stroke}%
\end{pgfscope}%
\begin{pgfscope}%
\definecolor{textcolor}{rgb}{0.000000,0.000000,0.000000}%
\pgfsetstrokecolor{textcolor}%
\pgfsetfillcolor{textcolor}%
\pgftext[x=2.521555in,y=3.686400in,left,base]{\color{textcolor}\rmfamily\fontsize{10.000000}{12.000000}\selectfont Ground truth}%
\end{pgfscope}%
\begin{pgfscope}%
\pgfsetbuttcap%
\pgfsetroundjoin%
\definecolor{currentfill}{rgb}{0.121569,0.466667,0.705882}%
\pgfsetfillcolor{currentfill}%
\pgfsetlinewidth{1.003750pt}%
\definecolor{currentstroke}{rgb}{0.121569,0.466667,0.705882}%
\pgfsetstrokecolor{currentstroke}%
\pgfsetdash{}{0pt}%
\pgfsys@defobject{currentmarker}{\pgfqpoint{-0.031056in}{-0.031056in}}{\pgfqpoint{0.031056in}{0.031056in}}{%
\pgfpathmoveto{\pgfqpoint{0.000000in}{-0.031056in}}%
\pgfpathcurveto{\pgfqpoint{0.008236in}{-0.031056in}}{\pgfqpoint{0.016136in}{-0.027784in}}{\pgfqpoint{0.021960in}{-0.021960in}}%
\pgfpathcurveto{\pgfqpoint{0.027784in}{-0.016136in}}{\pgfqpoint{0.031056in}{-0.008236in}}{\pgfqpoint{0.031056in}{0.000000in}}%
\pgfpathcurveto{\pgfqpoint{0.031056in}{0.008236in}}{\pgfqpoint{0.027784in}{0.016136in}}{\pgfqpoint{0.021960in}{0.021960in}}%
\pgfpathcurveto{\pgfqpoint{0.016136in}{0.027784in}}{\pgfqpoint{0.008236in}{0.031056in}}{\pgfqpoint{0.000000in}{0.031056in}}%
\pgfpathcurveto{\pgfqpoint{-0.008236in}{0.031056in}}{\pgfqpoint{-0.016136in}{0.027784in}}{\pgfqpoint{-0.021960in}{0.021960in}}%
\pgfpathcurveto{\pgfqpoint{-0.027784in}{0.016136in}}{\pgfqpoint{-0.031056in}{0.008236in}}{\pgfqpoint{-0.031056in}{0.000000in}}%
\pgfpathcurveto{\pgfqpoint{-0.031056in}{-0.008236in}}{\pgfqpoint{-0.027784in}{-0.016136in}}{\pgfqpoint{-0.021960in}{-0.021960in}}%
\pgfpathcurveto{\pgfqpoint{-0.016136in}{-0.027784in}}{\pgfqpoint{-0.008236in}{-0.031056in}}{\pgfqpoint{0.000000in}{-0.031056in}}%
\pgfpathclose%
\pgfusepath{stroke,fill}%
}%
\begin{pgfscope}%
\pgfsys@transformshift{2.271555in}{3.529248in}%
\pgfsys@useobject{currentmarker}{}%
\end{pgfscope}%
\end{pgfscope}%
\begin{pgfscope}%
\definecolor{textcolor}{rgb}{0.000000,0.000000,0.000000}%
\pgfsetstrokecolor{textcolor}%
\pgfsetfillcolor{textcolor}%
\pgftext[x=2.521555in,y=3.492789in,left,base]{\color{textcolor}\rmfamily\fontsize{10.000000}{12.000000}\selectfont Estimated position}%
\end{pgfscope}%
\begin{pgfscope}%
\pgfsetbuttcap%
\pgfsetroundjoin%
\definecolor{currentfill}{rgb}{1.000000,0.498039,0.054902}%
\pgfsetfillcolor{currentfill}%
\pgfsetlinewidth{1.003750pt}%
\definecolor{currentstroke}{rgb}{1.000000,0.498039,0.054902}%
\pgfsetstrokecolor{currentstroke}%
\pgfsetdash{}{0pt}%
\pgfsys@defobject{currentmarker}{\pgfqpoint{-0.031056in}{-0.031056in}}{\pgfqpoint{0.031056in}{0.031056in}}{%
\pgfpathmoveto{\pgfqpoint{0.000000in}{-0.031056in}}%
\pgfpathcurveto{\pgfqpoint{0.008236in}{-0.031056in}}{\pgfqpoint{0.016136in}{-0.027784in}}{\pgfqpoint{0.021960in}{-0.021960in}}%
\pgfpathcurveto{\pgfqpoint{0.027784in}{-0.016136in}}{\pgfqpoint{0.031056in}{-0.008236in}}{\pgfqpoint{0.031056in}{0.000000in}}%
\pgfpathcurveto{\pgfqpoint{0.031056in}{0.008236in}}{\pgfqpoint{0.027784in}{0.016136in}}{\pgfqpoint{0.021960in}{0.021960in}}%
\pgfpathcurveto{\pgfqpoint{0.016136in}{0.027784in}}{\pgfqpoint{0.008236in}{0.031056in}}{\pgfqpoint{0.000000in}{0.031056in}}%
\pgfpathcurveto{\pgfqpoint{-0.008236in}{0.031056in}}{\pgfqpoint{-0.016136in}{0.027784in}}{\pgfqpoint{-0.021960in}{0.021960in}}%
\pgfpathcurveto{\pgfqpoint{-0.027784in}{0.016136in}}{\pgfqpoint{-0.031056in}{0.008236in}}{\pgfqpoint{-0.031056in}{0.000000in}}%
\pgfpathcurveto{\pgfqpoint{-0.031056in}{-0.008236in}}{\pgfqpoint{-0.027784in}{-0.016136in}}{\pgfqpoint{-0.021960in}{-0.021960in}}%
\pgfpathcurveto{\pgfqpoint{-0.016136in}{-0.027784in}}{\pgfqpoint{-0.008236in}{-0.031056in}}{\pgfqpoint{0.000000in}{-0.031056in}}%
\pgfpathclose%
\pgfusepath{stroke,fill}%
}%
\begin{pgfscope}%
\pgfsys@transformshift{2.271555in}{3.335637in}%
\pgfsys@useobject{currentmarker}{}%
\end{pgfscope}%
\end{pgfscope}%
\begin{pgfscope}%
\definecolor{textcolor}{rgb}{0.000000,0.000000,0.000000}%
\pgfsetstrokecolor{textcolor}%
\pgfsetfillcolor{textcolor}%
\pgftext[x=2.521555in,y=3.299178in,left,base]{\color{textcolor}\rmfamily\fontsize{10.000000}{12.000000}\selectfont Estimated turn}%
\end{pgfscope}%
\end{pgfpicture}%
\makeatother%
\endgroup%
}
%         \caption{ROLEQ's 3D position estimation had the lowest turn error for the 4-meter line experiment.}
%         \label{fig:line28_3D}
%     \end{subfigure}
%     \caption{Position estimation by the best performing algorithms in the 4-meter line experiment.}
%     \label{fig:line28}
% \end{figure}

% \subsection{Triangle}

% The line shape consisted of moving the inertial system in a straight line for a determined distance. 3-line distances were tested: 4, 16, and 28 meter. The results are shown below:

% \subsubsection{4 meter}

% For the 16-meter line experiment, the Mahony algorithm which had the lowest displacement error with an average of 0.48 meters (16\% of error margin), and ROLEQ with an average of 0.24 meters of turn error (7\% of error margin).

% \begin{figure}[!h]
%     \centering
%     \begin{table}[H]
    \begin{center}
        \begin{tabular}[t]{lcccc}
            \hline
            Algorithm                   & Displacement Error[$m$] & Displacement Error[\%]      & Turn Error[$m$]  & Turn Error[\%]             \\
            \hline 
            AngularRate            & 2.36  & 19.70 & 8.02 & 66.85              \\            AQUA            & 2.05  & 17.09 & 8.48 & 70.67              \\            Complementary            & 0.79  & 6.56 & 3.47 & 28.93              \\            Davenport            & 2.10  & 17.53 & 2.77 & 23.08              \\            EKF            & 1.05  & 8.78 & 2.89 & 24.12              \\            FAMC            & 0.50  & 4.20 & 7.09 & 59.12              \\            FLAE            & 2.11  & 17.57 & 2.76 & 23.03              \\            Fourati            & 2.63  & 21.90 & 7.62 & 63.48              \\            Madgwick            & 1.64  & 13.66 & 2.29 & 19.09              \\            Mahony            & 0.77  & 6.38 & 2.87 & 23.96              \\            OLEQ            & 0.76  & 6.29 & 3.10 & 25.81              \\            QUEST            & 2.00  & 16.69 & 6.87 & 57.27              \\            ROLEQ            & 0.65  & 5.44 & 3.36 & 27.96              \\            SAAM            & 2.11  & 17.56 & 2.88 & 24.01              \\            Tilt            & 2.11  & 17.56 & 2.88 & 24.01              \\
            \hline
            Average & 1.58 & 13.13 & 4.49 & 37.43
        \end{tabular}
        \caption{Accelerometer Specifications. }
        \label{tab:accelerometer_specification}
    \end{center}
\end{table}
% \end{figure}

% \begin{figure}[!h]
%     \centering
%     \begin{subfigure}{0.49\textwidth}
%         \centering
%         \resizebox{1\linewidth}{!}{%% Creator: Matplotlib, PGF backend
%%
%% To include the figure in your LaTeX document, write
%%   \input{<filename>.pgf}
%%
%% Make sure the required packages are loaded in your preamble
%%   \usepackage{pgf}
%%
%% and, on pdftex
%%   \usepackage[utf8]{inputenc}\DeclareUnicodeCharacter{2212}{-}
%%
%% or, on luatex and xetex
%%   \usepackage{unicode-math}
%%
%% Figures using additional raster images can only be included by \input if
%% they are in the same directory as the main LaTeX file. For loading figures
%% from other directories you can use the `import` package
%%   \usepackage{import}
%%
%% and then include the figures with
%%   \import{<path to file>}{<filename>.pgf}
%%
%% Matplotlib used the following preamble
%%   \usepackage{fontspec}
%%
\begingroup%
\makeatletter%
\begin{pgfpicture}%
\pgfpathrectangle{\pgfpointorigin}{\pgfqpoint{4.342355in}{4.209289in}}%
\pgfusepath{use as bounding box, clip}%
\begin{pgfscope}%
\pgfsetbuttcap%
\pgfsetmiterjoin%
\definecolor{currentfill}{rgb}{1.000000,1.000000,1.000000}%
\pgfsetfillcolor{currentfill}%
\pgfsetlinewidth{0.000000pt}%
\definecolor{currentstroke}{rgb}{1.000000,1.000000,1.000000}%
\pgfsetstrokecolor{currentstroke}%
\pgfsetdash{}{0pt}%
\pgfpathmoveto{\pgfqpoint{0.000000in}{0.000000in}}%
\pgfpathlineto{\pgfqpoint{4.342355in}{0.000000in}}%
\pgfpathlineto{\pgfqpoint{4.342355in}{4.209289in}}%
\pgfpathlineto{\pgfqpoint{0.000000in}{4.209289in}}%
\pgfpathclose%
\pgfusepath{fill}%
\end{pgfscope}%
\begin{pgfscope}%
\pgfsetbuttcap%
\pgfsetmiterjoin%
\definecolor{currentfill}{rgb}{1.000000,1.000000,1.000000}%
\pgfsetfillcolor{currentfill}%
\pgfsetlinewidth{0.000000pt}%
\definecolor{currentstroke}{rgb}{0.000000,0.000000,0.000000}%
\pgfsetstrokecolor{currentstroke}%
\pgfsetstrokeopacity{0.000000}%
\pgfsetdash{}{0pt}%
\pgfpathmoveto{\pgfqpoint{0.100000in}{0.212622in}}%
\pgfpathlineto{\pgfqpoint{3.796000in}{0.212622in}}%
\pgfpathlineto{\pgfqpoint{3.796000in}{3.908622in}}%
\pgfpathlineto{\pgfqpoint{0.100000in}{3.908622in}}%
\pgfpathclose%
\pgfusepath{fill}%
\end{pgfscope}%
\begin{pgfscope}%
\pgfsetbuttcap%
\pgfsetmiterjoin%
\definecolor{currentfill}{rgb}{0.950000,0.950000,0.950000}%
\pgfsetfillcolor{currentfill}%
\pgfsetfillopacity{0.500000}%
\pgfsetlinewidth{1.003750pt}%
\definecolor{currentstroke}{rgb}{0.950000,0.950000,0.950000}%
\pgfsetstrokecolor{currentstroke}%
\pgfsetstrokeopacity{0.500000}%
\pgfsetdash{}{0pt}%
\pgfpathmoveto{\pgfqpoint{0.379073in}{1.123938in}}%
\pgfpathlineto{\pgfqpoint{1.599613in}{2.147018in}}%
\pgfpathlineto{\pgfqpoint{1.582647in}{3.622484in}}%
\pgfpathlineto{\pgfqpoint{0.303698in}{2.689165in}}%
\pgfusepath{stroke,fill}%
\end{pgfscope}%
\begin{pgfscope}%
\pgfsetbuttcap%
\pgfsetmiterjoin%
\definecolor{currentfill}{rgb}{0.900000,0.900000,0.900000}%
\pgfsetfillcolor{currentfill}%
\pgfsetfillopacity{0.500000}%
\pgfsetlinewidth{1.003750pt}%
\definecolor{currentstroke}{rgb}{0.900000,0.900000,0.900000}%
\pgfsetstrokecolor{currentstroke}%
\pgfsetstrokeopacity{0.500000}%
\pgfsetdash{}{0pt}%
\pgfpathmoveto{\pgfqpoint{1.599613in}{2.147018in}}%
\pgfpathlineto{\pgfqpoint{3.558144in}{1.577751in}}%
\pgfpathlineto{\pgfqpoint{3.628038in}{3.104037in}}%
\pgfpathlineto{\pgfqpoint{1.582647in}{3.622484in}}%
\pgfusepath{stroke,fill}%
\end{pgfscope}%
\begin{pgfscope}%
\pgfsetbuttcap%
\pgfsetmiterjoin%
\definecolor{currentfill}{rgb}{0.925000,0.925000,0.925000}%
\pgfsetfillcolor{currentfill}%
\pgfsetfillopacity{0.500000}%
\pgfsetlinewidth{1.003750pt}%
\definecolor{currentstroke}{rgb}{0.925000,0.925000,0.925000}%
\pgfsetstrokecolor{currentstroke}%
\pgfsetstrokeopacity{0.500000}%
\pgfsetdash{}{0pt}%
\pgfpathmoveto{\pgfqpoint{0.379073in}{1.123938in}}%
\pgfpathlineto{\pgfqpoint{2.455212in}{0.445871in}}%
\pgfpathlineto{\pgfqpoint{3.558144in}{1.577751in}}%
\pgfpathlineto{\pgfqpoint{1.599613in}{2.147018in}}%
\pgfusepath{stroke,fill}%
\end{pgfscope}%
\begin{pgfscope}%
\pgfsetrectcap%
\pgfsetroundjoin%
\pgfsetlinewidth{0.803000pt}%
\definecolor{currentstroke}{rgb}{0.000000,0.000000,0.000000}%
\pgfsetstrokecolor{currentstroke}%
\pgfsetdash{}{0pt}%
\pgfpathmoveto{\pgfqpoint{0.379073in}{1.123938in}}%
\pgfpathlineto{\pgfqpoint{2.455212in}{0.445871in}}%
\pgfusepath{stroke}%
\end{pgfscope}%
\begin{pgfscope}%
\definecolor{textcolor}{rgb}{0.000000,0.000000,0.000000}%
\pgfsetstrokecolor{textcolor}%
\pgfsetfillcolor{textcolor}%
\pgftext[x=0.730374in, y=0.408886in, left, base,rotate=341.912962]{\color{textcolor}\rmfamily\fontsize{10.000000}{12.000000}\selectfont Position X [\(\displaystyle m\)]}%
\end{pgfscope}%
\begin{pgfscope}%
\pgfsetbuttcap%
\pgfsetroundjoin%
\pgfsetlinewidth{0.803000pt}%
\definecolor{currentstroke}{rgb}{0.690196,0.690196,0.690196}%
\pgfsetstrokecolor{currentstroke}%
\pgfsetdash{}{0pt}%
\pgfpathmoveto{\pgfqpoint{0.765197in}{0.997830in}}%
\pgfpathlineto{\pgfqpoint{1.965175in}{2.040764in}}%
\pgfpathlineto{\pgfqpoint{1.963766in}{3.525881in}}%
\pgfusepath{stroke}%
\end{pgfscope}%
\begin{pgfscope}%
\pgfsetbuttcap%
\pgfsetroundjoin%
\pgfsetlinewidth{0.803000pt}%
\definecolor{currentstroke}{rgb}{0.690196,0.690196,0.690196}%
\pgfsetstrokecolor{currentstroke}%
\pgfsetdash{}{0pt}%
\pgfpathmoveto{\pgfqpoint{1.119554in}{0.882097in}}%
\pgfpathlineto{\pgfqpoint{2.300133in}{1.943405in}}%
\pgfpathlineto{\pgfqpoint{2.313242in}{3.437299in}}%
\pgfusepath{stroke}%
\end{pgfscope}%
\begin{pgfscope}%
\pgfsetbuttcap%
\pgfsetroundjoin%
\pgfsetlinewidth{0.803000pt}%
\definecolor{currentstroke}{rgb}{0.690196,0.690196,0.690196}%
\pgfsetstrokecolor{currentstroke}%
\pgfsetdash{}{0pt}%
\pgfpathmoveto{\pgfqpoint{1.480456in}{0.764226in}}%
\pgfpathlineto{\pgfqpoint{2.640759in}{1.844398in}}%
\pgfpathlineto{\pgfqpoint{2.668892in}{3.347152in}}%
\pgfusepath{stroke}%
\end{pgfscope}%
\begin{pgfscope}%
\pgfsetbuttcap%
\pgfsetroundjoin%
\pgfsetlinewidth{0.803000pt}%
\definecolor{currentstroke}{rgb}{0.690196,0.690196,0.690196}%
\pgfsetstrokecolor{currentstroke}%
\pgfsetdash{}{0pt}%
\pgfpathmoveto{\pgfqpoint{1.848087in}{0.644158in}}%
\pgfpathlineto{\pgfqpoint{2.987199in}{1.743702in}}%
\pgfpathlineto{\pgfqpoint{3.030878in}{3.255399in}}%
\pgfusepath{stroke}%
\end{pgfscope}%
\begin{pgfscope}%
\pgfsetbuttcap%
\pgfsetroundjoin%
\pgfsetlinewidth{0.803000pt}%
\definecolor{currentstroke}{rgb}{0.690196,0.690196,0.690196}%
\pgfsetstrokecolor{currentstroke}%
\pgfsetdash{}{0pt}%
\pgfpathmoveto{\pgfqpoint{2.222635in}{0.521830in}}%
\pgfpathlineto{\pgfqpoint{3.339601in}{1.641273in}}%
\pgfpathlineto{\pgfqpoint{3.399374in}{3.161996in}}%
\pgfusepath{stroke}%
\end{pgfscope}%
\begin{pgfscope}%
\pgfsetrectcap%
\pgfsetroundjoin%
\pgfsetlinewidth{0.803000pt}%
\definecolor{currentstroke}{rgb}{0.000000,0.000000,0.000000}%
\pgfsetstrokecolor{currentstroke}%
\pgfsetdash{}{0pt}%
\pgfpathmoveto{\pgfqpoint{0.775652in}{1.006916in}}%
\pgfpathlineto{\pgfqpoint{0.744242in}{0.979617in}}%
\pgfusepath{stroke}%
\end{pgfscope}%
\begin{pgfscope}%
\definecolor{textcolor}{rgb}{0.000000,0.000000,0.000000}%
\pgfsetstrokecolor{textcolor}%
\pgfsetfillcolor{textcolor}%
\pgftext[x=0.660891in,y=0.777824in,,top]{\color{textcolor}\rmfamily\fontsize{10.000000}{12.000000}\selectfont \(\displaystyle {0}\)}%
\end{pgfscope}%
\begin{pgfscope}%
\pgfsetrectcap%
\pgfsetroundjoin%
\pgfsetlinewidth{0.803000pt}%
\definecolor{currentstroke}{rgb}{0.000000,0.000000,0.000000}%
\pgfsetstrokecolor{currentstroke}%
\pgfsetdash{}{0pt}%
\pgfpathmoveto{\pgfqpoint{1.129848in}{0.891350in}}%
\pgfpathlineto{\pgfqpoint{1.098922in}{0.863549in}}%
\pgfusepath{stroke}%
\end{pgfscope}%
\begin{pgfscope}%
\definecolor{textcolor}{rgb}{0.000000,0.000000,0.000000}%
\pgfsetstrokecolor{textcolor}%
\pgfsetfillcolor{textcolor}%
\pgftext[x=1.015628in,y=0.659633in,,top]{\color{textcolor}\rmfamily\fontsize{10.000000}{12.000000}\selectfont \(\displaystyle {1}\)}%
\end{pgfscope}%
\begin{pgfscope}%
\pgfsetrectcap%
\pgfsetroundjoin%
\pgfsetlinewidth{0.803000pt}%
\definecolor{currentstroke}{rgb}{0.000000,0.000000,0.000000}%
\pgfsetstrokecolor{currentstroke}%
\pgfsetdash{}{0pt}%
\pgfpathmoveto{\pgfqpoint{1.490581in}{0.773651in}}%
\pgfpathlineto{\pgfqpoint{1.460162in}{0.745334in}}%
\pgfusepath{stroke}%
\end{pgfscope}%
\begin{pgfscope}%
\definecolor{textcolor}{rgb}{0.000000,0.000000,0.000000}%
\pgfsetstrokecolor{textcolor}%
\pgfsetfillcolor{textcolor}%
\pgftext[x=1.376943in,y=0.539251in,,top]{\color{textcolor}\rmfamily\fontsize{10.000000}{12.000000}\selectfont \(\displaystyle {2}\)}%
\end{pgfscope}%
\begin{pgfscope}%
\pgfsetrectcap%
\pgfsetroundjoin%
\pgfsetlinewidth{0.803000pt}%
\definecolor{currentstroke}{rgb}{0.000000,0.000000,0.000000}%
\pgfsetstrokecolor{currentstroke}%
\pgfsetdash{}{0pt}%
\pgfpathmoveto{\pgfqpoint{1.858034in}{0.653760in}}%
\pgfpathlineto{\pgfqpoint{1.828148in}{0.624911in}}%
\pgfusepath{stroke}%
\end{pgfscope}%
\begin{pgfscope}%
\definecolor{textcolor}{rgb}{0.000000,0.000000,0.000000}%
\pgfsetstrokecolor{textcolor}%
\pgfsetfillcolor{textcolor}%
\pgftext[x=1.745019in,y=0.416616in,,top]{\color{textcolor}\rmfamily\fontsize{10.000000}{12.000000}\selectfont \(\displaystyle {3}\)}%
\end{pgfscope}%
\begin{pgfscope}%
\pgfsetrectcap%
\pgfsetroundjoin%
\pgfsetlinewidth{0.803000pt}%
\definecolor{currentstroke}{rgb}{0.000000,0.000000,0.000000}%
\pgfsetstrokecolor{currentstroke}%
\pgfsetdash{}{0pt}%
\pgfpathmoveto{\pgfqpoint{2.232397in}{0.531614in}}%
\pgfpathlineto{\pgfqpoint{2.203068in}{0.502220in}}%
\pgfusepath{stroke}%
\end{pgfscope}%
\begin{pgfscope}%
\definecolor{textcolor}{rgb}{0.000000,0.000000,0.000000}%
\pgfsetstrokecolor{textcolor}%
\pgfsetfillcolor{textcolor}%
\pgftext[x=2.120049in,y=0.291664in,,top]{\color{textcolor}\rmfamily\fontsize{10.000000}{12.000000}\selectfont \(\displaystyle {4}\)}%
\end{pgfscope}%
\begin{pgfscope}%
\pgfsetrectcap%
\pgfsetroundjoin%
\pgfsetlinewidth{0.803000pt}%
\definecolor{currentstroke}{rgb}{0.000000,0.000000,0.000000}%
\pgfsetstrokecolor{currentstroke}%
\pgfsetdash{}{0pt}%
\pgfpathmoveto{\pgfqpoint{3.558144in}{1.577751in}}%
\pgfpathlineto{\pgfqpoint{2.455212in}{0.445871in}}%
\pgfusepath{stroke}%
\end{pgfscope}%
\begin{pgfscope}%
\definecolor{textcolor}{rgb}{0.000000,0.000000,0.000000}%
\pgfsetstrokecolor{textcolor}%
\pgfsetfillcolor{textcolor}%
\pgftext[x=3.120747in, y=0.305657in, left, base,rotate=45.742112]{\color{textcolor}\rmfamily\fontsize{10.000000}{12.000000}\selectfont Position Y [\(\displaystyle m\)]}%
\end{pgfscope}%
\begin{pgfscope}%
\pgfsetbuttcap%
\pgfsetroundjoin%
\pgfsetlinewidth{0.803000pt}%
\definecolor{currentstroke}{rgb}{0.690196,0.690196,0.690196}%
\pgfsetstrokecolor{currentstroke}%
\pgfsetdash{}{0pt}%
\pgfpathmoveto{\pgfqpoint{0.484243in}{2.820918in}}%
\pgfpathlineto{\pgfqpoint{0.550812in}{1.267892in}}%
\pgfpathlineto{\pgfqpoint{2.610992in}{0.605739in}}%
\pgfusepath{stroke}%
\end{pgfscope}%
\begin{pgfscope}%
\pgfsetbuttcap%
\pgfsetroundjoin%
\pgfsetlinewidth{0.803000pt}%
\definecolor{currentstroke}{rgb}{0.690196,0.690196,0.690196}%
\pgfsetstrokecolor{currentstroke}%
\pgfsetdash{}{0pt}%
\pgfpathmoveto{\pgfqpoint{0.655961in}{2.946231in}}%
\pgfpathlineto{\pgfqpoint{0.714325in}{1.404952in}}%
\pgfpathlineto{\pgfqpoint{2.759131in}{0.757766in}}%
\pgfusepath{stroke}%
\end{pgfscope}%
\begin{pgfscope}%
\pgfsetbuttcap%
\pgfsetroundjoin%
\pgfsetlinewidth{0.803000pt}%
\definecolor{currentstroke}{rgb}{0.690196,0.690196,0.690196}%
\pgfsetstrokecolor{currentstroke}%
\pgfsetdash{}{0pt}%
\pgfpathmoveto{\pgfqpoint{0.823749in}{3.068675in}}%
\pgfpathlineto{\pgfqpoint{0.874257in}{1.539010in}}%
\pgfpathlineto{\pgfqpoint{2.903855in}{0.906289in}}%
\pgfusepath{stroke}%
\end{pgfscope}%
\begin{pgfscope}%
\pgfsetbuttcap%
\pgfsetroundjoin%
\pgfsetlinewidth{0.803000pt}%
\definecolor{currentstroke}{rgb}{0.690196,0.690196,0.690196}%
\pgfsetstrokecolor{currentstroke}%
\pgfsetdash{}{0pt}%
\pgfpathmoveto{\pgfqpoint{0.987740in}{3.188348in}}%
\pgfpathlineto{\pgfqpoint{1.030723in}{1.670163in}}%
\pgfpathlineto{\pgfqpoint{3.045282in}{1.051428in}}%
\pgfusepath{stroke}%
\end{pgfscope}%
\begin{pgfscope}%
\pgfsetbuttcap%
\pgfsetroundjoin%
\pgfsetlinewidth{0.803000pt}%
\definecolor{currentstroke}{rgb}{0.690196,0.690196,0.690196}%
\pgfsetstrokecolor{currentstroke}%
\pgfsetdash{}{0pt}%
\pgfpathmoveto{\pgfqpoint{1.148062in}{3.305343in}}%
\pgfpathlineto{\pgfqpoint{1.183835in}{1.798505in}}%
\pgfpathlineto{\pgfqpoint{3.183522in}{1.193296in}}%
\pgfusepath{stroke}%
\end{pgfscope}%
\begin{pgfscope}%
\pgfsetbuttcap%
\pgfsetroundjoin%
\pgfsetlinewidth{0.803000pt}%
\definecolor{currentstroke}{rgb}{0.690196,0.690196,0.690196}%
\pgfsetstrokecolor{currentstroke}%
\pgfsetdash{}{0pt}%
\pgfpathmoveto{\pgfqpoint{1.304835in}{3.419750in}}%
\pgfpathlineto{\pgfqpoint{1.333701in}{1.924125in}}%
\pgfpathlineto{\pgfqpoint{3.318683in}{1.332004in}}%
\pgfusepath{stroke}%
\end{pgfscope}%
\begin{pgfscope}%
\pgfsetbuttcap%
\pgfsetroundjoin%
\pgfsetlinewidth{0.803000pt}%
\definecolor{currentstroke}{rgb}{0.690196,0.690196,0.690196}%
\pgfsetstrokecolor{currentstroke}%
\pgfsetdash{}{0pt}%
\pgfpathmoveto{\pgfqpoint{1.458177in}{3.531652in}}%
\pgfpathlineto{\pgfqpoint{1.480421in}{2.047109in}}%
\pgfpathlineto{\pgfqpoint{3.450865in}{1.467656in}}%
\pgfusepath{stroke}%
\end{pgfscope}%
\begin{pgfscope}%
\pgfsetrectcap%
\pgfsetroundjoin%
\pgfsetlinewidth{0.803000pt}%
\definecolor{currentstroke}{rgb}{0.000000,0.000000,0.000000}%
\pgfsetstrokecolor{currentstroke}%
\pgfsetdash{}{0pt}%
\pgfpathmoveto{\pgfqpoint{2.593636in}{0.611317in}}%
\pgfpathlineto{\pgfqpoint{2.645748in}{0.594568in}}%
\pgfusepath{stroke}%
\end{pgfscope}%
\begin{pgfscope}%
\definecolor{textcolor}{rgb}{0.000000,0.000000,0.000000}%
\pgfsetstrokecolor{textcolor}%
\pgfsetfillcolor{textcolor}%
\pgftext[x=2.788824in,y=0.420513in,,top]{\color{textcolor}\rmfamily\fontsize{10.000000}{12.000000}\selectfont \(\displaystyle {-1}\)}%
\end{pgfscope}%
\begin{pgfscope}%
\pgfsetrectcap%
\pgfsetroundjoin%
\pgfsetlinewidth{0.803000pt}%
\definecolor{currentstroke}{rgb}{0.000000,0.000000,0.000000}%
\pgfsetstrokecolor{currentstroke}%
\pgfsetdash{}{0pt}%
\pgfpathmoveto{\pgfqpoint{2.741915in}{0.763215in}}%
\pgfpathlineto{\pgfqpoint{2.793607in}{0.746855in}}%
\pgfusepath{stroke}%
\end{pgfscope}%
\begin{pgfscope}%
\definecolor{textcolor}{rgb}{0.000000,0.000000,0.000000}%
\pgfsetstrokecolor{textcolor}%
\pgfsetfillcolor{textcolor}%
\pgftext[x=2.934976in,y=0.574790in,,top]{\color{textcolor}\rmfamily\fontsize{10.000000}{12.000000}\selectfont \(\displaystyle {0}\)}%
\end{pgfscope}%
\begin{pgfscope}%
\pgfsetrectcap%
\pgfsetroundjoin%
\pgfsetlinewidth{0.803000pt}%
\definecolor{currentstroke}{rgb}{0.000000,0.000000,0.000000}%
\pgfsetstrokecolor{currentstroke}%
\pgfsetdash{}{0pt}%
\pgfpathmoveto{\pgfqpoint{2.886777in}{0.911613in}}%
\pgfpathlineto{\pgfqpoint{2.938055in}{0.895628in}}%
\pgfusepath{stroke}%
\end{pgfscope}%
\begin{pgfscope}%
\definecolor{textcolor}{rgb}{0.000000,0.000000,0.000000}%
\pgfsetstrokecolor{textcolor}%
\pgfsetfillcolor{textcolor}%
\pgftext[x=3.077757in,y=0.725509in,,top]{\color{textcolor}\rmfamily\fontsize{10.000000}{12.000000}\selectfont \(\displaystyle {1}\)}%
\end{pgfscope}%
\begin{pgfscope}%
\pgfsetrectcap%
\pgfsetroundjoin%
\pgfsetlinewidth{0.803000pt}%
\definecolor{currentstroke}{rgb}{0.000000,0.000000,0.000000}%
\pgfsetstrokecolor{currentstroke}%
\pgfsetdash{}{0pt}%
\pgfpathmoveto{\pgfqpoint{3.028340in}{1.056631in}}%
\pgfpathlineto{\pgfqpoint{3.079209in}{1.041008in}}%
\pgfusepath{stroke}%
\end{pgfscope}%
\begin{pgfscope}%
\definecolor{textcolor}{rgb}{0.000000,0.000000,0.000000}%
\pgfsetstrokecolor{textcolor}%
\pgfsetfillcolor{textcolor}%
\pgftext[x=3.217282in,y=0.872792in,,top]{\color{textcolor}\rmfamily\fontsize{10.000000}{12.000000}\selectfont \(\displaystyle {2}\)}%
\end{pgfscope}%
\begin{pgfscope}%
\pgfsetrectcap%
\pgfsetroundjoin%
\pgfsetlinewidth{0.803000pt}%
\definecolor{currentstroke}{rgb}{0.000000,0.000000,0.000000}%
\pgfsetstrokecolor{currentstroke}%
\pgfsetdash{}{0pt}%
\pgfpathmoveto{\pgfqpoint{3.166715in}{1.198383in}}%
\pgfpathlineto{\pgfqpoint{3.217180in}{1.183110in}}%
\pgfusepath{stroke}%
\end{pgfscope}%
\begin{pgfscope}%
\definecolor{textcolor}{rgb}{0.000000,0.000000,0.000000}%
\pgfsetstrokecolor{textcolor}%
\pgfsetfillcolor{textcolor}%
\pgftext[x=3.353662in,y=1.016755in,,top]{\color{textcolor}\rmfamily\fontsize{10.000000}{12.000000}\selectfont \(\displaystyle {3}\)}%
\end{pgfscope}%
\begin{pgfscope}%
\pgfsetrectcap%
\pgfsetroundjoin%
\pgfsetlinewidth{0.803000pt}%
\definecolor{currentstroke}{rgb}{0.000000,0.000000,0.000000}%
\pgfsetstrokecolor{currentstroke}%
\pgfsetdash{}{0pt}%
\pgfpathmoveto{\pgfqpoint{3.302008in}{1.336979in}}%
\pgfpathlineto{\pgfqpoint{3.352074in}{1.322044in}}%
\pgfusepath{stroke}%
\end{pgfscope}%
\begin{pgfscope}%
\definecolor{textcolor}{rgb}{0.000000,0.000000,0.000000}%
\pgfsetstrokecolor{textcolor}%
\pgfsetfillcolor{textcolor}%
\pgftext[x=3.487003in,y=1.157508in,,top]{\color{textcolor}\rmfamily\fontsize{10.000000}{12.000000}\selectfont \(\displaystyle {4}\)}%
\end{pgfscope}%
\begin{pgfscope}%
\pgfsetrectcap%
\pgfsetroundjoin%
\pgfsetlinewidth{0.803000pt}%
\definecolor{currentstroke}{rgb}{0.000000,0.000000,0.000000}%
\pgfsetstrokecolor{currentstroke}%
\pgfsetdash{}{0pt}%
\pgfpathmoveto{\pgfqpoint{3.434321in}{1.472521in}}%
\pgfpathlineto{\pgfqpoint{3.483994in}{1.457914in}}%
\pgfusepath{stroke}%
\end{pgfscope}%
\begin{pgfscope}%
\definecolor{textcolor}{rgb}{0.000000,0.000000,0.000000}%
\pgfsetstrokecolor{textcolor}%
\pgfsetfillcolor{textcolor}%
\pgftext[x=3.617403in,y=1.295159in,,top]{\color{textcolor}\rmfamily\fontsize{10.000000}{12.000000}\selectfont \(\displaystyle {5}\)}%
\end{pgfscope}%
\begin{pgfscope}%
\pgfsetrectcap%
\pgfsetroundjoin%
\pgfsetlinewidth{0.803000pt}%
\definecolor{currentstroke}{rgb}{0.000000,0.000000,0.000000}%
\pgfsetstrokecolor{currentstroke}%
\pgfsetdash{}{0pt}%
\pgfpathmoveto{\pgfqpoint{3.558144in}{1.577751in}}%
\pgfpathlineto{\pgfqpoint{3.628038in}{3.104037in}}%
\pgfusepath{stroke}%
\end{pgfscope}%
\begin{pgfscope}%
\definecolor{textcolor}{rgb}{0.000000,0.000000,0.000000}%
\pgfsetstrokecolor{textcolor}%
\pgfsetfillcolor{textcolor}%
\pgftext[x=4.167903in, y=1.963517in, left, base,rotate=87.378092]{\color{textcolor}\rmfamily\fontsize{10.000000}{12.000000}\selectfont Position Z [\(\displaystyle m\)]}%
\end{pgfscope}%
\begin{pgfscope}%
\pgfsetbuttcap%
\pgfsetroundjoin%
\pgfsetlinewidth{0.803000pt}%
\definecolor{currentstroke}{rgb}{0.690196,0.690196,0.690196}%
\pgfsetstrokecolor{currentstroke}%
\pgfsetdash{}{0pt}%
\pgfpathmoveto{\pgfqpoint{3.569211in}{1.819427in}}%
\pgfpathlineto{\pgfqpoint{1.596922in}{2.381062in}}%
\pgfpathlineto{\pgfqpoint{0.367155in}{1.371428in}}%
\pgfusepath{stroke}%
\end{pgfscope}%
\begin{pgfscope}%
\pgfsetbuttcap%
\pgfsetroundjoin%
\pgfsetlinewidth{0.803000pt}%
\definecolor{currentstroke}{rgb}{0.690196,0.690196,0.690196}%
\pgfsetstrokecolor{currentstroke}%
\pgfsetdash{}{0pt}%
\pgfpathmoveto{\pgfqpoint{3.579160in}{2.036667in}}%
\pgfpathlineto{\pgfqpoint{1.594504in}{2.591309in}}%
\pgfpathlineto{\pgfqpoint{0.356437in}{1.594007in}}%
\pgfusepath{stroke}%
\end{pgfscope}%
\begin{pgfscope}%
\pgfsetbuttcap%
\pgfsetroundjoin%
\pgfsetlinewidth{0.803000pt}%
\definecolor{currentstroke}{rgb}{0.690196,0.690196,0.690196}%
\pgfsetstrokecolor{currentstroke}%
\pgfsetdash{}{0pt}%
\pgfpathmoveto{\pgfqpoint{3.589234in}{2.256675in}}%
\pgfpathlineto{\pgfqpoint{1.592057in}{2.804106in}}%
\pgfpathlineto{\pgfqpoint{0.345576in}{1.819532in}}%
\pgfusepath{stroke}%
\end{pgfscope}%
\begin{pgfscope}%
\pgfsetbuttcap%
\pgfsetroundjoin%
\pgfsetlinewidth{0.803000pt}%
\definecolor{currentstroke}{rgb}{0.690196,0.690196,0.690196}%
\pgfsetstrokecolor{currentstroke}%
\pgfsetdash{}{0pt}%
\pgfpathmoveto{\pgfqpoint{3.599438in}{2.479505in}}%
\pgfpathlineto{\pgfqpoint{1.589581in}{3.019500in}}%
\pgfpathlineto{\pgfqpoint{0.334571in}{2.048061in}}%
\pgfusepath{stroke}%
\end{pgfscope}%
\begin{pgfscope}%
\pgfsetbuttcap%
\pgfsetroundjoin%
\pgfsetlinewidth{0.803000pt}%
\definecolor{currentstroke}{rgb}{0.690196,0.690196,0.690196}%
\pgfsetstrokecolor{currentstroke}%
\pgfsetdash{}{0pt}%
\pgfpathmoveto{\pgfqpoint{3.609774in}{2.705211in}}%
\pgfpathlineto{\pgfqpoint{1.587073in}{3.237539in}}%
\pgfpathlineto{\pgfqpoint{0.323418in}{2.279654in}}%
\pgfusepath{stroke}%
\end{pgfscope}%
\begin{pgfscope}%
\pgfsetbuttcap%
\pgfsetroundjoin%
\pgfsetlinewidth{0.803000pt}%
\definecolor{currentstroke}{rgb}{0.690196,0.690196,0.690196}%
\pgfsetstrokecolor{currentstroke}%
\pgfsetdash{}{0pt}%
\pgfpathmoveto{\pgfqpoint{3.620244in}{2.933850in}}%
\pgfpathlineto{\pgfqpoint{1.584535in}{3.458272in}}%
\pgfpathlineto{\pgfqpoint{0.312115in}{2.514374in}}%
\pgfusepath{stroke}%
\end{pgfscope}%
\begin{pgfscope}%
\pgfsetrectcap%
\pgfsetroundjoin%
\pgfsetlinewidth{0.803000pt}%
\definecolor{currentstroke}{rgb}{0.000000,0.000000,0.000000}%
\pgfsetstrokecolor{currentstroke}%
\pgfsetdash{}{0pt}%
\pgfpathmoveto{\pgfqpoint{3.552653in}{1.824142in}}%
\pgfpathlineto{\pgfqpoint{3.602368in}{1.809985in}}%
\pgfusepath{stroke}%
\end{pgfscope}%
\begin{pgfscope}%
\definecolor{textcolor}{rgb}{0.000000,0.000000,0.000000}%
\pgfsetstrokecolor{textcolor}%
\pgfsetfillcolor{textcolor}%
\pgftext[x=3.824424in,y=1.855133in,,top]{\color{textcolor}\rmfamily\fontsize{10.000000}{12.000000}\selectfont \(\displaystyle {-2}\)}%
\end{pgfscope}%
\begin{pgfscope}%
\pgfsetrectcap%
\pgfsetroundjoin%
\pgfsetlinewidth{0.803000pt}%
\definecolor{currentstroke}{rgb}{0.000000,0.000000,0.000000}%
\pgfsetstrokecolor{currentstroke}%
\pgfsetdash{}{0pt}%
\pgfpathmoveto{\pgfqpoint{3.562492in}{2.041325in}}%
\pgfpathlineto{\pgfqpoint{3.612534in}{2.027340in}}%
\pgfusepath{stroke}%
\end{pgfscope}%
\begin{pgfscope}%
\definecolor{textcolor}{rgb}{0.000000,0.000000,0.000000}%
\pgfsetstrokecolor{textcolor}%
\pgfsetfillcolor{textcolor}%
\pgftext[x=3.835953in,y=2.071939in,,top]{\color{textcolor}\rmfamily\fontsize{10.000000}{12.000000}\selectfont \(\displaystyle {-1}\)}%
\end{pgfscope}%
\begin{pgfscope}%
\pgfsetrectcap%
\pgfsetroundjoin%
\pgfsetlinewidth{0.803000pt}%
\definecolor{currentstroke}{rgb}{0.000000,0.000000,0.000000}%
\pgfsetstrokecolor{currentstroke}%
\pgfsetdash{}{0pt}%
\pgfpathmoveto{\pgfqpoint{3.572457in}{2.261274in}}%
\pgfpathlineto{\pgfqpoint{3.622830in}{2.247466in}}%
\pgfusepath{stroke}%
\end{pgfscope}%
\begin{pgfscope}%
\definecolor{textcolor}{rgb}{0.000000,0.000000,0.000000}%
\pgfsetstrokecolor{textcolor}%
\pgfsetfillcolor{textcolor}%
\pgftext[x=3.847628in,y=2.291498in,,top]{\color{textcolor}\rmfamily\fontsize{10.000000}{12.000000}\selectfont \(\displaystyle {0}\)}%
\end{pgfscope}%
\begin{pgfscope}%
\pgfsetrectcap%
\pgfsetroundjoin%
\pgfsetlinewidth{0.803000pt}%
\definecolor{currentstroke}{rgb}{0.000000,0.000000,0.000000}%
\pgfsetstrokecolor{currentstroke}%
\pgfsetdash{}{0pt}%
\pgfpathmoveto{\pgfqpoint{3.582549in}{2.484043in}}%
\pgfpathlineto{\pgfqpoint{3.633258in}{2.470419in}}%
\pgfusepath{stroke}%
\end{pgfscope}%
\begin{pgfscope}%
\definecolor{textcolor}{rgb}{0.000000,0.000000,0.000000}%
\pgfsetstrokecolor{textcolor}%
\pgfsetfillcolor{textcolor}%
\pgftext[x=3.859452in,y=2.513865in,,top]{\color{textcolor}\rmfamily\fontsize{10.000000}{12.000000}\selectfont \(\displaystyle {1}\)}%
\end{pgfscope}%
\begin{pgfscope}%
\pgfsetrectcap%
\pgfsetroundjoin%
\pgfsetlinewidth{0.803000pt}%
\definecolor{currentstroke}{rgb}{0.000000,0.000000,0.000000}%
\pgfsetstrokecolor{currentstroke}%
\pgfsetdash{}{0pt}%
\pgfpathmoveto{\pgfqpoint{3.592772in}{2.709686in}}%
\pgfpathlineto{\pgfqpoint{3.643821in}{2.696251in}}%
\pgfusepath{stroke}%
\end{pgfscope}%
\begin{pgfscope}%
\definecolor{textcolor}{rgb}{0.000000,0.000000,0.000000}%
\pgfsetstrokecolor{textcolor}%
\pgfsetfillcolor{textcolor}%
\pgftext[x=3.871429in,y=2.739093in,,top]{\color{textcolor}\rmfamily\fontsize{10.000000}{12.000000}\selectfont \(\displaystyle {2}\)}%
\end{pgfscope}%
\begin{pgfscope}%
\pgfsetrectcap%
\pgfsetroundjoin%
\pgfsetlinewidth{0.803000pt}%
\definecolor{currentstroke}{rgb}{0.000000,0.000000,0.000000}%
\pgfsetstrokecolor{currentstroke}%
\pgfsetdash{}{0pt}%
\pgfpathmoveto{\pgfqpoint{3.603127in}{2.938259in}}%
\pgfpathlineto{\pgfqpoint{3.654521in}{2.925020in}}%
\pgfusepath{stroke}%
\end{pgfscope}%
\begin{pgfscope}%
\definecolor{textcolor}{rgb}{0.000000,0.000000,0.000000}%
\pgfsetstrokecolor{textcolor}%
\pgfsetfillcolor{textcolor}%
\pgftext[x=3.883561in,y=2.967239in,,top]{\color{textcolor}\rmfamily\fontsize{10.000000}{12.000000}\selectfont \(\displaystyle {3}\)}%
\end{pgfscope}%
\begin{pgfscope}%
\pgfpathrectangle{\pgfqpoint{0.100000in}{0.212622in}}{\pgfqpoint{3.696000in}{3.696000in}}%
\pgfusepath{clip}%
\pgfsetrectcap%
\pgfsetroundjoin%
\pgfsetlinewidth{1.505625pt}%
\definecolor{currentstroke}{rgb}{0.121569,0.466667,0.705882}%
\pgfsetstrokecolor{currentstroke}%
\pgfsetdash{}{0pt}%
\pgfpathmoveto{\pgfqpoint{1.076558in}{1.974241in}}%
\pgfpathlineto{\pgfqpoint{1.698168in}{2.483690in}}%
\pgfpathlineto{\pgfqpoint{2.541591in}{1.536224in}}%
\pgfpathlineto{\pgfqpoint{1.076558in}{1.974241in}}%
\pgfusepath{stroke}%
\end{pgfscope}%
\begin{pgfscope}%
\pgfpathrectangle{\pgfqpoint{0.100000in}{0.212622in}}{\pgfqpoint{3.696000in}{3.696000in}}%
\pgfusepath{clip}%
\pgfsetrectcap%
\pgfsetroundjoin%
\pgfsetlinewidth{1.505625pt}%
\definecolor{currentstroke}{rgb}{1.000000,0.000000,0.000000}%
\pgfsetstrokecolor{currentstroke}%
\pgfsetdash{}{0pt}%
\pgfpathmoveto{\pgfqpoint{1.073814in}{1.974757in}}%
\pgfpathlineto{\pgfqpoint{1.076558in}{1.974241in}}%
\pgfusepath{stroke}%
\end{pgfscope}%
\begin{pgfscope}%
\pgfpathrectangle{\pgfqpoint{0.100000in}{0.212622in}}{\pgfqpoint{3.696000in}{3.696000in}}%
\pgfusepath{clip}%
\pgfsetrectcap%
\pgfsetroundjoin%
\pgfsetlinewidth{1.505625pt}%
\definecolor{currentstroke}{rgb}{1.000000,0.000000,0.000000}%
\pgfsetstrokecolor{currentstroke}%
\pgfsetdash{}{0pt}%
\pgfpathmoveto{\pgfqpoint{1.077490in}{1.972642in}}%
\pgfpathlineto{\pgfqpoint{1.083121in}{1.979619in}}%
\pgfusepath{stroke}%
\end{pgfscope}%
\begin{pgfscope}%
\pgfpathrectangle{\pgfqpoint{0.100000in}{0.212622in}}{\pgfqpoint{3.696000in}{3.696000in}}%
\pgfusepath{clip}%
\pgfsetrectcap%
\pgfsetroundjoin%
\pgfsetlinewidth{1.505625pt}%
\definecolor{currentstroke}{rgb}{1.000000,0.000000,0.000000}%
\pgfsetstrokecolor{currentstroke}%
\pgfsetdash{}{0pt}%
\pgfpathmoveto{\pgfqpoint{1.085260in}{1.968838in}}%
\pgfpathlineto{\pgfqpoint{1.089677in}{1.984993in}}%
\pgfusepath{stroke}%
\end{pgfscope}%
\begin{pgfscope}%
\pgfpathrectangle{\pgfqpoint{0.100000in}{0.212622in}}{\pgfqpoint{3.696000in}{3.696000in}}%
\pgfusepath{clip}%
\pgfsetrectcap%
\pgfsetroundjoin%
\pgfsetlinewidth{1.505625pt}%
\definecolor{currentstroke}{rgb}{1.000000,0.000000,0.000000}%
\pgfsetstrokecolor{currentstroke}%
\pgfsetdash{}{0pt}%
\pgfpathmoveto{\pgfqpoint{1.096823in}{1.963464in}}%
\pgfpathlineto{\pgfqpoint{1.096228in}{1.990361in}}%
\pgfusepath{stroke}%
\end{pgfscope}%
\begin{pgfscope}%
\pgfpathrectangle{\pgfqpoint{0.100000in}{0.212622in}}{\pgfqpoint{3.696000in}{3.696000in}}%
\pgfusepath{clip}%
\pgfsetrectcap%
\pgfsetroundjoin%
\pgfsetlinewidth{1.505625pt}%
\definecolor{currentstroke}{rgb}{1.000000,0.000000,0.000000}%
\pgfsetstrokecolor{currentstroke}%
\pgfsetdash{}{0pt}%
\pgfpathmoveto{\pgfqpoint{1.112115in}{1.956558in}}%
\pgfpathlineto{\pgfqpoint{1.109310in}{2.001083in}}%
\pgfusepath{stroke}%
\end{pgfscope}%
\begin{pgfscope}%
\pgfpathrectangle{\pgfqpoint{0.100000in}{0.212622in}}{\pgfqpoint{3.696000in}{3.696000in}}%
\pgfusepath{clip}%
\pgfsetrectcap%
\pgfsetroundjoin%
\pgfsetlinewidth{1.505625pt}%
\definecolor{currentstroke}{rgb}{1.000000,0.000000,0.000000}%
\pgfsetstrokecolor{currentstroke}%
\pgfsetdash{}{0pt}%
\pgfpathmoveto{\pgfqpoint{1.131927in}{1.947651in}}%
\pgfpathlineto{\pgfqpoint{1.115843in}{2.006437in}}%
\pgfusepath{stroke}%
\end{pgfscope}%
\begin{pgfscope}%
\pgfpathrectangle{\pgfqpoint{0.100000in}{0.212622in}}{\pgfqpoint{3.696000in}{3.696000in}}%
\pgfusepath{clip}%
\pgfsetrectcap%
\pgfsetroundjoin%
\pgfsetlinewidth{1.505625pt}%
\definecolor{currentstroke}{rgb}{1.000000,0.000000,0.000000}%
\pgfsetstrokecolor{currentstroke}%
\pgfsetdash{}{0pt}%
\pgfpathmoveto{\pgfqpoint{1.144439in}{1.941850in}}%
\pgfpathlineto{\pgfqpoint{1.122369in}{2.011786in}}%
\pgfusepath{stroke}%
\end{pgfscope}%
\begin{pgfscope}%
\pgfpathrectangle{\pgfqpoint{0.100000in}{0.212622in}}{\pgfqpoint{3.696000in}{3.696000in}}%
\pgfusepath{clip}%
\pgfsetrectcap%
\pgfsetroundjoin%
\pgfsetlinewidth{1.505625pt}%
\definecolor{currentstroke}{rgb}{1.000000,0.000000,0.000000}%
\pgfsetstrokecolor{currentstroke}%
\pgfsetdash{}{0pt}%
\pgfpathmoveto{\pgfqpoint{1.152329in}{1.937921in}}%
\pgfpathlineto{\pgfqpoint{1.128889in}{2.017129in}}%
\pgfusepath{stroke}%
\end{pgfscope}%
\begin{pgfscope}%
\pgfpathrectangle{\pgfqpoint{0.100000in}{0.212622in}}{\pgfqpoint{3.696000in}{3.696000in}}%
\pgfusepath{clip}%
\pgfsetrectcap%
\pgfsetroundjoin%
\pgfsetlinewidth{1.505625pt}%
\definecolor{currentstroke}{rgb}{1.000000,0.000000,0.000000}%
\pgfsetstrokecolor{currentstroke}%
\pgfsetdash{}{0pt}%
\pgfpathmoveto{\pgfqpoint{1.157103in}{1.935213in}}%
\pgfpathlineto{\pgfqpoint{1.128889in}{2.017129in}}%
\pgfusepath{stroke}%
\end{pgfscope}%
\begin{pgfscope}%
\pgfpathrectangle{\pgfqpoint{0.100000in}{0.212622in}}{\pgfqpoint{3.696000in}{3.696000in}}%
\pgfusepath{clip}%
\pgfsetrectcap%
\pgfsetroundjoin%
\pgfsetlinewidth{1.505625pt}%
\definecolor{currentstroke}{rgb}{1.000000,0.000000,0.000000}%
\pgfsetstrokecolor{currentstroke}%
\pgfsetdash{}{0pt}%
\pgfpathmoveto{\pgfqpoint{1.159760in}{1.933434in}}%
\pgfpathlineto{\pgfqpoint{1.128889in}{2.017129in}}%
\pgfusepath{stroke}%
\end{pgfscope}%
\begin{pgfscope}%
\pgfpathrectangle{\pgfqpoint{0.100000in}{0.212622in}}{\pgfqpoint{3.696000in}{3.696000in}}%
\pgfusepath{clip}%
\pgfsetrectcap%
\pgfsetroundjoin%
\pgfsetlinewidth{1.505625pt}%
\definecolor{currentstroke}{rgb}{1.000000,0.000000,0.000000}%
\pgfsetstrokecolor{currentstroke}%
\pgfsetdash{}{0pt}%
\pgfpathmoveto{\pgfqpoint{1.161090in}{1.932301in}}%
\pgfpathlineto{\pgfqpoint{1.128889in}{2.017129in}}%
\pgfusepath{stroke}%
\end{pgfscope}%
\begin{pgfscope}%
\pgfpathrectangle{\pgfqpoint{0.100000in}{0.212622in}}{\pgfqpoint{3.696000in}{3.696000in}}%
\pgfusepath{clip}%
\pgfsetrectcap%
\pgfsetroundjoin%
\pgfsetlinewidth{1.505625pt}%
\definecolor{currentstroke}{rgb}{1.000000,0.000000,0.000000}%
\pgfsetstrokecolor{currentstroke}%
\pgfsetdash{}{0pt}%
\pgfpathmoveto{\pgfqpoint{1.161677in}{1.931619in}}%
\pgfpathlineto{\pgfqpoint{1.128889in}{2.017129in}}%
\pgfusepath{stroke}%
\end{pgfscope}%
\begin{pgfscope}%
\pgfpathrectangle{\pgfqpoint{0.100000in}{0.212622in}}{\pgfqpoint{3.696000in}{3.696000in}}%
\pgfusepath{clip}%
\pgfsetrectcap%
\pgfsetroundjoin%
\pgfsetlinewidth{1.505625pt}%
\definecolor{currentstroke}{rgb}{1.000000,0.000000,0.000000}%
\pgfsetstrokecolor{currentstroke}%
\pgfsetdash{}{0pt}%
\pgfpathmoveto{\pgfqpoint{1.161894in}{1.931228in}}%
\pgfpathlineto{\pgfqpoint{1.128889in}{2.017129in}}%
\pgfusepath{stroke}%
\end{pgfscope}%
\begin{pgfscope}%
\pgfpathrectangle{\pgfqpoint{0.100000in}{0.212622in}}{\pgfqpoint{3.696000in}{3.696000in}}%
\pgfusepath{clip}%
\pgfsetrectcap%
\pgfsetroundjoin%
\pgfsetlinewidth{1.505625pt}%
\definecolor{currentstroke}{rgb}{1.000000,0.000000,0.000000}%
\pgfsetstrokecolor{currentstroke}%
\pgfsetdash{}{0pt}%
\pgfpathmoveto{\pgfqpoint{1.161965in}{1.931014in}}%
\pgfpathlineto{\pgfqpoint{1.128889in}{2.017129in}}%
\pgfusepath{stroke}%
\end{pgfscope}%
\begin{pgfscope}%
\pgfpathrectangle{\pgfqpoint{0.100000in}{0.212622in}}{\pgfqpoint{3.696000in}{3.696000in}}%
\pgfusepath{clip}%
\pgfsetrectcap%
\pgfsetroundjoin%
\pgfsetlinewidth{1.505625pt}%
\definecolor{currentstroke}{rgb}{1.000000,0.000000,0.000000}%
\pgfsetstrokecolor{currentstroke}%
\pgfsetdash{}{0pt}%
\pgfpathmoveto{\pgfqpoint{1.161984in}{1.930896in}}%
\pgfpathlineto{\pgfqpoint{1.128889in}{2.017129in}}%
\pgfusepath{stroke}%
\end{pgfscope}%
\begin{pgfscope}%
\pgfpathrectangle{\pgfqpoint{0.100000in}{0.212622in}}{\pgfqpoint{3.696000in}{3.696000in}}%
\pgfusepath{clip}%
\pgfsetrectcap%
\pgfsetroundjoin%
\pgfsetlinewidth{1.505625pt}%
\definecolor{currentstroke}{rgb}{1.000000,0.000000,0.000000}%
\pgfsetstrokecolor{currentstroke}%
\pgfsetdash{}{0pt}%
\pgfpathmoveto{\pgfqpoint{1.161984in}{1.930832in}}%
\pgfpathlineto{\pgfqpoint{1.128889in}{2.017129in}}%
\pgfusepath{stroke}%
\end{pgfscope}%
\begin{pgfscope}%
\pgfpathrectangle{\pgfqpoint{0.100000in}{0.212622in}}{\pgfqpoint{3.696000in}{3.696000in}}%
\pgfusepath{clip}%
\pgfsetrectcap%
\pgfsetroundjoin%
\pgfsetlinewidth{1.505625pt}%
\definecolor{currentstroke}{rgb}{1.000000,0.000000,0.000000}%
\pgfsetstrokecolor{currentstroke}%
\pgfsetdash{}{0pt}%
\pgfpathmoveto{\pgfqpoint{1.161980in}{1.930798in}}%
\pgfpathlineto{\pgfqpoint{1.128889in}{2.017129in}}%
\pgfusepath{stroke}%
\end{pgfscope}%
\begin{pgfscope}%
\pgfpathrectangle{\pgfqpoint{0.100000in}{0.212622in}}{\pgfqpoint{3.696000in}{3.696000in}}%
\pgfusepath{clip}%
\pgfsetrectcap%
\pgfsetroundjoin%
\pgfsetlinewidth{1.505625pt}%
\definecolor{currentstroke}{rgb}{1.000000,0.000000,0.000000}%
\pgfsetstrokecolor{currentstroke}%
\pgfsetdash{}{0pt}%
\pgfpathmoveto{\pgfqpoint{1.161977in}{1.930780in}}%
\pgfpathlineto{\pgfqpoint{1.128889in}{2.017129in}}%
\pgfusepath{stroke}%
\end{pgfscope}%
\begin{pgfscope}%
\pgfpathrectangle{\pgfqpoint{0.100000in}{0.212622in}}{\pgfqpoint{3.696000in}{3.696000in}}%
\pgfusepath{clip}%
\pgfsetrectcap%
\pgfsetroundjoin%
\pgfsetlinewidth{1.505625pt}%
\definecolor{currentstroke}{rgb}{1.000000,0.000000,0.000000}%
\pgfsetstrokecolor{currentstroke}%
\pgfsetdash{}{0pt}%
\pgfpathmoveto{\pgfqpoint{1.161974in}{1.930770in}}%
\pgfpathlineto{\pgfqpoint{1.128889in}{2.017129in}}%
\pgfusepath{stroke}%
\end{pgfscope}%
\begin{pgfscope}%
\pgfpathrectangle{\pgfqpoint{0.100000in}{0.212622in}}{\pgfqpoint{3.696000in}{3.696000in}}%
\pgfusepath{clip}%
\pgfsetrectcap%
\pgfsetroundjoin%
\pgfsetlinewidth{1.505625pt}%
\definecolor{currentstroke}{rgb}{1.000000,0.000000,0.000000}%
\pgfsetstrokecolor{currentstroke}%
\pgfsetdash{}{0pt}%
\pgfpathmoveto{\pgfqpoint{1.161973in}{1.930764in}}%
\pgfpathlineto{\pgfqpoint{1.128889in}{2.017129in}}%
\pgfusepath{stroke}%
\end{pgfscope}%
\begin{pgfscope}%
\pgfpathrectangle{\pgfqpoint{0.100000in}{0.212622in}}{\pgfqpoint{3.696000in}{3.696000in}}%
\pgfusepath{clip}%
\pgfsetrectcap%
\pgfsetroundjoin%
\pgfsetlinewidth{1.505625pt}%
\definecolor{currentstroke}{rgb}{1.000000,0.000000,0.000000}%
\pgfsetstrokecolor{currentstroke}%
\pgfsetdash{}{0pt}%
\pgfpathmoveto{\pgfqpoint{1.161972in}{1.930761in}}%
\pgfpathlineto{\pgfqpoint{1.128889in}{2.017129in}}%
\pgfusepath{stroke}%
\end{pgfscope}%
\begin{pgfscope}%
\pgfpathrectangle{\pgfqpoint{0.100000in}{0.212622in}}{\pgfqpoint{3.696000in}{3.696000in}}%
\pgfusepath{clip}%
\pgfsetrectcap%
\pgfsetroundjoin%
\pgfsetlinewidth{1.505625pt}%
\definecolor{currentstroke}{rgb}{1.000000,0.000000,0.000000}%
\pgfsetstrokecolor{currentstroke}%
\pgfsetdash{}{0pt}%
\pgfpathmoveto{\pgfqpoint{1.161971in}{1.930760in}}%
\pgfpathlineto{\pgfqpoint{1.128889in}{2.017129in}}%
\pgfusepath{stroke}%
\end{pgfscope}%
\begin{pgfscope}%
\pgfpathrectangle{\pgfqpoint{0.100000in}{0.212622in}}{\pgfqpoint{3.696000in}{3.696000in}}%
\pgfusepath{clip}%
\pgfsetrectcap%
\pgfsetroundjoin%
\pgfsetlinewidth{1.505625pt}%
\definecolor{currentstroke}{rgb}{1.000000,0.000000,0.000000}%
\pgfsetstrokecolor{currentstroke}%
\pgfsetdash{}{0pt}%
\pgfpathmoveto{\pgfqpoint{1.161360in}{1.929081in}}%
\pgfpathlineto{\pgfqpoint{1.128889in}{2.017129in}}%
\pgfusepath{stroke}%
\end{pgfscope}%
\begin{pgfscope}%
\pgfpathrectangle{\pgfqpoint{0.100000in}{0.212622in}}{\pgfqpoint{3.696000in}{3.696000in}}%
\pgfusepath{clip}%
\pgfsetrectcap%
\pgfsetroundjoin%
\pgfsetlinewidth{1.505625pt}%
\definecolor{currentstroke}{rgb}{1.000000,0.000000,0.000000}%
\pgfsetstrokecolor{currentstroke}%
\pgfsetdash{}{0pt}%
\pgfpathmoveto{\pgfqpoint{1.160996in}{1.928183in}}%
\pgfpathlineto{\pgfqpoint{1.128889in}{2.017129in}}%
\pgfusepath{stroke}%
\end{pgfscope}%
\begin{pgfscope}%
\pgfpathrectangle{\pgfqpoint{0.100000in}{0.212622in}}{\pgfqpoint{3.696000in}{3.696000in}}%
\pgfusepath{clip}%
\pgfsetrectcap%
\pgfsetroundjoin%
\pgfsetlinewidth{1.505625pt}%
\definecolor{currentstroke}{rgb}{1.000000,0.000000,0.000000}%
\pgfsetstrokecolor{currentstroke}%
\pgfsetdash{}{0pt}%
\pgfpathmoveto{\pgfqpoint{1.160803in}{1.927670in}}%
\pgfpathlineto{\pgfqpoint{1.128889in}{2.017129in}}%
\pgfusepath{stroke}%
\end{pgfscope}%
\begin{pgfscope}%
\pgfpathrectangle{\pgfqpoint{0.100000in}{0.212622in}}{\pgfqpoint{3.696000in}{3.696000in}}%
\pgfusepath{clip}%
\pgfsetrectcap%
\pgfsetroundjoin%
\pgfsetlinewidth{1.505625pt}%
\definecolor{currentstroke}{rgb}{1.000000,0.000000,0.000000}%
\pgfsetstrokecolor{currentstroke}%
\pgfsetdash{}{0pt}%
\pgfpathmoveto{\pgfqpoint{1.160690in}{1.927401in}}%
\pgfpathlineto{\pgfqpoint{1.128889in}{2.017129in}}%
\pgfusepath{stroke}%
\end{pgfscope}%
\begin{pgfscope}%
\pgfpathrectangle{\pgfqpoint{0.100000in}{0.212622in}}{\pgfqpoint{3.696000in}{3.696000in}}%
\pgfusepath{clip}%
\pgfsetrectcap%
\pgfsetroundjoin%
\pgfsetlinewidth{1.505625pt}%
\definecolor{currentstroke}{rgb}{1.000000,0.000000,0.000000}%
\pgfsetstrokecolor{currentstroke}%
\pgfsetdash{}{0pt}%
\pgfpathmoveto{\pgfqpoint{1.160627in}{1.927249in}}%
\pgfpathlineto{\pgfqpoint{1.128889in}{2.017129in}}%
\pgfusepath{stroke}%
\end{pgfscope}%
\begin{pgfscope}%
\pgfpathrectangle{\pgfqpoint{0.100000in}{0.212622in}}{\pgfqpoint{3.696000in}{3.696000in}}%
\pgfusepath{clip}%
\pgfsetrectcap%
\pgfsetroundjoin%
\pgfsetlinewidth{1.505625pt}%
\definecolor{currentstroke}{rgb}{1.000000,0.000000,0.000000}%
\pgfsetstrokecolor{currentstroke}%
\pgfsetdash{}{0pt}%
\pgfpathmoveto{\pgfqpoint{1.160594in}{1.927164in}}%
\pgfpathlineto{\pgfqpoint{1.128889in}{2.017129in}}%
\pgfusepath{stroke}%
\end{pgfscope}%
\begin{pgfscope}%
\pgfpathrectangle{\pgfqpoint{0.100000in}{0.212622in}}{\pgfqpoint{3.696000in}{3.696000in}}%
\pgfusepath{clip}%
\pgfsetrectcap%
\pgfsetroundjoin%
\pgfsetlinewidth{1.505625pt}%
\definecolor{currentstroke}{rgb}{1.000000,0.000000,0.000000}%
\pgfsetstrokecolor{currentstroke}%
\pgfsetdash{}{0pt}%
\pgfpathmoveto{\pgfqpoint{1.160575in}{1.927119in}}%
\pgfpathlineto{\pgfqpoint{1.128889in}{2.017129in}}%
\pgfusepath{stroke}%
\end{pgfscope}%
\begin{pgfscope}%
\pgfpathrectangle{\pgfqpoint{0.100000in}{0.212622in}}{\pgfqpoint{3.696000in}{3.696000in}}%
\pgfusepath{clip}%
\pgfsetrectcap%
\pgfsetroundjoin%
\pgfsetlinewidth{1.505625pt}%
\definecolor{currentstroke}{rgb}{1.000000,0.000000,0.000000}%
\pgfsetstrokecolor{currentstroke}%
\pgfsetdash{}{0pt}%
\pgfpathmoveto{\pgfqpoint{1.160564in}{1.927093in}}%
\pgfpathlineto{\pgfqpoint{1.128889in}{2.017129in}}%
\pgfusepath{stroke}%
\end{pgfscope}%
\begin{pgfscope}%
\pgfpathrectangle{\pgfqpoint{0.100000in}{0.212622in}}{\pgfqpoint{3.696000in}{3.696000in}}%
\pgfusepath{clip}%
\pgfsetrectcap%
\pgfsetroundjoin%
\pgfsetlinewidth{1.505625pt}%
\definecolor{currentstroke}{rgb}{1.000000,0.000000,0.000000}%
\pgfsetstrokecolor{currentstroke}%
\pgfsetdash{}{0pt}%
\pgfpathmoveto{\pgfqpoint{1.160559in}{1.927079in}}%
\pgfpathlineto{\pgfqpoint{1.128889in}{2.017129in}}%
\pgfusepath{stroke}%
\end{pgfscope}%
\begin{pgfscope}%
\pgfpathrectangle{\pgfqpoint{0.100000in}{0.212622in}}{\pgfqpoint{3.696000in}{3.696000in}}%
\pgfusepath{clip}%
\pgfsetrectcap%
\pgfsetroundjoin%
\pgfsetlinewidth{1.505625pt}%
\definecolor{currentstroke}{rgb}{1.000000,0.000000,0.000000}%
\pgfsetstrokecolor{currentstroke}%
\pgfsetdash{}{0pt}%
\pgfpathmoveto{\pgfqpoint{1.159766in}{1.925327in}}%
\pgfpathlineto{\pgfqpoint{1.128889in}{2.017129in}}%
\pgfusepath{stroke}%
\end{pgfscope}%
\begin{pgfscope}%
\pgfpathrectangle{\pgfqpoint{0.100000in}{0.212622in}}{\pgfqpoint{3.696000in}{3.696000in}}%
\pgfusepath{clip}%
\pgfsetrectcap%
\pgfsetroundjoin%
\pgfsetlinewidth{1.505625pt}%
\definecolor{currentstroke}{rgb}{1.000000,0.000000,0.000000}%
\pgfsetstrokecolor{currentstroke}%
\pgfsetdash{}{0pt}%
\pgfpathmoveto{\pgfqpoint{1.159339in}{1.924359in}}%
\pgfpathlineto{\pgfqpoint{1.122369in}{2.011786in}}%
\pgfusepath{stroke}%
\end{pgfscope}%
\begin{pgfscope}%
\pgfpathrectangle{\pgfqpoint{0.100000in}{0.212622in}}{\pgfqpoint{3.696000in}{3.696000in}}%
\pgfusepath{clip}%
\pgfsetrectcap%
\pgfsetroundjoin%
\pgfsetlinewidth{1.505625pt}%
\definecolor{currentstroke}{rgb}{1.000000,0.000000,0.000000}%
\pgfsetstrokecolor{currentstroke}%
\pgfsetdash{}{0pt}%
\pgfpathmoveto{\pgfqpoint{1.159091in}{1.923824in}}%
\pgfpathlineto{\pgfqpoint{1.122369in}{2.011786in}}%
\pgfusepath{stroke}%
\end{pgfscope}%
\begin{pgfscope}%
\pgfpathrectangle{\pgfqpoint{0.100000in}{0.212622in}}{\pgfqpoint{3.696000in}{3.696000in}}%
\pgfusepath{clip}%
\pgfsetrectcap%
\pgfsetroundjoin%
\pgfsetlinewidth{1.505625pt}%
\definecolor{currentstroke}{rgb}{1.000000,0.000000,0.000000}%
\pgfsetstrokecolor{currentstroke}%
\pgfsetdash{}{0pt}%
\pgfpathmoveto{\pgfqpoint{1.158940in}{1.923541in}}%
\pgfpathlineto{\pgfqpoint{1.122369in}{2.011786in}}%
\pgfusepath{stroke}%
\end{pgfscope}%
\begin{pgfscope}%
\pgfpathrectangle{\pgfqpoint{0.100000in}{0.212622in}}{\pgfqpoint{3.696000in}{3.696000in}}%
\pgfusepath{clip}%
\pgfsetrectcap%
\pgfsetroundjoin%
\pgfsetlinewidth{1.505625pt}%
\definecolor{currentstroke}{rgb}{1.000000,0.000000,0.000000}%
\pgfsetstrokecolor{currentstroke}%
\pgfsetdash{}{0pt}%
\pgfpathmoveto{\pgfqpoint{1.158856in}{1.923383in}}%
\pgfpathlineto{\pgfqpoint{1.122369in}{2.011786in}}%
\pgfusepath{stroke}%
\end{pgfscope}%
\begin{pgfscope}%
\pgfpathrectangle{\pgfqpoint{0.100000in}{0.212622in}}{\pgfqpoint{3.696000in}{3.696000in}}%
\pgfusepath{clip}%
\pgfsetrectcap%
\pgfsetroundjoin%
\pgfsetlinewidth{1.505625pt}%
\definecolor{currentstroke}{rgb}{1.000000,0.000000,0.000000}%
\pgfsetstrokecolor{currentstroke}%
\pgfsetdash{}{0pt}%
\pgfpathmoveto{\pgfqpoint{1.158802in}{1.923301in}}%
\pgfpathlineto{\pgfqpoint{1.122369in}{2.011786in}}%
\pgfusepath{stroke}%
\end{pgfscope}%
\begin{pgfscope}%
\pgfpathrectangle{\pgfqpoint{0.100000in}{0.212622in}}{\pgfqpoint{3.696000in}{3.696000in}}%
\pgfusepath{clip}%
\pgfsetrectcap%
\pgfsetroundjoin%
\pgfsetlinewidth{1.505625pt}%
\definecolor{currentstroke}{rgb}{1.000000,0.000000,0.000000}%
\pgfsetstrokecolor{currentstroke}%
\pgfsetdash{}{0pt}%
\pgfpathmoveto{\pgfqpoint{1.158767in}{1.923258in}}%
\pgfpathlineto{\pgfqpoint{1.122369in}{2.011786in}}%
\pgfusepath{stroke}%
\end{pgfscope}%
\begin{pgfscope}%
\pgfpathrectangle{\pgfqpoint{0.100000in}{0.212622in}}{\pgfqpoint{3.696000in}{3.696000in}}%
\pgfusepath{clip}%
\pgfsetrectcap%
\pgfsetroundjoin%
\pgfsetlinewidth{1.505625pt}%
\definecolor{currentstroke}{rgb}{1.000000,0.000000,0.000000}%
\pgfsetstrokecolor{currentstroke}%
\pgfsetdash{}{0pt}%
\pgfpathmoveto{\pgfqpoint{1.158741in}{1.923242in}}%
\pgfpathlineto{\pgfqpoint{1.122369in}{2.011786in}}%
\pgfusepath{stroke}%
\end{pgfscope}%
\begin{pgfscope}%
\pgfpathrectangle{\pgfqpoint{0.100000in}{0.212622in}}{\pgfqpoint{3.696000in}{3.696000in}}%
\pgfusepath{clip}%
\pgfsetrectcap%
\pgfsetroundjoin%
\pgfsetlinewidth{1.505625pt}%
\definecolor{currentstroke}{rgb}{1.000000,0.000000,0.000000}%
\pgfsetstrokecolor{currentstroke}%
\pgfsetdash{}{0pt}%
\pgfpathmoveto{\pgfqpoint{1.155170in}{1.921779in}}%
\pgfpathlineto{\pgfqpoint{1.122369in}{2.011786in}}%
\pgfusepath{stroke}%
\end{pgfscope}%
\begin{pgfscope}%
\pgfpathrectangle{\pgfqpoint{0.100000in}{0.212622in}}{\pgfqpoint{3.696000in}{3.696000in}}%
\pgfusepath{clip}%
\pgfsetrectcap%
\pgfsetroundjoin%
\pgfsetlinewidth{1.505625pt}%
\definecolor{currentstroke}{rgb}{1.000000,0.000000,0.000000}%
\pgfsetstrokecolor{currentstroke}%
\pgfsetdash{}{0pt}%
\pgfpathmoveto{\pgfqpoint{1.153108in}{1.921032in}}%
\pgfpathlineto{\pgfqpoint{1.122369in}{2.011786in}}%
\pgfusepath{stroke}%
\end{pgfscope}%
\begin{pgfscope}%
\pgfpathrectangle{\pgfqpoint{0.100000in}{0.212622in}}{\pgfqpoint{3.696000in}{3.696000in}}%
\pgfusepath{clip}%
\pgfsetrectcap%
\pgfsetroundjoin%
\pgfsetlinewidth{1.505625pt}%
\definecolor{currentstroke}{rgb}{1.000000,0.000000,0.000000}%
\pgfsetstrokecolor{currentstroke}%
\pgfsetdash{}{0pt}%
\pgfpathmoveto{\pgfqpoint{1.148380in}{1.921006in}}%
\pgfpathlineto{\pgfqpoint{1.122369in}{2.011786in}}%
\pgfusepath{stroke}%
\end{pgfscope}%
\begin{pgfscope}%
\pgfpathrectangle{\pgfqpoint{0.100000in}{0.212622in}}{\pgfqpoint{3.696000in}{3.696000in}}%
\pgfusepath{clip}%
\pgfsetrectcap%
\pgfsetroundjoin%
\pgfsetlinewidth{1.505625pt}%
\definecolor{currentstroke}{rgb}{1.000000,0.000000,0.000000}%
\pgfsetstrokecolor{currentstroke}%
\pgfsetdash{}{0pt}%
\pgfpathmoveto{\pgfqpoint{1.138987in}{1.924117in}}%
\pgfpathlineto{\pgfqpoint{1.122369in}{2.011786in}}%
\pgfusepath{stroke}%
\end{pgfscope}%
\begin{pgfscope}%
\pgfpathrectangle{\pgfqpoint{0.100000in}{0.212622in}}{\pgfqpoint{3.696000in}{3.696000in}}%
\pgfusepath{clip}%
\pgfsetrectcap%
\pgfsetroundjoin%
\pgfsetlinewidth{1.505625pt}%
\definecolor{currentstroke}{rgb}{1.000000,0.000000,0.000000}%
\pgfsetstrokecolor{currentstroke}%
\pgfsetdash{}{0pt}%
\pgfpathmoveto{\pgfqpoint{1.134266in}{1.926450in}}%
\pgfpathlineto{\pgfqpoint{1.122369in}{2.011786in}}%
\pgfusepath{stroke}%
\end{pgfscope}%
\begin{pgfscope}%
\pgfpathrectangle{\pgfqpoint{0.100000in}{0.212622in}}{\pgfqpoint{3.696000in}{3.696000in}}%
\pgfusepath{clip}%
\pgfsetrectcap%
\pgfsetroundjoin%
\pgfsetlinewidth{1.505625pt}%
\definecolor{currentstroke}{rgb}{1.000000,0.000000,0.000000}%
\pgfsetstrokecolor{currentstroke}%
\pgfsetdash{}{0pt}%
\pgfpathmoveto{\pgfqpoint{1.132125in}{1.928130in}}%
\pgfpathlineto{\pgfqpoint{1.122369in}{2.011786in}}%
\pgfusepath{stroke}%
\end{pgfscope}%
\begin{pgfscope}%
\pgfpathrectangle{\pgfqpoint{0.100000in}{0.212622in}}{\pgfqpoint{3.696000in}{3.696000in}}%
\pgfusepath{clip}%
\pgfsetrectcap%
\pgfsetroundjoin%
\pgfsetlinewidth{1.505625pt}%
\definecolor{currentstroke}{rgb}{1.000000,0.000000,0.000000}%
\pgfsetstrokecolor{currentstroke}%
\pgfsetdash{}{0pt}%
\pgfpathmoveto{\pgfqpoint{1.125320in}{1.937585in}}%
\pgfpathlineto{\pgfqpoint{1.122369in}{2.011786in}}%
\pgfusepath{stroke}%
\end{pgfscope}%
\begin{pgfscope}%
\pgfpathrectangle{\pgfqpoint{0.100000in}{0.212622in}}{\pgfqpoint{3.696000in}{3.696000in}}%
\pgfusepath{clip}%
\pgfsetrectcap%
\pgfsetroundjoin%
\pgfsetlinewidth{1.505625pt}%
\definecolor{currentstroke}{rgb}{1.000000,0.000000,0.000000}%
\pgfsetstrokecolor{currentstroke}%
\pgfsetdash{}{0pt}%
\pgfpathmoveto{\pgfqpoint{1.121749in}{1.950721in}}%
\pgfpathlineto{\pgfqpoint{1.128889in}{2.017129in}}%
\pgfusepath{stroke}%
\end{pgfscope}%
\begin{pgfscope}%
\pgfpathrectangle{\pgfqpoint{0.100000in}{0.212622in}}{\pgfqpoint{3.696000in}{3.696000in}}%
\pgfusepath{clip}%
\pgfsetrectcap%
\pgfsetroundjoin%
\pgfsetlinewidth{1.505625pt}%
\definecolor{currentstroke}{rgb}{1.000000,0.000000,0.000000}%
\pgfsetstrokecolor{currentstroke}%
\pgfsetdash{}{0pt}%
\pgfpathmoveto{\pgfqpoint{1.120458in}{1.966550in}}%
\pgfpathlineto{\pgfqpoint{1.135404in}{2.022468in}}%
\pgfusepath{stroke}%
\end{pgfscope}%
\begin{pgfscope}%
\pgfpathrectangle{\pgfqpoint{0.100000in}{0.212622in}}{\pgfqpoint{3.696000in}{3.696000in}}%
\pgfusepath{clip}%
\pgfsetrectcap%
\pgfsetroundjoin%
\pgfsetlinewidth{1.505625pt}%
\definecolor{currentstroke}{rgb}{1.000000,0.000000,0.000000}%
\pgfsetstrokecolor{currentstroke}%
\pgfsetdash{}{0pt}%
\pgfpathmoveto{\pgfqpoint{1.122899in}{1.986461in}}%
\pgfpathlineto{\pgfqpoint{1.148414in}{2.033132in}}%
\pgfusepath{stroke}%
\end{pgfscope}%
\begin{pgfscope}%
\pgfpathrectangle{\pgfqpoint{0.100000in}{0.212622in}}{\pgfqpoint{3.696000in}{3.696000in}}%
\pgfusepath{clip}%
\pgfsetrectcap%
\pgfsetroundjoin%
\pgfsetlinewidth{1.505625pt}%
\definecolor{currentstroke}{rgb}{1.000000,0.000000,0.000000}%
\pgfsetstrokecolor{currentstroke}%
\pgfsetdash{}{0pt}%
\pgfpathmoveto{\pgfqpoint{1.129270in}{2.010715in}}%
\pgfpathlineto{\pgfqpoint{1.161401in}{2.043775in}}%
\pgfusepath{stroke}%
\end{pgfscope}%
\begin{pgfscope}%
\pgfpathrectangle{\pgfqpoint{0.100000in}{0.212622in}}{\pgfqpoint{3.696000in}{3.696000in}}%
\pgfusepath{clip}%
\pgfsetrectcap%
\pgfsetroundjoin%
\pgfsetlinewidth{1.505625pt}%
\definecolor{currentstroke}{rgb}{1.000000,0.000000,0.000000}%
\pgfsetstrokecolor{currentstroke}%
\pgfsetdash{}{0pt}%
\pgfpathmoveto{\pgfqpoint{1.133529in}{2.023768in}}%
\pgfpathlineto{\pgfqpoint{1.167886in}{2.049090in}}%
\pgfusepath{stroke}%
\end{pgfscope}%
\begin{pgfscope}%
\pgfpathrectangle{\pgfqpoint{0.100000in}{0.212622in}}{\pgfqpoint{3.696000in}{3.696000in}}%
\pgfusepath{clip}%
\pgfsetrectcap%
\pgfsetroundjoin%
\pgfsetlinewidth{1.505625pt}%
\definecolor{currentstroke}{rgb}{1.000000,0.000000,0.000000}%
\pgfsetstrokecolor{currentstroke}%
\pgfsetdash{}{0pt}%
\pgfpathmoveto{\pgfqpoint{1.140493in}{2.041154in}}%
\pgfpathlineto{\pgfqpoint{1.174364in}{2.054399in}}%
\pgfusepath{stroke}%
\end{pgfscope}%
\begin{pgfscope}%
\pgfpathrectangle{\pgfqpoint{0.100000in}{0.212622in}}{\pgfqpoint{3.696000in}{3.696000in}}%
\pgfusepath{clip}%
\pgfsetrectcap%
\pgfsetroundjoin%
\pgfsetlinewidth{1.505625pt}%
\definecolor{currentstroke}{rgb}{1.000000,0.000000,0.000000}%
\pgfsetstrokecolor{currentstroke}%
\pgfsetdash{}{0pt}%
\pgfpathmoveto{\pgfqpoint{1.144915in}{2.050703in}}%
\pgfpathlineto{\pgfqpoint{1.180837in}{2.059704in}}%
\pgfusepath{stroke}%
\end{pgfscope}%
\begin{pgfscope}%
\pgfpathrectangle{\pgfqpoint{0.100000in}{0.212622in}}{\pgfqpoint{3.696000in}{3.696000in}}%
\pgfusepath{clip}%
\pgfsetrectcap%
\pgfsetroundjoin%
\pgfsetlinewidth{1.505625pt}%
\definecolor{currentstroke}{rgb}{1.000000,0.000000,0.000000}%
\pgfsetstrokecolor{currentstroke}%
\pgfsetdash{}{0pt}%
\pgfpathmoveto{\pgfqpoint{1.150401in}{2.062072in}}%
\pgfpathlineto{\pgfqpoint{1.187304in}{2.065004in}}%
\pgfusepath{stroke}%
\end{pgfscope}%
\begin{pgfscope}%
\pgfpathrectangle{\pgfqpoint{0.100000in}{0.212622in}}{\pgfqpoint{3.696000in}{3.696000in}}%
\pgfusepath{clip}%
\pgfsetrectcap%
\pgfsetroundjoin%
\pgfsetlinewidth{1.505625pt}%
\definecolor{currentstroke}{rgb}{1.000000,0.000000,0.000000}%
\pgfsetstrokecolor{currentstroke}%
\pgfsetdash{}{0pt}%
\pgfpathmoveto{\pgfqpoint{1.156919in}{2.074870in}}%
\pgfpathlineto{\pgfqpoint{1.193764in}{2.070299in}}%
\pgfusepath{stroke}%
\end{pgfscope}%
\begin{pgfscope}%
\pgfpathrectangle{\pgfqpoint{0.100000in}{0.212622in}}{\pgfqpoint{3.696000in}{3.696000in}}%
\pgfusepath{clip}%
\pgfsetrectcap%
\pgfsetroundjoin%
\pgfsetlinewidth{1.505625pt}%
\definecolor{currentstroke}{rgb}{1.000000,0.000000,0.000000}%
\pgfsetstrokecolor{currentstroke}%
\pgfsetdash{}{0pt}%
\pgfpathmoveto{\pgfqpoint{1.160545in}{2.081917in}}%
\pgfpathlineto{\pgfqpoint{1.200219in}{2.075589in}}%
\pgfusepath{stroke}%
\end{pgfscope}%
\begin{pgfscope}%
\pgfpathrectangle{\pgfqpoint{0.100000in}{0.212622in}}{\pgfqpoint{3.696000in}{3.696000in}}%
\pgfusepath{clip}%
\pgfsetrectcap%
\pgfsetroundjoin%
\pgfsetlinewidth{1.505625pt}%
\definecolor{currentstroke}{rgb}{1.000000,0.000000,0.000000}%
\pgfsetstrokecolor{currentstroke}%
\pgfsetdash{}{0pt}%
\pgfpathmoveto{\pgfqpoint{1.162529in}{2.085807in}}%
\pgfpathlineto{\pgfqpoint{1.200219in}{2.075589in}}%
\pgfusepath{stroke}%
\end{pgfscope}%
\begin{pgfscope}%
\pgfpathrectangle{\pgfqpoint{0.100000in}{0.212622in}}{\pgfqpoint{3.696000in}{3.696000in}}%
\pgfusepath{clip}%
\pgfsetrectcap%
\pgfsetroundjoin%
\pgfsetlinewidth{1.505625pt}%
\definecolor{currentstroke}{rgb}{1.000000,0.000000,0.000000}%
\pgfsetstrokecolor{currentstroke}%
\pgfsetdash{}{0pt}%
\pgfpathmoveto{\pgfqpoint{1.163655in}{2.087939in}}%
\pgfpathlineto{\pgfqpoint{1.200219in}{2.075589in}}%
\pgfusepath{stroke}%
\end{pgfscope}%
\begin{pgfscope}%
\pgfpathrectangle{\pgfqpoint{0.100000in}{0.212622in}}{\pgfqpoint{3.696000in}{3.696000in}}%
\pgfusepath{clip}%
\pgfsetrectcap%
\pgfsetroundjoin%
\pgfsetlinewidth{1.505625pt}%
\definecolor{currentstroke}{rgb}{1.000000,0.000000,0.000000}%
\pgfsetstrokecolor{currentstroke}%
\pgfsetdash{}{0pt}%
\pgfpathmoveto{\pgfqpoint{1.164263in}{2.089106in}}%
\pgfpathlineto{\pgfqpoint{1.200219in}{2.075589in}}%
\pgfusepath{stroke}%
\end{pgfscope}%
\begin{pgfscope}%
\pgfpathrectangle{\pgfqpoint{0.100000in}{0.212622in}}{\pgfqpoint{3.696000in}{3.696000in}}%
\pgfusepath{clip}%
\pgfsetrectcap%
\pgfsetroundjoin%
\pgfsetlinewidth{1.505625pt}%
\definecolor{currentstroke}{rgb}{1.000000,0.000000,0.000000}%
\pgfsetstrokecolor{currentstroke}%
\pgfsetdash{}{0pt}%
\pgfpathmoveto{\pgfqpoint{1.164600in}{2.089749in}}%
\pgfpathlineto{\pgfqpoint{1.200219in}{2.075589in}}%
\pgfusepath{stroke}%
\end{pgfscope}%
\begin{pgfscope}%
\pgfpathrectangle{\pgfqpoint{0.100000in}{0.212622in}}{\pgfqpoint{3.696000in}{3.696000in}}%
\pgfusepath{clip}%
\pgfsetrectcap%
\pgfsetroundjoin%
\pgfsetlinewidth{1.505625pt}%
\definecolor{currentstroke}{rgb}{1.000000,0.000000,0.000000}%
\pgfsetstrokecolor{currentstroke}%
\pgfsetdash{}{0pt}%
\pgfpathmoveto{\pgfqpoint{1.165794in}{2.091945in}}%
\pgfpathlineto{\pgfqpoint{1.206668in}{2.080874in}}%
\pgfusepath{stroke}%
\end{pgfscope}%
\begin{pgfscope}%
\pgfpathrectangle{\pgfqpoint{0.100000in}{0.212622in}}{\pgfqpoint{3.696000in}{3.696000in}}%
\pgfusepath{clip}%
\pgfsetrectcap%
\pgfsetroundjoin%
\pgfsetlinewidth{1.505625pt}%
\definecolor{currentstroke}{rgb}{1.000000,0.000000,0.000000}%
\pgfsetstrokecolor{currentstroke}%
\pgfsetdash{}{0pt}%
\pgfpathmoveto{\pgfqpoint{1.166426in}{2.093134in}}%
\pgfpathlineto{\pgfqpoint{1.206668in}{2.080874in}}%
\pgfusepath{stroke}%
\end{pgfscope}%
\begin{pgfscope}%
\pgfpathrectangle{\pgfqpoint{0.100000in}{0.212622in}}{\pgfqpoint{3.696000in}{3.696000in}}%
\pgfusepath{clip}%
\pgfsetrectcap%
\pgfsetroundjoin%
\pgfsetlinewidth{1.505625pt}%
\definecolor{currentstroke}{rgb}{1.000000,0.000000,0.000000}%
\pgfsetstrokecolor{currentstroke}%
\pgfsetdash{}{0pt}%
\pgfpathmoveto{\pgfqpoint{1.166779in}{2.093799in}}%
\pgfpathlineto{\pgfqpoint{1.206668in}{2.080874in}}%
\pgfusepath{stroke}%
\end{pgfscope}%
\begin{pgfscope}%
\pgfpathrectangle{\pgfqpoint{0.100000in}{0.212622in}}{\pgfqpoint{3.696000in}{3.696000in}}%
\pgfusepath{clip}%
\pgfsetrectcap%
\pgfsetroundjoin%
\pgfsetlinewidth{1.505625pt}%
\definecolor{currentstroke}{rgb}{1.000000,0.000000,0.000000}%
\pgfsetstrokecolor{currentstroke}%
\pgfsetdash{}{0pt}%
\pgfpathmoveto{\pgfqpoint{1.166965in}{2.094161in}}%
\pgfpathlineto{\pgfqpoint{1.206668in}{2.080874in}}%
\pgfusepath{stroke}%
\end{pgfscope}%
\begin{pgfscope}%
\pgfpathrectangle{\pgfqpoint{0.100000in}{0.212622in}}{\pgfqpoint{3.696000in}{3.696000in}}%
\pgfusepath{clip}%
\pgfsetrectcap%
\pgfsetroundjoin%
\pgfsetlinewidth{1.505625pt}%
\definecolor{currentstroke}{rgb}{1.000000,0.000000,0.000000}%
\pgfsetstrokecolor{currentstroke}%
\pgfsetdash{}{0pt}%
\pgfpathmoveto{\pgfqpoint{1.167063in}{2.094362in}}%
\pgfpathlineto{\pgfqpoint{1.206668in}{2.080874in}}%
\pgfusepath{stroke}%
\end{pgfscope}%
\begin{pgfscope}%
\pgfpathrectangle{\pgfqpoint{0.100000in}{0.212622in}}{\pgfqpoint{3.696000in}{3.696000in}}%
\pgfusepath{clip}%
\pgfsetrectcap%
\pgfsetroundjoin%
\pgfsetlinewidth{1.505625pt}%
\definecolor{currentstroke}{rgb}{1.000000,0.000000,0.000000}%
\pgfsetstrokecolor{currentstroke}%
\pgfsetdash{}{0pt}%
\pgfpathmoveto{\pgfqpoint{1.168297in}{2.096857in}}%
\pgfpathlineto{\pgfqpoint{1.206668in}{2.080874in}}%
\pgfusepath{stroke}%
\end{pgfscope}%
\begin{pgfscope}%
\pgfpathrectangle{\pgfqpoint{0.100000in}{0.212622in}}{\pgfqpoint{3.696000in}{3.696000in}}%
\pgfusepath{clip}%
\pgfsetrectcap%
\pgfsetroundjoin%
\pgfsetlinewidth{1.505625pt}%
\definecolor{currentstroke}{rgb}{1.000000,0.000000,0.000000}%
\pgfsetstrokecolor{currentstroke}%
\pgfsetdash{}{0pt}%
\pgfpathmoveto{\pgfqpoint{1.168984in}{2.098227in}}%
\pgfpathlineto{\pgfqpoint{1.206668in}{2.080874in}}%
\pgfusepath{stroke}%
\end{pgfscope}%
\begin{pgfscope}%
\pgfpathrectangle{\pgfqpoint{0.100000in}{0.212622in}}{\pgfqpoint{3.696000in}{3.696000in}}%
\pgfusepath{clip}%
\pgfsetrectcap%
\pgfsetroundjoin%
\pgfsetlinewidth{1.505625pt}%
\definecolor{currentstroke}{rgb}{1.000000,0.000000,0.000000}%
\pgfsetstrokecolor{currentstroke}%
\pgfsetdash{}{0pt}%
\pgfpathmoveto{\pgfqpoint{1.169369in}{2.098966in}}%
\pgfpathlineto{\pgfqpoint{1.206668in}{2.080874in}}%
\pgfusepath{stroke}%
\end{pgfscope}%
\begin{pgfscope}%
\pgfpathrectangle{\pgfqpoint{0.100000in}{0.212622in}}{\pgfqpoint{3.696000in}{3.696000in}}%
\pgfusepath{clip}%
\pgfsetrectcap%
\pgfsetroundjoin%
\pgfsetlinewidth{1.505625pt}%
\definecolor{currentstroke}{rgb}{1.000000,0.000000,0.000000}%
\pgfsetstrokecolor{currentstroke}%
\pgfsetdash{}{0pt}%
\pgfpathmoveto{\pgfqpoint{1.170719in}{2.101557in}}%
\pgfpathlineto{\pgfqpoint{1.206668in}{2.080874in}}%
\pgfusepath{stroke}%
\end{pgfscope}%
\begin{pgfscope}%
\pgfpathrectangle{\pgfqpoint{0.100000in}{0.212622in}}{\pgfqpoint{3.696000in}{3.696000in}}%
\pgfusepath{clip}%
\pgfsetrectcap%
\pgfsetroundjoin%
\pgfsetlinewidth{1.505625pt}%
\definecolor{currentstroke}{rgb}{1.000000,0.000000,0.000000}%
\pgfsetstrokecolor{currentstroke}%
\pgfsetdash{}{0pt}%
\pgfpathmoveto{\pgfqpoint{1.172871in}{2.105902in}}%
\pgfpathlineto{\pgfqpoint{1.213111in}{2.086155in}}%
\pgfusepath{stroke}%
\end{pgfscope}%
\begin{pgfscope}%
\pgfpathrectangle{\pgfqpoint{0.100000in}{0.212622in}}{\pgfqpoint{3.696000in}{3.696000in}}%
\pgfusepath{clip}%
\pgfsetrectcap%
\pgfsetroundjoin%
\pgfsetlinewidth{1.505625pt}%
\definecolor{currentstroke}{rgb}{1.000000,0.000000,0.000000}%
\pgfsetstrokecolor{currentstroke}%
\pgfsetdash{}{0pt}%
\pgfpathmoveto{\pgfqpoint{1.179168in}{2.119008in}}%
\pgfpathlineto{\pgfqpoint{1.219549in}{2.091431in}}%
\pgfusepath{stroke}%
\end{pgfscope}%
\begin{pgfscope}%
\pgfpathrectangle{\pgfqpoint{0.100000in}{0.212622in}}{\pgfqpoint{3.696000in}{3.696000in}}%
\pgfusepath{clip}%
\pgfsetrectcap%
\pgfsetroundjoin%
\pgfsetlinewidth{1.505625pt}%
\definecolor{currentstroke}{rgb}{1.000000,0.000000,0.000000}%
\pgfsetstrokecolor{currentstroke}%
\pgfsetdash{}{0pt}%
\pgfpathmoveto{\pgfqpoint{1.182725in}{2.126289in}}%
\pgfpathlineto{\pgfqpoint{1.219549in}{2.091431in}}%
\pgfusepath{stroke}%
\end{pgfscope}%
\begin{pgfscope}%
\pgfpathrectangle{\pgfqpoint{0.100000in}{0.212622in}}{\pgfqpoint{3.696000in}{3.696000in}}%
\pgfusepath{clip}%
\pgfsetrectcap%
\pgfsetroundjoin%
\pgfsetlinewidth{1.505625pt}%
\definecolor{currentstroke}{rgb}{1.000000,0.000000,0.000000}%
\pgfsetstrokecolor{currentstroke}%
\pgfsetdash{}{0pt}%
\pgfpathmoveto{\pgfqpoint{1.184533in}{2.130225in}}%
\pgfpathlineto{\pgfqpoint{1.225980in}{2.096702in}}%
\pgfusepath{stroke}%
\end{pgfscope}%
\begin{pgfscope}%
\pgfpathrectangle{\pgfqpoint{0.100000in}{0.212622in}}{\pgfqpoint{3.696000in}{3.696000in}}%
\pgfusepath{clip}%
\pgfsetrectcap%
\pgfsetroundjoin%
\pgfsetlinewidth{1.505625pt}%
\definecolor{currentstroke}{rgb}{1.000000,0.000000,0.000000}%
\pgfsetstrokecolor{currentstroke}%
\pgfsetdash{}{0pt}%
\pgfpathmoveto{\pgfqpoint{1.190632in}{2.142810in}}%
\pgfpathlineto{\pgfqpoint{1.232405in}{2.101968in}}%
\pgfusepath{stroke}%
\end{pgfscope}%
\begin{pgfscope}%
\pgfpathrectangle{\pgfqpoint{0.100000in}{0.212622in}}{\pgfqpoint{3.696000in}{3.696000in}}%
\pgfusepath{clip}%
\pgfsetrectcap%
\pgfsetroundjoin%
\pgfsetlinewidth{1.505625pt}%
\definecolor{currentstroke}{rgb}{1.000000,0.000000,0.000000}%
\pgfsetstrokecolor{currentstroke}%
\pgfsetdash{}{0pt}%
\pgfpathmoveto{\pgfqpoint{1.199333in}{2.160396in}}%
\pgfpathlineto{\pgfqpoint{1.238825in}{2.107229in}}%
\pgfusepath{stroke}%
\end{pgfscope}%
\begin{pgfscope}%
\pgfpathrectangle{\pgfqpoint{0.100000in}{0.212622in}}{\pgfqpoint{3.696000in}{3.696000in}}%
\pgfusepath{clip}%
\pgfsetrectcap%
\pgfsetroundjoin%
\pgfsetlinewidth{1.505625pt}%
\definecolor{currentstroke}{rgb}{1.000000,0.000000,0.000000}%
\pgfsetstrokecolor{currentstroke}%
\pgfsetdash{}{0pt}%
\pgfpathmoveto{\pgfqpoint{1.213676in}{2.187428in}}%
\pgfpathlineto{\pgfqpoint{1.258049in}{2.122984in}}%
\pgfusepath{stroke}%
\end{pgfscope}%
\begin{pgfscope}%
\pgfpathrectangle{\pgfqpoint{0.100000in}{0.212622in}}{\pgfqpoint{3.696000in}{3.696000in}}%
\pgfusepath{clip}%
\pgfsetrectcap%
\pgfsetroundjoin%
\pgfsetlinewidth{1.505625pt}%
\definecolor{currentstroke}{rgb}{1.000000,0.000000,0.000000}%
\pgfsetstrokecolor{currentstroke}%
\pgfsetdash{}{0pt}%
\pgfpathmoveto{\pgfqpoint{1.230524in}{2.221009in}}%
\pgfpathlineto{\pgfqpoint{1.270836in}{2.133464in}}%
\pgfusepath{stroke}%
\end{pgfscope}%
\begin{pgfscope}%
\pgfpathrectangle{\pgfqpoint{0.100000in}{0.212622in}}{\pgfqpoint{3.696000in}{3.696000in}}%
\pgfusepath{clip}%
\pgfsetrectcap%
\pgfsetroundjoin%
\pgfsetlinewidth{1.505625pt}%
\definecolor{currentstroke}{rgb}{1.000000,0.000000,0.000000}%
\pgfsetstrokecolor{currentstroke}%
\pgfsetdash{}{0pt}%
\pgfpathmoveto{\pgfqpoint{1.240061in}{2.239092in}}%
\pgfpathlineto{\pgfqpoint{1.283599in}{2.143924in}}%
\pgfusepath{stroke}%
\end{pgfscope}%
\begin{pgfscope}%
\pgfpathrectangle{\pgfqpoint{0.100000in}{0.212622in}}{\pgfqpoint{3.696000in}{3.696000in}}%
\pgfusepath{clip}%
\pgfsetrectcap%
\pgfsetroundjoin%
\pgfsetlinewidth{1.505625pt}%
\definecolor{currentstroke}{rgb}{1.000000,0.000000,0.000000}%
\pgfsetstrokecolor{currentstroke}%
\pgfsetdash{}{0pt}%
\pgfpathmoveto{\pgfqpoint{1.245126in}{2.248857in}}%
\pgfpathlineto{\pgfqpoint{1.289972in}{2.149147in}}%
\pgfusepath{stroke}%
\end{pgfscope}%
\begin{pgfscope}%
\pgfpathrectangle{\pgfqpoint{0.100000in}{0.212622in}}{\pgfqpoint{3.696000in}{3.696000in}}%
\pgfusepath{clip}%
\pgfsetrectcap%
\pgfsetroundjoin%
\pgfsetlinewidth{1.505625pt}%
\definecolor{currentstroke}{rgb}{1.000000,0.000000,0.000000}%
\pgfsetstrokecolor{currentstroke}%
\pgfsetdash{}{0pt}%
\pgfpathmoveto{\pgfqpoint{1.250840in}{2.261436in}}%
\pgfpathlineto{\pgfqpoint{1.296339in}{2.154366in}}%
\pgfusepath{stroke}%
\end{pgfscope}%
\begin{pgfscope}%
\pgfpathrectangle{\pgfqpoint{0.100000in}{0.212622in}}{\pgfqpoint{3.696000in}{3.696000in}}%
\pgfusepath{clip}%
\pgfsetrectcap%
\pgfsetroundjoin%
\pgfsetlinewidth{1.505625pt}%
\definecolor{currentstroke}{rgb}{1.000000,0.000000,0.000000}%
\pgfsetstrokecolor{currentstroke}%
\pgfsetdash{}{0pt}%
\pgfpathmoveto{\pgfqpoint{1.256178in}{2.275536in}}%
\pgfpathlineto{\pgfqpoint{1.302701in}{2.159579in}}%
\pgfusepath{stroke}%
\end{pgfscope}%
\begin{pgfscope}%
\pgfpathrectangle{\pgfqpoint{0.100000in}{0.212622in}}{\pgfqpoint{3.696000in}{3.696000in}}%
\pgfusepath{clip}%
\pgfsetrectcap%
\pgfsetroundjoin%
\pgfsetlinewidth{1.505625pt}%
\definecolor{currentstroke}{rgb}{1.000000,0.000000,0.000000}%
\pgfsetstrokecolor{currentstroke}%
\pgfsetdash{}{0pt}%
\pgfpathmoveto{\pgfqpoint{1.259316in}{2.283676in}}%
\pgfpathlineto{\pgfqpoint{1.309057in}{2.164788in}}%
\pgfusepath{stroke}%
\end{pgfscope}%
\begin{pgfscope}%
\pgfpathrectangle{\pgfqpoint{0.100000in}{0.212622in}}{\pgfqpoint{3.696000in}{3.696000in}}%
\pgfusepath{clip}%
\pgfsetrectcap%
\pgfsetroundjoin%
\pgfsetlinewidth{1.505625pt}%
\definecolor{currentstroke}{rgb}{1.000000,0.000000,0.000000}%
\pgfsetstrokecolor{currentstroke}%
\pgfsetdash{}{0pt}%
\pgfpathmoveto{\pgfqpoint{1.263287in}{2.294975in}}%
\pgfpathlineto{\pgfqpoint{1.315407in}{2.169992in}}%
\pgfusepath{stroke}%
\end{pgfscope}%
\begin{pgfscope}%
\pgfpathrectangle{\pgfqpoint{0.100000in}{0.212622in}}{\pgfqpoint{3.696000in}{3.696000in}}%
\pgfusepath{clip}%
\pgfsetrectcap%
\pgfsetroundjoin%
\pgfsetlinewidth{1.505625pt}%
\definecolor{currentstroke}{rgb}{1.000000,0.000000,0.000000}%
\pgfsetstrokecolor{currentstroke}%
\pgfsetdash{}{0pt}%
\pgfpathmoveto{\pgfqpoint{1.268416in}{2.310555in}}%
\pgfpathlineto{\pgfqpoint{1.321751in}{2.175192in}}%
\pgfusepath{stroke}%
\end{pgfscope}%
\begin{pgfscope}%
\pgfpathrectangle{\pgfqpoint{0.100000in}{0.212622in}}{\pgfqpoint{3.696000in}{3.696000in}}%
\pgfusepath{clip}%
\pgfsetrectcap%
\pgfsetroundjoin%
\pgfsetlinewidth{1.505625pt}%
\definecolor{currentstroke}{rgb}{1.000000,0.000000,0.000000}%
\pgfsetstrokecolor{currentstroke}%
\pgfsetdash{}{0pt}%
\pgfpathmoveto{\pgfqpoint{1.274531in}{2.327681in}}%
\pgfpathlineto{\pgfqpoint{1.328089in}{2.180387in}}%
\pgfusepath{stroke}%
\end{pgfscope}%
\begin{pgfscope}%
\pgfpathrectangle{\pgfqpoint{0.100000in}{0.212622in}}{\pgfqpoint{3.696000in}{3.696000in}}%
\pgfusepath{clip}%
\pgfsetrectcap%
\pgfsetroundjoin%
\pgfsetlinewidth{1.505625pt}%
\definecolor{currentstroke}{rgb}{1.000000,0.000000,0.000000}%
\pgfsetstrokecolor{currentstroke}%
\pgfsetdash{}{0pt}%
\pgfpathmoveto{\pgfqpoint{1.282437in}{2.349063in}}%
\pgfpathlineto{\pgfqpoint{1.340749in}{2.190762in}}%
\pgfusepath{stroke}%
\end{pgfscope}%
\begin{pgfscope}%
\pgfpathrectangle{\pgfqpoint{0.100000in}{0.212622in}}{\pgfqpoint{3.696000in}{3.696000in}}%
\pgfusepath{clip}%
\pgfsetrectcap%
\pgfsetroundjoin%
\pgfsetlinewidth{1.505625pt}%
\definecolor{currentstroke}{rgb}{1.000000,0.000000,0.000000}%
\pgfsetstrokecolor{currentstroke}%
\pgfsetdash{}{0pt}%
\pgfpathmoveto{\pgfqpoint{1.290291in}{2.371825in}}%
\pgfpathlineto{\pgfqpoint{1.353386in}{2.201119in}}%
\pgfusepath{stroke}%
\end{pgfscope}%
\begin{pgfscope}%
\pgfpathrectangle{\pgfqpoint{0.100000in}{0.212622in}}{\pgfqpoint{3.696000in}{3.696000in}}%
\pgfusepath{clip}%
\pgfsetrectcap%
\pgfsetroundjoin%
\pgfsetlinewidth{1.505625pt}%
\definecolor{currentstroke}{rgb}{1.000000,0.000000,0.000000}%
\pgfsetstrokecolor{currentstroke}%
\pgfsetdash{}{0pt}%
\pgfpathmoveto{\pgfqpoint{1.301486in}{2.399908in}}%
\pgfpathlineto{\pgfqpoint{1.366000in}{2.211457in}}%
\pgfusepath{stroke}%
\end{pgfscope}%
\begin{pgfscope}%
\pgfpathrectangle{\pgfqpoint{0.100000in}{0.212622in}}{\pgfqpoint{3.696000in}{3.696000in}}%
\pgfusepath{clip}%
\pgfsetrectcap%
\pgfsetroundjoin%
\pgfsetlinewidth{1.505625pt}%
\definecolor{currentstroke}{rgb}{1.000000,0.000000,0.000000}%
\pgfsetstrokecolor{currentstroke}%
\pgfsetdash{}{0pt}%
\pgfpathmoveto{\pgfqpoint{1.315750in}{2.432615in}}%
\pgfpathlineto{\pgfqpoint{1.384878in}{2.226929in}}%
\pgfusepath{stroke}%
\end{pgfscope}%
\begin{pgfscope}%
\pgfpathrectangle{\pgfqpoint{0.100000in}{0.212622in}}{\pgfqpoint{3.696000in}{3.696000in}}%
\pgfusepath{clip}%
\pgfsetrectcap%
\pgfsetroundjoin%
\pgfsetlinewidth{1.505625pt}%
\definecolor{currentstroke}{rgb}{1.000000,0.000000,0.000000}%
\pgfsetstrokecolor{currentstroke}%
\pgfsetdash{}{0pt}%
\pgfpathmoveto{\pgfqpoint{1.323902in}{2.449905in}}%
\pgfpathlineto{\pgfqpoint{1.391159in}{2.232077in}}%
\pgfusepath{stroke}%
\end{pgfscope}%
\begin{pgfscope}%
\pgfpathrectangle{\pgfqpoint{0.100000in}{0.212622in}}{\pgfqpoint{3.696000in}{3.696000in}}%
\pgfusepath{clip}%
\pgfsetrectcap%
\pgfsetroundjoin%
\pgfsetlinewidth{1.505625pt}%
\definecolor{currentstroke}{rgb}{1.000000,0.000000,0.000000}%
\pgfsetstrokecolor{currentstroke}%
\pgfsetdash{}{0pt}%
\pgfpathmoveto{\pgfqpoint{1.333536in}{2.470797in}}%
\pgfpathlineto{\pgfqpoint{1.403705in}{2.242359in}}%
\pgfusepath{stroke}%
\end{pgfscope}%
\begin{pgfscope}%
\pgfpathrectangle{\pgfqpoint{0.100000in}{0.212622in}}{\pgfqpoint{3.696000in}{3.696000in}}%
\pgfusepath{clip}%
\pgfsetrectcap%
\pgfsetroundjoin%
\pgfsetlinewidth{1.505625pt}%
\definecolor{currentstroke}{rgb}{1.000000,0.000000,0.000000}%
\pgfsetstrokecolor{currentstroke}%
\pgfsetdash{}{0pt}%
\pgfpathmoveto{\pgfqpoint{1.338585in}{2.482193in}}%
\pgfpathlineto{\pgfqpoint{1.409970in}{2.247493in}}%
\pgfusepath{stroke}%
\end{pgfscope}%
\begin{pgfscope}%
\pgfpathrectangle{\pgfqpoint{0.100000in}{0.212622in}}{\pgfqpoint{3.696000in}{3.696000in}}%
\pgfusepath{clip}%
\pgfsetrectcap%
\pgfsetroundjoin%
\pgfsetlinewidth{1.505625pt}%
\definecolor{currentstroke}{rgb}{1.000000,0.000000,0.000000}%
\pgfsetstrokecolor{currentstroke}%
\pgfsetdash{}{0pt}%
\pgfpathmoveto{\pgfqpoint{1.341421in}{2.488431in}}%
\pgfpathlineto{\pgfqpoint{1.409970in}{2.247493in}}%
\pgfusepath{stroke}%
\end{pgfscope}%
\begin{pgfscope}%
\pgfpathrectangle{\pgfqpoint{0.100000in}{0.212622in}}{\pgfqpoint{3.696000in}{3.696000in}}%
\pgfusepath{clip}%
\pgfsetrectcap%
\pgfsetroundjoin%
\pgfsetlinewidth{1.505625pt}%
\definecolor{currentstroke}{rgb}{1.000000,0.000000,0.000000}%
\pgfsetstrokecolor{currentstroke}%
\pgfsetdash{}{0pt}%
\pgfpathmoveto{\pgfqpoint{1.346441in}{2.499609in}}%
\pgfpathlineto{\pgfqpoint{1.416229in}{2.252623in}}%
\pgfusepath{stroke}%
\end{pgfscope}%
\begin{pgfscope}%
\pgfpathrectangle{\pgfqpoint{0.100000in}{0.212622in}}{\pgfqpoint{3.696000in}{3.696000in}}%
\pgfusepath{clip}%
\pgfsetrectcap%
\pgfsetroundjoin%
\pgfsetlinewidth{1.505625pt}%
\definecolor{currentstroke}{rgb}{1.000000,0.000000,0.000000}%
\pgfsetstrokecolor{currentstroke}%
\pgfsetdash{}{0pt}%
\pgfpathmoveto{\pgfqpoint{1.354030in}{2.516413in}}%
\pgfpathlineto{\pgfqpoint{1.428729in}{2.262868in}}%
\pgfusepath{stroke}%
\end{pgfscope}%
\begin{pgfscope}%
\pgfpathrectangle{\pgfqpoint{0.100000in}{0.212622in}}{\pgfqpoint{3.696000in}{3.696000in}}%
\pgfusepath{clip}%
\pgfsetrectcap%
\pgfsetroundjoin%
\pgfsetlinewidth{1.505625pt}%
\definecolor{currentstroke}{rgb}{1.000000,0.000000,0.000000}%
\pgfsetstrokecolor{currentstroke}%
\pgfsetdash{}{0pt}%
\pgfpathmoveto{\pgfqpoint{1.364218in}{2.537356in}}%
\pgfpathlineto{\pgfqpoint{1.434971in}{2.267983in}}%
\pgfusepath{stroke}%
\end{pgfscope}%
\begin{pgfscope}%
\pgfpathrectangle{\pgfqpoint{0.100000in}{0.212622in}}{\pgfqpoint{3.696000in}{3.696000in}}%
\pgfusepath{clip}%
\pgfsetrectcap%
\pgfsetroundjoin%
\pgfsetlinewidth{1.505625pt}%
\definecolor{currentstroke}{rgb}{1.000000,0.000000,0.000000}%
\pgfsetstrokecolor{currentstroke}%
\pgfsetdash{}{0pt}%
\pgfpathmoveto{\pgfqpoint{1.376159in}{2.562944in}}%
\pgfpathlineto{\pgfqpoint{1.453663in}{2.283303in}}%
\pgfusepath{stroke}%
\end{pgfscope}%
\begin{pgfscope}%
\pgfpathrectangle{\pgfqpoint{0.100000in}{0.212622in}}{\pgfqpoint{3.696000in}{3.696000in}}%
\pgfusepath{clip}%
\pgfsetrectcap%
\pgfsetroundjoin%
\pgfsetlinewidth{1.505625pt}%
\definecolor{currentstroke}{rgb}{1.000000,0.000000,0.000000}%
\pgfsetstrokecolor{currentstroke}%
\pgfsetdash{}{0pt}%
\pgfpathmoveto{\pgfqpoint{1.390669in}{2.594109in}}%
\pgfpathlineto{\pgfqpoint{1.466097in}{2.293493in}}%
\pgfusepath{stroke}%
\end{pgfscope}%
\begin{pgfscope}%
\pgfpathrectangle{\pgfqpoint{0.100000in}{0.212622in}}{\pgfqpoint{3.696000in}{3.696000in}}%
\pgfusepath{clip}%
\pgfsetrectcap%
\pgfsetroundjoin%
\pgfsetlinewidth{1.505625pt}%
\definecolor{currentstroke}{rgb}{1.000000,0.000000,0.000000}%
\pgfsetstrokecolor{currentstroke}%
\pgfsetdash{}{0pt}%
\pgfpathmoveto{\pgfqpoint{1.407401in}{2.632269in}}%
\pgfpathlineto{\pgfqpoint{1.484706in}{2.308744in}}%
\pgfusepath{stroke}%
\end{pgfscope}%
\begin{pgfscope}%
\pgfpathrectangle{\pgfqpoint{0.100000in}{0.212622in}}{\pgfqpoint{3.696000in}{3.696000in}}%
\pgfusepath{clip}%
\pgfsetrectcap%
\pgfsetroundjoin%
\pgfsetlinewidth{1.505625pt}%
\definecolor{currentstroke}{rgb}{1.000000,0.000000,0.000000}%
\pgfsetstrokecolor{currentstroke}%
\pgfsetdash{}{0pt}%
\pgfpathmoveto{\pgfqpoint{1.425680in}{2.673914in}}%
\pgfpathlineto{\pgfqpoint{1.509439in}{2.329015in}}%
\pgfusepath{stroke}%
\end{pgfscope}%
\begin{pgfscope}%
\pgfpathrectangle{\pgfqpoint{0.100000in}{0.212622in}}{\pgfqpoint{3.696000in}{3.696000in}}%
\pgfusepath{clip}%
\pgfsetrectcap%
\pgfsetroundjoin%
\pgfsetlinewidth{1.505625pt}%
\definecolor{currentstroke}{rgb}{1.000000,0.000000,0.000000}%
\pgfsetstrokecolor{currentstroke}%
\pgfsetdash{}{0pt}%
\pgfpathmoveto{\pgfqpoint{1.446653in}{2.722659in}}%
\pgfpathlineto{\pgfqpoint{1.534085in}{2.349213in}}%
\pgfusepath{stroke}%
\end{pgfscope}%
\begin{pgfscope}%
\pgfpathrectangle{\pgfqpoint{0.100000in}{0.212622in}}{\pgfqpoint{3.696000in}{3.696000in}}%
\pgfusepath{clip}%
\pgfsetrectcap%
\pgfsetroundjoin%
\pgfsetlinewidth{1.505625pt}%
\definecolor{currentstroke}{rgb}{1.000000,0.000000,0.000000}%
\pgfsetstrokecolor{currentstroke}%
\pgfsetdash{}{0pt}%
\pgfpathmoveto{\pgfqpoint{1.471727in}{2.772243in}}%
\pgfpathlineto{\pgfqpoint{1.558642in}{2.369340in}}%
\pgfusepath{stroke}%
\end{pgfscope}%
\begin{pgfscope}%
\pgfpathrectangle{\pgfqpoint{0.100000in}{0.212622in}}{\pgfqpoint{3.696000in}{3.696000in}}%
\pgfusepath{clip}%
\pgfsetrectcap%
\pgfsetroundjoin%
\pgfsetlinewidth{1.505625pt}%
\definecolor{currentstroke}{rgb}{1.000000,0.000000,0.000000}%
\pgfsetstrokecolor{currentstroke}%
\pgfsetdash{}{0pt}%
\pgfpathmoveto{\pgfqpoint{1.497090in}{2.826066in}}%
\pgfpathlineto{\pgfqpoint{1.589217in}{2.394398in}}%
\pgfusepath{stroke}%
\end{pgfscope}%
\begin{pgfscope}%
\pgfpathrectangle{\pgfqpoint{0.100000in}{0.212622in}}{\pgfqpoint{3.696000in}{3.696000in}}%
\pgfusepath{clip}%
\pgfsetrectcap%
\pgfsetroundjoin%
\pgfsetlinewidth{1.505625pt}%
\definecolor{currentstroke}{rgb}{1.000000,0.000000,0.000000}%
\pgfsetstrokecolor{currentstroke}%
\pgfsetdash{}{0pt}%
\pgfpathmoveto{\pgfqpoint{1.510401in}{2.855336in}}%
\pgfpathlineto{\pgfqpoint{1.607496in}{2.409379in}}%
\pgfusepath{stroke}%
\end{pgfscope}%
\begin{pgfscope}%
\pgfpathrectangle{\pgfqpoint{0.100000in}{0.212622in}}{\pgfqpoint{3.696000in}{3.696000in}}%
\pgfusepath{clip}%
\pgfsetrectcap%
\pgfsetroundjoin%
\pgfsetlinewidth{1.505625pt}%
\definecolor{currentstroke}{rgb}{1.000000,0.000000,0.000000}%
\pgfsetstrokecolor{currentstroke}%
\pgfsetdash{}{0pt}%
\pgfpathmoveto{\pgfqpoint{1.525110in}{2.886267in}}%
\pgfpathlineto{\pgfqpoint{1.619655in}{2.419344in}}%
\pgfusepath{stroke}%
\end{pgfscope}%
\begin{pgfscope}%
\pgfpathrectangle{\pgfqpoint{0.100000in}{0.212622in}}{\pgfqpoint{3.696000in}{3.696000in}}%
\pgfusepath{clip}%
\pgfsetrectcap%
\pgfsetroundjoin%
\pgfsetlinewidth{1.505625pt}%
\definecolor{currentstroke}{rgb}{1.000000,0.000000,0.000000}%
\pgfsetstrokecolor{currentstroke}%
\pgfsetdash{}{0pt}%
\pgfpathmoveto{\pgfqpoint{1.532791in}{2.903182in}}%
\pgfpathlineto{\pgfqpoint{1.631793in}{2.429292in}}%
\pgfusepath{stroke}%
\end{pgfscope}%
\begin{pgfscope}%
\pgfpathrectangle{\pgfqpoint{0.100000in}{0.212622in}}{\pgfqpoint{3.696000in}{3.696000in}}%
\pgfusepath{clip}%
\pgfsetrectcap%
\pgfsetroundjoin%
\pgfsetlinewidth{1.505625pt}%
\definecolor{currentstroke}{rgb}{1.000000,0.000000,0.000000}%
\pgfsetstrokecolor{currentstroke}%
\pgfsetdash{}{0pt}%
\pgfpathmoveto{\pgfqpoint{1.542254in}{2.923692in}}%
\pgfpathlineto{\pgfqpoint{1.637854in}{2.434259in}}%
\pgfusepath{stroke}%
\end{pgfscope}%
\begin{pgfscope}%
\pgfpathrectangle{\pgfqpoint{0.100000in}{0.212622in}}{\pgfqpoint{3.696000in}{3.696000in}}%
\pgfusepath{clip}%
\pgfsetrectcap%
\pgfsetroundjoin%
\pgfsetlinewidth{1.505625pt}%
\definecolor{currentstroke}{rgb}{1.000000,0.000000,0.000000}%
\pgfsetstrokecolor{currentstroke}%
\pgfsetdash{}{0pt}%
\pgfpathmoveto{\pgfqpoint{1.547052in}{2.934820in}}%
\pgfpathlineto{\pgfqpoint{1.643909in}{2.439222in}}%
\pgfusepath{stroke}%
\end{pgfscope}%
\begin{pgfscope}%
\pgfpathrectangle{\pgfqpoint{0.100000in}{0.212622in}}{\pgfqpoint{3.696000in}{3.696000in}}%
\pgfusepath{clip}%
\pgfsetrectcap%
\pgfsetroundjoin%
\pgfsetlinewidth{1.505625pt}%
\definecolor{currentstroke}{rgb}{1.000000,0.000000,0.000000}%
\pgfsetstrokecolor{currentstroke}%
\pgfsetdash{}{0pt}%
\pgfpathmoveto{\pgfqpoint{1.553125in}{2.948754in}}%
\pgfpathlineto{\pgfqpoint{1.656004in}{2.449134in}}%
\pgfusepath{stroke}%
\end{pgfscope}%
\begin{pgfscope}%
\pgfpathrectangle{\pgfqpoint{0.100000in}{0.212622in}}{\pgfqpoint{3.696000in}{3.696000in}}%
\pgfusepath{clip}%
\pgfsetrectcap%
\pgfsetroundjoin%
\pgfsetlinewidth{1.505625pt}%
\definecolor{currentstroke}{rgb}{1.000000,0.000000,0.000000}%
\pgfsetstrokecolor{currentstroke}%
\pgfsetdash{}{0pt}%
\pgfpathmoveto{\pgfqpoint{1.556425in}{2.956359in}}%
\pgfpathlineto{\pgfqpoint{1.656004in}{2.449134in}}%
\pgfusepath{stroke}%
\end{pgfscope}%
\begin{pgfscope}%
\pgfpathrectangle{\pgfqpoint{0.100000in}{0.212622in}}{\pgfqpoint{3.696000in}{3.696000in}}%
\pgfusepath{clip}%
\pgfsetrectcap%
\pgfsetroundjoin%
\pgfsetlinewidth{1.505625pt}%
\definecolor{currentstroke}{rgb}{1.000000,0.000000,0.000000}%
\pgfsetstrokecolor{currentstroke}%
\pgfsetdash{}{0pt}%
\pgfpathmoveto{\pgfqpoint{1.558316in}{2.960595in}}%
\pgfpathlineto{\pgfqpoint{1.662043in}{2.454084in}}%
\pgfusepath{stroke}%
\end{pgfscope}%
\begin{pgfscope}%
\pgfpathrectangle{\pgfqpoint{0.100000in}{0.212622in}}{\pgfqpoint{3.696000in}{3.696000in}}%
\pgfusepath{clip}%
\pgfsetrectcap%
\pgfsetroundjoin%
\pgfsetlinewidth{1.505625pt}%
\definecolor{currentstroke}{rgb}{1.000000,0.000000,0.000000}%
\pgfsetstrokecolor{currentstroke}%
\pgfsetdash{}{0pt}%
\pgfpathmoveto{\pgfqpoint{1.559324in}{2.962862in}}%
\pgfpathlineto{\pgfqpoint{1.662043in}{2.454084in}}%
\pgfusepath{stroke}%
\end{pgfscope}%
\begin{pgfscope}%
\pgfpathrectangle{\pgfqpoint{0.100000in}{0.212622in}}{\pgfqpoint{3.696000in}{3.696000in}}%
\pgfusepath{clip}%
\pgfsetrectcap%
\pgfsetroundjoin%
\pgfsetlinewidth{1.505625pt}%
\definecolor{currentstroke}{rgb}{1.000000,0.000000,0.000000}%
\pgfsetstrokecolor{currentstroke}%
\pgfsetdash{}{0pt}%
\pgfpathmoveto{\pgfqpoint{1.561171in}{2.966906in}}%
\pgfpathlineto{\pgfqpoint{1.662043in}{2.454084in}}%
\pgfusepath{stroke}%
\end{pgfscope}%
\begin{pgfscope}%
\pgfpathrectangle{\pgfqpoint{0.100000in}{0.212622in}}{\pgfqpoint{3.696000in}{3.696000in}}%
\pgfusepath{clip}%
\pgfsetrectcap%
\pgfsetroundjoin%
\pgfsetlinewidth{1.505625pt}%
\definecolor{currentstroke}{rgb}{1.000000,0.000000,0.000000}%
\pgfsetstrokecolor{currentstroke}%
\pgfsetdash{}{0pt}%
\pgfpathmoveto{\pgfqpoint{1.562174in}{2.969106in}}%
\pgfpathlineto{\pgfqpoint{1.662043in}{2.454084in}}%
\pgfusepath{stroke}%
\end{pgfscope}%
\begin{pgfscope}%
\pgfpathrectangle{\pgfqpoint{0.100000in}{0.212622in}}{\pgfqpoint{3.696000in}{3.696000in}}%
\pgfusepath{clip}%
\pgfsetrectcap%
\pgfsetroundjoin%
\pgfsetlinewidth{1.505625pt}%
\definecolor{currentstroke}{rgb}{1.000000,0.000000,0.000000}%
\pgfsetstrokecolor{currentstroke}%
\pgfsetdash{}{0pt}%
\pgfpathmoveto{\pgfqpoint{1.564144in}{2.973363in}}%
\pgfpathlineto{\pgfqpoint{1.668077in}{2.459029in}}%
\pgfusepath{stroke}%
\end{pgfscope}%
\begin{pgfscope}%
\pgfpathrectangle{\pgfqpoint{0.100000in}{0.212622in}}{\pgfqpoint{3.696000in}{3.696000in}}%
\pgfusepath{clip}%
\pgfsetrectcap%
\pgfsetroundjoin%
\pgfsetlinewidth{1.505625pt}%
\definecolor{currentstroke}{rgb}{1.000000,0.000000,0.000000}%
\pgfsetstrokecolor{currentstroke}%
\pgfsetdash{}{0pt}%
\pgfpathmoveto{\pgfqpoint{1.565263in}{2.975739in}}%
\pgfpathlineto{\pgfqpoint{1.668077in}{2.459029in}}%
\pgfusepath{stroke}%
\end{pgfscope}%
\begin{pgfscope}%
\pgfpathrectangle{\pgfqpoint{0.100000in}{0.212622in}}{\pgfqpoint{3.696000in}{3.696000in}}%
\pgfusepath{clip}%
\pgfsetrectcap%
\pgfsetroundjoin%
\pgfsetlinewidth{1.505625pt}%
\definecolor{currentstroke}{rgb}{1.000000,0.000000,0.000000}%
\pgfsetstrokecolor{currentstroke}%
\pgfsetdash{}{0pt}%
\pgfpathmoveto{\pgfqpoint{1.567109in}{2.979627in}}%
\pgfpathlineto{\pgfqpoint{1.668077in}{2.459029in}}%
\pgfusepath{stroke}%
\end{pgfscope}%
\begin{pgfscope}%
\pgfpathrectangle{\pgfqpoint{0.100000in}{0.212622in}}{\pgfqpoint{3.696000in}{3.696000in}}%
\pgfusepath{clip}%
\pgfsetrectcap%
\pgfsetroundjoin%
\pgfsetlinewidth{1.505625pt}%
\definecolor{currentstroke}{rgb}{1.000000,0.000000,0.000000}%
\pgfsetstrokecolor{currentstroke}%
\pgfsetdash{}{0pt}%
\pgfpathmoveto{\pgfqpoint{1.569669in}{2.985296in}}%
\pgfpathlineto{\pgfqpoint{1.674106in}{2.463970in}}%
\pgfusepath{stroke}%
\end{pgfscope}%
\begin{pgfscope}%
\pgfpathrectangle{\pgfqpoint{0.100000in}{0.212622in}}{\pgfqpoint{3.696000in}{3.696000in}}%
\pgfusepath{clip}%
\pgfsetrectcap%
\pgfsetroundjoin%
\pgfsetlinewidth{1.505625pt}%
\definecolor{currentstroke}{rgb}{1.000000,0.000000,0.000000}%
\pgfsetstrokecolor{currentstroke}%
\pgfsetdash{}{0pt}%
\pgfpathmoveto{\pgfqpoint{1.573088in}{2.993137in}}%
\pgfpathlineto{\pgfqpoint{1.674106in}{2.463970in}}%
\pgfusepath{stroke}%
\end{pgfscope}%
\begin{pgfscope}%
\pgfpathrectangle{\pgfqpoint{0.100000in}{0.212622in}}{\pgfqpoint{3.696000in}{3.696000in}}%
\pgfusepath{clip}%
\pgfsetrectcap%
\pgfsetroundjoin%
\pgfsetlinewidth{1.505625pt}%
\definecolor{currentstroke}{rgb}{1.000000,0.000000,0.000000}%
\pgfsetstrokecolor{currentstroke}%
\pgfsetdash{}{0pt}%
\pgfpathmoveto{\pgfqpoint{1.574743in}{2.997153in}}%
\pgfpathlineto{\pgfqpoint{1.680130in}{2.468906in}}%
\pgfusepath{stroke}%
\end{pgfscope}%
\begin{pgfscope}%
\pgfpathrectangle{\pgfqpoint{0.100000in}{0.212622in}}{\pgfqpoint{3.696000in}{3.696000in}}%
\pgfusepath{clip}%
\pgfsetrectcap%
\pgfsetroundjoin%
\pgfsetlinewidth{1.505625pt}%
\definecolor{currentstroke}{rgb}{1.000000,0.000000,0.000000}%
\pgfsetstrokecolor{currentstroke}%
\pgfsetdash{}{0pt}%
\pgfpathmoveto{\pgfqpoint{1.578017in}{3.003828in}}%
\pgfpathlineto{\pgfqpoint{1.680130in}{2.468906in}}%
\pgfusepath{stroke}%
\end{pgfscope}%
\begin{pgfscope}%
\pgfpathrectangle{\pgfqpoint{0.100000in}{0.212622in}}{\pgfqpoint{3.696000in}{3.696000in}}%
\pgfusepath{clip}%
\pgfsetrectcap%
\pgfsetroundjoin%
\pgfsetlinewidth{1.505625pt}%
\definecolor{currentstroke}{rgb}{1.000000,0.000000,0.000000}%
\pgfsetstrokecolor{currentstroke}%
\pgfsetdash{}{0pt}%
\pgfpathmoveto{\pgfqpoint{1.582163in}{3.012283in}}%
\pgfpathlineto{\pgfqpoint{1.686148in}{2.473839in}}%
\pgfusepath{stroke}%
\end{pgfscope}%
\begin{pgfscope}%
\pgfpathrectangle{\pgfqpoint{0.100000in}{0.212622in}}{\pgfqpoint{3.696000in}{3.696000in}}%
\pgfusepath{clip}%
\pgfsetrectcap%
\pgfsetroundjoin%
\pgfsetlinewidth{1.505625pt}%
\definecolor{currentstroke}{rgb}{1.000000,0.000000,0.000000}%
\pgfsetstrokecolor{currentstroke}%
\pgfsetdash{}{0pt}%
\pgfpathmoveto{\pgfqpoint{1.584409in}{3.016978in}}%
\pgfpathlineto{\pgfqpoint{1.692160in}{2.478766in}}%
\pgfusepath{stroke}%
\end{pgfscope}%
\begin{pgfscope}%
\pgfpathrectangle{\pgfqpoint{0.100000in}{0.212622in}}{\pgfqpoint{3.696000in}{3.696000in}}%
\pgfusepath{clip}%
\pgfsetrectcap%
\pgfsetroundjoin%
\pgfsetlinewidth{1.505625pt}%
\definecolor{currentstroke}{rgb}{1.000000,0.000000,0.000000}%
\pgfsetstrokecolor{currentstroke}%
\pgfsetdash{}{0pt}%
\pgfpathmoveto{\pgfqpoint{1.585590in}{3.019522in}}%
\pgfpathlineto{\pgfqpoint{1.692160in}{2.478766in}}%
\pgfusepath{stroke}%
\end{pgfscope}%
\begin{pgfscope}%
\pgfpathrectangle{\pgfqpoint{0.100000in}{0.212622in}}{\pgfqpoint{3.696000in}{3.696000in}}%
\pgfusepath{clip}%
\pgfsetrectcap%
\pgfsetroundjoin%
\pgfsetlinewidth{1.505625pt}%
\definecolor{currentstroke}{rgb}{1.000000,0.000000,0.000000}%
\pgfsetstrokecolor{currentstroke}%
\pgfsetdash{}{0pt}%
\pgfpathmoveto{\pgfqpoint{1.586274in}{3.020943in}}%
\pgfpathlineto{\pgfqpoint{1.692160in}{2.478766in}}%
\pgfusepath{stroke}%
\end{pgfscope}%
\begin{pgfscope}%
\pgfpathrectangle{\pgfqpoint{0.100000in}{0.212622in}}{\pgfqpoint{3.696000in}{3.696000in}}%
\pgfusepath{clip}%
\pgfsetrectcap%
\pgfsetroundjoin%
\pgfsetlinewidth{1.505625pt}%
\definecolor{currentstroke}{rgb}{1.000000,0.000000,0.000000}%
\pgfsetstrokecolor{currentstroke}%
\pgfsetdash{}{0pt}%
\pgfpathmoveto{\pgfqpoint{1.586645in}{3.021716in}}%
\pgfpathlineto{\pgfqpoint{1.692160in}{2.478766in}}%
\pgfusepath{stroke}%
\end{pgfscope}%
\begin{pgfscope}%
\pgfpathrectangle{\pgfqpoint{0.100000in}{0.212622in}}{\pgfqpoint{3.696000in}{3.696000in}}%
\pgfusepath{clip}%
\pgfsetrectcap%
\pgfsetroundjoin%
\pgfsetlinewidth{1.505625pt}%
\definecolor{currentstroke}{rgb}{1.000000,0.000000,0.000000}%
\pgfsetstrokecolor{currentstroke}%
\pgfsetdash{}{0pt}%
\pgfpathmoveto{\pgfqpoint{1.586850in}{3.022148in}}%
\pgfpathlineto{\pgfqpoint{1.692160in}{2.478766in}}%
\pgfusepath{stroke}%
\end{pgfscope}%
\begin{pgfscope}%
\pgfpathrectangle{\pgfqpoint{0.100000in}{0.212622in}}{\pgfqpoint{3.696000in}{3.696000in}}%
\pgfusepath{clip}%
\pgfsetrectcap%
\pgfsetroundjoin%
\pgfsetlinewidth{1.505625pt}%
\definecolor{currentstroke}{rgb}{1.000000,0.000000,0.000000}%
\pgfsetstrokecolor{currentstroke}%
\pgfsetdash{}{0pt}%
\pgfpathmoveto{\pgfqpoint{1.588025in}{3.024786in}}%
\pgfpathlineto{\pgfqpoint{1.692160in}{2.478766in}}%
\pgfusepath{stroke}%
\end{pgfscope}%
\begin{pgfscope}%
\pgfpathrectangle{\pgfqpoint{0.100000in}{0.212622in}}{\pgfqpoint{3.696000in}{3.696000in}}%
\pgfusepath{clip}%
\pgfsetrectcap%
\pgfsetroundjoin%
\pgfsetlinewidth{1.505625pt}%
\definecolor{currentstroke}{rgb}{1.000000,0.000000,0.000000}%
\pgfsetstrokecolor{currentstroke}%
\pgfsetdash{}{0pt}%
\pgfpathmoveto{\pgfqpoint{1.588689in}{3.026256in}}%
\pgfpathlineto{\pgfqpoint{1.692160in}{2.478766in}}%
\pgfusepath{stroke}%
\end{pgfscope}%
\begin{pgfscope}%
\pgfpathrectangle{\pgfqpoint{0.100000in}{0.212622in}}{\pgfqpoint{3.696000in}{3.696000in}}%
\pgfusepath{clip}%
\pgfsetrectcap%
\pgfsetroundjoin%
\pgfsetlinewidth{1.505625pt}%
\definecolor{currentstroke}{rgb}{1.000000,0.000000,0.000000}%
\pgfsetstrokecolor{currentstroke}%
\pgfsetdash{}{0pt}%
\pgfpathmoveto{\pgfqpoint{1.589054in}{3.027054in}}%
\pgfpathlineto{\pgfqpoint{1.692160in}{2.478766in}}%
\pgfusepath{stroke}%
\end{pgfscope}%
\begin{pgfscope}%
\pgfpathrectangle{\pgfqpoint{0.100000in}{0.212622in}}{\pgfqpoint{3.696000in}{3.696000in}}%
\pgfusepath{clip}%
\pgfsetrectcap%
\pgfsetroundjoin%
\pgfsetlinewidth{1.505625pt}%
\definecolor{currentstroke}{rgb}{1.000000,0.000000,0.000000}%
\pgfsetstrokecolor{currentstroke}%
\pgfsetdash{}{0pt}%
\pgfpathmoveto{\pgfqpoint{1.591125in}{3.031538in}}%
\pgfpathlineto{\pgfqpoint{1.698168in}{2.483690in}}%
\pgfusepath{stroke}%
\end{pgfscope}%
\begin{pgfscope}%
\pgfpathrectangle{\pgfqpoint{0.100000in}{0.212622in}}{\pgfqpoint{3.696000in}{3.696000in}}%
\pgfusepath{clip}%
\pgfsetrectcap%
\pgfsetroundjoin%
\pgfsetlinewidth{1.505625pt}%
\definecolor{currentstroke}{rgb}{1.000000,0.000000,0.000000}%
\pgfsetstrokecolor{currentstroke}%
\pgfsetdash{}{0pt}%
\pgfpathmoveto{\pgfqpoint{1.592243in}{3.034022in}}%
\pgfpathlineto{\pgfqpoint{1.698168in}{2.483690in}}%
\pgfusepath{stroke}%
\end{pgfscope}%
\begin{pgfscope}%
\pgfpathrectangle{\pgfqpoint{0.100000in}{0.212622in}}{\pgfqpoint{3.696000in}{3.696000in}}%
\pgfusepath{clip}%
\pgfsetrectcap%
\pgfsetroundjoin%
\pgfsetlinewidth{1.505625pt}%
\definecolor{currentstroke}{rgb}{1.000000,0.000000,0.000000}%
\pgfsetstrokecolor{currentstroke}%
\pgfsetdash{}{0pt}%
\pgfpathmoveto{\pgfqpoint{1.592853in}{3.035365in}}%
\pgfpathlineto{\pgfqpoint{1.698168in}{2.483690in}}%
\pgfusepath{stroke}%
\end{pgfscope}%
\begin{pgfscope}%
\pgfpathrectangle{\pgfqpoint{0.100000in}{0.212622in}}{\pgfqpoint{3.696000in}{3.696000in}}%
\pgfusepath{clip}%
\pgfsetrectcap%
\pgfsetroundjoin%
\pgfsetlinewidth{1.505625pt}%
\definecolor{currentstroke}{rgb}{1.000000,0.000000,0.000000}%
\pgfsetstrokecolor{currentstroke}%
\pgfsetdash{}{0pt}%
\pgfpathmoveto{\pgfqpoint{1.594169in}{3.038375in}}%
\pgfpathlineto{\pgfqpoint{1.698168in}{2.483690in}}%
\pgfusepath{stroke}%
\end{pgfscope}%
\begin{pgfscope}%
\pgfpathrectangle{\pgfqpoint{0.100000in}{0.212622in}}{\pgfqpoint{3.696000in}{3.696000in}}%
\pgfusepath{clip}%
\pgfsetrectcap%
\pgfsetroundjoin%
\pgfsetlinewidth{1.505625pt}%
\definecolor{currentstroke}{rgb}{1.000000,0.000000,0.000000}%
\pgfsetstrokecolor{currentstroke}%
\pgfsetdash{}{0pt}%
\pgfpathmoveto{\pgfqpoint{1.596563in}{3.043656in}}%
\pgfpathlineto{\pgfqpoint{1.698168in}{2.483690in}}%
\pgfusepath{stroke}%
\end{pgfscope}%
\begin{pgfscope}%
\pgfpathrectangle{\pgfqpoint{0.100000in}{0.212622in}}{\pgfqpoint{3.696000in}{3.696000in}}%
\pgfusepath{clip}%
\pgfsetrectcap%
\pgfsetroundjoin%
\pgfsetlinewidth{1.505625pt}%
\definecolor{currentstroke}{rgb}{1.000000,0.000000,0.000000}%
\pgfsetstrokecolor{currentstroke}%
\pgfsetdash{}{0pt}%
\pgfpathmoveto{\pgfqpoint{1.597932in}{3.046726in}}%
\pgfpathlineto{\pgfqpoint{1.698168in}{2.483690in}}%
\pgfusepath{stroke}%
\end{pgfscope}%
\begin{pgfscope}%
\pgfpathrectangle{\pgfqpoint{0.100000in}{0.212622in}}{\pgfqpoint{3.696000in}{3.696000in}}%
\pgfusepath{clip}%
\pgfsetrectcap%
\pgfsetroundjoin%
\pgfsetlinewidth{1.505625pt}%
\definecolor{currentstroke}{rgb}{1.000000,0.000000,0.000000}%
\pgfsetstrokecolor{currentstroke}%
\pgfsetdash{}{0pt}%
\pgfpathmoveto{\pgfqpoint{1.600084in}{3.051667in}}%
\pgfpathlineto{\pgfqpoint{1.698168in}{2.483690in}}%
\pgfusepath{stroke}%
\end{pgfscope}%
\begin{pgfscope}%
\pgfpathrectangle{\pgfqpoint{0.100000in}{0.212622in}}{\pgfqpoint{3.696000in}{3.696000in}}%
\pgfusepath{clip}%
\pgfsetrectcap%
\pgfsetroundjoin%
\pgfsetlinewidth{1.505625pt}%
\definecolor{currentstroke}{rgb}{1.000000,0.000000,0.000000}%
\pgfsetstrokecolor{currentstroke}%
\pgfsetdash{}{0pt}%
\pgfpathmoveto{\pgfqpoint{1.601325in}{3.054509in}}%
\pgfpathlineto{\pgfqpoint{1.698168in}{2.483690in}}%
\pgfusepath{stroke}%
\end{pgfscope}%
\begin{pgfscope}%
\pgfpathrectangle{\pgfqpoint{0.100000in}{0.212622in}}{\pgfqpoint{3.696000in}{3.696000in}}%
\pgfusepath{clip}%
\pgfsetrectcap%
\pgfsetroundjoin%
\pgfsetlinewidth{1.505625pt}%
\definecolor{currentstroke}{rgb}{1.000000,0.000000,0.000000}%
\pgfsetstrokecolor{currentstroke}%
\pgfsetdash{}{0pt}%
\pgfpathmoveto{\pgfqpoint{1.603537in}{3.059591in}}%
\pgfpathlineto{\pgfqpoint{1.698168in}{2.483690in}}%
\pgfusepath{stroke}%
\end{pgfscope}%
\begin{pgfscope}%
\pgfpathrectangle{\pgfqpoint{0.100000in}{0.212622in}}{\pgfqpoint{3.696000in}{3.696000in}}%
\pgfusepath{clip}%
\pgfsetrectcap%
\pgfsetroundjoin%
\pgfsetlinewidth{1.505625pt}%
\definecolor{currentstroke}{rgb}{1.000000,0.000000,0.000000}%
\pgfsetstrokecolor{currentstroke}%
\pgfsetdash{}{0pt}%
\pgfpathmoveto{\pgfqpoint{1.616584in}{3.087693in}}%
\pgfpathlineto{\pgfqpoint{1.698168in}{2.483690in}}%
\pgfusepath{stroke}%
\end{pgfscope}%
\begin{pgfscope}%
\pgfpathrectangle{\pgfqpoint{0.100000in}{0.212622in}}{\pgfqpoint{3.696000in}{3.696000in}}%
\pgfusepath{clip}%
\pgfsetrectcap%
\pgfsetroundjoin%
\pgfsetlinewidth{1.505625pt}%
\definecolor{currentstroke}{rgb}{1.000000,0.000000,0.000000}%
\pgfsetstrokecolor{currentstroke}%
\pgfsetdash{}{0pt}%
\pgfpathmoveto{\pgfqpoint{1.623269in}{3.102946in}}%
\pgfpathlineto{\pgfqpoint{1.698168in}{2.483690in}}%
\pgfusepath{stroke}%
\end{pgfscope}%
\begin{pgfscope}%
\pgfpathrectangle{\pgfqpoint{0.100000in}{0.212622in}}{\pgfqpoint{3.696000in}{3.696000in}}%
\pgfusepath{clip}%
\pgfsetrectcap%
\pgfsetroundjoin%
\pgfsetlinewidth{1.505625pt}%
\definecolor{currentstroke}{rgb}{1.000000,0.000000,0.000000}%
\pgfsetstrokecolor{currentstroke}%
\pgfsetdash{}{0pt}%
\pgfpathmoveto{\pgfqpoint{1.627133in}{3.111249in}}%
\pgfpathlineto{\pgfqpoint{1.698168in}{2.483690in}}%
\pgfusepath{stroke}%
\end{pgfscope}%
\begin{pgfscope}%
\pgfpathrectangle{\pgfqpoint{0.100000in}{0.212622in}}{\pgfqpoint{3.696000in}{3.696000in}}%
\pgfusepath{clip}%
\pgfsetrectcap%
\pgfsetroundjoin%
\pgfsetlinewidth{1.505625pt}%
\definecolor{currentstroke}{rgb}{1.000000,0.000000,0.000000}%
\pgfsetstrokecolor{currentstroke}%
\pgfsetdash{}{0pt}%
\pgfpathmoveto{\pgfqpoint{1.629223in}{3.115725in}}%
\pgfpathlineto{\pgfqpoint{1.698168in}{2.483690in}}%
\pgfusepath{stroke}%
\end{pgfscope}%
\begin{pgfscope}%
\pgfpathrectangle{\pgfqpoint{0.100000in}{0.212622in}}{\pgfqpoint{3.696000in}{3.696000in}}%
\pgfusepath{clip}%
\pgfsetrectcap%
\pgfsetroundjoin%
\pgfsetlinewidth{1.505625pt}%
\definecolor{currentstroke}{rgb}{1.000000,0.000000,0.000000}%
\pgfsetstrokecolor{currentstroke}%
\pgfsetdash{}{0pt}%
\pgfpathmoveto{\pgfqpoint{1.630368in}{3.118263in}}%
\pgfpathlineto{\pgfqpoint{1.698168in}{2.483690in}}%
\pgfusepath{stroke}%
\end{pgfscope}%
\begin{pgfscope}%
\pgfpathrectangle{\pgfqpoint{0.100000in}{0.212622in}}{\pgfqpoint{3.696000in}{3.696000in}}%
\pgfusepath{clip}%
\pgfsetrectcap%
\pgfsetroundjoin%
\pgfsetlinewidth{1.505625pt}%
\definecolor{currentstroke}{rgb}{1.000000,0.000000,0.000000}%
\pgfsetstrokecolor{currentstroke}%
\pgfsetdash{}{0pt}%
\pgfpathmoveto{\pgfqpoint{1.631001in}{3.119639in}}%
\pgfpathlineto{\pgfqpoint{1.698168in}{2.483690in}}%
\pgfusepath{stroke}%
\end{pgfscope}%
\begin{pgfscope}%
\pgfpathrectangle{\pgfqpoint{0.100000in}{0.212622in}}{\pgfqpoint{3.696000in}{3.696000in}}%
\pgfusepath{clip}%
\pgfsetrectcap%
\pgfsetroundjoin%
\pgfsetlinewidth{1.505625pt}%
\definecolor{currentstroke}{rgb}{1.000000,0.000000,0.000000}%
\pgfsetstrokecolor{currentstroke}%
\pgfsetdash{}{0pt}%
\pgfpathmoveto{\pgfqpoint{1.632701in}{3.123466in}}%
\pgfpathlineto{\pgfqpoint{1.698168in}{2.483690in}}%
\pgfusepath{stroke}%
\end{pgfscope}%
\begin{pgfscope}%
\pgfpathrectangle{\pgfqpoint{0.100000in}{0.212622in}}{\pgfqpoint{3.696000in}{3.696000in}}%
\pgfusepath{clip}%
\pgfsetrectcap%
\pgfsetroundjoin%
\pgfsetlinewidth{1.505625pt}%
\definecolor{currentstroke}{rgb}{1.000000,0.000000,0.000000}%
\pgfsetstrokecolor{currentstroke}%
\pgfsetdash{}{0pt}%
\pgfpathmoveto{\pgfqpoint{1.633694in}{3.125520in}}%
\pgfpathlineto{\pgfqpoint{1.698168in}{2.483690in}}%
\pgfusepath{stroke}%
\end{pgfscope}%
\begin{pgfscope}%
\pgfpathrectangle{\pgfqpoint{0.100000in}{0.212622in}}{\pgfqpoint{3.696000in}{3.696000in}}%
\pgfusepath{clip}%
\pgfsetrectcap%
\pgfsetroundjoin%
\pgfsetlinewidth{1.505625pt}%
\definecolor{currentstroke}{rgb}{1.000000,0.000000,0.000000}%
\pgfsetstrokecolor{currentstroke}%
\pgfsetdash{}{0pt}%
\pgfpathmoveto{\pgfqpoint{1.635980in}{3.130624in}}%
\pgfpathlineto{\pgfqpoint{1.698168in}{2.483690in}}%
\pgfusepath{stroke}%
\end{pgfscope}%
\begin{pgfscope}%
\pgfpathrectangle{\pgfqpoint{0.100000in}{0.212622in}}{\pgfqpoint{3.696000in}{3.696000in}}%
\pgfusepath{clip}%
\pgfsetrectcap%
\pgfsetroundjoin%
\pgfsetlinewidth{1.505625pt}%
\definecolor{currentstroke}{rgb}{1.000000,0.000000,0.000000}%
\pgfsetstrokecolor{currentstroke}%
\pgfsetdash{}{0pt}%
\pgfpathmoveto{\pgfqpoint{1.637228in}{3.133417in}}%
\pgfpathlineto{\pgfqpoint{1.698168in}{2.483690in}}%
\pgfusepath{stroke}%
\end{pgfscope}%
\begin{pgfscope}%
\pgfpathrectangle{\pgfqpoint{0.100000in}{0.212622in}}{\pgfqpoint{3.696000in}{3.696000in}}%
\pgfusepath{clip}%
\pgfsetrectcap%
\pgfsetroundjoin%
\pgfsetlinewidth{1.505625pt}%
\definecolor{currentstroke}{rgb}{1.000000,0.000000,0.000000}%
\pgfsetstrokecolor{currentstroke}%
\pgfsetdash{}{0pt}%
\pgfpathmoveto{\pgfqpoint{1.637908in}{3.134969in}}%
\pgfpathlineto{\pgfqpoint{1.698168in}{2.483690in}}%
\pgfusepath{stroke}%
\end{pgfscope}%
\begin{pgfscope}%
\pgfpathrectangle{\pgfqpoint{0.100000in}{0.212622in}}{\pgfqpoint{3.696000in}{3.696000in}}%
\pgfusepath{clip}%
\pgfsetrectcap%
\pgfsetroundjoin%
\pgfsetlinewidth{1.505625pt}%
\definecolor{currentstroke}{rgb}{1.000000,0.000000,0.000000}%
\pgfsetstrokecolor{currentstroke}%
\pgfsetdash{}{0pt}%
\pgfpathmoveto{\pgfqpoint{1.638281in}{3.135811in}}%
\pgfpathlineto{\pgfqpoint{1.698168in}{2.483690in}}%
\pgfusepath{stroke}%
\end{pgfscope}%
\begin{pgfscope}%
\pgfpathrectangle{\pgfqpoint{0.100000in}{0.212622in}}{\pgfqpoint{3.696000in}{3.696000in}}%
\pgfusepath{clip}%
\pgfsetrectcap%
\pgfsetroundjoin%
\pgfsetlinewidth{1.505625pt}%
\definecolor{currentstroke}{rgb}{1.000000,0.000000,0.000000}%
\pgfsetstrokecolor{currentstroke}%
\pgfsetdash{}{0pt}%
\pgfpathmoveto{\pgfqpoint{1.638502in}{3.136286in}}%
\pgfpathlineto{\pgfqpoint{1.698168in}{2.483690in}}%
\pgfusepath{stroke}%
\end{pgfscope}%
\begin{pgfscope}%
\pgfpathrectangle{\pgfqpoint{0.100000in}{0.212622in}}{\pgfqpoint{3.696000in}{3.696000in}}%
\pgfusepath{clip}%
\pgfsetrectcap%
\pgfsetroundjoin%
\pgfsetlinewidth{1.505625pt}%
\definecolor{currentstroke}{rgb}{1.000000,0.000000,0.000000}%
\pgfsetstrokecolor{currentstroke}%
\pgfsetdash{}{0pt}%
\pgfpathmoveto{\pgfqpoint{1.638620in}{3.136543in}}%
\pgfpathlineto{\pgfqpoint{1.698168in}{2.483690in}}%
\pgfusepath{stroke}%
\end{pgfscope}%
\begin{pgfscope}%
\pgfpathrectangle{\pgfqpoint{0.100000in}{0.212622in}}{\pgfqpoint{3.696000in}{3.696000in}}%
\pgfusepath{clip}%
\pgfsetrectcap%
\pgfsetroundjoin%
\pgfsetlinewidth{1.505625pt}%
\definecolor{currentstroke}{rgb}{1.000000,0.000000,0.000000}%
\pgfsetstrokecolor{currentstroke}%
\pgfsetdash{}{0pt}%
\pgfpathmoveto{\pgfqpoint{1.638686in}{3.136683in}}%
\pgfpathlineto{\pgfqpoint{1.698168in}{2.483690in}}%
\pgfusepath{stroke}%
\end{pgfscope}%
\begin{pgfscope}%
\pgfpathrectangle{\pgfqpoint{0.100000in}{0.212622in}}{\pgfqpoint{3.696000in}{3.696000in}}%
\pgfusepath{clip}%
\pgfsetrectcap%
\pgfsetroundjoin%
\pgfsetlinewidth{1.505625pt}%
\definecolor{currentstroke}{rgb}{1.000000,0.000000,0.000000}%
\pgfsetstrokecolor{currentstroke}%
\pgfsetdash{}{0pt}%
\pgfpathmoveto{\pgfqpoint{1.638722in}{3.136759in}}%
\pgfpathlineto{\pgfqpoint{1.698168in}{2.483690in}}%
\pgfusepath{stroke}%
\end{pgfscope}%
\begin{pgfscope}%
\pgfpathrectangle{\pgfqpoint{0.100000in}{0.212622in}}{\pgfqpoint{3.696000in}{3.696000in}}%
\pgfusepath{clip}%
\pgfsetrectcap%
\pgfsetroundjoin%
\pgfsetlinewidth{1.505625pt}%
\definecolor{currentstroke}{rgb}{1.000000,0.000000,0.000000}%
\pgfsetstrokecolor{currentstroke}%
\pgfsetdash{}{0pt}%
\pgfpathmoveto{\pgfqpoint{1.642749in}{3.145242in}}%
\pgfpathlineto{\pgfqpoint{1.698168in}{2.483690in}}%
\pgfusepath{stroke}%
\end{pgfscope}%
\begin{pgfscope}%
\pgfpathrectangle{\pgfqpoint{0.100000in}{0.212622in}}{\pgfqpoint{3.696000in}{3.696000in}}%
\pgfusepath{clip}%
\pgfsetrectcap%
\pgfsetroundjoin%
\pgfsetlinewidth{1.505625pt}%
\definecolor{currentstroke}{rgb}{1.000000,0.000000,0.000000}%
\pgfsetstrokecolor{currentstroke}%
\pgfsetdash{}{0pt}%
\pgfpathmoveto{\pgfqpoint{1.644866in}{3.149900in}}%
\pgfpathlineto{\pgfqpoint{1.698168in}{2.483690in}}%
\pgfusepath{stroke}%
\end{pgfscope}%
\begin{pgfscope}%
\pgfpathrectangle{\pgfqpoint{0.100000in}{0.212622in}}{\pgfqpoint{3.696000in}{3.696000in}}%
\pgfusepath{clip}%
\pgfsetrectcap%
\pgfsetroundjoin%
\pgfsetlinewidth{1.505625pt}%
\definecolor{currentstroke}{rgb}{1.000000,0.000000,0.000000}%
\pgfsetstrokecolor{currentstroke}%
\pgfsetdash{}{0pt}%
\pgfpathmoveto{\pgfqpoint{1.645986in}{3.152468in}}%
\pgfpathlineto{\pgfqpoint{1.698168in}{2.483690in}}%
\pgfusepath{stroke}%
\end{pgfscope}%
\begin{pgfscope}%
\pgfpathrectangle{\pgfqpoint{0.100000in}{0.212622in}}{\pgfqpoint{3.696000in}{3.696000in}}%
\pgfusepath{clip}%
\pgfsetrectcap%
\pgfsetroundjoin%
\pgfsetlinewidth{1.505625pt}%
\definecolor{currentstroke}{rgb}{1.000000,0.000000,0.000000}%
\pgfsetstrokecolor{currentstroke}%
\pgfsetdash{}{0pt}%
\pgfpathmoveto{\pgfqpoint{1.647826in}{3.157022in}}%
\pgfpathlineto{\pgfqpoint{1.698168in}{2.483690in}}%
\pgfusepath{stroke}%
\end{pgfscope}%
\begin{pgfscope}%
\pgfpathrectangle{\pgfqpoint{0.100000in}{0.212622in}}{\pgfqpoint{3.696000in}{3.696000in}}%
\pgfusepath{clip}%
\pgfsetrectcap%
\pgfsetroundjoin%
\pgfsetlinewidth{1.505625pt}%
\definecolor{currentstroke}{rgb}{1.000000,0.000000,0.000000}%
\pgfsetstrokecolor{currentstroke}%
\pgfsetdash{}{0pt}%
\pgfpathmoveto{\pgfqpoint{1.648840in}{3.159448in}}%
\pgfpathlineto{\pgfqpoint{1.698168in}{2.483690in}}%
\pgfusepath{stroke}%
\end{pgfscope}%
\begin{pgfscope}%
\pgfpathrectangle{\pgfqpoint{0.100000in}{0.212622in}}{\pgfqpoint{3.696000in}{3.696000in}}%
\pgfusepath{clip}%
\pgfsetrectcap%
\pgfsetroundjoin%
\pgfsetlinewidth{1.505625pt}%
\definecolor{currentstroke}{rgb}{1.000000,0.000000,0.000000}%
\pgfsetstrokecolor{currentstroke}%
\pgfsetdash{}{0pt}%
\pgfpathmoveto{\pgfqpoint{1.649458in}{3.160810in}}%
\pgfpathlineto{\pgfqpoint{1.698168in}{2.483690in}}%
\pgfusepath{stroke}%
\end{pgfscope}%
\begin{pgfscope}%
\pgfpathrectangle{\pgfqpoint{0.100000in}{0.212622in}}{\pgfqpoint{3.696000in}{3.696000in}}%
\pgfusepath{clip}%
\pgfsetrectcap%
\pgfsetroundjoin%
\pgfsetlinewidth{1.505625pt}%
\definecolor{currentstroke}{rgb}{1.000000,0.000000,0.000000}%
\pgfsetstrokecolor{currentstroke}%
\pgfsetdash{}{0pt}%
\pgfpathmoveto{\pgfqpoint{1.651361in}{3.165103in}}%
\pgfpathlineto{\pgfqpoint{1.698168in}{2.483690in}}%
\pgfusepath{stroke}%
\end{pgfscope}%
\begin{pgfscope}%
\pgfpathrectangle{\pgfqpoint{0.100000in}{0.212622in}}{\pgfqpoint{3.696000in}{3.696000in}}%
\pgfusepath{clip}%
\pgfsetrectcap%
\pgfsetroundjoin%
\pgfsetlinewidth{1.505625pt}%
\definecolor{currentstroke}{rgb}{1.000000,0.000000,0.000000}%
\pgfsetstrokecolor{currentstroke}%
\pgfsetdash{}{0pt}%
\pgfpathmoveto{\pgfqpoint{1.654179in}{3.171427in}}%
\pgfpathlineto{\pgfqpoint{1.698168in}{2.483690in}}%
\pgfusepath{stroke}%
\end{pgfscope}%
\begin{pgfscope}%
\pgfpathrectangle{\pgfqpoint{0.100000in}{0.212622in}}{\pgfqpoint{3.696000in}{3.696000in}}%
\pgfusepath{clip}%
\pgfsetrectcap%
\pgfsetroundjoin%
\pgfsetlinewidth{1.505625pt}%
\definecolor{currentstroke}{rgb}{1.000000,0.000000,0.000000}%
\pgfsetstrokecolor{currentstroke}%
\pgfsetdash{}{0pt}%
\pgfpathmoveto{\pgfqpoint{1.657732in}{3.179199in}}%
\pgfpathlineto{\pgfqpoint{1.698168in}{2.483690in}}%
\pgfusepath{stroke}%
\end{pgfscope}%
\begin{pgfscope}%
\pgfpathrectangle{\pgfqpoint{0.100000in}{0.212622in}}{\pgfqpoint{3.696000in}{3.696000in}}%
\pgfusepath{clip}%
\pgfsetrectcap%
\pgfsetroundjoin%
\pgfsetlinewidth{1.505625pt}%
\definecolor{currentstroke}{rgb}{1.000000,0.000000,0.000000}%
\pgfsetstrokecolor{currentstroke}%
\pgfsetdash{}{0pt}%
\pgfpathmoveto{\pgfqpoint{1.659616in}{3.183374in}}%
\pgfpathlineto{\pgfqpoint{1.698168in}{2.483690in}}%
\pgfusepath{stroke}%
\end{pgfscope}%
\begin{pgfscope}%
\pgfpathrectangle{\pgfqpoint{0.100000in}{0.212622in}}{\pgfqpoint{3.696000in}{3.696000in}}%
\pgfusepath{clip}%
\pgfsetrectcap%
\pgfsetroundjoin%
\pgfsetlinewidth{1.505625pt}%
\definecolor{currentstroke}{rgb}{1.000000,0.000000,0.000000}%
\pgfsetstrokecolor{currentstroke}%
\pgfsetdash{}{0pt}%
\pgfpathmoveto{\pgfqpoint{1.660703in}{3.185736in}}%
\pgfpathlineto{\pgfqpoint{1.698168in}{2.483690in}}%
\pgfusepath{stroke}%
\end{pgfscope}%
\begin{pgfscope}%
\pgfpathrectangle{\pgfqpoint{0.100000in}{0.212622in}}{\pgfqpoint{3.696000in}{3.696000in}}%
\pgfusepath{clip}%
\pgfsetrectcap%
\pgfsetroundjoin%
\pgfsetlinewidth{1.505625pt}%
\definecolor{currentstroke}{rgb}{1.000000,0.000000,0.000000}%
\pgfsetstrokecolor{currentstroke}%
\pgfsetdash{}{0pt}%
\pgfpathmoveto{\pgfqpoint{1.662488in}{3.189644in}}%
\pgfpathlineto{\pgfqpoint{1.698168in}{2.483690in}}%
\pgfusepath{stroke}%
\end{pgfscope}%
\begin{pgfscope}%
\pgfpathrectangle{\pgfqpoint{0.100000in}{0.212622in}}{\pgfqpoint{3.696000in}{3.696000in}}%
\pgfusepath{clip}%
\pgfsetrectcap%
\pgfsetroundjoin%
\pgfsetlinewidth{1.505625pt}%
\definecolor{currentstroke}{rgb}{1.000000,0.000000,0.000000}%
\pgfsetstrokecolor{currentstroke}%
\pgfsetdash{}{0pt}%
\pgfpathmoveto{\pgfqpoint{1.663512in}{3.191858in}}%
\pgfpathlineto{\pgfqpoint{1.698168in}{2.483690in}}%
\pgfusepath{stroke}%
\end{pgfscope}%
\begin{pgfscope}%
\pgfpathrectangle{\pgfqpoint{0.100000in}{0.212622in}}{\pgfqpoint{3.696000in}{3.696000in}}%
\pgfusepath{clip}%
\pgfsetrectcap%
\pgfsetroundjoin%
\pgfsetlinewidth{1.505625pt}%
\definecolor{currentstroke}{rgb}{1.000000,0.000000,0.000000}%
\pgfsetstrokecolor{currentstroke}%
\pgfsetdash{}{0pt}%
\pgfpathmoveto{\pgfqpoint{1.664048in}{3.193058in}}%
\pgfpathlineto{\pgfqpoint{1.698168in}{2.483690in}}%
\pgfusepath{stroke}%
\end{pgfscope}%
\begin{pgfscope}%
\pgfpathrectangle{\pgfqpoint{0.100000in}{0.212622in}}{\pgfqpoint{3.696000in}{3.696000in}}%
\pgfusepath{clip}%
\pgfsetrectcap%
\pgfsetroundjoin%
\pgfsetlinewidth{1.505625pt}%
\definecolor{currentstroke}{rgb}{1.000000,0.000000,0.000000}%
\pgfsetstrokecolor{currentstroke}%
\pgfsetdash{}{0pt}%
\pgfpathmoveto{\pgfqpoint{1.664329in}{3.193785in}}%
\pgfpathlineto{\pgfqpoint{1.698168in}{2.483690in}}%
\pgfusepath{stroke}%
\end{pgfscope}%
\begin{pgfscope}%
\pgfpathrectangle{\pgfqpoint{0.100000in}{0.212622in}}{\pgfqpoint{3.696000in}{3.696000in}}%
\pgfusepath{clip}%
\pgfsetrectcap%
\pgfsetroundjoin%
\pgfsetlinewidth{1.505625pt}%
\definecolor{currentstroke}{rgb}{1.000000,0.000000,0.000000}%
\pgfsetstrokecolor{currentstroke}%
\pgfsetdash{}{0pt}%
\pgfpathmoveto{\pgfqpoint{1.664457in}{3.194139in}}%
\pgfpathlineto{\pgfqpoint{1.698168in}{2.483690in}}%
\pgfusepath{stroke}%
\end{pgfscope}%
\begin{pgfscope}%
\pgfpathrectangle{\pgfqpoint{0.100000in}{0.212622in}}{\pgfqpoint{3.696000in}{3.696000in}}%
\pgfusepath{clip}%
\pgfsetrectcap%
\pgfsetroundjoin%
\pgfsetlinewidth{1.505625pt}%
\definecolor{currentstroke}{rgb}{1.000000,0.000000,0.000000}%
\pgfsetstrokecolor{currentstroke}%
\pgfsetdash{}{0pt}%
\pgfpathmoveto{\pgfqpoint{1.665601in}{3.196941in}}%
\pgfpathlineto{\pgfqpoint{1.698168in}{2.483690in}}%
\pgfusepath{stroke}%
\end{pgfscope}%
\begin{pgfscope}%
\pgfpathrectangle{\pgfqpoint{0.100000in}{0.212622in}}{\pgfqpoint{3.696000in}{3.696000in}}%
\pgfusepath{clip}%
\pgfsetrectcap%
\pgfsetroundjoin%
\pgfsetlinewidth{1.505625pt}%
\definecolor{currentstroke}{rgb}{1.000000,0.000000,0.000000}%
\pgfsetstrokecolor{currentstroke}%
\pgfsetdash{}{0pt}%
\pgfpathmoveto{\pgfqpoint{1.666193in}{3.198489in}}%
\pgfpathlineto{\pgfqpoint{1.698168in}{2.483690in}}%
\pgfusepath{stroke}%
\end{pgfscope}%
\begin{pgfscope}%
\pgfpathrectangle{\pgfqpoint{0.100000in}{0.212622in}}{\pgfqpoint{3.696000in}{3.696000in}}%
\pgfusepath{clip}%
\pgfsetrectcap%
\pgfsetroundjoin%
\pgfsetlinewidth{1.505625pt}%
\definecolor{currentstroke}{rgb}{1.000000,0.000000,0.000000}%
\pgfsetstrokecolor{currentstroke}%
\pgfsetdash{}{0pt}%
\pgfpathmoveto{\pgfqpoint{1.667412in}{3.201703in}}%
\pgfpathlineto{\pgfqpoint{1.698168in}{2.483690in}}%
\pgfusepath{stroke}%
\end{pgfscope}%
\begin{pgfscope}%
\pgfpathrectangle{\pgfqpoint{0.100000in}{0.212622in}}{\pgfqpoint{3.696000in}{3.696000in}}%
\pgfusepath{clip}%
\pgfsetrectcap%
\pgfsetroundjoin%
\pgfsetlinewidth{1.505625pt}%
\definecolor{currentstroke}{rgb}{1.000000,0.000000,0.000000}%
\pgfsetstrokecolor{currentstroke}%
\pgfsetdash{}{0pt}%
\pgfpathmoveto{\pgfqpoint{1.668093in}{3.203686in}}%
\pgfpathlineto{\pgfqpoint{1.698168in}{2.483690in}}%
\pgfusepath{stroke}%
\end{pgfscope}%
\begin{pgfscope}%
\pgfpathrectangle{\pgfqpoint{0.100000in}{0.212622in}}{\pgfqpoint{3.696000in}{3.696000in}}%
\pgfusepath{clip}%
\pgfsetrectcap%
\pgfsetroundjoin%
\pgfsetlinewidth{1.505625pt}%
\definecolor{currentstroke}{rgb}{1.000000,0.000000,0.000000}%
\pgfsetstrokecolor{currentstroke}%
\pgfsetdash{}{0pt}%
\pgfpathmoveto{\pgfqpoint{1.669533in}{3.207741in}}%
\pgfpathlineto{\pgfqpoint{1.698168in}{2.483690in}}%
\pgfusepath{stroke}%
\end{pgfscope}%
\begin{pgfscope}%
\pgfpathrectangle{\pgfqpoint{0.100000in}{0.212622in}}{\pgfqpoint{3.696000in}{3.696000in}}%
\pgfusepath{clip}%
\pgfsetrectcap%
\pgfsetroundjoin%
\pgfsetlinewidth{1.505625pt}%
\definecolor{currentstroke}{rgb}{1.000000,0.000000,0.000000}%
\pgfsetstrokecolor{currentstroke}%
\pgfsetdash{}{0pt}%
\pgfpathmoveto{\pgfqpoint{1.672348in}{3.214997in}}%
\pgfpathlineto{\pgfqpoint{1.698168in}{2.483690in}}%
\pgfusepath{stroke}%
\end{pgfscope}%
\begin{pgfscope}%
\pgfpathrectangle{\pgfqpoint{0.100000in}{0.212622in}}{\pgfqpoint{3.696000in}{3.696000in}}%
\pgfusepath{clip}%
\pgfsetrectcap%
\pgfsetroundjoin%
\pgfsetlinewidth{1.505625pt}%
\definecolor{currentstroke}{rgb}{1.000000,0.000000,0.000000}%
\pgfsetstrokecolor{currentstroke}%
\pgfsetdash{}{0pt}%
\pgfpathmoveto{\pgfqpoint{1.673967in}{3.218977in}}%
\pgfpathlineto{\pgfqpoint{1.698168in}{2.483690in}}%
\pgfusepath{stroke}%
\end{pgfscope}%
\begin{pgfscope}%
\pgfpathrectangle{\pgfqpoint{0.100000in}{0.212622in}}{\pgfqpoint{3.696000in}{3.696000in}}%
\pgfusepath{clip}%
\pgfsetrectcap%
\pgfsetroundjoin%
\pgfsetlinewidth{1.505625pt}%
\definecolor{currentstroke}{rgb}{1.000000,0.000000,0.000000}%
\pgfsetstrokecolor{currentstroke}%
\pgfsetdash{}{0pt}%
\pgfpathmoveto{\pgfqpoint{1.676708in}{3.225766in}}%
\pgfpathlineto{\pgfqpoint{1.698168in}{2.483690in}}%
\pgfusepath{stroke}%
\end{pgfscope}%
\begin{pgfscope}%
\pgfpathrectangle{\pgfqpoint{0.100000in}{0.212622in}}{\pgfqpoint{3.696000in}{3.696000in}}%
\pgfusepath{clip}%
\pgfsetrectcap%
\pgfsetroundjoin%
\pgfsetlinewidth{1.505625pt}%
\definecolor{currentstroke}{rgb}{1.000000,0.000000,0.000000}%
\pgfsetstrokecolor{currentstroke}%
\pgfsetdash{}{0pt}%
\pgfpathmoveto{\pgfqpoint{1.678316in}{3.229616in}}%
\pgfpathlineto{\pgfqpoint{1.698168in}{2.483690in}}%
\pgfusepath{stroke}%
\end{pgfscope}%
\begin{pgfscope}%
\pgfpathrectangle{\pgfqpoint{0.100000in}{0.212622in}}{\pgfqpoint{3.696000in}{3.696000in}}%
\pgfusepath{clip}%
\pgfsetrectcap%
\pgfsetroundjoin%
\pgfsetlinewidth{1.505625pt}%
\definecolor{currentstroke}{rgb}{1.000000,0.000000,0.000000}%
\pgfsetstrokecolor{currentstroke}%
\pgfsetdash{}{0pt}%
\pgfpathmoveto{\pgfqpoint{1.680776in}{3.235785in}}%
\pgfpathlineto{\pgfqpoint{1.698168in}{2.483690in}}%
\pgfusepath{stroke}%
\end{pgfscope}%
\begin{pgfscope}%
\pgfpathrectangle{\pgfqpoint{0.100000in}{0.212622in}}{\pgfqpoint{3.696000in}{3.696000in}}%
\pgfusepath{clip}%
\pgfsetrectcap%
\pgfsetroundjoin%
\pgfsetlinewidth{1.505625pt}%
\definecolor{currentstroke}{rgb}{1.000000,0.000000,0.000000}%
\pgfsetstrokecolor{currentstroke}%
\pgfsetdash{}{0pt}%
\pgfpathmoveto{\pgfqpoint{1.682176in}{3.239117in}}%
\pgfpathlineto{\pgfqpoint{1.698168in}{2.483690in}}%
\pgfusepath{stroke}%
\end{pgfscope}%
\begin{pgfscope}%
\pgfpathrectangle{\pgfqpoint{0.100000in}{0.212622in}}{\pgfqpoint{3.696000in}{3.696000in}}%
\pgfusepath{clip}%
\pgfsetrectcap%
\pgfsetroundjoin%
\pgfsetlinewidth{1.505625pt}%
\definecolor{currentstroke}{rgb}{1.000000,0.000000,0.000000}%
\pgfsetstrokecolor{currentstroke}%
\pgfsetdash{}{0pt}%
\pgfpathmoveto{\pgfqpoint{1.682951in}{3.240991in}}%
\pgfpathlineto{\pgfqpoint{1.698168in}{2.483690in}}%
\pgfusepath{stroke}%
\end{pgfscope}%
\begin{pgfscope}%
\pgfpathrectangle{\pgfqpoint{0.100000in}{0.212622in}}{\pgfqpoint{3.696000in}{3.696000in}}%
\pgfusepath{clip}%
\pgfsetrectcap%
\pgfsetroundjoin%
\pgfsetlinewidth{1.505625pt}%
\definecolor{currentstroke}{rgb}{1.000000,0.000000,0.000000}%
\pgfsetstrokecolor{currentstroke}%
\pgfsetdash{}{0pt}%
\pgfpathmoveto{\pgfqpoint{1.683390in}{3.242016in}}%
\pgfpathlineto{\pgfqpoint{1.698168in}{2.483690in}}%
\pgfusepath{stroke}%
\end{pgfscope}%
\begin{pgfscope}%
\pgfpathrectangle{\pgfqpoint{0.100000in}{0.212622in}}{\pgfqpoint{3.696000in}{3.696000in}}%
\pgfusepath{clip}%
\pgfsetrectcap%
\pgfsetroundjoin%
\pgfsetlinewidth{1.505625pt}%
\definecolor{currentstroke}{rgb}{1.000000,0.000000,0.000000}%
\pgfsetstrokecolor{currentstroke}%
\pgfsetdash{}{0pt}%
\pgfpathmoveto{\pgfqpoint{1.683637in}{3.242565in}}%
\pgfpathlineto{\pgfqpoint{1.698168in}{2.483690in}}%
\pgfusepath{stroke}%
\end{pgfscope}%
\begin{pgfscope}%
\pgfpathrectangle{\pgfqpoint{0.100000in}{0.212622in}}{\pgfqpoint{3.696000in}{3.696000in}}%
\pgfusepath{clip}%
\pgfsetrectcap%
\pgfsetroundjoin%
\pgfsetlinewidth{1.505625pt}%
\definecolor{currentstroke}{rgb}{1.000000,0.000000,0.000000}%
\pgfsetstrokecolor{currentstroke}%
\pgfsetdash{}{0pt}%
\pgfpathmoveto{\pgfqpoint{1.684695in}{3.244914in}}%
\pgfpathlineto{\pgfqpoint{1.698168in}{2.483690in}}%
\pgfusepath{stroke}%
\end{pgfscope}%
\begin{pgfscope}%
\pgfpathrectangle{\pgfqpoint{0.100000in}{0.212622in}}{\pgfqpoint{3.696000in}{3.696000in}}%
\pgfusepath{clip}%
\pgfsetrectcap%
\pgfsetroundjoin%
\pgfsetlinewidth{1.505625pt}%
\definecolor{currentstroke}{rgb}{1.000000,0.000000,0.000000}%
\pgfsetstrokecolor{currentstroke}%
\pgfsetdash{}{0pt}%
\pgfpathmoveto{\pgfqpoint{1.685264in}{3.246204in}}%
\pgfpathlineto{\pgfqpoint{1.698168in}{2.483690in}}%
\pgfusepath{stroke}%
\end{pgfscope}%
\begin{pgfscope}%
\pgfpathrectangle{\pgfqpoint{0.100000in}{0.212622in}}{\pgfqpoint{3.696000in}{3.696000in}}%
\pgfusepath{clip}%
\pgfsetrectcap%
\pgfsetroundjoin%
\pgfsetlinewidth{1.505625pt}%
\definecolor{currentstroke}{rgb}{1.000000,0.000000,0.000000}%
\pgfsetstrokecolor{currentstroke}%
\pgfsetdash{}{0pt}%
\pgfpathmoveto{\pgfqpoint{1.685583in}{3.246925in}}%
\pgfpathlineto{\pgfqpoint{1.698168in}{2.483690in}}%
\pgfusepath{stroke}%
\end{pgfscope}%
\begin{pgfscope}%
\pgfpathrectangle{\pgfqpoint{0.100000in}{0.212622in}}{\pgfqpoint{3.696000in}{3.696000in}}%
\pgfusepath{clip}%
\pgfsetrectcap%
\pgfsetroundjoin%
\pgfsetlinewidth{1.505625pt}%
\definecolor{currentstroke}{rgb}{1.000000,0.000000,0.000000}%
\pgfsetstrokecolor{currentstroke}%
\pgfsetdash{}{0pt}%
\pgfpathmoveto{\pgfqpoint{1.686728in}{3.249443in}}%
\pgfpathlineto{\pgfqpoint{1.698168in}{2.483690in}}%
\pgfusepath{stroke}%
\end{pgfscope}%
\begin{pgfscope}%
\pgfpathrectangle{\pgfqpoint{0.100000in}{0.212622in}}{\pgfqpoint{3.696000in}{3.696000in}}%
\pgfusepath{clip}%
\pgfsetrectcap%
\pgfsetroundjoin%
\pgfsetlinewidth{1.505625pt}%
\definecolor{currentstroke}{rgb}{1.000000,0.000000,0.000000}%
\pgfsetstrokecolor{currentstroke}%
\pgfsetdash{}{0pt}%
\pgfpathmoveto{\pgfqpoint{1.687360in}{3.250866in}}%
\pgfpathlineto{\pgfqpoint{1.698168in}{2.483690in}}%
\pgfusepath{stroke}%
\end{pgfscope}%
\begin{pgfscope}%
\pgfpathrectangle{\pgfqpoint{0.100000in}{0.212622in}}{\pgfqpoint{3.696000in}{3.696000in}}%
\pgfusepath{clip}%
\pgfsetrectcap%
\pgfsetroundjoin%
\pgfsetlinewidth{1.505625pt}%
\definecolor{currentstroke}{rgb}{1.000000,0.000000,0.000000}%
\pgfsetstrokecolor{currentstroke}%
\pgfsetdash{}{0pt}%
\pgfpathmoveto{\pgfqpoint{1.689366in}{3.255362in}}%
\pgfpathlineto{\pgfqpoint{1.698168in}{2.483690in}}%
\pgfusepath{stroke}%
\end{pgfscope}%
\begin{pgfscope}%
\pgfpathrectangle{\pgfqpoint{0.100000in}{0.212622in}}{\pgfqpoint{3.696000in}{3.696000in}}%
\pgfusepath{clip}%
\pgfsetrectcap%
\pgfsetroundjoin%
\pgfsetlinewidth{1.505625pt}%
\definecolor{currentstroke}{rgb}{1.000000,0.000000,0.000000}%
\pgfsetstrokecolor{currentstroke}%
\pgfsetdash{}{0pt}%
\pgfpathmoveto{\pgfqpoint{1.690337in}{3.257801in}}%
\pgfpathlineto{\pgfqpoint{1.698168in}{2.483690in}}%
\pgfusepath{stroke}%
\end{pgfscope}%
\begin{pgfscope}%
\pgfpathrectangle{\pgfqpoint{0.100000in}{0.212622in}}{\pgfqpoint{3.696000in}{3.696000in}}%
\pgfusepath{clip}%
\pgfsetrectcap%
\pgfsetroundjoin%
\pgfsetlinewidth{1.505625pt}%
\definecolor{currentstroke}{rgb}{1.000000,0.000000,0.000000}%
\pgfsetstrokecolor{currentstroke}%
\pgfsetdash{}{0pt}%
\pgfpathmoveto{\pgfqpoint{1.690938in}{3.259169in}}%
\pgfpathlineto{\pgfqpoint{1.698168in}{2.483690in}}%
\pgfusepath{stroke}%
\end{pgfscope}%
\begin{pgfscope}%
\pgfpathrectangle{\pgfqpoint{0.100000in}{0.212622in}}{\pgfqpoint{3.696000in}{3.696000in}}%
\pgfusepath{clip}%
\pgfsetrectcap%
\pgfsetroundjoin%
\pgfsetlinewidth{1.505625pt}%
\definecolor{currentstroke}{rgb}{1.000000,0.000000,0.000000}%
\pgfsetstrokecolor{currentstroke}%
\pgfsetdash{}{0pt}%
\pgfpathmoveto{\pgfqpoint{1.692590in}{3.263009in}}%
\pgfpathlineto{\pgfqpoint{1.698168in}{2.483690in}}%
\pgfusepath{stroke}%
\end{pgfscope}%
\begin{pgfscope}%
\pgfpathrectangle{\pgfqpoint{0.100000in}{0.212622in}}{\pgfqpoint{3.696000in}{3.696000in}}%
\pgfusepath{clip}%
\pgfsetrectcap%
\pgfsetroundjoin%
\pgfsetlinewidth{1.505625pt}%
\definecolor{currentstroke}{rgb}{1.000000,0.000000,0.000000}%
\pgfsetstrokecolor{currentstroke}%
\pgfsetdash{}{0pt}%
\pgfpathmoveto{\pgfqpoint{1.693526in}{3.265152in}}%
\pgfpathlineto{\pgfqpoint{1.698168in}{2.483690in}}%
\pgfusepath{stroke}%
\end{pgfscope}%
\begin{pgfscope}%
\pgfpathrectangle{\pgfqpoint{0.100000in}{0.212622in}}{\pgfqpoint{3.696000in}{3.696000in}}%
\pgfusepath{clip}%
\pgfsetrectcap%
\pgfsetroundjoin%
\pgfsetlinewidth{1.505625pt}%
\definecolor{currentstroke}{rgb}{1.000000,0.000000,0.000000}%
\pgfsetstrokecolor{currentstroke}%
\pgfsetdash{}{0pt}%
\pgfpathmoveto{\pgfqpoint{1.695642in}{3.270279in}}%
\pgfpathlineto{\pgfqpoint{1.698168in}{2.483690in}}%
\pgfusepath{stroke}%
\end{pgfscope}%
\begin{pgfscope}%
\pgfpathrectangle{\pgfqpoint{0.100000in}{0.212622in}}{\pgfqpoint{3.696000in}{3.696000in}}%
\pgfusepath{clip}%
\pgfsetrectcap%
\pgfsetroundjoin%
\pgfsetlinewidth{1.505625pt}%
\definecolor{currentstroke}{rgb}{1.000000,0.000000,0.000000}%
\pgfsetstrokecolor{currentstroke}%
\pgfsetdash{}{0pt}%
\pgfpathmoveto{\pgfqpoint{1.696862in}{3.273114in}}%
\pgfpathlineto{\pgfqpoint{1.698168in}{2.483690in}}%
\pgfusepath{stroke}%
\end{pgfscope}%
\begin{pgfscope}%
\pgfpathrectangle{\pgfqpoint{0.100000in}{0.212622in}}{\pgfqpoint{3.696000in}{3.696000in}}%
\pgfusepath{clip}%
\pgfsetrectcap%
\pgfsetroundjoin%
\pgfsetlinewidth{1.505625pt}%
\definecolor{currentstroke}{rgb}{1.000000,0.000000,0.000000}%
\pgfsetstrokecolor{currentstroke}%
\pgfsetdash{}{0pt}%
\pgfpathmoveto{\pgfqpoint{1.697497in}{3.274676in}}%
\pgfpathlineto{\pgfqpoint{1.698168in}{2.483690in}}%
\pgfusepath{stroke}%
\end{pgfscope}%
\begin{pgfscope}%
\pgfpathrectangle{\pgfqpoint{0.100000in}{0.212622in}}{\pgfqpoint{3.696000in}{3.696000in}}%
\pgfusepath{clip}%
\pgfsetrectcap%
\pgfsetroundjoin%
\pgfsetlinewidth{1.505625pt}%
\definecolor{currentstroke}{rgb}{1.000000,0.000000,0.000000}%
\pgfsetstrokecolor{currentstroke}%
\pgfsetdash{}{0pt}%
\pgfpathmoveto{\pgfqpoint{1.698938in}{3.278086in}}%
\pgfpathlineto{\pgfqpoint{1.698168in}{2.483690in}}%
\pgfusepath{stroke}%
\end{pgfscope}%
\begin{pgfscope}%
\pgfpathrectangle{\pgfqpoint{0.100000in}{0.212622in}}{\pgfqpoint{3.696000in}{3.696000in}}%
\pgfusepath{clip}%
\pgfsetrectcap%
\pgfsetroundjoin%
\pgfsetlinewidth{1.505625pt}%
\definecolor{currentstroke}{rgb}{1.000000,0.000000,0.000000}%
\pgfsetstrokecolor{currentstroke}%
\pgfsetdash{}{0pt}%
\pgfpathmoveto{\pgfqpoint{1.699715in}{3.279967in}}%
\pgfpathlineto{\pgfqpoint{1.698168in}{2.483690in}}%
\pgfusepath{stroke}%
\end{pgfscope}%
\begin{pgfscope}%
\pgfpathrectangle{\pgfqpoint{0.100000in}{0.212622in}}{\pgfqpoint{3.696000in}{3.696000in}}%
\pgfusepath{clip}%
\pgfsetrectcap%
\pgfsetroundjoin%
\pgfsetlinewidth{1.505625pt}%
\definecolor{currentstroke}{rgb}{1.000000,0.000000,0.000000}%
\pgfsetstrokecolor{currentstroke}%
\pgfsetdash{}{0pt}%
\pgfpathmoveto{\pgfqpoint{1.701693in}{3.284546in}}%
\pgfpathlineto{\pgfqpoint{1.698168in}{2.483690in}}%
\pgfusepath{stroke}%
\end{pgfscope}%
\begin{pgfscope}%
\pgfpathrectangle{\pgfqpoint{0.100000in}{0.212622in}}{\pgfqpoint{3.696000in}{3.696000in}}%
\pgfusepath{clip}%
\pgfsetrectcap%
\pgfsetroundjoin%
\pgfsetlinewidth{1.505625pt}%
\definecolor{currentstroke}{rgb}{1.000000,0.000000,0.000000}%
\pgfsetstrokecolor{currentstroke}%
\pgfsetdash{}{0pt}%
\pgfpathmoveto{\pgfqpoint{1.702768in}{3.287045in}}%
\pgfpathlineto{\pgfqpoint{1.698168in}{2.483690in}}%
\pgfusepath{stroke}%
\end{pgfscope}%
\begin{pgfscope}%
\pgfpathrectangle{\pgfqpoint{0.100000in}{0.212622in}}{\pgfqpoint{3.696000in}{3.696000in}}%
\pgfusepath{clip}%
\pgfsetrectcap%
\pgfsetroundjoin%
\pgfsetlinewidth{1.505625pt}%
\definecolor{currentstroke}{rgb}{1.000000,0.000000,0.000000}%
\pgfsetstrokecolor{currentstroke}%
\pgfsetdash{}{0pt}%
\pgfpathmoveto{\pgfqpoint{1.703385in}{3.288427in}}%
\pgfpathlineto{\pgfqpoint{1.698168in}{2.483690in}}%
\pgfusepath{stroke}%
\end{pgfscope}%
\begin{pgfscope}%
\pgfpathrectangle{\pgfqpoint{0.100000in}{0.212622in}}{\pgfqpoint{3.696000in}{3.696000in}}%
\pgfusepath{clip}%
\pgfsetrectcap%
\pgfsetroundjoin%
\pgfsetlinewidth{1.505625pt}%
\definecolor{currentstroke}{rgb}{1.000000,0.000000,0.000000}%
\pgfsetstrokecolor{currentstroke}%
\pgfsetdash{}{0pt}%
\pgfpathmoveto{\pgfqpoint{1.705535in}{3.292994in}}%
\pgfpathlineto{\pgfqpoint{1.698168in}{2.483690in}}%
\pgfusepath{stroke}%
\end{pgfscope}%
\begin{pgfscope}%
\pgfpathrectangle{\pgfqpoint{0.100000in}{0.212622in}}{\pgfqpoint{3.696000in}{3.696000in}}%
\pgfusepath{clip}%
\pgfsetrectcap%
\pgfsetroundjoin%
\pgfsetlinewidth{1.505625pt}%
\definecolor{currentstroke}{rgb}{1.000000,0.000000,0.000000}%
\pgfsetstrokecolor{currentstroke}%
\pgfsetdash{}{0pt}%
\pgfpathmoveto{\pgfqpoint{1.708469in}{3.299262in}}%
\pgfpathlineto{\pgfqpoint{1.698168in}{2.483690in}}%
\pgfusepath{stroke}%
\end{pgfscope}%
\begin{pgfscope}%
\pgfpathrectangle{\pgfqpoint{0.100000in}{0.212622in}}{\pgfqpoint{3.696000in}{3.696000in}}%
\pgfusepath{clip}%
\pgfsetrectcap%
\pgfsetroundjoin%
\pgfsetlinewidth{1.505625pt}%
\definecolor{currentstroke}{rgb}{1.000000,0.000000,0.000000}%
\pgfsetstrokecolor{currentstroke}%
\pgfsetdash{}{0pt}%
\pgfpathmoveto{\pgfqpoint{1.710030in}{3.302678in}}%
\pgfpathlineto{\pgfqpoint{1.698168in}{2.483690in}}%
\pgfusepath{stroke}%
\end{pgfscope}%
\begin{pgfscope}%
\pgfpathrectangle{\pgfqpoint{0.100000in}{0.212622in}}{\pgfqpoint{3.696000in}{3.696000in}}%
\pgfusepath{clip}%
\pgfsetrectcap%
\pgfsetroundjoin%
\pgfsetlinewidth{1.505625pt}%
\definecolor{currentstroke}{rgb}{1.000000,0.000000,0.000000}%
\pgfsetstrokecolor{currentstroke}%
\pgfsetdash{}{0pt}%
\pgfpathmoveto{\pgfqpoint{1.710906in}{3.304516in}}%
\pgfpathlineto{\pgfqpoint{1.698168in}{2.483690in}}%
\pgfusepath{stroke}%
\end{pgfscope}%
\begin{pgfscope}%
\pgfpathrectangle{\pgfqpoint{0.100000in}{0.212622in}}{\pgfqpoint{3.696000in}{3.696000in}}%
\pgfusepath{clip}%
\pgfsetrectcap%
\pgfsetroundjoin%
\pgfsetlinewidth{1.505625pt}%
\definecolor{currentstroke}{rgb}{1.000000,0.000000,0.000000}%
\pgfsetstrokecolor{currentstroke}%
\pgfsetdash{}{0pt}%
\pgfpathmoveto{\pgfqpoint{1.713192in}{3.309723in}}%
\pgfpathlineto{\pgfqpoint{1.698168in}{2.483690in}}%
\pgfusepath{stroke}%
\end{pgfscope}%
\begin{pgfscope}%
\pgfpathrectangle{\pgfqpoint{0.100000in}{0.212622in}}{\pgfqpoint{3.696000in}{3.696000in}}%
\pgfusepath{clip}%
\pgfsetrectcap%
\pgfsetroundjoin%
\pgfsetlinewidth{1.505625pt}%
\definecolor{currentstroke}{rgb}{1.000000,0.000000,0.000000}%
\pgfsetstrokecolor{currentstroke}%
\pgfsetdash{}{0pt}%
\pgfpathmoveto{\pgfqpoint{1.714462in}{3.312605in}}%
\pgfpathlineto{\pgfqpoint{1.698168in}{2.483690in}}%
\pgfusepath{stroke}%
\end{pgfscope}%
\begin{pgfscope}%
\pgfpathrectangle{\pgfqpoint{0.100000in}{0.212622in}}{\pgfqpoint{3.696000in}{3.696000in}}%
\pgfusepath{clip}%
\pgfsetrectcap%
\pgfsetroundjoin%
\pgfsetlinewidth{1.505625pt}%
\definecolor{currentstroke}{rgb}{1.000000,0.000000,0.000000}%
\pgfsetstrokecolor{currentstroke}%
\pgfsetdash{}{0pt}%
\pgfpathmoveto{\pgfqpoint{1.716522in}{3.317431in}}%
\pgfpathlineto{\pgfqpoint{1.698168in}{2.483690in}}%
\pgfusepath{stroke}%
\end{pgfscope}%
\begin{pgfscope}%
\pgfpathrectangle{\pgfqpoint{0.100000in}{0.212622in}}{\pgfqpoint{3.696000in}{3.696000in}}%
\pgfusepath{clip}%
\pgfsetrectcap%
\pgfsetroundjoin%
\pgfsetlinewidth{1.505625pt}%
\definecolor{currentstroke}{rgb}{1.000000,0.000000,0.000000}%
\pgfsetstrokecolor{currentstroke}%
\pgfsetdash{}{0pt}%
\pgfpathmoveto{\pgfqpoint{1.717627in}{3.320013in}}%
\pgfpathlineto{\pgfqpoint{1.698168in}{2.483690in}}%
\pgfusepath{stroke}%
\end{pgfscope}%
\begin{pgfscope}%
\pgfpathrectangle{\pgfqpoint{0.100000in}{0.212622in}}{\pgfqpoint{3.696000in}{3.696000in}}%
\pgfusepath{clip}%
\pgfsetrectcap%
\pgfsetroundjoin%
\pgfsetlinewidth{1.505625pt}%
\definecolor{currentstroke}{rgb}{1.000000,0.000000,0.000000}%
\pgfsetstrokecolor{currentstroke}%
\pgfsetdash{}{0pt}%
\pgfpathmoveto{\pgfqpoint{1.718237in}{3.321426in}}%
\pgfpathlineto{\pgfqpoint{1.698168in}{2.483690in}}%
\pgfusepath{stroke}%
\end{pgfscope}%
\begin{pgfscope}%
\pgfpathrectangle{\pgfqpoint{0.100000in}{0.212622in}}{\pgfqpoint{3.696000in}{3.696000in}}%
\pgfusepath{clip}%
\pgfsetrectcap%
\pgfsetroundjoin%
\pgfsetlinewidth{1.505625pt}%
\definecolor{currentstroke}{rgb}{1.000000,0.000000,0.000000}%
\pgfsetstrokecolor{currentstroke}%
\pgfsetdash{}{0pt}%
\pgfpathmoveto{\pgfqpoint{1.719952in}{3.325164in}}%
\pgfpathlineto{\pgfqpoint{1.698168in}{2.483690in}}%
\pgfusepath{stroke}%
\end{pgfscope}%
\begin{pgfscope}%
\pgfpathrectangle{\pgfqpoint{0.100000in}{0.212622in}}{\pgfqpoint{3.696000in}{3.696000in}}%
\pgfusepath{clip}%
\pgfsetrectcap%
\pgfsetroundjoin%
\pgfsetlinewidth{1.505625pt}%
\definecolor{currentstroke}{rgb}{1.000000,0.000000,0.000000}%
\pgfsetstrokecolor{currentstroke}%
\pgfsetdash{}{0pt}%
\pgfpathmoveto{\pgfqpoint{1.722309in}{3.330278in}}%
\pgfpathlineto{\pgfqpoint{1.698168in}{2.483690in}}%
\pgfusepath{stroke}%
\end{pgfscope}%
\begin{pgfscope}%
\pgfpathrectangle{\pgfqpoint{0.100000in}{0.212622in}}{\pgfqpoint{3.696000in}{3.696000in}}%
\pgfusepath{clip}%
\pgfsetrectcap%
\pgfsetroundjoin%
\pgfsetlinewidth{1.505625pt}%
\definecolor{currentstroke}{rgb}{1.000000,0.000000,0.000000}%
\pgfsetstrokecolor{currentstroke}%
\pgfsetdash{}{0pt}%
\pgfpathmoveto{\pgfqpoint{1.725794in}{3.337540in}}%
\pgfpathlineto{\pgfqpoint{1.698168in}{2.483690in}}%
\pgfusepath{stroke}%
\end{pgfscope}%
\begin{pgfscope}%
\pgfpathrectangle{\pgfqpoint{0.100000in}{0.212622in}}{\pgfqpoint{3.696000in}{3.696000in}}%
\pgfusepath{clip}%
\pgfsetrectcap%
\pgfsetroundjoin%
\pgfsetlinewidth{1.505625pt}%
\definecolor{currentstroke}{rgb}{1.000000,0.000000,0.000000}%
\pgfsetstrokecolor{currentstroke}%
\pgfsetdash{}{0pt}%
\pgfpathmoveto{\pgfqpoint{1.727595in}{3.341527in}}%
\pgfpathlineto{\pgfqpoint{1.698168in}{2.483690in}}%
\pgfusepath{stroke}%
\end{pgfscope}%
\begin{pgfscope}%
\pgfpathrectangle{\pgfqpoint{0.100000in}{0.212622in}}{\pgfqpoint{3.696000in}{3.696000in}}%
\pgfusepath{clip}%
\pgfsetrectcap%
\pgfsetroundjoin%
\pgfsetlinewidth{1.505625pt}%
\definecolor{currentstroke}{rgb}{1.000000,0.000000,0.000000}%
\pgfsetstrokecolor{currentstroke}%
\pgfsetdash{}{0pt}%
\pgfpathmoveto{\pgfqpoint{1.730000in}{3.347214in}}%
\pgfpathlineto{\pgfqpoint{1.698168in}{2.483690in}}%
\pgfusepath{stroke}%
\end{pgfscope}%
\begin{pgfscope}%
\pgfpathrectangle{\pgfqpoint{0.100000in}{0.212622in}}{\pgfqpoint{3.696000in}{3.696000in}}%
\pgfusepath{clip}%
\pgfsetrectcap%
\pgfsetroundjoin%
\pgfsetlinewidth{1.505625pt}%
\definecolor{currentstroke}{rgb}{1.000000,0.000000,0.000000}%
\pgfsetstrokecolor{currentstroke}%
\pgfsetdash{}{0pt}%
\pgfpathmoveto{\pgfqpoint{1.731295in}{3.350275in}}%
\pgfpathlineto{\pgfqpoint{1.698168in}{2.483690in}}%
\pgfusepath{stroke}%
\end{pgfscope}%
\begin{pgfscope}%
\pgfpathrectangle{\pgfqpoint{0.100000in}{0.212622in}}{\pgfqpoint{3.696000in}{3.696000in}}%
\pgfusepath{clip}%
\pgfsetrectcap%
\pgfsetroundjoin%
\pgfsetlinewidth{1.505625pt}%
\definecolor{currentstroke}{rgb}{1.000000,0.000000,0.000000}%
\pgfsetstrokecolor{currentstroke}%
\pgfsetdash{}{0pt}%
\pgfpathmoveto{\pgfqpoint{1.733152in}{3.354874in}}%
\pgfpathlineto{\pgfqpoint{1.698168in}{2.483690in}}%
\pgfusepath{stroke}%
\end{pgfscope}%
\begin{pgfscope}%
\pgfpathrectangle{\pgfqpoint{0.100000in}{0.212622in}}{\pgfqpoint{3.696000in}{3.696000in}}%
\pgfusepath{clip}%
\pgfsetrectcap%
\pgfsetroundjoin%
\pgfsetlinewidth{1.505625pt}%
\definecolor{currentstroke}{rgb}{1.000000,0.000000,0.000000}%
\pgfsetstrokecolor{currentstroke}%
\pgfsetdash{}{0pt}%
\pgfpathmoveto{\pgfqpoint{1.734189in}{3.357373in}}%
\pgfpathlineto{\pgfqpoint{1.698168in}{2.483690in}}%
\pgfusepath{stroke}%
\end{pgfscope}%
\begin{pgfscope}%
\pgfpathrectangle{\pgfqpoint{0.100000in}{0.212622in}}{\pgfqpoint{3.696000in}{3.696000in}}%
\pgfusepath{clip}%
\pgfsetrectcap%
\pgfsetroundjoin%
\pgfsetlinewidth{1.505625pt}%
\definecolor{currentstroke}{rgb}{1.000000,0.000000,0.000000}%
\pgfsetstrokecolor{currentstroke}%
\pgfsetdash{}{0pt}%
\pgfpathmoveto{\pgfqpoint{1.734761in}{3.358746in}}%
\pgfpathlineto{\pgfqpoint{1.698168in}{2.483690in}}%
\pgfusepath{stroke}%
\end{pgfscope}%
\begin{pgfscope}%
\pgfpathrectangle{\pgfqpoint{0.100000in}{0.212622in}}{\pgfqpoint{3.696000in}{3.696000in}}%
\pgfusepath{clip}%
\pgfsetrectcap%
\pgfsetroundjoin%
\pgfsetlinewidth{1.505625pt}%
\definecolor{currentstroke}{rgb}{1.000000,0.000000,0.000000}%
\pgfsetstrokecolor{currentstroke}%
\pgfsetdash{}{0pt}%
\pgfpathmoveto{\pgfqpoint{1.735075in}{3.359507in}}%
\pgfpathlineto{\pgfqpoint{1.698168in}{2.483690in}}%
\pgfusepath{stroke}%
\end{pgfscope}%
\begin{pgfscope}%
\pgfpathrectangle{\pgfqpoint{0.100000in}{0.212622in}}{\pgfqpoint{3.696000in}{3.696000in}}%
\pgfusepath{clip}%
\pgfsetrectcap%
\pgfsetroundjoin%
\pgfsetlinewidth{1.505625pt}%
\definecolor{currentstroke}{rgb}{1.000000,0.000000,0.000000}%
\pgfsetstrokecolor{currentstroke}%
\pgfsetdash{}{0pt}%
\pgfpathmoveto{\pgfqpoint{1.735252in}{3.359927in}}%
\pgfpathlineto{\pgfqpoint{1.698168in}{2.483690in}}%
\pgfusepath{stroke}%
\end{pgfscope}%
\begin{pgfscope}%
\pgfpathrectangle{\pgfqpoint{0.100000in}{0.212622in}}{\pgfqpoint{3.696000in}{3.696000in}}%
\pgfusepath{clip}%
\pgfsetrectcap%
\pgfsetroundjoin%
\pgfsetlinewidth{1.505625pt}%
\definecolor{currentstroke}{rgb}{1.000000,0.000000,0.000000}%
\pgfsetstrokecolor{currentstroke}%
\pgfsetdash{}{0pt}%
\pgfpathmoveto{\pgfqpoint{1.735348in}{3.360156in}}%
\pgfpathlineto{\pgfqpoint{1.698168in}{2.483690in}}%
\pgfusepath{stroke}%
\end{pgfscope}%
\begin{pgfscope}%
\pgfpathrectangle{\pgfqpoint{0.100000in}{0.212622in}}{\pgfqpoint{3.696000in}{3.696000in}}%
\pgfusepath{clip}%
\pgfsetrectcap%
\pgfsetroundjoin%
\pgfsetlinewidth{1.505625pt}%
\definecolor{currentstroke}{rgb}{1.000000,0.000000,0.000000}%
\pgfsetstrokecolor{currentstroke}%
\pgfsetdash{}{0pt}%
\pgfpathmoveto{\pgfqpoint{1.735399in}{3.360282in}}%
\pgfpathlineto{\pgfqpoint{1.698168in}{2.483690in}}%
\pgfusepath{stroke}%
\end{pgfscope}%
\begin{pgfscope}%
\pgfpathrectangle{\pgfqpoint{0.100000in}{0.212622in}}{\pgfqpoint{3.696000in}{3.696000in}}%
\pgfusepath{clip}%
\pgfsetrectcap%
\pgfsetroundjoin%
\pgfsetlinewidth{1.505625pt}%
\definecolor{currentstroke}{rgb}{1.000000,0.000000,0.000000}%
\pgfsetstrokecolor{currentstroke}%
\pgfsetdash{}{0pt}%
\pgfpathmoveto{\pgfqpoint{1.735429in}{3.360352in}}%
\pgfpathlineto{\pgfqpoint{1.698168in}{2.483690in}}%
\pgfusepath{stroke}%
\end{pgfscope}%
\begin{pgfscope}%
\pgfpathrectangle{\pgfqpoint{0.100000in}{0.212622in}}{\pgfqpoint{3.696000in}{3.696000in}}%
\pgfusepath{clip}%
\pgfsetrectcap%
\pgfsetroundjoin%
\pgfsetlinewidth{1.505625pt}%
\definecolor{currentstroke}{rgb}{1.000000,0.000000,0.000000}%
\pgfsetstrokecolor{currentstroke}%
\pgfsetdash{}{0pt}%
\pgfpathmoveto{\pgfqpoint{1.735445in}{3.360390in}}%
\pgfpathlineto{\pgfqpoint{1.698168in}{2.483690in}}%
\pgfusepath{stroke}%
\end{pgfscope}%
\begin{pgfscope}%
\pgfpathrectangle{\pgfqpoint{0.100000in}{0.212622in}}{\pgfqpoint{3.696000in}{3.696000in}}%
\pgfusepath{clip}%
\pgfsetrectcap%
\pgfsetroundjoin%
\pgfsetlinewidth{1.505625pt}%
\definecolor{currentstroke}{rgb}{1.000000,0.000000,0.000000}%
\pgfsetstrokecolor{currentstroke}%
\pgfsetdash{}{0pt}%
\pgfpathmoveto{\pgfqpoint{1.735454in}{3.360411in}}%
\pgfpathlineto{\pgfqpoint{1.698168in}{2.483690in}}%
\pgfusepath{stroke}%
\end{pgfscope}%
\begin{pgfscope}%
\pgfpathrectangle{\pgfqpoint{0.100000in}{0.212622in}}{\pgfqpoint{3.696000in}{3.696000in}}%
\pgfusepath{clip}%
\pgfsetrectcap%
\pgfsetroundjoin%
\pgfsetlinewidth{1.505625pt}%
\definecolor{currentstroke}{rgb}{1.000000,0.000000,0.000000}%
\pgfsetstrokecolor{currentstroke}%
\pgfsetdash{}{0pt}%
\pgfpathmoveto{\pgfqpoint{1.736219in}{3.362266in}}%
\pgfpathlineto{\pgfqpoint{1.698168in}{2.483690in}}%
\pgfusepath{stroke}%
\end{pgfscope}%
\begin{pgfscope}%
\pgfpathrectangle{\pgfqpoint{0.100000in}{0.212622in}}{\pgfqpoint{3.696000in}{3.696000in}}%
\pgfusepath{clip}%
\pgfsetrectcap%
\pgfsetroundjoin%
\pgfsetlinewidth{1.505625pt}%
\definecolor{currentstroke}{rgb}{1.000000,0.000000,0.000000}%
\pgfsetstrokecolor{currentstroke}%
\pgfsetdash{}{0pt}%
\pgfpathmoveto{\pgfqpoint{1.736631in}{3.363295in}}%
\pgfpathlineto{\pgfqpoint{1.698168in}{2.483690in}}%
\pgfusepath{stroke}%
\end{pgfscope}%
\begin{pgfscope}%
\pgfpathrectangle{\pgfqpoint{0.100000in}{0.212622in}}{\pgfqpoint{3.696000in}{3.696000in}}%
\pgfusepath{clip}%
\pgfsetrectcap%
\pgfsetroundjoin%
\pgfsetlinewidth{1.505625pt}%
\definecolor{currentstroke}{rgb}{1.000000,0.000000,0.000000}%
\pgfsetstrokecolor{currentstroke}%
\pgfsetdash{}{0pt}%
\pgfpathmoveto{\pgfqpoint{1.737673in}{3.365930in}}%
\pgfpathlineto{\pgfqpoint{1.698168in}{2.483690in}}%
\pgfusepath{stroke}%
\end{pgfscope}%
\begin{pgfscope}%
\pgfpathrectangle{\pgfqpoint{0.100000in}{0.212622in}}{\pgfqpoint{3.696000in}{3.696000in}}%
\pgfusepath{clip}%
\pgfsetrectcap%
\pgfsetroundjoin%
\pgfsetlinewidth{1.505625pt}%
\definecolor{currentstroke}{rgb}{1.000000,0.000000,0.000000}%
\pgfsetstrokecolor{currentstroke}%
\pgfsetdash{}{0pt}%
\pgfpathmoveto{\pgfqpoint{1.738268in}{3.367387in}}%
\pgfpathlineto{\pgfqpoint{1.698168in}{2.483690in}}%
\pgfusepath{stroke}%
\end{pgfscope}%
\begin{pgfscope}%
\pgfpathrectangle{\pgfqpoint{0.100000in}{0.212622in}}{\pgfqpoint{3.696000in}{3.696000in}}%
\pgfusepath{clip}%
\pgfsetrectcap%
\pgfsetroundjoin%
\pgfsetlinewidth{1.505625pt}%
\definecolor{currentstroke}{rgb}{1.000000,0.000000,0.000000}%
\pgfsetstrokecolor{currentstroke}%
\pgfsetdash{}{0pt}%
\pgfpathmoveto{\pgfqpoint{1.739447in}{3.370344in}}%
\pgfpathlineto{\pgfqpoint{1.698168in}{2.483690in}}%
\pgfusepath{stroke}%
\end{pgfscope}%
\begin{pgfscope}%
\pgfpathrectangle{\pgfqpoint{0.100000in}{0.212622in}}{\pgfqpoint{3.696000in}{3.696000in}}%
\pgfusepath{clip}%
\pgfsetrectcap%
\pgfsetroundjoin%
\pgfsetlinewidth{1.505625pt}%
\definecolor{currentstroke}{rgb}{1.000000,0.000000,0.000000}%
\pgfsetstrokecolor{currentstroke}%
\pgfsetdash{}{0pt}%
\pgfpathmoveto{\pgfqpoint{1.740145in}{3.371994in}}%
\pgfpathlineto{\pgfqpoint{1.698168in}{2.483690in}}%
\pgfusepath{stroke}%
\end{pgfscope}%
\begin{pgfscope}%
\pgfpathrectangle{\pgfqpoint{0.100000in}{0.212622in}}{\pgfqpoint{3.696000in}{3.696000in}}%
\pgfusepath{clip}%
\pgfsetrectcap%
\pgfsetroundjoin%
\pgfsetlinewidth{1.505625pt}%
\definecolor{currentstroke}{rgb}{1.000000,0.000000,0.000000}%
\pgfsetstrokecolor{currentstroke}%
\pgfsetdash{}{0pt}%
\pgfpathmoveto{\pgfqpoint{1.742331in}{3.376828in}}%
\pgfpathlineto{\pgfqpoint{1.698168in}{2.483690in}}%
\pgfusepath{stroke}%
\end{pgfscope}%
\begin{pgfscope}%
\pgfpathrectangle{\pgfqpoint{0.100000in}{0.212622in}}{\pgfqpoint{3.696000in}{3.696000in}}%
\pgfusepath{clip}%
\pgfsetrectcap%
\pgfsetroundjoin%
\pgfsetlinewidth{1.505625pt}%
\definecolor{currentstroke}{rgb}{1.000000,0.000000,0.000000}%
\pgfsetstrokecolor{currentstroke}%
\pgfsetdash{}{0pt}%
\pgfpathmoveto{\pgfqpoint{1.747183in}{3.386005in}}%
\pgfpathlineto{\pgfqpoint{1.698168in}{2.483690in}}%
\pgfusepath{stroke}%
\end{pgfscope}%
\begin{pgfscope}%
\pgfpathrectangle{\pgfqpoint{0.100000in}{0.212622in}}{\pgfqpoint{3.696000in}{3.696000in}}%
\pgfusepath{clip}%
\pgfsetrectcap%
\pgfsetroundjoin%
\pgfsetlinewidth{1.505625pt}%
\definecolor{currentstroke}{rgb}{1.000000,0.000000,0.000000}%
\pgfsetstrokecolor{currentstroke}%
\pgfsetdash{}{0pt}%
\pgfpathmoveto{\pgfqpoint{1.750214in}{3.390879in}}%
\pgfpathlineto{\pgfqpoint{1.698168in}{2.483690in}}%
\pgfusepath{stroke}%
\end{pgfscope}%
\begin{pgfscope}%
\pgfpathrectangle{\pgfqpoint{0.100000in}{0.212622in}}{\pgfqpoint{3.696000in}{3.696000in}}%
\pgfusepath{clip}%
\pgfsetrectcap%
\pgfsetroundjoin%
\pgfsetlinewidth{1.505625pt}%
\definecolor{currentstroke}{rgb}{1.000000,0.000000,0.000000}%
\pgfsetstrokecolor{currentstroke}%
\pgfsetdash{}{0pt}%
\pgfpathmoveto{\pgfqpoint{1.756666in}{3.398224in}}%
\pgfpathlineto{\pgfqpoint{1.698168in}{2.483690in}}%
\pgfusepath{stroke}%
\end{pgfscope}%
\begin{pgfscope}%
\pgfpathrectangle{\pgfqpoint{0.100000in}{0.212622in}}{\pgfqpoint{3.696000in}{3.696000in}}%
\pgfusepath{clip}%
\pgfsetrectcap%
\pgfsetroundjoin%
\pgfsetlinewidth{1.505625pt}%
\definecolor{currentstroke}{rgb}{1.000000,0.000000,0.000000}%
\pgfsetstrokecolor{currentstroke}%
\pgfsetdash{}{0pt}%
\pgfpathmoveto{\pgfqpoint{1.770205in}{3.401705in}}%
\pgfpathlineto{\pgfqpoint{1.698168in}{2.483690in}}%
\pgfusepath{stroke}%
\end{pgfscope}%
\begin{pgfscope}%
\pgfpathrectangle{\pgfqpoint{0.100000in}{0.212622in}}{\pgfqpoint{3.696000in}{3.696000in}}%
\pgfusepath{clip}%
\pgfsetrectcap%
\pgfsetroundjoin%
\pgfsetlinewidth{1.505625pt}%
\definecolor{currentstroke}{rgb}{1.000000,0.000000,0.000000}%
\pgfsetstrokecolor{currentstroke}%
\pgfsetdash{}{0pt}%
\pgfpathmoveto{\pgfqpoint{1.778209in}{3.399855in}}%
\pgfpathlineto{\pgfqpoint{1.698168in}{2.483690in}}%
\pgfusepath{stroke}%
\end{pgfscope}%
\begin{pgfscope}%
\pgfpathrectangle{\pgfqpoint{0.100000in}{0.212622in}}{\pgfqpoint{3.696000in}{3.696000in}}%
\pgfusepath{clip}%
\pgfsetrectcap%
\pgfsetroundjoin%
\pgfsetlinewidth{1.505625pt}%
\definecolor{currentstroke}{rgb}{1.000000,0.000000,0.000000}%
\pgfsetstrokecolor{currentstroke}%
\pgfsetdash{}{0pt}%
\pgfpathmoveto{\pgfqpoint{1.782299in}{3.398052in}}%
\pgfpathlineto{\pgfqpoint{1.698168in}{2.483690in}}%
\pgfusepath{stroke}%
\end{pgfscope}%
\begin{pgfscope}%
\pgfpathrectangle{\pgfqpoint{0.100000in}{0.212622in}}{\pgfqpoint{3.696000in}{3.696000in}}%
\pgfusepath{clip}%
\pgfsetrectcap%
\pgfsetroundjoin%
\pgfsetlinewidth{1.505625pt}%
\definecolor{currentstroke}{rgb}{1.000000,0.000000,0.000000}%
\pgfsetstrokecolor{currentstroke}%
\pgfsetdash{}{0pt}%
\pgfpathmoveto{\pgfqpoint{1.784318in}{3.396749in}}%
\pgfpathlineto{\pgfqpoint{1.698168in}{2.483690in}}%
\pgfusepath{stroke}%
\end{pgfscope}%
\begin{pgfscope}%
\pgfpathrectangle{\pgfqpoint{0.100000in}{0.212622in}}{\pgfqpoint{3.696000in}{3.696000in}}%
\pgfusepath{clip}%
\pgfsetrectcap%
\pgfsetroundjoin%
\pgfsetlinewidth{1.505625pt}%
\definecolor{currentstroke}{rgb}{1.000000,0.000000,0.000000}%
\pgfsetstrokecolor{currentstroke}%
\pgfsetdash{}{0pt}%
\pgfpathmoveto{\pgfqpoint{1.785424in}{3.396082in}}%
\pgfpathlineto{\pgfqpoint{1.698168in}{2.483690in}}%
\pgfusepath{stroke}%
\end{pgfscope}%
\begin{pgfscope}%
\pgfpathrectangle{\pgfqpoint{0.100000in}{0.212622in}}{\pgfqpoint{3.696000in}{3.696000in}}%
\pgfusepath{clip}%
\pgfsetrectcap%
\pgfsetroundjoin%
\pgfsetlinewidth{1.505625pt}%
\definecolor{currentstroke}{rgb}{1.000000,0.000000,0.000000}%
\pgfsetstrokecolor{currentstroke}%
\pgfsetdash{}{0pt}%
\pgfpathmoveto{\pgfqpoint{1.785941in}{3.395597in}}%
\pgfpathlineto{\pgfqpoint{1.698168in}{2.483690in}}%
\pgfusepath{stroke}%
\end{pgfscope}%
\begin{pgfscope}%
\pgfpathrectangle{\pgfqpoint{0.100000in}{0.212622in}}{\pgfqpoint{3.696000in}{3.696000in}}%
\pgfusepath{clip}%
\pgfsetrectcap%
\pgfsetroundjoin%
\pgfsetlinewidth{1.505625pt}%
\definecolor{currentstroke}{rgb}{1.000000,0.000000,0.000000}%
\pgfsetstrokecolor{currentstroke}%
\pgfsetdash{}{0pt}%
\pgfpathmoveto{\pgfqpoint{1.789495in}{3.392342in}}%
\pgfpathlineto{\pgfqpoint{1.698168in}{2.483690in}}%
\pgfusepath{stroke}%
\end{pgfscope}%
\begin{pgfscope}%
\pgfpathrectangle{\pgfqpoint{0.100000in}{0.212622in}}{\pgfqpoint{3.696000in}{3.696000in}}%
\pgfusepath{clip}%
\pgfsetrectcap%
\pgfsetroundjoin%
\pgfsetlinewidth{1.505625pt}%
\definecolor{currentstroke}{rgb}{1.000000,0.000000,0.000000}%
\pgfsetstrokecolor{currentstroke}%
\pgfsetdash{}{0pt}%
\pgfpathmoveto{\pgfqpoint{1.795191in}{3.385602in}}%
\pgfpathlineto{\pgfqpoint{1.698168in}{2.483690in}}%
\pgfusepath{stroke}%
\end{pgfscope}%
\begin{pgfscope}%
\pgfpathrectangle{\pgfqpoint{0.100000in}{0.212622in}}{\pgfqpoint{3.696000in}{3.696000in}}%
\pgfusepath{clip}%
\pgfsetrectcap%
\pgfsetroundjoin%
\pgfsetlinewidth{1.505625pt}%
\definecolor{currentstroke}{rgb}{1.000000,0.000000,0.000000}%
\pgfsetstrokecolor{currentstroke}%
\pgfsetdash{}{0pt}%
\pgfpathmoveto{\pgfqpoint{1.797895in}{3.381629in}}%
\pgfpathlineto{\pgfqpoint{1.698168in}{2.483690in}}%
\pgfusepath{stroke}%
\end{pgfscope}%
\begin{pgfscope}%
\pgfpathrectangle{\pgfqpoint{0.100000in}{0.212622in}}{\pgfqpoint{3.696000in}{3.696000in}}%
\pgfusepath{clip}%
\pgfsetrectcap%
\pgfsetroundjoin%
\pgfsetlinewidth{1.505625pt}%
\definecolor{currentstroke}{rgb}{1.000000,0.000000,0.000000}%
\pgfsetstrokecolor{currentstroke}%
\pgfsetdash{}{0pt}%
\pgfpathmoveto{\pgfqpoint{1.799254in}{3.379451in}}%
\pgfpathlineto{\pgfqpoint{1.698168in}{2.483690in}}%
\pgfusepath{stroke}%
\end{pgfscope}%
\begin{pgfscope}%
\pgfpathrectangle{\pgfqpoint{0.100000in}{0.212622in}}{\pgfqpoint{3.696000in}{3.696000in}}%
\pgfusepath{clip}%
\pgfsetrectcap%
\pgfsetroundjoin%
\pgfsetlinewidth{1.505625pt}%
\definecolor{currentstroke}{rgb}{1.000000,0.000000,0.000000}%
\pgfsetstrokecolor{currentstroke}%
\pgfsetdash{}{0pt}%
\pgfpathmoveto{\pgfqpoint{1.802005in}{3.374583in}}%
\pgfpathlineto{\pgfqpoint{1.698168in}{2.483690in}}%
\pgfusepath{stroke}%
\end{pgfscope}%
\begin{pgfscope}%
\pgfpathrectangle{\pgfqpoint{0.100000in}{0.212622in}}{\pgfqpoint{3.696000in}{3.696000in}}%
\pgfusepath{clip}%
\pgfsetrectcap%
\pgfsetroundjoin%
\pgfsetlinewidth{1.505625pt}%
\definecolor{currentstroke}{rgb}{1.000000,0.000000,0.000000}%
\pgfsetstrokecolor{currentstroke}%
\pgfsetdash{}{0pt}%
\pgfpathmoveto{\pgfqpoint{1.806218in}{3.366525in}}%
\pgfpathlineto{\pgfqpoint{1.698168in}{2.483690in}}%
\pgfusepath{stroke}%
\end{pgfscope}%
\begin{pgfscope}%
\pgfpathrectangle{\pgfqpoint{0.100000in}{0.212622in}}{\pgfqpoint{3.696000in}{3.696000in}}%
\pgfusepath{clip}%
\pgfsetrectcap%
\pgfsetroundjoin%
\pgfsetlinewidth{1.505625pt}%
\definecolor{currentstroke}{rgb}{1.000000,0.000000,0.000000}%
\pgfsetstrokecolor{currentstroke}%
\pgfsetdash{}{0pt}%
\pgfpathmoveto{\pgfqpoint{1.808443in}{3.362112in}}%
\pgfpathlineto{\pgfqpoint{1.698168in}{2.483690in}}%
\pgfusepath{stroke}%
\end{pgfscope}%
\begin{pgfscope}%
\pgfpathrectangle{\pgfqpoint{0.100000in}{0.212622in}}{\pgfqpoint{3.696000in}{3.696000in}}%
\pgfusepath{clip}%
\pgfsetrectcap%
\pgfsetroundjoin%
\pgfsetlinewidth{1.505625pt}%
\definecolor{currentstroke}{rgb}{1.000000,0.000000,0.000000}%
\pgfsetstrokecolor{currentstroke}%
\pgfsetdash{}{0pt}%
\pgfpathmoveto{\pgfqpoint{1.809510in}{3.359582in}}%
\pgfpathlineto{\pgfqpoint{1.698168in}{2.483690in}}%
\pgfusepath{stroke}%
\end{pgfscope}%
\begin{pgfscope}%
\pgfpathrectangle{\pgfqpoint{0.100000in}{0.212622in}}{\pgfqpoint{3.696000in}{3.696000in}}%
\pgfusepath{clip}%
\pgfsetrectcap%
\pgfsetroundjoin%
\pgfsetlinewidth{1.505625pt}%
\definecolor{currentstroke}{rgb}{1.000000,0.000000,0.000000}%
\pgfsetstrokecolor{currentstroke}%
\pgfsetdash{}{0pt}%
\pgfpathmoveto{\pgfqpoint{1.810132in}{3.358254in}}%
\pgfpathlineto{\pgfqpoint{1.698168in}{2.483690in}}%
\pgfusepath{stroke}%
\end{pgfscope}%
\begin{pgfscope}%
\pgfpathrectangle{\pgfqpoint{0.100000in}{0.212622in}}{\pgfqpoint{3.696000in}{3.696000in}}%
\pgfusepath{clip}%
\pgfsetrectcap%
\pgfsetroundjoin%
\pgfsetlinewidth{1.505625pt}%
\definecolor{currentstroke}{rgb}{1.000000,0.000000,0.000000}%
\pgfsetstrokecolor{currentstroke}%
\pgfsetdash{}{0pt}%
\pgfpathmoveto{\pgfqpoint{1.811630in}{3.354392in}}%
\pgfpathlineto{\pgfqpoint{1.698168in}{2.483690in}}%
\pgfusepath{stroke}%
\end{pgfscope}%
\begin{pgfscope}%
\pgfpathrectangle{\pgfqpoint{0.100000in}{0.212622in}}{\pgfqpoint{3.696000in}{3.696000in}}%
\pgfusepath{clip}%
\pgfsetrectcap%
\pgfsetroundjoin%
\pgfsetlinewidth{1.505625pt}%
\definecolor{currentstroke}{rgb}{1.000000,0.000000,0.000000}%
\pgfsetstrokecolor{currentstroke}%
\pgfsetdash{}{0pt}%
\pgfpathmoveto{\pgfqpoint{1.814010in}{3.348477in}}%
\pgfpathlineto{\pgfqpoint{1.698168in}{2.483690in}}%
\pgfusepath{stroke}%
\end{pgfscope}%
\begin{pgfscope}%
\pgfpathrectangle{\pgfqpoint{0.100000in}{0.212622in}}{\pgfqpoint{3.696000in}{3.696000in}}%
\pgfusepath{clip}%
\pgfsetrectcap%
\pgfsetroundjoin%
\pgfsetlinewidth{1.505625pt}%
\definecolor{currentstroke}{rgb}{1.000000,0.000000,0.000000}%
\pgfsetstrokecolor{currentstroke}%
\pgfsetdash{}{0pt}%
\pgfpathmoveto{\pgfqpoint{1.817590in}{3.338551in}}%
\pgfpathlineto{\pgfqpoint{1.698168in}{2.483690in}}%
\pgfusepath{stroke}%
\end{pgfscope}%
\begin{pgfscope}%
\pgfpathrectangle{\pgfqpoint{0.100000in}{0.212622in}}{\pgfqpoint{3.696000in}{3.696000in}}%
\pgfusepath{clip}%
\pgfsetrectcap%
\pgfsetroundjoin%
\pgfsetlinewidth{1.505625pt}%
\definecolor{currentstroke}{rgb}{1.000000,0.000000,0.000000}%
\pgfsetstrokecolor{currentstroke}%
\pgfsetdash{}{0pt}%
\pgfpathmoveto{\pgfqpoint{1.821643in}{3.326864in}}%
\pgfpathlineto{\pgfqpoint{1.698168in}{2.483690in}}%
\pgfusepath{stroke}%
\end{pgfscope}%
\begin{pgfscope}%
\pgfpathrectangle{\pgfqpoint{0.100000in}{0.212622in}}{\pgfqpoint{3.696000in}{3.696000in}}%
\pgfusepath{clip}%
\pgfsetrectcap%
\pgfsetroundjoin%
\pgfsetlinewidth{1.505625pt}%
\definecolor{currentstroke}{rgb}{1.000000,0.000000,0.000000}%
\pgfsetstrokecolor{currentstroke}%
\pgfsetdash{}{0pt}%
\pgfpathmoveto{\pgfqpoint{1.824014in}{3.320533in}}%
\pgfpathlineto{\pgfqpoint{1.698168in}{2.483690in}}%
\pgfusepath{stroke}%
\end{pgfscope}%
\begin{pgfscope}%
\pgfpathrectangle{\pgfqpoint{0.100000in}{0.212622in}}{\pgfqpoint{3.696000in}{3.696000in}}%
\pgfusepath{clip}%
\pgfsetrectcap%
\pgfsetroundjoin%
\pgfsetlinewidth{1.505625pt}%
\definecolor{currentstroke}{rgb}{1.000000,0.000000,0.000000}%
\pgfsetstrokecolor{currentstroke}%
\pgfsetdash{}{0pt}%
\pgfpathmoveto{\pgfqpoint{1.825139in}{3.316906in}}%
\pgfpathlineto{\pgfqpoint{1.698168in}{2.483690in}}%
\pgfusepath{stroke}%
\end{pgfscope}%
\begin{pgfscope}%
\pgfpathrectangle{\pgfqpoint{0.100000in}{0.212622in}}{\pgfqpoint{3.696000in}{3.696000in}}%
\pgfusepath{clip}%
\pgfsetrectcap%
\pgfsetroundjoin%
\pgfsetlinewidth{1.505625pt}%
\definecolor{currentstroke}{rgb}{1.000000,0.000000,0.000000}%
\pgfsetstrokecolor{currentstroke}%
\pgfsetdash{}{0pt}%
\pgfpathmoveto{\pgfqpoint{1.826839in}{3.311777in}}%
\pgfpathlineto{\pgfqpoint{1.698168in}{2.483690in}}%
\pgfusepath{stroke}%
\end{pgfscope}%
\begin{pgfscope}%
\pgfpathrectangle{\pgfqpoint{0.100000in}{0.212622in}}{\pgfqpoint{3.696000in}{3.696000in}}%
\pgfusepath{clip}%
\pgfsetrectcap%
\pgfsetroundjoin%
\pgfsetlinewidth{1.505625pt}%
\definecolor{currentstroke}{rgb}{1.000000,0.000000,0.000000}%
\pgfsetstrokecolor{currentstroke}%
\pgfsetdash{}{0pt}%
\pgfpathmoveto{\pgfqpoint{1.827640in}{3.308836in}}%
\pgfpathlineto{\pgfqpoint{1.698168in}{2.483690in}}%
\pgfusepath{stroke}%
\end{pgfscope}%
\begin{pgfscope}%
\pgfpathrectangle{\pgfqpoint{0.100000in}{0.212622in}}{\pgfqpoint{3.696000in}{3.696000in}}%
\pgfusepath{clip}%
\pgfsetrectcap%
\pgfsetroundjoin%
\pgfsetlinewidth{1.505625pt}%
\definecolor{currentstroke}{rgb}{1.000000,0.000000,0.000000}%
\pgfsetstrokecolor{currentstroke}%
\pgfsetdash{}{0pt}%
\pgfpathmoveto{\pgfqpoint{1.829041in}{3.304262in}}%
\pgfpathlineto{\pgfqpoint{1.698168in}{2.483690in}}%
\pgfusepath{stroke}%
\end{pgfscope}%
\begin{pgfscope}%
\pgfpathrectangle{\pgfqpoint{0.100000in}{0.212622in}}{\pgfqpoint{3.696000in}{3.696000in}}%
\pgfusepath{clip}%
\pgfsetrectcap%
\pgfsetroundjoin%
\pgfsetlinewidth{1.505625pt}%
\definecolor{currentstroke}{rgb}{1.000000,0.000000,0.000000}%
\pgfsetstrokecolor{currentstroke}%
\pgfsetdash{}{0pt}%
\pgfpathmoveto{\pgfqpoint{1.830947in}{3.296209in}}%
\pgfpathlineto{\pgfqpoint{1.698168in}{2.483690in}}%
\pgfusepath{stroke}%
\end{pgfscope}%
\begin{pgfscope}%
\pgfpathrectangle{\pgfqpoint{0.100000in}{0.212622in}}{\pgfqpoint{3.696000in}{3.696000in}}%
\pgfusepath{clip}%
\pgfsetrectcap%
\pgfsetroundjoin%
\pgfsetlinewidth{1.505625pt}%
\definecolor{currentstroke}{rgb}{1.000000,0.000000,0.000000}%
\pgfsetstrokecolor{currentstroke}%
\pgfsetdash{}{0pt}%
\pgfpathmoveto{\pgfqpoint{1.832187in}{3.291976in}}%
\pgfpathlineto{\pgfqpoint{1.698168in}{2.483690in}}%
\pgfusepath{stroke}%
\end{pgfscope}%
\begin{pgfscope}%
\pgfpathrectangle{\pgfqpoint{0.100000in}{0.212622in}}{\pgfqpoint{3.696000in}{3.696000in}}%
\pgfusepath{clip}%
\pgfsetrectcap%
\pgfsetroundjoin%
\pgfsetlinewidth{1.505625pt}%
\definecolor{currentstroke}{rgb}{1.000000,0.000000,0.000000}%
\pgfsetstrokecolor{currentstroke}%
\pgfsetdash{}{0pt}%
\pgfpathmoveto{\pgfqpoint{1.833989in}{3.284915in}}%
\pgfpathlineto{\pgfqpoint{1.698168in}{2.483690in}}%
\pgfusepath{stroke}%
\end{pgfscope}%
\begin{pgfscope}%
\pgfpathrectangle{\pgfqpoint{0.100000in}{0.212622in}}{\pgfqpoint{3.696000in}{3.696000in}}%
\pgfusepath{clip}%
\pgfsetrectcap%
\pgfsetroundjoin%
\pgfsetlinewidth{1.505625pt}%
\definecolor{currentstroke}{rgb}{1.000000,0.000000,0.000000}%
\pgfsetstrokecolor{currentstroke}%
\pgfsetdash{}{0pt}%
\pgfpathmoveto{\pgfqpoint{1.835278in}{3.281245in}}%
\pgfpathlineto{\pgfqpoint{1.698168in}{2.483690in}}%
\pgfusepath{stroke}%
\end{pgfscope}%
\begin{pgfscope}%
\pgfpathrectangle{\pgfqpoint{0.100000in}{0.212622in}}{\pgfqpoint{3.696000in}{3.696000in}}%
\pgfusepath{clip}%
\pgfsetrectcap%
\pgfsetroundjoin%
\pgfsetlinewidth{1.505625pt}%
\definecolor{currentstroke}{rgb}{1.000000,0.000000,0.000000}%
\pgfsetstrokecolor{currentstroke}%
\pgfsetdash{}{0pt}%
\pgfpathmoveto{\pgfqpoint{1.835832in}{3.279121in}}%
\pgfpathlineto{\pgfqpoint{1.698168in}{2.483690in}}%
\pgfusepath{stroke}%
\end{pgfscope}%
\begin{pgfscope}%
\pgfpathrectangle{\pgfqpoint{0.100000in}{0.212622in}}{\pgfqpoint{3.696000in}{3.696000in}}%
\pgfusepath{clip}%
\pgfsetrectcap%
\pgfsetroundjoin%
\pgfsetlinewidth{1.505625pt}%
\definecolor{currentstroke}{rgb}{1.000000,0.000000,0.000000}%
\pgfsetstrokecolor{currentstroke}%
\pgfsetdash{}{0pt}%
\pgfpathmoveto{\pgfqpoint{1.837255in}{3.274988in}}%
\pgfpathlineto{\pgfqpoint{1.698168in}{2.483690in}}%
\pgfusepath{stroke}%
\end{pgfscope}%
\begin{pgfscope}%
\pgfpathrectangle{\pgfqpoint{0.100000in}{0.212622in}}{\pgfqpoint{3.696000in}{3.696000in}}%
\pgfusepath{clip}%
\pgfsetrectcap%
\pgfsetroundjoin%
\pgfsetlinewidth{1.505625pt}%
\definecolor{currentstroke}{rgb}{1.000000,0.000000,0.000000}%
\pgfsetstrokecolor{currentstroke}%
\pgfsetdash{}{0pt}%
\pgfpathmoveto{\pgfqpoint{1.837919in}{3.272647in}}%
\pgfpathlineto{\pgfqpoint{1.698168in}{2.483690in}}%
\pgfusepath{stroke}%
\end{pgfscope}%
\begin{pgfscope}%
\pgfpathrectangle{\pgfqpoint{0.100000in}{0.212622in}}{\pgfqpoint{3.696000in}{3.696000in}}%
\pgfusepath{clip}%
\pgfsetrectcap%
\pgfsetroundjoin%
\pgfsetlinewidth{1.505625pt}%
\definecolor{currentstroke}{rgb}{1.000000,0.000000,0.000000}%
\pgfsetstrokecolor{currentstroke}%
\pgfsetdash{}{0pt}%
\pgfpathmoveto{\pgfqpoint{1.840168in}{3.265498in}}%
\pgfpathlineto{\pgfqpoint{1.698168in}{2.483690in}}%
\pgfusepath{stroke}%
\end{pgfscope}%
\begin{pgfscope}%
\pgfpathrectangle{\pgfqpoint{0.100000in}{0.212622in}}{\pgfqpoint{3.696000in}{3.696000in}}%
\pgfusepath{clip}%
\pgfsetrectcap%
\pgfsetroundjoin%
\pgfsetlinewidth{1.505625pt}%
\definecolor{currentstroke}{rgb}{1.000000,0.000000,0.000000}%
\pgfsetstrokecolor{currentstroke}%
\pgfsetdash{}{0pt}%
\pgfpathmoveto{\pgfqpoint{1.842613in}{3.254950in}}%
\pgfpathlineto{\pgfqpoint{1.698168in}{2.483690in}}%
\pgfusepath{stroke}%
\end{pgfscope}%
\begin{pgfscope}%
\pgfpathrectangle{\pgfqpoint{0.100000in}{0.212622in}}{\pgfqpoint{3.696000in}{3.696000in}}%
\pgfusepath{clip}%
\pgfsetrectcap%
\pgfsetroundjoin%
\pgfsetlinewidth{1.505625pt}%
\definecolor{currentstroke}{rgb}{1.000000,0.000000,0.000000}%
\pgfsetstrokecolor{currentstroke}%
\pgfsetdash{}{0pt}%
\pgfpathmoveto{\pgfqpoint{1.846243in}{3.240969in}}%
\pgfpathlineto{\pgfqpoint{1.712030in}{2.479788in}}%
\pgfusepath{stroke}%
\end{pgfscope}%
\begin{pgfscope}%
\pgfpathrectangle{\pgfqpoint{0.100000in}{0.212622in}}{\pgfqpoint{3.696000in}{3.696000in}}%
\pgfusepath{clip}%
\pgfsetrectcap%
\pgfsetroundjoin%
\pgfsetlinewidth{1.505625pt}%
\definecolor{currentstroke}{rgb}{1.000000,0.000000,0.000000}%
\pgfsetstrokecolor{currentstroke}%
\pgfsetdash{}{0pt}%
\pgfpathmoveto{\pgfqpoint{1.851093in}{3.221717in}}%
\pgfpathlineto{\pgfqpoint{1.725902in}{2.475884in}}%
\pgfusepath{stroke}%
\end{pgfscope}%
\begin{pgfscope}%
\pgfpathrectangle{\pgfqpoint{0.100000in}{0.212622in}}{\pgfqpoint{3.696000in}{3.696000in}}%
\pgfusepath{clip}%
\pgfsetrectcap%
\pgfsetroundjoin%
\pgfsetlinewidth{1.505625pt}%
\definecolor{currentstroke}{rgb}{1.000000,0.000000,0.000000}%
\pgfsetstrokecolor{currentstroke}%
\pgfsetdash{}{0pt}%
\pgfpathmoveto{\pgfqpoint{1.857850in}{3.198302in}}%
\pgfpathlineto{\pgfqpoint{1.739784in}{2.471977in}}%
\pgfusepath{stroke}%
\end{pgfscope}%
\begin{pgfscope}%
\pgfpathrectangle{\pgfqpoint{0.100000in}{0.212622in}}{\pgfqpoint{3.696000in}{3.696000in}}%
\pgfusepath{clip}%
\pgfsetrectcap%
\pgfsetroundjoin%
\pgfsetlinewidth{1.505625pt}%
\definecolor{currentstroke}{rgb}{1.000000,0.000000,0.000000}%
\pgfsetstrokecolor{currentstroke}%
\pgfsetdash{}{0pt}%
\pgfpathmoveto{\pgfqpoint{1.866944in}{3.172097in}}%
\pgfpathlineto{\pgfqpoint{1.753676in}{2.468067in}}%
\pgfusepath{stroke}%
\end{pgfscope}%
\begin{pgfscope}%
\pgfpathrectangle{\pgfqpoint{0.100000in}{0.212622in}}{\pgfqpoint{3.696000in}{3.696000in}}%
\pgfusepath{clip}%
\pgfsetrectcap%
\pgfsetroundjoin%
\pgfsetlinewidth{1.505625pt}%
\definecolor{currentstroke}{rgb}{1.000000,0.000000,0.000000}%
\pgfsetstrokecolor{currentstroke}%
\pgfsetdash{}{0pt}%
\pgfpathmoveto{\pgfqpoint{1.871547in}{3.157300in}}%
\pgfpathlineto{\pgfqpoint{1.767577in}{2.464155in}}%
\pgfusepath{stroke}%
\end{pgfscope}%
\begin{pgfscope}%
\pgfpathrectangle{\pgfqpoint{0.100000in}{0.212622in}}{\pgfqpoint{3.696000in}{3.696000in}}%
\pgfusepath{clip}%
\pgfsetrectcap%
\pgfsetroundjoin%
\pgfsetlinewidth{1.505625pt}%
\definecolor{currentstroke}{rgb}{1.000000,0.000000,0.000000}%
\pgfsetstrokecolor{currentstroke}%
\pgfsetdash{}{0pt}%
\pgfpathmoveto{\pgfqpoint{1.876804in}{3.140236in}}%
\pgfpathlineto{\pgfqpoint{1.781488in}{2.460240in}}%
\pgfusepath{stroke}%
\end{pgfscope}%
\begin{pgfscope}%
\pgfpathrectangle{\pgfqpoint{0.100000in}{0.212622in}}{\pgfqpoint{3.696000in}{3.696000in}}%
\pgfusepath{clip}%
\pgfsetrectcap%
\pgfsetroundjoin%
\pgfsetlinewidth{1.505625pt}%
\definecolor{currentstroke}{rgb}{1.000000,0.000000,0.000000}%
\pgfsetstrokecolor{currentstroke}%
\pgfsetdash{}{0pt}%
\pgfpathmoveto{\pgfqpoint{1.883399in}{3.119716in}}%
\pgfpathlineto{\pgfqpoint{1.795409in}{2.456322in}}%
\pgfusepath{stroke}%
\end{pgfscope}%
\begin{pgfscope}%
\pgfpathrectangle{\pgfqpoint{0.100000in}{0.212622in}}{\pgfqpoint{3.696000in}{3.696000in}}%
\pgfusepath{clip}%
\pgfsetrectcap%
\pgfsetroundjoin%
\pgfsetlinewidth{1.505625pt}%
\definecolor{currentstroke}{rgb}{1.000000,0.000000,0.000000}%
\pgfsetstrokecolor{currentstroke}%
\pgfsetdash{}{0pt}%
\pgfpathmoveto{\pgfqpoint{1.890692in}{3.094519in}}%
\pgfpathlineto{\pgfqpoint{1.809339in}{2.452401in}}%
\pgfusepath{stroke}%
\end{pgfscope}%
\begin{pgfscope}%
\pgfpathrectangle{\pgfqpoint{0.100000in}{0.212622in}}{\pgfqpoint{3.696000in}{3.696000in}}%
\pgfusepath{clip}%
\pgfsetrectcap%
\pgfsetroundjoin%
\pgfsetlinewidth{1.505625pt}%
\definecolor{currentstroke}{rgb}{1.000000,0.000000,0.000000}%
\pgfsetstrokecolor{currentstroke}%
\pgfsetdash{}{0pt}%
\pgfpathmoveto{\pgfqpoint{1.899557in}{3.066958in}}%
\pgfpathlineto{\pgfqpoint{1.837230in}{2.444551in}}%
\pgfusepath{stroke}%
\end{pgfscope}%
\begin{pgfscope}%
\pgfpathrectangle{\pgfqpoint{0.100000in}{0.212622in}}{\pgfqpoint{3.696000in}{3.696000in}}%
\pgfusepath{clip}%
\pgfsetrectcap%
\pgfsetroundjoin%
\pgfsetlinewidth{1.505625pt}%
\definecolor{currentstroke}{rgb}{1.000000,0.000000,0.000000}%
\pgfsetstrokecolor{currentstroke}%
\pgfsetdash{}{0pt}%
\pgfpathmoveto{\pgfqpoint{1.909322in}{3.033744in}}%
\pgfpathlineto{\pgfqpoint{1.851190in}{2.440622in}}%
\pgfusepath{stroke}%
\end{pgfscope}%
\begin{pgfscope}%
\pgfpathrectangle{\pgfqpoint{0.100000in}{0.212622in}}{\pgfqpoint{3.696000in}{3.696000in}}%
\pgfusepath{clip}%
\pgfsetrectcap%
\pgfsetroundjoin%
\pgfsetlinewidth{1.505625pt}%
\definecolor{currentstroke}{rgb}{1.000000,0.000000,0.000000}%
\pgfsetstrokecolor{currentstroke}%
\pgfsetdash{}{0pt}%
\pgfpathmoveto{\pgfqpoint{1.922033in}{2.998681in}}%
\pgfpathlineto{\pgfqpoint{1.879139in}{2.432756in}}%
\pgfusepath{stroke}%
\end{pgfscope}%
\begin{pgfscope}%
\pgfpathrectangle{\pgfqpoint{0.100000in}{0.212622in}}{\pgfqpoint{3.696000in}{3.696000in}}%
\pgfusepath{clip}%
\pgfsetrectcap%
\pgfsetroundjoin%
\pgfsetlinewidth{1.505625pt}%
\definecolor{currentstroke}{rgb}{1.000000,0.000000,0.000000}%
\pgfsetstrokecolor{currentstroke}%
\pgfsetdash{}{0pt}%
\pgfpathmoveto{\pgfqpoint{1.934223in}{2.955212in}}%
\pgfpathlineto{\pgfqpoint{1.907127in}{2.424879in}}%
\pgfusepath{stroke}%
\end{pgfscope}%
\begin{pgfscope}%
\pgfpathrectangle{\pgfqpoint{0.100000in}{0.212622in}}{\pgfqpoint{3.696000in}{3.696000in}}%
\pgfusepath{clip}%
\pgfsetrectcap%
\pgfsetroundjoin%
\pgfsetlinewidth{1.505625pt}%
\definecolor{currentstroke}{rgb}{1.000000,0.000000,0.000000}%
\pgfsetstrokecolor{currentstroke}%
\pgfsetdash{}{0pt}%
\pgfpathmoveto{\pgfqpoint{1.943664in}{2.933296in}}%
\pgfpathlineto{\pgfqpoint{1.921136in}{2.420936in}}%
\pgfusepath{stroke}%
\end{pgfscope}%
\begin{pgfscope}%
\pgfpathrectangle{\pgfqpoint{0.100000in}{0.212622in}}{\pgfqpoint{3.696000in}{3.696000in}}%
\pgfusepath{clip}%
\pgfsetrectcap%
\pgfsetroundjoin%
\pgfsetlinewidth{1.505625pt}%
\definecolor{currentstroke}{rgb}{1.000000,0.000000,0.000000}%
\pgfsetstrokecolor{currentstroke}%
\pgfsetdash{}{0pt}%
\pgfpathmoveto{\pgfqpoint{1.953462in}{2.904412in}}%
\pgfpathlineto{\pgfqpoint{1.949184in}{2.413042in}}%
\pgfusepath{stroke}%
\end{pgfscope}%
\begin{pgfscope}%
\pgfpathrectangle{\pgfqpoint{0.100000in}{0.212622in}}{\pgfqpoint{3.696000in}{3.696000in}}%
\pgfusepath{clip}%
\pgfsetrectcap%
\pgfsetroundjoin%
\pgfsetlinewidth{1.505625pt}%
\definecolor{currentstroke}{rgb}{1.000000,0.000000,0.000000}%
\pgfsetstrokecolor{currentstroke}%
\pgfsetdash{}{0pt}%
\pgfpathmoveto{\pgfqpoint{1.966350in}{2.873429in}}%
\pgfpathlineto{\pgfqpoint{1.963222in}{2.409091in}}%
\pgfusepath{stroke}%
\end{pgfscope}%
\begin{pgfscope}%
\pgfpathrectangle{\pgfqpoint{0.100000in}{0.212622in}}{\pgfqpoint{3.696000in}{3.696000in}}%
\pgfusepath{clip}%
\pgfsetrectcap%
\pgfsetroundjoin%
\pgfsetlinewidth{1.505625pt}%
\definecolor{currentstroke}{rgb}{1.000000,0.000000,0.000000}%
\pgfsetstrokecolor{currentstroke}%
\pgfsetdash{}{0pt}%
\pgfpathmoveto{\pgfqpoint{1.978543in}{2.837616in}}%
\pgfpathlineto{\pgfqpoint{1.991329in}{2.401180in}}%
\pgfusepath{stroke}%
\end{pgfscope}%
\begin{pgfscope}%
\pgfpathrectangle{\pgfqpoint{0.100000in}{0.212622in}}{\pgfqpoint{3.696000in}{3.696000in}}%
\pgfusepath{clip}%
\pgfsetrectcap%
\pgfsetroundjoin%
\pgfsetlinewidth{1.505625pt}%
\definecolor{currentstroke}{rgb}{1.000000,0.000000,0.000000}%
\pgfsetstrokecolor{currentstroke}%
\pgfsetdash{}{0pt}%
\pgfpathmoveto{\pgfqpoint{1.994589in}{2.799314in}}%
\pgfpathlineto{\pgfqpoint{2.019475in}{2.393258in}}%
\pgfusepath{stroke}%
\end{pgfscope}%
\begin{pgfscope}%
\pgfpathrectangle{\pgfqpoint{0.100000in}{0.212622in}}{\pgfqpoint{3.696000in}{3.696000in}}%
\pgfusepath{clip}%
\pgfsetrectcap%
\pgfsetroundjoin%
\pgfsetlinewidth{1.505625pt}%
\definecolor{currentstroke}{rgb}{1.000000,0.000000,0.000000}%
\pgfsetstrokecolor{currentstroke}%
\pgfsetdash{}{0pt}%
\pgfpathmoveto{\pgfqpoint{2.010927in}{2.756169in}}%
\pgfpathlineto{\pgfqpoint{2.061769in}{2.381355in}}%
\pgfusepath{stroke}%
\end{pgfscope}%
\begin{pgfscope}%
\pgfpathrectangle{\pgfqpoint{0.100000in}{0.212622in}}{\pgfqpoint{3.696000in}{3.696000in}}%
\pgfusepath{clip}%
\pgfsetrectcap%
\pgfsetroundjoin%
\pgfsetlinewidth{1.505625pt}%
\definecolor{currentstroke}{rgb}{1.000000,0.000000,0.000000}%
\pgfsetstrokecolor{currentstroke}%
\pgfsetdash{}{0pt}%
\pgfpathmoveto{\pgfqpoint{2.032405in}{2.710243in}}%
\pgfpathlineto{\pgfqpoint{2.090015in}{2.373405in}}%
\pgfusepath{stroke}%
\end{pgfscope}%
\begin{pgfscope}%
\pgfpathrectangle{\pgfqpoint{0.100000in}{0.212622in}}{\pgfqpoint{3.696000in}{3.696000in}}%
\pgfusepath{clip}%
\pgfsetrectcap%
\pgfsetroundjoin%
\pgfsetlinewidth{1.505625pt}%
\definecolor{currentstroke}{rgb}{1.000000,0.000000,0.000000}%
\pgfsetstrokecolor{currentstroke}%
\pgfsetdash{}{0pt}%
\pgfpathmoveto{\pgfqpoint{2.056721in}{2.661673in}}%
\pgfpathlineto{\pgfqpoint{2.132458in}{2.361460in}}%
\pgfusepath{stroke}%
\end{pgfscope}%
\begin{pgfscope}%
\pgfpathrectangle{\pgfqpoint{0.100000in}{0.212622in}}{\pgfqpoint{3.696000in}{3.696000in}}%
\pgfusepath{clip}%
\pgfsetrectcap%
\pgfsetroundjoin%
\pgfsetlinewidth{1.505625pt}%
\definecolor{currentstroke}{rgb}{1.000000,0.000000,0.000000}%
\pgfsetstrokecolor{currentstroke}%
\pgfsetdash{}{0pt}%
\pgfpathmoveto{\pgfqpoint{2.070179in}{2.634837in}}%
\pgfpathlineto{\pgfqpoint{2.160804in}{2.353482in}}%
\pgfusepath{stroke}%
\end{pgfscope}%
\begin{pgfscope}%
\pgfpathrectangle{\pgfqpoint{0.100000in}{0.212622in}}{\pgfqpoint{3.696000in}{3.696000in}}%
\pgfusepath{clip}%
\pgfsetrectcap%
\pgfsetroundjoin%
\pgfsetlinewidth{1.505625pt}%
\definecolor{currentstroke}{rgb}{1.000000,0.000000,0.000000}%
\pgfsetstrokecolor{currentstroke}%
\pgfsetdash{}{0pt}%
\pgfpathmoveto{\pgfqpoint{2.083780in}{2.605139in}}%
\pgfpathlineto{\pgfqpoint{2.174992in}{2.349489in}}%
\pgfusepath{stroke}%
\end{pgfscope}%
\begin{pgfscope}%
\pgfpathrectangle{\pgfqpoint{0.100000in}{0.212622in}}{\pgfqpoint{3.696000in}{3.696000in}}%
\pgfusepath{clip}%
\pgfsetrectcap%
\pgfsetroundjoin%
\pgfsetlinewidth{1.505625pt}%
\definecolor{currentstroke}{rgb}{1.000000,0.000000,0.000000}%
\pgfsetstrokecolor{currentstroke}%
\pgfsetdash{}{0pt}%
\pgfpathmoveto{\pgfqpoint{2.099315in}{2.573661in}}%
\pgfpathlineto{\pgfqpoint{2.203398in}{2.341494in}}%
\pgfusepath{stroke}%
\end{pgfscope}%
\begin{pgfscope}%
\pgfpathrectangle{\pgfqpoint{0.100000in}{0.212622in}}{\pgfqpoint{3.696000in}{3.696000in}}%
\pgfusepath{clip}%
\pgfsetrectcap%
\pgfsetroundjoin%
\pgfsetlinewidth{1.505625pt}%
\definecolor{currentstroke}{rgb}{1.000000,0.000000,0.000000}%
\pgfsetstrokecolor{currentstroke}%
\pgfsetdash{}{0pt}%
\pgfpathmoveto{\pgfqpoint{2.105413in}{2.554848in}}%
\pgfpathlineto{\pgfqpoint{2.217616in}{2.337492in}}%
\pgfusepath{stroke}%
\end{pgfscope}%
\begin{pgfscope}%
\pgfpathrectangle{\pgfqpoint{0.100000in}{0.212622in}}{\pgfqpoint{3.696000in}{3.696000in}}%
\pgfusepath{clip}%
\pgfsetrectcap%
\pgfsetroundjoin%
\pgfsetlinewidth{1.505625pt}%
\definecolor{currentstroke}{rgb}{1.000000,0.000000,0.000000}%
\pgfsetstrokecolor{currentstroke}%
\pgfsetdash{}{0pt}%
\pgfpathmoveto{\pgfqpoint{2.113770in}{2.535320in}}%
\pgfpathlineto{\pgfqpoint{2.231844in}{2.333488in}}%
\pgfusepath{stroke}%
\end{pgfscope}%
\begin{pgfscope}%
\pgfpathrectangle{\pgfqpoint{0.100000in}{0.212622in}}{\pgfqpoint{3.696000in}{3.696000in}}%
\pgfusepath{clip}%
\pgfsetrectcap%
\pgfsetroundjoin%
\pgfsetlinewidth{1.505625pt}%
\definecolor{currentstroke}{rgb}{1.000000,0.000000,0.000000}%
\pgfsetstrokecolor{currentstroke}%
\pgfsetdash{}{0pt}%
\pgfpathmoveto{\pgfqpoint{2.119648in}{2.512136in}}%
\pgfpathlineto{\pgfqpoint{2.246082in}{2.329480in}}%
\pgfusepath{stroke}%
\end{pgfscope}%
\begin{pgfscope}%
\pgfpathrectangle{\pgfqpoint{0.100000in}{0.212622in}}{\pgfqpoint{3.696000in}{3.696000in}}%
\pgfusepath{clip}%
\pgfsetrectcap%
\pgfsetroundjoin%
\pgfsetlinewidth{1.505625pt}%
\definecolor{currentstroke}{rgb}{1.000000,0.000000,0.000000}%
\pgfsetstrokecolor{currentstroke}%
\pgfsetdash{}{0pt}%
\pgfpathmoveto{\pgfqpoint{2.129149in}{2.488557in}}%
\pgfpathlineto{\pgfqpoint{2.260330in}{2.325470in}}%
\pgfusepath{stroke}%
\end{pgfscope}%
\begin{pgfscope}%
\pgfpathrectangle{\pgfqpoint{0.100000in}{0.212622in}}{\pgfqpoint{3.696000in}{3.696000in}}%
\pgfusepath{clip}%
\pgfsetrectcap%
\pgfsetroundjoin%
\pgfsetlinewidth{1.505625pt}%
\definecolor{currentstroke}{rgb}{1.000000,0.000000,0.000000}%
\pgfsetstrokecolor{currentstroke}%
\pgfsetdash{}{0pt}%
\pgfpathmoveto{\pgfqpoint{2.137307in}{2.459121in}}%
\pgfpathlineto{\pgfqpoint{2.288857in}{2.317441in}}%
\pgfusepath{stroke}%
\end{pgfscope}%
\begin{pgfscope}%
\pgfpathrectangle{\pgfqpoint{0.100000in}{0.212622in}}{\pgfqpoint{3.696000in}{3.696000in}}%
\pgfusepath{clip}%
\pgfsetrectcap%
\pgfsetroundjoin%
\pgfsetlinewidth{1.505625pt}%
\definecolor{currentstroke}{rgb}{1.000000,0.000000,0.000000}%
\pgfsetstrokecolor{currentstroke}%
\pgfsetdash{}{0pt}%
\pgfpathmoveto{\pgfqpoint{2.142559in}{2.443633in}}%
\pgfpathlineto{\pgfqpoint{2.303135in}{2.313423in}}%
\pgfusepath{stroke}%
\end{pgfscope}%
\begin{pgfscope}%
\pgfpathrectangle{\pgfqpoint{0.100000in}{0.212622in}}{\pgfqpoint{3.696000in}{3.696000in}}%
\pgfusepath{clip}%
\pgfsetrectcap%
\pgfsetroundjoin%
\pgfsetlinewidth{1.505625pt}%
\definecolor{currentstroke}{rgb}{1.000000,0.000000,0.000000}%
\pgfsetstrokecolor{currentstroke}%
\pgfsetdash{}{0pt}%
\pgfpathmoveto{\pgfqpoint{2.147667in}{2.422900in}}%
\pgfpathlineto{\pgfqpoint{2.303135in}{2.313423in}}%
\pgfusepath{stroke}%
\end{pgfscope}%
\begin{pgfscope}%
\pgfpathrectangle{\pgfqpoint{0.100000in}{0.212622in}}{\pgfqpoint{3.696000in}{3.696000in}}%
\pgfusepath{clip}%
\pgfsetrectcap%
\pgfsetroundjoin%
\pgfsetlinewidth{1.505625pt}%
\definecolor{currentstroke}{rgb}{1.000000,0.000000,0.000000}%
\pgfsetstrokecolor{currentstroke}%
\pgfsetdash{}{0pt}%
\pgfpathmoveto{\pgfqpoint{2.151607in}{2.412118in}}%
\pgfpathlineto{\pgfqpoint{2.317424in}{2.309401in}}%
\pgfusepath{stroke}%
\end{pgfscope}%
\begin{pgfscope}%
\pgfpathrectangle{\pgfqpoint{0.100000in}{0.212622in}}{\pgfqpoint{3.696000in}{3.696000in}}%
\pgfusepath{clip}%
\pgfsetrectcap%
\pgfsetroundjoin%
\pgfsetlinewidth{1.505625pt}%
\definecolor{currentstroke}{rgb}{1.000000,0.000000,0.000000}%
\pgfsetstrokecolor{currentstroke}%
\pgfsetdash{}{0pt}%
\pgfpathmoveto{\pgfqpoint{2.156086in}{2.396180in}}%
\pgfpathlineto{\pgfqpoint{2.331723in}{2.305377in}}%
\pgfusepath{stroke}%
\end{pgfscope}%
\begin{pgfscope}%
\pgfpathrectangle{\pgfqpoint{0.100000in}{0.212622in}}{\pgfqpoint{3.696000in}{3.696000in}}%
\pgfusepath{clip}%
\pgfsetrectcap%
\pgfsetroundjoin%
\pgfsetlinewidth{1.505625pt}%
\definecolor{currentstroke}{rgb}{1.000000,0.000000,0.000000}%
\pgfsetstrokecolor{currentstroke}%
\pgfsetdash{}{0pt}%
\pgfpathmoveto{\pgfqpoint{2.158775in}{2.387414in}}%
\pgfpathlineto{\pgfqpoint{2.331723in}{2.305377in}}%
\pgfusepath{stroke}%
\end{pgfscope}%
\begin{pgfscope}%
\pgfpathrectangle{\pgfqpoint{0.100000in}{0.212622in}}{\pgfqpoint{3.696000in}{3.696000in}}%
\pgfusepath{clip}%
\pgfsetrectcap%
\pgfsetroundjoin%
\pgfsetlinewidth{1.505625pt}%
\definecolor{currentstroke}{rgb}{1.000000,0.000000,0.000000}%
\pgfsetstrokecolor{currentstroke}%
\pgfsetdash{}{0pt}%
\pgfpathmoveto{\pgfqpoint{2.163671in}{2.371603in}}%
\pgfpathlineto{\pgfqpoint{2.346032in}{2.301350in}}%
\pgfusepath{stroke}%
\end{pgfscope}%
\begin{pgfscope}%
\pgfpathrectangle{\pgfqpoint{0.100000in}{0.212622in}}{\pgfqpoint{3.696000in}{3.696000in}}%
\pgfusepath{clip}%
\pgfsetrectcap%
\pgfsetroundjoin%
\pgfsetlinewidth{1.505625pt}%
\definecolor{currentstroke}{rgb}{1.000000,0.000000,0.000000}%
\pgfsetstrokecolor{currentstroke}%
\pgfsetdash{}{0pt}%
\pgfpathmoveto{\pgfqpoint{2.169907in}{2.354450in}}%
\pgfpathlineto{\pgfqpoint{2.360351in}{2.297319in}}%
\pgfusepath{stroke}%
\end{pgfscope}%
\begin{pgfscope}%
\pgfpathrectangle{\pgfqpoint{0.100000in}{0.212622in}}{\pgfqpoint{3.696000in}{3.696000in}}%
\pgfusepath{clip}%
\pgfsetrectcap%
\pgfsetroundjoin%
\pgfsetlinewidth{1.505625pt}%
\definecolor{currentstroke}{rgb}{1.000000,0.000000,0.000000}%
\pgfsetstrokecolor{currentstroke}%
\pgfsetdash{}{0pt}%
\pgfpathmoveto{\pgfqpoint{2.175421in}{2.332859in}}%
\pgfpathlineto{\pgfqpoint{2.374680in}{2.293286in}}%
\pgfusepath{stroke}%
\end{pgfscope}%
\begin{pgfscope}%
\pgfpathrectangle{\pgfqpoint{0.100000in}{0.212622in}}{\pgfqpoint{3.696000in}{3.696000in}}%
\pgfusepath{clip}%
\pgfsetrectcap%
\pgfsetroundjoin%
\pgfsetlinewidth{1.505625pt}%
\definecolor{currentstroke}{rgb}{1.000000,0.000000,0.000000}%
\pgfsetstrokecolor{currentstroke}%
\pgfsetdash{}{0pt}%
\pgfpathmoveto{\pgfqpoint{2.183564in}{2.310030in}}%
\pgfpathlineto{\pgfqpoint{2.389020in}{2.289251in}}%
\pgfusepath{stroke}%
\end{pgfscope}%
\begin{pgfscope}%
\pgfpathrectangle{\pgfqpoint{0.100000in}{0.212622in}}{\pgfqpoint{3.696000in}{3.696000in}}%
\pgfusepath{clip}%
\pgfsetrectcap%
\pgfsetroundjoin%
\pgfsetlinewidth{1.505625pt}%
\definecolor{currentstroke}{rgb}{1.000000,0.000000,0.000000}%
\pgfsetstrokecolor{currentstroke}%
\pgfsetdash{}{0pt}%
\pgfpathmoveto{\pgfqpoint{2.190501in}{2.280376in}}%
\pgfpathlineto{\pgfqpoint{2.403370in}{2.285212in}}%
\pgfusepath{stroke}%
\end{pgfscope}%
\begin{pgfscope}%
\pgfpathrectangle{\pgfqpoint{0.100000in}{0.212622in}}{\pgfqpoint{3.696000in}{3.696000in}}%
\pgfusepath{clip}%
\pgfsetrectcap%
\pgfsetroundjoin%
\pgfsetlinewidth{1.505625pt}%
\definecolor{currentstroke}{rgb}{1.000000,0.000000,0.000000}%
\pgfsetstrokecolor{currentstroke}%
\pgfsetdash{}{0pt}%
\pgfpathmoveto{\pgfqpoint{2.201368in}{2.249367in}}%
\pgfpathlineto{\pgfqpoint{2.432100in}{2.277126in}}%
\pgfusepath{stroke}%
\end{pgfscope}%
\begin{pgfscope}%
\pgfpathrectangle{\pgfqpoint{0.100000in}{0.212622in}}{\pgfqpoint{3.696000in}{3.696000in}}%
\pgfusepath{clip}%
\pgfsetrectcap%
\pgfsetroundjoin%
\pgfsetlinewidth{1.505625pt}%
\definecolor{currentstroke}{rgb}{1.000000,0.000000,0.000000}%
\pgfsetstrokecolor{currentstroke}%
\pgfsetdash{}{0pt}%
\pgfpathmoveto{\pgfqpoint{2.210770in}{2.209534in}}%
\pgfpathlineto{\pgfqpoint{2.446480in}{2.273078in}}%
\pgfusepath{stroke}%
\end{pgfscope}%
\begin{pgfscope}%
\pgfpathrectangle{\pgfqpoint{0.100000in}{0.212622in}}{\pgfqpoint{3.696000in}{3.696000in}}%
\pgfusepath{clip}%
\pgfsetrectcap%
\pgfsetroundjoin%
\pgfsetlinewidth{1.505625pt}%
\definecolor{currentstroke}{rgb}{1.000000,0.000000,0.000000}%
\pgfsetstrokecolor{currentstroke}%
\pgfsetdash{}{0pt}%
\pgfpathmoveto{\pgfqpoint{2.223755in}{2.166842in}}%
\pgfpathlineto{\pgfqpoint{2.475272in}{2.264975in}}%
\pgfusepath{stroke}%
\end{pgfscope}%
\begin{pgfscope}%
\pgfpathrectangle{\pgfqpoint{0.100000in}{0.212622in}}{\pgfqpoint{3.696000in}{3.696000in}}%
\pgfusepath{clip}%
\pgfsetrectcap%
\pgfsetroundjoin%
\pgfsetlinewidth{1.505625pt}%
\definecolor{currentstroke}{rgb}{1.000000,0.000000,0.000000}%
\pgfsetstrokecolor{currentstroke}%
\pgfsetdash{}{0pt}%
\pgfpathmoveto{\pgfqpoint{2.235571in}{2.118726in}}%
\pgfpathlineto{\pgfqpoint{2.857470in}{1.835625in}}%
\pgfusepath{stroke}%
\end{pgfscope}%
\begin{pgfscope}%
\pgfpathrectangle{\pgfqpoint{0.100000in}{0.212622in}}{\pgfqpoint{3.696000in}{3.696000in}}%
\pgfusepath{clip}%
\pgfsetrectcap%
\pgfsetroundjoin%
\pgfsetlinewidth{1.505625pt}%
\definecolor{currentstroke}{rgb}{1.000000,0.000000,0.000000}%
\pgfsetstrokecolor{currentstroke}%
\pgfsetdash{}{0pt}%
\pgfpathmoveto{\pgfqpoint{2.243759in}{2.093021in}}%
\pgfpathlineto{\pgfqpoint{2.845833in}{1.824596in}}%
\pgfusepath{stroke}%
\end{pgfscope}%
\begin{pgfscope}%
\pgfpathrectangle{\pgfqpoint{0.100000in}{0.212622in}}{\pgfqpoint{3.696000in}{3.696000in}}%
\pgfusepath{clip}%
\pgfsetrectcap%
\pgfsetroundjoin%
\pgfsetlinewidth{1.505625pt}%
\definecolor{currentstroke}{rgb}{1.000000,0.000000,0.000000}%
\pgfsetstrokecolor{currentstroke}%
\pgfsetdash{}{0pt}%
\pgfpathmoveto{\pgfqpoint{2.251242in}{2.060117in}}%
\pgfpathlineto{\pgfqpoint{2.834176in}{1.813546in}}%
\pgfusepath{stroke}%
\end{pgfscope}%
\begin{pgfscope}%
\pgfpathrectangle{\pgfqpoint{0.100000in}{0.212622in}}{\pgfqpoint{3.696000in}{3.696000in}}%
\pgfusepath{clip}%
\pgfsetrectcap%
\pgfsetroundjoin%
\pgfsetlinewidth{1.505625pt}%
\definecolor{currentstroke}{rgb}{1.000000,0.000000,0.000000}%
\pgfsetstrokecolor{currentstroke}%
\pgfsetdash{}{0pt}%
\pgfpathmoveto{\pgfqpoint{2.256328in}{2.042280in}}%
\pgfpathlineto{\pgfqpoint{2.828339in}{1.808014in}}%
\pgfusepath{stroke}%
\end{pgfscope}%
\begin{pgfscope}%
\pgfpathrectangle{\pgfqpoint{0.100000in}{0.212622in}}{\pgfqpoint{3.696000in}{3.696000in}}%
\pgfusepath{clip}%
\pgfsetrectcap%
\pgfsetroundjoin%
\pgfsetlinewidth{1.505625pt}%
\definecolor{currentstroke}{rgb}{1.000000,0.000000,0.000000}%
\pgfsetstrokecolor{currentstroke}%
\pgfsetdash{}{0pt}%
\pgfpathmoveto{\pgfqpoint{2.261386in}{2.015514in}}%
\pgfpathlineto{\pgfqpoint{2.816648in}{1.796933in}}%
\pgfusepath{stroke}%
\end{pgfscope}%
\begin{pgfscope}%
\pgfpathrectangle{\pgfqpoint{0.100000in}{0.212622in}}{\pgfqpoint{3.696000in}{3.696000in}}%
\pgfusepath{clip}%
\pgfsetrectcap%
\pgfsetroundjoin%
\pgfsetlinewidth{1.505625pt}%
\definecolor{currentstroke}{rgb}{1.000000,0.000000,0.000000}%
\pgfsetstrokecolor{currentstroke}%
\pgfsetdash{}{0pt}%
\pgfpathmoveto{\pgfqpoint{2.269286in}{1.984349in}}%
\pgfpathlineto{\pgfqpoint{2.810794in}{1.791385in}}%
\pgfusepath{stroke}%
\end{pgfscope}%
\begin{pgfscope}%
\pgfpathrectangle{\pgfqpoint{0.100000in}{0.212622in}}{\pgfqpoint{3.696000in}{3.696000in}}%
\pgfusepath{clip}%
\pgfsetrectcap%
\pgfsetroundjoin%
\pgfsetlinewidth{1.505625pt}%
\definecolor{currentstroke}{rgb}{1.000000,0.000000,0.000000}%
\pgfsetstrokecolor{currentstroke}%
\pgfsetdash{}{0pt}%
\pgfpathmoveto{\pgfqpoint{2.278192in}{1.946690in}}%
\pgfpathlineto{\pgfqpoint{2.793201in}{1.774709in}}%
\pgfusepath{stroke}%
\end{pgfscope}%
\begin{pgfscope}%
\pgfpathrectangle{\pgfqpoint{0.100000in}{0.212622in}}{\pgfqpoint{3.696000in}{3.696000in}}%
\pgfusepath{clip}%
\pgfsetrectcap%
\pgfsetroundjoin%
\pgfsetlinewidth{1.505625pt}%
\definecolor{currentstroke}{rgb}{1.000000,0.000000,0.000000}%
\pgfsetstrokecolor{currentstroke}%
\pgfsetdash{}{0pt}%
\pgfpathmoveto{\pgfqpoint{2.290825in}{1.908234in}}%
\pgfpathlineto{\pgfqpoint{2.781444in}{1.763566in}}%
\pgfusepath{stroke}%
\end{pgfscope}%
\begin{pgfscope}%
\pgfpathrectangle{\pgfqpoint{0.100000in}{0.212622in}}{\pgfqpoint{3.696000in}{3.696000in}}%
\pgfusepath{clip}%
\pgfsetrectcap%
\pgfsetroundjoin%
\pgfsetlinewidth{1.505625pt}%
\definecolor{currentstroke}{rgb}{1.000000,0.000000,0.000000}%
\pgfsetstrokecolor{currentstroke}%
\pgfsetdash{}{0pt}%
\pgfpathmoveto{\pgfqpoint{2.301074in}{1.863687in}}%
\pgfpathlineto{\pgfqpoint{2.763768in}{1.746812in}}%
\pgfusepath{stroke}%
\end{pgfscope}%
\begin{pgfscope}%
\pgfpathrectangle{\pgfqpoint{0.100000in}{0.212622in}}{\pgfqpoint{3.696000in}{3.696000in}}%
\pgfusepath{clip}%
\pgfsetrectcap%
\pgfsetroundjoin%
\pgfsetlinewidth{1.505625pt}%
\definecolor{currentstroke}{rgb}{1.000000,0.000000,0.000000}%
\pgfsetstrokecolor{currentstroke}%
\pgfsetdash{}{0pt}%
\pgfpathmoveto{\pgfqpoint{2.308332in}{1.840098in}}%
\pgfpathlineto{\pgfqpoint{2.757865in}{1.741216in}}%
\pgfusepath{stroke}%
\end{pgfscope}%
\begin{pgfscope}%
\pgfpathrectangle{\pgfqpoint{0.100000in}{0.212622in}}{\pgfqpoint{3.696000in}{3.696000in}}%
\pgfusepath{clip}%
\pgfsetrectcap%
\pgfsetroundjoin%
\pgfsetlinewidth{1.505625pt}%
\definecolor{currentstroke}{rgb}{1.000000,0.000000,0.000000}%
\pgfsetstrokecolor{currentstroke}%
\pgfsetdash{}{0pt}%
\pgfpathmoveto{\pgfqpoint{2.315268in}{1.811154in}}%
\pgfpathlineto{\pgfqpoint{2.746042in}{1.730010in}}%
\pgfusepath{stroke}%
\end{pgfscope}%
\begin{pgfscope}%
\pgfpathrectangle{\pgfqpoint{0.100000in}{0.212622in}}{\pgfqpoint{3.696000in}{3.696000in}}%
\pgfusepath{clip}%
\pgfsetrectcap%
\pgfsetroundjoin%
\pgfsetlinewidth{1.505625pt}%
\definecolor{currentstroke}{rgb}{1.000000,0.000000,0.000000}%
\pgfsetstrokecolor{currentstroke}%
\pgfsetdash{}{0pt}%
\pgfpathmoveto{\pgfqpoint{2.320070in}{1.795752in}}%
\pgfpathlineto{\pgfqpoint{2.740122in}{1.724399in}}%
\pgfusepath{stroke}%
\end{pgfscope}%
\begin{pgfscope}%
\pgfpathrectangle{\pgfqpoint{0.100000in}{0.212622in}}{\pgfqpoint{3.696000in}{3.696000in}}%
\pgfusepath{clip}%
\pgfsetrectcap%
\pgfsetroundjoin%
\pgfsetlinewidth{1.505625pt}%
\definecolor{currentstroke}{rgb}{1.000000,0.000000,0.000000}%
\pgfsetstrokecolor{currentstroke}%
\pgfsetdash{}{0pt}%
\pgfpathmoveto{\pgfqpoint{2.325595in}{1.774447in}}%
\pgfpathlineto{\pgfqpoint{2.734197in}{1.718783in}}%
\pgfusepath{stroke}%
\end{pgfscope}%
\begin{pgfscope}%
\pgfpathrectangle{\pgfqpoint{0.100000in}{0.212622in}}{\pgfqpoint{3.696000in}{3.696000in}}%
\pgfusepath{clip}%
\pgfsetrectcap%
\pgfsetroundjoin%
\pgfsetlinewidth{1.505625pt}%
\definecolor{currentstroke}{rgb}{1.000000,0.000000,0.000000}%
\pgfsetstrokecolor{currentstroke}%
\pgfsetdash{}{0pt}%
\pgfpathmoveto{\pgfqpoint{2.332253in}{1.749094in}}%
\pgfpathlineto{\pgfqpoint{2.722329in}{1.707534in}}%
\pgfusepath{stroke}%
\end{pgfscope}%
\begin{pgfscope}%
\pgfpathrectangle{\pgfqpoint{0.100000in}{0.212622in}}{\pgfqpoint{3.696000in}{3.696000in}}%
\pgfusepath{clip}%
\pgfsetrectcap%
\pgfsetroundjoin%
\pgfsetlinewidth{1.505625pt}%
\definecolor{currentstroke}{rgb}{1.000000,0.000000,0.000000}%
\pgfsetstrokecolor{currentstroke}%
\pgfsetdash{}{0pt}%
\pgfpathmoveto{\pgfqpoint{2.339239in}{1.719381in}}%
\pgfpathlineto{\pgfqpoint{2.710439in}{1.696265in}}%
\pgfusepath{stroke}%
\end{pgfscope}%
\begin{pgfscope}%
\pgfpathrectangle{\pgfqpoint{0.100000in}{0.212622in}}{\pgfqpoint{3.696000in}{3.696000in}}%
\pgfusepath{clip}%
\pgfsetrectcap%
\pgfsetroundjoin%
\pgfsetlinewidth{1.505625pt}%
\definecolor{currentstroke}{rgb}{1.000000,0.000000,0.000000}%
\pgfsetstrokecolor{currentstroke}%
\pgfsetdash{}{0pt}%
\pgfpathmoveto{\pgfqpoint{2.348849in}{1.684611in}}%
\pgfpathlineto{\pgfqpoint{2.698527in}{1.684974in}}%
\pgfusepath{stroke}%
\end{pgfscope}%
\begin{pgfscope}%
\pgfpathrectangle{\pgfqpoint{0.100000in}{0.212622in}}{\pgfqpoint{3.696000in}{3.696000in}}%
\pgfusepath{clip}%
\pgfsetrectcap%
\pgfsetroundjoin%
\pgfsetlinewidth{1.505625pt}%
\definecolor{currentstroke}{rgb}{1.000000,0.000000,0.000000}%
\pgfsetstrokecolor{currentstroke}%
\pgfsetdash{}{0pt}%
\pgfpathmoveto{\pgfqpoint{2.354114in}{1.665307in}}%
\pgfpathlineto{\pgfqpoint{2.692562in}{1.679320in}}%
\pgfusepath{stroke}%
\end{pgfscope}%
\begin{pgfscope}%
\pgfpathrectangle{\pgfqpoint{0.100000in}{0.212622in}}{\pgfqpoint{3.696000in}{3.696000in}}%
\pgfusepath{clip}%
\pgfsetrectcap%
\pgfsetroundjoin%
\pgfsetlinewidth{1.505625pt}%
\definecolor{currentstroke}{rgb}{1.000000,0.000000,0.000000}%
\pgfsetstrokecolor{currentstroke}%
\pgfsetdash{}{0pt}%
\pgfpathmoveto{\pgfqpoint{2.359625in}{1.643530in}}%
\pgfpathlineto{\pgfqpoint{2.686592in}{1.673661in}}%
\pgfusepath{stroke}%
\end{pgfscope}%
\begin{pgfscope}%
\pgfpathrectangle{\pgfqpoint{0.100000in}{0.212622in}}{\pgfqpoint{3.696000in}{3.696000in}}%
\pgfusepath{clip}%
\pgfsetrectcap%
\pgfsetroundjoin%
\pgfsetlinewidth{1.505625pt}%
\definecolor{currentstroke}{rgb}{1.000000,0.000000,0.000000}%
\pgfsetstrokecolor{currentstroke}%
\pgfsetdash{}{0pt}%
\pgfpathmoveto{\pgfqpoint{2.362838in}{1.631574in}}%
\pgfpathlineto{\pgfqpoint{2.680616in}{1.667997in}}%
\pgfusepath{stroke}%
\end{pgfscope}%
\begin{pgfscope}%
\pgfpathrectangle{\pgfqpoint{0.100000in}{0.212622in}}{\pgfqpoint{3.696000in}{3.696000in}}%
\pgfusepath{clip}%
\pgfsetrectcap%
\pgfsetroundjoin%
\pgfsetlinewidth{1.505625pt}%
\definecolor{currentstroke}{rgb}{1.000000,0.000000,0.000000}%
\pgfsetstrokecolor{currentstroke}%
\pgfsetdash{}{0pt}%
\pgfpathmoveto{\pgfqpoint{2.366489in}{1.615491in}}%
\pgfpathlineto{\pgfqpoint{2.674634in}{1.662327in}}%
\pgfusepath{stroke}%
\end{pgfscope}%
\begin{pgfscope}%
\pgfpathrectangle{\pgfqpoint{0.100000in}{0.212622in}}{\pgfqpoint{3.696000in}{3.696000in}}%
\pgfusepath{clip}%
\pgfsetrectcap%
\pgfsetroundjoin%
\pgfsetlinewidth{1.505625pt}%
\definecolor{currentstroke}{rgb}{1.000000,0.000000,0.000000}%
\pgfsetstrokecolor{currentstroke}%
\pgfsetdash{}{0pt}%
\pgfpathmoveto{\pgfqpoint{2.372408in}{1.592514in}}%
\pgfpathlineto{\pgfqpoint{2.668647in}{1.656652in}}%
\pgfusepath{stroke}%
\end{pgfscope}%
\begin{pgfscope}%
\pgfpathrectangle{\pgfqpoint{0.100000in}{0.212622in}}{\pgfqpoint{3.696000in}{3.696000in}}%
\pgfusepath{clip}%
\pgfsetrectcap%
\pgfsetroundjoin%
\pgfsetlinewidth{1.505625pt}%
\definecolor{currentstroke}{rgb}{1.000000,0.000000,0.000000}%
\pgfsetstrokecolor{currentstroke}%
\pgfsetdash{}{0pt}%
\pgfpathmoveto{\pgfqpoint{2.379025in}{1.567397in}}%
\pgfpathlineto{\pgfqpoint{2.656655in}{1.645286in}}%
\pgfusepath{stroke}%
\end{pgfscope}%
\begin{pgfscope}%
\pgfpathrectangle{\pgfqpoint{0.100000in}{0.212622in}}{\pgfqpoint{3.696000in}{3.696000in}}%
\pgfusepath{clip}%
\pgfsetrectcap%
\pgfsetroundjoin%
\pgfsetlinewidth{1.505625pt}%
\definecolor{currentstroke}{rgb}{1.000000,0.000000,0.000000}%
\pgfsetstrokecolor{currentstroke}%
\pgfsetdash{}{0pt}%
\pgfpathmoveto{\pgfqpoint{2.382675in}{1.553412in}}%
\pgfpathlineto{\pgfqpoint{2.656655in}{1.645286in}}%
\pgfusepath{stroke}%
\end{pgfscope}%
\begin{pgfscope}%
\pgfpathrectangle{\pgfqpoint{0.100000in}{0.212622in}}{\pgfqpoint{3.696000in}{3.696000in}}%
\pgfusepath{clip}%
\pgfsetrectcap%
\pgfsetroundjoin%
\pgfsetlinewidth{1.505625pt}%
\definecolor{currentstroke}{rgb}{1.000000,0.000000,0.000000}%
\pgfsetstrokecolor{currentstroke}%
\pgfsetdash{}{0pt}%
\pgfpathmoveto{\pgfqpoint{2.387391in}{1.535810in}}%
\pgfpathlineto{\pgfqpoint{2.644641in}{1.633899in}}%
\pgfusepath{stroke}%
\end{pgfscope}%
\begin{pgfscope}%
\pgfpathrectangle{\pgfqpoint{0.100000in}{0.212622in}}{\pgfqpoint{3.696000in}{3.696000in}}%
\pgfusepath{clip}%
\pgfsetrectcap%
\pgfsetroundjoin%
\pgfsetlinewidth{1.505625pt}%
\definecolor{currentstroke}{rgb}{1.000000,0.000000,0.000000}%
\pgfsetstrokecolor{currentstroke}%
\pgfsetdash{}{0pt}%
\pgfpathmoveto{\pgfqpoint{2.393234in}{1.514590in}}%
\pgfpathlineto{\pgfqpoint{2.638625in}{1.628197in}}%
\pgfusepath{stroke}%
\end{pgfscope}%
\begin{pgfscope}%
\pgfpathrectangle{\pgfqpoint{0.100000in}{0.212622in}}{\pgfqpoint{3.696000in}{3.696000in}}%
\pgfusepath{clip}%
\pgfsetrectcap%
\pgfsetroundjoin%
\pgfsetlinewidth{1.505625pt}%
\definecolor{currentstroke}{rgb}{1.000000,0.000000,0.000000}%
\pgfsetstrokecolor{currentstroke}%
\pgfsetdash{}{0pt}%
\pgfpathmoveto{\pgfqpoint{2.399867in}{1.489753in}}%
\pgfpathlineto{\pgfqpoint{2.312587in}{1.604691in}}%
\pgfusepath{stroke}%
\end{pgfscope}%
\begin{pgfscope}%
\pgfpathrectangle{\pgfqpoint{0.100000in}{0.212622in}}{\pgfqpoint{3.696000in}{3.696000in}}%
\pgfusepath{clip}%
\pgfsetrectcap%
\pgfsetroundjoin%
\pgfsetlinewidth{1.505625pt}%
\definecolor{currentstroke}{rgb}{1.000000,0.000000,0.000000}%
\pgfsetstrokecolor{currentstroke}%
\pgfsetdash{}{0pt}%
\pgfpathmoveto{\pgfqpoint{2.405736in}{1.462126in}}%
\pgfpathlineto{\pgfqpoint{2.327774in}{1.600151in}}%
\pgfusepath{stroke}%
\end{pgfscope}%
\begin{pgfscope}%
\pgfpathrectangle{\pgfqpoint{0.100000in}{0.212622in}}{\pgfqpoint{3.696000in}{3.696000in}}%
\pgfusepath{clip}%
\pgfsetrectcap%
\pgfsetroundjoin%
\pgfsetlinewidth{1.505625pt}%
\definecolor{currentstroke}{rgb}{1.000000,0.000000,0.000000}%
\pgfsetstrokecolor{currentstroke}%
\pgfsetdash{}{0pt}%
\pgfpathmoveto{\pgfqpoint{2.415073in}{1.430777in}}%
\pgfpathlineto{\pgfqpoint{2.342971in}{1.595607in}}%
\pgfusepath{stroke}%
\end{pgfscope}%
\begin{pgfscope}%
\pgfpathrectangle{\pgfqpoint{0.100000in}{0.212622in}}{\pgfqpoint{3.696000in}{3.696000in}}%
\pgfusepath{clip}%
\pgfsetrectcap%
\pgfsetroundjoin%
\pgfsetlinewidth{1.505625pt}%
\definecolor{currentstroke}{rgb}{1.000000,0.000000,0.000000}%
\pgfsetstrokecolor{currentstroke}%
\pgfsetdash{}{0pt}%
\pgfpathmoveto{\pgfqpoint{2.423471in}{1.395871in}}%
\pgfpathlineto{\pgfqpoint{2.373401in}{1.586509in}}%
\pgfusepath{stroke}%
\end{pgfscope}%
\begin{pgfscope}%
\pgfpathrectangle{\pgfqpoint{0.100000in}{0.212622in}}{\pgfqpoint{3.696000in}{3.696000in}}%
\pgfusepath{clip}%
\pgfsetrectcap%
\pgfsetroundjoin%
\pgfsetlinewidth{1.505625pt}%
\definecolor{currentstroke}{rgb}{1.000000,0.000000,0.000000}%
\pgfsetstrokecolor{currentstroke}%
\pgfsetdash{}{0pt}%
\pgfpathmoveto{\pgfqpoint{2.434279in}{1.359159in}}%
\pgfpathlineto{\pgfqpoint{2.388633in}{1.581955in}}%
\pgfusepath{stroke}%
\end{pgfscope}%
\begin{pgfscope}%
\pgfpathrectangle{\pgfqpoint{0.100000in}{0.212622in}}{\pgfqpoint{3.696000in}{3.696000in}}%
\pgfusepath{clip}%
\pgfsetrectcap%
\pgfsetroundjoin%
\pgfsetlinewidth{1.505625pt}%
\definecolor{currentstroke}{rgb}{1.000000,0.000000,0.000000}%
\pgfsetstrokecolor{currentstroke}%
\pgfsetdash{}{0pt}%
\pgfpathmoveto{\pgfqpoint{2.447144in}{1.318212in}}%
\pgfpathlineto{\pgfqpoint{2.419132in}{1.572836in}}%
\pgfusepath{stroke}%
\end{pgfscope}%
\begin{pgfscope}%
\pgfpathrectangle{\pgfqpoint{0.100000in}{0.212622in}}{\pgfqpoint{3.696000in}{3.696000in}}%
\pgfusepath{clip}%
\pgfsetrectcap%
\pgfsetroundjoin%
\pgfsetlinewidth{1.505625pt}%
\definecolor{currentstroke}{rgb}{1.000000,0.000000,0.000000}%
\pgfsetstrokecolor{currentstroke}%
\pgfsetdash{}{0pt}%
\pgfpathmoveto{\pgfqpoint{2.459923in}{1.273711in}}%
\pgfpathlineto{\pgfqpoint{2.449677in}{1.563704in}}%
\pgfusepath{stroke}%
\end{pgfscope}%
\begin{pgfscope}%
\pgfpathrectangle{\pgfqpoint{0.100000in}{0.212622in}}{\pgfqpoint{3.696000in}{3.696000in}}%
\pgfusepath{clip}%
\pgfsetrectcap%
\pgfsetroundjoin%
\pgfsetlinewidth{1.505625pt}%
\definecolor{currentstroke}{rgb}{1.000000,0.000000,0.000000}%
\pgfsetstrokecolor{currentstroke}%
\pgfsetdash{}{0pt}%
\pgfpathmoveto{\pgfqpoint{2.475233in}{1.225268in}}%
\pgfpathlineto{\pgfqpoint{2.480269in}{1.554558in}}%
\pgfusepath{stroke}%
\end{pgfscope}%
\begin{pgfscope}%
\pgfpathrectangle{\pgfqpoint{0.100000in}{0.212622in}}{\pgfqpoint{3.696000in}{3.696000in}}%
\pgfusepath{clip}%
\pgfsetrectcap%
\pgfsetroundjoin%
\pgfsetlinewidth{1.505625pt}%
\definecolor{currentstroke}{rgb}{1.000000,0.000000,0.000000}%
\pgfsetstrokecolor{currentstroke}%
\pgfsetdash{}{0pt}%
\pgfpathmoveto{\pgfqpoint{2.490334in}{1.171820in}}%
\pgfpathlineto{\pgfqpoint{2.526243in}{1.540812in}}%
\pgfusepath{stroke}%
\end{pgfscope}%
\begin{pgfscope}%
\pgfpathrectangle{\pgfqpoint{0.100000in}{0.212622in}}{\pgfqpoint{3.696000in}{3.696000in}}%
\pgfusepath{clip}%
\pgfsetrectcap%
\pgfsetroundjoin%
\pgfsetlinewidth{1.505625pt}%
\definecolor{currentstroke}{rgb}{1.000000,0.000000,0.000000}%
\pgfsetstrokecolor{currentstroke}%
\pgfsetdash{}{0pt}%
\pgfpathmoveto{\pgfqpoint{2.505404in}{1.113277in}}%
\pgfpathlineto{\pgfqpoint{2.541591in}{1.536224in}}%
\pgfusepath{stroke}%
\end{pgfscope}%
\begin{pgfscope}%
\pgfpathrectangle{\pgfqpoint{0.100000in}{0.212622in}}{\pgfqpoint{3.696000in}{3.696000in}}%
\pgfusepath{clip}%
\pgfsetrectcap%
\pgfsetroundjoin%
\pgfsetlinewidth{1.505625pt}%
\definecolor{currentstroke}{rgb}{1.000000,0.000000,0.000000}%
\pgfsetstrokecolor{currentstroke}%
\pgfsetdash{}{0pt}%
\pgfpathmoveto{\pgfqpoint{2.521718in}{1.050132in}}%
\pgfpathlineto{\pgfqpoint{2.541591in}{1.536224in}}%
\pgfusepath{stroke}%
\end{pgfscope}%
\begin{pgfscope}%
\pgfpathrectangle{\pgfqpoint{0.100000in}{0.212622in}}{\pgfqpoint{3.696000in}{3.696000in}}%
\pgfusepath{clip}%
\pgfsetrectcap%
\pgfsetroundjoin%
\pgfsetlinewidth{1.505625pt}%
\definecolor{currentstroke}{rgb}{1.000000,0.000000,0.000000}%
\pgfsetstrokecolor{currentstroke}%
\pgfsetdash{}{0pt}%
\pgfpathmoveto{\pgfqpoint{2.527446in}{1.016289in}}%
\pgfpathlineto{\pgfqpoint{2.541591in}{1.536224in}}%
\pgfusepath{stroke}%
\end{pgfscope}%
\begin{pgfscope}%
\pgfpathrectangle{\pgfqpoint{0.100000in}{0.212622in}}{\pgfqpoint{3.696000in}{3.696000in}}%
\pgfusepath{clip}%
\pgfsetrectcap%
\pgfsetroundjoin%
\pgfsetlinewidth{1.505625pt}%
\definecolor{currentstroke}{rgb}{1.000000,0.000000,0.000000}%
\pgfsetstrokecolor{currentstroke}%
\pgfsetdash{}{0pt}%
\pgfpathmoveto{\pgfqpoint{2.526094in}{0.973602in}}%
\pgfpathlineto{\pgfqpoint{2.541591in}{1.536224in}}%
\pgfusepath{stroke}%
\end{pgfscope}%
\begin{pgfscope}%
\pgfpathrectangle{\pgfqpoint{0.100000in}{0.212622in}}{\pgfqpoint{3.696000in}{3.696000in}}%
\pgfusepath{clip}%
\pgfsetrectcap%
\pgfsetroundjoin%
\pgfsetlinewidth{1.505625pt}%
\definecolor{currentstroke}{rgb}{1.000000,0.000000,0.000000}%
\pgfsetstrokecolor{currentstroke}%
\pgfsetdash{}{0pt}%
\pgfpathmoveto{\pgfqpoint{2.509847in}{0.927612in}}%
\pgfpathlineto{\pgfqpoint{2.541591in}{1.536224in}}%
\pgfusepath{stroke}%
\end{pgfscope}%
\begin{pgfscope}%
\pgfpathrectangle{\pgfqpoint{0.100000in}{0.212622in}}{\pgfqpoint{3.696000in}{3.696000in}}%
\pgfusepath{clip}%
\pgfsetrectcap%
\pgfsetroundjoin%
\pgfsetlinewidth{1.505625pt}%
\definecolor{currentstroke}{rgb}{1.000000,0.000000,0.000000}%
\pgfsetstrokecolor{currentstroke}%
\pgfsetdash{}{0pt}%
\pgfpathmoveto{\pgfqpoint{2.466514in}{0.894694in}}%
\pgfpathlineto{\pgfqpoint{2.541591in}{1.536224in}}%
\pgfusepath{stroke}%
\end{pgfscope}%
\begin{pgfscope}%
\pgfpathrectangle{\pgfqpoint{0.100000in}{0.212622in}}{\pgfqpoint{3.696000in}{3.696000in}}%
\pgfusepath{clip}%
\pgfsetrectcap%
\pgfsetroundjoin%
\pgfsetlinewidth{1.505625pt}%
\definecolor{currentstroke}{rgb}{1.000000,0.000000,0.000000}%
\pgfsetstrokecolor{currentstroke}%
\pgfsetdash{}{0pt}%
\pgfpathmoveto{\pgfqpoint{2.436381in}{0.882229in}}%
\pgfpathlineto{\pgfqpoint{2.541591in}{1.536224in}}%
\pgfusepath{stroke}%
\end{pgfscope}%
\begin{pgfscope}%
\pgfpathrectangle{\pgfqpoint{0.100000in}{0.212622in}}{\pgfqpoint{3.696000in}{3.696000in}}%
\pgfusepath{clip}%
\pgfsetrectcap%
\pgfsetroundjoin%
\pgfsetlinewidth{1.505625pt}%
\definecolor{currentstroke}{rgb}{1.000000,0.000000,0.000000}%
\pgfsetstrokecolor{currentstroke}%
\pgfsetdash{}{0pt}%
\pgfpathmoveto{\pgfqpoint{2.397767in}{0.878001in}}%
\pgfpathlineto{\pgfqpoint{2.541591in}{1.536224in}}%
\pgfusepath{stroke}%
\end{pgfscope}%
\begin{pgfscope}%
\pgfpathrectangle{\pgfqpoint{0.100000in}{0.212622in}}{\pgfqpoint{3.696000in}{3.696000in}}%
\pgfusepath{clip}%
\pgfsetrectcap%
\pgfsetroundjoin%
\pgfsetlinewidth{1.505625pt}%
\definecolor{currentstroke}{rgb}{1.000000,0.000000,0.000000}%
\pgfsetstrokecolor{currentstroke}%
\pgfsetdash{}{0pt}%
\pgfpathmoveto{\pgfqpoint{2.351594in}{0.881333in}}%
\pgfpathlineto{\pgfqpoint{2.526243in}{1.540812in}}%
\pgfusepath{stroke}%
\end{pgfscope}%
\begin{pgfscope}%
\pgfpathrectangle{\pgfqpoint{0.100000in}{0.212622in}}{\pgfqpoint{3.696000in}{3.696000in}}%
\pgfusepath{clip}%
\pgfsetrectcap%
\pgfsetroundjoin%
\pgfsetlinewidth{1.505625pt}%
\definecolor{currentstroke}{rgb}{1.000000,0.000000,0.000000}%
\pgfsetstrokecolor{currentstroke}%
\pgfsetdash{}{0pt}%
\pgfpathmoveto{\pgfqpoint{2.303027in}{0.889420in}}%
\pgfpathlineto{\pgfqpoint{2.480269in}{1.554558in}}%
\pgfusepath{stroke}%
\end{pgfscope}%
\begin{pgfscope}%
\pgfpathrectangle{\pgfqpoint{0.100000in}{0.212622in}}{\pgfqpoint{3.696000in}{3.696000in}}%
\pgfusepath{clip}%
\pgfsetrectcap%
\pgfsetroundjoin%
\pgfsetlinewidth{1.505625pt}%
\definecolor{currentstroke}{rgb}{1.000000,0.000000,0.000000}%
\pgfsetstrokecolor{currentstroke}%
\pgfsetdash{}{0pt}%
\pgfpathmoveto{\pgfqpoint{2.276290in}{0.892142in}}%
\pgfpathlineto{\pgfqpoint{2.464967in}{1.559133in}}%
\pgfusepath{stroke}%
\end{pgfscope}%
\begin{pgfscope}%
\pgfpathrectangle{\pgfqpoint{0.100000in}{0.212622in}}{\pgfqpoint{3.696000in}{3.696000in}}%
\pgfusepath{clip}%
\pgfsetrectcap%
\pgfsetroundjoin%
\pgfsetlinewidth{1.505625pt}%
\definecolor{currentstroke}{rgb}{1.000000,0.000000,0.000000}%
\pgfsetstrokecolor{currentstroke}%
\pgfsetdash{}{0pt}%
\pgfpathmoveto{\pgfqpoint{2.247446in}{0.894791in}}%
\pgfpathlineto{\pgfqpoint{2.434399in}{1.568272in}}%
\pgfusepath{stroke}%
\end{pgfscope}%
\begin{pgfscope}%
\pgfpathrectangle{\pgfqpoint{0.100000in}{0.212622in}}{\pgfqpoint{3.696000in}{3.696000in}}%
\pgfusepath{clip}%
\pgfsetrectcap%
\pgfsetroundjoin%
\pgfsetlinewidth{1.505625pt}%
\definecolor{currentstroke}{rgb}{1.000000,0.000000,0.000000}%
\pgfsetstrokecolor{currentstroke}%
\pgfsetdash{}{0pt}%
\pgfpathmoveto{\pgfqpoint{2.231769in}{0.897140in}}%
\pgfpathlineto{\pgfqpoint{2.419132in}{1.572836in}}%
\pgfusepath{stroke}%
\end{pgfscope}%
\begin{pgfscope}%
\pgfpathrectangle{\pgfqpoint{0.100000in}{0.212622in}}{\pgfqpoint{3.696000in}{3.696000in}}%
\pgfusepath{clip}%
\pgfsetrectcap%
\pgfsetroundjoin%
\pgfsetlinewidth{1.505625pt}%
\definecolor{currentstroke}{rgb}{1.000000,0.000000,0.000000}%
\pgfsetstrokecolor{currentstroke}%
\pgfsetdash{}{0pt}%
\pgfpathmoveto{\pgfqpoint{2.223125in}{0.898101in}}%
\pgfpathlineto{\pgfqpoint{2.419132in}{1.572836in}}%
\pgfusepath{stroke}%
\end{pgfscope}%
\begin{pgfscope}%
\pgfpathrectangle{\pgfqpoint{0.100000in}{0.212622in}}{\pgfqpoint{3.696000in}{3.696000in}}%
\pgfusepath{clip}%
\pgfsetrectcap%
\pgfsetroundjoin%
\pgfsetlinewidth{1.505625pt}%
\definecolor{currentstroke}{rgb}{1.000000,0.000000,0.000000}%
\pgfsetstrokecolor{currentstroke}%
\pgfsetdash{}{0pt}%
\pgfpathmoveto{\pgfqpoint{2.218437in}{0.898872in}}%
\pgfpathlineto{\pgfqpoint{2.403877in}{1.577397in}}%
\pgfusepath{stroke}%
\end{pgfscope}%
\begin{pgfscope}%
\pgfpathrectangle{\pgfqpoint{0.100000in}{0.212622in}}{\pgfqpoint{3.696000in}{3.696000in}}%
\pgfusepath{clip}%
\pgfsetrectcap%
\pgfsetroundjoin%
\pgfsetlinewidth{1.505625pt}%
\definecolor{currentstroke}{rgb}{1.000000,0.000000,0.000000}%
\pgfsetstrokecolor{currentstroke}%
\pgfsetdash{}{0pt}%
\pgfpathmoveto{\pgfqpoint{2.215865in}{0.899305in}}%
\pgfpathlineto{\pgfqpoint{2.403877in}{1.577397in}}%
\pgfusepath{stroke}%
\end{pgfscope}%
\begin{pgfscope}%
\pgfpathrectangle{\pgfqpoint{0.100000in}{0.212622in}}{\pgfqpoint{3.696000in}{3.696000in}}%
\pgfusepath{clip}%
\pgfsetrectcap%
\pgfsetroundjoin%
\pgfsetlinewidth{1.505625pt}%
\definecolor{currentstroke}{rgb}{1.000000,0.000000,0.000000}%
\pgfsetstrokecolor{currentstroke}%
\pgfsetdash{}{0pt}%
\pgfpathmoveto{\pgfqpoint{2.214465in}{0.899583in}}%
\pgfpathlineto{\pgfqpoint{2.403877in}{1.577397in}}%
\pgfusepath{stroke}%
\end{pgfscope}%
\begin{pgfscope}%
\pgfpathrectangle{\pgfqpoint{0.100000in}{0.212622in}}{\pgfqpoint{3.696000in}{3.696000in}}%
\pgfusepath{clip}%
\pgfsetrectcap%
\pgfsetroundjoin%
\pgfsetlinewidth{1.505625pt}%
\definecolor{currentstroke}{rgb}{1.000000,0.000000,0.000000}%
\pgfsetstrokecolor{currentstroke}%
\pgfsetdash{}{0pt}%
\pgfpathmoveto{\pgfqpoint{2.213695in}{0.899736in}}%
\pgfpathlineto{\pgfqpoint{2.403877in}{1.577397in}}%
\pgfusepath{stroke}%
\end{pgfscope}%
\begin{pgfscope}%
\pgfpathrectangle{\pgfqpoint{0.100000in}{0.212622in}}{\pgfqpoint{3.696000in}{3.696000in}}%
\pgfusepath{clip}%
\pgfsetrectcap%
\pgfsetroundjoin%
\pgfsetlinewidth{1.505625pt}%
\definecolor{currentstroke}{rgb}{1.000000,0.000000,0.000000}%
\pgfsetstrokecolor{currentstroke}%
\pgfsetdash{}{0pt}%
\pgfpathmoveto{\pgfqpoint{2.213273in}{0.899823in}}%
\pgfpathlineto{\pgfqpoint{2.403877in}{1.577397in}}%
\pgfusepath{stroke}%
\end{pgfscope}%
\begin{pgfscope}%
\pgfpathrectangle{\pgfqpoint{0.100000in}{0.212622in}}{\pgfqpoint{3.696000in}{3.696000in}}%
\pgfusepath{clip}%
\pgfsetrectcap%
\pgfsetroundjoin%
\pgfsetlinewidth{1.505625pt}%
\definecolor{currentstroke}{rgb}{1.000000,0.000000,0.000000}%
\pgfsetstrokecolor{currentstroke}%
\pgfsetdash{}{0pt}%
\pgfpathmoveto{\pgfqpoint{2.209237in}{0.901132in}}%
\pgfpathlineto{\pgfqpoint{2.403877in}{1.577397in}}%
\pgfusepath{stroke}%
\end{pgfscope}%
\begin{pgfscope}%
\pgfpathrectangle{\pgfqpoint{0.100000in}{0.212622in}}{\pgfqpoint{3.696000in}{3.696000in}}%
\pgfusepath{clip}%
\pgfsetrectcap%
\pgfsetroundjoin%
\pgfsetlinewidth{1.505625pt}%
\definecolor{currentstroke}{rgb}{1.000000,0.000000,0.000000}%
\pgfsetstrokecolor{currentstroke}%
\pgfsetdash{}{0pt}%
\pgfpathmoveto{\pgfqpoint{2.206944in}{0.901602in}}%
\pgfpathlineto{\pgfqpoint{2.403877in}{1.577397in}}%
\pgfusepath{stroke}%
\end{pgfscope}%
\begin{pgfscope}%
\pgfpathrectangle{\pgfqpoint{0.100000in}{0.212622in}}{\pgfqpoint{3.696000in}{3.696000in}}%
\pgfusepath{clip}%
\pgfsetrectcap%
\pgfsetroundjoin%
\pgfsetlinewidth{1.505625pt}%
\definecolor{currentstroke}{rgb}{1.000000,0.000000,0.000000}%
\pgfsetstrokecolor{currentstroke}%
\pgfsetdash{}{0pt}%
\pgfpathmoveto{\pgfqpoint{2.198170in}{0.903678in}}%
\pgfpathlineto{\pgfqpoint{2.388633in}{1.581955in}}%
\pgfusepath{stroke}%
\end{pgfscope}%
\begin{pgfscope}%
\pgfpathrectangle{\pgfqpoint{0.100000in}{0.212622in}}{\pgfqpoint{3.696000in}{3.696000in}}%
\pgfusepath{clip}%
\pgfsetrectcap%
\pgfsetroundjoin%
\pgfsetlinewidth{1.505625pt}%
\definecolor{currentstroke}{rgb}{1.000000,0.000000,0.000000}%
\pgfsetstrokecolor{currentstroke}%
\pgfsetdash{}{0pt}%
\pgfpathmoveto{\pgfqpoint{2.184668in}{0.906352in}}%
\pgfpathlineto{\pgfqpoint{2.373401in}{1.586509in}}%
\pgfusepath{stroke}%
\end{pgfscope}%
\begin{pgfscope}%
\pgfpathrectangle{\pgfqpoint{0.100000in}{0.212622in}}{\pgfqpoint{3.696000in}{3.696000in}}%
\pgfusepath{clip}%
\pgfsetrectcap%
\pgfsetroundjoin%
\pgfsetlinewidth{1.505625pt}%
\definecolor{currentstroke}{rgb}{1.000000,0.000000,0.000000}%
\pgfsetstrokecolor{currentstroke}%
\pgfsetdash{}{0pt}%
\pgfpathmoveto{\pgfqpoint{2.165681in}{0.912686in}}%
\pgfpathlineto{\pgfqpoint{2.358181in}{1.591060in}}%
\pgfusepath{stroke}%
\end{pgfscope}%
\begin{pgfscope}%
\pgfpathrectangle{\pgfqpoint{0.100000in}{0.212622in}}{\pgfqpoint{3.696000in}{3.696000in}}%
\pgfusepath{clip}%
\pgfsetrectcap%
\pgfsetroundjoin%
\pgfsetlinewidth{1.505625pt}%
\definecolor{currentstroke}{rgb}{1.000000,0.000000,0.000000}%
\pgfsetstrokecolor{currentstroke}%
\pgfsetdash{}{0pt}%
\pgfpathmoveto{\pgfqpoint{2.139231in}{0.917575in}}%
\pgfpathlineto{\pgfqpoint{2.327774in}{1.600151in}}%
\pgfusepath{stroke}%
\end{pgfscope}%
\begin{pgfscope}%
\pgfpathrectangle{\pgfqpoint{0.100000in}{0.212622in}}{\pgfqpoint{3.696000in}{3.696000in}}%
\pgfusepath{clip}%
\pgfsetrectcap%
\pgfsetroundjoin%
\pgfsetlinewidth{1.505625pt}%
\definecolor{currentstroke}{rgb}{1.000000,0.000000,0.000000}%
\pgfsetstrokecolor{currentstroke}%
\pgfsetdash{}{0pt}%
\pgfpathmoveto{\pgfqpoint{2.111546in}{0.927357in}}%
\pgfpathlineto{\pgfqpoint{2.297413in}{1.609228in}}%
\pgfusepath{stroke}%
\end{pgfscope}%
\begin{pgfscope}%
\pgfpathrectangle{\pgfqpoint{0.100000in}{0.212622in}}{\pgfqpoint{3.696000in}{3.696000in}}%
\pgfusepath{clip}%
\pgfsetrectcap%
\pgfsetroundjoin%
\pgfsetlinewidth{1.505625pt}%
\definecolor{currentstroke}{rgb}{1.000000,0.000000,0.000000}%
\pgfsetstrokecolor{currentstroke}%
\pgfsetdash{}{0pt}%
\pgfpathmoveto{\pgfqpoint{2.078524in}{0.933723in}}%
\pgfpathlineto{\pgfqpoint{2.267097in}{1.618292in}}%
\pgfusepath{stroke}%
\end{pgfscope}%
\begin{pgfscope}%
\pgfpathrectangle{\pgfqpoint{0.100000in}{0.212622in}}{\pgfqpoint{3.696000in}{3.696000in}}%
\pgfusepath{clip}%
\pgfsetrectcap%
\pgfsetroundjoin%
\pgfsetlinewidth{1.505625pt}%
\definecolor{currentstroke}{rgb}{1.000000,0.000000,0.000000}%
\pgfsetstrokecolor{currentstroke}%
\pgfsetdash{}{0pt}%
\pgfpathmoveto{\pgfqpoint{2.043809in}{0.941870in}}%
\pgfpathlineto{\pgfqpoint{2.236827in}{1.627342in}}%
\pgfusepath{stroke}%
\end{pgfscope}%
\begin{pgfscope}%
\pgfpathrectangle{\pgfqpoint{0.100000in}{0.212622in}}{\pgfqpoint{3.696000in}{3.696000in}}%
\pgfusepath{clip}%
\pgfsetrectcap%
\pgfsetroundjoin%
\pgfsetlinewidth{1.505625pt}%
\definecolor{currentstroke}{rgb}{1.000000,0.000000,0.000000}%
\pgfsetstrokecolor{currentstroke}%
\pgfsetdash{}{0pt}%
\pgfpathmoveto{\pgfqpoint{2.006314in}{0.949558in}}%
\pgfpathlineto{\pgfqpoint{2.206603in}{1.636379in}}%
\pgfusepath{stroke}%
\end{pgfscope}%
\begin{pgfscope}%
\pgfpathrectangle{\pgfqpoint{0.100000in}{0.212622in}}{\pgfqpoint{3.696000in}{3.696000in}}%
\pgfusepath{clip}%
\pgfsetrectcap%
\pgfsetroundjoin%
\pgfsetlinewidth{1.505625pt}%
\definecolor{currentstroke}{rgb}{1.000000,0.000000,0.000000}%
\pgfsetstrokecolor{currentstroke}%
\pgfsetdash{}{0pt}%
\pgfpathmoveto{\pgfqpoint{1.985986in}{0.954587in}}%
\pgfpathlineto{\pgfqpoint{2.176424in}{1.645402in}}%
\pgfusepath{stroke}%
\end{pgfscope}%
\begin{pgfscope}%
\pgfpathrectangle{\pgfqpoint{0.100000in}{0.212622in}}{\pgfqpoint{3.696000in}{3.696000in}}%
\pgfusepath{clip}%
\pgfsetrectcap%
\pgfsetroundjoin%
\pgfsetlinewidth{1.505625pt}%
\definecolor{currentstroke}{rgb}{1.000000,0.000000,0.000000}%
\pgfsetstrokecolor{currentstroke}%
\pgfsetdash{}{0pt}%
\pgfpathmoveto{\pgfqpoint{1.974703in}{0.956974in}}%
\pgfpathlineto{\pgfqpoint{2.176424in}{1.645402in}}%
\pgfusepath{stroke}%
\end{pgfscope}%
\begin{pgfscope}%
\pgfpathrectangle{\pgfqpoint{0.100000in}{0.212622in}}{\pgfqpoint{3.696000in}{3.696000in}}%
\pgfusepath{clip}%
\pgfsetrectcap%
\pgfsetroundjoin%
\pgfsetlinewidth{1.505625pt}%
\definecolor{currentstroke}{rgb}{1.000000,0.000000,0.000000}%
\pgfsetstrokecolor{currentstroke}%
\pgfsetdash{}{0pt}%
\pgfpathmoveto{\pgfqpoint{1.968519in}{0.958334in}}%
\pgfpathlineto{\pgfqpoint{2.161352in}{1.649908in}}%
\pgfusepath{stroke}%
\end{pgfscope}%
\begin{pgfscope}%
\pgfpathrectangle{\pgfqpoint{0.100000in}{0.212622in}}{\pgfqpoint{3.696000in}{3.696000in}}%
\pgfusepath{clip}%
\pgfsetrectcap%
\pgfsetroundjoin%
\pgfsetlinewidth{1.505625pt}%
\definecolor{currentstroke}{rgb}{1.000000,0.000000,0.000000}%
\pgfsetstrokecolor{currentstroke}%
\pgfsetdash{}{0pt}%
\pgfpathmoveto{\pgfqpoint{1.965138in}{0.959157in}}%
\pgfpathlineto{\pgfqpoint{2.161352in}{1.649908in}}%
\pgfusepath{stroke}%
\end{pgfscope}%
\begin{pgfscope}%
\pgfpathrectangle{\pgfqpoint{0.100000in}{0.212622in}}{\pgfqpoint{3.696000in}{3.696000in}}%
\pgfusepath{clip}%
\pgfsetrectcap%
\pgfsetroundjoin%
\pgfsetlinewidth{1.505625pt}%
\definecolor{currentstroke}{rgb}{1.000000,0.000000,0.000000}%
\pgfsetstrokecolor{currentstroke}%
\pgfsetdash{}{0pt}%
\pgfpathmoveto{\pgfqpoint{1.963275in}{0.959597in}}%
\pgfpathlineto{\pgfqpoint{2.161352in}{1.649908in}}%
\pgfusepath{stroke}%
\end{pgfscope}%
\begin{pgfscope}%
\pgfpathrectangle{\pgfqpoint{0.100000in}{0.212622in}}{\pgfqpoint{3.696000in}{3.696000in}}%
\pgfusepath{clip}%
\pgfsetrectcap%
\pgfsetroundjoin%
\pgfsetlinewidth{1.505625pt}%
\definecolor{currentstroke}{rgb}{1.000000,0.000000,0.000000}%
\pgfsetstrokecolor{currentstroke}%
\pgfsetdash{}{0pt}%
\pgfpathmoveto{\pgfqpoint{1.962258in}{0.959865in}}%
\pgfpathlineto{\pgfqpoint{2.161352in}{1.649908in}}%
\pgfusepath{stroke}%
\end{pgfscope}%
\begin{pgfscope}%
\pgfpathrectangle{\pgfqpoint{0.100000in}{0.212622in}}{\pgfqpoint{3.696000in}{3.696000in}}%
\pgfusepath{clip}%
\pgfsetrectcap%
\pgfsetroundjoin%
\pgfsetlinewidth{1.505625pt}%
\definecolor{currentstroke}{rgb}{1.000000,0.000000,0.000000}%
\pgfsetstrokecolor{currentstroke}%
\pgfsetdash{}{0pt}%
\pgfpathmoveto{\pgfqpoint{1.959004in}{0.960633in}}%
\pgfpathlineto{\pgfqpoint{2.161352in}{1.649908in}}%
\pgfusepath{stroke}%
\end{pgfscope}%
\begin{pgfscope}%
\pgfpathrectangle{\pgfqpoint{0.100000in}{0.212622in}}{\pgfqpoint{3.696000in}{3.696000in}}%
\pgfusepath{clip}%
\pgfsetrectcap%
\pgfsetroundjoin%
\pgfsetlinewidth{1.505625pt}%
\definecolor{currentstroke}{rgb}{1.000000,0.000000,0.000000}%
\pgfsetstrokecolor{currentstroke}%
\pgfsetdash{}{0pt}%
\pgfpathmoveto{\pgfqpoint{1.953881in}{0.962071in}}%
\pgfpathlineto{\pgfqpoint{2.146291in}{1.654411in}}%
\pgfusepath{stroke}%
\end{pgfscope}%
\begin{pgfscope}%
\pgfpathrectangle{\pgfqpoint{0.100000in}{0.212622in}}{\pgfqpoint{3.696000in}{3.696000in}}%
\pgfusepath{clip}%
\pgfsetrectcap%
\pgfsetroundjoin%
\pgfsetlinewidth{1.505625pt}%
\definecolor{currentstroke}{rgb}{1.000000,0.000000,0.000000}%
\pgfsetstrokecolor{currentstroke}%
\pgfsetdash{}{0pt}%
\pgfpathmoveto{\pgfqpoint{1.944968in}{0.963753in}}%
\pgfpathlineto{\pgfqpoint{2.146291in}{1.654411in}}%
\pgfusepath{stroke}%
\end{pgfscope}%
\begin{pgfscope}%
\pgfpathrectangle{\pgfqpoint{0.100000in}{0.212622in}}{\pgfqpoint{3.696000in}{3.696000in}}%
\pgfusepath{clip}%
\pgfsetrectcap%
\pgfsetroundjoin%
\pgfsetlinewidth{1.505625pt}%
\definecolor{currentstroke}{rgb}{1.000000,0.000000,0.000000}%
\pgfsetstrokecolor{currentstroke}%
\pgfsetdash{}{0pt}%
\pgfpathmoveto{\pgfqpoint{1.932352in}{0.966785in}}%
\pgfpathlineto{\pgfqpoint{2.131241in}{1.658911in}}%
\pgfusepath{stroke}%
\end{pgfscope}%
\begin{pgfscope}%
\pgfpathrectangle{\pgfqpoint{0.100000in}{0.212622in}}{\pgfqpoint{3.696000in}{3.696000in}}%
\pgfusepath{clip}%
\pgfsetrectcap%
\pgfsetroundjoin%
\pgfsetlinewidth{1.505625pt}%
\definecolor{currentstroke}{rgb}{1.000000,0.000000,0.000000}%
\pgfsetstrokecolor{currentstroke}%
\pgfsetdash{}{0pt}%
\pgfpathmoveto{\pgfqpoint{1.916437in}{0.971078in}}%
\pgfpathlineto{\pgfqpoint{2.116202in}{1.663407in}}%
\pgfusepath{stroke}%
\end{pgfscope}%
\begin{pgfscope}%
\pgfpathrectangle{\pgfqpoint{0.100000in}{0.212622in}}{\pgfqpoint{3.696000in}{3.696000in}}%
\pgfusepath{clip}%
\pgfsetrectcap%
\pgfsetroundjoin%
\pgfsetlinewidth{1.505625pt}%
\definecolor{currentstroke}{rgb}{1.000000,0.000000,0.000000}%
\pgfsetstrokecolor{currentstroke}%
\pgfsetdash{}{0pt}%
\pgfpathmoveto{\pgfqpoint{1.907694in}{0.973410in}}%
\pgfpathlineto{\pgfqpoint{2.101175in}{1.667900in}}%
\pgfusepath{stroke}%
\end{pgfscope}%
\begin{pgfscope}%
\pgfpathrectangle{\pgfqpoint{0.100000in}{0.212622in}}{\pgfqpoint{3.696000in}{3.696000in}}%
\pgfusepath{clip}%
\pgfsetrectcap%
\pgfsetroundjoin%
\pgfsetlinewidth{1.505625pt}%
\definecolor{currentstroke}{rgb}{1.000000,0.000000,0.000000}%
\pgfsetstrokecolor{currentstroke}%
\pgfsetdash{}{0pt}%
\pgfpathmoveto{\pgfqpoint{1.902801in}{0.974443in}}%
\pgfpathlineto{\pgfqpoint{2.101175in}{1.667900in}}%
\pgfusepath{stroke}%
\end{pgfscope}%
\begin{pgfscope}%
\pgfpathrectangle{\pgfqpoint{0.100000in}{0.212622in}}{\pgfqpoint{3.696000in}{3.696000in}}%
\pgfusepath{clip}%
\pgfsetrectcap%
\pgfsetroundjoin%
\pgfsetlinewidth{1.505625pt}%
\definecolor{currentstroke}{rgb}{1.000000,0.000000,0.000000}%
\pgfsetstrokecolor{currentstroke}%
\pgfsetdash{}{0pt}%
\pgfpathmoveto{\pgfqpoint{1.900091in}{0.974944in}}%
\pgfpathlineto{\pgfqpoint{2.101175in}{1.667900in}}%
\pgfusepath{stroke}%
\end{pgfscope}%
\begin{pgfscope}%
\pgfpathrectangle{\pgfqpoint{0.100000in}{0.212622in}}{\pgfqpoint{3.696000in}{3.696000in}}%
\pgfusepath{clip}%
\pgfsetrectcap%
\pgfsetroundjoin%
\pgfsetlinewidth{1.505625pt}%
\definecolor{currentstroke}{rgb}{1.000000,0.000000,0.000000}%
\pgfsetstrokecolor{currentstroke}%
\pgfsetdash{}{0pt}%
\pgfpathmoveto{\pgfqpoint{1.894738in}{0.976573in}}%
\pgfpathlineto{\pgfqpoint{2.086159in}{1.672389in}}%
\pgfusepath{stroke}%
\end{pgfscope}%
\begin{pgfscope}%
\pgfpathrectangle{\pgfqpoint{0.100000in}{0.212622in}}{\pgfqpoint{3.696000in}{3.696000in}}%
\pgfusepath{clip}%
\pgfsetrectcap%
\pgfsetroundjoin%
\pgfsetlinewidth{1.505625pt}%
\definecolor{currentstroke}{rgb}{1.000000,0.000000,0.000000}%
\pgfsetstrokecolor{currentstroke}%
\pgfsetdash{}{0pt}%
\pgfpathmoveto{\pgfqpoint{1.885241in}{0.978627in}}%
\pgfpathlineto{\pgfqpoint{2.086159in}{1.672389in}}%
\pgfusepath{stroke}%
\end{pgfscope}%
\begin{pgfscope}%
\pgfpathrectangle{\pgfqpoint{0.100000in}{0.212622in}}{\pgfqpoint{3.696000in}{3.696000in}}%
\pgfusepath{clip}%
\pgfsetrectcap%
\pgfsetroundjoin%
\pgfsetlinewidth{1.505625pt}%
\definecolor{currentstroke}{rgb}{1.000000,0.000000,0.000000}%
\pgfsetstrokecolor{currentstroke}%
\pgfsetdash{}{0pt}%
\pgfpathmoveto{\pgfqpoint{1.870696in}{0.982247in}}%
\pgfpathlineto{\pgfqpoint{2.071155in}{1.676875in}}%
\pgfusepath{stroke}%
\end{pgfscope}%
\begin{pgfscope}%
\pgfpathrectangle{\pgfqpoint{0.100000in}{0.212622in}}{\pgfqpoint{3.696000in}{3.696000in}}%
\pgfusepath{clip}%
\pgfsetrectcap%
\pgfsetroundjoin%
\pgfsetlinewidth{1.505625pt}%
\definecolor{currentstroke}{rgb}{1.000000,0.000000,0.000000}%
\pgfsetstrokecolor{currentstroke}%
\pgfsetdash{}{0pt}%
\pgfpathmoveto{\pgfqpoint{1.850827in}{0.986956in}}%
\pgfpathlineto{\pgfqpoint{2.041179in}{1.685837in}}%
\pgfusepath{stroke}%
\end{pgfscope}%
\begin{pgfscope}%
\pgfpathrectangle{\pgfqpoint{0.100000in}{0.212622in}}{\pgfqpoint{3.696000in}{3.696000in}}%
\pgfusepath{clip}%
\pgfsetrectcap%
\pgfsetroundjoin%
\pgfsetlinewidth{1.505625pt}%
\definecolor{currentstroke}{rgb}{1.000000,0.000000,0.000000}%
\pgfsetstrokecolor{currentstroke}%
\pgfsetdash{}{0pt}%
\pgfpathmoveto{\pgfqpoint{1.828583in}{0.992028in}}%
\pgfpathlineto{\pgfqpoint{2.026208in}{1.690314in}}%
\pgfusepath{stroke}%
\end{pgfscope}%
\begin{pgfscope}%
\pgfpathrectangle{\pgfqpoint{0.100000in}{0.212622in}}{\pgfqpoint{3.696000in}{3.696000in}}%
\pgfusepath{clip}%
\pgfsetrectcap%
\pgfsetroundjoin%
\pgfsetlinewidth{1.505625pt}%
\definecolor{currentstroke}{rgb}{1.000000,0.000000,0.000000}%
\pgfsetstrokecolor{currentstroke}%
\pgfsetdash{}{0pt}%
\pgfpathmoveto{\pgfqpoint{1.816263in}{0.994600in}}%
\pgfpathlineto{\pgfqpoint{2.011248in}{1.694786in}}%
\pgfusepath{stroke}%
\end{pgfscope}%
\begin{pgfscope}%
\pgfpathrectangle{\pgfqpoint{0.100000in}{0.212622in}}{\pgfqpoint{3.696000in}{3.696000in}}%
\pgfusepath{clip}%
\pgfsetrectcap%
\pgfsetroundjoin%
\pgfsetlinewidth{1.505625pt}%
\definecolor{currentstroke}{rgb}{1.000000,0.000000,0.000000}%
\pgfsetstrokecolor{currentstroke}%
\pgfsetdash{}{0pt}%
\pgfpathmoveto{\pgfqpoint{1.809490in}{0.996012in}}%
\pgfpathlineto{\pgfqpoint{2.011248in}{1.694786in}}%
\pgfusepath{stroke}%
\end{pgfscope}%
\begin{pgfscope}%
\pgfpathrectangle{\pgfqpoint{0.100000in}{0.212622in}}{\pgfqpoint{3.696000in}{3.696000in}}%
\pgfusepath{clip}%
\pgfsetrectcap%
\pgfsetroundjoin%
\pgfsetlinewidth{1.505625pt}%
\definecolor{currentstroke}{rgb}{1.000000,0.000000,0.000000}%
\pgfsetstrokecolor{currentstroke}%
\pgfsetdash{}{0pt}%
\pgfpathmoveto{\pgfqpoint{1.797805in}{0.998197in}}%
\pgfpathlineto{\pgfqpoint{1.996299in}{1.699256in}}%
\pgfusepath{stroke}%
\end{pgfscope}%
\begin{pgfscope}%
\pgfpathrectangle{\pgfqpoint{0.100000in}{0.212622in}}{\pgfqpoint{3.696000in}{3.696000in}}%
\pgfusepath{clip}%
\pgfsetrectcap%
\pgfsetroundjoin%
\pgfsetlinewidth{1.505625pt}%
\definecolor{currentstroke}{rgb}{1.000000,0.000000,0.000000}%
\pgfsetstrokecolor{currentstroke}%
\pgfsetdash{}{0pt}%
\pgfpathmoveto{\pgfqpoint{1.780638in}{1.000566in}}%
\pgfpathlineto{\pgfqpoint{1.981362in}{1.703722in}}%
\pgfusepath{stroke}%
\end{pgfscope}%
\begin{pgfscope}%
\pgfpathrectangle{\pgfqpoint{0.100000in}{0.212622in}}{\pgfqpoint{3.696000in}{3.696000in}}%
\pgfusepath{clip}%
\pgfsetrectcap%
\pgfsetroundjoin%
\pgfsetlinewidth{1.505625pt}%
\definecolor{currentstroke}{rgb}{1.000000,0.000000,0.000000}%
\pgfsetstrokecolor{currentstroke}%
\pgfsetdash{}{0pt}%
\pgfpathmoveto{\pgfqpoint{1.771328in}{1.002306in}}%
\pgfpathlineto{\pgfqpoint{1.966436in}{1.708184in}}%
\pgfusepath{stroke}%
\end{pgfscope}%
\begin{pgfscope}%
\pgfpathrectangle{\pgfqpoint{0.100000in}{0.212622in}}{\pgfqpoint{3.696000in}{3.696000in}}%
\pgfusepath{clip}%
\pgfsetrectcap%
\pgfsetroundjoin%
\pgfsetlinewidth{1.505625pt}%
\definecolor{currentstroke}{rgb}{1.000000,0.000000,0.000000}%
\pgfsetstrokecolor{currentstroke}%
\pgfsetdash{}{0pt}%
\pgfpathmoveto{\pgfqpoint{1.766246in}{1.003408in}}%
\pgfpathlineto{\pgfqpoint{1.966436in}{1.708184in}}%
\pgfusepath{stroke}%
\end{pgfscope}%
\begin{pgfscope}%
\pgfpathrectangle{\pgfqpoint{0.100000in}{0.212622in}}{\pgfqpoint{3.696000in}{3.696000in}}%
\pgfusepath{clip}%
\pgfsetrectcap%
\pgfsetroundjoin%
\pgfsetlinewidth{1.505625pt}%
\definecolor{currentstroke}{rgb}{1.000000,0.000000,0.000000}%
\pgfsetstrokecolor{currentstroke}%
\pgfsetdash{}{0pt}%
\pgfpathmoveto{\pgfqpoint{1.763479in}{1.004078in}}%
\pgfpathlineto{\pgfqpoint{1.966436in}{1.708184in}}%
\pgfusepath{stroke}%
\end{pgfscope}%
\begin{pgfscope}%
\pgfpathrectangle{\pgfqpoint{0.100000in}{0.212622in}}{\pgfqpoint{3.696000in}{3.696000in}}%
\pgfusepath{clip}%
\pgfsetrectcap%
\pgfsetroundjoin%
\pgfsetlinewidth{1.505625pt}%
\definecolor{currentstroke}{rgb}{1.000000,0.000000,0.000000}%
\pgfsetstrokecolor{currentstroke}%
\pgfsetdash{}{0pt}%
\pgfpathmoveto{\pgfqpoint{1.755571in}{1.006170in}}%
\pgfpathlineto{\pgfqpoint{1.951521in}{1.712644in}}%
\pgfusepath{stroke}%
\end{pgfscope}%
\begin{pgfscope}%
\pgfpathrectangle{\pgfqpoint{0.100000in}{0.212622in}}{\pgfqpoint{3.696000in}{3.696000in}}%
\pgfusepath{clip}%
\pgfsetrectcap%
\pgfsetroundjoin%
\pgfsetlinewidth{1.505625pt}%
\definecolor{currentstroke}{rgb}{1.000000,0.000000,0.000000}%
\pgfsetstrokecolor{currentstroke}%
\pgfsetdash{}{0pt}%
\pgfpathmoveto{\pgfqpoint{1.740948in}{1.009530in}}%
\pgfpathlineto{\pgfqpoint{1.936617in}{1.717100in}}%
\pgfusepath{stroke}%
\end{pgfscope}%
\begin{pgfscope}%
\pgfpathrectangle{\pgfqpoint{0.100000in}{0.212622in}}{\pgfqpoint{3.696000in}{3.696000in}}%
\pgfusepath{clip}%
\pgfsetrectcap%
\pgfsetroundjoin%
\pgfsetlinewidth{1.505625pt}%
\definecolor{currentstroke}{rgb}{1.000000,0.000000,0.000000}%
\pgfsetstrokecolor{currentstroke}%
\pgfsetdash{}{0pt}%
\pgfpathmoveto{\pgfqpoint{1.724464in}{1.013513in}}%
\pgfpathlineto{\pgfqpoint{1.921724in}{1.721552in}}%
\pgfusepath{stroke}%
\end{pgfscope}%
\begin{pgfscope}%
\pgfpathrectangle{\pgfqpoint{0.100000in}{0.212622in}}{\pgfqpoint{3.696000in}{3.696000in}}%
\pgfusepath{clip}%
\pgfsetrectcap%
\pgfsetroundjoin%
\pgfsetlinewidth{1.505625pt}%
\definecolor{currentstroke}{rgb}{1.000000,0.000000,0.000000}%
\pgfsetstrokecolor{currentstroke}%
\pgfsetdash{}{0pt}%
\pgfpathmoveto{\pgfqpoint{1.715233in}{1.015154in}}%
\pgfpathlineto{\pgfqpoint{1.921724in}{1.721552in}}%
\pgfusepath{stroke}%
\end{pgfscope}%
\begin{pgfscope}%
\pgfpathrectangle{\pgfqpoint{0.100000in}{0.212622in}}{\pgfqpoint{3.696000in}{3.696000in}}%
\pgfusepath{clip}%
\pgfsetrectcap%
\pgfsetroundjoin%
\pgfsetlinewidth{1.505625pt}%
\definecolor{currentstroke}{rgb}{1.000000,0.000000,0.000000}%
\pgfsetstrokecolor{currentstroke}%
\pgfsetdash{}{0pt}%
\pgfpathmoveto{\pgfqpoint{1.710160in}{1.016061in}}%
\pgfpathlineto{\pgfqpoint{1.906842in}{1.726002in}}%
\pgfusepath{stroke}%
\end{pgfscope}%
\begin{pgfscope}%
\pgfpathrectangle{\pgfqpoint{0.100000in}{0.212622in}}{\pgfqpoint{3.696000in}{3.696000in}}%
\pgfusepath{clip}%
\pgfsetrectcap%
\pgfsetroundjoin%
\pgfsetlinewidth{1.505625pt}%
\definecolor{currentstroke}{rgb}{1.000000,0.000000,0.000000}%
\pgfsetstrokecolor{currentstroke}%
\pgfsetdash{}{0pt}%
\pgfpathmoveto{\pgfqpoint{1.697167in}{1.019045in}}%
\pgfpathlineto{\pgfqpoint{1.891971in}{1.730448in}}%
\pgfusepath{stroke}%
\end{pgfscope}%
\begin{pgfscope}%
\pgfpathrectangle{\pgfqpoint{0.100000in}{0.212622in}}{\pgfqpoint{3.696000in}{3.696000in}}%
\pgfusepath{clip}%
\pgfsetrectcap%
\pgfsetroundjoin%
\pgfsetlinewidth{1.505625pt}%
\definecolor{currentstroke}{rgb}{1.000000,0.000000,0.000000}%
\pgfsetstrokecolor{currentstroke}%
\pgfsetdash{}{0pt}%
\pgfpathmoveto{\pgfqpoint{1.676854in}{1.022712in}}%
\pgfpathlineto{\pgfqpoint{1.877112in}{1.734891in}}%
\pgfusepath{stroke}%
\end{pgfscope}%
\begin{pgfscope}%
\pgfpathrectangle{\pgfqpoint{0.100000in}{0.212622in}}{\pgfqpoint{3.696000in}{3.696000in}}%
\pgfusepath{clip}%
\pgfsetrectcap%
\pgfsetroundjoin%
\pgfsetlinewidth{1.505625pt}%
\definecolor{currentstroke}{rgb}{1.000000,0.000000,0.000000}%
\pgfsetstrokecolor{currentstroke}%
\pgfsetdash{}{0pt}%
\pgfpathmoveto{\pgfqpoint{1.653382in}{1.026622in}}%
\pgfpathlineto{\pgfqpoint{1.862263in}{1.739330in}}%
\pgfusepath{stroke}%
\end{pgfscope}%
\begin{pgfscope}%
\pgfpathrectangle{\pgfqpoint{0.100000in}{0.212622in}}{\pgfqpoint{3.696000in}{3.696000in}}%
\pgfusepath{clip}%
\pgfsetrectcap%
\pgfsetroundjoin%
\pgfsetlinewidth{1.505625pt}%
\definecolor{currentstroke}{rgb}{1.000000,0.000000,0.000000}%
\pgfsetstrokecolor{currentstroke}%
\pgfsetdash{}{0pt}%
\pgfpathmoveto{\pgfqpoint{1.625921in}{1.031510in}}%
\pgfpathlineto{\pgfqpoint{1.832599in}{1.748199in}}%
\pgfusepath{stroke}%
\end{pgfscope}%
\begin{pgfscope}%
\pgfpathrectangle{\pgfqpoint{0.100000in}{0.212622in}}{\pgfqpoint{3.696000in}{3.696000in}}%
\pgfusepath{clip}%
\pgfsetrectcap%
\pgfsetroundjoin%
\pgfsetlinewidth{1.505625pt}%
\definecolor{currentstroke}{rgb}{1.000000,0.000000,0.000000}%
\pgfsetstrokecolor{currentstroke}%
\pgfsetdash{}{0pt}%
\pgfpathmoveto{\pgfqpoint{1.610722in}{1.033804in}}%
\pgfpathlineto{\pgfqpoint{1.817784in}{1.752628in}}%
\pgfusepath{stroke}%
\end{pgfscope}%
\begin{pgfscope}%
\pgfpathrectangle{\pgfqpoint{0.100000in}{0.212622in}}{\pgfqpoint{3.696000in}{3.696000in}}%
\pgfusepath{clip}%
\pgfsetrectcap%
\pgfsetroundjoin%
\pgfsetlinewidth{1.505625pt}%
\definecolor{currentstroke}{rgb}{1.000000,0.000000,0.000000}%
\pgfsetstrokecolor{currentstroke}%
\pgfsetdash{}{0pt}%
\pgfpathmoveto{\pgfqpoint{1.602429in}{1.035334in}}%
\pgfpathlineto{\pgfqpoint{1.802980in}{1.757055in}}%
\pgfusepath{stroke}%
\end{pgfscope}%
\begin{pgfscope}%
\pgfpathrectangle{\pgfqpoint{0.100000in}{0.212622in}}{\pgfqpoint{3.696000in}{3.696000in}}%
\pgfusepath{clip}%
\pgfsetrectcap%
\pgfsetroundjoin%
\pgfsetlinewidth{1.505625pt}%
\definecolor{currentstroke}{rgb}{1.000000,0.000000,0.000000}%
\pgfsetstrokecolor{currentstroke}%
\pgfsetdash{}{0pt}%
\pgfpathmoveto{\pgfqpoint{1.587689in}{1.037702in}}%
\pgfpathlineto{\pgfqpoint{1.788186in}{1.761478in}}%
\pgfusepath{stroke}%
\end{pgfscope}%
\begin{pgfscope}%
\pgfpathrectangle{\pgfqpoint{0.100000in}{0.212622in}}{\pgfqpoint{3.696000in}{3.696000in}}%
\pgfusepath{clip}%
\pgfsetrectcap%
\pgfsetroundjoin%
\pgfsetlinewidth{1.505625pt}%
\definecolor{currentstroke}{rgb}{1.000000,0.000000,0.000000}%
\pgfsetstrokecolor{currentstroke}%
\pgfsetdash{}{0pt}%
\pgfpathmoveto{\pgfqpoint{1.565668in}{1.042312in}}%
\pgfpathlineto{\pgfqpoint{1.773404in}{1.765897in}}%
\pgfusepath{stroke}%
\end{pgfscope}%
\begin{pgfscope}%
\pgfpathrectangle{\pgfqpoint{0.100000in}{0.212622in}}{\pgfqpoint{3.696000in}{3.696000in}}%
\pgfusepath{clip}%
\pgfsetrectcap%
\pgfsetroundjoin%
\pgfsetlinewidth{1.505625pt}%
\definecolor{currentstroke}{rgb}{1.000000,0.000000,0.000000}%
\pgfsetstrokecolor{currentstroke}%
\pgfsetdash{}{0pt}%
\pgfpathmoveto{\pgfqpoint{1.538340in}{1.046687in}}%
\pgfpathlineto{\pgfqpoint{1.743873in}{1.774727in}}%
\pgfusepath{stroke}%
\end{pgfscope}%
\begin{pgfscope}%
\pgfpathrectangle{\pgfqpoint{0.100000in}{0.212622in}}{\pgfqpoint{3.696000in}{3.696000in}}%
\pgfusepath{clip}%
\pgfsetrectcap%
\pgfsetroundjoin%
\pgfsetlinewidth{1.505625pt}%
\definecolor{currentstroke}{rgb}{1.000000,0.000000,0.000000}%
\pgfsetstrokecolor{currentstroke}%
\pgfsetdash{}{0pt}%
\pgfpathmoveto{\pgfqpoint{1.523390in}{1.049268in}}%
\pgfpathlineto{\pgfqpoint{1.729123in}{1.779136in}}%
\pgfusepath{stroke}%
\end{pgfscope}%
\begin{pgfscope}%
\pgfpathrectangle{\pgfqpoint{0.100000in}{0.212622in}}{\pgfqpoint{3.696000in}{3.696000in}}%
\pgfusepath{clip}%
\pgfsetrectcap%
\pgfsetroundjoin%
\pgfsetlinewidth{1.505625pt}%
\definecolor{currentstroke}{rgb}{1.000000,0.000000,0.000000}%
\pgfsetstrokecolor{currentstroke}%
\pgfsetdash{}{0pt}%
\pgfpathmoveto{\pgfqpoint{1.515218in}{1.050947in}}%
\pgfpathlineto{\pgfqpoint{1.729123in}{1.779136in}}%
\pgfusepath{stroke}%
\end{pgfscope}%
\begin{pgfscope}%
\pgfpathrectangle{\pgfqpoint{0.100000in}{0.212622in}}{\pgfqpoint{3.696000in}{3.696000in}}%
\pgfusepath{clip}%
\pgfsetrectcap%
\pgfsetroundjoin%
\pgfsetlinewidth{1.505625pt}%
\definecolor{currentstroke}{rgb}{1.000000,0.000000,0.000000}%
\pgfsetstrokecolor{currentstroke}%
\pgfsetdash{}{0pt}%
\pgfpathmoveto{\pgfqpoint{1.504688in}{1.052244in}}%
\pgfpathlineto{\pgfqpoint{1.714385in}{1.783543in}}%
\pgfusepath{stroke}%
\end{pgfscope}%
\begin{pgfscope}%
\pgfpathrectangle{\pgfqpoint{0.100000in}{0.212622in}}{\pgfqpoint{3.696000in}{3.696000in}}%
\pgfusepath{clip}%
\pgfsetrectcap%
\pgfsetroundjoin%
\pgfsetlinewidth{1.505625pt}%
\definecolor{currentstroke}{rgb}{1.000000,0.000000,0.000000}%
\pgfsetstrokecolor{currentstroke}%
\pgfsetdash{}{0pt}%
\pgfpathmoveto{\pgfqpoint{1.487238in}{1.056664in}}%
\pgfpathlineto{\pgfqpoint{1.699657in}{1.787946in}}%
\pgfusepath{stroke}%
\end{pgfscope}%
\begin{pgfscope}%
\pgfpathrectangle{\pgfqpoint{0.100000in}{0.212622in}}{\pgfqpoint{3.696000in}{3.696000in}}%
\pgfusepath{clip}%
\pgfsetrectcap%
\pgfsetroundjoin%
\pgfsetlinewidth{1.505625pt}%
\definecolor{currentstroke}{rgb}{1.000000,0.000000,0.000000}%
\pgfsetstrokecolor{currentstroke}%
\pgfsetdash{}{0pt}%
\pgfpathmoveto{\pgfqpoint{1.465529in}{1.060293in}}%
\pgfpathlineto{\pgfqpoint{1.670235in}{1.796743in}}%
\pgfusepath{stroke}%
\end{pgfscope}%
\begin{pgfscope}%
\pgfpathrectangle{\pgfqpoint{0.100000in}{0.212622in}}{\pgfqpoint{3.696000in}{3.696000in}}%
\pgfusepath{clip}%
\pgfsetrectcap%
\pgfsetroundjoin%
\pgfsetlinewidth{1.505625pt}%
\definecolor{currentstroke}{rgb}{1.000000,0.000000,0.000000}%
\pgfsetstrokecolor{currentstroke}%
\pgfsetdash{}{0pt}%
\pgfpathmoveto{\pgfqpoint{1.441126in}{1.063984in}}%
\pgfpathlineto{\pgfqpoint{1.655541in}{1.801136in}}%
\pgfusepath{stroke}%
\end{pgfscope}%
\begin{pgfscope}%
\pgfpathrectangle{\pgfqpoint{0.100000in}{0.212622in}}{\pgfqpoint{3.696000in}{3.696000in}}%
\pgfusepath{clip}%
\pgfsetrectcap%
\pgfsetroundjoin%
\pgfsetlinewidth{1.505625pt}%
\definecolor{currentstroke}{rgb}{1.000000,0.000000,0.000000}%
\pgfsetstrokecolor{currentstroke}%
\pgfsetdash{}{0pt}%
\pgfpathmoveto{\pgfqpoint{1.428033in}{1.067279in}}%
\pgfpathlineto{\pgfqpoint{1.640857in}{1.805526in}}%
\pgfusepath{stroke}%
\end{pgfscope}%
\begin{pgfscope}%
\pgfpathrectangle{\pgfqpoint{0.100000in}{0.212622in}}{\pgfqpoint{3.696000in}{3.696000in}}%
\pgfusepath{clip}%
\pgfsetrectcap%
\pgfsetroundjoin%
\pgfsetlinewidth{1.505625pt}%
\definecolor{currentstroke}{rgb}{1.000000,0.000000,0.000000}%
\pgfsetstrokecolor{currentstroke}%
\pgfsetdash{}{0pt}%
\pgfpathmoveto{\pgfqpoint{1.412312in}{1.070112in}}%
\pgfpathlineto{\pgfqpoint{1.626184in}{1.809913in}}%
\pgfusepath{stroke}%
\end{pgfscope}%
\begin{pgfscope}%
\pgfpathrectangle{\pgfqpoint{0.100000in}{0.212622in}}{\pgfqpoint{3.696000in}{3.696000in}}%
\pgfusepath{clip}%
\pgfsetrectcap%
\pgfsetroundjoin%
\pgfsetlinewidth{1.505625pt}%
\definecolor{currentstroke}{rgb}{1.000000,0.000000,0.000000}%
\pgfsetstrokecolor{currentstroke}%
\pgfsetdash{}{0pt}%
\pgfpathmoveto{\pgfqpoint{1.392397in}{1.075592in}}%
\pgfpathlineto{\pgfqpoint{1.611522in}{1.814297in}}%
\pgfusepath{stroke}%
\end{pgfscope}%
\begin{pgfscope}%
\pgfpathrectangle{\pgfqpoint{0.100000in}{0.212622in}}{\pgfqpoint{3.696000in}{3.696000in}}%
\pgfusepath{clip}%
\pgfsetrectcap%
\pgfsetroundjoin%
\pgfsetlinewidth{1.505625pt}%
\definecolor{currentstroke}{rgb}{1.000000,0.000000,0.000000}%
\pgfsetstrokecolor{currentstroke}%
\pgfsetdash{}{0pt}%
\pgfpathmoveto{\pgfqpoint{1.365671in}{1.080974in}}%
\pgfpathlineto{\pgfqpoint{1.582230in}{1.823055in}}%
\pgfusepath{stroke}%
\end{pgfscope}%
\begin{pgfscope}%
\pgfpathrectangle{\pgfqpoint{0.100000in}{0.212622in}}{\pgfqpoint{3.696000in}{3.696000in}}%
\pgfusepath{clip}%
\pgfsetrectcap%
\pgfsetroundjoin%
\pgfsetlinewidth{1.505625pt}%
\definecolor{currentstroke}{rgb}{1.000000,0.000000,0.000000}%
\pgfsetstrokecolor{currentstroke}%
\pgfsetdash{}{0pt}%
\pgfpathmoveto{\pgfqpoint{1.337506in}{1.087880in}}%
\pgfpathlineto{\pgfqpoint{1.552982in}{1.831799in}}%
\pgfusepath{stroke}%
\end{pgfscope}%
\begin{pgfscope}%
\pgfpathrectangle{\pgfqpoint{0.100000in}{0.212622in}}{\pgfqpoint{3.696000in}{3.696000in}}%
\pgfusepath{clip}%
\pgfsetrectcap%
\pgfsetroundjoin%
\pgfsetlinewidth{1.505625pt}%
\definecolor{currentstroke}{rgb}{1.000000,0.000000,0.000000}%
\pgfsetstrokecolor{currentstroke}%
\pgfsetdash{}{0pt}%
\pgfpathmoveto{\pgfqpoint{1.322066in}{1.091878in}}%
\pgfpathlineto{\pgfqpoint{1.538374in}{1.836167in}}%
\pgfusepath{stroke}%
\end{pgfscope}%
\begin{pgfscope}%
\pgfpathrectangle{\pgfqpoint{0.100000in}{0.212622in}}{\pgfqpoint{3.696000in}{3.696000in}}%
\pgfusepath{clip}%
\pgfsetrectcap%
\pgfsetroundjoin%
\pgfsetlinewidth{1.505625pt}%
\definecolor{currentstroke}{rgb}{1.000000,0.000000,0.000000}%
\pgfsetstrokecolor{currentstroke}%
\pgfsetdash{}{0pt}%
\pgfpathmoveto{\pgfqpoint{1.304052in}{1.095654in}}%
\pgfpathlineto{\pgfqpoint{1.523777in}{1.840531in}}%
\pgfusepath{stroke}%
\end{pgfscope}%
\begin{pgfscope}%
\pgfpathrectangle{\pgfqpoint{0.100000in}{0.212622in}}{\pgfqpoint{3.696000in}{3.696000in}}%
\pgfusepath{clip}%
\pgfsetrectcap%
\pgfsetroundjoin%
\pgfsetlinewidth{1.505625pt}%
\definecolor{currentstroke}{rgb}{1.000000,0.000000,0.000000}%
\pgfsetstrokecolor{currentstroke}%
\pgfsetdash{}{0pt}%
\pgfpathmoveto{\pgfqpoint{1.280853in}{1.100943in}}%
\pgfpathlineto{\pgfqpoint{1.494615in}{1.849250in}}%
\pgfusepath{stroke}%
\end{pgfscope}%
\begin{pgfscope}%
\pgfpathrectangle{\pgfqpoint{0.100000in}{0.212622in}}{\pgfqpoint{3.696000in}{3.696000in}}%
\pgfusepath{clip}%
\pgfsetrectcap%
\pgfsetroundjoin%
\pgfsetlinewidth{1.505625pt}%
\definecolor{currentstroke}{rgb}{1.000000,0.000000,0.000000}%
\pgfsetstrokecolor{currentstroke}%
\pgfsetdash{}{0pt}%
\pgfpathmoveto{\pgfqpoint{1.252941in}{1.107099in}}%
\pgfpathlineto{\pgfqpoint{1.465497in}{1.857956in}}%
\pgfusepath{stroke}%
\end{pgfscope}%
\begin{pgfscope}%
\pgfpathrectangle{\pgfqpoint{0.100000in}{0.212622in}}{\pgfqpoint{3.696000in}{3.696000in}}%
\pgfusepath{clip}%
\pgfsetrectcap%
\pgfsetroundjoin%
\pgfsetlinewidth{1.505625pt}%
\definecolor{currentstroke}{rgb}{1.000000,0.000000,0.000000}%
\pgfsetstrokecolor{currentstroke}%
\pgfsetdash{}{0pt}%
\pgfpathmoveto{\pgfqpoint{1.221750in}{1.112227in}}%
\pgfpathlineto{\pgfqpoint{1.436421in}{1.866649in}}%
\pgfusepath{stroke}%
\end{pgfscope}%
\begin{pgfscope}%
\pgfpathrectangle{\pgfqpoint{0.100000in}{0.212622in}}{\pgfqpoint{3.696000in}{3.696000in}}%
\pgfusepath{clip}%
\pgfsetrectcap%
\pgfsetroundjoin%
\pgfsetlinewidth{1.505625pt}%
\definecolor{currentstroke}{rgb}{1.000000,0.000000,0.000000}%
\pgfsetstrokecolor{currentstroke}%
\pgfsetdash{}{0pt}%
\pgfpathmoveto{\pgfqpoint{1.204655in}{1.115384in}}%
\pgfpathlineto{\pgfqpoint{1.421899in}{1.870991in}}%
\pgfusepath{stroke}%
\end{pgfscope}%
\begin{pgfscope}%
\pgfpathrectangle{\pgfqpoint{0.100000in}{0.212622in}}{\pgfqpoint{3.696000in}{3.696000in}}%
\pgfusepath{clip}%
\pgfsetrectcap%
\pgfsetroundjoin%
\pgfsetlinewidth{1.505625pt}%
\definecolor{currentstroke}{rgb}{1.000000,0.000000,0.000000}%
\pgfsetstrokecolor{currentstroke}%
\pgfsetdash{}{0pt}%
\pgfpathmoveto{\pgfqpoint{1.185358in}{1.119336in}}%
\pgfpathlineto{\pgfqpoint{1.407388in}{1.875329in}}%
\pgfusepath{stroke}%
\end{pgfscope}%
\begin{pgfscope}%
\pgfpathrectangle{\pgfqpoint{0.100000in}{0.212622in}}{\pgfqpoint{3.696000in}{3.696000in}}%
\pgfusepath{clip}%
\pgfsetrectcap%
\pgfsetroundjoin%
\pgfsetlinewidth{1.505625pt}%
\definecolor{currentstroke}{rgb}{1.000000,0.000000,0.000000}%
\pgfsetstrokecolor{currentstroke}%
\pgfsetdash{}{0pt}%
\pgfpathmoveto{\pgfqpoint{1.161678in}{1.125606in}}%
\pgfpathlineto{\pgfqpoint{1.378398in}{1.883997in}}%
\pgfusepath{stroke}%
\end{pgfscope}%
\begin{pgfscope}%
\pgfpathrectangle{\pgfqpoint{0.100000in}{0.212622in}}{\pgfqpoint{3.696000in}{3.696000in}}%
\pgfusepath{clip}%
\pgfsetrectcap%
\pgfsetroundjoin%
\pgfsetlinewidth{1.505625pt}%
\definecolor{currentstroke}{rgb}{1.000000,0.000000,0.000000}%
\pgfsetstrokecolor{currentstroke}%
\pgfsetdash{}{0pt}%
\pgfpathmoveto{\pgfqpoint{1.131878in}{1.132841in}}%
\pgfpathlineto{\pgfqpoint{1.349450in}{1.892652in}}%
\pgfusepath{stroke}%
\end{pgfscope}%
\begin{pgfscope}%
\pgfpathrectangle{\pgfqpoint{0.100000in}{0.212622in}}{\pgfqpoint{3.696000in}{3.696000in}}%
\pgfusepath{clip}%
\pgfsetrectcap%
\pgfsetroundjoin%
\pgfsetlinewidth{1.505625pt}%
\definecolor{currentstroke}{rgb}{1.000000,0.000000,0.000000}%
\pgfsetstrokecolor{currentstroke}%
\pgfsetdash{}{0pt}%
\pgfpathmoveto{\pgfqpoint{1.099805in}{1.140964in}}%
\pgfpathlineto{\pgfqpoint{1.320545in}{1.901294in}}%
\pgfusepath{stroke}%
\end{pgfscope}%
\begin{pgfscope}%
\pgfpathrectangle{\pgfqpoint{0.100000in}{0.212622in}}{\pgfqpoint{3.696000in}{3.696000in}}%
\pgfusepath{clip}%
\pgfsetrectcap%
\pgfsetroundjoin%
\pgfsetlinewidth{1.505625pt}%
\definecolor{currentstroke}{rgb}{1.000000,0.000000,0.000000}%
\pgfsetstrokecolor{currentstroke}%
\pgfsetdash{}{0pt}%
\pgfpathmoveto{\pgfqpoint{1.082196in}{1.145377in}}%
\pgfpathlineto{\pgfqpoint{1.306108in}{1.905610in}}%
\pgfusepath{stroke}%
\end{pgfscope}%
\begin{pgfscope}%
\pgfpathrectangle{\pgfqpoint{0.100000in}{0.212622in}}{\pgfqpoint{3.696000in}{3.696000in}}%
\pgfusepath{clip}%
\pgfsetrectcap%
\pgfsetroundjoin%
\pgfsetlinewidth{1.505625pt}%
\definecolor{currentstroke}{rgb}{1.000000,0.000000,0.000000}%
\pgfsetstrokecolor{currentstroke}%
\pgfsetdash{}{0pt}%
\pgfpathmoveto{\pgfqpoint{1.061894in}{1.149249in}}%
\pgfpathlineto{\pgfqpoint{1.277267in}{1.914233in}}%
\pgfusepath{stroke}%
\end{pgfscope}%
\begin{pgfscope}%
\pgfpathrectangle{\pgfqpoint{0.100000in}{0.212622in}}{\pgfqpoint{3.696000in}{3.696000in}}%
\pgfusepath{clip}%
\pgfsetrectcap%
\pgfsetroundjoin%
\pgfsetlinewidth{1.505625pt}%
\definecolor{currentstroke}{rgb}{1.000000,0.000000,0.000000}%
\pgfsetstrokecolor{currentstroke}%
\pgfsetdash{}{0pt}%
\pgfpathmoveto{\pgfqpoint{1.035273in}{1.154267in}}%
\pgfpathlineto{\pgfqpoint{1.262862in}{1.918540in}}%
\pgfusepath{stroke}%
\end{pgfscope}%
\begin{pgfscope}%
\pgfpathrectangle{\pgfqpoint{0.100000in}{0.212622in}}{\pgfqpoint{3.696000in}{3.696000in}}%
\pgfusepath{clip}%
\pgfsetrectcap%
\pgfsetroundjoin%
\pgfsetlinewidth{1.505625pt}%
\definecolor{currentstroke}{rgb}{1.000000,0.000000,0.000000}%
\pgfsetstrokecolor{currentstroke}%
\pgfsetdash{}{0pt}%
\pgfpathmoveto{\pgfqpoint{1.006504in}{1.162341in}}%
\pgfpathlineto{\pgfqpoint{1.234085in}{1.927144in}}%
\pgfusepath{stroke}%
\end{pgfscope}%
\begin{pgfscope}%
\pgfpathrectangle{\pgfqpoint{0.100000in}{0.212622in}}{\pgfqpoint{3.696000in}{3.696000in}}%
\pgfusepath{clip}%
\pgfsetrectcap%
\pgfsetroundjoin%
\pgfsetlinewidth{1.505625pt}%
\definecolor{currentstroke}{rgb}{1.000000,0.000000,0.000000}%
\pgfsetstrokecolor{currentstroke}%
\pgfsetdash{}{0pt}%
\pgfpathmoveto{\pgfqpoint{0.975070in}{1.169500in}}%
\pgfpathlineto{\pgfqpoint{1.190997in}{1.940026in}}%
\pgfusepath{stroke}%
\end{pgfscope}%
\begin{pgfscope}%
\pgfpathrectangle{\pgfqpoint{0.100000in}{0.212622in}}{\pgfqpoint{3.696000in}{3.696000in}}%
\pgfusepath{clip}%
\pgfsetrectcap%
\pgfsetroundjoin%
\pgfsetlinewidth{1.505625pt}%
\definecolor{currentstroke}{rgb}{1.000000,0.000000,0.000000}%
\pgfsetstrokecolor{currentstroke}%
\pgfsetdash{}{0pt}%
\pgfpathmoveto{\pgfqpoint{0.941798in}{1.177546in}}%
\pgfpathlineto{\pgfqpoint{1.162324in}{1.948599in}}%
\pgfusepath{stroke}%
\end{pgfscope}%
\begin{pgfscope}%
\pgfpathrectangle{\pgfqpoint{0.100000in}{0.212622in}}{\pgfqpoint{3.696000in}{3.696000in}}%
\pgfusepath{clip}%
\pgfsetrectcap%
\pgfsetroundjoin%
\pgfsetlinewidth{1.505625pt}%
\definecolor{currentstroke}{rgb}{1.000000,0.000000,0.000000}%
\pgfsetstrokecolor{currentstroke}%
\pgfsetdash{}{0pt}%
\pgfpathmoveto{\pgfqpoint{0.923187in}{1.180861in}}%
\pgfpathlineto{\pgfqpoint{1.148004in}{1.952880in}}%
\pgfusepath{stroke}%
\end{pgfscope}%
\begin{pgfscope}%
\pgfpathrectangle{\pgfqpoint{0.100000in}{0.212622in}}{\pgfqpoint{3.696000in}{3.696000in}}%
\pgfusepath{clip}%
\pgfsetrectcap%
\pgfsetroundjoin%
\pgfsetlinewidth{1.505625pt}%
\definecolor{currentstroke}{rgb}{1.000000,0.000000,0.000000}%
\pgfsetstrokecolor{currentstroke}%
\pgfsetdash{}{0pt}%
\pgfpathmoveto{\pgfqpoint{0.912862in}{1.182348in}}%
\pgfpathlineto{\pgfqpoint{1.133694in}{1.957159in}}%
\pgfusepath{stroke}%
\end{pgfscope}%
\begin{pgfscope}%
\pgfpathrectangle{\pgfqpoint{0.100000in}{0.212622in}}{\pgfqpoint{3.696000in}{3.696000in}}%
\pgfusepath{clip}%
\pgfsetrectcap%
\pgfsetroundjoin%
\pgfsetlinewidth{1.505625pt}%
\definecolor{currentstroke}{rgb}{1.000000,0.000000,0.000000}%
\pgfsetstrokecolor{currentstroke}%
\pgfsetdash{}{0pt}%
\pgfpathmoveto{\pgfqpoint{0.900401in}{1.184667in}}%
\pgfpathlineto{\pgfqpoint{1.119394in}{1.961434in}}%
\pgfusepath{stroke}%
\end{pgfscope}%
\begin{pgfscope}%
\pgfpathrectangle{\pgfqpoint{0.100000in}{0.212622in}}{\pgfqpoint{3.696000in}{3.696000in}}%
\pgfusepath{clip}%
\pgfsetrectcap%
\pgfsetroundjoin%
\pgfsetlinewidth{1.505625pt}%
\definecolor{currentstroke}{rgb}{1.000000,0.000000,0.000000}%
\pgfsetstrokecolor{currentstroke}%
\pgfsetdash{}{0pt}%
\pgfpathmoveto{\pgfqpoint{0.882033in}{1.189451in}}%
\pgfpathlineto{\pgfqpoint{1.105105in}{1.965706in}}%
\pgfusepath{stroke}%
\end{pgfscope}%
\begin{pgfscope}%
\pgfpathrectangle{\pgfqpoint{0.100000in}{0.212622in}}{\pgfqpoint{3.696000in}{3.696000in}}%
\pgfusepath{clip}%
\pgfsetrectcap%
\pgfsetroundjoin%
\pgfsetlinewidth{1.505625pt}%
\definecolor{currentstroke}{rgb}{1.000000,0.000000,0.000000}%
\pgfsetstrokecolor{currentstroke}%
\pgfsetdash{}{0pt}%
\pgfpathmoveto{\pgfqpoint{0.858409in}{1.193116in}}%
\pgfpathlineto{\pgfqpoint{1.090827in}{1.969975in}}%
\pgfusepath{stroke}%
\end{pgfscope}%
\begin{pgfscope}%
\pgfpathrectangle{\pgfqpoint{0.100000in}{0.212622in}}{\pgfqpoint{3.696000in}{3.696000in}}%
\pgfusepath{clip}%
\pgfsetrectcap%
\pgfsetroundjoin%
\pgfsetlinewidth{1.505625pt}%
\definecolor{currentstroke}{rgb}{1.000000,0.000000,0.000000}%
\pgfsetstrokecolor{currentstroke}%
\pgfsetdash{}{0pt}%
\pgfpathmoveto{\pgfqpoint{0.832709in}{1.197828in}}%
\pgfpathlineto{\pgfqpoint{1.076558in}{1.974241in}}%
\pgfusepath{stroke}%
\end{pgfscope}%
\begin{pgfscope}%
\pgfpathrectangle{\pgfqpoint{0.100000in}{0.212622in}}{\pgfqpoint{3.696000in}{3.696000in}}%
\pgfusepath{clip}%
\pgfsetrectcap%
\pgfsetroundjoin%
\pgfsetlinewidth{1.505625pt}%
\definecolor{currentstroke}{rgb}{1.000000,0.000000,0.000000}%
\pgfsetstrokecolor{currentstroke}%
\pgfsetdash{}{0pt}%
\pgfpathmoveto{\pgfqpoint{0.818736in}{1.201026in}}%
\pgfpathlineto{\pgfqpoint{1.076558in}{1.974241in}}%
\pgfusepath{stroke}%
\end{pgfscope}%
\begin{pgfscope}%
\pgfpathrectangle{\pgfqpoint{0.100000in}{0.212622in}}{\pgfqpoint{3.696000in}{3.696000in}}%
\pgfusepath{clip}%
\pgfsetrectcap%
\pgfsetroundjoin%
\pgfsetlinewidth{1.505625pt}%
\definecolor{currentstroke}{rgb}{1.000000,0.000000,0.000000}%
\pgfsetstrokecolor{currentstroke}%
\pgfsetdash{}{0pt}%
\pgfpathmoveto{\pgfqpoint{0.802279in}{1.205293in}}%
\pgfpathlineto{\pgfqpoint{1.076558in}{1.974241in}}%
\pgfusepath{stroke}%
\end{pgfscope}%
\begin{pgfscope}%
\pgfpathrectangle{\pgfqpoint{0.100000in}{0.212622in}}{\pgfqpoint{3.696000in}{3.696000in}}%
\pgfusepath{clip}%
\pgfsetrectcap%
\pgfsetroundjoin%
\pgfsetlinewidth{1.505625pt}%
\definecolor{currentstroke}{rgb}{1.000000,0.000000,0.000000}%
\pgfsetstrokecolor{currentstroke}%
\pgfsetdash{}{0pt}%
\pgfpathmoveto{\pgfqpoint{0.780321in}{1.208020in}}%
\pgfpathlineto{\pgfqpoint{1.076558in}{1.974241in}}%
\pgfusepath{stroke}%
\end{pgfscope}%
\begin{pgfscope}%
\pgfpathrectangle{\pgfqpoint{0.100000in}{0.212622in}}{\pgfqpoint{3.696000in}{3.696000in}}%
\pgfusepath{clip}%
\pgfsetrectcap%
\pgfsetroundjoin%
\pgfsetlinewidth{1.505625pt}%
\definecolor{currentstroke}{rgb}{1.000000,0.000000,0.000000}%
\pgfsetstrokecolor{currentstroke}%
\pgfsetdash{}{0pt}%
\pgfpathmoveto{\pgfqpoint{0.756299in}{1.216581in}}%
\pgfpathlineto{\pgfqpoint{1.076558in}{1.974241in}}%
\pgfusepath{stroke}%
\end{pgfscope}%
\begin{pgfscope}%
\pgfpathrectangle{\pgfqpoint{0.100000in}{0.212622in}}{\pgfqpoint{3.696000in}{3.696000in}}%
\pgfusepath{clip}%
\pgfsetrectcap%
\pgfsetroundjoin%
\pgfsetlinewidth{1.505625pt}%
\definecolor{currentstroke}{rgb}{1.000000,0.000000,0.000000}%
\pgfsetstrokecolor{currentstroke}%
\pgfsetdash{}{0pt}%
\pgfpathmoveto{\pgfqpoint{0.724636in}{1.221781in}}%
\pgfpathlineto{\pgfqpoint{1.076558in}{1.974241in}}%
\pgfusepath{stroke}%
\end{pgfscope}%
\begin{pgfscope}%
\pgfpathrectangle{\pgfqpoint{0.100000in}{0.212622in}}{\pgfqpoint{3.696000in}{3.696000in}}%
\pgfusepath{clip}%
\pgfsetrectcap%
\pgfsetroundjoin%
\pgfsetlinewidth{1.505625pt}%
\definecolor{currentstroke}{rgb}{1.000000,0.000000,0.000000}%
\pgfsetstrokecolor{currentstroke}%
\pgfsetdash{}{0pt}%
\pgfpathmoveto{\pgfqpoint{0.690958in}{1.228290in}}%
\pgfpathlineto{\pgfqpoint{1.076558in}{1.974241in}}%
\pgfusepath{stroke}%
\end{pgfscope}%
\begin{pgfscope}%
\pgfpathrectangle{\pgfqpoint{0.100000in}{0.212622in}}{\pgfqpoint{3.696000in}{3.696000in}}%
\pgfusepath{clip}%
\pgfsetrectcap%
\pgfsetroundjoin%
\pgfsetlinewidth{1.505625pt}%
\definecolor{currentstroke}{rgb}{1.000000,0.000000,0.000000}%
\pgfsetstrokecolor{currentstroke}%
\pgfsetdash{}{0pt}%
\pgfpathmoveto{\pgfqpoint{0.672594in}{1.232055in}}%
\pgfpathlineto{\pgfqpoint{1.076558in}{1.974241in}}%
\pgfusepath{stroke}%
\end{pgfscope}%
\begin{pgfscope}%
\pgfpathrectangle{\pgfqpoint{0.100000in}{0.212622in}}{\pgfqpoint{3.696000in}{3.696000in}}%
\pgfusepath{clip}%
\pgfsetrectcap%
\pgfsetroundjoin%
\pgfsetlinewidth{1.505625pt}%
\definecolor{currentstroke}{rgb}{1.000000,0.000000,0.000000}%
\pgfsetstrokecolor{currentstroke}%
\pgfsetdash{}{0pt}%
\pgfpathmoveto{\pgfqpoint{0.662515in}{1.233894in}}%
\pgfpathlineto{\pgfqpoint{1.076558in}{1.974241in}}%
\pgfusepath{stroke}%
\end{pgfscope}%
\begin{pgfscope}%
\pgfpathrectangle{\pgfqpoint{0.100000in}{0.212622in}}{\pgfqpoint{3.696000in}{3.696000in}}%
\pgfusepath{clip}%
\pgfsetrectcap%
\pgfsetroundjoin%
\pgfsetlinewidth{1.505625pt}%
\definecolor{currentstroke}{rgb}{1.000000,0.000000,0.000000}%
\pgfsetstrokecolor{currentstroke}%
\pgfsetdash{}{0pt}%
\pgfpathmoveto{\pgfqpoint{0.650934in}{1.236317in}}%
\pgfpathlineto{\pgfqpoint{1.076558in}{1.974241in}}%
\pgfusepath{stroke}%
\end{pgfscope}%
\begin{pgfscope}%
\pgfpathrectangle{\pgfqpoint{0.100000in}{0.212622in}}{\pgfqpoint{3.696000in}{3.696000in}}%
\pgfusepath{clip}%
\pgfsetrectcap%
\pgfsetroundjoin%
\pgfsetlinewidth{1.505625pt}%
\definecolor{currentstroke}{rgb}{1.000000,0.000000,0.000000}%
\pgfsetstrokecolor{currentstroke}%
\pgfsetdash{}{0pt}%
\pgfpathmoveto{\pgfqpoint{0.644699in}{1.237705in}}%
\pgfpathlineto{\pgfqpoint{1.076558in}{1.974241in}}%
\pgfusepath{stroke}%
\end{pgfscope}%
\begin{pgfscope}%
\pgfpathrectangle{\pgfqpoint{0.100000in}{0.212622in}}{\pgfqpoint{3.696000in}{3.696000in}}%
\pgfusepath{clip}%
\pgfsetrectcap%
\pgfsetroundjoin%
\pgfsetlinewidth{1.505625pt}%
\definecolor{currentstroke}{rgb}{1.000000,0.000000,0.000000}%
\pgfsetstrokecolor{currentstroke}%
\pgfsetdash{}{0pt}%
\pgfpathmoveto{\pgfqpoint{0.641233in}{1.238304in}}%
\pgfpathlineto{\pgfqpoint{1.076558in}{1.974241in}}%
\pgfusepath{stroke}%
\end{pgfscope}%
\begin{pgfscope}%
\pgfpathrectangle{\pgfqpoint{0.100000in}{0.212622in}}{\pgfqpoint{3.696000in}{3.696000in}}%
\pgfusepath{clip}%
\pgfsetrectcap%
\pgfsetroundjoin%
\pgfsetlinewidth{1.505625pt}%
\definecolor{currentstroke}{rgb}{1.000000,0.000000,0.000000}%
\pgfsetstrokecolor{currentstroke}%
\pgfsetdash{}{0pt}%
\pgfpathmoveto{\pgfqpoint{0.635126in}{1.239528in}}%
\pgfpathlineto{\pgfqpoint{1.076558in}{1.974241in}}%
\pgfusepath{stroke}%
\end{pgfscope}%
\begin{pgfscope}%
\pgfpathrectangle{\pgfqpoint{0.100000in}{0.212622in}}{\pgfqpoint{3.696000in}{3.696000in}}%
\pgfusepath{clip}%
\pgfsetrectcap%
\pgfsetroundjoin%
\pgfsetlinewidth{1.505625pt}%
\definecolor{currentstroke}{rgb}{1.000000,0.000000,0.000000}%
\pgfsetstrokecolor{currentstroke}%
\pgfsetdash{}{0pt}%
\pgfpathmoveto{\pgfqpoint{0.626362in}{1.242033in}}%
\pgfpathlineto{\pgfqpoint{1.076558in}{1.974241in}}%
\pgfusepath{stroke}%
\end{pgfscope}%
\begin{pgfscope}%
\pgfpathrectangle{\pgfqpoint{0.100000in}{0.212622in}}{\pgfqpoint{3.696000in}{3.696000in}}%
\pgfusepath{clip}%
\pgfsetrectcap%
\pgfsetroundjoin%
\pgfsetlinewidth{1.505625pt}%
\definecolor{currentstroke}{rgb}{1.000000,0.000000,0.000000}%
\pgfsetstrokecolor{currentstroke}%
\pgfsetdash{}{0pt}%
\pgfpathmoveto{\pgfqpoint{0.615354in}{1.245388in}}%
\pgfpathlineto{\pgfqpoint{1.076558in}{1.974241in}}%
\pgfusepath{stroke}%
\end{pgfscope}%
\begin{pgfscope}%
\pgfpathrectangle{\pgfqpoint{0.100000in}{0.212622in}}{\pgfqpoint{3.696000in}{3.696000in}}%
\pgfusepath{clip}%
\pgfsetrectcap%
\pgfsetroundjoin%
\pgfsetlinewidth{1.505625pt}%
\definecolor{currentstroke}{rgb}{1.000000,0.000000,0.000000}%
\pgfsetstrokecolor{currentstroke}%
\pgfsetdash{}{0pt}%
\pgfpathmoveto{\pgfqpoint{0.601883in}{1.247958in}}%
\pgfpathlineto{\pgfqpoint{1.076558in}{1.974241in}}%
\pgfusepath{stroke}%
\end{pgfscope}%
\begin{pgfscope}%
\pgfpathrectangle{\pgfqpoint{0.100000in}{0.212622in}}{\pgfqpoint{3.696000in}{3.696000in}}%
\pgfusepath{clip}%
\pgfsetrectcap%
\pgfsetroundjoin%
\pgfsetlinewidth{1.505625pt}%
\definecolor{currentstroke}{rgb}{1.000000,0.000000,0.000000}%
\pgfsetstrokecolor{currentstroke}%
\pgfsetdash{}{0pt}%
\pgfpathmoveto{\pgfqpoint{0.586945in}{1.251195in}}%
\pgfpathlineto{\pgfqpoint{1.076558in}{1.974241in}}%
\pgfusepath{stroke}%
\end{pgfscope}%
\begin{pgfscope}%
\pgfpathrectangle{\pgfqpoint{0.100000in}{0.212622in}}{\pgfqpoint{3.696000in}{3.696000in}}%
\pgfusepath{clip}%
\pgfsetbuttcap%
\pgfsetroundjoin%
\definecolor{currentfill}{rgb}{0.121569,0.466667,0.705882}%
\pgfsetfillcolor{currentfill}%
\pgfsetfillopacity{0.300000}%
\pgfsetlinewidth{1.003750pt}%
\definecolor{currentstroke}{rgb}{0.121569,0.466667,0.705882}%
\pgfsetstrokecolor{currentstroke}%
\pgfsetstrokeopacity{0.300000}%
\pgfsetdash{}{0pt}%
\pgfpathmoveto{\pgfqpoint{1.756666in}{3.367167in}}%
\pgfpathcurveto{\pgfqpoint{1.764903in}{3.367167in}}{\pgfqpoint{1.772803in}{3.370439in}}{\pgfqpoint{1.778627in}{3.376263in}}%
\pgfpathcurveto{\pgfqpoint{1.784451in}{3.382087in}}{\pgfqpoint{1.787723in}{3.389987in}}{\pgfqpoint{1.787723in}{3.398224in}}%
\pgfpathcurveto{\pgfqpoint{1.787723in}{3.406460in}}{\pgfqpoint{1.784451in}{3.414360in}}{\pgfqpoint{1.778627in}{3.420184in}}%
\pgfpathcurveto{\pgfqpoint{1.772803in}{3.426008in}}{\pgfqpoint{1.764903in}{3.429280in}}{\pgfqpoint{1.756666in}{3.429280in}}%
\pgfpathcurveto{\pgfqpoint{1.748430in}{3.429280in}}{\pgfqpoint{1.740530in}{3.426008in}}{\pgfqpoint{1.734706in}{3.420184in}}%
\pgfpathcurveto{\pgfqpoint{1.728882in}{3.414360in}}{\pgfqpoint{1.725610in}{3.406460in}}{\pgfqpoint{1.725610in}{3.398224in}}%
\pgfpathcurveto{\pgfqpoint{1.725610in}{3.389987in}}{\pgfqpoint{1.728882in}{3.382087in}}{\pgfqpoint{1.734706in}{3.376263in}}%
\pgfpathcurveto{\pgfqpoint{1.740530in}{3.370439in}}{\pgfqpoint{1.748430in}{3.367167in}}{\pgfqpoint{1.756666in}{3.367167in}}%
\pgfpathclose%
\pgfusepath{stroke,fill}%
\end{pgfscope}%
\begin{pgfscope}%
\pgfpathrectangle{\pgfqpoint{0.100000in}{0.212622in}}{\pgfqpoint{3.696000in}{3.696000in}}%
\pgfusepath{clip}%
\pgfsetbuttcap%
\pgfsetroundjoin%
\definecolor{currentfill}{rgb}{0.121569,0.466667,0.705882}%
\pgfsetfillcolor{currentfill}%
\pgfsetfillopacity{0.300025}%
\pgfsetlinewidth{1.003750pt}%
\definecolor{currentstroke}{rgb}{0.121569,0.466667,0.705882}%
\pgfsetstrokecolor{currentstroke}%
\pgfsetstrokeopacity{0.300025}%
\pgfsetdash{}{0pt}%
\pgfpathmoveto{\pgfqpoint{1.770205in}{3.370649in}}%
\pgfpathcurveto{\pgfqpoint{1.778441in}{3.370649in}}{\pgfqpoint{1.786341in}{3.373921in}}{\pgfqpoint{1.792165in}{3.379745in}}%
\pgfpathcurveto{\pgfqpoint{1.797989in}{3.385569in}}{\pgfqpoint{1.801261in}{3.393469in}}{\pgfqpoint{1.801261in}{3.401705in}}%
\pgfpathcurveto{\pgfqpoint{1.801261in}{3.409941in}}{\pgfqpoint{1.797989in}{3.417841in}}{\pgfqpoint{1.792165in}{3.423665in}}%
\pgfpathcurveto{\pgfqpoint{1.786341in}{3.429489in}}{\pgfqpoint{1.778441in}{3.432762in}}{\pgfqpoint{1.770205in}{3.432762in}}%
\pgfpathcurveto{\pgfqpoint{1.761968in}{3.432762in}}{\pgfqpoint{1.754068in}{3.429489in}}{\pgfqpoint{1.748244in}{3.423665in}}%
\pgfpathcurveto{\pgfqpoint{1.742420in}{3.417841in}}{\pgfqpoint{1.739148in}{3.409941in}}{\pgfqpoint{1.739148in}{3.401705in}}%
\pgfpathcurveto{\pgfqpoint{1.739148in}{3.393469in}}{\pgfqpoint{1.742420in}{3.385569in}}{\pgfqpoint{1.748244in}{3.379745in}}%
\pgfpathcurveto{\pgfqpoint{1.754068in}{3.373921in}}{\pgfqpoint{1.761968in}{3.370649in}}{\pgfqpoint{1.770205in}{3.370649in}}%
\pgfpathclose%
\pgfusepath{stroke,fill}%
\end{pgfscope}%
\begin{pgfscope}%
\pgfpathrectangle{\pgfqpoint{0.100000in}{0.212622in}}{\pgfqpoint{3.696000in}{3.696000in}}%
\pgfusepath{clip}%
\pgfsetbuttcap%
\pgfsetroundjoin%
\definecolor{currentfill}{rgb}{0.121569,0.466667,0.705882}%
\pgfsetfillcolor{currentfill}%
\pgfsetfillopacity{0.300928}%
\pgfsetlinewidth{1.003750pt}%
\definecolor{currentstroke}{rgb}{0.121569,0.466667,0.705882}%
\pgfsetstrokecolor{currentstroke}%
\pgfsetstrokeopacity{0.300928}%
\pgfsetdash{}{0pt}%
\pgfpathmoveto{\pgfqpoint{1.778209in}{3.368798in}}%
\pgfpathcurveto{\pgfqpoint{1.786445in}{3.368798in}}{\pgfqpoint{1.794345in}{3.372071in}}{\pgfqpoint{1.800169in}{3.377895in}}%
\pgfpathcurveto{\pgfqpoint{1.805993in}{3.383718in}}{\pgfqpoint{1.809266in}{3.391619in}}{\pgfqpoint{1.809266in}{3.399855in}}%
\pgfpathcurveto{\pgfqpoint{1.809266in}{3.408091in}}{\pgfqpoint{1.805993in}{3.415991in}}{\pgfqpoint{1.800169in}{3.421815in}}%
\pgfpathcurveto{\pgfqpoint{1.794345in}{3.427639in}}{\pgfqpoint{1.786445in}{3.430911in}}{\pgfqpoint{1.778209in}{3.430911in}}%
\pgfpathcurveto{\pgfqpoint{1.769973in}{3.430911in}}{\pgfqpoint{1.762073in}{3.427639in}}{\pgfqpoint{1.756249in}{3.421815in}}%
\pgfpathcurveto{\pgfqpoint{1.750425in}{3.415991in}}{\pgfqpoint{1.747153in}{3.408091in}}{\pgfqpoint{1.747153in}{3.399855in}}%
\pgfpathcurveto{\pgfqpoint{1.747153in}{3.391619in}}{\pgfqpoint{1.750425in}{3.383718in}}{\pgfqpoint{1.756249in}{3.377895in}}%
\pgfpathcurveto{\pgfqpoint{1.762073in}{3.372071in}}{\pgfqpoint{1.769973in}{3.368798in}}{\pgfqpoint{1.778209in}{3.368798in}}%
\pgfpathclose%
\pgfusepath{stroke,fill}%
\end{pgfscope}%
\begin{pgfscope}%
\pgfpathrectangle{\pgfqpoint{0.100000in}{0.212622in}}{\pgfqpoint{3.696000in}{3.696000in}}%
\pgfusepath{clip}%
\pgfsetbuttcap%
\pgfsetroundjoin%
\definecolor{currentfill}{rgb}{0.121569,0.466667,0.705882}%
\pgfsetfillcolor{currentfill}%
\pgfsetfillopacity{0.301113}%
\pgfsetlinewidth{1.003750pt}%
\definecolor{currentstroke}{rgb}{0.121569,0.466667,0.705882}%
\pgfsetstrokecolor{currentstroke}%
\pgfsetstrokeopacity{0.301113}%
\pgfsetdash{}{0pt}%
\pgfpathmoveto{\pgfqpoint{1.750214in}{3.359823in}}%
\pgfpathcurveto{\pgfqpoint{1.758450in}{3.359823in}}{\pgfqpoint{1.766350in}{3.363095in}}{\pgfqpoint{1.772174in}{3.368919in}}%
\pgfpathcurveto{\pgfqpoint{1.777998in}{3.374743in}}{\pgfqpoint{1.781270in}{3.382643in}}{\pgfqpoint{1.781270in}{3.390879in}}%
\pgfpathcurveto{\pgfqpoint{1.781270in}{3.399116in}}{\pgfqpoint{1.777998in}{3.407016in}}{\pgfqpoint{1.772174in}{3.412840in}}%
\pgfpathcurveto{\pgfqpoint{1.766350in}{3.418664in}}{\pgfqpoint{1.758450in}{3.421936in}}{\pgfqpoint{1.750214in}{3.421936in}}%
\pgfpathcurveto{\pgfqpoint{1.741977in}{3.421936in}}{\pgfqpoint{1.734077in}{3.418664in}}{\pgfqpoint{1.728253in}{3.412840in}}%
\pgfpathcurveto{\pgfqpoint{1.722429in}{3.407016in}}{\pgfqpoint{1.719157in}{3.399116in}}{\pgfqpoint{1.719157in}{3.390879in}}%
\pgfpathcurveto{\pgfqpoint{1.719157in}{3.382643in}}{\pgfqpoint{1.722429in}{3.374743in}}{\pgfqpoint{1.728253in}{3.368919in}}%
\pgfpathcurveto{\pgfqpoint{1.734077in}{3.363095in}}{\pgfqpoint{1.741977in}{3.359823in}}{\pgfqpoint{1.750214in}{3.359823in}}%
\pgfpathclose%
\pgfusepath{stroke,fill}%
\end{pgfscope}%
\begin{pgfscope}%
\pgfpathrectangle{\pgfqpoint{0.100000in}{0.212622in}}{\pgfqpoint{3.696000in}{3.696000in}}%
\pgfusepath{clip}%
\pgfsetbuttcap%
\pgfsetroundjoin%
\definecolor{currentfill}{rgb}{0.121569,0.466667,0.705882}%
\pgfsetfillcolor{currentfill}%
\pgfsetfillopacity{0.301646}%
\pgfsetlinewidth{1.003750pt}%
\definecolor{currentstroke}{rgb}{0.121569,0.466667,0.705882}%
\pgfsetstrokecolor{currentstroke}%
\pgfsetstrokeopacity{0.301646}%
\pgfsetdash{}{0pt}%
\pgfpathmoveto{\pgfqpoint{1.782299in}{3.366996in}}%
\pgfpathcurveto{\pgfqpoint{1.790536in}{3.366996in}}{\pgfqpoint{1.798436in}{3.370268in}}{\pgfqpoint{1.804260in}{3.376092in}}%
\pgfpathcurveto{\pgfqpoint{1.810084in}{3.381916in}}{\pgfqpoint{1.813356in}{3.389816in}}{\pgfqpoint{1.813356in}{3.398052in}}%
\pgfpathcurveto{\pgfqpoint{1.813356in}{3.406289in}}{\pgfqpoint{1.810084in}{3.414189in}}{\pgfqpoint{1.804260in}{3.420013in}}%
\pgfpathcurveto{\pgfqpoint{1.798436in}{3.425837in}}{\pgfqpoint{1.790536in}{3.429109in}}{\pgfqpoint{1.782299in}{3.429109in}}%
\pgfpathcurveto{\pgfqpoint{1.774063in}{3.429109in}}{\pgfqpoint{1.766163in}{3.425837in}}{\pgfqpoint{1.760339in}{3.420013in}}%
\pgfpathcurveto{\pgfqpoint{1.754515in}{3.414189in}}{\pgfqpoint{1.751243in}{3.406289in}}{\pgfqpoint{1.751243in}{3.398052in}}%
\pgfpathcurveto{\pgfqpoint{1.751243in}{3.389816in}}{\pgfqpoint{1.754515in}{3.381916in}}{\pgfqpoint{1.760339in}{3.376092in}}%
\pgfpathcurveto{\pgfqpoint{1.766163in}{3.370268in}}{\pgfqpoint{1.774063in}{3.366996in}}{\pgfqpoint{1.782299in}{3.366996in}}%
\pgfpathclose%
\pgfusepath{stroke,fill}%
\end{pgfscope}%
\begin{pgfscope}%
\pgfpathrectangle{\pgfqpoint{0.100000in}{0.212622in}}{\pgfqpoint{3.696000in}{3.696000in}}%
\pgfusepath{clip}%
\pgfsetbuttcap%
\pgfsetroundjoin%
\definecolor{currentfill}{rgb}{0.121569,0.466667,0.705882}%
\pgfsetfillcolor{currentfill}%
\pgfsetfillopacity{0.301998}%
\pgfsetlinewidth{1.003750pt}%
\definecolor{currentstroke}{rgb}{0.121569,0.466667,0.705882}%
\pgfsetstrokecolor{currentstroke}%
\pgfsetstrokeopacity{0.301998}%
\pgfsetdash{}{0pt}%
\pgfpathmoveto{\pgfqpoint{1.747183in}{3.354948in}}%
\pgfpathcurveto{\pgfqpoint{1.755419in}{3.354948in}}{\pgfqpoint{1.763319in}{3.358221in}}{\pgfqpoint{1.769143in}{3.364045in}}%
\pgfpathcurveto{\pgfqpoint{1.774967in}{3.369869in}}{\pgfqpoint{1.778239in}{3.377769in}}{\pgfqpoint{1.778239in}{3.386005in}}%
\pgfpathcurveto{\pgfqpoint{1.778239in}{3.394241in}}{\pgfqpoint{1.774967in}{3.402141in}}{\pgfqpoint{1.769143in}{3.407965in}}%
\pgfpathcurveto{\pgfqpoint{1.763319in}{3.413789in}}{\pgfqpoint{1.755419in}{3.417061in}}{\pgfqpoint{1.747183in}{3.417061in}}%
\pgfpathcurveto{\pgfqpoint{1.738946in}{3.417061in}}{\pgfqpoint{1.731046in}{3.413789in}}{\pgfqpoint{1.725222in}{3.407965in}}%
\pgfpathcurveto{\pgfqpoint{1.719398in}{3.402141in}}{\pgfqpoint{1.716126in}{3.394241in}}{\pgfqpoint{1.716126in}{3.386005in}}%
\pgfpathcurveto{\pgfqpoint{1.716126in}{3.377769in}}{\pgfqpoint{1.719398in}{3.369869in}}{\pgfqpoint{1.725222in}{3.364045in}}%
\pgfpathcurveto{\pgfqpoint{1.731046in}{3.358221in}}{\pgfqpoint{1.738946in}{3.354948in}}{\pgfqpoint{1.747183in}{3.354948in}}%
\pgfpathclose%
\pgfusepath{stroke,fill}%
\end{pgfscope}%
\begin{pgfscope}%
\pgfpathrectangle{\pgfqpoint{0.100000in}{0.212622in}}{\pgfqpoint{3.696000in}{3.696000in}}%
\pgfusepath{clip}%
\pgfsetbuttcap%
\pgfsetroundjoin%
\definecolor{currentfill}{rgb}{0.121569,0.466667,0.705882}%
\pgfsetfillcolor{currentfill}%
\pgfsetfillopacity{0.302110}%
\pgfsetlinewidth{1.003750pt}%
\definecolor{currentstroke}{rgb}{0.121569,0.466667,0.705882}%
\pgfsetstrokecolor{currentstroke}%
\pgfsetstrokeopacity{0.302110}%
\pgfsetdash{}{0pt}%
\pgfpathmoveto{\pgfqpoint{1.784318in}{3.365693in}}%
\pgfpathcurveto{\pgfqpoint{1.792554in}{3.365693in}}{\pgfqpoint{1.800454in}{3.368965in}}{\pgfqpoint{1.806278in}{3.374789in}}%
\pgfpathcurveto{\pgfqpoint{1.812102in}{3.380613in}}{\pgfqpoint{1.815374in}{3.388513in}}{\pgfqpoint{1.815374in}{3.396749in}}%
\pgfpathcurveto{\pgfqpoint{1.815374in}{3.404985in}}{\pgfqpoint{1.812102in}{3.412886in}}{\pgfqpoint{1.806278in}{3.418709in}}%
\pgfpathcurveto{\pgfqpoint{1.800454in}{3.424533in}}{\pgfqpoint{1.792554in}{3.427806in}}{\pgfqpoint{1.784318in}{3.427806in}}%
\pgfpathcurveto{\pgfqpoint{1.776081in}{3.427806in}}{\pgfqpoint{1.768181in}{3.424533in}}{\pgfqpoint{1.762357in}{3.418709in}}%
\pgfpathcurveto{\pgfqpoint{1.756534in}{3.412886in}}{\pgfqpoint{1.753261in}{3.404985in}}{\pgfqpoint{1.753261in}{3.396749in}}%
\pgfpathcurveto{\pgfqpoint{1.753261in}{3.388513in}}{\pgfqpoint{1.756534in}{3.380613in}}{\pgfqpoint{1.762357in}{3.374789in}}%
\pgfpathcurveto{\pgfqpoint{1.768181in}{3.368965in}}{\pgfqpoint{1.776081in}{3.365693in}}{\pgfqpoint{1.784318in}{3.365693in}}%
\pgfpathclose%
\pgfusepath{stroke,fill}%
\end{pgfscope}%
\begin{pgfscope}%
\pgfpathrectangle{\pgfqpoint{0.100000in}{0.212622in}}{\pgfqpoint{3.696000in}{3.696000in}}%
\pgfusepath{clip}%
\pgfsetbuttcap%
\pgfsetroundjoin%
\definecolor{currentfill}{rgb}{0.121569,0.466667,0.705882}%
\pgfsetfillcolor{currentfill}%
\pgfsetfillopacity{0.302378}%
\pgfsetlinewidth{1.003750pt}%
\definecolor{currentstroke}{rgb}{0.121569,0.466667,0.705882}%
\pgfsetstrokecolor{currentstroke}%
\pgfsetstrokeopacity{0.302378}%
\pgfsetdash{}{0pt}%
\pgfpathmoveto{\pgfqpoint{1.785424in}{3.365025in}}%
\pgfpathcurveto{\pgfqpoint{1.793661in}{3.365025in}}{\pgfqpoint{1.801561in}{3.368298in}}{\pgfqpoint{1.807385in}{3.374122in}}%
\pgfpathcurveto{\pgfqpoint{1.813209in}{3.379946in}}{\pgfqpoint{1.816481in}{3.387846in}}{\pgfqpoint{1.816481in}{3.396082in}}%
\pgfpathcurveto{\pgfqpoint{1.816481in}{3.404318in}}{\pgfqpoint{1.813209in}{3.412218in}}{\pgfqpoint{1.807385in}{3.418042in}}%
\pgfpathcurveto{\pgfqpoint{1.801561in}{3.423866in}}{\pgfqpoint{1.793661in}{3.427138in}}{\pgfqpoint{1.785424in}{3.427138in}}%
\pgfpathcurveto{\pgfqpoint{1.777188in}{3.427138in}}{\pgfqpoint{1.769288in}{3.423866in}}{\pgfqpoint{1.763464in}{3.418042in}}%
\pgfpathcurveto{\pgfqpoint{1.757640in}{3.412218in}}{\pgfqpoint{1.754368in}{3.404318in}}{\pgfqpoint{1.754368in}{3.396082in}}%
\pgfpathcurveto{\pgfqpoint{1.754368in}{3.387846in}}{\pgfqpoint{1.757640in}{3.379946in}}{\pgfqpoint{1.763464in}{3.374122in}}%
\pgfpathcurveto{\pgfqpoint{1.769288in}{3.368298in}}{\pgfqpoint{1.777188in}{3.365025in}}{\pgfqpoint{1.785424in}{3.365025in}}%
\pgfpathclose%
\pgfusepath{stroke,fill}%
\end{pgfscope}%
\begin{pgfscope}%
\pgfpathrectangle{\pgfqpoint{0.100000in}{0.212622in}}{\pgfqpoint{3.696000in}{3.696000in}}%
\pgfusepath{clip}%
\pgfsetbuttcap%
\pgfsetroundjoin%
\definecolor{currentfill}{rgb}{0.121569,0.466667,0.705882}%
\pgfsetfillcolor{currentfill}%
\pgfsetfillopacity{0.302531}%
\pgfsetlinewidth{1.003750pt}%
\definecolor{currentstroke}{rgb}{0.121569,0.466667,0.705882}%
\pgfsetstrokecolor{currentstroke}%
\pgfsetstrokeopacity{0.302531}%
\pgfsetdash{}{0pt}%
\pgfpathmoveto{\pgfqpoint{1.785941in}{3.364541in}}%
\pgfpathcurveto{\pgfqpoint{1.794177in}{3.364541in}}{\pgfqpoint{1.802077in}{3.367813in}}{\pgfqpoint{1.807901in}{3.373637in}}%
\pgfpathcurveto{\pgfqpoint{1.813725in}{3.379461in}}{\pgfqpoint{1.816997in}{3.387361in}}{\pgfqpoint{1.816997in}{3.395597in}}%
\pgfpathcurveto{\pgfqpoint{1.816997in}{3.403834in}}{\pgfqpoint{1.813725in}{3.411734in}}{\pgfqpoint{1.807901in}{3.417558in}}%
\pgfpathcurveto{\pgfqpoint{1.802077in}{3.423382in}}{\pgfqpoint{1.794177in}{3.426654in}}{\pgfqpoint{1.785941in}{3.426654in}}%
\pgfpathcurveto{\pgfqpoint{1.777704in}{3.426654in}}{\pgfqpoint{1.769804in}{3.423382in}}{\pgfqpoint{1.763980in}{3.417558in}}%
\pgfpathcurveto{\pgfqpoint{1.758156in}{3.411734in}}{\pgfqpoint{1.754884in}{3.403834in}}{\pgfqpoint{1.754884in}{3.395597in}}%
\pgfpathcurveto{\pgfqpoint{1.754884in}{3.387361in}}{\pgfqpoint{1.758156in}{3.379461in}}{\pgfqpoint{1.763980in}{3.373637in}}%
\pgfpathcurveto{\pgfqpoint{1.769804in}{3.367813in}}{\pgfqpoint{1.777704in}{3.364541in}}{\pgfqpoint{1.785941in}{3.364541in}}%
\pgfpathclose%
\pgfusepath{stroke,fill}%
\end{pgfscope}%
\begin{pgfscope}%
\pgfpathrectangle{\pgfqpoint{0.100000in}{0.212622in}}{\pgfqpoint{3.696000in}{3.696000in}}%
\pgfusepath{clip}%
\pgfsetbuttcap%
\pgfsetroundjoin%
\definecolor{currentfill}{rgb}{0.121569,0.466667,0.705882}%
\pgfsetfillcolor{currentfill}%
\pgfsetfillopacity{0.303639}%
\pgfsetlinewidth{1.003750pt}%
\definecolor{currentstroke}{rgb}{0.121569,0.466667,0.705882}%
\pgfsetstrokecolor{currentstroke}%
\pgfsetstrokeopacity{0.303639}%
\pgfsetdash{}{0pt}%
\pgfpathmoveto{\pgfqpoint{1.789495in}{3.361286in}}%
\pgfpathcurveto{\pgfqpoint{1.797732in}{3.361286in}}{\pgfqpoint{1.805632in}{3.364558in}}{\pgfqpoint{1.811456in}{3.370382in}}%
\pgfpathcurveto{\pgfqpoint{1.817280in}{3.376206in}}{\pgfqpoint{1.820552in}{3.384106in}}{\pgfqpoint{1.820552in}{3.392342in}}%
\pgfpathcurveto{\pgfqpoint{1.820552in}{3.400579in}}{\pgfqpoint{1.817280in}{3.408479in}}{\pgfqpoint{1.811456in}{3.414303in}}%
\pgfpathcurveto{\pgfqpoint{1.805632in}{3.420126in}}{\pgfqpoint{1.797732in}{3.423399in}}{\pgfqpoint{1.789495in}{3.423399in}}%
\pgfpathcurveto{\pgfqpoint{1.781259in}{3.423399in}}{\pgfqpoint{1.773359in}{3.420126in}}{\pgfqpoint{1.767535in}{3.414303in}}%
\pgfpathcurveto{\pgfqpoint{1.761711in}{3.408479in}}{\pgfqpoint{1.758439in}{3.400579in}}{\pgfqpoint{1.758439in}{3.392342in}}%
\pgfpathcurveto{\pgfqpoint{1.758439in}{3.384106in}}{\pgfqpoint{1.761711in}{3.376206in}}{\pgfqpoint{1.767535in}{3.370382in}}%
\pgfpathcurveto{\pgfqpoint{1.773359in}{3.364558in}}{\pgfqpoint{1.781259in}{3.361286in}}{\pgfqpoint{1.789495in}{3.361286in}}%
\pgfpathclose%
\pgfusepath{stroke,fill}%
\end{pgfscope}%
\begin{pgfscope}%
\pgfpathrectangle{\pgfqpoint{0.100000in}{0.212622in}}{\pgfqpoint{3.696000in}{3.696000in}}%
\pgfusepath{clip}%
\pgfsetbuttcap%
\pgfsetroundjoin%
\definecolor{currentfill}{rgb}{0.121569,0.466667,0.705882}%
\pgfsetfillcolor{currentfill}%
\pgfsetfillopacity{0.303729}%
\pgfsetlinewidth{1.003750pt}%
\definecolor{currentstroke}{rgb}{0.121569,0.466667,0.705882}%
\pgfsetstrokecolor{currentstroke}%
\pgfsetstrokeopacity{0.303729}%
\pgfsetdash{}{0pt}%
\pgfpathmoveto{\pgfqpoint{1.742331in}{3.345771in}}%
\pgfpathcurveto{\pgfqpoint{1.750567in}{3.345771in}}{\pgfqpoint{1.758467in}{3.349043in}}{\pgfqpoint{1.764291in}{3.354867in}}%
\pgfpathcurveto{\pgfqpoint{1.770115in}{3.360691in}}{\pgfqpoint{1.773387in}{3.368591in}}{\pgfqpoint{1.773387in}{3.376828in}}%
\pgfpathcurveto{\pgfqpoint{1.773387in}{3.385064in}}{\pgfqpoint{1.770115in}{3.392964in}}{\pgfqpoint{1.764291in}{3.398788in}}%
\pgfpathcurveto{\pgfqpoint{1.758467in}{3.404612in}}{\pgfqpoint{1.750567in}{3.407884in}}{\pgfqpoint{1.742331in}{3.407884in}}%
\pgfpathcurveto{\pgfqpoint{1.734094in}{3.407884in}}{\pgfqpoint{1.726194in}{3.404612in}}{\pgfqpoint{1.720370in}{3.398788in}}%
\pgfpathcurveto{\pgfqpoint{1.714546in}{3.392964in}}{\pgfqpoint{1.711274in}{3.385064in}}{\pgfqpoint{1.711274in}{3.376828in}}%
\pgfpathcurveto{\pgfqpoint{1.711274in}{3.368591in}}{\pgfqpoint{1.714546in}{3.360691in}}{\pgfqpoint{1.720370in}{3.354867in}}%
\pgfpathcurveto{\pgfqpoint{1.726194in}{3.349043in}}{\pgfqpoint{1.734094in}{3.345771in}}{\pgfqpoint{1.742331in}{3.345771in}}%
\pgfpathclose%
\pgfusepath{stroke,fill}%
\end{pgfscope}%
\begin{pgfscope}%
\pgfpathrectangle{\pgfqpoint{0.100000in}{0.212622in}}{\pgfqpoint{3.696000in}{3.696000in}}%
\pgfusepath{clip}%
\pgfsetbuttcap%
\pgfsetroundjoin%
\definecolor{currentfill}{rgb}{0.121569,0.466667,0.705882}%
\pgfsetfillcolor{currentfill}%
\pgfsetfillopacity{0.304743}%
\pgfsetlinewidth{1.003750pt}%
\definecolor{currentstroke}{rgb}{0.121569,0.466667,0.705882}%
\pgfsetstrokecolor{currentstroke}%
\pgfsetstrokeopacity{0.304743}%
\pgfsetdash{}{0pt}%
\pgfpathmoveto{\pgfqpoint{1.740145in}{3.340937in}}%
\pgfpathcurveto{\pgfqpoint{1.748382in}{3.340937in}}{\pgfqpoint{1.756282in}{3.344209in}}{\pgfqpoint{1.762106in}{3.350033in}}%
\pgfpathcurveto{\pgfqpoint{1.767929in}{3.355857in}}{\pgfqpoint{1.771202in}{3.363757in}}{\pgfqpoint{1.771202in}{3.371994in}}%
\pgfpathcurveto{\pgfqpoint{1.771202in}{3.380230in}}{\pgfqpoint{1.767929in}{3.388130in}}{\pgfqpoint{1.762106in}{3.393954in}}%
\pgfpathcurveto{\pgfqpoint{1.756282in}{3.399778in}}{\pgfqpoint{1.748382in}{3.403050in}}{\pgfqpoint{1.740145in}{3.403050in}}%
\pgfpathcurveto{\pgfqpoint{1.731909in}{3.403050in}}{\pgfqpoint{1.724009in}{3.399778in}}{\pgfqpoint{1.718185in}{3.393954in}}%
\pgfpathcurveto{\pgfqpoint{1.712361in}{3.388130in}}{\pgfqpoint{1.709089in}{3.380230in}}{\pgfqpoint{1.709089in}{3.371994in}}%
\pgfpathcurveto{\pgfqpoint{1.709089in}{3.363757in}}{\pgfqpoint{1.712361in}{3.355857in}}{\pgfqpoint{1.718185in}{3.350033in}}%
\pgfpathcurveto{\pgfqpoint{1.724009in}{3.344209in}}{\pgfqpoint{1.731909in}{3.340937in}}{\pgfqpoint{1.740145in}{3.340937in}}%
\pgfpathclose%
\pgfusepath{stroke,fill}%
\end{pgfscope}%
\begin{pgfscope}%
\pgfpathrectangle{\pgfqpoint{0.100000in}{0.212622in}}{\pgfqpoint{3.696000in}{3.696000in}}%
\pgfusepath{clip}%
\pgfsetbuttcap%
\pgfsetroundjoin%
\definecolor{currentfill}{rgb}{0.121569,0.466667,0.705882}%
\pgfsetfillcolor{currentfill}%
\pgfsetfillopacity{0.305095}%
\pgfsetlinewidth{1.003750pt}%
\definecolor{currentstroke}{rgb}{0.121569,0.466667,0.705882}%
\pgfsetstrokecolor{currentstroke}%
\pgfsetstrokeopacity{0.305095}%
\pgfsetdash{}{0pt}%
\pgfpathmoveto{\pgfqpoint{1.739447in}{3.339288in}}%
\pgfpathcurveto{\pgfqpoint{1.747683in}{3.339288in}}{\pgfqpoint{1.755583in}{3.342560in}}{\pgfqpoint{1.761407in}{3.348384in}}%
\pgfpathcurveto{\pgfqpoint{1.767231in}{3.354208in}}{\pgfqpoint{1.770504in}{3.362108in}}{\pgfqpoint{1.770504in}{3.370344in}}%
\pgfpathcurveto{\pgfqpoint{1.770504in}{3.378580in}}{\pgfqpoint{1.767231in}{3.386481in}}{\pgfqpoint{1.761407in}{3.392304in}}%
\pgfpathcurveto{\pgfqpoint{1.755583in}{3.398128in}}{\pgfqpoint{1.747683in}{3.401401in}}{\pgfqpoint{1.739447in}{3.401401in}}%
\pgfpathcurveto{\pgfqpoint{1.731211in}{3.401401in}}{\pgfqpoint{1.723311in}{3.398128in}}{\pgfqpoint{1.717487in}{3.392304in}}%
\pgfpathcurveto{\pgfqpoint{1.711663in}{3.386481in}}{\pgfqpoint{1.708391in}{3.378580in}}{\pgfqpoint{1.708391in}{3.370344in}}%
\pgfpathcurveto{\pgfqpoint{1.708391in}{3.362108in}}{\pgfqpoint{1.711663in}{3.354208in}}{\pgfqpoint{1.717487in}{3.348384in}}%
\pgfpathcurveto{\pgfqpoint{1.723311in}{3.342560in}}{\pgfqpoint{1.731211in}{3.339288in}}{\pgfqpoint{1.739447in}{3.339288in}}%
\pgfpathclose%
\pgfusepath{stroke,fill}%
\end{pgfscope}%
\begin{pgfscope}%
\pgfpathrectangle{\pgfqpoint{0.100000in}{0.212622in}}{\pgfqpoint{3.696000in}{3.696000in}}%
\pgfusepath{clip}%
\pgfsetbuttcap%
\pgfsetroundjoin%
\definecolor{currentfill}{rgb}{0.121569,0.466667,0.705882}%
\pgfsetfillcolor{currentfill}%
\pgfsetfillopacity{0.305674}%
\pgfsetlinewidth{1.003750pt}%
\definecolor{currentstroke}{rgb}{0.121569,0.466667,0.705882}%
\pgfsetstrokecolor{currentstroke}%
\pgfsetstrokeopacity{0.305674}%
\pgfsetdash{}{0pt}%
\pgfpathmoveto{\pgfqpoint{1.795191in}{3.354545in}}%
\pgfpathcurveto{\pgfqpoint{1.803427in}{3.354545in}}{\pgfqpoint{1.811328in}{3.357818in}}{\pgfqpoint{1.817151in}{3.363642in}}%
\pgfpathcurveto{\pgfqpoint{1.822975in}{3.369466in}}{\pgfqpoint{1.826248in}{3.377366in}}{\pgfqpoint{1.826248in}{3.385602in}}%
\pgfpathcurveto{\pgfqpoint{1.826248in}{3.393838in}}{\pgfqpoint{1.822975in}{3.401738in}}{\pgfqpoint{1.817151in}{3.407562in}}%
\pgfpathcurveto{\pgfqpoint{1.811328in}{3.413386in}}{\pgfqpoint{1.803427in}{3.416658in}}{\pgfqpoint{1.795191in}{3.416658in}}%
\pgfpathcurveto{\pgfqpoint{1.786955in}{3.416658in}}{\pgfqpoint{1.779055in}{3.413386in}}{\pgfqpoint{1.773231in}{3.407562in}}%
\pgfpathcurveto{\pgfqpoint{1.767407in}{3.401738in}}{\pgfqpoint{1.764135in}{3.393838in}}{\pgfqpoint{1.764135in}{3.385602in}}%
\pgfpathcurveto{\pgfqpoint{1.764135in}{3.377366in}}{\pgfqpoint{1.767407in}{3.369466in}}{\pgfqpoint{1.773231in}{3.363642in}}%
\pgfpathcurveto{\pgfqpoint{1.779055in}{3.357818in}}{\pgfqpoint{1.786955in}{3.354545in}}{\pgfqpoint{1.795191in}{3.354545in}}%
\pgfpathclose%
\pgfusepath{stroke,fill}%
\end{pgfscope}%
\begin{pgfscope}%
\pgfpathrectangle{\pgfqpoint{0.100000in}{0.212622in}}{\pgfqpoint{3.696000in}{3.696000in}}%
\pgfusepath{clip}%
\pgfsetbuttcap%
\pgfsetroundjoin%
\definecolor{currentfill}{rgb}{0.121569,0.466667,0.705882}%
\pgfsetfillcolor{currentfill}%
\pgfsetfillopacity{0.305769}%
\pgfsetlinewidth{1.003750pt}%
\definecolor{currentstroke}{rgb}{0.121569,0.466667,0.705882}%
\pgfsetstrokecolor{currentstroke}%
\pgfsetstrokeopacity{0.305769}%
\pgfsetdash{}{0pt}%
\pgfpathmoveto{\pgfqpoint{1.738268in}{3.336330in}}%
\pgfpathcurveto{\pgfqpoint{1.746504in}{3.336330in}}{\pgfqpoint{1.754404in}{3.339602in}}{\pgfqpoint{1.760228in}{3.345426in}}%
\pgfpathcurveto{\pgfqpoint{1.766052in}{3.351250in}}{\pgfqpoint{1.769324in}{3.359150in}}{\pgfqpoint{1.769324in}{3.367387in}}%
\pgfpathcurveto{\pgfqpoint{1.769324in}{3.375623in}}{\pgfqpoint{1.766052in}{3.383523in}}{\pgfqpoint{1.760228in}{3.389347in}}%
\pgfpathcurveto{\pgfqpoint{1.754404in}{3.395171in}}{\pgfqpoint{1.746504in}{3.398443in}}{\pgfqpoint{1.738268in}{3.398443in}}%
\pgfpathcurveto{\pgfqpoint{1.730031in}{3.398443in}}{\pgfqpoint{1.722131in}{3.395171in}}{\pgfqpoint{1.716307in}{3.389347in}}%
\pgfpathcurveto{\pgfqpoint{1.710483in}{3.383523in}}{\pgfqpoint{1.707211in}{3.375623in}}{\pgfqpoint{1.707211in}{3.367387in}}%
\pgfpathcurveto{\pgfqpoint{1.707211in}{3.359150in}}{\pgfqpoint{1.710483in}{3.351250in}}{\pgfqpoint{1.716307in}{3.345426in}}%
\pgfpathcurveto{\pgfqpoint{1.722131in}{3.339602in}}{\pgfqpoint{1.730031in}{3.336330in}}{\pgfqpoint{1.738268in}{3.336330in}}%
\pgfpathclose%
\pgfusepath{stroke,fill}%
\end{pgfscope}%
\begin{pgfscope}%
\pgfpathrectangle{\pgfqpoint{0.100000in}{0.212622in}}{\pgfqpoint{3.696000in}{3.696000in}}%
\pgfusepath{clip}%
\pgfsetbuttcap%
\pgfsetroundjoin%
\definecolor{currentfill}{rgb}{0.121569,0.466667,0.705882}%
\pgfsetfillcolor{currentfill}%
\pgfsetfillopacity{0.306093}%
\pgfsetlinewidth{1.003750pt}%
\definecolor{currentstroke}{rgb}{0.121569,0.466667,0.705882}%
\pgfsetstrokecolor{currentstroke}%
\pgfsetstrokeopacity{0.306093}%
\pgfsetdash{}{0pt}%
\pgfpathmoveto{\pgfqpoint{1.737673in}{3.334874in}}%
\pgfpathcurveto{\pgfqpoint{1.745909in}{3.334874in}}{\pgfqpoint{1.753809in}{3.338146in}}{\pgfqpoint{1.759633in}{3.343970in}}%
\pgfpathcurveto{\pgfqpoint{1.765457in}{3.349794in}}{\pgfqpoint{1.768729in}{3.357694in}}{\pgfqpoint{1.768729in}{3.365930in}}%
\pgfpathcurveto{\pgfqpoint{1.768729in}{3.374167in}}{\pgfqpoint{1.765457in}{3.382067in}}{\pgfqpoint{1.759633in}{3.387891in}}%
\pgfpathcurveto{\pgfqpoint{1.753809in}{3.393715in}}{\pgfqpoint{1.745909in}{3.396987in}}{\pgfqpoint{1.737673in}{3.396987in}}%
\pgfpathcurveto{\pgfqpoint{1.729436in}{3.396987in}}{\pgfqpoint{1.721536in}{3.393715in}}{\pgfqpoint{1.715712in}{3.387891in}}%
\pgfpathcurveto{\pgfqpoint{1.709888in}{3.382067in}}{\pgfqpoint{1.706616in}{3.374167in}}{\pgfqpoint{1.706616in}{3.365930in}}%
\pgfpathcurveto{\pgfqpoint{1.706616in}{3.357694in}}{\pgfqpoint{1.709888in}{3.349794in}}{\pgfqpoint{1.715712in}{3.343970in}}%
\pgfpathcurveto{\pgfqpoint{1.721536in}{3.338146in}}{\pgfqpoint{1.729436in}{3.334874in}}{\pgfqpoint{1.737673in}{3.334874in}}%
\pgfpathclose%
\pgfusepath{stroke,fill}%
\end{pgfscope}%
\begin{pgfscope}%
\pgfpathrectangle{\pgfqpoint{0.100000in}{0.212622in}}{\pgfqpoint{3.696000in}{3.696000in}}%
\pgfusepath{clip}%
\pgfsetbuttcap%
\pgfsetroundjoin%
\definecolor{currentfill}{rgb}{0.121569,0.466667,0.705882}%
\pgfsetfillcolor{currentfill}%
\pgfsetfillopacity{0.306696}%
\pgfsetlinewidth{1.003750pt}%
\definecolor{currentstroke}{rgb}{0.121569,0.466667,0.705882}%
\pgfsetstrokecolor{currentstroke}%
\pgfsetstrokeopacity{0.306696}%
\pgfsetdash{}{0pt}%
\pgfpathmoveto{\pgfqpoint{1.736631in}{3.332238in}}%
\pgfpathcurveto{\pgfqpoint{1.744867in}{3.332238in}}{\pgfqpoint{1.752767in}{3.335511in}}{\pgfqpoint{1.758591in}{3.341335in}}%
\pgfpathcurveto{\pgfqpoint{1.764415in}{3.347159in}}{\pgfqpoint{1.767687in}{3.355059in}}{\pgfqpoint{1.767687in}{3.363295in}}%
\pgfpathcurveto{\pgfqpoint{1.767687in}{3.371531in}}{\pgfqpoint{1.764415in}{3.379431in}}{\pgfqpoint{1.758591in}{3.385255in}}%
\pgfpathcurveto{\pgfqpoint{1.752767in}{3.391079in}}{\pgfqpoint{1.744867in}{3.394351in}}{\pgfqpoint{1.736631in}{3.394351in}}%
\pgfpathcurveto{\pgfqpoint{1.728395in}{3.394351in}}{\pgfqpoint{1.720495in}{3.391079in}}{\pgfqpoint{1.714671in}{3.385255in}}%
\pgfpathcurveto{\pgfqpoint{1.708847in}{3.379431in}}{\pgfqpoint{1.705574in}{3.371531in}}{\pgfqpoint{1.705574in}{3.363295in}}%
\pgfpathcurveto{\pgfqpoint{1.705574in}{3.355059in}}{\pgfqpoint{1.708847in}{3.347159in}}{\pgfqpoint{1.714671in}{3.341335in}}%
\pgfpathcurveto{\pgfqpoint{1.720495in}{3.335511in}}{\pgfqpoint{1.728395in}{3.332238in}}{\pgfqpoint{1.736631in}{3.332238in}}%
\pgfpathclose%
\pgfusepath{stroke,fill}%
\end{pgfscope}%
\begin{pgfscope}%
\pgfpathrectangle{\pgfqpoint{0.100000in}{0.212622in}}{\pgfqpoint{3.696000in}{3.696000in}}%
\pgfusepath{clip}%
\pgfsetbuttcap%
\pgfsetroundjoin%
\definecolor{currentfill}{rgb}{0.121569,0.466667,0.705882}%
\pgfsetfillcolor{currentfill}%
\pgfsetfillopacity{0.306800}%
\pgfsetlinewidth{1.003750pt}%
\definecolor{currentstroke}{rgb}{0.121569,0.466667,0.705882}%
\pgfsetstrokecolor{currentstroke}%
\pgfsetstrokeopacity{0.306800}%
\pgfsetdash{}{0pt}%
\pgfpathmoveto{\pgfqpoint{1.797895in}{3.350572in}}%
\pgfpathcurveto{\pgfqpoint{1.806132in}{3.350572in}}{\pgfqpoint{1.814032in}{3.353844in}}{\pgfqpoint{1.819856in}{3.359668in}}%
\pgfpathcurveto{\pgfqpoint{1.825680in}{3.365492in}}{\pgfqpoint{1.828952in}{3.373392in}}{\pgfqpoint{1.828952in}{3.381629in}}%
\pgfpathcurveto{\pgfqpoint{1.828952in}{3.389865in}}{\pgfqpoint{1.825680in}{3.397765in}}{\pgfqpoint{1.819856in}{3.403589in}}%
\pgfpathcurveto{\pgfqpoint{1.814032in}{3.409413in}}{\pgfqpoint{1.806132in}{3.412685in}}{\pgfqpoint{1.797895in}{3.412685in}}%
\pgfpathcurveto{\pgfqpoint{1.789659in}{3.412685in}}{\pgfqpoint{1.781759in}{3.409413in}}{\pgfqpoint{1.775935in}{3.403589in}}%
\pgfpathcurveto{\pgfqpoint{1.770111in}{3.397765in}}{\pgfqpoint{1.766839in}{3.389865in}}{\pgfqpoint{1.766839in}{3.381629in}}%
\pgfpathcurveto{\pgfqpoint{1.766839in}{3.373392in}}{\pgfqpoint{1.770111in}{3.365492in}}{\pgfqpoint{1.775935in}{3.359668in}}%
\pgfpathcurveto{\pgfqpoint{1.781759in}{3.353844in}}{\pgfqpoint{1.789659in}{3.350572in}}{\pgfqpoint{1.797895in}{3.350572in}}%
\pgfpathclose%
\pgfusepath{stroke,fill}%
\end{pgfscope}%
\begin{pgfscope}%
\pgfpathrectangle{\pgfqpoint{0.100000in}{0.212622in}}{\pgfqpoint{3.696000in}{3.696000in}}%
\pgfusepath{clip}%
\pgfsetbuttcap%
\pgfsetroundjoin%
\definecolor{currentfill}{rgb}{0.121569,0.466667,0.705882}%
\pgfsetfillcolor{currentfill}%
\pgfsetfillopacity{0.306931}%
\pgfsetlinewidth{1.003750pt}%
\definecolor{currentstroke}{rgb}{0.121569,0.466667,0.705882}%
\pgfsetstrokecolor{currentstroke}%
\pgfsetstrokeopacity{0.306931}%
\pgfsetdash{}{0pt}%
\pgfpathmoveto{\pgfqpoint{1.736219in}{3.331209in}}%
\pgfpathcurveto{\pgfqpoint{1.744455in}{3.331209in}}{\pgfqpoint{1.752355in}{3.334481in}}{\pgfqpoint{1.758179in}{3.340305in}}%
\pgfpathcurveto{\pgfqpoint{1.764003in}{3.346129in}}{\pgfqpoint{1.767276in}{3.354029in}}{\pgfqpoint{1.767276in}{3.362266in}}%
\pgfpathcurveto{\pgfqpoint{1.767276in}{3.370502in}}{\pgfqpoint{1.764003in}{3.378402in}}{\pgfqpoint{1.758179in}{3.384226in}}%
\pgfpathcurveto{\pgfqpoint{1.752355in}{3.390050in}}{\pgfqpoint{1.744455in}{3.393322in}}{\pgfqpoint{1.736219in}{3.393322in}}%
\pgfpathcurveto{\pgfqpoint{1.727983in}{3.393322in}}{\pgfqpoint{1.720083in}{3.390050in}}{\pgfqpoint{1.714259in}{3.384226in}}%
\pgfpathcurveto{\pgfqpoint{1.708435in}{3.378402in}}{\pgfqpoint{1.705163in}{3.370502in}}{\pgfqpoint{1.705163in}{3.362266in}}%
\pgfpathcurveto{\pgfqpoint{1.705163in}{3.354029in}}{\pgfqpoint{1.708435in}{3.346129in}}{\pgfqpoint{1.714259in}{3.340305in}}%
\pgfpathcurveto{\pgfqpoint{1.720083in}{3.334481in}}{\pgfqpoint{1.727983in}{3.331209in}}{\pgfqpoint{1.736219in}{3.331209in}}%
\pgfpathclose%
\pgfusepath{stroke,fill}%
\end{pgfscope}%
\begin{pgfscope}%
\pgfpathrectangle{\pgfqpoint{0.100000in}{0.212622in}}{\pgfqpoint{3.696000in}{3.696000in}}%
\pgfusepath{clip}%
\pgfsetbuttcap%
\pgfsetroundjoin%
\definecolor{currentfill}{rgb}{0.121569,0.466667,0.705882}%
\pgfsetfillcolor{currentfill}%
\pgfsetfillopacity{0.307358}%
\pgfsetlinewidth{1.003750pt}%
\definecolor{currentstroke}{rgb}{0.121569,0.466667,0.705882}%
\pgfsetstrokecolor{currentstroke}%
\pgfsetstrokeopacity{0.307358}%
\pgfsetdash{}{0pt}%
\pgfpathmoveto{\pgfqpoint{1.735454in}{3.329354in}}%
\pgfpathcurveto{\pgfqpoint{1.743690in}{3.329354in}}{\pgfqpoint{1.751590in}{3.332627in}}{\pgfqpoint{1.757414in}{3.338451in}}%
\pgfpathcurveto{\pgfqpoint{1.763238in}{3.344274in}}{\pgfqpoint{1.766510in}{3.352175in}}{\pgfqpoint{1.766510in}{3.360411in}}%
\pgfpathcurveto{\pgfqpoint{1.766510in}{3.368647in}}{\pgfqpoint{1.763238in}{3.376547in}}{\pgfqpoint{1.757414in}{3.382371in}}%
\pgfpathcurveto{\pgfqpoint{1.751590in}{3.388195in}}{\pgfqpoint{1.743690in}{3.391467in}}{\pgfqpoint{1.735454in}{3.391467in}}%
\pgfpathcurveto{\pgfqpoint{1.727217in}{3.391467in}}{\pgfqpoint{1.719317in}{3.388195in}}{\pgfqpoint{1.713493in}{3.382371in}}%
\pgfpathcurveto{\pgfqpoint{1.707669in}{3.376547in}}{\pgfqpoint{1.704397in}{3.368647in}}{\pgfqpoint{1.704397in}{3.360411in}}%
\pgfpathcurveto{\pgfqpoint{1.704397in}{3.352175in}}{\pgfqpoint{1.707669in}{3.344274in}}{\pgfqpoint{1.713493in}{3.338451in}}%
\pgfpathcurveto{\pgfqpoint{1.719317in}{3.332627in}}{\pgfqpoint{1.727217in}{3.329354in}}{\pgfqpoint{1.735454in}{3.329354in}}%
\pgfpathclose%
\pgfusepath{stroke,fill}%
\end{pgfscope}%
\begin{pgfscope}%
\pgfpathrectangle{\pgfqpoint{0.100000in}{0.212622in}}{\pgfqpoint{3.696000in}{3.696000in}}%
\pgfusepath{clip}%
\pgfsetbuttcap%
\pgfsetroundjoin%
\definecolor{currentfill}{rgb}{0.121569,0.466667,0.705882}%
\pgfsetfillcolor{currentfill}%
\pgfsetfillopacity{0.307362}%
\pgfsetlinewidth{1.003750pt}%
\definecolor{currentstroke}{rgb}{0.121569,0.466667,0.705882}%
\pgfsetstrokecolor{currentstroke}%
\pgfsetstrokeopacity{0.307362}%
\pgfsetdash{}{0pt}%
\pgfpathmoveto{\pgfqpoint{1.735445in}{3.329333in}}%
\pgfpathcurveto{\pgfqpoint{1.743681in}{3.329333in}}{\pgfqpoint{1.751581in}{3.332606in}}{\pgfqpoint{1.757405in}{3.338429in}}%
\pgfpathcurveto{\pgfqpoint{1.763229in}{3.344253in}}{\pgfqpoint{1.766501in}{3.352153in}}{\pgfqpoint{1.766501in}{3.360390in}}%
\pgfpathcurveto{\pgfqpoint{1.766501in}{3.368626in}}{\pgfqpoint{1.763229in}{3.376526in}}{\pgfqpoint{1.757405in}{3.382350in}}%
\pgfpathcurveto{\pgfqpoint{1.751581in}{3.388174in}}{\pgfqpoint{1.743681in}{3.391446in}}{\pgfqpoint{1.735445in}{3.391446in}}%
\pgfpathcurveto{\pgfqpoint{1.727208in}{3.391446in}}{\pgfqpoint{1.719308in}{3.388174in}}{\pgfqpoint{1.713484in}{3.382350in}}%
\pgfpathcurveto{\pgfqpoint{1.707660in}{3.376526in}}{\pgfqpoint{1.704388in}{3.368626in}}{\pgfqpoint{1.704388in}{3.360390in}}%
\pgfpathcurveto{\pgfqpoint{1.704388in}{3.352153in}}{\pgfqpoint{1.707660in}{3.344253in}}{\pgfqpoint{1.713484in}{3.338429in}}%
\pgfpathcurveto{\pgfqpoint{1.719308in}{3.332606in}}{\pgfqpoint{1.727208in}{3.329333in}}{\pgfqpoint{1.735445in}{3.329333in}}%
\pgfpathclose%
\pgfusepath{stroke,fill}%
\end{pgfscope}%
\begin{pgfscope}%
\pgfpathrectangle{\pgfqpoint{0.100000in}{0.212622in}}{\pgfqpoint{3.696000in}{3.696000in}}%
\pgfusepath{clip}%
\pgfsetbuttcap%
\pgfsetroundjoin%
\definecolor{currentfill}{rgb}{0.121569,0.466667,0.705882}%
\pgfsetfillcolor{currentfill}%
\pgfsetfillopacity{0.307371}%
\pgfsetlinewidth{1.003750pt}%
\definecolor{currentstroke}{rgb}{0.121569,0.466667,0.705882}%
\pgfsetstrokecolor{currentstroke}%
\pgfsetstrokeopacity{0.307371}%
\pgfsetdash{}{0pt}%
\pgfpathmoveto{\pgfqpoint{1.735429in}{3.329295in}}%
\pgfpathcurveto{\pgfqpoint{1.743665in}{3.329295in}}{\pgfqpoint{1.751565in}{3.332568in}}{\pgfqpoint{1.757389in}{3.338392in}}%
\pgfpathcurveto{\pgfqpoint{1.763213in}{3.344215in}}{\pgfqpoint{1.766485in}{3.352116in}}{\pgfqpoint{1.766485in}{3.360352in}}%
\pgfpathcurveto{\pgfqpoint{1.766485in}{3.368588in}}{\pgfqpoint{1.763213in}{3.376488in}}{\pgfqpoint{1.757389in}{3.382312in}}%
\pgfpathcurveto{\pgfqpoint{1.751565in}{3.388136in}}{\pgfqpoint{1.743665in}{3.391408in}}{\pgfqpoint{1.735429in}{3.391408in}}%
\pgfpathcurveto{\pgfqpoint{1.727192in}{3.391408in}}{\pgfqpoint{1.719292in}{3.388136in}}{\pgfqpoint{1.713468in}{3.382312in}}%
\pgfpathcurveto{\pgfqpoint{1.707644in}{3.376488in}}{\pgfqpoint{1.704372in}{3.368588in}}{\pgfqpoint{1.704372in}{3.360352in}}%
\pgfpathcurveto{\pgfqpoint{1.704372in}{3.352116in}}{\pgfqpoint{1.707644in}{3.344215in}}{\pgfqpoint{1.713468in}{3.338392in}}%
\pgfpathcurveto{\pgfqpoint{1.719292in}{3.332568in}}{\pgfqpoint{1.727192in}{3.329295in}}{\pgfqpoint{1.735429in}{3.329295in}}%
\pgfpathclose%
\pgfusepath{stroke,fill}%
\end{pgfscope}%
\begin{pgfscope}%
\pgfpathrectangle{\pgfqpoint{0.100000in}{0.212622in}}{\pgfqpoint{3.696000in}{3.696000in}}%
\pgfusepath{clip}%
\pgfsetbuttcap%
\pgfsetroundjoin%
\definecolor{currentfill}{rgb}{0.121569,0.466667,0.705882}%
\pgfsetfillcolor{currentfill}%
\pgfsetfillopacity{0.307387}%
\pgfsetlinewidth{1.003750pt}%
\definecolor{currentstroke}{rgb}{0.121569,0.466667,0.705882}%
\pgfsetstrokecolor{currentstroke}%
\pgfsetstrokeopacity{0.307387}%
\pgfsetdash{}{0pt}%
\pgfpathmoveto{\pgfqpoint{1.735399in}{3.329226in}}%
\pgfpathcurveto{\pgfqpoint{1.743636in}{3.329226in}}{\pgfqpoint{1.751536in}{3.332498in}}{\pgfqpoint{1.757360in}{3.338322in}}%
\pgfpathcurveto{\pgfqpoint{1.763184in}{3.344146in}}{\pgfqpoint{1.766456in}{3.352046in}}{\pgfqpoint{1.766456in}{3.360282in}}%
\pgfpathcurveto{\pgfqpoint{1.766456in}{3.368518in}}{\pgfqpoint{1.763184in}{3.376418in}}{\pgfqpoint{1.757360in}{3.382242in}}%
\pgfpathcurveto{\pgfqpoint{1.751536in}{3.388066in}}{\pgfqpoint{1.743636in}{3.391339in}}{\pgfqpoint{1.735399in}{3.391339in}}%
\pgfpathcurveto{\pgfqpoint{1.727163in}{3.391339in}}{\pgfqpoint{1.719263in}{3.388066in}}{\pgfqpoint{1.713439in}{3.382242in}}%
\pgfpathcurveto{\pgfqpoint{1.707615in}{3.376418in}}{\pgfqpoint{1.704343in}{3.368518in}}{\pgfqpoint{1.704343in}{3.360282in}}%
\pgfpathcurveto{\pgfqpoint{1.704343in}{3.352046in}}{\pgfqpoint{1.707615in}{3.344146in}}{\pgfqpoint{1.713439in}{3.338322in}}%
\pgfpathcurveto{\pgfqpoint{1.719263in}{3.332498in}}{\pgfqpoint{1.727163in}{3.329226in}}{\pgfqpoint{1.735399in}{3.329226in}}%
\pgfpathclose%
\pgfusepath{stroke,fill}%
\end{pgfscope}%
\begin{pgfscope}%
\pgfpathrectangle{\pgfqpoint{0.100000in}{0.212622in}}{\pgfqpoint{3.696000in}{3.696000in}}%
\pgfusepath{clip}%
\pgfsetbuttcap%
\pgfsetroundjoin%
\definecolor{currentfill}{rgb}{0.121569,0.466667,0.705882}%
\pgfsetfillcolor{currentfill}%
\pgfsetfillopacity{0.307415}%
\pgfsetlinewidth{1.003750pt}%
\definecolor{currentstroke}{rgb}{0.121569,0.466667,0.705882}%
\pgfsetstrokecolor{currentstroke}%
\pgfsetstrokeopacity{0.307415}%
\pgfsetdash{}{0pt}%
\pgfpathmoveto{\pgfqpoint{1.735348in}{3.329100in}}%
\pgfpathcurveto{\pgfqpoint{1.743585in}{3.329100in}}{\pgfqpoint{1.751485in}{3.332372in}}{\pgfqpoint{1.757309in}{3.338196in}}%
\pgfpathcurveto{\pgfqpoint{1.763133in}{3.344020in}}{\pgfqpoint{1.766405in}{3.351920in}}{\pgfqpoint{1.766405in}{3.360156in}}%
\pgfpathcurveto{\pgfqpoint{1.766405in}{3.368393in}}{\pgfqpoint{1.763133in}{3.376293in}}{\pgfqpoint{1.757309in}{3.382117in}}%
\pgfpathcurveto{\pgfqpoint{1.751485in}{3.387941in}}{\pgfqpoint{1.743585in}{3.391213in}}{\pgfqpoint{1.735348in}{3.391213in}}%
\pgfpathcurveto{\pgfqpoint{1.727112in}{3.391213in}}{\pgfqpoint{1.719212in}{3.387941in}}{\pgfqpoint{1.713388in}{3.382117in}}%
\pgfpathcurveto{\pgfqpoint{1.707564in}{3.376293in}}{\pgfqpoint{1.704292in}{3.368393in}}{\pgfqpoint{1.704292in}{3.360156in}}%
\pgfpathcurveto{\pgfqpoint{1.704292in}{3.351920in}}{\pgfqpoint{1.707564in}{3.344020in}}{\pgfqpoint{1.713388in}{3.338196in}}%
\pgfpathcurveto{\pgfqpoint{1.719212in}{3.332372in}}{\pgfqpoint{1.727112in}{3.329100in}}{\pgfqpoint{1.735348in}{3.329100in}}%
\pgfpathclose%
\pgfusepath{stroke,fill}%
\end{pgfscope}%
\begin{pgfscope}%
\pgfpathrectangle{\pgfqpoint{0.100000in}{0.212622in}}{\pgfqpoint{3.696000in}{3.696000in}}%
\pgfusepath{clip}%
\pgfsetbuttcap%
\pgfsetroundjoin%
\definecolor{currentfill}{rgb}{0.121569,0.466667,0.705882}%
\pgfsetfillcolor{currentfill}%
\pgfsetfillopacity{0.307437}%
\pgfsetlinewidth{1.003750pt}%
\definecolor{currentstroke}{rgb}{0.121569,0.466667,0.705882}%
\pgfsetstrokecolor{currentstroke}%
\pgfsetstrokeopacity{0.307437}%
\pgfsetdash{}{0pt}%
\pgfpathmoveto{\pgfqpoint{1.799254in}{3.348395in}}%
\pgfpathcurveto{\pgfqpoint{1.807491in}{3.348395in}}{\pgfqpoint{1.815391in}{3.351667in}}{\pgfqpoint{1.821215in}{3.357491in}}%
\pgfpathcurveto{\pgfqpoint{1.827038in}{3.363315in}}{\pgfqpoint{1.830311in}{3.371215in}}{\pgfqpoint{1.830311in}{3.379451in}}%
\pgfpathcurveto{\pgfqpoint{1.830311in}{3.387687in}}{\pgfqpoint{1.827038in}{3.395588in}}{\pgfqpoint{1.821215in}{3.401411in}}%
\pgfpathcurveto{\pgfqpoint{1.815391in}{3.407235in}}{\pgfqpoint{1.807491in}{3.410508in}}{\pgfqpoint{1.799254in}{3.410508in}}%
\pgfpathcurveto{\pgfqpoint{1.791018in}{3.410508in}}{\pgfqpoint{1.783118in}{3.407235in}}{\pgfqpoint{1.777294in}{3.401411in}}%
\pgfpathcurveto{\pgfqpoint{1.771470in}{3.395588in}}{\pgfqpoint{1.768198in}{3.387687in}}{\pgfqpoint{1.768198in}{3.379451in}}%
\pgfpathcurveto{\pgfqpoint{1.768198in}{3.371215in}}{\pgfqpoint{1.771470in}{3.363315in}}{\pgfqpoint{1.777294in}{3.357491in}}%
\pgfpathcurveto{\pgfqpoint{1.783118in}{3.351667in}}{\pgfqpoint{1.791018in}{3.348395in}}{\pgfqpoint{1.799254in}{3.348395in}}%
\pgfpathclose%
\pgfusepath{stroke,fill}%
\end{pgfscope}%
\begin{pgfscope}%
\pgfpathrectangle{\pgfqpoint{0.100000in}{0.212622in}}{\pgfqpoint{3.696000in}{3.696000in}}%
\pgfusepath{clip}%
\pgfsetbuttcap%
\pgfsetroundjoin%
\definecolor{currentfill}{rgb}{0.121569,0.466667,0.705882}%
\pgfsetfillcolor{currentfill}%
\pgfsetfillopacity{0.307467}%
\pgfsetlinewidth{1.003750pt}%
\definecolor{currentstroke}{rgb}{0.121569,0.466667,0.705882}%
\pgfsetstrokecolor{currentstroke}%
\pgfsetstrokeopacity{0.307467}%
\pgfsetdash{}{0pt}%
\pgfpathmoveto{\pgfqpoint{1.735252in}{3.328870in}}%
\pgfpathcurveto{\pgfqpoint{1.743488in}{3.328870in}}{\pgfqpoint{1.751388in}{3.332142in}}{\pgfqpoint{1.757212in}{3.337966in}}%
\pgfpathcurveto{\pgfqpoint{1.763036in}{3.343790in}}{\pgfqpoint{1.766309in}{3.351690in}}{\pgfqpoint{1.766309in}{3.359927in}}%
\pgfpathcurveto{\pgfqpoint{1.766309in}{3.368163in}}{\pgfqpoint{1.763036in}{3.376063in}}{\pgfqpoint{1.757212in}{3.381887in}}%
\pgfpathcurveto{\pgfqpoint{1.751388in}{3.387711in}}{\pgfqpoint{1.743488in}{3.390983in}}{\pgfqpoint{1.735252in}{3.390983in}}%
\pgfpathcurveto{\pgfqpoint{1.727016in}{3.390983in}}{\pgfqpoint{1.719116in}{3.387711in}}{\pgfqpoint{1.713292in}{3.381887in}}%
\pgfpathcurveto{\pgfqpoint{1.707468in}{3.376063in}}{\pgfqpoint{1.704196in}{3.368163in}}{\pgfqpoint{1.704196in}{3.359927in}}%
\pgfpathcurveto{\pgfqpoint{1.704196in}{3.351690in}}{\pgfqpoint{1.707468in}{3.343790in}}{\pgfqpoint{1.713292in}{3.337966in}}%
\pgfpathcurveto{\pgfqpoint{1.719116in}{3.332142in}}{\pgfqpoint{1.727016in}{3.328870in}}{\pgfqpoint{1.735252in}{3.328870in}}%
\pgfpathclose%
\pgfusepath{stroke,fill}%
\end{pgfscope}%
\begin{pgfscope}%
\pgfpathrectangle{\pgfqpoint{0.100000in}{0.212622in}}{\pgfqpoint{3.696000in}{3.696000in}}%
\pgfusepath{clip}%
\pgfsetbuttcap%
\pgfsetroundjoin%
\definecolor{currentfill}{rgb}{0.121569,0.466667,0.705882}%
\pgfsetfillcolor{currentfill}%
\pgfsetfillopacity{0.307559}%
\pgfsetlinewidth{1.003750pt}%
\definecolor{currentstroke}{rgb}{0.121569,0.466667,0.705882}%
\pgfsetstrokecolor{currentstroke}%
\pgfsetstrokeopacity{0.307559}%
\pgfsetdash{}{0pt}%
\pgfpathmoveto{\pgfqpoint{1.735075in}{3.328451in}}%
\pgfpathcurveto{\pgfqpoint{1.743311in}{3.328451in}}{\pgfqpoint{1.751211in}{3.331723in}}{\pgfqpoint{1.757035in}{3.337547in}}%
\pgfpathcurveto{\pgfqpoint{1.762859in}{3.343371in}}{\pgfqpoint{1.766131in}{3.351271in}}{\pgfqpoint{1.766131in}{3.359507in}}%
\pgfpathcurveto{\pgfqpoint{1.766131in}{3.367744in}}{\pgfqpoint{1.762859in}{3.375644in}}{\pgfqpoint{1.757035in}{3.381468in}}%
\pgfpathcurveto{\pgfqpoint{1.751211in}{3.387291in}}{\pgfqpoint{1.743311in}{3.390564in}}{\pgfqpoint{1.735075in}{3.390564in}}%
\pgfpathcurveto{\pgfqpoint{1.726839in}{3.390564in}}{\pgfqpoint{1.718939in}{3.387291in}}{\pgfqpoint{1.713115in}{3.381468in}}%
\pgfpathcurveto{\pgfqpoint{1.707291in}{3.375644in}}{\pgfqpoint{1.704018in}{3.367744in}}{\pgfqpoint{1.704018in}{3.359507in}}%
\pgfpathcurveto{\pgfqpoint{1.704018in}{3.351271in}}{\pgfqpoint{1.707291in}{3.343371in}}{\pgfqpoint{1.713115in}{3.337547in}}%
\pgfpathcurveto{\pgfqpoint{1.718939in}{3.331723in}}{\pgfqpoint{1.726839in}{3.328451in}}{\pgfqpoint{1.735075in}{3.328451in}}%
\pgfpathclose%
\pgfusepath{stroke,fill}%
\end{pgfscope}%
\begin{pgfscope}%
\pgfpathrectangle{\pgfqpoint{0.100000in}{0.212622in}}{\pgfqpoint{3.696000in}{3.696000in}}%
\pgfusepath{clip}%
\pgfsetbuttcap%
\pgfsetroundjoin%
\definecolor{currentfill}{rgb}{0.121569,0.466667,0.705882}%
\pgfsetfillcolor{currentfill}%
\pgfsetfillopacity{0.307730}%
\pgfsetlinewidth{1.003750pt}%
\definecolor{currentstroke}{rgb}{0.121569,0.466667,0.705882}%
\pgfsetstrokecolor{currentstroke}%
\pgfsetstrokeopacity{0.307730}%
\pgfsetdash{}{0pt}%
\pgfpathmoveto{\pgfqpoint{1.734761in}{3.327689in}}%
\pgfpathcurveto{\pgfqpoint{1.742998in}{3.327689in}}{\pgfqpoint{1.750898in}{3.330962in}}{\pgfqpoint{1.756722in}{3.336786in}}%
\pgfpathcurveto{\pgfqpoint{1.762546in}{3.342610in}}{\pgfqpoint{1.765818in}{3.350510in}}{\pgfqpoint{1.765818in}{3.358746in}}%
\pgfpathcurveto{\pgfqpoint{1.765818in}{3.366982in}}{\pgfqpoint{1.762546in}{3.374882in}}{\pgfqpoint{1.756722in}{3.380706in}}%
\pgfpathcurveto{\pgfqpoint{1.750898in}{3.386530in}}{\pgfqpoint{1.742998in}{3.389802in}}{\pgfqpoint{1.734761in}{3.389802in}}%
\pgfpathcurveto{\pgfqpoint{1.726525in}{3.389802in}}{\pgfqpoint{1.718625in}{3.386530in}}{\pgfqpoint{1.712801in}{3.380706in}}%
\pgfpathcurveto{\pgfqpoint{1.706977in}{3.374882in}}{\pgfqpoint{1.703705in}{3.366982in}}{\pgfqpoint{1.703705in}{3.358746in}}%
\pgfpathcurveto{\pgfqpoint{1.703705in}{3.350510in}}{\pgfqpoint{1.706977in}{3.342610in}}{\pgfqpoint{1.712801in}{3.336786in}}%
\pgfpathcurveto{\pgfqpoint{1.718625in}{3.330962in}}{\pgfqpoint{1.726525in}{3.327689in}}{\pgfqpoint{1.734761in}{3.327689in}}%
\pgfpathclose%
\pgfusepath{stroke,fill}%
\end{pgfscope}%
\begin{pgfscope}%
\pgfpathrectangle{\pgfqpoint{0.100000in}{0.212622in}}{\pgfqpoint{3.696000in}{3.696000in}}%
\pgfusepath{clip}%
\pgfsetbuttcap%
\pgfsetroundjoin%
\definecolor{currentfill}{rgb}{0.121569,0.466667,0.705882}%
\pgfsetfillcolor{currentfill}%
\pgfsetfillopacity{0.308042}%
\pgfsetlinewidth{1.003750pt}%
\definecolor{currentstroke}{rgb}{0.121569,0.466667,0.705882}%
\pgfsetstrokecolor{currentstroke}%
\pgfsetstrokeopacity{0.308042}%
\pgfsetdash{}{0pt}%
\pgfpathmoveto{\pgfqpoint{1.734189in}{3.326316in}}%
\pgfpathcurveto{\pgfqpoint{1.742425in}{3.326316in}}{\pgfqpoint{1.750325in}{3.329589in}}{\pgfqpoint{1.756149in}{3.335412in}}%
\pgfpathcurveto{\pgfqpoint{1.761973in}{3.341236in}}{\pgfqpoint{1.765245in}{3.349136in}}{\pgfqpoint{1.765245in}{3.357373in}}%
\pgfpathcurveto{\pgfqpoint{1.765245in}{3.365609in}}{\pgfqpoint{1.761973in}{3.373509in}}{\pgfqpoint{1.756149in}{3.379333in}}%
\pgfpathcurveto{\pgfqpoint{1.750325in}{3.385157in}}{\pgfqpoint{1.742425in}{3.388429in}}{\pgfqpoint{1.734189in}{3.388429in}}%
\pgfpathcurveto{\pgfqpoint{1.725952in}{3.388429in}}{\pgfqpoint{1.718052in}{3.385157in}}{\pgfqpoint{1.712228in}{3.379333in}}%
\pgfpathcurveto{\pgfqpoint{1.706404in}{3.373509in}}{\pgfqpoint{1.703132in}{3.365609in}}{\pgfqpoint{1.703132in}{3.357373in}}%
\pgfpathcurveto{\pgfqpoint{1.703132in}{3.349136in}}{\pgfqpoint{1.706404in}{3.341236in}}{\pgfqpoint{1.712228in}{3.335412in}}%
\pgfpathcurveto{\pgfqpoint{1.718052in}{3.329589in}}{\pgfqpoint{1.725952in}{3.326316in}}{\pgfqpoint{1.734189in}{3.326316in}}%
\pgfpathclose%
\pgfusepath{stroke,fill}%
\end{pgfscope}%
\begin{pgfscope}%
\pgfpathrectangle{\pgfqpoint{0.100000in}{0.212622in}}{\pgfqpoint{3.696000in}{3.696000in}}%
\pgfusepath{clip}%
\pgfsetbuttcap%
\pgfsetroundjoin%
\definecolor{currentfill}{rgb}{0.121569,0.466667,0.705882}%
\pgfsetfillcolor{currentfill}%
\pgfsetfillopacity{0.308611}%
\pgfsetlinewidth{1.003750pt}%
\definecolor{currentstroke}{rgb}{0.121569,0.466667,0.705882}%
\pgfsetstrokecolor{currentstroke}%
\pgfsetstrokeopacity{0.308611}%
\pgfsetdash{}{0pt}%
\pgfpathmoveto{\pgfqpoint{1.733152in}{3.323817in}}%
\pgfpathcurveto{\pgfqpoint{1.741388in}{3.323817in}}{\pgfqpoint{1.749288in}{3.327090in}}{\pgfqpoint{1.755112in}{3.332914in}}%
\pgfpathcurveto{\pgfqpoint{1.760936in}{3.338738in}}{\pgfqpoint{1.764208in}{3.346638in}}{\pgfqpoint{1.764208in}{3.354874in}}%
\pgfpathcurveto{\pgfqpoint{1.764208in}{3.363110in}}{\pgfqpoint{1.760936in}{3.371010in}}{\pgfqpoint{1.755112in}{3.376834in}}%
\pgfpathcurveto{\pgfqpoint{1.749288in}{3.382658in}}{\pgfqpoint{1.741388in}{3.385930in}}{\pgfqpoint{1.733152in}{3.385930in}}%
\pgfpathcurveto{\pgfqpoint{1.724915in}{3.385930in}}{\pgfqpoint{1.717015in}{3.382658in}}{\pgfqpoint{1.711191in}{3.376834in}}%
\pgfpathcurveto{\pgfqpoint{1.705368in}{3.371010in}}{\pgfqpoint{1.702095in}{3.363110in}}{\pgfqpoint{1.702095in}{3.354874in}}%
\pgfpathcurveto{\pgfqpoint{1.702095in}{3.346638in}}{\pgfqpoint{1.705368in}{3.338738in}}{\pgfqpoint{1.711191in}{3.332914in}}%
\pgfpathcurveto{\pgfqpoint{1.717015in}{3.327090in}}{\pgfqpoint{1.724915in}{3.323817in}}{\pgfqpoint{1.733152in}{3.323817in}}%
\pgfpathclose%
\pgfusepath{stroke,fill}%
\end{pgfscope}%
\begin{pgfscope}%
\pgfpathrectangle{\pgfqpoint{0.100000in}{0.212622in}}{\pgfqpoint{3.696000in}{3.696000in}}%
\pgfusepath{clip}%
\pgfsetbuttcap%
\pgfsetroundjoin%
\definecolor{currentfill}{rgb}{0.121569,0.466667,0.705882}%
\pgfsetfillcolor{currentfill}%
\pgfsetfillopacity{0.308885}%
\pgfsetlinewidth{1.003750pt}%
\definecolor{currentstroke}{rgb}{0.121569,0.466667,0.705882}%
\pgfsetstrokecolor{currentstroke}%
\pgfsetstrokeopacity{0.308885}%
\pgfsetdash{}{0pt}%
\pgfpathmoveto{\pgfqpoint{1.802005in}{3.343526in}}%
\pgfpathcurveto{\pgfqpoint{1.810241in}{3.343526in}}{\pgfqpoint{1.818141in}{3.346799in}}{\pgfqpoint{1.823965in}{3.352623in}}%
\pgfpathcurveto{\pgfqpoint{1.829789in}{3.358447in}}{\pgfqpoint{1.833061in}{3.366347in}}{\pgfqpoint{1.833061in}{3.374583in}}%
\pgfpathcurveto{\pgfqpoint{1.833061in}{3.382819in}}{\pgfqpoint{1.829789in}{3.390719in}}{\pgfqpoint{1.823965in}{3.396543in}}%
\pgfpathcurveto{\pgfqpoint{1.818141in}{3.402367in}}{\pgfqpoint{1.810241in}{3.405639in}}{\pgfqpoint{1.802005in}{3.405639in}}%
\pgfpathcurveto{\pgfqpoint{1.793768in}{3.405639in}}{\pgfqpoint{1.785868in}{3.402367in}}{\pgfqpoint{1.780044in}{3.396543in}}%
\pgfpathcurveto{\pgfqpoint{1.774220in}{3.390719in}}{\pgfqpoint{1.770948in}{3.382819in}}{\pgfqpoint{1.770948in}{3.374583in}}%
\pgfpathcurveto{\pgfqpoint{1.770948in}{3.366347in}}{\pgfqpoint{1.774220in}{3.358447in}}{\pgfqpoint{1.780044in}{3.352623in}}%
\pgfpathcurveto{\pgfqpoint{1.785868in}{3.346799in}}{\pgfqpoint{1.793768in}{3.343526in}}{\pgfqpoint{1.802005in}{3.343526in}}%
\pgfpathclose%
\pgfusepath{stroke,fill}%
\end{pgfscope}%
\begin{pgfscope}%
\pgfpathrectangle{\pgfqpoint{0.100000in}{0.212622in}}{\pgfqpoint{3.696000in}{3.696000in}}%
\pgfusepath{clip}%
\pgfsetbuttcap%
\pgfsetroundjoin%
\definecolor{currentfill}{rgb}{0.121569,0.466667,0.705882}%
\pgfsetfillcolor{currentfill}%
\pgfsetfillopacity{0.309638}%
\pgfsetlinewidth{1.003750pt}%
\definecolor{currentstroke}{rgb}{0.121569,0.466667,0.705882}%
\pgfsetstrokecolor{currentstroke}%
\pgfsetstrokeopacity{0.309638}%
\pgfsetdash{}{0pt}%
\pgfpathmoveto{\pgfqpoint{1.731295in}{3.319219in}}%
\pgfpathcurveto{\pgfqpoint{1.739531in}{3.319219in}}{\pgfqpoint{1.747431in}{3.322491in}}{\pgfqpoint{1.753255in}{3.328315in}}%
\pgfpathcurveto{\pgfqpoint{1.759079in}{3.334139in}}{\pgfqpoint{1.762351in}{3.342039in}}{\pgfqpoint{1.762351in}{3.350275in}}%
\pgfpathcurveto{\pgfqpoint{1.762351in}{3.358511in}}{\pgfqpoint{1.759079in}{3.366411in}}{\pgfqpoint{1.753255in}{3.372235in}}%
\pgfpathcurveto{\pgfqpoint{1.747431in}{3.378059in}}{\pgfqpoint{1.739531in}{3.381332in}}{\pgfqpoint{1.731295in}{3.381332in}}%
\pgfpathcurveto{\pgfqpoint{1.723059in}{3.381332in}}{\pgfqpoint{1.715159in}{3.378059in}}{\pgfqpoint{1.709335in}{3.372235in}}%
\pgfpathcurveto{\pgfqpoint{1.703511in}{3.366411in}}{\pgfqpoint{1.700238in}{3.358511in}}{\pgfqpoint{1.700238in}{3.350275in}}%
\pgfpathcurveto{\pgfqpoint{1.700238in}{3.342039in}}{\pgfqpoint{1.703511in}{3.334139in}}{\pgfqpoint{1.709335in}{3.328315in}}%
\pgfpathcurveto{\pgfqpoint{1.715159in}{3.322491in}}{\pgfqpoint{1.723059in}{3.319219in}}{\pgfqpoint{1.731295in}{3.319219in}}%
\pgfpathclose%
\pgfusepath{stroke,fill}%
\end{pgfscope}%
\begin{pgfscope}%
\pgfpathrectangle{\pgfqpoint{0.100000in}{0.212622in}}{\pgfqpoint{3.696000in}{3.696000in}}%
\pgfusepath{clip}%
\pgfsetbuttcap%
\pgfsetroundjoin%
\definecolor{currentfill}{rgb}{0.121569,0.466667,0.705882}%
\pgfsetfillcolor{currentfill}%
\pgfsetfillopacity{0.310348}%
\pgfsetlinewidth{1.003750pt}%
\definecolor{currentstroke}{rgb}{0.121569,0.466667,0.705882}%
\pgfsetstrokecolor{currentstroke}%
\pgfsetstrokeopacity{0.310348}%
\pgfsetdash{}{0pt}%
\pgfpathmoveto{\pgfqpoint{1.730000in}{3.316158in}}%
\pgfpathcurveto{\pgfqpoint{1.738236in}{3.316158in}}{\pgfqpoint{1.746136in}{3.319430in}}{\pgfqpoint{1.751960in}{3.325254in}}%
\pgfpathcurveto{\pgfqpoint{1.757784in}{3.331078in}}{\pgfqpoint{1.761056in}{3.338978in}}{\pgfqpoint{1.761056in}{3.347214in}}%
\pgfpathcurveto{\pgfqpoint{1.761056in}{3.355451in}}{\pgfqpoint{1.757784in}{3.363351in}}{\pgfqpoint{1.751960in}{3.369175in}}%
\pgfpathcurveto{\pgfqpoint{1.746136in}{3.374999in}}{\pgfqpoint{1.738236in}{3.378271in}}{\pgfqpoint{1.730000in}{3.378271in}}%
\pgfpathcurveto{\pgfqpoint{1.721764in}{3.378271in}}{\pgfqpoint{1.713864in}{3.374999in}}{\pgfqpoint{1.708040in}{3.369175in}}%
\pgfpathcurveto{\pgfqpoint{1.702216in}{3.363351in}}{\pgfqpoint{1.698943in}{3.355451in}}{\pgfqpoint{1.698943in}{3.347214in}}%
\pgfpathcurveto{\pgfqpoint{1.698943in}{3.338978in}}{\pgfqpoint{1.702216in}{3.331078in}}{\pgfqpoint{1.708040in}{3.325254in}}%
\pgfpathcurveto{\pgfqpoint{1.713864in}{3.319430in}}{\pgfqpoint{1.721764in}{3.316158in}}{\pgfqpoint{1.730000in}{3.316158in}}%
\pgfpathclose%
\pgfusepath{stroke,fill}%
\end{pgfscope}%
\begin{pgfscope}%
\pgfpathrectangle{\pgfqpoint{0.100000in}{0.212622in}}{\pgfqpoint{3.696000in}{3.696000in}}%
\pgfusepath{clip}%
\pgfsetbuttcap%
\pgfsetroundjoin%
\definecolor{currentfill}{rgb}{0.121569,0.466667,0.705882}%
\pgfsetfillcolor{currentfill}%
\pgfsetfillopacity{0.311255}%
\pgfsetlinewidth{1.003750pt}%
\definecolor{currentstroke}{rgb}{0.121569,0.466667,0.705882}%
\pgfsetstrokecolor{currentstroke}%
\pgfsetstrokeopacity{0.311255}%
\pgfsetdash{}{0pt}%
\pgfpathmoveto{\pgfqpoint{1.806218in}{3.335469in}}%
\pgfpathcurveto{\pgfqpoint{1.814454in}{3.335469in}}{\pgfqpoint{1.822354in}{3.338741in}}{\pgfqpoint{1.828178in}{3.344565in}}%
\pgfpathcurveto{\pgfqpoint{1.834002in}{3.350389in}}{\pgfqpoint{1.837275in}{3.358289in}}{\pgfqpoint{1.837275in}{3.366525in}}%
\pgfpathcurveto{\pgfqpoint{1.837275in}{3.374761in}}{\pgfqpoint{1.834002in}{3.382661in}}{\pgfqpoint{1.828178in}{3.388485in}}%
\pgfpathcurveto{\pgfqpoint{1.822354in}{3.394309in}}{\pgfqpoint{1.814454in}{3.397582in}}{\pgfqpoint{1.806218in}{3.397582in}}%
\pgfpathcurveto{\pgfqpoint{1.797982in}{3.397582in}}{\pgfqpoint{1.790082in}{3.394309in}}{\pgfqpoint{1.784258in}{3.388485in}}%
\pgfpathcurveto{\pgfqpoint{1.778434in}{3.382661in}}{\pgfqpoint{1.775162in}{3.374761in}}{\pgfqpoint{1.775162in}{3.366525in}}%
\pgfpathcurveto{\pgfqpoint{1.775162in}{3.358289in}}{\pgfqpoint{1.778434in}{3.350389in}}{\pgfqpoint{1.784258in}{3.344565in}}%
\pgfpathcurveto{\pgfqpoint{1.790082in}{3.338741in}}{\pgfqpoint{1.797982in}{3.335469in}}{\pgfqpoint{1.806218in}{3.335469in}}%
\pgfpathclose%
\pgfusepath{stroke,fill}%
\end{pgfscope}%
\begin{pgfscope}%
\pgfpathrectangle{\pgfqpoint{0.100000in}{0.212622in}}{\pgfqpoint{3.696000in}{3.696000in}}%
\pgfusepath{clip}%
\pgfsetbuttcap%
\pgfsetroundjoin%
\definecolor{currentfill}{rgb}{0.121569,0.466667,0.705882}%
\pgfsetfillcolor{currentfill}%
\pgfsetfillopacity{0.311599}%
\pgfsetlinewidth{1.003750pt}%
\definecolor{currentstroke}{rgb}{0.121569,0.466667,0.705882}%
\pgfsetstrokecolor{currentstroke}%
\pgfsetstrokeopacity{0.311599}%
\pgfsetdash{}{0pt}%
\pgfpathmoveto{\pgfqpoint{1.727595in}{3.310470in}}%
\pgfpathcurveto{\pgfqpoint{1.735831in}{3.310470in}}{\pgfqpoint{1.743731in}{3.313743in}}{\pgfqpoint{1.749555in}{3.319567in}}%
\pgfpathcurveto{\pgfqpoint{1.755379in}{3.325390in}}{\pgfqpoint{1.758651in}{3.333290in}}{\pgfqpoint{1.758651in}{3.341527in}}%
\pgfpathcurveto{\pgfqpoint{1.758651in}{3.349763in}}{\pgfqpoint{1.755379in}{3.357663in}}{\pgfqpoint{1.749555in}{3.363487in}}%
\pgfpathcurveto{\pgfqpoint{1.743731in}{3.369311in}}{\pgfqpoint{1.735831in}{3.372583in}}{\pgfqpoint{1.727595in}{3.372583in}}%
\pgfpathcurveto{\pgfqpoint{1.719358in}{3.372583in}}{\pgfqpoint{1.711458in}{3.369311in}}{\pgfqpoint{1.705634in}{3.363487in}}%
\pgfpathcurveto{\pgfqpoint{1.699810in}{3.357663in}}{\pgfqpoint{1.696538in}{3.349763in}}{\pgfqpoint{1.696538in}{3.341527in}}%
\pgfpathcurveto{\pgfqpoint{1.696538in}{3.333290in}}{\pgfqpoint{1.699810in}{3.325390in}}{\pgfqpoint{1.705634in}{3.319567in}}%
\pgfpathcurveto{\pgfqpoint{1.711458in}{3.313743in}}{\pgfqpoint{1.719358in}{3.310470in}}{\pgfqpoint{1.727595in}{3.310470in}}%
\pgfpathclose%
\pgfusepath{stroke,fill}%
\end{pgfscope}%
\begin{pgfscope}%
\pgfpathrectangle{\pgfqpoint{0.100000in}{0.212622in}}{\pgfqpoint{3.696000in}{3.696000in}}%
\pgfusepath{clip}%
\pgfsetbuttcap%
\pgfsetroundjoin%
\definecolor{currentfill}{rgb}{0.121569,0.466667,0.705882}%
\pgfsetfillcolor{currentfill}%
\pgfsetfillopacity{0.312476}%
\pgfsetlinewidth{1.003750pt}%
\definecolor{currentstroke}{rgb}{0.121569,0.466667,0.705882}%
\pgfsetstrokecolor{currentstroke}%
\pgfsetstrokeopacity{0.312476}%
\pgfsetdash{}{0pt}%
\pgfpathmoveto{\pgfqpoint{1.725794in}{3.306484in}}%
\pgfpathcurveto{\pgfqpoint{1.734030in}{3.306484in}}{\pgfqpoint{1.741930in}{3.309756in}}{\pgfqpoint{1.747754in}{3.315580in}}%
\pgfpathcurveto{\pgfqpoint{1.753578in}{3.321404in}}{\pgfqpoint{1.756850in}{3.329304in}}{\pgfqpoint{1.756850in}{3.337540in}}%
\pgfpathcurveto{\pgfqpoint{1.756850in}{3.345777in}}{\pgfqpoint{1.753578in}{3.353677in}}{\pgfqpoint{1.747754in}{3.359501in}}%
\pgfpathcurveto{\pgfqpoint{1.741930in}{3.365325in}}{\pgfqpoint{1.734030in}{3.368597in}}{\pgfqpoint{1.725794in}{3.368597in}}%
\pgfpathcurveto{\pgfqpoint{1.717557in}{3.368597in}}{\pgfqpoint{1.709657in}{3.365325in}}{\pgfqpoint{1.703833in}{3.359501in}}%
\pgfpathcurveto{\pgfqpoint{1.698010in}{3.353677in}}{\pgfqpoint{1.694737in}{3.345777in}}{\pgfqpoint{1.694737in}{3.337540in}}%
\pgfpathcurveto{\pgfqpoint{1.694737in}{3.329304in}}{\pgfqpoint{1.698010in}{3.321404in}}{\pgfqpoint{1.703833in}{3.315580in}}%
\pgfpathcurveto{\pgfqpoint{1.709657in}{3.309756in}}{\pgfqpoint{1.717557in}{3.306484in}}{\pgfqpoint{1.725794in}{3.306484in}}%
\pgfpathclose%
\pgfusepath{stroke,fill}%
\end{pgfscope}%
\begin{pgfscope}%
\pgfpathrectangle{\pgfqpoint{0.100000in}{0.212622in}}{\pgfqpoint{3.696000in}{3.696000in}}%
\pgfusepath{clip}%
\pgfsetbuttcap%
\pgfsetroundjoin%
\definecolor{currentfill}{rgb}{0.121569,0.466667,0.705882}%
\pgfsetfillcolor{currentfill}%
\pgfsetfillopacity{0.312571}%
\pgfsetlinewidth{1.003750pt}%
\definecolor{currentstroke}{rgb}{0.121569,0.466667,0.705882}%
\pgfsetstrokecolor{currentstroke}%
\pgfsetstrokeopacity{0.312571}%
\pgfsetdash{}{0pt}%
\pgfpathmoveto{\pgfqpoint{1.808443in}{3.331056in}}%
\pgfpathcurveto{\pgfqpoint{1.816679in}{3.331056in}}{\pgfqpoint{1.824579in}{3.334328in}}{\pgfqpoint{1.830403in}{3.340152in}}%
\pgfpathcurveto{\pgfqpoint{1.836227in}{3.345976in}}{\pgfqpoint{1.839500in}{3.353876in}}{\pgfqpoint{1.839500in}{3.362112in}}%
\pgfpathcurveto{\pgfqpoint{1.839500in}{3.370348in}}{\pgfqpoint{1.836227in}{3.378249in}}{\pgfqpoint{1.830403in}{3.384072in}}%
\pgfpathcurveto{\pgfqpoint{1.824579in}{3.389896in}}{\pgfqpoint{1.816679in}{3.393169in}}{\pgfqpoint{1.808443in}{3.393169in}}%
\pgfpathcurveto{\pgfqpoint{1.800207in}{3.393169in}}{\pgfqpoint{1.792307in}{3.389896in}}{\pgfqpoint{1.786483in}{3.384072in}}%
\pgfpathcurveto{\pgfqpoint{1.780659in}{3.378249in}}{\pgfqpoint{1.777387in}{3.370348in}}{\pgfqpoint{1.777387in}{3.362112in}}%
\pgfpathcurveto{\pgfqpoint{1.777387in}{3.353876in}}{\pgfqpoint{1.780659in}{3.345976in}}{\pgfqpoint{1.786483in}{3.340152in}}%
\pgfpathcurveto{\pgfqpoint{1.792307in}{3.334328in}}{\pgfqpoint{1.800207in}{3.331056in}}{\pgfqpoint{1.808443in}{3.331056in}}%
\pgfpathclose%
\pgfusepath{stroke,fill}%
\end{pgfscope}%
\begin{pgfscope}%
\pgfpathrectangle{\pgfqpoint{0.100000in}{0.212622in}}{\pgfqpoint{3.696000in}{3.696000in}}%
\pgfusepath{clip}%
\pgfsetbuttcap%
\pgfsetroundjoin%
\definecolor{currentfill}{rgb}{0.121569,0.466667,0.705882}%
\pgfsetfillcolor{currentfill}%
\pgfsetfillopacity{0.313276}%
\pgfsetlinewidth{1.003750pt}%
\definecolor{currentstroke}{rgb}{0.121569,0.466667,0.705882}%
\pgfsetstrokecolor{currentstroke}%
\pgfsetstrokeopacity{0.313276}%
\pgfsetdash{}{0pt}%
\pgfpathmoveto{\pgfqpoint{1.809510in}{3.328526in}}%
\pgfpathcurveto{\pgfqpoint{1.817746in}{3.328526in}}{\pgfqpoint{1.825646in}{3.331798in}}{\pgfqpoint{1.831470in}{3.337622in}}%
\pgfpathcurveto{\pgfqpoint{1.837294in}{3.343446in}}{\pgfqpoint{1.840566in}{3.351346in}}{\pgfqpoint{1.840566in}{3.359582in}}%
\pgfpathcurveto{\pgfqpoint{1.840566in}{3.367819in}}{\pgfqpoint{1.837294in}{3.375719in}}{\pgfqpoint{1.831470in}{3.381543in}}%
\pgfpathcurveto{\pgfqpoint{1.825646in}{3.387366in}}{\pgfqpoint{1.817746in}{3.390639in}}{\pgfqpoint{1.809510in}{3.390639in}}%
\pgfpathcurveto{\pgfqpoint{1.801273in}{3.390639in}}{\pgfqpoint{1.793373in}{3.387366in}}{\pgfqpoint{1.787549in}{3.381543in}}%
\pgfpathcurveto{\pgfqpoint{1.781725in}{3.375719in}}{\pgfqpoint{1.778453in}{3.367819in}}{\pgfqpoint{1.778453in}{3.359582in}}%
\pgfpathcurveto{\pgfqpoint{1.778453in}{3.351346in}}{\pgfqpoint{1.781725in}{3.343446in}}{\pgfqpoint{1.787549in}{3.337622in}}%
\pgfpathcurveto{\pgfqpoint{1.793373in}{3.331798in}}{\pgfqpoint{1.801273in}{3.328526in}}{\pgfqpoint{1.809510in}{3.328526in}}%
\pgfpathclose%
\pgfusepath{stroke,fill}%
\end{pgfscope}%
\begin{pgfscope}%
\pgfpathrectangle{\pgfqpoint{0.100000in}{0.212622in}}{\pgfqpoint{3.696000in}{3.696000in}}%
\pgfusepath{clip}%
\pgfsetbuttcap%
\pgfsetroundjoin%
\definecolor{currentfill}{rgb}{0.121569,0.466667,0.705882}%
\pgfsetfillcolor{currentfill}%
\pgfsetfillopacity{0.313679}%
\pgfsetlinewidth{1.003750pt}%
\definecolor{currentstroke}{rgb}{0.121569,0.466667,0.705882}%
\pgfsetstrokecolor{currentstroke}%
\pgfsetstrokeopacity{0.313679}%
\pgfsetdash{}{0pt}%
\pgfpathmoveto{\pgfqpoint{1.810132in}{3.327198in}}%
\pgfpathcurveto{\pgfqpoint{1.818369in}{3.327198in}}{\pgfqpoint{1.826269in}{3.330470in}}{\pgfqpoint{1.832093in}{3.336294in}}%
\pgfpathcurveto{\pgfqpoint{1.837917in}{3.342118in}}{\pgfqpoint{1.841189in}{3.350018in}}{\pgfqpoint{1.841189in}{3.358254in}}%
\pgfpathcurveto{\pgfqpoint{1.841189in}{3.366490in}}{\pgfqpoint{1.837917in}{3.374390in}}{\pgfqpoint{1.832093in}{3.380214in}}%
\pgfpathcurveto{\pgfqpoint{1.826269in}{3.386038in}}{\pgfqpoint{1.818369in}{3.389311in}}{\pgfqpoint{1.810132in}{3.389311in}}%
\pgfpathcurveto{\pgfqpoint{1.801896in}{3.389311in}}{\pgfqpoint{1.793996in}{3.386038in}}{\pgfqpoint{1.788172in}{3.380214in}}%
\pgfpathcurveto{\pgfqpoint{1.782348in}{3.374390in}}{\pgfqpoint{1.779076in}{3.366490in}}{\pgfqpoint{1.779076in}{3.358254in}}%
\pgfpathcurveto{\pgfqpoint{1.779076in}{3.350018in}}{\pgfqpoint{1.782348in}{3.342118in}}{\pgfqpoint{1.788172in}{3.336294in}}%
\pgfpathcurveto{\pgfqpoint{1.793996in}{3.330470in}}{\pgfqpoint{1.801896in}{3.327198in}}{\pgfqpoint{1.810132in}{3.327198in}}%
\pgfpathclose%
\pgfusepath{stroke,fill}%
\end{pgfscope}%
\begin{pgfscope}%
\pgfpathrectangle{\pgfqpoint{0.100000in}{0.212622in}}{\pgfqpoint{3.696000in}{3.696000in}}%
\pgfusepath{clip}%
\pgfsetbuttcap%
\pgfsetroundjoin%
\definecolor{currentfill}{rgb}{0.121569,0.466667,0.705882}%
\pgfsetfillcolor{currentfill}%
\pgfsetfillopacity{0.314013}%
\pgfsetlinewidth{1.003750pt}%
\definecolor{currentstroke}{rgb}{0.121569,0.466667,0.705882}%
\pgfsetstrokecolor{currentstroke}%
\pgfsetstrokeopacity{0.314013}%
\pgfsetdash{}{0pt}%
\pgfpathmoveto{\pgfqpoint{1.722309in}{3.299222in}}%
\pgfpathcurveto{\pgfqpoint{1.730546in}{3.299222in}}{\pgfqpoint{1.738446in}{3.302494in}}{\pgfqpoint{1.744270in}{3.308318in}}%
\pgfpathcurveto{\pgfqpoint{1.750093in}{3.314142in}}{\pgfqpoint{1.753366in}{3.322042in}}{\pgfqpoint{1.753366in}{3.330278in}}%
\pgfpathcurveto{\pgfqpoint{1.753366in}{3.338514in}}{\pgfqpoint{1.750093in}{3.346414in}}{\pgfqpoint{1.744270in}{3.352238in}}%
\pgfpathcurveto{\pgfqpoint{1.738446in}{3.358062in}}{\pgfqpoint{1.730546in}{3.361335in}}{\pgfqpoint{1.722309in}{3.361335in}}%
\pgfpathcurveto{\pgfqpoint{1.714073in}{3.361335in}}{\pgfqpoint{1.706173in}{3.358062in}}{\pgfqpoint{1.700349in}{3.352238in}}%
\pgfpathcurveto{\pgfqpoint{1.694525in}{3.346414in}}{\pgfqpoint{1.691253in}{3.338514in}}{\pgfqpoint{1.691253in}{3.330278in}}%
\pgfpathcurveto{\pgfqpoint{1.691253in}{3.322042in}}{\pgfqpoint{1.694525in}{3.314142in}}{\pgfqpoint{1.700349in}{3.308318in}}%
\pgfpathcurveto{\pgfqpoint{1.706173in}{3.302494in}}{\pgfqpoint{1.714073in}{3.299222in}}{\pgfqpoint{1.722309in}{3.299222in}}%
\pgfpathclose%
\pgfusepath{stroke,fill}%
\end{pgfscope}%
\begin{pgfscope}%
\pgfpathrectangle{\pgfqpoint{0.100000in}{0.212622in}}{\pgfqpoint{3.696000in}{3.696000in}}%
\pgfusepath{clip}%
\pgfsetbuttcap%
\pgfsetroundjoin%
\definecolor{currentfill}{rgb}{0.121569,0.466667,0.705882}%
\pgfsetfillcolor{currentfill}%
\pgfsetfillopacity{0.314757}%
\pgfsetlinewidth{1.003750pt}%
\definecolor{currentstroke}{rgb}{0.121569,0.466667,0.705882}%
\pgfsetstrokecolor{currentstroke}%
\pgfsetstrokeopacity{0.314757}%
\pgfsetdash{}{0pt}%
\pgfpathmoveto{\pgfqpoint{1.811630in}{3.323336in}}%
\pgfpathcurveto{\pgfqpoint{1.819866in}{3.323336in}}{\pgfqpoint{1.827766in}{3.326608in}}{\pgfqpoint{1.833590in}{3.332432in}}%
\pgfpathcurveto{\pgfqpoint{1.839414in}{3.338256in}}{\pgfqpoint{1.842686in}{3.346156in}}{\pgfqpoint{1.842686in}{3.354392in}}%
\pgfpathcurveto{\pgfqpoint{1.842686in}{3.362628in}}{\pgfqpoint{1.839414in}{3.370528in}}{\pgfqpoint{1.833590in}{3.376352in}}%
\pgfpathcurveto{\pgfqpoint{1.827766in}{3.382176in}}{\pgfqpoint{1.819866in}{3.385449in}}{\pgfqpoint{1.811630in}{3.385449in}}%
\pgfpathcurveto{\pgfqpoint{1.803393in}{3.385449in}}{\pgfqpoint{1.795493in}{3.382176in}}{\pgfqpoint{1.789669in}{3.376352in}}%
\pgfpathcurveto{\pgfqpoint{1.783845in}{3.370528in}}{\pgfqpoint{1.780573in}{3.362628in}}{\pgfqpoint{1.780573in}{3.354392in}}%
\pgfpathcurveto{\pgfqpoint{1.780573in}{3.346156in}}{\pgfqpoint{1.783845in}{3.338256in}}{\pgfqpoint{1.789669in}{3.332432in}}%
\pgfpathcurveto{\pgfqpoint{1.795493in}{3.326608in}}{\pgfqpoint{1.803393in}{3.323336in}}{\pgfqpoint{1.811630in}{3.323336in}}%
\pgfpathclose%
\pgfusepath{stroke,fill}%
\end{pgfscope}%
\begin{pgfscope}%
\pgfpathrectangle{\pgfqpoint{0.100000in}{0.212622in}}{\pgfqpoint{3.696000in}{3.696000in}}%
\pgfusepath{clip}%
\pgfsetbuttcap%
\pgfsetroundjoin%
\definecolor{currentfill}{rgb}{0.121569,0.466667,0.705882}%
\pgfsetfillcolor{currentfill}%
\pgfsetfillopacity{0.315181}%
\pgfsetlinewidth{1.003750pt}%
\definecolor{currentstroke}{rgb}{0.121569,0.466667,0.705882}%
\pgfsetstrokecolor{currentstroke}%
\pgfsetstrokeopacity{0.315181}%
\pgfsetdash{}{0pt}%
\pgfpathmoveto{\pgfqpoint{1.719952in}{3.294107in}}%
\pgfpathcurveto{\pgfqpoint{1.728188in}{3.294107in}}{\pgfqpoint{1.736088in}{3.297380in}}{\pgfqpoint{1.741912in}{3.303204in}}%
\pgfpathcurveto{\pgfqpoint{1.747736in}{3.309028in}}{\pgfqpoint{1.751008in}{3.316928in}}{\pgfqpoint{1.751008in}{3.325164in}}%
\pgfpathcurveto{\pgfqpoint{1.751008in}{3.333400in}}{\pgfqpoint{1.747736in}{3.341300in}}{\pgfqpoint{1.741912in}{3.347124in}}%
\pgfpathcurveto{\pgfqpoint{1.736088in}{3.352948in}}{\pgfqpoint{1.728188in}{3.356220in}}{\pgfqpoint{1.719952in}{3.356220in}}%
\pgfpathcurveto{\pgfqpoint{1.711715in}{3.356220in}}{\pgfqpoint{1.703815in}{3.352948in}}{\pgfqpoint{1.697991in}{3.347124in}}%
\pgfpathcurveto{\pgfqpoint{1.692167in}{3.341300in}}{\pgfqpoint{1.688895in}{3.333400in}}{\pgfqpoint{1.688895in}{3.325164in}}%
\pgfpathcurveto{\pgfqpoint{1.688895in}{3.316928in}}{\pgfqpoint{1.692167in}{3.309028in}}{\pgfqpoint{1.697991in}{3.303204in}}%
\pgfpathcurveto{\pgfqpoint{1.703815in}{3.297380in}}{\pgfqpoint{1.711715in}{3.294107in}}{\pgfqpoint{1.719952in}{3.294107in}}%
\pgfpathclose%
\pgfusepath{stroke,fill}%
\end{pgfscope}%
\begin{pgfscope}%
\pgfpathrectangle{\pgfqpoint{0.100000in}{0.212622in}}{\pgfqpoint{3.696000in}{3.696000in}}%
\pgfusepath{clip}%
\pgfsetbuttcap%
\pgfsetroundjoin%
\definecolor{currentfill}{rgb}{0.121569,0.466667,0.705882}%
\pgfsetfillcolor{currentfill}%
\pgfsetfillopacity{0.316003}%
\pgfsetlinewidth{1.003750pt}%
\definecolor{currentstroke}{rgb}{0.121569,0.466667,0.705882}%
\pgfsetstrokecolor{currentstroke}%
\pgfsetstrokeopacity{0.316003}%
\pgfsetdash{}{0pt}%
\pgfpathmoveto{\pgfqpoint{1.718237in}{3.290369in}}%
\pgfpathcurveto{\pgfqpoint{1.726474in}{3.290369in}}{\pgfqpoint{1.734374in}{3.293642in}}{\pgfqpoint{1.740198in}{3.299466in}}%
\pgfpathcurveto{\pgfqpoint{1.746022in}{3.305290in}}{\pgfqpoint{1.749294in}{3.313190in}}{\pgfqpoint{1.749294in}{3.321426in}}%
\pgfpathcurveto{\pgfqpoint{1.749294in}{3.329662in}}{\pgfqpoint{1.746022in}{3.337562in}}{\pgfqpoint{1.740198in}{3.343386in}}%
\pgfpathcurveto{\pgfqpoint{1.734374in}{3.349210in}}{\pgfqpoint{1.726474in}{3.352482in}}{\pgfqpoint{1.718237in}{3.352482in}}%
\pgfpathcurveto{\pgfqpoint{1.710001in}{3.352482in}}{\pgfqpoint{1.702101in}{3.349210in}}{\pgfqpoint{1.696277in}{3.343386in}}%
\pgfpathcurveto{\pgfqpoint{1.690453in}{3.337562in}}{\pgfqpoint{1.687181in}{3.329662in}}{\pgfqpoint{1.687181in}{3.321426in}}%
\pgfpathcurveto{\pgfqpoint{1.687181in}{3.313190in}}{\pgfqpoint{1.690453in}{3.305290in}}{\pgfqpoint{1.696277in}{3.299466in}}%
\pgfpathcurveto{\pgfqpoint{1.702101in}{3.293642in}}{\pgfqpoint{1.710001in}{3.290369in}}{\pgfqpoint{1.718237in}{3.290369in}}%
\pgfpathclose%
\pgfusepath{stroke,fill}%
\end{pgfscope}%
\begin{pgfscope}%
\pgfpathrectangle{\pgfqpoint{0.100000in}{0.212622in}}{\pgfqpoint{3.696000in}{3.696000in}}%
\pgfusepath{clip}%
\pgfsetbuttcap%
\pgfsetroundjoin%
\definecolor{currentfill}{rgb}{0.121569,0.466667,0.705882}%
\pgfsetfillcolor{currentfill}%
\pgfsetfillopacity{0.316325}%
\pgfsetlinewidth{1.003750pt}%
\definecolor{currentstroke}{rgb}{0.121569,0.466667,0.705882}%
\pgfsetstrokecolor{currentstroke}%
\pgfsetstrokeopacity{0.316325}%
\pgfsetdash{}{0pt}%
\pgfpathmoveto{\pgfqpoint{1.717627in}{3.288956in}}%
\pgfpathcurveto{\pgfqpoint{1.725863in}{3.288956in}}{\pgfqpoint{1.733763in}{3.292229in}}{\pgfqpoint{1.739587in}{3.298052in}}%
\pgfpathcurveto{\pgfqpoint{1.745411in}{3.303876in}}{\pgfqpoint{1.748683in}{3.311776in}}{\pgfqpoint{1.748683in}{3.320013in}}%
\pgfpathcurveto{\pgfqpoint{1.748683in}{3.328249in}}{\pgfqpoint{1.745411in}{3.336149in}}{\pgfqpoint{1.739587in}{3.341973in}}%
\pgfpathcurveto{\pgfqpoint{1.733763in}{3.347797in}}{\pgfqpoint{1.725863in}{3.351069in}}{\pgfqpoint{1.717627in}{3.351069in}}%
\pgfpathcurveto{\pgfqpoint{1.709390in}{3.351069in}}{\pgfqpoint{1.701490in}{3.347797in}}{\pgfqpoint{1.695667in}{3.341973in}}%
\pgfpathcurveto{\pgfqpoint{1.689843in}{3.336149in}}{\pgfqpoint{1.686570in}{3.328249in}}{\pgfqpoint{1.686570in}{3.320013in}}%
\pgfpathcurveto{\pgfqpoint{1.686570in}{3.311776in}}{\pgfqpoint{1.689843in}{3.303876in}}{\pgfqpoint{1.695667in}{3.298052in}}%
\pgfpathcurveto{\pgfqpoint{1.701490in}{3.292229in}}{\pgfqpoint{1.709390in}{3.288956in}}{\pgfqpoint{1.717627in}{3.288956in}}%
\pgfpathclose%
\pgfusepath{stroke,fill}%
\end{pgfscope}%
\begin{pgfscope}%
\pgfpathrectangle{\pgfqpoint{0.100000in}{0.212622in}}{\pgfqpoint{3.696000in}{3.696000in}}%
\pgfusepath{clip}%
\pgfsetbuttcap%
\pgfsetroundjoin%
\definecolor{currentfill}{rgb}{0.121569,0.466667,0.705882}%
\pgfsetfillcolor{currentfill}%
\pgfsetfillopacity{0.316530}%
\pgfsetlinewidth{1.003750pt}%
\definecolor{currentstroke}{rgb}{0.121569,0.466667,0.705882}%
\pgfsetstrokecolor{currentstroke}%
\pgfsetstrokeopacity{0.316530}%
\pgfsetdash{}{0pt}%
\pgfpathmoveto{\pgfqpoint{1.814010in}{3.317421in}}%
\pgfpathcurveto{\pgfqpoint{1.822246in}{3.317421in}}{\pgfqpoint{1.830147in}{3.320693in}}{\pgfqpoint{1.835970in}{3.326517in}}%
\pgfpathcurveto{\pgfqpoint{1.841794in}{3.332341in}}{\pgfqpoint{1.845067in}{3.340241in}}{\pgfqpoint{1.845067in}{3.348477in}}%
\pgfpathcurveto{\pgfqpoint{1.845067in}{3.356714in}}{\pgfqpoint{1.841794in}{3.364614in}}{\pgfqpoint{1.835970in}{3.370438in}}%
\pgfpathcurveto{\pgfqpoint{1.830147in}{3.376262in}}{\pgfqpoint{1.822246in}{3.379534in}}{\pgfqpoint{1.814010in}{3.379534in}}%
\pgfpathcurveto{\pgfqpoint{1.805774in}{3.379534in}}{\pgfqpoint{1.797874in}{3.376262in}}{\pgfqpoint{1.792050in}{3.370438in}}%
\pgfpathcurveto{\pgfqpoint{1.786226in}{3.364614in}}{\pgfqpoint{1.782954in}{3.356714in}}{\pgfqpoint{1.782954in}{3.348477in}}%
\pgfpathcurveto{\pgfqpoint{1.782954in}{3.340241in}}{\pgfqpoint{1.786226in}{3.332341in}}{\pgfqpoint{1.792050in}{3.326517in}}%
\pgfpathcurveto{\pgfqpoint{1.797874in}{3.320693in}}{\pgfqpoint{1.805774in}{3.317421in}}{\pgfqpoint{1.814010in}{3.317421in}}%
\pgfpathclose%
\pgfusepath{stroke,fill}%
\end{pgfscope}%
\begin{pgfscope}%
\pgfpathrectangle{\pgfqpoint{0.100000in}{0.212622in}}{\pgfqpoint{3.696000in}{3.696000in}}%
\pgfusepath{clip}%
\pgfsetbuttcap%
\pgfsetroundjoin%
\definecolor{currentfill}{rgb}{0.121569,0.466667,0.705882}%
\pgfsetfillcolor{currentfill}%
\pgfsetfillopacity{0.316909}%
\pgfsetlinewidth{1.003750pt}%
\definecolor{currentstroke}{rgb}{0.121569,0.466667,0.705882}%
\pgfsetstrokecolor{currentstroke}%
\pgfsetstrokeopacity{0.316909}%
\pgfsetdash{}{0pt}%
\pgfpathmoveto{\pgfqpoint{1.716522in}{3.286375in}}%
\pgfpathcurveto{\pgfqpoint{1.724758in}{3.286375in}}{\pgfqpoint{1.732659in}{3.289647in}}{\pgfqpoint{1.738482in}{3.295471in}}%
\pgfpathcurveto{\pgfqpoint{1.744306in}{3.301295in}}{\pgfqpoint{1.747579in}{3.309195in}}{\pgfqpoint{1.747579in}{3.317431in}}%
\pgfpathcurveto{\pgfqpoint{1.747579in}{3.325667in}}{\pgfqpoint{1.744306in}{3.333567in}}{\pgfqpoint{1.738482in}{3.339391in}}%
\pgfpathcurveto{\pgfqpoint{1.732659in}{3.345215in}}{\pgfqpoint{1.724758in}{3.348488in}}{\pgfqpoint{1.716522in}{3.348488in}}%
\pgfpathcurveto{\pgfqpoint{1.708286in}{3.348488in}}{\pgfqpoint{1.700386in}{3.345215in}}{\pgfqpoint{1.694562in}{3.339391in}}%
\pgfpathcurveto{\pgfqpoint{1.688738in}{3.333567in}}{\pgfqpoint{1.685466in}{3.325667in}}{\pgfqpoint{1.685466in}{3.317431in}}%
\pgfpathcurveto{\pgfqpoint{1.685466in}{3.309195in}}{\pgfqpoint{1.688738in}{3.301295in}}{\pgfqpoint{1.694562in}{3.295471in}}%
\pgfpathcurveto{\pgfqpoint{1.700386in}{3.289647in}}{\pgfqpoint{1.708286in}{3.286375in}}{\pgfqpoint{1.716522in}{3.286375in}}%
\pgfpathclose%
\pgfusepath{stroke,fill}%
\end{pgfscope}%
\begin{pgfscope}%
\pgfpathrectangle{\pgfqpoint{0.100000in}{0.212622in}}{\pgfqpoint{3.696000in}{3.696000in}}%
\pgfusepath{clip}%
\pgfsetbuttcap%
\pgfsetroundjoin%
\definecolor{currentfill}{rgb}{0.121569,0.466667,0.705882}%
\pgfsetfillcolor{currentfill}%
\pgfsetfillopacity{0.317927}%
\pgfsetlinewidth{1.003750pt}%
\definecolor{currentstroke}{rgb}{0.121569,0.466667,0.705882}%
\pgfsetstrokecolor{currentstroke}%
\pgfsetstrokeopacity{0.317927}%
\pgfsetdash{}{0pt}%
\pgfpathmoveto{\pgfqpoint{1.714462in}{3.281548in}}%
\pgfpathcurveto{\pgfqpoint{1.722698in}{3.281548in}}{\pgfqpoint{1.730598in}{3.284821in}}{\pgfqpoint{1.736422in}{3.290645in}}%
\pgfpathcurveto{\pgfqpoint{1.742246in}{3.296468in}}{\pgfqpoint{1.745518in}{3.304368in}}{\pgfqpoint{1.745518in}{3.312605in}}%
\pgfpathcurveto{\pgfqpoint{1.745518in}{3.320841in}}{\pgfqpoint{1.742246in}{3.328741in}}{\pgfqpoint{1.736422in}{3.334565in}}%
\pgfpathcurveto{\pgfqpoint{1.730598in}{3.340389in}}{\pgfqpoint{1.722698in}{3.343661in}}{\pgfqpoint{1.714462in}{3.343661in}}%
\pgfpathcurveto{\pgfqpoint{1.706225in}{3.343661in}}{\pgfqpoint{1.698325in}{3.340389in}}{\pgfqpoint{1.692501in}{3.334565in}}%
\pgfpathcurveto{\pgfqpoint{1.686678in}{3.328741in}}{\pgfqpoint{1.683405in}{3.320841in}}{\pgfqpoint{1.683405in}{3.312605in}}%
\pgfpathcurveto{\pgfqpoint{1.683405in}{3.304368in}}{\pgfqpoint{1.686678in}{3.296468in}}{\pgfqpoint{1.692501in}{3.290645in}}%
\pgfpathcurveto{\pgfqpoint{1.698325in}{3.284821in}}{\pgfqpoint{1.706225in}{3.281548in}}{\pgfqpoint{1.714462in}{3.281548in}}%
\pgfpathclose%
\pgfusepath{stroke,fill}%
\end{pgfscope}%
\begin{pgfscope}%
\pgfpathrectangle{\pgfqpoint{0.100000in}{0.212622in}}{\pgfqpoint{3.696000in}{3.696000in}}%
\pgfusepath{clip}%
\pgfsetbuttcap%
\pgfsetroundjoin%
\definecolor{currentfill}{rgb}{0.121569,0.466667,0.705882}%
\pgfsetfillcolor{currentfill}%
\pgfsetfillopacity{0.318563}%
\pgfsetlinewidth{1.003750pt}%
\definecolor{currentstroke}{rgb}{0.121569,0.466667,0.705882}%
\pgfsetstrokecolor{currentstroke}%
\pgfsetstrokeopacity{0.318563}%
\pgfsetdash{}{0pt}%
\pgfpathmoveto{\pgfqpoint{1.713192in}{3.278666in}}%
\pgfpathcurveto{\pgfqpoint{1.721428in}{3.278666in}}{\pgfqpoint{1.729328in}{3.281939in}}{\pgfqpoint{1.735152in}{3.287763in}}%
\pgfpathcurveto{\pgfqpoint{1.740976in}{3.293587in}}{\pgfqpoint{1.744248in}{3.301487in}}{\pgfqpoint{1.744248in}{3.309723in}}%
\pgfpathcurveto{\pgfqpoint{1.744248in}{3.317959in}}{\pgfqpoint{1.740976in}{3.325859in}}{\pgfqpoint{1.735152in}{3.331683in}}%
\pgfpathcurveto{\pgfqpoint{1.729328in}{3.337507in}}{\pgfqpoint{1.721428in}{3.340779in}}{\pgfqpoint{1.713192in}{3.340779in}}%
\pgfpathcurveto{\pgfqpoint{1.704955in}{3.340779in}}{\pgfqpoint{1.697055in}{3.337507in}}{\pgfqpoint{1.691232in}{3.331683in}}%
\pgfpathcurveto{\pgfqpoint{1.685408in}{3.325859in}}{\pgfqpoint{1.682135in}{3.317959in}}{\pgfqpoint{1.682135in}{3.309723in}}%
\pgfpathcurveto{\pgfqpoint{1.682135in}{3.301487in}}{\pgfqpoint{1.685408in}{3.293587in}}{\pgfqpoint{1.691232in}{3.287763in}}%
\pgfpathcurveto{\pgfqpoint{1.697055in}{3.281939in}}{\pgfqpoint{1.704955in}{3.278666in}}{\pgfqpoint{1.713192in}{3.278666in}}%
\pgfpathclose%
\pgfusepath{stroke,fill}%
\end{pgfscope}%
\begin{pgfscope}%
\pgfpathrectangle{\pgfqpoint{0.100000in}{0.212622in}}{\pgfqpoint{3.696000in}{3.696000in}}%
\pgfusepath{clip}%
\pgfsetbuttcap%
\pgfsetroundjoin%
\definecolor{currentfill}{rgb}{0.121569,0.466667,0.705882}%
\pgfsetfillcolor{currentfill}%
\pgfsetfillopacity{0.319374}%
\pgfsetlinewidth{1.003750pt}%
\definecolor{currentstroke}{rgb}{0.121569,0.466667,0.705882}%
\pgfsetstrokecolor{currentstroke}%
\pgfsetstrokeopacity{0.319374}%
\pgfsetdash{}{0pt}%
\pgfpathmoveto{\pgfqpoint{1.817590in}{3.307494in}}%
\pgfpathcurveto{\pgfqpoint{1.825826in}{3.307494in}}{\pgfqpoint{1.833726in}{3.310767in}}{\pgfqpoint{1.839550in}{3.316591in}}%
\pgfpathcurveto{\pgfqpoint{1.845374in}{3.322415in}}{\pgfqpoint{1.848646in}{3.330315in}}{\pgfqpoint{1.848646in}{3.338551in}}%
\pgfpathcurveto{\pgfqpoint{1.848646in}{3.346787in}}{\pgfqpoint{1.845374in}{3.354687in}}{\pgfqpoint{1.839550in}{3.360511in}}%
\pgfpathcurveto{\pgfqpoint{1.833726in}{3.366335in}}{\pgfqpoint{1.825826in}{3.369607in}}{\pgfqpoint{1.817590in}{3.369607in}}%
\pgfpathcurveto{\pgfqpoint{1.809353in}{3.369607in}}{\pgfqpoint{1.801453in}{3.366335in}}{\pgfqpoint{1.795629in}{3.360511in}}%
\pgfpathcurveto{\pgfqpoint{1.789805in}{3.354687in}}{\pgfqpoint{1.786533in}{3.346787in}}{\pgfqpoint{1.786533in}{3.338551in}}%
\pgfpathcurveto{\pgfqpoint{1.786533in}{3.330315in}}{\pgfqpoint{1.789805in}{3.322415in}}{\pgfqpoint{1.795629in}{3.316591in}}%
\pgfpathcurveto{\pgfqpoint{1.801453in}{3.310767in}}{\pgfqpoint{1.809353in}{3.307494in}}{\pgfqpoint{1.817590in}{3.307494in}}%
\pgfpathclose%
\pgfusepath{stroke,fill}%
\end{pgfscope}%
\begin{pgfscope}%
\pgfpathrectangle{\pgfqpoint{0.100000in}{0.212622in}}{\pgfqpoint{3.696000in}{3.696000in}}%
\pgfusepath{clip}%
\pgfsetbuttcap%
\pgfsetroundjoin%
\definecolor{currentfill}{rgb}{0.121569,0.466667,0.705882}%
\pgfsetfillcolor{currentfill}%
\pgfsetfillopacity{0.319734}%
\pgfsetlinewidth{1.003750pt}%
\definecolor{currentstroke}{rgb}{0.121569,0.466667,0.705882}%
\pgfsetstrokecolor{currentstroke}%
\pgfsetstrokeopacity{0.319734}%
\pgfsetdash{}{0pt}%
\pgfpathmoveto{\pgfqpoint{1.710906in}{3.273459in}}%
\pgfpathcurveto{\pgfqpoint{1.719143in}{3.273459in}}{\pgfqpoint{1.727043in}{3.276731in}}{\pgfqpoint{1.732867in}{3.282555in}}%
\pgfpathcurveto{\pgfqpoint{1.738691in}{3.288379in}}{\pgfqpoint{1.741963in}{3.296279in}}{\pgfqpoint{1.741963in}{3.304516in}}%
\pgfpathcurveto{\pgfqpoint{1.741963in}{3.312752in}}{\pgfqpoint{1.738691in}{3.320652in}}{\pgfqpoint{1.732867in}{3.326476in}}%
\pgfpathcurveto{\pgfqpoint{1.727043in}{3.332300in}}{\pgfqpoint{1.719143in}{3.335572in}}{\pgfqpoint{1.710906in}{3.335572in}}%
\pgfpathcurveto{\pgfqpoint{1.702670in}{3.335572in}}{\pgfqpoint{1.694770in}{3.332300in}}{\pgfqpoint{1.688946in}{3.326476in}}%
\pgfpathcurveto{\pgfqpoint{1.683122in}{3.320652in}}{\pgfqpoint{1.679850in}{3.312752in}}{\pgfqpoint{1.679850in}{3.304516in}}%
\pgfpathcurveto{\pgfqpoint{1.679850in}{3.296279in}}{\pgfqpoint{1.683122in}{3.288379in}}{\pgfqpoint{1.688946in}{3.282555in}}%
\pgfpathcurveto{\pgfqpoint{1.694770in}{3.276731in}}{\pgfqpoint{1.702670in}{3.273459in}}{\pgfqpoint{1.710906in}{3.273459in}}%
\pgfpathclose%
\pgfusepath{stroke,fill}%
\end{pgfscope}%
\begin{pgfscope}%
\pgfpathrectangle{\pgfqpoint{0.100000in}{0.212622in}}{\pgfqpoint{3.696000in}{3.696000in}}%
\pgfusepath{clip}%
\pgfsetbuttcap%
\pgfsetroundjoin%
\definecolor{currentfill}{rgb}{0.121569,0.466667,0.705882}%
\pgfsetfillcolor{currentfill}%
\pgfsetfillopacity{0.320150}%
\pgfsetlinewidth{1.003750pt}%
\definecolor{currentstroke}{rgb}{0.121569,0.466667,0.705882}%
\pgfsetstrokecolor{currentstroke}%
\pgfsetstrokeopacity{0.320150}%
\pgfsetdash{}{0pt}%
\pgfpathmoveto{\pgfqpoint{1.710030in}{3.271621in}}%
\pgfpathcurveto{\pgfqpoint{1.718266in}{3.271621in}}{\pgfqpoint{1.726166in}{3.274894in}}{\pgfqpoint{1.731990in}{3.280717in}}%
\pgfpathcurveto{\pgfqpoint{1.737814in}{3.286541in}}{\pgfqpoint{1.741086in}{3.294441in}}{\pgfqpoint{1.741086in}{3.302678in}}%
\pgfpathcurveto{\pgfqpoint{1.741086in}{3.310914in}}{\pgfqpoint{1.737814in}{3.318814in}}{\pgfqpoint{1.731990in}{3.324638in}}%
\pgfpathcurveto{\pgfqpoint{1.726166in}{3.330462in}}{\pgfqpoint{1.718266in}{3.333734in}}{\pgfqpoint{1.710030in}{3.333734in}}%
\pgfpathcurveto{\pgfqpoint{1.701794in}{3.333734in}}{\pgfqpoint{1.693894in}{3.330462in}}{\pgfqpoint{1.688070in}{3.324638in}}%
\pgfpathcurveto{\pgfqpoint{1.682246in}{3.318814in}}{\pgfqpoint{1.678973in}{3.310914in}}{\pgfqpoint{1.678973in}{3.302678in}}%
\pgfpathcurveto{\pgfqpoint{1.678973in}{3.294441in}}{\pgfqpoint{1.682246in}{3.286541in}}{\pgfqpoint{1.688070in}{3.280717in}}%
\pgfpathcurveto{\pgfqpoint{1.693894in}{3.274894in}}{\pgfqpoint{1.701794in}{3.271621in}}{\pgfqpoint{1.710030in}{3.271621in}}%
\pgfpathclose%
\pgfusepath{stroke,fill}%
\end{pgfscope}%
\begin{pgfscope}%
\pgfpathrectangle{\pgfqpoint{0.100000in}{0.212622in}}{\pgfqpoint{3.696000in}{3.696000in}}%
\pgfusepath{clip}%
\pgfsetbuttcap%
\pgfsetroundjoin%
\definecolor{currentfill}{rgb}{0.121569,0.466667,0.705882}%
\pgfsetfillcolor{currentfill}%
\pgfsetfillopacity{0.320897}%
\pgfsetlinewidth{1.003750pt}%
\definecolor{currentstroke}{rgb}{0.121569,0.466667,0.705882}%
\pgfsetstrokecolor{currentstroke}%
\pgfsetstrokeopacity{0.320897}%
\pgfsetdash{}{0pt}%
\pgfpathmoveto{\pgfqpoint{1.708469in}{3.268205in}}%
\pgfpathcurveto{\pgfqpoint{1.716705in}{3.268205in}}{\pgfqpoint{1.724605in}{3.271478in}}{\pgfqpoint{1.730429in}{3.277301in}}%
\pgfpathcurveto{\pgfqpoint{1.736253in}{3.283125in}}{\pgfqpoint{1.739526in}{3.291025in}}{\pgfqpoint{1.739526in}{3.299262in}}%
\pgfpathcurveto{\pgfqpoint{1.739526in}{3.307498in}}{\pgfqpoint{1.736253in}{3.315398in}}{\pgfqpoint{1.730429in}{3.321222in}}%
\pgfpathcurveto{\pgfqpoint{1.724605in}{3.327046in}}{\pgfqpoint{1.716705in}{3.330318in}}{\pgfqpoint{1.708469in}{3.330318in}}%
\pgfpathcurveto{\pgfqpoint{1.700233in}{3.330318in}}{\pgfqpoint{1.692333in}{3.327046in}}{\pgfqpoint{1.686509in}{3.321222in}}%
\pgfpathcurveto{\pgfqpoint{1.680685in}{3.315398in}}{\pgfqpoint{1.677413in}{3.307498in}}{\pgfqpoint{1.677413in}{3.299262in}}%
\pgfpathcurveto{\pgfqpoint{1.677413in}{3.291025in}}{\pgfqpoint{1.680685in}{3.283125in}}{\pgfqpoint{1.686509in}{3.277301in}}%
\pgfpathcurveto{\pgfqpoint{1.692333in}{3.271478in}}{\pgfqpoint{1.700233in}{3.268205in}}{\pgfqpoint{1.708469in}{3.268205in}}%
\pgfpathclose%
\pgfusepath{stroke,fill}%
\end{pgfscope}%
\begin{pgfscope}%
\pgfpathrectangle{\pgfqpoint{0.100000in}{0.212622in}}{\pgfqpoint{3.696000in}{3.696000in}}%
\pgfusepath{clip}%
\pgfsetbuttcap%
\pgfsetroundjoin%
\definecolor{currentfill}{rgb}{0.121569,0.466667,0.705882}%
\pgfsetfillcolor{currentfill}%
\pgfsetfillopacity{0.322216}%
\pgfsetlinewidth{1.003750pt}%
\definecolor{currentstroke}{rgb}{0.121569,0.466667,0.705882}%
\pgfsetstrokecolor{currentstroke}%
\pgfsetstrokeopacity{0.322216}%
\pgfsetdash{}{0pt}%
\pgfpathmoveto{\pgfqpoint{1.705535in}{3.261938in}}%
\pgfpathcurveto{\pgfqpoint{1.713771in}{3.261938in}}{\pgfqpoint{1.721672in}{3.265210in}}{\pgfqpoint{1.727495in}{3.271034in}}%
\pgfpathcurveto{\pgfqpoint{1.733319in}{3.276858in}}{\pgfqpoint{1.736592in}{3.284758in}}{\pgfqpoint{1.736592in}{3.292994in}}%
\pgfpathcurveto{\pgfqpoint{1.736592in}{3.301230in}}{\pgfqpoint{1.733319in}{3.309130in}}{\pgfqpoint{1.727495in}{3.314954in}}%
\pgfpathcurveto{\pgfqpoint{1.721672in}{3.320778in}}{\pgfqpoint{1.713771in}{3.324051in}}{\pgfqpoint{1.705535in}{3.324051in}}%
\pgfpathcurveto{\pgfqpoint{1.697299in}{3.324051in}}{\pgfqpoint{1.689399in}{3.320778in}}{\pgfqpoint{1.683575in}{3.314954in}}%
\pgfpathcurveto{\pgfqpoint{1.677751in}{3.309130in}}{\pgfqpoint{1.674479in}{3.301230in}}{\pgfqpoint{1.674479in}{3.292994in}}%
\pgfpathcurveto{\pgfqpoint{1.674479in}{3.284758in}}{\pgfqpoint{1.677751in}{3.276858in}}{\pgfqpoint{1.683575in}{3.271034in}}%
\pgfpathcurveto{\pgfqpoint{1.689399in}{3.265210in}}{\pgfqpoint{1.697299in}{3.261938in}}{\pgfqpoint{1.705535in}{3.261938in}}%
\pgfpathclose%
\pgfusepath{stroke,fill}%
\end{pgfscope}%
\begin{pgfscope}%
\pgfpathrectangle{\pgfqpoint{0.100000in}{0.212622in}}{\pgfqpoint{3.696000in}{3.696000in}}%
\pgfusepath{clip}%
\pgfsetbuttcap%
\pgfsetroundjoin%
\definecolor{currentfill}{rgb}{0.121569,0.466667,0.705882}%
\pgfsetfillcolor{currentfill}%
\pgfsetfillopacity{0.322743}%
\pgfsetlinewidth{1.003750pt}%
\definecolor{currentstroke}{rgb}{0.121569,0.466667,0.705882}%
\pgfsetstrokecolor{currentstroke}%
\pgfsetstrokeopacity{0.322743}%
\pgfsetdash{}{0pt}%
\pgfpathmoveto{\pgfqpoint{1.821643in}{3.295808in}}%
\pgfpathcurveto{\pgfqpoint{1.829879in}{3.295808in}}{\pgfqpoint{1.837779in}{3.299080in}}{\pgfqpoint{1.843603in}{3.304904in}}%
\pgfpathcurveto{\pgfqpoint{1.849427in}{3.310728in}}{\pgfqpoint{1.852699in}{3.318628in}}{\pgfqpoint{1.852699in}{3.326864in}}%
\pgfpathcurveto{\pgfqpoint{1.852699in}{3.335101in}}{\pgfqpoint{1.849427in}{3.343001in}}{\pgfqpoint{1.843603in}{3.348825in}}%
\pgfpathcurveto{\pgfqpoint{1.837779in}{3.354648in}}{\pgfqpoint{1.829879in}{3.357921in}}{\pgfqpoint{1.821643in}{3.357921in}}%
\pgfpathcurveto{\pgfqpoint{1.813407in}{3.357921in}}{\pgfqpoint{1.805507in}{3.354648in}}{\pgfqpoint{1.799683in}{3.348825in}}%
\pgfpathcurveto{\pgfqpoint{1.793859in}{3.343001in}}{\pgfqpoint{1.790586in}{3.335101in}}{\pgfqpoint{1.790586in}{3.326864in}}%
\pgfpathcurveto{\pgfqpoint{1.790586in}{3.318628in}}{\pgfqpoint{1.793859in}{3.310728in}}{\pgfqpoint{1.799683in}{3.304904in}}%
\pgfpathcurveto{\pgfqpoint{1.805507in}{3.299080in}}{\pgfqpoint{1.813407in}{3.295808in}}{\pgfqpoint{1.821643in}{3.295808in}}%
\pgfpathclose%
\pgfusepath{stroke,fill}%
\end{pgfscope}%
\begin{pgfscope}%
\pgfpathrectangle{\pgfqpoint{0.100000in}{0.212622in}}{\pgfqpoint{3.696000in}{3.696000in}}%
\pgfusepath{clip}%
\pgfsetbuttcap%
\pgfsetroundjoin%
\definecolor{currentfill}{rgb}{0.121569,0.466667,0.705882}%
\pgfsetfillcolor{currentfill}%
\pgfsetfillopacity{0.323250}%
\pgfsetlinewidth{1.003750pt}%
\definecolor{currentstroke}{rgb}{0.121569,0.466667,0.705882}%
\pgfsetstrokecolor{currentstroke}%
\pgfsetstrokeopacity{0.323250}%
\pgfsetdash{}{0pt}%
\pgfpathmoveto{\pgfqpoint{1.703385in}{3.257370in}}%
\pgfpathcurveto{\pgfqpoint{1.711621in}{3.257370in}}{\pgfqpoint{1.719521in}{3.260642in}}{\pgfqpoint{1.725345in}{3.266466in}}%
\pgfpathcurveto{\pgfqpoint{1.731169in}{3.272290in}}{\pgfqpoint{1.734442in}{3.280190in}}{\pgfqpoint{1.734442in}{3.288427in}}%
\pgfpathcurveto{\pgfqpoint{1.734442in}{3.296663in}}{\pgfqpoint{1.731169in}{3.304563in}}{\pgfqpoint{1.725345in}{3.310387in}}%
\pgfpathcurveto{\pgfqpoint{1.719521in}{3.316211in}}{\pgfqpoint{1.711621in}{3.319483in}}{\pgfqpoint{1.703385in}{3.319483in}}%
\pgfpathcurveto{\pgfqpoint{1.695149in}{3.319483in}}{\pgfqpoint{1.687249in}{3.316211in}}{\pgfqpoint{1.681425in}{3.310387in}}%
\pgfpathcurveto{\pgfqpoint{1.675601in}{3.304563in}}{\pgfqpoint{1.672329in}{3.296663in}}{\pgfqpoint{1.672329in}{3.288427in}}%
\pgfpathcurveto{\pgfqpoint{1.672329in}{3.280190in}}{\pgfqpoint{1.675601in}{3.272290in}}{\pgfqpoint{1.681425in}{3.266466in}}%
\pgfpathcurveto{\pgfqpoint{1.687249in}{3.260642in}}{\pgfqpoint{1.695149in}{3.257370in}}{\pgfqpoint{1.703385in}{3.257370in}}%
\pgfpathclose%
\pgfusepath{stroke,fill}%
\end{pgfscope}%
\begin{pgfscope}%
\pgfpathrectangle{\pgfqpoint{0.100000in}{0.212622in}}{\pgfqpoint{3.696000in}{3.696000in}}%
\pgfusepath{clip}%
\pgfsetbuttcap%
\pgfsetroundjoin%
\definecolor{currentfill}{rgb}{0.121569,0.466667,0.705882}%
\pgfsetfillcolor{currentfill}%
\pgfsetfillopacity{0.323546}%
\pgfsetlinewidth{1.003750pt}%
\definecolor{currentstroke}{rgb}{0.121569,0.466667,0.705882}%
\pgfsetstrokecolor{currentstroke}%
\pgfsetstrokeopacity{0.323546}%
\pgfsetdash{}{0pt}%
\pgfpathmoveto{\pgfqpoint{1.702768in}{3.255989in}}%
\pgfpathcurveto{\pgfqpoint{1.711004in}{3.255989in}}{\pgfqpoint{1.718904in}{3.259261in}}{\pgfqpoint{1.724728in}{3.265085in}}%
\pgfpathcurveto{\pgfqpoint{1.730552in}{3.270909in}}{\pgfqpoint{1.733825in}{3.278809in}}{\pgfqpoint{1.733825in}{3.287045in}}%
\pgfpathcurveto{\pgfqpoint{1.733825in}{3.295282in}}{\pgfqpoint{1.730552in}{3.303182in}}{\pgfqpoint{1.724728in}{3.309006in}}%
\pgfpathcurveto{\pgfqpoint{1.718904in}{3.314830in}}{\pgfqpoint{1.711004in}{3.318102in}}{\pgfqpoint{1.702768in}{3.318102in}}%
\pgfpathcurveto{\pgfqpoint{1.694532in}{3.318102in}}{\pgfqpoint{1.686632in}{3.314830in}}{\pgfqpoint{1.680808in}{3.309006in}}%
\pgfpathcurveto{\pgfqpoint{1.674984in}{3.303182in}}{\pgfqpoint{1.671712in}{3.295282in}}{\pgfqpoint{1.671712in}{3.287045in}}%
\pgfpathcurveto{\pgfqpoint{1.671712in}{3.278809in}}{\pgfqpoint{1.674984in}{3.270909in}}{\pgfqpoint{1.680808in}{3.265085in}}%
\pgfpathcurveto{\pgfqpoint{1.686632in}{3.259261in}}{\pgfqpoint{1.694532in}{3.255989in}}{\pgfqpoint{1.702768in}{3.255989in}}%
\pgfpathclose%
\pgfusepath{stroke,fill}%
\end{pgfscope}%
\begin{pgfscope}%
\pgfpathrectangle{\pgfqpoint{0.100000in}{0.212622in}}{\pgfqpoint{3.696000in}{3.696000in}}%
\pgfusepath{clip}%
\pgfsetbuttcap%
\pgfsetroundjoin%
\definecolor{currentfill}{rgb}{0.121569,0.466667,0.705882}%
\pgfsetfillcolor{currentfill}%
\pgfsetfillopacity{0.324099}%
\pgfsetlinewidth{1.003750pt}%
\definecolor{currentstroke}{rgb}{0.121569,0.466667,0.705882}%
\pgfsetstrokecolor{currentstroke}%
\pgfsetstrokeopacity{0.324099}%
\pgfsetdash{}{0pt}%
\pgfpathmoveto{\pgfqpoint{1.701693in}{3.253489in}}%
\pgfpathcurveto{\pgfqpoint{1.709929in}{3.253489in}}{\pgfqpoint{1.717829in}{3.256762in}}{\pgfqpoint{1.723653in}{3.262586in}}%
\pgfpathcurveto{\pgfqpoint{1.729477in}{3.268409in}}{\pgfqpoint{1.732749in}{3.276310in}}{\pgfqpoint{1.732749in}{3.284546in}}%
\pgfpathcurveto{\pgfqpoint{1.732749in}{3.292782in}}{\pgfqpoint{1.729477in}{3.300682in}}{\pgfqpoint{1.723653in}{3.306506in}}%
\pgfpathcurveto{\pgfqpoint{1.717829in}{3.312330in}}{\pgfqpoint{1.709929in}{3.315602in}}{\pgfqpoint{1.701693in}{3.315602in}}%
\pgfpathcurveto{\pgfqpoint{1.693456in}{3.315602in}}{\pgfqpoint{1.685556in}{3.312330in}}{\pgfqpoint{1.679732in}{3.306506in}}%
\pgfpathcurveto{\pgfqpoint{1.673909in}{3.300682in}}{\pgfqpoint{1.670636in}{3.292782in}}{\pgfqpoint{1.670636in}{3.284546in}}%
\pgfpathcurveto{\pgfqpoint{1.670636in}{3.276310in}}{\pgfqpoint{1.673909in}{3.268409in}}{\pgfqpoint{1.679732in}{3.262586in}}%
\pgfpathcurveto{\pgfqpoint{1.685556in}{3.256762in}}{\pgfqpoint{1.693456in}{3.253489in}}{\pgfqpoint{1.701693in}{3.253489in}}%
\pgfpathclose%
\pgfusepath{stroke,fill}%
\end{pgfscope}%
\begin{pgfscope}%
\pgfpathrectangle{\pgfqpoint{0.100000in}{0.212622in}}{\pgfqpoint{3.696000in}{3.696000in}}%
\pgfusepath{clip}%
\pgfsetbuttcap%
\pgfsetroundjoin%
\definecolor{currentfill}{rgb}{0.121569,0.466667,0.705882}%
\pgfsetfillcolor{currentfill}%
\pgfsetfillopacity{0.324620}%
\pgfsetlinewidth{1.003750pt}%
\definecolor{currentstroke}{rgb}{0.121569,0.466667,0.705882}%
\pgfsetstrokecolor{currentstroke}%
\pgfsetstrokeopacity{0.324620}%
\pgfsetdash{}{0pt}%
\pgfpathmoveto{\pgfqpoint{1.824014in}{3.289477in}}%
\pgfpathcurveto{\pgfqpoint{1.832250in}{3.289477in}}{\pgfqpoint{1.840150in}{3.292749in}}{\pgfqpoint{1.845974in}{3.298573in}}%
\pgfpathcurveto{\pgfqpoint{1.851798in}{3.304397in}}{\pgfqpoint{1.855070in}{3.312297in}}{\pgfqpoint{1.855070in}{3.320533in}}%
\pgfpathcurveto{\pgfqpoint{1.855070in}{3.328770in}}{\pgfqpoint{1.851798in}{3.336670in}}{\pgfqpoint{1.845974in}{3.342494in}}%
\pgfpathcurveto{\pgfqpoint{1.840150in}{3.348318in}}{\pgfqpoint{1.832250in}{3.351590in}}{\pgfqpoint{1.824014in}{3.351590in}}%
\pgfpathcurveto{\pgfqpoint{1.815777in}{3.351590in}}{\pgfqpoint{1.807877in}{3.348318in}}{\pgfqpoint{1.802053in}{3.342494in}}%
\pgfpathcurveto{\pgfqpoint{1.796230in}{3.336670in}}{\pgfqpoint{1.792957in}{3.328770in}}{\pgfqpoint{1.792957in}{3.320533in}}%
\pgfpathcurveto{\pgfqpoint{1.792957in}{3.312297in}}{\pgfqpoint{1.796230in}{3.304397in}}{\pgfqpoint{1.802053in}{3.298573in}}%
\pgfpathcurveto{\pgfqpoint{1.807877in}{3.292749in}}{\pgfqpoint{1.815777in}{3.289477in}}{\pgfqpoint{1.824014in}{3.289477in}}%
\pgfpathclose%
\pgfusepath{stroke,fill}%
\end{pgfscope}%
\begin{pgfscope}%
\pgfpathrectangle{\pgfqpoint{0.100000in}{0.212622in}}{\pgfqpoint{3.696000in}{3.696000in}}%
\pgfusepath{clip}%
\pgfsetbuttcap%
\pgfsetroundjoin%
\definecolor{currentfill}{rgb}{0.121569,0.466667,0.705882}%
\pgfsetfillcolor{currentfill}%
\pgfsetfillopacity{0.325091}%
\pgfsetlinewidth{1.003750pt}%
\definecolor{currentstroke}{rgb}{0.121569,0.466667,0.705882}%
\pgfsetstrokecolor{currentstroke}%
\pgfsetstrokeopacity{0.325091}%
\pgfsetdash{}{0pt}%
\pgfpathmoveto{\pgfqpoint{1.699715in}{3.248910in}}%
\pgfpathcurveto{\pgfqpoint{1.707951in}{3.248910in}}{\pgfqpoint{1.715852in}{3.252182in}}{\pgfqpoint{1.721675in}{3.258006in}}%
\pgfpathcurveto{\pgfqpoint{1.727499in}{3.263830in}}{\pgfqpoint{1.730772in}{3.271730in}}{\pgfqpoint{1.730772in}{3.279967in}}%
\pgfpathcurveto{\pgfqpoint{1.730772in}{3.288203in}}{\pgfqpoint{1.727499in}{3.296103in}}{\pgfqpoint{1.721675in}{3.301927in}}%
\pgfpathcurveto{\pgfqpoint{1.715852in}{3.307751in}}{\pgfqpoint{1.707951in}{3.311023in}}{\pgfqpoint{1.699715in}{3.311023in}}%
\pgfpathcurveto{\pgfqpoint{1.691479in}{3.311023in}}{\pgfqpoint{1.683579in}{3.307751in}}{\pgfqpoint{1.677755in}{3.301927in}}%
\pgfpathcurveto{\pgfqpoint{1.671931in}{3.296103in}}{\pgfqpoint{1.668659in}{3.288203in}}{\pgfqpoint{1.668659in}{3.279967in}}%
\pgfpathcurveto{\pgfqpoint{1.668659in}{3.271730in}}{\pgfqpoint{1.671931in}{3.263830in}}{\pgfqpoint{1.677755in}{3.258006in}}%
\pgfpathcurveto{\pgfqpoint{1.683579in}{3.252182in}}{\pgfqpoint{1.691479in}{3.248910in}}{\pgfqpoint{1.699715in}{3.248910in}}%
\pgfpathclose%
\pgfusepath{stroke,fill}%
\end{pgfscope}%
\begin{pgfscope}%
\pgfpathrectangle{\pgfqpoint{0.100000in}{0.212622in}}{\pgfqpoint{3.696000in}{3.696000in}}%
\pgfusepath{clip}%
\pgfsetbuttcap%
\pgfsetroundjoin%
\definecolor{currentfill}{rgb}{0.121569,0.466667,0.705882}%
\pgfsetfillcolor{currentfill}%
\pgfsetfillopacity{0.325522}%
\pgfsetlinewidth{1.003750pt}%
\definecolor{currentstroke}{rgb}{0.121569,0.466667,0.705882}%
\pgfsetstrokecolor{currentstroke}%
\pgfsetstrokeopacity{0.325522}%
\pgfsetdash{}{0pt}%
\pgfpathmoveto{\pgfqpoint{1.698938in}{3.247029in}}%
\pgfpathcurveto{\pgfqpoint{1.707174in}{3.247029in}}{\pgfqpoint{1.715074in}{3.250302in}}{\pgfqpoint{1.720898in}{3.256126in}}%
\pgfpathcurveto{\pgfqpoint{1.726722in}{3.261950in}}{\pgfqpoint{1.729994in}{3.269850in}}{\pgfqpoint{1.729994in}{3.278086in}}%
\pgfpathcurveto{\pgfqpoint{1.729994in}{3.286322in}}{\pgfqpoint{1.726722in}{3.294222in}}{\pgfqpoint{1.720898in}{3.300046in}}%
\pgfpathcurveto{\pgfqpoint{1.715074in}{3.305870in}}{\pgfqpoint{1.707174in}{3.309142in}}{\pgfqpoint{1.698938in}{3.309142in}}%
\pgfpathcurveto{\pgfqpoint{1.690701in}{3.309142in}}{\pgfqpoint{1.682801in}{3.305870in}}{\pgfqpoint{1.676977in}{3.300046in}}%
\pgfpathcurveto{\pgfqpoint{1.671153in}{3.294222in}}{\pgfqpoint{1.667881in}{3.286322in}}{\pgfqpoint{1.667881in}{3.278086in}}%
\pgfpathcurveto{\pgfqpoint{1.667881in}{3.269850in}}{\pgfqpoint{1.671153in}{3.261950in}}{\pgfqpoint{1.676977in}{3.256126in}}%
\pgfpathcurveto{\pgfqpoint{1.682801in}{3.250302in}}{\pgfqpoint{1.690701in}{3.247029in}}{\pgfqpoint{1.698938in}{3.247029in}}%
\pgfpathclose%
\pgfusepath{stroke,fill}%
\end{pgfscope}%
\begin{pgfscope}%
\pgfpathrectangle{\pgfqpoint{0.100000in}{0.212622in}}{\pgfqpoint{3.696000in}{3.696000in}}%
\pgfusepath{clip}%
\pgfsetbuttcap%
\pgfsetroundjoin%
\definecolor{currentfill}{rgb}{0.121569,0.466667,0.705882}%
\pgfsetfillcolor{currentfill}%
\pgfsetfillopacity{0.325619}%
\pgfsetlinewidth{1.003750pt}%
\definecolor{currentstroke}{rgb}{0.121569,0.466667,0.705882}%
\pgfsetstrokecolor{currentstroke}%
\pgfsetstrokeopacity{0.325619}%
\pgfsetdash{}{0pt}%
\pgfpathmoveto{\pgfqpoint{1.825139in}{3.285850in}}%
\pgfpathcurveto{\pgfqpoint{1.833375in}{3.285850in}}{\pgfqpoint{1.841275in}{3.289122in}}{\pgfqpoint{1.847099in}{3.294946in}}%
\pgfpathcurveto{\pgfqpoint{1.852923in}{3.300770in}}{\pgfqpoint{1.856195in}{3.308670in}}{\pgfqpoint{1.856195in}{3.316906in}}%
\pgfpathcurveto{\pgfqpoint{1.856195in}{3.325142in}}{\pgfqpoint{1.852923in}{3.333043in}}{\pgfqpoint{1.847099in}{3.338866in}}%
\pgfpathcurveto{\pgfqpoint{1.841275in}{3.344690in}}{\pgfqpoint{1.833375in}{3.347963in}}{\pgfqpoint{1.825139in}{3.347963in}}%
\pgfpathcurveto{\pgfqpoint{1.816903in}{3.347963in}}{\pgfqpoint{1.809003in}{3.344690in}}{\pgfqpoint{1.803179in}{3.338866in}}%
\pgfpathcurveto{\pgfqpoint{1.797355in}{3.333043in}}{\pgfqpoint{1.794082in}{3.325142in}}{\pgfqpoint{1.794082in}{3.316906in}}%
\pgfpathcurveto{\pgfqpoint{1.794082in}{3.308670in}}{\pgfqpoint{1.797355in}{3.300770in}}{\pgfqpoint{1.803179in}{3.294946in}}%
\pgfpathcurveto{\pgfqpoint{1.809003in}{3.289122in}}{\pgfqpoint{1.816903in}{3.285850in}}{\pgfqpoint{1.825139in}{3.285850in}}%
\pgfpathclose%
\pgfusepath{stroke,fill}%
\end{pgfscope}%
\begin{pgfscope}%
\pgfpathrectangle{\pgfqpoint{0.100000in}{0.212622in}}{\pgfqpoint{3.696000in}{3.696000in}}%
\pgfusepath{clip}%
\pgfsetbuttcap%
\pgfsetroundjoin%
\definecolor{currentfill}{rgb}{0.121569,0.466667,0.705882}%
\pgfsetfillcolor{currentfill}%
\pgfsetfillopacity{0.326300}%
\pgfsetlinewidth{1.003750pt}%
\definecolor{currentstroke}{rgb}{0.121569,0.466667,0.705882}%
\pgfsetstrokecolor{currentstroke}%
\pgfsetstrokeopacity{0.326300}%
\pgfsetdash{}{0pt}%
\pgfpathmoveto{\pgfqpoint{1.697497in}{3.243620in}}%
\pgfpathcurveto{\pgfqpoint{1.705733in}{3.243620in}}{\pgfqpoint{1.713633in}{3.246892in}}{\pgfqpoint{1.719457in}{3.252716in}}%
\pgfpathcurveto{\pgfqpoint{1.725281in}{3.258540in}}{\pgfqpoint{1.728553in}{3.266440in}}{\pgfqpoint{1.728553in}{3.274676in}}%
\pgfpathcurveto{\pgfqpoint{1.728553in}{3.282913in}}{\pgfqpoint{1.725281in}{3.290813in}}{\pgfqpoint{1.719457in}{3.296637in}}%
\pgfpathcurveto{\pgfqpoint{1.713633in}{3.302461in}}{\pgfqpoint{1.705733in}{3.305733in}}{\pgfqpoint{1.697497in}{3.305733in}}%
\pgfpathcurveto{\pgfqpoint{1.689260in}{3.305733in}}{\pgfqpoint{1.681360in}{3.302461in}}{\pgfqpoint{1.675536in}{3.296637in}}%
\pgfpathcurveto{\pgfqpoint{1.669712in}{3.290813in}}{\pgfqpoint{1.666440in}{3.282913in}}{\pgfqpoint{1.666440in}{3.274676in}}%
\pgfpathcurveto{\pgfqpoint{1.666440in}{3.266440in}}{\pgfqpoint{1.669712in}{3.258540in}}{\pgfqpoint{1.675536in}{3.252716in}}%
\pgfpathcurveto{\pgfqpoint{1.681360in}{3.246892in}}{\pgfqpoint{1.689260in}{3.243620in}}{\pgfqpoint{1.697497in}{3.243620in}}%
\pgfpathclose%
\pgfusepath{stroke,fill}%
\end{pgfscope}%
\begin{pgfscope}%
\pgfpathrectangle{\pgfqpoint{0.100000in}{0.212622in}}{\pgfqpoint{3.696000in}{3.696000in}}%
\pgfusepath{clip}%
\pgfsetbuttcap%
\pgfsetroundjoin%
\definecolor{currentfill}{rgb}{0.121569,0.466667,0.705882}%
\pgfsetfillcolor{currentfill}%
\pgfsetfillopacity{0.326654}%
\pgfsetlinewidth{1.003750pt}%
\definecolor{currentstroke}{rgb}{0.121569,0.466667,0.705882}%
\pgfsetstrokecolor{currentstroke}%
\pgfsetstrokeopacity{0.326654}%
\pgfsetdash{}{0pt}%
\pgfpathmoveto{\pgfqpoint{1.696862in}{3.242058in}}%
\pgfpathcurveto{\pgfqpoint{1.705098in}{3.242058in}}{\pgfqpoint{1.712998in}{3.245330in}}{\pgfqpoint{1.718822in}{3.251154in}}%
\pgfpathcurveto{\pgfqpoint{1.724646in}{3.256978in}}{\pgfqpoint{1.727918in}{3.264878in}}{\pgfqpoint{1.727918in}{3.273114in}}%
\pgfpathcurveto{\pgfqpoint{1.727918in}{3.281350in}}{\pgfqpoint{1.724646in}{3.289250in}}{\pgfqpoint{1.718822in}{3.295074in}}%
\pgfpathcurveto{\pgfqpoint{1.712998in}{3.300898in}}{\pgfqpoint{1.705098in}{3.304171in}}{\pgfqpoint{1.696862in}{3.304171in}}%
\pgfpathcurveto{\pgfqpoint{1.688625in}{3.304171in}}{\pgfqpoint{1.680725in}{3.300898in}}{\pgfqpoint{1.674901in}{3.295074in}}%
\pgfpathcurveto{\pgfqpoint{1.669077in}{3.289250in}}{\pgfqpoint{1.665805in}{3.281350in}}{\pgfqpoint{1.665805in}{3.273114in}}%
\pgfpathcurveto{\pgfqpoint{1.665805in}{3.264878in}}{\pgfqpoint{1.669077in}{3.256978in}}{\pgfqpoint{1.674901in}{3.251154in}}%
\pgfpathcurveto{\pgfqpoint{1.680725in}{3.245330in}}{\pgfqpoint{1.688625in}{3.242058in}}{\pgfqpoint{1.696862in}{3.242058in}}%
\pgfpathclose%
\pgfusepath{stroke,fill}%
\end{pgfscope}%
\begin{pgfscope}%
\pgfpathrectangle{\pgfqpoint{0.100000in}{0.212622in}}{\pgfqpoint{3.696000in}{3.696000in}}%
\pgfusepath{clip}%
\pgfsetbuttcap%
\pgfsetroundjoin%
\definecolor{currentfill}{rgb}{0.121569,0.466667,0.705882}%
\pgfsetfillcolor{currentfill}%
\pgfsetfillopacity{0.327123}%
\pgfsetlinewidth{1.003750pt}%
\definecolor{currentstroke}{rgb}{0.121569,0.466667,0.705882}%
\pgfsetstrokecolor{currentstroke}%
\pgfsetstrokeopacity{0.327123}%
\pgfsetdash{}{0pt}%
\pgfpathmoveto{\pgfqpoint{1.826839in}{3.280721in}}%
\pgfpathcurveto{\pgfqpoint{1.835076in}{3.280721in}}{\pgfqpoint{1.842976in}{3.283993in}}{\pgfqpoint{1.848800in}{3.289817in}}%
\pgfpathcurveto{\pgfqpoint{1.854623in}{3.295641in}}{\pgfqpoint{1.857896in}{3.303541in}}{\pgfqpoint{1.857896in}{3.311777in}}%
\pgfpathcurveto{\pgfqpoint{1.857896in}{3.320013in}}{\pgfqpoint{1.854623in}{3.327914in}}{\pgfqpoint{1.848800in}{3.333737in}}%
\pgfpathcurveto{\pgfqpoint{1.842976in}{3.339561in}}{\pgfqpoint{1.835076in}{3.342834in}}{\pgfqpoint{1.826839in}{3.342834in}}%
\pgfpathcurveto{\pgfqpoint{1.818603in}{3.342834in}}{\pgfqpoint{1.810703in}{3.339561in}}{\pgfqpoint{1.804879in}{3.333737in}}%
\pgfpathcurveto{\pgfqpoint{1.799055in}{3.327914in}}{\pgfqpoint{1.795783in}{3.320013in}}{\pgfqpoint{1.795783in}{3.311777in}}%
\pgfpathcurveto{\pgfqpoint{1.795783in}{3.303541in}}{\pgfqpoint{1.799055in}{3.295641in}}{\pgfqpoint{1.804879in}{3.289817in}}%
\pgfpathcurveto{\pgfqpoint{1.810703in}{3.283993in}}{\pgfqpoint{1.818603in}{3.280721in}}{\pgfqpoint{1.826839in}{3.280721in}}%
\pgfpathclose%
\pgfusepath{stroke,fill}%
\end{pgfscope}%
\begin{pgfscope}%
\pgfpathrectangle{\pgfqpoint{0.100000in}{0.212622in}}{\pgfqpoint{3.696000in}{3.696000in}}%
\pgfusepath{clip}%
\pgfsetbuttcap%
\pgfsetroundjoin%
\definecolor{currentfill}{rgb}{0.121569,0.466667,0.705882}%
\pgfsetfillcolor{currentfill}%
\pgfsetfillopacity{0.327283}%
\pgfsetlinewidth{1.003750pt}%
\definecolor{currentstroke}{rgb}{0.121569,0.466667,0.705882}%
\pgfsetstrokecolor{currentstroke}%
\pgfsetstrokeopacity{0.327283}%
\pgfsetdash{}{0pt}%
\pgfpathmoveto{\pgfqpoint{1.695642in}{3.239222in}}%
\pgfpathcurveto{\pgfqpoint{1.703879in}{3.239222in}}{\pgfqpoint{1.711779in}{3.242495in}}{\pgfqpoint{1.717603in}{3.248319in}}%
\pgfpathcurveto{\pgfqpoint{1.723427in}{3.254143in}}{\pgfqpoint{1.726699in}{3.262043in}}{\pgfqpoint{1.726699in}{3.270279in}}%
\pgfpathcurveto{\pgfqpoint{1.726699in}{3.278515in}}{\pgfqpoint{1.723427in}{3.286415in}}{\pgfqpoint{1.717603in}{3.292239in}}%
\pgfpathcurveto{\pgfqpoint{1.711779in}{3.298063in}}{\pgfqpoint{1.703879in}{3.301335in}}{\pgfqpoint{1.695642in}{3.301335in}}%
\pgfpathcurveto{\pgfqpoint{1.687406in}{3.301335in}}{\pgfqpoint{1.679506in}{3.298063in}}{\pgfqpoint{1.673682in}{3.292239in}}%
\pgfpathcurveto{\pgfqpoint{1.667858in}{3.286415in}}{\pgfqpoint{1.664586in}{3.278515in}}{\pgfqpoint{1.664586in}{3.270279in}}%
\pgfpathcurveto{\pgfqpoint{1.664586in}{3.262043in}}{\pgfqpoint{1.667858in}{3.254143in}}{\pgfqpoint{1.673682in}{3.248319in}}%
\pgfpathcurveto{\pgfqpoint{1.679506in}{3.242495in}}{\pgfqpoint{1.687406in}{3.239222in}}{\pgfqpoint{1.695642in}{3.239222in}}%
\pgfpathclose%
\pgfusepath{stroke,fill}%
\end{pgfscope}%
\begin{pgfscope}%
\pgfpathrectangle{\pgfqpoint{0.100000in}{0.212622in}}{\pgfqpoint{3.696000in}{3.696000in}}%
\pgfusepath{clip}%
\pgfsetbuttcap%
\pgfsetroundjoin%
\definecolor{currentfill}{rgb}{0.121569,0.466667,0.705882}%
\pgfsetfillcolor{currentfill}%
\pgfsetfillopacity{0.327920}%
\pgfsetlinewidth{1.003750pt}%
\definecolor{currentstroke}{rgb}{0.121569,0.466667,0.705882}%
\pgfsetstrokecolor{currentstroke}%
\pgfsetstrokeopacity{0.327920}%
\pgfsetdash{}{0pt}%
\pgfpathmoveto{\pgfqpoint{1.827640in}{3.277779in}}%
\pgfpathcurveto{\pgfqpoint{1.835877in}{3.277779in}}{\pgfqpoint{1.843777in}{3.281051in}}{\pgfqpoint{1.849601in}{3.286875in}}%
\pgfpathcurveto{\pgfqpoint{1.855425in}{3.292699in}}{\pgfqpoint{1.858697in}{3.300599in}}{\pgfqpoint{1.858697in}{3.308836in}}%
\pgfpathcurveto{\pgfqpoint{1.858697in}{3.317072in}}{\pgfqpoint{1.855425in}{3.324972in}}{\pgfqpoint{1.849601in}{3.330796in}}%
\pgfpathcurveto{\pgfqpoint{1.843777in}{3.336620in}}{\pgfqpoint{1.835877in}{3.339892in}}{\pgfqpoint{1.827640in}{3.339892in}}%
\pgfpathcurveto{\pgfqpoint{1.819404in}{3.339892in}}{\pgfqpoint{1.811504in}{3.336620in}}{\pgfqpoint{1.805680in}{3.330796in}}%
\pgfpathcurveto{\pgfqpoint{1.799856in}{3.324972in}}{\pgfqpoint{1.796584in}{3.317072in}}{\pgfqpoint{1.796584in}{3.308836in}}%
\pgfpathcurveto{\pgfqpoint{1.796584in}{3.300599in}}{\pgfqpoint{1.799856in}{3.292699in}}{\pgfqpoint{1.805680in}{3.286875in}}%
\pgfpathcurveto{\pgfqpoint{1.811504in}{3.281051in}}{\pgfqpoint{1.819404in}{3.277779in}}{\pgfqpoint{1.827640in}{3.277779in}}%
\pgfpathclose%
\pgfusepath{stroke,fill}%
\end{pgfscope}%
\begin{pgfscope}%
\pgfpathrectangle{\pgfqpoint{0.100000in}{0.212622in}}{\pgfqpoint{3.696000in}{3.696000in}}%
\pgfusepath{clip}%
\pgfsetbuttcap%
\pgfsetroundjoin%
\definecolor{currentfill}{rgb}{0.121569,0.466667,0.705882}%
\pgfsetfillcolor{currentfill}%
\pgfsetfillopacity{0.328461}%
\pgfsetlinewidth{1.003750pt}%
\definecolor{currentstroke}{rgb}{0.121569,0.466667,0.705882}%
\pgfsetstrokecolor{currentstroke}%
\pgfsetstrokeopacity{0.328461}%
\pgfsetdash{}{0pt}%
\pgfpathmoveto{\pgfqpoint{1.693526in}{3.234096in}}%
\pgfpathcurveto{\pgfqpoint{1.701763in}{3.234096in}}{\pgfqpoint{1.709663in}{3.237368in}}{\pgfqpoint{1.715487in}{3.243192in}}%
\pgfpathcurveto{\pgfqpoint{1.721311in}{3.249016in}}{\pgfqpoint{1.724583in}{3.256916in}}{\pgfqpoint{1.724583in}{3.265152in}}%
\pgfpathcurveto{\pgfqpoint{1.724583in}{3.273388in}}{\pgfqpoint{1.721311in}{3.281288in}}{\pgfqpoint{1.715487in}{3.287112in}}%
\pgfpathcurveto{\pgfqpoint{1.709663in}{3.292936in}}{\pgfqpoint{1.701763in}{3.296209in}}{\pgfqpoint{1.693526in}{3.296209in}}%
\pgfpathcurveto{\pgfqpoint{1.685290in}{3.296209in}}{\pgfqpoint{1.677390in}{3.292936in}}{\pgfqpoint{1.671566in}{3.287112in}}%
\pgfpathcurveto{\pgfqpoint{1.665742in}{3.281288in}}{\pgfqpoint{1.662470in}{3.273388in}}{\pgfqpoint{1.662470in}{3.265152in}}%
\pgfpathcurveto{\pgfqpoint{1.662470in}{3.256916in}}{\pgfqpoint{1.665742in}{3.249016in}}{\pgfqpoint{1.671566in}{3.243192in}}%
\pgfpathcurveto{\pgfqpoint{1.677390in}{3.237368in}}{\pgfqpoint{1.685290in}{3.234096in}}{\pgfqpoint{1.693526in}{3.234096in}}%
\pgfpathclose%
\pgfusepath{stroke,fill}%
\end{pgfscope}%
\begin{pgfscope}%
\pgfpathrectangle{\pgfqpoint{0.100000in}{0.212622in}}{\pgfqpoint{3.696000in}{3.696000in}}%
\pgfusepath{clip}%
\pgfsetbuttcap%
\pgfsetroundjoin%
\definecolor{currentfill}{rgb}{0.121569,0.466667,0.705882}%
\pgfsetfillcolor{currentfill}%
\pgfsetfillopacity{0.328923}%
\pgfsetlinewidth{1.003750pt}%
\definecolor{currentstroke}{rgb}{0.121569,0.466667,0.705882}%
\pgfsetstrokecolor{currentstroke}%
\pgfsetstrokeopacity{0.328923}%
\pgfsetdash{}{0pt}%
\pgfpathmoveto{\pgfqpoint{1.692590in}{3.231953in}}%
\pgfpathcurveto{\pgfqpoint{1.700826in}{3.231953in}}{\pgfqpoint{1.708726in}{3.235225in}}{\pgfqpoint{1.714550in}{3.241049in}}%
\pgfpathcurveto{\pgfqpoint{1.720374in}{3.246873in}}{\pgfqpoint{1.723647in}{3.254773in}}{\pgfqpoint{1.723647in}{3.263009in}}%
\pgfpathcurveto{\pgfqpoint{1.723647in}{3.271246in}}{\pgfqpoint{1.720374in}{3.279146in}}{\pgfqpoint{1.714550in}{3.284970in}}%
\pgfpathcurveto{\pgfqpoint{1.708726in}{3.290794in}}{\pgfqpoint{1.700826in}{3.294066in}}{\pgfqpoint{1.692590in}{3.294066in}}%
\pgfpathcurveto{\pgfqpoint{1.684354in}{3.294066in}}{\pgfqpoint{1.676454in}{3.290794in}}{\pgfqpoint{1.670630in}{3.284970in}}%
\pgfpathcurveto{\pgfqpoint{1.664806in}{3.279146in}}{\pgfqpoint{1.661534in}{3.271246in}}{\pgfqpoint{1.661534in}{3.263009in}}%
\pgfpathcurveto{\pgfqpoint{1.661534in}{3.254773in}}{\pgfqpoint{1.664806in}{3.246873in}}{\pgfqpoint{1.670630in}{3.241049in}}%
\pgfpathcurveto{\pgfqpoint{1.676454in}{3.235225in}}{\pgfqpoint{1.684354in}{3.231953in}}{\pgfqpoint{1.692590in}{3.231953in}}%
\pgfpathclose%
\pgfusepath{stroke,fill}%
\end{pgfscope}%
\begin{pgfscope}%
\pgfpathrectangle{\pgfqpoint{0.100000in}{0.212622in}}{\pgfqpoint{3.696000in}{3.696000in}}%
\pgfusepath{clip}%
\pgfsetbuttcap%
\pgfsetroundjoin%
\definecolor{currentfill}{rgb}{0.121569,0.466667,0.705882}%
\pgfsetfillcolor{currentfill}%
\pgfsetfillopacity{0.329235}%
\pgfsetlinewidth{1.003750pt}%
\definecolor{currentstroke}{rgb}{0.121569,0.466667,0.705882}%
\pgfsetstrokecolor{currentstroke}%
\pgfsetstrokeopacity{0.329235}%
\pgfsetdash{}{0pt}%
\pgfpathmoveto{\pgfqpoint{1.829041in}{3.273206in}}%
\pgfpathcurveto{\pgfqpoint{1.837277in}{3.273206in}}{\pgfqpoint{1.845178in}{3.276478in}}{\pgfqpoint{1.851001in}{3.282302in}}%
\pgfpathcurveto{\pgfqpoint{1.856825in}{3.288126in}}{\pgfqpoint{1.860098in}{3.296026in}}{\pgfqpoint{1.860098in}{3.304262in}}%
\pgfpathcurveto{\pgfqpoint{1.860098in}{3.312499in}}{\pgfqpoint{1.856825in}{3.320399in}}{\pgfqpoint{1.851001in}{3.326223in}}%
\pgfpathcurveto{\pgfqpoint{1.845178in}{3.332046in}}{\pgfqpoint{1.837277in}{3.335319in}}{\pgfqpoint{1.829041in}{3.335319in}}%
\pgfpathcurveto{\pgfqpoint{1.820805in}{3.335319in}}{\pgfqpoint{1.812905in}{3.332046in}}{\pgfqpoint{1.807081in}{3.326223in}}%
\pgfpathcurveto{\pgfqpoint{1.801257in}{3.320399in}}{\pgfqpoint{1.797985in}{3.312499in}}{\pgfqpoint{1.797985in}{3.304262in}}%
\pgfpathcurveto{\pgfqpoint{1.797985in}{3.296026in}}{\pgfqpoint{1.801257in}{3.288126in}}{\pgfqpoint{1.807081in}{3.282302in}}%
\pgfpathcurveto{\pgfqpoint{1.812905in}{3.276478in}}{\pgfqpoint{1.820805in}{3.273206in}}{\pgfqpoint{1.829041in}{3.273206in}}%
\pgfpathclose%
\pgfusepath{stroke,fill}%
\end{pgfscope}%
\begin{pgfscope}%
\pgfpathrectangle{\pgfqpoint{0.100000in}{0.212622in}}{\pgfqpoint{3.696000in}{3.696000in}}%
\pgfusepath{clip}%
\pgfsetbuttcap%
\pgfsetroundjoin%
\definecolor{currentfill}{rgb}{0.121569,0.466667,0.705882}%
\pgfsetfillcolor{currentfill}%
\pgfsetfillopacity{0.329790}%
\pgfsetlinewidth{1.003750pt}%
\definecolor{currentstroke}{rgb}{0.121569,0.466667,0.705882}%
\pgfsetstrokecolor{currentstroke}%
\pgfsetstrokeopacity{0.329790}%
\pgfsetdash{}{0pt}%
\pgfpathmoveto{\pgfqpoint{1.690938in}{3.228113in}}%
\pgfpathcurveto{\pgfqpoint{1.699174in}{3.228113in}}{\pgfqpoint{1.707074in}{3.231385in}}{\pgfqpoint{1.712898in}{3.237209in}}%
\pgfpathcurveto{\pgfqpoint{1.718722in}{3.243033in}}{\pgfqpoint{1.721994in}{3.250933in}}{\pgfqpoint{1.721994in}{3.259169in}}%
\pgfpathcurveto{\pgfqpoint{1.721994in}{3.267406in}}{\pgfqpoint{1.718722in}{3.275306in}}{\pgfqpoint{1.712898in}{3.281130in}}%
\pgfpathcurveto{\pgfqpoint{1.707074in}{3.286953in}}{\pgfqpoint{1.699174in}{3.290226in}}{\pgfqpoint{1.690938in}{3.290226in}}%
\pgfpathcurveto{\pgfqpoint{1.682702in}{3.290226in}}{\pgfqpoint{1.674802in}{3.286953in}}{\pgfqpoint{1.668978in}{3.281130in}}%
\pgfpathcurveto{\pgfqpoint{1.663154in}{3.275306in}}{\pgfqpoint{1.659881in}{3.267406in}}{\pgfqpoint{1.659881in}{3.259169in}}%
\pgfpathcurveto{\pgfqpoint{1.659881in}{3.250933in}}{\pgfqpoint{1.663154in}{3.243033in}}{\pgfqpoint{1.668978in}{3.237209in}}%
\pgfpathcurveto{\pgfqpoint{1.674802in}{3.231385in}}{\pgfqpoint{1.682702in}{3.228113in}}{\pgfqpoint{1.690938in}{3.228113in}}%
\pgfpathclose%
\pgfusepath{stroke,fill}%
\end{pgfscope}%
\begin{pgfscope}%
\pgfpathrectangle{\pgfqpoint{0.100000in}{0.212622in}}{\pgfqpoint{3.696000in}{3.696000in}}%
\pgfusepath{clip}%
\pgfsetbuttcap%
\pgfsetroundjoin%
\definecolor{currentfill}{rgb}{0.121569,0.466667,0.705882}%
\pgfsetfillcolor{currentfill}%
\pgfsetfillopacity{0.330084}%
\pgfsetlinewidth{1.003750pt}%
\definecolor{currentstroke}{rgb}{0.121569,0.466667,0.705882}%
\pgfsetstrokecolor{currentstroke}%
\pgfsetstrokeopacity{0.330084}%
\pgfsetdash{}{0pt}%
\pgfpathmoveto{\pgfqpoint{1.690337in}{3.226745in}}%
\pgfpathcurveto{\pgfqpoint{1.698573in}{3.226745in}}{\pgfqpoint{1.706473in}{3.230017in}}{\pgfqpoint{1.712297in}{3.235841in}}%
\pgfpathcurveto{\pgfqpoint{1.718121in}{3.241665in}}{\pgfqpoint{1.721393in}{3.249565in}}{\pgfqpoint{1.721393in}{3.257801in}}%
\pgfpathcurveto{\pgfqpoint{1.721393in}{3.266038in}}{\pgfqpoint{1.718121in}{3.273938in}}{\pgfqpoint{1.712297in}{3.279762in}}%
\pgfpathcurveto{\pgfqpoint{1.706473in}{3.285586in}}{\pgfqpoint{1.698573in}{3.288858in}}{\pgfqpoint{1.690337in}{3.288858in}}%
\pgfpathcurveto{\pgfqpoint{1.682100in}{3.288858in}}{\pgfqpoint{1.674200in}{3.285586in}}{\pgfqpoint{1.668376in}{3.279762in}}%
\pgfpathcurveto{\pgfqpoint{1.662552in}{3.273938in}}{\pgfqpoint{1.659280in}{3.266038in}}{\pgfqpoint{1.659280in}{3.257801in}}%
\pgfpathcurveto{\pgfqpoint{1.659280in}{3.249565in}}{\pgfqpoint{1.662552in}{3.241665in}}{\pgfqpoint{1.668376in}{3.235841in}}%
\pgfpathcurveto{\pgfqpoint{1.674200in}{3.230017in}}{\pgfqpoint{1.682100in}{3.226745in}}{\pgfqpoint{1.690337in}{3.226745in}}%
\pgfpathclose%
\pgfusepath{stroke,fill}%
\end{pgfscope}%
\begin{pgfscope}%
\pgfpathrectangle{\pgfqpoint{0.100000in}{0.212622in}}{\pgfqpoint{3.696000in}{3.696000in}}%
\pgfusepath{clip}%
\pgfsetbuttcap%
\pgfsetroundjoin%
\definecolor{currentfill}{rgb}{0.121569,0.466667,0.705882}%
\pgfsetfillcolor{currentfill}%
\pgfsetfillopacity{0.330660}%
\pgfsetlinewidth{1.003750pt}%
\definecolor{currentstroke}{rgb}{0.121569,0.466667,0.705882}%
\pgfsetstrokecolor{currentstroke}%
\pgfsetstrokeopacity{0.330660}%
\pgfsetdash{}{0pt}%
\pgfpathmoveto{\pgfqpoint{1.689366in}{3.224306in}}%
\pgfpathcurveto{\pgfqpoint{1.697602in}{3.224306in}}{\pgfqpoint{1.705502in}{3.227578in}}{\pgfqpoint{1.711326in}{3.233402in}}%
\pgfpathcurveto{\pgfqpoint{1.717150in}{3.239226in}}{\pgfqpoint{1.720422in}{3.247126in}}{\pgfqpoint{1.720422in}{3.255362in}}%
\pgfpathcurveto{\pgfqpoint{1.720422in}{3.263599in}}{\pgfqpoint{1.717150in}{3.271499in}}{\pgfqpoint{1.711326in}{3.277323in}}%
\pgfpathcurveto{\pgfqpoint{1.705502in}{3.283146in}}{\pgfqpoint{1.697602in}{3.286419in}}{\pgfqpoint{1.689366in}{3.286419in}}%
\pgfpathcurveto{\pgfqpoint{1.681130in}{3.286419in}}{\pgfqpoint{1.673229in}{3.283146in}}{\pgfqpoint{1.667406in}{3.277323in}}%
\pgfpathcurveto{\pgfqpoint{1.661582in}{3.271499in}}{\pgfqpoint{1.658309in}{3.263599in}}{\pgfqpoint{1.658309in}{3.255362in}}%
\pgfpathcurveto{\pgfqpoint{1.658309in}{3.247126in}}{\pgfqpoint{1.661582in}{3.239226in}}{\pgfqpoint{1.667406in}{3.233402in}}%
\pgfpathcurveto{\pgfqpoint{1.673229in}{3.227578in}}{\pgfqpoint{1.681130in}{3.224306in}}{\pgfqpoint{1.689366in}{3.224306in}}%
\pgfpathclose%
\pgfusepath{stroke,fill}%
\end{pgfscope}%
\begin{pgfscope}%
\pgfpathrectangle{\pgfqpoint{0.100000in}{0.212622in}}{\pgfqpoint{3.696000in}{3.696000in}}%
\pgfusepath{clip}%
\pgfsetbuttcap%
\pgfsetroundjoin%
\definecolor{currentfill}{rgb}{0.121569,0.466667,0.705882}%
\pgfsetfillcolor{currentfill}%
\pgfsetfillopacity{0.331353}%
\pgfsetlinewidth{1.003750pt}%
\definecolor{currentstroke}{rgb}{0.121569,0.466667,0.705882}%
\pgfsetstrokecolor{currentstroke}%
\pgfsetstrokeopacity{0.331353}%
\pgfsetdash{}{0pt}%
\pgfpathmoveto{\pgfqpoint{1.830947in}{3.265152in}}%
\pgfpathcurveto{\pgfqpoint{1.839183in}{3.265152in}}{\pgfqpoint{1.847083in}{3.268425in}}{\pgfqpoint{1.852907in}{3.274249in}}%
\pgfpathcurveto{\pgfqpoint{1.858731in}{3.280073in}}{\pgfqpoint{1.862003in}{3.287973in}}{\pgfqpoint{1.862003in}{3.296209in}}%
\pgfpathcurveto{\pgfqpoint{1.862003in}{3.304445in}}{\pgfqpoint{1.858731in}{3.312345in}}{\pgfqpoint{1.852907in}{3.318169in}}%
\pgfpathcurveto{\pgfqpoint{1.847083in}{3.323993in}}{\pgfqpoint{1.839183in}{3.327265in}}{\pgfqpoint{1.830947in}{3.327265in}}%
\pgfpathcurveto{\pgfqpoint{1.822710in}{3.327265in}}{\pgfqpoint{1.814810in}{3.323993in}}{\pgfqpoint{1.808986in}{3.318169in}}%
\pgfpathcurveto{\pgfqpoint{1.803163in}{3.312345in}}{\pgfqpoint{1.799890in}{3.304445in}}{\pgfqpoint{1.799890in}{3.296209in}}%
\pgfpathcurveto{\pgfqpoint{1.799890in}{3.287973in}}{\pgfqpoint{1.803163in}{3.280073in}}{\pgfqpoint{1.808986in}{3.274249in}}%
\pgfpathcurveto{\pgfqpoint{1.814810in}{3.268425in}}{\pgfqpoint{1.822710in}{3.265152in}}{\pgfqpoint{1.830947in}{3.265152in}}%
\pgfpathclose%
\pgfusepath{stroke,fill}%
\end{pgfscope}%
\begin{pgfscope}%
\pgfpathrectangle{\pgfqpoint{0.100000in}{0.212622in}}{\pgfqpoint{3.696000in}{3.696000in}}%
\pgfusepath{clip}%
\pgfsetbuttcap%
\pgfsetroundjoin%
\definecolor{currentfill}{rgb}{0.121569,0.466667,0.705882}%
\pgfsetfillcolor{currentfill}%
\pgfsetfillopacity{0.331634}%
\pgfsetlinewidth{1.003750pt}%
\definecolor{currentstroke}{rgb}{0.121569,0.466667,0.705882}%
\pgfsetstrokecolor{currentstroke}%
\pgfsetstrokeopacity{0.331634}%
\pgfsetdash{}{0pt}%
\pgfpathmoveto{\pgfqpoint{1.687360in}{3.219809in}}%
\pgfpathcurveto{\pgfqpoint{1.695597in}{3.219809in}}{\pgfqpoint{1.703497in}{3.223082in}}{\pgfqpoint{1.709321in}{3.228906in}}%
\pgfpathcurveto{\pgfqpoint{1.715145in}{3.234730in}}{\pgfqpoint{1.718417in}{3.242630in}}{\pgfqpoint{1.718417in}{3.250866in}}%
\pgfpathcurveto{\pgfqpoint{1.718417in}{3.259102in}}{\pgfqpoint{1.715145in}{3.267002in}}{\pgfqpoint{1.709321in}{3.272826in}}%
\pgfpathcurveto{\pgfqpoint{1.703497in}{3.278650in}}{\pgfqpoint{1.695597in}{3.281922in}}{\pgfqpoint{1.687360in}{3.281922in}}%
\pgfpathcurveto{\pgfqpoint{1.679124in}{3.281922in}}{\pgfqpoint{1.671224in}{3.278650in}}{\pgfqpoint{1.665400in}{3.272826in}}%
\pgfpathcurveto{\pgfqpoint{1.659576in}{3.267002in}}{\pgfqpoint{1.656304in}{3.259102in}}{\pgfqpoint{1.656304in}{3.250866in}}%
\pgfpathcurveto{\pgfqpoint{1.656304in}{3.242630in}}{\pgfqpoint{1.659576in}{3.234730in}}{\pgfqpoint{1.665400in}{3.228906in}}%
\pgfpathcurveto{\pgfqpoint{1.671224in}{3.223082in}}{\pgfqpoint{1.679124in}{3.219809in}}{\pgfqpoint{1.687360in}{3.219809in}}%
\pgfpathclose%
\pgfusepath{stroke,fill}%
\end{pgfscope}%
\begin{pgfscope}%
\pgfpathrectangle{\pgfqpoint{0.100000in}{0.212622in}}{\pgfqpoint{3.696000in}{3.696000in}}%
\pgfusepath{clip}%
\pgfsetbuttcap%
\pgfsetroundjoin%
\definecolor{currentfill}{rgb}{0.121569,0.466667,0.705882}%
\pgfsetfillcolor{currentfill}%
\pgfsetfillopacity{0.331934}%
\pgfsetlinewidth{1.003750pt}%
\definecolor{currentstroke}{rgb}{0.121569,0.466667,0.705882}%
\pgfsetstrokecolor{currentstroke}%
\pgfsetstrokeopacity{0.331934}%
\pgfsetdash{}{0pt}%
\pgfpathmoveto{\pgfqpoint{1.686728in}{3.218387in}}%
\pgfpathcurveto{\pgfqpoint{1.694965in}{3.218387in}}{\pgfqpoint{1.702865in}{3.221659in}}{\pgfqpoint{1.708689in}{3.227483in}}%
\pgfpathcurveto{\pgfqpoint{1.714513in}{3.233307in}}{\pgfqpoint{1.717785in}{3.241207in}}{\pgfqpoint{1.717785in}{3.249443in}}%
\pgfpathcurveto{\pgfqpoint{1.717785in}{3.257679in}}{\pgfqpoint{1.714513in}{3.265579in}}{\pgfqpoint{1.708689in}{3.271403in}}%
\pgfpathcurveto{\pgfqpoint{1.702865in}{3.277227in}}{\pgfqpoint{1.694965in}{3.280500in}}{\pgfqpoint{1.686728in}{3.280500in}}%
\pgfpathcurveto{\pgfqpoint{1.678492in}{3.280500in}}{\pgfqpoint{1.670592in}{3.277227in}}{\pgfqpoint{1.664768in}{3.271403in}}%
\pgfpathcurveto{\pgfqpoint{1.658944in}{3.265579in}}{\pgfqpoint{1.655672in}{3.257679in}}{\pgfqpoint{1.655672in}{3.249443in}}%
\pgfpathcurveto{\pgfqpoint{1.655672in}{3.241207in}}{\pgfqpoint{1.658944in}{3.233307in}}{\pgfqpoint{1.664768in}{3.227483in}}%
\pgfpathcurveto{\pgfqpoint{1.670592in}{3.221659in}}{\pgfqpoint{1.678492in}{3.218387in}}{\pgfqpoint{1.686728in}{3.218387in}}%
\pgfpathclose%
\pgfusepath{stroke,fill}%
\end{pgfscope}%
\begin{pgfscope}%
\pgfpathrectangle{\pgfqpoint{0.100000in}{0.212622in}}{\pgfqpoint{3.696000in}{3.696000in}}%
\pgfusepath{clip}%
\pgfsetbuttcap%
\pgfsetroundjoin%
\definecolor{currentfill}{rgb}{0.121569,0.466667,0.705882}%
\pgfsetfillcolor{currentfill}%
\pgfsetfillopacity{0.332497}%
\pgfsetlinewidth{1.003750pt}%
\definecolor{currentstroke}{rgb}{0.121569,0.466667,0.705882}%
\pgfsetstrokecolor{currentstroke}%
\pgfsetstrokeopacity{0.332497}%
\pgfsetdash{}{0pt}%
\pgfpathmoveto{\pgfqpoint{1.685583in}{3.215868in}}%
\pgfpathcurveto{\pgfqpoint{1.693820in}{3.215868in}}{\pgfqpoint{1.701720in}{3.219141in}}{\pgfqpoint{1.707544in}{3.224965in}}%
\pgfpathcurveto{\pgfqpoint{1.713368in}{3.230789in}}{\pgfqpoint{1.716640in}{3.238689in}}{\pgfqpoint{1.716640in}{3.246925in}}%
\pgfpathcurveto{\pgfqpoint{1.716640in}{3.255161in}}{\pgfqpoint{1.713368in}{3.263061in}}{\pgfqpoint{1.707544in}{3.268885in}}%
\pgfpathcurveto{\pgfqpoint{1.701720in}{3.274709in}}{\pgfqpoint{1.693820in}{3.277981in}}{\pgfqpoint{1.685583in}{3.277981in}}%
\pgfpathcurveto{\pgfqpoint{1.677347in}{3.277981in}}{\pgfqpoint{1.669447in}{3.274709in}}{\pgfqpoint{1.663623in}{3.268885in}}%
\pgfpathcurveto{\pgfqpoint{1.657799in}{3.263061in}}{\pgfqpoint{1.654527in}{3.255161in}}{\pgfqpoint{1.654527in}{3.246925in}}%
\pgfpathcurveto{\pgfqpoint{1.654527in}{3.238689in}}{\pgfqpoint{1.657799in}{3.230789in}}{\pgfqpoint{1.663623in}{3.224965in}}%
\pgfpathcurveto{\pgfqpoint{1.669447in}{3.219141in}}{\pgfqpoint{1.677347in}{3.215868in}}{\pgfqpoint{1.685583in}{3.215868in}}%
\pgfpathclose%
\pgfusepath{stroke,fill}%
\end{pgfscope}%
\begin{pgfscope}%
\pgfpathrectangle{\pgfqpoint{0.100000in}{0.212622in}}{\pgfqpoint{3.696000in}{3.696000in}}%
\pgfusepath{clip}%
\pgfsetbuttcap%
\pgfsetroundjoin%
\definecolor{currentfill}{rgb}{0.121569,0.466667,0.705882}%
\pgfsetfillcolor{currentfill}%
\pgfsetfillopacity{0.332571}%
\pgfsetlinewidth{1.003750pt}%
\definecolor{currentstroke}{rgb}{0.121569,0.466667,0.705882}%
\pgfsetstrokecolor{currentstroke}%
\pgfsetstrokeopacity{0.332571}%
\pgfsetdash{}{0pt}%
\pgfpathmoveto{\pgfqpoint{1.832187in}{3.260920in}}%
\pgfpathcurveto{\pgfqpoint{1.840424in}{3.260920in}}{\pgfqpoint{1.848324in}{3.264192in}}{\pgfqpoint{1.854148in}{3.270016in}}%
\pgfpathcurveto{\pgfqpoint{1.859972in}{3.275840in}}{\pgfqpoint{1.863244in}{3.283740in}}{\pgfqpoint{1.863244in}{3.291976in}}%
\pgfpathcurveto{\pgfqpoint{1.863244in}{3.300213in}}{\pgfqpoint{1.859972in}{3.308113in}}{\pgfqpoint{1.854148in}{3.313937in}}%
\pgfpathcurveto{\pgfqpoint{1.848324in}{3.319761in}}{\pgfqpoint{1.840424in}{3.323033in}}{\pgfqpoint{1.832187in}{3.323033in}}%
\pgfpathcurveto{\pgfqpoint{1.823951in}{3.323033in}}{\pgfqpoint{1.816051in}{3.319761in}}{\pgfqpoint{1.810227in}{3.313937in}}%
\pgfpathcurveto{\pgfqpoint{1.804403in}{3.308113in}}{\pgfqpoint{1.801131in}{3.300213in}}{\pgfqpoint{1.801131in}{3.291976in}}%
\pgfpathcurveto{\pgfqpoint{1.801131in}{3.283740in}}{\pgfqpoint{1.804403in}{3.275840in}}{\pgfqpoint{1.810227in}{3.270016in}}%
\pgfpathcurveto{\pgfqpoint{1.816051in}{3.264192in}}{\pgfqpoint{1.823951in}{3.260920in}}{\pgfqpoint{1.832187in}{3.260920in}}%
\pgfpathclose%
\pgfusepath{stroke,fill}%
\end{pgfscope}%
\begin{pgfscope}%
\pgfpathrectangle{\pgfqpoint{0.100000in}{0.212622in}}{\pgfqpoint{3.696000in}{3.696000in}}%
\pgfusepath{clip}%
\pgfsetbuttcap%
\pgfsetroundjoin%
\definecolor{currentfill}{rgb}{0.121569,0.466667,0.705882}%
\pgfsetfillcolor{currentfill}%
\pgfsetfillopacity{0.332654}%
\pgfsetlinewidth{1.003750pt}%
\definecolor{currentstroke}{rgb}{0.121569,0.466667,0.705882}%
\pgfsetstrokecolor{currentstroke}%
\pgfsetstrokeopacity{0.332654}%
\pgfsetdash{}{0pt}%
\pgfpathmoveto{\pgfqpoint{1.685264in}{3.215148in}}%
\pgfpathcurveto{\pgfqpoint{1.693500in}{3.215148in}}{\pgfqpoint{1.701400in}{3.218420in}}{\pgfqpoint{1.707224in}{3.224244in}}%
\pgfpathcurveto{\pgfqpoint{1.713048in}{3.230068in}}{\pgfqpoint{1.716320in}{3.237968in}}{\pgfqpoint{1.716320in}{3.246204in}}%
\pgfpathcurveto{\pgfqpoint{1.716320in}{3.254441in}}{\pgfqpoint{1.713048in}{3.262341in}}{\pgfqpoint{1.707224in}{3.268165in}}%
\pgfpathcurveto{\pgfqpoint{1.701400in}{3.273989in}}{\pgfqpoint{1.693500in}{3.277261in}}{\pgfqpoint{1.685264in}{3.277261in}}%
\pgfpathcurveto{\pgfqpoint{1.677028in}{3.277261in}}{\pgfqpoint{1.669128in}{3.273989in}}{\pgfqpoint{1.663304in}{3.268165in}}%
\pgfpathcurveto{\pgfqpoint{1.657480in}{3.262341in}}{\pgfqpoint{1.654207in}{3.254441in}}{\pgfqpoint{1.654207in}{3.246204in}}%
\pgfpathcurveto{\pgfqpoint{1.654207in}{3.237968in}}{\pgfqpoint{1.657480in}{3.230068in}}{\pgfqpoint{1.663304in}{3.224244in}}%
\pgfpathcurveto{\pgfqpoint{1.669128in}{3.218420in}}{\pgfqpoint{1.677028in}{3.215148in}}{\pgfqpoint{1.685264in}{3.215148in}}%
\pgfpathclose%
\pgfusepath{stroke,fill}%
\end{pgfscope}%
\begin{pgfscope}%
\pgfpathrectangle{\pgfqpoint{0.100000in}{0.212622in}}{\pgfqpoint{3.696000in}{3.696000in}}%
\pgfusepath{clip}%
\pgfsetbuttcap%
\pgfsetroundjoin%
\definecolor{currentfill}{rgb}{0.121569,0.466667,0.705882}%
\pgfsetfillcolor{currentfill}%
\pgfsetfillopacity{0.332947}%
\pgfsetlinewidth{1.003750pt}%
\definecolor{currentstroke}{rgb}{0.121569,0.466667,0.705882}%
\pgfsetstrokecolor{currentstroke}%
\pgfsetstrokeopacity{0.332947}%
\pgfsetdash{}{0pt}%
\pgfpathmoveto{\pgfqpoint{1.684695in}{3.213857in}}%
\pgfpathcurveto{\pgfqpoint{1.692931in}{3.213857in}}{\pgfqpoint{1.700831in}{3.217130in}}{\pgfqpoint{1.706655in}{3.222954in}}%
\pgfpathcurveto{\pgfqpoint{1.712479in}{3.228778in}}{\pgfqpoint{1.715751in}{3.236678in}}{\pgfqpoint{1.715751in}{3.244914in}}%
\pgfpathcurveto{\pgfqpoint{1.715751in}{3.253150in}}{\pgfqpoint{1.712479in}{3.261050in}}{\pgfqpoint{1.706655in}{3.266874in}}%
\pgfpathcurveto{\pgfqpoint{1.700831in}{3.272698in}}{\pgfqpoint{1.692931in}{3.275970in}}{\pgfqpoint{1.684695in}{3.275970in}}%
\pgfpathcurveto{\pgfqpoint{1.676458in}{3.275970in}}{\pgfqpoint{1.668558in}{3.272698in}}{\pgfqpoint{1.662734in}{3.266874in}}%
\pgfpathcurveto{\pgfqpoint{1.656910in}{3.261050in}}{\pgfqpoint{1.653638in}{3.253150in}}{\pgfqpoint{1.653638in}{3.244914in}}%
\pgfpathcurveto{\pgfqpoint{1.653638in}{3.236678in}}{\pgfqpoint{1.656910in}{3.228778in}}{\pgfqpoint{1.662734in}{3.222954in}}%
\pgfpathcurveto{\pgfqpoint{1.668558in}{3.217130in}}{\pgfqpoint{1.676458in}{3.213857in}}{\pgfqpoint{1.684695in}{3.213857in}}%
\pgfpathclose%
\pgfusepath{stroke,fill}%
\end{pgfscope}%
\begin{pgfscope}%
\pgfpathrectangle{\pgfqpoint{0.100000in}{0.212622in}}{\pgfqpoint{3.696000in}{3.696000in}}%
\pgfusepath{clip}%
\pgfsetbuttcap%
\pgfsetroundjoin%
\definecolor{currentfill}{rgb}{0.121569,0.466667,0.705882}%
\pgfsetfillcolor{currentfill}%
\pgfsetfillopacity{0.333473}%
\pgfsetlinewidth{1.003750pt}%
\definecolor{currentstroke}{rgb}{0.121569,0.466667,0.705882}%
\pgfsetstrokecolor{currentstroke}%
\pgfsetstrokeopacity{0.333473}%
\pgfsetdash{}{0pt}%
\pgfpathmoveto{\pgfqpoint{1.683637in}{3.211508in}}%
\pgfpathcurveto{\pgfqpoint{1.691874in}{3.211508in}}{\pgfqpoint{1.699774in}{3.214780in}}{\pgfqpoint{1.705598in}{3.220604in}}%
\pgfpathcurveto{\pgfqpoint{1.711421in}{3.226428in}}{\pgfqpoint{1.714694in}{3.234328in}}{\pgfqpoint{1.714694in}{3.242565in}}%
\pgfpathcurveto{\pgfqpoint{1.714694in}{3.250801in}}{\pgfqpoint{1.711421in}{3.258701in}}{\pgfqpoint{1.705598in}{3.264525in}}%
\pgfpathcurveto{\pgfqpoint{1.699774in}{3.270349in}}{\pgfqpoint{1.691874in}{3.273621in}}{\pgfqpoint{1.683637in}{3.273621in}}%
\pgfpathcurveto{\pgfqpoint{1.675401in}{3.273621in}}{\pgfqpoint{1.667501in}{3.270349in}}{\pgfqpoint{1.661677in}{3.264525in}}%
\pgfpathcurveto{\pgfqpoint{1.655853in}{3.258701in}}{\pgfqpoint{1.652581in}{3.250801in}}{\pgfqpoint{1.652581in}{3.242565in}}%
\pgfpathcurveto{\pgfqpoint{1.652581in}{3.234328in}}{\pgfqpoint{1.655853in}{3.226428in}}{\pgfqpoint{1.661677in}{3.220604in}}%
\pgfpathcurveto{\pgfqpoint{1.667501in}{3.214780in}}{\pgfqpoint{1.675401in}{3.211508in}}{\pgfqpoint{1.683637in}{3.211508in}}%
\pgfpathclose%
\pgfusepath{stroke,fill}%
\end{pgfscope}%
\begin{pgfscope}%
\pgfpathrectangle{\pgfqpoint{0.100000in}{0.212622in}}{\pgfqpoint{3.696000in}{3.696000in}}%
\pgfusepath{clip}%
\pgfsetbuttcap%
\pgfsetroundjoin%
\definecolor{currentfill}{rgb}{0.121569,0.466667,0.705882}%
\pgfsetfillcolor{currentfill}%
\pgfsetfillopacity{0.333598}%
\pgfsetlinewidth{1.003750pt}%
\definecolor{currentstroke}{rgb}{0.121569,0.466667,0.705882}%
\pgfsetstrokecolor{currentstroke}%
\pgfsetstrokeopacity{0.333598}%
\pgfsetdash{}{0pt}%
\pgfpathmoveto{\pgfqpoint{1.683390in}{3.210959in}}%
\pgfpathcurveto{\pgfqpoint{1.691626in}{3.210959in}}{\pgfqpoint{1.699526in}{3.214232in}}{\pgfqpoint{1.705350in}{3.220056in}}%
\pgfpathcurveto{\pgfqpoint{1.711174in}{3.225879in}}{\pgfqpoint{1.714446in}{3.233780in}}{\pgfqpoint{1.714446in}{3.242016in}}%
\pgfpathcurveto{\pgfqpoint{1.714446in}{3.250252in}}{\pgfqpoint{1.711174in}{3.258152in}}{\pgfqpoint{1.705350in}{3.263976in}}%
\pgfpathcurveto{\pgfqpoint{1.699526in}{3.269800in}}{\pgfqpoint{1.691626in}{3.273072in}}{\pgfqpoint{1.683390in}{3.273072in}}%
\pgfpathcurveto{\pgfqpoint{1.675154in}{3.273072in}}{\pgfqpoint{1.667253in}{3.269800in}}{\pgfqpoint{1.661430in}{3.263976in}}%
\pgfpathcurveto{\pgfqpoint{1.655606in}{3.258152in}}{\pgfqpoint{1.652333in}{3.250252in}}{\pgfqpoint{1.652333in}{3.242016in}}%
\pgfpathcurveto{\pgfqpoint{1.652333in}{3.233780in}}{\pgfqpoint{1.655606in}{3.225879in}}{\pgfqpoint{1.661430in}{3.220056in}}%
\pgfpathcurveto{\pgfqpoint{1.667253in}{3.214232in}}{\pgfqpoint{1.675154in}{3.210959in}}{\pgfqpoint{1.683390in}{3.210959in}}%
\pgfpathclose%
\pgfusepath{stroke,fill}%
\end{pgfscope}%
\begin{pgfscope}%
\pgfpathrectangle{\pgfqpoint{0.100000in}{0.212622in}}{\pgfqpoint{3.696000in}{3.696000in}}%
\pgfusepath{clip}%
\pgfsetbuttcap%
\pgfsetroundjoin%
\definecolor{currentfill}{rgb}{0.121569,0.466667,0.705882}%
\pgfsetfillcolor{currentfill}%
\pgfsetfillopacity{0.333821}%
\pgfsetlinewidth{1.003750pt}%
\definecolor{currentstroke}{rgb}{0.121569,0.466667,0.705882}%
\pgfsetstrokecolor{currentstroke}%
\pgfsetstrokeopacity{0.333821}%
\pgfsetdash{}{0pt}%
\pgfpathmoveto{\pgfqpoint{1.682951in}{3.209934in}}%
\pgfpathcurveto{\pgfqpoint{1.691187in}{3.209934in}}{\pgfqpoint{1.699087in}{3.213207in}}{\pgfqpoint{1.704911in}{3.219031in}}%
\pgfpathcurveto{\pgfqpoint{1.710735in}{3.224854in}}{\pgfqpoint{1.714007in}{3.232755in}}{\pgfqpoint{1.714007in}{3.240991in}}%
\pgfpathcurveto{\pgfqpoint{1.714007in}{3.249227in}}{\pgfqpoint{1.710735in}{3.257127in}}{\pgfqpoint{1.704911in}{3.262951in}}%
\pgfpathcurveto{\pgfqpoint{1.699087in}{3.268775in}}{\pgfqpoint{1.691187in}{3.272047in}}{\pgfqpoint{1.682951in}{3.272047in}}%
\pgfpathcurveto{\pgfqpoint{1.674715in}{3.272047in}}{\pgfqpoint{1.666815in}{3.268775in}}{\pgfqpoint{1.660991in}{3.262951in}}%
\pgfpathcurveto{\pgfqpoint{1.655167in}{3.257127in}}{\pgfqpoint{1.651894in}{3.249227in}}{\pgfqpoint{1.651894in}{3.240991in}}%
\pgfpathcurveto{\pgfqpoint{1.651894in}{3.232755in}}{\pgfqpoint{1.655167in}{3.224854in}}{\pgfqpoint{1.660991in}{3.219031in}}%
\pgfpathcurveto{\pgfqpoint{1.666815in}{3.213207in}}{\pgfqpoint{1.674715in}{3.209934in}}{\pgfqpoint{1.682951in}{3.209934in}}%
\pgfpathclose%
\pgfusepath{stroke,fill}%
\end{pgfscope}%
\begin{pgfscope}%
\pgfpathrectangle{\pgfqpoint{0.100000in}{0.212622in}}{\pgfqpoint{3.696000in}{3.696000in}}%
\pgfusepath{clip}%
\pgfsetbuttcap%
\pgfsetroundjoin%
\definecolor{currentfill}{rgb}{0.121569,0.466667,0.705882}%
\pgfsetfillcolor{currentfill}%
\pgfsetfillopacity{0.334229}%
\pgfsetlinewidth{1.003750pt}%
\definecolor{currentstroke}{rgb}{0.121569,0.466667,0.705882}%
\pgfsetstrokecolor{currentstroke}%
\pgfsetstrokeopacity{0.334229}%
\pgfsetdash{}{0pt}%
\pgfpathmoveto{\pgfqpoint{1.682176in}{3.208061in}}%
\pgfpathcurveto{\pgfqpoint{1.690412in}{3.208061in}}{\pgfqpoint{1.698312in}{3.211333in}}{\pgfqpoint{1.704136in}{3.217157in}}%
\pgfpathcurveto{\pgfqpoint{1.709960in}{3.222981in}}{\pgfqpoint{1.713232in}{3.230881in}}{\pgfqpoint{1.713232in}{3.239117in}}%
\pgfpathcurveto{\pgfqpoint{1.713232in}{3.247354in}}{\pgfqpoint{1.709960in}{3.255254in}}{\pgfqpoint{1.704136in}{3.261078in}}%
\pgfpathcurveto{\pgfqpoint{1.698312in}{3.266902in}}{\pgfqpoint{1.690412in}{3.270174in}}{\pgfqpoint{1.682176in}{3.270174in}}%
\pgfpathcurveto{\pgfqpoint{1.673939in}{3.270174in}}{\pgfqpoint{1.666039in}{3.266902in}}{\pgfqpoint{1.660215in}{3.261078in}}%
\pgfpathcurveto{\pgfqpoint{1.654391in}{3.255254in}}{\pgfqpoint{1.651119in}{3.247354in}}{\pgfqpoint{1.651119in}{3.239117in}}%
\pgfpathcurveto{\pgfqpoint{1.651119in}{3.230881in}}{\pgfqpoint{1.654391in}{3.222981in}}{\pgfqpoint{1.660215in}{3.217157in}}%
\pgfpathcurveto{\pgfqpoint{1.666039in}{3.211333in}}{\pgfqpoint{1.673939in}{3.208061in}}{\pgfqpoint{1.682176in}{3.208061in}}%
\pgfpathclose%
\pgfusepath{stroke,fill}%
\end{pgfscope}%
\begin{pgfscope}%
\pgfpathrectangle{\pgfqpoint{0.100000in}{0.212622in}}{\pgfqpoint{3.696000in}{3.696000in}}%
\pgfusepath{clip}%
\pgfsetbuttcap%
\pgfsetroundjoin%
\definecolor{currentfill}{rgb}{0.121569,0.466667,0.705882}%
\pgfsetfillcolor{currentfill}%
\pgfsetfillopacity{0.334426}%
\pgfsetlinewidth{1.003750pt}%
\definecolor{currentstroke}{rgb}{0.121569,0.466667,0.705882}%
\pgfsetstrokecolor{currentstroke}%
\pgfsetstrokeopacity{0.334426}%
\pgfsetdash{}{0pt}%
\pgfpathmoveto{\pgfqpoint{1.833989in}{3.253858in}}%
\pgfpathcurveto{\pgfqpoint{1.842225in}{3.253858in}}{\pgfqpoint{1.850125in}{3.257130in}}{\pgfqpoint{1.855949in}{3.262954in}}%
\pgfpathcurveto{\pgfqpoint{1.861773in}{3.268778in}}{\pgfqpoint{1.865045in}{3.276678in}}{\pgfqpoint{1.865045in}{3.284915in}}%
\pgfpathcurveto{\pgfqpoint{1.865045in}{3.293151in}}{\pgfqpoint{1.861773in}{3.301051in}}{\pgfqpoint{1.855949in}{3.306875in}}%
\pgfpathcurveto{\pgfqpoint{1.850125in}{3.312699in}}{\pgfqpoint{1.842225in}{3.315971in}}{\pgfqpoint{1.833989in}{3.315971in}}%
\pgfpathcurveto{\pgfqpoint{1.825752in}{3.315971in}}{\pgfqpoint{1.817852in}{3.312699in}}{\pgfqpoint{1.812028in}{3.306875in}}%
\pgfpathcurveto{\pgfqpoint{1.806204in}{3.301051in}}{\pgfqpoint{1.802932in}{3.293151in}}{\pgfqpoint{1.802932in}{3.284915in}}%
\pgfpathcurveto{\pgfqpoint{1.802932in}{3.276678in}}{\pgfqpoint{1.806204in}{3.268778in}}{\pgfqpoint{1.812028in}{3.262954in}}%
\pgfpathcurveto{\pgfqpoint{1.817852in}{3.257130in}}{\pgfqpoint{1.825752in}{3.253858in}}{\pgfqpoint{1.833989in}{3.253858in}}%
\pgfpathclose%
\pgfusepath{stroke,fill}%
\end{pgfscope}%
\begin{pgfscope}%
\pgfpathrectangle{\pgfqpoint{0.100000in}{0.212622in}}{\pgfqpoint{3.696000in}{3.696000in}}%
\pgfusepath{clip}%
\pgfsetbuttcap%
\pgfsetroundjoin%
\definecolor{currentfill}{rgb}{0.121569,0.466667,0.705882}%
\pgfsetfillcolor{currentfill}%
\pgfsetfillopacity{0.334992}%
\pgfsetlinewidth{1.003750pt}%
\definecolor{currentstroke}{rgb}{0.121569,0.466667,0.705882}%
\pgfsetstrokecolor{currentstroke}%
\pgfsetstrokeopacity{0.334992}%
\pgfsetdash{}{0pt}%
\pgfpathmoveto{\pgfqpoint{1.680776in}{3.204728in}}%
\pgfpathcurveto{\pgfqpoint{1.689012in}{3.204728in}}{\pgfqpoint{1.696912in}{3.208000in}}{\pgfqpoint{1.702736in}{3.213824in}}%
\pgfpathcurveto{\pgfqpoint{1.708560in}{3.219648in}}{\pgfqpoint{1.711833in}{3.227548in}}{\pgfqpoint{1.711833in}{3.235785in}}%
\pgfpathcurveto{\pgfqpoint{1.711833in}{3.244021in}}{\pgfqpoint{1.708560in}{3.251921in}}{\pgfqpoint{1.702736in}{3.257745in}}%
\pgfpathcurveto{\pgfqpoint{1.696912in}{3.263569in}}{\pgfqpoint{1.689012in}{3.266841in}}{\pgfqpoint{1.680776in}{3.266841in}}%
\pgfpathcurveto{\pgfqpoint{1.672540in}{3.266841in}}{\pgfqpoint{1.664640in}{3.263569in}}{\pgfqpoint{1.658816in}{3.257745in}}%
\pgfpathcurveto{\pgfqpoint{1.652992in}{3.251921in}}{\pgfqpoint{1.649720in}{3.244021in}}{\pgfqpoint{1.649720in}{3.235785in}}%
\pgfpathcurveto{\pgfqpoint{1.649720in}{3.227548in}}{\pgfqpoint{1.652992in}{3.219648in}}{\pgfqpoint{1.658816in}{3.213824in}}%
\pgfpathcurveto{\pgfqpoint{1.664640in}{3.208000in}}{\pgfqpoint{1.672540in}{3.204728in}}{\pgfqpoint{1.680776in}{3.204728in}}%
\pgfpathclose%
\pgfusepath{stroke,fill}%
\end{pgfscope}%
\begin{pgfscope}%
\pgfpathrectangle{\pgfqpoint{0.100000in}{0.212622in}}{\pgfqpoint{3.696000in}{3.696000in}}%
\pgfusepath{clip}%
\pgfsetbuttcap%
\pgfsetroundjoin%
\definecolor{currentfill}{rgb}{0.121569,0.466667,0.705882}%
\pgfsetfillcolor{currentfill}%
\pgfsetfillopacity{0.335502}%
\pgfsetlinewidth{1.003750pt}%
\definecolor{currentstroke}{rgb}{0.121569,0.466667,0.705882}%
\pgfsetstrokecolor{currentstroke}%
\pgfsetstrokeopacity{0.335502}%
\pgfsetdash{}{0pt}%
\pgfpathmoveto{\pgfqpoint{1.835278in}{3.250188in}}%
\pgfpathcurveto{\pgfqpoint{1.843514in}{3.250188in}}{\pgfqpoint{1.851414in}{3.253461in}}{\pgfqpoint{1.857238in}{3.259285in}}%
\pgfpathcurveto{\pgfqpoint{1.863062in}{3.265109in}}{\pgfqpoint{1.866334in}{3.273009in}}{\pgfqpoint{1.866334in}{3.281245in}}%
\pgfpathcurveto{\pgfqpoint{1.866334in}{3.289481in}}{\pgfqpoint{1.863062in}{3.297381in}}{\pgfqpoint{1.857238in}{3.303205in}}%
\pgfpathcurveto{\pgfqpoint{1.851414in}{3.309029in}}{\pgfqpoint{1.843514in}{3.312301in}}{\pgfqpoint{1.835278in}{3.312301in}}%
\pgfpathcurveto{\pgfqpoint{1.827042in}{3.312301in}}{\pgfqpoint{1.819142in}{3.309029in}}{\pgfqpoint{1.813318in}{3.303205in}}%
\pgfpathcurveto{\pgfqpoint{1.807494in}{3.297381in}}{\pgfqpoint{1.804221in}{3.289481in}}{\pgfqpoint{1.804221in}{3.281245in}}%
\pgfpathcurveto{\pgfqpoint{1.804221in}{3.273009in}}{\pgfqpoint{1.807494in}{3.265109in}}{\pgfqpoint{1.813318in}{3.259285in}}%
\pgfpathcurveto{\pgfqpoint{1.819142in}{3.253461in}}{\pgfqpoint{1.827042in}{3.250188in}}{\pgfqpoint{1.835278in}{3.250188in}}%
\pgfpathclose%
\pgfusepath{stroke,fill}%
\end{pgfscope}%
\begin{pgfscope}%
\pgfpathrectangle{\pgfqpoint{0.100000in}{0.212622in}}{\pgfqpoint{3.696000in}{3.696000in}}%
\pgfusepath{clip}%
\pgfsetbuttcap%
\pgfsetroundjoin%
\definecolor{currentfill}{rgb}{0.121569,0.466667,0.705882}%
\pgfsetfillcolor{currentfill}%
\pgfsetfillopacity{0.336067}%
\pgfsetlinewidth{1.003750pt}%
\definecolor{currentstroke}{rgb}{0.121569,0.466667,0.705882}%
\pgfsetstrokecolor{currentstroke}%
\pgfsetstrokeopacity{0.336067}%
\pgfsetdash{}{0pt}%
\pgfpathmoveto{\pgfqpoint{1.835832in}{3.248065in}}%
\pgfpathcurveto{\pgfqpoint{1.844068in}{3.248065in}}{\pgfqpoint{1.851968in}{3.251337in}}{\pgfqpoint{1.857792in}{3.257161in}}%
\pgfpathcurveto{\pgfqpoint{1.863616in}{3.262985in}}{\pgfqpoint{1.866888in}{3.270885in}}{\pgfqpoint{1.866888in}{3.279121in}}%
\pgfpathcurveto{\pgfqpoint{1.866888in}{3.287358in}}{\pgfqpoint{1.863616in}{3.295258in}}{\pgfqpoint{1.857792in}{3.301082in}}%
\pgfpathcurveto{\pgfqpoint{1.851968in}{3.306906in}}{\pgfqpoint{1.844068in}{3.310178in}}{\pgfqpoint{1.835832in}{3.310178in}}%
\pgfpathcurveto{\pgfqpoint{1.827596in}{3.310178in}}{\pgfqpoint{1.819696in}{3.306906in}}{\pgfqpoint{1.813872in}{3.301082in}}%
\pgfpathcurveto{\pgfqpoint{1.808048in}{3.295258in}}{\pgfqpoint{1.804775in}{3.287358in}}{\pgfqpoint{1.804775in}{3.279121in}}%
\pgfpathcurveto{\pgfqpoint{1.804775in}{3.270885in}}{\pgfqpoint{1.808048in}{3.262985in}}{\pgfqpoint{1.813872in}{3.257161in}}%
\pgfpathcurveto{\pgfqpoint{1.819696in}{3.251337in}}{\pgfqpoint{1.827596in}{3.248065in}}{\pgfqpoint{1.835832in}{3.248065in}}%
\pgfpathclose%
\pgfusepath{stroke,fill}%
\end{pgfscope}%
\begin{pgfscope}%
\pgfpathrectangle{\pgfqpoint{0.100000in}{0.212622in}}{\pgfqpoint{3.696000in}{3.696000in}}%
\pgfusepath{clip}%
\pgfsetbuttcap%
\pgfsetroundjoin%
\definecolor{currentfill}{rgb}{0.121569,0.466667,0.705882}%
\pgfsetfillcolor{currentfill}%
\pgfsetfillopacity{0.336374}%
\pgfsetlinewidth{1.003750pt}%
\definecolor{currentstroke}{rgb}{0.121569,0.466667,0.705882}%
\pgfsetstrokecolor{currentstroke}%
\pgfsetstrokeopacity{0.336374}%
\pgfsetdash{}{0pt}%
\pgfpathmoveto{\pgfqpoint{1.678316in}{3.198560in}}%
\pgfpathcurveto{\pgfqpoint{1.686552in}{3.198560in}}{\pgfqpoint{1.694452in}{3.201832in}}{\pgfqpoint{1.700276in}{3.207656in}}%
\pgfpathcurveto{\pgfqpoint{1.706100in}{3.213480in}}{\pgfqpoint{1.709372in}{3.221380in}}{\pgfqpoint{1.709372in}{3.229616in}}%
\pgfpathcurveto{\pgfqpoint{1.709372in}{3.237852in}}{\pgfqpoint{1.706100in}{3.245752in}}{\pgfqpoint{1.700276in}{3.251576in}}%
\pgfpathcurveto{\pgfqpoint{1.694452in}{3.257400in}}{\pgfqpoint{1.686552in}{3.260673in}}{\pgfqpoint{1.678316in}{3.260673in}}%
\pgfpathcurveto{\pgfqpoint{1.670080in}{3.260673in}}{\pgfqpoint{1.662180in}{3.257400in}}{\pgfqpoint{1.656356in}{3.251576in}}%
\pgfpathcurveto{\pgfqpoint{1.650532in}{3.245752in}}{\pgfqpoint{1.647259in}{3.237852in}}{\pgfqpoint{1.647259in}{3.229616in}}%
\pgfpathcurveto{\pgfqpoint{1.647259in}{3.221380in}}{\pgfqpoint{1.650532in}{3.213480in}}{\pgfqpoint{1.656356in}{3.207656in}}%
\pgfpathcurveto{\pgfqpoint{1.662180in}{3.201832in}}{\pgfqpoint{1.670080in}{3.198560in}}{\pgfqpoint{1.678316in}{3.198560in}}%
\pgfpathclose%
\pgfusepath{stroke,fill}%
\end{pgfscope}%
\begin{pgfscope}%
\pgfpathrectangle{\pgfqpoint{0.100000in}{0.212622in}}{\pgfqpoint{3.696000in}{3.696000in}}%
\pgfusepath{clip}%
\pgfsetbuttcap%
\pgfsetroundjoin%
\definecolor{currentfill}{rgb}{0.121569,0.466667,0.705882}%
\pgfsetfillcolor{currentfill}%
\pgfsetfillopacity{0.337177}%
\pgfsetlinewidth{1.003750pt}%
\definecolor{currentstroke}{rgb}{0.121569,0.466667,0.705882}%
\pgfsetstrokecolor{currentstroke}%
\pgfsetstrokeopacity{0.337177}%
\pgfsetdash{}{0pt}%
\pgfpathmoveto{\pgfqpoint{1.676708in}{3.194709in}}%
\pgfpathcurveto{\pgfqpoint{1.684944in}{3.194709in}}{\pgfqpoint{1.692844in}{3.197981in}}{\pgfqpoint{1.698668in}{3.203805in}}%
\pgfpathcurveto{\pgfqpoint{1.704492in}{3.209629in}}{\pgfqpoint{1.707765in}{3.217529in}}{\pgfqpoint{1.707765in}{3.225766in}}%
\pgfpathcurveto{\pgfqpoint{1.707765in}{3.234002in}}{\pgfqpoint{1.704492in}{3.241902in}}{\pgfqpoint{1.698668in}{3.247726in}}%
\pgfpathcurveto{\pgfqpoint{1.692844in}{3.253550in}}{\pgfqpoint{1.684944in}{3.256822in}}{\pgfqpoint{1.676708in}{3.256822in}}%
\pgfpathcurveto{\pgfqpoint{1.668472in}{3.256822in}}{\pgfqpoint{1.660572in}{3.253550in}}{\pgfqpoint{1.654748in}{3.247726in}}%
\pgfpathcurveto{\pgfqpoint{1.648924in}{3.241902in}}{\pgfqpoint{1.645652in}{3.234002in}}{\pgfqpoint{1.645652in}{3.225766in}}%
\pgfpathcurveto{\pgfqpoint{1.645652in}{3.217529in}}{\pgfqpoint{1.648924in}{3.209629in}}{\pgfqpoint{1.654748in}{3.203805in}}%
\pgfpathcurveto{\pgfqpoint{1.660572in}{3.197981in}}{\pgfqpoint{1.668472in}{3.194709in}}{\pgfqpoint{1.676708in}{3.194709in}}%
\pgfpathclose%
\pgfusepath{stroke,fill}%
\end{pgfscope}%
\begin{pgfscope}%
\pgfpathrectangle{\pgfqpoint{0.100000in}{0.212622in}}{\pgfqpoint{3.696000in}{3.696000in}}%
\pgfusepath{clip}%
\pgfsetbuttcap%
\pgfsetroundjoin%
\definecolor{currentfill}{rgb}{0.121569,0.466667,0.705882}%
\pgfsetfillcolor{currentfill}%
\pgfsetfillopacity{0.337288}%
\pgfsetlinewidth{1.003750pt}%
\definecolor{currentstroke}{rgb}{0.121569,0.466667,0.705882}%
\pgfsetstrokecolor{currentstroke}%
\pgfsetstrokeopacity{0.337288}%
\pgfsetdash{}{0pt}%
\pgfpathmoveto{\pgfqpoint{1.837255in}{3.243932in}}%
\pgfpathcurveto{\pgfqpoint{1.845491in}{3.243932in}}{\pgfqpoint{1.853391in}{3.247204in}}{\pgfqpoint{1.859215in}{3.253028in}}%
\pgfpathcurveto{\pgfqpoint{1.865039in}{3.258852in}}{\pgfqpoint{1.868311in}{3.266752in}}{\pgfqpoint{1.868311in}{3.274988in}}%
\pgfpathcurveto{\pgfqpoint{1.868311in}{3.283225in}}{\pgfqpoint{1.865039in}{3.291125in}}{\pgfqpoint{1.859215in}{3.296949in}}%
\pgfpathcurveto{\pgfqpoint{1.853391in}{3.302773in}}{\pgfqpoint{1.845491in}{3.306045in}}{\pgfqpoint{1.837255in}{3.306045in}}%
\pgfpathcurveto{\pgfqpoint{1.829019in}{3.306045in}}{\pgfqpoint{1.821118in}{3.302773in}}{\pgfqpoint{1.815295in}{3.296949in}}%
\pgfpathcurveto{\pgfqpoint{1.809471in}{3.291125in}}{\pgfqpoint{1.806198in}{3.283225in}}{\pgfqpoint{1.806198in}{3.274988in}}%
\pgfpathcurveto{\pgfqpoint{1.806198in}{3.266752in}}{\pgfqpoint{1.809471in}{3.258852in}}{\pgfqpoint{1.815295in}{3.253028in}}%
\pgfpathcurveto{\pgfqpoint{1.821118in}{3.247204in}}{\pgfqpoint{1.829019in}{3.243932in}}{\pgfqpoint{1.837255in}{3.243932in}}%
\pgfpathclose%
\pgfusepath{stroke,fill}%
\end{pgfscope}%
\begin{pgfscope}%
\pgfpathrectangle{\pgfqpoint{0.100000in}{0.212622in}}{\pgfqpoint{3.696000in}{3.696000in}}%
\pgfusepath{clip}%
\pgfsetbuttcap%
\pgfsetroundjoin%
\definecolor{currentfill}{rgb}{0.121569,0.466667,0.705882}%
\pgfsetfillcolor{currentfill}%
\pgfsetfillopacity{0.337942}%
\pgfsetlinewidth{1.003750pt}%
\definecolor{currentstroke}{rgb}{0.121569,0.466667,0.705882}%
\pgfsetstrokecolor{currentstroke}%
\pgfsetstrokeopacity{0.337942}%
\pgfsetdash{}{0pt}%
\pgfpathmoveto{\pgfqpoint{1.837919in}{3.241591in}}%
\pgfpathcurveto{\pgfqpoint{1.846156in}{3.241591in}}{\pgfqpoint{1.854056in}{3.244863in}}{\pgfqpoint{1.859879in}{3.250687in}}%
\pgfpathcurveto{\pgfqpoint{1.865703in}{3.256511in}}{\pgfqpoint{1.868976in}{3.264411in}}{\pgfqpoint{1.868976in}{3.272647in}}%
\pgfpathcurveto{\pgfqpoint{1.868976in}{3.280883in}}{\pgfqpoint{1.865703in}{3.288783in}}{\pgfqpoint{1.859879in}{3.294607in}}%
\pgfpathcurveto{\pgfqpoint{1.854056in}{3.300431in}}{\pgfqpoint{1.846156in}{3.303704in}}{\pgfqpoint{1.837919in}{3.303704in}}%
\pgfpathcurveto{\pgfqpoint{1.829683in}{3.303704in}}{\pgfqpoint{1.821783in}{3.300431in}}{\pgfqpoint{1.815959in}{3.294607in}}%
\pgfpathcurveto{\pgfqpoint{1.810135in}{3.288783in}}{\pgfqpoint{1.806863in}{3.280883in}}{\pgfqpoint{1.806863in}{3.272647in}}%
\pgfpathcurveto{\pgfqpoint{1.806863in}{3.264411in}}{\pgfqpoint{1.810135in}{3.256511in}}{\pgfqpoint{1.815959in}{3.250687in}}%
\pgfpathcurveto{\pgfqpoint{1.821783in}{3.244863in}}{\pgfqpoint{1.829683in}{3.241591in}}{\pgfqpoint{1.837919in}{3.241591in}}%
\pgfpathclose%
\pgfusepath{stroke,fill}%
\end{pgfscope}%
\begin{pgfscope}%
\pgfpathrectangle{\pgfqpoint{0.100000in}{0.212622in}}{\pgfqpoint{3.696000in}{3.696000in}}%
\pgfusepath{clip}%
\pgfsetbuttcap%
\pgfsetroundjoin%
\definecolor{currentfill}{rgb}{0.121569,0.466667,0.705882}%
\pgfsetfillcolor{currentfill}%
\pgfsetfillopacity{0.338734}%
\pgfsetlinewidth{1.003750pt}%
\definecolor{currentstroke}{rgb}{0.121569,0.466667,0.705882}%
\pgfsetstrokecolor{currentstroke}%
\pgfsetstrokeopacity{0.338734}%
\pgfsetdash{}{0pt}%
\pgfpathmoveto{\pgfqpoint{1.673967in}{3.187920in}}%
\pgfpathcurveto{\pgfqpoint{1.682204in}{3.187920in}}{\pgfqpoint{1.690104in}{3.191193in}}{\pgfqpoint{1.695928in}{3.197017in}}%
\pgfpathcurveto{\pgfqpoint{1.701752in}{3.202841in}}{\pgfqpoint{1.705024in}{3.210741in}}{\pgfqpoint{1.705024in}{3.218977in}}%
\pgfpathcurveto{\pgfqpoint{1.705024in}{3.227213in}}{\pgfqpoint{1.701752in}{3.235113in}}{\pgfqpoint{1.695928in}{3.240937in}}%
\pgfpathcurveto{\pgfqpoint{1.690104in}{3.246761in}}{\pgfqpoint{1.682204in}{3.250033in}}{\pgfqpoint{1.673967in}{3.250033in}}%
\pgfpathcurveto{\pgfqpoint{1.665731in}{3.250033in}}{\pgfqpoint{1.657831in}{3.246761in}}{\pgfqpoint{1.652007in}{3.240937in}}%
\pgfpathcurveto{\pgfqpoint{1.646183in}{3.235113in}}{\pgfqpoint{1.642911in}{3.227213in}}{\pgfqpoint{1.642911in}{3.218977in}}%
\pgfpathcurveto{\pgfqpoint{1.642911in}{3.210741in}}{\pgfqpoint{1.646183in}{3.202841in}}{\pgfqpoint{1.652007in}{3.197017in}}%
\pgfpathcurveto{\pgfqpoint{1.657831in}{3.191193in}}{\pgfqpoint{1.665731in}{3.187920in}}{\pgfqpoint{1.673967in}{3.187920in}}%
\pgfpathclose%
\pgfusepath{stroke,fill}%
\end{pgfscope}%
\begin{pgfscope}%
\pgfpathrectangle{\pgfqpoint{0.100000in}{0.212622in}}{\pgfqpoint{3.696000in}{3.696000in}}%
\pgfusepath{clip}%
\pgfsetbuttcap%
\pgfsetroundjoin%
\definecolor{currentfill}{rgb}{0.121569,0.466667,0.705882}%
\pgfsetfillcolor{currentfill}%
\pgfsetfillopacity{0.339584}%
\pgfsetlinewidth{1.003750pt}%
\definecolor{currentstroke}{rgb}{0.121569,0.466667,0.705882}%
\pgfsetstrokecolor{currentstroke}%
\pgfsetstrokeopacity{0.339584}%
\pgfsetdash{}{0pt}%
\pgfpathmoveto{\pgfqpoint{1.672348in}{3.183940in}}%
\pgfpathcurveto{\pgfqpoint{1.680585in}{3.183940in}}{\pgfqpoint{1.688485in}{3.187213in}}{\pgfqpoint{1.694309in}{3.193037in}}%
\pgfpathcurveto{\pgfqpoint{1.700133in}{3.198861in}}{\pgfqpoint{1.703405in}{3.206761in}}{\pgfqpoint{1.703405in}{3.214997in}}%
\pgfpathcurveto{\pgfqpoint{1.703405in}{3.223233in}}{\pgfqpoint{1.700133in}{3.231133in}}{\pgfqpoint{1.694309in}{3.236957in}}%
\pgfpathcurveto{\pgfqpoint{1.688485in}{3.242781in}}{\pgfqpoint{1.680585in}{3.246053in}}{\pgfqpoint{1.672348in}{3.246053in}}%
\pgfpathcurveto{\pgfqpoint{1.664112in}{3.246053in}}{\pgfqpoint{1.656212in}{3.242781in}}{\pgfqpoint{1.650388in}{3.236957in}}%
\pgfpathcurveto{\pgfqpoint{1.644564in}{3.231133in}}{\pgfqpoint{1.641292in}{3.223233in}}{\pgfqpoint{1.641292in}{3.214997in}}%
\pgfpathcurveto{\pgfqpoint{1.641292in}{3.206761in}}{\pgfqpoint{1.644564in}{3.198861in}}{\pgfqpoint{1.650388in}{3.193037in}}%
\pgfpathcurveto{\pgfqpoint{1.656212in}{3.187213in}}{\pgfqpoint{1.664112in}{3.183940in}}{\pgfqpoint{1.672348in}{3.183940in}}%
\pgfpathclose%
\pgfusepath{stroke,fill}%
\end{pgfscope}%
\begin{pgfscope}%
\pgfpathrectangle{\pgfqpoint{0.100000in}{0.212622in}}{\pgfqpoint{3.696000in}{3.696000in}}%
\pgfusepath{clip}%
\pgfsetbuttcap%
\pgfsetroundjoin%
\definecolor{currentfill}{rgb}{0.121569,0.466667,0.705882}%
\pgfsetfillcolor{currentfill}%
\pgfsetfillopacity{0.340019}%
\pgfsetlinewidth{1.003750pt}%
\definecolor{currentstroke}{rgb}{0.121569,0.466667,0.705882}%
\pgfsetstrokecolor{currentstroke}%
\pgfsetstrokeopacity{0.340019}%
\pgfsetdash{}{0pt}%
\pgfpathmoveto{\pgfqpoint{1.840168in}{3.234441in}}%
\pgfpathcurveto{\pgfqpoint{1.848405in}{3.234441in}}{\pgfqpoint{1.856305in}{3.237714in}}{\pgfqpoint{1.862129in}{3.243538in}}%
\pgfpathcurveto{\pgfqpoint{1.867953in}{3.249361in}}{\pgfqpoint{1.871225in}{3.257262in}}{\pgfqpoint{1.871225in}{3.265498in}}%
\pgfpathcurveto{\pgfqpoint{1.871225in}{3.273734in}}{\pgfqpoint{1.867953in}{3.281634in}}{\pgfqpoint{1.862129in}{3.287458in}}%
\pgfpathcurveto{\pgfqpoint{1.856305in}{3.293282in}}{\pgfqpoint{1.848405in}{3.296554in}}{\pgfqpoint{1.840168in}{3.296554in}}%
\pgfpathcurveto{\pgfqpoint{1.831932in}{3.296554in}}{\pgfqpoint{1.824032in}{3.293282in}}{\pgfqpoint{1.818208in}{3.287458in}}%
\pgfpathcurveto{\pgfqpoint{1.812384in}{3.281634in}}{\pgfqpoint{1.809112in}{3.273734in}}{\pgfqpoint{1.809112in}{3.265498in}}%
\pgfpathcurveto{\pgfqpoint{1.809112in}{3.257262in}}{\pgfqpoint{1.812384in}{3.249361in}}{\pgfqpoint{1.818208in}{3.243538in}}%
\pgfpathcurveto{\pgfqpoint{1.824032in}{3.237714in}}{\pgfqpoint{1.831932in}{3.234441in}}{\pgfqpoint{1.840168in}{3.234441in}}%
\pgfpathclose%
\pgfusepath{stroke,fill}%
\end{pgfscope}%
\begin{pgfscope}%
\pgfpathrectangle{\pgfqpoint{0.100000in}{0.212622in}}{\pgfqpoint{3.696000in}{3.696000in}}%
\pgfusepath{clip}%
\pgfsetbuttcap%
\pgfsetroundjoin%
\definecolor{currentfill}{rgb}{0.121569,0.466667,0.705882}%
\pgfsetfillcolor{currentfill}%
\pgfsetfillopacity{0.341157}%
\pgfsetlinewidth{1.003750pt}%
\definecolor{currentstroke}{rgb}{0.121569,0.466667,0.705882}%
\pgfsetstrokecolor{currentstroke}%
\pgfsetstrokeopacity{0.341157}%
\pgfsetdash{}{0pt}%
\pgfpathmoveto{\pgfqpoint{1.669533in}{3.176685in}}%
\pgfpathcurveto{\pgfqpoint{1.677770in}{3.176685in}}{\pgfqpoint{1.685670in}{3.179957in}}{\pgfqpoint{1.691493in}{3.185781in}}%
\pgfpathcurveto{\pgfqpoint{1.697317in}{3.191605in}}{\pgfqpoint{1.700590in}{3.199505in}}{\pgfqpoint{1.700590in}{3.207741in}}%
\pgfpathcurveto{\pgfqpoint{1.700590in}{3.215978in}}{\pgfqpoint{1.697317in}{3.223878in}}{\pgfqpoint{1.691493in}{3.229702in}}%
\pgfpathcurveto{\pgfqpoint{1.685670in}{3.235525in}}{\pgfqpoint{1.677770in}{3.238798in}}{\pgfqpoint{1.669533in}{3.238798in}}%
\pgfpathcurveto{\pgfqpoint{1.661297in}{3.238798in}}{\pgfqpoint{1.653397in}{3.235525in}}{\pgfqpoint{1.647573in}{3.229702in}}%
\pgfpathcurveto{\pgfqpoint{1.641749in}{3.223878in}}{\pgfqpoint{1.638477in}{3.215978in}}{\pgfqpoint{1.638477in}{3.207741in}}%
\pgfpathcurveto{\pgfqpoint{1.638477in}{3.199505in}}{\pgfqpoint{1.641749in}{3.191605in}}{\pgfqpoint{1.647573in}{3.185781in}}%
\pgfpathcurveto{\pgfqpoint{1.653397in}{3.179957in}}{\pgfqpoint{1.661297in}{3.176685in}}{\pgfqpoint{1.669533in}{3.176685in}}%
\pgfpathclose%
\pgfusepath{stroke,fill}%
\end{pgfscope}%
\begin{pgfscope}%
\pgfpathrectangle{\pgfqpoint{0.100000in}{0.212622in}}{\pgfqpoint{3.696000in}{3.696000in}}%
\pgfusepath{clip}%
\pgfsetbuttcap%
\pgfsetroundjoin%
\definecolor{currentfill}{rgb}{0.121569,0.466667,0.705882}%
\pgfsetfillcolor{currentfill}%
\pgfsetfillopacity{0.342127}%
\pgfsetlinewidth{1.003750pt}%
\definecolor{currentstroke}{rgb}{0.121569,0.466667,0.705882}%
\pgfsetstrokecolor{currentstroke}%
\pgfsetstrokeopacity{0.342127}%
\pgfsetdash{}{0pt}%
\pgfpathmoveto{\pgfqpoint{1.668093in}{3.172629in}}%
\pgfpathcurveto{\pgfqpoint{1.676329in}{3.172629in}}{\pgfqpoint{1.684229in}{3.175901in}}{\pgfqpoint{1.690053in}{3.181725in}}%
\pgfpathcurveto{\pgfqpoint{1.695877in}{3.187549in}}{\pgfqpoint{1.699149in}{3.195449in}}{\pgfqpoint{1.699149in}{3.203686in}}%
\pgfpathcurveto{\pgfqpoint{1.699149in}{3.211922in}}{\pgfqpoint{1.695877in}{3.219822in}}{\pgfqpoint{1.690053in}{3.225646in}}%
\pgfpathcurveto{\pgfqpoint{1.684229in}{3.231470in}}{\pgfqpoint{1.676329in}{3.234742in}}{\pgfqpoint{1.668093in}{3.234742in}}%
\pgfpathcurveto{\pgfqpoint{1.659857in}{3.234742in}}{\pgfqpoint{1.651957in}{3.231470in}}{\pgfqpoint{1.646133in}{3.225646in}}%
\pgfpathcurveto{\pgfqpoint{1.640309in}{3.219822in}}{\pgfqpoint{1.637036in}{3.211922in}}{\pgfqpoint{1.637036in}{3.203686in}}%
\pgfpathcurveto{\pgfqpoint{1.637036in}{3.195449in}}{\pgfqpoint{1.640309in}{3.187549in}}{\pgfqpoint{1.646133in}{3.181725in}}%
\pgfpathcurveto{\pgfqpoint{1.651957in}{3.175901in}}{\pgfqpoint{1.659857in}{3.172629in}}{\pgfqpoint{1.668093in}{3.172629in}}%
\pgfpathclose%
\pgfusepath{stroke,fill}%
\end{pgfscope}%
\begin{pgfscope}%
\pgfpathrectangle{\pgfqpoint{0.100000in}{0.212622in}}{\pgfqpoint{3.696000in}{3.696000in}}%
\pgfusepath{clip}%
\pgfsetbuttcap%
\pgfsetroundjoin%
\definecolor{currentfill}{rgb}{0.121569,0.466667,0.705882}%
\pgfsetfillcolor{currentfill}%
\pgfsetfillopacity{0.342488}%
\pgfsetlinewidth{1.003750pt}%
\definecolor{currentstroke}{rgb}{0.121569,0.466667,0.705882}%
\pgfsetstrokecolor{currentstroke}%
\pgfsetstrokeopacity{0.342488}%
\pgfsetdash{}{0pt}%
\pgfpathmoveto{\pgfqpoint{1.667412in}{3.170646in}}%
\pgfpathcurveto{\pgfqpoint{1.675649in}{3.170646in}}{\pgfqpoint{1.683549in}{3.173919in}}{\pgfqpoint{1.689373in}{3.179742in}}%
\pgfpathcurveto{\pgfqpoint{1.695197in}{3.185566in}}{\pgfqpoint{1.698469in}{3.193466in}}{\pgfqpoint{1.698469in}{3.201703in}}%
\pgfpathcurveto{\pgfqpoint{1.698469in}{3.209939in}}{\pgfqpoint{1.695197in}{3.217839in}}{\pgfqpoint{1.689373in}{3.223663in}}%
\pgfpathcurveto{\pgfqpoint{1.683549in}{3.229487in}}{\pgfqpoint{1.675649in}{3.232759in}}{\pgfqpoint{1.667412in}{3.232759in}}%
\pgfpathcurveto{\pgfqpoint{1.659176in}{3.232759in}}{\pgfqpoint{1.651276in}{3.229487in}}{\pgfqpoint{1.645452in}{3.223663in}}%
\pgfpathcurveto{\pgfqpoint{1.639628in}{3.217839in}}{\pgfqpoint{1.636356in}{3.209939in}}{\pgfqpoint{1.636356in}{3.201703in}}%
\pgfpathcurveto{\pgfqpoint{1.636356in}{3.193466in}}{\pgfqpoint{1.639628in}{3.185566in}}{\pgfqpoint{1.645452in}{3.179742in}}%
\pgfpathcurveto{\pgfqpoint{1.651276in}{3.173919in}}{\pgfqpoint{1.659176in}{3.170646in}}{\pgfqpoint{1.667412in}{3.170646in}}%
\pgfpathclose%
\pgfusepath{stroke,fill}%
\end{pgfscope}%
\begin{pgfscope}%
\pgfpathrectangle{\pgfqpoint{0.100000in}{0.212622in}}{\pgfqpoint{3.696000in}{3.696000in}}%
\pgfusepath{clip}%
\pgfsetbuttcap%
\pgfsetroundjoin%
\definecolor{currentfill}{rgb}{0.121569,0.466667,0.705882}%
\pgfsetfillcolor{currentfill}%
\pgfsetfillopacity{0.342851}%
\pgfsetlinewidth{1.003750pt}%
\definecolor{currentstroke}{rgb}{0.121569,0.466667,0.705882}%
\pgfsetstrokecolor{currentstroke}%
\pgfsetstrokeopacity{0.342851}%
\pgfsetdash{}{0pt}%
\pgfpathmoveto{\pgfqpoint{1.842613in}{3.223894in}}%
\pgfpathcurveto{\pgfqpoint{1.850849in}{3.223894in}}{\pgfqpoint{1.858749in}{3.227166in}}{\pgfqpoint{1.864573in}{3.232990in}}%
\pgfpathcurveto{\pgfqpoint{1.870397in}{3.238814in}}{\pgfqpoint{1.873670in}{3.246714in}}{\pgfqpoint{1.873670in}{3.254950in}}%
\pgfpathcurveto{\pgfqpoint{1.873670in}{3.263186in}}{\pgfqpoint{1.870397in}{3.271086in}}{\pgfqpoint{1.864573in}{3.276910in}}%
\pgfpathcurveto{\pgfqpoint{1.858749in}{3.282734in}}{\pgfqpoint{1.850849in}{3.286007in}}{\pgfqpoint{1.842613in}{3.286007in}}%
\pgfpathcurveto{\pgfqpoint{1.834377in}{3.286007in}}{\pgfqpoint{1.826477in}{3.282734in}}{\pgfqpoint{1.820653in}{3.276910in}}%
\pgfpathcurveto{\pgfqpoint{1.814829in}{3.271086in}}{\pgfqpoint{1.811557in}{3.263186in}}{\pgfqpoint{1.811557in}{3.254950in}}%
\pgfpathcurveto{\pgfqpoint{1.811557in}{3.246714in}}{\pgfqpoint{1.814829in}{3.238814in}}{\pgfqpoint{1.820653in}{3.232990in}}%
\pgfpathcurveto{\pgfqpoint{1.826477in}{3.227166in}}{\pgfqpoint{1.834377in}{3.223894in}}{\pgfqpoint{1.842613in}{3.223894in}}%
\pgfpathclose%
\pgfusepath{stroke,fill}%
\end{pgfscope}%
\begin{pgfscope}%
\pgfpathrectangle{\pgfqpoint{0.100000in}{0.212622in}}{\pgfqpoint{3.696000in}{3.696000in}}%
\pgfusepath{clip}%
\pgfsetbuttcap%
\pgfsetroundjoin%
\definecolor{currentfill}{rgb}{0.121569,0.466667,0.705882}%
\pgfsetfillcolor{currentfill}%
\pgfsetfillopacity{0.343247}%
\pgfsetlinewidth{1.003750pt}%
\definecolor{currentstroke}{rgb}{0.121569,0.466667,0.705882}%
\pgfsetstrokecolor{currentstroke}%
\pgfsetstrokeopacity{0.343247}%
\pgfsetdash{}{0pt}%
\pgfpathmoveto{\pgfqpoint{1.666193in}{3.167432in}}%
\pgfpathcurveto{\pgfqpoint{1.674430in}{3.167432in}}{\pgfqpoint{1.682330in}{3.170704in}}{\pgfqpoint{1.688154in}{3.176528in}}%
\pgfpathcurveto{\pgfqpoint{1.693978in}{3.182352in}}{\pgfqpoint{1.697250in}{3.190252in}}{\pgfqpoint{1.697250in}{3.198489in}}%
\pgfpathcurveto{\pgfqpoint{1.697250in}{3.206725in}}{\pgfqpoint{1.693978in}{3.214625in}}{\pgfqpoint{1.688154in}{3.220449in}}%
\pgfpathcurveto{\pgfqpoint{1.682330in}{3.226273in}}{\pgfqpoint{1.674430in}{3.229545in}}{\pgfqpoint{1.666193in}{3.229545in}}%
\pgfpathcurveto{\pgfqpoint{1.657957in}{3.229545in}}{\pgfqpoint{1.650057in}{3.226273in}}{\pgfqpoint{1.644233in}{3.220449in}}%
\pgfpathcurveto{\pgfqpoint{1.638409in}{3.214625in}}{\pgfqpoint{1.635137in}{3.206725in}}{\pgfqpoint{1.635137in}{3.198489in}}%
\pgfpathcurveto{\pgfqpoint{1.635137in}{3.190252in}}{\pgfqpoint{1.638409in}{3.182352in}}{\pgfqpoint{1.644233in}{3.176528in}}%
\pgfpathcurveto{\pgfqpoint{1.650057in}{3.170704in}}{\pgfqpoint{1.657957in}{3.167432in}}{\pgfqpoint{1.666193in}{3.167432in}}%
\pgfpathclose%
\pgfusepath{stroke,fill}%
\end{pgfscope}%
\begin{pgfscope}%
\pgfpathrectangle{\pgfqpoint{0.100000in}{0.212622in}}{\pgfqpoint{3.696000in}{3.696000in}}%
\pgfusepath{clip}%
\pgfsetbuttcap%
\pgfsetroundjoin%
\definecolor{currentfill}{rgb}{0.121569,0.466667,0.705882}%
\pgfsetfillcolor{currentfill}%
\pgfsetfillopacity{0.343598}%
\pgfsetlinewidth{1.003750pt}%
\definecolor{currentstroke}{rgb}{0.121569,0.466667,0.705882}%
\pgfsetstrokecolor{currentstroke}%
\pgfsetstrokeopacity{0.343598}%
\pgfsetdash{}{0pt}%
\pgfpathmoveto{\pgfqpoint{1.665601in}{3.165885in}}%
\pgfpathcurveto{\pgfqpoint{1.673838in}{3.165885in}}{\pgfqpoint{1.681738in}{3.169157in}}{\pgfqpoint{1.687562in}{3.174981in}}%
\pgfpathcurveto{\pgfqpoint{1.693386in}{3.180805in}}{\pgfqpoint{1.696658in}{3.188705in}}{\pgfqpoint{1.696658in}{3.196941in}}%
\pgfpathcurveto{\pgfqpoint{1.696658in}{3.205178in}}{\pgfqpoint{1.693386in}{3.213078in}}{\pgfqpoint{1.687562in}{3.218902in}}%
\pgfpathcurveto{\pgfqpoint{1.681738in}{3.224726in}}{\pgfqpoint{1.673838in}{3.227998in}}{\pgfqpoint{1.665601in}{3.227998in}}%
\pgfpathcurveto{\pgfqpoint{1.657365in}{3.227998in}}{\pgfqpoint{1.649465in}{3.224726in}}{\pgfqpoint{1.643641in}{3.218902in}}%
\pgfpathcurveto{\pgfqpoint{1.637817in}{3.213078in}}{\pgfqpoint{1.634545in}{3.205178in}}{\pgfqpoint{1.634545in}{3.196941in}}%
\pgfpathcurveto{\pgfqpoint{1.634545in}{3.188705in}}{\pgfqpoint{1.637817in}{3.180805in}}{\pgfqpoint{1.643641in}{3.174981in}}%
\pgfpathcurveto{\pgfqpoint{1.649465in}{3.169157in}}{\pgfqpoint{1.657365in}{3.165885in}}{\pgfqpoint{1.665601in}{3.165885in}}%
\pgfpathclose%
\pgfusepath{stroke,fill}%
\end{pgfscope}%
\begin{pgfscope}%
\pgfpathrectangle{\pgfqpoint{0.100000in}{0.212622in}}{\pgfqpoint{3.696000in}{3.696000in}}%
\pgfusepath{clip}%
\pgfsetbuttcap%
\pgfsetroundjoin%
\definecolor{currentfill}{rgb}{0.121569,0.466667,0.705882}%
\pgfsetfillcolor{currentfill}%
\pgfsetfillopacity{0.344221}%
\pgfsetlinewidth{1.003750pt}%
\definecolor{currentstroke}{rgb}{0.121569,0.466667,0.705882}%
\pgfsetstrokecolor{currentstroke}%
\pgfsetstrokeopacity{0.344221}%
\pgfsetdash{}{0pt}%
\pgfpathmoveto{\pgfqpoint{1.664457in}{3.163082in}}%
\pgfpathcurveto{\pgfqpoint{1.672694in}{3.163082in}}{\pgfqpoint{1.680594in}{3.166355in}}{\pgfqpoint{1.686418in}{3.172179in}}%
\pgfpathcurveto{\pgfqpoint{1.692242in}{3.178003in}}{\pgfqpoint{1.695514in}{3.185903in}}{\pgfqpoint{1.695514in}{3.194139in}}%
\pgfpathcurveto{\pgfqpoint{1.695514in}{3.202375in}}{\pgfqpoint{1.692242in}{3.210275in}}{\pgfqpoint{1.686418in}{3.216099in}}%
\pgfpathcurveto{\pgfqpoint{1.680594in}{3.221923in}}{\pgfqpoint{1.672694in}{3.225195in}}{\pgfqpoint{1.664457in}{3.225195in}}%
\pgfpathcurveto{\pgfqpoint{1.656221in}{3.225195in}}{\pgfqpoint{1.648321in}{3.221923in}}{\pgfqpoint{1.642497in}{3.216099in}}%
\pgfpathcurveto{\pgfqpoint{1.636673in}{3.210275in}}{\pgfqpoint{1.633401in}{3.202375in}}{\pgfqpoint{1.633401in}{3.194139in}}%
\pgfpathcurveto{\pgfqpoint{1.633401in}{3.185903in}}{\pgfqpoint{1.636673in}{3.178003in}}{\pgfqpoint{1.642497in}{3.172179in}}%
\pgfpathcurveto{\pgfqpoint{1.648321in}{3.166355in}}{\pgfqpoint{1.656221in}{3.163082in}}{\pgfqpoint{1.664457in}{3.163082in}}%
\pgfpathclose%
\pgfusepath{stroke,fill}%
\end{pgfscope}%
\begin{pgfscope}%
\pgfpathrectangle{\pgfqpoint{0.100000in}{0.212622in}}{\pgfqpoint{3.696000in}{3.696000in}}%
\pgfusepath{clip}%
\pgfsetbuttcap%
\pgfsetroundjoin%
\definecolor{currentfill}{rgb}{0.121569,0.466667,0.705882}%
\pgfsetfillcolor{currentfill}%
\pgfsetfillopacity{0.344311}%
\pgfsetlinewidth{1.003750pt}%
\definecolor{currentstroke}{rgb}{0.121569,0.466667,0.705882}%
\pgfsetstrokecolor{currentstroke}%
\pgfsetstrokeopacity{0.344311}%
\pgfsetdash{}{0pt}%
\pgfpathmoveto{\pgfqpoint{1.664329in}{3.162728in}}%
\pgfpathcurveto{\pgfqpoint{1.672566in}{3.162728in}}{\pgfqpoint{1.680466in}{3.166001in}}{\pgfqpoint{1.686290in}{3.171824in}}%
\pgfpathcurveto{\pgfqpoint{1.692114in}{3.177648in}}{\pgfqpoint{1.695386in}{3.185548in}}{\pgfqpoint{1.695386in}{3.193785in}}%
\pgfpathcurveto{\pgfqpoint{1.695386in}{3.202021in}}{\pgfqpoint{1.692114in}{3.209921in}}{\pgfqpoint{1.686290in}{3.215745in}}%
\pgfpathcurveto{\pgfqpoint{1.680466in}{3.221569in}}{\pgfqpoint{1.672566in}{3.224841in}}{\pgfqpoint{1.664329in}{3.224841in}}%
\pgfpathcurveto{\pgfqpoint{1.656093in}{3.224841in}}{\pgfqpoint{1.648193in}{3.221569in}}{\pgfqpoint{1.642369in}{3.215745in}}%
\pgfpathcurveto{\pgfqpoint{1.636545in}{3.209921in}}{\pgfqpoint{1.633273in}{3.202021in}}{\pgfqpoint{1.633273in}{3.193785in}}%
\pgfpathcurveto{\pgfqpoint{1.633273in}{3.185548in}}{\pgfqpoint{1.636545in}{3.177648in}}{\pgfqpoint{1.642369in}{3.171824in}}%
\pgfpathcurveto{\pgfqpoint{1.648193in}{3.166001in}}{\pgfqpoint{1.656093in}{3.162728in}}{\pgfqpoint{1.664329in}{3.162728in}}%
\pgfpathclose%
\pgfusepath{stroke,fill}%
\end{pgfscope}%
\begin{pgfscope}%
\pgfpathrectangle{\pgfqpoint{0.100000in}{0.212622in}}{\pgfqpoint{3.696000in}{3.696000in}}%
\pgfusepath{clip}%
\pgfsetbuttcap%
\pgfsetroundjoin%
\definecolor{currentfill}{rgb}{0.121569,0.466667,0.705882}%
\pgfsetfillcolor{currentfill}%
\pgfsetfillopacity{0.344440}%
\pgfsetlinewidth{1.003750pt}%
\definecolor{currentstroke}{rgb}{0.121569,0.466667,0.705882}%
\pgfsetstrokecolor{currentstroke}%
\pgfsetstrokeopacity{0.344440}%
\pgfsetdash{}{0pt}%
\pgfpathmoveto{\pgfqpoint{1.664048in}{3.162001in}}%
\pgfpathcurveto{\pgfqpoint{1.672284in}{3.162001in}}{\pgfqpoint{1.680184in}{3.165274in}}{\pgfqpoint{1.686008in}{3.171098in}}%
\pgfpathcurveto{\pgfqpoint{1.691832in}{3.176922in}}{\pgfqpoint{1.695105in}{3.184822in}}{\pgfqpoint{1.695105in}{3.193058in}}%
\pgfpathcurveto{\pgfqpoint{1.695105in}{3.201294in}}{\pgfqpoint{1.691832in}{3.209194in}}{\pgfqpoint{1.686008in}{3.215018in}}%
\pgfpathcurveto{\pgfqpoint{1.680184in}{3.220842in}}{\pgfqpoint{1.672284in}{3.224114in}}{\pgfqpoint{1.664048in}{3.224114in}}%
\pgfpathcurveto{\pgfqpoint{1.655812in}{3.224114in}}{\pgfqpoint{1.647912in}{3.220842in}}{\pgfqpoint{1.642088in}{3.215018in}}%
\pgfpathcurveto{\pgfqpoint{1.636264in}{3.209194in}}{\pgfqpoint{1.632992in}{3.201294in}}{\pgfqpoint{1.632992in}{3.193058in}}%
\pgfpathcurveto{\pgfqpoint{1.632992in}{3.184822in}}{\pgfqpoint{1.636264in}{3.176922in}}{\pgfqpoint{1.642088in}{3.171098in}}%
\pgfpathcurveto{\pgfqpoint{1.647912in}{3.165274in}}{\pgfqpoint{1.655812in}{3.162001in}}{\pgfqpoint{1.664048in}{3.162001in}}%
\pgfpathclose%
\pgfusepath{stroke,fill}%
\end{pgfscope}%
\begin{pgfscope}%
\pgfpathrectangle{\pgfqpoint{0.100000in}{0.212622in}}{\pgfqpoint{3.696000in}{3.696000in}}%
\pgfusepath{clip}%
\pgfsetbuttcap%
\pgfsetroundjoin%
\definecolor{currentfill}{rgb}{0.121569,0.466667,0.705882}%
\pgfsetfillcolor{currentfill}%
\pgfsetfillopacity{0.344701}%
\pgfsetlinewidth{1.003750pt}%
\definecolor{currentstroke}{rgb}{0.121569,0.466667,0.705882}%
\pgfsetstrokecolor{currentstroke}%
\pgfsetstrokeopacity{0.344701}%
\pgfsetdash{}{0pt}%
\pgfpathmoveto{\pgfqpoint{1.663512in}{3.160802in}}%
\pgfpathcurveto{\pgfqpoint{1.671748in}{3.160802in}}{\pgfqpoint{1.679648in}{3.164074in}}{\pgfqpoint{1.685472in}{3.169898in}}%
\pgfpathcurveto{\pgfqpoint{1.691296in}{3.175722in}}{\pgfqpoint{1.694568in}{3.183622in}}{\pgfqpoint{1.694568in}{3.191858in}}%
\pgfpathcurveto{\pgfqpoint{1.694568in}{3.200095in}}{\pgfqpoint{1.691296in}{3.207995in}}{\pgfqpoint{1.685472in}{3.213819in}}%
\pgfpathcurveto{\pgfqpoint{1.679648in}{3.219643in}}{\pgfqpoint{1.671748in}{3.222915in}}{\pgfqpoint{1.663512in}{3.222915in}}%
\pgfpathcurveto{\pgfqpoint{1.655276in}{3.222915in}}{\pgfqpoint{1.647376in}{3.219643in}}{\pgfqpoint{1.641552in}{3.213819in}}%
\pgfpathcurveto{\pgfqpoint{1.635728in}{3.207995in}}{\pgfqpoint{1.632455in}{3.200095in}}{\pgfqpoint{1.632455in}{3.191858in}}%
\pgfpathcurveto{\pgfqpoint{1.632455in}{3.183622in}}{\pgfqpoint{1.635728in}{3.175722in}}{\pgfqpoint{1.641552in}{3.169898in}}%
\pgfpathcurveto{\pgfqpoint{1.647376in}{3.164074in}}{\pgfqpoint{1.655276in}{3.160802in}}{\pgfqpoint{1.663512in}{3.160802in}}%
\pgfpathclose%
\pgfusepath{stroke,fill}%
\end{pgfscope}%
\begin{pgfscope}%
\pgfpathrectangle{\pgfqpoint{0.100000in}{0.212622in}}{\pgfqpoint{3.696000in}{3.696000in}}%
\pgfusepath{clip}%
\pgfsetbuttcap%
\pgfsetroundjoin%
\definecolor{currentfill}{rgb}{0.121569,0.466667,0.705882}%
\pgfsetfillcolor{currentfill}%
\pgfsetfillopacity{0.345153}%
\pgfsetlinewidth{1.003750pt}%
\definecolor{currentstroke}{rgb}{0.121569,0.466667,0.705882}%
\pgfsetstrokecolor{currentstroke}%
\pgfsetstrokeopacity{0.345153}%
\pgfsetdash{}{0pt}%
\pgfpathmoveto{\pgfqpoint{1.662488in}{3.158588in}}%
\pgfpathcurveto{\pgfqpoint{1.670724in}{3.158588in}}{\pgfqpoint{1.678624in}{3.161860in}}{\pgfqpoint{1.684448in}{3.167684in}}%
\pgfpathcurveto{\pgfqpoint{1.690272in}{3.173508in}}{\pgfqpoint{1.693544in}{3.181408in}}{\pgfqpoint{1.693544in}{3.189644in}}%
\pgfpathcurveto{\pgfqpoint{1.693544in}{3.197880in}}{\pgfqpoint{1.690272in}{3.205781in}}{\pgfqpoint{1.684448in}{3.211604in}}%
\pgfpathcurveto{\pgfqpoint{1.678624in}{3.217428in}}{\pgfqpoint{1.670724in}{3.220701in}}{\pgfqpoint{1.662488in}{3.220701in}}%
\pgfpathcurveto{\pgfqpoint{1.654251in}{3.220701in}}{\pgfqpoint{1.646351in}{3.217428in}}{\pgfqpoint{1.640527in}{3.211604in}}%
\pgfpathcurveto{\pgfqpoint{1.634704in}{3.205781in}}{\pgfqpoint{1.631431in}{3.197880in}}{\pgfqpoint{1.631431in}{3.189644in}}%
\pgfpathcurveto{\pgfqpoint{1.631431in}{3.181408in}}{\pgfqpoint{1.634704in}{3.173508in}}{\pgfqpoint{1.640527in}{3.167684in}}%
\pgfpathcurveto{\pgfqpoint{1.646351in}{3.161860in}}{\pgfqpoint{1.654251in}{3.158588in}}{\pgfqpoint{1.662488in}{3.158588in}}%
\pgfpathclose%
\pgfusepath{stroke,fill}%
\end{pgfscope}%
\begin{pgfscope}%
\pgfpathrectangle{\pgfqpoint{0.100000in}{0.212622in}}{\pgfqpoint{3.696000in}{3.696000in}}%
\pgfusepath{clip}%
\pgfsetbuttcap%
\pgfsetroundjoin%
\definecolor{currentfill}{rgb}{0.121569,0.466667,0.705882}%
\pgfsetfillcolor{currentfill}%
\pgfsetfillopacity{0.346024}%
\pgfsetlinewidth{1.003750pt}%
\definecolor{currentstroke}{rgb}{0.121569,0.466667,0.705882}%
\pgfsetstrokecolor{currentstroke}%
\pgfsetstrokeopacity{0.346024}%
\pgfsetdash{}{0pt}%
\pgfpathmoveto{\pgfqpoint{1.660703in}{3.154679in}}%
\pgfpathcurveto{\pgfqpoint{1.668939in}{3.154679in}}{\pgfqpoint{1.676839in}{3.157952in}}{\pgfqpoint{1.682663in}{3.163775in}}%
\pgfpathcurveto{\pgfqpoint{1.688487in}{3.169599in}}{\pgfqpoint{1.691760in}{3.177499in}}{\pgfqpoint{1.691760in}{3.185736in}}%
\pgfpathcurveto{\pgfqpoint{1.691760in}{3.193972in}}{\pgfqpoint{1.688487in}{3.201872in}}{\pgfqpoint{1.682663in}{3.207696in}}%
\pgfpathcurveto{\pgfqpoint{1.676839in}{3.213520in}}{\pgfqpoint{1.668939in}{3.216792in}}{\pgfqpoint{1.660703in}{3.216792in}}%
\pgfpathcurveto{\pgfqpoint{1.652467in}{3.216792in}}{\pgfqpoint{1.644567in}{3.213520in}}{\pgfqpoint{1.638743in}{3.207696in}}%
\pgfpathcurveto{\pgfqpoint{1.632919in}{3.201872in}}{\pgfqpoint{1.629647in}{3.193972in}}{\pgfqpoint{1.629647in}{3.185736in}}%
\pgfpathcurveto{\pgfqpoint{1.629647in}{3.177499in}}{\pgfqpoint{1.632919in}{3.169599in}}{\pgfqpoint{1.638743in}{3.163775in}}%
\pgfpathcurveto{\pgfqpoint{1.644567in}{3.157952in}}{\pgfqpoint{1.652467in}{3.154679in}}{\pgfqpoint{1.660703in}{3.154679in}}%
\pgfpathclose%
\pgfusepath{stroke,fill}%
\end{pgfscope}%
\begin{pgfscope}%
\pgfpathrectangle{\pgfqpoint{0.100000in}{0.212622in}}{\pgfqpoint{3.696000in}{3.696000in}}%
\pgfusepath{clip}%
\pgfsetbuttcap%
\pgfsetroundjoin%
\definecolor{currentfill}{rgb}{0.121569,0.466667,0.705882}%
\pgfsetfillcolor{currentfill}%
\pgfsetfillopacity{0.346530}%
\pgfsetlinewidth{1.003750pt}%
\definecolor{currentstroke}{rgb}{0.121569,0.466667,0.705882}%
\pgfsetstrokecolor{currentstroke}%
\pgfsetstrokeopacity{0.346530}%
\pgfsetdash{}{0pt}%
\pgfpathmoveto{\pgfqpoint{1.659616in}{3.152317in}}%
\pgfpathcurveto{\pgfqpoint{1.667853in}{3.152317in}}{\pgfqpoint{1.675753in}{3.155590in}}{\pgfqpoint{1.681577in}{3.161414in}}%
\pgfpathcurveto{\pgfqpoint{1.687401in}{3.167238in}}{\pgfqpoint{1.690673in}{3.175138in}}{\pgfqpoint{1.690673in}{3.183374in}}%
\pgfpathcurveto{\pgfqpoint{1.690673in}{3.191610in}}{\pgfqpoint{1.687401in}{3.199510in}}{\pgfqpoint{1.681577in}{3.205334in}}%
\pgfpathcurveto{\pgfqpoint{1.675753in}{3.211158in}}{\pgfqpoint{1.667853in}{3.214430in}}{\pgfqpoint{1.659616in}{3.214430in}}%
\pgfpathcurveto{\pgfqpoint{1.651380in}{3.214430in}}{\pgfqpoint{1.643480in}{3.211158in}}{\pgfqpoint{1.637656in}{3.205334in}}%
\pgfpathcurveto{\pgfqpoint{1.631832in}{3.199510in}}{\pgfqpoint{1.628560in}{3.191610in}}{\pgfqpoint{1.628560in}{3.183374in}}%
\pgfpathcurveto{\pgfqpoint{1.628560in}{3.175138in}}{\pgfqpoint{1.631832in}{3.167238in}}{\pgfqpoint{1.637656in}{3.161414in}}%
\pgfpathcurveto{\pgfqpoint{1.643480in}{3.155590in}}{\pgfqpoint{1.651380in}{3.152317in}}{\pgfqpoint{1.659616in}{3.152317in}}%
\pgfpathclose%
\pgfusepath{stroke,fill}%
\end{pgfscope}%
\begin{pgfscope}%
\pgfpathrectangle{\pgfqpoint{0.100000in}{0.212622in}}{\pgfqpoint{3.696000in}{3.696000in}}%
\pgfusepath{clip}%
\pgfsetbuttcap%
\pgfsetroundjoin%
\definecolor{currentfill}{rgb}{0.121569,0.466667,0.705882}%
\pgfsetfillcolor{currentfill}%
\pgfsetfillopacity{0.346751}%
\pgfsetlinewidth{1.003750pt}%
\definecolor{currentstroke}{rgb}{0.121569,0.466667,0.705882}%
\pgfsetstrokecolor{currentstroke}%
\pgfsetstrokeopacity{0.346751}%
\pgfsetdash{}{0pt}%
\pgfpathmoveto{\pgfqpoint{1.846243in}{3.209913in}}%
\pgfpathcurveto{\pgfqpoint{1.854479in}{3.209913in}}{\pgfqpoint{1.862379in}{3.213185in}}{\pgfqpoint{1.868203in}{3.219009in}}%
\pgfpathcurveto{\pgfqpoint{1.874027in}{3.224833in}}{\pgfqpoint{1.877299in}{3.232733in}}{\pgfqpoint{1.877299in}{3.240969in}}%
\pgfpathcurveto{\pgfqpoint{1.877299in}{3.249206in}}{\pgfqpoint{1.874027in}{3.257106in}}{\pgfqpoint{1.868203in}{3.262930in}}%
\pgfpathcurveto{\pgfqpoint{1.862379in}{3.268753in}}{\pgfqpoint{1.854479in}{3.272026in}}{\pgfqpoint{1.846243in}{3.272026in}}%
\pgfpathcurveto{\pgfqpoint{1.838007in}{3.272026in}}{\pgfqpoint{1.830107in}{3.268753in}}{\pgfqpoint{1.824283in}{3.262930in}}%
\pgfpathcurveto{\pgfqpoint{1.818459in}{3.257106in}}{\pgfqpoint{1.815186in}{3.249206in}}{\pgfqpoint{1.815186in}{3.240969in}}%
\pgfpathcurveto{\pgfqpoint{1.815186in}{3.232733in}}{\pgfqpoint{1.818459in}{3.224833in}}{\pgfqpoint{1.824283in}{3.219009in}}%
\pgfpathcurveto{\pgfqpoint{1.830107in}{3.213185in}}{\pgfqpoint{1.838007in}{3.209913in}}{\pgfqpoint{1.846243in}{3.209913in}}%
\pgfpathclose%
\pgfusepath{stroke,fill}%
\end{pgfscope}%
\begin{pgfscope}%
\pgfpathrectangle{\pgfqpoint{0.100000in}{0.212622in}}{\pgfqpoint{3.696000in}{3.696000in}}%
\pgfusepath{clip}%
\pgfsetbuttcap%
\pgfsetroundjoin%
\definecolor{currentfill}{rgb}{0.121569,0.466667,0.705882}%
\pgfsetfillcolor{currentfill}%
\pgfsetfillopacity{0.347502}%
\pgfsetlinewidth{1.003750pt}%
\definecolor{currentstroke}{rgb}{0.121569,0.466667,0.705882}%
\pgfsetstrokecolor{currentstroke}%
\pgfsetstrokeopacity{0.347502}%
\pgfsetdash{}{0pt}%
\pgfpathmoveto{\pgfqpoint{1.657732in}{3.148142in}}%
\pgfpathcurveto{\pgfqpoint{1.665969in}{3.148142in}}{\pgfqpoint{1.673869in}{3.151414in}}{\pgfqpoint{1.679692in}{3.157238in}}%
\pgfpathcurveto{\pgfqpoint{1.685516in}{3.163062in}}{\pgfqpoint{1.688789in}{3.170962in}}{\pgfqpoint{1.688789in}{3.179199in}}%
\pgfpathcurveto{\pgfqpoint{1.688789in}{3.187435in}}{\pgfqpoint{1.685516in}{3.195335in}}{\pgfqpoint{1.679692in}{3.201159in}}%
\pgfpathcurveto{\pgfqpoint{1.673869in}{3.206983in}}{\pgfqpoint{1.665969in}{3.210255in}}{\pgfqpoint{1.657732in}{3.210255in}}%
\pgfpathcurveto{\pgfqpoint{1.649496in}{3.210255in}}{\pgfqpoint{1.641596in}{3.206983in}}{\pgfqpoint{1.635772in}{3.201159in}}%
\pgfpathcurveto{\pgfqpoint{1.629948in}{3.195335in}}{\pgfqpoint{1.626676in}{3.187435in}}{\pgfqpoint{1.626676in}{3.179199in}}%
\pgfpathcurveto{\pgfqpoint{1.626676in}{3.170962in}}{\pgfqpoint{1.629948in}{3.163062in}}{\pgfqpoint{1.635772in}{3.157238in}}%
\pgfpathcurveto{\pgfqpoint{1.641596in}{3.151414in}}{\pgfqpoint{1.649496in}{3.148142in}}{\pgfqpoint{1.657732in}{3.148142in}}%
\pgfpathclose%
\pgfusepath{stroke,fill}%
\end{pgfscope}%
\begin{pgfscope}%
\pgfpathrectangle{\pgfqpoint{0.100000in}{0.212622in}}{\pgfqpoint{3.696000in}{3.696000in}}%
\pgfusepath{clip}%
\pgfsetbuttcap%
\pgfsetroundjoin%
\definecolor{currentfill}{rgb}{0.121569,0.466667,0.705882}%
\pgfsetfillcolor{currentfill}%
\pgfsetfillopacity{0.349195}%
\pgfsetlinewidth{1.003750pt}%
\definecolor{currentstroke}{rgb}{0.121569,0.466667,0.705882}%
\pgfsetstrokecolor{currentstroke}%
\pgfsetstrokeopacity{0.349195}%
\pgfsetdash{}{0pt}%
\pgfpathmoveto{\pgfqpoint{1.654179in}{3.140371in}}%
\pgfpathcurveto{\pgfqpoint{1.662415in}{3.140371in}}{\pgfqpoint{1.670315in}{3.143643in}}{\pgfqpoint{1.676139in}{3.149467in}}%
\pgfpathcurveto{\pgfqpoint{1.681963in}{3.155291in}}{\pgfqpoint{1.685235in}{3.163191in}}{\pgfqpoint{1.685235in}{3.171427in}}%
\pgfpathcurveto{\pgfqpoint{1.685235in}{3.179663in}}{\pgfqpoint{1.681963in}{3.187563in}}{\pgfqpoint{1.676139in}{3.193387in}}%
\pgfpathcurveto{\pgfqpoint{1.670315in}{3.199211in}}{\pgfqpoint{1.662415in}{3.202484in}}{\pgfqpoint{1.654179in}{3.202484in}}%
\pgfpathcurveto{\pgfqpoint{1.645943in}{3.202484in}}{\pgfqpoint{1.638043in}{3.199211in}}{\pgfqpoint{1.632219in}{3.193387in}}%
\pgfpathcurveto{\pgfqpoint{1.626395in}{3.187563in}}{\pgfqpoint{1.623122in}{3.179663in}}{\pgfqpoint{1.623122in}{3.171427in}}%
\pgfpathcurveto{\pgfqpoint{1.623122in}{3.163191in}}{\pgfqpoint{1.626395in}{3.155291in}}{\pgfqpoint{1.632219in}{3.149467in}}%
\pgfpathcurveto{\pgfqpoint{1.638043in}{3.143643in}}{\pgfqpoint{1.645943in}{3.140371in}}{\pgfqpoint{1.654179in}{3.140371in}}%
\pgfpathclose%
\pgfusepath{stroke,fill}%
\end{pgfscope}%
\begin{pgfscope}%
\pgfpathrectangle{\pgfqpoint{0.100000in}{0.212622in}}{\pgfqpoint{3.696000in}{3.696000in}}%
\pgfusepath{clip}%
\pgfsetbuttcap%
\pgfsetroundjoin%
\definecolor{currentfill}{rgb}{0.121569,0.466667,0.705882}%
\pgfsetfillcolor{currentfill}%
\pgfsetfillopacity{0.350541}%
\pgfsetlinewidth{1.003750pt}%
\definecolor{currentstroke}{rgb}{0.121569,0.466667,0.705882}%
\pgfsetstrokecolor{currentstroke}%
\pgfsetstrokeopacity{0.350541}%
\pgfsetdash{}{0pt}%
\pgfpathmoveto{\pgfqpoint{1.651361in}{3.134046in}}%
\pgfpathcurveto{\pgfqpoint{1.659597in}{3.134046in}}{\pgfqpoint{1.667497in}{3.137319in}}{\pgfqpoint{1.673321in}{3.143143in}}%
\pgfpathcurveto{\pgfqpoint{1.679145in}{3.148966in}}{\pgfqpoint{1.682417in}{3.156867in}}{\pgfqpoint{1.682417in}{3.165103in}}%
\pgfpathcurveto{\pgfqpoint{1.682417in}{3.173339in}}{\pgfqpoint{1.679145in}{3.181239in}}{\pgfqpoint{1.673321in}{3.187063in}}%
\pgfpathcurveto{\pgfqpoint{1.667497in}{3.192887in}}{\pgfqpoint{1.659597in}{3.196159in}}{\pgfqpoint{1.651361in}{3.196159in}}%
\pgfpathcurveto{\pgfqpoint{1.643124in}{3.196159in}}{\pgfqpoint{1.635224in}{3.192887in}}{\pgfqpoint{1.629400in}{3.187063in}}%
\pgfpathcurveto{\pgfqpoint{1.623576in}{3.181239in}}{\pgfqpoint{1.620304in}{3.173339in}}{\pgfqpoint{1.620304in}{3.165103in}}%
\pgfpathcurveto{\pgfqpoint{1.620304in}{3.156867in}}{\pgfqpoint{1.623576in}{3.148966in}}{\pgfqpoint{1.629400in}{3.143143in}}%
\pgfpathcurveto{\pgfqpoint{1.635224in}{3.137319in}}{\pgfqpoint{1.643124in}{3.134046in}}{\pgfqpoint{1.651361in}{3.134046in}}%
\pgfpathclose%
\pgfusepath{stroke,fill}%
\end{pgfscope}%
\begin{pgfscope}%
\pgfpathrectangle{\pgfqpoint{0.100000in}{0.212622in}}{\pgfqpoint{3.696000in}{3.696000in}}%
\pgfusepath{clip}%
\pgfsetbuttcap%
\pgfsetroundjoin%
\definecolor{currentfill}{rgb}{0.121569,0.466667,0.705882}%
\pgfsetfillcolor{currentfill}%
\pgfsetfillopacity{0.351490}%
\pgfsetlinewidth{1.003750pt}%
\definecolor{currentstroke}{rgb}{0.121569,0.466667,0.705882}%
\pgfsetstrokecolor{currentstroke}%
\pgfsetstrokeopacity{0.351490}%
\pgfsetdash{}{0pt}%
\pgfpathmoveto{\pgfqpoint{1.649458in}{3.129754in}}%
\pgfpathcurveto{\pgfqpoint{1.657694in}{3.129754in}}{\pgfqpoint{1.665594in}{3.133026in}}{\pgfqpoint{1.671418in}{3.138850in}}%
\pgfpathcurveto{\pgfqpoint{1.677242in}{3.144674in}}{\pgfqpoint{1.680514in}{3.152574in}}{\pgfqpoint{1.680514in}{3.160810in}}%
\pgfpathcurveto{\pgfqpoint{1.680514in}{3.169047in}}{\pgfqpoint{1.677242in}{3.176947in}}{\pgfqpoint{1.671418in}{3.182770in}}%
\pgfpathcurveto{\pgfqpoint{1.665594in}{3.188594in}}{\pgfqpoint{1.657694in}{3.191867in}}{\pgfqpoint{1.649458in}{3.191867in}}%
\pgfpathcurveto{\pgfqpoint{1.641222in}{3.191867in}}{\pgfqpoint{1.633322in}{3.188594in}}{\pgfqpoint{1.627498in}{3.182770in}}%
\pgfpathcurveto{\pgfqpoint{1.621674in}{3.176947in}}{\pgfqpoint{1.618401in}{3.169047in}}{\pgfqpoint{1.618401in}{3.160810in}}%
\pgfpathcurveto{\pgfqpoint{1.618401in}{3.152574in}}{\pgfqpoint{1.621674in}{3.144674in}}{\pgfqpoint{1.627498in}{3.138850in}}%
\pgfpathcurveto{\pgfqpoint{1.633322in}{3.133026in}}{\pgfqpoint{1.641222in}{3.129754in}}{\pgfqpoint{1.649458in}{3.129754in}}%
\pgfpathclose%
\pgfusepath{stroke,fill}%
\end{pgfscope}%
\begin{pgfscope}%
\pgfpathrectangle{\pgfqpoint{0.100000in}{0.212622in}}{\pgfqpoint{3.696000in}{3.696000in}}%
\pgfusepath{clip}%
\pgfsetbuttcap%
\pgfsetroundjoin%
\definecolor{currentfill}{rgb}{0.121569,0.466667,0.705882}%
\pgfsetfillcolor{currentfill}%
\pgfsetfillopacity{0.351779}%
\pgfsetlinewidth{1.003750pt}%
\definecolor{currentstroke}{rgb}{0.121569,0.466667,0.705882}%
\pgfsetstrokecolor{currentstroke}%
\pgfsetstrokeopacity{0.351779}%
\pgfsetdash{}{0pt}%
\pgfpathmoveto{\pgfqpoint{1.648840in}{3.128392in}}%
\pgfpathcurveto{\pgfqpoint{1.657077in}{3.128392in}}{\pgfqpoint{1.664977in}{3.131664in}}{\pgfqpoint{1.670801in}{3.137488in}}%
\pgfpathcurveto{\pgfqpoint{1.676625in}{3.143312in}}{\pgfqpoint{1.679897in}{3.151212in}}{\pgfqpoint{1.679897in}{3.159448in}}%
\pgfpathcurveto{\pgfqpoint{1.679897in}{3.167685in}}{\pgfqpoint{1.676625in}{3.175585in}}{\pgfqpoint{1.670801in}{3.181409in}}%
\pgfpathcurveto{\pgfqpoint{1.664977in}{3.187232in}}{\pgfqpoint{1.657077in}{3.190505in}}{\pgfqpoint{1.648840in}{3.190505in}}%
\pgfpathcurveto{\pgfqpoint{1.640604in}{3.190505in}}{\pgfqpoint{1.632704in}{3.187232in}}{\pgfqpoint{1.626880in}{3.181409in}}%
\pgfpathcurveto{\pgfqpoint{1.621056in}{3.175585in}}{\pgfqpoint{1.617784in}{3.167685in}}{\pgfqpoint{1.617784in}{3.159448in}}%
\pgfpathcurveto{\pgfqpoint{1.617784in}{3.151212in}}{\pgfqpoint{1.621056in}{3.143312in}}{\pgfqpoint{1.626880in}{3.137488in}}%
\pgfpathcurveto{\pgfqpoint{1.632704in}{3.131664in}}{\pgfqpoint{1.640604in}{3.128392in}}{\pgfqpoint{1.648840in}{3.128392in}}%
\pgfpathclose%
\pgfusepath{stroke,fill}%
\end{pgfscope}%
\begin{pgfscope}%
\pgfpathrectangle{\pgfqpoint{0.100000in}{0.212622in}}{\pgfqpoint{3.696000in}{3.696000in}}%
\pgfusepath{clip}%
\pgfsetbuttcap%
\pgfsetroundjoin%
\definecolor{currentfill}{rgb}{0.121569,0.466667,0.705882}%
\pgfsetfillcolor{currentfill}%
\pgfsetfillopacity{0.352175}%
\pgfsetlinewidth{1.003750pt}%
\definecolor{currentstroke}{rgb}{0.121569,0.466667,0.705882}%
\pgfsetstrokecolor{currentstroke}%
\pgfsetstrokeopacity{0.352175}%
\pgfsetdash{}{0pt}%
\pgfpathmoveto{\pgfqpoint{1.851093in}{3.190661in}}%
\pgfpathcurveto{\pgfqpoint{1.859329in}{3.190661in}}{\pgfqpoint{1.867229in}{3.193933in}}{\pgfqpoint{1.873053in}{3.199757in}}%
\pgfpathcurveto{\pgfqpoint{1.878877in}{3.205581in}}{\pgfqpoint{1.882149in}{3.213481in}}{\pgfqpoint{1.882149in}{3.221717in}}%
\pgfpathcurveto{\pgfqpoint{1.882149in}{3.229953in}}{\pgfqpoint{1.878877in}{3.237853in}}{\pgfqpoint{1.873053in}{3.243677in}}%
\pgfpathcurveto{\pgfqpoint{1.867229in}{3.249501in}}{\pgfqpoint{1.859329in}{3.252774in}}{\pgfqpoint{1.851093in}{3.252774in}}%
\pgfpathcurveto{\pgfqpoint{1.842856in}{3.252774in}}{\pgfqpoint{1.834956in}{3.249501in}}{\pgfqpoint{1.829132in}{3.243677in}}%
\pgfpathcurveto{\pgfqpoint{1.823309in}{3.237853in}}{\pgfqpoint{1.820036in}{3.229953in}}{\pgfqpoint{1.820036in}{3.221717in}}%
\pgfpathcurveto{\pgfqpoint{1.820036in}{3.213481in}}{\pgfqpoint{1.823309in}{3.205581in}}{\pgfqpoint{1.829132in}{3.199757in}}%
\pgfpathcurveto{\pgfqpoint{1.834956in}{3.193933in}}{\pgfqpoint{1.842856in}{3.190661in}}{\pgfqpoint{1.851093in}{3.190661in}}%
\pgfpathclose%
\pgfusepath{stroke,fill}%
\end{pgfscope}%
\begin{pgfscope}%
\pgfpathrectangle{\pgfqpoint{0.100000in}{0.212622in}}{\pgfqpoint{3.696000in}{3.696000in}}%
\pgfusepath{clip}%
\pgfsetbuttcap%
\pgfsetroundjoin%
\definecolor{currentfill}{rgb}{0.121569,0.466667,0.705882}%
\pgfsetfillcolor{currentfill}%
\pgfsetfillopacity{0.352343}%
\pgfsetlinewidth{1.003750pt}%
\definecolor{currentstroke}{rgb}{0.121569,0.466667,0.705882}%
\pgfsetstrokecolor{currentstroke}%
\pgfsetstrokeopacity{0.352343}%
\pgfsetdash{}{0pt}%
\pgfpathmoveto{\pgfqpoint{1.647826in}{3.125965in}}%
\pgfpathcurveto{\pgfqpoint{1.656062in}{3.125965in}}{\pgfqpoint{1.663962in}{3.129237in}}{\pgfqpoint{1.669786in}{3.135061in}}%
\pgfpathcurveto{\pgfqpoint{1.675610in}{3.140885in}}{\pgfqpoint{1.678882in}{3.148785in}}{\pgfqpoint{1.678882in}{3.157022in}}%
\pgfpathcurveto{\pgfqpoint{1.678882in}{3.165258in}}{\pgfqpoint{1.675610in}{3.173158in}}{\pgfqpoint{1.669786in}{3.178982in}}%
\pgfpathcurveto{\pgfqpoint{1.663962in}{3.184806in}}{\pgfqpoint{1.656062in}{3.188078in}}{\pgfqpoint{1.647826in}{3.188078in}}%
\pgfpathcurveto{\pgfqpoint{1.639590in}{3.188078in}}{\pgfqpoint{1.631690in}{3.184806in}}{\pgfqpoint{1.625866in}{3.178982in}}%
\pgfpathcurveto{\pgfqpoint{1.620042in}{3.173158in}}{\pgfqpoint{1.616769in}{3.165258in}}{\pgfqpoint{1.616769in}{3.157022in}}%
\pgfpathcurveto{\pgfqpoint{1.616769in}{3.148785in}}{\pgfqpoint{1.620042in}{3.140885in}}{\pgfqpoint{1.625866in}{3.135061in}}%
\pgfpathcurveto{\pgfqpoint{1.631690in}{3.129237in}}{\pgfqpoint{1.639590in}{3.125965in}}{\pgfqpoint{1.647826in}{3.125965in}}%
\pgfpathclose%
\pgfusepath{stroke,fill}%
\end{pgfscope}%
\begin{pgfscope}%
\pgfpathrectangle{\pgfqpoint{0.100000in}{0.212622in}}{\pgfqpoint{3.696000in}{3.696000in}}%
\pgfusepath{clip}%
\pgfsetbuttcap%
\pgfsetroundjoin%
\definecolor{currentfill}{rgb}{0.121569,0.466667,0.705882}%
\pgfsetfillcolor{currentfill}%
\pgfsetfillopacity{0.353336}%
\pgfsetlinewidth{1.003750pt}%
\definecolor{currentstroke}{rgb}{0.121569,0.466667,0.705882}%
\pgfsetstrokecolor{currentstroke}%
\pgfsetstrokeopacity{0.353336}%
\pgfsetdash{}{0pt}%
\pgfpathmoveto{\pgfqpoint{1.645986in}{3.121411in}}%
\pgfpathcurveto{\pgfqpoint{1.654222in}{3.121411in}}{\pgfqpoint{1.662122in}{3.124684in}}{\pgfqpoint{1.667946in}{3.130508in}}%
\pgfpathcurveto{\pgfqpoint{1.673770in}{3.136332in}}{\pgfqpoint{1.677042in}{3.144232in}}{\pgfqpoint{1.677042in}{3.152468in}}%
\pgfpathcurveto{\pgfqpoint{1.677042in}{3.160704in}}{\pgfqpoint{1.673770in}{3.168604in}}{\pgfqpoint{1.667946in}{3.174428in}}%
\pgfpathcurveto{\pgfqpoint{1.662122in}{3.180252in}}{\pgfqpoint{1.654222in}{3.183524in}}{\pgfqpoint{1.645986in}{3.183524in}}%
\pgfpathcurveto{\pgfqpoint{1.637749in}{3.183524in}}{\pgfqpoint{1.629849in}{3.180252in}}{\pgfqpoint{1.624025in}{3.174428in}}%
\pgfpathcurveto{\pgfqpoint{1.618201in}{3.168604in}}{\pgfqpoint{1.614929in}{3.160704in}}{\pgfqpoint{1.614929in}{3.152468in}}%
\pgfpathcurveto{\pgfqpoint{1.614929in}{3.144232in}}{\pgfqpoint{1.618201in}{3.136332in}}{\pgfqpoint{1.624025in}{3.130508in}}%
\pgfpathcurveto{\pgfqpoint{1.629849in}{3.124684in}}{\pgfqpoint{1.637749in}{3.121411in}}{\pgfqpoint{1.645986in}{3.121411in}}%
\pgfpathclose%
\pgfusepath{stroke,fill}%
\end{pgfscope}%
\begin{pgfscope}%
\pgfpathrectangle{\pgfqpoint{0.100000in}{0.212622in}}{\pgfqpoint{3.696000in}{3.696000in}}%
\pgfusepath{clip}%
\pgfsetbuttcap%
\pgfsetroundjoin%
\definecolor{currentfill}{rgb}{0.121569,0.466667,0.705882}%
\pgfsetfillcolor{currentfill}%
\pgfsetfillopacity{0.353894}%
\pgfsetlinewidth{1.003750pt}%
\definecolor{currentstroke}{rgb}{0.121569,0.466667,0.705882}%
\pgfsetstrokecolor{currentstroke}%
\pgfsetstrokeopacity{0.353894}%
\pgfsetdash{}{0pt}%
\pgfpathmoveto{\pgfqpoint{1.644866in}{3.118843in}}%
\pgfpathcurveto{\pgfqpoint{1.653103in}{3.118843in}}{\pgfqpoint{1.661003in}{3.122115in}}{\pgfqpoint{1.666827in}{3.127939in}}%
\pgfpathcurveto{\pgfqpoint{1.672651in}{3.133763in}}{\pgfqpoint{1.675923in}{3.141663in}}{\pgfqpoint{1.675923in}{3.149900in}}%
\pgfpathcurveto{\pgfqpoint{1.675923in}{3.158136in}}{\pgfqpoint{1.672651in}{3.166036in}}{\pgfqpoint{1.666827in}{3.171860in}}%
\pgfpathcurveto{\pgfqpoint{1.661003in}{3.177684in}}{\pgfqpoint{1.653103in}{3.180956in}}{\pgfqpoint{1.644866in}{3.180956in}}%
\pgfpathcurveto{\pgfqpoint{1.636630in}{3.180956in}}{\pgfqpoint{1.628730in}{3.177684in}}{\pgfqpoint{1.622906in}{3.171860in}}%
\pgfpathcurveto{\pgfqpoint{1.617082in}{3.166036in}}{\pgfqpoint{1.613810in}{3.158136in}}{\pgfqpoint{1.613810in}{3.149900in}}%
\pgfpathcurveto{\pgfqpoint{1.613810in}{3.141663in}}{\pgfqpoint{1.617082in}{3.133763in}}{\pgfqpoint{1.622906in}{3.127939in}}%
\pgfpathcurveto{\pgfqpoint{1.628730in}{3.122115in}}{\pgfqpoint{1.636630in}{3.118843in}}{\pgfqpoint{1.644866in}{3.118843in}}%
\pgfpathclose%
\pgfusepath{stroke,fill}%
\end{pgfscope}%
\begin{pgfscope}%
\pgfpathrectangle{\pgfqpoint{0.100000in}{0.212622in}}{\pgfqpoint{3.696000in}{3.696000in}}%
\pgfusepath{clip}%
\pgfsetbuttcap%
\pgfsetroundjoin%
\definecolor{currentfill}{rgb}{0.121569,0.466667,0.705882}%
\pgfsetfillcolor{currentfill}%
\pgfsetfillopacity{0.354892}%
\pgfsetlinewidth{1.003750pt}%
\definecolor{currentstroke}{rgb}{0.121569,0.466667,0.705882}%
\pgfsetstrokecolor{currentstroke}%
\pgfsetstrokeopacity{0.354892}%
\pgfsetdash{}{0pt}%
\pgfpathmoveto{\pgfqpoint{1.642749in}{3.114186in}}%
\pgfpathcurveto{\pgfqpoint{1.650985in}{3.114186in}}{\pgfqpoint{1.658885in}{3.117458in}}{\pgfqpoint{1.664709in}{3.123282in}}%
\pgfpathcurveto{\pgfqpoint{1.670533in}{3.129106in}}{\pgfqpoint{1.673805in}{3.137006in}}{\pgfqpoint{1.673805in}{3.145242in}}%
\pgfpathcurveto{\pgfqpoint{1.673805in}{3.153479in}}{\pgfqpoint{1.670533in}{3.161379in}}{\pgfqpoint{1.664709in}{3.167203in}}%
\pgfpathcurveto{\pgfqpoint{1.658885in}{3.173027in}}{\pgfqpoint{1.650985in}{3.176299in}}{\pgfqpoint{1.642749in}{3.176299in}}%
\pgfpathcurveto{\pgfqpoint{1.634512in}{3.176299in}}{\pgfqpoint{1.626612in}{3.173027in}}{\pgfqpoint{1.620788in}{3.167203in}}%
\pgfpathcurveto{\pgfqpoint{1.614964in}{3.161379in}}{\pgfqpoint{1.611692in}{3.153479in}}{\pgfqpoint{1.611692in}{3.145242in}}%
\pgfpathcurveto{\pgfqpoint{1.611692in}{3.137006in}}{\pgfqpoint{1.614964in}{3.129106in}}{\pgfqpoint{1.620788in}{3.123282in}}%
\pgfpathcurveto{\pgfqpoint{1.626612in}{3.117458in}}{\pgfqpoint{1.634512in}{3.114186in}}{\pgfqpoint{1.642749in}{3.114186in}}%
\pgfpathclose%
\pgfusepath{stroke,fill}%
\end{pgfscope}%
\begin{pgfscope}%
\pgfpathrectangle{\pgfqpoint{0.100000in}{0.212622in}}{\pgfqpoint{3.696000in}{3.696000in}}%
\pgfusepath{clip}%
\pgfsetbuttcap%
\pgfsetroundjoin%
\definecolor{currentfill}{rgb}{0.121569,0.466667,0.705882}%
\pgfsetfillcolor{currentfill}%
\pgfsetfillopacity{0.356657}%
\pgfsetlinewidth{1.003750pt}%
\definecolor{currentstroke}{rgb}{0.121569,0.466667,0.705882}%
\pgfsetstrokecolor{currentstroke}%
\pgfsetstrokeopacity{0.356657}%
\pgfsetdash{}{0pt}%
\pgfpathmoveto{\pgfqpoint{1.638722in}{3.105703in}}%
\pgfpathcurveto{\pgfqpoint{1.646958in}{3.105703in}}{\pgfqpoint{1.654859in}{3.108975in}}{\pgfqpoint{1.660682in}{3.114799in}}%
\pgfpathcurveto{\pgfqpoint{1.666506in}{3.120623in}}{\pgfqpoint{1.669779in}{3.128523in}}{\pgfqpoint{1.669779in}{3.136759in}}%
\pgfpathcurveto{\pgfqpoint{1.669779in}{3.144995in}}{\pgfqpoint{1.666506in}{3.152895in}}{\pgfqpoint{1.660682in}{3.158719in}}%
\pgfpathcurveto{\pgfqpoint{1.654859in}{3.164543in}}{\pgfqpoint{1.646958in}{3.167816in}}{\pgfqpoint{1.638722in}{3.167816in}}%
\pgfpathcurveto{\pgfqpoint{1.630486in}{3.167816in}}{\pgfqpoint{1.622586in}{3.164543in}}{\pgfqpoint{1.616762in}{3.158719in}}%
\pgfpathcurveto{\pgfqpoint{1.610938in}{3.152895in}}{\pgfqpoint{1.607666in}{3.144995in}}{\pgfqpoint{1.607666in}{3.136759in}}%
\pgfpathcurveto{\pgfqpoint{1.607666in}{3.128523in}}{\pgfqpoint{1.610938in}{3.120623in}}{\pgfqpoint{1.616762in}{3.114799in}}%
\pgfpathcurveto{\pgfqpoint{1.622586in}{3.108975in}}{\pgfqpoint{1.630486in}{3.105703in}}{\pgfqpoint{1.638722in}{3.105703in}}%
\pgfpathclose%
\pgfusepath{stroke,fill}%
\end{pgfscope}%
\begin{pgfscope}%
\pgfpathrectangle{\pgfqpoint{0.100000in}{0.212622in}}{\pgfqpoint{3.696000in}{3.696000in}}%
\pgfusepath{clip}%
\pgfsetbuttcap%
\pgfsetroundjoin%
\definecolor{currentfill}{rgb}{0.121569,0.466667,0.705882}%
\pgfsetfillcolor{currentfill}%
\pgfsetfillopacity{0.356674}%
\pgfsetlinewidth{1.003750pt}%
\definecolor{currentstroke}{rgb}{0.121569,0.466667,0.705882}%
\pgfsetstrokecolor{currentstroke}%
\pgfsetstrokeopacity{0.356674}%
\pgfsetdash{}{0pt}%
\pgfpathmoveto{\pgfqpoint{1.638686in}{3.105626in}}%
\pgfpathcurveto{\pgfqpoint{1.646922in}{3.105626in}}{\pgfqpoint{1.654822in}{3.108899in}}{\pgfqpoint{1.660646in}{3.114723in}}%
\pgfpathcurveto{\pgfqpoint{1.666470in}{3.120547in}}{\pgfqpoint{1.669742in}{3.128447in}}{\pgfqpoint{1.669742in}{3.136683in}}%
\pgfpathcurveto{\pgfqpoint{1.669742in}{3.144919in}}{\pgfqpoint{1.666470in}{3.152819in}}{\pgfqpoint{1.660646in}{3.158643in}}%
\pgfpathcurveto{\pgfqpoint{1.654822in}{3.164467in}}{\pgfqpoint{1.646922in}{3.167739in}}{\pgfqpoint{1.638686in}{3.167739in}}%
\pgfpathcurveto{\pgfqpoint{1.630449in}{3.167739in}}{\pgfqpoint{1.622549in}{3.164467in}}{\pgfqpoint{1.616725in}{3.158643in}}%
\pgfpathcurveto{\pgfqpoint{1.610901in}{3.152819in}}{\pgfqpoint{1.607629in}{3.144919in}}{\pgfqpoint{1.607629in}{3.136683in}}%
\pgfpathcurveto{\pgfqpoint{1.607629in}{3.128447in}}{\pgfqpoint{1.610901in}{3.120547in}}{\pgfqpoint{1.616725in}{3.114723in}}%
\pgfpathcurveto{\pgfqpoint{1.622549in}{3.108899in}}{\pgfqpoint{1.630449in}{3.105626in}}{\pgfqpoint{1.638686in}{3.105626in}}%
\pgfpathclose%
\pgfusepath{stroke,fill}%
\end{pgfscope}%
\begin{pgfscope}%
\pgfpathrectangle{\pgfqpoint{0.100000in}{0.212622in}}{\pgfqpoint{3.696000in}{3.696000in}}%
\pgfusepath{clip}%
\pgfsetbuttcap%
\pgfsetroundjoin%
\definecolor{currentfill}{rgb}{0.121569,0.466667,0.705882}%
\pgfsetfillcolor{currentfill}%
\pgfsetfillopacity{0.356705}%
\pgfsetlinewidth{1.003750pt}%
\definecolor{currentstroke}{rgb}{0.121569,0.466667,0.705882}%
\pgfsetstrokecolor{currentstroke}%
\pgfsetstrokeopacity{0.356705}%
\pgfsetdash{}{0pt}%
\pgfpathmoveto{\pgfqpoint{1.638620in}{3.105486in}}%
\pgfpathcurveto{\pgfqpoint{1.646857in}{3.105486in}}{\pgfqpoint{1.654757in}{3.108759in}}{\pgfqpoint{1.660581in}{3.114583in}}%
\pgfpathcurveto{\pgfqpoint{1.666405in}{3.120407in}}{\pgfqpoint{1.669677in}{3.128307in}}{\pgfqpoint{1.669677in}{3.136543in}}%
\pgfpathcurveto{\pgfqpoint{1.669677in}{3.144779in}}{\pgfqpoint{1.666405in}{3.152679in}}{\pgfqpoint{1.660581in}{3.158503in}}%
\pgfpathcurveto{\pgfqpoint{1.654757in}{3.164327in}}{\pgfqpoint{1.646857in}{3.167599in}}{\pgfqpoint{1.638620in}{3.167599in}}%
\pgfpathcurveto{\pgfqpoint{1.630384in}{3.167599in}}{\pgfqpoint{1.622484in}{3.164327in}}{\pgfqpoint{1.616660in}{3.158503in}}%
\pgfpathcurveto{\pgfqpoint{1.610836in}{3.152679in}}{\pgfqpoint{1.607564in}{3.144779in}}{\pgfqpoint{1.607564in}{3.136543in}}%
\pgfpathcurveto{\pgfqpoint{1.607564in}{3.128307in}}{\pgfqpoint{1.610836in}{3.120407in}}{\pgfqpoint{1.616660in}{3.114583in}}%
\pgfpathcurveto{\pgfqpoint{1.622484in}{3.108759in}}{\pgfqpoint{1.630384in}{3.105486in}}{\pgfqpoint{1.638620in}{3.105486in}}%
\pgfpathclose%
\pgfusepath{stroke,fill}%
\end{pgfscope}%
\begin{pgfscope}%
\pgfpathrectangle{\pgfqpoint{0.100000in}{0.212622in}}{\pgfqpoint{3.696000in}{3.696000in}}%
\pgfusepath{clip}%
\pgfsetbuttcap%
\pgfsetroundjoin%
\definecolor{currentfill}{rgb}{0.121569,0.466667,0.705882}%
\pgfsetfillcolor{currentfill}%
\pgfsetfillopacity{0.356761}%
\pgfsetlinewidth{1.003750pt}%
\definecolor{currentstroke}{rgb}{0.121569,0.466667,0.705882}%
\pgfsetstrokecolor{currentstroke}%
\pgfsetstrokeopacity{0.356761}%
\pgfsetdash{}{0pt}%
\pgfpathmoveto{\pgfqpoint{1.638502in}{3.105230in}}%
\pgfpathcurveto{\pgfqpoint{1.646738in}{3.105230in}}{\pgfqpoint{1.654639in}{3.108502in}}{\pgfqpoint{1.660462in}{3.114326in}}%
\pgfpathcurveto{\pgfqpoint{1.666286in}{3.120150in}}{\pgfqpoint{1.669559in}{3.128050in}}{\pgfqpoint{1.669559in}{3.136286in}}%
\pgfpathcurveto{\pgfqpoint{1.669559in}{3.144522in}}{\pgfqpoint{1.666286in}{3.152422in}}{\pgfqpoint{1.660462in}{3.158246in}}%
\pgfpathcurveto{\pgfqpoint{1.654639in}{3.164070in}}{\pgfqpoint{1.646738in}{3.167343in}}{\pgfqpoint{1.638502in}{3.167343in}}%
\pgfpathcurveto{\pgfqpoint{1.630266in}{3.167343in}}{\pgfqpoint{1.622366in}{3.164070in}}{\pgfqpoint{1.616542in}{3.158246in}}%
\pgfpathcurveto{\pgfqpoint{1.610718in}{3.152422in}}{\pgfqpoint{1.607446in}{3.144522in}}{\pgfqpoint{1.607446in}{3.136286in}}%
\pgfpathcurveto{\pgfqpoint{1.607446in}{3.128050in}}{\pgfqpoint{1.610718in}{3.120150in}}{\pgfqpoint{1.616542in}{3.114326in}}%
\pgfpathcurveto{\pgfqpoint{1.622366in}{3.108502in}}{\pgfqpoint{1.630266in}{3.105230in}}{\pgfqpoint{1.638502in}{3.105230in}}%
\pgfpathclose%
\pgfusepath{stroke,fill}%
\end{pgfscope}%
\begin{pgfscope}%
\pgfpathrectangle{\pgfqpoint{0.100000in}{0.212622in}}{\pgfqpoint{3.696000in}{3.696000in}}%
\pgfusepath{clip}%
\pgfsetbuttcap%
\pgfsetroundjoin%
\definecolor{currentfill}{rgb}{0.121569,0.466667,0.705882}%
\pgfsetfillcolor{currentfill}%
\pgfsetfillopacity{0.356859}%
\pgfsetlinewidth{1.003750pt}%
\definecolor{currentstroke}{rgb}{0.121569,0.466667,0.705882}%
\pgfsetstrokecolor{currentstroke}%
\pgfsetstrokeopacity{0.356859}%
\pgfsetdash{}{0pt}%
\pgfpathmoveto{\pgfqpoint{1.638281in}{3.104754in}}%
\pgfpathcurveto{\pgfqpoint{1.646517in}{3.104754in}}{\pgfqpoint{1.654417in}{3.108027in}}{\pgfqpoint{1.660241in}{3.113851in}}%
\pgfpathcurveto{\pgfqpoint{1.666065in}{3.119675in}}{\pgfqpoint{1.669337in}{3.127575in}}{\pgfqpoint{1.669337in}{3.135811in}}%
\pgfpathcurveto{\pgfqpoint{1.669337in}{3.144047in}}{\pgfqpoint{1.666065in}{3.151947in}}{\pgfqpoint{1.660241in}{3.157771in}}%
\pgfpathcurveto{\pgfqpoint{1.654417in}{3.163595in}}{\pgfqpoint{1.646517in}{3.166867in}}{\pgfqpoint{1.638281in}{3.166867in}}%
\pgfpathcurveto{\pgfqpoint{1.630045in}{3.166867in}}{\pgfqpoint{1.622145in}{3.163595in}}{\pgfqpoint{1.616321in}{3.157771in}}%
\pgfpathcurveto{\pgfqpoint{1.610497in}{3.151947in}}{\pgfqpoint{1.607224in}{3.144047in}}{\pgfqpoint{1.607224in}{3.135811in}}%
\pgfpathcurveto{\pgfqpoint{1.607224in}{3.127575in}}{\pgfqpoint{1.610497in}{3.119675in}}{\pgfqpoint{1.616321in}{3.113851in}}%
\pgfpathcurveto{\pgfqpoint{1.622145in}{3.108027in}}{\pgfqpoint{1.630045in}{3.104754in}}{\pgfqpoint{1.638281in}{3.104754in}}%
\pgfpathclose%
\pgfusepath{stroke,fill}%
\end{pgfscope}%
\begin{pgfscope}%
\pgfpathrectangle{\pgfqpoint{0.100000in}{0.212622in}}{\pgfqpoint{3.696000in}{3.696000in}}%
\pgfusepath{clip}%
\pgfsetbuttcap%
\pgfsetroundjoin%
\definecolor{currentfill}{rgb}{0.121569,0.466667,0.705882}%
\pgfsetfillcolor{currentfill}%
\pgfsetfillopacity{0.357050}%
\pgfsetlinewidth{1.003750pt}%
\definecolor{currentstroke}{rgb}{0.121569,0.466667,0.705882}%
\pgfsetstrokecolor{currentstroke}%
\pgfsetstrokeopacity{0.357050}%
\pgfsetdash{}{0pt}%
\pgfpathmoveto{\pgfqpoint{1.637908in}{3.103913in}}%
\pgfpathcurveto{\pgfqpoint{1.646144in}{3.103913in}}{\pgfqpoint{1.654044in}{3.107185in}}{\pgfqpoint{1.659868in}{3.113009in}}%
\pgfpathcurveto{\pgfqpoint{1.665692in}{3.118833in}}{\pgfqpoint{1.668964in}{3.126733in}}{\pgfqpoint{1.668964in}{3.134969in}}%
\pgfpathcurveto{\pgfqpoint{1.668964in}{3.143205in}}{\pgfqpoint{1.665692in}{3.151105in}}{\pgfqpoint{1.659868in}{3.156929in}}%
\pgfpathcurveto{\pgfqpoint{1.654044in}{3.162753in}}{\pgfqpoint{1.646144in}{3.166026in}}{\pgfqpoint{1.637908in}{3.166026in}}%
\pgfpathcurveto{\pgfqpoint{1.629672in}{3.166026in}}{\pgfqpoint{1.621772in}{3.162753in}}{\pgfqpoint{1.615948in}{3.156929in}}%
\pgfpathcurveto{\pgfqpoint{1.610124in}{3.151105in}}{\pgfqpoint{1.606851in}{3.143205in}}{\pgfqpoint{1.606851in}{3.134969in}}%
\pgfpathcurveto{\pgfqpoint{1.606851in}{3.126733in}}{\pgfqpoint{1.610124in}{3.118833in}}{\pgfqpoint{1.615948in}{3.113009in}}%
\pgfpathcurveto{\pgfqpoint{1.621772in}{3.107185in}}{\pgfqpoint{1.629672in}{3.103913in}}{\pgfqpoint{1.637908in}{3.103913in}}%
\pgfpathclose%
\pgfusepath{stroke,fill}%
\end{pgfscope}%
\begin{pgfscope}%
\pgfpathrectangle{\pgfqpoint{0.100000in}{0.212622in}}{\pgfqpoint{3.696000in}{3.696000in}}%
\pgfusepath{clip}%
\pgfsetbuttcap%
\pgfsetroundjoin%
\definecolor{currentfill}{rgb}{0.121569,0.466667,0.705882}%
\pgfsetfillcolor{currentfill}%
\pgfsetfillopacity{0.357391}%
\pgfsetlinewidth{1.003750pt}%
\definecolor{currentstroke}{rgb}{0.121569,0.466667,0.705882}%
\pgfsetstrokecolor{currentstroke}%
\pgfsetstrokeopacity{0.357391}%
\pgfsetdash{}{0pt}%
\pgfpathmoveto{\pgfqpoint{1.637228in}{3.102360in}}%
\pgfpathcurveto{\pgfqpoint{1.645464in}{3.102360in}}{\pgfqpoint{1.653364in}{3.105633in}}{\pgfqpoint{1.659188in}{3.111456in}}%
\pgfpathcurveto{\pgfqpoint{1.665012in}{3.117280in}}{\pgfqpoint{1.668284in}{3.125180in}}{\pgfqpoint{1.668284in}{3.133417in}}%
\pgfpathcurveto{\pgfqpoint{1.668284in}{3.141653in}}{\pgfqpoint{1.665012in}{3.149553in}}{\pgfqpoint{1.659188in}{3.155377in}}%
\pgfpathcurveto{\pgfqpoint{1.653364in}{3.161201in}}{\pgfqpoint{1.645464in}{3.164473in}}{\pgfqpoint{1.637228in}{3.164473in}}%
\pgfpathcurveto{\pgfqpoint{1.628992in}{3.164473in}}{\pgfqpoint{1.621092in}{3.161201in}}{\pgfqpoint{1.615268in}{3.155377in}}%
\pgfpathcurveto{\pgfqpoint{1.609444in}{3.149553in}}{\pgfqpoint{1.606172in}{3.141653in}}{\pgfqpoint{1.606172in}{3.133417in}}%
\pgfpathcurveto{\pgfqpoint{1.606172in}{3.125180in}}{\pgfqpoint{1.609444in}{3.117280in}}{\pgfqpoint{1.615268in}{3.111456in}}%
\pgfpathcurveto{\pgfqpoint{1.621092in}{3.105633in}}{\pgfqpoint{1.628992in}{3.102360in}}{\pgfqpoint{1.637228in}{3.102360in}}%
\pgfpathclose%
\pgfusepath{stroke,fill}%
\end{pgfscope}%
\begin{pgfscope}%
\pgfpathrectangle{\pgfqpoint{0.100000in}{0.212622in}}{\pgfqpoint{3.696000in}{3.696000in}}%
\pgfusepath{clip}%
\pgfsetbuttcap%
\pgfsetroundjoin%
\definecolor{currentfill}{rgb}{0.121569,0.466667,0.705882}%
\pgfsetfillcolor{currentfill}%
\pgfsetfillopacity{0.358017}%
\pgfsetlinewidth{1.003750pt}%
\definecolor{currentstroke}{rgb}{0.121569,0.466667,0.705882}%
\pgfsetstrokecolor{currentstroke}%
\pgfsetstrokeopacity{0.358017}%
\pgfsetdash{}{0pt}%
\pgfpathmoveto{\pgfqpoint{1.635980in}{3.099567in}}%
\pgfpathcurveto{\pgfqpoint{1.644217in}{3.099567in}}{\pgfqpoint{1.652117in}{3.102839in}}{\pgfqpoint{1.657941in}{3.108663in}}%
\pgfpathcurveto{\pgfqpoint{1.663764in}{3.114487in}}{\pgfqpoint{1.667037in}{3.122387in}}{\pgfqpoint{1.667037in}{3.130624in}}%
\pgfpathcurveto{\pgfqpoint{1.667037in}{3.138860in}}{\pgfqpoint{1.663764in}{3.146760in}}{\pgfqpoint{1.657941in}{3.152584in}}%
\pgfpathcurveto{\pgfqpoint{1.652117in}{3.158408in}}{\pgfqpoint{1.644217in}{3.161680in}}{\pgfqpoint{1.635980in}{3.161680in}}%
\pgfpathcurveto{\pgfqpoint{1.627744in}{3.161680in}}{\pgfqpoint{1.619844in}{3.158408in}}{\pgfqpoint{1.614020in}{3.152584in}}%
\pgfpathcurveto{\pgfqpoint{1.608196in}{3.146760in}}{\pgfqpoint{1.604924in}{3.138860in}}{\pgfqpoint{1.604924in}{3.130624in}}%
\pgfpathcurveto{\pgfqpoint{1.604924in}{3.122387in}}{\pgfqpoint{1.608196in}{3.114487in}}{\pgfqpoint{1.614020in}{3.108663in}}%
\pgfpathcurveto{\pgfqpoint{1.619844in}{3.102839in}}{\pgfqpoint{1.627744in}{3.099567in}}{\pgfqpoint{1.635980in}{3.099567in}}%
\pgfpathclose%
\pgfusepath{stroke,fill}%
\end{pgfscope}%
\begin{pgfscope}%
\pgfpathrectangle{\pgfqpoint{0.100000in}{0.212622in}}{\pgfqpoint{3.696000in}{3.696000in}}%
\pgfusepath{clip}%
\pgfsetbuttcap%
\pgfsetroundjoin%
\definecolor{currentfill}{rgb}{0.121569,0.466667,0.705882}%
\pgfsetfillcolor{currentfill}%
\pgfsetfillopacity{0.358914}%
\pgfsetlinewidth{1.003750pt}%
\definecolor{currentstroke}{rgb}{0.121569,0.466667,0.705882}%
\pgfsetstrokecolor{currentstroke}%
\pgfsetstrokeopacity{0.358914}%
\pgfsetdash{}{0pt}%
\pgfpathmoveto{\pgfqpoint{1.857850in}{3.167246in}}%
\pgfpathcurveto{\pgfqpoint{1.866086in}{3.167246in}}{\pgfqpoint{1.873986in}{3.170518in}}{\pgfqpoint{1.879810in}{3.176342in}}%
\pgfpathcurveto{\pgfqpoint{1.885634in}{3.182166in}}{\pgfqpoint{1.888906in}{3.190066in}}{\pgfqpoint{1.888906in}{3.198302in}}%
\pgfpathcurveto{\pgfqpoint{1.888906in}{3.206539in}}{\pgfqpoint{1.885634in}{3.214439in}}{\pgfqpoint{1.879810in}{3.220263in}}%
\pgfpathcurveto{\pgfqpoint{1.873986in}{3.226087in}}{\pgfqpoint{1.866086in}{3.229359in}}{\pgfqpoint{1.857850in}{3.229359in}}%
\pgfpathcurveto{\pgfqpoint{1.849614in}{3.229359in}}{\pgfqpoint{1.841714in}{3.226087in}}{\pgfqpoint{1.835890in}{3.220263in}}%
\pgfpathcurveto{\pgfqpoint{1.830066in}{3.214439in}}{\pgfqpoint{1.826793in}{3.206539in}}{\pgfqpoint{1.826793in}{3.198302in}}%
\pgfpathcurveto{\pgfqpoint{1.826793in}{3.190066in}}{\pgfqpoint{1.830066in}{3.182166in}}{\pgfqpoint{1.835890in}{3.176342in}}%
\pgfpathcurveto{\pgfqpoint{1.841714in}{3.170518in}}{\pgfqpoint{1.849614in}{3.167246in}}{\pgfqpoint{1.857850in}{3.167246in}}%
\pgfpathclose%
\pgfusepath{stroke,fill}%
\end{pgfscope}%
\begin{pgfscope}%
\pgfpathrectangle{\pgfqpoint{0.100000in}{0.212622in}}{\pgfqpoint{3.696000in}{3.696000in}}%
\pgfusepath{clip}%
\pgfsetbuttcap%
\pgfsetroundjoin%
\definecolor{currentfill}{rgb}{0.121569,0.466667,0.705882}%
\pgfsetfillcolor{currentfill}%
\pgfsetfillopacity{0.359144}%
\pgfsetlinewidth{1.003750pt}%
\definecolor{currentstroke}{rgb}{0.121569,0.466667,0.705882}%
\pgfsetstrokecolor{currentstroke}%
\pgfsetstrokeopacity{0.359144}%
\pgfsetdash{}{0pt}%
\pgfpathmoveto{\pgfqpoint{1.633694in}{3.094464in}}%
\pgfpathcurveto{\pgfqpoint{1.641931in}{3.094464in}}{\pgfqpoint{1.649831in}{3.097736in}}{\pgfqpoint{1.655655in}{3.103560in}}%
\pgfpathcurveto{\pgfqpoint{1.661479in}{3.109384in}}{\pgfqpoint{1.664751in}{3.117284in}}{\pgfqpoint{1.664751in}{3.125520in}}%
\pgfpathcurveto{\pgfqpoint{1.664751in}{3.133757in}}{\pgfqpoint{1.661479in}{3.141657in}}{\pgfqpoint{1.655655in}{3.147481in}}%
\pgfpathcurveto{\pgfqpoint{1.649831in}{3.153305in}}{\pgfqpoint{1.641931in}{3.156577in}}{\pgfqpoint{1.633694in}{3.156577in}}%
\pgfpathcurveto{\pgfqpoint{1.625458in}{3.156577in}}{\pgfqpoint{1.617558in}{3.153305in}}{\pgfqpoint{1.611734in}{3.147481in}}%
\pgfpathcurveto{\pgfqpoint{1.605910in}{3.141657in}}{\pgfqpoint{1.602638in}{3.133757in}}{\pgfqpoint{1.602638in}{3.125520in}}%
\pgfpathcurveto{\pgfqpoint{1.602638in}{3.117284in}}{\pgfqpoint{1.605910in}{3.109384in}}{\pgfqpoint{1.611734in}{3.103560in}}%
\pgfpathcurveto{\pgfqpoint{1.617558in}{3.097736in}}{\pgfqpoint{1.625458in}{3.094464in}}{\pgfqpoint{1.633694in}{3.094464in}}%
\pgfpathclose%
\pgfusepath{stroke,fill}%
\end{pgfscope}%
\begin{pgfscope}%
\pgfpathrectangle{\pgfqpoint{0.100000in}{0.212622in}}{\pgfqpoint{3.696000in}{3.696000in}}%
\pgfusepath{clip}%
\pgfsetbuttcap%
\pgfsetroundjoin%
\definecolor{currentfill}{rgb}{0.121569,0.466667,0.705882}%
\pgfsetfillcolor{currentfill}%
\pgfsetfillopacity{0.359619}%
\pgfsetlinewidth{1.003750pt}%
\definecolor{currentstroke}{rgb}{0.121569,0.466667,0.705882}%
\pgfsetstrokecolor{currentstroke}%
\pgfsetstrokeopacity{0.359619}%
\pgfsetdash{}{0pt}%
\pgfpathmoveto{\pgfqpoint{1.632701in}{3.092410in}}%
\pgfpathcurveto{\pgfqpoint{1.640937in}{3.092410in}}{\pgfqpoint{1.648837in}{3.095682in}}{\pgfqpoint{1.654661in}{3.101506in}}%
\pgfpathcurveto{\pgfqpoint{1.660485in}{3.107330in}}{\pgfqpoint{1.663757in}{3.115230in}}{\pgfqpoint{1.663757in}{3.123466in}}%
\pgfpathcurveto{\pgfqpoint{1.663757in}{3.131702in}}{\pgfqpoint{1.660485in}{3.139602in}}{\pgfqpoint{1.654661in}{3.145426in}}%
\pgfpathcurveto{\pgfqpoint{1.648837in}{3.151250in}}{\pgfqpoint{1.640937in}{3.154523in}}{\pgfqpoint{1.632701in}{3.154523in}}%
\pgfpathcurveto{\pgfqpoint{1.624464in}{3.154523in}}{\pgfqpoint{1.616564in}{3.151250in}}{\pgfqpoint{1.610740in}{3.145426in}}%
\pgfpathcurveto{\pgfqpoint{1.604916in}{3.139602in}}{\pgfqpoint{1.601644in}{3.131702in}}{\pgfqpoint{1.601644in}{3.123466in}}%
\pgfpathcurveto{\pgfqpoint{1.601644in}{3.115230in}}{\pgfqpoint{1.604916in}{3.107330in}}{\pgfqpoint{1.610740in}{3.101506in}}%
\pgfpathcurveto{\pgfqpoint{1.616564in}{3.095682in}}{\pgfqpoint{1.624464in}{3.092410in}}{\pgfqpoint{1.632701in}{3.092410in}}%
\pgfpathclose%
\pgfusepath{stroke,fill}%
\end{pgfscope}%
\begin{pgfscope}%
\pgfpathrectangle{\pgfqpoint{0.100000in}{0.212622in}}{\pgfqpoint{3.696000in}{3.696000in}}%
\pgfusepath{clip}%
\pgfsetbuttcap%
\pgfsetroundjoin%
\definecolor{currentfill}{rgb}{0.121569,0.466667,0.705882}%
\pgfsetfillcolor{currentfill}%
\pgfsetfillopacity{0.360487}%
\pgfsetlinewidth{1.003750pt}%
\definecolor{currentstroke}{rgb}{0.121569,0.466667,0.705882}%
\pgfsetstrokecolor{currentstroke}%
\pgfsetstrokeopacity{0.360487}%
\pgfsetdash{}{0pt}%
\pgfpathmoveto{\pgfqpoint{1.631001in}{3.088582in}}%
\pgfpathcurveto{\pgfqpoint{1.639237in}{3.088582in}}{\pgfqpoint{1.647137in}{3.091855in}}{\pgfqpoint{1.652961in}{3.097679in}}%
\pgfpathcurveto{\pgfqpoint{1.658785in}{3.103503in}}{\pgfqpoint{1.662057in}{3.111403in}}{\pgfqpoint{1.662057in}{3.119639in}}%
\pgfpathcurveto{\pgfqpoint{1.662057in}{3.127875in}}{\pgfqpoint{1.658785in}{3.135775in}}{\pgfqpoint{1.652961in}{3.141599in}}%
\pgfpathcurveto{\pgfqpoint{1.647137in}{3.147423in}}{\pgfqpoint{1.639237in}{3.150695in}}{\pgfqpoint{1.631001in}{3.150695in}}%
\pgfpathcurveto{\pgfqpoint{1.622764in}{3.150695in}}{\pgfqpoint{1.614864in}{3.147423in}}{\pgfqpoint{1.609040in}{3.141599in}}%
\pgfpathcurveto{\pgfqpoint{1.603216in}{3.135775in}}{\pgfqpoint{1.599944in}{3.127875in}}{\pgfqpoint{1.599944in}{3.119639in}}%
\pgfpathcurveto{\pgfqpoint{1.599944in}{3.111403in}}{\pgfqpoint{1.603216in}{3.103503in}}{\pgfqpoint{1.609040in}{3.097679in}}%
\pgfpathcurveto{\pgfqpoint{1.614864in}{3.091855in}}{\pgfqpoint{1.622764in}{3.088582in}}{\pgfqpoint{1.631001in}{3.088582in}}%
\pgfpathclose%
\pgfusepath{stroke,fill}%
\end{pgfscope}%
\begin{pgfscope}%
\pgfpathrectangle{\pgfqpoint{0.100000in}{0.212622in}}{\pgfqpoint{3.696000in}{3.696000in}}%
\pgfusepath{clip}%
\pgfsetbuttcap%
\pgfsetroundjoin%
\definecolor{currentfill}{rgb}{0.121569,0.466667,0.705882}%
\pgfsetfillcolor{currentfill}%
\pgfsetfillopacity{0.360800}%
\pgfsetlinewidth{1.003750pt}%
\definecolor{currentstroke}{rgb}{0.121569,0.466667,0.705882}%
\pgfsetstrokecolor{currentstroke}%
\pgfsetstrokeopacity{0.360800}%
\pgfsetdash{}{0pt}%
\pgfpathmoveto{\pgfqpoint{1.630368in}{3.087207in}}%
\pgfpathcurveto{\pgfqpoint{1.638604in}{3.087207in}}{\pgfqpoint{1.646504in}{3.090479in}}{\pgfqpoint{1.652328in}{3.096303in}}%
\pgfpathcurveto{\pgfqpoint{1.658152in}{3.102127in}}{\pgfqpoint{1.661424in}{3.110027in}}{\pgfqpoint{1.661424in}{3.118263in}}%
\pgfpathcurveto{\pgfqpoint{1.661424in}{3.126499in}}{\pgfqpoint{1.658152in}{3.134399in}}{\pgfqpoint{1.652328in}{3.140223in}}%
\pgfpathcurveto{\pgfqpoint{1.646504in}{3.146047in}}{\pgfqpoint{1.638604in}{3.149320in}}{\pgfqpoint{1.630368in}{3.149320in}}%
\pgfpathcurveto{\pgfqpoint{1.622132in}{3.149320in}}{\pgfqpoint{1.614232in}{3.146047in}}{\pgfqpoint{1.608408in}{3.140223in}}%
\pgfpathcurveto{\pgfqpoint{1.602584in}{3.134399in}}{\pgfqpoint{1.599311in}{3.126499in}}{\pgfqpoint{1.599311in}{3.118263in}}%
\pgfpathcurveto{\pgfqpoint{1.599311in}{3.110027in}}{\pgfqpoint{1.602584in}{3.102127in}}{\pgfqpoint{1.608408in}{3.096303in}}%
\pgfpathcurveto{\pgfqpoint{1.614232in}{3.090479in}}{\pgfqpoint{1.622132in}{3.087207in}}{\pgfqpoint{1.630368in}{3.087207in}}%
\pgfpathclose%
\pgfusepath{stroke,fill}%
\end{pgfscope}%
\begin{pgfscope}%
\pgfpathrectangle{\pgfqpoint{0.100000in}{0.212622in}}{\pgfqpoint{3.696000in}{3.696000in}}%
\pgfusepath{clip}%
\pgfsetbuttcap%
\pgfsetroundjoin%
\definecolor{currentfill}{rgb}{0.121569,0.466667,0.705882}%
\pgfsetfillcolor{currentfill}%
\pgfsetfillopacity{0.361360}%
\pgfsetlinewidth{1.003750pt}%
\definecolor{currentstroke}{rgb}{0.121569,0.466667,0.705882}%
\pgfsetstrokecolor{currentstroke}%
\pgfsetstrokeopacity{0.361360}%
\pgfsetdash{}{0pt}%
\pgfpathmoveto{\pgfqpoint{1.629223in}{3.084669in}}%
\pgfpathcurveto{\pgfqpoint{1.637460in}{3.084669in}}{\pgfqpoint{1.645360in}{3.087941in}}{\pgfqpoint{1.651184in}{3.093765in}}%
\pgfpathcurveto{\pgfqpoint{1.657007in}{3.099589in}}{\pgfqpoint{1.660280in}{3.107489in}}{\pgfqpoint{1.660280in}{3.115725in}}%
\pgfpathcurveto{\pgfqpoint{1.660280in}{3.123962in}}{\pgfqpoint{1.657007in}{3.131862in}}{\pgfqpoint{1.651184in}{3.137686in}}%
\pgfpathcurveto{\pgfqpoint{1.645360in}{3.143509in}}{\pgfqpoint{1.637460in}{3.146782in}}{\pgfqpoint{1.629223in}{3.146782in}}%
\pgfpathcurveto{\pgfqpoint{1.620987in}{3.146782in}}{\pgfqpoint{1.613087in}{3.143509in}}{\pgfqpoint{1.607263in}{3.137686in}}%
\pgfpathcurveto{\pgfqpoint{1.601439in}{3.131862in}}{\pgfqpoint{1.598167in}{3.123962in}}{\pgfqpoint{1.598167in}{3.115725in}}%
\pgfpathcurveto{\pgfqpoint{1.598167in}{3.107489in}}{\pgfqpoint{1.601439in}{3.099589in}}{\pgfqpoint{1.607263in}{3.093765in}}%
\pgfpathcurveto{\pgfqpoint{1.613087in}{3.087941in}}{\pgfqpoint{1.620987in}{3.084669in}}{\pgfqpoint{1.629223in}{3.084669in}}%
\pgfpathclose%
\pgfusepath{stroke,fill}%
\end{pgfscope}%
\begin{pgfscope}%
\pgfpathrectangle{\pgfqpoint{0.100000in}{0.212622in}}{\pgfqpoint{3.696000in}{3.696000in}}%
\pgfusepath{clip}%
\pgfsetbuttcap%
\pgfsetroundjoin%
\definecolor{currentfill}{rgb}{0.121569,0.466667,0.705882}%
\pgfsetfillcolor{currentfill}%
\pgfsetfillopacity{0.362412}%
\pgfsetlinewidth{1.003750pt}%
\definecolor{currentstroke}{rgb}{0.121569,0.466667,0.705882}%
\pgfsetstrokecolor{currentstroke}%
\pgfsetstrokeopacity{0.362412}%
\pgfsetdash{}{0pt}%
\pgfpathmoveto{\pgfqpoint{1.627133in}{3.080193in}}%
\pgfpathcurveto{\pgfqpoint{1.635369in}{3.080193in}}{\pgfqpoint{1.643269in}{3.083465in}}{\pgfqpoint{1.649093in}{3.089289in}}%
\pgfpathcurveto{\pgfqpoint{1.654917in}{3.095113in}}{\pgfqpoint{1.658189in}{3.103013in}}{\pgfqpoint{1.658189in}{3.111249in}}%
\pgfpathcurveto{\pgfqpoint{1.658189in}{3.119485in}}{\pgfqpoint{1.654917in}{3.127385in}}{\pgfqpoint{1.649093in}{3.133209in}}%
\pgfpathcurveto{\pgfqpoint{1.643269in}{3.139033in}}{\pgfqpoint{1.635369in}{3.142306in}}{\pgfqpoint{1.627133in}{3.142306in}}%
\pgfpathcurveto{\pgfqpoint{1.618896in}{3.142306in}}{\pgfqpoint{1.610996in}{3.139033in}}{\pgfqpoint{1.605172in}{3.133209in}}%
\pgfpathcurveto{\pgfqpoint{1.599348in}{3.127385in}}{\pgfqpoint{1.596076in}{3.119485in}}{\pgfqpoint{1.596076in}{3.111249in}}%
\pgfpathcurveto{\pgfqpoint{1.596076in}{3.103013in}}{\pgfqpoint{1.599348in}{3.095113in}}{\pgfqpoint{1.605172in}{3.089289in}}%
\pgfpathcurveto{\pgfqpoint{1.610996in}{3.083465in}}{\pgfqpoint{1.618896in}{3.080193in}}{\pgfqpoint{1.627133in}{3.080193in}}%
\pgfpathclose%
\pgfusepath{stroke,fill}%
\end{pgfscope}%
\begin{pgfscope}%
\pgfpathrectangle{\pgfqpoint{0.100000in}{0.212622in}}{\pgfqpoint{3.696000in}{3.696000in}}%
\pgfusepath{clip}%
\pgfsetbuttcap%
\pgfsetroundjoin%
\definecolor{currentfill}{rgb}{0.121569,0.466667,0.705882}%
\pgfsetfillcolor{currentfill}%
\pgfsetfillopacity{0.364269}%
\pgfsetlinewidth{1.003750pt}%
\definecolor{currentstroke}{rgb}{0.121569,0.466667,0.705882}%
\pgfsetstrokecolor{currentstroke}%
\pgfsetstrokeopacity{0.364269}%
\pgfsetdash{}{0pt}%
\pgfpathmoveto{\pgfqpoint{1.623269in}{3.071890in}}%
\pgfpathcurveto{\pgfqpoint{1.631506in}{3.071890in}}{\pgfqpoint{1.639406in}{3.075162in}}{\pgfqpoint{1.645230in}{3.080986in}}%
\pgfpathcurveto{\pgfqpoint{1.651054in}{3.086810in}}{\pgfqpoint{1.654326in}{3.094710in}}{\pgfqpoint{1.654326in}{3.102946in}}%
\pgfpathcurveto{\pgfqpoint{1.654326in}{3.111183in}}{\pgfqpoint{1.651054in}{3.119083in}}{\pgfqpoint{1.645230in}{3.124907in}}%
\pgfpathcurveto{\pgfqpoint{1.639406in}{3.130731in}}{\pgfqpoint{1.631506in}{3.134003in}}{\pgfqpoint{1.623269in}{3.134003in}}%
\pgfpathcurveto{\pgfqpoint{1.615033in}{3.134003in}}{\pgfqpoint{1.607133in}{3.130731in}}{\pgfqpoint{1.601309in}{3.124907in}}%
\pgfpathcurveto{\pgfqpoint{1.595485in}{3.119083in}}{\pgfqpoint{1.592213in}{3.111183in}}{\pgfqpoint{1.592213in}{3.102946in}}%
\pgfpathcurveto{\pgfqpoint{1.592213in}{3.094710in}}{\pgfqpoint{1.595485in}{3.086810in}}{\pgfqpoint{1.601309in}{3.080986in}}%
\pgfpathcurveto{\pgfqpoint{1.607133in}{3.075162in}}{\pgfqpoint{1.615033in}{3.071890in}}{\pgfqpoint{1.623269in}{3.071890in}}%
\pgfpathclose%
\pgfusepath{stroke,fill}%
\end{pgfscope}%
\begin{pgfscope}%
\pgfpathrectangle{\pgfqpoint{0.100000in}{0.212622in}}{\pgfqpoint{3.696000in}{3.696000in}}%
\pgfusepath{clip}%
\pgfsetbuttcap%
\pgfsetroundjoin%
\definecolor{currentfill}{rgb}{0.121569,0.466667,0.705882}%
\pgfsetfillcolor{currentfill}%
\pgfsetfillopacity{0.366903}%
\pgfsetlinewidth{1.003750pt}%
\definecolor{currentstroke}{rgb}{0.121569,0.466667,0.705882}%
\pgfsetstrokecolor{currentstroke}%
\pgfsetstrokeopacity{0.366903}%
\pgfsetdash{}{0pt}%
\pgfpathmoveto{\pgfqpoint{1.866944in}{3.141040in}}%
\pgfpathcurveto{\pgfqpoint{1.875180in}{3.141040in}}{\pgfqpoint{1.883080in}{3.144312in}}{\pgfqpoint{1.888904in}{3.150136in}}%
\pgfpathcurveto{\pgfqpoint{1.894728in}{3.155960in}}{\pgfqpoint{1.898000in}{3.163860in}}{\pgfqpoint{1.898000in}{3.172097in}}%
\pgfpathcurveto{\pgfqpoint{1.898000in}{3.180333in}}{\pgfqpoint{1.894728in}{3.188233in}}{\pgfqpoint{1.888904in}{3.194057in}}%
\pgfpathcurveto{\pgfqpoint{1.883080in}{3.199881in}}{\pgfqpoint{1.875180in}{3.203153in}}{\pgfqpoint{1.866944in}{3.203153in}}%
\pgfpathcurveto{\pgfqpoint{1.858707in}{3.203153in}}{\pgfqpoint{1.850807in}{3.199881in}}{\pgfqpoint{1.844983in}{3.194057in}}%
\pgfpathcurveto{\pgfqpoint{1.839159in}{3.188233in}}{\pgfqpoint{1.835887in}{3.180333in}}{\pgfqpoint{1.835887in}{3.172097in}}%
\pgfpathcurveto{\pgfqpoint{1.835887in}{3.163860in}}{\pgfqpoint{1.839159in}{3.155960in}}{\pgfqpoint{1.844983in}{3.150136in}}%
\pgfpathcurveto{\pgfqpoint{1.850807in}{3.144312in}}{\pgfqpoint{1.858707in}{3.141040in}}{\pgfqpoint{1.866944in}{3.141040in}}%
\pgfpathclose%
\pgfusepath{stroke,fill}%
\end{pgfscope}%
\begin{pgfscope}%
\pgfpathrectangle{\pgfqpoint{0.100000in}{0.212622in}}{\pgfqpoint{3.696000in}{3.696000in}}%
\pgfusepath{clip}%
\pgfsetbuttcap%
\pgfsetroundjoin%
\definecolor{currentfill}{rgb}{0.121569,0.466667,0.705882}%
\pgfsetfillcolor{currentfill}%
\pgfsetfillopacity{0.367700}%
\pgfsetlinewidth{1.003750pt}%
\definecolor{currentstroke}{rgb}{0.121569,0.466667,0.705882}%
\pgfsetstrokecolor{currentstroke}%
\pgfsetstrokeopacity{0.367700}%
\pgfsetdash{}{0pt}%
\pgfpathmoveto{\pgfqpoint{1.616584in}{3.056637in}}%
\pgfpathcurveto{\pgfqpoint{1.624820in}{3.056637in}}{\pgfqpoint{1.632720in}{3.059909in}}{\pgfqpoint{1.638544in}{3.065733in}}%
\pgfpathcurveto{\pgfqpoint{1.644368in}{3.071557in}}{\pgfqpoint{1.647641in}{3.079457in}}{\pgfqpoint{1.647641in}{3.087693in}}%
\pgfpathcurveto{\pgfqpoint{1.647641in}{3.095930in}}{\pgfqpoint{1.644368in}{3.103830in}}{\pgfqpoint{1.638544in}{3.109654in}}%
\pgfpathcurveto{\pgfqpoint{1.632720in}{3.115478in}}{\pgfqpoint{1.624820in}{3.118750in}}{\pgfqpoint{1.616584in}{3.118750in}}%
\pgfpathcurveto{\pgfqpoint{1.608348in}{3.118750in}}{\pgfqpoint{1.600448in}{3.115478in}}{\pgfqpoint{1.594624in}{3.109654in}}%
\pgfpathcurveto{\pgfqpoint{1.588800in}{3.103830in}}{\pgfqpoint{1.585528in}{3.095930in}}{\pgfqpoint{1.585528in}{3.087693in}}%
\pgfpathcurveto{\pgfqpoint{1.585528in}{3.079457in}}{\pgfqpoint{1.588800in}{3.071557in}}{\pgfqpoint{1.594624in}{3.065733in}}%
\pgfpathcurveto{\pgfqpoint{1.600448in}{3.059909in}}{\pgfqpoint{1.608348in}{3.056637in}}{\pgfqpoint{1.616584in}{3.056637in}}%
\pgfpathclose%
\pgfusepath{stroke,fill}%
\end{pgfscope}%
\begin{pgfscope}%
\pgfpathrectangle{\pgfqpoint{0.100000in}{0.212622in}}{\pgfqpoint{3.696000in}{3.696000in}}%
\pgfusepath{clip}%
\pgfsetbuttcap%
\pgfsetroundjoin%
\definecolor{currentfill}{rgb}{0.121569,0.466667,0.705882}%
\pgfsetfillcolor{currentfill}%
\pgfsetfillopacity{0.371211}%
\pgfsetlinewidth{1.003750pt}%
\definecolor{currentstroke}{rgb}{0.121569,0.466667,0.705882}%
\pgfsetstrokecolor{currentstroke}%
\pgfsetstrokeopacity{0.371211}%
\pgfsetdash{}{0pt}%
\pgfpathmoveto{\pgfqpoint{1.871547in}{3.126244in}}%
\pgfpathcurveto{\pgfqpoint{1.879783in}{3.126244in}}{\pgfqpoint{1.887683in}{3.129516in}}{\pgfqpoint{1.893507in}{3.135340in}}%
\pgfpathcurveto{\pgfqpoint{1.899331in}{3.141164in}}{\pgfqpoint{1.902603in}{3.149064in}}{\pgfqpoint{1.902603in}{3.157300in}}%
\pgfpathcurveto{\pgfqpoint{1.902603in}{3.165537in}}{\pgfqpoint{1.899331in}{3.173437in}}{\pgfqpoint{1.893507in}{3.179261in}}%
\pgfpathcurveto{\pgfqpoint{1.887683in}{3.185084in}}{\pgfqpoint{1.879783in}{3.188357in}}{\pgfqpoint{1.871547in}{3.188357in}}%
\pgfpathcurveto{\pgfqpoint{1.863310in}{3.188357in}}{\pgfqpoint{1.855410in}{3.185084in}}{\pgfqpoint{1.849586in}{3.179261in}}%
\pgfpathcurveto{\pgfqpoint{1.843763in}{3.173437in}}{\pgfqpoint{1.840490in}{3.165537in}}{\pgfqpoint{1.840490in}{3.157300in}}%
\pgfpathcurveto{\pgfqpoint{1.840490in}{3.149064in}}{\pgfqpoint{1.843763in}{3.141164in}}{\pgfqpoint{1.849586in}{3.135340in}}%
\pgfpathcurveto{\pgfqpoint{1.855410in}{3.129516in}}{\pgfqpoint{1.863310in}{3.126244in}}{\pgfqpoint{1.871547in}{3.126244in}}%
\pgfpathclose%
\pgfusepath{stroke,fill}%
\end{pgfscope}%
\begin{pgfscope}%
\pgfpathrectangle{\pgfqpoint{0.100000in}{0.212622in}}{\pgfqpoint{3.696000in}{3.696000in}}%
\pgfusepath{clip}%
\pgfsetbuttcap%
\pgfsetroundjoin%
\definecolor{currentfill}{rgb}{0.121569,0.466667,0.705882}%
\pgfsetfillcolor{currentfill}%
\pgfsetfillopacity{0.373629}%
\pgfsetlinewidth{1.003750pt}%
\definecolor{currentstroke}{rgb}{0.121569,0.466667,0.705882}%
\pgfsetstrokecolor{currentstroke}%
\pgfsetstrokeopacity{0.373629}%
\pgfsetdash{}{0pt}%
\pgfpathmoveto{\pgfqpoint{1.603537in}{3.028535in}}%
\pgfpathcurveto{\pgfqpoint{1.611773in}{3.028535in}}{\pgfqpoint{1.619674in}{3.031807in}}{\pgfqpoint{1.625497in}{3.037631in}}%
\pgfpathcurveto{\pgfqpoint{1.631321in}{3.043455in}}{\pgfqpoint{1.634594in}{3.051355in}}{\pgfqpoint{1.634594in}{3.059591in}}%
\pgfpathcurveto{\pgfqpoint{1.634594in}{3.067828in}}{\pgfqpoint{1.631321in}{3.075728in}}{\pgfqpoint{1.625497in}{3.081552in}}%
\pgfpathcurveto{\pgfqpoint{1.619674in}{3.087376in}}{\pgfqpoint{1.611773in}{3.090648in}}{\pgfqpoint{1.603537in}{3.090648in}}%
\pgfpathcurveto{\pgfqpoint{1.595301in}{3.090648in}}{\pgfqpoint{1.587401in}{3.087376in}}{\pgfqpoint{1.581577in}{3.081552in}}%
\pgfpathcurveto{\pgfqpoint{1.575753in}{3.075728in}}{\pgfqpoint{1.572481in}{3.067828in}}{\pgfqpoint{1.572481in}{3.059591in}}%
\pgfpathcurveto{\pgfqpoint{1.572481in}{3.051355in}}{\pgfqpoint{1.575753in}{3.043455in}}{\pgfqpoint{1.581577in}{3.037631in}}%
\pgfpathcurveto{\pgfqpoint{1.587401in}{3.031807in}}{\pgfqpoint{1.595301in}{3.028535in}}{\pgfqpoint{1.603537in}{3.028535in}}%
\pgfpathclose%
\pgfusepath{stroke,fill}%
\end{pgfscope}%
\begin{pgfscope}%
\pgfpathrectangle{\pgfqpoint{0.100000in}{0.212622in}}{\pgfqpoint{3.696000in}{3.696000in}}%
\pgfusepath{clip}%
\pgfsetbuttcap%
\pgfsetroundjoin%
\definecolor{currentfill}{rgb}{0.121569,0.466667,0.705882}%
\pgfsetfillcolor{currentfill}%
\pgfsetfillopacity{0.374774}%
\pgfsetlinewidth{1.003750pt}%
\definecolor{currentstroke}{rgb}{0.121569,0.466667,0.705882}%
\pgfsetstrokecolor{currentstroke}%
\pgfsetstrokeopacity{0.374774}%
\pgfsetdash{}{0pt}%
\pgfpathmoveto{\pgfqpoint{1.601325in}{3.023452in}}%
\pgfpathcurveto{\pgfqpoint{1.609562in}{3.023452in}}{\pgfqpoint{1.617462in}{3.026725in}}{\pgfqpoint{1.623286in}{3.032549in}}%
\pgfpathcurveto{\pgfqpoint{1.629110in}{3.038373in}}{\pgfqpoint{1.632382in}{3.046273in}}{\pgfqpoint{1.632382in}{3.054509in}}%
\pgfpathcurveto{\pgfqpoint{1.632382in}{3.062745in}}{\pgfqpoint{1.629110in}{3.070645in}}{\pgfqpoint{1.623286in}{3.076469in}}%
\pgfpathcurveto{\pgfqpoint{1.617462in}{3.082293in}}{\pgfqpoint{1.609562in}{3.085565in}}{\pgfqpoint{1.601325in}{3.085565in}}%
\pgfpathcurveto{\pgfqpoint{1.593089in}{3.085565in}}{\pgfqpoint{1.585189in}{3.082293in}}{\pgfqpoint{1.579365in}{3.076469in}}%
\pgfpathcurveto{\pgfqpoint{1.573541in}{3.070645in}}{\pgfqpoint{1.570269in}{3.062745in}}{\pgfqpoint{1.570269in}{3.054509in}}%
\pgfpathcurveto{\pgfqpoint{1.570269in}{3.046273in}}{\pgfqpoint{1.573541in}{3.038373in}}{\pgfqpoint{1.579365in}{3.032549in}}%
\pgfpathcurveto{\pgfqpoint{1.585189in}{3.026725in}}{\pgfqpoint{1.593089in}{3.023452in}}{\pgfqpoint{1.601325in}{3.023452in}}%
\pgfpathclose%
\pgfusepath{stroke,fill}%
\end{pgfscope}%
\begin{pgfscope}%
\pgfpathrectangle{\pgfqpoint{0.100000in}{0.212622in}}{\pgfqpoint{3.696000in}{3.696000in}}%
\pgfusepath{clip}%
\pgfsetbuttcap%
\pgfsetroundjoin%
\definecolor{currentfill}{rgb}{0.121569,0.466667,0.705882}%
\pgfsetfillcolor{currentfill}%
\pgfsetfillopacity{0.375389}%
\pgfsetlinewidth{1.003750pt}%
\definecolor{currentstroke}{rgb}{0.121569,0.466667,0.705882}%
\pgfsetstrokecolor{currentstroke}%
\pgfsetstrokeopacity{0.375389}%
\pgfsetdash{}{0pt}%
\pgfpathmoveto{\pgfqpoint{1.600084in}{3.020611in}}%
\pgfpathcurveto{\pgfqpoint{1.608320in}{3.020611in}}{\pgfqpoint{1.616220in}{3.023883in}}{\pgfqpoint{1.622044in}{3.029707in}}%
\pgfpathcurveto{\pgfqpoint{1.627868in}{3.035531in}}{\pgfqpoint{1.631140in}{3.043431in}}{\pgfqpoint{1.631140in}{3.051667in}}%
\pgfpathcurveto{\pgfqpoint{1.631140in}{3.059904in}}{\pgfqpoint{1.627868in}{3.067804in}}{\pgfqpoint{1.622044in}{3.073628in}}%
\pgfpathcurveto{\pgfqpoint{1.616220in}{3.079452in}}{\pgfqpoint{1.608320in}{3.082724in}}{\pgfqpoint{1.600084in}{3.082724in}}%
\pgfpathcurveto{\pgfqpoint{1.591847in}{3.082724in}}{\pgfqpoint{1.583947in}{3.079452in}}{\pgfqpoint{1.578123in}{3.073628in}}%
\pgfpathcurveto{\pgfqpoint{1.572299in}{3.067804in}}{\pgfqpoint{1.569027in}{3.059904in}}{\pgfqpoint{1.569027in}{3.051667in}}%
\pgfpathcurveto{\pgfqpoint{1.569027in}{3.043431in}}{\pgfqpoint{1.572299in}{3.035531in}}{\pgfqpoint{1.578123in}{3.029707in}}%
\pgfpathcurveto{\pgfqpoint{1.583947in}{3.023883in}}{\pgfqpoint{1.591847in}{3.020611in}}{\pgfqpoint{1.600084in}{3.020611in}}%
\pgfpathclose%
\pgfusepath{stroke,fill}%
\end{pgfscope}%
\begin{pgfscope}%
\pgfpathrectangle{\pgfqpoint{0.100000in}{0.212622in}}{\pgfqpoint{3.696000in}{3.696000in}}%
\pgfusepath{clip}%
\pgfsetbuttcap%
\pgfsetroundjoin%
\definecolor{currentfill}{rgb}{0.121569,0.466667,0.705882}%
\pgfsetfillcolor{currentfill}%
\pgfsetfillopacity{0.376131}%
\pgfsetlinewidth{1.003750pt}%
\definecolor{currentstroke}{rgb}{0.121569,0.466667,0.705882}%
\pgfsetstrokecolor{currentstroke}%
\pgfsetstrokeopacity{0.376131}%
\pgfsetdash{}{0pt}%
\pgfpathmoveto{\pgfqpoint{1.876804in}{3.109179in}}%
\pgfpathcurveto{\pgfqpoint{1.885040in}{3.109179in}}{\pgfqpoint{1.892940in}{3.112452in}}{\pgfqpoint{1.898764in}{3.118276in}}%
\pgfpathcurveto{\pgfqpoint{1.904588in}{3.124099in}}{\pgfqpoint{1.907860in}{3.132000in}}{\pgfqpoint{1.907860in}{3.140236in}}%
\pgfpathcurveto{\pgfqpoint{1.907860in}{3.148472in}}{\pgfqpoint{1.904588in}{3.156372in}}{\pgfqpoint{1.898764in}{3.162196in}}%
\pgfpathcurveto{\pgfqpoint{1.892940in}{3.168020in}}{\pgfqpoint{1.885040in}{3.171292in}}{\pgfqpoint{1.876804in}{3.171292in}}%
\pgfpathcurveto{\pgfqpoint{1.868568in}{3.171292in}}{\pgfqpoint{1.860668in}{3.168020in}}{\pgfqpoint{1.854844in}{3.162196in}}%
\pgfpathcurveto{\pgfqpoint{1.849020in}{3.156372in}}{\pgfqpoint{1.845747in}{3.148472in}}{\pgfqpoint{1.845747in}{3.140236in}}%
\pgfpathcurveto{\pgfqpoint{1.845747in}{3.132000in}}{\pgfqpoint{1.849020in}{3.124099in}}{\pgfqpoint{1.854844in}{3.118276in}}%
\pgfpathcurveto{\pgfqpoint{1.860668in}{3.112452in}}{\pgfqpoint{1.868568in}{3.109179in}}{\pgfqpoint{1.876804in}{3.109179in}}%
\pgfpathclose%
\pgfusepath{stroke,fill}%
\end{pgfscope}%
\begin{pgfscope}%
\pgfpathrectangle{\pgfqpoint{0.100000in}{0.212622in}}{\pgfqpoint{3.696000in}{3.696000in}}%
\pgfusepath{clip}%
\pgfsetbuttcap%
\pgfsetroundjoin%
\definecolor{currentfill}{rgb}{0.121569,0.466667,0.705882}%
\pgfsetfillcolor{currentfill}%
\pgfsetfillopacity{0.376591}%
\pgfsetlinewidth{1.003750pt}%
\definecolor{currentstroke}{rgb}{0.121569,0.466667,0.705882}%
\pgfsetstrokecolor{currentstroke}%
\pgfsetstrokeopacity{0.376591}%
\pgfsetdash{}{0pt}%
\pgfpathmoveto{\pgfqpoint{1.597932in}{3.015670in}}%
\pgfpathcurveto{\pgfqpoint{1.606168in}{3.015670in}}{\pgfqpoint{1.614068in}{3.018942in}}{\pgfqpoint{1.619892in}{3.024766in}}%
\pgfpathcurveto{\pgfqpoint{1.625716in}{3.030590in}}{\pgfqpoint{1.628989in}{3.038490in}}{\pgfqpoint{1.628989in}{3.046726in}}%
\pgfpathcurveto{\pgfqpoint{1.628989in}{3.054963in}}{\pgfqpoint{1.625716in}{3.062863in}}{\pgfqpoint{1.619892in}{3.068687in}}%
\pgfpathcurveto{\pgfqpoint{1.614068in}{3.074510in}}{\pgfqpoint{1.606168in}{3.077783in}}{\pgfqpoint{1.597932in}{3.077783in}}%
\pgfpathcurveto{\pgfqpoint{1.589696in}{3.077783in}}{\pgfqpoint{1.581796in}{3.074510in}}{\pgfqpoint{1.575972in}{3.068687in}}%
\pgfpathcurveto{\pgfqpoint{1.570148in}{3.062863in}}{\pgfqpoint{1.566876in}{3.054963in}}{\pgfqpoint{1.566876in}{3.046726in}}%
\pgfpathcurveto{\pgfqpoint{1.566876in}{3.038490in}}{\pgfqpoint{1.570148in}{3.030590in}}{\pgfqpoint{1.575972in}{3.024766in}}%
\pgfpathcurveto{\pgfqpoint{1.581796in}{3.018942in}}{\pgfqpoint{1.589696in}{3.015670in}}{\pgfqpoint{1.597932in}{3.015670in}}%
\pgfpathclose%
\pgfusepath{stroke,fill}%
\end{pgfscope}%
\begin{pgfscope}%
\pgfpathrectangle{\pgfqpoint{0.100000in}{0.212622in}}{\pgfqpoint{3.696000in}{3.696000in}}%
\pgfusepath{clip}%
\pgfsetbuttcap%
\pgfsetroundjoin%
\definecolor{currentfill}{rgb}{0.121569,0.466667,0.705882}%
\pgfsetfillcolor{currentfill}%
\pgfsetfillopacity{0.377202}%
\pgfsetlinewidth{1.003750pt}%
\definecolor{currentstroke}{rgb}{0.121569,0.466667,0.705882}%
\pgfsetstrokecolor{currentstroke}%
\pgfsetstrokeopacity{0.377202}%
\pgfsetdash{}{0pt}%
\pgfpathmoveto{\pgfqpoint{1.596563in}{3.012599in}}%
\pgfpathcurveto{\pgfqpoint{1.604799in}{3.012599in}}{\pgfqpoint{1.612699in}{3.015871in}}{\pgfqpoint{1.618523in}{3.021695in}}%
\pgfpathcurveto{\pgfqpoint{1.624347in}{3.027519in}}{\pgfqpoint{1.627619in}{3.035419in}}{\pgfqpoint{1.627619in}{3.043656in}}%
\pgfpathcurveto{\pgfqpoint{1.627619in}{3.051892in}}{\pgfqpoint{1.624347in}{3.059792in}}{\pgfqpoint{1.618523in}{3.065616in}}%
\pgfpathcurveto{\pgfqpoint{1.612699in}{3.071440in}}{\pgfqpoint{1.604799in}{3.074712in}}{\pgfqpoint{1.596563in}{3.074712in}}%
\pgfpathcurveto{\pgfqpoint{1.588327in}{3.074712in}}{\pgfqpoint{1.580427in}{3.071440in}}{\pgfqpoint{1.574603in}{3.065616in}}%
\pgfpathcurveto{\pgfqpoint{1.568779in}{3.059792in}}{\pgfqpoint{1.565506in}{3.051892in}}{\pgfqpoint{1.565506in}{3.043656in}}%
\pgfpathcurveto{\pgfqpoint{1.565506in}{3.035419in}}{\pgfqpoint{1.568779in}{3.027519in}}{\pgfqpoint{1.574603in}{3.021695in}}%
\pgfpathcurveto{\pgfqpoint{1.580427in}{3.015871in}}{\pgfqpoint{1.588327in}{3.012599in}}{\pgfqpoint{1.596563in}{3.012599in}}%
\pgfpathclose%
\pgfusepath{stroke,fill}%
\end{pgfscope}%
\begin{pgfscope}%
\pgfpathrectangle{\pgfqpoint{0.100000in}{0.212622in}}{\pgfqpoint{3.696000in}{3.696000in}}%
\pgfusepath{clip}%
\pgfsetbuttcap%
\pgfsetroundjoin%
\definecolor{currentfill}{rgb}{0.121569,0.466667,0.705882}%
\pgfsetfillcolor{currentfill}%
\pgfsetfillopacity{0.378414}%
\pgfsetlinewidth{1.003750pt}%
\definecolor{currentstroke}{rgb}{0.121569,0.466667,0.705882}%
\pgfsetstrokecolor{currentstroke}%
\pgfsetstrokeopacity{0.378414}%
\pgfsetdash{}{0pt}%
\pgfpathmoveto{\pgfqpoint{1.594169in}{3.007318in}}%
\pgfpathcurveto{\pgfqpoint{1.602405in}{3.007318in}}{\pgfqpoint{1.610306in}{3.010591in}}{\pgfqpoint{1.616129in}{3.016415in}}%
\pgfpathcurveto{\pgfqpoint{1.621953in}{3.022239in}}{\pgfqpoint{1.625226in}{3.030139in}}{\pgfqpoint{1.625226in}{3.038375in}}%
\pgfpathcurveto{\pgfqpoint{1.625226in}{3.046611in}}{\pgfqpoint{1.621953in}{3.054511in}}{\pgfqpoint{1.616129in}{3.060335in}}%
\pgfpathcurveto{\pgfqpoint{1.610306in}{3.066159in}}{\pgfqpoint{1.602405in}{3.069431in}}{\pgfqpoint{1.594169in}{3.069431in}}%
\pgfpathcurveto{\pgfqpoint{1.585933in}{3.069431in}}{\pgfqpoint{1.578033in}{3.066159in}}{\pgfqpoint{1.572209in}{3.060335in}}%
\pgfpathcurveto{\pgfqpoint{1.566385in}{3.054511in}}{\pgfqpoint{1.563113in}{3.046611in}}{\pgfqpoint{1.563113in}{3.038375in}}%
\pgfpathcurveto{\pgfqpoint{1.563113in}{3.030139in}}{\pgfqpoint{1.566385in}{3.022239in}}{\pgfqpoint{1.572209in}{3.016415in}}%
\pgfpathcurveto{\pgfqpoint{1.578033in}{3.010591in}}{\pgfqpoint{1.585933in}{3.007318in}}{\pgfqpoint{1.594169in}{3.007318in}}%
\pgfpathclose%
\pgfusepath{stroke,fill}%
\end{pgfscope}%
\begin{pgfscope}%
\pgfpathrectangle{\pgfqpoint{0.100000in}{0.212622in}}{\pgfqpoint{3.696000in}{3.696000in}}%
\pgfusepath{clip}%
\pgfsetbuttcap%
\pgfsetroundjoin%
\definecolor{currentfill}{rgb}{0.121569,0.466667,0.705882}%
\pgfsetfillcolor{currentfill}%
\pgfsetfillopacity{0.379082}%
\pgfsetlinewidth{1.003750pt}%
\definecolor{currentstroke}{rgb}{0.121569,0.466667,0.705882}%
\pgfsetstrokecolor{currentstroke}%
\pgfsetstrokeopacity{0.379082}%
\pgfsetdash{}{0pt}%
\pgfpathmoveto{\pgfqpoint{1.592853in}{3.004308in}}%
\pgfpathcurveto{\pgfqpoint{1.601090in}{3.004308in}}{\pgfqpoint{1.608990in}{3.007581in}}{\pgfqpoint{1.614814in}{3.013405in}}%
\pgfpathcurveto{\pgfqpoint{1.620638in}{3.019229in}}{\pgfqpoint{1.623910in}{3.027129in}}{\pgfqpoint{1.623910in}{3.035365in}}%
\pgfpathcurveto{\pgfqpoint{1.623910in}{3.043601in}}{\pgfqpoint{1.620638in}{3.051501in}}{\pgfqpoint{1.614814in}{3.057325in}}%
\pgfpathcurveto{\pgfqpoint{1.608990in}{3.063149in}}{\pgfqpoint{1.601090in}{3.066421in}}{\pgfqpoint{1.592853in}{3.066421in}}%
\pgfpathcurveto{\pgfqpoint{1.584617in}{3.066421in}}{\pgfqpoint{1.576717in}{3.063149in}}{\pgfqpoint{1.570893in}{3.057325in}}%
\pgfpathcurveto{\pgfqpoint{1.565069in}{3.051501in}}{\pgfqpoint{1.561797in}{3.043601in}}{\pgfqpoint{1.561797in}{3.035365in}}%
\pgfpathcurveto{\pgfqpoint{1.561797in}{3.027129in}}{\pgfqpoint{1.565069in}{3.019229in}}{\pgfqpoint{1.570893in}{3.013405in}}%
\pgfpathcurveto{\pgfqpoint{1.576717in}{3.007581in}}{\pgfqpoint{1.584617in}{3.004308in}}{\pgfqpoint{1.592853in}{3.004308in}}%
\pgfpathclose%
\pgfusepath{stroke,fill}%
\end{pgfscope}%
\begin{pgfscope}%
\pgfpathrectangle{\pgfqpoint{0.100000in}{0.212622in}}{\pgfqpoint{3.696000in}{3.696000in}}%
\pgfusepath{clip}%
\pgfsetbuttcap%
\pgfsetroundjoin%
\definecolor{currentfill}{rgb}{0.121569,0.466667,0.705882}%
\pgfsetfillcolor{currentfill}%
\pgfsetfillopacity{0.379395}%
\pgfsetlinewidth{1.003750pt}%
\definecolor{currentstroke}{rgb}{0.121569,0.466667,0.705882}%
\pgfsetstrokecolor{currentstroke}%
\pgfsetstrokeopacity{0.379395}%
\pgfsetdash{}{0pt}%
\pgfpathmoveto{\pgfqpoint{1.592243in}{3.002965in}}%
\pgfpathcurveto{\pgfqpoint{1.600479in}{3.002965in}}{\pgfqpoint{1.608379in}{3.006238in}}{\pgfqpoint{1.614203in}{3.012062in}}%
\pgfpathcurveto{\pgfqpoint{1.620027in}{3.017886in}}{\pgfqpoint{1.623299in}{3.025786in}}{\pgfqpoint{1.623299in}{3.034022in}}%
\pgfpathcurveto{\pgfqpoint{1.623299in}{3.042258in}}{\pgfqpoint{1.620027in}{3.050158in}}{\pgfqpoint{1.614203in}{3.055982in}}%
\pgfpathcurveto{\pgfqpoint{1.608379in}{3.061806in}}{\pgfqpoint{1.600479in}{3.065078in}}{\pgfqpoint{1.592243in}{3.065078in}}%
\pgfpathcurveto{\pgfqpoint{1.584007in}{3.065078in}}{\pgfqpoint{1.576107in}{3.061806in}}{\pgfqpoint{1.570283in}{3.055982in}}%
\pgfpathcurveto{\pgfqpoint{1.564459in}{3.050158in}}{\pgfqpoint{1.561186in}{3.042258in}}{\pgfqpoint{1.561186in}{3.034022in}}%
\pgfpathcurveto{\pgfqpoint{1.561186in}{3.025786in}}{\pgfqpoint{1.564459in}{3.017886in}}{\pgfqpoint{1.570283in}{3.012062in}}%
\pgfpathcurveto{\pgfqpoint{1.576107in}{3.006238in}}{\pgfqpoint{1.584007in}{3.002965in}}{\pgfqpoint{1.592243in}{3.002965in}}%
\pgfpathclose%
\pgfusepath{stroke,fill}%
\end{pgfscope}%
\begin{pgfscope}%
\pgfpathrectangle{\pgfqpoint{0.100000in}{0.212622in}}{\pgfqpoint{3.696000in}{3.696000in}}%
\pgfusepath{clip}%
\pgfsetbuttcap%
\pgfsetroundjoin%
\definecolor{currentfill}{rgb}{0.121569,0.466667,0.705882}%
\pgfsetfillcolor{currentfill}%
\pgfsetfillopacity{0.379952}%
\pgfsetlinewidth{1.003750pt}%
\definecolor{currentstroke}{rgb}{0.121569,0.466667,0.705882}%
\pgfsetstrokecolor{currentstroke}%
\pgfsetstrokeopacity{0.379952}%
\pgfsetdash{}{0pt}%
\pgfpathmoveto{\pgfqpoint{1.591125in}{3.000482in}}%
\pgfpathcurveto{\pgfqpoint{1.599361in}{3.000482in}}{\pgfqpoint{1.607261in}{3.003754in}}{\pgfqpoint{1.613085in}{3.009578in}}%
\pgfpathcurveto{\pgfqpoint{1.618909in}{3.015402in}}{\pgfqpoint{1.622182in}{3.023302in}}{\pgfqpoint{1.622182in}{3.031538in}}%
\pgfpathcurveto{\pgfqpoint{1.622182in}{3.039775in}}{\pgfqpoint{1.618909in}{3.047675in}}{\pgfqpoint{1.613085in}{3.053499in}}%
\pgfpathcurveto{\pgfqpoint{1.607261in}{3.059323in}}{\pgfqpoint{1.599361in}{3.062595in}}{\pgfqpoint{1.591125in}{3.062595in}}%
\pgfpathcurveto{\pgfqpoint{1.582889in}{3.062595in}}{\pgfqpoint{1.574989in}{3.059323in}}{\pgfqpoint{1.569165in}{3.053499in}}%
\pgfpathcurveto{\pgfqpoint{1.563341in}{3.047675in}}{\pgfqpoint{1.560069in}{3.039775in}}{\pgfqpoint{1.560069in}{3.031538in}}%
\pgfpathcurveto{\pgfqpoint{1.560069in}{3.023302in}}{\pgfqpoint{1.563341in}{3.015402in}}{\pgfqpoint{1.569165in}{3.009578in}}%
\pgfpathcurveto{\pgfqpoint{1.574989in}{3.003754in}}{\pgfqpoint{1.582889in}{3.000482in}}{\pgfqpoint{1.591125in}{3.000482in}}%
\pgfpathclose%
\pgfusepath{stroke,fill}%
\end{pgfscope}%
\begin{pgfscope}%
\pgfpathrectangle{\pgfqpoint{0.100000in}{0.212622in}}{\pgfqpoint{3.696000in}{3.696000in}}%
\pgfusepath{clip}%
\pgfsetbuttcap%
\pgfsetroundjoin%
\definecolor{currentfill}{rgb}{0.121569,0.466667,0.705882}%
\pgfsetfillcolor{currentfill}%
\pgfsetfillopacity{0.380964}%
\pgfsetlinewidth{1.003750pt}%
\definecolor{currentstroke}{rgb}{0.121569,0.466667,0.705882}%
\pgfsetstrokecolor{currentstroke}%
\pgfsetstrokeopacity{0.380964}%
\pgfsetdash{}{0pt}%
\pgfpathmoveto{\pgfqpoint{1.589054in}{2.995998in}}%
\pgfpathcurveto{\pgfqpoint{1.597290in}{2.995998in}}{\pgfqpoint{1.605190in}{2.999270in}}{\pgfqpoint{1.611014in}{3.005094in}}%
\pgfpathcurveto{\pgfqpoint{1.616838in}{3.010918in}}{\pgfqpoint{1.620110in}{3.018818in}}{\pgfqpoint{1.620110in}{3.027054in}}%
\pgfpathcurveto{\pgfqpoint{1.620110in}{3.035290in}}{\pgfqpoint{1.616838in}{3.043190in}}{\pgfqpoint{1.611014in}{3.049014in}}%
\pgfpathcurveto{\pgfqpoint{1.605190in}{3.054838in}}{\pgfqpoint{1.597290in}{3.058111in}}{\pgfqpoint{1.589054in}{3.058111in}}%
\pgfpathcurveto{\pgfqpoint{1.580817in}{3.058111in}}{\pgfqpoint{1.572917in}{3.054838in}}{\pgfqpoint{1.567093in}{3.049014in}}%
\pgfpathcurveto{\pgfqpoint{1.561269in}{3.043190in}}{\pgfqpoint{1.557997in}{3.035290in}}{\pgfqpoint{1.557997in}{3.027054in}}%
\pgfpathcurveto{\pgfqpoint{1.557997in}{3.018818in}}{\pgfqpoint{1.561269in}{3.010918in}}{\pgfqpoint{1.567093in}{3.005094in}}%
\pgfpathcurveto{\pgfqpoint{1.572917in}{2.999270in}}{\pgfqpoint{1.580817in}{2.995998in}}{\pgfqpoint{1.589054in}{2.995998in}}%
\pgfpathclose%
\pgfusepath{stroke,fill}%
\end{pgfscope}%
\begin{pgfscope}%
\pgfpathrectangle{\pgfqpoint{0.100000in}{0.212622in}}{\pgfqpoint{3.696000in}{3.696000in}}%
\pgfusepath{clip}%
\pgfsetbuttcap%
\pgfsetroundjoin%
\definecolor{currentfill}{rgb}{0.121569,0.466667,0.705882}%
\pgfsetfillcolor{currentfill}%
\pgfsetfillopacity{0.381144}%
\pgfsetlinewidth{1.003750pt}%
\definecolor{currentstroke}{rgb}{0.121569,0.466667,0.705882}%
\pgfsetstrokecolor{currentstroke}%
\pgfsetstrokeopacity{0.381144}%
\pgfsetdash{}{0pt}%
\pgfpathmoveto{\pgfqpoint{1.588689in}{2.995200in}}%
\pgfpathcurveto{\pgfqpoint{1.596925in}{2.995200in}}{\pgfqpoint{1.604825in}{2.998472in}}{\pgfqpoint{1.610649in}{3.004296in}}%
\pgfpathcurveto{\pgfqpoint{1.616473in}{3.010120in}}{\pgfqpoint{1.619745in}{3.018020in}}{\pgfqpoint{1.619745in}{3.026256in}}%
\pgfpathcurveto{\pgfqpoint{1.619745in}{3.034492in}}{\pgfqpoint{1.616473in}{3.042393in}}{\pgfqpoint{1.610649in}{3.048216in}}%
\pgfpathcurveto{\pgfqpoint{1.604825in}{3.054040in}}{\pgfqpoint{1.596925in}{3.057313in}}{\pgfqpoint{1.588689in}{3.057313in}}%
\pgfpathcurveto{\pgfqpoint{1.580453in}{3.057313in}}{\pgfqpoint{1.572553in}{3.054040in}}{\pgfqpoint{1.566729in}{3.048216in}}%
\pgfpathcurveto{\pgfqpoint{1.560905in}{3.042393in}}{\pgfqpoint{1.557632in}{3.034492in}}{\pgfqpoint{1.557632in}{3.026256in}}%
\pgfpathcurveto{\pgfqpoint{1.557632in}{3.018020in}}{\pgfqpoint{1.560905in}{3.010120in}}{\pgfqpoint{1.566729in}{3.004296in}}%
\pgfpathcurveto{\pgfqpoint{1.572553in}{2.998472in}}{\pgfqpoint{1.580453in}{2.995200in}}{\pgfqpoint{1.588689in}{2.995200in}}%
\pgfpathclose%
\pgfusepath{stroke,fill}%
\end{pgfscope}%
\begin{pgfscope}%
\pgfpathrectangle{\pgfqpoint{0.100000in}{0.212622in}}{\pgfqpoint{3.696000in}{3.696000in}}%
\pgfusepath{clip}%
\pgfsetbuttcap%
\pgfsetroundjoin%
\definecolor{currentfill}{rgb}{0.121569,0.466667,0.705882}%
\pgfsetfillcolor{currentfill}%
\pgfsetfillopacity{0.381466}%
\pgfsetlinewidth{1.003750pt}%
\definecolor{currentstroke}{rgb}{0.121569,0.466667,0.705882}%
\pgfsetstrokecolor{currentstroke}%
\pgfsetstrokeopacity{0.381466}%
\pgfsetdash{}{0pt}%
\pgfpathmoveto{\pgfqpoint{1.588025in}{2.993730in}}%
\pgfpathcurveto{\pgfqpoint{1.596261in}{2.993730in}}{\pgfqpoint{1.604161in}{2.997002in}}{\pgfqpoint{1.609985in}{3.002826in}}%
\pgfpathcurveto{\pgfqpoint{1.615809in}{3.008650in}}{\pgfqpoint{1.619081in}{3.016550in}}{\pgfqpoint{1.619081in}{3.024786in}}%
\pgfpathcurveto{\pgfqpoint{1.619081in}{3.033023in}}{\pgfqpoint{1.615809in}{3.040923in}}{\pgfqpoint{1.609985in}{3.046747in}}%
\pgfpathcurveto{\pgfqpoint{1.604161in}{3.052571in}}{\pgfqpoint{1.596261in}{3.055843in}}{\pgfqpoint{1.588025in}{3.055843in}}%
\pgfpathcurveto{\pgfqpoint{1.579788in}{3.055843in}}{\pgfqpoint{1.571888in}{3.052571in}}{\pgfqpoint{1.566064in}{3.046747in}}%
\pgfpathcurveto{\pgfqpoint{1.560240in}{3.040923in}}{\pgfqpoint{1.556968in}{3.033023in}}{\pgfqpoint{1.556968in}{3.024786in}}%
\pgfpathcurveto{\pgfqpoint{1.556968in}{3.016550in}}{\pgfqpoint{1.560240in}{3.008650in}}{\pgfqpoint{1.566064in}{3.002826in}}%
\pgfpathcurveto{\pgfqpoint{1.571888in}{2.997002in}}{\pgfqpoint{1.579788in}{2.993730in}}{\pgfqpoint{1.588025in}{2.993730in}}%
\pgfpathclose%
\pgfusepath{stroke,fill}%
\end{pgfscope}%
\begin{pgfscope}%
\pgfpathrectangle{\pgfqpoint{0.100000in}{0.212622in}}{\pgfqpoint{3.696000in}{3.696000in}}%
\pgfusepath{clip}%
\pgfsetbuttcap%
\pgfsetroundjoin%
\definecolor{currentfill}{rgb}{0.121569,0.466667,0.705882}%
\pgfsetfillcolor{currentfill}%
\pgfsetfillopacity{0.382069}%
\pgfsetlinewidth{1.003750pt}%
\definecolor{currentstroke}{rgb}{0.121569,0.466667,0.705882}%
\pgfsetstrokecolor{currentstroke}%
\pgfsetstrokeopacity{0.382069}%
\pgfsetdash{}{0pt}%
\pgfpathmoveto{\pgfqpoint{1.586850in}{2.991092in}}%
\pgfpathcurveto{\pgfqpoint{1.595086in}{2.991092in}}{\pgfqpoint{1.602986in}{2.994364in}}{\pgfqpoint{1.608810in}{3.000188in}}%
\pgfpathcurveto{\pgfqpoint{1.614634in}{3.006012in}}{\pgfqpoint{1.617907in}{3.013912in}}{\pgfqpoint{1.617907in}{3.022148in}}%
\pgfpathcurveto{\pgfqpoint{1.617907in}{3.030385in}}{\pgfqpoint{1.614634in}{3.038285in}}{\pgfqpoint{1.608810in}{3.044109in}}%
\pgfpathcurveto{\pgfqpoint{1.602986in}{3.049933in}}{\pgfqpoint{1.595086in}{3.053205in}}{\pgfqpoint{1.586850in}{3.053205in}}%
\pgfpathcurveto{\pgfqpoint{1.578614in}{3.053205in}}{\pgfqpoint{1.570714in}{3.049933in}}{\pgfqpoint{1.564890in}{3.044109in}}%
\pgfpathcurveto{\pgfqpoint{1.559066in}{3.038285in}}{\pgfqpoint{1.555794in}{3.030385in}}{\pgfqpoint{1.555794in}{3.022148in}}%
\pgfpathcurveto{\pgfqpoint{1.555794in}{3.013912in}}{\pgfqpoint{1.559066in}{3.006012in}}{\pgfqpoint{1.564890in}{3.000188in}}%
\pgfpathcurveto{\pgfqpoint{1.570714in}{2.994364in}}{\pgfqpoint{1.578614in}{2.991092in}}{\pgfqpoint{1.586850in}{2.991092in}}%
\pgfpathclose%
\pgfusepath{stroke,fill}%
\end{pgfscope}%
\begin{pgfscope}%
\pgfpathrectangle{\pgfqpoint{0.100000in}{0.212622in}}{\pgfqpoint{3.696000in}{3.696000in}}%
\pgfusepath{clip}%
\pgfsetbuttcap%
\pgfsetroundjoin%
\definecolor{currentfill}{rgb}{0.121569,0.466667,0.705882}%
\pgfsetfillcolor{currentfill}%
\pgfsetfillopacity{0.382159}%
\pgfsetlinewidth{1.003750pt}%
\definecolor{currentstroke}{rgb}{0.121569,0.466667,0.705882}%
\pgfsetstrokecolor{currentstroke}%
\pgfsetstrokeopacity{0.382159}%
\pgfsetdash{}{0pt}%
\pgfpathmoveto{\pgfqpoint{1.883399in}{3.088659in}}%
\pgfpathcurveto{\pgfqpoint{1.891636in}{3.088659in}}{\pgfqpoint{1.899536in}{3.091931in}}{\pgfqpoint{1.905360in}{3.097755in}}%
\pgfpathcurveto{\pgfqpoint{1.911184in}{3.103579in}}{\pgfqpoint{1.914456in}{3.111479in}}{\pgfqpoint{1.914456in}{3.119716in}}%
\pgfpathcurveto{\pgfqpoint{1.914456in}{3.127952in}}{\pgfqpoint{1.911184in}{3.135852in}}{\pgfqpoint{1.905360in}{3.141676in}}%
\pgfpathcurveto{\pgfqpoint{1.899536in}{3.147500in}}{\pgfqpoint{1.891636in}{3.150772in}}{\pgfqpoint{1.883399in}{3.150772in}}%
\pgfpathcurveto{\pgfqpoint{1.875163in}{3.150772in}}{\pgfqpoint{1.867263in}{3.147500in}}{\pgfqpoint{1.861439in}{3.141676in}}%
\pgfpathcurveto{\pgfqpoint{1.855615in}{3.135852in}}{\pgfqpoint{1.852343in}{3.127952in}}{\pgfqpoint{1.852343in}{3.119716in}}%
\pgfpathcurveto{\pgfqpoint{1.852343in}{3.111479in}}{\pgfqpoint{1.855615in}{3.103579in}}{\pgfqpoint{1.861439in}{3.097755in}}%
\pgfpathcurveto{\pgfqpoint{1.867263in}{3.091931in}}{\pgfqpoint{1.875163in}{3.088659in}}{\pgfqpoint{1.883399in}{3.088659in}}%
\pgfpathclose%
\pgfusepath{stroke,fill}%
\end{pgfscope}%
\begin{pgfscope}%
\pgfpathrectangle{\pgfqpoint{0.100000in}{0.212622in}}{\pgfqpoint{3.696000in}{3.696000in}}%
\pgfusepath{clip}%
\pgfsetbuttcap%
\pgfsetroundjoin%
\definecolor{currentfill}{rgb}{0.121569,0.466667,0.705882}%
\pgfsetfillcolor{currentfill}%
\pgfsetfillopacity{0.382160}%
\pgfsetlinewidth{1.003750pt}%
\definecolor{currentstroke}{rgb}{0.121569,0.466667,0.705882}%
\pgfsetstrokecolor{currentstroke}%
\pgfsetstrokeopacity{0.382160}%
\pgfsetdash{}{0pt}%
\pgfpathmoveto{\pgfqpoint{1.586645in}{2.990659in}}%
\pgfpathcurveto{\pgfqpoint{1.594882in}{2.990659in}}{\pgfqpoint{1.602782in}{2.993931in}}{\pgfqpoint{1.608606in}{2.999755in}}%
\pgfpathcurveto{\pgfqpoint{1.614430in}{3.005579in}}{\pgfqpoint{1.617702in}{3.013479in}}{\pgfqpoint{1.617702in}{3.021716in}}%
\pgfpathcurveto{\pgfqpoint{1.617702in}{3.029952in}}{\pgfqpoint{1.614430in}{3.037852in}}{\pgfqpoint{1.608606in}{3.043676in}}%
\pgfpathcurveto{\pgfqpoint{1.602782in}{3.049500in}}{\pgfqpoint{1.594882in}{3.052772in}}{\pgfqpoint{1.586645in}{3.052772in}}%
\pgfpathcurveto{\pgfqpoint{1.578409in}{3.052772in}}{\pgfqpoint{1.570509in}{3.049500in}}{\pgfqpoint{1.564685in}{3.043676in}}%
\pgfpathcurveto{\pgfqpoint{1.558861in}{3.037852in}}{\pgfqpoint{1.555589in}{3.029952in}}{\pgfqpoint{1.555589in}{3.021716in}}%
\pgfpathcurveto{\pgfqpoint{1.555589in}{3.013479in}}{\pgfqpoint{1.558861in}{3.005579in}}{\pgfqpoint{1.564685in}{2.999755in}}%
\pgfpathcurveto{\pgfqpoint{1.570509in}{2.993931in}}{\pgfqpoint{1.578409in}{2.990659in}}{\pgfqpoint{1.586645in}{2.990659in}}%
\pgfpathclose%
\pgfusepath{stroke,fill}%
\end{pgfscope}%
\begin{pgfscope}%
\pgfpathrectangle{\pgfqpoint{0.100000in}{0.212622in}}{\pgfqpoint{3.696000in}{3.696000in}}%
\pgfusepath{clip}%
\pgfsetbuttcap%
\pgfsetroundjoin%
\definecolor{currentfill}{rgb}{0.121569,0.466667,0.705882}%
\pgfsetfillcolor{currentfill}%
\pgfsetfillopacity{0.382331}%
\pgfsetlinewidth{1.003750pt}%
\definecolor{currentstroke}{rgb}{0.121569,0.466667,0.705882}%
\pgfsetstrokecolor{currentstroke}%
\pgfsetstrokeopacity{0.382331}%
\pgfsetdash{}{0pt}%
\pgfpathmoveto{\pgfqpoint{1.586274in}{2.989886in}}%
\pgfpathcurveto{\pgfqpoint{1.594510in}{2.989886in}}{\pgfqpoint{1.602410in}{2.993159in}}{\pgfqpoint{1.608234in}{2.998983in}}%
\pgfpathcurveto{\pgfqpoint{1.614058in}{3.004807in}}{\pgfqpoint{1.617330in}{3.012707in}}{\pgfqpoint{1.617330in}{3.020943in}}%
\pgfpathcurveto{\pgfqpoint{1.617330in}{3.029179in}}{\pgfqpoint{1.614058in}{3.037079in}}{\pgfqpoint{1.608234in}{3.042903in}}%
\pgfpathcurveto{\pgfqpoint{1.602410in}{3.048727in}}{\pgfqpoint{1.594510in}{3.051999in}}{\pgfqpoint{1.586274in}{3.051999in}}%
\pgfpathcurveto{\pgfqpoint{1.578037in}{3.051999in}}{\pgfqpoint{1.570137in}{3.048727in}}{\pgfqpoint{1.564313in}{3.042903in}}%
\pgfpathcurveto{\pgfqpoint{1.558489in}{3.037079in}}{\pgfqpoint{1.555217in}{3.029179in}}{\pgfqpoint{1.555217in}{3.020943in}}%
\pgfpathcurveto{\pgfqpoint{1.555217in}{3.012707in}}{\pgfqpoint{1.558489in}{3.004807in}}{\pgfqpoint{1.564313in}{2.998983in}}%
\pgfpathcurveto{\pgfqpoint{1.570137in}{2.993159in}}{\pgfqpoint{1.578037in}{2.989886in}}{\pgfqpoint{1.586274in}{2.989886in}}%
\pgfpathclose%
\pgfusepath{stroke,fill}%
\end{pgfscope}%
\begin{pgfscope}%
\pgfpathrectangle{\pgfqpoint{0.100000in}{0.212622in}}{\pgfqpoint{3.696000in}{3.696000in}}%
\pgfusepath{clip}%
\pgfsetbuttcap%
\pgfsetroundjoin%
\definecolor{currentfill}{rgb}{0.121569,0.466667,0.705882}%
\pgfsetfillcolor{currentfill}%
\pgfsetfillopacity{0.382635}%
\pgfsetlinewidth{1.003750pt}%
\definecolor{currentstroke}{rgb}{0.121569,0.466667,0.705882}%
\pgfsetstrokecolor{currentstroke}%
\pgfsetstrokeopacity{0.382635}%
\pgfsetdash{}{0pt}%
\pgfpathmoveto{\pgfqpoint{1.585590in}{2.988465in}}%
\pgfpathcurveto{\pgfqpoint{1.593826in}{2.988465in}}{\pgfqpoint{1.601726in}{2.991738in}}{\pgfqpoint{1.607550in}{2.997562in}}%
\pgfpathcurveto{\pgfqpoint{1.613374in}{3.003386in}}{\pgfqpoint{1.616647in}{3.011286in}}{\pgfqpoint{1.616647in}{3.019522in}}%
\pgfpathcurveto{\pgfqpoint{1.616647in}{3.027758in}}{\pgfqpoint{1.613374in}{3.035658in}}{\pgfqpoint{1.607550in}{3.041482in}}%
\pgfpathcurveto{\pgfqpoint{1.601726in}{3.047306in}}{\pgfqpoint{1.593826in}{3.050578in}}{\pgfqpoint{1.585590in}{3.050578in}}%
\pgfpathcurveto{\pgfqpoint{1.577354in}{3.050578in}}{\pgfqpoint{1.569454in}{3.047306in}}{\pgfqpoint{1.563630in}{3.041482in}}%
\pgfpathcurveto{\pgfqpoint{1.557806in}{3.035658in}}{\pgfqpoint{1.554534in}{3.027758in}}{\pgfqpoint{1.554534in}{3.019522in}}%
\pgfpathcurveto{\pgfqpoint{1.554534in}{3.011286in}}{\pgfqpoint{1.557806in}{3.003386in}}{\pgfqpoint{1.563630in}{2.997562in}}%
\pgfpathcurveto{\pgfqpoint{1.569454in}{2.991738in}}{\pgfqpoint{1.577354in}{2.988465in}}{\pgfqpoint{1.585590in}{2.988465in}}%
\pgfpathclose%
\pgfusepath{stroke,fill}%
\end{pgfscope}%
\begin{pgfscope}%
\pgfpathrectangle{\pgfqpoint{0.100000in}{0.212622in}}{\pgfqpoint{3.696000in}{3.696000in}}%
\pgfusepath{clip}%
\pgfsetbuttcap%
\pgfsetroundjoin%
\definecolor{currentfill}{rgb}{0.121569,0.466667,0.705882}%
\pgfsetfillcolor{currentfill}%
\pgfsetfillopacity{0.383215}%
\pgfsetlinewidth{1.003750pt}%
\definecolor{currentstroke}{rgb}{0.121569,0.466667,0.705882}%
\pgfsetstrokecolor{currentstroke}%
\pgfsetstrokeopacity{0.383215}%
\pgfsetdash{}{0pt}%
\pgfpathmoveto{\pgfqpoint{1.584409in}{2.985921in}}%
\pgfpathcurveto{\pgfqpoint{1.592646in}{2.985921in}}{\pgfqpoint{1.600546in}{2.989194in}}{\pgfqpoint{1.606370in}{2.995018in}}%
\pgfpathcurveto{\pgfqpoint{1.612194in}{3.000841in}}{\pgfqpoint{1.615466in}{3.008742in}}{\pgfqpoint{1.615466in}{3.016978in}}%
\pgfpathcurveto{\pgfqpoint{1.615466in}{3.025214in}}{\pgfqpoint{1.612194in}{3.033114in}}{\pgfqpoint{1.606370in}{3.038938in}}%
\pgfpathcurveto{\pgfqpoint{1.600546in}{3.044762in}}{\pgfqpoint{1.592646in}{3.048034in}}{\pgfqpoint{1.584409in}{3.048034in}}%
\pgfpathcurveto{\pgfqpoint{1.576173in}{3.048034in}}{\pgfqpoint{1.568273in}{3.044762in}}{\pgfqpoint{1.562449in}{3.038938in}}%
\pgfpathcurveto{\pgfqpoint{1.556625in}{3.033114in}}{\pgfqpoint{1.553353in}{3.025214in}}{\pgfqpoint{1.553353in}{3.016978in}}%
\pgfpathcurveto{\pgfqpoint{1.553353in}{3.008742in}}{\pgfqpoint{1.556625in}{3.000841in}}{\pgfqpoint{1.562449in}{2.995018in}}%
\pgfpathcurveto{\pgfqpoint{1.568273in}{2.989194in}}{\pgfqpoint{1.576173in}{2.985921in}}{\pgfqpoint{1.584409in}{2.985921in}}%
\pgfpathclose%
\pgfusepath{stroke,fill}%
\end{pgfscope}%
\begin{pgfscope}%
\pgfpathrectangle{\pgfqpoint{0.100000in}{0.212622in}}{\pgfqpoint{3.696000in}{3.696000in}}%
\pgfusepath{clip}%
\pgfsetbuttcap%
\pgfsetroundjoin%
\definecolor{currentfill}{rgb}{0.121569,0.466667,0.705882}%
\pgfsetfillcolor{currentfill}%
\pgfsetfillopacity{0.384224}%
\pgfsetlinewidth{1.003750pt}%
\definecolor{currentstroke}{rgb}{0.121569,0.466667,0.705882}%
\pgfsetstrokecolor{currentstroke}%
\pgfsetstrokeopacity{0.384224}%
\pgfsetdash{}{0pt}%
\pgfpathmoveto{\pgfqpoint{1.582163in}{2.981226in}}%
\pgfpathcurveto{\pgfqpoint{1.590399in}{2.981226in}}{\pgfqpoint{1.598299in}{2.984498in}}{\pgfqpoint{1.604123in}{2.990322in}}%
\pgfpathcurveto{\pgfqpoint{1.609947in}{2.996146in}}{\pgfqpoint{1.613219in}{3.004046in}}{\pgfqpoint{1.613219in}{3.012283in}}%
\pgfpathcurveto{\pgfqpoint{1.613219in}{3.020519in}}{\pgfqpoint{1.609947in}{3.028419in}}{\pgfqpoint{1.604123in}{3.034243in}}%
\pgfpathcurveto{\pgfqpoint{1.598299in}{3.040067in}}{\pgfqpoint{1.590399in}{3.043339in}}{\pgfqpoint{1.582163in}{3.043339in}}%
\pgfpathcurveto{\pgfqpoint{1.573926in}{3.043339in}}{\pgfqpoint{1.566026in}{3.040067in}}{\pgfqpoint{1.560202in}{3.034243in}}%
\pgfpathcurveto{\pgfqpoint{1.554378in}{3.028419in}}{\pgfqpoint{1.551106in}{3.020519in}}{\pgfqpoint{1.551106in}{3.012283in}}%
\pgfpathcurveto{\pgfqpoint{1.551106in}{3.004046in}}{\pgfqpoint{1.554378in}{2.996146in}}{\pgfqpoint{1.560202in}{2.990322in}}%
\pgfpathcurveto{\pgfqpoint{1.566026in}{2.984498in}}{\pgfqpoint{1.573926in}{2.981226in}}{\pgfqpoint{1.582163in}{2.981226in}}%
\pgfpathclose%
\pgfusepath{stroke,fill}%
\end{pgfscope}%
\begin{pgfscope}%
\pgfpathrectangle{\pgfqpoint{0.100000in}{0.212622in}}{\pgfqpoint{3.696000in}{3.696000in}}%
\pgfusepath{clip}%
\pgfsetbuttcap%
\pgfsetroundjoin%
\definecolor{currentfill}{rgb}{0.121569,0.466667,0.705882}%
\pgfsetfillcolor{currentfill}%
\pgfsetfillopacity{0.386067}%
\pgfsetlinewidth{1.003750pt}%
\definecolor{currentstroke}{rgb}{0.121569,0.466667,0.705882}%
\pgfsetstrokecolor{currentstroke}%
\pgfsetstrokeopacity{0.386067}%
\pgfsetdash{}{0pt}%
\pgfpathmoveto{\pgfqpoint{1.578017in}{2.972771in}}%
\pgfpathcurveto{\pgfqpoint{1.586253in}{2.972771in}}{\pgfqpoint{1.594153in}{2.976044in}}{\pgfqpoint{1.599977in}{2.981868in}}%
\pgfpathcurveto{\pgfqpoint{1.605801in}{2.987692in}}{\pgfqpoint{1.609073in}{2.995592in}}{\pgfqpoint{1.609073in}{3.003828in}}%
\pgfpathcurveto{\pgfqpoint{1.609073in}{3.012064in}}{\pgfqpoint{1.605801in}{3.019964in}}{\pgfqpoint{1.599977in}{3.025788in}}%
\pgfpathcurveto{\pgfqpoint{1.594153in}{3.031612in}}{\pgfqpoint{1.586253in}{3.034884in}}{\pgfqpoint{1.578017in}{3.034884in}}%
\pgfpathcurveto{\pgfqpoint{1.569781in}{3.034884in}}{\pgfqpoint{1.561881in}{3.031612in}}{\pgfqpoint{1.556057in}{3.025788in}}%
\pgfpathcurveto{\pgfqpoint{1.550233in}{3.019964in}}{\pgfqpoint{1.546960in}{3.012064in}}{\pgfqpoint{1.546960in}{3.003828in}}%
\pgfpathcurveto{\pgfqpoint{1.546960in}{2.995592in}}{\pgfqpoint{1.550233in}{2.987692in}}{\pgfqpoint{1.556057in}{2.981868in}}%
\pgfpathcurveto{\pgfqpoint{1.561881in}{2.976044in}}{\pgfqpoint{1.569781in}{2.972771in}}{\pgfqpoint{1.578017in}{2.972771in}}%
\pgfpathclose%
\pgfusepath{stroke,fill}%
\end{pgfscope}%
\begin{pgfscope}%
\pgfpathrectangle{\pgfqpoint{0.100000in}{0.212622in}}{\pgfqpoint{3.696000in}{3.696000in}}%
\pgfusepath{clip}%
\pgfsetbuttcap%
\pgfsetroundjoin%
\definecolor{currentfill}{rgb}{0.121569,0.466667,0.705882}%
\pgfsetfillcolor{currentfill}%
\pgfsetfillopacity{0.387458}%
\pgfsetlinewidth{1.003750pt}%
\definecolor{currentstroke}{rgb}{0.121569,0.466667,0.705882}%
\pgfsetstrokecolor{currentstroke}%
\pgfsetstrokeopacity{0.387458}%
\pgfsetdash{}{0pt}%
\pgfpathmoveto{\pgfqpoint{1.574743in}{2.966097in}}%
\pgfpathcurveto{\pgfqpoint{1.582980in}{2.966097in}}{\pgfqpoint{1.590880in}{2.969369in}}{\pgfqpoint{1.596704in}{2.975193in}}%
\pgfpathcurveto{\pgfqpoint{1.602527in}{2.981017in}}{\pgfqpoint{1.605800in}{2.988917in}}{\pgfqpoint{1.605800in}{2.997153in}}%
\pgfpathcurveto{\pgfqpoint{1.605800in}{3.005390in}}{\pgfqpoint{1.602527in}{3.013290in}}{\pgfqpoint{1.596704in}{3.019114in}}%
\pgfpathcurveto{\pgfqpoint{1.590880in}{3.024937in}}{\pgfqpoint{1.582980in}{3.028210in}}{\pgfqpoint{1.574743in}{3.028210in}}%
\pgfpathcurveto{\pgfqpoint{1.566507in}{3.028210in}}{\pgfqpoint{1.558607in}{3.024937in}}{\pgfqpoint{1.552783in}{3.019114in}}%
\pgfpathcurveto{\pgfqpoint{1.546959in}{3.013290in}}{\pgfqpoint{1.543687in}{3.005390in}}{\pgfqpoint{1.543687in}{2.997153in}}%
\pgfpathcurveto{\pgfqpoint{1.543687in}{2.988917in}}{\pgfqpoint{1.546959in}{2.981017in}}{\pgfqpoint{1.552783in}{2.975193in}}%
\pgfpathcurveto{\pgfqpoint{1.558607in}{2.969369in}}{\pgfqpoint{1.566507in}{2.966097in}}{\pgfqpoint{1.574743in}{2.966097in}}%
\pgfpathclose%
\pgfusepath{stroke,fill}%
\end{pgfscope}%
\begin{pgfscope}%
\pgfpathrectangle{\pgfqpoint{0.100000in}{0.212622in}}{\pgfqpoint{3.696000in}{3.696000in}}%
\pgfusepath{clip}%
\pgfsetbuttcap%
\pgfsetroundjoin%
\definecolor{currentfill}{rgb}{0.121569,0.466667,0.705882}%
\pgfsetfillcolor{currentfill}%
\pgfsetfillopacity{0.388586}%
\pgfsetlinewidth{1.003750pt}%
\definecolor{currentstroke}{rgb}{0.121569,0.466667,0.705882}%
\pgfsetstrokecolor{currentstroke}%
\pgfsetstrokeopacity{0.388586}%
\pgfsetdash{}{0pt}%
\pgfpathmoveto{\pgfqpoint{1.573088in}{2.962080in}}%
\pgfpathcurveto{\pgfqpoint{1.581324in}{2.962080in}}{\pgfqpoint{1.589224in}{2.965353in}}{\pgfqpoint{1.595048in}{2.971177in}}%
\pgfpathcurveto{\pgfqpoint{1.600872in}{2.977000in}}{\pgfqpoint{1.604144in}{2.984901in}}{\pgfqpoint{1.604144in}{2.993137in}}%
\pgfpathcurveto{\pgfqpoint{1.604144in}{3.001373in}}{\pgfqpoint{1.600872in}{3.009273in}}{\pgfqpoint{1.595048in}{3.015097in}}%
\pgfpathcurveto{\pgfqpoint{1.589224in}{3.020921in}}{\pgfqpoint{1.581324in}{3.024193in}}{\pgfqpoint{1.573088in}{3.024193in}}%
\pgfpathcurveto{\pgfqpoint{1.564851in}{3.024193in}}{\pgfqpoint{1.556951in}{3.020921in}}{\pgfqpoint{1.551127in}{3.015097in}}%
\pgfpathcurveto{\pgfqpoint{1.545303in}{3.009273in}}{\pgfqpoint{1.542031in}{3.001373in}}{\pgfqpoint{1.542031in}{2.993137in}}%
\pgfpathcurveto{\pgfqpoint{1.542031in}{2.984901in}}{\pgfqpoint{1.545303in}{2.977000in}}{\pgfqpoint{1.551127in}{2.971177in}}%
\pgfpathcurveto{\pgfqpoint{1.556951in}{2.965353in}}{\pgfqpoint{1.564851in}{2.962080in}}{\pgfqpoint{1.573088in}{2.962080in}}%
\pgfpathclose%
\pgfusepath{stroke,fill}%
\end{pgfscope}%
\begin{pgfscope}%
\pgfpathrectangle{\pgfqpoint{0.100000in}{0.212622in}}{\pgfqpoint{3.696000in}{3.696000in}}%
\pgfusepath{clip}%
\pgfsetbuttcap%
\pgfsetroundjoin%
\definecolor{currentfill}{rgb}{0.121569,0.466667,0.705882}%
\pgfsetfillcolor{currentfill}%
\pgfsetfillopacity{0.389185}%
\pgfsetlinewidth{1.003750pt}%
\definecolor{currentstroke}{rgb}{0.121569,0.466667,0.705882}%
\pgfsetstrokecolor{currentstroke}%
\pgfsetstrokeopacity{0.389185}%
\pgfsetdash{}{0pt}%
\pgfpathmoveto{\pgfqpoint{1.890692in}{3.063463in}}%
\pgfpathcurveto{\pgfqpoint{1.898929in}{3.063463in}}{\pgfqpoint{1.906829in}{3.066735in}}{\pgfqpoint{1.912653in}{3.072559in}}%
\pgfpathcurveto{\pgfqpoint{1.918477in}{3.078383in}}{\pgfqpoint{1.921749in}{3.086283in}}{\pgfqpoint{1.921749in}{3.094519in}}%
\pgfpathcurveto{\pgfqpoint{1.921749in}{3.102755in}}{\pgfqpoint{1.918477in}{3.110656in}}{\pgfqpoint{1.912653in}{3.116479in}}%
\pgfpathcurveto{\pgfqpoint{1.906829in}{3.122303in}}{\pgfqpoint{1.898929in}{3.125576in}}{\pgfqpoint{1.890692in}{3.125576in}}%
\pgfpathcurveto{\pgfqpoint{1.882456in}{3.125576in}}{\pgfqpoint{1.874556in}{3.122303in}}{\pgfqpoint{1.868732in}{3.116479in}}%
\pgfpathcurveto{\pgfqpoint{1.862908in}{3.110656in}}{\pgfqpoint{1.859636in}{3.102755in}}{\pgfqpoint{1.859636in}{3.094519in}}%
\pgfpathcurveto{\pgfqpoint{1.859636in}{3.086283in}}{\pgfqpoint{1.862908in}{3.078383in}}{\pgfqpoint{1.868732in}{3.072559in}}%
\pgfpathcurveto{\pgfqpoint{1.874556in}{3.066735in}}{\pgfqpoint{1.882456in}{3.063463in}}{\pgfqpoint{1.890692in}{3.063463in}}%
\pgfpathclose%
\pgfusepath{stroke,fill}%
\end{pgfscope}%
\begin{pgfscope}%
\pgfpathrectangle{\pgfqpoint{0.100000in}{0.212622in}}{\pgfqpoint{3.696000in}{3.696000in}}%
\pgfusepath{clip}%
\pgfsetbuttcap%
\pgfsetroundjoin%
\definecolor{currentfill}{rgb}{0.121569,0.466667,0.705882}%
\pgfsetfillcolor{currentfill}%
\pgfsetfillopacity{0.390403}%
\pgfsetlinewidth{1.003750pt}%
\definecolor{currentstroke}{rgb}{0.121569,0.466667,0.705882}%
\pgfsetstrokecolor{currentstroke}%
\pgfsetstrokeopacity{0.390403}%
\pgfsetdash{}{0pt}%
\pgfpathmoveto{\pgfqpoint{1.569669in}{2.954239in}}%
\pgfpathcurveto{\pgfqpoint{1.577905in}{2.954239in}}{\pgfqpoint{1.585805in}{2.957512in}}{\pgfqpoint{1.591629in}{2.963336in}}%
\pgfpathcurveto{\pgfqpoint{1.597453in}{2.969160in}}{\pgfqpoint{1.600725in}{2.977060in}}{\pgfqpoint{1.600725in}{2.985296in}}%
\pgfpathcurveto{\pgfqpoint{1.600725in}{2.993532in}}{\pgfqpoint{1.597453in}{3.001432in}}{\pgfqpoint{1.591629in}{3.007256in}}%
\pgfpathcurveto{\pgfqpoint{1.585805in}{3.013080in}}{\pgfqpoint{1.577905in}{3.016352in}}{\pgfqpoint{1.569669in}{3.016352in}}%
\pgfpathcurveto{\pgfqpoint{1.561433in}{3.016352in}}{\pgfqpoint{1.553533in}{3.013080in}}{\pgfqpoint{1.547709in}{3.007256in}}%
\pgfpathcurveto{\pgfqpoint{1.541885in}{3.001432in}}{\pgfqpoint{1.538612in}{2.993532in}}{\pgfqpoint{1.538612in}{2.985296in}}%
\pgfpathcurveto{\pgfqpoint{1.538612in}{2.977060in}}{\pgfqpoint{1.541885in}{2.969160in}}{\pgfqpoint{1.547709in}{2.963336in}}%
\pgfpathcurveto{\pgfqpoint{1.553533in}{2.957512in}}{\pgfqpoint{1.561433in}{2.954239in}}{\pgfqpoint{1.569669in}{2.954239in}}%
\pgfpathclose%
\pgfusepath{stroke,fill}%
\end{pgfscope}%
\begin{pgfscope}%
\pgfpathrectangle{\pgfqpoint{0.100000in}{0.212622in}}{\pgfqpoint{3.696000in}{3.696000in}}%
\pgfusepath{clip}%
\pgfsetbuttcap%
\pgfsetroundjoin%
\definecolor{currentfill}{rgb}{0.121569,0.466667,0.705882}%
\pgfsetfillcolor{currentfill}%
\pgfsetfillopacity{0.391512}%
\pgfsetlinewidth{1.003750pt}%
\definecolor{currentstroke}{rgb}{0.121569,0.466667,0.705882}%
\pgfsetstrokecolor{currentstroke}%
\pgfsetstrokeopacity{0.391512}%
\pgfsetdash{}{0pt}%
\pgfpathmoveto{\pgfqpoint{1.567109in}{2.948571in}}%
\pgfpathcurveto{\pgfqpoint{1.575345in}{2.948571in}}{\pgfqpoint{1.583245in}{2.951843in}}{\pgfqpoint{1.589069in}{2.957667in}}%
\pgfpathcurveto{\pgfqpoint{1.594893in}{2.963491in}}{\pgfqpoint{1.598165in}{2.971391in}}{\pgfqpoint{1.598165in}{2.979627in}}%
\pgfpathcurveto{\pgfqpoint{1.598165in}{2.987863in}}{\pgfqpoint{1.594893in}{2.995764in}}{\pgfqpoint{1.589069in}{3.001587in}}%
\pgfpathcurveto{\pgfqpoint{1.583245in}{3.007411in}}{\pgfqpoint{1.575345in}{3.010684in}}{\pgfqpoint{1.567109in}{3.010684in}}%
\pgfpathcurveto{\pgfqpoint{1.558872in}{3.010684in}}{\pgfqpoint{1.550972in}{3.007411in}}{\pgfqpoint{1.545148in}{3.001587in}}%
\pgfpathcurveto{\pgfqpoint{1.539324in}{2.995764in}}{\pgfqpoint{1.536052in}{2.987863in}}{\pgfqpoint{1.536052in}{2.979627in}}%
\pgfpathcurveto{\pgfqpoint{1.536052in}{2.971391in}}{\pgfqpoint{1.539324in}{2.963491in}}{\pgfqpoint{1.545148in}{2.957667in}}%
\pgfpathcurveto{\pgfqpoint{1.550972in}{2.951843in}}{\pgfqpoint{1.558872in}{2.948571in}}{\pgfqpoint{1.567109in}{2.948571in}}%
\pgfpathclose%
\pgfusepath{stroke,fill}%
\end{pgfscope}%
\begin{pgfscope}%
\pgfpathrectangle{\pgfqpoint{0.100000in}{0.212622in}}{\pgfqpoint{3.696000in}{3.696000in}}%
\pgfusepath{clip}%
\pgfsetbuttcap%
\pgfsetroundjoin%
\definecolor{currentfill}{rgb}{0.121569,0.466667,0.705882}%
\pgfsetfillcolor{currentfill}%
\pgfsetfillopacity{0.392371}%
\pgfsetlinewidth{1.003750pt}%
\definecolor{currentstroke}{rgb}{0.121569,0.466667,0.705882}%
\pgfsetstrokecolor{currentstroke}%
\pgfsetstrokeopacity{0.392371}%
\pgfsetdash{}{0pt}%
\pgfpathmoveto{\pgfqpoint{1.565263in}{2.944683in}}%
\pgfpathcurveto{\pgfqpoint{1.573499in}{2.944683in}}{\pgfqpoint{1.581399in}{2.947955in}}{\pgfqpoint{1.587223in}{2.953779in}}%
\pgfpathcurveto{\pgfqpoint{1.593047in}{2.959603in}}{\pgfqpoint{1.596320in}{2.967503in}}{\pgfqpoint{1.596320in}{2.975739in}}%
\pgfpathcurveto{\pgfqpoint{1.596320in}{2.983976in}}{\pgfqpoint{1.593047in}{2.991876in}}{\pgfqpoint{1.587223in}{2.997700in}}%
\pgfpathcurveto{\pgfqpoint{1.581399in}{3.003524in}}{\pgfqpoint{1.573499in}{3.006796in}}{\pgfqpoint{1.565263in}{3.006796in}}%
\pgfpathcurveto{\pgfqpoint{1.557027in}{3.006796in}}{\pgfqpoint{1.549127in}{3.003524in}}{\pgfqpoint{1.543303in}{2.997700in}}%
\pgfpathcurveto{\pgfqpoint{1.537479in}{2.991876in}}{\pgfqpoint{1.534207in}{2.983976in}}{\pgfqpoint{1.534207in}{2.975739in}}%
\pgfpathcurveto{\pgfqpoint{1.534207in}{2.967503in}}{\pgfqpoint{1.537479in}{2.959603in}}{\pgfqpoint{1.543303in}{2.953779in}}%
\pgfpathcurveto{\pgfqpoint{1.549127in}{2.947955in}}{\pgfqpoint{1.557027in}{2.944683in}}{\pgfqpoint{1.565263in}{2.944683in}}%
\pgfpathclose%
\pgfusepath{stroke,fill}%
\end{pgfscope}%
\begin{pgfscope}%
\pgfpathrectangle{\pgfqpoint{0.100000in}{0.212622in}}{\pgfqpoint{3.696000in}{3.696000in}}%
\pgfusepath{clip}%
\pgfsetbuttcap%
\pgfsetroundjoin%
\definecolor{currentfill}{rgb}{0.121569,0.466667,0.705882}%
\pgfsetfillcolor{currentfill}%
\pgfsetfillopacity{0.392873}%
\pgfsetlinewidth{1.003750pt}%
\definecolor{currentstroke}{rgb}{0.121569,0.466667,0.705882}%
\pgfsetstrokecolor{currentstroke}%
\pgfsetstrokeopacity{0.392873}%
\pgfsetdash{}{0pt}%
\pgfpathmoveto{\pgfqpoint{1.564144in}{2.942307in}}%
\pgfpathcurveto{\pgfqpoint{1.572380in}{2.942307in}}{\pgfqpoint{1.580280in}{2.945579in}}{\pgfqpoint{1.586104in}{2.951403in}}%
\pgfpathcurveto{\pgfqpoint{1.591928in}{2.957227in}}{\pgfqpoint{1.595200in}{2.965127in}}{\pgfqpoint{1.595200in}{2.973363in}}%
\pgfpathcurveto{\pgfqpoint{1.595200in}{2.981600in}}{\pgfqpoint{1.591928in}{2.989500in}}{\pgfqpoint{1.586104in}{2.995324in}}%
\pgfpathcurveto{\pgfqpoint{1.580280in}{3.001148in}}{\pgfqpoint{1.572380in}{3.004420in}}{\pgfqpoint{1.564144in}{3.004420in}}%
\pgfpathcurveto{\pgfqpoint{1.555907in}{3.004420in}}{\pgfqpoint{1.548007in}{3.001148in}}{\pgfqpoint{1.542183in}{2.995324in}}%
\pgfpathcurveto{\pgfqpoint{1.536360in}{2.989500in}}{\pgfqpoint{1.533087in}{2.981600in}}{\pgfqpoint{1.533087in}{2.973363in}}%
\pgfpathcurveto{\pgfqpoint{1.533087in}{2.965127in}}{\pgfqpoint{1.536360in}{2.957227in}}{\pgfqpoint{1.542183in}{2.951403in}}%
\pgfpathcurveto{\pgfqpoint{1.548007in}{2.945579in}}{\pgfqpoint{1.555907in}{2.942307in}}{\pgfqpoint{1.564144in}{2.942307in}}%
\pgfpathclose%
\pgfusepath{stroke,fill}%
\end{pgfscope}%
\begin{pgfscope}%
\pgfpathrectangle{\pgfqpoint{0.100000in}{0.212622in}}{\pgfqpoint{3.696000in}{3.696000in}}%
\pgfusepath{clip}%
\pgfsetbuttcap%
\pgfsetroundjoin%
\definecolor{currentfill}{rgb}{0.121569,0.466667,0.705882}%
\pgfsetfillcolor{currentfill}%
\pgfsetfillopacity{0.393821}%
\pgfsetlinewidth{1.003750pt}%
\definecolor{currentstroke}{rgb}{0.121569,0.466667,0.705882}%
\pgfsetstrokecolor{currentstroke}%
\pgfsetstrokeopacity{0.393821}%
\pgfsetdash{}{0pt}%
\pgfpathmoveto{\pgfqpoint{1.562174in}{2.938049in}}%
\pgfpathcurveto{\pgfqpoint{1.570410in}{2.938049in}}{\pgfqpoint{1.578310in}{2.941322in}}{\pgfqpoint{1.584134in}{2.947145in}}%
\pgfpathcurveto{\pgfqpoint{1.589958in}{2.952969in}}{\pgfqpoint{1.593230in}{2.960869in}}{\pgfqpoint{1.593230in}{2.969106in}}%
\pgfpathcurveto{\pgfqpoint{1.593230in}{2.977342in}}{\pgfqpoint{1.589958in}{2.985242in}}{\pgfqpoint{1.584134in}{2.991066in}}%
\pgfpathcurveto{\pgfqpoint{1.578310in}{2.996890in}}{\pgfqpoint{1.570410in}{3.000162in}}{\pgfqpoint{1.562174in}{3.000162in}}%
\pgfpathcurveto{\pgfqpoint{1.553937in}{3.000162in}}{\pgfqpoint{1.546037in}{2.996890in}}{\pgfqpoint{1.540214in}{2.991066in}}%
\pgfpathcurveto{\pgfqpoint{1.534390in}{2.985242in}}{\pgfqpoint{1.531117in}{2.977342in}}{\pgfqpoint{1.531117in}{2.969106in}}%
\pgfpathcurveto{\pgfqpoint{1.531117in}{2.960869in}}{\pgfqpoint{1.534390in}{2.952969in}}{\pgfqpoint{1.540214in}{2.947145in}}%
\pgfpathcurveto{\pgfqpoint{1.546037in}{2.941322in}}{\pgfqpoint{1.553937in}{2.938049in}}{\pgfqpoint{1.562174in}{2.938049in}}%
\pgfpathclose%
\pgfusepath{stroke,fill}%
\end{pgfscope}%
\begin{pgfscope}%
\pgfpathrectangle{\pgfqpoint{0.100000in}{0.212622in}}{\pgfqpoint{3.696000in}{3.696000in}}%
\pgfusepath{clip}%
\pgfsetbuttcap%
\pgfsetroundjoin%
\definecolor{currentfill}{rgb}{0.121569,0.466667,0.705882}%
\pgfsetfillcolor{currentfill}%
\pgfsetfillopacity{0.394340}%
\pgfsetlinewidth{1.003750pt}%
\definecolor{currentstroke}{rgb}{0.121569,0.466667,0.705882}%
\pgfsetstrokecolor{currentstroke}%
\pgfsetstrokeopacity{0.394340}%
\pgfsetdash{}{0pt}%
\pgfpathmoveto{\pgfqpoint{1.561171in}{2.935849in}}%
\pgfpathcurveto{\pgfqpoint{1.569407in}{2.935849in}}{\pgfqpoint{1.577307in}{2.939121in}}{\pgfqpoint{1.583131in}{2.944945in}}%
\pgfpathcurveto{\pgfqpoint{1.588955in}{2.950769in}}{\pgfqpoint{1.592228in}{2.958669in}}{\pgfqpoint{1.592228in}{2.966906in}}%
\pgfpathcurveto{\pgfqpoint{1.592228in}{2.975142in}}{\pgfqpoint{1.588955in}{2.983042in}}{\pgfqpoint{1.583131in}{2.988866in}}%
\pgfpathcurveto{\pgfqpoint{1.577307in}{2.994690in}}{\pgfqpoint{1.569407in}{2.997962in}}{\pgfqpoint{1.561171in}{2.997962in}}%
\pgfpathcurveto{\pgfqpoint{1.552935in}{2.997962in}}{\pgfqpoint{1.545035in}{2.994690in}}{\pgfqpoint{1.539211in}{2.988866in}}%
\pgfpathcurveto{\pgfqpoint{1.533387in}{2.983042in}}{\pgfqpoint{1.530115in}{2.975142in}}{\pgfqpoint{1.530115in}{2.966906in}}%
\pgfpathcurveto{\pgfqpoint{1.530115in}{2.958669in}}{\pgfqpoint{1.533387in}{2.950769in}}{\pgfqpoint{1.539211in}{2.944945in}}%
\pgfpathcurveto{\pgfqpoint{1.545035in}{2.939121in}}{\pgfqpoint{1.552935in}{2.935849in}}{\pgfqpoint{1.561171in}{2.935849in}}%
\pgfpathclose%
\pgfusepath{stroke,fill}%
\end{pgfscope}%
\begin{pgfscope}%
\pgfpathrectangle{\pgfqpoint{0.100000in}{0.212622in}}{\pgfqpoint{3.696000in}{3.696000in}}%
\pgfusepath{clip}%
\pgfsetbuttcap%
\pgfsetroundjoin%
\definecolor{currentfill}{rgb}{0.121569,0.466667,0.705882}%
\pgfsetfillcolor{currentfill}%
\pgfsetfillopacity{0.395267}%
\pgfsetlinewidth{1.003750pt}%
\definecolor{currentstroke}{rgb}{0.121569,0.466667,0.705882}%
\pgfsetstrokecolor{currentstroke}%
\pgfsetstrokeopacity{0.395267}%
\pgfsetdash{}{0pt}%
\pgfpathmoveto{\pgfqpoint{1.559324in}{2.931805in}}%
\pgfpathcurveto{\pgfqpoint{1.567561in}{2.931805in}}{\pgfqpoint{1.575461in}{2.935078in}}{\pgfqpoint{1.581285in}{2.940902in}}%
\pgfpathcurveto{\pgfqpoint{1.587109in}{2.946726in}}{\pgfqpoint{1.590381in}{2.954626in}}{\pgfqpoint{1.590381in}{2.962862in}}%
\pgfpathcurveto{\pgfqpoint{1.590381in}{2.971098in}}{\pgfqpoint{1.587109in}{2.978998in}}{\pgfqpoint{1.581285in}{2.984822in}}%
\pgfpathcurveto{\pgfqpoint{1.575461in}{2.990646in}}{\pgfqpoint{1.567561in}{2.993918in}}{\pgfqpoint{1.559324in}{2.993918in}}%
\pgfpathcurveto{\pgfqpoint{1.551088in}{2.993918in}}{\pgfqpoint{1.543188in}{2.990646in}}{\pgfqpoint{1.537364in}{2.984822in}}%
\pgfpathcurveto{\pgfqpoint{1.531540in}{2.978998in}}{\pgfqpoint{1.528268in}{2.971098in}}{\pgfqpoint{1.528268in}{2.962862in}}%
\pgfpathcurveto{\pgfqpoint{1.528268in}{2.954626in}}{\pgfqpoint{1.531540in}{2.946726in}}{\pgfqpoint{1.537364in}{2.940902in}}%
\pgfpathcurveto{\pgfqpoint{1.543188in}{2.935078in}}{\pgfqpoint{1.551088in}{2.931805in}}{\pgfqpoint{1.559324in}{2.931805in}}%
\pgfpathclose%
\pgfusepath{stroke,fill}%
\end{pgfscope}%
\begin{pgfscope}%
\pgfpathrectangle{\pgfqpoint{0.100000in}{0.212622in}}{\pgfqpoint{3.696000in}{3.696000in}}%
\pgfusepath{clip}%
\pgfsetbuttcap%
\pgfsetroundjoin%
\definecolor{currentfill}{rgb}{0.121569,0.466667,0.705882}%
\pgfsetfillcolor{currentfill}%
\pgfsetfillopacity{0.395791}%
\pgfsetlinewidth{1.003750pt}%
\definecolor{currentstroke}{rgb}{0.121569,0.466667,0.705882}%
\pgfsetstrokecolor{currentstroke}%
\pgfsetstrokeopacity{0.395791}%
\pgfsetdash{}{0pt}%
\pgfpathmoveto{\pgfqpoint{1.558316in}{2.929538in}}%
\pgfpathcurveto{\pgfqpoint{1.566552in}{2.929538in}}{\pgfqpoint{1.574452in}{2.932811in}}{\pgfqpoint{1.580276in}{2.938634in}}%
\pgfpathcurveto{\pgfqpoint{1.586100in}{2.944458in}}{\pgfqpoint{1.589373in}{2.952358in}}{\pgfqpoint{1.589373in}{2.960595in}}%
\pgfpathcurveto{\pgfqpoint{1.589373in}{2.968831in}}{\pgfqpoint{1.586100in}{2.976731in}}{\pgfqpoint{1.580276in}{2.982555in}}%
\pgfpathcurveto{\pgfqpoint{1.574452in}{2.988379in}}{\pgfqpoint{1.566552in}{2.991651in}}{\pgfqpoint{1.558316in}{2.991651in}}%
\pgfpathcurveto{\pgfqpoint{1.550080in}{2.991651in}}{\pgfqpoint{1.542180in}{2.988379in}}{\pgfqpoint{1.536356in}{2.982555in}}%
\pgfpathcurveto{\pgfqpoint{1.530532in}{2.976731in}}{\pgfqpoint{1.527260in}{2.968831in}}{\pgfqpoint{1.527260in}{2.960595in}}%
\pgfpathcurveto{\pgfqpoint{1.527260in}{2.952358in}}{\pgfqpoint{1.530532in}{2.944458in}}{\pgfqpoint{1.536356in}{2.938634in}}%
\pgfpathcurveto{\pgfqpoint{1.542180in}{2.932811in}}{\pgfqpoint{1.550080in}{2.929538in}}{\pgfqpoint{1.558316in}{2.929538in}}%
\pgfpathclose%
\pgfusepath{stroke,fill}%
\end{pgfscope}%
\begin{pgfscope}%
\pgfpathrectangle{\pgfqpoint{0.100000in}{0.212622in}}{\pgfqpoint{3.696000in}{3.696000in}}%
\pgfusepath{clip}%
\pgfsetbuttcap%
\pgfsetroundjoin%
\definecolor{currentfill}{rgb}{0.121569,0.466667,0.705882}%
\pgfsetfillcolor{currentfill}%
\pgfsetfillopacity{0.396700}%
\pgfsetlinewidth{1.003750pt}%
\definecolor{currentstroke}{rgb}{0.121569,0.466667,0.705882}%
\pgfsetstrokecolor{currentstroke}%
\pgfsetstrokeopacity{0.396700}%
\pgfsetdash{}{0pt}%
\pgfpathmoveto{\pgfqpoint{1.556425in}{2.925303in}}%
\pgfpathcurveto{\pgfqpoint{1.564661in}{2.925303in}}{\pgfqpoint{1.572561in}{2.928575in}}{\pgfqpoint{1.578385in}{2.934399in}}%
\pgfpathcurveto{\pgfqpoint{1.584209in}{2.940223in}}{\pgfqpoint{1.587481in}{2.948123in}}{\pgfqpoint{1.587481in}{2.956359in}}%
\pgfpathcurveto{\pgfqpoint{1.587481in}{2.964595in}}{\pgfqpoint{1.584209in}{2.972495in}}{\pgfqpoint{1.578385in}{2.978319in}}%
\pgfpathcurveto{\pgfqpoint{1.572561in}{2.984143in}}{\pgfqpoint{1.564661in}{2.987416in}}{\pgfqpoint{1.556425in}{2.987416in}}%
\pgfpathcurveto{\pgfqpoint{1.548189in}{2.987416in}}{\pgfqpoint{1.540288in}{2.984143in}}{\pgfqpoint{1.534465in}{2.978319in}}%
\pgfpathcurveto{\pgfqpoint{1.528641in}{2.972495in}}{\pgfqpoint{1.525368in}{2.964595in}}{\pgfqpoint{1.525368in}{2.956359in}}%
\pgfpathcurveto{\pgfqpoint{1.525368in}{2.948123in}}{\pgfqpoint{1.528641in}{2.940223in}}{\pgfqpoint{1.534465in}{2.934399in}}%
\pgfpathcurveto{\pgfqpoint{1.540288in}{2.928575in}}{\pgfqpoint{1.548189in}{2.925303in}}{\pgfqpoint{1.556425in}{2.925303in}}%
\pgfpathclose%
\pgfusepath{stroke,fill}%
\end{pgfscope}%
\begin{pgfscope}%
\pgfpathrectangle{\pgfqpoint{0.100000in}{0.212622in}}{\pgfqpoint{3.696000in}{3.696000in}}%
\pgfusepath{clip}%
\pgfsetbuttcap%
\pgfsetroundjoin%
\definecolor{currentfill}{rgb}{0.121569,0.466667,0.705882}%
\pgfsetfillcolor{currentfill}%
\pgfsetfillopacity{0.397325}%
\pgfsetlinewidth{1.003750pt}%
\definecolor{currentstroke}{rgb}{0.121569,0.466667,0.705882}%
\pgfsetstrokecolor{currentstroke}%
\pgfsetstrokeopacity{0.397325}%
\pgfsetdash{}{0pt}%
\pgfpathmoveto{\pgfqpoint{1.899557in}{3.035901in}}%
\pgfpathcurveto{\pgfqpoint{1.907793in}{3.035901in}}{\pgfqpoint{1.915693in}{3.039174in}}{\pgfqpoint{1.921517in}{3.044998in}}%
\pgfpathcurveto{\pgfqpoint{1.927341in}{3.050822in}}{\pgfqpoint{1.930613in}{3.058722in}}{\pgfqpoint{1.930613in}{3.066958in}}%
\pgfpathcurveto{\pgfqpoint{1.930613in}{3.075194in}}{\pgfqpoint{1.927341in}{3.083094in}}{\pgfqpoint{1.921517in}{3.088918in}}%
\pgfpathcurveto{\pgfqpoint{1.915693in}{3.094742in}}{\pgfqpoint{1.907793in}{3.098014in}}{\pgfqpoint{1.899557in}{3.098014in}}%
\pgfpathcurveto{\pgfqpoint{1.891320in}{3.098014in}}{\pgfqpoint{1.883420in}{3.094742in}}{\pgfqpoint{1.877596in}{3.088918in}}%
\pgfpathcurveto{\pgfqpoint{1.871772in}{3.083094in}}{\pgfqpoint{1.868500in}{3.075194in}}{\pgfqpoint{1.868500in}{3.066958in}}%
\pgfpathcurveto{\pgfqpoint{1.868500in}{3.058722in}}{\pgfqpoint{1.871772in}{3.050822in}}{\pgfqpoint{1.877596in}{3.044998in}}%
\pgfpathcurveto{\pgfqpoint{1.883420in}{3.039174in}}{\pgfqpoint{1.891320in}{3.035901in}}{\pgfqpoint{1.899557in}{3.035901in}}%
\pgfpathclose%
\pgfusepath{stroke,fill}%
\end{pgfscope}%
\begin{pgfscope}%
\pgfpathrectangle{\pgfqpoint{0.100000in}{0.212622in}}{\pgfqpoint{3.696000in}{3.696000in}}%
\pgfusepath{clip}%
\pgfsetbuttcap%
\pgfsetroundjoin%
\definecolor{currentfill}{rgb}{0.121569,0.466667,0.705882}%
\pgfsetfillcolor{currentfill}%
\pgfsetfillopacity{0.398415}%
\pgfsetlinewidth{1.003750pt}%
\definecolor{currentstroke}{rgb}{0.121569,0.466667,0.705882}%
\pgfsetstrokecolor{currentstroke}%
\pgfsetstrokeopacity{0.398415}%
\pgfsetdash{}{0pt}%
\pgfpathmoveto{\pgfqpoint{1.553125in}{2.917697in}}%
\pgfpathcurveto{\pgfqpoint{1.561362in}{2.917697in}}{\pgfqpoint{1.569262in}{2.920970in}}{\pgfqpoint{1.575086in}{2.926794in}}%
\pgfpathcurveto{\pgfqpoint{1.580910in}{2.932618in}}{\pgfqpoint{1.584182in}{2.940518in}}{\pgfqpoint{1.584182in}{2.948754in}}%
\pgfpathcurveto{\pgfqpoint{1.584182in}{2.956990in}}{\pgfqpoint{1.580910in}{2.964890in}}{\pgfqpoint{1.575086in}{2.970714in}}%
\pgfpathcurveto{\pgfqpoint{1.569262in}{2.976538in}}{\pgfqpoint{1.561362in}{2.979810in}}{\pgfqpoint{1.553125in}{2.979810in}}%
\pgfpathcurveto{\pgfqpoint{1.544889in}{2.979810in}}{\pgfqpoint{1.536989in}{2.976538in}}{\pgfqpoint{1.531165in}{2.970714in}}%
\pgfpathcurveto{\pgfqpoint{1.525341in}{2.964890in}}{\pgfqpoint{1.522069in}{2.956990in}}{\pgfqpoint{1.522069in}{2.948754in}}%
\pgfpathcurveto{\pgfqpoint{1.522069in}{2.940518in}}{\pgfqpoint{1.525341in}{2.932618in}}{\pgfqpoint{1.531165in}{2.926794in}}%
\pgfpathcurveto{\pgfqpoint{1.536989in}{2.920970in}}{\pgfqpoint{1.544889in}{2.917697in}}{\pgfqpoint{1.553125in}{2.917697in}}%
\pgfpathclose%
\pgfusepath{stroke,fill}%
\end{pgfscope}%
\begin{pgfscope}%
\pgfpathrectangle{\pgfqpoint{0.100000in}{0.212622in}}{\pgfqpoint{3.696000in}{3.696000in}}%
\pgfusepath{clip}%
\pgfsetbuttcap%
\pgfsetroundjoin%
\definecolor{currentfill}{rgb}{0.121569,0.466667,0.705882}%
\pgfsetfillcolor{currentfill}%
\pgfsetfillopacity{0.401490}%
\pgfsetlinewidth{1.003750pt}%
\definecolor{currentstroke}{rgb}{0.121569,0.466667,0.705882}%
\pgfsetstrokecolor{currentstroke}%
\pgfsetstrokeopacity{0.401490}%
\pgfsetdash{}{0pt}%
\pgfpathmoveto{\pgfqpoint{1.547052in}{2.903764in}}%
\pgfpathcurveto{\pgfqpoint{1.555288in}{2.903764in}}{\pgfqpoint{1.563188in}{2.907036in}}{\pgfqpoint{1.569012in}{2.912860in}}%
\pgfpathcurveto{\pgfqpoint{1.574836in}{2.918684in}}{\pgfqpoint{1.578108in}{2.926584in}}{\pgfqpoint{1.578108in}{2.934820in}}%
\pgfpathcurveto{\pgfqpoint{1.578108in}{2.943057in}}{\pgfqpoint{1.574836in}{2.950957in}}{\pgfqpoint{1.569012in}{2.956781in}}%
\pgfpathcurveto{\pgfqpoint{1.563188in}{2.962604in}}{\pgfqpoint{1.555288in}{2.965877in}}{\pgfqpoint{1.547052in}{2.965877in}}%
\pgfpathcurveto{\pgfqpoint{1.538815in}{2.965877in}}{\pgfqpoint{1.530915in}{2.962604in}}{\pgfqpoint{1.525091in}{2.956781in}}%
\pgfpathcurveto{\pgfqpoint{1.519268in}{2.950957in}}{\pgfqpoint{1.515995in}{2.943057in}}{\pgfqpoint{1.515995in}{2.934820in}}%
\pgfpathcurveto{\pgfqpoint{1.515995in}{2.926584in}}{\pgfqpoint{1.519268in}{2.918684in}}{\pgfqpoint{1.525091in}{2.912860in}}%
\pgfpathcurveto{\pgfqpoint{1.530915in}{2.907036in}}{\pgfqpoint{1.538815in}{2.903764in}}{\pgfqpoint{1.547052in}{2.903764in}}%
\pgfpathclose%
\pgfusepath{stroke,fill}%
\end{pgfscope}%
\begin{pgfscope}%
\pgfpathrectangle{\pgfqpoint{0.100000in}{0.212622in}}{\pgfqpoint{3.696000in}{3.696000in}}%
\pgfusepath{clip}%
\pgfsetbuttcap%
\pgfsetroundjoin%
\definecolor{currentfill}{rgb}{0.121569,0.466667,0.705882}%
\pgfsetfillcolor{currentfill}%
\pgfsetfillopacity{0.403994}%
\pgfsetlinewidth{1.003750pt}%
\definecolor{currentstroke}{rgb}{0.121569,0.466667,0.705882}%
\pgfsetstrokecolor{currentstroke}%
\pgfsetstrokeopacity{0.403994}%
\pgfsetdash{}{0pt}%
\pgfpathmoveto{\pgfqpoint{1.542254in}{2.892636in}}%
\pgfpathcurveto{\pgfqpoint{1.550490in}{2.892636in}}{\pgfqpoint{1.558391in}{2.895908in}}{\pgfqpoint{1.564214in}{2.901732in}}%
\pgfpathcurveto{\pgfqpoint{1.570038in}{2.907556in}}{\pgfqpoint{1.573311in}{2.915456in}}{\pgfqpoint{1.573311in}{2.923692in}}%
\pgfpathcurveto{\pgfqpoint{1.573311in}{2.931929in}}{\pgfqpoint{1.570038in}{2.939829in}}{\pgfqpoint{1.564214in}{2.945653in}}%
\pgfpathcurveto{\pgfqpoint{1.558391in}{2.951477in}}{\pgfqpoint{1.550490in}{2.954749in}}{\pgfqpoint{1.542254in}{2.954749in}}%
\pgfpathcurveto{\pgfqpoint{1.534018in}{2.954749in}}{\pgfqpoint{1.526118in}{2.951477in}}{\pgfqpoint{1.520294in}{2.945653in}}%
\pgfpathcurveto{\pgfqpoint{1.514470in}{2.939829in}}{\pgfqpoint{1.511198in}{2.931929in}}{\pgfqpoint{1.511198in}{2.923692in}}%
\pgfpathcurveto{\pgfqpoint{1.511198in}{2.915456in}}{\pgfqpoint{1.514470in}{2.907556in}}{\pgfqpoint{1.520294in}{2.901732in}}%
\pgfpathcurveto{\pgfqpoint{1.526118in}{2.895908in}}{\pgfqpoint{1.534018in}{2.892636in}}{\pgfqpoint{1.542254in}{2.892636in}}%
\pgfpathclose%
\pgfusepath{stroke,fill}%
\end{pgfscope}%
\begin{pgfscope}%
\pgfpathrectangle{\pgfqpoint{0.100000in}{0.212622in}}{\pgfqpoint{3.696000in}{3.696000in}}%
\pgfusepath{clip}%
\pgfsetbuttcap%
\pgfsetroundjoin%
\definecolor{currentfill}{rgb}{0.121569,0.466667,0.705882}%
\pgfsetfillcolor{currentfill}%
\pgfsetfillopacity{0.406455}%
\pgfsetlinewidth{1.003750pt}%
\definecolor{currentstroke}{rgb}{0.121569,0.466667,0.705882}%
\pgfsetstrokecolor{currentstroke}%
\pgfsetstrokeopacity{0.406455}%
\pgfsetdash{}{0pt}%
\pgfpathmoveto{\pgfqpoint{1.909322in}{3.002688in}}%
\pgfpathcurveto{\pgfqpoint{1.917558in}{3.002688in}}{\pgfqpoint{1.925458in}{3.005960in}}{\pgfqpoint{1.931282in}{3.011784in}}%
\pgfpathcurveto{\pgfqpoint{1.937106in}{3.017608in}}{\pgfqpoint{1.940378in}{3.025508in}}{\pgfqpoint{1.940378in}{3.033744in}}%
\pgfpathcurveto{\pgfqpoint{1.940378in}{3.041981in}}{\pgfqpoint{1.937106in}{3.049881in}}{\pgfqpoint{1.931282in}{3.055705in}}%
\pgfpathcurveto{\pgfqpoint{1.925458in}{3.061529in}}{\pgfqpoint{1.917558in}{3.064801in}}{\pgfqpoint{1.909322in}{3.064801in}}%
\pgfpathcurveto{\pgfqpoint{1.901085in}{3.064801in}}{\pgfqpoint{1.893185in}{3.061529in}}{\pgfqpoint{1.887361in}{3.055705in}}%
\pgfpathcurveto{\pgfqpoint{1.881537in}{3.049881in}}{\pgfqpoint{1.878265in}{3.041981in}}{\pgfqpoint{1.878265in}{3.033744in}}%
\pgfpathcurveto{\pgfqpoint{1.878265in}{3.025508in}}{\pgfqpoint{1.881537in}{3.017608in}}{\pgfqpoint{1.887361in}{3.011784in}}%
\pgfpathcurveto{\pgfqpoint{1.893185in}{3.005960in}}{\pgfqpoint{1.901085in}{3.002688in}}{\pgfqpoint{1.909322in}{3.002688in}}%
\pgfpathclose%
\pgfusepath{stroke,fill}%
\end{pgfscope}%
\begin{pgfscope}%
\pgfpathrectangle{\pgfqpoint{0.100000in}{0.212622in}}{\pgfqpoint{3.696000in}{3.696000in}}%
\pgfusepath{clip}%
\pgfsetbuttcap%
\pgfsetroundjoin%
\definecolor{currentfill}{rgb}{0.121569,0.466667,0.705882}%
\pgfsetfillcolor{currentfill}%
\pgfsetfillopacity{0.408290}%
\pgfsetlinewidth{1.003750pt}%
\definecolor{currentstroke}{rgb}{0.121569,0.466667,0.705882}%
\pgfsetstrokecolor{currentstroke}%
\pgfsetstrokeopacity{0.408290}%
\pgfsetdash{}{0pt}%
\pgfpathmoveto{\pgfqpoint{1.532791in}{2.872126in}}%
\pgfpathcurveto{\pgfqpoint{1.541027in}{2.872126in}}{\pgfqpoint{1.548927in}{2.875398in}}{\pgfqpoint{1.554751in}{2.881222in}}%
\pgfpathcurveto{\pgfqpoint{1.560575in}{2.887046in}}{\pgfqpoint{1.563847in}{2.894946in}}{\pgfqpoint{1.563847in}{2.903182in}}%
\pgfpathcurveto{\pgfqpoint{1.563847in}{2.911419in}}{\pgfqpoint{1.560575in}{2.919319in}}{\pgfqpoint{1.554751in}{2.925143in}}%
\pgfpathcurveto{\pgfqpoint{1.548927in}{2.930967in}}{\pgfqpoint{1.541027in}{2.934239in}}{\pgfqpoint{1.532791in}{2.934239in}}%
\pgfpathcurveto{\pgfqpoint{1.524555in}{2.934239in}}{\pgfqpoint{1.516654in}{2.930967in}}{\pgfqpoint{1.510831in}{2.925143in}}%
\pgfpathcurveto{\pgfqpoint{1.505007in}{2.919319in}}{\pgfqpoint{1.501734in}{2.911419in}}{\pgfqpoint{1.501734in}{2.903182in}}%
\pgfpathcurveto{\pgfqpoint{1.501734in}{2.894946in}}{\pgfqpoint{1.505007in}{2.887046in}}{\pgfqpoint{1.510831in}{2.881222in}}%
\pgfpathcurveto{\pgfqpoint{1.516654in}{2.875398in}}{\pgfqpoint{1.524555in}{2.872126in}}{\pgfqpoint{1.532791in}{2.872126in}}%
\pgfpathclose%
\pgfusepath{stroke,fill}%
\end{pgfscope}%
\begin{pgfscope}%
\pgfpathrectangle{\pgfqpoint{0.100000in}{0.212622in}}{\pgfqpoint{3.696000in}{3.696000in}}%
\pgfusepath{clip}%
\pgfsetbuttcap%
\pgfsetroundjoin%
\definecolor{currentfill}{rgb}{0.121569,0.466667,0.705882}%
\pgfsetfillcolor{currentfill}%
\pgfsetfillopacity{0.412062}%
\pgfsetlinewidth{1.003750pt}%
\definecolor{currentstroke}{rgb}{0.121569,0.466667,0.705882}%
\pgfsetstrokecolor{currentstroke}%
\pgfsetstrokeopacity{0.412062}%
\pgfsetdash{}{0pt}%
\pgfpathmoveto{\pgfqpoint{1.525110in}{2.855211in}}%
\pgfpathcurveto{\pgfqpoint{1.533347in}{2.855211in}}{\pgfqpoint{1.541247in}{2.858483in}}{\pgfqpoint{1.547071in}{2.864307in}}%
\pgfpathcurveto{\pgfqpoint{1.552895in}{2.870131in}}{\pgfqpoint{1.556167in}{2.878031in}}{\pgfqpoint{1.556167in}{2.886267in}}%
\pgfpathcurveto{\pgfqpoint{1.556167in}{2.894504in}}{\pgfqpoint{1.552895in}{2.902404in}}{\pgfqpoint{1.547071in}{2.908228in}}%
\pgfpathcurveto{\pgfqpoint{1.541247in}{2.914051in}}{\pgfqpoint{1.533347in}{2.917324in}}{\pgfqpoint{1.525110in}{2.917324in}}%
\pgfpathcurveto{\pgfqpoint{1.516874in}{2.917324in}}{\pgfqpoint{1.508974in}{2.914051in}}{\pgfqpoint{1.503150in}{2.908228in}}%
\pgfpathcurveto{\pgfqpoint{1.497326in}{2.902404in}}{\pgfqpoint{1.494054in}{2.894504in}}{\pgfqpoint{1.494054in}{2.886267in}}%
\pgfpathcurveto{\pgfqpoint{1.494054in}{2.878031in}}{\pgfqpoint{1.497326in}{2.870131in}}{\pgfqpoint{1.503150in}{2.864307in}}%
\pgfpathcurveto{\pgfqpoint{1.508974in}{2.858483in}}{\pgfqpoint{1.516874in}{2.855211in}}{\pgfqpoint{1.525110in}{2.855211in}}%
\pgfpathclose%
\pgfusepath{stroke,fill}%
\end{pgfscope}%
\begin{pgfscope}%
\pgfpathrectangle{\pgfqpoint{0.100000in}{0.212622in}}{\pgfqpoint{3.696000in}{3.696000in}}%
\pgfusepath{clip}%
\pgfsetbuttcap%
\pgfsetroundjoin%
\definecolor{currentfill}{rgb}{0.121569,0.466667,0.705882}%
\pgfsetfillcolor{currentfill}%
\pgfsetfillopacity{0.416802}%
\pgfsetlinewidth{1.003750pt}%
\definecolor{currentstroke}{rgb}{0.121569,0.466667,0.705882}%
\pgfsetstrokecolor{currentstroke}%
\pgfsetstrokeopacity{0.416802}%
\pgfsetdash{}{0pt}%
\pgfpathmoveto{\pgfqpoint{1.922033in}{2.967625in}}%
\pgfpathcurveto{\pgfqpoint{1.930269in}{2.967625in}}{\pgfqpoint{1.938169in}{2.970897in}}{\pgfqpoint{1.943993in}{2.976721in}}%
\pgfpathcurveto{\pgfqpoint{1.949817in}{2.982545in}}{\pgfqpoint{1.953089in}{2.990445in}}{\pgfqpoint{1.953089in}{2.998681in}}%
\pgfpathcurveto{\pgfqpoint{1.953089in}{3.006917in}}{\pgfqpoint{1.949817in}{3.014817in}}{\pgfqpoint{1.943993in}{3.020641in}}%
\pgfpathcurveto{\pgfqpoint{1.938169in}{3.026465in}}{\pgfqpoint{1.930269in}{3.029738in}}{\pgfqpoint{1.922033in}{3.029738in}}%
\pgfpathcurveto{\pgfqpoint{1.913797in}{3.029738in}}{\pgfqpoint{1.905897in}{3.026465in}}{\pgfqpoint{1.900073in}{3.020641in}}%
\pgfpathcurveto{\pgfqpoint{1.894249in}{3.014817in}}{\pgfqpoint{1.890976in}{3.006917in}}{\pgfqpoint{1.890976in}{2.998681in}}%
\pgfpathcurveto{\pgfqpoint{1.890976in}{2.990445in}}{\pgfqpoint{1.894249in}{2.982545in}}{\pgfqpoint{1.900073in}{2.976721in}}%
\pgfpathcurveto{\pgfqpoint{1.905897in}{2.970897in}}{\pgfqpoint{1.913797in}{2.967625in}}{\pgfqpoint{1.922033in}{2.967625in}}%
\pgfpathclose%
\pgfusepath{stroke,fill}%
\end{pgfscope}%
\begin{pgfscope}%
\pgfpathrectangle{\pgfqpoint{0.100000in}{0.212622in}}{\pgfqpoint{3.696000in}{3.696000in}}%
\pgfusepath{clip}%
\pgfsetbuttcap%
\pgfsetroundjoin%
\definecolor{currentfill}{rgb}{0.121569,0.466667,0.705882}%
\pgfsetfillcolor{currentfill}%
\pgfsetfillopacity{0.418696}%
\pgfsetlinewidth{1.003750pt}%
\definecolor{currentstroke}{rgb}{0.121569,0.466667,0.705882}%
\pgfsetstrokecolor{currentstroke}%
\pgfsetstrokeopacity{0.418696}%
\pgfsetdash{}{0pt}%
\pgfpathmoveto{\pgfqpoint{1.510401in}{2.824280in}}%
\pgfpathcurveto{\pgfqpoint{1.518637in}{2.824280in}}{\pgfqpoint{1.526537in}{2.827552in}}{\pgfqpoint{1.532361in}{2.833376in}}%
\pgfpathcurveto{\pgfqpoint{1.538185in}{2.839200in}}{\pgfqpoint{1.541458in}{2.847100in}}{\pgfqpoint{1.541458in}{2.855336in}}%
\pgfpathcurveto{\pgfqpoint{1.541458in}{2.863572in}}{\pgfqpoint{1.538185in}{2.871472in}}{\pgfqpoint{1.532361in}{2.877296in}}%
\pgfpathcurveto{\pgfqpoint{1.526537in}{2.883120in}}{\pgfqpoint{1.518637in}{2.886393in}}{\pgfqpoint{1.510401in}{2.886393in}}%
\pgfpathcurveto{\pgfqpoint{1.502165in}{2.886393in}}{\pgfqpoint{1.494265in}{2.883120in}}{\pgfqpoint{1.488441in}{2.877296in}}%
\pgfpathcurveto{\pgfqpoint{1.482617in}{2.871472in}}{\pgfqpoint{1.479345in}{2.863572in}}{\pgfqpoint{1.479345in}{2.855336in}}%
\pgfpathcurveto{\pgfqpoint{1.479345in}{2.847100in}}{\pgfqpoint{1.482617in}{2.839200in}}{\pgfqpoint{1.488441in}{2.833376in}}%
\pgfpathcurveto{\pgfqpoint{1.494265in}{2.827552in}}{\pgfqpoint{1.502165in}{2.824280in}}{\pgfqpoint{1.510401in}{2.824280in}}%
\pgfpathclose%
\pgfusepath{stroke,fill}%
\end{pgfscope}%
\begin{pgfscope}%
\pgfpathrectangle{\pgfqpoint{0.100000in}{0.212622in}}{\pgfqpoint{3.696000in}{3.696000in}}%
\pgfusepath{clip}%
\pgfsetbuttcap%
\pgfsetroundjoin%
\definecolor{currentfill}{rgb}{0.121569,0.466667,0.705882}%
\pgfsetfillcolor{currentfill}%
\pgfsetfillopacity{0.425203}%
\pgfsetlinewidth{1.003750pt}%
\definecolor{currentstroke}{rgb}{0.121569,0.466667,0.705882}%
\pgfsetstrokecolor{currentstroke}%
\pgfsetstrokeopacity{0.425203}%
\pgfsetdash{}{0pt}%
\pgfpathmoveto{\pgfqpoint{1.497090in}{2.795010in}}%
\pgfpathcurveto{\pgfqpoint{1.505326in}{2.795010in}}{\pgfqpoint{1.513226in}{2.798282in}}{\pgfqpoint{1.519050in}{2.804106in}}%
\pgfpathcurveto{\pgfqpoint{1.524874in}{2.809930in}}{\pgfqpoint{1.528146in}{2.817830in}}{\pgfqpoint{1.528146in}{2.826066in}}%
\pgfpathcurveto{\pgfqpoint{1.528146in}{2.834303in}}{\pgfqpoint{1.524874in}{2.842203in}}{\pgfqpoint{1.519050in}{2.848027in}}%
\pgfpathcurveto{\pgfqpoint{1.513226in}{2.853851in}}{\pgfqpoint{1.505326in}{2.857123in}}{\pgfqpoint{1.497090in}{2.857123in}}%
\pgfpathcurveto{\pgfqpoint{1.488854in}{2.857123in}}{\pgfqpoint{1.480954in}{2.853851in}}{\pgfqpoint{1.475130in}{2.848027in}}%
\pgfpathcurveto{\pgfqpoint{1.469306in}{2.842203in}}{\pgfqpoint{1.466033in}{2.834303in}}{\pgfqpoint{1.466033in}{2.826066in}}%
\pgfpathcurveto{\pgfqpoint{1.466033in}{2.817830in}}{\pgfqpoint{1.469306in}{2.809930in}}{\pgfqpoint{1.475130in}{2.804106in}}%
\pgfpathcurveto{\pgfqpoint{1.480954in}{2.798282in}}{\pgfqpoint{1.488854in}{2.795010in}}{\pgfqpoint{1.497090in}{2.795010in}}%
\pgfpathclose%
\pgfusepath{stroke,fill}%
\end{pgfscope}%
\begin{pgfscope}%
\pgfpathrectangle{\pgfqpoint{0.100000in}{0.212622in}}{\pgfqpoint{3.696000in}{3.696000in}}%
\pgfusepath{clip}%
\pgfsetbuttcap%
\pgfsetroundjoin%
\definecolor{currentfill}{rgb}{0.121569,0.466667,0.705882}%
\pgfsetfillcolor{currentfill}%
\pgfsetfillopacity{0.428119}%
\pgfsetlinewidth{1.003750pt}%
\definecolor{currentstroke}{rgb}{0.121569,0.466667,0.705882}%
\pgfsetstrokecolor{currentstroke}%
\pgfsetstrokeopacity{0.428119}%
\pgfsetdash{}{0pt}%
\pgfpathmoveto{\pgfqpoint{1.934223in}{2.924155in}}%
\pgfpathcurveto{\pgfqpoint{1.942459in}{2.924155in}}{\pgfqpoint{1.950359in}{2.927428in}}{\pgfqpoint{1.956183in}{2.933252in}}%
\pgfpathcurveto{\pgfqpoint{1.962007in}{2.939075in}}{\pgfqpoint{1.965279in}{2.946976in}}{\pgfqpoint{1.965279in}{2.955212in}}%
\pgfpathcurveto{\pgfqpoint{1.965279in}{2.963448in}}{\pgfqpoint{1.962007in}{2.971348in}}{\pgfqpoint{1.956183in}{2.977172in}}%
\pgfpathcurveto{\pgfqpoint{1.950359in}{2.982996in}}{\pgfqpoint{1.942459in}{2.986268in}}{\pgfqpoint{1.934223in}{2.986268in}}%
\pgfpathcurveto{\pgfqpoint{1.925986in}{2.986268in}}{\pgfqpoint{1.918086in}{2.982996in}}{\pgfqpoint{1.912262in}{2.977172in}}%
\pgfpathcurveto{\pgfqpoint{1.906438in}{2.971348in}}{\pgfqpoint{1.903166in}{2.963448in}}{\pgfqpoint{1.903166in}{2.955212in}}%
\pgfpathcurveto{\pgfqpoint{1.903166in}{2.946976in}}{\pgfqpoint{1.906438in}{2.939075in}}{\pgfqpoint{1.912262in}{2.933252in}}%
\pgfpathcurveto{\pgfqpoint{1.918086in}{2.927428in}}{\pgfqpoint{1.925986in}{2.924155in}}{\pgfqpoint{1.934223in}{2.924155in}}%
\pgfpathclose%
\pgfusepath{stroke,fill}%
\end{pgfscope}%
\begin{pgfscope}%
\pgfpathrectangle{\pgfqpoint{0.100000in}{0.212622in}}{\pgfqpoint{3.696000in}{3.696000in}}%
\pgfusepath{clip}%
\pgfsetbuttcap%
\pgfsetroundjoin%
\definecolor{currentfill}{rgb}{0.121569,0.466667,0.705882}%
\pgfsetfillcolor{currentfill}%
\pgfsetfillopacity{0.434846}%
\pgfsetlinewidth{1.003750pt}%
\definecolor{currentstroke}{rgb}{0.121569,0.466667,0.705882}%
\pgfsetstrokecolor{currentstroke}%
\pgfsetstrokeopacity{0.434846}%
\pgfsetdash{}{0pt}%
\pgfpathmoveto{\pgfqpoint{1.943664in}{2.902240in}}%
\pgfpathcurveto{\pgfqpoint{1.951900in}{2.902240in}}{\pgfqpoint{1.959800in}{2.905512in}}{\pgfqpoint{1.965624in}{2.911336in}}%
\pgfpathcurveto{\pgfqpoint{1.971448in}{2.917160in}}{\pgfqpoint{1.974720in}{2.925060in}}{\pgfqpoint{1.974720in}{2.933296in}}%
\pgfpathcurveto{\pgfqpoint{1.974720in}{2.941533in}}{\pgfqpoint{1.971448in}{2.949433in}}{\pgfqpoint{1.965624in}{2.955257in}}%
\pgfpathcurveto{\pgfqpoint{1.959800in}{2.961080in}}{\pgfqpoint{1.951900in}{2.964353in}}{\pgfqpoint{1.943664in}{2.964353in}}%
\pgfpathcurveto{\pgfqpoint{1.935427in}{2.964353in}}{\pgfqpoint{1.927527in}{2.961080in}}{\pgfqpoint{1.921703in}{2.955257in}}%
\pgfpathcurveto{\pgfqpoint{1.915879in}{2.949433in}}{\pgfqpoint{1.912607in}{2.941533in}}{\pgfqpoint{1.912607in}{2.933296in}}%
\pgfpathcurveto{\pgfqpoint{1.912607in}{2.925060in}}{\pgfqpoint{1.915879in}{2.917160in}}{\pgfqpoint{1.921703in}{2.911336in}}%
\pgfpathcurveto{\pgfqpoint{1.927527in}{2.905512in}}{\pgfqpoint{1.935427in}{2.902240in}}{\pgfqpoint{1.943664in}{2.902240in}}%
\pgfpathclose%
\pgfusepath{stroke,fill}%
\end{pgfscope}%
\begin{pgfscope}%
\pgfpathrectangle{\pgfqpoint{0.100000in}{0.212622in}}{\pgfqpoint{3.696000in}{3.696000in}}%
\pgfusepath{clip}%
\pgfsetbuttcap%
\pgfsetroundjoin%
\definecolor{currentfill}{rgb}{0.121569,0.466667,0.705882}%
\pgfsetfillcolor{currentfill}%
\pgfsetfillopacity{0.436632}%
\pgfsetlinewidth{1.003750pt}%
\definecolor{currentstroke}{rgb}{0.121569,0.466667,0.705882}%
\pgfsetstrokecolor{currentstroke}%
\pgfsetstrokeopacity{0.436632}%
\pgfsetdash{}{0pt}%
\pgfpathmoveto{\pgfqpoint{1.471727in}{2.741186in}}%
\pgfpathcurveto{\pgfqpoint{1.479963in}{2.741186in}}{\pgfqpoint{1.487863in}{2.744458in}}{\pgfqpoint{1.493687in}{2.750282in}}%
\pgfpathcurveto{\pgfqpoint{1.499511in}{2.756106in}}{\pgfqpoint{1.502783in}{2.764006in}}{\pgfqpoint{1.502783in}{2.772243in}}%
\pgfpathcurveto{\pgfqpoint{1.502783in}{2.780479in}}{\pgfqpoint{1.499511in}{2.788379in}}{\pgfqpoint{1.493687in}{2.794203in}}%
\pgfpathcurveto{\pgfqpoint{1.487863in}{2.800027in}}{\pgfqpoint{1.479963in}{2.803299in}}{\pgfqpoint{1.471727in}{2.803299in}}%
\pgfpathcurveto{\pgfqpoint{1.463491in}{2.803299in}}{\pgfqpoint{1.455591in}{2.800027in}}{\pgfqpoint{1.449767in}{2.794203in}}%
\pgfpathcurveto{\pgfqpoint{1.443943in}{2.788379in}}{\pgfqpoint{1.440670in}{2.780479in}}{\pgfqpoint{1.440670in}{2.772243in}}%
\pgfpathcurveto{\pgfqpoint{1.440670in}{2.764006in}}{\pgfqpoint{1.443943in}{2.756106in}}{\pgfqpoint{1.449767in}{2.750282in}}%
\pgfpathcurveto{\pgfqpoint{1.455591in}{2.744458in}}{\pgfqpoint{1.463491in}{2.741186in}}{\pgfqpoint{1.471727in}{2.741186in}}%
\pgfpathclose%
\pgfusepath{stroke,fill}%
\end{pgfscope}%
\begin{pgfscope}%
\pgfpathrectangle{\pgfqpoint{0.100000in}{0.212622in}}{\pgfqpoint{3.696000in}{3.696000in}}%
\pgfusepath{clip}%
\pgfsetbuttcap%
\pgfsetroundjoin%
\definecolor{currentfill}{rgb}{0.121569,0.466667,0.705882}%
\pgfsetfillcolor{currentfill}%
\pgfsetfillopacity{0.442776}%
\pgfsetlinewidth{1.003750pt}%
\definecolor{currentstroke}{rgb}{0.121569,0.466667,0.705882}%
\pgfsetstrokecolor{currentstroke}%
\pgfsetstrokeopacity{0.442776}%
\pgfsetdash{}{0pt}%
\pgfpathmoveto{\pgfqpoint{1.953462in}{2.873355in}}%
\pgfpathcurveto{\pgfqpoint{1.961699in}{2.873355in}}{\pgfqpoint{1.969599in}{2.876628in}}{\pgfqpoint{1.975423in}{2.882452in}}%
\pgfpathcurveto{\pgfqpoint{1.981247in}{2.888276in}}{\pgfqpoint{1.984519in}{2.896176in}}{\pgfqpoint{1.984519in}{2.904412in}}%
\pgfpathcurveto{\pgfqpoint{1.984519in}{2.912648in}}{\pgfqpoint{1.981247in}{2.920548in}}{\pgfqpoint{1.975423in}{2.926372in}}%
\pgfpathcurveto{\pgfqpoint{1.969599in}{2.932196in}}{\pgfqpoint{1.961699in}{2.935468in}}{\pgfqpoint{1.953462in}{2.935468in}}%
\pgfpathcurveto{\pgfqpoint{1.945226in}{2.935468in}}{\pgfqpoint{1.937326in}{2.932196in}}{\pgfqpoint{1.931502in}{2.926372in}}%
\pgfpathcurveto{\pgfqpoint{1.925678in}{2.920548in}}{\pgfqpoint{1.922406in}{2.912648in}}{\pgfqpoint{1.922406in}{2.904412in}}%
\pgfpathcurveto{\pgfqpoint{1.922406in}{2.896176in}}{\pgfqpoint{1.925678in}{2.888276in}}{\pgfqpoint{1.931502in}{2.882452in}}%
\pgfpathcurveto{\pgfqpoint{1.937326in}{2.876628in}}{\pgfqpoint{1.945226in}{2.873355in}}{\pgfqpoint{1.953462in}{2.873355in}}%
\pgfpathclose%
\pgfusepath{stroke,fill}%
\end{pgfscope}%
\begin{pgfscope}%
\pgfpathrectangle{\pgfqpoint{0.100000in}{0.212622in}}{\pgfqpoint{3.696000in}{3.696000in}}%
\pgfusepath{clip}%
\pgfsetbuttcap%
\pgfsetroundjoin%
\definecolor{currentfill}{rgb}{0.121569,0.466667,0.705882}%
\pgfsetfillcolor{currentfill}%
\pgfsetfillopacity{0.447130}%
\pgfsetlinewidth{1.003750pt}%
\definecolor{currentstroke}{rgb}{0.121569,0.466667,0.705882}%
\pgfsetstrokecolor{currentstroke}%
\pgfsetstrokeopacity{0.447130}%
\pgfsetdash{}{0pt}%
\pgfpathmoveto{\pgfqpoint{1.446653in}{2.691602in}}%
\pgfpathcurveto{\pgfqpoint{1.454890in}{2.691602in}}{\pgfqpoint{1.462790in}{2.694875in}}{\pgfqpoint{1.468614in}{2.700698in}}%
\pgfpathcurveto{\pgfqpoint{1.474438in}{2.706522in}}{\pgfqpoint{1.477710in}{2.714422in}}{\pgfqpoint{1.477710in}{2.722659in}}%
\pgfpathcurveto{\pgfqpoint{1.477710in}{2.730895in}}{\pgfqpoint{1.474438in}{2.738795in}}{\pgfqpoint{1.468614in}{2.744619in}}%
\pgfpathcurveto{\pgfqpoint{1.462790in}{2.750443in}}{\pgfqpoint{1.454890in}{2.753715in}}{\pgfqpoint{1.446653in}{2.753715in}}%
\pgfpathcurveto{\pgfqpoint{1.438417in}{2.753715in}}{\pgfqpoint{1.430517in}{2.750443in}}{\pgfqpoint{1.424693in}{2.744619in}}%
\pgfpathcurveto{\pgfqpoint{1.418869in}{2.738795in}}{\pgfqpoint{1.415597in}{2.730895in}}{\pgfqpoint{1.415597in}{2.722659in}}%
\pgfpathcurveto{\pgfqpoint{1.415597in}{2.714422in}}{\pgfqpoint{1.418869in}{2.706522in}}{\pgfqpoint{1.424693in}{2.700698in}}%
\pgfpathcurveto{\pgfqpoint{1.430517in}{2.694875in}}{\pgfqpoint{1.438417in}{2.691602in}}{\pgfqpoint{1.446653in}{2.691602in}}%
\pgfpathclose%
\pgfusepath{stroke,fill}%
\end{pgfscope}%
\begin{pgfscope}%
\pgfpathrectangle{\pgfqpoint{0.100000in}{0.212622in}}{\pgfqpoint{3.696000in}{3.696000in}}%
\pgfusepath{clip}%
\pgfsetbuttcap%
\pgfsetroundjoin%
\definecolor{currentfill}{rgb}{0.121569,0.466667,0.705882}%
\pgfsetfillcolor{currentfill}%
\pgfsetfillopacity{0.451970}%
\pgfsetlinewidth{1.003750pt}%
\definecolor{currentstroke}{rgb}{0.121569,0.466667,0.705882}%
\pgfsetstrokecolor{currentstroke}%
\pgfsetstrokeopacity{0.451970}%
\pgfsetdash{}{0pt}%
\pgfpathmoveto{\pgfqpoint{1.966350in}{2.842372in}}%
\pgfpathcurveto{\pgfqpoint{1.974586in}{2.842372in}}{\pgfqpoint{1.982487in}{2.845645in}}{\pgfqpoint{1.988310in}{2.851468in}}%
\pgfpathcurveto{\pgfqpoint{1.994134in}{2.857292in}}{\pgfqpoint{1.997407in}{2.865192in}}{\pgfqpoint{1.997407in}{2.873429in}}%
\pgfpathcurveto{\pgfqpoint{1.997407in}{2.881665in}}{\pgfqpoint{1.994134in}{2.889565in}}{\pgfqpoint{1.988310in}{2.895389in}}%
\pgfpathcurveto{\pgfqpoint{1.982487in}{2.901213in}}{\pgfqpoint{1.974586in}{2.904485in}}{\pgfqpoint{1.966350in}{2.904485in}}%
\pgfpathcurveto{\pgfqpoint{1.958114in}{2.904485in}}{\pgfqpoint{1.950214in}{2.901213in}}{\pgfqpoint{1.944390in}{2.895389in}}%
\pgfpathcurveto{\pgfqpoint{1.938566in}{2.889565in}}{\pgfqpoint{1.935294in}{2.881665in}}{\pgfqpoint{1.935294in}{2.873429in}}%
\pgfpathcurveto{\pgfqpoint{1.935294in}{2.865192in}}{\pgfqpoint{1.938566in}{2.857292in}}{\pgfqpoint{1.944390in}{2.851468in}}%
\pgfpathcurveto{\pgfqpoint{1.950214in}{2.845645in}}{\pgfqpoint{1.958114in}{2.842372in}}{\pgfqpoint{1.966350in}{2.842372in}}%
\pgfpathclose%
\pgfusepath{stroke,fill}%
\end{pgfscope}%
\begin{pgfscope}%
\pgfpathrectangle{\pgfqpoint{0.100000in}{0.212622in}}{\pgfqpoint{3.696000in}{3.696000in}}%
\pgfusepath{clip}%
\pgfsetbuttcap%
\pgfsetroundjoin%
\definecolor{currentfill}{rgb}{0.121569,0.466667,0.705882}%
\pgfsetfillcolor{currentfill}%
\pgfsetfillopacity{0.457457}%
\pgfsetlinewidth{1.003750pt}%
\definecolor{currentstroke}{rgb}{0.121569,0.466667,0.705882}%
\pgfsetstrokecolor{currentstroke}%
\pgfsetstrokeopacity{0.457457}%
\pgfsetdash{}{0pt}%
\pgfpathmoveto{\pgfqpoint{1.425680in}{2.642857in}}%
\pgfpathcurveto{\pgfqpoint{1.433916in}{2.642857in}}{\pgfqpoint{1.441816in}{2.646130in}}{\pgfqpoint{1.447640in}{2.651954in}}%
\pgfpathcurveto{\pgfqpoint{1.453464in}{2.657778in}}{\pgfqpoint{1.456736in}{2.665678in}}{\pgfqpoint{1.456736in}{2.673914in}}%
\pgfpathcurveto{\pgfqpoint{1.456736in}{2.682150in}}{\pgfqpoint{1.453464in}{2.690050in}}{\pgfqpoint{1.447640in}{2.695874in}}%
\pgfpathcurveto{\pgfqpoint{1.441816in}{2.701698in}}{\pgfqpoint{1.433916in}{2.704970in}}{\pgfqpoint{1.425680in}{2.704970in}}%
\pgfpathcurveto{\pgfqpoint{1.417444in}{2.704970in}}{\pgfqpoint{1.409544in}{2.701698in}}{\pgfqpoint{1.403720in}{2.695874in}}%
\pgfpathcurveto{\pgfqpoint{1.397896in}{2.690050in}}{\pgfqpoint{1.394623in}{2.682150in}}{\pgfqpoint{1.394623in}{2.673914in}}%
\pgfpathcurveto{\pgfqpoint{1.394623in}{2.665678in}}{\pgfqpoint{1.397896in}{2.657778in}}{\pgfqpoint{1.403720in}{2.651954in}}%
\pgfpathcurveto{\pgfqpoint{1.409544in}{2.646130in}}{\pgfqpoint{1.417444in}{2.642857in}}{\pgfqpoint{1.425680in}{2.642857in}}%
\pgfpathclose%
\pgfusepath{stroke,fill}%
\end{pgfscope}%
\begin{pgfscope}%
\pgfpathrectangle{\pgfqpoint{0.100000in}{0.212622in}}{\pgfqpoint{3.696000in}{3.696000in}}%
\pgfusepath{clip}%
\pgfsetbuttcap%
\pgfsetroundjoin%
\definecolor{currentfill}{rgb}{0.121569,0.466667,0.705882}%
\pgfsetfillcolor{currentfill}%
\pgfsetfillopacity{0.461550}%
\pgfsetlinewidth{1.003750pt}%
\definecolor{currentstroke}{rgb}{0.121569,0.466667,0.705882}%
\pgfsetstrokecolor{currentstroke}%
\pgfsetstrokeopacity{0.461550}%
\pgfsetdash{}{0pt}%
\pgfpathmoveto{\pgfqpoint{1.978543in}{2.806559in}}%
\pgfpathcurveto{\pgfqpoint{1.986779in}{2.806559in}}{\pgfqpoint{1.994679in}{2.809832in}}{\pgfqpoint{2.000503in}{2.815656in}}%
\pgfpathcurveto{\pgfqpoint{2.006327in}{2.821480in}}{\pgfqpoint{2.009600in}{2.829380in}}{\pgfqpoint{2.009600in}{2.837616in}}%
\pgfpathcurveto{\pgfqpoint{2.009600in}{2.845852in}}{\pgfqpoint{2.006327in}{2.853752in}}{\pgfqpoint{2.000503in}{2.859576in}}%
\pgfpathcurveto{\pgfqpoint{1.994679in}{2.865400in}}{\pgfqpoint{1.986779in}{2.868672in}}{\pgfqpoint{1.978543in}{2.868672in}}%
\pgfpathcurveto{\pgfqpoint{1.970307in}{2.868672in}}{\pgfqpoint{1.962407in}{2.865400in}}{\pgfqpoint{1.956583in}{2.859576in}}%
\pgfpathcurveto{\pgfqpoint{1.950759in}{2.853752in}}{\pgfqpoint{1.947487in}{2.845852in}}{\pgfqpoint{1.947487in}{2.837616in}}%
\pgfpathcurveto{\pgfqpoint{1.947487in}{2.829380in}}{\pgfqpoint{1.950759in}{2.821480in}}{\pgfqpoint{1.956583in}{2.815656in}}%
\pgfpathcurveto{\pgfqpoint{1.962407in}{2.809832in}}{\pgfqpoint{1.970307in}{2.806559in}}{\pgfqpoint{1.978543in}{2.806559in}}%
\pgfpathclose%
\pgfusepath{stroke,fill}%
\end{pgfscope}%
\begin{pgfscope}%
\pgfpathrectangle{\pgfqpoint{0.100000in}{0.212622in}}{\pgfqpoint{3.696000in}{3.696000in}}%
\pgfusepath{clip}%
\pgfsetbuttcap%
\pgfsetroundjoin%
\definecolor{currentfill}{rgb}{0.121569,0.466667,0.705882}%
\pgfsetfillcolor{currentfill}%
\pgfsetfillopacity{0.466660}%
\pgfsetlinewidth{1.003750pt}%
\definecolor{currentstroke}{rgb}{0.121569,0.466667,0.705882}%
\pgfsetstrokecolor{currentstroke}%
\pgfsetstrokeopacity{0.466660}%
\pgfsetdash{}{0pt}%
\pgfpathmoveto{\pgfqpoint{1.407401in}{2.601212in}}%
\pgfpathcurveto{\pgfqpoint{1.415637in}{2.601212in}}{\pgfqpoint{1.423537in}{2.604485in}}{\pgfqpoint{1.429361in}{2.610309in}}%
\pgfpathcurveto{\pgfqpoint{1.435185in}{2.616133in}}{\pgfqpoint{1.438457in}{2.624033in}}{\pgfqpoint{1.438457in}{2.632269in}}%
\pgfpathcurveto{\pgfqpoint{1.438457in}{2.640505in}}{\pgfqpoint{1.435185in}{2.648405in}}{\pgfqpoint{1.429361in}{2.654229in}}%
\pgfpathcurveto{\pgfqpoint{1.423537in}{2.660053in}}{\pgfqpoint{1.415637in}{2.663325in}}{\pgfqpoint{1.407401in}{2.663325in}}%
\pgfpathcurveto{\pgfqpoint{1.399165in}{2.663325in}}{\pgfqpoint{1.391265in}{2.660053in}}{\pgfqpoint{1.385441in}{2.654229in}}%
\pgfpathcurveto{\pgfqpoint{1.379617in}{2.648405in}}{\pgfqpoint{1.376344in}{2.640505in}}{\pgfqpoint{1.376344in}{2.632269in}}%
\pgfpathcurveto{\pgfqpoint{1.376344in}{2.624033in}}{\pgfqpoint{1.379617in}{2.616133in}}{\pgfqpoint{1.385441in}{2.610309in}}%
\pgfpathcurveto{\pgfqpoint{1.391265in}{2.604485in}}{\pgfqpoint{1.399165in}{2.601212in}}{\pgfqpoint{1.407401in}{2.601212in}}%
\pgfpathclose%
\pgfusepath{stroke,fill}%
\end{pgfscope}%
\begin{pgfscope}%
\pgfpathrectangle{\pgfqpoint{0.100000in}{0.212622in}}{\pgfqpoint{3.696000in}{3.696000in}}%
\pgfusepath{clip}%
\pgfsetbuttcap%
\pgfsetroundjoin%
\definecolor{currentfill}{rgb}{0.121569,0.466667,0.705882}%
\pgfsetfillcolor{currentfill}%
\pgfsetfillopacity{0.472828}%
\pgfsetlinewidth{1.003750pt}%
\definecolor{currentstroke}{rgb}{0.121569,0.466667,0.705882}%
\pgfsetstrokecolor{currentstroke}%
\pgfsetstrokeopacity{0.472828}%
\pgfsetdash{}{0pt}%
\pgfpathmoveto{\pgfqpoint{1.994589in}{2.768258in}}%
\pgfpathcurveto{\pgfqpoint{2.002825in}{2.768258in}}{\pgfqpoint{2.010725in}{2.771530in}}{\pgfqpoint{2.016549in}{2.777354in}}%
\pgfpathcurveto{\pgfqpoint{2.022373in}{2.783178in}}{\pgfqpoint{2.025645in}{2.791078in}}{\pgfqpoint{2.025645in}{2.799314in}}%
\pgfpathcurveto{\pgfqpoint{2.025645in}{2.807550in}}{\pgfqpoint{2.022373in}{2.815450in}}{\pgfqpoint{2.016549in}{2.821274in}}%
\pgfpathcurveto{\pgfqpoint{2.010725in}{2.827098in}}{\pgfqpoint{2.002825in}{2.830371in}}{\pgfqpoint{1.994589in}{2.830371in}}%
\pgfpathcurveto{\pgfqpoint{1.986353in}{2.830371in}}{\pgfqpoint{1.978453in}{2.827098in}}{\pgfqpoint{1.972629in}{2.821274in}}%
\pgfpathcurveto{\pgfqpoint{1.966805in}{2.815450in}}{\pgfqpoint{1.963532in}{2.807550in}}{\pgfqpoint{1.963532in}{2.799314in}}%
\pgfpathcurveto{\pgfqpoint{1.963532in}{2.791078in}}{\pgfqpoint{1.966805in}{2.783178in}}{\pgfqpoint{1.972629in}{2.777354in}}%
\pgfpathcurveto{\pgfqpoint{1.978453in}{2.771530in}}{\pgfqpoint{1.986353in}{2.768258in}}{\pgfqpoint{1.994589in}{2.768258in}}%
\pgfpathclose%
\pgfusepath{stroke,fill}%
\end{pgfscope}%
\begin{pgfscope}%
\pgfpathrectangle{\pgfqpoint{0.100000in}{0.212622in}}{\pgfqpoint{3.696000in}{3.696000in}}%
\pgfusepath{clip}%
\pgfsetbuttcap%
\pgfsetroundjoin%
\definecolor{currentfill}{rgb}{0.121569,0.466667,0.705882}%
\pgfsetfillcolor{currentfill}%
\pgfsetfillopacity{0.474660}%
\pgfsetlinewidth{1.003750pt}%
\definecolor{currentstroke}{rgb}{0.121569,0.466667,0.705882}%
\pgfsetstrokecolor{currentstroke}%
\pgfsetstrokeopacity{0.474660}%
\pgfsetdash{}{0pt}%
\pgfpathmoveto{\pgfqpoint{1.390669in}{2.563053in}}%
\pgfpathcurveto{\pgfqpoint{1.398905in}{2.563053in}}{\pgfqpoint{1.406805in}{2.566325in}}{\pgfqpoint{1.412629in}{2.572149in}}%
\pgfpathcurveto{\pgfqpoint{1.418453in}{2.577973in}}{\pgfqpoint{1.421725in}{2.585873in}}{\pgfqpoint{1.421725in}{2.594109in}}%
\pgfpathcurveto{\pgfqpoint{1.421725in}{2.602345in}}{\pgfqpoint{1.418453in}{2.610245in}}{\pgfqpoint{1.412629in}{2.616069in}}%
\pgfpathcurveto{\pgfqpoint{1.406805in}{2.621893in}}{\pgfqpoint{1.398905in}{2.625166in}}{\pgfqpoint{1.390669in}{2.625166in}}%
\pgfpathcurveto{\pgfqpoint{1.382433in}{2.625166in}}{\pgfqpoint{1.374533in}{2.621893in}}{\pgfqpoint{1.368709in}{2.616069in}}%
\pgfpathcurveto{\pgfqpoint{1.362885in}{2.610245in}}{\pgfqpoint{1.359612in}{2.602345in}}{\pgfqpoint{1.359612in}{2.594109in}}%
\pgfpathcurveto{\pgfqpoint{1.359612in}{2.585873in}}{\pgfqpoint{1.362885in}{2.577973in}}{\pgfqpoint{1.368709in}{2.572149in}}%
\pgfpathcurveto{\pgfqpoint{1.374533in}{2.566325in}}{\pgfqpoint{1.382433in}{2.563053in}}{\pgfqpoint{1.390669in}{2.563053in}}%
\pgfpathclose%
\pgfusepath{stroke,fill}%
\end{pgfscope}%
\begin{pgfscope}%
\pgfpathrectangle{\pgfqpoint{0.100000in}{0.212622in}}{\pgfqpoint{3.696000in}{3.696000in}}%
\pgfusepath{clip}%
\pgfsetbuttcap%
\pgfsetroundjoin%
\definecolor{currentfill}{rgb}{0.121569,0.466667,0.705882}%
\pgfsetfillcolor{currentfill}%
\pgfsetfillopacity{0.480932}%
\pgfsetlinewidth{1.003750pt}%
\definecolor{currentstroke}{rgb}{0.121569,0.466667,0.705882}%
\pgfsetstrokecolor{currentstroke}%
\pgfsetstrokeopacity{0.480932}%
\pgfsetdash{}{0pt}%
\pgfpathmoveto{\pgfqpoint{1.376159in}{2.531887in}}%
\pgfpathcurveto{\pgfqpoint{1.384395in}{2.531887in}}{\pgfqpoint{1.392295in}{2.535159in}}{\pgfqpoint{1.398119in}{2.540983in}}%
\pgfpathcurveto{\pgfqpoint{1.403943in}{2.546807in}}{\pgfqpoint{1.407215in}{2.554707in}}{\pgfqpoint{1.407215in}{2.562944in}}%
\pgfpathcurveto{\pgfqpoint{1.407215in}{2.571180in}}{\pgfqpoint{1.403943in}{2.579080in}}{\pgfqpoint{1.398119in}{2.584904in}}%
\pgfpathcurveto{\pgfqpoint{1.392295in}{2.590728in}}{\pgfqpoint{1.384395in}{2.594000in}}{\pgfqpoint{1.376159in}{2.594000in}}%
\pgfpathcurveto{\pgfqpoint{1.367923in}{2.594000in}}{\pgfqpoint{1.360022in}{2.590728in}}{\pgfqpoint{1.354199in}{2.584904in}}%
\pgfpathcurveto{\pgfqpoint{1.348375in}{2.579080in}}{\pgfqpoint{1.345102in}{2.571180in}}{\pgfqpoint{1.345102in}{2.562944in}}%
\pgfpathcurveto{\pgfqpoint{1.345102in}{2.554707in}}{\pgfqpoint{1.348375in}{2.546807in}}{\pgfqpoint{1.354199in}{2.540983in}}%
\pgfpathcurveto{\pgfqpoint{1.360022in}{2.535159in}}{\pgfqpoint{1.367923in}{2.531887in}}{\pgfqpoint{1.376159in}{2.531887in}}%
\pgfpathclose%
\pgfusepath{stroke,fill}%
\end{pgfscope}%
\begin{pgfscope}%
\pgfpathrectangle{\pgfqpoint{0.100000in}{0.212622in}}{\pgfqpoint{3.696000in}{3.696000in}}%
\pgfusepath{clip}%
\pgfsetbuttcap%
\pgfsetroundjoin%
\definecolor{currentfill}{rgb}{0.121569,0.466667,0.705882}%
\pgfsetfillcolor{currentfill}%
\pgfsetfillopacity{0.485124}%
\pgfsetlinewidth{1.003750pt}%
\definecolor{currentstroke}{rgb}{0.121569,0.466667,0.705882}%
\pgfsetstrokecolor{currentstroke}%
\pgfsetstrokeopacity{0.485124}%
\pgfsetdash{}{0pt}%
\pgfpathmoveto{\pgfqpoint{2.010927in}{2.725113in}}%
\pgfpathcurveto{\pgfqpoint{2.019163in}{2.725113in}}{\pgfqpoint{2.027063in}{2.728385in}}{\pgfqpoint{2.032887in}{2.734209in}}%
\pgfpathcurveto{\pgfqpoint{2.038711in}{2.740033in}}{\pgfqpoint{2.041984in}{2.747933in}}{\pgfqpoint{2.041984in}{2.756169in}}%
\pgfpathcurveto{\pgfqpoint{2.041984in}{2.764406in}}{\pgfqpoint{2.038711in}{2.772306in}}{\pgfqpoint{2.032887in}{2.778130in}}%
\pgfpathcurveto{\pgfqpoint{2.027063in}{2.783954in}}{\pgfqpoint{2.019163in}{2.787226in}}{\pgfqpoint{2.010927in}{2.787226in}}%
\pgfpathcurveto{\pgfqpoint{2.002691in}{2.787226in}}{\pgfqpoint{1.994791in}{2.783954in}}{\pgfqpoint{1.988967in}{2.778130in}}%
\pgfpathcurveto{\pgfqpoint{1.983143in}{2.772306in}}{\pgfqpoint{1.979871in}{2.764406in}}{\pgfqpoint{1.979871in}{2.756169in}}%
\pgfpathcurveto{\pgfqpoint{1.979871in}{2.747933in}}{\pgfqpoint{1.983143in}{2.740033in}}{\pgfqpoint{1.988967in}{2.734209in}}%
\pgfpathcurveto{\pgfqpoint{1.994791in}{2.728385in}}{\pgfqpoint{2.002691in}{2.725113in}}{\pgfqpoint{2.010927in}{2.725113in}}%
\pgfpathclose%
\pgfusepath{stroke,fill}%
\end{pgfscope}%
\begin{pgfscope}%
\pgfpathrectangle{\pgfqpoint{0.100000in}{0.212622in}}{\pgfqpoint{3.696000in}{3.696000in}}%
\pgfusepath{clip}%
\pgfsetbuttcap%
\pgfsetroundjoin%
\definecolor{currentfill}{rgb}{0.121569,0.466667,0.705882}%
\pgfsetfillcolor{currentfill}%
\pgfsetfillopacity{0.486108}%
\pgfsetlinewidth{1.003750pt}%
\definecolor{currentstroke}{rgb}{0.121569,0.466667,0.705882}%
\pgfsetstrokecolor{currentstroke}%
\pgfsetstrokeopacity{0.486108}%
\pgfsetdash{}{0pt}%
\pgfpathmoveto{\pgfqpoint{1.364218in}{2.506299in}}%
\pgfpathcurveto{\pgfqpoint{1.372454in}{2.506299in}}{\pgfqpoint{1.380354in}{2.509571in}}{\pgfqpoint{1.386178in}{2.515395in}}%
\pgfpathcurveto{\pgfqpoint{1.392002in}{2.521219in}}{\pgfqpoint{1.395274in}{2.529119in}}{\pgfqpoint{1.395274in}{2.537356in}}%
\pgfpathcurveto{\pgfqpoint{1.395274in}{2.545592in}}{\pgfqpoint{1.392002in}{2.553492in}}{\pgfqpoint{1.386178in}{2.559316in}}%
\pgfpathcurveto{\pgfqpoint{1.380354in}{2.565140in}}{\pgfqpoint{1.372454in}{2.568412in}}{\pgfqpoint{1.364218in}{2.568412in}}%
\pgfpathcurveto{\pgfqpoint{1.355981in}{2.568412in}}{\pgfqpoint{1.348081in}{2.565140in}}{\pgfqpoint{1.342257in}{2.559316in}}%
\pgfpathcurveto{\pgfqpoint{1.336434in}{2.553492in}}{\pgfqpoint{1.333161in}{2.545592in}}{\pgfqpoint{1.333161in}{2.537356in}}%
\pgfpathcurveto{\pgfqpoint{1.333161in}{2.529119in}}{\pgfqpoint{1.336434in}{2.521219in}}{\pgfqpoint{1.342257in}{2.515395in}}%
\pgfpathcurveto{\pgfqpoint{1.348081in}{2.509571in}}{\pgfqpoint{1.355981in}{2.506299in}}{\pgfqpoint{1.364218in}{2.506299in}}%
\pgfpathclose%
\pgfusepath{stroke,fill}%
\end{pgfscope}%
\begin{pgfscope}%
\pgfpathrectangle{\pgfqpoint{0.100000in}{0.212622in}}{\pgfqpoint{3.696000in}{3.696000in}}%
\pgfusepath{clip}%
\pgfsetbuttcap%
\pgfsetroundjoin%
\definecolor{currentfill}{rgb}{0.121569,0.466667,0.705882}%
\pgfsetfillcolor{currentfill}%
\pgfsetfillopacity{0.490394}%
\pgfsetlinewidth{1.003750pt}%
\definecolor{currentstroke}{rgb}{0.121569,0.466667,0.705882}%
\pgfsetstrokecolor{currentstroke}%
\pgfsetstrokeopacity{0.490394}%
\pgfsetdash{}{0pt}%
\pgfpathmoveto{\pgfqpoint{1.354030in}{2.485357in}}%
\pgfpathcurveto{\pgfqpoint{1.362267in}{2.485357in}}{\pgfqpoint{1.370167in}{2.488629in}}{\pgfqpoint{1.375991in}{2.494453in}}%
\pgfpathcurveto{\pgfqpoint{1.381815in}{2.500277in}}{\pgfqpoint{1.385087in}{2.508177in}}{\pgfqpoint{1.385087in}{2.516413in}}%
\pgfpathcurveto{\pgfqpoint{1.385087in}{2.524650in}}{\pgfqpoint{1.381815in}{2.532550in}}{\pgfqpoint{1.375991in}{2.538373in}}%
\pgfpathcurveto{\pgfqpoint{1.370167in}{2.544197in}}{\pgfqpoint{1.362267in}{2.547470in}}{\pgfqpoint{1.354030in}{2.547470in}}%
\pgfpathcurveto{\pgfqpoint{1.345794in}{2.547470in}}{\pgfqpoint{1.337894in}{2.544197in}}{\pgfqpoint{1.332070in}{2.538373in}}%
\pgfpathcurveto{\pgfqpoint{1.326246in}{2.532550in}}{\pgfqpoint{1.322974in}{2.524650in}}{\pgfqpoint{1.322974in}{2.516413in}}%
\pgfpathcurveto{\pgfqpoint{1.322974in}{2.508177in}}{\pgfqpoint{1.326246in}{2.500277in}}{\pgfqpoint{1.332070in}{2.494453in}}%
\pgfpathcurveto{\pgfqpoint{1.337894in}{2.488629in}}{\pgfqpoint{1.345794in}{2.485357in}}{\pgfqpoint{1.354030in}{2.485357in}}%
\pgfpathclose%
\pgfusepath{stroke,fill}%
\end{pgfscope}%
\begin{pgfscope}%
\pgfpathrectangle{\pgfqpoint{0.100000in}{0.212622in}}{\pgfqpoint{3.696000in}{3.696000in}}%
\pgfusepath{clip}%
\pgfsetbuttcap%
\pgfsetroundjoin%
\definecolor{currentfill}{rgb}{0.121569,0.466667,0.705882}%
\pgfsetfillcolor{currentfill}%
\pgfsetfillopacity{0.493765}%
\pgfsetlinewidth{1.003750pt}%
\definecolor{currentstroke}{rgb}{0.121569,0.466667,0.705882}%
\pgfsetstrokecolor{currentstroke}%
\pgfsetstrokeopacity{0.493765}%
\pgfsetdash{}{0pt}%
\pgfpathmoveto{\pgfqpoint{1.346441in}{2.468553in}}%
\pgfpathcurveto{\pgfqpoint{1.354677in}{2.468553in}}{\pgfqpoint{1.362577in}{2.471825in}}{\pgfqpoint{1.368401in}{2.477649in}}%
\pgfpathcurveto{\pgfqpoint{1.374225in}{2.483473in}}{\pgfqpoint{1.377498in}{2.491373in}}{\pgfqpoint{1.377498in}{2.499609in}}%
\pgfpathcurveto{\pgfqpoint{1.377498in}{2.507846in}}{\pgfqpoint{1.374225in}{2.515746in}}{\pgfqpoint{1.368401in}{2.521570in}}%
\pgfpathcurveto{\pgfqpoint{1.362577in}{2.527394in}}{\pgfqpoint{1.354677in}{2.530666in}}{\pgfqpoint{1.346441in}{2.530666in}}%
\pgfpathcurveto{\pgfqpoint{1.338205in}{2.530666in}}{\pgfqpoint{1.330305in}{2.527394in}}{\pgfqpoint{1.324481in}{2.521570in}}%
\pgfpathcurveto{\pgfqpoint{1.318657in}{2.515746in}}{\pgfqpoint{1.315385in}{2.507846in}}{\pgfqpoint{1.315385in}{2.499609in}}%
\pgfpathcurveto{\pgfqpoint{1.315385in}{2.491373in}}{\pgfqpoint{1.318657in}{2.483473in}}{\pgfqpoint{1.324481in}{2.477649in}}%
\pgfpathcurveto{\pgfqpoint{1.330305in}{2.471825in}}{\pgfqpoint{1.338205in}{2.468553in}}{\pgfqpoint{1.346441in}{2.468553in}}%
\pgfpathclose%
\pgfusepath{stroke,fill}%
\end{pgfscope}%
\begin{pgfscope}%
\pgfpathrectangle{\pgfqpoint{0.100000in}{0.212622in}}{\pgfqpoint{3.696000in}{3.696000in}}%
\pgfusepath{clip}%
\pgfsetbuttcap%
\pgfsetroundjoin%
\definecolor{currentfill}{rgb}{0.121569,0.466667,0.705882}%
\pgfsetfillcolor{currentfill}%
\pgfsetfillopacity{0.496062}%
\pgfsetlinewidth{1.003750pt}%
\definecolor{currentstroke}{rgb}{0.121569,0.466667,0.705882}%
\pgfsetstrokecolor{currentstroke}%
\pgfsetstrokeopacity{0.496062}%
\pgfsetdash{}{0pt}%
\pgfpathmoveto{\pgfqpoint{1.341421in}{2.457374in}}%
\pgfpathcurveto{\pgfqpoint{1.349658in}{2.457374in}}{\pgfqpoint{1.357558in}{2.460646in}}{\pgfqpoint{1.363382in}{2.466470in}}%
\pgfpathcurveto{\pgfqpoint{1.369206in}{2.472294in}}{\pgfqpoint{1.372478in}{2.480194in}}{\pgfqpoint{1.372478in}{2.488431in}}%
\pgfpathcurveto{\pgfqpoint{1.372478in}{2.496667in}}{\pgfqpoint{1.369206in}{2.504567in}}{\pgfqpoint{1.363382in}{2.510391in}}%
\pgfpathcurveto{\pgfqpoint{1.357558in}{2.516215in}}{\pgfqpoint{1.349658in}{2.519487in}}{\pgfqpoint{1.341421in}{2.519487in}}%
\pgfpathcurveto{\pgfqpoint{1.333185in}{2.519487in}}{\pgfqpoint{1.325285in}{2.516215in}}{\pgfqpoint{1.319461in}{2.510391in}}%
\pgfpathcurveto{\pgfqpoint{1.313637in}{2.504567in}}{\pgfqpoint{1.310365in}{2.496667in}}{\pgfqpoint{1.310365in}{2.488431in}}%
\pgfpathcurveto{\pgfqpoint{1.310365in}{2.480194in}}{\pgfqpoint{1.313637in}{2.472294in}}{\pgfqpoint{1.319461in}{2.466470in}}%
\pgfpathcurveto{\pgfqpoint{1.325285in}{2.460646in}}{\pgfqpoint{1.333185in}{2.457374in}}{\pgfqpoint{1.341421in}{2.457374in}}%
\pgfpathclose%
\pgfusepath{stroke,fill}%
\end{pgfscope}%
\begin{pgfscope}%
\pgfpathrectangle{\pgfqpoint{0.100000in}{0.212622in}}{\pgfqpoint{3.696000in}{3.696000in}}%
\pgfusepath{clip}%
\pgfsetbuttcap%
\pgfsetroundjoin%
\definecolor{currentfill}{rgb}{0.121569,0.466667,0.705882}%
\pgfsetfillcolor{currentfill}%
\pgfsetfillopacity{0.497384}%
\pgfsetlinewidth{1.003750pt}%
\definecolor{currentstroke}{rgb}{0.121569,0.466667,0.705882}%
\pgfsetstrokecolor{currentstroke}%
\pgfsetstrokeopacity{0.497384}%
\pgfsetdash{}{0pt}%
\pgfpathmoveto{\pgfqpoint{1.338585in}{2.451136in}}%
\pgfpathcurveto{\pgfqpoint{1.346821in}{2.451136in}}{\pgfqpoint{1.354721in}{2.454409in}}{\pgfqpoint{1.360545in}{2.460233in}}%
\pgfpathcurveto{\pgfqpoint{1.366369in}{2.466057in}}{\pgfqpoint{1.369641in}{2.473957in}}{\pgfqpoint{1.369641in}{2.482193in}}%
\pgfpathcurveto{\pgfqpoint{1.369641in}{2.490429in}}{\pgfqpoint{1.366369in}{2.498329in}}{\pgfqpoint{1.360545in}{2.504153in}}%
\pgfpathcurveto{\pgfqpoint{1.354721in}{2.509977in}}{\pgfqpoint{1.346821in}{2.513249in}}{\pgfqpoint{1.338585in}{2.513249in}}%
\pgfpathcurveto{\pgfqpoint{1.330349in}{2.513249in}}{\pgfqpoint{1.322448in}{2.509977in}}{\pgfqpoint{1.316625in}{2.504153in}}%
\pgfpathcurveto{\pgfqpoint{1.310801in}{2.498329in}}{\pgfqpoint{1.307528in}{2.490429in}}{\pgfqpoint{1.307528in}{2.482193in}}%
\pgfpathcurveto{\pgfqpoint{1.307528in}{2.473957in}}{\pgfqpoint{1.310801in}{2.466057in}}{\pgfqpoint{1.316625in}{2.460233in}}%
\pgfpathcurveto{\pgfqpoint{1.322448in}{2.454409in}}{\pgfqpoint{1.330349in}{2.451136in}}{\pgfqpoint{1.338585in}{2.451136in}}%
\pgfpathclose%
\pgfusepath{stroke,fill}%
\end{pgfscope}%
\begin{pgfscope}%
\pgfpathrectangle{\pgfqpoint{0.100000in}{0.212622in}}{\pgfqpoint{3.696000in}{3.696000in}}%
\pgfusepath{clip}%
\pgfsetbuttcap%
\pgfsetroundjoin%
\definecolor{currentfill}{rgb}{0.121569,0.466667,0.705882}%
\pgfsetfillcolor{currentfill}%
\pgfsetfillopacity{0.499225}%
\pgfsetlinewidth{1.003750pt}%
\definecolor{currentstroke}{rgb}{0.121569,0.466667,0.705882}%
\pgfsetstrokecolor{currentstroke}%
\pgfsetstrokeopacity{0.499225}%
\pgfsetdash{}{0pt}%
\pgfpathmoveto{\pgfqpoint{2.032405in}{2.679187in}}%
\pgfpathcurveto{\pgfqpoint{2.040641in}{2.679187in}}{\pgfqpoint{2.048541in}{2.682459in}}{\pgfqpoint{2.054365in}{2.688283in}}%
\pgfpathcurveto{\pgfqpoint{2.060189in}{2.694107in}}{\pgfqpoint{2.063462in}{2.702007in}}{\pgfqpoint{2.063462in}{2.710243in}}%
\pgfpathcurveto{\pgfqpoint{2.063462in}{2.718480in}}{\pgfqpoint{2.060189in}{2.726380in}}{\pgfqpoint{2.054365in}{2.732204in}}%
\pgfpathcurveto{\pgfqpoint{2.048541in}{2.738028in}}{\pgfqpoint{2.040641in}{2.741300in}}{\pgfqpoint{2.032405in}{2.741300in}}%
\pgfpathcurveto{\pgfqpoint{2.024169in}{2.741300in}}{\pgfqpoint{2.016269in}{2.738028in}}{\pgfqpoint{2.010445in}{2.732204in}}%
\pgfpathcurveto{\pgfqpoint{2.004621in}{2.726380in}}{\pgfqpoint{2.001349in}{2.718480in}}{\pgfqpoint{2.001349in}{2.710243in}}%
\pgfpathcurveto{\pgfqpoint{2.001349in}{2.702007in}}{\pgfqpoint{2.004621in}{2.694107in}}{\pgfqpoint{2.010445in}{2.688283in}}%
\pgfpathcurveto{\pgfqpoint{2.016269in}{2.682459in}}{\pgfqpoint{2.024169in}{2.679187in}}{\pgfqpoint{2.032405in}{2.679187in}}%
\pgfpathclose%
\pgfusepath{stroke,fill}%
\end{pgfscope}%
\begin{pgfscope}%
\pgfpathrectangle{\pgfqpoint{0.100000in}{0.212622in}}{\pgfqpoint{3.696000in}{3.696000in}}%
\pgfusepath{clip}%
\pgfsetbuttcap%
\pgfsetroundjoin%
\definecolor{currentfill}{rgb}{0.121569,0.466667,0.705882}%
\pgfsetfillcolor{currentfill}%
\pgfsetfillopacity{0.499815}%
\pgfsetlinewidth{1.003750pt}%
\definecolor{currentstroke}{rgb}{0.121569,0.466667,0.705882}%
\pgfsetstrokecolor{currentstroke}%
\pgfsetstrokeopacity{0.499815}%
\pgfsetdash{}{0pt}%
\pgfpathmoveto{\pgfqpoint{1.333536in}{2.439740in}}%
\pgfpathcurveto{\pgfqpoint{1.341772in}{2.439740in}}{\pgfqpoint{1.349672in}{2.443013in}}{\pgfqpoint{1.355496in}{2.448837in}}%
\pgfpathcurveto{\pgfqpoint{1.361320in}{2.454661in}}{\pgfqpoint{1.364592in}{2.462561in}}{\pgfqpoint{1.364592in}{2.470797in}}%
\pgfpathcurveto{\pgfqpoint{1.364592in}{2.479033in}}{\pgfqpoint{1.361320in}{2.486933in}}{\pgfqpoint{1.355496in}{2.492757in}}%
\pgfpathcurveto{\pgfqpoint{1.349672in}{2.498581in}}{\pgfqpoint{1.341772in}{2.501853in}}{\pgfqpoint{1.333536in}{2.501853in}}%
\pgfpathcurveto{\pgfqpoint{1.325299in}{2.501853in}}{\pgfqpoint{1.317399in}{2.498581in}}{\pgfqpoint{1.311575in}{2.492757in}}%
\pgfpathcurveto{\pgfqpoint{1.305752in}{2.486933in}}{\pgfqpoint{1.302479in}{2.479033in}}{\pgfqpoint{1.302479in}{2.470797in}}%
\pgfpathcurveto{\pgfqpoint{1.302479in}{2.462561in}}{\pgfqpoint{1.305752in}{2.454661in}}{\pgfqpoint{1.311575in}{2.448837in}}%
\pgfpathcurveto{\pgfqpoint{1.317399in}{2.443013in}}{\pgfqpoint{1.325299in}{2.439740in}}{\pgfqpoint{1.333536in}{2.439740in}}%
\pgfpathclose%
\pgfusepath{stroke,fill}%
\end{pgfscope}%
\begin{pgfscope}%
\pgfpathrectangle{\pgfqpoint{0.100000in}{0.212622in}}{\pgfqpoint{3.696000in}{3.696000in}}%
\pgfusepath{clip}%
\pgfsetbuttcap%
\pgfsetroundjoin%
\definecolor{currentfill}{rgb}{0.121569,0.466667,0.705882}%
\pgfsetfillcolor{currentfill}%
\pgfsetfillopacity{0.504079}%
\pgfsetlinewidth{1.003750pt}%
\definecolor{currentstroke}{rgb}{0.121569,0.466667,0.705882}%
\pgfsetstrokecolor{currentstroke}%
\pgfsetstrokeopacity{0.504079}%
\pgfsetdash{}{0pt}%
\pgfpathmoveto{\pgfqpoint{1.323902in}{2.418848in}}%
\pgfpathcurveto{\pgfqpoint{1.332138in}{2.418848in}}{\pgfqpoint{1.340038in}{2.422120in}}{\pgfqpoint{1.345862in}{2.427944in}}%
\pgfpathcurveto{\pgfqpoint{1.351686in}{2.433768in}}{\pgfqpoint{1.354958in}{2.441668in}}{\pgfqpoint{1.354958in}{2.449905in}}%
\pgfpathcurveto{\pgfqpoint{1.354958in}{2.458141in}}{\pgfqpoint{1.351686in}{2.466041in}}{\pgfqpoint{1.345862in}{2.471865in}}%
\pgfpathcurveto{\pgfqpoint{1.340038in}{2.477689in}}{\pgfqpoint{1.332138in}{2.480961in}}{\pgfqpoint{1.323902in}{2.480961in}}%
\pgfpathcurveto{\pgfqpoint{1.315666in}{2.480961in}}{\pgfqpoint{1.307766in}{2.477689in}}{\pgfqpoint{1.301942in}{2.471865in}}%
\pgfpathcurveto{\pgfqpoint{1.296118in}{2.466041in}}{\pgfqpoint{1.292845in}{2.458141in}}{\pgfqpoint{1.292845in}{2.449905in}}%
\pgfpathcurveto{\pgfqpoint{1.292845in}{2.441668in}}{\pgfqpoint{1.296118in}{2.433768in}}{\pgfqpoint{1.301942in}{2.427944in}}%
\pgfpathcurveto{\pgfqpoint{1.307766in}{2.422120in}}{\pgfqpoint{1.315666in}{2.418848in}}{\pgfqpoint{1.323902in}{2.418848in}}%
\pgfpathclose%
\pgfusepath{stroke,fill}%
\end{pgfscope}%
\begin{pgfscope}%
\pgfpathrectangle{\pgfqpoint{0.100000in}{0.212622in}}{\pgfqpoint{3.696000in}{3.696000in}}%
\pgfusepath{clip}%
\pgfsetbuttcap%
\pgfsetroundjoin%
\definecolor{currentfill}{rgb}{0.121569,0.466667,0.705882}%
\pgfsetfillcolor{currentfill}%
\pgfsetfillopacity{0.507956}%
\pgfsetlinewidth{1.003750pt}%
\definecolor{currentstroke}{rgb}{0.121569,0.466667,0.705882}%
\pgfsetstrokecolor{currentstroke}%
\pgfsetstrokeopacity{0.507956}%
\pgfsetdash{}{0pt}%
\pgfpathmoveto{\pgfqpoint{1.315750in}{2.401558in}}%
\pgfpathcurveto{\pgfqpoint{1.323986in}{2.401558in}}{\pgfqpoint{1.331886in}{2.404831in}}{\pgfqpoint{1.337710in}{2.410655in}}%
\pgfpathcurveto{\pgfqpoint{1.343534in}{2.416479in}}{\pgfqpoint{1.346806in}{2.424379in}}{\pgfqpoint{1.346806in}{2.432615in}}%
\pgfpathcurveto{\pgfqpoint{1.346806in}{2.440851in}}{\pgfqpoint{1.343534in}{2.448751in}}{\pgfqpoint{1.337710in}{2.454575in}}%
\pgfpathcurveto{\pgfqpoint{1.331886in}{2.460399in}}{\pgfqpoint{1.323986in}{2.463671in}}{\pgfqpoint{1.315750in}{2.463671in}}%
\pgfpathcurveto{\pgfqpoint{1.307514in}{2.463671in}}{\pgfqpoint{1.299614in}{2.460399in}}{\pgfqpoint{1.293790in}{2.454575in}}%
\pgfpathcurveto{\pgfqpoint{1.287966in}{2.448751in}}{\pgfqpoint{1.284693in}{2.440851in}}{\pgfqpoint{1.284693in}{2.432615in}}%
\pgfpathcurveto{\pgfqpoint{1.284693in}{2.424379in}}{\pgfqpoint{1.287966in}{2.416479in}}{\pgfqpoint{1.293790in}{2.410655in}}%
\pgfpathcurveto{\pgfqpoint{1.299614in}{2.404831in}}{\pgfqpoint{1.307514in}{2.401558in}}{\pgfqpoint{1.315750in}{2.401558in}}%
\pgfpathclose%
\pgfusepath{stroke,fill}%
\end{pgfscope}%
\begin{pgfscope}%
\pgfpathrectangle{\pgfqpoint{0.100000in}{0.212622in}}{\pgfqpoint{3.696000in}{3.696000in}}%
\pgfusepath{clip}%
\pgfsetbuttcap%
\pgfsetroundjoin%
\definecolor{currentfill}{rgb}{0.121569,0.466667,0.705882}%
\pgfsetfillcolor{currentfill}%
\pgfsetfillopacity{0.514092}%
\pgfsetlinewidth{1.003750pt}%
\definecolor{currentstroke}{rgb}{0.121569,0.466667,0.705882}%
\pgfsetstrokecolor{currentstroke}%
\pgfsetstrokeopacity{0.514092}%
\pgfsetdash{}{0pt}%
\pgfpathmoveto{\pgfqpoint{2.056721in}{2.630616in}}%
\pgfpathcurveto{\pgfqpoint{2.064957in}{2.630616in}}{\pgfqpoint{2.072857in}{2.633889in}}{\pgfqpoint{2.078681in}{2.639713in}}%
\pgfpathcurveto{\pgfqpoint{2.084505in}{2.645537in}}{\pgfqpoint{2.087778in}{2.653437in}}{\pgfqpoint{2.087778in}{2.661673in}}%
\pgfpathcurveto{\pgfqpoint{2.087778in}{2.669909in}}{\pgfqpoint{2.084505in}{2.677809in}}{\pgfqpoint{2.078681in}{2.683633in}}%
\pgfpathcurveto{\pgfqpoint{2.072857in}{2.689457in}}{\pgfqpoint{2.064957in}{2.692729in}}{\pgfqpoint{2.056721in}{2.692729in}}%
\pgfpathcurveto{\pgfqpoint{2.048485in}{2.692729in}}{\pgfqpoint{2.040585in}{2.689457in}}{\pgfqpoint{2.034761in}{2.683633in}}%
\pgfpathcurveto{\pgfqpoint{2.028937in}{2.677809in}}{\pgfqpoint{2.025665in}{2.669909in}}{\pgfqpoint{2.025665in}{2.661673in}}%
\pgfpathcurveto{\pgfqpoint{2.025665in}{2.653437in}}{\pgfqpoint{2.028937in}{2.645537in}}{\pgfqpoint{2.034761in}{2.639713in}}%
\pgfpathcurveto{\pgfqpoint{2.040585in}{2.633889in}}{\pgfqpoint{2.048485in}{2.630616in}}{\pgfqpoint{2.056721in}{2.630616in}}%
\pgfpathclose%
\pgfusepath{stroke,fill}%
\end{pgfscope}%
\begin{pgfscope}%
\pgfpathrectangle{\pgfqpoint{0.100000in}{0.212622in}}{\pgfqpoint{3.696000in}{3.696000in}}%
\pgfusepath{clip}%
\pgfsetbuttcap%
\pgfsetroundjoin%
\definecolor{currentfill}{rgb}{0.121569,0.466667,0.705882}%
\pgfsetfillcolor{currentfill}%
\pgfsetfillopacity{0.514913}%
\pgfsetlinewidth{1.003750pt}%
\definecolor{currentstroke}{rgb}{0.121569,0.466667,0.705882}%
\pgfsetstrokecolor{currentstroke}%
\pgfsetstrokeopacity{0.514913}%
\pgfsetdash{}{0pt}%
\pgfpathmoveto{\pgfqpoint{1.301486in}{2.368851in}}%
\pgfpathcurveto{\pgfqpoint{1.309722in}{2.368851in}}{\pgfqpoint{1.317622in}{2.372123in}}{\pgfqpoint{1.323446in}{2.377947in}}%
\pgfpathcurveto{\pgfqpoint{1.329270in}{2.383771in}}{\pgfqpoint{1.332543in}{2.391671in}}{\pgfqpoint{1.332543in}{2.399908in}}%
\pgfpathcurveto{\pgfqpoint{1.332543in}{2.408144in}}{\pgfqpoint{1.329270in}{2.416044in}}{\pgfqpoint{1.323446in}{2.421868in}}%
\pgfpathcurveto{\pgfqpoint{1.317622in}{2.427692in}}{\pgfqpoint{1.309722in}{2.430964in}}{\pgfqpoint{1.301486in}{2.430964in}}%
\pgfpathcurveto{\pgfqpoint{1.293250in}{2.430964in}}{\pgfqpoint{1.285350in}{2.427692in}}{\pgfqpoint{1.279526in}{2.421868in}}%
\pgfpathcurveto{\pgfqpoint{1.273702in}{2.416044in}}{\pgfqpoint{1.270430in}{2.408144in}}{\pgfqpoint{1.270430in}{2.399908in}}%
\pgfpathcurveto{\pgfqpoint{1.270430in}{2.391671in}}{\pgfqpoint{1.273702in}{2.383771in}}{\pgfqpoint{1.279526in}{2.377947in}}%
\pgfpathcurveto{\pgfqpoint{1.285350in}{2.372123in}}{\pgfqpoint{1.293250in}{2.368851in}}{\pgfqpoint{1.301486in}{2.368851in}}%
\pgfpathclose%
\pgfusepath{stroke,fill}%
\end{pgfscope}%
\begin{pgfscope}%
\pgfpathrectangle{\pgfqpoint{0.100000in}{0.212622in}}{\pgfqpoint{3.696000in}{3.696000in}}%
\pgfusepath{clip}%
\pgfsetbuttcap%
\pgfsetroundjoin%
\definecolor{currentfill}{rgb}{0.121569,0.466667,0.705882}%
\pgfsetfillcolor{currentfill}%
\pgfsetfillopacity{0.521491}%
\pgfsetlinewidth{1.003750pt}%
\definecolor{currentstroke}{rgb}{0.121569,0.466667,0.705882}%
\pgfsetstrokecolor{currentstroke}%
\pgfsetstrokeopacity{0.521491}%
\pgfsetdash{}{0pt}%
\pgfpathmoveto{\pgfqpoint{1.290291in}{2.340768in}}%
\pgfpathcurveto{\pgfqpoint{1.298527in}{2.340768in}}{\pgfqpoint{1.306427in}{2.344041in}}{\pgfqpoint{1.312251in}{2.349865in}}%
\pgfpathcurveto{\pgfqpoint{1.318075in}{2.355689in}}{\pgfqpoint{1.321347in}{2.363589in}}{\pgfqpoint{1.321347in}{2.371825in}}%
\pgfpathcurveto{\pgfqpoint{1.321347in}{2.380061in}}{\pgfqpoint{1.318075in}{2.387961in}}{\pgfqpoint{1.312251in}{2.393785in}}%
\pgfpathcurveto{\pgfqpoint{1.306427in}{2.399609in}}{\pgfqpoint{1.298527in}{2.402881in}}{\pgfqpoint{1.290291in}{2.402881in}}%
\pgfpathcurveto{\pgfqpoint{1.282054in}{2.402881in}}{\pgfqpoint{1.274154in}{2.399609in}}{\pgfqpoint{1.268330in}{2.393785in}}%
\pgfpathcurveto{\pgfqpoint{1.262506in}{2.387961in}}{\pgfqpoint{1.259234in}{2.380061in}}{\pgfqpoint{1.259234in}{2.371825in}}%
\pgfpathcurveto{\pgfqpoint{1.259234in}{2.363589in}}{\pgfqpoint{1.262506in}{2.355689in}}{\pgfqpoint{1.268330in}{2.349865in}}%
\pgfpathcurveto{\pgfqpoint{1.274154in}{2.344041in}}{\pgfqpoint{1.282054in}{2.340768in}}{\pgfqpoint{1.290291in}{2.340768in}}%
\pgfpathclose%
\pgfusepath{stroke,fill}%
\end{pgfscope}%
\begin{pgfscope}%
\pgfpathrectangle{\pgfqpoint{0.100000in}{0.212622in}}{\pgfqpoint{3.696000in}{3.696000in}}%
\pgfusepath{clip}%
\pgfsetbuttcap%
\pgfsetroundjoin%
\definecolor{currentfill}{rgb}{0.121569,0.466667,0.705882}%
\pgfsetfillcolor{currentfill}%
\pgfsetfillopacity{0.522274}%
\pgfsetlinewidth{1.003750pt}%
\definecolor{currentstroke}{rgb}{0.121569,0.466667,0.705882}%
\pgfsetstrokecolor{currentstroke}%
\pgfsetstrokeopacity{0.522274}%
\pgfsetdash{}{0pt}%
\pgfpathmoveto{\pgfqpoint{2.070179in}{2.603781in}}%
\pgfpathcurveto{\pgfqpoint{2.078416in}{2.603781in}}{\pgfqpoint{2.086316in}{2.607053in}}{\pgfqpoint{2.092140in}{2.612877in}}%
\pgfpathcurveto{\pgfqpoint{2.097964in}{2.618701in}}{\pgfqpoint{2.101236in}{2.626601in}}{\pgfqpoint{2.101236in}{2.634837in}}%
\pgfpathcurveto{\pgfqpoint{2.101236in}{2.643073in}}{\pgfqpoint{2.097964in}{2.650973in}}{\pgfqpoint{2.092140in}{2.656797in}}%
\pgfpathcurveto{\pgfqpoint{2.086316in}{2.662621in}}{\pgfqpoint{2.078416in}{2.665894in}}{\pgfqpoint{2.070179in}{2.665894in}}%
\pgfpathcurveto{\pgfqpoint{2.061943in}{2.665894in}}{\pgfqpoint{2.054043in}{2.662621in}}{\pgfqpoint{2.048219in}{2.656797in}}%
\pgfpathcurveto{\pgfqpoint{2.042395in}{2.650973in}}{\pgfqpoint{2.039123in}{2.643073in}}{\pgfqpoint{2.039123in}{2.634837in}}%
\pgfpathcurveto{\pgfqpoint{2.039123in}{2.626601in}}{\pgfqpoint{2.042395in}{2.618701in}}{\pgfqpoint{2.048219in}{2.612877in}}%
\pgfpathcurveto{\pgfqpoint{2.054043in}{2.607053in}}{\pgfqpoint{2.061943in}{2.603781in}}{\pgfqpoint{2.070179in}{2.603781in}}%
\pgfpathclose%
\pgfusepath{stroke,fill}%
\end{pgfscope}%
\begin{pgfscope}%
\pgfpathrectangle{\pgfqpoint{0.100000in}{0.212622in}}{\pgfqpoint{3.696000in}{3.696000in}}%
\pgfusepath{clip}%
\pgfsetbuttcap%
\pgfsetroundjoin%
\definecolor{currentfill}{rgb}{0.121569,0.466667,0.705882}%
\pgfsetfillcolor{currentfill}%
\pgfsetfillopacity{0.526846}%
\pgfsetlinewidth{1.003750pt}%
\definecolor{currentstroke}{rgb}{0.121569,0.466667,0.705882}%
\pgfsetstrokecolor{currentstroke}%
\pgfsetstrokeopacity{0.526846}%
\pgfsetdash{}{0pt}%
\pgfpathmoveto{\pgfqpoint{1.282437in}{2.318006in}}%
\pgfpathcurveto{\pgfqpoint{1.290674in}{2.318006in}}{\pgfqpoint{1.298574in}{2.321278in}}{\pgfqpoint{1.304398in}{2.327102in}}%
\pgfpathcurveto{\pgfqpoint{1.310222in}{2.332926in}}{\pgfqpoint{1.313494in}{2.340826in}}{\pgfqpoint{1.313494in}{2.349063in}}%
\pgfpathcurveto{\pgfqpoint{1.313494in}{2.357299in}}{\pgfqpoint{1.310222in}{2.365199in}}{\pgfqpoint{1.304398in}{2.371023in}}%
\pgfpathcurveto{\pgfqpoint{1.298574in}{2.376847in}}{\pgfqpoint{1.290674in}{2.380119in}}{\pgfqpoint{1.282437in}{2.380119in}}%
\pgfpathcurveto{\pgfqpoint{1.274201in}{2.380119in}}{\pgfqpoint{1.266301in}{2.376847in}}{\pgfqpoint{1.260477in}{2.371023in}}%
\pgfpathcurveto{\pgfqpoint{1.254653in}{2.365199in}}{\pgfqpoint{1.251381in}{2.357299in}}{\pgfqpoint{1.251381in}{2.349063in}}%
\pgfpathcurveto{\pgfqpoint{1.251381in}{2.340826in}}{\pgfqpoint{1.254653in}{2.332926in}}{\pgfqpoint{1.260477in}{2.327102in}}%
\pgfpathcurveto{\pgfqpoint{1.266301in}{2.321278in}}{\pgfqpoint{1.274201in}{2.318006in}}{\pgfqpoint{1.282437in}{2.318006in}}%
\pgfpathclose%
\pgfusepath{stroke,fill}%
\end{pgfscope}%
\begin{pgfscope}%
\pgfpathrectangle{\pgfqpoint{0.100000in}{0.212622in}}{\pgfqpoint{3.696000in}{3.696000in}}%
\pgfusepath{clip}%
\pgfsetbuttcap%
\pgfsetroundjoin%
\definecolor{currentfill}{rgb}{0.121569,0.466667,0.705882}%
\pgfsetfillcolor{currentfill}%
\pgfsetfillopacity{0.531020}%
\pgfsetlinewidth{1.003750pt}%
\definecolor{currentstroke}{rgb}{0.121569,0.466667,0.705882}%
\pgfsetstrokecolor{currentstroke}%
\pgfsetstrokeopacity{0.531020}%
\pgfsetdash{}{0pt}%
\pgfpathmoveto{\pgfqpoint{2.083780in}{2.574082in}}%
\pgfpathcurveto{\pgfqpoint{2.092017in}{2.574082in}}{\pgfqpoint{2.099917in}{2.577354in}}{\pgfqpoint{2.105741in}{2.583178in}}%
\pgfpathcurveto{\pgfqpoint{2.111565in}{2.589002in}}{\pgfqpoint{2.114837in}{2.596902in}}{\pgfqpoint{2.114837in}{2.605139in}}%
\pgfpathcurveto{\pgfqpoint{2.114837in}{2.613375in}}{\pgfqpoint{2.111565in}{2.621275in}}{\pgfqpoint{2.105741in}{2.627099in}}%
\pgfpathcurveto{\pgfqpoint{2.099917in}{2.632923in}}{\pgfqpoint{2.092017in}{2.636195in}}{\pgfqpoint{2.083780in}{2.636195in}}%
\pgfpathcurveto{\pgfqpoint{2.075544in}{2.636195in}}{\pgfqpoint{2.067644in}{2.632923in}}{\pgfqpoint{2.061820in}{2.627099in}}%
\pgfpathcurveto{\pgfqpoint{2.055996in}{2.621275in}}{\pgfqpoint{2.052724in}{2.613375in}}{\pgfqpoint{2.052724in}{2.605139in}}%
\pgfpathcurveto{\pgfqpoint{2.052724in}{2.596902in}}{\pgfqpoint{2.055996in}{2.589002in}}{\pgfqpoint{2.061820in}{2.583178in}}%
\pgfpathcurveto{\pgfqpoint{2.067644in}{2.577354in}}{\pgfqpoint{2.075544in}{2.574082in}}{\pgfqpoint{2.083780in}{2.574082in}}%
\pgfpathclose%
\pgfusepath{stroke,fill}%
\end{pgfscope}%
\begin{pgfscope}%
\pgfpathrectangle{\pgfqpoint{0.100000in}{0.212622in}}{\pgfqpoint{3.696000in}{3.696000in}}%
\pgfusepath{clip}%
\pgfsetbuttcap%
\pgfsetroundjoin%
\definecolor{currentfill}{rgb}{0.121569,0.466667,0.705882}%
\pgfsetfillcolor{currentfill}%
\pgfsetfillopacity{0.531106}%
\pgfsetlinewidth{1.003750pt}%
\definecolor{currentstroke}{rgb}{0.121569,0.466667,0.705882}%
\pgfsetstrokecolor{currentstroke}%
\pgfsetstrokeopacity{0.531106}%
\pgfsetdash{}{0pt}%
\pgfpathmoveto{\pgfqpoint{1.274531in}{2.296625in}}%
\pgfpathcurveto{\pgfqpoint{1.282767in}{2.296625in}}{\pgfqpoint{1.290667in}{2.299897in}}{\pgfqpoint{1.296491in}{2.305721in}}%
\pgfpathcurveto{\pgfqpoint{1.302315in}{2.311545in}}{\pgfqpoint{1.305587in}{2.319445in}}{\pgfqpoint{1.305587in}{2.327681in}}%
\pgfpathcurveto{\pgfqpoint{1.305587in}{2.335918in}}{\pgfqpoint{1.302315in}{2.343818in}}{\pgfqpoint{1.296491in}{2.349642in}}%
\pgfpathcurveto{\pgfqpoint{1.290667in}{2.355465in}}{\pgfqpoint{1.282767in}{2.358738in}}{\pgfqpoint{1.274531in}{2.358738in}}%
\pgfpathcurveto{\pgfqpoint{1.266294in}{2.358738in}}{\pgfqpoint{1.258394in}{2.355465in}}{\pgfqpoint{1.252570in}{2.349642in}}%
\pgfpathcurveto{\pgfqpoint{1.246747in}{2.343818in}}{\pgfqpoint{1.243474in}{2.335918in}}{\pgfqpoint{1.243474in}{2.327681in}}%
\pgfpathcurveto{\pgfqpoint{1.243474in}{2.319445in}}{\pgfqpoint{1.246747in}{2.311545in}}{\pgfqpoint{1.252570in}{2.305721in}}%
\pgfpathcurveto{\pgfqpoint{1.258394in}{2.299897in}}{\pgfqpoint{1.266294in}{2.296625in}}{\pgfqpoint{1.274531in}{2.296625in}}%
\pgfpathclose%
\pgfusepath{stroke,fill}%
\end{pgfscope}%
\begin{pgfscope}%
\pgfpathrectangle{\pgfqpoint{0.100000in}{0.212622in}}{\pgfqpoint{3.696000in}{3.696000in}}%
\pgfusepath{clip}%
\pgfsetbuttcap%
\pgfsetroundjoin%
\definecolor{currentfill}{rgb}{0.121569,0.466667,0.705882}%
\pgfsetfillcolor{currentfill}%
\pgfsetfillopacity{0.535028}%
\pgfsetlinewidth{1.003750pt}%
\definecolor{currentstroke}{rgb}{0.121569,0.466667,0.705882}%
\pgfsetstrokecolor{currentstroke}%
\pgfsetstrokeopacity{0.535028}%
\pgfsetdash{}{0pt}%
\pgfpathmoveto{\pgfqpoint{1.268416in}{2.279499in}}%
\pgfpathcurveto{\pgfqpoint{1.276652in}{2.279499in}}{\pgfqpoint{1.284552in}{2.282771in}}{\pgfqpoint{1.290376in}{2.288595in}}%
\pgfpathcurveto{\pgfqpoint{1.296200in}{2.294419in}}{\pgfqpoint{1.299473in}{2.302319in}}{\pgfqpoint{1.299473in}{2.310555in}}%
\pgfpathcurveto{\pgfqpoint{1.299473in}{2.318791in}}{\pgfqpoint{1.296200in}{2.326691in}}{\pgfqpoint{1.290376in}{2.332515in}}%
\pgfpathcurveto{\pgfqpoint{1.284552in}{2.338339in}}{\pgfqpoint{1.276652in}{2.341612in}}{\pgfqpoint{1.268416in}{2.341612in}}%
\pgfpathcurveto{\pgfqpoint{1.260180in}{2.341612in}}{\pgfqpoint{1.252280in}{2.338339in}}{\pgfqpoint{1.246456in}{2.332515in}}%
\pgfpathcurveto{\pgfqpoint{1.240632in}{2.326691in}}{\pgfqpoint{1.237360in}{2.318791in}}{\pgfqpoint{1.237360in}{2.310555in}}%
\pgfpathcurveto{\pgfqpoint{1.237360in}{2.302319in}}{\pgfqpoint{1.240632in}{2.294419in}}{\pgfqpoint{1.246456in}{2.288595in}}%
\pgfpathcurveto{\pgfqpoint{1.252280in}{2.282771in}}{\pgfqpoint{1.260180in}{2.279499in}}{\pgfqpoint{1.268416in}{2.279499in}}%
\pgfpathclose%
\pgfusepath{stroke,fill}%
\end{pgfscope}%
\begin{pgfscope}%
\pgfpathrectangle{\pgfqpoint{0.100000in}{0.212622in}}{\pgfqpoint{3.696000in}{3.696000in}}%
\pgfusepath{clip}%
\pgfsetbuttcap%
\pgfsetroundjoin%
\definecolor{currentfill}{rgb}{0.121569,0.466667,0.705882}%
\pgfsetfillcolor{currentfill}%
\pgfsetfillopacity{0.537772}%
\pgfsetlinewidth{1.003750pt}%
\definecolor{currentstroke}{rgb}{0.121569,0.466667,0.705882}%
\pgfsetstrokecolor{currentstroke}%
\pgfsetstrokeopacity{0.537772}%
\pgfsetdash{}{0pt}%
\pgfpathmoveto{\pgfqpoint{1.263287in}{2.263918in}}%
\pgfpathcurveto{\pgfqpoint{1.271524in}{2.263918in}}{\pgfqpoint{1.279424in}{2.267191in}}{\pgfqpoint{1.285248in}{2.273015in}}%
\pgfpathcurveto{\pgfqpoint{1.291071in}{2.278839in}}{\pgfqpoint{1.294344in}{2.286739in}}{\pgfqpoint{1.294344in}{2.294975in}}%
\pgfpathcurveto{\pgfqpoint{1.294344in}{2.303211in}}{\pgfqpoint{1.291071in}{2.311111in}}{\pgfqpoint{1.285248in}{2.316935in}}%
\pgfpathcurveto{\pgfqpoint{1.279424in}{2.322759in}}{\pgfqpoint{1.271524in}{2.326031in}}{\pgfqpoint{1.263287in}{2.326031in}}%
\pgfpathcurveto{\pgfqpoint{1.255051in}{2.326031in}}{\pgfqpoint{1.247151in}{2.322759in}}{\pgfqpoint{1.241327in}{2.316935in}}%
\pgfpathcurveto{\pgfqpoint{1.235503in}{2.311111in}}{\pgfqpoint{1.232231in}{2.303211in}}{\pgfqpoint{1.232231in}{2.294975in}}%
\pgfpathcurveto{\pgfqpoint{1.232231in}{2.286739in}}{\pgfqpoint{1.235503in}{2.278839in}}{\pgfqpoint{1.241327in}{2.273015in}}%
\pgfpathcurveto{\pgfqpoint{1.247151in}{2.267191in}}{\pgfqpoint{1.255051in}{2.263918in}}{\pgfqpoint{1.263287in}{2.263918in}}%
\pgfpathclose%
\pgfusepath{stroke,fill}%
\end{pgfscope}%
\begin{pgfscope}%
\pgfpathrectangle{\pgfqpoint{0.100000in}{0.212622in}}{\pgfqpoint{3.696000in}{3.696000in}}%
\pgfusepath{clip}%
\pgfsetbuttcap%
\pgfsetroundjoin%
\definecolor{currentfill}{rgb}{0.121569,0.466667,0.705882}%
\pgfsetfillcolor{currentfill}%
\pgfsetfillopacity{0.540492}%
\pgfsetlinewidth{1.003750pt}%
\definecolor{currentstroke}{rgb}{0.121569,0.466667,0.705882}%
\pgfsetstrokecolor{currentstroke}%
\pgfsetstrokeopacity{0.540492}%
\pgfsetdash{}{0pt}%
\pgfpathmoveto{\pgfqpoint{1.259316in}{2.252620in}}%
\pgfpathcurveto{\pgfqpoint{1.267552in}{2.252620in}}{\pgfqpoint{1.275452in}{2.255892in}}{\pgfqpoint{1.281276in}{2.261716in}}%
\pgfpathcurveto{\pgfqpoint{1.287100in}{2.267540in}}{\pgfqpoint{1.290372in}{2.275440in}}{\pgfqpoint{1.290372in}{2.283676in}}%
\pgfpathcurveto{\pgfqpoint{1.290372in}{2.291913in}}{\pgfqpoint{1.287100in}{2.299813in}}{\pgfqpoint{1.281276in}{2.305637in}}%
\pgfpathcurveto{\pgfqpoint{1.275452in}{2.311461in}}{\pgfqpoint{1.267552in}{2.314733in}}{\pgfqpoint{1.259316in}{2.314733in}}%
\pgfpathcurveto{\pgfqpoint{1.251079in}{2.314733in}}{\pgfqpoint{1.243179in}{2.311461in}}{\pgfqpoint{1.237355in}{2.305637in}}%
\pgfpathcurveto{\pgfqpoint{1.231531in}{2.299813in}}{\pgfqpoint{1.228259in}{2.291913in}}{\pgfqpoint{1.228259in}{2.283676in}}%
\pgfpathcurveto{\pgfqpoint{1.228259in}{2.275440in}}{\pgfqpoint{1.231531in}{2.267540in}}{\pgfqpoint{1.237355in}{2.261716in}}%
\pgfpathcurveto{\pgfqpoint{1.243179in}{2.255892in}}{\pgfqpoint{1.251079in}{2.252620in}}{\pgfqpoint{1.259316in}{2.252620in}}%
\pgfpathclose%
\pgfusepath{stroke,fill}%
\end{pgfscope}%
\begin{pgfscope}%
\pgfpathrectangle{\pgfqpoint{0.100000in}{0.212622in}}{\pgfqpoint{3.696000in}{3.696000in}}%
\pgfusepath{clip}%
\pgfsetbuttcap%
\pgfsetroundjoin%
\definecolor{currentfill}{rgb}{0.121569,0.466667,0.705882}%
\pgfsetfillcolor{currentfill}%
\pgfsetfillopacity{0.540719}%
\pgfsetlinewidth{1.003750pt}%
\definecolor{currentstroke}{rgb}{0.121569,0.466667,0.705882}%
\pgfsetstrokecolor{currentstroke}%
\pgfsetstrokeopacity{0.540719}%
\pgfsetdash{}{0pt}%
\pgfpathmoveto{\pgfqpoint{2.099315in}{2.542605in}}%
\pgfpathcurveto{\pgfqpoint{2.107551in}{2.542605in}}{\pgfqpoint{2.115451in}{2.545877in}}{\pgfqpoint{2.121275in}{2.551701in}}%
\pgfpathcurveto{\pgfqpoint{2.127099in}{2.557525in}}{\pgfqpoint{2.130372in}{2.565425in}}{\pgfqpoint{2.130372in}{2.573661in}}%
\pgfpathcurveto{\pgfqpoint{2.130372in}{2.581898in}}{\pgfqpoint{2.127099in}{2.589798in}}{\pgfqpoint{2.121275in}{2.595621in}}%
\pgfpathcurveto{\pgfqpoint{2.115451in}{2.601445in}}{\pgfqpoint{2.107551in}{2.604718in}}{\pgfqpoint{2.099315in}{2.604718in}}%
\pgfpathcurveto{\pgfqpoint{2.091079in}{2.604718in}}{\pgfqpoint{2.083179in}{2.601445in}}{\pgfqpoint{2.077355in}{2.595621in}}%
\pgfpathcurveto{\pgfqpoint{2.071531in}{2.589798in}}{\pgfqpoint{2.068259in}{2.581898in}}{\pgfqpoint{2.068259in}{2.573661in}}%
\pgfpathcurveto{\pgfqpoint{2.068259in}{2.565425in}}{\pgfqpoint{2.071531in}{2.557525in}}{\pgfqpoint{2.077355in}{2.551701in}}%
\pgfpathcurveto{\pgfqpoint{2.083179in}{2.545877in}}{\pgfqpoint{2.091079in}{2.542605in}}{\pgfqpoint{2.099315in}{2.542605in}}%
\pgfpathclose%
\pgfusepath{stroke,fill}%
\end{pgfscope}%
\begin{pgfscope}%
\pgfpathrectangle{\pgfqpoint{0.100000in}{0.212622in}}{\pgfqpoint{3.696000in}{3.696000in}}%
\pgfusepath{clip}%
\pgfsetbuttcap%
\pgfsetroundjoin%
\definecolor{currentfill}{rgb}{0.121569,0.466667,0.705882}%
\pgfsetfillcolor{currentfill}%
\pgfsetfillopacity{0.542076}%
\pgfsetlinewidth{1.003750pt}%
\definecolor{currentstroke}{rgb}{0.121569,0.466667,0.705882}%
\pgfsetstrokecolor{currentstroke}%
\pgfsetstrokeopacity{0.542076}%
\pgfsetdash{}{0pt}%
\pgfpathmoveto{\pgfqpoint{1.256178in}{2.244479in}}%
\pgfpathcurveto{\pgfqpoint{1.264415in}{2.244479in}}{\pgfqpoint{1.272315in}{2.247751in}}{\pgfqpoint{1.278139in}{2.253575in}}%
\pgfpathcurveto{\pgfqpoint{1.283963in}{2.259399in}}{\pgfqpoint{1.287235in}{2.267299in}}{\pgfqpoint{1.287235in}{2.275536in}}%
\pgfpathcurveto{\pgfqpoint{1.287235in}{2.283772in}}{\pgfqpoint{1.283963in}{2.291672in}}{\pgfqpoint{1.278139in}{2.297496in}}%
\pgfpathcurveto{\pgfqpoint{1.272315in}{2.303320in}}{\pgfqpoint{1.264415in}{2.306592in}}{\pgfqpoint{1.256178in}{2.306592in}}%
\pgfpathcurveto{\pgfqpoint{1.247942in}{2.306592in}}{\pgfqpoint{1.240042in}{2.303320in}}{\pgfqpoint{1.234218in}{2.297496in}}%
\pgfpathcurveto{\pgfqpoint{1.228394in}{2.291672in}}{\pgfqpoint{1.225122in}{2.283772in}}{\pgfqpoint{1.225122in}{2.275536in}}%
\pgfpathcurveto{\pgfqpoint{1.225122in}{2.267299in}}{\pgfqpoint{1.228394in}{2.259399in}}{\pgfqpoint{1.234218in}{2.253575in}}%
\pgfpathcurveto{\pgfqpoint{1.240042in}{2.247751in}}{\pgfqpoint{1.247942in}{2.244479in}}{\pgfqpoint{1.256178in}{2.244479in}}%
\pgfpathclose%
\pgfusepath{stroke,fill}%
\end{pgfscope}%
\begin{pgfscope}%
\pgfpathrectangle{\pgfqpoint{0.100000in}{0.212622in}}{\pgfqpoint{3.696000in}{3.696000in}}%
\pgfusepath{clip}%
\pgfsetbuttcap%
\pgfsetroundjoin%
\definecolor{currentfill}{rgb}{0.121569,0.466667,0.705882}%
\pgfsetfillcolor{currentfill}%
\pgfsetfillopacity{0.545238}%
\pgfsetlinewidth{1.003750pt}%
\definecolor{currentstroke}{rgb}{0.121569,0.466667,0.705882}%
\pgfsetstrokecolor{currentstroke}%
\pgfsetstrokeopacity{0.545238}%
\pgfsetdash{}{0pt}%
\pgfpathmoveto{\pgfqpoint{1.250840in}{2.230379in}}%
\pgfpathcurveto{\pgfqpoint{1.259076in}{2.230379in}}{\pgfqpoint{1.266977in}{2.233651in}}{\pgfqpoint{1.272800in}{2.239475in}}%
\pgfpathcurveto{\pgfqpoint{1.278624in}{2.245299in}}{\pgfqpoint{1.281897in}{2.253199in}}{\pgfqpoint{1.281897in}{2.261436in}}%
\pgfpathcurveto{\pgfqpoint{1.281897in}{2.269672in}}{\pgfqpoint{1.278624in}{2.277572in}}{\pgfqpoint{1.272800in}{2.283396in}}%
\pgfpathcurveto{\pgfqpoint{1.266977in}{2.289220in}}{\pgfqpoint{1.259076in}{2.292492in}}{\pgfqpoint{1.250840in}{2.292492in}}%
\pgfpathcurveto{\pgfqpoint{1.242604in}{2.292492in}}{\pgfqpoint{1.234704in}{2.289220in}}{\pgfqpoint{1.228880in}{2.283396in}}%
\pgfpathcurveto{\pgfqpoint{1.223056in}{2.277572in}}{\pgfqpoint{1.219784in}{2.269672in}}{\pgfqpoint{1.219784in}{2.261436in}}%
\pgfpathcurveto{\pgfqpoint{1.219784in}{2.253199in}}{\pgfqpoint{1.223056in}{2.245299in}}{\pgfqpoint{1.228880in}{2.239475in}}%
\pgfpathcurveto{\pgfqpoint{1.234704in}{2.233651in}}{\pgfqpoint{1.242604in}{2.230379in}}{\pgfqpoint{1.250840in}{2.230379in}}%
\pgfpathclose%
\pgfusepath{stroke,fill}%
\end{pgfscope}%
\begin{pgfscope}%
\pgfpathrectangle{\pgfqpoint{0.100000in}{0.212622in}}{\pgfqpoint{3.696000in}{3.696000in}}%
\pgfusepath{clip}%
\pgfsetbuttcap%
\pgfsetroundjoin%
\definecolor{currentfill}{rgb}{0.121569,0.466667,0.705882}%
\pgfsetfillcolor{currentfill}%
\pgfsetfillopacity{0.545752}%
\pgfsetlinewidth{1.003750pt}%
\definecolor{currentstroke}{rgb}{0.121569,0.466667,0.705882}%
\pgfsetstrokecolor{currentstroke}%
\pgfsetstrokeopacity{0.545752}%
\pgfsetdash{}{0pt}%
\pgfpathmoveto{\pgfqpoint{2.105413in}{2.523792in}}%
\pgfpathcurveto{\pgfqpoint{2.113649in}{2.523792in}}{\pgfqpoint{2.121549in}{2.527064in}}{\pgfqpoint{2.127373in}{2.532888in}}%
\pgfpathcurveto{\pgfqpoint{2.133197in}{2.538712in}}{\pgfqpoint{2.136469in}{2.546612in}}{\pgfqpoint{2.136469in}{2.554848in}}%
\pgfpathcurveto{\pgfqpoint{2.136469in}{2.563085in}}{\pgfqpoint{2.133197in}{2.570985in}}{\pgfqpoint{2.127373in}{2.576809in}}%
\pgfpathcurveto{\pgfqpoint{2.121549in}{2.582633in}}{\pgfqpoint{2.113649in}{2.585905in}}{\pgfqpoint{2.105413in}{2.585905in}}%
\pgfpathcurveto{\pgfqpoint{2.097177in}{2.585905in}}{\pgfqpoint{2.089277in}{2.582633in}}{\pgfqpoint{2.083453in}{2.576809in}}%
\pgfpathcurveto{\pgfqpoint{2.077629in}{2.570985in}}{\pgfqpoint{2.074356in}{2.563085in}}{\pgfqpoint{2.074356in}{2.554848in}}%
\pgfpathcurveto{\pgfqpoint{2.074356in}{2.546612in}}{\pgfqpoint{2.077629in}{2.538712in}}{\pgfqpoint{2.083453in}{2.532888in}}%
\pgfpathcurveto{\pgfqpoint{2.089277in}{2.527064in}}{\pgfqpoint{2.097177in}{2.523792in}}{\pgfqpoint{2.105413in}{2.523792in}}%
\pgfpathclose%
\pgfusepath{stroke,fill}%
\end{pgfscope}%
\begin{pgfscope}%
\pgfpathrectangle{\pgfqpoint{0.100000in}{0.212622in}}{\pgfqpoint{3.696000in}{3.696000in}}%
\pgfusepath{clip}%
\pgfsetbuttcap%
\pgfsetroundjoin%
\definecolor{currentfill}{rgb}{0.121569,0.466667,0.705882}%
\pgfsetfillcolor{currentfill}%
\pgfsetfillopacity{0.547450}%
\pgfsetlinewidth{1.003750pt}%
\definecolor{currentstroke}{rgb}{0.121569,0.466667,0.705882}%
\pgfsetstrokecolor{currentstroke}%
\pgfsetstrokeopacity{0.547450}%
\pgfsetdash{}{0pt}%
\pgfpathmoveto{\pgfqpoint{1.245126in}{2.217801in}}%
\pgfpathcurveto{\pgfqpoint{1.253362in}{2.217801in}}{\pgfqpoint{1.261262in}{2.221073in}}{\pgfqpoint{1.267086in}{2.226897in}}%
\pgfpathcurveto{\pgfqpoint{1.272910in}{2.232721in}}{\pgfqpoint{1.276182in}{2.240621in}}{\pgfqpoint{1.276182in}{2.248857in}}%
\pgfpathcurveto{\pgfqpoint{1.276182in}{2.257094in}}{\pgfqpoint{1.272910in}{2.264994in}}{\pgfqpoint{1.267086in}{2.270818in}}%
\pgfpathcurveto{\pgfqpoint{1.261262in}{2.276642in}}{\pgfqpoint{1.253362in}{2.279914in}}{\pgfqpoint{1.245126in}{2.279914in}}%
\pgfpathcurveto{\pgfqpoint{1.236889in}{2.279914in}}{\pgfqpoint{1.228989in}{2.276642in}}{\pgfqpoint{1.223165in}{2.270818in}}%
\pgfpathcurveto{\pgfqpoint{1.217341in}{2.264994in}}{\pgfqpoint{1.214069in}{2.257094in}}{\pgfqpoint{1.214069in}{2.248857in}}%
\pgfpathcurveto{\pgfqpoint{1.214069in}{2.240621in}}{\pgfqpoint{1.217341in}{2.232721in}}{\pgfqpoint{1.223165in}{2.226897in}}%
\pgfpathcurveto{\pgfqpoint{1.228989in}{2.221073in}}{\pgfqpoint{1.236889in}{2.217801in}}{\pgfqpoint{1.245126in}{2.217801in}}%
\pgfpathclose%
\pgfusepath{stroke,fill}%
\end{pgfscope}%
\begin{pgfscope}%
\pgfpathrectangle{\pgfqpoint{0.100000in}{0.212622in}}{\pgfqpoint{3.696000in}{3.696000in}}%
\pgfusepath{clip}%
\pgfsetbuttcap%
\pgfsetroundjoin%
\definecolor{currentfill}{rgb}{0.121569,0.466667,0.705882}%
\pgfsetfillcolor{currentfill}%
\pgfsetfillopacity{0.549552}%
\pgfsetlinewidth{1.003750pt}%
\definecolor{currentstroke}{rgb}{0.121569,0.466667,0.705882}%
\pgfsetstrokecolor{currentstroke}%
\pgfsetstrokeopacity{0.549552}%
\pgfsetdash{}{0pt}%
\pgfpathmoveto{\pgfqpoint{1.240061in}{2.208035in}}%
\pgfpathcurveto{\pgfqpoint{1.248297in}{2.208035in}}{\pgfqpoint{1.256197in}{2.211308in}}{\pgfqpoint{1.262021in}{2.217132in}}%
\pgfpathcurveto{\pgfqpoint{1.267845in}{2.222956in}}{\pgfqpoint{1.271117in}{2.230856in}}{\pgfqpoint{1.271117in}{2.239092in}}%
\pgfpathcurveto{\pgfqpoint{1.271117in}{2.247328in}}{\pgfqpoint{1.267845in}{2.255228in}}{\pgfqpoint{1.262021in}{2.261052in}}%
\pgfpathcurveto{\pgfqpoint{1.256197in}{2.266876in}}{\pgfqpoint{1.248297in}{2.270148in}}{\pgfqpoint{1.240061in}{2.270148in}}%
\pgfpathcurveto{\pgfqpoint{1.231824in}{2.270148in}}{\pgfqpoint{1.223924in}{2.266876in}}{\pgfqpoint{1.218100in}{2.261052in}}%
\pgfpathcurveto{\pgfqpoint{1.212276in}{2.255228in}}{\pgfqpoint{1.209004in}{2.247328in}}{\pgfqpoint{1.209004in}{2.239092in}}%
\pgfpathcurveto{\pgfqpoint{1.209004in}{2.230856in}}{\pgfqpoint{1.212276in}{2.222956in}}{\pgfqpoint{1.218100in}{2.217132in}}%
\pgfpathcurveto{\pgfqpoint{1.223924in}{2.211308in}}{\pgfqpoint{1.231824in}{2.208035in}}{\pgfqpoint{1.240061in}{2.208035in}}%
\pgfpathclose%
\pgfusepath{stroke,fill}%
\end{pgfscope}%
\begin{pgfscope}%
\pgfpathrectangle{\pgfqpoint{0.100000in}{0.212622in}}{\pgfqpoint{3.696000in}{3.696000in}}%
\pgfusepath{clip}%
\pgfsetbuttcap%
\pgfsetroundjoin%
\definecolor{currentfill}{rgb}{0.121569,0.466667,0.705882}%
\pgfsetfillcolor{currentfill}%
\pgfsetfillopacity{0.551665}%
\pgfsetlinewidth{1.003750pt}%
\definecolor{currentstroke}{rgb}{0.121569,0.466667,0.705882}%
\pgfsetstrokecolor{currentstroke}%
\pgfsetstrokeopacity{0.551665}%
\pgfsetdash{}{0pt}%
\pgfpathmoveto{\pgfqpoint{2.113770in}{2.504264in}}%
\pgfpathcurveto{\pgfqpoint{2.122006in}{2.504264in}}{\pgfqpoint{2.129906in}{2.507536in}}{\pgfqpoint{2.135730in}{2.513360in}}%
\pgfpathcurveto{\pgfqpoint{2.141554in}{2.519184in}}{\pgfqpoint{2.144827in}{2.527084in}}{\pgfqpoint{2.144827in}{2.535320in}}%
\pgfpathcurveto{\pgfqpoint{2.144827in}{2.543556in}}{\pgfqpoint{2.141554in}{2.551456in}}{\pgfqpoint{2.135730in}{2.557280in}}%
\pgfpathcurveto{\pgfqpoint{2.129906in}{2.563104in}}{\pgfqpoint{2.122006in}{2.566377in}}{\pgfqpoint{2.113770in}{2.566377in}}%
\pgfpathcurveto{\pgfqpoint{2.105534in}{2.566377in}}{\pgfqpoint{2.097634in}{2.563104in}}{\pgfqpoint{2.091810in}{2.557280in}}%
\pgfpathcurveto{\pgfqpoint{2.085986in}{2.551456in}}{\pgfqpoint{2.082714in}{2.543556in}}{\pgfqpoint{2.082714in}{2.535320in}}%
\pgfpathcurveto{\pgfqpoint{2.082714in}{2.527084in}}{\pgfqpoint{2.085986in}{2.519184in}}{\pgfqpoint{2.091810in}{2.513360in}}%
\pgfpathcurveto{\pgfqpoint{2.097634in}{2.507536in}}{\pgfqpoint{2.105534in}{2.504264in}}{\pgfqpoint{2.113770in}{2.504264in}}%
\pgfpathclose%
\pgfusepath{stroke,fill}%
\end{pgfscope}%
\begin{pgfscope}%
\pgfpathrectangle{\pgfqpoint{0.100000in}{0.212622in}}{\pgfqpoint{3.696000in}{3.696000in}}%
\pgfusepath{clip}%
\pgfsetbuttcap%
\pgfsetroundjoin%
\definecolor{currentfill}{rgb}{0.121569,0.466667,0.705882}%
\pgfsetfillcolor{currentfill}%
\pgfsetfillopacity{0.553196}%
\pgfsetlinewidth{1.003750pt}%
\definecolor{currentstroke}{rgb}{0.121569,0.466667,0.705882}%
\pgfsetstrokecolor{currentstroke}%
\pgfsetstrokeopacity{0.553196}%
\pgfsetdash{}{0pt}%
\pgfpathmoveto{\pgfqpoint{1.230524in}{2.189952in}}%
\pgfpathcurveto{\pgfqpoint{1.238760in}{2.189952in}}{\pgfqpoint{1.246660in}{2.193224in}}{\pgfqpoint{1.252484in}{2.199048in}}%
\pgfpathcurveto{\pgfqpoint{1.258308in}{2.204872in}}{\pgfqpoint{1.261580in}{2.212772in}}{\pgfqpoint{1.261580in}{2.221009in}}%
\pgfpathcurveto{\pgfqpoint{1.261580in}{2.229245in}}{\pgfqpoint{1.258308in}{2.237145in}}{\pgfqpoint{1.252484in}{2.242969in}}%
\pgfpathcurveto{\pgfqpoint{1.246660in}{2.248793in}}{\pgfqpoint{1.238760in}{2.252065in}}{\pgfqpoint{1.230524in}{2.252065in}}%
\pgfpathcurveto{\pgfqpoint{1.222287in}{2.252065in}}{\pgfqpoint{1.214387in}{2.248793in}}{\pgfqpoint{1.208563in}{2.242969in}}%
\pgfpathcurveto{\pgfqpoint{1.202740in}{2.237145in}}{\pgfqpoint{1.199467in}{2.229245in}}{\pgfqpoint{1.199467in}{2.221009in}}%
\pgfpathcurveto{\pgfqpoint{1.199467in}{2.212772in}}{\pgfqpoint{1.202740in}{2.204872in}}{\pgfqpoint{1.208563in}{2.199048in}}%
\pgfpathcurveto{\pgfqpoint{1.214387in}{2.193224in}}{\pgfqpoint{1.222287in}{2.189952in}}{\pgfqpoint{1.230524in}{2.189952in}}%
\pgfpathclose%
\pgfusepath{stroke,fill}%
\end{pgfscope}%
\begin{pgfscope}%
\pgfpathrectangle{\pgfqpoint{0.100000in}{0.212622in}}{\pgfqpoint{3.696000in}{3.696000in}}%
\pgfusepath{clip}%
\pgfsetbuttcap%
\pgfsetroundjoin%
\definecolor{currentfill}{rgb}{0.121569,0.466667,0.705882}%
\pgfsetfillcolor{currentfill}%
\pgfsetfillopacity{0.557655}%
\pgfsetlinewidth{1.003750pt}%
\definecolor{currentstroke}{rgb}{0.121569,0.466667,0.705882}%
\pgfsetstrokecolor{currentstroke}%
\pgfsetstrokeopacity{0.557655}%
\pgfsetdash{}{0pt}%
\pgfpathmoveto{\pgfqpoint{2.119648in}{2.481079in}}%
\pgfpathcurveto{\pgfqpoint{2.127884in}{2.481079in}}{\pgfqpoint{2.135784in}{2.484352in}}{\pgfqpoint{2.141608in}{2.490175in}}%
\pgfpathcurveto{\pgfqpoint{2.147432in}{2.495999in}}{\pgfqpoint{2.150704in}{2.503899in}}{\pgfqpoint{2.150704in}{2.512136in}}%
\pgfpathcurveto{\pgfqpoint{2.150704in}{2.520372in}}{\pgfqpoint{2.147432in}{2.528272in}}{\pgfqpoint{2.141608in}{2.534096in}}%
\pgfpathcurveto{\pgfqpoint{2.135784in}{2.539920in}}{\pgfqpoint{2.127884in}{2.543192in}}{\pgfqpoint{2.119648in}{2.543192in}}%
\pgfpathcurveto{\pgfqpoint{2.111412in}{2.543192in}}{\pgfqpoint{2.103512in}{2.539920in}}{\pgfqpoint{2.097688in}{2.534096in}}%
\pgfpathcurveto{\pgfqpoint{2.091864in}{2.528272in}}{\pgfqpoint{2.088591in}{2.520372in}}{\pgfqpoint{2.088591in}{2.512136in}}%
\pgfpathcurveto{\pgfqpoint{2.088591in}{2.503899in}}{\pgfqpoint{2.091864in}{2.495999in}}{\pgfqpoint{2.097688in}{2.490175in}}%
\pgfpathcurveto{\pgfqpoint{2.103512in}{2.484352in}}{\pgfqpoint{2.111412in}{2.481079in}}{\pgfqpoint{2.119648in}{2.481079in}}%
\pgfpathclose%
\pgfusepath{stroke,fill}%
\end{pgfscope}%
\begin{pgfscope}%
\pgfpathrectangle{\pgfqpoint{0.100000in}{0.212622in}}{\pgfqpoint{3.696000in}{3.696000in}}%
\pgfusepath{clip}%
\pgfsetbuttcap%
\pgfsetroundjoin%
\definecolor{currentfill}{rgb}{0.121569,0.466667,0.705882}%
\pgfsetfillcolor{currentfill}%
\pgfsetfillopacity{0.559808}%
\pgfsetlinewidth{1.003750pt}%
\definecolor{currentstroke}{rgb}{0.121569,0.466667,0.705882}%
\pgfsetstrokecolor{currentstroke}%
\pgfsetstrokeopacity{0.559808}%
\pgfsetdash{}{0pt}%
\pgfpathmoveto{\pgfqpoint{1.213676in}{2.156372in}}%
\pgfpathcurveto{\pgfqpoint{1.221912in}{2.156372in}}{\pgfqpoint{1.229812in}{2.159644in}}{\pgfqpoint{1.235636in}{2.165468in}}%
\pgfpathcurveto{\pgfqpoint{1.241460in}{2.171292in}}{\pgfqpoint{1.244732in}{2.179192in}}{\pgfqpoint{1.244732in}{2.187428in}}%
\pgfpathcurveto{\pgfqpoint{1.244732in}{2.195665in}}{\pgfqpoint{1.241460in}{2.203565in}}{\pgfqpoint{1.235636in}{2.209389in}}%
\pgfpathcurveto{\pgfqpoint{1.229812in}{2.215213in}}{\pgfqpoint{1.221912in}{2.218485in}}{\pgfqpoint{1.213676in}{2.218485in}}%
\pgfpathcurveto{\pgfqpoint{1.205439in}{2.218485in}}{\pgfqpoint{1.197539in}{2.215213in}}{\pgfqpoint{1.191715in}{2.209389in}}%
\pgfpathcurveto{\pgfqpoint{1.185891in}{2.203565in}}{\pgfqpoint{1.182619in}{2.195665in}}{\pgfqpoint{1.182619in}{2.187428in}}%
\pgfpathcurveto{\pgfqpoint{1.182619in}{2.179192in}}{\pgfqpoint{1.185891in}{2.171292in}}{\pgfqpoint{1.191715in}{2.165468in}}%
\pgfpathcurveto{\pgfqpoint{1.197539in}{2.159644in}}{\pgfqpoint{1.205439in}{2.156372in}}{\pgfqpoint{1.213676in}{2.156372in}}%
\pgfpathclose%
\pgfusepath{stroke,fill}%
\end{pgfscope}%
\begin{pgfscope}%
\pgfpathrectangle{\pgfqpoint{0.100000in}{0.212622in}}{\pgfqpoint{3.696000in}{3.696000in}}%
\pgfusepath{clip}%
\pgfsetbuttcap%
\pgfsetroundjoin%
\definecolor{currentfill}{rgb}{0.121569,0.466667,0.705882}%
\pgfsetfillcolor{currentfill}%
\pgfsetfillopacity{0.564900}%
\pgfsetlinewidth{1.003750pt}%
\definecolor{currentstroke}{rgb}{0.121569,0.466667,0.705882}%
\pgfsetstrokecolor{currentstroke}%
\pgfsetstrokeopacity{0.564900}%
\pgfsetdash{}{0pt}%
\pgfpathmoveto{\pgfqpoint{2.129149in}{2.457501in}}%
\pgfpathcurveto{\pgfqpoint{2.137386in}{2.457501in}}{\pgfqpoint{2.145286in}{2.460773in}}{\pgfqpoint{2.151110in}{2.466597in}}%
\pgfpathcurveto{\pgfqpoint{2.156933in}{2.472421in}}{\pgfqpoint{2.160206in}{2.480321in}}{\pgfqpoint{2.160206in}{2.488557in}}%
\pgfpathcurveto{\pgfqpoint{2.160206in}{2.496794in}}{\pgfqpoint{2.156933in}{2.504694in}}{\pgfqpoint{2.151110in}{2.510518in}}%
\pgfpathcurveto{\pgfqpoint{2.145286in}{2.516342in}}{\pgfqpoint{2.137386in}{2.519614in}}{\pgfqpoint{2.129149in}{2.519614in}}%
\pgfpathcurveto{\pgfqpoint{2.120913in}{2.519614in}}{\pgfqpoint{2.113013in}{2.516342in}}{\pgfqpoint{2.107189in}{2.510518in}}%
\pgfpathcurveto{\pgfqpoint{2.101365in}{2.504694in}}{\pgfqpoint{2.098093in}{2.496794in}}{\pgfqpoint{2.098093in}{2.488557in}}%
\pgfpathcurveto{\pgfqpoint{2.098093in}{2.480321in}}{\pgfqpoint{2.101365in}{2.472421in}}{\pgfqpoint{2.107189in}{2.466597in}}%
\pgfpathcurveto{\pgfqpoint{2.113013in}{2.460773in}}{\pgfqpoint{2.120913in}{2.457501in}}{\pgfqpoint{2.129149in}{2.457501in}}%
\pgfpathclose%
\pgfusepath{stroke,fill}%
\end{pgfscope}%
\begin{pgfscope}%
\pgfpathrectangle{\pgfqpoint{0.100000in}{0.212622in}}{\pgfqpoint{3.696000in}{3.696000in}}%
\pgfusepath{clip}%
\pgfsetbuttcap%
\pgfsetroundjoin%
\definecolor{currentfill}{rgb}{0.121569,0.466667,0.705882}%
\pgfsetfillcolor{currentfill}%
\pgfsetfillopacity{0.565582}%
\pgfsetlinewidth{1.003750pt}%
\definecolor{currentstroke}{rgb}{0.121569,0.466667,0.705882}%
\pgfsetstrokecolor{currentstroke}%
\pgfsetstrokeopacity{0.565582}%
\pgfsetdash{}{0pt}%
\pgfpathmoveto{\pgfqpoint{1.199333in}{2.129339in}}%
\pgfpathcurveto{\pgfqpoint{1.207569in}{2.129339in}}{\pgfqpoint{1.215469in}{2.132612in}}{\pgfqpoint{1.221293in}{2.138436in}}%
\pgfpathcurveto{\pgfqpoint{1.227117in}{2.144259in}}{\pgfqpoint{1.230390in}{2.152159in}}{\pgfqpoint{1.230390in}{2.160396in}}%
\pgfpathcurveto{\pgfqpoint{1.230390in}{2.168632in}}{\pgfqpoint{1.227117in}{2.176532in}}{\pgfqpoint{1.221293in}{2.182356in}}%
\pgfpathcurveto{\pgfqpoint{1.215469in}{2.188180in}}{\pgfqpoint{1.207569in}{2.191452in}}{\pgfqpoint{1.199333in}{2.191452in}}%
\pgfpathcurveto{\pgfqpoint{1.191097in}{2.191452in}}{\pgfqpoint{1.183197in}{2.188180in}}{\pgfqpoint{1.177373in}{2.182356in}}%
\pgfpathcurveto{\pgfqpoint{1.171549in}{2.176532in}}{\pgfqpoint{1.168277in}{2.168632in}}{\pgfqpoint{1.168277in}{2.160396in}}%
\pgfpathcurveto{\pgfqpoint{1.168277in}{2.152159in}}{\pgfqpoint{1.171549in}{2.144259in}}{\pgfqpoint{1.177373in}{2.138436in}}%
\pgfpathcurveto{\pgfqpoint{1.183197in}{2.132612in}}{\pgfqpoint{1.191097in}{2.129339in}}{\pgfqpoint{1.199333in}{2.129339in}}%
\pgfpathclose%
\pgfusepath{stroke,fill}%
\end{pgfscope}%
\begin{pgfscope}%
\pgfpathrectangle{\pgfqpoint{0.100000in}{0.212622in}}{\pgfqpoint{3.696000in}{3.696000in}}%
\pgfusepath{clip}%
\pgfsetbuttcap%
\pgfsetroundjoin%
\definecolor{currentfill}{rgb}{0.121569,0.466667,0.705882}%
\pgfsetfillcolor{currentfill}%
\pgfsetfillopacity{0.569210}%
\pgfsetlinewidth{1.003750pt}%
\definecolor{currentstroke}{rgb}{0.121569,0.466667,0.705882}%
\pgfsetstrokecolor{currentstroke}%
\pgfsetstrokeopacity{0.569210}%
\pgfsetdash{}{0pt}%
\pgfpathmoveto{\pgfqpoint{1.190632in}{2.111754in}}%
\pgfpathcurveto{\pgfqpoint{1.198869in}{2.111754in}}{\pgfqpoint{1.206769in}{2.115026in}}{\pgfqpoint{1.212593in}{2.120850in}}%
\pgfpathcurveto{\pgfqpoint{1.218417in}{2.126674in}}{\pgfqpoint{1.221689in}{2.134574in}}{\pgfqpoint{1.221689in}{2.142810in}}%
\pgfpathcurveto{\pgfqpoint{1.221689in}{2.151046in}}{\pgfqpoint{1.218417in}{2.158946in}}{\pgfqpoint{1.212593in}{2.164770in}}%
\pgfpathcurveto{\pgfqpoint{1.206769in}{2.170594in}}{\pgfqpoint{1.198869in}{2.173867in}}{\pgfqpoint{1.190632in}{2.173867in}}%
\pgfpathcurveto{\pgfqpoint{1.182396in}{2.173867in}}{\pgfqpoint{1.174496in}{2.170594in}}{\pgfqpoint{1.168672in}{2.164770in}}%
\pgfpathcurveto{\pgfqpoint{1.162848in}{2.158946in}}{\pgfqpoint{1.159576in}{2.151046in}}{\pgfqpoint{1.159576in}{2.142810in}}%
\pgfpathcurveto{\pgfqpoint{1.159576in}{2.134574in}}{\pgfqpoint{1.162848in}{2.126674in}}{\pgfqpoint{1.168672in}{2.120850in}}%
\pgfpathcurveto{\pgfqpoint{1.174496in}{2.115026in}}{\pgfqpoint{1.182396in}{2.111754in}}{\pgfqpoint{1.190632in}{2.111754in}}%
\pgfpathclose%
\pgfusepath{stroke,fill}%
\end{pgfscope}%
\begin{pgfscope}%
\pgfpathrectangle{\pgfqpoint{0.100000in}{0.212622in}}{\pgfqpoint{3.696000in}{3.696000in}}%
\pgfusepath{clip}%
\pgfsetbuttcap%
\pgfsetroundjoin%
\definecolor{currentfill}{rgb}{0.121569,0.466667,0.705882}%
\pgfsetfillcolor{currentfill}%
\pgfsetfillopacity{0.571923}%
\pgfsetlinewidth{1.003750pt}%
\definecolor{currentstroke}{rgb}{0.121569,0.466667,0.705882}%
\pgfsetstrokecolor{currentstroke}%
\pgfsetstrokeopacity{0.571923}%
\pgfsetdash{}{0pt}%
\pgfpathmoveto{\pgfqpoint{1.184533in}{2.099168in}}%
\pgfpathcurveto{\pgfqpoint{1.192769in}{2.099168in}}{\pgfqpoint{1.200669in}{2.102441in}}{\pgfqpoint{1.206493in}{2.108265in}}%
\pgfpathcurveto{\pgfqpoint{1.212317in}{2.114089in}}{\pgfqpoint{1.215589in}{2.121989in}}{\pgfqpoint{1.215589in}{2.130225in}}%
\pgfpathcurveto{\pgfqpoint{1.215589in}{2.138461in}}{\pgfqpoint{1.212317in}{2.146361in}}{\pgfqpoint{1.206493in}{2.152185in}}%
\pgfpathcurveto{\pgfqpoint{1.200669in}{2.158009in}}{\pgfqpoint{1.192769in}{2.161281in}}{\pgfqpoint{1.184533in}{2.161281in}}%
\pgfpathcurveto{\pgfqpoint{1.176296in}{2.161281in}}{\pgfqpoint{1.168396in}{2.158009in}}{\pgfqpoint{1.162572in}{2.152185in}}%
\pgfpathcurveto{\pgfqpoint{1.156748in}{2.146361in}}{\pgfqpoint{1.153476in}{2.138461in}}{\pgfqpoint{1.153476in}{2.130225in}}%
\pgfpathcurveto{\pgfqpoint{1.153476in}{2.121989in}}{\pgfqpoint{1.156748in}{2.114089in}}{\pgfqpoint{1.162572in}{2.108265in}}%
\pgfpathcurveto{\pgfqpoint{1.168396in}{2.102441in}}{\pgfqpoint{1.176296in}{2.099168in}}{\pgfqpoint{1.184533in}{2.099168in}}%
\pgfpathclose%
\pgfusepath{stroke,fill}%
\end{pgfscope}%
\begin{pgfscope}%
\pgfpathrectangle{\pgfqpoint{0.100000in}{0.212622in}}{\pgfqpoint{3.696000in}{3.696000in}}%
\pgfusepath{clip}%
\pgfsetbuttcap%
\pgfsetroundjoin%
\definecolor{currentfill}{rgb}{0.121569,0.466667,0.705882}%
\pgfsetfillcolor{currentfill}%
\pgfsetfillopacity{0.572631}%
\pgfsetlinewidth{1.003750pt}%
\definecolor{currentstroke}{rgb}{0.121569,0.466667,0.705882}%
\pgfsetstrokecolor{currentstroke}%
\pgfsetstrokeopacity{0.572631}%
\pgfsetdash{}{0pt}%
\pgfpathmoveto{\pgfqpoint{2.137307in}{2.428065in}}%
\pgfpathcurveto{\pgfqpoint{2.145543in}{2.428065in}}{\pgfqpoint{2.153443in}{2.431337in}}{\pgfqpoint{2.159267in}{2.437161in}}%
\pgfpathcurveto{\pgfqpoint{2.165091in}{2.442985in}}{\pgfqpoint{2.168363in}{2.450885in}}{\pgfqpoint{2.168363in}{2.459121in}}%
\pgfpathcurveto{\pgfqpoint{2.168363in}{2.467357in}}{\pgfqpoint{2.165091in}{2.475258in}}{\pgfqpoint{2.159267in}{2.481081in}}%
\pgfpathcurveto{\pgfqpoint{2.153443in}{2.486905in}}{\pgfqpoint{2.145543in}{2.490178in}}{\pgfqpoint{2.137307in}{2.490178in}}%
\pgfpathcurveto{\pgfqpoint{2.129070in}{2.490178in}}{\pgfqpoint{2.121170in}{2.486905in}}{\pgfqpoint{2.115346in}{2.481081in}}%
\pgfpathcurveto{\pgfqpoint{2.109522in}{2.475258in}}{\pgfqpoint{2.106250in}{2.467357in}}{\pgfqpoint{2.106250in}{2.459121in}}%
\pgfpathcurveto{\pgfqpoint{2.106250in}{2.450885in}}{\pgfqpoint{2.109522in}{2.442985in}}{\pgfqpoint{2.115346in}{2.437161in}}%
\pgfpathcurveto{\pgfqpoint{2.121170in}{2.431337in}}{\pgfqpoint{2.129070in}{2.428065in}}{\pgfqpoint{2.137307in}{2.428065in}}%
\pgfpathclose%
\pgfusepath{stroke,fill}%
\end{pgfscope}%
\begin{pgfscope}%
\pgfpathrectangle{\pgfqpoint{0.100000in}{0.212622in}}{\pgfqpoint{3.696000in}{3.696000in}}%
\pgfusepath{clip}%
\pgfsetbuttcap%
\pgfsetroundjoin%
\definecolor{currentfill}{rgb}{0.121569,0.466667,0.705882}%
\pgfsetfillcolor{currentfill}%
\pgfsetfillopacity{0.572738}%
\pgfsetlinewidth{1.003750pt}%
\definecolor{currentstroke}{rgb}{0.121569,0.466667,0.705882}%
\pgfsetstrokecolor{currentstroke}%
\pgfsetstrokeopacity{0.572738}%
\pgfsetdash{}{0pt}%
\pgfpathmoveto{\pgfqpoint{1.182725in}{2.095232in}}%
\pgfpathcurveto{\pgfqpoint{1.190961in}{2.095232in}}{\pgfqpoint{1.198861in}{2.098505in}}{\pgfqpoint{1.204685in}{2.104329in}}%
\pgfpathcurveto{\pgfqpoint{1.210509in}{2.110153in}}{\pgfqpoint{1.213781in}{2.118053in}}{\pgfqpoint{1.213781in}{2.126289in}}%
\pgfpathcurveto{\pgfqpoint{1.213781in}{2.134525in}}{\pgfqpoint{1.210509in}{2.142425in}}{\pgfqpoint{1.204685in}{2.148249in}}%
\pgfpathcurveto{\pgfqpoint{1.198861in}{2.154073in}}{\pgfqpoint{1.190961in}{2.157345in}}{\pgfqpoint{1.182725in}{2.157345in}}%
\pgfpathcurveto{\pgfqpoint{1.174488in}{2.157345in}}{\pgfqpoint{1.166588in}{2.154073in}}{\pgfqpoint{1.160764in}{2.148249in}}%
\pgfpathcurveto{\pgfqpoint{1.154941in}{2.142425in}}{\pgfqpoint{1.151668in}{2.134525in}}{\pgfqpoint{1.151668in}{2.126289in}}%
\pgfpathcurveto{\pgfqpoint{1.151668in}{2.118053in}}{\pgfqpoint{1.154941in}{2.110153in}}{\pgfqpoint{1.160764in}{2.104329in}}%
\pgfpathcurveto{\pgfqpoint{1.166588in}{2.098505in}}{\pgfqpoint{1.174488in}{2.095232in}}{\pgfqpoint{1.182725in}{2.095232in}}%
\pgfpathclose%
\pgfusepath{stroke,fill}%
\end{pgfscope}%
\begin{pgfscope}%
\pgfpathrectangle{\pgfqpoint{0.100000in}{0.212622in}}{\pgfqpoint{3.696000in}{3.696000in}}%
\pgfusepath{clip}%
\pgfsetbuttcap%
\pgfsetroundjoin%
\definecolor{currentfill}{rgb}{0.121569,0.466667,0.705882}%
\pgfsetfillcolor{currentfill}%
\pgfsetfillopacity{0.574110}%
\pgfsetlinewidth{1.003750pt}%
\definecolor{currentstroke}{rgb}{0.121569,0.466667,0.705882}%
\pgfsetstrokecolor{currentstroke}%
\pgfsetstrokeopacity{0.574110}%
\pgfsetdash{}{0pt}%
\pgfpathmoveto{\pgfqpoint{1.179168in}{2.087952in}}%
\pgfpathcurveto{\pgfqpoint{1.187404in}{2.087952in}}{\pgfqpoint{1.195305in}{2.091224in}}{\pgfqpoint{1.201128in}{2.097048in}}%
\pgfpathcurveto{\pgfqpoint{1.206952in}{2.102872in}}{\pgfqpoint{1.210225in}{2.110772in}}{\pgfqpoint{1.210225in}{2.119008in}}%
\pgfpathcurveto{\pgfqpoint{1.210225in}{2.127245in}}{\pgfqpoint{1.206952in}{2.135145in}}{\pgfqpoint{1.201128in}{2.140969in}}%
\pgfpathcurveto{\pgfqpoint{1.195305in}{2.146792in}}{\pgfqpoint{1.187404in}{2.150065in}}{\pgfqpoint{1.179168in}{2.150065in}}%
\pgfpathcurveto{\pgfqpoint{1.170932in}{2.150065in}}{\pgfqpoint{1.163032in}{2.146792in}}{\pgfqpoint{1.157208in}{2.140969in}}%
\pgfpathcurveto{\pgfqpoint{1.151384in}{2.135145in}}{\pgfqpoint{1.148112in}{2.127245in}}{\pgfqpoint{1.148112in}{2.119008in}}%
\pgfpathcurveto{\pgfqpoint{1.148112in}{2.110772in}}{\pgfqpoint{1.151384in}{2.102872in}}{\pgfqpoint{1.157208in}{2.097048in}}%
\pgfpathcurveto{\pgfqpoint{1.163032in}{2.091224in}}{\pgfqpoint{1.170932in}{2.087952in}}{\pgfqpoint{1.179168in}{2.087952in}}%
\pgfpathclose%
\pgfusepath{stroke,fill}%
\end{pgfscope}%
\begin{pgfscope}%
\pgfpathrectangle{\pgfqpoint{0.100000in}{0.212622in}}{\pgfqpoint{3.696000in}{3.696000in}}%
\pgfusepath{clip}%
\pgfsetbuttcap%
\pgfsetroundjoin%
\definecolor{currentfill}{rgb}{0.121569,0.466667,0.705882}%
\pgfsetfillcolor{currentfill}%
\pgfsetfillopacity{0.576692}%
\pgfsetlinewidth{1.003750pt}%
\definecolor{currentstroke}{rgb}{0.121569,0.466667,0.705882}%
\pgfsetstrokecolor{currentstroke}%
\pgfsetstrokeopacity{0.576692}%
\pgfsetdash{}{0pt}%
\pgfpathmoveto{\pgfqpoint{1.172871in}{2.074846in}}%
\pgfpathcurveto{\pgfqpoint{1.181107in}{2.074846in}}{\pgfqpoint{1.189007in}{2.078118in}}{\pgfqpoint{1.194831in}{2.083942in}}%
\pgfpathcurveto{\pgfqpoint{1.200655in}{2.089766in}}{\pgfqpoint{1.203927in}{2.097666in}}{\pgfqpoint{1.203927in}{2.105902in}}%
\pgfpathcurveto{\pgfqpoint{1.203927in}{2.114139in}}{\pgfqpoint{1.200655in}{2.122039in}}{\pgfqpoint{1.194831in}{2.127863in}}%
\pgfpathcurveto{\pgfqpoint{1.189007in}{2.133687in}}{\pgfqpoint{1.181107in}{2.136959in}}{\pgfqpoint{1.172871in}{2.136959in}}%
\pgfpathcurveto{\pgfqpoint{1.164635in}{2.136959in}}{\pgfqpoint{1.156735in}{2.133687in}}{\pgfqpoint{1.150911in}{2.127863in}}%
\pgfpathcurveto{\pgfqpoint{1.145087in}{2.122039in}}{\pgfqpoint{1.141814in}{2.114139in}}{\pgfqpoint{1.141814in}{2.105902in}}%
\pgfpathcurveto{\pgfqpoint{1.141814in}{2.097666in}}{\pgfqpoint{1.145087in}{2.089766in}}{\pgfqpoint{1.150911in}{2.083942in}}%
\pgfpathcurveto{\pgfqpoint{1.156735in}{2.078118in}}{\pgfqpoint{1.164635in}{2.074846in}}{\pgfqpoint{1.172871in}{2.074846in}}%
\pgfpathclose%
\pgfusepath{stroke,fill}%
\end{pgfscope}%
\begin{pgfscope}%
\pgfpathrectangle{\pgfqpoint{0.100000in}{0.212622in}}{\pgfqpoint{3.696000in}{3.696000in}}%
\pgfusepath{clip}%
\pgfsetbuttcap%
\pgfsetroundjoin%
\definecolor{currentfill}{rgb}{0.121569,0.466667,0.705882}%
\pgfsetfillcolor{currentfill}%
\pgfsetfillopacity{0.577082}%
\pgfsetlinewidth{1.003750pt}%
\definecolor{currentstroke}{rgb}{0.121569,0.466667,0.705882}%
\pgfsetstrokecolor{currentstroke}%
\pgfsetstrokeopacity{0.577082}%
\pgfsetdash{}{0pt}%
\pgfpathmoveto{\pgfqpoint{2.142559in}{2.412577in}}%
\pgfpathcurveto{\pgfqpoint{2.150795in}{2.412577in}}{\pgfqpoint{2.158695in}{2.415849in}}{\pgfqpoint{2.164519in}{2.421673in}}%
\pgfpathcurveto{\pgfqpoint{2.170343in}{2.427497in}}{\pgfqpoint{2.173615in}{2.435397in}}{\pgfqpoint{2.173615in}{2.443633in}}%
\pgfpathcurveto{\pgfqpoint{2.173615in}{2.451870in}}{\pgfqpoint{2.170343in}{2.459770in}}{\pgfqpoint{2.164519in}{2.465594in}}%
\pgfpathcurveto{\pgfqpoint{2.158695in}{2.471418in}}{\pgfqpoint{2.150795in}{2.474690in}}{\pgfqpoint{2.142559in}{2.474690in}}%
\pgfpathcurveto{\pgfqpoint{2.134322in}{2.474690in}}{\pgfqpoint{2.126422in}{2.471418in}}{\pgfqpoint{2.120598in}{2.465594in}}%
\pgfpathcurveto{\pgfqpoint{2.114775in}{2.459770in}}{\pgfqpoint{2.111502in}{2.451870in}}{\pgfqpoint{2.111502in}{2.443633in}}%
\pgfpathcurveto{\pgfqpoint{2.111502in}{2.435397in}}{\pgfqpoint{2.114775in}{2.427497in}}{\pgfqpoint{2.120598in}{2.421673in}}%
\pgfpathcurveto{\pgfqpoint{2.126422in}{2.415849in}}{\pgfqpoint{2.134322in}{2.412577in}}{\pgfqpoint{2.142559in}{2.412577in}}%
\pgfpathclose%
\pgfusepath{stroke,fill}%
\end{pgfscope}%
\begin{pgfscope}%
\pgfpathrectangle{\pgfqpoint{0.100000in}{0.212622in}}{\pgfqpoint{3.696000in}{3.696000in}}%
\pgfusepath{clip}%
\pgfsetbuttcap%
\pgfsetroundjoin%
\definecolor{currentfill}{rgb}{0.121569,0.466667,0.705882}%
\pgfsetfillcolor{currentfill}%
\pgfsetfillopacity{0.577612}%
\pgfsetlinewidth{1.003750pt}%
\definecolor{currentstroke}{rgb}{0.121569,0.466667,0.705882}%
\pgfsetstrokecolor{currentstroke}%
\pgfsetstrokeopacity{0.577612}%
\pgfsetdash{}{0pt}%
\pgfpathmoveto{\pgfqpoint{1.170719in}{2.070500in}}%
\pgfpathcurveto{\pgfqpoint{1.178956in}{2.070500in}}{\pgfqpoint{1.186856in}{2.073773in}}{\pgfqpoint{1.192680in}{2.079597in}}%
\pgfpathcurveto{\pgfqpoint{1.198503in}{2.085420in}}{\pgfqpoint{1.201776in}{2.093321in}}{\pgfqpoint{1.201776in}{2.101557in}}%
\pgfpathcurveto{\pgfqpoint{1.201776in}{2.109793in}}{\pgfqpoint{1.198503in}{2.117693in}}{\pgfqpoint{1.192680in}{2.123517in}}%
\pgfpathcurveto{\pgfqpoint{1.186856in}{2.129341in}}{\pgfqpoint{1.178956in}{2.132613in}}{\pgfqpoint{1.170719in}{2.132613in}}%
\pgfpathcurveto{\pgfqpoint{1.162483in}{2.132613in}}{\pgfqpoint{1.154583in}{2.129341in}}{\pgfqpoint{1.148759in}{2.123517in}}%
\pgfpathcurveto{\pgfqpoint{1.142935in}{2.117693in}}{\pgfqpoint{1.139663in}{2.109793in}}{\pgfqpoint{1.139663in}{2.101557in}}%
\pgfpathcurveto{\pgfqpoint{1.139663in}{2.093321in}}{\pgfqpoint{1.142935in}{2.085420in}}{\pgfqpoint{1.148759in}{2.079597in}}%
\pgfpathcurveto{\pgfqpoint{1.154583in}{2.073773in}}{\pgfqpoint{1.162483in}{2.070500in}}{\pgfqpoint{1.170719in}{2.070500in}}%
\pgfpathclose%
\pgfusepath{stroke,fill}%
\end{pgfscope}%
\begin{pgfscope}%
\pgfpathrectangle{\pgfqpoint{0.100000in}{0.212622in}}{\pgfqpoint{3.696000in}{3.696000in}}%
\pgfusepath{clip}%
\pgfsetbuttcap%
\pgfsetroundjoin%
\definecolor{currentfill}{rgb}{0.121569,0.466667,0.705882}%
\pgfsetfillcolor{currentfill}%
\pgfsetfillopacity{0.578138}%
\pgfsetlinewidth{1.003750pt}%
\definecolor{currentstroke}{rgb}{0.121569,0.466667,0.705882}%
\pgfsetstrokecolor{currentstroke}%
\pgfsetstrokeopacity{0.578138}%
\pgfsetdash{}{0pt}%
\pgfpathmoveto{\pgfqpoint{1.169369in}{2.067909in}}%
\pgfpathcurveto{\pgfqpoint{1.177605in}{2.067909in}}{\pgfqpoint{1.185505in}{2.071182in}}{\pgfqpoint{1.191329in}{2.077006in}}%
\pgfpathcurveto{\pgfqpoint{1.197153in}{2.082829in}}{\pgfqpoint{1.200425in}{2.090730in}}{\pgfqpoint{1.200425in}{2.098966in}}%
\pgfpathcurveto{\pgfqpoint{1.200425in}{2.107202in}}{\pgfqpoint{1.197153in}{2.115102in}}{\pgfqpoint{1.191329in}{2.120926in}}%
\pgfpathcurveto{\pgfqpoint{1.185505in}{2.126750in}}{\pgfqpoint{1.177605in}{2.130022in}}{\pgfqpoint{1.169369in}{2.130022in}}%
\pgfpathcurveto{\pgfqpoint{1.161133in}{2.130022in}}{\pgfqpoint{1.153233in}{2.126750in}}{\pgfqpoint{1.147409in}{2.120926in}}%
\pgfpathcurveto{\pgfqpoint{1.141585in}{2.115102in}}{\pgfqpoint{1.138312in}{2.107202in}}{\pgfqpoint{1.138312in}{2.098966in}}%
\pgfpathcurveto{\pgfqpoint{1.138312in}{2.090730in}}{\pgfqpoint{1.141585in}{2.082829in}}{\pgfqpoint{1.147409in}{2.077006in}}%
\pgfpathcurveto{\pgfqpoint{1.153233in}{2.071182in}}{\pgfqpoint{1.161133in}{2.067909in}}{\pgfqpoint{1.169369in}{2.067909in}}%
\pgfpathclose%
\pgfusepath{stroke,fill}%
\end{pgfscope}%
\begin{pgfscope}%
\pgfpathrectangle{\pgfqpoint{0.100000in}{0.212622in}}{\pgfqpoint{3.696000in}{3.696000in}}%
\pgfusepath{clip}%
\pgfsetbuttcap%
\pgfsetroundjoin%
\definecolor{currentfill}{rgb}{0.121569,0.466667,0.705882}%
\pgfsetfillcolor{currentfill}%
\pgfsetfillopacity{0.578297}%
\pgfsetlinewidth{1.003750pt}%
\definecolor{currentstroke}{rgb}{0.121569,0.466667,0.705882}%
\pgfsetstrokecolor{currentstroke}%
\pgfsetstrokeopacity{0.578297}%
\pgfsetdash{}{0pt}%
\pgfpathmoveto{\pgfqpoint{1.168984in}{2.067171in}}%
\pgfpathcurveto{\pgfqpoint{1.177221in}{2.067171in}}{\pgfqpoint{1.185121in}{2.070443in}}{\pgfqpoint{1.190945in}{2.076267in}}%
\pgfpathcurveto{\pgfqpoint{1.196768in}{2.082091in}}{\pgfqpoint{1.200041in}{2.089991in}}{\pgfqpoint{1.200041in}{2.098227in}}%
\pgfpathcurveto{\pgfqpoint{1.200041in}{2.106464in}}{\pgfqpoint{1.196768in}{2.114364in}}{\pgfqpoint{1.190945in}{2.120188in}}%
\pgfpathcurveto{\pgfqpoint{1.185121in}{2.126012in}}{\pgfqpoint{1.177221in}{2.129284in}}{\pgfqpoint{1.168984in}{2.129284in}}%
\pgfpathcurveto{\pgfqpoint{1.160748in}{2.129284in}}{\pgfqpoint{1.152848in}{2.126012in}}{\pgfqpoint{1.147024in}{2.120188in}}%
\pgfpathcurveto{\pgfqpoint{1.141200in}{2.114364in}}{\pgfqpoint{1.137928in}{2.106464in}}{\pgfqpoint{1.137928in}{2.098227in}}%
\pgfpathcurveto{\pgfqpoint{1.137928in}{2.089991in}}{\pgfqpoint{1.141200in}{2.082091in}}{\pgfqpoint{1.147024in}{2.076267in}}%
\pgfpathcurveto{\pgfqpoint{1.152848in}{2.070443in}}{\pgfqpoint{1.160748in}{2.067171in}}{\pgfqpoint{1.168984in}{2.067171in}}%
\pgfpathclose%
\pgfusepath{stroke,fill}%
\end{pgfscope}%
\begin{pgfscope}%
\pgfpathrectangle{\pgfqpoint{0.100000in}{0.212622in}}{\pgfqpoint{3.696000in}{3.696000in}}%
\pgfusepath{clip}%
\pgfsetbuttcap%
\pgfsetroundjoin%
\definecolor{currentfill}{rgb}{0.121569,0.466667,0.705882}%
\pgfsetfillcolor{currentfill}%
\pgfsetfillopacity{0.578581}%
\pgfsetlinewidth{1.003750pt}%
\definecolor{currentstroke}{rgb}{0.121569,0.466667,0.705882}%
\pgfsetstrokecolor{currentstroke}%
\pgfsetstrokeopacity{0.578581}%
\pgfsetdash{}{0pt}%
\pgfpathmoveto{\pgfqpoint{1.168297in}{2.065800in}}%
\pgfpathcurveto{\pgfqpoint{1.176534in}{2.065800in}}{\pgfqpoint{1.184434in}{2.069073in}}{\pgfqpoint{1.190258in}{2.074897in}}%
\pgfpathcurveto{\pgfqpoint{1.196081in}{2.080721in}}{\pgfqpoint{1.199354in}{2.088621in}}{\pgfqpoint{1.199354in}{2.096857in}}%
\pgfpathcurveto{\pgfqpoint{1.199354in}{2.105093in}}{\pgfqpoint{1.196081in}{2.112993in}}{\pgfqpoint{1.190258in}{2.118817in}}%
\pgfpathcurveto{\pgfqpoint{1.184434in}{2.124641in}}{\pgfqpoint{1.176534in}{2.127913in}}{\pgfqpoint{1.168297in}{2.127913in}}%
\pgfpathcurveto{\pgfqpoint{1.160061in}{2.127913in}}{\pgfqpoint{1.152161in}{2.124641in}}{\pgfqpoint{1.146337in}{2.118817in}}%
\pgfpathcurveto{\pgfqpoint{1.140513in}{2.112993in}}{\pgfqpoint{1.137241in}{2.105093in}}{\pgfqpoint{1.137241in}{2.096857in}}%
\pgfpathcurveto{\pgfqpoint{1.137241in}{2.088621in}}{\pgfqpoint{1.140513in}{2.080721in}}{\pgfqpoint{1.146337in}{2.074897in}}%
\pgfpathcurveto{\pgfqpoint{1.152161in}{2.069073in}}{\pgfqpoint{1.160061in}{2.065800in}}{\pgfqpoint{1.168297in}{2.065800in}}%
\pgfpathclose%
\pgfusepath{stroke,fill}%
\end{pgfscope}%
\begin{pgfscope}%
\pgfpathrectangle{\pgfqpoint{0.100000in}{0.212622in}}{\pgfqpoint{3.696000in}{3.696000in}}%
\pgfusepath{clip}%
\pgfsetbuttcap%
\pgfsetroundjoin%
\definecolor{currentfill}{rgb}{0.121569,0.466667,0.705882}%
\pgfsetfillcolor{currentfill}%
\pgfsetfillopacity{0.579103}%
\pgfsetlinewidth{1.003750pt}%
\definecolor{currentstroke}{rgb}{0.121569,0.466667,0.705882}%
\pgfsetstrokecolor{currentstroke}%
\pgfsetstrokeopacity{0.579103}%
\pgfsetdash{}{0pt}%
\pgfpathmoveto{\pgfqpoint{1.167063in}{2.063305in}}%
\pgfpathcurveto{\pgfqpoint{1.175300in}{2.063305in}}{\pgfqpoint{1.183200in}{2.066578in}}{\pgfqpoint{1.189024in}{2.072401in}}%
\pgfpathcurveto{\pgfqpoint{1.194848in}{2.078225in}}{\pgfqpoint{1.198120in}{2.086125in}}{\pgfqpoint{1.198120in}{2.094362in}}%
\pgfpathcurveto{\pgfqpoint{1.198120in}{2.102598in}}{\pgfqpoint{1.194848in}{2.110498in}}{\pgfqpoint{1.189024in}{2.116322in}}%
\pgfpathcurveto{\pgfqpoint{1.183200in}{2.122146in}}{\pgfqpoint{1.175300in}{2.125418in}}{\pgfqpoint{1.167063in}{2.125418in}}%
\pgfpathcurveto{\pgfqpoint{1.158827in}{2.125418in}}{\pgfqpoint{1.150927in}{2.122146in}}{\pgfqpoint{1.145103in}{2.116322in}}%
\pgfpathcurveto{\pgfqpoint{1.139279in}{2.110498in}}{\pgfqpoint{1.136007in}{2.102598in}}{\pgfqpoint{1.136007in}{2.094362in}}%
\pgfpathcurveto{\pgfqpoint{1.136007in}{2.086125in}}{\pgfqpoint{1.139279in}{2.078225in}}{\pgfqpoint{1.145103in}{2.072401in}}%
\pgfpathcurveto{\pgfqpoint{1.150927in}{2.066578in}}{\pgfqpoint{1.158827in}{2.063305in}}{\pgfqpoint{1.167063in}{2.063305in}}%
\pgfpathclose%
\pgfusepath{stroke,fill}%
\end{pgfscope}%
\begin{pgfscope}%
\pgfpathrectangle{\pgfqpoint{0.100000in}{0.212622in}}{\pgfqpoint{3.696000in}{3.696000in}}%
\pgfusepath{clip}%
\pgfsetbuttcap%
\pgfsetroundjoin%
\definecolor{currentfill}{rgb}{0.121569,0.466667,0.705882}%
\pgfsetfillcolor{currentfill}%
\pgfsetfillopacity{0.579145}%
\pgfsetlinewidth{1.003750pt}%
\definecolor{currentstroke}{rgb}{0.121569,0.466667,0.705882}%
\pgfsetstrokecolor{currentstroke}%
\pgfsetstrokeopacity{0.579145}%
\pgfsetdash{}{0pt}%
\pgfpathmoveto{\pgfqpoint{1.166965in}{2.063104in}}%
\pgfpathcurveto{\pgfqpoint{1.175201in}{2.063104in}}{\pgfqpoint{1.183101in}{2.066376in}}{\pgfqpoint{1.188925in}{2.072200in}}%
\pgfpathcurveto{\pgfqpoint{1.194749in}{2.078024in}}{\pgfqpoint{1.198021in}{2.085924in}}{\pgfqpoint{1.198021in}{2.094161in}}%
\pgfpathcurveto{\pgfqpoint{1.198021in}{2.102397in}}{\pgfqpoint{1.194749in}{2.110297in}}{\pgfqpoint{1.188925in}{2.116121in}}%
\pgfpathcurveto{\pgfqpoint{1.183101in}{2.121945in}}{\pgfqpoint{1.175201in}{2.125217in}}{\pgfqpoint{1.166965in}{2.125217in}}%
\pgfpathcurveto{\pgfqpoint{1.158728in}{2.125217in}}{\pgfqpoint{1.150828in}{2.121945in}}{\pgfqpoint{1.145004in}{2.116121in}}%
\pgfpathcurveto{\pgfqpoint{1.139180in}{2.110297in}}{\pgfqpoint{1.135908in}{2.102397in}}{\pgfqpoint{1.135908in}{2.094161in}}%
\pgfpathcurveto{\pgfqpoint{1.135908in}{2.085924in}}{\pgfqpoint{1.139180in}{2.078024in}}{\pgfqpoint{1.145004in}{2.072200in}}%
\pgfpathcurveto{\pgfqpoint{1.150828in}{2.066376in}}{\pgfqpoint{1.158728in}{2.063104in}}{\pgfqpoint{1.166965in}{2.063104in}}%
\pgfpathclose%
\pgfusepath{stroke,fill}%
\end{pgfscope}%
\begin{pgfscope}%
\pgfpathrectangle{\pgfqpoint{0.100000in}{0.212622in}}{\pgfqpoint{3.696000in}{3.696000in}}%
\pgfusepath{clip}%
\pgfsetbuttcap%
\pgfsetroundjoin%
\definecolor{currentfill}{rgb}{0.121569,0.466667,0.705882}%
\pgfsetfillcolor{currentfill}%
\pgfsetfillopacity{0.579222}%
\pgfsetlinewidth{1.003750pt}%
\definecolor{currentstroke}{rgb}{0.121569,0.466667,0.705882}%
\pgfsetstrokecolor{currentstroke}%
\pgfsetstrokeopacity{0.579222}%
\pgfsetdash{}{0pt}%
\pgfpathmoveto{\pgfqpoint{1.166779in}{2.062743in}}%
\pgfpathcurveto{\pgfqpoint{1.175016in}{2.062743in}}{\pgfqpoint{1.182916in}{2.066015in}}{\pgfqpoint{1.188740in}{2.071839in}}%
\pgfpathcurveto{\pgfqpoint{1.194564in}{2.077663in}}{\pgfqpoint{1.197836in}{2.085563in}}{\pgfqpoint{1.197836in}{2.093799in}}%
\pgfpathcurveto{\pgfqpoint{1.197836in}{2.102036in}}{\pgfqpoint{1.194564in}{2.109936in}}{\pgfqpoint{1.188740in}{2.115760in}}%
\pgfpathcurveto{\pgfqpoint{1.182916in}{2.121583in}}{\pgfqpoint{1.175016in}{2.124856in}}{\pgfqpoint{1.166779in}{2.124856in}}%
\pgfpathcurveto{\pgfqpoint{1.158543in}{2.124856in}}{\pgfqpoint{1.150643in}{2.121583in}}{\pgfqpoint{1.144819in}{2.115760in}}%
\pgfpathcurveto{\pgfqpoint{1.138995in}{2.109936in}}{\pgfqpoint{1.135723in}{2.102036in}}{\pgfqpoint{1.135723in}{2.093799in}}%
\pgfpathcurveto{\pgfqpoint{1.135723in}{2.085563in}}{\pgfqpoint{1.138995in}{2.077663in}}{\pgfqpoint{1.144819in}{2.071839in}}%
\pgfpathcurveto{\pgfqpoint{1.150643in}{2.066015in}}{\pgfqpoint{1.158543in}{2.062743in}}{\pgfqpoint{1.166779in}{2.062743in}}%
\pgfpathclose%
\pgfusepath{stroke,fill}%
\end{pgfscope}%
\begin{pgfscope}%
\pgfpathrectangle{\pgfqpoint{0.100000in}{0.212622in}}{\pgfqpoint{3.696000in}{3.696000in}}%
\pgfusepath{clip}%
\pgfsetbuttcap%
\pgfsetroundjoin%
\definecolor{currentfill}{rgb}{0.121569,0.466667,0.705882}%
\pgfsetfillcolor{currentfill}%
\pgfsetfillopacity{0.579356}%
\pgfsetlinewidth{1.003750pt}%
\definecolor{currentstroke}{rgb}{0.121569,0.466667,0.705882}%
\pgfsetstrokecolor{currentstroke}%
\pgfsetstrokeopacity{0.579356}%
\pgfsetdash{}{0pt}%
\pgfpathmoveto{\pgfqpoint{1.166426in}{2.062078in}}%
\pgfpathcurveto{\pgfqpoint{1.174662in}{2.062078in}}{\pgfqpoint{1.182562in}{2.065350in}}{\pgfqpoint{1.188386in}{2.071174in}}%
\pgfpathcurveto{\pgfqpoint{1.194210in}{2.076998in}}{\pgfqpoint{1.197482in}{2.084898in}}{\pgfqpoint{1.197482in}{2.093134in}}%
\pgfpathcurveto{\pgfqpoint{1.197482in}{2.101371in}}{\pgfqpoint{1.194210in}{2.109271in}}{\pgfqpoint{1.188386in}{2.115095in}}%
\pgfpathcurveto{\pgfqpoint{1.182562in}{2.120918in}}{\pgfqpoint{1.174662in}{2.124191in}}{\pgfqpoint{1.166426in}{2.124191in}}%
\pgfpathcurveto{\pgfqpoint{1.158190in}{2.124191in}}{\pgfqpoint{1.150290in}{2.120918in}}{\pgfqpoint{1.144466in}{2.115095in}}%
\pgfpathcurveto{\pgfqpoint{1.138642in}{2.109271in}}{\pgfqpoint{1.135369in}{2.101371in}}{\pgfqpoint{1.135369in}{2.093134in}}%
\pgfpathcurveto{\pgfqpoint{1.135369in}{2.084898in}}{\pgfqpoint{1.138642in}{2.076998in}}{\pgfqpoint{1.144466in}{2.071174in}}%
\pgfpathcurveto{\pgfqpoint{1.150290in}{2.065350in}}{\pgfqpoint{1.158190in}{2.062078in}}{\pgfqpoint{1.166426in}{2.062078in}}%
\pgfpathclose%
\pgfusepath{stroke,fill}%
\end{pgfscope}%
\begin{pgfscope}%
\pgfpathrectangle{\pgfqpoint{0.100000in}{0.212622in}}{\pgfqpoint{3.696000in}{3.696000in}}%
\pgfusepath{clip}%
\pgfsetbuttcap%
\pgfsetroundjoin%
\definecolor{currentfill}{rgb}{0.121569,0.466667,0.705882}%
\pgfsetfillcolor{currentfill}%
\pgfsetfillopacity{0.579605}%
\pgfsetlinewidth{1.003750pt}%
\definecolor{currentstroke}{rgb}{0.121569,0.466667,0.705882}%
\pgfsetstrokecolor{currentstroke}%
\pgfsetstrokeopacity{0.579605}%
\pgfsetdash{}{0pt}%
\pgfpathmoveto{\pgfqpoint{1.165794in}{2.060888in}}%
\pgfpathcurveto{\pgfqpoint{1.174030in}{2.060888in}}{\pgfqpoint{1.181930in}{2.064160in}}{\pgfqpoint{1.187754in}{2.069984in}}%
\pgfpathcurveto{\pgfqpoint{1.193578in}{2.075808in}}{\pgfqpoint{1.196850in}{2.083708in}}{\pgfqpoint{1.196850in}{2.091945in}}%
\pgfpathcurveto{\pgfqpoint{1.196850in}{2.100181in}}{\pgfqpoint{1.193578in}{2.108081in}}{\pgfqpoint{1.187754in}{2.113905in}}%
\pgfpathcurveto{\pgfqpoint{1.181930in}{2.119729in}}{\pgfqpoint{1.174030in}{2.123001in}}{\pgfqpoint{1.165794in}{2.123001in}}%
\pgfpathcurveto{\pgfqpoint{1.157558in}{2.123001in}}{\pgfqpoint{1.149657in}{2.119729in}}{\pgfqpoint{1.143834in}{2.113905in}}%
\pgfpathcurveto{\pgfqpoint{1.138010in}{2.108081in}}{\pgfqpoint{1.134737in}{2.100181in}}{\pgfqpoint{1.134737in}{2.091945in}}%
\pgfpathcurveto{\pgfqpoint{1.134737in}{2.083708in}}{\pgfqpoint{1.138010in}{2.075808in}}{\pgfqpoint{1.143834in}{2.069984in}}%
\pgfpathcurveto{\pgfqpoint{1.149657in}{2.064160in}}{\pgfqpoint{1.157558in}{2.060888in}}{\pgfqpoint{1.165794in}{2.060888in}}%
\pgfpathclose%
\pgfusepath{stroke,fill}%
\end{pgfscope}%
\begin{pgfscope}%
\pgfpathrectangle{\pgfqpoint{0.100000in}{0.212622in}}{\pgfqpoint{3.696000in}{3.696000in}}%
\pgfusepath{clip}%
\pgfsetbuttcap%
\pgfsetroundjoin%
\definecolor{currentfill}{rgb}{0.121569,0.466667,0.705882}%
\pgfsetfillcolor{currentfill}%
\pgfsetfillopacity{0.580037}%
\pgfsetlinewidth{1.003750pt}%
\definecolor{currentstroke}{rgb}{0.121569,0.466667,0.705882}%
\pgfsetstrokecolor{currentstroke}%
\pgfsetstrokeopacity{0.580037}%
\pgfsetdash{}{0pt}%
\pgfpathmoveto{\pgfqpoint{1.164600in}{2.058692in}}%
\pgfpathcurveto{\pgfqpoint{1.172836in}{2.058692in}}{\pgfqpoint{1.180736in}{2.061965in}}{\pgfqpoint{1.186560in}{2.067789in}}%
\pgfpathcurveto{\pgfqpoint{1.192384in}{2.073612in}}{\pgfqpoint{1.195656in}{2.081513in}}{\pgfqpoint{1.195656in}{2.089749in}}%
\pgfpathcurveto{\pgfqpoint{1.195656in}{2.097985in}}{\pgfqpoint{1.192384in}{2.105885in}}{\pgfqpoint{1.186560in}{2.111709in}}%
\pgfpathcurveto{\pgfqpoint{1.180736in}{2.117533in}}{\pgfqpoint{1.172836in}{2.120805in}}{\pgfqpoint{1.164600in}{2.120805in}}%
\pgfpathcurveto{\pgfqpoint{1.156364in}{2.120805in}}{\pgfqpoint{1.148464in}{2.117533in}}{\pgfqpoint{1.142640in}{2.111709in}}%
\pgfpathcurveto{\pgfqpoint{1.136816in}{2.105885in}}{\pgfqpoint{1.133543in}{2.097985in}}{\pgfqpoint{1.133543in}{2.089749in}}%
\pgfpathcurveto{\pgfqpoint{1.133543in}{2.081513in}}{\pgfqpoint{1.136816in}{2.073612in}}{\pgfqpoint{1.142640in}{2.067789in}}%
\pgfpathcurveto{\pgfqpoint{1.148464in}{2.061965in}}{\pgfqpoint{1.156364in}{2.058692in}}{\pgfqpoint{1.164600in}{2.058692in}}%
\pgfpathclose%
\pgfusepath{stroke,fill}%
\end{pgfscope}%
\begin{pgfscope}%
\pgfpathrectangle{\pgfqpoint{0.100000in}{0.212622in}}{\pgfqpoint{3.696000in}{3.696000in}}%
\pgfusepath{clip}%
\pgfsetbuttcap%
\pgfsetroundjoin%
\definecolor{currentfill}{rgb}{0.121569,0.466667,0.705882}%
\pgfsetfillcolor{currentfill}%
\pgfsetfillopacity{0.580167}%
\pgfsetlinewidth{1.003750pt}%
\definecolor{currentstroke}{rgb}{0.121569,0.466667,0.705882}%
\pgfsetstrokecolor{currentstroke}%
\pgfsetstrokeopacity{0.580167}%
\pgfsetdash{}{0pt}%
\pgfpathmoveto{\pgfqpoint{1.164263in}{2.058050in}}%
\pgfpathcurveto{\pgfqpoint{1.172499in}{2.058050in}}{\pgfqpoint{1.180399in}{2.061322in}}{\pgfqpoint{1.186223in}{2.067146in}}%
\pgfpathcurveto{\pgfqpoint{1.192047in}{2.072970in}}{\pgfqpoint{1.195320in}{2.080870in}}{\pgfqpoint{1.195320in}{2.089106in}}%
\pgfpathcurveto{\pgfqpoint{1.195320in}{2.097343in}}{\pgfqpoint{1.192047in}{2.105243in}}{\pgfqpoint{1.186223in}{2.111067in}}%
\pgfpathcurveto{\pgfqpoint{1.180399in}{2.116890in}}{\pgfqpoint{1.172499in}{2.120163in}}{\pgfqpoint{1.164263in}{2.120163in}}%
\pgfpathcurveto{\pgfqpoint{1.156027in}{2.120163in}}{\pgfqpoint{1.148127in}{2.116890in}}{\pgfqpoint{1.142303in}{2.111067in}}%
\pgfpathcurveto{\pgfqpoint{1.136479in}{2.105243in}}{\pgfqpoint{1.133207in}{2.097343in}}{\pgfqpoint{1.133207in}{2.089106in}}%
\pgfpathcurveto{\pgfqpoint{1.133207in}{2.080870in}}{\pgfqpoint{1.136479in}{2.072970in}}{\pgfqpoint{1.142303in}{2.067146in}}%
\pgfpathcurveto{\pgfqpoint{1.148127in}{2.061322in}}{\pgfqpoint{1.156027in}{2.058050in}}{\pgfqpoint{1.164263in}{2.058050in}}%
\pgfpathclose%
\pgfusepath{stroke,fill}%
\end{pgfscope}%
\begin{pgfscope}%
\pgfpathrectangle{\pgfqpoint{0.100000in}{0.212622in}}{\pgfqpoint{3.696000in}{3.696000in}}%
\pgfusepath{clip}%
\pgfsetbuttcap%
\pgfsetroundjoin%
\definecolor{currentfill}{rgb}{0.121569,0.466667,0.705882}%
\pgfsetfillcolor{currentfill}%
\pgfsetfillopacity{0.580405}%
\pgfsetlinewidth{1.003750pt}%
\definecolor{currentstroke}{rgb}{0.121569,0.466667,0.705882}%
\pgfsetstrokecolor{currentstroke}%
\pgfsetstrokeopacity{0.580405}%
\pgfsetdash{}{0pt}%
\pgfpathmoveto{\pgfqpoint{1.163655in}{2.056883in}}%
\pgfpathcurveto{\pgfqpoint{1.171892in}{2.056883in}}{\pgfqpoint{1.179792in}{2.060155in}}{\pgfqpoint{1.185616in}{2.065979in}}%
\pgfpathcurveto{\pgfqpoint{1.191440in}{2.071803in}}{\pgfqpoint{1.194712in}{2.079703in}}{\pgfqpoint{1.194712in}{2.087939in}}%
\pgfpathcurveto{\pgfqpoint{1.194712in}{2.096176in}}{\pgfqpoint{1.191440in}{2.104076in}}{\pgfqpoint{1.185616in}{2.109900in}}%
\pgfpathcurveto{\pgfqpoint{1.179792in}{2.115723in}}{\pgfqpoint{1.171892in}{2.118996in}}{\pgfqpoint{1.163655in}{2.118996in}}%
\pgfpathcurveto{\pgfqpoint{1.155419in}{2.118996in}}{\pgfqpoint{1.147519in}{2.115723in}}{\pgfqpoint{1.141695in}{2.109900in}}%
\pgfpathcurveto{\pgfqpoint{1.135871in}{2.104076in}}{\pgfqpoint{1.132599in}{2.096176in}}{\pgfqpoint{1.132599in}{2.087939in}}%
\pgfpathcurveto{\pgfqpoint{1.132599in}{2.079703in}}{\pgfqpoint{1.135871in}{2.071803in}}{\pgfqpoint{1.141695in}{2.065979in}}%
\pgfpathcurveto{\pgfqpoint{1.147519in}{2.060155in}}{\pgfqpoint{1.155419in}{2.056883in}}{\pgfqpoint{1.163655in}{2.056883in}}%
\pgfpathclose%
\pgfusepath{stroke,fill}%
\end{pgfscope}%
\begin{pgfscope}%
\pgfpathrectangle{\pgfqpoint{0.100000in}{0.212622in}}{\pgfqpoint{3.696000in}{3.696000in}}%
\pgfusepath{clip}%
\pgfsetbuttcap%
\pgfsetroundjoin%
\definecolor{currentfill}{rgb}{0.121569,0.466667,0.705882}%
\pgfsetfillcolor{currentfill}%
\pgfsetfillopacity{0.580829}%
\pgfsetlinewidth{1.003750pt}%
\definecolor{currentstroke}{rgb}{0.121569,0.466667,0.705882}%
\pgfsetstrokecolor{currentstroke}%
\pgfsetstrokeopacity{0.580829}%
\pgfsetdash{}{0pt}%
\pgfpathmoveto{\pgfqpoint{1.162529in}{2.054750in}}%
\pgfpathcurveto{\pgfqpoint{1.170765in}{2.054750in}}{\pgfqpoint{1.178665in}{2.058022in}}{\pgfqpoint{1.184489in}{2.063846in}}%
\pgfpathcurveto{\pgfqpoint{1.190313in}{2.069670in}}{\pgfqpoint{1.193585in}{2.077570in}}{\pgfqpoint{1.193585in}{2.085807in}}%
\pgfpathcurveto{\pgfqpoint{1.193585in}{2.094043in}}{\pgfqpoint{1.190313in}{2.101943in}}{\pgfqpoint{1.184489in}{2.107767in}}%
\pgfpathcurveto{\pgfqpoint{1.178665in}{2.113591in}}{\pgfqpoint{1.170765in}{2.116863in}}{\pgfqpoint{1.162529in}{2.116863in}}%
\pgfpathcurveto{\pgfqpoint{1.154292in}{2.116863in}}{\pgfqpoint{1.146392in}{2.113591in}}{\pgfqpoint{1.140568in}{2.107767in}}%
\pgfpathcurveto{\pgfqpoint{1.134745in}{2.101943in}}{\pgfqpoint{1.131472in}{2.094043in}}{\pgfqpoint{1.131472in}{2.085807in}}%
\pgfpathcurveto{\pgfqpoint{1.131472in}{2.077570in}}{\pgfqpoint{1.134745in}{2.069670in}}{\pgfqpoint{1.140568in}{2.063846in}}%
\pgfpathcurveto{\pgfqpoint{1.146392in}{2.058022in}}{\pgfqpoint{1.154292in}{2.054750in}}{\pgfqpoint{1.162529in}{2.054750in}}%
\pgfpathclose%
\pgfusepath{stroke,fill}%
\end{pgfscope}%
\begin{pgfscope}%
\pgfpathrectangle{\pgfqpoint{0.100000in}{0.212622in}}{\pgfqpoint{3.696000in}{3.696000in}}%
\pgfusepath{clip}%
\pgfsetbuttcap%
\pgfsetroundjoin%
\definecolor{currentfill}{rgb}{0.121569,0.466667,0.705882}%
\pgfsetfillcolor{currentfill}%
\pgfsetfillopacity{0.581616}%
\pgfsetlinewidth{1.003750pt}%
\definecolor{currentstroke}{rgb}{0.121569,0.466667,0.705882}%
\pgfsetstrokecolor{currentstroke}%
\pgfsetstrokeopacity{0.581616}%
\pgfsetdash{}{0pt}%
\pgfpathmoveto{\pgfqpoint{1.160545in}{2.050860in}}%
\pgfpathcurveto{\pgfqpoint{1.168781in}{2.050860in}}{\pgfqpoint{1.176681in}{2.054132in}}{\pgfqpoint{1.182505in}{2.059956in}}%
\pgfpathcurveto{\pgfqpoint{1.188329in}{2.065780in}}{\pgfqpoint{1.191601in}{2.073680in}}{\pgfqpoint{1.191601in}{2.081917in}}%
\pgfpathcurveto{\pgfqpoint{1.191601in}{2.090153in}}{\pgfqpoint{1.188329in}{2.098053in}}{\pgfqpoint{1.182505in}{2.103877in}}%
\pgfpathcurveto{\pgfqpoint{1.176681in}{2.109701in}}{\pgfqpoint{1.168781in}{2.112973in}}{\pgfqpoint{1.160545in}{2.112973in}}%
\pgfpathcurveto{\pgfqpoint{1.152308in}{2.112973in}}{\pgfqpoint{1.144408in}{2.109701in}}{\pgfqpoint{1.138584in}{2.103877in}}%
\pgfpathcurveto{\pgfqpoint{1.132761in}{2.098053in}}{\pgfqpoint{1.129488in}{2.090153in}}{\pgfqpoint{1.129488in}{2.081917in}}%
\pgfpathcurveto{\pgfqpoint{1.129488in}{2.073680in}}{\pgfqpoint{1.132761in}{2.065780in}}{\pgfqpoint{1.138584in}{2.059956in}}%
\pgfpathcurveto{\pgfqpoint{1.144408in}{2.054132in}}{\pgfqpoint{1.152308in}{2.050860in}}{\pgfqpoint{1.160545in}{2.050860in}}%
\pgfpathclose%
\pgfusepath{stroke,fill}%
\end{pgfscope}%
\begin{pgfscope}%
\pgfpathrectangle{\pgfqpoint{0.100000in}{0.212622in}}{\pgfqpoint{3.696000in}{3.696000in}}%
\pgfusepath{clip}%
\pgfsetbuttcap%
\pgfsetroundjoin%
\definecolor{currentfill}{rgb}{0.121569,0.466667,0.705882}%
\pgfsetfillcolor{currentfill}%
\pgfsetfillopacity{0.582464}%
\pgfsetlinewidth{1.003750pt}%
\definecolor{currentstroke}{rgb}{0.121569,0.466667,0.705882}%
\pgfsetstrokecolor{currentstroke}%
\pgfsetstrokeopacity{0.582464}%
\pgfsetdash{}{0pt}%
\pgfpathmoveto{\pgfqpoint{2.147667in}{2.391844in}}%
\pgfpathcurveto{\pgfqpoint{2.155904in}{2.391844in}}{\pgfqpoint{2.163804in}{2.395116in}}{\pgfqpoint{2.169628in}{2.400940in}}%
\pgfpathcurveto{\pgfqpoint{2.175452in}{2.406764in}}{\pgfqpoint{2.178724in}{2.414664in}}{\pgfqpoint{2.178724in}{2.422900in}}%
\pgfpathcurveto{\pgfqpoint{2.178724in}{2.431137in}}{\pgfqpoint{2.175452in}{2.439037in}}{\pgfqpoint{2.169628in}{2.444860in}}%
\pgfpathcurveto{\pgfqpoint{2.163804in}{2.450684in}}{\pgfqpoint{2.155904in}{2.453957in}}{\pgfqpoint{2.147667in}{2.453957in}}%
\pgfpathcurveto{\pgfqpoint{2.139431in}{2.453957in}}{\pgfqpoint{2.131531in}{2.450684in}}{\pgfqpoint{2.125707in}{2.444860in}}%
\pgfpathcurveto{\pgfqpoint{2.119883in}{2.439037in}}{\pgfqpoint{2.116611in}{2.431137in}}{\pgfqpoint{2.116611in}{2.422900in}}%
\pgfpathcurveto{\pgfqpoint{2.116611in}{2.414664in}}{\pgfqpoint{2.119883in}{2.406764in}}{\pgfqpoint{2.125707in}{2.400940in}}%
\pgfpathcurveto{\pgfqpoint{2.131531in}{2.395116in}}{\pgfqpoint{2.139431in}{2.391844in}}{\pgfqpoint{2.147667in}{2.391844in}}%
\pgfpathclose%
\pgfusepath{stroke,fill}%
\end{pgfscope}%
\begin{pgfscope}%
\pgfpathrectangle{\pgfqpoint{0.100000in}{0.212622in}}{\pgfqpoint{3.696000in}{3.696000in}}%
\pgfusepath{clip}%
\pgfsetbuttcap%
\pgfsetroundjoin%
\definecolor{currentfill}{rgb}{0.121569,0.466667,0.705882}%
\pgfsetfillcolor{currentfill}%
\pgfsetfillopacity{0.583050}%
\pgfsetlinewidth{1.003750pt}%
\definecolor{currentstroke}{rgb}{0.121569,0.466667,0.705882}%
\pgfsetstrokecolor{currentstroke}%
\pgfsetstrokeopacity{0.583050}%
\pgfsetdash{}{0pt}%
\pgfpathmoveto{\pgfqpoint{1.156919in}{2.043813in}}%
\pgfpathcurveto{\pgfqpoint{1.165155in}{2.043813in}}{\pgfqpoint{1.173055in}{2.047086in}}{\pgfqpoint{1.178879in}{2.052910in}}%
\pgfpathcurveto{\pgfqpoint{1.184703in}{2.058734in}}{\pgfqpoint{1.187976in}{2.066634in}}{\pgfqpoint{1.187976in}{2.074870in}}%
\pgfpathcurveto{\pgfqpoint{1.187976in}{2.083106in}}{\pgfqpoint{1.184703in}{2.091006in}}{\pgfqpoint{1.178879in}{2.096830in}}%
\pgfpathcurveto{\pgfqpoint{1.173055in}{2.102654in}}{\pgfqpoint{1.165155in}{2.105926in}}{\pgfqpoint{1.156919in}{2.105926in}}%
\pgfpathcurveto{\pgfqpoint{1.148683in}{2.105926in}}{\pgfqpoint{1.140783in}{2.102654in}}{\pgfqpoint{1.134959in}{2.096830in}}%
\pgfpathcurveto{\pgfqpoint{1.129135in}{2.091006in}}{\pgfqpoint{1.125863in}{2.083106in}}{\pgfqpoint{1.125863in}{2.074870in}}%
\pgfpathcurveto{\pgfqpoint{1.125863in}{2.066634in}}{\pgfqpoint{1.129135in}{2.058734in}}{\pgfqpoint{1.134959in}{2.052910in}}%
\pgfpathcurveto{\pgfqpoint{1.140783in}{2.047086in}}{\pgfqpoint{1.148683in}{2.043813in}}{\pgfqpoint{1.156919in}{2.043813in}}%
\pgfpathclose%
\pgfusepath{stroke,fill}%
\end{pgfscope}%
\begin{pgfscope}%
\pgfpathrectangle{\pgfqpoint{0.100000in}{0.212622in}}{\pgfqpoint{3.696000in}{3.696000in}}%
\pgfusepath{clip}%
\pgfsetbuttcap%
\pgfsetroundjoin%
\definecolor{currentfill}{rgb}{0.121569,0.466667,0.705882}%
\pgfsetfillcolor{currentfill}%
\pgfsetfillopacity{0.585600}%
\pgfsetlinewidth{1.003750pt}%
\definecolor{currentstroke}{rgb}{0.121569,0.466667,0.705882}%
\pgfsetstrokecolor{currentstroke}%
\pgfsetstrokeopacity{0.585600}%
\pgfsetdash{}{0pt}%
\pgfpathmoveto{\pgfqpoint{2.151607in}{2.381061in}}%
\pgfpathcurveto{\pgfqpoint{2.159843in}{2.381061in}}{\pgfqpoint{2.167743in}{2.384334in}}{\pgfqpoint{2.173567in}{2.390158in}}%
\pgfpathcurveto{\pgfqpoint{2.179391in}{2.395982in}}{\pgfqpoint{2.182664in}{2.403882in}}{\pgfqpoint{2.182664in}{2.412118in}}%
\pgfpathcurveto{\pgfqpoint{2.182664in}{2.420354in}}{\pgfqpoint{2.179391in}{2.428254in}}{\pgfqpoint{2.173567in}{2.434078in}}%
\pgfpathcurveto{\pgfqpoint{2.167743in}{2.439902in}}{\pgfqpoint{2.159843in}{2.443174in}}{\pgfqpoint{2.151607in}{2.443174in}}%
\pgfpathcurveto{\pgfqpoint{2.143371in}{2.443174in}}{\pgfqpoint{2.135471in}{2.439902in}}{\pgfqpoint{2.129647in}{2.434078in}}%
\pgfpathcurveto{\pgfqpoint{2.123823in}{2.428254in}}{\pgfqpoint{2.120551in}{2.420354in}}{\pgfqpoint{2.120551in}{2.412118in}}%
\pgfpathcurveto{\pgfqpoint{2.120551in}{2.403882in}}{\pgfqpoint{2.123823in}{2.395982in}}{\pgfqpoint{2.129647in}{2.390158in}}%
\pgfpathcurveto{\pgfqpoint{2.135471in}{2.384334in}}{\pgfqpoint{2.143371in}{2.381061in}}{\pgfqpoint{2.151607in}{2.381061in}}%
\pgfpathclose%
\pgfusepath{stroke,fill}%
\end{pgfscope}%
\begin{pgfscope}%
\pgfpathrectangle{\pgfqpoint{0.100000in}{0.212622in}}{\pgfqpoint{3.696000in}{3.696000in}}%
\pgfusepath{clip}%
\pgfsetbuttcap%
\pgfsetroundjoin%
\definecolor{currentfill}{rgb}{0.121569,0.466667,0.705882}%
\pgfsetfillcolor{currentfill}%
\pgfsetfillopacity{0.585686}%
\pgfsetlinewidth{1.003750pt}%
\definecolor{currentstroke}{rgb}{0.121569,0.466667,0.705882}%
\pgfsetstrokecolor{currentstroke}%
\pgfsetstrokeopacity{0.585686}%
\pgfsetdash{}{0pt}%
\pgfpathmoveto{\pgfqpoint{1.150401in}{2.031016in}}%
\pgfpathcurveto{\pgfqpoint{1.158637in}{2.031016in}}{\pgfqpoint{1.166537in}{2.034288in}}{\pgfqpoint{1.172361in}{2.040112in}}%
\pgfpathcurveto{\pgfqpoint{1.178185in}{2.045936in}}{\pgfqpoint{1.181457in}{2.053836in}}{\pgfqpoint{1.181457in}{2.062072in}}%
\pgfpathcurveto{\pgfqpoint{1.181457in}{2.070309in}}{\pgfqpoint{1.178185in}{2.078209in}}{\pgfqpoint{1.172361in}{2.084033in}}%
\pgfpathcurveto{\pgfqpoint{1.166537in}{2.089856in}}{\pgfqpoint{1.158637in}{2.093129in}}{\pgfqpoint{1.150401in}{2.093129in}}%
\pgfpathcurveto{\pgfqpoint{1.142165in}{2.093129in}}{\pgfqpoint{1.134265in}{2.089856in}}{\pgfqpoint{1.128441in}{2.084033in}}%
\pgfpathcurveto{\pgfqpoint{1.122617in}{2.078209in}}{\pgfqpoint{1.119344in}{2.070309in}}{\pgfqpoint{1.119344in}{2.062072in}}%
\pgfpathcurveto{\pgfqpoint{1.119344in}{2.053836in}}{\pgfqpoint{1.122617in}{2.045936in}}{\pgfqpoint{1.128441in}{2.040112in}}%
\pgfpathcurveto{\pgfqpoint{1.134265in}{2.034288in}}{\pgfqpoint{1.142165in}{2.031016in}}{\pgfqpoint{1.150401in}{2.031016in}}%
\pgfpathclose%
\pgfusepath{stroke,fill}%
\end{pgfscope}%
\begin{pgfscope}%
\pgfpathrectangle{\pgfqpoint{0.100000in}{0.212622in}}{\pgfqpoint{3.696000in}{3.696000in}}%
\pgfusepath{clip}%
\pgfsetbuttcap%
\pgfsetroundjoin%
\definecolor{currentfill}{rgb}{0.121569,0.466667,0.705882}%
\pgfsetfillcolor{currentfill}%
\pgfsetfillopacity{0.588065}%
\pgfsetlinewidth{1.003750pt}%
\definecolor{currentstroke}{rgb}{0.121569,0.466667,0.705882}%
\pgfsetstrokecolor{currentstroke}%
\pgfsetstrokeopacity{0.588065}%
\pgfsetdash{}{0pt}%
\pgfpathmoveto{\pgfqpoint{1.144915in}{2.019646in}}%
\pgfpathcurveto{\pgfqpoint{1.153151in}{2.019646in}}{\pgfqpoint{1.161051in}{2.022918in}}{\pgfqpoint{1.166875in}{2.028742in}}%
\pgfpathcurveto{\pgfqpoint{1.172699in}{2.034566in}}{\pgfqpoint{1.175971in}{2.042466in}}{\pgfqpoint{1.175971in}{2.050703in}}%
\pgfpathcurveto{\pgfqpoint{1.175971in}{2.058939in}}{\pgfqpoint{1.172699in}{2.066839in}}{\pgfqpoint{1.166875in}{2.072663in}}%
\pgfpathcurveto{\pgfqpoint{1.161051in}{2.078487in}}{\pgfqpoint{1.153151in}{2.081759in}}{\pgfqpoint{1.144915in}{2.081759in}}%
\pgfpathcurveto{\pgfqpoint{1.136679in}{2.081759in}}{\pgfqpoint{1.128778in}{2.078487in}}{\pgfqpoint{1.122955in}{2.072663in}}%
\pgfpathcurveto{\pgfqpoint{1.117131in}{2.066839in}}{\pgfqpoint{1.113858in}{2.058939in}}{\pgfqpoint{1.113858in}{2.050703in}}%
\pgfpathcurveto{\pgfqpoint{1.113858in}{2.042466in}}{\pgfqpoint{1.117131in}{2.034566in}}{\pgfqpoint{1.122955in}{2.028742in}}%
\pgfpathcurveto{\pgfqpoint{1.128778in}{2.022918in}}{\pgfqpoint{1.136679in}{2.019646in}}{\pgfqpoint{1.144915in}{2.019646in}}%
\pgfpathclose%
\pgfusepath{stroke,fill}%
\end{pgfscope}%
\begin{pgfscope}%
\pgfpathrectangle{\pgfqpoint{0.100000in}{0.212622in}}{\pgfqpoint{3.696000in}{3.696000in}}%
\pgfusepath{clip}%
\pgfsetbuttcap%
\pgfsetroundjoin%
\definecolor{currentfill}{rgb}{0.121569,0.466667,0.705882}%
\pgfsetfillcolor{currentfill}%
\pgfsetfillopacity{0.589908}%
\pgfsetlinewidth{1.003750pt}%
\definecolor{currentstroke}{rgb}{0.121569,0.466667,0.705882}%
\pgfsetstrokecolor{currentstroke}%
\pgfsetstrokeopacity{0.589908}%
\pgfsetdash{}{0pt}%
\pgfpathmoveto{\pgfqpoint{2.156086in}{2.365123in}}%
\pgfpathcurveto{\pgfqpoint{2.164322in}{2.365123in}}{\pgfqpoint{2.172222in}{2.368395in}}{\pgfqpoint{2.178046in}{2.374219in}}%
\pgfpathcurveto{\pgfqpoint{2.183870in}{2.380043in}}{\pgfqpoint{2.187142in}{2.387943in}}{\pgfqpoint{2.187142in}{2.396180in}}%
\pgfpathcurveto{\pgfqpoint{2.187142in}{2.404416in}}{\pgfqpoint{2.183870in}{2.412316in}}{\pgfqpoint{2.178046in}{2.418140in}}%
\pgfpathcurveto{\pgfqpoint{2.172222in}{2.423964in}}{\pgfqpoint{2.164322in}{2.427236in}}{\pgfqpoint{2.156086in}{2.427236in}}%
\pgfpathcurveto{\pgfqpoint{2.147849in}{2.427236in}}{\pgfqpoint{2.139949in}{2.423964in}}{\pgfqpoint{2.134125in}{2.418140in}}%
\pgfpathcurveto{\pgfqpoint{2.128301in}{2.412316in}}{\pgfqpoint{2.125029in}{2.404416in}}{\pgfqpoint{2.125029in}{2.396180in}}%
\pgfpathcurveto{\pgfqpoint{2.125029in}{2.387943in}}{\pgfqpoint{2.128301in}{2.380043in}}{\pgfqpoint{2.134125in}{2.374219in}}%
\pgfpathcurveto{\pgfqpoint{2.139949in}{2.368395in}}{\pgfqpoint{2.147849in}{2.365123in}}{\pgfqpoint{2.156086in}{2.365123in}}%
\pgfpathclose%
\pgfusepath{stroke,fill}%
\end{pgfscope}%
\begin{pgfscope}%
\pgfpathrectangle{\pgfqpoint{0.100000in}{0.212622in}}{\pgfqpoint{3.696000in}{3.696000in}}%
\pgfusepath{clip}%
\pgfsetbuttcap%
\pgfsetroundjoin%
\definecolor{currentfill}{rgb}{0.121569,0.466667,0.705882}%
\pgfsetfillcolor{currentfill}%
\pgfsetfillopacity{0.590093}%
\pgfsetlinewidth{1.003750pt}%
\definecolor{currentstroke}{rgb}{0.121569,0.466667,0.705882}%
\pgfsetstrokecolor{currentstroke}%
\pgfsetstrokeopacity{0.590093}%
\pgfsetdash{}{0pt}%
\pgfpathmoveto{\pgfqpoint{1.140493in}{2.010098in}}%
\pgfpathcurveto{\pgfqpoint{1.148729in}{2.010098in}}{\pgfqpoint{1.156629in}{2.013370in}}{\pgfqpoint{1.162453in}{2.019194in}}%
\pgfpathcurveto{\pgfqpoint{1.168277in}{2.025018in}}{\pgfqpoint{1.171549in}{2.032918in}}{\pgfqpoint{1.171549in}{2.041154in}}%
\pgfpathcurveto{\pgfqpoint{1.171549in}{2.049391in}}{\pgfqpoint{1.168277in}{2.057291in}}{\pgfqpoint{1.162453in}{2.063115in}}%
\pgfpathcurveto{\pgfqpoint{1.156629in}{2.068939in}}{\pgfqpoint{1.148729in}{2.072211in}}{\pgfqpoint{1.140493in}{2.072211in}}%
\pgfpathcurveto{\pgfqpoint{1.132257in}{2.072211in}}{\pgfqpoint{1.124357in}{2.068939in}}{\pgfqpoint{1.118533in}{2.063115in}}%
\pgfpathcurveto{\pgfqpoint{1.112709in}{2.057291in}}{\pgfqpoint{1.109436in}{2.049391in}}{\pgfqpoint{1.109436in}{2.041154in}}%
\pgfpathcurveto{\pgfqpoint{1.109436in}{2.032918in}}{\pgfqpoint{1.112709in}{2.025018in}}{\pgfqpoint{1.118533in}{2.019194in}}%
\pgfpathcurveto{\pgfqpoint{1.124357in}{2.013370in}}{\pgfqpoint{1.132257in}{2.010098in}}{\pgfqpoint{1.140493in}{2.010098in}}%
\pgfpathclose%
\pgfusepath{stroke,fill}%
\end{pgfscope}%
\begin{pgfscope}%
\pgfpathrectangle{\pgfqpoint{0.100000in}{0.212622in}}{\pgfqpoint{3.696000in}{3.696000in}}%
\pgfusepath{clip}%
\pgfsetbuttcap%
\pgfsetroundjoin%
\definecolor{currentfill}{rgb}{0.121569,0.466667,0.705882}%
\pgfsetfillcolor{currentfill}%
\pgfsetfillopacity{0.592284}%
\pgfsetlinewidth{1.003750pt}%
\definecolor{currentstroke}{rgb}{0.121569,0.466667,0.705882}%
\pgfsetstrokecolor{currentstroke}%
\pgfsetstrokeopacity{0.592284}%
\pgfsetdash{}{0pt}%
\pgfpathmoveto{\pgfqpoint{2.158775in}{2.356358in}}%
\pgfpathcurveto{\pgfqpoint{2.167011in}{2.356358in}}{\pgfqpoint{2.174911in}{2.359630in}}{\pgfqpoint{2.180735in}{2.365454in}}%
\pgfpathcurveto{\pgfqpoint{2.186559in}{2.371278in}}{\pgfqpoint{2.189831in}{2.379178in}}{\pgfqpoint{2.189831in}{2.387414in}}%
\pgfpathcurveto{\pgfqpoint{2.189831in}{2.395651in}}{\pgfqpoint{2.186559in}{2.403551in}}{\pgfqpoint{2.180735in}{2.409375in}}%
\pgfpathcurveto{\pgfqpoint{2.174911in}{2.415199in}}{\pgfqpoint{2.167011in}{2.418471in}}{\pgfqpoint{2.158775in}{2.418471in}}%
\pgfpathcurveto{\pgfqpoint{2.150538in}{2.418471in}}{\pgfqpoint{2.142638in}{2.415199in}}{\pgfqpoint{2.136814in}{2.409375in}}%
\pgfpathcurveto{\pgfqpoint{2.130990in}{2.403551in}}{\pgfqpoint{2.127718in}{2.395651in}}{\pgfqpoint{2.127718in}{2.387414in}}%
\pgfpathcurveto{\pgfqpoint{2.127718in}{2.379178in}}{\pgfqpoint{2.130990in}{2.371278in}}{\pgfqpoint{2.136814in}{2.365454in}}%
\pgfpathcurveto{\pgfqpoint{2.142638in}{2.359630in}}{\pgfqpoint{2.150538in}{2.356358in}}{\pgfqpoint{2.158775in}{2.356358in}}%
\pgfpathclose%
\pgfusepath{stroke,fill}%
\end{pgfscope}%
\begin{pgfscope}%
\pgfpathrectangle{\pgfqpoint{0.100000in}{0.212622in}}{\pgfqpoint{3.696000in}{3.696000in}}%
\pgfusepath{clip}%
\pgfsetbuttcap%
\pgfsetroundjoin%
\definecolor{currentfill}{rgb}{0.121569,0.466667,0.705882}%
\pgfsetfillcolor{currentfill}%
\pgfsetfillopacity{0.594073}%
\pgfsetlinewidth{1.003750pt}%
\definecolor{currentstroke}{rgb}{0.121569,0.466667,0.705882}%
\pgfsetstrokecolor{currentstroke}%
\pgfsetstrokeopacity{0.594073}%
\pgfsetdash{}{0pt}%
\pgfpathmoveto{\pgfqpoint{1.133529in}{1.992711in}}%
\pgfpathcurveto{\pgfqpoint{1.141765in}{1.992711in}}{\pgfqpoint{1.149665in}{1.995984in}}{\pgfqpoint{1.155489in}{2.001808in}}%
\pgfpathcurveto{\pgfqpoint{1.161313in}{2.007631in}}{\pgfqpoint{1.164586in}{2.015532in}}{\pgfqpoint{1.164586in}{2.023768in}}%
\pgfpathcurveto{\pgfqpoint{1.164586in}{2.032004in}}{\pgfqpoint{1.161313in}{2.039904in}}{\pgfqpoint{1.155489in}{2.045728in}}%
\pgfpathcurveto{\pgfqpoint{1.149665in}{2.051552in}}{\pgfqpoint{1.141765in}{2.054824in}}{\pgfqpoint{1.133529in}{2.054824in}}%
\pgfpathcurveto{\pgfqpoint{1.125293in}{2.054824in}}{\pgfqpoint{1.117393in}{2.051552in}}{\pgfqpoint{1.111569in}{2.045728in}}%
\pgfpathcurveto{\pgfqpoint{1.105745in}{2.039904in}}{\pgfqpoint{1.102473in}{2.032004in}}{\pgfqpoint{1.102473in}{2.023768in}}%
\pgfpathcurveto{\pgfqpoint{1.102473in}{2.015532in}}{\pgfqpoint{1.105745in}{2.007631in}}{\pgfqpoint{1.111569in}{2.001808in}}%
\pgfpathcurveto{\pgfqpoint{1.117393in}{1.995984in}}{\pgfqpoint{1.125293in}{1.992711in}}{\pgfqpoint{1.133529in}{1.992711in}}%
\pgfpathclose%
\pgfusepath{stroke,fill}%
\end{pgfscope}%
\begin{pgfscope}%
\pgfpathrectangle{\pgfqpoint{0.100000in}{0.212622in}}{\pgfqpoint{3.696000in}{3.696000in}}%
\pgfusepath{clip}%
\pgfsetbuttcap%
\pgfsetroundjoin%
\definecolor{currentfill}{rgb}{0.121569,0.466667,0.705882}%
\pgfsetfillcolor{currentfill}%
\pgfsetfillopacity{0.596592}%
\pgfsetlinewidth{1.003750pt}%
\definecolor{currentstroke}{rgb}{0.121569,0.466667,0.705882}%
\pgfsetstrokecolor{currentstroke}%
\pgfsetstrokeopacity{0.596592}%
\pgfsetdash{}{0pt}%
\pgfpathmoveto{\pgfqpoint{2.163671in}{2.340546in}}%
\pgfpathcurveto{\pgfqpoint{2.171908in}{2.340546in}}{\pgfqpoint{2.179808in}{2.343818in}}{\pgfqpoint{2.185632in}{2.349642in}}%
\pgfpathcurveto{\pgfqpoint{2.191456in}{2.355466in}}{\pgfqpoint{2.194728in}{2.363366in}}{\pgfqpoint{2.194728in}{2.371603in}}%
\pgfpathcurveto{\pgfqpoint{2.194728in}{2.379839in}}{\pgfqpoint{2.191456in}{2.387739in}}{\pgfqpoint{2.185632in}{2.393563in}}%
\pgfpathcurveto{\pgfqpoint{2.179808in}{2.399387in}}{\pgfqpoint{2.171908in}{2.402659in}}{\pgfqpoint{2.163671in}{2.402659in}}%
\pgfpathcurveto{\pgfqpoint{2.155435in}{2.402659in}}{\pgfqpoint{2.147535in}{2.399387in}}{\pgfqpoint{2.141711in}{2.393563in}}%
\pgfpathcurveto{\pgfqpoint{2.135887in}{2.387739in}}{\pgfqpoint{2.132615in}{2.379839in}}{\pgfqpoint{2.132615in}{2.371603in}}%
\pgfpathcurveto{\pgfqpoint{2.132615in}{2.363366in}}{\pgfqpoint{2.135887in}{2.355466in}}{\pgfqpoint{2.141711in}{2.349642in}}%
\pgfpathcurveto{\pgfqpoint{2.147535in}{2.343818in}}{\pgfqpoint{2.155435in}{2.340546in}}{\pgfqpoint{2.163671in}{2.340546in}}%
\pgfpathclose%
\pgfusepath{stroke,fill}%
\end{pgfscope}%
\begin{pgfscope}%
\pgfpathrectangle{\pgfqpoint{0.100000in}{0.212622in}}{\pgfqpoint{3.696000in}{3.696000in}}%
\pgfusepath{clip}%
\pgfsetbuttcap%
\pgfsetroundjoin%
\definecolor{currentfill}{rgb}{0.121569,0.466667,0.705882}%
\pgfsetfillcolor{currentfill}%
\pgfsetfillopacity{0.597486}%
\pgfsetlinewidth{1.003750pt}%
\definecolor{currentstroke}{rgb}{0.121569,0.466667,0.705882}%
\pgfsetstrokecolor{currentstroke}%
\pgfsetstrokeopacity{0.597486}%
\pgfsetdash{}{0pt}%
\pgfpathmoveto{\pgfqpoint{1.129270in}{1.979658in}}%
\pgfpathcurveto{\pgfqpoint{1.137506in}{1.979658in}}{\pgfqpoint{1.145406in}{1.982931in}}{\pgfqpoint{1.151230in}{1.988755in}}%
\pgfpathcurveto{\pgfqpoint{1.157054in}{1.994579in}}{\pgfqpoint{1.160327in}{2.002479in}}{\pgfqpoint{1.160327in}{2.010715in}}%
\pgfpathcurveto{\pgfqpoint{1.160327in}{2.018951in}}{\pgfqpoint{1.157054in}{2.026851in}}{\pgfqpoint{1.151230in}{2.032675in}}%
\pgfpathcurveto{\pgfqpoint{1.145406in}{2.038499in}}{\pgfqpoint{1.137506in}{2.041771in}}{\pgfqpoint{1.129270in}{2.041771in}}%
\pgfpathcurveto{\pgfqpoint{1.121034in}{2.041771in}}{\pgfqpoint{1.113134in}{2.038499in}}{\pgfqpoint{1.107310in}{2.032675in}}%
\pgfpathcurveto{\pgfqpoint{1.101486in}{2.026851in}}{\pgfqpoint{1.098214in}{2.018951in}}{\pgfqpoint{1.098214in}{2.010715in}}%
\pgfpathcurveto{\pgfqpoint{1.098214in}{2.002479in}}{\pgfqpoint{1.101486in}{1.994579in}}{\pgfqpoint{1.107310in}{1.988755in}}%
\pgfpathcurveto{\pgfqpoint{1.113134in}{1.982931in}}{\pgfqpoint{1.121034in}{1.979658in}}{\pgfqpoint{1.129270in}{1.979658in}}%
\pgfpathclose%
\pgfusepath{stroke,fill}%
\end{pgfscope}%
\begin{pgfscope}%
\pgfpathrectangle{\pgfqpoint{0.100000in}{0.212622in}}{\pgfqpoint{3.696000in}{3.696000in}}%
\pgfusepath{clip}%
\pgfsetbuttcap%
\pgfsetroundjoin%
\definecolor{currentfill}{rgb}{0.121569,0.466667,0.705882}%
\pgfsetfillcolor{currentfill}%
\pgfsetfillopacity{0.601551}%
\pgfsetlinewidth{1.003750pt}%
\definecolor{currentstroke}{rgb}{0.121569,0.466667,0.705882}%
\pgfsetstrokecolor{currentstroke}%
\pgfsetstrokeopacity{0.601551}%
\pgfsetdash{}{0pt}%
\pgfpathmoveto{\pgfqpoint{2.169907in}{2.323393in}}%
\pgfpathcurveto{\pgfqpoint{2.178143in}{2.323393in}}{\pgfqpoint{2.186043in}{2.326666in}}{\pgfqpoint{2.191867in}{2.332490in}}%
\pgfpathcurveto{\pgfqpoint{2.197691in}{2.338314in}}{\pgfqpoint{2.200963in}{2.346214in}}{\pgfqpoint{2.200963in}{2.354450in}}%
\pgfpathcurveto{\pgfqpoint{2.200963in}{2.362686in}}{\pgfqpoint{2.197691in}{2.370586in}}{\pgfqpoint{2.191867in}{2.376410in}}%
\pgfpathcurveto{\pgfqpoint{2.186043in}{2.382234in}}{\pgfqpoint{2.178143in}{2.385506in}}{\pgfqpoint{2.169907in}{2.385506in}}%
\pgfpathcurveto{\pgfqpoint{2.161670in}{2.385506in}}{\pgfqpoint{2.153770in}{2.382234in}}{\pgfqpoint{2.147946in}{2.376410in}}%
\pgfpathcurveto{\pgfqpoint{2.142123in}{2.370586in}}{\pgfqpoint{2.138850in}{2.362686in}}{\pgfqpoint{2.138850in}{2.354450in}}%
\pgfpathcurveto{\pgfqpoint{2.138850in}{2.346214in}}{\pgfqpoint{2.142123in}{2.338314in}}{\pgfqpoint{2.147946in}{2.332490in}}%
\pgfpathcurveto{\pgfqpoint{2.153770in}{2.326666in}}{\pgfqpoint{2.161670in}{2.323393in}}{\pgfqpoint{2.169907in}{2.323393in}}%
\pgfpathclose%
\pgfusepath{stroke,fill}%
\end{pgfscope}%
\begin{pgfscope}%
\pgfpathrectangle{\pgfqpoint{0.100000in}{0.212622in}}{\pgfqpoint{3.696000in}{3.696000in}}%
\pgfusepath{clip}%
\pgfsetbuttcap%
\pgfsetroundjoin%
\definecolor{currentfill}{rgb}{0.121569,0.466667,0.705882}%
\pgfsetfillcolor{currentfill}%
\pgfsetfillopacity{0.603864}%
\pgfsetlinewidth{1.003750pt}%
\definecolor{currentstroke}{rgb}{0.121569,0.466667,0.705882}%
\pgfsetstrokecolor{currentstroke}%
\pgfsetstrokeopacity{0.603864}%
\pgfsetdash{}{0pt}%
\pgfpathmoveto{\pgfqpoint{1.122899in}{1.955404in}}%
\pgfpathcurveto{\pgfqpoint{1.131136in}{1.955404in}}{\pgfqpoint{1.139036in}{1.958676in}}{\pgfqpoint{1.144860in}{1.964500in}}%
\pgfpathcurveto{\pgfqpoint{1.150684in}{1.970324in}}{\pgfqpoint{1.153956in}{1.978224in}}{\pgfqpoint{1.153956in}{1.986461in}}%
\pgfpathcurveto{\pgfqpoint{1.153956in}{1.994697in}}{\pgfqpoint{1.150684in}{2.002597in}}{\pgfqpoint{1.144860in}{2.008421in}}%
\pgfpathcurveto{\pgfqpoint{1.139036in}{2.014245in}}{\pgfqpoint{1.131136in}{2.017517in}}{\pgfqpoint{1.122899in}{2.017517in}}%
\pgfpathcurveto{\pgfqpoint{1.114663in}{2.017517in}}{\pgfqpoint{1.106763in}{2.014245in}}{\pgfqpoint{1.100939in}{2.008421in}}%
\pgfpathcurveto{\pgfqpoint{1.095115in}{2.002597in}}{\pgfqpoint{1.091843in}{1.994697in}}{\pgfqpoint{1.091843in}{1.986461in}}%
\pgfpathcurveto{\pgfqpoint{1.091843in}{1.978224in}}{\pgfqpoint{1.095115in}{1.970324in}}{\pgfqpoint{1.100939in}{1.964500in}}%
\pgfpathcurveto{\pgfqpoint{1.106763in}{1.958676in}}{\pgfqpoint{1.114663in}{1.955404in}}{\pgfqpoint{1.122899in}{1.955404in}}%
\pgfpathclose%
\pgfusepath{stroke,fill}%
\end{pgfscope}%
\begin{pgfscope}%
\pgfpathrectangle{\pgfqpoint{0.100000in}{0.212622in}}{\pgfqpoint{3.696000in}{3.696000in}}%
\pgfusepath{clip}%
\pgfsetbuttcap%
\pgfsetroundjoin%
\definecolor{currentfill}{rgb}{0.121569,0.466667,0.705882}%
\pgfsetfillcolor{currentfill}%
\pgfsetfillopacity{0.605923}%
\pgfsetlinewidth{1.003750pt}%
\definecolor{currentstroke}{rgb}{0.121569,0.466667,0.705882}%
\pgfsetstrokecolor{currentstroke}%
\pgfsetstrokeopacity{0.605923}%
\pgfsetdash{}{0pt}%
\pgfpathmoveto{\pgfqpoint{0.586945in}{1.220138in}}%
\pgfpathcurveto{\pgfqpoint{0.595181in}{1.220138in}}{\pgfqpoint{0.603081in}{1.223410in}}{\pgfqpoint{0.608905in}{1.229234in}}%
\pgfpathcurveto{\pgfqpoint{0.614729in}{1.235058in}}{\pgfqpoint{0.618001in}{1.242958in}}{\pgfqpoint{0.618001in}{1.251195in}}%
\pgfpathcurveto{\pgfqpoint{0.618001in}{1.259431in}}{\pgfqpoint{0.614729in}{1.267331in}}{\pgfqpoint{0.608905in}{1.273155in}}%
\pgfpathcurveto{\pgfqpoint{0.603081in}{1.278979in}}{\pgfqpoint{0.595181in}{1.282251in}}{\pgfqpoint{0.586945in}{1.282251in}}%
\pgfpathcurveto{\pgfqpoint{0.578708in}{1.282251in}}{\pgfqpoint{0.570808in}{1.278979in}}{\pgfqpoint{0.564985in}{1.273155in}}%
\pgfpathcurveto{\pgfqpoint{0.559161in}{1.267331in}}{\pgfqpoint{0.555888in}{1.259431in}}{\pgfqpoint{0.555888in}{1.251195in}}%
\pgfpathcurveto{\pgfqpoint{0.555888in}{1.242958in}}{\pgfqpoint{0.559161in}{1.235058in}}{\pgfqpoint{0.564985in}{1.229234in}}%
\pgfpathcurveto{\pgfqpoint{0.570808in}{1.223410in}}{\pgfqpoint{0.578708in}{1.220138in}}{\pgfqpoint{0.586945in}{1.220138in}}%
\pgfpathclose%
\pgfusepath{stroke,fill}%
\end{pgfscope}%
\begin{pgfscope}%
\pgfpathrectangle{\pgfqpoint{0.100000in}{0.212622in}}{\pgfqpoint{3.696000in}{3.696000in}}%
\pgfusepath{clip}%
\pgfsetbuttcap%
\pgfsetroundjoin%
\definecolor{currentfill}{rgb}{0.121569,0.466667,0.705882}%
\pgfsetfillcolor{currentfill}%
\pgfsetfillopacity{0.607176}%
\pgfsetlinewidth{1.003750pt}%
\definecolor{currentstroke}{rgb}{0.121569,0.466667,0.705882}%
\pgfsetstrokecolor{currentstroke}%
\pgfsetstrokeopacity{0.607176}%
\pgfsetdash{}{0pt}%
\pgfpathmoveto{\pgfqpoint{2.175421in}{2.301802in}}%
\pgfpathcurveto{\pgfqpoint{2.183657in}{2.301802in}}{\pgfqpoint{2.191557in}{2.305075in}}{\pgfqpoint{2.197381in}{2.310899in}}%
\pgfpathcurveto{\pgfqpoint{2.203205in}{2.316723in}}{\pgfqpoint{2.206478in}{2.324623in}}{\pgfqpoint{2.206478in}{2.332859in}}%
\pgfpathcurveto{\pgfqpoint{2.206478in}{2.341095in}}{\pgfqpoint{2.203205in}{2.348995in}}{\pgfqpoint{2.197381in}{2.354819in}}%
\pgfpathcurveto{\pgfqpoint{2.191557in}{2.360643in}}{\pgfqpoint{2.183657in}{2.363915in}}{\pgfqpoint{2.175421in}{2.363915in}}%
\pgfpathcurveto{\pgfqpoint{2.167185in}{2.363915in}}{\pgfqpoint{2.159285in}{2.360643in}}{\pgfqpoint{2.153461in}{2.354819in}}%
\pgfpathcurveto{\pgfqpoint{2.147637in}{2.348995in}}{\pgfqpoint{2.144365in}{2.341095in}}{\pgfqpoint{2.144365in}{2.332859in}}%
\pgfpathcurveto{\pgfqpoint{2.144365in}{2.324623in}}{\pgfqpoint{2.147637in}{2.316723in}}{\pgfqpoint{2.153461in}{2.310899in}}%
\pgfpathcurveto{\pgfqpoint{2.159285in}{2.305075in}}{\pgfqpoint{2.167185in}{2.301802in}}{\pgfqpoint{2.175421in}{2.301802in}}%
\pgfpathclose%
\pgfusepath{stroke,fill}%
\end{pgfscope}%
\begin{pgfscope}%
\pgfpathrectangle{\pgfqpoint{0.100000in}{0.212622in}}{\pgfqpoint{3.696000in}{3.696000in}}%
\pgfusepath{clip}%
\pgfsetbuttcap%
\pgfsetroundjoin%
\definecolor{currentfill}{rgb}{0.121569,0.466667,0.705882}%
\pgfsetfillcolor{currentfill}%
\pgfsetfillopacity{0.609394}%
\pgfsetlinewidth{1.003750pt}%
\definecolor{currentstroke}{rgb}{0.121569,0.466667,0.705882}%
\pgfsetstrokecolor{currentstroke}%
\pgfsetstrokeopacity{0.609394}%
\pgfsetdash{}{0pt}%
\pgfpathmoveto{\pgfqpoint{1.120458in}{1.935494in}}%
\pgfpathcurveto{\pgfqpoint{1.128694in}{1.935494in}}{\pgfqpoint{1.136594in}{1.938766in}}{\pgfqpoint{1.142418in}{1.944590in}}%
\pgfpathcurveto{\pgfqpoint{1.148242in}{1.950414in}}{\pgfqpoint{1.151515in}{1.958314in}}{\pgfqpoint{1.151515in}{1.966550in}}%
\pgfpathcurveto{\pgfqpoint{1.151515in}{1.974786in}}{\pgfqpoint{1.148242in}{1.982687in}}{\pgfqpoint{1.142418in}{1.988510in}}%
\pgfpathcurveto{\pgfqpoint{1.136594in}{1.994334in}}{\pgfqpoint{1.128694in}{1.997607in}}{\pgfqpoint{1.120458in}{1.997607in}}%
\pgfpathcurveto{\pgfqpoint{1.112222in}{1.997607in}}{\pgfqpoint{1.104322in}{1.994334in}}{\pgfqpoint{1.098498in}{1.988510in}}%
\pgfpathcurveto{\pgfqpoint{1.092674in}{1.982687in}}{\pgfqpoint{1.089402in}{1.974786in}}{\pgfqpoint{1.089402in}{1.966550in}}%
\pgfpathcurveto{\pgfqpoint{1.089402in}{1.958314in}}{\pgfqpoint{1.092674in}{1.950414in}}{\pgfqpoint{1.098498in}{1.944590in}}%
\pgfpathcurveto{\pgfqpoint{1.104322in}{1.938766in}}{\pgfqpoint{1.112222in}{1.935494in}}{\pgfqpoint{1.120458in}{1.935494in}}%
\pgfpathclose%
\pgfusepath{stroke,fill}%
\end{pgfscope}%
\begin{pgfscope}%
\pgfpathrectangle{\pgfqpoint{0.100000in}{0.212622in}}{\pgfqpoint{3.696000in}{3.696000in}}%
\pgfusepath{clip}%
\pgfsetbuttcap%
\pgfsetroundjoin%
\definecolor{currentfill}{rgb}{0.121569,0.466667,0.705882}%
\pgfsetfillcolor{currentfill}%
\pgfsetfillopacity{0.610651}%
\pgfsetlinewidth{1.003750pt}%
\definecolor{currentstroke}{rgb}{0.121569,0.466667,0.705882}%
\pgfsetstrokecolor{currentstroke}%
\pgfsetstrokeopacity{0.610651}%
\pgfsetdash{}{0pt}%
\pgfpathmoveto{\pgfqpoint{0.601883in}{1.216902in}}%
\pgfpathcurveto{\pgfqpoint{0.610120in}{1.216902in}}{\pgfqpoint{0.618020in}{1.220174in}}{\pgfqpoint{0.623844in}{1.225998in}}%
\pgfpathcurveto{\pgfqpoint{0.629668in}{1.231822in}}{\pgfqpoint{0.632940in}{1.239722in}}{\pgfqpoint{0.632940in}{1.247958in}}%
\pgfpathcurveto{\pgfqpoint{0.632940in}{1.256194in}}{\pgfqpoint{0.629668in}{1.264094in}}{\pgfqpoint{0.623844in}{1.269918in}}%
\pgfpathcurveto{\pgfqpoint{0.618020in}{1.275742in}}{\pgfqpoint{0.610120in}{1.279015in}}{\pgfqpoint{0.601883in}{1.279015in}}%
\pgfpathcurveto{\pgfqpoint{0.593647in}{1.279015in}}{\pgfqpoint{0.585747in}{1.275742in}}{\pgfqpoint{0.579923in}{1.269918in}}%
\pgfpathcurveto{\pgfqpoint{0.574099in}{1.264094in}}{\pgfqpoint{0.570827in}{1.256194in}}{\pgfqpoint{0.570827in}{1.247958in}}%
\pgfpathcurveto{\pgfqpoint{0.570827in}{1.239722in}}{\pgfqpoint{0.574099in}{1.231822in}}{\pgfqpoint{0.579923in}{1.225998in}}%
\pgfpathcurveto{\pgfqpoint{0.585747in}{1.220174in}}{\pgfqpoint{0.593647in}{1.216902in}}{\pgfqpoint{0.601883in}{1.216902in}}%
\pgfpathclose%
\pgfusepath{stroke,fill}%
\end{pgfscope}%
\begin{pgfscope}%
\pgfpathrectangle{\pgfqpoint{0.100000in}{0.212622in}}{\pgfqpoint{3.696000in}{3.696000in}}%
\pgfusepath{clip}%
\pgfsetbuttcap%
\pgfsetroundjoin%
\definecolor{currentfill}{rgb}{0.121569,0.466667,0.705882}%
\pgfsetfillcolor{currentfill}%
\pgfsetfillopacity{0.613867}%
\pgfsetlinewidth{1.003750pt}%
\definecolor{currentstroke}{rgb}{0.121569,0.466667,0.705882}%
\pgfsetstrokecolor{currentstroke}%
\pgfsetstrokeopacity{0.613867}%
\pgfsetdash{}{0pt}%
\pgfpathmoveto{\pgfqpoint{2.183564in}{2.278973in}}%
\pgfpathcurveto{\pgfqpoint{2.191801in}{2.278973in}}{\pgfqpoint{2.199701in}{2.282246in}}{\pgfqpoint{2.205525in}{2.288070in}}%
\pgfpathcurveto{\pgfqpoint{2.211349in}{2.293894in}}{\pgfqpoint{2.214621in}{2.301794in}}{\pgfqpoint{2.214621in}{2.310030in}}%
\pgfpathcurveto{\pgfqpoint{2.214621in}{2.318266in}}{\pgfqpoint{2.211349in}{2.326166in}}{\pgfqpoint{2.205525in}{2.331990in}}%
\pgfpathcurveto{\pgfqpoint{2.199701in}{2.337814in}}{\pgfqpoint{2.191801in}{2.341086in}}{\pgfqpoint{2.183564in}{2.341086in}}%
\pgfpathcurveto{\pgfqpoint{2.175328in}{2.341086in}}{\pgfqpoint{2.167428in}{2.337814in}}{\pgfqpoint{2.161604in}{2.331990in}}%
\pgfpathcurveto{\pgfqpoint{2.155780in}{2.326166in}}{\pgfqpoint{2.152508in}{2.318266in}}{\pgfqpoint{2.152508in}{2.310030in}}%
\pgfpathcurveto{\pgfqpoint{2.152508in}{2.301794in}}{\pgfqpoint{2.155780in}{2.293894in}}{\pgfqpoint{2.161604in}{2.288070in}}%
\pgfpathcurveto{\pgfqpoint{2.167428in}{2.282246in}}{\pgfqpoint{2.175328in}{2.278973in}}{\pgfqpoint{2.183564in}{2.278973in}}%
\pgfpathclose%
\pgfusepath{stroke,fill}%
\end{pgfscope}%
\begin{pgfscope}%
\pgfpathrectangle{\pgfqpoint{0.100000in}{0.212622in}}{\pgfqpoint{3.696000in}{3.696000in}}%
\pgfusepath{clip}%
\pgfsetbuttcap%
\pgfsetroundjoin%
\definecolor{currentfill}{rgb}{0.121569,0.466667,0.705882}%
\pgfsetfillcolor{currentfill}%
\pgfsetfillopacity{0.614309}%
\pgfsetlinewidth{1.003750pt}%
\definecolor{currentstroke}{rgb}{0.121569,0.466667,0.705882}%
\pgfsetstrokecolor{currentstroke}%
\pgfsetstrokeopacity{0.614309}%
\pgfsetdash{}{0pt}%
\pgfpathmoveto{\pgfqpoint{1.121749in}{1.919665in}}%
\pgfpathcurveto{\pgfqpoint{1.129985in}{1.919665in}}{\pgfqpoint{1.137885in}{1.922937in}}{\pgfqpoint{1.143709in}{1.928761in}}%
\pgfpathcurveto{\pgfqpoint{1.149533in}{1.934585in}}{\pgfqpoint{1.152805in}{1.942485in}}{\pgfqpoint{1.152805in}{1.950721in}}%
\pgfpathcurveto{\pgfqpoint{1.152805in}{1.958957in}}{\pgfqpoint{1.149533in}{1.966858in}}{\pgfqpoint{1.143709in}{1.972681in}}%
\pgfpathcurveto{\pgfqpoint{1.137885in}{1.978505in}}{\pgfqpoint{1.129985in}{1.981778in}}{\pgfqpoint{1.121749in}{1.981778in}}%
\pgfpathcurveto{\pgfqpoint{1.113512in}{1.981778in}}{\pgfqpoint{1.105612in}{1.978505in}}{\pgfqpoint{1.099788in}{1.972681in}}%
\pgfpathcurveto{\pgfqpoint{1.093964in}{1.966858in}}{\pgfqpoint{1.090692in}{1.958957in}}{\pgfqpoint{1.090692in}{1.950721in}}%
\pgfpathcurveto{\pgfqpoint{1.090692in}{1.942485in}}{\pgfqpoint{1.093964in}{1.934585in}}{\pgfqpoint{1.099788in}{1.928761in}}%
\pgfpathcurveto{\pgfqpoint{1.105612in}{1.922937in}}{\pgfqpoint{1.113512in}{1.919665in}}{\pgfqpoint{1.121749in}{1.919665in}}%
\pgfpathclose%
\pgfusepath{stroke,fill}%
\end{pgfscope}%
\begin{pgfscope}%
\pgfpathrectangle{\pgfqpoint{0.100000in}{0.212622in}}{\pgfqpoint{3.696000in}{3.696000in}}%
\pgfusepath{clip}%
\pgfsetbuttcap%
\pgfsetroundjoin%
\definecolor{currentfill}{rgb}{0.121569,0.466667,0.705882}%
\pgfsetfillcolor{currentfill}%
\pgfsetfillopacity{0.614712}%
\pgfsetlinewidth{1.003750pt}%
\definecolor{currentstroke}{rgb}{0.121569,0.466667,0.705882}%
\pgfsetstrokecolor{currentstroke}%
\pgfsetstrokeopacity{0.614712}%
\pgfsetdash{}{0pt}%
\pgfpathmoveto{\pgfqpoint{0.615354in}{1.214331in}}%
\pgfpathcurveto{\pgfqpoint{0.623590in}{1.214331in}}{\pgfqpoint{0.631490in}{1.217604in}}{\pgfqpoint{0.637314in}{1.223427in}}%
\pgfpathcurveto{\pgfqpoint{0.643138in}{1.229251in}}{\pgfqpoint{0.646410in}{1.237151in}}{\pgfqpoint{0.646410in}{1.245388in}}%
\pgfpathcurveto{\pgfqpoint{0.646410in}{1.253624in}}{\pgfqpoint{0.643138in}{1.261524in}}{\pgfqpoint{0.637314in}{1.267348in}}%
\pgfpathcurveto{\pgfqpoint{0.631490in}{1.273172in}}{\pgfqpoint{0.623590in}{1.276444in}}{\pgfqpoint{0.615354in}{1.276444in}}%
\pgfpathcurveto{\pgfqpoint{0.607118in}{1.276444in}}{\pgfqpoint{0.599218in}{1.273172in}}{\pgfqpoint{0.593394in}{1.267348in}}%
\pgfpathcurveto{\pgfqpoint{0.587570in}{1.261524in}}{\pgfqpoint{0.584297in}{1.253624in}}{\pgfqpoint{0.584297in}{1.245388in}}%
\pgfpathcurveto{\pgfqpoint{0.584297in}{1.237151in}}{\pgfqpoint{0.587570in}{1.229251in}}{\pgfqpoint{0.593394in}{1.223427in}}%
\pgfpathcurveto{\pgfqpoint{0.599218in}{1.217604in}}{\pgfqpoint{0.607118in}{1.214331in}}{\pgfqpoint{0.615354in}{1.214331in}}%
\pgfpathclose%
\pgfusepath{stroke,fill}%
\end{pgfscope}%
\begin{pgfscope}%
\pgfpathrectangle{\pgfqpoint{0.100000in}{0.212622in}}{\pgfqpoint{3.696000in}{3.696000in}}%
\pgfusepath{clip}%
\pgfsetbuttcap%
\pgfsetroundjoin%
\definecolor{currentfill}{rgb}{0.121569,0.466667,0.705882}%
\pgfsetfillcolor{currentfill}%
\pgfsetfillopacity{0.618188}%
\pgfsetlinewidth{1.003750pt}%
\definecolor{currentstroke}{rgb}{0.121569,0.466667,0.705882}%
\pgfsetstrokecolor{currentstroke}%
\pgfsetstrokeopacity{0.618188}%
\pgfsetdash{}{0pt}%
\pgfpathmoveto{\pgfqpoint{0.626362in}{1.210977in}}%
\pgfpathcurveto{\pgfqpoint{0.634598in}{1.210977in}}{\pgfqpoint{0.642498in}{1.214249in}}{\pgfqpoint{0.648322in}{1.220073in}}%
\pgfpathcurveto{\pgfqpoint{0.654146in}{1.225897in}}{\pgfqpoint{0.657418in}{1.233797in}}{\pgfqpoint{0.657418in}{1.242033in}}%
\pgfpathcurveto{\pgfqpoint{0.657418in}{1.250270in}}{\pgfqpoint{0.654146in}{1.258170in}}{\pgfqpoint{0.648322in}{1.263994in}}%
\pgfpathcurveto{\pgfqpoint{0.642498in}{1.269818in}}{\pgfqpoint{0.634598in}{1.273090in}}{\pgfqpoint{0.626362in}{1.273090in}}%
\pgfpathcurveto{\pgfqpoint{0.618125in}{1.273090in}}{\pgfqpoint{0.610225in}{1.269818in}}{\pgfqpoint{0.604401in}{1.263994in}}%
\pgfpathcurveto{\pgfqpoint{0.598577in}{1.258170in}}{\pgfqpoint{0.595305in}{1.250270in}}{\pgfqpoint{0.595305in}{1.242033in}}%
\pgfpathcurveto{\pgfqpoint{0.595305in}{1.233797in}}{\pgfqpoint{0.598577in}{1.225897in}}{\pgfqpoint{0.604401in}{1.220073in}}%
\pgfpathcurveto{\pgfqpoint{0.610225in}{1.214249in}}{\pgfqpoint{0.618125in}{1.210977in}}{\pgfqpoint{0.626362in}{1.210977in}}%
\pgfpathclose%
\pgfusepath{stroke,fill}%
\end{pgfscope}%
\begin{pgfscope}%
\pgfpathrectangle{\pgfqpoint{0.100000in}{0.212622in}}{\pgfqpoint{3.696000in}{3.696000in}}%
\pgfusepath{clip}%
\pgfsetbuttcap%
\pgfsetroundjoin%
\definecolor{currentfill}{rgb}{0.121569,0.466667,0.705882}%
\pgfsetfillcolor{currentfill}%
\pgfsetfillopacity{0.618239}%
\pgfsetlinewidth{1.003750pt}%
\definecolor{currentstroke}{rgb}{0.121569,0.466667,0.705882}%
\pgfsetstrokecolor{currentstroke}%
\pgfsetstrokeopacity{0.618239}%
\pgfsetdash{}{0pt}%
\pgfpathmoveto{\pgfqpoint{1.125320in}{1.906528in}}%
\pgfpathcurveto{\pgfqpoint{1.133556in}{1.906528in}}{\pgfqpoint{1.141456in}{1.909800in}}{\pgfqpoint{1.147280in}{1.915624in}}%
\pgfpathcurveto{\pgfqpoint{1.153104in}{1.921448in}}{\pgfqpoint{1.156376in}{1.929348in}}{\pgfqpoint{1.156376in}{1.937585in}}%
\pgfpathcurveto{\pgfqpoint{1.156376in}{1.945821in}}{\pgfqpoint{1.153104in}{1.953721in}}{\pgfqpoint{1.147280in}{1.959545in}}%
\pgfpathcurveto{\pgfqpoint{1.141456in}{1.965369in}}{\pgfqpoint{1.133556in}{1.968641in}}{\pgfqpoint{1.125320in}{1.968641in}}%
\pgfpathcurveto{\pgfqpoint{1.117084in}{1.968641in}}{\pgfqpoint{1.109184in}{1.965369in}}{\pgfqpoint{1.103360in}{1.959545in}}%
\pgfpathcurveto{\pgfqpoint{1.097536in}{1.953721in}}{\pgfqpoint{1.094263in}{1.945821in}}{\pgfqpoint{1.094263in}{1.937585in}}%
\pgfpathcurveto{\pgfqpoint{1.094263in}{1.929348in}}{\pgfqpoint{1.097536in}{1.921448in}}{\pgfqpoint{1.103360in}{1.915624in}}%
\pgfpathcurveto{\pgfqpoint{1.109184in}{1.909800in}}{\pgfqpoint{1.117084in}{1.906528in}}{\pgfqpoint{1.125320in}{1.906528in}}%
\pgfpathclose%
\pgfusepath{stroke,fill}%
\end{pgfscope}%
\begin{pgfscope}%
\pgfpathrectangle{\pgfqpoint{0.100000in}{0.212622in}}{\pgfqpoint{3.696000in}{3.696000in}}%
\pgfusepath{clip}%
\pgfsetbuttcap%
\pgfsetroundjoin%
\definecolor{currentfill}{rgb}{0.121569,0.466667,0.705882}%
\pgfsetfillcolor{currentfill}%
\pgfsetfillopacity{0.620883}%
\pgfsetlinewidth{1.003750pt}%
\definecolor{currentstroke}{rgb}{0.121569,0.466667,0.705882}%
\pgfsetstrokecolor{currentstroke}%
\pgfsetstrokeopacity{0.620883}%
\pgfsetdash{}{0pt}%
\pgfpathmoveto{\pgfqpoint{0.635126in}{1.208471in}}%
\pgfpathcurveto{\pgfqpoint{0.643362in}{1.208471in}}{\pgfqpoint{0.651262in}{1.211743in}}{\pgfqpoint{0.657086in}{1.217567in}}%
\pgfpathcurveto{\pgfqpoint{0.662910in}{1.223391in}}{\pgfqpoint{0.666182in}{1.231291in}}{\pgfqpoint{0.666182in}{1.239528in}}%
\pgfpathcurveto{\pgfqpoint{0.666182in}{1.247764in}}{\pgfqpoint{0.662910in}{1.255664in}}{\pgfqpoint{0.657086in}{1.261488in}}%
\pgfpathcurveto{\pgfqpoint{0.651262in}{1.267312in}}{\pgfqpoint{0.643362in}{1.270584in}}{\pgfqpoint{0.635126in}{1.270584in}}%
\pgfpathcurveto{\pgfqpoint{0.626889in}{1.270584in}}{\pgfqpoint{0.618989in}{1.267312in}}{\pgfqpoint{0.613165in}{1.261488in}}%
\pgfpathcurveto{\pgfqpoint{0.607341in}{1.255664in}}{\pgfqpoint{0.604069in}{1.247764in}}{\pgfqpoint{0.604069in}{1.239528in}}%
\pgfpathcurveto{\pgfqpoint{0.604069in}{1.231291in}}{\pgfqpoint{0.607341in}{1.223391in}}{\pgfqpoint{0.613165in}{1.217567in}}%
\pgfpathcurveto{\pgfqpoint{0.618989in}{1.211743in}}{\pgfqpoint{0.626889in}{1.208471in}}{\pgfqpoint{0.635126in}{1.208471in}}%
\pgfpathclose%
\pgfusepath{stroke,fill}%
\end{pgfscope}%
\begin{pgfscope}%
\pgfpathrectangle{\pgfqpoint{0.100000in}{0.212622in}}{\pgfqpoint{3.696000in}{3.696000in}}%
\pgfusepath{clip}%
\pgfsetbuttcap%
\pgfsetroundjoin%
\definecolor{currentfill}{rgb}{0.121569,0.466667,0.705882}%
\pgfsetfillcolor{currentfill}%
\pgfsetfillopacity{0.621641}%
\pgfsetlinewidth{1.003750pt}%
\definecolor{currentstroke}{rgb}{0.121569,0.466667,0.705882}%
\pgfsetstrokecolor{currentstroke}%
\pgfsetstrokeopacity{0.621641}%
\pgfsetdash{}{0pt}%
\pgfpathmoveto{\pgfqpoint{2.190501in}{2.249320in}}%
\pgfpathcurveto{\pgfqpoint{2.198738in}{2.249320in}}{\pgfqpoint{2.206638in}{2.252592in}}{\pgfqpoint{2.212462in}{2.258416in}}%
\pgfpathcurveto{\pgfqpoint{2.218286in}{2.264240in}}{\pgfqpoint{2.221558in}{2.272140in}}{\pgfqpoint{2.221558in}{2.280376in}}%
\pgfpathcurveto{\pgfqpoint{2.221558in}{2.288613in}}{\pgfqpoint{2.218286in}{2.296513in}}{\pgfqpoint{2.212462in}{2.302337in}}%
\pgfpathcurveto{\pgfqpoint{2.206638in}{2.308160in}}{\pgfqpoint{2.198738in}{2.311433in}}{\pgfqpoint{2.190501in}{2.311433in}}%
\pgfpathcurveto{\pgfqpoint{2.182265in}{2.311433in}}{\pgfqpoint{2.174365in}{2.308160in}}{\pgfqpoint{2.168541in}{2.302337in}}%
\pgfpathcurveto{\pgfqpoint{2.162717in}{2.296513in}}{\pgfqpoint{2.159445in}{2.288613in}}{\pgfqpoint{2.159445in}{2.280376in}}%
\pgfpathcurveto{\pgfqpoint{2.159445in}{2.272140in}}{\pgfqpoint{2.162717in}{2.264240in}}{\pgfqpoint{2.168541in}{2.258416in}}%
\pgfpathcurveto{\pgfqpoint{2.174365in}{2.252592in}}{\pgfqpoint{2.182265in}{2.249320in}}{\pgfqpoint{2.190501in}{2.249320in}}%
\pgfpathclose%
\pgfusepath{stroke,fill}%
\end{pgfscope}%
\begin{pgfscope}%
\pgfpathrectangle{\pgfqpoint{0.100000in}{0.212622in}}{\pgfqpoint{3.696000in}{3.696000in}}%
\pgfusepath{clip}%
\pgfsetbuttcap%
\pgfsetroundjoin%
\definecolor{currentfill}{rgb}{0.121569,0.466667,0.705882}%
\pgfsetfillcolor{currentfill}%
\pgfsetfillopacity{0.621746}%
\pgfsetlinewidth{1.003750pt}%
\definecolor{currentstroke}{rgb}{0.121569,0.466667,0.705882}%
\pgfsetstrokecolor{currentstroke}%
\pgfsetstrokeopacity{0.621746}%
\pgfsetdash{}{0pt}%
\pgfpathmoveto{\pgfqpoint{1.132125in}{1.897074in}}%
\pgfpathcurveto{\pgfqpoint{1.140361in}{1.897074in}}{\pgfqpoint{1.148261in}{1.900346in}}{\pgfqpoint{1.154085in}{1.906170in}}%
\pgfpathcurveto{\pgfqpoint{1.159909in}{1.911994in}}{\pgfqpoint{1.163181in}{1.919894in}}{\pgfqpoint{1.163181in}{1.928130in}}%
\pgfpathcurveto{\pgfqpoint{1.163181in}{1.936366in}}{\pgfqpoint{1.159909in}{1.944266in}}{\pgfqpoint{1.154085in}{1.950090in}}%
\pgfpathcurveto{\pgfqpoint{1.148261in}{1.955914in}}{\pgfqpoint{1.140361in}{1.959187in}}{\pgfqpoint{1.132125in}{1.959187in}}%
\pgfpathcurveto{\pgfqpoint{1.123889in}{1.959187in}}{\pgfqpoint{1.115989in}{1.955914in}}{\pgfqpoint{1.110165in}{1.950090in}}%
\pgfpathcurveto{\pgfqpoint{1.104341in}{1.944266in}}{\pgfqpoint{1.101068in}{1.936366in}}{\pgfqpoint{1.101068in}{1.928130in}}%
\pgfpathcurveto{\pgfqpoint{1.101068in}{1.919894in}}{\pgfqpoint{1.104341in}{1.911994in}}{\pgfqpoint{1.110165in}{1.906170in}}%
\pgfpathcurveto{\pgfqpoint{1.115989in}{1.900346in}}{\pgfqpoint{1.123889in}{1.897074in}}{\pgfqpoint{1.132125in}{1.897074in}}%
\pgfpathclose%
\pgfusepath{stroke,fill}%
\end{pgfscope}%
\begin{pgfscope}%
\pgfpathrectangle{\pgfqpoint{0.100000in}{0.212622in}}{\pgfqpoint{3.696000in}{3.696000in}}%
\pgfusepath{clip}%
\pgfsetbuttcap%
\pgfsetroundjoin%
\definecolor{currentfill}{rgb}{0.121569,0.466667,0.705882}%
\pgfsetfillcolor{currentfill}%
\pgfsetfillopacity{0.622437}%
\pgfsetlinewidth{1.003750pt}%
\definecolor{currentstroke}{rgb}{0.121569,0.466667,0.705882}%
\pgfsetstrokecolor{currentstroke}%
\pgfsetstrokeopacity{0.622437}%
\pgfsetdash{}{0pt}%
\pgfpathmoveto{\pgfqpoint{1.134266in}{1.895393in}}%
\pgfpathcurveto{\pgfqpoint{1.142502in}{1.895393in}}{\pgfqpoint{1.150402in}{1.898666in}}{\pgfqpoint{1.156226in}{1.904490in}}%
\pgfpathcurveto{\pgfqpoint{1.162050in}{1.910314in}}{\pgfqpoint{1.165322in}{1.918214in}}{\pgfqpoint{1.165322in}{1.926450in}}%
\pgfpathcurveto{\pgfqpoint{1.165322in}{1.934686in}}{\pgfqpoint{1.162050in}{1.942586in}}{\pgfqpoint{1.156226in}{1.948410in}}%
\pgfpathcurveto{\pgfqpoint{1.150402in}{1.954234in}}{\pgfqpoint{1.142502in}{1.957506in}}{\pgfqpoint{1.134266in}{1.957506in}}%
\pgfpathcurveto{\pgfqpoint{1.126029in}{1.957506in}}{\pgfqpoint{1.118129in}{1.954234in}}{\pgfqpoint{1.112305in}{1.948410in}}%
\pgfpathcurveto{\pgfqpoint{1.106481in}{1.942586in}}{\pgfqpoint{1.103209in}{1.934686in}}{\pgfqpoint{1.103209in}{1.926450in}}%
\pgfpathcurveto{\pgfqpoint{1.103209in}{1.918214in}}{\pgfqpoint{1.106481in}{1.910314in}}{\pgfqpoint{1.112305in}{1.904490in}}%
\pgfpathcurveto{\pgfqpoint{1.118129in}{1.898666in}}{\pgfqpoint{1.126029in}{1.895393in}}{\pgfqpoint{1.134266in}{1.895393in}}%
\pgfpathclose%
\pgfusepath{stroke,fill}%
\end{pgfscope}%
\begin{pgfscope}%
\pgfpathrectangle{\pgfqpoint{0.100000in}{0.212622in}}{\pgfqpoint{3.696000in}{3.696000in}}%
\pgfusepath{clip}%
\pgfsetbuttcap%
\pgfsetroundjoin%
\definecolor{currentfill}{rgb}{0.121569,0.466667,0.705882}%
\pgfsetfillcolor{currentfill}%
\pgfsetfillopacity{0.622673}%
\pgfsetlinewidth{1.003750pt}%
\definecolor{currentstroke}{rgb}{0.121569,0.466667,0.705882}%
\pgfsetstrokecolor{currentstroke}%
\pgfsetstrokeopacity{0.622673}%
\pgfsetdash{}{0pt}%
\pgfpathmoveto{\pgfqpoint{0.641233in}{1.207248in}}%
\pgfpathcurveto{\pgfqpoint{0.649469in}{1.207248in}}{\pgfqpoint{0.657369in}{1.210520in}}{\pgfqpoint{0.663193in}{1.216344in}}%
\pgfpathcurveto{\pgfqpoint{0.669017in}{1.222168in}}{\pgfqpoint{0.672289in}{1.230068in}}{\pgfqpoint{0.672289in}{1.238304in}}%
\pgfpathcurveto{\pgfqpoint{0.672289in}{1.246541in}}{\pgfqpoint{0.669017in}{1.254441in}}{\pgfqpoint{0.663193in}{1.260265in}}%
\pgfpathcurveto{\pgfqpoint{0.657369in}{1.266088in}}{\pgfqpoint{0.649469in}{1.269361in}}{\pgfqpoint{0.641233in}{1.269361in}}%
\pgfpathcurveto{\pgfqpoint{0.632996in}{1.269361in}}{\pgfqpoint{0.625096in}{1.266088in}}{\pgfqpoint{0.619272in}{1.260265in}}%
\pgfpathcurveto{\pgfqpoint{0.613448in}{1.254441in}}{\pgfqpoint{0.610176in}{1.246541in}}{\pgfqpoint{0.610176in}{1.238304in}}%
\pgfpathcurveto{\pgfqpoint{0.610176in}{1.230068in}}{\pgfqpoint{0.613448in}{1.222168in}}{\pgfqpoint{0.619272in}{1.216344in}}%
\pgfpathcurveto{\pgfqpoint{0.625096in}{1.210520in}}{\pgfqpoint{0.632996in}{1.207248in}}{\pgfqpoint{0.641233in}{1.207248in}}%
\pgfpathclose%
\pgfusepath{stroke,fill}%
\end{pgfscope}%
\begin{pgfscope}%
\pgfpathrectangle{\pgfqpoint{0.100000in}{0.212622in}}{\pgfqpoint{3.696000in}{3.696000in}}%
\pgfusepath{clip}%
\pgfsetbuttcap%
\pgfsetroundjoin%
\definecolor{currentfill}{rgb}{0.121569,0.466667,0.705882}%
\pgfsetfillcolor{currentfill}%
\pgfsetfillopacity{0.622940}%
\pgfsetlinewidth{1.003750pt}%
\definecolor{currentstroke}{rgb}{0.121569,0.466667,0.705882}%
\pgfsetstrokecolor{currentstroke}%
\pgfsetstrokeopacity{0.622940}%
\pgfsetdash{}{0pt}%
\pgfpathmoveto{\pgfqpoint{1.159760in}{1.902377in}}%
\pgfpathcurveto{\pgfqpoint{1.167997in}{1.902377in}}{\pgfqpoint{1.175897in}{1.905649in}}{\pgfqpoint{1.181721in}{1.911473in}}%
\pgfpathcurveto{\pgfqpoint{1.187545in}{1.917297in}}{\pgfqpoint{1.190817in}{1.925197in}}{\pgfqpoint{1.190817in}{1.933434in}}%
\pgfpathcurveto{\pgfqpoint{1.190817in}{1.941670in}}{\pgfqpoint{1.187545in}{1.949570in}}{\pgfqpoint{1.181721in}{1.955394in}}%
\pgfpathcurveto{\pgfqpoint{1.175897in}{1.961218in}}{\pgfqpoint{1.167997in}{1.964490in}}{\pgfqpoint{1.159760in}{1.964490in}}%
\pgfpathcurveto{\pgfqpoint{1.151524in}{1.964490in}}{\pgfqpoint{1.143624in}{1.961218in}}{\pgfqpoint{1.137800in}{1.955394in}}%
\pgfpathcurveto{\pgfqpoint{1.131976in}{1.949570in}}{\pgfqpoint{1.128704in}{1.941670in}}{\pgfqpoint{1.128704in}{1.933434in}}%
\pgfpathcurveto{\pgfqpoint{1.128704in}{1.925197in}}{\pgfqpoint{1.131976in}{1.917297in}}{\pgfqpoint{1.137800in}{1.911473in}}%
\pgfpathcurveto{\pgfqpoint{1.143624in}{1.905649in}}{\pgfqpoint{1.151524in}{1.902377in}}{\pgfqpoint{1.159760in}{1.902377in}}%
\pgfpathclose%
\pgfusepath{stroke,fill}%
\end{pgfscope}%
\begin{pgfscope}%
\pgfpathrectangle{\pgfqpoint{0.100000in}{0.212622in}}{\pgfqpoint{3.696000in}{3.696000in}}%
\pgfusepath{clip}%
\pgfsetbuttcap%
\pgfsetroundjoin%
\definecolor{currentfill}{rgb}{0.121569,0.466667,0.705882}%
\pgfsetfillcolor{currentfill}%
\pgfsetfillopacity{0.623008}%
\pgfsetlinewidth{1.003750pt}%
\definecolor{currentstroke}{rgb}{0.121569,0.466667,0.705882}%
\pgfsetstrokecolor{currentstroke}%
\pgfsetstrokeopacity{0.623008}%
\pgfsetdash{}{0pt}%
\pgfpathmoveto{\pgfqpoint{1.157103in}{1.904156in}}%
\pgfpathcurveto{\pgfqpoint{1.165340in}{1.904156in}}{\pgfqpoint{1.173240in}{1.907428in}}{\pgfqpoint{1.179064in}{1.913252in}}%
\pgfpathcurveto{\pgfqpoint{1.184887in}{1.919076in}}{\pgfqpoint{1.188160in}{1.926976in}}{\pgfqpoint{1.188160in}{1.935213in}}%
\pgfpathcurveto{\pgfqpoint{1.188160in}{1.943449in}}{\pgfqpoint{1.184887in}{1.951349in}}{\pgfqpoint{1.179064in}{1.957173in}}%
\pgfpathcurveto{\pgfqpoint{1.173240in}{1.962997in}}{\pgfqpoint{1.165340in}{1.966269in}}{\pgfqpoint{1.157103in}{1.966269in}}%
\pgfpathcurveto{\pgfqpoint{1.148867in}{1.966269in}}{\pgfqpoint{1.140967in}{1.962997in}}{\pgfqpoint{1.135143in}{1.957173in}}%
\pgfpathcurveto{\pgfqpoint{1.129319in}{1.951349in}}{\pgfqpoint{1.126047in}{1.943449in}}{\pgfqpoint{1.126047in}{1.935213in}}%
\pgfpathcurveto{\pgfqpoint{1.126047in}{1.926976in}}{\pgfqpoint{1.129319in}{1.919076in}}{\pgfqpoint{1.135143in}{1.913252in}}%
\pgfpathcurveto{\pgfqpoint{1.140967in}{1.907428in}}{\pgfqpoint{1.148867in}{1.904156in}}{\pgfqpoint{1.157103in}{1.904156in}}%
\pgfpathclose%
\pgfusepath{stroke,fill}%
\end{pgfscope}%
\begin{pgfscope}%
\pgfpathrectangle{\pgfqpoint{0.100000in}{0.212622in}}{\pgfqpoint{3.696000in}{3.696000in}}%
\pgfusepath{clip}%
\pgfsetbuttcap%
\pgfsetroundjoin%
\definecolor{currentfill}{rgb}{0.121569,0.466667,0.705882}%
\pgfsetfillcolor{currentfill}%
\pgfsetfillopacity{0.623057}%
\pgfsetlinewidth{1.003750pt}%
\definecolor{currentstroke}{rgb}{0.121569,0.466667,0.705882}%
\pgfsetstrokecolor{currentstroke}%
\pgfsetstrokeopacity{0.623057}%
\pgfsetdash{}{0pt}%
\pgfpathmoveto{\pgfqpoint{1.161090in}{1.901245in}}%
\pgfpathcurveto{\pgfqpoint{1.169327in}{1.901245in}}{\pgfqpoint{1.177227in}{1.904517in}}{\pgfqpoint{1.183051in}{1.910341in}}%
\pgfpathcurveto{\pgfqpoint{1.188875in}{1.916165in}}{\pgfqpoint{1.192147in}{1.924065in}}{\pgfqpoint{1.192147in}{1.932301in}}%
\pgfpathcurveto{\pgfqpoint{1.192147in}{1.940538in}}{\pgfqpoint{1.188875in}{1.948438in}}{\pgfqpoint{1.183051in}{1.954262in}}%
\pgfpathcurveto{\pgfqpoint{1.177227in}{1.960085in}}{\pgfqpoint{1.169327in}{1.963358in}}{\pgfqpoint{1.161090in}{1.963358in}}%
\pgfpathcurveto{\pgfqpoint{1.152854in}{1.963358in}}{\pgfqpoint{1.144954in}{1.960085in}}{\pgfqpoint{1.139130in}{1.954262in}}%
\pgfpathcurveto{\pgfqpoint{1.133306in}{1.948438in}}{\pgfqpoint{1.130034in}{1.940538in}}{\pgfqpoint{1.130034in}{1.932301in}}%
\pgfpathcurveto{\pgfqpoint{1.130034in}{1.924065in}}{\pgfqpoint{1.133306in}{1.916165in}}{\pgfqpoint{1.139130in}{1.910341in}}%
\pgfpathcurveto{\pgfqpoint{1.144954in}{1.904517in}}{\pgfqpoint{1.152854in}{1.901245in}}{\pgfqpoint{1.161090in}{1.901245in}}%
\pgfpathclose%
\pgfusepath{stroke,fill}%
\end{pgfscope}%
\begin{pgfscope}%
\pgfpathrectangle{\pgfqpoint{0.100000in}{0.212622in}}{\pgfqpoint{3.696000in}{3.696000in}}%
\pgfusepath{clip}%
\pgfsetbuttcap%
\pgfsetroundjoin%
\definecolor{currentfill}{rgb}{0.121569,0.466667,0.705882}%
\pgfsetfillcolor{currentfill}%
\pgfsetfillopacity{0.623183}%
\pgfsetlinewidth{1.003750pt}%
\definecolor{currentstroke}{rgb}{0.121569,0.466667,0.705882}%
\pgfsetstrokecolor{currentstroke}%
\pgfsetstrokeopacity{0.623183}%
\pgfsetdash{}{0pt}%
\pgfpathmoveto{\pgfqpoint{1.161677in}{1.900562in}}%
\pgfpathcurveto{\pgfqpoint{1.169913in}{1.900562in}}{\pgfqpoint{1.177813in}{1.903834in}}{\pgfqpoint{1.183637in}{1.909658in}}%
\pgfpathcurveto{\pgfqpoint{1.189461in}{1.915482in}}{\pgfqpoint{1.192733in}{1.923382in}}{\pgfqpoint{1.192733in}{1.931619in}}%
\pgfpathcurveto{\pgfqpoint{1.192733in}{1.939855in}}{\pgfqpoint{1.189461in}{1.947755in}}{\pgfqpoint{1.183637in}{1.953579in}}%
\pgfpathcurveto{\pgfqpoint{1.177813in}{1.959403in}}{\pgfqpoint{1.169913in}{1.962675in}}{\pgfqpoint{1.161677in}{1.962675in}}%
\pgfpathcurveto{\pgfqpoint{1.153441in}{1.962675in}}{\pgfqpoint{1.145541in}{1.959403in}}{\pgfqpoint{1.139717in}{1.953579in}}%
\pgfpathcurveto{\pgfqpoint{1.133893in}{1.947755in}}{\pgfqpoint{1.130620in}{1.939855in}}{\pgfqpoint{1.130620in}{1.931619in}}%
\pgfpathcurveto{\pgfqpoint{1.130620in}{1.923382in}}{\pgfqpoint{1.133893in}{1.915482in}}{\pgfqpoint{1.139717in}{1.909658in}}%
\pgfpathcurveto{\pgfqpoint{1.145541in}{1.903834in}}{\pgfqpoint{1.153441in}{1.900562in}}{\pgfqpoint{1.161677in}{1.900562in}}%
\pgfpathclose%
\pgfusepath{stroke,fill}%
\end{pgfscope}%
\begin{pgfscope}%
\pgfpathrectangle{\pgfqpoint{0.100000in}{0.212622in}}{\pgfqpoint{3.696000in}{3.696000in}}%
\pgfusepath{clip}%
\pgfsetbuttcap%
\pgfsetroundjoin%
\definecolor{currentfill}{rgb}{0.121569,0.466667,0.705882}%
\pgfsetfillcolor{currentfill}%
\pgfsetfillopacity{0.623272}%
\pgfsetlinewidth{1.003750pt}%
\definecolor{currentstroke}{rgb}{0.121569,0.466667,0.705882}%
\pgfsetstrokecolor{currentstroke}%
\pgfsetstrokeopacity{0.623272}%
\pgfsetdash{}{0pt}%
\pgfpathmoveto{\pgfqpoint{1.161894in}{1.900172in}}%
\pgfpathcurveto{\pgfqpoint{1.170131in}{1.900172in}}{\pgfqpoint{1.178031in}{1.903444in}}{\pgfqpoint{1.183855in}{1.909268in}}%
\pgfpathcurveto{\pgfqpoint{1.189679in}{1.915092in}}{\pgfqpoint{1.192951in}{1.922992in}}{\pgfqpoint{1.192951in}{1.931228in}}%
\pgfpathcurveto{\pgfqpoint{1.192951in}{1.939465in}}{\pgfqpoint{1.189679in}{1.947365in}}{\pgfqpoint{1.183855in}{1.953189in}}%
\pgfpathcurveto{\pgfqpoint{1.178031in}{1.959012in}}{\pgfqpoint{1.170131in}{1.962285in}}{\pgfqpoint{1.161894in}{1.962285in}}%
\pgfpathcurveto{\pgfqpoint{1.153658in}{1.962285in}}{\pgfqpoint{1.145758in}{1.959012in}}{\pgfqpoint{1.139934in}{1.953189in}}%
\pgfpathcurveto{\pgfqpoint{1.134110in}{1.947365in}}{\pgfqpoint{1.130838in}{1.939465in}}{\pgfqpoint{1.130838in}{1.931228in}}%
\pgfpathcurveto{\pgfqpoint{1.130838in}{1.922992in}}{\pgfqpoint{1.134110in}{1.915092in}}{\pgfqpoint{1.139934in}{1.909268in}}%
\pgfpathcurveto{\pgfqpoint{1.145758in}{1.903444in}}{\pgfqpoint{1.153658in}{1.900172in}}{\pgfqpoint{1.161894in}{1.900172in}}%
\pgfpathclose%
\pgfusepath{stroke,fill}%
\end{pgfscope}%
\begin{pgfscope}%
\pgfpathrectangle{\pgfqpoint{0.100000in}{0.212622in}}{\pgfqpoint{3.696000in}{3.696000in}}%
\pgfusepath{clip}%
\pgfsetbuttcap%
\pgfsetroundjoin%
\definecolor{currentfill}{rgb}{0.121569,0.466667,0.705882}%
\pgfsetfillcolor{currentfill}%
\pgfsetfillopacity{0.623326}%
\pgfsetlinewidth{1.003750pt}%
\definecolor{currentstroke}{rgb}{0.121569,0.466667,0.705882}%
\pgfsetstrokecolor{currentstroke}%
\pgfsetstrokeopacity{0.623326}%
\pgfsetdash{}{0pt}%
\pgfpathmoveto{\pgfqpoint{1.161965in}{1.899958in}}%
\pgfpathcurveto{\pgfqpoint{1.170201in}{1.899958in}}{\pgfqpoint{1.178101in}{1.903230in}}{\pgfqpoint{1.183925in}{1.909054in}}%
\pgfpathcurveto{\pgfqpoint{1.189749in}{1.914878in}}{\pgfqpoint{1.193021in}{1.922778in}}{\pgfqpoint{1.193021in}{1.931014in}}%
\pgfpathcurveto{\pgfqpoint{1.193021in}{1.939250in}}{\pgfqpoint{1.189749in}{1.947150in}}{\pgfqpoint{1.183925in}{1.952974in}}%
\pgfpathcurveto{\pgfqpoint{1.178101in}{1.958798in}}{\pgfqpoint{1.170201in}{1.962071in}}{\pgfqpoint{1.161965in}{1.962071in}}%
\pgfpathcurveto{\pgfqpoint{1.153728in}{1.962071in}}{\pgfqpoint{1.145828in}{1.958798in}}{\pgfqpoint{1.140004in}{1.952974in}}%
\pgfpathcurveto{\pgfqpoint{1.134180in}{1.947150in}}{\pgfqpoint{1.130908in}{1.939250in}}{\pgfqpoint{1.130908in}{1.931014in}}%
\pgfpathcurveto{\pgfqpoint{1.130908in}{1.922778in}}{\pgfqpoint{1.134180in}{1.914878in}}{\pgfqpoint{1.140004in}{1.909054in}}%
\pgfpathcurveto{\pgfqpoint{1.145828in}{1.903230in}}{\pgfqpoint{1.153728in}{1.899958in}}{\pgfqpoint{1.161965in}{1.899958in}}%
\pgfpathclose%
\pgfusepath{stroke,fill}%
\end{pgfscope}%
\begin{pgfscope}%
\pgfpathrectangle{\pgfqpoint{0.100000in}{0.212622in}}{\pgfqpoint{3.696000in}{3.696000in}}%
\pgfusepath{clip}%
\pgfsetbuttcap%
\pgfsetroundjoin%
\definecolor{currentfill}{rgb}{0.121569,0.466667,0.705882}%
\pgfsetfillcolor{currentfill}%
\pgfsetfillopacity{0.623354}%
\pgfsetlinewidth{1.003750pt}%
\definecolor{currentstroke}{rgb}{0.121569,0.466667,0.705882}%
\pgfsetstrokecolor{currentstroke}%
\pgfsetstrokeopacity{0.623354}%
\pgfsetdash{}{0pt}%
\pgfpathmoveto{\pgfqpoint{1.161984in}{1.899839in}}%
\pgfpathcurveto{\pgfqpoint{1.170220in}{1.899839in}}{\pgfqpoint{1.178120in}{1.903111in}}{\pgfqpoint{1.183944in}{1.908935in}}%
\pgfpathcurveto{\pgfqpoint{1.189768in}{1.914759in}}{\pgfqpoint{1.193040in}{1.922659in}}{\pgfqpoint{1.193040in}{1.930896in}}%
\pgfpathcurveto{\pgfqpoint{1.193040in}{1.939132in}}{\pgfqpoint{1.189768in}{1.947032in}}{\pgfqpoint{1.183944in}{1.952856in}}%
\pgfpathcurveto{\pgfqpoint{1.178120in}{1.958680in}}{\pgfqpoint{1.170220in}{1.961952in}}{\pgfqpoint{1.161984in}{1.961952in}}%
\pgfpathcurveto{\pgfqpoint{1.153747in}{1.961952in}}{\pgfqpoint{1.145847in}{1.958680in}}{\pgfqpoint{1.140023in}{1.952856in}}%
\pgfpathcurveto{\pgfqpoint{1.134200in}{1.947032in}}{\pgfqpoint{1.130927in}{1.939132in}}{\pgfqpoint{1.130927in}{1.930896in}}%
\pgfpathcurveto{\pgfqpoint{1.130927in}{1.922659in}}{\pgfqpoint{1.134200in}{1.914759in}}{\pgfqpoint{1.140023in}{1.908935in}}%
\pgfpathcurveto{\pgfqpoint{1.145847in}{1.903111in}}{\pgfqpoint{1.153747in}{1.899839in}}{\pgfqpoint{1.161984in}{1.899839in}}%
\pgfpathclose%
\pgfusepath{stroke,fill}%
\end{pgfscope}%
\begin{pgfscope}%
\pgfpathrectangle{\pgfqpoint{0.100000in}{0.212622in}}{\pgfqpoint{3.696000in}{3.696000in}}%
\pgfusepath{clip}%
\pgfsetbuttcap%
\pgfsetroundjoin%
\definecolor{currentfill}{rgb}{0.121569,0.466667,0.705882}%
\pgfsetfillcolor{currentfill}%
\pgfsetfillopacity{0.623370}%
\pgfsetlinewidth{1.003750pt}%
\definecolor{currentstroke}{rgb}{0.121569,0.466667,0.705882}%
\pgfsetstrokecolor{currentstroke}%
\pgfsetstrokeopacity{0.623370}%
\pgfsetdash{}{0pt}%
\pgfpathmoveto{\pgfqpoint{1.161984in}{1.899776in}}%
\pgfpathcurveto{\pgfqpoint{1.170221in}{1.899776in}}{\pgfqpoint{1.178121in}{1.903048in}}{\pgfqpoint{1.183945in}{1.908872in}}%
\pgfpathcurveto{\pgfqpoint{1.189768in}{1.914696in}}{\pgfqpoint{1.193041in}{1.922596in}}{\pgfqpoint{1.193041in}{1.930832in}}%
\pgfpathcurveto{\pgfqpoint{1.193041in}{1.939069in}}{\pgfqpoint{1.189768in}{1.946969in}}{\pgfqpoint{1.183945in}{1.952793in}}%
\pgfpathcurveto{\pgfqpoint{1.178121in}{1.958616in}}{\pgfqpoint{1.170221in}{1.961889in}}{\pgfqpoint{1.161984in}{1.961889in}}%
\pgfpathcurveto{\pgfqpoint{1.153748in}{1.961889in}}{\pgfqpoint{1.145848in}{1.958616in}}{\pgfqpoint{1.140024in}{1.952793in}}%
\pgfpathcurveto{\pgfqpoint{1.134200in}{1.946969in}}{\pgfqpoint{1.130928in}{1.939069in}}{\pgfqpoint{1.130928in}{1.930832in}}%
\pgfpathcurveto{\pgfqpoint{1.130928in}{1.922596in}}{\pgfqpoint{1.134200in}{1.914696in}}{\pgfqpoint{1.140024in}{1.908872in}}%
\pgfpathcurveto{\pgfqpoint{1.145848in}{1.903048in}}{\pgfqpoint{1.153748in}{1.899776in}}{\pgfqpoint{1.161984in}{1.899776in}}%
\pgfpathclose%
\pgfusepath{stroke,fill}%
\end{pgfscope}%
\begin{pgfscope}%
\pgfpathrectangle{\pgfqpoint{0.100000in}{0.212622in}}{\pgfqpoint{3.696000in}{3.696000in}}%
\pgfusepath{clip}%
\pgfsetbuttcap%
\pgfsetroundjoin%
\definecolor{currentfill}{rgb}{0.121569,0.466667,0.705882}%
\pgfsetfillcolor{currentfill}%
\pgfsetfillopacity{0.623378}%
\pgfsetlinewidth{1.003750pt}%
\definecolor{currentstroke}{rgb}{0.121569,0.466667,0.705882}%
\pgfsetstrokecolor{currentstroke}%
\pgfsetstrokeopacity{0.623378}%
\pgfsetdash{}{0pt}%
\pgfpathmoveto{\pgfqpoint{1.161980in}{1.899742in}}%
\pgfpathcurveto{\pgfqpoint{1.170216in}{1.899742in}}{\pgfqpoint{1.178116in}{1.903014in}}{\pgfqpoint{1.183940in}{1.908838in}}%
\pgfpathcurveto{\pgfqpoint{1.189764in}{1.914662in}}{\pgfqpoint{1.193037in}{1.922562in}}{\pgfqpoint{1.193037in}{1.930798in}}%
\pgfpathcurveto{\pgfqpoint{1.193037in}{1.939035in}}{\pgfqpoint{1.189764in}{1.946935in}}{\pgfqpoint{1.183940in}{1.952759in}}%
\pgfpathcurveto{\pgfqpoint{1.178116in}{1.958583in}}{\pgfqpoint{1.170216in}{1.961855in}}{\pgfqpoint{1.161980in}{1.961855in}}%
\pgfpathcurveto{\pgfqpoint{1.153744in}{1.961855in}}{\pgfqpoint{1.145844in}{1.958583in}}{\pgfqpoint{1.140020in}{1.952759in}}%
\pgfpathcurveto{\pgfqpoint{1.134196in}{1.946935in}}{\pgfqpoint{1.130924in}{1.939035in}}{\pgfqpoint{1.130924in}{1.930798in}}%
\pgfpathcurveto{\pgfqpoint{1.130924in}{1.922562in}}{\pgfqpoint{1.134196in}{1.914662in}}{\pgfqpoint{1.140020in}{1.908838in}}%
\pgfpathcurveto{\pgfqpoint{1.145844in}{1.903014in}}{\pgfqpoint{1.153744in}{1.899742in}}{\pgfqpoint{1.161980in}{1.899742in}}%
\pgfpathclose%
\pgfusepath{stroke,fill}%
\end{pgfscope}%
\begin{pgfscope}%
\pgfpathrectangle{\pgfqpoint{0.100000in}{0.212622in}}{\pgfqpoint{3.696000in}{3.696000in}}%
\pgfusepath{clip}%
\pgfsetbuttcap%
\pgfsetroundjoin%
\definecolor{currentfill}{rgb}{0.121569,0.466667,0.705882}%
\pgfsetfillcolor{currentfill}%
\pgfsetfillopacity{0.623382}%
\pgfsetlinewidth{1.003750pt}%
\definecolor{currentstroke}{rgb}{0.121569,0.466667,0.705882}%
\pgfsetstrokecolor{currentstroke}%
\pgfsetstrokeopacity{0.623382}%
\pgfsetdash{}{0pt}%
\pgfpathmoveto{\pgfqpoint{1.161977in}{1.899723in}}%
\pgfpathcurveto{\pgfqpoint{1.170213in}{1.899723in}}{\pgfqpoint{1.178113in}{1.902996in}}{\pgfqpoint{1.183937in}{1.908820in}}%
\pgfpathcurveto{\pgfqpoint{1.189761in}{1.914643in}}{\pgfqpoint{1.193033in}{1.922544in}}{\pgfqpoint{1.193033in}{1.930780in}}%
\pgfpathcurveto{\pgfqpoint{1.193033in}{1.939016in}}{\pgfqpoint{1.189761in}{1.946916in}}{\pgfqpoint{1.183937in}{1.952740in}}%
\pgfpathcurveto{\pgfqpoint{1.178113in}{1.958564in}}{\pgfqpoint{1.170213in}{1.961836in}}{\pgfqpoint{1.161977in}{1.961836in}}%
\pgfpathcurveto{\pgfqpoint{1.153740in}{1.961836in}}{\pgfqpoint{1.145840in}{1.958564in}}{\pgfqpoint{1.140016in}{1.952740in}}%
\pgfpathcurveto{\pgfqpoint{1.134192in}{1.946916in}}{\pgfqpoint{1.130920in}{1.939016in}}{\pgfqpoint{1.130920in}{1.930780in}}%
\pgfpathcurveto{\pgfqpoint{1.130920in}{1.922544in}}{\pgfqpoint{1.134192in}{1.914643in}}{\pgfqpoint{1.140016in}{1.908820in}}%
\pgfpathcurveto{\pgfqpoint{1.145840in}{1.902996in}}{\pgfqpoint{1.153740in}{1.899723in}}{\pgfqpoint{1.161977in}{1.899723in}}%
\pgfpathclose%
\pgfusepath{stroke,fill}%
\end{pgfscope}%
\begin{pgfscope}%
\pgfpathrectangle{\pgfqpoint{0.100000in}{0.212622in}}{\pgfqpoint{3.696000in}{3.696000in}}%
\pgfusepath{clip}%
\pgfsetbuttcap%
\pgfsetroundjoin%
\definecolor{currentfill}{rgb}{0.121569,0.466667,0.705882}%
\pgfsetfillcolor{currentfill}%
\pgfsetfillopacity{0.623385}%
\pgfsetlinewidth{1.003750pt}%
\definecolor{currentstroke}{rgb}{0.121569,0.466667,0.705882}%
\pgfsetstrokecolor{currentstroke}%
\pgfsetstrokeopacity{0.623385}%
\pgfsetdash{}{0pt}%
\pgfpathmoveto{\pgfqpoint{1.161974in}{1.899713in}}%
\pgfpathcurveto{\pgfqpoint{1.170210in}{1.899713in}}{\pgfqpoint{1.178111in}{1.902986in}}{\pgfqpoint{1.183934in}{1.908810in}}%
\pgfpathcurveto{\pgfqpoint{1.189758in}{1.914633in}}{\pgfqpoint{1.193031in}{1.922534in}}{\pgfqpoint{1.193031in}{1.930770in}}%
\pgfpathcurveto{\pgfqpoint{1.193031in}{1.939006in}}{\pgfqpoint{1.189758in}{1.946906in}}{\pgfqpoint{1.183934in}{1.952730in}}%
\pgfpathcurveto{\pgfqpoint{1.178111in}{1.958554in}}{\pgfqpoint{1.170210in}{1.961826in}}{\pgfqpoint{1.161974in}{1.961826in}}%
\pgfpathcurveto{\pgfqpoint{1.153738in}{1.961826in}}{\pgfqpoint{1.145838in}{1.958554in}}{\pgfqpoint{1.140014in}{1.952730in}}%
\pgfpathcurveto{\pgfqpoint{1.134190in}{1.946906in}}{\pgfqpoint{1.130918in}{1.939006in}}{\pgfqpoint{1.130918in}{1.930770in}}%
\pgfpathcurveto{\pgfqpoint{1.130918in}{1.922534in}}{\pgfqpoint{1.134190in}{1.914633in}}{\pgfqpoint{1.140014in}{1.908810in}}%
\pgfpathcurveto{\pgfqpoint{1.145838in}{1.902986in}}{\pgfqpoint{1.153738in}{1.899713in}}{\pgfqpoint{1.161974in}{1.899713in}}%
\pgfpathclose%
\pgfusepath{stroke,fill}%
\end{pgfscope}%
\begin{pgfscope}%
\pgfpathrectangle{\pgfqpoint{0.100000in}{0.212622in}}{\pgfqpoint{3.696000in}{3.696000in}}%
\pgfusepath{clip}%
\pgfsetbuttcap%
\pgfsetroundjoin%
\definecolor{currentfill}{rgb}{0.121569,0.466667,0.705882}%
\pgfsetfillcolor{currentfill}%
\pgfsetfillopacity{0.623386}%
\pgfsetlinewidth{1.003750pt}%
\definecolor{currentstroke}{rgb}{0.121569,0.466667,0.705882}%
\pgfsetstrokecolor{currentstroke}%
\pgfsetstrokeopacity{0.623386}%
\pgfsetdash{}{0pt}%
\pgfpathmoveto{\pgfqpoint{1.161973in}{1.899708in}}%
\pgfpathcurveto{\pgfqpoint{1.170209in}{1.899708in}}{\pgfqpoint{1.178109in}{1.902980in}}{\pgfqpoint{1.183933in}{1.908804in}}%
\pgfpathcurveto{\pgfqpoint{1.189757in}{1.914628in}}{\pgfqpoint{1.193029in}{1.922528in}}{\pgfqpoint{1.193029in}{1.930764in}}%
\pgfpathcurveto{\pgfqpoint{1.193029in}{1.939001in}}{\pgfqpoint{1.189757in}{1.946901in}}{\pgfqpoint{1.183933in}{1.952725in}}%
\pgfpathcurveto{\pgfqpoint{1.178109in}{1.958548in}}{\pgfqpoint{1.170209in}{1.961821in}}{\pgfqpoint{1.161973in}{1.961821in}}%
\pgfpathcurveto{\pgfqpoint{1.153736in}{1.961821in}}{\pgfqpoint{1.145836in}{1.958548in}}{\pgfqpoint{1.140012in}{1.952725in}}%
\pgfpathcurveto{\pgfqpoint{1.134188in}{1.946901in}}{\pgfqpoint{1.130916in}{1.939001in}}{\pgfqpoint{1.130916in}{1.930764in}}%
\pgfpathcurveto{\pgfqpoint{1.130916in}{1.922528in}}{\pgfqpoint{1.134188in}{1.914628in}}{\pgfqpoint{1.140012in}{1.908804in}}%
\pgfpathcurveto{\pgfqpoint{1.145836in}{1.902980in}}{\pgfqpoint{1.153736in}{1.899708in}}{\pgfqpoint{1.161973in}{1.899708in}}%
\pgfpathclose%
\pgfusepath{stroke,fill}%
\end{pgfscope}%
\begin{pgfscope}%
\pgfpathrectangle{\pgfqpoint{0.100000in}{0.212622in}}{\pgfqpoint{3.696000in}{3.696000in}}%
\pgfusepath{clip}%
\pgfsetbuttcap%
\pgfsetroundjoin%
\definecolor{currentfill}{rgb}{0.121569,0.466667,0.705882}%
\pgfsetfillcolor{currentfill}%
\pgfsetfillopacity{0.623387}%
\pgfsetlinewidth{1.003750pt}%
\definecolor{currentstroke}{rgb}{0.121569,0.466667,0.705882}%
\pgfsetstrokecolor{currentstroke}%
\pgfsetstrokeopacity{0.623387}%
\pgfsetdash{}{0pt}%
\pgfpathmoveto{\pgfqpoint{1.161972in}{1.899705in}}%
\pgfpathcurveto{\pgfqpoint{1.170208in}{1.899705in}}{\pgfqpoint{1.178108in}{1.902977in}}{\pgfqpoint{1.183932in}{1.908801in}}%
\pgfpathcurveto{\pgfqpoint{1.189756in}{1.914625in}}{\pgfqpoint{1.193028in}{1.922525in}}{\pgfqpoint{1.193028in}{1.930761in}}%
\pgfpathcurveto{\pgfqpoint{1.193028in}{1.938998in}}{\pgfqpoint{1.189756in}{1.946898in}}{\pgfqpoint{1.183932in}{1.952722in}}%
\pgfpathcurveto{\pgfqpoint{1.178108in}{1.958546in}}{\pgfqpoint{1.170208in}{1.961818in}}{\pgfqpoint{1.161972in}{1.961818in}}%
\pgfpathcurveto{\pgfqpoint{1.153735in}{1.961818in}}{\pgfqpoint{1.145835in}{1.958546in}}{\pgfqpoint{1.140011in}{1.952722in}}%
\pgfpathcurveto{\pgfqpoint{1.134187in}{1.946898in}}{\pgfqpoint{1.130915in}{1.938998in}}{\pgfqpoint{1.130915in}{1.930761in}}%
\pgfpathcurveto{\pgfqpoint{1.130915in}{1.922525in}}{\pgfqpoint{1.134187in}{1.914625in}}{\pgfqpoint{1.140011in}{1.908801in}}%
\pgfpathcurveto{\pgfqpoint{1.145835in}{1.902977in}}{\pgfqpoint{1.153735in}{1.899705in}}{\pgfqpoint{1.161972in}{1.899705in}}%
\pgfpathclose%
\pgfusepath{stroke,fill}%
\end{pgfscope}%
\begin{pgfscope}%
\pgfpathrectangle{\pgfqpoint{0.100000in}{0.212622in}}{\pgfqpoint{3.696000in}{3.696000in}}%
\pgfusepath{clip}%
\pgfsetbuttcap%
\pgfsetroundjoin%
\definecolor{currentfill}{rgb}{0.121569,0.466667,0.705882}%
\pgfsetfillcolor{currentfill}%
\pgfsetfillopacity{0.623387}%
\pgfsetlinewidth{1.003750pt}%
\definecolor{currentstroke}{rgb}{0.121569,0.466667,0.705882}%
\pgfsetstrokecolor{currentstroke}%
\pgfsetstrokeopacity{0.623387}%
\pgfsetdash{}{0pt}%
\pgfpathmoveto{\pgfqpoint{1.161971in}{1.899703in}}%
\pgfpathcurveto{\pgfqpoint{1.170207in}{1.899703in}}{\pgfqpoint{1.178107in}{1.902976in}}{\pgfqpoint{1.183931in}{1.908799in}}%
\pgfpathcurveto{\pgfqpoint{1.189755in}{1.914623in}}{\pgfqpoint{1.193027in}{1.922523in}}{\pgfqpoint{1.193027in}{1.930760in}}%
\pgfpathcurveto{\pgfqpoint{1.193027in}{1.938996in}}{\pgfqpoint{1.189755in}{1.946896in}}{\pgfqpoint{1.183931in}{1.952720in}}%
\pgfpathcurveto{\pgfqpoint{1.178107in}{1.958544in}}{\pgfqpoint{1.170207in}{1.961816in}}{\pgfqpoint{1.161971in}{1.961816in}}%
\pgfpathcurveto{\pgfqpoint{1.153735in}{1.961816in}}{\pgfqpoint{1.145835in}{1.958544in}}{\pgfqpoint{1.140011in}{1.952720in}}%
\pgfpathcurveto{\pgfqpoint{1.134187in}{1.946896in}}{\pgfqpoint{1.130914in}{1.938996in}}{\pgfqpoint{1.130914in}{1.930760in}}%
\pgfpathcurveto{\pgfqpoint{1.130914in}{1.922523in}}{\pgfqpoint{1.134187in}{1.914623in}}{\pgfqpoint{1.140011in}{1.908799in}}%
\pgfpathcurveto{\pgfqpoint{1.145835in}{1.902976in}}{\pgfqpoint{1.153735in}{1.899703in}}{\pgfqpoint{1.161971in}{1.899703in}}%
\pgfpathclose%
\pgfusepath{stroke,fill}%
\end{pgfscope}%
\begin{pgfscope}%
\pgfpathrectangle{\pgfqpoint{0.100000in}{0.212622in}}{\pgfqpoint{3.696000in}{3.696000in}}%
\pgfusepath{clip}%
\pgfsetbuttcap%
\pgfsetroundjoin%
\definecolor{currentfill}{rgb}{0.121569,0.466667,0.705882}%
\pgfsetfillcolor{currentfill}%
\pgfsetfillopacity{0.623472}%
\pgfsetlinewidth{1.003750pt}%
\definecolor{currentstroke}{rgb}{0.121569,0.466667,0.705882}%
\pgfsetstrokecolor{currentstroke}%
\pgfsetstrokeopacity{0.623472}%
\pgfsetdash{}{0pt}%
\pgfpathmoveto{\pgfqpoint{1.138987in}{1.893060in}}%
\pgfpathcurveto{\pgfqpoint{1.147223in}{1.893060in}}{\pgfqpoint{1.155123in}{1.896333in}}{\pgfqpoint{1.160947in}{1.902157in}}%
\pgfpathcurveto{\pgfqpoint{1.166771in}{1.907981in}}{\pgfqpoint{1.170044in}{1.915881in}}{\pgfqpoint{1.170044in}{1.924117in}}%
\pgfpathcurveto{\pgfqpoint{1.170044in}{1.932353in}}{\pgfqpoint{1.166771in}{1.940253in}}{\pgfqpoint{1.160947in}{1.946077in}}%
\pgfpathcurveto{\pgfqpoint{1.155123in}{1.951901in}}{\pgfqpoint{1.147223in}{1.955173in}}{\pgfqpoint{1.138987in}{1.955173in}}%
\pgfpathcurveto{\pgfqpoint{1.130751in}{1.955173in}}{\pgfqpoint{1.122851in}{1.951901in}}{\pgfqpoint{1.117027in}{1.946077in}}%
\pgfpathcurveto{\pgfqpoint{1.111203in}{1.940253in}}{\pgfqpoint{1.107931in}{1.932353in}}{\pgfqpoint{1.107931in}{1.924117in}}%
\pgfpathcurveto{\pgfqpoint{1.107931in}{1.915881in}}{\pgfqpoint{1.111203in}{1.907981in}}{\pgfqpoint{1.117027in}{1.902157in}}%
\pgfpathcurveto{\pgfqpoint{1.122851in}{1.896333in}}{\pgfqpoint{1.130751in}{1.893060in}}{\pgfqpoint{1.138987in}{1.893060in}}%
\pgfpathclose%
\pgfusepath{stroke,fill}%
\end{pgfscope}%
\begin{pgfscope}%
\pgfpathrectangle{\pgfqpoint{0.100000in}{0.212622in}}{\pgfqpoint{3.696000in}{3.696000in}}%
\pgfusepath{clip}%
\pgfsetbuttcap%
\pgfsetroundjoin%
\definecolor{currentfill}{rgb}{0.121569,0.466667,0.705882}%
\pgfsetfillcolor{currentfill}%
\pgfsetfillopacity{0.623599}%
\pgfsetlinewidth{1.003750pt}%
\definecolor{currentstroke}{rgb}{0.121569,0.466667,0.705882}%
\pgfsetstrokecolor{currentstroke}%
\pgfsetstrokeopacity{0.623599}%
\pgfsetdash{}{0pt}%
\pgfpathmoveto{\pgfqpoint{1.152329in}{1.906865in}}%
\pgfpathcurveto{\pgfqpoint{1.160566in}{1.906865in}}{\pgfqpoint{1.168466in}{1.910137in}}{\pgfqpoint{1.174290in}{1.915961in}}%
\pgfpathcurveto{\pgfqpoint{1.180114in}{1.921785in}}{\pgfqpoint{1.183386in}{1.929685in}}{\pgfqpoint{1.183386in}{1.937921in}}%
\pgfpathcurveto{\pgfqpoint{1.183386in}{1.946157in}}{\pgfqpoint{1.180114in}{1.954058in}}{\pgfqpoint{1.174290in}{1.959881in}}%
\pgfpathcurveto{\pgfqpoint{1.168466in}{1.965705in}}{\pgfqpoint{1.160566in}{1.968978in}}{\pgfqpoint{1.152329in}{1.968978in}}%
\pgfpathcurveto{\pgfqpoint{1.144093in}{1.968978in}}{\pgfqpoint{1.136193in}{1.965705in}}{\pgfqpoint{1.130369in}{1.959881in}}%
\pgfpathcurveto{\pgfqpoint{1.124545in}{1.954058in}}{\pgfqpoint{1.121273in}{1.946157in}}{\pgfqpoint{1.121273in}{1.937921in}}%
\pgfpathcurveto{\pgfqpoint{1.121273in}{1.929685in}}{\pgfqpoint{1.124545in}{1.921785in}}{\pgfqpoint{1.130369in}{1.915961in}}%
\pgfpathcurveto{\pgfqpoint{1.136193in}{1.910137in}}{\pgfqpoint{1.144093in}{1.906865in}}{\pgfqpoint{1.152329in}{1.906865in}}%
\pgfpathclose%
\pgfusepath{stroke,fill}%
\end{pgfscope}%
\begin{pgfscope}%
\pgfpathrectangle{\pgfqpoint{0.100000in}{0.212622in}}{\pgfqpoint{3.696000in}{3.696000in}}%
\pgfusepath{clip}%
\pgfsetbuttcap%
\pgfsetroundjoin%
\definecolor{currentfill}{rgb}{0.121569,0.466667,0.705882}%
\pgfsetfillcolor{currentfill}%
\pgfsetfillopacity{0.623657}%
\pgfsetlinewidth{1.003750pt}%
\definecolor{currentstroke}{rgb}{0.121569,0.466667,0.705882}%
\pgfsetstrokecolor{currentstroke}%
\pgfsetstrokeopacity{0.623657}%
\pgfsetdash{}{0pt}%
\pgfpathmoveto{\pgfqpoint{0.644699in}{1.206649in}}%
\pgfpathcurveto{\pgfqpoint{0.652935in}{1.206649in}}{\pgfqpoint{0.660835in}{1.209921in}}{\pgfqpoint{0.666659in}{1.215745in}}%
\pgfpathcurveto{\pgfqpoint{0.672483in}{1.221569in}}{\pgfqpoint{0.675755in}{1.229469in}}{\pgfqpoint{0.675755in}{1.237705in}}%
\pgfpathcurveto{\pgfqpoint{0.675755in}{1.245941in}}{\pgfqpoint{0.672483in}{1.253841in}}{\pgfqpoint{0.666659in}{1.259665in}}%
\pgfpathcurveto{\pgfqpoint{0.660835in}{1.265489in}}{\pgfqpoint{0.652935in}{1.268762in}}{\pgfqpoint{0.644699in}{1.268762in}}%
\pgfpathcurveto{\pgfqpoint{0.636462in}{1.268762in}}{\pgfqpoint{0.628562in}{1.265489in}}{\pgfqpoint{0.622738in}{1.259665in}}%
\pgfpathcurveto{\pgfqpoint{0.616914in}{1.253841in}}{\pgfqpoint{0.613642in}{1.245941in}}{\pgfqpoint{0.613642in}{1.237705in}}%
\pgfpathcurveto{\pgfqpoint{0.613642in}{1.229469in}}{\pgfqpoint{0.616914in}{1.221569in}}{\pgfqpoint{0.622738in}{1.215745in}}%
\pgfpathcurveto{\pgfqpoint{0.628562in}{1.209921in}}{\pgfqpoint{0.636462in}{1.206649in}}{\pgfqpoint{0.644699in}{1.206649in}}%
\pgfpathclose%
\pgfusepath{stroke,fill}%
\end{pgfscope}%
\begin{pgfscope}%
\pgfpathrectangle{\pgfqpoint{0.100000in}{0.212622in}}{\pgfqpoint{3.696000in}{3.696000in}}%
\pgfusepath{clip}%
\pgfsetbuttcap%
\pgfsetroundjoin%
\definecolor{currentfill}{rgb}{0.121569,0.466667,0.705882}%
\pgfsetfillcolor{currentfill}%
\pgfsetfillopacity{0.623749}%
\pgfsetlinewidth{1.003750pt}%
\definecolor{currentstroke}{rgb}{0.121569,0.466667,0.705882}%
\pgfsetstrokecolor{currentstroke}%
\pgfsetstrokeopacity{0.623749}%
\pgfsetdash{}{0pt}%
\pgfpathmoveto{\pgfqpoint{1.161360in}{1.898025in}}%
\pgfpathcurveto{\pgfqpoint{1.169596in}{1.898025in}}{\pgfqpoint{1.177496in}{1.901297in}}{\pgfqpoint{1.183320in}{1.907121in}}%
\pgfpathcurveto{\pgfqpoint{1.189144in}{1.912945in}}{\pgfqpoint{1.192416in}{1.920845in}}{\pgfqpoint{1.192416in}{1.929081in}}%
\pgfpathcurveto{\pgfqpoint{1.192416in}{1.937317in}}{\pgfqpoint{1.189144in}{1.945217in}}{\pgfqpoint{1.183320in}{1.951041in}}%
\pgfpathcurveto{\pgfqpoint{1.177496in}{1.956865in}}{\pgfqpoint{1.169596in}{1.960138in}}{\pgfqpoint{1.161360in}{1.960138in}}%
\pgfpathcurveto{\pgfqpoint{1.153124in}{1.960138in}}{\pgfqpoint{1.145223in}{1.956865in}}{\pgfqpoint{1.139400in}{1.951041in}}%
\pgfpathcurveto{\pgfqpoint{1.133576in}{1.945217in}}{\pgfqpoint{1.130303in}{1.937317in}}{\pgfqpoint{1.130303in}{1.929081in}}%
\pgfpathcurveto{\pgfqpoint{1.130303in}{1.920845in}}{\pgfqpoint{1.133576in}{1.912945in}}{\pgfqpoint{1.139400in}{1.907121in}}%
\pgfpathcurveto{\pgfqpoint{1.145223in}{1.901297in}}{\pgfqpoint{1.153124in}{1.898025in}}{\pgfqpoint{1.161360in}{1.898025in}}%
\pgfpathclose%
\pgfusepath{stroke,fill}%
\end{pgfscope}%
\begin{pgfscope}%
\pgfpathrectangle{\pgfqpoint{0.100000in}{0.212622in}}{\pgfqpoint{3.696000in}{3.696000in}}%
\pgfusepath{clip}%
\pgfsetbuttcap%
\pgfsetroundjoin%
\definecolor{currentfill}{rgb}{0.121569,0.466667,0.705882}%
\pgfsetfillcolor{currentfill}%
\pgfsetfillopacity{0.623948}%
\pgfsetlinewidth{1.003750pt}%
\definecolor{currentstroke}{rgb}{0.121569,0.466667,0.705882}%
\pgfsetstrokecolor{currentstroke}%
\pgfsetstrokeopacity{0.623948}%
\pgfsetdash{}{0pt}%
\pgfpathmoveto{\pgfqpoint{1.160996in}{1.897126in}}%
\pgfpathcurveto{\pgfqpoint{1.169232in}{1.897126in}}{\pgfqpoint{1.177132in}{1.900398in}}{\pgfqpoint{1.182956in}{1.906222in}}%
\pgfpathcurveto{\pgfqpoint{1.188780in}{1.912046in}}{\pgfqpoint{1.192052in}{1.919946in}}{\pgfqpoint{1.192052in}{1.928183in}}%
\pgfpathcurveto{\pgfqpoint{1.192052in}{1.936419in}}{\pgfqpoint{1.188780in}{1.944319in}}{\pgfqpoint{1.182956in}{1.950143in}}%
\pgfpathcurveto{\pgfqpoint{1.177132in}{1.955967in}}{\pgfqpoint{1.169232in}{1.959239in}}{\pgfqpoint{1.160996in}{1.959239in}}%
\pgfpathcurveto{\pgfqpoint{1.152760in}{1.959239in}}{\pgfqpoint{1.144859in}{1.955967in}}{\pgfqpoint{1.139036in}{1.950143in}}%
\pgfpathcurveto{\pgfqpoint{1.133212in}{1.944319in}}{\pgfqpoint{1.129939in}{1.936419in}}{\pgfqpoint{1.129939in}{1.928183in}}%
\pgfpathcurveto{\pgfqpoint{1.129939in}{1.919946in}}{\pgfqpoint{1.133212in}{1.912046in}}{\pgfqpoint{1.139036in}{1.906222in}}%
\pgfpathcurveto{\pgfqpoint{1.144859in}{1.900398in}}{\pgfqpoint{1.152760in}{1.897126in}}{\pgfqpoint{1.160996in}{1.897126in}}%
\pgfpathclose%
\pgfusepath{stroke,fill}%
\end{pgfscope}%
\begin{pgfscope}%
\pgfpathrectangle{\pgfqpoint{0.100000in}{0.212622in}}{\pgfqpoint{3.696000in}{3.696000in}}%
\pgfusepath{clip}%
\pgfsetbuttcap%
\pgfsetroundjoin%
\definecolor{currentfill}{rgb}{0.121569,0.466667,0.705882}%
\pgfsetfillcolor{currentfill}%
\pgfsetfillopacity{0.624055}%
\pgfsetlinewidth{1.003750pt}%
\definecolor{currentstroke}{rgb}{0.121569,0.466667,0.705882}%
\pgfsetstrokecolor{currentstroke}%
\pgfsetstrokeopacity{0.624055}%
\pgfsetdash{}{0pt}%
\pgfpathmoveto{\pgfqpoint{1.160803in}{1.896614in}}%
\pgfpathcurveto{\pgfqpoint{1.169040in}{1.896614in}}{\pgfqpoint{1.176940in}{1.899886in}}{\pgfqpoint{1.182764in}{1.905710in}}%
\pgfpathcurveto{\pgfqpoint{1.188588in}{1.911534in}}{\pgfqpoint{1.191860in}{1.919434in}}{\pgfqpoint{1.191860in}{1.927670in}}%
\pgfpathcurveto{\pgfqpoint{1.191860in}{1.935906in}}{\pgfqpoint{1.188588in}{1.943807in}}{\pgfqpoint{1.182764in}{1.949630in}}%
\pgfpathcurveto{\pgfqpoint{1.176940in}{1.955454in}}{\pgfqpoint{1.169040in}{1.958727in}}{\pgfqpoint{1.160803in}{1.958727in}}%
\pgfpathcurveto{\pgfqpoint{1.152567in}{1.958727in}}{\pgfqpoint{1.144667in}{1.955454in}}{\pgfqpoint{1.138843in}{1.949630in}}%
\pgfpathcurveto{\pgfqpoint{1.133019in}{1.943807in}}{\pgfqpoint{1.129747in}{1.935906in}}{\pgfqpoint{1.129747in}{1.927670in}}%
\pgfpathcurveto{\pgfqpoint{1.129747in}{1.919434in}}{\pgfqpoint{1.133019in}{1.911534in}}{\pgfqpoint{1.138843in}{1.905710in}}%
\pgfpathcurveto{\pgfqpoint{1.144667in}{1.899886in}}{\pgfqpoint{1.152567in}{1.896614in}}{\pgfqpoint{1.160803in}{1.896614in}}%
\pgfpathclose%
\pgfusepath{stroke,fill}%
\end{pgfscope}%
\begin{pgfscope}%
\pgfpathrectangle{\pgfqpoint{0.100000in}{0.212622in}}{\pgfqpoint{3.696000in}{3.696000in}}%
\pgfusepath{clip}%
\pgfsetbuttcap%
\pgfsetroundjoin%
\definecolor{currentfill}{rgb}{0.121569,0.466667,0.705882}%
\pgfsetfillcolor{currentfill}%
\pgfsetfillopacity{0.624115}%
\pgfsetlinewidth{1.003750pt}%
\definecolor{currentstroke}{rgb}{0.121569,0.466667,0.705882}%
\pgfsetstrokecolor{currentstroke}%
\pgfsetstrokeopacity{0.624115}%
\pgfsetdash{}{0pt}%
\pgfpathmoveto{\pgfqpoint{1.160690in}{1.896344in}}%
\pgfpathcurveto{\pgfqpoint{1.168926in}{1.896344in}}{\pgfqpoint{1.176827in}{1.899616in}}{\pgfqpoint{1.182650in}{1.905440in}}%
\pgfpathcurveto{\pgfqpoint{1.188474in}{1.911264in}}{\pgfqpoint{1.191747in}{1.919164in}}{\pgfqpoint{1.191747in}{1.927401in}}%
\pgfpathcurveto{\pgfqpoint{1.191747in}{1.935637in}}{\pgfqpoint{1.188474in}{1.943537in}}{\pgfqpoint{1.182650in}{1.949361in}}%
\pgfpathcurveto{\pgfqpoint{1.176827in}{1.955185in}}{\pgfqpoint{1.168926in}{1.958457in}}{\pgfqpoint{1.160690in}{1.958457in}}%
\pgfpathcurveto{\pgfqpoint{1.152454in}{1.958457in}}{\pgfqpoint{1.144554in}{1.955185in}}{\pgfqpoint{1.138730in}{1.949361in}}%
\pgfpathcurveto{\pgfqpoint{1.132906in}{1.943537in}}{\pgfqpoint{1.129634in}{1.935637in}}{\pgfqpoint{1.129634in}{1.927401in}}%
\pgfpathcurveto{\pgfqpoint{1.129634in}{1.919164in}}{\pgfqpoint{1.132906in}{1.911264in}}{\pgfqpoint{1.138730in}{1.905440in}}%
\pgfpathcurveto{\pgfqpoint{1.144554in}{1.899616in}}{\pgfqpoint{1.152454in}{1.896344in}}{\pgfqpoint{1.160690in}{1.896344in}}%
\pgfpathclose%
\pgfusepath{stroke,fill}%
\end{pgfscope}%
\begin{pgfscope}%
\pgfpathrectangle{\pgfqpoint{0.100000in}{0.212622in}}{\pgfqpoint{3.696000in}{3.696000in}}%
\pgfusepath{clip}%
\pgfsetbuttcap%
\pgfsetroundjoin%
\definecolor{currentfill}{rgb}{0.121569,0.466667,0.705882}%
\pgfsetfillcolor{currentfill}%
\pgfsetfillopacity{0.624146}%
\pgfsetlinewidth{1.003750pt}%
\definecolor{currentstroke}{rgb}{0.121569,0.466667,0.705882}%
\pgfsetstrokecolor{currentstroke}%
\pgfsetstrokeopacity{0.624146}%
\pgfsetdash{}{0pt}%
\pgfpathmoveto{\pgfqpoint{1.160627in}{1.896192in}}%
\pgfpathcurveto{\pgfqpoint{1.168864in}{1.896192in}}{\pgfqpoint{1.176764in}{1.899464in}}{\pgfqpoint{1.182588in}{1.905288in}}%
\pgfpathcurveto{\pgfqpoint{1.188412in}{1.911112in}}{\pgfqpoint{1.191684in}{1.919012in}}{\pgfqpoint{1.191684in}{1.927249in}}%
\pgfpathcurveto{\pgfqpoint{1.191684in}{1.935485in}}{\pgfqpoint{1.188412in}{1.943385in}}{\pgfqpoint{1.182588in}{1.949209in}}%
\pgfpathcurveto{\pgfqpoint{1.176764in}{1.955033in}}{\pgfqpoint{1.168864in}{1.958305in}}{\pgfqpoint{1.160627in}{1.958305in}}%
\pgfpathcurveto{\pgfqpoint{1.152391in}{1.958305in}}{\pgfqpoint{1.144491in}{1.955033in}}{\pgfqpoint{1.138667in}{1.949209in}}%
\pgfpathcurveto{\pgfqpoint{1.132843in}{1.943385in}}{\pgfqpoint{1.129571in}{1.935485in}}{\pgfqpoint{1.129571in}{1.927249in}}%
\pgfpathcurveto{\pgfqpoint{1.129571in}{1.919012in}}{\pgfqpoint{1.132843in}{1.911112in}}{\pgfqpoint{1.138667in}{1.905288in}}%
\pgfpathcurveto{\pgfqpoint{1.144491in}{1.899464in}}{\pgfqpoint{1.152391in}{1.896192in}}{\pgfqpoint{1.160627in}{1.896192in}}%
\pgfpathclose%
\pgfusepath{stroke,fill}%
\end{pgfscope}%
\begin{pgfscope}%
\pgfpathrectangle{\pgfqpoint{0.100000in}{0.212622in}}{\pgfqpoint{3.696000in}{3.696000in}}%
\pgfusepath{clip}%
\pgfsetbuttcap%
\pgfsetroundjoin%
\definecolor{currentfill}{rgb}{0.121569,0.466667,0.705882}%
\pgfsetfillcolor{currentfill}%
\pgfsetfillopacity{0.624164}%
\pgfsetlinewidth{1.003750pt}%
\definecolor{currentstroke}{rgb}{0.121569,0.466667,0.705882}%
\pgfsetstrokecolor{currentstroke}%
\pgfsetstrokeopacity{0.624164}%
\pgfsetdash{}{0pt}%
\pgfpathmoveto{\pgfqpoint{1.160594in}{1.896108in}}%
\pgfpathcurveto{\pgfqpoint{1.168830in}{1.896108in}}{\pgfqpoint{1.176730in}{1.899380in}}{\pgfqpoint{1.182554in}{1.905204in}}%
\pgfpathcurveto{\pgfqpoint{1.188378in}{1.911028in}}{\pgfqpoint{1.191650in}{1.918928in}}{\pgfqpoint{1.191650in}{1.927164in}}%
\pgfpathcurveto{\pgfqpoint{1.191650in}{1.935400in}}{\pgfqpoint{1.188378in}{1.943300in}}{\pgfqpoint{1.182554in}{1.949124in}}%
\pgfpathcurveto{\pgfqpoint{1.176730in}{1.954948in}}{\pgfqpoint{1.168830in}{1.958221in}}{\pgfqpoint{1.160594in}{1.958221in}}%
\pgfpathcurveto{\pgfqpoint{1.152358in}{1.958221in}}{\pgfqpoint{1.144457in}{1.954948in}}{\pgfqpoint{1.138634in}{1.949124in}}%
\pgfpathcurveto{\pgfqpoint{1.132810in}{1.943300in}}{\pgfqpoint{1.129537in}{1.935400in}}{\pgfqpoint{1.129537in}{1.927164in}}%
\pgfpathcurveto{\pgfqpoint{1.129537in}{1.918928in}}{\pgfqpoint{1.132810in}{1.911028in}}{\pgfqpoint{1.138634in}{1.905204in}}%
\pgfpathcurveto{\pgfqpoint{1.144457in}{1.899380in}}{\pgfqpoint{1.152358in}{1.896108in}}{\pgfqpoint{1.160594in}{1.896108in}}%
\pgfpathclose%
\pgfusepath{stroke,fill}%
\end{pgfscope}%
\begin{pgfscope}%
\pgfpathrectangle{\pgfqpoint{0.100000in}{0.212622in}}{\pgfqpoint{3.696000in}{3.696000in}}%
\pgfusepath{clip}%
\pgfsetbuttcap%
\pgfsetroundjoin%
\definecolor{currentfill}{rgb}{0.121569,0.466667,0.705882}%
\pgfsetfillcolor{currentfill}%
\pgfsetfillopacity{0.624174}%
\pgfsetlinewidth{1.003750pt}%
\definecolor{currentstroke}{rgb}{0.121569,0.466667,0.705882}%
\pgfsetstrokecolor{currentstroke}%
\pgfsetstrokeopacity{0.624174}%
\pgfsetdash{}{0pt}%
\pgfpathmoveto{\pgfqpoint{1.160575in}{1.896062in}}%
\pgfpathcurveto{\pgfqpoint{1.168811in}{1.896062in}}{\pgfqpoint{1.176711in}{1.899334in}}{\pgfqpoint{1.182535in}{1.905158in}}%
\pgfpathcurveto{\pgfqpoint{1.188359in}{1.910982in}}{\pgfqpoint{1.191631in}{1.918882in}}{\pgfqpoint{1.191631in}{1.927119in}}%
\pgfpathcurveto{\pgfqpoint{1.191631in}{1.935355in}}{\pgfqpoint{1.188359in}{1.943255in}}{\pgfqpoint{1.182535in}{1.949079in}}%
\pgfpathcurveto{\pgfqpoint{1.176711in}{1.954903in}}{\pgfqpoint{1.168811in}{1.958175in}}{\pgfqpoint{1.160575in}{1.958175in}}%
\pgfpathcurveto{\pgfqpoint{1.152338in}{1.958175in}}{\pgfqpoint{1.144438in}{1.954903in}}{\pgfqpoint{1.138614in}{1.949079in}}%
\pgfpathcurveto{\pgfqpoint{1.132790in}{1.943255in}}{\pgfqpoint{1.129518in}{1.935355in}}{\pgfqpoint{1.129518in}{1.927119in}}%
\pgfpathcurveto{\pgfqpoint{1.129518in}{1.918882in}}{\pgfqpoint{1.132790in}{1.910982in}}{\pgfqpoint{1.138614in}{1.905158in}}%
\pgfpathcurveto{\pgfqpoint{1.144438in}{1.899334in}}{\pgfqpoint{1.152338in}{1.896062in}}{\pgfqpoint{1.160575in}{1.896062in}}%
\pgfpathclose%
\pgfusepath{stroke,fill}%
\end{pgfscope}%
\begin{pgfscope}%
\pgfpathrectangle{\pgfqpoint{0.100000in}{0.212622in}}{\pgfqpoint{3.696000in}{3.696000in}}%
\pgfusepath{clip}%
\pgfsetbuttcap%
\pgfsetroundjoin%
\definecolor{currentfill}{rgb}{0.121569,0.466667,0.705882}%
\pgfsetfillcolor{currentfill}%
\pgfsetfillopacity{0.624179}%
\pgfsetlinewidth{1.003750pt}%
\definecolor{currentstroke}{rgb}{0.121569,0.466667,0.705882}%
\pgfsetstrokecolor{currentstroke}%
\pgfsetstrokeopacity{0.624179}%
\pgfsetdash{}{0pt}%
\pgfpathmoveto{\pgfqpoint{1.160564in}{1.896037in}}%
\pgfpathcurveto{\pgfqpoint{1.168800in}{1.896037in}}{\pgfqpoint{1.176701in}{1.899309in}}{\pgfqpoint{1.182524in}{1.905133in}}%
\pgfpathcurveto{\pgfqpoint{1.188348in}{1.910957in}}{\pgfqpoint{1.191621in}{1.918857in}}{\pgfqpoint{1.191621in}{1.927093in}}%
\pgfpathcurveto{\pgfqpoint{1.191621in}{1.935329in}}{\pgfqpoint{1.188348in}{1.943230in}}{\pgfqpoint{1.182524in}{1.949053in}}%
\pgfpathcurveto{\pgfqpoint{1.176701in}{1.954877in}}{\pgfqpoint{1.168800in}{1.958150in}}{\pgfqpoint{1.160564in}{1.958150in}}%
\pgfpathcurveto{\pgfqpoint{1.152328in}{1.958150in}}{\pgfqpoint{1.144428in}{1.954877in}}{\pgfqpoint{1.138604in}{1.949053in}}%
\pgfpathcurveto{\pgfqpoint{1.132780in}{1.943230in}}{\pgfqpoint{1.129508in}{1.935329in}}{\pgfqpoint{1.129508in}{1.927093in}}%
\pgfpathcurveto{\pgfqpoint{1.129508in}{1.918857in}}{\pgfqpoint{1.132780in}{1.910957in}}{\pgfqpoint{1.138604in}{1.905133in}}%
\pgfpathcurveto{\pgfqpoint{1.144428in}{1.899309in}}{\pgfqpoint{1.152328in}{1.896037in}}{\pgfqpoint{1.160564in}{1.896037in}}%
\pgfpathclose%
\pgfusepath{stroke,fill}%
\end{pgfscope}%
\begin{pgfscope}%
\pgfpathrectangle{\pgfqpoint{0.100000in}{0.212622in}}{\pgfqpoint{3.696000in}{3.696000in}}%
\pgfusepath{clip}%
\pgfsetbuttcap%
\pgfsetroundjoin%
\definecolor{currentfill}{rgb}{0.121569,0.466667,0.705882}%
\pgfsetfillcolor{currentfill}%
\pgfsetfillopacity{0.624182}%
\pgfsetlinewidth{1.003750pt}%
\definecolor{currentstroke}{rgb}{0.121569,0.466667,0.705882}%
\pgfsetstrokecolor{currentstroke}%
\pgfsetstrokeopacity{0.624182}%
\pgfsetdash{}{0pt}%
\pgfpathmoveto{\pgfqpoint{1.160559in}{1.896023in}}%
\pgfpathcurveto{\pgfqpoint{1.168795in}{1.896023in}}{\pgfqpoint{1.176695in}{1.899295in}}{\pgfqpoint{1.182519in}{1.905119in}}%
\pgfpathcurveto{\pgfqpoint{1.188343in}{1.910943in}}{\pgfqpoint{1.191615in}{1.918843in}}{\pgfqpoint{1.191615in}{1.927079in}}%
\pgfpathcurveto{\pgfqpoint{1.191615in}{1.935316in}}{\pgfqpoint{1.188343in}{1.943216in}}{\pgfqpoint{1.182519in}{1.949040in}}%
\pgfpathcurveto{\pgfqpoint{1.176695in}{1.954864in}}{\pgfqpoint{1.168795in}{1.958136in}}{\pgfqpoint{1.160559in}{1.958136in}}%
\pgfpathcurveto{\pgfqpoint{1.152322in}{1.958136in}}{\pgfqpoint{1.144422in}{1.954864in}}{\pgfqpoint{1.138598in}{1.949040in}}%
\pgfpathcurveto{\pgfqpoint{1.132774in}{1.943216in}}{\pgfqpoint{1.129502in}{1.935316in}}{\pgfqpoint{1.129502in}{1.927079in}}%
\pgfpathcurveto{\pgfqpoint{1.129502in}{1.918843in}}{\pgfqpoint{1.132774in}{1.910943in}}{\pgfqpoint{1.138598in}{1.905119in}}%
\pgfpathcurveto{\pgfqpoint{1.144422in}{1.899295in}}{\pgfqpoint{1.152322in}{1.896023in}}{\pgfqpoint{1.160559in}{1.896023in}}%
\pgfpathclose%
\pgfusepath{stroke,fill}%
\end{pgfscope}%
\begin{pgfscope}%
\pgfpathrectangle{\pgfqpoint{0.100000in}{0.212622in}}{\pgfqpoint{3.696000in}{3.696000in}}%
\pgfusepath{clip}%
\pgfsetbuttcap%
\pgfsetroundjoin%
\definecolor{currentfill}{rgb}{0.121569,0.466667,0.705882}%
\pgfsetfillcolor{currentfill}%
\pgfsetfillopacity{0.624550}%
\pgfsetlinewidth{1.003750pt}%
\definecolor{currentstroke}{rgb}{0.121569,0.466667,0.705882}%
\pgfsetstrokecolor{currentstroke}%
\pgfsetstrokeopacity{0.624550}%
\pgfsetdash{}{0pt}%
\pgfpathmoveto{\pgfqpoint{1.159766in}{1.894271in}}%
\pgfpathcurveto{\pgfqpoint{1.168003in}{1.894271in}}{\pgfqpoint{1.175903in}{1.897543in}}{\pgfqpoint{1.181727in}{1.903367in}}%
\pgfpathcurveto{\pgfqpoint{1.187550in}{1.909191in}}{\pgfqpoint{1.190823in}{1.917091in}}{\pgfqpoint{1.190823in}{1.925327in}}%
\pgfpathcurveto{\pgfqpoint{1.190823in}{1.933564in}}{\pgfqpoint{1.187550in}{1.941464in}}{\pgfqpoint{1.181727in}{1.947288in}}%
\pgfpathcurveto{\pgfqpoint{1.175903in}{1.953112in}}{\pgfqpoint{1.168003in}{1.956384in}}{\pgfqpoint{1.159766in}{1.956384in}}%
\pgfpathcurveto{\pgfqpoint{1.151530in}{1.956384in}}{\pgfqpoint{1.143630in}{1.953112in}}{\pgfqpoint{1.137806in}{1.947288in}}%
\pgfpathcurveto{\pgfqpoint{1.131982in}{1.941464in}}{\pgfqpoint{1.128710in}{1.933564in}}{\pgfqpoint{1.128710in}{1.925327in}}%
\pgfpathcurveto{\pgfqpoint{1.128710in}{1.917091in}}{\pgfqpoint{1.131982in}{1.909191in}}{\pgfqpoint{1.137806in}{1.903367in}}%
\pgfpathcurveto{\pgfqpoint{1.143630in}{1.897543in}}{\pgfqpoint{1.151530in}{1.894271in}}{\pgfqpoint{1.159766in}{1.894271in}}%
\pgfpathclose%
\pgfusepath{stroke,fill}%
\end{pgfscope}%
\begin{pgfscope}%
\pgfpathrectangle{\pgfqpoint{0.100000in}{0.212622in}}{\pgfqpoint{3.696000in}{3.696000in}}%
\pgfusepath{clip}%
\pgfsetbuttcap%
\pgfsetroundjoin%
\definecolor{currentfill}{rgb}{0.121569,0.466667,0.705882}%
\pgfsetfillcolor{currentfill}%
\pgfsetfillopacity{0.624753}%
\pgfsetlinewidth{1.003750pt}%
\definecolor{currentstroke}{rgb}{0.121569,0.466667,0.705882}%
\pgfsetstrokecolor{currentstroke}%
\pgfsetstrokeopacity{0.624753}%
\pgfsetdash{}{0pt}%
\pgfpathmoveto{\pgfqpoint{1.159339in}{1.893302in}}%
\pgfpathcurveto{\pgfqpoint{1.167575in}{1.893302in}}{\pgfqpoint{1.175475in}{1.896575in}}{\pgfqpoint{1.181299in}{1.902399in}}%
\pgfpathcurveto{\pgfqpoint{1.187123in}{1.908222in}}{\pgfqpoint{1.190395in}{1.916123in}}{\pgfqpoint{1.190395in}{1.924359in}}%
\pgfpathcurveto{\pgfqpoint{1.190395in}{1.932595in}}{\pgfqpoint{1.187123in}{1.940495in}}{\pgfqpoint{1.181299in}{1.946319in}}%
\pgfpathcurveto{\pgfqpoint{1.175475in}{1.952143in}}{\pgfqpoint{1.167575in}{1.955415in}}{\pgfqpoint{1.159339in}{1.955415in}}%
\pgfpathcurveto{\pgfqpoint{1.151103in}{1.955415in}}{\pgfqpoint{1.143203in}{1.952143in}}{\pgfqpoint{1.137379in}{1.946319in}}%
\pgfpathcurveto{\pgfqpoint{1.131555in}{1.940495in}}{\pgfqpoint{1.128282in}{1.932595in}}{\pgfqpoint{1.128282in}{1.924359in}}%
\pgfpathcurveto{\pgfqpoint{1.128282in}{1.916123in}}{\pgfqpoint{1.131555in}{1.908222in}}{\pgfqpoint{1.137379in}{1.902399in}}%
\pgfpathcurveto{\pgfqpoint{1.143203in}{1.896575in}}{\pgfqpoint{1.151103in}{1.893302in}}{\pgfqpoint{1.159339in}{1.893302in}}%
\pgfpathclose%
\pgfusepath{stroke,fill}%
\end{pgfscope}%
\begin{pgfscope}%
\pgfpathrectangle{\pgfqpoint{0.100000in}{0.212622in}}{\pgfqpoint{3.696000in}{3.696000in}}%
\pgfusepath{clip}%
\pgfsetbuttcap%
\pgfsetroundjoin%
\definecolor{currentfill}{rgb}{0.121569,0.466667,0.705882}%
\pgfsetfillcolor{currentfill}%
\pgfsetfillopacity{0.624861}%
\pgfsetlinewidth{1.003750pt}%
\definecolor{currentstroke}{rgb}{0.121569,0.466667,0.705882}%
\pgfsetstrokecolor{currentstroke}%
\pgfsetstrokeopacity{0.624861}%
\pgfsetdash{}{0pt}%
\pgfpathmoveto{\pgfqpoint{1.159091in}{1.892768in}}%
\pgfpathcurveto{\pgfqpoint{1.167327in}{1.892768in}}{\pgfqpoint{1.175227in}{1.896040in}}{\pgfqpoint{1.181051in}{1.901864in}}%
\pgfpathcurveto{\pgfqpoint{1.186875in}{1.907688in}}{\pgfqpoint{1.190147in}{1.915588in}}{\pgfqpoint{1.190147in}{1.923824in}}%
\pgfpathcurveto{\pgfqpoint{1.190147in}{1.932061in}}{\pgfqpoint{1.186875in}{1.939961in}}{\pgfqpoint{1.181051in}{1.945785in}}%
\pgfpathcurveto{\pgfqpoint{1.175227in}{1.951609in}}{\pgfqpoint{1.167327in}{1.954881in}}{\pgfqpoint{1.159091in}{1.954881in}}%
\pgfpathcurveto{\pgfqpoint{1.150854in}{1.954881in}}{\pgfqpoint{1.142954in}{1.951609in}}{\pgfqpoint{1.137130in}{1.945785in}}%
\pgfpathcurveto{\pgfqpoint{1.131306in}{1.939961in}}{\pgfqpoint{1.128034in}{1.932061in}}{\pgfqpoint{1.128034in}{1.923824in}}%
\pgfpathcurveto{\pgfqpoint{1.128034in}{1.915588in}}{\pgfqpoint{1.131306in}{1.907688in}}{\pgfqpoint{1.137130in}{1.901864in}}%
\pgfpathcurveto{\pgfqpoint{1.142954in}{1.896040in}}{\pgfqpoint{1.150854in}{1.892768in}}{\pgfqpoint{1.159091in}{1.892768in}}%
\pgfpathclose%
\pgfusepath{stroke,fill}%
\end{pgfscope}%
\begin{pgfscope}%
\pgfpathrectangle{\pgfqpoint{0.100000in}{0.212622in}}{\pgfqpoint{3.696000in}{3.696000in}}%
\pgfusepath{clip}%
\pgfsetbuttcap%
\pgfsetroundjoin%
\definecolor{currentfill}{rgb}{0.121569,0.466667,0.705882}%
\pgfsetfillcolor{currentfill}%
\pgfsetfillopacity{0.624868}%
\pgfsetlinewidth{1.003750pt}%
\definecolor{currentstroke}{rgb}{0.121569,0.466667,0.705882}%
\pgfsetstrokecolor{currentstroke}%
\pgfsetstrokeopacity{0.624868}%
\pgfsetdash{}{0pt}%
\pgfpathmoveto{\pgfqpoint{1.148380in}{1.889950in}}%
\pgfpathcurveto{\pgfqpoint{1.156616in}{1.889950in}}{\pgfqpoint{1.164516in}{1.893222in}}{\pgfqpoint{1.170340in}{1.899046in}}%
\pgfpathcurveto{\pgfqpoint{1.176164in}{1.904870in}}{\pgfqpoint{1.179437in}{1.912770in}}{\pgfqpoint{1.179437in}{1.921006in}}%
\pgfpathcurveto{\pgfqpoint{1.179437in}{1.929242in}}{\pgfqpoint{1.176164in}{1.937142in}}{\pgfqpoint{1.170340in}{1.942966in}}%
\pgfpathcurveto{\pgfqpoint{1.164516in}{1.948790in}}{\pgfqpoint{1.156616in}{1.952063in}}{\pgfqpoint{1.148380in}{1.952063in}}%
\pgfpathcurveto{\pgfqpoint{1.140144in}{1.952063in}}{\pgfqpoint{1.132244in}{1.948790in}}{\pgfqpoint{1.126420in}{1.942966in}}%
\pgfpathcurveto{\pgfqpoint{1.120596in}{1.937142in}}{\pgfqpoint{1.117324in}{1.929242in}}{\pgfqpoint{1.117324in}{1.921006in}}%
\pgfpathcurveto{\pgfqpoint{1.117324in}{1.912770in}}{\pgfqpoint{1.120596in}{1.904870in}}{\pgfqpoint{1.126420in}{1.899046in}}%
\pgfpathcurveto{\pgfqpoint{1.132244in}{1.893222in}}{\pgfqpoint{1.140144in}{1.889950in}}{\pgfqpoint{1.148380in}{1.889950in}}%
\pgfpathclose%
\pgfusepath{stroke,fill}%
\end{pgfscope}%
\begin{pgfscope}%
\pgfpathrectangle{\pgfqpoint{0.100000in}{0.212622in}}{\pgfqpoint{3.696000in}{3.696000in}}%
\pgfusepath{clip}%
\pgfsetbuttcap%
\pgfsetroundjoin%
\definecolor{currentfill}{rgb}{0.121569,0.466667,0.705882}%
\pgfsetfillcolor{currentfill}%
\pgfsetfillopacity{0.624918}%
\pgfsetlinewidth{1.003750pt}%
\definecolor{currentstroke}{rgb}{0.121569,0.466667,0.705882}%
\pgfsetstrokecolor{currentstroke}%
\pgfsetstrokeopacity{0.624918}%
\pgfsetdash{}{0pt}%
\pgfpathmoveto{\pgfqpoint{1.158940in}{1.892485in}}%
\pgfpathcurveto{\pgfqpoint{1.167176in}{1.892485in}}{\pgfqpoint{1.175076in}{1.895757in}}{\pgfqpoint{1.180900in}{1.901581in}}%
\pgfpathcurveto{\pgfqpoint{1.186724in}{1.907405in}}{\pgfqpoint{1.189996in}{1.915305in}}{\pgfqpoint{1.189996in}{1.923541in}}%
\pgfpathcurveto{\pgfqpoint{1.189996in}{1.931778in}}{\pgfqpoint{1.186724in}{1.939678in}}{\pgfqpoint{1.180900in}{1.945502in}}%
\pgfpathcurveto{\pgfqpoint{1.175076in}{1.951326in}}{\pgfqpoint{1.167176in}{1.954598in}}{\pgfqpoint{1.158940in}{1.954598in}}%
\pgfpathcurveto{\pgfqpoint{1.150703in}{1.954598in}}{\pgfqpoint{1.142803in}{1.951326in}}{\pgfqpoint{1.136979in}{1.945502in}}%
\pgfpathcurveto{\pgfqpoint{1.131155in}{1.939678in}}{\pgfqpoint{1.127883in}{1.931778in}}{\pgfqpoint{1.127883in}{1.923541in}}%
\pgfpathcurveto{\pgfqpoint{1.127883in}{1.915305in}}{\pgfqpoint{1.131155in}{1.907405in}}{\pgfqpoint{1.136979in}{1.901581in}}%
\pgfpathcurveto{\pgfqpoint{1.142803in}{1.895757in}}{\pgfqpoint{1.150703in}{1.892485in}}{\pgfqpoint{1.158940in}{1.892485in}}%
\pgfpathclose%
\pgfusepath{stroke,fill}%
\end{pgfscope}%
\begin{pgfscope}%
\pgfpathrectangle{\pgfqpoint{0.100000in}{0.212622in}}{\pgfqpoint{3.696000in}{3.696000in}}%
\pgfusepath{clip}%
\pgfsetbuttcap%
\pgfsetroundjoin%
\definecolor{currentfill}{rgb}{0.121569,0.466667,0.705882}%
\pgfsetfillcolor{currentfill}%
\pgfsetfillopacity{0.624949}%
\pgfsetlinewidth{1.003750pt}%
\definecolor{currentstroke}{rgb}{0.121569,0.466667,0.705882}%
\pgfsetstrokecolor{currentstroke}%
\pgfsetstrokeopacity{0.624949}%
\pgfsetdash{}{0pt}%
\pgfpathmoveto{\pgfqpoint{1.158856in}{1.892327in}}%
\pgfpathcurveto{\pgfqpoint{1.167092in}{1.892327in}}{\pgfqpoint{1.174992in}{1.895599in}}{\pgfqpoint{1.180816in}{1.901423in}}%
\pgfpathcurveto{\pgfqpoint{1.186640in}{1.907247in}}{\pgfqpoint{1.189912in}{1.915147in}}{\pgfqpoint{1.189912in}{1.923383in}}%
\pgfpathcurveto{\pgfqpoint{1.189912in}{1.931619in}}{\pgfqpoint{1.186640in}{1.939519in}}{\pgfqpoint{1.180816in}{1.945343in}}%
\pgfpathcurveto{\pgfqpoint{1.174992in}{1.951167in}}{\pgfqpoint{1.167092in}{1.954440in}}{\pgfqpoint{1.158856in}{1.954440in}}%
\pgfpathcurveto{\pgfqpoint{1.150619in}{1.954440in}}{\pgfqpoint{1.142719in}{1.951167in}}{\pgfqpoint{1.136895in}{1.945343in}}%
\pgfpathcurveto{\pgfqpoint{1.131071in}{1.939519in}}{\pgfqpoint{1.127799in}{1.931619in}}{\pgfqpoint{1.127799in}{1.923383in}}%
\pgfpathcurveto{\pgfqpoint{1.127799in}{1.915147in}}{\pgfqpoint{1.131071in}{1.907247in}}{\pgfqpoint{1.136895in}{1.901423in}}%
\pgfpathcurveto{\pgfqpoint{1.142719in}{1.895599in}}{\pgfqpoint{1.150619in}{1.892327in}}{\pgfqpoint{1.158856in}{1.892327in}}%
\pgfpathclose%
\pgfusepath{stroke,fill}%
\end{pgfscope}%
\begin{pgfscope}%
\pgfpathrectangle{\pgfqpoint{0.100000in}{0.212622in}}{\pgfqpoint{3.696000in}{3.696000in}}%
\pgfusepath{clip}%
\pgfsetbuttcap%
\pgfsetroundjoin%
\definecolor{currentfill}{rgb}{0.121569,0.466667,0.705882}%
\pgfsetfillcolor{currentfill}%
\pgfsetfillopacity{0.624965}%
\pgfsetlinewidth{1.003750pt}%
\definecolor{currentstroke}{rgb}{0.121569,0.466667,0.705882}%
\pgfsetstrokecolor{currentstroke}%
\pgfsetstrokeopacity{0.624965}%
\pgfsetdash{}{0pt}%
\pgfpathmoveto{\pgfqpoint{1.158802in}{1.892244in}}%
\pgfpathcurveto{\pgfqpoint{1.167038in}{1.892244in}}{\pgfqpoint{1.174938in}{1.895517in}}{\pgfqpoint{1.180762in}{1.901341in}}%
\pgfpathcurveto{\pgfqpoint{1.186586in}{1.907164in}}{\pgfqpoint{1.189858in}{1.915065in}}{\pgfqpoint{1.189858in}{1.923301in}}%
\pgfpathcurveto{\pgfqpoint{1.189858in}{1.931537in}}{\pgfqpoint{1.186586in}{1.939437in}}{\pgfqpoint{1.180762in}{1.945261in}}%
\pgfpathcurveto{\pgfqpoint{1.174938in}{1.951085in}}{\pgfqpoint{1.167038in}{1.954357in}}{\pgfqpoint{1.158802in}{1.954357in}}%
\pgfpathcurveto{\pgfqpoint{1.150566in}{1.954357in}}{\pgfqpoint{1.142666in}{1.951085in}}{\pgfqpoint{1.136842in}{1.945261in}}%
\pgfpathcurveto{\pgfqpoint{1.131018in}{1.939437in}}{\pgfqpoint{1.127745in}{1.931537in}}{\pgfqpoint{1.127745in}{1.923301in}}%
\pgfpathcurveto{\pgfqpoint{1.127745in}{1.915065in}}{\pgfqpoint{1.131018in}{1.907164in}}{\pgfqpoint{1.136842in}{1.901341in}}%
\pgfpathcurveto{\pgfqpoint{1.142666in}{1.895517in}}{\pgfqpoint{1.150566in}{1.892244in}}{\pgfqpoint{1.158802in}{1.892244in}}%
\pgfpathclose%
\pgfusepath{stroke,fill}%
\end{pgfscope}%
\begin{pgfscope}%
\pgfpathrectangle{\pgfqpoint{0.100000in}{0.212622in}}{\pgfqpoint{3.696000in}{3.696000in}}%
\pgfusepath{clip}%
\pgfsetbuttcap%
\pgfsetroundjoin%
\definecolor{currentfill}{rgb}{0.121569,0.466667,0.705882}%
\pgfsetfillcolor{currentfill}%
\pgfsetfillopacity{0.624972}%
\pgfsetlinewidth{1.003750pt}%
\definecolor{currentstroke}{rgb}{0.121569,0.466667,0.705882}%
\pgfsetstrokecolor{currentstroke}%
\pgfsetstrokeopacity{0.624972}%
\pgfsetdash{}{0pt}%
\pgfpathmoveto{\pgfqpoint{1.158767in}{1.892202in}}%
\pgfpathcurveto{\pgfqpoint{1.167003in}{1.892202in}}{\pgfqpoint{1.174903in}{1.895474in}}{\pgfqpoint{1.180727in}{1.901298in}}%
\pgfpathcurveto{\pgfqpoint{1.186551in}{1.907122in}}{\pgfqpoint{1.189823in}{1.915022in}}{\pgfqpoint{1.189823in}{1.923258in}}%
\pgfpathcurveto{\pgfqpoint{1.189823in}{1.931495in}}{\pgfqpoint{1.186551in}{1.939395in}}{\pgfqpoint{1.180727in}{1.945219in}}%
\pgfpathcurveto{\pgfqpoint{1.174903in}{1.951043in}}{\pgfqpoint{1.167003in}{1.954315in}}{\pgfqpoint{1.158767in}{1.954315in}}%
\pgfpathcurveto{\pgfqpoint{1.150530in}{1.954315in}}{\pgfqpoint{1.142630in}{1.951043in}}{\pgfqpoint{1.136806in}{1.945219in}}%
\pgfpathcurveto{\pgfqpoint{1.130983in}{1.939395in}}{\pgfqpoint{1.127710in}{1.931495in}}{\pgfqpoint{1.127710in}{1.923258in}}%
\pgfpathcurveto{\pgfqpoint{1.127710in}{1.915022in}}{\pgfqpoint{1.130983in}{1.907122in}}{\pgfqpoint{1.136806in}{1.901298in}}%
\pgfpathcurveto{\pgfqpoint{1.142630in}{1.895474in}}{\pgfqpoint{1.150530in}{1.892202in}}{\pgfqpoint{1.158767in}{1.892202in}}%
\pgfpathclose%
\pgfusepath{stroke,fill}%
\end{pgfscope}%
\begin{pgfscope}%
\pgfpathrectangle{\pgfqpoint{0.100000in}{0.212622in}}{\pgfqpoint{3.696000in}{3.696000in}}%
\pgfusepath{clip}%
\pgfsetbuttcap%
\pgfsetroundjoin%
\definecolor{currentfill}{rgb}{0.121569,0.466667,0.705882}%
\pgfsetfillcolor{currentfill}%
\pgfsetfillopacity{0.624975}%
\pgfsetlinewidth{1.003750pt}%
\definecolor{currentstroke}{rgb}{0.121569,0.466667,0.705882}%
\pgfsetstrokecolor{currentstroke}%
\pgfsetstrokeopacity{0.624975}%
\pgfsetdash{}{0pt}%
\pgfpathmoveto{\pgfqpoint{1.158741in}{1.892185in}}%
\pgfpathcurveto{\pgfqpoint{1.166978in}{1.892185in}}{\pgfqpoint{1.174878in}{1.895458in}}{\pgfqpoint{1.180702in}{1.901282in}}%
\pgfpathcurveto{\pgfqpoint{1.186525in}{1.907105in}}{\pgfqpoint{1.189798in}{1.915006in}}{\pgfqpoint{1.189798in}{1.923242in}}%
\pgfpathcurveto{\pgfqpoint{1.189798in}{1.931478in}}{\pgfqpoint{1.186525in}{1.939378in}}{\pgfqpoint{1.180702in}{1.945202in}}%
\pgfpathcurveto{\pgfqpoint{1.174878in}{1.951026in}}{\pgfqpoint{1.166978in}{1.954298in}}{\pgfqpoint{1.158741in}{1.954298in}}%
\pgfpathcurveto{\pgfqpoint{1.150505in}{1.954298in}}{\pgfqpoint{1.142605in}{1.951026in}}{\pgfqpoint{1.136781in}{1.945202in}}%
\pgfpathcurveto{\pgfqpoint{1.130957in}{1.939378in}}{\pgfqpoint{1.127685in}{1.931478in}}{\pgfqpoint{1.127685in}{1.923242in}}%
\pgfpathcurveto{\pgfqpoint{1.127685in}{1.915006in}}{\pgfqpoint{1.130957in}{1.907105in}}{\pgfqpoint{1.136781in}{1.901282in}}%
\pgfpathcurveto{\pgfqpoint{1.142605in}{1.895458in}}{\pgfqpoint{1.150505in}{1.892185in}}{\pgfqpoint{1.158741in}{1.892185in}}%
\pgfpathclose%
\pgfusepath{stroke,fill}%
\end{pgfscope}%
\begin{pgfscope}%
\pgfpathrectangle{\pgfqpoint{0.100000in}{0.212622in}}{\pgfqpoint{3.696000in}{3.696000in}}%
\pgfusepath{clip}%
\pgfsetbuttcap%
\pgfsetroundjoin%
\definecolor{currentfill}{rgb}{0.121569,0.466667,0.705882}%
\pgfsetfillcolor{currentfill}%
\pgfsetfillopacity{0.625148}%
\pgfsetlinewidth{1.003750pt}%
\definecolor{currentstroke}{rgb}{0.121569,0.466667,0.705882}%
\pgfsetstrokecolor{currentstroke}%
\pgfsetstrokeopacity{0.625148}%
\pgfsetdash{}{0pt}%
\pgfpathmoveto{\pgfqpoint{1.155170in}{1.890723in}}%
\pgfpathcurveto{\pgfqpoint{1.163406in}{1.890723in}}{\pgfqpoint{1.171306in}{1.893995in}}{\pgfqpoint{1.177130in}{1.899819in}}%
\pgfpathcurveto{\pgfqpoint{1.182954in}{1.905643in}}{\pgfqpoint{1.186226in}{1.913543in}}{\pgfqpoint{1.186226in}{1.921779in}}%
\pgfpathcurveto{\pgfqpoint{1.186226in}{1.930016in}}{\pgfqpoint{1.182954in}{1.937916in}}{\pgfqpoint{1.177130in}{1.943739in}}%
\pgfpathcurveto{\pgfqpoint{1.171306in}{1.949563in}}{\pgfqpoint{1.163406in}{1.952836in}}{\pgfqpoint{1.155170in}{1.952836in}}%
\pgfpathcurveto{\pgfqpoint{1.146933in}{1.952836in}}{\pgfqpoint{1.139033in}{1.949563in}}{\pgfqpoint{1.133209in}{1.943739in}}%
\pgfpathcurveto{\pgfqpoint{1.127385in}{1.937916in}}{\pgfqpoint{1.124113in}{1.930016in}}{\pgfqpoint{1.124113in}{1.921779in}}%
\pgfpathcurveto{\pgfqpoint{1.124113in}{1.913543in}}{\pgfqpoint{1.127385in}{1.905643in}}{\pgfqpoint{1.133209in}{1.899819in}}%
\pgfpathcurveto{\pgfqpoint{1.139033in}{1.893995in}}{\pgfqpoint{1.146933in}{1.890723in}}{\pgfqpoint{1.155170in}{1.890723in}}%
\pgfpathclose%
\pgfusepath{stroke,fill}%
\end{pgfscope}%
\begin{pgfscope}%
\pgfpathrectangle{\pgfqpoint{0.100000in}{0.212622in}}{\pgfqpoint{3.696000in}{3.696000in}}%
\pgfusepath{clip}%
\pgfsetbuttcap%
\pgfsetroundjoin%
\definecolor{currentfill}{rgb}{0.121569,0.466667,0.705882}%
\pgfsetfillcolor{currentfill}%
\pgfsetfillopacity{0.625167}%
\pgfsetlinewidth{1.003750pt}%
\definecolor{currentstroke}{rgb}{0.121569,0.466667,0.705882}%
\pgfsetstrokecolor{currentstroke}%
\pgfsetstrokeopacity{0.625167}%
\pgfsetdash{}{0pt}%
\pgfpathmoveto{\pgfqpoint{1.153108in}{1.889976in}}%
\pgfpathcurveto{\pgfqpoint{1.161345in}{1.889976in}}{\pgfqpoint{1.169245in}{1.893248in}}{\pgfqpoint{1.175069in}{1.899072in}}%
\pgfpathcurveto{\pgfqpoint{1.180892in}{1.904896in}}{\pgfqpoint{1.184165in}{1.912796in}}{\pgfqpoint{1.184165in}{1.921032in}}%
\pgfpathcurveto{\pgfqpoint{1.184165in}{1.929268in}}{\pgfqpoint{1.180892in}{1.937169in}}{\pgfqpoint{1.175069in}{1.942992in}}%
\pgfpathcurveto{\pgfqpoint{1.169245in}{1.948816in}}{\pgfqpoint{1.161345in}{1.952089in}}{\pgfqpoint{1.153108in}{1.952089in}}%
\pgfpathcurveto{\pgfqpoint{1.144872in}{1.952089in}}{\pgfqpoint{1.136972in}{1.948816in}}{\pgfqpoint{1.131148in}{1.942992in}}%
\pgfpathcurveto{\pgfqpoint{1.125324in}{1.937169in}}{\pgfqpoint{1.122052in}{1.929268in}}{\pgfqpoint{1.122052in}{1.921032in}}%
\pgfpathcurveto{\pgfqpoint{1.122052in}{1.912796in}}{\pgfqpoint{1.125324in}{1.904896in}}{\pgfqpoint{1.131148in}{1.899072in}}%
\pgfpathcurveto{\pgfqpoint{1.136972in}{1.893248in}}{\pgfqpoint{1.144872in}{1.889976in}}{\pgfqpoint{1.153108in}{1.889976in}}%
\pgfpathclose%
\pgfusepath{stroke,fill}%
\end{pgfscope}%
\begin{pgfscope}%
\pgfpathrectangle{\pgfqpoint{0.100000in}{0.212622in}}{\pgfqpoint{3.696000in}{3.696000in}}%
\pgfusepath{clip}%
\pgfsetbuttcap%
\pgfsetroundjoin%
\definecolor{currentfill}{rgb}{0.121569,0.466667,0.705882}%
\pgfsetfillcolor{currentfill}%
\pgfsetfillopacity{0.625429}%
\pgfsetlinewidth{1.003750pt}%
\definecolor{currentstroke}{rgb}{0.121569,0.466667,0.705882}%
\pgfsetstrokecolor{currentstroke}%
\pgfsetstrokeopacity{0.625429}%
\pgfsetdash{}{0pt}%
\pgfpathmoveto{\pgfqpoint{0.650934in}{1.205261in}}%
\pgfpathcurveto{\pgfqpoint{0.659170in}{1.205261in}}{\pgfqpoint{0.667070in}{1.208533in}}{\pgfqpoint{0.672894in}{1.214357in}}%
\pgfpathcurveto{\pgfqpoint{0.678718in}{1.220181in}}{\pgfqpoint{0.681990in}{1.228081in}}{\pgfqpoint{0.681990in}{1.236317in}}%
\pgfpathcurveto{\pgfqpoint{0.681990in}{1.244554in}}{\pgfqpoint{0.678718in}{1.252454in}}{\pgfqpoint{0.672894in}{1.258278in}}%
\pgfpathcurveto{\pgfqpoint{0.667070in}{1.264101in}}{\pgfqpoint{0.659170in}{1.267374in}}{\pgfqpoint{0.650934in}{1.267374in}}%
\pgfpathcurveto{\pgfqpoint{0.642698in}{1.267374in}}{\pgfqpoint{0.634798in}{1.264101in}}{\pgfqpoint{0.628974in}{1.258278in}}%
\pgfpathcurveto{\pgfqpoint{0.623150in}{1.252454in}}{\pgfqpoint{0.619877in}{1.244554in}}{\pgfqpoint{0.619877in}{1.236317in}}%
\pgfpathcurveto{\pgfqpoint{0.619877in}{1.228081in}}{\pgfqpoint{0.623150in}{1.220181in}}{\pgfqpoint{0.628974in}{1.214357in}}%
\pgfpathcurveto{\pgfqpoint{0.634798in}{1.208533in}}{\pgfqpoint{0.642698in}{1.205261in}}{\pgfqpoint{0.650934in}{1.205261in}}%
\pgfpathclose%
\pgfusepath{stroke,fill}%
\end{pgfscope}%
\begin{pgfscope}%
\pgfpathrectangle{\pgfqpoint{0.100000in}{0.212622in}}{\pgfqpoint{3.696000in}{3.696000in}}%
\pgfusepath{clip}%
\pgfsetbuttcap%
\pgfsetroundjoin%
\definecolor{currentfill}{rgb}{0.121569,0.466667,0.705882}%
\pgfsetfillcolor{currentfill}%
\pgfsetfillopacity{0.625529}%
\pgfsetlinewidth{1.003750pt}%
\definecolor{currentstroke}{rgb}{0.121569,0.466667,0.705882}%
\pgfsetstrokecolor{currentstroke}%
\pgfsetstrokeopacity{0.625529}%
\pgfsetdash{}{0pt}%
\pgfpathmoveto{\pgfqpoint{1.144439in}{1.910793in}}%
\pgfpathcurveto{\pgfqpoint{1.152676in}{1.910793in}}{\pgfqpoint{1.160576in}{1.914065in}}{\pgfqpoint{1.166400in}{1.919889in}}%
\pgfpathcurveto{\pgfqpoint{1.172224in}{1.925713in}}{\pgfqpoint{1.175496in}{1.933613in}}{\pgfqpoint{1.175496in}{1.941850in}}%
\pgfpathcurveto{\pgfqpoint{1.175496in}{1.950086in}}{\pgfqpoint{1.172224in}{1.957986in}}{\pgfqpoint{1.166400in}{1.963810in}}%
\pgfpathcurveto{\pgfqpoint{1.160576in}{1.969634in}}{\pgfqpoint{1.152676in}{1.972906in}}{\pgfqpoint{1.144439in}{1.972906in}}%
\pgfpathcurveto{\pgfqpoint{1.136203in}{1.972906in}}{\pgfqpoint{1.128303in}{1.969634in}}{\pgfqpoint{1.122479in}{1.963810in}}%
\pgfpathcurveto{\pgfqpoint{1.116655in}{1.957986in}}{\pgfqpoint{1.113383in}{1.950086in}}{\pgfqpoint{1.113383in}{1.941850in}}%
\pgfpathcurveto{\pgfqpoint{1.113383in}{1.933613in}}{\pgfqpoint{1.116655in}{1.925713in}}{\pgfqpoint{1.122479in}{1.919889in}}%
\pgfpathcurveto{\pgfqpoint{1.128303in}{1.914065in}}{\pgfqpoint{1.136203in}{1.910793in}}{\pgfqpoint{1.144439in}{1.910793in}}%
\pgfpathclose%
\pgfusepath{stroke,fill}%
\end{pgfscope}%
\begin{pgfscope}%
\pgfpathrectangle{\pgfqpoint{0.100000in}{0.212622in}}{\pgfqpoint{3.696000in}{3.696000in}}%
\pgfusepath{clip}%
\pgfsetbuttcap%
\pgfsetroundjoin%
\definecolor{currentfill}{rgb}{0.121569,0.466667,0.705882}%
\pgfsetfillcolor{currentfill}%
\pgfsetfillopacity{0.628491}%
\pgfsetlinewidth{1.003750pt}%
\definecolor{currentstroke}{rgb}{0.121569,0.466667,0.705882}%
\pgfsetstrokecolor{currentstroke}%
\pgfsetstrokeopacity{0.628491}%
\pgfsetdash{}{0pt}%
\pgfpathmoveto{\pgfqpoint{0.662515in}{1.202838in}}%
\pgfpathcurveto{\pgfqpoint{0.670751in}{1.202838in}}{\pgfqpoint{0.678651in}{1.206110in}}{\pgfqpoint{0.684475in}{1.211934in}}%
\pgfpathcurveto{\pgfqpoint{0.690299in}{1.217758in}}{\pgfqpoint{0.693571in}{1.225658in}}{\pgfqpoint{0.693571in}{1.233894in}}%
\pgfpathcurveto{\pgfqpoint{0.693571in}{1.242131in}}{\pgfqpoint{0.690299in}{1.250031in}}{\pgfqpoint{0.684475in}{1.255855in}}%
\pgfpathcurveto{\pgfqpoint{0.678651in}{1.261678in}}{\pgfqpoint{0.670751in}{1.264951in}}{\pgfqpoint{0.662515in}{1.264951in}}%
\pgfpathcurveto{\pgfqpoint{0.654279in}{1.264951in}}{\pgfqpoint{0.646379in}{1.261678in}}{\pgfqpoint{0.640555in}{1.255855in}}%
\pgfpathcurveto{\pgfqpoint{0.634731in}{1.250031in}}{\pgfqpoint{0.631458in}{1.242131in}}{\pgfqpoint{0.631458in}{1.233894in}}%
\pgfpathcurveto{\pgfqpoint{0.631458in}{1.225658in}}{\pgfqpoint{0.634731in}{1.217758in}}{\pgfqpoint{0.640555in}{1.211934in}}%
\pgfpathcurveto{\pgfqpoint{0.646379in}{1.206110in}}{\pgfqpoint{0.654279in}{1.202838in}}{\pgfqpoint{0.662515in}{1.202838in}}%
\pgfpathclose%
\pgfusepath{stroke,fill}%
\end{pgfscope}%
\begin{pgfscope}%
\pgfpathrectangle{\pgfqpoint{0.100000in}{0.212622in}}{\pgfqpoint{3.696000in}{3.696000in}}%
\pgfusepath{clip}%
\pgfsetbuttcap%
\pgfsetroundjoin%
\definecolor{currentfill}{rgb}{0.121569,0.466667,0.705882}%
\pgfsetfillcolor{currentfill}%
\pgfsetfillopacity{0.630139}%
\pgfsetlinewidth{1.003750pt}%
\definecolor{currentstroke}{rgb}{0.121569,0.466667,0.705882}%
\pgfsetstrokecolor{currentstroke}%
\pgfsetstrokeopacity{0.630139}%
\pgfsetdash{}{0pt}%
\pgfpathmoveto{\pgfqpoint{1.131927in}{1.916594in}}%
\pgfpathcurveto{\pgfqpoint{1.140163in}{1.916594in}}{\pgfqpoint{1.148063in}{1.919867in}}{\pgfqpoint{1.153887in}{1.925691in}}%
\pgfpathcurveto{\pgfqpoint{1.159711in}{1.931514in}}{\pgfqpoint{1.162984in}{1.939415in}}{\pgfqpoint{1.162984in}{1.947651in}}%
\pgfpathcurveto{\pgfqpoint{1.162984in}{1.955887in}}{\pgfqpoint{1.159711in}{1.963787in}}{\pgfqpoint{1.153887in}{1.969611in}}%
\pgfpathcurveto{\pgfqpoint{1.148063in}{1.975435in}}{\pgfqpoint{1.140163in}{1.978707in}}{\pgfqpoint{1.131927in}{1.978707in}}%
\pgfpathcurveto{\pgfqpoint{1.123691in}{1.978707in}}{\pgfqpoint{1.115791in}{1.975435in}}{\pgfqpoint{1.109967in}{1.969611in}}%
\pgfpathcurveto{\pgfqpoint{1.104143in}{1.963787in}}{\pgfqpoint{1.100871in}{1.955887in}}{\pgfqpoint{1.100871in}{1.947651in}}%
\pgfpathcurveto{\pgfqpoint{1.100871in}{1.939415in}}{\pgfqpoint{1.104143in}{1.931514in}}{\pgfqpoint{1.109967in}{1.925691in}}%
\pgfpathcurveto{\pgfqpoint{1.115791in}{1.919867in}}{\pgfqpoint{1.123691in}{1.916594in}}{\pgfqpoint{1.131927in}{1.916594in}}%
\pgfpathclose%
\pgfusepath{stroke,fill}%
\end{pgfscope}%
\begin{pgfscope}%
\pgfpathrectangle{\pgfqpoint{0.100000in}{0.212622in}}{\pgfqpoint{3.696000in}{3.696000in}}%
\pgfusepath{clip}%
\pgfsetbuttcap%
\pgfsetroundjoin%
\definecolor{currentfill}{rgb}{0.121569,0.466667,0.705882}%
\pgfsetfillcolor{currentfill}%
\pgfsetfillopacity{0.630914}%
\pgfsetlinewidth{1.003750pt}%
\definecolor{currentstroke}{rgb}{0.121569,0.466667,0.705882}%
\pgfsetstrokecolor{currentstroke}%
\pgfsetstrokeopacity{0.630914}%
\pgfsetdash{}{0pt}%
\pgfpathmoveto{\pgfqpoint{2.201368in}{2.218311in}}%
\pgfpathcurveto{\pgfqpoint{2.209604in}{2.218311in}}{\pgfqpoint{2.217504in}{2.221583in}}{\pgfqpoint{2.223328in}{2.227407in}}%
\pgfpathcurveto{\pgfqpoint{2.229152in}{2.233231in}}{\pgfqpoint{2.232424in}{2.241131in}}{\pgfqpoint{2.232424in}{2.249367in}}%
\pgfpathcurveto{\pgfqpoint{2.232424in}{2.257603in}}{\pgfqpoint{2.229152in}{2.265504in}}{\pgfqpoint{2.223328in}{2.271327in}}%
\pgfpathcurveto{\pgfqpoint{2.217504in}{2.277151in}}{\pgfqpoint{2.209604in}{2.280424in}}{\pgfqpoint{2.201368in}{2.280424in}}%
\pgfpathcurveto{\pgfqpoint{2.193131in}{2.280424in}}{\pgfqpoint{2.185231in}{2.277151in}}{\pgfqpoint{2.179407in}{2.271327in}}%
\pgfpathcurveto{\pgfqpoint{2.173583in}{2.265504in}}{\pgfqpoint{2.170311in}{2.257603in}}{\pgfqpoint{2.170311in}{2.249367in}}%
\pgfpathcurveto{\pgfqpoint{2.170311in}{2.241131in}}{\pgfqpoint{2.173583in}{2.233231in}}{\pgfqpoint{2.179407in}{2.227407in}}%
\pgfpathcurveto{\pgfqpoint{2.185231in}{2.221583in}}{\pgfqpoint{2.193131in}{2.218311in}}{\pgfqpoint{2.201368in}{2.218311in}}%
\pgfpathclose%
\pgfusepath{stroke,fill}%
\end{pgfscope}%
\begin{pgfscope}%
\pgfpathrectangle{\pgfqpoint{0.100000in}{0.212622in}}{\pgfqpoint{3.696000in}{3.696000in}}%
\pgfusepath{clip}%
\pgfsetbuttcap%
\pgfsetroundjoin%
\definecolor{currentfill}{rgb}{0.121569,0.466667,0.705882}%
\pgfsetfillcolor{currentfill}%
\pgfsetfillopacity{0.630921}%
\pgfsetlinewidth{1.003750pt}%
\definecolor{currentstroke}{rgb}{0.121569,0.466667,0.705882}%
\pgfsetstrokecolor{currentstroke}%
\pgfsetstrokeopacity{0.630921}%
\pgfsetdash{}{0pt}%
\pgfpathmoveto{\pgfqpoint{0.672594in}{1.200999in}}%
\pgfpathcurveto{\pgfqpoint{0.680831in}{1.200999in}}{\pgfqpoint{0.688731in}{1.204271in}}{\pgfqpoint{0.694555in}{1.210095in}}%
\pgfpathcurveto{\pgfqpoint{0.700378in}{1.215919in}}{\pgfqpoint{0.703651in}{1.223819in}}{\pgfqpoint{0.703651in}{1.232055in}}%
\pgfpathcurveto{\pgfqpoint{0.703651in}{1.240292in}}{\pgfqpoint{0.700378in}{1.248192in}}{\pgfqpoint{0.694555in}{1.254016in}}%
\pgfpathcurveto{\pgfqpoint{0.688731in}{1.259840in}}{\pgfqpoint{0.680831in}{1.263112in}}{\pgfqpoint{0.672594in}{1.263112in}}%
\pgfpathcurveto{\pgfqpoint{0.664358in}{1.263112in}}{\pgfqpoint{0.656458in}{1.259840in}}{\pgfqpoint{0.650634in}{1.254016in}}%
\pgfpathcurveto{\pgfqpoint{0.644810in}{1.248192in}}{\pgfqpoint{0.641538in}{1.240292in}}{\pgfqpoint{0.641538in}{1.232055in}}%
\pgfpathcurveto{\pgfqpoint{0.641538in}{1.223819in}}{\pgfqpoint{0.644810in}{1.215919in}}{\pgfqpoint{0.650634in}{1.210095in}}%
\pgfpathcurveto{\pgfqpoint{0.656458in}{1.204271in}}{\pgfqpoint{0.664358in}{1.200999in}}{\pgfqpoint{0.672594in}{1.200999in}}%
\pgfpathclose%
\pgfusepath{stroke,fill}%
\end{pgfscope}%
\begin{pgfscope}%
\pgfpathrectangle{\pgfqpoint{0.100000in}{0.212622in}}{\pgfqpoint{3.696000in}{3.696000in}}%
\pgfusepath{clip}%
\pgfsetbuttcap%
\pgfsetroundjoin%
\definecolor{currentfill}{rgb}{0.121569,0.466667,0.705882}%
\pgfsetfillcolor{currentfill}%
\pgfsetfillopacity{0.635222}%
\pgfsetlinewidth{1.003750pt}%
\definecolor{currentstroke}{rgb}{0.121569,0.466667,0.705882}%
\pgfsetstrokecolor{currentstroke}%
\pgfsetstrokeopacity{0.635222}%
\pgfsetdash{}{0pt}%
\pgfpathmoveto{\pgfqpoint{0.690958in}{1.197233in}}%
\pgfpathcurveto{\pgfqpoint{0.699195in}{1.197233in}}{\pgfqpoint{0.707095in}{1.200505in}}{\pgfqpoint{0.712919in}{1.206329in}}%
\pgfpathcurveto{\pgfqpoint{0.718743in}{1.212153in}}{\pgfqpoint{0.722015in}{1.220053in}}{\pgfqpoint{0.722015in}{1.228290in}}%
\pgfpathcurveto{\pgfqpoint{0.722015in}{1.236526in}}{\pgfqpoint{0.718743in}{1.244426in}}{\pgfqpoint{0.712919in}{1.250250in}}%
\pgfpathcurveto{\pgfqpoint{0.707095in}{1.256074in}}{\pgfqpoint{0.699195in}{1.259346in}}{\pgfqpoint{0.690958in}{1.259346in}}%
\pgfpathcurveto{\pgfqpoint{0.682722in}{1.259346in}}{\pgfqpoint{0.674822in}{1.256074in}}{\pgfqpoint{0.668998in}{1.250250in}}%
\pgfpathcurveto{\pgfqpoint{0.663174in}{1.244426in}}{\pgfqpoint{0.659902in}{1.236526in}}{\pgfqpoint{0.659902in}{1.228290in}}%
\pgfpathcurveto{\pgfqpoint{0.659902in}{1.220053in}}{\pgfqpoint{0.663174in}{1.212153in}}{\pgfqpoint{0.668998in}{1.206329in}}%
\pgfpathcurveto{\pgfqpoint{0.674822in}{1.200505in}}{\pgfqpoint{0.682722in}{1.197233in}}{\pgfqpoint{0.690958in}{1.197233in}}%
\pgfpathclose%
\pgfusepath{stroke,fill}%
\end{pgfscope}%
\begin{pgfscope}%
\pgfpathrectangle{\pgfqpoint{0.100000in}{0.212622in}}{\pgfqpoint{3.696000in}{3.696000in}}%
\pgfusepath{clip}%
\pgfsetbuttcap%
\pgfsetroundjoin%
\definecolor{currentfill}{rgb}{0.121569,0.466667,0.705882}%
\pgfsetfillcolor{currentfill}%
\pgfsetfillopacity{0.639810}%
\pgfsetlinewidth{1.003750pt}%
\definecolor{currentstroke}{rgb}{0.121569,0.466667,0.705882}%
\pgfsetstrokecolor{currentstroke}%
\pgfsetstrokeopacity{0.639810}%
\pgfsetdash{}{0pt}%
\pgfpathmoveto{\pgfqpoint{1.112115in}{1.925502in}}%
\pgfpathcurveto{\pgfqpoint{1.120351in}{1.925502in}}{\pgfqpoint{1.128251in}{1.928774in}}{\pgfqpoint{1.134075in}{1.934598in}}%
\pgfpathcurveto{\pgfqpoint{1.139899in}{1.940422in}}{\pgfqpoint{1.143171in}{1.948322in}}{\pgfqpoint{1.143171in}{1.956558in}}%
\pgfpathcurveto{\pgfqpoint{1.143171in}{1.964795in}}{\pgfqpoint{1.139899in}{1.972695in}}{\pgfqpoint{1.134075in}{1.978519in}}%
\pgfpathcurveto{\pgfqpoint{1.128251in}{1.984343in}}{\pgfqpoint{1.120351in}{1.987615in}}{\pgfqpoint{1.112115in}{1.987615in}}%
\pgfpathcurveto{\pgfqpoint{1.103878in}{1.987615in}}{\pgfqpoint{1.095978in}{1.984343in}}{\pgfqpoint{1.090154in}{1.978519in}}%
\pgfpathcurveto{\pgfqpoint{1.084330in}{1.972695in}}{\pgfqpoint{1.081058in}{1.964795in}}{\pgfqpoint{1.081058in}{1.956558in}}%
\pgfpathcurveto{\pgfqpoint{1.081058in}{1.948322in}}{\pgfqpoint{1.084330in}{1.940422in}}{\pgfqpoint{1.090154in}{1.934598in}}%
\pgfpathcurveto{\pgfqpoint{1.095978in}{1.928774in}}{\pgfqpoint{1.103878in}{1.925502in}}{\pgfqpoint{1.112115in}{1.925502in}}%
\pgfpathclose%
\pgfusepath{stroke,fill}%
\end{pgfscope}%
\begin{pgfscope}%
\pgfpathrectangle{\pgfqpoint{0.100000in}{0.212622in}}{\pgfqpoint{3.696000in}{3.696000in}}%
\pgfusepath{clip}%
\pgfsetbuttcap%
\pgfsetroundjoin%
\definecolor{currentfill}{rgb}{0.121569,0.466667,0.705882}%
\pgfsetfillcolor{currentfill}%
\pgfsetfillopacity{0.641489}%
\pgfsetlinewidth{1.003750pt}%
\definecolor{currentstroke}{rgb}{0.121569,0.466667,0.705882}%
\pgfsetstrokecolor{currentstroke}%
\pgfsetstrokeopacity{0.641489}%
\pgfsetdash{}{0pt}%
\pgfpathmoveto{\pgfqpoint{2.210770in}{2.178477in}}%
\pgfpathcurveto{\pgfqpoint{2.219006in}{2.178477in}}{\pgfqpoint{2.226907in}{2.181749in}}{\pgfqpoint{2.232730in}{2.187573in}}%
\pgfpathcurveto{\pgfqpoint{2.238554in}{2.193397in}}{\pgfqpoint{2.241827in}{2.201297in}}{\pgfqpoint{2.241827in}{2.209534in}}%
\pgfpathcurveto{\pgfqpoint{2.241827in}{2.217770in}}{\pgfqpoint{2.238554in}{2.225670in}}{\pgfqpoint{2.232730in}{2.231494in}}%
\pgfpathcurveto{\pgfqpoint{2.226907in}{2.237318in}}{\pgfqpoint{2.219006in}{2.240590in}}{\pgfqpoint{2.210770in}{2.240590in}}%
\pgfpathcurveto{\pgfqpoint{2.202534in}{2.240590in}}{\pgfqpoint{2.194634in}{2.237318in}}{\pgfqpoint{2.188810in}{2.231494in}}%
\pgfpathcurveto{\pgfqpoint{2.182986in}{2.225670in}}{\pgfqpoint{2.179714in}{2.217770in}}{\pgfqpoint{2.179714in}{2.209534in}}%
\pgfpathcurveto{\pgfqpoint{2.179714in}{2.201297in}}{\pgfqpoint{2.182986in}{2.193397in}}{\pgfqpoint{2.188810in}{2.187573in}}%
\pgfpathcurveto{\pgfqpoint{2.194634in}{2.181749in}}{\pgfqpoint{2.202534in}{2.178477in}}{\pgfqpoint{2.210770in}{2.178477in}}%
\pgfpathclose%
\pgfusepath{stroke,fill}%
\end{pgfscope}%
\begin{pgfscope}%
\pgfpathrectangle{\pgfqpoint{0.100000in}{0.212622in}}{\pgfqpoint{3.696000in}{3.696000in}}%
\pgfusepath{clip}%
\pgfsetbuttcap%
\pgfsetroundjoin%
\definecolor{currentfill}{rgb}{0.121569,0.466667,0.705882}%
\pgfsetfillcolor{currentfill}%
\pgfsetfillopacity{0.642887}%
\pgfsetlinewidth{1.003750pt}%
\definecolor{currentstroke}{rgb}{0.121569,0.466667,0.705882}%
\pgfsetstrokecolor{currentstroke}%
\pgfsetstrokeopacity{0.642887}%
\pgfsetdash{}{0pt}%
\pgfpathmoveto{\pgfqpoint{0.724636in}{1.190724in}}%
\pgfpathcurveto{\pgfqpoint{0.732872in}{1.190724in}}{\pgfqpoint{0.740772in}{1.193997in}}{\pgfqpoint{0.746596in}{1.199821in}}%
\pgfpathcurveto{\pgfqpoint{0.752420in}{1.205645in}}{\pgfqpoint{0.755692in}{1.213545in}}{\pgfqpoint{0.755692in}{1.221781in}}%
\pgfpathcurveto{\pgfqpoint{0.755692in}{1.230017in}}{\pgfqpoint{0.752420in}{1.237917in}}{\pgfqpoint{0.746596in}{1.243741in}}%
\pgfpathcurveto{\pgfqpoint{0.740772in}{1.249565in}}{\pgfqpoint{0.732872in}{1.252837in}}{\pgfqpoint{0.724636in}{1.252837in}}%
\pgfpathcurveto{\pgfqpoint{0.716400in}{1.252837in}}{\pgfqpoint{0.708500in}{1.249565in}}{\pgfqpoint{0.702676in}{1.243741in}}%
\pgfpathcurveto{\pgfqpoint{0.696852in}{1.237917in}}{\pgfqpoint{0.693579in}{1.230017in}}{\pgfqpoint{0.693579in}{1.221781in}}%
\pgfpathcurveto{\pgfqpoint{0.693579in}{1.213545in}}{\pgfqpoint{0.696852in}{1.205645in}}{\pgfqpoint{0.702676in}{1.199821in}}%
\pgfpathcurveto{\pgfqpoint{0.708500in}{1.193997in}}{\pgfqpoint{0.716400in}{1.190724in}}{\pgfqpoint{0.724636in}{1.190724in}}%
\pgfpathclose%
\pgfusepath{stroke,fill}%
\end{pgfscope}%
\begin{pgfscope}%
\pgfpathrectangle{\pgfqpoint{0.100000in}{0.212622in}}{\pgfqpoint{3.696000in}{3.696000in}}%
\pgfusepath{clip}%
\pgfsetbuttcap%
\pgfsetroundjoin%
\definecolor{currentfill}{rgb}{0.121569,0.466667,0.705882}%
\pgfsetfillcolor{currentfill}%
\pgfsetfillopacity{0.649658}%
\pgfsetlinewidth{1.003750pt}%
\definecolor{currentstroke}{rgb}{0.121569,0.466667,0.705882}%
\pgfsetstrokecolor{currentstroke}%
\pgfsetstrokeopacity{0.649658}%
\pgfsetdash{}{0pt}%
\pgfpathmoveto{\pgfqpoint{1.096823in}{1.932407in}}%
\pgfpathcurveto{\pgfqpoint{1.105059in}{1.932407in}}{\pgfqpoint{1.112959in}{1.935679in}}{\pgfqpoint{1.118783in}{1.941503in}}%
\pgfpathcurveto{\pgfqpoint{1.124607in}{1.947327in}}{\pgfqpoint{1.127879in}{1.955227in}}{\pgfqpoint{1.127879in}{1.963464in}}%
\pgfpathcurveto{\pgfqpoint{1.127879in}{1.971700in}}{\pgfqpoint{1.124607in}{1.979600in}}{\pgfqpoint{1.118783in}{1.985424in}}%
\pgfpathcurveto{\pgfqpoint{1.112959in}{1.991248in}}{\pgfqpoint{1.105059in}{1.994520in}}{\pgfqpoint{1.096823in}{1.994520in}}%
\pgfpathcurveto{\pgfqpoint{1.088587in}{1.994520in}}{\pgfqpoint{1.080687in}{1.991248in}}{\pgfqpoint{1.074863in}{1.985424in}}%
\pgfpathcurveto{\pgfqpoint{1.069039in}{1.979600in}}{\pgfqpoint{1.065766in}{1.971700in}}{\pgfqpoint{1.065766in}{1.963464in}}%
\pgfpathcurveto{\pgfqpoint{1.065766in}{1.955227in}}{\pgfqpoint{1.069039in}{1.947327in}}{\pgfqpoint{1.074863in}{1.941503in}}%
\pgfpathcurveto{\pgfqpoint{1.080687in}{1.935679in}}{\pgfqpoint{1.088587in}{1.932407in}}{\pgfqpoint{1.096823in}{1.932407in}}%
\pgfpathclose%
\pgfusepath{stroke,fill}%
\end{pgfscope}%
\begin{pgfscope}%
\pgfpathrectangle{\pgfqpoint{0.100000in}{0.212622in}}{\pgfqpoint{3.696000in}{3.696000in}}%
\pgfusepath{clip}%
\pgfsetbuttcap%
\pgfsetroundjoin%
\definecolor{currentfill}{rgb}{0.121569,0.466667,0.705882}%
\pgfsetfillcolor{currentfill}%
\pgfsetfillopacity{0.649873}%
\pgfsetlinewidth{1.003750pt}%
\definecolor{currentstroke}{rgb}{0.121569,0.466667,0.705882}%
\pgfsetstrokecolor{currentstroke}%
\pgfsetstrokeopacity{0.649873}%
\pgfsetdash{}{0pt}%
\pgfpathmoveto{\pgfqpoint{0.756299in}{1.185524in}}%
\pgfpathcurveto{\pgfqpoint{0.764536in}{1.185524in}}{\pgfqpoint{0.772436in}{1.188797in}}{\pgfqpoint{0.778260in}{1.194621in}}%
\pgfpathcurveto{\pgfqpoint{0.784084in}{1.200444in}}{\pgfqpoint{0.787356in}{1.208344in}}{\pgfqpoint{0.787356in}{1.216581in}}%
\pgfpathcurveto{\pgfqpoint{0.787356in}{1.224817in}}{\pgfqpoint{0.784084in}{1.232717in}}{\pgfqpoint{0.778260in}{1.238541in}}%
\pgfpathcurveto{\pgfqpoint{0.772436in}{1.244365in}}{\pgfqpoint{0.764536in}{1.247637in}}{\pgfqpoint{0.756299in}{1.247637in}}%
\pgfpathcurveto{\pgfqpoint{0.748063in}{1.247637in}}{\pgfqpoint{0.740163in}{1.244365in}}{\pgfqpoint{0.734339in}{1.238541in}}%
\pgfpathcurveto{\pgfqpoint{0.728515in}{1.232717in}}{\pgfqpoint{0.725243in}{1.224817in}}{\pgfqpoint{0.725243in}{1.216581in}}%
\pgfpathcurveto{\pgfqpoint{0.725243in}{1.208344in}}{\pgfqpoint{0.728515in}{1.200444in}}{\pgfqpoint{0.734339in}{1.194621in}}%
\pgfpathcurveto{\pgfqpoint{0.740163in}{1.188797in}}{\pgfqpoint{0.748063in}{1.185524in}}{\pgfqpoint{0.756299in}{1.185524in}}%
\pgfpathclose%
\pgfusepath{stroke,fill}%
\end{pgfscope}%
\begin{pgfscope}%
\pgfpathrectangle{\pgfqpoint{0.100000in}{0.212622in}}{\pgfqpoint{3.696000in}{3.696000in}}%
\pgfusepath{clip}%
\pgfsetbuttcap%
\pgfsetroundjoin%
\definecolor{currentfill}{rgb}{0.121569,0.466667,0.705882}%
\pgfsetfillcolor{currentfill}%
\pgfsetfillopacity{0.653577}%
\pgfsetlinewidth{1.003750pt}%
\definecolor{currentstroke}{rgb}{0.121569,0.466667,0.705882}%
\pgfsetstrokecolor{currentstroke}%
\pgfsetstrokeopacity{0.653577}%
\pgfsetdash{}{0pt}%
\pgfpathmoveto{\pgfqpoint{2.223755in}{2.135786in}}%
\pgfpathcurveto{\pgfqpoint{2.231991in}{2.135786in}}{\pgfqpoint{2.239891in}{2.139058in}}{\pgfqpoint{2.245715in}{2.144882in}}%
\pgfpathcurveto{\pgfqpoint{2.251539in}{2.150706in}}{\pgfqpoint{2.254812in}{2.158606in}}{\pgfqpoint{2.254812in}{2.166842in}}%
\pgfpathcurveto{\pgfqpoint{2.254812in}{2.175078in}}{\pgfqpoint{2.251539in}{2.182978in}}{\pgfqpoint{2.245715in}{2.188802in}}%
\pgfpathcurveto{\pgfqpoint{2.239891in}{2.194626in}}{\pgfqpoint{2.231991in}{2.197899in}}{\pgfqpoint{2.223755in}{2.197899in}}%
\pgfpathcurveto{\pgfqpoint{2.215519in}{2.197899in}}{\pgfqpoint{2.207619in}{2.194626in}}{\pgfqpoint{2.201795in}{2.188802in}}%
\pgfpathcurveto{\pgfqpoint{2.195971in}{2.182978in}}{\pgfqpoint{2.192699in}{2.175078in}}{\pgfqpoint{2.192699in}{2.166842in}}%
\pgfpathcurveto{\pgfqpoint{2.192699in}{2.158606in}}{\pgfqpoint{2.195971in}{2.150706in}}{\pgfqpoint{2.201795in}{2.144882in}}%
\pgfpathcurveto{\pgfqpoint{2.207619in}{2.139058in}}{\pgfqpoint{2.215519in}{2.135786in}}{\pgfqpoint{2.223755in}{2.135786in}}%
\pgfpathclose%
\pgfusepath{stroke,fill}%
\end{pgfscope}%
\begin{pgfscope}%
\pgfpathrectangle{\pgfqpoint{0.100000in}{0.212622in}}{\pgfqpoint{3.696000in}{3.696000in}}%
\pgfusepath{clip}%
\pgfsetbuttcap%
\pgfsetroundjoin%
\definecolor{currentfill}{rgb}{0.121569,0.466667,0.705882}%
\pgfsetfillcolor{currentfill}%
\pgfsetfillopacity{0.655766}%
\pgfsetlinewidth{1.003750pt}%
\definecolor{currentstroke}{rgb}{0.121569,0.466667,0.705882}%
\pgfsetstrokecolor{currentstroke}%
\pgfsetstrokeopacity{0.655766}%
\pgfsetdash{}{0pt}%
\pgfpathmoveto{\pgfqpoint{0.780321in}{1.176963in}}%
\pgfpathcurveto{\pgfqpoint{0.788557in}{1.176963in}}{\pgfqpoint{0.796457in}{1.180236in}}{\pgfqpoint{0.802281in}{1.186059in}}%
\pgfpathcurveto{\pgfqpoint{0.808105in}{1.191883in}}{\pgfqpoint{0.811377in}{1.199783in}}{\pgfqpoint{0.811377in}{1.208020in}}%
\pgfpathcurveto{\pgfqpoint{0.811377in}{1.216256in}}{\pgfqpoint{0.808105in}{1.224156in}}{\pgfqpoint{0.802281in}{1.229980in}}%
\pgfpathcurveto{\pgfqpoint{0.796457in}{1.235804in}}{\pgfqpoint{0.788557in}{1.239076in}}{\pgfqpoint{0.780321in}{1.239076in}}%
\pgfpathcurveto{\pgfqpoint{0.772085in}{1.239076in}}{\pgfqpoint{0.764185in}{1.235804in}}{\pgfqpoint{0.758361in}{1.229980in}}%
\pgfpathcurveto{\pgfqpoint{0.752537in}{1.224156in}}{\pgfqpoint{0.749264in}{1.216256in}}{\pgfqpoint{0.749264in}{1.208020in}}%
\pgfpathcurveto{\pgfqpoint{0.749264in}{1.199783in}}{\pgfqpoint{0.752537in}{1.191883in}}{\pgfqpoint{0.758361in}{1.186059in}}%
\pgfpathcurveto{\pgfqpoint{0.764185in}{1.180236in}}{\pgfqpoint{0.772085in}{1.176963in}}{\pgfqpoint{0.780321in}{1.176963in}}%
\pgfpathclose%
\pgfusepath{stroke,fill}%
\end{pgfscope}%
\begin{pgfscope}%
\pgfpathrectangle{\pgfqpoint{0.100000in}{0.212622in}}{\pgfqpoint{3.696000in}{3.696000in}}%
\pgfusepath{clip}%
\pgfsetbuttcap%
\pgfsetroundjoin%
\definecolor{currentfill}{rgb}{0.121569,0.466667,0.705882}%
\pgfsetfillcolor{currentfill}%
\pgfsetfillopacity{0.658248}%
\pgfsetlinewidth{1.003750pt}%
\definecolor{currentstroke}{rgb}{0.121569,0.466667,0.705882}%
\pgfsetstrokecolor{currentstroke}%
\pgfsetstrokeopacity{0.658248}%
\pgfsetdash{}{0pt}%
\pgfpathmoveto{\pgfqpoint{1.085260in}{1.937782in}}%
\pgfpathcurveto{\pgfqpoint{1.093497in}{1.937782in}}{\pgfqpoint{1.101397in}{1.941054in}}{\pgfqpoint{1.107221in}{1.946878in}}%
\pgfpathcurveto{\pgfqpoint{1.113044in}{1.952702in}}{\pgfqpoint{1.116317in}{1.960602in}}{\pgfqpoint{1.116317in}{1.968838in}}%
\pgfpathcurveto{\pgfqpoint{1.116317in}{1.977074in}}{\pgfqpoint{1.113044in}{1.984974in}}{\pgfqpoint{1.107221in}{1.990798in}}%
\pgfpathcurveto{\pgfqpoint{1.101397in}{1.996622in}}{\pgfqpoint{1.093497in}{1.999895in}}{\pgfqpoint{1.085260in}{1.999895in}}%
\pgfpathcurveto{\pgfqpoint{1.077024in}{1.999895in}}{\pgfqpoint{1.069124in}{1.996622in}}{\pgfqpoint{1.063300in}{1.990798in}}%
\pgfpathcurveto{\pgfqpoint{1.057476in}{1.984974in}}{\pgfqpoint{1.054204in}{1.977074in}}{\pgfqpoint{1.054204in}{1.968838in}}%
\pgfpathcurveto{\pgfqpoint{1.054204in}{1.960602in}}{\pgfqpoint{1.057476in}{1.952702in}}{\pgfqpoint{1.063300in}{1.946878in}}%
\pgfpathcurveto{\pgfqpoint{1.069124in}{1.941054in}}{\pgfqpoint{1.077024in}{1.937782in}}{\pgfqpoint{1.085260in}{1.937782in}}%
\pgfpathclose%
\pgfusepath{stroke,fill}%
\end{pgfscope}%
\begin{pgfscope}%
\pgfpathrectangle{\pgfqpoint{0.100000in}{0.212622in}}{\pgfqpoint{3.696000in}{3.696000in}}%
\pgfusepath{clip}%
\pgfsetbuttcap%
\pgfsetroundjoin%
\definecolor{currentfill}{rgb}{0.121569,0.466667,0.705882}%
\pgfsetfillcolor{currentfill}%
\pgfsetfillopacity{0.660400}%
\pgfsetlinewidth{1.003750pt}%
\definecolor{currentstroke}{rgb}{0.121569,0.466667,0.705882}%
\pgfsetstrokecolor{currentstroke}%
\pgfsetstrokeopacity{0.660400}%
\pgfsetdash{}{0pt}%
\pgfpathmoveto{\pgfqpoint{0.802279in}{1.174237in}}%
\pgfpathcurveto{\pgfqpoint{0.810515in}{1.174237in}}{\pgfqpoint{0.818416in}{1.177509in}}{\pgfqpoint{0.824239in}{1.183333in}}%
\pgfpathcurveto{\pgfqpoint{0.830063in}{1.189157in}}{\pgfqpoint{0.833336in}{1.197057in}}{\pgfqpoint{0.833336in}{1.205293in}}%
\pgfpathcurveto{\pgfqpoint{0.833336in}{1.213529in}}{\pgfqpoint{0.830063in}{1.221429in}}{\pgfqpoint{0.824239in}{1.227253in}}%
\pgfpathcurveto{\pgfqpoint{0.818416in}{1.233077in}}{\pgfqpoint{0.810515in}{1.236350in}}{\pgfqpoint{0.802279in}{1.236350in}}%
\pgfpathcurveto{\pgfqpoint{0.794043in}{1.236350in}}{\pgfqpoint{0.786143in}{1.233077in}}{\pgfqpoint{0.780319in}{1.227253in}}%
\pgfpathcurveto{\pgfqpoint{0.774495in}{1.221429in}}{\pgfqpoint{0.771223in}{1.213529in}}{\pgfqpoint{0.771223in}{1.205293in}}%
\pgfpathcurveto{\pgfqpoint{0.771223in}{1.197057in}}{\pgfqpoint{0.774495in}{1.189157in}}{\pgfqpoint{0.780319in}{1.183333in}}%
\pgfpathcurveto{\pgfqpoint{0.786143in}{1.177509in}}{\pgfqpoint{0.794043in}{1.174237in}}{\pgfqpoint{0.802279in}{1.174237in}}%
\pgfpathclose%
\pgfusepath{stroke,fill}%
\end{pgfscope}%
\begin{pgfscope}%
\pgfpathrectangle{\pgfqpoint{0.100000in}{0.212622in}}{\pgfqpoint{3.696000in}{3.696000in}}%
\pgfusepath{clip}%
\pgfsetbuttcap%
\pgfsetroundjoin%
\definecolor{currentfill}{rgb}{0.121569,0.466667,0.705882}%
\pgfsetfillcolor{currentfill}%
\pgfsetfillopacity{0.664241}%
\pgfsetlinewidth{1.003750pt}%
\definecolor{currentstroke}{rgb}{0.121569,0.466667,0.705882}%
\pgfsetstrokecolor{currentstroke}%
\pgfsetstrokeopacity{0.664241}%
\pgfsetdash{}{0pt}%
\pgfpathmoveto{\pgfqpoint{0.818736in}{1.169970in}}%
\pgfpathcurveto{\pgfqpoint{0.826972in}{1.169970in}}{\pgfqpoint{0.834872in}{1.173242in}}{\pgfqpoint{0.840696in}{1.179066in}}%
\pgfpathcurveto{\pgfqpoint{0.846520in}{1.184890in}}{\pgfqpoint{0.849792in}{1.192790in}}{\pgfqpoint{0.849792in}{1.201026in}}%
\pgfpathcurveto{\pgfqpoint{0.849792in}{1.209263in}}{\pgfqpoint{0.846520in}{1.217163in}}{\pgfqpoint{0.840696in}{1.222987in}}%
\pgfpathcurveto{\pgfqpoint{0.834872in}{1.228810in}}{\pgfqpoint{0.826972in}{1.232083in}}{\pgfqpoint{0.818736in}{1.232083in}}%
\pgfpathcurveto{\pgfqpoint{0.810499in}{1.232083in}}{\pgfqpoint{0.802599in}{1.228810in}}{\pgfqpoint{0.796776in}{1.222987in}}%
\pgfpathcurveto{\pgfqpoint{0.790952in}{1.217163in}}{\pgfqpoint{0.787679in}{1.209263in}}{\pgfqpoint{0.787679in}{1.201026in}}%
\pgfpathcurveto{\pgfqpoint{0.787679in}{1.192790in}}{\pgfqpoint{0.790952in}{1.184890in}}{\pgfqpoint{0.796776in}{1.179066in}}%
\pgfpathcurveto{\pgfqpoint{0.802599in}{1.173242in}}{\pgfqpoint{0.810499in}{1.169970in}}{\pgfqpoint{0.818736in}{1.169970in}}%
\pgfpathclose%
\pgfusepath{stroke,fill}%
\end{pgfscope}%
\begin{pgfscope}%
\pgfpathrectangle{\pgfqpoint{0.100000in}{0.212622in}}{\pgfqpoint{3.696000in}{3.696000in}}%
\pgfusepath{clip}%
\pgfsetbuttcap%
\pgfsetroundjoin%
\definecolor{currentfill}{rgb}{0.121569,0.466667,0.705882}%
\pgfsetfillcolor{currentfill}%
\pgfsetfillopacity{0.664856}%
\pgfsetlinewidth{1.003750pt}%
\definecolor{currentstroke}{rgb}{0.121569,0.466667,0.705882}%
\pgfsetstrokecolor{currentstroke}%
\pgfsetstrokeopacity{0.664856}%
\pgfsetdash{}{0pt}%
\pgfpathmoveto{\pgfqpoint{1.077490in}{1.941586in}}%
\pgfpathcurveto{\pgfqpoint{1.085727in}{1.941586in}}{\pgfqpoint{1.093627in}{1.944858in}}{\pgfqpoint{1.099451in}{1.950682in}}%
\pgfpathcurveto{\pgfqpoint{1.105275in}{1.956506in}}{\pgfqpoint{1.108547in}{1.964406in}}{\pgfqpoint{1.108547in}{1.972642in}}%
\pgfpathcurveto{\pgfqpoint{1.108547in}{1.980879in}}{\pgfqpoint{1.105275in}{1.988779in}}{\pgfqpoint{1.099451in}{1.994603in}}%
\pgfpathcurveto{\pgfqpoint{1.093627in}{2.000427in}}{\pgfqpoint{1.085727in}{2.003699in}}{\pgfqpoint{1.077490in}{2.003699in}}%
\pgfpathcurveto{\pgfqpoint{1.069254in}{2.003699in}}{\pgfqpoint{1.061354in}{2.000427in}}{\pgfqpoint{1.055530in}{1.994603in}}%
\pgfpathcurveto{\pgfqpoint{1.049706in}{1.988779in}}{\pgfqpoint{1.046434in}{1.980879in}}{\pgfqpoint{1.046434in}{1.972642in}}%
\pgfpathcurveto{\pgfqpoint{1.046434in}{1.964406in}}{\pgfqpoint{1.049706in}{1.956506in}}{\pgfqpoint{1.055530in}{1.950682in}}%
\pgfpathcurveto{\pgfqpoint{1.061354in}{1.944858in}}{\pgfqpoint{1.069254in}{1.941586in}}{\pgfqpoint{1.077490in}{1.941586in}}%
\pgfpathclose%
\pgfusepath{stroke,fill}%
\end{pgfscope}%
\begin{pgfscope}%
\pgfpathrectangle{\pgfqpoint{0.100000in}{0.212622in}}{\pgfqpoint{3.696000in}{3.696000in}}%
\pgfusepath{clip}%
\pgfsetbuttcap%
\pgfsetroundjoin%
\definecolor{currentfill}{rgb}{0.121569,0.466667,0.705882}%
\pgfsetfillcolor{currentfill}%
\pgfsetfillopacity{0.666458}%
\pgfsetlinewidth{1.003750pt}%
\definecolor{currentstroke}{rgb}{0.121569,0.466667,0.705882}%
\pgfsetstrokecolor{currentstroke}%
\pgfsetstrokeopacity{0.666458}%
\pgfsetdash{}{0pt}%
\pgfpathmoveto{\pgfqpoint{2.235571in}{2.087670in}}%
\pgfpathcurveto{\pgfqpoint{2.243807in}{2.087670in}}{\pgfqpoint{2.251707in}{2.090942in}}{\pgfqpoint{2.257531in}{2.096766in}}%
\pgfpathcurveto{\pgfqpoint{2.263355in}{2.102590in}}{\pgfqpoint{2.266627in}{2.110490in}}{\pgfqpoint{2.266627in}{2.118726in}}%
\pgfpathcurveto{\pgfqpoint{2.266627in}{2.126963in}}{\pgfqpoint{2.263355in}{2.134863in}}{\pgfqpoint{2.257531in}{2.140687in}}%
\pgfpathcurveto{\pgfqpoint{2.251707in}{2.146510in}}{\pgfqpoint{2.243807in}{2.149783in}}{\pgfqpoint{2.235571in}{2.149783in}}%
\pgfpathcurveto{\pgfqpoint{2.227334in}{2.149783in}}{\pgfqpoint{2.219434in}{2.146510in}}{\pgfqpoint{2.213610in}{2.140687in}}%
\pgfpathcurveto{\pgfqpoint{2.207786in}{2.134863in}}{\pgfqpoint{2.204514in}{2.126963in}}{\pgfqpoint{2.204514in}{2.118726in}}%
\pgfpathcurveto{\pgfqpoint{2.204514in}{2.110490in}}{\pgfqpoint{2.207786in}{2.102590in}}{\pgfqpoint{2.213610in}{2.096766in}}%
\pgfpathcurveto{\pgfqpoint{2.219434in}{2.090942in}}{\pgfqpoint{2.227334in}{2.087670in}}{\pgfqpoint{2.235571in}{2.087670in}}%
\pgfpathclose%
\pgfusepath{stroke,fill}%
\end{pgfscope}%
\begin{pgfscope}%
\pgfpathrectangle{\pgfqpoint{0.100000in}{0.212622in}}{\pgfqpoint{3.696000in}{3.696000in}}%
\pgfusepath{clip}%
\pgfsetbuttcap%
\pgfsetroundjoin%
\definecolor{currentfill}{rgb}{0.121569,0.466667,0.705882}%
\pgfsetfillcolor{currentfill}%
\pgfsetfillopacity{0.667380}%
\pgfsetlinewidth{1.003750pt}%
\definecolor{currentstroke}{rgb}{0.121569,0.466667,0.705882}%
\pgfsetstrokecolor{currentstroke}%
\pgfsetstrokeopacity{0.667380}%
\pgfsetdash{}{0pt}%
\pgfpathmoveto{\pgfqpoint{0.832709in}{1.166772in}}%
\pgfpathcurveto{\pgfqpoint{0.840945in}{1.166772in}}{\pgfqpoint{0.848845in}{1.170044in}}{\pgfqpoint{0.854669in}{1.175868in}}%
\pgfpathcurveto{\pgfqpoint{0.860493in}{1.181692in}}{\pgfqpoint{0.863765in}{1.189592in}}{\pgfqpoint{0.863765in}{1.197828in}}%
\pgfpathcurveto{\pgfqpoint{0.863765in}{1.206065in}}{\pgfqpoint{0.860493in}{1.213965in}}{\pgfqpoint{0.854669in}{1.219789in}}%
\pgfpathcurveto{\pgfqpoint{0.848845in}{1.225613in}}{\pgfqpoint{0.840945in}{1.228885in}}{\pgfqpoint{0.832709in}{1.228885in}}%
\pgfpathcurveto{\pgfqpoint{0.824472in}{1.228885in}}{\pgfqpoint{0.816572in}{1.225613in}}{\pgfqpoint{0.810748in}{1.219789in}}%
\pgfpathcurveto{\pgfqpoint{0.804924in}{1.213965in}}{\pgfqpoint{0.801652in}{1.206065in}}{\pgfqpoint{0.801652in}{1.197828in}}%
\pgfpathcurveto{\pgfqpoint{0.801652in}{1.189592in}}{\pgfqpoint{0.804924in}{1.181692in}}{\pgfqpoint{0.810748in}{1.175868in}}%
\pgfpathcurveto{\pgfqpoint{0.816572in}{1.170044in}}{\pgfqpoint{0.824472in}{1.166772in}}{\pgfqpoint{0.832709in}{1.166772in}}%
\pgfpathclose%
\pgfusepath{stroke,fill}%
\end{pgfscope}%
\begin{pgfscope}%
\pgfpathrectangle{\pgfqpoint{0.100000in}{0.212622in}}{\pgfqpoint{3.696000in}{3.696000in}}%
\pgfusepath{clip}%
\pgfsetbuttcap%
\pgfsetroundjoin%
\definecolor{currentfill}{rgb}{0.121569,0.466667,0.705882}%
\pgfsetfillcolor{currentfill}%
\pgfsetfillopacity{0.668953}%
\pgfsetlinewidth{1.003750pt}%
\definecolor{currentstroke}{rgb}{0.121569,0.466667,0.705882}%
\pgfsetstrokecolor{currentstroke}%
\pgfsetstrokeopacity{0.668953}%
\pgfsetdash{}{0pt}%
\pgfpathmoveto{\pgfqpoint{1.073814in}{1.943701in}}%
\pgfpathcurveto{\pgfqpoint{1.082050in}{1.943701in}}{\pgfqpoint{1.089950in}{1.946973in}}{\pgfqpoint{1.095774in}{1.952797in}}%
\pgfpathcurveto{\pgfqpoint{1.101598in}{1.958621in}}{\pgfqpoint{1.104870in}{1.966521in}}{\pgfqpoint{1.104870in}{1.974757in}}%
\pgfpathcurveto{\pgfqpoint{1.104870in}{1.982994in}}{\pgfqpoint{1.101598in}{1.990894in}}{\pgfqpoint{1.095774in}{1.996718in}}%
\pgfpathcurveto{\pgfqpoint{1.089950in}{2.002542in}}{\pgfqpoint{1.082050in}{2.005814in}}{\pgfqpoint{1.073814in}{2.005814in}}%
\pgfpathcurveto{\pgfqpoint{1.065577in}{2.005814in}}{\pgfqpoint{1.057677in}{2.002542in}}{\pgfqpoint{1.051853in}{1.996718in}}%
\pgfpathcurveto{\pgfqpoint{1.046029in}{1.990894in}}{\pgfqpoint{1.042757in}{1.982994in}}{\pgfqpoint{1.042757in}{1.974757in}}%
\pgfpathcurveto{\pgfqpoint{1.042757in}{1.966521in}}{\pgfqpoint{1.046029in}{1.958621in}}{\pgfqpoint{1.051853in}{1.952797in}}%
\pgfpathcurveto{\pgfqpoint{1.057677in}{1.946973in}}{\pgfqpoint{1.065577in}{1.943701in}}{\pgfqpoint{1.073814in}{1.943701in}}%
\pgfpathclose%
\pgfusepath{stroke,fill}%
\end{pgfscope}%
\begin{pgfscope}%
\pgfpathrectangle{\pgfqpoint{0.100000in}{0.212622in}}{\pgfqpoint{3.696000in}{3.696000in}}%
\pgfusepath{clip}%
\pgfsetbuttcap%
\pgfsetroundjoin%
\definecolor{currentfill}{rgb}{0.121569,0.466667,0.705882}%
\pgfsetfillcolor{currentfill}%
\pgfsetfillopacity{0.673063}%
\pgfsetlinewidth{1.003750pt}%
\definecolor{currentstroke}{rgb}{0.121569,0.466667,0.705882}%
\pgfsetstrokecolor{currentstroke}%
\pgfsetstrokeopacity{0.673063}%
\pgfsetdash{}{0pt}%
\pgfpathmoveto{\pgfqpoint{0.858409in}{1.162059in}}%
\pgfpathcurveto{\pgfqpoint{0.866645in}{1.162059in}}{\pgfqpoint{0.874545in}{1.165331in}}{\pgfqpoint{0.880369in}{1.171155in}}%
\pgfpathcurveto{\pgfqpoint{0.886193in}{1.176979in}}{\pgfqpoint{0.889466in}{1.184879in}}{\pgfqpoint{0.889466in}{1.193116in}}%
\pgfpathcurveto{\pgfqpoint{0.889466in}{1.201352in}}{\pgfqpoint{0.886193in}{1.209252in}}{\pgfqpoint{0.880369in}{1.215076in}}%
\pgfpathcurveto{\pgfqpoint{0.874545in}{1.220900in}}{\pgfqpoint{0.866645in}{1.224172in}}{\pgfqpoint{0.858409in}{1.224172in}}%
\pgfpathcurveto{\pgfqpoint{0.850173in}{1.224172in}}{\pgfqpoint{0.842273in}{1.220900in}}{\pgfqpoint{0.836449in}{1.215076in}}%
\pgfpathcurveto{\pgfqpoint{0.830625in}{1.209252in}}{\pgfqpoint{0.827353in}{1.201352in}}{\pgfqpoint{0.827353in}{1.193116in}}%
\pgfpathcurveto{\pgfqpoint{0.827353in}{1.184879in}}{\pgfqpoint{0.830625in}{1.176979in}}{\pgfqpoint{0.836449in}{1.171155in}}%
\pgfpathcurveto{\pgfqpoint{0.842273in}{1.165331in}}{\pgfqpoint{0.850173in}{1.162059in}}{\pgfqpoint{0.858409in}{1.162059in}}%
\pgfpathclose%
\pgfusepath{stroke,fill}%
\end{pgfscope}%
\begin{pgfscope}%
\pgfpathrectangle{\pgfqpoint{0.100000in}{0.212622in}}{\pgfqpoint{3.696000in}{3.696000in}}%
\pgfusepath{clip}%
\pgfsetbuttcap%
\pgfsetroundjoin%
\definecolor{currentfill}{rgb}{0.121569,0.466667,0.705882}%
\pgfsetfillcolor{currentfill}%
\pgfsetfillopacity{0.673804}%
\pgfsetlinewidth{1.003750pt}%
\definecolor{currentstroke}{rgb}{0.121569,0.466667,0.705882}%
\pgfsetstrokecolor{currentstroke}%
\pgfsetstrokeopacity{0.673804}%
\pgfsetdash{}{0pt}%
\pgfpathmoveto{\pgfqpoint{2.243759in}{2.061964in}}%
\pgfpathcurveto{\pgfqpoint{2.251995in}{2.061964in}}{\pgfqpoint{2.259895in}{2.065237in}}{\pgfqpoint{2.265719in}{2.071061in}}%
\pgfpathcurveto{\pgfqpoint{2.271543in}{2.076884in}}{\pgfqpoint{2.274815in}{2.084785in}}{\pgfqpoint{2.274815in}{2.093021in}}%
\pgfpathcurveto{\pgfqpoint{2.274815in}{2.101257in}}{\pgfqpoint{2.271543in}{2.109157in}}{\pgfqpoint{2.265719in}{2.114981in}}%
\pgfpathcurveto{\pgfqpoint{2.259895in}{2.120805in}}{\pgfqpoint{2.251995in}{2.124077in}}{\pgfqpoint{2.243759in}{2.124077in}}%
\pgfpathcurveto{\pgfqpoint{2.235522in}{2.124077in}}{\pgfqpoint{2.227622in}{2.120805in}}{\pgfqpoint{2.221798in}{2.114981in}}%
\pgfpathcurveto{\pgfqpoint{2.215974in}{2.109157in}}{\pgfqpoint{2.212702in}{2.101257in}}{\pgfqpoint{2.212702in}{2.093021in}}%
\pgfpathcurveto{\pgfqpoint{2.212702in}{2.084785in}}{\pgfqpoint{2.215974in}{2.076884in}}{\pgfqpoint{2.221798in}{2.071061in}}%
\pgfpathcurveto{\pgfqpoint{2.227622in}{2.065237in}}{\pgfqpoint{2.235522in}{2.061964in}}{\pgfqpoint{2.243759in}{2.061964in}}%
\pgfpathclose%
\pgfusepath{stroke,fill}%
\end{pgfscope}%
\begin{pgfscope}%
\pgfpathrectangle{\pgfqpoint{0.100000in}{0.212622in}}{\pgfqpoint{3.696000in}{3.696000in}}%
\pgfusepath{clip}%
\pgfsetbuttcap%
\pgfsetroundjoin%
\definecolor{currentfill}{rgb}{0.121569,0.466667,0.705882}%
\pgfsetfillcolor{currentfill}%
\pgfsetfillopacity{0.678120}%
\pgfsetlinewidth{1.003750pt}%
\definecolor{currentstroke}{rgb}{0.121569,0.466667,0.705882}%
\pgfsetstrokecolor{currentstroke}%
\pgfsetstrokeopacity{0.678120}%
\pgfsetdash{}{0pt}%
\pgfpathmoveto{\pgfqpoint{0.882033in}{1.158394in}}%
\pgfpathcurveto{\pgfqpoint{0.890269in}{1.158394in}}{\pgfqpoint{0.898169in}{1.161667in}}{\pgfqpoint{0.903993in}{1.167490in}}%
\pgfpathcurveto{\pgfqpoint{0.909817in}{1.173314in}}{\pgfqpoint{0.913090in}{1.181214in}}{\pgfqpoint{0.913090in}{1.189451in}}%
\pgfpathcurveto{\pgfqpoint{0.913090in}{1.197687in}}{\pgfqpoint{0.909817in}{1.205587in}}{\pgfqpoint{0.903993in}{1.211411in}}%
\pgfpathcurveto{\pgfqpoint{0.898169in}{1.217235in}}{\pgfqpoint{0.890269in}{1.220507in}}{\pgfqpoint{0.882033in}{1.220507in}}%
\pgfpathcurveto{\pgfqpoint{0.873797in}{1.220507in}}{\pgfqpoint{0.865897in}{1.217235in}}{\pgfqpoint{0.860073in}{1.211411in}}%
\pgfpathcurveto{\pgfqpoint{0.854249in}{1.205587in}}{\pgfqpoint{0.850977in}{1.197687in}}{\pgfqpoint{0.850977in}{1.189451in}}%
\pgfpathcurveto{\pgfqpoint{0.850977in}{1.181214in}}{\pgfqpoint{0.854249in}{1.173314in}}{\pgfqpoint{0.860073in}{1.167490in}}%
\pgfpathcurveto{\pgfqpoint{0.865897in}{1.161667in}}{\pgfqpoint{0.873797in}{1.158394in}}{\pgfqpoint{0.882033in}{1.158394in}}%
\pgfpathclose%
\pgfusepath{stroke,fill}%
\end{pgfscope}%
\begin{pgfscope}%
\pgfpathrectangle{\pgfqpoint{0.100000in}{0.212622in}}{\pgfqpoint{3.696000in}{3.696000in}}%
\pgfusepath{clip}%
\pgfsetbuttcap%
\pgfsetroundjoin%
\definecolor{currentfill}{rgb}{0.121569,0.466667,0.705882}%
\pgfsetfillcolor{currentfill}%
\pgfsetfillopacity{0.682363}%
\pgfsetlinewidth{1.003750pt}%
\definecolor{currentstroke}{rgb}{0.121569,0.466667,0.705882}%
\pgfsetstrokecolor{currentstroke}%
\pgfsetstrokeopacity{0.682363}%
\pgfsetdash{}{0pt}%
\pgfpathmoveto{\pgfqpoint{0.900401in}{1.153611in}}%
\pgfpathcurveto{\pgfqpoint{0.908637in}{1.153611in}}{\pgfqpoint{0.916537in}{1.156883in}}{\pgfqpoint{0.922361in}{1.162707in}}%
\pgfpathcurveto{\pgfqpoint{0.928185in}{1.168531in}}{\pgfqpoint{0.931457in}{1.176431in}}{\pgfqpoint{0.931457in}{1.184667in}}%
\pgfpathcurveto{\pgfqpoint{0.931457in}{1.192903in}}{\pgfqpoint{0.928185in}{1.200804in}}{\pgfqpoint{0.922361in}{1.206627in}}%
\pgfpathcurveto{\pgfqpoint{0.916537in}{1.212451in}}{\pgfqpoint{0.908637in}{1.215724in}}{\pgfqpoint{0.900401in}{1.215724in}}%
\pgfpathcurveto{\pgfqpoint{0.892164in}{1.215724in}}{\pgfqpoint{0.884264in}{1.212451in}}{\pgfqpoint{0.878440in}{1.206627in}}%
\pgfpathcurveto{\pgfqpoint{0.872616in}{1.200804in}}{\pgfqpoint{0.869344in}{1.192903in}}{\pgfqpoint{0.869344in}{1.184667in}}%
\pgfpathcurveto{\pgfqpoint{0.869344in}{1.176431in}}{\pgfqpoint{0.872616in}{1.168531in}}{\pgfqpoint{0.878440in}{1.162707in}}%
\pgfpathcurveto{\pgfqpoint{0.884264in}{1.156883in}}{\pgfqpoint{0.892164in}{1.153611in}}{\pgfqpoint{0.900401in}{1.153611in}}%
\pgfpathclose%
\pgfusepath{stroke,fill}%
\end{pgfscope}%
\begin{pgfscope}%
\pgfpathrectangle{\pgfqpoint{0.100000in}{0.212622in}}{\pgfqpoint{3.696000in}{3.696000in}}%
\pgfusepath{clip}%
\pgfsetbuttcap%
\pgfsetroundjoin%
\definecolor{currentfill}{rgb}{0.121569,0.466667,0.705882}%
\pgfsetfillcolor{currentfill}%
\pgfsetfillopacity{0.682722}%
\pgfsetlinewidth{1.003750pt}%
\definecolor{currentstroke}{rgb}{0.121569,0.466667,0.705882}%
\pgfsetstrokecolor{currentstroke}%
\pgfsetstrokeopacity{0.682722}%
\pgfsetdash{}{0pt}%
\pgfpathmoveto{\pgfqpoint{2.251242in}{2.029060in}}%
\pgfpathcurveto{\pgfqpoint{2.259478in}{2.029060in}}{\pgfqpoint{2.267378in}{2.032333in}}{\pgfqpoint{2.273202in}{2.038157in}}%
\pgfpathcurveto{\pgfqpoint{2.279026in}{2.043981in}}{\pgfqpoint{2.282298in}{2.051881in}}{\pgfqpoint{2.282298in}{2.060117in}}%
\pgfpathcurveto{\pgfqpoint{2.282298in}{2.068353in}}{\pgfqpoint{2.279026in}{2.076253in}}{\pgfqpoint{2.273202in}{2.082077in}}%
\pgfpathcurveto{\pgfqpoint{2.267378in}{2.087901in}}{\pgfqpoint{2.259478in}{2.091173in}}{\pgfqpoint{2.251242in}{2.091173in}}%
\pgfpathcurveto{\pgfqpoint{2.243005in}{2.091173in}}{\pgfqpoint{2.235105in}{2.087901in}}{\pgfqpoint{2.229281in}{2.082077in}}%
\pgfpathcurveto{\pgfqpoint{2.223457in}{2.076253in}}{\pgfqpoint{2.220185in}{2.068353in}}{\pgfqpoint{2.220185in}{2.060117in}}%
\pgfpathcurveto{\pgfqpoint{2.220185in}{2.051881in}}{\pgfqpoint{2.223457in}{2.043981in}}{\pgfqpoint{2.229281in}{2.038157in}}%
\pgfpathcurveto{\pgfqpoint{2.235105in}{2.032333in}}{\pgfqpoint{2.243005in}{2.029060in}}{\pgfqpoint{2.251242in}{2.029060in}}%
\pgfpathclose%
\pgfusepath{stroke,fill}%
\end{pgfscope}%
\begin{pgfscope}%
\pgfpathrectangle{\pgfqpoint{0.100000in}{0.212622in}}{\pgfqpoint{3.696000in}{3.696000in}}%
\pgfusepath{clip}%
\pgfsetbuttcap%
\pgfsetroundjoin%
\definecolor{currentfill}{rgb}{0.121569,0.466667,0.705882}%
\pgfsetfillcolor{currentfill}%
\pgfsetfillopacity{0.685103}%
\pgfsetlinewidth{1.003750pt}%
\definecolor{currentstroke}{rgb}{0.121569,0.466667,0.705882}%
\pgfsetstrokecolor{currentstroke}%
\pgfsetstrokeopacity{0.685103}%
\pgfsetdash{}{0pt}%
\pgfpathmoveto{\pgfqpoint{0.912862in}{1.151291in}}%
\pgfpathcurveto{\pgfqpoint{0.921098in}{1.151291in}}{\pgfqpoint{0.928998in}{1.154563in}}{\pgfqpoint{0.934822in}{1.160387in}}%
\pgfpathcurveto{\pgfqpoint{0.940646in}{1.166211in}}{\pgfqpoint{0.943918in}{1.174111in}}{\pgfqpoint{0.943918in}{1.182348in}}%
\pgfpathcurveto{\pgfqpoint{0.943918in}{1.190584in}}{\pgfqpoint{0.940646in}{1.198484in}}{\pgfqpoint{0.934822in}{1.204308in}}%
\pgfpathcurveto{\pgfqpoint{0.928998in}{1.210132in}}{\pgfqpoint{0.921098in}{1.213404in}}{\pgfqpoint{0.912862in}{1.213404in}}%
\pgfpathcurveto{\pgfqpoint{0.904626in}{1.213404in}}{\pgfqpoint{0.896725in}{1.210132in}}{\pgfqpoint{0.890902in}{1.204308in}}%
\pgfpathcurveto{\pgfqpoint{0.885078in}{1.198484in}}{\pgfqpoint{0.881805in}{1.190584in}}{\pgfqpoint{0.881805in}{1.182348in}}%
\pgfpathcurveto{\pgfqpoint{0.881805in}{1.174111in}}{\pgfqpoint{0.885078in}{1.166211in}}{\pgfqpoint{0.890902in}{1.160387in}}%
\pgfpathcurveto{\pgfqpoint{0.896725in}{1.154563in}}{\pgfqpoint{0.904626in}{1.151291in}}{\pgfqpoint{0.912862in}{1.151291in}}%
\pgfpathclose%
\pgfusepath{stroke,fill}%
\end{pgfscope}%
\begin{pgfscope}%
\pgfpathrectangle{\pgfqpoint{0.100000in}{0.212622in}}{\pgfqpoint{3.696000in}{3.696000in}}%
\pgfusepath{clip}%
\pgfsetbuttcap%
\pgfsetroundjoin%
\definecolor{currentfill}{rgb}{0.121569,0.466667,0.705882}%
\pgfsetfillcolor{currentfill}%
\pgfsetfillopacity{0.687344}%
\pgfsetlinewidth{1.003750pt}%
\definecolor{currentstroke}{rgb}{0.121569,0.466667,0.705882}%
\pgfsetstrokecolor{currentstroke}%
\pgfsetstrokeopacity{0.687344}%
\pgfsetdash{}{0pt}%
\pgfpathmoveto{\pgfqpoint{0.923187in}{1.149805in}}%
\pgfpathcurveto{\pgfqpoint{0.931423in}{1.149805in}}{\pgfqpoint{0.939323in}{1.153077in}}{\pgfqpoint{0.945147in}{1.158901in}}%
\pgfpathcurveto{\pgfqpoint{0.950971in}{1.164725in}}{\pgfqpoint{0.954244in}{1.172625in}}{\pgfqpoint{0.954244in}{1.180861in}}%
\pgfpathcurveto{\pgfqpoint{0.954244in}{1.189097in}}{\pgfqpoint{0.950971in}{1.196997in}}{\pgfqpoint{0.945147in}{1.202821in}}%
\pgfpathcurveto{\pgfqpoint{0.939323in}{1.208645in}}{\pgfqpoint{0.931423in}{1.211918in}}{\pgfqpoint{0.923187in}{1.211918in}}%
\pgfpathcurveto{\pgfqpoint{0.914951in}{1.211918in}}{\pgfqpoint{0.907051in}{1.208645in}}{\pgfqpoint{0.901227in}{1.202821in}}%
\pgfpathcurveto{\pgfqpoint{0.895403in}{1.196997in}}{\pgfqpoint{0.892131in}{1.189097in}}{\pgfqpoint{0.892131in}{1.180861in}}%
\pgfpathcurveto{\pgfqpoint{0.892131in}{1.172625in}}{\pgfqpoint{0.895403in}{1.164725in}}{\pgfqpoint{0.901227in}{1.158901in}}%
\pgfpathcurveto{\pgfqpoint{0.907051in}{1.153077in}}{\pgfqpoint{0.914951in}{1.149805in}}{\pgfqpoint{0.923187in}{1.149805in}}%
\pgfpathclose%
\pgfusepath{stroke,fill}%
\end{pgfscope}%
\begin{pgfscope}%
\pgfpathrectangle{\pgfqpoint{0.100000in}{0.212622in}}{\pgfqpoint{3.696000in}{3.696000in}}%
\pgfusepath{clip}%
\pgfsetbuttcap%
\pgfsetroundjoin%
\definecolor{currentfill}{rgb}{0.121569,0.466667,0.705882}%
\pgfsetfillcolor{currentfill}%
\pgfsetfillopacity{0.687738}%
\pgfsetlinewidth{1.003750pt}%
\definecolor{currentstroke}{rgb}{0.121569,0.466667,0.705882}%
\pgfsetstrokecolor{currentstroke}%
\pgfsetstrokeopacity{0.687738}%
\pgfsetdash{}{0pt}%
\pgfpathmoveto{\pgfqpoint{2.256328in}{2.011223in}}%
\pgfpathcurveto{\pgfqpoint{2.264564in}{2.011223in}}{\pgfqpoint{2.272464in}{2.014495in}}{\pgfqpoint{2.278288in}{2.020319in}}%
\pgfpathcurveto{\pgfqpoint{2.284112in}{2.026143in}}{\pgfqpoint{2.287385in}{2.034043in}}{\pgfqpoint{2.287385in}{2.042280in}}%
\pgfpathcurveto{\pgfqpoint{2.287385in}{2.050516in}}{\pgfqpoint{2.284112in}{2.058416in}}{\pgfqpoint{2.278288in}{2.064240in}}%
\pgfpathcurveto{\pgfqpoint{2.272464in}{2.070064in}}{\pgfqpoint{2.264564in}{2.073336in}}{\pgfqpoint{2.256328in}{2.073336in}}%
\pgfpathcurveto{\pgfqpoint{2.248092in}{2.073336in}}{\pgfqpoint{2.240192in}{2.070064in}}{\pgfqpoint{2.234368in}{2.064240in}}%
\pgfpathcurveto{\pgfqpoint{2.228544in}{2.058416in}}{\pgfqpoint{2.225272in}{2.050516in}}{\pgfqpoint{2.225272in}{2.042280in}}%
\pgfpathcurveto{\pgfqpoint{2.225272in}{2.034043in}}{\pgfqpoint{2.228544in}{2.026143in}}{\pgfqpoint{2.234368in}{2.020319in}}%
\pgfpathcurveto{\pgfqpoint{2.240192in}{2.014495in}}{\pgfqpoint{2.248092in}{2.011223in}}{\pgfqpoint{2.256328in}{2.011223in}}%
\pgfpathclose%
\pgfusepath{stroke,fill}%
\end{pgfscope}%
\begin{pgfscope}%
\pgfpathrectangle{\pgfqpoint{0.100000in}{0.212622in}}{\pgfqpoint{3.696000in}{3.696000in}}%
\pgfusepath{clip}%
\pgfsetbuttcap%
\pgfsetroundjoin%
\definecolor{currentfill}{rgb}{0.121569,0.466667,0.705882}%
\pgfsetfillcolor{currentfill}%
\pgfsetfillopacity{0.691488}%
\pgfsetlinewidth{1.003750pt}%
\definecolor{currentstroke}{rgb}{0.121569,0.466667,0.705882}%
\pgfsetstrokecolor{currentstroke}%
\pgfsetstrokeopacity{0.691488}%
\pgfsetdash{}{0pt}%
\pgfpathmoveto{\pgfqpoint{0.941798in}{1.146490in}}%
\pgfpathcurveto{\pgfqpoint{0.950034in}{1.146490in}}{\pgfqpoint{0.957934in}{1.149762in}}{\pgfqpoint{0.963758in}{1.155586in}}%
\pgfpathcurveto{\pgfqpoint{0.969582in}{1.161410in}}{\pgfqpoint{0.972855in}{1.169310in}}{\pgfqpoint{0.972855in}{1.177546in}}%
\pgfpathcurveto{\pgfqpoint{0.972855in}{1.185782in}}{\pgfqpoint{0.969582in}{1.193682in}}{\pgfqpoint{0.963758in}{1.199506in}}%
\pgfpathcurveto{\pgfqpoint{0.957934in}{1.205330in}}{\pgfqpoint{0.950034in}{1.208603in}}{\pgfqpoint{0.941798in}{1.208603in}}%
\pgfpathcurveto{\pgfqpoint{0.933562in}{1.208603in}}{\pgfqpoint{0.925662in}{1.205330in}}{\pgfqpoint{0.919838in}{1.199506in}}%
\pgfpathcurveto{\pgfqpoint{0.914014in}{1.193682in}}{\pgfqpoint{0.910742in}{1.185782in}}{\pgfqpoint{0.910742in}{1.177546in}}%
\pgfpathcurveto{\pgfqpoint{0.910742in}{1.169310in}}{\pgfqpoint{0.914014in}{1.161410in}}{\pgfqpoint{0.919838in}{1.155586in}}%
\pgfpathcurveto{\pgfqpoint{0.925662in}{1.149762in}}{\pgfqpoint{0.933562in}{1.146490in}}{\pgfqpoint{0.941798in}{1.146490in}}%
\pgfpathclose%
\pgfusepath{stroke,fill}%
\end{pgfscope}%
\begin{pgfscope}%
\pgfpathrectangle{\pgfqpoint{0.100000in}{0.212622in}}{\pgfqpoint{3.696000in}{3.696000in}}%
\pgfusepath{clip}%
\pgfsetbuttcap%
\pgfsetroundjoin%
\definecolor{currentfill}{rgb}{0.121569,0.466667,0.705882}%
\pgfsetfillcolor{currentfill}%
\pgfsetfillopacity{0.694486}%
\pgfsetlinewidth{1.003750pt}%
\definecolor{currentstroke}{rgb}{0.121569,0.466667,0.705882}%
\pgfsetstrokecolor{currentstroke}%
\pgfsetstrokeopacity{0.694486}%
\pgfsetdash{}{0pt}%
\pgfpathmoveto{\pgfqpoint{2.261386in}{1.984458in}}%
\pgfpathcurveto{\pgfqpoint{2.269623in}{1.984458in}}{\pgfqpoint{2.277523in}{1.987730in}}{\pgfqpoint{2.283347in}{1.993554in}}%
\pgfpathcurveto{\pgfqpoint{2.289170in}{1.999378in}}{\pgfqpoint{2.292443in}{2.007278in}}{\pgfqpoint{2.292443in}{2.015514in}}%
\pgfpathcurveto{\pgfqpoint{2.292443in}{2.023751in}}{\pgfqpoint{2.289170in}{2.031651in}}{\pgfqpoint{2.283347in}{2.037475in}}%
\pgfpathcurveto{\pgfqpoint{2.277523in}{2.043299in}}{\pgfqpoint{2.269623in}{2.046571in}}{\pgfqpoint{2.261386in}{2.046571in}}%
\pgfpathcurveto{\pgfqpoint{2.253150in}{2.046571in}}{\pgfqpoint{2.245250in}{2.043299in}}{\pgfqpoint{2.239426in}{2.037475in}}%
\pgfpathcurveto{\pgfqpoint{2.233602in}{2.031651in}}{\pgfqpoint{2.230330in}{2.023751in}}{\pgfqpoint{2.230330in}{2.015514in}}%
\pgfpathcurveto{\pgfqpoint{2.230330in}{2.007278in}}{\pgfqpoint{2.233602in}{1.999378in}}{\pgfqpoint{2.239426in}{1.993554in}}%
\pgfpathcurveto{\pgfqpoint{2.245250in}{1.987730in}}{\pgfqpoint{2.253150in}{1.984458in}}{\pgfqpoint{2.261386in}{1.984458in}}%
\pgfpathclose%
\pgfusepath{stroke,fill}%
\end{pgfscope}%
\begin{pgfscope}%
\pgfpathrectangle{\pgfqpoint{0.100000in}{0.212622in}}{\pgfqpoint{3.696000in}{3.696000in}}%
\pgfusepath{clip}%
\pgfsetbuttcap%
\pgfsetroundjoin%
\definecolor{currentfill}{rgb}{0.121569,0.466667,0.705882}%
\pgfsetfillcolor{currentfill}%
\pgfsetfillopacity{0.699195}%
\pgfsetlinewidth{1.003750pt}%
\definecolor{currentstroke}{rgb}{0.121569,0.466667,0.705882}%
\pgfsetstrokecolor{currentstroke}%
\pgfsetstrokeopacity{0.699195}%
\pgfsetdash{}{0pt}%
\pgfpathmoveto{\pgfqpoint{0.975070in}{1.138444in}}%
\pgfpathcurveto{\pgfqpoint{0.983306in}{1.138444in}}{\pgfqpoint{0.991206in}{1.141716in}}{\pgfqpoint{0.997030in}{1.147540in}}%
\pgfpathcurveto{\pgfqpoint{1.002854in}{1.153364in}}{\pgfqpoint{1.006127in}{1.161264in}}{\pgfqpoint{1.006127in}{1.169500in}}%
\pgfpathcurveto{\pgfqpoint{1.006127in}{1.177736in}}{\pgfqpoint{1.002854in}{1.185637in}}{\pgfqpoint{0.997030in}{1.191460in}}%
\pgfpathcurveto{\pgfqpoint{0.991206in}{1.197284in}}{\pgfqpoint{0.983306in}{1.200557in}}{\pgfqpoint{0.975070in}{1.200557in}}%
\pgfpathcurveto{\pgfqpoint{0.966834in}{1.200557in}}{\pgfqpoint{0.958934in}{1.197284in}}{\pgfqpoint{0.953110in}{1.191460in}}%
\pgfpathcurveto{\pgfqpoint{0.947286in}{1.185637in}}{\pgfqpoint{0.944014in}{1.177736in}}{\pgfqpoint{0.944014in}{1.169500in}}%
\pgfpathcurveto{\pgfqpoint{0.944014in}{1.161264in}}{\pgfqpoint{0.947286in}{1.153364in}}{\pgfqpoint{0.953110in}{1.147540in}}%
\pgfpathcurveto{\pgfqpoint{0.958934in}{1.141716in}}{\pgfqpoint{0.966834in}{1.138444in}}{\pgfqpoint{0.975070in}{1.138444in}}%
\pgfpathclose%
\pgfusepath{stroke,fill}%
\end{pgfscope}%
\begin{pgfscope}%
\pgfpathrectangle{\pgfqpoint{0.100000in}{0.212622in}}{\pgfqpoint{3.696000in}{3.696000in}}%
\pgfusepath{clip}%
\pgfsetbuttcap%
\pgfsetroundjoin%
\definecolor{currentfill}{rgb}{0.121569,0.466667,0.705882}%
\pgfsetfillcolor{currentfill}%
\pgfsetfillopacity{0.702703}%
\pgfsetlinewidth{1.003750pt}%
\definecolor{currentstroke}{rgb}{0.121569,0.466667,0.705882}%
\pgfsetstrokecolor{currentstroke}%
\pgfsetstrokeopacity{0.702703}%
\pgfsetdash{}{0pt}%
\pgfpathmoveto{\pgfqpoint{2.269286in}{1.953292in}}%
\pgfpathcurveto{\pgfqpoint{2.277523in}{1.953292in}}{\pgfqpoint{2.285423in}{1.956564in}}{\pgfqpoint{2.291247in}{1.962388in}}%
\pgfpathcurveto{\pgfqpoint{2.297071in}{1.968212in}}{\pgfqpoint{2.300343in}{1.976112in}}{\pgfqpoint{2.300343in}{1.984349in}}%
\pgfpathcurveto{\pgfqpoint{2.300343in}{1.992585in}}{\pgfqpoint{2.297071in}{2.000485in}}{\pgfqpoint{2.291247in}{2.006309in}}%
\pgfpathcurveto{\pgfqpoint{2.285423in}{2.012133in}}{\pgfqpoint{2.277523in}{2.015405in}}{\pgfqpoint{2.269286in}{2.015405in}}%
\pgfpathcurveto{\pgfqpoint{2.261050in}{2.015405in}}{\pgfqpoint{2.253150in}{2.012133in}}{\pgfqpoint{2.247326in}{2.006309in}}%
\pgfpathcurveto{\pgfqpoint{2.241502in}{2.000485in}}{\pgfqpoint{2.238230in}{1.992585in}}{\pgfqpoint{2.238230in}{1.984349in}}%
\pgfpathcurveto{\pgfqpoint{2.238230in}{1.976112in}}{\pgfqpoint{2.241502in}{1.968212in}}{\pgfqpoint{2.247326in}{1.962388in}}%
\pgfpathcurveto{\pgfqpoint{2.253150in}{1.956564in}}{\pgfqpoint{2.261050in}{1.953292in}}{\pgfqpoint{2.269286in}{1.953292in}}%
\pgfpathclose%
\pgfusepath{stroke,fill}%
\end{pgfscope}%
\begin{pgfscope}%
\pgfpathrectangle{\pgfqpoint{0.100000in}{0.212622in}}{\pgfqpoint{3.696000in}{3.696000in}}%
\pgfusepath{clip}%
\pgfsetbuttcap%
\pgfsetroundjoin%
\definecolor{currentfill}{rgb}{0.121569,0.466667,0.705882}%
\pgfsetfillcolor{currentfill}%
\pgfsetfillopacity{0.706435}%
\pgfsetlinewidth{1.003750pt}%
\definecolor{currentstroke}{rgb}{0.121569,0.466667,0.705882}%
\pgfsetstrokecolor{currentstroke}%
\pgfsetstrokeopacity{0.706435}%
\pgfsetdash{}{0pt}%
\pgfpathmoveto{\pgfqpoint{1.006504in}{1.131284in}}%
\pgfpathcurveto{\pgfqpoint{1.014740in}{1.131284in}}{\pgfqpoint{1.022640in}{1.134557in}}{\pgfqpoint{1.028464in}{1.140380in}}%
\pgfpathcurveto{\pgfqpoint{1.034288in}{1.146204in}}{\pgfqpoint{1.037560in}{1.154104in}}{\pgfqpoint{1.037560in}{1.162341in}}%
\pgfpathcurveto{\pgfqpoint{1.037560in}{1.170577in}}{\pgfqpoint{1.034288in}{1.178477in}}{\pgfqpoint{1.028464in}{1.184301in}}%
\pgfpathcurveto{\pgfqpoint{1.022640in}{1.190125in}}{\pgfqpoint{1.014740in}{1.193397in}}{\pgfqpoint{1.006504in}{1.193397in}}%
\pgfpathcurveto{\pgfqpoint{0.998267in}{1.193397in}}{\pgfqpoint{0.990367in}{1.190125in}}{\pgfqpoint{0.984543in}{1.184301in}}%
\pgfpathcurveto{\pgfqpoint{0.978720in}{1.178477in}}{\pgfqpoint{0.975447in}{1.170577in}}{\pgfqpoint{0.975447in}{1.162341in}}%
\pgfpathcurveto{\pgfqpoint{0.975447in}{1.154104in}}{\pgfqpoint{0.978720in}{1.146204in}}{\pgfqpoint{0.984543in}{1.140380in}}%
\pgfpathcurveto{\pgfqpoint{0.990367in}{1.134557in}}{\pgfqpoint{0.998267in}{1.131284in}}{\pgfqpoint{1.006504in}{1.131284in}}%
\pgfpathclose%
\pgfusepath{stroke,fill}%
\end{pgfscope}%
\begin{pgfscope}%
\pgfpathrectangle{\pgfqpoint{0.100000in}{0.212622in}}{\pgfqpoint{3.696000in}{3.696000in}}%
\pgfusepath{clip}%
\pgfsetbuttcap%
\pgfsetroundjoin%
\definecolor{currentfill}{rgb}{0.121569,0.466667,0.705882}%
\pgfsetfillcolor{currentfill}%
\pgfsetfillopacity{0.712518}%
\pgfsetlinewidth{1.003750pt}%
\definecolor{currentstroke}{rgb}{0.121569,0.466667,0.705882}%
\pgfsetstrokecolor{currentstroke}%
\pgfsetstrokeopacity{0.712518}%
\pgfsetdash{}{0pt}%
\pgfpathmoveto{\pgfqpoint{2.278192in}{1.915634in}}%
\pgfpathcurveto{\pgfqpoint{2.286428in}{1.915634in}}{\pgfqpoint{2.294328in}{1.918906in}}{\pgfqpoint{2.300152in}{1.924730in}}%
\pgfpathcurveto{\pgfqpoint{2.305976in}{1.930554in}}{\pgfqpoint{2.309249in}{1.938454in}}{\pgfqpoint{2.309249in}{1.946690in}}%
\pgfpathcurveto{\pgfqpoint{2.309249in}{1.954926in}}{\pgfqpoint{2.305976in}{1.962827in}}{\pgfqpoint{2.300152in}{1.968650in}}%
\pgfpathcurveto{\pgfqpoint{2.294328in}{1.974474in}}{\pgfqpoint{2.286428in}{1.977747in}}{\pgfqpoint{2.278192in}{1.977747in}}%
\pgfpathcurveto{\pgfqpoint{2.269956in}{1.977747in}}{\pgfqpoint{2.262056in}{1.974474in}}{\pgfqpoint{2.256232in}{1.968650in}}%
\pgfpathcurveto{\pgfqpoint{2.250408in}{1.962827in}}{\pgfqpoint{2.247136in}{1.954926in}}{\pgfqpoint{2.247136in}{1.946690in}}%
\pgfpathcurveto{\pgfqpoint{2.247136in}{1.938454in}}{\pgfqpoint{2.250408in}{1.930554in}}{\pgfqpoint{2.256232in}{1.924730in}}%
\pgfpathcurveto{\pgfqpoint{2.262056in}{1.918906in}}{\pgfqpoint{2.269956in}{1.915634in}}{\pgfqpoint{2.278192in}{1.915634in}}%
\pgfpathclose%
\pgfusepath{stroke,fill}%
\end{pgfscope}%
\begin{pgfscope}%
\pgfpathrectangle{\pgfqpoint{0.100000in}{0.212622in}}{\pgfqpoint{3.696000in}{3.696000in}}%
\pgfusepath{clip}%
\pgfsetbuttcap%
\pgfsetroundjoin%
\definecolor{currentfill}{rgb}{0.121569,0.466667,0.705882}%
\pgfsetfillcolor{currentfill}%
\pgfsetfillopacity{0.713307}%
\pgfsetlinewidth{1.003750pt}%
\definecolor{currentstroke}{rgb}{0.121569,0.466667,0.705882}%
\pgfsetstrokecolor{currentstroke}%
\pgfsetstrokeopacity{0.713307}%
\pgfsetdash{}{0pt}%
\pgfpathmoveto{\pgfqpoint{1.035273in}{1.123211in}}%
\pgfpathcurveto{\pgfqpoint{1.043509in}{1.123211in}}{\pgfqpoint{1.051409in}{1.126483in}}{\pgfqpoint{1.057233in}{1.132307in}}%
\pgfpathcurveto{\pgfqpoint{1.063057in}{1.138131in}}{\pgfqpoint{1.066330in}{1.146031in}}{\pgfqpoint{1.066330in}{1.154267in}}%
\pgfpathcurveto{\pgfqpoint{1.066330in}{1.162504in}}{\pgfqpoint{1.063057in}{1.170404in}}{\pgfqpoint{1.057233in}{1.176228in}}%
\pgfpathcurveto{\pgfqpoint{1.051409in}{1.182052in}}{\pgfqpoint{1.043509in}{1.185324in}}{\pgfqpoint{1.035273in}{1.185324in}}%
\pgfpathcurveto{\pgfqpoint{1.027037in}{1.185324in}}{\pgfqpoint{1.019137in}{1.182052in}}{\pgfqpoint{1.013313in}{1.176228in}}%
\pgfpathcurveto{\pgfqpoint{1.007489in}{1.170404in}}{\pgfqpoint{1.004217in}{1.162504in}}{\pgfqpoint{1.004217in}{1.154267in}}%
\pgfpathcurveto{\pgfqpoint{1.004217in}{1.146031in}}{\pgfqpoint{1.007489in}{1.138131in}}{\pgfqpoint{1.013313in}{1.132307in}}%
\pgfpathcurveto{\pgfqpoint{1.019137in}{1.126483in}}{\pgfqpoint{1.027037in}{1.123211in}}{\pgfqpoint{1.035273in}{1.123211in}}%
\pgfpathclose%
\pgfusepath{stroke,fill}%
\end{pgfscope}%
\begin{pgfscope}%
\pgfpathrectangle{\pgfqpoint{0.100000in}{0.212622in}}{\pgfqpoint{3.696000in}{3.696000in}}%
\pgfusepath{clip}%
\pgfsetbuttcap%
\pgfsetroundjoin%
\definecolor{currentfill}{rgb}{0.121569,0.466667,0.705882}%
\pgfsetfillcolor{currentfill}%
\pgfsetfillopacity{0.719249}%
\pgfsetlinewidth{1.003750pt}%
\definecolor{currentstroke}{rgb}{0.121569,0.466667,0.705882}%
\pgfsetstrokecolor{currentstroke}%
\pgfsetstrokeopacity{0.719249}%
\pgfsetdash{}{0pt}%
\pgfpathmoveto{\pgfqpoint{1.061894in}{1.118192in}}%
\pgfpathcurveto{\pgfqpoint{1.070130in}{1.118192in}}{\pgfqpoint{1.078030in}{1.121464in}}{\pgfqpoint{1.083854in}{1.127288in}}%
\pgfpathcurveto{\pgfqpoint{1.089678in}{1.133112in}}{\pgfqpoint{1.092951in}{1.141012in}}{\pgfqpoint{1.092951in}{1.149249in}}%
\pgfpathcurveto{\pgfqpoint{1.092951in}{1.157485in}}{\pgfqpoint{1.089678in}{1.165385in}}{\pgfqpoint{1.083854in}{1.171209in}}%
\pgfpathcurveto{\pgfqpoint{1.078030in}{1.177033in}}{\pgfqpoint{1.070130in}{1.180305in}}{\pgfqpoint{1.061894in}{1.180305in}}%
\pgfpathcurveto{\pgfqpoint{1.053658in}{1.180305in}}{\pgfqpoint{1.045758in}{1.177033in}}{\pgfqpoint{1.039934in}{1.171209in}}%
\pgfpathcurveto{\pgfqpoint{1.034110in}{1.165385in}}{\pgfqpoint{1.030838in}{1.157485in}}{\pgfqpoint{1.030838in}{1.149249in}}%
\pgfpathcurveto{\pgfqpoint{1.030838in}{1.141012in}}{\pgfqpoint{1.034110in}{1.133112in}}{\pgfqpoint{1.039934in}{1.127288in}}%
\pgfpathcurveto{\pgfqpoint{1.045758in}{1.121464in}}{\pgfqpoint{1.053658in}{1.118192in}}{\pgfqpoint{1.061894in}{1.118192in}}%
\pgfpathclose%
\pgfusepath{stroke,fill}%
\end{pgfscope}%
\begin{pgfscope}%
\pgfpathrectangle{\pgfqpoint{0.100000in}{0.212622in}}{\pgfqpoint{3.696000in}{3.696000in}}%
\pgfusepath{clip}%
\pgfsetbuttcap%
\pgfsetroundjoin%
\definecolor{currentfill}{rgb}{0.121569,0.466667,0.705882}%
\pgfsetfillcolor{currentfill}%
\pgfsetfillopacity{0.723617}%
\pgfsetlinewidth{1.003750pt}%
\definecolor{currentstroke}{rgb}{0.121569,0.466667,0.705882}%
\pgfsetstrokecolor{currentstroke}%
\pgfsetstrokeopacity{0.723617}%
\pgfsetdash{}{0pt}%
\pgfpathmoveto{\pgfqpoint{2.290825in}{1.877178in}}%
\pgfpathcurveto{\pgfqpoint{2.299062in}{1.877178in}}{\pgfqpoint{2.306962in}{1.880450in}}{\pgfqpoint{2.312785in}{1.886274in}}%
\pgfpathcurveto{\pgfqpoint{2.318609in}{1.892098in}}{\pgfqpoint{2.321882in}{1.899998in}}{\pgfqpoint{2.321882in}{1.908234in}}%
\pgfpathcurveto{\pgfqpoint{2.321882in}{1.916470in}}{\pgfqpoint{2.318609in}{1.924371in}}{\pgfqpoint{2.312785in}{1.930194in}}%
\pgfpathcurveto{\pgfqpoint{2.306962in}{1.936018in}}{\pgfqpoint{2.299062in}{1.939291in}}{\pgfqpoint{2.290825in}{1.939291in}}%
\pgfpathcurveto{\pgfqpoint{2.282589in}{1.939291in}}{\pgfqpoint{2.274689in}{1.936018in}}{\pgfqpoint{2.268865in}{1.930194in}}%
\pgfpathcurveto{\pgfqpoint{2.263041in}{1.924371in}}{\pgfqpoint{2.259769in}{1.916470in}}{\pgfqpoint{2.259769in}{1.908234in}}%
\pgfpathcurveto{\pgfqpoint{2.259769in}{1.899998in}}{\pgfqpoint{2.263041in}{1.892098in}}{\pgfqpoint{2.268865in}{1.886274in}}%
\pgfpathcurveto{\pgfqpoint{2.274689in}{1.880450in}}{\pgfqpoint{2.282589in}{1.877178in}}{\pgfqpoint{2.290825in}{1.877178in}}%
\pgfpathclose%
\pgfusepath{stroke,fill}%
\end{pgfscope}%
\begin{pgfscope}%
\pgfpathrectangle{\pgfqpoint{0.100000in}{0.212622in}}{\pgfqpoint{3.696000in}{3.696000in}}%
\pgfusepath{clip}%
\pgfsetbuttcap%
\pgfsetroundjoin%
\definecolor{currentfill}{rgb}{0.121569,0.466667,0.705882}%
\pgfsetfillcolor{currentfill}%
\pgfsetfillopacity{0.723840}%
\pgfsetlinewidth{1.003750pt}%
\definecolor{currentstroke}{rgb}{0.121569,0.466667,0.705882}%
\pgfsetstrokecolor{currentstroke}%
\pgfsetstrokeopacity{0.723840}%
\pgfsetdash{}{0pt}%
\pgfpathmoveto{\pgfqpoint{1.082196in}{1.114321in}}%
\pgfpathcurveto{\pgfqpoint{1.090432in}{1.114321in}}{\pgfqpoint{1.098332in}{1.117593in}}{\pgfqpoint{1.104156in}{1.123417in}}%
\pgfpathcurveto{\pgfqpoint{1.109980in}{1.129241in}}{\pgfqpoint{1.113252in}{1.137141in}}{\pgfqpoint{1.113252in}{1.145377in}}%
\pgfpathcurveto{\pgfqpoint{1.113252in}{1.153614in}}{\pgfqpoint{1.109980in}{1.161514in}}{\pgfqpoint{1.104156in}{1.167338in}}%
\pgfpathcurveto{\pgfqpoint{1.098332in}{1.173162in}}{\pgfqpoint{1.090432in}{1.176434in}}{\pgfqpoint{1.082196in}{1.176434in}}%
\pgfpathcurveto{\pgfqpoint{1.073959in}{1.176434in}}{\pgfqpoint{1.066059in}{1.173162in}}{\pgfqpoint{1.060235in}{1.167338in}}%
\pgfpathcurveto{\pgfqpoint{1.054412in}{1.161514in}}{\pgfqpoint{1.051139in}{1.153614in}}{\pgfqpoint{1.051139in}{1.145377in}}%
\pgfpathcurveto{\pgfqpoint{1.051139in}{1.137141in}}{\pgfqpoint{1.054412in}{1.129241in}}{\pgfqpoint{1.060235in}{1.123417in}}%
\pgfpathcurveto{\pgfqpoint{1.066059in}{1.117593in}}{\pgfqpoint{1.073959in}{1.114321in}}{\pgfqpoint{1.082196in}{1.114321in}}%
\pgfpathclose%
\pgfusepath{stroke,fill}%
\end{pgfscope}%
\begin{pgfscope}%
\pgfpathrectangle{\pgfqpoint{0.100000in}{0.212622in}}{\pgfqpoint{3.696000in}{3.696000in}}%
\pgfusepath{clip}%
\pgfsetbuttcap%
\pgfsetroundjoin%
\definecolor{currentfill}{rgb}{0.121569,0.466667,0.705882}%
\pgfsetfillcolor{currentfill}%
\pgfsetfillopacity{0.728002}%
\pgfsetlinewidth{1.003750pt}%
\definecolor{currentstroke}{rgb}{0.121569,0.466667,0.705882}%
\pgfsetstrokecolor{currentstroke}%
\pgfsetstrokeopacity{0.728002}%
\pgfsetdash{}{0pt}%
\pgfpathmoveto{\pgfqpoint{1.099805in}{1.109907in}}%
\pgfpathcurveto{\pgfqpoint{1.108041in}{1.109907in}}{\pgfqpoint{1.115941in}{1.113179in}}{\pgfqpoint{1.121765in}{1.119003in}}%
\pgfpathcurveto{\pgfqpoint{1.127589in}{1.124827in}}{\pgfqpoint{1.130861in}{1.132727in}}{\pgfqpoint{1.130861in}{1.140964in}}%
\pgfpathcurveto{\pgfqpoint{1.130861in}{1.149200in}}{\pgfqpoint{1.127589in}{1.157100in}}{\pgfqpoint{1.121765in}{1.162924in}}%
\pgfpathcurveto{\pgfqpoint{1.115941in}{1.168748in}}{\pgfqpoint{1.108041in}{1.172020in}}{\pgfqpoint{1.099805in}{1.172020in}}%
\pgfpathcurveto{\pgfqpoint{1.091569in}{1.172020in}}{\pgfqpoint{1.083669in}{1.168748in}}{\pgfqpoint{1.077845in}{1.162924in}}%
\pgfpathcurveto{\pgfqpoint{1.072021in}{1.157100in}}{\pgfqpoint{1.068748in}{1.149200in}}{\pgfqpoint{1.068748in}{1.140964in}}%
\pgfpathcurveto{\pgfqpoint{1.068748in}{1.132727in}}{\pgfqpoint{1.072021in}{1.124827in}}{\pgfqpoint{1.077845in}{1.119003in}}%
\pgfpathcurveto{\pgfqpoint{1.083669in}{1.113179in}}{\pgfqpoint{1.091569in}{1.109907in}}{\pgfqpoint{1.099805in}{1.109907in}}%
\pgfpathclose%
\pgfusepath{stroke,fill}%
\end{pgfscope}%
\begin{pgfscope}%
\pgfpathrectangle{\pgfqpoint{0.100000in}{0.212622in}}{\pgfqpoint{3.696000in}{3.696000in}}%
\pgfusepath{clip}%
\pgfsetbuttcap%
\pgfsetroundjoin%
\definecolor{currentfill}{rgb}{0.121569,0.466667,0.705882}%
\pgfsetfillcolor{currentfill}%
\pgfsetfillopacity{0.735282}%
\pgfsetlinewidth{1.003750pt}%
\definecolor{currentstroke}{rgb}{0.121569,0.466667,0.705882}%
\pgfsetstrokecolor{currentstroke}%
\pgfsetstrokeopacity{0.735282}%
\pgfsetdash{}{0pt}%
\pgfpathmoveto{\pgfqpoint{2.301074in}{1.832631in}}%
\pgfpathcurveto{\pgfqpoint{2.309311in}{1.832631in}}{\pgfqpoint{2.317211in}{1.835903in}}{\pgfqpoint{2.323035in}{1.841727in}}%
\pgfpathcurveto{\pgfqpoint{2.328859in}{1.847551in}}{\pgfqpoint{2.332131in}{1.855451in}}{\pgfqpoint{2.332131in}{1.863687in}}%
\pgfpathcurveto{\pgfqpoint{2.332131in}{1.871924in}}{\pgfqpoint{2.328859in}{1.879824in}}{\pgfqpoint{2.323035in}{1.885647in}}%
\pgfpathcurveto{\pgfqpoint{2.317211in}{1.891471in}}{\pgfqpoint{2.309311in}{1.894744in}}{\pgfqpoint{2.301074in}{1.894744in}}%
\pgfpathcurveto{\pgfqpoint{2.292838in}{1.894744in}}{\pgfqpoint{2.284938in}{1.891471in}}{\pgfqpoint{2.279114in}{1.885647in}}%
\pgfpathcurveto{\pgfqpoint{2.273290in}{1.879824in}}{\pgfqpoint{2.270018in}{1.871924in}}{\pgfqpoint{2.270018in}{1.863687in}}%
\pgfpathcurveto{\pgfqpoint{2.270018in}{1.855451in}}{\pgfqpoint{2.273290in}{1.847551in}}{\pgfqpoint{2.279114in}{1.841727in}}%
\pgfpathcurveto{\pgfqpoint{2.284938in}{1.835903in}}{\pgfqpoint{2.292838in}{1.832631in}}{\pgfqpoint{2.301074in}{1.832631in}}%
\pgfpathclose%
\pgfusepath{stroke,fill}%
\end{pgfscope}%
\begin{pgfscope}%
\pgfpathrectangle{\pgfqpoint{0.100000in}{0.212622in}}{\pgfqpoint{3.696000in}{3.696000in}}%
\pgfusepath{clip}%
\pgfsetbuttcap%
\pgfsetroundjoin%
\definecolor{currentfill}{rgb}{0.121569,0.466667,0.705882}%
\pgfsetfillcolor{currentfill}%
\pgfsetfillopacity{0.735542}%
\pgfsetlinewidth{1.003750pt}%
\definecolor{currentstroke}{rgb}{0.121569,0.466667,0.705882}%
\pgfsetstrokecolor{currentstroke}%
\pgfsetstrokeopacity{0.735542}%
\pgfsetdash{}{0pt}%
\pgfpathmoveto{\pgfqpoint{1.131878in}{1.101784in}}%
\pgfpathcurveto{\pgfqpoint{1.140114in}{1.101784in}}{\pgfqpoint{1.148014in}{1.105057in}}{\pgfqpoint{1.153838in}{1.110881in}}%
\pgfpathcurveto{\pgfqpoint{1.159662in}{1.116704in}}{\pgfqpoint{1.162934in}{1.124605in}}{\pgfqpoint{1.162934in}{1.132841in}}%
\pgfpathcurveto{\pgfqpoint{1.162934in}{1.141077in}}{\pgfqpoint{1.159662in}{1.148977in}}{\pgfqpoint{1.153838in}{1.154801in}}%
\pgfpathcurveto{\pgfqpoint{1.148014in}{1.160625in}}{\pgfqpoint{1.140114in}{1.163897in}}{\pgfqpoint{1.131878in}{1.163897in}}%
\pgfpathcurveto{\pgfqpoint{1.123641in}{1.163897in}}{\pgfqpoint{1.115741in}{1.160625in}}{\pgfqpoint{1.109917in}{1.154801in}}%
\pgfpathcurveto{\pgfqpoint{1.104093in}{1.148977in}}{\pgfqpoint{1.100821in}{1.141077in}}{\pgfqpoint{1.100821in}{1.132841in}}%
\pgfpathcurveto{\pgfqpoint{1.100821in}{1.124605in}}{\pgfqpoint{1.104093in}{1.116704in}}{\pgfqpoint{1.109917in}{1.110881in}}%
\pgfpathcurveto{\pgfqpoint{1.115741in}{1.105057in}}{\pgfqpoint{1.123641in}{1.101784in}}{\pgfqpoint{1.131878in}{1.101784in}}%
\pgfpathclose%
\pgfusepath{stroke,fill}%
\end{pgfscope}%
\begin{pgfscope}%
\pgfpathrectangle{\pgfqpoint{0.100000in}{0.212622in}}{\pgfqpoint{3.696000in}{3.696000in}}%
\pgfusepath{clip}%
\pgfsetbuttcap%
\pgfsetroundjoin%
\definecolor{currentfill}{rgb}{0.121569,0.466667,0.705882}%
\pgfsetfillcolor{currentfill}%
\pgfsetfillopacity{0.742005}%
\pgfsetlinewidth{1.003750pt}%
\definecolor{currentstroke}{rgb}{0.121569,0.466667,0.705882}%
\pgfsetstrokecolor{currentstroke}%
\pgfsetstrokeopacity{0.742005}%
\pgfsetdash{}{0pt}%
\pgfpathmoveto{\pgfqpoint{2.308332in}{1.809041in}}%
\pgfpathcurveto{\pgfqpoint{2.316568in}{1.809041in}}{\pgfqpoint{2.324468in}{1.812314in}}{\pgfqpoint{2.330292in}{1.818138in}}%
\pgfpathcurveto{\pgfqpoint{2.336116in}{1.823962in}}{\pgfqpoint{2.339388in}{1.831862in}}{\pgfqpoint{2.339388in}{1.840098in}}%
\pgfpathcurveto{\pgfqpoint{2.339388in}{1.848334in}}{\pgfqpoint{2.336116in}{1.856234in}}{\pgfqpoint{2.330292in}{1.862058in}}%
\pgfpathcurveto{\pgfqpoint{2.324468in}{1.867882in}}{\pgfqpoint{2.316568in}{1.871154in}}{\pgfqpoint{2.308332in}{1.871154in}}%
\pgfpathcurveto{\pgfqpoint{2.300095in}{1.871154in}}{\pgfqpoint{2.292195in}{1.867882in}}{\pgfqpoint{2.286372in}{1.862058in}}%
\pgfpathcurveto{\pgfqpoint{2.280548in}{1.856234in}}{\pgfqpoint{2.277275in}{1.848334in}}{\pgfqpoint{2.277275in}{1.840098in}}%
\pgfpathcurveto{\pgfqpoint{2.277275in}{1.831862in}}{\pgfqpoint{2.280548in}{1.823962in}}{\pgfqpoint{2.286372in}{1.818138in}}%
\pgfpathcurveto{\pgfqpoint{2.292195in}{1.812314in}}{\pgfqpoint{2.300095in}{1.809041in}}{\pgfqpoint{2.308332in}{1.809041in}}%
\pgfpathclose%
\pgfusepath{stroke,fill}%
\end{pgfscope}%
\begin{pgfscope}%
\pgfpathrectangle{\pgfqpoint{0.100000in}{0.212622in}}{\pgfqpoint{3.696000in}{3.696000in}}%
\pgfusepath{clip}%
\pgfsetbuttcap%
\pgfsetroundjoin%
\definecolor{currentfill}{rgb}{0.121569,0.466667,0.705882}%
\pgfsetfillcolor{currentfill}%
\pgfsetfillopacity{0.742386}%
\pgfsetlinewidth{1.003750pt}%
\definecolor{currentstroke}{rgb}{0.121569,0.466667,0.705882}%
\pgfsetstrokecolor{currentstroke}%
\pgfsetstrokeopacity{0.742386}%
\pgfsetdash{}{0pt}%
\pgfpathmoveto{\pgfqpoint{1.161678in}{1.094550in}}%
\pgfpathcurveto{\pgfqpoint{1.169914in}{1.094550in}}{\pgfqpoint{1.177814in}{1.097822in}}{\pgfqpoint{1.183638in}{1.103646in}}%
\pgfpathcurveto{\pgfqpoint{1.189462in}{1.109470in}}{\pgfqpoint{1.192734in}{1.117370in}}{\pgfqpoint{1.192734in}{1.125606in}}%
\pgfpathcurveto{\pgfqpoint{1.192734in}{1.133843in}}{\pgfqpoint{1.189462in}{1.141743in}}{\pgfqpoint{1.183638in}{1.147567in}}%
\pgfpathcurveto{\pgfqpoint{1.177814in}{1.153390in}}{\pgfqpoint{1.169914in}{1.156663in}}{\pgfqpoint{1.161678in}{1.156663in}}%
\pgfpathcurveto{\pgfqpoint{1.153442in}{1.156663in}}{\pgfqpoint{1.145542in}{1.153390in}}{\pgfqpoint{1.139718in}{1.147567in}}%
\pgfpathcurveto{\pgfqpoint{1.133894in}{1.141743in}}{\pgfqpoint{1.130621in}{1.133843in}}{\pgfqpoint{1.130621in}{1.125606in}}%
\pgfpathcurveto{\pgfqpoint{1.130621in}{1.117370in}}{\pgfqpoint{1.133894in}{1.109470in}}{\pgfqpoint{1.139718in}{1.103646in}}%
\pgfpathcurveto{\pgfqpoint{1.145542in}{1.097822in}}{\pgfqpoint{1.153442in}{1.094550in}}{\pgfqpoint{1.161678in}{1.094550in}}%
\pgfpathclose%
\pgfusepath{stroke,fill}%
\end{pgfscope}%
\begin{pgfscope}%
\pgfpathrectangle{\pgfqpoint{0.100000in}{0.212622in}}{\pgfqpoint{3.696000in}{3.696000in}}%
\pgfusepath{clip}%
\pgfsetbuttcap%
\pgfsetroundjoin%
\definecolor{currentfill}{rgb}{0.121569,0.466667,0.705882}%
\pgfsetfillcolor{currentfill}%
\pgfsetfillopacity{0.747835}%
\pgfsetlinewidth{1.003750pt}%
\definecolor{currentstroke}{rgb}{0.121569,0.466667,0.705882}%
\pgfsetstrokecolor{currentstroke}%
\pgfsetstrokeopacity{0.747835}%
\pgfsetdash{}{0pt}%
\pgfpathmoveto{\pgfqpoint{1.185358in}{1.088279in}}%
\pgfpathcurveto{\pgfqpoint{1.193594in}{1.088279in}}{\pgfqpoint{1.201495in}{1.091552in}}{\pgfqpoint{1.207318in}{1.097376in}}%
\pgfpathcurveto{\pgfqpoint{1.213142in}{1.103199in}}{\pgfqpoint{1.216415in}{1.111100in}}{\pgfqpoint{1.216415in}{1.119336in}}%
\pgfpathcurveto{\pgfqpoint{1.216415in}{1.127572in}}{\pgfqpoint{1.213142in}{1.135472in}}{\pgfqpoint{1.207318in}{1.141296in}}%
\pgfpathcurveto{\pgfqpoint{1.201495in}{1.147120in}}{\pgfqpoint{1.193594in}{1.150392in}}{\pgfqpoint{1.185358in}{1.150392in}}%
\pgfpathcurveto{\pgfqpoint{1.177122in}{1.150392in}}{\pgfqpoint{1.169222in}{1.147120in}}{\pgfqpoint{1.163398in}{1.141296in}}%
\pgfpathcurveto{\pgfqpoint{1.157574in}{1.135472in}}{\pgfqpoint{1.154302in}{1.127572in}}{\pgfqpoint{1.154302in}{1.119336in}}%
\pgfpathcurveto{\pgfqpoint{1.154302in}{1.111100in}}{\pgfqpoint{1.157574in}{1.103199in}}{\pgfqpoint{1.163398in}{1.097376in}}%
\pgfpathcurveto{\pgfqpoint{1.169222in}{1.091552in}}{\pgfqpoint{1.177122in}{1.088279in}}{\pgfqpoint{1.185358in}{1.088279in}}%
\pgfpathclose%
\pgfusepath{stroke,fill}%
\end{pgfscope}%
\begin{pgfscope}%
\pgfpathrectangle{\pgfqpoint{0.100000in}{0.212622in}}{\pgfqpoint{3.696000in}{3.696000in}}%
\pgfusepath{clip}%
\pgfsetbuttcap%
\pgfsetroundjoin%
\definecolor{currentfill}{rgb}{0.121569,0.466667,0.705882}%
\pgfsetfillcolor{currentfill}%
\pgfsetfillopacity{0.749745}%
\pgfsetlinewidth{1.003750pt}%
\definecolor{currentstroke}{rgb}{0.121569,0.466667,0.705882}%
\pgfsetstrokecolor{currentstroke}%
\pgfsetstrokeopacity{0.749745}%
\pgfsetdash{}{0pt}%
\pgfpathmoveto{\pgfqpoint{2.315268in}{1.780098in}}%
\pgfpathcurveto{\pgfqpoint{2.323504in}{1.780098in}}{\pgfqpoint{2.331404in}{1.783370in}}{\pgfqpoint{2.337228in}{1.789194in}}%
\pgfpathcurveto{\pgfqpoint{2.343052in}{1.795018in}}{\pgfqpoint{2.346324in}{1.802918in}}{\pgfqpoint{2.346324in}{1.811154in}}%
\pgfpathcurveto{\pgfqpoint{2.346324in}{1.819391in}}{\pgfqpoint{2.343052in}{1.827291in}}{\pgfqpoint{2.337228in}{1.833115in}}%
\pgfpathcurveto{\pgfqpoint{2.331404in}{1.838938in}}{\pgfqpoint{2.323504in}{1.842211in}}{\pgfqpoint{2.315268in}{1.842211in}}%
\pgfpathcurveto{\pgfqpoint{2.307032in}{1.842211in}}{\pgfqpoint{2.299132in}{1.838938in}}{\pgfqpoint{2.293308in}{1.833115in}}%
\pgfpathcurveto{\pgfqpoint{2.287484in}{1.827291in}}{\pgfqpoint{2.284211in}{1.819391in}}{\pgfqpoint{2.284211in}{1.811154in}}%
\pgfpathcurveto{\pgfqpoint{2.284211in}{1.802918in}}{\pgfqpoint{2.287484in}{1.795018in}}{\pgfqpoint{2.293308in}{1.789194in}}%
\pgfpathcurveto{\pgfqpoint{2.299132in}{1.783370in}}{\pgfqpoint{2.307032in}{1.780098in}}{\pgfqpoint{2.315268in}{1.780098in}}%
\pgfpathclose%
\pgfusepath{stroke,fill}%
\end{pgfscope}%
\begin{pgfscope}%
\pgfpathrectangle{\pgfqpoint{0.100000in}{0.212622in}}{\pgfqpoint{3.696000in}{3.696000in}}%
\pgfusepath{clip}%
\pgfsetbuttcap%
\pgfsetroundjoin%
\definecolor{currentfill}{rgb}{0.121569,0.466667,0.705882}%
\pgfsetfillcolor{currentfill}%
\pgfsetfillopacity{0.752150}%
\pgfsetlinewidth{1.003750pt}%
\definecolor{currentstroke}{rgb}{0.121569,0.466667,0.705882}%
\pgfsetstrokecolor{currentstroke}%
\pgfsetstrokeopacity{0.752150}%
\pgfsetdash{}{0pt}%
\pgfpathmoveto{\pgfqpoint{1.204655in}{1.084328in}}%
\pgfpathcurveto{\pgfqpoint{1.212892in}{1.084328in}}{\pgfqpoint{1.220792in}{1.087600in}}{\pgfqpoint{1.226616in}{1.093424in}}%
\pgfpathcurveto{\pgfqpoint{1.232439in}{1.099248in}}{\pgfqpoint{1.235712in}{1.107148in}}{\pgfqpoint{1.235712in}{1.115384in}}%
\pgfpathcurveto{\pgfqpoint{1.235712in}{1.123621in}}{\pgfqpoint{1.232439in}{1.131521in}}{\pgfqpoint{1.226616in}{1.137345in}}%
\pgfpathcurveto{\pgfqpoint{1.220792in}{1.143169in}}{\pgfqpoint{1.212892in}{1.146441in}}{\pgfqpoint{1.204655in}{1.146441in}}%
\pgfpathcurveto{\pgfqpoint{1.196419in}{1.146441in}}{\pgfqpoint{1.188519in}{1.143169in}}{\pgfqpoint{1.182695in}{1.137345in}}%
\pgfpathcurveto{\pgfqpoint{1.176871in}{1.131521in}}{\pgfqpoint{1.173599in}{1.123621in}}{\pgfqpoint{1.173599in}{1.115384in}}%
\pgfpathcurveto{\pgfqpoint{1.173599in}{1.107148in}}{\pgfqpoint{1.176871in}{1.099248in}}{\pgfqpoint{1.182695in}{1.093424in}}%
\pgfpathcurveto{\pgfqpoint{1.188519in}{1.087600in}}{\pgfqpoint{1.196419in}{1.084328in}}{\pgfqpoint{1.204655in}{1.084328in}}%
\pgfpathclose%
\pgfusepath{stroke,fill}%
\end{pgfscope}%
\begin{pgfscope}%
\pgfpathrectangle{\pgfqpoint{0.100000in}{0.212622in}}{\pgfqpoint{3.696000in}{3.696000in}}%
\pgfusepath{clip}%
\pgfsetbuttcap%
\pgfsetroundjoin%
\definecolor{currentfill}{rgb}{0.121569,0.466667,0.705882}%
\pgfsetfillcolor{currentfill}%
\pgfsetfillopacity{0.754174}%
\pgfsetlinewidth{1.003750pt}%
\definecolor{currentstroke}{rgb}{0.121569,0.466667,0.705882}%
\pgfsetstrokecolor{currentstroke}%
\pgfsetstrokeopacity{0.754174}%
\pgfsetdash{}{0pt}%
\pgfpathmoveto{\pgfqpoint{2.320070in}{1.764695in}}%
\pgfpathcurveto{\pgfqpoint{2.328307in}{1.764695in}}{\pgfqpoint{2.336207in}{1.767968in}}{\pgfqpoint{2.342031in}{1.773792in}}%
\pgfpathcurveto{\pgfqpoint{2.347854in}{1.779616in}}{\pgfqpoint{2.351127in}{1.787516in}}{\pgfqpoint{2.351127in}{1.795752in}}%
\pgfpathcurveto{\pgfqpoint{2.351127in}{1.803988in}}{\pgfqpoint{2.347854in}{1.811888in}}{\pgfqpoint{2.342031in}{1.817712in}}%
\pgfpathcurveto{\pgfqpoint{2.336207in}{1.823536in}}{\pgfqpoint{2.328307in}{1.826808in}}{\pgfqpoint{2.320070in}{1.826808in}}%
\pgfpathcurveto{\pgfqpoint{2.311834in}{1.826808in}}{\pgfqpoint{2.303934in}{1.823536in}}{\pgfqpoint{2.298110in}{1.817712in}}%
\pgfpathcurveto{\pgfqpoint{2.292286in}{1.811888in}}{\pgfqpoint{2.289014in}{1.803988in}}{\pgfqpoint{2.289014in}{1.795752in}}%
\pgfpathcurveto{\pgfqpoint{2.289014in}{1.787516in}}{\pgfqpoint{2.292286in}{1.779616in}}{\pgfqpoint{2.298110in}{1.773792in}}%
\pgfpathcurveto{\pgfqpoint{2.303934in}{1.767968in}}{\pgfqpoint{2.311834in}{1.764695in}}{\pgfqpoint{2.320070in}{1.764695in}}%
\pgfpathclose%
\pgfusepath{stroke,fill}%
\end{pgfscope}%
\begin{pgfscope}%
\pgfpathrectangle{\pgfqpoint{0.100000in}{0.212622in}}{\pgfqpoint{3.696000in}{3.696000in}}%
\pgfusepath{clip}%
\pgfsetbuttcap%
\pgfsetroundjoin%
\definecolor{currentfill}{rgb}{0.121569,0.466667,0.705882}%
\pgfsetfillcolor{currentfill}%
\pgfsetfillopacity{0.755834}%
\pgfsetlinewidth{1.003750pt}%
\definecolor{currentstroke}{rgb}{0.121569,0.466667,0.705882}%
\pgfsetstrokecolor{currentstroke}%
\pgfsetstrokeopacity{0.755834}%
\pgfsetdash{}{0pt}%
\pgfpathmoveto{\pgfqpoint{1.221750in}{1.081171in}}%
\pgfpathcurveto{\pgfqpoint{1.229987in}{1.081171in}}{\pgfqpoint{1.237887in}{1.084443in}}{\pgfqpoint{1.243711in}{1.090267in}}%
\pgfpathcurveto{\pgfqpoint{1.249534in}{1.096091in}}{\pgfqpoint{1.252807in}{1.103991in}}{\pgfqpoint{1.252807in}{1.112227in}}%
\pgfpathcurveto{\pgfqpoint{1.252807in}{1.120463in}}{\pgfqpoint{1.249534in}{1.128363in}}{\pgfqpoint{1.243711in}{1.134187in}}%
\pgfpathcurveto{\pgfqpoint{1.237887in}{1.140011in}}{\pgfqpoint{1.229987in}{1.143284in}}{\pgfqpoint{1.221750in}{1.143284in}}%
\pgfpathcurveto{\pgfqpoint{1.213514in}{1.143284in}}{\pgfqpoint{1.205614in}{1.140011in}}{\pgfqpoint{1.199790in}{1.134187in}}%
\pgfpathcurveto{\pgfqpoint{1.193966in}{1.128363in}}{\pgfqpoint{1.190694in}{1.120463in}}{\pgfqpoint{1.190694in}{1.112227in}}%
\pgfpathcurveto{\pgfqpoint{1.190694in}{1.103991in}}{\pgfqpoint{1.193966in}{1.096091in}}{\pgfqpoint{1.199790in}{1.090267in}}%
\pgfpathcurveto{\pgfqpoint{1.205614in}{1.084443in}}{\pgfqpoint{1.213514in}{1.081171in}}{\pgfqpoint{1.221750in}{1.081171in}}%
\pgfpathclose%
\pgfusepath{stroke,fill}%
\end{pgfscope}%
\begin{pgfscope}%
\pgfpathrectangle{\pgfqpoint{0.100000in}{0.212622in}}{\pgfqpoint{3.696000in}{3.696000in}}%
\pgfusepath{clip}%
\pgfsetbuttcap%
\pgfsetroundjoin%
\definecolor{currentfill}{rgb}{0.121569,0.466667,0.705882}%
\pgfsetfillcolor{currentfill}%
\pgfsetfillopacity{0.760012}%
\pgfsetlinewidth{1.003750pt}%
\definecolor{currentstroke}{rgb}{0.121569,0.466667,0.705882}%
\pgfsetstrokecolor{currentstroke}%
\pgfsetstrokeopacity{0.760012}%
\pgfsetdash{}{0pt}%
\pgfpathmoveto{\pgfqpoint{2.325595in}{1.743391in}}%
\pgfpathcurveto{\pgfqpoint{2.333831in}{1.743391in}}{\pgfqpoint{2.341731in}{1.746663in}}{\pgfqpoint{2.347555in}{1.752487in}}%
\pgfpathcurveto{\pgfqpoint{2.353379in}{1.758311in}}{\pgfqpoint{2.356652in}{1.766211in}}{\pgfqpoint{2.356652in}{1.774447in}}%
\pgfpathcurveto{\pgfqpoint{2.356652in}{1.782683in}}{\pgfqpoint{2.353379in}{1.790584in}}{\pgfqpoint{2.347555in}{1.796407in}}%
\pgfpathcurveto{\pgfqpoint{2.341731in}{1.802231in}}{\pgfqpoint{2.333831in}{1.805504in}}{\pgfqpoint{2.325595in}{1.805504in}}%
\pgfpathcurveto{\pgfqpoint{2.317359in}{1.805504in}}{\pgfqpoint{2.309459in}{1.802231in}}{\pgfqpoint{2.303635in}{1.796407in}}%
\pgfpathcurveto{\pgfqpoint{2.297811in}{1.790584in}}{\pgfqpoint{2.294539in}{1.782683in}}{\pgfqpoint{2.294539in}{1.774447in}}%
\pgfpathcurveto{\pgfqpoint{2.294539in}{1.766211in}}{\pgfqpoint{2.297811in}{1.758311in}}{\pgfqpoint{2.303635in}{1.752487in}}%
\pgfpathcurveto{\pgfqpoint{2.309459in}{1.746663in}}{\pgfqpoint{2.317359in}{1.743391in}}{\pgfqpoint{2.325595in}{1.743391in}}%
\pgfpathclose%
\pgfusepath{stroke,fill}%
\end{pgfscope}%
\begin{pgfscope}%
\pgfpathrectangle{\pgfqpoint{0.100000in}{0.212622in}}{\pgfqpoint{3.696000in}{3.696000in}}%
\pgfusepath{clip}%
\pgfsetbuttcap%
\pgfsetroundjoin%
\definecolor{currentfill}{rgb}{0.121569,0.466667,0.705882}%
\pgfsetfillcolor{currentfill}%
\pgfsetfillopacity{0.762601}%
\pgfsetlinewidth{1.003750pt}%
\definecolor{currentstroke}{rgb}{0.121569,0.466667,0.705882}%
\pgfsetstrokecolor{currentstroke}%
\pgfsetstrokeopacity{0.762601}%
\pgfsetdash{}{0pt}%
\pgfpathmoveto{\pgfqpoint{1.252941in}{1.076042in}}%
\pgfpathcurveto{\pgfqpoint{1.261177in}{1.076042in}}{\pgfqpoint{1.269077in}{1.079315in}}{\pgfqpoint{1.274901in}{1.085139in}}%
\pgfpathcurveto{\pgfqpoint{1.280725in}{1.090962in}}{\pgfqpoint{1.283997in}{1.098862in}}{\pgfqpoint{1.283997in}{1.107099in}}%
\pgfpathcurveto{\pgfqpoint{1.283997in}{1.115335in}}{\pgfqpoint{1.280725in}{1.123235in}}{\pgfqpoint{1.274901in}{1.129059in}}%
\pgfpathcurveto{\pgfqpoint{1.269077in}{1.134883in}}{\pgfqpoint{1.261177in}{1.138155in}}{\pgfqpoint{1.252941in}{1.138155in}}%
\pgfpathcurveto{\pgfqpoint{1.244704in}{1.138155in}}{\pgfqpoint{1.236804in}{1.134883in}}{\pgfqpoint{1.230980in}{1.129059in}}%
\pgfpathcurveto{\pgfqpoint{1.225157in}{1.123235in}}{\pgfqpoint{1.221884in}{1.115335in}}{\pgfqpoint{1.221884in}{1.107099in}}%
\pgfpathcurveto{\pgfqpoint{1.221884in}{1.098862in}}{\pgfqpoint{1.225157in}{1.090962in}}{\pgfqpoint{1.230980in}{1.085139in}}%
\pgfpathcurveto{\pgfqpoint{1.236804in}{1.079315in}}{\pgfqpoint{1.244704in}{1.076042in}}{\pgfqpoint{1.252941in}{1.076042in}}%
\pgfpathclose%
\pgfusepath{stroke,fill}%
\end{pgfscope}%
\begin{pgfscope}%
\pgfpathrectangle{\pgfqpoint{0.100000in}{0.212622in}}{\pgfqpoint{3.696000in}{3.696000in}}%
\pgfusepath{clip}%
\pgfsetbuttcap%
\pgfsetroundjoin%
\definecolor{currentfill}{rgb}{0.121569,0.466667,0.705882}%
\pgfsetfillcolor{currentfill}%
\pgfsetfillopacity{0.766936}%
\pgfsetlinewidth{1.003750pt}%
\definecolor{currentstroke}{rgb}{0.121569,0.466667,0.705882}%
\pgfsetstrokecolor{currentstroke}%
\pgfsetstrokeopacity{0.766936}%
\pgfsetdash{}{0pt}%
\pgfpathmoveto{\pgfqpoint{2.332253in}{1.718037in}}%
\pgfpathcurveto{\pgfqpoint{2.340490in}{1.718037in}}{\pgfqpoint{2.348390in}{1.721310in}}{\pgfqpoint{2.354214in}{1.727134in}}%
\pgfpathcurveto{\pgfqpoint{2.360038in}{1.732958in}}{\pgfqpoint{2.363310in}{1.740858in}}{\pgfqpoint{2.363310in}{1.749094in}}%
\pgfpathcurveto{\pgfqpoint{2.363310in}{1.757330in}}{\pgfqpoint{2.360038in}{1.765230in}}{\pgfqpoint{2.354214in}{1.771054in}}%
\pgfpathcurveto{\pgfqpoint{2.348390in}{1.776878in}}{\pgfqpoint{2.340490in}{1.780150in}}{\pgfqpoint{2.332253in}{1.780150in}}%
\pgfpathcurveto{\pgfqpoint{2.324017in}{1.780150in}}{\pgfqpoint{2.316117in}{1.776878in}}{\pgfqpoint{2.310293in}{1.771054in}}%
\pgfpathcurveto{\pgfqpoint{2.304469in}{1.765230in}}{\pgfqpoint{2.301197in}{1.757330in}}{\pgfqpoint{2.301197in}{1.749094in}}%
\pgfpathcurveto{\pgfqpoint{2.301197in}{1.740858in}}{\pgfqpoint{2.304469in}{1.732958in}}{\pgfqpoint{2.310293in}{1.727134in}}%
\pgfpathcurveto{\pgfqpoint{2.316117in}{1.721310in}}{\pgfqpoint{2.324017in}{1.718037in}}{\pgfqpoint{2.332253in}{1.718037in}}%
\pgfpathclose%
\pgfusepath{stroke,fill}%
\end{pgfscope}%
\begin{pgfscope}%
\pgfpathrectangle{\pgfqpoint{0.100000in}{0.212622in}}{\pgfqpoint{3.696000in}{3.696000in}}%
\pgfusepath{clip}%
\pgfsetbuttcap%
\pgfsetroundjoin%
\definecolor{currentfill}{rgb}{0.121569,0.466667,0.705882}%
\pgfsetfillcolor{currentfill}%
\pgfsetfillopacity{0.768890}%
\pgfsetlinewidth{1.003750pt}%
\definecolor{currentstroke}{rgb}{0.121569,0.466667,0.705882}%
\pgfsetstrokecolor{currentstroke}%
\pgfsetstrokeopacity{0.768890}%
\pgfsetdash{}{0pt}%
\pgfpathmoveto{\pgfqpoint{1.280853in}{1.069886in}}%
\pgfpathcurveto{\pgfqpoint{1.289089in}{1.069886in}}{\pgfqpoint{1.296989in}{1.073158in}}{\pgfqpoint{1.302813in}{1.078982in}}%
\pgfpathcurveto{\pgfqpoint{1.308637in}{1.084806in}}{\pgfqpoint{1.311909in}{1.092706in}}{\pgfqpoint{1.311909in}{1.100943in}}%
\pgfpathcurveto{\pgfqpoint{1.311909in}{1.109179in}}{\pgfqpoint{1.308637in}{1.117079in}}{\pgfqpoint{1.302813in}{1.122903in}}%
\pgfpathcurveto{\pgfqpoint{1.296989in}{1.128727in}}{\pgfqpoint{1.289089in}{1.131999in}}{\pgfqpoint{1.280853in}{1.131999in}}%
\pgfpathcurveto{\pgfqpoint{1.272617in}{1.131999in}}{\pgfqpoint{1.264716in}{1.128727in}}{\pgfqpoint{1.258893in}{1.122903in}}%
\pgfpathcurveto{\pgfqpoint{1.253069in}{1.117079in}}{\pgfqpoint{1.249796in}{1.109179in}}{\pgfqpoint{1.249796in}{1.100943in}}%
\pgfpathcurveto{\pgfqpoint{1.249796in}{1.092706in}}{\pgfqpoint{1.253069in}{1.084806in}}{\pgfqpoint{1.258893in}{1.078982in}}%
\pgfpathcurveto{\pgfqpoint{1.264716in}{1.073158in}}{\pgfqpoint{1.272617in}{1.069886in}}{\pgfqpoint{1.280853in}{1.069886in}}%
\pgfpathclose%
\pgfusepath{stroke,fill}%
\end{pgfscope}%
\begin{pgfscope}%
\pgfpathrectangle{\pgfqpoint{0.100000in}{0.212622in}}{\pgfqpoint{3.696000in}{3.696000in}}%
\pgfusepath{clip}%
\pgfsetbuttcap%
\pgfsetroundjoin%
\definecolor{currentfill}{rgb}{0.121569,0.466667,0.705882}%
\pgfsetfillcolor{currentfill}%
\pgfsetfillopacity{0.774075}%
\pgfsetlinewidth{1.003750pt}%
\definecolor{currentstroke}{rgb}{0.121569,0.466667,0.705882}%
\pgfsetstrokecolor{currentstroke}%
\pgfsetstrokeopacity{0.774075}%
\pgfsetdash{}{0pt}%
\pgfpathmoveto{\pgfqpoint{1.304052in}{1.064597in}}%
\pgfpathcurveto{\pgfqpoint{1.312288in}{1.064597in}}{\pgfqpoint{1.320188in}{1.067869in}}{\pgfqpoint{1.326012in}{1.073693in}}%
\pgfpathcurveto{\pgfqpoint{1.331836in}{1.079517in}}{\pgfqpoint{1.335108in}{1.087417in}}{\pgfqpoint{1.335108in}{1.095654in}}%
\pgfpathcurveto{\pgfqpoint{1.335108in}{1.103890in}}{\pgfqpoint{1.331836in}{1.111790in}}{\pgfqpoint{1.326012in}{1.117614in}}%
\pgfpathcurveto{\pgfqpoint{1.320188in}{1.123438in}}{\pgfqpoint{1.312288in}{1.126710in}}{\pgfqpoint{1.304052in}{1.126710in}}%
\pgfpathcurveto{\pgfqpoint{1.295815in}{1.126710in}}{\pgfqpoint{1.287915in}{1.123438in}}{\pgfqpoint{1.282091in}{1.117614in}}%
\pgfpathcurveto{\pgfqpoint{1.276267in}{1.111790in}}{\pgfqpoint{1.272995in}{1.103890in}}{\pgfqpoint{1.272995in}{1.095654in}}%
\pgfpathcurveto{\pgfqpoint{1.272995in}{1.087417in}}{\pgfqpoint{1.276267in}{1.079517in}}{\pgfqpoint{1.282091in}{1.073693in}}%
\pgfpathcurveto{\pgfqpoint{1.287915in}{1.067869in}}{\pgfqpoint{1.295815in}{1.064597in}}{\pgfqpoint{1.304052in}{1.064597in}}%
\pgfpathclose%
\pgfusepath{stroke,fill}%
\end{pgfscope}%
\begin{pgfscope}%
\pgfpathrectangle{\pgfqpoint{0.100000in}{0.212622in}}{\pgfqpoint{3.696000in}{3.696000in}}%
\pgfusepath{clip}%
\pgfsetbuttcap%
\pgfsetroundjoin%
\definecolor{currentfill}{rgb}{0.121569,0.466667,0.705882}%
\pgfsetfillcolor{currentfill}%
\pgfsetfillopacity{0.774954}%
\pgfsetlinewidth{1.003750pt}%
\definecolor{currentstroke}{rgb}{0.121569,0.466667,0.705882}%
\pgfsetstrokecolor{currentstroke}%
\pgfsetstrokeopacity{0.774954}%
\pgfsetdash{}{0pt}%
\pgfpathmoveto{\pgfqpoint{2.339239in}{1.688325in}}%
\pgfpathcurveto{\pgfqpoint{2.347476in}{1.688325in}}{\pgfqpoint{2.355376in}{1.691597in}}{\pgfqpoint{2.361200in}{1.697421in}}%
\pgfpathcurveto{\pgfqpoint{2.367024in}{1.703245in}}{\pgfqpoint{2.370296in}{1.711145in}}{\pgfqpoint{2.370296in}{1.719381in}}%
\pgfpathcurveto{\pgfqpoint{2.370296in}{1.727617in}}{\pgfqpoint{2.367024in}{1.735517in}}{\pgfqpoint{2.361200in}{1.741341in}}%
\pgfpathcurveto{\pgfqpoint{2.355376in}{1.747165in}}{\pgfqpoint{2.347476in}{1.750438in}}{\pgfqpoint{2.339239in}{1.750438in}}%
\pgfpathcurveto{\pgfqpoint{2.331003in}{1.750438in}}{\pgfqpoint{2.323103in}{1.747165in}}{\pgfqpoint{2.317279in}{1.741341in}}%
\pgfpathcurveto{\pgfqpoint{2.311455in}{1.735517in}}{\pgfqpoint{2.308183in}{1.727617in}}{\pgfqpoint{2.308183in}{1.719381in}}%
\pgfpathcurveto{\pgfqpoint{2.308183in}{1.711145in}}{\pgfqpoint{2.311455in}{1.703245in}}{\pgfqpoint{2.317279in}{1.697421in}}%
\pgfpathcurveto{\pgfqpoint{2.323103in}{1.691597in}}{\pgfqpoint{2.331003in}{1.688325in}}{\pgfqpoint{2.339239in}{1.688325in}}%
\pgfpathclose%
\pgfusepath{stroke,fill}%
\end{pgfscope}%
\begin{pgfscope}%
\pgfpathrectangle{\pgfqpoint{0.100000in}{0.212622in}}{\pgfqpoint{3.696000in}{3.696000in}}%
\pgfusepath{clip}%
\pgfsetbuttcap%
\pgfsetroundjoin%
\definecolor{currentfill}{rgb}{0.121569,0.466667,0.705882}%
\pgfsetfillcolor{currentfill}%
\pgfsetfillopacity{0.778077}%
\pgfsetlinewidth{1.003750pt}%
\definecolor{currentstroke}{rgb}{0.121569,0.466667,0.705882}%
\pgfsetstrokecolor{currentstroke}%
\pgfsetstrokeopacity{0.778077}%
\pgfsetdash{}{0pt}%
\pgfpathmoveto{\pgfqpoint{1.322066in}{1.060821in}}%
\pgfpathcurveto{\pgfqpoint{1.330303in}{1.060821in}}{\pgfqpoint{1.338203in}{1.064094in}}{\pgfqpoint{1.344027in}{1.069918in}}%
\pgfpathcurveto{\pgfqpoint{1.349851in}{1.075742in}}{\pgfqpoint{1.353123in}{1.083642in}}{\pgfqpoint{1.353123in}{1.091878in}}%
\pgfpathcurveto{\pgfqpoint{1.353123in}{1.100114in}}{\pgfqpoint{1.349851in}{1.108014in}}{\pgfqpoint{1.344027in}{1.113838in}}%
\pgfpathcurveto{\pgfqpoint{1.338203in}{1.119662in}}{\pgfqpoint{1.330303in}{1.122934in}}{\pgfqpoint{1.322066in}{1.122934in}}%
\pgfpathcurveto{\pgfqpoint{1.313830in}{1.122934in}}{\pgfqpoint{1.305930in}{1.119662in}}{\pgfqpoint{1.300106in}{1.113838in}}%
\pgfpathcurveto{\pgfqpoint{1.294282in}{1.108014in}}{\pgfqpoint{1.291010in}{1.100114in}}{\pgfqpoint{1.291010in}{1.091878in}}%
\pgfpathcurveto{\pgfqpoint{1.291010in}{1.083642in}}{\pgfqpoint{1.294282in}{1.075742in}}{\pgfqpoint{1.300106in}{1.069918in}}%
\pgfpathcurveto{\pgfqpoint{1.305930in}{1.064094in}}{\pgfqpoint{1.313830in}{1.060821in}}{\pgfqpoint{1.322066in}{1.060821in}}%
\pgfpathclose%
\pgfusepath{stroke,fill}%
\end{pgfscope}%
\begin{pgfscope}%
\pgfpathrectangle{\pgfqpoint{0.100000in}{0.212622in}}{\pgfqpoint{3.696000in}{3.696000in}}%
\pgfusepath{clip}%
\pgfsetbuttcap%
\pgfsetroundjoin%
\definecolor{currentfill}{rgb}{0.121569,0.466667,0.705882}%
\pgfsetfillcolor{currentfill}%
\pgfsetfillopacity{0.781575}%
\pgfsetlinewidth{1.003750pt}%
\definecolor{currentstroke}{rgb}{0.121569,0.466667,0.705882}%
\pgfsetstrokecolor{currentstroke}%
\pgfsetstrokeopacity{0.781575}%
\pgfsetdash{}{0pt}%
\pgfpathmoveto{\pgfqpoint{1.337506in}{1.056824in}}%
\pgfpathcurveto{\pgfqpoint{1.345743in}{1.056824in}}{\pgfqpoint{1.353643in}{1.060096in}}{\pgfqpoint{1.359467in}{1.065920in}}%
\pgfpathcurveto{\pgfqpoint{1.365291in}{1.071744in}}{\pgfqpoint{1.368563in}{1.079644in}}{\pgfqpoint{1.368563in}{1.087880in}}%
\pgfpathcurveto{\pgfqpoint{1.368563in}{1.096117in}}{\pgfqpoint{1.365291in}{1.104017in}}{\pgfqpoint{1.359467in}{1.109841in}}%
\pgfpathcurveto{\pgfqpoint{1.353643in}{1.115664in}}{\pgfqpoint{1.345743in}{1.118937in}}{\pgfqpoint{1.337506in}{1.118937in}}%
\pgfpathcurveto{\pgfqpoint{1.329270in}{1.118937in}}{\pgfqpoint{1.321370in}{1.115664in}}{\pgfqpoint{1.315546in}{1.109841in}}%
\pgfpathcurveto{\pgfqpoint{1.309722in}{1.104017in}}{\pgfqpoint{1.306450in}{1.096117in}}{\pgfqpoint{1.306450in}{1.087880in}}%
\pgfpathcurveto{\pgfqpoint{1.306450in}{1.079644in}}{\pgfqpoint{1.309722in}{1.071744in}}{\pgfqpoint{1.315546in}{1.065920in}}%
\pgfpathcurveto{\pgfqpoint{1.321370in}{1.060096in}}{\pgfqpoint{1.329270in}{1.056824in}}{\pgfqpoint{1.337506in}{1.056824in}}%
\pgfpathclose%
\pgfusepath{stroke,fill}%
\end{pgfscope}%
\begin{pgfscope}%
\pgfpathrectangle{\pgfqpoint{0.100000in}{0.212622in}}{\pgfqpoint{3.696000in}{3.696000in}}%
\pgfusepath{clip}%
\pgfsetbuttcap%
\pgfsetroundjoin%
\definecolor{currentfill}{rgb}{0.121569,0.466667,0.705882}%
\pgfsetfillcolor{currentfill}%
\pgfsetfillopacity{0.784849}%
\pgfsetlinewidth{1.003750pt}%
\definecolor{currentstroke}{rgb}{0.121569,0.466667,0.705882}%
\pgfsetstrokecolor{currentstroke}%
\pgfsetstrokeopacity{0.784849}%
\pgfsetdash{}{0pt}%
\pgfpathmoveto{\pgfqpoint{2.348849in}{1.653555in}}%
\pgfpathcurveto{\pgfqpoint{2.357085in}{1.653555in}}{\pgfqpoint{2.364985in}{1.656827in}}{\pgfqpoint{2.370809in}{1.662651in}}%
\pgfpathcurveto{\pgfqpoint{2.376633in}{1.668475in}}{\pgfqpoint{2.379906in}{1.676375in}}{\pgfqpoint{2.379906in}{1.684611in}}%
\pgfpathcurveto{\pgfqpoint{2.379906in}{1.692848in}}{\pgfqpoint{2.376633in}{1.700748in}}{\pgfqpoint{2.370809in}{1.706572in}}%
\pgfpathcurveto{\pgfqpoint{2.364985in}{1.712396in}}{\pgfqpoint{2.357085in}{1.715668in}}{\pgfqpoint{2.348849in}{1.715668in}}%
\pgfpathcurveto{\pgfqpoint{2.340613in}{1.715668in}}{\pgfqpoint{2.332713in}{1.712396in}}{\pgfqpoint{2.326889in}{1.706572in}}%
\pgfpathcurveto{\pgfqpoint{2.321065in}{1.700748in}}{\pgfqpoint{2.317793in}{1.692848in}}{\pgfqpoint{2.317793in}{1.684611in}}%
\pgfpathcurveto{\pgfqpoint{2.317793in}{1.676375in}}{\pgfqpoint{2.321065in}{1.668475in}}{\pgfqpoint{2.326889in}{1.662651in}}%
\pgfpathcurveto{\pgfqpoint{2.332713in}{1.656827in}}{\pgfqpoint{2.340613in}{1.653555in}}{\pgfqpoint{2.348849in}{1.653555in}}%
\pgfpathclose%
\pgfusepath{stroke,fill}%
\end{pgfscope}%
\begin{pgfscope}%
\pgfpathrectangle{\pgfqpoint{0.100000in}{0.212622in}}{\pgfqpoint{3.696000in}{3.696000in}}%
\pgfusepath{clip}%
\pgfsetbuttcap%
\pgfsetroundjoin%
\definecolor{currentfill}{rgb}{0.121569,0.466667,0.705882}%
\pgfsetfillcolor{currentfill}%
\pgfsetfillopacity{0.787971}%
\pgfsetlinewidth{1.003750pt}%
\definecolor{currentstroke}{rgb}{0.121569,0.466667,0.705882}%
\pgfsetstrokecolor{currentstroke}%
\pgfsetstrokeopacity{0.787971}%
\pgfsetdash{}{0pt}%
\pgfpathmoveto{\pgfqpoint{1.365671in}{1.049918in}}%
\pgfpathcurveto{\pgfqpoint{1.373908in}{1.049918in}}{\pgfqpoint{1.381808in}{1.053190in}}{\pgfqpoint{1.387632in}{1.059014in}}%
\pgfpathcurveto{\pgfqpoint{1.393456in}{1.064838in}}{\pgfqpoint{1.396728in}{1.072738in}}{\pgfqpoint{1.396728in}{1.080974in}}%
\pgfpathcurveto{\pgfqpoint{1.396728in}{1.089210in}}{\pgfqpoint{1.393456in}{1.097110in}}{\pgfqpoint{1.387632in}{1.102934in}}%
\pgfpathcurveto{\pgfqpoint{1.381808in}{1.108758in}}{\pgfqpoint{1.373908in}{1.112031in}}{\pgfqpoint{1.365671in}{1.112031in}}%
\pgfpathcurveto{\pgfqpoint{1.357435in}{1.112031in}}{\pgfqpoint{1.349535in}{1.108758in}}{\pgfqpoint{1.343711in}{1.102934in}}%
\pgfpathcurveto{\pgfqpoint{1.337887in}{1.097110in}}{\pgfqpoint{1.334615in}{1.089210in}}{\pgfqpoint{1.334615in}{1.080974in}}%
\pgfpathcurveto{\pgfqpoint{1.334615in}{1.072738in}}{\pgfqpoint{1.337887in}{1.064838in}}{\pgfqpoint{1.343711in}{1.059014in}}%
\pgfpathcurveto{\pgfqpoint{1.349535in}{1.053190in}}{\pgfqpoint{1.357435in}{1.049918in}}{\pgfqpoint{1.365671in}{1.049918in}}%
\pgfpathclose%
\pgfusepath{stroke,fill}%
\end{pgfscope}%
\begin{pgfscope}%
\pgfpathrectangle{\pgfqpoint{0.100000in}{0.212622in}}{\pgfqpoint{3.696000in}{3.696000in}}%
\pgfusepath{clip}%
\pgfsetbuttcap%
\pgfsetroundjoin%
\definecolor{currentfill}{rgb}{0.121569,0.466667,0.705882}%
\pgfsetfillcolor{currentfill}%
\pgfsetfillopacity{0.790259}%
\pgfsetlinewidth{1.003750pt}%
\definecolor{currentstroke}{rgb}{0.121569,0.466667,0.705882}%
\pgfsetstrokecolor{currentstroke}%
\pgfsetstrokeopacity{0.790259}%
\pgfsetdash{}{0pt}%
\pgfpathmoveto{\pgfqpoint{2.354114in}{1.634250in}}%
\pgfpathcurveto{\pgfqpoint{2.362350in}{1.634250in}}{\pgfqpoint{2.370250in}{1.637523in}}{\pgfqpoint{2.376074in}{1.643347in}}%
\pgfpathcurveto{\pgfqpoint{2.381898in}{1.649170in}}{\pgfqpoint{2.385170in}{1.657070in}}{\pgfqpoint{2.385170in}{1.665307in}}%
\pgfpathcurveto{\pgfqpoint{2.385170in}{1.673543in}}{\pgfqpoint{2.381898in}{1.681443in}}{\pgfqpoint{2.376074in}{1.687267in}}%
\pgfpathcurveto{\pgfqpoint{2.370250in}{1.693091in}}{\pgfqpoint{2.362350in}{1.696363in}}{\pgfqpoint{2.354114in}{1.696363in}}%
\pgfpathcurveto{\pgfqpoint{2.345878in}{1.696363in}}{\pgfqpoint{2.337978in}{1.693091in}}{\pgfqpoint{2.332154in}{1.687267in}}%
\pgfpathcurveto{\pgfqpoint{2.326330in}{1.681443in}}{\pgfqpoint{2.323057in}{1.673543in}}{\pgfqpoint{2.323057in}{1.665307in}}%
\pgfpathcurveto{\pgfqpoint{2.323057in}{1.657070in}}{\pgfqpoint{2.326330in}{1.649170in}}{\pgfqpoint{2.332154in}{1.643347in}}%
\pgfpathcurveto{\pgfqpoint{2.337978in}{1.637523in}}{\pgfqpoint{2.345878in}{1.634250in}}{\pgfqpoint{2.354114in}{1.634250in}}%
\pgfpathclose%
\pgfusepath{stroke,fill}%
\end{pgfscope}%
\begin{pgfscope}%
\pgfpathrectangle{\pgfqpoint{0.100000in}{0.212622in}}{\pgfqpoint{3.696000in}{3.696000in}}%
\pgfusepath{clip}%
\pgfsetbuttcap%
\pgfsetroundjoin%
\definecolor{currentfill}{rgb}{0.121569,0.466667,0.705882}%
\pgfsetfillcolor{currentfill}%
\pgfsetfillopacity{0.793788}%
\pgfsetlinewidth{1.003750pt}%
\definecolor{currentstroke}{rgb}{0.121569,0.466667,0.705882}%
\pgfsetstrokecolor{currentstroke}%
\pgfsetstrokeopacity{0.793788}%
\pgfsetdash{}{0pt}%
\pgfpathmoveto{\pgfqpoint{1.392397in}{1.044536in}}%
\pgfpathcurveto{\pgfqpoint{1.400633in}{1.044536in}}{\pgfqpoint{1.408533in}{1.047808in}}{\pgfqpoint{1.414357in}{1.053632in}}%
\pgfpathcurveto{\pgfqpoint{1.420181in}{1.059456in}}{\pgfqpoint{1.423454in}{1.067356in}}{\pgfqpoint{1.423454in}{1.075592in}}%
\pgfpathcurveto{\pgfqpoint{1.423454in}{1.083828in}}{\pgfqpoint{1.420181in}{1.091728in}}{\pgfqpoint{1.414357in}{1.097552in}}%
\pgfpathcurveto{\pgfqpoint{1.408533in}{1.103376in}}{\pgfqpoint{1.400633in}{1.106649in}}{\pgfqpoint{1.392397in}{1.106649in}}%
\pgfpathcurveto{\pgfqpoint{1.384161in}{1.106649in}}{\pgfqpoint{1.376261in}{1.103376in}}{\pgfqpoint{1.370437in}{1.097552in}}%
\pgfpathcurveto{\pgfqpoint{1.364613in}{1.091728in}}{\pgfqpoint{1.361341in}{1.083828in}}{\pgfqpoint{1.361341in}{1.075592in}}%
\pgfpathcurveto{\pgfqpoint{1.361341in}{1.067356in}}{\pgfqpoint{1.364613in}{1.059456in}}{\pgfqpoint{1.370437in}{1.053632in}}%
\pgfpathcurveto{\pgfqpoint{1.376261in}{1.047808in}}{\pgfqpoint{1.384161in}{1.044536in}}{\pgfqpoint{1.392397in}{1.044536in}}%
\pgfpathclose%
\pgfusepath{stroke,fill}%
\end{pgfscope}%
\begin{pgfscope}%
\pgfpathrectangle{\pgfqpoint{0.100000in}{0.212622in}}{\pgfqpoint{3.696000in}{3.696000in}}%
\pgfusepath{clip}%
\pgfsetbuttcap%
\pgfsetroundjoin%
\definecolor{currentfill}{rgb}{0.121569,0.466667,0.705882}%
\pgfsetfillcolor{currentfill}%
\pgfsetfillopacity{0.796250}%
\pgfsetlinewidth{1.003750pt}%
\definecolor{currentstroke}{rgb}{0.121569,0.466667,0.705882}%
\pgfsetstrokecolor{currentstroke}%
\pgfsetstrokeopacity{0.796250}%
\pgfsetdash{}{0pt}%
\pgfpathmoveto{\pgfqpoint{2.359625in}{1.612474in}}%
\pgfpathcurveto{\pgfqpoint{2.367861in}{1.612474in}}{\pgfqpoint{2.375762in}{1.615746in}}{\pgfqpoint{2.381585in}{1.621570in}}%
\pgfpathcurveto{\pgfqpoint{2.387409in}{1.627394in}}{\pgfqpoint{2.390682in}{1.635294in}}{\pgfqpoint{2.390682in}{1.643530in}}%
\pgfpathcurveto{\pgfqpoint{2.390682in}{1.651766in}}{\pgfqpoint{2.387409in}{1.659666in}}{\pgfqpoint{2.381585in}{1.665490in}}%
\pgfpathcurveto{\pgfqpoint{2.375762in}{1.671314in}}{\pgfqpoint{2.367861in}{1.674587in}}{\pgfqpoint{2.359625in}{1.674587in}}%
\pgfpathcurveto{\pgfqpoint{2.351389in}{1.674587in}}{\pgfqpoint{2.343489in}{1.671314in}}{\pgfqpoint{2.337665in}{1.665490in}}%
\pgfpathcurveto{\pgfqpoint{2.331841in}{1.659666in}}{\pgfqpoint{2.328569in}{1.651766in}}{\pgfqpoint{2.328569in}{1.643530in}}%
\pgfpathcurveto{\pgfqpoint{2.328569in}{1.635294in}}{\pgfqpoint{2.331841in}{1.627394in}}{\pgfqpoint{2.337665in}{1.621570in}}%
\pgfpathcurveto{\pgfqpoint{2.343489in}{1.615746in}}{\pgfqpoint{2.351389in}{1.612474in}}{\pgfqpoint{2.359625in}{1.612474in}}%
\pgfpathclose%
\pgfusepath{stroke,fill}%
\end{pgfscope}%
\begin{pgfscope}%
\pgfpathrectangle{\pgfqpoint{0.100000in}{0.212622in}}{\pgfqpoint{3.696000in}{3.696000in}}%
\pgfusepath{clip}%
\pgfsetbuttcap%
\pgfsetroundjoin%
\definecolor{currentfill}{rgb}{0.121569,0.466667,0.705882}%
\pgfsetfillcolor{currentfill}%
\pgfsetfillopacity{0.798354}%
\pgfsetlinewidth{1.003750pt}%
\definecolor{currentstroke}{rgb}{0.121569,0.466667,0.705882}%
\pgfsetstrokecolor{currentstroke}%
\pgfsetstrokeopacity{0.798354}%
\pgfsetdash{}{0pt}%
\pgfpathmoveto{\pgfqpoint{1.412312in}{1.039055in}}%
\pgfpathcurveto{\pgfqpoint{1.420548in}{1.039055in}}{\pgfqpoint{1.428448in}{1.042328in}}{\pgfqpoint{1.434272in}{1.048151in}}%
\pgfpathcurveto{\pgfqpoint{1.440096in}{1.053975in}}{\pgfqpoint{1.443368in}{1.061875in}}{\pgfqpoint{1.443368in}{1.070112in}}%
\pgfpathcurveto{\pgfqpoint{1.443368in}{1.078348in}}{\pgfqpoint{1.440096in}{1.086248in}}{\pgfqpoint{1.434272in}{1.092072in}}%
\pgfpathcurveto{\pgfqpoint{1.428448in}{1.097896in}}{\pgfqpoint{1.420548in}{1.101168in}}{\pgfqpoint{1.412312in}{1.101168in}}%
\pgfpathcurveto{\pgfqpoint{1.404076in}{1.101168in}}{\pgfqpoint{1.396175in}{1.097896in}}{\pgfqpoint{1.390352in}{1.092072in}}%
\pgfpathcurveto{\pgfqpoint{1.384528in}{1.086248in}}{\pgfqpoint{1.381255in}{1.078348in}}{\pgfqpoint{1.381255in}{1.070112in}}%
\pgfpathcurveto{\pgfqpoint{1.381255in}{1.061875in}}{\pgfqpoint{1.384528in}{1.053975in}}{\pgfqpoint{1.390352in}{1.048151in}}%
\pgfpathcurveto{\pgfqpoint{1.396175in}{1.042328in}}{\pgfqpoint{1.404076in}{1.039055in}}{\pgfqpoint{1.412312in}{1.039055in}}%
\pgfpathclose%
\pgfusepath{stroke,fill}%
\end{pgfscope}%
\begin{pgfscope}%
\pgfpathrectangle{\pgfqpoint{0.100000in}{0.212622in}}{\pgfqpoint{3.696000in}{3.696000in}}%
\pgfusepath{clip}%
\pgfsetbuttcap%
\pgfsetroundjoin%
\definecolor{currentfill}{rgb}{0.121569,0.466667,0.705882}%
\pgfsetfillcolor{currentfill}%
\pgfsetfillopacity{0.799562}%
\pgfsetlinewidth{1.003750pt}%
\definecolor{currentstroke}{rgb}{0.121569,0.466667,0.705882}%
\pgfsetstrokecolor{currentstroke}%
\pgfsetstrokeopacity{0.799562}%
\pgfsetdash{}{0pt}%
\pgfpathmoveto{\pgfqpoint{2.362838in}{1.600518in}}%
\pgfpathcurveto{\pgfqpoint{2.371075in}{1.600518in}}{\pgfqpoint{2.378975in}{1.603790in}}{\pgfqpoint{2.384799in}{1.609614in}}%
\pgfpathcurveto{\pgfqpoint{2.390623in}{1.615438in}}{\pgfqpoint{2.393895in}{1.623338in}}{\pgfqpoint{2.393895in}{1.631574in}}%
\pgfpathcurveto{\pgfqpoint{2.393895in}{1.639810in}}{\pgfqpoint{2.390623in}{1.647710in}}{\pgfqpoint{2.384799in}{1.653534in}}%
\pgfpathcurveto{\pgfqpoint{2.378975in}{1.659358in}}{\pgfqpoint{2.371075in}{1.662631in}}{\pgfqpoint{2.362838in}{1.662631in}}%
\pgfpathcurveto{\pgfqpoint{2.354602in}{1.662631in}}{\pgfqpoint{2.346702in}{1.659358in}}{\pgfqpoint{2.340878in}{1.653534in}}%
\pgfpathcurveto{\pgfqpoint{2.335054in}{1.647710in}}{\pgfqpoint{2.331782in}{1.639810in}}{\pgfqpoint{2.331782in}{1.631574in}}%
\pgfpathcurveto{\pgfqpoint{2.331782in}{1.623338in}}{\pgfqpoint{2.335054in}{1.615438in}}{\pgfqpoint{2.340878in}{1.609614in}}%
\pgfpathcurveto{\pgfqpoint{2.346702in}{1.603790in}}{\pgfqpoint{2.354602in}{1.600518in}}{\pgfqpoint{2.362838in}{1.600518in}}%
\pgfpathclose%
\pgfusepath{stroke,fill}%
\end{pgfscope}%
\begin{pgfscope}%
\pgfpathrectangle{\pgfqpoint{0.100000in}{0.212622in}}{\pgfqpoint{3.696000in}{3.696000in}}%
\pgfusepath{clip}%
\pgfsetbuttcap%
\pgfsetroundjoin%
\definecolor{currentfill}{rgb}{0.121569,0.466667,0.705882}%
\pgfsetfillcolor{currentfill}%
\pgfsetfillopacity{0.801707}%
\pgfsetlinewidth{1.003750pt}%
\definecolor{currentstroke}{rgb}{0.121569,0.466667,0.705882}%
\pgfsetstrokecolor{currentstroke}%
\pgfsetstrokeopacity{0.801707}%
\pgfsetdash{}{0pt}%
\pgfpathmoveto{\pgfqpoint{1.428033in}{1.036222in}}%
\pgfpathcurveto{\pgfqpoint{1.436269in}{1.036222in}}{\pgfqpoint{1.444169in}{1.039495in}}{\pgfqpoint{1.449993in}{1.045319in}}%
\pgfpathcurveto{\pgfqpoint{1.455817in}{1.051142in}}{\pgfqpoint{1.459090in}{1.059043in}}{\pgfqpoint{1.459090in}{1.067279in}}%
\pgfpathcurveto{\pgfqpoint{1.459090in}{1.075515in}}{\pgfqpoint{1.455817in}{1.083415in}}{\pgfqpoint{1.449993in}{1.089239in}}%
\pgfpathcurveto{\pgfqpoint{1.444169in}{1.095063in}}{\pgfqpoint{1.436269in}{1.098335in}}{\pgfqpoint{1.428033in}{1.098335in}}%
\pgfpathcurveto{\pgfqpoint{1.419797in}{1.098335in}}{\pgfqpoint{1.411897in}{1.095063in}}{\pgfqpoint{1.406073in}{1.089239in}}%
\pgfpathcurveto{\pgfqpoint{1.400249in}{1.083415in}}{\pgfqpoint{1.396977in}{1.075515in}}{\pgfqpoint{1.396977in}{1.067279in}}%
\pgfpathcurveto{\pgfqpoint{1.396977in}{1.059043in}}{\pgfqpoint{1.400249in}{1.051142in}}{\pgfqpoint{1.406073in}{1.045319in}}%
\pgfpathcurveto{\pgfqpoint{1.411897in}{1.039495in}}{\pgfqpoint{1.419797in}{1.036222in}}{\pgfqpoint{1.428033in}{1.036222in}}%
\pgfpathclose%
\pgfusepath{stroke,fill}%
\end{pgfscope}%
\begin{pgfscope}%
\pgfpathrectangle{\pgfqpoint{0.100000in}{0.212622in}}{\pgfqpoint{3.696000in}{3.696000in}}%
\pgfusepath{clip}%
\pgfsetbuttcap%
\pgfsetroundjoin%
\definecolor{currentfill}{rgb}{0.121569,0.466667,0.705882}%
\pgfsetfillcolor{currentfill}%
\pgfsetfillopacity{0.803705}%
\pgfsetlinewidth{1.003750pt}%
\definecolor{currentstroke}{rgb}{0.121569,0.466667,0.705882}%
\pgfsetstrokecolor{currentstroke}%
\pgfsetstrokeopacity{0.803705}%
\pgfsetdash{}{0pt}%
\pgfpathmoveto{\pgfqpoint{2.366489in}{1.584435in}}%
\pgfpathcurveto{\pgfqpoint{2.374725in}{1.584435in}}{\pgfqpoint{2.382625in}{1.587707in}}{\pgfqpoint{2.388449in}{1.593531in}}%
\pgfpathcurveto{\pgfqpoint{2.394273in}{1.599355in}}{\pgfqpoint{2.397545in}{1.607255in}}{\pgfqpoint{2.397545in}{1.615491in}}%
\pgfpathcurveto{\pgfqpoint{2.397545in}{1.623728in}}{\pgfqpoint{2.394273in}{1.631628in}}{\pgfqpoint{2.388449in}{1.637452in}}%
\pgfpathcurveto{\pgfqpoint{2.382625in}{1.643276in}}{\pgfqpoint{2.374725in}{1.646548in}}{\pgfqpoint{2.366489in}{1.646548in}}%
\pgfpathcurveto{\pgfqpoint{2.358252in}{1.646548in}}{\pgfqpoint{2.350352in}{1.643276in}}{\pgfqpoint{2.344528in}{1.637452in}}%
\pgfpathcurveto{\pgfqpoint{2.338704in}{1.631628in}}{\pgfqpoint{2.335432in}{1.623728in}}{\pgfqpoint{2.335432in}{1.615491in}}%
\pgfpathcurveto{\pgfqpoint{2.335432in}{1.607255in}}{\pgfqpoint{2.338704in}{1.599355in}}{\pgfqpoint{2.344528in}{1.593531in}}%
\pgfpathcurveto{\pgfqpoint{2.350352in}{1.587707in}}{\pgfqpoint{2.358252in}{1.584435in}}{\pgfqpoint{2.366489in}{1.584435in}}%
\pgfpathclose%
\pgfusepath{stroke,fill}%
\end{pgfscope}%
\begin{pgfscope}%
\pgfpathrectangle{\pgfqpoint{0.100000in}{0.212622in}}{\pgfqpoint{3.696000in}{3.696000in}}%
\pgfusepath{clip}%
\pgfsetbuttcap%
\pgfsetroundjoin%
\definecolor{currentfill}{rgb}{0.121569,0.466667,0.705882}%
\pgfsetfillcolor{currentfill}%
\pgfsetfillopacity{0.804555}%
\pgfsetlinewidth{1.003750pt}%
\definecolor{currentstroke}{rgb}{0.121569,0.466667,0.705882}%
\pgfsetstrokecolor{currentstroke}%
\pgfsetstrokeopacity{0.804555}%
\pgfsetdash{}{0pt}%
\pgfpathmoveto{\pgfqpoint{1.441126in}{1.032928in}}%
\pgfpathcurveto{\pgfqpoint{1.449362in}{1.032928in}}{\pgfqpoint{1.457262in}{1.036200in}}{\pgfqpoint{1.463086in}{1.042024in}}%
\pgfpathcurveto{\pgfqpoint{1.468910in}{1.047848in}}{\pgfqpoint{1.472182in}{1.055748in}}{\pgfqpoint{1.472182in}{1.063984in}}%
\pgfpathcurveto{\pgfqpoint{1.472182in}{1.072220in}}{\pgfqpoint{1.468910in}{1.080120in}}{\pgfqpoint{1.463086in}{1.085944in}}%
\pgfpathcurveto{\pgfqpoint{1.457262in}{1.091768in}}{\pgfqpoint{1.449362in}{1.095041in}}{\pgfqpoint{1.441126in}{1.095041in}}%
\pgfpathcurveto{\pgfqpoint{1.432889in}{1.095041in}}{\pgfqpoint{1.424989in}{1.091768in}}{\pgfqpoint{1.419165in}{1.085944in}}%
\pgfpathcurveto{\pgfqpoint{1.413341in}{1.080120in}}{\pgfqpoint{1.410069in}{1.072220in}}{\pgfqpoint{1.410069in}{1.063984in}}%
\pgfpathcurveto{\pgfqpoint{1.410069in}{1.055748in}}{\pgfqpoint{1.413341in}{1.047848in}}{\pgfqpoint{1.419165in}{1.042024in}}%
\pgfpathcurveto{\pgfqpoint{1.424989in}{1.036200in}}{\pgfqpoint{1.432889in}{1.032928in}}{\pgfqpoint{1.441126in}{1.032928in}}%
\pgfpathclose%
\pgfusepath{stroke,fill}%
\end{pgfscope}%
\begin{pgfscope}%
\pgfpathrectangle{\pgfqpoint{0.100000in}{0.212622in}}{\pgfqpoint{3.696000in}{3.696000in}}%
\pgfusepath{clip}%
\pgfsetbuttcap%
\pgfsetroundjoin%
\definecolor{currentfill}{rgb}{0.121569,0.466667,0.705882}%
\pgfsetfillcolor{currentfill}%
\pgfsetfillopacity{0.809585}%
\pgfsetlinewidth{1.003750pt}%
\definecolor{currentstroke}{rgb}{0.121569,0.466667,0.705882}%
\pgfsetstrokecolor{currentstroke}%
\pgfsetstrokeopacity{0.809585}%
\pgfsetdash{}{0pt}%
\pgfpathmoveto{\pgfqpoint{1.465529in}{1.029236in}}%
\pgfpathcurveto{\pgfqpoint{1.473765in}{1.029236in}}{\pgfqpoint{1.481665in}{1.032509in}}{\pgfqpoint{1.487489in}{1.038332in}}%
\pgfpathcurveto{\pgfqpoint{1.493313in}{1.044156in}}{\pgfqpoint{1.496586in}{1.052056in}}{\pgfqpoint{1.496586in}{1.060293in}}%
\pgfpathcurveto{\pgfqpoint{1.496586in}{1.068529in}}{\pgfqpoint{1.493313in}{1.076429in}}{\pgfqpoint{1.487489in}{1.082253in}}%
\pgfpathcurveto{\pgfqpoint{1.481665in}{1.088077in}}{\pgfqpoint{1.473765in}{1.091349in}}{\pgfqpoint{1.465529in}{1.091349in}}%
\pgfpathcurveto{\pgfqpoint{1.457293in}{1.091349in}}{\pgfqpoint{1.449393in}{1.088077in}}{\pgfqpoint{1.443569in}{1.082253in}}%
\pgfpathcurveto{\pgfqpoint{1.437745in}{1.076429in}}{\pgfqpoint{1.434473in}{1.068529in}}{\pgfqpoint{1.434473in}{1.060293in}}%
\pgfpathcurveto{\pgfqpoint{1.434473in}{1.052056in}}{\pgfqpoint{1.437745in}{1.044156in}}{\pgfqpoint{1.443569in}{1.038332in}}%
\pgfpathcurveto{\pgfqpoint{1.449393in}{1.032509in}}{\pgfqpoint{1.457293in}{1.029236in}}{\pgfqpoint{1.465529in}{1.029236in}}%
\pgfpathclose%
\pgfusepath{stroke,fill}%
\end{pgfscope}%
\begin{pgfscope}%
\pgfpathrectangle{\pgfqpoint{0.100000in}{0.212622in}}{\pgfqpoint{3.696000in}{3.696000in}}%
\pgfusepath{clip}%
\pgfsetbuttcap%
\pgfsetroundjoin%
\definecolor{currentfill}{rgb}{0.121569,0.466667,0.705882}%
\pgfsetfillcolor{currentfill}%
\pgfsetfillopacity{0.809726}%
\pgfsetlinewidth{1.003750pt}%
\definecolor{currentstroke}{rgb}{0.121569,0.466667,0.705882}%
\pgfsetstrokecolor{currentstroke}%
\pgfsetstrokeopacity{0.809726}%
\pgfsetdash{}{0pt}%
\pgfpathmoveto{\pgfqpoint{2.372408in}{1.561457in}}%
\pgfpathcurveto{\pgfqpoint{2.380644in}{1.561457in}}{\pgfqpoint{2.388545in}{1.564730in}}{\pgfqpoint{2.394368in}{1.570554in}}%
\pgfpathcurveto{\pgfqpoint{2.400192in}{1.576378in}}{\pgfqpoint{2.403465in}{1.584278in}}{\pgfqpoint{2.403465in}{1.592514in}}%
\pgfpathcurveto{\pgfqpoint{2.403465in}{1.600750in}}{\pgfqpoint{2.400192in}{1.608650in}}{\pgfqpoint{2.394368in}{1.614474in}}%
\pgfpathcurveto{\pgfqpoint{2.388545in}{1.620298in}}{\pgfqpoint{2.380644in}{1.623570in}}{\pgfqpoint{2.372408in}{1.623570in}}%
\pgfpathcurveto{\pgfqpoint{2.364172in}{1.623570in}}{\pgfqpoint{2.356272in}{1.620298in}}{\pgfqpoint{2.350448in}{1.614474in}}%
\pgfpathcurveto{\pgfqpoint{2.344624in}{1.608650in}}{\pgfqpoint{2.341352in}{1.600750in}}{\pgfqpoint{2.341352in}{1.592514in}}%
\pgfpathcurveto{\pgfqpoint{2.341352in}{1.584278in}}{\pgfqpoint{2.344624in}{1.576378in}}{\pgfqpoint{2.350448in}{1.570554in}}%
\pgfpathcurveto{\pgfqpoint{2.356272in}{1.564730in}}{\pgfqpoint{2.364172in}{1.561457in}}{\pgfqpoint{2.372408in}{1.561457in}}%
\pgfpathclose%
\pgfusepath{stroke,fill}%
\end{pgfscope}%
\begin{pgfscope}%
\pgfpathrectangle{\pgfqpoint{0.100000in}{0.212622in}}{\pgfqpoint{3.696000in}{3.696000in}}%
\pgfusepath{clip}%
\pgfsetbuttcap%
\pgfsetroundjoin%
\definecolor{currentfill}{rgb}{0.121569,0.466667,0.705882}%
\pgfsetfillcolor{currentfill}%
\pgfsetfillopacity{0.814112}%
\pgfsetlinewidth{1.003750pt}%
\definecolor{currentstroke}{rgb}{0.121569,0.466667,0.705882}%
\pgfsetstrokecolor{currentstroke}%
\pgfsetstrokeopacity{0.814112}%
\pgfsetdash{}{0pt}%
\pgfpathmoveto{\pgfqpoint{1.487238in}{1.025608in}}%
\pgfpathcurveto{\pgfqpoint{1.495475in}{1.025608in}}{\pgfqpoint{1.503375in}{1.028880in}}{\pgfqpoint{1.509199in}{1.034704in}}%
\pgfpathcurveto{\pgfqpoint{1.515023in}{1.040528in}}{\pgfqpoint{1.518295in}{1.048428in}}{\pgfqpoint{1.518295in}{1.056664in}}%
\pgfpathcurveto{\pgfqpoint{1.518295in}{1.064900in}}{\pgfqpoint{1.515023in}{1.072801in}}{\pgfqpoint{1.509199in}{1.078624in}}%
\pgfpathcurveto{\pgfqpoint{1.503375in}{1.084448in}}{\pgfqpoint{1.495475in}{1.087721in}}{\pgfqpoint{1.487238in}{1.087721in}}%
\pgfpathcurveto{\pgfqpoint{1.479002in}{1.087721in}}{\pgfqpoint{1.471102in}{1.084448in}}{\pgfqpoint{1.465278in}{1.078624in}}%
\pgfpathcurveto{\pgfqpoint{1.459454in}{1.072801in}}{\pgfqpoint{1.456182in}{1.064900in}}{\pgfqpoint{1.456182in}{1.056664in}}%
\pgfpathcurveto{\pgfqpoint{1.456182in}{1.048428in}}{\pgfqpoint{1.459454in}{1.040528in}}{\pgfqpoint{1.465278in}{1.034704in}}%
\pgfpathcurveto{\pgfqpoint{1.471102in}{1.028880in}}{\pgfqpoint{1.479002in}{1.025608in}}{\pgfqpoint{1.487238in}{1.025608in}}%
\pgfpathclose%
\pgfusepath{stroke,fill}%
\end{pgfscope}%
\begin{pgfscope}%
\pgfpathrectangle{\pgfqpoint{0.100000in}{0.212622in}}{\pgfqpoint{3.696000in}{3.696000in}}%
\pgfusepath{clip}%
\pgfsetbuttcap%
\pgfsetroundjoin%
\definecolor{currentfill}{rgb}{0.121569,0.466667,0.705882}%
\pgfsetfillcolor{currentfill}%
\pgfsetfillopacity{0.816694}%
\pgfsetlinewidth{1.003750pt}%
\definecolor{currentstroke}{rgb}{0.121569,0.466667,0.705882}%
\pgfsetstrokecolor{currentstroke}%
\pgfsetstrokeopacity{0.816694}%
\pgfsetdash{}{0pt}%
\pgfpathmoveto{\pgfqpoint{2.379025in}{1.536340in}}%
\pgfpathcurveto{\pgfqpoint{2.387261in}{1.536340in}}{\pgfqpoint{2.395161in}{1.539613in}}{\pgfqpoint{2.400985in}{1.545437in}}%
\pgfpathcurveto{\pgfqpoint{2.406809in}{1.551261in}}{\pgfqpoint{2.410082in}{1.559161in}}{\pgfqpoint{2.410082in}{1.567397in}}%
\pgfpathcurveto{\pgfqpoint{2.410082in}{1.575633in}}{\pgfqpoint{2.406809in}{1.583533in}}{\pgfqpoint{2.400985in}{1.589357in}}%
\pgfpathcurveto{\pgfqpoint{2.395161in}{1.595181in}}{\pgfqpoint{2.387261in}{1.598453in}}{\pgfqpoint{2.379025in}{1.598453in}}%
\pgfpathcurveto{\pgfqpoint{2.370789in}{1.598453in}}{\pgfqpoint{2.362889in}{1.595181in}}{\pgfqpoint{2.357065in}{1.589357in}}%
\pgfpathcurveto{\pgfqpoint{2.351241in}{1.583533in}}{\pgfqpoint{2.347969in}{1.575633in}}{\pgfqpoint{2.347969in}{1.567397in}}%
\pgfpathcurveto{\pgfqpoint{2.347969in}{1.559161in}}{\pgfqpoint{2.351241in}{1.551261in}}{\pgfqpoint{2.357065in}{1.545437in}}%
\pgfpathcurveto{\pgfqpoint{2.362889in}{1.539613in}}{\pgfqpoint{2.370789in}{1.536340in}}{\pgfqpoint{2.379025in}{1.536340in}}%
\pgfpathclose%
\pgfusepath{stroke,fill}%
\end{pgfscope}%
\begin{pgfscope}%
\pgfpathrectangle{\pgfqpoint{0.100000in}{0.212622in}}{\pgfqpoint{3.696000in}{3.696000in}}%
\pgfusepath{clip}%
\pgfsetbuttcap%
\pgfsetroundjoin%
\definecolor{currentfill}{rgb}{0.121569,0.466667,0.705882}%
\pgfsetfillcolor{currentfill}%
\pgfsetfillopacity{0.818005}%
\pgfsetlinewidth{1.003750pt}%
\definecolor{currentstroke}{rgb}{0.121569,0.466667,0.705882}%
\pgfsetstrokecolor{currentstroke}%
\pgfsetstrokeopacity{0.818005}%
\pgfsetdash{}{0pt}%
\pgfpathmoveto{\pgfqpoint{1.504688in}{1.021187in}}%
\pgfpathcurveto{\pgfqpoint{1.512924in}{1.021187in}}{\pgfqpoint{1.520824in}{1.024459in}}{\pgfqpoint{1.526648in}{1.030283in}}%
\pgfpathcurveto{\pgfqpoint{1.532472in}{1.036107in}}{\pgfqpoint{1.535745in}{1.044007in}}{\pgfqpoint{1.535745in}{1.052244in}}%
\pgfpathcurveto{\pgfqpoint{1.535745in}{1.060480in}}{\pgfqpoint{1.532472in}{1.068380in}}{\pgfqpoint{1.526648in}{1.074204in}}%
\pgfpathcurveto{\pgfqpoint{1.520824in}{1.080028in}}{\pgfqpoint{1.512924in}{1.083300in}}{\pgfqpoint{1.504688in}{1.083300in}}%
\pgfpathcurveto{\pgfqpoint{1.496452in}{1.083300in}}{\pgfqpoint{1.488552in}{1.080028in}}{\pgfqpoint{1.482728in}{1.074204in}}%
\pgfpathcurveto{\pgfqpoint{1.476904in}{1.068380in}}{\pgfqpoint{1.473632in}{1.060480in}}{\pgfqpoint{1.473632in}{1.052244in}}%
\pgfpathcurveto{\pgfqpoint{1.473632in}{1.044007in}}{\pgfqpoint{1.476904in}{1.036107in}}{\pgfqpoint{1.482728in}{1.030283in}}%
\pgfpathcurveto{\pgfqpoint{1.488552in}{1.024459in}}{\pgfqpoint{1.496452in}{1.021187in}}{\pgfqpoint{1.504688in}{1.021187in}}%
\pgfpathclose%
\pgfusepath{stroke,fill}%
\end{pgfscope}%
\begin{pgfscope}%
\pgfpathrectangle{\pgfqpoint{0.100000in}{0.212622in}}{\pgfqpoint{3.696000in}{3.696000in}}%
\pgfusepath{clip}%
\pgfsetbuttcap%
\pgfsetroundjoin%
\definecolor{currentfill}{rgb}{0.121569,0.466667,0.705882}%
\pgfsetfillcolor{currentfill}%
\pgfsetfillopacity{0.820166}%
\pgfsetlinewidth{1.003750pt}%
\definecolor{currentstroke}{rgb}{0.121569,0.466667,0.705882}%
\pgfsetstrokecolor{currentstroke}%
\pgfsetstrokeopacity{0.820166}%
\pgfsetdash{}{0pt}%
\pgfpathmoveto{\pgfqpoint{1.515218in}{1.019890in}}%
\pgfpathcurveto{\pgfqpoint{1.523455in}{1.019890in}}{\pgfqpoint{1.531355in}{1.023163in}}{\pgfqpoint{1.537179in}{1.028987in}}%
\pgfpathcurveto{\pgfqpoint{1.543003in}{1.034810in}}{\pgfqpoint{1.546275in}{1.042710in}}{\pgfqpoint{1.546275in}{1.050947in}}%
\pgfpathcurveto{\pgfqpoint{1.546275in}{1.059183in}}{\pgfqpoint{1.543003in}{1.067083in}}{\pgfqpoint{1.537179in}{1.072907in}}%
\pgfpathcurveto{\pgfqpoint{1.531355in}{1.078731in}}{\pgfqpoint{1.523455in}{1.082003in}}{\pgfqpoint{1.515218in}{1.082003in}}%
\pgfpathcurveto{\pgfqpoint{1.506982in}{1.082003in}}{\pgfqpoint{1.499082in}{1.078731in}}{\pgfqpoint{1.493258in}{1.072907in}}%
\pgfpathcurveto{\pgfqpoint{1.487434in}{1.067083in}}{\pgfqpoint{1.484162in}{1.059183in}}{\pgfqpoint{1.484162in}{1.050947in}}%
\pgfpathcurveto{\pgfqpoint{1.484162in}{1.042710in}}{\pgfqpoint{1.487434in}{1.034810in}}{\pgfqpoint{1.493258in}{1.028987in}}%
\pgfpathcurveto{\pgfqpoint{1.499082in}{1.023163in}}{\pgfqpoint{1.506982in}{1.019890in}}{\pgfqpoint{1.515218in}{1.019890in}}%
\pgfpathclose%
\pgfusepath{stroke,fill}%
\end{pgfscope}%
\begin{pgfscope}%
\pgfpathrectangle{\pgfqpoint{0.100000in}{0.212622in}}{\pgfqpoint{3.696000in}{3.696000in}}%
\pgfusepath{clip}%
\pgfsetbuttcap%
\pgfsetroundjoin%
\definecolor{currentfill}{rgb}{0.121569,0.466667,0.705882}%
\pgfsetfillcolor{currentfill}%
\pgfsetfillopacity{0.820489}%
\pgfsetlinewidth{1.003750pt}%
\definecolor{currentstroke}{rgb}{0.121569,0.466667,0.705882}%
\pgfsetstrokecolor{currentstroke}%
\pgfsetstrokeopacity{0.820489}%
\pgfsetdash{}{0pt}%
\pgfpathmoveto{\pgfqpoint{2.382675in}{1.522356in}}%
\pgfpathcurveto{\pgfqpoint{2.390912in}{1.522356in}}{\pgfqpoint{2.398812in}{1.525628in}}{\pgfqpoint{2.404636in}{1.531452in}}%
\pgfpathcurveto{\pgfqpoint{2.410460in}{1.537276in}}{\pgfqpoint{2.413732in}{1.545176in}}{\pgfqpoint{2.413732in}{1.553412in}}%
\pgfpathcurveto{\pgfqpoint{2.413732in}{1.561648in}}{\pgfqpoint{2.410460in}{1.569548in}}{\pgfqpoint{2.404636in}{1.575372in}}%
\pgfpathcurveto{\pgfqpoint{2.398812in}{1.581196in}}{\pgfqpoint{2.390912in}{1.584469in}}{\pgfqpoint{2.382675in}{1.584469in}}%
\pgfpathcurveto{\pgfqpoint{2.374439in}{1.584469in}}{\pgfqpoint{2.366539in}{1.581196in}}{\pgfqpoint{2.360715in}{1.575372in}}%
\pgfpathcurveto{\pgfqpoint{2.354891in}{1.569548in}}{\pgfqpoint{2.351619in}{1.561648in}}{\pgfqpoint{2.351619in}{1.553412in}}%
\pgfpathcurveto{\pgfqpoint{2.351619in}{1.545176in}}{\pgfqpoint{2.354891in}{1.537276in}}{\pgfqpoint{2.360715in}{1.531452in}}%
\pgfpathcurveto{\pgfqpoint{2.366539in}{1.525628in}}{\pgfqpoint{2.374439in}{1.522356in}}{\pgfqpoint{2.382675in}{1.522356in}}%
\pgfpathclose%
\pgfusepath{stroke,fill}%
\end{pgfscope}%
\begin{pgfscope}%
\pgfpathrectangle{\pgfqpoint{0.100000in}{0.212622in}}{\pgfqpoint{3.696000in}{3.696000in}}%
\pgfusepath{clip}%
\pgfsetbuttcap%
\pgfsetroundjoin%
\definecolor{currentfill}{rgb}{0.121569,0.466667,0.705882}%
\pgfsetfillcolor{currentfill}%
\pgfsetfillopacity{0.821927}%
\pgfsetlinewidth{1.003750pt}%
\definecolor{currentstroke}{rgb}{0.121569,0.466667,0.705882}%
\pgfsetstrokecolor{currentstroke}%
\pgfsetstrokeopacity{0.821927}%
\pgfsetdash{}{0pt}%
\pgfpathmoveto{\pgfqpoint{1.523390in}{1.018211in}}%
\pgfpathcurveto{\pgfqpoint{1.531626in}{1.018211in}}{\pgfqpoint{1.539526in}{1.021484in}}{\pgfqpoint{1.545350in}{1.027307in}}%
\pgfpathcurveto{\pgfqpoint{1.551174in}{1.033131in}}{\pgfqpoint{1.554446in}{1.041031in}}{\pgfqpoint{1.554446in}{1.049268in}}%
\pgfpathcurveto{\pgfqpoint{1.554446in}{1.057504in}}{\pgfqpoint{1.551174in}{1.065404in}}{\pgfqpoint{1.545350in}{1.071228in}}%
\pgfpathcurveto{\pgfqpoint{1.539526in}{1.077052in}}{\pgfqpoint{1.531626in}{1.080324in}}{\pgfqpoint{1.523390in}{1.080324in}}%
\pgfpathcurveto{\pgfqpoint{1.515154in}{1.080324in}}{\pgfqpoint{1.507254in}{1.077052in}}{\pgfqpoint{1.501430in}{1.071228in}}%
\pgfpathcurveto{\pgfqpoint{1.495606in}{1.065404in}}{\pgfqpoint{1.492333in}{1.057504in}}{\pgfqpoint{1.492333in}{1.049268in}}%
\pgfpathcurveto{\pgfqpoint{1.492333in}{1.041031in}}{\pgfqpoint{1.495606in}{1.033131in}}{\pgfqpoint{1.501430in}{1.027307in}}%
\pgfpathcurveto{\pgfqpoint{1.507254in}{1.021484in}}{\pgfqpoint{1.515154in}{1.018211in}}{\pgfqpoint{1.523390in}{1.018211in}}%
\pgfpathclose%
\pgfusepath{stroke,fill}%
\end{pgfscope}%
\begin{pgfscope}%
\pgfpathrectangle{\pgfqpoint{0.100000in}{0.212622in}}{\pgfqpoint{3.696000in}{3.696000in}}%
\pgfusepath{clip}%
\pgfsetbuttcap%
\pgfsetroundjoin%
\definecolor{currentfill}{rgb}{0.121569,0.466667,0.705882}%
\pgfsetfillcolor{currentfill}%
\pgfsetfillopacity{0.825142}%
\pgfsetlinewidth{1.003750pt}%
\definecolor{currentstroke}{rgb}{0.121569,0.466667,0.705882}%
\pgfsetstrokecolor{currentstroke}%
\pgfsetstrokeopacity{0.825142}%
\pgfsetdash{}{0pt}%
\pgfpathmoveto{\pgfqpoint{1.538340in}{1.015630in}}%
\pgfpathcurveto{\pgfqpoint{1.546576in}{1.015630in}}{\pgfqpoint{1.554477in}{1.018902in}}{\pgfqpoint{1.560300in}{1.024726in}}%
\pgfpathcurveto{\pgfqpoint{1.566124in}{1.030550in}}{\pgfqpoint{1.569397in}{1.038450in}}{\pgfqpoint{1.569397in}{1.046687in}}%
\pgfpathcurveto{\pgfqpoint{1.569397in}{1.054923in}}{\pgfqpoint{1.566124in}{1.062823in}}{\pgfqpoint{1.560300in}{1.068647in}}%
\pgfpathcurveto{\pgfqpoint{1.554477in}{1.074471in}}{\pgfqpoint{1.546576in}{1.077743in}}{\pgfqpoint{1.538340in}{1.077743in}}%
\pgfpathcurveto{\pgfqpoint{1.530104in}{1.077743in}}{\pgfqpoint{1.522204in}{1.074471in}}{\pgfqpoint{1.516380in}{1.068647in}}%
\pgfpathcurveto{\pgfqpoint{1.510556in}{1.062823in}}{\pgfqpoint{1.507284in}{1.054923in}}{\pgfqpoint{1.507284in}{1.046687in}}%
\pgfpathcurveto{\pgfqpoint{1.507284in}{1.038450in}}{\pgfqpoint{1.510556in}{1.030550in}}{\pgfqpoint{1.516380in}{1.024726in}}%
\pgfpathcurveto{\pgfqpoint{1.522204in}{1.018902in}}{\pgfqpoint{1.530104in}{1.015630in}}{\pgfqpoint{1.538340in}{1.015630in}}%
\pgfpathclose%
\pgfusepath{stroke,fill}%
\end{pgfscope}%
\begin{pgfscope}%
\pgfpathrectangle{\pgfqpoint{0.100000in}{0.212622in}}{\pgfqpoint{3.696000in}{3.696000in}}%
\pgfusepath{clip}%
\pgfsetbuttcap%
\pgfsetroundjoin%
\definecolor{currentfill}{rgb}{0.121569,0.466667,0.705882}%
\pgfsetfillcolor{currentfill}%
\pgfsetfillopacity{0.825358}%
\pgfsetlinewidth{1.003750pt}%
\definecolor{currentstroke}{rgb}{0.121569,0.466667,0.705882}%
\pgfsetstrokecolor{currentstroke}%
\pgfsetstrokeopacity{0.825358}%
\pgfsetdash{}{0pt}%
\pgfpathmoveto{\pgfqpoint{2.387391in}{1.504754in}}%
\pgfpathcurveto{\pgfqpoint{2.395627in}{1.504754in}}{\pgfqpoint{2.403527in}{1.508026in}}{\pgfqpoint{2.409351in}{1.513850in}}%
\pgfpathcurveto{\pgfqpoint{2.415175in}{1.519674in}}{\pgfqpoint{2.418447in}{1.527574in}}{\pgfqpoint{2.418447in}{1.535810in}}%
\pgfpathcurveto{\pgfqpoint{2.418447in}{1.544047in}}{\pgfqpoint{2.415175in}{1.551947in}}{\pgfqpoint{2.409351in}{1.557771in}}%
\pgfpathcurveto{\pgfqpoint{2.403527in}{1.563595in}}{\pgfqpoint{2.395627in}{1.566867in}}{\pgfqpoint{2.387391in}{1.566867in}}%
\pgfpathcurveto{\pgfqpoint{2.379155in}{1.566867in}}{\pgfqpoint{2.371255in}{1.563595in}}{\pgfqpoint{2.365431in}{1.557771in}}%
\pgfpathcurveto{\pgfqpoint{2.359607in}{1.551947in}}{\pgfqpoint{2.356334in}{1.544047in}}{\pgfqpoint{2.356334in}{1.535810in}}%
\pgfpathcurveto{\pgfqpoint{2.356334in}{1.527574in}}{\pgfqpoint{2.359607in}{1.519674in}}{\pgfqpoint{2.365431in}{1.513850in}}%
\pgfpathcurveto{\pgfqpoint{2.371255in}{1.508026in}}{\pgfqpoint{2.379155in}{1.504754in}}{\pgfqpoint{2.387391in}{1.504754in}}%
\pgfpathclose%
\pgfusepath{stroke,fill}%
\end{pgfscope}%
\begin{pgfscope}%
\pgfpathrectangle{\pgfqpoint{0.100000in}{0.212622in}}{\pgfqpoint{3.696000in}{3.696000in}}%
\pgfusepath{clip}%
\pgfsetbuttcap%
\pgfsetroundjoin%
\definecolor{currentfill}{rgb}{0.121569,0.466667,0.705882}%
\pgfsetfillcolor{currentfill}%
\pgfsetfillopacity{0.830914}%
\pgfsetlinewidth{1.003750pt}%
\definecolor{currentstroke}{rgb}{0.121569,0.466667,0.705882}%
\pgfsetstrokecolor{currentstroke}%
\pgfsetstrokeopacity{0.830914}%
\pgfsetdash{}{0pt}%
\pgfpathmoveto{\pgfqpoint{1.565668in}{1.011255in}}%
\pgfpathcurveto{\pgfqpoint{1.573904in}{1.011255in}}{\pgfqpoint{1.581804in}{1.014527in}}{\pgfqpoint{1.587628in}{1.020351in}}%
\pgfpathcurveto{\pgfqpoint{1.593452in}{1.026175in}}{\pgfqpoint{1.596724in}{1.034075in}}{\pgfqpoint{1.596724in}{1.042312in}}%
\pgfpathcurveto{\pgfqpoint{1.596724in}{1.050548in}}{\pgfqpoint{1.593452in}{1.058448in}}{\pgfqpoint{1.587628in}{1.064272in}}%
\pgfpathcurveto{\pgfqpoint{1.581804in}{1.070096in}}{\pgfqpoint{1.573904in}{1.073368in}}{\pgfqpoint{1.565668in}{1.073368in}}%
\pgfpathcurveto{\pgfqpoint{1.557431in}{1.073368in}}{\pgfqpoint{1.549531in}{1.070096in}}{\pgfqpoint{1.543707in}{1.064272in}}%
\pgfpathcurveto{\pgfqpoint{1.537883in}{1.058448in}}{\pgfqpoint{1.534611in}{1.050548in}}{\pgfqpoint{1.534611in}{1.042312in}}%
\pgfpathcurveto{\pgfqpoint{1.534611in}{1.034075in}}{\pgfqpoint{1.537883in}{1.026175in}}{\pgfqpoint{1.543707in}{1.020351in}}%
\pgfpathcurveto{\pgfqpoint{1.549531in}{1.014527in}}{\pgfqpoint{1.557431in}{1.011255in}}{\pgfqpoint{1.565668in}{1.011255in}}%
\pgfpathclose%
\pgfusepath{stroke,fill}%
\end{pgfscope}%
\begin{pgfscope}%
\pgfpathrectangle{\pgfqpoint{0.100000in}{0.212622in}}{\pgfqpoint{3.696000in}{3.696000in}}%
\pgfusepath{clip}%
\pgfsetbuttcap%
\pgfsetroundjoin%
\definecolor{currentfill}{rgb}{0.121569,0.466667,0.705882}%
\pgfsetfillcolor{currentfill}%
\pgfsetfillopacity{0.831236}%
\pgfsetlinewidth{1.003750pt}%
\definecolor{currentstroke}{rgb}{0.121569,0.466667,0.705882}%
\pgfsetstrokecolor{currentstroke}%
\pgfsetstrokeopacity{0.831236}%
\pgfsetdash{}{0pt}%
\pgfpathmoveto{\pgfqpoint{2.393234in}{1.483534in}}%
\pgfpathcurveto{\pgfqpoint{2.401470in}{1.483534in}}{\pgfqpoint{2.409370in}{1.486806in}}{\pgfqpoint{2.415194in}{1.492630in}}%
\pgfpathcurveto{\pgfqpoint{2.421018in}{1.498454in}}{\pgfqpoint{2.424290in}{1.506354in}}{\pgfqpoint{2.424290in}{1.514590in}}%
\pgfpathcurveto{\pgfqpoint{2.424290in}{1.522827in}}{\pgfqpoint{2.421018in}{1.530727in}}{\pgfqpoint{2.415194in}{1.536551in}}%
\pgfpathcurveto{\pgfqpoint{2.409370in}{1.542375in}}{\pgfqpoint{2.401470in}{1.545647in}}{\pgfqpoint{2.393234in}{1.545647in}}%
\pgfpathcurveto{\pgfqpoint{2.384998in}{1.545647in}}{\pgfqpoint{2.377098in}{1.542375in}}{\pgfqpoint{2.371274in}{1.536551in}}%
\pgfpathcurveto{\pgfqpoint{2.365450in}{1.530727in}}{\pgfqpoint{2.362177in}{1.522827in}}{\pgfqpoint{2.362177in}{1.514590in}}%
\pgfpathcurveto{\pgfqpoint{2.362177in}{1.506354in}}{\pgfqpoint{2.365450in}{1.498454in}}{\pgfqpoint{2.371274in}{1.492630in}}%
\pgfpathcurveto{\pgfqpoint{2.377098in}{1.486806in}}{\pgfqpoint{2.384998in}{1.483534in}}{\pgfqpoint{2.393234in}{1.483534in}}%
\pgfpathclose%
\pgfusepath{stroke,fill}%
\end{pgfscope}%
\begin{pgfscope}%
\pgfpathrectangle{\pgfqpoint{0.100000in}{0.212622in}}{\pgfqpoint{3.696000in}{3.696000in}}%
\pgfusepath{clip}%
\pgfsetbuttcap%
\pgfsetroundjoin%
\definecolor{currentfill}{rgb}{0.121569,0.466667,0.705882}%
\pgfsetfillcolor{currentfill}%
\pgfsetfillopacity{0.835760}%
\pgfsetlinewidth{1.003750pt}%
\definecolor{currentstroke}{rgb}{0.121569,0.466667,0.705882}%
\pgfsetstrokecolor{currentstroke}%
\pgfsetstrokeopacity{0.835760}%
\pgfsetdash{}{0pt}%
\pgfpathmoveto{\pgfqpoint{1.587689in}{1.006646in}}%
\pgfpathcurveto{\pgfqpoint{1.595926in}{1.006646in}}{\pgfqpoint{1.603826in}{1.009918in}}{\pgfqpoint{1.609650in}{1.015742in}}%
\pgfpathcurveto{\pgfqpoint{1.615474in}{1.021566in}}{\pgfqpoint{1.618746in}{1.029466in}}{\pgfqpoint{1.618746in}{1.037702in}}%
\pgfpathcurveto{\pgfqpoint{1.618746in}{1.045939in}}{\pgfqpoint{1.615474in}{1.053839in}}{\pgfqpoint{1.609650in}{1.059663in}}%
\pgfpathcurveto{\pgfqpoint{1.603826in}{1.065487in}}{\pgfqpoint{1.595926in}{1.068759in}}{\pgfqpoint{1.587689in}{1.068759in}}%
\pgfpathcurveto{\pgfqpoint{1.579453in}{1.068759in}}{\pgfqpoint{1.571553in}{1.065487in}}{\pgfqpoint{1.565729in}{1.059663in}}%
\pgfpathcurveto{\pgfqpoint{1.559905in}{1.053839in}}{\pgfqpoint{1.556633in}{1.045939in}}{\pgfqpoint{1.556633in}{1.037702in}}%
\pgfpathcurveto{\pgfqpoint{1.556633in}{1.029466in}}{\pgfqpoint{1.559905in}{1.021566in}}{\pgfqpoint{1.565729in}{1.015742in}}%
\pgfpathcurveto{\pgfqpoint{1.571553in}{1.009918in}}{\pgfqpoint{1.579453in}{1.006646in}}{\pgfqpoint{1.587689in}{1.006646in}}%
\pgfpathclose%
\pgfusepath{stroke,fill}%
\end{pgfscope}%
\begin{pgfscope}%
\pgfpathrectangle{\pgfqpoint{0.100000in}{0.212622in}}{\pgfqpoint{3.696000in}{3.696000in}}%
\pgfusepath{clip}%
\pgfsetbuttcap%
\pgfsetroundjoin%
\definecolor{currentfill}{rgb}{0.121569,0.466667,0.705882}%
\pgfsetfillcolor{currentfill}%
\pgfsetfillopacity{0.838231}%
\pgfsetlinewidth{1.003750pt}%
\definecolor{currentstroke}{rgb}{0.121569,0.466667,0.705882}%
\pgfsetstrokecolor{currentstroke}%
\pgfsetstrokeopacity{0.838231}%
\pgfsetdash{}{0pt}%
\pgfpathmoveto{\pgfqpoint{2.399867in}{1.458696in}}%
\pgfpathcurveto{\pgfqpoint{2.408103in}{1.458696in}}{\pgfqpoint{2.416003in}{1.461969in}}{\pgfqpoint{2.421827in}{1.467793in}}%
\pgfpathcurveto{\pgfqpoint{2.427651in}{1.473616in}}{\pgfqpoint{2.430923in}{1.481516in}}{\pgfqpoint{2.430923in}{1.489753in}}%
\pgfpathcurveto{\pgfqpoint{2.430923in}{1.497989in}}{\pgfqpoint{2.427651in}{1.505889in}}{\pgfqpoint{2.421827in}{1.511713in}}%
\pgfpathcurveto{\pgfqpoint{2.416003in}{1.517537in}}{\pgfqpoint{2.408103in}{1.520809in}}{\pgfqpoint{2.399867in}{1.520809in}}%
\pgfpathcurveto{\pgfqpoint{2.391630in}{1.520809in}}{\pgfqpoint{2.383730in}{1.517537in}}{\pgfqpoint{2.377906in}{1.511713in}}%
\pgfpathcurveto{\pgfqpoint{2.372082in}{1.505889in}}{\pgfqpoint{2.368810in}{1.497989in}}{\pgfqpoint{2.368810in}{1.489753in}}%
\pgfpathcurveto{\pgfqpoint{2.368810in}{1.481516in}}{\pgfqpoint{2.372082in}{1.473616in}}{\pgfqpoint{2.377906in}{1.467793in}}%
\pgfpathcurveto{\pgfqpoint{2.383730in}{1.461969in}}{\pgfqpoint{2.391630in}{1.458696in}}{\pgfqpoint{2.399867in}{1.458696in}}%
\pgfpathclose%
\pgfusepath{stroke,fill}%
\end{pgfscope}%
\begin{pgfscope}%
\pgfpathrectangle{\pgfqpoint{0.100000in}{0.212622in}}{\pgfqpoint{3.696000in}{3.696000in}}%
\pgfusepath{clip}%
\pgfsetbuttcap%
\pgfsetroundjoin%
\definecolor{currentfill}{rgb}{0.121569,0.466667,0.705882}%
\pgfsetfillcolor{currentfill}%
\pgfsetfillopacity{0.838916}%
\pgfsetlinewidth{1.003750pt}%
\definecolor{currentstroke}{rgb}{0.121569,0.466667,0.705882}%
\pgfsetstrokecolor{currentstroke}%
\pgfsetstrokeopacity{0.838916}%
\pgfsetdash{}{0pt}%
\pgfpathmoveto{\pgfqpoint{1.602429in}{1.004277in}}%
\pgfpathcurveto{\pgfqpoint{1.610665in}{1.004277in}}{\pgfqpoint{1.618565in}{1.007550in}}{\pgfqpoint{1.624389in}{1.013374in}}%
\pgfpathcurveto{\pgfqpoint{1.630213in}{1.019198in}}{\pgfqpoint{1.633485in}{1.027098in}}{\pgfqpoint{1.633485in}{1.035334in}}%
\pgfpathcurveto{\pgfqpoint{1.633485in}{1.043570in}}{\pgfqpoint{1.630213in}{1.051470in}}{\pgfqpoint{1.624389in}{1.057294in}}%
\pgfpathcurveto{\pgfqpoint{1.618565in}{1.063118in}}{\pgfqpoint{1.610665in}{1.066390in}}{\pgfqpoint{1.602429in}{1.066390in}}%
\pgfpathcurveto{\pgfqpoint{1.594192in}{1.066390in}}{\pgfqpoint{1.586292in}{1.063118in}}{\pgfqpoint{1.580468in}{1.057294in}}%
\pgfpathcurveto{\pgfqpoint{1.574644in}{1.051470in}}{\pgfqpoint{1.571372in}{1.043570in}}{\pgfqpoint{1.571372in}{1.035334in}}%
\pgfpathcurveto{\pgfqpoint{1.571372in}{1.027098in}}{\pgfqpoint{1.574644in}{1.019198in}}{\pgfqpoint{1.580468in}{1.013374in}}%
\pgfpathcurveto{\pgfqpoint{1.586292in}{1.007550in}}{\pgfqpoint{1.594192in}{1.004277in}}{\pgfqpoint{1.602429in}{1.004277in}}%
\pgfpathclose%
\pgfusepath{stroke,fill}%
\end{pgfscope}%
\begin{pgfscope}%
\pgfpathrectangle{\pgfqpoint{0.100000in}{0.212622in}}{\pgfqpoint{3.696000in}{3.696000in}}%
\pgfusepath{clip}%
\pgfsetbuttcap%
\pgfsetroundjoin%
\definecolor{currentfill}{rgb}{0.121569,0.466667,0.705882}%
\pgfsetfillcolor{currentfill}%
\pgfsetfillopacity{0.840696}%
\pgfsetlinewidth{1.003750pt}%
\definecolor{currentstroke}{rgb}{0.121569,0.466667,0.705882}%
\pgfsetstrokecolor{currentstroke}%
\pgfsetstrokeopacity{0.840696}%
\pgfsetdash{}{0pt}%
\pgfpathmoveto{\pgfqpoint{1.610722in}{1.002747in}}%
\pgfpathcurveto{\pgfqpoint{1.618959in}{1.002747in}}{\pgfqpoint{1.626859in}{1.006020in}}{\pgfqpoint{1.632683in}{1.011844in}}%
\pgfpathcurveto{\pgfqpoint{1.638506in}{1.017668in}}{\pgfqpoint{1.641779in}{1.025568in}}{\pgfqpoint{1.641779in}{1.033804in}}%
\pgfpathcurveto{\pgfqpoint{1.641779in}{1.042040in}}{\pgfqpoint{1.638506in}{1.049940in}}{\pgfqpoint{1.632683in}{1.055764in}}%
\pgfpathcurveto{\pgfqpoint{1.626859in}{1.061588in}}{\pgfqpoint{1.618959in}{1.064860in}}{\pgfqpoint{1.610722in}{1.064860in}}%
\pgfpathcurveto{\pgfqpoint{1.602486in}{1.064860in}}{\pgfqpoint{1.594586in}{1.061588in}}{\pgfqpoint{1.588762in}{1.055764in}}%
\pgfpathcurveto{\pgfqpoint{1.582938in}{1.049940in}}{\pgfqpoint{1.579666in}{1.042040in}}{\pgfqpoint{1.579666in}{1.033804in}}%
\pgfpathcurveto{\pgfqpoint{1.579666in}{1.025568in}}{\pgfqpoint{1.582938in}{1.017668in}}{\pgfqpoint{1.588762in}{1.011844in}}%
\pgfpathcurveto{\pgfqpoint{1.594586in}{1.006020in}}{\pgfqpoint{1.602486in}{1.002747in}}{\pgfqpoint{1.610722in}{1.002747in}}%
\pgfpathclose%
\pgfusepath{stroke,fill}%
\end{pgfscope}%
\begin{pgfscope}%
\pgfpathrectangle{\pgfqpoint{0.100000in}{0.212622in}}{\pgfqpoint{3.696000in}{3.696000in}}%
\pgfusepath{clip}%
\pgfsetbuttcap%
\pgfsetroundjoin%
\definecolor{currentfill}{rgb}{0.121569,0.466667,0.705882}%
\pgfsetfillcolor{currentfill}%
\pgfsetfillopacity{0.843906}%
\pgfsetlinewidth{1.003750pt}%
\definecolor{currentstroke}{rgb}{0.121569,0.466667,0.705882}%
\pgfsetstrokecolor{currentstroke}%
\pgfsetstrokeopacity{0.843906}%
\pgfsetdash{}{0pt}%
\pgfpathmoveto{\pgfqpoint{1.625921in}{1.000453in}}%
\pgfpathcurveto{\pgfqpoint{1.634157in}{1.000453in}}{\pgfqpoint{1.642057in}{1.003726in}}{\pgfqpoint{1.647881in}{1.009550in}}%
\pgfpathcurveto{\pgfqpoint{1.653705in}{1.015373in}}{\pgfqpoint{1.656978in}{1.023274in}}{\pgfqpoint{1.656978in}{1.031510in}}%
\pgfpathcurveto{\pgfqpoint{1.656978in}{1.039746in}}{\pgfqpoint{1.653705in}{1.047646in}}{\pgfqpoint{1.647881in}{1.053470in}}%
\pgfpathcurveto{\pgfqpoint{1.642057in}{1.059294in}}{\pgfqpoint{1.634157in}{1.062566in}}{\pgfqpoint{1.625921in}{1.062566in}}%
\pgfpathcurveto{\pgfqpoint{1.617685in}{1.062566in}}{\pgfqpoint{1.609785in}{1.059294in}}{\pgfqpoint{1.603961in}{1.053470in}}%
\pgfpathcurveto{\pgfqpoint{1.598137in}{1.047646in}}{\pgfqpoint{1.594865in}{1.039746in}}{\pgfqpoint{1.594865in}{1.031510in}}%
\pgfpathcurveto{\pgfqpoint{1.594865in}{1.023274in}}{\pgfqpoint{1.598137in}{1.015373in}}{\pgfqpoint{1.603961in}{1.009550in}}%
\pgfpathcurveto{\pgfqpoint{1.609785in}{1.003726in}}{\pgfqpoint{1.617685in}{1.000453in}}{\pgfqpoint{1.625921in}{1.000453in}}%
\pgfpathclose%
\pgfusepath{stroke,fill}%
\end{pgfscope}%
\begin{pgfscope}%
\pgfpathrectangle{\pgfqpoint{0.100000in}{0.212622in}}{\pgfqpoint{3.696000in}{3.696000in}}%
\pgfusepath{clip}%
\pgfsetbuttcap%
\pgfsetroundjoin%
\definecolor{currentfill}{rgb}{0.121569,0.466667,0.705882}%
\pgfsetfillcolor{currentfill}%
\pgfsetfillopacity{0.845478}%
\pgfsetlinewidth{1.003750pt}%
\definecolor{currentstroke}{rgb}{0.121569,0.466667,0.705882}%
\pgfsetstrokecolor{currentstroke}%
\pgfsetstrokeopacity{0.845478}%
\pgfsetdash{}{0pt}%
\pgfpathmoveto{\pgfqpoint{2.405736in}{1.431069in}}%
\pgfpathcurveto{\pgfqpoint{2.413973in}{1.431069in}}{\pgfqpoint{2.421873in}{1.434342in}}{\pgfqpoint{2.427696in}{1.440166in}}%
\pgfpathcurveto{\pgfqpoint{2.433520in}{1.445990in}}{\pgfqpoint{2.436793in}{1.453890in}}{\pgfqpoint{2.436793in}{1.462126in}}%
\pgfpathcurveto{\pgfqpoint{2.436793in}{1.470362in}}{\pgfqpoint{2.433520in}{1.478262in}}{\pgfqpoint{2.427696in}{1.484086in}}%
\pgfpathcurveto{\pgfqpoint{2.421873in}{1.489910in}}{\pgfqpoint{2.413973in}{1.493182in}}{\pgfqpoint{2.405736in}{1.493182in}}%
\pgfpathcurveto{\pgfqpoint{2.397500in}{1.493182in}}{\pgfqpoint{2.389600in}{1.489910in}}{\pgfqpoint{2.383776in}{1.484086in}}%
\pgfpathcurveto{\pgfqpoint{2.377952in}{1.478262in}}{\pgfqpoint{2.374680in}{1.470362in}}{\pgfqpoint{2.374680in}{1.462126in}}%
\pgfpathcurveto{\pgfqpoint{2.374680in}{1.453890in}}{\pgfqpoint{2.377952in}{1.445990in}}{\pgfqpoint{2.383776in}{1.440166in}}%
\pgfpathcurveto{\pgfqpoint{2.389600in}{1.434342in}}{\pgfqpoint{2.397500in}{1.431069in}}{\pgfqpoint{2.405736in}{1.431069in}}%
\pgfpathclose%
\pgfusepath{stroke,fill}%
\end{pgfscope}%
\begin{pgfscope}%
\pgfpathrectangle{\pgfqpoint{0.100000in}{0.212622in}}{\pgfqpoint{3.696000in}{3.696000in}}%
\pgfusepath{clip}%
\pgfsetbuttcap%
\pgfsetroundjoin%
\definecolor{currentfill}{rgb}{0.121569,0.466667,0.705882}%
\pgfsetfillcolor{currentfill}%
\pgfsetfillopacity{0.849845}%
\pgfsetlinewidth{1.003750pt}%
\definecolor{currentstroke}{rgb}{0.121569,0.466667,0.705882}%
\pgfsetstrokecolor{currentstroke}%
\pgfsetstrokeopacity{0.849845}%
\pgfsetdash{}{0pt}%
\pgfpathmoveto{\pgfqpoint{1.653382in}{0.995565in}}%
\pgfpathcurveto{\pgfqpoint{1.661618in}{0.995565in}}{\pgfqpoint{1.669518in}{0.998838in}}{\pgfqpoint{1.675342in}{1.004662in}}%
\pgfpathcurveto{\pgfqpoint{1.681166in}{1.010485in}}{\pgfqpoint{1.684438in}{1.018386in}}{\pgfqpoint{1.684438in}{1.026622in}}%
\pgfpathcurveto{\pgfqpoint{1.684438in}{1.034858in}}{\pgfqpoint{1.681166in}{1.042758in}}{\pgfqpoint{1.675342in}{1.048582in}}%
\pgfpathcurveto{\pgfqpoint{1.669518in}{1.054406in}}{\pgfqpoint{1.661618in}{1.057678in}}{\pgfqpoint{1.653382in}{1.057678in}}%
\pgfpathcurveto{\pgfqpoint{1.645145in}{1.057678in}}{\pgfqpoint{1.637245in}{1.054406in}}{\pgfqpoint{1.631421in}{1.048582in}}%
\pgfpathcurveto{\pgfqpoint{1.625597in}{1.042758in}}{\pgfqpoint{1.622325in}{1.034858in}}{\pgfqpoint{1.622325in}{1.026622in}}%
\pgfpathcurveto{\pgfqpoint{1.622325in}{1.018386in}}{\pgfqpoint{1.625597in}{1.010485in}}{\pgfqpoint{1.631421in}{1.004662in}}%
\pgfpathcurveto{\pgfqpoint{1.637245in}{0.998838in}}{\pgfqpoint{1.645145in}{0.995565in}}{\pgfqpoint{1.653382in}{0.995565in}}%
\pgfpathclose%
\pgfusepath{stroke,fill}%
\end{pgfscope}%
\begin{pgfscope}%
\pgfpathrectangle{\pgfqpoint{0.100000in}{0.212622in}}{\pgfqpoint{3.696000in}{3.696000in}}%
\pgfusepath{clip}%
\pgfsetbuttcap%
\pgfsetroundjoin%
\definecolor{currentfill}{rgb}{0.121569,0.466667,0.705882}%
\pgfsetfillcolor{currentfill}%
\pgfsetfillopacity{0.854391}%
\pgfsetlinewidth{1.003750pt}%
\definecolor{currentstroke}{rgb}{0.121569,0.466667,0.705882}%
\pgfsetstrokecolor{currentstroke}%
\pgfsetstrokeopacity{0.854391}%
\pgfsetdash{}{0pt}%
\pgfpathmoveto{\pgfqpoint{2.415073in}{1.399721in}}%
\pgfpathcurveto{\pgfqpoint{2.423309in}{1.399721in}}{\pgfqpoint{2.431209in}{1.402993in}}{\pgfqpoint{2.437033in}{1.408817in}}%
\pgfpathcurveto{\pgfqpoint{2.442857in}{1.414641in}}{\pgfqpoint{2.446129in}{1.422541in}}{\pgfqpoint{2.446129in}{1.430777in}}%
\pgfpathcurveto{\pgfqpoint{2.446129in}{1.439014in}}{\pgfqpoint{2.442857in}{1.446914in}}{\pgfqpoint{2.437033in}{1.452738in}}%
\pgfpathcurveto{\pgfqpoint{2.431209in}{1.458562in}}{\pgfqpoint{2.423309in}{1.461834in}}{\pgfqpoint{2.415073in}{1.461834in}}%
\pgfpathcurveto{\pgfqpoint{2.406837in}{1.461834in}}{\pgfqpoint{2.398936in}{1.458562in}}{\pgfqpoint{2.393113in}{1.452738in}}%
\pgfpathcurveto{\pgfqpoint{2.387289in}{1.446914in}}{\pgfqpoint{2.384016in}{1.439014in}}{\pgfqpoint{2.384016in}{1.430777in}}%
\pgfpathcurveto{\pgfqpoint{2.384016in}{1.422541in}}{\pgfqpoint{2.387289in}{1.414641in}}{\pgfqpoint{2.393113in}{1.408817in}}%
\pgfpathcurveto{\pgfqpoint{2.398936in}{1.402993in}}{\pgfqpoint{2.406837in}{1.399721in}}{\pgfqpoint{2.415073in}{1.399721in}}%
\pgfpathclose%
\pgfusepath{stroke,fill}%
\end{pgfscope}%
\begin{pgfscope}%
\pgfpathrectangle{\pgfqpoint{0.100000in}{0.212622in}}{\pgfqpoint{3.696000in}{3.696000in}}%
\pgfusepath{clip}%
\pgfsetbuttcap%
\pgfsetroundjoin%
\definecolor{currentfill}{rgb}{0.121569,0.466667,0.705882}%
\pgfsetfillcolor{currentfill}%
\pgfsetfillopacity{0.854851}%
\pgfsetlinewidth{1.003750pt}%
\definecolor{currentstroke}{rgb}{0.121569,0.466667,0.705882}%
\pgfsetstrokecolor{currentstroke}%
\pgfsetstrokeopacity{0.854851}%
\pgfsetdash{}{0pt}%
\pgfpathmoveto{\pgfqpoint{1.676854in}{0.991655in}}%
\pgfpathcurveto{\pgfqpoint{1.685091in}{0.991655in}}{\pgfqpoint{1.692991in}{0.994928in}}{\pgfqpoint{1.698815in}{1.000751in}}%
\pgfpathcurveto{\pgfqpoint{1.704639in}{1.006575in}}{\pgfqpoint{1.707911in}{1.014475in}}{\pgfqpoint{1.707911in}{1.022712in}}%
\pgfpathcurveto{\pgfqpoint{1.707911in}{1.030948in}}{\pgfqpoint{1.704639in}{1.038848in}}{\pgfqpoint{1.698815in}{1.044672in}}%
\pgfpathcurveto{\pgfqpoint{1.692991in}{1.050496in}}{\pgfqpoint{1.685091in}{1.053768in}}{\pgfqpoint{1.676854in}{1.053768in}}%
\pgfpathcurveto{\pgfqpoint{1.668618in}{1.053768in}}{\pgfqpoint{1.660718in}{1.050496in}}{\pgfqpoint{1.654894in}{1.044672in}}%
\pgfpathcurveto{\pgfqpoint{1.649070in}{1.038848in}}{\pgfqpoint{1.645798in}{1.030948in}}{\pgfqpoint{1.645798in}{1.022712in}}%
\pgfpathcurveto{\pgfqpoint{1.645798in}{1.014475in}}{\pgfqpoint{1.649070in}{1.006575in}}{\pgfqpoint{1.654894in}{1.000751in}}%
\pgfpathcurveto{\pgfqpoint{1.660718in}{0.994928in}}{\pgfqpoint{1.668618in}{0.991655in}}{\pgfqpoint{1.676854in}{0.991655in}}%
\pgfpathclose%
\pgfusepath{stroke,fill}%
\end{pgfscope}%
\begin{pgfscope}%
\pgfpathrectangle{\pgfqpoint{0.100000in}{0.212622in}}{\pgfqpoint{3.696000in}{3.696000in}}%
\pgfusepath{clip}%
\pgfsetbuttcap%
\pgfsetroundjoin%
\definecolor{currentfill}{rgb}{0.121569,0.466667,0.705882}%
\pgfsetfillcolor{currentfill}%
\pgfsetfillopacity{0.859227}%
\pgfsetlinewidth{1.003750pt}%
\definecolor{currentstroke}{rgb}{0.121569,0.466667,0.705882}%
\pgfsetstrokecolor{currentstroke}%
\pgfsetstrokeopacity{0.859227}%
\pgfsetdash{}{0pt}%
\pgfpathmoveto{\pgfqpoint{1.697167in}{0.987989in}}%
\pgfpathcurveto{\pgfqpoint{1.705403in}{0.987989in}}{\pgfqpoint{1.713303in}{0.991261in}}{\pgfqpoint{1.719127in}{0.997085in}}%
\pgfpathcurveto{\pgfqpoint{1.724951in}{1.002909in}}{\pgfqpoint{1.728224in}{1.010809in}}{\pgfqpoint{1.728224in}{1.019045in}}%
\pgfpathcurveto{\pgfqpoint{1.728224in}{1.027281in}}{\pgfqpoint{1.724951in}{1.035182in}}{\pgfqpoint{1.719127in}{1.041005in}}%
\pgfpathcurveto{\pgfqpoint{1.713303in}{1.046829in}}{\pgfqpoint{1.705403in}{1.050102in}}{\pgfqpoint{1.697167in}{1.050102in}}%
\pgfpathcurveto{\pgfqpoint{1.688931in}{1.050102in}}{\pgfqpoint{1.681031in}{1.046829in}}{\pgfqpoint{1.675207in}{1.041005in}}%
\pgfpathcurveto{\pgfqpoint{1.669383in}{1.035182in}}{\pgfqpoint{1.666111in}{1.027281in}}{\pgfqpoint{1.666111in}{1.019045in}}%
\pgfpathcurveto{\pgfqpoint{1.666111in}{1.010809in}}{\pgfqpoint{1.669383in}{1.002909in}}{\pgfqpoint{1.675207in}{0.997085in}}%
\pgfpathcurveto{\pgfqpoint{1.681031in}{0.991261in}}{\pgfqpoint{1.688931in}{0.987989in}}{\pgfqpoint{1.697167in}{0.987989in}}%
\pgfpathclose%
\pgfusepath{stroke,fill}%
\end{pgfscope}%
\begin{pgfscope}%
\pgfpathrectangle{\pgfqpoint{0.100000in}{0.212622in}}{\pgfqpoint{3.696000in}{3.696000in}}%
\pgfusepath{clip}%
\pgfsetbuttcap%
\pgfsetroundjoin%
\definecolor{currentfill}{rgb}{0.121569,0.466667,0.705882}%
\pgfsetfillcolor{currentfill}%
\pgfsetfillopacity{0.862132}%
\pgfsetlinewidth{1.003750pt}%
\definecolor{currentstroke}{rgb}{0.121569,0.466667,0.705882}%
\pgfsetstrokecolor{currentstroke}%
\pgfsetstrokeopacity{0.862132}%
\pgfsetdash{}{0pt}%
\pgfpathmoveto{\pgfqpoint{1.710160in}{0.985005in}}%
\pgfpathcurveto{\pgfqpoint{1.718396in}{0.985005in}}{\pgfqpoint{1.726296in}{0.988277in}}{\pgfqpoint{1.732120in}{0.994101in}}%
\pgfpathcurveto{\pgfqpoint{1.737944in}{0.999925in}}{\pgfqpoint{1.741217in}{1.007825in}}{\pgfqpoint{1.741217in}{1.016061in}}%
\pgfpathcurveto{\pgfqpoint{1.741217in}{1.024298in}}{\pgfqpoint{1.737944in}{1.032198in}}{\pgfqpoint{1.732120in}{1.038022in}}%
\pgfpathcurveto{\pgfqpoint{1.726296in}{1.043845in}}{\pgfqpoint{1.718396in}{1.047118in}}{\pgfqpoint{1.710160in}{1.047118in}}%
\pgfpathcurveto{\pgfqpoint{1.701924in}{1.047118in}}{\pgfqpoint{1.694024in}{1.043845in}}{\pgfqpoint{1.688200in}{1.038022in}}%
\pgfpathcurveto{\pgfqpoint{1.682376in}{1.032198in}}{\pgfqpoint{1.679104in}{1.024298in}}{\pgfqpoint{1.679104in}{1.016061in}}%
\pgfpathcurveto{\pgfqpoint{1.679104in}{1.007825in}}{\pgfqpoint{1.682376in}{0.999925in}}{\pgfqpoint{1.688200in}{0.994101in}}%
\pgfpathcurveto{\pgfqpoint{1.694024in}{0.988277in}}{\pgfqpoint{1.701924in}{0.985005in}}{\pgfqpoint{1.710160in}{0.985005in}}%
\pgfpathclose%
\pgfusepath{stroke,fill}%
\end{pgfscope}%
\begin{pgfscope}%
\pgfpathrectangle{\pgfqpoint{0.100000in}{0.212622in}}{\pgfqpoint{3.696000in}{3.696000in}}%
\pgfusepath{clip}%
\pgfsetbuttcap%
\pgfsetroundjoin%
\definecolor{currentfill}{rgb}{0.121569,0.466667,0.705882}%
\pgfsetfillcolor{currentfill}%
\pgfsetfillopacity{0.863251}%
\pgfsetlinewidth{1.003750pt}%
\definecolor{currentstroke}{rgb}{0.121569,0.466667,0.705882}%
\pgfsetstrokecolor{currentstroke}%
\pgfsetstrokeopacity{0.863251}%
\pgfsetdash{}{0pt}%
\pgfpathmoveto{\pgfqpoint{1.715233in}{0.984098in}}%
\pgfpathcurveto{\pgfqpoint{1.723470in}{0.984098in}}{\pgfqpoint{1.731370in}{0.987370in}}{\pgfqpoint{1.737194in}{0.993194in}}%
\pgfpathcurveto{\pgfqpoint{1.743018in}{0.999018in}}{\pgfqpoint{1.746290in}{1.006918in}}{\pgfqpoint{1.746290in}{1.015154in}}%
\pgfpathcurveto{\pgfqpoint{1.746290in}{1.023390in}}{\pgfqpoint{1.743018in}{1.031291in}}{\pgfqpoint{1.737194in}{1.037114in}}%
\pgfpathcurveto{\pgfqpoint{1.731370in}{1.042938in}}{\pgfqpoint{1.723470in}{1.046211in}}{\pgfqpoint{1.715233in}{1.046211in}}%
\pgfpathcurveto{\pgfqpoint{1.706997in}{1.046211in}}{\pgfqpoint{1.699097in}{1.042938in}}{\pgfqpoint{1.693273in}{1.037114in}}%
\pgfpathcurveto{\pgfqpoint{1.687449in}{1.031291in}}{\pgfqpoint{1.684177in}{1.023390in}}{\pgfqpoint{1.684177in}{1.015154in}}%
\pgfpathcurveto{\pgfqpoint{1.684177in}{1.006918in}}{\pgfqpoint{1.687449in}{0.999018in}}{\pgfqpoint{1.693273in}{0.993194in}}%
\pgfpathcurveto{\pgfqpoint{1.699097in}{0.987370in}}{\pgfqpoint{1.706997in}{0.984098in}}{\pgfqpoint{1.715233in}{0.984098in}}%
\pgfpathclose%
\pgfusepath{stroke,fill}%
\end{pgfscope}%
\begin{pgfscope}%
\pgfpathrectangle{\pgfqpoint{0.100000in}{0.212622in}}{\pgfqpoint{3.696000in}{3.696000in}}%
\pgfusepath{clip}%
\pgfsetbuttcap%
\pgfsetroundjoin%
\definecolor{currentfill}{rgb}{0.121569,0.466667,0.705882}%
\pgfsetfillcolor{currentfill}%
\pgfsetfillopacity{0.863527}%
\pgfsetlinewidth{1.003750pt}%
\definecolor{currentstroke}{rgb}{0.121569,0.466667,0.705882}%
\pgfsetstrokecolor{currentstroke}%
\pgfsetstrokeopacity{0.863527}%
\pgfsetdash{}{0pt}%
\pgfpathmoveto{\pgfqpoint{2.423471in}{1.364815in}}%
\pgfpathcurveto{\pgfqpoint{2.431707in}{1.364815in}}{\pgfqpoint{2.439607in}{1.368087in}}{\pgfqpoint{2.445431in}{1.373911in}}%
\pgfpathcurveto{\pgfqpoint{2.451255in}{1.379735in}}{\pgfqpoint{2.454527in}{1.387635in}}{\pgfqpoint{2.454527in}{1.395871in}}%
\pgfpathcurveto{\pgfqpoint{2.454527in}{1.404108in}}{\pgfqpoint{2.451255in}{1.412008in}}{\pgfqpoint{2.445431in}{1.417832in}}%
\pgfpathcurveto{\pgfqpoint{2.439607in}{1.423656in}}{\pgfqpoint{2.431707in}{1.426928in}}{\pgfqpoint{2.423471in}{1.426928in}}%
\pgfpathcurveto{\pgfqpoint{2.415234in}{1.426928in}}{\pgfqpoint{2.407334in}{1.423656in}}{\pgfqpoint{2.401510in}{1.417832in}}%
\pgfpathcurveto{\pgfqpoint{2.395686in}{1.412008in}}{\pgfqpoint{2.392414in}{1.404108in}}{\pgfqpoint{2.392414in}{1.395871in}}%
\pgfpathcurveto{\pgfqpoint{2.392414in}{1.387635in}}{\pgfqpoint{2.395686in}{1.379735in}}{\pgfqpoint{2.401510in}{1.373911in}}%
\pgfpathcurveto{\pgfqpoint{2.407334in}{1.368087in}}{\pgfqpoint{2.415234in}{1.364815in}}{\pgfqpoint{2.423471in}{1.364815in}}%
\pgfpathclose%
\pgfusepath{stroke,fill}%
\end{pgfscope}%
\begin{pgfscope}%
\pgfpathrectangle{\pgfqpoint{0.100000in}{0.212622in}}{\pgfqpoint{3.696000in}{3.696000in}}%
\pgfusepath{clip}%
\pgfsetbuttcap%
\pgfsetroundjoin%
\definecolor{currentfill}{rgb}{0.121569,0.466667,0.705882}%
\pgfsetfillcolor{currentfill}%
\pgfsetfillopacity{0.865284}%
\pgfsetlinewidth{1.003750pt}%
\definecolor{currentstroke}{rgb}{0.121569,0.466667,0.705882}%
\pgfsetstrokecolor{currentstroke}%
\pgfsetstrokeopacity{0.865284}%
\pgfsetdash{}{0pt}%
\pgfpathmoveto{\pgfqpoint{1.724464in}{0.982457in}}%
\pgfpathcurveto{\pgfqpoint{1.732700in}{0.982457in}}{\pgfqpoint{1.740600in}{0.985729in}}{\pgfqpoint{1.746424in}{0.991553in}}%
\pgfpathcurveto{\pgfqpoint{1.752248in}{0.997377in}}{\pgfqpoint{1.755520in}{1.005277in}}{\pgfqpoint{1.755520in}{1.013513in}}%
\pgfpathcurveto{\pgfqpoint{1.755520in}{1.021749in}}{\pgfqpoint{1.752248in}{1.029649in}}{\pgfqpoint{1.746424in}{1.035473in}}%
\pgfpathcurveto{\pgfqpoint{1.740600in}{1.041297in}}{\pgfqpoint{1.732700in}{1.044570in}}{\pgfqpoint{1.724464in}{1.044570in}}%
\pgfpathcurveto{\pgfqpoint{1.716227in}{1.044570in}}{\pgfqpoint{1.708327in}{1.041297in}}{\pgfqpoint{1.702503in}{1.035473in}}%
\pgfpathcurveto{\pgfqpoint{1.696679in}{1.029649in}}{\pgfqpoint{1.693407in}{1.021749in}}{\pgfqpoint{1.693407in}{1.013513in}}%
\pgfpathcurveto{\pgfqpoint{1.693407in}{1.005277in}}{\pgfqpoint{1.696679in}{0.997377in}}{\pgfqpoint{1.702503in}{0.991553in}}%
\pgfpathcurveto{\pgfqpoint{1.708327in}{0.985729in}}{\pgfqpoint{1.716227in}{0.982457in}}{\pgfqpoint{1.724464in}{0.982457in}}%
\pgfpathclose%
\pgfusepath{stroke,fill}%
\end{pgfscope}%
\begin{pgfscope}%
\pgfpathrectangle{\pgfqpoint{0.100000in}{0.212622in}}{\pgfqpoint{3.696000in}{3.696000in}}%
\pgfusepath{clip}%
\pgfsetbuttcap%
\pgfsetroundjoin%
\definecolor{currentfill}{rgb}{0.121569,0.466667,0.705882}%
\pgfsetfillcolor{currentfill}%
\pgfsetfillopacity{0.869084}%
\pgfsetlinewidth{1.003750pt}%
\definecolor{currentstroke}{rgb}{0.121569,0.466667,0.705882}%
\pgfsetstrokecolor{currentstroke}%
\pgfsetstrokeopacity{0.869084}%
\pgfsetdash{}{0pt}%
\pgfpathmoveto{\pgfqpoint{1.740948in}{0.978474in}}%
\pgfpathcurveto{\pgfqpoint{1.749184in}{0.978474in}}{\pgfqpoint{1.757084in}{0.981746in}}{\pgfqpoint{1.762908in}{0.987570in}}%
\pgfpathcurveto{\pgfqpoint{1.768732in}{0.993394in}}{\pgfqpoint{1.772004in}{1.001294in}}{\pgfqpoint{1.772004in}{1.009530in}}%
\pgfpathcurveto{\pgfqpoint{1.772004in}{1.017767in}}{\pgfqpoint{1.768732in}{1.025667in}}{\pgfqpoint{1.762908in}{1.031491in}}%
\pgfpathcurveto{\pgfqpoint{1.757084in}{1.037315in}}{\pgfqpoint{1.749184in}{1.040587in}}{\pgfqpoint{1.740948in}{1.040587in}}%
\pgfpathcurveto{\pgfqpoint{1.732712in}{1.040587in}}{\pgfqpoint{1.724812in}{1.037315in}}{\pgfqpoint{1.718988in}{1.031491in}}%
\pgfpathcurveto{\pgfqpoint{1.713164in}{1.025667in}}{\pgfqpoint{1.709891in}{1.017767in}}{\pgfqpoint{1.709891in}{1.009530in}}%
\pgfpathcurveto{\pgfqpoint{1.709891in}{1.001294in}}{\pgfqpoint{1.713164in}{0.993394in}}{\pgfqpoint{1.718988in}{0.987570in}}%
\pgfpathcurveto{\pgfqpoint{1.724812in}{0.981746in}}{\pgfqpoint{1.732712in}{0.978474in}}{\pgfqpoint{1.740948in}{0.978474in}}%
\pgfpathclose%
\pgfusepath{stroke,fill}%
\end{pgfscope}%
\begin{pgfscope}%
\pgfpathrectangle{\pgfqpoint{0.100000in}{0.212622in}}{\pgfqpoint{3.696000in}{3.696000in}}%
\pgfusepath{clip}%
\pgfsetbuttcap%
\pgfsetroundjoin%
\definecolor{currentfill}{rgb}{0.121569,0.466667,0.705882}%
\pgfsetfillcolor{currentfill}%
\pgfsetfillopacity{0.872411}%
\pgfsetlinewidth{1.003750pt}%
\definecolor{currentstroke}{rgb}{0.121569,0.466667,0.705882}%
\pgfsetstrokecolor{currentstroke}%
\pgfsetstrokeopacity{0.872411}%
\pgfsetdash{}{0pt}%
\pgfpathmoveto{\pgfqpoint{1.755571in}{0.975114in}}%
\pgfpathcurveto{\pgfqpoint{1.763807in}{0.975114in}}{\pgfqpoint{1.771707in}{0.978386in}}{\pgfqpoint{1.777531in}{0.984210in}}%
\pgfpathcurveto{\pgfqpoint{1.783355in}{0.990034in}}{\pgfqpoint{1.786627in}{0.997934in}}{\pgfqpoint{1.786627in}{1.006170in}}%
\pgfpathcurveto{\pgfqpoint{1.786627in}{1.014406in}}{\pgfqpoint{1.783355in}{1.022306in}}{\pgfqpoint{1.777531in}{1.028130in}}%
\pgfpathcurveto{\pgfqpoint{1.771707in}{1.033954in}}{\pgfqpoint{1.763807in}{1.037227in}}{\pgfqpoint{1.755571in}{1.037227in}}%
\pgfpathcurveto{\pgfqpoint{1.747334in}{1.037227in}}{\pgfqpoint{1.739434in}{1.033954in}}{\pgfqpoint{1.733610in}{1.028130in}}%
\pgfpathcurveto{\pgfqpoint{1.727786in}{1.022306in}}{\pgfqpoint{1.724514in}{1.014406in}}{\pgfqpoint{1.724514in}{1.006170in}}%
\pgfpathcurveto{\pgfqpoint{1.724514in}{0.997934in}}{\pgfqpoint{1.727786in}{0.990034in}}{\pgfqpoint{1.733610in}{0.984210in}}%
\pgfpathcurveto{\pgfqpoint{1.739434in}{0.978386in}}{\pgfqpoint{1.747334in}{0.975114in}}{\pgfqpoint{1.755571in}{0.975114in}}%
\pgfpathclose%
\pgfusepath{stroke,fill}%
\end{pgfscope}%
\begin{pgfscope}%
\pgfpathrectangle{\pgfqpoint{0.100000in}{0.212622in}}{\pgfqpoint{3.696000in}{3.696000in}}%
\pgfusepath{clip}%
\pgfsetbuttcap%
\pgfsetroundjoin%
\definecolor{currentfill}{rgb}{0.121569,0.466667,0.705882}%
\pgfsetfillcolor{currentfill}%
\pgfsetfillopacity{0.873606}%
\pgfsetlinewidth{1.003750pt}%
\definecolor{currentstroke}{rgb}{0.121569,0.466667,0.705882}%
\pgfsetstrokecolor{currentstroke}%
\pgfsetstrokeopacity{0.873606}%
\pgfsetdash{}{0pt}%
\pgfpathmoveto{\pgfqpoint{2.434279in}{1.328103in}}%
\pgfpathcurveto{\pgfqpoint{2.442515in}{1.328103in}}{\pgfqpoint{2.450415in}{1.331375in}}{\pgfqpoint{2.456239in}{1.337199in}}%
\pgfpathcurveto{\pgfqpoint{2.462063in}{1.343023in}}{\pgfqpoint{2.465335in}{1.350923in}}{\pgfqpoint{2.465335in}{1.359159in}}%
\pgfpathcurveto{\pgfqpoint{2.465335in}{1.367395in}}{\pgfqpoint{2.462063in}{1.375296in}}{\pgfqpoint{2.456239in}{1.381119in}}%
\pgfpathcurveto{\pgfqpoint{2.450415in}{1.386943in}}{\pgfqpoint{2.442515in}{1.390216in}}{\pgfqpoint{2.434279in}{1.390216in}}%
\pgfpathcurveto{\pgfqpoint{2.426042in}{1.390216in}}{\pgfqpoint{2.418142in}{1.386943in}}{\pgfqpoint{2.412318in}{1.381119in}}%
\pgfpathcurveto{\pgfqpoint{2.406494in}{1.375296in}}{\pgfqpoint{2.403222in}{1.367395in}}{\pgfqpoint{2.403222in}{1.359159in}}%
\pgfpathcurveto{\pgfqpoint{2.403222in}{1.350923in}}{\pgfqpoint{2.406494in}{1.343023in}}{\pgfqpoint{2.412318in}{1.337199in}}%
\pgfpathcurveto{\pgfqpoint{2.418142in}{1.331375in}}{\pgfqpoint{2.426042in}{1.328103in}}{\pgfqpoint{2.434279in}{1.328103in}}%
\pgfpathclose%
\pgfusepath{stroke,fill}%
\end{pgfscope}%
\begin{pgfscope}%
\pgfpathrectangle{\pgfqpoint{0.100000in}{0.212622in}}{\pgfqpoint{3.696000in}{3.696000in}}%
\pgfusepath{clip}%
\pgfsetbuttcap%
\pgfsetroundjoin%
\definecolor{currentfill}{rgb}{0.121569,0.466667,0.705882}%
\pgfsetfillcolor{currentfill}%
\pgfsetfillopacity{0.874237}%
\pgfsetlinewidth{1.003750pt}%
\definecolor{currentstroke}{rgb}{0.121569,0.466667,0.705882}%
\pgfsetstrokecolor{currentstroke}%
\pgfsetstrokeopacity{0.874237}%
\pgfsetdash{}{0pt}%
\pgfpathmoveto{\pgfqpoint{1.763479in}{0.973021in}}%
\pgfpathcurveto{\pgfqpoint{1.771715in}{0.973021in}}{\pgfqpoint{1.779615in}{0.976293in}}{\pgfqpoint{1.785439in}{0.982117in}}%
\pgfpathcurveto{\pgfqpoint{1.791263in}{0.987941in}}{\pgfqpoint{1.794535in}{0.995841in}}{\pgfqpoint{1.794535in}{1.004078in}}%
\pgfpathcurveto{\pgfqpoint{1.794535in}{1.012314in}}{\pgfqpoint{1.791263in}{1.020214in}}{\pgfqpoint{1.785439in}{1.026038in}}%
\pgfpathcurveto{\pgfqpoint{1.779615in}{1.031862in}}{\pgfqpoint{1.771715in}{1.035134in}}{\pgfqpoint{1.763479in}{1.035134in}}%
\pgfpathcurveto{\pgfqpoint{1.755243in}{1.035134in}}{\pgfqpoint{1.747342in}{1.031862in}}{\pgfqpoint{1.741519in}{1.026038in}}%
\pgfpathcurveto{\pgfqpoint{1.735695in}{1.020214in}}{\pgfqpoint{1.732422in}{1.012314in}}{\pgfqpoint{1.732422in}{1.004078in}}%
\pgfpathcurveto{\pgfqpoint{1.732422in}{0.995841in}}{\pgfqpoint{1.735695in}{0.987941in}}{\pgfqpoint{1.741519in}{0.982117in}}%
\pgfpathcurveto{\pgfqpoint{1.747342in}{0.976293in}}{\pgfqpoint{1.755243in}{0.973021in}}{\pgfqpoint{1.763479in}{0.973021in}}%
\pgfpathclose%
\pgfusepath{stroke,fill}%
\end{pgfscope}%
\begin{pgfscope}%
\pgfpathrectangle{\pgfqpoint{0.100000in}{0.212622in}}{\pgfqpoint{3.696000in}{3.696000in}}%
\pgfusepath{clip}%
\pgfsetbuttcap%
\pgfsetroundjoin%
\definecolor{currentfill}{rgb}{0.121569,0.466667,0.705882}%
\pgfsetfillcolor{currentfill}%
\pgfsetfillopacity{0.874860}%
\pgfsetlinewidth{1.003750pt}%
\definecolor{currentstroke}{rgb}{0.121569,0.466667,0.705882}%
\pgfsetstrokecolor{currentstroke}%
\pgfsetstrokeopacity{0.874860}%
\pgfsetdash{}{0pt}%
\pgfpathmoveto{\pgfqpoint{1.766246in}{0.972351in}}%
\pgfpathcurveto{\pgfqpoint{1.774482in}{0.972351in}}{\pgfqpoint{1.782382in}{0.975624in}}{\pgfqpoint{1.788206in}{0.981448in}}%
\pgfpathcurveto{\pgfqpoint{1.794030in}{0.987272in}}{\pgfqpoint{1.797302in}{0.995172in}}{\pgfqpoint{1.797302in}{1.003408in}}%
\pgfpathcurveto{\pgfqpoint{1.797302in}{1.011644in}}{\pgfqpoint{1.794030in}{1.019544in}}{\pgfqpoint{1.788206in}{1.025368in}}%
\pgfpathcurveto{\pgfqpoint{1.782382in}{1.031192in}}{\pgfqpoint{1.774482in}{1.034464in}}{\pgfqpoint{1.766246in}{1.034464in}}%
\pgfpathcurveto{\pgfqpoint{1.758009in}{1.034464in}}{\pgfqpoint{1.750109in}{1.031192in}}{\pgfqpoint{1.744285in}{1.025368in}}%
\pgfpathcurveto{\pgfqpoint{1.738461in}{1.019544in}}{\pgfqpoint{1.735189in}{1.011644in}}{\pgfqpoint{1.735189in}{1.003408in}}%
\pgfpathcurveto{\pgfqpoint{1.735189in}{0.995172in}}{\pgfqpoint{1.738461in}{0.987272in}}{\pgfqpoint{1.744285in}{0.981448in}}%
\pgfpathcurveto{\pgfqpoint{1.750109in}{0.975624in}}{\pgfqpoint{1.758009in}{0.972351in}}{\pgfqpoint{1.766246in}{0.972351in}}%
\pgfpathclose%
\pgfusepath{stroke,fill}%
\end{pgfscope}%
\begin{pgfscope}%
\pgfpathrectangle{\pgfqpoint{0.100000in}{0.212622in}}{\pgfqpoint{3.696000in}{3.696000in}}%
\pgfusepath{clip}%
\pgfsetbuttcap%
\pgfsetroundjoin%
\definecolor{currentfill}{rgb}{0.121569,0.466667,0.705882}%
\pgfsetfillcolor{currentfill}%
\pgfsetfillopacity{0.875964}%
\pgfsetlinewidth{1.003750pt}%
\definecolor{currentstroke}{rgb}{0.121569,0.466667,0.705882}%
\pgfsetstrokecolor{currentstroke}%
\pgfsetstrokeopacity{0.875964}%
\pgfsetdash{}{0pt}%
\pgfpathmoveto{\pgfqpoint{1.771328in}{0.971250in}}%
\pgfpathcurveto{\pgfqpoint{1.779565in}{0.971250in}}{\pgfqpoint{1.787465in}{0.974522in}}{\pgfqpoint{1.793289in}{0.980346in}}%
\pgfpathcurveto{\pgfqpoint{1.799113in}{0.986170in}}{\pgfqpoint{1.802385in}{0.994070in}}{\pgfqpoint{1.802385in}{1.002306in}}%
\pgfpathcurveto{\pgfqpoint{1.802385in}{1.010542in}}{\pgfqpoint{1.799113in}{1.018442in}}{\pgfqpoint{1.793289in}{1.024266in}}%
\pgfpathcurveto{\pgfqpoint{1.787465in}{1.030090in}}{\pgfqpoint{1.779565in}{1.033363in}}{\pgfqpoint{1.771328in}{1.033363in}}%
\pgfpathcurveto{\pgfqpoint{1.763092in}{1.033363in}}{\pgfqpoint{1.755192in}{1.030090in}}{\pgfqpoint{1.749368in}{1.024266in}}%
\pgfpathcurveto{\pgfqpoint{1.743544in}{1.018442in}}{\pgfqpoint{1.740272in}{1.010542in}}{\pgfqpoint{1.740272in}{1.002306in}}%
\pgfpathcurveto{\pgfqpoint{1.740272in}{0.994070in}}{\pgfqpoint{1.743544in}{0.986170in}}{\pgfqpoint{1.749368in}{0.980346in}}%
\pgfpathcurveto{\pgfqpoint{1.755192in}{0.974522in}}{\pgfqpoint{1.763092in}{0.971250in}}{\pgfqpoint{1.771328in}{0.971250in}}%
\pgfpathclose%
\pgfusepath{stroke,fill}%
\end{pgfscope}%
\begin{pgfscope}%
\pgfpathrectangle{\pgfqpoint{0.100000in}{0.212622in}}{\pgfqpoint{3.696000in}{3.696000in}}%
\pgfusepath{clip}%
\pgfsetbuttcap%
\pgfsetroundjoin%
\definecolor{currentfill}{rgb}{0.121569,0.466667,0.705882}%
\pgfsetfillcolor{currentfill}%
\pgfsetfillopacity{0.877957}%
\pgfsetlinewidth{1.003750pt}%
\definecolor{currentstroke}{rgb}{0.121569,0.466667,0.705882}%
\pgfsetstrokecolor{currentstroke}%
\pgfsetstrokeopacity{0.877957}%
\pgfsetdash{}{0pt}%
\pgfpathmoveto{\pgfqpoint{1.780638in}{0.969509in}}%
\pgfpathcurveto{\pgfqpoint{1.788874in}{0.969509in}}{\pgfqpoint{1.796774in}{0.972782in}}{\pgfqpoint{1.802598in}{0.978605in}}%
\pgfpathcurveto{\pgfqpoint{1.808422in}{0.984429in}}{\pgfqpoint{1.811695in}{0.992329in}}{\pgfqpoint{1.811695in}{1.000566in}}%
\pgfpathcurveto{\pgfqpoint{1.811695in}{1.008802in}}{\pgfqpoint{1.808422in}{1.016702in}}{\pgfqpoint{1.802598in}{1.022526in}}%
\pgfpathcurveto{\pgfqpoint{1.796774in}{1.028350in}}{\pgfqpoint{1.788874in}{1.031622in}}{\pgfqpoint{1.780638in}{1.031622in}}%
\pgfpathcurveto{\pgfqpoint{1.772402in}{1.031622in}}{\pgfqpoint{1.764502in}{1.028350in}}{\pgfqpoint{1.758678in}{1.022526in}}%
\pgfpathcurveto{\pgfqpoint{1.752854in}{1.016702in}}{\pgfqpoint{1.749582in}{1.008802in}}{\pgfqpoint{1.749582in}{1.000566in}}%
\pgfpathcurveto{\pgfqpoint{1.749582in}{0.992329in}}{\pgfqpoint{1.752854in}{0.984429in}}{\pgfqpoint{1.758678in}{0.978605in}}%
\pgfpathcurveto{\pgfqpoint{1.764502in}{0.972782in}}{\pgfqpoint{1.772402in}{0.969509in}}{\pgfqpoint{1.780638in}{0.969509in}}%
\pgfpathclose%
\pgfusepath{stroke,fill}%
\end{pgfscope}%
\begin{pgfscope}%
\pgfpathrectangle{\pgfqpoint{0.100000in}{0.212622in}}{\pgfqpoint{3.696000in}{3.696000in}}%
\pgfusepath{clip}%
\pgfsetbuttcap%
\pgfsetroundjoin%
\definecolor{currentfill}{rgb}{0.121569,0.466667,0.705882}%
\pgfsetfillcolor{currentfill}%
\pgfsetfillopacity{0.881449}%
\pgfsetlinewidth{1.003750pt}%
\definecolor{currentstroke}{rgb}{0.121569,0.466667,0.705882}%
\pgfsetstrokecolor{currentstroke}%
\pgfsetstrokeopacity{0.881449}%
\pgfsetdash{}{0pt}%
\pgfpathmoveto{\pgfqpoint{1.797805in}{0.967141in}}%
\pgfpathcurveto{\pgfqpoint{1.806042in}{0.967141in}}{\pgfqpoint{1.813942in}{0.970413in}}{\pgfqpoint{1.819766in}{0.976237in}}%
\pgfpathcurveto{\pgfqpoint{1.825590in}{0.982061in}}{\pgfqpoint{1.828862in}{0.989961in}}{\pgfqpoint{1.828862in}{0.998197in}}%
\pgfpathcurveto{\pgfqpoint{1.828862in}{1.006434in}}{\pgfqpoint{1.825590in}{1.014334in}}{\pgfqpoint{1.819766in}{1.020158in}}%
\pgfpathcurveto{\pgfqpoint{1.813942in}{1.025982in}}{\pgfqpoint{1.806042in}{1.029254in}}{\pgfqpoint{1.797805in}{1.029254in}}%
\pgfpathcurveto{\pgfqpoint{1.789569in}{1.029254in}}{\pgfqpoint{1.781669in}{1.025982in}}{\pgfqpoint{1.775845in}{1.020158in}}%
\pgfpathcurveto{\pgfqpoint{1.770021in}{1.014334in}}{\pgfqpoint{1.766749in}{1.006434in}}{\pgfqpoint{1.766749in}{0.998197in}}%
\pgfpathcurveto{\pgfqpoint{1.766749in}{0.989961in}}{\pgfqpoint{1.770021in}{0.982061in}}{\pgfqpoint{1.775845in}{0.976237in}}%
\pgfpathcurveto{\pgfqpoint{1.781669in}{0.970413in}}{\pgfqpoint{1.789569in}{0.967141in}}{\pgfqpoint{1.797805in}{0.967141in}}%
\pgfpathclose%
\pgfusepath{stroke,fill}%
\end{pgfscope}%
\begin{pgfscope}%
\pgfpathrectangle{\pgfqpoint{0.100000in}{0.212622in}}{\pgfqpoint{3.696000in}{3.696000in}}%
\pgfusepath{clip}%
\pgfsetbuttcap%
\pgfsetroundjoin%
\definecolor{currentfill}{rgb}{0.121569,0.466667,0.705882}%
\pgfsetfillcolor{currentfill}%
\pgfsetfillopacity{0.883943}%
\pgfsetlinewidth{1.003750pt}%
\definecolor{currentstroke}{rgb}{0.121569,0.466667,0.705882}%
\pgfsetstrokecolor{currentstroke}%
\pgfsetstrokeopacity{0.883943}%
\pgfsetdash{}{0pt}%
\pgfpathmoveto{\pgfqpoint{1.809490in}{0.964956in}}%
\pgfpathcurveto{\pgfqpoint{1.817726in}{0.964956in}}{\pgfqpoint{1.825626in}{0.968228in}}{\pgfqpoint{1.831450in}{0.974052in}}%
\pgfpathcurveto{\pgfqpoint{1.837274in}{0.979876in}}{\pgfqpoint{1.840546in}{0.987776in}}{\pgfqpoint{1.840546in}{0.996012in}}%
\pgfpathcurveto{\pgfqpoint{1.840546in}{1.004249in}}{\pgfqpoint{1.837274in}{1.012149in}}{\pgfqpoint{1.831450in}{1.017973in}}%
\pgfpathcurveto{\pgfqpoint{1.825626in}{1.023796in}}{\pgfqpoint{1.817726in}{1.027069in}}{\pgfqpoint{1.809490in}{1.027069in}}%
\pgfpathcurveto{\pgfqpoint{1.801254in}{1.027069in}}{\pgfqpoint{1.793353in}{1.023796in}}{\pgfqpoint{1.787530in}{1.017973in}}%
\pgfpathcurveto{\pgfqpoint{1.781706in}{1.012149in}}{\pgfqpoint{1.778433in}{1.004249in}}{\pgfqpoint{1.778433in}{0.996012in}}%
\pgfpathcurveto{\pgfqpoint{1.778433in}{0.987776in}}{\pgfqpoint{1.781706in}{0.979876in}}{\pgfqpoint{1.787530in}{0.974052in}}%
\pgfpathcurveto{\pgfqpoint{1.793353in}{0.968228in}}{\pgfqpoint{1.801254in}{0.964956in}}{\pgfqpoint{1.809490in}{0.964956in}}%
\pgfpathclose%
\pgfusepath{stroke,fill}%
\end{pgfscope}%
\begin{pgfscope}%
\pgfpathrectangle{\pgfqpoint{0.100000in}{0.212622in}}{\pgfqpoint{3.696000in}{3.696000in}}%
\pgfusepath{clip}%
\pgfsetbuttcap%
\pgfsetroundjoin%
\definecolor{currentfill}{rgb}{0.121569,0.466667,0.705882}%
\pgfsetfillcolor{currentfill}%
\pgfsetfillopacity{0.885395}%
\pgfsetlinewidth{1.003750pt}%
\definecolor{currentstroke}{rgb}{0.121569,0.466667,0.705882}%
\pgfsetstrokecolor{currentstroke}%
\pgfsetstrokeopacity{0.885395}%
\pgfsetdash{}{0pt}%
\pgfpathmoveto{\pgfqpoint{1.816263in}{0.963543in}}%
\pgfpathcurveto{\pgfqpoint{1.824499in}{0.963543in}}{\pgfqpoint{1.832399in}{0.966815in}}{\pgfqpoint{1.838223in}{0.972639in}}%
\pgfpathcurveto{\pgfqpoint{1.844047in}{0.978463in}}{\pgfqpoint{1.847320in}{0.986363in}}{\pgfqpoint{1.847320in}{0.994600in}}%
\pgfpathcurveto{\pgfqpoint{1.847320in}{1.002836in}}{\pgfqpoint{1.844047in}{1.010736in}}{\pgfqpoint{1.838223in}{1.016560in}}%
\pgfpathcurveto{\pgfqpoint{1.832399in}{1.022384in}}{\pgfqpoint{1.824499in}{1.025656in}}{\pgfqpoint{1.816263in}{1.025656in}}%
\pgfpathcurveto{\pgfqpoint{1.808027in}{1.025656in}}{\pgfqpoint{1.800127in}{1.022384in}}{\pgfqpoint{1.794303in}{1.016560in}}%
\pgfpathcurveto{\pgfqpoint{1.788479in}{1.010736in}}{\pgfqpoint{1.785207in}{1.002836in}}{\pgfqpoint{1.785207in}{0.994600in}}%
\pgfpathcurveto{\pgfqpoint{1.785207in}{0.986363in}}{\pgfqpoint{1.788479in}{0.978463in}}{\pgfqpoint{1.794303in}{0.972639in}}%
\pgfpathcurveto{\pgfqpoint{1.800127in}{0.966815in}}{\pgfqpoint{1.808027in}{0.963543in}}{\pgfqpoint{1.816263in}{0.963543in}}%
\pgfpathclose%
\pgfusepath{stroke,fill}%
\end{pgfscope}%
\begin{pgfscope}%
\pgfpathrectangle{\pgfqpoint{0.100000in}{0.212622in}}{\pgfqpoint{3.696000in}{3.696000in}}%
\pgfusepath{clip}%
\pgfsetbuttcap%
\pgfsetroundjoin%
\definecolor{currentfill}{rgb}{0.121569,0.466667,0.705882}%
\pgfsetfillcolor{currentfill}%
\pgfsetfillopacity{0.885509}%
\pgfsetlinewidth{1.003750pt}%
\definecolor{currentstroke}{rgb}{0.121569,0.466667,0.705882}%
\pgfsetstrokecolor{currentstroke}%
\pgfsetstrokeopacity{0.885509}%
\pgfsetdash{}{0pt}%
\pgfpathmoveto{\pgfqpoint{2.447144in}{1.287155in}}%
\pgfpathcurveto{\pgfqpoint{2.455381in}{1.287155in}}{\pgfqpoint{2.463281in}{1.290427in}}{\pgfqpoint{2.469105in}{1.296251in}}%
\pgfpathcurveto{\pgfqpoint{2.474928in}{1.302075in}}{\pgfqpoint{2.478201in}{1.309975in}}{\pgfqpoint{2.478201in}{1.318212in}}%
\pgfpathcurveto{\pgfqpoint{2.478201in}{1.326448in}}{\pgfqpoint{2.474928in}{1.334348in}}{\pgfqpoint{2.469105in}{1.340172in}}%
\pgfpathcurveto{\pgfqpoint{2.463281in}{1.345996in}}{\pgfqpoint{2.455381in}{1.349268in}}{\pgfqpoint{2.447144in}{1.349268in}}%
\pgfpathcurveto{\pgfqpoint{2.438908in}{1.349268in}}{\pgfqpoint{2.431008in}{1.345996in}}{\pgfqpoint{2.425184in}{1.340172in}}%
\pgfpathcurveto{\pgfqpoint{2.419360in}{1.334348in}}{\pgfqpoint{2.416088in}{1.326448in}}{\pgfqpoint{2.416088in}{1.318212in}}%
\pgfpathcurveto{\pgfqpoint{2.416088in}{1.309975in}}{\pgfqpoint{2.419360in}{1.302075in}}{\pgfqpoint{2.425184in}{1.296251in}}%
\pgfpathcurveto{\pgfqpoint{2.431008in}{1.290427in}}{\pgfqpoint{2.438908in}{1.287155in}}{\pgfqpoint{2.447144in}{1.287155in}}%
\pgfpathclose%
\pgfusepath{stroke,fill}%
\end{pgfscope}%
\begin{pgfscope}%
\pgfpathrectangle{\pgfqpoint{0.100000in}{0.212622in}}{\pgfqpoint{3.696000in}{3.696000in}}%
\pgfusepath{clip}%
\pgfsetbuttcap%
\pgfsetroundjoin%
\definecolor{currentfill}{rgb}{0.121569,0.466667,0.705882}%
\pgfsetfillcolor{currentfill}%
\pgfsetfillopacity{0.888038}%
\pgfsetlinewidth{1.003750pt}%
\definecolor{currentstroke}{rgb}{0.121569,0.466667,0.705882}%
\pgfsetstrokecolor{currentstroke}%
\pgfsetstrokeopacity{0.888038}%
\pgfsetdash{}{0pt}%
\pgfpathmoveto{\pgfqpoint{1.828583in}{0.960971in}}%
\pgfpathcurveto{\pgfqpoint{1.836819in}{0.960971in}}{\pgfqpoint{1.844719in}{0.964244in}}{\pgfqpoint{1.850543in}{0.970068in}}%
\pgfpathcurveto{\pgfqpoint{1.856367in}{0.975891in}}{\pgfqpoint{1.859639in}{0.983792in}}{\pgfqpoint{1.859639in}{0.992028in}}%
\pgfpathcurveto{\pgfqpoint{1.859639in}{1.000264in}}{\pgfqpoint{1.856367in}{1.008164in}}{\pgfqpoint{1.850543in}{1.013988in}}%
\pgfpathcurveto{\pgfqpoint{1.844719in}{1.019812in}}{\pgfqpoint{1.836819in}{1.023084in}}{\pgfqpoint{1.828583in}{1.023084in}}%
\pgfpathcurveto{\pgfqpoint{1.820346in}{1.023084in}}{\pgfqpoint{1.812446in}{1.019812in}}{\pgfqpoint{1.806622in}{1.013988in}}%
\pgfpathcurveto{\pgfqpoint{1.800798in}{1.008164in}}{\pgfqpoint{1.797526in}{1.000264in}}{\pgfqpoint{1.797526in}{0.992028in}}%
\pgfpathcurveto{\pgfqpoint{1.797526in}{0.983792in}}{\pgfqpoint{1.800798in}{0.975891in}}{\pgfqpoint{1.806622in}{0.970068in}}%
\pgfpathcurveto{\pgfqpoint{1.812446in}{0.964244in}}{\pgfqpoint{1.820346in}{0.960971in}}{\pgfqpoint{1.828583in}{0.960971in}}%
\pgfpathclose%
\pgfusepath{stroke,fill}%
\end{pgfscope}%
\begin{pgfscope}%
\pgfpathrectangle{\pgfqpoint{0.100000in}{0.212622in}}{\pgfqpoint{3.696000in}{3.696000in}}%
\pgfusepath{clip}%
\pgfsetbuttcap%
\pgfsetroundjoin%
\definecolor{currentfill}{rgb}{0.121569,0.466667,0.705882}%
\pgfsetfillcolor{currentfill}%
\pgfsetfillopacity{0.892959}%
\pgfsetlinewidth{1.003750pt}%
\definecolor{currentstroke}{rgb}{0.121569,0.466667,0.705882}%
\pgfsetstrokecolor{currentstroke}%
\pgfsetstrokeopacity{0.892959}%
\pgfsetdash{}{0pt}%
\pgfpathmoveto{\pgfqpoint{1.850827in}{0.955900in}}%
\pgfpathcurveto{\pgfqpoint{1.859063in}{0.955900in}}{\pgfqpoint{1.866963in}{0.959172in}}{\pgfqpoint{1.872787in}{0.964996in}}%
\pgfpathcurveto{\pgfqpoint{1.878611in}{0.970820in}}{\pgfqpoint{1.881883in}{0.978720in}}{\pgfqpoint{1.881883in}{0.986956in}}%
\pgfpathcurveto{\pgfqpoint{1.881883in}{0.995193in}}{\pgfqpoint{1.878611in}{1.003093in}}{\pgfqpoint{1.872787in}{1.008917in}}%
\pgfpathcurveto{\pgfqpoint{1.866963in}{1.014741in}}{\pgfqpoint{1.859063in}{1.018013in}}{\pgfqpoint{1.850827in}{1.018013in}}%
\pgfpathcurveto{\pgfqpoint{1.842590in}{1.018013in}}{\pgfqpoint{1.834690in}{1.014741in}}{\pgfqpoint{1.828866in}{1.008917in}}%
\pgfpathcurveto{\pgfqpoint{1.823042in}{1.003093in}}{\pgfqpoint{1.819770in}{0.995193in}}{\pgfqpoint{1.819770in}{0.986956in}}%
\pgfpathcurveto{\pgfqpoint{1.819770in}{0.978720in}}{\pgfqpoint{1.823042in}{0.970820in}}{\pgfqpoint{1.828866in}{0.964996in}}%
\pgfpathcurveto{\pgfqpoint{1.834690in}{0.959172in}}{\pgfqpoint{1.842590in}{0.955900in}}{\pgfqpoint{1.850827in}{0.955900in}}%
\pgfpathclose%
\pgfusepath{stroke,fill}%
\end{pgfscope}%
\begin{pgfscope}%
\pgfpathrectangle{\pgfqpoint{0.100000in}{0.212622in}}{\pgfqpoint{3.696000in}{3.696000in}}%
\pgfusepath{clip}%
\pgfsetbuttcap%
\pgfsetroundjoin%
\definecolor{currentfill}{rgb}{0.121569,0.466667,0.705882}%
\pgfsetfillcolor{currentfill}%
\pgfsetfillopacity{0.897355}%
\pgfsetlinewidth{1.003750pt}%
\definecolor{currentstroke}{rgb}{0.121569,0.466667,0.705882}%
\pgfsetstrokecolor{currentstroke}%
\pgfsetstrokeopacity{0.897355}%
\pgfsetdash{}{0pt}%
\pgfpathmoveto{\pgfqpoint{1.870696in}{0.951191in}}%
\pgfpathcurveto{\pgfqpoint{1.878932in}{0.951191in}}{\pgfqpoint{1.886832in}{0.954463in}}{\pgfqpoint{1.892656in}{0.960287in}}%
\pgfpathcurveto{\pgfqpoint{1.898480in}{0.966111in}}{\pgfqpoint{1.901752in}{0.974011in}}{\pgfqpoint{1.901752in}{0.982247in}}%
\pgfpathcurveto{\pgfqpoint{1.901752in}{0.990483in}}{\pgfqpoint{1.898480in}{0.998384in}}{\pgfqpoint{1.892656in}{1.004207in}}%
\pgfpathcurveto{\pgfqpoint{1.886832in}{1.010031in}}{\pgfqpoint{1.878932in}{1.013304in}}{\pgfqpoint{1.870696in}{1.013304in}}%
\pgfpathcurveto{\pgfqpoint{1.862459in}{1.013304in}}{\pgfqpoint{1.854559in}{1.010031in}}{\pgfqpoint{1.848735in}{1.004207in}}%
\pgfpathcurveto{\pgfqpoint{1.842911in}{0.998384in}}{\pgfqpoint{1.839639in}{0.990483in}}{\pgfqpoint{1.839639in}{0.982247in}}%
\pgfpathcurveto{\pgfqpoint{1.839639in}{0.974011in}}{\pgfqpoint{1.842911in}{0.966111in}}{\pgfqpoint{1.848735in}{0.960287in}}%
\pgfpathcurveto{\pgfqpoint{1.854559in}{0.954463in}}{\pgfqpoint{1.862459in}{0.951191in}}{\pgfqpoint{1.870696in}{0.951191in}}%
\pgfpathclose%
\pgfusepath{stroke,fill}%
\end{pgfscope}%
\begin{pgfscope}%
\pgfpathrectangle{\pgfqpoint{0.100000in}{0.212622in}}{\pgfqpoint{3.696000in}{3.696000in}}%
\pgfusepath{clip}%
\pgfsetbuttcap%
\pgfsetroundjoin%
\definecolor{currentfill}{rgb}{0.121569,0.466667,0.705882}%
\pgfsetfillcolor{currentfill}%
\pgfsetfillopacity{0.898118}%
\pgfsetlinewidth{1.003750pt}%
\definecolor{currentstroke}{rgb}{0.121569,0.466667,0.705882}%
\pgfsetstrokecolor{currentstroke}%
\pgfsetstrokeopacity{0.898118}%
\pgfsetdash{}{0pt}%
\pgfpathmoveto{\pgfqpoint{2.459923in}{1.242654in}}%
\pgfpathcurveto{\pgfqpoint{2.468159in}{1.242654in}}{\pgfqpoint{2.476059in}{1.245927in}}{\pgfqpoint{2.481883in}{1.251751in}}%
\pgfpathcurveto{\pgfqpoint{2.487707in}{1.257575in}}{\pgfqpoint{2.490979in}{1.265475in}}{\pgfqpoint{2.490979in}{1.273711in}}%
\pgfpathcurveto{\pgfqpoint{2.490979in}{1.281947in}}{\pgfqpoint{2.487707in}{1.289847in}}{\pgfqpoint{2.481883in}{1.295671in}}%
\pgfpathcurveto{\pgfqpoint{2.476059in}{1.301495in}}{\pgfqpoint{2.468159in}{1.304767in}}{\pgfqpoint{2.459923in}{1.304767in}}%
\pgfpathcurveto{\pgfqpoint{2.451686in}{1.304767in}}{\pgfqpoint{2.443786in}{1.301495in}}{\pgfqpoint{2.437962in}{1.295671in}}%
\pgfpathcurveto{\pgfqpoint{2.432139in}{1.289847in}}{\pgfqpoint{2.428866in}{1.281947in}}{\pgfqpoint{2.428866in}{1.273711in}}%
\pgfpathcurveto{\pgfqpoint{2.428866in}{1.265475in}}{\pgfqpoint{2.432139in}{1.257575in}}{\pgfqpoint{2.437962in}{1.251751in}}%
\pgfpathcurveto{\pgfqpoint{2.443786in}{1.245927in}}{\pgfqpoint{2.451686in}{1.242654in}}{\pgfqpoint{2.459923in}{1.242654in}}%
\pgfpathclose%
\pgfusepath{stroke,fill}%
\end{pgfscope}%
\begin{pgfscope}%
\pgfpathrectangle{\pgfqpoint{0.100000in}{0.212622in}}{\pgfqpoint{3.696000in}{3.696000in}}%
\pgfusepath{clip}%
\pgfsetbuttcap%
\pgfsetroundjoin%
\definecolor{currentfill}{rgb}{0.121569,0.466667,0.705882}%
\pgfsetfillcolor{currentfill}%
\pgfsetfillopacity{0.900597}%
\pgfsetlinewidth{1.003750pt}%
\definecolor{currentstroke}{rgb}{0.121569,0.466667,0.705882}%
\pgfsetstrokecolor{currentstroke}%
\pgfsetstrokeopacity{0.900597}%
\pgfsetdash{}{0pt}%
\pgfpathmoveto{\pgfqpoint{1.885241in}{0.947571in}}%
\pgfpathcurveto{\pgfqpoint{1.893477in}{0.947571in}}{\pgfqpoint{1.901377in}{0.950843in}}{\pgfqpoint{1.907201in}{0.956667in}}%
\pgfpathcurveto{\pgfqpoint{1.913025in}{0.962491in}}{\pgfqpoint{1.916297in}{0.970391in}}{\pgfqpoint{1.916297in}{0.978627in}}%
\pgfpathcurveto{\pgfqpoint{1.916297in}{0.986864in}}{\pgfqpoint{1.913025in}{0.994764in}}{\pgfqpoint{1.907201in}{1.000588in}}%
\pgfpathcurveto{\pgfqpoint{1.901377in}{1.006412in}}{\pgfqpoint{1.893477in}{1.009684in}}{\pgfqpoint{1.885241in}{1.009684in}}%
\pgfpathcurveto{\pgfqpoint{1.877004in}{1.009684in}}{\pgfqpoint{1.869104in}{1.006412in}}{\pgfqpoint{1.863280in}{1.000588in}}%
\pgfpathcurveto{\pgfqpoint{1.857456in}{0.994764in}}{\pgfqpoint{1.854184in}{0.986864in}}{\pgfqpoint{1.854184in}{0.978627in}}%
\pgfpathcurveto{\pgfqpoint{1.854184in}{0.970391in}}{\pgfqpoint{1.857456in}{0.962491in}}{\pgfqpoint{1.863280in}{0.956667in}}%
\pgfpathcurveto{\pgfqpoint{1.869104in}{0.950843in}}{\pgfqpoint{1.877004in}{0.947571in}}{\pgfqpoint{1.885241in}{0.947571in}}%
\pgfpathclose%
\pgfusepath{stroke,fill}%
\end{pgfscope}%
\begin{pgfscope}%
\pgfpathrectangle{\pgfqpoint{0.100000in}{0.212622in}}{\pgfqpoint{3.696000in}{3.696000in}}%
\pgfusepath{clip}%
\pgfsetbuttcap%
\pgfsetroundjoin%
\definecolor{currentfill}{rgb}{0.121569,0.466667,0.705882}%
\pgfsetfillcolor{currentfill}%
\pgfsetfillopacity{0.902685}%
\pgfsetlinewidth{1.003750pt}%
\definecolor{currentstroke}{rgb}{0.121569,0.466667,0.705882}%
\pgfsetstrokecolor{currentstroke}%
\pgfsetstrokeopacity{0.902685}%
\pgfsetdash{}{0pt}%
\pgfpathmoveto{\pgfqpoint{1.894738in}{0.945517in}}%
\pgfpathcurveto{\pgfqpoint{1.902974in}{0.945517in}}{\pgfqpoint{1.910874in}{0.948789in}}{\pgfqpoint{1.916698in}{0.954613in}}%
\pgfpathcurveto{\pgfqpoint{1.922522in}{0.960437in}}{\pgfqpoint{1.925795in}{0.968337in}}{\pgfqpoint{1.925795in}{0.976573in}}%
\pgfpathcurveto{\pgfqpoint{1.925795in}{0.984810in}}{\pgfqpoint{1.922522in}{0.992710in}}{\pgfqpoint{1.916698in}{0.998533in}}%
\pgfpathcurveto{\pgfqpoint{1.910874in}{1.004357in}}{\pgfqpoint{1.902974in}{1.007630in}}{\pgfqpoint{1.894738in}{1.007630in}}%
\pgfpathcurveto{\pgfqpoint{1.886502in}{1.007630in}}{\pgfqpoint{1.878602in}{1.004357in}}{\pgfqpoint{1.872778in}{0.998533in}}%
\pgfpathcurveto{\pgfqpoint{1.866954in}{0.992710in}}{\pgfqpoint{1.863682in}{0.984810in}}{\pgfqpoint{1.863682in}{0.976573in}}%
\pgfpathcurveto{\pgfqpoint{1.863682in}{0.968337in}}{\pgfqpoint{1.866954in}{0.960437in}}{\pgfqpoint{1.872778in}{0.954613in}}%
\pgfpathcurveto{\pgfqpoint{1.878602in}{0.948789in}}{\pgfqpoint{1.886502in}{0.945517in}}{\pgfqpoint{1.894738in}{0.945517in}}%
\pgfpathclose%
\pgfusepath{stroke,fill}%
\end{pgfscope}%
\begin{pgfscope}%
\pgfpathrectangle{\pgfqpoint{0.100000in}{0.212622in}}{\pgfqpoint{3.696000in}{3.696000in}}%
\pgfusepath{clip}%
\pgfsetbuttcap%
\pgfsetroundjoin%
\definecolor{currentfill}{rgb}{0.121569,0.466667,0.705882}%
\pgfsetfillcolor{currentfill}%
\pgfsetfillopacity{0.903941}%
\pgfsetlinewidth{1.003750pt}%
\definecolor{currentstroke}{rgb}{0.121569,0.466667,0.705882}%
\pgfsetstrokecolor{currentstroke}%
\pgfsetstrokeopacity{0.903941}%
\pgfsetdash{}{0pt}%
\pgfpathmoveto{\pgfqpoint{1.900091in}{0.943888in}}%
\pgfpathcurveto{\pgfqpoint{1.908327in}{0.943888in}}{\pgfqpoint{1.916227in}{0.947160in}}{\pgfqpoint{1.922051in}{0.952984in}}%
\pgfpathcurveto{\pgfqpoint{1.927875in}{0.958808in}}{\pgfqpoint{1.931147in}{0.966708in}}{\pgfqpoint{1.931147in}{0.974944in}}%
\pgfpathcurveto{\pgfqpoint{1.931147in}{0.983180in}}{\pgfqpoint{1.927875in}{0.991080in}}{\pgfqpoint{1.922051in}{0.996904in}}%
\pgfpathcurveto{\pgfqpoint{1.916227in}{1.002728in}}{\pgfqpoint{1.908327in}{1.006001in}}{\pgfqpoint{1.900091in}{1.006001in}}%
\pgfpathcurveto{\pgfqpoint{1.891855in}{1.006001in}}{\pgfqpoint{1.883955in}{1.002728in}}{\pgfqpoint{1.878131in}{0.996904in}}%
\pgfpathcurveto{\pgfqpoint{1.872307in}{0.991080in}}{\pgfqpoint{1.869034in}{0.983180in}}{\pgfqpoint{1.869034in}{0.974944in}}%
\pgfpathcurveto{\pgfqpoint{1.869034in}{0.966708in}}{\pgfqpoint{1.872307in}{0.958808in}}{\pgfqpoint{1.878131in}{0.952984in}}%
\pgfpathcurveto{\pgfqpoint{1.883955in}{0.947160in}}{\pgfqpoint{1.891855in}{0.943888in}}{\pgfqpoint{1.900091in}{0.943888in}}%
\pgfpathclose%
\pgfusepath{stroke,fill}%
\end{pgfscope}%
\begin{pgfscope}%
\pgfpathrectangle{\pgfqpoint{0.100000in}{0.212622in}}{\pgfqpoint{3.696000in}{3.696000in}}%
\pgfusepath{clip}%
\pgfsetbuttcap%
\pgfsetroundjoin%
\definecolor{currentfill}{rgb}{0.121569,0.466667,0.705882}%
\pgfsetfillcolor{currentfill}%
\pgfsetfillopacity{0.904520}%
\pgfsetlinewidth{1.003750pt}%
\definecolor{currentstroke}{rgb}{0.121569,0.466667,0.705882}%
\pgfsetstrokecolor{currentstroke}%
\pgfsetstrokeopacity{0.904520}%
\pgfsetdash{}{0pt}%
\pgfpathmoveto{\pgfqpoint{1.902801in}{0.943386in}}%
\pgfpathcurveto{\pgfqpoint{1.911037in}{0.943386in}}{\pgfqpoint{1.918937in}{0.946659in}}{\pgfqpoint{1.924761in}{0.952482in}}%
\pgfpathcurveto{\pgfqpoint{1.930585in}{0.958306in}}{\pgfqpoint{1.933857in}{0.966206in}}{\pgfqpoint{1.933857in}{0.974443in}}%
\pgfpathcurveto{\pgfqpoint{1.933857in}{0.982679in}}{\pgfqpoint{1.930585in}{0.990579in}}{\pgfqpoint{1.924761in}{0.996403in}}%
\pgfpathcurveto{\pgfqpoint{1.918937in}{1.002227in}}{\pgfqpoint{1.911037in}{1.005499in}}{\pgfqpoint{1.902801in}{1.005499in}}%
\pgfpathcurveto{\pgfqpoint{1.894565in}{1.005499in}}{\pgfqpoint{1.886665in}{1.002227in}}{\pgfqpoint{1.880841in}{0.996403in}}%
\pgfpathcurveto{\pgfqpoint{1.875017in}{0.990579in}}{\pgfqpoint{1.871744in}{0.982679in}}{\pgfqpoint{1.871744in}{0.974443in}}%
\pgfpathcurveto{\pgfqpoint{1.871744in}{0.966206in}}{\pgfqpoint{1.875017in}{0.958306in}}{\pgfqpoint{1.880841in}{0.952482in}}%
\pgfpathcurveto{\pgfqpoint{1.886665in}{0.946659in}}{\pgfqpoint{1.894565in}{0.943386in}}{\pgfqpoint{1.902801in}{0.943386in}}%
\pgfpathclose%
\pgfusepath{stroke,fill}%
\end{pgfscope}%
\begin{pgfscope}%
\pgfpathrectangle{\pgfqpoint{0.100000in}{0.212622in}}{\pgfqpoint{3.696000in}{3.696000in}}%
\pgfusepath{clip}%
\pgfsetbuttcap%
\pgfsetroundjoin%
\definecolor{currentfill}{rgb}{0.121569,0.466667,0.705882}%
\pgfsetfillcolor{currentfill}%
\pgfsetfillopacity{0.905585}%
\pgfsetlinewidth{1.003750pt}%
\definecolor{currentstroke}{rgb}{0.121569,0.466667,0.705882}%
\pgfsetstrokecolor{currentstroke}%
\pgfsetstrokeopacity{0.905585}%
\pgfsetdash{}{0pt}%
\pgfpathmoveto{\pgfqpoint{1.907694in}{0.942354in}}%
\pgfpathcurveto{\pgfqpoint{1.915931in}{0.942354in}}{\pgfqpoint{1.923831in}{0.945626in}}{\pgfqpoint{1.929655in}{0.951450in}}%
\pgfpathcurveto{\pgfqpoint{1.935479in}{0.957274in}}{\pgfqpoint{1.938751in}{0.965174in}}{\pgfqpoint{1.938751in}{0.973410in}}%
\pgfpathcurveto{\pgfqpoint{1.938751in}{0.981646in}}{\pgfqpoint{1.935479in}{0.989546in}}{\pgfqpoint{1.929655in}{0.995370in}}%
\pgfpathcurveto{\pgfqpoint{1.923831in}{1.001194in}}{\pgfqpoint{1.915931in}{1.004467in}}{\pgfqpoint{1.907694in}{1.004467in}}%
\pgfpathcurveto{\pgfqpoint{1.899458in}{1.004467in}}{\pgfqpoint{1.891558in}{1.001194in}}{\pgfqpoint{1.885734in}{0.995370in}}%
\pgfpathcurveto{\pgfqpoint{1.879910in}{0.989546in}}{\pgfqpoint{1.876638in}{0.981646in}}{\pgfqpoint{1.876638in}{0.973410in}}%
\pgfpathcurveto{\pgfqpoint{1.876638in}{0.965174in}}{\pgfqpoint{1.879910in}{0.957274in}}{\pgfqpoint{1.885734in}{0.951450in}}%
\pgfpathcurveto{\pgfqpoint{1.891558in}{0.945626in}}{\pgfqpoint{1.899458in}{0.942354in}}{\pgfqpoint{1.907694in}{0.942354in}}%
\pgfpathclose%
\pgfusepath{stroke,fill}%
\end{pgfscope}%
\begin{pgfscope}%
\pgfpathrectangle{\pgfqpoint{0.100000in}{0.212622in}}{\pgfqpoint{3.696000in}{3.696000in}}%
\pgfusepath{clip}%
\pgfsetbuttcap%
\pgfsetroundjoin%
\definecolor{currentfill}{rgb}{0.121569,0.466667,0.705882}%
\pgfsetfillcolor{currentfill}%
\pgfsetfillopacity{0.907582}%
\pgfsetlinewidth{1.003750pt}%
\definecolor{currentstroke}{rgb}{0.121569,0.466667,0.705882}%
\pgfsetstrokecolor{currentstroke}%
\pgfsetstrokeopacity{0.907582}%
\pgfsetdash{}{0pt}%
\pgfpathmoveto{\pgfqpoint{1.916437in}{0.940022in}}%
\pgfpathcurveto{\pgfqpoint{1.924673in}{0.940022in}}{\pgfqpoint{1.932573in}{0.943294in}}{\pgfqpoint{1.938397in}{0.949118in}}%
\pgfpathcurveto{\pgfqpoint{1.944221in}{0.954942in}}{\pgfqpoint{1.947493in}{0.962842in}}{\pgfqpoint{1.947493in}{0.971078in}}%
\pgfpathcurveto{\pgfqpoint{1.947493in}{0.979314in}}{\pgfqpoint{1.944221in}{0.987214in}}{\pgfqpoint{1.938397in}{0.993038in}}%
\pgfpathcurveto{\pgfqpoint{1.932573in}{0.998862in}}{\pgfqpoint{1.924673in}{1.002135in}}{\pgfqpoint{1.916437in}{1.002135in}}%
\pgfpathcurveto{\pgfqpoint{1.908201in}{1.002135in}}{\pgfqpoint{1.900301in}{0.998862in}}{\pgfqpoint{1.894477in}{0.993038in}}%
\pgfpathcurveto{\pgfqpoint{1.888653in}{0.987214in}}{\pgfqpoint{1.885380in}{0.979314in}}{\pgfqpoint{1.885380in}{0.971078in}}%
\pgfpathcurveto{\pgfqpoint{1.885380in}{0.962842in}}{\pgfqpoint{1.888653in}{0.954942in}}{\pgfqpoint{1.894477in}{0.949118in}}%
\pgfpathcurveto{\pgfqpoint{1.900301in}{0.943294in}}{\pgfqpoint{1.908201in}{0.940022in}}{\pgfqpoint{1.916437in}{0.940022in}}%
\pgfpathclose%
\pgfusepath{stroke,fill}%
\end{pgfscope}%
\begin{pgfscope}%
\pgfpathrectangle{\pgfqpoint{0.100000in}{0.212622in}}{\pgfqpoint{3.696000in}{3.696000in}}%
\pgfusepath{clip}%
\pgfsetbuttcap%
\pgfsetroundjoin%
\definecolor{currentfill}{rgb}{0.121569,0.466667,0.705882}%
\pgfsetfillcolor{currentfill}%
\pgfsetfillopacity{0.911192}%
\pgfsetlinewidth{1.003750pt}%
\definecolor{currentstroke}{rgb}{0.121569,0.466667,0.705882}%
\pgfsetstrokecolor{currentstroke}%
\pgfsetstrokeopacity{0.911192}%
\pgfsetdash{}{0pt}%
\pgfpathmoveto{\pgfqpoint{1.932352in}{0.935728in}}%
\pgfpathcurveto{\pgfqpoint{1.940589in}{0.935728in}}{\pgfqpoint{1.948489in}{0.939000in}}{\pgfqpoint{1.954313in}{0.944824in}}%
\pgfpathcurveto{\pgfqpoint{1.960136in}{0.950648in}}{\pgfqpoint{1.963409in}{0.958548in}}{\pgfqpoint{1.963409in}{0.966785in}}%
\pgfpathcurveto{\pgfqpoint{1.963409in}{0.975021in}}{\pgfqpoint{1.960136in}{0.982921in}}{\pgfqpoint{1.954313in}{0.988745in}}%
\pgfpathcurveto{\pgfqpoint{1.948489in}{0.994569in}}{\pgfqpoint{1.940589in}{0.997841in}}{\pgfqpoint{1.932352in}{0.997841in}}%
\pgfpathcurveto{\pgfqpoint{1.924116in}{0.997841in}}{\pgfqpoint{1.916216in}{0.994569in}}{\pgfqpoint{1.910392in}{0.988745in}}%
\pgfpathcurveto{\pgfqpoint{1.904568in}{0.982921in}}{\pgfqpoint{1.901296in}{0.975021in}}{\pgfqpoint{1.901296in}{0.966785in}}%
\pgfpathcurveto{\pgfqpoint{1.901296in}{0.958548in}}{\pgfqpoint{1.904568in}{0.950648in}}{\pgfqpoint{1.910392in}{0.944824in}}%
\pgfpathcurveto{\pgfqpoint{1.916216in}{0.939000in}}{\pgfqpoint{1.924116in}{0.935728in}}{\pgfqpoint{1.932352in}{0.935728in}}%
\pgfpathclose%
\pgfusepath{stroke,fill}%
\end{pgfscope}%
\begin{pgfscope}%
\pgfpathrectangle{\pgfqpoint{0.100000in}{0.212622in}}{\pgfqpoint{3.696000in}{3.696000in}}%
\pgfusepath{clip}%
\pgfsetbuttcap%
\pgfsetroundjoin%
\definecolor{currentfill}{rgb}{0.121569,0.466667,0.705882}%
\pgfsetfillcolor{currentfill}%
\pgfsetfillopacity{0.912059}%
\pgfsetlinewidth{1.003750pt}%
\definecolor{currentstroke}{rgb}{0.121569,0.466667,0.705882}%
\pgfsetstrokecolor{currentstroke}%
\pgfsetstrokeopacity{0.912059}%
\pgfsetdash{}{0pt}%
\pgfpathmoveto{\pgfqpoint{2.475233in}{1.194211in}}%
\pgfpathcurveto{\pgfqpoint{2.483470in}{1.194211in}}{\pgfqpoint{2.491370in}{1.197484in}}{\pgfqpoint{2.497194in}{1.203308in}}%
\pgfpathcurveto{\pgfqpoint{2.503018in}{1.209132in}}{\pgfqpoint{2.506290in}{1.217032in}}{\pgfqpoint{2.506290in}{1.225268in}}%
\pgfpathcurveto{\pgfqpoint{2.506290in}{1.233504in}}{\pgfqpoint{2.503018in}{1.241404in}}{\pgfqpoint{2.497194in}{1.247228in}}%
\pgfpathcurveto{\pgfqpoint{2.491370in}{1.253052in}}{\pgfqpoint{2.483470in}{1.256324in}}{\pgfqpoint{2.475233in}{1.256324in}}%
\pgfpathcurveto{\pgfqpoint{2.466997in}{1.256324in}}{\pgfqpoint{2.459097in}{1.253052in}}{\pgfqpoint{2.453273in}{1.247228in}}%
\pgfpathcurveto{\pgfqpoint{2.447449in}{1.241404in}}{\pgfqpoint{2.444177in}{1.233504in}}{\pgfqpoint{2.444177in}{1.225268in}}%
\pgfpathcurveto{\pgfqpoint{2.444177in}{1.217032in}}{\pgfqpoint{2.447449in}{1.209132in}}{\pgfqpoint{2.453273in}{1.203308in}}%
\pgfpathcurveto{\pgfqpoint{2.459097in}{1.197484in}}{\pgfqpoint{2.466997in}{1.194211in}}{\pgfqpoint{2.475233in}{1.194211in}}%
\pgfpathclose%
\pgfusepath{stroke,fill}%
\end{pgfscope}%
\begin{pgfscope}%
\pgfpathrectangle{\pgfqpoint{0.100000in}{0.212622in}}{\pgfqpoint{3.696000in}{3.696000in}}%
\pgfusepath{clip}%
\pgfsetbuttcap%
\pgfsetroundjoin%
\definecolor{currentfill}{rgb}{0.121569,0.466667,0.705882}%
\pgfsetfillcolor{currentfill}%
\pgfsetfillopacity{0.913936}%
\pgfsetlinewidth{1.003750pt}%
\definecolor{currentstroke}{rgb}{0.121569,0.466667,0.705882}%
\pgfsetstrokecolor{currentstroke}%
\pgfsetstrokeopacity{0.913936}%
\pgfsetdash{}{0pt}%
\pgfpathmoveto{\pgfqpoint{1.944968in}{0.932697in}}%
\pgfpathcurveto{\pgfqpoint{1.953204in}{0.932697in}}{\pgfqpoint{1.961104in}{0.935969in}}{\pgfqpoint{1.966928in}{0.941793in}}%
\pgfpathcurveto{\pgfqpoint{1.972752in}{0.947617in}}{\pgfqpoint{1.976025in}{0.955517in}}{\pgfqpoint{1.976025in}{0.963753in}}%
\pgfpathcurveto{\pgfqpoint{1.976025in}{0.971990in}}{\pgfqpoint{1.972752in}{0.979890in}}{\pgfqpoint{1.966928in}{0.985714in}}%
\pgfpathcurveto{\pgfqpoint{1.961104in}{0.991538in}}{\pgfqpoint{1.953204in}{0.994810in}}{\pgfqpoint{1.944968in}{0.994810in}}%
\pgfpathcurveto{\pgfqpoint{1.936732in}{0.994810in}}{\pgfqpoint{1.928832in}{0.991538in}}{\pgfqpoint{1.923008in}{0.985714in}}%
\pgfpathcurveto{\pgfqpoint{1.917184in}{0.979890in}}{\pgfqpoint{1.913912in}{0.971990in}}{\pgfqpoint{1.913912in}{0.963753in}}%
\pgfpathcurveto{\pgfqpoint{1.913912in}{0.955517in}}{\pgfqpoint{1.917184in}{0.947617in}}{\pgfqpoint{1.923008in}{0.941793in}}%
\pgfpathcurveto{\pgfqpoint{1.928832in}{0.935969in}}{\pgfqpoint{1.936732in}{0.932697in}}{\pgfqpoint{1.944968in}{0.932697in}}%
\pgfpathclose%
\pgfusepath{stroke,fill}%
\end{pgfscope}%
\begin{pgfscope}%
\pgfpathrectangle{\pgfqpoint{0.100000in}{0.212622in}}{\pgfqpoint{3.696000in}{3.696000in}}%
\pgfusepath{clip}%
\pgfsetbuttcap%
\pgfsetroundjoin%
\definecolor{currentfill}{rgb}{0.121569,0.466667,0.705882}%
\pgfsetfillcolor{currentfill}%
\pgfsetfillopacity{0.915820}%
\pgfsetlinewidth{1.003750pt}%
\definecolor{currentstroke}{rgb}{0.121569,0.466667,0.705882}%
\pgfsetstrokecolor{currentstroke}%
\pgfsetstrokeopacity{0.915820}%
\pgfsetdash{}{0pt}%
\pgfpathmoveto{\pgfqpoint{1.953881in}{0.931014in}}%
\pgfpathcurveto{\pgfqpoint{1.962118in}{0.931014in}}{\pgfqpoint{1.970018in}{0.934287in}}{\pgfqpoint{1.975842in}{0.940111in}}%
\pgfpathcurveto{\pgfqpoint{1.981666in}{0.945935in}}{\pgfqpoint{1.984938in}{0.953835in}}{\pgfqpoint{1.984938in}{0.962071in}}%
\pgfpathcurveto{\pgfqpoint{1.984938in}{0.970307in}}{\pgfqpoint{1.981666in}{0.978207in}}{\pgfqpoint{1.975842in}{0.984031in}}%
\pgfpathcurveto{\pgfqpoint{1.970018in}{0.989855in}}{\pgfqpoint{1.962118in}{0.993127in}}{\pgfqpoint{1.953881in}{0.993127in}}%
\pgfpathcurveto{\pgfqpoint{1.945645in}{0.993127in}}{\pgfqpoint{1.937745in}{0.989855in}}{\pgfqpoint{1.931921in}{0.984031in}}%
\pgfpathcurveto{\pgfqpoint{1.926097in}{0.978207in}}{\pgfqpoint{1.922825in}{0.970307in}}{\pgfqpoint{1.922825in}{0.962071in}}%
\pgfpathcurveto{\pgfqpoint{1.922825in}{0.953835in}}{\pgfqpoint{1.926097in}{0.945935in}}{\pgfqpoint{1.931921in}{0.940111in}}%
\pgfpathcurveto{\pgfqpoint{1.937745in}{0.934287in}}{\pgfqpoint{1.945645in}{0.931014in}}{\pgfqpoint{1.953881in}{0.931014in}}%
\pgfpathclose%
\pgfusepath{stroke,fill}%
\end{pgfscope}%
\begin{pgfscope}%
\pgfpathrectangle{\pgfqpoint{0.100000in}{0.212622in}}{\pgfqpoint{3.696000in}{3.696000in}}%
\pgfusepath{clip}%
\pgfsetbuttcap%
\pgfsetroundjoin%
\definecolor{currentfill}{rgb}{0.121569,0.466667,0.705882}%
\pgfsetfillcolor{currentfill}%
\pgfsetfillopacity{0.916987}%
\pgfsetlinewidth{1.003750pt}%
\definecolor{currentstroke}{rgb}{0.121569,0.466667,0.705882}%
\pgfsetstrokecolor{currentstroke}%
\pgfsetstrokeopacity{0.916987}%
\pgfsetdash{}{0pt}%
\pgfpathmoveto{\pgfqpoint{1.959004in}{0.929576in}}%
\pgfpathcurveto{\pgfqpoint{1.967241in}{0.929576in}}{\pgfqpoint{1.975141in}{0.932848in}}{\pgfqpoint{1.980965in}{0.938672in}}%
\pgfpathcurveto{\pgfqpoint{1.986788in}{0.944496in}}{\pgfqpoint{1.990061in}{0.952396in}}{\pgfqpoint{1.990061in}{0.960633in}}%
\pgfpathcurveto{\pgfqpoint{1.990061in}{0.968869in}}{\pgfqpoint{1.986788in}{0.976769in}}{\pgfqpoint{1.980965in}{0.982593in}}%
\pgfpathcurveto{\pgfqpoint{1.975141in}{0.988417in}}{\pgfqpoint{1.967241in}{0.991689in}}{\pgfqpoint{1.959004in}{0.991689in}}%
\pgfpathcurveto{\pgfqpoint{1.950768in}{0.991689in}}{\pgfqpoint{1.942868in}{0.988417in}}{\pgfqpoint{1.937044in}{0.982593in}}%
\pgfpathcurveto{\pgfqpoint{1.931220in}{0.976769in}}{\pgfqpoint{1.927948in}{0.968869in}}{\pgfqpoint{1.927948in}{0.960633in}}%
\pgfpathcurveto{\pgfqpoint{1.927948in}{0.952396in}}{\pgfqpoint{1.931220in}{0.944496in}}{\pgfqpoint{1.937044in}{0.938672in}}%
\pgfpathcurveto{\pgfqpoint{1.942868in}{0.932848in}}{\pgfqpoint{1.950768in}{0.929576in}}{\pgfqpoint{1.959004in}{0.929576in}}%
\pgfpathclose%
\pgfusepath{stroke,fill}%
\end{pgfscope}%
\begin{pgfscope}%
\pgfpathrectangle{\pgfqpoint{0.100000in}{0.212622in}}{\pgfqpoint{3.696000in}{3.696000in}}%
\pgfusepath{clip}%
\pgfsetbuttcap%
\pgfsetroundjoin%
\definecolor{currentfill}{rgb}{0.121569,0.466667,0.705882}%
\pgfsetfillcolor{currentfill}%
\pgfsetfillopacity{0.917707}%
\pgfsetlinewidth{1.003750pt}%
\definecolor{currentstroke}{rgb}{0.121569,0.466667,0.705882}%
\pgfsetstrokecolor{currentstroke}%
\pgfsetstrokeopacity{0.917707}%
\pgfsetdash{}{0pt}%
\pgfpathmoveto{\pgfqpoint{1.962258in}{0.928809in}}%
\pgfpathcurveto{\pgfqpoint{1.970495in}{0.928809in}}{\pgfqpoint{1.978395in}{0.932081in}}{\pgfqpoint{1.984219in}{0.937905in}}%
\pgfpathcurveto{\pgfqpoint{1.990043in}{0.943729in}}{\pgfqpoint{1.993315in}{0.951629in}}{\pgfqpoint{1.993315in}{0.959865in}}%
\pgfpathcurveto{\pgfqpoint{1.993315in}{0.968101in}}{\pgfqpoint{1.990043in}{0.976002in}}{\pgfqpoint{1.984219in}{0.981825in}}%
\pgfpathcurveto{\pgfqpoint{1.978395in}{0.987649in}}{\pgfqpoint{1.970495in}{0.990922in}}{\pgfqpoint{1.962258in}{0.990922in}}%
\pgfpathcurveto{\pgfqpoint{1.954022in}{0.990922in}}{\pgfqpoint{1.946122in}{0.987649in}}{\pgfqpoint{1.940298in}{0.981825in}}%
\pgfpathcurveto{\pgfqpoint{1.934474in}{0.976002in}}{\pgfqpoint{1.931202in}{0.968101in}}{\pgfqpoint{1.931202in}{0.959865in}}%
\pgfpathcurveto{\pgfqpoint{1.931202in}{0.951629in}}{\pgfqpoint{1.934474in}{0.943729in}}{\pgfqpoint{1.940298in}{0.937905in}}%
\pgfpathcurveto{\pgfqpoint{1.946122in}{0.932081in}}{\pgfqpoint{1.954022in}{0.928809in}}{\pgfqpoint{1.962258in}{0.928809in}}%
\pgfpathclose%
\pgfusepath{stroke,fill}%
\end{pgfscope}%
\begin{pgfscope}%
\pgfpathrectangle{\pgfqpoint{0.100000in}{0.212622in}}{\pgfqpoint{3.696000in}{3.696000in}}%
\pgfusepath{clip}%
\pgfsetbuttcap%
\pgfsetroundjoin%
\definecolor{currentfill}{rgb}{0.121569,0.466667,0.705882}%
\pgfsetfillcolor{currentfill}%
\pgfsetfillopacity{0.917935}%
\pgfsetlinewidth{1.003750pt}%
\definecolor{currentstroke}{rgb}{0.121569,0.466667,0.705882}%
\pgfsetstrokecolor{currentstroke}%
\pgfsetstrokeopacity{0.917935}%
\pgfsetdash{}{0pt}%
\pgfpathmoveto{\pgfqpoint{1.963275in}{0.928540in}}%
\pgfpathcurveto{\pgfqpoint{1.971511in}{0.928540in}}{\pgfqpoint{1.979411in}{0.931813in}}{\pgfqpoint{1.985235in}{0.937637in}}%
\pgfpathcurveto{\pgfqpoint{1.991059in}{0.943461in}}{\pgfqpoint{1.994331in}{0.951361in}}{\pgfqpoint{1.994331in}{0.959597in}}%
\pgfpathcurveto{\pgfqpoint{1.994331in}{0.967833in}}{\pgfqpoint{1.991059in}{0.975733in}}{\pgfqpoint{1.985235in}{0.981557in}}%
\pgfpathcurveto{\pgfqpoint{1.979411in}{0.987381in}}{\pgfqpoint{1.971511in}{0.990653in}}{\pgfqpoint{1.963275in}{0.990653in}}%
\pgfpathcurveto{\pgfqpoint{1.955038in}{0.990653in}}{\pgfqpoint{1.947138in}{0.987381in}}{\pgfqpoint{1.941314in}{0.981557in}}%
\pgfpathcurveto{\pgfqpoint{1.935490in}{0.975733in}}{\pgfqpoint{1.932218in}{0.967833in}}{\pgfqpoint{1.932218in}{0.959597in}}%
\pgfpathcurveto{\pgfqpoint{1.932218in}{0.951361in}}{\pgfqpoint{1.935490in}{0.943461in}}{\pgfqpoint{1.941314in}{0.937637in}}%
\pgfpathcurveto{\pgfqpoint{1.947138in}{0.931813in}}{\pgfqpoint{1.955038in}{0.928540in}}{\pgfqpoint{1.963275in}{0.928540in}}%
\pgfpathclose%
\pgfusepath{stroke,fill}%
\end{pgfscope}%
\begin{pgfscope}%
\pgfpathrectangle{\pgfqpoint{0.100000in}{0.212622in}}{\pgfqpoint{3.696000in}{3.696000in}}%
\pgfusepath{clip}%
\pgfsetbuttcap%
\pgfsetroundjoin%
\definecolor{currentfill}{rgb}{0.121569,0.466667,0.705882}%
\pgfsetfillcolor{currentfill}%
\pgfsetfillopacity{0.918344}%
\pgfsetlinewidth{1.003750pt}%
\definecolor{currentstroke}{rgb}{0.121569,0.466667,0.705882}%
\pgfsetstrokecolor{currentstroke}%
\pgfsetstrokeopacity{0.918344}%
\pgfsetdash{}{0pt}%
\pgfpathmoveto{\pgfqpoint{1.965138in}{0.928100in}}%
\pgfpathcurveto{\pgfqpoint{1.973374in}{0.928100in}}{\pgfqpoint{1.981274in}{0.931372in}}{\pgfqpoint{1.987098in}{0.937196in}}%
\pgfpathcurveto{\pgfqpoint{1.992922in}{0.943020in}}{\pgfqpoint{1.996195in}{0.950920in}}{\pgfqpoint{1.996195in}{0.959157in}}%
\pgfpathcurveto{\pgfqpoint{1.996195in}{0.967393in}}{\pgfqpoint{1.992922in}{0.975293in}}{\pgfqpoint{1.987098in}{0.981117in}}%
\pgfpathcurveto{\pgfqpoint{1.981274in}{0.986941in}}{\pgfqpoint{1.973374in}{0.990213in}}{\pgfqpoint{1.965138in}{0.990213in}}%
\pgfpathcurveto{\pgfqpoint{1.956902in}{0.990213in}}{\pgfqpoint{1.949002in}{0.986941in}}{\pgfqpoint{1.943178in}{0.981117in}}%
\pgfpathcurveto{\pgfqpoint{1.937354in}{0.975293in}}{\pgfqpoint{1.934082in}{0.967393in}}{\pgfqpoint{1.934082in}{0.959157in}}%
\pgfpathcurveto{\pgfqpoint{1.934082in}{0.950920in}}{\pgfqpoint{1.937354in}{0.943020in}}{\pgfqpoint{1.943178in}{0.937196in}}%
\pgfpathcurveto{\pgfqpoint{1.949002in}{0.931372in}}{\pgfqpoint{1.956902in}{0.928100in}}{\pgfqpoint{1.965138in}{0.928100in}}%
\pgfpathclose%
\pgfusepath{stroke,fill}%
\end{pgfscope}%
\begin{pgfscope}%
\pgfpathrectangle{\pgfqpoint{0.100000in}{0.212622in}}{\pgfqpoint{3.696000in}{3.696000in}}%
\pgfusepath{clip}%
\pgfsetbuttcap%
\pgfsetroundjoin%
\definecolor{currentfill}{rgb}{0.121569,0.466667,0.705882}%
\pgfsetfillcolor{currentfill}%
\pgfsetfillopacity{0.919090}%
\pgfsetlinewidth{1.003750pt}%
\definecolor{currentstroke}{rgb}{0.121569,0.466667,0.705882}%
\pgfsetstrokecolor{currentstroke}%
\pgfsetstrokeopacity{0.919090}%
\pgfsetdash{}{0pt}%
\pgfpathmoveto{\pgfqpoint{1.968519in}{0.927278in}}%
\pgfpathcurveto{\pgfqpoint{1.976755in}{0.927278in}}{\pgfqpoint{1.984655in}{0.930550in}}{\pgfqpoint{1.990479in}{0.936374in}}%
\pgfpathcurveto{\pgfqpoint{1.996303in}{0.942198in}}{\pgfqpoint{1.999576in}{0.950098in}}{\pgfqpoint{1.999576in}{0.958334in}}%
\pgfpathcurveto{\pgfqpoint{1.999576in}{0.966570in}}{\pgfqpoint{1.996303in}{0.974471in}}{\pgfqpoint{1.990479in}{0.980294in}}%
\pgfpathcurveto{\pgfqpoint{1.984655in}{0.986118in}}{\pgfqpoint{1.976755in}{0.989391in}}{\pgfqpoint{1.968519in}{0.989391in}}%
\pgfpathcurveto{\pgfqpoint{1.960283in}{0.989391in}}{\pgfqpoint{1.952383in}{0.986118in}}{\pgfqpoint{1.946559in}{0.980294in}}%
\pgfpathcurveto{\pgfqpoint{1.940735in}{0.974471in}}{\pgfqpoint{1.937463in}{0.966570in}}{\pgfqpoint{1.937463in}{0.958334in}}%
\pgfpathcurveto{\pgfqpoint{1.937463in}{0.950098in}}{\pgfqpoint{1.940735in}{0.942198in}}{\pgfqpoint{1.946559in}{0.936374in}}%
\pgfpathcurveto{\pgfqpoint{1.952383in}{0.930550in}}{\pgfqpoint{1.960283in}{0.927278in}}{\pgfqpoint{1.968519in}{0.927278in}}%
\pgfpathclose%
\pgfusepath{stroke,fill}%
\end{pgfscope}%
\begin{pgfscope}%
\pgfpathrectangle{\pgfqpoint{0.100000in}{0.212622in}}{\pgfqpoint{3.696000in}{3.696000in}}%
\pgfusepath{clip}%
\pgfsetbuttcap%
\pgfsetroundjoin%
\definecolor{currentfill}{rgb}{0.121569,0.466667,0.705882}%
\pgfsetfillcolor{currentfill}%
\pgfsetfillopacity{0.920442}%
\pgfsetlinewidth{1.003750pt}%
\definecolor{currentstroke}{rgb}{0.121569,0.466667,0.705882}%
\pgfsetstrokecolor{currentstroke}%
\pgfsetstrokeopacity{0.920442}%
\pgfsetdash{}{0pt}%
\pgfpathmoveto{\pgfqpoint{1.974703in}{0.925917in}}%
\pgfpathcurveto{\pgfqpoint{1.982939in}{0.925917in}}{\pgfqpoint{1.990839in}{0.929190in}}{\pgfqpoint{1.996663in}{0.935014in}}%
\pgfpathcurveto{\pgfqpoint{2.002487in}{0.940838in}}{\pgfqpoint{2.005760in}{0.948738in}}{\pgfqpoint{2.005760in}{0.956974in}}%
\pgfpathcurveto{\pgfqpoint{2.005760in}{0.965210in}}{\pgfqpoint{2.002487in}{0.973110in}}{\pgfqpoint{1.996663in}{0.978934in}}%
\pgfpathcurveto{\pgfqpoint{1.990839in}{0.984758in}}{\pgfqpoint{1.982939in}{0.988030in}}{\pgfqpoint{1.974703in}{0.988030in}}%
\pgfpathcurveto{\pgfqpoint{1.966467in}{0.988030in}}{\pgfqpoint{1.958567in}{0.984758in}}{\pgfqpoint{1.952743in}{0.978934in}}%
\pgfpathcurveto{\pgfqpoint{1.946919in}{0.973110in}}{\pgfqpoint{1.943647in}{0.965210in}}{\pgfqpoint{1.943647in}{0.956974in}}%
\pgfpathcurveto{\pgfqpoint{1.943647in}{0.948738in}}{\pgfqpoint{1.946919in}{0.940838in}}{\pgfqpoint{1.952743in}{0.935014in}}%
\pgfpathcurveto{\pgfqpoint{1.958567in}{0.929190in}}{\pgfqpoint{1.966467in}{0.925917in}}{\pgfqpoint{1.974703in}{0.925917in}}%
\pgfpathclose%
\pgfusepath{stroke,fill}%
\end{pgfscope}%
\begin{pgfscope}%
\pgfpathrectangle{\pgfqpoint{0.100000in}{0.212622in}}{\pgfqpoint{3.696000in}{3.696000in}}%
\pgfusepath{clip}%
\pgfsetbuttcap%
\pgfsetroundjoin%
\definecolor{currentfill}{rgb}{0.121569,0.466667,0.705882}%
\pgfsetfillcolor{currentfill}%
\pgfsetfillopacity{0.922879}%
\pgfsetlinewidth{1.003750pt}%
\definecolor{currentstroke}{rgb}{0.121569,0.466667,0.705882}%
\pgfsetstrokecolor{currentstroke}%
\pgfsetstrokeopacity{0.922879}%
\pgfsetdash{}{0pt}%
\pgfpathmoveto{\pgfqpoint{1.985986in}{0.923530in}}%
\pgfpathcurveto{\pgfqpoint{1.994222in}{0.923530in}}{\pgfqpoint{2.002122in}{0.926802in}}{\pgfqpoint{2.007946in}{0.932626in}}%
\pgfpathcurveto{\pgfqpoint{2.013770in}{0.938450in}}{\pgfqpoint{2.017043in}{0.946350in}}{\pgfqpoint{2.017043in}{0.954587in}}%
\pgfpathcurveto{\pgfqpoint{2.017043in}{0.962823in}}{\pgfqpoint{2.013770in}{0.970723in}}{\pgfqpoint{2.007946in}{0.976547in}}%
\pgfpathcurveto{\pgfqpoint{2.002122in}{0.982371in}}{\pgfqpoint{1.994222in}{0.985643in}}{\pgfqpoint{1.985986in}{0.985643in}}%
\pgfpathcurveto{\pgfqpoint{1.977750in}{0.985643in}}{\pgfqpoint{1.969850in}{0.982371in}}{\pgfqpoint{1.964026in}{0.976547in}}%
\pgfpathcurveto{\pgfqpoint{1.958202in}{0.970723in}}{\pgfqpoint{1.954930in}{0.962823in}}{\pgfqpoint{1.954930in}{0.954587in}}%
\pgfpathcurveto{\pgfqpoint{1.954930in}{0.946350in}}{\pgfqpoint{1.958202in}{0.938450in}}{\pgfqpoint{1.964026in}{0.932626in}}%
\pgfpathcurveto{\pgfqpoint{1.969850in}{0.926802in}}{\pgfqpoint{1.977750in}{0.923530in}}{\pgfqpoint{1.985986in}{0.923530in}}%
\pgfpathclose%
\pgfusepath{stroke,fill}%
\end{pgfscope}%
\begin{pgfscope}%
\pgfpathrectangle{\pgfqpoint{0.100000in}{0.212622in}}{\pgfqpoint{3.696000in}{3.696000in}}%
\pgfusepath{clip}%
\pgfsetbuttcap%
\pgfsetroundjoin%
\definecolor{currentfill}{rgb}{0.121569,0.466667,0.705882}%
\pgfsetfillcolor{currentfill}%
\pgfsetfillopacity{0.926971}%
\pgfsetlinewidth{1.003750pt}%
\definecolor{currentstroke}{rgb}{0.121569,0.466667,0.705882}%
\pgfsetstrokecolor{currentstroke}%
\pgfsetstrokeopacity{0.926971}%
\pgfsetdash{}{0pt}%
\pgfpathmoveto{\pgfqpoint{2.490334in}{1.140763in}}%
\pgfpathcurveto{\pgfqpoint{2.498571in}{1.140763in}}{\pgfqpoint{2.506471in}{1.144036in}}{\pgfqpoint{2.512295in}{1.149860in}}%
\pgfpathcurveto{\pgfqpoint{2.518119in}{1.155684in}}{\pgfqpoint{2.521391in}{1.163584in}}{\pgfqpoint{2.521391in}{1.171820in}}%
\pgfpathcurveto{\pgfqpoint{2.521391in}{1.180056in}}{\pgfqpoint{2.518119in}{1.187956in}}{\pgfqpoint{2.512295in}{1.193780in}}%
\pgfpathcurveto{\pgfqpoint{2.506471in}{1.199604in}}{\pgfqpoint{2.498571in}{1.202876in}}{\pgfqpoint{2.490334in}{1.202876in}}%
\pgfpathcurveto{\pgfqpoint{2.482098in}{1.202876in}}{\pgfqpoint{2.474198in}{1.199604in}}{\pgfqpoint{2.468374in}{1.193780in}}%
\pgfpathcurveto{\pgfqpoint{2.462550in}{1.187956in}}{\pgfqpoint{2.459278in}{1.180056in}}{\pgfqpoint{2.459278in}{1.171820in}}%
\pgfpathcurveto{\pgfqpoint{2.459278in}{1.163584in}}{\pgfqpoint{2.462550in}{1.155684in}}{\pgfqpoint{2.468374in}{1.149860in}}%
\pgfpathcurveto{\pgfqpoint{2.474198in}{1.144036in}}{\pgfqpoint{2.482098in}{1.140763in}}{\pgfqpoint{2.490334in}{1.140763in}}%
\pgfpathclose%
\pgfusepath{stroke,fill}%
\end{pgfscope}%
\begin{pgfscope}%
\pgfpathrectangle{\pgfqpoint{0.100000in}{0.212622in}}{\pgfqpoint{3.696000in}{3.696000in}}%
\pgfusepath{clip}%
\pgfsetbuttcap%
\pgfsetroundjoin%
\definecolor{currentfill}{rgb}{0.121569,0.466667,0.705882}%
\pgfsetfillcolor{currentfill}%
\pgfsetfillopacity{0.927375}%
\pgfsetlinewidth{1.003750pt}%
\definecolor{currentstroke}{rgb}{0.121569,0.466667,0.705882}%
\pgfsetstrokecolor{currentstroke}%
\pgfsetstrokeopacity{0.927375}%
\pgfsetdash{}{0pt}%
\pgfpathmoveto{\pgfqpoint{2.006314in}{0.918502in}}%
\pgfpathcurveto{\pgfqpoint{2.014551in}{0.918502in}}{\pgfqpoint{2.022451in}{0.921774in}}{\pgfqpoint{2.028275in}{0.927598in}}%
\pgfpathcurveto{\pgfqpoint{2.034099in}{0.933422in}}{\pgfqpoint{2.037371in}{0.941322in}}{\pgfqpoint{2.037371in}{0.949558in}}%
\pgfpathcurveto{\pgfqpoint{2.037371in}{0.957795in}}{\pgfqpoint{2.034099in}{0.965695in}}{\pgfqpoint{2.028275in}{0.971519in}}%
\pgfpathcurveto{\pgfqpoint{2.022451in}{0.977343in}}{\pgfqpoint{2.014551in}{0.980615in}}{\pgfqpoint{2.006314in}{0.980615in}}%
\pgfpathcurveto{\pgfqpoint{1.998078in}{0.980615in}}{\pgfqpoint{1.990178in}{0.977343in}}{\pgfqpoint{1.984354in}{0.971519in}}%
\pgfpathcurveto{\pgfqpoint{1.978530in}{0.965695in}}{\pgfqpoint{1.975258in}{0.957795in}}{\pgfqpoint{1.975258in}{0.949558in}}%
\pgfpathcurveto{\pgfqpoint{1.975258in}{0.941322in}}{\pgfqpoint{1.978530in}{0.933422in}}{\pgfqpoint{1.984354in}{0.927598in}}%
\pgfpathcurveto{\pgfqpoint{1.990178in}{0.921774in}}{\pgfqpoint{1.998078in}{0.918502in}}{\pgfqpoint{2.006314in}{0.918502in}}%
\pgfpathclose%
\pgfusepath{stroke,fill}%
\end{pgfscope}%
\begin{pgfscope}%
\pgfpathrectangle{\pgfqpoint{0.100000in}{0.212622in}}{\pgfqpoint{3.696000in}{3.696000in}}%
\pgfusepath{clip}%
\pgfsetbuttcap%
\pgfsetroundjoin%
\definecolor{currentfill}{rgb}{0.121569,0.466667,0.705882}%
\pgfsetfillcolor{currentfill}%
\pgfsetfillopacity{0.935350}%
\pgfsetlinewidth{1.003750pt}%
\definecolor{currentstroke}{rgb}{0.121569,0.466667,0.705882}%
\pgfsetstrokecolor{currentstroke}%
\pgfsetstrokeopacity{0.935350}%
\pgfsetdash{}{0pt}%
\pgfpathmoveto{\pgfqpoint{2.043809in}{0.910813in}}%
\pgfpathcurveto{\pgfqpoint{2.052045in}{0.910813in}}{\pgfqpoint{2.059945in}{0.914085in}}{\pgfqpoint{2.065769in}{0.919909in}}%
\pgfpathcurveto{\pgfqpoint{2.071593in}{0.925733in}}{\pgfqpoint{2.074866in}{0.933633in}}{\pgfqpoint{2.074866in}{0.941870in}}%
\pgfpathcurveto{\pgfqpoint{2.074866in}{0.950106in}}{\pgfqpoint{2.071593in}{0.958006in}}{\pgfqpoint{2.065769in}{0.963830in}}%
\pgfpathcurveto{\pgfqpoint{2.059945in}{0.969654in}}{\pgfqpoint{2.052045in}{0.972926in}}{\pgfqpoint{2.043809in}{0.972926in}}%
\pgfpathcurveto{\pgfqpoint{2.035573in}{0.972926in}}{\pgfqpoint{2.027673in}{0.969654in}}{\pgfqpoint{2.021849in}{0.963830in}}%
\pgfpathcurveto{\pgfqpoint{2.016025in}{0.958006in}}{\pgfqpoint{2.012753in}{0.950106in}}{\pgfqpoint{2.012753in}{0.941870in}}%
\pgfpathcurveto{\pgfqpoint{2.012753in}{0.933633in}}{\pgfqpoint{2.016025in}{0.925733in}}{\pgfqpoint{2.021849in}{0.919909in}}%
\pgfpathcurveto{\pgfqpoint{2.027673in}{0.914085in}}{\pgfqpoint{2.035573in}{0.910813in}}{\pgfqpoint{2.043809in}{0.910813in}}%
\pgfpathclose%
\pgfusepath{stroke,fill}%
\end{pgfscope}%
\begin{pgfscope}%
\pgfpathrectangle{\pgfqpoint{0.100000in}{0.212622in}}{\pgfqpoint{3.696000in}{3.696000in}}%
\pgfusepath{clip}%
\pgfsetbuttcap%
\pgfsetroundjoin%
\definecolor{currentfill}{rgb}{0.121569,0.466667,0.705882}%
\pgfsetfillcolor{currentfill}%
\pgfsetfillopacity{0.942928}%
\pgfsetlinewidth{1.003750pt}%
\definecolor{currentstroke}{rgb}{0.121569,0.466667,0.705882}%
\pgfsetstrokecolor{currentstroke}%
\pgfsetstrokeopacity{0.942928}%
\pgfsetdash{}{0pt}%
\pgfpathmoveto{\pgfqpoint{2.078524in}{0.902667in}}%
\pgfpathcurveto{\pgfqpoint{2.086760in}{0.902667in}}{\pgfqpoint{2.094660in}{0.905939in}}{\pgfqpoint{2.100484in}{0.911763in}}%
\pgfpathcurveto{\pgfqpoint{2.106308in}{0.917587in}}{\pgfqpoint{2.109580in}{0.925487in}}{\pgfqpoint{2.109580in}{0.933723in}}%
\pgfpathcurveto{\pgfqpoint{2.109580in}{0.941960in}}{\pgfqpoint{2.106308in}{0.949860in}}{\pgfqpoint{2.100484in}{0.955684in}}%
\pgfpathcurveto{\pgfqpoint{2.094660in}{0.961508in}}{\pgfqpoint{2.086760in}{0.964780in}}{\pgfqpoint{2.078524in}{0.964780in}}%
\pgfpathcurveto{\pgfqpoint{2.070288in}{0.964780in}}{\pgfqpoint{2.062388in}{0.961508in}}{\pgfqpoint{2.056564in}{0.955684in}}%
\pgfpathcurveto{\pgfqpoint{2.050740in}{0.949860in}}{\pgfqpoint{2.047467in}{0.941960in}}{\pgfqpoint{2.047467in}{0.933723in}}%
\pgfpathcurveto{\pgfqpoint{2.047467in}{0.925487in}}{\pgfqpoint{2.050740in}{0.917587in}}{\pgfqpoint{2.056564in}{0.911763in}}%
\pgfpathcurveto{\pgfqpoint{2.062388in}{0.905939in}}{\pgfqpoint{2.070288in}{0.902667in}}{\pgfqpoint{2.078524in}{0.902667in}}%
\pgfpathclose%
\pgfusepath{stroke,fill}%
\end{pgfscope}%
\begin{pgfscope}%
\pgfpathrectangle{\pgfqpoint{0.100000in}{0.212622in}}{\pgfqpoint{3.696000in}{3.696000in}}%
\pgfusepath{clip}%
\pgfsetbuttcap%
\pgfsetroundjoin%
\definecolor{currentfill}{rgb}{0.121569,0.466667,0.705882}%
\pgfsetfillcolor{currentfill}%
\pgfsetfillopacity{0.943185}%
\pgfsetlinewidth{1.003750pt}%
\definecolor{currentstroke}{rgb}{0.121569,0.466667,0.705882}%
\pgfsetstrokecolor{currentstroke}%
\pgfsetstrokeopacity{0.943185}%
\pgfsetdash{}{0pt}%
\pgfpathmoveto{\pgfqpoint{2.505404in}{1.082221in}}%
\pgfpathcurveto{\pgfqpoint{2.513640in}{1.082221in}}{\pgfqpoint{2.521540in}{1.085493in}}{\pgfqpoint{2.527364in}{1.091317in}}%
\pgfpathcurveto{\pgfqpoint{2.533188in}{1.097141in}}{\pgfqpoint{2.536460in}{1.105041in}}{\pgfqpoint{2.536460in}{1.113277in}}%
\pgfpathcurveto{\pgfqpoint{2.536460in}{1.121514in}}{\pgfqpoint{2.533188in}{1.129414in}}{\pgfqpoint{2.527364in}{1.135238in}}%
\pgfpathcurveto{\pgfqpoint{2.521540in}{1.141061in}}{\pgfqpoint{2.513640in}{1.144334in}}{\pgfqpoint{2.505404in}{1.144334in}}%
\pgfpathcurveto{\pgfqpoint{2.497167in}{1.144334in}}{\pgfqpoint{2.489267in}{1.141061in}}{\pgfqpoint{2.483443in}{1.135238in}}%
\pgfpathcurveto{\pgfqpoint{2.477619in}{1.129414in}}{\pgfqpoint{2.474347in}{1.121514in}}{\pgfqpoint{2.474347in}{1.113277in}}%
\pgfpathcurveto{\pgfqpoint{2.474347in}{1.105041in}}{\pgfqpoint{2.477619in}{1.097141in}}{\pgfqpoint{2.483443in}{1.091317in}}%
\pgfpathcurveto{\pgfqpoint{2.489267in}{1.085493in}}{\pgfqpoint{2.497167in}{1.082221in}}{\pgfqpoint{2.505404in}{1.082221in}}%
\pgfpathclose%
\pgfusepath{stroke,fill}%
\end{pgfscope}%
\begin{pgfscope}%
\pgfpathrectangle{\pgfqpoint{0.100000in}{0.212622in}}{\pgfqpoint{3.696000in}{3.696000in}}%
\pgfusepath{clip}%
\pgfsetbuttcap%
\pgfsetroundjoin%
\definecolor{currentfill}{rgb}{0.121569,0.466667,0.705882}%
\pgfsetfillcolor{currentfill}%
\pgfsetfillopacity{0.949861}%
\pgfsetlinewidth{1.003750pt}%
\definecolor{currentstroke}{rgb}{0.121569,0.466667,0.705882}%
\pgfsetstrokecolor{currentstroke}%
\pgfsetstrokeopacity{0.949861}%
\pgfsetdash{}{0pt}%
\pgfpathmoveto{\pgfqpoint{2.111546in}{0.896301in}}%
\pgfpathcurveto{\pgfqpoint{2.119782in}{0.896301in}}{\pgfqpoint{2.127682in}{0.899573in}}{\pgfqpoint{2.133506in}{0.905397in}}%
\pgfpathcurveto{\pgfqpoint{2.139330in}{0.911221in}}{\pgfqpoint{2.142602in}{0.919121in}}{\pgfqpoint{2.142602in}{0.927357in}}%
\pgfpathcurveto{\pgfqpoint{2.142602in}{0.935594in}}{\pgfqpoint{2.139330in}{0.943494in}}{\pgfqpoint{2.133506in}{0.949318in}}%
\pgfpathcurveto{\pgfqpoint{2.127682in}{0.955142in}}{\pgfqpoint{2.119782in}{0.958414in}}{\pgfqpoint{2.111546in}{0.958414in}}%
\pgfpathcurveto{\pgfqpoint{2.103310in}{0.958414in}}{\pgfqpoint{2.095410in}{0.955142in}}{\pgfqpoint{2.089586in}{0.949318in}}%
\pgfpathcurveto{\pgfqpoint{2.083762in}{0.943494in}}{\pgfqpoint{2.080489in}{0.935594in}}{\pgfqpoint{2.080489in}{0.927357in}}%
\pgfpathcurveto{\pgfqpoint{2.080489in}{0.919121in}}{\pgfqpoint{2.083762in}{0.911221in}}{\pgfqpoint{2.089586in}{0.905397in}}%
\pgfpathcurveto{\pgfqpoint{2.095410in}{0.899573in}}{\pgfqpoint{2.103310in}{0.896301in}}{\pgfqpoint{2.111546in}{0.896301in}}%
\pgfpathclose%
\pgfusepath{stroke,fill}%
\end{pgfscope}%
\begin{pgfscope}%
\pgfpathrectangle{\pgfqpoint{0.100000in}{0.212622in}}{\pgfqpoint{3.696000in}{3.696000in}}%
\pgfusepath{clip}%
\pgfsetbuttcap%
\pgfsetroundjoin%
\definecolor{currentfill}{rgb}{0.121569,0.466667,0.705882}%
\pgfsetfillcolor{currentfill}%
\pgfsetfillopacity{0.956454}%
\pgfsetlinewidth{1.003750pt}%
\definecolor{currentstroke}{rgb}{0.121569,0.466667,0.705882}%
\pgfsetstrokecolor{currentstroke}%
\pgfsetstrokeopacity{0.956454}%
\pgfsetdash{}{0pt}%
\pgfpathmoveto{\pgfqpoint{2.139231in}{0.886519in}}%
\pgfpathcurveto{\pgfqpoint{2.147467in}{0.886519in}}{\pgfqpoint{2.155367in}{0.889791in}}{\pgfqpoint{2.161191in}{0.895615in}}%
\pgfpathcurveto{\pgfqpoint{2.167015in}{0.901439in}}{\pgfqpoint{2.170287in}{0.909339in}}{\pgfqpoint{2.170287in}{0.917575in}}%
\pgfpathcurveto{\pgfqpoint{2.170287in}{0.925812in}}{\pgfqpoint{2.167015in}{0.933712in}}{\pgfqpoint{2.161191in}{0.939536in}}%
\pgfpathcurveto{\pgfqpoint{2.155367in}{0.945360in}}{\pgfqpoint{2.147467in}{0.948632in}}{\pgfqpoint{2.139231in}{0.948632in}}%
\pgfpathcurveto{\pgfqpoint{2.130995in}{0.948632in}}{\pgfqpoint{2.123095in}{0.945360in}}{\pgfqpoint{2.117271in}{0.939536in}}%
\pgfpathcurveto{\pgfqpoint{2.111447in}{0.933712in}}{\pgfqpoint{2.108174in}{0.925812in}}{\pgfqpoint{2.108174in}{0.917575in}}%
\pgfpathcurveto{\pgfqpoint{2.108174in}{0.909339in}}{\pgfqpoint{2.111447in}{0.901439in}}{\pgfqpoint{2.117271in}{0.895615in}}%
\pgfpathcurveto{\pgfqpoint{2.123095in}{0.889791in}}{\pgfqpoint{2.130995in}{0.886519in}}{\pgfqpoint{2.139231in}{0.886519in}}%
\pgfpathclose%
\pgfusepath{stroke,fill}%
\end{pgfscope}%
\begin{pgfscope}%
\pgfpathrectangle{\pgfqpoint{0.100000in}{0.212622in}}{\pgfqpoint{3.696000in}{3.696000in}}%
\pgfusepath{clip}%
\pgfsetbuttcap%
\pgfsetroundjoin%
\definecolor{currentfill}{rgb}{0.121569,0.466667,0.705882}%
\pgfsetfillcolor{currentfill}%
\pgfsetfillopacity{0.960622}%
\pgfsetlinewidth{1.003750pt}%
\definecolor{currentstroke}{rgb}{0.121569,0.466667,0.705882}%
\pgfsetstrokecolor{currentstroke}%
\pgfsetstrokeopacity{0.960622}%
\pgfsetdash{}{0pt}%
\pgfpathmoveto{\pgfqpoint{2.521718in}{1.019075in}}%
\pgfpathcurveto{\pgfqpoint{2.529955in}{1.019075in}}{\pgfqpoint{2.537855in}{1.022348in}}{\pgfqpoint{2.543679in}{1.028172in}}%
\pgfpathcurveto{\pgfqpoint{2.549503in}{1.033995in}}{\pgfqpoint{2.552775in}{1.041896in}}{\pgfqpoint{2.552775in}{1.050132in}}%
\pgfpathcurveto{\pgfqpoint{2.552775in}{1.058368in}}{\pgfqpoint{2.549503in}{1.066268in}}{\pgfqpoint{2.543679in}{1.072092in}}%
\pgfpathcurveto{\pgfqpoint{2.537855in}{1.077916in}}{\pgfqpoint{2.529955in}{1.081188in}}{\pgfqpoint{2.521718in}{1.081188in}}%
\pgfpathcurveto{\pgfqpoint{2.513482in}{1.081188in}}{\pgfqpoint{2.505582in}{1.077916in}}{\pgfqpoint{2.499758in}{1.072092in}}%
\pgfpathcurveto{\pgfqpoint{2.493934in}{1.066268in}}{\pgfqpoint{2.490662in}{1.058368in}}{\pgfqpoint{2.490662in}{1.050132in}}%
\pgfpathcurveto{\pgfqpoint{2.490662in}{1.041896in}}{\pgfqpoint{2.493934in}{1.033995in}}{\pgfqpoint{2.499758in}{1.028172in}}%
\pgfpathcurveto{\pgfqpoint{2.505582in}{1.022348in}}{\pgfqpoint{2.513482in}{1.019075in}}{\pgfqpoint{2.521718in}{1.019075in}}%
\pgfpathclose%
\pgfusepath{stroke,fill}%
\end{pgfscope}%
\begin{pgfscope}%
\pgfpathrectangle{\pgfqpoint{0.100000in}{0.212622in}}{\pgfqpoint{3.696000in}{3.696000in}}%
\pgfusepath{clip}%
\pgfsetbuttcap%
\pgfsetroundjoin%
\definecolor{currentfill}{rgb}{0.121569,0.466667,0.705882}%
\pgfsetfillcolor{currentfill}%
\pgfsetfillopacity{0.962008}%
\pgfsetlinewidth{1.003750pt}%
\definecolor{currentstroke}{rgb}{0.121569,0.466667,0.705882}%
\pgfsetstrokecolor{currentstroke}%
\pgfsetstrokeopacity{0.962008}%
\pgfsetdash{}{0pt}%
\pgfpathmoveto{\pgfqpoint{2.165681in}{0.881629in}}%
\pgfpathcurveto{\pgfqpoint{2.173918in}{0.881629in}}{\pgfqpoint{2.181818in}{0.884901in}}{\pgfqpoint{2.187642in}{0.890725in}}%
\pgfpathcurveto{\pgfqpoint{2.193466in}{0.896549in}}{\pgfqpoint{2.196738in}{0.904449in}}{\pgfqpoint{2.196738in}{0.912686in}}%
\pgfpathcurveto{\pgfqpoint{2.196738in}{0.920922in}}{\pgfqpoint{2.193466in}{0.928822in}}{\pgfqpoint{2.187642in}{0.934646in}}%
\pgfpathcurveto{\pgfqpoint{2.181818in}{0.940470in}}{\pgfqpoint{2.173918in}{0.943742in}}{\pgfqpoint{2.165681in}{0.943742in}}%
\pgfpathcurveto{\pgfqpoint{2.157445in}{0.943742in}}{\pgfqpoint{2.149545in}{0.940470in}}{\pgfqpoint{2.143721in}{0.934646in}}%
\pgfpathcurveto{\pgfqpoint{2.137897in}{0.928822in}}{\pgfqpoint{2.134625in}{0.920922in}}{\pgfqpoint{2.134625in}{0.912686in}}%
\pgfpathcurveto{\pgfqpoint{2.134625in}{0.904449in}}{\pgfqpoint{2.137897in}{0.896549in}}{\pgfqpoint{2.143721in}{0.890725in}}%
\pgfpathcurveto{\pgfqpoint{2.149545in}{0.884901in}}{\pgfqpoint{2.157445in}{0.881629in}}{\pgfqpoint{2.165681in}{0.881629in}}%
\pgfpathclose%
\pgfusepath{stroke,fill}%
\end{pgfscope}%
\begin{pgfscope}%
\pgfpathrectangle{\pgfqpoint{0.100000in}{0.212622in}}{\pgfqpoint{3.696000in}{3.696000in}}%
\pgfusepath{clip}%
\pgfsetbuttcap%
\pgfsetroundjoin%
\definecolor{currentfill}{rgb}{0.121569,0.466667,0.705882}%
\pgfsetfillcolor{currentfill}%
\pgfsetfillopacity{0.966525}%
\pgfsetlinewidth{1.003750pt}%
\definecolor{currentstroke}{rgb}{0.121569,0.466667,0.705882}%
\pgfsetstrokecolor{currentstroke}%
\pgfsetstrokeopacity{0.966525}%
\pgfsetdash{}{0pt}%
\pgfpathmoveto{\pgfqpoint{2.184668in}{0.875295in}}%
\pgfpathcurveto{\pgfqpoint{2.192905in}{0.875295in}}{\pgfqpoint{2.200805in}{0.878567in}}{\pgfqpoint{2.206629in}{0.884391in}}%
\pgfpathcurveto{\pgfqpoint{2.212453in}{0.890215in}}{\pgfqpoint{2.215725in}{0.898115in}}{\pgfqpoint{2.215725in}{0.906352in}}%
\pgfpathcurveto{\pgfqpoint{2.215725in}{0.914588in}}{\pgfqpoint{2.212453in}{0.922488in}}{\pgfqpoint{2.206629in}{0.928312in}}%
\pgfpathcurveto{\pgfqpoint{2.200805in}{0.934136in}}{\pgfqpoint{2.192905in}{0.937408in}}{\pgfqpoint{2.184668in}{0.937408in}}%
\pgfpathcurveto{\pgfqpoint{2.176432in}{0.937408in}}{\pgfqpoint{2.168532in}{0.934136in}}{\pgfqpoint{2.162708in}{0.928312in}}%
\pgfpathcurveto{\pgfqpoint{2.156884in}{0.922488in}}{\pgfqpoint{2.153612in}{0.914588in}}{\pgfqpoint{2.153612in}{0.906352in}}%
\pgfpathcurveto{\pgfqpoint{2.153612in}{0.898115in}}{\pgfqpoint{2.156884in}{0.890215in}}{\pgfqpoint{2.162708in}{0.884391in}}%
\pgfpathcurveto{\pgfqpoint{2.168532in}{0.878567in}}{\pgfqpoint{2.176432in}{0.875295in}}{\pgfqpoint{2.184668in}{0.875295in}}%
\pgfpathclose%
\pgfusepath{stroke,fill}%
\end{pgfscope}%
\begin{pgfscope}%
\pgfpathrectangle{\pgfqpoint{0.100000in}{0.212622in}}{\pgfqpoint{3.696000in}{3.696000in}}%
\pgfusepath{clip}%
\pgfsetbuttcap%
\pgfsetroundjoin%
\definecolor{currentfill}{rgb}{0.121569,0.466667,0.705882}%
\pgfsetfillcolor{currentfill}%
\pgfsetfillopacity{0.969426}%
\pgfsetlinewidth{1.003750pt}%
\definecolor{currentstroke}{rgb}{0.121569,0.466667,0.705882}%
\pgfsetstrokecolor{currentstroke}%
\pgfsetstrokeopacity{0.969426}%
\pgfsetdash{}{0pt}%
\pgfpathmoveto{\pgfqpoint{2.198170in}{0.872622in}}%
\pgfpathcurveto{\pgfqpoint{2.206407in}{0.872622in}}{\pgfqpoint{2.214307in}{0.875894in}}{\pgfqpoint{2.220131in}{0.881718in}}%
\pgfpathcurveto{\pgfqpoint{2.225955in}{0.887542in}}{\pgfqpoint{2.229227in}{0.895442in}}{\pgfqpoint{2.229227in}{0.903678in}}%
\pgfpathcurveto{\pgfqpoint{2.229227in}{0.911915in}}{\pgfqpoint{2.225955in}{0.919815in}}{\pgfqpoint{2.220131in}{0.925639in}}%
\pgfpathcurveto{\pgfqpoint{2.214307in}{0.931462in}}{\pgfqpoint{2.206407in}{0.934735in}}{\pgfqpoint{2.198170in}{0.934735in}}%
\pgfpathcurveto{\pgfqpoint{2.189934in}{0.934735in}}{\pgfqpoint{2.182034in}{0.931462in}}{\pgfqpoint{2.176210in}{0.925639in}}%
\pgfpathcurveto{\pgfqpoint{2.170386in}{0.919815in}}{\pgfqpoint{2.167114in}{0.911915in}}{\pgfqpoint{2.167114in}{0.903678in}}%
\pgfpathcurveto{\pgfqpoint{2.167114in}{0.895442in}}{\pgfqpoint{2.170386in}{0.887542in}}{\pgfqpoint{2.176210in}{0.881718in}}%
\pgfpathcurveto{\pgfqpoint{2.182034in}{0.875894in}}{\pgfqpoint{2.189934in}{0.872622in}}{\pgfqpoint{2.198170in}{0.872622in}}%
\pgfpathclose%
\pgfusepath{stroke,fill}%
\end{pgfscope}%
\begin{pgfscope}%
\pgfpathrectangle{\pgfqpoint{0.100000in}{0.212622in}}{\pgfqpoint{3.696000in}{3.696000in}}%
\pgfusepath{clip}%
\pgfsetbuttcap%
\pgfsetroundjoin%
\definecolor{currentfill}{rgb}{0.121569,0.466667,0.705882}%
\pgfsetfillcolor{currentfill}%
\pgfsetfillopacity{0.970354}%
\pgfsetlinewidth{1.003750pt}%
\definecolor{currentstroke}{rgb}{0.121569,0.466667,0.705882}%
\pgfsetstrokecolor{currentstroke}%
\pgfsetstrokeopacity{0.970354}%
\pgfsetdash{}{0pt}%
\pgfpathmoveto{\pgfqpoint{2.527446in}{0.985232in}}%
\pgfpathcurveto{\pgfqpoint{2.535682in}{0.985232in}}{\pgfqpoint{2.543582in}{0.988505in}}{\pgfqpoint{2.549406in}{0.994329in}}%
\pgfpathcurveto{\pgfqpoint{2.555230in}{1.000153in}}{\pgfqpoint{2.558502in}{1.008053in}}{\pgfqpoint{2.558502in}{1.016289in}}%
\pgfpathcurveto{\pgfqpoint{2.558502in}{1.024525in}}{\pgfqpoint{2.555230in}{1.032425in}}{\pgfqpoint{2.549406in}{1.038249in}}%
\pgfpathcurveto{\pgfqpoint{2.543582in}{1.044073in}}{\pgfqpoint{2.535682in}{1.047345in}}{\pgfqpoint{2.527446in}{1.047345in}}%
\pgfpathcurveto{\pgfqpoint{2.519210in}{1.047345in}}{\pgfqpoint{2.511310in}{1.044073in}}{\pgfqpoint{2.505486in}{1.038249in}}%
\pgfpathcurveto{\pgfqpoint{2.499662in}{1.032425in}}{\pgfqpoint{2.496389in}{1.024525in}}{\pgfqpoint{2.496389in}{1.016289in}}%
\pgfpathcurveto{\pgfqpoint{2.496389in}{1.008053in}}{\pgfqpoint{2.499662in}{1.000153in}}{\pgfqpoint{2.505486in}{0.994329in}}%
\pgfpathcurveto{\pgfqpoint{2.511310in}{0.988505in}}{\pgfqpoint{2.519210in}{0.985232in}}{\pgfqpoint{2.527446in}{0.985232in}}%
\pgfpathclose%
\pgfusepath{stroke,fill}%
\end{pgfscope}%
\begin{pgfscope}%
\pgfpathrectangle{\pgfqpoint{0.100000in}{0.212622in}}{\pgfqpoint{3.696000in}{3.696000in}}%
\pgfusepath{clip}%
\pgfsetbuttcap%
\pgfsetroundjoin%
\definecolor{currentfill}{rgb}{0.121569,0.466667,0.705882}%
\pgfsetfillcolor{currentfill}%
\pgfsetfillopacity{0.971388}%
\pgfsetlinewidth{1.003750pt}%
\definecolor{currentstroke}{rgb}{0.121569,0.466667,0.705882}%
\pgfsetstrokecolor{currentstroke}%
\pgfsetstrokeopacity{0.971388}%
\pgfsetdash{}{0pt}%
\pgfpathmoveto{\pgfqpoint{2.206944in}{0.870545in}}%
\pgfpathcurveto{\pgfqpoint{2.215181in}{0.870545in}}{\pgfqpoint{2.223081in}{0.873818in}}{\pgfqpoint{2.228905in}{0.879641in}}%
\pgfpathcurveto{\pgfqpoint{2.234729in}{0.885465in}}{\pgfqpoint{2.238001in}{0.893365in}}{\pgfqpoint{2.238001in}{0.901602in}}%
\pgfpathcurveto{\pgfqpoint{2.238001in}{0.909838in}}{\pgfqpoint{2.234729in}{0.917738in}}{\pgfqpoint{2.228905in}{0.923562in}}%
\pgfpathcurveto{\pgfqpoint{2.223081in}{0.929386in}}{\pgfqpoint{2.215181in}{0.932658in}}{\pgfqpoint{2.206944in}{0.932658in}}%
\pgfpathcurveto{\pgfqpoint{2.198708in}{0.932658in}}{\pgfqpoint{2.190808in}{0.929386in}}{\pgfqpoint{2.184984in}{0.923562in}}%
\pgfpathcurveto{\pgfqpoint{2.179160in}{0.917738in}}{\pgfqpoint{2.175888in}{0.909838in}}{\pgfqpoint{2.175888in}{0.901602in}}%
\pgfpathcurveto{\pgfqpoint{2.175888in}{0.893365in}}{\pgfqpoint{2.179160in}{0.885465in}}{\pgfqpoint{2.184984in}{0.879641in}}%
\pgfpathcurveto{\pgfqpoint{2.190808in}{0.873818in}}{\pgfqpoint{2.198708in}{0.870545in}}{\pgfqpoint{2.206944in}{0.870545in}}%
\pgfpathclose%
\pgfusepath{stroke,fill}%
\end{pgfscope}%
\begin{pgfscope}%
\pgfpathrectangle{\pgfqpoint{0.100000in}{0.212622in}}{\pgfqpoint{3.696000in}{3.696000in}}%
\pgfusepath{clip}%
\pgfsetbuttcap%
\pgfsetroundjoin%
\definecolor{currentfill}{rgb}{0.121569,0.466667,0.705882}%
\pgfsetfillcolor{currentfill}%
\pgfsetfillopacity{0.971885}%
\pgfsetlinewidth{1.003750pt}%
\definecolor{currentstroke}{rgb}{0.121569,0.466667,0.705882}%
\pgfsetstrokecolor{currentstroke}%
\pgfsetstrokeopacity{0.971885}%
\pgfsetdash{}{0pt}%
\pgfpathmoveto{\pgfqpoint{2.209237in}{0.870076in}}%
\pgfpathcurveto{\pgfqpoint{2.217474in}{0.870076in}}{\pgfqpoint{2.225374in}{0.873348in}}{\pgfqpoint{2.231198in}{0.879172in}}%
\pgfpathcurveto{\pgfqpoint{2.237022in}{0.884996in}}{\pgfqpoint{2.240294in}{0.892896in}}{\pgfqpoint{2.240294in}{0.901132in}}%
\pgfpathcurveto{\pgfqpoint{2.240294in}{0.909369in}}{\pgfqpoint{2.237022in}{0.917269in}}{\pgfqpoint{2.231198in}{0.923093in}}%
\pgfpathcurveto{\pgfqpoint{2.225374in}{0.928917in}}{\pgfqpoint{2.217474in}{0.932189in}}{\pgfqpoint{2.209237in}{0.932189in}}%
\pgfpathcurveto{\pgfqpoint{2.201001in}{0.932189in}}{\pgfqpoint{2.193101in}{0.928917in}}{\pgfqpoint{2.187277in}{0.923093in}}%
\pgfpathcurveto{\pgfqpoint{2.181453in}{0.917269in}}{\pgfqpoint{2.178181in}{0.909369in}}{\pgfqpoint{2.178181in}{0.901132in}}%
\pgfpathcurveto{\pgfqpoint{2.178181in}{0.892896in}}{\pgfqpoint{2.181453in}{0.884996in}}{\pgfqpoint{2.187277in}{0.879172in}}%
\pgfpathcurveto{\pgfqpoint{2.193101in}{0.873348in}}{\pgfqpoint{2.201001in}{0.870076in}}{\pgfqpoint{2.209237in}{0.870076in}}%
\pgfpathclose%
\pgfusepath{stroke,fill}%
\end{pgfscope}%
\begin{pgfscope}%
\pgfpathrectangle{\pgfqpoint{0.100000in}{0.212622in}}{\pgfqpoint{3.696000in}{3.696000in}}%
\pgfusepath{clip}%
\pgfsetbuttcap%
\pgfsetroundjoin%
\definecolor{currentfill}{rgb}{0.121569,0.466667,0.705882}%
\pgfsetfillcolor{currentfill}%
\pgfsetfillopacity{0.972814}%
\pgfsetlinewidth{1.003750pt}%
\definecolor{currentstroke}{rgb}{0.121569,0.466667,0.705882}%
\pgfsetstrokecolor{currentstroke}%
\pgfsetstrokeopacity{0.972814}%
\pgfsetdash{}{0pt}%
\pgfpathmoveto{\pgfqpoint{2.213273in}{0.868766in}}%
\pgfpathcurveto{\pgfqpoint{2.221510in}{0.868766in}}{\pgfqpoint{2.229410in}{0.872039in}}{\pgfqpoint{2.235234in}{0.877863in}}%
\pgfpathcurveto{\pgfqpoint{2.241058in}{0.883687in}}{\pgfqpoint{2.244330in}{0.891587in}}{\pgfqpoint{2.244330in}{0.899823in}}%
\pgfpathcurveto{\pgfqpoint{2.244330in}{0.908059in}}{\pgfqpoint{2.241058in}{0.915959in}}{\pgfqpoint{2.235234in}{0.921783in}}%
\pgfpathcurveto{\pgfqpoint{2.229410in}{0.927607in}}{\pgfqpoint{2.221510in}{0.930879in}}{\pgfqpoint{2.213273in}{0.930879in}}%
\pgfpathcurveto{\pgfqpoint{2.205037in}{0.930879in}}{\pgfqpoint{2.197137in}{0.927607in}}{\pgfqpoint{2.191313in}{0.921783in}}%
\pgfpathcurveto{\pgfqpoint{2.185489in}{0.915959in}}{\pgfqpoint{2.182217in}{0.908059in}}{\pgfqpoint{2.182217in}{0.899823in}}%
\pgfpathcurveto{\pgfqpoint{2.182217in}{0.891587in}}{\pgfqpoint{2.185489in}{0.883687in}}{\pgfqpoint{2.191313in}{0.877863in}}%
\pgfpathcurveto{\pgfqpoint{2.197137in}{0.872039in}}{\pgfqpoint{2.205037in}{0.868766in}}{\pgfqpoint{2.213273in}{0.868766in}}%
\pgfpathclose%
\pgfusepath{stroke,fill}%
\end{pgfscope}%
\begin{pgfscope}%
\pgfpathrectangle{\pgfqpoint{0.100000in}{0.212622in}}{\pgfqpoint{3.696000in}{3.696000in}}%
\pgfusepath{clip}%
\pgfsetbuttcap%
\pgfsetroundjoin%
\definecolor{currentfill}{rgb}{0.121569,0.466667,0.705882}%
\pgfsetfillcolor{currentfill}%
\pgfsetfillopacity{0.972905}%
\pgfsetlinewidth{1.003750pt}%
\definecolor{currentstroke}{rgb}{0.121569,0.466667,0.705882}%
\pgfsetstrokecolor{currentstroke}%
\pgfsetstrokeopacity{0.972905}%
\pgfsetdash{}{0pt}%
\pgfpathmoveto{\pgfqpoint{2.213695in}{0.868679in}}%
\pgfpathcurveto{\pgfqpoint{2.221932in}{0.868679in}}{\pgfqpoint{2.229832in}{0.871951in}}{\pgfqpoint{2.235656in}{0.877775in}}%
\pgfpathcurveto{\pgfqpoint{2.241479in}{0.883599in}}{\pgfqpoint{2.244752in}{0.891499in}}{\pgfqpoint{2.244752in}{0.899736in}}%
\pgfpathcurveto{\pgfqpoint{2.244752in}{0.907972in}}{\pgfqpoint{2.241479in}{0.915872in}}{\pgfqpoint{2.235656in}{0.921696in}}%
\pgfpathcurveto{\pgfqpoint{2.229832in}{0.927520in}}{\pgfqpoint{2.221932in}{0.930792in}}{\pgfqpoint{2.213695in}{0.930792in}}%
\pgfpathcurveto{\pgfqpoint{2.205459in}{0.930792in}}{\pgfqpoint{2.197559in}{0.927520in}}{\pgfqpoint{2.191735in}{0.921696in}}%
\pgfpathcurveto{\pgfqpoint{2.185911in}{0.915872in}}{\pgfqpoint{2.182639in}{0.907972in}}{\pgfqpoint{2.182639in}{0.899736in}}%
\pgfpathcurveto{\pgfqpoint{2.182639in}{0.891499in}}{\pgfqpoint{2.185911in}{0.883599in}}{\pgfqpoint{2.191735in}{0.877775in}}%
\pgfpathcurveto{\pgfqpoint{2.197559in}{0.871951in}}{\pgfqpoint{2.205459in}{0.868679in}}{\pgfqpoint{2.213695in}{0.868679in}}%
\pgfpathclose%
\pgfusepath{stroke,fill}%
\end{pgfscope}%
\begin{pgfscope}%
\pgfpathrectangle{\pgfqpoint{0.100000in}{0.212622in}}{\pgfqpoint{3.696000in}{3.696000in}}%
\pgfusepath{clip}%
\pgfsetbuttcap%
\pgfsetroundjoin%
\definecolor{currentfill}{rgb}{0.121569,0.466667,0.705882}%
\pgfsetfillcolor{currentfill}%
\pgfsetfillopacity{0.973069}%
\pgfsetlinewidth{1.003750pt}%
\definecolor{currentstroke}{rgb}{0.121569,0.466667,0.705882}%
\pgfsetstrokecolor{currentstroke}%
\pgfsetstrokeopacity{0.973069}%
\pgfsetdash{}{0pt}%
\pgfpathmoveto{\pgfqpoint{2.214465in}{0.868527in}}%
\pgfpathcurveto{\pgfqpoint{2.222701in}{0.868527in}}{\pgfqpoint{2.230601in}{0.871799in}}{\pgfqpoint{2.236425in}{0.877623in}}%
\pgfpathcurveto{\pgfqpoint{2.242249in}{0.883447in}}{\pgfqpoint{2.245521in}{0.891347in}}{\pgfqpoint{2.245521in}{0.899583in}}%
\pgfpathcurveto{\pgfqpoint{2.245521in}{0.907820in}}{\pgfqpoint{2.242249in}{0.915720in}}{\pgfqpoint{2.236425in}{0.921544in}}%
\pgfpathcurveto{\pgfqpoint{2.230601in}{0.927368in}}{\pgfqpoint{2.222701in}{0.930640in}}{\pgfqpoint{2.214465in}{0.930640in}}%
\pgfpathcurveto{\pgfqpoint{2.206228in}{0.930640in}}{\pgfqpoint{2.198328in}{0.927368in}}{\pgfqpoint{2.192504in}{0.921544in}}%
\pgfpathcurveto{\pgfqpoint{2.186680in}{0.915720in}}{\pgfqpoint{2.183408in}{0.907820in}}{\pgfqpoint{2.183408in}{0.899583in}}%
\pgfpathcurveto{\pgfqpoint{2.183408in}{0.891347in}}{\pgfqpoint{2.186680in}{0.883447in}}{\pgfqpoint{2.192504in}{0.877623in}}%
\pgfpathcurveto{\pgfqpoint{2.198328in}{0.871799in}}{\pgfqpoint{2.206228in}{0.868527in}}{\pgfqpoint{2.214465in}{0.868527in}}%
\pgfpathclose%
\pgfusepath{stroke,fill}%
\end{pgfscope}%
\begin{pgfscope}%
\pgfpathrectangle{\pgfqpoint{0.100000in}{0.212622in}}{\pgfqpoint{3.696000in}{3.696000in}}%
\pgfusepath{clip}%
\pgfsetbuttcap%
\pgfsetroundjoin%
\definecolor{currentfill}{rgb}{0.121569,0.466667,0.705882}%
\pgfsetfillcolor{currentfill}%
\pgfsetfillopacity{0.973365}%
\pgfsetlinewidth{1.003750pt}%
\definecolor{currentstroke}{rgb}{0.121569,0.466667,0.705882}%
\pgfsetstrokecolor{currentstroke}%
\pgfsetstrokeopacity{0.973365}%
\pgfsetdash{}{0pt}%
\pgfpathmoveto{\pgfqpoint{2.215865in}{0.868249in}}%
\pgfpathcurveto{\pgfqpoint{2.224101in}{0.868249in}}{\pgfqpoint{2.232001in}{0.871521in}}{\pgfqpoint{2.237825in}{0.877345in}}%
\pgfpathcurveto{\pgfqpoint{2.243649in}{0.883169in}}{\pgfqpoint{2.246922in}{0.891069in}}{\pgfqpoint{2.246922in}{0.899305in}}%
\pgfpathcurveto{\pgfqpoint{2.246922in}{0.907542in}}{\pgfqpoint{2.243649in}{0.915442in}}{\pgfqpoint{2.237825in}{0.921266in}}%
\pgfpathcurveto{\pgfqpoint{2.232001in}{0.927090in}}{\pgfqpoint{2.224101in}{0.930362in}}{\pgfqpoint{2.215865in}{0.930362in}}%
\pgfpathcurveto{\pgfqpoint{2.207629in}{0.930362in}}{\pgfqpoint{2.199729in}{0.927090in}}{\pgfqpoint{2.193905in}{0.921266in}}%
\pgfpathcurveto{\pgfqpoint{2.188081in}{0.915442in}}{\pgfqpoint{2.184809in}{0.907542in}}{\pgfqpoint{2.184809in}{0.899305in}}%
\pgfpathcurveto{\pgfqpoint{2.184809in}{0.891069in}}{\pgfqpoint{2.188081in}{0.883169in}}{\pgfqpoint{2.193905in}{0.877345in}}%
\pgfpathcurveto{\pgfqpoint{2.199729in}{0.871521in}}{\pgfqpoint{2.207629in}{0.868249in}}{\pgfqpoint{2.215865in}{0.868249in}}%
\pgfpathclose%
\pgfusepath{stroke,fill}%
\end{pgfscope}%
\begin{pgfscope}%
\pgfpathrectangle{\pgfqpoint{0.100000in}{0.212622in}}{\pgfqpoint{3.696000in}{3.696000in}}%
\pgfusepath{clip}%
\pgfsetbuttcap%
\pgfsetroundjoin%
\definecolor{currentfill}{rgb}{0.121569,0.466667,0.705882}%
\pgfsetfillcolor{currentfill}%
\pgfsetfillopacity{0.973882}%
\pgfsetlinewidth{1.003750pt}%
\definecolor{currentstroke}{rgb}{0.121569,0.466667,0.705882}%
\pgfsetstrokecolor{currentstroke}%
\pgfsetstrokeopacity{0.973882}%
\pgfsetdash{}{0pt}%
\pgfpathmoveto{\pgfqpoint{2.218437in}{0.867816in}}%
\pgfpathcurveto{\pgfqpoint{2.226674in}{0.867816in}}{\pgfqpoint{2.234574in}{0.871088in}}{\pgfqpoint{2.240398in}{0.876912in}}%
\pgfpathcurveto{\pgfqpoint{2.246222in}{0.882736in}}{\pgfqpoint{2.249494in}{0.890636in}}{\pgfqpoint{2.249494in}{0.898872in}}%
\pgfpathcurveto{\pgfqpoint{2.249494in}{0.907108in}}{\pgfqpoint{2.246222in}{0.915008in}}{\pgfqpoint{2.240398in}{0.920832in}}%
\pgfpathcurveto{\pgfqpoint{2.234574in}{0.926656in}}{\pgfqpoint{2.226674in}{0.929929in}}{\pgfqpoint{2.218437in}{0.929929in}}%
\pgfpathcurveto{\pgfqpoint{2.210201in}{0.929929in}}{\pgfqpoint{2.202301in}{0.926656in}}{\pgfqpoint{2.196477in}{0.920832in}}%
\pgfpathcurveto{\pgfqpoint{2.190653in}{0.915008in}}{\pgfqpoint{2.187381in}{0.907108in}}{\pgfqpoint{2.187381in}{0.898872in}}%
\pgfpathcurveto{\pgfqpoint{2.187381in}{0.890636in}}{\pgfqpoint{2.190653in}{0.882736in}}{\pgfqpoint{2.196477in}{0.876912in}}%
\pgfpathcurveto{\pgfqpoint{2.202301in}{0.871088in}}{\pgfqpoint{2.210201in}{0.867816in}}{\pgfqpoint{2.218437in}{0.867816in}}%
\pgfpathclose%
\pgfusepath{stroke,fill}%
\end{pgfscope}%
\begin{pgfscope}%
\pgfpathrectangle{\pgfqpoint{0.100000in}{0.212622in}}{\pgfqpoint{3.696000in}{3.696000in}}%
\pgfusepath{clip}%
\pgfsetbuttcap%
\pgfsetroundjoin%
\definecolor{currentfill}{rgb}{0.121569,0.466667,0.705882}%
\pgfsetfillcolor{currentfill}%
\pgfsetfillopacity{0.974811}%
\pgfsetlinewidth{1.003750pt}%
\definecolor{currentstroke}{rgb}{0.121569,0.466667,0.705882}%
\pgfsetstrokecolor{currentstroke}%
\pgfsetstrokeopacity{0.974811}%
\pgfsetdash{}{0pt}%
\pgfpathmoveto{\pgfqpoint{2.223125in}{0.867045in}}%
\pgfpathcurveto{\pgfqpoint{2.231362in}{0.867045in}}{\pgfqpoint{2.239262in}{0.870317in}}{\pgfqpoint{2.245086in}{0.876141in}}%
\pgfpathcurveto{\pgfqpoint{2.250909in}{0.881965in}}{\pgfqpoint{2.254182in}{0.889865in}}{\pgfqpoint{2.254182in}{0.898101in}}%
\pgfpathcurveto{\pgfqpoint{2.254182in}{0.906338in}}{\pgfqpoint{2.250909in}{0.914238in}}{\pgfqpoint{2.245086in}{0.920062in}}%
\pgfpathcurveto{\pgfqpoint{2.239262in}{0.925886in}}{\pgfqpoint{2.231362in}{0.929158in}}{\pgfqpoint{2.223125in}{0.929158in}}%
\pgfpathcurveto{\pgfqpoint{2.214889in}{0.929158in}}{\pgfqpoint{2.206989in}{0.925886in}}{\pgfqpoint{2.201165in}{0.920062in}}%
\pgfpathcurveto{\pgfqpoint{2.195341in}{0.914238in}}{\pgfqpoint{2.192069in}{0.906338in}}{\pgfqpoint{2.192069in}{0.898101in}}%
\pgfpathcurveto{\pgfqpoint{2.192069in}{0.889865in}}{\pgfqpoint{2.195341in}{0.881965in}}{\pgfqpoint{2.201165in}{0.876141in}}%
\pgfpathcurveto{\pgfqpoint{2.206989in}{0.870317in}}{\pgfqpoint{2.214889in}{0.867045in}}{\pgfqpoint{2.223125in}{0.867045in}}%
\pgfpathclose%
\pgfusepath{stroke,fill}%
\end{pgfscope}%
\begin{pgfscope}%
\pgfpathrectangle{\pgfqpoint{0.100000in}{0.212622in}}{\pgfqpoint{3.696000in}{3.696000in}}%
\pgfusepath{clip}%
\pgfsetbuttcap%
\pgfsetroundjoin%
\definecolor{currentfill}{rgb}{0.121569,0.466667,0.705882}%
\pgfsetfillcolor{currentfill}%
\pgfsetfillopacity{0.976375}%
\pgfsetlinewidth{1.003750pt}%
\definecolor{currentstroke}{rgb}{0.121569,0.466667,0.705882}%
\pgfsetstrokecolor{currentstroke}%
\pgfsetstrokeopacity{0.976375}%
\pgfsetdash{}{0pt}%
\pgfpathmoveto{\pgfqpoint{2.231769in}{0.866083in}}%
\pgfpathcurveto{\pgfqpoint{2.240005in}{0.866083in}}{\pgfqpoint{2.247905in}{0.869356in}}{\pgfqpoint{2.253729in}{0.875179in}}%
\pgfpathcurveto{\pgfqpoint{2.259553in}{0.881003in}}{\pgfqpoint{2.262826in}{0.888903in}}{\pgfqpoint{2.262826in}{0.897140in}}%
\pgfpathcurveto{\pgfqpoint{2.262826in}{0.905376in}}{\pgfqpoint{2.259553in}{0.913276in}}{\pgfqpoint{2.253729in}{0.919100in}}%
\pgfpathcurveto{\pgfqpoint{2.247905in}{0.924924in}}{\pgfqpoint{2.240005in}{0.928196in}}{\pgfqpoint{2.231769in}{0.928196in}}%
\pgfpathcurveto{\pgfqpoint{2.223533in}{0.928196in}}{\pgfqpoint{2.215633in}{0.924924in}}{\pgfqpoint{2.209809in}{0.919100in}}%
\pgfpathcurveto{\pgfqpoint{2.203985in}{0.913276in}}{\pgfqpoint{2.200713in}{0.905376in}}{\pgfqpoint{2.200713in}{0.897140in}}%
\pgfpathcurveto{\pgfqpoint{2.200713in}{0.888903in}}{\pgfqpoint{2.203985in}{0.881003in}}{\pgfqpoint{2.209809in}{0.875179in}}%
\pgfpathcurveto{\pgfqpoint{2.215633in}{0.869356in}}{\pgfqpoint{2.223533in}{0.866083in}}{\pgfqpoint{2.231769in}{0.866083in}}%
\pgfpathclose%
\pgfusepath{stroke,fill}%
\end{pgfscope}%
\begin{pgfscope}%
\pgfpathrectangle{\pgfqpoint{0.100000in}{0.212622in}}{\pgfqpoint{3.696000in}{3.696000in}}%
\pgfusepath{clip}%
\pgfsetbuttcap%
\pgfsetroundjoin%
\definecolor{currentfill}{rgb}{0.121569,0.466667,0.705882}%
\pgfsetfillcolor{currentfill}%
\pgfsetfillopacity{0.979157}%
\pgfsetlinewidth{1.003750pt}%
\definecolor{currentstroke}{rgb}{0.121569,0.466667,0.705882}%
\pgfsetstrokecolor{currentstroke}%
\pgfsetstrokeopacity{0.979157}%
\pgfsetdash{}{0pt}%
\pgfpathmoveto{\pgfqpoint{2.247446in}{0.863734in}}%
\pgfpathcurveto{\pgfqpoint{2.255683in}{0.863734in}}{\pgfqpoint{2.263583in}{0.867006in}}{\pgfqpoint{2.269407in}{0.872830in}}%
\pgfpathcurveto{\pgfqpoint{2.275231in}{0.878654in}}{\pgfqpoint{2.278503in}{0.886554in}}{\pgfqpoint{2.278503in}{0.894791in}}%
\pgfpathcurveto{\pgfqpoint{2.278503in}{0.903027in}}{\pgfqpoint{2.275231in}{0.910927in}}{\pgfqpoint{2.269407in}{0.916751in}}%
\pgfpathcurveto{\pgfqpoint{2.263583in}{0.922575in}}{\pgfqpoint{2.255683in}{0.925847in}}{\pgfqpoint{2.247446in}{0.925847in}}%
\pgfpathcurveto{\pgfqpoint{2.239210in}{0.925847in}}{\pgfqpoint{2.231310in}{0.922575in}}{\pgfqpoint{2.225486in}{0.916751in}}%
\pgfpathcurveto{\pgfqpoint{2.219662in}{0.910927in}}{\pgfqpoint{2.216390in}{0.903027in}}{\pgfqpoint{2.216390in}{0.894791in}}%
\pgfpathcurveto{\pgfqpoint{2.216390in}{0.886554in}}{\pgfqpoint{2.219662in}{0.878654in}}{\pgfqpoint{2.225486in}{0.872830in}}%
\pgfpathcurveto{\pgfqpoint{2.231310in}{0.867006in}}{\pgfqpoint{2.239210in}{0.863734in}}{\pgfqpoint{2.247446in}{0.863734in}}%
\pgfpathclose%
\pgfusepath{stroke,fill}%
\end{pgfscope}%
\begin{pgfscope}%
\pgfpathrectangle{\pgfqpoint{0.100000in}{0.212622in}}{\pgfqpoint{3.696000in}{3.696000in}}%
\pgfusepath{clip}%
\pgfsetbuttcap%
\pgfsetroundjoin%
\definecolor{currentfill}{rgb}{0.121569,0.466667,0.705882}%
\pgfsetfillcolor{currentfill}%
\pgfsetfillopacity{0.979970}%
\pgfsetlinewidth{1.003750pt}%
\definecolor{currentstroke}{rgb}{0.121569,0.466667,0.705882}%
\pgfsetstrokecolor{currentstroke}%
\pgfsetstrokeopacity{0.979970}%
\pgfsetdash{}{0pt}%
\pgfpathmoveto{\pgfqpoint{2.526094in}{0.942545in}}%
\pgfpathcurveto{\pgfqpoint{2.534330in}{0.942545in}}{\pgfqpoint{2.542230in}{0.945817in}}{\pgfqpoint{2.548054in}{0.951641in}}%
\pgfpathcurveto{\pgfqpoint{2.553878in}{0.957465in}}{\pgfqpoint{2.557150in}{0.965365in}}{\pgfqpoint{2.557150in}{0.973602in}}%
\pgfpathcurveto{\pgfqpoint{2.557150in}{0.981838in}}{\pgfqpoint{2.553878in}{0.989738in}}{\pgfqpoint{2.548054in}{0.995562in}}%
\pgfpathcurveto{\pgfqpoint{2.542230in}{1.001386in}}{\pgfqpoint{2.534330in}{1.004658in}}{\pgfqpoint{2.526094in}{1.004658in}}%
\pgfpathcurveto{\pgfqpoint{2.517858in}{1.004658in}}{\pgfqpoint{2.509958in}{1.001386in}}{\pgfqpoint{2.504134in}{0.995562in}}%
\pgfpathcurveto{\pgfqpoint{2.498310in}{0.989738in}}{\pgfqpoint{2.495037in}{0.981838in}}{\pgfqpoint{2.495037in}{0.973602in}}%
\pgfpathcurveto{\pgfqpoint{2.495037in}{0.965365in}}{\pgfqpoint{2.498310in}{0.957465in}}{\pgfqpoint{2.504134in}{0.951641in}}%
\pgfpathcurveto{\pgfqpoint{2.509958in}{0.945817in}}{\pgfqpoint{2.517858in}{0.942545in}}{\pgfqpoint{2.526094in}{0.942545in}}%
\pgfpathclose%
\pgfusepath{stroke,fill}%
\end{pgfscope}%
\begin{pgfscope}%
\pgfpathrectangle{\pgfqpoint{0.100000in}{0.212622in}}{\pgfqpoint{3.696000in}{3.696000in}}%
\pgfusepath{clip}%
\pgfsetbuttcap%
\pgfsetroundjoin%
\definecolor{currentfill}{rgb}{0.121569,0.466667,0.705882}%
\pgfsetfillcolor{currentfill}%
\pgfsetfillopacity{0.983743}%
\pgfsetlinewidth{1.003750pt}%
\definecolor{currentstroke}{rgb}{0.121569,0.466667,0.705882}%
\pgfsetstrokecolor{currentstroke}%
\pgfsetstrokeopacity{0.983743}%
\pgfsetdash{}{0pt}%
\pgfpathmoveto{\pgfqpoint{2.276290in}{0.861085in}}%
\pgfpathcurveto{\pgfqpoint{2.284526in}{0.861085in}}{\pgfqpoint{2.292427in}{0.864358in}}{\pgfqpoint{2.298250in}{0.870182in}}%
\pgfpathcurveto{\pgfqpoint{2.304074in}{0.876006in}}{\pgfqpoint{2.307347in}{0.883906in}}{\pgfqpoint{2.307347in}{0.892142in}}%
\pgfpathcurveto{\pgfqpoint{2.307347in}{0.900378in}}{\pgfqpoint{2.304074in}{0.908278in}}{\pgfqpoint{2.298250in}{0.914102in}}%
\pgfpathcurveto{\pgfqpoint{2.292427in}{0.919926in}}{\pgfqpoint{2.284526in}{0.923198in}}{\pgfqpoint{2.276290in}{0.923198in}}%
\pgfpathcurveto{\pgfqpoint{2.268054in}{0.923198in}}{\pgfqpoint{2.260154in}{0.919926in}}{\pgfqpoint{2.254330in}{0.914102in}}%
\pgfpathcurveto{\pgfqpoint{2.248506in}{0.908278in}}{\pgfqpoint{2.245234in}{0.900378in}}{\pgfqpoint{2.245234in}{0.892142in}}%
\pgfpathcurveto{\pgfqpoint{2.245234in}{0.883906in}}{\pgfqpoint{2.248506in}{0.876006in}}{\pgfqpoint{2.254330in}{0.870182in}}%
\pgfpathcurveto{\pgfqpoint{2.260154in}{0.864358in}}{\pgfqpoint{2.268054in}{0.861085in}}{\pgfqpoint{2.276290in}{0.861085in}}%
\pgfpathclose%
\pgfusepath{stroke,fill}%
\end{pgfscope}%
\begin{pgfscope}%
\pgfpathrectangle{\pgfqpoint{0.100000in}{0.212622in}}{\pgfqpoint{3.696000in}{3.696000in}}%
\pgfusepath{clip}%
\pgfsetbuttcap%
\pgfsetroundjoin%
\definecolor{currentfill}{rgb}{0.121569,0.466667,0.705882}%
\pgfsetfillcolor{currentfill}%
\pgfsetfillopacity{0.987598}%
\pgfsetlinewidth{1.003750pt}%
\definecolor{currentstroke}{rgb}{0.121569,0.466667,0.705882}%
\pgfsetstrokecolor{currentstroke}%
\pgfsetstrokeopacity{0.987598}%
\pgfsetdash{}{0pt}%
\pgfpathmoveto{\pgfqpoint{2.303027in}{0.858364in}}%
\pgfpathcurveto{\pgfqpoint{2.311263in}{0.858364in}}{\pgfqpoint{2.319163in}{0.861636in}}{\pgfqpoint{2.324987in}{0.867460in}}%
\pgfpathcurveto{\pgfqpoint{2.330811in}{0.873284in}}{\pgfqpoint{2.334083in}{0.881184in}}{\pgfqpoint{2.334083in}{0.889420in}}%
\pgfpathcurveto{\pgfqpoint{2.334083in}{0.897656in}}{\pgfqpoint{2.330811in}{0.905557in}}{\pgfqpoint{2.324987in}{0.911380in}}%
\pgfpathcurveto{\pgfqpoint{2.319163in}{0.917204in}}{\pgfqpoint{2.311263in}{0.920477in}}{\pgfqpoint{2.303027in}{0.920477in}}%
\pgfpathcurveto{\pgfqpoint{2.294790in}{0.920477in}}{\pgfqpoint{2.286890in}{0.917204in}}{\pgfqpoint{2.281066in}{0.911380in}}%
\pgfpathcurveto{\pgfqpoint{2.275242in}{0.905557in}}{\pgfqpoint{2.271970in}{0.897656in}}{\pgfqpoint{2.271970in}{0.889420in}}%
\pgfpathcurveto{\pgfqpoint{2.271970in}{0.881184in}}{\pgfqpoint{2.275242in}{0.873284in}}{\pgfqpoint{2.281066in}{0.867460in}}%
\pgfpathcurveto{\pgfqpoint{2.286890in}{0.861636in}}{\pgfqpoint{2.294790in}{0.858364in}}{\pgfqpoint{2.303027in}{0.858364in}}%
\pgfpathclose%
\pgfusepath{stroke,fill}%
\end{pgfscope}%
\begin{pgfscope}%
\pgfpathrectangle{\pgfqpoint{0.100000in}{0.212622in}}{\pgfqpoint{3.696000in}{3.696000in}}%
\pgfusepath{clip}%
\pgfsetbuttcap%
\pgfsetroundjoin%
\definecolor{currentfill}{rgb}{0.121569,0.466667,0.705882}%
\pgfsetfillcolor{currentfill}%
\pgfsetfillopacity{0.990726}%
\pgfsetlinewidth{1.003750pt}%
\definecolor{currentstroke}{rgb}{0.121569,0.466667,0.705882}%
\pgfsetstrokecolor{currentstroke}%
\pgfsetstrokeopacity{0.990726}%
\pgfsetdash{}{0pt}%
\pgfpathmoveto{\pgfqpoint{2.509847in}{0.896556in}}%
\pgfpathcurveto{\pgfqpoint{2.518083in}{0.896556in}}{\pgfqpoint{2.525983in}{0.899828in}}{\pgfqpoint{2.531807in}{0.905652in}}%
\pgfpathcurveto{\pgfqpoint{2.537631in}{0.911476in}}{\pgfqpoint{2.540903in}{0.919376in}}{\pgfqpoint{2.540903in}{0.927612in}}%
\pgfpathcurveto{\pgfqpoint{2.540903in}{0.935849in}}{\pgfqpoint{2.537631in}{0.943749in}}{\pgfqpoint{2.531807in}{0.949573in}}%
\pgfpathcurveto{\pgfqpoint{2.525983in}{0.955396in}}{\pgfqpoint{2.518083in}{0.958669in}}{\pgfqpoint{2.509847in}{0.958669in}}%
\pgfpathcurveto{\pgfqpoint{2.501611in}{0.958669in}}{\pgfqpoint{2.493711in}{0.955396in}}{\pgfqpoint{2.487887in}{0.949573in}}%
\pgfpathcurveto{\pgfqpoint{2.482063in}{0.943749in}}{\pgfqpoint{2.478790in}{0.935849in}}{\pgfqpoint{2.478790in}{0.927612in}}%
\pgfpathcurveto{\pgfqpoint{2.478790in}{0.919376in}}{\pgfqpoint{2.482063in}{0.911476in}}{\pgfqpoint{2.487887in}{0.905652in}}%
\pgfpathcurveto{\pgfqpoint{2.493711in}{0.899828in}}{\pgfqpoint{2.501611in}{0.896556in}}{\pgfqpoint{2.509847in}{0.896556in}}%
\pgfpathclose%
\pgfusepath{stroke,fill}%
\end{pgfscope}%
\begin{pgfscope}%
\pgfpathrectangle{\pgfqpoint{0.100000in}{0.212622in}}{\pgfqpoint{3.696000in}{3.696000in}}%
\pgfusepath{clip}%
\pgfsetbuttcap%
\pgfsetroundjoin%
\definecolor{currentfill}{rgb}{0.121569,0.466667,0.705882}%
\pgfsetfillcolor{currentfill}%
\pgfsetfillopacity{0.994086}%
\pgfsetlinewidth{1.003750pt}%
\definecolor{currentstroke}{rgb}{0.121569,0.466667,0.705882}%
\pgfsetstrokecolor{currentstroke}%
\pgfsetstrokeopacity{0.994086}%
\pgfsetdash{}{0pt}%
\pgfpathmoveto{\pgfqpoint{2.351594in}{0.850276in}}%
\pgfpathcurveto{\pgfqpoint{2.359831in}{0.850276in}}{\pgfqpoint{2.367731in}{0.853549in}}{\pgfqpoint{2.373555in}{0.859373in}}%
\pgfpathcurveto{\pgfqpoint{2.379379in}{0.865196in}}{\pgfqpoint{2.382651in}{0.873097in}}{\pgfqpoint{2.382651in}{0.881333in}}%
\pgfpathcurveto{\pgfqpoint{2.382651in}{0.889569in}}{\pgfqpoint{2.379379in}{0.897469in}}{\pgfqpoint{2.373555in}{0.903293in}}%
\pgfpathcurveto{\pgfqpoint{2.367731in}{0.909117in}}{\pgfqpoint{2.359831in}{0.912389in}}{\pgfqpoint{2.351594in}{0.912389in}}%
\pgfpathcurveto{\pgfqpoint{2.343358in}{0.912389in}}{\pgfqpoint{2.335458in}{0.909117in}}{\pgfqpoint{2.329634in}{0.903293in}}%
\pgfpathcurveto{\pgfqpoint{2.323810in}{0.897469in}}{\pgfqpoint{2.320538in}{0.889569in}}{\pgfqpoint{2.320538in}{0.881333in}}%
\pgfpathcurveto{\pgfqpoint{2.320538in}{0.873097in}}{\pgfqpoint{2.323810in}{0.865196in}}{\pgfqpoint{2.329634in}{0.859373in}}%
\pgfpathcurveto{\pgfqpoint{2.335458in}{0.853549in}}{\pgfqpoint{2.343358in}{0.850276in}}{\pgfqpoint{2.351594in}{0.850276in}}%
\pgfpathclose%
\pgfusepath{stroke,fill}%
\end{pgfscope}%
\begin{pgfscope}%
\pgfpathrectangle{\pgfqpoint{0.100000in}{0.212622in}}{\pgfqpoint{3.696000in}{3.696000in}}%
\pgfusepath{clip}%
\pgfsetbuttcap%
\pgfsetroundjoin%
\definecolor{currentfill}{rgb}{0.121569,0.466667,0.705882}%
\pgfsetfillcolor{currentfill}%
\pgfsetfillopacity{0.998366}%
\pgfsetlinewidth{1.003750pt}%
\definecolor{currentstroke}{rgb}{0.121569,0.466667,0.705882}%
\pgfsetstrokecolor{currentstroke}%
\pgfsetstrokeopacity{0.998366}%
\pgfsetdash{}{0pt}%
\pgfpathmoveto{\pgfqpoint{2.466514in}{0.863638in}}%
\pgfpathcurveto{\pgfqpoint{2.474750in}{0.863638in}}{\pgfqpoint{2.482650in}{0.866910in}}{\pgfqpoint{2.488474in}{0.872734in}}%
\pgfpathcurveto{\pgfqpoint{2.494298in}{0.878558in}}{\pgfqpoint{2.497570in}{0.886458in}}{\pgfqpoint{2.497570in}{0.894694in}}%
\pgfpathcurveto{\pgfqpoint{2.497570in}{0.902930in}}{\pgfqpoint{2.494298in}{0.910830in}}{\pgfqpoint{2.488474in}{0.916654in}}%
\pgfpathcurveto{\pgfqpoint{2.482650in}{0.922478in}}{\pgfqpoint{2.474750in}{0.925751in}}{\pgfqpoint{2.466514in}{0.925751in}}%
\pgfpathcurveto{\pgfqpoint{2.458277in}{0.925751in}}{\pgfqpoint{2.450377in}{0.922478in}}{\pgfqpoint{2.444553in}{0.916654in}}%
\pgfpathcurveto{\pgfqpoint{2.438730in}{0.910830in}}{\pgfqpoint{2.435457in}{0.902930in}}{\pgfqpoint{2.435457in}{0.894694in}}%
\pgfpathcurveto{\pgfqpoint{2.435457in}{0.886458in}}{\pgfqpoint{2.438730in}{0.878558in}}{\pgfqpoint{2.444553in}{0.872734in}}%
\pgfpathcurveto{\pgfqpoint{2.450377in}{0.866910in}}{\pgfqpoint{2.458277in}{0.863638in}}{\pgfqpoint{2.466514in}{0.863638in}}%
\pgfpathclose%
\pgfusepath{stroke,fill}%
\end{pgfscope}%
\begin{pgfscope}%
\pgfpathrectangle{\pgfqpoint{0.100000in}{0.212622in}}{\pgfqpoint{3.696000in}{3.696000in}}%
\pgfusepath{clip}%
\pgfsetbuttcap%
\pgfsetroundjoin%
\definecolor{currentfill}{rgb}{0.121569,0.466667,0.705882}%
\pgfsetfillcolor{currentfill}%
\pgfsetfillopacity{0.998612}%
\pgfsetlinewidth{1.003750pt}%
\definecolor{currentstroke}{rgb}{0.121569,0.466667,0.705882}%
\pgfsetstrokecolor{currentstroke}%
\pgfsetstrokeopacity{0.998612}%
\pgfsetdash{}{0pt}%
\pgfpathmoveto{\pgfqpoint{2.397767in}{0.846945in}}%
\pgfpathcurveto{\pgfqpoint{2.406003in}{0.846945in}}{\pgfqpoint{2.413903in}{0.850217in}}{\pgfqpoint{2.419727in}{0.856041in}}%
\pgfpathcurveto{\pgfqpoint{2.425551in}{0.861865in}}{\pgfqpoint{2.428823in}{0.869765in}}{\pgfqpoint{2.428823in}{0.878001in}}%
\pgfpathcurveto{\pgfqpoint{2.428823in}{0.886238in}}{\pgfqpoint{2.425551in}{0.894138in}}{\pgfqpoint{2.419727in}{0.899962in}}%
\pgfpathcurveto{\pgfqpoint{2.413903in}{0.905786in}}{\pgfqpoint{2.406003in}{0.909058in}}{\pgfqpoint{2.397767in}{0.909058in}}%
\pgfpathcurveto{\pgfqpoint{2.389531in}{0.909058in}}{\pgfqpoint{2.381631in}{0.905786in}}{\pgfqpoint{2.375807in}{0.899962in}}%
\pgfpathcurveto{\pgfqpoint{2.369983in}{0.894138in}}{\pgfqpoint{2.366710in}{0.886238in}}{\pgfqpoint{2.366710in}{0.878001in}}%
\pgfpathcurveto{\pgfqpoint{2.366710in}{0.869765in}}{\pgfqpoint{2.369983in}{0.861865in}}{\pgfqpoint{2.375807in}{0.856041in}}%
\pgfpathcurveto{\pgfqpoint{2.381631in}{0.850217in}}{\pgfqpoint{2.389531in}{0.846945in}}{\pgfqpoint{2.397767in}{0.846945in}}%
\pgfpathclose%
\pgfusepath{stroke,fill}%
\end{pgfscope}%
\begin{pgfscope}%
\pgfpathrectangle{\pgfqpoint{0.100000in}{0.212622in}}{\pgfqpoint{3.696000in}{3.696000in}}%
\pgfusepath{clip}%
\pgfsetbuttcap%
\pgfsetroundjoin%
\definecolor{currentfill}{rgb}{0.121569,0.466667,0.705882}%
\pgfsetfillcolor{currentfill}%
\pgfsetlinewidth{1.003750pt}%
\definecolor{currentstroke}{rgb}{0.121569,0.466667,0.705882}%
\pgfsetstrokecolor{currentstroke}%
\pgfsetdash{}{0pt}%
\pgfpathmoveto{\pgfqpoint{2.436381in}{0.851172in}}%
\pgfpathcurveto{\pgfqpoint{2.444618in}{0.851172in}}{\pgfqpoint{2.452518in}{0.854445in}}{\pgfqpoint{2.458342in}{0.860269in}}%
\pgfpathcurveto{\pgfqpoint{2.464166in}{0.866092in}}{\pgfqpoint{2.467438in}{0.873992in}}{\pgfqpoint{2.467438in}{0.882229in}}%
\pgfpathcurveto{\pgfqpoint{2.467438in}{0.890465in}}{\pgfqpoint{2.464166in}{0.898365in}}{\pgfqpoint{2.458342in}{0.904189in}}%
\pgfpathcurveto{\pgfqpoint{2.452518in}{0.910013in}}{\pgfqpoint{2.444618in}{0.913285in}}{\pgfqpoint{2.436381in}{0.913285in}}%
\pgfpathcurveto{\pgfqpoint{2.428145in}{0.913285in}}{\pgfqpoint{2.420245in}{0.910013in}}{\pgfqpoint{2.414421in}{0.904189in}}%
\pgfpathcurveto{\pgfqpoint{2.408597in}{0.898365in}}{\pgfqpoint{2.405325in}{0.890465in}}{\pgfqpoint{2.405325in}{0.882229in}}%
\pgfpathcurveto{\pgfqpoint{2.405325in}{0.873992in}}{\pgfqpoint{2.408597in}{0.866092in}}{\pgfqpoint{2.414421in}{0.860269in}}%
\pgfpathcurveto{\pgfqpoint{2.420245in}{0.854445in}}{\pgfqpoint{2.428145in}{0.851172in}}{\pgfqpoint{2.436381in}{0.851172in}}%
\pgfpathclose%
\pgfusepath{stroke,fill}%
\end{pgfscope}%
\begin{pgfscope}%
\definecolor{textcolor}{rgb}{0.000000,0.000000,0.000000}%
\pgfsetstrokecolor{textcolor}%
\pgfsetfillcolor{textcolor}%
\pgftext[x=1.948000in,y=3.991956in,,base]{\color{textcolor}\rmfamily\fontsize{12.000000}{14.400000}\selectfont ROLEQ}%
\end{pgfscope}%
\begin{pgfscope}%
\pgfsetbuttcap%
\pgfsetmiterjoin%
\definecolor{currentfill}{rgb}{1.000000,1.000000,1.000000}%
\pgfsetfillcolor{currentfill}%
\pgfsetfillopacity{0.800000}%
\pgfsetlinewidth{1.003750pt}%
\definecolor{currentstroke}{rgb}{0.800000,0.800000,0.800000}%
\pgfsetstrokecolor{currentstroke}%
\pgfsetstrokeopacity{0.800000}%
\pgfsetdash{}{0pt}%
\pgfpathmoveto{\pgfqpoint{2.104889in}{3.410289in}}%
\pgfpathlineto{\pgfqpoint{3.698778in}{3.410289in}}%
\pgfpathquadraticcurveto{\pgfqpoint{3.726556in}{3.410289in}}{\pgfqpoint{3.726556in}{3.438067in}}%
\pgfpathlineto{\pgfqpoint{3.726556in}{3.811400in}}%
\pgfpathquadraticcurveto{\pgfqpoint{3.726556in}{3.839178in}}{\pgfqpoint{3.698778in}{3.839178in}}%
\pgfpathlineto{\pgfqpoint{2.104889in}{3.839178in}}%
\pgfpathquadraticcurveto{\pgfqpoint{2.077111in}{3.839178in}}{\pgfqpoint{2.077111in}{3.811400in}}%
\pgfpathlineto{\pgfqpoint{2.077111in}{3.438067in}}%
\pgfpathquadraticcurveto{\pgfqpoint{2.077111in}{3.410289in}}{\pgfqpoint{2.104889in}{3.410289in}}%
\pgfpathclose%
\pgfusepath{stroke,fill}%
\end{pgfscope}%
\begin{pgfscope}%
\pgfsetrectcap%
\pgfsetroundjoin%
\pgfsetlinewidth{1.505625pt}%
\definecolor{currentstroke}{rgb}{0.121569,0.466667,0.705882}%
\pgfsetstrokecolor{currentstroke}%
\pgfsetdash{}{0pt}%
\pgfpathmoveto{\pgfqpoint{2.132667in}{3.735011in}}%
\pgfpathlineto{\pgfqpoint{2.410444in}{3.735011in}}%
\pgfusepath{stroke}%
\end{pgfscope}%
\begin{pgfscope}%
\definecolor{textcolor}{rgb}{0.000000,0.000000,0.000000}%
\pgfsetstrokecolor{textcolor}%
\pgfsetfillcolor{textcolor}%
\pgftext[x=2.521555in,y=3.686400in,left,base]{\color{textcolor}\rmfamily\fontsize{10.000000}{12.000000}\selectfont Ground truth}%
\end{pgfscope}%
\begin{pgfscope}%
\pgfsetbuttcap%
\pgfsetroundjoin%
\definecolor{currentfill}{rgb}{0.121569,0.466667,0.705882}%
\pgfsetfillcolor{currentfill}%
\pgfsetlinewidth{1.003750pt}%
\definecolor{currentstroke}{rgb}{0.121569,0.466667,0.705882}%
\pgfsetstrokecolor{currentstroke}%
\pgfsetdash{}{0pt}%
\pgfsys@defobject{currentmarker}{\pgfqpoint{-0.031056in}{-0.031056in}}{\pgfqpoint{0.031056in}{0.031056in}}{%
\pgfpathmoveto{\pgfqpoint{0.000000in}{-0.031056in}}%
\pgfpathcurveto{\pgfqpoint{0.008236in}{-0.031056in}}{\pgfqpoint{0.016136in}{-0.027784in}}{\pgfqpoint{0.021960in}{-0.021960in}}%
\pgfpathcurveto{\pgfqpoint{0.027784in}{-0.016136in}}{\pgfqpoint{0.031056in}{-0.008236in}}{\pgfqpoint{0.031056in}{0.000000in}}%
\pgfpathcurveto{\pgfqpoint{0.031056in}{0.008236in}}{\pgfqpoint{0.027784in}{0.016136in}}{\pgfqpoint{0.021960in}{0.021960in}}%
\pgfpathcurveto{\pgfqpoint{0.016136in}{0.027784in}}{\pgfqpoint{0.008236in}{0.031056in}}{\pgfqpoint{0.000000in}{0.031056in}}%
\pgfpathcurveto{\pgfqpoint{-0.008236in}{0.031056in}}{\pgfqpoint{-0.016136in}{0.027784in}}{\pgfqpoint{-0.021960in}{0.021960in}}%
\pgfpathcurveto{\pgfqpoint{-0.027784in}{0.016136in}}{\pgfqpoint{-0.031056in}{0.008236in}}{\pgfqpoint{-0.031056in}{0.000000in}}%
\pgfpathcurveto{\pgfqpoint{-0.031056in}{-0.008236in}}{\pgfqpoint{-0.027784in}{-0.016136in}}{\pgfqpoint{-0.021960in}{-0.021960in}}%
\pgfpathcurveto{\pgfqpoint{-0.016136in}{-0.027784in}}{\pgfqpoint{-0.008236in}{-0.031056in}}{\pgfqpoint{0.000000in}{-0.031056in}}%
\pgfpathclose%
\pgfusepath{stroke,fill}%
}%
\begin{pgfscope}%
\pgfsys@transformshift{2.271555in}{3.529248in}%
\pgfsys@useobject{currentmarker}{}%
\end{pgfscope}%
\end{pgfscope}%
\begin{pgfscope}%
\definecolor{textcolor}{rgb}{0.000000,0.000000,0.000000}%
\pgfsetstrokecolor{textcolor}%
\pgfsetfillcolor{textcolor}%
\pgftext[x=2.521555in,y=3.492789in,left,base]{\color{textcolor}\rmfamily\fontsize{10.000000}{12.000000}\selectfont Estimated position}%
\end{pgfscope}%
\end{pgfpicture}%
\makeatother%
\endgroup%
}
%         \caption{INS Hardware}
%         \label{fig:triangle4_2D}
%     \end{subfigure}
%     \begin{subfigure}{0.49\textwidth}
%         \centering
%         \resizebox{1\linewidth}{!}{%% Creator: Matplotlib, PGF backend
%%
%% To include the figure in your LaTeX document, write
%%   \input{<filename>.pgf}
%%
%% Make sure the required packages are loaded in your preamble
%%   \usepackage{pgf}
%%
%% and, on pdftex
%%   \usepackage[utf8]{inputenc}\DeclareUnicodeCharacter{2212}{-}
%%
%% or, on luatex and xetex
%%   \usepackage{unicode-math}
%%
%% Figures using additional raster images can only be included by \input if
%% they are in the same directory as the main LaTeX file. For loading figures
%% from other directories you can use the `import` package
%%   \usepackage{import}
%%
%% and then include the figures with
%%   \import{<path to file>}{<filename>.pgf}
%%
%% Matplotlib used the following preamble
%%   \usepackage{fontspec}
%%
\begingroup%
\makeatletter%
\begin{pgfpicture}%
\pgfpathrectangle{\pgfpointorigin}{\pgfqpoint{4.342355in}{4.008622in}}%
\pgfusepath{use as bounding box, clip}%
\begin{pgfscope}%
\pgfsetbuttcap%
\pgfsetmiterjoin%
\definecolor{currentfill}{rgb}{1.000000,1.000000,1.000000}%
\pgfsetfillcolor{currentfill}%
\pgfsetlinewidth{0.000000pt}%
\definecolor{currentstroke}{rgb}{1.000000,1.000000,1.000000}%
\pgfsetstrokecolor{currentstroke}%
\pgfsetdash{}{0pt}%
\pgfpathmoveto{\pgfqpoint{0.000000in}{-0.000000in}}%
\pgfpathlineto{\pgfqpoint{4.342355in}{-0.000000in}}%
\pgfpathlineto{\pgfqpoint{4.342355in}{4.008622in}}%
\pgfpathlineto{\pgfqpoint{0.000000in}{4.008622in}}%
\pgfpathclose%
\pgfusepath{fill}%
\end{pgfscope}%
\begin{pgfscope}%
\pgfsetbuttcap%
\pgfsetmiterjoin%
\definecolor{currentfill}{rgb}{1.000000,1.000000,1.000000}%
\pgfsetfillcolor{currentfill}%
\pgfsetlinewidth{0.000000pt}%
\definecolor{currentstroke}{rgb}{0.000000,0.000000,0.000000}%
\pgfsetstrokecolor{currentstroke}%
\pgfsetstrokeopacity{0.000000}%
\pgfsetdash{}{0pt}%
\pgfpathmoveto{\pgfqpoint{0.100000in}{0.212622in}}%
\pgfpathlineto{\pgfqpoint{3.796000in}{0.212622in}}%
\pgfpathlineto{\pgfqpoint{3.796000in}{3.908622in}}%
\pgfpathlineto{\pgfqpoint{0.100000in}{3.908622in}}%
\pgfpathclose%
\pgfusepath{fill}%
\end{pgfscope}%
\begin{pgfscope}%
\pgfsetbuttcap%
\pgfsetmiterjoin%
\definecolor{currentfill}{rgb}{0.950000,0.950000,0.950000}%
\pgfsetfillcolor{currentfill}%
\pgfsetfillopacity{0.500000}%
\pgfsetlinewidth{1.003750pt}%
\definecolor{currentstroke}{rgb}{0.950000,0.950000,0.950000}%
\pgfsetstrokecolor{currentstroke}%
\pgfsetstrokeopacity{0.500000}%
\pgfsetdash{}{0pt}%
\pgfpathmoveto{\pgfqpoint{0.379073in}{1.123938in}}%
\pgfpathlineto{\pgfqpoint{1.599613in}{2.147018in}}%
\pgfpathlineto{\pgfqpoint{1.582647in}{3.622484in}}%
\pgfpathlineto{\pgfqpoint{0.303698in}{2.689165in}}%
\pgfusepath{stroke,fill}%
\end{pgfscope}%
\begin{pgfscope}%
\pgfsetbuttcap%
\pgfsetmiterjoin%
\definecolor{currentfill}{rgb}{0.900000,0.900000,0.900000}%
\pgfsetfillcolor{currentfill}%
\pgfsetfillopacity{0.500000}%
\pgfsetlinewidth{1.003750pt}%
\definecolor{currentstroke}{rgb}{0.900000,0.900000,0.900000}%
\pgfsetstrokecolor{currentstroke}%
\pgfsetstrokeopacity{0.500000}%
\pgfsetdash{}{0pt}%
\pgfpathmoveto{\pgfqpoint{1.599613in}{2.147018in}}%
\pgfpathlineto{\pgfqpoint{3.558144in}{1.577751in}}%
\pgfpathlineto{\pgfqpoint{3.628038in}{3.104037in}}%
\pgfpathlineto{\pgfqpoint{1.582647in}{3.622484in}}%
\pgfusepath{stroke,fill}%
\end{pgfscope}%
\begin{pgfscope}%
\pgfsetbuttcap%
\pgfsetmiterjoin%
\definecolor{currentfill}{rgb}{0.925000,0.925000,0.925000}%
\pgfsetfillcolor{currentfill}%
\pgfsetfillopacity{0.500000}%
\pgfsetlinewidth{1.003750pt}%
\definecolor{currentstroke}{rgb}{0.925000,0.925000,0.925000}%
\pgfsetstrokecolor{currentstroke}%
\pgfsetstrokeopacity{0.500000}%
\pgfsetdash{}{0pt}%
\pgfpathmoveto{\pgfqpoint{0.379073in}{1.123938in}}%
\pgfpathlineto{\pgfqpoint{2.455212in}{0.445871in}}%
\pgfpathlineto{\pgfqpoint{3.558144in}{1.577751in}}%
\pgfpathlineto{\pgfqpoint{1.599613in}{2.147018in}}%
\pgfusepath{stroke,fill}%
\end{pgfscope}%
\begin{pgfscope}%
\pgfsetrectcap%
\pgfsetroundjoin%
\pgfsetlinewidth{0.803000pt}%
\definecolor{currentstroke}{rgb}{0.000000,0.000000,0.000000}%
\pgfsetstrokecolor{currentstroke}%
\pgfsetdash{}{0pt}%
\pgfpathmoveto{\pgfqpoint{0.379073in}{1.123938in}}%
\pgfpathlineto{\pgfqpoint{2.455212in}{0.445871in}}%
\pgfusepath{stroke}%
\end{pgfscope}%
\begin{pgfscope}%
\definecolor{textcolor}{rgb}{0.000000,0.000000,0.000000}%
\pgfsetstrokecolor{textcolor}%
\pgfsetfillcolor{textcolor}%
\pgftext[x=0.730374in, y=0.408886in, left, base,rotate=341.912962]{\color{textcolor}\rmfamily\fontsize{10.000000}{12.000000}\selectfont Position X [\(\displaystyle m\)]}%
\end{pgfscope}%
\begin{pgfscope}%
\pgfsetbuttcap%
\pgfsetroundjoin%
\pgfsetlinewidth{0.803000pt}%
\definecolor{currentstroke}{rgb}{0.690196,0.690196,0.690196}%
\pgfsetstrokecolor{currentstroke}%
\pgfsetdash{}{0pt}%
\pgfpathmoveto{\pgfqpoint{0.504815in}{1.082870in}}%
\pgfpathlineto{\pgfqpoint{1.718725in}{2.112397in}}%
\pgfpathlineto{\pgfqpoint{1.706795in}{3.591016in}}%
\pgfusepath{stroke}%
\end{pgfscope}%
\begin{pgfscope}%
\pgfsetbuttcap%
\pgfsetroundjoin%
\pgfsetlinewidth{0.803000pt}%
\definecolor{currentstroke}{rgb}{0.690196,0.690196,0.690196}%
\pgfsetstrokecolor{currentstroke}%
\pgfsetdash{}{0pt}%
\pgfpathmoveto{\pgfqpoint{0.942263in}{0.940000in}}%
\pgfpathlineto{\pgfqpoint{2.132611in}{1.992097in}}%
\pgfpathlineto{\pgfqpoint{2.138428in}{3.481609in}}%
\pgfusepath{stroke}%
\end{pgfscope}%
\begin{pgfscope}%
\pgfsetbuttcap%
\pgfsetroundjoin%
\pgfsetlinewidth{0.803000pt}%
\definecolor{currentstroke}{rgb}{0.690196,0.690196,0.690196}%
\pgfsetstrokecolor{currentstroke}%
\pgfsetdash{}{0pt}%
\pgfpathmoveto{\pgfqpoint{1.389776in}{0.793842in}}%
\pgfpathlineto{\pgfqpoint{2.555223in}{1.869260in}}%
\pgfpathlineto{\pgfqpoint{2.579558in}{3.369796in}}%
\pgfusepath{stroke}%
\end{pgfscope}%
\begin{pgfscope}%
\pgfsetbuttcap%
\pgfsetroundjoin%
\pgfsetlinewidth{0.803000pt}%
\definecolor{currentstroke}{rgb}{0.690196,0.690196,0.690196}%
\pgfsetstrokecolor{currentstroke}%
\pgfsetdash{}{0pt}%
\pgfpathmoveto{\pgfqpoint{1.847705in}{0.644283in}}%
\pgfpathlineto{\pgfqpoint{2.986839in}{1.743807in}}%
\pgfpathlineto{\pgfqpoint{3.030503in}{3.255494in}}%
\pgfusepath{stroke}%
\end{pgfscope}%
\begin{pgfscope}%
\pgfsetbuttcap%
\pgfsetroundjoin%
\pgfsetlinewidth{0.803000pt}%
\definecolor{currentstroke}{rgb}{0.690196,0.690196,0.690196}%
\pgfsetstrokecolor{currentstroke}%
\pgfsetdash{}{0pt}%
\pgfpathmoveto{\pgfqpoint{2.316418in}{0.491201in}}%
\pgfpathlineto{\pgfqpoint{3.427751in}{1.615651in}}%
\pgfpathlineto{\pgfqpoint{3.491592in}{3.138622in}}%
\pgfusepath{stroke}%
\end{pgfscope}%
\begin{pgfscope}%
\pgfsetrectcap%
\pgfsetroundjoin%
\pgfsetlinewidth{0.803000pt}%
\definecolor{currentstroke}{rgb}{0.000000,0.000000,0.000000}%
\pgfsetstrokecolor{currentstroke}%
\pgfsetdash{}{0pt}%
\pgfpathmoveto{\pgfqpoint{0.515386in}{1.091835in}}%
\pgfpathlineto{\pgfqpoint{0.483629in}{1.064902in}}%
\pgfusepath{stroke}%
\end{pgfscope}%
\begin{pgfscope}%
\definecolor{textcolor}{rgb}{0.000000,0.000000,0.000000}%
\pgfsetstrokecolor{textcolor}%
\pgfsetfillcolor{textcolor}%
\pgftext[x=0.400245in,y=0.864666in,,top]{\color{textcolor}\rmfamily\fontsize{10.000000}{12.000000}\selectfont \(\displaystyle {0}\)}%
\end{pgfscope}%
\begin{pgfscope}%
\pgfsetrectcap%
\pgfsetroundjoin%
\pgfsetlinewidth{0.803000pt}%
\definecolor{currentstroke}{rgb}{0.000000,0.000000,0.000000}%
\pgfsetstrokecolor{currentstroke}%
\pgfsetdash{}{0pt}%
\pgfpathmoveto{\pgfqpoint{0.952638in}{0.949170in}}%
\pgfpathlineto{\pgfqpoint{0.921468in}{0.921620in}}%
\pgfusepath{stroke}%
\end{pgfscope}%
\begin{pgfscope}%
\definecolor{textcolor}{rgb}{0.000000,0.000000,0.000000}%
\pgfsetstrokecolor{textcolor}%
\pgfsetfillcolor{textcolor}%
\pgftext[x=0.838144in,y=0.718767in,,top]{\color{textcolor}\rmfamily\fontsize{10.000000}{12.000000}\selectfont \(\displaystyle {1}\)}%
\end{pgfscope}%
\begin{pgfscope}%
\pgfsetrectcap%
\pgfsetroundjoin%
\pgfsetlinewidth{0.803000pt}%
\definecolor{currentstroke}{rgb}{0.000000,0.000000,0.000000}%
\pgfsetstrokecolor{currentstroke}%
\pgfsetdash{}{0pt}%
\pgfpathmoveto{\pgfqpoint{1.399944in}{0.803224in}}%
\pgfpathlineto{\pgfqpoint{1.369396in}{0.775037in}}%
\pgfusepath{stroke}%
\end{pgfscope}%
\begin{pgfscope}%
\definecolor{textcolor}{rgb}{0.000000,0.000000,0.000000}%
\pgfsetstrokecolor{textcolor}%
\pgfsetfillcolor{textcolor}%
\pgftext[x=1.286157in,y=0.569499in,,top]{\color{textcolor}\rmfamily\fontsize{10.000000}{12.000000}\selectfont \(\displaystyle {2}\)}%
\end{pgfscope}%
\begin{pgfscope}%
\pgfsetrectcap%
\pgfsetroundjoin%
\pgfsetlinewidth{0.803000pt}%
\definecolor{currentstroke}{rgb}{0.000000,0.000000,0.000000}%
\pgfsetstrokecolor{currentstroke}%
\pgfsetdash{}{0pt}%
\pgfpathmoveto{\pgfqpoint{1.857653in}{0.653884in}}%
\pgfpathlineto{\pgfqpoint{1.827765in}{0.625036in}}%
\pgfusepath{stroke}%
\end{pgfscope}%
\begin{pgfscope}%
\definecolor{textcolor}{rgb}{0.000000,0.000000,0.000000}%
\pgfsetstrokecolor{textcolor}%
\pgfsetfillcolor{textcolor}%
\pgftext[x=1.744637in,y=0.416743in,,top]{\color{textcolor}\rmfamily\fontsize{10.000000}{12.000000}\selectfont \(\displaystyle {3}\)}%
\end{pgfscope}%
\begin{pgfscope}%
\pgfsetrectcap%
\pgfsetroundjoin%
\pgfsetlinewidth{0.803000pt}%
\definecolor{currentstroke}{rgb}{0.000000,0.000000,0.000000}%
\pgfsetstrokecolor{currentstroke}%
\pgfsetdash{}{0pt}%
\pgfpathmoveto{\pgfqpoint{2.326132in}{0.501030in}}%
\pgfpathlineto{\pgfqpoint{2.296945in}{0.471499in}}%
\pgfusepath{stroke}%
\end{pgfscope}%
\begin{pgfscope}%
\definecolor{textcolor}{rgb}{0.000000,0.000000,0.000000}%
\pgfsetstrokecolor{textcolor}%
\pgfsetfillcolor{textcolor}%
\pgftext[x=2.213956in,y=0.260376in,,top]{\color{textcolor}\rmfamily\fontsize{10.000000}{12.000000}\selectfont \(\displaystyle {4}\)}%
\end{pgfscope}%
\begin{pgfscope}%
\pgfsetrectcap%
\pgfsetroundjoin%
\pgfsetlinewidth{0.803000pt}%
\definecolor{currentstroke}{rgb}{0.000000,0.000000,0.000000}%
\pgfsetstrokecolor{currentstroke}%
\pgfsetdash{}{0pt}%
\pgfpathmoveto{\pgfqpoint{3.558144in}{1.577751in}}%
\pgfpathlineto{\pgfqpoint{2.455212in}{0.445871in}}%
\pgfusepath{stroke}%
\end{pgfscope}%
\begin{pgfscope}%
\definecolor{textcolor}{rgb}{0.000000,0.000000,0.000000}%
\pgfsetstrokecolor{textcolor}%
\pgfsetfillcolor{textcolor}%
\pgftext[x=3.120747in, y=0.305657in, left, base,rotate=45.742112]{\color{textcolor}\rmfamily\fontsize{10.000000}{12.000000}\selectfont Position Y [\(\displaystyle m\)]}%
\end{pgfscope}%
\begin{pgfscope}%
\pgfsetbuttcap%
\pgfsetroundjoin%
\pgfsetlinewidth{0.803000pt}%
\definecolor{currentstroke}{rgb}{0.690196,0.690196,0.690196}%
\pgfsetstrokecolor{currentstroke}%
\pgfsetdash{}{0pt}%
\pgfpathmoveto{\pgfqpoint{0.479186in}{2.817228in}}%
\pgfpathlineto{\pgfqpoint{0.545999in}{1.263858in}}%
\pgfpathlineto{\pgfqpoint{2.606629in}{0.601262in}}%
\pgfusepath{stroke}%
\end{pgfscope}%
\begin{pgfscope}%
\pgfsetbuttcap%
\pgfsetroundjoin%
\pgfsetlinewidth{0.803000pt}%
\definecolor{currentstroke}{rgb}{0.690196,0.690196,0.690196}%
\pgfsetstrokecolor{currentstroke}%
\pgfsetdash{}{0pt}%
\pgfpathmoveto{\pgfqpoint{0.677085in}{2.961646in}}%
\pgfpathlineto{\pgfqpoint{0.734451in}{1.421822in}}%
\pgfpathlineto{\pgfqpoint{2.777352in}{0.776466in}}%
\pgfusepath{stroke}%
\end{pgfscope}%
\begin{pgfscope}%
\pgfsetbuttcap%
\pgfsetroundjoin%
\pgfsetlinewidth{0.803000pt}%
\definecolor{currentstroke}{rgb}{0.690196,0.690196,0.690196}%
\pgfsetstrokecolor{currentstroke}%
\pgfsetdash{}{0pt}%
\pgfpathmoveto{\pgfqpoint{0.869774in}{3.102262in}}%
\pgfpathlineto{\pgfqpoint{0.918154in}{1.575806in}}%
\pgfpathlineto{\pgfqpoint{2.943550in}{0.947025in}}%
\pgfusepath{stroke}%
\end{pgfscope}%
\begin{pgfscope}%
\pgfsetbuttcap%
\pgfsetroundjoin%
\pgfsetlinewidth{0.803000pt}%
\definecolor{currentstroke}{rgb}{0.690196,0.690196,0.690196}%
\pgfsetstrokecolor{currentstroke}%
\pgfsetdash{}{0pt}%
\pgfpathmoveto{\pgfqpoint{1.057457in}{3.239224in}}%
\pgfpathlineto{\pgfqpoint{1.097287in}{1.725958in}}%
\pgfpathlineto{\pgfqpoint{3.105399in}{1.113123in}}%
\pgfusepath{stroke}%
\end{pgfscope}%
\begin{pgfscope}%
\pgfsetbuttcap%
\pgfsetroundjoin%
\pgfsetlinewidth{0.803000pt}%
\definecolor{currentstroke}{rgb}{0.690196,0.690196,0.690196}%
\pgfsetstrokecolor{currentstroke}%
\pgfsetdash{}{0pt}%
\pgfpathmoveto{\pgfqpoint{1.240326in}{3.372673in}}%
\pgfpathlineto{\pgfqpoint{1.272017in}{1.872420in}}%
\pgfpathlineto{\pgfqpoint{3.263069in}{1.274931in}}%
\pgfusepath{stroke}%
\end{pgfscope}%
\begin{pgfscope}%
\pgfsetbuttcap%
\pgfsetroundjoin%
\pgfsetlinewidth{0.803000pt}%
\definecolor{currentstroke}{rgb}{0.690196,0.690196,0.690196}%
\pgfsetstrokecolor{currentstroke}%
\pgfsetdash{}{0pt}%
\pgfpathmoveto{\pgfqpoint{1.418564in}{3.502743in}}%
\pgfpathlineto{\pgfqpoint{1.442505in}{2.015327in}}%
\pgfpathlineto{\pgfqpoint{3.416720in}{1.432614in}}%
\pgfusepath{stroke}%
\end{pgfscope}%
\begin{pgfscope}%
\pgfsetrectcap%
\pgfsetroundjoin%
\pgfsetlinewidth{0.803000pt}%
\definecolor{currentstroke}{rgb}{0.000000,0.000000,0.000000}%
\pgfsetstrokecolor{currentstroke}%
\pgfsetdash{}{0pt}%
\pgfpathmoveto{\pgfqpoint{2.589270in}{0.606844in}}%
\pgfpathlineto{\pgfqpoint{2.641394in}{0.590084in}}%
\pgfusepath{stroke}%
\end{pgfscope}%
\begin{pgfscope}%
\definecolor{textcolor}{rgb}{0.000000,0.000000,0.000000}%
\pgfsetstrokecolor{textcolor}%
\pgfsetfillcolor{textcolor}%
\pgftext[x=2.784520in,y=0.415969in,,top]{\color{textcolor}\rmfamily\fontsize{10.000000}{12.000000}\selectfont \(\displaystyle {−1}\)}%
\end{pgfscope}%
\begin{pgfscope}%
\pgfsetrectcap%
\pgfsetroundjoin%
\pgfsetlinewidth{0.803000pt}%
\definecolor{currentstroke}{rgb}{0.000000,0.000000,0.000000}%
\pgfsetstrokecolor{currentstroke}%
\pgfsetdash{}{0pt}%
\pgfpathmoveto{\pgfqpoint{2.760153in}{0.781899in}}%
\pgfpathlineto{\pgfqpoint{2.811794in}{0.765586in}}%
\pgfusepath{stroke}%
\end{pgfscope}%
\begin{pgfscope}%
\definecolor{textcolor}{rgb}{0.000000,0.000000,0.000000}%
\pgfsetstrokecolor{textcolor}%
\pgfsetfillcolor{textcolor}%
\pgftext[x=2.952953in,y=0.593766in,,top]{\color{textcolor}\rmfamily\fontsize{10.000000}{12.000000}\selectfont \(\displaystyle {0}\)}%
\end{pgfscope}%
\begin{pgfscope}%
\pgfsetrectcap%
\pgfsetroundjoin%
\pgfsetlinewidth{0.803000pt}%
\definecolor{currentstroke}{rgb}{0.000000,0.000000,0.000000}%
\pgfsetstrokecolor{currentstroke}%
\pgfsetdash{}{0pt}%
\pgfpathmoveto{\pgfqpoint{2.926510in}{0.952315in}}%
\pgfpathlineto{\pgfqpoint{2.977673in}{0.936432in}}%
\pgfusepath{stroke}%
\end{pgfscope}%
\begin{pgfscope}%
\definecolor{textcolor}{rgb}{0.000000,0.000000,0.000000}%
\pgfsetstrokecolor{textcolor}%
\pgfsetfillcolor{textcolor}%
\pgftext[x=3.116918in,y=0.766847in,,top]{\color{textcolor}\rmfamily\fontsize{10.000000}{12.000000}\selectfont \(\displaystyle {1}\)}%
\end{pgfscope}%
\begin{pgfscope}%
\pgfsetrectcap%
\pgfsetroundjoin%
\pgfsetlinewidth{0.803000pt}%
\definecolor{currentstroke}{rgb}{0.000000,0.000000,0.000000}%
\pgfsetstrokecolor{currentstroke}%
\pgfsetdash{}{0pt}%
\pgfpathmoveto{\pgfqpoint{3.088515in}{1.118275in}}%
\pgfpathlineto{\pgfqpoint{3.139209in}{1.102805in}}%
\pgfusepath{stroke}%
\end{pgfscope}%
\begin{pgfscope}%
\definecolor{textcolor}{rgb}{0.000000,0.000000,0.000000}%
\pgfsetstrokecolor{textcolor}%
\pgfsetfillcolor{textcolor}%
\pgftext[x=3.276591in,y=0.935398in,,top]{\color{textcolor}\rmfamily\fontsize{10.000000}{12.000000}\selectfont \(\displaystyle {2}\)}%
\end{pgfscope}%
\begin{pgfscope}%
\pgfsetrectcap%
\pgfsetroundjoin%
\pgfsetlinewidth{0.803000pt}%
\definecolor{currentstroke}{rgb}{0.000000,0.000000,0.000000}%
\pgfsetstrokecolor{currentstroke}%
\pgfsetdash{}{0pt}%
\pgfpathmoveto{\pgfqpoint{3.246339in}{1.279951in}}%
\pgfpathlineto{\pgfqpoint{3.296570in}{1.264878in}}%
\pgfusepath{stroke}%
\end{pgfscope}%
\begin{pgfscope}%
\definecolor{textcolor}{rgb}{0.000000,0.000000,0.000000}%
\pgfsetstrokecolor{textcolor}%
\pgfsetfillcolor{textcolor}%
\pgftext[x=3.432138in,y=1.099593in,,top]{\color{textcolor}\rmfamily\fontsize{10.000000}{12.000000}\selectfont \(\displaystyle {3}\)}%
\end{pgfscope}%
\begin{pgfscope}%
\pgfsetrectcap%
\pgfsetroundjoin%
\pgfsetlinewidth{0.803000pt}%
\definecolor{currentstroke}{rgb}{0.000000,0.000000,0.000000}%
\pgfsetstrokecolor{currentstroke}%
\pgfsetdash{}{0pt}%
\pgfpathmoveto{\pgfqpoint{3.400141in}{1.437507in}}%
\pgfpathlineto{\pgfqpoint{3.449916in}{1.422816in}}%
\pgfusepath{stroke}%
\end{pgfscope}%
\begin{pgfscope}%
\definecolor{textcolor}{rgb}{0.000000,0.000000,0.000000}%
\pgfsetstrokecolor{textcolor}%
\pgfsetfillcolor{textcolor}%
\pgftext[x=3.583718in,y=1.259601in,,top]{\color{textcolor}\rmfamily\fontsize{10.000000}{12.000000}\selectfont \(\displaystyle {4}\)}%
\end{pgfscope}%
\begin{pgfscope}%
\pgfsetrectcap%
\pgfsetroundjoin%
\pgfsetlinewidth{0.803000pt}%
\definecolor{currentstroke}{rgb}{0.000000,0.000000,0.000000}%
\pgfsetstrokecolor{currentstroke}%
\pgfsetdash{}{0pt}%
\pgfpathmoveto{\pgfqpoint{3.558144in}{1.577751in}}%
\pgfpathlineto{\pgfqpoint{3.628038in}{3.104037in}}%
\pgfusepath{stroke}%
\end{pgfscope}%
\begin{pgfscope}%
\definecolor{textcolor}{rgb}{0.000000,0.000000,0.000000}%
\pgfsetstrokecolor{textcolor}%
\pgfsetfillcolor{textcolor}%
\pgftext[x=4.167903in, y=1.963517in, left, base,rotate=87.378092]{\color{textcolor}\rmfamily\fontsize{10.000000}{12.000000}\selectfont Position Z [\(\displaystyle m\)]}%
\end{pgfscope}%
\begin{pgfscope}%
\pgfsetbuttcap%
\pgfsetroundjoin%
\pgfsetlinewidth{0.803000pt}%
\definecolor{currentstroke}{rgb}{0.690196,0.690196,0.690196}%
\pgfsetstrokecolor{currentstroke}%
\pgfsetdash{}{0pt}%
\pgfpathmoveto{\pgfqpoint{3.561286in}{1.646348in}}%
\pgfpathlineto{\pgfqpoint{1.598849in}{2.213466in}}%
\pgfpathlineto{\pgfqpoint{0.375691in}{1.194172in}}%
\pgfusepath{stroke}%
\end{pgfscope}%
\begin{pgfscope}%
\pgfsetbuttcap%
\pgfsetroundjoin%
\pgfsetlinewidth{0.803000pt}%
\definecolor{currentstroke}{rgb}{0.690196,0.690196,0.690196}%
\pgfsetstrokecolor{currentstroke}%
\pgfsetdash{}{0pt}%
\pgfpathmoveto{\pgfqpoint{3.570042in}{1.837559in}}%
\pgfpathlineto{\pgfqpoint{1.596720in}{2.398616in}}%
\pgfpathlineto{\pgfqpoint{0.366261in}{1.390002in}}%
\pgfusepath{stroke}%
\end{pgfscope}%
\begin{pgfscope}%
\pgfsetbuttcap%
\pgfsetroundjoin%
\pgfsetlinewidth{0.803000pt}%
\definecolor{currentstroke}{rgb}{0.690196,0.690196,0.690196}%
\pgfsetstrokecolor{currentstroke}%
\pgfsetdash{}{0pt}%
\pgfpathmoveto{\pgfqpoint{3.578897in}{2.030924in}}%
\pgfpathlineto{\pgfqpoint{1.594568in}{2.585753in}}%
\pgfpathlineto{\pgfqpoint{0.356720in}{1.588123in}}%
\pgfusepath{stroke}%
\end{pgfscope}%
\begin{pgfscope}%
\pgfsetbuttcap%
\pgfsetroundjoin%
\pgfsetlinewidth{0.803000pt}%
\definecolor{currentstroke}{rgb}{0.690196,0.690196,0.690196}%
\pgfsetstrokecolor{currentstroke}%
\pgfsetdash{}{0pt}%
\pgfpathmoveto{\pgfqpoint{3.587852in}{2.226480in}}%
\pgfpathlineto{\pgfqpoint{1.592393in}{2.774908in}}%
\pgfpathlineto{\pgfqpoint{0.347067in}{1.788573in}}%
\pgfusepath{stroke}%
\end{pgfscope}%
\begin{pgfscope}%
\pgfsetbuttcap%
\pgfsetroundjoin%
\pgfsetlinewidth{0.803000pt}%
\definecolor{currentstroke}{rgb}{0.690196,0.690196,0.690196}%
\pgfsetstrokecolor{currentstroke}%
\pgfsetdash{}{0pt}%
\pgfpathmoveto{\pgfqpoint{3.596909in}{2.424263in}}%
\pgfpathlineto{\pgfqpoint{1.590194in}{2.966114in}}%
\pgfpathlineto{\pgfqpoint{0.337300in}{1.991396in}}%
\pgfusepath{stroke}%
\end{pgfscope}%
\begin{pgfscope}%
\pgfsetbuttcap%
\pgfsetroundjoin%
\pgfsetlinewidth{0.803000pt}%
\definecolor{currentstroke}{rgb}{0.690196,0.690196,0.690196}%
\pgfsetstrokecolor{currentstroke}%
\pgfsetdash{}{0pt}%
\pgfpathmoveto{\pgfqpoint{3.606070in}{2.624313in}}%
\pgfpathlineto{\pgfqpoint{1.587972in}{3.159404in}}%
\pgfpathlineto{\pgfqpoint{0.327416in}{2.196633in}}%
\pgfusepath{stroke}%
\end{pgfscope}%
\begin{pgfscope}%
\pgfsetbuttcap%
\pgfsetroundjoin%
\pgfsetlinewidth{0.803000pt}%
\definecolor{currentstroke}{rgb}{0.690196,0.690196,0.690196}%
\pgfsetstrokecolor{currentstroke}%
\pgfsetdash{}{0pt}%
\pgfpathmoveto{\pgfqpoint{3.615336in}{2.826669in}}%
\pgfpathlineto{\pgfqpoint{1.585725in}{3.354814in}}%
\pgfpathlineto{\pgfqpoint{0.317414in}{2.404327in}}%
\pgfusepath{stroke}%
\end{pgfscope}%
\begin{pgfscope}%
\pgfsetbuttcap%
\pgfsetroundjoin%
\pgfsetlinewidth{0.803000pt}%
\definecolor{currentstroke}{rgb}{0.690196,0.690196,0.690196}%
\pgfsetstrokecolor{currentstroke}%
\pgfsetdash{}{0pt}%
\pgfpathmoveto{\pgfqpoint{3.624710in}{3.031369in}}%
\pgfpathlineto{\pgfqpoint{1.583453in}{3.552377in}}%
\pgfpathlineto{\pgfqpoint{0.307292in}{2.614524in}}%
\pgfusepath{stroke}%
\end{pgfscope}%
\begin{pgfscope}%
\pgfsetrectcap%
\pgfsetroundjoin%
\pgfsetlinewidth{0.803000pt}%
\definecolor{currentstroke}{rgb}{0.000000,0.000000,0.000000}%
\pgfsetstrokecolor{currentstroke}%
\pgfsetdash{}{0pt}%
\pgfpathmoveto{\pgfqpoint{3.544814in}{1.651108in}}%
\pgfpathlineto{\pgfqpoint{3.594269in}{1.636817in}}%
\pgfusepath{stroke}%
\end{pgfscope}%
\begin{pgfscope}%
\definecolor{textcolor}{rgb}{0.000000,0.000000,0.000000}%
\pgfsetstrokecolor{textcolor}%
\pgfsetfillcolor{textcolor}%
\pgftext[x=3.815238in,y=1.682395in,,top]{\color{textcolor}\rmfamily\fontsize{10.000000}{12.000000}\selectfont \(\displaystyle {−0.01}\)}%
\end{pgfscope}%
\begin{pgfscope}%
\pgfsetrectcap%
\pgfsetroundjoin%
\pgfsetlinewidth{0.803000pt}%
\definecolor{currentstroke}{rgb}{0.000000,0.000000,0.000000}%
\pgfsetstrokecolor{currentstroke}%
\pgfsetdash{}{0pt}%
\pgfpathmoveto{\pgfqpoint{3.553474in}{1.842270in}}%
\pgfpathlineto{\pgfqpoint{3.603217in}{1.828127in}}%
\pgfusepath{stroke}%
\end{pgfscope}%
\begin{pgfscope}%
\definecolor{textcolor}{rgb}{0.000000,0.000000,0.000000}%
\pgfsetstrokecolor{textcolor}%
\pgfsetfillcolor{textcolor}%
\pgftext[x=3.825386in,y=1.873230in,,top]{\color{textcolor}\rmfamily\fontsize{10.000000}{12.000000}\selectfont \(\displaystyle {0.00}\)}%
\end{pgfscope}%
\begin{pgfscope}%
\pgfsetrectcap%
\pgfsetroundjoin%
\pgfsetlinewidth{0.803000pt}%
\definecolor{currentstroke}{rgb}{0.000000,0.000000,0.000000}%
\pgfsetstrokecolor{currentstroke}%
\pgfsetdash{}{0pt}%
\pgfpathmoveto{\pgfqpoint{3.562232in}{2.035584in}}%
\pgfpathlineto{\pgfqpoint{3.612265in}{2.021594in}}%
\pgfusepath{stroke}%
\end{pgfscope}%
\begin{pgfscope}%
\definecolor{textcolor}{rgb}{0.000000,0.000000,0.000000}%
\pgfsetstrokecolor{textcolor}%
\pgfsetfillcolor{textcolor}%
\pgftext[x=3.835648in,y=2.066208in,,top]{\color{textcolor}\rmfamily\fontsize{10.000000}{12.000000}\selectfont \(\displaystyle {0.01}\)}%
\end{pgfscope}%
\begin{pgfscope}%
\pgfsetrectcap%
\pgfsetroundjoin%
\pgfsetlinewidth{0.803000pt}%
\definecolor{currentstroke}{rgb}{0.000000,0.000000,0.000000}%
\pgfsetstrokecolor{currentstroke}%
\pgfsetdash{}{0pt}%
\pgfpathmoveto{\pgfqpoint{3.571089in}{2.231087in}}%
\pgfpathlineto{\pgfqpoint{3.621417in}{2.217255in}}%
\pgfusepath{stroke}%
\end{pgfscope}%
\begin{pgfscope}%
\definecolor{textcolor}{rgb}{0.000000,0.000000,0.000000}%
\pgfsetstrokecolor{textcolor}%
\pgfsetfillcolor{textcolor}%
\pgftext[x=3.846025in,y=2.261365in,,top]{\color{textcolor}\rmfamily\fontsize{10.000000}{12.000000}\selectfont \(\displaystyle {0.02}\)}%
\end{pgfscope}%
\begin{pgfscope}%
\pgfsetrectcap%
\pgfsetroundjoin%
\pgfsetlinewidth{0.803000pt}%
\definecolor{currentstroke}{rgb}{0.000000,0.000000,0.000000}%
\pgfsetstrokecolor{currentstroke}%
\pgfsetdash{}{0pt}%
\pgfpathmoveto{\pgfqpoint{3.580047in}{2.428816in}}%
\pgfpathlineto{\pgfqpoint{3.630673in}{2.415146in}}%
\pgfusepath{stroke}%
\end{pgfscope}%
\begin{pgfscope}%
\definecolor{textcolor}{rgb}{0.000000,0.000000,0.000000}%
\pgfsetstrokecolor{textcolor}%
\pgfsetfillcolor{textcolor}%
\pgftext[x=3.856521in,y=2.458739in,,top]{\color{textcolor}\rmfamily\fontsize{10.000000}{12.000000}\selectfont \(\displaystyle {0.03}\)}%
\end{pgfscope}%
\begin{pgfscope}%
\pgfsetrectcap%
\pgfsetroundjoin%
\pgfsetlinewidth{0.803000pt}%
\definecolor{currentstroke}{rgb}{0.000000,0.000000,0.000000}%
\pgfsetstrokecolor{currentstroke}%
\pgfsetdash{}{0pt}%
\pgfpathmoveto{\pgfqpoint{3.589108in}{2.628811in}}%
\pgfpathlineto{\pgfqpoint{3.640035in}{2.615307in}}%
\pgfusepath{stroke}%
\end{pgfscope}%
\begin{pgfscope}%
\definecolor{textcolor}{rgb}{0.000000,0.000000,0.000000}%
\pgfsetstrokecolor{textcolor}%
\pgfsetfillcolor{textcolor}%
\pgftext[x=3.867136in,y=2.658368in,,top]{\color{textcolor}\rmfamily\fontsize{10.000000}{12.000000}\selectfont \(\displaystyle {0.04}\)}%
\end{pgfscope}%
\begin{pgfscope}%
\pgfsetrectcap%
\pgfsetroundjoin%
\pgfsetlinewidth{0.803000pt}%
\definecolor{currentstroke}{rgb}{0.000000,0.000000,0.000000}%
\pgfsetstrokecolor{currentstroke}%
\pgfsetdash{}{0pt}%
\pgfpathmoveto{\pgfqpoint{3.598273in}{2.831109in}}%
\pgfpathlineto{\pgfqpoint{3.649505in}{2.817777in}}%
\pgfusepath{stroke}%
\end{pgfscope}%
\begin{pgfscope}%
\definecolor{textcolor}{rgb}{0.000000,0.000000,0.000000}%
\pgfsetstrokecolor{textcolor}%
\pgfsetfillcolor{textcolor}%
\pgftext[x=3.877873in,y=2.860290in,,top]{\color{textcolor}\rmfamily\fontsize{10.000000}{12.000000}\selectfont \(\displaystyle {0.05}\)}%
\end{pgfscope}%
\begin{pgfscope}%
\pgfsetrectcap%
\pgfsetroundjoin%
\pgfsetlinewidth{0.803000pt}%
\definecolor{currentstroke}{rgb}{0.000000,0.000000,0.000000}%
\pgfsetstrokecolor{currentstroke}%
\pgfsetdash{}{0pt}%
\pgfpathmoveto{\pgfqpoint{3.607544in}{3.035751in}}%
\pgfpathlineto{\pgfqpoint{3.659084in}{3.022596in}}%
\pgfusepath{stroke}%
\end{pgfscope}%
\begin{pgfscope}%
\definecolor{textcolor}{rgb}{0.000000,0.000000,0.000000}%
\pgfsetstrokecolor{textcolor}%
\pgfsetfillcolor{textcolor}%
\pgftext[x=3.888735in,y=3.064545in,,top]{\color{textcolor}\rmfamily\fontsize{10.000000}{12.000000}\selectfont \(\displaystyle {0.06}\)}%
\end{pgfscope}%
\begin{pgfscope}%
\pgfpathrectangle{\pgfqpoint{0.100000in}{0.212622in}}{\pgfqpoint{3.696000in}{3.696000in}}%
\pgfusepath{clip}%
\pgfsetrectcap%
\pgfsetroundjoin%
\pgfsetlinewidth{1.505625pt}%
\definecolor{currentstroke}{rgb}{0.121569,0.466667,0.705882}%
\pgfsetstrokecolor{currentstroke}%
\pgfsetdash{}{0pt}%
\pgfpathmoveto{\pgfqpoint{0.849498in}{1.645218in}}%
\pgfpathlineto{\pgfqpoint{1.559298in}{2.233957in}}%
\pgfpathlineto{\pgfqpoint{2.646158in}{1.089674in}}%
\pgfpathlineto{\pgfqpoint{0.849498in}{1.645218in}}%
\pgfusepath{stroke}%
\end{pgfscope}%
\begin{pgfscope}%
\pgfpathrectangle{\pgfqpoint{0.100000in}{0.212622in}}{\pgfqpoint{3.696000in}{3.696000in}}%
\pgfusepath{clip}%
\pgfsetbuttcap%
\pgfsetroundjoin%
\definecolor{currentfill}{rgb}{0.121569,0.466667,0.705882}%
\pgfsetfillcolor{currentfill}%
\pgfsetfillopacity{0.300000}%
\pgfsetlinewidth{1.003750pt}%
\definecolor{currentstroke}{rgb}{0.121569,0.466667,0.705882}%
\pgfsetstrokecolor{currentstroke}%
\pgfsetstrokeopacity{0.300000}%
\pgfsetdash{}{0pt}%
\pgfpathmoveto{\pgfqpoint{1.904781in}{2.063880in}}%
\pgfpathcurveto{\pgfqpoint{1.913018in}{2.063880in}}{\pgfqpoint{1.920918in}{2.067153in}}{\pgfqpoint{1.926742in}{2.072977in}}%
\pgfpathcurveto{\pgfqpoint{1.932566in}{2.078801in}}{\pgfqpoint{1.935838in}{2.086701in}}{\pgfqpoint{1.935838in}{2.094937in}}%
\pgfpathcurveto{\pgfqpoint{1.935838in}{2.103173in}}{\pgfqpoint{1.932566in}{2.111073in}}{\pgfqpoint{1.926742in}{2.116897in}}%
\pgfpathcurveto{\pgfqpoint{1.920918in}{2.122721in}}{\pgfqpoint{1.913018in}{2.125993in}}{\pgfqpoint{1.904781in}{2.125993in}}%
\pgfpathcurveto{\pgfqpoint{1.896545in}{2.125993in}}{\pgfqpoint{1.888645in}{2.122721in}}{\pgfqpoint{1.882821in}{2.116897in}}%
\pgfpathcurveto{\pgfqpoint{1.876997in}{2.111073in}}{\pgfqpoint{1.873725in}{2.103173in}}{\pgfqpoint{1.873725in}{2.094937in}}%
\pgfpathcurveto{\pgfqpoint{1.873725in}{2.086701in}}{\pgfqpoint{1.876997in}{2.078801in}}{\pgfqpoint{1.882821in}{2.072977in}}%
\pgfpathcurveto{\pgfqpoint{1.888645in}{2.067153in}}{\pgfqpoint{1.896545in}{2.063880in}}{\pgfqpoint{1.904781in}{2.063880in}}%
\pgfpathclose%
\pgfusepath{stroke,fill}%
\end{pgfscope}%
\begin{pgfscope}%
\pgfpathrectangle{\pgfqpoint{0.100000in}{0.212622in}}{\pgfqpoint{3.696000in}{3.696000in}}%
\pgfusepath{clip}%
\pgfsetbuttcap%
\pgfsetroundjoin%
\definecolor{currentfill}{rgb}{0.121569,0.466667,0.705882}%
\pgfsetfillcolor{currentfill}%
\pgfsetfillopacity{0.300110}%
\pgfsetlinewidth{1.003750pt}%
\definecolor{currentstroke}{rgb}{0.121569,0.466667,0.705882}%
\pgfsetstrokecolor{currentstroke}%
\pgfsetstrokeopacity{0.300110}%
\pgfsetdash{}{0pt}%
\pgfpathmoveto{\pgfqpoint{1.893749in}{2.066736in}}%
\pgfpathcurveto{\pgfqpoint{1.901985in}{2.066736in}}{\pgfqpoint{1.909885in}{2.070009in}}{\pgfqpoint{1.915709in}{2.075832in}}%
\pgfpathcurveto{\pgfqpoint{1.921533in}{2.081656in}}{\pgfqpoint{1.924805in}{2.089556in}}{\pgfqpoint{1.924805in}{2.097793in}}%
\pgfpathcurveto{\pgfqpoint{1.924805in}{2.106029in}}{\pgfqpoint{1.921533in}{2.113929in}}{\pgfqpoint{1.915709in}{2.119753in}}%
\pgfpathcurveto{\pgfqpoint{1.909885in}{2.125577in}}{\pgfqpoint{1.901985in}{2.128849in}}{\pgfqpoint{1.893749in}{2.128849in}}%
\pgfpathcurveto{\pgfqpoint{1.885512in}{2.128849in}}{\pgfqpoint{1.877612in}{2.125577in}}{\pgfqpoint{1.871788in}{2.119753in}}%
\pgfpathcurveto{\pgfqpoint{1.865964in}{2.113929in}}{\pgfqpoint{1.862692in}{2.106029in}}{\pgfqpoint{1.862692in}{2.097793in}}%
\pgfpathcurveto{\pgfqpoint{1.862692in}{2.089556in}}{\pgfqpoint{1.865964in}{2.081656in}}{\pgfqpoint{1.871788in}{2.075832in}}%
\pgfpathcurveto{\pgfqpoint{1.877612in}{2.070009in}}{\pgfqpoint{1.885512in}{2.066736in}}{\pgfqpoint{1.893749in}{2.066736in}}%
\pgfpathclose%
\pgfusepath{stroke,fill}%
\end{pgfscope}%
\begin{pgfscope}%
\pgfpathrectangle{\pgfqpoint{0.100000in}{0.212622in}}{\pgfqpoint{3.696000in}{3.696000in}}%
\pgfusepath{clip}%
\pgfsetbuttcap%
\pgfsetroundjoin%
\definecolor{currentfill}{rgb}{0.121569,0.466667,0.705882}%
\pgfsetfillcolor{currentfill}%
\pgfsetfillopacity{0.300117}%
\pgfsetlinewidth{1.003750pt}%
\definecolor{currentstroke}{rgb}{0.121569,0.466667,0.705882}%
\pgfsetstrokecolor{currentstroke}%
\pgfsetstrokeopacity{0.300117}%
\pgfsetdash{}{0pt}%
\pgfpathmoveto{\pgfqpoint{1.910859in}{2.063083in}}%
\pgfpathcurveto{\pgfqpoint{1.919096in}{2.063083in}}{\pgfqpoint{1.926996in}{2.066355in}}{\pgfqpoint{1.932820in}{2.072179in}}%
\pgfpathcurveto{\pgfqpoint{1.938643in}{2.078003in}}{\pgfqpoint{1.941916in}{2.085903in}}{\pgfqpoint{1.941916in}{2.094140in}}%
\pgfpathcurveto{\pgfqpoint{1.941916in}{2.102376in}}{\pgfqpoint{1.938643in}{2.110276in}}{\pgfqpoint{1.932820in}{2.116100in}}%
\pgfpathcurveto{\pgfqpoint{1.926996in}{2.121924in}}{\pgfqpoint{1.919096in}{2.125196in}}{\pgfqpoint{1.910859in}{2.125196in}}%
\pgfpathcurveto{\pgfqpoint{1.902623in}{2.125196in}}{\pgfqpoint{1.894723in}{2.121924in}}{\pgfqpoint{1.888899in}{2.116100in}}%
\pgfpathcurveto{\pgfqpoint{1.883075in}{2.110276in}}{\pgfqpoint{1.879803in}{2.102376in}}{\pgfqpoint{1.879803in}{2.094140in}}%
\pgfpathcurveto{\pgfqpoint{1.879803in}{2.085903in}}{\pgfqpoint{1.883075in}{2.078003in}}{\pgfqpoint{1.888899in}{2.072179in}}%
\pgfpathcurveto{\pgfqpoint{1.894723in}{2.066355in}}{\pgfqpoint{1.902623in}{2.063083in}}{\pgfqpoint{1.910859in}{2.063083in}}%
\pgfpathclose%
\pgfusepath{stroke,fill}%
\end{pgfscope}%
\begin{pgfscope}%
\pgfpathrectangle{\pgfqpoint{0.100000in}{0.212622in}}{\pgfqpoint{3.696000in}{3.696000in}}%
\pgfusepath{clip}%
\pgfsetbuttcap%
\pgfsetroundjoin%
\definecolor{currentfill}{rgb}{0.121569,0.466667,0.705882}%
\pgfsetfillcolor{currentfill}%
\pgfsetfillopacity{0.300145}%
\pgfsetlinewidth{1.003750pt}%
\definecolor{currentstroke}{rgb}{0.121569,0.466667,0.705882}%
\pgfsetstrokecolor{currentstroke}%
\pgfsetstrokeopacity{0.300145}%
\pgfsetdash{}{0pt}%
\pgfpathmoveto{\pgfqpoint{1.911745in}{2.062750in}}%
\pgfpathcurveto{\pgfqpoint{1.919981in}{2.062750in}}{\pgfqpoint{1.927881in}{2.066022in}}{\pgfqpoint{1.933705in}{2.071846in}}%
\pgfpathcurveto{\pgfqpoint{1.939529in}{2.077670in}}{\pgfqpoint{1.942801in}{2.085570in}}{\pgfqpoint{1.942801in}{2.093806in}}%
\pgfpathcurveto{\pgfqpoint{1.942801in}{2.102042in}}{\pgfqpoint{1.939529in}{2.109943in}}{\pgfqpoint{1.933705in}{2.115766in}}%
\pgfpathcurveto{\pgfqpoint{1.927881in}{2.121590in}}{\pgfqpoint{1.919981in}{2.124863in}}{\pgfqpoint{1.911745in}{2.124863in}}%
\pgfpathcurveto{\pgfqpoint{1.903508in}{2.124863in}}{\pgfqpoint{1.895608in}{2.121590in}}{\pgfqpoint{1.889785in}{2.115766in}}%
\pgfpathcurveto{\pgfqpoint{1.883961in}{2.109943in}}{\pgfqpoint{1.880688in}{2.102042in}}{\pgfqpoint{1.880688in}{2.093806in}}%
\pgfpathcurveto{\pgfqpoint{1.880688in}{2.085570in}}{\pgfqpoint{1.883961in}{2.077670in}}{\pgfqpoint{1.889785in}{2.071846in}}%
\pgfpathcurveto{\pgfqpoint{1.895608in}{2.066022in}}{\pgfqpoint{1.903508in}{2.062750in}}{\pgfqpoint{1.911745in}{2.062750in}}%
\pgfpathclose%
\pgfusepath{stroke,fill}%
\end{pgfscope}%
\begin{pgfscope}%
\pgfpathrectangle{\pgfqpoint{0.100000in}{0.212622in}}{\pgfqpoint{3.696000in}{3.696000in}}%
\pgfusepath{clip}%
\pgfsetbuttcap%
\pgfsetroundjoin%
\definecolor{currentfill}{rgb}{0.121569,0.466667,0.705882}%
\pgfsetfillcolor{currentfill}%
\pgfsetfillopacity{0.300212}%
\pgfsetlinewidth{1.003750pt}%
\definecolor{currentstroke}{rgb}{0.121569,0.466667,0.705882}%
\pgfsetstrokecolor{currentstroke}%
\pgfsetstrokeopacity{0.300212}%
\pgfsetdash{}{0pt}%
\pgfpathmoveto{\pgfqpoint{1.913353in}{2.062476in}}%
\pgfpathcurveto{\pgfqpoint{1.921590in}{2.062476in}}{\pgfqpoint{1.929490in}{2.065748in}}{\pgfqpoint{1.935314in}{2.071572in}}%
\pgfpathcurveto{\pgfqpoint{1.941138in}{2.077396in}}{\pgfqpoint{1.944410in}{2.085296in}}{\pgfqpoint{1.944410in}{2.093533in}}%
\pgfpathcurveto{\pgfqpoint{1.944410in}{2.101769in}}{\pgfqpoint{1.941138in}{2.109669in}}{\pgfqpoint{1.935314in}{2.115493in}}%
\pgfpathcurveto{\pgfqpoint{1.929490in}{2.121317in}}{\pgfqpoint{1.921590in}{2.124589in}}{\pgfqpoint{1.913353in}{2.124589in}}%
\pgfpathcurveto{\pgfqpoint{1.905117in}{2.124589in}}{\pgfqpoint{1.897217in}{2.121317in}}{\pgfqpoint{1.891393in}{2.115493in}}%
\pgfpathcurveto{\pgfqpoint{1.885569in}{2.109669in}}{\pgfqpoint{1.882297in}{2.101769in}}{\pgfqpoint{1.882297in}{2.093533in}}%
\pgfpathcurveto{\pgfqpoint{1.882297in}{2.085296in}}{\pgfqpoint{1.885569in}{2.077396in}}{\pgfqpoint{1.891393in}{2.071572in}}%
\pgfpathcurveto{\pgfqpoint{1.897217in}{2.065748in}}{\pgfqpoint{1.905117in}{2.062476in}}{\pgfqpoint{1.913353in}{2.062476in}}%
\pgfpathclose%
\pgfusepath{stroke,fill}%
\end{pgfscope}%
\begin{pgfscope}%
\pgfpathrectangle{\pgfqpoint{0.100000in}{0.212622in}}{\pgfqpoint{3.696000in}{3.696000in}}%
\pgfusepath{clip}%
\pgfsetbuttcap%
\pgfsetroundjoin%
\definecolor{currentfill}{rgb}{0.121569,0.466667,0.705882}%
\pgfsetfillcolor{currentfill}%
\pgfsetfillopacity{0.300277}%
\pgfsetlinewidth{1.003750pt}%
\definecolor{currentstroke}{rgb}{0.121569,0.466667,0.705882}%
\pgfsetstrokecolor{currentstroke}%
\pgfsetstrokeopacity{0.300277}%
\pgfsetdash{}{0pt}%
\pgfpathmoveto{\pgfqpoint{1.888138in}{2.066794in}}%
\pgfpathcurveto{\pgfqpoint{1.896374in}{2.066794in}}{\pgfqpoint{1.904274in}{2.070067in}}{\pgfqpoint{1.910098in}{2.075891in}}%
\pgfpathcurveto{\pgfqpoint{1.915922in}{2.081714in}}{\pgfqpoint{1.919194in}{2.089615in}}{\pgfqpoint{1.919194in}{2.097851in}}%
\pgfpathcurveto{\pgfqpoint{1.919194in}{2.106087in}}{\pgfqpoint{1.915922in}{2.113987in}}{\pgfqpoint{1.910098in}{2.119811in}}%
\pgfpathcurveto{\pgfqpoint{1.904274in}{2.125635in}}{\pgfqpoint{1.896374in}{2.128907in}}{\pgfqpoint{1.888138in}{2.128907in}}%
\pgfpathcurveto{\pgfqpoint{1.879901in}{2.128907in}}{\pgfqpoint{1.872001in}{2.125635in}}{\pgfqpoint{1.866177in}{2.119811in}}%
\pgfpathcurveto{\pgfqpoint{1.860353in}{2.113987in}}{\pgfqpoint{1.857081in}{2.106087in}}{\pgfqpoint{1.857081in}{2.097851in}}%
\pgfpathcurveto{\pgfqpoint{1.857081in}{2.089615in}}{\pgfqpoint{1.860353in}{2.081714in}}{\pgfqpoint{1.866177in}{2.075891in}}%
\pgfpathcurveto{\pgfqpoint{1.872001in}{2.070067in}}{\pgfqpoint{1.879901in}{2.066794in}}{\pgfqpoint{1.888138in}{2.066794in}}%
\pgfpathclose%
\pgfusepath{stroke,fill}%
\end{pgfscope}%
\begin{pgfscope}%
\pgfpathrectangle{\pgfqpoint{0.100000in}{0.212622in}}{\pgfqpoint{3.696000in}{3.696000in}}%
\pgfusepath{clip}%
\pgfsetbuttcap%
\pgfsetroundjoin%
\definecolor{currentfill}{rgb}{0.121569,0.466667,0.705882}%
\pgfsetfillcolor{currentfill}%
\pgfsetfillopacity{0.300362}%
\pgfsetlinewidth{1.003750pt}%
\definecolor{currentstroke}{rgb}{0.121569,0.466667,0.705882}%
\pgfsetstrokecolor{currentstroke}%
\pgfsetstrokeopacity{0.300362}%
\pgfsetdash{}{0pt}%
\pgfpathmoveto{\pgfqpoint{1.916129in}{2.060586in}}%
\pgfpathcurveto{\pgfqpoint{1.924365in}{2.060586in}}{\pgfqpoint{1.932265in}{2.063858in}}{\pgfqpoint{1.938089in}{2.069682in}}%
\pgfpathcurveto{\pgfqpoint{1.943913in}{2.075506in}}{\pgfqpoint{1.947185in}{2.083406in}}{\pgfqpoint{1.947185in}{2.091642in}}%
\pgfpathcurveto{\pgfqpoint{1.947185in}{2.099878in}}{\pgfqpoint{1.943913in}{2.107778in}}{\pgfqpoint{1.938089in}{2.113602in}}%
\pgfpathcurveto{\pgfqpoint{1.932265in}{2.119426in}}{\pgfqpoint{1.924365in}{2.122699in}}{\pgfqpoint{1.916129in}{2.122699in}}%
\pgfpathcurveto{\pgfqpoint{1.907892in}{2.122699in}}{\pgfqpoint{1.899992in}{2.119426in}}{\pgfqpoint{1.894168in}{2.113602in}}%
\pgfpathcurveto{\pgfqpoint{1.888344in}{2.107778in}}{\pgfqpoint{1.885072in}{2.099878in}}{\pgfqpoint{1.885072in}{2.091642in}}%
\pgfpathcurveto{\pgfqpoint{1.885072in}{2.083406in}}{\pgfqpoint{1.888344in}{2.075506in}}{\pgfqpoint{1.894168in}{2.069682in}}%
\pgfpathcurveto{\pgfqpoint{1.899992in}{2.063858in}}{\pgfqpoint{1.907892in}{2.060586in}}{\pgfqpoint{1.916129in}{2.060586in}}%
\pgfpathclose%
\pgfusepath{stroke,fill}%
\end{pgfscope}%
\begin{pgfscope}%
\pgfpathrectangle{\pgfqpoint{0.100000in}{0.212622in}}{\pgfqpoint{3.696000in}{3.696000in}}%
\pgfusepath{clip}%
\pgfsetbuttcap%
\pgfsetroundjoin%
\definecolor{currentfill}{rgb}{0.121569,0.466667,0.705882}%
\pgfsetfillcolor{currentfill}%
\pgfsetfillopacity{0.300363}%
\pgfsetlinewidth{1.003750pt}%
\definecolor{currentstroke}{rgb}{0.121569,0.466667,0.705882}%
\pgfsetstrokecolor{currentstroke}%
\pgfsetstrokeopacity{0.300363}%
\pgfsetdash{}{0pt}%
\pgfpathmoveto{\pgfqpoint{1.884851in}{2.067702in}}%
\pgfpathcurveto{\pgfqpoint{1.893087in}{2.067702in}}{\pgfqpoint{1.900987in}{2.070975in}}{\pgfqpoint{1.906811in}{2.076799in}}%
\pgfpathcurveto{\pgfqpoint{1.912635in}{2.082623in}}{\pgfqpoint{1.915907in}{2.090523in}}{\pgfqpoint{1.915907in}{2.098759in}}%
\pgfpathcurveto{\pgfqpoint{1.915907in}{2.106995in}}{\pgfqpoint{1.912635in}{2.114895in}}{\pgfqpoint{1.906811in}{2.120719in}}%
\pgfpathcurveto{\pgfqpoint{1.900987in}{2.126543in}}{\pgfqpoint{1.893087in}{2.129815in}}{\pgfqpoint{1.884851in}{2.129815in}}%
\pgfpathcurveto{\pgfqpoint{1.876615in}{2.129815in}}{\pgfqpoint{1.868715in}{2.126543in}}{\pgfqpoint{1.862891in}{2.120719in}}%
\pgfpathcurveto{\pgfqpoint{1.857067in}{2.114895in}}{\pgfqpoint{1.853794in}{2.106995in}}{\pgfqpoint{1.853794in}{2.098759in}}%
\pgfpathcurveto{\pgfqpoint{1.853794in}{2.090523in}}{\pgfqpoint{1.857067in}{2.082623in}}{\pgfqpoint{1.862891in}{2.076799in}}%
\pgfpathcurveto{\pgfqpoint{1.868715in}{2.070975in}}{\pgfqpoint{1.876615in}{2.067702in}}{\pgfqpoint{1.884851in}{2.067702in}}%
\pgfpathclose%
\pgfusepath{stroke,fill}%
\end{pgfscope}%
\begin{pgfscope}%
\pgfpathrectangle{\pgfqpoint{0.100000in}{0.212622in}}{\pgfqpoint{3.696000in}{3.696000in}}%
\pgfusepath{clip}%
\pgfsetbuttcap%
\pgfsetroundjoin%
\definecolor{currentfill}{rgb}{0.121569,0.466667,0.705882}%
\pgfsetfillcolor{currentfill}%
\pgfsetfillopacity{0.300671}%
\pgfsetlinewidth{1.003750pt}%
\definecolor{currentstroke}{rgb}{0.121569,0.466667,0.705882}%
\pgfsetstrokecolor{currentstroke}%
\pgfsetstrokeopacity{0.300671}%
\pgfsetdash{}{0pt}%
\pgfpathmoveto{\pgfqpoint{1.878280in}{2.067405in}}%
\pgfpathcurveto{\pgfqpoint{1.886517in}{2.067405in}}{\pgfqpoint{1.894417in}{2.070678in}}{\pgfqpoint{1.900241in}{2.076502in}}%
\pgfpathcurveto{\pgfqpoint{1.906065in}{2.082326in}}{\pgfqpoint{1.909337in}{2.090226in}}{\pgfqpoint{1.909337in}{2.098462in}}%
\pgfpathcurveto{\pgfqpoint{1.909337in}{2.106698in}}{\pgfqpoint{1.906065in}{2.114598in}}{\pgfqpoint{1.900241in}{2.120422in}}%
\pgfpathcurveto{\pgfqpoint{1.894417in}{2.126246in}}{\pgfqpoint{1.886517in}{2.129518in}}{\pgfqpoint{1.878280in}{2.129518in}}%
\pgfpathcurveto{\pgfqpoint{1.870044in}{2.129518in}}{\pgfqpoint{1.862144in}{2.126246in}}{\pgfqpoint{1.856320in}{2.120422in}}%
\pgfpathcurveto{\pgfqpoint{1.850496in}{2.114598in}}{\pgfqpoint{1.847224in}{2.106698in}}{\pgfqpoint{1.847224in}{2.098462in}}%
\pgfpathcurveto{\pgfqpoint{1.847224in}{2.090226in}}{\pgfqpoint{1.850496in}{2.082326in}}{\pgfqpoint{1.856320in}{2.076502in}}%
\pgfpathcurveto{\pgfqpoint{1.862144in}{2.070678in}}{\pgfqpoint{1.870044in}{2.067405in}}{\pgfqpoint{1.878280in}{2.067405in}}%
\pgfpathclose%
\pgfusepath{stroke,fill}%
\end{pgfscope}%
\begin{pgfscope}%
\pgfpathrectangle{\pgfqpoint{0.100000in}{0.212622in}}{\pgfqpoint{3.696000in}{3.696000in}}%
\pgfusepath{clip}%
\pgfsetbuttcap%
\pgfsetroundjoin%
\definecolor{currentfill}{rgb}{0.121569,0.466667,0.705882}%
\pgfsetfillcolor{currentfill}%
\pgfsetfillopacity{0.300933}%
\pgfsetlinewidth{1.003750pt}%
\definecolor{currentstroke}{rgb}{0.121569,0.466667,0.705882}%
\pgfsetstrokecolor{currentstroke}%
\pgfsetstrokeopacity{0.300933}%
\pgfsetdash{}{0pt}%
\pgfpathmoveto{\pgfqpoint{1.921443in}{2.061447in}}%
\pgfpathcurveto{\pgfqpoint{1.929679in}{2.061447in}}{\pgfqpoint{1.937579in}{2.064719in}}{\pgfqpoint{1.943403in}{2.070543in}}%
\pgfpathcurveto{\pgfqpoint{1.949227in}{2.076367in}}{\pgfqpoint{1.952499in}{2.084267in}}{\pgfqpoint{1.952499in}{2.092504in}}%
\pgfpathcurveto{\pgfqpoint{1.952499in}{2.100740in}}{\pgfqpoint{1.949227in}{2.108640in}}{\pgfqpoint{1.943403in}{2.114464in}}%
\pgfpathcurveto{\pgfqpoint{1.937579in}{2.120288in}}{\pgfqpoint{1.929679in}{2.123560in}}{\pgfqpoint{1.921443in}{2.123560in}}%
\pgfpathcurveto{\pgfqpoint{1.913206in}{2.123560in}}{\pgfqpoint{1.905306in}{2.120288in}}{\pgfqpoint{1.899482in}{2.114464in}}%
\pgfpathcurveto{\pgfqpoint{1.893658in}{2.108640in}}{\pgfqpoint{1.890386in}{2.100740in}}{\pgfqpoint{1.890386in}{2.092504in}}%
\pgfpathcurveto{\pgfqpoint{1.890386in}{2.084267in}}{\pgfqpoint{1.893658in}{2.076367in}}{\pgfqpoint{1.899482in}{2.070543in}}%
\pgfpathcurveto{\pgfqpoint{1.905306in}{2.064719in}}{\pgfqpoint{1.913206in}{2.061447in}}{\pgfqpoint{1.921443in}{2.061447in}}%
\pgfpathclose%
\pgfusepath{stroke,fill}%
\end{pgfscope}%
\begin{pgfscope}%
\pgfpathrectangle{\pgfqpoint{0.100000in}{0.212622in}}{\pgfqpoint{3.696000in}{3.696000in}}%
\pgfusepath{clip}%
\pgfsetbuttcap%
\pgfsetroundjoin%
\definecolor{currentfill}{rgb}{0.121569,0.466667,0.705882}%
\pgfsetfillcolor{currentfill}%
\pgfsetfillopacity{0.301068}%
\pgfsetlinewidth{1.003750pt}%
\definecolor{currentstroke}{rgb}{0.121569,0.466667,0.705882}%
\pgfsetstrokecolor{currentstroke}%
\pgfsetstrokeopacity{0.301068}%
\pgfsetdash{}{0pt}%
\pgfpathmoveto{\pgfqpoint{1.866271in}{2.071306in}}%
\pgfpathcurveto{\pgfqpoint{1.874507in}{2.071306in}}{\pgfqpoint{1.882407in}{2.074578in}}{\pgfqpoint{1.888231in}{2.080402in}}%
\pgfpathcurveto{\pgfqpoint{1.894055in}{2.086226in}}{\pgfqpoint{1.897327in}{2.094126in}}{\pgfqpoint{1.897327in}{2.102362in}}%
\pgfpathcurveto{\pgfqpoint{1.897327in}{2.110599in}}{\pgfqpoint{1.894055in}{2.118499in}}{\pgfqpoint{1.888231in}{2.124323in}}%
\pgfpathcurveto{\pgfqpoint{1.882407in}{2.130147in}}{\pgfqpoint{1.874507in}{2.133419in}}{\pgfqpoint{1.866271in}{2.133419in}}%
\pgfpathcurveto{\pgfqpoint{1.858034in}{2.133419in}}{\pgfqpoint{1.850134in}{2.130147in}}{\pgfqpoint{1.844310in}{2.124323in}}%
\pgfpathcurveto{\pgfqpoint{1.838487in}{2.118499in}}{\pgfqpoint{1.835214in}{2.110599in}}{\pgfqpoint{1.835214in}{2.102362in}}%
\pgfpathcurveto{\pgfqpoint{1.835214in}{2.094126in}}{\pgfqpoint{1.838487in}{2.086226in}}{\pgfqpoint{1.844310in}{2.080402in}}%
\pgfpathcurveto{\pgfqpoint{1.850134in}{2.074578in}}{\pgfqpoint{1.858034in}{2.071306in}}{\pgfqpoint{1.866271in}{2.071306in}}%
\pgfpathclose%
\pgfusepath{stroke,fill}%
\end{pgfscope}%
\begin{pgfscope}%
\pgfpathrectangle{\pgfqpoint{0.100000in}{0.212622in}}{\pgfqpoint{3.696000in}{3.696000in}}%
\pgfusepath{clip}%
\pgfsetbuttcap%
\pgfsetroundjoin%
\definecolor{currentfill}{rgb}{0.121569,0.466667,0.705882}%
\pgfsetfillcolor{currentfill}%
\pgfsetfillopacity{0.301455}%
\pgfsetlinewidth{1.003750pt}%
\definecolor{currentstroke}{rgb}{0.121569,0.466667,0.705882}%
\pgfsetstrokecolor{currentstroke}%
\pgfsetstrokeopacity{0.301455}%
\pgfsetdash{}{0pt}%
\pgfpathmoveto{\pgfqpoint{1.860498in}{2.071399in}}%
\pgfpathcurveto{\pgfqpoint{1.868734in}{2.071399in}}{\pgfqpoint{1.876634in}{2.074671in}}{\pgfqpoint{1.882458in}{2.080495in}}%
\pgfpathcurveto{\pgfqpoint{1.888282in}{2.086319in}}{\pgfqpoint{1.891554in}{2.094219in}}{\pgfqpoint{1.891554in}{2.102455in}}%
\pgfpathcurveto{\pgfqpoint{1.891554in}{2.110691in}}{\pgfqpoint{1.888282in}{2.118591in}}{\pgfqpoint{1.882458in}{2.124415in}}%
\pgfpathcurveto{\pgfqpoint{1.876634in}{2.130239in}}{\pgfqpoint{1.868734in}{2.133512in}}{\pgfqpoint{1.860498in}{2.133512in}}%
\pgfpathcurveto{\pgfqpoint{1.852261in}{2.133512in}}{\pgfqpoint{1.844361in}{2.130239in}}{\pgfqpoint{1.838537in}{2.124415in}}%
\pgfpathcurveto{\pgfqpoint{1.832713in}{2.118591in}}{\pgfqpoint{1.829441in}{2.110691in}}{\pgfqpoint{1.829441in}{2.102455in}}%
\pgfpathcurveto{\pgfqpoint{1.829441in}{2.094219in}}{\pgfqpoint{1.832713in}{2.086319in}}{\pgfqpoint{1.838537in}{2.080495in}}%
\pgfpathcurveto{\pgfqpoint{1.844361in}{2.074671in}}{\pgfqpoint{1.852261in}{2.071399in}}{\pgfqpoint{1.860498in}{2.071399in}}%
\pgfpathclose%
\pgfusepath{stroke,fill}%
\end{pgfscope}%
\begin{pgfscope}%
\pgfpathrectangle{\pgfqpoint{0.100000in}{0.212622in}}{\pgfqpoint{3.696000in}{3.696000in}}%
\pgfusepath{clip}%
\pgfsetbuttcap%
\pgfsetroundjoin%
\definecolor{currentfill}{rgb}{0.121569,0.466667,0.705882}%
\pgfsetfillcolor{currentfill}%
\pgfsetfillopacity{0.301605}%
\pgfsetlinewidth{1.003750pt}%
\definecolor{currentstroke}{rgb}{0.121569,0.466667,0.705882}%
\pgfsetstrokecolor{currentstroke}%
\pgfsetstrokeopacity{0.301605}%
\pgfsetdash{}{0pt}%
\pgfpathmoveto{\pgfqpoint{1.857095in}{2.071739in}}%
\pgfpathcurveto{\pgfqpoint{1.865332in}{2.071739in}}{\pgfqpoint{1.873232in}{2.075011in}}{\pgfqpoint{1.879056in}{2.080835in}}%
\pgfpathcurveto{\pgfqpoint{1.884879in}{2.086659in}}{\pgfqpoint{1.888152in}{2.094559in}}{\pgfqpoint{1.888152in}{2.102795in}}%
\pgfpathcurveto{\pgfqpoint{1.888152in}{2.111032in}}{\pgfqpoint{1.884879in}{2.118932in}}{\pgfqpoint{1.879056in}{2.124756in}}%
\pgfpathcurveto{\pgfqpoint{1.873232in}{2.130579in}}{\pgfqpoint{1.865332in}{2.133852in}}{\pgfqpoint{1.857095in}{2.133852in}}%
\pgfpathcurveto{\pgfqpoint{1.848859in}{2.133852in}}{\pgfqpoint{1.840959in}{2.130579in}}{\pgfqpoint{1.835135in}{2.124756in}}%
\pgfpathcurveto{\pgfqpoint{1.829311in}{2.118932in}}{\pgfqpoint{1.826039in}{2.111032in}}{\pgfqpoint{1.826039in}{2.102795in}}%
\pgfpathcurveto{\pgfqpoint{1.826039in}{2.094559in}}{\pgfqpoint{1.829311in}{2.086659in}}{\pgfqpoint{1.835135in}{2.080835in}}%
\pgfpathcurveto{\pgfqpoint{1.840959in}{2.075011in}}{\pgfqpoint{1.848859in}{2.071739in}}{\pgfqpoint{1.857095in}{2.071739in}}%
\pgfpathclose%
\pgfusepath{stroke,fill}%
\end{pgfscope}%
\begin{pgfscope}%
\pgfpathrectangle{\pgfqpoint{0.100000in}{0.212622in}}{\pgfqpoint{3.696000in}{3.696000in}}%
\pgfusepath{clip}%
\pgfsetbuttcap%
\pgfsetroundjoin%
\definecolor{currentfill}{rgb}{0.121569,0.466667,0.705882}%
\pgfsetfillcolor{currentfill}%
\pgfsetfillopacity{0.301728}%
\pgfsetlinewidth{1.003750pt}%
\definecolor{currentstroke}{rgb}{0.121569,0.466667,0.705882}%
\pgfsetstrokecolor{currentstroke}%
\pgfsetstrokeopacity{0.301728}%
\pgfsetdash{}{0pt}%
\pgfpathmoveto{\pgfqpoint{1.855201in}{2.072305in}}%
\pgfpathcurveto{\pgfqpoint{1.863437in}{2.072305in}}{\pgfqpoint{1.871337in}{2.075577in}}{\pgfqpoint{1.877161in}{2.081401in}}%
\pgfpathcurveto{\pgfqpoint{1.882985in}{2.087225in}}{\pgfqpoint{1.886257in}{2.095125in}}{\pgfqpoint{1.886257in}{2.103362in}}%
\pgfpathcurveto{\pgfqpoint{1.886257in}{2.111598in}}{\pgfqpoint{1.882985in}{2.119498in}}{\pgfqpoint{1.877161in}{2.125322in}}%
\pgfpathcurveto{\pgfqpoint{1.871337in}{2.131146in}}{\pgfqpoint{1.863437in}{2.134418in}}{\pgfqpoint{1.855201in}{2.134418in}}%
\pgfpathcurveto{\pgfqpoint{1.846965in}{2.134418in}}{\pgfqpoint{1.839065in}{2.131146in}}{\pgfqpoint{1.833241in}{2.125322in}}%
\pgfpathcurveto{\pgfqpoint{1.827417in}{2.119498in}}{\pgfqpoint{1.824144in}{2.111598in}}{\pgfqpoint{1.824144in}{2.103362in}}%
\pgfpathcurveto{\pgfqpoint{1.824144in}{2.095125in}}{\pgfqpoint{1.827417in}{2.087225in}}{\pgfqpoint{1.833241in}{2.081401in}}%
\pgfpathcurveto{\pgfqpoint{1.839065in}{2.075577in}}{\pgfqpoint{1.846965in}{2.072305in}}{\pgfqpoint{1.855201in}{2.072305in}}%
\pgfpathclose%
\pgfusepath{stroke,fill}%
\end{pgfscope}%
\begin{pgfscope}%
\pgfpathrectangle{\pgfqpoint{0.100000in}{0.212622in}}{\pgfqpoint{3.696000in}{3.696000in}}%
\pgfusepath{clip}%
\pgfsetbuttcap%
\pgfsetroundjoin%
\definecolor{currentfill}{rgb}{0.121569,0.466667,0.705882}%
\pgfsetfillcolor{currentfill}%
\pgfsetfillopacity{0.301936}%
\pgfsetlinewidth{1.003750pt}%
\definecolor{currentstroke}{rgb}{0.121569,0.466667,0.705882}%
\pgfsetstrokecolor{currentstroke}%
\pgfsetstrokeopacity{0.301936}%
\pgfsetdash{}{0pt}%
\pgfpathmoveto{\pgfqpoint{1.929628in}{2.056182in}}%
\pgfpathcurveto{\pgfqpoint{1.937865in}{2.056182in}}{\pgfqpoint{1.945765in}{2.059454in}}{\pgfqpoint{1.951589in}{2.065278in}}%
\pgfpathcurveto{\pgfqpoint{1.957412in}{2.071102in}}{\pgfqpoint{1.960685in}{2.079002in}}{\pgfqpoint{1.960685in}{2.087239in}}%
\pgfpathcurveto{\pgfqpoint{1.960685in}{2.095475in}}{\pgfqpoint{1.957412in}{2.103375in}}{\pgfqpoint{1.951589in}{2.109199in}}%
\pgfpathcurveto{\pgfqpoint{1.945765in}{2.115023in}}{\pgfqpoint{1.937865in}{2.118295in}}{\pgfqpoint{1.929628in}{2.118295in}}%
\pgfpathcurveto{\pgfqpoint{1.921392in}{2.118295in}}{\pgfqpoint{1.913492in}{2.115023in}}{\pgfqpoint{1.907668in}{2.109199in}}%
\pgfpathcurveto{\pgfqpoint{1.901844in}{2.103375in}}{\pgfqpoint{1.898572in}{2.095475in}}{\pgfqpoint{1.898572in}{2.087239in}}%
\pgfpathcurveto{\pgfqpoint{1.898572in}{2.079002in}}{\pgfqpoint{1.901844in}{2.071102in}}{\pgfqpoint{1.907668in}{2.065278in}}%
\pgfpathcurveto{\pgfqpoint{1.913492in}{2.059454in}}{\pgfqpoint{1.921392in}{2.056182in}}{\pgfqpoint{1.929628in}{2.056182in}}%
\pgfpathclose%
\pgfusepath{stroke,fill}%
\end{pgfscope}%
\begin{pgfscope}%
\pgfpathrectangle{\pgfqpoint{0.100000in}{0.212622in}}{\pgfqpoint{3.696000in}{3.696000in}}%
\pgfusepath{clip}%
\pgfsetbuttcap%
\pgfsetroundjoin%
\definecolor{currentfill}{rgb}{0.121569,0.466667,0.705882}%
\pgfsetfillcolor{currentfill}%
\pgfsetfillopacity{0.301962}%
\pgfsetlinewidth{1.003750pt}%
\definecolor{currentstroke}{rgb}{0.121569,0.466667,0.705882}%
\pgfsetstrokecolor{currentstroke}%
\pgfsetstrokeopacity{0.301962}%
\pgfsetdash{}{0pt}%
\pgfpathmoveto{\pgfqpoint{1.850671in}{2.070892in}}%
\pgfpathcurveto{\pgfqpoint{1.858907in}{2.070892in}}{\pgfqpoint{1.866807in}{2.074164in}}{\pgfqpoint{1.872631in}{2.079988in}}%
\pgfpathcurveto{\pgfqpoint{1.878455in}{2.085812in}}{\pgfqpoint{1.881727in}{2.093712in}}{\pgfqpoint{1.881727in}{2.101948in}}%
\pgfpathcurveto{\pgfqpoint{1.881727in}{2.110185in}}{\pgfqpoint{1.878455in}{2.118085in}}{\pgfqpoint{1.872631in}{2.123909in}}%
\pgfpathcurveto{\pgfqpoint{1.866807in}{2.129733in}}{\pgfqpoint{1.858907in}{2.133005in}}{\pgfqpoint{1.850671in}{2.133005in}}%
\pgfpathcurveto{\pgfqpoint{1.842435in}{2.133005in}}{\pgfqpoint{1.834535in}{2.129733in}}{\pgfqpoint{1.828711in}{2.123909in}}%
\pgfpathcurveto{\pgfqpoint{1.822887in}{2.118085in}}{\pgfqpoint{1.819614in}{2.110185in}}{\pgfqpoint{1.819614in}{2.101948in}}%
\pgfpathcurveto{\pgfqpoint{1.819614in}{2.093712in}}{\pgfqpoint{1.822887in}{2.085812in}}{\pgfqpoint{1.828711in}{2.079988in}}%
\pgfpathcurveto{\pgfqpoint{1.834535in}{2.074164in}}{\pgfqpoint{1.842435in}{2.070892in}}{\pgfqpoint{1.850671in}{2.070892in}}%
\pgfpathclose%
\pgfusepath{stroke,fill}%
\end{pgfscope}%
\begin{pgfscope}%
\pgfpathrectangle{\pgfqpoint{0.100000in}{0.212622in}}{\pgfqpoint{3.696000in}{3.696000in}}%
\pgfusepath{clip}%
\pgfsetbuttcap%
\pgfsetroundjoin%
\definecolor{currentfill}{rgb}{0.121569,0.466667,0.705882}%
\pgfsetfillcolor{currentfill}%
\pgfsetfillopacity{0.302498}%
\pgfsetlinewidth{1.003750pt}%
\definecolor{currentstroke}{rgb}{0.121569,0.466667,0.705882}%
\pgfsetstrokecolor{currentstroke}%
\pgfsetstrokeopacity{0.302498}%
\pgfsetdash{}{0pt}%
\pgfpathmoveto{\pgfqpoint{1.841717in}{2.076355in}}%
\pgfpathcurveto{\pgfqpoint{1.849954in}{2.076355in}}{\pgfqpoint{1.857854in}{2.079627in}}{\pgfqpoint{1.863678in}{2.085451in}}%
\pgfpathcurveto{\pgfqpoint{1.869501in}{2.091275in}}{\pgfqpoint{1.872774in}{2.099175in}}{\pgfqpoint{1.872774in}{2.107411in}}%
\pgfpathcurveto{\pgfqpoint{1.872774in}{2.115648in}}{\pgfqpoint{1.869501in}{2.123548in}}{\pgfqpoint{1.863678in}{2.129372in}}%
\pgfpathcurveto{\pgfqpoint{1.857854in}{2.135196in}}{\pgfqpoint{1.849954in}{2.138468in}}{\pgfqpoint{1.841717in}{2.138468in}}%
\pgfpathcurveto{\pgfqpoint{1.833481in}{2.138468in}}{\pgfqpoint{1.825581in}{2.135196in}}{\pgfqpoint{1.819757in}{2.129372in}}%
\pgfpathcurveto{\pgfqpoint{1.813933in}{2.123548in}}{\pgfqpoint{1.810661in}{2.115648in}}{\pgfqpoint{1.810661in}{2.107411in}}%
\pgfpathcurveto{\pgfqpoint{1.810661in}{2.099175in}}{\pgfqpoint{1.813933in}{2.091275in}}{\pgfqpoint{1.819757in}{2.085451in}}%
\pgfpathcurveto{\pgfqpoint{1.825581in}{2.079627in}}{\pgfqpoint{1.833481in}{2.076355in}}{\pgfqpoint{1.841717in}{2.076355in}}%
\pgfpathclose%
\pgfusepath{stroke,fill}%
\end{pgfscope}%
\begin{pgfscope}%
\pgfpathrectangle{\pgfqpoint{0.100000in}{0.212622in}}{\pgfqpoint{3.696000in}{3.696000in}}%
\pgfusepath{clip}%
\pgfsetbuttcap%
\pgfsetroundjoin%
\definecolor{currentfill}{rgb}{0.121569,0.466667,0.705882}%
\pgfsetfillcolor{currentfill}%
\pgfsetfillopacity{0.303397}%
\pgfsetlinewidth{1.003750pt}%
\definecolor{currentstroke}{rgb}{0.121569,0.466667,0.705882}%
\pgfsetstrokecolor{currentstroke}%
\pgfsetstrokeopacity{0.303397}%
\pgfsetdash{}{0pt}%
\pgfpathmoveto{\pgfqpoint{1.830174in}{2.074156in}}%
\pgfpathcurveto{\pgfqpoint{1.838410in}{2.074156in}}{\pgfqpoint{1.846311in}{2.077428in}}{\pgfqpoint{1.852134in}{2.083252in}}%
\pgfpathcurveto{\pgfqpoint{1.857958in}{2.089076in}}{\pgfqpoint{1.861231in}{2.096976in}}{\pgfqpoint{1.861231in}{2.105213in}}%
\pgfpathcurveto{\pgfqpoint{1.861231in}{2.113449in}}{\pgfqpoint{1.857958in}{2.121349in}}{\pgfqpoint{1.852134in}{2.127173in}}%
\pgfpathcurveto{\pgfqpoint{1.846311in}{2.132997in}}{\pgfqpoint{1.838410in}{2.136269in}}{\pgfqpoint{1.830174in}{2.136269in}}%
\pgfpathcurveto{\pgfqpoint{1.821938in}{2.136269in}}{\pgfqpoint{1.814038in}{2.132997in}}{\pgfqpoint{1.808214in}{2.127173in}}%
\pgfpathcurveto{\pgfqpoint{1.802390in}{2.121349in}}{\pgfqpoint{1.799118in}{2.113449in}}{\pgfqpoint{1.799118in}{2.105213in}}%
\pgfpathcurveto{\pgfqpoint{1.799118in}{2.096976in}}{\pgfqpoint{1.802390in}{2.089076in}}{\pgfqpoint{1.808214in}{2.083252in}}%
\pgfpathcurveto{\pgfqpoint{1.814038in}{2.077428in}}{\pgfqpoint{1.821938in}{2.074156in}}{\pgfqpoint{1.830174in}{2.074156in}}%
\pgfpathclose%
\pgfusepath{stroke,fill}%
\end{pgfscope}%
\begin{pgfscope}%
\pgfpathrectangle{\pgfqpoint{0.100000in}{0.212622in}}{\pgfqpoint{3.696000in}{3.696000in}}%
\pgfusepath{clip}%
\pgfsetbuttcap%
\pgfsetroundjoin%
\definecolor{currentfill}{rgb}{0.121569,0.466667,0.705882}%
\pgfsetfillcolor{currentfill}%
\pgfsetfillopacity{0.304332}%
\pgfsetlinewidth{1.003750pt}%
\definecolor{currentstroke}{rgb}{0.121569,0.466667,0.705882}%
\pgfsetstrokecolor{currentstroke}%
\pgfsetstrokeopacity{0.304332}%
\pgfsetdash{}{0pt}%
\pgfpathmoveto{\pgfqpoint{1.813367in}{2.081168in}}%
\pgfpathcurveto{\pgfqpoint{1.821603in}{2.081168in}}{\pgfqpoint{1.829503in}{2.084440in}}{\pgfqpoint{1.835327in}{2.090264in}}%
\pgfpathcurveto{\pgfqpoint{1.841151in}{2.096088in}}{\pgfqpoint{1.844423in}{2.103988in}}{\pgfqpoint{1.844423in}{2.112224in}}%
\pgfpathcurveto{\pgfqpoint{1.844423in}{2.120461in}}{\pgfqpoint{1.841151in}{2.128361in}}{\pgfqpoint{1.835327in}{2.134185in}}%
\pgfpathcurveto{\pgfqpoint{1.829503in}{2.140009in}}{\pgfqpoint{1.821603in}{2.143281in}}{\pgfqpoint{1.813367in}{2.143281in}}%
\pgfpathcurveto{\pgfqpoint{1.805131in}{2.143281in}}{\pgfqpoint{1.797230in}{2.140009in}}{\pgfqpoint{1.791407in}{2.134185in}}%
\pgfpathcurveto{\pgfqpoint{1.785583in}{2.128361in}}{\pgfqpoint{1.782310in}{2.120461in}}{\pgfqpoint{1.782310in}{2.112224in}}%
\pgfpathcurveto{\pgfqpoint{1.782310in}{2.103988in}}{\pgfqpoint{1.785583in}{2.096088in}}{\pgfqpoint{1.791407in}{2.090264in}}%
\pgfpathcurveto{\pgfqpoint{1.797230in}{2.084440in}}{\pgfqpoint{1.805131in}{2.081168in}}{\pgfqpoint{1.813367in}{2.081168in}}%
\pgfpathclose%
\pgfusepath{stroke,fill}%
\end{pgfscope}%
\begin{pgfscope}%
\pgfpathrectangle{\pgfqpoint{0.100000in}{0.212622in}}{\pgfqpoint{3.696000in}{3.696000in}}%
\pgfusepath{clip}%
\pgfsetbuttcap%
\pgfsetroundjoin%
\definecolor{currentfill}{rgb}{0.121569,0.466667,0.705882}%
\pgfsetfillcolor{currentfill}%
\pgfsetfillopacity{0.304977}%
\pgfsetlinewidth{1.003750pt}%
\definecolor{currentstroke}{rgb}{0.121569,0.466667,0.705882}%
\pgfsetstrokecolor{currentstroke}%
\pgfsetstrokeopacity{0.304977}%
\pgfsetdash{}{0pt}%
\pgfpathmoveto{\pgfqpoint{1.805593in}{2.080861in}}%
\pgfpathcurveto{\pgfqpoint{1.813829in}{2.080861in}}{\pgfqpoint{1.821729in}{2.084134in}}{\pgfqpoint{1.827553in}{2.089958in}}%
\pgfpathcurveto{\pgfqpoint{1.833377in}{2.095782in}}{\pgfqpoint{1.836650in}{2.103682in}}{\pgfqpoint{1.836650in}{2.111918in}}%
\pgfpathcurveto{\pgfqpoint{1.836650in}{2.120154in}}{\pgfqpoint{1.833377in}{2.128054in}}{\pgfqpoint{1.827553in}{2.133878in}}%
\pgfpathcurveto{\pgfqpoint{1.821729in}{2.139702in}}{\pgfqpoint{1.813829in}{2.142974in}}{\pgfqpoint{1.805593in}{2.142974in}}%
\pgfpathcurveto{\pgfqpoint{1.797357in}{2.142974in}}{\pgfqpoint{1.789457in}{2.139702in}}{\pgfqpoint{1.783633in}{2.133878in}}%
\pgfpathcurveto{\pgfqpoint{1.777809in}{2.128054in}}{\pgfqpoint{1.774537in}{2.120154in}}{\pgfqpoint{1.774537in}{2.111918in}}%
\pgfpathcurveto{\pgfqpoint{1.774537in}{2.103682in}}{\pgfqpoint{1.777809in}{2.095782in}}{\pgfqpoint{1.783633in}{2.089958in}}%
\pgfpathcurveto{\pgfqpoint{1.789457in}{2.084134in}}{\pgfqpoint{1.797357in}{2.080861in}}{\pgfqpoint{1.805593in}{2.080861in}}%
\pgfpathclose%
\pgfusepath{stroke,fill}%
\end{pgfscope}%
\begin{pgfscope}%
\pgfpathrectangle{\pgfqpoint{0.100000in}{0.212622in}}{\pgfqpoint{3.696000in}{3.696000in}}%
\pgfusepath{clip}%
\pgfsetbuttcap%
\pgfsetroundjoin%
\definecolor{currentfill}{rgb}{0.121569,0.466667,0.705882}%
\pgfsetfillcolor{currentfill}%
\pgfsetfillopacity{0.305244}%
\pgfsetlinewidth{1.003750pt}%
\definecolor{currentstroke}{rgb}{0.121569,0.466667,0.705882}%
\pgfsetstrokecolor{currentstroke}%
\pgfsetstrokeopacity{0.305244}%
\pgfsetdash{}{0pt}%
\pgfpathmoveto{\pgfqpoint{1.800972in}{2.081043in}}%
\pgfpathcurveto{\pgfqpoint{1.809209in}{2.081043in}}{\pgfqpoint{1.817109in}{2.084315in}}{\pgfqpoint{1.822933in}{2.090139in}}%
\pgfpathcurveto{\pgfqpoint{1.828757in}{2.095963in}}{\pgfqpoint{1.832029in}{2.103863in}}{\pgfqpoint{1.832029in}{2.112100in}}%
\pgfpathcurveto{\pgfqpoint{1.832029in}{2.120336in}}{\pgfqpoint{1.828757in}{2.128236in}}{\pgfqpoint{1.822933in}{2.134060in}}%
\pgfpathcurveto{\pgfqpoint{1.817109in}{2.139884in}}{\pgfqpoint{1.809209in}{2.143156in}}{\pgfqpoint{1.800972in}{2.143156in}}%
\pgfpathcurveto{\pgfqpoint{1.792736in}{2.143156in}}{\pgfqpoint{1.784836in}{2.139884in}}{\pgfqpoint{1.779012in}{2.134060in}}%
\pgfpathcurveto{\pgfqpoint{1.773188in}{2.128236in}}{\pgfqpoint{1.769916in}{2.120336in}}{\pgfqpoint{1.769916in}{2.112100in}}%
\pgfpathcurveto{\pgfqpoint{1.769916in}{2.103863in}}{\pgfqpoint{1.773188in}{2.095963in}}{\pgfqpoint{1.779012in}{2.090139in}}%
\pgfpathcurveto{\pgfqpoint{1.784836in}{2.084315in}}{\pgfqpoint{1.792736in}{2.081043in}}{\pgfqpoint{1.800972in}{2.081043in}}%
\pgfpathclose%
\pgfusepath{stroke,fill}%
\end{pgfscope}%
\begin{pgfscope}%
\pgfpathrectangle{\pgfqpoint{0.100000in}{0.212622in}}{\pgfqpoint{3.696000in}{3.696000in}}%
\pgfusepath{clip}%
\pgfsetbuttcap%
\pgfsetroundjoin%
\definecolor{currentfill}{rgb}{0.121569,0.466667,0.705882}%
\pgfsetfillcolor{currentfill}%
\pgfsetfillopacity{0.305545}%
\pgfsetlinewidth{1.003750pt}%
\definecolor{currentstroke}{rgb}{0.121569,0.466667,0.705882}%
\pgfsetstrokecolor{currentstroke}%
\pgfsetstrokeopacity{0.305545}%
\pgfsetdash{}{0pt}%
\pgfpathmoveto{\pgfqpoint{1.945443in}{2.061676in}}%
\pgfpathcurveto{\pgfqpoint{1.953679in}{2.061676in}}{\pgfqpoint{1.961579in}{2.064949in}}{\pgfqpoint{1.967403in}{2.070773in}}%
\pgfpathcurveto{\pgfqpoint{1.973227in}{2.076596in}}{\pgfqpoint{1.976499in}{2.084497in}}{\pgfqpoint{1.976499in}{2.092733in}}%
\pgfpathcurveto{\pgfqpoint{1.976499in}{2.100969in}}{\pgfqpoint{1.973227in}{2.108869in}}{\pgfqpoint{1.967403in}{2.114693in}}%
\pgfpathcurveto{\pgfqpoint{1.961579in}{2.120517in}}{\pgfqpoint{1.953679in}{2.123789in}}{\pgfqpoint{1.945443in}{2.123789in}}%
\pgfpathcurveto{\pgfqpoint{1.937206in}{2.123789in}}{\pgfqpoint{1.929306in}{2.120517in}}{\pgfqpoint{1.923482in}{2.114693in}}%
\pgfpathcurveto{\pgfqpoint{1.917658in}{2.108869in}}{\pgfqpoint{1.914386in}{2.100969in}}{\pgfqpoint{1.914386in}{2.092733in}}%
\pgfpathcurveto{\pgfqpoint{1.914386in}{2.084497in}}{\pgfqpoint{1.917658in}{2.076596in}}{\pgfqpoint{1.923482in}{2.070773in}}%
\pgfpathcurveto{\pgfqpoint{1.929306in}{2.064949in}}{\pgfqpoint{1.937206in}{2.061676in}}{\pgfqpoint{1.945443in}{2.061676in}}%
\pgfpathclose%
\pgfusepath{stroke,fill}%
\end{pgfscope}%
\begin{pgfscope}%
\pgfpathrectangle{\pgfqpoint{0.100000in}{0.212622in}}{\pgfqpoint{3.696000in}{3.696000in}}%
\pgfusepath{clip}%
\pgfsetbuttcap%
\pgfsetroundjoin%
\definecolor{currentfill}{rgb}{0.121569,0.466667,0.705882}%
\pgfsetfillcolor{currentfill}%
\pgfsetfillopacity{0.305792}%
\pgfsetlinewidth{1.003750pt}%
\definecolor{currentstroke}{rgb}{0.121569,0.466667,0.705882}%
\pgfsetstrokecolor{currentstroke}%
\pgfsetstrokeopacity{0.305792}%
\pgfsetdash{}{0pt}%
\pgfpathmoveto{\pgfqpoint{1.794158in}{2.082604in}}%
\pgfpathcurveto{\pgfqpoint{1.802395in}{2.082604in}}{\pgfqpoint{1.810295in}{2.085876in}}{\pgfqpoint{1.816119in}{2.091700in}}%
\pgfpathcurveto{\pgfqpoint{1.821943in}{2.097524in}}{\pgfqpoint{1.825215in}{2.105424in}}{\pgfqpoint{1.825215in}{2.113660in}}%
\pgfpathcurveto{\pgfqpoint{1.825215in}{2.121897in}}{\pgfqpoint{1.821943in}{2.129797in}}{\pgfqpoint{1.816119in}{2.135621in}}%
\pgfpathcurveto{\pgfqpoint{1.810295in}{2.141444in}}{\pgfqpoint{1.802395in}{2.144717in}}{\pgfqpoint{1.794158in}{2.144717in}}%
\pgfpathcurveto{\pgfqpoint{1.785922in}{2.144717in}}{\pgfqpoint{1.778022in}{2.141444in}}{\pgfqpoint{1.772198in}{2.135621in}}%
\pgfpathcurveto{\pgfqpoint{1.766374in}{2.129797in}}{\pgfqpoint{1.763102in}{2.121897in}}{\pgfqpoint{1.763102in}{2.113660in}}%
\pgfpathcurveto{\pgfqpoint{1.763102in}{2.105424in}}{\pgfqpoint{1.766374in}{2.097524in}}{\pgfqpoint{1.772198in}{2.091700in}}%
\pgfpathcurveto{\pgfqpoint{1.778022in}{2.085876in}}{\pgfqpoint{1.785922in}{2.082604in}}{\pgfqpoint{1.794158in}{2.082604in}}%
\pgfpathclose%
\pgfusepath{stroke,fill}%
\end{pgfscope}%
\begin{pgfscope}%
\pgfpathrectangle{\pgfqpoint{0.100000in}{0.212622in}}{\pgfqpoint{3.696000in}{3.696000in}}%
\pgfusepath{clip}%
\pgfsetbuttcap%
\pgfsetroundjoin%
\definecolor{currentfill}{rgb}{0.121569,0.466667,0.705882}%
\pgfsetfillcolor{currentfill}%
\pgfsetfillopacity{0.306026}%
\pgfsetlinewidth{1.003750pt}%
\definecolor{currentstroke}{rgb}{0.121569,0.466667,0.705882}%
\pgfsetstrokecolor{currentstroke}%
\pgfsetstrokeopacity{0.306026}%
\pgfsetdash{}{0pt}%
\pgfpathmoveto{\pgfqpoint{1.790704in}{2.082132in}}%
\pgfpathcurveto{\pgfqpoint{1.798940in}{2.082132in}}{\pgfqpoint{1.806840in}{2.085404in}}{\pgfqpoint{1.812664in}{2.091228in}}%
\pgfpathcurveto{\pgfqpoint{1.818488in}{2.097052in}}{\pgfqpoint{1.821760in}{2.104952in}}{\pgfqpoint{1.821760in}{2.113189in}}%
\pgfpathcurveto{\pgfqpoint{1.821760in}{2.121425in}}{\pgfqpoint{1.818488in}{2.129325in}}{\pgfqpoint{1.812664in}{2.135149in}}%
\pgfpathcurveto{\pgfqpoint{1.806840in}{2.140973in}}{\pgfqpoint{1.798940in}{2.144245in}}{\pgfqpoint{1.790704in}{2.144245in}}%
\pgfpathcurveto{\pgfqpoint{1.782468in}{2.144245in}}{\pgfqpoint{1.774567in}{2.140973in}}{\pgfqpoint{1.768744in}{2.135149in}}%
\pgfpathcurveto{\pgfqpoint{1.762920in}{2.129325in}}{\pgfqpoint{1.759647in}{2.121425in}}{\pgfqpoint{1.759647in}{2.113189in}}%
\pgfpathcurveto{\pgfqpoint{1.759647in}{2.104952in}}{\pgfqpoint{1.762920in}{2.097052in}}{\pgfqpoint{1.768744in}{2.091228in}}%
\pgfpathcurveto{\pgfqpoint{1.774567in}{2.085404in}}{\pgfqpoint{1.782468in}{2.082132in}}{\pgfqpoint{1.790704in}{2.082132in}}%
\pgfpathclose%
\pgfusepath{stroke,fill}%
\end{pgfscope}%
\begin{pgfscope}%
\pgfpathrectangle{\pgfqpoint{0.100000in}{0.212622in}}{\pgfqpoint{3.696000in}{3.696000in}}%
\pgfusepath{clip}%
\pgfsetbuttcap%
\pgfsetroundjoin%
\definecolor{currentfill}{rgb}{0.121569,0.466667,0.705882}%
\pgfsetfillcolor{currentfill}%
\pgfsetfillopacity{0.306565}%
\pgfsetlinewidth{1.003750pt}%
\definecolor{currentstroke}{rgb}{0.121569,0.466667,0.705882}%
\pgfsetstrokecolor{currentstroke}%
\pgfsetstrokeopacity{0.306565}%
\pgfsetdash{}{0pt}%
\pgfpathmoveto{\pgfqpoint{1.784329in}{2.085239in}}%
\pgfpathcurveto{\pgfqpoint{1.792565in}{2.085239in}}{\pgfqpoint{1.800465in}{2.088511in}}{\pgfqpoint{1.806289in}{2.094335in}}%
\pgfpathcurveto{\pgfqpoint{1.812113in}{2.100159in}}{\pgfqpoint{1.815386in}{2.108059in}}{\pgfqpoint{1.815386in}{2.116296in}}%
\pgfpathcurveto{\pgfqpoint{1.815386in}{2.124532in}}{\pgfqpoint{1.812113in}{2.132432in}}{\pgfqpoint{1.806289in}{2.138256in}}%
\pgfpathcurveto{\pgfqpoint{1.800465in}{2.144080in}}{\pgfqpoint{1.792565in}{2.147352in}}{\pgfqpoint{1.784329in}{2.147352in}}%
\pgfpathcurveto{\pgfqpoint{1.776093in}{2.147352in}}{\pgfqpoint{1.768193in}{2.144080in}}{\pgfqpoint{1.762369in}{2.138256in}}%
\pgfpathcurveto{\pgfqpoint{1.756545in}{2.132432in}}{\pgfqpoint{1.753273in}{2.124532in}}{\pgfqpoint{1.753273in}{2.116296in}}%
\pgfpathcurveto{\pgfqpoint{1.753273in}{2.108059in}}{\pgfqpoint{1.756545in}{2.100159in}}{\pgfqpoint{1.762369in}{2.094335in}}%
\pgfpathcurveto{\pgfqpoint{1.768193in}{2.088511in}}{\pgfqpoint{1.776093in}{2.085239in}}{\pgfqpoint{1.784329in}{2.085239in}}%
\pgfpathclose%
\pgfusepath{stroke,fill}%
\end{pgfscope}%
\begin{pgfscope}%
\pgfpathrectangle{\pgfqpoint{0.100000in}{0.212622in}}{\pgfqpoint{3.696000in}{3.696000in}}%
\pgfusepath{clip}%
\pgfsetbuttcap%
\pgfsetroundjoin%
\definecolor{currentfill}{rgb}{0.121569,0.466667,0.705882}%
\pgfsetfillcolor{currentfill}%
\pgfsetfillopacity{0.307538}%
\pgfsetlinewidth{1.003750pt}%
\definecolor{currentstroke}{rgb}{0.121569,0.466667,0.705882}%
\pgfsetstrokecolor{currentstroke}%
\pgfsetstrokeopacity{0.307538}%
\pgfsetdash{}{0pt}%
\pgfpathmoveto{\pgfqpoint{1.775861in}{2.084367in}}%
\pgfpathcurveto{\pgfqpoint{1.784098in}{2.084367in}}{\pgfqpoint{1.791998in}{2.087639in}}{\pgfqpoint{1.797822in}{2.093463in}}%
\pgfpathcurveto{\pgfqpoint{1.803646in}{2.099287in}}{\pgfqpoint{1.806918in}{2.107187in}}{\pgfqpoint{1.806918in}{2.115423in}}%
\pgfpathcurveto{\pgfqpoint{1.806918in}{2.123660in}}{\pgfqpoint{1.803646in}{2.131560in}}{\pgfqpoint{1.797822in}{2.137384in}}%
\pgfpathcurveto{\pgfqpoint{1.791998in}{2.143208in}}{\pgfqpoint{1.784098in}{2.146480in}}{\pgfqpoint{1.775861in}{2.146480in}}%
\pgfpathcurveto{\pgfqpoint{1.767625in}{2.146480in}}{\pgfqpoint{1.759725in}{2.143208in}}{\pgfqpoint{1.753901in}{2.137384in}}%
\pgfpathcurveto{\pgfqpoint{1.748077in}{2.131560in}}{\pgfqpoint{1.744805in}{2.123660in}}{\pgfqpoint{1.744805in}{2.115423in}}%
\pgfpathcurveto{\pgfqpoint{1.744805in}{2.107187in}}{\pgfqpoint{1.748077in}{2.099287in}}{\pgfqpoint{1.753901in}{2.093463in}}%
\pgfpathcurveto{\pgfqpoint{1.759725in}{2.087639in}}{\pgfqpoint{1.767625in}{2.084367in}}{\pgfqpoint{1.775861in}{2.084367in}}%
\pgfpathclose%
\pgfusepath{stroke,fill}%
\end{pgfscope}%
\begin{pgfscope}%
\pgfpathrectangle{\pgfqpoint{0.100000in}{0.212622in}}{\pgfqpoint{3.696000in}{3.696000in}}%
\pgfusepath{clip}%
\pgfsetbuttcap%
\pgfsetroundjoin%
\definecolor{currentfill}{rgb}{0.121569,0.466667,0.705882}%
\pgfsetfillcolor{currentfill}%
\pgfsetfillopacity{0.307769}%
\pgfsetlinewidth{1.003750pt}%
\definecolor{currentstroke}{rgb}{0.121569,0.466667,0.705882}%
\pgfsetstrokecolor{currentstroke}%
\pgfsetstrokeopacity{0.307769}%
\pgfsetdash{}{0pt}%
\pgfpathmoveto{\pgfqpoint{1.950065in}{2.055921in}}%
\pgfpathcurveto{\pgfqpoint{1.958301in}{2.055921in}}{\pgfqpoint{1.966201in}{2.059193in}}{\pgfqpoint{1.972025in}{2.065017in}}%
\pgfpathcurveto{\pgfqpoint{1.977849in}{2.070841in}}{\pgfqpoint{1.981121in}{2.078741in}}{\pgfqpoint{1.981121in}{2.086977in}}%
\pgfpathcurveto{\pgfqpoint{1.981121in}{2.095213in}}{\pgfqpoint{1.977849in}{2.103113in}}{\pgfqpoint{1.972025in}{2.108937in}}%
\pgfpathcurveto{\pgfqpoint{1.966201in}{2.114761in}}{\pgfqpoint{1.958301in}{2.118034in}}{\pgfqpoint{1.950065in}{2.118034in}}%
\pgfpathcurveto{\pgfqpoint{1.941828in}{2.118034in}}{\pgfqpoint{1.933928in}{2.114761in}}{\pgfqpoint{1.928104in}{2.108937in}}%
\pgfpathcurveto{\pgfqpoint{1.922280in}{2.103113in}}{\pgfqpoint{1.919008in}{2.095213in}}{\pgfqpoint{1.919008in}{2.086977in}}%
\pgfpathcurveto{\pgfqpoint{1.919008in}{2.078741in}}{\pgfqpoint{1.922280in}{2.070841in}}{\pgfqpoint{1.928104in}{2.065017in}}%
\pgfpathcurveto{\pgfqpoint{1.933928in}{2.059193in}}{\pgfqpoint{1.941828in}{2.055921in}}{\pgfqpoint{1.950065in}{2.055921in}}%
\pgfpathclose%
\pgfusepath{stroke,fill}%
\end{pgfscope}%
\begin{pgfscope}%
\pgfpathrectangle{\pgfqpoint{0.100000in}{0.212622in}}{\pgfqpoint{3.696000in}{3.696000in}}%
\pgfusepath{clip}%
\pgfsetbuttcap%
\pgfsetroundjoin%
\definecolor{currentfill}{rgb}{0.121569,0.466667,0.705882}%
\pgfsetfillcolor{currentfill}%
\pgfsetfillopacity{0.308066}%
\pgfsetlinewidth{1.003750pt}%
\definecolor{currentstroke}{rgb}{0.121569,0.466667,0.705882}%
\pgfsetstrokecolor{currentstroke}%
\pgfsetstrokeopacity{0.308066}%
\pgfsetdash{}{0pt}%
\pgfpathmoveto{\pgfqpoint{1.769750in}{2.088398in}}%
\pgfpathcurveto{\pgfqpoint{1.777987in}{2.088398in}}{\pgfqpoint{1.785887in}{2.091671in}}{\pgfqpoint{1.791711in}{2.097495in}}%
\pgfpathcurveto{\pgfqpoint{1.797535in}{2.103319in}}{\pgfqpoint{1.800807in}{2.111219in}}{\pgfqpoint{1.800807in}{2.119455in}}%
\pgfpathcurveto{\pgfqpoint{1.800807in}{2.127691in}}{\pgfqpoint{1.797535in}{2.135591in}}{\pgfqpoint{1.791711in}{2.141415in}}%
\pgfpathcurveto{\pgfqpoint{1.785887in}{2.147239in}}{\pgfqpoint{1.777987in}{2.150511in}}{\pgfqpoint{1.769750in}{2.150511in}}%
\pgfpathcurveto{\pgfqpoint{1.761514in}{2.150511in}}{\pgfqpoint{1.753614in}{2.147239in}}{\pgfqpoint{1.747790in}{2.141415in}}%
\pgfpathcurveto{\pgfqpoint{1.741966in}{2.135591in}}{\pgfqpoint{1.738694in}{2.127691in}}{\pgfqpoint{1.738694in}{2.119455in}}%
\pgfpathcurveto{\pgfqpoint{1.738694in}{2.111219in}}{\pgfqpoint{1.741966in}{2.103319in}}{\pgfqpoint{1.747790in}{2.097495in}}%
\pgfpathcurveto{\pgfqpoint{1.753614in}{2.091671in}}{\pgfqpoint{1.761514in}{2.088398in}}{\pgfqpoint{1.769750in}{2.088398in}}%
\pgfpathclose%
\pgfusepath{stroke,fill}%
\end{pgfscope}%
\begin{pgfscope}%
\pgfpathrectangle{\pgfqpoint{0.100000in}{0.212622in}}{\pgfqpoint{3.696000in}{3.696000in}}%
\pgfusepath{clip}%
\pgfsetbuttcap%
\pgfsetroundjoin%
\definecolor{currentfill}{rgb}{0.121569,0.466667,0.705882}%
\pgfsetfillcolor{currentfill}%
\pgfsetfillopacity{0.308797}%
\pgfsetlinewidth{1.003750pt}%
\definecolor{currentstroke}{rgb}{0.121569,0.466667,0.705882}%
\pgfsetstrokecolor{currentstroke}%
\pgfsetstrokeopacity{0.308797}%
\pgfsetdash{}{0pt}%
\pgfpathmoveto{\pgfqpoint{1.761844in}{2.087907in}}%
\pgfpathcurveto{\pgfqpoint{1.770080in}{2.087907in}}{\pgfqpoint{1.777980in}{2.091179in}}{\pgfqpoint{1.783804in}{2.097003in}}%
\pgfpathcurveto{\pgfqpoint{1.789628in}{2.102827in}}{\pgfqpoint{1.792900in}{2.110727in}}{\pgfqpoint{1.792900in}{2.118963in}}%
\pgfpathcurveto{\pgfqpoint{1.792900in}{2.127200in}}{\pgfqpoint{1.789628in}{2.135100in}}{\pgfqpoint{1.783804in}{2.140924in}}%
\pgfpathcurveto{\pgfqpoint{1.777980in}{2.146748in}}{\pgfqpoint{1.770080in}{2.150020in}}{\pgfqpoint{1.761844in}{2.150020in}}%
\pgfpathcurveto{\pgfqpoint{1.753608in}{2.150020in}}{\pgfqpoint{1.745708in}{2.146748in}}{\pgfqpoint{1.739884in}{2.140924in}}%
\pgfpathcurveto{\pgfqpoint{1.734060in}{2.135100in}}{\pgfqpoint{1.730787in}{2.127200in}}{\pgfqpoint{1.730787in}{2.118963in}}%
\pgfpathcurveto{\pgfqpoint{1.730787in}{2.110727in}}{\pgfqpoint{1.734060in}{2.102827in}}{\pgfqpoint{1.739884in}{2.097003in}}%
\pgfpathcurveto{\pgfqpoint{1.745708in}{2.091179in}}{\pgfqpoint{1.753608in}{2.087907in}}{\pgfqpoint{1.761844in}{2.087907in}}%
\pgfpathclose%
\pgfusepath{stroke,fill}%
\end{pgfscope}%
\begin{pgfscope}%
\pgfpathrectangle{\pgfqpoint{0.100000in}{0.212622in}}{\pgfqpoint{3.696000in}{3.696000in}}%
\pgfusepath{clip}%
\pgfsetbuttcap%
\pgfsetroundjoin%
\definecolor{currentfill}{rgb}{0.121569,0.466667,0.705882}%
\pgfsetfillcolor{currentfill}%
\pgfsetfillopacity{0.309215}%
\pgfsetlinewidth{1.003750pt}%
\definecolor{currentstroke}{rgb}{0.121569,0.466667,0.705882}%
\pgfsetstrokecolor{currentstroke}%
\pgfsetstrokeopacity{0.309215}%
\pgfsetdash{}{0pt}%
\pgfpathmoveto{\pgfqpoint{1.756567in}{2.090766in}}%
\pgfpathcurveto{\pgfqpoint{1.764804in}{2.090766in}}{\pgfqpoint{1.772704in}{2.094038in}}{\pgfqpoint{1.778528in}{2.099862in}}%
\pgfpathcurveto{\pgfqpoint{1.784352in}{2.105686in}}{\pgfqpoint{1.787624in}{2.113586in}}{\pgfqpoint{1.787624in}{2.121822in}}%
\pgfpathcurveto{\pgfqpoint{1.787624in}{2.130058in}}{\pgfqpoint{1.784352in}{2.137959in}}{\pgfqpoint{1.778528in}{2.143782in}}%
\pgfpathcurveto{\pgfqpoint{1.772704in}{2.149606in}}{\pgfqpoint{1.764804in}{2.152879in}}{\pgfqpoint{1.756567in}{2.152879in}}%
\pgfpathcurveto{\pgfqpoint{1.748331in}{2.152879in}}{\pgfqpoint{1.740431in}{2.149606in}}{\pgfqpoint{1.734607in}{2.143782in}}%
\pgfpathcurveto{\pgfqpoint{1.728783in}{2.137959in}}{\pgfqpoint{1.725511in}{2.130058in}}{\pgfqpoint{1.725511in}{2.121822in}}%
\pgfpathcurveto{\pgfqpoint{1.725511in}{2.113586in}}{\pgfqpoint{1.728783in}{2.105686in}}{\pgfqpoint{1.734607in}{2.099862in}}%
\pgfpathcurveto{\pgfqpoint{1.740431in}{2.094038in}}{\pgfqpoint{1.748331in}{2.090766in}}{\pgfqpoint{1.756567in}{2.090766in}}%
\pgfpathclose%
\pgfusepath{stroke,fill}%
\end{pgfscope}%
\begin{pgfscope}%
\pgfpathrectangle{\pgfqpoint{0.100000in}{0.212622in}}{\pgfqpoint{3.696000in}{3.696000in}}%
\pgfusepath{clip}%
\pgfsetbuttcap%
\pgfsetroundjoin%
\definecolor{currentfill}{rgb}{0.121569,0.466667,0.705882}%
\pgfsetfillcolor{currentfill}%
\pgfsetfillopacity{0.309424}%
\pgfsetlinewidth{1.003750pt}%
\definecolor{currentstroke}{rgb}{0.121569,0.466667,0.705882}%
\pgfsetstrokecolor{currentstroke}%
\pgfsetstrokeopacity{0.309424}%
\pgfsetdash{}{0pt}%
\pgfpathmoveto{\pgfqpoint{1.754192in}{2.090496in}}%
\pgfpathcurveto{\pgfqpoint{1.762428in}{2.090496in}}{\pgfqpoint{1.770328in}{2.093768in}}{\pgfqpoint{1.776152in}{2.099592in}}%
\pgfpathcurveto{\pgfqpoint{1.781976in}{2.105416in}}{\pgfqpoint{1.785248in}{2.113316in}}{\pgfqpoint{1.785248in}{2.121552in}}%
\pgfpathcurveto{\pgfqpoint{1.785248in}{2.129789in}}{\pgfqpoint{1.781976in}{2.137689in}}{\pgfqpoint{1.776152in}{2.143513in}}%
\pgfpathcurveto{\pgfqpoint{1.770328in}{2.149337in}}{\pgfqpoint{1.762428in}{2.152609in}}{\pgfqpoint{1.754192in}{2.152609in}}%
\pgfpathcurveto{\pgfqpoint{1.745955in}{2.152609in}}{\pgfqpoint{1.738055in}{2.149337in}}{\pgfqpoint{1.732231in}{2.143513in}}%
\pgfpathcurveto{\pgfqpoint{1.726407in}{2.137689in}}{\pgfqpoint{1.723135in}{2.129789in}}{\pgfqpoint{1.723135in}{2.121552in}}%
\pgfpathcurveto{\pgfqpoint{1.723135in}{2.113316in}}{\pgfqpoint{1.726407in}{2.105416in}}{\pgfqpoint{1.732231in}{2.099592in}}%
\pgfpathcurveto{\pgfqpoint{1.738055in}{2.093768in}}{\pgfqpoint{1.745955in}{2.090496in}}{\pgfqpoint{1.754192in}{2.090496in}}%
\pgfpathclose%
\pgfusepath{stroke,fill}%
\end{pgfscope}%
\begin{pgfscope}%
\pgfpathrectangle{\pgfqpoint{0.100000in}{0.212622in}}{\pgfqpoint{3.696000in}{3.696000in}}%
\pgfusepath{clip}%
\pgfsetbuttcap%
\pgfsetroundjoin%
\definecolor{currentfill}{rgb}{0.121569,0.466667,0.705882}%
\pgfsetfillcolor{currentfill}%
\pgfsetfillopacity{0.309590}%
\pgfsetlinewidth{1.003750pt}%
\definecolor{currentstroke}{rgb}{0.121569,0.466667,0.705882}%
\pgfsetstrokecolor{currentstroke}%
\pgfsetstrokeopacity{0.309590}%
\pgfsetdash{}{0pt}%
\pgfpathmoveto{\pgfqpoint{1.952259in}{2.055876in}}%
\pgfpathcurveto{\pgfqpoint{1.960495in}{2.055876in}}{\pgfqpoint{1.968396in}{2.059148in}}{\pgfqpoint{1.974219in}{2.064972in}}%
\pgfpathcurveto{\pgfqpoint{1.980043in}{2.070796in}}{\pgfqpoint{1.983316in}{2.078696in}}{\pgfqpoint{1.983316in}{2.086933in}}%
\pgfpathcurveto{\pgfqpoint{1.983316in}{2.095169in}}{\pgfqpoint{1.980043in}{2.103069in}}{\pgfqpoint{1.974219in}{2.108893in}}%
\pgfpathcurveto{\pgfqpoint{1.968396in}{2.114717in}}{\pgfqpoint{1.960495in}{2.117989in}}{\pgfqpoint{1.952259in}{2.117989in}}%
\pgfpathcurveto{\pgfqpoint{1.944023in}{2.117989in}}{\pgfqpoint{1.936123in}{2.114717in}}{\pgfqpoint{1.930299in}{2.108893in}}%
\pgfpathcurveto{\pgfqpoint{1.924475in}{2.103069in}}{\pgfqpoint{1.921203in}{2.095169in}}{\pgfqpoint{1.921203in}{2.086933in}}%
\pgfpathcurveto{\pgfqpoint{1.921203in}{2.078696in}}{\pgfqpoint{1.924475in}{2.070796in}}{\pgfqpoint{1.930299in}{2.064972in}}%
\pgfpathcurveto{\pgfqpoint{1.936123in}{2.059148in}}{\pgfqpoint{1.944023in}{2.055876in}}{\pgfqpoint{1.952259in}{2.055876in}}%
\pgfpathclose%
\pgfusepath{stroke,fill}%
\end{pgfscope}%
\begin{pgfscope}%
\pgfpathrectangle{\pgfqpoint{0.100000in}{0.212622in}}{\pgfqpoint{3.696000in}{3.696000in}}%
\pgfusepath{clip}%
\pgfsetbuttcap%
\pgfsetroundjoin%
\definecolor{currentfill}{rgb}{0.121569,0.466667,0.705882}%
\pgfsetfillcolor{currentfill}%
\pgfsetfillopacity{0.309856}%
\pgfsetlinewidth{1.003750pt}%
\definecolor{currentstroke}{rgb}{0.121569,0.466667,0.705882}%
\pgfsetstrokecolor{currentstroke}%
\pgfsetstrokeopacity{0.309856}%
\pgfsetdash{}{0pt}%
\pgfpathmoveto{\pgfqpoint{1.748188in}{2.093631in}}%
\pgfpathcurveto{\pgfqpoint{1.756424in}{2.093631in}}{\pgfqpoint{1.764324in}{2.096904in}}{\pgfqpoint{1.770148in}{2.102728in}}%
\pgfpathcurveto{\pgfqpoint{1.775972in}{2.108552in}}{\pgfqpoint{1.779245in}{2.116452in}}{\pgfqpoint{1.779245in}{2.124688in}}%
\pgfpathcurveto{\pgfqpoint{1.779245in}{2.132924in}}{\pgfqpoint{1.775972in}{2.140824in}}{\pgfqpoint{1.770148in}{2.146648in}}%
\pgfpathcurveto{\pgfqpoint{1.764324in}{2.152472in}}{\pgfqpoint{1.756424in}{2.155744in}}{\pgfqpoint{1.748188in}{2.155744in}}%
\pgfpathcurveto{\pgfqpoint{1.739952in}{2.155744in}}{\pgfqpoint{1.732052in}{2.152472in}}{\pgfqpoint{1.726228in}{2.146648in}}%
\pgfpathcurveto{\pgfqpoint{1.720404in}{2.140824in}}{\pgfqpoint{1.717132in}{2.132924in}}{\pgfqpoint{1.717132in}{2.124688in}}%
\pgfpathcurveto{\pgfqpoint{1.717132in}{2.116452in}}{\pgfqpoint{1.720404in}{2.108552in}}{\pgfqpoint{1.726228in}{2.102728in}}%
\pgfpathcurveto{\pgfqpoint{1.732052in}{2.096904in}}{\pgfqpoint{1.739952in}{2.093631in}}{\pgfqpoint{1.748188in}{2.093631in}}%
\pgfpathclose%
\pgfusepath{stroke,fill}%
\end{pgfscope}%
\begin{pgfscope}%
\pgfpathrectangle{\pgfqpoint{0.100000in}{0.212622in}}{\pgfqpoint{3.696000in}{3.696000in}}%
\pgfusepath{clip}%
\pgfsetbuttcap%
\pgfsetroundjoin%
\definecolor{currentfill}{rgb}{0.121569,0.466667,0.705882}%
\pgfsetfillcolor{currentfill}%
\pgfsetfillopacity{0.310093}%
\pgfsetlinewidth{1.003750pt}%
\definecolor{currentstroke}{rgb}{0.121569,0.466667,0.705882}%
\pgfsetstrokecolor{currentstroke}%
\pgfsetstrokeopacity{0.310093}%
\pgfsetdash{}{0pt}%
\pgfpathmoveto{\pgfqpoint{1.745559in}{2.093193in}}%
\pgfpathcurveto{\pgfqpoint{1.753795in}{2.093193in}}{\pgfqpoint{1.761695in}{2.096466in}}{\pgfqpoint{1.767519in}{2.102290in}}%
\pgfpathcurveto{\pgfqpoint{1.773343in}{2.108114in}}{\pgfqpoint{1.776615in}{2.116014in}}{\pgfqpoint{1.776615in}{2.124250in}}%
\pgfpathcurveto{\pgfqpoint{1.776615in}{2.132486in}}{\pgfqpoint{1.773343in}{2.140386in}}{\pgfqpoint{1.767519in}{2.146210in}}%
\pgfpathcurveto{\pgfqpoint{1.761695in}{2.152034in}}{\pgfqpoint{1.753795in}{2.155306in}}{\pgfqpoint{1.745559in}{2.155306in}}%
\pgfpathcurveto{\pgfqpoint{1.737322in}{2.155306in}}{\pgfqpoint{1.729422in}{2.152034in}}{\pgfqpoint{1.723598in}{2.146210in}}%
\pgfpathcurveto{\pgfqpoint{1.717775in}{2.140386in}}{\pgfqpoint{1.714502in}{2.132486in}}{\pgfqpoint{1.714502in}{2.124250in}}%
\pgfpathcurveto{\pgfqpoint{1.714502in}{2.116014in}}{\pgfqpoint{1.717775in}{2.108114in}}{\pgfqpoint{1.723598in}{2.102290in}}%
\pgfpathcurveto{\pgfqpoint{1.729422in}{2.096466in}}{\pgfqpoint{1.737322in}{2.093193in}}{\pgfqpoint{1.745559in}{2.093193in}}%
\pgfpathclose%
\pgfusepath{stroke,fill}%
\end{pgfscope}%
\begin{pgfscope}%
\pgfpathrectangle{\pgfqpoint{0.100000in}{0.212622in}}{\pgfqpoint{3.696000in}{3.696000in}}%
\pgfusepath{clip}%
\pgfsetbuttcap%
\pgfsetroundjoin%
\definecolor{currentfill}{rgb}{0.121569,0.466667,0.705882}%
\pgfsetfillcolor{currentfill}%
\pgfsetfillopacity{0.310967}%
\pgfsetlinewidth{1.003750pt}%
\definecolor{currentstroke}{rgb}{0.121569,0.466667,0.705882}%
\pgfsetstrokecolor{currentstroke}%
\pgfsetstrokeopacity{0.310967}%
\pgfsetdash{}{0pt}%
\pgfpathmoveto{\pgfqpoint{1.735700in}{2.097641in}}%
\pgfpathcurveto{\pgfqpoint{1.743936in}{2.097641in}}{\pgfqpoint{1.751836in}{2.100913in}}{\pgfqpoint{1.757660in}{2.106737in}}%
\pgfpathcurveto{\pgfqpoint{1.763484in}{2.112561in}}{\pgfqpoint{1.766757in}{2.120461in}}{\pgfqpoint{1.766757in}{2.128698in}}%
\pgfpathcurveto{\pgfqpoint{1.766757in}{2.136934in}}{\pgfqpoint{1.763484in}{2.144834in}}{\pgfqpoint{1.757660in}{2.150658in}}%
\pgfpathcurveto{\pgfqpoint{1.751836in}{2.156482in}}{\pgfqpoint{1.743936in}{2.159754in}}{\pgfqpoint{1.735700in}{2.159754in}}%
\pgfpathcurveto{\pgfqpoint{1.727464in}{2.159754in}}{\pgfqpoint{1.719564in}{2.156482in}}{\pgfqpoint{1.713740in}{2.150658in}}%
\pgfpathcurveto{\pgfqpoint{1.707916in}{2.144834in}}{\pgfqpoint{1.704644in}{2.136934in}}{\pgfqpoint{1.704644in}{2.128698in}}%
\pgfpathcurveto{\pgfqpoint{1.704644in}{2.120461in}}{\pgfqpoint{1.707916in}{2.112561in}}{\pgfqpoint{1.713740in}{2.106737in}}%
\pgfpathcurveto{\pgfqpoint{1.719564in}{2.100913in}}{\pgfqpoint{1.727464in}{2.097641in}}{\pgfqpoint{1.735700in}{2.097641in}}%
\pgfpathclose%
\pgfusepath{stroke,fill}%
\end{pgfscope}%
\begin{pgfscope}%
\pgfpathrectangle{\pgfqpoint{0.100000in}{0.212622in}}{\pgfqpoint{3.696000in}{3.696000in}}%
\pgfusepath{clip}%
\pgfsetbuttcap%
\pgfsetroundjoin%
\definecolor{currentfill}{rgb}{0.121569,0.466667,0.705882}%
\pgfsetfillcolor{currentfill}%
\pgfsetfillopacity{0.311614}%
\pgfsetlinewidth{1.003750pt}%
\definecolor{currentstroke}{rgb}{0.121569,0.466667,0.705882}%
\pgfsetstrokecolor{currentstroke}%
\pgfsetstrokeopacity{0.311614}%
\pgfsetdash{}{0pt}%
\pgfpathmoveto{\pgfqpoint{1.949283in}{2.049867in}}%
\pgfpathcurveto{\pgfqpoint{1.957519in}{2.049867in}}{\pgfqpoint{1.965419in}{2.053139in}}{\pgfqpoint{1.971243in}{2.058963in}}%
\pgfpathcurveto{\pgfqpoint{1.977067in}{2.064787in}}{\pgfqpoint{1.980339in}{2.072687in}}{\pgfqpoint{1.980339in}{2.080923in}}%
\pgfpathcurveto{\pgfqpoint{1.980339in}{2.089160in}}{\pgfqpoint{1.977067in}{2.097060in}}{\pgfqpoint{1.971243in}{2.102884in}}%
\pgfpathcurveto{\pgfqpoint{1.965419in}{2.108707in}}{\pgfqpoint{1.957519in}{2.111980in}}{\pgfqpoint{1.949283in}{2.111980in}}%
\pgfpathcurveto{\pgfqpoint{1.941047in}{2.111980in}}{\pgfqpoint{1.933147in}{2.108707in}}{\pgfqpoint{1.927323in}{2.102884in}}%
\pgfpathcurveto{\pgfqpoint{1.921499in}{2.097060in}}{\pgfqpoint{1.918226in}{2.089160in}}{\pgfqpoint{1.918226in}{2.080923in}}%
\pgfpathcurveto{\pgfqpoint{1.918226in}{2.072687in}}{\pgfqpoint{1.921499in}{2.064787in}}{\pgfqpoint{1.927323in}{2.058963in}}%
\pgfpathcurveto{\pgfqpoint{1.933147in}{2.053139in}}{\pgfqpoint{1.941047in}{2.049867in}}{\pgfqpoint{1.949283in}{2.049867in}}%
\pgfpathclose%
\pgfusepath{stroke,fill}%
\end{pgfscope}%
\begin{pgfscope}%
\pgfpathrectangle{\pgfqpoint{0.100000in}{0.212622in}}{\pgfqpoint{3.696000in}{3.696000in}}%
\pgfusepath{clip}%
\pgfsetbuttcap%
\pgfsetroundjoin%
\definecolor{currentfill}{rgb}{0.121569,0.466667,0.705882}%
\pgfsetfillcolor{currentfill}%
\pgfsetfillopacity{0.312059}%
\pgfsetlinewidth{1.003750pt}%
\definecolor{currentstroke}{rgb}{0.121569,0.466667,0.705882}%
\pgfsetstrokecolor{currentstroke}%
\pgfsetstrokeopacity{0.312059}%
\pgfsetdash{}{0pt}%
\pgfpathmoveto{\pgfqpoint{1.724366in}{2.095356in}}%
\pgfpathcurveto{\pgfqpoint{1.732602in}{2.095356in}}{\pgfqpoint{1.740502in}{2.098629in}}{\pgfqpoint{1.746326in}{2.104453in}}%
\pgfpathcurveto{\pgfqpoint{1.752150in}{2.110276in}}{\pgfqpoint{1.755423in}{2.118176in}}{\pgfqpoint{1.755423in}{2.126413in}}%
\pgfpathcurveto{\pgfqpoint{1.755423in}{2.134649in}}{\pgfqpoint{1.752150in}{2.142549in}}{\pgfqpoint{1.746326in}{2.148373in}}%
\pgfpathcurveto{\pgfqpoint{1.740502in}{2.154197in}}{\pgfqpoint{1.732602in}{2.157469in}}{\pgfqpoint{1.724366in}{2.157469in}}%
\pgfpathcurveto{\pgfqpoint{1.716130in}{2.157469in}}{\pgfqpoint{1.708230in}{2.154197in}}{\pgfqpoint{1.702406in}{2.148373in}}%
\pgfpathcurveto{\pgfqpoint{1.696582in}{2.142549in}}{\pgfqpoint{1.693310in}{2.134649in}}{\pgfqpoint{1.693310in}{2.126413in}}%
\pgfpathcurveto{\pgfqpoint{1.693310in}{2.118176in}}{\pgfqpoint{1.696582in}{2.110276in}}{\pgfqpoint{1.702406in}{2.104453in}}%
\pgfpathcurveto{\pgfqpoint{1.708230in}{2.098629in}}{\pgfqpoint{1.716130in}{2.095356in}}{\pgfqpoint{1.724366in}{2.095356in}}%
\pgfpathclose%
\pgfusepath{stroke,fill}%
\end{pgfscope}%
\begin{pgfscope}%
\pgfpathrectangle{\pgfqpoint{0.100000in}{0.212622in}}{\pgfqpoint{3.696000in}{3.696000in}}%
\pgfusepath{clip}%
\pgfsetbuttcap%
\pgfsetroundjoin%
\definecolor{currentfill}{rgb}{0.121569,0.466667,0.705882}%
\pgfsetfillcolor{currentfill}%
\pgfsetfillopacity{0.312646}%
\pgfsetlinewidth{1.003750pt}%
\definecolor{currentstroke}{rgb}{0.121569,0.466667,0.705882}%
\pgfsetstrokecolor{currentstroke}%
\pgfsetstrokeopacity{0.312646}%
\pgfsetdash{}{0pt}%
\pgfpathmoveto{\pgfqpoint{1.946899in}{2.047241in}}%
\pgfpathcurveto{\pgfqpoint{1.955135in}{2.047241in}}{\pgfqpoint{1.963035in}{2.050513in}}{\pgfqpoint{1.968859in}{2.056337in}}%
\pgfpathcurveto{\pgfqpoint{1.974683in}{2.062161in}}{\pgfqpoint{1.977955in}{2.070061in}}{\pgfqpoint{1.977955in}{2.078298in}}%
\pgfpathcurveto{\pgfqpoint{1.977955in}{2.086534in}}{\pgfqpoint{1.974683in}{2.094434in}}{\pgfqpoint{1.968859in}{2.100258in}}%
\pgfpathcurveto{\pgfqpoint{1.963035in}{2.106082in}}{\pgfqpoint{1.955135in}{2.109354in}}{\pgfqpoint{1.946899in}{2.109354in}}%
\pgfpathcurveto{\pgfqpoint{1.938662in}{2.109354in}}{\pgfqpoint{1.930762in}{2.106082in}}{\pgfqpoint{1.924939in}{2.100258in}}%
\pgfpathcurveto{\pgfqpoint{1.919115in}{2.094434in}}{\pgfqpoint{1.915842in}{2.086534in}}{\pgfqpoint{1.915842in}{2.078298in}}%
\pgfpathcurveto{\pgfqpoint{1.915842in}{2.070061in}}{\pgfqpoint{1.919115in}{2.062161in}}{\pgfqpoint{1.924939in}{2.056337in}}%
\pgfpathcurveto{\pgfqpoint{1.930762in}{2.050513in}}{\pgfqpoint{1.938662in}{2.047241in}}{\pgfqpoint{1.946899in}{2.047241in}}%
\pgfpathclose%
\pgfusepath{stroke,fill}%
\end{pgfscope}%
\begin{pgfscope}%
\pgfpathrectangle{\pgfqpoint{0.100000in}{0.212622in}}{\pgfqpoint{3.696000in}{3.696000in}}%
\pgfusepath{clip}%
\pgfsetbuttcap%
\pgfsetroundjoin%
\definecolor{currentfill}{rgb}{0.121569,0.466667,0.705882}%
\pgfsetfillcolor{currentfill}%
\pgfsetfillopacity{0.312964}%
\pgfsetlinewidth{1.003750pt}%
\definecolor{currentstroke}{rgb}{0.121569,0.466667,0.705882}%
\pgfsetstrokecolor{currentstroke}%
\pgfsetstrokeopacity{0.312964}%
\pgfsetdash{}{0pt}%
\pgfpathmoveto{\pgfqpoint{1.945932in}{2.046291in}}%
\pgfpathcurveto{\pgfqpoint{1.954168in}{2.046291in}}{\pgfqpoint{1.962068in}{2.049563in}}{\pgfqpoint{1.967892in}{2.055387in}}%
\pgfpathcurveto{\pgfqpoint{1.973716in}{2.061211in}}{\pgfqpoint{1.976988in}{2.069111in}}{\pgfqpoint{1.976988in}{2.077348in}}%
\pgfpathcurveto{\pgfqpoint{1.976988in}{2.085584in}}{\pgfqpoint{1.973716in}{2.093484in}}{\pgfqpoint{1.967892in}{2.099308in}}%
\pgfpathcurveto{\pgfqpoint{1.962068in}{2.105132in}}{\pgfqpoint{1.954168in}{2.108404in}}{\pgfqpoint{1.945932in}{2.108404in}}%
\pgfpathcurveto{\pgfqpoint{1.937695in}{2.108404in}}{\pgfqpoint{1.929795in}{2.105132in}}{\pgfqpoint{1.923971in}{2.099308in}}%
\pgfpathcurveto{\pgfqpoint{1.918147in}{2.093484in}}{\pgfqpoint{1.914875in}{2.085584in}}{\pgfqpoint{1.914875in}{2.077348in}}%
\pgfpathcurveto{\pgfqpoint{1.914875in}{2.069111in}}{\pgfqpoint{1.918147in}{2.061211in}}{\pgfqpoint{1.923971in}{2.055387in}}%
\pgfpathcurveto{\pgfqpoint{1.929795in}{2.049563in}}{\pgfqpoint{1.937695in}{2.046291in}}{\pgfqpoint{1.945932in}{2.046291in}}%
\pgfpathclose%
\pgfusepath{stroke,fill}%
\end{pgfscope}%
\begin{pgfscope}%
\pgfpathrectangle{\pgfqpoint{0.100000in}{0.212622in}}{\pgfqpoint{3.696000in}{3.696000in}}%
\pgfusepath{clip}%
\pgfsetbuttcap%
\pgfsetroundjoin%
\definecolor{currentfill}{rgb}{0.121569,0.466667,0.705882}%
\pgfsetfillcolor{currentfill}%
\pgfsetfillopacity{0.313482}%
\pgfsetlinewidth{1.003750pt}%
\definecolor{currentstroke}{rgb}{0.121569,0.466667,0.705882}%
\pgfsetstrokecolor{currentstroke}%
\pgfsetstrokeopacity{0.313482}%
\pgfsetdash{}{0pt}%
\pgfpathmoveto{\pgfqpoint{1.943500in}{2.045010in}}%
\pgfpathcurveto{\pgfqpoint{1.951737in}{2.045010in}}{\pgfqpoint{1.959637in}{2.048282in}}{\pgfqpoint{1.965461in}{2.054106in}}%
\pgfpathcurveto{\pgfqpoint{1.971284in}{2.059930in}}{\pgfqpoint{1.974557in}{2.067830in}}{\pgfqpoint{1.974557in}{2.076066in}}%
\pgfpathcurveto{\pgfqpoint{1.974557in}{2.084303in}}{\pgfqpoint{1.971284in}{2.092203in}}{\pgfqpoint{1.965461in}{2.098027in}}%
\pgfpathcurveto{\pgfqpoint{1.959637in}{2.103851in}}{\pgfqpoint{1.951737in}{2.107123in}}{\pgfqpoint{1.943500in}{2.107123in}}%
\pgfpathcurveto{\pgfqpoint{1.935264in}{2.107123in}}{\pgfqpoint{1.927364in}{2.103851in}}{\pgfqpoint{1.921540in}{2.098027in}}%
\pgfpathcurveto{\pgfqpoint{1.915716in}{2.092203in}}{\pgfqpoint{1.912444in}{2.084303in}}{\pgfqpoint{1.912444in}{2.076066in}}%
\pgfpathcurveto{\pgfqpoint{1.912444in}{2.067830in}}{\pgfqpoint{1.915716in}{2.059930in}}{\pgfqpoint{1.921540in}{2.054106in}}%
\pgfpathcurveto{\pgfqpoint{1.927364in}{2.048282in}}{\pgfqpoint{1.935264in}{2.045010in}}{\pgfqpoint{1.943500in}{2.045010in}}%
\pgfpathclose%
\pgfusepath{stroke,fill}%
\end{pgfscope}%
\begin{pgfscope}%
\pgfpathrectangle{\pgfqpoint{0.100000in}{0.212622in}}{\pgfqpoint{3.696000in}{3.696000in}}%
\pgfusepath{clip}%
\pgfsetbuttcap%
\pgfsetroundjoin%
\definecolor{currentfill}{rgb}{0.121569,0.466667,0.705882}%
\pgfsetfillcolor{currentfill}%
\pgfsetfillopacity{0.313568}%
\pgfsetlinewidth{1.003750pt}%
\definecolor{currentstroke}{rgb}{0.121569,0.466667,0.705882}%
\pgfsetstrokecolor{currentstroke}%
\pgfsetstrokeopacity{0.313568}%
\pgfsetdash{}{0pt}%
\pgfpathmoveto{\pgfqpoint{1.705365in}{2.103153in}}%
\pgfpathcurveto{\pgfqpoint{1.713601in}{2.103153in}}{\pgfqpoint{1.721501in}{2.106425in}}{\pgfqpoint{1.727325in}{2.112249in}}%
\pgfpathcurveto{\pgfqpoint{1.733149in}{2.118073in}}{\pgfqpoint{1.736421in}{2.125973in}}{\pgfqpoint{1.736421in}{2.134209in}}%
\pgfpathcurveto{\pgfqpoint{1.736421in}{2.142446in}}{\pgfqpoint{1.733149in}{2.150346in}}{\pgfqpoint{1.727325in}{2.156169in}}%
\pgfpathcurveto{\pgfqpoint{1.721501in}{2.161993in}}{\pgfqpoint{1.713601in}{2.165266in}}{\pgfqpoint{1.705365in}{2.165266in}}%
\pgfpathcurveto{\pgfqpoint{1.697128in}{2.165266in}}{\pgfqpoint{1.689228in}{2.161993in}}{\pgfqpoint{1.683404in}{2.156169in}}%
\pgfpathcurveto{\pgfqpoint{1.677581in}{2.150346in}}{\pgfqpoint{1.674308in}{2.142446in}}{\pgfqpoint{1.674308in}{2.134209in}}%
\pgfpathcurveto{\pgfqpoint{1.674308in}{2.125973in}}{\pgfqpoint{1.677581in}{2.118073in}}{\pgfqpoint{1.683404in}{2.112249in}}%
\pgfpathcurveto{\pgfqpoint{1.689228in}{2.106425in}}{\pgfqpoint{1.697128in}{2.103153in}}{\pgfqpoint{1.705365in}{2.103153in}}%
\pgfpathclose%
\pgfusepath{stroke,fill}%
\end{pgfscope}%
\begin{pgfscope}%
\pgfpathrectangle{\pgfqpoint{0.100000in}{0.212622in}}{\pgfqpoint{3.696000in}{3.696000in}}%
\pgfusepath{clip}%
\pgfsetbuttcap%
\pgfsetroundjoin%
\definecolor{currentfill}{rgb}{0.121569,0.466667,0.705882}%
\pgfsetfillcolor{currentfill}%
\pgfsetfillopacity{0.313729}%
\pgfsetlinewidth{1.003750pt}%
\definecolor{currentstroke}{rgb}{0.121569,0.466667,0.705882}%
\pgfsetstrokecolor{currentstroke}%
\pgfsetstrokeopacity{0.313729}%
\pgfsetdash{}{0pt}%
\pgfpathmoveto{\pgfqpoint{1.942440in}{2.044143in}}%
\pgfpathcurveto{\pgfqpoint{1.950676in}{2.044143in}}{\pgfqpoint{1.958576in}{2.047415in}}{\pgfqpoint{1.964400in}{2.053239in}}%
\pgfpathcurveto{\pgfqpoint{1.970224in}{2.059063in}}{\pgfqpoint{1.973496in}{2.066963in}}{\pgfqpoint{1.973496in}{2.075199in}}%
\pgfpathcurveto{\pgfqpoint{1.973496in}{2.083435in}}{\pgfqpoint{1.970224in}{2.091336in}}{\pgfqpoint{1.964400in}{2.097159in}}%
\pgfpathcurveto{\pgfqpoint{1.958576in}{2.102983in}}{\pgfqpoint{1.950676in}{2.106256in}}{\pgfqpoint{1.942440in}{2.106256in}}%
\pgfpathcurveto{\pgfqpoint{1.934204in}{2.106256in}}{\pgfqpoint{1.926303in}{2.102983in}}{\pgfqpoint{1.920480in}{2.097159in}}%
\pgfpathcurveto{\pgfqpoint{1.914656in}{2.091336in}}{\pgfqpoint{1.911383in}{2.083435in}}{\pgfqpoint{1.911383in}{2.075199in}}%
\pgfpathcurveto{\pgfqpoint{1.911383in}{2.066963in}}{\pgfqpoint{1.914656in}{2.059063in}}{\pgfqpoint{1.920480in}{2.053239in}}%
\pgfpathcurveto{\pgfqpoint{1.926303in}{2.047415in}}{\pgfqpoint{1.934204in}{2.044143in}}{\pgfqpoint{1.942440in}{2.044143in}}%
\pgfpathclose%
\pgfusepath{stroke,fill}%
\end{pgfscope}%
\begin{pgfscope}%
\pgfpathrectangle{\pgfqpoint{0.100000in}{0.212622in}}{\pgfqpoint{3.696000in}{3.696000in}}%
\pgfusepath{clip}%
\pgfsetbuttcap%
\pgfsetroundjoin%
\definecolor{currentfill}{rgb}{0.121569,0.466667,0.705882}%
\pgfsetfillcolor{currentfill}%
\pgfsetfillopacity{0.314129}%
\pgfsetlinewidth{1.003750pt}%
\definecolor{currentstroke}{rgb}{0.121569,0.466667,0.705882}%
\pgfsetstrokecolor{currentstroke}%
\pgfsetstrokeopacity{0.314129}%
\pgfsetdash{}{0pt}%
\pgfpathmoveto{\pgfqpoint{1.940034in}{2.042923in}}%
\pgfpathcurveto{\pgfqpoint{1.948270in}{2.042923in}}{\pgfqpoint{1.956171in}{2.046195in}}{\pgfqpoint{1.961994in}{2.052019in}}%
\pgfpathcurveto{\pgfqpoint{1.967818in}{2.057843in}}{\pgfqpoint{1.971091in}{2.065743in}}{\pgfqpoint{1.971091in}{2.073979in}}%
\pgfpathcurveto{\pgfqpoint{1.971091in}{2.082216in}}{\pgfqpoint{1.967818in}{2.090116in}}{\pgfqpoint{1.961994in}{2.095940in}}%
\pgfpathcurveto{\pgfqpoint{1.956171in}{2.101764in}}{\pgfqpoint{1.948270in}{2.105036in}}{\pgfqpoint{1.940034in}{2.105036in}}%
\pgfpathcurveto{\pgfqpoint{1.931798in}{2.105036in}}{\pgfqpoint{1.923898in}{2.101764in}}{\pgfqpoint{1.918074in}{2.095940in}}%
\pgfpathcurveto{\pgfqpoint{1.912250in}{2.090116in}}{\pgfqpoint{1.908978in}{2.082216in}}{\pgfqpoint{1.908978in}{2.073979in}}%
\pgfpathcurveto{\pgfqpoint{1.908978in}{2.065743in}}{\pgfqpoint{1.912250in}{2.057843in}}{\pgfqpoint{1.918074in}{2.052019in}}%
\pgfpathcurveto{\pgfqpoint{1.923898in}{2.046195in}}{\pgfqpoint{1.931798in}{2.042923in}}{\pgfqpoint{1.940034in}{2.042923in}}%
\pgfpathclose%
\pgfusepath{stroke,fill}%
\end{pgfscope}%
\begin{pgfscope}%
\pgfpathrectangle{\pgfqpoint{0.100000in}{0.212622in}}{\pgfqpoint{3.696000in}{3.696000in}}%
\pgfusepath{clip}%
\pgfsetbuttcap%
\pgfsetroundjoin%
\definecolor{currentfill}{rgb}{0.121569,0.466667,0.705882}%
\pgfsetfillcolor{currentfill}%
\pgfsetfillopacity{0.314293}%
\pgfsetlinewidth{1.003750pt}%
\definecolor{currentstroke}{rgb}{0.121569,0.466667,0.705882}%
\pgfsetstrokecolor{currentstroke}%
\pgfsetstrokeopacity{0.314293}%
\pgfsetdash{}{0pt}%
\pgfpathmoveto{\pgfqpoint{1.939277in}{2.042237in}}%
\pgfpathcurveto{\pgfqpoint{1.947513in}{2.042237in}}{\pgfqpoint{1.955413in}{2.045510in}}{\pgfqpoint{1.961237in}{2.051334in}}%
\pgfpathcurveto{\pgfqpoint{1.967061in}{2.057157in}}{\pgfqpoint{1.970333in}{2.065058in}}{\pgfqpoint{1.970333in}{2.073294in}}%
\pgfpathcurveto{\pgfqpoint{1.970333in}{2.081530in}}{\pgfqpoint{1.967061in}{2.089430in}}{\pgfqpoint{1.961237in}{2.095254in}}%
\pgfpathcurveto{\pgfqpoint{1.955413in}{2.101078in}}{\pgfqpoint{1.947513in}{2.104350in}}{\pgfqpoint{1.939277in}{2.104350in}}%
\pgfpathcurveto{\pgfqpoint{1.931040in}{2.104350in}}{\pgfqpoint{1.923140in}{2.101078in}}{\pgfqpoint{1.917316in}{2.095254in}}%
\pgfpathcurveto{\pgfqpoint{1.911492in}{2.089430in}}{\pgfqpoint{1.908220in}{2.081530in}}{\pgfqpoint{1.908220in}{2.073294in}}%
\pgfpathcurveto{\pgfqpoint{1.908220in}{2.065058in}}{\pgfqpoint{1.911492in}{2.057157in}}{\pgfqpoint{1.917316in}{2.051334in}}%
\pgfpathcurveto{\pgfqpoint{1.923140in}{2.045510in}}{\pgfqpoint{1.931040in}{2.042237in}}{\pgfqpoint{1.939277in}{2.042237in}}%
\pgfpathclose%
\pgfusepath{stroke,fill}%
\end{pgfscope}%
\begin{pgfscope}%
\pgfpathrectangle{\pgfqpoint{0.100000in}{0.212622in}}{\pgfqpoint{3.696000in}{3.696000in}}%
\pgfusepath{clip}%
\pgfsetbuttcap%
\pgfsetroundjoin%
\definecolor{currentfill}{rgb}{0.121569,0.466667,0.705882}%
\pgfsetfillcolor{currentfill}%
\pgfsetfillopacity{0.314573}%
\pgfsetlinewidth{1.003750pt}%
\definecolor{currentstroke}{rgb}{0.121569,0.466667,0.705882}%
\pgfsetstrokecolor{currentstroke}%
\pgfsetstrokeopacity{0.314573}%
\pgfsetdash{}{0pt}%
\pgfpathmoveto{\pgfqpoint{1.937510in}{2.041431in}}%
\pgfpathcurveto{\pgfqpoint{1.945746in}{2.041431in}}{\pgfqpoint{1.953646in}{2.044703in}}{\pgfqpoint{1.959470in}{2.050527in}}%
\pgfpathcurveto{\pgfqpoint{1.965294in}{2.056351in}}{\pgfqpoint{1.968567in}{2.064251in}}{\pgfqpoint{1.968567in}{2.072488in}}%
\pgfpathcurveto{\pgfqpoint{1.968567in}{2.080724in}}{\pgfqpoint{1.965294in}{2.088624in}}{\pgfqpoint{1.959470in}{2.094448in}}%
\pgfpathcurveto{\pgfqpoint{1.953646in}{2.100272in}}{\pgfqpoint{1.945746in}{2.103544in}}{\pgfqpoint{1.937510in}{2.103544in}}%
\pgfpathcurveto{\pgfqpoint{1.929274in}{2.103544in}}{\pgfqpoint{1.921374in}{2.100272in}}{\pgfqpoint{1.915550in}{2.094448in}}%
\pgfpathcurveto{\pgfqpoint{1.909726in}{2.088624in}}{\pgfqpoint{1.906454in}{2.080724in}}{\pgfqpoint{1.906454in}{2.072488in}}%
\pgfpathcurveto{\pgfqpoint{1.906454in}{2.064251in}}{\pgfqpoint{1.909726in}{2.056351in}}{\pgfqpoint{1.915550in}{2.050527in}}%
\pgfpathcurveto{\pgfqpoint{1.921374in}{2.044703in}}{\pgfqpoint{1.929274in}{2.041431in}}{\pgfqpoint{1.937510in}{2.041431in}}%
\pgfpathclose%
\pgfusepath{stroke,fill}%
\end{pgfscope}%
\begin{pgfscope}%
\pgfpathrectangle{\pgfqpoint{0.100000in}{0.212622in}}{\pgfqpoint{3.696000in}{3.696000in}}%
\pgfusepath{clip}%
\pgfsetbuttcap%
\pgfsetroundjoin%
\definecolor{currentfill}{rgb}{0.121569,0.466667,0.705882}%
\pgfsetfillcolor{currentfill}%
\pgfsetfillopacity{0.314576}%
\pgfsetlinewidth{1.003750pt}%
\definecolor{currentstroke}{rgb}{0.121569,0.466667,0.705882}%
\pgfsetstrokecolor{currentstroke}%
\pgfsetstrokeopacity{0.314576}%
\pgfsetdash{}{0pt}%
\pgfpathmoveto{\pgfqpoint{1.937495in}{2.041417in}}%
\pgfpathcurveto{\pgfqpoint{1.945731in}{2.041417in}}{\pgfqpoint{1.953632in}{2.044689in}}{\pgfqpoint{1.959455in}{2.050513in}}%
\pgfpathcurveto{\pgfqpoint{1.965279in}{2.056337in}}{\pgfqpoint{1.968552in}{2.064237in}}{\pgfqpoint{1.968552in}{2.072474in}}%
\pgfpathcurveto{\pgfqpoint{1.968552in}{2.080710in}}{\pgfqpoint{1.965279in}{2.088610in}}{\pgfqpoint{1.959455in}{2.094434in}}%
\pgfpathcurveto{\pgfqpoint{1.953632in}{2.100258in}}{\pgfqpoint{1.945731in}{2.103530in}}{\pgfqpoint{1.937495in}{2.103530in}}%
\pgfpathcurveto{\pgfqpoint{1.929259in}{2.103530in}}{\pgfqpoint{1.921359in}{2.100258in}}{\pgfqpoint{1.915535in}{2.094434in}}%
\pgfpathcurveto{\pgfqpoint{1.909711in}{2.088610in}}{\pgfqpoint{1.906439in}{2.080710in}}{\pgfqpoint{1.906439in}{2.072474in}}%
\pgfpathcurveto{\pgfqpoint{1.906439in}{2.064237in}}{\pgfqpoint{1.909711in}{2.056337in}}{\pgfqpoint{1.915535in}{2.050513in}}%
\pgfpathcurveto{\pgfqpoint{1.921359in}{2.044689in}}{\pgfqpoint{1.929259in}{2.041417in}}{\pgfqpoint{1.937495in}{2.041417in}}%
\pgfpathclose%
\pgfusepath{stroke,fill}%
\end{pgfscope}%
\begin{pgfscope}%
\pgfpathrectangle{\pgfqpoint{0.100000in}{0.212622in}}{\pgfqpoint{3.696000in}{3.696000in}}%
\pgfusepath{clip}%
\pgfsetbuttcap%
\pgfsetroundjoin%
\definecolor{currentfill}{rgb}{0.121569,0.466667,0.705882}%
\pgfsetfillcolor{currentfill}%
\pgfsetfillopacity{0.314582}%
\pgfsetlinewidth{1.003750pt}%
\definecolor{currentstroke}{rgb}{0.121569,0.466667,0.705882}%
\pgfsetstrokecolor{currentstroke}%
\pgfsetstrokeopacity{0.314582}%
\pgfsetdash{}{0pt}%
\pgfpathmoveto{\pgfqpoint{1.937461in}{2.041398in}}%
\pgfpathcurveto{\pgfqpoint{1.945697in}{2.041398in}}{\pgfqpoint{1.953597in}{2.044671in}}{\pgfqpoint{1.959421in}{2.050495in}}%
\pgfpathcurveto{\pgfqpoint{1.965245in}{2.056319in}}{\pgfqpoint{1.968517in}{2.064219in}}{\pgfqpoint{1.968517in}{2.072455in}}%
\pgfpathcurveto{\pgfqpoint{1.968517in}{2.080691in}}{\pgfqpoint{1.965245in}{2.088591in}}{\pgfqpoint{1.959421in}{2.094415in}}%
\pgfpathcurveto{\pgfqpoint{1.953597in}{2.100239in}}{\pgfqpoint{1.945697in}{2.103511in}}{\pgfqpoint{1.937461in}{2.103511in}}%
\pgfpathcurveto{\pgfqpoint{1.929225in}{2.103511in}}{\pgfqpoint{1.921325in}{2.100239in}}{\pgfqpoint{1.915501in}{2.094415in}}%
\pgfpathcurveto{\pgfqpoint{1.909677in}{2.088591in}}{\pgfqpoint{1.906404in}{2.080691in}}{\pgfqpoint{1.906404in}{2.072455in}}%
\pgfpathcurveto{\pgfqpoint{1.906404in}{2.064219in}}{\pgfqpoint{1.909677in}{2.056319in}}{\pgfqpoint{1.915501in}{2.050495in}}%
\pgfpathcurveto{\pgfqpoint{1.921325in}{2.044671in}}{\pgfqpoint{1.929225in}{2.041398in}}{\pgfqpoint{1.937461in}{2.041398in}}%
\pgfpathclose%
\pgfusepath{stroke,fill}%
\end{pgfscope}%
\begin{pgfscope}%
\pgfpathrectangle{\pgfqpoint{0.100000in}{0.212622in}}{\pgfqpoint{3.696000in}{3.696000in}}%
\pgfusepath{clip}%
\pgfsetbuttcap%
\pgfsetroundjoin%
\definecolor{currentfill}{rgb}{0.121569,0.466667,0.705882}%
\pgfsetfillcolor{currentfill}%
\pgfsetfillopacity{0.314594}%
\pgfsetlinewidth{1.003750pt}%
\definecolor{currentstroke}{rgb}{0.121569,0.466667,0.705882}%
\pgfsetstrokecolor{currentstroke}%
\pgfsetstrokeopacity{0.314594}%
\pgfsetdash{}{0pt}%
\pgfpathmoveto{\pgfqpoint{1.937413in}{2.041355in}}%
\pgfpathcurveto{\pgfqpoint{1.945650in}{2.041355in}}{\pgfqpoint{1.953550in}{2.044628in}}{\pgfqpoint{1.959374in}{2.050452in}}%
\pgfpathcurveto{\pgfqpoint{1.965198in}{2.056276in}}{\pgfqpoint{1.968470in}{2.064176in}}{\pgfqpoint{1.968470in}{2.072412in}}%
\pgfpathcurveto{\pgfqpoint{1.968470in}{2.080648in}}{\pgfqpoint{1.965198in}{2.088548in}}{\pgfqpoint{1.959374in}{2.094372in}}%
\pgfpathcurveto{\pgfqpoint{1.953550in}{2.100196in}}{\pgfqpoint{1.945650in}{2.103468in}}{\pgfqpoint{1.937413in}{2.103468in}}%
\pgfpathcurveto{\pgfqpoint{1.929177in}{2.103468in}}{\pgfqpoint{1.921277in}{2.100196in}}{\pgfqpoint{1.915453in}{2.094372in}}%
\pgfpathcurveto{\pgfqpoint{1.909629in}{2.088548in}}{\pgfqpoint{1.906357in}{2.080648in}}{\pgfqpoint{1.906357in}{2.072412in}}%
\pgfpathcurveto{\pgfqpoint{1.906357in}{2.064176in}}{\pgfqpoint{1.909629in}{2.056276in}}{\pgfqpoint{1.915453in}{2.050452in}}%
\pgfpathcurveto{\pgfqpoint{1.921277in}{2.044628in}}{\pgfqpoint{1.929177in}{2.041355in}}{\pgfqpoint{1.937413in}{2.041355in}}%
\pgfpathclose%
\pgfusepath{stroke,fill}%
\end{pgfscope}%
\begin{pgfscope}%
\pgfpathrectangle{\pgfqpoint{0.100000in}{0.212622in}}{\pgfqpoint{3.696000in}{3.696000in}}%
\pgfusepath{clip}%
\pgfsetbuttcap%
\pgfsetroundjoin%
\definecolor{currentfill}{rgb}{0.121569,0.466667,0.705882}%
\pgfsetfillcolor{currentfill}%
\pgfsetfillopacity{0.314613}%
\pgfsetlinewidth{1.003750pt}%
\definecolor{currentstroke}{rgb}{0.121569,0.466667,0.705882}%
\pgfsetstrokecolor{currentstroke}%
\pgfsetstrokeopacity{0.314613}%
\pgfsetdash{}{0pt}%
\pgfpathmoveto{\pgfqpoint{1.937297in}{2.041292in}}%
\pgfpathcurveto{\pgfqpoint{1.945534in}{2.041292in}}{\pgfqpoint{1.953434in}{2.044564in}}{\pgfqpoint{1.959258in}{2.050388in}}%
\pgfpathcurveto{\pgfqpoint{1.965082in}{2.056212in}}{\pgfqpoint{1.968354in}{2.064112in}}{\pgfqpoint{1.968354in}{2.072348in}}%
\pgfpathcurveto{\pgfqpoint{1.968354in}{2.080585in}}{\pgfqpoint{1.965082in}{2.088485in}}{\pgfqpoint{1.959258in}{2.094309in}}%
\pgfpathcurveto{\pgfqpoint{1.953434in}{2.100132in}}{\pgfqpoint{1.945534in}{2.103405in}}{\pgfqpoint{1.937297in}{2.103405in}}%
\pgfpathcurveto{\pgfqpoint{1.929061in}{2.103405in}}{\pgfqpoint{1.921161in}{2.100132in}}{\pgfqpoint{1.915337in}{2.094309in}}%
\pgfpathcurveto{\pgfqpoint{1.909513in}{2.088485in}}{\pgfqpoint{1.906241in}{2.080585in}}{\pgfqpoint{1.906241in}{2.072348in}}%
\pgfpathcurveto{\pgfqpoint{1.906241in}{2.064112in}}{\pgfqpoint{1.909513in}{2.056212in}}{\pgfqpoint{1.915337in}{2.050388in}}%
\pgfpathcurveto{\pgfqpoint{1.921161in}{2.044564in}}{\pgfqpoint{1.929061in}{2.041292in}}{\pgfqpoint{1.937297in}{2.041292in}}%
\pgfpathclose%
\pgfusepath{stroke,fill}%
\end{pgfscope}%
\begin{pgfscope}%
\pgfpathrectangle{\pgfqpoint{0.100000in}{0.212622in}}{\pgfqpoint{3.696000in}{3.696000in}}%
\pgfusepath{clip}%
\pgfsetbuttcap%
\pgfsetroundjoin%
\definecolor{currentfill}{rgb}{0.121569,0.466667,0.705882}%
\pgfsetfillcolor{currentfill}%
\pgfsetfillopacity{0.314651}%
\pgfsetlinewidth{1.003750pt}%
\definecolor{currentstroke}{rgb}{0.121569,0.466667,0.705882}%
\pgfsetstrokecolor{currentstroke}%
\pgfsetstrokeopacity{0.314651}%
\pgfsetdash{}{0pt}%
\pgfpathmoveto{\pgfqpoint{1.937134in}{2.041141in}}%
\pgfpathcurveto{\pgfqpoint{1.945370in}{2.041141in}}{\pgfqpoint{1.953270in}{2.044413in}}{\pgfqpoint{1.959094in}{2.050237in}}%
\pgfpathcurveto{\pgfqpoint{1.964918in}{2.056061in}}{\pgfqpoint{1.968190in}{2.063961in}}{\pgfqpoint{1.968190in}{2.072197in}}%
\pgfpathcurveto{\pgfqpoint{1.968190in}{2.080434in}}{\pgfqpoint{1.964918in}{2.088334in}}{\pgfqpoint{1.959094in}{2.094158in}}%
\pgfpathcurveto{\pgfqpoint{1.953270in}{2.099982in}}{\pgfqpoint{1.945370in}{2.103254in}}{\pgfqpoint{1.937134in}{2.103254in}}%
\pgfpathcurveto{\pgfqpoint{1.928898in}{2.103254in}}{\pgfqpoint{1.920998in}{2.099982in}}{\pgfqpoint{1.915174in}{2.094158in}}%
\pgfpathcurveto{\pgfqpoint{1.909350in}{2.088334in}}{\pgfqpoint{1.906077in}{2.080434in}}{\pgfqpoint{1.906077in}{2.072197in}}%
\pgfpathcurveto{\pgfqpoint{1.906077in}{2.063961in}}{\pgfqpoint{1.909350in}{2.056061in}}{\pgfqpoint{1.915174in}{2.050237in}}%
\pgfpathcurveto{\pgfqpoint{1.920998in}{2.044413in}}{\pgfqpoint{1.928898in}{2.041141in}}{\pgfqpoint{1.937134in}{2.041141in}}%
\pgfpathclose%
\pgfusepath{stroke,fill}%
\end{pgfscope}%
\begin{pgfscope}%
\pgfpathrectangle{\pgfqpoint{0.100000in}{0.212622in}}{\pgfqpoint{3.696000in}{3.696000in}}%
\pgfusepath{clip}%
\pgfsetbuttcap%
\pgfsetroundjoin%
\definecolor{currentfill}{rgb}{0.121569,0.466667,0.705882}%
\pgfsetfillcolor{currentfill}%
\pgfsetfillopacity{0.314715}%
\pgfsetlinewidth{1.003750pt}%
\definecolor{currentstroke}{rgb}{0.121569,0.466667,0.705882}%
\pgfsetstrokecolor{currentstroke}%
\pgfsetstrokeopacity{0.314715}%
\pgfsetdash{}{0pt}%
\pgfpathmoveto{\pgfqpoint{1.936737in}{2.040974in}}%
\pgfpathcurveto{\pgfqpoint{1.944974in}{2.040974in}}{\pgfqpoint{1.952874in}{2.044247in}}{\pgfqpoint{1.958698in}{2.050070in}}%
\pgfpathcurveto{\pgfqpoint{1.964522in}{2.055894in}}{\pgfqpoint{1.967794in}{2.063794in}}{\pgfqpoint{1.967794in}{2.072031in}}%
\pgfpathcurveto{\pgfqpoint{1.967794in}{2.080267in}}{\pgfqpoint{1.964522in}{2.088167in}}{\pgfqpoint{1.958698in}{2.093991in}}%
\pgfpathcurveto{\pgfqpoint{1.952874in}{2.099815in}}{\pgfqpoint{1.944974in}{2.103087in}}{\pgfqpoint{1.936737in}{2.103087in}}%
\pgfpathcurveto{\pgfqpoint{1.928501in}{2.103087in}}{\pgfqpoint{1.920601in}{2.099815in}}{\pgfqpoint{1.914777in}{2.093991in}}%
\pgfpathcurveto{\pgfqpoint{1.908953in}{2.088167in}}{\pgfqpoint{1.905681in}{2.080267in}}{\pgfqpoint{1.905681in}{2.072031in}}%
\pgfpathcurveto{\pgfqpoint{1.905681in}{2.063794in}}{\pgfqpoint{1.908953in}{2.055894in}}{\pgfqpoint{1.914777in}{2.050070in}}%
\pgfpathcurveto{\pgfqpoint{1.920601in}{2.044247in}}{\pgfqpoint{1.928501in}{2.040974in}}{\pgfqpoint{1.936737in}{2.040974in}}%
\pgfpathclose%
\pgfusepath{stroke,fill}%
\end{pgfscope}%
\begin{pgfscope}%
\pgfpathrectangle{\pgfqpoint{0.100000in}{0.212622in}}{\pgfqpoint{3.696000in}{3.696000in}}%
\pgfusepath{clip}%
\pgfsetbuttcap%
\pgfsetroundjoin%
\definecolor{currentfill}{rgb}{0.121569,0.466667,0.705882}%
\pgfsetfillcolor{currentfill}%
\pgfsetfillopacity{0.314840}%
\pgfsetlinewidth{1.003750pt}%
\definecolor{currentstroke}{rgb}{0.121569,0.466667,0.705882}%
\pgfsetstrokecolor{currentstroke}%
\pgfsetstrokeopacity{0.314840}%
\pgfsetdash{}{0pt}%
\pgfpathmoveto{\pgfqpoint{1.936223in}{2.040437in}}%
\pgfpathcurveto{\pgfqpoint{1.944459in}{2.040437in}}{\pgfqpoint{1.952359in}{2.043709in}}{\pgfqpoint{1.958183in}{2.049533in}}%
\pgfpathcurveto{\pgfqpoint{1.964007in}{2.055357in}}{\pgfqpoint{1.967279in}{2.063257in}}{\pgfqpoint{1.967279in}{2.071493in}}%
\pgfpathcurveto{\pgfqpoint{1.967279in}{2.079729in}}{\pgfqpoint{1.964007in}{2.087629in}}{\pgfqpoint{1.958183in}{2.093453in}}%
\pgfpathcurveto{\pgfqpoint{1.952359in}{2.099277in}}{\pgfqpoint{1.944459in}{2.102550in}}{\pgfqpoint{1.936223in}{2.102550in}}%
\pgfpathcurveto{\pgfqpoint{1.927986in}{2.102550in}}{\pgfqpoint{1.920086in}{2.099277in}}{\pgfqpoint{1.914262in}{2.093453in}}%
\pgfpathcurveto{\pgfqpoint{1.908438in}{2.087629in}}{\pgfqpoint{1.905166in}{2.079729in}}{\pgfqpoint{1.905166in}{2.071493in}}%
\pgfpathcurveto{\pgfqpoint{1.905166in}{2.063257in}}{\pgfqpoint{1.908438in}{2.055357in}}{\pgfqpoint{1.914262in}{2.049533in}}%
\pgfpathcurveto{\pgfqpoint{1.920086in}{2.043709in}}{\pgfqpoint{1.927986in}{2.040437in}}{\pgfqpoint{1.936223in}{2.040437in}}%
\pgfpathclose%
\pgfusepath{stroke,fill}%
\end{pgfscope}%
\begin{pgfscope}%
\pgfpathrectangle{\pgfqpoint{0.100000in}{0.212622in}}{\pgfqpoint{3.696000in}{3.696000in}}%
\pgfusepath{clip}%
\pgfsetbuttcap%
\pgfsetroundjoin%
\definecolor{currentfill}{rgb}{0.121569,0.466667,0.705882}%
\pgfsetfillcolor{currentfill}%
\pgfsetfillopacity{0.315045}%
\pgfsetlinewidth{1.003750pt}%
\definecolor{currentstroke}{rgb}{0.121569,0.466667,0.705882}%
\pgfsetstrokecolor{currentstroke}%
\pgfsetstrokeopacity{0.315045}%
\pgfsetdash{}{0pt}%
\pgfpathmoveto{\pgfqpoint{1.934964in}{2.039758in}}%
\pgfpathcurveto{\pgfqpoint{1.943200in}{2.039758in}}{\pgfqpoint{1.951101in}{2.043031in}}{\pgfqpoint{1.956924in}{2.048855in}}%
\pgfpathcurveto{\pgfqpoint{1.962748in}{2.054679in}}{\pgfqpoint{1.966021in}{2.062579in}}{\pgfqpoint{1.966021in}{2.070815in}}%
\pgfpathcurveto{\pgfqpoint{1.966021in}{2.079051in}}{\pgfqpoint{1.962748in}{2.086951in}}{\pgfqpoint{1.956924in}{2.092775in}}%
\pgfpathcurveto{\pgfqpoint{1.951101in}{2.098599in}}{\pgfqpoint{1.943200in}{2.101871in}}{\pgfqpoint{1.934964in}{2.101871in}}%
\pgfpathcurveto{\pgfqpoint{1.926728in}{2.101871in}}{\pgfqpoint{1.918828in}{2.098599in}}{\pgfqpoint{1.913004in}{2.092775in}}%
\pgfpathcurveto{\pgfqpoint{1.907180in}{2.086951in}}{\pgfqpoint{1.903908in}{2.079051in}}{\pgfqpoint{1.903908in}{2.070815in}}%
\pgfpathcurveto{\pgfqpoint{1.903908in}{2.062579in}}{\pgfqpoint{1.907180in}{2.054679in}}{\pgfqpoint{1.913004in}{2.048855in}}%
\pgfpathcurveto{\pgfqpoint{1.918828in}{2.043031in}}{\pgfqpoint{1.926728in}{2.039758in}}{\pgfqpoint{1.934964in}{2.039758in}}%
\pgfpathclose%
\pgfusepath{stroke,fill}%
\end{pgfscope}%
\begin{pgfscope}%
\pgfpathrectangle{\pgfqpoint{0.100000in}{0.212622in}}{\pgfqpoint{3.696000in}{3.696000in}}%
\pgfusepath{clip}%
\pgfsetbuttcap%
\pgfsetroundjoin%
\definecolor{currentfill}{rgb}{0.121569,0.466667,0.705882}%
\pgfsetfillcolor{currentfill}%
\pgfsetfillopacity{0.315295}%
\pgfsetlinewidth{1.003750pt}%
\definecolor{currentstroke}{rgb}{0.121569,0.466667,0.705882}%
\pgfsetstrokecolor{currentstroke}%
\pgfsetstrokeopacity{0.315295}%
\pgfsetdash{}{0pt}%
\pgfpathmoveto{\pgfqpoint{1.682999in}{2.099400in}}%
\pgfpathcurveto{\pgfqpoint{1.691235in}{2.099400in}}{\pgfqpoint{1.699135in}{2.102672in}}{\pgfqpoint{1.704959in}{2.108496in}}%
\pgfpathcurveto{\pgfqpoint{1.710783in}{2.114320in}}{\pgfqpoint{1.714056in}{2.122220in}}{\pgfqpoint{1.714056in}{2.130456in}}%
\pgfpathcurveto{\pgfqpoint{1.714056in}{2.138692in}}{\pgfqpoint{1.710783in}{2.146592in}}{\pgfqpoint{1.704959in}{2.152416in}}%
\pgfpathcurveto{\pgfqpoint{1.699135in}{2.158240in}}{\pgfqpoint{1.691235in}{2.161513in}}{\pgfqpoint{1.682999in}{2.161513in}}%
\pgfpathcurveto{\pgfqpoint{1.674763in}{2.161513in}}{\pgfqpoint{1.666863in}{2.158240in}}{\pgfqpoint{1.661039in}{2.152416in}}%
\pgfpathcurveto{\pgfqpoint{1.655215in}{2.146592in}}{\pgfqpoint{1.651943in}{2.138692in}}{\pgfqpoint{1.651943in}{2.130456in}}%
\pgfpathcurveto{\pgfqpoint{1.651943in}{2.122220in}}{\pgfqpoint{1.655215in}{2.114320in}}{\pgfqpoint{1.661039in}{2.108496in}}%
\pgfpathcurveto{\pgfqpoint{1.666863in}{2.102672in}}{\pgfqpoint{1.674763in}{2.099400in}}{\pgfqpoint{1.682999in}{2.099400in}}%
\pgfpathclose%
\pgfusepath{stroke,fill}%
\end{pgfscope}%
\begin{pgfscope}%
\pgfpathrectangle{\pgfqpoint{0.100000in}{0.212622in}}{\pgfqpoint{3.696000in}{3.696000in}}%
\pgfusepath{clip}%
\pgfsetbuttcap%
\pgfsetroundjoin%
\definecolor{currentfill}{rgb}{0.121569,0.466667,0.705882}%
\pgfsetfillcolor{currentfill}%
\pgfsetfillopacity{0.315474}%
\pgfsetlinewidth{1.003750pt}%
\definecolor{currentstroke}{rgb}{0.121569,0.466667,0.705882}%
\pgfsetstrokecolor{currentstroke}%
\pgfsetstrokeopacity{0.315474}%
\pgfsetdash{}{0pt}%
\pgfpathmoveto{\pgfqpoint{1.933186in}{2.038176in}}%
\pgfpathcurveto{\pgfqpoint{1.941423in}{2.038176in}}{\pgfqpoint{1.949323in}{2.041448in}}{\pgfqpoint{1.955147in}{2.047272in}}%
\pgfpathcurveto{\pgfqpoint{1.960971in}{2.053096in}}{\pgfqpoint{1.964243in}{2.060996in}}{\pgfqpoint{1.964243in}{2.069232in}}%
\pgfpathcurveto{\pgfqpoint{1.964243in}{2.077468in}}{\pgfqpoint{1.960971in}{2.085368in}}{\pgfqpoint{1.955147in}{2.091192in}}%
\pgfpathcurveto{\pgfqpoint{1.949323in}{2.097016in}}{\pgfqpoint{1.941423in}{2.100289in}}{\pgfqpoint{1.933186in}{2.100289in}}%
\pgfpathcurveto{\pgfqpoint{1.924950in}{2.100289in}}{\pgfqpoint{1.917050in}{2.097016in}}{\pgfqpoint{1.911226in}{2.091192in}}%
\pgfpathcurveto{\pgfqpoint{1.905402in}{2.085368in}}{\pgfqpoint{1.902130in}{2.077468in}}{\pgfqpoint{1.902130in}{2.069232in}}%
\pgfpathcurveto{\pgfqpoint{1.902130in}{2.060996in}}{\pgfqpoint{1.905402in}{2.053096in}}{\pgfqpoint{1.911226in}{2.047272in}}%
\pgfpathcurveto{\pgfqpoint{1.917050in}{2.041448in}}{\pgfqpoint{1.924950in}{2.038176in}}{\pgfqpoint{1.933186in}{2.038176in}}%
\pgfpathclose%
\pgfusepath{stroke,fill}%
\end{pgfscope}%
\begin{pgfscope}%
\pgfpathrectangle{\pgfqpoint{0.100000in}{0.212622in}}{\pgfqpoint{3.696000in}{3.696000in}}%
\pgfusepath{clip}%
\pgfsetbuttcap%
\pgfsetroundjoin%
\definecolor{currentfill}{rgb}{0.121569,0.466667,0.705882}%
\pgfsetfillcolor{currentfill}%
\pgfsetfillopacity{0.316126}%
\pgfsetlinewidth{1.003750pt}%
\definecolor{currentstroke}{rgb}{0.121569,0.466667,0.705882}%
\pgfsetstrokecolor{currentstroke}%
\pgfsetstrokeopacity{0.316126}%
\pgfsetdash{}{0pt}%
\pgfpathmoveto{\pgfqpoint{1.929001in}{2.035778in}}%
\pgfpathcurveto{\pgfqpoint{1.937238in}{2.035778in}}{\pgfqpoint{1.945138in}{2.039051in}}{\pgfqpoint{1.950962in}{2.044875in}}%
\pgfpathcurveto{\pgfqpoint{1.956786in}{2.050698in}}{\pgfqpoint{1.960058in}{2.058599in}}{\pgfqpoint{1.960058in}{2.066835in}}%
\pgfpathcurveto{\pgfqpoint{1.960058in}{2.075071in}}{\pgfqpoint{1.956786in}{2.082971in}}{\pgfqpoint{1.950962in}{2.088795in}}%
\pgfpathcurveto{\pgfqpoint{1.945138in}{2.094619in}}{\pgfqpoint{1.937238in}{2.097891in}}{\pgfqpoint{1.929001in}{2.097891in}}%
\pgfpathcurveto{\pgfqpoint{1.920765in}{2.097891in}}{\pgfqpoint{1.912865in}{2.094619in}}{\pgfqpoint{1.907041in}{2.088795in}}%
\pgfpathcurveto{\pgfqpoint{1.901217in}{2.082971in}}{\pgfqpoint{1.897945in}{2.075071in}}{\pgfqpoint{1.897945in}{2.066835in}}%
\pgfpathcurveto{\pgfqpoint{1.897945in}{2.058599in}}{\pgfqpoint{1.901217in}{2.050698in}}{\pgfqpoint{1.907041in}{2.044875in}}%
\pgfpathcurveto{\pgfqpoint{1.912865in}{2.039051in}}{\pgfqpoint{1.920765in}{2.035778in}}{\pgfqpoint{1.929001in}{2.035778in}}%
\pgfpathclose%
\pgfusepath{stroke,fill}%
\end{pgfscope}%
\begin{pgfscope}%
\pgfpathrectangle{\pgfqpoint{0.100000in}{0.212622in}}{\pgfqpoint{3.696000in}{3.696000in}}%
\pgfusepath{clip}%
\pgfsetbuttcap%
\pgfsetroundjoin%
\definecolor{currentfill}{rgb}{0.121569,0.466667,0.705882}%
\pgfsetfillcolor{currentfill}%
\pgfsetfillopacity{0.316663}%
\pgfsetlinewidth{1.003750pt}%
\definecolor{currentstroke}{rgb}{0.121569,0.466667,0.705882}%
\pgfsetstrokecolor{currentstroke}%
\pgfsetstrokeopacity{0.316663}%
\pgfsetdash{}{0pt}%
\pgfpathmoveto{\pgfqpoint{1.926770in}{2.033904in}}%
\pgfpathcurveto{\pgfqpoint{1.935006in}{2.033904in}}{\pgfqpoint{1.942906in}{2.037176in}}{\pgfqpoint{1.948730in}{2.043000in}}%
\pgfpathcurveto{\pgfqpoint{1.954554in}{2.048824in}}{\pgfqpoint{1.957826in}{2.056724in}}{\pgfqpoint{1.957826in}{2.064961in}}%
\pgfpathcurveto{\pgfqpoint{1.957826in}{2.073197in}}{\pgfqpoint{1.954554in}{2.081097in}}{\pgfqpoint{1.948730in}{2.086921in}}%
\pgfpathcurveto{\pgfqpoint{1.942906in}{2.092745in}}{\pgfqpoint{1.935006in}{2.096017in}}{\pgfqpoint{1.926770in}{2.096017in}}%
\pgfpathcurveto{\pgfqpoint{1.918534in}{2.096017in}}{\pgfqpoint{1.910634in}{2.092745in}}{\pgfqpoint{1.904810in}{2.086921in}}%
\pgfpathcurveto{\pgfqpoint{1.898986in}{2.081097in}}{\pgfqpoint{1.895713in}{2.073197in}}{\pgfqpoint{1.895713in}{2.064961in}}%
\pgfpathcurveto{\pgfqpoint{1.895713in}{2.056724in}}{\pgfqpoint{1.898986in}{2.048824in}}{\pgfqpoint{1.904810in}{2.043000in}}%
\pgfpathcurveto{\pgfqpoint{1.910634in}{2.037176in}}{\pgfqpoint{1.918534in}{2.033904in}}{\pgfqpoint{1.926770in}{2.033904in}}%
\pgfpathclose%
\pgfusepath{stroke,fill}%
\end{pgfscope}%
\begin{pgfscope}%
\pgfpathrectangle{\pgfqpoint{0.100000in}{0.212622in}}{\pgfqpoint{3.696000in}{3.696000in}}%
\pgfusepath{clip}%
\pgfsetbuttcap%
\pgfsetroundjoin%
\definecolor{currentfill}{rgb}{0.121569,0.466667,0.705882}%
\pgfsetfillcolor{currentfill}%
\pgfsetfillopacity{0.317147}%
\pgfsetlinewidth{1.003750pt}%
\definecolor{currentstroke}{rgb}{0.121569,0.466667,0.705882}%
\pgfsetstrokecolor{currentstroke}%
\pgfsetstrokeopacity{0.317147}%
\pgfsetdash{}{0pt}%
\pgfpathmoveto{\pgfqpoint{1.649876in}{2.111205in}}%
\pgfpathcurveto{\pgfqpoint{1.658112in}{2.111205in}}{\pgfqpoint{1.666012in}{2.114477in}}{\pgfqpoint{1.671836in}{2.120301in}}%
\pgfpathcurveto{\pgfqpoint{1.677660in}{2.126125in}}{\pgfqpoint{1.680932in}{2.134025in}}{\pgfqpoint{1.680932in}{2.142261in}}%
\pgfpathcurveto{\pgfqpoint{1.680932in}{2.150497in}}{\pgfqpoint{1.677660in}{2.158397in}}{\pgfqpoint{1.671836in}{2.164221in}}%
\pgfpathcurveto{\pgfqpoint{1.666012in}{2.170045in}}{\pgfqpoint{1.658112in}{2.173318in}}{\pgfqpoint{1.649876in}{2.173318in}}%
\pgfpathcurveto{\pgfqpoint{1.641639in}{2.173318in}}{\pgfqpoint{1.633739in}{2.170045in}}{\pgfqpoint{1.627915in}{2.164221in}}%
\pgfpathcurveto{\pgfqpoint{1.622092in}{2.158397in}}{\pgfqpoint{1.618819in}{2.150497in}}{\pgfqpoint{1.618819in}{2.142261in}}%
\pgfpathcurveto{\pgfqpoint{1.618819in}{2.134025in}}{\pgfqpoint{1.622092in}{2.126125in}}{\pgfqpoint{1.627915in}{2.120301in}}%
\pgfpathcurveto{\pgfqpoint{1.633739in}{2.114477in}}{\pgfqpoint{1.641639in}{2.111205in}}{\pgfqpoint{1.649876in}{2.111205in}}%
\pgfpathclose%
\pgfusepath{stroke,fill}%
\end{pgfscope}%
\begin{pgfscope}%
\pgfpathrectangle{\pgfqpoint{0.100000in}{0.212622in}}{\pgfqpoint{3.696000in}{3.696000in}}%
\pgfusepath{clip}%
\pgfsetbuttcap%
\pgfsetroundjoin%
\definecolor{currentfill}{rgb}{0.121569,0.466667,0.705882}%
\pgfsetfillcolor{currentfill}%
\pgfsetfillopacity{0.317450}%
\pgfsetlinewidth{1.003750pt}%
\definecolor{currentstroke}{rgb}{0.121569,0.466667,0.705882}%
\pgfsetstrokecolor{currentstroke}%
\pgfsetstrokeopacity{0.317450}%
\pgfsetdash{}{0pt}%
\pgfpathmoveto{\pgfqpoint{1.921269in}{2.031309in}}%
\pgfpathcurveto{\pgfqpoint{1.929505in}{2.031309in}}{\pgfqpoint{1.937405in}{2.034581in}}{\pgfqpoint{1.943229in}{2.040405in}}%
\pgfpathcurveto{\pgfqpoint{1.949053in}{2.046229in}}{\pgfqpoint{1.952325in}{2.054129in}}{\pgfqpoint{1.952325in}{2.062365in}}%
\pgfpathcurveto{\pgfqpoint{1.952325in}{2.070601in}}{\pgfqpoint{1.949053in}{2.078501in}}{\pgfqpoint{1.943229in}{2.084325in}}%
\pgfpathcurveto{\pgfqpoint{1.937405in}{2.090149in}}{\pgfqpoint{1.929505in}{2.093422in}}{\pgfqpoint{1.921269in}{2.093422in}}%
\pgfpathcurveto{\pgfqpoint{1.913032in}{2.093422in}}{\pgfqpoint{1.905132in}{2.090149in}}{\pgfqpoint{1.899308in}{2.084325in}}%
\pgfpathcurveto{\pgfqpoint{1.893484in}{2.078501in}}{\pgfqpoint{1.890212in}{2.070601in}}{\pgfqpoint{1.890212in}{2.062365in}}%
\pgfpathcurveto{\pgfqpoint{1.890212in}{2.054129in}}{\pgfqpoint{1.893484in}{2.046229in}}{\pgfqpoint{1.899308in}{2.040405in}}%
\pgfpathcurveto{\pgfqpoint{1.905132in}{2.034581in}}{\pgfqpoint{1.913032in}{2.031309in}}{\pgfqpoint{1.921269in}{2.031309in}}%
\pgfpathclose%
\pgfusepath{stroke,fill}%
\end{pgfscope}%
\begin{pgfscope}%
\pgfpathrectangle{\pgfqpoint{0.100000in}{0.212622in}}{\pgfqpoint{3.696000in}{3.696000in}}%
\pgfusepath{clip}%
\pgfsetbuttcap%
\pgfsetroundjoin%
\definecolor{currentfill}{rgb}{0.121569,0.466667,0.705882}%
\pgfsetfillcolor{currentfill}%
\pgfsetfillopacity{0.318175}%
\pgfsetlinewidth{1.003750pt}%
\definecolor{currentstroke}{rgb}{0.121569,0.466667,0.705882}%
\pgfsetstrokecolor{currentstroke}%
\pgfsetstrokeopacity{0.318175}%
\pgfsetdash{}{0pt}%
\pgfpathmoveto{\pgfqpoint{1.918601in}{2.028838in}}%
\pgfpathcurveto{\pgfqpoint{1.926837in}{2.028838in}}{\pgfqpoint{1.934737in}{2.032110in}}{\pgfqpoint{1.940561in}{2.037934in}}%
\pgfpathcurveto{\pgfqpoint{1.946385in}{2.043758in}}{\pgfqpoint{1.949657in}{2.051658in}}{\pgfqpoint{1.949657in}{2.059894in}}%
\pgfpathcurveto{\pgfqpoint{1.949657in}{2.068131in}}{\pgfqpoint{1.946385in}{2.076031in}}{\pgfqpoint{1.940561in}{2.081855in}}%
\pgfpathcurveto{\pgfqpoint{1.934737in}{2.087678in}}{\pgfqpoint{1.926837in}{2.090951in}}{\pgfqpoint{1.918601in}{2.090951in}}%
\pgfpathcurveto{\pgfqpoint{1.910365in}{2.090951in}}{\pgfqpoint{1.902465in}{2.087678in}}{\pgfqpoint{1.896641in}{2.081855in}}%
\pgfpathcurveto{\pgfqpoint{1.890817in}{2.076031in}}{\pgfqpoint{1.887544in}{2.068131in}}{\pgfqpoint{1.887544in}{2.059894in}}%
\pgfpathcurveto{\pgfqpoint{1.887544in}{2.051658in}}{\pgfqpoint{1.890817in}{2.043758in}}{\pgfqpoint{1.896641in}{2.037934in}}%
\pgfpathcurveto{\pgfqpoint{1.902465in}{2.032110in}}{\pgfqpoint{1.910365in}{2.028838in}}{\pgfqpoint{1.918601in}{2.028838in}}%
\pgfpathclose%
\pgfusepath{stroke,fill}%
\end{pgfscope}%
\begin{pgfscope}%
\pgfpathrectangle{\pgfqpoint{0.100000in}{0.212622in}}{\pgfqpoint{3.696000in}{3.696000in}}%
\pgfusepath{clip}%
\pgfsetbuttcap%
\pgfsetroundjoin%
\definecolor{currentfill}{rgb}{0.121569,0.466667,0.705882}%
\pgfsetfillcolor{currentfill}%
\pgfsetfillopacity{0.319292}%
\pgfsetlinewidth{1.003750pt}%
\definecolor{currentstroke}{rgb}{0.121569,0.466667,0.705882}%
\pgfsetstrokecolor{currentstroke}%
\pgfsetstrokeopacity{0.319292}%
\pgfsetdash{}{0pt}%
\pgfpathmoveto{\pgfqpoint{1.911258in}{2.026634in}}%
\pgfpathcurveto{\pgfqpoint{1.919494in}{2.026634in}}{\pgfqpoint{1.927394in}{2.029906in}}{\pgfqpoint{1.933218in}{2.035730in}}%
\pgfpathcurveto{\pgfqpoint{1.939042in}{2.041554in}}{\pgfqpoint{1.942314in}{2.049454in}}{\pgfqpoint{1.942314in}{2.057690in}}%
\pgfpathcurveto{\pgfqpoint{1.942314in}{2.065927in}}{\pgfqpoint{1.939042in}{2.073827in}}{\pgfqpoint{1.933218in}{2.079651in}}%
\pgfpathcurveto{\pgfqpoint{1.927394in}{2.085475in}}{\pgfqpoint{1.919494in}{2.088747in}}{\pgfqpoint{1.911258in}{2.088747in}}%
\pgfpathcurveto{\pgfqpoint{1.903022in}{2.088747in}}{\pgfqpoint{1.895122in}{2.085475in}}{\pgfqpoint{1.889298in}{2.079651in}}%
\pgfpathcurveto{\pgfqpoint{1.883474in}{2.073827in}}{\pgfqpoint{1.880201in}{2.065927in}}{\pgfqpoint{1.880201in}{2.057690in}}%
\pgfpathcurveto{\pgfqpoint{1.880201in}{2.049454in}}{\pgfqpoint{1.883474in}{2.041554in}}{\pgfqpoint{1.889298in}{2.035730in}}%
\pgfpathcurveto{\pgfqpoint{1.895122in}{2.029906in}}{\pgfqpoint{1.903022in}{2.026634in}}{\pgfqpoint{1.911258in}{2.026634in}}%
\pgfpathclose%
\pgfusepath{stroke,fill}%
\end{pgfscope}%
\begin{pgfscope}%
\pgfpathrectangle{\pgfqpoint{0.100000in}{0.212622in}}{\pgfqpoint{3.696000in}{3.696000in}}%
\pgfusepath{clip}%
\pgfsetbuttcap%
\pgfsetroundjoin%
\definecolor{currentfill}{rgb}{0.121569,0.466667,0.705882}%
\pgfsetfillcolor{currentfill}%
\pgfsetfillopacity{0.319455}%
\pgfsetlinewidth{1.003750pt}%
\definecolor{currentstroke}{rgb}{0.121569,0.466667,0.705882}%
\pgfsetstrokecolor{currentstroke}%
\pgfsetstrokeopacity{0.319455}%
\pgfsetdash{}{0pt}%
\pgfpathmoveto{\pgfqpoint{1.617569in}{2.106523in}}%
\pgfpathcurveto{\pgfqpoint{1.625805in}{2.106523in}}{\pgfqpoint{1.633706in}{2.109796in}}{\pgfqpoint{1.639529in}{2.115620in}}%
\pgfpathcurveto{\pgfqpoint{1.645353in}{2.121444in}}{\pgfqpoint{1.648626in}{2.129344in}}{\pgfqpoint{1.648626in}{2.137580in}}%
\pgfpathcurveto{\pgfqpoint{1.648626in}{2.145816in}}{\pgfqpoint{1.645353in}{2.153716in}}{\pgfqpoint{1.639529in}{2.159540in}}%
\pgfpathcurveto{\pgfqpoint{1.633706in}{2.165364in}}{\pgfqpoint{1.625805in}{2.168636in}}{\pgfqpoint{1.617569in}{2.168636in}}%
\pgfpathcurveto{\pgfqpoint{1.609333in}{2.168636in}}{\pgfqpoint{1.601433in}{2.165364in}}{\pgfqpoint{1.595609in}{2.159540in}}%
\pgfpathcurveto{\pgfqpoint{1.589785in}{2.153716in}}{\pgfqpoint{1.586513in}{2.145816in}}{\pgfqpoint{1.586513in}{2.137580in}}%
\pgfpathcurveto{\pgfqpoint{1.586513in}{2.129344in}}{\pgfqpoint{1.589785in}{2.121444in}}{\pgfqpoint{1.595609in}{2.115620in}}%
\pgfpathcurveto{\pgfqpoint{1.601433in}{2.109796in}}{\pgfqpoint{1.609333in}{2.106523in}}{\pgfqpoint{1.617569in}{2.106523in}}%
\pgfpathclose%
\pgfusepath{stroke,fill}%
\end{pgfscope}%
\begin{pgfscope}%
\pgfpathrectangle{\pgfqpoint{0.100000in}{0.212622in}}{\pgfqpoint{3.696000in}{3.696000in}}%
\pgfusepath{clip}%
\pgfsetbuttcap%
\pgfsetroundjoin%
\definecolor{currentfill}{rgb}{0.121569,0.466667,0.705882}%
\pgfsetfillcolor{currentfill}%
\pgfsetfillopacity{0.320333}%
\pgfsetlinewidth{1.003750pt}%
\definecolor{currentstroke}{rgb}{0.121569,0.466667,0.705882}%
\pgfsetstrokecolor{currentstroke}%
\pgfsetstrokeopacity{0.320333}%
\pgfsetdash{}{0pt}%
\pgfpathmoveto{\pgfqpoint{1.908450in}{2.023280in}}%
\pgfpathcurveto{\pgfqpoint{1.916687in}{2.023280in}}{\pgfqpoint{1.924587in}{2.026553in}}{\pgfqpoint{1.930411in}{2.032376in}}%
\pgfpathcurveto{\pgfqpoint{1.936235in}{2.038200in}}{\pgfqpoint{1.939507in}{2.046100in}}{\pgfqpoint{1.939507in}{2.054337in}}%
\pgfpathcurveto{\pgfqpoint{1.939507in}{2.062573in}}{\pgfqpoint{1.936235in}{2.070473in}}{\pgfqpoint{1.930411in}{2.076297in}}%
\pgfpathcurveto{\pgfqpoint{1.924587in}{2.082121in}}{\pgfqpoint{1.916687in}{2.085393in}}{\pgfqpoint{1.908450in}{2.085393in}}%
\pgfpathcurveto{\pgfqpoint{1.900214in}{2.085393in}}{\pgfqpoint{1.892314in}{2.082121in}}{\pgfqpoint{1.886490in}{2.076297in}}%
\pgfpathcurveto{\pgfqpoint{1.880666in}{2.070473in}}{\pgfqpoint{1.877394in}{2.062573in}}{\pgfqpoint{1.877394in}{2.054337in}}%
\pgfpathcurveto{\pgfqpoint{1.877394in}{2.046100in}}{\pgfqpoint{1.880666in}{2.038200in}}{\pgfqpoint{1.886490in}{2.032376in}}%
\pgfpathcurveto{\pgfqpoint{1.892314in}{2.026553in}}{\pgfqpoint{1.900214in}{2.023280in}}{\pgfqpoint{1.908450in}{2.023280in}}%
\pgfpathclose%
\pgfusepath{stroke,fill}%
\end{pgfscope}%
\begin{pgfscope}%
\pgfpathrectangle{\pgfqpoint{0.100000in}{0.212622in}}{\pgfqpoint{3.696000in}{3.696000in}}%
\pgfusepath{clip}%
\pgfsetbuttcap%
\pgfsetroundjoin%
\definecolor{currentfill}{rgb}{0.121569,0.466667,0.705882}%
\pgfsetfillcolor{currentfill}%
\pgfsetfillopacity{0.320681}%
\pgfsetlinewidth{1.003750pt}%
\definecolor{currentstroke}{rgb}{0.121569,0.466667,0.705882}%
\pgfsetstrokecolor{currentstroke}%
\pgfsetstrokeopacity{0.320681}%
\pgfsetdash{}{0pt}%
\pgfpathmoveto{\pgfqpoint{1.596793in}{2.113279in}}%
\pgfpathcurveto{\pgfqpoint{1.605029in}{2.113279in}}{\pgfqpoint{1.612929in}{2.116551in}}{\pgfqpoint{1.618753in}{2.122375in}}%
\pgfpathcurveto{\pgfqpoint{1.624577in}{2.128199in}}{\pgfqpoint{1.627849in}{2.136099in}}{\pgfqpoint{1.627849in}{2.144335in}}%
\pgfpathcurveto{\pgfqpoint{1.627849in}{2.152571in}}{\pgfqpoint{1.624577in}{2.160471in}}{\pgfqpoint{1.618753in}{2.166295in}}%
\pgfpathcurveto{\pgfqpoint{1.612929in}{2.172119in}}{\pgfqpoint{1.605029in}{2.175392in}}{\pgfqpoint{1.596793in}{2.175392in}}%
\pgfpathcurveto{\pgfqpoint{1.588557in}{2.175392in}}{\pgfqpoint{1.580657in}{2.172119in}}{\pgfqpoint{1.574833in}{2.166295in}}%
\pgfpathcurveto{\pgfqpoint{1.569009in}{2.160471in}}{\pgfqpoint{1.565736in}{2.152571in}}{\pgfqpoint{1.565736in}{2.144335in}}%
\pgfpathcurveto{\pgfqpoint{1.565736in}{2.136099in}}{\pgfqpoint{1.569009in}{2.128199in}}{\pgfqpoint{1.574833in}{2.122375in}}%
\pgfpathcurveto{\pgfqpoint{1.580657in}{2.116551in}}{\pgfqpoint{1.588557in}{2.113279in}}{\pgfqpoint{1.596793in}{2.113279in}}%
\pgfpathclose%
\pgfusepath{stroke,fill}%
\end{pgfscope}%
\begin{pgfscope}%
\pgfpathrectangle{\pgfqpoint{0.100000in}{0.212622in}}{\pgfqpoint{3.696000in}{3.696000in}}%
\pgfusepath{clip}%
\pgfsetbuttcap%
\pgfsetroundjoin%
\definecolor{currentfill}{rgb}{0.121569,0.466667,0.705882}%
\pgfsetfillcolor{currentfill}%
\pgfsetfillopacity{0.320889}%
\pgfsetlinewidth{1.003750pt}%
\definecolor{currentstroke}{rgb}{0.121569,0.466667,0.705882}%
\pgfsetstrokecolor{currentstroke}%
\pgfsetstrokeopacity{0.320889}%
\pgfsetdash{}{0pt}%
\pgfpathmoveto{\pgfqpoint{1.904610in}{2.022063in}}%
\pgfpathcurveto{\pgfqpoint{1.912846in}{2.022063in}}{\pgfqpoint{1.920746in}{2.025335in}}{\pgfqpoint{1.926570in}{2.031159in}}%
\pgfpathcurveto{\pgfqpoint{1.932394in}{2.036983in}}{\pgfqpoint{1.935666in}{2.044883in}}{\pgfqpoint{1.935666in}{2.053119in}}%
\pgfpathcurveto{\pgfqpoint{1.935666in}{2.061356in}}{\pgfqpoint{1.932394in}{2.069256in}}{\pgfqpoint{1.926570in}{2.075080in}}%
\pgfpathcurveto{\pgfqpoint{1.920746in}{2.080904in}}{\pgfqpoint{1.912846in}{2.084176in}}{\pgfqpoint{1.904610in}{2.084176in}}%
\pgfpathcurveto{\pgfqpoint{1.896374in}{2.084176in}}{\pgfqpoint{1.888474in}{2.080904in}}{\pgfqpoint{1.882650in}{2.075080in}}%
\pgfpathcurveto{\pgfqpoint{1.876826in}{2.069256in}}{\pgfqpoint{1.873553in}{2.061356in}}{\pgfqpoint{1.873553in}{2.053119in}}%
\pgfpathcurveto{\pgfqpoint{1.873553in}{2.044883in}}{\pgfqpoint{1.876826in}{2.036983in}}{\pgfqpoint{1.882650in}{2.031159in}}%
\pgfpathcurveto{\pgfqpoint{1.888474in}{2.025335in}}{\pgfqpoint{1.896374in}{2.022063in}}{\pgfqpoint{1.904610in}{2.022063in}}%
\pgfpathclose%
\pgfusepath{stroke,fill}%
\end{pgfscope}%
\begin{pgfscope}%
\pgfpathrectangle{\pgfqpoint{0.100000in}{0.212622in}}{\pgfqpoint{3.696000in}{3.696000in}}%
\pgfusepath{clip}%
\pgfsetbuttcap%
\pgfsetroundjoin%
\definecolor{currentfill}{rgb}{0.121569,0.466667,0.705882}%
\pgfsetfillcolor{currentfill}%
\pgfsetfillopacity{0.321163}%
\pgfsetlinewidth{1.003750pt}%
\definecolor{currentstroke}{rgb}{0.121569,0.466667,0.705882}%
\pgfsetstrokecolor{currentstroke}%
\pgfsetstrokeopacity{0.321163}%
\pgfsetdash{}{0pt}%
\pgfpathmoveto{\pgfqpoint{1.903793in}{2.021133in}}%
\pgfpathcurveto{\pgfqpoint{1.912029in}{2.021133in}}{\pgfqpoint{1.919929in}{2.024405in}}{\pgfqpoint{1.925753in}{2.030229in}}%
\pgfpathcurveto{\pgfqpoint{1.931577in}{2.036053in}}{\pgfqpoint{1.934849in}{2.043953in}}{\pgfqpoint{1.934849in}{2.052189in}}%
\pgfpathcurveto{\pgfqpoint{1.934849in}{2.060425in}}{\pgfqpoint{1.931577in}{2.068326in}}{\pgfqpoint{1.925753in}{2.074149in}}%
\pgfpathcurveto{\pgfqpoint{1.919929in}{2.079973in}}{\pgfqpoint{1.912029in}{2.083246in}}{\pgfqpoint{1.903793in}{2.083246in}}%
\pgfpathcurveto{\pgfqpoint{1.895556in}{2.083246in}}{\pgfqpoint{1.887656in}{2.079973in}}{\pgfqpoint{1.881832in}{2.074149in}}%
\pgfpathcurveto{\pgfqpoint{1.876008in}{2.068326in}}{\pgfqpoint{1.872736in}{2.060425in}}{\pgfqpoint{1.872736in}{2.052189in}}%
\pgfpathcurveto{\pgfqpoint{1.872736in}{2.043953in}}{\pgfqpoint{1.876008in}{2.036053in}}{\pgfqpoint{1.881832in}{2.030229in}}%
\pgfpathcurveto{\pgfqpoint{1.887656in}{2.024405in}}{\pgfqpoint{1.895556in}{2.021133in}}{\pgfqpoint{1.903793in}{2.021133in}}%
\pgfpathclose%
\pgfusepath{stroke,fill}%
\end{pgfscope}%
\begin{pgfscope}%
\pgfpathrectangle{\pgfqpoint{0.100000in}{0.212622in}}{\pgfqpoint{3.696000in}{3.696000in}}%
\pgfusepath{clip}%
\pgfsetbuttcap%
\pgfsetroundjoin%
\definecolor{currentfill}{rgb}{0.121569,0.466667,0.705882}%
\pgfsetfillcolor{currentfill}%
\pgfsetfillopacity{0.321549}%
\pgfsetlinewidth{1.003750pt}%
\definecolor{currentstroke}{rgb}{0.121569,0.466667,0.705882}%
\pgfsetstrokecolor{currentstroke}%
\pgfsetstrokeopacity{0.321549}%
\pgfsetdash{}{0pt}%
\pgfpathmoveto{\pgfqpoint{1.900959in}{2.020667in}}%
\pgfpathcurveto{\pgfqpoint{1.909195in}{2.020667in}}{\pgfqpoint{1.917095in}{2.023940in}}{\pgfqpoint{1.922919in}{2.029764in}}%
\pgfpathcurveto{\pgfqpoint{1.928743in}{2.035588in}}{\pgfqpoint{1.932015in}{2.043488in}}{\pgfqpoint{1.932015in}{2.051724in}}%
\pgfpathcurveto{\pgfqpoint{1.932015in}{2.059960in}}{\pgfqpoint{1.928743in}{2.067860in}}{\pgfqpoint{1.922919in}{2.073684in}}%
\pgfpathcurveto{\pgfqpoint{1.917095in}{2.079508in}}{\pgfqpoint{1.909195in}{2.082780in}}{\pgfqpoint{1.900959in}{2.082780in}}%
\pgfpathcurveto{\pgfqpoint{1.892722in}{2.082780in}}{\pgfqpoint{1.884822in}{2.079508in}}{\pgfqpoint{1.878998in}{2.073684in}}%
\pgfpathcurveto{\pgfqpoint{1.873174in}{2.067860in}}{\pgfqpoint{1.869902in}{2.059960in}}{\pgfqpoint{1.869902in}{2.051724in}}%
\pgfpathcurveto{\pgfqpoint{1.869902in}{2.043488in}}{\pgfqpoint{1.873174in}{2.035588in}}{\pgfqpoint{1.878998in}{2.029764in}}%
\pgfpathcurveto{\pgfqpoint{1.884822in}{2.023940in}}{\pgfqpoint{1.892722in}{2.020667in}}{\pgfqpoint{1.900959in}{2.020667in}}%
\pgfpathclose%
\pgfusepath{stroke,fill}%
\end{pgfscope}%
\begin{pgfscope}%
\pgfpathrectangle{\pgfqpoint{0.100000in}{0.212622in}}{\pgfqpoint{3.696000in}{3.696000in}}%
\pgfusepath{clip}%
\pgfsetbuttcap%
\pgfsetroundjoin%
\definecolor{currentfill}{rgb}{0.121569,0.466667,0.705882}%
\pgfsetfillcolor{currentfill}%
\pgfsetfillopacity{0.321945}%
\pgfsetlinewidth{1.003750pt}%
\definecolor{currentstroke}{rgb}{0.121569,0.466667,0.705882}%
\pgfsetstrokecolor{currentstroke}%
\pgfsetstrokeopacity{0.321945}%
\pgfsetdash{}{0pt}%
\pgfpathmoveto{\pgfqpoint{1.575770in}{2.110658in}}%
\pgfpathcurveto{\pgfqpoint{1.584006in}{2.110658in}}{\pgfqpoint{1.591906in}{2.113930in}}{\pgfqpoint{1.597730in}{2.119754in}}%
\pgfpathcurveto{\pgfqpoint{1.603554in}{2.125578in}}{\pgfqpoint{1.606827in}{2.133478in}}{\pgfqpoint{1.606827in}{2.141715in}}%
\pgfpathcurveto{\pgfqpoint{1.606827in}{2.149951in}}{\pgfqpoint{1.603554in}{2.157851in}}{\pgfqpoint{1.597730in}{2.163675in}}%
\pgfpathcurveto{\pgfqpoint{1.591906in}{2.169499in}}{\pgfqpoint{1.584006in}{2.172771in}}{\pgfqpoint{1.575770in}{2.172771in}}%
\pgfpathcurveto{\pgfqpoint{1.567534in}{2.172771in}}{\pgfqpoint{1.559634in}{2.169499in}}{\pgfqpoint{1.553810in}{2.163675in}}%
\pgfpathcurveto{\pgfqpoint{1.547986in}{2.157851in}}{\pgfqpoint{1.544714in}{2.149951in}}{\pgfqpoint{1.544714in}{2.141715in}}%
\pgfpathcurveto{\pgfqpoint{1.544714in}{2.133478in}}{\pgfqpoint{1.547986in}{2.125578in}}{\pgfqpoint{1.553810in}{2.119754in}}%
\pgfpathcurveto{\pgfqpoint{1.559634in}{2.113930in}}{\pgfqpoint{1.567534in}{2.110658in}}{\pgfqpoint{1.575770in}{2.110658in}}%
\pgfpathclose%
\pgfusepath{stroke,fill}%
\end{pgfscope}%
\begin{pgfscope}%
\pgfpathrectangle{\pgfqpoint{0.100000in}{0.212622in}}{\pgfqpoint{3.696000in}{3.696000in}}%
\pgfusepath{clip}%
\pgfsetbuttcap%
\pgfsetroundjoin%
\definecolor{currentfill}{rgb}{0.121569,0.466667,0.705882}%
\pgfsetfillcolor{currentfill}%
\pgfsetfillopacity{0.322113}%
\pgfsetlinewidth{1.003750pt}%
\definecolor{currentstroke}{rgb}{0.121569,0.466667,0.705882}%
\pgfsetstrokecolor{currentstroke}%
\pgfsetstrokeopacity{0.322113}%
\pgfsetdash{}{0pt}%
\pgfpathmoveto{\pgfqpoint{1.897273in}{2.016382in}}%
\pgfpathcurveto{\pgfqpoint{1.905509in}{2.016382in}}{\pgfqpoint{1.913409in}{2.019654in}}{\pgfqpoint{1.919233in}{2.025478in}}%
\pgfpathcurveto{\pgfqpoint{1.925057in}{2.031302in}}{\pgfqpoint{1.928329in}{2.039202in}}{\pgfqpoint{1.928329in}{2.047438in}}%
\pgfpathcurveto{\pgfqpoint{1.928329in}{2.055674in}}{\pgfqpoint{1.925057in}{2.063574in}}{\pgfqpoint{1.919233in}{2.069398in}}%
\pgfpathcurveto{\pgfqpoint{1.913409in}{2.075222in}}{\pgfqpoint{1.905509in}{2.078495in}}{\pgfqpoint{1.897273in}{2.078495in}}%
\pgfpathcurveto{\pgfqpoint{1.889036in}{2.078495in}}{\pgfqpoint{1.881136in}{2.075222in}}{\pgfqpoint{1.875312in}{2.069398in}}%
\pgfpathcurveto{\pgfqpoint{1.869488in}{2.063574in}}{\pgfqpoint{1.866216in}{2.055674in}}{\pgfqpoint{1.866216in}{2.047438in}}%
\pgfpathcurveto{\pgfqpoint{1.866216in}{2.039202in}}{\pgfqpoint{1.869488in}{2.031302in}}{\pgfqpoint{1.875312in}{2.025478in}}%
\pgfpathcurveto{\pgfqpoint{1.881136in}{2.019654in}}{\pgfqpoint{1.889036in}{2.016382in}}{\pgfqpoint{1.897273in}{2.016382in}}%
\pgfpathclose%
\pgfusepath{stroke,fill}%
\end{pgfscope}%
\begin{pgfscope}%
\pgfpathrectangle{\pgfqpoint{0.100000in}{0.212622in}}{\pgfqpoint{3.696000in}{3.696000in}}%
\pgfusepath{clip}%
\pgfsetbuttcap%
\pgfsetroundjoin%
\definecolor{currentfill}{rgb}{0.121569,0.466667,0.705882}%
\pgfsetfillcolor{currentfill}%
\pgfsetfillopacity{0.322631}%
\pgfsetlinewidth{1.003750pt}%
\definecolor{currentstroke}{rgb}{0.121569,0.466667,0.705882}%
\pgfsetstrokecolor{currentstroke}%
\pgfsetstrokeopacity{0.322631}%
\pgfsetdash{}{0pt}%
\pgfpathmoveto{\pgfqpoint{1.895043in}{2.014973in}}%
\pgfpathcurveto{\pgfqpoint{1.903280in}{2.014973in}}{\pgfqpoint{1.911180in}{2.018245in}}{\pgfqpoint{1.917004in}{2.024069in}}%
\pgfpathcurveto{\pgfqpoint{1.922828in}{2.029893in}}{\pgfqpoint{1.926100in}{2.037793in}}{\pgfqpoint{1.926100in}{2.046029in}}%
\pgfpathcurveto{\pgfqpoint{1.926100in}{2.054266in}}{\pgfqpoint{1.922828in}{2.062166in}}{\pgfqpoint{1.917004in}{2.067990in}}%
\pgfpathcurveto{\pgfqpoint{1.911180in}{2.073814in}}{\pgfqpoint{1.903280in}{2.077086in}}{\pgfqpoint{1.895043in}{2.077086in}}%
\pgfpathcurveto{\pgfqpoint{1.886807in}{2.077086in}}{\pgfqpoint{1.878907in}{2.073814in}}{\pgfqpoint{1.873083in}{2.067990in}}%
\pgfpathcurveto{\pgfqpoint{1.867259in}{2.062166in}}{\pgfqpoint{1.863987in}{2.054266in}}{\pgfqpoint{1.863987in}{2.046029in}}%
\pgfpathcurveto{\pgfqpoint{1.863987in}{2.037793in}}{\pgfqpoint{1.867259in}{2.029893in}}{\pgfqpoint{1.873083in}{2.024069in}}%
\pgfpathcurveto{\pgfqpoint{1.878907in}{2.018245in}}{\pgfqpoint{1.886807in}{2.014973in}}{\pgfqpoint{1.895043in}{2.014973in}}%
\pgfpathclose%
\pgfusepath{stroke,fill}%
\end{pgfscope}%
\begin{pgfscope}%
\pgfpathrectangle{\pgfqpoint{0.100000in}{0.212622in}}{\pgfqpoint{3.696000in}{3.696000in}}%
\pgfusepath{clip}%
\pgfsetbuttcap%
\pgfsetroundjoin%
\definecolor{currentfill}{rgb}{0.121569,0.466667,0.705882}%
\pgfsetfillcolor{currentfill}%
\pgfsetfillopacity{0.323306}%
\pgfsetlinewidth{1.003750pt}%
\definecolor{currentstroke}{rgb}{0.121569,0.466667,0.705882}%
\pgfsetstrokecolor{currentstroke}%
\pgfsetstrokeopacity{0.323306}%
\pgfsetdash{}{0pt}%
\pgfpathmoveto{\pgfqpoint{1.889354in}{2.013200in}}%
\pgfpathcurveto{\pgfqpoint{1.897591in}{2.013200in}}{\pgfqpoint{1.905491in}{2.016472in}}{\pgfqpoint{1.911315in}{2.022296in}}%
\pgfpathcurveto{\pgfqpoint{1.917138in}{2.028120in}}{\pgfqpoint{1.920411in}{2.036020in}}{\pgfqpoint{1.920411in}{2.044256in}}%
\pgfpathcurveto{\pgfqpoint{1.920411in}{2.052493in}}{\pgfqpoint{1.917138in}{2.060393in}}{\pgfqpoint{1.911315in}{2.066217in}}%
\pgfpathcurveto{\pgfqpoint{1.905491in}{2.072041in}}{\pgfqpoint{1.897591in}{2.075313in}}{\pgfqpoint{1.889354in}{2.075313in}}%
\pgfpathcurveto{\pgfqpoint{1.881118in}{2.075313in}}{\pgfqpoint{1.873218in}{2.072041in}}{\pgfqpoint{1.867394in}{2.066217in}}%
\pgfpathcurveto{\pgfqpoint{1.861570in}{2.060393in}}{\pgfqpoint{1.858298in}{2.052493in}}{\pgfqpoint{1.858298in}{2.044256in}}%
\pgfpathcurveto{\pgfqpoint{1.858298in}{2.036020in}}{\pgfqpoint{1.861570in}{2.028120in}}{\pgfqpoint{1.867394in}{2.022296in}}%
\pgfpathcurveto{\pgfqpoint{1.873218in}{2.016472in}}{\pgfqpoint{1.881118in}{2.013200in}}{\pgfqpoint{1.889354in}{2.013200in}}%
\pgfpathclose%
\pgfusepath{stroke,fill}%
\end{pgfscope}%
\begin{pgfscope}%
\pgfpathrectangle{\pgfqpoint{0.100000in}{0.212622in}}{\pgfqpoint{3.696000in}{3.696000in}}%
\pgfusepath{clip}%
\pgfsetbuttcap%
\pgfsetroundjoin%
\definecolor{currentfill}{rgb}{0.121569,0.466667,0.705882}%
\pgfsetfillcolor{currentfill}%
\pgfsetfillopacity{0.323733}%
\pgfsetlinewidth{1.003750pt}%
\definecolor{currentstroke}{rgb}{0.121569,0.466667,0.705882}%
\pgfsetstrokecolor{currentstroke}%
\pgfsetstrokeopacity{0.323733}%
\pgfsetdash{}{0pt}%
\pgfpathmoveto{\pgfqpoint{1.888512in}{2.012179in}}%
\pgfpathcurveto{\pgfqpoint{1.896748in}{2.012179in}}{\pgfqpoint{1.904648in}{2.015451in}}{\pgfqpoint{1.910472in}{2.021275in}}%
\pgfpathcurveto{\pgfqpoint{1.916296in}{2.027099in}}{\pgfqpoint{1.919568in}{2.034999in}}{\pgfqpoint{1.919568in}{2.043236in}}%
\pgfpathcurveto{\pgfqpoint{1.919568in}{2.051472in}}{\pgfqpoint{1.916296in}{2.059372in}}{\pgfqpoint{1.910472in}{2.065196in}}%
\pgfpathcurveto{\pgfqpoint{1.904648in}{2.071020in}}{\pgfqpoint{1.896748in}{2.074292in}}{\pgfqpoint{1.888512in}{2.074292in}}%
\pgfpathcurveto{\pgfqpoint{1.880276in}{2.074292in}}{\pgfqpoint{1.872376in}{2.071020in}}{\pgfqpoint{1.866552in}{2.065196in}}%
\pgfpathcurveto{\pgfqpoint{1.860728in}{2.059372in}}{\pgfqpoint{1.857455in}{2.051472in}}{\pgfqpoint{1.857455in}{2.043236in}}%
\pgfpathcurveto{\pgfqpoint{1.857455in}{2.034999in}}{\pgfqpoint{1.860728in}{2.027099in}}{\pgfqpoint{1.866552in}{2.021275in}}%
\pgfpathcurveto{\pgfqpoint{1.872376in}{2.015451in}}{\pgfqpoint{1.880276in}{2.012179in}}{\pgfqpoint{1.888512in}{2.012179in}}%
\pgfpathclose%
\pgfusepath{stroke,fill}%
\end{pgfscope}%
\begin{pgfscope}%
\pgfpathrectangle{\pgfqpoint{0.100000in}{0.212622in}}{\pgfqpoint{3.696000in}{3.696000in}}%
\pgfusepath{clip}%
\pgfsetbuttcap%
\pgfsetroundjoin%
\definecolor{currentfill}{rgb}{0.121569,0.466667,0.705882}%
\pgfsetfillcolor{currentfill}%
\pgfsetfillopacity{0.323852}%
\pgfsetlinewidth{1.003750pt}%
\definecolor{currentstroke}{rgb}{0.121569,0.466667,0.705882}%
\pgfsetstrokecolor{currentstroke}%
\pgfsetstrokeopacity{0.323852}%
\pgfsetdash{}{0pt}%
\pgfpathmoveto{\pgfqpoint{1.546833in}{2.121517in}}%
\pgfpathcurveto{\pgfqpoint{1.555070in}{2.121517in}}{\pgfqpoint{1.562970in}{2.124790in}}{\pgfqpoint{1.568794in}{2.130614in}}%
\pgfpathcurveto{\pgfqpoint{1.574617in}{2.136438in}}{\pgfqpoint{1.577890in}{2.144338in}}{\pgfqpoint{1.577890in}{2.152574in}}%
\pgfpathcurveto{\pgfqpoint{1.577890in}{2.160810in}}{\pgfqpoint{1.574617in}{2.168710in}}{\pgfqpoint{1.568794in}{2.174534in}}%
\pgfpathcurveto{\pgfqpoint{1.562970in}{2.180358in}}{\pgfqpoint{1.555070in}{2.183630in}}{\pgfqpoint{1.546833in}{2.183630in}}%
\pgfpathcurveto{\pgfqpoint{1.538597in}{2.183630in}}{\pgfqpoint{1.530697in}{2.180358in}}{\pgfqpoint{1.524873in}{2.174534in}}%
\pgfpathcurveto{\pgfqpoint{1.519049in}{2.168710in}}{\pgfqpoint{1.515777in}{2.160810in}}{\pgfqpoint{1.515777in}{2.152574in}}%
\pgfpathcurveto{\pgfqpoint{1.515777in}{2.144338in}}{\pgfqpoint{1.519049in}{2.136438in}}{\pgfqpoint{1.524873in}{2.130614in}}%
\pgfpathcurveto{\pgfqpoint{1.530697in}{2.124790in}}{\pgfqpoint{1.538597in}{2.121517in}}{\pgfqpoint{1.546833in}{2.121517in}}%
\pgfpathclose%
\pgfusepath{stroke,fill}%
\end{pgfscope}%
\begin{pgfscope}%
\pgfpathrectangle{\pgfqpoint{0.100000in}{0.212622in}}{\pgfqpoint{3.696000in}{3.696000in}}%
\pgfusepath{clip}%
\pgfsetbuttcap%
\pgfsetroundjoin%
\definecolor{currentfill}{rgb}{0.121569,0.466667,0.705882}%
\pgfsetfillcolor{currentfill}%
\pgfsetfillopacity{0.324122}%
\pgfsetlinewidth{1.003750pt}%
\definecolor{currentstroke}{rgb}{0.121569,0.466667,0.705882}%
\pgfsetstrokecolor{currentstroke}%
\pgfsetstrokeopacity{0.324122}%
\pgfsetdash{}{0pt}%
\pgfpathmoveto{\pgfqpoint{1.885211in}{2.009874in}}%
\pgfpathcurveto{\pgfqpoint{1.893447in}{2.009874in}}{\pgfqpoint{1.901347in}{2.013147in}}{\pgfqpoint{1.907171in}{2.018971in}}%
\pgfpathcurveto{\pgfqpoint{1.912995in}{2.024795in}}{\pgfqpoint{1.916267in}{2.032695in}}{\pgfqpoint{1.916267in}{2.040931in}}%
\pgfpathcurveto{\pgfqpoint{1.916267in}{2.049167in}}{\pgfqpoint{1.912995in}{2.057067in}}{\pgfqpoint{1.907171in}{2.062891in}}%
\pgfpathcurveto{\pgfqpoint{1.901347in}{2.068715in}}{\pgfqpoint{1.893447in}{2.071987in}}{\pgfqpoint{1.885211in}{2.071987in}}%
\pgfpathcurveto{\pgfqpoint{1.876974in}{2.071987in}}{\pgfqpoint{1.869074in}{2.068715in}}{\pgfqpoint{1.863250in}{2.062891in}}%
\pgfpathcurveto{\pgfqpoint{1.857427in}{2.057067in}}{\pgfqpoint{1.854154in}{2.049167in}}{\pgfqpoint{1.854154in}{2.040931in}}%
\pgfpathcurveto{\pgfqpoint{1.854154in}{2.032695in}}{\pgfqpoint{1.857427in}{2.024795in}}{\pgfqpoint{1.863250in}{2.018971in}}%
\pgfpathcurveto{\pgfqpoint{1.869074in}{2.013147in}}{\pgfqpoint{1.876974in}{2.009874in}}{\pgfqpoint{1.885211in}{2.009874in}}%
\pgfpathclose%
\pgfusepath{stroke,fill}%
\end{pgfscope}%
\begin{pgfscope}%
\pgfpathrectangle{\pgfqpoint{0.100000in}{0.212622in}}{\pgfqpoint{3.696000in}{3.696000in}}%
\pgfusepath{clip}%
\pgfsetbuttcap%
\pgfsetroundjoin%
\definecolor{currentfill}{rgb}{0.121569,0.466667,0.705882}%
\pgfsetfillcolor{currentfill}%
\pgfsetfillopacity{0.324741}%
\pgfsetlinewidth{1.003750pt}%
\definecolor{currentstroke}{rgb}{0.121569,0.466667,0.705882}%
\pgfsetstrokecolor{currentstroke}%
\pgfsetstrokeopacity{0.324741}%
\pgfsetdash{}{0pt}%
\pgfpathmoveto{\pgfqpoint{1.879186in}{2.005081in}}%
\pgfpathcurveto{\pgfqpoint{1.887422in}{2.005081in}}{\pgfqpoint{1.895322in}{2.008353in}}{\pgfqpoint{1.901146in}{2.014177in}}%
\pgfpathcurveto{\pgfqpoint{1.906970in}{2.020001in}}{\pgfqpoint{1.910242in}{2.027901in}}{\pgfqpoint{1.910242in}{2.036138in}}%
\pgfpathcurveto{\pgfqpoint{1.910242in}{2.044374in}}{\pgfqpoint{1.906970in}{2.052274in}}{\pgfqpoint{1.901146in}{2.058098in}}%
\pgfpathcurveto{\pgfqpoint{1.895322in}{2.063922in}}{\pgfqpoint{1.887422in}{2.067194in}}{\pgfqpoint{1.879186in}{2.067194in}}%
\pgfpathcurveto{\pgfqpoint{1.870950in}{2.067194in}}{\pgfqpoint{1.863049in}{2.063922in}}{\pgfqpoint{1.857226in}{2.058098in}}%
\pgfpathcurveto{\pgfqpoint{1.851402in}{2.052274in}}{\pgfqpoint{1.848129in}{2.044374in}}{\pgfqpoint{1.848129in}{2.036138in}}%
\pgfpathcurveto{\pgfqpoint{1.848129in}{2.027901in}}{\pgfqpoint{1.851402in}{2.020001in}}{\pgfqpoint{1.857226in}{2.014177in}}%
\pgfpathcurveto{\pgfqpoint{1.863049in}{2.008353in}}{\pgfqpoint{1.870950in}{2.005081in}}{\pgfqpoint{1.879186in}{2.005081in}}%
\pgfpathclose%
\pgfusepath{stroke,fill}%
\end{pgfscope}%
\begin{pgfscope}%
\pgfpathrectangle{\pgfqpoint{0.100000in}{0.212622in}}{\pgfqpoint{3.696000in}{3.696000in}}%
\pgfusepath{clip}%
\pgfsetbuttcap%
\pgfsetroundjoin%
\definecolor{currentfill}{rgb}{0.121569,0.466667,0.705882}%
\pgfsetfillcolor{currentfill}%
\pgfsetfillopacity{0.325575}%
\pgfsetlinewidth{1.003750pt}%
\definecolor{currentstroke}{rgb}{0.121569,0.466667,0.705882}%
\pgfsetstrokecolor{currentstroke}%
\pgfsetstrokeopacity{0.325575}%
\pgfsetdash{}{0pt}%
\pgfpathmoveto{\pgfqpoint{1.875928in}{2.002272in}}%
\pgfpathcurveto{\pgfqpoint{1.884164in}{2.002272in}}{\pgfqpoint{1.892064in}{2.005545in}}{\pgfqpoint{1.897888in}{2.011369in}}%
\pgfpathcurveto{\pgfqpoint{1.903712in}{2.017193in}}{\pgfqpoint{1.906984in}{2.025093in}}{\pgfqpoint{1.906984in}{2.033329in}}%
\pgfpathcurveto{\pgfqpoint{1.906984in}{2.041565in}}{\pgfqpoint{1.903712in}{2.049465in}}{\pgfqpoint{1.897888in}{2.055289in}}%
\pgfpathcurveto{\pgfqpoint{1.892064in}{2.061113in}}{\pgfqpoint{1.884164in}{2.064385in}}{\pgfqpoint{1.875928in}{2.064385in}}%
\pgfpathcurveto{\pgfqpoint{1.867691in}{2.064385in}}{\pgfqpoint{1.859791in}{2.061113in}}{\pgfqpoint{1.853967in}{2.055289in}}%
\pgfpathcurveto{\pgfqpoint{1.848143in}{2.049465in}}{\pgfqpoint{1.844871in}{2.041565in}}{\pgfqpoint{1.844871in}{2.033329in}}%
\pgfpathcurveto{\pgfqpoint{1.844871in}{2.025093in}}{\pgfqpoint{1.848143in}{2.017193in}}{\pgfqpoint{1.853967in}{2.011369in}}%
\pgfpathcurveto{\pgfqpoint{1.859791in}{2.005545in}}{\pgfqpoint{1.867691in}{2.002272in}}{\pgfqpoint{1.875928in}{2.002272in}}%
\pgfpathclose%
\pgfusepath{stroke,fill}%
\end{pgfscope}%
\begin{pgfscope}%
\pgfpathrectangle{\pgfqpoint{0.100000in}{0.212622in}}{\pgfqpoint{3.696000in}{3.696000in}}%
\pgfusepath{clip}%
\pgfsetbuttcap%
\pgfsetroundjoin%
\definecolor{currentfill}{rgb}{0.121569,0.466667,0.705882}%
\pgfsetfillcolor{currentfill}%
\pgfsetfillopacity{0.325751}%
\pgfsetlinewidth{1.003750pt}%
\definecolor{currentstroke}{rgb}{0.121569,0.466667,0.705882}%
\pgfsetstrokecolor{currentstroke}%
\pgfsetstrokeopacity{0.325751}%
\pgfsetdash{}{0pt}%
\pgfpathmoveto{\pgfqpoint{1.874593in}{2.001527in}}%
\pgfpathcurveto{\pgfqpoint{1.882829in}{2.001527in}}{\pgfqpoint{1.890729in}{2.004800in}}{\pgfqpoint{1.896553in}{2.010624in}}%
\pgfpathcurveto{\pgfqpoint{1.902377in}{2.016448in}}{\pgfqpoint{1.905649in}{2.024348in}}{\pgfqpoint{1.905649in}{2.032584in}}%
\pgfpathcurveto{\pgfqpoint{1.905649in}{2.040820in}}{\pgfqpoint{1.902377in}{2.048720in}}{\pgfqpoint{1.896553in}{2.054544in}}%
\pgfpathcurveto{\pgfqpoint{1.890729in}{2.060368in}}{\pgfqpoint{1.882829in}{2.063640in}}{\pgfqpoint{1.874593in}{2.063640in}}%
\pgfpathcurveto{\pgfqpoint{1.866356in}{2.063640in}}{\pgfqpoint{1.858456in}{2.060368in}}{\pgfqpoint{1.852632in}{2.054544in}}%
\pgfpathcurveto{\pgfqpoint{1.846808in}{2.048720in}}{\pgfqpoint{1.843536in}{2.040820in}}{\pgfqpoint{1.843536in}{2.032584in}}%
\pgfpathcurveto{\pgfqpoint{1.843536in}{2.024348in}}{\pgfqpoint{1.846808in}{2.016448in}}{\pgfqpoint{1.852632in}{2.010624in}}%
\pgfpathcurveto{\pgfqpoint{1.858456in}{2.004800in}}{\pgfqpoint{1.866356in}{2.001527in}}{\pgfqpoint{1.874593in}{2.001527in}}%
\pgfpathclose%
\pgfusepath{stroke,fill}%
\end{pgfscope}%
\begin{pgfscope}%
\pgfpathrectangle{\pgfqpoint{0.100000in}{0.212622in}}{\pgfqpoint{3.696000in}{3.696000in}}%
\pgfusepath{clip}%
\pgfsetbuttcap%
\pgfsetroundjoin%
\definecolor{currentfill}{rgb}{0.121569,0.466667,0.705882}%
\pgfsetfillcolor{currentfill}%
\pgfsetfillopacity{0.325868}%
\pgfsetlinewidth{1.003750pt}%
\definecolor{currentstroke}{rgb}{0.121569,0.466667,0.705882}%
\pgfsetstrokecolor{currentstroke}%
\pgfsetstrokeopacity{0.325868}%
\pgfsetdash{}{0pt}%
\pgfpathmoveto{\pgfqpoint{1.517767in}{2.115375in}}%
\pgfpathcurveto{\pgfqpoint{1.526003in}{2.115375in}}{\pgfqpoint{1.533903in}{2.118648in}}{\pgfqpoint{1.539727in}{2.124472in}}%
\pgfpathcurveto{\pgfqpoint{1.545551in}{2.130296in}}{\pgfqpoint{1.548823in}{2.138196in}}{\pgfqpoint{1.548823in}{2.146432in}}%
\pgfpathcurveto{\pgfqpoint{1.548823in}{2.154668in}}{\pgfqpoint{1.545551in}{2.162568in}}{\pgfqpoint{1.539727in}{2.168392in}}%
\pgfpathcurveto{\pgfqpoint{1.533903in}{2.174216in}}{\pgfqpoint{1.526003in}{2.177488in}}{\pgfqpoint{1.517767in}{2.177488in}}%
\pgfpathcurveto{\pgfqpoint{1.509530in}{2.177488in}}{\pgfqpoint{1.501630in}{2.174216in}}{\pgfqpoint{1.495806in}{2.168392in}}%
\pgfpathcurveto{\pgfqpoint{1.489982in}{2.162568in}}{\pgfqpoint{1.486710in}{2.154668in}}{\pgfqpoint{1.486710in}{2.146432in}}%
\pgfpathcurveto{\pgfqpoint{1.486710in}{2.138196in}}{\pgfqpoint{1.489982in}{2.130296in}}{\pgfqpoint{1.495806in}{2.124472in}}%
\pgfpathcurveto{\pgfqpoint{1.501630in}{2.118648in}}{\pgfqpoint{1.509530in}{2.115375in}}{\pgfqpoint{1.517767in}{2.115375in}}%
\pgfpathclose%
\pgfusepath{stroke,fill}%
\end{pgfscope}%
\begin{pgfscope}%
\pgfpathrectangle{\pgfqpoint{0.100000in}{0.212622in}}{\pgfqpoint{3.696000in}{3.696000in}}%
\pgfusepath{clip}%
\pgfsetbuttcap%
\pgfsetroundjoin%
\definecolor{currentfill}{rgb}{0.121569,0.466667,0.705882}%
\pgfsetfillcolor{currentfill}%
\pgfsetfillopacity{0.326188}%
\pgfsetlinewidth{1.003750pt}%
\definecolor{currentstroke}{rgb}{0.121569,0.466667,0.705882}%
\pgfsetstrokecolor{currentstroke}%
\pgfsetstrokeopacity{0.326188}%
\pgfsetdash{}{0pt}%
\pgfpathmoveto{\pgfqpoint{1.872889in}{1.999935in}}%
\pgfpathcurveto{\pgfqpoint{1.881125in}{1.999935in}}{\pgfqpoint{1.889025in}{2.003208in}}{\pgfqpoint{1.894849in}{2.009032in}}%
\pgfpathcurveto{\pgfqpoint{1.900673in}{2.014856in}}{\pgfqpoint{1.903946in}{2.022756in}}{\pgfqpoint{1.903946in}{2.030992in}}%
\pgfpathcurveto{\pgfqpoint{1.903946in}{2.039228in}}{\pgfqpoint{1.900673in}{2.047128in}}{\pgfqpoint{1.894849in}{2.052952in}}%
\pgfpathcurveto{\pgfqpoint{1.889025in}{2.058776in}}{\pgfqpoint{1.881125in}{2.062048in}}{\pgfqpoint{1.872889in}{2.062048in}}%
\pgfpathcurveto{\pgfqpoint{1.864653in}{2.062048in}}{\pgfqpoint{1.856753in}{2.058776in}}{\pgfqpoint{1.850929in}{2.052952in}}%
\pgfpathcurveto{\pgfqpoint{1.845105in}{2.047128in}}{\pgfqpoint{1.841833in}{2.039228in}}{\pgfqpoint{1.841833in}{2.030992in}}%
\pgfpathcurveto{\pgfqpoint{1.841833in}{2.022756in}}{\pgfqpoint{1.845105in}{2.014856in}}{\pgfqpoint{1.850929in}{2.009032in}}%
\pgfpathcurveto{\pgfqpoint{1.856753in}{2.003208in}}{\pgfqpoint{1.864653in}{1.999935in}}{\pgfqpoint{1.872889in}{1.999935in}}%
\pgfpathclose%
\pgfusepath{stroke,fill}%
\end{pgfscope}%
\begin{pgfscope}%
\pgfpathrectangle{\pgfqpoint{0.100000in}{0.212622in}}{\pgfqpoint{3.696000in}{3.696000in}}%
\pgfusepath{clip}%
\pgfsetbuttcap%
\pgfsetroundjoin%
\definecolor{currentfill}{rgb}{0.121569,0.466667,0.705882}%
\pgfsetfillcolor{currentfill}%
\pgfsetfillopacity{0.326811}%
\pgfsetlinewidth{1.003750pt}%
\definecolor{currentstroke}{rgb}{0.121569,0.466667,0.705882}%
\pgfsetstrokecolor{currentstroke}%
\pgfsetstrokeopacity{0.326811}%
\pgfsetdash{}{0pt}%
\pgfpathmoveto{\pgfqpoint{1.868160in}{1.998320in}}%
\pgfpathcurveto{\pgfqpoint{1.876396in}{1.998320in}}{\pgfqpoint{1.884296in}{2.001592in}}{\pgfqpoint{1.890120in}{2.007416in}}%
\pgfpathcurveto{\pgfqpoint{1.895944in}{2.013240in}}{\pgfqpoint{1.899217in}{2.021140in}}{\pgfqpoint{1.899217in}{2.029376in}}%
\pgfpathcurveto{\pgfqpoint{1.899217in}{2.037613in}}{\pgfqpoint{1.895944in}{2.045513in}}{\pgfqpoint{1.890120in}{2.051337in}}%
\pgfpathcurveto{\pgfqpoint{1.884296in}{2.057161in}}{\pgfqpoint{1.876396in}{2.060433in}}{\pgfqpoint{1.868160in}{2.060433in}}%
\pgfpathcurveto{\pgfqpoint{1.859924in}{2.060433in}}{\pgfqpoint{1.852024in}{2.057161in}}{\pgfqpoint{1.846200in}{2.051337in}}%
\pgfpathcurveto{\pgfqpoint{1.840376in}{2.045513in}}{\pgfqpoint{1.837104in}{2.037613in}}{\pgfqpoint{1.837104in}{2.029376in}}%
\pgfpathcurveto{\pgfqpoint{1.837104in}{2.021140in}}{\pgfqpoint{1.840376in}{2.013240in}}{\pgfqpoint{1.846200in}{2.007416in}}%
\pgfpathcurveto{\pgfqpoint{1.852024in}{2.001592in}}{\pgfqpoint{1.859924in}{1.998320in}}{\pgfqpoint{1.868160in}{1.998320in}}%
\pgfpathclose%
\pgfusepath{stroke,fill}%
\end{pgfscope}%
\begin{pgfscope}%
\pgfpathrectangle{\pgfqpoint{0.100000in}{0.212622in}}{\pgfqpoint{3.696000in}{3.696000in}}%
\pgfusepath{clip}%
\pgfsetbuttcap%
\pgfsetroundjoin%
\definecolor{currentfill}{rgb}{0.121569,0.466667,0.705882}%
\pgfsetfillcolor{currentfill}%
\pgfsetfillopacity{0.327082}%
\pgfsetlinewidth{1.003750pt}%
\definecolor{currentstroke}{rgb}{0.121569,0.466667,0.705882}%
\pgfsetstrokecolor{currentstroke}%
\pgfsetstrokeopacity{0.327082}%
\pgfsetdash{}{0pt}%
\pgfpathmoveto{\pgfqpoint{1.866683in}{1.996938in}}%
\pgfpathcurveto{\pgfqpoint{1.874920in}{1.996938in}}{\pgfqpoint{1.882820in}{2.000210in}}{\pgfqpoint{1.888644in}{2.006034in}}%
\pgfpathcurveto{\pgfqpoint{1.894468in}{2.011858in}}{\pgfqpoint{1.897740in}{2.019758in}}{\pgfqpoint{1.897740in}{2.027994in}}%
\pgfpathcurveto{\pgfqpoint{1.897740in}{2.036231in}}{\pgfqpoint{1.894468in}{2.044131in}}{\pgfqpoint{1.888644in}{2.049955in}}%
\pgfpathcurveto{\pgfqpoint{1.882820in}{2.055778in}}{\pgfqpoint{1.874920in}{2.059051in}}{\pgfqpoint{1.866683in}{2.059051in}}%
\pgfpathcurveto{\pgfqpoint{1.858447in}{2.059051in}}{\pgfqpoint{1.850547in}{2.055778in}}{\pgfqpoint{1.844723in}{2.049955in}}%
\pgfpathcurveto{\pgfqpoint{1.838899in}{2.044131in}}{\pgfqpoint{1.835627in}{2.036231in}}{\pgfqpoint{1.835627in}{2.027994in}}%
\pgfpathcurveto{\pgfqpoint{1.835627in}{2.019758in}}{\pgfqpoint{1.838899in}{2.011858in}}{\pgfqpoint{1.844723in}{2.006034in}}%
\pgfpathcurveto{\pgfqpoint{1.850547in}{2.000210in}}{\pgfqpoint{1.858447in}{1.996938in}}{\pgfqpoint{1.866683in}{1.996938in}}%
\pgfpathclose%
\pgfusepath{stroke,fill}%
\end{pgfscope}%
\begin{pgfscope}%
\pgfpathrectangle{\pgfqpoint{0.100000in}{0.212622in}}{\pgfqpoint{3.696000in}{3.696000in}}%
\pgfusepath{clip}%
\pgfsetbuttcap%
\pgfsetroundjoin%
\definecolor{currentfill}{rgb}{0.121569,0.466667,0.705882}%
\pgfsetfillcolor{currentfill}%
\pgfsetfillopacity{0.327517}%
\pgfsetlinewidth{1.003750pt}%
\definecolor{currentstroke}{rgb}{0.121569,0.466667,0.705882}%
\pgfsetstrokecolor{currentstroke}%
\pgfsetstrokeopacity{0.327517}%
\pgfsetdash{}{0pt}%
\pgfpathmoveto{\pgfqpoint{1.863329in}{1.995055in}}%
\pgfpathcurveto{\pgfqpoint{1.871565in}{1.995055in}}{\pgfqpoint{1.879465in}{1.998328in}}{\pgfqpoint{1.885289in}{2.004152in}}%
\pgfpathcurveto{\pgfqpoint{1.891113in}{2.009976in}}{\pgfqpoint{1.894385in}{2.017876in}}{\pgfqpoint{1.894385in}{2.026112in}}%
\pgfpathcurveto{\pgfqpoint{1.894385in}{2.034348in}}{\pgfqpoint{1.891113in}{2.042248in}}{\pgfqpoint{1.885289in}{2.048072in}}%
\pgfpathcurveto{\pgfqpoint{1.879465in}{2.053896in}}{\pgfqpoint{1.871565in}{2.057168in}}{\pgfqpoint{1.863329in}{2.057168in}}%
\pgfpathcurveto{\pgfqpoint{1.855092in}{2.057168in}}{\pgfqpoint{1.847192in}{2.053896in}}{\pgfqpoint{1.841368in}{2.048072in}}%
\pgfpathcurveto{\pgfqpoint{1.835544in}{2.042248in}}{\pgfqpoint{1.832272in}{2.034348in}}{\pgfqpoint{1.832272in}{2.026112in}}%
\pgfpathcurveto{\pgfqpoint{1.832272in}{2.017876in}}{\pgfqpoint{1.835544in}{2.009976in}}{\pgfqpoint{1.841368in}{2.004152in}}%
\pgfpathcurveto{\pgfqpoint{1.847192in}{1.998328in}}{\pgfqpoint{1.855092in}{1.995055in}}{\pgfqpoint{1.863329in}{1.995055in}}%
\pgfpathclose%
\pgfusepath{stroke,fill}%
\end{pgfscope}%
\begin{pgfscope}%
\pgfpathrectangle{\pgfqpoint{0.100000in}{0.212622in}}{\pgfqpoint{3.696000in}{3.696000in}}%
\pgfusepath{clip}%
\pgfsetbuttcap%
\pgfsetroundjoin%
\definecolor{currentfill}{rgb}{0.121569,0.466667,0.705882}%
\pgfsetfillcolor{currentfill}%
\pgfsetfillopacity{0.327688}%
\pgfsetlinewidth{1.003750pt}%
\definecolor{currentstroke}{rgb}{0.121569,0.466667,0.705882}%
\pgfsetstrokecolor{currentstroke}%
\pgfsetstrokeopacity{0.327688}%
\pgfsetdash{}{0pt}%
\pgfpathmoveto{\pgfqpoint{1.861841in}{1.993953in}}%
\pgfpathcurveto{\pgfqpoint{1.870077in}{1.993953in}}{\pgfqpoint{1.877978in}{1.997226in}}{\pgfqpoint{1.883801in}{2.003050in}}%
\pgfpathcurveto{\pgfqpoint{1.889625in}{2.008874in}}{\pgfqpoint{1.892898in}{2.016774in}}{\pgfqpoint{1.892898in}{2.025010in}}%
\pgfpathcurveto{\pgfqpoint{1.892898in}{2.033246in}}{\pgfqpoint{1.889625in}{2.041146in}}{\pgfqpoint{1.883801in}{2.046970in}}%
\pgfpathcurveto{\pgfqpoint{1.877978in}{2.052794in}}{\pgfqpoint{1.870077in}{2.056066in}}{\pgfqpoint{1.861841in}{2.056066in}}%
\pgfpathcurveto{\pgfqpoint{1.853605in}{2.056066in}}{\pgfqpoint{1.845705in}{2.052794in}}{\pgfqpoint{1.839881in}{2.046970in}}%
\pgfpathcurveto{\pgfqpoint{1.834057in}{2.041146in}}{\pgfqpoint{1.830785in}{2.033246in}}{\pgfqpoint{1.830785in}{2.025010in}}%
\pgfpathcurveto{\pgfqpoint{1.830785in}{2.016774in}}{\pgfqpoint{1.834057in}{2.008874in}}{\pgfqpoint{1.839881in}{2.003050in}}%
\pgfpathcurveto{\pgfqpoint{1.845705in}{1.997226in}}{\pgfqpoint{1.853605in}{1.993953in}}{\pgfqpoint{1.861841in}{1.993953in}}%
\pgfpathclose%
\pgfusepath{stroke,fill}%
\end{pgfscope}%
\begin{pgfscope}%
\pgfpathrectangle{\pgfqpoint{0.100000in}{0.212622in}}{\pgfqpoint{3.696000in}{3.696000in}}%
\pgfusepath{clip}%
\pgfsetbuttcap%
\pgfsetroundjoin%
\definecolor{currentfill}{rgb}{0.121569,0.466667,0.705882}%
\pgfsetfillcolor{currentfill}%
\pgfsetfillopacity{0.328060}%
\pgfsetlinewidth{1.003750pt}%
\definecolor{currentstroke}{rgb}{0.121569,0.466667,0.705882}%
\pgfsetstrokecolor{currentstroke}%
\pgfsetstrokeopacity{0.328060}%
\pgfsetdash{}{0pt}%
\pgfpathmoveto{\pgfqpoint{1.859397in}{1.991963in}}%
\pgfpathcurveto{\pgfqpoint{1.867633in}{1.991963in}}{\pgfqpoint{1.875533in}{1.995235in}}{\pgfqpoint{1.881357in}{2.001059in}}%
\pgfpathcurveto{\pgfqpoint{1.887181in}{2.006883in}}{\pgfqpoint{1.890453in}{2.014783in}}{\pgfqpoint{1.890453in}{2.023019in}}%
\pgfpathcurveto{\pgfqpoint{1.890453in}{2.031255in}}{\pgfqpoint{1.887181in}{2.039155in}}{\pgfqpoint{1.881357in}{2.044979in}}%
\pgfpathcurveto{\pgfqpoint{1.875533in}{2.050803in}}{\pgfqpoint{1.867633in}{2.054076in}}{\pgfqpoint{1.859397in}{2.054076in}}%
\pgfpathcurveto{\pgfqpoint{1.851161in}{2.054076in}}{\pgfqpoint{1.843260in}{2.050803in}}{\pgfqpoint{1.837437in}{2.044979in}}%
\pgfpathcurveto{\pgfqpoint{1.831613in}{2.039155in}}{\pgfqpoint{1.828340in}{2.031255in}}{\pgfqpoint{1.828340in}{2.023019in}}%
\pgfpathcurveto{\pgfqpoint{1.828340in}{2.014783in}}{\pgfqpoint{1.831613in}{2.006883in}}{\pgfqpoint{1.837437in}{2.001059in}}%
\pgfpathcurveto{\pgfqpoint{1.843260in}{1.995235in}}{\pgfqpoint{1.851161in}{1.991963in}}{\pgfqpoint{1.859397in}{1.991963in}}%
\pgfpathclose%
\pgfusepath{stroke,fill}%
\end{pgfscope}%
\begin{pgfscope}%
\pgfpathrectangle{\pgfqpoint{0.100000in}{0.212622in}}{\pgfqpoint{3.696000in}{3.696000in}}%
\pgfusepath{clip}%
\pgfsetbuttcap%
\pgfsetroundjoin%
\definecolor{currentfill}{rgb}{0.121569,0.466667,0.705882}%
\pgfsetfillcolor{currentfill}%
\pgfsetfillopacity{0.328812}%
\pgfsetlinewidth{1.003750pt}%
\definecolor{currentstroke}{rgb}{0.121569,0.466667,0.705882}%
\pgfsetstrokecolor{currentstroke}%
\pgfsetstrokeopacity{0.328812}%
\pgfsetdash{}{0pt}%
\pgfpathmoveto{\pgfqpoint{1.854484in}{1.989636in}}%
\pgfpathcurveto{\pgfqpoint{1.862720in}{1.989636in}}{\pgfqpoint{1.870621in}{1.992909in}}{\pgfqpoint{1.876444in}{1.998732in}}%
\pgfpathcurveto{\pgfqpoint{1.882268in}{2.004556in}}{\pgfqpoint{1.885541in}{2.012456in}}{\pgfqpoint{1.885541in}{2.020693in}}%
\pgfpathcurveto{\pgfqpoint{1.885541in}{2.028929in}}{\pgfqpoint{1.882268in}{2.036829in}}{\pgfqpoint{1.876444in}{2.042653in}}%
\pgfpathcurveto{\pgfqpoint{1.870621in}{2.048477in}}{\pgfqpoint{1.862720in}{2.051749in}}{\pgfqpoint{1.854484in}{2.051749in}}%
\pgfpathcurveto{\pgfqpoint{1.846248in}{2.051749in}}{\pgfqpoint{1.838348in}{2.048477in}}{\pgfqpoint{1.832524in}{2.042653in}}%
\pgfpathcurveto{\pgfqpoint{1.826700in}{2.036829in}}{\pgfqpoint{1.823428in}{2.028929in}}{\pgfqpoint{1.823428in}{2.020693in}}%
\pgfpathcurveto{\pgfqpoint{1.823428in}{2.012456in}}{\pgfqpoint{1.826700in}{2.004556in}}{\pgfqpoint{1.832524in}{1.998732in}}%
\pgfpathcurveto{\pgfqpoint{1.838348in}{1.992909in}}{\pgfqpoint{1.846248in}{1.989636in}}{\pgfqpoint{1.854484in}{1.989636in}}%
\pgfpathclose%
\pgfusepath{stroke,fill}%
\end{pgfscope}%
\begin{pgfscope}%
\pgfpathrectangle{\pgfqpoint{0.100000in}{0.212622in}}{\pgfqpoint{3.696000in}{3.696000in}}%
\pgfusepath{clip}%
\pgfsetbuttcap%
\pgfsetroundjoin%
\definecolor{currentfill}{rgb}{0.121569,0.466667,0.705882}%
\pgfsetfillcolor{currentfill}%
\pgfsetfillopacity{0.329028}%
\pgfsetlinewidth{1.003750pt}%
\definecolor{currentstroke}{rgb}{0.121569,0.466667,0.705882}%
\pgfsetstrokecolor{currentstroke}%
\pgfsetstrokeopacity{0.329028}%
\pgfsetdash{}{0pt}%
\pgfpathmoveto{\pgfqpoint{1.852465in}{1.987988in}}%
\pgfpathcurveto{\pgfqpoint{1.860701in}{1.987988in}}{\pgfqpoint{1.868601in}{1.991260in}}{\pgfqpoint{1.874425in}{1.997084in}}%
\pgfpathcurveto{\pgfqpoint{1.880249in}{2.002908in}}{\pgfqpoint{1.883522in}{2.010808in}}{\pgfqpoint{1.883522in}{2.019045in}}%
\pgfpathcurveto{\pgfqpoint{1.883522in}{2.027281in}}{\pgfqpoint{1.880249in}{2.035181in}}{\pgfqpoint{1.874425in}{2.041005in}}%
\pgfpathcurveto{\pgfqpoint{1.868601in}{2.046829in}}{\pgfqpoint{1.860701in}{2.050101in}}{\pgfqpoint{1.852465in}{2.050101in}}%
\pgfpathcurveto{\pgfqpoint{1.844229in}{2.050101in}}{\pgfqpoint{1.836329in}{2.046829in}}{\pgfqpoint{1.830505in}{2.041005in}}%
\pgfpathcurveto{\pgfqpoint{1.824681in}{2.035181in}}{\pgfqpoint{1.821409in}{2.027281in}}{\pgfqpoint{1.821409in}{2.019045in}}%
\pgfpathcurveto{\pgfqpoint{1.821409in}{2.010808in}}{\pgfqpoint{1.824681in}{2.002908in}}{\pgfqpoint{1.830505in}{1.997084in}}%
\pgfpathcurveto{\pgfqpoint{1.836329in}{1.991260in}}{\pgfqpoint{1.844229in}{1.987988in}}{\pgfqpoint{1.852465in}{1.987988in}}%
\pgfpathclose%
\pgfusepath{stroke,fill}%
\end{pgfscope}%
\begin{pgfscope}%
\pgfpathrectangle{\pgfqpoint{0.100000in}{0.212622in}}{\pgfqpoint{3.696000in}{3.696000in}}%
\pgfusepath{clip}%
\pgfsetbuttcap%
\pgfsetroundjoin%
\definecolor{currentfill}{rgb}{0.121569,0.466667,0.705882}%
\pgfsetfillcolor{currentfill}%
\pgfsetfillopacity{0.329651}%
\pgfsetlinewidth{1.003750pt}%
\definecolor{currentstroke}{rgb}{0.121569,0.466667,0.705882}%
\pgfsetstrokecolor{currentstroke}%
\pgfsetstrokeopacity{0.329651}%
\pgfsetdash{}{0pt}%
\pgfpathmoveto{\pgfqpoint{1.849310in}{1.985840in}}%
\pgfpathcurveto{\pgfqpoint{1.857546in}{1.985840in}}{\pgfqpoint{1.865446in}{1.989112in}}{\pgfqpoint{1.871270in}{1.994936in}}%
\pgfpathcurveto{\pgfqpoint{1.877094in}{2.000760in}}{\pgfqpoint{1.880366in}{2.008660in}}{\pgfqpoint{1.880366in}{2.016896in}}%
\pgfpathcurveto{\pgfqpoint{1.880366in}{2.025133in}}{\pgfqpoint{1.877094in}{2.033033in}}{\pgfqpoint{1.871270in}{2.038856in}}%
\pgfpathcurveto{\pgfqpoint{1.865446in}{2.044680in}}{\pgfqpoint{1.857546in}{2.047953in}}{\pgfqpoint{1.849310in}{2.047953in}}%
\pgfpathcurveto{\pgfqpoint{1.841074in}{2.047953in}}{\pgfqpoint{1.833174in}{2.044680in}}{\pgfqpoint{1.827350in}{2.038856in}}%
\pgfpathcurveto{\pgfqpoint{1.821526in}{2.033033in}}{\pgfqpoint{1.818253in}{2.025133in}}{\pgfqpoint{1.818253in}{2.016896in}}%
\pgfpathcurveto{\pgfqpoint{1.818253in}{2.008660in}}{\pgfqpoint{1.821526in}{2.000760in}}{\pgfqpoint{1.827350in}{1.994936in}}%
\pgfpathcurveto{\pgfqpoint{1.833174in}{1.989112in}}{\pgfqpoint{1.841074in}{1.985840in}}{\pgfqpoint{1.849310in}{1.985840in}}%
\pgfpathclose%
\pgfusepath{stroke,fill}%
\end{pgfscope}%
\begin{pgfscope}%
\pgfpathrectangle{\pgfqpoint{0.100000in}{0.212622in}}{\pgfqpoint{3.696000in}{3.696000in}}%
\pgfusepath{clip}%
\pgfsetbuttcap%
\pgfsetroundjoin%
\definecolor{currentfill}{rgb}{0.121569,0.466667,0.705882}%
\pgfsetfillcolor{currentfill}%
\pgfsetfillopacity{0.329797}%
\pgfsetlinewidth{1.003750pt}%
\definecolor{currentstroke}{rgb}{0.121569,0.466667,0.705882}%
\pgfsetstrokecolor{currentstroke}%
\pgfsetstrokeopacity{0.329797}%
\pgfsetdash{}{0pt}%
\pgfpathmoveto{\pgfqpoint{1.847858in}{1.985147in}}%
\pgfpathcurveto{\pgfqpoint{1.856094in}{1.985147in}}{\pgfqpoint{1.863994in}{1.988419in}}{\pgfqpoint{1.869818in}{1.994243in}}%
\pgfpathcurveto{\pgfqpoint{1.875642in}{2.000067in}}{\pgfqpoint{1.878914in}{2.007967in}}{\pgfqpoint{1.878914in}{2.016203in}}%
\pgfpathcurveto{\pgfqpoint{1.878914in}{2.024440in}}{\pgfqpoint{1.875642in}{2.032340in}}{\pgfqpoint{1.869818in}{2.038164in}}%
\pgfpathcurveto{\pgfqpoint{1.863994in}{2.043988in}}{\pgfqpoint{1.856094in}{2.047260in}}{\pgfqpoint{1.847858in}{2.047260in}}%
\pgfpathcurveto{\pgfqpoint{1.839622in}{2.047260in}}{\pgfqpoint{1.831722in}{2.043988in}}{\pgfqpoint{1.825898in}{2.038164in}}%
\pgfpathcurveto{\pgfqpoint{1.820074in}{2.032340in}}{\pgfqpoint{1.816801in}{2.024440in}}{\pgfqpoint{1.816801in}{2.016203in}}%
\pgfpathcurveto{\pgfqpoint{1.816801in}{2.007967in}}{\pgfqpoint{1.820074in}{2.000067in}}{\pgfqpoint{1.825898in}{1.994243in}}%
\pgfpathcurveto{\pgfqpoint{1.831722in}{1.988419in}}{\pgfqpoint{1.839622in}{1.985147in}}{\pgfqpoint{1.847858in}{1.985147in}}%
\pgfpathclose%
\pgfusepath{stroke,fill}%
\end{pgfscope}%
\begin{pgfscope}%
\pgfpathrectangle{\pgfqpoint{0.100000in}{0.212622in}}{\pgfqpoint{3.696000in}{3.696000in}}%
\pgfusepath{clip}%
\pgfsetbuttcap%
\pgfsetroundjoin%
\definecolor{currentfill}{rgb}{0.121569,0.466667,0.705882}%
\pgfsetfillcolor{currentfill}%
\pgfsetfillopacity{0.330260}%
\pgfsetlinewidth{1.003750pt}%
\definecolor{currentstroke}{rgb}{0.121569,0.466667,0.705882}%
\pgfsetstrokecolor{currentstroke}%
\pgfsetstrokeopacity{0.330260}%
\pgfsetdash{}{0pt}%
\pgfpathmoveto{\pgfqpoint{1.846226in}{1.983727in}}%
\pgfpathcurveto{\pgfqpoint{1.854463in}{1.983727in}}{\pgfqpoint{1.862363in}{1.987000in}}{\pgfqpoint{1.868187in}{1.992823in}}%
\pgfpathcurveto{\pgfqpoint{1.874010in}{1.998647in}}{\pgfqpoint{1.877283in}{2.006547in}}{\pgfqpoint{1.877283in}{2.014784in}}%
\pgfpathcurveto{\pgfqpoint{1.877283in}{2.023020in}}{\pgfqpoint{1.874010in}{2.030920in}}{\pgfqpoint{1.868187in}{2.036744in}}%
\pgfpathcurveto{\pgfqpoint{1.862363in}{2.042568in}}{\pgfqpoint{1.854463in}{2.045840in}}{\pgfqpoint{1.846226in}{2.045840in}}%
\pgfpathcurveto{\pgfqpoint{1.837990in}{2.045840in}}{\pgfqpoint{1.830090in}{2.042568in}}{\pgfqpoint{1.824266in}{2.036744in}}%
\pgfpathcurveto{\pgfqpoint{1.818442in}{2.030920in}}{\pgfqpoint{1.815170in}{2.023020in}}{\pgfqpoint{1.815170in}{2.014784in}}%
\pgfpathcurveto{\pgfqpoint{1.815170in}{2.006547in}}{\pgfqpoint{1.818442in}{1.998647in}}{\pgfqpoint{1.824266in}{1.992823in}}%
\pgfpathcurveto{\pgfqpoint{1.830090in}{1.987000in}}{\pgfqpoint{1.837990in}{1.983727in}}{\pgfqpoint{1.846226in}{1.983727in}}%
\pgfpathclose%
\pgfusepath{stroke,fill}%
\end{pgfscope}%
\begin{pgfscope}%
\pgfpathrectangle{\pgfqpoint{0.100000in}{0.212622in}}{\pgfqpoint{3.696000in}{3.696000in}}%
\pgfusepath{clip}%
\pgfsetbuttcap%
\pgfsetroundjoin%
\definecolor{currentfill}{rgb}{0.121569,0.466667,0.705882}%
\pgfsetfillcolor{currentfill}%
\pgfsetfillopacity{0.330828}%
\pgfsetlinewidth{1.003750pt}%
\definecolor{currentstroke}{rgb}{0.121569,0.466667,0.705882}%
\pgfsetstrokecolor{currentstroke}%
\pgfsetstrokeopacity{0.330828}%
\pgfsetdash{}{0pt}%
\pgfpathmoveto{\pgfqpoint{1.840979in}{1.982951in}}%
\pgfpathcurveto{\pgfqpoint{1.849216in}{1.982951in}}{\pgfqpoint{1.857116in}{1.986223in}}{\pgfqpoint{1.862940in}{1.992047in}}%
\pgfpathcurveto{\pgfqpoint{1.868763in}{1.997871in}}{\pgfqpoint{1.872036in}{2.005771in}}{\pgfqpoint{1.872036in}{2.014007in}}%
\pgfpathcurveto{\pgfqpoint{1.872036in}{2.022243in}}{\pgfqpoint{1.868763in}{2.030144in}}{\pgfqpoint{1.862940in}{2.035967in}}%
\pgfpathcurveto{\pgfqpoint{1.857116in}{2.041791in}}{\pgfqpoint{1.849216in}{2.045064in}}{\pgfqpoint{1.840979in}{2.045064in}}%
\pgfpathcurveto{\pgfqpoint{1.832743in}{2.045064in}}{\pgfqpoint{1.824843in}{2.041791in}}{\pgfqpoint{1.819019in}{2.035967in}}%
\pgfpathcurveto{\pgfqpoint{1.813195in}{2.030144in}}{\pgfqpoint{1.809923in}{2.022243in}}{\pgfqpoint{1.809923in}{2.014007in}}%
\pgfpathcurveto{\pgfqpoint{1.809923in}{2.005771in}}{\pgfqpoint{1.813195in}{1.997871in}}{\pgfqpoint{1.819019in}{1.992047in}}%
\pgfpathcurveto{\pgfqpoint{1.824843in}{1.986223in}}{\pgfqpoint{1.832743in}{1.982951in}}{\pgfqpoint{1.840979in}{1.982951in}}%
\pgfpathclose%
\pgfusepath{stroke,fill}%
\end{pgfscope}%
\begin{pgfscope}%
\pgfpathrectangle{\pgfqpoint{0.100000in}{0.212622in}}{\pgfqpoint{3.696000in}{3.696000in}}%
\pgfusepath{clip}%
\pgfsetbuttcap%
\pgfsetroundjoin%
\definecolor{currentfill}{rgb}{0.121569,0.466667,0.705882}%
\pgfsetfillcolor{currentfill}%
\pgfsetfillopacity{0.331056}%
\pgfsetlinewidth{1.003750pt}%
\definecolor{currentstroke}{rgb}{0.121569,0.466667,0.705882}%
\pgfsetstrokecolor{currentstroke}%
\pgfsetstrokeopacity{0.331056}%
\pgfsetdash{}{0pt}%
\pgfpathmoveto{\pgfqpoint{1.840068in}{1.981867in}}%
\pgfpathcurveto{\pgfqpoint{1.848304in}{1.981867in}}{\pgfqpoint{1.856204in}{1.985140in}}{\pgfqpoint{1.862028in}{1.990964in}}%
\pgfpathcurveto{\pgfqpoint{1.867852in}{1.996787in}}{\pgfqpoint{1.871124in}{2.004687in}}{\pgfqpoint{1.871124in}{2.012924in}}%
\pgfpathcurveto{\pgfqpoint{1.871124in}{2.021160in}}{\pgfqpoint{1.867852in}{2.029060in}}{\pgfqpoint{1.862028in}{2.034884in}}%
\pgfpathcurveto{\pgfqpoint{1.856204in}{2.040708in}}{\pgfqpoint{1.848304in}{2.043980in}}{\pgfqpoint{1.840068in}{2.043980in}}%
\pgfpathcurveto{\pgfqpoint{1.831831in}{2.043980in}}{\pgfqpoint{1.823931in}{2.040708in}}{\pgfqpoint{1.818107in}{2.034884in}}%
\pgfpathcurveto{\pgfqpoint{1.812283in}{2.029060in}}{\pgfqpoint{1.809011in}{2.021160in}}{\pgfqpoint{1.809011in}{2.012924in}}%
\pgfpathcurveto{\pgfqpoint{1.809011in}{2.004687in}}{\pgfqpoint{1.812283in}{1.996787in}}{\pgfqpoint{1.818107in}{1.990964in}}%
\pgfpathcurveto{\pgfqpoint{1.823931in}{1.985140in}}{\pgfqpoint{1.831831in}{1.981867in}}{\pgfqpoint{1.840068in}{1.981867in}}%
\pgfpathclose%
\pgfusepath{stroke,fill}%
\end{pgfscope}%
\begin{pgfscope}%
\pgfpathrectangle{\pgfqpoint{0.100000in}{0.212622in}}{\pgfqpoint{3.696000in}{3.696000in}}%
\pgfusepath{clip}%
\pgfsetbuttcap%
\pgfsetroundjoin%
\definecolor{currentfill}{rgb}{0.121569,0.466667,0.705882}%
\pgfsetfillcolor{currentfill}%
\pgfsetfillopacity{0.331484}%
\pgfsetlinewidth{1.003750pt}%
\definecolor{currentstroke}{rgb}{0.121569,0.466667,0.705882}%
\pgfsetstrokecolor{currentstroke}%
\pgfsetstrokeopacity{0.331484}%
\pgfsetdash{}{0pt}%
\pgfpathmoveto{\pgfqpoint{1.837496in}{1.981354in}}%
\pgfpathcurveto{\pgfqpoint{1.845732in}{1.981354in}}{\pgfqpoint{1.853632in}{1.984626in}}{\pgfqpoint{1.859456in}{1.990450in}}%
\pgfpathcurveto{\pgfqpoint{1.865280in}{1.996274in}}{\pgfqpoint{1.868552in}{2.004174in}}{\pgfqpoint{1.868552in}{2.012410in}}%
\pgfpathcurveto{\pgfqpoint{1.868552in}{2.020647in}}{\pgfqpoint{1.865280in}{2.028547in}}{\pgfqpoint{1.859456in}{2.034371in}}%
\pgfpathcurveto{\pgfqpoint{1.853632in}{2.040195in}}{\pgfqpoint{1.845732in}{2.043467in}}{\pgfqpoint{1.837496in}{2.043467in}}%
\pgfpathcurveto{\pgfqpoint{1.829260in}{2.043467in}}{\pgfqpoint{1.821360in}{2.040195in}}{\pgfqpoint{1.815536in}{2.034371in}}%
\pgfpathcurveto{\pgfqpoint{1.809712in}{2.028547in}}{\pgfqpoint{1.806439in}{2.020647in}}{\pgfqpoint{1.806439in}{2.012410in}}%
\pgfpathcurveto{\pgfqpoint{1.806439in}{2.004174in}}{\pgfqpoint{1.809712in}{1.996274in}}{\pgfqpoint{1.815536in}{1.990450in}}%
\pgfpathcurveto{\pgfqpoint{1.821360in}{1.984626in}}{\pgfqpoint{1.829260in}{1.981354in}}{\pgfqpoint{1.837496in}{1.981354in}}%
\pgfpathclose%
\pgfusepath{stroke,fill}%
\end{pgfscope}%
\begin{pgfscope}%
\pgfpathrectangle{\pgfqpoint{0.100000in}{0.212622in}}{\pgfqpoint{3.696000in}{3.696000in}}%
\pgfusepath{clip}%
\pgfsetbuttcap%
\pgfsetroundjoin%
\definecolor{currentfill}{rgb}{0.121569,0.466667,0.705882}%
\pgfsetfillcolor{currentfill}%
\pgfsetfillopacity{0.331567}%
\pgfsetlinewidth{1.003750pt}%
\definecolor{currentstroke}{rgb}{0.121569,0.466667,0.705882}%
\pgfsetstrokecolor{currentstroke}%
\pgfsetstrokeopacity{0.331567}%
\pgfsetdash{}{0pt}%
\pgfpathmoveto{\pgfqpoint{1.836879in}{1.980767in}}%
\pgfpathcurveto{\pgfqpoint{1.845115in}{1.980767in}}{\pgfqpoint{1.853015in}{1.984039in}}{\pgfqpoint{1.858839in}{1.989863in}}%
\pgfpathcurveto{\pgfqpoint{1.864663in}{1.995687in}}{\pgfqpoint{1.867935in}{2.003587in}}{\pgfqpoint{1.867935in}{2.011823in}}%
\pgfpathcurveto{\pgfqpoint{1.867935in}{2.020059in}}{\pgfqpoint{1.864663in}{2.027959in}}{\pgfqpoint{1.858839in}{2.033783in}}%
\pgfpathcurveto{\pgfqpoint{1.853015in}{2.039607in}}{\pgfqpoint{1.845115in}{2.042880in}}{\pgfqpoint{1.836879in}{2.042880in}}%
\pgfpathcurveto{\pgfqpoint{1.828643in}{2.042880in}}{\pgfqpoint{1.820742in}{2.039607in}}{\pgfqpoint{1.814919in}{2.033783in}}%
\pgfpathcurveto{\pgfqpoint{1.809095in}{2.027959in}}{\pgfqpoint{1.805822in}{2.020059in}}{\pgfqpoint{1.805822in}{2.011823in}}%
\pgfpathcurveto{\pgfqpoint{1.805822in}{2.003587in}}{\pgfqpoint{1.809095in}{1.995687in}}{\pgfqpoint{1.814919in}{1.989863in}}%
\pgfpathcurveto{\pgfqpoint{1.820742in}{1.984039in}}{\pgfqpoint{1.828643in}{1.980767in}}{\pgfqpoint{1.836879in}{1.980767in}}%
\pgfpathclose%
\pgfusepath{stroke,fill}%
\end{pgfscope}%
\begin{pgfscope}%
\pgfpathrectangle{\pgfqpoint{0.100000in}{0.212622in}}{\pgfqpoint{3.696000in}{3.696000in}}%
\pgfusepath{clip}%
\pgfsetbuttcap%
\pgfsetroundjoin%
\definecolor{currentfill}{rgb}{0.121569,0.466667,0.705882}%
\pgfsetfillcolor{currentfill}%
\pgfsetfillopacity{0.331732}%
\pgfsetlinewidth{1.003750pt}%
\definecolor{currentstroke}{rgb}{0.121569,0.466667,0.705882}%
\pgfsetstrokecolor{currentstroke}%
\pgfsetstrokeopacity{0.331732}%
\pgfsetdash{}{0pt}%
\pgfpathmoveto{\pgfqpoint{1.835673in}{1.979927in}}%
\pgfpathcurveto{\pgfqpoint{1.843909in}{1.979927in}}{\pgfqpoint{1.851810in}{1.983199in}}{\pgfqpoint{1.857633in}{1.989023in}}%
\pgfpathcurveto{\pgfqpoint{1.863457in}{1.994847in}}{\pgfqpoint{1.866730in}{2.002747in}}{\pgfqpoint{1.866730in}{2.010983in}}%
\pgfpathcurveto{\pgfqpoint{1.866730in}{2.019220in}}{\pgfqpoint{1.863457in}{2.027120in}}{\pgfqpoint{1.857633in}{2.032944in}}%
\pgfpathcurveto{\pgfqpoint{1.851810in}{2.038768in}}{\pgfqpoint{1.843909in}{2.042040in}}{\pgfqpoint{1.835673in}{2.042040in}}%
\pgfpathcurveto{\pgfqpoint{1.827437in}{2.042040in}}{\pgfqpoint{1.819537in}{2.038768in}}{\pgfqpoint{1.813713in}{2.032944in}}%
\pgfpathcurveto{\pgfqpoint{1.807889in}{2.027120in}}{\pgfqpoint{1.804617in}{2.019220in}}{\pgfqpoint{1.804617in}{2.010983in}}%
\pgfpathcurveto{\pgfqpoint{1.804617in}{2.002747in}}{\pgfqpoint{1.807889in}{1.994847in}}{\pgfqpoint{1.813713in}{1.989023in}}%
\pgfpathcurveto{\pgfqpoint{1.819537in}{1.983199in}}{\pgfqpoint{1.827437in}{1.979927in}}{\pgfqpoint{1.835673in}{1.979927in}}%
\pgfpathclose%
\pgfusepath{stroke,fill}%
\end{pgfscope}%
\begin{pgfscope}%
\pgfpathrectangle{\pgfqpoint{0.100000in}{0.212622in}}{\pgfqpoint{3.696000in}{3.696000in}}%
\pgfusepath{clip}%
\pgfsetbuttcap%
\pgfsetroundjoin%
\definecolor{currentfill}{rgb}{0.121569,0.466667,0.705882}%
\pgfsetfillcolor{currentfill}%
\pgfsetfillopacity{0.332068}%
\pgfsetlinewidth{1.003750pt}%
\definecolor{currentstroke}{rgb}{0.121569,0.466667,0.705882}%
\pgfsetstrokecolor{currentstroke}%
\pgfsetstrokeopacity{0.332068}%
\pgfsetdash{}{0pt}%
\pgfpathmoveto{\pgfqpoint{1.833694in}{1.978320in}}%
\pgfpathcurveto{\pgfqpoint{1.841930in}{1.978320in}}{\pgfqpoint{1.849830in}{1.981592in}}{\pgfqpoint{1.855654in}{1.987416in}}%
\pgfpathcurveto{\pgfqpoint{1.861478in}{1.993240in}}{\pgfqpoint{1.864751in}{2.001140in}}{\pgfqpoint{1.864751in}{2.009376in}}%
\pgfpathcurveto{\pgfqpoint{1.864751in}{2.017613in}}{\pgfqpoint{1.861478in}{2.025513in}}{\pgfqpoint{1.855654in}{2.031337in}}%
\pgfpathcurveto{\pgfqpoint{1.849830in}{2.037160in}}{\pgfqpoint{1.841930in}{2.040433in}}{\pgfqpoint{1.833694in}{2.040433in}}%
\pgfpathcurveto{\pgfqpoint{1.825458in}{2.040433in}}{\pgfqpoint{1.817558in}{2.037160in}}{\pgfqpoint{1.811734in}{2.031337in}}%
\pgfpathcurveto{\pgfqpoint{1.805910in}{2.025513in}}{\pgfqpoint{1.802638in}{2.017613in}}{\pgfqpoint{1.802638in}{2.009376in}}%
\pgfpathcurveto{\pgfqpoint{1.802638in}{2.001140in}}{\pgfqpoint{1.805910in}{1.993240in}}{\pgfqpoint{1.811734in}{1.987416in}}%
\pgfpathcurveto{\pgfqpoint{1.817558in}{1.981592in}}{\pgfqpoint{1.825458in}{1.978320in}}{\pgfqpoint{1.833694in}{1.978320in}}%
\pgfpathclose%
\pgfusepath{stroke,fill}%
\end{pgfscope}%
\begin{pgfscope}%
\pgfpathrectangle{\pgfqpoint{0.100000in}{0.212622in}}{\pgfqpoint{3.696000in}{3.696000in}}%
\pgfusepath{clip}%
\pgfsetbuttcap%
\pgfsetroundjoin%
\definecolor{currentfill}{rgb}{0.121569,0.466667,0.705882}%
\pgfsetfillcolor{currentfill}%
\pgfsetfillopacity{0.332137}%
\pgfsetlinewidth{1.003750pt}%
\definecolor{currentstroke}{rgb}{0.121569,0.466667,0.705882}%
\pgfsetstrokecolor{currentstroke}%
\pgfsetstrokeopacity{0.332137}%
\pgfsetdash{}{0pt}%
\pgfpathmoveto{\pgfqpoint{1.833179in}{1.977949in}}%
\pgfpathcurveto{\pgfqpoint{1.841415in}{1.977949in}}{\pgfqpoint{1.849315in}{1.981221in}}{\pgfqpoint{1.855139in}{1.987045in}}%
\pgfpathcurveto{\pgfqpoint{1.860963in}{1.992869in}}{\pgfqpoint{1.864235in}{2.000769in}}{\pgfqpoint{1.864235in}{2.009005in}}%
\pgfpathcurveto{\pgfqpoint{1.864235in}{2.017241in}}{\pgfqpoint{1.860963in}{2.025141in}}{\pgfqpoint{1.855139in}{2.030965in}}%
\pgfpathcurveto{\pgfqpoint{1.849315in}{2.036789in}}{\pgfqpoint{1.841415in}{2.040062in}}{\pgfqpoint{1.833179in}{2.040062in}}%
\pgfpathcurveto{\pgfqpoint{1.824942in}{2.040062in}}{\pgfqpoint{1.817042in}{2.036789in}}{\pgfqpoint{1.811219in}{2.030965in}}%
\pgfpathcurveto{\pgfqpoint{1.805395in}{2.025141in}}{\pgfqpoint{1.802122in}{2.017241in}}{\pgfqpoint{1.802122in}{2.009005in}}%
\pgfpathcurveto{\pgfqpoint{1.802122in}{2.000769in}}{\pgfqpoint{1.805395in}{1.992869in}}{\pgfqpoint{1.811219in}{1.987045in}}%
\pgfpathcurveto{\pgfqpoint{1.817042in}{1.981221in}}{\pgfqpoint{1.824942in}{1.977949in}}{\pgfqpoint{1.833179in}{1.977949in}}%
\pgfpathclose%
\pgfusepath{stroke,fill}%
\end{pgfscope}%
\begin{pgfscope}%
\pgfpathrectangle{\pgfqpoint{0.100000in}{0.212622in}}{\pgfqpoint{3.696000in}{3.696000in}}%
\pgfusepath{clip}%
\pgfsetbuttcap%
\pgfsetroundjoin%
\definecolor{currentfill}{rgb}{0.121569,0.466667,0.705882}%
\pgfsetfillcolor{currentfill}%
\pgfsetfillopacity{0.332283}%
\pgfsetlinewidth{1.003750pt}%
\definecolor{currentstroke}{rgb}{0.121569,0.466667,0.705882}%
\pgfsetstrokecolor{currentstroke}%
\pgfsetstrokeopacity{0.332283}%
\pgfsetdash{}{0pt}%
\pgfpathmoveto{\pgfqpoint{1.832421in}{1.977157in}}%
\pgfpathcurveto{\pgfqpoint{1.840657in}{1.977157in}}{\pgfqpoint{1.848557in}{1.980430in}}{\pgfqpoint{1.854381in}{1.986254in}}%
\pgfpathcurveto{\pgfqpoint{1.860205in}{1.992078in}}{\pgfqpoint{1.863477in}{1.999978in}}{\pgfqpoint{1.863477in}{2.008214in}}%
\pgfpathcurveto{\pgfqpoint{1.863477in}{2.016450in}}{\pgfqpoint{1.860205in}{2.024350in}}{\pgfqpoint{1.854381in}{2.030174in}}%
\pgfpathcurveto{\pgfqpoint{1.848557in}{2.035998in}}{\pgfqpoint{1.840657in}{2.039270in}}{\pgfqpoint{1.832421in}{2.039270in}}%
\pgfpathcurveto{\pgfqpoint{1.824184in}{2.039270in}}{\pgfqpoint{1.816284in}{2.035998in}}{\pgfqpoint{1.810460in}{2.030174in}}%
\pgfpathcurveto{\pgfqpoint{1.804636in}{2.024350in}}{\pgfqpoint{1.801364in}{2.016450in}}{\pgfqpoint{1.801364in}{2.008214in}}%
\pgfpathcurveto{\pgfqpoint{1.801364in}{1.999978in}}{\pgfqpoint{1.804636in}{1.992078in}}{\pgfqpoint{1.810460in}{1.986254in}}%
\pgfpathcurveto{\pgfqpoint{1.816284in}{1.980430in}}{\pgfqpoint{1.824184in}{1.977157in}}{\pgfqpoint{1.832421in}{1.977157in}}%
\pgfpathclose%
\pgfusepath{stroke,fill}%
\end{pgfscope}%
\begin{pgfscope}%
\pgfpathrectangle{\pgfqpoint{0.100000in}{0.212622in}}{\pgfqpoint{3.696000in}{3.696000in}}%
\pgfusepath{clip}%
\pgfsetbuttcap%
\pgfsetroundjoin%
\definecolor{currentfill}{rgb}{0.121569,0.466667,0.705882}%
\pgfsetfillcolor{currentfill}%
\pgfsetfillopacity{0.332514}%
\pgfsetlinewidth{1.003750pt}%
\definecolor{currentstroke}{rgb}{0.121569,0.466667,0.705882}%
\pgfsetstrokecolor{currentstroke}%
\pgfsetstrokeopacity{0.332514}%
\pgfsetdash{}{0pt}%
\pgfpathmoveto{\pgfqpoint{1.830629in}{1.976104in}}%
\pgfpathcurveto{\pgfqpoint{1.838865in}{1.976104in}}{\pgfqpoint{1.846765in}{1.979376in}}{\pgfqpoint{1.852589in}{1.985200in}}%
\pgfpathcurveto{\pgfqpoint{1.858413in}{1.991024in}}{\pgfqpoint{1.861685in}{1.998924in}}{\pgfqpoint{1.861685in}{2.007160in}}%
\pgfpathcurveto{\pgfqpoint{1.861685in}{2.015397in}}{\pgfqpoint{1.858413in}{2.023297in}}{\pgfqpoint{1.852589in}{2.029121in}}%
\pgfpathcurveto{\pgfqpoint{1.846765in}{2.034944in}}{\pgfqpoint{1.838865in}{2.038217in}}{\pgfqpoint{1.830629in}{2.038217in}}%
\pgfpathcurveto{\pgfqpoint{1.822392in}{2.038217in}}{\pgfqpoint{1.814492in}{2.034944in}}{\pgfqpoint{1.808668in}{2.029121in}}%
\pgfpathcurveto{\pgfqpoint{1.802845in}{2.023297in}}{\pgfqpoint{1.799572in}{2.015397in}}{\pgfqpoint{1.799572in}{2.007160in}}%
\pgfpathcurveto{\pgfqpoint{1.799572in}{1.998924in}}{\pgfqpoint{1.802845in}{1.991024in}}{\pgfqpoint{1.808668in}{1.985200in}}%
\pgfpathcurveto{\pgfqpoint{1.814492in}{1.979376in}}{\pgfqpoint{1.822392in}{1.976104in}}{\pgfqpoint{1.830629in}{1.976104in}}%
\pgfpathclose%
\pgfusepath{stroke,fill}%
\end{pgfscope}%
\begin{pgfscope}%
\pgfpathrectangle{\pgfqpoint{0.100000in}{0.212622in}}{\pgfqpoint{3.696000in}{3.696000in}}%
\pgfusepath{clip}%
\pgfsetbuttcap%
\pgfsetroundjoin%
\definecolor{currentfill}{rgb}{0.121569,0.466667,0.705882}%
\pgfsetfillcolor{currentfill}%
\pgfsetfillopacity{0.333070}%
\pgfsetlinewidth{1.003750pt}%
\definecolor{currentstroke}{rgb}{0.121569,0.466667,0.705882}%
\pgfsetstrokecolor{currentstroke}%
\pgfsetstrokeopacity{0.333070}%
\pgfsetdash{}{0pt}%
\pgfpathmoveto{\pgfqpoint{1.828329in}{1.973734in}}%
\pgfpathcurveto{\pgfqpoint{1.836566in}{1.973734in}}{\pgfqpoint{1.844466in}{1.977006in}}{\pgfqpoint{1.850290in}{1.982830in}}%
\pgfpathcurveto{\pgfqpoint{1.856113in}{1.988654in}}{\pgfqpoint{1.859386in}{1.996554in}}{\pgfqpoint{1.859386in}{2.004790in}}%
\pgfpathcurveto{\pgfqpoint{1.859386in}{2.013026in}}{\pgfqpoint{1.856113in}{2.020926in}}{\pgfqpoint{1.850290in}{2.026750in}}%
\pgfpathcurveto{\pgfqpoint{1.844466in}{2.032574in}}{\pgfqpoint{1.836566in}{2.035847in}}{\pgfqpoint{1.828329in}{2.035847in}}%
\pgfpathcurveto{\pgfqpoint{1.820093in}{2.035847in}}{\pgfqpoint{1.812193in}{2.032574in}}{\pgfqpoint{1.806369in}{2.026750in}}%
\pgfpathcurveto{\pgfqpoint{1.800545in}{2.020926in}}{\pgfqpoint{1.797273in}{2.013026in}}{\pgfqpoint{1.797273in}{2.004790in}}%
\pgfpathcurveto{\pgfqpoint{1.797273in}{1.996554in}}{\pgfqpoint{1.800545in}{1.988654in}}{\pgfqpoint{1.806369in}{1.982830in}}%
\pgfpathcurveto{\pgfqpoint{1.812193in}{1.977006in}}{\pgfqpoint{1.820093in}{1.973734in}}{\pgfqpoint{1.828329in}{1.973734in}}%
\pgfpathclose%
\pgfusepath{stroke,fill}%
\end{pgfscope}%
\begin{pgfscope}%
\pgfpathrectangle{\pgfqpoint{0.100000in}{0.212622in}}{\pgfqpoint{3.696000in}{3.696000in}}%
\pgfusepath{clip}%
\pgfsetbuttcap%
\pgfsetroundjoin%
\definecolor{currentfill}{rgb}{0.121569,0.466667,0.705882}%
\pgfsetfillcolor{currentfill}%
\pgfsetfillopacity{0.334037}%
\pgfsetlinewidth{1.003750pt}%
\definecolor{currentstroke}{rgb}{0.121569,0.466667,0.705882}%
\pgfsetstrokecolor{currentstroke}%
\pgfsetstrokeopacity{0.334037}%
\pgfsetdash{}{0pt}%
\pgfpathmoveto{\pgfqpoint{1.821844in}{1.972704in}}%
\pgfpathcurveto{\pgfqpoint{1.830080in}{1.972704in}}{\pgfqpoint{1.837980in}{1.975977in}}{\pgfqpoint{1.843804in}{1.981800in}}%
\pgfpathcurveto{\pgfqpoint{1.849628in}{1.987624in}}{\pgfqpoint{1.852900in}{1.995524in}}{\pgfqpoint{1.852900in}{2.003761in}}%
\pgfpathcurveto{\pgfqpoint{1.852900in}{2.011997in}}{\pgfqpoint{1.849628in}{2.019897in}}{\pgfqpoint{1.843804in}{2.025721in}}%
\pgfpathcurveto{\pgfqpoint{1.837980in}{2.031545in}}{\pgfqpoint{1.830080in}{2.034817in}}{\pgfqpoint{1.821844in}{2.034817in}}%
\pgfpathcurveto{\pgfqpoint{1.813607in}{2.034817in}}{\pgfqpoint{1.805707in}{2.031545in}}{\pgfqpoint{1.799883in}{2.025721in}}%
\pgfpathcurveto{\pgfqpoint{1.794059in}{2.019897in}}{\pgfqpoint{1.790787in}{2.011997in}}{\pgfqpoint{1.790787in}{2.003761in}}%
\pgfpathcurveto{\pgfqpoint{1.790787in}{1.995524in}}{\pgfqpoint{1.794059in}{1.987624in}}{\pgfqpoint{1.799883in}{1.981800in}}%
\pgfpathcurveto{\pgfqpoint{1.805707in}{1.975977in}}{\pgfqpoint{1.813607in}{1.972704in}}{\pgfqpoint{1.821844in}{1.972704in}}%
\pgfpathclose%
\pgfusepath{stroke,fill}%
\end{pgfscope}%
\begin{pgfscope}%
\pgfpathrectangle{\pgfqpoint{0.100000in}{0.212622in}}{\pgfqpoint{3.696000in}{3.696000in}}%
\pgfusepath{clip}%
\pgfsetbuttcap%
\pgfsetroundjoin%
\definecolor{currentfill}{rgb}{0.121569,0.466667,0.705882}%
\pgfsetfillcolor{currentfill}%
\pgfsetfillopacity{0.334682}%
\pgfsetlinewidth{1.003750pt}%
\definecolor{currentstroke}{rgb}{0.121569,0.466667,0.705882}%
\pgfsetstrokecolor{currentstroke}%
\pgfsetstrokeopacity{0.334682}%
\pgfsetdash{}{0pt}%
\pgfpathmoveto{\pgfqpoint{1.819662in}{1.969793in}}%
\pgfpathcurveto{\pgfqpoint{1.827898in}{1.969793in}}{\pgfqpoint{1.835798in}{1.973065in}}{\pgfqpoint{1.841622in}{1.978889in}}%
\pgfpathcurveto{\pgfqpoint{1.847446in}{1.984713in}}{\pgfqpoint{1.850719in}{1.992613in}}{\pgfqpoint{1.850719in}{2.000849in}}%
\pgfpathcurveto{\pgfqpoint{1.850719in}{2.009086in}}{\pgfqpoint{1.847446in}{2.016986in}}{\pgfqpoint{1.841622in}{2.022810in}}%
\pgfpathcurveto{\pgfqpoint{1.835798in}{2.028633in}}{\pgfqpoint{1.827898in}{2.031906in}}{\pgfqpoint{1.819662in}{2.031906in}}%
\pgfpathcurveto{\pgfqpoint{1.811426in}{2.031906in}}{\pgfqpoint{1.803526in}{2.028633in}}{\pgfqpoint{1.797702in}{2.022810in}}%
\pgfpathcurveto{\pgfqpoint{1.791878in}{2.016986in}}{\pgfqpoint{1.788606in}{2.009086in}}{\pgfqpoint{1.788606in}{2.000849in}}%
\pgfpathcurveto{\pgfqpoint{1.788606in}{1.992613in}}{\pgfqpoint{1.791878in}{1.984713in}}{\pgfqpoint{1.797702in}{1.978889in}}%
\pgfpathcurveto{\pgfqpoint{1.803526in}{1.973065in}}{\pgfqpoint{1.811426in}{1.969793in}}{\pgfqpoint{1.819662in}{1.969793in}}%
\pgfpathclose%
\pgfusepath{stroke,fill}%
\end{pgfscope}%
\begin{pgfscope}%
\pgfpathrectangle{\pgfqpoint{0.100000in}{0.212622in}}{\pgfqpoint{3.696000in}{3.696000in}}%
\pgfusepath{clip}%
\pgfsetbuttcap%
\pgfsetroundjoin%
\definecolor{currentfill}{rgb}{0.121569,0.466667,0.705882}%
\pgfsetfillcolor{currentfill}%
\pgfsetfillopacity{0.334829}%
\pgfsetlinewidth{1.003750pt}%
\definecolor{currentstroke}{rgb}{0.121569,0.466667,0.705882}%
\pgfsetstrokecolor{currentstroke}%
\pgfsetstrokeopacity{0.334829}%
\pgfsetdash{}{0pt}%
\pgfpathmoveto{\pgfqpoint{1.529027in}{2.121612in}}%
\pgfpathcurveto{\pgfqpoint{1.537263in}{2.121612in}}{\pgfqpoint{1.545163in}{2.124884in}}{\pgfqpoint{1.550987in}{2.130708in}}%
\pgfpathcurveto{\pgfqpoint{1.556811in}{2.136532in}}{\pgfqpoint{1.560083in}{2.144432in}}{\pgfqpoint{1.560083in}{2.152668in}}%
\pgfpathcurveto{\pgfqpoint{1.560083in}{2.160904in}}{\pgfqpoint{1.556811in}{2.168804in}}{\pgfqpoint{1.550987in}{2.174628in}}%
\pgfpathcurveto{\pgfqpoint{1.545163in}{2.180452in}}{\pgfqpoint{1.537263in}{2.183725in}}{\pgfqpoint{1.529027in}{2.183725in}}%
\pgfpathcurveto{\pgfqpoint{1.520790in}{2.183725in}}{\pgfqpoint{1.512890in}{2.180452in}}{\pgfqpoint{1.507066in}{2.174628in}}%
\pgfpathcurveto{\pgfqpoint{1.501242in}{2.168804in}}{\pgfqpoint{1.497970in}{2.160904in}}{\pgfqpoint{1.497970in}{2.152668in}}%
\pgfpathcurveto{\pgfqpoint{1.497970in}{2.144432in}}{\pgfqpoint{1.501242in}{2.136532in}}{\pgfqpoint{1.507066in}{2.130708in}}%
\pgfpathcurveto{\pgfqpoint{1.512890in}{2.124884in}}{\pgfqpoint{1.520790in}{2.121612in}}{\pgfqpoint{1.529027in}{2.121612in}}%
\pgfpathclose%
\pgfusepath{stroke,fill}%
\end{pgfscope}%
\begin{pgfscope}%
\pgfpathrectangle{\pgfqpoint{0.100000in}{0.212622in}}{\pgfqpoint{3.696000in}{3.696000in}}%
\pgfusepath{clip}%
\pgfsetbuttcap%
\pgfsetroundjoin%
\definecolor{currentfill}{rgb}{0.121569,0.466667,0.705882}%
\pgfsetfillcolor{currentfill}%
\pgfsetfillopacity{0.335939}%
\pgfsetlinewidth{1.003750pt}%
\definecolor{currentstroke}{rgb}{0.121569,0.466667,0.705882}%
\pgfsetstrokecolor{currentstroke}%
\pgfsetstrokeopacity{0.335939}%
\pgfsetdash{}{0pt}%
\pgfpathmoveto{\pgfqpoint{1.813044in}{1.968845in}}%
\pgfpathcurveto{\pgfqpoint{1.821280in}{1.968845in}}{\pgfqpoint{1.829180in}{1.972118in}}{\pgfqpoint{1.835004in}{1.977942in}}%
\pgfpathcurveto{\pgfqpoint{1.840828in}{1.983766in}}{\pgfqpoint{1.844100in}{1.991666in}}{\pgfqpoint{1.844100in}{1.999902in}}%
\pgfpathcurveto{\pgfqpoint{1.844100in}{2.008138in}}{\pgfqpoint{1.840828in}{2.016038in}}{\pgfqpoint{1.835004in}{2.021862in}}%
\pgfpathcurveto{\pgfqpoint{1.829180in}{2.027686in}}{\pgfqpoint{1.821280in}{2.030958in}}{\pgfqpoint{1.813044in}{2.030958in}}%
\pgfpathcurveto{\pgfqpoint{1.804807in}{2.030958in}}{\pgfqpoint{1.796907in}{2.027686in}}{\pgfqpoint{1.791083in}{2.021862in}}%
\pgfpathcurveto{\pgfqpoint{1.785259in}{2.016038in}}{\pgfqpoint{1.781987in}{2.008138in}}{\pgfqpoint{1.781987in}{1.999902in}}%
\pgfpathcurveto{\pgfqpoint{1.781987in}{1.991666in}}{\pgfqpoint{1.785259in}{1.983766in}}{\pgfqpoint{1.791083in}{1.977942in}}%
\pgfpathcurveto{\pgfqpoint{1.796907in}{1.972118in}}{\pgfqpoint{1.804807in}{1.968845in}}{\pgfqpoint{1.813044in}{1.968845in}}%
\pgfpathclose%
\pgfusepath{stroke,fill}%
\end{pgfscope}%
\begin{pgfscope}%
\pgfpathrectangle{\pgfqpoint{0.100000in}{0.212622in}}{\pgfqpoint{3.696000in}{3.696000in}}%
\pgfusepath{clip}%
\pgfsetbuttcap%
\pgfsetroundjoin%
\definecolor{currentfill}{rgb}{0.121569,0.466667,0.705882}%
\pgfsetfillcolor{currentfill}%
\pgfsetfillopacity{0.336315}%
\pgfsetlinewidth{1.003750pt}%
\definecolor{currentstroke}{rgb}{0.121569,0.466667,0.705882}%
\pgfsetstrokecolor{currentstroke}%
\pgfsetstrokeopacity{0.336315}%
\pgfsetdash{}{0pt}%
\pgfpathmoveto{\pgfqpoint{1.809706in}{1.965151in}}%
\pgfpathcurveto{\pgfqpoint{1.817942in}{1.965151in}}{\pgfqpoint{1.825842in}{1.968424in}}{\pgfqpoint{1.831666in}{1.974248in}}%
\pgfpathcurveto{\pgfqpoint{1.837490in}{1.980071in}}{\pgfqpoint{1.840762in}{1.987971in}}{\pgfqpoint{1.840762in}{1.996208in}}%
\pgfpathcurveto{\pgfqpoint{1.840762in}{2.004444in}}{\pgfqpoint{1.837490in}{2.012344in}}{\pgfqpoint{1.831666in}{2.018168in}}%
\pgfpathcurveto{\pgfqpoint{1.825842in}{2.023992in}}{\pgfqpoint{1.817942in}{2.027264in}}{\pgfqpoint{1.809706in}{2.027264in}}%
\pgfpathcurveto{\pgfqpoint{1.801469in}{2.027264in}}{\pgfqpoint{1.793569in}{2.023992in}}{\pgfqpoint{1.787745in}{2.018168in}}%
\pgfpathcurveto{\pgfqpoint{1.781921in}{2.012344in}}{\pgfqpoint{1.778649in}{2.004444in}}{\pgfqpoint{1.778649in}{1.996208in}}%
\pgfpathcurveto{\pgfqpoint{1.778649in}{1.987971in}}{\pgfqpoint{1.781921in}{1.980071in}}{\pgfqpoint{1.787745in}{1.974248in}}%
\pgfpathcurveto{\pgfqpoint{1.793569in}{1.968424in}}{\pgfqpoint{1.801469in}{1.965151in}}{\pgfqpoint{1.809706in}{1.965151in}}%
\pgfpathclose%
\pgfusepath{stroke,fill}%
\end{pgfscope}%
\begin{pgfscope}%
\pgfpathrectangle{\pgfqpoint{0.100000in}{0.212622in}}{\pgfqpoint{3.696000in}{3.696000in}}%
\pgfusepath{clip}%
\pgfsetbuttcap%
\pgfsetroundjoin%
\definecolor{currentfill}{rgb}{0.121569,0.466667,0.705882}%
\pgfsetfillcolor{currentfill}%
\pgfsetfillopacity{0.337860}%
\pgfsetlinewidth{1.003750pt}%
\definecolor{currentstroke}{rgb}{0.121569,0.466667,0.705882}%
\pgfsetstrokecolor{currentstroke}%
\pgfsetstrokeopacity{0.337860}%
\pgfsetdash{}{0pt}%
\pgfpathmoveto{\pgfqpoint{1.805106in}{1.962549in}}%
\pgfpathcurveto{\pgfqpoint{1.813343in}{1.962549in}}{\pgfqpoint{1.821243in}{1.965821in}}{\pgfqpoint{1.827067in}{1.971645in}}%
\pgfpathcurveto{\pgfqpoint{1.832891in}{1.977469in}}{\pgfqpoint{1.836163in}{1.985369in}}{\pgfqpoint{1.836163in}{1.993605in}}%
\pgfpathcurveto{\pgfqpoint{1.836163in}{2.001841in}}{\pgfqpoint{1.832891in}{2.009741in}}{\pgfqpoint{1.827067in}{2.015565in}}%
\pgfpathcurveto{\pgfqpoint{1.821243in}{2.021389in}}{\pgfqpoint{1.813343in}{2.024662in}}{\pgfqpoint{1.805106in}{2.024662in}}%
\pgfpathcurveto{\pgfqpoint{1.796870in}{2.024662in}}{\pgfqpoint{1.788970in}{2.021389in}}{\pgfqpoint{1.783146in}{2.015565in}}%
\pgfpathcurveto{\pgfqpoint{1.777322in}{2.009741in}}{\pgfqpoint{1.774050in}{2.001841in}}{\pgfqpoint{1.774050in}{1.993605in}}%
\pgfpathcurveto{\pgfqpoint{1.774050in}{1.985369in}}{\pgfqpoint{1.777322in}{1.977469in}}{\pgfqpoint{1.783146in}{1.971645in}}%
\pgfpathcurveto{\pgfqpoint{1.788970in}{1.965821in}}{\pgfqpoint{1.796870in}{1.962549in}}{\pgfqpoint{1.805106in}{1.962549in}}%
\pgfpathclose%
\pgfusepath{stroke,fill}%
\end{pgfscope}%
\begin{pgfscope}%
\pgfpathrectangle{\pgfqpoint{0.100000in}{0.212622in}}{\pgfqpoint{3.696000in}{3.696000in}}%
\pgfusepath{clip}%
\pgfsetbuttcap%
\pgfsetroundjoin%
\definecolor{currentfill}{rgb}{0.121569,0.466667,0.705882}%
\pgfsetfillcolor{currentfill}%
\pgfsetfillopacity{0.338428}%
\pgfsetlinewidth{1.003750pt}%
\definecolor{currentstroke}{rgb}{0.121569,0.466667,0.705882}%
\pgfsetstrokecolor{currentstroke}%
\pgfsetstrokeopacity{0.338428}%
\pgfsetdash{}{0pt}%
\pgfpathmoveto{\pgfqpoint{1.801400in}{1.960299in}}%
\pgfpathcurveto{\pgfqpoint{1.809636in}{1.960299in}}{\pgfqpoint{1.817536in}{1.963571in}}{\pgfqpoint{1.823360in}{1.969395in}}%
\pgfpathcurveto{\pgfqpoint{1.829184in}{1.975219in}}{\pgfqpoint{1.832456in}{1.983119in}}{\pgfqpoint{1.832456in}{1.991355in}}%
\pgfpathcurveto{\pgfqpoint{1.832456in}{1.999592in}}{\pgfqpoint{1.829184in}{2.007492in}}{\pgfqpoint{1.823360in}{2.013316in}}%
\pgfpathcurveto{\pgfqpoint{1.817536in}{2.019140in}}{\pgfqpoint{1.809636in}{2.022412in}}{\pgfqpoint{1.801400in}{2.022412in}}%
\pgfpathcurveto{\pgfqpoint{1.793164in}{2.022412in}}{\pgfqpoint{1.785263in}{2.019140in}}{\pgfqpoint{1.779440in}{2.013316in}}%
\pgfpathcurveto{\pgfqpoint{1.773616in}{2.007492in}}{\pgfqpoint{1.770343in}{1.999592in}}{\pgfqpoint{1.770343in}{1.991355in}}%
\pgfpathcurveto{\pgfqpoint{1.770343in}{1.983119in}}{\pgfqpoint{1.773616in}{1.975219in}}{\pgfqpoint{1.779440in}{1.969395in}}%
\pgfpathcurveto{\pgfqpoint{1.785263in}{1.963571in}}{\pgfqpoint{1.793164in}{1.960299in}}{\pgfqpoint{1.801400in}{1.960299in}}%
\pgfpathclose%
\pgfusepath{stroke,fill}%
\end{pgfscope}%
\begin{pgfscope}%
\pgfpathrectangle{\pgfqpoint{0.100000in}{0.212622in}}{\pgfqpoint{3.696000in}{3.696000in}}%
\pgfusepath{clip}%
\pgfsetbuttcap%
\pgfsetroundjoin%
\definecolor{currentfill}{rgb}{0.121569,0.466667,0.705882}%
\pgfsetfillcolor{currentfill}%
\pgfsetfillopacity{0.338983}%
\pgfsetlinewidth{1.003750pt}%
\definecolor{currentstroke}{rgb}{0.121569,0.466667,0.705882}%
\pgfsetstrokecolor{currentstroke}%
\pgfsetstrokeopacity{0.338983}%
\pgfsetdash{}{0pt}%
\pgfpathmoveto{\pgfqpoint{1.800995in}{1.960098in}}%
\pgfpathcurveto{\pgfqpoint{1.809231in}{1.960098in}}{\pgfqpoint{1.817131in}{1.963370in}}{\pgfqpoint{1.822955in}{1.969194in}}%
\pgfpathcurveto{\pgfqpoint{1.828779in}{1.975018in}}{\pgfqpoint{1.832051in}{1.982918in}}{\pgfqpoint{1.832051in}{1.991154in}}%
\pgfpathcurveto{\pgfqpoint{1.832051in}{1.999390in}}{\pgfqpoint{1.828779in}{2.007290in}}{\pgfqpoint{1.822955in}{2.013114in}}%
\pgfpathcurveto{\pgfqpoint{1.817131in}{2.018938in}}{\pgfqpoint{1.809231in}{2.022211in}}{\pgfqpoint{1.800995in}{2.022211in}}%
\pgfpathcurveto{\pgfqpoint{1.792758in}{2.022211in}}{\pgfqpoint{1.784858in}{2.018938in}}{\pgfqpoint{1.779034in}{2.013114in}}%
\pgfpathcurveto{\pgfqpoint{1.773210in}{2.007290in}}{\pgfqpoint{1.769938in}{1.999390in}}{\pgfqpoint{1.769938in}{1.991154in}}%
\pgfpathcurveto{\pgfqpoint{1.769938in}{1.982918in}}{\pgfqpoint{1.773210in}{1.975018in}}{\pgfqpoint{1.779034in}{1.969194in}}%
\pgfpathcurveto{\pgfqpoint{1.784858in}{1.963370in}}{\pgfqpoint{1.792758in}{1.960098in}}{\pgfqpoint{1.800995in}{1.960098in}}%
\pgfpathclose%
\pgfusepath{stroke,fill}%
\end{pgfscope}%
\begin{pgfscope}%
\pgfpathrectangle{\pgfqpoint{0.100000in}{0.212622in}}{\pgfqpoint{3.696000in}{3.696000in}}%
\pgfusepath{clip}%
\pgfsetbuttcap%
\pgfsetroundjoin%
\definecolor{currentfill}{rgb}{0.121569,0.466667,0.705882}%
\pgfsetfillcolor{currentfill}%
\pgfsetfillopacity{0.339527}%
\pgfsetlinewidth{1.003750pt}%
\definecolor{currentstroke}{rgb}{0.121569,0.466667,0.705882}%
\pgfsetstrokecolor{currentstroke}%
\pgfsetstrokeopacity{0.339527}%
\pgfsetdash{}{0pt}%
\pgfpathmoveto{\pgfqpoint{1.797024in}{1.960949in}}%
\pgfpathcurveto{\pgfqpoint{1.805261in}{1.960949in}}{\pgfqpoint{1.813161in}{1.964222in}}{\pgfqpoint{1.818985in}{1.970046in}}%
\pgfpathcurveto{\pgfqpoint{1.824809in}{1.975870in}}{\pgfqpoint{1.828081in}{1.983770in}}{\pgfqpoint{1.828081in}{1.992006in}}%
\pgfpathcurveto{\pgfqpoint{1.828081in}{2.000242in}}{\pgfqpoint{1.824809in}{2.008142in}}{\pgfqpoint{1.818985in}{2.013966in}}%
\pgfpathcurveto{\pgfqpoint{1.813161in}{2.019790in}}{\pgfqpoint{1.805261in}{2.023062in}}{\pgfqpoint{1.797024in}{2.023062in}}%
\pgfpathcurveto{\pgfqpoint{1.788788in}{2.023062in}}{\pgfqpoint{1.780888in}{2.019790in}}{\pgfqpoint{1.775064in}{2.013966in}}%
\pgfpathcurveto{\pgfqpoint{1.769240in}{2.008142in}}{\pgfqpoint{1.765968in}{2.000242in}}{\pgfqpoint{1.765968in}{1.992006in}}%
\pgfpathcurveto{\pgfqpoint{1.765968in}{1.983770in}}{\pgfqpoint{1.769240in}{1.975870in}}{\pgfqpoint{1.775064in}{1.970046in}}%
\pgfpathcurveto{\pgfqpoint{1.780888in}{1.964222in}}{\pgfqpoint{1.788788in}{1.960949in}}{\pgfqpoint{1.797024in}{1.960949in}}%
\pgfpathclose%
\pgfusepath{stroke,fill}%
\end{pgfscope}%
\begin{pgfscope}%
\pgfpathrectangle{\pgfqpoint{0.100000in}{0.212622in}}{\pgfqpoint{3.696000in}{3.696000in}}%
\pgfusepath{clip}%
\pgfsetbuttcap%
\pgfsetroundjoin%
\definecolor{currentfill}{rgb}{0.121569,0.466667,0.705882}%
\pgfsetfillcolor{currentfill}%
\pgfsetfillopacity{0.339705}%
\pgfsetlinewidth{1.003750pt}%
\definecolor{currentstroke}{rgb}{0.121569,0.466667,0.705882}%
\pgfsetstrokecolor{currentstroke}%
\pgfsetstrokeopacity{0.339705}%
\pgfsetdash{}{0pt}%
\pgfpathmoveto{\pgfqpoint{1.795887in}{1.959455in}}%
\pgfpathcurveto{\pgfqpoint{1.804123in}{1.959455in}}{\pgfqpoint{1.812023in}{1.962727in}}{\pgfqpoint{1.817847in}{1.968551in}}%
\pgfpathcurveto{\pgfqpoint{1.823671in}{1.974375in}}{\pgfqpoint{1.826943in}{1.982275in}}{\pgfqpoint{1.826943in}{1.990512in}}%
\pgfpathcurveto{\pgfqpoint{1.826943in}{1.998748in}}{\pgfqpoint{1.823671in}{2.006648in}}{\pgfqpoint{1.817847in}{2.012472in}}%
\pgfpathcurveto{\pgfqpoint{1.812023in}{2.018296in}}{\pgfqpoint{1.804123in}{2.021568in}}{\pgfqpoint{1.795887in}{2.021568in}}%
\pgfpathcurveto{\pgfqpoint{1.787650in}{2.021568in}}{\pgfqpoint{1.779750in}{2.018296in}}{\pgfqpoint{1.773926in}{2.012472in}}%
\pgfpathcurveto{\pgfqpoint{1.768102in}{2.006648in}}{\pgfqpoint{1.764830in}{1.998748in}}{\pgfqpoint{1.764830in}{1.990512in}}%
\pgfpathcurveto{\pgfqpoint{1.764830in}{1.982275in}}{\pgfqpoint{1.768102in}{1.974375in}}{\pgfqpoint{1.773926in}{1.968551in}}%
\pgfpathcurveto{\pgfqpoint{1.779750in}{1.962727in}}{\pgfqpoint{1.787650in}{1.959455in}}{\pgfqpoint{1.795887in}{1.959455in}}%
\pgfpathclose%
\pgfusepath{stroke,fill}%
\end{pgfscope}%
\begin{pgfscope}%
\pgfpathrectangle{\pgfqpoint{0.100000in}{0.212622in}}{\pgfqpoint{3.696000in}{3.696000in}}%
\pgfusepath{clip}%
\pgfsetbuttcap%
\pgfsetroundjoin%
\definecolor{currentfill}{rgb}{0.121569,0.466667,0.705882}%
\pgfsetfillcolor{currentfill}%
\pgfsetfillopacity{0.340160}%
\pgfsetlinewidth{1.003750pt}%
\definecolor{currentstroke}{rgb}{0.121569,0.466667,0.705882}%
\pgfsetstrokecolor{currentstroke}%
\pgfsetstrokeopacity{0.340160}%
\pgfsetdash{}{0pt}%
\pgfpathmoveto{\pgfqpoint{1.793605in}{1.957970in}}%
\pgfpathcurveto{\pgfqpoint{1.801841in}{1.957970in}}{\pgfqpoint{1.809741in}{1.961242in}}{\pgfqpoint{1.815565in}{1.967066in}}%
\pgfpathcurveto{\pgfqpoint{1.821389in}{1.972890in}}{\pgfqpoint{1.824661in}{1.980790in}}{\pgfqpoint{1.824661in}{1.989026in}}%
\pgfpathcurveto{\pgfqpoint{1.824661in}{1.997263in}}{\pgfqpoint{1.821389in}{2.005163in}}{\pgfqpoint{1.815565in}{2.010987in}}%
\pgfpathcurveto{\pgfqpoint{1.809741in}{2.016811in}}{\pgfqpoint{1.801841in}{2.020083in}}{\pgfqpoint{1.793605in}{2.020083in}}%
\pgfpathcurveto{\pgfqpoint{1.785368in}{2.020083in}}{\pgfqpoint{1.777468in}{2.016811in}}{\pgfqpoint{1.771644in}{2.010987in}}%
\pgfpathcurveto{\pgfqpoint{1.765820in}{2.005163in}}{\pgfqpoint{1.762548in}{1.997263in}}{\pgfqpoint{1.762548in}{1.989026in}}%
\pgfpathcurveto{\pgfqpoint{1.762548in}{1.980790in}}{\pgfqpoint{1.765820in}{1.972890in}}{\pgfqpoint{1.771644in}{1.967066in}}%
\pgfpathcurveto{\pgfqpoint{1.777468in}{1.961242in}}{\pgfqpoint{1.785368in}{1.957970in}}{\pgfqpoint{1.793605in}{1.957970in}}%
\pgfpathclose%
\pgfusepath{stroke,fill}%
\end{pgfscope}%
\begin{pgfscope}%
\pgfpathrectangle{\pgfqpoint{0.100000in}{0.212622in}}{\pgfqpoint{3.696000in}{3.696000in}}%
\pgfusepath{clip}%
\pgfsetbuttcap%
\pgfsetroundjoin%
\definecolor{currentfill}{rgb}{0.121569,0.466667,0.705882}%
\pgfsetfillcolor{currentfill}%
\pgfsetfillopacity{0.340206}%
\pgfsetlinewidth{1.003750pt}%
\definecolor{currentstroke}{rgb}{0.121569,0.466667,0.705882}%
\pgfsetstrokecolor{currentstroke}%
\pgfsetstrokeopacity{0.340206}%
\pgfsetdash{}{0pt}%
\pgfpathmoveto{\pgfqpoint{1.793263in}{1.957750in}}%
\pgfpathcurveto{\pgfqpoint{1.801500in}{1.957750in}}{\pgfqpoint{1.809400in}{1.961022in}}{\pgfqpoint{1.815224in}{1.966846in}}%
\pgfpathcurveto{\pgfqpoint{1.821048in}{1.972670in}}{\pgfqpoint{1.824320in}{1.980570in}}{\pgfqpoint{1.824320in}{1.988806in}}%
\pgfpathcurveto{\pgfqpoint{1.824320in}{1.997042in}}{\pgfqpoint{1.821048in}{2.004942in}}{\pgfqpoint{1.815224in}{2.010766in}}%
\pgfpathcurveto{\pgfqpoint{1.809400in}{2.016590in}}{\pgfqpoint{1.801500in}{2.019863in}}{\pgfqpoint{1.793263in}{2.019863in}}%
\pgfpathcurveto{\pgfqpoint{1.785027in}{2.019863in}}{\pgfqpoint{1.777127in}{2.016590in}}{\pgfqpoint{1.771303in}{2.010766in}}%
\pgfpathcurveto{\pgfqpoint{1.765479in}{2.004942in}}{\pgfqpoint{1.762207in}{1.997042in}}{\pgfqpoint{1.762207in}{1.988806in}}%
\pgfpathcurveto{\pgfqpoint{1.762207in}{1.980570in}}{\pgfqpoint{1.765479in}{1.972670in}}{\pgfqpoint{1.771303in}{1.966846in}}%
\pgfpathcurveto{\pgfqpoint{1.777127in}{1.961022in}}{\pgfqpoint{1.785027in}{1.957750in}}{\pgfqpoint{1.793263in}{1.957750in}}%
\pgfpathclose%
\pgfusepath{stroke,fill}%
\end{pgfscope}%
\begin{pgfscope}%
\pgfpathrectangle{\pgfqpoint{0.100000in}{0.212622in}}{\pgfqpoint{3.696000in}{3.696000in}}%
\pgfusepath{clip}%
\pgfsetbuttcap%
\pgfsetroundjoin%
\definecolor{currentfill}{rgb}{0.121569,0.466667,0.705882}%
\pgfsetfillcolor{currentfill}%
\pgfsetfillopacity{0.340388}%
\pgfsetlinewidth{1.003750pt}%
\definecolor{currentstroke}{rgb}{0.121569,0.466667,0.705882}%
\pgfsetstrokecolor{currentstroke}%
\pgfsetstrokeopacity{0.340388}%
\pgfsetdash{}{0pt}%
\pgfpathmoveto{\pgfqpoint{1.793019in}{1.957580in}}%
\pgfpathcurveto{\pgfqpoint{1.801256in}{1.957580in}}{\pgfqpoint{1.809156in}{1.960852in}}{\pgfqpoint{1.814980in}{1.966676in}}%
\pgfpathcurveto{\pgfqpoint{1.820804in}{1.972500in}}{\pgfqpoint{1.824076in}{1.980400in}}{\pgfqpoint{1.824076in}{1.988636in}}%
\pgfpathcurveto{\pgfqpoint{1.824076in}{1.996872in}}{\pgfqpoint{1.820804in}{2.004772in}}{\pgfqpoint{1.814980in}{2.010596in}}%
\pgfpathcurveto{\pgfqpoint{1.809156in}{2.016420in}}{\pgfqpoint{1.801256in}{2.019693in}}{\pgfqpoint{1.793019in}{2.019693in}}%
\pgfpathcurveto{\pgfqpoint{1.784783in}{2.019693in}}{\pgfqpoint{1.776883in}{2.016420in}}{\pgfqpoint{1.771059in}{2.010596in}}%
\pgfpathcurveto{\pgfqpoint{1.765235in}{2.004772in}}{\pgfqpoint{1.761963in}{1.996872in}}{\pgfqpoint{1.761963in}{1.988636in}}%
\pgfpathcurveto{\pgfqpoint{1.761963in}{1.980400in}}{\pgfqpoint{1.765235in}{1.972500in}}{\pgfqpoint{1.771059in}{1.966676in}}%
\pgfpathcurveto{\pgfqpoint{1.776883in}{1.960852in}}{\pgfqpoint{1.784783in}{1.957580in}}{\pgfqpoint{1.793019in}{1.957580in}}%
\pgfpathclose%
\pgfusepath{stroke,fill}%
\end{pgfscope}%
\begin{pgfscope}%
\pgfpathrectangle{\pgfqpoint{0.100000in}{0.212622in}}{\pgfqpoint{3.696000in}{3.696000in}}%
\pgfusepath{clip}%
\pgfsetbuttcap%
\pgfsetroundjoin%
\definecolor{currentfill}{rgb}{0.121569,0.466667,0.705882}%
\pgfsetfillcolor{currentfill}%
\pgfsetfillopacity{0.340608}%
\pgfsetlinewidth{1.003750pt}%
\definecolor{currentstroke}{rgb}{0.121569,0.466667,0.705882}%
\pgfsetstrokecolor{currentstroke}%
\pgfsetstrokeopacity{0.340608}%
\pgfsetdash{}{0pt}%
\pgfpathmoveto{\pgfqpoint{1.791462in}{1.958231in}}%
\pgfpathcurveto{\pgfqpoint{1.799699in}{1.958231in}}{\pgfqpoint{1.807599in}{1.961504in}}{\pgfqpoint{1.813423in}{1.967328in}}%
\pgfpathcurveto{\pgfqpoint{1.819247in}{1.973151in}}{\pgfqpoint{1.822519in}{1.981051in}}{\pgfqpoint{1.822519in}{1.989288in}}%
\pgfpathcurveto{\pgfqpoint{1.822519in}{1.997524in}}{\pgfqpoint{1.819247in}{2.005424in}}{\pgfqpoint{1.813423in}{2.011248in}}%
\pgfpathcurveto{\pgfqpoint{1.807599in}{2.017072in}}{\pgfqpoint{1.799699in}{2.020344in}}{\pgfqpoint{1.791462in}{2.020344in}}%
\pgfpathcurveto{\pgfqpoint{1.783226in}{2.020344in}}{\pgfqpoint{1.775326in}{2.017072in}}{\pgfqpoint{1.769502in}{2.011248in}}%
\pgfpathcurveto{\pgfqpoint{1.763678in}{2.005424in}}{\pgfqpoint{1.760406in}{1.997524in}}{\pgfqpoint{1.760406in}{1.989288in}}%
\pgfpathcurveto{\pgfqpoint{1.760406in}{1.981051in}}{\pgfqpoint{1.763678in}{1.973151in}}{\pgfqpoint{1.769502in}{1.967328in}}%
\pgfpathcurveto{\pgfqpoint{1.775326in}{1.961504in}}{\pgfqpoint{1.783226in}{1.958231in}}{\pgfqpoint{1.791462in}{1.958231in}}%
\pgfpathclose%
\pgfusepath{stroke,fill}%
\end{pgfscope}%
\begin{pgfscope}%
\pgfpathrectangle{\pgfqpoint{0.100000in}{0.212622in}}{\pgfqpoint{3.696000in}{3.696000in}}%
\pgfusepath{clip}%
\pgfsetbuttcap%
\pgfsetroundjoin%
\definecolor{currentfill}{rgb}{0.121569,0.466667,0.705882}%
\pgfsetfillcolor{currentfill}%
\pgfsetfillopacity{0.340854}%
\pgfsetlinewidth{1.003750pt}%
\definecolor{currentstroke}{rgb}{0.121569,0.466667,0.705882}%
\pgfsetstrokecolor{currentstroke}%
\pgfsetstrokeopacity{0.340854}%
\pgfsetdash{}{0pt}%
\pgfpathmoveto{\pgfqpoint{1.789784in}{1.956121in}}%
\pgfpathcurveto{\pgfqpoint{1.798021in}{1.956121in}}{\pgfqpoint{1.805921in}{1.959393in}}{\pgfqpoint{1.811745in}{1.965217in}}%
\pgfpathcurveto{\pgfqpoint{1.817568in}{1.971041in}}{\pgfqpoint{1.820841in}{1.978941in}}{\pgfqpoint{1.820841in}{1.987177in}}%
\pgfpathcurveto{\pgfqpoint{1.820841in}{1.995413in}}{\pgfqpoint{1.817568in}{2.003313in}}{\pgfqpoint{1.811745in}{2.009137in}}%
\pgfpathcurveto{\pgfqpoint{1.805921in}{2.014961in}}{\pgfqpoint{1.798021in}{2.018234in}}{\pgfqpoint{1.789784in}{2.018234in}}%
\pgfpathcurveto{\pgfqpoint{1.781548in}{2.018234in}}{\pgfqpoint{1.773648in}{2.014961in}}{\pgfqpoint{1.767824in}{2.009137in}}%
\pgfpathcurveto{\pgfqpoint{1.762000in}{2.003313in}}{\pgfqpoint{1.758728in}{1.995413in}}{\pgfqpoint{1.758728in}{1.987177in}}%
\pgfpathcurveto{\pgfqpoint{1.758728in}{1.978941in}}{\pgfqpoint{1.762000in}{1.971041in}}{\pgfqpoint{1.767824in}{1.965217in}}%
\pgfpathcurveto{\pgfqpoint{1.773648in}{1.959393in}}{\pgfqpoint{1.781548in}{1.956121in}}{\pgfqpoint{1.789784in}{1.956121in}}%
\pgfpathclose%
\pgfusepath{stroke,fill}%
\end{pgfscope}%
\begin{pgfscope}%
\pgfpathrectangle{\pgfqpoint{0.100000in}{0.212622in}}{\pgfqpoint{3.696000in}{3.696000in}}%
\pgfusepath{clip}%
\pgfsetbuttcap%
\pgfsetroundjoin%
\definecolor{currentfill}{rgb}{0.121569,0.466667,0.705882}%
\pgfsetfillcolor{currentfill}%
\pgfsetfillopacity{0.341544}%
\pgfsetlinewidth{1.003750pt}%
\definecolor{currentstroke}{rgb}{0.121569,0.466667,0.705882}%
\pgfsetstrokecolor{currentstroke}%
\pgfsetstrokeopacity{0.341544}%
\pgfsetdash{}{0pt}%
\pgfpathmoveto{\pgfqpoint{1.786050in}{1.955056in}}%
\pgfpathcurveto{\pgfqpoint{1.794286in}{1.955056in}}{\pgfqpoint{1.802186in}{1.958328in}}{\pgfqpoint{1.808010in}{1.964152in}}%
\pgfpathcurveto{\pgfqpoint{1.813834in}{1.969976in}}{\pgfqpoint{1.817106in}{1.977876in}}{\pgfqpoint{1.817106in}{1.986112in}}%
\pgfpathcurveto{\pgfqpoint{1.817106in}{1.994349in}}{\pgfqpoint{1.813834in}{2.002249in}}{\pgfqpoint{1.808010in}{2.008073in}}%
\pgfpathcurveto{\pgfqpoint{1.802186in}{2.013896in}}{\pgfqpoint{1.794286in}{2.017169in}}{\pgfqpoint{1.786050in}{2.017169in}}%
\pgfpathcurveto{\pgfqpoint{1.777814in}{2.017169in}}{\pgfqpoint{1.769914in}{2.013896in}}{\pgfqpoint{1.764090in}{2.008073in}}%
\pgfpathcurveto{\pgfqpoint{1.758266in}{2.002249in}}{\pgfqpoint{1.754993in}{1.994349in}}{\pgfqpoint{1.754993in}{1.986112in}}%
\pgfpathcurveto{\pgfqpoint{1.754993in}{1.977876in}}{\pgfqpoint{1.758266in}{1.969976in}}{\pgfqpoint{1.764090in}{1.964152in}}%
\pgfpathcurveto{\pgfqpoint{1.769914in}{1.958328in}}{\pgfqpoint{1.777814in}{1.955056in}}{\pgfqpoint{1.786050in}{1.955056in}}%
\pgfpathclose%
\pgfusepath{stroke,fill}%
\end{pgfscope}%
\begin{pgfscope}%
\pgfpathrectangle{\pgfqpoint{0.100000in}{0.212622in}}{\pgfqpoint{3.696000in}{3.696000in}}%
\pgfusepath{clip}%
\pgfsetbuttcap%
\pgfsetroundjoin%
\definecolor{currentfill}{rgb}{0.121569,0.466667,0.705882}%
\pgfsetfillcolor{currentfill}%
\pgfsetfillopacity{0.341766}%
\pgfsetlinewidth{1.003750pt}%
\definecolor{currentstroke}{rgb}{0.121569,0.466667,0.705882}%
\pgfsetstrokecolor{currentstroke}%
\pgfsetstrokeopacity{0.341766}%
\pgfsetdash{}{0pt}%
\pgfpathmoveto{\pgfqpoint{1.783735in}{1.953229in}}%
\pgfpathcurveto{\pgfqpoint{1.791971in}{1.953229in}}{\pgfqpoint{1.799871in}{1.956502in}}{\pgfqpoint{1.805695in}{1.962326in}}%
\pgfpathcurveto{\pgfqpoint{1.811519in}{1.968150in}}{\pgfqpoint{1.814791in}{1.976050in}}{\pgfqpoint{1.814791in}{1.984286in}}%
\pgfpathcurveto{\pgfqpoint{1.814791in}{1.992522in}}{\pgfqpoint{1.811519in}{2.000422in}}{\pgfqpoint{1.805695in}{2.006246in}}%
\pgfpathcurveto{\pgfqpoint{1.799871in}{2.012070in}}{\pgfqpoint{1.791971in}{2.015342in}}{\pgfqpoint{1.783735in}{2.015342in}}%
\pgfpathcurveto{\pgfqpoint{1.775499in}{2.015342in}}{\pgfqpoint{1.767599in}{2.012070in}}{\pgfqpoint{1.761775in}{2.006246in}}%
\pgfpathcurveto{\pgfqpoint{1.755951in}{2.000422in}}{\pgfqpoint{1.752678in}{1.992522in}}{\pgfqpoint{1.752678in}{1.984286in}}%
\pgfpathcurveto{\pgfqpoint{1.752678in}{1.976050in}}{\pgfqpoint{1.755951in}{1.968150in}}{\pgfqpoint{1.761775in}{1.962326in}}%
\pgfpathcurveto{\pgfqpoint{1.767599in}{1.956502in}}{\pgfqpoint{1.775499in}{1.953229in}}{\pgfqpoint{1.783735in}{1.953229in}}%
\pgfpathclose%
\pgfusepath{stroke,fill}%
\end{pgfscope}%
\begin{pgfscope}%
\pgfpathrectangle{\pgfqpoint{0.100000in}{0.212622in}}{\pgfqpoint{3.696000in}{3.696000in}}%
\pgfusepath{clip}%
\pgfsetbuttcap%
\pgfsetroundjoin%
\definecolor{currentfill}{rgb}{0.121569,0.466667,0.705882}%
\pgfsetfillcolor{currentfill}%
\pgfsetfillopacity{0.342498}%
\pgfsetlinewidth{1.003750pt}%
\definecolor{currentstroke}{rgb}{0.121569,0.466667,0.705882}%
\pgfsetstrokecolor{currentstroke}%
\pgfsetstrokeopacity{0.342498}%
\pgfsetdash{}{0pt}%
\pgfpathmoveto{\pgfqpoint{1.780506in}{1.950722in}}%
\pgfpathcurveto{\pgfqpoint{1.788742in}{1.950722in}}{\pgfqpoint{1.796642in}{1.953994in}}{\pgfqpoint{1.802466in}{1.959818in}}%
\pgfpathcurveto{\pgfqpoint{1.808290in}{1.965642in}}{\pgfqpoint{1.811562in}{1.973542in}}{\pgfqpoint{1.811562in}{1.981778in}}%
\pgfpathcurveto{\pgfqpoint{1.811562in}{1.990014in}}{\pgfqpoint{1.808290in}{1.997914in}}{\pgfqpoint{1.802466in}{2.003738in}}%
\pgfpathcurveto{\pgfqpoint{1.796642in}{2.009562in}}{\pgfqpoint{1.788742in}{2.012835in}}{\pgfqpoint{1.780506in}{2.012835in}}%
\pgfpathcurveto{\pgfqpoint{1.772269in}{2.012835in}}{\pgfqpoint{1.764369in}{2.009562in}}{\pgfqpoint{1.758545in}{2.003738in}}%
\pgfpathcurveto{\pgfqpoint{1.752722in}{1.997914in}}{\pgfqpoint{1.749449in}{1.990014in}}{\pgfqpoint{1.749449in}{1.981778in}}%
\pgfpathcurveto{\pgfqpoint{1.749449in}{1.973542in}}{\pgfqpoint{1.752722in}{1.965642in}}{\pgfqpoint{1.758545in}{1.959818in}}%
\pgfpathcurveto{\pgfqpoint{1.764369in}{1.953994in}}{\pgfqpoint{1.772269in}{1.950722in}}{\pgfqpoint{1.780506in}{1.950722in}}%
\pgfpathclose%
\pgfusepath{stroke,fill}%
\end{pgfscope}%
\begin{pgfscope}%
\pgfpathrectangle{\pgfqpoint{0.100000in}{0.212622in}}{\pgfqpoint{3.696000in}{3.696000in}}%
\pgfusepath{clip}%
\pgfsetbuttcap%
\pgfsetroundjoin%
\definecolor{currentfill}{rgb}{0.121569,0.466667,0.705882}%
\pgfsetfillcolor{currentfill}%
\pgfsetfillopacity{0.343034}%
\pgfsetlinewidth{1.003750pt}%
\definecolor{currentstroke}{rgb}{0.121569,0.466667,0.705882}%
\pgfsetstrokecolor{currentstroke}%
\pgfsetstrokeopacity{0.343034}%
\pgfsetdash{}{0pt}%
\pgfpathmoveto{\pgfqpoint{1.537191in}{2.101379in}}%
\pgfpathcurveto{\pgfqpoint{1.545427in}{2.101379in}}{\pgfqpoint{1.553327in}{2.104651in}}{\pgfqpoint{1.559151in}{2.110475in}}%
\pgfpathcurveto{\pgfqpoint{1.564975in}{2.116299in}}{\pgfqpoint{1.568247in}{2.124199in}}{\pgfqpoint{1.568247in}{2.132435in}}%
\pgfpathcurveto{\pgfqpoint{1.568247in}{2.140671in}}{\pgfqpoint{1.564975in}{2.148571in}}{\pgfqpoint{1.559151in}{2.154395in}}%
\pgfpathcurveto{\pgfqpoint{1.553327in}{2.160219in}}{\pgfqpoint{1.545427in}{2.163492in}}{\pgfqpoint{1.537191in}{2.163492in}}%
\pgfpathcurveto{\pgfqpoint{1.528954in}{2.163492in}}{\pgfqpoint{1.521054in}{2.160219in}}{\pgfqpoint{1.515230in}{2.154395in}}%
\pgfpathcurveto{\pgfqpoint{1.509406in}{2.148571in}}{\pgfqpoint{1.506134in}{2.140671in}}{\pgfqpoint{1.506134in}{2.132435in}}%
\pgfpathcurveto{\pgfqpoint{1.506134in}{2.124199in}}{\pgfqpoint{1.509406in}{2.116299in}}{\pgfqpoint{1.515230in}{2.110475in}}%
\pgfpathcurveto{\pgfqpoint{1.521054in}{2.104651in}}{\pgfqpoint{1.528954in}{2.101379in}}{\pgfqpoint{1.537191in}{2.101379in}}%
\pgfpathclose%
\pgfusepath{stroke,fill}%
\end{pgfscope}%
\begin{pgfscope}%
\pgfpathrectangle{\pgfqpoint{0.100000in}{0.212622in}}{\pgfqpoint{3.696000in}{3.696000in}}%
\pgfusepath{clip}%
\pgfsetbuttcap%
\pgfsetroundjoin%
\definecolor{currentfill}{rgb}{0.121569,0.466667,0.705882}%
\pgfsetfillcolor{currentfill}%
\pgfsetfillopacity{0.343322}%
\pgfsetlinewidth{1.003750pt}%
\definecolor{currentstroke}{rgb}{0.121569,0.466667,0.705882}%
\pgfsetstrokecolor{currentstroke}%
\pgfsetstrokeopacity{0.343322}%
\pgfsetdash{}{0pt}%
\pgfpathmoveto{\pgfqpoint{1.772699in}{1.945577in}}%
\pgfpathcurveto{\pgfqpoint{1.780935in}{1.945577in}}{\pgfqpoint{1.788835in}{1.948849in}}{\pgfqpoint{1.794659in}{1.954673in}}%
\pgfpathcurveto{\pgfqpoint{1.800483in}{1.960497in}}{\pgfqpoint{1.803755in}{1.968397in}}{\pgfqpoint{1.803755in}{1.976633in}}%
\pgfpathcurveto{\pgfqpoint{1.803755in}{1.984870in}}{\pgfqpoint{1.800483in}{1.992770in}}{\pgfqpoint{1.794659in}{1.998594in}}%
\pgfpathcurveto{\pgfqpoint{1.788835in}{2.004418in}}{\pgfqpoint{1.780935in}{2.007690in}}{\pgfqpoint{1.772699in}{2.007690in}}%
\pgfpathcurveto{\pgfqpoint{1.764462in}{2.007690in}}{\pgfqpoint{1.756562in}{2.004418in}}{\pgfqpoint{1.750738in}{1.998594in}}%
\pgfpathcurveto{\pgfqpoint{1.744914in}{1.992770in}}{\pgfqpoint{1.741642in}{1.984870in}}{\pgfqpoint{1.741642in}{1.976633in}}%
\pgfpathcurveto{\pgfqpoint{1.741642in}{1.968397in}}{\pgfqpoint{1.744914in}{1.960497in}}{\pgfqpoint{1.750738in}{1.954673in}}%
\pgfpathcurveto{\pgfqpoint{1.756562in}{1.948849in}}{\pgfqpoint{1.764462in}{1.945577in}}{\pgfqpoint{1.772699in}{1.945577in}}%
\pgfpathclose%
\pgfusepath{stroke,fill}%
\end{pgfscope}%
\begin{pgfscope}%
\pgfpathrectangle{\pgfqpoint{0.100000in}{0.212622in}}{\pgfqpoint{3.696000in}{3.696000in}}%
\pgfusepath{clip}%
\pgfsetbuttcap%
\pgfsetroundjoin%
\definecolor{currentfill}{rgb}{0.121569,0.466667,0.705882}%
\pgfsetfillcolor{currentfill}%
\pgfsetfillopacity{0.344221}%
\pgfsetlinewidth{1.003750pt}%
\definecolor{currentstroke}{rgb}{0.121569,0.466667,0.705882}%
\pgfsetstrokecolor{currentstroke}%
\pgfsetstrokeopacity{0.344221}%
\pgfsetdash{}{0pt}%
\pgfpathmoveto{\pgfqpoint{1.768028in}{1.940647in}}%
\pgfpathcurveto{\pgfqpoint{1.776264in}{1.940647in}}{\pgfqpoint{1.784164in}{1.943920in}}{\pgfqpoint{1.789988in}{1.949743in}}%
\pgfpathcurveto{\pgfqpoint{1.795812in}{1.955567in}}{\pgfqpoint{1.799085in}{1.963467in}}{\pgfqpoint{1.799085in}{1.971704in}}%
\pgfpathcurveto{\pgfqpoint{1.799085in}{1.979940in}}{\pgfqpoint{1.795812in}{1.987840in}}{\pgfqpoint{1.789988in}{1.993664in}}%
\pgfpathcurveto{\pgfqpoint{1.784164in}{1.999488in}}{\pgfqpoint{1.776264in}{2.002760in}}{\pgfqpoint{1.768028in}{2.002760in}}%
\pgfpathcurveto{\pgfqpoint{1.759792in}{2.002760in}}{\pgfqpoint{1.751892in}{1.999488in}}{\pgfqpoint{1.746068in}{1.993664in}}%
\pgfpathcurveto{\pgfqpoint{1.740244in}{1.987840in}}{\pgfqpoint{1.736972in}{1.979940in}}{\pgfqpoint{1.736972in}{1.971704in}}%
\pgfpathcurveto{\pgfqpoint{1.736972in}{1.963467in}}{\pgfqpoint{1.740244in}{1.955567in}}{\pgfqpoint{1.746068in}{1.949743in}}%
\pgfpathcurveto{\pgfqpoint{1.751892in}{1.943920in}}{\pgfqpoint{1.759792in}{1.940647in}}{\pgfqpoint{1.768028in}{1.940647in}}%
\pgfpathclose%
\pgfusepath{stroke,fill}%
\end{pgfscope}%
\begin{pgfscope}%
\pgfpathrectangle{\pgfqpoint{0.100000in}{0.212622in}}{\pgfqpoint{3.696000in}{3.696000in}}%
\pgfusepath{clip}%
\pgfsetbuttcap%
\pgfsetroundjoin%
\definecolor{currentfill}{rgb}{0.121569,0.466667,0.705882}%
\pgfsetfillcolor{currentfill}%
\pgfsetfillopacity{0.345038}%
\pgfsetlinewidth{1.003750pt}%
\definecolor{currentstroke}{rgb}{0.121569,0.466667,0.705882}%
\pgfsetstrokecolor{currentstroke}%
\pgfsetstrokeopacity{0.345038}%
\pgfsetdash{}{0pt}%
\pgfpathmoveto{\pgfqpoint{1.764944in}{1.938500in}}%
\pgfpathcurveto{\pgfqpoint{1.773181in}{1.938500in}}{\pgfqpoint{1.781081in}{1.941772in}}{\pgfqpoint{1.786905in}{1.947596in}}%
\pgfpathcurveto{\pgfqpoint{1.792729in}{1.953420in}}{\pgfqpoint{1.796001in}{1.961320in}}{\pgfqpoint{1.796001in}{1.969557in}}%
\pgfpathcurveto{\pgfqpoint{1.796001in}{1.977793in}}{\pgfqpoint{1.792729in}{1.985693in}}{\pgfqpoint{1.786905in}{1.991517in}}%
\pgfpathcurveto{\pgfqpoint{1.781081in}{1.997341in}}{\pgfqpoint{1.773181in}{2.000613in}}{\pgfqpoint{1.764944in}{2.000613in}}%
\pgfpathcurveto{\pgfqpoint{1.756708in}{2.000613in}}{\pgfqpoint{1.748808in}{1.997341in}}{\pgfqpoint{1.742984in}{1.991517in}}%
\pgfpathcurveto{\pgfqpoint{1.737160in}{1.985693in}}{\pgfqpoint{1.733888in}{1.977793in}}{\pgfqpoint{1.733888in}{1.969557in}}%
\pgfpathcurveto{\pgfqpoint{1.733888in}{1.961320in}}{\pgfqpoint{1.737160in}{1.953420in}}{\pgfqpoint{1.742984in}{1.947596in}}%
\pgfpathcurveto{\pgfqpoint{1.748808in}{1.941772in}}{\pgfqpoint{1.756708in}{1.938500in}}{\pgfqpoint{1.764944in}{1.938500in}}%
\pgfpathclose%
\pgfusepath{stroke,fill}%
\end{pgfscope}%
\begin{pgfscope}%
\pgfpathrectangle{\pgfqpoint{0.100000in}{0.212622in}}{\pgfqpoint{3.696000in}{3.696000in}}%
\pgfusepath{clip}%
\pgfsetbuttcap%
\pgfsetroundjoin%
\definecolor{currentfill}{rgb}{0.121569,0.466667,0.705882}%
\pgfsetfillcolor{currentfill}%
\pgfsetfillopacity{0.345215}%
\pgfsetlinewidth{1.003750pt}%
\definecolor{currentstroke}{rgb}{0.121569,0.466667,0.705882}%
\pgfsetstrokecolor{currentstroke}%
\pgfsetstrokeopacity{0.345215}%
\pgfsetdash{}{0pt}%
\pgfpathmoveto{\pgfqpoint{1.763555in}{1.937914in}}%
\pgfpathcurveto{\pgfqpoint{1.771792in}{1.937914in}}{\pgfqpoint{1.779692in}{1.941187in}}{\pgfqpoint{1.785516in}{1.947011in}}%
\pgfpathcurveto{\pgfqpoint{1.791339in}{1.952834in}}{\pgfqpoint{1.794612in}{1.960735in}}{\pgfqpoint{1.794612in}{1.968971in}}%
\pgfpathcurveto{\pgfqpoint{1.794612in}{1.977207in}}{\pgfqpoint{1.791339in}{1.985107in}}{\pgfqpoint{1.785516in}{1.990931in}}%
\pgfpathcurveto{\pgfqpoint{1.779692in}{1.996755in}}{\pgfqpoint{1.771792in}{2.000027in}}{\pgfqpoint{1.763555in}{2.000027in}}%
\pgfpathcurveto{\pgfqpoint{1.755319in}{2.000027in}}{\pgfqpoint{1.747419in}{1.996755in}}{\pgfqpoint{1.741595in}{1.990931in}}%
\pgfpathcurveto{\pgfqpoint{1.735771in}{1.985107in}}{\pgfqpoint{1.732499in}{1.977207in}}{\pgfqpoint{1.732499in}{1.968971in}}%
\pgfpathcurveto{\pgfqpoint{1.732499in}{1.960735in}}{\pgfqpoint{1.735771in}{1.952834in}}{\pgfqpoint{1.741595in}{1.947011in}}%
\pgfpathcurveto{\pgfqpoint{1.747419in}{1.941187in}}{\pgfqpoint{1.755319in}{1.937914in}}{\pgfqpoint{1.763555in}{1.937914in}}%
\pgfpathclose%
\pgfusepath{stroke,fill}%
\end{pgfscope}%
\begin{pgfscope}%
\pgfpathrectangle{\pgfqpoint{0.100000in}{0.212622in}}{\pgfqpoint{3.696000in}{3.696000in}}%
\pgfusepath{clip}%
\pgfsetbuttcap%
\pgfsetroundjoin%
\definecolor{currentfill}{rgb}{0.121569,0.466667,0.705882}%
\pgfsetfillcolor{currentfill}%
\pgfsetfillopacity{0.345726}%
\pgfsetlinewidth{1.003750pt}%
\definecolor{currentstroke}{rgb}{0.121569,0.466667,0.705882}%
\pgfsetstrokecolor{currentstroke}%
\pgfsetstrokeopacity{0.345726}%
\pgfsetdash{}{0pt}%
\pgfpathmoveto{\pgfqpoint{1.762293in}{1.936423in}}%
\pgfpathcurveto{\pgfqpoint{1.770529in}{1.936423in}}{\pgfqpoint{1.778429in}{1.939696in}}{\pgfqpoint{1.784253in}{1.945520in}}%
\pgfpathcurveto{\pgfqpoint{1.790077in}{1.951344in}}{\pgfqpoint{1.793349in}{1.959244in}}{\pgfqpoint{1.793349in}{1.967480in}}%
\pgfpathcurveto{\pgfqpoint{1.793349in}{1.975716in}}{\pgfqpoint{1.790077in}{1.983616in}}{\pgfqpoint{1.784253in}{1.989440in}}%
\pgfpathcurveto{\pgfqpoint{1.778429in}{1.995264in}}{\pgfqpoint{1.770529in}{1.998536in}}{\pgfqpoint{1.762293in}{1.998536in}}%
\pgfpathcurveto{\pgfqpoint{1.754056in}{1.998536in}}{\pgfqpoint{1.746156in}{1.995264in}}{\pgfqpoint{1.740332in}{1.989440in}}%
\pgfpathcurveto{\pgfqpoint{1.734509in}{1.983616in}}{\pgfqpoint{1.731236in}{1.975716in}}{\pgfqpoint{1.731236in}{1.967480in}}%
\pgfpathcurveto{\pgfqpoint{1.731236in}{1.959244in}}{\pgfqpoint{1.734509in}{1.951344in}}{\pgfqpoint{1.740332in}{1.945520in}}%
\pgfpathcurveto{\pgfqpoint{1.746156in}{1.939696in}}{\pgfqpoint{1.754056in}{1.936423in}}{\pgfqpoint{1.762293in}{1.936423in}}%
\pgfpathclose%
\pgfusepath{stroke,fill}%
\end{pgfscope}%
\begin{pgfscope}%
\pgfpathrectangle{\pgfqpoint{0.100000in}{0.212622in}}{\pgfqpoint{3.696000in}{3.696000in}}%
\pgfusepath{clip}%
\pgfsetbuttcap%
\pgfsetroundjoin%
\definecolor{currentfill}{rgb}{0.121569,0.466667,0.705882}%
\pgfsetfillcolor{currentfill}%
\pgfsetfillopacity{0.346365}%
\pgfsetlinewidth{1.003750pt}%
\definecolor{currentstroke}{rgb}{0.121569,0.466667,0.705882}%
\pgfsetstrokecolor{currentstroke}%
\pgfsetstrokeopacity{0.346365}%
\pgfsetdash{}{0pt}%
\pgfpathmoveto{\pgfqpoint{1.757665in}{1.934937in}}%
\pgfpathcurveto{\pgfqpoint{1.765902in}{1.934937in}}{\pgfqpoint{1.773802in}{1.938210in}}{\pgfqpoint{1.779626in}{1.944034in}}%
\pgfpathcurveto{\pgfqpoint{1.785450in}{1.949858in}}{\pgfqpoint{1.788722in}{1.957758in}}{\pgfqpoint{1.788722in}{1.965994in}}%
\pgfpathcurveto{\pgfqpoint{1.788722in}{1.974230in}}{\pgfqpoint{1.785450in}{1.982130in}}{\pgfqpoint{1.779626in}{1.987954in}}%
\pgfpathcurveto{\pgfqpoint{1.773802in}{1.993778in}}{\pgfqpoint{1.765902in}{1.997050in}}{\pgfqpoint{1.757665in}{1.997050in}}%
\pgfpathcurveto{\pgfqpoint{1.749429in}{1.997050in}}{\pgfqpoint{1.741529in}{1.993778in}}{\pgfqpoint{1.735705in}{1.987954in}}%
\pgfpathcurveto{\pgfqpoint{1.729881in}{1.982130in}}{\pgfqpoint{1.726609in}{1.974230in}}{\pgfqpoint{1.726609in}{1.965994in}}%
\pgfpathcurveto{\pgfqpoint{1.726609in}{1.957758in}}{\pgfqpoint{1.729881in}{1.949858in}}{\pgfqpoint{1.735705in}{1.944034in}}%
\pgfpathcurveto{\pgfqpoint{1.741529in}{1.938210in}}{\pgfqpoint{1.749429in}{1.934937in}}{\pgfqpoint{1.757665in}{1.934937in}}%
\pgfpathclose%
\pgfusepath{stroke,fill}%
\end{pgfscope}%
\begin{pgfscope}%
\pgfpathrectangle{\pgfqpoint{0.100000in}{0.212622in}}{\pgfqpoint{3.696000in}{3.696000in}}%
\pgfusepath{clip}%
\pgfsetbuttcap%
\pgfsetroundjoin%
\definecolor{currentfill}{rgb}{0.121569,0.466667,0.705882}%
\pgfsetfillcolor{currentfill}%
\pgfsetfillopacity{0.347138}%
\pgfsetlinewidth{1.003750pt}%
\definecolor{currentstroke}{rgb}{0.121569,0.466667,0.705882}%
\pgfsetstrokecolor{currentstroke}%
\pgfsetstrokeopacity{0.347138}%
\pgfsetdash{}{0pt}%
\pgfpathmoveto{\pgfqpoint{1.757079in}{1.934452in}}%
\pgfpathcurveto{\pgfqpoint{1.765315in}{1.934452in}}{\pgfqpoint{1.773216in}{1.937724in}}{\pgfqpoint{1.779039in}{1.943548in}}%
\pgfpathcurveto{\pgfqpoint{1.784863in}{1.949372in}}{\pgfqpoint{1.788136in}{1.957272in}}{\pgfqpoint{1.788136in}{1.965508in}}%
\pgfpathcurveto{\pgfqpoint{1.788136in}{1.973745in}}{\pgfqpoint{1.784863in}{1.981645in}}{\pgfqpoint{1.779039in}{1.987469in}}%
\pgfpathcurveto{\pgfqpoint{1.773216in}{1.993292in}}{\pgfqpoint{1.765315in}{1.996565in}}{\pgfqpoint{1.757079in}{1.996565in}}%
\pgfpathcurveto{\pgfqpoint{1.748843in}{1.996565in}}{\pgfqpoint{1.740943in}{1.993292in}}{\pgfqpoint{1.735119in}{1.987469in}}%
\pgfpathcurveto{\pgfqpoint{1.729295in}{1.981645in}}{\pgfqpoint{1.726023in}{1.973745in}}{\pgfqpoint{1.726023in}{1.965508in}}%
\pgfpathcurveto{\pgfqpoint{1.726023in}{1.957272in}}{\pgfqpoint{1.729295in}{1.949372in}}{\pgfqpoint{1.735119in}{1.943548in}}%
\pgfpathcurveto{\pgfqpoint{1.740943in}{1.937724in}}{\pgfqpoint{1.748843in}{1.934452in}}{\pgfqpoint{1.757079in}{1.934452in}}%
\pgfpathclose%
\pgfusepath{stroke,fill}%
\end{pgfscope}%
\begin{pgfscope}%
\pgfpathrectangle{\pgfqpoint{0.100000in}{0.212622in}}{\pgfqpoint{3.696000in}{3.696000in}}%
\pgfusepath{clip}%
\pgfsetbuttcap%
\pgfsetroundjoin%
\definecolor{currentfill}{rgb}{0.121569,0.466667,0.705882}%
\pgfsetfillcolor{currentfill}%
\pgfsetfillopacity{0.347917}%
\pgfsetlinewidth{1.003750pt}%
\definecolor{currentstroke}{rgb}{0.121569,0.466667,0.705882}%
\pgfsetstrokecolor{currentstroke}%
\pgfsetstrokeopacity{0.347917}%
\pgfsetdash{}{0pt}%
\pgfpathmoveto{\pgfqpoint{1.751900in}{1.934445in}}%
\pgfpathcurveto{\pgfqpoint{1.760136in}{1.934445in}}{\pgfqpoint{1.768036in}{1.937717in}}{\pgfqpoint{1.773860in}{1.943541in}}%
\pgfpathcurveto{\pgfqpoint{1.779684in}{1.949365in}}{\pgfqpoint{1.782956in}{1.957265in}}{\pgfqpoint{1.782956in}{1.965501in}}%
\pgfpathcurveto{\pgfqpoint{1.782956in}{1.973738in}}{\pgfqpoint{1.779684in}{1.981638in}}{\pgfqpoint{1.773860in}{1.987462in}}%
\pgfpathcurveto{\pgfqpoint{1.768036in}{1.993286in}}{\pgfqpoint{1.760136in}{1.996558in}}{\pgfqpoint{1.751900in}{1.996558in}}%
\pgfpathcurveto{\pgfqpoint{1.743664in}{1.996558in}}{\pgfqpoint{1.735764in}{1.993286in}}{\pgfqpoint{1.729940in}{1.987462in}}%
\pgfpathcurveto{\pgfqpoint{1.724116in}{1.981638in}}{\pgfqpoint{1.720843in}{1.973738in}}{\pgfqpoint{1.720843in}{1.965501in}}%
\pgfpathcurveto{\pgfqpoint{1.720843in}{1.957265in}}{\pgfqpoint{1.724116in}{1.949365in}}{\pgfqpoint{1.729940in}{1.943541in}}%
\pgfpathcurveto{\pgfqpoint{1.735764in}{1.937717in}}{\pgfqpoint{1.743664in}{1.934445in}}{\pgfqpoint{1.751900in}{1.934445in}}%
\pgfpathclose%
\pgfusepath{stroke,fill}%
\end{pgfscope}%
\begin{pgfscope}%
\pgfpathrectangle{\pgfqpoint{0.100000in}{0.212622in}}{\pgfqpoint{3.696000in}{3.696000in}}%
\pgfusepath{clip}%
\pgfsetbuttcap%
\pgfsetroundjoin%
\definecolor{currentfill}{rgb}{0.121569,0.466667,0.705882}%
\pgfsetfillcolor{currentfill}%
\pgfsetfillopacity{0.349897}%
\pgfsetlinewidth{1.003750pt}%
\definecolor{currentstroke}{rgb}{0.121569,0.466667,0.705882}%
\pgfsetstrokecolor{currentstroke}%
\pgfsetstrokeopacity{0.349897}%
\pgfsetdash{}{0pt}%
\pgfpathmoveto{\pgfqpoint{1.748519in}{1.929928in}}%
\pgfpathcurveto{\pgfqpoint{1.756756in}{1.929928in}}{\pgfqpoint{1.764656in}{1.933201in}}{\pgfqpoint{1.770480in}{1.939025in}}%
\pgfpathcurveto{\pgfqpoint{1.776304in}{1.944849in}}{\pgfqpoint{1.779576in}{1.952749in}}{\pgfqpoint{1.779576in}{1.960985in}}%
\pgfpathcurveto{\pgfqpoint{1.779576in}{1.969221in}}{\pgfqpoint{1.776304in}{1.977121in}}{\pgfqpoint{1.770480in}{1.982945in}}%
\pgfpathcurveto{\pgfqpoint{1.764656in}{1.988769in}}{\pgfqpoint{1.756756in}{1.992041in}}{\pgfqpoint{1.748519in}{1.992041in}}%
\pgfpathcurveto{\pgfqpoint{1.740283in}{1.992041in}}{\pgfqpoint{1.732383in}{1.988769in}}{\pgfqpoint{1.726559in}{1.982945in}}%
\pgfpathcurveto{\pgfqpoint{1.720735in}{1.977121in}}{\pgfqpoint{1.717463in}{1.969221in}}{\pgfqpoint{1.717463in}{1.960985in}}%
\pgfpathcurveto{\pgfqpoint{1.717463in}{1.952749in}}{\pgfqpoint{1.720735in}{1.944849in}}{\pgfqpoint{1.726559in}{1.939025in}}%
\pgfpathcurveto{\pgfqpoint{1.732383in}{1.933201in}}{\pgfqpoint{1.740283in}{1.929928in}}{\pgfqpoint{1.748519in}{1.929928in}}%
\pgfpathclose%
\pgfusepath{stroke,fill}%
\end{pgfscope}%
\begin{pgfscope}%
\pgfpathrectangle{\pgfqpoint{0.100000in}{0.212622in}}{\pgfqpoint{3.696000in}{3.696000in}}%
\pgfusepath{clip}%
\pgfsetbuttcap%
\pgfsetroundjoin%
\definecolor{currentfill}{rgb}{0.121569,0.466667,0.705882}%
\pgfsetfillcolor{currentfill}%
\pgfsetfillopacity{0.349913}%
\pgfsetlinewidth{1.003750pt}%
\definecolor{currentstroke}{rgb}{0.121569,0.466667,0.705882}%
\pgfsetstrokecolor{currentstroke}%
\pgfsetstrokeopacity{0.349913}%
\pgfsetdash{}{0pt}%
\pgfpathmoveto{\pgfqpoint{1.748438in}{1.929935in}}%
\pgfpathcurveto{\pgfqpoint{1.756675in}{1.929935in}}{\pgfqpoint{1.764575in}{1.933208in}}{\pgfqpoint{1.770399in}{1.939032in}}%
\pgfpathcurveto{\pgfqpoint{1.776222in}{1.944856in}}{\pgfqpoint{1.779495in}{1.952756in}}{\pgfqpoint{1.779495in}{1.960992in}}%
\pgfpathcurveto{\pgfqpoint{1.779495in}{1.969228in}}{\pgfqpoint{1.776222in}{1.977128in}}{\pgfqpoint{1.770399in}{1.982952in}}%
\pgfpathcurveto{\pgfqpoint{1.764575in}{1.988776in}}{\pgfqpoint{1.756675in}{1.992048in}}{\pgfqpoint{1.748438in}{1.992048in}}%
\pgfpathcurveto{\pgfqpoint{1.740202in}{1.992048in}}{\pgfqpoint{1.732302in}{1.988776in}}{\pgfqpoint{1.726478in}{1.982952in}}%
\pgfpathcurveto{\pgfqpoint{1.720654in}{1.977128in}}{\pgfqpoint{1.717382in}{1.969228in}}{\pgfqpoint{1.717382in}{1.960992in}}%
\pgfpathcurveto{\pgfqpoint{1.717382in}{1.952756in}}{\pgfqpoint{1.720654in}{1.944856in}}{\pgfqpoint{1.726478in}{1.939032in}}%
\pgfpathcurveto{\pgfqpoint{1.732302in}{1.933208in}}{\pgfqpoint{1.740202in}{1.929935in}}{\pgfqpoint{1.748438in}{1.929935in}}%
\pgfpathclose%
\pgfusepath{stroke,fill}%
\end{pgfscope}%
\begin{pgfscope}%
\pgfpathrectangle{\pgfqpoint{0.100000in}{0.212622in}}{\pgfqpoint{3.696000in}{3.696000in}}%
\pgfusepath{clip}%
\pgfsetbuttcap%
\pgfsetroundjoin%
\definecolor{currentfill}{rgb}{0.121569,0.466667,0.705882}%
\pgfsetfillcolor{currentfill}%
\pgfsetfillopacity{0.349940}%
\pgfsetlinewidth{1.003750pt}%
\definecolor{currentstroke}{rgb}{0.121569,0.466667,0.705882}%
\pgfsetstrokecolor{currentstroke}%
\pgfsetstrokeopacity{0.349940}%
\pgfsetdash{}{0pt}%
\pgfpathmoveto{\pgfqpoint{1.748370in}{1.929831in}}%
\pgfpathcurveto{\pgfqpoint{1.756607in}{1.929831in}}{\pgfqpoint{1.764507in}{1.933104in}}{\pgfqpoint{1.770330in}{1.938927in}}%
\pgfpathcurveto{\pgfqpoint{1.776154in}{1.944751in}}{\pgfqpoint{1.779427in}{1.952651in}}{\pgfqpoint{1.779427in}{1.960888in}}%
\pgfpathcurveto{\pgfqpoint{1.779427in}{1.969124in}}{\pgfqpoint{1.776154in}{1.977024in}}{\pgfqpoint{1.770330in}{1.982848in}}%
\pgfpathcurveto{\pgfqpoint{1.764507in}{1.988672in}}{\pgfqpoint{1.756607in}{1.991944in}}{\pgfqpoint{1.748370in}{1.991944in}}%
\pgfpathcurveto{\pgfqpoint{1.740134in}{1.991944in}}{\pgfqpoint{1.732234in}{1.988672in}}{\pgfqpoint{1.726410in}{1.982848in}}%
\pgfpathcurveto{\pgfqpoint{1.720586in}{1.977024in}}{\pgfqpoint{1.717314in}{1.969124in}}{\pgfqpoint{1.717314in}{1.960888in}}%
\pgfpathcurveto{\pgfqpoint{1.717314in}{1.952651in}}{\pgfqpoint{1.720586in}{1.944751in}}{\pgfqpoint{1.726410in}{1.938927in}}%
\pgfpathcurveto{\pgfqpoint{1.732234in}{1.933104in}}{\pgfqpoint{1.740134in}{1.929831in}}{\pgfqpoint{1.748370in}{1.929831in}}%
\pgfpathclose%
\pgfusepath{stroke,fill}%
\end{pgfscope}%
\begin{pgfscope}%
\pgfpathrectangle{\pgfqpoint{0.100000in}{0.212622in}}{\pgfqpoint{3.696000in}{3.696000in}}%
\pgfusepath{clip}%
\pgfsetbuttcap%
\pgfsetroundjoin%
\definecolor{currentfill}{rgb}{0.121569,0.466667,0.705882}%
\pgfsetfillcolor{currentfill}%
\pgfsetfillopacity{0.349980}%
\pgfsetlinewidth{1.003750pt}%
\definecolor{currentstroke}{rgb}{0.121569,0.466667,0.705882}%
\pgfsetstrokecolor{currentstroke}%
\pgfsetstrokeopacity{0.349980}%
\pgfsetdash{}{0pt}%
\pgfpathmoveto{\pgfqpoint{1.748120in}{1.929734in}}%
\pgfpathcurveto{\pgfqpoint{1.756356in}{1.929734in}}{\pgfqpoint{1.764256in}{1.933007in}}{\pgfqpoint{1.770080in}{1.938831in}}%
\pgfpathcurveto{\pgfqpoint{1.775904in}{1.944655in}}{\pgfqpoint{1.779176in}{1.952555in}}{\pgfqpoint{1.779176in}{1.960791in}}%
\pgfpathcurveto{\pgfqpoint{1.779176in}{1.969027in}}{\pgfqpoint{1.775904in}{1.976927in}}{\pgfqpoint{1.770080in}{1.982751in}}%
\pgfpathcurveto{\pgfqpoint{1.764256in}{1.988575in}}{\pgfqpoint{1.756356in}{1.991847in}}{\pgfqpoint{1.748120in}{1.991847in}}%
\pgfpathcurveto{\pgfqpoint{1.739883in}{1.991847in}}{\pgfqpoint{1.731983in}{1.988575in}}{\pgfqpoint{1.726160in}{1.982751in}}%
\pgfpathcurveto{\pgfqpoint{1.720336in}{1.976927in}}{\pgfqpoint{1.717063in}{1.969027in}}{\pgfqpoint{1.717063in}{1.960791in}}%
\pgfpathcurveto{\pgfqpoint{1.717063in}{1.952555in}}{\pgfqpoint{1.720336in}{1.944655in}}{\pgfqpoint{1.726160in}{1.938831in}}%
\pgfpathcurveto{\pgfqpoint{1.731983in}{1.933007in}}{\pgfqpoint{1.739883in}{1.929734in}}{\pgfqpoint{1.748120in}{1.929734in}}%
\pgfpathclose%
\pgfusepath{stroke,fill}%
\end{pgfscope}%
\begin{pgfscope}%
\pgfpathrectangle{\pgfqpoint{0.100000in}{0.212622in}}{\pgfqpoint{3.696000in}{3.696000in}}%
\pgfusepath{clip}%
\pgfsetbuttcap%
\pgfsetroundjoin%
\definecolor{currentfill}{rgb}{0.121569,0.466667,0.705882}%
\pgfsetfillcolor{currentfill}%
\pgfsetfillopacity{0.350047}%
\pgfsetlinewidth{1.003750pt}%
\definecolor{currentstroke}{rgb}{0.121569,0.466667,0.705882}%
\pgfsetstrokecolor{currentstroke}%
\pgfsetstrokeopacity{0.350047}%
\pgfsetdash{}{0pt}%
\pgfpathmoveto{\pgfqpoint{1.747802in}{1.929323in}}%
\pgfpathcurveto{\pgfqpoint{1.756038in}{1.929323in}}{\pgfqpoint{1.763938in}{1.932596in}}{\pgfqpoint{1.769762in}{1.938419in}}%
\pgfpathcurveto{\pgfqpoint{1.775586in}{1.944243in}}{\pgfqpoint{1.778858in}{1.952143in}}{\pgfqpoint{1.778858in}{1.960380in}}%
\pgfpathcurveto{\pgfqpoint{1.778858in}{1.968616in}}{\pgfqpoint{1.775586in}{1.976516in}}{\pgfqpoint{1.769762in}{1.982340in}}%
\pgfpathcurveto{\pgfqpoint{1.763938in}{1.988164in}}{\pgfqpoint{1.756038in}{1.991436in}}{\pgfqpoint{1.747802in}{1.991436in}}%
\pgfpathcurveto{\pgfqpoint{1.739565in}{1.991436in}}{\pgfqpoint{1.731665in}{1.988164in}}{\pgfqpoint{1.725841in}{1.982340in}}%
\pgfpathcurveto{\pgfqpoint{1.720017in}{1.976516in}}{\pgfqpoint{1.716745in}{1.968616in}}{\pgfqpoint{1.716745in}{1.960380in}}%
\pgfpathcurveto{\pgfqpoint{1.716745in}{1.952143in}}{\pgfqpoint{1.720017in}{1.944243in}}{\pgfqpoint{1.725841in}{1.938419in}}%
\pgfpathcurveto{\pgfqpoint{1.731665in}{1.932596in}}{\pgfqpoint{1.739565in}{1.929323in}}{\pgfqpoint{1.747802in}{1.929323in}}%
\pgfpathclose%
\pgfusepath{stroke,fill}%
\end{pgfscope}%
\begin{pgfscope}%
\pgfpathrectangle{\pgfqpoint{0.100000in}{0.212622in}}{\pgfqpoint{3.696000in}{3.696000in}}%
\pgfusepath{clip}%
\pgfsetbuttcap%
\pgfsetroundjoin%
\definecolor{currentfill}{rgb}{0.121569,0.466667,0.705882}%
\pgfsetfillcolor{currentfill}%
\pgfsetfillopacity{0.350199}%
\pgfsetlinewidth{1.003750pt}%
\definecolor{currentstroke}{rgb}{0.121569,0.466667,0.705882}%
\pgfsetstrokecolor{currentstroke}%
\pgfsetstrokeopacity{0.350199}%
\pgfsetdash{}{0pt}%
\pgfpathmoveto{\pgfqpoint{1.747123in}{1.928918in}}%
\pgfpathcurveto{\pgfqpoint{1.755359in}{1.928918in}}{\pgfqpoint{1.763259in}{1.932190in}}{\pgfqpoint{1.769083in}{1.938014in}}%
\pgfpathcurveto{\pgfqpoint{1.774907in}{1.943838in}}{\pgfqpoint{1.778179in}{1.951738in}}{\pgfqpoint{1.778179in}{1.959974in}}%
\pgfpathcurveto{\pgfqpoint{1.778179in}{1.968210in}}{\pgfqpoint{1.774907in}{1.976110in}}{\pgfqpoint{1.769083in}{1.981934in}}%
\pgfpathcurveto{\pgfqpoint{1.763259in}{1.987758in}}{\pgfqpoint{1.755359in}{1.991031in}}{\pgfqpoint{1.747123in}{1.991031in}}%
\pgfpathcurveto{\pgfqpoint{1.738886in}{1.991031in}}{\pgfqpoint{1.730986in}{1.987758in}}{\pgfqpoint{1.725162in}{1.981934in}}%
\pgfpathcurveto{\pgfqpoint{1.719339in}{1.976110in}}{\pgfqpoint{1.716066in}{1.968210in}}{\pgfqpoint{1.716066in}{1.959974in}}%
\pgfpathcurveto{\pgfqpoint{1.716066in}{1.951738in}}{\pgfqpoint{1.719339in}{1.943838in}}{\pgfqpoint{1.725162in}{1.938014in}}%
\pgfpathcurveto{\pgfqpoint{1.730986in}{1.932190in}}{\pgfqpoint{1.738886in}{1.928918in}}{\pgfqpoint{1.747123in}{1.928918in}}%
\pgfpathclose%
\pgfusepath{stroke,fill}%
\end{pgfscope}%
\begin{pgfscope}%
\pgfpathrectangle{\pgfqpoint{0.100000in}{0.212622in}}{\pgfqpoint{3.696000in}{3.696000in}}%
\pgfusepath{clip}%
\pgfsetbuttcap%
\pgfsetroundjoin%
\definecolor{currentfill}{rgb}{0.121569,0.466667,0.705882}%
\pgfsetfillcolor{currentfill}%
\pgfsetfillopacity{0.350375}%
\pgfsetlinewidth{1.003750pt}%
\definecolor{currentstroke}{rgb}{0.121569,0.466667,0.705882}%
\pgfsetstrokecolor{currentstroke}%
\pgfsetstrokeopacity{0.350375}%
\pgfsetdash{}{0pt}%
\pgfpathmoveto{\pgfqpoint{1.745730in}{1.927708in}}%
\pgfpathcurveto{\pgfqpoint{1.753967in}{1.927708in}}{\pgfqpoint{1.761867in}{1.930981in}}{\pgfqpoint{1.767691in}{1.936805in}}%
\pgfpathcurveto{\pgfqpoint{1.773515in}{1.942629in}}{\pgfqpoint{1.776787in}{1.950529in}}{\pgfqpoint{1.776787in}{1.958765in}}%
\pgfpathcurveto{\pgfqpoint{1.776787in}{1.967001in}}{\pgfqpoint{1.773515in}{1.974901in}}{\pgfqpoint{1.767691in}{1.980725in}}%
\pgfpathcurveto{\pgfqpoint{1.761867in}{1.986549in}}{\pgfqpoint{1.753967in}{1.989821in}}{\pgfqpoint{1.745730in}{1.989821in}}%
\pgfpathcurveto{\pgfqpoint{1.737494in}{1.989821in}}{\pgfqpoint{1.729594in}{1.986549in}}{\pgfqpoint{1.723770in}{1.980725in}}%
\pgfpathcurveto{\pgfqpoint{1.717946in}{1.974901in}}{\pgfqpoint{1.714674in}{1.967001in}}{\pgfqpoint{1.714674in}{1.958765in}}%
\pgfpathcurveto{\pgfqpoint{1.714674in}{1.950529in}}{\pgfqpoint{1.717946in}{1.942629in}}{\pgfqpoint{1.723770in}{1.936805in}}%
\pgfpathcurveto{\pgfqpoint{1.729594in}{1.930981in}}{\pgfqpoint{1.737494in}{1.927708in}}{\pgfqpoint{1.745730in}{1.927708in}}%
\pgfpathclose%
\pgfusepath{stroke,fill}%
\end{pgfscope}%
\begin{pgfscope}%
\pgfpathrectangle{\pgfqpoint{0.100000in}{0.212622in}}{\pgfqpoint{3.696000in}{3.696000in}}%
\pgfusepath{clip}%
\pgfsetbuttcap%
\pgfsetroundjoin%
\definecolor{currentfill}{rgb}{0.121569,0.466667,0.705882}%
\pgfsetfillcolor{currentfill}%
\pgfsetfillopacity{0.350895}%
\pgfsetlinewidth{1.003750pt}%
\definecolor{currentstroke}{rgb}{0.121569,0.466667,0.705882}%
\pgfsetstrokecolor{currentstroke}%
\pgfsetstrokeopacity{0.350895}%
\pgfsetdash{}{0pt}%
\pgfpathmoveto{\pgfqpoint{1.743979in}{1.925852in}}%
\pgfpathcurveto{\pgfqpoint{1.752216in}{1.925852in}}{\pgfqpoint{1.760116in}{1.929124in}}{\pgfqpoint{1.765940in}{1.934948in}}%
\pgfpathcurveto{\pgfqpoint{1.771764in}{1.940772in}}{\pgfqpoint{1.775036in}{1.948672in}}{\pgfqpoint{1.775036in}{1.956908in}}%
\pgfpathcurveto{\pgfqpoint{1.775036in}{1.965144in}}{\pgfqpoint{1.771764in}{1.973044in}}{\pgfqpoint{1.765940in}{1.978868in}}%
\pgfpathcurveto{\pgfqpoint{1.760116in}{1.984692in}}{\pgfqpoint{1.752216in}{1.987965in}}{\pgfqpoint{1.743979in}{1.987965in}}%
\pgfpathcurveto{\pgfqpoint{1.735743in}{1.987965in}}{\pgfqpoint{1.727843in}{1.984692in}}{\pgfqpoint{1.722019in}{1.978868in}}%
\pgfpathcurveto{\pgfqpoint{1.716195in}{1.973044in}}{\pgfqpoint{1.712923in}{1.965144in}}{\pgfqpoint{1.712923in}{1.956908in}}%
\pgfpathcurveto{\pgfqpoint{1.712923in}{1.948672in}}{\pgfqpoint{1.716195in}{1.940772in}}{\pgfqpoint{1.722019in}{1.934948in}}%
\pgfpathcurveto{\pgfqpoint{1.727843in}{1.929124in}}{\pgfqpoint{1.735743in}{1.925852in}}{\pgfqpoint{1.743979in}{1.925852in}}%
\pgfpathclose%
\pgfusepath{stroke,fill}%
\end{pgfscope}%
\begin{pgfscope}%
\pgfpathrectangle{\pgfqpoint{0.100000in}{0.212622in}}{\pgfqpoint{3.696000in}{3.696000in}}%
\pgfusepath{clip}%
\pgfsetbuttcap%
\pgfsetroundjoin%
\definecolor{currentfill}{rgb}{0.121569,0.466667,0.705882}%
\pgfsetfillcolor{currentfill}%
\pgfsetfillopacity{0.351634}%
\pgfsetlinewidth{1.003750pt}%
\definecolor{currentstroke}{rgb}{0.121569,0.466667,0.705882}%
\pgfsetstrokecolor{currentstroke}%
\pgfsetstrokeopacity{0.351634}%
\pgfsetdash{}{0pt}%
\pgfpathmoveto{\pgfqpoint{1.739427in}{1.922888in}}%
\pgfpathcurveto{\pgfqpoint{1.747663in}{1.922888in}}{\pgfqpoint{1.755563in}{1.926160in}}{\pgfqpoint{1.761387in}{1.931984in}}%
\pgfpathcurveto{\pgfqpoint{1.767211in}{1.937808in}}{\pgfqpoint{1.770484in}{1.945708in}}{\pgfqpoint{1.770484in}{1.953944in}}%
\pgfpathcurveto{\pgfqpoint{1.770484in}{1.962181in}}{\pgfqpoint{1.767211in}{1.970081in}}{\pgfqpoint{1.761387in}{1.975905in}}%
\pgfpathcurveto{\pgfqpoint{1.755563in}{1.981729in}}{\pgfqpoint{1.747663in}{1.985001in}}{\pgfqpoint{1.739427in}{1.985001in}}%
\pgfpathcurveto{\pgfqpoint{1.731191in}{1.985001in}}{\pgfqpoint{1.723291in}{1.981729in}}{\pgfqpoint{1.717467in}{1.975905in}}%
\pgfpathcurveto{\pgfqpoint{1.711643in}{1.970081in}}{\pgfqpoint{1.708371in}{1.962181in}}{\pgfqpoint{1.708371in}{1.953944in}}%
\pgfpathcurveto{\pgfqpoint{1.708371in}{1.945708in}}{\pgfqpoint{1.711643in}{1.937808in}}{\pgfqpoint{1.717467in}{1.931984in}}%
\pgfpathcurveto{\pgfqpoint{1.723291in}{1.926160in}}{\pgfqpoint{1.731191in}{1.922888in}}{\pgfqpoint{1.739427in}{1.922888in}}%
\pgfpathclose%
\pgfusepath{stroke,fill}%
\end{pgfscope}%
\begin{pgfscope}%
\pgfpathrectangle{\pgfqpoint{0.100000in}{0.212622in}}{\pgfqpoint{3.696000in}{3.696000in}}%
\pgfusepath{clip}%
\pgfsetbuttcap%
\pgfsetroundjoin%
\definecolor{currentfill}{rgb}{0.121569,0.466667,0.705882}%
\pgfsetfillcolor{currentfill}%
\pgfsetfillopacity{0.352180}%
\pgfsetlinewidth{1.003750pt}%
\definecolor{currentstroke}{rgb}{0.121569,0.466667,0.705882}%
\pgfsetstrokecolor{currentstroke}%
\pgfsetstrokeopacity{0.352180}%
\pgfsetdash{}{0pt}%
\pgfpathmoveto{\pgfqpoint{1.738793in}{1.921869in}}%
\pgfpathcurveto{\pgfqpoint{1.747030in}{1.921869in}}{\pgfqpoint{1.754930in}{1.925142in}}{\pgfqpoint{1.760754in}{1.930966in}}%
\pgfpathcurveto{\pgfqpoint{1.766578in}{1.936790in}}{\pgfqpoint{1.769850in}{1.944690in}}{\pgfqpoint{1.769850in}{1.952926in}}%
\pgfpathcurveto{\pgfqpoint{1.769850in}{1.961162in}}{\pgfqpoint{1.766578in}{1.969062in}}{\pgfqpoint{1.760754in}{1.974886in}}%
\pgfpathcurveto{\pgfqpoint{1.754930in}{1.980710in}}{\pgfqpoint{1.747030in}{1.983982in}}{\pgfqpoint{1.738793in}{1.983982in}}%
\pgfpathcurveto{\pgfqpoint{1.730557in}{1.983982in}}{\pgfqpoint{1.722657in}{1.980710in}}{\pgfqpoint{1.716833in}{1.974886in}}%
\pgfpathcurveto{\pgfqpoint{1.711009in}{1.969062in}}{\pgfqpoint{1.707737in}{1.961162in}}{\pgfqpoint{1.707737in}{1.952926in}}%
\pgfpathcurveto{\pgfqpoint{1.707737in}{1.944690in}}{\pgfqpoint{1.711009in}{1.936790in}}{\pgfqpoint{1.716833in}{1.930966in}}%
\pgfpathcurveto{\pgfqpoint{1.722657in}{1.925142in}}{\pgfqpoint{1.730557in}{1.921869in}}{\pgfqpoint{1.738793in}{1.921869in}}%
\pgfpathclose%
\pgfusepath{stroke,fill}%
\end{pgfscope}%
\begin{pgfscope}%
\pgfpathrectangle{\pgfqpoint{0.100000in}{0.212622in}}{\pgfqpoint{3.696000in}{3.696000in}}%
\pgfusepath{clip}%
\pgfsetbuttcap%
\pgfsetroundjoin%
\definecolor{currentfill}{rgb}{0.121569,0.466667,0.705882}%
\pgfsetfillcolor{currentfill}%
\pgfsetfillopacity{0.352661}%
\pgfsetlinewidth{1.003750pt}%
\definecolor{currentstroke}{rgb}{0.121569,0.466667,0.705882}%
\pgfsetstrokecolor{currentstroke}%
\pgfsetstrokeopacity{0.352661}%
\pgfsetdash{}{0pt}%
\pgfpathmoveto{\pgfqpoint{1.736239in}{1.917832in}}%
\pgfpathcurveto{\pgfqpoint{1.744475in}{1.917832in}}{\pgfqpoint{1.752376in}{1.921105in}}{\pgfqpoint{1.758199in}{1.926929in}}%
\pgfpathcurveto{\pgfqpoint{1.764023in}{1.932753in}}{\pgfqpoint{1.767296in}{1.940653in}}{\pgfqpoint{1.767296in}{1.948889in}}%
\pgfpathcurveto{\pgfqpoint{1.767296in}{1.957125in}}{\pgfqpoint{1.764023in}{1.965025in}}{\pgfqpoint{1.758199in}{1.970849in}}%
\pgfpathcurveto{\pgfqpoint{1.752376in}{1.976673in}}{\pgfqpoint{1.744475in}{1.979945in}}{\pgfqpoint{1.736239in}{1.979945in}}%
\pgfpathcurveto{\pgfqpoint{1.728003in}{1.979945in}}{\pgfqpoint{1.720103in}{1.976673in}}{\pgfqpoint{1.714279in}{1.970849in}}%
\pgfpathcurveto{\pgfqpoint{1.708455in}{1.965025in}}{\pgfqpoint{1.705183in}{1.957125in}}{\pgfqpoint{1.705183in}{1.948889in}}%
\pgfpathcurveto{\pgfqpoint{1.705183in}{1.940653in}}{\pgfqpoint{1.708455in}{1.932753in}}{\pgfqpoint{1.714279in}{1.926929in}}%
\pgfpathcurveto{\pgfqpoint{1.720103in}{1.921105in}}{\pgfqpoint{1.728003in}{1.917832in}}{\pgfqpoint{1.736239in}{1.917832in}}%
\pgfpathclose%
\pgfusepath{stroke,fill}%
\end{pgfscope}%
\begin{pgfscope}%
\pgfpathrectangle{\pgfqpoint{0.100000in}{0.212622in}}{\pgfqpoint{3.696000in}{3.696000in}}%
\pgfusepath{clip}%
\pgfsetbuttcap%
\pgfsetroundjoin%
\definecolor{currentfill}{rgb}{0.121569,0.466667,0.705882}%
\pgfsetfillcolor{currentfill}%
\pgfsetfillopacity{0.352853}%
\pgfsetlinewidth{1.003750pt}%
\definecolor{currentstroke}{rgb}{0.121569,0.466667,0.705882}%
\pgfsetstrokecolor{currentstroke}%
\pgfsetstrokeopacity{0.352853}%
\pgfsetdash{}{0pt}%
\pgfpathmoveto{\pgfqpoint{1.735051in}{1.916853in}}%
\pgfpathcurveto{\pgfqpoint{1.743287in}{1.916853in}}{\pgfqpoint{1.751187in}{1.920125in}}{\pgfqpoint{1.757011in}{1.925949in}}%
\pgfpathcurveto{\pgfqpoint{1.762835in}{1.931773in}}{\pgfqpoint{1.766108in}{1.939673in}}{\pgfqpoint{1.766108in}{1.947910in}}%
\pgfpathcurveto{\pgfqpoint{1.766108in}{1.956146in}}{\pgfqpoint{1.762835in}{1.964046in}}{\pgfqpoint{1.757011in}{1.969870in}}%
\pgfpathcurveto{\pgfqpoint{1.751187in}{1.975694in}}{\pgfqpoint{1.743287in}{1.978966in}}{\pgfqpoint{1.735051in}{1.978966in}}%
\pgfpathcurveto{\pgfqpoint{1.726815in}{1.978966in}}{\pgfqpoint{1.718915in}{1.975694in}}{\pgfqpoint{1.713091in}{1.969870in}}%
\pgfpathcurveto{\pgfqpoint{1.707267in}{1.964046in}}{\pgfqpoint{1.703995in}{1.956146in}}{\pgfqpoint{1.703995in}{1.947910in}}%
\pgfpathcurveto{\pgfqpoint{1.703995in}{1.939673in}}{\pgfqpoint{1.707267in}{1.931773in}}{\pgfqpoint{1.713091in}{1.925949in}}%
\pgfpathcurveto{\pgfqpoint{1.718915in}{1.920125in}}{\pgfqpoint{1.726815in}{1.916853in}}{\pgfqpoint{1.735051in}{1.916853in}}%
\pgfpathclose%
\pgfusepath{stroke,fill}%
\end{pgfscope}%
\begin{pgfscope}%
\pgfpathrectangle{\pgfqpoint{0.100000in}{0.212622in}}{\pgfqpoint{3.696000in}{3.696000in}}%
\pgfusepath{clip}%
\pgfsetbuttcap%
\pgfsetroundjoin%
\definecolor{currentfill}{rgb}{0.121569,0.466667,0.705882}%
\pgfsetfillcolor{currentfill}%
\pgfsetfillopacity{0.353287}%
\pgfsetlinewidth{1.003750pt}%
\definecolor{currentstroke}{rgb}{0.121569,0.466667,0.705882}%
\pgfsetstrokecolor{currentstroke}%
\pgfsetstrokeopacity{0.353287}%
\pgfsetdash{}{0pt}%
\pgfpathmoveto{\pgfqpoint{1.733392in}{1.914994in}}%
\pgfpathcurveto{\pgfqpoint{1.741628in}{1.914994in}}{\pgfqpoint{1.749528in}{1.918267in}}{\pgfqpoint{1.755352in}{1.924091in}}%
\pgfpathcurveto{\pgfqpoint{1.761176in}{1.929915in}}{\pgfqpoint{1.764448in}{1.937815in}}{\pgfqpoint{1.764448in}{1.946051in}}%
\pgfpathcurveto{\pgfqpoint{1.764448in}{1.954287in}}{\pgfqpoint{1.761176in}{1.962187in}}{\pgfqpoint{1.755352in}{1.968011in}}%
\pgfpathcurveto{\pgfqpoint{1.749528in}{1.973835in}}{\pgfqpoint{1.741628in}{1.977107in}}{\pgfqpoint{1.733392in}{1.977107in}}%
\pgfpathcurveto{\pgfqpoint{1.725156in}{1.977107in}}{\pgfqpoint{1.717256in}{1.973835in}}{\pgfqpoint{1.711432in}{1.968011in}}%
\pgfpathcurveto{\pgfqpoint{1.705608in}{1.962187in}}{\pgfqpoint{1.702335in}{1.954287in}}{\pgfqpoint{1.702335in}{1.946051in}}%
\pgfpathcurveto{\pgfqpoint{1.702335in}{1.937815in}}{\pgfqpoint{1.705608in}{1.929915in}}{\pgfqpoint{1.711432in}{1.924091in}}%
\pgfpathcurveto{\pgfqpoint{1.717256in}{1.918267in}}{\pgfqpoint{1.725156in}{1.914994in}}{\pgfqpoint{1.733392in}{1.914994in}}%
\pgfpathclose%
\pgfusepath{stroke,fill}%
\end{pgfscope}%
\begin{pgfscope}%
\pgfpathrectangle{\pgfqpoint{0.100000in}{0.212622in}}{\pgfqpoint{3.696000in}{3.696000in}}%
\pgfusepath{clip}%
\pgfsetbuttcap%
\pgfsetroundjoin%
\definecolor{currentfill}{rgb}{0.121569,0.466667,0.705882}%
\pgfsetfillcolor{currentfill}%
\pgfsetfillopacity{0.353965}%
\pgfsetlinewidth{1.003750pt}%
\definecolor{currentstroke}{rgb}{0.121569,0.466667,0.705882}%
\pgfsetstrokecolor{currentstroke}%
\pgfsetstrokeopacity{0.353965}%
\pgfsetdash{}{0pt}%
\pgfpathmoveto{\pgfqpoint{1.729035in}{1.912747in}}%
\pgfpathcurveto{\pgfqpoint{1.737271in}{1.912747in}}{\pgfqpoint{1.745171in}{1.916019in}}{\pgfqpoint{1.750995in}{1.921843in}}%
\pgfpathcurveto{\pgfqpoint{1.756819in}{1.927667in}}{\pgfqpoint{1.760091in}{1.935567in}}{\pgfqpoint{1.760091in}{1.943804in}}%
\pgfpathcurveto{\pgfqpoint{1.760091in}{1.952040in}}{\pgfqpoint{1.756819in}{1.959940in}}{\pgfqpoint{1.750995in}{1.965764in}}%
\pgfpathcurveto{\pgfqpoint{1.745171in}{1.971588in}}{\pgfqpoint{1.737271in}{1.974860in}}{\pgfqpoint{1.729035in}{1.974860in}}%
\pgfpathcurveto{\pgfqpoint{1.720798in}{1.974860in}}{\pgfqpoint{1.712898in}{1.971588in}}{\pgfqpoint{1.707074in}{1.965764in}}%
\pgfpathcurveto{\pgfqpoint{1.701251in}{1.959940in}}{\pgfqpoint{1.697978in}{1.952040in}}{\pgfqpoint{1.697978in}{1.943804in}}%
\pgfpathcurveto{\pgfqpoint{1.697978in}{1.935567in}}{\pgfqpoint{1.701251in}{1.927667in}}{\pgfqpoint{1.707074in}{1.921843in}}%
\pgfpathcurveto{\pgfqpoint{1.712898in}{1.916019in}}{\pgfqpoint{1.720798in}{1.912747in}}{\pgfqpoint{1.729035in}{1.912747in}}%
\pgfpathclose%
\pgfusepath{stroke,fill}%
\end{pgfscope}%
\begin{pgfscope}%
\pgfpathrectangle{\pgfqpoint{0.100000in}{0.212622in}}{\pgfqpoint{3.696000in}{3.696000in}}%
\pgfusepath{clip}%
\pgfsetbuttcap%
\pgfsetroundjoin%
\definecolor{currentfill}{rgb}{0.121569,0.466667,0.705882}%
\pgfsetfillcolor{currentfill}%
\pgfsetfillopacity{0.355456}%
\pgfsetlinewidth{1.003750pt}%
\definecolor{currentstroke}{rgb}{0.121569,0.466667,0.705882}%
\pgfsetstrokecolor{currentstroke}%
\pgfsetstrokeopacity{0.355456}%
\pgfsetdash{}{0pt}%
\pgfpathmoveto{\pgfqpoint{1.724345in}{1.906060in}}%
\pgfpathcurveto{\pgfqpoint{1.732581in}{1.906060in}}{\pgfqpoint{1.740481in}{1.909332in}}{\pgfqpoint{1.746305in}{1.915156in}}%
\pgfpathcurveto{\pgfqpoint{1.752129in}{1.920980in}}{\pgfqpoint{1.755401in}{1.928880in}}{\pgfqpoint{1.755401in}{1.937116in}}%
\pgfpathcurveto{\pgfqpoint{1.755401in}{1.945352in}}{\pgfqpoint{1.752129in}{1.953252in}}{\pgfqpoint{1.746305in}{1.959076in}}%
\pgfpathcurveto{\pgfqpoint{1.740481in}{1.964900in}}{\pgfqpoint{1.732581in}{1.968173in}}{\pgfqpoint{1.724345in}{1.968173in}}%
\pgfpathcurveto{\pgfqpoint{1.716109in}{1.968173in}}{\pgfqpoint{1.708209in}{1.964900in}}{\pgfqpoint{1.702385in}{1.959076in}}%
\pgfpathcurveto{\pgfqpoint{1.696561in}{1.953252in}}{\pgfqpoint{1.693288in}{1.945352in}}{\pgfqpoint{1.693288in}{1.937116in}}%
\pgfpathcurveto{\pgfqpoint{1.693288in}{1.928880in}}{\pgfqpoint{1.696561in}{1.920980in}}{\pgfqpoint{1.702385in}{1.915156in}}%
\pgfpathcurveto{\pgfqpoint{1.708209in}{1.909332in}}{\pgfqpoint{1.716109in}{1.906060in}}{\pgfqpoint{1.724345in}{1.906060in}}%
\pgfpathclose%
\pgfusepath{stroke,fill}%
\end{pgfscope}%
\begin{pgfscope}%
\pgfpathrectangle{\pgfqpoint{0.100000in}{0.212622in}}{\pgfqpoint{3.696000in}{3.696000in}}%
\pgfusepath{clip}%
\pgfsetbuttcap%
\pgfsetroundjoin%
\definecolor{currentfill}{rgb}{0.121569,0.466667,0.705882}%
\pgfsetfillcolor{currentfill}%
\pgfsetfillopacity{0.356571}%
\pgfsetlinewidth{1.003750pt}%
\definecolor{currentstroke}{rgb}{0.121569,0.466667,0.705882}%
\pgfsetstrokecolor{currentstroke}%
\pgfsetstrokeopacity{0.356571}%
\pgfsetdash{}{0pt}%
\pgfpathmoveto{\pgfqpoint{1.563110in}{2.112407in}}%
\pgfpathcurveto{\pgfqpoint{1.571346in}{2.112407in}}{\pgfqpoint{1.579246in}{2.115680in}}{\pgfqpoint{1.585070in}{2.121504in}}%
\pgfpathcurveto{\pgfqpoint{1.590894in}{2.127328in}}{\pgfqpoint{1.594166in}{2.135228in}}{\pgfqpoint{1.594166in}{2.143464in}}%
\pgfpathcurveto{\pgfqpoint{1.594166in}{2.151700in}}{\pgfqpoint{1.590894in}{2.159600in}}{\pgfqpoint{1.585070in}{2.165424in}}%
\pgfpathcurveto{\pgfqpoint{1.579246in}{2.171248in}}{\pgfqpoint{1.571346in}{2.174520in}}{\pgfqpoint{1.563110in}{2.174520in}}%
\pgfpathcurveto{\pgfqpoint{1.554874in}{2.174520in}}{\pgfqpoint{1.546974in}{2.171248in}}{\pgfqpoint{1.541150in}{2.165424in}}%
\pgfpathcurveto{\pgfqpoint{1.535326in}{2.159600in}}{\pgfqpoint{1.532053in}{2.151700in}}{\pgfqpoint{1.532053in}{2.143464in}}%
\pgfpathcurveto{\pgfqpoint{1.532053in}{2.135228in}}{\pgfqpoint{1.535326in}{2.127328in}}{\pgfqpoint{1.541150in}{2.121504in}}%
\pgfpathcurveto{\pgfqpoint{1.546974in}{2.115680in}}{\pgfqpoint{1.554874in}{2.112407in}}{\pgfqpoint{1.563110in}{2.112407in}}%
\pgfpathclose%
\pgfusepath{stroke,fill}%
\end{pgfscope}%
\begin{pgfscope}%
\pgfpathrectangle{\pgfqpoint{0.100000in}{0.212622in}}{\pgfqpoint{3.696000in}{3.696000in}}%
\pgfusepath{clip}%
\pgfsetbuttcap%
\pgfsetroundjoin%
\definecolor{currentfill}{rgb}{0.121569,0.466667,0.705882}%
\pgfsetfillcolor{currentfill}%
\pgfsetfillopacity{0.357954}%
\pgfsetlinewidth{1.003750pt}%
\definecolor{currentstroke}{rgb}{0.121569,0.466667,0.705882}%
\pgfsetstrokecolor{currentstroke}%
\pgfsetstrokeopacity{0.357954}%
\pgfsetdash{}{0pt}%
\pgfpathmoveto{\pgfqpoint{1.709075in}{1.901860in}}%
\pgfpathcurveto{\pgfqpoint{1.717311in}{1.901860in}}{\pgfqpoint{1.725211in}{1.905132in}}{\pgfqpoint{1.731035in}{1.910956in}}%
\pgfpathcurveto{\pgfqpoint{1.736859in}{1.916780in}}{\pgfqpoint{1.740131in}{1.924680in}}{\pgfqpoint{1.740131in}{1.932917in}}%
\pgfpathcurveto{\pgfqpoint{1.740131in}{1.941153in}}{\pgfqpoint{1.736859in}{1.949053in}}{\pgfqpoint{1.731035in}{1.954877in}}%
\pgfpathcurveto{\pgfqpoint{1.725211in}{1.960701in}}{\pgfqpoint{1.717311in}{1.963973in}}{\pgfqpoint{1.709075in}{1.963973in}}%
\pgfpathcurveto{\pgfqpoint{1.700839in}{1.963973in}}{\pgfqpoint{1.692938in}{1.960701in}}{\pgfqpoint{1.687115in}{1.954877in}}%
\pgfpathcurveto{\pgfqpoint{1.681291in}{1.949053in}}{\pgfqpoint{1.678018in}{1.941153in}}{\pgfqpoint{1.678018in}{1.932917in}}%
\pgfpathcurveto{\pgfqpoint{1.678018in}{1.924680in}}{\pgfqpoint{1.681291in}{1.916780in}}{\pgfqpoint{1.687115in}{1.910956in}}%
\pgfpathcurveto{\pgfqpoint{1.692938in}{1.905132in}}{\pgfqpoint{1.700839in}{1.901860in}}{\pgfqpoint{1.709075in}{1.901860in}}%
\pgfpathclose%
\pgfusepath{stroke,fill}%
\end{pgfscope}%
\begin{pgfscope}%
\pgfpathrectangle{\pgfqpoint{0.100000in}{0.212622in}}{\pgfqpoint{3.696000in}{3.696000in}}%
\pgfusepath{clip}%
\pgfsetbuttcap%
\pgfsetroundjoin%
\definecolor{currentfill}{rgb}{0.121569,0.466667,0.705882}%
\pgfsetfillcolor{currentfill}%
\pgfsetfillopacity{0.363590}%
\pgfsetlinewidth{1.003750pt}%
\definecolor{currentstroke}{rgb}{0.121569,0.466667,0.705882}%
\pgfsetstrokecolor{currentstroke}%
\pgfsetstrokeopacity{0.363590}%
\pgfsetdash{}{0pt}%
\pgfpathmoveto{\pgfqpoint{1.694095in}{1.883997in}}%
\pgfpathcurveto{\pgfqpoint{1.702331in}{1.883997in}}{\pgfqpoint{1.710231in}{1.887269in}}{\pgfqpoint{1.716055in}{1.893093in}}%
\pgfpathcurveto{\pgfqpoint{1.721879in}{1.898917in}}{\pgfqpoint{1.725152in}{1.906817in}}{\pgfqpoint{1.725152in}{1.915054in}}%
\pgfpathcurveto{\pgfqpoint{1.725152in}{1.923290in}}{\pgfqpoint{1.721879in}{1.931190in}}{\pgfqpoint{1.716055in}{1.937014in}}%
\pgfpathcurveto{\pgfqpoint{1.710231in}{1.942838in}}{\pgfqpoint{1.702331in}{1.946110in}}{\pgfqpoint{1.694095in}{1.946110in}}%
\pgfpathcurveto{\pgfqpoint{1.685859in}{1.946110in}}{\pgfqpoint{1.677959in}{1.942838in}}{\pgfqpoint{1.672135in}{1.937014in}}%
\pgfpathcurveto{\pgfqpoint{1.666311in}{1.931190in}}{\pgfqpoint{1.663039in}{1.923290in}}{\pgfqpoint{1.663039in}{1.915054in}}%
\pgfpathcurveto{\pgfqpoint{1.663039in}{1.906817in}}{\pgfqpoint{1.666311in}{1.898917in}}{\pgfqpoint{1.672135in}{1.893093in}}%
\pgfpathcurveto{\pgfqpoint{1.677959in}{1.887269in}}{\pgfqpoint{1.685859in}{1.883997in}}{\pgfqpoint{1.694095in}{1.883997in}}%
\pgfpathclose%
\pgfusepath{stroke,fill}%
\end{pgfscope}%
\begin{pgfscope}%
\pgfpathrectangle{\pgfqpoint{0.100000in}{0.212622in}}{\pgfqpoint{3.696000in}{3.696000in}}%
\pgfusepath{clip}%
\pgfsetbuttcap%
\pgfsetroundjoin%
\definecolor{currentfill}{rgb}{0.121569,0.466667,0.705882}%
\pgfsetfillcolor{currentfill}%
\pgfsetfillopacity{0.364528}%
\pgfsetlinewidth{1.003750pt}%
\definecolor{currentstroke}{rgb}{0.121569,0.466667,0.705882}%
\pgfsetstrokecolor{currentstroke}%
\pgfsetstrokeopacity{0.364528}%
\pgfsetdash{}{0pt}%
\pgfpathmoveto{\pgfqpoint{1.689105in}{1.883165in}}%
\pgfpathcurveto{\pgfqpoint{1.697341in}{1.883165in}}{\pgfqpoint{1.705241in}{1.886437in}}{\pgfqpoint{1.711065in}{1.892261in}}%
\pgfpathcurveto{\pgfqpoint{1.716889in}{1.898085in}}{\pgfqpoint{1.720161in}{1.905985in}}{\pgfqpoint{1.720161in}{1.914221in}}%
\pgfpathcurveto{\pgfqpoint{1.720161in}{1.922458in}}{\pgfqpoint{1.716889in}{1.930358in}}{\pgfqpoint{1.711065in}{1.936182in}}%
\pgfpathcurveto{\pgfqpoint{1.705241in}{1.942006in}}{\pgfqpoint{1.697341in}{1.945278in}}{\pgfqpoint{1.689105in}{1.945278in}}%
\pgfpathcurveto{\pgfqpoint{1.680868in}{1.945278in}}{\pgfqpoint{1.672968in}{1.942006in}}{\pgfqpoint{1.667144in}{1.936182in}}%
\pgfpathcurveto{\pgfqpoint{1.661321in}{1.930358in}}{\pgfqpoint{1.658048in}{1.922458in}}{\pgfqpoint{1.658048in}{1.914221in}}%
\pgfpathcurveto{\pgfqpoint{1.658048in}{1.905985in}}{\pgfqpoint{1.661321in}{1.898085in}}{\pgfqpoint{1.667144in}{1.892261in}}%
\pgfpathcurveto{\pgfqpoint{1.672968in}{1.886437in}}{\pgfqpoint{1.680868in}{1.883165in}}{\pgfqpoint{1.689105in}{1.883165in}}%
\pgfpathclose%
\pgfusepath{stroke,fill}%
\end{pgfscope}%
\begin{pgfscope}%
\pgfpathrectangle{\pgfqpoint{0.100000in}{0.212622in}}{\pgfqpoint{3.696000in}{3.696000in}}%
\pgfusepath{clip}%
\pgfsetbuttcap%
\pgfsetroundjoin%
\definecolor{currentfill}{rgb}{0.121569,0.466667,0.705882}%
\pgfsetfillcolor{currentfill}%
\pgfsetfillopacity{0.365145}%
\pgfsetlinewidth{1.003750pt}%
\definecolor{currentstroke}{rgb}{0.121569,0.466667,0.705882}%
\pgfsetstrokecolor{currentstroke}%
\pgfsetstrokeopacity{0.365145}%
\pgfsetdash{}{0pt}%
\pgfpathmoveto{\pgfqpoint{1.687802in}{1.881487in}}%
\pgfpathcurveto{\pgfqpoint{1.696038in}{1.881487in}}{\pgfqpoint{1.703938in}{1.884759in}}{\pgfqpoint{1.709762in}{1.890583in}}%
\pgfpathcurveto{\pgfqpoint{1.715586in}{1.896407in}}{\pgfqpoint{1.718858in}{1.904307in}}{\pgfqpoint{1.718858in}{1.912543in}}%
\pgfpathcurveto{\pgfqpoint{1.718858in}{1.920780in}}{\pgfqpoint{1.715586in}{1.928680in}}{\pgfqpoint{1.709762in}{1.934504in}}%
\pgfpathcurveto{\pgfqpoint{1.703938in}{1.940328in}}{\pgfqpoint{1.696038in}{1.943600in}}{\pgfqpoint{1.687802in}{1.943600in}}%
\pgfpathcurveto{\pgfqpoint{1.679566in}{1.943600in}}{\pgfqpoint{1.671665in}{1.940328in}}{\pgfqpoint{1.665842in}{1.934504in}}%
\pgfpathcurveto{\pgfqpoint{1.660018in}{1.928680in}}{\pgfqpoint{1.656745in}{1.920780in}}{\pgfqpoint{1.656745in}{1.912543in}}%
\pgfpathcurveto{\pgfqpoint{1.656745in}{1.904307in}}{\pgfqpoint{1.660018in}{1.896407in}}{\pgfqpoint{1.665842in}{1.890583in}}%
\pgfpathcurveto{\pgfqpoint{1.671665in}{1.884759in}}{\pgfqpoint{1.679566in}{1.881487in}}{\pgfqpoint{1.687802in}{1.881487in}}%
\pgfpathclose%
\pgfusepath{stroke,fill}%
\end{pgfscope}%
\begin{pgfscope}%
\pgfpathrectangle{\pgfqpoint{0.100000in}{0.212622in}}{\pgfqpoint{3.696000in}{3.696000in}}%
\pgfusepath{clip}%
\pgfsetbuttcap%
\pgfsetroundjoin%
\definecolor{currentfill}{rgb}{0.121569,0.466667,0.705882}%
\pgfsetfillcolor{currentfill}%
\pgfsetfillopacity{0.366018}%
\pgfsetlinewidth{1.003750pt}%
\definecolor{currentstroke}{rgb}{0.121569,0.466667,0.705882}%
\pgfsetstrokecolor{currentstroke}%
\pgfsetstrokeopacity{0.366018}%
\pgfsetdash{}{0pt}%
\pgfpathmoveto{\pgfqpoint{1.682394in}{1.880916in}}%
\pgfpathcurveto{\pgfqpoint{1.690630in}{1.880916in}}{\pgfqpoint{1.698530in}{1.884189in}}{\pgfqpoint{1.704354in}{1.890012in}}%
\pgfpathcurveto{\pgfqpoint{1.710178in}{1.895836in}}{\pgfqpoint{1.713450in}{1.903736in}}{\pgfqpoint{1.713450in}{1.911973in}}%
\pgfpathcurveto{\pgfqpoint{1.713450in}{1.920209in}}{\pgfqpoint{1.710178in}{1.928109in}}{\pgfqpoint{1.704354in}{1.933933in}}%
\pgfpathcurveto{\pgfqpoint{1.698530in}{1.939757in}}{\pgfqpoint{1.690630in}{1.943029in}}{\pgfqpoint{1.682394in}{1.943029in}}%
\pgfpathcurveto{\pgfqpoint{1.674157in}{1.943029in}}{\pgfqpoint{1.666257in}{1.939757in}}{\pgfqpoint{1.660433in}{1.933933in}}%
\pgfpathcurveto{\pgfqpoint{1.654609in}{1.928109in}}{\pgfqpoint{1.651337in}{1.920209in}}{\pgfqpoint{1.651337in}{1.911973in}}%
\pgfpathcurveto{\pgfqpoint{1.651337in}{1.903736in}}{\pgfqpoint{1.654609in}{1.895836in}}{\pgfqpoint{1.660433in}{1.890012in}}%
\pgfpathcurveto{\pgfqpoint{1.666257in}{1.884189in}}{\pgfqpoint{1.674157in}{1.880916in}}{\pgfqpoint{1.682394in}{1.880916in}}%
\pgfpathclose%
\pgfusepath{stroke,fill}%
\end{pgfscope}%
\begin{pgfscope}%
\pgfpathrectangle{\pgfqpoint{0.100000in}{0.212622in}}{\pgfqpoint{3.696000in}{3.696000in}}%
\pgfusepath{clip}%
\pgfsetbuttcap%
\pgfsetroundjoin%
\definecolor{currentfill}{rgb}{0.121569,0.466667,0.705882}%
\pgfsetfillcolor{currentfill}%
\pgfsetfillopacity{0.366413}%
\pgfsetlinewidth{1.003750pt}%
\definecolor{currentstroke}{rgb}{0.121569,0.466667,0.705882}%
\pgfsetstrokecolor{currentstroke}%
\pgfsetstrokeopacity{0.366413}%
\pgfsetdash{}{0pt}%
\pgfpathmoveto{\pgfqpoint{1.680293in}{1.878189in}}%
\pgfpathcurveto{\pgfqpoint{1.688529in}{1.878189in}}{\pgfqpoint{1.696429in}{1.881462in}}{\pgfqpoint{1.702253in}{1.887285in}}%
\pgfpathcurveto{\pgfqpoint{1.708077in}{1.893109in}}{\pgfqpoint{1.711349in}{1.901009in}}{\pgfqpoint{1.711349in}{1.909246in}}%
\pgfpathcurveto{\pgfqpoint{1.711349in}{1.917482in}}{\pgfqpoint{1.708077in}{1.925382in}}{\pgfqpoint{1.702253in}{1.931206in}}%
\pgfpathcurveto{\pgfqpoint{1.696429in}{1.937030in}}{\pgfqpoint{1.688529in}{1.940302in}}{\pgfqpoint{1.680293in}{1.940302in}}%
\pgfpathcurveto{\pgfqpoint{1.672056in}{1.940302in}}{\pgfqpoint{1.664156in}{1.937030in}}{\pgfqpoint{1.658332in}{1.931206in}}%
\pgfpathcurveto{\pgfqpoint{1.652508in}{1.925382in}}{\pgfqpoint{1.649236in}{1.917482in}}{\pgfqpoint{1.649236in}{1.909246in}}%
\pgfpathcurveto{\pgfqpoint{1.649236in}{1.901009in}}{\pgfqpoint{1.652508in}{1.893109in}}{\pgfqpoint{1.658332in}{1.887285in}}%
\pgfpathcurveto{\pgfqpoint{1.664156in}{1.881462in}}{\pgfqpoint{1.672056in}{1.878189in}}{\pgfqpoint{1.680293in}{1.878189in}}%
\pgfpathclose%
\pgfusepath{stroke,fill}%
\end{pgfscope}%
\begin{pgfscope}%
\pgfpathrectangle{\pgfqpoint{0.100000in}{0.212622in}}{\pgfqpoint{3.696000in}{3.696000in}}%
\pgfusepath{clip}%
\pgfsetbuttcap%
\pgfsetroundjoin%
\definecolor{currentfill}{rgb}{0.121569,0.466667,0.705882}%
\pgfsetfillcolor{currentfill}%
\pgfsetfillopacity{0.366570}%
\pgfsetlinewidth{1.003750pt}%
\definecolor{currentstroke}{rgb}{0.121569,0.466667,0.705882}%
\pgfsetstrokecolor{currentstroke}%
\pgfsetstrokeopacity{0.366570}%
\pgfsetdash{}{0pt}%
\pgfpathmoveto{\pgfqpoint{1.573988in}{2.082983in}}%
\pgfpathcurveto{\pgfqpoint{1.582224in}{2.082983in}}{\pgfqpoint{1.590124in}{2.086255in}}{\pgfqpoint{1.595948in}{2.092079in}}%
\pgfpathcurveto{\pgfqpoint{1.601772in}{2.097903in}}{\pgfqpoint{1.605045in}{2.105803in}}{\pgfqpoint{1.605045in}{2.114039in}}%
\pgfpathcurveto{\pgfqpoint{1.605045in}{2.122275in}}{\pgfqpoint{1.601772in}{2.130175in}}{\pgfqpoint{1.595948in}{2.135999in}}%
\pgfpathcurveto{\pgfqpoint{1.590124in}{2.141823in}}{\pgfqpoint{1.582224in}{2.145096in}}{\pgfqpoint{1.573988in}{2.145096in}}%
\pgfpathcurveto{\pgfqpoint{1.565752in}{2.145096in}}{\pgfqpoint{1.557852in}{2.141823in}}{\pgfqpoint{1.552028in}{2.135999in}}%
\pgfpathcurveto{\pgfqpoint{1.546204in}{2.130175in}}{\pgfqpoint{1.542932in}{2.122275in}}{\pgfqpoint{1.542932in}{2.114039in}}%
\pgfpathcurveto{\pgfqpoint{1.542932in}{2.105803in}}{\pgfqpoint{1.546204in}{2.097903in}}{\pgfqpoint{1.552028in}{2.092079in}}%
\pgfpathcurveto{\pgfqpoint{1.557852in}{2.086255in}}{\pgfqpoint{1.565752in}{2.082983in}}{\pgfqpoint{1.573988in}{2.082983in}}%
\pgfpathclose%
\pgfusepath{stroke,fill}%
\end{pgfscope}%
\begin{pgfscope}%
\pgfpathrectangle{\pgfqpoint{0.100000in}{0.212622in}}{\pgfqpoint{3.696000in}{3.696000in}}%
\pgfusepath{clip}%
\pgfsetbuttcap%
\pgfsetroundjoin%
\definecolor{currentfill}{rgb}{0.121569,0.466667,0.705882}%
\pgfsetfillcolor{currentfill}%
\pgfsetfillopacity{0.367534}%
\pgfsetlinewidth{1.003750pt}%
\definecolor{currentstroke}{rgb}{0.121569,0.466667,0.705882}%
\pgfsetstrokecolor{currentstroke}%
\pgfsetstrokeopacity{0.367534}%
\pgfsetdash{}{0pt}%
\pgfpathmoveto{\pgfqpoint{1.675600in}{1.877306in}}%
\pgfpathcurveto{\pgfqpoint{1.683836in}{1.877306in}}{\pgfqpoint{1.691736in}{1.880578in}}{\pgfqpoint{1.697560in}{1.886402in}}%
\pgfpathcurveto{\pgfqpoint{1.703384in}{1.892226in}}{\pgfqpoint{1.706656in}{1.900126in}}{\pgfqpoint{1.706656in}{1.908363in}}%
\pgfpathcurveto{\pgfqpoint{1.706656in}{1.916599in}}{\pgfqpoint{1.703384in}{1.924499in}}{\pgfqpoint{1.697560in}{1.930323in}}%
\pgfpathcurveto{\pgfqpoint{1.691736in}{1.936147in}}{\pgfqpoint{1.683836in}{1.939419in}}{\pgfqpoint{1.675600in}{1.939419in}}%
\pgfpathcurveto{\pgfqpoint{1.667363in}{1.939419in}}{\pgfqpoint{1.659463in}{1.936147in}}{\pgfqpoint{1.653639in}{1.930323in}}%
\pgfpathcurveto{\pgfqpoint{1.647816in}{1.924499in}}{\pgfqpoint{1.644543in}{1.916599in}}{\pgfqpoint{1.644543in}{1.908363in}}%
\pgfpathcurveto{\pgfqpoint{1.644543in}{1.900126in}}{\pgfqpoint{1.647816in}{1.892226in}}{\pgfqpoint{1.653639in}{1.886402in}}%
\pgfpathcurveto{\pgfqpoint{1.659463in}{1.880578in}}{\pgfqpoint{1.667363in}{1.877306in}}{\pgfqpoint{1.675600in}{1.877306in}}%
\pgfpathclose%
\pgfusepath{stroke,fill}%
\end{pgfscope}%
\begin{pgfscope}%
\pgfpathrectangle{\pgfqpoint{0.100000in}{0.212622in}}{\pgfqpoint{3.696000in}{3.696000in}}%
\pgfusepath{clip}%
\pgfsetbuttcap%
\pgfsetroundjoin%
\definecolor{currentfill}{rgb}{0.121569,0.466667,0.705882}%
\pgfsetfillcolor{currentfill}%
\pgfsetfillopacity{0.367861}%
\pgfsetlinewidth{1.003750pt}%
\definecolor{currentstroke}{rgb}{0.121569,0.466667,0.705882}%
\pgfsetstrokecolor{currentstroke}%
\pgfsetstrokeopacity{0.367861}%
\pgfsetdash{}{0pt}%
\pgfpathmoveto{\pgfqpoint{1.672880in}{1.874858in}}%
\pgfpathcurveto{\pgfqpoint{1.681117in}{1.874858in}}{\pgfqpoint{1.689017in}{1.878131in}}{\pgfqpoint{1.694841in}{1.883955in}}%
\pgfpathcurveto{\pgfqpoint{1.700665in}{1.889779in}}{\pgfqpoint{1.703937in}{1.897679in}}{\pgfqpoint{1.703937in}{1.905915in}}%
\pgfpathcurveto{\pgfqpoint{1.703937in}{1.914151in}}{\pgfqpoint{1.700665in}{1.922051in}}{\pgfqpoint{1.694841in}{1.927875in}}%
\pgfpathcurveto{\pgfqpoint{1.689017in}{1.933699in}}{\pgfqpoint{1.681117in}{1.936971in}}{\pgfqpoint{1.672880in}{1.936971in}}%
\pgfpathcurveto{\pgfqpoint{1.664644in}{1.936971in}}{\pgfqpoint{1.656744in}{1.933699in}}{\pgfqpoint{1.650920in}{1.927875in}}%
\pgfpathcurveto{\pgfqpoint{1.645096in}{1.922051in}}{\pgfqpoint{1.641824in}{1.914151in}}{\pgfqpoint{1.641824in}{1.905915in}}%
\pgfpathcurveto{\pgfqpoint{1.641824in}{1.897679in}}{\pgfqpoint{1.645096in}{1.889779in}}{\pgfqpoint{1.650920in}{1.883955in}}%
\pgfpathcurveto{\pgfqpoint{1.656744in}{1.878131in}}{\pgfqpoint{1.664644in}{1.874858in}}{\pgfqpoint{1.672880in}{1.874858in}}%
\pgfpathclose%
\pgfusepath{stroke,fill}%
\end{pgfscope}%
\begin{pgfscope}%
\pgfpathrectangle{\pgfqpoint{0.100000in}{0.212622in}}{\pgfqpoint{3.696000in}{3.696000in}}%
\pgfusepath{clip}%
\pgfsetbuttcap%
\pgfsetroundjoin%
\definecolor{currentfill}{rgb}{0.121569,0.466667,0.705882}%
\pgfsetfillcolor{currentfill}%
\pgfsetfillopacity{0.368104}%
\pgfsetlinewidth{1.003750pt}%
\definecolor{currentstroke}{rgb}{0.121569,0.466667,0.705882}%
\pgfsetstrokecolor{currentstroke}%
\pgfsetstrokeopacity{0.368104}%
\pgfsetdash{}{0pt}%
\pgfpathmoveto{\pgfqpoint{1.671923in}{1.873976in}}%
\pgfpathcurveto{\pgfqpoint{1.680159in}{1.873976in}}{\pgfqpoint{1.688059in}{1.877248in}}{\pgfqpoint{1.693883in}{1.883072in}}%
\pgfpathcurveto{\pgfqpoint{1.699707in}{1.888896in}}{\pgfqpoint{1.702979in}{1.896796in}}{\pgfqpoint{1.702979in}{1.905032in}}%
\pgfpathcurveto{\pgfqpoint{1.702979in}{1.913268in}}{\pgfqpoint{1.699707in}{1.921168in}}{\pgfqpoint{1.693883in}{1.926992in}}%
\pgfpathcurveto{\pgfqpoint{1.688059in}{1.932816in}}{\pgfqpoint{1.680159in}{1.936089in}}{\pgfqpoint{1.671923in}{1.936089in}}%
\pgfpathcurveto{\pgfqpoint{1.663686in}{1.936089in}}{\pgfqpoint{1.655786in}{1.932816in}}{\pgfqpoint{1.649962in}{1.926992in}}%
\pgfpathcurveto{\pgfqpoint{1.644138in}{1.921168in}}{\pgfqpoint{1.640866in}{1.913268in}}{\pgfqpoint{1.640866in}{1.905032in}}%
\pgfpathcurveto{\pgfqpoint{1.640866in}{1.896796in}}{\pgfqpoint{1.644138in}{1.888896in}}{\pgfqpoint{1.649962in}{1.883072in}}%
\pgfpathcurveto{\pgfqpoint{1.655786in}{1.877248in}}{\pgfqpoint{1.663686in}{1.873976in}}{\pgfqpoint{1.671923in}{1.873976in}}%
\pgfpathclose%
\pgfusepath{stroke,fill}%
\end{pgfscope}%
\begin{pgfscope}%
\pgfpathrectangle{\pgfqpoint{0.100000in}{0.212622in}}{\pgfqpoint{3.696000in}{3.696000in}}%
\pgfusepath{clip}%
\pgfsetbuttcap%
\pgfsetroundjoin%
\definecolor{currentfill}{rgb}{0.121569,0.466667,0.705882}%
\pgfsetfillcolor{currentfill}%
\pgfsetfillopacity{0.368428}%
\pgfsetlinewidth{1.003750pt}%
\definecolor{currentstroke}{rgb}{0.121569,0.466667,0.705882}%
\pgfsetstrokecolor{currentstroke}%
\pgfsetstrokeopacity{0.368428}%
\pgfsetdash{}{0pt}%
\pgfpathmoveto{\pgfqpoint{1.669339in}{1.872842in}}%
\pgfpathcurveto{\pgfqpoint{1.677576in}{1.872842in}}{\pgfqpoint{1.685476in}{1.876115in}}{\pgfqpoint{1.691300in}{1.881938in}}%
\pgfpathcurveto{\pgfqpoint{1.697124in}{1.887762in}}{\pgfqpoint{1.700396in}{1.895662in}}{\pgfqpoint{1.700396in}{1.903899in}}%
\pgfpathcurveto{\pgfqpoint{1.700396in}{1.912135in}}{\pgfqpoint{1.697124in}{1.920035in}}{\pgfqpoint{1.691300in}{1.925859in}}%
\pgfpathcurveto{\pgfqpoint{1.685476in}{1.931683in}}{\pgfqpoint{1.677576in}{1.934955in}}{\pgfqpoint{1.669339in}{1.934955in}}%
\pgfpathcurveto{\pgfqpoint{1.661103in}{1.934955in}}{\pgfqpoint{1.653203in}{1.931683in}}{\pgfqpoint{1.647379in}{1.925859in}}%
\pgfpathcurveto{\pgfqpoint{1.641555in}{1.920035in}}{\pgfqpoint{1.638283in}{1.912135in}}{\pgfqpoint{1.638283in}{1.903899in}}%
\pgfpathcurveto{\pgfqpoint{1.638283in}{1.895662in}}{\pgfqpoint{1.641555in}{1.887762in}}{\pgfqpoint{1.647379in}{1.881938in}}%
\pgfpathcurveto{\pgfqpoint{1.653203in}{1.876115in}}{\pgfqpoint{1.661103in}{1.872842in}}{\pgfqpoint{1.669339in}{1.872842in}}%
\pgfpathclose%
\pgfusepath{stroke,fill}%
\end{pgfscope}%
\begin{pgfscope}%
\pgfpathrectangle{\pgfqpoint{0.100000in}{0.212622in}}{\pgfqpoint{3.696000in}{3.696000in}}%
\pgfusepath{clip}%
\pgfsetbuttcap%
\pgfsetroundjoin%
\definecolor{currentfill}{rgb}{0.121569,0.466667,0.705882}%
\pgfsetfillcolor{currentfill}%
\pgfsetfillopacity{0.369546}%
\pgfsetlinewidth{1.003750pt}%
\definecolor{currentstroke}{rgb}{0.121569,0.466667,0.705882}%
\pgfsetstrokecolor{currentstroke}%
\pgfsetstrokeopacity{0.369546}%
\pgfsetdash{}{0pt}%
\pgfpathmoveto{\pgfqpoint{1.667457in}{1.870729in}}%
\pgfpathcurveto{\pgfqpoint{1.675693in}{1.870729in}}{\pgfqpoint{1.683593in}{1.874001in}}{\pgfqpoint{1.689417in}{1.879825in}}%
\pgfpathcurveto{\pgfqpoint{1.695241in}{1.885649in}}{\pgfqpoint{1.698513in}{1.893549in}}{\pgfqpoint{1.698513in}{1.901785in}}%
\pgfpathcurveto{\pgfqpoint{1.698513in}{1.910021in}}{\pgfqpoint{1.695241in}{1.917922in}}{\pgfqpoint{1.689417in}{1.923745in}}%
\pgfpathcurveto{\pgfqpoint{1.683593in}{1.929569in}}{\pgfqpoint{1.675693in}{1.932842in}}{\pgfqpoint{1.667457in}{1.932842in}}%
\pgfpathcurveto{\pgfqpoint{1.659220in}{1.932842in}}{\pgfqpoint{1.651320in}{1.929569in}}{\pgfqpoint{1.645496in}{1.923745in}}%
\pgfpathcurveto{\pgfqpoint{1.639672in}{1.917922in}}{\pgfqpoint{1.636400in}{1.910021in}}{\pgfqpoint{1.636400in}{1.901785in}}%
\pgfpathcurveto{\pgfqpoint{1.636400in}{1.893549in}}{\pgfqpoint{1.639672in}{1.885649in}}{\pgfqpoint{1.645496in}{1.879825in}}%
\pgfpathcurveto{\pgfqpoint{1.651320in}{1.874001in}}{\pgfqpoint{1.659220in}{1.870729in}}{\pgfqpoint{1.667457in}{1.870729in}}%
\pgfpathclose%
\pgfusepath{stroke,fill}%
\end{pgfscope}%
\begin{pgfscope}%
\pgfpathrectangle{\pgfqpoint{0.100000in}{0.212622in}}{\pgfqpoint{3.696000in}{3.696000in}}%
\pgfusepath{clip}%
\pgfsetbuttcap%
\pgfsetroundjoin%
\definecolor{currentfill}{rgb}{0.121569,0.466667,0.705882}%
\pgfsetfillcolor{currentfill}%
\pgfsetfillopacity{0.369677}%
\pgfsetlinewidth{1.003750pt}%
\definecolor{currentstroke}{rgb}{0.121569,0.466667,0.705882}%
\pgfsetstrokecolor{currentstroke}%
\pgfsetstrokeopacity{0.369677}%
\pgfsetdash{}{0pt}%
\pgfpathmoveto{\pgfqpoint{1.666577in}{1.870630in}}%
\pgfpathcurveto{\pgfqpoint{1.674813in}{1.870630in}}{\pgfqpoint{1.682713in}{1.873903in}}{\pgfqpoint{1.688537in}{1.879727in}}%
\pgfpathcurveto{\pgfqpoint{1.694361in}{1.885550in}}{\pgfqpoint{1.697634in}{1.893451in}}{\pgfqpoint{1.697634in}{1.901687in}}%
\pgfpathcurveto{\pgfqpoint{1.697634in}{1.909923in}}{\pgfqpoint{1.694361in}{1.917823in}}{\pgfqpoint{1.688537in}{1.923647in}}%
\pgfpathcurveto{\pgfqpoint{1.682713in}{1.929471in}}{\pgfqpoint{1.674813in}{1.932743in}}{\pgfqpoint{1.666577in}{1.932743in}}%
\pgfpathcurveto{\pgfqpoint{1.658341in}{1.932743in}}{\pgfqpoint{1.650441in}{1.929471in}}{\pgfqpoint{1.644617in}{1.923647in}}%
\pgfpathcurveto{\pgfqpoint{1.638793in}{1.917823in}}{\pgfqpoint{1.635521in}{1.909923in}}{\pgfqpoint{1.635521in}{1.901687in}}%
\pgfpathcurveto{\pgfqpoint{1.635521in}{1.893451in}}{\pgfqpoint{1.638793in}{1.885550in}}{\pgfqpoint{1.644617in}{1.879727in}}%
\pgfpathcurveto{\pgfqpoint{1.650441in}{1.873903in}}{\pgfqpoint{1.658341in}{1.870630in}}{\pgfqpoint{1.666577in}{1.870630in}}%
\pgfpathclose%
\pgfusepath{stroke,fill}%
\end{pgfscope}%
\begin{pgfscope}%
\pgfpathrectangle{\pgfqpoint{0.100000in}{0.212622in}}{\pgfqpoint{3.696000in}{3.696000in}}%
\pgfusepath{clip}%
\pgfsetbuttcap%
\pgfsetroundjoin%
\definecolor{currentfill}{rgb}{0.121569,0.466667,0.705882}%
\pgfsetfillcolor{currentfill}%
\pgfsetfillopacity{0.369996}%
\pgfsetlinewidth{1.003750pt}%
\definecolor{currentstroke}{rgb}{0.121569,0.466667,0.705882}%
\pgfsetstrokecolor{currentstroke}%
\pgfsetstrokeopacity{0.369996}%
\pgfsetdash{}{0pt}%
\pgfpathmoveto{\pgfqpoint{1.665914in}{1.869708in}}%
\pgfpathcurveto{\pgfqpoint{1.674150in}{1.869708in}}{\pgfqpoint{1.682050in}{1.872980in}}{\pgfqpoint{1.687874in}{1.878804in}}%
\pgfpathcurveto{\pgfqpoint{1.693698in}{1.884628in}}{\pgfqpoint{1.696971in}{1.892528in}}{\pgfqpoint{1.696971in}{1.900765in}}%
\pgfpathcurveto{\pgfqpoint{1.696971in}{1.909001in}}{\pgfqpoint{1.693698in}{1.916901in}}{\pgfqpoint{1.687874in}{1.922725in}}%
\pgfpathcurveto{\pgfqpoint{1.682050in}{1.928549in}}{\pgfqpoint{1.674150in}{1.931821in}}{\pgfqpoint{1.665914in}{1.931821in}}%
\pgfpathcurveto{\pgfqpoint{1.657678in}{1.931821in}}{\pgfqpoint{1.649778in}{1.928549in}}{\pgfqpoint{1.643954in}{1.922725in}}%
\pgfpathcurveto{\pgfqpoint{1.638130in}{1.916901in}}{\pgfqpoint{1.634858in}{1.909001in}}{\pgfqpoint{1.634858in}{1.900765in}}%
\pgfpathcurveto{\pgfqpoint{1.634858in}{1.892528in}}{\pgfqpoint{1.638130in}{1.884628in}}{\pgfqpoint{1.643954in}{1.878804in}}%
\pgfpathcurveto{\pgfqpoint{1.649778in}{1.872980in}}{\pgfqpoint{1.657678in}{1.869708in}}{\pgfqpoint{1.665914in}{1.869708in}}%
\pgfpathclose%
\pgfusepath{stroke,fill}%
\end{pgfscope}%
\begin{pgfscope}%
\pgfpathrectangle{\pgfqpoint{0.100000in}{0.212622in}}{\pgfqpoint{3.696000in}{3.696000in}}%
\pgfusepath{clip}%
\pgfsetbuttcap%
\pgfsetroundjoin%
\definecolor{currentfill}{rgb}{0.121569,0.466667,0.705882}%
\pgfsetfillcolor{currentfill}%
\pgfsetfillopacity{0.370474}%
\pgfsetlinewidth{1.003750pt}%
\definecolor{currentstroke}{rgb}{0.121569,0.466667,0.705882}%
\pgfsetstrokecolor{currentstroke}%
\pgfsetstrokeopacity{0.370474}%
\pgfsetdash{}{0pt}%
\pgfpathmoveto{\pgfqpoint{1.662982in}{1.869751in}}%
\pgfpathcurveto{\pgfqpoint{1.671218in}{1.869751in}}{\pgfqpoint{1.679118in}{1.873023in}}{\pgfqpoint{1.684942in}{1.878847in}}%
\pgfpathcurveto{\pgfqpoint{1.690766in}{1.884671in}}{\pgfqpoint{1.694039in}{1.892571in}}{\pgfqpoint{1.694039in}{1.900808in}}%
\pgfpathcurveto{\pgfqpoint{1.694039in}{1.909044in}}{\pgfqpoint{1.690766in}{1.916944in}}{\pgfqpoint{1.684942in}{1.922768in}}%
\pgfpathcurveto{\pgfqpoint{1.679118in}{1.928592in}}{\pgfqpoint{1.671218in}{1.931864in}}{\pgfqpoint{1.662982in}{1.931864in}}%
\pgfpathcurveto{\pgfqpoint{1.654746in}{1.931864in}}{\pgfqpoint{1.646846in}{1.928592in}}{\pgfqpoint{1.641022in}{1.922768in}}%
\pgfpathcurveto{\pgfqpoint{1.635198in}{1.916944in}}{\pgfqpoint{1.631926in}{1.909044in}}{\pgfqpoint{1.631926in}{1.900808in}}%
\pgfpathcurveto{\pgfqpoint{1.631926in}{1.892571in}}{\pgfqpoint{1.635198in}{1.884671in}}{\pgfqpoint{1.641022in}{1.878847in}}%
\pgfpathcurveto{\pgfqpoint{1.646846in}{1.873023in}}{\pgfqpoint{1.654746in}{1.869751in}}{\pgfqpoint{1.662982in}{1.869751in}}%
\pgfpathclose%
\pgfusepath{stroke,fill}%
\end{pgfscope}%
\begin{pgfscope}%
\pgfpathrectangle{\pgfqpoint{0.100000in}{0.212622in}}{\pgfqpoint{3.696000in}{3.696000in}}%
\pgfusepath{clip}%
\pgfsetbuttcap%
\pgfsetroundjoin%
\definecolor{currentfill}{rgb}{0.121569,0.466667,0.705882}%
\pgfsetfillcolor{currentfill}%
\pgfsetfillopacity{0.370562}%
\pgfsetlinewidth{1.003750pt}%
\definecolor{currentstroke}{rgb}{0.121569,0.466667,0.705882}%
\pgfsetstrokecolor{currentstroke}%
\pgfsetstrokeopacity{0.370562}%
\pgfsetdash{}{0pt}%
\pgfpathmoveto{\pgfqpoint{1.662770in}{1.869453in}}%
\pgfpathcurveto{\pgfqpoint{1.671006in}{1.869453in}}{\pgfqpoint{1.678906in}{1.872725in}}{\pgfqpoint{1.684730in}{1.878549in}}%
\pgfpathcurveto{\pgfqpoint{1.690554in}{1.884373in}}{\pgfqpoint{1.693826in}{1.892273in}}{\pgfqpoint{1.693826in}{1.900510in}}%
\pgfpathcurveto{\pgfqpoint{1.693826in}{1.908746in}}{\pgfqpoint{1.690554in}{1.916646in}}{\pgfqpoint{1.684730in}{1.922470in}}%
\pgfpathcurveto{\pgfqpoint{1.678906in}{1.928294in}}{\pgfqpoint{1.671006in}{1.931566in}}{\pgfqpoint{1.662770in}{1.931566in}}%
\pgfpathcurveto{\pgfqpoint{1.654533in}{1.931566in}}{\pgfqpoint{1.646633in}{1.928294in}}{\pgfqpoint{1.640809in}{1.922470in}}%
\pgfpathcurveto{\pgfqpoint{1.634986in}{1.916646in}}{\pgfqpoint{1.631713in}{1.908746in}}{\pgfqpoint{1.631713in}{1.900510in}}%
\pgfpathcurveto{\pgfqpoint{1.631713in}{1.892273in}}{\pgfqpoint{1.634986in}{1.884373in}}{\pgfqpoint{1.640809in}{1.878549in}}%
\pgfpathcurveto{\pgfqpoint{1.646633in}{1.872725in}}{\pgfqpoint{1.654533in}{1.869453in}}{\pgfqpoint{1.662770in}{1.869453in}}%
\pgfpathclose%
\pgfusepath{stroke,fill}%
\end{pgfscope}%
\begin{pgfscope}%
\pgfpathrectangle{\pgfqpoint{0.100000in}{0.212622in}}{\pgfqpoint{3.696000in}{3.696000in}}%
\pgfusepath{clip}%
\pgfsetbuttcap%
\pgfsetroundjoin%
\definecolor{currentfill}{rgb}{0.121569,0.466667,0.705882}%
\pgfsetfillcolor{currentfill}%
\pgfsetfillopacity{0.370712}%
\pgfsetlinewidth{1.003750pt}%
\definecolor{currentstroke}{rgb}{0.121569,0.466667,0.705882}%
\pgfsetstrokecolor{currentstroke}%
\pgfsetstrokeopacity{0.370712}%
\pgfsetdash{}{0pt}%
\pgfpathmoveto{\pgfqpoint{1.661923in}{1.869502in}}%
\pgfpathcurveto{\pgfqpoint{1.670159in}{1.869502in}}{\pgfqpoint{1.678059in}{1.872775in}}{\pgfqpoint{1.683883in}{1.878599in}}%
\pgfpathcurveto{\pgfqpoint{1.689707in}{1.884423in}}{\pgfqpoint{1.692979in}{1.892323in}}{\pgfqpoint{1.692979in}{1.900559in}}%
\pgfpathcurveto{\pgfqpoint{1.692979in}{1.908795in}}{\pgfqpoint{1.689707in}{1.916695in}}{\pgfqpoint{1.683883in}{1.922519in}}%
\pgfpathcurveto{\pgfqpoint{1.678059in}{1.928343in}}{\pgfqpoint{1.670159in}{1.931615in}}{\pgfqpoint{1.661923in}{1.931615in}}%
\pgfpathcurveto{\pgfqpoint{1.653687in}{1.931615in}}{\pgfqpoint{1.645787in}{1.928343in}}{\pgfqpoint{1.639963in}{1.922519in}}%
\pgfpathcurveto{\pgfqpoint{1.634139in}{1.916695in}}{\pgfqpoint{1.630866in}{1.908795in}}{\pgfqpoint{1.630866in}{1.900559in}}%
\pgfpathcurveto{\pgfqpoint{1.630866in}{1.892323in}}{\pgfqpoint{1.634139in}{1.884423in}}{\pgfqpoint{1.639963in}{1.878599in}}%
\pgfpathcurveto{\pgfqpoint{1.645787in}{1.872775in}}{\pgfqpoint{1.653687in}{1.869502in}}{\pgfqpoint{1.661923in}{1.869502in}}%
\pgfpathclose%
\pgfusepath{stroke,fill}%
\end{pgfscope}%
\begin{pgfscope}%
\pgfpathrectangle{\pgfqpoint{0.100000in}{0.212622in}}{\pgfqpoint{3.696000in}{3.696000in}}%
\pgfusepath{clip}%
\pgfsetbuttcap%
\pgfsetroundjoin%
\definecolor{currentfill}{rgb}{0.121569,0.466667,0.705882}%
\pgfsetfillcolor{currentfill}%
\pgfsetfillopacity{0.370976}%
\pgfsetlinewidth{1.003750pt}%
\definecolor{currentstroke}{rgb}{0.121569,0.466667,0.705882}%
\pgfsetstrokecolor{currentstroke}%
\pgfsetstrokeopacity{0.370976}%
\pgfsetdash{}{0pt}%
\pgfpathmoveto{\pgfqpoint{1.661130in}{1.868438in}}%
\pgfpathcurveto{\pgfqpoint{1.669366in}{1.868438in}}{\pgfqpoint{1.677266in}{1.871710in}}{\pgfqpoint{1.683090in}{1.877534in}}%
\pgfpathcurveto{\pgfqpoint{1.688914in}{1.883358in}}{\pgfqpoint{1.692186in}{1.891258in}}{\pgfqpoint{1.692186in}{1.899495in}}%
\pgfpathcurveto{\pgfqpoint{1.692186in}{1.907731in}}{\pgfqpoint{1.688914in}{1.915631in}}{\pgfqpoint{1.683090in}{1.921455in}}%
\pgfpathcurveto{\pgfqpoint{1.677266in}{1.927279in}}{\pgfqpoint{1.669366in}{1.930551in}}{\pgfqpoint{1.661130in}{1.930551in}}%
\pgfpathcurveto{\pgfqpoint{1.652893in}{1.930551in}}{\pgfqpoint{1.644993in}{1.927279in}}{\pgfqpoint{1.639169in}{1.921455in}}%
\pgfpathcurveto{\pgfqpoint{1.633346in}{1.915631in}}{\pgfqpoint{1.630073in}{1.907731in}}{\pgfqpoint{1.630073in}{1.899495in}}%
\pgfpathcurveto{\pgfqpoint{1.630073in}{1.891258in}}{\pgfqpoint{1.633346in}{1.883358in}}{\pgfqpoint{1.639169in}{1.877534in}}%
\pgfpathcurveto{\pgfqpoint{1.644993in}{1.871710in}}{\pgfqpoint{1.652893in}{1.868438in}}{\pgfqpoint{1.661130in}{1.868438in}}%
\pgfpathclose%
\pgfusepath{stroke,fill}%
\end{pgfscope}%
\begin{pgfscope}%
\pgfpathrectangle{\pgfqpoint{0.100000in}{0.212622in}}{\pgfqpoint{3.696000in}{3.696000in}}%
\pgfusepath{clip}%
\pgfsetbuttcap%
\pgfsetroundjoin%
\definecolor{currentfill}{rgb}{0.121569,0.466667,0.705882}%
\pgfsetfillcolor{currentfill}%
\pgfsetfillopacity{0.371378}%
\pgfsetlinewidth{1.003750pt}%
\definecolor{currentstroke}{rgb}{0.121569,0.466667,0.705882}%
\pgfsetstrokecolor{currentstroke}%
\pgfsetstrokeopacity{0.371378}%
\pgfsetdash{}{0pt}%
\pgfpathmoveto{\pgfqpoint{1.658814in}{1.867113in}}%
\pgfpathcurveto{\pgfqpoint{1.667050in}{1.867113in}}{\pgfqpoint{1.674950in}{1.870386in}}{\pgfqpoint{1.680774in}{1.876210in}}%
\pgfpathcurveto{\pgfqpoint{1.686598in}{1.882033in}}{\pgfqpoint{1.689870in}{1.889934in}}{\pgfqpoint{1.689870in}{1.898170in}}%
\pgfpathcurveto{\pgfqpoint{1.689870in}{1.906406in}}{\pgfqpoint{1.686598in}{1.914306in}}{\pgfqpoint{1.680774in}{1.920130in}}%
\pgfpathcurveto{\pgfqpoint{1.674950in}{1.925954in}}{\pgfqpoint{1.667050in}{1.929226in}}{\pgfqpoint{1.658814in}{1.929226in}}%
\pgfpathcurveto{\pgfqpoint{1.650578in}{1.929226in}}{\pgfqpoint{1.642678in}{1.925954in}}{\pgfqpoint{1.636854in}{1.920130in}}%
\pgfpathcurveto{\pgfqpoint{1.631030in}{1.914306in}}{\pgfqpoint{1.627757in}{1.906406in}}{\pgfqpoint{1.627757in}{1.898170in}}%
\pgfpathcurveto{\pgfqpoint{1.627757in}{1.889934in}}{\pgfqpoint{1.631030in}{1.882033in}}{\pgfqpoint{1.636854in}{1.876210in}}%
\pgfpathcurveto{\pgfqpoint{1.642678in}{1.870386in}}{\pgfqpoint{1.650578in}{1.867113in}}{\pgfqpoint{1.658814in}{1.867113in}}%
\pgfpathclose%
\pgfusepath{stroke,fill}%
\end{pgfscope}%
\begin{pgfscope}%
\pgfpathrectangle{\pgfqpoint{0.100000in}{0.212622in}}{\pgfqpoint{3.696000in}{3.696000in}}%
\pgfusepath{clip}%
\pgfsetbuttcap%
\pgfsetroundjoin%
\definecolor{currentfill}{rgb}{0.121569,0.466667,0.705882}%
\pgfsetfillcolor{currentfill}%
\pgfsetfillopacity{0.371979}%
\pgfsetlinewidth{1.003750pt}%
\definecolor{currentstroke}{rgb}{0.121569,0.466667,0.705882}%
\pgfsetstrokecolor{currentstroke}%
\pgfsetstrokeopacity{0.371979}%
\pgfsetdash{}{0pt}%
\pgfpathmoveto{\pgfqpoint{1.654752in}{1.863555in}}%
\pgfpathcurveto{\pgfqpoint{1.662989in}{1.863555in}}{\pgfqpoint{1.670889in}{1.866827in}}{\pgfqpoint{1.676713in}{1.872651in}}%
\pgfpathcurveto{\pgfqpoint{1.682537in}{1.878475in}}{\pgfqpoint{1.685809in}{1.886375in}}{\pgfqpoint{1.685809in}{1.894611in}}%
\pgfpathcurveto{\pgfqpoint{1.685809in}{1.902848in}}{\pgfqpoint{1.682537in}{1.910748in}}{\pgfqpoint{1.676713in}{1.916572in}}%
\pgfpathcurveto{\pgfqpoint{1.670889in}{1.922396in}}{\pgfqpoint{1.662989in}{1.925668in}}{\pgfqpoint{1.654752in}{1.925668in}}%
\pgfpathcurveto{\pgfqpoint{1.646516in}{1.925668in}}{\pgfqpoint{1.638616in}{1.922396in}}{\pgfqpoint{1.632792in}{1.916572in}}%
\pgfpathcurveto{\pgfqpoint{1.626968in}{1.910748in}}{\pgfqpoint{1.623696in}{1.902848in}}{\pgfqpoint{1.623696in}{1.894611in}}%
\pgfpathcurveto{\pgfqpoint{1.623696in}{1.886375in}}{\pgfqpoint{1.626968in}{1.878475in}}{\pgfqpoint{1.632792in}{1.872651in}}%
\pgfpathcurveto{\pgfqpoint{1.638616in}{1.866827in}}{\pgfqpoint{1.646516in}{1.863555in}}{\pgfqpoint{1.654752in}{1.863555in}}%
\pgfpathclose%
\pgfusepath{stroke,fill}%
\end{pgfscope}%
\begin{pgfscope}%
\pgfpathrectangle{\pgfqpoint{0.100000in}{0.212622in}}{\pgfqpoint{3.696000in}{3.696000in}}%
\pgfusepath{clip}%
\pgfsetbuttcap%
\pgfsetroundjoin%
\definecolor{currentfill}{rgb}{0.121569,0.466667,0.705882}%
\pgfsetfillcolor{currentfill}%
\pgfsetfillopacity{0.373251}%
\pgfsetlinewidth{1.003750pt}%
\definecolor{currentstroke}{rgb}{0.121569,0.466667,0.705882}%
\pgfsetstrokecolor{currentstroke}%
\pgfsetstrokeopacity{0.373251}%
\pgfsetdash{}{0pt}%
\pgfpathmoveto{\pgfqpoint{1.647878in}{1.857579in}}%
\pgfpathcurveto{\pgfqpoint{1.656114in}{1.857579in}}{\pgfqpoint{1.664014in}{1.860851in}}{\pgfqpoint{1.669838in}{1.866675in}}%
\pgfpathcurveto{\pgfqpoint{1.675662in}{1.872499in}}{\pgfqpoint{1.678935in}{1.880399in}}{\pgfqpoint{1.678935in}{1.888635in}}%
\pgfpathcurveto{\pgfqpoint{1.678935in}{1.896871in}}{\pgfqpoint{1.675662in}{1.904771in}}{\pgfqpoint{1.669838in}{1.910595in}}%
\pgfpathcurveto{\pgfqpoint{1.664014in}{1.916419in}}{\pgfqpoint{1.656114in}{1.919692in}}{\pgfqpoint{1.647878in}{1.919692in}}%
\pgfpathcurveto{\pgfqpoint{1.639642in}{1.919692in}}{\pgfqpoint{1.631742in}{1.916419in}}{\pgfqpoint{1.625918in}{1.910595in}}%
\pgfpathcurveto{\pgfqpoint{1.620094in}{1.904771in}}{\pgfqpoint{1.616822in}{1.896871in}}{\pgfqpoint{1.616822in}{1.888635in}}%
\pgfpathcurveto{\pgfqpoint{1.616822in}{1.880399in}}{\pgfqpoint{1.620094in}{1.872499in}}{\pgfqpoint{1.625918in}{1.866675in}}%
\pgfpathcurveto{\pgfqpoint{1.631742in}{1.860851in}}{\pgfqpoint{1.639642in}{1.857579in}}{\pgfqpoint{1.647878in}{1.857579in}}%
\pgfpathclose%
\pgfusepath{stroke,fill}%
\end{pgfscope}%
\begin{pgfscope}%
\pgfpathrectangle{\pgfqpoint{0.100000in}{0.212622in}}{\pgfqpoint{3.696000in}{3.696000in}}%
\pgfusepath{clip}%
\pgfsetbuttcap%
\pgfsetroundjoin%
\definecolor{currentfill}{rgb}{0.121569,0.466667,0.705882}%
\pgfsetfillcolor{currentfill}%
\pgfsetfillopacity{0.374056}%
\pgfsetlinewidth{1.003750pt}%
\definecolor{currentstroke}{rgb}{0.121569,0.466667,0.705882}%
\pgfsetstrokecolor{currentstroke}%
\pgfsetstrokeopacity{0.374056}%
\pgfsetdash{}{0pt}%
\pgfpathmoveto{\pgfqpoint{1.640914in}{1.854289in}}%
\pgfpathcurveto{\pgfqpoint{1.649150in}{1.854289in}}{\pgfqpoint{1.657050in}{1.857561in}}{\pgfqpoint{1.662874in}{1.863385in}}%
\pgfpathcurveto{\pgfqpoint{1.668698in}{1.869209in}}{\pgfqpoint{1.671970in}{1.877109in}}{\pgfqpoint{1.671970in}{1.885345in}}%
\pgfpathcurveto{\pgfqpoint{1.671970in}{1.893581in}}{\pgfqpoint{1.668698in}{1.901481in}}{\pgfqpoint{1.662874in}{1.907305in}}%
\pgfpathcurveto{\pgfqpoint{1.657050in}{1.913129in}}{\pgfqpoint{1.649150in}{1.916402in}}{\pgfqpoint{1.640914in}{1.916402in}}%
\pgfpathcurveto{\pgfqpoint{1.632677in}{1.916402in}}{\pgfqpoint{1.624777in}{1.913129in}}{\pgfqpoint{1.618953in}{1.907305in}}%
\pgfpathcurveto{\pgfqpoint{1.613130in}{1.901481in}}{\pgfqpoint{1.609857in}{1.893581in}}{\pgfqpoint{1.609857in}{1.885345in}}%
\pgfpathcurveto{\pgfqpoint{1.609857in}{1.877109in}}{\pgfqpoint{1.613130in}{1.869209in}}{\pgfqpoint{1.618953in}{1.863385in}}%
\pgfpathcurveto{\pgfqpoint{1.624777in}{1.857561in}}{\pgfqpoint{1.632677in}{1.854289in}}{\pgfqpoint{1.640914in}{1.854289in}}%
\pgfpathclose%
\pgfusepath{stroke,fill}%
\end{pgfscope}%
\begin{pgfscope}%
\pgfpathrectangle{\pgfqpoint{0.100000in}{0.212622in}}{\pgfqpoint{3.696000in}{3.696000in}}%
\pgfusepath{clip}%
\pgfsetbuttcap%
\pgfsetroundjoin%
\definecolor{currentfill}{rgb}{0.121569,0.466667,0.705882}%
\pgfsetfillcolor{currentfill}%
\pgfsetfillopacity{0.374703}%
\pgfsetlinewidth{1.003750pt}%
\definecolor{currentstroke}{rgb}{0.121569,0.466667,0.705882}%
\pgfsetstrokecolor{currentstroke}%
\pgfsetstrokeopacity{0.374703}%
\pgfsetdash{}{0pt}%
\pgfpathmoveto{\pgfqpoint{1.628191in}{1.842314in}}%
\pgfpathcurveto{\pgfqpoint{1.636427in}{1.842314in}}{\pgfqpoint{1.644327in}{1.845586in}}{\pgfqpoint{1.650151in}{1.851410in}}%
\pgfpathcurveto{\pgfqpoint{1.655975in}{1.857234in}}{\pgfqpoint{1.659247in}{1.865134in}}{\pgfqpoint{1.659247in}{1.873371in}}%
\pgfpathcurveto{\pgfqpoint{1.659247in}{1.881607in}}{\pgfqpoint{1.655975in}{1.889507in}}{\pgfqpoint{1.650151in}{1.895331in}}%
\pgfpathcurveto{\pgfqpoint{1.644327in}{1.901155in}}{\pgfqpoint{1.636427in}{1.904427in}}{\pgfqpoint{1.628191in}{1.904427in}}%
\pgfpathcurveto{\pgfqpoint{1.619955in}{1.904427in}}{\pgfqpoint{1.612055in}{1.901155in}}{\pgfqpoint{1.606231in}{1.895331in}}%
\pgfpathcurveto{\pgfqpoint{1.600407in}{1.889507in}}{\pgfqpoint{1.597134in}{1.881607in}}{\pgfqpoint{1.597134in}{1.873371in}}%
\pgfpathcurveto{\pgfqpoint{1.597134in}{1.865134in}}{\pgfqpoint{1.600407in}{1.857234in}}{\pgfqpoint{1.606231in}{1.851410in}}%
\pgfpathcurveto{\pgfqpoint{1.612055in}{1.845586in}}{\pgfqpoint{1.619955in}{1.842314in}}{\pgfqpoint{1.628191in}{1.842314in}}%
\pgfpathclose%
\pgfusepath{stroke,fill}%
\end{pgfscope}%
\begin{pgfscope}%
\pgfpathrectangle{\pgfqpoint{0.100000in}{0.212622in}}{\pgfqpoint{3.696000in}{3.696000in}}%
\pgfusepath{clip}%
\pgfsetbuttcap%
\pgfsetroundjoin%
\definecolor{currentfill}{rgb}{0.121569,0.466667,0.705882}%
\pgfsetfillcolor{currentfill}%
\pgfsetfillopacity{0.374722}%
\pgfsetlinewidth{1.003750pt}%
\definecolor{currentstroke}{rgb}{0.121569,0.466667,0.705882}%
\pgfsetstrokecolor{currentstroke}%
\pgfsetstrokeopacity{0.374722}%
\pgfsetdash{}{0pt}%
\pgfpathmoveto{\pgfqpoint{1.637141in}{1.851769in}}%
\pgfpathcurveto{\pgfqpoint{1.645377in}{1.851769in}}{\pgfqpoint{1.653277in}{1.855041in}}{\pgfqpoint{1.659101in}{1.860865in}}%
\pgfpathcurveto{\pgfqpoint{1.664925in}{1.866689in}}{\pgfqpoint{1.668198in}{1.874589in}}{\pgfqpoint{1.668198in}{1.882826in}}%
\pgfpathcurveto{\pgfqpoint{1.668198in}{1.891062in}}{\pgfqpoint{1.664925in}{1.898962in}}{\pgfqpoint{1.659101in}{1.904786in}}%
\pgfpathcurveto{\pgfqpoint{1.653277in}{1.910610in}}{\pgfqpoint{1.645377in}{1.913882in}}{\pgfqpoint{1.637141in}{1.913882in}}%
\pgfpathcurveto{\pgfqpoint{1.628905in}{1.913882in}}{\pgfqpoint{1.621005in}{1.910610in}}{\pgfqpoint{1.615181in}{1.904786in}}%
\pgfpathcurveto{\pgfqpoint{1.609357in}{1.898962in}}{\pgfqpoint{1.606085in}{1.891062in}}{\pgfqpoint{1.606085in}{1.882826in}}%
\pgfpathcurveto{\pgfqpoint{1.606085in}{1.874589in}}{\pgfqpoint{1.609357in}{1.866689in}}{\pgfqpoint{1.615181in}{1.860865in}}%
\pgfpathcurveto{\pgfqpoint{1.621005in}{1.855041in}}{\pgfqpoint{1.628905in}{1.851769in}}{\pgfqpoint{1.637141in}{1.851769in}}%
\pgfpathclose%
\pgfusepath{stroke,fill}%
\end{pgfscope}%
\begin{pgfscope}%
\pgfpathrectangle{\pgfqpoint{0.100000in}{0.212622in}}{\pgfqpoint{3.696000in}{3.696000in}}%
\pgfusepath{clip}%
\pgfsetbuttcap%
\pgfsetroundjoin%
\definecolor{currentfill}{rgb}{0.121569,0.466667,0.705882}%
\pgfsetfillcolor{currentfill}%
\pgfsetfillopacity{0.375371}%
\pgfsetlinewidth{1.003750pt}%
\definecolor{currentstroke}{rgb}{0.121569,0.466667,0.705882}%
\pgfsetstrokecolor{currentstroke}%
\pgfsetstrokeopacity{0.375371}%
\pgfsetdash{}{0pt}%
\pgfpathmoveto{\pgfqpoint{1.593935in}{2.092399in}}%
\pgfpathcurveto{\pgfqpoint{1.602171in}{2.092399in}}{\pgfqpoint{1.610071in}{2.095671in}}{\pgfqpoint{1.615895in}{2.101495in}}%
\pgfpathcurveto{\pgfqpoint{1.621719in}{2.107319in}}{\pgfqpoint{1.624991in}{2.115219in}}{\pgfqpoint{1.624991in}{2.123456in}}%
\pgfpathcurveto{\pgfqpoint{1.624991in}{2.131692in}}{\pgfqpoint{1.621719in}{2.139592in}}{\pgfqpoint{1.615895in}{2.145416in}}%
\pgfpathcurveto{\pgfqpoint{1.610071in}{2.151240in}}{\pgfqpoint{1.602171in}{2.154512in}}{\pgfqpoint{1.593935in}{2.154512in}}%
\pgfpathcurveto{\pgfqpoint{1.585698in}{2.154512in}}{\pgfqpoint{1.577798in}{2.151240in}}{\pgfqpoint{1.571974in}{2.145416in}}%
\pgfpathcurveto{\pgfqpoint{1.566150in}{2.139592in}}{\pgfqpoint{1.562878in}{2.131692in}}{\pgfqpoint{1.562878in}{2.123456in}}%
\pgfpathcurveto{\pgfqpoint{1.562878in}{2.115219in}}{\pgfqpoint{1.566150in}{2.107319in}}{\pgfqpoint{1.571974in}{2.101495in}}%
\pgfpathcurveto{\pgfqpoint{1.577798in}{2.095671in}}{\pgfqpoint{1.585698in}{2.092399in}}{\pgfqpoint{1.593935in}{2.092399in}}%
\pgfpathclose%
\pgfusepath{stroke,fill}%
\end{pgfscope}%
\begin{pgfscope}%
\pgfpathrectangle{\pgfqpoint{0.100000in}{0.212622in}}{\pgfqpoint{3.696000in}{3.696000in}}%
\pgfusepath{clip}%
\pgfsetbuttcap%
\pgfsetroundjoin%
\definecolor{currentfill}{rgb}{0.121569,0.466667,0.705882}%
\pgfsetfillcolor{currentfill}%
\pgfsetfillopacity{0.376170}%
\pgfsetlinewidth{1.003750pt}%
\definecolor{currentstroke}{rgb}{0.121569,0.466667,0.705882}%
\pgfsetstrokecolor{currentstroke}%
\pgfsetstrokeopacity{0.376170}%
\pgfsetdash{}{0pt}%
\pgfpathmoveto{\pgfqpoint{1.626915in}{1.839856in}}%
\pgfpathcurveto{\pgfqpoint{1.635151in}{1.839856in}}{\pgfqpoint{1.643051in}{1.843129in}}{\pgfqpoint{1.648875in}{1.848952in}}%
\pgfpathcurveto{\pgfqpoint{1.654699in}{1.854776in}}{\pgfqpoint{1.657971in}{1.862676in}}{\pgfqpoint{1.657971in}{1.870913in}}%
\pgfpathcurveto{\pgfqpoint{1.657971in}{1.879149in}}{\pgfqpoint{1.654699in}{1.887049in}}{\pgfqpoint{1.648875in}{1.892873in}}%
\pgfpathcurveto{\pgfqpoint{1.643051in}{1.898697in}}{\pgfqpoint{1.635151in}{1.901969in}}{\pgfqpoint{1.626915in}{1.901969in}}%
\pgfpathcurveto{\pgfqpoint{1.618679in}{1.901969in}}{\pgfqpoint{1.610778in}{1.898697in}}{\pgfqpoint{1.604955in}{1.892873in}}%
\pgfpathcurveto{\pgfqpoint{1.599131in}{1.887049in}}{\pgfqpoint{1.595858in}{1.879149in}}{\pgfqpoint{1.595858in}{1.870913in}}%
\pgfpathcurveto{\pgfqpoint{1.595858in}{1.862676in}}{\pgfqpoint{1.599131in}{1.854776in}}{\pgfqpoint{1.604955in}{1.848952in}}%
\pgfpathcurveto{\pgfqpoint{1.610778in}{1.843129in}}{\pgfqpoint{1.618679in}{1.839856in}}{\pgfqpoint{1.626915in}{1.839856in}}%
\pgfpathclose%
\pgfusepath{stroke,fill}%
\end{pgfscope}%
\begin{pgfscope}%
\pgfpathrectangle{\pgfqpoint{0.100000in}{0.212622in}}{\pgfqpoint{3.696000in}{3.696000in}}%
\pgfusepath{clip}%
\pgfsetbuttcap%
\pgfsetroundjoin%
\definecolor{currentfill}{rgb}{0.121569,0.466667,0.705882}%
\pgfsetfillcolor{currentfill}%
\pgfsetfillopacity{0.376974}%
\pgfsetlinewidth{1.003750pt}%
\definecolor{currentstroke}{rgb}{0.121569,0.466667,0.705882}%
\pgfsetstrokecolor{currentstroke}%
\pgfsetstrokeopacity{0.376974}%
\pgfsetdash{}{0pt}%
\pgfpathmoveto{\pgfqpoint{1.622521in}{1.840743in}}%
\pgfpathcurveto{\pgfqpoint{1.630757in}{1.840743in}}{\pgfqpoint{1.638657in}{1.844015in}}{\pgfqpoint{1.644481in}{1.849839in}}%
\pgfpathcurveto{\pgfqpoint{1.650305in}{1.855663in}}{\pgfqpoint{1.653577in}{1.863563in}}{\pgfqpoint{1.653577in}{1.871799in}}%
\pgfpathcurveto{\pgfqpoint{1.653577in}{1.880036in}}{\pgfqpoint{1.650305in}{1.887936in}}{\pgfqpoint{1.644481in}{1.893760in}}%
\pgfpathcurveto{\pgfqpoint{1.638657in}{1.899584in}}{\pgfqpoint{1.630757in}{1.902856in}}{\pgfqpoint{1.622521in}{1.902856in}}%
\pgfpathcurveto{\pgfqpoint{1.614284in}{1.902856in}}{\pgfqpoint{1.606384in}{1.899584in}}{\pgfqpoint{1.600560in}{1.893760in}}%
\pgfpathcurveto{\pgfqpoint{1.594736in}{1.887936in}}{\pgfqpoint{1.591464in}{1.880036in}}{\pgfqpoint{1.591464in}{1.871799in}}%
\pgfpathcurveto{\pgfqpoint{1.591464in}{1.863563in}}{\pgfqpoint{1.594736in}{1.855663in}}{\pgfqpoint{1.600560in}{1.849839in}}%
\pgfpathcurveto{\pgfqpoint{1.606384in}{1.844015in}}{\pgfqpoint{1.614284in}{1.840743in}}{\pgfqpoint{1.622521in}{1.840743in}}%
\pgfpathclose%
\pgfusepath{stroke,fill}%
\end{pgfscope}%
\begin{pgfscope}%
\pgfpathrectangle{\pgfqpoint{0.100000in}{0.212622in}}{\pgfqpoint{3.696000in}{3.696000in}}%
\pgfusepath{clip}%
\pgfsetbuttcap%
\pgfsetroundjoin%
\definecolor{currentfill}{rgb}{0.121569,0.466667,0.705882}%
\pgfsetfillcolor{currentfill}%
\pgfsetfillopacity{0.377534}%
\pgfsetlinewidth{1.003750pt}%
\definecolor{currentstroke}{rgb}{0.121569,0.466667,0.705882}%
\pgfsetstrokecolor{currentstroke}%
\pgfsetstrokeopacity{0.377534}%
\pgfsetdash{}{0pt}%
\pgfpathmoveto{\pgfqpoint{1.621564in}{1.839471in}}%
\pgfpathcurveto{\pgfqpoint{1.629800in}{1.839471in}}{\pgfqpoint{1.637700in}{1.842743in}}{\pgfqpoint{1.643524in}{1.848567in}}%
\pgfpathcurveto{\pgfqpoint{1.649348in}{1.854391in}}{\pgfqpoint{1.652621in}{1.862291in}}{\pgfqpoint{1.652621in}{1.870528in}}%
\pgfpathcurveto{\pgfqpoint{1.652621in}{1.878764in}}{\pgfqpoint{1.649348in}{1.886664in}}{\pgfqpoint{1.643524in}{1.892488in}}%
\pgfpathcurveto{\pgfqpoint{1.637700in}{1.898312in}}{\pgfqpoint{1.629800in}{1.901584in}}{\pgfqpoint{1.621564in}{1.901584in}}%
\pgfpathcurveto{\pgfqpoint{1.613328in}{1.901584in}}{\pgfqpoint{1.605428in}{1.898312in}}{\pgfqpoint{1.599604in}{1.892488in}}%
\pgfpathcurveto{\pgfqpoint{1.593780in}{1.886664in}}{\pgfqpoint{1.590508in}{1.878764in}}{\pgfqpoint{1.590508in}{1.870528in}}%
\pgfpathcurveto{\pgfqpoint{1.590508in}{1.862291in}}{\pgfqpoint{1.593780in}{1.854391in}}{\pgfqpoint{1.599604in}{1.848567in}}%
\pgfpathcurveto{\pgfqpoint{1.605428in}{1.842743in}}{\pgfqpoint{1.613328in}{1.839471in}}{\pgfqpoint{1.621564in}{1.839471in}}%
\pgfpathclose%
\pgfusepath{stroke,fill}%
\end{pgfscope}%
\begin{pgfscope}%
\pgfpathrectangle{\pgfqpoint{0.100000in}{0.212622in}}{\pgfqpoint{3.696000in}{3.696000in}}%
\pgfusepath{clip}%
\pgfsetbuttcap%
\pgfsetroundjoin%
\definecolor{currentfill}{rgb}{0.121569,0.466667,0.705882}%
\pgfsetfillcolor{currentfill}%
\pgfsetfillopacity{0.378321}%
\pgfsetlinewidth{1.003750pt}%
\definecolor{currentstroke}{rgb}{0.121569,0.466667,0.705882}%
\pgfsetstrokecolor{currentstroke}%
\pgfsetstrokeopacity{0.378321}%
\pgfsetdash{}{0pt}%
\pgfpathmoveto{\pgfqpoint{1.616826in}{1.839680in}}%
\pgfpathcurveto{\pgfqpoint{1.625062in}{1.839680in}}{\pgfqpoint{1.632962in}{1.842953in}}{\pgfqpoint{1.638786in}{1.848777in}}%
\pgfpathcurveto{\pgfqpoint{1.644610in}{1.854600in}}{\pgfqpoint{1.647883in}{1.862501in}}{\pgfqpoint{1.647883in}{1.870737in}}%
\pgfpathcurveto{\pgfqpoint{1.647883in}{1.878973in}}{\pgfqpoint{1.644610in}{1.886873in}}{\pgfqpoint{1.638786in}{1.892697in}}%
\pgfpathcurveto{\pgfqpoint{1.632962in}{1.898521in}}{\pgfqpoint{1.625062in}{1.901793in}}{\pgfqpoint{1.616826in}{1.901793in}}%
\pgfpathcurveto{\pgfqpoint{1.608590in}{1.901793in}}{\pgfqpoint{1.600690in}{1.898521in}}{\pgfqpoint{1.594866in}{1.892697in}}%
\pgfpathcurveto{\pgfqpoint{1.589042in}{1.886873in}}{\pgfqpoint{1.585770in}{1.878973in}}{\pgfqpoint{1.585770in}{1.870737in}}%
\pgfpathcurveto{\pgfqpoint{1.585770in}{1.862501in}}{\pgfqpoint{1.589042in}{1.854600in}}{\pgfqpoint{1.594866in}{1.848777in}}%
\pgfpathcurveto{\pgfqpoint{1.600690in}{1.842953in}}{\pgfqpoint{1.608590in}{1.839680in}}{\pgfqpoint{1.616826in}{1.839680in}}%
\pgfpathclose%
\pgfusepath{stroke,fill}%
\end{pgfscope}%
\begin{pgfscope}%
\pgfpathrectangle{\pgfqpoint{0.100000in}{0.212622in}}{\pgfqpoint{3.696000in}{3.696000in}}%
\pgfusepath{clip}%
\pgfsetbuttcap%
\pgfsetroundjoin%
\definecolor{currentfill}{rgb}{0.121569,0.466667,0.705882}%
\pgfsetfillcolor{currentfill}%
\pgfsetfillopacity{0.378828}%
\pgfsetlinewidth{1.003750pt}%
\definecolor{currentstroke}{rgb}{0.121569,0.466667,0.705882}%
\pgfsetstrokecolor{currentstroke}%
\pgfsetstrokeopacity{0.378828}%
\pgfsetdash{}{0pt}%
\pgfpathmoveto{\pgfqpoint{1.615786in}{1.838367in}}%
\pgfpathcurveto{\pgfqpoint{1.624022in}{1.838367in}}{\pgfqpoint{1.631922in}{1.841640in}}{\pgfqpoint{1.637746in}{1.847464in}}%
\pgfpathcurveto{\pgfqpoint{1.643570in}{1.853288in}}{\pgfqpoint{1.646842in}{1.861188in}}{\pgfqpoint{1.646842in}{1.869424in}}%
\pgfpathcurveto{\pgfqpoint{1.646842in}{1.877660in}}{\pgfqpoint{1.643570in}{1.885560in}}{\pgfqpoint{1.637746in}{1.891384in}}%
\pgfpathcurveto{\pgfqpoint{1.631922in}{1.897208in}}{\pgfqpoint{1.624022in}{1.900480in}}{\pgfqpoint{1.615786in}{1.900480in}}%
\pgfpathcurveto{\pgfqpoint{1.607549in}{1.900480in}}{\pgfqpoint{1.599649in}{1.897208in}}{\pgfqpoint{1.593825in}{1.891384in}}%
\pgfpathcurveto{\pgfqpoint{1.588001in}{1.885560in}}{\pgfqpoint{1.584729in}{1.877660in}}{\pgfqpoint{1.584729in}{1.869424in}}%
\pgfpathcurveto{\pgfqpoint{1.584729in}{1.861188in}}{\pgfqpoint{1.588001in}{1.853288in}}{\pgfqpoint{1.593825in}{1.847464in}}%
\pgfpathcurveto{\pgfqpoint{1.599649in}{1.841640in}}{\pgfqpoint{1.607549in}{1.838367in}}{\pgfqpoint{1.615786in}{1.838367in}}%
\pgfpathclose%
\pgfusepath{stroke,fill}%
\end{pgfscope}%
\begin{pgfscope}%
\pgfpathrectangle{\pgfqpoint{0.100000in}{0.212622in}}{\pgfqpoint{3.696000in}{3.696000in}}%
\pgfusepath{clip}%
\pgfsetbuttcap%
\pgfsetroundjoin%
\definecolor{currentfill}{rgb}{0.121569,0.466667,0.705882}%
\pgfsetfillcolor{currentfill}%
\pgfsetfillopacity{0.379438}%
\pgfsetlinewidth{1.003750pt}%
\definecolor{currentstroke}{rgb}{0.121569,0.466667,0.705882}%
\pgfsetstrokecolor{currentstroke}%
\pgfsetstrokeopacity{0.379438}%
\pgfsetdash{}{0pt}%
\pgfpathmoveto{\pgfqpoint{1.611285in}{1.837492in}}%
\pgfpathcurveto{\pgfqpoint{1.619521in}{1.837492in}}{\pgfqpoint{1.627421in}{1.840765in}}{\pgfqpoint{1.633245in}{1.846589in}}%
\pgfpathcurveto{\pgfqpoint{1.639069in}{1.852413in}}{\pgfqpoint{1.642342in}{1.860313in}}{\pgfqpoint{1.642342in}{1.868549in}}%
\pgfpathcurveto{\pgfqpoint{1.642342in}{1.876785in}}{\pgfqpoint{1.639069in}{1.884685in}}{\pgfqpoint{1.633245in}{1.890509in}}%
\pgfpathcurveto{\pgfqpoint{1.627421in}{1.896333in}}{\pgfqpoint{1.619521in}{1.899605in}}{\pgfqpoint{1.611285in}{1.899605in}}%
\pgfpathcurveto{\pgfqpoint{1.603049in}{1.899605in}}{\pgfqpoint{1.595149in}{1.896333in}}{\pgfqpoint{1.589325in}{1.890509in}}%
\pgfpathcurveto{\pgfqpoint{1.583501in}{1.884685in}}{\pgfqpoint{1.580229in}{1.876785in}}{\pgfqpoint{1.580229in}{1.868549in}}%
\pgfpathcurveto{\pgfqpoint{1.580229in}{1.860313in}}{\pgfqpoint{1.583501in}{1.852413in}}{\pgfqpoint{1.589325in}{1.846589in}}%
\pgfpathcurveto{\pgfqpoint{1.595149in}{1.840765in}}{\pgfqpoint{1.603049in}{1.837492in}}{\pgfqpoint{1.611285in}{1.837492in}}%
\pgfpathclose%
\pgfusepath{stroke,fill}%
\end{pgfscope}%
\begin{pgfscope}%
\pgfpathrectangle{\pgfqpoint{0.100000in}{0.212622in}}{\pgfqpoint{3.696000in}{3.696000in}}%
\pgfusepath{clip}%
\pgfsetbuttcap%
\pgfsetroundjoin%
\definecolor{currentfill}{rgb}{0.121569,0.466667,0.705882}%
\pgfsetfillcolor{currentfill}%
\pgfsetfillopacity{0.379960}%
\pgfsetlinewidth{1.003750pt}%
\definecolor{currentstroke}{rgb}{0.121569,0.466667,0.705882}%
\pgfsetstrokecolor{currentstroke}%
\pgfsetstrokeopacity{0.379960}%
\pgfsetdash{}{0pt}%
\pgfpathmoveto{\pgfqpoint{1.610243in}{1.836188in}}%
\pgfpathcurveto{\pgfqpoint{1.618479in}{1.836188in}}{\pgfqpoint{1.626379in}{1.839460in}}{\pgfqpoint{1.632203in}{1.845284in}}%
\pgfpathcurveto{\pgfqpoint{1.638027in}{1.851108in}}{\pgfqpoint{1.641299in}{1.859008in}}{\pgfqpoint{1.641299in}{1.867244in}}%
\pgfpathcurveto{\pgfqpoint{1.641299in}{1.875480in}}{\pgfqpoint{1.638027in}{1.883380in}}{\pgfqpoint{1.632203in}{1.889204in}}%
\pgfpathcurveto{\pgfqpoint{1.626379in}{1.895028in}}{\pgfqpoint{1.618479in}{1.898301in}}{\pgfqpoint{1.610243in}{1.898301in}}%
\pgfpathcurveto{\pgfqpoint{1.602006in}{1.898301in}}{\pgfqpoint{1.594106in}{1.895028in}}{\pgfqpoint{1.588282in}{1.889204in}}%
\pgfpathcurveto{\pgfqpoint{1.582459in}{1.883380in}}{\pgfqpoint{1.579186in}{1.875480in}}{\pgfqpoint{1.579186in}{1.867244in}}%
\pgfpathcurveto{\pgfqpoint{1.579186in}{1.859008in}}{\pgfqpoint{1.582459in}{1.851108in}}{\pgfqpoint{1.588282in}{1.845284in}}%
\pgfpathcurveto{\pgfqpoint{1.594106in}{1.839460in}}{\pgfqpoint{1.602006in}{1.836188in}}{\pgfqpoint{1.610243in}{1.836188in}}%
\pgfpathclose%
\pgfusepath{stroke,fill}%
\end{pgfscope}%
\begin{pgfscope}%
\pgfpathrectangle{\pgfqpoint{0.100000in}{0.212622in}}{\pgfqpoint{3.696000in}{3.696000in}}%
\pgfusepath{clip}%
\pgfsetbuttcap%
\pgfsetroundjoin%
\definecolor{currentfill}{rgb}{0.121569,0.466667,0.705882}%
\pgfsetfillcolor{currentfill}%
\pgfsetfillopacity{0.380654}%
\pgfsetlinewidth{1.003750pt}%
\definecolor{currentstroke}{rgb}{0.121569,0.466667,0.705882}%
\pgfsetstrokecolor{currentstroke}%
\pgfsetstrokeopacity{0.380654}%
\pgfsetdash{}{0pt}%
\pgfpathmoveto{\pgfqpoint{1.605565in}{1.836001in}}%
\pgfpathcurveto{\pgfqpoint{1.613802in}{1.836001in}}{\pgfqpoint{1.621702in}{1.839273in}}{\pgfqpoint{1.627526in}{1.845097in}}%
\pgfpathcurveto{\pgfqpoint{1.633349in}{1.850921in}}{\pgfqpoint{1.636622in}{1.858821in}}{\pgfqpoint{1.636622in}{1.867058in}}%
\pgfpathcurveto{\pgfqpoint{1.636622in}{1.875294in}}{\pgfqpoint{1.633349in}{1.883194in}}{\pgfqpoint{1.627526in}{1.889018in}}%
\pgfpathcurveto{\pgfqpoint{1.621702in}{1.894842in}}{\pgfqpoint{1.613802in}{1.898114in}}{\pgfqpoint{1.605565in}{1.898114in}}%
\pgfpathcurveto{\pgfqpoint{1.597329in}{1.898114in}}{\pgfqpoint{1.589429in}{1.894842in}}{\pgfqpoint{1.583605in}{1.889018in}}%
\pgfpathcurveto{\pgfqpoint{1.577781in}{1.883194in}}{\pgfqpoint{1.574509in}{1.875294in}}{\pgfqpoint{1.574509in}{1.867058in}}%
\pgfpathcurveto{\pgfqpoint{1.574509in}{1.858821in}}{\pgfqpoint{1.577781in}{1.850921in}}{\pgfqpoint{1.583605in}{1.845097in}}%
\pgfpathcurveto{\pgfqpoint{1.589429in}{1.839273in}}{\pgfqpoint{1.597329in}{1.836001in}}{\pgfqpoint{1.605565in}{1.836001in}}%
\pgfpathclose%
\pgfusepath{stroke,fill}%
\end{pgfscope}%
\begin{pgfscope}%
\pgfpathrectangle{\pgfqpoint{0.100000in}{0.212622in}}{\pgfqpoint{3.696000in}{3.696000in}}%
\pgfusepath{clip}%
\pgfsetbuttcap%
\pgfsetroundjoin%
\definecolor{currentfill}{rgb}{0.121569,0.466667,0.705882}%
\pgfsetfillcolor{currentfill}%
\pgfsetfillopacity{0.382421}%
\pgfsetlinewidth{1.003750pt}%
\definecolor{currentstroke}{rgb}{0.121569,0.466667,0.705882}%
\pgfsetstrokecolor{currentstroke}%
\pgfsetstrokeopacity{0.382421}%
\pgfsetdash{}{0pt}%
\pgfpathmoveto{\pgfqpoint{1.601877in}{1.832195in}}%
\pgfpathcurveto{\pgfqpoint{1.610114in}{1.832195in}}{\pgfqpoint{1.618014in}{1.835468in}}{\pgfqpoint{1.623838in}{1.841291in}}%
\pgfpathcurveto{\pgfqpoint{1.629662in}{1.847115in}}{\pgfqpoint{1.632934in}{1.855015in}}{\pgfqpoint{1.632934in}{1.863252in}}%
\pgfpathcurveto{\pgfqpoint{1.632934in}{1.871488in}}{\pgfqpoint{1.629662in}{1.879388in}}{\pgfqpoint{1.623838in}{1.885212in}}%
\pgfpathcurveto{\pgfqpoint{1.618014in}{1.891036in}}{\pgfqpoint{1.610114in}{1.894308in}}{\pgfqpoint{1.601877in}{1.894308in}}%
\pgfpathcurveto{\pgfqpoint{1.593641in}{1.894308in}}{\pgfqpoint{1.585741in}{1.891036in}}{\pgfqpoint{1.579917in}{1.885212in}}%
\pgfpathcurveto{\pgfqpoint{1.574093in}{1.879388in}}{\pgfqpoint{1.570821in}{1.871488in}}{\pgfqpoint{1.570821in}{1.863252in}}%
\pgfpathcurveto{\pgfqpoint{1.570821in}{1.855015in}}{\pgfqpoint{1.574093in}{1.847115in}}{\pgfqpoint{1.579917in}{1.841291in}}%
\pgfpathcurveto{\pgfqpoint{1.585741in}{1.835468in}}{\pgfqpoint{1.593641in}{1.832195in}}{\pgfqpoint{1.601877in}{1.832195in}}%
\pgfpathclose%
\pgfusepath{stroke,fill}%
\end{pgfscope}%
\begin{pgfscope}%
\pgfpathrectangle{\pgfqpoint{0.100000in}{0.212622in}}{\pgfqpoint{3.696000in}{3.696000in}}%
\pgfusepath{clip}%
\pgfsetbuttcap%
\pgfsetroundjoin%
\definecolor{currentfill}{rgb}{0.121569,0.466667,0.705882}%
\pgfsetfillcolor{currentfill}%
\pgfsetfillopacity{0.382567}%
\pgfsetlinewidth{1.003750pt}%
\definecolor{currentstroke}{rgb}{0.121569,0.466667,0.705882}%
\pgfsetstrokecolor{currentstroke}%
\pgfsetstrokeopacity{0.382567}%
\pgfsetdash{}{0pt}%
\pgfpathmoveto{\pgfqpoint{1.602976in}{2.074426in}}%
\pgfpathcurveto{\pgfqpoint{1.611212in}{2.074426in}}{\pgfqpoint{1.619113in}{2.077699in}}{\pgfqpoint{1.624936in}{2.083522in}}%
\pgfpathcurveto{\pgfqpoint{1.630760in}{2.089346in}}{\pgfqpoint{1.634033in}{2.097246in}}{\pgfqpoint{1.634033in}{2.105483in}}%
\pgfpathcurveto{\pgfqpoint{1.634033in}{2.113719in}}{\pgfqpoint{1.630760in}{2.121619in}}{\pgfqpoint{1.624936in}{2.127443in}}%
\pgfpathcurveto{\pgfqpoint{1.619113in}{2.133267in}}{\pgfqpoint{1.611212in}{2.136539in}}{\pgfqpoint{1.602976in}{2.136539in}}%
\pgfpathcurveto{\pgfqpoint{1.594740in}{2.136539in}}{\pgfqpoint{1.586840in}{2.133267in}}{\pgfqpoint{1.581016in}{2.127443in}}%
\pgfpathcurveto{\pgfqpoint{1.575192in}{2.121619in}}{\pgfqpoint{1.571920in}{2.113719in}}{\pgfqpoint{1.571920in}{2.105483in}}%
\pgfpathcurveto{\pgfqpoint{1.571920in}{2.097246in}}{\pgfqpoint{1.575192in}{2.089346in}}{\pgfqpoint{1.581016in}{2.083522in}}%
\pgfpathcurveto{\pgfqpoint{1.586840in}{2.077699in}}{\pgfqpoint{1.594740in}{2.074426in}}{\pgfqpoint{1.602976in}{2.074426in}}%
\pgfpathclose%
\pgfusepath{stroke,fill}%
\end{pgfscope}%
\begin{pgfscope}%
\pgfpathrectangle{\pgfqpoint{0.100000in}{0.212622in}}{\pgfqpoint{3.696000in}{3.696000in}}%
\pgfusepath{clip}%
\pgfsetbuttcap%
\pgfsetroundjoin%
\definecolor{currentfill}{rgb}{0.121569,0.466667,0.705882}%
\pgfsetfillcolor{currentfill}%
\pgfsetfillopacity{0.384662}%
\pgfsetlinewidth{1.003750pt}%
\definecolor{currentstroke}{rgb}{0.121569,0.466667,0.705882}%
\pgfsetstrokecolor{currentstroke}%
\pgfsetstrokeopacity{0.384662}%
\pgfsetdash{}{0pt}%
\pgfpathmoveto{\pgfqpoint{1.587261in}{1.829520in}}%
\pgfpathcurveto{\pgfqpoint{1.595498in}{1.829520in}}{\pgfqpoint{1.603398in}{1.832792in}}{\pgfqpoint{1.609222in}{1.838616in}}%
\pgfpathcurveto{\pgfqpoint{1.615046in}{1.844440in}}{\pgfqpoint{1.618318in}{1.852340in}}{\pgfqpoint{1.618318in}{1.860576in}}%
\pgfpathcurveto{\pgfqpoint{1.618318in}{1.868813in}}{\pgfqpoint{1.615046in}{1.876713in}}{\pgfqpoint{1.609222in}{1.882537in}}%
\pgfpathcurveto{\pgfqpoint{1.603398in}{1.888361in}}{\pgfqpoint{1.595498in}{1.891633in}}{\pgfqpoint{1.587261in}{1.891633in}}%
\pgfpathcurveto{\pgfqpoint{1.579025in}{1.891633in}}{\pgfqpoint{1.571125in}{1.888361in}}{\pgfqpoint{1.565301in}{1.882537in}}%
\pgfpathcurveto{\pgfqpoint{1.559477in}{1.876713in}}{\pgfqpoint{1.556205in}{1.868813in}}{\pgfqpoint{1.556205in}{1.860576in}}%
\pgfpathcurveto{\pgfqpoint{1.556205in}{1.852340in}}{\pgfqpoint{1.559477in}{1.844440in}}{\pgfqpoint{1.565301in}{1.838616in}}%
\pgfpathcurveto{\pgfqpoint{1.571125in}{1.832792in}}{\pgfqpoint{1.579025in}{1.829520in}}{\pgfqpoint{1.587261in}{1.829520in}}%
\pgfpathclose%
\pgfusepath{stroke,fill}%
\end{pgfscope}%
\begin{pgfscope}%
\pgfpathrectangle{\pgfqpoint{0.100000in}{0.212622in}}{\pgfqpoint{3.696000in}{3.696000in}}%
\pgfusepath{clip}%
\pgfsetbuttcap%
\pgfsetroundjoin%
\definecolor{currentfill}{rgb}{0.121569,0.466667,0.705882}%
\pgfsetfillcolor{currentfill}%
\pgfsetfillopacity{0.386708}%
\pgfsetlinewidth{1.003750pt}%
\definecolor{currentstroke}{rgb}{0.121569,0.466667,0.705882}%
\pgfsetstrokecolor{currentstroke}%
\pgfsetstrokeopacity{0.386708}%
\pgfsetdash{}{0pt}%
\pgfpathmoveto{\pgfqpoint{1.580613in}{1.821573in}}%
\pgfpathcurveto{\pgfqpoint{1.588849in}{1.821573in}}{\pgfqpoint{1.596749in}{1.824845in}}{\pgfqpoint{1.602573in}{1.830669in}}%
\pgfpathcurveto{\pgfqpoint{1.608397in}{1.836493in}}{\pgfqpoint{1.611669in}{1.844393in}}{\pgfqpoint{1.611669in}{1.852629in}}%
\pgfpathcurveto{\pgfqpoint{1.611669in}{1.860865in}}{\pgfqpoint{1.608397in}{1.868765in}}{\pgfqpoint{1.602573in}{1.874589in}}%
\pgfpathcurveto{\pgfqpoint{1.596749in}{1.880413in}}{\pgfqpoint{1.588849in}{1.883686in}}{\pgfqpoint{1.580613in}{1.883686in}}%
\pgfpathcurveto{\pgfqpoint{1.572376in}{1.883686in}}{\pgfqpoint{1.564476in}{1.880413in}}{\pgfqpoint{1.558652in}{1.874589in}}%
\pgfpathcurveto{\pgfqpoint{1.552829in}{1.868765in}}{\pgfqpoint{1.549556in}{1.860865in}}{\pgfqpoint{1.549556in}{1.852629in}}%
\pgfpathcurveto{\pgfqpoint{1.549556in}{1.844393in}}{\pgfqpoint{1.552829in}{1.836493in}}{\pgfqpoint{1.558652in}{1.830669in}}%
\pgfpathcurveto{\pgfqpoint{1.564476in}{1.824845in}}{\pgfqpoint{1.572376in}{1.821573in}}{\pgfqpoint{1.580613in}{1.821573in}}%
\pgfpathclose%
\pgfusepath{stroke,fill}%
\end{pgfscope}%
\begin{pgfscope}%
\pgfpathrectangle{\pgfqpoint{0.100000in}{0.212622in}}{\pgfqpoint{3.696000in}{3.696000in}}%
\pgfusepath{clip}%
\pgfsetbuttcap%
\pgfsetroundjoin%
\definecolor{currentfill}{rgb}{0.121569,0.466667,0.705882}%
\pgfsetfillcolor{currentfill}%
\pgfsetfillopacity{0.390067}%
\pgfsetlinewidth{1.003750pt}%
\definecolor{currentstroke}{rgb}{0.121569,0.466667,0.705882}%
\pgfsetstrokecolor{currentstroke}%
\pgfsetstrokeopacity{0.390067}%
\pgfsetdash{}{0pt}%
\pgfpathmoveto{\pgfqpoint{1.559459in}{1.817857in}}%
\pgfpathcurveto{\pgfqpoint{1.567696in}{1.817857in}}{\pgfqpoint{1.575596in}{1.821129in}}{\pgfqpoint{1.581420in}{1.826953in}}%
\pgfpathcurveto{\pgfqpoint{1.587244in}{1.832777in}}{\pgfqpoint{1.590516in}{1.840677in}}{\pgfqpoint{1.590516in}{1.848913in}}%
\pgfpathcurveto{\pgfqpoint{1.590516in}{1.857150in}}{\pgfqpoint{1.587244in}{1.865050in}}{\pgfqpoint{1.581420in}{1.870874in}}%
\pgfpathcurveto{\pgfqpoint{1.575596in}{1.876697in}}{\pgfqpoint{1.567696in}{1.879970in}}{\pgfqpoint{1.559459in}{1.879970in}}%
\pgfpathcurveto{\pgfqpoint{1.551223in}{1.879970in}}{\pgfqpoint{1.543323in}{1.876697in}}{\pgfqpoint{1.537499in}{1.870874in}}%
\pgfpathcurveto{\pgfqpoint{1.531675in}{1.865050in}}{\pgfqpoint{1.528403in}{1.857150in}}{\pgfqpoint{1.528403in}{1.848913in}}%
\pgfpathcurveto{\pgfqpoint{1.528403in}{1.840677in}}{\pgfqpoint{1.531675in}{1.832777in}}{\pgfqpoint{1.537499in}{1.826953in}}%
\pgfpathcurveto{\pgfqpoint{1.543323in}{1.821129in}}{\pgfqpoint{1.551223in}{1.817857in}}{\pgfqpoint{1.559459in}{1.817857in}}%
\pgfpathclose%
\pgfusepath{stroke,fill}%
\end{pgfscope}%
\begin{pgfscope}%
\pgfpathrectangle{\pgfqpoint{0.100000in}{0.212622in}}{\pgfqpoint{3.696000in}{3.696000in}}%
\pgfusepath{clip}%
\pgfsetbuttcap%
\pgfsetroundjoin%
\definecolor{currentfill}{rgb}{0.121569,0.466667,0.705882}%
\pgfsetfillcolor{currentfill}%
\pgfsetfillopacity{0.393507}%
\pgfsetlinewidth{1.003750pt}%
\definecolor{currentstroke}{rgb}{0.121569,0.466667,0.705882}%
\pgfsetstrokecolor{currentstroke}%
\pgfsetstrokeopacity{0.393507}%
\pgfsetdash{}{0pt}%
\pgfpathmoveto{\pgfqpoint{1.550283in}{1.806696in}}%
\pgfpathcurveto{\pgfqpoint{1.558520in}{1.806696in}}{\pgfqpoint{1.566420in}{1.809968in}}{\pgfqpoint{1.572244in}{1.815792in}}%
\pgfpathcurveto{\pgfqpoint{1.578068in}{1.821616in}}{\pgfqpoint{1.581340in}{1.829516in}}{\pgfqpoint{1.581340in}{1.837753in}}%
\pgfpathcurveto{\pgfqpoint{1.581340in}{1.845989in}}{\pgfqpoint{1.578068in}{1.853889in}}{\pgfqpoint{1.572244in}{1.859713in}}%
\pgfpathcurveto{\pgfqpoint{1.566420in}{1.865537in}}{\pgfqpoint{1.558520in}{1.868809in}}{\pgfqpoint{1.550283in}{1.868809in}}%
\pgfpathcurveto{\pgfqpoint{1.542047in}{1.868809in}}{\pgfqpoint{1.534147in}{1.865537in}}{\pgfqpoint{1.528323in}{1.859713in}}%
\pgfpathcurveto{\pgfqpoint{1.522499in}{1.853889in}}{\pgfqpoint{1.519227in}{1.845989in}}{\pgfqpoint{1.519227in}{1.837753in}}%
\pgfpathcurveto{\pgfqpoint{1.519227in}{1.829516in}}{\pgfqpoint{1.522499in}{1.821616in}}{\pgfqpoint{1.528323in}{1.815792in}}%
\pgfpathcurveto{\pgfqpoint{1.534147in}{1.809968in}}{\pgfqpoint{1.542047in}{1.806696in}}{\pgfqpoint{1.550283in}{1.806696in}}%
\pgfpathclose%
\pgfusepath{stroke,fill}%
\end{pgfscope}%
\begin{pgfscope}%
\pgfpathrectangle{\pgfqpoint{0.100000in}{0.212622in}}{\pgfqpoint{3.696000in}{3.696000in}}%
\pgfusepath{clip}%
\pgfsetbuttcap%
\pgfsetroundjoin%
\definecolor{currentfill}{rgb}{0.121569,0.466667,0.705882}%
\pgfsetfillcolor{currentfill}%
\pgfsetfillopacity{0.394640}%
\pgfsetlinewidth{1.003750pt}%
\definecolor{currentstroke}{rgb}{0.121569,0.466667,0.705882}%
\pgfsetstrokecolor{currentstroke}%
\pgfsetstrokeopacity{0.394640}%
\pgfsetdash{}{0pt}%
\pgfpathmoveto{\pgfqpoint{1.629479in}{2.085417in}}%
\pgfpathcurveto{\pgfqpoint{1.637716in}{2.085417in}}{\pgfqpoint{1.645616in}{2.088689in}}{\pgfqpoint{1.651440in}{2.094513in}}%
\pgfpathcurveto{\pgfqpoint{1.657264in}{2.100337in}}{\pgfqpoint{1.660536in}{2.108237in}}{\pgfqpoint{1.660536in}{2.116474in}}%
\pgfpathcurveto{\pgfqpoint{1.660536in}{2.124710in}}{\pgfqpoint{1.657264in}{2.132610in}}{\pgfqpoint{1.651440in}{2.138434in}}%
\pgfpathcurveto{\pgfqpoint{1.645616in}{2.144258in}}{\pgfqpoint{1.637716in}{2.147530in}}{\pgfqpoint{1.629479in}{2.147530in}}%
\pgfpathcurveto{\pgfqpoint{1.621243in}{2.147530in}}{\pgfqpoint{1.613343in}{2.144258in}}{\pgfqpoint{1.607519in}{2.138434in}}%
\pgfpathcurveto{\pgfqpoint{1.601695in}{2.132610in}}{\pgfqpoint{1.598423in}{2.124710in}}{\pgfqpoint{1.598423in}{2.116474in}}%
\pgfpathcurveto{\pgfqpoint{1.598423in}{2.108237in}}{\pgfqpoint{1.601695in}{2.100337in}}{\pgfqpoint{1.607519in}{2.094513in}}%
\pgfpathcurveto{\pgfqpoint{1.613343in}{2.088689in}}{\pgfqpoint{1.621243in}{2.085417in}}{\pgfqpoint{1.629479in}{2.085417in}}%
\pgfpathclose%
\pgfusepath{stroke,fill}%
\end{pgfscope}%
\begin{pgfscope}%
\pgfpathrectangle{\pgfqpoint{0.100000in}{0.212622in}}{\pgfqpoint{3.696000in}{3.696000in}}%
\pgfusepath{clip}%
\pgfsetbuttcap%
\pgfsetroundjoin%
\definecolor{currentfill}{rgb}{0.121569,0.466667,0.705882}%
\pgfsetfillcolor{currentfill}%
\pgfsetfillopacity{0.398696}%
\pgfsetlinewidth{1.003750pt}%
\definecolor{currentstroke}{rgb}{0.121569,0.466667,0.705882}%
\pgfsetstrokecolor{currentstroke}%
\pgfsetstrokeopacity{0.398696}%
\pgfsetdash{}{0pt}%
\pgfpathmoveto{\pgfqpoint{1.517417in}{1.802155in}}%
\pgfpathcurveto{\pgfqpoint{1.525653in}{1.802155in}}{\pgfqpoint{1.533553in}{1.805427in}}{\pgfqpoint{1.539377in}{1.811251in}}%
\pgfpathcurveto{\pgfqpoint{1.545201in}{1.817075in}}{\pgfqpoint{1.548473in}{1.824975in}}{\pgfqpoint{1.548473in}{1.833211in}}%
\pgfpathcurveto{\pgfqpoint{1.548473in}{1.841447in}}{\pgfqpoint{1.545201in}{1.849347in}}{\pgfqpoint{1.539377in}{1.855171in}}%
\pgfpathcurveto{\pgfqpoint{1.533553in}{1.860995in}}{\pgfqpoint{1.525653in}{1.864268in}}{\pgfqpoint{1.517417in}{1.864268in}}%
\pgfpathcurveto{\pgfqpoint{1.509181in}{1.864268in}}{\pgfqpoint{1.501281in}{1.860995in}}{\pgfqpoint{1.495457in}{1.855171in}}%
\pgfpathcurveto{\pgfqpoint{1.489633in}{1.849347in}}{\pgfqpoint{1.486360in}{1.841447in}}{\pgfqpoint{1.486360in}{1.833211in}}%
\pgfpathcurveto{\pgfqpoint{1.486360in}{1.824975in}}{\pgfqpoint{1.489633in}{1.817075in}}{\pgfqpoint{1.495457in}{1.811251in}}%
\pgfpathcurveto{\pgfqpoint{1.501281in}{1.805427in}}{\pgfqpoint{1.509181in}{1.802155in}}{\pgfqpoint{1.517417in}{1.802155in}}%
\pgfpathclose%
\pgfusepath{stroke,fill}%
\end{pgfscope}%
\begin{pgfscope}%
\pgfpathrectangle{\pgfqpoint{0.100000in}{0.212622in}}{\pgfqpoint{3.696000in}{3.696000in}}%
\pgfusepath{clip}%
\pgfsetbuttcap%
\pgfsetroundjoin%
\definecolor{currentfill}{rgb}{0.121569,0.466667,0.705882}%
\pgfsetfillcolor{currentfill}%
\pgfsetfillopacity{0.403496}%
\pgfsetlinewidth{1.003750pt}%
\definecolor{currentstroke}{rgb}{0.121569,0.466667,0.705882}%
\pgfsetstrokecolor{currentstroke}%
\pgfsetstrokeopacity{0.403496}%
\pgfsetdash{}{0pt}%
\pgfpathmoveto{\pgfqpoint{1.641081in}{2.063434in}}%
\pgfpathcurveto{\pgfqpoint{1.649318in}{2.063434in}}{\pgfqpoint{1.657218in}{2.066706in}}{\pgfqpoint{1.663042in}{2.072530in}}%
\pgfpathcurveto{\pgfqpoint{1.668866in}{2.078354in}}{\pgfqpoint{1.672138in}{2.086254in}}{\pgfqpoint{1.672138in}{2.094490in}}%
\pgfpathcurveto{\pgfqpoint{1.672138in}{2.102726in}}{\pgfqpoint{1.668866in}{2.110627in}}{\pgfqpoint{1.663042in}{2.116450in}}%
\pgfpathcurveto{\pgfqpoint{1.657218in}{2.122274in}}{\pgfqpoint{1.649318in}{2.125547in}}{\pgfqpoint{1.641081in}{2.125547in}}%
\pgfpathcurveto{\pgfqpoint{1.632845in}{2.125547in}}{\pgfqpoint{1.624945in}{2.122274in}}{\pgfqpoint{1.619121in}{2.116450in}}%
\pgfpathcurveto{\pgfqpoint{1.613297in}{2.110627in}}{\pgfqpoint{1.610025in}{2.102726in}}{\pgfqpoint{1.610025in}{2.094490in}}%
\pgfpathcurveto{\pgfqpoint{1.610025in}{2.086254in}}{\pgfqpoint{1.613297in}{2.078354in}}{\pgfqpoint{1.619121in}{2.072530in}}%
\pgfpathcurveto{\pgfqpoint{1.624945in}{2.066706in}}{\pgfqpoint{1.632845in}{2.063434in}}{\pgfqpoint{1.641081in}{2.063434in}}%
\pgfpathclose%
\pgfusepath{stroke,fill}%
\end{pgfscope}%
\begin{pgfscope}%
\pgfpathrectangle{\pgfqpoint{0.100000in}{0.212622in}}{\pgfqpoint{3.696000in}{3.696000in}}%
\pgfusepath{clip}%
\pgfsetbuttcap%
\pgfsetroundjoin%
\definecolor{currentfill}{rgb}{0.121569,0.466667,0.705882}%
\pgfsetfillcolor{currentfill}%
\pgfsetfillopacity{0.405411}%
\pgfsetlinewidth{1.003750pt}%
\definecolor{currentstroke}{rgb}{0.121569,0.466667,0.705882}%
\pgfsetstrokecolor{currentstroke}%
\pgfsetstrokeopacity{0.405411}%
\pgfsetdash{}{0pt}%
\pgfpathmoveto{\pgfqpoint{1.504140in}{1.785401in}}%
\pgfpathcurveto{\pgfqpoint{1.512377in}{1.785401in}}{\pgfqpoint{1.520277in}{1.788673in}}{\pgfqpoint{1.526101in}{1.794497in}}%
\pgfpathcurveto{\pgfqpoint{1.531925in}{1.800321in}}{\pgfqpoint{1.535197in}{1.808221in}}{\pgfqpoint{1.535197in}{1.816457in}}%
\pgfpathcurveto{\pgfqpoint{1.535197in}{1.824693in}}{\pgfqpoint{1.531925in}{1.832593in}}{\pgfqpoint{1.526101in}{1.838417in}}%
\pgfpathcurveto{\pgfqpoint{1.520277in}{1.844241in}}{\pgfqpoint{1.512377in}{1.847514in}}{\pgfqpoint{1.504140in}{1.847514in}}%
\pgfpathcurveto{\pgfqpoint{1.495904in}{1.847514in}}{\pgfqpoint{1.488004in}{1.844241in}}{\pgfqpoint{1.482180in}{1.838417in}}%
\pgfpathcurveto{\pgfqpoint{1.476356in}{1.832593in}}{\pgfqpoint{1.473084in}{1.824693in}}{\pgfqpoint{1.473084in}{1.816457in}}%
\pgfpathcurveto{\pgfqpoint{1.473084in}{1.808221in}}{\pgfqpoint{1.476356in}{1.800321in}}{\pgfqpoint{1.482180in}{1.794497in}}%
\pgfpathcurveto{\pgfqpoint{1.488004in}{1.788673in}}{\pgfqpoint{1.495904in}{1.785401in}}{\pgfqpoint{1.504140in}{1.785401in}}%
\pgfpathclose%
\pgfusepath{stroke,fill}%
\end{pgfscope}%
\begin{pgfscope}%
\pgfpathrectangle{\pgfqpoint{0.100000in}{0.212622in}}{\pgfqpoint{3.696000in}{3.696000in}}%
\pgfusepath{clip}%
\pgfsetbuttcap%
\pgfsetroundjoin%
\definecolor{currentfill}{rgb}{0.121569,0.466667,0.705882}%
\pgfsetfillcolor{currentfill}%
\pgfsetfillopacity{0.414164}%
\pgfsetlinewidth{1.003750pt}%
\definecolor{currentstroke}{rgb}{0.121569,0.466667,0.705882}%
\pgfsetstrokecolor{currentstroke}%
\pgfsetstrokeopacity{0.414164}%
\pgfsetdash{}{0pt}%
\pgfpathmoveto{\pgfqpoint{1.443295in}{1.783512in}}%
\pgfpathcurveto{\pgfqpoint{1.451531in}{1.783512in}}{\pgfqpoint{1.459431in}{1.786785in}}{\pgfqpoint{1.465255in}{1.792609in}}%
\pgfpathcurveto{\pgfqpoint{1.471079in}{1.798433in}}{\pgfqpoint{1.474351in}{1.806333in}}{\pgfqpoint{1.474351in}{1.814569in}}%
\pgfpathcurveto{\pgfqpoint{1.474351in}{1.822805in}}{\pgfqpoint{1.471079in}{1.830705in}}{\pgfqpoint{1.465255in}{1.836529in}}%
\pgfpathcurveto{\pgfqpoint{1.459431in}{1.842353in}}{\pgfqpoint{1.451531in}{1.845625in}}{\pgfqpoint{1.443295in}{1.845625in}}%
\pgfpathcurveto{\pgfqpoint{1.435058in}{1.845625in}}{\pgfqpoint{1.427158in}{1.842353in}}{\pgfqpoint{1.421334in}{1.836529in}}%
\pgfpathcurveto{\pgfqpoint{1.415510in}{1.830705in}}{\pgfqpoint{1.412238in}{1.822805in}}{\pgfqpoint{1.412238in}{1.814569in}}%
\pgfpathcurveto{\pgfqpoint{1.412238in}{1.806333in}}{\pgfqpoint{1.415510in}{1.798433in}}{\pgfqpoint{1.421334in}{1.792609in}}%
\pgfpathcurveto{\pgfqpoint{1.427158in}{1.786785in}}{\pgfqpoint{1.435058in}{1.783512in}}{\pgfqpoint{1.443295in}{1.783512in}}%
\pgfpathclose%
\pgfusepath{stroke,fill}%
\end{pgfscope}%
\begin{pgfscope}%
\pgfpathrectangle{\pgfqpoint{0.100000in}{0.212622in}}{\pgfqpoint{3.696000in}{3.696000in}}%
\pgfusepath{clip}%
\pgfsetbuttcap%
\pgfsetroundjoin%
\definecolor{currentfill}{rgb}{0.121569,0.466667,0.705882}%
\pgfsetfillcolor{currentfill}%
\pgfsetfillopacity{0.417218}%
\pgfsetlinewidth{1.003750pt}%
\definecolor{currentstroke}{rgb}{0.121569,0.466667,0.705882}%
\pgfsetstrokecolor{currentstroke}%
\pgfsetstrokeopacity{0.417218}%
\pgfsetdash{}{0pt}%
\pgfpathmoveto{\pgfqpoint{1.673121in}{2.067885in}}%
\pgfpathcurveto{\pgfqpoint{1.681358in}{2.067885in}}{\pgfqpoint{1.689258in}{2.071157in}}{\pgfqpoint{1.695082in}{2.076981in}}%
\pgfpathcurveto{\pgfqpoint{1.700905in}{2.082805in}}{\pgfqpoint{1.704178in}{2.090705in}}{\pgfqpoint{1.704178in}{2.098942in}}%
\pgfpathcurveto{\pgfqpoint{1.704178in}{2.107178in}}{\pgfqpoint{1.700905in}{2.115078in}}{\pgfqpoint{1.695082in}{2.120902in}}%
\pgfpathcurveto{\pgfqpoint{1.689258in}{2.126726in}}{\pgfqpoint{1.681358in}{2.129998in}}{\pgfqpoint{1.673121in}{2.129998in}}%
\pgfpathcurveto{\pgfqpoint{1.664885in}{2.129998in}}{\pgfqpoint{1.656985in}{2.126726in}}{\pgfqpoint{1.651161in}{2.120902in}}%
\pgfpathcurveto{\pgfqpoint{1.645337in}{2.115078in}}{\pgfqpoint{1.642065in}{2.107178in}}{\pgfqpoint{1.642065in}{2.098942in}}%
\pgfpathcurveto{\pgfqpoint{1.642065in}{2.090705in}}{\pgfqpoint{1.645337in}{2.082805in}}{\pgfqpoint{1.651161in}{2.076981in}}%
\pgfpathcurveto{\pgfqpoint{1.656985in}{2.071157in}}{\pgfqpoint{1.664885in}{2.067885in}}{\pgfqpoint{1.673121in}{2.067885in}}%
\pgfpathclose%
\pgfusepath{stroke,fill}%
\end{pgfscope}%
\begin{pgfscope}%
\pgfpathrectangle{\pgfqpoint{0.100000in}{0.212622in}}{\pgfqpoint{3.696000in}{3.696000in}}%
\pgfusepath{clip}%
\pgfsetbuttcap%
\pgfsetroundjoin%
\definecolor{currentfill}{rgb}{0.121569,0.466667,0.705882}%
\pgfsetfillcolor{currentfill}%
\pgfsetfillopacity{0.426593}%
\pgfsetlinewidth{1.003750pt}%
\definecolor{currentstroke}{rgb}{0.121569,0.466667,0.705882}%
\pgfsetstrokecolor{currentstroke}%
\pgfsetstrokeopacity{0.426593}%
\pgfsetdash{}{0pt}%
\pgfpathmoveto{\pgfqpoint{1.422689in}{1.759464in}}%
\pgfpathcurveto{\pgfqpoint{1.430926in}{1.759464in}}{\pgfqpoint{1.438826in}{1.762737in}}{\pgfqpoint{1.444650in}{1.768561in}}%
\pgfpathcurveto{\pgfqpoint{1.450474in}{1.774385in}}{\pgfqpoint{1.453746in}{1.782285in}}{\pgfqpoint{1.453746in}{1.790521in}}%
\pgfpathcurveto{\pgfqpoint{1.453746in}{1.798757in}}{\pgfqpoint{1.450474in}{1.806657in}}{\pgfqpoint{1.444650in}{1.812481in}}%
\pgfpathcurveto{\pgfqpoint{1.438826in}{1.818305in}}{\pgfqpoint{1.430926in}{1.821577in}}{\pgfqpoint{1.422689in}{1.821577in}}%
\pgfpathcurveto{\pgfqpoint{1.414453in}{1.821577in}}{\pgfqpoint{1.406553in}{1.818305in}}{\pgfqpoint{1.400729in}{1.812481in}}%
\pgfpathcurveto{\pgfqpoint{1.394905in}{1.806657in}}{\pgfqpoint{1.391633in}{1.798757in}}{\pgfqpoint{1.391633in}{1.790521in}}%
\pgfpathcurveto{\pgfqpoint{1.391633in}{1.782285in}}{\pgfqpoint{1.394905in}{1.774385in}}{\pgfqpoint{1.400729in}{1.768561in}}%
\pgfpathcurveto{\pgfqpoint{1.406553in}{1.762737in}}{\pgfqpoint{1.414453in}{1.759464in}}{\pgfqpoint{1.422689in}{1.759464in}}%
\pgfpathclose%
\pgfusepath{stroke,fill}%
\end{pgfscope}%
\begin{pgfscope}%
\pgfpathrectangle{\pgfqpoint{0.100000in}{0.212622in}}{\pgfqpoint{3.696000in}{3.696000in}}%
\pgfusepath{clip}%
\pgfsetbuttcap%
\pgfsetroundjoin%
\definecolor{currentfill}{rgb}{0.121569,0.466667,0.705882}%
\pgfsetfillcolor{currentfill}%
\pgfsetfillopacity{0.428337}%
\pgfsetlinewidth{1.003750pt}%
\definecolor{currentstroke}{rgb}{0.121569,0.466667,0.705882}%
\pgfsetstrokecolor{currentstroke}%
\pgfsetstrokeopacity{0.428337}%
\pgfsetdash{}{0pt}%
\pgfpathmoveto{\pgfqpoint{1.693056in}{2.040164in}}%
\pgfpathcurveto{\pgfqpoint{1.701292in}{2.040164in}}{\pgfqpoint{1.709192in}{2.043437in}}{\pgfqpoint{1.715016in}{2.049261in}}%
\pgfpathcurveto{\pgfqpoint{1.720840in}{2.055085in}}{\pgfqpoint{1.724113in}{2.062985in}}{\pgfqpoint{1.724113in}{2.071221in}}%
\pgfpathcurveto{\pgfqpoint{1.724113in}{2.079457in}}{\pgfqpoint{1.720840in}{2.087357in}}{\pgfqpoint{1.715016in}{2.093181in}}%
\pgfpathcurveto{\pgfqpoint{1.709192in}{2.099005in}}{\pgfqpoint{1.701292in}{2.102277in}}{\pgfqpoint{1.693056in}{2.102277in}}%
\pgfpathcurveto{\pgfqpoint{1.684820in}{2.102277in}}{\pgfqpoint{1.676920in}{2.099005in}}{\pgfqpoint{1.671096in}{2.093181in}}%
\pgfpathcurveto{\pgfqpoint{1.665272in}{2.087357in}}{\pgfqpoint{1.662000in}{2.079457in}}{\pgfqpoint{1.662000in}{2.071221in}}%
\pgfpathcurveto{\pgfqpoint{1.662000in}{2.062985in}}{\pgfqpoint{1.665272in}{2.055085in}}{\pgfqpoint{1.671096in}{2.049261in}}%
\pgfpathcurveto{\pgfqpoint{1.676920in}{2.043437in}}{\pgfqpoint{1.684820in}{2.040164in}}{\pgfqpoint{1.693056in}{2.040164in}}%
\pgfpathclose%
\pgfusepath{stroke,fill}%
\end{pgfscope}%
\begin{pgfscope}%
\pgfpathrectangle{\pgfqpoint{0.100000in}{0.212622in}}{\pgfqpoint{3.696000in}{3.696000in}}%
\pgfusepath{clip}%
\pgfsetbuttcap%
\pgfsetroundjoin%
\definecolor{currentfill}{rgb}{0.121569,0.466667,0.705882}%
\pgfsetfillcolor{currentfill}%
\pgfsetfillopacity{0.434964}%
\pgfsetlinewidth{1.003750pt}%
\definecolor{currentstroke}{rgb}{0.121569,0.466667,0.705882}%
\pgfsetstrokecolor{currentstroke}%
\pgfsetstrokeopacity{0.434964}%
\pgfsetdash{}{0pt}%
\pgfpathmoveto{\pgfqpoint{1.371310in}{1.756102in}}%
\pgfpathcurveto{\pgfqpoint{1.379546in}{1.756102in}}{\pgfqpoint{1.387446in}{1.759374in}}{\pgfqpoint{1.393270in}{1.765198in}}%
\pgfpathcurveto{\pgfqpoint{1.399094in}{1.771022in}}{\pgfqpoint{1.402366in}{1.778922in}}{\pgfqpoint{1.402366in}{1.787158in}}%
\pgfpathcurveto{\pgfqpoint{1.402366in}{1.795394in}}{\pgfqpoint{1.399094in}{1.803294in}}{\pgfqpoint{1.393270in}{1.809118in}}%
\pgfpathcurveto{\pgfqpoint{1.387446in}{1.814942in}}{\pgfqpoint{1.379546in}{1.818215in}}{\pgfqpoint{1.371310in}{1.818215in}}%
\pgfpathcurveto{\pgfqpoint{1.363073in}{1.818215in}}{\pgfqpoint{1.355173in}{1.814942in}}{\pgfqpoint{1.349349in}{1.809118in}}%
\pgfpathcurveto{\pgfqpoint{1.343525in}{1.803294in}}{\pgfqpoint{1.340253in}{1.795394in}}{\pgfqpoint{1.340253in}{1.787158in}}%
\pgfpathcurveto{\pgfqpoint{1.340253in}{1.778922in}}{\pgfqpoint{1.343525in}{1.771022in}}{\pgfqpoint{1.349349in}{1.765198in}}%
\pgfpathcurveto{\pgfqpoint{1.355173in}{1.759374in}}{\pgfqpoint{1.363073in}{1.756102in}}{\pgfqpoint{1.371310in}{1.756102in}}%
\pgfpathclose%
\pgfusepath{stroke,fill}%
\end{pgfscope}%
\begin{pgfscope}%
\pgfpathrectangle{\pgfqpoint{0.100000in}{0.212622in}}{\pgfqpoint{3.696000in}{3.696000in}}%
\pgfusepath{clip}%
\pgfsetbuttcap%
\pgfsetroundjoin%
\definecolor{currentfill}{rgb}{0.121569,0.466667,0.705882}%
\pgfsetfillcolor{currentfill}%
\pgfsetfillopacity{0.445056}%
\pgfsetlinewidth{1.003750pt}%
\definecolor{currentstroke}{rgb}{0.121569,0.466667,0.705882}%
\pgfsetstrokecolor{currentstroke}%
\pgfsetstrokeopacity{0.445056}%
\pgfsetdash{}{0pt}%
\pgfpathmoveto{\pgfqpoint{1.738751in}{2.047820in}}%
\pgfpathcurveto{\pgfqpoint{1.746987in}{2.047820in}}{\pgfqpoint{1.754887in}{2.051092in}}{\pgfqpoint{1.760711in}{2.056916in}}%
\pgfpathcurveto{\pgfqpoint{1.766535in}{2.062740in}}{\pgfqpoint{1.769808in}{2.070640in}}{\pgfqpoint{1.769808in}{2.078876in}}%
\pgfpathcurveto{\pgfqpoint{1.769808in}{2.087112in}}{\pgfqpoint{1.766535in}{2.095012in}}{\pgfqpoint{1.760711in}{2.100836in}}%
\pgfpathcurveto{\pgfqpoint{1.754887in}{2.106660in}}{\pgfqpoint{1.746987in}{2.109933in}}{\pgfqpoint{1.738751in}{2.109933in}}%
\pgfpathcurveto{\pgfqpoint{1.730515in}{2.109933in}}{\pgfqpoint{1.722615in}{2.106660in}}{\pgfqpoint{1.716791in}{2.100836in}}%
\pgfpathcurveto{\pgfqpoint{1.710967in}{2.095012in}}{\pgfqpoint{1.707695in}{2.087112in}}{\pgfqpoint{1.707695in}{2.078876in}}%
\pgfpathcurveto{\pgfqpoint{1.707695in}{2.070640in}}{\pgfqpoint{1.710967in}{2.062740in}}{\pgfqpoint{1.716791in}{2.056916in}}%
\pgfpathcurveto{\pgfqpoint{1.722615in}{2.051092in}}{\pgfqpoint{1.730515in}{2.047820in}}{\pgfqpoint{1.738751in}{2.047820in}}%
\pgfpathclose%
\pgfusepath{stroke,fill}%
\end{pgfscope}%
\begin{pgfscope}%
\pgfpathrectangle{\pgfqpoint{0.100000in}{0.212622in}}{\pgfqpoint{3.696000in}{3.696000in}}%
\pgfusepath{clip}%
\pgfsetbuttcap%
\pgfsetroundjoin%
\definecolor{currentfill}{rgb}{0.121569,0.466667,0.705882}%
\pgfsetfillcolor{currentfill}%
\pgfsetfillopacity{0.445512}%
\pgfsetlinewidth{1.003750pt}%
\definecolor{currentstroke}{rgb}{0.121569,0.466667,0.705882}%
\pgfsetstrokecolor{currentstroke}%
\pgfsetstrokeopacity{0.445512}%
\pgfsetdash{}{0pt}%
\pgfpathmoveto{\pgfqpoint{1.351329in}{1.740424in}}%
\pgfpathcurveto{\pgfqpoint{1.359565in}{1.740424in}}{\pgfqpoint{1.367465in}{1.743697in}}{\pgfqpoint{1.373289in}{1.749521in}}%
\pgfpathcurveto{\pgfqpoint{1.379113in}{1.755345in}}{\pgfqpoint{1.382386in}{1.763245in}}{\pgfqpoint{1.382386in}{1.771481in}}%
\pgfpathcurveto{\pgfqpoint{1.382386in}{1.779717in}}{\pgfqpoint{1.379113in}{1.787617in}}{\pgfqpoint{1.373289in}{1.793441in}}%
\pgfpathcurveto{\pgfqpoint{1.367465in}{1.799265in}}{\pgfqpoint{1.359565in}{1.802537in}}{\pgfqpoint{1.351329in}{1.802537in}}%
\pgfpathcurveto{\pgfqpoint{1.343093in}{1.802537in}}{\pgfqpoint{1.335193in}{1.799265in}}{\pgfqpoint{1.329369in}{1.793441in}}%
\pgfpathcurveto{\pgfqpoint{1.323545in}{1.787617in}}{\pgfqpoint{1.320273in}{1.779717in}}{\pgfqpoint{1.320273in}{1.771481in}}%
\pgfpathcurveto{\pgfqpoint{1.320273in}{1.763245in}}{\pgfqpoint{1.323545in}{1.755345in}}{\pgfqpoint{1.329369in}{1.749521in}}%
\pgfpathcurveto{\pgfqpoint{1.335193in}{1.743697in}}{\pgfqpoint{1.343093in}{1.740424in}}{\pgfqpoint{1.351329in}{1.740424in}}%
\pgfpathclose%
\pgfusepath{stroke,fill}%
\end{pgfscope}%
\begin{pgfscope}%
\pgfpathrectangle{\pgfqpoint{0.100000in}{0.212622in}}{\pgfqpoint{3.696000in}{3.696000in}}%
\pgfusepath{clip}%
\pgfsetbuttcap%
\pgfsetroundjoin%
\definecolor{currentfill}{rgb}{0.121569,0.466667,0.705882}%
\pgfsetfillcolor{currentfill}%
\pgfsetfillopacity{0.452528}%
\pgfsetlinewidth{1.003750pt}%
\definecolor{currentstroke}{rgb}{0.121569,0.466667,0.705882}%
\pgfsetstrokecolor{currentstroke}%
\pgfsetstrokeopacity{0.452528}%
\pgfsetdash{}{0pt}%
\pgfpathmoveto{\pgfqpoint{1.306332in}{1.750471in}}%
\pgfpathcurveto{\pgfqpoint{1.314568in}{1.750471in}}{\pgfqpoint{1.322468in}{1.753744in}}{\pgfqpoint{1.328292in}{1.759568in}}%
\pgfpathcurveto{\pgfqpoint{1.334116in}{1.765391in}}{\pgfqpoint{1.337389in}{1.773291in}}{\pgfqpoint{1.337389in}{1.781528in}}%
\pgfpathcurveto{\pgfqpoint{1.337389in}{1.789764in}}{\pgfqpoint{1.334116in}{1.797664in}}{\pgfqpoint{1.328292in}{1.803488in}}%
\pgfpathcurveto{\pgfqpoint{1.322468in}{1.809312in}}{\pgfqpoint{1.314568in}{1.812584in}}{\pgfqpoint{1.306332in}{1.812584in}}%
\pgfpathcurveto{\pgfqpoint{1.298096in}{1.812584in}}{\pgfqpoint{1.290196in}{1.809312in}}{\pgfqpoint{1.284372in}{1.803488in}}%
\pgfpathcurveto{\pgfqpoint{1.278548in}{1.797664in}}{\pgfqpoint{1.275276in}{1.789764in}}{\pgfqpoint{1.275276in}{1.781528in}}%
\pgfpathcurveto{\pgfqpoint{1.275276in}{1.773291in}}{\pgfqpoint{1.278548in}{1.765391in}}{\pgfqpoint{1.284372in}{1.759568in}}%
\pgfpathcurveto{\pgfqpoint{1.290196in}{1.753744in}}{\pgfqpoint{1.298096in}{1.750471in}}{\pgfqpoint{1.306332in}{1.750471in}}%
\pgfpathclose%
\pgfusepath{stroke,fill}%
\end{pgfscope}%
\begin{pgfscope}%
\pgfpathrectangle{\pgfqpoint{0.100000in}{0.212622in}}{\pgfqpoint{3.696000in}{3.696000in}}%
\pgfusepath{clip}%
\pgfsetbuttcap%
\pgfsetroundjoin%
\definecolor{currentfill}{rgb}{0.121569,0.466667,0.705882}%
\pgfsetfillcolor{currentfill}%
\pgfsetfillopacity{0.458249}%
\pgfsetlinewidth{1.003750pt}%
\definecolor{currentstroke}{rgb}{0.121569,0.466667,0.705882}%
\pgfsetstrokecolor{currentstroke}%
\pgfsetstrokeopacity{0.458249}%
\pgfsetdash{}{0pt}%
\pgfpathmoveto{\pgfqpoint{1.764457in}{2.015273in}}%
\pgfpathcurveto{\pgfqpoint{1.772693in}{2.015273in}}{\pgfqpoint{1.780593in}{2.018545in}}{\pgfqpoint{1.786417in}{2.024369in}}%
\pgfpathcurveto{\pgfqpoint{1.792241in}{2.030193in}}{\pgfqpoint{1.795513in}{2.038093in}}{\pgfqpoint{1.795513in}{2.046329in}}%
\pgfpathcurveto{\pgfqpoint{1.795513in}{2.054565in}}{\pgfqpoint{1.792241in}{2.062465in}}{\pgfqpoint{1.786417in}{2.068289in}}%
\pgfpathcurveto{\pgfqpoint{1.780593in}{2.074113in}}{\pgfqpoint{1.772693in}{2.077386in}}{\pgfqpoint{1.764457in}{2.077386in}}%
\pgfpathcurveto{\pgfqpoint{1.756220in}{2.077386in}}{\pgfqpoint{1.748320in}{2.074113in}}{\pgfqpoint{1.742496in}{2.068289in}}%
\pgfpathcurveto{\pgfqpoint{1.736672in}{2.062465in}}{\pgfqpoint{1.733400in}{2.054565in}}{\pgfqpoint{1.733400in}{2.046329in}}%
\pgfpathcurveto{\pgfqpoint{1.733400in}{2.038093in}}{\pgfqpoint{1.736672in}{2.030193in}}{\pgfqpoint{1.742496in}{2.024369in}}%
\pgfpathcurveto{\pgfqpoint{1.748320in}{2.018545in}}{\pgfqpoint{1.756220in}{2.015273in}}{\pgfqpoint{1.764457in}{2.015273in}}%
\pgfpathclose%
\pgfusepath{stroke,fill}%
\end{pgfscope}%
\begin{pgfscope}%
\pgfpathrectangle{\pgfqpoint{0.100000in}{0.212622in}}{\pgfqpoint{3.696000in}{3.696000in}}%
\pgfusepath{clip}%
\pgfsetbuttcap%
\pgfsetroundjoin%
\definecolor{currentfill}{rgb}{0.121569,0.466667,0.705882}%
\pgfsetfillcolor{currentfill}%
\pgfsetfillopacity{0.460324}%
\pgfsetlinewidth{1.003750pt}%
\definecolor{currentstroke}{rgb}{0.121569,0.466667,0.705882}%
\pgfsetstrokecolor{currentstroke}%
\pgfsetstrokeopacity{0.460324}%
\pgfsetdash{}{0pt}%
\pgfpathmoveto{\pgfqpoint{1.293585in}{1.737357in}}%
\pgfpathcurveto{\pgfqpoint{1.301821in}{1.737357in}}{\pgfqpoint{1.309721in}{1.740629in}}{\pgfqpoint{1.315545in}{1.746453in}}%
\pgfpathcurveto{\pgfqpoint{1.321369in}{1.752277in}}{\pgfqpoint{1.324641in}{1.760177in}}{\pgfqpoint{1.324641in}{1.768413in}}%
\pgfpathcurveto{\pgfqpoint{1.324641in}{1.776649in}}{\pgfqpoint{1.321369in}{1.784550in}}{\pgfqpoint{1.315545in}{1.790373in}}%
\pgfpathcurveto{\pgfqpoint{1.309721in}{1.796197in}}{\pgfqpoint{1.301821in}{1.799470in}}{\pgfqpoint{1.293585in}{1.799470in}}%
\pgfpathcurveto{\pgfqpoint{1.285348in}{1.799470in}}{\pgfqpoint{1.277448in}{1.796197in}}{\pgfqpoint{1.271624in}{1.790373in}}%
\pgfpathcurveto{\pgfqpoint{1.265800in}{1.784550in}}{\pgfqpoint{1.262528in}{1.776649in}}{\pgfqpoint{1.262528in}{1.768413in}}%
\pgfpathcurveto{\pgfqpoint{1.262528in}{1.760177in}}{\pgfqpoint{1.265800in}{1.752277in}}{\pgfqpoint{1.271624in}{1.746453in}}%
\pgfpathcurveto{\pgfqpoint{1.277448in}{1.740629in}}{\pgfqpoint{1.285348in}{1.737357in}}{\pgfqpoint{1.293585in}{1.737357in}}%
\pgfpathclose%
\pgfusepath{stroke,fill}%
\end{pgfscope}%
\begin{pgfscope}%
\pgfpathrectangle{\pgfqpoint{0.100000in}{0.212622in}}{\pgfqpoint{3.696000in}{3.696000in}}%
\pgfusepath{clip}%
\pgfsetbuttcap%
\pgfsetroundjoin%
\definecolor{currentfill}{rgb}{0.121569,0.466667,0.705882}%
\pgfsetfillcolor{currentfill}%
\pgfsetfillopacity{0.465727}%
\pgfsetlinewidth{1.003750pt}%
\definecolor{currentstroke}{rgb}{0.121569,0.466667,0.705882}%
\pgfsetstrokecolor{currentstroke}%
\pgfsetstrokeopacity{0.465727}%
\pgfsetdash{}{0pt}%
\pgfpathmoveto{\pgfqpoint{1.264776in}{1.745548in}}%
\pgfpathcurveto{\pgfqpoint{1.273012in}{1.745548in}}{\pgfqpoint{1.280912in}{1.748821in}}{\pgfqpoint{1.286736in}{1.754645in}}%
\pgfpathcurveto{\pgfqpoint{1.292560in}{1.760469in}}{\pgfqpoint{1.295832in}{1.768369in}}{\pgfqpoint{1.295832in}{1.776605in}}%
\pgfpathcurveto{\pgfqpoint{1.295832in}{1.784841in}}{\pgfqpoint{1.292560in}{1.792741in}}{\pgfqpoint{1.286736in}{1.798565in}}%
\pgfpathcurveto{\pgfqpoint{1.280912in}{1.804389in}}{\pgfqpoint{1.273012in}{1.807661in}}{\pgfqpoint{1.264776in}{1.807661in}}%
\pgfpathcurveto{\pgfqpoint{1.256539in}{1.807661in}}{\pgfqpoint{1.248639in}{1.804389in}}{\pgfqpoint{1.242815in}{1.798565in}}%
\pgfpathcurveto{\pgfqpoint{1.236991in}{1.792741in}}{\pgfqpoint{1.233719in}{1.784841in}}{\pgfqpoint{1.233719in}{1.776605in}}%
\pgfpathcurveto{\pgfqpoint{1.233719in}{1.768369in}}{\pgfqpoint{1.236991in}{1.760469in}}{\pgfqpoint{1.242815in}{1.754645in}}%
\pgfpathcurveto{\pgfqpoint{1.248639in}{1.748821in}}{\pgfqpoint{1.256539in}{1.745548in}}{\pgfqpoint{1.264776in}{1.745548in}}%
\pgfpathclose%
\pgfusepath{stroke,fill}%
\end{pgfscope}%
\begin{pgfscope}%
\pgfpathrectangle{\pgfqpoint{0.100000in}{0.212622in}}{\pgfqpoint{3.696000in}{3.696000in}}%
\pgfusepath{clip}%
\pgfsetbuttcap%
\pgfsetroundjoin%
\definecolor{currentfill}{rgb}{0.121569,0.466667,0.705882}%
\pgfsetfillcolor{currentfill}%
\pgfsetfillopacity{0.468067}%
\pgfsetlinewidth{1.003750pt}%
\definecolor{currentstroke}{rgb}{0.121569,0.466667,0.705882}%
\pgfsetstrokecolor{currentstroke}%
\pgfsetstrokeopacity{0.468067}%
\pgfsetdash{}{0pt}%
\pgfpathmoveto{\pgfqpoint{1.790813in}{2.019394in}}%
\pgfpathcurveto{\pgfqpoint{1.799050in}{2.019394in}}{\pgfqpoint{1.806950in}{2.022666in}}{\pgfqpoint{1.812774in}{2.028490in}}%
\pgfpathcurveto{\pgfqpoint{1.818598in}{2.034314in}}{\pgfqpoint{1.821870in}{2.042214in}}{\pgfqpoint{1.821870in}{2.050450in}}%
\pgfpathcurveto{\pgfqpoint{1.821870in}{2.058687in}}{\pgfqpoint{1.818598in}{2.066587in}}{\pgfqpoint{1.812774in}{2.072411in}}%
\pgfpathcurveto{\pgfqpoint{1.806950in}{2.078234in}}{\pgfqpoint{1.799050in}{2.081507in}}{\pgfqpoint{1.790813in}{2.081507in}}%
\pgfpathcurveto{\pgfqpoint{1.782577in}{2.081507in}}{\pgfqpoint{1.774677in}{2.078234in}}{\pgfqpoint{1.768853in}{2.072411in}}%
\pgfpathcurveto{\pgfqpoint{1.763029in}{2.066587in}}{\pgfqpoint{1.759757in}{2.058687in}}{\pgfqpoint{1.759757in}{2.050450in}}%
\pgfpathcurveto{\pgfqpoint{1.759757in}{2.042214in}}{\pgfqpoint{1.763029in}{2.034314in}}{\pgfqpoint{1.768853in}{2.028490in}}%
\pgfpathcurveto{\pgfqpoint{1.774677in}{2.022666in}}{\pgfqpoint{1.782577in}{2.019394in}}{\pgfqpoint{1.790813in}{2.019394in}}%
\pgfpathclose%
\pgfusepath{stroke,fill}%
\end{pgfscope}%
\begin{pgfscope}%
\pgfpathrectangle{\pgfqpoint{0.100000in}{0.212622in}}{\pgfqpoint{3.696000in}{3.696000in}}%
\pgfusepath{clip}%
\pgfsetbuttcap%
\pgfsetroundjoin%
\definecolor{currentfill}{rgb}{0.121569,0.466667,0.705882}%
\pgfsetfillcolor{currentfill}%
\pgfsetfillopacity{0.471328}%
\pgfsetlinewidth{1.003750pt}%
\definecolor{currentstroke}{rgb}{0.121569,0.466667,0.705882}%
\pgfsetstrokecolor{currentstroke}%
\pgfsetstrokeopacity{0.471328}%
\pgfsetdash{}{0pt}%
\pgfpathmoveto{\pgfqpoint{1.256648in}{1.737848in}}%
\pgfpathcurveto{\pgfqpoint{1.264884in}{1.737848in}}{\pgfqpoint{1.272784in}{1.741120in}}{\pgfqpoint{1.278608in}{1.746944in}}%
\pgfpathcurveto{\pgfqpoint{1.284432in}{1.752768in}}{\pgfqpoint{1.287704in}{1.760668in}}{\pgfqpoint{1.287704in}{1.768904in}}%
\pgfpathcurveto{\pgfqpoint{1.287704in}{1.777141in}}{\pgfqpoint{1.284432in}{1.785041in}}{\pgfqpoint{1.278608in}{1.790865in}}%
\pgfpathcurveto{\pgfqpoint{1.272784in}{1.796689in}}{\pgfqpoint{1.264884in}{1.799961in}}{\pgfqpoint{1.256648in}{1.799961in}}%
\pgfpathcurveto{\pgfqpoint{1.248411in}{1.799961in}}{\pgfqpoint{1.240511in}{1.796689in}}{\pgfqpoint{1.234687in}{1.790865in}}%
\pgfpathcurveto{\pgfqpoint{1.228863in}{1.785041in}}{\pgfqpoint{1.225591in}{1.777141in}}{\pgfqpoint{1.225591in}{1.768904in}}%
\pgfpathcurveto{\pgfqpoint{1.225591in}{1.760668in}}{\pgfqpoint{1.228863in}{1.752768in}}{\pgfqpoint{1.234687in}{1.746944in}}%
\pgfpathcurveto{\pgfqpoint{1.240511in}{1.741120in}}{\pgfqpoint{1.248411in}{1.737848in}}{\pgfqpoint{1.256648in}{1.737848in}}%
\pgfpathclose%
\pgfusepath{stroke,fill}%
\end{pgfscope}%
\begin{pgfscope}%
\pgfpathrectangle{\pgfqpoint{0.100000in}{0.212622in}}{\pgfqpoint{3.696000in}{3.696000in}}%
\pgfusepath{clip}%
\pgfsetbuttcap%
\pgfsetroundjoin%
\definecolor{currentfill}{rgb}{0.121569,0.466667,0.705882}%
\pgfsetfillcolor{currentfill}%
\pgfsetfillopacity{0.474845}%
\pgfsetlinewidth{1.003750pt}%
\definecolor{currentstroke}{rgb}{0.121569,0.466667,0.705882}%
\pgfsetstrokecolor{currentstroke}%
\pgfsetstrokeopacity{0.474845}%
\pgfsetdash{}{0pt}%
\pgfpathmoveto{\pgfqpoint{1.238765in}{1.741402in}}%
\pgfpathcurveto{\pgfqpoint{1.247002in}{1.741402in}}{\pgfqpoint{1.254902in}{1.744674in}}{\pgfqpoint{1.260726in}{1.750498in}}%
\pgfpathcurveto{\pgfqpoint{1.266550in}{1.756322in}}{\pgfqpoint{1.269822in}{1.764222in}}{\pgfqpoint{1.269822in}{1.772458in}}%
\pgfpathcurveto{\pgfqpoint{1.269822in}{1.780695in}}{\pgfqpoint{1.266550in}{1.788595in}}{\pgfqpoint{1.260726in}{1.794419in}}%
\pgfpathcurveto{\pgfqpoint{1.254902in}{1.800243in}}{\pgfqpoint{1.247002in}{1.803515in}}{\pgfqpoint{1.238765in}{1.803515in}}%
\pgfpathcurveto{\pgfqpoint{1.230529in}{1.803515in}}{\pgfqpoint{1.222629in}{1.800243in}}{\pgfqpoint{1.216805in}{1.794419in}}%
\pgfpathcurveto{\pgfqpoint{1.210981in}{1.788595in}}{\pgfqpoint{1.207709in}{1.780695in}}{\pgfqpoint{1.207709in}{1.772458in}}%
\pgfpathcurveto{\pgfqpoint{1.207709in}{1.764222in}}{\pgfqpoint{1.210981in}{1.756322in}}{\pgfqpoint{1.216805in}{1.750498in}}%
\pgfpathcurveto{\pgfqpoint{1.222629in}{1.744674in}}{\pgfqpoint{1.230529in}{1.741402in}}{\pgfqpoint{1.238765in}{1.741402in}}%
\pgfpathclose%
\pgfusepath{stroke,fill}%
\end{pgfscope}%
\begin{pgfscope}%
\pgfpathrectangle{\pgfqpoint{0.100000in}{0.212622in}}{\pgfqpoint{3.696000in}{3.696000in}}%
\pgfusepath{clip}%
\pgfsetbuttcap%
\pgfsetroundjoin%
\definecolor{currentfill}{rgb}{0.121569,0.466667,0.705882}%
\pgfsetfillcolor{currentfill}%
\pgfsetfillopacity{0.475443}%
\pgfsetlinewidth{1.003750pt}%
\definecolor{currentstroke}{rgb}{0.121569,0.466667,0.705882}%
\pgfsetstrokecolor{currentstroke}%
\pgfsetstrokeopacity{0.475443}%
\pgfsetdash{}{0pt}%
\pgfpathmoveto{\pgfqpoint{1.804915in}{1.996216in}}%
\pgfpathcurveto{\pgfqpoint{1.813152in}{1.996216in}}{\pgfqpoint{1.821052in}{1.999488in}}{\pgfqpoint{1.826876in}{2.005312in}}%
\pgfpathcurveto{\pgfqpoint{1.832700in}{2.011136in}}{\pgfqpoint{1.835972in}{2.019036in}}{\pgfqpoint{1.835972in}{2.027272in}}%
\pgfpathcurveto{\pgfqpoint{1.835972in}{2.035509in}}{\pgfqpoint{1.832700in}{2.043409in}}{\pgfqpoint{1.826876in}{2.049233in}}%
\pgfpathcurveto{\pgfqpoint{1.821052in}{2.055057in}}{\pgfqpoint{1.813152in}{2.058329in}}{\pgfqpoint{1.804915in}{2.058329in}}%
\pgfpathcurveto{\pgfqpoint{1.796679in}{2.058329in}}{\pgfqpoint{1.788779in}{2.055057in}}{\pgfqpoint{1.782955in}{2.049233in}}%
\pgfpathcurveto{\pgfqpoint{1.777131in}{2.043409in}}{\pgfqpoint{1.773859in}{2.035509in}}{\pgfqpoint{1.773859in}{2.027272in}}%
\pgfpathcurveto{\pgfqpoint{1.773859in}{2.019036in}}{\pgfqpoint{1.777131in}{2.011136in}}{\pgfqpoint{1.782955in}{2.005312in}}%
\pgfpathcurveto{\pgfqpoint{1.788779in}{1.999488in}}{\pgfqpoint{1.796679in}{1.996216in}}{\pgfqpoint{1.804915in}{1.996216in}}%
\pgfpathclose%
\pgfusepath{stroke,fill}%
\end{pgfscope}%
\begin{pgfscope}%
\pgfpathrectangle{\pgfqpoint{0.100000in}{0.212622in}}{\pgfqpoint{3.696000in}{3.696000in}}%
\pgfusepath{clip}%
\pgfsetbuttcap%
\pgfsetroundjoin%
\definecolor{currentfill}{rgb}{0.121569,0.466667,0.705882}%
\pgfsetfillcolor{currentfill}%
\pgfsetfillopacity{0.478076}%
\pgfsetlinewidth{1.003750pt}%
\definecolor{currentstroke}{rgb}{0.121569,0.466667,0.705882}%
\pgfsetstrokecolor{currentstroke}%
\pgfsetstrokeopacity{0.478076}%
\pgfsetdash{}{0pt}%
\pgfpathmoveto{\pgfqpoint{1.235539in}{1.738392in}}%
\pgfpathcurveto{\pgfqpoint{1.243776in}{1.738392in}}{\pgfqpoint{1.251676in}{1.741665in}}{\pgfqpoint{1.257500in}{1.747488in}}%
\pgfpathcurveto{\pgfqpoint{1.263324in}{1.753312in}}{\pgfqpoint{1.266596in}{1.761212in}}{\pgfqpoint{1.266596in}{1.769449in}}%
\pgfpathcurveto{\pgfqpoint{1.266596in}{1.777685in}}{\pgfqpoint{1.263324in}{1.785585in}}{\pgfqpoint{1.257500in}{1.791409in}}%
\pgfpathcurveto{\pgfqpoint{1.251676in}{1.797233in}}{\pgfqpoint{1.243776in}{1.800505in}}{\pgfqpoint{1.235539in}{1.800505in}}%
\pgfpathcurveto{\pgfqpoint{1.227303in}{1.800505in}}{\pgfqpoint{1.219403in}{1.797233in}}{\pgfqpoint{1.213579in}{1.791409in}}%
\pgfpathcurveto{\pgfqpoint{1.207755in}{1.785585in}}{\pgfqpoint{1.204483in}{1.777685in}}{\pgfqpoint{1.204483in}{1.769449in}}%
\pgfpathcurveto{\pgfqpoint{1.204483in}{1.761212in}}{\pgfqpoint{1.207755in}{1.753312in}}{\pgfqpoint{1.213579in}{1.747488in}}%
\pgfpathcurveto{\pgfqpoint{1.219403in}{1.741665in}}{\pgfqpoint{1.227303in}{1.738392in}}{\pgfqpoint{1.235539in}{1.738392in}}%
\pgfpathclose%
\pgfusepath{stroke,fill}%
\end{pgfscope}%
\begin{pgfscope}%
\pgfpathrectangle{\pgfqpoint{0.100000in}{0.212622in}}{\pgfqpoint{3.696000in}{3.696000in}}%
\pgfusepath{clip}%
\pgfsetbuttcap%
\pgfsetroundjoin%
\definecolor{currentfill}{rgb}{0.121569,0.466667,0.705882}%
\pgfsetfillcolor{currentfill}%
\pgfsetfillopacity{0.479356}%
\pgfsetlinewidth{1.003750pt}%
\definecolor{currentstroke}{rgb}{0.121569,0.466667,0.705882}%
\pgfsetstrokecolor{currentstroke}%
\pgfsetstrokeopacity{0.479356}%
\pgfsetdash{}{0pt}%
\pgfpathmoveto{\pgfqpoint{1.228596in}{1.739699in}}%
\pgfpathcurveto{\pgfqpoint{1.236832in}{1.739699in}}{\pgfqpoint{1.244733in}{1.742971in}}{\pgfqpoint{1.250556in}{1.748795in}}%
\pgfpathcurveto{\pgfqpoint{1.256380in}{1.754619in}}{\pgfqpoint{1.259653in}{1.762519in}}{\pgfqpoint{1.259653in}{1.770755in}}%
\pgfpathcurveto{\pgfqpoint{1.259653in}{1.778991in}}{\pgfqpoint{1.256380in}{1.786891in}}{\pgfqpoint{1.250556in}{1.792715in}}%
\pgfpathcurveto{\pgfqpoint{1.244733in}{1.798539in}}{\pgfqpoint{1.236832in}{1.801812in}}{\pgfqpoint{1.228596in}{1.801812in}}%
\pgfpathcurveto{\pgfqpoint{1.220360in}{1.801812in}}{\pgfqpoint{1.212460in}{1.798539in}}{\pgfqpoint{1.206636in}{1.792715in}}%
\pgfpathcurveto{\pgfqpoint{1.200812in}{1.786891in}}{\pgfqpoint{1.197540in}{1.778991in}}{\pgfqpoint{1.197540in}{1.770755in}}%
\pgfpathcurveto{\pgfqpoint{1.197540in}{1.762519in}}{\pgfqpoint{1.200812in}{1.754619in}}{\pgfqpoint{1.206636in}{1.748795in}}%
\pgfpathcurveto{\pgfqpoint{1.212460in}{1.742971in}}{\pgfqpoint{1.220360in}{1.739699in}}{\pgfqpoint{1.228596in}{1.739699in}}%
\pgfpathclose%
\pgfusepath{stroke,fill}%
\end{pgfscope}%
\begin{pgfscope}%
\pgfpathrectangle{\pgfqpoint{0.100000in}{0.212622in}}{\pgfqpoint{3.696000in}{3.696000in}}%
\pgfusepath{clip}%
\pgfsetbuttcap%
\pgfsetroundjoin%
\definecolor{currentfill}{rgb}{0.121569,0.466667,0.705882}%
\pgfsetfillcolor{currentfill}%
\pgfsetfillopacity{0.482517}%
\pgfsetlinewidth{1.003750pt}%
\definecolor{currentstroke}{rgb}{0.121569,0.466667,0.705882}%
\pgfsetstrokecolor{currentstroke}%
\pgfsetstrokeopacity{0.482517}%
\pgfsetdash{}{0pt}%
\pgfpathmoveto{\pgfqpoint{1.224693in}{1.735903in}}%
\pgfpathcurveto{\pgfqpoint{1.232930in}{1.735903in}}{\pgfqpoint{1.240830in}{1.739175in}}{\pgfqpoint{1.246654in}{1.744999in}}%
\pgfpathcurveto{\pgfqpoint{1.252478in}{1.750823in}}{\pgfqpoint{1.255750in}{1.758723in}}{\pgfqpoint{1.255750in}{1.766959in}}%
\pgfpathcurveto{\pgfqpoint{1.255750in}{1.775195in}}{\pgfqpoint{1.252478in}{1.783095in}}{\pgfqpoint{1.246654in}{1.788919in}}%
\pgfpathcurveto{\pgfqpoint{1.240830in}{1.794743in}}{\pgfqpoint{1.232930in}{1.798016in}}{\pgfqpoint{1.224693in}{1.798016in}}%
\pgfpathcurveto{\pgfqpoint{1.216457in}{1.798016in}}{\pgfqpoint{1.208557in}{1.794743in}}{\pgfqpoint{1.202733in}{1.788919in}}%
\pgfpathcurveto{\pgfqpoint{1.196909in}{1.783095in}}{\pgfqpoint{1.193637in}{1.775195in}}{\pgfqpoint{1.193637in}{1.766959in}}%
\pgfpathcurveto{\pgfqpoint{1.193637in}{1.758723in}}{\pgfqpoint{1.196909in}{1.750823in}}{\pgfqpoint{1.202733in}{1.744999in}}%
\pgfpathcurveto{\pgfqpoint{1.208557in}{1.739175in}}{\pgfqpoint{1.216457in}{1.735903in}}{\pgfqpoint{1.224693in}{1.735903in}}%
\pgfpathclose%
\pgfusepath{stroke,fill}%
\end{pgfscope}%
\begin{pgfscope}%
\pgfpathrectangle{\pgfqpoint{0.100000in}{0.212622in}}{\pgfqpoint{3.696000in}{3.696000in}}%
\pgfusepath{clip}%
\pgfsetbuttcap%
\pgfsetroundjoin%
\definecolor{currentfill}{rgb}{0.121569,0.466667,0.705882}%
\pgfsetfillcolor{currentfill}%
\pgfsetfillopacity{0.486950}%
\pgfsetlinewidth{1.003750pt}%
\definecolor{currentstroke}{rgb}{0.121569,0.466667,0.705882}%
\pgfsetstrokecolor{currentstroke}%
\pgfsetstrokeopacity{0.486950}%
\pgfsetdash{}{0pt}%
\pgfpathmoveto{\pgfqpoint{1.201662in}{1.741435in}}%
\pgfpathcurveto{\pgfqpoint{1.209898in}{1.741435in}}{\pgfqpoint{1.217798in}{1.744707in}}{\pgfqpoint{1.223622in}{1.750531in}}%
\pgfpathcurveto{\pgfqpoint{1.229446in}{1.756355in}}{\pgfqpoint{1.232718in}{1.764255in}}{\pgfqpoint{1.232718in}{1.772492in}}%
\pgfpathcurveto{\pgfqpoint{1.232718in}{1.780728in}}{\pgfqpoint{1.229446in}{1.788628in}}{\pgfqpoint{1.223622in}{1.794452in}}%
\pgfpathcurveto{\pgfqpoint{1.217798in}{1.800276in}}{\pgfqpoint{1.209898in}{1.803548in}}{\pgfqpoint{1.201662in}{1.803548in}}%
\pgfpathcurveto{\pgfqpoint{1.193425in}{1.803548in}}{\pgfqpoint{1.185525in}{1.800276in}}{\pgfqpoint{1.179701in}{1.794452in}}%
\pgfpathcurveto{\pgfqpoint{1.173877in}{1.788628in}}{\pgfqpoint{1.170605in}{1.780728in}}{\pgfqpoint{1.170605in}{1.772492in}}%
\pgfpathcurveto{\pgfqpoint{1.170605in}{1.764255in}}{\pgfqpoint{1.173877in}{1.756355in}}{\pgfqpoint{1.179701in}{1.750531in}}%
\pgfpathcurveto{\pgfqpoint{1.185525in}{1.744707in}}{\pgfqpoint{1.193425in}{1.741435in}}{\pgfqpoint{1.201662in}{1.741435in}}%
\pgfpathclose%
\pgfusepath{stroke,fill}%
\end{pgfscope}%
\begin{pgfscope}%
\pgfpathrectangle{\pgfqpoint{0.100000in}{0.212622in}}{\pgfqpoint{3.696000in}{3.696000in}}%
\pgfusepath{clip}%
\pgfsetbuttcap%
\pgfsetroundjoin%
\definecolor{currentfill}{rgb}{0.121569,0.466667,0.705882}%
\pgfsetfillcolor{currentfill}%
\pgfsetfillopacity{0.487783}%
\pgfsetlinewidth{1.003750pt}%
\definecolor{currentstroke}{rgb}{0.121569,0.466667,0.705882}%
\pgfsetstrokecolor{currentstroke}%
\pgfsetstrokeopacity{0.487783}%
\pgfsetdash{}{0pt}%
\pgfpathmoveto{\pgfqpoint{1.831320in}{2.003278in}}%
\pgfpathcurveto{\pgfqpoint{1.839556in}{2.003278in}}{\pgfqpoint{1.847456in}{2.006550in}}{\pgfqpoint{1.853280in}{2.012374in}}%
\pgfpathcurveto{\pgfqpoint{1.859104in}{2.018198in}}{\pgfqpoint{1.862377in}{2.026098in}}{\pgfqpoint{1.862377in}{2.034335in}}%
\pgfpathcurveto{\pgfqpoint{1.862377in}{2.042571in}}{\pgfqpoint{1.859104in}{2.050471in}}{\pgfqpoint{1.853280in}{2.056295in}}%
\pgfpathcurveto{\pgfqpoint{1.847456in}{2.062119in}}{\pgfqpoint{1.839556in}{2.065391in}}{\pgfqpoint{1.831320in}{2.065391in}}%
\pgfpathcurveto{\pgfqpoint{1.823084in}{2.065391in}}{\pgfqpoint{1.815184in}{2.062119in}}{\pgfqpoint{1.809360in}{2.056295in}}%
\pgfpathcurveto{\pgfqpoint{1.803536in}{2.050471in}}{\pgfqpoint{1.800264in}{2.042571in}}{\pgfqpoint{1.800264in}{2.034335in}}%
\pgfpathcurveto{\pgfqpoint{1.800264in}{2.026098in}}{\pgfqpoint{1.803536in}{2.018198in}}{\pgfqpoint{1.809360in}{2.012374in}}%
\pgfpathcurveto{\pgfqpoint{1.815184in}{2.006550in}}{\pgfqpoint{1.823084in}{2.003278in}}{\pgfqpoint{1.831320in}{2.003278in}}%
\pgfpathclose%
\pgfusepath{stroke,fill}%
\end{pgfscope}%
\begin{pgfscope}%
\pgfpathrectangle{\pgfqpoint{0.100000in}{0.212622in}}{\pgfqpoint{3.696000in}{3.696000in}}%
\pgfusepath{clip}%
\pgfsetbuttcap%
\pgfsetroundjoin%
\definecolor{currentfill}{rgb}{0.121569,0.466667,0.705882}%
\pgfsetfillcolor{currentfill}%
\pgfsetfillopacity{0.492036}%
\pgfsetlinewidth{1.003750pt}%
\definecolor{currentstroke}{rgb}{0.121569,0.466667,0.705882}%
\pgfsetstrokecolor{currentstroke}%
\pgfsetstrokeopacity{0.492036}%
\pgfsetdash{}{0pt}%
\pgfpathmoveto{\pgfqpoint{1.196250in}{1.735788in}}%
\pgfpathcurveto{\pgfqpoint{1.204487in}{1.735788in}}{\pgfqpoint{1.212387in}{1.739060in}}{\pgfqpoint{1.218211in}{1.744884in}}%
\pgfpathcurveto{\pgfqpoint{1.224035in}{1.750708in}}{\pgfqpoint{1.227307in}{1.758608in}}{\pgfqpoint{1.227307in}{1.766844in}}%
\pgfpathcurveto{\pgfqpoint{1.227307in}{1.775081in}}{\pgfqpoint{1.224035in}{1.782981in}}{\pgfqpoint{1.218211in}{1.788805in}}%
\pgfpathcurveto{\pgfqpoint{1.212387in}{1.794628in}}{\pgfqpoint{1.204487in}{1.797901in}}{\pgfqpoint{1.196250in}{1.797901in}}%
\pgfpathcurveto{\pgfqpoint{1.188014in}{1.797901in}}{\pgfqpoint{1.180114in}{1.794628in}}{\pgfqpoint{1.174290in}{1.788805in}}%
\pgfpathcurveto{\pgfqpoint{1.168466in}{1.782981in}}{\pgfqpoint{1.165194in}{1.775081in}}{\pgfqpoint{1.165194in}{1.766844in}}%
\pgfpathcurveto{\pgfqpoint{1.165194in}{1.758608in}}{\pgfqpoint{1.168466in}{1.750708in}}{\pgfqpoint{1.174290in}{1.744884in}}%
\pgfpathcurveto{\pgfqpoint{1.180114in}{1.739060in}}{\pgfqpoint{1.188014in}{1.735788in}}{\pgfqpoint{1.196250in}{1.735788in}}%
\pgfpathclose%
\pgfusepath{stroke,fill}%
\end{pgfscope}%
\begin{pgfscope}%
\pgfpathrectangle{\pgfqpoint{0.100000in}{0.212622in}}{\pgfqpoint{3.696000in}{3.696000in}}%
\pgfusepath{clip}%
\pgfsetbuttcap%
\pgfsetroundjoin%
\definecolor{currentfill}{rgb}{0.121569,0.466667,0.705882}%
\pgfsetfillcolor{currentfill}%
\pgfsetfillopacity{0.492591}%
\pgfsetlinewidth{1.003750pt}%
\definecolor{currentstroke}{rgb}{0.121569,0.466667,0.705882}%
\pgfsetstrokecolor{currentstroke}%
\pgfsetstrokeopacity{0.492591}%
\pgfsetdash{}{0pt}%
\pgfpathmoveto{\pgfqpoint{1.837222in}{1.992218in}}%
\pgfpathcurveto{\pgfqpoint{1.845458in}{1.992218in}}{\pgfqpoint{1.853358in}{1.995490in}}{\pgfqpoint{1.859182in}{2.001314in}}%
\pgfpathcurveto{\pgfqpoint{1.865006in}{2.007138in}}{\pgfqpoint{1.868278in}{2.015038in}}{\pgfqpoint{1.868278in}{2.023274in}}%
\pgfpathcurveto{\pgfqpoint{1.868278in}{2.031511in}}{\pgfqpoint{1.865006in}{2.039411in}}{\pgfqpoint{1.859182in}{2.045235in}}%
\pgfpathcurveto{\pgfqpoint{1.853358in}{2.051059in}}{\pgfqpoint{1.845458in}{2.054331in}}{\pgfqpoint{1.837222in}{2.054331in}}%
\pgfpathcurveto{\pgfqpoint{1.828985in}{2.054331in}}{\pgfqpoint{1.821085in}{2.051059in}}{\pgfqpoint{1.815261in}{2.045235in}}%
\pgfpathcurveto{\pgfqpoint{1.809437in}{2.039411in}}{\pgfqpoint{1.806165in}{2.031511in}}{\pgfqpoint{1.806165in}{2.023274in}}%
\pgfpathcurveto{\pgfqpoint{1.806165in}{2.015038in}}{\pgfqpoint{1.809437in}{2.007138in}}{\pgfqpoint{1.815261in}{2.001314in}}%
\pgfpathcurveto{\pgfqpoint{1.821085in}{1.995490in}}{\pgfqpoint{1.828985in}{1.992218in}}{\pgfqpoint{1.837222in}{1.992218in}}%
\pgfpathclose%
\pgfusepath{stroke,fill}%
\end{pgfscope}%
\begin{pgfscope}%
\pgfpathrectangle{\pgfqpoint{0.100000in}{0.212622in}}{\pgfqpoint{3.696000in}{3.696000in}}%
\pgfusepath{clip}%
\pgfsetbuttcap%
\pgfsetroundjoin%
\definecolor{currentfill}{rgb}{0.121569,0.466667,0.705882}%
\pgfsetfillcolor{currentfill}%
\pgfsetfillopacity{0.498618}%
\pgfsetlinewidth{1.003750pt}%
\definecolor{currentstroke}{rgb}{0.121569,0.466667,0.705882}%
\pgfsetstrokecolor{currentstroke}%
\pgfsetstrokeopacity{0.498618}%
\pgfsetdash{}{0pt}%
\pgfpathmoveto{\pgfqpoint{1.161977in}{1.738132in}}%
\pgfpathcurveto{\pgfqpoint{1.170213in}{1.738132in}}{\pgfqpoint{1.178113in}{1.741404in}}{\pgfqpoint{1.183937in}{1.747228in}}%
\pgfpathcurveto{\pgfqpoint{1.189761in}{1.753052in}}{\pgfqpoint{1.193033in}{1.760952in}}{\pgfqpoint{1.193033in}{1.769188in}}%
\pgfpathcurveto{\pgfqpoint{1.193033in}{1.777424in}}{\pgfqpoint{1.189761in}{1.785324in}}{\pgfqpoint{1.183937in}{1.791148in}}%
\pgfpathcurveto{\pgfqpoint{1.178113in}{1.796972in}}{\pgfqpoint{1.170213in}{1.800244in}}{\pgfqpoint{1.161977in}{1.800244in}}%
\pgfpathcurveto{\pgfqpoint{1.153741in}{1.800244in}}{\pgfqpoint{1.145841in}{1.796972in}}{\pgfqpoint{1.140017in}{1.791148in}}%
\pgfpathcurveto{\pgfqpoint{1.134193in}{1.785324in}}{\pgfqpoint{1.130920in}{1.777424in}}{\pgfqpoint{1.130920in}{1.769188in}}%
\pgfpathcurveto{\pgfqpoint{1.130920in}{1.760952in}}{\pgfqpoint{1.134193in}{1.753052in}}{\pgfqpoint{1.140017in}{1.747228in}}%
\pgfpathcurveto{\pgfqpoint{1.145841in}{1.741404in}}{\pgfqpoint{1.153741in}{1.738132in}}{\pgfqpoint{1.161977in}{1.738132in}}%
\pgfpathclose%
\pgfusepath{stroke,fill}%
\end{pgfscope}%
\begin{pgfscope}%
\pgfpathrectangle{\pgfqpoint{0.100000in}{0.212622in}}{\pgfqpoint{3.696000in}{3.696000in}}%
\pgfusepath{clip}%
\pgfsetbuttcap%
\pgfsetroundjoin%
\definecolor{currentfill}{rgb}{0.121569,0.466667,0.705882}%
\pgfsetfillcolor{currentfill}%
\pgfsetfillopacity{0.500108}%
\pgfsetlinewidth{1.003750pt}%
\definecolor{currentstroke}{rgb}{0.121569,0.466667,0.705882}%
\pgfsetstrokecolor{currentstroke}%
\pgfsetstrokeopacity{0.500108}%
\pgfsetdash{}{0pt}%
\pgfpathmoveto{\pgfqpoint{1.850540in}{1.995143in}}%
\pgfpathcurveto{\pgfqpoint{1.858776in}{1.995143in}}{\pgfqpoint{1.866676in}{1.998415in}}{\pgfqpoint{1.872500in}{2.004239in}}%
\pgfpathcurveto{\pgfqpoint{1.878324in}{2.010063in}}{\pgfqpoint{1.881597in}{2.017963in}}{\pgfqpoint{1.881597in}{2.026199in}}%
\pgfpathcurveto{\pgfqpoint{1.881597in}{2.034435in}}{\pgfqpoint{1.878324in}{2.042335in}}{\pgfqpoint{1.872500in}{2.048159in}}%
\pgfpathcurveto{\pgfqpoint{1.866676in}{2.053983in}}{\pgfqpoint{1.858776in}{2.057256in}}{\pgfqpoint{1.850540in}{2.057256in}}%
\pgfpathcurveto{\pgfqpoint{1.842304in}{2.057256in}}{\pgfqpoint{1.834404in}{2.053983in}}{\pgfqpoint{1.828580in}{2.048159in}}%
\pgfpathcurveto{\pgfqpoint{1.822756in}{2.042335in}}{\pgfqpoint{1.819484in}{2.034435in}}{\pgfqpoint{1.819484in}{2.026199in}}%
\pgfpathcurveto{\pgfqpoint{1.819484in}{2.017963in}}{\pgfqpoint{1.822756in}{2.010063in}}{\pgfqpoint{1.828580in}{2.004239in}}%
\pgfpathcurveto{\pgfqpoint{1.834404in}{1.998415in}}{\pgfqpoint{1.842304in}{1.995143in}}{\pgfqpoint{1.850540in}{1.995143in}}%
\pgfpathclose%
\pgfusepath{stroke,fill}%
\end{pgfscope}%
\begin{pgfscope}%
\pgfpathrectangle{\pgfqpoint{0.100000in}{0.212622in}}{\pgfqpoint{3.696000in}{3.696000in}}%
\pgfusepath{clip}%
\pgfsetbuttcap%
\pgfsetroundjoin%
\definecolor{currentfill}{rgb}{0.121569,0.466667,0.705882}%
\pgfsetfillcolor{currentfill}%
\pgfsetfillopacity{0.505935}%
\pgfsetlinewidth{1.003750pt}%
\definecolor{currentstroke}{rgb}{0.121569,0.466667,0.705882}%
\pgfsetstrokecolor{currentstroke}%
\pgfsetstrokeopacity{0.505935}%
\pgfsetdash{}{0pt}%
\pgfpathmoveto{\pgfqpoint{1.856915in}{1.981754in}}%
\pgfpathcurveto{\pgfqpoint{1.865152in}{1.981754in}}{\pgfqpoint{1.873052in}{1.985026in}}{\pgfqpoint{1.878876in}{1.990850in}}%
\pgfpathcurveto{\pgfqpoint{1.884700in}{1.996674in}}{\pgfqpoint{1.887972in}{2.004574in}}{\pgfqpoint{1.887972in}{2.012810in}}%
\pgfpathcurveto{\pgfqpoint{1.887972in}{2.021047in}}{\pgfqpoint{1.884700in}{2.028947in}}{\pgfqpoint{1.878876in}{2.034771in}}%
\pgfpathcurveto{\pgfqpoint{1.873052in}{2.040595in}}{\pgfqpoint{1.865152in}{2.043867in}}{\pgfqpoint{1.856915in}{2.043867in}}%
\pgfpathcurveto{\pgfqpoint{1.848679in}{2.043867in}}{\pgfqpoint{1.840779in}{2.040595in}}{\pgfqpoint{1.834955in}{2.034771in}}%
\pgfpathcurveto{\pgfqpoint{1.829131in}{2.028947in}}{\pgfqpoint{1.825859in}{2.021047in}}{\pgfqpoint{1.825859in}{2.012810in}}%
\pgfpathcurveto{\pgfqpoint{1.825859in}{2.004574in}}{\pgfqpoint{1.829131in}{1.996674in}}{\pgfqpoint{1.834955in}{1.990850in}}%
\pgfpathcurveto{\pgfqpoint{1.840779in}{1.985026in}}{\pgfqpoint{1.848679in}{1.981754in}}{\pgfqpoint{1.856915in}{1.981754in}}%
\pgfpathclose%
\pgfusepath{stroke,fill}%
\end{pgfscope}%
\begin{pgfscope}%
\pgfpathrectangle{\pgfqpoint{0.100000in}{0.212622in}}{\pgfqpoint{3.696000in}{3.696000in}}%
\pgfusepath{clip}%
\pgfsetbuttcap%
\pgfsetroundjoin%
\definecolor{currentfill}{rgb}{0.121569,0.466667,0.705882}%
\pgfsetfillcolor{currentfill}%
\pgfsetfillopacity{0.506591}%
\pgfsetlinewidth{1.003750pt}%
\definecolor{currentstroke}{rgb}{0.121569,0.466667,0.705882}%
\pgfsetstrokecolor{currentstroke}%
\pgfsetstrokeopacity{0.506591}%
\pgfsetdash{}{0pt}%
\pgfpathmoveto{\pgfqpoint{1.154572in}{1.726807in}}%
\pgfpathcurveto{\pgfqpoint{1.162808in}{1.726807in}}{\pgfqpoint{1.170709in}{1.730080in}}{\pgfqpoint{1.176532in}{1.735903in}}%
\pgfpathcurveto{\pgfqpoint{1.182356in}{1.741727in}}{\pgfqpoint{1.185629in}{1.749627in}}{\pgfqpoint{1.185629in}{1.757864in}}%
\pgfpathcurveto{\pgfqpoint{1.185629in}{1.766100in}}{\pgfqpoint{1.182356in}{1.774000in}}{\pgfqpoint{1.176532in}{1.779824in}}%
\pgfpathcurveto{\pgfqpoint{1.170709in}{1.785648in}}{\pgfqpoint{1.162808in}{1.788920in}}{\pgfqpoint{1.154572in}{1.788920in}}%
\pgfpathcurveto{\pgfqpoint{1.146336in}{1.788920in}}{\pgfqpoint{1.138436in}{1.785648in}}{\pgfqpoint{1.132612in}{1.779824in}}%
\pgfpathcurveto{\pgfqpoint{1.126788in}{1.774000in}}{\pgfqpoint{1.123516in}{1.766100in}}{\pgfqpoint{1.123516in}{1.757864in}}%
\pgfpathcurveto{\pgfqpoint{1.123516in}{1.749627in}}{\pgfqpoint{1.126788in}{1.741727in}}{\pgfqpoint{1.132612in}{1.735903in}}%
\pgfpathcurveto{\pgfqpoint{1.138436in}{1.730080in}}{\pgfqpoint{1.146336in}{1.726807in}}{\pgfqpoint{1.154572in}{1.726807in}}%
\pgfpathclose%
\pgfusepath{stroke,fill}%
\end{pgfscope}%
\begin{pgfscope}%
\pgfpathrectangle{\pgfqpoint{0.100000in}{0.212622in}}{\pgfqpoint{3.696000in}{3.696000in}}%
\pgfusepath{clip}%
\pgfsetbuttcap%
\pgfsetroundjoin%
\definecolor{currentfill}{rgb}{0.121569,0.466667,0.705882}%
\pgfsetfillcolor{currentfill}%
\pgfsetfillopacity{0.510820}%
\pgfsetlinewidth{1.003750pt}%
\definecolor{currentstroke}{rgb}{0.121569,0.466667,0.705882}%
\pgfsetstrokecolor{currentstroke}%
\pgfsetstrokeopacity{0.510820}%
\pgfsetdash{}{0pt}%
\pgfpathmoveto{\pgfqpoint{1.133765in}{1.720747in}}%
\pgfpathcurveto{\pgfqpoint{1.142002in}{1.720747in}}{\pgfqpoint{1.149902in}{1.724019in}}{\pgfqpoint{1.155726in}{1.729843in}}%
\pgfpathcurveto{\pgfqpoint{1.161549in}{1.735667in}}{\pgfqpoint{1.164822in}{1.743567in}}{\pgfqpoint{1.164822in}{1.751803in}}%
\pgfpathcurveto{\pgfqpoint{1.164822in}{1.760039in}}{\pgfqpoint{1.161549in}{1.767940in}}{\pgfqpoint{1.155726in}{1.773763in}}%
\pgfpathcurveto{\pgfqpoint{1.149902in}{1.779587in}}{\pgfqpoint{1.142002in}{1.782860in}}{\pgfqpoint{1.133765in}{1.782860in}}%
\pgfpathcurveto{\pgfqpoint{1.125529in}{1.782860in}}{\pgfqpoint{1.117629in}{1.779587in}}{\pgfqpoint{1.111805in}{1.773763in}}%
\pgfpathcurveto{\pgfqpoint{1.105981in}{1.767940in}}{\pgfqpoint{1.102709in}{1.760039in}}{\pgfqpoint{1.102709in}{1.751803in}}%
\pgfpathcurveto{\pgfqpoint{1.102709in}{1.743567in}}{\pgfqpoint{1.105981in}{1.735667in}}{\pgfqpoint{1.111805in}{1.729843in}}%
\pgfpathcurveto{\pgfqpoint{1.117629in}{1.724019in}}{\pgfqpoint{1.125529in}{1.720747in}}{\pgfqpoint{1.133765in}{1.720747in}}%
\pgfpathclose%
\pgfusepath{stroke,fill}%
\end{pgfscope}%
\begin{pgfscope}%
\pgfpathrectangle{\pgfqpoint{0.100000in}{0.212622in}}{\pgfqpoint{3.696000in}{3.696000in}}%
\pgfusepath{clip}%
\pgfsetbuttcap%
\pgfsetroundjoin%
\definecolor{currentfill}{rgb}{0.121569,0.466667,0.705882}%
\pgfsetfillcolor{currentfill}%
\pgfsetfillopacity{0.515622}%
\pgfsetlinewidth{1.003750pt}%
\definecolor{currentstroke}{rgb}{0.121569,0.466667,0.705882}%
\pgfsetstrokecolor{currentstroke}%
\pgfsetstrokeopacity{0.515622}%
\pgfsetdash{}{0pt}%
\pgfpathmoveto{\pgfqpoint{1.876817in}{1.990228in}}%
\pgfpathcurveto{\pgfqpoint{1.885054in}{1.990228in}}{\pgfqpoint{1.892954in}{1.993500in}}{\pgfqpoint{1.898778in}{1.999324in}}%
\pgfpathcurveto{\pgfqpoint{1.904602in}{2.005148in}}{\pgfqpoint{1.907874in}{2.013048in}}{\pgfqpoint{1.907874in}{2.021284in}}%
\pgfpathcurveto{\pgfqpoint{1.907874in}{2.029520in}}{\pgfqpoint{1.904602in}{2.037420in}}{\pgfqpoint{1.898778in}{2.043244in}}%
\pgfpathcurveto{\pgfqpoint{1.892954in}{2.049068in}}{\pgfqpoint{1.885054in}{2.052341in}}{\pgfqpoint{1.876817in}{2.052341in}}%
\pgfpathcurveto{\pgfqpoint{1.868581in}{2.052341in}}{\pgfqpoint{1.860681in}{2.049068in}}{\pgfqpoint{1.854857in}{2.043244in}}%
\pgfpathcurveto{\pgfqpoint{1.849033in}{2.037420in}}{\pgfqpoint{1.845761in}{2.029520in}}{\pgfqpoint{1.845761in}{2.021284in}}%
\pgfpathcurveto{\pgfqpoint{1.845761in}{2.013048in}}{\pgfqpoint{1.849033in}{2.005148in}}{\pgfqpoint{1.854857in}{1.999324in}}%
\pgfpathcurveto{\pgfqpoint{1.860681in}{1.993500in}}{\pgfqpoint{1.868581in}{1.990228in}}{\pgfqpoint{1.876817in}{1.990228in}}%
\pgfpathclose%
\pgfusepath{stroke,fill}%
\end{pgfscope}%
\begin{pgfscope}%
\pgfpathrectangle{\pgfqpoint{0.100000in}{0.212622in}}{\pgfqpoint{3.696000in}{3.696000in}}%
\pgfusepath{clip}%
\pgfsetbuttcap%
\pgfsetroundjoin%
\definecolor{currentfill}{rgb}{0.121569,0.466667,0.705882}%
\pgfsetfillcolor{currentfill}%
\pgfsetfillopacity{0.517072}%
\pgfsetlinewidth{1.003750pt}%
\definecolor{currentstroke}{rgb}{0.121569,0.466667,0.705882}%
\pgfsetstrokecolor{currentstroke}%
\pgfsetstrokeopacity{0.517072}%
\pgfsetdash{}{0pt}%
\pgfpathmoveto{\pgfqpoint{1.130144in}{1.715589in}}%
\pgfpathcurveto{\pgfqpoint{1.138380in}{1.715589in}}{\pgfqpoint{1.146280in}{1.718861in}}{\pgfqpoint{1.152104in}{1.724685in}}%
\pgfpathcurveto{\pgfqpoint{1.157928in}{1.730509in}}{\pgfqpoint{1.161201in}{1.738409in}}{\pgfqpoint{1.161201in}{1.746645in}}%
\pgfpathcurveto{\pgfqpoint{1.161201in}{1.754881in}}{\pgfqpoint{1.157928in}{1.762781in}}{\pgfqpoint{1.152104in}{1.768605in}}%
\pgfpathcurveto{\pgfqpoint{1.146280in}{1.774429in}}{\pgfqpoint{1.138380in}{1.777702in}}{\pgfqpoint{1.130144in}{1.777702in}}%
\pgfpathcurveto{\pgfqpoint{1.121908in}{1.777702in}}{\pgfqpoint{1.114008in}{1.774429in}}{\pgfqpoint{1.108184in}{1.768605in}}%
\pgfpathcurveto{\pgfqpoint{1.102360in}{1.762781in}}{\pgfqpoint{1.099088in}{1.754881in}}{\pgfqpoint{1.099088in}{1.746645in}}%
\pgfpathcurveto{\pgfqpoint{1.099088in}{1.738409in}}{\pgfqpoint{1.102360in}{1.730509in}}{\pgfqpoint{1.108184in}{1.724685in}}%
\pgfpathcurveto{\pgfqpoint{1.114008in}{1.718861in}}{\pgfqpoint{1.121908in}{1.715589in}}{\pgfqpoint{1.130144in}{1.715589in}}%
\pgfpathclose%
\pgfusepath{stroke,fill}%
\end{pgfscope}%
\begin{pgfscope}%
\pgfpathrectangle{\pgfqpoint{0.100000in}{0.212622in}}{\pgfqpoint{3.696000in}{3.696000in}}%
\pgfusepath{clip}%
\pgfsetbuttcap%
\pgfsetroundjoin%
\definecolor{currentfill}{rgb}{0.121569,0.466667,0.705882}%
\pgfsetfillcolor{currentfill}%
\pgfsetfillopacity{0.520928}%
\pgfsetlinewidth{1.003750pt}%
\definecolor{currentstroke}{rgb}{0.121569,0.466667,0.705882}%
\pgfsetstrokecolor{currentstroke}%
\pgfsetstrokeopacity{0.520928}%
\pgfsetdash{}{0pt}%
\pgfpathmoveto{\pgfqpoint{1.113677in}{1.717037in}}%
\pgfpathcurveto{\pgfqpoint{1.121914in}{1.717037in}}{\pgfqpoint{1.129814in}{1.720309in}}{\pgfqpoint{1.135638in}{1.726133in}}%
\pgfpathcurveto{\pgfqpoint{1.141462in}{1.731957in}}{\pgfqpoint{1.144734in}{1.739857in}}{\pgfqpoint{1.144734in}{1.748093in}}%
\pgfpathcurveto{\pgfqpoint{1.144734in}{1.756330in}}{\pgfqpoint{1.141462in}{1.764230in}}{\pgfqpoint{1.135638in}{1.770054in}}%
\pgfpathcurveto{\pgfqpoint{1.129814in}{1.775878in}}{\pgfqpoint{1.121914in}{1.779150in}}{\pgfqpoint{1.113677in}{1.779150in}}%
\pgfpathcurveto{\pgfqpoint{1.105441in}{1.779150in}}{\pgfqpoint{1.097541in}{1.775878in}}{\pgfqpoint{1.091717in}{1.770054in}}%
\pgfpathcurveto{\pgfqpoint{1.085893in}{1.764230in}}{\pgfqpoint{1.082621in}{1.756330in}}{\pgfqpoint{1.082621in}{1.748093in}}%
\pgfpathcurveto{\pgfqpoint{1.082621in}{1.739857in}}{\pgfqpoint{1.085893in}{1.731957in}}{\pgfqpoint{1.091717in}{1.726133in}}%
\pgfpathcurveto{\pgfqpoint{1.097541in}{1.720309in}}{\pgfqpoint{1.105441in}{1.717037in}}{\pgfqpoint{1.113677in}{1.717037in}}%
\pgfpathclose%
\pgfusepath{stroke,fill}%
\end{pgfscope}%
\begin{pgfscope}%
\pgfpathrectangle{\pgfqpoint{0.100000in}{0.212622in}}{\pgfqpoint{3.696000in}{3.696000in}}%
\pgfusepath{clip}%
\pgfsetbuttcap%
\pgfsetroundjoin%
\definecolor{currentfill}{rgb}{0.121569,0.466667,0.705882}%
\pgfsetfillcolor{currentfill}%
\pgfsetfillopacity{0.523664}%
\pgfsetlinewidth{1.003750pt}%
\definecolor{currentstroke}{rgb}{0.121569,0.466667,0.705882}%
\pgfsetstrokecolor{currentstroke}%
\pgfsetstrokeopacity{0.523664}%
\pgfsetdash{}{0pt}%
\pgfpathmoveto{\pgfqpoint{1.882119in}{1.977810in}}%
\pgfpathcurveto{\pgfqpoint{1.890355in}{1.977810in}}{\pgfqpoint{1.898255in}{1.981083in}}{\pgfqpoint{1.904079in}{1.986906in}}%
\pgfpathcurveto{\pgfqpoint{1.909903in}{1.992730in}}{\pgfqpoint{1.913176in}{2.000630in}}{\pgfqpoint{1.913176in}{2.008867in}}%
\pgfpathcurveto{\pgfqpoint{1.913176in}{2.017103in}}{\pgfqpoint{1.909903in}{2.025003in}}{\pgfqpoint{1.904079in}{2.030827in}}%
\pgfpathcurveto{\pgfqpoint{1.898255in}{2.036651in}}{\pgfqpoint{1.890355in}{2.039923in}}{\pgfqpoint{1.882119in}{2.039923in}}%
\pgfpathcurveto{\pgfqpoint{1.873883in}{2.039923in}}{\pgfqpoint{1.865983in}{2.036651in}}{\pgfqpoint{1.860159in}{2.030827in}}%
\pgfpathcurveto{\pgfqpoint{1.854335in}{2.025003in}}{\pgfqpoint{1.851063in}{2.017103in}}{\pgfqpoint{1.851063in}{2.008867in}}%
\pgfpathcurveto{\pgfqpoint{1.851063in}{2.000630in}}{\pgfqpoint{1.854335in}{1.992730in}}{\pgfqpoint{1.860159in}{1.986906in}}%
\pgfpathcurveto{\pgfqpoint{1.865983in}{1.981083in}}{\pgfqpoint{1.873883in}{1.977810in}}{\pgfqpoint{1.882119in}{1.977810in}}%
\pgfpathclose%
\pgfusepath{stroke,fill}%
\end{pgfscope}%
\begin{pgfscope}%
\pgfpathrectangle{\pgfqpoint{0.100000in}{0.212622in}}{\pgfqpoint{3.696000in}{3.696000in}}%
\pgfusepath{clip}%
\pgfsetbuttcap%
\pgfsetroundjoin%
\definecolor{currentfill}{rgb}{0.121569,0.466667,0.705882}%
\pgfsetfillcolor{currentfill}%
\pgfsetfillopacity{0.524699}%
\pgfsetlinewidth{1.003750pt}%
\definecolor{currentstroke}{rgb}{0.121569,0.466667,0.705882}%
\pgfsetstrokecolor{currentstroke}%
\pgfsetstrokeopacity{0.524699}%
\pgfsetdash{}{0pt}%
\pgfpathmoveto{\pgfqpoint{1.108814in}{1.711803in}}%
\pgfpathcurveto{\pgfqpoint{1.117051in}{1.711803in}}{\pgfqpoint{1.124951in}{1.715075in}}{\pgfqpoint{1.130775in}{1.720899in}}%
\pgfpathcurveto{\pgfqpoint{1.136599in}{1.726723in}}{\pgfqpoint{1.139871in}{1.734623in}}{\pgfqpoint{1.139871in}{1.742859in}}%
\pgfpathcurveto{\pgfqpoint{1.139871in}{1.751095in}}{\pgfqpoint{1.136599in}{1.758995in}}{\pgfqpoint{1.130775in}{1.764819in}}%
\pgfpathcurveto{\pgfqpoint{1.124951in}{1.770643in}}{\pgfqpoint{1.117051in}{1.773916in}}{\pgfqpoint{1.108814in}{1.773916in}}%
\pgfpathcurveto{\pgfqpoint{1.100578in}{1.773916in}}{\pgfqpoint{1.092678in}{1.770643in}}{\pgfqpoint{1.086854in}{1.764819in}}%
\pgfpathcurveto{\pgfqpoint{1.081030in}{1.758995in}}{\pgfqpoint{1.077758in}{1.751095in}}{\pgfqpoint{1.077758in}{1.742859in}}%
\pgfpathcurveto{\pgfqpoint{1.077758in}{1.734623in}}{\pgfqpoint{1.081030in}{1.726723in}}{\pgfqpoint{1.086854in}{1.720899in}}%
\pgfpathcurveto{\pgfqpoint{1.092678in}{1.715075in}}{\pgfqpoint{1.100578in}{1.711803in}}{\pgfqpoint{1.108814in}{1.711803in}}%
\pgfpathclose%
\pgfusepath{stroke,fill}%
\end{pgfscope}%
\begin{pgfscope}%
\pgfpathrectangle{\pgfqpoint{0.100000in}{0.212622in}}{\pgfqpoint{3.696000in}{3.696000in}}%
\pgfusepath{clip}%
\pgfsetbuttcap%
\pgfsetroundjoin%
\definecolor{currentfill}{rgb}{0.121569,0.466667,0.705882}%
\pgfsetfillcolor{currentfill}%
\pgfsetfillopacity{0.528308}%
\pgfsetlinewidth{1.003750pt}%
\definecolor{currentstroke}{rgb}{0.121569,0.466667,0.705882}%
\pgfsetstrokecolor{currentstroke}%
\pgfsetstrokeopacity{0.528308}%
\pgfsetdash{}{0pt}%
\pgfpathmoveto{\pgfqpoint{1.096993in}{1.720727in}}%
\pgfpathcurveto{\pgfqpoint{1.105229in}{1.720727in}}{\pgfqpoint{1.113129in}{1.723999in}}{\pgfqpoint{1.118953in}{1.729823in}}%
\pgfpathcurveto{\pgfqpoint{1.124777in}{1.735647in}}{\pgfqpoint{1.128049in}{1.743547in}}{\pgfqpoint{1.128049in}{1.751783in}}%
\pgfpathcurveto{\pgfqpoint{1.128049in}{1.760019in}}{\pgfqpoint{1.124777in}{1.767919in}}{\pgfqpoint{1.118953in}{1.773743in}}%
\pgfpathcurveto{\pgfqpoint{1.113129in}{1.779567in}}{\pgfqpoint{1.105229in}{1.782840in}}{\pgfqpoint{1.096993in}{1.782840in}}%
\pgfpathcurveto{\pgfqpoint{1.088756in}{1.782840in}}{\pgfqpoint{1.080856in}{1.779567in}}{\pgfqpoint{1.075032in}{1.773743in}}%
\pgfpathcurveto{\pgfqpoint{1.069208in}{1.767919in}}{\pgfqpoint{1.065936in}{1.760019in}}{\pgfqpoint{1.065936in}{1.751783in}}%
\pgfpathcurveto{\pgfqpoint{1.065936in}{1.743547in}}{\pgfqpoint{1.069208in}{1.735647in}}{\pgfqpoint{1.075032in}{1.729823in}}%
\pgfpathcurveto{\pgfqpoint{1.080856in}{1.723999in}}{\pgfqpoint{1.088756in}{1.720727in}}{\pgfqpoint{1.096993in}{1.720727in}}%
\pgfpathclose%
\pgfusepath{stroke,fill}%
\end{pgfscope}%
\begin{pgfscope}%
\pgfpathrectangle{\pgfqpoint{0.100000in}{0.212622in}}{\pgfqpoint{3.696000in}{3.696000in}}%
\pgfusepath{clip}%
\pgfsetbuttcap%
\pgfsetroundjoin%
\definecolor{currentfill}{rgb}{0.121569,0.466667,0.705882}%
\pgfsetfillcolor{currentfill}%
\pgfsetfillopacity{0.529638}%
\pgfsetlinewidth{1.003750pt}%
\definecolor{currentstroke}{rgb}{0.121569,0.466667,0.705882}%
\pgfsetstrokecolor{currentstroke}%
\pgfsetstrokeopacity{0.529638}%
\pgfsetdash{}{0pt}%
\pgfpathmoveto{\pgfqpoint{1.892961in}{1.981246in}}%
\pgfpathcurveto{\pgfqpoint{1.901197in}{1.981246in}}{\pgfqpoint{1.909097in}{1.984519in}}{\pgfqpoint{1.914921in}{1.990343in}}%
\pgfpathcurveto{\pgfqpoint{1.920745in}{1.996167in}}{\pgfqpoint{1.924017in}{2.004067in}}{\pgfqpoint{1.924017in}{2.012303in}}%
\pgfpathcurveto{\pgfqpoint{1.924017in}{2.020539in}}{\pgfqpoint{1.920745in}{2.028439in}}{\pgfqpoint{1.914921in}{2.034263in}}%
\pgfpathcurveto{\pgfqpoint{1.909097in}{2.040087in}}{\pgfqpoint{1.901197in}{2.043359in}}{\pgfqpoint{1.892961in}{2.043359in}}%
\pgfpathcurveto{\pgfqpoint{1.884724in}{2.043359in}}{\pgfqpoint{1.876824in}{2.040087in}}{\pgfqpoint{1.871000in}{2.034263in}}%
\pgfpathcurveto{\pgfqpoint{1.865176in}{2.028439in}}{\pgfqpoint{1.861904in}{2.020539in}}{\pgfqpoint{1.861904in}{2.012303in}}%
\pgfpathcurveto{\pgfqpoint{1.861904in}{2.004067in}}{\pgfqpoint{1.865176in}{1.996167in}}{\pgfqpoint{1.871000in}{1.990343in}}%
\pgfpathcurveto{\pgfqpoint{1.876824in}{1.984519in}}{\pgfqpoint{1.884724in}{1.981246in}}{\pgfqpoint{1.892961in}{1.981246in}}%
\pgfpathclose%
\pgfusepath{stroke,fill}%
\end{pgfscope}%
\begin{pgfscope}%
\pgfpathrectangle{\pgfqpoint{0.100000in}{0.212622in}}{\pgfqpoint{3.696000in}{3.696000in}}%
\pgfusepath{clip}%
\pgfsetbuttcap%
\pgfsetroundjoin%
\definecolor{currentfill}{rgb}{0.121569,0.466667,0.705882}%
\pgfsetfillcolor{currentfill}%
\pgfsetfillopacity{0.530503}%
\pgfsetlinewidth{1.003750pt}%
\definecolor{currentstroke}{rgb}{0.121569,0.466667,0.705882}%
\pgfsetstrokecolor{currentstroke}%
\pgfsetstrokeopacity{0.530503}%
\pgfsetdash{}{0pt}%
\pgfpathmoveto{\pgfqpoint{1.094986in}{1.717911in}}%
\pgfpathcurveto{\pgfqpoint{1.103223in}{1.717911in}}{\pgfqpoint{1.111123in}{1.721183in}}{\pgfqpoint{1.116947in}{1.727007in}}%
\pgfpathcurveto{\pgfqpoint{1.122771in}{1.732831in}}{\pgfqpoint{1.126043in}{1.740731in}}{\pgfqpoint{1.126043in}{1.748968in}}%
\pgfpathcurveto{\pgfqpoint{1.126043in}{1.757204in}}{\pgfqpoint{1.122771in}{1.765104in}}{\pgfqpoint{1.116947in}{1.770928in}}%
\pgfpathcurveto{\pgfqpoint{1.111123in}{1.776752in}}{\pgfqpoint{1.103223in}{1.780024in}}{\pgfqpoint{1.094986in}{1.780024in}}%
\pgfpathcurveto{\pgfqpoint{1.086750in}{1.780024in}}{\pgfqpoint{1.078850in}{1.776752in}}{\pgfqpoint{1.073026in}{1.770928in}}%
\pgfpathcurveto{\pgfqpoint{1.067202in}{1.765104in}}{\pgfqpoint{1.063930in}{1.757204in}}{\pgfqpoint{1.063930in}{1.748968in}}%
\pgfpathcurveto{\pgfqpoint{1.063930in}{1.740731in}}{\pgfqpoint{1.067202in}{1.732831in}}{\pgfqpoint{1.073026in}{1.727007in}}%
\pgfpathcurveto{\pgfqpoint{1.078850in}{1.721183in}}{\pgfqpoint{1.086750in}{1.717911in}}{\pgfqpoint{1.094986in}{1.717911in}}%
\pgfpathclose%
\pgfusepath{stroke,fill}%
\end{pgfscope}%
\begin{pgfscope}%
\pgfpathrectangle{\pgfqpoint{0.100000in}{0.212622in}}{\pgfqpoint{3.696000in}{3.696000in}}%
\pgfusepath{clip}%
\pgfsetbuttcap%
\pgfsetroundjoin%
\definecolor{currentfill}{rgb}{0.121569,0.466667,0.705882}%
\pgfsetfillcolor{currentfill}%
\pgfsetfillopacity{0.533818}%
\pgfsetlinewidth{1.003750pt}%
\definecolor{currentstroke}{rgb}{0.121569,0.466667,0.705882}%
\pgfsetstrokecolor{currentstroke}%
\pgfsetstrokeopacity{0.533818}%
\pgfsetdash{}{0pt}%
\pgfpathmoveto{\pgfqpoint{1.080715in}{1.721117in}}%
\pgfpathcurveto{\pgfqpoint{1.088951in}{1.721117in}}{\pgfqpoint{1.096851in}{1.724390in}}{\pgfqpoint{1.102675in}{1.730214in}}%
\pgfpathcurveto{\pgfqpoint{1.108499in}{1.736038in}}{\pgfqpoint{1.111771in}{1.743938in}}{\pgfqpoint{1.111771in}{1.752174in}}%
\pgfpathcurveto{\pgfqpoint{1.111771in}{1.760410in}}{\pgfqpoint{1.108499in}{1.768310in}}{\pgfqpoint{1.102675in}{1.774134in}}%
\pgfpathcurveto{\pgfqpoint{1.096851in}{1.779958in}}{\pgfqpoint{1.088951in}{1.783230in}}{\pgfqpoint{1.080715in}{1.783230in}}%
\pgfpathcurveto{\pgfqpoint{1.072479in}{1.783230in}}{\pgfqpoint{1.064579in}{1.779958in}}{\pgfqpoint{1.058755in}{1.774134in}}%
\pgfpathcurveto{\pgfqpoint{1.052931in}{1.768310in}}{\pgfqpoint{1.049658in}{1.760410in}}{\pgfqpoint{1.049658in}{1.752174in}}%
\pgfpathcurveto{\pgfqpoint{1.049658in}{1.743938in}}{\pgfqpoint{1.052931in}{1.736038in}}{\pgfqpoint{1.058755in}{1.730214in}}%
\pgfpathcurveto{\pgfqpoint{1.064579in}{1.724390in}}{\pgfqpoint{1.072479in}{1.721117in}}{\pgfqpoint{1.080715in}{1.721117in}}%
\pgfpathclose%
\pgfusepath{stroke,fill}%
\end{pgfscope}%
\begin{pgfscope}%
\pgfpathrectangle{\pgfqpoint{0.100000in}{0.212622in}}{\pgfqpoint{3.696000in}{3.696000in}}%
\pgfusepath{clip}%
\pgfsetbuttcap%
\pgfsetroundjoin%
\definecolor{currentfill}{rgb}{0.121569,0.466667,0.705882}%
\pgfsetfillcolor{currentfill}%
\pgfsetfillopacity{0.535148}%
\pgfsetlinewidth{1.003750pt}%
\definecolor{currentstroke}{rgb}{0.121569,0.466667,0.705882}%
\pgfsetstrokecolor{currentstroke}%
\pgfsetstrokeopacity{0.535148}%
\pgfsetdash{}{0pt}%
\pgfpathmoveto{\pgfqpoint{1.897557in}{1.971273in}}%
\pgfpathcurveto{\pgfqpoint{1.905793in}{1.971273in}}{\pgfqpoint{1.913693in}{1.974545in}}{\pgfqpoint{1.919517in}{1.980369in}}%
\pgfpathcurveto{\pgfqpoint{1.925341in}{1.986193in}}{\pgfqpoint{1.928614in}{1.994093in}}{\pgfqpoint{1.928614in}{2.002329in}}%
\pgfpathcurveto{\pgfqpoint{1.928614in}{2.010565in}}{\pgfqpoint{1.925341in}{2.018465in}}{\pgfqpoint{1.919517in}{2.024289in}}%
\pgfpathcurveto{\pgfqpoint{1.913693in}{2.030113in}}{\pgfqpoint{1.905793in}{2.033386in}}{\pgfqpoint{1.897557in}{2.033386in}}%
\pgfpathcurveto{\pgfqpoint{1.889321in}{2.033386in}}{\pgfqpoint{1.881421in}{2.030113in}}{\pgfqpoint{1.875597in}{2.024289in}}%
\pgfpathcurveto{\pgfqpoint{1.869773in}{2.018465in}}{\pgfqpoint{1.866501in}{2.010565in}}{\pgfqpoint{1.866501in}{2.002329in}}%
\pgfpathcurveto{\pgfqpoint{1.866501in}{1.994093in}}{\pgfqpoint{1.869773in}{1.986193in}}{\pgfqpoint{1.875597in}{1.980369in}}%
\pgfpathcurveto{\pgfqpoint{1.881421in}{1.974545in}}{\pgfqpoint{1.889321in}{1.971273in}}{\pgfqpoint{1.897557in}{1.971273in}}%
\pgfpathclose%
\pgfusepath{stroke,fill}%
\end{pgfscope}%
\begin{pgfscope}%
\pgfpathrectangle{\pgfqpoint{0.100000in}{0.212622in}}{\pgfqpoint{3.696000in}{3.696000in}}%
\pgfusepath{clip}%
\pgfsetbuttcap%
\pgfsetroundjoin%
\definecolor{currentfill}{rgb}{0.121569,0.466667,0.705882}%
\pgfsetfillcolor{currentfill}%
\pgfsetfillopacity{0.537410}%
\pgfsetlinewidth{1.003750pt}%
\definecolor{currentstroke}{rgb}{0.121569,0.466667,0.705882}%
\pgfsetstrokecolor{currentstroke}%
\pgfsetstrokeopacity{0.537410}%
\pgfsetdash{}{0pt}%
\pgfpathmoveto{\pgfqpoint{1.077678in}{1.718021in}}%
\pgfpathcurveto{\pgfqpoint{1.085914in}{1.718021in}}{\pgfqpoint{1.093814in}{1.721293in}}{\pgfqpoint{1.099638in}{1.727117in}}%
\pgfpathcurveto{\pgfqpoint{1.105462in}{1.732941in}}{\pgfqpoint{1.108734in}{1.740841in}}{\pgfqpoint{1.108734in}{1.749078in}}%
\pgfpathcurveto{\pgfqpoint{1.108734in}{1.757314in}}{\pgfqpoint{1.105462in}{1.765214in}}{\pgfqpoint{1.099638in}{1.771038in}}%
\pgfpathcurveto{\pgfqpoint{1.093814in}{1.776862in}}{\pgfqpoint{1.085914in}{1.780134in}}{\pgfqpoint{1.077678in}{1.780134in}}%
\pgfpathcurveto{\pgfqpoint{1.069441in}{1.780134in}}{\pgfqpoint{1.061541in}{1.776862in}}{\pgfqpoint{1.055717in}{1.771038in}}%
\pgfpathcurveto{\pgfqpoint{1.049893in}{1.765214in}}{\pgfqpoint{1.046621in}{1.757314in}}{\pgfqpoint{1.046621in}{1.749078in}}%
\pgfpathcurveto{\pgfqpoint{1.046621in}{1.740841in}}{\pgfqpoint{1.049893in}{1.732941in}}{\pgfqpoint{1.055717in}{1.727117in}}%
\pgfpathcurveto{\pgfqpoint{1.061541in}{1.721293in}}{\pgfqpoint{1.069441in}{1.718021in}}{\pgfqpoint{1.077678in}{1.718021in}}%
\pgfpathclose%
\pgfusepath{stroke,fill}%
\end{pgfscope}%
\begin{pgfscope}%
\pgfpathrectangle{\pgfqpoint{0.100000in}{0.212622in}}{\pgfqpoint{3.696000in}{3.696000in}}%
\pgfusepath{clip}%
\pgfsetbuttcap%
\pgfsetroundjoin%
\definecolor{currentfill}{rgb}{0.121569,0.466667,0.705882}%
\pgfsetfillcolor{currentfill}%
\pgfsetfillopacity{0.539278}%
\pgfsetlinewidth{1.003750pt}%
\definecolor{currentstroke}{rgb}{0.121569,0.466667,0.705882}%
\pgfsetstrokecolor{currentstroke}%
\pgfsetstrokeopacity{0.539278}%
\pgfsetdash{}{0pt}%
\pgfpathmoveto{\pgfqpoint{1.905599in}{1.973354in}}%
\pgfpathcurveto{\pgfqpoint{1.913836in}{1.973354in}}{\pgfqpoint{1.921736in}{1.976626in}}{\pgfqpoint{1.927560in}{1.982450in}}%
\pgfpathcurveto{\pgfqpoint{1.933384in}{1.988274in}}{\pgfqpoint{1.936656in}{1.996174in}}{\pgfqpoint{1.936656in}{2.004411in}}%
\pgfpathcurveto{\pgfqpoint{1.936656in}{2.012647in}}{\pgfqpoint{1.933384in}{2.020547in}}{\pgfqpoint{1.927560in}{2.026371in}}%
\pgfpathcurveto{\pgfqpoint{1.921736in}{2.032195in}}{\pgfqpoint{1.913836in}{2.035467in}}{\pgfqpoint{1.905599in}{2.035467in}}%
\pgfpathcurveto{\pgfqpoint{1.897363in}{2.035467in}}{\pgfqpoint{1.889463in}{2.032195in}}{\pgfqpoint{1.883639in}{2.026371in}}%
\pgfpathcurveto{\pgfqpoint{1.877815in}{2.020547in}}{\pgfqpoint{1.874543in}{2.012647in}}{\pgfqpoint{1.874543in}{2.004411in}}%
\pgfpathcurveto{\pgfqpoint{1.874543in}{1.996174in}}{\pgfqpoint{1.877815in}{1.988274in}}{\pgfqpoint{1.883639in}{1.982450in}}%
\pgfpathcurveto{\pgfqpoint{1.889463in}{1.976626in}}{\pgfqpoint{1.897363in}{1.973354in}}{\pgfqpoint{1.905599in}{1.973354in}}%
\pgfpathclose%
\pgfusepath{stroke,fill}%
\end{pgfscope}%
\begin{pgfscope}%
\pgfpathrectangle{\pgfqpoint{0.100000in}{0.212622in}}{\pgfqpoint{3.696000in}{3.696000in}}%
\pgfusepath{clip}%
\pgfsetbuttcap%
\pgfsetroundjoin%
\definecolor{currentfill}{rgb}{0.121569,0.466667,0.705882}%
\pgfsetfillcolor{currentfill}%
\pgfsetfillopacity{0.539721}%
\pgfsetlinewidth{1.003750pt}%
\definecolor{currentstroke}{rgb}{0.121569,0.466667,0.705882}%
\pgfsetstrokecolor{currentstroke}%
\pgfsetstrokeopacity{0.539721}%
\pgfsetdash{}{0pt}%
\pgfpathmoveto{\pgfqpoint{1.066819in}{1.720564in}}%
\pgfpathcurveto{\pgfqpoint{1.075055in}{1.720564in}}{\pgfqpoint{1.082955in}{1.723837in}}{\pgfqpoint{1.088779in}{1.729661in}}%
\pgfpathcurveto{\pgfqpoint{1.094603in}{1.735485in}}{\pgfqpoint{1.097875in}{1.743385in}}{\pgfqpoint{1.097875in}{1.751621in}}%
\pgfpathcurveto{\pgfqpoint{1.097875in}{1.759857in}}{\pgfqpoint{1.094603in}{1.767757in}}{\pgfqpoint{1.088779in}{1.773581in}}%
\pgfpathcurveto{\pgfqpoint{1.082955in}{1.779405in}}{\pgfqpoint{1.075055in}{1.782677in}}{\pgfqpoint{1.066819in}{1.782677in}}%
\pgfpathcurveto{\pgfqpoint{1.058583in}{1.782677in}}{\pgfqpoint{1.050682in}{1.779405in}}{\pgfqpoint{1.044859in}{1.773581in}}%
\pgfpathcurveto{\pgfqpoint{1.039035in}{1.767757in}}{\pgfqpoint{1.035762in}{1.759857in}}{\pgfqpoint{1.035762in}{1.751621in}}%
\pgfpathcurveto{\pgfqpoint{1.035762in}{1.743385in}}{\pgfqpoint{1.039035in}{1.735485in}}{\pgfqpoint{1.044859in}{1.729661in}}%
\pgfpathcurveto{\pgfqpoint{1.050682in}{1.723837in}}{\pgfqpoint{1.058583in}{1.720564in}}{\pgfqpoint{1.066819in}{1.720564in}}%
\pgfpathclose%
\pgfusepath{stroke,fill}%
\end{pgfscope}%
\begin{pgfscope}%
\pgfpathrectangle{\pgfqpoint{0.100000in}{0.212622in}}{\pgfqpoint{3.696000in}{3.696000in}}%
\pgfusepath{clip}%
\pgfsetbuttcap%
\pgfsetroundjoin%
\definecolor{currentfill}{rgb}{0.121569,0.466667,0.705882}%
\pgfsetfillcolor{currentfill}%
\pgfsetfillopacity{0.543851}%
\pgfsetlinewidth{1.003750pt}%
\definecolor{currentstroke}{rgb}{0.121569,0.466667,0.705882}%
\pgfsetstrokecolor{currentstroke}%
\pgfsetstrokeopacity{0.543851}%
\pgfsetdash{}{0pt}%
\pgfpathmoveto{\pgfqpoint{1.909303in}{1.967199in}}%
\pgfpathcurveto{\pgfqpoint{1.917539in}{1.967199in}}{\pgfqpoint{1.925439in}{1.970472in}}{\pgfqpoint{1.931263in}{1.976295in}}%
\pgfpathcurveto{\pgfqpoint{1.937087in}{1.982119in}}{\pgfqpoint{1.940360in}{1.990019in}}{\pgfqpoint{1.940360in}{1.998256in}}%
\pgfpathcurveto{\pgfqpoint{1.940360in}{2.006492in}}{\pgfqpoint{1.937087in}{2.014392in}}{\pgfqpoint{1.931263in}{2.020216in}}%
\pgfpathcurveto{\pgfqpoint{1.925439in}{2.026040in}}{\pgfqpoint{1.917539in}{2.029312in}}{\pgfqpoint{1.909303in}{2.029312in}}%
\pgfpathcurveto{\pgfqpoint{1.901067in}{2.029312in}}{\pgfqpoint{1.893167in}{2.026040in}}{\pgfqpoint{1.887343in}{2.020216in}}%
\pgfpathcurveto{\pgfqpoint{1.881519in}{2.014392in}}{\pgfqpoint{1.878247in}{2.006492in}}{\pgfqpoint{1.878247in}{1.998256in}}%
\pgfpathcurveto{\pgfqpoint{1.878247in}{1.990019in}}{\pgfqpoint{1.881519in}{1.982119in}}{\pgfqpoint{1.887343in}{1.976295in}}%
\pgfpathcurveto{\pgfqpoint{1.893167in}{1.970472in}}{\pgfqpoint{1.901067in}{1.967199in}}{\pgfqpoint{1.909303in}{1.967199in}}%
\pgfpathclose%
\pgfusepath{stroke,fill}%
\end{pgfscope}%
\begin{pgfscope}%
\pgfpathrectangle{\pgfqpoint{0.100000in}{0.212622in}}{\pgfqpoint{3.696000in}{3.696000in}}%
\pgfusepath{clip}%
\pgfsetbuttcap%
\pgfsetroundjoin%
\definecolor{currentfill}{rgb}{0.121569,0.466667,0.705882}%
\pgfsetfillcolor{currentfill}%
\pgfsetfillopacity{0.543990}%
\pgfsetlinewidth{1.003750pt}%
\definecolor{currentstroke}{rgb}{0.121569,0.466667,0.705882}%
\pgfsetstrokecolor{currentstroke}%
\pgfsetstrokeopacity{0.543990}%
\pgfsetdash{}{0pt}%
\pgfpathmoveto{\pgfqpoint{1.058870in}{1.709573in}}%
\pgfpathcurveto{\pgfqpoint{1.067106in}{1.709573in}}{\pgfqpoint{1.075006in}{1.712845in}}{\pgfqpoint{1.080830in}{1.718669in}}%
\pgfpathcurveto{\pgfqpoint{1.086654in}{1.724493in}}{\pgfqpoint{1.089927in}{1.732393in}}{\pgfqpoint{1.089927in}{1.740629in}}%
\pgfpathcurveto{\pgfqpoint{1.089927in}{1.748866in}}{\pgfqpoint{1.086654in}{1.756766in}}{\pgfqpoint{1.080830in}{1.762590in}}%
\pgfpathcurveto{\pgfqpoint{1.075006in}{1.768414in}}{\pgfqpoint{1.067106in}{1.771686in}}{\pgfqpoint{1.058870in}{1.771686in}}%
\pgfpathcurveto{\pgfqpoint{1.050634in}{1.771686in}}{\pgfqpoint{1.042734in}{1.768414in}}{\pgfqpoint{1.036910in}{1.762590in}}%
\pgfpathcurveto{\pgfqpoint{1.031086in}{1.756766in}}{\pgfqpoint{1.027814in}{1.748866in}}{\pgfqpoint{1.027814in}{1.740629in}}%
\pgfpathcurveto{\pgfqpoint{1.027814in}{1.732393in}}{\pgfqpoint{1.031086in}{1.724493in}}{\pgfqpoint{1.036910in}{1.718669in}}%
\pgfpathcurveto{\pgfqpoint{1.042734in}{1.712845in}}{\pgfqpoint{1.050634in}{1.709573in}}{\pgfqpoint{1.058870in}{1.709573in}}%
\pgfpathclose%
\pgfusepath{stroke,fill}%
\end{pgfscope}%
\begin{pgfscope}%
\pgfpathrectangle{\pgfqpoint{0.100000in}{0.212622in}}{\pgfqpoint{3.696000in}{3.696000in}}%
\pgfusepath{clip}%
\pgfsetbuttcap%
\pgfsetroundjoin%
\definecolor{currentfill}{rgb}{0.121569,0.466667,0.705882}%
\pgfsetfillcolor{currentfill}%
\pgfsetfillopacity{0.547134}%
\pgfsetlinewidth{1.003750pt}%
\definecolor{currentstroke}{rgb}{0.121569,0.466667,0.705882}%
\pgfsetstrokecolor{currentstroke}%
\pgfsetstrokeopacity{0.547134}%
\pgfsetdash{}{0pt}%
\pgfpathmoveto{\pgfqpoint{1.916229in}{1.969375in}}%
\pgfpathcurveto{\pgfqpoint{1.924465in}{1.969375in}}{\pgfqpoint{1.932365in}{1.972647in}}{\pgfqpoint{1.938189in}{1.978471in}}%
\pgfpathcurveto{\pgfqpoint{1.944013in}{1.984295in}}{\pgfqpoint{1.947285in}{1.992195in}}{\pgfqpoint{1.947285in}{2.000431in}}%
\pgfpathcurveto{\pgfqpoint{1.947285in}{2.008667in}}{\pgfqpoint{1.944013in}{2.016567in}}{\pgfqpoint{1.938189in}{2.022391in}}%
\pgfpathcurveto{\pgfqpoint{1.932365in}{2.028215in}}{\pgfqpoint{1.924465in}{2.031488in}}{\pgfqpoint{1.916229in}{2.031488in}}%
\pgfpathcurveto{\pgfqpoint{1.907992in}{2.031488in}}{\pgfqpoint{1.900092in}{2.028215in}}{\pgfqpoint{1.894268in}{2.022391in}}%
\pgfpathcurveto{\pgfqpoint{1.888444in}{2.016567in}}{\pgfqpoint{1.885172in}{2.008667in}}{\pgfqpoint{1.885172in}{2.000431in}}%
\pgfpathcurveto{\pgfqpoint{1.885172in}{1.992195in}}{\pgfqpoint{1.888444in}{1.984295in}}{\pgfqpoint{1.894268in}{1.978471in}}%
\pgfpathcurveto{\pgfqpoint{1.900092in}{1.972647in}}{\pgfqpoint{1.907992in}{1.969375in}}{\pgfqpoint{1.916229in}{1.969375in}}%
\pgfpathclose%
\pgfusepath{stroke,fill}%
\end{pgfscope}%
\begin{pgfscope}%
\pgfpathrectangle{\pgfqpoint{0.100000in}{0.212622in}}{\pgfqpoint{3.696000in}{3.696000in}}%
\pgfusepath{clip}%
\pgfsetbuttcap%
\pgfsetroundjoin%
\definecolor{currentfill}{rgb}{0.121569,0.466667,0.705882}%
\pgfsetfillcolor{currentfill}%
\pgfsetfillopacity{0.551278}%
\pgfsetlinewidth{1.003750pt}%
\definecolor{currentstroke}{rgb}{0.121569,0.466667,0.705882}%
\pgfsetstrokecolor{currentstroke}%
\pgfsetstrokeopacity{0.551278}%
\pgfsetdash{}{0pt}%
\pgfpathmoveto{\pgfqpoint{1.920292in}{1.960281in}}%
\pgfpathcurveto{\pgfqpoint{1.928529in}{1.960281in}}{\pgfqpoint{1.936429in}{1.963554in}}{\pgfqpoint{1.942253in}{1.969377in}}%
\pgfpathcurveto{\pgfqpoint{1.948077in}{1.975201in}}{\pgfqpoint{1.951349in}{1.983101in}}{\pgfqpoint{1.951349in}{1.991338in}}%
\pgfpathcurveto{\pgfqpoint{1.951349in}{1.999574in}}{\pgfqpoint{1.948077in}{2.007474in}}{\pgfqpoint{1.942253in}{2.013298in}}%
\pgfpathcurveto{\pgfqpoint{1.936429in}{2.019122in}}{\pgfqpoint{1.928529in}{2.022394in}}{\pgfqpoint{1.920292in}{2.022394in}}%
\pgfpathcurveto{\pgfqpoint{1.912056in}{2.022394in}}{\pgfqpoint{1.904156in}{2.019122in}}{\pgfqpoint{1.898332in}{2.013298in}}%
\pgfpathcurveto{\pgfqpoint{1.892508in}{2.007474in}}{\pgfqpoint{1.889236in}{1.999574in}}{\pgfqpoint{1.889236in}{1.991338in}}%
\pgfpathcurveto{\pgfqpoint{1.889236in}{1.983101in}}{\pgfqpoint{1.892508in}{1.975201in}}{\pgfqpoint{1.898332in}{1.969377in}}%
\pgfpathcurveto{\pgfqpoint{1.904156in}{1.963554in}}{\pgfqpoint{1.912056in}{1.960281in}}{\pgfqpoint{1.920292in}{1.960281in}}%
\pgfpathclose%
\pgfusepath{stroke,fill}%
\end{pgfscope}%
\begin{pgfscope}%
\pgfpathrectangle{\pgfqpoint{0.100000in}{0.212622in}}{\pgfqpoint{3.696000in}{3.696000in}}%
\pgfusepath{clip}%
\pgfsetbuttcap%
\pgfsetroundjoin%
\definecolor{currentfill}{rgb}{0.121569,0.466667,0.705882}%
\pgfsetfillcolor{currentfill}%
\pgfsetfillopacity{0.551637}%
\pgfsetlinewidth{1.003750pt}%
\definecolor{currentstroke}{rgb}{0.121569,0.466667,0.705882}%
\pgfsetstrokecolor{currentstroke}%
\pgfsetstrokeopacity{0.551637}%
\pgfsetdash{}{0pt}%
\pgfpathmoveto{\pgfqpoint{1.024434in}{1.714963in}}%
\pgfpathcurveto{\pgfqpoint{1.032670in}{1.714963in}}{\pgfqpoint{1.040570in}{1.718236in}}{\pgfqpoint{1.046394in}{1.724060in}}%
\pgfpathcurveto{\pgfqpoint{1.052218in}{1.729884in}}{\pgfqpoint{1.055490in}{1.737784in}}{\pgfqpoint{1.055490in}{1.746020in}}%
\pgfpathcurveto{\pgfqpoint{1.055490in}{1.754256in}}{\pgfqpoint{1.052218in}{1.762156in}}{\pgfqpoint{1.046394in}{1.767980in}}%
\pgfpathcurveto{\pgfqpoint{1.040570in}{1.773804in}}{\pgfqpoint{1.032670in}{1.777076in}}{\pgfqpoint{1.024434in}{1.777076in}}%
\pgfpathcurveto{\pgfqpoint{1.016197in}{1.777076in}}{\pgfqpoint{1.008297in}{1.773804in}}{\pgfqpoint{1.002473in}{1.767980in}}%
\pgfpathcurveto{\pgfqpoint{0.996649in}{1.762156in}}{\pgfqpoint{0.993377in}{1.754256in}}{\pgfqpoint{0.993377in}{1.746020in}}%
\pgfpathcurveto{\pgfqpoint{0.993377in}{1.737784in}}{\pgfqpoint{0.996649in}{1.729884in}}{\pgfqpoint{1.002473in}{1.724060in}}%
\pgfpathcurveto{\pgfqpoint{1.008297in}{1.718236in}}{\pgfqpoint{1.016197in}{1.714963in}}{\pgfqpoint{1.024434in}{1.714963in}}%
\pgfpathclose%
\pgfusepath{stroke,fill}%
\end{pgfscope}%
\begin{pgfscope}%
\pgfpathrectangle{\pgfqpoint{0.100000in}{0.212622in}}{\pgfqpoint{3.696000in}{3.696000in}}%
\pgfusepath{clip}%
\pgfsetbuttcap%
\pgfsetroundjoin%
\definecolor{currentfill}{rgb}{0.121569,0.466667,0.705882}%
\pgfsetfillcolor{currentfill}%
\pgfsetfillopacity{0.558020}%
\pgfsetlinewidth{1.003750pt}%
\definecolor{currentstroke}{rgb}{0.121569,0.466667,0.705882}%
\pgfsetstrokecolor{currentstroke}%
\pgfsetstrokeopacity{0.558020}%
\pgfsetdash{}{0pt}%
\pgfpathmoveto{\pgfqpoint{1.932856in}{1.964669in}}%
\pgfpathcurveto{\pgfqpoint{1.941093in}{1.964669in}}{\pgfqpoint{1.948993in}{1.967942in}}{\pgfqpoint{1.954817in}{1.973766in}}%
\pgfpathcurveto{\pgfqpoint{1.960641in}{1.979589in}}{\pgfqpoint{1.963913in}{1.987489in}}{\pgfqpoint{1.963913in}{1.995726in}}%
\pgfpathcurveto{\pgfqpoint{1.963913in}{2.003962in}}{\pgfqpoint{1.960641in}{2.011862in}}{\pgfqpoint{1.954817in}{2.017686in}}%
\pgfpathcurveto{\pgfqpoint{1.948993in}{2.023510in}}{\pgfqpoint{1.941093in}{2.026782in}}{\pgfqpoint{1.932856in}{2.026782in}}%
\pgfpathcurveto{\pgfqpoint{1.924620in}{2.026782in}}{\pgfqpoint{1.916720in}{2.023510in}}{\pgfqpoint{1.910896in}{2.017686in}}%
\pgfpathcurveto{\pgfqpoint{1.905072in}{2.011862in}}{\pgfqpoint{1.901800in}{2.003962in}}{\pgfqpoint{1.901800in}{1.995726in}}%
\pgfpathcurveto{\pgfqpoint{1.901800in}{1.987489in}}{\pgfqpoint{1.905072in}{1.979589in}}{\pgfqpoint{1.910896in}{1.973766in}}%
\pgfpathcurveto{\pgfqpoint{1.916720in}{1.967942in}}{\pgfqpoint{1.924620in}{1.964669in}}{\pgfqpoint{1.932856in}{1.964669in}}%
\pgfpathclose%
\pgfusepath{stroke,fill}%
\end{pgfscope}%
\begin{pgfscope}%
\pgfpathrectangle{\pgfqpoint{0.100000in}{0.212622in}}{\pgfqpoint{3.696000in}{3.696000in}}%
\pgfusepath{clip}%
\pgfsetbuttcap%
\pgfsetroundjoin%
\definecolor{currentfill}{rgb}{0.121569,0.466667,0.705882}%
\pgfsetfillcolor{currentfill}%
\pgfsetfillopacity{0.559125}%
\pgfsetlinewidth{1.003750pt}%
\definecolor{currentstroke}{rgb}{0.121569,0.466667,0.705882}%
\pgfsetstrokecolor{currentstroke}%
\pgfsetstrokeopacity{0.559125}%
\pgfsetdash{}{0pt}%
\pgfpathmoveto{\pgfqpoint{1.012339in}{1.704425in}}%
\pgfpathcurveto{\pgfqpoint{1.020575in}{1.704425in}}{\pgfqpoint{1.028475in}{1.707698in}}{\pgfqpoint{1.034299in}{1.713522in}}%
\pgfpathcurveto{\pgfqpoint{1.040123in}{1.719346in}}{\pgfqpoint{1.043395in}{1.727246in}}{\pgfqpoint{1.043395in}{1.735482in}}%
\pgfpathcurveto{\pgfqpoint{1.043395in}{1.743718in}}{\pgfqpoint{1.040123in}{1.751618in}}{\pgfqpoint{1.034299in}{1.757442in}}%
\pgfpathcurveto{\pgfqpoint{1.028475in}{1.763266in}}{\pgfqpoint{1.020575in}{1.766538in}}{\pgfqpoint{1.012339in}{1.766538in}}%
\pgfpathcurveto{\pgfqpoint{1.004103in}{1.766538in}}{\pgfqpoint{0.996203in}{1.763266in}}{\pgfqpoint{0.990379in}{1.757442in}}%
\pgfpathcurveto{\pgfqpoint{0.984555in}{1.751618in}}{\pgfqpoint{0.981282in}{1.743718in}}{\pgfqpoint{0.981282in}{1.735482in}}%
\pgfpathcurveto{\pgfqpoint{0.981282in}{1.727246in}}{\pgfqpoint{0.984555in}{1.719346in}}{\pgfqpoint{0.990379in}{1.713522in}}%
\pgfpathcurveto{\pgfqpoint{0.996203in}{1.707698in}}{\pgfqpoint{1.004103in}{1.704425in}}{\pgfqpoint{1.012339in}{1.704425in}}%
\pgfpathclose%
\pgfusepath{stroke,fill}%
\end{pgfscope}%
\begin{pgfscope}%
\pgfpathrectangle{\pgfqpoint{0.100000in}{0.212622in}}{\pgfqpoint{3.696000in}{3.696000in}}%
\pgfusepath{clip}%
\pgfsetbuttcap%
\pgfsetroundjoin%
\definecolor{currentfill}{rgb}{0.121569,0.466667,0.705882}%
\pgfsetfillcolor{currentfill}%
\pgfsetfillopacity{0.562726}%
\pgfsetlinewidth{1.003750pt}%
\definecolor{currentstroke}{rgb}{0.121569,0.466667,0.705882}%
\pgfsetstrokecolor{currentstroke}%
\pgfsetstrokeopacity{0.562726}%
\pgfsetdash{}{0pt}%
\pgfpathmoveto{\pgfqpoint{0.992791in}{1.706649in}}%
\pgfpathcurveto{\pgfqpoint{1.001027in}{1.706649in}}{\pgfqpoint{1.008927in}{1.709921in}}{\pgfqpoint{1.014751in}{1.715745in}}%
\pgfpathcurveto{\pgfqpoint{1.020575in}{1.721569in}}{\pgfqpoint{1.023848in}{1.729469in}}{\pgfqpoint{1.023848in}{1.737705in}}%
\pgfpathcurveto{\pgfqpoint{1.023848in}{1.745941in}}{\pgfqpoint{1.020575in}{1.753841in}}{\pgfqpoint{1.014751in}{1.759665in}}%
\pgfpathcurveto{\pgfqpoint{1.008927in}{1.765489in}}{\pgfqpoint{1.001027in}{1.768762in}}{\pgfqpoint{0.992791in}{1.768762in}}%
\pgfpathcurveto{\pgfqpoint{0.984555in}{1.768762in}}{\pgfqpoint{0.976655in}{1.765489in}}{\pgfqpoint{0.970831in}{1.759665in}}%
\pgfpathcurveto{\pgfqpoint{0.965007in}{1.753841in}}{\pgfqpoint{0.961735in}{1.745941in}}{\pgfqpoint{0.961735in}{1.737705in}}%
\pgfpathcurveto{\pgfqpoint{0.961735in}{1.729469in}}{\pgfqpoint{0.965007in}{1.721569in}}{\pgfqpoint{0.970831in}{1.715745in}}%
\pgfpathcurveto{\pgfqpoint{0.976655in}{1.709921in}}{\pgfqpoint{0.984555in}{1.706649in}}{\pgfqpoint{0.992791in}{1.706649in}}%
\pgfpathclose%
\pgfusepath{stroke,fill}%
\end{pgfscope}%
\begin{pgfscope}%
\pgfpathrectangle{\pgfqpoint{0.100000in}{0.212622in}}{\pgfqpoint{3.696000in}{3.696000in}}%
\pgfusepath{clip}%
\pgfsetbuttcap%
\pgfsetroundjoin%
\definecolor{currentfill}{rgb}{0.121569,0.466667,0.705882}%
\pgfsetfillcolor{currentfill}%
\pgfsetfillopacity{0.563028}%
\pgfsetlinewidth{1.003750pt}%
\definecolor{currentstroke}{rgb}{0.121569,0.466667,0.705882}%
\pgfsetstrokecolor{currentstroke}%
\pgfsetstrokeopacity{0.563028}%
\pgfsetdash{}{0pt}%
\pgfpathmoveto{\pgfqpoint{0.853186in}{1.615634in}}%
\pgfpathcurveto{\pgfqpoint{0.861423in}{1.615634in}}{\pgfqpoint{0.869323in}{1.618906in}}{\pgfqpoint{0.875147in}{1.624730in}}%
\pgfpathcurveto{\pgfqpoint{0.880971in}{1.630554in}}{\pgfqpoint{0.884243in}{1.638454in}}{\pgfqpoint{0.884243in}{1.646690in}}%
\pgfpathcurveto{\pgfqpoint{0.884243in}{1.654927in}}{\pgfqpoint{0.880971in}{1.662827in}}{\pgfqpoint{0.875147in}{1.668651in}}%
\pgfpathcurveto{\pgfqpoint{0.869323in}{1.674474in}}{\pgfqpoint{0.861423in}{1.677747in}}{\pgfqpoint{0.853186in}{1.677747in}}%
\pgfpathcurveto{\pgfqpoint{0.844950in}{1.677747in}}{\pgfqpoint{0.837050in}{1.674474in}}{\pgfqpoint{0.831226in}{1.668651in}}%
\pgfpathcurveto{\pgfqpoint{0.825402in}{1.662827in}}{\pgfqpoint{0.822130in}{1.654927in}}{\pgfqpoint{0.822130in}{1.646690in}}%
\pgfpathcurveto{\pgfqpoint{0.822130in}{1.638454in}}{\pgfqpoint{0.825402in}{1.630554in}}{\pgfqpoint{0.831226in}{1.624730in}}%
\pgfpathcurveto{\pgfqpoint{0.837050in}{1.618906in}}{\pgfqpoint{0.844950in}{1.615634in}}{\pgfqpoint{0.853186in}{1.615634in}}%
\pgfpathclose%
\pgfusepath{stroke,fill}%
\end{pgfscope}%
\begin{pgfscope}%
\pgfpathrectangle{\pgfqpoint{0.100000in}{0.212622in}}{\pgfqpoint{3.696000in}{3.696000in}}%
\pgfusepath{clip}%
\pgfsetbuttcap%
\pgfsetroundjoin%
\definecolor{currentfill}{rgb}{0.121569,0.466667,0.705882}%
\pgfsetfillcolor{currentfill}%
\pgfsetfillopacity{0.563667}%
\pgfsetlinewidth{1.003750pt}%
\definecolor{currentstroke}{rgb}{0.121569,0.466667,0.705882}%
\pgfsetstrokecolor{currentstroke}%
\pgfsetstrokeopacity{0.563667}%
\pgfsetdash{}{0pt}%
\pgfpathmoveto{\pgfqpoint{1.938294in}{1.953153in}}%
\pgfpathcurveto{\pgfqpoint{1.946530in}{1.953153in}}{\pgfqpoint{1.954430in}{1.956426in}}{\pgfqpoint{1.960254in}{1.962250in}}%
\pgfpathcurveto{\pgfqpoint{1.966078in}{1.968074in}}{\pgfqpoint{1.969351in}{1.975974in}}{\pgfqpoint{1.969351in}{1.984210in}}%
\pgfpathcurveto{\pgfqpoint{1.969351in}{1.992446in}}{\pgfqpoint{1.966078in}{2.000346in}}{\pgfqpoint{1.960254in}{2.006170in}}%
\pgfpathcurveto{\pgfqpoint{1.954430in}{2.011994in}}{\pgfqpoint{1.946530in}{2.015266in}}{\pgfqpoint{1.938294in}{2.015266in}}%
\pgfpathcurveto{\pgfqpoint{1.930058in}{2.015266in}}{\pgfqpoint{1.922158in}{2.011994in}}{\pgfqpoint{1.916334in}{2.006170in}}%
\pgfpathcurveto{\pgfqpoint{1.910510in}{2.000346in}}{\pgfqpoint{1.907238in}{1.992446in}}{\pgfqpoint{1.907238in}{1.984210in}}%
\pgfpathcurveto{\pgfqpoint{1.907238in}{1.975974in}}{\pgfqpoint{1.910510in}{1.968074in}}{\pgfqpoint{1.916334in}{1.962250in}}%
\pgfpathcurveto{\pgfqpoint{1.922158in}{1.956426in}}{\pgfqpoint{1.930058in}{1.953153in}}{\pgfqpoint{1.938294in}{1.953153in}}%
\pgfpathclose%
\pgfusepath{stroke,fill}%
\end{pgfscope}%
\begin{pgfscope}%
\pgfpathrectangle{\pgfqpoint{0.100000in}{0.212622in}}{\pgfqpoint{3.696000in}{3.696000in}}%
\pgfusepath{clip}%
\pgfsetbuttcap%
\pgfsetroundjoin%
\definecolor{currentfill}{rgb}{0.121569,0.466667,0.705882}%
\pgfsetfillcolor{currentfill}%
\pgfsetfillopacity{0.564954}%
\pgfsetlinewidth{1.003750pt}%
\definecolor{currentstroke}{rgb}{0.121569,0.466667,0.705882}%
\pgfsetstrokecolor{currentstroke}%
\pgfsetstrokeopacity{0.564954}%
\pgfsetdash{}{0pt}%
\pgfpathmoveto{\pgfqpoint{0.849088in}{1.613530in}}%
\pgfpathcurveto{\pgfqpoint{0.857324in}{1.613530in}}{\pgfqpoint{0.865225in}{1.616802in}}{\pgfqpoint{0.871048in}{1.622626in}}%
\pgfpathcurveto{\pgfqpoint{0.876872in}{1.628450in}}{\pgfqpoint{0.880145in}{1.636350in}}{\pgfqpoint{0.880145in}{1.644586in}}%
\pgfpathcurveto{\pgfqpoint{0.880145in}{1.652822in}}{\pgfqpoint{0.876872in}{1.660722in}}{\pgfqpoint{0.871048in}{1.666546in}}%
\pgfpathcurveto{\pgfqpoint{0.865225in}{1.672370in}}{\pgfqpoint{0.857324in}{1.675643in}}{\pgfqpoint{0.849088in}{1.675643in}}%
\pgfpathcurveto{\pgfqpoint{0.840852in}{1.675643in}}{\pgfqpoint{0.832952in}{1.672370in}}{\pgfqpoint{0.827128in}{1.666546in}}%
\pgfpathcurveto{\pgfqpoint{0.821304in}{1.660722in}}{\pgfqpoint{0.818032in}{1.652822in}}{\pgfqpoint{0.818032in}{1.644586in}}%
\pgfpathcurveto{\pgfqpoint{0.818032in}{1.636350in}}{\pgfqpoint{0.821304in}{1.628450in}}{\pgfqpoint{0.827128in}{1.622626in}}%
\pgfpathcurveto{\pgfqpoint{0.832952in}{1.616802in}}{\pgfqpoint{0.840852in}{1.613530in}}{\pgfqpoint{0.849088in}{1.613530in}}%
\pgfpathclose%
\pgfusepath{stroke,fill}%
\end{pgfscope}%
\begin{pgfscope}%
\pgfpathrectangle{\pgfqpoint{0.100000in}{0.212622in}}{\pgfqpoint{3.696000in}{3.696000in}}%
\pgfusepath{clip}%
\pgfsetbuttcap%
\pgfsetroundjoin%
\definecolor{currentfill}{rgb}{0.121569,0.466667,0.705882}%
\pgfsetfillcolor{currentfill}%
\pgfsetfillopacity{0.565985}%
\pgfsetlinewidth{1.003750pt}%
\definecolor{currentstroke}{rgb}{0.121569,0.466667,0.705882}%
\pgfsetstrokecolor{currentstroke}%
\pgfsetstrokeopacity{0.565985}%
\pgfsetdash{}{0pt}%
\pgfpathmoveto{\pgfqpoint{0.988173in}{1.699871in}}%
\pgfpathcurveto{\pgfqpoint{0.996409in}{1.699871in}}{\pgfqpoint{1.004309in}{1.703143in}}{\pgfqpoint{1.010133in}{1.708967in}}%
\pgfpathcurveto{\pgfqpoint{1.015957in}{1.714791in}}{\pgfqpoint{1.019229in}{1.722691in}}{\pgfqpoint{1.019229in}{1.730928in}}%
\pgfpathcurveto{\pgfqpoint{1.019229in}{1.739164in}}{\pgfqpoint{1.015957in}{1.747064in}}{\pgfqpoint{1.010133in}{1.752888in}}%
\pgfpathcurveto{\pgfqpoint{1.004309in}{1.758712in}}{\pgfqpoint{0.996409in}{1.761984in}}{\pgfqpoint{0.988173in}{1.761984in}}%
\pgfpathcurveto{\pgfqpoint{0.979936in}{1.761984in}}{\pgfqpoint{0.972036in}{1.758712in}}{\pgfqpoint{0.966212in}{1.752888in}}%
\pgfpathcurveto{\pgfqpoint{0.960388in}{1.747064in}}{\pgfqpoint{0.957116in}{1.739164in}}{\pgfqpoint{0.957116in}{1.730928in}}%
\pgfpathcurveto{\pgfqpoint{0.957116in}{1.722691in}}{\pgfqpoint{0.960388in}{1.714791in}}{\pgfqpoint{0.966212in}{1.708967in}}%
\pgfpathcurveto{\pgfqpoint{0.972036in}{1.703143in}}{\pgfqpoint{0.979936in}{1.699871in}}{\pgfqpoint{0.988173in}{1.699871in}}%
\pgfpathclose%
\pgfusepath{stroke,fill}%
\end{pgfscope}%
\begin{pgfscope}%
\pgfpathrectangle{\pgfqpoint{0.100000in}{0.212622in}}{\pgfqpoint{3.696000in}{3.696000in}}%
\pgfusepath{clip}%
\pgfsetbuttcap%
\pgfsetroundjoin%
\definecolor{currentfill}{rgb}{0.121569,0.466667,0.705882}%
\pgfsetfillcolor{currentfill}%
\pgfsetfillopacity{0.566757}%
\pgfsetlinewidth{1.003750pt}%
\definecolor{currentstroke}{rgb}{0.121569,0.466667,0.705882}%
\pgfsetstrokecolor{currentstroke}%
\pgfsetstrokeopacity{0.566757}%
\pgfsetdash{}{0pt}%
\pgfpathmoveto{\pgfqpoint{0.983984in}{1.699980in}}%
\pgfpathcurveto{\pgfqpoint{0.992220in}{1.699980in}}{\pgfqpoint{1.000120in}{1.703252in}}{\pgfqpoint{1.005944in}{1.709076in}}%
\pgfpathcurveto{\pgfqpoint{1.011768in}{1.714900in}}{\pgfqpoint{1.015040in}{1.722800in}}{\pgfqpoint{1.015040in}{1.731037in}}%
\pgfpathcurveto{\pgfqpoint{1.015040in}{1.739273in}}{\pgfqpoint{1.011768in}{1.747173in}}{\pgfqpoint{1.005944in}{1.752997in}}%
\pgfpathcurveto{\pgfqpoint{1.000120in}{1.758821in}}{\pgfqpoint{0.992220in}{1.762093in}}{\pgfqpoint{0.983984in}{1.762093in}}%
\pgfpathcurveto{\pgfqpoint{0.975748in}{1.762093in}}{\pgfqpoint{0.967848in}{1.758821in}}{\pgfqpoint{0.962024in}{1.752997in}}%
\pgfpathcurveto{\pgfqpoint{0.956200in}{1.747173in}}{\pgfqpoint{0.952927in}{1.739273in}}{\pgfqpoint{0.952927in}{1.731037in}}%
\pgfpathcurveto{\pgfqpoint{0.952927in}{1.722800in}}{\pgfqpoint{0.956200in}{1.714900in}}{\pgfqpoint{0.962024in}{1.709076in}}%
\pgfpathcurveto{\pgfqpoint{0.967848in}{1.703252in}}{\pgfqpoint{0.975748in}{1.699980in}}{\pgfqpoint{0.983984in}{1.699980in}}%
\pgfpathclose%
\pgfusepath{stroke,fill}%
\end{pgfscope}%
\begin{pgfscope}%
\pgfpathrectangle{\pgfqpoint{0.100000in}{0.212622in}}{\pgfqpoint{3.696000in}{3.696000in}}%
\pgfusepath{clip}%
\pgfsetbuttcap%
\pgfsetroundjoin%
\definecolor{currentfill}{rgb}{0.121569,0.466667,0.705882}%
\pgfsetfillcolor{currentfill}%
\pgfsetfillopacity{0.568538}%
\pgfsetlinewidth{1.003750pt}%
\definecolor{currentstroke}{rgb}{0.121569,0.466667,0.705882}%
\pgfsetstrokecolor{currentstroke}%
\pgfsetstrokeopacity{0.568538}%
\pgfsetdash{}{0pt}%
\pgfpathmoveto{\pgfqpoint{0.842188in}{1.611582in}}%
\pgfpathcurveto{\pgfqpoint{0.850425in}{1.611582in}}{\pgfqpoint{0.858325in}{1.614854in}}{\pgfqpoint{0.864149in}{1.620678in}}%
\pgfpathcurveto{\pgfqpoint{0.869973in}{1.626502in}}{\pgfqpoint{0.873245in}{1.634402in}}{\pgfqpoint{0.873245in}{1.642638in}}%
\pgfpathcurveto{\pgfqpoint{0.873245in}{1.650874in}}{\pgfqpoint{0.869973in}{1.658775in}}{\pgfqpoint{0.864149in}{1.664598in}}%
\pgfpathcurveto{\pgfqpoint{0.858325in}{1.670422in}}{\pgfqpoint{0.850425in}{1.673695in}}{\pgfqpoint{0.842188in}{1.673695in}}%
\pgfpathcurveto{\pgfqpoint{0.833952in}{1.673695in}}{\pgfqpoint{0.826052in}{1.670422in}}{\pgfqpoint{0.820228in}{1.664598in}}%
\pgfpathcurveto{\pgfqpoint{0.814404in}{1.658775in}}{\pgfqpoint{0.811132in}{1.650874in}}{\pgfqpoint{0.811132in}{1.642638in}}%
\pgfpathcurveto{\pgfqpoint{0.811132in}{1.634402in}}{\pgfqpoint{0.814404in}{1.626502in}}{\pgfqpoint{0.820228in}{1.620678in}}%
\pgfpathcurveto{\pgfqpoint{0.826052in}{1.614854in}}{\pgfqpoint{0.833952in}{1.611582in}}{\pgfqpoint{0.842188in}{1.611582in}}%
\pgfpathclose%
\pgfusepath{stroke,fill}%
\end{pgfscope}%
\begin{pgfscope}%
\pgfpathrectangle{\pgfqpoint{0.100000in}{0.212622in}}{\pgfqpoint{3.696000in}{3.696000in}}%
\pgfusepath{clip}%
\pgfsetbuttcap%
\pgfsetroundjoin%
\definecolor{currentfill}{rgb}{0.121569,0.466667,0.705882}%
\pgfsetfillcolor{currentfill}%
\pgfsetfillopacity{0.568749}%
\pgfsetlinewidth{1.003750pt}%
\definecolor{currentstroke}{rgb}{0.121569,0.466667,0.705882}%
\pgfsetstrokecolor{currentstroke}%
\pgfsetstrokeopacity{0.568749}%
\pgfsetdash{}{0pt}%
\pgfpathmoveto{\pgfqpoint{0.981192in}{1.697639in}}%
\pgfpathcurveto{\pgfqpoint{0.989428in}{1.697639in}}{\pgfqpoint{0.997329in}{1.700912in}}{\pgfqpoint{1.003152in}{1.706736in}}%
\pgfpathcurveto{\pgfqpoint{1.008976in}{1.712559in}}{\pgfqpoint{1.012249in}{1.720460in}}{\pgfqpoint{1.012249in}{1.728696in}}%
\pgfpathcurveto{\pgfqpoint{1.012249in}{1.736932in}}{\pgfqpoint{1.008976in}{1.744832in}}{\pgfqpoint{1.003152in}{1.750656in}}%
\pgfpathcurveto{\pgfqpoint{0.997329in}{1.756480in}}{\pgfqpoint{0.989428in}{1.759752in}}{\pgfqpoint{0.981192in}{1.759752in}}%
\pgfpathcurveto{\pgfqpoint{0.972956in}{1.759752in}}{\pgfqpoint{0.965056in}{1.756480in}}{\pgfqpoint{0.959232in}{1.750656in}}%
\pgfpathcurveto{\pgfqpoint{0.953408in}{1.744832in}}{\pgfqpoint{0.950136in}{1.736932in}}{\pgfqpoint{0.950136in}{1.728696in}}%
\pgfpathcurveto{\pgfqpoint{0.950136in}{1.720460in}}{\pgfqpoint{0.953408in}{1.712559in}}{\pgfqpoint{0.959232in}{1.706736in}}%
\pgfpathcurveto{\pgfqpoint{0.965056in}{1.700912in}}{\pgfqpoint{0.972956in}{1.697639in}}{\pgfqpoint{0.981192in}{1.697639in}}%
\pgfpathclose%
\pgfusepath{stroke,fill}%
\end{pgfscope}%
\begin{pgfscope}%
\pgfpathrectangle{\pgfqpoint{0.100000in}{0.212622in}}{\pgfqpoint{3.696000in}{3.696000in}}%
\pgfusepath{clip}%
\pgfsetbuttcap%
\pgfsetroundjoin%
\definecolor{currentfill}{rgb}{0.121569,0.466667,0.705882}%
\pgfsetfillcolor{currentfill}%
\pgfsetfillopacity{0.571981}%
\pgfsetlinewidth{1.003750pt}%
\definecolor{currentstroke}{rgb}{0.121569,0.466667,0.705882}%
\pgfsetstrokecolor{currentstroke}%
\pgfsetstrokeopacity{0.571981}%
\pgfsetdash{}{0pt}%
\pgfpathmoveto{\pgfqpoint{0.966749in}{1.703449in}}%
\pgfpathcurveto{\pgfqpoint{0.974986in}{1.703449in}}{\pgfqpoint{0.982886in}{1.706721in}}{\pgfqpoint{0.988710in}{1.712545in}}%
\pgfpathcurveto{\pgfqpoint{0.994534in}{1.718369in}}{\pgfqpoint{0.997806in}{1.726269in}}{\pgfqpoint{0.997806in}{1.734506in}}%
\pgfpathcurveto{\pgfqpoint{0.997806in}{1.742742in}}{\pgfqpoint{0.994534in}{1.750642in}}{\pgfqpoint{0.988710in}{1.756466in}}%
\pgfpathcurveto{\pgfqpoint{0.982886in}{1.762290in}}{\pgfqpoint{0.974986in}{1.765562in}}{\pgfqpoint{0.966749in}{1.765562in}}%
\pgfpathcurveto{\pgfqpoint{0.958513in}{1.765562in}}{\pgfqpoint{0.950613in}{1.762290in}}{\pgfqpoint{0.944789in}{1.756466in}}%
\pgfpathcurveto{\pgfqpoint{0.938965in}{1.750642in}}{\pgfqpoint{0.935693in}{1.742742in}}{\pgfqpoint{0.935693in}{1.734506in}}%
\pgfpathcurveto{\pgfqpoint{0.935693in}{1.726269in}}{\pgfqpoint{0.938965in}{1.718369in}}{\pgfqpoint{0.944789in}{1.712545in}}%
\pgfpathcurveto{\pgfqpoint{0.950613in}{1.706721in}}{\pgfqpoint{0.958513in}{1.703449in}}{\pgfqpoint{0.966749in}{1.703449in}}%
\pgfpathclose%
\pgfusepath{stroke,fill}%
\end{pgfscope}%
\begin{pgfscope}%
\pgfpathrectangle{\pgfqpoint{0.100000in}{0.212622in}}{\pgfqpoint{3.696000in}{3.696000in}}%
\pgfusepath{clip}%
\pgfsetbuttcap%
\pgfsetroundjoin%
\definecolor{currentfill}{rgb}{0.121569,0.466667,0.705882}%
\pgfsetfillcolor{currentfill}%
\pgfsetfillopacity{0.572619}%
\pgfsetlinewidth{1.003750pt}%
\definecolor{currentstroke}{rgb}{0.121569,0.466667,0.705882}%
\pgfsetstrokecolor{currentstroke}%
\pgfsetstrokeopacity{0.572619}%
\pgfsetdash{}{0pt}%
\pgfpathmoveto{\pgfqpoint{1.955838in}{1.958499in}}%
\pgfpathcurveto{\pgfqpoint{1.964074in}{1.958499in}}{\pgfqpoint{1.971974in}{1.961771in}}{\pgfqpoint{1.977798in}{1.967595in}}%
\pgfpathcurveto{\pgfqpoint{1.983622in}{1.973419in}}{\pgfqpoint{1.986895in}{1.981319in}}{\pgfqpoint{1.986895in}{1.989556in}}%
\pgfpathcurveto{\pgfqpoint{1.986895in}{1.997792in}}{\pgfqpoint{1.983622in}{2.005692in}}{\pgfqpoint{1.977798in}{2.011516in}}%
\pgfpathcurveto{\pgfqpoint{1.971974in}{2.017340in}}{\pgfqpoint{1.964074in}{2.020612in}}{\pgfqpoint{1.955838in}{2.020612in}}%
\pgfpathcurveto{\pgfqpoint{1.947602in}{2.020612in}}{\pgfqpoint{1.939702in}{2.017340in}}{\pgfqpoint{1.933878in}{2.011516in}}%
\pgfpathcurveto{\pgfqpoint{1.928054in}{2.005692in}}{\pgfqpoint{1.924782in}{1.997792in}}{\pgfqpoint{1.924782in}{1.989556in}}%
\pgfpathcurveto{\pgfqpoint{1.924782in}{1.981319in}}{\pgfqpoint{1.928054in}{1.973419in}}{\pgfqpoint{1.933878in}{1.967595in}}%
\pgfpathcurveto{\pgfqpoint{1.939702in}{1.961771in}}{\pgfqpoint{1.947602in}{1.958499in}}{\pgfqpoint{1.955838in}{1.958499in}}%
\pgfpathclose%
\pgfusepath{stroke,fill}%
\end{pgfscope}%
\begin{pgfscope}%
\pgfpathrectangle{\pgfqpoint{0.100000in}{0.212622in}}{\pgfqpoint{3.696000in}{3.696000in}}%
\pgfusepath{clip}%
\pgfsetbuttcap%
\pgfsetroundjoin%
\definecolor{currentfill}{rgb}{0.121569,0.466667,0.705882}%
\pgfsetfillcolor{currentfill}%
\pgfsetfillopacity{0.573137}%
\pgfsetlinewidth{1.003750pt}%
\definecolor{currentstroke}{rgb}{0.121569,0.466667,0.705882}%
\pgfsetstrokecolor{currentstroke}%
\pgfsetstrokeopacity{0.573137}%
\pgfsetdash{}{0pt}%
\pgfpathmoveto{\pgfqpoint{0.964973in}{1.701470in}}%
\pgfpathcurveto{\pgfqpoint{0.973210in}{1.701470in}}{\pgfqpoint{0.981110in}{1.704742in}}{\pgfqpoint{0.986934in}{1.710566in}}%
\pgfpathcurveto{\pgfqpoint{0.992758in}{1.716390in}}{\pgfqpoint{0.996030in}{1.724290in}}{\pgfqpoint{0.996030in}{1.732526in}}%
\pgfpathcurveto{\pgfqpoint{0.996030in}{1.740762in}}{\pgfqpoint{0.992758in}{1.748662in}}{\pgfqpoint{0.986934in}{1.754486in}}%
\pgfpathcurveto{\pgfqpoint{0.981110in}{1.760310in}}{\pgfqpoint{0.973210in}{1.763583in}}{\pgfqpoint{0.964973in}{1.763583in}}%
\pgfpathcurveto{\pgfqpoint{0.956737in}{1.763583in}}{\pgfqpoint{0.948837in}{1.760310in}}{\pgfqpoint{0.943013in}{1.754486in}}%
\pgfpathcurveto{\pgfqpoint{0.937189in}{1.748662in}}{\pgfqpoint{0.933917in}{1.740762in}}{\pgfqpoint{0.933917in}{1.732526in}}%
\pgfpathcurveto{\pgfqpoint{0.933917in}{1.724290in}}{\pgfqpoint{0.937189in}{1.716390in}}{\pgfqpoint{0.943013in}{1.710566in}}%
\pgfpathcurveto{\pgfqpoint{0.948837in}{1.704742in}}{\pgfqpoint{0.956737in}{1.701470in}}{\pgfqpoint{0.964973in}{1.701470in}}%
\pgfpathclose%
\pgfusepath{stroke,fill}%
\end{pgfscope}%
\begin{pgfscope}%
\pgfpathrectangle{\pgfqpoint{0.100000in}{0.212622in}}{\pgfqpoint{3.696000in}{3.696000in}}%
\pgfusepath{clip}%
\pgfsetbuttcap%
\pgfsetroundjoin%
\definecolor{currentfill}{rgb}{0.121569,0.466667,0.705882}%
\pgfsetfillcolor{currentfill}%
\pgfsetfillopacity{0.573580}%
\pgfsetlinewidth{1.003750pt}%
\definecolor{currentstroke}{rgb}{0.121569,0.466667,0.705882}%
\pgfsetstrokecolor{currentstroke}%
\pgfsetstrokeopacity{0.573580}%
\pgfsetdash{}{0pt}%
\pgfpathmoveto{\pgfqpoint{0.962644in}{1.700199in}}%
\pgfpathcurveto{\pgfqpoint{0.970881in}{1.700199in}}{\pgfqpoint{0.978781in}{1.703471in}}{\pgfqpoint{0.984605in}{1.709295in}}%
\pgfpathcurveto{\pgfqpoint{0.990428in}{1.715119in}}{\pgfqpoint{0.993701in}{1.723019in}}{\pgfqpoint{0.993701in}{1.731256in}}%
\pgfpathcurveto{\pgfqpoint{0.993701in}{1.739492in}}{\pgfqpoint{0.990428in}{1.747392in}}{\pgfqpoint{0.984605in}{1.753216in}}%
\pgfpathcurveto{\pgfqpoint{0.978781in}{1.759040in}}{\pgfqpoint{0.970881in}{1.762312in}}{\pgfqpoint{0.962644in}{1.762312in}}%
\pgfpathcurveto{\pgfqpoint{0.954408in}{1.762312in}}{\pgfqpoint{0.946508in}{1.759040in}}{\pgfqpoint{0.940684in}{1.753216in}}%
\pgfpathcurveto{\pgfqpoint{0.934860in}{1.747392in}}{\pgfqpoint{0.931588in}{1.739492in}}{\pgfqpoint{0.931588in}{1.731256in}}%
\pgfpathcurveto{\pgfqpoint{0.931588in}{1.723019in}}{\pgfqpoint{0.934860in}{1.715119in}}{\pgfqpoint{0.940684in}{1.709295in}}%
\pgfpathcurveto{\pgfqpoint{0.946508in}{1.703471in}}{\pgfqpoint{0.954408in}{1.700199in}}{\pgfqpoint{0.962644in}{1.700199in}}%
\pgfpathclose%
\pgfusepath{stroke,fill}%
\end{pgfscope}%
\begin{pgfscope}%
\pgfpathrectangle{\pgfqpoint{0.100000in}{0.212622in}}{\pgfqpoint{3.696000in}{3.696000in}}%
\pgfusepath{clip}%
\pgfsetbuttcap%
\pgfsetroundjoin%
\definecolor{currentfill}{rgb}{0.121569,0.466667,0.705882}%
\pgfsetfillcolor{currentfill}%
\pgfsetfillopacity{0.573595}%
\pgfsetlinewidth{1.003750pt}%
\definecolor{currentstroke}{rgb}{0.121569,0.466667,0.705882}%
\pgfsetstrokecolor{currentstroke}%
\pgfsetstrokeopacity{0.573595}%
\pgfsetdash{}{0pt}%
\pgfpathmoveto{\pgfqpoint{0.833604in}{1.609531in}}%
\pgfpathcurveto{\pgfqpoint{0.841840in}{1.609531in}}{\pgfqpoint{0.849740in}{1.612804in}}{\pgfqpoint{0.855564in}{1.618628in}}%
\pgfpathcurveto{\pgfqpoint{0.861388in}{1.624451in}}{\pgfqpoint{0.864661in}{1.632352in}}{\pgfqpoint{0.864661in}{1.640588in}}%
\pgfpathcurveto{\pgfqpoint{0.864661in}{1.648824in}}{\pgfqpoint{0.861388in}{1.656724in}}{\pgfqpoint{0.855564in}{1.662548in}}%
\pgfpathcurveto{\pgfqpoint{0.849740in}{1.668372in}}{\pgfqpoint{0.841840in}{1.671644in}}{\pgfqpoint{0.833604in}{1.671644in}}%
\pgfpathcurveto{\pgfqpoint{0.825368in}{1.671644in}}{\pgfqpoint{0.817468in}{1.668372in}}{\pgfqpoint{0.811644in}{1.662548in}}%
\pgfpathcurveto{\pgfqpoint{0.805820in}{1.656724in}}{\pgfqpoint{0.802548in}{1.648824in}}{\pgfqpoint{0.802548in}{1.640588in}}%
\pgfpathcurveto{\pgfqpoint{0.802548in}{1.632352in}}{\pgfqpoint{0.805820in}{1.624451in}}{\pgfqpoint{0.811644in}{1.618628in}}%
\pgfpathcurveto{\pgfqpoint{0.817468in}{1.612804in}}{\pgfqpoint{0.825368in}{1.609531in}}{\pgfqpoint{0.833604in}{1.609531in}}%
\pgfpathclose%
\pgfusepath{stroke,fill}%
\end{pgfscope}%
\begin{pgfscope}%
\pgfpathrectangle{\pgfqpoint{0.100000in}{0.212622in}}{\pgfqpoint{3.696000in}{3.696000in}}%
\pgfusepath{clip}%
\pgfsetbuttcap%
\pgfsetroundjoin%
\definecolor{currentfill}{rgb}{0.121569,0.466667,0.705882}%
\pgfsetfillcolor{currentfill}%
\pgfsetfillopacity{0.573753}%
\pgfsetlinewidth{1.003750pt}%
\definecolor{currentstroke}{rgb}{0.121569,0.466667,0.705882}%
\pgfsetstrokecolor{currentstroke}%
\pgfsetstrokeopacity{0.573753}%
\pgfsetdash{}{0pt}%
\pgfpathmoveto{\pgfqpoint{0.962243in}{1.699772in}}%
\pgfpathcurveto{\pgfqpoint{0.970480in}{1.699772in}}{\pgfqpoint{0.978380in}{1.703045in}}{\pgfqpoint{0.984204in}{1.708869in}}%
\pgfpathcurveto{\pgfqpoint{0.990028in}{1.714693in}}{\pgfqpoint{0.993300in}{1.722593in}}{\pgfqpoint{0.993300in}{1.730829in}}%
\pgfpathcurveto{\pgfqpoint{0.993300in}{1.739065in}}{\pgfqpoint{0.990028in}{1.746965in}}{\pgfqpoint{0.984204in}{1.752789in}}%
\pgfpathcurveto{\pgfqpoint{0.978380in}{1.758613in}}{\pgfqpoint{0.970480in}{1.761885in}}{\pgfqpoint{0.962243in}{1.761885in}}%
\pgfpathcurveto{\pgfqpoint{0.954007in}{1.761885in}}{\pgfqpoint{0.946107in}{1.758613in}}{\pgfqpoint{0.940283in}{1.752789in}}%
\pgfpathcurveto{\pgfqpoint{0.934459in}{1.746965in}}{\pgfqpoint{0.931187in}{1.739065in}}{\pgfqpoint{0.931187in}{1.730829in}}%
\pgfpathcurveto{\pgfqpoint{0.931187in}{1.722593in}}{\pgfqpoint{0.934459in}{1.714693in}}{\pgfqpoint{0.940283in}{1.708869in}}%
\pgfpathcurveto{\pgfqpoint{0.946107in}{1.703045in}}{\pgfqpoint{0.954007in}{1.699772in}}{\pgfqpoint{0.962243in}{1.699772in}}%
\pgfpathclose%
\pgfusepath{stroke,fill}%
\end{pgfscope}%
\begin{pgfscope}%
\pgfpathrectangle{\pgfqpoint{0.100000in}{0.212622in}}{\pgfqpoint{3.696000in}{3.696000in}}%
\pgfusepath{clip}%
\pgfsetbuttcap%
\pgfsetroundjoin%
\definecolor{currentfill}{rgb}{0.121569,0.466667,0.705882}%
\pgfsetfillcolor{currentfill}%
\pgfsetfillopacity{0.573992}%
\pgfsetlinewidth{1.003750pt}%
\definecolor{currentstroke}{rgb}{0.121569,0.466667,0.705882}%
\pgfsetstrokecolor{currentstroke}%
\pgfsetstrokeopacity{0.573992}%
\pgfsetdash{}{0pt}%
\pgfpathmoveto{\pgfqpoint{0.961142in}{1.698945in}}%
\pgfpathcurveto{\pgfqpoint{0.969379in}{1.698945in}}{\pgfqpoint{0.977279in}{1.702217in}}{\pgfqpoint{0.983102in}{1.708041in}}%
\pgfpathcurveto{\pgfqpoint{0.988926in}{1.713865in}}{\pgfqpoint{0.992199in}{1.721765in}}{\pgfqpoint{0.992199in}{1.730001in}}%
\pgfpathcurveto{\pgfqpoint{0.992199in}{1.738238in}}{\pgfqpoint{0.988926in}{1.746138in}}{\pgfqpoint{0.983102in}{1.751962in}}%
\pgfpathcurveto{\pgfqpoint{0.977279in}{1.757786in}}{\pgfqpoint{0.969379in}{1.761058in}}{\pgfqpoint{0.961142in}{1.761058in}}%
\pgfpathcurveto{\pgfqpoint{0.952906in}{1.761058in}}{\pgfqpoint{0.945006in}{1.757786in}}{\pgfqpoint{0.939182in}{1.751962in}}%
\pgfpathcurveto{\pgfqpoint{0.933358in}{1.746138in}}{\pgfqpoint{0.930086in}{1.738238in}}{\pgfqpoint{0.930086in}{1.730001in}}%
\pgfpathcurveto{\pgfqpoint{0.930086in}{1.721765in}}{\pgfqpoint{0.933358in}{1.713865in}}{\pgfqpoint{0.939182in}{1.708041in}}%
\pgfpathcurveto{\pgfqpoint{0.945006in}{1.702217in}}{\pgfqpoint{0.952906in}{1.698945in}}{\pgfqpoint{0.961142in}{1.698945in}}%
\pgfpathclose%
\pgfusepath{stroke,fill}%
\end{pgfscope}%
\begin{pgfscope}%
\pgfpathrectangle{\pgfqpoint{0.100000in}{0.212622in}}{\pgfqpoint{3.696000in}{3.696000in}}%
\pgfusepath{clip}%
\pgfsetbuttcap%
\pgfsetroundjoin%
\definecolor{currentfill}{rgb}{0.121569,0.466667,0.705882}%
\pgfsetfillcolor{currentfill}%
\pgfsetfillopacity{0.574528}%
\pgfsetlinewidth{1.003750pt}%
\definecolor{currentstroke}{rgb}{0.121569,0.466667,0.705882}%
\pgfsetstrokecolor{currentstroke}%
\pgfsetstrokeopacity{0.574528}%
\pgfsetdash{}{0pt}%
\pgfpathmoveto{\pgfqpoint{0.959662in}{1.697460in}}%
\pgfpathcurveto{\pgfqpoint{0.967898in}{1.697460in}}{\pgfqpoint{0.975799in}{1.700733in}}{\pgfqpoint{0.981622in}{1.706556in}}%
\pgfpathcurveto{\pgfqpoint{0.987446in}{1.712380in}}{\pgfqpoint{0.990719in}{1.720280in}}{\pgfqpoint{0.990719in}{1.728517in}}%
\pgfpathcurveto{\pgfqpoint{0.990719in}{1.736753in}}{\pgfqpoint{0.987446in}{1.744653in}}{\pgfqpoint{0.981622in}{1.750477in}}%
\pgfpathcurveto{\pgfqpoint{0.975799in}{1.756301in}}{\pgfqpoint{0.967898in}{1.759573in}}{\pgfqpoint{0.959662in}{1.759573in}}%
\pgfpathcurveto{\pgfqpoint{0.951426in}{1.759573in}}{\pgfqpoint{0.943526in}{1.756301in}}{\pgfqpoint{0.937702in}{1.750477in}}%
\pgfpathcurveto{\pgfqpoint{0.931878in}{1.744653in}}{\pgfqpoint{0.928606in}{1.736753in}}{\pgfqpoint{0.928606in}{1.728517in}}%
\pgfpathcurveto{\pgfqpoint{0.928606in}{1.720280in}}{\pgfqpoint{0.931878in}{1.712380in}}{\pgfqpoint{0.937702in}{1.706556in}}%
\pgfpathcurveto{\pgfqpoint{0.943526in}{1.700733in}}{\pgfqpoint{0.951426in}{1.697460in}}{\pgfqpoint{0.959662in}{1.697460in}}%
\pgfpathclose%
\pgfusepath{stroke,fill}%
\end{pgfscope}%
\begin{pgfscope}%
\pgfpathrectangle{\pgfqpoint{0.100000in}{0.212622in}}{\pgfqpoint{3.696000in}{3.696000in}}%
\pgfusepath{clip}%
\pgfsetbuttcap%
\pgfsetroundjoin%
\definecolor{currentfill}{rgb}{0.121569,0.466667,0.705882}%
\pgfsetfillcolor{currentfill}%
\pgfsetfillopacity{0.574565}%
\pgfsetlinewidth{1.003750pt}%
\definecolor{currentstroke}{rgb}{0.121569,0.466667,0.705882}%
\pgfsetstrokecolor{currentstroke}%
\pgfsetstrokeopacity{0.574565}%
\pgfsetdash{}{0pt}%
\pgfpathmoveto{\pgfqpoint{0.959491in}{1.697363in}}%
\pgfpathcurveto{\pgfqpoint{0.967727in}{1.697363in}}{\pgfqpoint{0.975627in}{1.700635in}}{\pgfqpoint{0.981451in}{1.706459in}}%
\pgfpathcurveto{\pgfqpoint{0.987275in}{1.712283in}}{\pgfqpoint{0.990548in}{1.720183in}}{\pgfqpoint{0.990548in}{1.728419in}}%
\pgfpathcurveto{\pgfqpoint{0.990548in}{1.736655in}}{\pgfqpoint{0.987275in}{1.744555in}}{\pgfqpoint{0.981451in}{1.750379in}}%
\pgfpathcurveto{\pgfqpoint{0.975627in}{1.756203in}}{\pgfqpoint{0.967727in}{1.759476in}}{\pgfqpoint{0.959491in}{1.759476in}}%
\pgfpathcurveto{\pgfqpoint{0.951255in}{1.759476in}}{\pgfqpoint{0.943355in}{1.756203in}}{\pgfqpoint{0.937531in}{1.750379in}}%
\pgfpathcurveto{\pgfqpoint{0.931707in}{1.744555in}}{\pgfqpoint{0.928435in}{1.736655in}}{\pgfqpoint{0.928435in}{1.728419in}}%
\pgfpathcurveto{\pgfqpoint{0.928435in}{1.720183in}}{\pgfqpoint{0.931707in}{1.712283in}}{\pgfqpoint{0.937531in}{1.706459in}}%
\pgfpathcurveto{\pgfqpoint{0.943355in}{1.700635in}}{\pgfqpoint{0.951255in}{1.697363in}}{\pgfqpoint{0.959491in}{1.697363in}}%
\pgfpathclose%
\pgfusepath{stroke,fill}%
\end{pgfscope}%
\begin{pgfscope}%
\pgfpathrectangle{\pgfqpoint{0.100000in}{0.212622in}}{\pgfqpoint{3.696000in}{3.696000in}}%
\pgfusepath{clip}%
\pgfsetbuttcap%
\pgfsetroundjoin%
\definecolor{currentfill}{rgb}{0.121569,0.466667,0.705882}%
\pgfsetfillcolor{currentfill}%
\pgfsetfillopacity{0.574640}%
\pgfsetlinewidth{1.003750pt}%
\definecolor{currentstroke}{rgb}{0.121569,0.466667,0.705882}%
\pgfsetstrokecolor{currentstroke}%
\pgfsetstrokeopacity{0.574640}%
\pgfsetdash{}{0pt}%
\pgfpathmoveto{\pgfqpoint{0.959265in}{1.697132in}}%
\pgfpathcurveto{\pgfqpoint{0.967502in}{1.697132in}}{\pgfqpoint{0.975402in}{1.700404in}}{\pgfqpoint{0.981226in}{1.706228in}}%
\pgfpathcurveto{\pgfqpoint{0.987050in}{1.712052in}}{\pgfqpoint{0.990322in}{1.719952in}}{\pgfqpoint{0.990322in}{1.728188in}}%
\pgfpathcurveto{\pgfqpoint{0.990322in}{1.736425in}}{\pgfqpoint{0.987050in}{1.744325in}}{\pgfqpoint{0.981226in}{1.750149in}}%
\pgfpathcurveto{\pgfqpoint{0.975402in}{1.755973in}}{\pgfqpoint{0.967502in}{1.759245in}}{\pgfqpoint{0.959265in}{1.759245in}}%
\pgfpathcurveto{\pgfqpoint{0.951029in}{1.759245in}}{\pgfqpoint{0.943129in}{1.755973in}}{\pgfqpoint{0.937305in}{1.750149in}}%
\pgfpathcurveto{\pgfqpoint{0.931481in}{1.744325in}}{\pgfqpoint{0.928209in}{1.736425in}}{\pgfqpoint{0.928209in}{1.728188in}}%
\pgfpathcurveto{\pgfqpoint{0.928209in}{1.719952in}}{\pgfqpoint{0.931481in}{1.712052in}}{\pgfqpoint{0.937305in}{1.706228in}}%
\pgfpathcurveto{\pgfqpoint{0.943129in}{1.700404in}}{\pgfqpoint{0.951029in}{1.697132in}}{\pgfqpoint{0.959265in}{1.697132in}}%
\pgfpathclose%
\pgfusepath{stroke,fill}%
\end{pgfscope}%
\begin{pgfscope}%
\pgfpathrectangle{\pgfqpoint{0.100000in}{0.212622in}}{\pgfqpoint{3.696000in}{3.696000in}}%
\pgfusepath{clip}%
\pgfsetbuttcap%
\pgfsetroundjoin%
\definecolor{currentfill}{rgb}{0.121569,0.466667,0.705882}%
\pgfsetfillcolor{currentfill}%
\pgfsetfillopacity{0.574760}%
\pgfsetlinewidth{1.003750pt}%
\definecolor{currentstroke}{rgb}{0.121569,0.466667,0.705882}%
\pgfsetstrokecolor{currentstroke}%
\pgfsetstrokeopacity{0.574760}%
\pgfsetdash{}{0pt}%
\pgfpathmoveto{\pgfqpoint{0.958685in}{1.696816in}}%
\pgfpathcurveto{\pgfqpoint{0.966921in}{1.696816in}}{\pgfqpoint{0.974821in}{1.700089in}}{\pgfqpoint{0.980645in}{1.705913in}}%
\pgfpathcurveto{\pgfqpoint{0.986469in}{1.711737in}}{\pgfqpoint{0.989741in}{1.719637in}}{\pgfqpoint{0.989741in}{1.727873in}}%
\pgfpathcurveto{\pgfqpoint{0.989741in}{1.736109in}}{\pgfqpoint{0.986469in}{1.744009in}}{\pgfqpoint{0.980645in}{1.749833in}}%
\pgfpathcurveto{\pgfqpoint{0.974821in}{1.755657in}}{\pgfqpoint{0.966921in}{1.758929in}}{\pgfqpoint{0.958685in}{1.758929in}}%
\pgfpathcurveto{\pgfqpoint{0.950449in}{1.758929in}}{\pgfqpoint{0.942549in}{1.755657in}}{\pgfqpoint{0.936725in}{1.749833in}}%
\pgfpathcurveto{\pgfqpoint{0.930901in}{1.744009in}}{\pgfqpoint{0.927628in}{1.736109in}}{\pgfqpoint{0.927628in}{1.727873in}}%
\pgfpathcurveto{\pgfqpoint{0.927628in}{1.719637in}}{\pgfqpoint{0.930901in}{1.711737in}}{\pgfqpoint{0.936725in}{1.705913in}}%
\pgfpathcurveto{\pgfqpoint{0.942549in}{1.700089in}}{\pgfqpoint{0.950449in}{1.696816in}}{\pgfqpoint{0.958685in}{1.696816in}}%
\pgfpathclose%
\pgfusepath{stroke,fill}%
\end{pgfscope}%
\begin{pgfscope}%
\pgfpathrectangle{\pgfqpoint{0.100000in}{0.212622in}}{\pgfqpoint{3.696000in}{3.696000in}}%
\pgfusepath{clip}%
\pgfsetbuttcap%
\pgfsetroundjoin%
\definecolor{currentfill}{rgb}{0.121569,0.466667,0.705882}%
\pgfsetfillcolor{currentfill}%
\pgfsetfillopacity{0.575020}%
\pgfsetlinewidth{1.003750pt}%
\definecolor{currentstroke}{rgb}{0.121569,0.466667,0.705882}%
\pgfsetstrokecolor{currentstroke}%
\pgfsetstrokeopacity{0.575020}%
\pgfsetdash{}{0pt}%
\pgfpathmoveto{\pgfqpoint{0.958039in}{1.696009in}}%
\pgfpathcurveto{\pgfqpoint{0.966275in}{1.696009in}}{\pgfqpoint{0.974175in}{1.699281in}}{\pgfqpoint{0.979999in}{1.705105in}}%
\pgfpathcurveto{\pgfqpoint{0.985823in}{1.710929in}}{\pgfqpoint{0.989096in}{1.718829in}}{\pgfqpoint{0.989096in}{1.727066in}}%
\pgfpathcurveto{\pgfqpoint{0.989096in}{1.735302in}}{\pgfqpoint{0.985823in}{1.743202in}}{\pgfqpoint{0.979999in}{1.749026in}}%
\pgfpathcurveto{\pgfqpoint{0.974175in}{1.754850in}}{\pgfqpoint{0.966275in}{1.758122in}}{\pgfqpoint{0.958039in}{1.758122in}}%
\pgfpathcurveto{\pgfqpoint{0.949803in}{1.758122in}}{\pgfqpoint{0.941903in}{1.754850in}}{\pgfqpoint{0.936079in}{1.749026in}}%
\pgfpathcurveto{\pgfqpoint{0.930255in}{1.743202in}}{\pgfqpoint{0.926983in}{1.735302in}}{\pgfqpoint{0.926983in}{1.727066in}}%
\pgfpathcurveto{\pgfqpoint{0.926983in}{1.718829in}}{\pgfqpoint{0.930255in}{1.710929in}}{\pgfqpoint{0.936079in}{1.705105in}}%
\pgfpathcurveto{\pgfqpoint{0.941903in}{1.699281in}}{\pgfqpoint{0.949803in}{1.696009in}}{\pgfqpoint{0.958039in}{1.696009in}}%
\pgfpathclose%
\pgfusepath{stroke,fill}%
\end{pgfscope}%
\begin{pgfscope}%
\pgfpathrectangle{\pgfqpoint{0.100000in}{0.212622in}}{\pgfqpoint{3.696000in}{3.696000in}}%
\pgfusepath{clip}%
\pgfsetbuttcap%
\pgfsetroundjoin%
\definecolor{currentfill}{rgb}{0.121569,0.466667,0.705882}%
\pgfsetfillcolor{currentfill}%
\pgfsetfillopacity{0.575450}%
\pgfsetlinewidth{1.003750pt}%
\definecolor{currentstroke}{rgb}{0.121569,0.466667,0.705882}%
\pgfsetstrokecolor{currentstroke}%
\pgfsetstrokeopacity{0.575450}%
\pgfsetdash{}{0pt}%
\pgfpathmoveto{\pgfqpoint{0.956272in}{1.694980in}}%
\pgfpathcurveto{\pgfqpoint{0.964508in}{1.694980in}}{\pgfqpoint{0.972408in}{1.698252in}}{\pgfqpoint{0.978232in}{1.704076in}}%
\pgfpathcurveto{\pgfqpoint{0.984056in}{1.709900in}}{\pgfqpoint{0.987329in}{1.717800in}}{\pgfqpoint{0.987329in}{1.726036in}}%
\pgfpathcurveto{\pgfqpoint{0.987329in}{1.734272in}}{\pgfqpoint{0.984056in}{1.742172in}}{\pgfqpoint{0.978232in}{1.747996in}}%
\pgfpathcurveto{\pgfqpoint{0.972408in}{1.753820in}}{\pgfqpoint{0.964508in}{1.757093in}}{\pgfqpoint{0.956272in}{1.757093in}}%
\pgfpathcurveto{\pgfqpoint{0.948036in}{1.757093in}}{\pgfqpoint{0.940136in}{1.753820in}}{\pgfqpoint{0.934312in}{1.747996in}}%
\pgfpathcurveto{\pgfqpoint{0.928488in}{1.742172in}}{\pgfqpoint{0.925216in}{1.734272in}}{\pgfqpoint{0.925216in}{1.726036in}}%
\pgfpathcurveto{\pgfqpoint{0.925216in}{1.717800in}}{\pgfqpoint{0.928488in}{1.709900in}}{\pgfqpoint{0.934312in}{1.704076in}}%
\pgfpathcurveto{\pgfqpoint{0.940136in}{1.698252in}}{\pgfqpoint{0.948036in}{1.694980in}}{\pgfqpoint{0.956272in}{1.694980in}}%
\pgfpathclose%
\pgfusepath{stroke,fill}%
\end{pgfscope}%
\begin{pgfscope}%
\pgfpathrectangle{\pgfqpoint{0.100000in}{0.212622in}}{\pgfqpoint{3.696000in}{3.696000in}}%
\pgfusepath{clip}%
\pgfsetbuttcap%
\pgfsetroundjoin%
\definecolor{currentfill}{rgb}{0.121569,0.466667,0.705882}%
\pgfsetfillcolor{currentfill}%
\pgfsetfillopacity{0.575601}%
\pgfsetlinewidth{1.003750pt}%
\definecolor{currentstroke}{rgb}{0.121569,0.466667,0.705882}%
\pgfsetstrokecolor{currentstroke}%
\pgfsetstrokeopacity{0.575601}%
\pgfsetdash{}{0pt}%
\pgfpathmoveto{\pgfqpoint{0.955949in}{1.694615in}}%
\pgfpathcurveto{\pgfqpoint{0.964185in}{1.694615in}}{\pgfqpoint{0.972085in}{1.697887in}}{\pgfqpoint{0.977909in}{1.703711in}}%
\pgfpathcurveto{\pgfqpoint{0.983733in}{1.709535in}}{\pgfqpoint{0.987006in}{1.717435in}}{\pgfqpoint{0.987006in}{1.725671in}}%
\pgfpathcurveto{\pgfqpoint{0.987006in}{1.733907in}}{\pgfqpoint{0.983733in}{1.741807in}}{\pgfqpoint{0.977909in}{1.747631in}}%
\pgfpathcurveto{\pgfqpoint{0.972085in}{1.753455in}}{\pgfqpoint{0.964185in}{1.756728in}}{\pgfqpoint{0.955949in}{1.756728in}}%
\pgfpathcurveto{\pgfqpoint{0.947713in}{1.756728in}}{\pgfqpoint{0.939813in}{1.753455in}}{\pgfqpoint{0.933989in}{1.747631in}}%
\pgfpathcurveto{\pgfqpoint{0.928165in}{1.741807in}}{\pgfqpoint{0.924893in}{1.733907in}}{\pgfqpoint{0.924893in}{1.725671in}}%
\pgfpathcurveto{\pgfqpoint{0.924893in}{1.717435in}}{\pgfqpoint{0.928165in}{1.709535in}}{\pgfqpoint{0.933989in}{1.703711in}}%
\pgfpathcurveto{\pgfqpoint{0.939813in}{1.697887in}}{\pgfqpoint{0.947713in}{1.694615in}}{\pgfqpoint{0.955949in}{1.694615in}}%
\pgfpathclose%
\pgfusepath{stroke,fill}%
\end{pgfscope}%
\begin{pgfscope}%
\pgfpathrectangle{\pgfqpoint{0.100000in}{0.212622in}}{\pgfqpoint{3.696000in}{3.696000in}}%
\pgfusepath{clip}%
\pgfsetbuttcap%
\pgfsetroundjoin%
\definecolor{currentfill}{rgb}{0.121569,0.466667,0.705882}%
\pgfsetfillcolor{currentfill}%
\pgfsetfillopacity{0.575838}%
\pgfsetlinewidth{1.003750pt}%
\definecolor{currentstroke}{rgb}{0.121569,0.466667,0.705882}%
\pgfsetstrokecolor{currentstroke}%
\pgfsetstrokeopacity{0.575838}%
\pgfsetdash{}{0pt}%
\pgfpathmoveto{\pgfqpoint{0.955074in}{1.694031in}}%
\pgfpathcurveto{\pgfqpoint{0.963310in}{1.694031in}}{\pgfqpoint{0.971210in}{1.697304in}}{\pgfqpoint{0.977034in}{1.703128in}}%
\pgfpathcurveto{\pgfqpoint{0.982858in}{1.708951in}}{\pgfqpoint{0.986130in}{1.716852in}}{\pgfqpoint{0.986130in}{1.725088in}}%
\pgfpathcurveto{\pgfqpoint{0.986130in}{1.733324in}}{\pgfqpoint{0.982858in}{1.741224in}}{\pgfqpoint{0.977034in}{1.747048in}}%
\pgfpathcurveto{\pgfqpoint{0.971210in}{1.752872in}}{\pgfqpoint{0.963310in}{1.756144in}}{\pgfqpoint{0.955074in}{1.756144in}}%
\pgfpathcurveto{\pgfqpoint{0.946838in}{1.756144in}}{\pgfqpoint{0.938938in}{1.752872in}}{\pgfqpoint{0.933114in}{1.747048in}}%
\pgfpathcurveto{\pgfqpoint{0.927290in}{1.741224in}}{\pgfqpoint{0.924017in}{1.733324in}}{\pgfqpoint{0.924017in}{1.725088in}}%
\pgfpathcurveto{\pgfqpoint{0.924017in}{1.716852in}}{\pgfqpoint{0.927290in}{1.708951in}}{\pgfqpoint{0.933114in}{1.703128in}}%
\pgfpathcurveto{\pgfqpoint{0.938938in}{1.697304in}}{\pgfqpoint{0.946838in}{1.694031in}}{\pgfqpoint{0.955074in}{1.694031in}}%
\pgfpathclose%
\pgfusepath{stroke,fill}%
\end{pgfscope}%
\begin{pgfscope}%
\pgfpathrectangle{\pgfqpoint{0.100000in}{0.212622in}}{\pgfqpoint{3.696000in}{3.696000in}}%
\pgfusepath{clip}%
\pgfsetbuttcap%
\pgfsetroundjoin%
\definecolor{currentfill}{rgb}{0.121569,0.466667,0.705882}%
\pgfsetfillcolor{currentfill}%
\pgfsetfillopacity{0.576327}%
\pgfsetlinewidth{1.003750pt}%
\definecolor{currentstroke}{rgb}{0.121569,0.466667,0.705882}%
\pgfsetstrokecolor{currentstroke}%
\pgfsetstrokeopacity{0.576327}%
\pgfsetdash{}{0pt}%
\pgfpathmoveto{\pgfqpoint{0.953853in}{1.692913in}}%
\pgfpathcurveto{\pgfqpoint{0.962089in}{1.692913in}}{\pgfqpoint{0.969989in}{1.696185in}}{\pgfqpoint{0.975813in}{1.702009in}}%
\pgfpathcurveto{\pgfqpoint{0.981637in}{1.707833in}}{\pgfqpoint{0.984909in}{1.715733in}}{\pgfqpoint{0.984909in}{1.723969in}}%
\pgfpathcurveto{\pgfqpoint{0.984909in}{1.732206in}}{\pgfqpoint{0.981637in}{1.740106in}}{\pgfqpoint{0.975813in}{1.745930in}}%
\pgfpathcurveto{\pgfqpoint{0.969989in}{1.751754in}}{\pgfqpoint{0.962089in}{1.755026in}}{\pgfqpoint{0.953853in}{1.755026in}}%
\pgfpathcurveto{\pgfqpoint{0.945616in}{1.755026in}}{\pgfqpoint{0.937716in}{1.751754in}}{\pgfqpoint{0.931892in}{1.745930in}}%
\pgfpathcurveto{\pgfqpoint{0.926068in}{1.740106in}}{\pgfqpoint{0.922796in}{1.732206in}}{\pgfqpoint{0.922796in}{1.723969in}}%
\pgfpathcurveto{\pgfqpoint{0.922796in}{1.715733in}}{\pgfqpoint{0.926068in}{1.707833in}}{\pgfqpoint{0.931892in}{1.702009in}}%
\pgfpathcurveto{\pgfqpoint{0.937716in}{1.696185in}}{\pgfqpoint{0.945616in}{1.692913in}}{\pgfqpoint{0.953853in}{1.692913in}}%
\pgfpathclose%
\pgfusepath{stroke,fill}%
\end{pgfscope}%
\begin{pgfscope}%
\pgfpathrectangle{\pgfqpoint{0.100000in}{0.212622in}}{\pgfqpoint{3.696000in}{3.696000in}}%
\pgfusepath{clip}%
\pgfsetbuttcap%
\pgfsetroundjoin%
\definecolor{currentfill}{rgb}{0.121569,0.466667,0.705882}%
\pgfsetfillcolor{currentfill}%
\pgfsetfillopacity{0.577100}%
\pgfsetlinewidth{1.003750pt}%
\definecolor{currentstroke}{rgb}{0.121569,0.466667,0.705882}%
\pgfsetstrokecolor{currentstroke}%
\pgfsetstrokeopacity{0.577100}%
\pgfsetdash{}{0pt}%
\pgfpathmoveto{\pgfqpoint{0.951067in}{1.690778in}}%
\pgfpathcurveto{\pgfqpoint{0.959303in}{1.690778in}}{\pgfqpoint{0.967203in}{1.694050in}}{\pgfqpoint{0.973027in}{1.699874in}}%
\pgfpathcurveto{\pgfqpoint{0.978851in}{1.705698in}}{\pgfqpoint{0.982123in}{1.713598in}}{\pgfqpoint{0.982123in}{1.721834in}}%
\pgfpathcurveto{\pgfqpoint{0.982123in}{1.730070in}}{\pgfqpoint{0.978851in}{1.737970in}}{\pgfqpoint{0.973027in}{1.743794in}}%
\pgfpathcurveto{\pgfqpoint{0.967203in}{1.749618in}}{\pgfqpoint{0.959303in}{1.752891in}}{\pgfqpoint{0.951067in}{1.752891in}}%
\pgfpathcurveto{\pgfqpoint{0.942831in}{1.752891in}}{\pgfqpoint{0.934930in}{1.749618in}}{\pgfqpoint{0.929107in}{1.743794in}}%
\pgfpathcurveto{\pgfqpoint{0.923283in}{1.737970in}}{\pgfqpoint{0.920010in}{1.730070in}}{\pgfqpoint{0.920010in}{1.721834in}}%
\pgfpathcurveto{\pgfqpoint{0.920010in}{1.713598in}}{\pgfqpoint{0.923283in}{1.705698in}}{\pgfqpoint{0.929107in}{1.699874in}}%
\pgfpathcurveto{\pgfqpoint{0.934930in}{1.694050in}}{\pgfqpoint{0.942831in}{1.690778in}}{\pgfqpoint{0.951067in}{1.690778in}}%
\pgfpathclose%
\pgfusepath{stroke,fill}%
\end{pgfscope}%
\begin{pgfscope}%
\pgfpathrectangle{\pgfqpoint{0.100000in}{0.212622in}}{\pgfqpoint{3.696000in}{3.696000in}}%
\pgfusepath{clip}%
\pgfsetbuttcap%
\pgfsetroundjoin%
\definecolor{currentfill}{rgb}{0.121569,0.466667,0.705882}%
\pgfsetfillcolor{currentfill}%
\pgfsetfillopacity{0.578744}%
\pgfsetlinewidth{1.003750pt}%
\definecolor{currentstroke}{rgb}{0.121569,0.466667,0.705882}%
\pgfsetstrokecolor{currentstroke}%
\pgfsetstrokeopacity{0.578744}%
\pgfsetdash{}{0pt}%
\pgfpathmoveto{\pgfqpoint{0.947105in}{1.687164in}}%
\pgfpathcurveto{\pgfqpoint{0.955341in}{1.687164in}}{\pgfqpoint{0.963241in}{1.690436in}}{\pgfqpoint{0.969065in}{1.696260in}}%
\pgfpathcurveto{\pgfqpoint{0.974889in}{1.702084in}}{\pgfqpoint{0.978161in}{1.709984in}}{\pgfqpoint{0.978161in}{1.718220in}}%
\pgfpathcurveto{\pgfqpoint{0.978161in}{1.726456in}}{\pgfqpoint{0.974889in}{1.734357in}}{\pgfqpoint{0.969065in}{1.740180in}}%
\pgfpathcurveto{\pgfqpoint{0.963241in}{1.746004in}}{\pgfqpoint{0.955341in}{1.749277in}}{\pgfqpoint{0.947105in}{1.749277in}}%
\pgfpathcurveto{\pgfqpoint{0.938868in}{1.749277in}}{\pgfqpoint{0.930968in}{1.746004in}}{\pgfqpoint{0.925144in}{1.740180in}}%
\pgfpathcurveto{\pgfqpoint{0.919320in}{1.734357in}}{\pgfqpoint{0.916048in}{1.726456in}}{\pgfqpoint{0.916048in}{1.718220in}}%
\pgfpathcurveto{\pgfqpoint{0.916048in}{1.709984in}}{\pgfqpoint{0.919320in}{1.702084in}}{\pgfqpoint{0.925144in}{1.696260in}}%
\pgfpathcurveto{\pgfqpoint{0.930968in}{1.690436in}}{\pgfqpoint{0.938868in}{1.687164in}}{\pgfqpoint{0.947105in}{1.687164in}}%
\pgfpathclose%
\pgfusepath{stroke,fill}%
\end{pgfscope}%
\begin{pgfscope}%
\pgfpathrectangle{\pgfqpoint{0.100000in}{0.212622in}}{\pgfqpoint{3.696000in}{3.696000in}}%
\pgfusepath{clip}%
\pgfsetbuttcap%
\pgfsetroundjoin%
\definecolor{currentfill}{rgb}{0.121569,0.466667,0.705882}%
\pgfsetfillcolor{currentfill}%
\pgfsetfillopacity{0.579075}%
\pgfsetlinewidth{1.003750pt}%
\definecolor{currentstroke}{rgb}{0.121569,0.466667,0.705882}%
\pgfsetstrokecolor{currentstroke}%
\pgfsetstrokeopacity{0.579075}%
\pgfsetdash{}{0pt}%
\pgfpathmoveto{\pgfqpoint{0.821273in}{1.605245in}}%
\pgfpathcurveto{\pgfqpoint{0.829510in}{1.605245in}}{\pgfqpoint{0.837410in}{1.608517in}}{\pgfqpoint{0.843234in}{1.614341in}}%
\pgfpathcurveto{\pgfqpoint{0.849058in}{1.620165in}}{\pgfqpoint{0.852330in}{1.628065in}}{\pgfqpoint{0.852330in}{1.636301in}}%
\pgfpathcurveto{\pgfqpoint{0.852330in}{1.644537in}}{\pgfqpoint{0.849058in}{1.652437in}}{\pgfqpoint{0.843234in}{1.658261in}}%
\pgfpathcurveto{\pgfqpoint{0.837410in}{1.664085in}}{\pgfqpoint{0.829510in}{1.667358in}}{\pgfqpoint{0.821273in}{1.667358in}}%
\pgfpathcurveto{\pgfqpoint{0.813037in}{1.667358in}}{\pgfqpoint{0.805137in}{1.664085in}}{\pgfqpoint{0.799313in}{1.658261in}}%
\pgfpathcurveto{\pgfqpoint{0.793489in}{1.652437in}}{\pgfqpoint{0.790217in}{1.644537in}}{\pgfqpoint{0.790217in}{1.636301in}}%
\pgfpathcurveto{\pgfqpoint{0.790217in}{1.628065in}}{\pgfqpoint{0.793489in}{1.620165in}}{\pgfqpoint{0.799313in}{1.614341in}}%
\pgfpathcurveto{\pgfqpoint{0.805137in}{1.608517in}}{\pgfqpoint{0.813037in}{1.605245in}}{\pgfqpoint{0.821273in}{1.605245in}}%
\pgfpathclose%
\pgfusepath{stroke,fill}%
\end{pgfscope}%
\begin{pgfscope}%
\pgfpathrectangle{\pgfqpoint{0.100000in}{0.212622in}}{\pgfqpoint{3.696000in}{3.696000in}}%
\pgfusepath{clip}%
\pgfsetbuttcap%
\pgfsetroundjoin%
\definecolor{currentfill}{rgb}{0.121569,0.466667,0.705882}%
\pgfsetfillcolor{currentfill}%
\pgfsetfillopacity{0.580268}%
\pgfsetlinewidth{1.003750pt}%
\definecolor{currentstroke}{rgb}{0.121569,0.466667,0.705882}%
\pgfsetstrokecolor{currentstroke}%
\pgfsetstrokeopacity{0.580268}%
\pgfsetdash{}{0pt}%
\pgfpathmoveto{\pgfqpoint{1.964038in}{1.941287in}}%
\pgfpathcurveto{\pgfqpoint{1.972275in}{1.941287in}}{\pgfqpoint{1.980175in}{1.944559in}}{\pgfqpoint{1.985999in}{1.950383in}}%
\pgfpathcurveto{\pgfqpoint{1.991823in}{1.956207in}}{\pgfqpoint{1.995095in}{1.964107in}}{\pgfqpoint{1.995095in}{1.972343in}}%
\pgfpathcurveto{\pgfqpoint{1.995095in}{1.980580in}}{\pgfqpoint{1.991823in}{1.988480in}}{\pgfqpoint{1.985999in}{1.994304in}}%
\pgfpathcurveto{\pgfqpoint{1.980175in}{2.000128in}}{\pgfqpoint{1.972275in}{2.003400in}}{\pgfqpoint{1.964038in}{2.003400in}}%
\pgfpathcurveto{\pgfqpoint{1.955802in}{2.003400in}}{\pgfqpoint{1.947902in}{2.000128in}}{\pgfqpoint{1.942078in}{1.994304in}}%
\pgfpathcurveto{\pgfqpoint{1.936254in}{1.988480in}}{\pgfqpoint{1.932982in}{1.980580in}}{\pgfqpoint{1.932982in}{1.972343in}}%
\pgfpathcurveto{\pgfqpoint{1.932982in}{1.964107in}}{\pgfqpoint{1.936254in}{1.956207in}}{\pgfqpoint{1.942078in}{1.950383in}}%
\pgfpathcurveto{\pgfqpoint{1.947902in}{1.944559in}}{\pgfqpoint{1.955802in}{1.941287in}}{\pgfqpoint{1.964038in}{1.941287in}}%
\pgfpathclose%
\pgfusepath{stroke,fill}%
\end{pgfscope}%
\begin{pgfscope}%
\pgfpathrectangle{\pgfqpoint{0.100000in}{0.212622in}}{\pgfqpoint{3.696000in}{3.696000in}}%
\pgfusepath{clip}%
\pgfsetbuttcap%
\pgfsetroundjoin%
\definecolor{currentfill}{rgb}{0.121569,0.466667,0.705882}%
\pgfsetfillcolor{currentfill}%
\pgfsetfillopacity{0.581509}%
\pgfsetlinewidth{1.003750pt}%
\definecolor{currentstroke}{rgb}{0.121569,0.466667,0.705882}%
\pgfsetstrokecolor{currentstroke}%
\pgfsetstrokeopacity{0.581509}%
\pgfsetdash{}{0pt}%
\pgfpathmoveto{\pgfqpoint{0.938053in}{1.681238in}}%
\pgfpathcurveto{\pgfqpoint{0.946289in}{1.681238in}}{\pgfqpoint{0.954189in}{1.684510in}}{\pgfqpoint{0.960013in}{1.690334in}}%
\pgfpathcurveto{\pgfqpoint{0.965837in}{1.696158in}}{\pgfqpoint{0.969109in}{1.704058in}}{\pgfqpoint{0.969109in}{1.712295in}}%
\pgfpathcurveto{\pgfqpoint{0.969109in}{1.720531in}}{\pgfqpoint{0.965837in}{1.728431in}}{\pgfqpoint{0.960013in}{1.734255in}}%
\pgfpathcurveto{\pgfqpoint{0.954189in}{1.740079in}}{\pgfqpoint{0.946289in}{1.743351in}}{\pgfqpoint{0.938053in}{1.743351in}}%
\pgfpathcurveto{\pgfqpoint{0.929817in}{1.743351in}}{\pgfqpoint{0.921917in}{1.740079in}}{\pgfqpoint{0.916093in}{1.734255in}}%
\pgfpathcurveto{\pgfqpoint{0.910269in}{1.728431in}}{\pgfqpoint{0.906996in}{1.720531in}}{\pgfqpoint{0.906996in}{1.712295in}}%
\pgfpathcurveto{\pgfqpoint{0.906996in}{1.704058in}}{\pgfqpoint{0.910269in}{1.696158in}}{\pgfqpoint{0.916093in}{1.690334in}}%
\pgfpathcurveto{\pgfqpoint{0.921917in}{1.684510in}}{\pgfqpoint{0.929817in}{1.681238in}}{\pgfqpoint{0.938053in}{1.681238in}}%
\pgfpathclose%
\pgfusepath{stroke,fill}%
\end{pgfscope}%
\begin{pgfscope}%
\pgfpathrectangle{\pgfqpoint{0.100000in}{0.212622in}}{\pgfqpoint{3.696000in}{3.696000in}}%
\pgfusepath{clip}%
\pgfsetbuttcap%
\pgfsetroundjoin%
\definecolor{currentfill}{rgb}{0.121569,0.466667,0.705882}%
\pgfsetfillcolor{currentfill}%
\pgfsetfillopacity{0.584112}%
\pgfsetlinewidth{1.003750pt}%
\definecolor{currentstroke}{rgb}{0.121569,0.466667,0.705882}%
\pgfsetstrokecolor{currentstroke}%
\pgfsetstrokeopacity{0.584112}%
\pgfsetdash{}{0pt}%
\pgfpathmoveto{\pgfqpoint{0.932325in}{1.674540in}}%
\pgfpathcurveto{\pgfqpoint{0.940561in}{1.674540in}}{\pgfqpoint{0.948461in}{1.677812in}}{\pgfqpoint{0.954285in}{1.683636in}}%
\pgfpathcurveto{\pgfqpoint{0.960109in}{1.689460in}}{\pgfqpoint{0.963381in}{1.697360in}}{\pgfqpoint{0.963381in}{1.705596in}}%
\pgfpathcurveto{\pgfqpoint{0.963381in}{1.713833in}}{\pgfqpoint{0.960109in}{1.721733in}}{\pgfqpoint{0.954285in}{1.727557in}}%
\pgfpathcurveto{\pgfqpoint{0.948461in}{1.733381in}}{\pgfqpoint{0.940561in}{1.736653in}}{\pgfqpoint{0.932325in}{1.736653in}}%
\pgfpathcurveto{\pgfqpoint{0.924088in}{1.736653in}}{\pgfqpoint{0.916188in}{1.733381in}}{\pgfqpoint{0.910364in}{1.727557in}}%
\pgfpathcurveto{\pgfqpoint{0.904540in}{1.721733in}}{\pgfqpoint{0.901268in}{1.713833in}}{\pgfqpoint{0.901268in}{1.705596in}}%
\pgfpathcurveto{\pgfqpoint{0.901268in}{1.697360in}}{\pgfqpoint{0.904540in}{1.689460in}}{\pgfqpoint{0.910364in}{1.683636in}}%
\pgfpathcurveto{\pgfqpoint{0.916188in}{1.677812in}}{\pgfqpoint{0.924088in}{1.674540in}}{\pgfqpoint{0.932325in}{1.674540in}}%
\pgfpathclose%
\pgfusepath{stroke,fill}%
\end{pgfscope}%
\begin{pgfscope}%
\pgfpathrectangle{\pgfqpoint{0.100000in}{0.212622in}}{\pgfqpoint{3.696000in}{3.696000in}}%
\pgfusepath{clip}%
\pgfsetbuttcap%
\pgfsetroundjoin%
\definecolor{currentfill}{rgb}{0.121569,0.466667,0.705882}%
\pgfsetfillcolor{currentfill}%
\pgfsetfillopacity{0.585438}%
\pgfsetlinewidth{1.003750pt}%
\definecolor{currentstroke}{rgb}{0.121569,0.466667,0.705882}%
\pgfsetstrokecolor{currentstroke}%
\pgfsetstrokeopacity{0.585438}%
\pgfsetdash{}{0pt}%
\pgfpathmoveto{\pgfqpoint{0.808427in}{1.602153in}}%
\pgfpathcurveto{\pgfqpoint{0.816663in}{1.602153in}}{\pgfqpoint{0.824563in}{1.605425in}}{\pgfqpoint{0.830387in}{1.611249in}}%
\pgfpathcurveto{\pgfqpoint{0.836211in}{1.617073in}}{\pgfqpoint{0.839483in}{1.624973in}}{\pgfqpoint{0.839483in}{1.633210in}}%
\pgfpathcurveto{\pgfqpoint{0.839483in}{1.641446in}}{\pgfqpoint{0.836211in}{1.649346in}}{\pgfqpoint{0.830387in}{1.655170in}}%
\pgfpathcurveto{\pgfqpoint{0.824563in}{1.660994in}}{\pgfqpoint{0.816663in}{1.664266in}}{\pgfqpoint{0.808427in}{1.664266in}}%
\pgfpathcurveto{\pgfqpoint{0.800190in}{1.664266in}}{\pgfqpoint{0.792290in}{1.660994in}}{\pgfqpoint{0.786466in}{1.655170in}}%
\pgfpathcurveto{\pgfqpoint{0.780642in}{1.649346in}}{\pgfqpoint{0.777370in}{1.641446in}}{\pgfqpoint{0.777370in}{1.633210in}}%
\pgfpathcurveto{\pgfqpoint{0.777370in}{1.624973in}}{\pgfqpoint{0.780642in}{1.617073in}}{\pgfqpoint{0.786466in}{1.611249in}}%
\pgfpathcurveto{\pgfqpoint{0.792290in}{1.605425in}}{\pgfqpoint{0.800190in}{1.602153in}}{\pgfqpoint{0.808427in}{1.602153in}}%
\pgfpathclose%
\pgfusepath{stroke,fill}%
\end{pgfscope}%
\begin{pgfscope}%
\pgfpathrectangle{\pgfqpoint{0.100000in}{0.212622in}}{\pgfqpoint{3.696000in}{3.696000in}}%
\pgfusepath{clip}%
\pgfsetbuttcap%
\pgfsetroundjoin%
\definecolor{currentfill}{rgb}{0.121569,0.466667,0.705882}%
\pgfsetfillcolor{currentfill}%
\pgfsetfillopacity{0.586150}%
\pgfsetlinewidth{1.003750pt}%
\definecolor{currentstroke}{rgb}{0.121569,0.466667,0.705882}%
\pgfsetstrokecolor{currentstroke}%
\pgfsetstrokeopacity{0.586150}%
\pgfsetdash{}{0pt}%
\pgfpathmoveto{\pgfqpoint{0.925103in}{1.670837in}}%
\pgfpathcurveto{\pgfqpoint{0.933339in}{1.670837in}}{\pgfqpoint{0.941239in}{1.674110in}}{\pgfqpoint{0.947063in}{1.679934in}}%
\pgfpathcurveto{\pgfqpoint{0.952887in}{1.685758in}}{\pgfqpoint{0.956159in}{1.693658in}}{\pgfqpoint{0.956159in}{1.701894in}}%
\pgfpathcurveto{\pgfqpoint{0.956159in}{1.710130in}}{\pgfqpoint{0.952887in}{1.718030in}}{\pgfqpoint{0.947063in}{1.723854in}}%
\pgfpathcurveto{\pgfqpoint{0.941239in}{1.729678in}}{\pgfqpoint{0.933339in}{1.732950in}}{\pgfqpoint{0.925103in}{1.732950in}}%
\pgfpathcurveto{\pgfqpoint{0.916866in}{1.732950in}}{\pgfqpoint{0.908966in}{1.729678in}}{\pgfqpoint{0.903142in}{1.723854in}}%
\pgfpathcurveto{\pgfqpoint{0.897318in}{1.718030in}}{\pgfqpoint{0.894046in}{1.710130in}}{\pgfqpoint{0.894046in}{1.701894in}}%
\pgfpathcurveto{\pgfqpoint{0.894046in}{1.693658in}}{\pgfqpoint{0.897318in}{1.685758in}}{\pgfqpoint{0.903142in}{1.679934in}}%
\pgfpathcurveto{\pgfqpoint{0.908966in}{1.674110in}}{\pgfqpoint{0.916866in}{1.670837in}}{\pgfqpoint{0.925103in}{1.670837in}}%
\pgfpathclose%
\pgfusepath{stroke,fill}%
\end{pgfscope}%
\begin{pgfscope}%
\pgfpathrectangle{\pgfqpoint{0.100000in}{0.212622in}}{\pgfqpoint{3.696000in}{3.696000in}}%
\pgfusepath{clip}%
\pgfsetbuttcap%
\pgfsetroundjoin%
\definecolor{currentfill}{rgb}{0.121569,0.466667,0.705882}%
\pgfsetfillcolor{currentfill}%
\pgfsetfillopacity{0.588943}%
\pgfsetlinewidth{1.003750pt}%
\definecolor{currentstroke}{rgb}{0.121569,0.466667,0.705882}%
\pgfsetstrokecolor{currentstroke}%
\pgfsetstrokeopacity{0.588943}%
\pgfsetdash{}{0pt}%
\pgfpathmoveto{\pgfqpoint{0.801096in}{1.600716in}}%
\pgfpathcurveto{\pgfqpoint{0.809332in}{1.600716in}}{\pgfqpoint{0.817232in}{1.603988in}}{\pgfqpoint{0.823056in}{1.609812in}}%
\pgfpathcurveto{\pgfqpoint{0.828880in}{1.615636in}}{\pgfqpoint{0.832152in}{1.623536in}}{\pgfqpoint{0.832152in}{1.631773in}}%
\pgfpathcurveto{\pgfqpoint{0.832152in}{1.640009in}}{\pgfqpoint{0.828880in}{1.647909in}}{\pgfqpoint{0.823056in}{1.653733in}}%
\pgfpathcurveto{\pgfqpoint{0.817232in}{1.659557in}}{\pgfqpoint{0.809332in}{1.662829in}}{\pgfqpoint{0.801096in}{1.662829in}}%
\pgfpathcurveto{\pgfqpoint{0.792859in}{1.662829in}}{\pgfqpoint{0.784959in}{1.659557in}}{\pgfqpoint{0.779135in}{1.653733in}}%
\pgfpathcurveto{\pgfqpoint{0.773311in}{1.647909in}}{\pgfqpoint{0.770039in}{1.640009in}}{\pgfqpoint{0.770039in}{1.631773in}}%
\pgfpathcurveto{\pgfqpoint{0.770039in}{1.623536in}}{\pgfqpoint{0.773311in}{1.615636in}}{\pgfqpoint{0.779135in}{1.609812in}}%
\pgfpathcurveto{\pgfqpoint{0.784959in}{1.603988in}}{\pgfqpoint{0.792859in}{1.600716in}}{\pgfqpoint{0.801096in}{1.600716in}}%
\pgfpathclose%
\pgfusepath{stroke,fill}%
\end{pgfscope}%
\begin{pgfscope}%
\pgfpathrectangle{\pgfqpoint{0.100000in}{0.212622in}}{\pgfqpoint{3.696000in}{3.696000in}}%
\pgfusepath{clip}%
\pgfsetbuttcap%
\pgfsetroundjoin%
\definecolor{currentfill}{rgb}{0.121569,0.466667,0.705882}%
\pgfsetfillcolor{currentfill}%
\pgfsetfillopacity{0.589854}%
\pgfsetlinewidth{1.003750pt}%
\definecolor{currentstroke}{rgb}{0.121569,0.466667,0.705882}%
\pgfsetstrokecolor{currentstroke}%
\pgfsetstrokeopacity{0.589854}%
\pgfsetdash{}{0pt}%
\pgfpathmoveto{\pgfqpoint{0.914992in}{1.660382in}}%
\pgfpathcurveto{\pgfqpoint{0.923228in}{1.660382in}}{\pgfqpoint{0.931128in}{1.663655in}}{\pgfqpoint{0.936952in}{1.669479in}}%
\pgfpathcurveto{\pgfqpoint{0.942776in}{1.675302in}}{\pgfqpoint{0.946049in}{1.683203in}}{\pgfqpoint{0.946049in}{1.691439in}}%
\pgfpathcurveto{\pgfqpoint{0.946049in}{1.699675in}}{\pgfqpoint{0.942776in}{1.707575in}}{\pgfqpoint{0.936952in}{1.713399in}}%
\pgfpathcurveto{\pgfqpoint{0.931128in}{1.719223in}}{\pgfqpoint{0.923228in}{1.722495in}}{\pgfqpoint{0.914992in}{1.722495in}}%
\pgfpathcurveto{\pgfqpoint{0.906756in}{1.722495in}}{\pgfqpoint{0.898856in}{1.719223in}}{\pgfqpoint{0.893032in}{1.713399in}}%
\pgfpathcurveto{\pgfqpoint{0.887208in}{1.707575in}}{\pgfqpoint{0.883936in}{1.699675in}}{\pgfqpoint{0.883936in}{1.691439in}}%
\pgfpathcurveto{\pgfqpoint{0.883936in}{1.683203in}}{\pgfqpoint{0.887208in}{1.675302in}}{\pgfqpoint{0.893032in}{1.669479in}}%
\pgfpathcurveto{\pgfqpoint{0.898856in}{1.663655in}}{\pgfqpoint{0.906756in}{1.660382in}}{\pgfqpoint{0.914992in}{1.660382in}}%
\pgfpathclose%
\pgfusepath{stroke,fill}%
\end{pgfscope}%
\begin{pgfscope}%
\pgfpathrectangle{\pgfqpoint{0.100000in}{0.212622in}}{\pgfqpoint{3.696000in}{3.696000in}}%
\pgfusepath{clip}%
\pgfsetbuttcap%
\pgfsetroundjoin%
\definecolor{currentfill}{rgb}{0.121569,0.466667,0.705882}%
\pgfsetfillcolor{currentfill}%
\pgfsetfillopacity{0.590734}%
\pgfsetlinewidth{1.003750pt}%
\definecolor{currentstroke}{rgb}{0.121569,0.466667,0.705882}%
\pgfsetstrokecolor{currentstroke}%
\pgfsetstrokeopacity{0.590734}%
\pgfsetdash{}{0pt}%
\pgfpathmoveto{\pgfqpoint{0.796287in}{1.599932in}}%
\pgfpathcurveto{\pgfqpoint{0.804524in}{1.599932in}}{\pgfqpoint{0.812424in}{1.603204in}}{\pgfqpoint{0.818248in}{1.609028in}}%
\pgfpathcurveto{\pgfqpoint{0.824072in}{1.614852in}}{\pgfqpoint{0.827344in}{1.622752in}}{\pgfqpoint{0.827344in}{1.630988in}}%
\pgfpathcurveto{\pgfqpoint{0.827344in}{1.639225in}}{\pgfqpoint{0.824072in}{1.647125in}}{\pgfqpoint{0.818248in}{1.652949in}}%
\pgfpathcurveto{\pgfqpoint{0.812424in}{1.658773in}}{\pgfqpoint{0.804524in}{1.662045in}}{\pgfqpoint{0.796287in}{1.662045in}}%
\pgfpathcurveto{\pgfqpoint{0.788051in}{1.662045in}}{\pgfqpoint{0.780151in}{1.658773in}}{\pgfqpoint{0.774327in}{1.652949in}}%
\pgfpathcurveto{\pgfqpoint{0.768503in}{1.647125in}}{\pgfqpoint{0.765231in}{1.639225in}}{\pgfqpoint{0.765231in}{1.630988in}}%
\pgfpathcurveto{\pgfqpoint{0.765231in}{1.622752in}}{\pgfqpoint{0.768503in}{1.614852in}}{\pgfqpoint{0.774327in}{1.609028in}}%
\pgfpathcurveto{\pgfqpoint{0.780151in}{1.603204in}}{\pgfqpoint{0.788051in}{1.599932in}}{\pgfqpoint{0.796287in}{1.599932in}}%
\pgfpathclose%
\pgfusepath{stroke,fill}%
\end{pgfscope}%
\begin{pgfscope}%
\pgfpathrectangle{\pgfqpoint{0.100000in}{0.212622in}}{\pgfqpoint{3.696000in}{3.696000in}}%
\pgfusepath{clip}%
\pgfsetbuttcap%
\pgfsetroundjoin%
\definecolor{currentfill}{rgb}{0.121569,0.466667,0.705882}%
\pgfsetfillcolor{currentfill}%
\pgfsetfillopacity{0.591912}%
\pgfsetlinewidth{1.003750pt}%
\definecolor{currentstroke}{rgb}{0.121569,0.466667,0.705882}%
\pgfsetstrokecolor{currentstroke}%
\pgfsetstrokeopacity{0.591912}%
\pgfsetdash{}{0pt}%
\pgfpathmoveto{\pgfqpoint{0.794315in}{1.599985in}}%
\pgfpathcurveto{\pgfqpoint{0.802551in}{1.599985in}}{\pgfqpoint{0.810451in}{1.603257in}}{\pgfqpoint{0.816275in}{1.609081in}}%
\pgfpathcurveto{\pgfqpoint{0.822099in}{1.614905in}}{\pgfqpoint{0.825371in}{1.622805in}}{\pgfqpoint{0.825371in}{1.631041in}}%
\pgfpathcurveto{\pgfqpoint{0.825371in}{1.639277in}}{\pgfqpoint{0.822099in}{1.647177in}}{\pgfqpoint{0.816275in}{1.653001in}}%
\pgfpathcurveto{\pgfqpoint{0.810451in}{1.658825in}}{\pgfqpoint{0.802551in}{1.662098in}}{\pgfqpoint{0.794315in}{1.662098in}}%
\pgfpathcurveto{\pgfqpoint{0.786078in}{1.662098in}}{\pgfqpoint{0.778178in}{1.658825in}}{\pgfqpoint{0.772354in}{1.653001in}}%
\pgfpathcurveto{\pgfqpoint{0.766530in}{1.647177in}}{\pgfqpoint{0.763258in}{1.639277in}}{\pgfqpoint{0.763258in}{1.631041in}}%
\pgfpathcurveto{\pgfqpoint{0.763258in}{1.622805in}}{\pgfqpoint{0.766530in}{1.614905in}}{\pgfqpoint{0.772354in}{1.609081in}}%
\pgfpathcurveto{\pgfqpoint{0.778178in}{1.603257in}}{\pgfqpoint{0.786078in}{1.599985in}}{\pgfqpoint{0.794315in}{1.599985in}}%
\pgfpathclose%
\pgfusepath{stroke,fill}%
\end{pgfscope}%
\begin{pgfscope}%
\pgfpathrectangle{\pgfqpoint{0.100000in}{0.212622in}}{\pgfqpoint{3.696000in}{3.696000in}}%
\pgfusepath{clip}%
\pgfsetbuttcap%
\pgfsetroundjoin%
\definecolor{currentfill}{rgb}{0.121569,0.466667,0.705882}%
\pgfsetfillcolor{currentfill}%
\pgfsetfillopacity{0.592256}%
\pgfsetlinewidth{1.003750pt}%
\definecolor{currentstroke}{rgb}{0.121569,0.466667,0.705882}%
\pgfsetstrokecolor{currentstroke}%
\pgfsetstrokeopacity{0.592256}%
\pgfsetdash{}{0pt}%
\pgfpathmoveto{\pgfqpoint{0.902736in}{1.655374in}}%
\pgfpathcurveto{\pgfqpoint{0.910972in}{1.655374in}}{\pgfqpoint{0.918873in}{1.658646in}}{\pgfqpoint{0.924696in}{1.664470in}}%
\pgfpathcurveto{\pgfqpoint{0.930520in}{1.670294in}}{\pgfqpoint{0.933793in}{1.678194in}}{\pgfqpoint{0.933793in}{1.686431in}}%
\pgfpathcurveto{\pgfqpoint{0.933793in}{1.694667in}}{\pgfqpoint{0.930520in}{1.702567in}}{\pgfqpoint{0.924696in}{1.708391in}}%
\pgfpathcurveto{\pgfqpoint{0.918873in}{1.714215in}}{\pgfqpoint{0.910972in}{1.717487in}}{\pgfqpoint{0.902736in}{1.717487in}}%
\pgfpathcurveto{\pgfqpoint{0.894500in}{1.717487in}}{\pgfqpoint{0.886600in}{1.714215in}}{\pgfqpoint{0.880776in}{1.708391in}}%
\pgfpathcurveto{\pgfqpoint{0.874952in}{1.702567in}}{\pgfqpoint{0.871680in}{1.694667in}}{\pgfqpoint{0.871680in}{1.686431in}}%
\pgfpathcurveto{\pgfqpoint{0.871680in}{1.678194in}}{\pgfqpoint{0.874952in}{1.670294in}}{\pgfqpoint{0.880776in}{1.664470in}}%
\pgfpathcurveto{\pgfqpoint{0.886600in}{1.658646in}}{\pgfqpoint{0.894500in}{1.655374in}}{\pgfqpoint{0.902736in}{1.655374in}}%
\pgfpathclose%
\pgfusepath{stroke,fill}%
\end{pgfscope}%
\begin{pgfscope}%
\pgfpathrectangle{\pgfqpoint{0.100000in}{0.212622in}}{\pgfqpoint{3.696000in}{3.696000in}}%
\pgfusepath{clip}%
\pgfsetbuttcap%
\pgfsetroundjoin%
\definecolor{currentfill}{rgb}{0.121569,0.466667,0.705882}%
\pgfsetfillcolor{currentfill}%
\pgfsetfillopacity{0.592455}%
\pgfsetlinewidth{1.003750pt}%
\definecolor{currentstroke}{rgb}{0.121569,0.466667,0.705882}%
\pgfsetstrokecolor{currentstroke}%
\pgfsetstrokeopacity{0.592455}%
\pgfsetdash{}{0pt}%
\pgfpathmoveto{\pgfqpoint{1.987576in}{1.947368in}}%
\pgfpathcurveto{\pgfqpoint{1.995812in}{1.947368in}}{\pgfqpoint{2.003712in}{1.950640in}}{\pgfqpoint{2.009536in}{1.956464in}}%
\pgfpathcurveto{\pgfqpoint{2.015360in}{1.962288in}}{\pgfqpoint{2.018632in}{1.970188in}}{\pgfqpoint{2.018632in}{1.978424in}}%
\pgfpathcurveto{\pgfqpoint{2.018632in}{1.986661in}}{\pgfqpoint{2.015360in}{1.994561in}}{\pgfqpoint{2.009536in}{2.000385in}}%
\pgfpathcurveto{\pgfqpoint{2.003712in}{2.006209in}}{\pgfqpoint{1.995812in}{2.009481in}}{\pgfqpoint{1.987576in}{2.009481in}}%
\pgfpathcurveto{\pgfqpoint{1.979340in}{2.009481in}}{\pgfqpoint{1.971440in}{2.006209in}}{\pgfqpoint{1.965616in}{2.000385in}}%
\pgfpathcurveto{\pgfqpoint{1.959792in}{1.994561in}}{\pgfqpoint{1.956519in}{1.986661in}}{\pgfqpoint{1.956519in}{1.978424in}}%
\pgfpathcurveto{\pgfqpoint{1.956519in}{1.970188in}}{\pgfqpoint{1.959792in}{1.962288in}}{\pgfqpoint{1.965616in}{1.956464in}}%
\pgfpathcurveto{\pgfqpoint{1.971440in}{1.950640in}}{\pgfqpoint{1.979340in}{1.947368in}}{\pgfqpoint{1.987576in}{1.947368in}}%
\pgfpathclose%
\pgfusepath{stroke,fill}%
\end{pgfscope}%
\begin{pgfscope}%
\pgfpathrectangle{\pgfqpoint{0.100000in}{0.212622in}}{\pgfqpoint{3.696000in}{3.696000in}}%
\pgfusepath{clip}%
\pgfsetbuttcap%
\pgfsetroundjoin%
\definecolor{currentfill}{rgb}{0.121569,0.466667,0.705882}%
\pgfsetfillcolor{currentfill}%
\pgfsetfillopacity{0.592458}%
\pgfsetlinewidth{1.003750pt}%
\definecolor{currentstroke}{rgb}{0.121569,0.466667,0.705882}%
\pgfsetstrokecolor{currentstroke}%
\pgfsetstrokeopacity{0.592458}%
\pgfsetdash{}{0pt}%
\pgfpathmoveto{\pgfqpoint{0.793032in}{1.599567in}}%
\pgfpathcurveto{\pgfqpoint{0.801268in}{1.599567in}}{\pgfqpoint{0.809168in}{1.602839in}}{\pgfqpoint{0.814992in}{1.608663in}}%
\pgfpathcurveto{\pgfqpoint{0.820816in}{1.614487in}}{\pgfqpoint{0.824088in}{1.622387in}}{\pgfqpoint{0.824088in}{1.630623in}}%
\pgfpathcurveto{\pgfqpoint{0.824088in}{1.638860in}}{\pgfqpoint{0.820816in}{1.646760in}}{\pgfqpoint{0.814992in}{1.652584in}}%
\pgfpathcurveto{\pgfqpoint{0.809168in}{1.658408in}}{\pgfqpoint{0.801268in}{1.661680in}}{\pgfqpoint{0.793032in}{1.661680in}}%
\pgfpathcurveto{\pgfqpoint{0.784796in}{1.661680in}}{\pgfqpoint{0.776896in}{1.658408in}}{\pgfqpoint{0.771072in}{1.652584in}}%
\pgfpathcurveto{\pgfqpoint{0.765248in}{1.646760in}}{\pgfqpoint{0.761975in}{1.638860in}}{\pgfqpoint{0.761975in}{1.630623in}}%
\pgfpathcurveto{\pgfqpoint{0.761975in}{1.622387in}}{\pgfqpoint{0.765248in}{1.614487in}}{\pgfqpoint{0.771072in}{1.608663in}}%
\pgfpathcurveto{\pgfqpoint{0.776896in}{1.602839in}}{\pgfqpoint{0.784796in}{1.599567in}}{\pgfqpoint{0.793032in}{1.599567in}}%
\pgfpathclose%
\pgfusepath{stroke,fill}%
\end{pgfscope}%
\begin{pgfscope}%
\pgfpathrectangle{\pgfqpoint{0.100000in}{0.212622in}}{\pgfqpoint{3.696000in}{3.696000in}}%
\pgfusepath{clip}%
\pgfsetbuttcap%
\pgfsetroundjoin%
\definecolor{currentfill}{rgb}{0.121569,0.466667,0.705882}%
\pgfsetfillcolor{currentfill}%
\pgfsetfillopacity{0.592782}%
\pgfsetlinewidth{1.003750pt}%
\definecolor{currentstroke}{rgb}{0.121569,0.466667,0.705882}%
\pgfsetstrokecolor{currentstroke}%
\pgfsetstrokeopacity{0.592782}%
\pgfsetdash{}{0pt}%
\pgfpathmoveto{\pgfqpoint{0.792352in}{1.599455in}}%
\pgfpathcurveto{\pgfqpoint{0.800589in}{1.599455in}}{\pgfqpoint{0.808489in}{1.602727in}}{\pgfqpoint{0.814313in}{1.608551in}}%
\pgfpathcurveto{\pgfqpoint{0.820136in}{1.614375in}}{\pgfqpoint{0.823409in}{1.622275in}}{\pgfqpoint{0.823409in}{1.630511in}}%
\pgfpathcurveto{\pgfqpoint{0.823409in}{1.638747in}}{\pgfqpoint{0.820136in}{1.646647in}}{\pgfqpoint{0.814313in}{1.652471in}}%
\pgfpathcurveto{\pgfqpoint{0.808489in}{1.658295in}}{\pgfqpoint{0.800589in}{1.661568in}}{\pgfqpoint{0.792352in}{1.661568in}}%
\pgfpathcurveto{\pgfqpoint{0.784116in}{1.661568in}}{\pgfqpoint{0.776216in}{1.658295in}}{\pgfqpoint{0.770392in}{1.652471in}}%
\pgfpathcurveto{\pgfqpoint{0.764568in}{1.646647in}}{\pgfqpoint{0.761296in}{1.638747in}}{\pgfqpoint{0.761296in}{1.630511in}}%
\pgfpathcurveto{\pgfqpoint{0.761296in}{1.622275in}}{\pgfqpoint{0.764568in}{1.614375in}}{\pgfqpoint{0.770392in}{1.608551in}}%
\pgfpathcurveto{\pgfqpoint{0.776216in}{1.602727in}}{\pgfqpoint{0.784116in}{1.599455in}}{\pgfqpoint{0.792352in}{1.599455in}}%
\pgfpathclose%
\pgfusepath{stroke,fill}%
\end{pgfscope}%
\begin{pgfscope}%
\pgfpathrectangle{\pgfqpoint{0.100000in}{0.212622in}}{\pgfqpoint{3.696000in}{3.696000in}}%
\pgfusepath{clip}%
\pgfsetbuttcap%
\pgfsetroundjoin%
\definecolor{currentfill}{rgb}{0.121569,0.466667,0.705882}%
\pgfsetfillcolor{currentfill}%
\pgfsetfillopacity{0.592967}%
\pgfsetlinewidth{1.003750pt}%
\definecolor{currentstroke}{rgb}{0.121569,0.466667,0.705882}%
\pgfsetstrokecolor{currentstroke}%
\pgfsetstrokeopacity{0.592967}%
\pgfsetdash{}{0pt}%
\pgfpathmoveto{\pgfqpoint{0.792078in}{1.599342in}}%
\pgfpathcurveto{\pgfqpoint{0.800314in}{1.599342in}}{\pgfqpoint{0.808214in}{1.602614in}}{\pgfqpoint{0.814038in}{1.608438in}}%
\pgfpathcurveto{\pgfqpoint{0.819862in}{1.614262in}}{\pgfqpoint{0.823134in}{1.622162in}}{\pgfqpoint{0.823134in}{1.630398in}}%
\pgfpathcurveto{\pgfqpoint{0.823134in}{1.638635in}}{\pgfqpoint{0.819862in}{1.646535in}}{\pgfqpoint{0.814038in}{1.652359in}}%
\pgfpathcurveto{\pgfqpoint{0.808214in}{1.658183in}}{\pgfqpoint{0.800314in}{1.661455in}}{\pgfqpoint{0.792078in}{1.661455in}}%
\pgfpathcurveto{\pgfqpoint{0.783842in}{1.661455in}}{\pgfqpoint{0.775942in}{1.658183in}}{\pgfqpoint{0.770118in}{1.652359in}}%
\pgfpathcurveto{\pgfqpoint{0.764294in}{1.646535in}}{\pgfqpoint{0.761021in}{1.638635in}}{\pgfqpoint{0.761021in}{1.630398in}}%
\pgfpathcurveto{\pgfqpoint{0.761021in}{1.622162in}}{\pgfqpoint{0.764294in}{1.614262in}}{\pgfqpoint{0.770118in}{1.608438in}}%
\pgfpathcurveto{\pgfqpoint{0.775942in}{1.602614in}}{\pgfqpoint{0.783842in}{1.599342in}}{\pgfqpoint{0.792078in}{1.599342in}}%
\pgfpathclose%
\pgfusepath{stroke,fill}%
\end{pgfscope}%
\begin{pgfscope}%
\pgfpathrectangle{\pgfqpoint{0.100000in}{0.212622in}}{\pgfqpoint{3.696000in}{3.696000in}}%
\pgfusepath{clip}%
\pgfsetbuttcap%
\pgfsetroundjoin%
\definecolor{currentfill}{rgb}{0.121569,0.466667,0.705882}%
\pgfsetfillcolor{currentfill}%
\pgfsetfillopacity{0.593061}%
\pgfsetlinewidth{1.003750pt}%
\definecolor{currentstroke}{rgb}{0.121569,0.466667,0.705882}%
\pgfsetstrokecolor{currentstroke}%
\pgfsetstrokeopacity{0.593061}%
\pgfsetdash{}{0pt}%
\pgfpathmoveto{\pgfqpoint{0.791872in}{1.599284in}}%
\pgfpathcurveto{\pgfqpoint{0.800108in}{1.599284in}}{\pgfqpoint{0.808008in}{1.602556in}}{\pgfqpoint{0.813832in}{1.608380in}}%
\pgfpathcurveto{\pgfqpoint{0.819656in}{1.614204in}}{\pgfqpoint{0.822928in}{1.622104in}}{\pgfqpoint{0.822928in}{1.630340in}}%
\pgfpathcurveto{\pgfqpoint{0.822928in}{1.638577in}}{\pgfqpoint{0.819656in}{1.646477in}}{\pgfqpoint{0.813832in}{1.652301in}}%
\pgfpathcurveto{\pgfqpoint{0.808008in}{1.658125in}}{\pgfqpoint{0.800108in}{1.661397in}}{\pgfqpoint{0.791872in}{1.661397in}}%
\pgfpathcurveto{\pgfqpoint{0.783635in}{1.661397in}}{\pgfqpoint{0.775735in}{1.658125in}}{\pgfqpoint{0.769912in}{1.652301in}}%
\pgfpathcurveto{\pgfqpoint{0.764088in}{1.646477in}}{\pgfqpoint{0.760815in}{1.638577in}}{\pgfqpoint{0.760815in}{1.630340in}}%
\pgfpathcurveto{\pgfqpoint{0.760815in}{1.622104in}}{\pgfqpoint{0.764088in}{1.614204in}}{\pgfqpoint{0.769912in}{1.608380in}}%
\pgfpathcurveto{\pgfqpoint{0.775735in}{1.602556in}}{\pgfqpoint{0.783635in}{1.599284in}}{\pgfqpoint{0.791872in}{1.599284in}}%
\pgfpathclose%
\pgfusepath{stroke,fill}%
\end{pgfscope}%
\begin{pgfscope}%
\pgfpathrectangle{\pgfqpoint{0.100000in}{0.212622in}}{\pgfqpoint{3.696000in}{3.696000in}}%
\pgfusepath{clip}%
\pgfsetbuttcap%
\pgfsetroundjoin%
\definecolor{currentfill}{rgb}{0.121569,0.466667,0.705882}%
\pgfsetfillcolor{currentfill}%
\pgfsetfillopacity{0.593115}%
\pgfsetlinewidth{1.003750pt}%
\definecolor{currentstroke}{rgb}{0.121569,0.466667,0.705882}%
\pgfsetstrokecolor{currentstroke}%
\pgfsetstrokeopacity{0.593115}%
\pgfsetdash{}{0pt}%
\pgfpathmoveto{\pgfqpoint{0.791755in}{1.599273in}}%
\pgfpathcurveto{\pgfqpoint{0.799991in}{1.599273in}}{\pgfqpoint{0.807891in}{1.602545in}}{\pgfqpoint{0.813715in}{1.608369in}}%
\pgfpathcurveto{\pgfqpoint{0.819539in}{1.614193in}}{\pgfqpoint{0.822811in}{1.622093in}}{\pgfqpoint{0.822811in}{1.630329in}}%
\pgfpathcurveto{\pgfqpoint{0.822811in}{1.638566in}}{\pgfqpoint{0.819539in}{1.646466in}}{\pgfqpoint{0.813715in}{1.652290in}}%
\pgfpathcurveto{\pgfqpoint{0.807891in}{1.658114in}}{\pgfqpoint{0.799991in}{1.661386in}}{\pgfqpoint{0.791755in}{1.661386in}}%
\pgfpathcurveto{\pgfqpoint{0.783518in}{1.661386in}}{\pgfqpoint{0.775618in}{1.658114in}}{\pgfqpoint{0.769794in}{1.652290in}}%
\pgfpathcurveto{\pgfqpoint{0.763971in}{1.646466in}}{\pgfqpoint{0.760698in}{1.638566in}}{\pgfqpoint{0.760698in}{1.630329in}}%
\pgfpathcurveto{\pgfqpoint{0.760698in}{1.622093in}}{\pgfqpoint{0.763971in}{1.614193in}}{\pgfqpoint{0.769794in}{1.608369in}}%
\pgfpathcurveto{\pgfqpoint{0.775618in}{1.602545in}}{\pgfqpoint{0.783518in}{1.599273in}}{\pgfqpoint{0.791755in}{1.599273in}}%
\pgfpathclose%
\pgfusepath{stroke,fill}%
\end{pgfscope}%
\begin{pgfscope}%
\pgfpathrectangle{\pgfqpoint{0.100000in}{0.212622in}}{\pgfqpoint{3.696000in}{3.696000in}}%
\pgfusepath{clip}%
\pgfsetbuttcap%
\pgfsetroundjoin%
\definecolor{currentfill}{rgb}{0.121569,0.466667,0.705882}%
\pgfsetfillcolor{currentfill}%
\pgfsetfillopacity{0.593143}%
\pgfsetlinewidth{1.003750pt}%
\definecolor{currentstroke}{rgb}{0.121569,0.466667,0.705882}%
\pgfsetstrokecolor{currentstroke}%
\pgfsetstrokeopacity{0.593143}%
\pgfsetdash{}{0pt}%
\pgfpathmoveto{\pgfqpoint{0.791691in}{1.599253in}}%
\pgfpathcurveto{\pgfqpoint{0.799927in}{1.599253in}}{\pgfqpoint{0.807827in}{1.602525in}}{\pgfqpoint{0.813651in}{1.608349in}}%
\pgfpathcurveto{\pgfqpoint{0.819475in}{1.614173in}}{\pgfqpoint{0.822748in}{1.622073in}}{\pgfqpoint{0.822748in}{1.630309in}}%
\pgfpathcurveto{\pgfqpoint{0.822748in}{1.638546in}}{\pgfqpoint{0.819475in}{1.646446in}}{\pgfqpoint{0.813651in}{1.652270in}}%
\pgfpathcurveto{\pgfqpoint{0.807827in}{1.658094in}}{\pgfqpoint{0.799927in}{1.661366in}}{\pgfqpoint{0.791691in}{1.661366in}}%
\pgfpathcurveto{\pgfqpoint{0.783455in}{1.661366in}}{\pgfqpoint{0.775555in}{1.658094in}}{\pgfqpoint{0.769731in}{1.652270in}}%
\pgfpathcurveto{\pgfqpoint{0.763907in}{1.646446in}}{\pgfqpoint{0.760635in}{1.638546in}}{\pgfqpoint{0.760635in}{1.630309in}}%
\pgfpathcurveto{\pgfqpoint{0.760635in}{1.622073in}}{\pgfqpoint{0.763907in}{1.614173in}}{\pgfqpoint{0.769731in}{1.608349in}}%
\pgfpathcurveto{\pgfqpoint{0.775555in}{1.602525in}}{\pgfqpoint{0.783455in}{1.599253in}}{\pgfqpoint{0.791691in}{1.599253in}}%
\pgfpathclose%
\pgfusepath{stroke,fill}%
\end{pgfscope}%
\begin{pgfscope}%
\pgfpathrectangle{\pgfqpoint{0.100000in}{0.212622in}}{\pgfqpoint{3.696000in}{3.696000in}}%
\pgfusepath{clip}%
\pgfsetbuttcap%
\pgfsetroundjoin%
\definecolor{currentfill}{rgb}{0.121569,0.466667,0.705882}%
\pgfsetfillcolor{currentfill}%
\pgfsetfillopacity{0.593160}%
\pgfsetlinewidth{1.003750pt}%
\definecolor{currentstroke}{rgb}{0.121569,0.466667,0.705882}%
\pgfsetstrokecolor{currentstroke}%
\pgfsetstrokeopacity{0.593160}%
\pgfsetdash{}{0pt}%
\pgfpathmoveto{\pgfqpoint{0.791661in}{1.599245in}}%
\pgfpathcurveto{\pgfqpoint{0.799898in}{1.599245in}}{\pgfqpoint{0.807798in}{1.602517in}}{\pgfqpoint{0.813622in}{1.608341in}}%
\pgfpathcurveto{\pgfqpoint{0.819446in}{1.614165in}}{\pgfqpoint{0.822718in}{1.622065in}}{\pgfqpoint{0.822718in}{1.630301in}}%
\pgfpathcurveto{\pgfqpoint{0.822718in}{1.638537in}}{\pgfqpoint{0.819446in}{1.646437in}}{\pgfqpoint{0.813622in}{1.652261in}}%
\pgfpathcurveto{\pgfqpoint{0.807798in}{1.658085in}}{\pgfqpoint{0.799898in}{1.661358in}}{\pgfqpoint{0.791661in}{1.661358in}}%
\pgfpathcurveto{\pgfqpoint{0.783425in}{1.661358in}}{\pgfqpoint{0.775525in}{1.658085in}}{\pgfqpoint{0.769701in}{1.652261in}}%
\pgfpathcurveto{\pgfqpoint{0.763877in}{1.646437in}}{\pgfqpoint{0.760605in}{1.638537in}}{\pgfqpoint{0.760605in}{1.630301in}}%
\pgfpathcurveto{\pgfqpoint{0.760605in}{1.622065in}}{\pgfqpoint{0.763877in}{1.614165in}}{\pgfqpoint{0.769701in}{1.608341in}}%
\pgfpathcurveto{\pgfqpoint{0.775525in}{1.602517in}}{\pgfqpoint{0.783425in}{1.599245in}}{\pgfqpoint{0.791661in}{1.599245in}}%
\pgfpathclose%
\pgfusepath{stroke,fill}%
\end{pgfscope}%
\begin{pgfscope}%
\pgfpathrectangle{\pgfqpoint{0.100000in}{0.212622in}}{\pgfqpoint{3.696000in}{3.696000in}}%
\pgfusepath{clip}%
\pgfsetbuttcap%
\pgfsetroundjoin%
\definecolor{currentfill}{rgb}{0.121569,0.466667,0.705882}%
\pgfsetfillcolor{currentfill}%
\pgfsetfillopacity{0.593169}%
\pgfsetlinewidth{1.003750pt}%
\definecolor{currentstroke}{rgb}{0.121569,0.466667,0.705882}%
\pgfsetstrokecolor{currentstroke}%
\pgfsetstrokeopacity{0.593169}%
\pgfsetdash{}{0pt}%
\pgfpathmoveto{\pgfqpoint{0.791643in}{1.599240in}}%
\pgfpathcurveto{\pgfqpoint{0.799880in}{1.599240in}}{\pgfqpoint{0.807780in}{1.602513in}}{\pgfqpoint{0.813604in}{1.608337in}}%
\pgfpathcurveto{\pgfqpoint{0.819427in}{1.614161in}}{\pgfqpoint{0.822700in}{1.622061in}}{\pgfqpoint{0.822700in}{1.630297in}}%
\pgfpathcurveto{\pgfqpoint{0.822700in}{1.638533in}}{\pgfqpoint{0.819427in}{1.646433in}}{\pgfqpoint{0.813604in}{1.652257in}}%
\pgfpathcurveto{\pgfqpoint{0.807780in}{1.658081in}}{\pgfqpoint{0.799880in}{1.661353in}}{\pgfqpoint{0.791643in}{1.661353in}}%
\pgfpathcurveto{\pgfqpoint{0.783407in}{1.661353in}}{\pgfqpoint{0.775507in}{1.658081in}}{\pgfqpoint{0.769683in}{1.652257in}}%
\pgfpathcurveto{\pgfqpoint{0.763859in}{1.646433in}}{\pgfqpoint{0.760587in}{1.638533in}}{\pgfqpoint{0.760587in}{1.630297in}}%
\pgfpathcurveto{\pgfqpoint{0.760587in}{1.622061in}}{\pgfqpoint{0.763859in}{1.614161in}}{\pgfqpoint{0.769683in}{1.608337in}}%
\pgfpathcurveto{\pgfqpoint{0.775507in}{1.602513in}}{\pgfqpoint{0.783407in}{1.599240in}}{\pgfqpoint{0.791643in}{1.599240in}}%
\pgfpathclose%
\pgfusepath{stroke,fill}%
\end{pgfscope}%
\begin{pgfscope}%
\pgfpathrectangle{\pgfqpoint{0.100000in}{0.212622in}}{\pgfqpoint{3.696000in}{3.696000in}}%
\pgfusepath{clip}%
\pgfsetbuttcap%
\pgfsetroundjoin%
\definecolor{currentfill}{rgb}{0.121569,0.466667,0.705882}%
\pgfsetfillcolor{currentfill}%
\pgfsetfillopacity{0.593173}%
\pgfsetlinewidth{1.003750pt}%
\definecolor{currentstroke}{rgb}{0.121569,0.466667,0.705882}%
\pgfsetstrokecolor{currentstroke}%
\pgfsetstrokeopacity{0.593173}%
\pgfsetdash{}{0pt}%
\pgfpathmoveto{\pgfqpoint{0.791632in}{1.599238in}}%
\pgfpathcurveto{\pgfqpoint{0.799868in}{1.599238in}}{\pgfqpoint{0.807768in}{1.602510in}}{\pgfqpoint{0.813592in}{1.608334in}}%
\pgfpathcurveto{\pgfqpoint{0.819416in}{1.614158in}}{\pgfqpoint{0.822689in}{1.622058in}}{\pgfqpoint{0.822689in}{1.630294in}}%
\pgfpathcurveto{\pgfqpoint{0.822689in}{1.638530in}}{\pgfqpoint{0.819416in}{1.646431in}}{\pgfqpoint{0.813592in}{1.652254in}}%
\pgfpathcurveto{\pgfqpoint{0.807768in}{1.658078in}}{\pgfqpoint{0.799868in}{1.661351in}}{\pgfqpoint{0.791632in}{1.661351in}}%
\pgfpathcurveto{\pgfqpoint{0.783396in}{1.661351in}}{\pgfqpoint{0.775496in}{1.658078in}}{\pgfqpoint{0.769672in}{1.652254in}}%
\pgfpathcurveto{\pgfqpoint{0.763848in}{1.646431in}}{\pgfqpoint{0.760576in}{1.638530in}}{\pgfqpoint{0.760576in}{1.630294in}}%
\pgfpathcurveto{\pgfqpoint{0.760576in}{1.622058in}}{\pgfqpoint{0.763848in}{1.614158in}}{\pgfqpoint{0.769672in}{1.608334in}}%
\pgfpathcurveto{\pgfqpoint{0.775496in}{1.602510in}}{\pgfqpoint{0.783396in}{1.599238in}}{\pgfqpoint{0.791632in}{1.599238in}}%
\pgfpathclose%
\pgfusepath{stroke,fill}%
\end{pgfscope}%
\begin{pgfscope}%
\pgfpathrectangle{\pgfqpoint{0.100000in}{0.212622in}}{\pgfqpoint{3.696000in}{3.696000in}}%
\pgfusepath{clip}%
\pgfsetbuttcap%
\pgfsetroundjoin%
\definecolor{currentfill}{rgb}{0.121569,0.466667,0.705882}%
\pgfsetfillcolor{currentfill}%
\pgfsetfillopacity{0.593176}%
\pgfsetlinewidth{1.003750pt}%
\definecolor{currentstroke}{rgb}{0.121569,0.466667,0.705882}%
\pgfsetstrokecolor{currentstroke}%
\pgfsetstrokeopacity{0.593176}%
\pgfsetdash{}{0pt}%
\pgfpathmoveto{\pgfqpoint{0.791627in}{1.599237in}}%
\pgfpathcurveto{\pgfqpoint{0.799863in}{1.599237in}}{\pgfqpoint{0.807763in}{1.602509in}}{\pgfqpoint{0.813587in}{1.608333in}}%
\pgfpathcurveto{\pgfqpoint{0.819411in}{1.614157in}}{\pgfqpoint{0.822684in}{1.622057in}}{\pgfqpoint{0.822684in}{1.630293in}}%
\pgfpathcurveto{\pgfqpoint{0.822684in}{1.638530in}}{\pgfqpoint{0.819411in}{1.646430in}}{\pgfqpoint{0.813587in}{1.652254in}}%
\pgfpathcurveto{\pgfqpoint{0.807763in}{1.658078in}}{\pgfqpoint{0.799863in}{1.661350in}}{\pgfqpoint{0.791627in}{1.661350in}}%
\pgfpathcurveto{\pgfqpoint{0.783391in}{1.661350in}}{\pgfqpoint{0.775491in}{1.658078in}}{\pgfqpoint{0.769667in}{1.652254in}}%
\pgfpathcurveto{\pgfqpoint{0.763843in}{1.646430in}}{\pgfqpoint{0.760571in}{1.638530in}}{\pgfqpoint{0.760571in}{1.630293in}}%
\pgfpathcurveto{\pgfqpoint{0.760571in}{1.622057in}}{\pgfqpoint{0.763843in}{1.614157in}}{\pgfqpoint{0.769667in}{1.608333in}}%
\pgfpathcurveto{\pgfqpoint{0.775491in}{1.602509in}}{\pgfqpoint{0.783391in}{1.599237in}}{\pgfqpoint{0.791627in}{1.599237in}}%
\pgfpathclose%
\pgfusepath{stroke,fill}%
\end{pgfscope}%
\begin{pgfscope}%
\pgfpathrectangle{\pgfqpoint{0.100000in}{0.212622in}}{\pgfqpoint{3.696000in}{3.696000in}}%
\pgfusepath{clip}%
\pgfsetbuttcap%
\pgfsetroundjoin%
\definecolor{currentfill}{rgb}{0.121569,0.466667,0.705882}%
\pgfsetfillcolor{currentfill}%
\pgfsetfillopacity{0.593178}%
\pgfsetlinewidth{1.003750pt}%
\definecolor{currentstroke}{rgb}{0.121569,0.466667,0.705882}%
\pgfsetstrokecolor{currentstroke}%
\pgfsetstrokeopacity{0.593178}%
\pgfsetdash{}{0pt}%
\pgfpathmoveto{\pgfqpoint{0.791624in}{1.599236in}}%
\pgfpathcurveto{\pgfqpoint{0.799860in}{1.599236in}}{\pgfqpoint{0.807760in}{1.602508in}}{\pgfqpoint{0.813584in}{1.608332in}}%
\pgfpathcurveto{\pgfqpoint{0.819408in}{1.614156in}}{\pgfqpoint{0.822680in}{1.622056in}}{\pgfqpoint{0.822680in}{1.630293in}}%
\pgfpathcurveto{\pgfqpoint{0.822680in}{1.638529in}}{\pgfqpoint{0.819408in}{1.646429in}}{\pgfqpoint{0.813584in}{1.652253in}}%
\pgfpathcurveto{\pgfqpoint{0.807760in}{1.658077in}}{\pgfqpoint{0.799860in}{1.661349in}}{\pgfqpoint{0.791624in}{1.661349in}}%
\pgfpathcurveto{\pgfqpoint{0.783388in}{1.661349in}}{\pgfqpoint{0.775488in}{1.658077in}}{\pgfqpoint{0.769664in}{1.652253in}}%
\pgfpathcurveto{\pgfqpoint{0.763840in}{1.646429in}}{\pgfqpoint{0.760567in}{1.638529in}}{\pgfqpoint{0.760567in}{1.630293in}}%
\pgfpathcurveto{\pgfqpoint{0.760567in}{1.622056in}}{\pgfqpoint{0.763840in}{1.614156in}}{\pgfqpoint{0.769664in}{1.608332in}}%
\pgfpathcurveto{\pgfqpoint{0.775488in}{1.602508in}}{\pgfqpoint{0.783388in}{1.599236in}}{\pgfqpoint{0.791624in}{1.599236in}}%
\pgfpathclose%
\pgfusepath{stroke,fill}%
\end{pgfscope}%
\begin{pgfscope}%
\pgfpathrectangle{\pgfqpoint{0.100000in}{0.212622in}}{\pgfqpoint{3.696000in}{3.696000in}}%
\pgfusepath{clip}%
\pgfsetbuttcap%
\pgfsetroundjoin%
\definecolor{currentfill}{rgb}{0.121569,0.466667,0.705882}%
\pgfsetfillcolor{currentfill}%
\pgfsetfillopacity{0.593179}%
\pgfsetlinewidth{1.003750pt}%
\definecolor{currentstroke}{rgb}{0.121569,0.466667,0.705882}%
\pgfsetstrokecolor{currentstroke}%
\pgfsetstrokeopacity{0.593179}%
\pgfsetdash{}{0pt}%
\pgfpathmoveto{\pgfqpoint{0.791622in}{1.599236in}}%
\pgfpathcurveto{\pgfqpoint{0.799859in}{1.599236in}}{\pgfqpoint{0.807759in}{1.602508in}}{\pgfqpoint{0.813583in}{1.608332in}}%
\pgfpathcurveto{\pgfqpoint{0.819407in}{1.614156in}}{\pgfqpoint{0.822679in}{1.622056in}}{\pgfqpoint{0.822679in}{1.630292in}}%
\pgfpathcurveto{\pgfqpoint{0.822679in}{1.638529in}}{\pgfqpoint{0.819407in}{1.646429in}}{\pgfqpoint{0.813583in}{1.652253in}}%
\pgfpathcurveto{\pgfqpoint{0.807759in}{1.658077in}}{\pgfqpoint{0.799859in}{1.661349in}}{\pgfqpoint{0.791622in}{1.661349in}}%
\pgfpathcurveto{\pgfqpoint{0.783386in}{1.661349in}}{\pgfqpoint{0.775486in}{1.658077in}}{\pgfqpoint{0.769662in}{1.652253in}}%
\pgfpathcurveto{\pgfqpoint{0.763838in}{1.646429in}}{\pgfqpoint{0.760566in}{1.638529in}}{\pgfqpoint{0.760566in}{1.630292in}}%
\pgfpathcurveto{\pgfqpoint{0.760566in}{1.622056in}}{\pgfqpoint{0.763838in}{1.614156in}}{\pgfqpoint{0.769662in}{1.608332in}}%
\pgfpathcurveto{\pgfqpoint{0.775486in}{1.602508in}}{\pgfqpoint{0.783386in}{1.599236in}}{\pgfqpoint{0.791622in}{1.599236in}}%
\pgfpathclose%
\pgfusepath{stroke,fill}%
\end{pgfscope}%
\begin{pgfscope}%
\pgfpathrectangle{\pgfqpoint{0.100000in}{0.212622in}}{\pgfqpoint{3.696000in}{3.696000in}}%
\pgfusepath{clip}%
\pgfsetbuttcap%
\pgfsetroundjoin%
\definecolor{currentfill}{rgb}{0.121569,0.466667,0.705882}%
\pgfsetfillcolor{currentfill}%
\pgfsetfillopacity{0.593179}%
\pgfsetlinewidth{1.003750pt}%
\definecolor{currentstroke}{rgb}{0.121569,0.466667,0.705882}%
\pgfsetstrokecolor{currentstroke}%
\pgfsetstrokeopacity{0.593179}%
\pgfsetdash{}{0pt}%
\pgfpathmoveto{\pgfqpoint{0.791621in}{1.599236in}}%
\pgfpathcurveto{\pgfqpoint{0.799858in}{1.599236in}}{\pgfqpoint{0.807758in}{1.602508in}}{\pgfqpoint{0.813582in}{1.608332in}}%
\pgfpathcurveto{\pgfqpoint{0.819406in}{1.614156in}}{\pgfqpoint{0.822678in}{1.622056in}}{\pgfqpoint{0.822678in}{1.630292in}}%
\pgfpathcurveto{\pgfqpoint{0.822678in}{1.638529in}}{\pgfqpoint{0.819406in}{1.646429in}}{\pgfqpoint{0.813582in}{1.652253in}}%
\pgfpathcurveto{\pgfqpoint{0.807758in}{1.658076in}}{\pgfqpoint{0.799858in}{1.661349in}}{\pgfqpoint{0.791621in}{1.661349in}}%
\pgfpathcurveto{\pgfqpoint{0.783385in}{1.661349in}}{\pgfqpoint{0.775485in}{1.658076in}}{\pgfqpoint{0.769661in}{1.652253in}}%
\pgfpathcurveto{\pgfqpoint{0.763837in}{1.646429in}}{\pgfqpoint{0.760565in}{1.638529in}}{\pgfqpoint{0.760565in}{1.630292in}}%
\pgfpathcurveto{\pgfqpoint{0.760565in}{1.622056in}}{\pgfqpoint{0.763837in}{1.614156in}}{\pgfqpoint{0.769661in}{1.608332in}}%
\pgfpathcurveto{\pgfqpoint{0.775485in}{1.602508in}}{\pgfqpoint{0.783385in}{1.599236in}}{\pgfqpoint{0.791621in}{1.599236in}}%
\pgfpathclose%
\pgfusepath{stroke,fill}%
\end{pgfscope}%
\begin{pgfscope}%
\pgfpathrectangle{\pgfqpoint{0.100000in}{0.212622in}}{\pgfqpoint{3.696000in}{3.696000in}}%
\pgfusepath{clip}%
\pgfsetbuttcap%
\pgfsetroundjoin%
\definecolor{currentfill}{rgb}{0.121569,0.466667,0.705882}%
\pgfsetfillcolor{currentfill}%
\pgfsetfillopacity{0.593638}%
\pgfsetlinewidth{1.003750pt}%
\definecolor{currentstroke}{rgb}{0.121569,0.466667,0.705882}%
\pgfsetstrokecolor{currentstroke}%
\pgfsetstrokeopacity{0.593638}%
\pgfsetdash{}{0pt}%
\pgfpathmoveto{\pgfqpoint{0.790843in}{1.598847in}}%
\pgfpathcurveto{\pgfqpoint{0.799079in}{1.598847in}}{\pgfqpoint{0.806980in}{1.602119in}}{\pgfqpoint{0.812803in}{1.607943in}}%
\pgfpathcurveto{\pgfqpoint{0.818627in}{1.613767in}}{\pgfqpoint{0.821900in}{1.621667in}}{\pgfqpoint{0.821900in}{1.629903in}}%
\pgfpathcurveto{\pgfqpoint{0.821900in}{1.638140in}}{\pgfqpoint{0.818627in}{1.646040in}}{\pgfqpoint{0.812803in}{1.651864in}}%
\pgfpathcurveto{\pgfqpoint{0.806980in}{1.657688in}}{\pgfqpoint{0.799079in}{1.660960in}}{\pgfqpoint{0.790843in}{1.660960in}}%
\pgfpathcurveto{\pgfqpoint{0.782607in}{1.660960in}}{\pgfqpoint{0.774707in}{1.657688in}}{\pgfqpoint{0.768883in}{1.651864in}}%
\pgfpathcurveto{\pgfqpoint{0.763059in}{1.646040in}}{\pgfqpoint{0.759787in}{1.638140in}}{\pgfqpoint{0.759787in}{1.629903in}}%
\pgfpathcurveto{\pgfqpoint{0.759787in}{1.621667in}}{\pgfqpoint{0.763059in}{1.613767in}}{\pgfqpoint{0.768883in}{1.607943in}}%
\pgfpathcurveto{\pgfqpoint{0.774707in}{1.602119in}}{\pgfqpoint{0.782607in}{1.598847in}}{\pgfqpoint{0.790843in}{1.598847in}}%
\pgfpathclose%
\pgfusepath{stroke,fill}%
\end{pgfscope}%
\begin{pgfscope}%
\pgfpathrectangle{\pgfqpoint{0.100000in}{0.212622in}}{\pgfqpoint{3.696000in}{3.696000in}}%
\pgfusepath{clip}%
\pgfsetbuttcap%
\pgfsetroundjoin%
\definecolor{currentfill}{rgb}{0.121569,0.466667,0.705882}%
\pgfsetfillcolor{currentfill}%
\pgfsetfillopacity{0.593887}%
\pgfsetlinewidth{1.003750pt}%
\definecolor{currentstroke}{rgb}{0.121569,0.466667,0.705882}%
\pgfsetstrokecolor{currentstroke}%
\pgfsetstrokeopacity{0.593887}%
\pgfsetdash{}{0pt}%
\pgfpathmoveto{\pgfqpoint{0.790228in}{1.598811in}}%
\pgfpathcurveto{\pgfqpoint{0.798464in}{1.598811in}}{\pgfqpoint{0.806364in}{1.602084in}}{\pgfqpoint{0.812188in}{1.607908in}}%
\pgfpathcurveto{\pgfqpoint{0.818012in}{1.613732in}}{\pgfqpoint{0.821284in}{1.621632in}}{\pgfqpoint{0.821284in}{1.629868in}}%
\pgfpathcurveto{\pgfqpoint{0.821284in}{1.638104in}}{\pgfqpoint{0.818012in}{1.646004in}}{\pgfqpoint{0.812188in}{1.651828in}}%
\pgfpathcurveto{\pgfqpoint{0.806364in}{1.657652in}}{\pgfqpoint{0.798464in}{1.660924in}}{\pgfqpoint{0.790228in}{1.660924in}}%
\pgfpathcurveto{\pgfqpoint{0.781991in}{1.660924in}}{\pgfqpoint{0.774091in}{1.657652in}}{\pgfqpoint{0.768267in}{1.651828in}}%
\pgfpathcurveto{\pgfqpoint{0.762443in}{1.646004in}}{\pgfqpoint{0.759171in}{1.638104in}}{\pgfqpoint{0.759171in}{1.629868in}}%
\pgfpathcurveto{\pgfqpoint{0.759171in}{1.621632in}}{\pgfqpoint{0.762443in}{1.613732in}}{\pgfqpoint{0.768267in}{1.607908in}}%
\pgfpathcurveto{\pgfqpoint{0.774091in}{1.602084in}}{\pgfqpoint{0.781991in}{1.598811in}}{\pgfqpoint{0.790228in}{1.598811in}}%
\pgfpathclose%
\pgfusepath{stroke,fill}%
\end{pgfscope}%
\begin{pgfscope}%
\pgfpathrectangle{\pgfqpoint{0.100000in}{0.212622in}}{\pgfqpoint{3.696000in}{3.696000in}}%
\pgfusepath{clip}%
\pgfsetbuttcap%
\pgfsetroundjoin%
\definecolor{currentfill}{rgb}{0.121569,0.466667,0.705882}%
\pgfsetfillcolor{currentfill}%
\pgfsetfillopacity{0.594026}%
\pgfsetlinewidth{1.003750pt}%
\definecolor{currentstroke}{rgb}{0.121569,0.466667,0.705882}%
\pgfsetstrokecolor{currentstroke}%
\pgfsetstrokeopacity{0.594026}%
\pgfsetdash{}{0pt}%
\pgfpathmoveto{\pgfqpoint{0.789978in}{1.598708in}}%
\pgfpathcurveto{\pgfqpoint{0.798214in}{1.598708in}}{\pgfqpoint{0.806115in}{1.601980in}}{\pgfqpoint{0.811938in}{1.607804in}}%
\pgfpathcurveto{\pgfqpoint{0.817762in}{1.613628in}}{\pgfqpoint{0.821035in}{1.621528in}}{\pgfqpoint{0.821035in}{1.629765in}}%
\pgfpathcurveto{\pgfqpoint{0.821035in}{1.638001in}}{\pgfqpoint{0.817762in}{1.645901in}}{\pgfqpoint{0.811938in}{1.651725in}}%
\pgfpathcurveto{\pgfqpoint{0.806115in}{1.657549in}}{\pgfqpoint{0.798214in}{1.660821in}}{\pgfqpoint{0.789978in}{1.660821in}}%
\pgfpathcurveto{\pgfqpoint{0.781742in}{1.660821in}}{\pgfqpoint{0.773842in}{1.657549in}}{\pgfqpoint{0.768018in}{1.651725in}}%
\pgfpathcurveto{\pgfqpoint{0.762194in}{1.645901in}}{\pgfqpoint{0.758922in}{1.638001in}}{\pgfqpoint{0.758922in}{1.629765in}}%
\pgfpathcurveto{\pgfqpoint{0.758922in}{1.621528in}}{\pgfqpoint{0.762194in}{1.613628in}}{\pgfqpoint{0.768018in}{1.607804in}}%
\pgfpathcurveto{\pgfqpoint{0.773842in}{1.601980in}}{\pgfqpoint{0.781742in}{1.598708in}}{\pgfqpoint{0.789978in}{1.598708in}}%
\pgfpathclose%
\pgfusepath{stroke,fill}%
\end{pgfscope}%
\begin{pgfscope}%
\pgfpathrectangle{\pgfqpoint{0.100000in}{0.212622in}}{\pgfqpoint{3.696000in}{3.696000in}}%
\pgfusepath{clip}%
\pgfsetbuttcap%
\pgfsetroundjoin%
\definecolor{currentfill}{rgb}{0.121569,0.466667,0.705882}%
\pgfsetfillcolor{currentfill}%
\pgfsetfillopacity{0.594106}%
\pgfsetlinewidth{1.003750pt}%
\definecolor{currentstroke}{rgb}{0.121569,0.466667,0.705882}%
\pgfsetstrokecolor{currentstroke}%
\pgfsetstrokeopacity{0.594106}%
\pgfsetdash{}{0pt}%
\pgfpathmoveto{\pgfqpoint{0.789839in}{1.598676in}}%
\pgfpathcurveto{\pgfqpoint{0.798075in}{1.598676in}}{\pgfqpoint{0.805975in}{1.601948in}}{\pgfqpoint{0.811799in}{1.607772in}}%
\pgfpathcurveto{\pgfqpoint{0.817623in}{1.613596in}}{\pgfqpoint{0.820895in}{1.621496in}}{\pgfqpoint{0.820895in}{1.629732in}}%
\pgfpathcurveto{\pgfqpoint{0.820895in}{1.637969in}}{\pgfqpoint{0.817623in}{1.645869in}}{\pgfqpoint{0.811799in}{1.651693in}}%
\pgfpathcurveto{\pgfqpoint{0.805975in}{1.657517in}}{\pgfqpoint{0.798075in}{1.660789in}}{\pgfqpoint{0.789839in}{1.660789in}}%
\pgfpathcurveto{\pgfqpoint{0.781603in}{1.660789in}}{\pgfqpoint{0.773702in}{1.657517in}}{\pgfqpoint{0.767879in}{1.651693in}}%
\pgfpathcurveto{\pgfqpoint{0.762055in}{1.645869in}}{\pgfqpoint{0.758782in}{1.637969in}}{\pgfqpoint{0.758782in}{1.629732in}}%
\pgfpathcurveto{\pgfqpoint{0.758782in}{1.621496in}}{\pgfqpoint{0.762055in}{1.613596in}}{\pgfqpoint{0.767879in}{1.607772in}}%
\pgfpathcurveto{\pgfqpoint{0.773702in}{1.601948in}}{\pgfqpoint{0.781603in}{1.598676in}}{\pgfqpoint{0.789839in}{1.598676in}}%
\pgfpathclose%
\pgfusepath{stroke,fill}%
\end{pgfscope}%
\begin{pgfscope}%
\pgfpathrectangle{\pgfqpoint{0.100000in}{0.212622in}}{\pgfqpoint{3.696000in}{3.696000in}}%
\pgfusepath{clip}%
\pgfsetbuttcap%
\pgfsetroundjoin%
\definecolor{currentfill}{rgb}{0.121569,0.466667,0.705882}%
\pgfsetfillcolor{currentfill}%
\pgfsetfillopacity{0.594148}%
\pgfsetlinewidth{1.003750pt}%
\definecolor{currentstroke}{rgb}{0.121569,0.466667,0.705882}%
\pgfsetstrokecolor{currentstroke}%
\pgfsetstrokeopacity{0.594148}%
\pgfsetdash{}{0pt}%
\pgfpathmoveto{\pgfqpoint{0.789747in}{1.598665in}}%
\pgfpathcurveto{\pgfqpoint{0.797983in}{1.598665in}}{\pgfqpoint{0.805883in}{1.601937in}}{\pgfqpoint{0.811707in}{1.607761in}}%
\pgfpathcurveto{\pgfqpoint{0.817531in}{1.613585in}}{\pgfqpoint{0.820803in}{1.621485in}}{\pgfqpoint{0.820803in}{1.629721in}}%
\pgfpathcurveto{\pgfqpoint{0.820803in}{1.637958in}}{\pgfqpoint{0.817531in}{1.645858in}}{\pgfqpoint{0.811707in}{1.651682in}}%
\pgfpathcurveto{\pgfqpoint{0.805883in}{1.657505in}}{\pgfqpoint{0.797983in}{1.660778in}}{\pgfqpoint{0.789747in}{1.660778in}}%
\pgfpathcurveto{\pgfqpoint{0.781510in}{1.660778in}}{\pgfqpoint{0.773610in}{1.657505in}}{\pgfqpoint{0.767786in}{1.651682in}}%
\pgfpathcurveto{\pgfqpoint{0.761963in}{1.645858in}}{\pgfqpoint{0.758690in}{1.637958in}}{\pgfqpoint{0.758690in}{1.629721in}}%
\pgfpathcurveto{\pgfqpoint{0.758690in}{1.621485in}}{\pgfqpoint{0.761963in}{1.613585in}}{\pgfqpoint{0.767786in}{1.607761in}}%
\pgfpathcurveto{\pgfqpoint{0.773610in}{1.601937in}}{\pgfqpoint{0.781510in}{1.598665in}}{\pgfqpoint{0.789747in}{1.598665in}}%
\pgfpathclose%
\pgfusepath{stroke,fill}%
\end{pgfscope}%
\begin{pgfscope}%
\pgfpathrectangle{\pgfqpoint{0.100000in}{0.212622in}}{\pgfqpoint{3.696000in}{3.696000in}}%
\pgfusepath{clip}%
\pgfsetbuttcap%
\pgfsetroundjoin%
\definecolor{currentfill}{rgb}{0.121569,0.466667,0.705882}%
\pgfsetfillcolor{currentfill}%
\pgfsetfillopacity{0.594171}%
\pgfsetlinewidth{1.003750pt}%
\definecolor{currentstroke}{rgb}{0.121569,0.466667,0.705882}%
\pgfsetstrokecolor{currentstroke}%
\pgfsetstrokeopacity{0.594171}%
\pgfsetdash{}{0pt}%
\pgfpathmoveto{\pgfqpoint{0.789704in}{1.598648in}}%
\pgfpathcurveto{\pgfqpoint{0.797940in}{1.598648in}}{\pgfqpoint{0.805840in}{1.601921in}}{\pgfqpoint{0.811664in}{1.607744in}}%
\pgfpathcurveto{\pgfqpoint{0.817488in}{1.613568in}}{\pgfqpoint{0.820761in}{1.621468in}}{\pgfqpoint{0.820761in}{1.629705in}}%
\pgfpathcurveto{\pgfqpoint{0.820761in}{1.637941in}}{\pgfqpoint{0.817488in}{1.645841in}}{\pgfqpoint{0.811664in}{1.651665in}}%
\pgfpathcurveto{\pgfqpoint{0.805840in}{1.657489in}}{\pgfqpoint{0.797940in}{1.660761in}}{\pgfqpoint{0.789704in}{1.660761in}}%
\pgfpathcurveto{\pgfqpoint{0.781468in}{1.660761in}}{\pgfqpoint{0.773568in}{1.657489in}}{\pgfqpoint{0.767744in}{1.651665in}}%
\pgfpathcurveto{\pgfqpoint{0.761920in}{1.645841in}}{\pgfqpoint{0.758648in}{1.637941in}}{\pgfqpoint{0.758648in}{1.629705in}}%
\pgfpathcurveto{\pgfqpoint{0.758648in}{1.621468in}}{\pgfqpoint{0.761920in}{1.613568in}}{\pgfqpoint{0.767744in}{1.607744in}}%
\pgfpathcurveto{\pgfqpoint{0.773568in}{1.601921in}}{\pgfqpoint{0.781468in}{1.598648in}}{\pgfqpoint{0.789704in}{1.598648in}}%
\pgfpathclose%
\pgfusepath{stroke,fill}%
\end{pgfscope}%
\begin{pgfscope}%
\pgfpathrectangle{\pgfqpoint{0.100000in}{0.212622in}}{\pgfqpoint{3.696000in}{3.696000in}}%
\pgfusepath{clip}%
\pgfsetbuttcap%
\pgfsetroundjoin%
\definecolor{currentfill}{rgb}{0.121569,0.466667,0.705882}%
\pgfsetfillcolor{currentfill}%
\pgfsetfillopacity{0.594184}%
\pgfsetlinewidth{1.003750pt}%
\definecolor{currentstroke}{rgb}{0.121569,0.466667,0.705882}%
\pgfsetstrokecolor{currentstroke}%
\pgfsetstrokeopacity{0.594184}%
\pgfsetdash{}{0pt}%
\pgfpathmoveto{\pgfqpoint{0.789677in}{1.598646in}}%
\pgfpathcurveto{\pgfqpoint{0.797913in}{1.598646in}}{\pgfqpoint{0.805813in}{1.601918in}}{\pgfqpoint{0.811637in}{1.607742in}}%
\pgfpathcurveto{\pgfqpoint{0.817461in}{1.613566in}}{\pgfqpoint{0.820733in}{1.621466in}}{\pgfqpoint{0.820733in}{1.629703in}}%
\pgfpathcurveto{\pgfqpoint{0.820733in}{1.637939in}}{\pgfqpoint{0.817461in}{1.645839in}}{\pgfqpoint{0.811637in}{1.651663in}}%
\pgfpathcurveto{\pgfqpoint{0.805813in}{1.657487in}}{\pgfqpoint{0.797913in}{1.660759in}}{\pgfqpoint{0.789677in}{1.660759in}}%
\pgfpathcurveto{\pgfqpoint{0.781440in}{1.660759in}}{\pgfqpoint{0.773540in}{1.657487in}}{\pgfqpoint{0.767716in}{1.651663in}}%
\pgfpathcurveto{\pgfqpoint{0.761893in}{1.645839in}}{\pgfqpoint{0.758620in}{1.637939in}}{\pgfqpoint{0.758620in}{1.629703in}}%
\pgfpathcurveto{\pgfqpoint{0.758620in}{1.621466in}}{\pgfqpoint{0.761893in}{1.613566in}}{\pgfqpoint{0.767716in}{1.607742in}}%
\pgfpathcurveto{\pgfqpoint{0.773540in}{1.601918in}}{\pgfqpoint{0.781440in}{1.598646in}}{\pgfqpoint{0.789677in}{1.598646in}}%
\pgfpathclose%
\pgfusepath{stroke,fill}%
\end{pgfscope}%
\begin{pgfscope}%
\pgfpathrectangle{\pgfqpoint{0.100000in}{0.212622in}}{\pgfqpoint{3.696000in}{3.696000in}}%
\pgfusepath{clip}%
\pgfsetbuttcap%
\pgfsetroundjoin%
\definecolor{currentfill}{rgb}{0.121569,0.466667,0.705882}%
\pgfsetfillcolor{currentfill}%
\pgfsetfillopacity{0.594191}%
\pgfsetlinewidth{1.003750pt}%
\definecolor{currentstroke}{rgb}{0.121569,0.466667,0.705882}%
\pgfsetstrokecolor{currentstroke}%
\pgfsetstrokeopacity{0.594191}%
\pgfsetdash{}{0pt}%
\pgfpathmoveto{\pgfqpoint{0.789662in}{1.598643in}}%
\pgfpathcurveto{\pgfqpoint{0.797898in}{1.598643in}}{\pgfqpoint{0.805798in}{1.601916in}}{\pgfqpoint{0.811622in}{1.607740in}}%
\pgfpathcurveto{\pgfqpoint{0.817446in}{1.613564in}}{\pgfqpoint{0.820718in}{1.621464in}}{\pgfqpoint{0.820718in}{1.629700in}}%
\pgfpathcurveto{\pgfqpoint{0.820718in}{1.637936in}}{\pgfqpoint{0.817446in}{1.645836in}}{\pgfqpoint{0.811622in}{1.651660in}}%
\pgfpathcurveto{\pgfqpoint{0.805798in}{1.657484in}}{\pgfqpoint{0.797898in}{1.660756in}}{\pgfqpoint{0.789662in}{1.660756in}}%
\pgfpathcurveto{\pgfqpoint{0.781426in}{1.660756in}}{\pgfqpoint{0.773525in}{1.657484in}}{\pgfqpoint{0.767702in}{1.651660in}}%
\pgfpathcurveto{\pgfqpoint{0.761878in}{1.645836in}}{\pgfqpoint{0.758605in}{1.637936in}}{\pgfqpoint{0.758605in}{1.629700in}}%
\pgfpathcurveto{\pgfqpoint{0.758605in}{1.621464in}}{\pgfqpoint{0.761878in}{1.613564in}}{\pgfqpoint{0.767702in}{1.607740in}}%
\pgfpathcurveto{\pgfqpoint{0.773525in}{1.601916in}}{\pgfqpoint{0.781426in}{1.598643in}}{\pgfqpoint{0.789662in}{1.598643in}}%
\pgfpathclose%
\pgfusepath{stroke,fill}%
\end{pgfscope}%
\begin{pgfscope}%
\pgfpathrectangle{\pgfqpoint{0.100000in}{0.212622in}}{\pgfqpoint{3.696000in}{3.696000in}}%
\pgfusepath{clip}%
\pgfsetbuttcap%
\pgfsetroundjoin%
\definecolor{currentfill}{rgb}{0.121569,0.466667,0.705882}%
\pgfsetfillcolor{currentfill}%
\pgfsetfillopacity{0.594195}%
\pgfsetlinewidth{1.003750pt}%
\definecolor{currentstroke}{rgb}{0.121569,0.466667,0.705882}%
\pgfsetstrokecolor{currentstroke}%
\pgfsetstrokeopacity{0.594195}%
\pgfsetdash{}{0pt}%
\pgfpathmoveto{\pgfqpoint{0.789656in}{1.598641in}}%
\pgfpathcurveto{\pgfqpoint{0.797892in}{1.598641in}}{\pgfqpoint{0.805792in}{1.601913in}}{\pgfqpoint{0.811616in}{1.607737in}}%
\pgfpathcurveto{\pgfqpoint{0.817440in}{1.613561in}}{\pgfqpoint{0.820712in}{1.621461in}}{\pgfqpoint{0.820712in}{1.629697in}}%
\pgfpathcurveto{\pgfqpoint{0.820712in}{1.637934in}}{\pgfqpoint{0.817440in}{1.645834in}}{\pgfqpoint{0.811616in}{1.651658in}}%
\pgfpathcurveto{\pgfqpoint{0.805792in}{1.657482in}}{\pgfqpoint{0.797892in}{1.660754in}}{\pgfqpoint{0.789656in}{1.660754in}}%
\pgfpathcurveto{\pgfqpoint{0.781419in}{1.660754in}}{\pgfqpoint{0.773519in}{1.657482in}}{\pgfqpoint{0.767695in}{1.651658in}}%
\pgfpathcurveto{\pgfqpoint{0.761872in}{1.645834in}}{\pgfqpoint{0.758599in}{1.637934in}}{\pgfqpoint{0.758599in}{1.629697in}}%
\pgfpathcurveto{\pgfqpoint{0.758599in}{1.621461in}}{\pgfqpoint{0.761872in}{1.613561in}}{\pgfqpoint{0.767695in}{1.607737in}}%
\pgfpathcurveto{\pgfqpoint{0.773519in}{1.601913in}}{\pgfqpoint{0.781419in}{1.598641in}}{\pgfqpoint{0.789656in}{1.598641in}}%
\pgfpathclose%
\pgfusepath{stroke,fill}%
\end{pgfscope}%
\begin{pgfscope}%
\pgfpathrectangle{\pgfqpoint{0.100000in}{0.212622in}}{\pgfqpoint{3.696000in}{3.696000in}}%
\pgfusepath{clip}%
\pgfsetbuttcap%
\pgfsetroundjoin%
\definecolor{currentfill}{rgb}{0.121569,0.466667,0.705882}%
\pgfsetfillcolor{currentfill}%
\pgfsetfillopacity{0.594704}%
\pgfsetlinewidth{1.003750pt}%
\definecolor{currentstroke}{rgb}{0.121569,0.466667,0.705882}%
\pgfsetstrokecolor{currentstroke}%
\pgfsetstrokeopacity{0.594704}%
\pgfsetdash{}{0pt}%
\pgfpathmoveto{\pgfqpoint{0.788687in}{1.598451in}}%
\pgfpathcurveto{\pgfqpoint{0.796923in}{1.598451in}}{\pgfqpoint{0.804823in}{1.601724in}}{\pgfqpoint{0.810647in}{1.607547in}}%
\pgfpathcurveto{\pgfqpoint{0.816471in}{1.613371in}}{\pgfqpoint{0.819743in}{1.621271in}}{\pgfqpoint{0.819743in}{1.629508in}}%
\pgfpathcurveto{\pgfqpoint{0.819743in}{1.637744in}}{\pgfqpoint{0.816471in}{1.645644in}}{\pgfqpoint{0.810647in}{1.651468in}}%
\pgfpathcurveto{\pgfqpoint{0.804823in}{1.657292in}}{\pgfqpoint{0.796923in}{1.660564in}}{\pgfqpoint{0.788687in}{1.660564in}}%
\pgfpathcurveto{\pgfqpoint{0.780450in}{1.660564in}}{\pgfqpoint{0.772550in}{1.657292in}}{\pgfqpoint{0.766727in}{1.651468in}}%
\pgfpathcurveto{\pgfqpoint{0.760903in}{1.645644in}}{\pgfqpoint{0.757630in}{1.637744in}}{\pgfqpoint{0.757630in}{1.629508in}}%
\pgfpathcurveto{\pgfqpoint{0.757630in}{1.621271in}}{\pgfqpoint{0.760903in}{1.613371in}}{\pgfqpoint{0.766727in}{1.607547in}}%
\pgfpathcurveto{\pgfqpoint{0.772550in}{1.601724in}}{\pgfqpoint{0.780450in}{1.598451in}}{\pgfqpoint{0.788687in}{1.598451in}}%
\pgfpathclose%
\pgfusepath{stroke,fill}%
\end{pgfscope}%
\begin{pgfscope}%
\pgfpathrectangle{\pgfqpoint{0.100000in}{0.212622in}}{\pgfqpoint{3.696000in}{3.696000in}}%
\pgfusepath{clip}%
\pgfsetbuttcap%
\pgfsetroundjoin%
\definecolor{currentfill}{rgb}{0.121569,0.466667,0.705882}%
\pgfsetfillcolor{currentfill}%
\pgfsetfillopacity{0.595002}%
\pgfsetlinewidth{1.003750pt}%
\definecolor{currentstroke}{rgb}{0.121569,0.466667,0.705882}%
\pgfsetstrokecolor{currentstroke}%
\pgfsetstrokeopacity{0.595002}%
\pgfsetdash{}{0pt}%
\pgfpathmoveto{\pgfqpoint{0.788306in}{1.598316in}}%
\pgfpathcurveto{\pgfqpoint{0.796542in}{1.598316in}}{\pgfqpoint{0.804442in}{1.601589in}}{\pgfqpoint{0.810266in}{1.607413in}}%
\pgfpathcurveto{\pgfqpoint{0.816090in}{1.613237in}}{\pgfqpoint{0.819362in}{1.621137in}}{\pgfqpoint{0.819362in}{1.629373in}}%
\pgfpathcurveto{\pgfqpoint{0.819362in}{1.637609in}}{\pgfqpoint{0.816090in}{1.645509in}}{\pgfqpoint{0.810266in}{1.651333in}}%
\pgfpathcurveto{\pgfqpoint{0.804442in}{1.657157in}}{\pgfqpoint{0.796542in}{1.660429in}}{\pgfqpoint{0.788306in}{1.660429in}}%
\pgfpathcurveto{\pgfqpoint{0.780070in}{1.660429in}}{\pgfqpoint{0.772170in}{1.657157in}}{\pgfqpoint{0.766346in}{1.651333in}}%
\pgfpathcurveto{\pgfqpoint{0.760522in}{1.645509in}}{\pgfqpoint{0.757249in}{1.637609in}}{\pgfqpoint{0.757249in}{1.629373in}}%
\pgfpathcurveto{\pgfqpoint{0.757249in}{1.621137in}}{\pgfqpoint{0.760522in}{1.613237in}}{\pgfqpoint{0.766346in}{1.607413in}}%
\pgfpathcurveto{\pgfqpoint{0.772170in}{1.601589in}}{\pgfqpoint{0.780070in}{1.598316in}}{\pgfqpoint{0.788306in}{1.598316in}}%
\pgfpathclose%
\pgfusepath{stroke,fill}%
\end{pgfscope}%
\begin{pgfscope}%
\pgfpathrectangle{\pgfqpoint{0.100000in}{0.212622in}}{\pgfqpoint{3.696000in}{3.696000in}}%
\pgfusepath{clip}%
\pgfsetbuttcap%
\pgfsetroundjoin%
\definecolor{currentfill}{rgb}{0.121569,0.466667,0.705882}%
\pgfsetfillcolor{currentfill}%
\pgfsetfillopacity{0.595162}%
\pgfsetlinewidth{1.003750pt}%
\definecolor{currentstroke}{rgb}{0.121569,0.466667,0.705882}%
\pgfsetstrokecolor{currentstroke}%
\pgfsetstrokeopacity{0.595162}%
\pgfsetdash{}{0pt}%
\pgfpathmoveto{\pgfqpoint{0.788122in}{1.598203in}}%
\pgfpathcurveto{\pgfqpoint{0.796358in}{1.598203in}}{\pgfqpoint{0.804258in}{1.601475in}}{\pgfqpoint{0.810082in}{1.607299in}}%
\pgfpathcurveto{\pgfqpoint{0.815906in}{1.613123in}}{\pgfqpoint{0.819179in}{1.621023in}}{\pgfqpoint{0.819179in}{1.629259in}}%
\pgfpathcurveto{\pgfqpoint{0.819179in}{1.637495in}}{\pgfqpoint{0.815906in}{1.645395in}}{\pgfqpoint{0.810082in}{1.651219in}}%
\pgfpathcurveto{\pgfqpoint{0.804258in}{1.657043in}}{\pgfqpoint{0.796358in}{1.660316in}}{\pgfqpoint{0.788122in}{1.660316in}}%
\pgfpathcurveto{\pgfqpoint{0.779886in}{1.660316in}}{\pgfqpoint{0.771986in}{1.657043in}}{\pgfqpoint{0.766162in}{1.651219in}}%
\pgfpathcurveto{\pgfqpoint{0.760338in}{1.645395in}}{\pgfqpoint{0.757066in}{1.637495in}}{\pgfqpoint{0.757066in}{1.629259in}}%
\pgfpathcurveto{\pgfqpoint{0.757066in}{1.621023in}}{\pgfqpoint{0.760338in}{1.613123in}}{\pgfqpoint{0.766162in}{1.607299in}}%
\pgfpathcurveto{\pgfqpoint{0.771986in}{1.601475in}}{\pgfqpoint{0.779886in}{1.598203in}}{\pgfqpoint{0.788122in}{1.598203in}}%
\pgfpathclose%
\pgfusepath{stroke,fill}%
\end{pgfscope}%
\begin{pgfscope}%
\pgfpathrectangle{\pgfqpoint{0.100000in}{0.212622in}}{\pgfqpoint{3.696000in}{3.696000in}}%
\pgfusepath{clip}%
\pgfsetbuttcap%
\pgfsetroundjoin%
\definecolor{currentfill}{rgb}{0.121569,0.466667,0.705882}%
\pgfsetfillcolor{currentfill}%
\pgfsetfillopacity{0.595259}%
\pgfsetlinewidth{1.003750pt}%
\definecolor{currentstroke}{rgb}{0.121569,0.466667,0.705882}%
\pgfsetstrokecolor{currentstroke}%
\pgfsetstrokeopacity{0.595259}%
\pgfsetdash{}{0pt}%
\pgfpathmoveto{\pgfqpoint{0.788032in}{1.598185in}}%
\pgfpathcurveto{\pgfqpoint{0.796268in}{1.598185in}}{\pgfqpoint{0.804168in}{1.601457in}}{\pgfqpoint{0.809992in}{1.607281in}}%
\pgfpathcurveto{\pgfqpoint{0.815816in}{1.613105in}}{\pgfqpoint{0.819089in}{1.621005in}}{\pgfqpoint{0.819089in}{1.629242in}}%
\pgfpathcurveto{\pgfqpoint{0.819089in}{1.637478in}}{\pgfqpoint{0.815816in}{1.645378in}}{\pgfqpoint{0.809992in}{1.651202in}}%
\pgfpathcurveto{\pgfqpoint{0.804168in}{1.657026in}}{\pgfqpoint{0.796268in}{1.660298in}}{\pgfqpoint{0.788032in}{1.660298in}}%
\pgfpathcurveto{\pgfqpoint{0.779796in}{1.660298in}}{\pgfqpoint{0.771896in}{1.657026in}}{\pgfqpoint{0.766072in}{1.651202in}}%
\pgfpathcurveto{\pgfqpoint{0.760248in}{1.645378in}}{\pgfqpoint{0.756976in}{1.637478in}}{\pgfqpoint{0.756976in}{1.629242in}}%
\pgfpathcurveto{\pgfqpoint{0.756976in}{1.621005in}}{\pgfqpoint{0.760248in}{1.613105in}}{\pgfqpoint{0.766072in}{1.607281in}}%
\pgfpathcurveto{\pgfqpoint{0.771896in}{1.601457in}}{\pgfqpoint{0.779796in}{1.598185in}}{\pgfqpoint{0.788032in}{1.598185in}}%
\pgfpathclose%
\pgfusepath{stroke,fill}%
\end{pgfscope}%
\begin{pgfscope}%
\pgfpathrectangle{\pgfqpoint{0.100000in}{0.212622in}}{\pgfqpoint{3.696000in}{3.696000in}}%
\pgfusepath{clip}%
\pgfsetbuttcap%
\pgfsetroundjoin%
\definecolor{currentfill}{rgb}{0.121569,0.466667,0.705882}%
\pgfsetfillcolor{currentfill}%
\pgfsetfillopacity{0.595313}%
\pgfsetlinewidth{1.003750pt}%
\definecolor{currentstroke}{rgb}{0.121569,0.466667,0.705882}%
\pgfsetstrokecolor{currentstroke}%
\pgfsetstrokeopacity{0.595313}%
\pgfsetdash{}{0pt}%
\pgfpathmoveto{\pgfqpoint{0.788026in}{1.598149in}}%
\pgfpathcurveto{\pgfqpoint{0.796263in}{1.598149in}}{\pgfqpoint{0.804163in}{1.601421in}}{\pgfqpoint{0.809987in}{1.607245in}}%
\pgfpathcurveto{\pgfqpoint{0.815811in}{1.613069in}}{\pgfqpoint{0.819083in}{1.620969in}}{\pgfqpoint{0.819083in}{1.629205in}}%
\pgfpathcurveto{\pgfqpoint{0.819083in}{1.637442in}}{\pgfqpoint{0.815811in}{1.645342in}}{\pgfqpoint{0.809987in}{1.651166in}}%
\pgfpathcurveto{\pgfqpoint{0.804163in}{1.656989in}}{\pgfqpoint{0.796263in}{1.660262in}}{\pgfqpoint{0.788026in}{1.660262in}}%
\pgfpathcurveto{\pgfqpoint{0.779790in}{1.660262in}}{\pgfqpoint{0.771890in}{1.656989in}}{\pgfqpoint{0.766066in}{1.651166in}}%
\pgfpathcurveto{\pgfqpoint{0.760242in}{1.645342in}}{\pgfqpoint{0.756970in}{1.637442in}}{\pgfqpoint{0.756970in}{1.629205in}}%
\pgfpathcurveto{\pgfqpoint{0.756970in}{1.620969in}}{\pgfqpoint{0.760242in}{1.613069in}}{\pgfqpoint{0.766066in}{1.607245in}}%
\pgfpathcurveto{\pgfqpoint{0.771890in}{1.601421in}}{\pgfqpoint{0.779790in}{1.598149in}}{\pgfqpoint{0.788026in}{1.598149in}}%
\pgfpathclose%
\pgfusepath{stroke,fill}%
\end{pgfscope}%
\begin{pgfscope}%
\pgfpathrectangle{\pgfqpoint{0.100000in}{0.212622in}}{\pgfqpoint{3.696000in}{3.696000in}}%
\pgfusepath{clip}%
\pgfsetbuttcap%
\pgfsetroundjoin%
\definecolor{currentfill}{rgb}{0.121569,0.466667,0.705882}%
\pgfsetfillcolor{currentfill}%
\pgfsetfillopacity{0.595345}%
\pgfsetlinewidth{1.003750pt}%
\definecolor{currentstroke}{rgb}{0.121569,0.466667,0.705882}%
\pgfsetstrokecolor{currentstroke}%
\pgfsetstrokeopacity{0.595345}%
\pgfsetdash{}{0pt}%
\pgfpathmoveto{\pgfqpoint{0.788046in}{1.598140in}}%
\pgfpathcurveto{\pgfqpoint{0.796282in}{1.598140in}}{\pgfqpoint{0.804182in}{1.601412in}}{\pgfqpoint{0.810006in}{1.607236in}}%
\pgfpathcurveto{\pgfqpoint{0.815830in}{1.613060in}}{\pgfqpoint{0.819102in}{1.620960in}}{\pgfqpoint{0.819102in}{1.629196in}}%
\pgfpathcurveto{\pgfqpoint{0.819102in}{1.637432in}}{\pgfqpoint{0.815830in}{1.645332in}}{\pgfqpoint{0.810006in}{1.651156in}}%
\pgfpathcurveto{\pgfqpoint{0.804182in}{1.656980in}}{\pgfqpoint{0.796282in}{1.660253in}}{\pgfqpoint{0.788046in}{1.660253in}}%
\pgfpathcurveto{\pgfqpoint{0.779809in}{1.660253in}}{\pgfqpoint{0.771909in}{1.656980in}}{\pgfqpoint{0.766085in}{1.651156in}}%
\pgfpathcurveto{\pgfqpoint{0.760261in}{1.645332in}}{\pgfqpoint{0.756989in}{1.637432in}}{\pgfqpoint{0.756989in}{1.629196in}}%
\pgfpathcurveto{\pgfqpoint{0.756989in}{1.620960in}}{\pgfqpoint{0.760261in}{1.613060in}}{\pgfqpoint{0.766085in}{1.607236in}}%
\pgfpathcurveto{\pgfqpoint{0.771909in}{1.601412in}}{\pgfqpoint{0.779809in}{1.598140in}}{\pgfqpoint{0.788046in}{1.598140in}}%
\pgfpathclose%
\pgfusepath{stroke,fill}%
\end{pgfscope}%
\begin{pgfscope}%
\pgfpathrectangle{\pgfqpoint{0.100000in}{0.212622in}}{\pgfqpoint{3.696000in}{3.696000in}}%
\pgfusepath{clip}%
\pgfsetbuttcap%
\pgfsetroundjoin%
\definecolor{currentfill}{rgb}{0.121569,0.466667,0.705882}%
\pgfsetfillcolor{currentfill}%
\pgfsetfillopacity{0.595358}%
\pgfsetlinewidth{1.003750pt}%
\definecolor{currentstroke}{rgb}{0.121569,0.466667,0.705882}%
\pgfsetstrokecolor{currentstroke}%
\pgfsetstrokeopacity{0.595358}%
\pgfsetdash{}{0pt}%
\pgfpathmoveto{\pgfqpoint{0.788077in}{1.598105in}}%
\pgfpathcurveto{\pgfqpoint{0.796314in}{1.598105in}}{\pgfqpoint{0.804214in}{1.601377in}}{\pgfqpoint{0.810038in}{1.607201in}}%
\pgfpathcurveto{\pgfqpoint{0.815861in}{1.613025in}}{\pgfqpoint{0.819134in}{1.620925in}}{\pgfqpoint{0.819134in}{1.629161in}}%
\pgfpathcurveto{\pgfqpoint{0.819134in}{1.637397in}}{\pgfqpoint{0.815861in}{1.645297in}}{\pgfqpoint{0.810038in}{1.651121in}}%
\pgfpathcurveto{\pgfqpoint{0.804214in}{1.656945in}}{\pgfqpoint{0.796314in}{1.660218in}}{\pgfqpoint{0.788077in}{1.660218in}}%
\pgfpathcurveto{\pgfqpoint{0.779841in}{1.660218in}}{\pgfqpoint{0.771941in}{1.656945in}}{\pgfqpoint{0.766117in}{1.651121in}}%
\pgfpathcurveto{\pgfqpoint{0.760293in}{1.645297in}}{\pgfqpoint{0.757021in}{1.637397in}}{\pgfqpoint{0.757021in}{1.629161in}}%
\pgfpathcurveto{\pgfqpoint{0.757021in}{1.620925in}}{\pgfqpoint{0.760293in}{1.613025in}}{\pgfqpoint{0.766117in}{1.607201in}}%
\pgfpathcurveto{\pgfqpoint{0.771941in}{1.601377in}}{\pgfqpoint{0.779841in}{1.598105in}}{\pgfqpoint{0.788077in}{1.598105in}}%
\pgfpathclose%
\pgfusepath{stroke,fill}%
\end{pgfscope}%
\begin{pgfscope}%
\pgfpathrectangle{\pgfqpoint{0.100000in}{0.212622in}}{\pgfqpoint{3.696000in}{3.696000in}}%
\pgfusepath{clip}%
\pgfsetbuttcap%
\pgfsetroundjoin%
\definecolor{currentfill}{rgb}{0.121569,0.466667,0.705882}%
\pgfsetfillcolor{currentfill}%
\pgfsetfillopacity{0.595366}%
\pgfsetlinewidth{1.003750pt}%
\definecolor{currentstroke}{rgb}{0.121569,0.466667,0.705882}%
\pgfsetstrokecolor{currentstroke}%
\pgfsetstrokeopacity{0.595366}%
\pgfsetdash{}{0pt}%
\pgfpathmoveto{\pgfqpoint{0.788102in}{1.598094in}}%
\pgfpathcurveto{\pgfqpoint{0.796338in}{1.598094in}}{\pgfqpoint{0.804238in}{1.601366in}}{\pgfqpoint{0.810062in}{1.607190in}}%
\pgfpathcurveto{\pgfqpoint{0.815886in}{1.613014in}}{\pgfqpoint{0.819159in}{1.620914in}}{\pgfqpoint{0.819159in}{1.629151in}}%
\pgfpathcurveto{\pgfqpoint{0.819159in}{1.637387in}}{\pgfqpoint{0.815886in}{1.645287in}}{\pgfqpoint{0.810062in}{1.651111in}}%
\pgfpathcurveto{\pgfqpoint{0.804238in}{1.656935in}}{\pgfqpoint{0.796338in}{1.660207in}}{\pgfqpoint{0.788102in}{1.660207in}}%
\pgfpathcurveto{\pgfqpoint{0.779866in}{1.660207in}}{\pgfqpoint{0.771966in}{1.656935in}}{\pgfqpoint{0.766142in}{1.651111in}}%
\pgfpathcurveto{\pgfqpoint{0.760318in}{1.645287in}}{\pgfqpoint{0.757046in}{1.637387in}}{\pgfqpoint{0.757046in}{1.629151in}}%
\pgfpathcurveto{\pgfqpoint{0.757046in}{1.620914in}}{\pgfqpoint{0.760318in}{1.613014in}}{\pgfqpoint{0.766142in}{1.607190in}}%
\pgfpathcurveto{\pgfqpoint{0.771966in}{1.601366in}}{\pgfqpoint{0.779866in}{1.598094in}}{\pgfqpoint{0.788102in}{1.598094in}}%
\pgfpathclose%
\pgfusepath{stroke,fill}%
\end{pgfscope}%
\begin{pgfscope}%
\pgfpathrectangle{\pgfqpoint{0.100000in}{0.212622in}}{\pgfqpoint{3.696000in}{3.696000in}}%
\pgfusepath{clip}%
\pgfsetbuttcap%
\pgfsetroundjoin%
\definecolor{currentfill}{rgb}{0.121569,0.466667,0.705882}%
\pgfsetfillcolor{currentfill}%
\pgfsetfillopacity{0.596019}%
\pgfsetlinewidth{1.003750pt}%
\definecolor{currentstroke}{rgb}{0.121569,0.466667,0.705882}%
\pgfsetstrokecolor{currentstroke}%
\pgfsetstrokeopacity{0.596019}%
\pgfsetdash{}{0pt}%
\pgfpathmoveto{\pgfqpoint{0.886407in}{1.633963in}}%
\pgfpathcurveto{\pgfqpoint{0.894643in}{1.633963in}}{\pgfqpoint{0.902543in}{1.637235in}}{\pgfqpoint{0.908367in}{1.643059in}}%
\pgfpathcurveto{\pgfqpoint{0.914191in}{1.648883in}}{\pgfqpoint{0.917463in}{1.656783in}}{\pgfqpoint{0.917463in}{1.665020in}}%
\pgfpathcurveto{\pgfqpoint{0.917463in}{1.673256in}}{\pgfqpoint{0.914191in}{1.681156in}}{\pgfqpoint{0.908367in}{1.686980in}}%
\pgfpathcurveto{\pgfqpoint{0.902543in}{1.692804in}}{\pgfqpoint{0.894643in}{1.696076in}}{\pgfqpoint{0.886407in}{1.696076in}}%
\pgfpathcurveto{\pgfqpoint{0.878171in}{1.696076in}}{\pgfqpoint{0.870271in}{1.692804in}}{\pgfqpoint{0.864447in}{1.686980in}}%
\pgfpathcurveto{\pgfqpoint{0.858623in}{1.681156in}}{\pgfqpoint{0.855350in}{1.673256in}}{\pgfqpoint{0.855350in}{1.665020in}}%
\pgfpathcurveto{\pgfqpoint{0.855350in}{1.656783in}}{\pgfqpoint{0.858623in}{1.648883in}}{\pgfqpoint{0.864447in}{1.643059in}}%
\pgfpathcurveto{\pgfqpoint{0.870271in}{1.637235in}}{\pgfqpoint{0.878171in}{1.633963in}}{\pgfqpoint{0.886407in}{1.633963in}}%
\pgfpathclose%
\pgfusepath{stroke,fill}%
\end{pgfscope}%
\begin{pgfscope}%
\pgfpathrectangle{\pgfqpoint{0.100000in}{0.212622in}}{\pgfqpoint{3.696000in}{3.696000in}}%
\pgfusepath{clip}%
\pgfsetbuttcap%
\pgfsetroundjoin%
\definecolor{currentfill}{rgb}{0.121569,0.466667,0.705882}%
\pgfsetfillcolor{currentfill}%
\pgfsetfillopacity{0.596445}%
\pgfsetlinewidth{1.003750pt}%
\definecolor{currentstroke}{rgb}{0.121569,0.466667,0.705882}%
\pgfsetstrokecolor{currentstroke}%
\pgfsetstrokeopacity{0.596445}%
\pgfsetdash{}{0pt}%
\pgfpathmoveto{\pgfqpoint{0.791596in}{1.597776in}}%
\pgfpathcurveto{\pgfqpoint{0.799832in}{1.597776in}}{\pgfqpoint{0.807732in}{1.601048in}}{\pgfqpoint{0.813556in}{1.606872in}}%
\pgfpathcurveto{\pgfqpoint{0.819380in}{1.612696in}}{\pgfqpoint{0.822652in}{1.620596in}}{\pgfqpoint{0.822652in}{1.628833in}}%
\pgfpathcurveto{\pgfqpoint{0.822652in}{1.637069in}}{\pgfqpoint{0.819380in}{1.644969in}}{\pgfqpoint{0.813556in}{1.650793in}}%
\pgfpathcurveto{\pgfqpoint{0.807732in}{1.656617in}}{\pgfqpoint{0.799832in}{1.659889in}}{\pgfqpoint{0.791596in}{1.659889in}}%
\pgfpathcurveto{\pgfqpoint{0.783359in}{1.659889in}}{\pgfqpoint{0.775459in}{1.656617in}}{\pgfqpoint{0.769635in}{1.650793in}}%
\pgfpathcurveto{\pgfqpoint{0.763812in}{1.644969in}}{\pgfqpoint{0.760539in}{1.637069in}}{\pgfqpoint{0.760539in}{1.628833in}}%
\pgfpathcurveto{\pgfqpoint{0.760539in}{1.620596in}}{\pgfqpoint{0.763812in}{1.612696in}}{\pgfqpoint{0.769635in}{1.606872in}}%
\pgfpathcurveto{\pgfqpoint{0.775459in}{1.601048in}}{\pgfqpoint{0.783359in}{1.597776in}}{\pgfqpoint{0.791596in}{1.597776in}}%
\pgfpathclose%
\pgfusepath{stroke,fill}%
\end{pgfscope}%
\begin{pgfscope}%
\pgfpathrectangle{\pgfqpoint{0.100000in}{0.212622in}}{\pgfqpoint{3.696000in}{3.696000in}}%
\pgfusepath{clip}%
\pgfsetbuttcap%
\pgfsetroundjoin%
\definecolor{currentfill}{rgb}{0.121569,0.466667,0.705882}%
\pgfsetfillcolor{currentfill}%
\pgfsetfillopacity{0.596820}%
\pgfsetlinewidth{1.003750pt}%
\definecolor{currentstroke}{rgb}{0.121569,0.466667,0.705882}%
\pgfsetstrokecolor{currentstroke}%
\pgfsetstrokeopacity{0.596820}%
\pgfsetdash{}{0pt}%
\pgfpathmoveto{\pgfqpoint{0.793745in}{1.596343in}}%
\pgfpathcurveto{\pgfqpoint{0.801981in}{1.596343in}}{\pgfqpoint{0.809881in}{1.599615in}}{\pgfqpoint{0.815705in}{1.605439in}}%
\pgfpathcurveto{\pgfqpoint{0.821529in}{1.611263in}}{\pgfqpoint{0.824801in}{1.619163in}}{\pgfqpoint{0.824801in}{1.627399in}}%
\pgfpathcurveto{\pgfqpoint{0.824801in}{1.635635in}}{\pgfqpoint{0.821529in}{1.643535in}}{\pgfqpoint{0.815705in}{1.649359in}}%
\pgfpathcurveto{\pgfqpoint{0.809881in}{1.655183in}}{\pgfqpoint{0.801981in}{1.658456in}}{\pgfqpoint{0.793745in}{1.658456in}}%
\pgfpathcurveto{\pgfqpoint{0.785508in}{1.658456in}}{\pgfqpoint{0.777608in}{1.655183in}}{\pgfqpoint{0.771784in}{1.649359in}}%
\pgfpathcurveto{\pgfqpoint{0.765961in}{1.643535in}}{\pgfqpoint{0.762688in}{1.635635in}}{\pgfqpoint{0.762688in}{1.627399in}}%
\pgfpathcurveto{\pgfqpoint{0.762688in}{1.619163in}}{\pgfqpoint{0.765961in}{1.611263in}}{\pgfqpoint{0.771784in}{1.605439in}}%
\pgfpathcurveto{\pgfqpoint{0.777608in}{1.599615in}}{\pgfqpoint{0.785508in}{1.596343in}}{\pgfqpoint{0.793745in}{1.596343in}}%
\pgfpathclose%
\pgfusepath{stroke,fill}%
\end{pgfscope}%
\begin{pgfscope}%
\pgfpathrectangle{\pgfqpoint{0.100000in}{0.212622in}}{\pgfqpoint{3.696000in}{3.696000in}}%
\pgfusepath{clip}%
\pgfsetbuttcap%
\pgfsetroundjoin%
\definecolor{currentfill}{rgb}{0.121569,0.466667,0.705882}%
\pgfsetfillcolor{currentfill}%
\pgfsetfillopacity{0.597834}%
\pgfsetlinewidth{1.003750pt}%
\definecolor{currentstroke}{rgb}{0.121569,0.466667,0.705882}%
\pgfsetstrokecolor{currentstroke}%
\pgfsetstrokeopacity{0.597834}%
\pgfsetdash{}{0pt}%
\pgfpathmoveto{\pgfqpoint{0.798872in}{1.596510in}}%
\pgfpathcurveto{\pgfqpoint{0.807109in}{1.596510in}}{\pgfqpoint{0.815009in}{1.599782in}}{\pgfqpoint{0.820833in}{1.605606in}}%
\pgfpathcurveto{\pgfqpoint{0.826656in}{1.611430in}}{\pgfqpoint{0.829929in}{1.619330in}}{\pgfqpoint{0.829929in}{1.627566in}}%
\pgfpathcurveto{\pgfqpoint{0.829929in}{1.635802in}}{\pgfqpoint{0.826656in}{1.643702in}}{\pgfqpoint{0.820833in}{1.649526in}}%
\pgfpathcurveto{\pgfqpoint{0.815009in}{1.655350in}}{\pgfqpoint{0.807109in}{1.658623in}}{\pgfqpoint{0.798872in}{1.658623in}}%
\pgfpathcurveto{\pgfqpoint{0.790636in}{1.658623in}}{\pgfqpoint{0.782736in}{1.655350in}}{\pgfqpoint{0.776912in}{1.649526in}}%
\pgfpathcurveto{\pgfqpoint{0.771088in}{1.643702in}}{\pgfqpoint{0.767816in}{1.635802in}}{\pgfqpoint{0.767816in}{1.627566in}}%
\pgfpathcurveto{\pgfqpoint{0.767816in}{1.619330in}}{\pgfqpoint{0.771088in}{1.611430in}}{\pgfqpoint{0.776912in}{1.605606in}}%
\pgfpathcurveto{\pgfqpoint{0.782736in}{1.599782in}}{\pgfqpoint{0.790636in}{1.596510in}}{\pgfqpoint{0.798872in}{1.596510in}}%
\pgfpathclose%
\pgfusepath{stroke,fill}%
\end{pgfscope}%
\begin{pgfscope}%
\pgfpathrectangle{\pgfqpoint{0.100000in}{0.212622in}}{\pgfqpoint{3.696000in}{3.696000in}}%
\pgfusepath{clip}%
\pgfsetbuttcap%
\pgfsetroundjoin%
\definecolor{currentfill}{rgb}{0.121569,0.466667,0.705882}%
\pgfsetfillcolor{currentfill}%
\pgfsetfillopacity{0.598528}%
\pgfsetlinewidth{1.003750pt}%
\definecolor{currentstroke}{rgb}{0.121569,0.466667,0.705882}%
\pgfsetstrokecolor{currentstroke}%
\pgfsetstrokeopacity{0.598528}%
\pgfsetdash{}{0pt}%
\pgfpathmoveto{\pgfqpoint{0.865675in}{1.624861in}}%
\pgfpathcurveto{\pgfqpoint{0.873912in}{1.624861in}}{\pgfqpoint{0.881812in}{1.628133in}}{\pgfqpoint{0.887636in}{1.633957in}}%
\pgfpathcurveto{\pgfqpoint{0.893459in}{1.639781in}}{\pgfqpoint{0.896732in}{1.647681in}}{\pgfqpoint{0.896732in}{1.655917in}}%
\pgfpathcurveto{\pgfqpoint{0.896732in}{1.664153in}}{\pgfqpoint{0.893459in}{1.672053in}}{\pgfqpoint{0.887636in}{1.677877in}}%
\pgfpathcurveto{\pgfqpoint{0.881812in}{1.683701in}}{\pgfqpoint{0.873912in}{1.686974in}}{\pgfqpoint{0.865675in}{1.686974in}}%
\pgfpathcurveto{\pgfqpoint{0.857439in}{1.686974in}}{\pgfqpoint{0.849539in}{1.683701in}}{\pgfqpoint{0.843715in}{1.677877in}}%
\pgfpathcurveto{\pgfqpoint{0.837891in}{1.672053in}}{\pgfqpoint{0.834619in}{1.664153in}}{\pgfqpoint{0.834619in}{1.655917in}}%
\pgfpathcurveto{\pgfqpoint{0.834619in}{1.647681in}}{\pgfqpoint{0.837891in}{1.639781in}}{\pgfqpoint{0.843715in}{1.633957in}}%
\pgfpathcurveto{\pgfqpoint{0.849539in}{1.628133in}}{\pgfqpoint{0.857439in}{1.624861in}}{\pgfqpoint{0.865675in}{1.624861in}}%
\pgfpathclose%
\pgfusepath{stroke,fill}%
\end{pgfscope}%
\begin{pgfscope}%
\pgfpathrectangle{\pgfqpoint{0.100000in}{0.212622in}}{\pgfqpoint{3.696000in}{3.696000in}}%
\pgfusepath{clip}%
\pgfsetbuttcap%
\pgfsetroundjoin%
\definecolor{currentfill}{rgb}{0.121569,0.466667,0.705882}%
\pgfsetfillcolor{currentfill}%
\pgfsetfillopacity{0.599261}%
\pgfsetlinewidth{1.003750pt}%
\definecolor{currentstroke}{rgb}{0.121569,0.466667,0.705882}%
\pgfsetstrokecolor{currentstroke}%
\pgfsetstrokeopacity{0.599261}%
\pgfsetdash{}{0pt}%
\pgfpathmoveto{\pgfqpoint{0.810266in}{1.595463in}}%
\pgfpathcurveto{\pgfqpoint{0.818502in}{1.595463in}}{\pgfqpoint{0.826402in}{1.598736in}}{\pgfqpoint{0.832226in}{1.604560in}}%
\pgfpathcurveto{\pgfqpoint{0.838050in}{1.610383in}}{\pgfqpoint{0.841323in}{1.618283in}}{\pgfqpoint{0.841323in}{1.626520in}}%
\pgfpathcurveto{\pgfqpoint{0.841323in}{1.634756in}}{\pgfqpoint{0.838050in}{1.642656in}}{\pgfqpoint{0.832226in}{1.648480in}}%
\pgfpathcurveto{\pgfqpoint{0.826402in}{1.654304in}}{\pgfqpoint{0.818502in}{1.657576in}}{\pgfqpoint{0.810266in}{1.657576in}}%
\pgfpathcurveto{\pgfqpoint{0.802030in}{1.657576in}}{\pgfqpoint{0.794130in}{1.654304in}}{\pgfqpoint{0.788306in}{1.648480in}}%
\pgfpathcurveto{\pgfqpoint{0.782482in}{1.642656in}}{\pgfqpoint{0.779210in}{1.634756in}}{\pgfqpoint{0.779210in}{1.626520in}}%
\pgfpathcurveto{\pgfqpoint{0.779210in}{1.618283in}}{\pgfqpoint{0.782482in}{1.610383in}}{\pgfqpoint{0.788306in}{1.604560in}}%
\pgfpathcurveto{\pgfqpoint{0.794130in}{1.598736in}}{\pgfqpoint{0.802030in}{1.595463in}}{\pgfqpoint{0.810266in}{1.595463in}}%
\pgfpathclose%
\pgfusepath{stroke,fill}%
\end{pgfscope}%
\begin{pgfscope}%
\pgfpathrectangle{\pgfqpoint{0.100000in}{0.212622in}}{\pgfqpoint{3.696000in}{3.696000in}}%
\pgfusepath{clip}%
\pgfsetbuttcap%
\pgfsetroundjoin%
\definecolor{currentfill}{rgb}{0.121569,0.466667,0.705882}%
\pgfsetfillcolor{currentfill}%
\pgfsetfillopacity{0.600204}%
\pgfsetlinewidth{1.003750pt}%
\definecolor{currentstroke}{rgb}{0.121569,0.466667,0.705882}%
\pgfsetstrokecolor{currentstroke}%
\pgfsetstrokeopacity{0.600204}%
\pgfsetdash{}{0pt}%
\pgfpathmoveto{\pgfqpoint{0.816662in}{1.598681in}}%
\pgfpathcurveto{\pgfqpoint{0.824898in}{1.598681in}}{\pgfqpoint{0.832798in}{1.601953in}}{\pgfqpoint{0.838622in}{1.607777in}}%
\pgfpathcurveto{\pgfqpoint{0.844446in}{1.613601in}}{\pgfqpoint{0.847718in}{1.621501in}}{\pgfqpoint{0.847718in}{1.629737in}}%
\pgfpathcurveto{\pgfqpoint{0.847718in}{1.637974in}}{\pgfqpoint{0.844446in}{1.645874in}}{\pgfqpoint{0.838622in}{1.651698in}}%
\pgfpathcurveto{\pgfqpoint{0.832798in}{1.657521in}}{\pgfqpoint{0.824898in}{1.660794in}}{\pgfqpoint{0.816662in}{1.660794in}}%
\pgfpathcurveto{\pgfqpoint{0.808425in}{1.660794in}}{\pgfqpoint{0.800525in}{1.657521in}}{\pgfqpoint{0.794701in}{1.651698in}}%
\pgfpathcurveto{\pgfqpoint{0.788877in}{1.645874in}}{\pgfqpoint{0.785605in}{1.637974in}}{\pgfqpoint{0.785605in}{1.629737in}}%
\pgfpathcurveto{\pgfqpoint{0.785605in}{1.621501in}}{\pgfqpoint{0.788877in}{1.613601in}}{\pgfqpoint{0.794701in}{1.607777in}}%
\pgfpathcurveto{\pgfqpoint{0.800525in}{1.601953in}}{\pgfqpoint{0.808425in}{1.598681in}}{\pgfqpoint{0.816662in}{1.598681in}}%
\pgfpathclose%
\pgfusepath{stroke,fill}%
\end{pgfscope}%
\begin{pgfscope}%
\pgfpathrectangle{\pgfqpoint{0.100000in}{0.212622in}}{\pgfqpoint{3.696000in}{3.696000in}}%
\pgfusepath{clip}%
\pgfsetbuttcap%
\pgfsetroundjoin%
\definecolor{currentfill}{rgb}{0.121569,0.466667,0.705882}%
\pgfsetfillcolor{currentfill}%
\pgfsetfillopacity{0.600463}%
\pgfsetlinewidth{1.003750pt}%
\definecolor{currentstroke}{rgb}{0.121569,0.466667,0.705882}%
\pgfsetstrokecolor{currentstroke}%
\pgfsetstrokeopacity{0.600463}%
\pgfsetdash{}{0pt}%
\pgfpathmoveto{\pgfqpoint{0.820193in}{1.599410in}}%
\pgfpathcurveto{\pgfqpoint{0.828430in}{1.599410in}}{\pgfqpoint{0.836330in}{1.602682in}}{\pgfqpoint{0.842154in}{1.608506in}}%
\pgfpathcurveto{\pgfqpoint{0.847978in}{1.614330in}}{\pgfqpoint{0.851250in}{1.622230in}}{\pgfqpoint{0.851250in}{1.630466in}}%
\pgfpathcurveto{\pgfqpoint{0.851250in}{1.638703in}}{\pgfqpoint{0.847978in}{1.646603in}}{\pgfqpoint{0.842154in}{1.652427in}}%
\pgfpathcurveto{\pgfqpoint{0.836330in}{1.658250in}}{\pgfqpoint{0.828430in}{1.661523in}}{\pgfqpoint{0.820193in}{1.661523in}}%
\pgfpathcurveto{\pgfqpoint{0.811957in}{1.661523in}}{\pgfqpoint{0.804057in}{1.658250in}}{\pgfqpoint{0.798233in}{1.652427in}}%
\pgfpathcurveto{\pgfqpoint{0.792409in}{1.646603in}}{\pgfqpoint{0.789137in}{1.638703in}}{\pgfqpoint{0.789137in}{1.630466in}}%
\pgfpathcurveto{\pgfqpoint{0.789137in}{1.622230in}}{\pgfqpoint{0.792409in}{1.614330in}}{\pgfqpoint{0.798233in}{1.608506in}}%
\pgfpathcurveto{\pgfqpoint{0.804057in}{1.602682in}}{\pgfqpoint{0.811957in}{1.599410in}}{\pgfqpoint{0.820193in}{1.599410in}}%
\pgfpathclose%
\pgfusepath{stroke,fill}%
\end{pgfscope}%
\begin{pgfscope}%
\pgfpathrectangle{\pgfqpoint{0.100000in}{0.212622in}}{\pgfqpoint{3.696000in}{3.696000in}}%
\pgfusepath{clip}%
\pgfsetbuttcap%
\pgfsetroundjoin%
\definecolor{currentfill}{rgb}{0.121569,0.466667,0.705882}%
\pgfsetfillcolor{currentfill}%
\pgfsetfillopacity{0.600535}%
\pgfsetlinewidth{1.003750pt}%
\definecolor{currentstroke}{rgb}{0.121569,0.466667,0.705882}%
\pgfsetstrokecolor{currentstroke}%
\pgfsetstrokeopacity{0.600535}%
\pgfsetdash{}{0pt}%
\pgfpathmoveto{\pgfqpoint{0.851778in}{1.612619in}}%
\pgfpathcurveto{\pgfqpoint{0.860014in}{1.612619in}}{\pgfqpoint{0.867914in}{1.615891in}}{\pgfqpoint{0.873738in}{1.621715in}}%
\pgfpathcurveto{\pgfqpoint{0.879562in}{1.627539in}}{\pgfqpoint{0.882834in}{1.635439in}}{\pgfqpoint{0.882834in}{1.643676in}}%
\pgfpathcurveto{\pgfqpoint{0.882834in}{1.651912in}}{\pgfqpoint{0.879562in}{1.659812in}}{\pgfqpoint{0.873738in}{1.665636in}}%
\pgfpathcurveto{\pgfqpoint{0.867914in}{1.671460in}}{\pgfqpoint{0.860014in}{1.674732in}}{\pgfqpoint{0.851778in}{1.674732in}}%
\pgfpathcurveto{\pgfqpoint{0.843541in}{1.674732in}}{\pgfqpoint{0.835641in}{1.671460in}}{\pgfqpoint{0.829817in}{1.665636in}}%
\pgfpathcurveto{\pgfqpoint{0.823993in}{1.659812in}}{\pgfqpoint{0.820721in}{1.651912in}}{\pgfqpoint{0.820721in}{1.643676in}}%
\pgfpathcurveto{\pgfqpoint{0.820721in}{1.635439in}}{\pgfqpoint{0.823993in}{1.627539in}}{\pgfqpoint{0.829817in}{1.621715in}}%
\pgfpathcurveto{\pgfqpoint{0.835641in}{1.615891in}}{\pgfqpoint{0.843541in}{1.612619in}}{\pgfqpoint{0.851778in}{1.612619in}}%
\pgfpathclose%
\pgfusepath{stroke,fill}%
\end{pgfscope}%
\begin{pgfscope}%
\pgfpathrectangle{\pgfqpoint{0.100000in}{0.212622in}}{\pgfqpoint{3.696000in}{3.696000in}}%
\pgfusepath{clip}%
\pgfsetbuttcap%
\pgfsetroundjoin%
\definecolor{currentfill}{rgb}{0.121569,0.466667,0.705882}%
\pgfsetfillcolor{currentfill}%
\pgfsetfillopacity{0.600721}%
\pgfsetlinewidth{1.003750pt}%
\definecolor{currentstroke}{rgb}{0.121569,0.466667,0.705882}%
\pgfsetstrokecolor{currentstroke}%
\pgfsetstrokeopacity{0.600721}%
\pgfsetdash{}{0pt}%
\pgfpathmoveto{\pgfqpoint{0.834451in}{1.607343in}}%
\pgfpathcurveto{\pgfqpoint{0.842688in}{1.607343in}}{\pgfqpoint{0.850588in}{1.610615in}}{\pgfqpoint{0.856412in}{1.616439in}}%
\pgfpathcurveto{\pgfqpoint{0.862235in}{1.622263in}}{\pgfqpoint{0.865508in}{1.630163in}}{\pgfqpoint{0.865508in}{1.638399in}}%
\pgfpathcurveto{\pgfqpoint{0.865508in}{1.646635in}}{\pgfqpoint{0.862235in}{1.654536in}}{\pgfqpoint{0.856412in}{1.660359in}}%
\pgfpathcurveto{\pgfqpoint{0.850588in}{1.666183in}}{\pgfqpoint{0.842688in}{1.669456in}}{\pgfqpoint{0.834451in}{1.669456in}}%
\pgfpathcurveto{\pgfqpoint{0.826215in}{1.669456in}}{\pgfqpoint{0.818315in}{1.666183in}}{\pgfqpoint{0.812491in}{1.660359in}}%
\pgfpathcurveto{\pgfqpoint{0.806667in}{1.654536in}}{\pgfqpoint{0.803395in}{1.646635in}}{\pgfqpoint{0.803395in}{1.638399in}}%
\pgfpathcurveto{\pgfqpoint{0.803395in}{1.630163in}}{\pgfqpoint{0.806667in}{1.622263in}}{\pgfqpoint{0.812491in}{1.616439in}}%
\pgfpathcurveto{\pgfqpoint{0.818315in}{1.610615in}}{\pgfqpoint{0.826215in}{1.607343in}}{\pgfqpoint{0.834451in}{1.607343in}}%
\pgfpathclose%
\pgfusepath{stroke,fill}%
\end{pgfscope}%
\begin{pgfscope}%
\pgfpathrectangle{\pgfqpoint{0.100000in}{0.212622in}}{\pgfqpoint{3.696000in}{3.696000in}}%
\pgfusepath{clip}%
\pgfsetbuttcap%
\pgfsetroundjoin%
\definecolor{currentfill}{rgb}{0.121569,0.466667,0.705882}%
\pgfsetfillcolor{currentfill}%
\pgfsetfillopacity{0.603634}%
\pgfsetlinewidth{1.003750pt}%
\definecolor{currentstroke}{rgb}{0.121569,0.466667,0.705882}%
\pgfsetstrokecolor{currentstroke}%
\pgfsetstrokeopacity{0.603634}%
\pgfsetdash{}{0pt}%
\pgfpathmoveto{\pgfqpoint{1.996525in}{1.930068in}}%
\pgfpathcurveto{\pgfqpoint{2.004762in}{1.930068in}}{\pgfqpoint{2.012662in}{1.933340in}}{\pgfqpoint{2.018486in}{1.939164in}}%
\pgfpathcurveto{\pgfqpoint{2.024310in}{1.944988in}}{\pgfqpoint{2.027582in}{1.952888in}}{\pgfqpoint{2.027582in}{1.961124in}}%
\pgfpathcurveto{\pgfqpoint{2.027582in}{1.969360in}}{\pgfqpoint{2.024310in}{1.977260in}}{\pgfqpoint{2.018486in}{1.983084in}}%
\pgfpathcurveto{\pgfqpoint{2.012662in}{1.988908in}}{\pgfqpoint{2.004762in}{1.992181in}}{\pgfqpoint{1.996525in}{1.992181in}}%
\pgfpathcurveto{\pgfqpoint{1.988289in}{1.992181in}}{\pgfqpoint{1.980389in}{1.988908in}}{\pgfqpoint{1.974565in}{1.983084in}}%
\pgfpathcurveto{\pgfqpoint{1.968741in}{1.977260in}}{\pgfqpoint{1.965469in}{1.969360in}}{\pgfqpoint{1.965469in}{1.961124in}}%
\pgfpathcurveto{\pgfqpoint{1.965469in}{1.952888in}}{\pgfqpoint{1.968741in}{1.944988in}}{\pgfqpoint{1.974565in}{1.939164in}}%
\pgfpathcurveto{\pgfqpoint{1.980389in}{1.933340in}}{\pgfqpoint{1.988289in}{1.930068in}}{\pgfqpoint{1.996525in}{1.930068in}}%
\pgfpathclose%
\pgfusepath{stroke,fill}%
\end{pgfscope}%
\begin{pgfscope}%
\pgfpathrectangle{\pgfqpoint{0.100000in}{0.212622in}}{\pgfqpoint{3.696000in}{3.696000in}}%
\pgfusepath{clip}%
\pgfsetbuttcap%
\pgfsetroundjoin%
\definecolor{currentfill}{rgb}{0.121569,0.466667,0.705882}%
\pgfsetfillcolor{currentfill}%
\pgfsetfillopacity{0.620298}%
\pgfsetlinewidth{1.003750pt}%
\definecolor{currentstroke}{rgb}{0.121569,0.466667,0.705882}%
\pgfsetstrokecolor{currentstroke}%
\pgfsetstrokeopacity{0.620298}%
\pgfsetdash{}{0pt}%
\pgfpathmoveto{\pgfqpoint{2.027294in}{1.939710in}}%
\pgfpathcurveto{\pgfqpoint{2.035531in}{1.939710in}}{\pgfqpoint{2.043431in}{1.942982in}}{\pgfqpoint{2.049255in}{1.948806in}}%
\pgfpathcurveto{\pgfqpoint{2.055079in}{1.954630in}}{\pgfqpoint{2.058351in}{1.962530in}}{\pgfqpoint{2.058351in}{1.970766in}}%
\pgfpathcurveto{\pgfqpoint{2.058351in}{1.979003in}}{\pgfqpoint{2.055079in}{1.986903in}}{\pgfqpoint{2.049255in}{1.992727in}}%
\pgfpathcurveto{\pgfqpoint{2.043431in}{1.998551in}}{\pgfqpoint{2.035531in}{2.001823in}}{\pgfqpoint{2.027294in}{2.001823in}}%
\pgfpathcurveto{\pgfqpoint{2.019058in}{2.001823in}}{\pgfqpoint{2.011158in}{1.998551in}}{\pgfqpoint{2.005334in}{1.992727in}}%
\pgfpathcurveto{\pgfqpoint{1.999510in}{1.986903in}}{\pgfqpoint{1.996238in}{1.979003in}}{\pgfqpoint{1.996238in}{1.970766in}}%
\pgfpathcurveto{\pgfqpoint{1.996238in}{1.962530in}}{\pgfqpoint{1.999510in}{1.954630in}}{\pgfqpoint{2.005334in}{1.948806in}}%
\pgfpathcurveto{\pgfqpoint{2.011158in}{1.942982in}}{\pgfqpoint{2.019058in}{1.939710in}}{\pgfqpoint{2.027294in}{1.939710in}}%
\pgfpathclose%
\pgfusepath{stroke,fill}%
\end{pgfscope}%
\begin{pgfscope}%
\pgfpathrectangle{\pgfqpoint{0.100000in}{0.212622in}}{\pgfqpoint{3.696000in}{3.696000in}}%
\pgfusepath{clip}%
\pgfsetbuttcap%
\pgfsetroundjoin%
\definecolor{currentfill}{rgb}{0.121569,0.466667,0.705882}%
\pgfsetfillcolor{currentfill}%
\pgfsetfillopacity{0.633339}%
\pgfsetlinewidth{1.003750pt}%
\definecolor{currentstroke}{rgb}{0.121569,0.466667,0.705882}%
\pgfsetstrokecolor{currentstroke}%
\pgfsetstrokeopacity{0.633339}%
\pgfsetdash{}{0pt}%
\pgfpathmoveto{\pgfqpoint{2.039648in}{1.914272in}}%
\pgfpathcurveto{\pgfqpoint{2.047884in}{1.914272in}}{\pgfqpoint{2.055784in}{1.917545in}}{\pgfqpoint{2.061608in}{1.923369in}}%
\pgfpathcurveto{\pgfqpoint{2.067432in}{1.929193in}}{\pgfqpoint{2.070704in}{1.937093in}}{\pgfqpoint{2.070704in}{1.945329in}}%
\pgfpathcurveto{\pgfqpoint{2.070704in}{1.953565in}}{\pgfqpoint{2.067432in}{1.961465in}}{\pgfqpoint{2.061608in}{1.967289in}}%
\pgfpathcurveto{\pgfqpoint{2.055784in}{1.973113in}}{\pgfqpoint{2.047884in}{1.976385in}}{\pgfqpoint{2.039648in}{1.976385in}}%
\pgfpathcurveto{\pgfqpoint{2.031411in}{1.976385in}}{\pgfqpoint{2.023511in}{1.973113in}}{\pgfqpoint{2.017687in}{1.967289in}}%
\pgfpathcurveto{\pgfqpoint{2.011863in}{1.961465in}}{\pgfqpoint{2.008591in}{1.953565in}}{\pgfqpoint{2.008591in}{1.945329in}}%
\pgfpathcurveto{\pgfqpoint{2.008591in}{1.937093in}}{\pgfqpoint{2.011863in}{1.929193in}}{\pgfqpoint{2.017687in}{1.923369in}}%
\pgfpathcurveto{\pgfqpoint{2.023511in}{1.917545in}}{\pgfqpoint{2.031411in}{1.914272in}}{\pgfqpoint{2.039648in}{1.914272in}}%
\pgfpathclose%
\pgfusepath{stroke,fill}%
\end{pgfscope}%
\begin{pgfscope}%
\pgfpathrectangle{\pgfqpoint{0.100000in}{0.212622in}}{\pgfqpoint{3.696000in}{3.696000in}}%
\pgfusepath{clip}%
\pgfsetbuttcap%
\pgfsetroundjoin%
\definecolor{currentfill}{rgb}{0.121569,0.466667,0.705882}%
\pgfsetfillcolor{currentfill}%
\pgfsetfillopacity{0.642723}%
\pgfsetlinewidth{1.003750pt}%
\definecolor{currentstroke}{rgb}{0.121569,0.466667,0.705882}%
\pgfsetstrokecolor{currentstroke}%
\pgfsetstrokeopacity{0.642723}%
\pgfsetdash{}{0pt}%
\pgfpathmoveto{\pgfqpoint{2.056371in}{1.914502in}}%
\pgfpathcurveto{\pgfqpoint{2.064608in}{1.914502in}}{\pgfqpoint{2.072508in}{1.917774in}}{\pgfqpoint{2.078332in}{1.923598in}}%
\pgfpathcurveto{\pgfqpoint{2.084156in}{1.929422in}}{\pgfqpoint{2.087428in}{1.937322in}}{\pgfqpoint{2.087428in}{1.945558in}}%
\pgfpathcurveto{\pgfqpoint{2.087428in}{1.953794in}}{\pgfqpoint{2.084156in}{1.961694in}}{\pgfqpoint{2.078332in}{1.967518in}}%
\pgfpathcurveto{\pgfqpoint{2.072508in}{1.973342in}}{\pgfqpoint{2.064608in}{1.976615in}}{\pgfqpoint{2.056371in}{1.976615in}}%
\pgfpathcurveto{\pgfqpoint{2.048135in}{1.976615in}}{\pgfqpoint{2.040235in}{1.973342in}}{\pgfqpoint{2.034411in}{1.967518in}}%
\pgfpathcurveto{\pgfqpoint{2.028587in}{1.961694in}}{\pgfqpoint{2.025315in}{1.953794in}}{\pgfqpoint{2.025315in}{1.945558in}}%
\pgfpathcurveto{\pgfqpoint{2.025315in}{1.937322in}}{\pgfqpoint{2.028587in}{1.929422in}}{\pgfqpoint{2.034411in}{1.923598in}}%
\pgfpathcurveto{\pgfqpoint{2.040235in}{1.917774in}}{\pgfqpoint{2.048135in}{1.914502in}}{\pgfqpoint{2.056371in}{1.914502in}}%
\pgfpathclose%
\pgfusepath{stroke,fill}%
\end{pgfscope}%
\begin{pgfscope}%
\pgfpathrectangle{\pgfqpoint{0.100000in}{0.212622in}}{\pgfqpoint{3.696000in}{3.696000in}}%
\pgfusepath{clip}%
\pgfsetbuttcap%
\pgfsetroundjoin%
\definecolor{currentfill}{rgb}{0.121569,0.466667,0.705882}%
\pgfsetfillcolor{currentfill}%
\pgfsetfillopacity{0.651887}%
\pgfsetlinewidth{1.003750pt}%
\definecolor{currentstroke}{rgb}{0.121569,0.466667,0.705882}%
\pgfsetstrokecolor{currentstroke}%
\pgfsetstrokeopacity{0.651887}%
\pgfsetdash{}{0pt}%
\pgfpathmoveto{\pgfqpoint{2.062831in}{1.898534in}}%
\pgfpathcurveto{\pgfqpoint{2.071067in}{1.898534in}}{\pgfqpoint{2.078967in}{1.901806in}}{\pgfqpoint{2.084791in}{1.907630in}}%
\pgfpathcurveto{\pgfqpoint{2.090615in}{1.913454in}}{\pgfqpoint{2.093887in}{1.921354in}}{\pgfqpoint{2.093887in}{1.929590in}}%
\pgfpathcurveto{\pgfqpoint{2.093887in}{1.937827in}}{\pgfqpoint{2.090615in}{1.945727in}}{\pgfqpoint{2.084791in}{1.951551in}}%
\pgfpathcurveto{\pgfqpoint{2.078967in}{1.957374in}}{\pgfqpoint{2.071067in}{1.960647in}}{\pgfqpoint{2.062831in}{1.960647in}}%
\pgfpathcurveto{\pgfqpoint{2.054595in}{1.960647in}}{\pgfqpoint{2.046695in}{1.957374in}}{\pgfqpoint{2.040871in}{1.951551in}}%
\pgfpathcurveto{\pgfqpoint{2.035047in}{1.945727in}}{\pgfqpoint{2.031774in}{1.937827in}}{\pgfqpoint{2.031774in}{1.929590in}}%
\pgfpathcurveto{\pgfqpoint{2.031774in}{1.921354in}}{\pgfqpoint{2.035047in}{1.913454in}}{\pgfqpoint{2.040871in}{1.907630in}}%
\pgfpathcurveto{\pgfqpoint{2.046695in}{1.901806in}}{\pgfqpoint{2.054595in}{1.898534in}}{\pgfqpoint{2.062831in}{1.898534in}}%
\pgfpathclose%
\pgfusepath{stroke,fill}%
\end{pgfscope}%
\begin{pgfscope}%
\pgfpathrectangle{\pgfqpoint{0.100000in}{0.212622in}}{\pgfqpoint{3.696000in}{3.696000in}}%
\pgfusepath{clip}%
\pgfsetbuttcap%
\pgfsetroundjoin%
\definecolor{currentfill}{rgb}{0.121569,0.466667,0.705882}%
\pgfsetfillcolor{currentfill}%
\pgfsetfillopacity{0.658992}%
\pgfsetlinewidth{1.003750pt}%
\definecolor{currentstroke}{rgb}{0.121569,0.466667,0.705882}%
\pgfsetstrokecolor{currentstroke}%
\pgfsetstrokeopacity{0.658992}%
\pgfsetdash{}{0pt}%
\pgfpathmoveto{\pgfqpoint{2.075924in}{1.903261in}}%
\pgfpathcurveto{\pgfqpoint{2.084160in}{1.903261in}}{\pgfqpoint{2.092060in}{1.906533in}}{\pgfqpoint{2.097884in}{1.912357in}}%
\pgfpathcurveto{\pgfqpoint{2.103708in}{1.918181in}}{\pgfqpoint{2.106980in}{1.926081in}}{\pgfqpoint{2.106980in}{1.934317in}}%
\pgfpathcurveto{\pgfqpoint{2.106980in}{1.942554in}}{\pgfqpoint{2.103708in}{1.950454in}}{\pgfqpoint{2.097884in}{1.956278in}}%
\pgfpathcurveto{\pgfqpoint{2.092060in}{1.962102in}}{\pgfqpoint{2.084160in}{1.965374in}}{\pgfqpoint{2.075924in}{1.965374in}}%
\pgfpathcurveto{\pgfqpoint{2.067687in}{1.965374in}}{\pgfqpoint{2.059787in}{1.962102in}}{\pgfqpoint{2.053963in}{1.956278in}}%
\pgfpathcurveto{\pgfqpoint{2.048139in}{1.950454in}}{\pgfqpoint{2.044867in}{1.942554in}}{\pgfqpoint{2.044867in}{1.934317in}}%
\pgfpathcurveto{\pgfqpoint{2.044867in}{1.926081in}}{\pgfqpoint{2.048139in}{1.918181in}}{\pgfqpoint{2.053963in}{1.912357in}}%
\pgfpathcurveto{\pgfqpoint{2.059787in}{1.906533in}}{\pgfqpoint{2.067687in}{1.903261in}}{\pgfqpoint{2.075924in}{1.903261in}}%
\pgfpathclose%
\pgfusepath{stroke,fill}%
\end{pgfscope}%
\begin{pgfscope}%
\pgfpathrectangle{\pgfqpoint{0.100000in}{0.212622in}}{\pgfqpoint{3.696000in}{3.696000in}}%
\pgfusepath{clip}%
\pgfsetbuttcap%
\pgfsetroundjoin%
\definecolor{currentfill}{rgb}{0.121569,0.466667,0.705882}%
\pgfsetfillcolor{currentfill}%
\pgfsetfillopacity{0.667086}%
\pgfsetlinewidth{1.003750pt}%
\definecolor{currentstroke}{rgb}{0.121569,0.466667,0.705882}%
\pgfsetstrokecolor{currentstroke}%
\pgfsetstrokeopacity{0.667086}%
\pgfsetdash{}{0pt}%
\pgfpathmoveto{\pgfqpoint{2.080003in}{1.896829in}}%
\pgfpathcurveto{\pgfqpoint{2.088239in}{1.896829in}}{\pgfqpoint{2.096139in}{1.900102in}}{\pgfqpoint{2.101963in}{1.905926in}}%
\pgfpathcurveto{\pgfqpoint{2.107787in}{1.911750in}}{\pgfqpoint{2.111060in}{1.919650in}}{\pgfqpoint{2.111060in}{1.927886in}}%
\pgfpathcurveto{\pgfqpoint{2.111060in}{1.936122in}}{\pgfqpoint{2.107787in}{1.944022in}}{\pgfqpoint{2.101963in}{1.949846in}}%
\pgfpathcurveto{\pgfqpoint{2.096139in}{1.955670in}}{\pgfqpoint{2.088239in}{1.958942in}}{\pgfqpoint{2.080003in}{1.958942in}}%
\pgfpathcurveto{\pgfqpoint{2.071767in}{1.958942in}}{\pgfqpoint{2.063867in}{1.955670in}}{\pgfqpoint{2.058043in}{1.949846in}}%
\pgfpathcurveto{\pgfqpoint{2.052219in}{1.944022in}}{\pgfqpoint{2.048947in}{1.936122in}}{\pgfqpoint{2.048947in}{1.927886in}}%
\pgfpathcurveto{\pgfqpoint{2.048947in}{1.919650in}}{\pgfqpoint{2.052219in}{1.911750in}}{\pgfqpoint{2.058043in}{1.905926in}}%
\pgfpathcurveto{\pgfqpoint{2.063867in}{1.900102in}}{\pgfqpoint{2.071767in}{1.896829in}}{\pgfqpoint{2.080003in}{1.896829in}}%
\pgfpathclose%
\pgfusepath{stroke,fill}%
\end{pgfscope}%
\begin{pgfscope}%
\pgfpathrectangle{\pgfqpoint{0.100000in}{0.212622in}}{\pgfqpoint{3.696000in}{3.696000in}}%
\pgfusepath{clip}%
\pgfsetbuttcap%
\pgfsetroundjoin%
\definecolor{currentfill}{rgb}{0.121569,0.466667,0.705882}%
\pgfsetfillcolor{currentfill}%
\pgfsetfillopacity{0.679138}%
\pgfsetlinewidth{1.003750pt}%
\definecolor{currentstroke}{rgb}{0.121569,0.466667,0.705882}%
\pgfsetstrokecolor{currentstroke}%
\pgfsetstrokeopacity{0.679138}%
\pgfsetdash{}{0pt}%
\pgfpathmoveto{\pgfqpoint{2.101567in}{1.904480in}}%
\pgfpathcurveto{\pgfqpoint{2.109803in}{1.904480in}}{\pgfqpoint{2.117703in}{1.907752in}}{\pgfqpoint{2.123527in}{1.913576in}}%
\pgfpathcurveto{\pgfqpoint{2.129351in}{1.919400in}}{\pgfqpoint{2.132623in}{1.927300in}}{\pgfqpoint{2.132623in}{1.935537in}}%
\pgfpathcurveto{\pgfqpoint{2.132623in}{1.943773in}}{\pgfqpoint{2.129351in}{1.951673in}}{\pgfqpoint{2.123527in}{1.957497in}}%
\pgfpathcurveto{\pgfqpoint{2.117703in}{1.963321in}}{\pgfqpoint{2.109803in}{1.966593in}}{\pgfqpoint{2.101567in}{1.966593in}}%
\pgfpathcurveto{\pgfqpoint{2.093330in}{1.966593in}}{\pgfqpoint{2.085430in}{1.963321in}}{\pgfqpoint{2.079606in}{1.957497in}}%
\pgfpathcurveto{\pgfqpoint{2.073782in}{1.951673in}}{\pgfqpoint{2.070510in}{1.943773in}}{\pgfqpoint{2.070510in}{1.935537in}}%
\pgfpathcurveto{\pgfqpoint{2.070510in}{1.927300in}}{\pgfqpoint{2.073782in}{1.919400in}}{\pgfqpoint{2.079606in}{1.913576in}}%
\pgfpathcurveto{\pgfqpoint{2.085430in}{1.907752in}}{\pgfqpoint{2.093330in}{1.904480in}}{\pgfqpoint{2.101567in}{1.904480in}}%
\pgfpathclose%
\pgfusepath{stroke,fill}%
\end{pgfscope}%
\begin{pgfscope}%
\pgfpathrectangle{\pgfqpoint{0.100000in}{0.212622in}}{\pgfqpoint{3.696000in}{3.696000in}}%
\pgfusepath{clip}%
\pgfsetbuttcap%
\pgfsetroundjoin%
\definecolor{currentfill}{rgb}{0.121569,0.466667,0.705882}%
\pgfsetfillcolor{currentfill}%
\pgfsetfillopacity{0.689306}%
\pgfsetlinewidth{1.003750pt}%
\definecolor{currentstroke}{rgb}{0.121569,0.466667,0.705882}%
\pgfsetstrokecolor{currentstroke}%
\pgfsetstrokeopacity{0.689306}%
\pgfsetdash{}{0pt}%
\pgfpathmoveto{\pgfqpoint{2.110469in}{1.884773in}}%
\pgfpathcurveto{\pgfqpoint{2.118705in}{1.884773in}}{\pgfqpoint{2.126605in}{1.888045in}}{\pgfqpoint{2.132429in}{1.893869in}}%
\pgfpathcurveto{\pgfqpoint{2.138253in}{1.899693in}}{\pgfqpoint{2.141526in}{1.907593in}}{\pgfqpoint{2.141526in}{1.915829in}}%
\pgfpathcurveto{\pgfqpoint{2.141526in}{1.924066in}}{\pgfqpoint{2.138253in}{1.931966in}}{\pgfqpoint{2.132429in}{1.937790in}}%
\pgfpathcurveto{\pgfqpoint{2.126605in}{1.943614in}}{\pgfqpoint{2.118705in}{1.946886in}}{\pgfqpoint{2.110469in}{1.946886in}}%
\pgfpathcurveto{\pgfqpoint{2.102233in}{1.946886in}}{\pgfqpoint{2.094333in}{1.943614in}}{\pgfqpoint{2.088509in}{1.937790in}}%
\pgfpathcurveto{\pgfqpoint{2.082685in}{1.931966in}}{\pgfqpoint{2.079413in}{1.924066in}}{\pgfqpoint{2.079413in}{1.915829in}}%
\pgfpathcurveto{\pgfqpoint{2.079413in}{1.907593in}}{\pgfqpoint{2.082685in}{1.899693in}}{\pgfqpoint{2.088509in}{1.893869in}}%
\pgfpathcurveto{\pgfqpoint{2.094333in}{1.888045in}}{\pgfqpoint{2.102233in}{1.884773in}}{\pgfqpoint{2.110469in}{1.884773in}}%
\pgfpathclose%
\pgfusepath{stroke,fill}%
\end{pgfscope}%
\begin{pgfscope}%
\pgfpathrectangle{\pgfqpoint{0.100000in}{0.212622in}}{\pgfqpoint{3.696000in}{3.696000in}}%
\pgfusepath{clip}%
\pgfsetbuttcap%
\pgfsetroundjoin%
\definecolor{currentfill}{rgb}{0.121569,0.466667,0.705882}%
\pgfsetfillcolor{currentfill}%
\pgfsetfillopacity{0.704747}%
\pgfsetlinewidth{1.003750pt}%
\definecolor{currentstroke}{rgb}{0.121569,0.466667,0.705882}%
\pgfsetstrokecolor{currentstroke}%
\pgfsetstrokeopacity{0.704747}%
\pgfsetdash{}{0pt}%
\pgfpathmoveto{\pgfqpoint{2.140587in}{1.894141in}}%
\pgfpathcurveto{\pgfqpoint{2.148823in}{1.894141in}}{\pgfqpoint{2.156723in}{1.897413in}}{\pgfqpoint{2.162547in}{1.903237in}}%
\pgfpathcurveto{\pgfqpoint{2.168371in}{1.909061in}}{\pgfqpoint{2.171643in}{1.916961in}}{\pgfqpoint{2.171643in}{1.925197in}}%
\pgfpathcurveto{\pgfqpoint{2.171643in}{1.933434in}}{\pgfqpoint{2.168371in}{1.941334in}}{\pgfqpoint{2.162547in}{1.947157in}}%
\pgfpathcurveto{\pgfqpoint{2.156723in}{1.952981in}}{\pgfqpoint{2.148823in}{1.956254in}}{\pgfqpoint{2.140587in}{1.956254in}}%
\pgfpathcurveto{\pgfqpoint{2.132350in}{1.956254in}}{\pgfqpoint{2.124450in}{1.952981in}}{\pgfqpoint{2.118627in}{1.947157in}}%
\pgfpathcurveto{\pgfqpoint{2.112803in}{1.941334in}}{\pgfqpoint{2.109530in}{1.933434in}}{\pgfqpoint{2.109530in}{1.925197in}}%
\pgfpathcurveto{\pgfqpoint{2.109530in}{1.916961in}}{\pgfqpoint{2.112803in}{1.909061in}}{\pgfqpoint{2.118627in}{1.903237in}}%
\pgfpathcurveto{\pgfqpoint{2.124450in}{1.897413in}}{\pgfqpoint{2.132350in}{1.894141in}}{\pgfqpoint{2.140587in}{1.894141in}}%
\pgfpathclose%
\pgfusepath{stroke,fill}%
\end{pgfscope}%
\begin{pgfscope}%
\pgfpathrectangle{\pgfqpoint{0.100000in}{0.212622in}}{\pgfqpoint{3.696000in}{3.696000in}}%
\pgfusepath{clip}%
\pgfsetbuttcap%
\pgfsetroundjoin%
\definecolor{currentfill}{rgb}{0.121569,0.466667,0.705882}%
\pgfsetfillcolor{currentfill}%
\pgfsetfillopacity{0.717376}%
\pgfsetlinewidth{1.003750pt}%
\definecolor{currentstroke}{rgb}{0.121569,0.466667,0.705882}%
\pgfsetstrokecolor{currentstroke}%
\pgfsetstrokeopacity{0.717376}%
\pgfsetdash{}{0pt}%
\pgfpathmoveto{\pgfqpoint{2.150795in}{1.874182in}}%
\pgfpathcurveto{\pgfqpoint{2.159032in}{1.874182in}}{\pgfqpoint{2.166932in}{1.877454in}}{\pgfqpoint{2.172756in}{1.883278in}}%
\pgfpathcurveto{\pgfqpoint{2.178580in}{1.889102in}}{\pgfqpoint{2.181852in}{1.897002in}}{\pgfqpoint{2.181852in}{1.905238in}}%
\pgfpathcurveto{\pgfqpoint{2.181852in}{1.913475in}}{\pgfqpoint{2.178580in}{1.921375in}}{\pgfqpoint{2.172756in}{1.927199in}}%
\pgfpathcurveto{\pgfqpoint{2.166932in}{1.933023in}}{\pgfqpoint{2.159032in}{1.936295in}}{\pgfqpoint{2.150795in}{1.936295in}}%
\pgfpathcurveto{\pgfqpoint{2.142559in}{1.936295in}}{\pgfqpoint{2.134659in}{1.933023in}}{\pgfqpoint{2.128835in}{1.927199in}}%
\pgfpathcurveto{\pgfqpoint{2.123011in}{1.921375in}}{\pgfqpoint{2.119739in}{1.913475in}}{\pgfqpoint{2.119739in}{1.905238in}}%
\pgfpathcurveto{\pgfqpoint{2.119739in}{1.897002in}}{\pgfqpoint{2.123011in}{1.889102in}}{\pgfqpoint{2.128835in}{1.883278in}}%
\pgfpathcurveto{\pgfqpoint{2.134659in}{1.877454in}}{\pgfqpoint{2.142559in}{1.874182in}}{\pgfqpoint{2.150795in}{1.874182in}}%
\pgfpathclose%
\pgfusepath{stroke,fill}%
\end{pgfscope}%
\begin{pgfscope}%
\pgfpathrectangle{\pgfqpoint{0.100000in}{0.212622in}}{\pgfqpoint{3.696000in}{3.696000in}}%
\pgfusepath{clip}%
\pgfsetbuttcap%
\pgfsetroundjoin%
\definecolor{currentfill}{rgb}{0.121569,0.466667,0.705882}%
\pgfsetfillcolor{currentfill}%
\pgfsetfillopacity{0.726480}%
\pgfsetlinewidth{1.003750pt}%
\definecolor{currentstroke}{rgb}{0.121569,0.466667,0.705882}%
\pgfsetstrokecolor{currentstroke}%
\pgfsetstrokeopacity{0.726480}%
\pgfsetdash{}{0pt}%
\pgfpathmoveto{\pgfqpoint{2.166553in}{1.876717in}}%
\pgfpathcurveto{\pgfqpoint{2.174789in}{1.876717in}}{\pgfqpoint{2.182689in}{1.879989in}}{\pgfqpoint{2.188513in}{1.885813in}}%
\pgfpathcurveto{\pgfqpoint{2.194337in}{1.891637in}}{\pgfqpoint{2.197609in}{1.899537in}}{\pgfqpoint{2.197609in}{1.907773in}}%
\pgfpathcurveto{\pgfqpoint{2.197609in}{1.916010in}}{\pgfqpoint{2.194337in}{1.923910in}}{\pgfqpoint{2.188513in}{1.929734in}}%
\pgfpathcurveto{\pgfqpoint{2.182689in}{1.935558in}}{\pgfqpoint{2.174789in}{1.938830in}}{\pgfqpoint{2.166553in}{1.938830in}}%
\pgfpathcurveto{\pgfqpoint{2.158316in}{1.938830in}}{\pgfqpoint{2.150416in}{1.935558in}}{\pgfqpoint{2.144592in}{1.929734in}}%
\pgfpathcurveto{\pgfqpoint{2.138768in}{1.923910in}}{\pgfqpoint{2.135496in}{1.916010in}}{\pgfqpoint{2.135496in}{1.907773in}}%
\pgfpathcurveto{\pgfqpoint{2.135496in}{1.899537in}}{\pgfqpoint{2.138768in}{1.891637in}}{\pgfqpoint{2.144592in}{1.885813in}}%
\pgfpathcurveto{\pgfqpoint{2.150416in}{1.879989in}}{\pgfqpoint{2.158316in}{1.876717in}}{\pgfqpoint{2.166553in}{1.876717in}}%
\pgfpathclose%
\pgfusepath{stroke,fill}%
\end{pgfscope}%
\begin{pgfscope}%
\pgfpathrectangle{\pgfqpoint{0.100000in}{0.212622in}}{\pgfqpoint{3.696000in}{3.696000in}}%
\pgfusepath{clip}%
\pgfsetbuttcap%
\pgfsetroundjoin%
\definecolor{currentfill}{rgb}{0.121569,0.466667,0.705882}%
\pgfsetfillcolor{currentfill}%
\pgfsetfillopacity{0.734634}%
\pgfsetlinewidth{1.003750pt}%
\definecolor{currentstroke}{rgb}{0.121569,0.466667,0.705882}%
\pgfsetstrokecolor{currentstroke}%
\pgfsetstrokeopacity{0.734634}%
\pgfsetdash{}{0pt}%
\pgfpathmoveto{\pgfqpoint{2.174421in}{1.862166in}}%
\pgfpathcurveto{\pgfqpoint{2.182657in}{1.862166in}}{\pgfqpoint{2.190557in}{1.865439in}}{\pgfqpoint{2.196381in}{1.871263in}}%
\pgfpathcurveto{\pgfqpoint{2.202205in}{1.877087in}}{\pgfqpoint{2.205477in}{1.884987in}}{\pgfqpoint{2.205477in}{1.893223in}}%
\pgfpathcurveto{\pgfqpoint{2.205477in}{1.901459in}}{\pgfqpoint{2.202205in}{1.909359in}}{\pgfqpoint{2.196381in}{1.915183in}}%
\pgfpathcurveto{\pgfqpoint{2.190557in}{1.921007in}}{\pgfqpoint{2.182657in}{1.924279in}}{\pgfqpoint{2.174421in}{1.924279in}}%
\pgfpathcurveto{\pgfqpoint{2.166184in}{1.924279in}}{\pgfqpoint{2.158284in}{1.921007in}}{\pgfqpoint{2.152460in}{1.915183in}}%
\pgfpathcurveto{\pgfqpoint{2.146636in}{1.909359in}}{\pgfqpoint{2.143364in}{1.901459in}}{\pgfqpoint{2.143364in}{1.893223in}}%
\pgfpathcurveto{\pgfqpoint{2.143364in}{1.884987in}}{\pgfqpoint{2.146636in}{1.877087in}}{\pgfqpoint{2.152460in}{1.871263in}}%
\pgfpathcurveto{\pgfqpoint{2.158284in}{1.865439in}}{\pgfqpoint{2.166184in}{1.862166in}}{\pgfqpoint{2.174421in}{1.862166in}}%
\pgfpathclose%
\pgfusepath{stroke,fill}%
\end{pgfscope}%
\begin{pgfscope}%
\pgfpathrectangle{\pgfqpoint{0.100000in}{0.212622in}}{\pgfqpoint{3.696000in}{3.696000in}}%
\pgfusepath{clip}%
\pgfsetbuttcap%
\pgfsetroundjoin%
\definecolor{currentfill}{rgb}{0.121569,0.466667,0.705882}%
\pgfsetfillcolor{currentfill}%
\pgfsetfillopacity{0.740660}%
\pgfsetlinewidth{1.003750pt}%
\definecolor{currentstroke}{rgb}{0.121569,0.466667,0.705882}%
\pgfsetstrokecolor{currentstroke}%
\pgfsetstrokeopacity{0.740660}%
\pgfsetdash{}{0pt}%
\pgfpathmoveto{\pgfqpoint{2.185622in}{1.864263in}}%
\pgfpathcurveto{\pgfqpoint{2.193858in}{1.864263in}}{\pgfqpoint{2.201758in}{1.867535in}}{\pgfqpoint{2.207582in}{1.873359in}}%
\pgfpathcurveto{\pgfqpoint{2.213406in}{1.879183in}}{\pgfqpoint{2.216679in}{1.887083in}}{\pgfqpoint{2.216679in}{1.895319in}}%
\pgfpathcurveto{\pgfqpoint{2.216679in}{1.903556in}}{\pgfqpoint{2.213406in}{1.911456in}}{\pgfqpoint{2.207582in}{1.917280in}}%
\pgfpathcurveto{\pgfqpoint{2.201758in}{1.923104in}}{\pgfqpoint{2.193858in}{1.926376in}}{\pgfqpoint{2.185622in}{1.926376in}}%
\pgfpathcurveto{\pgfqpoint{2.177386in}{1.926376in}}{\pgfqpoint{2.169486in}{1.923104in}}{\pgfqpoint{2.163662in}{1.917280in}}%
\pgfpathcurveto{\pgfqpoint{2.157838in}{1.911456in}}{\pgfqpoint{2.154566in}{1.903556in}}{\pgfqpoint{2.154566in}{1.895319in}}%
\pgfpathcurveto{\pgfqpoint{2.154566in}{1.887083in}}{\pgfqpoint{2.157838in}{1.879183in}}{\pgfqpoint{2.163662in}{1.873359in}}%
\pgfpathcurveto{\pgfqpoint{2.169486in}{1.867535in}}{\pgfqpoint{2.177386in}{1.864263in}}{\pgfqpoint{2.185622in}{1.864263in}}%
\pgfpathclose%
\pgfusepath{stroke,fill}%
\end{pgfscope}%
\begin{pgfscope}%
\pgfpathrectangle{\pgfqpoint{0.100000in}{0.212622in}}{\pgfqpoint{3.696000in}{3.696000in}}%
\pgfusepath{clip}%
\pgfsetbuttcap%
\pgfsetroundjoin%
\definecolor{currentfill}{rgb}{0.121569,0.466667,0.705882}%
\pgfsetfillcolor{currentfill}%
\pgfsetfillopacity{0.746950}%
\pgfsetlinewidth{1.003750pt}%
\definecolor{currentstroke}{rgb}{0.121569,0.466667,0.705882}%
\pgfsetstrokecolor{currentstroke}%
\pgfsetstrokeopacity{0.746950}%
\pgfsetdash{}{0pt}%
\pgfpathmoveto{\pgfqpoint{2.189387in}{1.855439in}}%
\pgfpathcurveto{\pgfqpoint{2.197623in}{1.855439in}}{\pgfqpoint{2.205523in}{1.858711in}}{\pgfqpoint{2.211347in}{1.864535in}}%
\pgfpathcurveto{\pgfqpoint{2.217171in}{1.870359in}}{\pgfqpoint{2.220444in}{1.878259in}}{\pgfqpoint{2.220444in}{1.886496in}}%
\pgfpathcurveto{\pgfqpoint{2.220444in}{1.894732in}}{\pgfqpoint{2.217171in}{1.902632in}}{\pgfqpoint{2.211347in}{1.908456in}}%
\pgfpathcurveto{\pgfqpoint{2.205523in}{1.914280in}}{\pgfqpoint{2.197623in}{1.917552in}}{\pgfqpoint{2.189387in}{1.917552in}}%
\pgfpathcurveto{\pgfqpoint{2.181151in}{1.917552in}}{\pgfqpoint{2.173251in}{1.914280in}}{\pgfqpoint{2.167427in}{1.908456in}}%
\pgfpathcurveto{\pgfqpoint{2.161603in}{1.902632in}}{\pgfqpoint{2.158331in}{1.894732in}}{\pgfqpoint{2.158331in}{1.886496in}}%
\pgfpathcurveto{\pgfqpoint{2.158331in}{1.878259in}}{\pgfqpoint{2.161603in}{1.870359in}}{\pgfqpoint{2.167427in}{1.864535in}}%
\pgfpathcurveto{\pgfqpoint{2.173251in}{1.858711in}}{\pgfqpoint{2.181151in}{1.855439in}}{\pgfqpoint{2.189387in}{1.855439in}}%
\pgfpathclose%
\pgfusepath{stroke,fill}%
\end{pgfscope}%
\begin{pgfscope}%
\pgfpathrectangle{\pgfqpoint{0.100000in}{0.212622in}}{\pgfqpoint{3.696000in}{3.696000in}}%
\pgfusepath{clip}%
\pgfsetbuttcap%
\pgfsetroundjoin%
\definecolor{currentfill}{rgb}{0.121569,0.466667,0.705882}%
\pgfsetfillcolor{currentfill}%
\pgfsetfillopacity{0.756545}%
\pgfsetlinewidth{1.003750pt}%
\definecolor{currentstroke}{rgb}{0.121569,0.466667,0.705882}%
\pgfsetstrokecolor{currentstroke}%
\pgfsetstrokeopacity{0.756545}%
\pgfsetdash{}{0pt}%
\pgfpathmoveto{\pgfqpoint{2.206323in}{1.857801in}}%
\pgfpathcurveto{\pgfqpoint{2.214559in}{1.857801in}}{\pgfqpoint{2.222459in}{1.861073in}}{\pgfqpoint{2.228283in}{1.866897in}}%
\pgfpathcurveto{\pgfqpoint{2.234107in}{1.872721in}}{\pgfqpoint{2.237379in}{1.880621in}}{\pgfqpoint{2.237379in}{1.888858in}}%
\pgfpathcurveto{\pgfqpoint{2.237379in}{1.897094in}}{\pgfqpoint{2.234107in}{1.904994in}}{\pgfqpoint{2.228283in}{1.910818in}}%
\pgfpathcurveto{\pgfqpoint{2.222459in}{1.916642in}}{\pgfqpoint{2.214559in}{1.919914in}}{\pgfqpoint{2.206323in}{1.919914in}}%
\pgfpathcurveto{\pgfqpoint{2.198086in}{1.919914in}}{\pgfqpoint{2.190186in}{1.916642in}}{\pgfqpoint{2.184362in}{1.910818in}}%
\pgfpathcurveto{\pgfqpoint{2.178538in}{1.904994in}}{\pgfqpoint{2.175266in}{1.897094in}}{\pgfqpoint{2.175266in}{1.888858in}}%
\pgfpathcurveto{\pgfqpoint{2.175266in}{1.880621in}}{\pgfqpoint{2.178538in}{1.872721in}}{\pgfqpoint{2.184362in}{1.866897in}}%
\pgfpathcurveto{\pgfqpoint{2.190186in}{1.861073in}}{\pgfqpoint{2.198086in}{1.857801in}}{\pgfqpoint{2.206323in}{1.857801in}}%
\pgfpathclose%
\pgfusepath{stroke,fill}%
\end{pgfscope}%
\begin{pgfscope}%
\pgfpathrectangle{\pgfqpoint{0.100000in}{0.212622in}}{\pgfqpoint{3.696000in}{3.696000in}}%
\pgfusepath{clip}%
\pgfsetbuttcap%
\pgfsetroundjoin%
\definecolor{currentfill}{rgb}{0.121569,0.466667,0.705882}%
\pgfsetfillcolor{currentfill}%
\pgfsetfillopacity{0.764725}%
\pgfsetlinewidth{1.003750pt}%
\definecolor{currentstroke}{rgb}{0.121569,0.466667,0.705882}%
\pgfsetstrokecolor{currentstroke}%
\pgfsetstrokeopacity{0.764725}%
\pgfsetdash{}{0pt}%
\pgfpathmoveto{\pgfqpoint{2.211875in}{1.841697in}}%
\pgfpathcurveto{\pgfqpoint{2.220111in}{1.841697in}}{\pgfqpoint{2.228011in}{1.844969in}}{\pgfqpoint{2.233835in}{1.850793in}}%
\pgfpathcurveto{\pgfqpoint{2.239659in}{1.856617in}}{\pgfqpoint{2.242931in}{1.864517in}}{\pgfqpoint{2.242931in}{1.872753in}}%
\pgfpathcurveto{\pgfqpoint{2.242931in}{1.880990in}}{\pgfqpoint{2.239659in}{1.888890in}}{\pgfqpoint{2.233835in}{1.894714in}}%
\pgfpathcurveto{\pgfqpoint{2.228011in}{1.900538in}}{\pgfqpoint{2.220111in}{1.903810in}}{\pgfqpoint{2.211875in}{1.903810in}}%
\pgfpathcurveto{\pgfqpoint{2.203639in}{1.903810in}}{\pgfqpoint{2.195739in}{1.900538in}}{\pgfqpoint{2.189915in}{1.894714in}}%
\pgfpathcurveto{\pgfqpoint{2.184091in}{1.888890in}}{\pgfqpoint{2.180818in}{1.880990in}}{\pgfqpoint{2.180818in}{1.872753in}}%
\pgfpathcurveto{\pgfqpoint{2.180818in}{1.864517in}}{\pgfqpoint{2.184091in}{1.856617in}}{\pgfqpoint{2.189915in}{1.850793in}}%
\pgfpathcurveto{\pgfqpoint{2.195739in}{1.844969in}}{\pgfqpoint{2.203639in}{1.841697in}}{\pgfqpoint{2.211875in}{1.841697in}}%
\pgfpathclose%
\pgfusepath{stroke,fill}%
\end{pgfscope}%
\begin{pgfscope}%
\pgfpathrectangle{\pgfqpoint{0.100000in}{0.212622in}}{\pgfqpoint{3.696000in}{3.696000in}}%
\pgfusepath{clip}%
\pgfsetbuttcap%
\pgfsetroundjoin%
\definecolor{currentfill}{rgb}{0.121569,0.466667,0.705882}%
\pgfsetfillcolor{currentfill}%
\pgfsetfillopacity{0.777498}%
\pgfsetlinewidth{1.003750pt}%
\definecolor{currentstroke}{rgb}{0.121569,0.466667,0.705882}%
\pgfsetstrokecolor{currentstroke}%
\pgfsetstrokeopacity{0.777498}%
\pgfsetdash{}{0pt}%
\pgfpathmoveto{\pgfqpoint{2.234520in}{1.840064in}}%
\pgfpathcurveto{\pgfqpoint{2.242756in}{1.840064in}}{\pgfqpoint{2.250656in}{1.843336in}}{\pgfqpoint{2.256480in}{1.849160in}}%
\pgfpathcurveto{\pgfqpoint{2.262304in}{1.854984in}}{\pgfqpoint{2.265576in}{1.862884in}}{\pgfqpoint{2.265576in}{1.871121in}}%
\pgfpathcurveto{\pgfqpoint{2.265576in}{1.879357in}}{\pgfqpoint{2.262304in}{1.887257in}}{\pgfqpoint{2.256480in}{1.893081in}}%
\pgfpathcurveto{\pgfqpoint{2.250656in}{1.898905in}}{\pgfqpoint{2.242756in}{1.902177in}}{\pgfqpoint{2.234520in}{1.902177in}}%
\pgfpathcurveto{\pgfqpoint{2.226284in}{1.902177in}}{\pgfqpoint{2.218384in}{1.898905in}}{\pgfqpoint{2.212560in}{1.893081in}}%
\pgfpathcurveto{\pgfqpoint{2.206736in}{1.887257in}}{\pgfqpoint{2.203463in}{1.879357in}}{\pgfqpoint{2.203463in}{1.871121in}}%
\pgfpathcurveto{\pgfqpoint{2.203463in}{1.862884in}}{\pgfqpoint{2.206736in}{1.854984in}}{\pgfqpoint{2.212560in}{1.849160in}}%
\pgfpathcurveto{\pgfqpoint{2.218384in}{1.843336in}}{\pgfqpoint{2.226284in}{1.840064in}}{\pgfqpoint{2.234520in}{1.840064in}}%
\pgfpathclose%
\pgfusepath{stroke,fill}%
\end{pgfscope}%
\begin{pgfscope}%
\pgfpathrectangle{\pgfqpoint{0.100000in}{0.212622in}}{\pgfqpoint{3.696000in}{3.696000in}}%
\pgfusepath{clip}%
\pgfsetbuttcap%
\pgfsetroundjoin%
\definecolor{currentfill}{rgb}{0.121569,0.466667,0.705882}%
\pgfsetfillcolor{currentfill}%
\pgfsetfillopacity{0.782844}%
\pgfsetlinewidth{1.003750pt}%
\definecolor{currentstroke}{rgb}{0.121569,0.466667,0.705882}%
\pgfsetstrokecolor{currentstroke}%
\pgfsetstrokeopacity{0.782844}%
\pgfsetdash{}{0pt}%
\pgfpathmoveto{\pgfqpoint{2.238597in}{1.828735in}}%
\pgfpathcurveto{\pgfqpoint{2.246834in}{1.828735in}}{\pgfqpoint{2.254734in}{1.832007in}}{\pgfqpoint{2.260558in}{1.837831in}}%
\pgfpathcurveto{\pgfqpoint{2.266382in}{1.843655in}}{\pgfqpoint{2.269654in}{1.851555in}}{\pgfqpoint{2.269654in}{1.859791in}}%
\pgfpathcurveto{\pgfqpoint{2.269654in}{1.868028in}}{\pgfqpoint{2.266382in}{1.875928in}}{\pgfqpoint{2.260558in}{1.881752in}}%
\pgfpathcurveto{\pgfqpoint{2.254734in}{1.887575in}}{\pgfqpoint{2.246834in}{1.890848in}}{\pgfqpoint{2.238597in}{1.890848in}}%
\pgfpathcurveto{\pgfqpoint{2.230361in}{1.890848in}}{\pgfqpoint{2.222461in}{1.887575in}}{\pgfqpoint{2.216637in}{1.881752in}}%
\pgfpathcurveto{\pgfqpoint{2.210813in}{1.875928in}}{\pgfqpoint{2.207541in}{1.868028in}}{\pgfqpoint{2.207541in}{1.859791in}}%
\pgfpathcurveto{\pgfqpoint{2.207541in}{1.851555in}}{\pgfqpoint{2.210813in}{1.843655in}}{\pgfqpoint{2.216637in}{1.837831in}}%
\pgfpathcurveto{\pgfqpoint{2.222461in}{1.832007in}}{\pgfqpoint{2.230361in}{1.828735in}}{\pgfqpoint{2.238597in}{1.828735in}}%
\pgfpathclose%
\pgfusepath{stroke,fill}%
\end{pgfscope}%
\begin{pgfscope}%
\pgfpathrectangle{\pgfqpoint{0.100000in}{0.212622in}}{\pgfqpoint{3.696000in}{3.696000in}}%
\pgfusepath{clip}%
\pgfsetbuttcap%
\pgfsetroundjoin%
\definecolor{currentfill}{rgb}{0.121569,0.466667,0.705882}%
\pgfsetfillcolor{currentfill}%
\pgfsetfillopacity{0.789994}%
\pgfsetlinewidth{1.003750pt}%
\definecolor{currentstroke}{rgb}{0.121569,0.466667,0.705882}%
\pgfsetstrokecolor{currentstroke}%
\pgfsetstrokeopacity{0.789994}%
\pgfsetdash{}{0pt}%
\pgfpathmoveto{\pgfqpoint{2.247508in}{1.822690in}}%
\pgfpathcurveto{\pgfqpoint{2.255745in}{1.822690in}}{\pgfqpoint{2.263645in}{1.825963in}}{\pgfqpoint{2.269469in}{1.831787in}}%
\pgfpathcurveto{\pgfqpoint{2.275293in}{1.837610in}}{\pgfqpoint{2.278565in}{1.845511in}}{\pgfqpoint{2.278565in}{1.853747in}}%
\pgfpathcurveto{\pgfqpoint{2.278565in}{1.861983in}}{\pgfqpoint{2.275293in}{1.869883in}}{\pgfqpoint{2.269469in}{1.875707in}}%
\pgfpathcurveto{\pgfqpoint{2.263645in}{1.881531in}}{\pgfqpoint{2.255745in}{1.884803in}}{\pgfqpoint{2.247508in}{1.884803in}}%
\pgfpathcurveto{\pgfqpoint{2.239272in}{1.884803in}}{\pgfqpoint{2.231372in}{1.881531in}}{\pgfqpoint{2.225548in}{1.875707in}}%
\pgfpathcurveto{\pgfqpoint{2.219724in}{1.869883in}}{\pgfqpoint{2.216452in}{1.861983in}}{\pgfqpoint{2.216452in}{1.853747in}}%
\pgfpathcurveto{\pgfqpoint{2.216452in}{1.845511in}}{\pgfqpoint{2.219724in}{1.837610in}}{\pgfqpoint{2.225548in}{1.831787in}}%
\pgfpathcurveto{\pgfqpoint{2.231372in}{1.825963in}}{\pgfqpoint{2.239272in}{1.822690in}}{\pgfqpoint{2.247508in}{1.822690in}}%
\pgfpathclose%
\pgfusepath{stroke,fill}%
\end{pgfscope}%
\begin{pgfscope}%
\pgfpathrectangle{\pgfqpoint{0.100000in}{0.212622in}}{\pgfqpoint{3.696000in}{3.696000in}}%
\pgfusepath{clip}%
\pgfsetbuttcap%
\pgfsetroundjoin%
\definecolor{currentfill}{rgb}{0.121569,0.466667,0.705882}%
\pgfsetfillcolor{currentfill}%
\pgfsetfillopacity{0.794117}%
\pgfsetlinewidth{1.003750pt}%
\definecolor{currentstroke}{rgb}{0.121569,0.466667,0.705882}%
\pgfsetstrokecolor{currentstroke}%
\pgfsetstrokeopacity{0.794117}%
\pgfsetdash{}{0pt}%
\pgfpathmoveto{\pgfqpoint{2.252528in}{1.820541in}}%
\pgfpathcurveto{\pgfqpoint{2.260764in}{1.820541in}}{\pgfqpoint{2.268665in}{1.823814in}}{\pgfqpoint{2.274488in}{1.829638in}}%
\pgfpathcurveto{\pgfqpoint{2.280312in}{1.835462in}}{\pgfqpoint{2.283585in}{1.843362in}}{\pgfqpoint{2.283585in}{1.851598in}}%
\pgfpathcurveto{\pgfqpoint{2.283585in}{1.859834in}}{\pgfqpoint{2.280312in}{1.867734in}}{\pgfqpoint{2.274488in}{1.873558in}}%
\pgfpathcurveto{\pgfqpoint{2.268665in}{1.879382in}}{\pgfqpoint{2.260764in}{1.882654in}}{\pgfqpoint{2.252528in}{1.882654in}}%
\pgfpathcurveto{\pgfqpoint{2.244292in}{1.882654in}}{\pgfqpoint{2.236392in}{1.879382in}}{\pgfqpoint{2.230568in}{1.873558in}}%
\pgfpathcurveto{\pgfqpoint{2.224744in}{1.867734in}}{\pgfqpoint{2.221472in}{1.859834in}}{\pgfqpoint{2.221472in}{1.851598in}}%
\pgfpathcurveto{\pgfqpoint{2.221472in}{1.843362in}}{\pgfqpoint{2.224744in}{1.835462in}}{\pgfqpoint{2.230568in}{1.829638in}}%
\pgfpathcurveto{\pgfqpoint{2.236392in}{1.823814in}}{\pgfqpoint{2.244292in}{1.820541in}}{\pgfqpoint{2.252528in}{1.820541in}}%
\pgfpathclose%
\pgfusepath{stroke,fill}%
\end{pgfscope}%
\begin{pgfscope}%
\pgfpathrectangle{\pgfqpoint{0.100000in}{0.212622in}}{\pgfqpoint{3.696000in}{3.696000in}}%
\pgfusepath{clip}%
\pgfsetbuttcap%
\pgfsetroundjoin%
\definecolor{currentfill}{rgb}{0.121569,0.466667,0.705882}%
\pgfsetfillcolor{currentfill}%
\pgfsetfillopacity{0.798981}%
\pgfsetlinewidth{1.003750pt}%
\definecolor{currentstroke}{rgb}{0.121569,0.466667,0.705882}%
\pgfsetstrokecolor{currentstroke}%
\pgfsetstrokeopacity{0.798981}%
\pgfsetdash{}{0pt}%
\pgfpathmoveto{\pgfqpoint{2.253736in}{1.816048in}}%
\pgfpathcurveto{\pgfqpoint{2.261972in}{1.816048in}}{\pgfqpoint{2.269872in}{1.819320in}}{\pgfqpoint{2.275696in}{1.825144in}}%
\pgfpathcurveto{\pgfqpoint{2.281520in}{1.830968in}}{\pgfqpoint{2.284793in}{1.838868in}}{\pgfqpoint{2.284793in}{1.847104in}}%
\pgfpathcurveto{\pgfqpoint{2.284793in}{1.855341in}}{\pgfqpoint{2.281520in}{1.863241in}}{\pgfqpoint{2.275696in}{1.869065in}}%
\pgfpathcurveto{\pgfqpoint{2.269872in}{1.874888in}}{\pgfqpoint{2.261972in}{1.878161in}}{\pgfqpoint{2.253736in}{1.878161in}}%
\pgfpathcurveto{\pgfqpoint{2.245500in}{1.878161in}}{\pgfqpoint{2.237600in}{1.874888in}}{\pgfqpoint{2.231776in}{1.869065in}}%
\pgfpathcurveto{\pgfqpoint{2.225952in}{1.863241in}}{\pgfqpoint{2.222680in}{1.855341in}}{\pgfqpoint{2.222680in}{1.847104in}}%
\pgfpathcurveto{\pgfqpoint{2.222680in}{1.838868in}}{\pgfqpoint{2.225952in}{1.830968in}}{\pgfqpoint{2.231776in}{1.825144in}}%
\pgfpathcurveto{\pgfqpoint{2.237600in}{1.819320in}}{\pgfqpoint{2.245500in}{1.816048in}}{\pgfqpoint{2.253736in}{1.816048in}}%
\pgfpathclose%
\pgfusepath{stroke,fill}%
\end{pgfscope}%
\begin{pgfscope}%
\pgfpathrectangle{\pgfqpoint{0.100000in}{0.212622in}}{\pgfqpoint{3.696000in}{3.696000in}}%
\pgfusepath{clip}%
\pgfsetbuttcap%
\pgfsetroundjoin%
\definecolor{currentfill}{rgb}{0.121569,0.466667,0.705882}%
\pgfsetfillcolor{currentfill}%
\pgfsetfillopacity{0.807518}%
\pgfsetlinewidth{1.003750pt}%
\definecolor{currentstroke}{rgb}{0.121569,0.466667,0.705882}%
\pgfsetstrokecolor{currentstroke}%
\pgfsetstrokeopacity{0.807518}%
\pgfsetdash{}{0pt}%
\pgfpathmoveto{\pgfqpoint{2.268915in}{1.817631in}}%
\pgfpathcurveto{\pgfqpoint{2.277151in}{1.817631in}}{\pgfqpoint{2.285051in}{1.820903in}}{\pgfqpoint{2.290875in}{1.826727in}}%
\pgfpathcurveto{\pgfqpoint{2.296699in}{1.832551in}}{\pgfqpoint{2.299971in}{1.840451in}}{\pgfqpoint{2.299971in}{1.848687in}}%
\pgfpathcurveto{\pgfqpoint{2.299971in}{1.856923in}}{\pgfqpoint{2.296699in}{1.864823in}}{\pgfqpoint{2.290875in}{1.870647in}}%
\pgfpathcurveto{\pgfqpoint{2.285051in}{1.876471in}}{\pgfqpoint{2.277151in}{1.879744in}}{\pgfqpoint{2.268915in}{1.879744in}}%
\pgfpathcurveto{\pgfqpoint{2.260678in}{1.879744in}}{\pgfqpoint{2.252778in}{1.876471in}}{\pgfqpoint{2.246955in}{1.870647in}}%
\pgfpathcurveto{\pgfqpoint{2.241131in}{1.864823in}}{\pgfqpoint{2.237858in}{1.856923in}}{\pgfqpoint{2.237858in}{1.848687in}}%
\pgfpathcurveto{\pgfqpoint{2.237858in}{1.840451in}}{\pgfqpoint{2.241131in}{1.832551in}}{\pgfqpoint{2.246955in}{1.826727in}}%
\pgfpathcurveto{\pgfqpoint{2.252778in}{1.820903in}}{\pgfqpoint{2.260678in}{1.817631in}}{\pgfqpoint{2.268915in}{1.817631in}}%
\pgfpathclose%
\pgfusepath{stroke,fill}%
\end{pgfscope}%
\begin{pgfscope}%
\pgfpathrectangle{\pgfqpoint{0.100000in}{0.212622in}}{\pgfqpoint{3.696000in}{3.696000in}}%
\pgfusepath{clip}%
\pgfsetbuttcap%
\pgfsetroundjoin%
\definecolor{currentfill}{rgb}{0.121569,0.466667,0.705882}%
\pgfsetfillcolor{currentfill}%
\pgfsetfillopacity{0.813342}%
\pgfsetlinewidth{1.003750pt}%
\definecolor{currentstroke}{rgb}{0.121569,0.466667,0.705882}%
\pgfsetstrokecolor{currentstroke}%
\pgfsetstrokeopacity{0.813342}%
\pgfsetdash{}{0pt}%
\pgfpathmoveto{\pgfqpoint{0.542668in}{2.429983in}}%
\pgfpathcurveto{\pgfqpoint{0.550904in}{2.429983in}}{\pgfqpoint{0.558804in}{2.433256in}}{\pgfqpoint{0.564628in}{2.439080in}}%
\pgfpathcurveto{\pgfqpoint{0.570452in}{2.444904in}}{\pgfqpoint{0.573725in}{2.452804in}}{\pgfqpoint{0.573725in}{2.461040in}}%
\pgfpathcurveto{\pgfqpoint{0.573725in}{2.469276in}}{\pgfqpoint{0.570452in}{2.477176in}}{\pgfqpoint{0.564628in}{2.483000in}}%
\pgfpathcurveto{\pgfqpoint{0.558804in}{2.488824in}}{\pgfqpoint{0.550904in}{2.492096in}}{\pgfqpoint{0.542668in}{2.492096in}}%
\pgfpathcurveto{\pgfqpoint{0.534432in}{2.492096in}}{\pgfqpoint{0.526532in}{2.488824in}}{\pgfqpoint{0.520708in}{2.483000in}}%
\pgfpathcurveto{\pgfqpoint{0.514884in}{2.477176in}}{\pgfqpoint{0.511612in}{2.469276in}}{\pgfqpoint{0.511612in}{2.461040in}}%
\pgfpathcurveto{\pgfqpoint{0.511612in}{2.452804in}}{\pgfqpoint{0.514884in}{2.444904in}}{\pgfqpoint{0.520708in}{2.439080in}}%
\pgfpathcurveto{\pgfqpoint{0.526532in}{2.433256in}}{\pgfqpoint{0.534432in}{2.429983in}}{\pgfqpoint{0.542668in}{2.429983in}}%
\pgfpathclose%
\pgfusepath{stroke,fill}%
\end{pgfscope}%
\begin{pgfscope}%
\pgfpathrectangle{\pgfqpoint{0.100000in}{0.212622in}}{\pgfqpoint{3.696000in}{3.696000in}}%
\pgfusepath{clip}%
\pgfsetbuttcap%
\pgfsetroundjoin%
\definecolor{currentfill}{rgb}{0.121569,0.466667,0.705882}%
\pgfsetfillcolor{currentfill}%
\pgfsetfillopacity{0.815309}%
\pgfsetlinewidth{1.003750pt}%
\definecolor{currentstroke}{rgb}{0.121569,0.466667,0.705882}%
\pgfsetstrokecolor{currentstroke}%
\pgfsetstrokeopacity{0.815309}%
\pgfsetdash{}{0pt}%
\pgfpathmoveto{\pgfqpoint{2.275679in}{1.807869in}}%
\pgfpathcurveto{\pgfqpoint{2.283915in}{1.807869in}}{\pgfqpoint{2.291815in}{1.811142in}}{\pgfqpoint{2.297639in}{1.816966in}}%
\pgfpathcurveto{\pgfqpoint{2.303463in}{1.822790in}}{\pgfqpoint{2.306735in}{1.830690in}}{\pgfqpoint{2.306735in}{1.838926in}}%
\pgfpathcurveto{\pgfqpoint{2.306735in}{1.847162in}}{\pgfqpoint{2.303463in}{1.855062in}}{\pgfqpoint{2.297639in}{1.860886in}}%
\pgfpathcurveto{\pgfqpoint{2.291815in}{1.866710in}}{\pgfqpoint{2.283915in}{1.869982in}}{\pgfqpoint{2.275679in}{1.869982in}}%
\pgfpathcurveto{\pgfqpoint{2.267442in}{1.869982in}}{\pgfqpoint{2.259542in}{1.866710in}}{\pgfqpoint{2.253718in}{1.860886in}}%
\pgfpathcurveto{\pgfqpoint{2.247894in}{1.855062in}}{\pgfqpoint{2.244622in}{1.847162in}}{\pgfqpoint{2.244622in}{1.838926in}}%
\pgfpathcurveto{\pgfqpoint{2.244622in}{1.830690in}}{\pgfqpoint{2.247894in}{1.822790in}}{\pgfqpoint{2.253718in}{1.816966in}}%
\pgfpathcurveto{\pgfqpoint{2.259542in}{1.811142in}}{\pgfqpoint{2.267442in}{1.807869in}}{\pgfqpoint{2.275679in}{1.807869in}}%
\pgfpathclose%
\pgfusepath{stroke,fill}%
\end{pgfscope}%
\begin{pgfscope}%
\pgfpathrectangle{\pgfqpoint{0.100000in}{0.212622in}}{\pgfqpoint{3.696000in}{3.696000in}}%
\pgfusepath{clip}%
\pgfsetbuttcap%
\pgfsetroundjoin%
\definecolor{currentfill}{rgb}{0.121569,0.466667,0.705882}%
\pgfsetfillcolor{currentfill}%
\pgfsetfillopacity{0.815444}%
\pgfsetlinewidth{1.003750pt}%
\definecolor{currentstroke}{rgb}{0.121569,0.466667,0.705882}%
\pgfsetstrokecolor{currentstroke}%
\pgfsetstrokeopacity{0.815444}%
\pgfsetdash{}{0pt}%
\pgfpathmoveto{\pgfqpoint{0.553151in}{2.428211in}}%
\pgfpathcurveto{\pgfqpoint{0.561388in}{2.428211in}}{\pgfqpoint{0.569288in}{2.431483in}}{\pgfqpoint{0.575112in}{2.437307in}}%
\pgfpathcurveto{\pgfqpoint{0.580936in}{2.443131in}}{\pgfqpoint{0.584208in}{2.451031in}}{\pgfqpoint{0.584208in}{2.459267in}}%
\pgfpathcurveto{\pgfqpoint{0.584208in}{2.467504in}}{\pgfqpoint{0.580936in}{2.475404in}}{\pgfqpoint{0.575112in}{2.481228in}}%
\pgfpathcurveto{\pgfqpoint{0.569288in}{2.487051in}}{\pgfqpoint{0.561388in}{2.490324in}}{\pgfqpoint{0.553151in}{2.490324in}}%
\pgfpathcurveto{\pgfqpoint{0.544915in}{2.490324in}}{\pgfqpoint{0.537015in}{2.487051in}}{\pgfqpoint{0.531191in}{2.481228in}}%
\pgfpathcurveto{\pgfqpoint{0.525367in}{2.475404in}}{\pgfqpoint{0.522095in}{2.467504in}}{\pgfqpoint{0.522095in}{2.459267in}}%
\pgfpathcurveto{\pgfqpoint{0.522095in}{2.451031in}}{\pgfqpoint{0.525367in}{2.443131in}}{\pgfqpoint{0.531191in}{2.437307in}}%
\pgfpathcurveto{\pgfqpoint{0.537015in}{2.431483in}}{\pgfqpoint{0.544915in}{2.428211in}}{\pgfqpoint{0.553151in}{2.428211in}}%
\pgfpathclose%
\pgfusepath{stroke,fill}%
\end{pgfscope}%
\begin{pgfscope}%
\pgfpathrectangle{\pgfqpoint{0.100000in}{0.212622in}}{\pgfqpoint{3.696000in}{3.696000in}}%
\pgfusepath{clip}%
\pgfsetbuttcap%
\pgfsetroundjoin%
\definecolor{currentfill}{rgb}{0.121569,0.466667,0.705882}%
\pgfsetfillcolor{currentfill}%
\pgfsetfillopacity{0.817831}%
\pgfsetlinewidth{1.003750pt}%
\definecolor{currentstroke}{rgb}{0.121569,0.466667,0.705882}%
\pgfsetstrokecolor{currentstroke}%
\pgfsetstrokeopacity{0.817831}%
\pgfsetdash{}{0pt}%
\pgfpathmoveto{\pgfqpoint{0.582097in}{2.409907in}}%
\pgfpathcurveto{\pgfqpoint{0.590333in}{2.409907in}}{\pgfqpoint{0.598233in}{2.413180in}}{\pgfqpoint{0.604057in}{2.419004in}}%
\pgfpathcurveto{\pgfqpoint{0.609881in}{2.424827in}}{\pgfqpoint{0.613153in}{2.432728in}}{\pgfqpoint{0.613153in}{2.440964in}}%
\pgfpathcurveto{\pgfqpoint{0.613153in}{2.449200in}}{\pgfqpoint{0.609881in}{2.457100in}}{\pgfqpoint{0.604057in}{2.462924in}}%
\pgfpathcurveto{\pgfqpoint{0.598233in}{2.468748in}}{\pgfqpoint{0.590333in}{2.472020in}}{\pgfqpoint{0.582097in}{2.472020in}}%
\pgfpathcurveto{\pgfqpoint{0.573860in}{2.472020in}}{\pgfqpoint{0.565960in}{2.468748in}}{\pgfqpoint{0.560136in}{2.462924in}}%
\pgfpathcurveto{\pgfqpoint{0.554312in}{2.457100in}}{\pgfqpoint{0.551040in}{2.449200in}}{\pgfqpoint{0.551040in}{2.440964in}}%
\pgfpathcurveto{\pgfqpoint{0.551040in}{2.432728in}}{\pgfqpoint{0.554312in}{2.424827in}}{\pgfqpoint{0.560136in}{2.419004in}}%
\pgfpathcurveto{\pgfqpoint{0.565960in}{2.413180in}}{\pgfqpoint{0.573860in}{2.409907in}}{\pgfqpoint{0.582097in}{2.409907in}}%
\pgfpathclose%
\pgfusepath{stroke,fill}%
\end{pgfscope}%
\begin{pgfscope}%
\pgfpathrectangle{\pgfqpoint{0.100000in}{0.212622in}}{\pgfqpoint{3.696000in}{3.696000in}}%
\pgfusepath{clip}%
\pgfsetbuttcap%
\pgfsetroundjoin%
\definecolor{currentfill}{rgb}{0.121569,0.466667,0.705882}%
\pgfsetfillcolor{currentfill}%
\pgfsetfillopacity{0.819639}%
\pgfsetlinewidth{1.003750pt}%
\definecolor{currentstroke}{rgb}{0.121569,0.466667,0.705882}%
\pgfsetstrokecolor{currentstroke}%
\pgfsetstrokeopacity{0.819639}%
\pgfsetdash{}{0pt}%
\pgfpathmoveto{\pgfqpoint{0.577169in}{2.424450in}}%
\pgfpathcurveto{\pgfqpoint{0.585406in}{2.424450in}}{\pgfqpoint{0.593306in}{2.427723in}}{\pgfqpoint{0.599130in}{2.433547in}}%
\pgfpathcurveto{\pgfqpoint{0.604954in}{2.439371in}}{\pgfqpoint{0.608226in}{2.447271in}}{\pgfqpoint{0.608226in}{2.455507in}}%
\pgfpathcurveto{\pgfqpoint{0.608226in}{2.463743in}}{\pgfqpoint{0.604954in}{2.471643in}}{\pgfqpoint{0.599130in}{2.477467in}}%
\pgfpathcurveto{\pgfqpoint{0.593306in}{2.483291in}}{\pgfqpoint{0.585406in}{2.486563in}}{\pgfqpoint{0.577169in}{2.486563in}}%
\pgfpathcurveto{\pgfqpoint{0.568933in}{2.486563in}}{\pgfqpoint{0.561033in}{2.483291in}}{\pgfqpoint{0.555209in}{2.477467in}}%
\pgfpathcurveto{\pgfqpoint{0.549385in}{2.471643in}}{\pgfqpoint{0.546113in}{2.463743in}}{\pgfqpoint{0.546113in}{2.455507in}}%
\pgfpathcurveto{\pgfqpoint{0.546113in}{2.447271in}}{\pgfqpoint{0.549385in}{2.439371in}}{\pgfqpoint{0.555209in}{2.433547in}}%
\pgfpathcurveto{\pgfqpoint{0.561033in}{2.427723in}}{\pgfqpoint{0.568933in}{2.424450in}}{\pgfqpoint{0.577169in}{2.424450in}}%
\pgfpathclose%
\pgfusepath{stroke,fill}%
\end{pgfscope}%
\begin{pgfscope}%
\pgfpathrectangle{\pgfqpoint{0.100000in}{0.212622in}}{\pgfqpoint{3.696000in}{3.696000in}}%
\pgfusepath{clip}%
\pgfsetbuttcap%
\pgfsetroundjoin%
\definecolor{currentfill}{rgb}{0.121569,0.466667,0.705882}%
\pgfsetfillcolor{currentfill}%
\pgfsetfillopacity{0.820387}%
\pgfsetlinewidth{1.003750pt}%
\definecolor{currentstroke}{rgb}{0.121569,0.466667,0.705882}%
\pgfsetstrokecolor{currentstroke}%
\pgfsetstrokeopacity{0.820387}%
\pgfsetdash{}{0pt}%
\pgfpathmoveto{\pgfqpoint{2.284659in}{1.807289in}}%
\pgfpathcurveto{\pgfqpoint{2.292895in}{1.807289in}}{\pgfqpoint{2.300795in}{1.810562in}}{\pgfqpoint{2.306619in}{1.816386in}}%
\pgfpathcurveto{\pgfqpoint{2.312443in}{1.822210in}}{\pgfqpoint{2.315716in}{1.830110in}}{\pgfqpoint{2.315716in}{1.838346in}}%
\pgfpathcurveto{\pgfqpoint{2.315716in}{1.846582in}}{\pgfqpoint{2.312443in}{1.854482in}}{\pgfqpoint{2.306619in}{1.860306in}}%
\pgfpathcurveto{\pgfqpoint{2.300795in}{1.866130in}}{\pgfqpoint{2.292895in}{1.869402in}}{\pgfqpoint{2.284659in}{1.869402in}}%
\pgfpathcurveto{\pgfqpoint{2.276423in}{1.869402in}}{\pgfqpoint{2.268523in}{1.866130in}}{\pgfqpoint{2.262699in}{1.860306in}}%
\pgfpathcurveto{\pgfqpoint{2.256875in}{1.854482in}}{\pgfqpoint{2.253603in}{1.846582in}}{\pgfqpoint{2.253603in}{1.838346in}}%
\pgfpathcurveto{\pgfqpoint{2.253603in}{1.830110in}}{\pgfqpoint{2.256875in}{1.822210in}}{\pgfqpoint{2.262699in}{1.816386in}}%
\pgfpathcurveto{\pgfqpoint{2.268523in}{1.810562in}}{\pgfqpoint{2.276423in}{1.807289in}}{\pgfqpoint{2.284659in}{1.807289in}}%
\pgfpathclose%
\pgfusepath{stroke,fill}%
\end{pgfscope}%
\begin{pgfscope}%
\pgfpathrectangle{\pgfqpoint{0.100000in}{0.212622in}}{\pgfqpoint{3.696000in}{3.696000in}}%
\pgfusepath{clip}%
\pgfsetbuttcap%
\pgfsetroundjoin%
\definecolor{currentfill}{rgb}{0.121569,0.466667,0.705882}%
\pgfsetfillcolor{currentfill}%
\pgfsetfillopacity{0.822348}%
\pgfsetlinewidth{1.003750pt}%
\definecolor{currentstroke}{rgb}{0.121569,0.466667,0.705882}%
\pgfsetstrokecolor{currentstroke}%
\pgfsetstrokeopacity{0.822348}%
\pgfsetdash{}{0pt}%
\pgfpathmoveto{\pgfqpoint{0.618950in}{2.395570in}}%
\pgfpathcurveto{\pgfqpoint{0.627186in}{2.395570in}}{\pgfqpoint{0.635086in}{2.398842in}}{\pgfqpoint{0.640910in}{2.404666in}}%
\pgfpathcurveto{\pgfqpoint{0.646734in}{2.410490in}}{\pgfqpoint{0.650006in}{2.418390in}}{\pgfqpoint{0.650006in}{2.426626in}}%
\pgfpathcurveto{\pgfqpoint{0.650006in}{2.434863in}}{\pgfqpoint{0.646734in}{2.442763in}}{\pgfqpoint{0.640910in}{2.448587in}}%
\pgfpathcurveto{\pgfqpoint{0.635086in}{2.454411in}}{\pgfqpoint{0.627186in}{2.457683in}}{\pgfqpoint{0.618950in}{2.457683in}}%
\pgfpathcurveto{\pgfqpoint{0.610713in}{2.457683in}}{\pgfqpoint{0.602813in}{2.454411in}}{\pgfqpoint{0.596989in}{2.448587in}}%
\pgfpathcurveto{\pgfqpoint{0.591165in}{2.442763in}}{\pgfqpoint{0.587893in}{2.434863in}}{\pgfqpoint{0.587893in}{2.426626in}}%
\pgfpathcurveto{\pgfqpoint{0.587893in}{2.418390in}}{\pgfqpoint{0.591165in}{2.410490in}}{\pgfqpoint{0.596989in}{2.404666in}}%
\pgfpathcurveto{\pgfqpoint{0.602813in}{2.398842in}}{\pgfqpoint{0.610713in}{2.395570in}}{\pgfqpoint{0.618950in}{2.395570in}}%
\pgfpathclose%
\pgfusepath{stroke,fill}%
\end{pgfscope}%
\begin{pgfscope}%
\pgfpathrectangle{\pgfqpoint{0.100000in}{0.212622in}}{\pgfqpoint{3.696000in}{3.696000in}}%
\pgfusepath{clip}%
\pgfsetbuttcap%
\pgfsetroundjoin%
\definecolor{currentfill}{rgb}{0.121569,0.466667,0.705882}%
\pgfsetfillcolor{currentfill}%
\pgfsetfillopacity{0.824474}%
\pgfsetlinewidth{1.003750pt}%
\definecolor{currentstroke}{rgb}{0.121569,0.466667,0.705882}%
\pgfsetstrokecolor{currentstroke}%
\pgfsetstrokeopacity{0.824474}%
\pgfsetdash{}{0pt}%
\pgfpathmoveto{\pgfqpoint{0.610098in}{2.419305in}}%
\pgfpathcurveto{\pgfqpoint{0.618334in}{2.419305in}}{\pgfqpoint{0.626234in}{2.422577in}}{\pgfqpoint{0.632058in}{2.428401in}}%
\pgfpathcurveto{\pgfqpoint{0.637882in}{2.434225in}}{\pgfqpoint{0.641154in}{2.442125in}}{\pgfqpoint{0.641154in}{2.450361in}}%
\pgfpathcurveto{\pgfqpoint{0.641154in}{2.458597in}}{\pgfqpoint{0.637882in}{2.466498in}}{\pgfqpoint{0.632058in}{2.472321in}}%
\pgfpathcurveto{\pgfqpoint{0.626234in}{2.478145in}}{\pgfqpoint{0.618334in}{2.481418in}}{\pgfqpoint{0.610098in}{2.481418in}}%
\pgfpathcurveto{\pgfqpoint{0.601861in}{2.481418in}}{\pgfqpoint{0.593961in}{2.478145in}}{\pgfqpoint{0.588138in}{2.472321in}}%
\pgfpathcurveto{\pgfqpoint{0.582314in}{2.466498in}}{\pgfqpoint{0.579041in}{2.458597in}}{\pgfqpoint{0.579041in}{2.450361in}}%
\pgfpathcurveto{\pgfqpoint{0.579041in}{2.442125in}}{\pgfqpoint{0.582314in}{2.434225in}}{\pgfqpoint{0.588138in}{2.428401in}}%
\pgfpathcurveto{\pgfqpoint{0.593961in}{2.422577in}}{\pgfqpoint{0.601861in}{2.419305in}}{\pgfqpoint{0.610098in}{2.419305in}}%
\pgfpathclose%
\pgfusepath{stroke,fill}%
\end{pgfscope}%
\begin{pgfscope}%
\pgfpathrectangle{\pgfqpoint{0.100000in}{0.212622in}}{\pgfqpoint{3.696000in}{3.696000in}}%
\pgfusepath{clip}%
\pgfsetbuttcap%
\pgfsetroundjoin%
\definecolor{currentfill}{rgb}{0.121569,0.466667,0.705882}%
\pgfsetfillcolor{currentfill}%
\pgfsetfillopacity{0.825480}%
\pgfsetlinewidth{1.003750pt}%
\definecolor{currentstroke}{rgb}{0.121569,0.466667,0.705882}%
\pgfsetstrokecolor{currentstroke}%
\pgfsetstrokeopacity{0.825480}%
\pgfsetdash{}{0pt}%
\pgfpathmoveto{\pgfqpoint{2.288777in}{1.798225in}}%
\pgfpathcurveto{\pgfqpoint{2.297013in}{1.798225in}}{\pgfqpoint{2.304913in}{1.801497in}}{\pgfqpoint{2.310737in}{1.807321in}}%
\pgfpathcurveto{\pgfqpoint{2.316561in}{1.813145in}}{\pgfqpoint{2.319833in}{1.821045in}}{\pgfqpoint{2.319833in}{1.829281in}}%
\pgfpathcurveto{\pgfqpoint{2.319833in}{1.837517in}}{\pgfqpoint{2.316561in}{1.845418in}}{\pgfqpoint{2.310737in}{1.851241in}}%
\pgfpathcurveto{\pgfqpoint{2.304913in}{1.857065in}}{\pgfqpoint{2.297013in}{1.860338in}}{\pgfqpoint{2.288777in}{1.860338in}}%
\pgfpathcurveto{\pgfqpoint{2.280540in}{1.860338in}}{\pgfqpoint{2.272640in}{1.857065in}}{\pgfqpoint{2.266816in}{1.851241in}}%
\pgfpathcurveto{\pgfqpoint{2.260992in}{1.845418in}}{\pgfqpoint{2.257720in}{1.837517in}}{\pgfqpoint{2.257720in}{1.829281in}}%
\pgfpathcurveto{\pgfqpoint{2.257720in}{1.821045in}}{\pgfqpoint{2.260992in}{1.813145in}}{\pgfqpoint{2.266816in}{1.807321in}}%
\pgfpathcurveto{\pgfqpoint{2.272640in}{1.801497in}}{\pgfqpoint{2.280540in}{1.798225in}}{\pgfqpoint{2.288777in}{1.798225in}}%
\pgfpathclose%
\pgfusepath{stroke,fill}%
\end{pgfscope}%
\begin{pgfscope}%
\pgfpathrectangle{\pgfqpoint{0.100000in}{0.212622in}}{\pgfqpoint{3.696000in}{3.696000in}}%
\pgfusepath{clip}%
\pgfsetbuttcap%
\pgfsetroundjoin%
\definecolor{currentfill}{rgb}{0.121569,0.466667,0.705882}%
\pgfsetfillcolor{currentfill}%
\pgfsetfillopacity{0.825589}%
\pgfsetlinewidth{1.003750pt}%
\definecolor{currentstroke}{rgb}{0.121569,0.466667,0.705882}%
\pgfsetstrokecolor{currentstroke}%
\pgfsetstrokeopacity{0.825589}%
\pgfsetdash{}{0pt}%
\pgfpathmoveto{\pgfqpoint{0.647555in}{2.378938in}}%
\pgfpathcurveto{\pgfqpoint{0.655791in}{2.378938in}}{\pgfqpoint{0.663691in}{2.382210in}}{\pgfqpoint{0.669515in}{2.388034in}}%
\pgfpathcurveto{\pgfqpoint{0.675339in}{2.393858in}}{\pgfqpoint{0.678611in}{2.401758in}}{\pgfqpoint{0.678611in}{2.409995in}}%
\pgfpathcurveto{\pgfqpoint{0.678611in}{2.418231in}}{\pgfqpoint{0.675339in}{2.426131in}}{\pgfqpoint{0.669515in}{2.431955in}}%
\pgfpathcurveto{\pgfqpoint{0.663691in}{2.437779in}}{\pgfqpoint{0.655791in}{2.441051in}}{\pgfqpoint{0.647555in}{2.441051in}}%
\pgfpathcurveto{\pgfqpoint{0.639319in}{2.441051in}}{\pgfqpoint{0.631419in}{2.437779in}}{\pgfqpoint{0.625595in}{2.431955in}}%
\pgfpathcurveto{\pgfqpoint{0.619771in}{2.426131in}}{\pgfqpoint{0.616498in}{2.418231in}}{\pgfqpoint{0.616498in}{2.409995in}}%
\pgfpathcurveto{\pgfqpoint{0.616498in}{2.401758in}}{\pgfqpoint{0.619771in}{2.393858in}}{\pgfqpoint{0.625595in}{2.388034in}}%
\pgfpathcurveto{\pgfqpoint{0.631419in}{2.382210in}}{\pgfqpoint{0.639319in}{2.378938in}}{\pgfqpoint{0.647555in}{2.378938in}}%
\pgfpathclose%
\pgfusepath{stroke,fill}%
\end{pgfscope}%
\begin{pgfscope}%
\pgfpathrectangle{\pgfqpoint{0.100000in}{0.212622in}}{\pgfqpoint{3.696000in}{3.696000in}}%
\pgfusepath{clip}%
\pgfsetbuttcap%
\pgfsetroundjoin%
\definecolor{currentfill}{rgb}{0.121569,0.466667,0.705882}%
\pgfsetfillcolor{currentfill}%
\pgfsetfillopacity{0.828661}%
\pgfsetlinewidth{1.003750pt}%
\definecolor{currentstroke}{rgb}{0.121569,0.466667,0.705882}%
\pgfsetstrokecolor{currentstroke}%
\pgfsetstrokeopacity{0.828661}%
\pgfsetdash{}{0pt}%
\pgfpathmoveto{\pgfqpoint{0.671537in}{2.372537in}}%
\pgfpathcurveto{\pgfqpoint{0.679773in}{2.372537in}}{\pgfqpoint{0.687673in}{2.375809in}}{\pgfqpoint{0.693497in}{2.381633in}}%
\pgfpathcurveto{\pgfqpoint{0.699321in}{2.387457in}}{\pgfqpoint{0.702593in}{2.395357in}}{\pgfqpoint{0.702593in}{2.403593in}}%
\pgfpathcurveto{\pgfqpoint{0.702593in}{2.411829in}}{\pgfqpoint{0.699321in}{2.419729in}}{\pgfqpoint{0.693497in}{2.425553in}}%
\pgfpathcurveto{\pgfqpoint{0.687673in}{2.431377in}}{\pgfqpoint{0.679773in}{2.434650in}}{\pgfqpoint{0.671537in}{2.434650in}}%
\pgfpathcurveto{\pgfqpoint{0.663301in}{2.434650in}}{\pgfqpoint{0.655401in}{2.431377in}}{\pgfqpoint{0.649577in}{2.425553in}}%
\pgfpathcurveto{\pgfqpoint{0.643753in}{2.419729in}}{\pgfqpoint{0.640480in}{2.411829in}}{\pgfqpoint{0.640480in}{2.403593in}}%
\pgfpathcurveto{\pgfqpoint{0.640480in}{2.395357in}}{\pgfqpoint{0.643753in}{2.387457in}}{\pgfqpoint{0.649577in}{2.381633in}}%
\pgfpathcurveto{\pgfqpoint{0.655401in}{2.375809in}}{\pgfqpoint{0.663301in}{2.372537in}}{\pgfqpoint{0.671537in}{2.372537in}}%
\pgfpathclose%
\pgfusepath{stroke,fill}%
\end{pgfscope}%
\begin{pgfscope}%
\pgfpathrectangle{\pgfqpoint{0.100000in}{0.212622in}}{\pgfqpoint{3.696000in}{3.696000in}}%
\pgfusepath{clip}%
\pgfsetbuttcap%
\pgfsetroundjoin%
\definecolor{currentfill}{rgb}{0.121569,0.466667,0.705882}%
\pgfsetfillcolor{currentfill}%
\pgfsetfillopacity{0.830133}%
\pgfsetlinewidth{1.003750pt}%
\definecolor{currentstroke}{rgb}{0.121569,0.466667,0.705882}%
\pgfsetstrokecolor{currentstroke}%
\pgfsetstrokeopacity{0.830133}%
\pgfsetdash{}{0pt}%
\pgfpathmoveto{\pgfqpoint{0.646372in}{2.413621in}}%
\pgfpathcurveto{\pgfqpoint{0.654609in}{2.413621in}}{\pgfqpoint{0.662509in}{2.416893in}}{\pgfqpoint{0.668333in}{2.422717in}}%
\pgfpathcurveto{\pgfqpoint{0.674157in}{2.428541in}}{\pgfqpoint{0.677429in}{2.436441in}}{\pgfqpoint{0.677429in}{2.444677in}}%
\pgfpathcurveto{\pgfqpoint{0.677429in}{2.452914in}}{\pgfqpoint{0.674157in}{2.460814in}}{\pgfqpoint{0.668333in}{2.466638in}}%
\pgfpathcurveto{\pgfqpoint{0.662509in}{2.472462in}}{\pgfqpoint{0.654609in}{2.475734in}}{\pgfqpoint{0.646372in}{2.475734in}}%
\pgfpathcurveto{\pgfqpoint{0.638136in}{2.475734in}}{\pgfqpoint{0.630236in}{2.472462in}}{\pgfqpoint{0.624412in}{2.466638in}}%
\pgfpathcurveto{\pgfqpoint{0.618588in}{2.460814in}}{\pgfqpoint{0.615316in}{2.452914in}}{\pgfqpoint{0.615316in}{2.444677in}}%
\pgfpathcurveto{\pgfqpoint{0.615316in}{2.436441in}}{\pgfqpoint{0.618588in}{2.428541in}}{\pgfqpoint{0.624412in}{2.422717in}}%
\pgfpathcurveto{\pgfqpoint{0.630236in}{2.416893in}}{\pgfqpoint{0.638136in}{2.413621in}}{\pgfqpoint{0.646372in}{2.413621in}}%
\pgfpathclose%
\pgfusepath{stroke,fill}%
\end{pgfscope}%
\begin{pgfscope}%
\pgfpathrectangle{\pgfqpoint{0.100000in}{0.212622in}}{\pgfqpoint{3.696000in}{3.696000in}}%
\pgfusepath{clip}%
\pgfsetbuttcap%
\pgfsetroundjoin%
\definecolor{currentfill}{rgb}{0.121569,0.466667,0.705882}%
\pgfsetfillcolor{currentfill}%
\pgfsetfillopacity{0.830802}%
\pgfsetlinewidth{1.003750pt}%
\definecolor{currentstroke}{rgb}{0.121569,0.466667,0.705882}%
\pgfsetstrokecolor{currentstroke}%
\pgfsetstrokeopacity{0.830802}%
\pgfsetdash{}{0pt}%
\pgfpathmoveto{\pgfqpoint{0.692175in}{2.361248in}}%
\pgfpathcurveto{\pgfqpoint{0.700411in}{2.361248in}}{\pgfqpoint{0.708311in}{2.364520in}}{\pgfqpoint{0.714135in}{2.370344in}}%
\pgfpathcurveto{\pgfqpoint{0.719959in}{2.376168in}}{\pgfqpoint{0.723231in}{2.384068in}}{\pgfqpoint{0.723231in}{2.392305in}}%
\pgfpathcurveto{\pgfqpoint{0.723231in}{2.400541in}}{\pgfqpoint{0.719959in}{2.408441in}}{\pgfqpoint{0.714135in}{2.414265in}}%
\pgfpathcurveto{\pgfqpoint{0.708311in}{2.420089in}}{\pgfqpoint{0.700411in}{2.423361in}}{\pgfqpoint{0.692175in}{2.423361in}}%
\pgfpathcurveto{\pgfqpoint{0.683939in}{2.423361in}}{\pgfqpoint{0.676039in}{2.420089in}}{\pgfqpoint{0.670215in}{2.414265in}}%
\pgfpathcurveto{\pgfqpoint{0.664391in}{2.408441in}}{\pgfqpoint{0.661118in}{2.400541in}}{\pgfqpoint{0.661118in}{2.392305in}}%
\pgfpathcurveto{\pgfqpoint{0.661118in}{2.384068in}}{\pgfqpoint{0.664391in}{2.376168in}}{\pgfqpoint{0.670215in}{2.370344in}}%
\pgfpathcurveto{\pgfqpoint{0.676039in}{2.364520in}}{\pgfqpoint{0.683939in}{2.361248in}}{\pgfqpoint{0.692175in}{2.361248in}}%
\pgfpathclose%
\pgfusepath{stroke,fill}%
\end{pgfscope}%
\begin{pgfscope}%
\pgfpathrectangle{\pgfqpoint{0.100000in}{0.212622in}}{\pgfqpoint{3.696000in}{3.696000in}}%
\pgfusepath{clip}%
\pgfsetbuttcap%
\pgfsetroundjoin%
\definecolor{currentfill}{rgb}{0.121569,0.466667,0.705882}%
\pgfsetfillcolor{currentfill}%
\pgfsetfillopacity{0.832081}%
\pgfsetlinewidth{1.003750pt}%
\definecolor{currentstroke}{rgb}{0.121569,0.466667,0.705882}%
\pgfsetstrokecolor{currentstroke}%
\pgfsetstrokeopacity{0.832081}%
\pgfsetdash{}{0pt}%
\pgfpathmoveto{\pgfqpoint{2.296532in}{1.789544in}}%
\pgfpathcurveto{\pgfqpoint{2.304769in}{1.789544in}}{\pgfqpoint{2.312669in}{1.792817in}}{\pgfqpoint{2.318493in}{1.798641in}}%
\pgfpathcurveto{\pgfqpoint{2.324316in}{1.804465in}}{\pgfqpoint{2.327589in}{1.812365in}}{\pgfqpoint{2.327589in}{1.820601in}}%
\pgfpathcurveto{\pgfqpoint{2.327589in}{1.828837in}}{\pgfqpoint{2.324316in}{1.836737in}}{\pgfqpoint{2.318493in}{1.842561in}}%
\pgfpathcurveto{\pgfqpoint{2.312669in}{1.848385in}}{\pgfqpoint{2.304769in}{1.851657in}}{\pgfqpoint{2.296532in}{1.851657in}}%
\pgfpathcurveto{\pgfqpoint{2.288296in}{1.851657in}}{\pgfqpoint{2.280396in}{1.848385in}}{\pgfqpoint{2.274572in}{1.842561in}}%
\pgfpathcurveto{\pgfqpoint{2.268748in}{1.836737in}}{\pgfqpoint{2.265476in}{1.828837in}}{\pgfqpoint{2.265476in}{1.820601in}}%
\pgfpathcurveto{\pgfqpoint{2.265476in}{1.812365in}}{\pgfqpoint{2.268748in}{1.804465in}}{\pgfqpoint{2.274572in}{1.798641in}}%
\pgfpathcurveto{\pgfqpoint{2.280396in}{1.792817in}}{\pgfqpoint{2.288296in}{1.789544in}}{\pgfqpoint{2.296532in}{1.789544in}}%
\pgfpathclose%
\pgfusepath{stroke,fill}%
\end{pgfscope}%
\begin{pgfscope}%
\pgfpathrectangle{\pgfqpoint{0.100000in}{0.212622in}}{\pgfqpoint{3.696000in}{3.696000in}}%
\pgfusepath{clip}%
\pgfsetbuttcap%
\pgfsetroundjoin%
\definecolor{currentfill}{rgb}{0.121569,0.466667,0.705882}%
\pgfsetfillcolor{currentfill}%
\pgfsetfillopacity{0.834848}%
\pgfsetlinewidth{1.003750pt}%
\definecolor{currentstroke}{rgb}{0.121569,0.466667,0.705882}%
\pgfsetstrokecolor{currentstroke}%
\pgfsetstrokeopacity{0.834848}%
\pgfsetdash{}{0pt}%
\pgfpathmoveto{\pgfqpoint{0.730617in}{2.344894in}}%
\pgfpathcurveto{\pgfqpoint{0.738853in}{2.344894in}}{\pgfqpoint{0.746753in}{2.348166in}}{\pgfqpoint{0.752577in}{2.353990in}}%
\pgfpathcurveto{\pgfqpoint{0.758401in}{2.359814in}}{\pgfqpoint{0.761674in}{2.367714in}}{\pgfqpoint{0.761674in}{2.375950in}}%
\pgfpathcurveto{\pgfqpoint{0.761674in}{2.384186in}}{\pgfqpoint{0.758401in}{2.392086in}}{\pgfqpoint{0.752577in}{2.397910in}}%
\pgfpathcurveto{\pgfqpoint{0.746753in}{2.403734in}}{\pgfqpoint{0.738853in}{2.407007in}}{\pgfqpoint{0.730617in}{2.407007in}}%
\pgfpathcurveto{\pgfqpoint{0.722381in}{2.407007in}}{\pgfqpoint{0.714481in}{2.403734in}}{\pgfqpoint{0.708657in}{2.397910in}}%
\pgfpathcurveto{\pgfqpoint{0.702833in}{2.392086in}}{\pgfqpoint{0.699561in}{2.384186in}}{\pgfqpoint{0.699561in}{2.375950in}}%
\pgfpathcurveto{\pgfqpoint{0.699561in}{2.367714in}}{\pgfqpoint{0.702833in}{2.359814in}}{\pgfqpoint{0.708657in}{2.353990in}}%
\pgfpathcurveto{\pgfqpoint{0.714481in}{2.348166in}}{\pgfqpoint{0.722381in}{2.344894in}}{\pgfqpoint{0.730617in}{2.344894in}}%
\pgfpathclose%
\pgfusepath{stroke,fill}%
\end{pgfscope}%
\begin{pgfscope}%
\pgfpathrectangle{\pgfqpoint{0.100000in}{0.212622in}}{\pgfqpoint{3.696000in}{3.696000in}}%
\pgfusepath{clip}%
\pgfsetbuttcap%
\pgfsetroundjoin%
\definecolor{currentfill}{rgb}{0.121569,0.466667,0.705882}%
\pgfsetfillcolor{currentfill}%
\pgfsetfillopacity{0.836217}%
\pgfsetlinewidth{1.003750pt}%
\definecolor{currentstroke}{rgb}{0.121569,0.466667,0.705882}%
\pgfsetstrokecolor{currentstroke}%
\pgfsetstrokeopacity{0.836217}%
\pgfsetdash{}{0pt}%
\pgfpathmoveto{\pgfqpoint{0.685285in}{2.406719in}}%
\pgfpathcurveto{\pgfqpoint{0.693521in}{2.406719in}}{\pgfqpoint{0.701421in}{2.409991in}}{\pgfqpoint{0.707245in}{2.415815in}}%
\pgfpathcurveto{\pgfqpoint{0.713069in}{2.421639in}}{\pgfqpoint{0.716342in}{2.429539in}}{\pgfqpoint{0.716342in}{2.437775in}}%
\pgfpathcurveto{\pgfqpoint{0.716342in}{2.446012in}}{\pgfqpoint{0.713069in}{2.453912in}}{\pgfqpoint{0.707245in}{2.459736in}}%
\pgfpathcurveto{\pgfqpoint{0.701421in}{2.465559in}}{\pgfqpoint{0.693521in}{2.468832in}}{\pgfqpoint{0.685285in}{2.468832in}}%
\pgfpathcurveto{\pgfqpoint{0.677049in}{2.468832in}}{\pgfqpoint{0.669149in}{2.465559in}}{\pgfqpoint{0.663325in}{2.459736in}}%
\pgfpathcurveto{\pgfqpoint{0.657501in}{2.453912in}}{\pgfqpoint{0.654229in}{2.446012in}}{\pgfqpoint{0.654229in}{2.437775in}}%
\pgfpathcurveto{\pgfqpoint{0.654229in}{2.429539in}}{\pgfqpoint{0.657501in}{2.421639in}}{\pgfqpoint{0.663325in}{2.415815in}}%
\pgfpathcurveto{\pgfqpoint{0.669149in}{2.409991in}}{\pgfqpoint{0.677049in}{2.406719in}}{\pgfqpoint{0.685285in}{2.406719in}}%
\pgfpathclose%
\pgfusepath{stroke,fill}%
\end{pgfscope}%
\begin{pgfscope}%
\pgfpathrectangle{\pgfqpoint{0.100000in}{0.212622in}}{\pgfqpoint{3.696000in}{3.696000in}}%
\pgfusepath{clip}%
\pgfsetbuttcap%
\pgfsetroundjoin%
\definecolor{currentfill}{rgb}{0.121569,0.466667,0.705882}%
\pgfsetfillcolor{currentfill}%
\pgfsetfillopacity{0.838938}%
\pgfsetlinewidth{1.003750pt}%
\definecolor{currentstroke}{rgb}{0.121569,0.466667,0.705882}%
\pgfsetstrokecolor{currentstroke}%
\pgfsetstrokeopacity{0.838938}%
\pgfsetdash{}{0pt}%
\pgfpathmoveto{\pgfqpoint{0.765966in}{2.334924in}}%
\pgfpathcurveto{\pgfqpoint{0.774203in}{2.334924in}}{\pgfqpoint{0.782103in}{2.338196in}}{\pgfqpoint{0.787927in}{2.344020in}}%
\pgfpathcurveto{\pgfqpoint{0.793751in}{2.349844in}}{\pgfqpoint{0.797023in}{2.357744in}}{\pgfqpoint{0.797023in}{2.365980in}}%
\pgfpathcurveto{\pgfqpoint{0.797023in}{2.374217in}}{\pgfqpoint{0.793751in}{2.382117in}}{\pgfqpoint{0.787927in}{2.387941in}}%
\pgfpathcurveto{\pgfqpoint{0.782103in}{2.393764in}}{\pgfqpoint{0.774203in}{2.397037in}}{\pgfqpoint{0.765966in}{2.397037in}}%
\pgfpathcurveto{\pgfqpoint{0.757730in}{2.397037in}}{\pgfqpoint{0.749830in}{2.393764in}}{\pgfqpoint{0.744006in}{2.387941in}}%
\pgfpathcurveto{\pgfqpoint{0.738182in}{2.382117in}}{\pgfqpoint{0.734910in}{2.374217in}}{\pgfqpoint{0.734910in}{2.365980in}}%
\pgfpathcurveto{\pgfqpoint{0.734910in}{2.357744in}}{\pgfqpoint{0.738182in}{2.349844in}}{\pgfqpoint{0.744006in}{2.344020in}}%
\pgfpathcurveto{\pgfqpoint{0.749830in}{2.338196in}}{\pgfqpoint{0.757730in}{2.334924in}}{\pgfqpoint{0.765966in}{2.334924in}}%
\pgfpathclose%
\pgfusepath{stroke,fill}%
\end{pgfscope}%
\begin{pgfscope}%
\pgfpathrectangle{\pgfqpoint{0.100000in}{0.212622in}}{\pgfqpoint{3.696000in}{3.696000in}}%
\pgfusepath{clip}%
\pgfsetbuttcap%
\pgfsetroundjoin%
\definecolor{currentfill}{rgb}{0.121569,0.466667,0.705882}%
\pgfsetfillcolor{currentfill}%
\pgfsetfillopacity{0.841145}%
\pgfsetlinewidth{1.003750pt}%
\definecolor{currentstroke}{rgb}{0.121569,0.466667,0.705882}%
\pgfsetstrokecolor{currentstroke}%
\pgfsetstrokeopacity{0.841145}%
\pgfsetdash{}{0pt}%
\pgfpathmoveto{\pgfqpoint{2.309858in}{1.787128in}}%
\pgfpathcurveto{\pgfqpoint{2.318095in}{1.787128in}}{\pgfqpoint{2.325995in}{1.790400in}}{\pgfqpoint{2.331819in}{1.796224in}}%
\pgfpathcurveto{\pgfqpoint{2.337643in}{1.802048in}}{\pgfqpoint{2.340915in}{1.809948in}}{\pgfqpoint{2.340915in}{1.818184in}}%
\pgfpathcurveto{\pgfqpoint{2.340915in}{1.826420in}}{\pgfqpoint{2.337643in}{1.834320in}}{\pgfqpoint{2.331819in}{1.840144in}}%
\pgfpathcurveto{\pgfqpoint{2.325995in}{1.845968in}}{\pgfqpoint{2.318095in}{1.849241in}}{\pgfqpoint{2.309858in}{1.849241in}}%
\pgfpathcurveto{\pgfqpoint{2.301622in}{1.849241in}}{\pgfqpoint{2.293722in}{1.845968in}}{\pgfqpoint{2.287898in}{1.840144in}}%
\pgfpathcurveto{\pgfqpoint{2.282074in}{1.834320in}}{\pgfqpoint{2.278802in}{1.826420in}}{\pgfqpoint{2.278802in}{1.818184in}}%
\pgfpathcurveto{\pgfqpoint{2.278802in}{1.809948in}}{\pgfqpoint{2.282074in}{1.802048in}}{\pgfqpoint{2.287898in}{1.796224in}}%
\pgfpathcurveto{\pgfqpoint{2.293722in}{1.790400in}}{\pgfqpoint{2.301622in}{1.787128in}}{\pgfqpoint{2.309858in}{1.787128in}}%
\pgfpathclose%
\pgfusepath{stroke,fill}%
\end{pgfscope}%
\begin{pgfscope}%
\pgfpathrectangle{\pgfqpoint{0.100000in}{0.212622in}}{\pgfqpoint{3.696000in}{3.696000in}}%
\pgfusepath{clip}%
\pgfsetbuttcap%
\pgfsetroundjoin%
\definecolor{currentfill}{rgb}{0.121569,0.466667,0.705882}%
\pgfsetfillcolor{currentfill}%
\pgfsetfillopacity{0.841975}%
\pgfsetlinewidth{1.003750pt}%
\definecolor{currentstroke}{rgb}{0.121569,0.466667,0.705882}%
\pgfsetstrokecolor{currentstroke}%
\pgfsetstrokeopacity{0.841975}%
\pgfsetdash{}{0pt}%
\pgfpathmoveto{\pgfqpoint{0.793873in}{2.318654in}}%
\pgfpathcurveto{\pgfqpoint{0.802109in}{2.318654in}}{\pgfqpoint{0.810009in}{2.321926in}}{\pgfqpoint{0.815833in}{2.327750in}}%
\pgfpathcurveto{\pgfqpoint{0.821657in}{2.333574in}}{\pgfqpoint{0.824930in}{2.341474in}}{\pgfqpoint{0.824930in}{2.349710in}}%
\pgfpathcurveto{\pgfqpoint{0.824930in}{2.357946in}}{\pgfqpoint{0.821657in}{2.365847in}}{\pgfqpoint{0.815833in}{2.371670in}}%
\pgfpathcurveto{\pgfqpoint{0.810009in}{2.377494in}}{\pgfqpoint{0.802109in}{2.380767in}}{\pgfqpoint{0.793873in}{2.380767in}}%
\pgfpathcurveto{\pgfqpoint{0.785637in}{2.380767in}}{\pgfqpoint{0.777737in}{2.377494in}}{\pgfqpoint{0.771913in}{2.371670in}}%
\pgfpathcurveto{\pgfqpoint{0.766089in}{2.365847in}}{\pgfqpoint{0.762817in}{2.357946in}}{\pgfqpoint{0.762817in}{2.349710in}}%
\pgfpathcurveto{\pgfqpoint{0.762817in}{2.341474in}}{\pgfqpoint{0.766089in}{2.333574in}}{\pgfqpoint{0.771913in}{2.327750in}}%
\pgfpathcurveto{\pgfqpoint{0.777737in}{2.321926in}}{\pgfqpoint{0.785637in}{2.318654in}}{\pgfqpoint{0.793873in}{2.318654in}}%
\pgfpathclose%
\pgfusepath{stroke,fill}%
\end{pgfscope}%
\begin{pgfscope}%
\pgfpathrectangle{\pgfqpoint{0.100000in}{0.212622in}}{\pgfqpoint{3.696000in}{3.696000in}}%
\pgfusepath{clip}%
\pgfsetbuttcap%
\pgfsetroundjoin%
\definecolor{currentfill}{rgb}{0.121569,0.466667,0.705882}%
\pgfsetfillcolor{currentfill}%
\pgfsetfillopacity{0.843393}%
\pgfsetlinewidth{1.003750pt}%
\definecolor{currentstroke}{rgb}{0.121569,0.466667,0.705882}%
\pgfsetstrokecolor{currentstroke}%
\pgfsetstrokeopacity{0.843393}%
\pgfsetdash{}{0pt}%
\pgfpathmoveto{\pgfqpoint{0.726441in}{2.403253in}}%
\pgfpathcurveto{\pgfqpoint{0.734677in}{2.403253in}}{\pgfqpoint{0.742577in}{2.406526in}}{\pgfqpoint{0.748401in}{2.412350in}}%
\pgfpathcurveto{\pgfqpoint{0.754225in}{2.418174in}}{\pgfqpoint{0.757497in}{2.426074in}}{\pgfqpoint{0.757497in}{2.434310in}}%
\pgfpathcurveto{\pgfqpoint{0.757497in}{2.442546in}}{\pgfqpoint{0.754225in}{2.450446in}}{\pgfqpoint{0.748401in}{2.456270in}}%
\pgfpathcurveto{\pgfqpoint{0.742577in}{2.462094in}}{\pgfqpoint{0.734677in}{2.465366in}}{\pgfqpoint{0.726441in}{2.465366in}}%
\pgfpathcurveto{\pgfqpoint{0.718204in}{2.465366in}}{\pgfqpoint{0.710304in}{2.462094in}}{\pgfqpoint{0.704481in}{2.456270in}}%
\pgfpathcurveto{\pgfqpoint{0.698657in}{2.450446in}}{\pgfqpoint{0.695384in}{2.442546in}}{\pgfqpoint{0.695384in}{2.434310in}}%
\pgfpathcurveto{\pgfqpoint{0.695384in}{2.426074in}}{\pgfqpoint{0.698657in}{2.418174in}}{\pgfqpoint{0.704481in}{2.412350in}}%
\pgfpathcurveto{\pgfqpoint{0.710304in}{2.406526in}}{\pgfqpoint{0.718204in}{2.403253in}}{\pgfqpoint{0.726441in}{2.403253in}}%
\pgfpathclose%
\pgfusepath{stroke,fill}%
\end{pgfscope}%
\begin{pgfscope}%
\pgfpathrectangle{\pgfqpoint{0.100000in}{0.212622in}}{\pgfqpoint{3.696000in}{3.696000in}}%
\pgfusepath{clip}%
\pgfsetbuttcap%
\pgfsetroundjoin%
\definecolor{currentfill}{rgb}{0.121569,0.466667,0.705882}%
\pgfsetfillcolor{currentfill}%
\pgfsetfillopacity{0.844774}%
\pgfsetlinewidth{1.003750pt}%
\definecolor{currentstroke}{rgb}{0.121569,0.466667,0.705882}%
\pgfsetstrokecolor{currentstroke}%
\pgfsetstrokeopacity{0.844774}%
\pgfsetdash{}{0pt}%
\pgfpathmoveto{\pgfqpoint{0.816288in}{2.312187in}}%
\pgfpathcurveto{\pgfqpoint{0.824524in}{2.312187in}}{\pgfqpoint{0.832424in}{2.315459in}}{\pgfqpoint{0.838248in}{2.321283in}}%
\pgfpathcurveto{\pgfqpoint{0.844072in}{2.327107in}}{\pgfqpoint{0.847344in}{2.335007in}}{\pgfqpoint{0.847344in}{2.343244in}}%
\pgfpathcurveto{\pgfqpoint{0.847344in}{2.351480in}}{\pgfqpoint{0.844072in}{2.359380in}}{\pgfqpoint{0.838248in}{2.365204in}}%
\pgfpathcurveto{\pgfqpoint{0.832424in}{2.371028in}}{\pgfqpoint{0.824524in}{2.374300in}}{\pgfqpoint{0.816288in}{2.374300in}}%
\pgfpathcurveto{\pgfqpoint{0.808051in}{2.374300in}}{\pgfqpoint{0.800151in}{2.371028in}}{\pgfqpoint{0.794327in}{2.365204in}}%
\pgfpathcurveto{\pgfqpoint{0.788504in}{2.359380in}}{\pgfqpoint{0.785231in}{2.351480in}}{\pgfqpoint{0.785231in}{2.343244in}}%
\pgfpathcurveto{\pgfqpoint{0.785231in}{2.335007in}}{\pgfqpoint{0.788504in}{2.327107in}}{\pgfqpoint{0.794327in}{2.321283in}}%
\pgfpathcurveto{\pgfqpoint{0.800151in}{2.315459in}}{\pgfqpoint{0.808051in}{2.312187in}}{\pgfqpoint{0.816288in}{2.312187in}}%
\pgfpathclose%
\pgfusepath{stroke,fill}%
\end{pgfscope}%
\begin{pgfscope}%
\pgfpathrectangle{\pgfqpoint{0.100000in}{0.212622in}}{\pgfqpoint{3.696000in}{3.696000in}}%
\pgfusepath{clip}%
\pgfsetbuttcap%
\pgfsetroundjoin%
\definecolor{currentfill}{rgb}{0.121569,0.466667,0.705882}%
\pgfsetfillcolor{currentfill}%
\pgfsetfillopacity{0.846813}%
\pgfsetlinewidth{1.003750pt}%
\definecolor{currentstroke}{rgb}{0.121569,0.466667,0.705882}%
\pgfsetstrokecolor{currentstroke}%
\pgfsetstrokeopacity{0.846813}%
\pgfsetdash{}{0pt}%
\pgfpathmoveto{\pgfqpoint{0.835285in}{2.302095in}}%
\pgfpathcurveto{\pgfqpoint{0.843522in}{2.302095in}}{\pgfqpoint{0.851422in}{2.305367in}}{\pgfqpoint{0.857246in}{2.311191in}}%
\pgfpathcurveto{\pgfqpoint{0.863070in}{2.317015in}}{\pgfqpoint{0.866342in}{2.324915in}}{\pgfqpoint{0.866342in}{2.333151in}}%
\pgfpathcurveto{\pgfqpoint{0.866342in}{2.341388in}}{\pgfqpoint{0.863070in}{2.349288in}}{\pgfqpoint{0.857246in}{2.355112in}}%
\pgfpathcurveto{\pgfqpoint{0.851422in}{2.360936in}}{\pgfqpoint{0.843522in}{2.364208in}}{\pgfqpoint{0.835285in}{2.364208in}}%
\pgfpathcurveto{\pgfqpoint{0.827049in}{2.364208in}}{\pgfqpoint{0.819149in}{2.360936in}}{\pgfqpoint{0.813325in}{2.355112in}}%
\pgfpathcurveto{\pgfqpoint{0.807501in}{2.349288in}}{\pgfqpoint{0.804229in}{2.341388in}}{\pgfqpoint{0.804229in}{2.333151in}}%
\pgfpathcurveto{\pgfqpoint{0.804229in}{2.324915in}}{\pgfqpoint{0.807501in}{2.317015in}}{\pgfqpoint{0.813325in}{2.311191in}}%
\pgfpathcurveto{\pgfqpoint{0.819149in}{2.305367in}}{\pgfqpoint{0.827049in}{2.302095in}}{\pgfqpoint{0.835285in}{2.302095in}}%
\pgfpathclose%
\pgfusepath{stroke,fill}%
\end{pgfscope}%
\begin{pgfscope}%
\pgfpathrectangle{\pgfqpoint{0.100000in}{0.212622in}}{\pgfqpoint{3.696000in}{3.696000in}}%
\pgfusepath{clip}%
\pgfsetbuttcap%
\pgfsetroundjoin%
\definecolor{currentfill}{rgb}{0.121569,0.466667,0.705882}%
\pgfsetfillcolor{currentfill}%
\pgfsetfillopacity{0.847419}%
\pgfsetlinewidth{1.003750pt}%
\definecolor{currentstroke}{rgb}{0.121569,0.466667,0.705882}%
\pgfsetstrokecolor{currentstroke}%
\pgfsetstrokeopacity{0.847419}%
\pgfsetdash{}{0pt}%
\pgfpathmoveto{\pgfqpoint{0.748659in}{2.400436in}}%
\pgfpathcurveto{\pgfqpoint{0.756895in}{2.400436in}}{\pgfqpoint{0.764795in}{2.403709in}}{\pgfqpoint{0.770619in}{2.409533in}}%
\pgfpathcurveto{\pgfqpoint{0.776443in}{2.415357in}}{\pgfqpoint{0.779716in}{2.423257in}}{\pgfqpoint{0.779716in}{2.431493in}}%
\pgfpathcurveto{\pgfqpoint{0.779716in}{2.439729in}}{\pgfqpoint{0.776443in}{2.447629in}}{\pgfqpoint{0.770619in}{2.453453in}}%
\pgfpathcurveto{\pgfqpoint{0.764795in}{2.459277in}}{\pgfqpoint{0.756895in}{2.462549in}}{\pgfqpoint{0.748659in}{2.462549in}}%
\pgfpathcurveto{\pgfqpoint{0.740423in}{2.462549in}}{\pgfqpoint{0.732523in}{2.459277in}}{\pgfqpoint{0.726699in}{2.453453in}}%
\pgfpathcurveto{\pgfqpoint{0.720875in}{2.447629in}}{\pgfqpoint{0.717603in}{2.439729in}}{\pgfqpoint{0.717603in}{2.431493in}}%
\pgfpathcurveto{\pgfqpoint{0.717603in}{2.423257in}}{\pgfqpoint{0.720875in}{2.415357in}}{\pgfqpoint{0.726699in}{2.409533in}}%
\pgfpathcurveto{\pgfqpoint{0.732523in}{2.403709in}}{\pgfqpoint{0.740423in}{2.400436in}}{\pgfqpoint{0.748659in}{2.400436in}}%
\pgfpathclose%
\pgfusepath{stroke,fill}%
\end{pgfscope}%
\begin{pgfscope}%
\pgfpathrectangle{\pgfqpoint{0.100000in}{0.212622in}}{\pgfqpoint{3.696000in}{3.696000in}}%
\pgfusepath{clip}%
\pgfsetbuttcap%
\pgfsetroundjoin%
\definecolor{currentfill}{rgb}{0.121569,0.466667,0.705882}%
\pgfsetfillcolor{currentfill}%
\pgfsetfillopacity{0.848970}%
\pgfsetlinewidth{1.003750pt}%
\definecolor{currentstroke}{rgb}{0.121569,0.466667,0.705882}%
\pgfsetstrokecolor{currentstroke}%
\pgfsetstrokeopacity{0.848970}%
\pgfsetdash{}{0pt}%
\pgfpathmoveto{\pgfqpoint{2.314150in}{1.773805in}}%
\pgfpathcurveto{\pgfqpoint{2.322386in}{1.773805in}}{\pgfqpoint{2.330286in}{1.777078in}}{\pgfqpoint{2.336110in}{1.782901in}}%
\pgfpathcurveto{\pgfqpoint{2.341934in}{1.788725in}}{\pgfqpoint{2.345206in}{1.796625in}}{\pgfqpoint{2.345206in}{1.804862in}}%
\pgfpathcurveto{\pgfqpoint{2.345206in}{1.813098in}}{\pgfqpoint{2.341934in}{1.820998in}}{\pgfqpoint{2.336110in}{1.826822in}}%
\pgfpathcurveto{\pgfqpoint{2.330286in}{1.832646in}}{\pgfqpoint{2.322386in}{1.835918in}}{\pgfqpoint{2.314150in}{1.835918in}}%
\pgfpathcurveto{\pgfqpoint{2.305913in}{1.835918in}}{\pgfqpoint{2.298013in}{1.832646in}}{\pgfqpoint{2.292189in}{1.826822in}}%
\pgfpathcurveto{\pgfqpoint{2.286365in}{1.820998in}}{\pgfqpoint{2.283093in}{1.813098in}}{\pgfqpoint{2.283093in}{1.804862in}}%
\pgfpathcurveto{\pgfqpoint{2.283093in}{1.796625in}}{\pgfqpoint{2.286365in}{1.788725in}}{\pgfqpoint{2.292189in}{1.782901in}}%
\pgfpathcurveto{\pgfqpoint{2.298013in}{1.777078in}}{\pgfqpoint{2.305913in}{1.773805in}}{\pgfqpoint{2.314150in}{1.773805in}}%
\pgfpathclose%
\pgfusepath{stroke,fill}%
\end{pgfscope}%
\begin{pgfscope}%
\pgfpathrectangle{\pgfqpoint{0.100000in}{0.212622in}}{\pgfqpoint{3.696000in}{3.696000in}}%
\pgfusepath{clip}%
\pgfsetbuttcap%
\pgfsetroundjoin%
\definecolor{currentfill}{rgb}{0.121569,0.466667,0.705882}%
\pgfsetfillcolor{currentfill}%
\pgfsetfillopacity{0.849508}%
\pgfsetlinewidth{1.003750pt}%
\definecolor{currentstroke}{rgb}{0.121569,0.466667,0.705882}%
\pgfsetstrokecolor{currentstroke}%
\pgfsetstrokeopacity{0.849508}%
\pgfsetdash{}{0pt}%
\pgfpathmoveto{\pgfqpoint{0.760922in}{2.398046in}}%
\pgfpathcurveto{\pgfqpoint{0.769158in}{2.398046in}}{\pgfqpoint{0.777058in}{2.401318in}}{\pgfqpoint{0.782882in}{2.407142in}}%
\pgfpathcurveto{\pgfqpoint{0.788706in}{2.412966in}}{\pgfqpoint{0.791978in}{2.420866in}}{\pgfqpoint{0.791978in}{2.429102in}}%
\pgfpathcurveto{\pgfqpoint{0.791978in}{2.437338in}}{\pgfqpoint{0.788706in}{2.445238in}}{\pgfqpoint{0.782882in}{2.451062in}}%
\pgfpathcurveto{\pgfqpoint{0.777058in}{2.456886in}}{\pgfqpoint{0.769158in}{2.460159in}}{\pgfqpoint{0.760922in}{2.460159in}}%
\pgfpathcurveto{\pgfqpoint{0.752686in}{2.460159in}}{\pgfqpoint{0.744786in}{2.456886in}}{\pgfqpoint{0.738962in}{2.451062in}}%
\pgfpathcurveto{\pgfqpoint{0.733138in}{2.445238in}}{\pgfqpoint{0.729865in}{2.437338in}}{\pgfqpoint{0.729865in}{2.429102in}}%
\pgfpathcurveto{\pgfqpoint{0.729865in}{2.420866in}}{\pgfqpoint{0.733138in}{2.412966in}}{\pgfqpoint{0.738962in}{2.407142in}}%
\pgfpathcurveto{\pgfqpoint{0.744786in}{2.401318in}}{\pgfqpoint{0.752686in}{2.398046in}}{\pgfqpoint{0.760922in}{2.398046in}}%
\pgfpathclose%
\pgfusepath{stroke,fill}%
\end{pgfscope}%
\begin{pgfscope}%
\pgfpathrectangle{\pgfqpoint{0.100000in}{0.212622in}}{\pgfqpoint{3.696000in}{3.696000in}}%
\pgfusepath{clip}%
\pgfsetbuttcap%
\pgfsetroundjoin%
\definecolor{currentfill}{rgb}{0.121569,0.466667,0.705882}%
\pgfsetfillcolor{currentfill}%
\pgfsetfillopacity{0.851182}%
\pgfsetlinewidth{1.003750pt}%
\definecolor{currentstroke}{rgb}{0.121569,0.466667,0.705882}%
\pgfsetstrokecolor{currentstroke}%
\pgfsetstrokeopacity{0.851182}%
\pgfsetdash{}{0pt}%
\pgfpathmoveto{\pgfqpoint{0.870301in}{2.289966in}}%
\pgfpathcurveto{\pgfqpoint{0.878537in}{2.289966in}}{\pgfqpoint{0.886437in}{2.293238in}}{\pgfqpoint{0.892261in}{2.299062in}}%
\pgfpathcurveto{\pgfqpoint{0.898085in}{2.304886in}}{\pgfqpoint{0.901357in}{2.312786in}}{\pgfqpoint{0.901357in}{2.321022in}}%
\pgfpathcurveto{\pgfqpoint{0.901357in}{2.329259in}}{\pgfqpoint{0.898085in}{2.337159in}}{\pgfqpoint{0.892261in}{2.342983in}}%
\pgfpathcurveto{\pgfqpoint{0.886437in}{2.348807in}}{\pgfqpoint{0.878537in}{2.352079in}}{\pgfqpoint{0.870301in}{2.352079in}}%
\pgfpathcurveto{\pgfqpoint{0.862064in}{2.352079in}}{\pgfqpoint{0.854164in}{2.348807in}}{\pgfqpoint{0.848340in}{2.342983in}}%
\pgfpathcurveto{\pgfqpoint{0.842517in}{2.337159in}}{\pgfqpoint{0.839244in}{2.329259in}}{\pgfqpoint{0.839244in}{2.321022in}}%
\pgfpathcurveto{\pgfqpoint{0.839244in}{2.312786in}}{\pgfqpoint{0.842517in}{2.304886in}}{\pgfqpoint{0.848340in}{2.299062in}}%
\pgfpathcurveto{\pgfqpoint{0.854164in}{2.293238in}}{\pgfqpoint{0.862064in}{2.289966in}}{\pgfqpoint{0.870301in}{2.289966in}}%
\pgfpathclose%
\pgfusepath{stroke,fill}%
\end{pgfscope}%
\begin{pgfscope}%
\pgfpathrectangle{\pgfqpoint{0.100000in}{0.212622in}}{\pgfqpoint{3.696000in}{3.696000in}}%
\pgfusepath{clip}%
\pgfsetbuttcap%
\pgfsetroundjoin%
\definecolor{currentfill}{rgb}{0.121569,0.466667,0.705882}%
\pgfsetfillcolor{currentfill}%
\pgfsetfillopacity{0.851995}%
\pgfsetlinewidth{1.003750pt}%
\definecolor{currentstroke}{rgb}{0.121569,0.466667,0.705882}%
\pgfsetstrokecolor{currentstroke}%
\pgfsetstrokeopacity{0.851995}%
\pgfsetdash{}{0pt}%
\pgfpathmoveto{\pgfqpoint{0.776122in}{2.395442in}}%
\pgfpathcurveto{\pgfqpoint{0.784358in}{2.395442in}}{\pgfqpoint{0.792258in}{2.398714in}}{\pgfqpoint{0.798082in}{2.404538in}}%
\pgfpathcurveto{\pgfqpoint{0.803906in}{2.410362in}}{\pgfqpoint{0.807178in}{2.418262in}}{\pgfqpoint{0.807178in}{2.426498in}}%
\pgfpathcurveto{\pgfqpoint{0.807178in}{2.434735in}}{\pgfqpoint{0.803906in}{2.442635in}}{\pgfqpoint{0.798082in}{2.448459in}}%
\pgfpathcurveto{\pgfqpoint{0.792258in}{2.454283in}}{\pgfqpoint{0.784358in}{2.457555in}}{\pgfqpoint{0.776122in}{2.457555in}}%
\pgfpathcurveto{\pgfqpoint{0.767886in}{2.457555in}}{\pgfqpoint{0.759986in}{2.454283in}}{\pgfqpoint{0.754162in}{2.448459in}}%
\pgfpathcurveto{\pgfqpoint{0.748338in}{2.442635in}}{\pgfqpoint{0.745065in}{2.434735in}}{\pgfqpoint{0.745065in}{2.426498in}}%
\pgfpathcurveto{\pgfqpoint{0.745065in}{2.418262in}}{\pgfqpoint{0.748338in}{2.410362in}}{\pgfqpoint{0.754162in}{2.404538in}}%
\pgfpathcurveto{\pgfqpoint{0.759986in}{2.398714in}}{\pgfqpoint{0.767886in}{2.395442in}}{\pgfqpoint{0.776122in}{2.395442in}}%
\pgfpathclose%
\pgfusepath{stroke,fill}%
\end{pgfscope}%
\begin{pgfscope}%
\pgfpathrectangle{\pgfqpoint{0.100000in}{0.212622in}}{\pgfqpoint{3.696000in}{3.696000in}}%
\pgfusepath{clip}%
\pgfsetbuttcap%
\pgfsetroundjoin%
\definecolor{currentfill}{rgb}{0.121569,0.466667,0.705882}%
\pgfsetfillcolor{currentfill}%
\pgfsetfillopacity{0.855054}%
\pgfsetlinewidth{1.003750pt}%
\definecolor{currentstroke}{rgb}{0.121569,0.466667,0.705882}%
\pgfsetstrokecolor{currentstroke}%
\pgfsetstrokeopacity{0.855054}%
\pgfsetdash{}{0pt}%
\pgfpathmoveto{\pgfqpoint{0.903176in}{2.277507in}}%
\pgfpathcurveto{\pgfqpoint{0.911412in}{2.277507in}}{\pgfqpoint{0.919312in}{2.280780in}}{\pgfqpoint{0.925136in}{2.286604in}}%
\pgfpathcurveto{\pgfqpoint{0.930960in}{2.292428in}}{\pgfqpoint{0.934232in}{2.300328in}}{\pgfqpoint{0.934232in}{2.308564in}}%
\pgfpathcurveto{\pgfqpoint{0.934232in}{2.316800in}}{\pgfqpoint{0.930960in}{2.324700in}}{\pgfqpoint{0.925136in}{2.330524in}}%
\pgfpathcurveto{\pgfqpoint{0.919312in}{2.336348in}}{\pgfqpoint{0.911412in}{2.339620in}}{\pgfqpoint{0.903176in}{2.339620in}}%
\pgfpathcurveto{\pgfqpoint{0.894940in}{2.339620in}}{\pgfqpoint{0.887040in}{2.336348in}}{\pgfqpoint{0.881216in}{2.330524in}}%
\pgfpathcurveto{\pgfqpoint{0.875392in}{2.324700in}}{\pgfqpoint{0.872119in}{2.316800in}}{\pgfqpoint{0.872119in}{2.308564in}}%
\pgfpathcurveto{\pgfqpoint{0.872119in}{2.300328in}}{\pgfqpoint{0.875392in}{2.292428in}}{\pgfqpoint{0.881216in}{2.286604in}}%
\pgfpathcurveto{\pgfqpoint{0.887040in}{2.280780in}}{\pgfqpoint{0.894940in}{2.277507in}}{\pgfqpoint{0.903176in}{2.277507in}}%
\pgfpathclose%
\pgfusepath{stroke,fill}%
\end{pgfscope}%
\begin{pgfscope}%
\pgfpathrectangle{\pgfqpoint{0.100000in}{0.212622in}}{\pgfqpoint{3.696000in}{3.696000in}}%
\pgfusepath{clip}%
\pgfsetbuttcap%
\pgfsetroundjoin%
\definecolor{currentfill}{rgb}{0.121569,0.466667,0.705882}%
\pgfsetfillcolor{currentfill}%
\pgfsetfillopacity{0.856150}%
\pgfsetlinewidth{1.003750pt}%
\definecolor{currentstroke}{rgb}{0.121569,0.466667,0.705882}%
\pgfsetstrokecolor{currentstroke}%
\pgfsetstrokeopacity{0.856150}%
\pgfsetdash{}{0pt}%
\pgfpathmoveto{\pgfqpoint{0.798781in}{2.393533in}}%
\pgfpathcurveto{\pgfqpoint{0.807017in}{2.393533in}}{\pgfqpoint{0.814917in}{2.396806in}}{\pgfqpoint{0.820741in}{2.402630in}}%
\pgfpathcurveto{\pgfqpoint{0.826565in}{2.408454in}}{\pgfqpoint{0.829837in}{2.416354in}}{\pgfqpoint{0.829837in}{2.424590in}}%
\pgfpathcurveto{\pgfqpoint{0.829837in}{2.432826in}}{\pgfqpoint{0.826565in}{2.440726in}}{\pgfqpoint{0.820741in}{2.446550in}}%
\pgfpathcurveto{\pgfqpoint{0.814917in}{2.452374in}}{\pgfqpoint{0.807017in}{2.455646in}}{\pgfqpoint{0.798781in}{2.455646in}}%
\pgfpathcurveto{\pgfqpoint{0.790544in}{2.455646in}}{\pgfqpoint{0.782644in}{2.452374in}}{\pgfqpoint{0.776820in}{2.446550in}}%
\pgfpathcurveto{\pgfqpoint{0.770997in}{2.440726in}}{\pgfqpoint{0.767724in}{2.432826in}}{\pgfqpoint{0.767724in}{2.424590in}}%
\pgfpathcurveto{\pgfqpoint{0.767724in}{2.416354in}}{\pgfqpoint{0.770997in}{2.408454in}}{\pgfqpoint{0.776820in}{2.402630in}}%
\pgfpathcurveto{\pgfqpoint{0.782644in}{2.396806in}}{\pgfqpoint{0.790544in}{2.393533in}}{\pgfqpoint{0.798781in}{2.393533in}}%
\pgfpathclose%
\pgfusepath{stroke,fill}%
\end{pgfscope}%
\begin{pgfscope}%
\pgfpathrectangle{\pgfqpoint{0.100000in}{0.212622in}}{\pgfqpoint{3.696000in}{3.696000in}}%
\pgfusepath{clip}%
\pgfsetbuttcap%
\pgfsetroundjoin%
\definecolor{currentfill}{rgb}{0.121569,0.466667,0.705882}%
\pgfsetfillcolor{currentfill}%
\pgfsetfillopacity{0.857688}%
\pgfsetlinewidth{1.003750pt}%
\definecolor{currentstroke}{rgb}{0.121569,0.466667,0.705882}%
\pgfsetstrokecolor{currentstroke}%
\pgfsetstrokeopacity{0.857688}%
\pgfsetdash{}{0pt}%
\pgfpathmoveto{\pgfqpoint{0.927971in}{2.264844in}}%
\pgfpathcurveto{\pgfqpoint{0.936208in}{2.264844in}}{\pgfqpoint{0.944108in}{2.268117in}}{\pgfqpoint{0.949931in}{2.273941in}}%
\pgfpathcurveto{\pgfqpoint{0.955755in}{2.279764in}}{\pgfqpoint{0.959028in}{2.287664in}}{\pgfqpoint{0.959028in}{2.295901in}}%
\pgfpathcurveto{\pgfqpoint{0.959028in}{2.304137in}}{\pgfqpoint{0.955755in}{2.312037in}}{\pgfqpoint{0.949931in}{2.317861in}}%
\pgfpathcurveto{\pgfqpoint{0.944108in}{2.323685in}}{\pgfqpoint{0.936208in}{2.326957in}}{\pgfqpoint{0.927971in}{2.326957in}}%
\pgfpathcurveto{\pgfqpoint{0.919735in}{2.326957in}}{\pgfqpoint{0.911835in}{2.323685in}}{\pgfqpoint{0.906011in}{2.317861in}}%
\pgfpathcurveto{\pgfqpoint{0.900187in}{2.312037in}}{\pgfqpoint{0.896915in}{2.304137in}}{\pgfqpoint{0.896915in}{2.295901in}}%
\pgfpathcurveto{\pgfqpoint{0.896915in}{2.287664in}}{\pgfqpoint{0.900187in}{2.279764in}}{\pgfqpoint{0.906011in}{2.273941in}}%
\pgfpathcurveto{\pgfqpoint{0.911835in}{2.268117in}}{\pgfqpoint{0.919735in}{2.264844in}}{\pgfqpoint{0.927971in}{2.264844in}}%
\pgfpathclose%
\pgfusepath{stroke,fill}%
\end{pgfscope}%
\begin{pgfscope}%
\pgfpathrectangle{\pgfqpoint{0.100000in}{0.212622in}}{\pgfqpoint{3.696000in}{3.696000in}}%
\pgfusepath{clip}%
\pgfsetbuttcap%
\pgfsetroundjoin%
\definecolor{currentfill}{rgb}{0.121569,0.466667,0.705882}%
\pgfsetfillcolor{currentfill}%
\pgfsetfillopacity{0.860147}%
\pgfsetlinewidth{1.003750pt}%
\definecolor{currentstroke}{rgb}{0.121569,0.466667,0.705882}%
\pgfsetstrokecolor{currentstroke}%
\pgfsetstrokeopacity{0.860147}%
\pgfsetdash{}{0pt}%
\pgfpathmoveto{\pgfqpoint{0.947223in}{2.259096in}}%
\pgfpathcurveto{\pgfqpoint{0.955459in}{2.259096in}}{\pgfqpoint{0.963359in}{2.262368in}}{\pgfqpoint{0.969183in}{2.268192in}}%
\pgfpathcurveto{\pgfqpoint{0.975007in}{2.274016in}}{\pgfqpoint{0.978280in}{2.281916in}}{\pgfqpoint{0.978280in}{2.290152in}}%
\pgfpathcurveto{\pgfqpoint{0.978280in}{2.298389in}}{\pgfqpoint{0.975007in}{2.306289in}}{\pgfqpoint{0.969183in}{2.312113in}}%
\pgfpathcurveto{\pgfqpoint{0.963359in}{2.317937in}}{\pgfqpoint{0.955459in}{2.321209in}}{\pgfqpoint{0.947223in}{2.321209in}}%
\pgfpathcurveto{\pgfqpoint{0.938987in}{2.321209in}}{\pgfqpoint{0.931087in}{2.317937in}}{\pgfqpoint{0.925263in}{2.312113in}}%
\pgfpathcurveto{\pgfqpoint{0.919439in}{2.306289in}}{\pgfqpoint{0.916167in}{2.298389in}}{\pgfqpoint{0.916167in}{2.290152in}}%
\pgfpathcurveto{\pgfqpoint{0.916167in}{2.281916in}}{\pgfqpoint{0.919439in}{2.274016in}}{\pgfqpoint{0.925263in}{2.268192in}}%
\pgfpathcurveto{\pgfqpoint{0.931087in}{2.262368in}}{\pgfqpoint{0.938987in}{2.259096in}}{\pgfqpoint{0.947223in}{2.259096in}}%
\pgfpathclose%
\pgfusepath{stroke,fill}%
\end{pgfscope}%
\begin{pgfscope}%
\pgfpathrectangle{\pgfqpoint{0.100000in}{0.212622in}}{\pgfqpoint{3.696000in}{3.696000in}}%
\pgfusepath{clip}%
\pgfsetbuttcap%
\pgfsetroundjoin%
\definecolor{currentfill}{rgb}{0.121569,0.466667,0.705882}%
\pgfsetfillcolor{currentfill}%
\pgfsetfillopacity{0.861155}%
\pgfsetlinewidth{1.003750pt}%
\definecolor{currentstroke}{rgb}{0.121569,0.466667,0.705882}%
\pgfsetstrokecolor{currentstroke}%
\pgfsetstrokeopacity{0.861155}%
\pgfsetdash{}{0pt}%
\pgfpathmoveto{\pgfqpoint{0.827246in}{2.390171in}}%
\pgfpathcurveto{\pgfqpoint{0.835482in}{2.390171in}}{\pgfqpoint{0.843382in}{2.393443in}}{\pgfqpoint{0.849206in}{2.399267in}}%
\pgfpathcurveto{\pgfqpoint{0.855030in}{2.405091in}}{\pgfqpoint{0.858302in}{2.412991in}}{\pgfqpoint{0.858302in}{2.421228in}}%
\pgfpathcurveto{\pgfqpoint{0.858302in}{2.429464in}}{\pgfqpoint{0.855030in}{2.437364in}}{\pgfqpoint{0.849206in}{2.443188in}}%
\pgfpathcurveto{\pgfqpoint{0.843382in}{2.449012in}}{\pgfqpoint{0.835482in}{2.452284in}}{\pgfqpoint{0.827246in}{2.452284in}}%
\pgfpathcurveto{\pgfqpoint{0.819009in}{2.452284in}}{\pgfqpoint{0.811109in}{2.449012in}}{\pgfqpoint{0.805285in}{2.443188in}}%
\pgfpathcurveto{\pgfqpoint{0.799461in}{2.437364in}}{\pgfqpoint{0.796189in}{2.429464in}}{\pgfqpoint{0.796189in}{2.421228in}}%
\pgfpathcurveto{\pgfqpoint{0.796189in}{2.412991in}}{\pgfqpoint{0.799461in}{2.405091in}}{\pgfqpoint{0.805285in}{2.399267in}}%
\pgfpathcurveto{\pgfqpoint{0.811109in}{2.393443in}}{\pgfqpoint{0.819009in}{2.390171in}}{\pgfqpoint{0.827246in}{2.390171in}}%
\pgfpathclose%
\pgfusepath{stroke,fill}%
\end{pgfscope}%
\begin{pgfscope}%
\pgfpathrectangle{\pgfqpoint{0.100000in}{0.212622in}}{\pgfqpoint{3.696000in}{3.696000in}}%
\pgfusepath{clip}%
\pgfsetbuttcap%
\pgfsetroundjoin%
\definecolor{currentfill}{rgb}{0.121569,0.466667,0.705882}%
\pgfsetfillcolor{currentfill}%
\pgfsetfillopacity{0.861769}%
\pgfsetlinewidth{1.003750pt}%
\definecolor{currentstroke}{rgb}{0.121569,0.466667,0.705882}%
\pgfsetstrokecolor{currentstroke}%
\pgfsetstrokeopacity{0.861769}%
\pgfsetdash{}{0pt}%
\pgfpathmoveto{\pgfqpoint{0.963305in}{2.250237in}}%
\pgfpathcurveto{\pgfqpoint{0.971541in}{2.250237in}}{\pgfqpoint{0.979441in}{2.253510in}}{\pgfqpoint{0.985265in}{2.259334in}}%
\pgfpathcurveto{\pgfqpoint{0.991089in}{2.265157in}}{\pgfqpoint{0.994362in}{2.273058in}}{\pgfqpoint{0.994362in}{2.281294in}}%
\pgfpathcurveto{\pgfqpoint{0.994362in}{2.289530in}}{\pgfqpoint{0.991089in}{2.297430in}}{\pgfqpoint{0.985265in}{2.303254in}}%
\pgfpathcurveto{\pgfqpoint{0.979441in}{2.309078in}}{\pgfqpoint{0.971541in}{2.312350in}}{\pgfqpoint{0.963305in}{2.312350in}}%
\pgfpathcurveto{\pgfqpoint{0.955069in}{2.312350in}}{\pgfqpoint{0.947169in}{2.309078in}}{\pgfqpoint{0.941345in}{2.303254in}}%
\pgfpathcurveto{\pgfqpoint{0.935521in}{2.297430in}}{\pgfqpoint{0.932249in}{2.289530in}}{\pgfqpoint{0.932249in}{2.281294in}}%
\pgfpathcurveto{\pgfqpoint{0.932249in}{2.273058in}}{\pgfqpoint{0.935521in}{2.265157in}}{\pgfqpoint{0.941345in}{2.259334in}}%
\pgfpathcurveto{\pgfqpoint{0.947169in}{2.253510in}}{\pgfqpoint{0.955069in}{2.250237in}}{\pgfqpoint{0.963305in}{2.250237in}}%
\pgfpathclose%
\pgfusepath{stroke,fill}%
\end{pgfscope}%
\begin{pgfscope}%
\pgfpathrectangle{\pgfqpoint{0.100000in}{0.212622in}}{\pgfqpoint{3.696000in}{3.696000in}}%
\pgfusepath{clip}%
\pgfsetbuttcap%
\pgfsetroundjoin%
\definecolor{currentfill}{rgb}{0.121569,0.466667,0.705882}%
\pgfsetfillcolor{currentfill}%
\pgfsetfillopacity{0.862193}%
\pgfsetlinewidth{1.003750pt}%
\definecolor{currentstroke}{rgb}{0.121569,0.466667,0.705882}%
\pgfsetstrokecolor{currentstroke}%
\pgfsetstrokeopacity{0.862193}%
\pgfsetdash{}{0pt}%
\pgfpathmoveto{\pgfqpoint{2.339485in}{1.783396in}}%
\pgfpathcurveto{\pgfqpoint{2.347721in}{1.783396in}}{\pgfqpoint{2.355621in}{1.786668in}}{\pgfqpoint{2.361445in}{1.792492in}}%
\pgfpathcurveto{\pgfqpoint{2.367269in}{1.798316in}}{\pgfqpoint{2.370542in}{1.806216in}}{\pgfqpoint{2.370542in}{1.814452in}}%
\pgfpathcurveto{\pgfqpoint{2.370542in}{1.822689in}}{\pgfqpoint{2.367269in}{1.830589in}}{\pgfqpoint{2.361445in}{1.836413in}}%
\pgfpathcurveto{\pgfqpoint{2.355621in}{1.842237in}}{\pgfqpoint{2.347721in}{1.845509in}}{\pgfqpoint{2.339485in}{1.845509in}}%
\pgfpathcurveto{\pgfqpoint{2.331249in}{1.845509in}}{\pgfqpoint{2.323349in}{1.842237in}}{\pgfqpoint{2.317525in}{1.836413in}}%
\pgfpathcurveto{\pgfqpoint{2.311701in}{1.830589in}}{\pgfqpoint{2.308429in}{1.822689in}}{\pgfqpoint{2.308429in}{1.814452in}}%
\pgfpathcurveto{\pgfqpoint{2.308429in}{1.806216in}}{\pgfqpoint{2.311701in}{1.798316in}}{\pgfqpoint{2.317525in}{1.792492in}}%
\pgfpathcurveto{\pgfqpoint{2.323349in}{1.786668in}}{\pgfqpoint{2.331249in}{1.783396in}}{\pgfqpoint{2.339485in}{1.783396in}}%
\pgfpathclose%
\pgfusepath{stroke,fill}%
\end{pgfscope}%
\begin{pgfscope}%
\pgfpathrectangle{\pgfqpoint{0.100000in}{0.212622in}}{\pgfqpoint{3.696000in}{3.696000in}}%
\pgfusepath{clip}%
\pgfsetbuttcap%
\pgfsetroundjoin%
\definecolor{currentfill}{rgb}{0.121569,0.466667,0.705882}%
\pgfsetfillcolor{currentfill}%
\pgfsetfillopacity{0.865417}%
\pgfsetlinewidth{1.003750pt}%
\definecolor{currentstroke}{rgb}{0.121569,0.466667,0.705882}%
\pgfsetstrokecolor{currentstroke}%
\pgfsetstrokeopacity{0.865417}%
\pgfsetdash{}{0pt}%
\pgfpathmoveto{\pgfqpoint{0.992770in}{2.239785in}}%
\pgfpathcurveto{\pgfqpoint{1.001006in}{2.239785in}}{\pgfqpoint{1.008906in}{2.243058in}}{\pgfqpoint{1.014730in}{2.248881in}}%
\pgfpathcurveto{\pgfqpoint{1.020554in}{2.254705in}}{\pgfqpoint{1.023826in}{2.262605in}}{\pgfqpoint{1.023826in}{2.270842in}}%
\pgfpathcurveto{\pgfqpoint{1.023826in}{2.279078in}}{\pgfqpoint{1.020554in}{2.286978in}}{\pgfqpoint{1.014730in}{2.292802in}}%
\pgfpathcurveto{\pgfqpoint{1.008906in}{2.298626in}}{\pgfqpoint{1.001006in}{2.301898in}}{\pgfqpoint{0.992770in}{2.301898in}}%
\pgfpathcurveto{\pgfqpoint{0.984533in}{2.301898in}}{\pgfqpoint{0.976633in}{2.298626in}}{\pgfqpoint{0.970809in}{2.292802in}}%
\pgfpathcurveto{\pgfqpoint{0.964986in}{2.286978in}}{\pgfqpoint{0.961713in}{2.279078in}}{\pgfqpoint{0.961713in}{2.270842in}}%
\pgfpathcurveto{\pgfqpoint{0.961713in}{2.262605in}}{\pgfqpoint{0.964986in}{2.254705in}}{\pgfqpoint{0.970809in}{2.248881in}}%
\pgfpathcurveto{\pgfqpoint{0.976633in}{2.243058in}}{\pgfqpoint{0.984533in}{2.239785in}}{\pgfqpoint{0.992770in}{2.239785in}}%
\pgfpathclose%
\pgfusepath{stroke,fill}%
\end{pgfscope}%
\begin{pgfscope}%
\pgfpathrectangle{\pgfqpoint{0.100000in}{0.212622in}}{\pgfqpoint{3.696000in}{3.696000in}}%
\pgfusepath{clip}%
\pgfsetbuttcap%
\pgfsetroundjoin%
\definecolor{currentfill}{rgb}{0.121569,0.466667,0.705882}%
\pgfsetfillcolor{currentfill}%
\pgfsetfillopacity{0.866095}%
\pgfsetlinewidth{1.003750pt}%
\definecolor{currentstroke}{rgb}{0.121569,0.466667,0.705882}%
\pgfsetstrokecolor{currentstroke}%
\pgfsetstrokeopacity{0.866095}%
\pgfsetdash{}{0pt}%
\pgfpathmoveto{\pgfqpoint{0.858699in}{2.382619in}}%
\pgfpathcurveto{\pgfqpoint{0.866935in}{2.382619in}}{\pgfqpoint{0.874835in}{2.385891in}}{\pgfqpoint{0.880659in}{2.391715in}}%
\pgfpathcurveto{\pgfqpoint{0.886483in}{2.397539in}}{\pgfqpoint{0.889756in}{2.405439in}}{\pgfqpoint{0.889756in}{2.413675in}}%
\pgfpathcurveto{\pgfqpoint{0.889756in}{2.421912in}}{\pgfqpoint{0.886483in}{2.429812in}}{\pgfqpoint{0.880659in}{2.435636in}}%
\pgfpathcurveto{\pgfqpoint{0.874835in}{2.441460in}}{\pgfqpoint{0.866935in}{2.444732in}}{\pgfqpoint{0.858699in}{2.444732in}}%
\pgfpathcurveto{\pgfqpoint{0.850463in}{2.444732in}}{\pgfqpoint{0.842563in}{2.441460in}}{\pgfqpoint{0.836739in}{2.435636in}}%
\pgfpathcurveto{\pgfqpoint{0.830915in}{2.429812in}}{\pgfqpoint{0.827643in}{2.421912in}}{\pgfqpoint{0.827643in}{2.413675in}}%
\pgfpathcurveto{\pgfqpoint{0.827643in}{2.405439in}}{\pgfqpoint{0.830915in}{2.397539in}}{\pgfqpoint{0.836739in}{2.391715in}}%
\pgfpathcurveto{\pgfqpoint{0.842563in}{2.385891in}}{\pgfqpoint{0.850463in}{2.382619in}}{\pgfqpoint{0.858699in}{2.382619in}}%
\pgfpathclose%
\pgfusepath{stroke,fill}%
\end{pgfscope}%
\begin{pgfscope}%
\pgfpathrectangle{\pgfqpoint{0.100000in}{0.212622in}}{\pgfqpoint{3.696000in}{3.696000in}}%
\pgfusepath{clip}%
\pgfsetbuttcap%
\pgfsetroundjoin%
\definecolor{currentfill}{rgb}{0.121569,0.466667,0.705882}%
\pgfsetfillcolor{currentfill}%
\pgfsetfillopacity{0.868331}%
\pgfsetlinewidth{1.003750pt}%
\definecolor{currentstroke}{rgb}{0.121569,0.466667,0.705882}%
\pgfsetstrokecolor{currentstroke}%
\pgfsetstrokeopacity{0.868331}%
\pgfsetdash{}{0pt}%
\pgfpathmoveto{\pgfqpoint{1.020162in}{2.232729in}}%
\pgfpathcurveto{\pgfqpoint{1.028399in}{2.232729in}}{\pgfqpoint{1.036299in}{2.236001in}}{\pgfqpoint{1.042123in}{2.241825in}}%
\pgfpathcurveto{\pgfqpoint{1.047947in}{2.247649in}}{\pgfqpoint{1.051219in}{2.255549in}}{\pgfqpoint{1.051219in}{2.263785in}}%
\pgfpathcurveto{\pgfqpoint{1.051219in}{2.272021in}}{\pgfqpoint{1.047947in}{2.279921in}}{\pgfqpoint{1.042123in}{2.285745in}}%
\pgfpathcurveto{\pgfqpoint{1.036299in}{2.291569in}}{\pgfqpoint{1.028399in}{2.294842in}}{\pgfqpoint{1.020162in}{2.294842in}}%
\pgfpathcurveto{\pgfqpoint{1.011926in}{2.294842in}}{\pgfqpoint{1.004026in}{2.291569in}}{\pgfqpoint{0.998202in}{2.285745in}}%
\pgfpathcurveto{\pgfqpoint{0.992378in}{2.279921in}}{\pgfqpoint{0.989106in}{2.272021in}}{\pgfqpoint{0.989106in}{2.263785in}}%
\pgfpathcurveto{\pgfqpoint{0.989106in}{2.255549in}}{\pgfqpoint{0.992378in}{2.247649in}}{\pgfqpoint{0.998202in}{2.241825in}}%
\pgfpathcurveto{\pgfqpoint{1.004026in}{2.236001in}}{\pgfqpoint{1.011926in}{2.232729in}}{\pgfqpoint{1.020162in}{2.232729in}}%
\pgfpathclose%
\pgfusepath{stroke,fill}%
\end{pgfscope}%
\begin{pgfscope}%
\pgfpathrectangle{\pgfqpoint{0.100000in}{0.212622in}}{\pgfqpoint{3.696000in}{3.696000in}}%
\pgfusepath{clip}%
\pgfsetbuttcap%
\pgfsetroundjoin%
\definecolor{currentfill}{rgb}{0.121569,0.466667,0.705882}%
\pgfsetfillcolor{currentfill}%
\pgfsetfillopacity{0.869007}%
\pgfsetlinewidth{1.003750pt}%
\definecolor{currentstroke}{rgb}{0.121569,0.466667,0.705882}%
\pgfsetstrokecolor{currentstroke}%
\pgfsetstrokeopacity{0.869007}%
\pgfsetdash{}{0pt}%
\pgfpathmoveto{\pgfqpoint{0.876862in}{2.383055in}}%
\pgfpathcurveto{\pgfqpoint{0.885098in}{2.383055in}}{\pgfqpoint{0.892998in}{2.386327in}}{\pgfqpoint{0.898822in}{2.392151in}}%
\pgfpathcurveto{\pgfqpoint{0.904646in}{2.397975in}}{\pgfqpoint{0.907919in}{2.405875in}}{\pgfqpoint{0.907919in}{2.414111in}}%
\pgfpathcurveto{\pgfqpoint{0.907919in}{2.422348in}}{\pgfqpoint{0.904646in}{2.430248in}}{\pgfqpoint{0.898822in}{2.436072in}}%
\pgfpathcurveto{\pgfqpoint{0.892998in}{2.441896in}}{\pgfqpoint{0.885098in}{2.445168in}}{\pgfqpoint{0.876862in}{2.445168in}}%
\pgfpathcurveto{\pgfqpoint{0.868626in}{2.445168in}}{\pgfqpoint{0.860726in}{2.441896in}}{\pgfqpoint{0.854902in}{2.436072in}}%
\pgfpathcurveto{\pgfqpoint{0.849078in}{2.430248in}}{\pgfqpoint{0.845806in}{2.422348in}}{\pgfqpoint{0.845806in}{2.414111in}}%
\pgfpathcurveto{\pgfqpoint{0.845806in}{2.405875in}}{\pgfqpoint{0.849078in}{2.397975in}}{\pgfqpoint{0.854902in}{2.392151in}}%
\pgfpathcurveto{\pgfqpoint{0.860726in}{2.386327in}}{\pgfqpoint{0.868626in}{2.383055in}}{\pgfqpoint{0.876862in}{2.383055in}}%
\pgfpathclose%
\pgfusepath{stroke,fill}%
\end{pgfscope}%
\begin{pgfscope}%
\pgfpathrectangle{\pgfqpoint{0.100000in}{0.212622in}}{\pgfqpoint{3.696000in}{3.696000in}}%
\pgfusepath{clip}%
\pgfsetbuttcap%
\pgfsetroundjoin%
\definecolor{currentfill}{rgb}{0.121569,0.466667,0.705882}%
\pgfsetfillcolor{currentfill}%
\pgfsetfillopacity{0.870459}%
\pgfsetlinewidth{1.003750pt}%
\definecolor{currentstroke}{rgb}{0.121569,0.466667,0.705882}%
\pgfsetstrokecolor{currentstroke}%
\pgfsetstrokeopacity{0.870459}%
\pgfsetdash{}{0pt}%
\pgfpathmoveto{\pgfqpoint{1.041903in}{2.221264in}}%
\pgfpathcurveto{\pgfqpoint{1.050139in}{2.221264in}}{\pgfqpoint{1.058039in}{2.224536in}}{\pgfqpoint{1.063863in}{2.230360in}}%
\pgfpathcurveto{\pgfqpoint{1.069687in}{2.236184in}}{\pgfqpoint{1.072959in}{2.244084in}}{\pgfqpoint{1.072959in}{2.252320in}}%
\pgfpathcurveto{\pgfqpoint{1.072959in}{2.260556in}}{\pgfqpoint{1.069687in}{2.268456in}}{\pgfqpoint{1.063863in}{2.274280in}}%
\pgfpathcurveto{\pgfqpoint{1.058039in}{2.280104in}}{\pgfqpoint{1.050139in}{2.283377in}}{\pgfqpoint{1.041903in}{2.283377in}}%
\pgfpathcurveto{\pgfqpoint{1.033667in}{2.283377in}}{\pgfqpoint{1.025767in}{2.280104in}}{\pgfqpoint{1.019943in}{2.274280in}}%
\pgfpathcurveto{\pgfqpoint{1.014119in}{2.268456in}}{\pgfqpoint{1.010846in}{2.260556in}}{\pgfqpoint{1.010846in}{2.252320in}}%
\pgfpathcurveto{\pgfqpoint{1.010846in}{2.244084in}}{\pgfqpoint{1.014119in}{2.236184in}}{\pgfqpoint{1.019943in}{2.230360in}}%
\pgfpathcurveto{\pgfqpoint{1.025767in}{2.224536in}}{\pgfqpoint{1.033667in}{2.221264in}}{\pgfqpoint{1.041903in}{2.221264in}}%
\pgfpathclose%
\pgfusepath{stroke,fill}%
\end{pgfscope}%
\begin{pgfscope}%
\pgfpathrectangle{\pgfqpoint{0.100000in}{0.212622in}}{\pgfqpoint{3.696000in}{3.696000in}}%
\pgfusepath{clip}%
\pgfsetbuttcap%
\pgfsetroundjoin%
\definecolor{currentfill}{rgb}{0.121569,0.466667,0.705882}%
\pgfsetfillcolor{currentfill}%
\pgfsetfillopacity{0.872031}%
\pgfsetlinewidth{1.003750pt}%
\definecolor{currentstroke}{rgb}{0.121569,0.466667,0.705882}%
\pgfsetstrokecolor{currentstroke}%
\pgfsetstrokeopacity{0.872031}%
\pgfsetdash{}{0pt}%
\pgfpathmoveto{\pgfqpoint{1.054884in}{2.218099in}}%
\pgfpathcurveto{\pgfqpoint{1.063120in}{2.218099in}}{\pgfqpoint{1.071020in}{2.221371in}}{\pgfqpoint{1.076844in}{2.227195in}}%
\pgfpathcurveto{\pgfqpoint{1.082668in}{2.233019in}}{\pgfqpoint{1.085940in}{2.240919in}}{\pgfqpoint{1.085940in}{2.249155in}}%
\pgfpathcurveto{\pgfqpoint{1.085940in}{2.257392in}}{\pgfqpoint{1.082668in}{2.265292in}}{\pgfqpoint{1.076844in}{2.271116in}}%
\pgfpathcurveto{\pgfqpoint{1.071020in}{2.276940in}}{\pgfqpoint{1.063120in}{2.280212in}}{\pgfqpoint{1.054884in}{2.280212in}}%
\pgfpathcurveto{\pgfqpoint{1.046647in}{2.280212in}}{\pgfqpoint{1.038747in}{2.276940in}}{\pgfqpoint{1.032923in}{2.271116in}}%
\pgfpathcurveto{\pgfqpoint{1.027099in}{2.265292in}}{\pgfqpoint{1.023827in}{2.257392in}}{\pgfqpoint{1.023827in}{2.249155in}}%
\pgfpathcurveto{\pgfqpoint{1.023827in}{2.240919in}}{\pgfqpoint{1.027099in}{2.233019in}}{\pgfqpoint{1.032923in}{2.227195in}}%
\pgfpathcurveto{\pgfqpoint{1.038747in}{2.221371in}}{\pgfqpoint{1.046647in}{2.218099in}}{\pgfqpoint{1.054884in}{2.218099in}}%
\pgfpathclose%
\pgfusepath{stroke,fill}%
\end{pgfscope}%
\begin{pgfscope}%
\pgfpathrectangle{\pgfqpoint{0.100000in}{0.212622in}}{\pgfqpoint{3.696000in}{3.696000in}}%
\pgfusepath{clip}%
\pgfsetbuttcap%
\pgfsetroundjoin%
\definecolor{currentfill}{rgb}{0.121569,0.466667,0.705882}%
\pgfsetfillcolor{currentfill}%
\pgfsetfillopacity{0.872269}%
\pgfsetlinewidth{1.003750pt}%
\definecolor{currentstroke}{rgb}{0.121569,0.466667,0.705882}%
\pgfsetstrokecolor{currentstroke}%
\pgfsetstrokeopacity{0.872269}%
\pgfsetdash{}{0pt}%
\pgfpathmoveto{\pgfqpoint{0.897412in}{2.377702in}}%
\pgfpathcurveto{\pgfqpoint{0.905649in}{2.377702in}}{\pgfqpoint{0.913549in}{2.380974in}}{\pgfqpoint{0.919373in}{2.386798in}}%
\pgfpathcurveto{\pgfqpoint{0.925197in}{2.392622in}}{\pgfqpoint{0.928469in}{2.400522in}}{\pgfqpoint{0.928469in}{2.408759in}}%
\pgfpathcurveto{\pgfqpoint{0.928469in}{2.416995in}}{\pgfqpoint{0.925197in}{2.424895in}}{\pgfqpoint{0.919373in}{2.430719in}}%
\pgfpathcurveto{\pgfqpoint{0.913549in}{2.436543in}}{\pgfqpoint{0.905649in}{2.439815in}}{\pgfqpoint{0.897412in}{2.439815in}}%
\pgfpathcurveto{\pgfqpoint{0.889176in}{2.439815in}}{\pgfqpoint{0.881276in}{2.436543in}}{\pgfqpoint{0.875452in}{2.430719in}}%
\pgfpathcurveto{\pgfqpoint{0.869628in}{2.424895in}}{\pgfqpoint{0.866356in}{2.416995in}}{\pgfqpoint{0.866356in}{2.408759in}}%
\pgfpathcurveto{\pgfqpoint{0.866356in}{2.400522in}}{\pgfqpoint{0.869628in}{2.392622in}}{\pgfqpoint{0.875452in}{2.386798in}}%
\pgfpathcurveto{\pgfqpoint{0.881276in}{2.380974in}}{\pgfqpoint{0.889176in}{2.377702in}}{\pgfqpoint{0.897412in}{2.377702in}}%
\pgfpathclose%
\pgfusepath{stroke,fill}%
\end{pgfscope}%
\begin{pgfscope}%
\pgfpathrectangle{\pgfqpoint{0.100000in}{0.212622in}}{\pgfqpoint{3.696000in}{3.696000in}}%
\pgfusepath{clip}%
\pgfsetbuttcap%
\pgfsetroundjoin%
\definecolor{currentfill}{rgb}{0.121569,0.466667,0.705882}%
\pgfsetfillcolor{currentfill}%
\pgfsetfillopacity{0.872873}%
\pgfsetlinewidth{1.003750pt}%
\definecolor{currentstroke}{rgb}{0.121569,0.466667,0.705882}%
\pgfsetstrokecolor{currentstroke}%
\pgfsetstrokeopacity{0.872873}%
\pgfsetdash{}{0pt}%
\pgfpathmoveto{\pgfqpoint{2.346047in}{1.771211in}}%
\pgfpathcurveto{\pgfqpoint{2.354283in}{1.771211in}}{\pgfqpoint{2.362183in}{1.774483in}}{\pgfqpoint{2.368007in}{1.780307in}}%
\pgfpathcurveto{\pgfqpoint{2.373831in}{1.786131in}}{\pgfqpoint{2.377103in}{1.794031in}}{\pgfqpoint{2.377103in}{1.802267in}}%
\pgfpathcurveto{\pgfqpoint{2.377103in}{1.810503in}}{\pgfqpoint{2.373831in}{1.818403in}}{\pgfqpoint{2.368007in}{1.824227in}}%
\pgfpathcurveto{\pgfqpoint{2.362183in}{1.830051in}}{\pgfqpoint{2.354283in}{1.833324in}}{\pgfqpoint{2.346047in}{1.833324in}}%
\pgfpathcurveto{\pgfqpoint{2.337810in}{1.833324in}}{\pgfqpoint{2.329910in}{1.830051in}}{\pgfqpoint{2.324086in}{1.824227in}}%
\pgfpathcurveto{\pgfqpoint{2.318262in}{1.818403in}}{\pgfqpoint{2.314990in}{1.810503in}}{\pgfqpoint{2.314990in}{1.802267in}}%
\pgfpathcurveto{\pgfqpoint{2.314990in}{1.794031in}}{\pgfqpoint{2.318262in}{1.786131in}}{\pgfqpoint{2.324086in}{1.780307in}}%
\pgfpathcurveto{\pgfqpoint{2.329910in}{1.774483in}}{\pgfqpoint{2.337810in}{1.771211in}}{\pgfqpoint{2.346047in}{1.771211in}}%
\pgfpathclose%
\pgfusepath{stroke,fill}%
\end{pgfscope}%
\begin{pgfscope}%
\pgfpathrectangle{\pgfqpoint{0.100000in}{0.212622in}}{\pgfqpoint{3.696000in}{3.696000in}}%
\pgfusepath{clip}%
\pgfsetbuttcap%
\pgfsetroundjoin%
\definecolor{currentfill}{rgb}{0.121569,0.466667,0.705882}%
\pgfsetfillcolor{currentfill}%
\pgfsetfillopacity{0.872960}%
\pgfsetlinewidth{1.003750pt}%
\definecolor{currentstroke}{rgb}{0.121569,0.466667,0.705882}%
\pgfsetstrokecolor{currentstroke}%
\pgfsetstrokeopacity{0.872960}%
\pgfsetdash{}{0pt}%
\pgfpathmoveto{\pgfqpoint{1.065019in}{2.212766in}}%
\pgfpathcurveto{\pgfqpoint{1.073255in}{2.212766in}}{\pgfqpoint{1.081155in}{2.216038in}}{\pgfqpoint{1.086979in}{2.221862in}}%
\pgfpathcurveto{\pgfqpoint{1.092803in}{2.227686in}}{\pgfqpoint{1.096075in}{2.235586in}}{\pgfqpoint{1.096075in}{2.243822in}}%
\pgfpathcurveto{\pgfqpoint{1.096075in}{2.252058in}}{\pgfqpoint{1.092803in}{2.259958in}}{\pgfqpoint{1.086979in}{2.265782in}}%
\pgfpathcurveto{\pgfqpoint{1.081155in}{2.271606in}}{\pgfqpoint{1.073255in}{2.274879in}}{\pgfqpoint{1.065019in}{2.274879in}}%
\pgfpathcurveto{\pgfqpoint{1.056782in}{2.274879in}}{\pgfqpoint{1.048882in}{2.271606in}}{\pgfqpoint{1.043058in}{2.265782in}}%
\pgfpathcurveto{\pgfqpoint{1.037234in}{2.259958in}}{\pgfqpoint{1.033962in}{2.252058in}}{\pgfqpoint{1.033962in}{2.243822in}}%
\pgfpathcurveto{\pgfqpoint{1.033962in}{2.235586in}}{\pgfqpoint{1.037234in}{2.227686in}}{\pgfqpoint{1.043058in}{2.221862in}}%
\pgfpathcurveto{\pgfqpoint{1.048882in}{2.216038in}}{\pgfqpoint{1.056782in}{2.212766in}}{\pgfqpoint{1.065019in}{2.212766in}}%
\pgfpathclose%
\pgfusepath{stroke,fill}%
\end{pgfscope}%
\begin{pgfscope}%
\pgfpathrectangle{\pgfqpoint{0.100000in}{0.212622in}}{\pgfqpoint{3.696000in}{3.696000in}}%
\pgfusepath{clip}%
\pgfsetbuttcap%
\pgfsetroundjoin%
\definecolor{currentfill}{rgb}{0.121569,0.466667,0.705882}%
\pgfsetfillcolor{currentfill}%
\pgfsetfillopacity{0.875053}%
\pgfsetlinewidth{1.003750pt}%
\definecolor{currentstroke}{rgb}{0.121569,0.466667,0.705882}%
\pgfsetstrokecolor{currentstroke}%
\pgfsetstrokeopacity{0.875053}%
\pgfsetdash{}{0pt}%
\pgfpathmoveto{\pgfqpoint{1.083587in}{2.206504in}}%
\pgfpathcurveto{\pgfqpoint{1.091823in}{2.206504in}}{\pgfqpoint{1.099723in}{2.209777in}}{\pgfqpoint{1.105547in}{2.215600in}}%
\pgfpathcurveto{\pgfqpoint{1.111371in}{2.221424in}}{\pgfqpoint{1.114643in}{2.229324in}}{\pgfqpoint{1.114643in}{2.237561in}}%
\pgfpathcurveto{\pgfqpoint{1.114643in}{2.245797in}}{\pgfqpoint{1.111371in}{2.253697in}}{\pgfqpoint{1.105547in}{2.259521in}}%
\pgfpathcurveto{\pgfqpoint{1.099723in}{2.265345in}}{\pgfqpoint{1.091823in}{2.268617in}}{\pgfqpoint{1.083587in}{2.268617in}}%
\pgfpathcurveto{\pgfqpoint{1.075351in}{2.268617in}}{\pgfqpoint{1.067450in}{2.265345in}}{\pgfqpoint{1.061627in}{2.259521in}}%
\pgfpathcurveto{\pgfqpoint{1.055803in}{2.253697in}}{\pgfqpoint{1.052530in}{2.245797in}}{\pgfqpoint{1.052530in}{2.237561in}}%
\pgfpathcurveto{\pgfqpoint{1.052530in}{2.229324in}}{\pgfqpoint{1.055803in}{2.221424in}}{\pgfqpoint{1.061627in}{2.215600in}}%
\pgfpathcurveto{\pgfqpoint{1.067450in}{2.209777in}}{\pgfqpoint{1.075351in}{2.206504in}}{\pgfqpoint{1.083587in}{2.206504in}}%
\pgfpathclose%
\pgfusepath{stroke,fill}%
\end{pgfscope}%
\begin{pgfscope}%
\pgfpathrectangle{\pgfqpoint{0.100000in}{0.212622in}}{\pgfqpoint{3.696000in}{3.696000in}}%
\pgfusepath{clip}%
\pgfsetbuttcap%
\pgfsetroundjoin%
\definecolor{currentfill}{rgb}{0.121569,0.466667,0.705882}%
\pgfsetfillcolor{currentfill}%
\pgfsetfillopacity{0.875801}%
\pgfsetlinewidth{1.003750pt}%
\definecolor{currentstroke}{rgb}{0.121569,0.466667,0.705882}%
\pgfsetstrokecolor{currentstroke}%
\pgfsetstrokeopacity{0.875801}%
\pgfsetdash{}{0pt}%
\pgfpathmoveto{\pgfqpoint{0.925272in}{2.371062in}}%
\pgfpathcurveto{\pgfqpoint{0.933508in}{2.371062in}}{\pgfqpoint{0.941408in}{2.374334in}}{\pgfqpoint{0.947232in}{2.380158in}}%
\pgfpathcurveto{\pgfqpoint{0.953056in}{2.385982in}}{\pgfqpoint{0.956328in}{2.393882in}}{\pgfqpoint{0.956328in}{2.402118in}}%
\pgfpathcurveto{\pgfqpoint{0.956328in}{2.410354in}}{\pgfqpoint{0.953056in}{2.418255in}}{\pgfqpoint{0.947232in}{2.424078in}}%
\pgfpathcurveto{\pgfqpoint{0.941408in}{2.429902in}}{\pgfqpoint{0.933508in}{2.433175in}}{\pgfqpoint{0.925272in}{2.433175in}}%
\pgfpathcurveto{\pgfqpoint{0.917035in}{2.433175in}}{\pgfqpoint{0.909135in}{2.429902in}}{\pgfqpoint{0.903311in}{2.424078in}}%
\pgfpathcurveto{\pgfqpoint{0.897487in}{2.418255in}}{\pgfqpoint{0.894215in}{2.410354in}}{\pgfqpoint{0.894215in}{2.402118in}}%
\pgfpathcurveto{\pgfqpoint{0.894215in}{2.393882in}}{\pgfqpoint{0.897487in}{2.385982in}}{\pgfqpoint{0.903311in}{2.380158in}}%
\pgfpathcurveto{\pgfqpoint{0.909135in}{2.374334in}}{\pgfqpoint{0.917035in}{2.371062in}}{\pgfqpoint{0.925272in}{2.371062in}}%
\pgfpathclose%
\pgfusepath{stroke,fill}%
\end{pgfscope}%
\begin{pgfscope}%
\pgfpathrectangle{\pgfqpoint{0.100000in}{0.212622in}}{\pgfqpoint{3.696000in}{3.696000in}}%
\pgfusepath{clip}%
\pgfsetbuttcap%
\pgfsetroundjoin%
\definecolor{currentfill}{rgb}{0.121569,0.466667,0.705882}%
\pgfsetfillcolor{currentfill}%
\pgfsetfillopacity{0.878603}%
\pgfsetlinewidth{1.003750pt}%
\definecolor{currentstroke}{rgb}{0.121569,0.466667,0.705882}%
\pgfsetstrokecolor{currentstroke}%
\pgfsetstrokeopacity{0.878603}%
\pgfsetdash{}{0pt}%
\pgfpathmoveto{\pgfqpoint{1.117629in}{2.194314in}}%
\pgfpathcurveto{\pgfqpoint{1.125865in}{2.194314in}}{\pgfqpoint{1.133765in}{2.197586in}}{\pgfqpoint{1.139589in}{2.203410in}}%
\pgfpathcurveto{\pgfqpoint{1.145413in}{2.209234in}}{\pgfqpoint{1.148685in}{2.217134in}}{\pgfqpoint{1.148685in}{2.225370in}}%
\pgfpathcurveto{\pgfqpoint{1.148685in}{2.233606in}}{\pgfqpoint{1.145413in}{2.241506in}}{\pgfqpoint{1.139589in}{2.247330in}}%
\pgfpathcurveto{\pgfqpoint{1.133765in}{2.253154in}}{\pgfqpoint{1.125865in}{2.256427in}}{\pgfqpoint{1.117629in}{2.256427in}}%
\pgfpathcurveto{\pgfqpoint{1.109392in}{2.256427in}}{\pgfqpoint{1.101492in}{2.253154in}}{\pgfqpoint{1.095668in}{2.247330in}}%
\pgfpathcurveto{\pgfqpoint{1.089844in}{2.241506in}}{\pgfqpoint{1.086572in}{2.233606in}}{\pgfqpoint{1.086572in}{2.225370in}}%
\pgfpathcurveto{\pgfqpoint{1.086572in}{2.217134in}}{\pgfqpoint{1.089844in}{2.209234in}}{\pgfqpoint{1.095668in}{2.203410in}}%
\pgfpathcurveto{\pgfqpoint{1.101492in}{2.197586in}}{\pgfqpoint{1.109392in}{2.194314in}}{\pgfqpoint{1.117629in}{2.194314in}}%
\pgfpathclose%
\pgfusepath{stroke,fill}%
\end{pgfscope}%
\begin{pgfscope}%
\pgfpathrectangle{\pgfqpoint{0.100000in}{0.212622in}}{\pgfqpoint{3.696000in}{3.696000in}}%
\pgfusepath{clip}%
\pgfsetbuttcap%
\pgfsetroundjoin%
\definecolor{currentfill}{rgb}{0.121569,0.466667,0.705882}%
\pgfsetfillcolor{currentfill}%
\pgfsetfillopacity{0.881187}%
\pgfsetlinewidth{1.003750pt}%
\definecolor{currentstroke}{rgb}{0.121569,0.466667,0.705882}%
\pgfsetstrokecolor{currentstroke}%
\pgfsetstrokeopacity{0.881187}%
\pgfsetdash{}{0pt}%
\pgfpathmoveto{\pgfqpoint{0.957688in}{2.370765in}}%
\pgfpathcurveto{\pgfqpoint{0.965925in}{2.370765in}}{\pgfqpoint{0.973825in}{2.374038in}}{\pgfqpoint{0.979649in}{2.379862in}}%
\pgfpathcurveto{\pgfqpoint{0.985473in}{2.385686in}}{\pgfqpoint{0.988745in}{2.393586in}}{\pgfqpoint{0.988745in}{2.401822in}}%
\pgfpathcurveto{\pgfqpoint{0.988745in}{2.410058in}}{\pgfqpoint{0.985473in}{2.417958in}}{\pgfqpoint{0.979649in}{2.423782in}}%
\pgfpathcurveto{\pgfqpoint{0.973825in}{2.429606in}}{\pgfqpoint{0.965925in}{2.432878in}}{\pgfqpoint{0.957688in}{2.432878in}}%
\pgfpathcurveto{\pgfqpoint{0.949452in}{2.432878in}}{\pgfqpoint{0.941552in}{2.429606in}}{\pgfqpoint{0.935728in}{2.423782in}}%
\pgfpathcurveto{\pgfqpoint{0.929904in}{2.417958in}}{\pgfqpoint{0.926632in}{2.410058in}}{\pgfqpoint{0.926632in}{2.401822in}}%
\pgfpathcurveto{\pgfqpoint{0.926632in}{2.393586in}}{\pgfqpoint{0.929904in}{2.385686in}}{\pgfqpoint{0.935728in}{2.379862in}}%
\pgfpathcurveto{\pgfqpoint{0.941552in}{2.374038in}}{\pgfqpoint{0.949452in}{2.370765in}}{\pgfqpoint{0.957688in}{2.370765in}}%
\pgfpathclose%
\pgfusepath{stroke,fill}%
\end{pgfscope}%
\begin{pgfscope}%
\pgfpathrectangle{\pgfqpoint{0.100000in}{0.212622in}}{\pgfqpoint{3.696000in}{3.696000in}}%
\pgfusepath{clip}%
\pgfsetbuttcap%
\pgfsetroundjoin%
\definecolor{currentfill}{rgb}{0.121569,0.466667,0.705882}%
\pgfsetfillcolor{currentfill}%
\pgfsetfillopacity{0.881189}%
\pgfsetlinewidth{1.003750pt}%
\definecolor{currentstroke}{rgb}{0.121569,0.466667,0.705882}%
\pgfsetstrokecolor{currentstroke}%
\pgfsetstrokeopacity{0.881189}%
\pgfsetdash{}{0pt}%
\pgfpathmoveto{\pgfqpoint{1.144859in}{2.179728in}}%
\pgfpathcurveto{\pgfqpoint{1.153095in}{2.179728in}}{\pgfqpoint{1.160995in}{2.183000in}}{\pgfqpoint{1.166819in}{2.188824in}}%
\pgfpathcurveto{\pgfqpoint{1.172643in}{2.194648in}}{\pgfqpoint{1.175915in}{2.202548in}}{\pgfqpoint{1.175915in}{2.210785in}}%
\pgfpathcurveto{\pgfqpoint{1.175915in}{2.219021in}}{\pgfqpoint{1.172643in}{2.226921in}}{\pgfqpoint{1.166819in}{2.232745in}}%
\pgfpathcurveto{\pgfqpoint{1.160995in}{2.238569in}}{\pgfqpoint{1.153095in}{2.241841in}}{\pgfqpoint{1.144859in}{2.241841in}}%
\pgfpathcurveto{\pgfqpoint{1.136622in}{2.241841in}}{\pgfqpoint{1.128722in}{2.238569in}}{\pgfqpoint{1.122898in}{2.232745in}}%
\pgfpathcurveto{\pgfqpoint{1.117074in}{2.226921in}}{\pgfqpoint{1.113802in}{2.219021in}}{\pgfqpoint{1.113802in}{2.210785in}}%
\pgfpathcurveto{\pgfqpoint{1.113802in}{2.202548in}}{\pgfqpoint{1.117074in}{2.194648in}}{\pgfqpoint{1.122898in}{2.188824in}}%
\pgfpathcurveto{\pgfqpoint{1.128722in}{2.183000in}}{\pgfqpoint{1.136622in}{2.179728in}}{\pgfqpoint{1.144859in}{2.179728in}}%
\pgfpathclose%
\pgfusepath{stroke,fill}%
\end{pgfscope}%
\begin{pgfscope}%
\pgfpathrectangle{\pgfqpoint{0.100000in}{0.212622in}}{\pgfqpoint{3.696000in}{3.696000in}}%
\pgfusepath{clip}%
\pgfsetbuttcap%
\pgfsetroundjoin%
\definecolor{currentfill}{rgb}{0.121569,0.466667,0.705882}%
\pgfsetfillcolor{currentfill}%
\pgfsetfillopacity{0.883343}%
\pgfsetlinewidth{1.003750pt}%
\definecolor{currentstroke}{rgb}{0.121569,0.466667,0.705882}%
\pgfsetstrokecolor{currentstroke}%
\pgfsetstrokeopacity{0.883343}%
\pgfsetdash{}{0pt}%
\pgfpathmoveto{\pgfqpoint{1.163251in}{2.174828in}}%
\pgfpathcurveto{\pgfqpoint{1.171488in}{2.174828in}}{\pgfqpoint{1.179388in}{2.178100in}}{\pgfqpoint{1.185212in}{2.183924in}}%
\pgfpathcurveto{\pgfqpoint{1.191036in}{2.189748in}}{\pgfqpoint{1.194308in}{2.197648in}}{\pgfqpoint{1.194308in}{2.205884in}}%
\pgfpathcurveto{\pgfqpoint{1.194308in}{2.214120in}}{\pgfqpoint{1.191036in}{2.222020in}}{\pgfqpoint{1.185212in}{2.227844in}}%
\pgfpathcurveto{\pgfqpoint{1.179388in}{2.233668in}}{\pgfqpoint{1.171488in}{2.236941in}}{\pgfqpoint{1.163251in}{2.236941in}}%
\pgfpathcurveto{\pgfqpoint{1.155015in}{2.236941in}}{\pgfqpoint{1.147115in}{2.233668in}}{\pgfqpoint{1.141291in}{2.227844in}}%
\pgfpathcurveto{\pgfqpoint{1.135467in}{2.222020in}}{\pgfqpoint{1.132195in}{2.214120in}}{\pgfqpoint{1.132195in}{2.205884in}}%
\pgfpathcurveto{\pgfqpoint{1.132195in}{2.197648in}}{\pgfqpoint{1.135467in}{2.189748in}}{\pgfqpoint{1.141291in}{2.183924in}}%
\pgfpathcurveto{\pgfqpoint{1.147115in}{2.178100in}}{\pgfqpoint{1.155015in}{2.174828in}}{\pgfqpoint{1.163251in}{2.174828in}}%
\pgfpathclose%
\pgfusepath{stroke,fill}%
\end{pgfscope}%
\begin{pgfscope}%
\pgfpathrectangle{\pgfqpoint{0.100000in}{0.212622in}}{\pgfqpoint{3.696000in}{3.696000in}}%
\pgfusepath{clip}%
\pgfsetbuttcap%
\pgfsetroundjoin%
\definecolor{currentfill}{rgb}{0.121569,0.466667,0.705882}%
\pgfsetfillcolor{currentfill}%
\pgfsetfillopacity{0.884302}%
\pgfsetlinewidth{1.003750pt}%
\definecolor{currentstroke}{rgb}{0.121569,0.466667,0.705882}%
\pgfsetstrokecolor{currentstroke}%
\pgfsetstrokeopacity{0.884302}%
\pgfsetdash{}{0pt}%
\pgfpathmoveto{\pgfqpoint{1.173461in}{2.169448in}}%
\pgfpathcurveto{\pgfqpoint{1.181697in}{2.169448in}}{\pgfqpoint{1.189597in}{2.172720in}}{\pgfqpoint{1.195421in}{2.178544in}}%
\pgfpathcurveto{\pgfqpoint{1.201245in}{2.184368in}}{\pgfqpoint{1.204517in}{2.192268in}}{\pgfqpoint{1.204517in}{2.200504in}}%
\pgfpathcurveto{\pgfqpoint{1.204517in}{2.208740in}}{\pgfqpoint{1.201245in}{2.216641in}}{\pgfqpoint{1.195421in}{2.222464in}}%
\pgfpathcurveto{\pgfqpoint{1.189597in}{2.228288in}}{\pgfqpoint{1.181697in}{2.231561in}}{\pgfqpoint{1.173461in}{2.231561in}}%
\pgfpathcurveto{\pgfqpoint{1.165225in}{2.231561in}}{\pgfqpoint{1.157325in}{2.228288in}}{\pgfqpoint{1.151501in}{2.222464in}}%
\pgfpathcurveto{\pgfqpoint{1.145677in}{2.216641in}}{\pgfqpoint{1.142404in}{2.208740in}}{\pgfqpoint{1.142404in}{2.200504in}}%
\pgfpathcurveto{\pgfqpoint{1.142404in}{2.192268in}}{\pgfqpoint{1.145677in}{2.184368in}}{\pgfqpoint{1.151501in}{2.178544in}}%
\pgfpathcurveto{\pgfqpoint{1.157325in}{2.172720in}}{\pgfqpoint{1.165225in}{2.169448in}}{\pgfqpoint{1.173461in}{2.169448in}}%
\pgfpathclose%
\pgfusepath{stroke,fill}%
\end{pgfscope}%
\begin{pgfscope}%
\pgfpathrectangle{\pgfqpoint{0.100000in}{0.212622in}}{\pgfqpoint{3.696000in}{3.696000in}}%
\pgfusepath{clip}%
\pgfsetbuttcap%
\pgfsetroundjoin%
\definecolor{currentfill}{rgb}{0.121569,0.466667,0.705882}%
\pgfsetfillcolor{currentfill}%
\pgfsetfillopacity{0.885165}%
\pgfsetlinewidth{1.003750pt}%
\definecolor{currentstroke}{rgb}{0.121569,0.466667,0.705882}%
\pgfsetstrokecolor{currentstroke}%
\pgfsetstrokeopacity{0.885165}%
\pgfsetdash{}{0pt}%
\pgfpathmoveto{\pgfqpoint{2.370250in}{1.761524in}}%
\pgfpathcurveto{\pgfqpoint{2.378487in}{1.761524in}}{\pgfqpoint{2.386387in}{1.764796in}}{\pgfqpoint{2.392211in}{1.770620in}}%
\pgfpathcurveto{\pgfqpoint{2.398035in}{1.776444in}}{\pgfqpoint{2.401307in}{1.784344in}}{\pgfqpoint{2.401307in}{1.792580in}}%
\pgfpathcurveto{\pgfqpoint{2.401307in}{1.800816in}}{\pgfqpoint{2.398035in}{1.808717in}}{\pgfqpoint{2.392211in}{1.814540in}}%
\pgfpathcurveto{\pgfqpoint{2.386387in}{1.820364in}}{\pgfqpoint{2.378487in}{1.823637in}}{\pgfqpoint{2.370250in}{1.823637in}}%
\pgfpathcurveto{\pgfqpoint{2.362014in}{1.823637in}}{\pgfqpoint{2.354114in}{1.820364in}}{\pgfqpoint{2.348290in}{1.814540in}}%
\pgfpathcurveto{\pgfqpoint{2.342466in}{1.808717in}}{\pgfqpoint{2.339194in}{1.800816in}}{\pgfqpoint{2.339194in}{1.792580in}}%
\pgfpathcurveto{\pgfqpoint{2.339194in}{1.784344in}}{\pgfqpoint{2.342466in}{1.776444in}}{\pgfqpoint{2.348290in}{1.770620in}}%
\pgfpathcurveto{\pgfqpoint{2.354114in}{1.764796in}}{\pgfqpoint{2.362014in}{1.761524in}}{\pgfqpoint{2.370250in}{1.761524in}}%
\pgfpathclose%
\pgfusepath{stroke,fill}%
\end{pgfscope}%
\begin{pgfscope}%
\pgfpathrectangle{\pgfqpoint{0.100000in}{0.212622in}}{\pgfqpoint{3.696000in}{3.696000in}}%
\pgfusepath{clip}%
\pgfsetbuttcap%
\pgfsetroundjoin%
\definecolor{currentfill}{rgb}{0.121569,0.466667,0.705882}%
\pgfsetfillcolor{currentfill}%
\pgfsetfillopacity{0.886150}%
\pgfsetlinewidth{1.003750pt}%
\definecolor{currentstroke}{rgb}{0.121569,0.466667,0.705882}%
\pgfsetstrokecolor{currentstroke}%
\pgfsetstrokeopacity{0.886150}%
\pgfsetdash{}{0pt}%
\pgfpathmoveto{\pgfqpoint{1.192199in}{2.161048in}}%
\pgfpathcurveto{\pgfqpoint{1.200436in}{2.161048in}}{\pgfqpoint{1.208336in}{2.164321in}}{\pgfqpoint{1.214160in}{2.170145in}}%
\pgfpathcurveto{\pgfqpoint{1.219984in}{2.175969in}}{\pgfqpoint{1.223256in}{2.183869in}}{\pgfqpoint{1.223256in}{2.192105in}}%
\pgfpathcurveto{\pgfqpoint{1.223256in}{2.200341in}}{\pgfqpoint{1.219984in}{2.208241in}}{\pgfqpoint{1.214160in}{2.214065in}}%
\pgfpathcurveto{\pgfqpoint{1.208336in}{2.219889in}}{\pgfqpoint{1.200436in}{2.223161in}}{\pgfqpoint{1.192199in}{2.223161in}}%
\pgfpathcurveto{\pgfqpoint{1.183963in}{2.223161in}}{\pgfqpoint{1.176063in}{2.219889in}}{\pgfqpoint{1.170239in}{2.214065in}}%
\pgfpathcurveto{\pgfqpoint{1.164415in}{2.208241in}}{\pgfqpoint{1.161143in}{2.200341in}}{\pgfqpoint{1.161143in}{2.192105in}}%
\pgfpathcurveto{\pgfqpoint{1.161143in}{2.183869in}}{\pgfqpoint{1.164415in}{2.175969in}}{\pgfqpoint{1.170239in}{2.170145in}}%
\pgfpathcurveto{\pgfqpoint{1.176063in}{2.164321in}}{\pgfqpoint{1.183963in}{2.161048in}}{\pgfqpoint{1.192199in}{2.161048in}}%
\pgfpathclose%
\pgfusepath{stroke,fill}%
\end{pgfscope}%
\begin{pgfscope}%
\pgfpathrectangle{\pgfqpoint{0.100000in}{0.212622in}}{\pgfqpoint{3.696000in}{3.696000in}}%
\pgfusepath{clip}%
\pgfsetbuttcap%
\pgfsetroundjoin%
\definecolor{currentfill}{rgb}{0.121569,0.466667,0.705882}%
\pgfsetfillcolor{currentfill}%
\pgfsetfillopacity{0.887794}%
\pgfsetlinewidth{1.003750pt}%
\definecolor{currentstroke}{rgb}{0.121569,0.466667,0.705882}%
\pgfsetstrokecolor{currentstroke}%
\pgfsetstrokeopacity{0.887794}%
\pgfsetdash{}{0pt}%
\pgfpathmoveto{\pgfqpoint{0.996912in}{2.364990in}}%
\pgfpathcurveto{\pgfqpoint{1.005149in}{2.364990in}}{\pgfqpoint{1.013049in}{2.368262in}}{\pgfqpoint{1.018873in}{2.374086in}}%
\pgfpathcurveto{\pgfqpoint{1.024696in}{2.379910in}}{\pgfqpoint{1.027969in}{2.387810in}}{\pgfqpoint{1.027969in}{2.396047in}}%
\pgfpathcurveto{\pgfqpoint{1.027969in}{2.404283in}}{\pgfqpoint{1.024696in}{2.412183in}}{\pgfqpoint{1.018873in}{2.418007in}}%
\pgfpathcurveto{\pgfqpoint{1.013049in}{2.423831in}}{\pgfqpoint{1.005149in}{2.427103in}}{\pgfqpoint{0.996912in}{2.427103in}}%
\pgfpathcurveto{\pgfqpoint{0.988676in}{2.427103in}}{\pgfqpoint{0.980776in}{2.423831in}}{\pgfqpoint{0.974952in}{2.418007in}}%
\pgfpathcurveto{\pgfqpoint{0.969128in}{2.412183in}}{\pgfqpoint{0.965856in}{2.404283in}}{\pgfqpoint{0.965856in}{2.396047in}}%
\pgfpathcurveto{\pgfqpoint{0.965856in}{2.387810in}}{\pgfqpoint{0.969128in}{2.379910in}}{\pgfqpoint{0.974952in}{2.374086in}}%
\pgfpathcurveto{\pgfqpoint{0.980776in}{2.368262in}}{\pgfqpoint{0.988676in}{2.364990in}}{\pgfqpoint{0.996912in}{2.364990in}}%
\pgfpathclose%
\pgfusepath{stroke,fill}%
\end{pgfscope}%
\begin{pgfscope}%
\pgfpathrectangle{\pgfqpoint{0.100000in}{0.212622in}}{\pgfqpoint{3.696000in}{3.696000in}}%
\pgfusepath{clip}%
\pgfsetbuttcap%
\pgfsetroundjoin%
\definecolor{currentfill}{rgb}{0.121569,0.466667,0.705882}%
\pgfsetfillcolor{currentfill}%
\pgfsetfillopacity{0.889759}%
\pgfsetlinewidth{1.003750pt}%
\definecolor{currentstroke}{rgb}{0.121569,0.466667,0.705882}%
\pgfsetstrokecolor{currentstroke}%
\pgfsetstrokeopacity{0.889759}%
\pgfsetdash{}{0pt}%
\pgfpathmoveto{\pgfqpoint{1.227120in}{2.151466in}}%
\pgfpathcurveto{\pgfqpoint{1.235356in}{2.151466in}}{\pgfqpoint{1.243256in}{2.154739in}}{\pgfqpoint{1.249080in}{2.160563in}}%
\pgfpathcurveto{\pgfqpoint{1.254904in}{2.166387in}}{\pgfqpoint{1.258176in}{2.174287in}}{\pgfqpoint{1.258176in}{2.182523in}}%
\pgfpathcurveto{\pgfqpoint{1.258176in}{2.190759in}}{\pgfqpoint{1.254904in}{2.198659in}}{\pgfqpoint{1.249080in}{2.204483in}}%
\pgfpathcurveto{\pgfqpoint{1.243256in}{2.210307in}}{\pgfqpoint{1.235356in}{2.213579in}}{\pgfqpoint{1.227120in}{2.213579in}}%
\pgfpathcurveto{\pgfqpoint{1.218884in}{2.213579in}}{\pgfqpoint{1.210984in}{2.210307in}}{\pgfqpoint{1.205160in}{2.204483in}}%
\pgfpathcurveto{\pgfqpoint{1.199336in}{2.198659in}}{\pgfqpoint{1.196063in}{2.190759in}}{\pgfqpoint{1.196063in}{2.182523in}}%
\pgfpathcurveto{\pgfqpoint{1.196063in}{2.174287in}}{\pgfqpoint{1.199336in}{2.166387in}}{\pgfqpoint{1.205160in}{2.160563in}}%
\pgfpathcurveto{\pgfqpoint{1.210984in}{2.154739in}}{\pgfqpoint{1.218884in}{2.151466in}}{\pgfqpoint{1.227120in}{2.151466in}}%
\pgfpathclose%
\pgfusepath{stroke,fill}%
\end{pgfscope}%
\begin{pgfscope}%
\pgfpathrectangle{\pgfqpoint{0.100000in}{0.212622in}}{\pgfqpoint{3.696000in}{3.696000in}}%
\pgfusepath{clip}%
\pgfsetbuttcap%
\pgfsetroundjoin%
\definecolor{currentfill}{rgb}{0.121569,0.466667,0.705882}%
\pgfsetfillcolor{currentfill}%
\pgfsetfillopacity{0.892545}%
\pgfsetlinewidth{1.003750pt}%
\definecolor{currentstroke}{rgb}{0.121569,0.466667,0.705882}%
\pgfsetstrokecolor{currentstroke}%
\pgfsetstrokeopacity{0.892545}%
\pgfsetdash{}{0pt}%
\pgfpathmoveto{\pgfqpoint{1.255503in}{2.135469in}}%
\pgfpathcurveto{\pgfqpoint{1.263740in}{2.135469in}}{\pgfqpoint{1.271640in}{2.138741in}}{\pgfqpoint{1.277464in}{2.144565in}}%
\pgfpathcurveto{\pgfqpoint{1.283288in}{2.150389in}}{\pgfqpoint{1.286560in}{2.158289in}}{\pgfqpoint{1.286560in}{2.166525in}}%
\pgfpathcurveto{\pgfqpoint{1.286560in}{2.174762in}}{\pgfqpoint{1.283288in}{2.182662in}}{\pgfqpoint{1.277464in}{2.188486in}}%
\pgfpathcurveto{\pgfqpoint{1.271640in}{2.194309in}}{\pgfqpoint{1.263740in}{2.197582in}}{\pgfqpoint{1.255503in}{2.197582in}}%
\pgfpathcurveto{\pgfqpoint{1.247267in}{2.197582in}}{\pgfqpoint{1.239367in}{2.194309in}}{\pgfqpoint{1.233543in}{2.188486in}}%
\pgfpathcurveto{\pgfqpoint{1.227719in}{2.182662in}}{\pgfqpoint{1.224447in}{2.174762in}}{\pgfqpoint{1.224447in}{2.166525in}}%
\pgfpathcurveto{\pgfqpoint{1.224447in}{2.158289in}}{\pgfqpoint{1.227719in}{2.150389in}}{\pgfqpoint{1.233543in}{2.144565in}}%
\pgfpathcurveto{\pgfqpoint{1.239367in}{2.138741in}}{\pgfqpoint{1.247267in}{2.135469in}}{\pgfqpoint{1.255503in}{2.135469in}}%
\pgfpathclose%
\pgfusepath{stroke,fill}%
\end{pgfscope}%
\begin{pgfscope}%
\pgfpathrectangle{\pgfqpoint{0.100000in}{0.212622in}}{\pgfqpoint{3.696000in}{3.696000in}}%
\pgfusepath{clip}%
\pgfsetbuttcap%
\pgfsetroundjoin%
\definecolor{currentfill}{rgb}{0.121569,0.466667,0.705882}%
\pgfsetfillcolor{currentfill}%
\pgfsetfillopacity{0.895367}%
\pgfsetlinewidth{1.003750pt}%
\definecolor{currentstroke}{rgb}{0.121569,0.466667,0.705882}%
\pgfsetstrokecolor{currentstroke}%
\pgfsetstrokeopacity{0.895367}%
\pgfsetdash{}{0pt}%
\pgfpathmoveto{\pgfqpoint{1.281014in}{2.127791in}}%
\pgfpathcurveto{\pgfqpoint{1.289250in}{2.127791in}}{\pgfqpoint{1.297150in}{2.131063in}}{\pgfqpoint{1.302974in}{2.136887in}}%
\pgfpathcurveto{\pgfqpoint{1.308798in}{2.142711in}}{\pgfqpoint{1.312070in}{2.150611in}}{\pgfqpoint{1.312070in}{2.158847in}}%
\pgfpathcurveto{\pgfqpoint{1.312070in}{2.167084in}}{\pgfqpoint{1.308798in}{2.174984in}}{\pgfqpoint{1.302974in}{2.180808in}}%
\pgfpathcurveto{\pgfqpoint{1.297150in}{2.186632in}}{\pgfqpoint{1.289250in}{2.189904in}}{\pgfqpoint{1.281014in}{2.189904in}}%
\pgfpathcurveto{\pgfqpoint{1.272777in}{2.189904in}}{\pgfqpoint{1.264877in}{2.186632in}}{\pgfqpoint{1.259053in}{2.180808in}}%
\pgfpathcurveto{\pgfqpoint{1.253229in}{2.174984in}}{\pgfqpoint{1.249957in}{2.167084in}}{\pgfqpoint{1.249957in}{2.158847in}}%
\pgfpathcurveto{\pgfqpoint{1.249957in}{2.150611in}}{\pgfqpoint{1.253229in}{2.142711in}}{\pgfqpoint{1.259053in}{2.136887in}}%
\pgfpathcurveto{\pgfqpoint{1.264877in}{2.131063in}}{\pgfqpoint{1.272777in}{2.127791in}}{\pgfqpoint{1.281014in}{2.127791in}}%
\pgfpathclose%
\pgfusepath{stroke,fill}%
\end{pgfscope}%
\begin{pgfscope}%
\pgfpathrectangle{\pgfqpoint{0.100000in}{0.212622in}}{\pgfqpoint{3.696000in}{3.696000in}}%
\pgfusepath{clip}%
\pgfsetbuttcap%
\pgfsetroundjoin%
\definecolor{currentfill}{rgb}{0.121569,0.466667,0.705882}%
\pgfsetfillcolor{currentfill}%
\pgfsetfillopacity{0.895413}%
\pgfsetlinewidth{1.003750pt}%
\definecolor{currentstroke}{rgb}{0.121569,0.466667,0.705882}%
\pgfsetstrokecolor{currentstroke}%
\pgfsetstrokeopacity{0.895413}%
\pgfsetdash{}{0pt}%
\pgfpathmoveto{\pgfqpoint{1.040372in}{2.366330in}}%
\pgfpathcurveto{\pgfqpoint{1.048609in}{2.366330in}}{\pgfqpoint{1.056509in}{2.369602in}}{\pgfqpoint{1.062333in}{2.375426in}}%
\pgfpathcurveto{\pgfqpoint{1.068157in}{2.381250in}}{\pgfqpoint{1.071429in}{2.389150in}}{\pgfqpoint{1.071429in}{2.397386in}}%
\pgfpathcurveto{\pgfqpoint{1.071429in}{2.405623in}}{\pgfqpoint{1.068157in}{2.413523in}}{\pgfqpoint{1.062333in}{2.419347in}}%
\pgfpathcurveto{\pgfqpoint{1.056509in}{2.425171in}}{\pgfqpoint{1.048609in}{2.428443in}}{\pgfqpoint{1.040372in}{2.428443in}}%
\pgfpathcurveto{\pgfqpoint{1.032136in}{2.428443in}}{\pgfqpoint{1.024236in}{2.425171in}}{\pgfqpoint{1.018412in}{2.419347in}}%
\pgfpathcurveto{\pgfqpoint{1.012588in}{2.413523in}}{\pgfqpoint{1.009316in}{2.405623in}}{\pgfqpoint{1.009316in}{2.397386in}}%
\pgfpathcurveto{\pgfqpoint{1.009316in}{2.389150in}}{\pgfqpoint{1.012588in}{2.381250in}}{\pgfqpoint{1.018412in}{2.375426in}}%
\pgfpathcurveto{\pgfqpoint{1.024236in}{2.369602in}}{\pgfqpoint{1.032136in}{2.366330in}}{\pgfqpoint{1.040372in}{2.366330in}}%
\pgfpathclose%
\pgfusepath{stroke,fill}%
\end{pgfscope}%
\begin{pgfscope}%
\pgfpathrectangle{\pgfqpoint{0.100000in}{0.212622in}}{\pgfqpoint{3.696000in}{3.696000in}}%
\pgfusepath{clip}%
\pgfsetbuttcap%
\pgfsetroundjoin%
\definecolor{currentfill}{rgb}{0.121569,0.466667,0.705882}%
\pgfsetfillcolor{currentfill}%
\pgfsetfillopacity{0.896961}%
\pgfsetlinewidth{1.003750pt}%
\definecolor{currentstroke}{rgb}{0.121569,0.466667,0.705882}%
\pgfsetstrokecolor{currentstroke}%
\pgfsetstrokeopacity{0.896961}%
\pgfsetdash{}{0pt}%
\pgfpathmoveto{\pgfqpoint{1.297060in}{2.119141in}}%
\pgfpathcurveto{\pgfqpoint{1.305296in}{2.119141in}}{\pgfqpoint{1.313196in}{2.122413in}}{\pgfqpoint{1.319020in}{2.128237in}}%
\pgfpathcurveto{\pgfqpoint{1.324844in}{2.134061in}}{\pgfqpoint{1.328116in}{2.141961in}}{\pgfqpoint{1.328116in}{2.150197in}}%
\pgfpathcurveto{\pgfqpoint{1.328116in}{2.158433in}}{\pgfqpoint{1.324844in}{2.166334in}}{\pgfqpoint{1.319020in}{2.172157in}}%
\pgfpathcurveto{\pgfqpoint{1.313196in}{2.177981in}}{\pgfqpoint{1.305296in}{2.181254in}}{\pgfqpoint{1.297060in}{2.181254in}}%
\pgfpathcurveto{\pgfqpoint{1.288823in}{2.181254in}}{\pgfqpoint{1.280923in}{2.177981in}}{\pgfqpoint{1.275099in}{2.172157in}}%
\pgfpathcurveto{\pgfqpoint{1.269275in}{2.166334in}}{\pgfqpoint{1.266003in}{2.158433in}}{\pgfqpoint{1.266003in}{2.150197in}}%
\pgfpathcurveto{\pgfqpoint{1.266003in}{2.141961in}}{\pgfqpoint{1.269275in}{2.134061in}}{\pgfqpoint{1.275099in}{2.128237in}}%
\pgfpathcurveto{\pgfqpoint{1.280923in}{2.122413in}}{\pgfqpoint{1.288823in}{2.119141in}}{\pgfqpoint{1.297060in}{2.119141in}}%
\pgfpathclose%
\pgfusepath{stroke,fill}%
\end{pgfscope}%
\begin{pgfscope}%
\pgfpathrectangle{\pgfqpoint{0.100000in}{0.212622in}}{\pgfqpoint{3.696000in}{3.696000in}}%
\pgfusepath{clip}%
\pgfsetbuttcap%
\pgfsetroundjoin%
\definecolor{currentfill}{rgb}{0.121569,0.466667,0.705882}%
\pgfsetfillcolor{currentfill}%
\pgfsetfillopacity{0.897712}%
\pgfsetlinewidth{1.003750pt}%
\definecolor{currentstroke}{rgb}{0.121569,0.466667,0.705882}%
\pgfsetstrokecolor{currentstroke}%
\pgfsetstrokeopacity{0.897712}%
\pgfsetdash{}{0pt}%
\pgfpathmoveto{\pgfqpoint{1.303441in}{2.117488in}}%
\pgfpathcurveto{\pgfqpoint{1.311678in}{2.117488in}}{\pgfqpoint{1.319578in}{2.120761in}}{\pgfqpoint{1.325402in}{2.126584in}}%
\pgfpathcurveto{\pgfqpoint{1.331226in}{2.132408in}}{\pgfqpoint{1.334498in}{2.140308in}}{\pgfqpoint{1.334498in}{2.148545in}}%
\pgfpathcurveto{\pgfqpoint{1.334498in}{2.156781in}}{\pgfqpoint{1.331226in}{2.164681in}}{\pgfqpoint{1.325402in}{2.170505in}}%
\pgfpathcurveto{\pgfqpoint{1.319578in}{2.176329in}}{\pgfqpoint{1.311678in}{2.179601in}}{\pgfqpoint{1.303441in}{2.179601in}}%
\pgfpathcurveto{\pgfqpoint{1.295205in}{2.179601in}}{\pgfqpoint{1.287305in}{2.176329in}}{\pgfqpoint{1.281481in}{2.170505in}}%
\pgfpathcurveto{\pgfqpoint{1.275657in}{2.164681in}}{\pgfqpoint{1.272385in}{2.156781in}}{\pgfqpoint{1.272385in}{2.148545in}}%
\pgfpathcurveto{\pgfqpoint{1.272385in}{2.140308in}}{\pgfqpoint{1.275657in}{2.132408in}}{\pgfqpoint{1.281481in}{2.126584in}}%
\pgfpathcurveto{\pgfqpoint{1.287305in}{2.120761in}}{\pgfqpoint{1.295205in}{2.117488in}}{\pgfqpoint{1.303441in}{2.117488in}}%
\pgfpathclose%
\pgfusepath{stroke,fill}%
\end{pgfscope}%
\begin{pgfscope}%
\pgfpathrectangle{\pgfqpoint{0.100000in}{0.212622in}}{\pgfqpoint{3.696000in}{3.696000in}}%
\pgfusepath{clip}%
\pgfsetbuttcap%
\pgfsetroundjoin%
\definecolor{currentfill}{rgb}{0.121569,0.466667,0.705882}%
\pgfsetfillcolor{currentfill}%
\pgfsetfillopacity{0.897857}%
\pgfsetlinewidth{1.003750pt}%
\definecolor{currentstroke}{rgb}{0.121569,0.466667,0.705882}%
\pgfsetstrokecolor{currentstroke}%
\pgfsetstrokeopacity{0.897857}%
\pgfsetdash{}{0pt}%
\pgfpathmoveto{\pgfqpoint{2.384230in}{1.741257in}}%
\pgfpathcurveto{\pgfqpoint{2.392467in}{1.741257in}}{\pgfqpoint{2.400367in}{1.744529in}}{\pgfqpoint{2.406191in}{1.750353in}}%
\pgfpathcurveto{\pgfqpoint{2.412015in}{1.756177in}}{\pgfqpoint{2.415287in}{1.764077in}}{\pgfqpoint{2.415287in}{1.772313in}}%
\pgfpathcurveto{\pgfqpoint{2.415287in}{1.780549in}}{\pgfqpoint{2.412015in}{1.788449in}}{\pgfqpoint{2.406191in}{1.794273in}}%
\pgfpathcurveto{\pgfqpoint{2.400367in}{1.800097in}}{\pgfqpoint{2.392467in}{1.803370in}}{\pgfqpoint{2.384230in}{1.803370in}}%
\pgfpathcurveto{\pgfqpoint{2.375994in}{1.803370in}}{\pgfqpoint{2.368094in}{1.800097in}}{\pgfqpoint{2.362270in}{1.794273in}}%
\pgfpathcurveto{\pgfqpoint{2.356446in}{1.788449in}}{\pgfqpoint{2.353174in}{1.780549in}}{\pgfqpoint{2.353174in}{1.772313in}}%
\pgfpathcurveto{\pgfqpoint{2.353174in}{1.764077in}}{\pgfqpoint{2.356446in}{1.756177in}}{\pgfqpoint{2.362270in}{1.750353in}}%
\pgfpathcurveto{\pgfqpoint{2.368094in}{1.744529in}}{\pgfqpoint{2.375994in}{1.741257in}}{\pgfqpoint{2.384230in}{1.741257in}}%
\pgfpathclose%
\pgfusepath{stroke,fill}%
\end{pgfscope}%
\begin{pgfscope}%
\pgfpathrectangle{\pgfqpoint{0.100000in}{0.212622in}}{\pgfqpoint{3.696000in}{3.696000in}}%
\pgfusepath{clip}%
\pgfsetbuttcap%
\pgfsetroundjoin%
\definecolor{currentfill}{rgb}{0.121569,0.466667,0.705882}%
\pgfsetfillcolor{currentfill}%
\pgfsetfillopacity{0.898889}%
\pgfsetlinewidth{1.003750pt}%
\definecolor{currentstroke}{rgb}{0.121569,0.466667,0.705882}%
\pgfsetstrokecolor{currentstroke}%
\pgfsetstrokeopacity{0.898889}%
\pgfsetdash{}{0pt}%
\pgfpathmoveto{\pgfqpoint{1.314988in}{2.112870in}}%
\pgfpathcurveto{\pgfqpoint{1.323224in}{2.112870in}}{\pgfqpoint{1.331124in}{2.116142in}}{\pgfqpoint{1.336948in}{2.121966in}}%
\pgfpathcurveto{\pgfqpoint{1.342772in}{2.127790in}}{\pgfqpoint{1.346044in}{2.135690in}}{\pgfqpoint{1.346044in}{2.143927in}}%
\pgfpathcurveto{\pgfqpoint{1.346044in}{2.152163in}}{\pgfqpoint{1.342772in}{2.160063in}}{\pgfqpoint{1.336948in}{2.165887in}}%
\pgfpathcurveto{\pgfqpoint{1.331124in}{2.171711in}}{\pgfqpoint{1.323224in}{2.174983in}}{\pgfqpoint{1.314988in}{2.174983in}}%
\pgfpathcurveto{\pgfqpoint{1.306751in}{2.174983in}}{\pgfqpoint{1.298851in}{2.171711in}}{\pgfqpoint{1.293027in}{2.165887in}}%
\pgfpathcurveto{\pgfqpoint{1.287203in}{2.160063in}}{\pgfqpoint{1.283931in}{2.152163in}}{\pgfqpoint{1.283931in}{2.143927in}}%
\pgfpathcurveto{\pgfqpoint{1.283931in}{2.135690in}}{\pgfqpoint{1.287203in}{2.127790in}}{\pgfqpoint{1.293027in}{2.121966in}}%
\pgfpathcurveto{\pgfqpoint{1.298851in}{2.116142in}}{\pgfqpoint{1.306751in}{2.112870in}}{\pgfqpoint{1.314988in}{2.112870in}}%
\pgfpathclose%
\pgfusepath{stroke,fill}%
\end{pgfscope}%
\begin{pgfscope}%
\pgfpathrectangle{\pgfqpoint{0.100000in}{0.212622in}}{\pgfqpoint{3.696000in}{3.696000in}}%
\pgfusepath{clip}%
\pgfsetbuttcap%
\pgfsetroundjoin%
\definecolor{currentfill}{rgb}{0.121569,0.466667,0.705882}%
\pgfsetfillcolor{currentfill}%
\pgfsetfillopacity{0.899752}%
\pgfsetlinewidth{1.003750pt}%
\definecolor{currentstroke}{rgb}{0.121569,0.466667,0.705882}%
\pgfsetstrokecolor{currentstroke}%
\pgfsetstrokeopacity{0.899752}%
\pgfsetdash{}{0pt}%
\pgfpathmoveto{\pgfqpoint{1.064391in}{2.368339in}}%
\pgfpathcurveto{\pgfqpoint{1.072628in}{2.368339in}}{\pgfqpoint{1.080528in}{2.371611in}}{\pgfqpoint{1.086352in}{2.377435in}}%
\pgfpathcurveto{\pgfqpoint{1.092175in}{2.383259in}}{\pgfqpoint{1.095448in}{2.391159in}}{\pgfqpoint{1.095448in}{2.399395in}}%
\pgfpathcurveto{\pgfqpoint{1.095448in}{2.407631in}}{\pgfqpoint{1.092175in}{2.415531in}}{\pgfqpoint{1.086352in}{2.421355in}}%
\pgfpathcurveto{\pgfqpoint{1.080528in}{2.427179in}}{\pgfqpoint{1.072628in}{2.430452in}}{\pgfqpoint{1.064391in}{2.430452in}}%
\pgfpathcurveto{\pgfqpoint{1.056155in}{2.430452in}}{\pgfqpoint{1.048255in}{2.427179in}}{\pgfqpoint{1.042431in}{2.421355in}}%
\pgfpathcurveto{\pgfqpoint{1.036607in}{2.415531in}}{\pgfqpoint{1.033335in}{2.407631in}}{\pgfqpoint{1.033335in}{2.399395in}}%
\pgfpathcurveto{\pgfqpoint{1.033335in}{2.391159in}}{\pgfqpoint{1.036607in}{2.383259in}}{\pgfqpoint{1.042431in}{2.377435in}}%
\pgfpathcurveto{\pgfqpoint{1.048255in}{2.371611in}}{\pgfqpoint{1.056155in}{2.368339in}}{\pgfqpoint{1.064391in}{2.368339in}}%
\pgfpathclose%
\pgfusepath{stroke,fill}%
\end{pgfscope}%
\begin{pgfscope}%
\pgfpathrectangle{\pgfqpoint{0.100000in}{0.212622in}}{\pgfqpoint{3.696000in}{3.696000in}}%
\pgfusepath{clip}%
\pgfsetbuttcap%
\pgfsetroundjoin%
\definecolor{currentfill}{rgb}{0.121569,0.466667,0.705882}%
\pgfsetfillcolor{currentfill}%
\pgfsetfillopacity{0.900922}%
\pgfsetlinewidth{1.003750pt}%
\definecolor{currentstroke}{rgb}{0.121569,0.466667,0.705882}%
\pgfsetstrokecolor{currentstroke}%
\pgfsetstrokeopacity{0.900922}%
\pgfsetdash{}{0pt}%
\pgfpathmoveto{\pgfqpoint{1.335689in}{2.102493in}}%
\pgfpathcurveto{\pgfqpoint{1.343925in}{2.102493in}}{\pgfqpoint{1.351825in}{2.105765in}}{\pgfqpoint{1.357649in}{2.111589in}}%
\pgfpathcurveto{\pgfqpoint{1.363473in}{2.117413in}}{\pgfqpoint{1.366745in}{2.125313in}}{\pgfqpoint{1.366745in}{2.133549in}}%
\pgfpathcurveto{\pgfqpoint{1.366745in}{2.141786in}}{\pgfqpoint{1.363473in}{2.149686in}}{\pgfqpoint{1.357649in}{2.155510in}}%
\pgfpathcurveto{\pgfqpoint{1.351825in}{2.161334in}}{\pgfqpoint{1.343925in}{2.164606in}}{\pgfqpoint{1.335689in}{2.164606in}}%
\pgfpathcurveto{\pgfqpoint{1.327453in}{2.164606in}}{\pgfqpoint{1.319552in}{2.161334in}}{\pgfqpoint{1.313729in}{2.155510in}}%
\pgfpathcurveto{\pgfqpoint{1.307905in}{2.149686in}}{\pgfqpoint{1.304632in}{2.141786in}}{\pgfqpoint{1.304632in}{2.133549in}}%
\pgfpathcurveto{\pgfqpoint{1.304632in}{2.125313in}}{\pgfqpoint{1.307905in}{2.117413in}}{\pgfqpoint{1.313729in}{2.111589in}}%
\pgfpathcurveto{\pgfqpoint{1.319552in}{2.105765in}}{\pgfqpoint{1.327453in}{2.102493in}}{\pgfqpoint{1.335689in}{2.102493in}}%
\pgfpathclose%
\pgfusepath{stroke,fill}%
\end{pgfscope}%
\begin{pgfscope}%
\pgfpathrectangle{\pgfqpoint{0.100000in}{0.212622in}}{\pgfqpoint{3.696000in}{3.696000in}}%
\pgfusepath{clip}%
\pgfsetbuttcap%
\pgfsetroundjoin%
\definecolor{currentfill}{rgb}{0.121569,0.466667,0.705882}%
\pgfsetfillcolor{currentfill}%
\pgfsetfillopacity{0.902173}%
\pgfsetlinewidth{1.003750pt}%
\definecolor{currentstroke}{rgb}{0.121569,0.466667,0.705882}%
\pgfsetstrokecolor{currentstroke}%
\pgfsetstrokeopacity{0.902173}%
\pgfsetdash{}{0pt}%
\pgfpathmoveto{\pgfqpoint{1.078029in}{2.371863in}}%
\pgfpathcurveto{\pgfqpoint{1.086265in}{2.371863in}}{\pgfqpoint{1.094165in}{2.375136in}}{\pgfqpoint{1.099989in}{2.380960in}}%
\pgfpathcurveto{\pgfqpoint{1.105813in}{2.386784in}}{\pgfqpoint{1.109085in}{2.394684in}}{\pgfqpoint{1.109085in}{2.402920in}}%
\pgfpathcurveto{\pgfqpoint{1.109085in}{2.411156in}}{\pgfqpoint{1.105813in}{2.419056in}}{\pgfqpoint{1.099989in}{2.424880in}}%
\pgfpathcurveto{\pgfqpoint{1.094165in}{2.430704in}}{\pgfqpoint{1.086265in}{2.433976in}}{\pgfqpoint{1.078029in}{2.433976in}}%
\pgfpathcurveto{\pgfqpoint{1.069793in}{2.433976in}}{\pgfqpoint{1.061893in}{2.430704in}}{\pgfqpoint{1.056069in}{2.424880in}}%
\pgfpathcurveto{\pgfqpoint{1.050245in}{2.419056in}}{\pgfqpoint{1.046972in}{2.411156in}}{\pgfqpoint{1.046972in}{2.402920in}}%
\pgfpathcurveto{\pgfqpoint{1.046972in}{2.394684in}}{\pgfqpoint{1.050245in}{2.386784in}}{\pgfqpoint{1.056069in}{2.380960in}}%
\pgfpathcurveto{\pgfqpoint{1.061893in}{2.375136in}}{\pgfqpoint{1.069793in}{2.371863in}}{\pgfqpoint{1.078029in}{2.371863in}}%
\pgfpathclose%
\pgfusepath{stroke,fill}%
\end{pgfscope}%
\begin{pgfscope}%
\pgfpathrectangle{\pgfqpoint{0.100000in}{0.212622in}}{\pgfqpoint{3.696000in}{3.696000in}}%
\pgfusepath{clip}%
\pgfsetbuttcap%
\pgfsetroundjoin%
\definecolor{currentfill}{rgb}{0.121569,0.466667,0.705882}%
\pgfsetfillcolor{currentfill}%
\pgfsetfillopacity{0.903125}%
\pgfsetlinewidth{1.003750pt}%
\definecolor{currentstroke}{rgb}{0.121569,0.466667,0.705882}%
\pgfsetstrokecolor{currentstroke}%
\pgfsetstrokeopacity{0.903125}%
\pgfsetdash{}{0pt}%
\pgfpathmoveto{\pgfqpoint{1.353928in}{2.095982in}}%
\pgfpathcurveto{\pgfqpoint{1.362164in}{2.095982in}}{\pgfqpoint{1.370064in}{2.099254in}}{\pgfqpoint{1.375888in}{2.105078in}}%
\pgfpathcurveto{\pgfqpoint{1.381712in}{2.110902in}}{\pgfqpoint{1.384984in}{2.118802in}}{\pgfqpoint{1.384984in}{2.127038in}}%
\pgfpathcurveto{\pgfqpoint{1.384984in}{2.135275in}}{\pgfqpoint{1.381712in}{2.143175in}}{\pgfqpoint{1.375888in}{2.148999in}}%
\pgfpathcurveto{\pgfqpoint{1.370064in}{2.154822in}}{\pgfqpoint{1.362164in}{2.158095in}}{\pgfqpoint{1.353928in}{2.158095in}}%
\pgfpathcurveto{\pgfqpoint{1.345691in}{2.158095in}}{\pgfqpoint{1.337791in}{2.154822in}}{\pgfqpoint{1.331967in}{2.148999in}}%
\pgfpathcurveto{\pgfqpoint{1.326143in}{2.143175in}}{\pgfqpoint{1.322871in}{2.135275in}}{\pgfqpoint{1.322871in}{2.127038in}}%
\pgfpathcurveto{\pgfqpoint{1.322871in}{2.118802in}}{\pgfqpoint{1.326143in}{2.110902in}}{\pgfqpoint{1.331967in}{2.105078in}}%
\pgfpathcurveto{\pgfqpoint{1.337791in}{2.099254in}}{\pgfqpoint{1.345691in}{2.095982in}}{\pgfqpoint{1.353928in}{2.095982in}}%
\pgfpathclose%
\pgfusepath{stroke,fill}%
\end{pgfscope}%
\begin{pgfscope}%
\pgfpathrectangle{\pgfqpoint{0.100000in}{0.212622in}}{\pgfqpoint{3.696000in}{3.696000in}}%
\pgfusepath{clip}%
\pgfsetbuttcap%
\pgfsetroundjoin%
\definecolor{currentfill}{rgb}{0.121569,0.466667,0.705882}%
\pgfsetfillcolor{currentfill}%
\pgfsetfillopacity{0.904213}%
\pgfsetlinewidth{1.003750pt}%
\definecolor{currentstroke}{rgb}{0.121569,0.466667,0.705882}%
\pgfsetstrokecolor{currentstroke}%
\pgfsetstrokeopacity{0.904213}%
\pgfsetdash{}{0pt}%
\pgfpathmoveto{\pgfqpoint{1.363711in}{2.091053in}}%
\pgfpathcurveto{\pgfqpoint{1.371948in}{2.091053in}}{\pgfqpoint{1.379848in}{2.094325in}}{\pgfqpoint{1.385672in}{2.100149in}}%
\pgfpathcurveto{\pgfqpoint{1.391496in}{2.105973in}}{\pgfqpoint{1.394768in}{2.113873in}}{\pgfqpoint{1.394768in}{2.122110in}}%
\pgfpathcurveto{\pgfqpoint{1.394768in}{2.130346in}}{\pgfqpoint{1.391496in}{2.138246in}}{\pgfqpoint{1.385672in}{2.144070in}}%
\pgfpathcurveto{\pgfqpoint{1.379848in}{2.149894in}}{\pgfqpoint{1.371948in}{2.153166in}}{\pgfqpoint{1.363711in}{2.153166in}}%
\pgfpathcurveto{\pgfqpoint{1.355475in}{2.153166in}}{\pgfqpoint{1.347575in}{2.149894in}}{\pgfqpoint{1.341751in}{2.144070in}}%
\pgfpathcurveto{\pgfqpoint{1.335927in}{2.138246in}}{\pgfqpoint{1.332655in}{2.130346in}}{\pgfqpoint{1.332655in}{2.122110in}}%
\pgfpathcurveto{\pgfqpoint{1.332655in}{2.113873in}}{\pgfqpoint{1.335927in}{2.105973in}}{\pgfqpoint{1.341751in}{2.100149in}}%
\pgfpathcurveto{\pgfqpoint{1.347575in}{2.094325in}}{\pgfqpoint{1.355475in}{2.091053in}}{\pgfqpoint{1.363711in}{2.091053in}}%
\pgfpathclose%
\pgfusepath{stroke,fill}%
\end{pgfscope}%
\begin{pgfscope}%
\pgfpathrectangle{\pgfqpoint{0.100000in}{0.212622in}}{\pgfqpoint{3.696000in}{3.696000in}}%
\pgfusepath{clip}%
\pgfsetbuttcap%
\pgfsetroundjoin%
\definecolor{currentfill}{rgb}{0.121569,0.466667,0.705882}%
\pgfsetfillcolor{currentfill}%
\pgfsetfillopacity{0.904649}%
\pgfsetlinewidth{1.003750pt}%
\definecolor{currentstroke}{rgb}{0.121569,0.466667,0.705882}%
\pgfsetstrokecolor{currentstroke}%
\pgfsetstrokeopacity{0.904649}%
\pgfsetdash{}{0pt}%
\pgfpathmoveto{\pgfqpoint{1.367136in}{2.089844in}}%
\pgfpathcurveto{\pgfqpoint{1.375372in}{2.089844in}}{\pgfqpoint{1.383272in}{2.093117in}}{\pgfqpoint{1.389096in}{2.098941in}}%
\pgfpathcurveto{\pgfqpoint{1.394920in}{2.104765in}}{\pgfqpoint{1.398192in}{2.112665in}}{\pgfqpoint{1.398192in}{2.120901in}}%
\pgfpathcurveto{\pgfqpoint{1.398192in}{2.129137in}}{\pgfqpoint{1.394920in}{2.137037in}}{\pgfqpoint{1.389096in}{2.142861in}}%
\pgfpathcurveto{\pgfqpoint{1.383272in}{2.148685in}}{\pgfqpoint{1.375372in}{2.151957in}}{\pgfqpoint{1.367136in}{2.151957in}}%
\pgfpathcurveto{\pgfqpoint{1.358899in}{2.151957in}}{\pgfqpoint{1.350999in}{2.148685in}}{\pgfqpoint{1.345175in}{2.142861in}}%
\pgfpathcurveto{\pgfqpoint{1.339351in}{2.137037in}}{\pgfqpoint{1.336079in}{2.129137in}}{\pgfqpoint{1.336079in}{2.120901in}}%
\pgfpathcurveto{\pgfqpoint{1.336079in}{2.112665in}}{\pgfqpoint{1.339351in}{2.104765in}}{\pgfqpoint{1.345175in}{2.098941in}}%
\pgfpathcurveto{\pgfqpoint{1.350999in}{2.093117in}}{\pgfqpoint{1.358899in}{2.089844in}}{\pgfqpoint{1.367136in}{2.089844in}}%
\pgfpathclose%
\pgfusepath{stroke,fill}%
\end{pgfscope}%
\begin{pgfscope}%
\pgfpathrectangle{\pgfqpoint{0.100000in}{0.212622in}}{\pgfqpoint{3.696000in}{3.696000in}}%
\pgfusepath{clip}%
\pgfsetbuttcap%
\pgfsetroundjoin%
\definecolor{currentfill}{rgb}{0.121569,0.466667,0.705882}%
\pgfsetfillcolor{currentfill}%
\pgfsetfillopacity{0.905069}%
\pgfsetlinewidth{1.003750pt}%
\definecolor{currentstroke}{rgb}{0.121569,0.466667,0.705882}%
\pgfsetstrokecolor{currentstroke}%
\pgfsetstrokeopacity{0.905069}%
\pgfsetdash{}{0pt}%
\pgfpathmoveto{\pgfqpoint{1.094139in}{2.376507in}}%
\pgfpathcurveto{\pgfqpoint{1.102376in}{2.376507in}}{\pgfqpoint{1.110276in}{2.379779in}}{\pgfqpoint{1.116100in}{2.385603in}}%
\pgfpathcurveto{\pgfqpoint{1.121924in}{2.391427in}}{\pgfqpoint{1.125196in}{2.399327in}}{\pgfqpoint{1.125196in}{2.407564in}}%
\pgfpathcurveto{\pgfqpoint{1.125196in}{2.415800in}}{\pgfqpoint{1.121924in}{2.423700in}}{\pgfqpoint{1.116100in}{2.429524in}}%
\pgfpathcurveto{\pgfqpoint{1.110276in}{2.435348in}}{\pgfqpoint{1.102376in}{2.438620in}}{\pgfqpoint{1.094139in}{2.438620in}}%
\pgfpathcurveto{\pgfqpoint{1.085903in}{2.438620in}}{\pgfqpoint{1.078003in}{2.435348in}}{\pgfqpoint{1.072179in}{2.429524in}}%
\pgfpathcurveto{\pgfqpoint{1.066355in}{2.423700in}}{\pgfqpoint{1.063083in}{2.415800in}}{\pgfqpoint{1.063083in}{2.407564in}}%
\pgfpathcurveto{\pgfqpoint{1.063083in}{2.399327in}}{\pgfqpoint{1.066355in}{2.391427in}}{\pgfqpoint{1.072179in}{2.385603in}}%
\pgfpathcurveto{\pgfqpoint{1.078003in}{2.379779in}}{\pgfqpoint{1.085903in}{2.376507in}}{\pgfqpoint{1.094139in}{2.376507in}}%
\pgfpathclose%
\pgfusepath{stroke,fill}%
\end{pgfscope}%
\begin{pgfscope}%
\pgfpathrectangle{\pgfqpoint{0.100000in}{0.212622in}}{\pgfqpoint{3.696000in}{3.696000in}}%
\pgfusepath{clip}%
\pgfsetbuttcap%
\pgfsetroundjoin%
\definecolor{currentfill}{rgb}{0.121569,0.466667,0.705882}%
\pgfsetfillcolor{currentfill}%
\pgfsetfillopacity{0.905334}%
\pgfsetlinewidth{1.003750pt}%
\definecolor{currentstroke}{rgb}{0.121569,0.466667,0.705882}%
\pgfsetstrokecolor{currentstroke}%
\pgfsetstrokeopacity{0.905334}%
\pgfsetdash{}{0pt}%
\pgfpathmoveto{\pgfqpoint{1.373255in}{2.086567in}}%
\pgfpathcurveto{\pgfqpoint{1.381492in}{2.086567in}}{\pgfqpoint{1.389392in}{2.089839in}}{\pgfqpoint{1.395216in}{2.095663in}}%
\pgfpathcurveto{\pgfqpoint{1.401040in}{2.101487in}}{\pgfqpoint{1.404312in}{2.109387in}}{\pgfqpoint{1.404312in}{2.117623in}}%
\pgfpathcurveto{\pgfqpoint{1.404312in}{2.125859in}}{\pgfqpoint{1.401040in}{2.133759in}}{\pgfqpoint{1.395216in}{2.139583in}}%
\pgfpathcurveto{\pgfqpoint{1.389392in}{2.145407in}}{\pgfqpoint{1.381492in}{2.148680in}}{\pgfqpoint{1.373255in}{2.148680in}}%
\pgfpathcurveto{\pgfqpoint{1.365019in}{2.148680in}}{\pgfqpoint{1.357119in}{2.145407in}}{\pgfqpoint{1.351295in}{2.139583in}}%
\pgfpathcurveto{\pgfqpoint{1.345471in}{2.133759in}}{\pgfqpoint{1.342199in}{2.125859in}}{\pgfqpoint{1.342199in}{2.117623in}}%
\pgfpathcurveto{\pgfqpoint{1.342199in}{2.109387in}}{\pgfqpoint{1.345471in}{2.101487in}}{\pgfqpoint{1.351295in}{2.095663in}}%
\pgfpathcurveto{\pgfqpoint{1.357119in}{2.089839in}}{\pgfqpoint{1.365019in}{2.086567in}}{\pgfqpoint{1.373255in}{2.086567in}}%
\pgfpathclose%
\pgfusepath{stroke,fill}%
\end{pgfscope}%
\begin{pgfscope}%
\pgfpathrectangle{\pgfqpoint{0.100000in}{0.212622in}}{\pgfqpoint{3.696000in}{3.696000in}}%
\pgfusepath{clip}%
\pgfsetbuttcap%
\pgfsetroundjoin%
\definecolor{currentfill}{rgb}{0.121569,0.466667,0.705882}%
\pgfsetfillcolor{currentfill}%
\pgfsetfillopacity{0.906655}%
\pgfsetlinewidth{1.003750pt}%
\definecolor{currentstroke}{rgb}{0.121569,0.466667,0.705882}%
\pgfsetstrokecolor{currentstroke}%
\pgfsetstrokeopacity{0.906655}%
\pgfsetdash{}{0pt}%
\pgfpathmoveto{\pgfqpoint{1.384785in}{2.082487in}}%
\pgfpathcurveto{\pgfqpoint{1.393022in}{2.082487in}}{\pgfqpoint{1.400922in}{2.085759in}}{\pgfqpoint{1.406746in}{2.091583in}}%
\pgfpathcurveto{\pgfqpoint{1.412570in}{2.097407in}}{\pgfqpoint{1.415842in}{2.105307in}}{\pgfqpoint{1.415842in}{2.113543in}}%
\pgfpathcurveto{\pgfqpoint{1.415842in}{2.121779in}}{\pgfqpoint{1.412570in}{2.129679in}}{\pgfqpoint{1.406746in}{2.135503in}}%
\pgfpathcurveto{\pgfqpoint{1.400922in}{2.141327in}}{\pgfqpoint{1.393022in}{2.144600in}}{\pgfqpoint{1.384785in}{2.144600in}}%
\pgfpathcurveto{\pgfqpoint{1.376549in}{2.144600in}}{\pgfqpoint{1.368649in}{2.141327in}}{\pgfqpoint{1.362825in}{2.135503in}}%
\pgfpathcurveto{\pgfqpoint{1.357001in}{2.129679in}}{\pgfqpoint{1.353729in}{2.121779in}}{\pgfqpoint{1.353729in}{2.113543in}}%
\pgfpathcurveto{\pgfqpoint{1.353729in}{2.105307in}}{\pgfqpoint{1.357001in}{2.097407in}}{\pgfqpoint{1.362825in}{2.091583in}}%
\pgfpathcurveto{\pgfqpoint{1.368649in}{2.085759in}}{\pgfqpoint{1.376549in}{2.082487in}}{\pgfqpoint{1.384785in}{2.082487in}}%
\pgfpathclose%
\pgfusepath{stroke,fill}%
\end{pgfscope}%
\begin{pgfscope}%
\pgfpathrectangle{\pgfqpoint{0.100000in}{0.212622in}}{\pgfqpoint{3.696000in}{3.696000in}}%
\pgfusepath{clip}%
\pgfsetbuttcap%
\pgfsetroundjoin%
\definecolor{currentfill}{rgb}{0.121569,0.466667,0.705882}%
\pgfsetfillcolor{currentfill}%
\pgfsetfillopacity{0.906731}%
\pgfsetlinewidth{1.003750pt}%
\definecolor{currentstroke}{rgb}{0.121569,0.466667,0.705882}%
\pgfsetstrokecolor{currentstroke}%
\pgfsetstrokeopacity{0.906731}%
\pgfsetdash{}{0pt}%
\pgfpathmoveto{\pgfqpoint{1.102698in}{2.377888in}}%
\pgfpathcurveto{\pgfqpoint{1.110934in}{2.377888in}}{\pgfqpoint{1.118834in}{2.381160in}}{\pgfqpoint{1.124658in}{2.386984in}}%
\pgfpathcurveto{\pgfqpoint{1.130482in}{2.392808in}}{\pgfqpoint{1.133754in}{2.400708in}}{\pgfqpoint{1.133754in}{2.408945in}}%
\pgfpathcurveto{\pgfqpoint{1.133754in}{2.417181in}}{\pgfqpoint{1.130482in}{2.425081in}}{\pgfqpoint{1.124658in}{2.430905in}}%
\pgfpathcurveto{\pgfqpoint{1.118834in}{2.436729in}}{\pgfqpoint{1.110934in}{2.440001in}}{\pgfqpoint{1.102698in}{2.440001in}}%
\pgfpathcurveto{\pgfqpoint{1.094461in}{2.440001in}}{\pgfqpoint{1.086561in}{2.436729in}}{\pgfqpoint{1.080737in}{2.430905in}}%
\pgfpathcurveto{\pgfqpoint{1.074914in}{2.425081in}}{\pgfqpoint{1.071641in}{2.417181in}}{\pgfqpoint{1.071641in}{2.408945in}}%
\pgfpathcurveto{\pgfqpoint{1.071641in}{2.400708in}}{\pgfqpoint{1.074914in}{2.392808in}}{\pgfqpoint{1.080737in}{2.386984in}}%
\pgfpathcurveto{\pgfqpoint{1.086561in}{2.381160in}}{\pgfqpoint{1.094461in}{2.377888in}}{\pgfqpoint{1.102698in}{2.377888in}}%
\pgfpathclose%
\pgfusepath{stroke,fill}%
\end{pgfscope}%
\begin{pgfscope}%
\pgfpathrectangle{\pgfqpoint{0.100000in}{0.212622in}}{\pgfqpoint{3.696000in}{3.696000in}}%
\pgfusepath{clip}%
\pgfsetbuttcap%
\pgfsetroundjoin%
\definecolor{currentfill}{rgb}{0.121569,0.466667,0.705882}%
\pgfsetfillcolor{currentfill}%
\pgfsetfillopacity{0.907647}%
\pgfsetlinewidth{1.003750pt}%
\definecolor{currentstroke}{rgb}{0.121569,0.466667,0.705882}%
\pgfsetstrokecolor{currentstroke}%
\pgfsetstrokeopacity{0.907647}%
\pgfsetdash{}{0pt}%
\pgfpathmoveto{\pgfqpoint{1.107530in}{2.379273in}}%
\pgfpathcurveto{\pgfqpoint{1.115766in}{2.379273in}}{\pgfqpoint{1.123666in}{2.382545in}}{\pgfqpoint{1.129490in}{2.388369in}}%
\pgfpathcurveto{\pgfqpoint{1.135314in}{2.394193in}}{\pgfqpoint{1.138586in}{2.402093in}}{\pgfqpoint{1.138586in}{2.410329in}}%
\pgfpathcurveto{\pgfqpoint{1.138586in}{2.418566in}}{\pgfqpoint{1.135314in}{2.426466in}}{\pgfqpoint{1.129490in}{2.432290in}}%
\pgfpathcurveto{\pgfqpoint{1.123666in}{2.438114in}}{\pgfqpoint{1.115766in}{2.441386in}}{\pgfqpoint{1.107530in}{2.441386in}}%
\pgfpathcurveto{\pgfqpoint{1.099293in}{2.441386in}}{\pgfqpoint{1.091393in}{2.438114in}}{\pgfqpoint{1.085569in}{2.432290in}}%
\pgfpathcurveto{\pgfqpoint{1.079746in}{2.426466in}}{\pgfqpoint{1.076473in}{2.418566in}}{\pgfqpoint{1.076473in}{2.410329in}}%
\pgfpathcurveto{\pgfqpoint{1.076473in}{2.402093in}}{\pgfqpoint{1.079746in}{2.394193in}}{\pgfqpoint{1.085569in}{2.388369in}}%
\pgfpathcurveto{\pgfqpoint{1.091393in}{2.382545in}}{\pgfqpoint{1.099293in}{2.379273in}}{\pgfqpoint{1.107530in}{2.379273in}}%
\pgfpathclose%
\pgfusepath{stroke,fill}%
\end{pgfscope}%
\begin{pgfscope}%
\pgfpathrectangle{\pgfqpoint{0.100000in}{0.212622in}}{\pgfqpoint{3.696000in}{3.696000in}}%
\pgfusepath{clip}%
\pgfsetbuttcap%
\pgfsetroundjoin%
\definecolor{currentfill}{rgb}{0.121569,0.466667,0.705882}%
\pgfsetfillcolor{currentfill}%
\pgfsetfillopacity{0.908807}%
\pgfsetlinewidth{1.003750pt}%
\definecolor{currentstroke}{rgb}{0.121569,0.466667,0.705882}%
\pgfsetstrokecolor{currentstroke}%
\pgfsetstrokeopacity{0.908807}%
\pgfsetdash{}{0pt}%
\pgfpathmoveto{\pgfqpoint{1.405813in}{2.073529in}}%
\pgfpathcurveto{\pgfqpoint{1.414049in}{2.073529in}}{\pgfqpoint{1.421949in}{2.076802in}}{\pgfqpoint{1.427773in}{2.082626in}}%
\pgfpathcurveto{\pgfqpoint{1.433597in}{2.088450in}}{\pgfqpoint{1.436869in}{2.096350in}}{\pgfqpoint{1.436869in}{2.104586in}}%
\pgfpathcurveto{\pgfqpoint{1.436869in}{2.112822in}}{\pgfqpoint{1.433597in}{2.120722in}}{\pgfqpoint{1.427773in}{2.126546in}}%
\pgfpathcurveto{\pgfqpoint{1.421949in}{2.132370in}}{\pgfqpoint{1.414049in}{2.135642in}}{\pgfqpoint{1.405813in}{2.135642in}}%
\pgfpathcurveto{\pgfqpoint{1.397577in}{2.135642in}}{\pgfqpoint{1.389677in}{2.132370in}}{\pgfqpoint{1.383853in}{2.126546in}}%
\pgfpathcurveto{\pgfqpoint{1.378029in}{2.120722in}}{\pgfqpoint{1.374756in}{2.112822in}}{\pgfqpoint{1.374756in}{2.104586in}}%
\pgfpathcurveto{\pgfqpoint{1.374756in}{2.096350in}}{\pgfqpoint{1.378029in}{2.088450in}}{\pgfqpoint{1.383853in}{2.082626in}}%
\pgfpathcurveto{\pgfqpoint{1.389677in}{2.076802in}}{\pgfqpoint{1.397577in}{2.073529in}}{\pgfqpoint{1.405813in}{2.073529in}}%
\pgfpathclose%
\pgfusepath{stroke,fill}%
\end{pgfscope}%
\begin{pgfscope}%
\pgfpathrectangle{\pgfqpoint{0.100000in}{0.212622in}}{\pgfqpoint{3.696000in}{3.696000in}}%
\pgfusepath{clip}%
\pgfsetbuttcap%
\pgfsetroundjoin%
\definecolor{currentfill}{rgb}{0.121569,0.466667,0.705882}%
\pgfsetfillcolor{currentfill}%
\pgfsetfillopacity{0.909088}%
\pgfsetlinewidth{1.003750pt}%
\definecolor{currentstroke}{rgb}{0.121569,0.466667,0.705882}%
\pgfsetstrokecolor{currentstroke}%
\pgfsetstrokeopacity{0.909088}%
\pgfsetdash{}{0pt}%
\pgfpathmoveto{\pgfqpoint{1.116140in}{2.380089in}}%
\pgfpathcurveto{\pgfqpoint{1.124376in}{2.380089in}}{\pgfqpoint{1.132276in}{2.383361in}}{\pgfqpoint{1.138100in}{2.389185in}}%
\pgfpathcurveto{\pgfqpoint{1.143924in}{2.395009in}}{\pgfqpoint{1.147196in}{2.402909in}}{\pgfqpoint{1.147196in}{2.411146in}}%
\pgfpathcurveto{\pgfqpoint{1.147196in}{2.419382in}}{\pgfqpoint{1.143924in}{2.427282in}}{\pgfqpoint{1.138100in}{2.433106in}}%
\pgfpathcurveto{\pgfqpoint{1.132276in}{2.438930in}}{\pgfqpoint{1.124376in}{2.442202in}}{\pgfqpoint{1.116140in}{2.442202in}}%
\pgfpathcurveto{\pgfqpoint{1.107903in}{2.442202in}}{\pgfqpoint{1.100003in}{2.438930in}}{\pgfqpoint{1.094179in}{2.433106in}}%
\pgfpathcurveto{\pgfqpoint{1.088355in}{2.427282in}}{\pgfqpoint{1.085083in}{2.419382in}}{\pgfqpoint{1.085083in}{2.411146in}}%
\pgfpathcurveto{\pgfqpoint{1.085083in}{2.402909in}}{\pgfqpoint{1.088355in}{2.395009in}}{\pgfqpoint{1.094179in}{2.389185in}}%
\pgfpathcurveto{\pgfqpoint{1.100003in}{2.383361in}}{\pgfqpoint{1.107903in}{2.380089in}}{\pgfqpoint{1.116140in}{2.380089in}}%
\pgfpathclose%
\pgfusepath{stroke,fill}%
\end{pgfscope}%
\begin{pgfscope}%
\pgfpathrectangle{\pgfqpoint{0.100000in}{0.212622in}}{\pgfqpoint{3.696000in}{3.696000in}}%
\pgfusepath{clip}%
\pgfsetbuttcap%
\pgfsetroundjoin%
\definecolor{currentfill}{rgb}{0.121569,0.466667,0.705882}%
\pgfsetfillcolor{currentfill}%
\pgfsetfillopacity{0.910185}%
\pgfsetlinewidth{1.003750pt}%
\definecolor{currentstroke}{rgb}{0.121569,0.466667,0.705882}%
\pgfsetstrokecolor{currentstroke}%
\pgfsetstrokeopacity{0.910185}%
\pgfsetdash{}{0pt}%
\pgfpathmoveto{\pgfqpoint{1.420445in}{2.067187in}}%
\pgfpathcurveto{\pgfqpoint{1.428682in}{2.067187in}}{\pgfqpoint{1.436582in}{2.070459in}}{\pgfqpoint{1.442406in}{2.076283in}}%
\pgfpathcurveto{\pgfqpoint{1.448230in}{2.082107in}}{\pgfqpoint{1.451502in}{2.090007in}}{\pgfqpoint{1.451502in}{2.098243in}}%
\pgfpathcurveto{\pgfqpoint{1.451502in}{2.106480in}}{\pgfqpoint{1.448230in}{2.114380in}}{\pgfqpoint{1.442406in}{2.120204in}}%
\pgfpathcurveto{\pgfqpoint{1.436582in}{2.126028in}}{\pgfqpoint{1.428682in}{2.129300in}}{\pgfqpoint{1.420445in}{2.129300in}}%
\pgfpathcurveto{\pgfqpoint{1.412209in}{2.129300in}}{\pgfqpoint{1.404309in}{2.126028in}}{\pgfqpoint{1.398485in}{2.120204in}}%
\pgfpathcurveto{\pgfqpoint{1.392661in}{2.114380in}}{\pgfqpoint{1.389389in}{2.106480in}}{\pgfqpoint{1.389389in}{2.098243in}}%
\pgfpathcurveto{\pgfqpoint{1.389389in}{2.090007in}}{\pgfqpoint{1.392661in}{2.082107in}}{\pgfqpoint{1.398485in}{2.076283in}}%
\pgfpathcurveto{\pgfqpoint{1.404309in}{2.070459in}}{\pgfqpoint{1.412209in}{2.067187in}}{\pgfqpoint{1.420445in}{2.067187in}}%
\pgfpathclose%
\pgfusepath{stroke,fill}%
\end{pgfscope}%
\begin{pgfscope}%
\pgfpathrectangle{\pgfqpoint{0.100000in}{0.212622in}}{\pgfqpoint{3.696000in}{3.696000in}}%
\pgfusepath{clip}%
\pgfsetbuttcap%
\pgfsetroundjoin%
\definecolor{currentfill}{rgb}{0.121569,0.466667,0.705882}%
\pgfsetfillcolor{currentfill}%
\pgfsetfillopacity{0.911111}%
\pgfsetlinewidth{1.003750pt}%
\definecolor{currentstroke}{rgb}{0.121569,0.466667,0.705882}%
\pgfsetstrokecolor{currentstroke}%
\pgfsetstrokeopacity{0.911111}%
\pgfsetdash{}{0pt}%
\pgfpathmoveto{\pgfqpoint{1.428866in}{2.063939in}}%
\pgfpathcurveto{\pgfqpoint{1.437103in}{2.063939in}}{\pgfqpoint{1.445003in}{2.067212in}}{\pgfqpoint{1.450827in}{2.073036in}}%
\pgfpathcurveto{\pgfqpoint{1.456651in}{2.078860in}}{\pgfqpoint{1.459923in}{2.086760in}}{\pgfqpoint{1.459923in}{2.094996in}}%
\pgfpathcurveto{\pgfqpoint{1.459923in}{2.103232in}}{\pgfqpoint{1.456651in}{2.111132in}}{\pgfqpoint{1.450827in}{2.116956in}}%
\pgfpathcurveto{\pgfqpoint{1.445003in}{2.122780in}}{\pgfqpoint{1.437103in}{2.126052in}}{\pgfqpoint{1.428866in}{2.126052in}}%
\pgfpathcurveto{\pgfqpoint{1.420630in}{2.126052in}}{\pgfqpoint{1.412730in}{2.122780in}}{\pgfqpoint{1.406906in}{2.116956in}}%
\pgfpathcurveto{\pgfqpoint{1.401082in}{2.111132in}}{\pgfqpoint{1.397810in}{2.103232in}}{\pgfqpoint{1.397810in}{2.094996in}}%
\pgfpathcurveto{\pgfqpoint{1.397810in}{2.086760in}}{\pgfqpoint{1.401082in}{2.078860in}}{\pgfqpoint{1.406906in}{2.073036in}}%
\pgfpathcurveto{\pgfqpoint{1.412730in}{2.067212in}}{\pgfqpoint{1.420630in}{2.063939in}}{\pgfqpoint{1.428866in}{2.063939in}}%
\pgfpathclose%
\pgfusepath{stroke,fill}%
\end{pgfscope}%
\begin{pgfscope}%
\pgfpathrectangle{\pgfqpoint{0.100000in}{0.212622in}}{\pgfqpoint{3.696000in}{3.696000in}}%
\pgfusepath{clip}%
\pgfsetbuttcap%
\pgfsetroundjoin%
\definecolor{currentfill}{rgb}{0.121569,0.466667,0.705882}%
\pgfsetfillcolor{currentfill}%
\pgfsetfillopacity{0.911237}%
\pgfsetlinewidth{1.003750pt}%
\definecolor{currentstroke}{rgb}{0.121569,0.466667,0.705882}%
\pgfsetstrokecolor{currentstroke}%
\pgfsetstrokeopacity{0.911237}%
\pgfsetdash{}{0pt}%
\pgfpathmoveto{\pgfqpoint{1.129223in}{2.383268in}}%
\pgfpathcurveto{\pgfqpoint{1.137459in}{2.383268in}}{\pgfqpoint{1.145359in}{2.386540in}}{\pgfqpoint{1.151183in}{2.392364in}}%
\pgfpathcurveto{\pgfqpoint{1.157007in}{2.398188in}}{\pgfqpoint{1.160280in}{2.406088in}}{\pgfqpoint{1.160280in}{2.414324in}}%
\pgfpathcurveto{\pgfqpoint{1.160280in}{2.422561in}}{\pgfqpoint{1.157007in}{2.430461in}}{\pgfqpoint{1.151183in}{2.436285in}}%
\pgfpathcurveto{\pgfqpoint{1.145359in}{2.442109in}}{\pgfqpoint{1.137459in}{2.445381in}}{\pgfqpoint{1.129223in}{2.445381in}}%
\pgfpathcurveto{\pgfqpoint{1.120987in}{2.445381in}}{\pgfqpoint{1.113087in}{2.442109in}}{\pgfqpoint{1.107263in}{2.436285in}}%
\pgfpathcurveto{\pgfqpoint{1.101439in}{2.430461in}}{\pgfqpoint{1.098167in}{2.422561in}}{\pgfqpoint{1.098167in}{2.414324in}}%
\pgfpathcurveto{\pgfqpoint{1.098167in}{2.406088in}}{\pgfqpoint{1.101439in}{2.398188in}}{\pgfqpoint{1.107263in}{2.392364in}}%
\pgfpathcurveto{\pgfqpoint{1.113087in}{2.386540in}}{\pgfqpoint{1.120987in}{2.383268in}}{\pgfqpoint{1.129223in}{2.383268in}}%
\pgfpathclose%
\pgfusepath{stroke,fill}%
\end{pgfscope}%
\begin{pgfscope}%
\pgfpathrectangle{\pgfqpoint{0.100000in}{0.212622in}}{\pgfqpoint{3.696000in}{3.696000in}}%
\pgfusepath{clip}%
\pgfsetbuttcap%
\pgfsetroundjoin%
\definecolor{currentfill}{rgb}{0.121569,0.466667,0.705882}%
\pgfsetfillcolor{currentfill}%
\pgfsetfillopacity{0.912805}%
\pgfsetlinewidth{1.003750pt}%
\definecolor{currentstroke}{rgb}{0.121569,0.466667,0.705882}%
\pgfsetstrokecolor{currentstroke}%
\pgfsetstrokeopacity{0.912805}%
\pgfsetdash{}{0pt}%
\pgfpathmoveto{\pgfqpoint{1.444051in}{2.057571in}}%
\pgfpathcurveto{\pgfqpoint{1.452287in}{2.057571in}}{\pgfqpoint{1.460187in}{2.060844in}}{\pgfqpoint{1.466011in}{2.066667in}}%
\pgfpathcurveto{\pgfqpoint{1.471835in}{2.072491in}}{\pgfqpoint{1.475107in}{2.080391in}}{\pgfqpoint{1.475107in}{2.088628in}}%
\pgfpathcurveto{\pgfqpoint{1.475107in}{2.096864in}}{\pgfqpoint{1.471835in}{2.104764in}}{\pgfqpoint{1.466011in}{2.110588in}}%
\pgfpathcurveto{\pgfqpoint{1.460187in}{2.116412in}}{\pgfqpoint{1.452287in}{2.119684in}}{\pgfqpoint{1.444051in}{2.119684in}}%
\pgfpathcurveto{\pgfqpoint{1.435815in}{2.119684in}}{\pgfqpoint{1.427915in}{2.116412in}}{\pgfqpoint{1.422091in}{2.110588in}}%
\pgfpathcurveto{\pgfqpoint{1.416267in}{2.104764in}}{\pgfqpoint{1.412994in}{2.096864in}}{\pgfqpoint{1.412994in}{2.088628in}}%
\pgfpathcurveto{\pgfqpoint{1.412994in}{2.080391in}}{\pgfqpoint{1.416267in}{2.072491in}}{\pgfqpoint{1.422091in}{2.066667in}}%
\pgfpathcurveto{\pgfqpoint{1.427915in}{2.060844in}}{\pgfqpoint{1.435815in}{2.057571in}}{\pgfqpoint{1.444051in}{2.057571in}}%
\pgfpathclose%
\pgfusepath{stroke,fill}%
\end{pgfscope}%
\begin{pgfscope}%
\pgfpathrectangle{\pgfqpoint{0.100000in}{0.212622in}}{\pgfqpoint{3.696000in}{3.696000in}}%
\pgfusepath{clip}%
\pgfsetbuttcap%
\pgfsetroundjoin%
\definecolor{currentfill}{rgb}{0.121569,0.466667,0.705882}%
\pgfsetfillcolor{currentfill}%
\pgfsetfillopacity{0.913889}%
\pgfsetlinewidth{1.003750pt}%
\definecolor{currentstroke}{rgb}{0.121569,0.466667,0.705882}%
\pgfsetstrokecolor{currentstroke}%
\pgfsetstrokeopacity{0.913889}%
\pgfsetdash{}{0pt}%
\pgfpathmoveto{\pgfqpoint{2.414659in}{1.734517in}}%
\pgfpathcurveto{\pgfqpoint{2.422895in}{1.734517in}}{\pgfqpoint{2.430795in}{1.737790in}}{\pgfqpoint{2.436619in}{1.743614in}}%
\pgfpathcurveto{\pgfqpoint{2.442443in}{1.749438in}}{\pgfqpoint{2.445716in}{1.757338in}}{\pgfqpoint{2.445716in}{1.765574in}}%
\pgfpathcurveto{\pgfqpoint{2.445716in}{1.773810in}}{\pgfqpoint{2.442443in}{1.781710in}}{\pgfqpoint{2.436619in}{1.787534in}}%
\pgfpathcurveto{\pgfqpoint{2.430795in}{1.793358in}}{\pgfqpoint{2.422895in}{1.796630in}}{\pgfqpoint{2.414659in}{1.796630in}}%
\pgfpathcurveto{\pgfqpoint{2.406423in}{1.796630in}}{\pgfqpoint{2.398523in}{1.793358in}}{\pgfqpoint{2.392699in}{1.787534in}}%
\pgfpathcurveto{\pgfqpoint{2.386875in}{1.781710in}}{\pgfqpoint{2.383603in}{1.773810in}}{\pgfqpoint{2.383603in}{1.765574in}}%
\pgfpathcurveto{\pgfqpoint{2.383603in}{1.757338in}}{\pgfqpoint{2.386875in}{1.749438in}}{\pgfqpoint{2.392699in}{1.743614in}}%
\pgfpathcurveto{\pgfqpoint{2.398523in}{1.737790in}}{\pgfqpoint{2.406423in}{1.734517in}}{\pgfqpoint{2.414659in}{1.734517in}}%
\pgfpathclose%
\pgfusepath{stroke,fill}%
\end{pgfscope}%
\begin{pgfscope}%
\pgfpathrectangle{\pgfqpoint{0.100000in}{0.212622in}}{\pgfqpoint{3.696000in}{3.696000in}}%
\pgfusepath{clip}%
\pgfsetbuttcap%
\pgfsetroundjoin%
\definecolor{currentfill}{rgb}{0.121569,0.466667,0.705882}%
\pgfsetfillcolor{currentfill}%
\pgfsetfillopacity{0.914423}%
\pgfsetlinewidth{1.003750pt}%
\definecolor{currentstroke}{rgb}{0.121569,0.466667,0.705882}%
\pgfsetstrokecolor{currentstroke}%
\pgfsetstrokeopacity{0.914423}%
\pgfsetdash{}{0pt}%
\pgfpathmoveto{\pgfqpoint{1.145206in}{2.386563in}}%
\pgfpathcurveto{\pgfqpoint{1.153443in}{2.386563in}}{\pgfqpoint{1.161343in}{2.389836in}}{\pgfqpoint{1.167167in}{2.395660in}}%
\pgfpathcurveto{\pgfqpoint{1.172990in}{2.401484in}}{\pgfqpoint{1.176263in}{2.409384in}}{\pgfqpoint{1.176263in}{2.417620in}}%
\pgfpathcurveto{\pgfqpoint{1.176263in}{2.425856in}}{\pgfqpoint{1.172990in}{2.433756in}}{\pgfqpoint{1.167167in}{2.439580in}}%
\pgfpathcurveto{\pgfqpoint{1.161343in}{2.445404in}}{\pgfqpoint{1.153443in}{2.448676in}}{\pgfqpoint{1.145206in}{2.448676in}}%
\pgfpathcurveto{\pgfqpoint{1.136970in}{2.448676in}}{\pgfqpoint{1.129070in}{2.445404in}}{\pgfqpoint{1.123246in}{2.439580in}}%
\pgfpathcurveto{\pgfqpoint{1.117422in}{2.433756in}}{\pgfqpoint{1.114150in}{2.425856in}}{\pgfqpoint{1.114150in}{2.417620in}}%
\pgfpathcurveto{\pgfqpoint{1.114150in}{2.409384in}}{\pgfqpoint{1.117422in}{2.401484in}}{\pgfqpoint{1.123246in}{2.395660in}}%
\pgfpathcurveto{\pgfqpoint{1.129070in}{2.389836in}}{\pgfqpoint{1.136970in}{2.386563in}}{\pgfqpoint{1.145206in}{2.386563in}}%
\pgfpathclose%
\pgfusepath{stroke,fill}%
\end{pgfscope}%
\begin{pgfscope}%
\pgfpathrectangle{\pgfqpoint{0.100000in}{0.212622in}}{\pgfqpoint{3.696000in}{3.696000in}}%
\pgfusepath{clip}%
\pgfsetbuttcap%
\pgfsetroundjoin%
\definecolor{currentfill}{rgb}{0.121569,0.466667,0.705882}%
\pgfsetfillcolor{currentfill}%
\pgfsetfillopacity{0.916030}%
\pgfsetlinewidth{1.003750pt}%
\definecolor{currentstroke}{rgb}{0.121569,0.466667,0.705882}%
\pgfsetstrokecolor{currentstroke}%
\pgfsetstrokeopacity{0.916030}%
\pgfsetdash{}{0pt}%
\pgfpathmoveto{\pgfqpoint{1.471998in}{2.048171in}}%
\pgfpathcurveto{\pgfqpoint{1.480234in}{2.048171in}}{\pgfqpoint{1.488134in}{2.051444in}}{\pgfqpoint{1.493958in}{2.057268in}}%
\pgfpathcurveto{\pgfqpoint{1.499782in}{2.063092in}}{\pgfqpoint{1.503054in}{2.070992in}}{\pgfqpoint{1.503054in}{2.079228in}}%
\pgfpathcurveto{\pgfqpoint{1.503054in}{2.087464in}}{\pgfqpoint{1.499782in}{2.095364in}}{\pgfqpoint{1.493958in}{2.101188in}}%
\pgfpathcurveto{\pgfqpoint{1.488134in}{2.107012in}}{\pgfqpoint{1.480234in}{2.110284in}}{\pgfqpoint{1.471998in}{2.110284in}}%
\pgfpathcurveto{\pgfqpoint{1.463761in}{2.110284in}}{\pgfqpoint{1.455861in}{2.107012in}}{\pgfqpoint{1.450037in}{2.101188in}}%
\pgfpathcurveto{\pgfqpoint{1.444213in}{2.095364in}}{\pgfqpoint{1.440941in}{2.087464in}}{\pgfqpoint{1.440941in}{2.079228in}}%
\pgfpathcurveto{\pgfqpoint{1.440941in}{2.070992in}}{\pgfqpoint{1.444213in}{2.063092in}}{\pgfqpoint{1.450037in}{2.057268in}}%
\pgfpathcurveto{\pgfqpoint{1.455861in}{2.051444in}}{\pgfqpoint{1.463761in}{2.048171in}}{\pgfqpoint{1.471998in}{2.048171in}}%
\pgfpathclose%
\pgfusepath{stroke,fill}%
\end{pgfscope}%
\begin{pgfscope}%
\pgfpathrectangle{\pgfqpoint{0.100000in}{0.212622in}}{\pgfqpoint{3.696000in}{3.696000in}}%
\pgfusepath{clip}%
\pgfsetbuttcap%
\pgfsetroundjoin%
\definecolor{currentfill}{rgb}{0.121569,0.466667,0.705882}%
\pgfsetfillcolor{currentfill}%
\pgfsetfillopacity{0.918084}%
\pgfsetlinewidth{1.003750pt}%
\definecolor{currentstroke}{rgb}{0.121569,0.466667,0.705882}%
\pgfsetstrokecolor{currentstroke}%
\pgfsetstrokeopacity{0.918084}%
\pgfsetdash{}{0pt}%
\pgfpathmoveto{\pgfqpoint{1.164825in}{2.394842in}}%
\pgfpathcurveto{\pgfqpoint{1.173061in}{2.394842in}}{\pgfqpoint{1.180961in}{2.398114in}}{\pgfqpoint{1.186785in}{2.403938in}}%
\pgfpathcurveto{\pgfqpoint{1.192609in}{2.409762in}}{\pgfqpoint{1.195882in}{2.417662in}}{\pgfqpoint{1.195882in}{2.425898in}}%
\pgfpathcurveto{\pgfqpoint{1.195882in}{2.434135in}}{\pgfqpoint{1.192609in}{2.442035in}}{\pgfqpoint{1.186785in}{2.447859in}}%
\pgfpathcurveto{\pgfqpoint{1.180961in}{2.453682in}}{\pgfqpoint{1.173061in}{2.456955in}}{\pgfqpoint{1.164825in}{2.456955in}}%
\pgfpathcurveto{\pgfqpoint{1.156589in}{2.456955in}}{\pgfqpoint{1.148689in}{2.453682in}}{\pgfqpoint{1.142865in}{2.447859in}}%
\pgfpathcurveto{\pgfqpoint{1.137041in}{2.442035in}}{\pgfqpoint{1.133769in}{2.434135in}}{\pgfqpoint{1.133769in}{2.425898in}}%
\pgfpathcurveto{\pgfqpoint{1.133769in}{2.417662in}}{\pgfqpoint{1.137041in}{2.409762in}}{\pgfqpoint{1.142865in}{2.403938in}}%
\pgfpathcurveto{\pgfqpoint{1.148689in}{2.398114in}}{\pgfqpoint{1.156589in}{2.394842in}}{\pgfqpoint{1.164825in}{2.394842in}}%
\pgfpathclose%
\pgfusepath{stroke,fill}%
\end{pgfscope}%
\begin{pgfscope}%
\pgfpathrectangle{\pgfqpoint{0.100000in}{0.212622in}}{\pgfqpoint{3.696000in}{3.696000in}}%
\pgfusepath{clip}%
\pgfsetbuttcap%
\pgfsetroundjoin%
\definecolor{currentfill}{rgb}{0.121569,0.466667,0.705882}%
\pgfsetfillcolor{currentfill}%
\pgfsetfillopacity{0.918996}%
\pgfsetlinewidth{1.003750pt}%
\definecolor{currentstroke}{rgb}{0.121569,0.466667,0.705882}%
\pgfsetstrokecolor{currentstroke}%
\pgfsetstrokeopacity{0.918996}%
\pgfsetdash{}{0pt}%
\pgfpathmoveto{\pgfqpoint{1.496824in}{2.039583in}}%
\pgfpathcurveto{\pgfqpoint{1.505060in}{2.039583in}}{\pgfqpoint{1.512960in}{2.042855in}}{\pgfqpoint{1.518784in}{2.048679in}}%
\pgfpathcurveto{\pgfqpoint{1.524608in}{2.054503in}}{\pgfqpoint{1.527880in}{2.062403in}}{\pgfqpoint{1.527880in}{2.070639in}}%
\pgfpathcurveto{\pgfqpoint{1.527880in}{2.078876in}}{\pgfqpoint{1.524608in}{2.086776in}}{\pgfqpoint{1.518784in}{2.092600in}}%
\pgfpathcurveto{\pgfqpoint{1.512960in}{2.098424in}}{\pgfqpoint{1.505060in}{2.101696in}}{\pgfqpoint{1.496824in}{2.101696in}}%
\pgfpathcurveto{\pgfqpoint{1.488588in}{2.101696in}}{\pgfqpoint{1.480688in}{2.098424in}}{\pgfqpoint{1.474864in}{2.092600in}}%
\pgfpathcurveto{\pgfqpoint{1.469040in}{2.086776in}}{\pgfqpoint{1.465767in}{2.078876in}}{\pgfqpoint{1.465767in}{2.070639in}}%
\pgfpathcurveto{\pgfqpoint{1.465767in}{2.062403in}}{\pgfqpoint{1.469040in}{2.054503in}}{\pgfqpoint{1.474864in}{2.048679in}}%
\pgfpathcurveto{\pgfqpoint{1.480688in}{2.042855in}}{\pgfqpoint{1.488588in}{2.039583in}}{\pgfqpoint{1.496824in}{2.039583in}}%
\pgfpathclose%
\pgfusepath{stroke,fill}%
\end{pgfscope}%
\begin{pgfscope}%
\pgfpathrectangle{\pgfqpoint{0.100000in}{0.212622in}}{\pgfqpoint{3.696000in}{3.696000in}}%
\pgfusepath{clip}%
\pgfsetbuttcap%
\pgfsetroundjoin%
\definecolor{currentfill}{rgb}{0.121569,0.466667,0.705882}%
\pgfsetfillcolor{currentfill}%
\pgfsetfillopacity{0.921061}%
\pgfsetlinewidth{1.003750pt}%
\definecolor{currentstroke}{rgb}{0.121569,0.466667,0.705882}%
\pgfsetstrokecolor{currentstroke}%
\pgfsetstrokeopacity{0.921061}%
\pgfsetdash{}{0pt}%
\pgfpathmoveto{\pgfqpoint{1.514777in}{2.031496in}}%
\pgfpathcurveto{\pgfqpoint{1.523013in}{2.031496in}}{\pgfqpoint{1.530913in}{2.034768in}}{\pgfqpoint{1.536737in}{2.040592in}}%
\pgfpathcurveto{\pgfqpoint{1.542561in}{2.046416in}}{\pgfqpoint{1.545834in}{2.054316in}}{\pgfqpoint{1.545834in}{2.062552in}}%
\pgfpathcurveto{\pgfqpoint{1.545834in}{2.070789in}}{\pgfqpoint{1.542561in}{2.078689in}}{\pgfqpoint{1.536737in}{2.084513in}}%
\pgfpathcurveto{\pgfqpoint{1.530913in}{2.090336in}}{\pgfqpoint{1.523013in}{2.093609in}}{\pgfqpoint{1.514777in}{2.093609in}}%
\pgfpathcurveto{\pgfqpoint{1.506541in}{2.093609in}}{\pgfqpoint{1.498641in}{2.090336in}}{\pgfqpoint{1.492817in}{2.084513in}}%
\pgfpathcurveto{\pgfqpoint{1.486993in}{2.078689in}}{\pgfqpoint{1.483721in}{2.070789in}}{\pgfqpoint{1.483721in}{2.062552in}}%
\pgfpathcurveto{\pgfqpoint{1.483721in}{2.054316in}}{\pgfqpoint{1.486993in}{2.046416in}}{\pgfqpoint{1.492817in}{2.040592in}}%
\pgfpathcurveto{\pgfqpoint{1.498641in}{2.034768in}}{\pgfqpoint{1.506541in}{2.031496in}}{\pgfqpoint{1.514777in}{2.031496in}}%
\pgfpathclose%
\pgfusepath{stroke,fill}%
\end{pgfscope}%
\begin{pgfscope}%
\pgfpathrectangle{\pgfqpoint{0.100000in}{0.212622in}}{\pgfqpoint{3.696000in}{3.696000in}}%
\pgfusepath{clip}%
\pgfsetbuttcap%
\pgfsetroundjoin%
\definecolor{currentfill}{rgb}{0.121569,0.466667,0.705882}%
\pgfsetfillcolor{currentfill}%
\pgfsetfillopacity{0.922238}%
\pgfsetlinewidth{1.003750pt}%
\definecolor{currentstroke}{rgb}{0.121569,0.466667,0.705882}%
\pgfsetstrokecolor{currentstroke}%
\pgfsetstrokeopacity{0.922238}%
\pgfsetdash{}{0pt}%
\pgfpathmoveto{\pgfqpoint{1.187089in}{2.404236in}}%
\pgfpathcurveto{\pgfqpoint{1.195325in}{2.404236in}}{\pgfqpoint{1.203225in}{2.407508in}}{\pgfqpoint{1.209049in}{2.413332in}}%
\pgfpathcurveto{\pgfqpoint{1.214873in}{2.419156in}}{\pgfqpoint{1.218145in}{2.427056in}}{\pgfqpoint{1.218145in}{2.435293in}}%
\pgfpathcurveto{\pgfqpoint{1.218145in}{2.443529in}}{\pgfqpoint{1.214873in}{2.451429in}}{\pgfqpoint{1.209049in}{2.457253in}}%
\pgfpathcurveto{\pgfqpoint{1.203225in}{2.463077in}}{\pgfqpoint{1.195325in}{2.466349in}}{\pgfqpoint{1.187089in}{2.466349in}}%
\pgfpathcurveto{\pgfqpoint{1.178853in}{2.466349in}}{\pgfqpoint{1.170953in}{2.463077in}}{\pgfqpoint{1.165129in}{2.457253in}}%
\pgfpathcurveto{\pgfqpoint{1.159305in}{2.451429in}}{\pgfqpoint{1.156032in}{2.443529in}}{\pgfqpoint{1.156032in}{2.435293in}}%
\pgfpathcurveto{\pgfqpoint{1.156032in}{2.427056in}}{\pgfqpoint{1.159305in}{2.419156in}}{\pgfqpoint{1.165129in}{2.413332in}}%
\pgfpathcurveto{\pgfqpoint{1.170953in}{2.407508in}}{\pgfqpoint{1.178853in}{2.404236in}}{\pgfqpoint{1.187089in}{2.404236in}}%
\pgfpathclose%
\pgfusepath{stroke,fill}%
\end{pgfscope}%
\begin{pgfscope}%
\pgfpathrectangle{\pgfqpoint{0.100000in}{0.212622in}}{\pgfqpoint{3.696000in}{3.696000in}}%
\pgfusepath{clip}%
\pgfsetbuttcap%
\pgfsetroundjoin%
\definecolor{currentfill}{rgb}{0.121569,0.466667,0.705882}%
\pgfsetfillcolor{currentfill}%
\pgfsetfillopacity{0.922467}%
\pgfsetlinewidth{1.003750pt}%
\definecolor{currentstroke}{rgb}{0.121569,0.466667,0.705882}%
\pgfsetstrokecolor{currentstroke}%
\pgfsetstrokeopacity{0.922467}%
\pgfsetdash{}{0pt}%
\pgfpathmoveto{\pgfqpoint{1.526649in}{2.027599in}}%
\pgfpathcurveto{\pgfqpoint{1.534885in}{2.027599in}}{\pgfqpoint{1.542785in}{2.030872in}}{\pgfqpoint{1.548609in}{2.036695in}}%
\pgfpathcurveto{\pgfqpoint{1.554433in}{2.042519in}}{\pgfqpoint{1.557705in}{2.050419in}}{\pgfqpoint{1.557705in}{2.058656in}}%
\pgfpathcurveto{\pgfqpoint{1.557705in}{2.066892in}}{\pgfqpoint{1.554433in}{2.074792in}}{\pgfqpoint{1.548609in}{2.080616in}}%
\pgfpathcurveto{\pgfqpoint{1.542785in}{2.086440in}}{\pgfqpoint{1.534885in}{2.089712in}}{\pgfqpoint{1.526649in}{2.089712in}}%
\pgfpathcurveto{\pgfqpoint{1.518412in}{2.089712in}}{\pgfqpoint{1.510512in}{2.086440in}}{\pgfqpoint{1.504688in}{2.080616in}}%
\pgfpathcurveto{\pgfqpoint{1.498864in}{2.074792in}}{\pgfqpoint{1.495592in}{2.066892in}}{\pgfqpoint{1.495592in}{2.058656in}}%
\pgfpathcurveto{\pgfqpoint{1.495592in}{2.050419in}}{\pgfqpoint{1.498864in}{2.042519in}}{\pgfqpoint{1.504688in}{2.036695in}}%
\pgfpathcurveto{\pgfqpoint{1.510512in}{2.030872in}}{\pgfqpoint{1.518412in}{2.027599in}}{\pgfqpoint{1.526649in}{2.027599in}}%
\pgfpathclose%
\pgfusepath{stroke,fill}%
\end{pgfscope}%
\begin{pgfscope}%
\pgfpathrectangle{\pgfqpoint{0.100000in}{0.212622in}}{\pgfqpoint{3.696000in}{3.696000in}}%
\pgfusepath{clip}%
\pgfsetbuttcap%
\pgfsetroundjoin%
\definecolor{currentfill}{rgb}{0.121569,0.466667,0.705882}%
\pgfsetfillcolor{currentfill}%
\pgfsetfillopacity{0.923235}%
\pgfsetlinewidth{1.003750pt}%
\definecolor{currentstroke}{rgb}{0.121569,0.466667,0.705882}%
\pgfsetstrokecolor{currentstroke}%
\pgfsetstrokeopacity{0.923235}%
\pgfsetdash{}{0pt}%
\pgfpathmoveto{\pgfqpoint{1.533560in}{2.025144in}}%
\pgfpathcurveto{\pgfqpoint{1.541797in}{2.025144in}}{\pgfqpoint{1.549697in}{2.028417in}}{\pgfqpoint{1.555521in}{2.034241in}}%
\pgfpathcurveto{\pgfqpoint{1.561345in}{2.040065in}}{\pgfqpoint{1.564617in}{2.047965in}}{\pgfqpoint{1.564617in}{2.056201in}}%
\pgfpathcurveto{\pgfqpoint{1.564617in}{2.064437in}}{\pgfqpoint{1.561345in}{2.072337in}}{\pgfqpoint{1.555521in}{2.078161in}}%
\pgfpathcurveto{\pgfqpoint{1.549697in}{2.083985in}}{\pgfqpoint{1.541797in}{2.087257in}}{\pgfqpoint{1.533560in}{2.087257in}}%
\pgfpathcurveto{\pgfqpoint{1.525324in}{2.087257in}}{\pgfqpoint{1.517424in}{2.083985in}}{\pgfqpoint{1.511600in}{2.078161in}}%
\pgfpathcurveto{\pgfqpoint{1.505776in}{2.072337in}}{\pgfqpoint{1.502504in}{2.064437in}}{\pgfqpoint{1.502504in}{2.056201in}}%
\pgfpathcurveto{\pgfqpoint{1.502504in}{2.047965in}}{\pgfqpoint{1.505776in}{2.040065in}}{\pgfqpoint{1.511600in}{2.034241in}}%
\pgfpathcurveto{\pgfqpoint{1.517424in}{2.028417in}}{\pgfqpoint{1.525324in}{2.025144in}}{\pgfqpoint{1.533560in}{2.025144in}}%
\pgfpathclose%
\pgfusepath{stroke,fill}%
\end{pgfscope}%
\begin{pgfscope}%
\pgfpathrectangle{\pgfqpoint{0.100000in}{0.212622in}}{\pgfqpoint{3.696000in}{3.696000in}}%
\pgfusepath{clip}%
\pgfsetbuttcap%
\pgfsetroundjoin%
\definecolor{currentfill}{rgb}{0.121569,0.466667,0.705882}%
\pgfsetfillcolor{currentfill}%
\pgfsetfillopacity{0.923615}%
\pgfsetlinewidth{1.003750pt}%
\definecolor{currentstroke}{rgb}{0.121569,0.466667,0.705882}%
\pgfsetstrokecolor{currentstroke}%
\pgfsetstrokeopacity{0.923615}%
\pgfsetdash{}{0pt}%
\pgfpathmoveto{\pgfqpoint{1.536827in}{2.023607in}}%
\pgfpathcurveto{\pgfqpoint{1.545063in}{2.023607in}}{\pgfqpoint{1.552963in}{2.026879in}}{\pgfqpoint{1.558787in}{2.032703in}}%
\pgfpathcurveto{\pgfqpoint{1.564611in}{2.038527in}}{\pgfqpoint{1.567883in}{2.046427in}}{\pgfqpoint{1.567883in}{2.054663in}}%
\pgfpathcurveto{\pgfqpoint{1.567883in}{2.062900in}}{\pgfqpoint{1.564611in}{2.070800in}}{\pgfqpoint{1.558787in}{2.076624in}}%
\pgfpathcurveto{\pgfqpoint{1.552963in}{2.082448in}}{\pgfqpoint{1.545063in}{2.085720in}}{\pgfqpoint{1.536827in}{2.085720in}}%
\pgfpathcurveto{\pgfqpoint{1.528590in}{2.085720in}}{\pgfqpoint{1.520690in}{2.082448in}}{\pgfqpoint{1.514866in}{2.076624in}}%
\pgfpathcurveto{\pgfqpoint{1.509042in}{2.070800in}}{\pgfqpoint{1.505770in}{2.062900in}}{\pgfqpoint{1.505770in}{2.054663in}}%
\pgfpathcurveto{\pgfqpoint{1.505770in}{2.046427in}}{\pgfqpoint{1.509042in}{2.038527in}}{\pgfqpoint{1.514866in}{2.032703in}}%
\pgfpathcurveto{\pgfqpoint{1.520690in}{2.026879in}}{\pgfqpoint{1.528590in}{2.023607in}}{\pgfqpoint{1.536827in}{2.023607in}}%
\pgfpathclose%
\pgfusepath{stroke,fill}%
\end{pgfscope}%
\begin{pgfscope}%
\pgfpathrectangle{\pgfqpoint{0.100000in}{0.212622in}}{\pgfqpoint{3.696000in}{3.696000in}}%
\pgfusepath{clip}%
\pgfsetbuttcap%
\pgfsetroundjoin%
\definecolor{currentfill}{rgb}{0.121569,0.466667,0.705882}%
\pgfsetfillcolor{currentfill}%
\pgfsetfillopacity{0.924296}%
\pgfsetlinewidth{1.003750pt}%
\definecolor{currentstroke}{rgb}{0.121569,0.466667,0.705882}%
\pgfsetstrokecolor{currentstroke}%
\pgfsetstrokeopacity{0.924296}%
\pgfsetdash{}{0pt}%
\pgfpathmoveto{\pgfqpoint{1.542999in}{2.021623in}}%
\pgfpathcurveto{\pgfqpoint{1.551235in}{2.021623in}}{\pgfqpoint{1.559135in}{2.024896in}}{\pgfqpoint{1.564959in}{2.030720in}}%
\pgfpathcurveto{\pgfqpoint{1.570783in}{2.036544in}}{\pgfqpoint{1.574056in}{2.044444in}}{\pgfqpoint{1.574056in}{2.052680in}}%
\pgfpathcurveto{\pgfqpoint{1.574056in}{2.060916in}}{\pgfqpoint{1.570783in}{2.068816in}}{\pgfqpoint{1.564959in}{2.074640in}}%
\pgfpathcurveto{\pgfqpoint{1.559135in}{2.080464in}}{\pgfqpoint{1.551235in}{2.083736in}}{\pgfqpoint{1.542999in}{2.083736in}}%
\pgfpathcurveto{\pgfqpoint{1.534763in}{2.083736in}}{\pgfqpoint{1.526863in}{2.080464in}}{\pgfqpoint{1.521039in}{2.074640in}}%
\pgfpathcurveto{\pgfqpoint{1.515215in}{2.068816in}}{\pgfqpoint{1.511943in}{2.060916in}}{\pgfqpoint{1.511943in}{2.052680in}}%
\pgfpathcurveto{\pgfqpoint{1.511943in}{2.044444in}}{\pgfqpoint{1.515215in}{2.036544in}}{\pgfqpoint{1.521039in}{2.030720in}}%
\pgfpathcurveto{\pgfqpoint{1.526863in}{2.024896in}}{\pgfqpoint{1.534763in}{2.021623in}}{\pgfqpoint{1.542999in}{2.021623in}}%
\pgfpathclose%
\pgfusepath{stroke,fill}%
\end{pgfscope}%
\begin{pgfscope}%
\pgfpathrectangle{\pgfqpoint{0.100000in}{0.212622in}}{\pgfqpoint{3.696000in}{3.696000in}}%
\pgfusepath{clip}%
\pgfsetbuttcap%
\pgfsetroundjoin%
\definecolor{currentfill}{rgb}{0.121569,0.466667,0.705882}%
\pgfsetfillcolor{currentfill}%
\pgfsetfillopacity{0.925502}%
\pgfsetlinewidth{1.003750pt}%
\definecolor{currentstroke}{rgb}{0.121569,0.466667,0.705882}%
\pgfsetstrokecolor{currentstroke}%
\pgfsetstrokeopacity{0.925502}%
\pgfsetdash{}{0pt}%
\pgfpathmoveto{\pgfqpoint{1.553937in}{2.016638in}}%
\pgfpathcurveto{\pgfqpoint{1.562173in}{2.016638in}}{\pgfqpoint{1.570073in}{2.019910in}}{\pgfqpoint{1.575897in}{2.025734in}}%
\pgfpathcurveto{\pgfqpoint{1.581721in}{2.031558in}}{\pgfqpoint{1.584993in}{2.039458in}}{\pgfqpoint{1.584993in}{2.047694in}}%
\pgfpathcurveto{\pgfqpoint{1.584993in}{2.055931in}}{\pgfqpoint{1.581721in}{2.063831in}}{\pgfqpoint{1.575897in}{2.069655in}}%
\pgfpathcurveto{\pgfqpoint{1.570073in}{2.075478in}}{\pgfqpoint{1.562173in}{2.078751in}}{\pgfqpoint{1.553937in}{2.078751in}}%
\pgfpathcurveto{\pgfqpoint{1.545700in}{2.078751in}}{\pgfqpoint{1.537800in}{2.075478in}}{\pgfqpoint{1.531976in}{2.069655in}}%
\pgfpathcurveto{\pgfqpoint{1.526152in}{2.063831in}}{\pgfqpoint{1.522880in}{2.055931in}}{\pgfqpoint{1.522880in}{2.047694in}}%
\pgfpathcurveto{\pgfqpoint{1.522880in}{2.039458in}}{\pgfqpoint{1.526152in}{2.031558in}}{\pgfqpoint{1.531976in}{2.025734in}}%
\pgfpathcurveto{\pgfqpoint{1.537800in}{2.019910in}}{\pgfqpoint{1.545700in}{2.016638in}}{\pgfqpoint{1.553937in}{2.016638in}}%
\pgfpathclose%
\pgfusepath{stroke,fill}%
\end{pgfscope}%
\begin{pgfscope}%
\pgfpathrectangle{\pgfqpoint{0.100000in}{0.212622in}}{\pgfqpoint{3.696000in}{3.696000in}}%
\pgfusepath{clip}%
\pgfsetbuttcap%
\pgfsetroundjoin%
\definecolor{currentfill}{rgb}{0.121569,0.466667,0.705882}%
\pgfsetfillcolor{currentfill}%
\pgfsetfillopacity{0.928173}%
\pgfsetlinewidth{1.003750pt}%
\definecolor{currentstroke}{rgb}{0.121569,0.466667,0.705882}%
\pgfsetstrokecolor{currentstroke}%
\pgfsetstrokeopacity{0.928173}%
\pgfsetdash{}{0pt}%
\pgfpathmoveto{\pgfqpoint{1.574046in}{2.011546in}}%
\pgfpathcurveto{\pgfqpoint{1.582283in}{2.011546in}}{\pgfqpoint{1.590183in}{2.014818in}}{\pgfqpoint{1.596007in}{2.020642in}}%
\pgfpathcurveto{\pgfqpoint{1.601830in}{2.026466in}}{\pgfqpoint{1.605103in}{2.034366in}}{\pgfqpoint{1.605103in}{2.042602in}}%
\pgfpathcurveto{\pgfqpoint{1.605103in}{2.050839in}}{\pgfqpoint{1.601830in}{2.058739in}}{\pgfqpoint{1.596007in}{2.064563in}}%
\pgfpathcurveto{\pgfqpoint{1.590183in}{2.070387in}}{\pgfqpoint{1.582283in}{2.073659in}}{\pgfqpoint{1.574046in}{2.073659in}}%
\pgfpathcurveto{\pgfqpoint{1.565810in}{2.073659in}}{\pgfqpoint{1.557910in}{2.070387in}}{\pgfqpoint{1.552086in}{2.064563in}}%
\pgfpathcurveto{\pgfqpoint{1.546262in}{2.058739in}}{\pgfqpoint{1.542990in}{2.050839in}}{\pgfqpoint{1.542990in}{2.042602in}}%
\pgfpathcurveto{\pgfqpoint{1.542990in}{2.034366in}}{\pgfqpoint{1.546262in}{2.026466in}}{\pgfqpoint{1.552086in}{2.020642in}}%
\pgfpathcurveto{\pgfqpoint{1.557910in}{2.014818in}}{\pgfqpoint{1.565810in}{2.011546in}}{\pgfqpoint{1.574046in}{2.011546in}}%
\pgfpathclose%
\pgfusepath{stroke,fill}%
\end{pgfscope}%
\begin{pgfscope}%
\pgfpathrectangle{\pgfqpoint{0.100000in}{0.212622in}}{\pgfqpoint{3.696000in}{3.696000in}}%
\pgfusepath{clip}%
\pgfsetbuttcap%
\pgfsetroundjoin%
\definecolor{currentfill}{rgb}{0.121569,0.466667,0.705882}%
\pgfsetfillcolor{currentfill}%
\pgfsetfillopacity{0.928748}%
\pgfsetlinewidth{1.003750pt}%
\definecolor{currentstroke}{rgb}{0.121569,0.466667,0.705882}%
\pgfsetstrokecolor{currentstroke}%
\pgfsetstrokeopacity{0.928748}%
\pgfsetdash{}{0pt}%
\pgfpathmoveto{\pgfqpoint{2.430161in}{1.709790in}}%
\pgfpathcurveto{\pgfqpoint{2.438397in}{1.709790in}}{\pgfqpoint{2.446297in}{1.713063in}}{\pgfqpoint{2.452121in}{1.718887in}}%
\pgfpathcurveto{\pgfqpoint{2.457945in}{1.724711in}}{\pgfqpoint{2.461217in}{1.732611in}}{\pgfqpoint{2.461217in}{1.740847in}}%
\pgfpathcurveto{\pgfqpoint{2.461217in}{1.749083in}}{\pgfqpoint{2.457945in}{1.756983in}}{\pgfqpoint{2.452121in}{1.762807in}}%
\pgfpathcurveto{\pgfqpoint{2.446297in}{1.768631in}}{\pgfqpoint{2.438397in}{1.771903in}}{\pgfqpoint{2.430161in}{1.771903in}}%
\pgfpathcurveto{\pgfqpoint{2.421924in}{1.771903in}}{\pgfqpoint{2.414024in}{1.768631in}}{\pgfqpoint{2.408200in}{1.762807in}}%
\pgfpathcurveto{\pgfqpoint{2.402376in}{1.756983in}}{\pgfqpoint{2.399104in}{1.749083in}}{\pgfqpoint{2.399104in}{1.740847in}}%
\pgfpathcurveto{\pgfqpoint{2.399104in}{1.732611in}}{\pgfqpoint{2.402376in}{1.724711in}}{\pgfqpoint{2.408200in}{1.718887in}}%
\pgfpathcurveto{\pgfqpoint{2.414024in}{1.713063in}}{\pgfqpoint{2.421924in}{1.709790in}}{\pgfqpoint{2.430161in}{1.709790in}}%
\pgfpathclose%
\pgfusepath{stroke,fill}%
\end{pgfscope}%
\begin{pgfscope}%
\pgfpathrectangle{\pgfqpoint{0.100000in}{0.212622in}}{\pgfqpoint{3.696000in}{3.696000in}}%
\pgfusepath{clip}%
\pgfsetbuttcap%
\pgfsetroundjoin%
\definecolor{currentfill}{rgb}{0.121569,0.466667,0.705882}%
\pgfsetfillcolor{currentfill}%
\pgfsetfillopacity{0.929900}%
\pgfsetlinewidth{1.003750pt}%
\definecolor{currentstroke}{rgb}{0.121569,0.466667,0.705882}%
\pgfsetstrokecolor{currentstroke}%
\pgfsetstrokeopacity{0.929900}%
\pgfsetdash{}{0pt}%
\pgfpathmoveto{\pgfqpoint{1.589035in}{2.002700in}}%
\pgfpathcurveto{\pgfqpoint{1.597271in}{2.002700in}}{\pgfqpoint{1.605172in}{2.005972in}}{\pgfqpoint{1.610995in}{2.011796in}}%
\pgfpathcurveto{\pgfqpoint{1.616819in}{2.017620in}}{\pgfqpoint{1.620092in}{2.025520in}}{\pgfqpoint{1.620092in}{2.033756in}}%
\pgfpathcurveto{\pgfqpoint{1.620092in}{2.041993in}}{\pgfqpoint{1.616819in}{2.049893in}}{\pgfqpoint{1.610995in}{2.055717in}}%
\pgfpathcurveto{\pgfqpoint{1.605172in}{2.061541in}}{\pgfqpoint{1.597271in}{2.064813in}}{\pgfqpoint{1.589035in}{2.064813in}}%
\pgfpathcurveto{\pgfqpoint{1.580799in}{2.064813in}}{\pgfqpoint{1.572899in}{2.061541in}}{\pgfqpoint{1.567075in}{2.055717in}}%
\pgfpathcurveto{\pgfqpoint{1.561251in}{2.049893in}}{\pgfqpoint{1.557979in}{2.041993in}}{\pgfqpoint{1.557979in}{2.033756in}}%
\pgfpathcurveto{\pgfqpoint{1.557979in}{2.025520in}}{\pgfqpoint{1.561251in}{2.017620in}}{\pgfqpoint{1.567075in}{2.011796in}}%
\pgfpathcurveto{\pgfqpoint{1.572899in}{2.005972in}}{\pgfqpoint{1.580799in}{2.002700in}}{\pgfqpoint{1.589035in}{2.002700in}}%
\pgfpathclose%
\pgfusepath{stroke,fill}%
\end{pgfscope}%
\begin{pgfscope}%
\pgfpathrectangle{\pgfqpoint{0.100000in}{0.212622in}}{\pgfqpoint{3.696000in}{3.696000in}}%
\pgfusepath{clip}%
\pgfsetbuttcap%
\pgfsetroundjoin%
\definecolor{currentfill}{rgb}{0.121569,0.466667,0.705882}%
\pgfsetfillcolor{currentfill}%
\pgfsetfillopacity{0.931275}%
\pgfsetlinewidth{1.003750pt}%
\definecolor{currentstroke}{rgb}{0.121569,0.466667,0.705882}%
\pgfsetstrokecolor{currentstroke}%
\pgfsetstrokeopacity{0.931275}%
\pgfsetdash{}{0pt}%
\pgfpathmoveto{\pgfqpoint{1.600145in}{1.999840in}}%
\pgfpathcurveto{\pgfqpoint{1.608381in}{1.999840in}}{\pgfqpoint{1.616281in}{2.003112in}}{\pgfqpoint{1.622105in}{2.008936in}}%
\pgfpathcurveto{\pgfqpoint{1.627929in}{2.014760in}}{\pgfqpoint{1.631201in}{2.022660in}}{\pgfqpoint{1.631201in}{2.030897in}}%
\pgfpathcurveto{\pgfqpoint{1.631201in}{2.039133in}}{\pgfqpoint{1.627929in}{2.047033in}}{\pgfqpoint{1.622105in}{2.052857in}}%
\pgfpathcurveto{\pgfqpoint{1.616281in}{2.058681in}}{\pgfqpoint{1.608381in}{2.061953in}}{\pgfqpoint{1.600145in}{2.061953in}}%
\pgfpathcurveto{\pgfqpoint{1.591909in}{2.061953in}}{\pgfqpoint{1.584009in}{2.058681in}}{\pgfqpoint{1.578185in}{2.052857in}}%
\pgfpathcurveto{\pgfqpoint{1.572361in}{2.047033in}}{\pgfqpoint{1.569088in}{2.039133in}}{\pgfqpoint{1.569088in}{2.030897in}}%
\pgfpathcurveto{\pgfqpoint{1.569088in}{2.022660in}}{\pgfqpoint{1.572361in}{2.014760in}}{\pgfqpoint{1.578185in}{2.008936in}}%
\pgfpathcurveto{\pgfqpoint{1.584009in}{2.003112in}}{\pgfqpoint{1.591909in}{1.999840in}}{\pgfqpoint{1.600145in}{1.999840in}}%
\pgfpathclose%
\pgfusepath{stroke,fill}%
\end{pgfscope}%
\begin{pgfscope}%
\pgfpathrectangle{\pgfqpoint{0.100000in}{0.212622in}}{\pgfqpoint{3.696000in}{3.696000in}}%
\pgfusepath{clip}%
\pgfsetbuttcap%
\pgfsetroundjoin%
\definecolor{currentfill}{rgb}{0.121569,0.466667,0.705882}%
\pgfsetfillcolor{currentfill}%
\pgfsetfillopacity{0.931965}%
\pgfsetlinewidth{1.003750pt}%
\definecolor{currentstroke}{rgb}{0.121569,0.466667,0.705882}%
\pgfsetstrokecolor{currentstroke}%
\pgfsetstrokeopacity{0.931965}%
\pgfsetdash{}{0pt}%
\pgfpathmoveto{\pgfqpoint{1.606646in}{1.997065in}}%
\pgfpathcurveto{\pgfqpoint{1.614882in}{1.997065in}}{\pgfqpoint{1.622782in}{2.000337in}}{\pgfqpoint{1.628606in}{2.006161in}}%
\pgfpathcurveto{\pgfqpoint{1.634430in}{2.011985in}}{\pgfqpoint{1.637703in}{2.019885in}}{\pgfqpoint{1.637703in}{2.028121in}}%
\pgfpathcurveto{\pgfqpoint{1.637703in}{2.036358in}}{\pgfqpoint{1.634430in}{2.044258in}}{\pgfqpoint{1.628606in}{2.050082in}}%
\pgfpathcurveto{\pgfqpoint{1.622782in}{2.055906in}}{\pgfqpoint{1.614882in}{2.059178in}}{\pgfqpoint{1.606646in}{2.059178in}}%
\pgfpathcurveto{\pgfqpoint{1.598410in}{2.059178in}}{\pgfqpoint{1.590510in}{2.055906in}}{\pgfqpoint{1.584686in}{2.050082in}}%
\pgfpathcurveto{\pgfqpoint{1.578862in}{2.044258in}}{\pgfqpoint{1.575590in}{2.036358in}}{\pgfqpoint{1.575590in}{2.028121in}}%
\pgfpathcurveto{\pgfqpoint{1.575590in}{2.019885in}}{\pgfqpoint{1.578862in}{2.011985in}}{\pgfqpoint{1.584686in}{2.006161in}}%
\pgfpathcurveto{\pgfqpoint{1.590510in}{2.000337in}}{\pgfqpoint{1.598410in}{1.997065in}}{\pgfqpoint{1.606646in}{1.997065in}}%
\pgfpathclose%
\pgfusepath{stroke,fill}%
\end{pgfscope}%
\begin{pgfscope}%
\pgfpathrectangle{\pgfqpoint{0.100000in}{0.212622in}}{\pgfqpoint{3.696000in}{3.696000in}}%
\pgfusepath{clip}%
\pgfsetbuttcap%
\pgfsetroundjoin%
\definecolor{currentfill}{rgb}{0.121569,0.466667,0.705882}%
\pgfsetfillcolor{currentfill}%
\pgfsetfillopacity{0.932490}%
\pgfsetlinewidth{1.003750pt}%
\definecolor{currentstroke}{rgb}{0.121569,0.466667,0.705882}%
\pgfsetstrokecolor{currentstroke}%
\pgfsetstrokeopacity{0.932490}%
\pgfsetdash{}{0pt}%
\pgfpathmoveto{\pgfqpoint{1.610684in}{1.995811in}}%
\pgfpathcurveto{\pgfqpoint{1.618921in}{1.995811in}}{\pgfqpoint{1.626821in}{1.999084in}}{\pgfqpoint{1.632645in}{2.004908in}}%
\pgfpathcurveto{\pgfqpoint{1.638469in}{2.010731in}}{\pgfqpoint{1.641741in}{2.018632in}}{\pgfqpoint{1.641741in}{2.026868in}}%
\pgfpathcurveto{\pgfqpoint{1.641741in}{2.035104in}}{\pgfqpoint{1.638469in}{2.043004in}}{\pgfqpoint{1.632645in}{2.048828in}}%
\pgfpathcurveto{\pgfqpoint{1.626821in}{2.054652in}}{\pgfqpoint{1.618921in}{2.057924in}}{\pgfqpoint{1.610684in}{2.057924in}}%
\pgfpathcurveto{\pgfqpoint{1.602448in}{2.057924in}}{\pgfqpoint{1.594548in}{2.054652in}}{\pgfqpoint{1.588724in}{2.048828in}}%
\pgfpathcurveto{\pgfqpoint{1.582900in}{2.043004in}}{\pgfqpoint{1.579628in}{2.035104in}}{\pgfqpoint{1.579628in}{2.026868in}}%
\pgfpathcurveto{\pgfqpoint{1.579628in}{2.018632in}}{\pgfqpoint{1.582900in}{2.010731in}}{\pgfqpoint{1.588724in}{2.004908in}}%
\pgfpathcurveto{\pgfqpoint{1.594548in}{1.999084in}}{\pgfqpoint{1.602448in}{1.995811in}}{\pgfqpoint{1.610684in}{1.995811in}}%
\pgfpathclose%
\pgfusepath{stroke,fill}%
\end{pgfscope}%
\begin{pgfscope}%
\pgfpathrectangle{\pgfqpoint{0.100000in}{0.212622in}}{\pgfqpoint{3.696000in}{3.696000in}}%
\pgfusepath{clip}%
\pgfsetbuttcap%
\pgfsetroundjoin%
\definecolor{currentfill}{rgb}{0.121569,0.466667,0.705882}%
\pgfsetfillcolor{currentfill}%
\pgfsetfillopacity{0.932637}%
\pgfsetlinewidth{1.003750pt}%
\definecolor{currentstroke}{rgb}{0.121569,0.466667,0.705882}%
\pgfsetstrokecolor{currentstroke}%
\pgfsetstrokeopacity{0.932637}%
\pgfsetdash{}{0pt}%
\pgfpathmoveto{\pgfqpoint{1.611948in}{1.995267in}}%
\pgfpathcurveto{\pgfqpoint{1.620184in}{1.995267in}}{\pgfqpoint{1.628084in}{1.998539in}}{\pgfqpoint{1.633908in}{2.004363in}}%
\pgfpathcurveto{\pgfqpoint{1.639732in}{2.010187in}}{\pgfqpoint{1.643004in}{2.018087in}}{\pgfqpoint{1.643004in}{2.026323in}}%
\pgfpathcurveto{\pgfqpoint{1.643004in}{2.034559in}}{\pgfqpoint{1.639732in}{2.042459in}}{\pgfqpoint{1.633908in}{2.048283in}}%
\pgfpathcurveto{\pgfqpoint{1.628084in}{2.054107in}}{\pgfqpoint{1.620184in}{2.057380in}}{\pgfqpoint{1.611948in}{2.057380in}}%
\pgfpathcurveto{\pgfqpoint{1.603711in}{2.057380in}}{\pgfqpoint{1.595811in}{2.054107in}}{\pgfqpoint{1.589987in}{2.048283in}}%
\pgfpathcurveto{\pgfqpoint{1.584163in}{2.042459in}}{\pgfqpoint{1.580891in}{2.034559in}}{\pgfqpoint{1.580891in}{2.026323in}}%
\pgfpathcurveto{\pgfqpoint{1.580891in}{2.018087in}}{\pgfqpoint{1.584163in}{2.010187in}}{\pgfqpoint{1.589987in}{2.004363in}}%
\pgfpathcurveto{\pgfqpoint{1.595811in}{1.998539in}}{\pgfqpoint{1.603711in}{1.995267in}}{\pgfqpoint{1.611948in}{1.995267in}}%
\pgfpathclose%
\pgfusepath{stroke,fill}%
\end{pgfscope}%
\begin{pgfscope}%
\pgfpathrectangle{\pgfqpoint{0.100000in}{0.212622in}}{\pgfqpoint{3.696000in}{3.696000in}}%
\pgfusepath{clip}%
\pgfsetbuttcap%
\pgfsetroundjoin%
\definecolor{currentfill}{rgb}{0.121569,0.466667,0.705882}%
\pgfsetfillcolor{currentfill}%
\pgfsetfillopacity{0.932923}%
\pgfsetlinewidth{1.003750pt}%
\definecolor{currentstroke}{rgb}{0.121569,0.466667,0.705882}%
\pgfsetstrokecolor{currentstroke}%
\pgfsetstrokeopacity{0.932923}%
\pgfsetdash{}{0pt}%
\pgfpathmoveto{\pgfqpoint{1.614251in}{1.994429in}}%
\pgfpathcurveto{\pgfqpoint{1.622487in}{1.994429in}}{\pgfqpoint{1.630387in}{1.997701in}}{\pgfqpoint{1.636211in}{2.003525in}}%
\pgfpathcurveto{\pgfqpoint{1.642035in}{2.009349in}}{\pgfqpoint{1.645307in}{2.017249in}}{\pgfqpoint{1.645307in}{2.025485in}}%
\pgfpathcurveto{\pgfqpoint{1.645307in}{2.033722in}}{\pgfqpoint{1.642035in}{2.041622in}}{\pgfqpoint{1.636211in}{2.047446in}}%
\pgfpathcurveto{\pgfqpoint{1.630387in}{2.053270in}}{\pgfqpoint{1.622487in}{2.056542in}}{\pgfqpoint{1.614251in}{2.056542in}}%
\pgfpathcurveto{\pgfqpoint{1.606014in}{2.056542in}}{\pgfqpoint{1.598114in}{2.053270in}}{\pgfqpoint{1.592290in}{2.047446in}}%
\pgfpathcurveto{\pgfqpoint{1.586466in}{2.041622in}}{\pgfqpoint{1.583194in}{2.033722in}}{\pgfqpoint{1.583194in}{2.025485in}}%
\pgfpathcurveto{\pgfqpoint{1.583194in}{2.017249in}}{\pgfqpoint{1.586466in}{2.009349in}}{\pgfqpoint{1.592290in}{2.003525in}}%
\pgfpathcurveto{\pgfqpoint{1.598114in}{1.997701in}}{\pgfqpoint{1.606014in}{1.994429in}}{\pgfqpoint{1.614251in}{1.994429in}}%
\pgfpathclose%
\pgfusepath{stroke,fill}%
\end{pgfscope}%
\begin{pgfscope}%
\pgfpathrectangle{\pgfqpoint{0.100000in}{0.212622in}}{\pgfqpoint{3.696000in}{3.696000in}}%
\pgfusepath{clip}%
\pgfsetbuttcap%
\pgfsetroundjoin%
\definecolor{currentfill}{rgb}{0.121569,0.466667,0.705882}%
\pgfsetfillcolor{currentfill}%
\pgfsetfillopacity{0.933416}%
\pgfsetlinewidth{1.003750pt}%
\definecolor{currentstroke}{rgb}{0.121569,0.466667,0.705882}%
\pgfsetstrokecolor{currentstroke}%
\pgfsetstrokeopacity{0.933416}%
\pgfsetdash{}{0pt}%
\pgfpathmoveto{\pgfqpoint{1.618447in}{1.992751in}}%
\pgfpathcurveto{\pgfqpoint{1.626683in}{1.992751in}}{\pgfqpoint{1.634583in}{1.996024in}}{\pgfqpoint{1.640407in}{2.001847in}}%
\pgfpathcurveto{\pgfqpoint{1.646231in}{2.007671in}}{\pgfqpoint{1.649503in}{2.015571in}}{\pgfqpoint{1.649503in}{2.023808in}}%
\pgfpathcurveto{\pgfqpoint{1.649503in}{2.032044in}}{\pgfqpoint{1.646231in}{2.039944in}}{\pgfqpoint{1.640407in}{2.045768in}}%
\pgfpathcurveto{\pgfqpoint{1.634583in}{2.051592in}}{\pgfqpoint{1.626683in}{2.054864in}}{\pgfqpoint{1.618447in}{2.054864in}}%
\pgfpathcurveto{\pgfqpoint{1.610211in}{2.054864in}}{\pgfqpoint{1.602311in}{2.051592in}}{\pgfqpoint{1.596487in}{2.045768in}}%
\pgfpathcurveto{\pgfqpoint{1.590663in}{2.039944in}}{\pgfqpoint{1.587390in}{2.032044in}}{\pgfqpoint{1.587390in}{2.023808in}}%
\pgfpathcurveto{\pgfqpoint{1.587390in}{2.015571in}}{\pgfqpoint{1.590663in}{2.007671in}}{\pgfqpoint{1.596487in}{2.001847in}}%
\pgfpathcurveto{\pgfqpoint{1.602311in}{1.996024in}}{\pgfqpoint{1.610211in}{1.992751in}}{\pgfqpoint{1.618447in}{1.992751in}}%
\pgfpathclose%
\pgfusepath{stroke,fill}%
\end{pgfscope}%
\begin{pgfscope}%
\pgfpathrectangle{\pgfqpoint{0.100000in}{0.212622in}}{\pgfqpoint{3.696000in}{3.696000in}}%
\pgfusepath{clip}%
\pgfsetbuttcap%
\pgfsetroundjoin%
\definecolor{currentfill}{rgb}{0.121569,0.466667,0.705882}%
\pgfsetfillcolor{currentfill}%
\pgfsetfillopacity{0.934355}%
\pgfsetlinewidth{1.003750pt}%
\definecolor{currentstroke}{rgb}{0.121569,0.466667,0.705882}%
\pgfsetstrokecolor{currentstroke}%
\pgfsetstrokeopacity{0.934355}%
\pgfsetdash{}{0pt}%
\pgfpathmoveto{\pgfqpoint{1.626146in}{1.990232in}}%
\pgfpathcurveto{\pgfqpoint{1.634383in}{1.990232in}}{\pgfqpoint{1.642283in}{1.993504in}}{\pgfqpoint{1.648107in}{1.999328in}}%
\pgfpathcurveto{\pgfqpoint{1.653931in}{2.005152in}}{\pgfqpoint{1.657203in}{2.013052in}}{\pgfqpoint{1.657203in}{2.021289in}}%
\pgfpathcurveto{\pgfqpoint{1.657203in}{2.029525in}}{\pgfqpoint{1.653931in}{2.037425in}}{\pgfqpoint{1.648107in}{2.043249in}}%
\pgfpathcurveto{\pgfqpoint{1.642283in}{2.049073in}}{\pgfqpoint{1.634383in}{2.052345in}}{\pgfqpoint{1.626146in}{2.052345in}}%
\pgfpathcurveto{\pgfqpoint{1.617910in}{2.052345in}}{\pgfqpoint{1.610010in}{2.049073in}}{\pgfqpoint{1.604186in}{2.043249in}}%
\pgfpathcurveto{\pgfqpoint{1.598362in}{2.037425in}}{\pgfqpoint{1.595090in}{2.029525in}}{\pgfqpoint{1.595090in}{2.021289in}}%
\pgfpathcurveto{\pgfqpoint{1.595090in}{2.013052in}}{\pgfqpoint{1.598362in}{2.005152in}}{\pgfqpoint{1.604186in}{1.999328in}}%
\pgfpathcurveto{\pgfqpoint{1.610010in}{1.993504in}}{\pgfqpoint{1.617910in}{1.990232in}}{\pgfqpoint{1.626146in}{1.990232in}}%
\pgfpathclose%
\pgfusepath{stroke,fill}%
\end{pgfscope}%
\begin{pgfscope}%
\pgfpathrectangle{\pgfqpoint{0.100000in}{0.212622in}}{\pgfqpoint{3.696000in}{3.696000in}}%
\pgfusepath{clip}%
\pgfsetbuttcap%
\pgfsetroundjoin%
\definecolor{currentfill}{rgb}{0.121569,0.466667,0.705882}%
\pgfsetfillcolor{currentfill}%
\pgfsetfillopacity{0.935923}%
\pgfsetlinewidth{1.003750pt}%
\definecolor{currentstroke}{rgb}{0.121569,0.466667,0.705882}%
\pgfsetstrokecolor{currentstroke}%
\pgfsetstrokeopacity{0.935923}%
\pgfsetdash{}{0pt}%
\pgfpathmoveto{\pgfqpoint{1.640070in}{1.984409in}}%
\pgfpathcurveto{\pgfqpoint{1.648307in}{1.984409in}}{\pgfqpoint{1.656207in}{1.987681in}}{\pgfqpoint{1.662031in}{1.993505in}}%
\pgfpathcurveto{\pgfqpoint{1.667854in}{1.999329in}}{\pgfqpoint{1.671127in}{2.007229in}}{\pgfqpoint{1.671127in}{2.015465in}}%
\pgfpathcurveto{\pgfqpoint{1.671127in}{2.023701in}}{\pgfqpoint{1.667854in}{2.031601in}}{\pgfqpoint{1.662031in}{2.037425in}}%
\pgfpathcurveto{\pgfqpoint{1.656207in}{2.043249in}}{\pgfqpoint{1.648307in}{2.046522in}}{\pgfqpoint{1.640070in}{2.046522in}}%
\pgfpathcurveto{\pgfqpoint{1.631834in}{2.046522in}}{\pgfqpoint{1.623934in}{2.043249in}}{\pgfqpoint{1.618110in}{2.037425in}}%
\pgfpathcurveto{\pgfqpoint{1.612286in}{2.031601in}}{\pgfqpoint{1.609014in}{2.023701in}}{\pgfqpoint{1.609014in}{2.015465in}}%
\pgfpathcurveto{\pgfqpoint{1.609014in}{2.007229in}}{\pgfqpoint{1.612286in}{1.999329in}}{\pgfqpoint{1.618110in}{1.993505in}}%
\pgfpathcurveto{\pgfqpoint{1.623934in}{1.987681in}}{\pgfqpoint{1.631834in}{1.984409in}}{\pgfqpoint{1.640070in}{1.984409in}}%
\pgfpathclose%
\pgfusepath{stroke,fill}%
\end{pgfscope}%
\begin{pgfscope}%
\pgfpathrectangle{\pgfqpoint{0.100000in}{0.212622in}}{\pgfqpoint{3.696000in}{3.696000in}}%
\pgfusepath{clip}%
\pgfsetbuttcap%
\pgfsetroundjoin%
\definecolor{currentfill}{rgb}{0.121569,0.466667,0.705882}%
\pgfsetfillcolor{currentfill}%
\pgfsetfillopacity{0.938856}%
\pgfsetlinewidth{1.003750pt}%
\definecolor{currentstroke}{rgb}{0.121569,0.466667,0.705882}%
\pgfsetstrokecolor{currentstroke}%
\pgfsetstrokeopacity{0.938856}%
\pgfsetdash{}{0pt}%
\pgfpathmoveto{\pgfqpoint{1.665620in}{1.975152in}}%
\pgfpathcurveto{\pgfqpoint{1.673856in}{1.975152in}}{\pgfqpoint{1.681756in}{1.978424in}}{\pgfqpoint{1.687580in}{1.984248in}}%
\pgfpathcurveto{\pgfqpoint{1.693404in}{1.990072in}}{\pgfqpoint{1.696676in}{1.997972in}}{\pgfqpoint{1.696676in}{2.006208in}}%
\pgfpathcurveto{\pgfqpoint{1.696676in}{2.014444in}}{\pgfqpoint{1.693404in}{2.022345in}}{\pgfqpoint{1.687580in}{2.028168in}}%
\pgfpathcurveto{\pgfqpoint{1.681756in}{2.033992in}}{\pgfqpoint{1.673856in}{2.037265in}}{\pgfqpoint{1.665620in}{2.037265in}}%
\pgfpathcurveto{\pgfqpoint{1.657383in}{2.037265in}}{\pgfqpoint{1.649483in}{2.033992in}}{\pgfqpoint{1.643660in}{2.028168in}}%
\pgfpathcurveto{\pgfqpoint{1.637836in}{2.022345in}}{\pgfqpoint{1.634563in}{2.014444in}}{\pgfqpoint{1.634563in}{2.006208in}}%
\pgfpathcurveto{\pgfqpoint{1.634563in}{1.997972in}}{\pgfqpoint{1.637836in}{1.990072in}}{\pgfqpoint{1.643660in}{1.984248in}}%
\pgfpathcurveto{\pgfqpoint{1.649483in}{1.978424in}}{\pgfqpoint{1.657383in}{1.975152in}}{\pgfqpoint{1.665620in}{1.975152in}}%
\pgfpathclose%
\pgfusepath{stroke,fill}%
\end{pgfscope}%
\begin{pgfscope}%
\pgfpathrectangle{\pgfqpoint{0.100000in}{0.212622in}}{\pgfqpoint{3.696000in}{3.696000in}}%
\pgfusepath{clip}%
\pgfsetbuttcap%
\pgfsetroundjoin%
\definecolor{currentfill}{rgb}{0.121569,0.466667,0.705882}%
\pgfsetfillcolor{currentfill}%
\pgfsetfillopacity{0.944280}%
\pgfsetlinewidth{1.003750pt}%
\definecolor{currentstroke}{rgb}{0.121569,0.466667,0.705882}%
\pgfsetstrokecolor{currentstroke}%
\pgfsetstrokeopacity{0.944280}%
\pgfsetdash{}{0pt}%
\pgfpathmoveto{\pgfqpoint{1.711428in}{1.956212in}}%
\pgfpathcurveto{\pgfqpoint{1.719664in}{1.956212in}}{\pgfqpoint{1.727564in}{1.959484in}}{\pgfqpoint{1.733388in}{1.965308in}}%
\pgfpathcurveto{\pgfqpoint{1.739212in}{1.971132in}}{\pgfqpoint{1.742484in}{1.979032in}}{\pgfqpoint{1.742484in}{1.987269in}}%
\pgfpathcurveto{\pgfqpoint{1.742484in}{1.995505in}}{\pgfqpoint{1.739212in}{2.003405in}}{\pgfqpoint{1.733388in}{2.009229in}}%
\pgfpathcurveto{\pgfqpoint{1.727564in}{2.015053in}}{\pgfqpoint{1.719664in}{2.018325in}}{\pgfqpoint{1.711428in}{2.018325in}}%
\pgfpathcurveto{\pgfqpoint{1.703192in}{2.018325in}}{\pgfqpoint{1.695292in}{2.015053in}}{\pgfqpoint{1.689468in}{2.009229in}}%
\pgfpathcurveto{\pgfqpoint{1.683644in}{2.003405in}}{\pgfqpoint{1.680371in}{1.995505in}}{\pgfqpoint{1.680371in}{1.987269in}}%
\pgfpathcurveto{\pgfqpoint{1.680371in}{1.979032in}}{\pgfqpoint{1.683644in}{1.971132in}}{\pgfqpoint{1.689468in}{1.965308in}}%
\pgfpathcurveto{\pgfqpoint{1.695292in}{1.959484in}}{\pgfqpoint{1.703192in}{1.956212in}}{\pgfqpoint{1.711428in}{1.956212in}}%
\pgfpathclose%
\pgfusepath{stroke,fill}%
\end{pgfscope}%
\begin{pgfscope}%
\pgfpathrectangle{\pgfqpoint{0.100000in}{0.212622in}}{\pgfqpoint{3.696000in}{3.696000in}}%
\pgfusepath{clip}%
\pgfsetbuttcap%
\pgfsetroundjoin%
\definecolor{currentfill}{rgb}{0.121569,0.466667,0.705882}%
\pgfsetfillcolor{currentfill}%
\pgfsetfillopacity{0.947391}%
\pgfsetlinewidth{1.003750pt}%
\definecolor{currentstroke}{rgb}{0.121569,0.466667,0.705882}%
\pgfsetstrokecolor{currentstroke}%
\pgfsetstrokeopacity{0.947391}%
\pgfsetdash{}{0pt}%
\pgfpathmoveto{\pgfqpoint{2.456401in}{1.697083in}}%
\pgfpathcurveto{\pgfqpoint{2.464638in}{1.697083in}}{\pgfqpoint{2.472538in}{1.700355in}}{\pgfqpoint{2.478362in}{1.706179in}}%
\pgfpathcurveto{\pgfqpoint{2.484186in}{1.712003in}}{\pgfqpoint{2.487458in}{1.719903in}}{\pgfqpoint{2.487458in}{1.728139in}}%
\pgfpathcurveto{\pgfqpoint{2.487458in}{1.736376in}}{\pgfqpoint{2.484186in}{1.744276in}}{\pgfqpoint{2.478362in}{1.750100in}}%
\pgfpathcurveto{\pgfqpoint{2.472538in}{1.755924in}}{\pgfqpoint{2.464638in}{1.759196in}}{\pgfqpoint{2.456401in}{1.759196in}}%
\pgfpathcurveto{\pgfqpoint{2.448165in}{1.759196in}}{\pgfqpoint{2.440265in}{1.755924in}}{\pgfqpoint{2.434441in}{1.750100in}}%
\pgfpathcurveto{\pgfqpoint{2.428617in}{1.744276in}}{\pgfqpoint{2.425345in}{1.736376in}}{\pgfqpoint{2.425345in}{1.728139in}}%
\pgfpathcurveto{\pgfqpoint{2.425345in}{1.719903in}}{\pgfqpoint{2.428617in}{1.712003in}}{\pgfqpoint{2.434441in}{1.706179in}}%
\pgfpathcurveto{\pgfqpoint{2.440265in}{1.700355in}}{\pgfqpoint{2.448165in}{1.697083in}}{\pgfqpoint{2.456401in}{1.697083in}}%
\pgfpathclose%
\pgfusepath{stroke,fill}%
\end{pgfscope}%
\begin{pgfscope}%
\pgfpathrectangle{\pgfqpoint{0.100000in}{0.212622in}}{\pgfqpoint{3.696000in}{3.696000in}}%
\pgfusepath{clip}%
\pgfsetbuttcap%
\pgfsetroundjoin%
\definecolor{currentfill}{rgb}{0.121569,0.466667,0.705882}%
\pgfsetfillcolor{currentfill}%
\pgfsetfillopacity{0.949216}%
\pgfsetlinewidth{1.003750pt}%
\definecolor{currentstroke}{rgb}{0.121569,0.466667,0.705882}%
\pgfsetstrokecolor{currentstroke}%
\pgfsetstrokeopacity{0.949216}%
\pgfsetdash{}{0pt}%
\pgfpathmoveto{\pgfqpoint{1.755036in}{1.940695in}}%
\pgfpathcurveto{\pgfqpoint{1.763272in}{1.940695in}}{\pgfqpoint{1.771172in}{1.943967in}}{\pgfqpoint{1.776996in}{1.949791in}}%
\pgfpathcurveto{\pgfqpoint{1.782820in}{1.955615in}}{\pgfqpoint{1.786093in}{1.963515in}}{\pgfqpoint{1.786093in}{1.971751in}}%
\pgfpathcurveto{\pgfqpoint{1.786093in}{1.979987in}}{\pgfqpoint{1.782820in}{1.987887in}}{\pgfqpoint{1.776996in}{1.993711in}}%
\pgfpathcurveto{\pgfqpoint{1.771172in}{1.999535in}}{\pgfqpoint{1.763272in}{2.002808in}}{\pgfqpoint{1.755036in}{2.002808in}}%
\pgfpathcurveto{\pgfqpoint{1.746800in}{2.002808in}}{\pgfqpoint{1.738900in}{1.999535in}}{\pgfqpoint{1.733076in}{1.993711in}}%
\pgfpathcurveto{\pgfqpoint{1.727252in}{1.987887in}}{\pgfqpoint{1.723980in}{1.979987in}}{\pgfqpoint{1.723980in}{1.971751in}}%
\pgfpathcurveto{\pgfqpoint{1.723980in}{1.963515in}}{\pgfqpoint{1.727252in}{1.955615in}}{\pgfqpoint{1.733076in}{1.949791in}}%
\pgfpathcurveto{\pgfqpoint{1.738900in}{1.943967in}}{\pgfqpoint{1.746800in}{1.940695in}}{\pgfqpoint{1.755036in}{1.940695in}}%
\pgfpathclose%
\pgfusepath{stroke,fill}%
\end{pgfscope}%
\begin{pgfscope}%
\pgfpathrectangle{\pgfqpoint{0.100000in}{0.212622in}}{\pgfqpoint{3.696000in}{3.696000in}}%
\pgfusepath{clip}%
\pgfsetbuttcap%
\pgfsetroundjoin%
\definecolor{currentfill}{rgb}{0.121569,0.466667,0.705882}%
\pgfsetfillcolor{currentfill}%
\pgfsetfillopacity{0.953783}%
\pgfsetlinewidth{1.003750pt}%
\definecolor{currentstroke}{rgb}{0.121569,0.466667,0.705882}%
\pgfsetstrokecolor{currentstroke}%
\pgfsetstrokeopacity{0.953783}%
\pgfsetdash{}{0pt}%
\pgfpathmoveto{\pgfqpoint{1.795688in}{1.924354in}}%
\pgfpathcurveto{\pgfqpoint{1.803924in}{1.924354in}}{\pgfqpoint{1.811825in}{1.927627in}}{\pgfqpoint{1.817648in}{1.933451in}}%
\pgfpathcurveto{\pgfqpoint{1.823472in}{1.939275in}}{\pgfqpoint{1.826745in}{1.947175in}}{\pgfqpoint{1.826745in}{1.955411in}}%
\pgfpathcurveto{\pgfqpoint{1.826745in}{1.963647in}}{\pgfqpoint{1.823472in}{1.971547in}}{\pgfqpoint{1.817648in}{1.977371in}}%
\pgfpathcurveto{\pgfqpoint{1.811825in}{1.983195in}}{\pgfqpoint{1.803924in}{1.986467in}}{\pgfqpoint{1.795688in}{1.986467in}}%
\pgfpathcurveto{\pgfqpoint{1.787452in}{1.986467in}}{\pgfqpoint{1.779552in}{1.983195in}}{\pgfqpoint{1.773728in}{1.977371in}}%
\pgfpathcurveto{\pgfqpoint{1.767904in}{1.971547in}}{\pgfqpoint{1.764632in}{1.963647in}}{\pgfqpoint{1.764632in}{1.955411in}}%
\pgfpathcurveto{\pgfqpoint{1.764632in}{1.947175in}}{\pgfqpoint{1.767904in}{1.939275in}}{\pgfqpoint{1.773728in}{1.933451in}}%
\pgfpathcurveto{\pgfqpoint{1.779552in}{1.927627in}}{\pgfqpoint{1.787452in}{1.924354in}}{\pgfqpoint{1.795688in}{1.924354in}}%
\pgfpathclose%
\pgfusepath{stroke,fill}%
\end{pgfscope}%
\begin{pgfscope}%
\pgfpathrectangle{\pgfqpoint{0.100000in}{0.212622in}}{\pgfqpoint{3.696000in}{3.696000in}}%
\pgfusepath{clip}%
\pgfsetbuttcap%
\pgfsetroundjoin%
\definecolor{currentfill}{rgb}{0.121569,0.466667,0.705882}%
\pgfsetfillcolor{currentfill}%
\pgfsetfillopacity{0.957733}%
\pgfsetlinewidth{1.003750pt}%
\definecolor{currentstroke}{rgb}{0.121569,0.466667,0.705882}%
\pgfsetstrokecolor{currentstroke}%
\pgfsetstrokeopacity{0.957733}%
\pgfsetdash{}{0pt}%
\pgfpathmoveto{\pgfqpoint{1.832112in}{1.911632in}}%
\pgfpathcurveto{\pgfqpoint{1.840348in}{1.911632in}}{\pgfqpoint{1.848248in}{1.914904in}}{\pgfqpoint{1.854072in}{1.920728in}}%
\pgfpathcurveto{\pgfqpoint{1.859896in}{1.926552in}}{\pgfqpoint{1.863169in}{1.934452in}}{\pgfqpoint{1.863169in}{1.942688in}}%
\pgfpathcurveto{\pgfqpoint{1.863169in}{1.950924in}}{\pgfqpoint{1.859896in}{1.958824in}}{\pgfqpoint{1.854072in}{1.964648in}}%
\pgfpathcurveto{\pgfqpoint{1.848248in}{1.970472in}}{\pgfqpoint{1.840348in}{1.973745in}}{\pgfqpoint{1.832112in}{1.973745in}}%
\pgfpathcurveto{\pgfqpoint{1.823876in}{1.973745in}}{\pgfqpoint{1.815976in}{1.970472in}}{\pgfqpoint{1.810152in}{1.964648in}}%
\pgfpathcurveto{\pgfqpoint{1.804328in}{1.958824in}}{\pgfqpoint{1.801056in}{1.950924in}}{\pgfqpoint{1.801056in}{1.942688in}}%
\pgfpathcurveto{\pgfqpoint{1.801056in}{1.934452in}}{\pgfqpoint{1.804328in}{1.926552in}}{\pgfqpoint{1.810152in}{1.920728in}}%
\pgfpathcurveto{\pgfqpoint{1.815976in}{1.914904in}}{\pgfqpoint{1.823876in}{1.911632in}}{\pgfqpoint{1.832112in}{1.911632in}}%
\pgfpathclose%
\pgfusepath{stroke,fill}%
\end{pgfscope}%
\begin{pgfscope}%
\pgfpathrectangle{\pgfqpoint{0.100000in}{0.212622in}}{\pgfqpoint{3.696000in}{3.696000in}}%
\pgfusepath{clip}%
\pgfsetbuttcap%
\pgfsetroundjoin%
\definecolor{currentfill}{rgb}{0.121569,0.466667,0.705882}%
\pgfsetfillcolor{currentfill}%
\pgfsetfillopacity{0.961901}%
\pgfsetlinewidth{1.003750pt}%
\definecolor{currentstroke}{rgb}{0.121569,0.466667,0.705882}%
\pgfsetstrokecolor{currentstroke}%
\pgfsetstrokeopacity{0.961901}%
\pgfsetdash{}{0pt}%
\pgfpathmoveto{\pgfqpoint{1.864170in}{1.900185in}}%
\pgfpathcurveto{\pgfqpoint{1.872406in}{1.900185in}}{\pgfqpoint{1.880306in}{1.903457in}}{\pgfqpoint{1.886130in}{1.909281in}}%
\pgfpathcurveto{\pgfqpoint{1.891954in}{1.915105in}}{\pgfqpoint{1.895226in}{1.923005in}}{\pgfqpoint{1.895226in}{1.931241in}}%
\pgfpathcurveto{\pgfqpoint{1.895226in}{1.939478in}}{\pgfqpoint{1.891954in}{1.947378in}}{\pgfqpoint{1.886130in}{1.953202in}}%
\pgfpathcurveto{\pgfqpoint{1.880306in}{1.959026in}}{\pgfqpoint{1.872406in}{1.962298in}}{\pgfqpoint{1.864170in}{1.962298in}}%
\pgfpathcurveto{\pgfqpoint{1.855933in}{1.962298in}}{\pgfqpoint{1.848033in}{1.959026in}}{\pgfqpoint{1.842209in}{1.953202in}}%
\pgfpathcurveto{\pgfqpoint{1.836385in}{1.947378in}}{\pgfqpoint{1.833113in}{1.939478in}}{\pgfqpoint{1.833113in}{1.931241in}}%
\pgfpathcurveto{\pgfqpoint{1.833113in}{1.923005in}}{\pgfqpoint{1.836385in}{1.915105in}}{\pgfqpoint{1.842209in}{1.909281in}}%
\pgfpathcurveto{\pgfqpoint{1.848033in}{1.903457in}}{\pgfqpoint{1.855933in}{1.900185in}}{\pgfqpoint{1.864170in}{1.900185in}}%
\pgfpathclose%
\pgfusepath{stroke,fill}%
\end{pgfscope}%
\begin{pgfscope}%
\pgfpathrectangle{\pgfqpoint{0.100000in}{0.212622in}}{\pgfqpoint{3.696000in}{3.696000in}}%
\pgfusepath{clip}%
\pgfsetbuttcap%
\pgfsetroundjoin%
\definecolor{currentfill}{rgb}{0.121569,0.466667,0.705882}%
\pgfsetfillcolor{currentfill}%
\pgfsetfillopacity{0.964508}%
\pgfsetlinewidth{1.003750pt}%
\definecolor{currentstroke}{rgb}{0.121569,0.466667,0.705882}%
\pgfsetstrokecolor{currentstroke}%
\pgfsetstrokeopacity{0.964508}%
\pgfsetdash{}{0pt}%
\pgfpathmoveto{\pgfqpoint{1.888919in}{1.890396in}}%
\pgfpathcurveto{\pgfqpoint{1.897155in}{1.890396in}}{\pgfqpoint{1.905055in}{1.893668in}}{\pgfqpoint{1.910879in}{1.899492in}}%
\pgfpathcurveto{\pgfqpoint{1.916703in}{1.905316in}}{\pgfqpoint{1.919975in}{1.913216in}}{\pgfqpoint{1.919975in}{1.921453in}}%
\pgfpathcurveto{\pgfqpoint{1.919975in}{1.929689in}}{\pgfqpoint{1.916703in}{1.937589in}}{\pgfqpoint{1.910879in}{1.943413in}}%
\pgfpathcurveto{\pgfqpoint{1.905055in}{1.949237in}}{\pgfqpoint{1.897155in}{1.952509in}}{\pgfqpoint{1.888919in}{1.952509in}}%
\pgfpathcurveto{\pgfqpoint{1.880683in}{1.952509in}}{\pgfqpoint{1.872783in}{1.949237in}}{\pgfqpoint{1.866959in}{1.943413in}}%
\pgfpathcurveto{\pgfqpoint{1.861135in}{1.937589in}}{\pgfqpoint{1.857862in}{1.929689in}}{\pgfqpoint{1.857862in}{1.921453in}}%
\pgfpathcurveto{\pgfqpoint{1.857862in}{1.913216in}}{\pgfqpoint{1.861135in}{1.905316in}}{\pgfqpoint{1.866959in}{1.899492in}}%
\pgfpathcurveto{\pgfqpoint{1.872783in}{1.893668in}}{\pgfqpoint{1.880683in}{1.890396in}}{\pgfqpoint{1.888919in}{1.890396in}}%
\pgfpathclose%
\pgfusepath{stroke,fill}%
\end{pgfscope}%
\begin{pgfscope}%
\pgfpathrectangle{\pgfqpoint{0.100000in}{0.212622in}}{\pgfqpoint{3.696000in}{3.696000in}}%
\pgfusepath{clip}%
\pgfsetbuttcap%
\pgfsetroundjoin%
\definecolor{currentfill}{rgb}{0.121569,0.466667,0.705882}%
\pgfsetfillcolor{currentfill}%
\pgfsetfillopacity{0.966650}%
\pgfsetlinewidth{1.003750pt}%
\definecolor{currentstroke}{rgb}{0.121569,0.466667,0.705882}%
\pgfsetstrokecolor{currentstroke}%
\pgfsetstrokeopacity{0.966650}%
\pgfsetdash{}{0pt}%
\pgfpathmoveto{\pgfqpoint{1.905433in}{1.884752in}}%
\pgfpathcurveto{\pgfqpoint{1.913669in}{1.884752in}}{\pgfqpoint{1.921569in}{1.888024in}}{\pgfqpoint{1.927393in}{1.893848in}}%
\pgfpathcurveto{\pgfqpoint{1.933217in}{1.899672in}}{\pgfqpoint{1.936489in}{1.907572in}}{\pgfqpoint{1.936489in}{1.915808in}}%
\pgfpathcurveto{\pgfqpoint{1.936489in}{1.924044in}}{\pgfqpoint{1.933217in}{1.931944in}}{\pgfqpoint{1.927393in}{1.937768in}}%
\pgfpathcurveto{\pgfqpoint{1.921569in}{1.943592in}}{\pgfqpoint{1.913669in}{1.946865in}}{\pgfqpoint{1.905433in}{1.946865in}}%
\pgfpathcurveto{\pgfqpoint{1.897196in}{1.946865in}}{\pgfqpoint{1.889296in}{1.943592in}}{\pgfqpoint{1.883472in}{1.937768in}}%
\pgfpathcurveto{\pgfqpoint{1.877649in}{1.931944in}}{\pgfqpoint{1.874376in}{1.924044in}}{\pgfqpoint{1.874376in}{1.915808in}}%
\pgfpathcurveto{\pgfqpoint{1.874376in}{1.907572in}}{\pgfqpoint{1.877649in}{1.899672in}}{\pgfqpoint{1.883472in}{1.893848in}}%
\pgfpathcurveto{\pgfqpoint{1.889296in}{1.888024in}}{\pgfqpoint{1.897196in}{1.884752in}}{\pgfqpoint{1.905433in}{1.884752in}}%
\pgfpathclose%
\pgfusepath{stroke,fill}%
\end{pgfscope}%
\begin{pgfscope}%
\pgfpathrectangle{\pgfqpoint{0.100000in}{0.212622in}}{\pgfqpoint{3.696000in}{3.696000in}}%
\pgfusepath{clip}%
\pgfsetbuttcap%
\pgfsetroundjoin%
\definecolor{currentfill}{rgb}{0.121569,0.466667,0.705882}%
\pgfsetfillcolor{currentfill}%
\pgfsetfillopacity{0.967190}%
\pgfsetlinewidth{1.003750pt}%
\definecolor{currentstroke}{rgb}{0.121569,0.466667,0.705882}%
\pgfsetstrokecolor{currentstroke}%
\pgfsetstrokeopacity{0.967190}%
\pgfsetdash{}{0pt}%
\pgfpathmoveto{\pgfqpoint{2.476803in}{1.680064in}}%
\pgfpathcurveto{\pgfqpoint{2.485039in}{1.680064in}}{\pgfqpoint{2.492939in}{1.683336in}}{\pgfqpoint{2.498763in}{1.689160in}}%
\pgfpathcurveto{\pgfqpoint{2.504587in}{1.694984in}}{\pgfqpoint{2.507859in}{1.702884in}}{\pgfqpoint{2.507859in}{1.711120in}}%
\pgfpathcurveto{\pgfqpoint{2.507859in}{1.719357in}}{\pgfqpoint{2.504587in}{1.727257in}}{\pgfqpoint{2.498763in}{1.733081in}}%
\pgfpathcurveto{\pgfqpoint{2.492939in}{1.738904in}}{\pgfqpoint{2.485039in}{1.742177in}}{\pgfqpoint{2.476803in}{1.742177in}}%
\pgfpathcurveto{\pgfqpoint{2.468566in}{1.742177in}}{\pgfqpoint{2.460666in}{1.738904in}}{\pgfqpoint{2.454843in}{1.733081in}}%
\pgfpathcurveto{\pgfqpoint{2.449019in}{1.727257in}}{\pgfqpoint{2.445746in}{1.719357in}}{\pgfqpoint{2.445746in}{1.711120in}}%
\pgfpathcurveto{\pgfqpoint{2.445746in}{1.702884in}}{\pgfqpoint{2.449019in}{1.694984in}}{\pgfqpoint{2.454843in}{1.689160in}}%
\pgfpathcurveto{\pgfqpoint{2.460666in}{1.683336in}}{\pgfqpoint{2.468566in}{1.680064in}}{\pgfqpoint{2.476803in}{1.680064in}}%
\pgfpathclose%
\pgfusepath{stroke,fill}%
\end{pgfscope}%
\begin{pgfscope}%
\pgfpathrectangle{\pgfqpoint{0.100000in}{0.212622in}}{\pgfqpoint{3.696000in}{3.696000in}}%
\pgfusepath{clip}%
\pgfsetbuttcap%
\pgfsetroundjoin%
\definecolor{currentfill}{rgb}{0.121569,0.466667,0.705882}%
\pgfsetfillcolor{currentfill}%
\pgfsetfillopacity{0.967837}%
\pgfsetlinewidth{1.003750pt}%
\definecolor{currentstroke}{rgb}{0.121569,0.466667,0.705882}%
\pgfsetstrokecolor{currentstroke}%
\pgfsetstrokeopacity{0.967837}%
\pgfsetdash{}{0pt}%
\pgfpathmoveto{\pgfqpoint{1.916433in}{1.880094in}}%
\pgfpathcurveto{\pgfqpoint{1.924670in}{1.880094in}}{\pgfqpoint{1.932570in}{1.883367in}}{\pgfqpoint{1.938394in}{1.889191in}}%
\pgfpathcurveto{\pgfqpoint{1.944218in}{1.895014in}}{\pgfqpoint{1.947490in}{1.902915in}}{\pgfqpoint{1.947490in}{1.911151in}}%
\pgfpathcurveto{\pgfqpoint{1.947490in}{1.919387in}}{\pgfqpoint{1.944218in}{1.927287in}}{\pgfqpoint{1.938394in}{1.933111in}}%
\pgfpathcurveto{\pgfqpoint{1.932570in}{1.938935in}}{\pgfqpoint{1.924670in}{1.942207in}}{\pgfqpoint{1.916433in}{1.942207in}}%
\pgfpathcurveto{\pgfqpoint{1.908197in}{1.942207in}}{\pgfqpoint{1.900297in}{1.938935in}}{\pgfqpoint{1.894473in}{1.933111in}}%
\pgfpathcurveto{\pgfqpoint{1.888649in}{1.927287in}}{\pgfqpoint{1.885377in}{1.919387in}}{\pgfqpoint{1.885377in}{1.911151in}}%
\pgfpathcurveto{\pgfqpoint{1.885377in}{1.902915in}}{\pgfqpoint{1.888649in}{1.895014in}}{\pgfqpoint{1.894473in}{1.889191in}}%
\pgfpathcurveto{\pgfqpoint{1.900297in}{1.883367in}}{\pgfqpoint{1.908197in}{1.880094in}}{\pgfqpoint{1.916433in}{1.880094in}}%
\pgfpathclose%
\pgfusepath{stroke,fill}%
\end{pgfscope}%
\begin{pgfscope}%
\pgfpathrectangle{\pgfqpoint{0.100000in}{0.212622in}}{\pgfqpoint{3.696000in}{3.696000in}}%
\pgfusepath{clip}%
\pgfsetbuttcap%
\pgfsetroundjoin%
\definecolor{currentfill}{rgb}{0.121569,0.466667,0.705882}%
\pgfsetfillcolor{currentfill}%
\pgfsetfillopacity{0.968206}%
\pgfsetlinewidth{1.003750pt}%
\definecolor{currentstroke}{rgb}{0.121569,0.466667,0.705882}%
\pgfsetstrokecolor{currentstroke}%
\pgfsetstrokeopacity{0.968206}%
\pgfsetdash{}{0pt}%
\pgfpathmoveto{\pgfqpoint{1.919265in}{1.879237in}}%
\pgfpathcurveto{\pgfqpoint{1.927501in}{1.879237in}}{\pgfqpoint{1.935402in}{1.882509in}}{\pgfqpoint{1.941225in}{1.888333in}}%
\pgfpathcurveto{\pgfqpoint{1.947049in}{1.894157in}}{\pgfqpoint{1.950322in}{1.902057in}}{\pgfqpoint{1.950322in}{1.910293in}}%
\pgfpathcurveto{\pgfqpoint{1.950322in}{1.918530in}}{\pgfqpoint{1.947049in}{1.926430in}}{\pgfqpoint{1.941225in}{1.932254in}}%
\pgfpathcurveto{\pgfqpoint{1.935402in}{1.938078in}}{\pgfqpoint{1.927501in}{1.941350in}}{\pgfqpoint{1.919265in}{1.941350in}}%
\pgfpathcurveto{\pgfqpoint{1.911029in}{1.941350in}}{\pgfqpoint{1.903129in}{1.938078in}}{\pgfqpoint{1.897305in}{1.932254in}}%
\pgfpathcurveto{\pgfqpoint{1.891481in}{1.926430in}}{\pgfqpoint{1.888209in}{1.918530in}}{\pgfqpoint{1.888209in}{1.910293in}}%
\pgfpathcurveto{\pgfqpoint{1.888209in}{1.902057in}}{\pgfqpoint{1.891481in}{1.894157in}}{\pgfqpoint{1.897305in}{1.888333in}}%
\pgfpathcurveto{\pgfqpoint{1.903129in}{1.882509in}}{\pgfqpoint{1.911029in}{1.879237in}}{\pgfqpoint{1.919265in}{1.879237in}}%
\pgfpathclose%
\pgfusepath{stroke,fill}%
\end{pgfscope}%
\begin{pgfscope}%
\pgfpathrectangle{\pgfqpoint{0.100000in}{0.212622in}}{\pgfqpoint{3.696000in}{3.696000in}}%
\pgfusepath{clip}%
\pgfsetbuttcap%
\pgfsetroundjoin%
\definecolor{currentfill}{rgb}{0.121569,0.466667,0.705882}%
\pgfsetfillcolor{currentfill}%
\pgfsetfillopacity{0.968734}%
\pgfsetlinewidth{1.003750pt}%
\definecolor{currentstroke}{rgb}{0.121569,0.466667,0.705882}%
\pgfsetstrokecolor{currentstroke}%
\pgfsetstrokeopacity{0.968734}%
\pgfsetdash{}{0pt}%
\pgfpathmoveto{\pgfqpoint{1.924242in}{1.876206in}}%
\pgfpathcurveto{\pgfqpoint{1.932478in}{1.876206in}}{\pgfqpoint{1.940378in}{1.879478in}}{\pgfqpoint{1.946202in}{1.885302in}}%
\pgfpathcurveto{\pgfqpoint{1.952026in}{1.891126in}}{\pgfqpoint{1.955298in}{1.899026in}}{\pgfqpoint{1.955298in}{1.907263in}}%
\pgfpathcurveto{\pgfqpoint{1.955298in}{1.915499in}}{\pgfqpoint{1.952026in}{1.923399in}}{\pgfqpoint{1.946202in}{1.929223in}}%
\pgfpathcurveto{\pgfqpoint{1.940378in}{1.935047in}}{\pgfqpoint{1.932478in}{1.938319in}}{\pgfqpoint{1.924242in}{1.938319in}}%
\pgfpathcurveto{\pgfqpoint{1.916005in}{1.938319in}}{\pgfqpoint{1.908105in}{1.935047in}}{\pgfqpoint{1.902281in}{1.929223in}}%
\pgfpathcurveto{\pgfqpoint{1.896458in}{1.923399in}}{\pgfqpoint{1.893185in}{1.915499in}}{\pgfqpoint{1.893185in}{1.907263in}}%
\pgfpathcurveto{\pgfqpoint{1.893185in}{1.899026in}}{\pgfqpoint{1.896458in}{1.891126in}}{\pgfqpoint{1.902281in}{1.885302in}}%
\pgfpathcurveto{\pgfqpoint{1.908105in}{1.879478in}}{\pgfqpoint{1.916005in}{1.876206in}}{\pgfqpoint{1.924242in}{1.876206in}}%
\pgfpathclose%
\pgfusepath{stroke,fill}%
\end{pgfscope}%
\begin{pgfscope}%
\pgfpathrectangle{\pgfqpoint{0.100000in}{0.212622in}}{\pgfqpoint{3.696000in}{3.696000in}}%
\pgfusepath{clip}%
\pgfsetbuttcap%
\pgfsetroundjoin%
\definecolor{currentfill}{rgb}{0.121569,0.466667,0.705882}%
\pgfsetfillcolor{currentfill}%
\pgfsetfillopacity{0.968806}%
\pgfsetlinewidth{1.003750pt}%
\definecolor{currentstroke}{rgb}{0.121569,0.466667,0.705882}%
\pgfsetstrokecolor{currentstroke}%
\pgfsetstrokeopacity{0.968806}%
\pgfsetdash{}{0pt}%
\pgfpathmoveto{\pgfqpoint{1.924764in}{1.876083in}}%
\pgfpathcurveto{\pgfqpoint{1.933001in}{1.876083in}}{\pgfqpoint{1.940901in}{1.879355in}}{\pgfqpoint{1.946725in}{1.885179in}}%
\pgfpathcurveto{\pgfqpoint{1.952549in}{1.891003in}}{\pgfqpoint{1.955821in}{1.898903in}}{\pgfqpoint{1.955821in}{1.907139in}}%
\pgfpathcurveto{\pgfqpoint{1.955821in}{1.915376in}}{\pgfqpoint{1.952549in}{1.923276in}}{\pgfqpoint{1.946725in}{1.929100in}}%
\pgfpathcurveto{\pgfqpoint{1.940901in}{1.934924in}}{\pgfqpoint{1.933001in}{1.938196in}}{\pgfqpoint{1.924764in}{1.938196in}}%
\pgfpathcurveto{\pgfqpoint{1.916528in}{1.938196in}}{\pgfqpoint{1.908628in}{1.934924in}}{\pgfqpoint{1.902804in}{1.929100in}}%
\pgfpathcurveto{\pgfqpoint{1.896980in}{1.923276in}}{\pgfqpoint{1.893708in}{1.915376in}}{\pgfqpoint{1.893708in}{1.907139in}}%
\pgfpathcurveto{\pgfqpoint{1.893708in}{1.898903in}}{\pgfqpoint{1.896980in}{1.891003in}}{\pgfqpoint{1.902804in}{1.885179in}}%
\pgfpathcurveto{\pgfqpoint{1.908628in}{1.879355in}}{\pgfqpoint{1.916528in}{1.876083in}}{\pgfqpoint{1.924764in}{1.876083in}}%
\pgfpathclose%
\pgfusepath{stroke,fill}%
\end{pgfscope}%
\begin{pgfscope}%
\pgfpathrectangle{\pgfqpoint{0.100000in}{0.212622in}}{\pgfqpoint{3.696000in}{3.696000in}}%
\pgfusepath{clip}%
\pgfsetbuttcap%
\pgfsetroundjoin%
\definecolor{currentfill}{rgb}{0.121569,0.466667,0.705882}%
\pgfsetfillcolor{currentfill}%
\pgfsetfillopacity{0.968908}%
\pgfsetlinewidth{1.003750pt}%
\definecolor{currentstroke}{rgb}{0.121569,0.466667,0.705882}%
\pgfsetstrokecolor{currentstroke}%
\pgfsetstrokeopacity{0.968908}%
\pgfsetdash{}{0pt}%
\pgfpathmoveto{\pgfqpoint{1.925710in}{1.875652in}}%
\pgfpathcurveto{\pgfqpoint{1.933946in}{1.875652in}}{\pgfqpoint{1.941846in}{1.878924in}}{\pgfqpoint{1.947670in}{1.884748in}}%
\pgfpathcurveto{\pgfqpoint{1.953494in}{1.890572in}}{\pgfqpoint{1.956767in}{1.898472in}}{\pgfqpoint{1.956767in}{1.906708in}}%
\pgfpathcurveto{\pgfqpoint{1.956767in}{1.914944in}}{\pgfqpoint{1.953494in}{1.922844in}}{\pgfqpoint{1.947670in}{1.928668in}}%
\pgfpathcurveto{\pgfqpoint{1.941846in}{1.934492in}}{\pgfqpoint{1.933946in}{1.937765in}}{\pgfqpoint{1.925710in}{1.937765in}}%
\pgfpathcurveto{\pgfqpoint{1.917474in}{1.937765in}}{\pgfqpoint{1.909574in}{1.934492in}}{\pgfqpoint{1.903750in}{1.928668in}}%
\pgfpathcurveto{\pgfqpoint{1.897926in}{1.922844in}}{\pgfqpoint{1.894654in}{1.914944in}}{\pgfqpoint{1.894654in}{1.906708in}}%
\pgfpathcurveto{\pgfqpoint{1.894654in}{1.898472in}}{\pgfqpoint{1.897926in}{1.890572in}}{\pgfqpoint{1.903750in}{1.884748in}}%
\pgfpathcurveto{\pgfqpoint{1.909574in}{1.878924in}}{\pgfqpoint{1.917474in}{1.875652in}}{\pgfqpoint{1.925710in}{1.875652in}}%
\pgfpathclose%
\pgfusepath{stroke,fill}%
\end{pgfscope}%
\begin{pgfscope}%
\pgfpathrectangle{\pgfqpoint{0.100000in}{0.212622in}}{\pgfqpoint{3.696000in}{3.696000in}}%
\pgfusepath{clip}%
\pgfsetbuttcap%
\pgfsetroundjoin%
\definecolor{currentfill}{rgb}{0.121569,0.466667,0.705882}%
\pgfsetfillcolor{currentfill}%
\pgfsetfillopacity{0.969104}%
\pgfsetlinewidth{1.003750pt}%
\definecolor{currentstroke}{rgb}{0.121569,0.466667,0.705882}%
\pgfsetstrokecolor{currentstroke}%
\pgfsetstrokeopacity{0.969104}%
\pgfsetdash{}{0pt}%
\pgfpathmoveto{\pgfqpoint{1.927459in}{1.875052in}}%
\pgfpathcurveto{\pgfqpoint{1.935695in}{1.875052in}}{\pgfqpoint{1.943595in}{1.878324in}}{\pgfqpoint{1.949419in}{1.884148in}}%
\pgfpathcurveto{\pgfqpoint{1.955243in}{1.889972in}}{\pgfqpoint{1.958515in}{1.897872in}}{\pgfqpoint{1.958515in}{1.906108in}}%
\pgfpathcurveto{\pgfqpoint{1.958515in}{1.914345in}}{\pgfqpoint{1.955243in}{1.922245in}}{\pgfqpoint{1.949419in}{1.928068in}}%
\pgfpathcurveto{\pgfqpoint{1.943595in}{1.933892in}}{\pgfqpoint{1.935695in}{1.937165in}}{\pgfqpoint{1.927459in}{1.937165in}}%
\pgfpathcurveto{\pgfqpoint{1.919222in}{1.937165in}}{\pgfqpoint{1.911322in}{1.933892in}}{\pgfqpoint{1.905498in}{1.928068in}}%
\pgfpathcurveto{\pgfqpoint{1.899675in}{1.922245in}}{\pgfqpoint{1.896402in}{1.914345in}}{\pgfqpoint{1.896402in}{1.906108in}}%
\pgfpathcurveto{\pgfqpoint{1.896402in}{1.897872in}}{\pgfqpoint{1.899675in}{1.889972in}}{\pgfqpoint{1.905498in}{1.884148in}}%
\pgfpathcurveto{\pgfqpoint{1.911322in}{1.878324in}}{\pgfqpoint{1.919222in}{1.875052in}}{\pgfqpoint{1.927459in}{1.875052in}}%
\pgfpathclose%
\pgfusepath{stroke,fill}%
\end{pgfscope}%
\begin{pgfscope}%
\pgfpathrectangle{\pgfqpoint{0.100000in}{0.212622in}}{\pgfqpoint{3.696000in}{3.696000in}}%
\pgfusepath{clip}%
\pgfsetbuttcap%
\pgfsetroundjoin%
\definecolor{currentfill}{rgb}{0.121569,0.466667,0.705882}%
\pgfsetfillcolor{currentfill}%
\pgfsetfillopacity{0.969424}%
\pgfsetlinewidth{1.003750pt}%
\definecolor{currentstroke}{rgb}{0.121569,0.466667,0.705882}%
\pgfsetstrokecolor{currentstroke}%
\pgfsetstrokeopacity{0.969424}%
\pgfsetdash{}{0pt}%
\pgfpathmoveto{\pgfqpoint{1.930565in}{1.873449in}}%
\pgfpathcurveto{\pgfqpoint{1.938801in}{1.873449in}}{\pgfqpoint{1.946701in}{1.876721in}}{\pgfqpoint{1.952525in}{1.882545in}}%
\pgfpathcurveto{\pgfqpoint{1.958349in}{1.888369in}}{\pgfqpoint{1.961622in}{1.896269in}}{\pgfqpoint{1.961622in}{1.904505in}}%
\pgfpathcurveto{\pgfqpoint{1.961622in}{1.912742in}}{\pgfqpoint{1.958349in}{1.920642in}}{\pgfqpoint{1.952525in}{1.926466in}}%
\pgfpathcurveto{\pgfqpoint{1.946701in}{1.932289in}}{\pgfqpoint{1.938801in}{1.935562in}}{\pgfqpoint{1.930565in}{1.935562in}}%
\pgfpathcurveto{\pgfqpoint{1.922329in}{1.935562in}}{\pgfqpoint{1.914429in}{1.932289in}}{\pgfqpoint{1.908605in}{1.926466in}}%
\pgfpathcurveto{\pgfqpoint{1.902781in}{1.920642in}}{\pgfqpoint{1.899509in}{1.912742in}}{\pgfqpoint{1.899509in}{1.904505in}}%
\pgfpathcurveto{\pgfqpoint{1.899509in}{1.896269in}}{\pgfqpoint{1.902781in}{1.888369in}}{\pgfqpoint{1.908605in}{1.882545in}}%
\pgfpathcurveto{\pgfqpoint{1.914429in}{1.876721in}}{\pgfqpoint{1.922329in}{1.873449in}}{\pgfqpoint{1.930565in}{1.873449in}}%
\pgfpathclose%
\pgfusepath{stroke,fill}%
\end{pgfscope}%
\begin{pgfscope}%
\pgfpathrectangle{\pgfqpoint{0.100000in}{0.212622in}}{\pgfqpoint{3.696000in}{3.696000in}}%
\pgfusepath{clip}%
\pgfsetbuttcap%
\pgfsetroundjoin%
\definecolor{currentfill}{rgb}{0.121569,0.466667,0.705882}%
\pgfsetfillcolor{currentfill}%
\pgfsetfillopacity{0.970046}%
\pgfsetlinewidth{1.003750pt}%
\definecolor{currentstroke}{rgb}{0.121569,0.466667,0.705882}%
\pgfsetstrokecolor{currentstroke}%
\pgfsetstrokeopacity{0.970046}%
\pgfsetdash{}{0pt}%
\pgfpathmoveto{\pgfqpoint{1.936395in}{1.871508in}}%
\pgfpathcurveto{\pgfqpoint{1.944631in}{1.871508in}}{\pgfqpoint{1.952531in}{1.874780in}}{\pgfqpoint{1.958355in}{1.880604in}}%
\pgfpathcurveto{\pgfqpoint{1.964179in}{1.886428in}}{\pgfqpoint{1.967452in}{1.894328in}}{\pgfqpoint{1.967452in}{1.902564in}}%
\pgfpathcurveto{\pgfqpoint{1.967452in}{1.910800in}}{\pgfqpoint{1.964179in}{1.918701in}}{\pgfqpoint{1.958355in}{1.924524in}}%
\pgfpathcurveto{\pgfqpoint{1.952531in}{1.930348in}}{\pgfqpoint{1.944631in}{1.933621in}}{\pgfqpoint{1.936395in}{1.933621in}}%
\pgfpathcurveto{\pgfqpoint{1.928159in}{1.933621in}}{\pgfqpoint{1.920259in}{1.930348in}}{\pgfqpoint{1.914435in}{1.924524in}}%
\pgfpathcurveto{\pgfqpoint{1.908611in}{1.918701in}}{\pgfqpoint{1.905339in}{1.910800in}}{\pgfqpoint{1.905339in}{1.902564in}}%
\pgfpathcurveto{\pgfqpoint{1.905339in}{1.894328in}}{\pgfqpoint{1.908611in}{1.886428in}}{\pgfqpoint{1.914435in}{1.880604in}}%
\pgfpathcurveto{\pgfqpoint{1.920259in}{1.874780in}}{\pgfqpoint{1.928159in}{1.871508in}}{\pgfqpoint{1.936395in}{1.871508in}}%
\pgfpathclose%
\pgfusepath{stroke,fill}%
\end{pgfscope}%
\begin{pgfscope}%
\pgfpathrectangle{\pgfqpoint{0.100000in}{0.212622in}}{\pgfqpoint{3.696000in}{3.696000in}}%
\pgfusepath{clip}%
\pgfsetbuttcap%
\pgfsetroundjoin%
\definecolor{currentfill}{rgb}{0.121569,0.466667,0.705882}%
\pgfsetfillcolor{currentfill}%
\pgfsetfillopacity{0.971076}%
\pgfsetlinewidth{1.003750pt}%
\definecolor{currentstroke}{rgb}{0.121569,0.466667,0.705882}%
\pgfsetstrokecolor{currentstroke}%
\pgfsetstrokeopacity{0.971076}%
\pgfsetdash{}{0pt}%
\pgfpathmoveto{\pgfqpoint{1.946801in}{1.866497in}}%
\pgfpathcurveto{\pgfqpoint{1.955037in}{1.866497in}}{\pgfqpoint{1.962937in}{1.869769in}}{\pgfqpoint{1.968761in}{1.875593in}}%
\pgfpathcurveto{\pgfqpoint{1.974585in}{1.881417in}}{\pgfqpoint{1.977857in}{1.889317in}}{\pgfqpoint{1.977857in}{1.897554in}}%
\pgfpathcurveto{\pgfqpoint{1.977857in}{1.905790in}}{\pgfqpoint{1.974585in}{1.913690in}}{\pgfqpoint{1.968761in}{1.919514in}}%
\pgfpathcurveto{\pgfqpoint{1.962937in}{1.925338in}}{\pgfqpoint{1.955037in}{1.928610in}}{\pgfqpoint{1.946801in}{1.928610in}}%
\pgfpathcurveto{\pgfqpoint{1.938565in}{1.928610in}}{\pgfqpoint{1.930665in}{1.925338in}}{\pgfqpoint{1.924841in}{1.919514in}}%
\pgfpathcurveto{\pgfqpoint{1.919017in}{1.913690in}}{\pgfqpoint{1.915744in}{1.905790in}}{\pgfqpoint{1.915744in}{1.897554in}}%
\pgfpathcurveto{\pgfqpoint{1.915744in}{1.889317in}}{\pgfqpoint{1.919017in}{1.881417in}}{\pgfqpoint{1.924841in}{1.875593in}}%
\pgfpathcurveto{\pgfqpoint{1.930665in}{1.869769in}}{\pgfqpoint{1.938565in}{1.866497in}}{\pgfqpoint{1.946801in}{1.866497in}}%
\pgfpathclose%
\pgfusepath{stroke,fill}%
\end{pgfscope}%
\begin{pgfscope}%
\pgfpathrectangle{\pgfqpoint{0.100000in}{0.212622in}}{\pgfqpoint{3.696000in}{3.696000in}}%
\pgfusepath{clip}%
\pgfsetbuttcap%
\pgfsetroundjoin%
\definecolor{currentfill}{rgb}{0.121569,0.466667,0.705882}%
\pgfsetfillcolor{currentfill}%
\pgfsetfillopacity{0.972836}%
\pgfsetlinewidth{1.003750pt}%
\definecolor{currentstroke}{rgb}{0.121569,0.466667,0.705882}%
\pgfsetstrokecolor{currentstroke}%
\pgfsetstrokeopacity{0.972836}%
\pgfsetdash{}{0pt}%
\pgfpathmoveto{\pgfqpoint{1.966100in}{1.858228in}}%
\pgfpathcurveto{\pgfqpoint{1.974337in}{1.858228in}}{\pgfqpoint{1.982237in}{1.861500in}}{\pgfqpoint{1.988061in}{1.867324in}}%
\pgfpathcurveto{\pgfqpoint{1.993885in}{1.873148in}}{\pgfqpoint{1.997157in}{1.881048in}}{\pgfqpoint{1.997157in}{1.889285in}}%
\pgfpathcurveto{\pgfqpoint{1.997157in}{1.897521in}}{\pgfqpoint{1.993885in}{1.905421in}}{\pgfqpoint{1.988061in}{1.911245in}}%
\pgfpathcurveto{\pgfqpoint{1.982237in}{1.917069in}}{\pgfqpoint{1.974337in}{1.920341in}}{\pgfqpoint{1.966100in}{1.920341in}}%
\pgfpathcurveto{\pgfqpoint{1.957864in}{1.920341in}}{\pgfqpoint{1.949964in}{1.917069in}}{\pgfqpoint{1.944140in}{1.911245in}}%
\pgfpathcurveto{\pgfqpoint{1.938316in}{1.905421in}}{\pgfqpoint{1.935044in}{1.897521in}}{\pgfqpoint{1.935044in}{1.889285in}}%
\pgfpathcurveto{\pgfqpoint{1.935044in}{1.881048in}}{\pgfqpoint{1.938316in}{1.873148in}}{\pgfqpoint{1.944140in}{1.867324in}}%
\pgfpathcurveto{\pgfqpoint{1.949964in}{1.861500in}}{\pgfqpoint{1.957864in}{1.858228in}}{\pgfqpoint{1.966100in}{1.858228in}}%
\pgfpathclose%
\pgfusepath{stroke,fill}%
\end{pgfscope}%
\begin{pgfscope}%
\pgfpathrectangle{\pgfqpoint{0.100000in}{0.212622in}}{\pgfqpoint{3.696000in}{3.696000in}}%
\pgfusepath{clip}%
\pgfsetbuttcap%
\pgfsetroundjoin%
\definecolor{currentfill}{rgb}{0.121569,0.466667,0.705882}%
\pgfsetfillcolor{currentfill}%
\pgfsetfillopacity{0.976396}%
\pgfsetlinewidth{1.003750pt}%
\definecolor{currentstroke}{rgb}{0.121569,0.466667,0.705882}%
\pgfsetstrokecolor{currentstroke}%
\pgfsetstrokeopacity{0.976396}%
\pgfsetdash{}{0pt}%
\pgfpathmoveto{\pgfqpoint{2.000991in}{1.844383in}}%
\pgfpathcurveto{\pgfqpoint{2.009227in}{1.844383in}}{\pgfqpoint{2.017127in}{1.847656in}}{\pgfqpoint{2.022951in}{1.853480in}}%
\pgfpathcurveto{\pgfqpoint{2.028775in}{1.859304in}}{\pgfqpoint{2.032048in}{1.867204in}}{\pgfqpoint{2.032048in}{1.875440in}}%
\pgfpathcurveto{\pgfqpoint{2.032048in}{1.883676in}}{\pgfqpoint{2.028775in}{1.891576in}}{\pgfqpoint{2.022951in}{1.897400in}}%
\pgfpathcurveto{\pgfqpoint{2.017127in}{1.903224in}}{\pgfqpoint{2.009227in}{1.906496in}}{\pgfqpoint{2.000991in}{1.906496in}}%
\pgfpathcurveto{\pgfqpoint{1.992755in}{1.906496in}}{\pgfqpoint{1.984855in}{1.903224in}}{\pgfqpoint{1.979031in}{1.897400in}}%
\pgfpathcurveto{\pgfqpoint{1.973207in}{1.891576in}}{\pgfqpoint{1.969935in}{1.883676in}}{\pgfqpoint{1.969935in}{1.875440in}}%
\pgfpathcurveto{\pgfqpoint{1.969935in}{1.867204in}}{\pgfqpoint{1.973207in}{1.859304in}}{\pgfqpoint{1.979031in}{1.853480in}}%
\pgfpathcurveto{\pgfqpoint{1.984855in}{1.847656in}}{\pgfqpoint{1.992755in}{1.844383in}}{\pgfqpoint{2.000991in}{1.844383in}}%
\pgfpathclose%
\pgfusepath{stroke,fill}%
\end{pgfscope}%
\begin{pgfscope}%
\pgfpathrectangle{\pgfqpoint{0.100000in}{0.212622in}}{\pgfqpoint{3.696000in}{3.696000in}}%
\pgfusepath{clip}%
\pgfsetbuttcap%
\pgfsetroundjoin%
\definecolor{currentfill}{rgb}{0.121569,0.466667,0.705882}%
\pgfsetfillcolor{currentfill}%
\pgfsetfillopacity{0.979277}%
\pgfsetlinewidth{1.003750pt}%
\definecolor{currentstroke}{rgb}{0.121569,0.466667,0.705882}%
\pgfsetstrokecolor{currentstroke}%
\pgfsetstrokeopacity{0.979277}%
\pgfsetdash{}{0pt}%
\pgfpathmoveto{\pgfqpoint{2.033468in}{1.829884in}}%
\pgfpathcurveto{\pgfqpoint{2.041704in}{1.829884in}}{\pgfqpoint{2.049604in}{1.833157in}}{\pgfqpoint{2.055428in}{1.838980in}}%
\pgfpathcurveto{\pgfqpoint{2.061252in}{1.844804in}}{\pgfqpoint{2.064524in}{1.852704in}}{\pgfqpoint{2.064524in}{1.860941in}}%
\pgfpathcurveto{\pgfqpoint{2.064524in}{1.869177in}}{\pgfqpoint{2.061252in}{1.877077in}}{\pgfqpoint{2.055428in}{1.882901in}}%
\pgfpathcurveto{\pgfqpoint{2.049604in}{1.888725in}}{\pgfqpoint{2.041704in}{1.891997in}}{\pgfqpoint{2.033468in}{1.891997in}}%
\pgfpathcurveto{\pgfqpoint{2.025232in}{1.891997in}}{\pgfqpoint{2.017332in}{1.888725in}}{\pgfqpoint{2.011508in}{1.882901in}}%
\pgfpathcurveto{\pgfqpoint{2.005684in}{1.877077in}}{\pgfqpoint{2.002411in}{1.869177in}}{\pgfqpoint{2.002411in}{1.860941in}}%
\pgfpathcurveto{\pgfqpoint{2.002411in}{1.852704in}}{\pgfqpoint{2.005684in}{1.844804in}}{\pgfqpoint{2.011508in}{1.838980in}}%
\pgfpathcurveto{\pgfqpoint{2.017332in}{1.833157in}}{\pgfqpoint{2.025232in}{1.829884in}}{\pgfqpoint{2.033468in}{1.829884in}}%
\pgfpathclose%
\pgfusepath{stroke,fill}%
\end{pgfscope}%
\begin{pgfscope}%
\pgfpathrectangle{\pgfqpoint{0.100000in}{0.212622in}}{\pgfqpoint{3.696000in}{3.696000in}}%
\pgfusepath{clip}%
\pgfsetbuttcap%
\pgfsetroundjoin%
\definecolor{currentfill}{rgb}{0.121569,0.466667,0.705882}%
\pgfsetfillcolor{currentfill}%
\pgfsetfillopacity{0.982160}%
\pgfsetlinewidth{1.003750pt}%
\definecolor{currentstroke}{rgb}{0.121569,0.466667,0.705882}%
\pgfsetstrokecolor{currentstroke}%
\pgfsetstrokeopacity{0.982160}%
\pgfsetdash{}{0pt}%
\pgfpathmoveto{\pgfqpoint{2.475823in}{1.630199in}}%
\pgfpathcurveto{\pgfqpoint{2.484059in}{1.630199in}}{\pgfqpoint{2.491959in}{1.633471in}}{\pgfqpoint{2.497783in}{1.639295in}}%
\pgfpathcurveto{\pgfqpoint{2.503607in}{1.645119in}}{\pgfqpoint{2.506879in}{1.653019in}}{\pgfqpoint{2.506879in}{1.661255in}}%
\pgfpathcurveto{\pgfqpoint{2.506879in}{1.669491in}}{\pgfqpoint{2.503607in}{1.677391in}}{\pgfqpoint{2.497783in}{1.683215in}}%
\pgfpathcurveto{\pgfqpoint{2.491959in}{1.689039in}}{\pgfqpoint{2.484059in}{1.692312in}}{\pgfqpoint{2.475823in}{1.692312in}}%
\pgfpathcurveto{\pgfqpoint{2.467587in}{1.692312in}}{\pgfqpoint{2.459686in}{1.689039in}}{\pgfqpoint{2.453863in}{1.683215in}}%
\pgfpathcurveto{\pgfqpoint{2.448039in}{1.677391in}}{\pgfqpoint{2.444766in}{1.669491in}}{\pgfqpoint{2.444766in}{1.661255in}}%
\pgfpathcurveto{\pgfqpoint{2.444766in}{1.653019in}}{\pgfqpoint{2.448039in}{1.645119in}}{\pgfqpoint{2.453863in}{1.639295in}}%
\pgfpathcurveto{\pgfqpoint{2.459686in}{1.633471in}}{\pgfqpoint{2.467587in}{1.630199in}}{\pgfqpoint{2.475823in}{1.630199in}}%
\pgfpathclose%
\pgfusepath{stroke,fill}%
\end{pgfscope}%
\begin{pgfscope}%
\pgfpathrectangle{\pgfqpoint{0.100000in}{0.212622in}}{\pgfqpoint{3.696000in}{3.696000in}}%
\pgfusepath{clip}%
\pgfsetbuttcap%
\pgfsetroundjoin%
\definecolor{currentfill}{rgb}{0.121569,0.466667,0.705882}%
\pgfsetfillcolor{currentfill}%
\pgfsetfillopacity{0.984284}%
\pgfsetlinewidth{1.003750pt}%
\definecolor{currentstroke}{rgb}{0.121569,0.466667,0.705882}%
\pgfsetstrokecolor{currentstroke}%
\pgfsetstrokeopacity{0.984284}%
\pgfsetdash{}{0pt}%
\pgfpathmoveto{\pgfqpoint{2.092391in}{1.801005in}}%
\pgfpathcurveto{\pgfqpoint{2.100627in}{1.801005in}}{\pgfqpoint{2.108527in}{1.804277in}}{\pgfqpoint{2.114351in}{1.810101in}}%
\pgfpathcurveto{\pgfqpoint{2.120175in}{1.815925in}}{\pgfqpoint{2.123448in}{1.823825in}}{\pgfqpoint{2.123448in}{1.832061in}}%
\pgfpathcurveto{\pgfqpoint{2.123448in}{1.840297in}}{\pgfqpoint{2.120175in}{1.848197in}}{\pgfqpoint{2.114351in}{1.854021in}}%
\pgfpathcurveto{\pgfqpoint{2.108527in}{1.859845in}}{\pgfqpoint{2.100627in}{1.863118in}}{\pgfqpoint{2.092391in}{1.863118in}}%
\pgfpathcurveto{\pgfqpoint{2.084155in}{1.863118in}}{\pgfqpoint{2.076255in}{1.859845in}}{\pgfqpoint{2.070431in}{1.854021in}}%
\pgfpathcurveto{\pgfqpoint{2.064607in}{1.848197in}}{\pgfqpoint{2.061335in}{1.840297in}}{\pgfqpoint{2.061335in}{1.832061in}}%
\pgfpathcurveto{\pgfqpoint{2.061335in}{1.823825in}}{\pgfqpoint{2.064607in}{1.815925in}}{\pgfqpoint{2.070431in}{1.810101in}}%
\pgfpathcurveto{\pgfqpoint{2.076255in}{1.804277in}}{\pgfqpoint{2.084155in}{1.801005in}}{\pgfqpoint{2.092391in}{1.801005in}}%
\pgfpathclose%
\pgfusepath{stroke,fill}%
\end{pgfscope}%
\begin{pgfscope}%
\pgfpathrectangle{\pgfqpoint{0.100000in}{0.212622in}}{\pgfqpoint{3.696000in}{3.696000in}}%
\pgfusepath{clip}%
\pgfsetbuttcap%
\pgfsetroundjoin%
\definecolor{currentfill}{rgb}{0.121569,0.466667,0.705882}%
\pgfsetfillcolor{currentfill}%
\pgfsetfillopacity{0.990796}%
\pgfsetlinewidth{1.003750pt}%
\definecolor{currentstroke}{rgb}{0.121569,0.466667,0.705882}%
\pgfsetstrokecolor{currentstroke}%
\pgfsetstrokeopacity{0.990796}%
\pgfsetdash{}{0pt}%
\pgfpathmoveto{\pgfqpoint{2.147725in}{1.782566in}}%
\pgfpathcurveto{\pgfqpoint{2.155962in}{1.782566in}}{\pgfqpoint{2.163862in}{1.785839in}}{\pgfqpoint{2.169686in}{1.791662in}}%
\pgfpathcurveto{\pgfqpoint{2.175510in}{1.797486in}}{\pgfqpoint{2.178782in}{1.805386in}}{\pgfqpoint{2.178782in}{1.813623in}}%
\pgfpathcurveto{\pgfqpoint{2.178782in}{1.821859in}}{\pgfqpoint{2.175510in}{1.829759in}}{\pgfqpoint{2.169686in}{1.835583in}}%
\pgfpathcurveto{\pgfqpoint{2.163862in}{1.841407in}}{\pgfqpoint{2.155962in}{1.844679in}}{\pgfqpoint{2.147725in}{1.844679in}}%
\pgfpathcurveto{\pgfqpoint{2.139489in}{1.844679in}}{\pgfqpoint{2.131589in}{1.841407in}}{\pgfqpoint{2.125765in}{1.835583in}}%
\pgfpathcurveto{\pgfqpoint{2.119941in}{1.829759in}}{\pgfqpoint{2.116669in}{1.821859in}}{\pgfqpoint{2.116669in}{1.813623in}}%
\pgfpathcurveto{\pgfqpoint{2.116669in}{1.805386in}}{\pgfqpoint{2.119941in}{1.797486in}}{\pgfqpoint{2.125765in}{1.791662in}}%
\pgfpathcurveto{\pgfqpoint{2.131589in}{1.785839in}}{\pgfqpoint{2.139489in}{1.782566in}}{\pgfqpoint{2.147725in}{1.782566in}}%
\pgfpathclose%
\pgfusepath{stroke,fill}%
\end{pgfscope}%
\begin{pgfscope}%
\pgfpathrectangle{\pgfqpoint{0.100000in}{0.212622in}}{\pgfqpoint{3.696000in}{3.696000in}}%
\pgfusepath{clip}%
\pgfsetbuttcap%
\pgfsetroundjoin%
\definecolor{currentfill}{rgb}{0.121569,0.466667,0.705882}%
\pgfsetfillcolor{currentfill}%
\pgfsetfillopacity{0.992622}%
\pgfsetlinewidth{1.003750pt}%
\definecolor{currentstroke}{rgb}{0.121569,0.466667,0.705882}%
\pgfsetstrokecolor{currentstroke}%
\pgfsetstrokeopacity{0.992622}%
\pgfsetdash{}{0pt}%
\pgfpathmoveto{\pgfqpoint{2.450023in}{1.642000in}}%
\pgfpathcurveto{\pgfqpoint{2.458259in}{1.642000in}}{\pgfqpoint{2.466159in}{1.645273in}}{\pgfqpoint{2.471983in}{1.651097in}}%
\pgfpathcurveto{\pgfqpoint{2.477807in}{1.656921in}}{\pgfqpoint{2.481080in}{1.664821in}}{\pgfqpoint{2.481080in}{1.673057in}}%
\pgfpathcurveto{\pgfqpoint{2.481080in}{1.681293in}}{\pgfqpoint{2.477807in}{1.689193in}}{\pgfqpoint{2.471983in}{1.695017in}}%
\pgfpathcurveto{\pgfqpoint{2.466159in}{1.700841in}}{\pgfqpoint{2.458259in}{1.704113in}}{\pgfqpoint{2.450023in}{1.704113in}}%
\pgfpathcurveto{\pgfqpoint{2.441787in}{1.704113in}}{\pgfqpoint{2.433887in}{1.700841in}}{\pgfqpoint{2.428063in}{1.695017in}}%
\pgfpathcurveto{\pgfqpoint{2.422239in}{1.689193in}}{\pgfqpoint{2.418967in}{1.681293in}}{\pgfqpoint{2.418967in}{1.673057in}}%
\pgfpathcurveto{\pgfqpoint{2.418967in}{1.664821in}}{\pgfqpoint{2.422239in}{1.656921in}}{\pgfqpoint{2.428063in}{1.651097in}}%
\pgfpathcurveto{\pgfqpoint{2.433887in}{1.645273in}}{\pgfqpoint{2.441787in}{1.642000in}}{\pgfqpoint{2.450023in}{1.642000in}}%
\pgfpathclose%
\pgfusepath{stroke,fill}%
\end{pgfscope}%
\begin{pgfscope}%
\pgfpathrectangle{\pgfqpoint{0.100000in}{0.212622in}}{\pgfqpoint{3.696000in}{3.696000in}}%
\pgfusepath{clip}%
\pgfsetbuttcap%
\pgfsetroundjoin%
\definecolor{currentfill}{rgb}{0.121569,0.466667,0.705882}%
\pgfsetfillcolor{currentfill}%
\pgfsetfillopacity{0.995241}%
\pgfsetlinewidth{1.003750pt}%
\definecolor{currentstroke}{rgb}{0.121569,0.466667,0.705882}%
\pgfsetstrokecolor{currentstroke}%
\pgfsetstrokeopacity{0.995241}%
\pgfsetdash{}{0pt}%
\pgfpathmoveto{\pgfqpoint{2.196342in}{1.761325in}}%
\pgfpathcurveto{\pgfqpoint{2.204578in}{1.761325in}}{\pgfqpoint{2.212478in}{1.764597in}}{\pgfqpoint{2.218302in}{1.770421in}}%
\pgfpathcurveto{\pgfqpoint{2.224126in}{1.776245in}}{\pgfqpoint{2.227398in}{1.784145in}}{\pgfqpoint{2.227398in}{1.792381in}}%
\pgfpathcurveto{\pgfqpoint{2.227398in}{1.800618in}}{\pgfqpoint{2.224126in}{1.808518in}}{\pgfqpoint{2.218302in}{1.814342in}}%
\pgfpathcurveto{\pgfqpoint{2.212478in}{1.820166in}}{\pgfqpoint{2.204578in}{1.823438in}}{\pgfqpoint{2.196342in}{1.823438in}}%
\pgfpathcurveto{\pgfqpoint{2.188105in}{1.823438in}}{\pgfqpoint{2.180205in}{1.820166in}}{\pgfqpoint{2.174381in}{1.814342in}}%
\pgfpathcurveto{\pgfqpoint{2.168557in}{1.808518in}}{\pgfqpoint{2.165285in}{1.800618in}}{\pgfqpoint{2.165285in}{1.792381in}}%
\pgfpathcurveto{\pgfqpoint{2.165285in}{1.784145in}}{\pgfqpoint{2.168557in}{1.776245in}}{\pgfqpoint{2.174381in}{1.770421in}}%
\pgfpathcurveto{\pgfqpoint{2.180205in}{1.764597in}}{\pgfqpoint{2.188105in}{1.761325in}}{\pgfqpoint{2.196342in}{1.761325in}}%
\pgfpathclose%
\pgfusepath{stroke,fill}%
\end{pgfscope}%
\begin{pgfscope}%
\pgfpathrectangle{\pgfqpoint{0.100000in}{0.212622in}}{\pgfqpoint{3.696000in}{3.696000in}}%
\pgfusepath{clip}%
\pgfsetbuttcap%
\pgfsetroundjoin%
\definecolor{currentfill}{rgb}{0.121569,0.466667,0.705882}%
\pgfsetfillcolor{currentfill}%
\pgfsetfillopacity{0.997864}%
\pgfsetlinewidth{1.003750pt}%
\definecolor{currentstroke}{rgb}{0.121569,0.466667,0.705882}%
\pgfsetstrokecolor{currentstroke}%
\pgfsetstrokeopacity{0.997864}%
\pgfsetdash{}{0pt}%
\pgfpathmoveto{\pgfqpoint{2.241151in}{1.739280in}}%
\pgfpathcurveto{\pgfqpoint{2.249387in}{1.739280in}}{\pgfqpoint{2.257287in}{1.742552in}}{\pgfqpoint{2.263111in}{1.748376in}}%
\pgfpathcurveto{\pgfqpoint{2.268935in}{1.754200in}}{\pgfqpoint{2.272207in}{1.762100in}}{\pgfqpoint{2.272207in}{1.770336in}}%
\pgfpathcurveto{\pgfqpoint{2.272207in}{1.778573in}}{\pgfqpoint{2.268935in}{1.786473in}}{\pgfqpoint{2.263111in}{1.792297in}}%
\pgfpathcurveto{\pgfqpoint{2.257287in}{1.798120in}}{\pgfqpoint{2.249387in}{1.801393in}}{\pgfqpoint{2.241151in}{1.801393in}}%
\pgfpathcurveto{\pgfqpoint{2.232914in}{1.801393in}}{\pgfqpoint{2.225014in}{1.798120in}}{\pgfqpoint{2.219190in}{1.792297in}}%
\pgfpathcurveto{\pgfqpoint{2.213367in}{1.786473in}}{\pgfqpoint{2.210094in}{1.778573in}}{\pgfqpoint{2.210094in}{1.770336in}}%
\pgfpathcurveto{\pgfqpoint{2.210094in}{1.762100in}}{\pgfqpoint{2.213367in}{1.754200in}}{\pgfqpoint{2.219190in}{1.748376in}}%
\pgfpathcurveto{\pgfqpoint{2.225014in}{1.742552in}}{\pgfqpoint{2.232914in}{1.739280in}}{\pgfqpoint{2.241151in}{1.739280in}}%
\pgfpathclose%
\pgfusepath{stroke,fill}%
\end{pgfscope}%
\begin{pgfscope}%
\pgfpathrectangle{\pgfqpoint{0.100000in}{0.212622in}}{\pgfqpoint{3.696000in}{3.696000in}}%
\pgfusepath{clip}%
\pgfsetbuttcap%
\pgfsetroundjoin%
\definecolor{currentfill}{rgb}{0.121569,0.466667,0.705882}%
\pgfsetfillcolor{currentfill}%
\pgfsetfillopacity{0.999152}%
\pgfsetlinewidth{1.003750pt}%
\definecolor{currentstroke}{rgb}{0.121569,0.466667,0.705882}%
\pgfsetstrokecolor{currentstroke}%
\pgfsetstrokeopacity{0.999152}%
\pgfsetdash{}{0pt}%
\pgfpathmoveto{\pgfqpoint{2.398033in}{1.666544in}}%
\pgfpathcurveto{\pgfqpoint{2.406269in}{1.666544in}}{\pgfqpoint{2.414170in}{1.669816in}}{\pgfqpoint{2.419993in}{1.675640in}}%
\pgfpathcurveto{\pgfqpoint{2.425817in}{1.681464in}}{\pgfqpoint{2.429090in}{1.689364in}}{\pgfqpoint{2.429090in}{1.697600in}}%
\pgfpathcurveto{\pgfqpoint{2.429090in}{1.705837in}}{\pgfqpoint{2.425817in}{1.713737in}}{\pgfqpoint{2.419993in}{1.719561in}}%
\pgfpathcurveto{\pgfqpoint{2.414170in}{1.725384in}}{\pgfqpoint{2.406269in}{1.728657in}}{\pgfqpoint{2.398033in}{1.728657in}}%
\pgfpathcurveto{\pgfqpoint{2.389797in}{1.728657in}}{\pgfqpoint{2.381897in}{1.725384in}}{\pgfqpoint{2.376073in}{1.719561in}}%
\pgfpathcurveto{\pgfqpoint{2.370249in}{1.713737in}}{\pgfqpoint{2.366977in}{1.705837in}}{\pgfqpoint{2.366977in}{1.697600in}}%
\pgfpathcurveto{\pgfqpoint{2.366977in}{1.689364in}}{\pgfqpoint{2.370249in}{1.681464in}}{\pgfqpoint{2.376073in}{1.675640in}}%
\pgfpathcurveto{\pgfqpoint{2.381897in}{1.669816in}}{\pgfqpoint{2.389797in}{1.666544in}}{\pgfqpoint{2.398033in}{1.666544in}}%
\pgfpathclose%
\pgfusepath{stroke,fill}%
\end{pgfscope}%
\begin{pgfscope}%
\pgfpathrectangle{\pgfqpoint{0.100000in}{0.212622in}}{\pgfqpoint{3.696000in}{3.696000in}}%
\pgfusepath{clip}%
\pgfsetbuttcap%
\pgfsetroundjoin%
\definecolor{currentfill}{rgb}{0.121569,0.466667,0.705882}%
\pgfsetfillcolor{currentfill}%
\pgfsetlinewidth{1.003750pt}%
\definecolor{currentstroke}{rgb}{0.121569,0.466667,0.705882}%
\pgfsetstrokecolor{currentstroke}%
\pgfsetdash{}{0pt}%
\pgfpathmoveto{\pgfqpoint{2.324154in}{1.711381in}}%
\pgfpathcurveto{\pgfqpoint{2.332390in}{1.711381in}}{\pgfqpoint{2.340290in}{1.714653in}}{\pgfqpoint{2.346114in}{1.720477in}}%
\pgfpathcurveto{\pgfqpoint{2.351938in}{1.726301in}}{\pgfqpoint{2.355211in}{1.734201in}}{\pgfqpoint{2.355211in}{1.742437in}}%
\pgfpathcurveto{\pgfqpoint{2.355211in}{1.750673in}}{\pgfqpoint{2.351938in}{1.758573in}}{\pgfqpoint{2.346114in}{1.764397in}}%
\pgfpathcurveto{\pgfqpoint{2.340290in}{1.770221in}}{\pgfqpoint{2.332390in}{1.773494in}}{\pgfqpoint{2.324154in}{1.773494in}}%
\pgfpathcurveto{\pgfqpoint{2.315918in}{1.773494in}}{\pgfqpoint{2.308018in}{1.770221in}}{\pgfqpoint{2.302194in}{1.764397in}}%
\pgfpathcurveto{\pgfqpoint{2.296370in}{1.758573in}}{\pgfqpoint{2.293098in}{1.750673in}}{\pgfqpoint{2.293098in}{1.742437in}}%
\pgfpathcurveto{\pgfqpoint{2.293098in}{1.734201in}}{\pgfqpoint{2.296370in}{1.726301in}}{\pgfqpoint{2.302194in}{1.720477in}}%
\pgfpathcurveto{\pgfqpoint{2.308018in}{1.714653in}}{\pgfqpoint{2.315918in}{1.711381in}}{\pgfqpoint{2.324154in}{1.711381in}}%
\pgfpathclose%
\pgfusepath{stroke,fill}%
\end{pgfscope}%
\begin{pgfscope}%
\pgfsetbuttcap%
\pgfsetmiterjoin%
\definecolor{currentfill}{rgb}{1.000000,1.000000,1.000000}%
\pgfsetfillcolor{currentfill}%
\pgfsetfillopacity{0.800000}%
\pgfsetlinewidth{1.003750pt}%
\definecolor{currentstroke}{rgb}{0.800000,0.800000,0.800000}%
\pgfsetstrokecolor{currentstroke}%
\pgfsetstrokeopacity{0.800000}%
\pgfsetdash{}{0pt}%
\pgfpathmoveto{\pgfqpoint{2.104889in}{3.410289in}}%
\pgfpathlineto{\pgfqpoint{3.698778in}{3.410289in}}%
\pgfpathquadraticcurveto{\pgfqpoint{3.726556in}{3.410289in}}{\pgfqpoint{3.726556in}{3.438067in}}%
\pgfpathlineto{\pgfqpoint{3.726556in}{3.811400in}}%
\pgfpathquadraticcurveto{\pgfqpoint{3.726556in}{3.839178in}}{\pgfqpoint{3.698778in}{3.839178in}}%
\pgfpathlineto{\pgfqpoint{2.104889in}{3.839178in}}%
\pgfpathquadraticcurveto{\pgfqpoint{2.077111in}{3.839178in}}{\pgfqpoint{2.077111in}{3.811400in}}%
\pgfpathlineto{\pgfqpoint{2.077111in}{3.438067in}}%
\pgfpathquadraticcurveto{\pgfqpoint{2.077111in}{3.410289in}}{\pgfqpoint{2.104889in}{3.410289in}}%
\pgfpathclose%
\pgfusepath{stroke,fill}%
\end{pgfscope}%
\begin{pgfscope}%
\pgfsetrectcap%
\pgfsetroundjoin%
\pgfsetlinewidth{1.505625pt}%
\definecolor{currentstroke}{rgb}{0.121569,0.466667,0.705882}%
\pgfsetstrokecolor{currentstroke}%
\pgfsetdash{}{0pt}%
\pgfpathmoveto{\pgfqpoint{2.132667in}{3.735011in}}%
\pgfpathlineto{\pgfqpoint{2.410444in}{3.735011in}}%
\pgfusepath{stroke}%
\end{pgfscope}%
\begin{pgfscope}%
\definecolor{textcolor}{rgb}{0.000000,0.000000,0.000000}%
\pgfsetstrokecolor{textcolor}%
\pgfsetfillcolor{textcolor}%
\pgftext[x=2.521555in,y=3.686400in,left,base]{\color{textcolor}\rmfamily\fontsize{10.000000}{12.000000}\selectfont Ground truth}%
\end{pgfscope}%
\begin{pgfscope}%
\pgfsetbuttcap%
\pgfsetroundjoin%
\definecolor{currentfill}{rgb}{0.121569,0.466667,0.705882}%
\pgfsetfillcolor{currentfill}%
\pgfsetlinewidth{1.003750pt}%
\definecolor{currentstroke}{rgb}{0.121569,0.466667,0.705882}%
\pgfsetstrokecolor{currentstroke}%
\pgfsetdash{}{0pt}%
\pgfsys@defobject{currentmarker}{\pgfqpoint{-0.031056in}{-0.031056in}}{\pgfqpoint{0.031056in}{0.031056in}}{%
\pgfpathmoveto{\pgfqpoint{0.000000in}{-0.031056in}}%
\pgfpathcurveto{\pgfqpoint{0.008236in}{-0.031056in}}{\pgfqpoint{0.016136in}{-0.027784in}}{\pgfqpoint{0.021960in}{-0.021960in}}%
\pgfpathcurveto{\pgfqpoint{0.027784in}{-0.016136in}}{\pgfqpoint{0.031056in}{-0.008236in}}{\pgfqpoint{0.031056in}{0.000000in}}%
\pgfpathcurveto{\pgfqpoint{0.031056in}{0.008236in}}{\pgfqpoint{0.027784in}{0.016136in}}{\pgfqpoint{0.021960in}{0.021960in}}%
\pgfpathcurveto{\pgfqpoint{0.016136in}{0.027784in}}{\pgfqpoint{0.008236in}{0.031056in}}{\pgfqpoint{0.000000in}{0.031056in}}%
\pgfpathcurveto{\pgfqpoint{-0.008236in}{0.031056in}}{\pgfqpoint{-0.016136in}{0.027784in}}{\pgfqpoint{-0.021960in}{0.021960in}}%
\pgfpathcurveto{\pgfqpoint{-0.027784in}{0.016136in}}{\pgfqpoint{-0.031056in}{0.008236in}}{\pgfqpoint{-0.031056in}{0.000000in}}%
\pgfpathcurveto{\pgfqpoint{-0.031056in}{-0.008236in}}{\pgfqpoint{-0.027784in}{-0.016136in}}{\pgfqpoint{-0.021960in}{-0.021960in}}%
\pgfpathcurveto{\pgfqpoint{-0.016136in}{-0.027784in}}{\pgfqpoint{-0.008236in}{-0.031056in}}{\pgfqpoint{0.000000in}{-0.031056in}}%
\pgfpathclose%
\pgfusepath{stroke,fill}%
}%
\begin{pgfscope}%
\pgfsys@transformshift{2.271555in}{3.529248in}%
\pgfsys@useobject{currentmarker}{}%
\end{pgfscope}%
\end{pgfscope}%
\begin{pgfscope}%
\definecolor{textcolor}{rgb}{0.000000,0.000000,0.000000}%
\pgfsetstrokecolor{textcolor}%
\pgfsetfillcolor{textcolor}%
\pgftext[x=2.521555in,y=3.492789in,left,base]{\color{textcolor}\rmfamily\fontsize{10.000000}{12.000000}\selectfont Estimated position}%
\end{pgfscope}%
\end{pgfpicture}%
\makeatother%
\endgroup%
}
%         \caption{MPU-9250 Breakout}
%         \label{fig:triangle4_3D}
%     \end{subfigure}
%     \caption{Position estimation by the best performing algorithms in the 4-meter line experiment.}
%     \label{fig:triangle4}
% \end{figure}

% \subsubsection{16 meter}

% For the 16-meter line experiment, the Mahony algorithm which had the lowest displacement error with an average of 0.48 meters (16\% of error margin), and ROLEQ with an average of 0.24 meters of turn error (7\% of error margin).

% \begin{figure}[!h]
%     \centering
%     \begin{table}[H]
    \begin{center}
        \resizebox{1\linewidth}{!}{
            \begin{tabular}[t]{lcccc}
                \hline
                Algorithm   & Displacement Error[$m$] & Displacement Error[\%] & Turn Error[$m$] & Turn Error[\%] \\
                \hline
                AngularRate & 12.53                   & 26.10                  & 19.26           & 40.12          \\            AQUA            & 7.48  & 15.58 & 12.02 & 25.04              \\            Complementary            & 8.07  & 16.82 & 9.07 & 18.89              \\            Davenport            & 5.43  & 11.31 & 5.31 & 11.07              \\            EKF            & 1.96  & 4.09 & 1.94 & 4.04              \\            FAMC            & 11.55  & 24.05 & 19.62 & 40.87              \\            FLAE            & 5.41  & 11.27 & 5.17 & 10.76              \\            Fourati            & 10.51  & 21.90 & 19.72 & 41.09              \\            Madgwick            & 3.70  & 7.71 & 7.11 & 14.81              \\            Mahony            & 1.62  & 3.37 & 2.39 & 4.98              \\            OLEQ            & 1.66  & 3.45 & 2.36 & 4.92              \\            QUEST            & 8.39  & 17.48 & 18.56 & 38.66              \\            ROLEQ            & 1.79  & 3.73 & 2.93 & 6.11              \\            SAAM            & 5.76  & 12.01 & 4.60 & 9.59              \\            Tilt            & 5.76  & 12.01 & 4.60 & 9.59              \\
                \hline
                Average     & 6.11                    & 12.72                  & 8.98            & 18.70
            \end{tabular}
        }
        \caption{Accelerometer Specifications. }
        \label{tab:accelerometer_specification}
    \end{center}
\end{table}
% \end{figure}

% \begin{figure}[!h]
%     \centering
%     \begin{subfigure}{0.49\textwidth}
%         \centering
%         \resizebox{1\linewidth}{!}{%% Creator: Matplotlib, PGF backend
%%
%% To include the figure in your LaTeX document, write
%%   \input{<filename>.pgf}
%%
%% Make sure the required packages are loaded in your preamble
%%   \usepackage{pgf}
%%
%% and, on pdftex
%%   \usepackage[utf8]{inputenc}\DeclareUnicodeCharacter{2212}{-}
%%
%% or, on luatex and xetex
%%   \usepackage{unicode-math}
%%
%% Figures using additional raster images can only be included by \input if
%% they are in the same directory as the main LaTeX file. For loading figures
%% from other directories you can use the `import` package
%%   \usepackage{import}
%%
%% and then include the figures with
%%   \import{<path to file>}{<filename>.pgf}
%%
%% Matplotlib used the following preamble
%%   \usepackage{fontspec}
%%
\begingroup%
\makeatletter%
\begin{pgfpicture}%
\pgfpathrectangle{\pgfpointorigin}{\pgfqpoint{4.342355in}{4.207622in}}%
\pgfusepath{use as bounding box, clip}%
\begin{pgfscope}%
\pgfsetbuttcap%
\pgfsetmiterjoin%
\definecolor{currentfill}{rgb}{1.000000,1.000000,1.000000}%
\pgfsetfillcolor{currentfill}%
\pgfsetlinewidth{0.000000pt}%
\definecolor{currentstroke}{rgb}{1.000000,1.000000,1.000000}%
\pgfsetstrokecolor{currentstroke}%
\pgfsetdash{}{0pt}%
\pgfpathmoveto{\pgfqpoint{0.000000in}{0.000000in}}%
\pgfpathlineto{\pgfqpoint{4.342355in}{0.000000in}}%
\pgfpathlineto{\pgfqpoint{4.342355in}{4.207622in}}%
\pgfpathlineto{\pgfqpoint{0.000000in}{4.207622in}}%
\pgfpathclose%
\pgfusepath{fill}%
\end{pgfscope}%
\begin{pgfscope}%
\pgfsetbuttcap%
\pgfsetmiterjoin%
\definecolor{currentfill}{rgb}{1.000000,1.000000,1.000000}%
\pgfsetfillcolor{currentfill}%
\pgfsetlinewidth{0.000000pt}%
\definecolor{currentstroke}{rgb}{0.000000,0.000000,0.000000}%
\pgfsetstrokecolor{currentstroke}%
\pgfsetstrokeopacity{0.000000}%
\pgfsetdash{}{0pt}%
\pgfpathmoveto{\pgfqpoint{0.100000in}{0.212622in}}%
\pgfpathlineto{\pgfqpoint{3.796000in}{0.212622in}}%
\pgfpathlineto{\pgfqpoint{3.796000in}{3.908622in}}%
\pgfpathlineto{\pgfqpoint{0.100000in}{3.908622in}}%
\pgfpathclose%
\pgfusepath{fill}%
\end{pgfscope}%
\begin{pgfscope}%
\pgfsetbuttcap%
\pgfsetmiterjoin%
\definecolor{currentfill}{rgb}{0.950000,0.950000,0.950000}%
\pgfsetfillcolor{currentfill}%
\pgfsetfillopacity{0.500000}%
\pgfsetlinewidth{1.003750pt}%
\definecolor{currentstroke}{rgb}{0.950000,0.950000,0.950000}%
\pgfsetstrokecolor{currentstroke}%
\pgfsetstrokeopacity{0.500000}%
\pgfsetdash{}{0pt}%
\pgfpathmoveto{\pgfqpoint{0.379073in}{1.123938in}}%
\pgfpathlineto{\pgfqpoint{1.599613in}{2.147018in}}%
\pgfpathlineto{\pgfqpoint{1.582647in}{3.622484in}}%
\pgfpathlineto{\pgfqpoint{0.303698in}{2.689165in}}%
\pgfusepath{stroke,fill}%
\end{pgfscope}%
\begin{pgfscope}%
\pgfsetbuttcap%
\pgfsetmiterjoin%
\definecolor{currentfill}{rgb}{0.900000,0.900000,0.900000}%
\pgfsetfillcolor{currentfill}%
\pgfsetfillopacity{0.500000}%
\pgfsetlinewidth{1.003750pt}%
\definecolor{currentstroke}{rgb}{0.900000,0.900000,0.900000}%
\pgfsetstrokecolor{currentstroke}%
\pgfsetstrokeopacity{0.500000}%
\pgfsetdash{}{0pt}%
\pgfpathmoveto{\pgfqpoint{1.599613in}{2.147018in}}%
\pgfpathlineto{\pgfqpoint{3.558144in}{1.577751in}}%
\pgfpathlineto{\pgfqpoint{3.628038in}{3.104037in}}%
\pgfpathlineto{\pgfqpoint{1.582647in}{3.622484in}}%
\pgfusepath{stroke,fill}%
\end{pgfscope}%
\begin{pgfscope}%
\pgfsetbuttcap%
\pgfsetmiterjoin%
\definecolor{currentfill}{rgb}{0.925000,0.925000,0.925000}%
\pgfsetfillcolor{currentfill}%
\pgfsetfillopacity{0.500000}%
\pgfsetlinewidth{1.003750pt}%
\definecolor{currentstroke}{rgb}{0.925000,0.925000,0.925000}%
\pgfsetstrokecolor{currentstroke}%
\pgfsetstrokeopacity{0.500000}%
\pgfsetdash{}{0pt}%
\pgfpathmoveto{\pgfqpoint{0.379073in}{1.123938in}}%
\pgfpathlineto{\pgfqpoint{2.455212in}{0.445871in}}%
\pgfpathlineto{\pgfqpoint{3.558144in}{1.577751in}}%
\pgfpathlineto{\pgfqpoint{1.599613in}{2.147018in}}%
\pgfusepath{stroke,fill}%
\end{pgfscope}%
\begin{pgfscope}%
\pgfsetrectcap%
\pgfsetroundjoin%
\pgfsetlinewidth{0.803000pt}%
\definecolor{currentstroke}{rgb}{0.000000,0.000000,0.000000}%
\pgfsetstrokecolor{currentstroke}%
\pgfsetdash{}{0pt}%
\pgfpathmoveto{\pgfqpoint{0.379073in}{1.123938in}}%
\pgfpathlineto{\pgfqpoint{2.455212in}{0.445871in}}%
\pgfusepath{stroke}%
\end{pgfscope}%
\begin{pgfscope}%
\definecolor{textcolor}{rgb}{0.000000,0.000000,0.000000}%
\pgfsetstrokecolor{textcolor}%
\pgfsetfillcolor{textcolor}%
\pgftext[x=0.730374in, y=0.408886in, left, base,rotate=341.912962]{\color{textcolor}\rmfamily\fontsize{10.000000}{12.000000}\selectfont Position X [\(\displaystyle m\)]}%
\end{pgfscope}%
\begin{pgfscope}%
\pgfsetbuttcap%
\pgfsetroundjoin%
\pgfsetlinewidth{0.803000pt}%
\definecolor{currentstroke}{rgb}{0.690196,0.690196,0.690196}%
\pgfsetstrokecolor{currentstroke}%
\pgfsetdash{}{0pt}%
\pgfpathmoveto{\pgfqpoint{0.688752in}{1.022797in}}%
\pgfpathlineto{\pgfqpoint{1.892848in}{2.061786in}}%
\pgfpathlineto{\pgfqpoint{1.888337in}{3.545000in}}%
\pgfusepath{stroke}%
\end{pgfscope}%
\begin{pgfscope}%
\pgfsetbuttcap%
\pgfsetroundjoin%
\pgfsetlinewidth{0.803000pt}%
\definecolor{currentstroke}{rgb}{0.690196,0.690196,0.690196}%
\pgfsetstrokecolor{currentstroke}%
\pgfsetdash{}{0pt}%
\pgfpathmoveto{\pgfqpoint{1.113001in}{0.884237in}}%
\pgfpathlineto{\pgfqpoint{2.293944in}{1.945204in}}%
\pgfpathlineto{\pgfqpoint{2.306782in}{3.438936in}}%
\pgfusepath{stroke}%
\end{pgfscope}%
\begin{pgfscope}%
\pgfsetbuttcap%
\pgfsetroundjoin%
\pgfsetlinewidth{0.803000pt}%
\definecolor{currentstroke}{rgb}{0.690196,0.690196,0.690196}%
\pgfsetstrokecolor{currentstroke}%
\pgfsetdash{}{0pt}%
\pgfpathmoveto{\pgfqpoint{1.546668in}{0.742601in}}%
\pgfpathlineto{\pgfqpoint{2.703195in}{1.826251in}}%
\pgfpathlineto{\pgfqpoint{2.734109in}{3.330622in}}%
\pgfusepath{stroke}%
\end{pgfscope}%
\begin{pgfscope}%
\pgfsetbuttcap%
\pgfsetroundjoin%
\pgfsetlinewidth{0.803000pt}%
\definecolor{currentstroke}{rgb}{0.690196,0.690196,0.690196}%
\pgfsetstrokecolor{currentstroke}%
\pgfsetdash{}{0pt}%
\pgfpathmoveto{\pgfqpoint{1.990070in}{0.597786in}}%
\pgfpathlineto{\pgfqpoint{3.120853in}{1.704854in}}%
\pgfpathlineto{\pgfqpoint{3.170603in}{3.219983in}}%
\pgfusepath{stroke}%
\end{pgfscope}%
\begin{pgfscope}%
\pgfsetrectcap%
\pgfsetroundjoin%
\pgfsetlinewidth{0.803000pt}%
\definecolor{currentstroke}{rgb}{0.000000,0.000000,0.000000}%
\pgfsetstrokecolor{currentstroke}%
\pgfsetdash{}{0pt}%
\pgfpathmoveto{\pgfqpoint{0.699241in}{1.031848in}}%
\pgfpathlineto{\pgfqpoint{0.667728in}{1.004656in}}%
\pgfusepath{stroke}%
\end{pgfscope}%
\begin{pgfscope}%
\definecolor{textcolor}{rgb}{0.000000,0.000000,0.000000}%
\pgfsetstrokecolor{textcolor}%
\pgfsetfillcolor{textcolor}%
\pgftext[x=0.584367in,y=0.803321in,,top]{\color{textcolor}\rmfamily\fontsize{10.000000}{12.000000}\selectfont \(\displaystyle {0}\)}%
\end{pgfscope}%
\begin{pgfscope}%
\pgfsetrectcap%
\pgfsetroundjoin%
\pgfsetlinewidth{0.803000pt}%
\definecolor{currentstroke}{rgb}{0.000000,0.000000,0.000000}%
\pgfsetstrokecolor{currentstroke}%
\pgfsetdash{}{0pt}%
\pgfpathmoveto{\pgfqpoint{1.123298in}{0.893487in}}%
\pgfpathlineto{\pgfqpoint{1.092363in}{0.865695in}}%
\pgfusepath{stroke}%
\end{pgfscope}%
\begin{pgfscope}%
\definecolor{textcolor}{rgb}{0.000000,0.000000,0.000000}%
\pgfsetstrokecolor{textcolor}%
\pgfsetfillcolor{textcolor}%
\pgftext[x=1.009068in,y=0.661819in,,top]{\color{textcolor}\rmfamily\fontsize{10.000000}{12.000000}\selectfont \(\displaystyle {5}\)}%
\end{pgfscope}%
\begin{pgfscope}%
\pgfsetrectcap%
\pgfsetroundjoin%
\pgfsetlinewidth{0.803000pt}%
\definecolor{currentstroke}{rgb}{0.000000,0.000000,0.000000}%
\pgfsetstrokecolor{currentstroke}%
\pgfsetdash{}{0pt}%
\pgfpathmoveto{\pgfqpoint{1.556761in}{0.752058in}}%
\pgfpathlineto{\pgfqpoint{1.526438in}{0.723645in}}%
\pgfusepath{stroke}%
\end{pgfscope}%
\begin{pgfscope}%
\definecolor{textcolor}{rgb}{0.000000,0.000000,0.000000}%
\pgfsetstrokecolor{textcolor}%
\pgfsetfillcolor{textcolor}%
\pgftext[x=1.443233in,y=0.517165in,,top]{\color{textcolor}\rmfamily\fontsize{10.000000}{12.000000}\selectfont \(\displaystyle {10}\)}%
\end{pgfscope}%
\begin{pgfscope}%
\pgfsetrectcap%
\pgfsetroundjoin%
\pgfsetlinewidth{0.803000pt}%
\definecolor{currentstroke}{rgb}{0.000000,0.000000,0.000000}%
\pgfsetstrokecolor{currentstroke}%
\pgfsetdash{}{0pt}%
\pgfpathmoveto{\pgfqpoint{1.999948in}{0.607457in}}%
\pgfpathlineto{\pgfqpoint{1.970271in}{0.578402in}}%
\pgfusepath{stroke}%
\end{pgfscope}%
\begin{pgfscope}%
\definecolor{textcolor}{rgb}{0.000000,0.000000,0.000000}%
\pgfsetstrokecolor{textcolor}%
\pgfsetfillcolor{textcolor}%
\pgftext[x=1.887182in,y=0.369250in,,top]{\color{textcolor}\rmfamily\fontsize{10.000000}{12.000000}\selectfont \(\displaystyle {15}\)}%
\end{pgfscope}%
\begin{pgfscope}%
\pgfsetrectcap%
\pgfsetroundjoin%
\pgfsetlinewidth{0.803000pt}%
\definecolor{currentstroke}{rgb}{0.000000,0.000000,0.000000}%
\pgfsetstrokecolor{currentstroke}%
\pgfsetdash{}{0pt}%
\pgfpathmoveto{\pgfqpoint{3.558144in}{1.577751in}}%
\pgfpathlineto{\pgfqpoint{2.455212in}{0.445871in}}%
\pgfusepath{stroke}%
\end{pgfscope}%
\begin{pgfscope}%
\definecolor{textcolor}{rgb}{0.000000,0.000000,0.000000}%
\pgfsetstrokecolor{textcolor}%
\pgfsetfillcolor{textcolor}%
\pgftext[x=3.120747in, y=0.305657in, left, base,rotate=45.742112]{\color{textcolor}\rmfamily\fontsize{10.000000}{12.000000}\selectfont Position Y [\(\displaystyle m\)]}%
\end{pgfscope}%
\begin{pgfscope}%
\pgfsetbuttcap%
\pgfsetroundjoin%
\pgfsetlinewidth{0.803000pt}%
\definecolor{currentstroke}{rgb}{0.690196,0.690196,0.690196}%
\pgfsetstrokecolor{currentstroke}%
\pgfsetdash{}{0pt}%
\pgfpathmoveto{\pgfqpoint{0.519825in}{2.846885in}}%
\pgfpathlineto{\pgfqpoint{0.584681in}{1.296282in}}%
\pgfpathlineto{\pgfqpoint{2.641691in}{0.637244in}}%
\pgfusepath{stroke}%
\end{pgfscope}%
\begin{pgfscope}%
\pgfsetbuttcap%
\pgfsetroundjoin%
\pgfsetlinewidth{0.803000pt}%
\definecolor{currentstroke}{rgb}{0.690196,0.690196,0.690196}%
\pgfsetstrokecolor{currentstroke}%
\pgfsetdash{}{0pt}%
\pgfpathmoveto{\pgfqpoint{0.825695in}{3.070095in}}%
\pgfpathlineto{\pgfqpoint{0.876112in}{1.540565in}}%
\pgfpathlineto{\pgfqpoint{2.905534in}{0.908011in}}%
\pgfusepath{stroke}%
\end{pgfscope}%
\begin{pgfscope}%
\pgfsetbuttcap%
\pgfsetroundjoin%
\pgfsetlinewidth{0.803000pt}%
\definecolor{currentstroke}{rgb}{0.690196,0.690196,0.690196}%
\pgfsetstrokecolor{currentstroke}%
\pgfsetdash{}{0pt}%
\pgfpathmoveto{\pgfqpoint{1.119175in}{3.284263in}}%
\pgfpathlineto{\pgfqpoint{1.156237in}{1.775372in}}%
\pgfpathlineto{\pgfqpoint{3.158616in}{1.167736in}}%
\pgfusepath{stroke}%
\end{pgfscope}%
\begin{pgfscope}%
\pgfsetbuttcap%
\pgfsetroundjoin%
\pgfsetlinewidth{0.803000pt}%
\definecolor{currentstroke}{rgb}{0.690196,0.690196,0.690196}%
\pgfsetstrokecolor{currentstroke}%
\pgfsetdash{}{0pt}%
\pgfpathmoveto{\pgfqpoint{1.401004in}{3.489929in}}%
\pgfpathlineto{\pgfqpoint{1.425701in}{2.001242in}}%
\pgfpathlineto{\pgfqpoint{3.401584in}{1.417081in}}%
\pgfusepath{stroke}%
\end{pgfscope}%
\begin{pgfscope}%
\pgfsetrectcap%
\pgfsetroundjoin%
\pgfsetlinewidth{0.803000pt}%
\definecolor{currentstroke}{rgb}{0.000000,0.000000,0.000000}%
\pgfsetstrokecolor{currentstroke}%
\pgfsetdash{}{0pt}%
\pgfpathmoveto{\pgfqpoint{2.624364in}{0.642795in}}%
\pgfpathlineto{\pgfqpoint{2.676389in}{0.626127in}}%
\pgfusepath{stroke}%
\end{pgfscope}%
\begin{pgfscope}%
\definecolor{textcolor}{rgb}{0.000000,0.000000,0.000000}%
\pgfsetstrokecolor{textcolor}%
\pgfsetfillcolor{textcolor}%
\pgftext[x=2.819111in,y=0.452484in,,top]{\color{textcolor}\rmfamily\fontsize{10.000000}{12.000000}\selectfont \(\displaystyle {0}\)}%
\end{pgfscope}%
\begin{pgfscope}%
\pgfsetrectcap%
\pgfsetroundjoin%
\pgfsetlinewidth{0.803000pt}%
\definecolor{currentstroke}{rgb}{0.000000,0.000000,0.000000}%
\pgfsetstrokecolor{currentstroke}%
\pgfsetdash{}{0pt}%
\pgfpathmoveto{\pgfqpoint{2.888457in}{0.913334in}}%
\pgfpathlineto{\pgfqpoint{2.939730in}{0.897353in}}%
\pgfusepath{stroke}%
\end{pgfscope}%
\begin{pgfscope}%
\definecolor{textcolor}{rgb}{0.000000,0.000000,0.000000}%
\pgfsetstrokecolor{textcolor}%
\pgfsetfillcolor{textcolor}%
\pgftext[x=3.079412in,y=0.727257in,,top]{\color{textcolor}\rmfamily\fontsize{10.000000}{12.000000}\selectfont \(\displaystyle {5}\)}%
\end{pgfscope}%
\begin{pgfscope}%
\pgfsetrectcap%
\pgfsetroundjoin%
\pgfsetlinewidth{0.803000pt}%
\definecolor{currentstroke}{rgb}{0.000000,0.000000,0.000000}%
\pgfsetstrokecolor{currentstroke}%
\pgfsetdash{}{0pt}%
\pgfpathmoveto{\pgfqpoint{3.141784in}{1.172844in}}%
\pgfpathlineto{\pgfqpoint{3.192322in}{1.157508in}}%
\pgfusepath{stroke}%
\end{pgfscope}%
\begin{pgfscope}%
\definecolor{textcolor}{rgb}{0.000000,0.000000,0.000000}%
\pgfsetstrokecolor{textcolor}%
\pgfsetfillcolor{textcolor}%
\pgftext[x=3.329091in,y=0.990818in,,top]{\color{textcolor}\rmfamily\fontsize{10.000000}{12.000000}\selectfont \(\displaystyle {10}\)}%
\end{pgfscope}%
\begin{pgfscope}%
\pgfsetrectcap%
\pgfsetroundjoin%
\pgfsetlinewidth{0.803000pt}%
\definecolor{currentstroke}{rgb}{0.000000,0.000000,0.000000}%
\pgfsetstrokecolor{currentstroke}%
\pgfsetdash{}{0pt}%
\pgfpathmoveto{\pgfqpoint{3.384990in}{1.421987in}}%
\pgfpathlineto{\pgfqpoint{3.434811in}{1.407258in}}%
\pgfusepath{stroke}%
\end{pgfscope}%
\begin{pgfscope}%
\definecolor{textcolor}{rgb}{0.000000,0.000000,0.000000}%
\pgfsetstrokecolor{textcolor}%
\pgfsetfillcolor{textcolor}%
\pgftext[x=3.568786in,y=1.243839in,,top]{\color{textcolor}\rmfamily\fontsize{10.000000}{12.000000}\selectfont \(\displaystyle {15}\)}%
\end{pgfscope}%
\begin{pgfscope}%
\pgfsetrectcap%
\pgfsetroundjoin%
\pgfsetlinewidth{0.803000pt}%
\definecolor{currentstroke}{rgb}{0.000000,0.000000,0.000000}%
\pgfsetstrokecolor{currentstroke}%
\pgfsetdash{}{0pt}%
\pgfpathmoveto{\pgfqpoint{3.558144in}{1.577751in}}%
\pgfpathlineto{\pgfqpoint{3.628038in}{3.104037in}}%
\pgfusepath{stroke}%
\end{pgfscope}%
\begin{pgfscope}%
\definecolor{textcolor}{rgb}{0.000000,0.000000,0.000000}%
\pgfsetstrokecolor{textcolor}%
\pgfsetfillcolor{textcolor}%
\pgftext[x=4.167903in, y=1.963517in, left, base,rotate=87.378092]{\color{textcolor}\rmfamily\fontsize{10.000000}{12.000000}\selectfont Position Z [\(\displaystyle m\)]}%
\end{pgfscope}%
\begin{pgfscope}%
\pgfsetbuttcap%
\pgfsetroundjoin%
\pgfsetlinewidth{0.803000pt}%
\definecolor{currentstroke}{rgb}{0.690196,0.690196,0.690196}%
\pgfsetstrokecolor{currentstroke}%
\pgfsetdash{}{0pt}%
\pgfpathmoveto{\pgfqpoint{3.568957in}{1.813878in}}%
\pgfpathlineto{\pgfqpoint{1.596984in}{2.375691in}}%
\pgfpathlineto{\pgfqpoint{0.367429in}{1.365745in}}%
\pgfusepath{stroke}%
\end{pgfscope}%
\begin{pgfscope}%
\pgfsetbuttcap%
\pgfsetroundjoin%
\pgfsetlinewidth{0.803000pt}%
\definecolor{currentstroke}{rgb}{0.690196,0.690196,0.690196}%
\pgfsetstrokecolor{currentstroke}%
\pgfsetdash{}{0pt}%
\pgfpathmoveto{\pgfqpoint{3.582824in}{2.116684in}}%
\pgfpathlineto{\pgfqpoint{1.593614in}{2.668718in}}%
\pgfpathlineto{\pgfqpoint{0.352487in}{1.676018in}}%
\pgfusepath{stroke}%
\end{pgfscope}%
\begin{pgfscope}%
\pgfsetbuttcap%
\pgfsetroundjoin%
\pgfsetlinewidth{0.803000pt}%
\definecolor{currentstroke}{rgb}{0.690196,0.690196,0.690196}%
\pgfsetstrokecolor{currentstroke}%
\pgfsetdash{}{0pt}%
\pgfpathmoveto{\pgfqpoint{3.596937in}{2.424882in}}%
\pgfpathlineto{\pgfqpoint{1.590188in}{2.966712in}}%
\pgfpathlineto{\pgfqpoint{0.337269in}{1.992031in}}%
\pgfusepath{stroke}%
\end{pgfscope}%
\begin{pgfscope}%
\pgfsetbuttcap%
\pgfsetroundjoin%
\pgfsetlinewidth{0.803000pt}%
\definecolor{currentstroke}{rgb}{0.690196,0.690196,0.690196}%
\pgfsetstrokecolor{currentstroke}%
\pgfsetdash{}{0pt}%
\pgfpathmoveto{\pgfqpoint{3.611304in}{2.738619in}}%
\pgfpathlineto{\pgfqpoint{1.586702in}{3.269800in}}%
\pgfpathlineto{\pgfqpoint{0.321767in}{2.313943in}}%
\pgfusepath{stroke}%
\end{pgfscope}%
\begin{pgfscope}%
\pgfsetbuttcap%
\pgfsetroundjoin%
\pgfsetlinewidth{0.803000pt}%
\definecolor{currentstroke}{rgb}{0.690196,0.690196,0.690196}%
\pgfsetstrokecolor{currentstroke}%
\pgfsetdash{}{0pt}%
\pgfpathmoveto{\pgfqpoint{3.625931in}{3.058045in}}%
\pgfpathlineto{\pgfqpoint{1.583157in}{3.578114in}}%
\pgfpathlineto{\pgfqpoint{0.305973in}{2.641922in}}%
\pgfusepath{stroke}%
\end{pgfscope}%
\begin{pgfscope}%
\pgfsetrectcap%
\pgfsetroundjoin%
\pgfsetlinewidth{0.803000pt}%
\definecolor{currentstroke}{rgb}{0.000000,0.000000,0.000000}%
\pgfsetstrokecolor{currentstroke}%
\pgfsetdash{}{0pt}%
\pgfpathmoveto{\pgfqpoint{3.552402in}{1.818595in}}%
\pgfpathlineto{\pgfqpoint{3.602108in}{1.804434in}}%
\pgfusepath{stroke}%
\end{pgfscope}%
\begin{pgfscope}%
\definecolor{textcolor}{rgb}{0.000000,0.000000,0.000000}%
\pgfsetstrokecolor{textcolor}%
\pgfsetfillcolor{textcolor}%
\pgftext[x=3.824129in,y=1.849596in,,top]{\color{textcolor}\rmfamily\fontsize{10.000000}{12.000000}\selectfont \(\displaystyle {0}\)}%
\end{pgfscope}%
\begin{pgfscope}%
\pgfsetrectcap%
\pgfsetroundjoin%
\pgfsetlinewidth{0.803000pt}%
\definecolor{currentstroke}{rgb}{0.000000,0.000000,0.000000}%
\pgfsetstrokecolor{currentstroke}%
\pgfsetdash{}{0pt}%
\pgfpathmoveto{\pgfqpoint{3.566116in}{2.121320in}}%
\pgfpathlineto{\pgfqpoint{3.616279in}{2.107400in}}%
\pgfusepath{stroke}%
\end{pgfscope}%
\begin{pgfscope}%
\definecolor{textcolor}{rgb}{0.000000,0.000000,0.000000}%
\pgfsetstrokecolor{textcolor}%
\pgfsetfillcolor{textcolor}%
\pgftext[x=3.840199in,y=2.151794in,,top]{\color{textcolor}\rmfamily\fontsize{10.000000}{12.000000}\selectfont \(\displaystyle {1}\)}%
\end{pgfscope}%
\begin{pgfscope}%
\pgfsetrectcap%
\pgfsetroundjoin%
\pgfsetlinewidth{0.803000pt}%
\definecolor{currentstroke}{rgb}{0.000000,0.000000,0.000000}%
\pgfsetstrokecolor{currentstroke}%
\pgfsetdash{}{0pt}%
\pgfpathmoveto{\pgfqpoint{3.580075in}{2.429435in}}%
\pgfpathlineto{\pgfqpoint{3.630702in}{2.415766in}}%
\pgfusepath{stroke}%
\end{pgfscope}%
\begin{pgfscope}%
\definecolor{textcolor}{rgb}{0.000000,0.000000,0.000000}%
\pgfsetstrokecolor{textcolor}%
\pgfsetfillcolor{textcolor}%
\pgftext[x=3.856554in,y=2.459357in,,top]{\color{textcolor}\rmfamily\fontsize{10.000000}{12.000000}\selectfont \(\displaystyle {2}\)}%
\end{pgfscope}%
\begin{pgfscope}%
\pgfsetrectcap%
\pgfsetroundjoin%
\pgfsetlinewidth{0.803000pt}%
\definecolor{currentstroke}{rgb}{0.000000,0.000000,0.000000}%
\pgfsetstrokecolor{currentstroke}%
\pgfsetdash{}{0pt}%
\pgfpathmoveto{\pgfqpoint{3.594285in}{2.743084in}}%
\pgfpathlineto{\pgfqpoint{3.645384in}{2.729678in}}%
\pgfusepath{stroke}%
\end{pgfscope}%
\begin{pgfscope}%
\definecolor{textcolor}{rgb}{0.000000,0.000000,0.000000}%
\pgfsetstrokecolor{textcolor}%
\pgfsetfillcolor{textcolor}%
\pgftext[x=3.873201in,y=2.772430in,,top]{\color{textcolor}\rmfamily\fontsize{10.000000}{12.000000}\selectfont \(\displaystyle {3}\)}%
\end{pgfscope}%
\begin{pgfscope}%
\pgfsetrectcap%
\pgfsetroundjoin%
\pgfsetlinewidth{0.803000pt}%
\definecolor{currentstroke}{rgb}{0.000000,0.000000,0.000000}%
\pgfsetstrokecolor{currentstroke}%
\pgfsetdash{}{0pt}%
\pgfpathmoveto{\pgfqpoint{3.608752in}{3.062418in}}%
\pgfpathlineto{\pgfqpoint{3.660333in}{3.049286in}}%
\pgfusepath{stroke}%
\end{pgfscope}%
\begin{pgfscope}%
\definecolor{textcolor}{rgb}{0.000000,0.000000,0.000000}%
\pgfsetstrokecolor{textcolor}%
\pgfsetfillcolor{textcolor}%
\pgftext[x=3.890150in,y=3.091162in,,top]{\color{textcolor}\rmfamily\fontsize{10.000000}{12.000000}\selectfont \(\displaystyle {4}\)}%
\end{pgfscope}%
\begin{pgfscope}%
\pgfpathrectangle{\pgfqpoint{0.100000in}{0.212622in}}{\pgfqpoint{3.696000in}{3.696000in}}%
\pgfusepath{clip}%
\pgfsetrectcap%
\pgfsetroundjoin%
\pgfsetlinewidth{1.505625pt}%
\definecolor{currentstroke}{rgb}{0.121569,0.466667,0.705882}%
\pgfsetstrokecolor{currentstroke}%
\pgfsetdash{}{0pt}%
\pgfpathmoveto{\pgfqpoint{0.883791in}{1.438814in}}%
\pgfpathlineto{\pgfqpoint{1.771800in}{2.189817in}}%
\pgfpathlineto{\pgfqpoint{2.272126in}{1.002530in}}%
\pgfpathlineto{\pgfqpoint{0.883791in}{1.438814in}}%
\pgfusepath{stroke}%
\end{pgfscope}%
\begin{pgfscope}%
\pgfpathrectangle{\pgfqpoint{0.100000in}{0.212622in}}{\pgfqpoint{3.696000in}{3.696000in}}%
\pgfusepath{clip}%
\pgfsetrectcap%
\pgfsetroundjoin%
\pgfsetlinewidth{1.505625pt}%
\definecolor{currentstroke}{rgb}{1.000000,0.000000,0.000000}%
\pgfsetstrokecolor{currentstroke}%
\pgfsetdash{}{0pt}%
\pgfpathmoveto{\pgfqpoint{0.883216in}{1.438396in}}%
\pgfpathlineto{\pgfqpoint{0.883791in}{1.438814in}}%
\pgfusepath{stroke}%
\end{pgfscope}%
\begin{pgfscope}%
\pgfpathrectangle{\pgfqpoint{0.100000in}{0.212622in}}{\pgfqpoint{3.696000in}{3.696000in}}%
\pgfusepath{clip}%
\pgfsetrectcap%
\pgfsetroundjoin%
\pgfsetlinewidth{1.505625pt}%
\definecolor{currentstroke}{rgb}{1.000000,0.000000,0.000000}%
\pgfsetstrokecolor{currentstroke}%
\pgfsetdash{}{0pt}%
\pgfpathmoveto{\pgfqpoint{0.883030in}{1.438590in}}%
\pgfpathlineto{\pgfqpoint{0.883791in}{1.438814in}}%
\pgfusepath{stroke}%
\end{pgfscope}%
\begin{pgfscope}%
\pgfpathrectangle{\pgfqpoint{0.100000in}{0.212622in}}{\pgfqpoint{3.696000in}{3.696000in}}%
\pgfusepath{clip}%
\pgfsetrectcap%
\pgfsetroundjoin%
\pgfsetlinewidth{1.505625pt}%
\definecolor{currentstroke}{rgb}{1.000000,0.000000,0.000000}%
\pgfsetstrokecolor{currentstroke}%
\pgfsetdash{}{0pt}%
\pgfpathmoveto{\pgfqpoint{0.882371in}{1.438798in}}%
\pgfpathlineto{\pgfqpoint{0.883791in}{1.438814in}}%
\pgfusepath{stroke}%
\end{pgfscope}%
\begin{pgfscope}%
\pgfpathrectangle{\pgfqpoint{0.100000in}{0.212622in}}{\pgfqpoint{3.696000in}{3.696000in}}%
\pgfusepath{clip}%
\pgfsetrectcap%
\pgfsetroundjoin%
\pgfsetlinewidth{1.505625pt}%
\definecolor{currentstroke}{rgb}{1.000000,0.000000,0.000000}%
\pgfsetstrokecolor{currentstroke}%
\pgfsetdash{}{0pt}%
\pgfpathmoveto{\pgfqpoint{0.881305in}{1.439235in}}%
\pgfpathlineto{\pgfqpoint{0.883791in}{1.438814in}}%
\pgfusepath{stroke}%
\end{pgfscope}%
\begin{pgfscope}%
\pgfpathrectangle{\pgfqpoint{0.100000in}{0.212622in}}{\pgfqpoint{3.696000in}{3.696000in}}%
\pgfusepath{clip}%
\pgfsetrectcap%
\pgfsetroundjoin%
\pgfsetlinewidth{1.505625pt}%
\definecolor{currentstroke}{rgb}{1.000000,0.000000,0.000000}%
\pgfsetstrokecolor{currentstroke}%
\pgfsetdash{}{0pt}%
\pgfpathmoveto{\pgfqpoint{0.879009in}{1.440050in}}%
\pgfpathlineto{\pgfqpoint{0.883791in}{1.438814in}}%
\pgfusepath{stroke}%
\end{pgfscope}%
\begin{pgfscope}%
\pgfpathrectangle{\pgfqpoint{0.100000in}{0.212622in}}{\pgfqpoint{3.696000in}{3.696000in}}%
\pgfusepath{clip}%
\pgfsetrectcap%
\pgfsetroundjoin%
\pgfsetlinewidth{1.505625pt}%
\definecolor{currentstroke}{rgb}{1.000000,0.000000,0.000000}%
\pgfsetstrokecolor{currentstroke}%
\pgfsetdash{}{0pt}%
\pgfpathmoveto{\pgfqpoint{0.876546in}{1.441090in}}%
\pgfpathlineto{\pgfqpoint{0.883791in}{1.438814in}}%
\pgfusepath{stroke}%
\end{pgfscope}%
\begin{pgfscope}%
\pgfpathrectangle{\pgfqpoint{0.100000in}{0.212622in}}{\pgfqpoint{3.696000in}{3.696000in}}%
\pgfusepath{clip}%
\pgfsetrectcap%
\pgfsetroundjoin%
\pgfsetlinewidth{1.505625pt}%
\definecolor{currentstroke}{rgb}{1.000000,0.000000,0.000000}%
\pgfsetstrokecolor{currentstroke}%
\pgfsetdash{}{0pt}%
\pgfpathmoveto{\pgfqpoint{0.872935in}{1.442523in}}%
\pgfpathlineto{\pgfqpoint{0.883791in}{1.438814in}}%
\pgfusepath{stroke}%
\end{pgfscope}%
\begin{pgfscope}%
\pgfpathrectangle{\pgfqpoint{0.100000in}{0.212622in}}{\pgfqpoint{3.696000in}{3.696000in}}%
\pgfusepath{clip}%
\pgfsetrectcap%
\pgfsetroundjoin%
\pgfsetlinewidth{1.505625pt}%
\definecolor{currentstroke}{rgb}{1.000000,0.000000,0.000000}%
\pgfsetstrokecolor{currentstroke}%
\pgfsetdash{}{0pt}%
\pgfpathmoveto{\pgfqpoint{0.869345in}{1.443588in}}%
\pgfpathlineto{\pgfqpoint{0.883791in}{1.438814in}}%
\pgfusepath{stroke}%
\end{pgfscope}%
\begin{pgfscope}%
\pgfpathrectangle{\pgfqpoint{0.100000in}{0.212622in}}{\pgfqpoint{3.696000in}{3.696000in}}%
\pgfusepath{clip}%
\pgfsetrectcap%
\pgfsetroundjoin%
\pgfsetlinewidth{1.505625pt}%
\definecolor{currentstroke}{rgb}{1.000000,0.000000,0.000000}%
\pgfsetstrokecolor{currentstroke}%
\pgfsetdash{}{0pt}%
\pgfpathmoveto{\pgfqpoint{0.867054in}{1.444649in}}%
\pgfpathlineto{\pgfqpoint{0.883791in}{1.438814in}}%
\pgfusepath{stroke}%
\end{pgfscope}%
\begin{pgfscope}%
\pgfpathrectangle{\pgfqpoint{0.100000in}{0.212622in}}{\pgfqpoint{3.696000in}{3.696000in}}%
\pgfusepath{clip}%
\pgfsetrectcap%
\pgfsetroundjoin%
\pgfsetlinewidth{1.505625pt}%
\definecolor{currentstroke}{rgb}{1.000000,0.000000,0.000000}%
\pgfsetstrokecolor{currentstroke}%
\pgfsetdash{}{0pt}%
\pgfpathmoveto{\pgfqpoint{0.866012in}{1.445058in}}%
\pgfpathlineto{\pgfqpoint{0.883791in}{1.438814in}}%
\pgfusepath{stroke}%
\end{pgfscope}%
\begin{pgfscope}%
\pgfpathrectangle{\pgfqpoint{0.100000in}{0.212622in}}{\pgfqpoint{3.696000in}{3.696000in}}%
\pgfusepath{clip}%
\pgfsetrectcap%
\pgfsetroundjoin%
\pgfsetlinewidth{1.505625pt}%
\definecolor{currentstroke}{rgb}{1.000000,0.000000,0.000000}%
\pgfsetstrokecolor{currentstroke}%
\pgfsetdash{}{0pt}%
\pgfpathmoveto{\pgfqpoint{0.865352in}{1.445379in}}%
\pgfpathlineto{\pgfqpoint{0.883791in}{1.438814in}}%
\pgfusepath{stroke}%
\end{pgfscope}%
\begin{pgfscope}%
\pgfpathrectangle{\pgfqpoint{0.100000in}{0.212622in}}{\pgfqpoint{3.696000in}{3.696000in}}%
\pgfusepath{clip}%
\pgfsetrectcap%
\pgfsetroundjoin%
\pgfsetlinewidth{1.505625pt}%
\definecolor{currentstroke}{rgb}{1.000000,0.000000,0.000000}%
\pgfsetstrokecolor{currentstroke}%
\pgfsetdash{}{0pt}%
\pgfpathmoveto{\pgfqpoint{0.865009in}{1.445486in}}%
\pgfpathlineto{\pgfqpoint{0.883791in}{1.438814in}}%
\pgfusepath{stroke}%
\end{pgfscope}%
\begin{pgfscope}%
\pgfpathrectangle{\pgfqpoint{0.100000in}{0.212622in}}{\pgfqpoint{3.696000in}{3.696000in}}%
\pgfusepath{clip}%
\pgfsetrectcap%
\pgfsetroundjoin%
\pgfsetlinewidth{1.505625pt}%
\definecolor{currentstroke}{rgb}{1.000000,0.000000,0.000000}%
\pgfsetstrokecolor{currentstroke}%
\pgfsetdash{}{0pt}%
\pgfpathmoveto{\pgfqpoint{0.864820in}{1.445590in}}%
\pgfpathlineto{\pgfqpoint{0.883791in}{1.438814in}}%
\pgfusepath{stroke}%
\end{pgfscope}%
\begin{pgfscope}%
\pgfpathrectangle{\pgfqpoint{0.100000in}{0.212622in}}{\pgfqpoint{3.696000in}{3.696000in}}%
\pgfusepath{clip}%
\pgfsetrectcap%
\pgfsetroundjoin%
\pgfsetlinewidth{1.505625pt}%
\definecolor{currentstroke}{rgb}{1.000000,0.000000,0.000000}%
\pgfsetstrokecolor{currentstroke}%
\pgfsetdash{}{0pt}%
\pgfpathmoveto{\pgfqpoint{0.864719in}{1.445626in}}%
\pgfpathlineto{\pgfqpoint{0.883791in}{1.438814in}}%
\pgfusepath{stroke}%
\end{pgfscope}%
\begin{pgfscope}%
\pgfpathrectangle{\pgfqpoint{0.100000in}{0.212622in}}{\pgfqpoint{3.696000in}{3.696000in}}%
\pgfusepath{clip}%
\pgfsetrectcap%
\pgfsetroundjoin%
\pgfsetlinewidth{1.505625pt}%
\definecolor{currentstroke}{rgb}{1.000000,0.000000,0.000000}%
\pgfsetstrokecolor{currentstroke}%
\pgfsetdash{}{0pt}%
\pgfpathmoveto{\pgfqpoint{0.864659in}{1.445657in}}%
\pgfpathlineto{\pgfqpoint{0.883791in}{1.438814in}}%
\pgfusepath{stroke}%
\end{pgfscope}%
\begin{pgfscope}%
\pgfpathrectangle{\pgfqpoint{0.100000in}{0.212622in}}{\pgfqpoint{3.696000in}{3.696000in}}%
\pgfusepath{clip}%
\pgfsetrectcap%
\pgfsetroundjoin%
\pgfsetlinewidth{1.505625pt}%
\definecolor{currentstroke}{rgb}{1.000000,0.000000,0.000000}%
\pgfsetstrokecolor{currentstroke}%
\pgfsetdash{}{0pt}%
\pgfpathmoveto{\pgfqpoint{0.864628in}{1.445673in}}%
\pgfpathlineto{\pgfqpoint{0.883791in}{1.438814in}}%
\pgfusepath{stroke}%
\end{pgfscope}%
\begin{pgfscope}%
\pgfpathrectangle{\pgfqpoint{0.100000in}{0.212622in}}{\pgfqpoint{3.696000in}{3.696000in}}%
\pgfusepath{clip}%
\pgfsetrectcap%
\pgfsetroundjoin%
\pgfsetlinewidth{1.505625pt}%
\definecolor{currentstroke}{rgb}{1.000000,0.000000,0.000000}%
\pgfsetstrokecolor{currentstroke}%
\pgfsetdash{}{0pt}%
\pgfpathmoveto{\pgfqpoint{0.864610in}{1.445684in}}%
\pgfpathlineto{\pgfqpoint{0.883791in}{1.438814in}}%
\pgfusepath{stroke}%
\end{pgfscope}%
\begin{pgfscope}%
\pgfpathrectangle{\pgfqpoint{0.100000in}{0.212622in}}{\pgfqpoint{3.696000in}{3.696000in}}%
\pgfusepath{clip}%
\pgfsetrectcap%
\pgfsetroundjoin%
\pgfsetlinewidth{1.505625pt}%
\definecolor{currentstroke}{rgb}{1.000000,0.000000,0.000000}%
\pgfsetstrokecolor{currentstroke}%
\pgfsetdash{}{0pt}%
\pgfpathmoveto{\pgfqpoint{0.864600in}{1.445688in}}%
\pgfpathlineto{\pgfqpoint{0.883791in}{1.438814in}}%
\pgfusepath{stroke}%
\end{pgfscope}%
\begin{pgfscope}%
\pgfpathrectangle{\pgfqpoint{0.100000in}{0.212622in}}{\pgfqpoint{3.696000in}{3.696000in}}%
\pgfusepath{clip}%
\pgfsetrectcap%
\pgfsetroundjoin%
\pgfsetlinewidth{1.505625pt}%
\definecolor{currentstroke}{rgb}{1.000000,0.000000,0.000000}%
\pgfsetstrokecolor{currentstroke}%
\pgfsetdash{}{0pt}%
\pgfpathmoveto{\pgfqpoint{0.864595in}{1.445692in}}%
\pgfpathlineto{\pgfqpoint{0.883791in}{1.438814in}}%
\pgfusepath{stroke}%
\end{pgfscope}%
\begin{pgfscope}%
\pgfpathrectangle{\pgfqpoint{0.100000in}{0.212622in}}{\pgfqpoint{3.696000in}{3.696000in}}%
\pgfusepath{clip}%
\pgfsetrectcap%
\pgfsetroundjoin%
\pgfsetlinewidth{1.505625pt}%
\definecolor{currentstroke}{rgb}{1.000000,0.000000,0.000000}%
\pgfsetstrokecolor{currentstroke}%
\pgfsetdash{}{0pt}%
\pgfpathmoveto{\pgfqpoint{0.864592in}{1.445693in}}%
\pgfpathlineto{\pgfqpoint{0.883791in}{1.438814in}}%
\pgfusepath{stroke}%
\end{pgfscope}%
\begin{pgfscope}%
\pgfpathrectangle{\pgfqpoint{0.100000in}{0.212622in}}{\pgfqpoint{3.696000in}{3.696000in}}%
\pgfusepath{clip}%
\pgfsetrectcap%
\pgfsetroundjoin%
\pgfsetlinewidth{1.505625pt}%
\definecolor{currentstroke}{rgb}{1.000000,0.000000,0.000000}%
\pgfsetstrokecolor{currentstroke}%
\pgfsetdash{}{0pt}%
\pgfpathmoveto{\pgfqpoint{0.864590in}{1.445693in}}%
\pgfpathlineto{\pgfqpoint{0.883791in}{1.438814in}}%
\pgfusepath{stroke}%
\end{pgfscope}%
\begin{pgfscope}%
\pgfpathrectangle{\pgfqpoint{0.100000in}{0.212622in}}{\pgfqpoint{3.696000in}{3.696000in}}%
\pgfusepath{clip}%
\pgfsetrectcap%
\pgfsetroundjoin%
\pgfsetlinewidth{1.505625pt}%
\definecolor{currentstroke}{rgb}{1.000000,0.000000,0.000000}%
\pgfsetstrokecolor{currentstroke}%
\pgfsetdash{}{0pt}%
\pgfpathmoveto{\pgfqpoint{0.864589in}{1.445694in}}%
\pgfpathlineto{\pgfqpoint{0.883791in}{1.438814in}}%
\pgfusepath{stroke}%
\end{pgfscope}%
\begin{pgfscope}%
\pgfpathrectangle{\pgfqpoint{0.100000in}{0.212622in}}{\pgfqpoint{3.696000in}{3.696000in}}%
\pgfusepath{clip}%
\pgfsetrectcap%
\pgfsetroundjoin%
\pgfsetlinewidth{1.505625pt}%
\definecolor{currentstroke}{rgb}{1.000000,0.000000,0.000000}%
\pgfsetstrokecolor{currentstroke}%
\pgfsetdash{}{0pt}%
\pgfpathmoveto{\pgfqpoint{0.864589in}{1.445694in}}%
\pgfpathlineto{\pgfqpoint{0.883791in}{1.438814in}}%
\pgfusepath{stroke}%
\end{pgfscope}%
\begin{pgfscope}%
\pgfpathrectangle{\pgfqpoint{0.100000in}{0.212622in}}{\pgfqpoint{3.696000in}{3.696000in}}%
\pgfusepath{clip}%
\pgfsetrectcap%
\pgfsetroundjoin%
\pgfsetlinewidth{1.505625pt}%
\definecolor{currentstroke}{rgb}{1.000000,0.000000,0.000000}%
\pgfsetstrokecolor{currentstroke}%
\pgfsetdash{}{0pt}%
\pgfpathmoveto{\pgfqpoint{0.864589in}{1.445694in}}%
\pgfpathlineto{\pgfqpoint{0.883791in}{1.438814in}}%
\pgfusepath{stroke}%
\end{pgfscope}%
\begin{pgfscope}%
\pgfpathrectangle{\pgfqpoint{0.100000in}{0.212622in}}{\pgfqpoint{3.696000in}{3.696000in}}%
\pgfusepath{clip}%
\pgfsetrectcap%
\pgfsetroundjoin%
\pgfsetlinewidth{1.505625pt}%
\definecolor{currentstroke}{rgb}{1.000000,0.000000,0.000000}%
\pgfsetstrokecolor{currentstroke}%
\pgfsetdash{}{0pt}%
\pgfpathmoveto{\pgfqpoint{0.864588in}{1.445694in}}%
\pgfpathlineto{\pgfqpoint{0.883791in}{1.438814in}}%
\pgfusepath{stroke}%
\end{pgfscope}%
\begin{pgfscope}%
\pgfpathrectangle{\pgfqpoint{0.100000in}{0.212622in}}{\pgfqpoint{3.696000in}{3.696000in}}%
\pgfusepath{clip}%
\pgfsetrectcap%
\pgfsetroundjoin%
\pgfsetlinewidth{1.505625pt}%
\definecolor{currentstroke}{rgb}{1.000000,0.000000,0.000000}%
\pgfsetstrokecolor{currentstroke}%
\pgfsetdash{}{0pt}%
\pgfpathmoveto{\pgfqpoint{0.864588in}{1.445694in}}%
\pgfpathlineto{\pgfqpoint{0.883791in}{1.438814in}}%
\pgfusepath{stroke}%
\end{pgfscope}%
\begin{pgfscope}%
\pgfpathrectangle{\pgfqpoint{0.100000in}{0.212622in}}{\pgfqpoint{3.696000in}{3.696000in}}%
\pgfusepath{clip}%
\pgfsetrectcap%
\pgfsetroundjoin%
\pgfsetlinewidth{1.505625pt}%
\definecolor{currentstroke}{rgb}{1.000000,0.000000,0.000000}%
\pgfsetstrokecolor{currentstroke}%
\pgfsetdash{}{0pt}%
\pgfpathmoveto{\pgfqpoint{0.864588in}{1.445694in}}%
\pgfpathlineto{\pgfqpoint{0.883791in}{1.438814in}}%
\pgfusepath{stroke}%
\end{pgfscope}%
\begin{pgfscope}%
\pgfpathrectangle{\pgfqpoint{0.100000in}{0.212622in}}{\pgfqpoint{3.696000in}{3.696000in}}%
\pgfusepath{clip}%
\pgfsetrectcap%
\pgfsetroundjoin%
\pgfsetlinewidth{1.505625pt}%
\definecolor{currentstroke}{rgb}{1.000000,0.000000,0.000000}%
\pgfsetstrokecolor{currentstroke}%
\pgfsetdash{}{0pt}%
\pgfpathmoveto{\pgfqpoint{0.864588in}{1.445694in}}%
\pgfpathlineto{\pgfqpoint{0.883791in}{1.438814in}}%
\pgfusepath{stroke}%
\end{pgfscope}%
\begin{pgfscope}%
\pgfpathrectangle{\pgfqpoint{0.100000in}{0.212622in}}{\pgfqpoint{3.696000in}{3.696000in}}%
\pgfusepath{clip}%
\pgfsetrectcap%
\pgfsetroundjoin%
\pgfsetlinewidth{1.505625pt}%
\definecolor{currentstroke}{rgb}{1.000000,0.000000,0.000000}%
\pgfsetstrokecolor{currentstroke}%
\pgfsetdash{}{0pt}%
\pgfpathmoveto{\pgfqpoint{0.864588in}{1.445694in}}%
\pgfpathlineto{\pgfqpoint{0.883791in}{1.438814in}}%
\pgfusepath{stroke}%
\end{pgfscope}%
\begin{pgfscope}%
\pgfpathrectangle{\pgfqpoint{0.100000in}{0.212622in}}{\pgfqpoint{3.696000in}{3.696000in}}%
\pgfusepath{clip}%
\pgfsetrectcap%
\pgfsetroundjoin%
\pgfsetlinewidth{1.505625pt}%
\definecolor{currentstroke}{rgb}{1.000000,0.000000,0.000000}%
\pgfsetstrokecolor{currentstroke}%
\pgfsetdash{}{0pt}%
\pgfpathmoveto{\pgfqpoint{0.864588in}{1.445694in}}%
\pgfpathlineto{\pgfqpoint{0.883791in}{1.438814in}}%
\pgfusepath{stroke}%
\end{pgfscope}%
\begin{pgfscope}%
\pgfpathrectangle{\pgfqpoint{0.100000in}{0.212622in}}{\pgfqpoint{3.696000in}{3.696000in}}%
\pgfusepath{clip}%
\pgfsetrectcap%
\pgfsetroundjoin%
\pgfsetlinewidth{1.505625pt}%
\definecolor{currentstroke}{rgb}{1.000000,0.000000,0.000000}%
\pgfsetstrokecolor{currentstroke}%
\pgfsetdash{}{0pt}%
\pgfpathmoveto{\pgfqpoint{0.864588in}{1.445694in}}%
\pgfpathlineto{\pgfqpoint{0.883791in}{1.438814in}}%
\pgfusepath{stroke}%
\end{pgfscope}%
\begin{pgfscope}%
\pgfpathrectangle{\pgfqpoint{0.100000in}{0.212622in}}{\pgfqpoint{3.696000in}{3.696000in}}%
\pgfusepath{clip}%
\pgfsetrectcap%
\pgfsetroundjoin%
\pgfsetlinewidth{1.505625pt}%
\definecolor{currentstroke}{rgb}{1.000000,0.000000,0.000000}%
\pgfsetstrokecolor{currentstroke}%
\pgfsetdash{}{0pt}%
\pgfpathmoveto{\pgfqpoint{0.864588in}{1.445694in}}%
\pgfpathlineto{\pgfqpoint{0.883791in}{1.438814in}}%
\pgfusepath{stroke}%
\end{pgfscope}%
\begin{pgfscope}%
\pgfpathrectangle{\pgfqpoint{0.100000in}{0.212622in}}{\pgfqpoint{3.696000in}{3.696000in}}%
\pgfusepath{clip}%
\pgfsetrectcap%
\pgfsetroundjoin%
\pgfsetlinewidth{1.505625pt}%
\definecolor{currentstroke}{rgb}{1.000000,0.000000,0.000000}%
\pgfsetstrokecolor{currentstroke}%
\pgfsetdash{}{0pt}%
\pgfpathmoveto{\pgfqpoint{0.864588in}{1.445694in}}%
\pgfpathlineto{\pgfqpoint{0.883791in}{1.438814in}}%
\pgfusepath{stroke}%
\end{pgfscope}%
\begin{pgfscope}%
\pgfpathrectangle{\pgfqpoint{0.100000in}{0.212622in}}{\pgfqpoint{3.696000in}{3.696000in}}%
\pgfusepath{clip}%
\pgfsetrectcap%
\pgfsetroundjoin%
\pgfsetlinewidth{1.505625pt}%
\definecolor{currentstroke}{rgb}{1.000000,0.000000,0.000000}%
\pgfsetstrokecolor{currentstroke}%
\pgfsetdash{}{0pt}%
\pgfpathmoveto{\pgfqpoint{0.864588in}{1.445694in}}%
\pgfpathlineto{\pgfqpoint{0.883791in}{1.438814in}}%
\pgfusepath{stroke}%
\end{pgfscope}%
\begin{pgfscope}%
\pgfpathrectangle{\pgfqpoint{0.100000in}{0.212622in}}{\pgfqpoint{3.696000in}{3.696000in}}%
\pgfusepath{clip}%
\pgfsetrectcap%
\pgfsetroundjoin%
\pgfsetlinewidth{1.505625pt}%
\definecolor{currentstroke}{rgb}{1.000000,0.000000,0.000000}%
\pgfsetstrokecolor{currentstroke}%
\pgfsetdash{}{0pt}%
\pgfpathmoveto{\pgfqpoint{0.864588in}{1.445694in}}%
\pgfpathlineto{\pgfqpoint{0.883791in}{1.438814in}}%
\pgfusepath{stroke}%
\end{pgfscope}%
\begin{pgfscope}%
\pgfpathrectangle{\pgfqpoint{0.100000in}{0.212622in}}{\pgfqpoint{3.696000in}{3.696000in}}%
\pgfusepath{clip}%
\pgfsetrectcap%
\pgfsetroundjoin%
\pgfsetlinewidth{1.505625pt}%
\definecolor{currentstroke}{rgb}{1.000000,0.000000,0.000000}%
\pgfsetstrokecolor{currentstroke}%
\pgfsetdash{}{0pt}%
\pgfpathmoveto{\pgfqpoint{0.864588in}{1.445694in}}%
\pgfpathlineto{\pgfqpoint{0.883791in}{1.438814in}}%
\pgfusepath{stroke}%
\end{pgfscope}%
\begin{pgfscope}%
\pgfpathrectangle{\pgfqpoint{0.100000in}{0.212622in}}{\pgfqpoint{3.696000in}{3.696000in}}%
\pgfusepath{clip}%
\pgfsetrectcap%
\pgfsetroundjoin%
\pgfsetlinewidth{1.505625pt}%
\definecolor{currentstroke}{rgb}{1.000000,0.000000,0.000000}%
\pgfsetstrokecolor{currentstroke}%
\pgfsetdash{}{0pt}%
\pgfpathmoveto{\pgfqpoint{0.864588in}{1.445694in}}%
\pgfpathlineto{\pgfqpoint{0.883791in}{1.438814in}}%
\pgfusepath{stroke}%
\end{pgfscope}%
\begin{pgfscope}%
\pgfpathrectangle{\pgfqpoint{0.100000in}{0.212622in}}{\pgfqpoint{3.696000in}{3.696000in}}%
\pgfusepath{clip}%
\pgfsetrectcap%
\pgfsetroundjoin%
\pgfsetlinewidth{1.505625pt}%
\definecolor{currentstroke}{rgb}{1.000000,0.000000,0.000000}%
\pgfsetstrokecolor{currentstroke}%
\pgfsetdash{}{0pt}%
\pgfpathmoveto{\pgfqpoint{0.864588in}{1.445694in}}%
\pgfpathlineto{\pgfqpoint{0.883791in}{1.438814in}}%
\pgfusepath{stroke}%
\end{pgfscope}%
\begin{pgfscope}%
\pgfpathrectangle{\pgfqpoint{0.100000in}{0.212622in}}{\pgfqpoint{3.696000in}{3.696000in}}%
\pgfusepath{clip}%
\pgfsetrectcap%
\pgfsetroundjoin%
\pgfsetlinewidth{1.505625pt}%
\definecolor{currentstroke}{rgb}{1.000000,0.000000,0.000000}%
\pgfsetstrokecolor{currentstroke}%
\pgfsetdash{}{0pt}%
\pgfpathmoveto{\pgfqpoint{0.864588in}{1.445694in}}%
\pgfpathlineto{\pgfqpoint{0.883791in}{1.438814in}}%
\pgfusepath{stroke}%
\end{pgfscope}%
\begin{pgfscope}%
\pgfpathrectangle{\pgfqpoint{0.100000in}{0.212622in}}{\pgfqpoint{3.696000in}{3.696000in}}%
\pgfusepath{clip}%
\pgfsetrectcap%
\pgfsetroundjoin%
\pgfsetlinewidth{1.505625pt}%
\definecolor{currentstroke}{rgb}{1.000000,0.000000,0.000000}%
\pgfsetstrokecolor{currentstroke}%
\pgfsetdash{}{0pt}%
\pgfpathmoveto{\pgfqpoint{0.864588in}{1.445694in}}%
\pgfpathlineto{\pgfqpoint{0.883791in}{1.438814in}}%
\pgfusepath{stroke}%
\end{pgfscope}%
\begin{pgfscope}%
\pgfpathrectangle{\pgfqpoint{0.100000in}{0.212622in}}{\pgfqpoint{3.696000in}{3.696000in}}%
\pgfusepath{clip}%
\pgfsetrectcap%
\pgfsetroundjoin%
\pgfsetlinewidth{1.505625pt}%
\definecolor{currentstroke}{rgb}{1.000000,0.000000,0.000000}%
\pgfsetstrokecolor{currentstroke}%
\pgfsetdash{}{0pt}%
\pgfpathmoveto{\pgfqpoint{0.864588in}{1.445694in}}%
\pgfpathlineto{\pgfqpoint{0.883791in}{1.438814in}}%
\pgfusepath{stroke}%
\end{pgfscope}%
\begin{pgfscope}%
\pgfpathrectangle{\pgfqpoint{0.100000in}{0.212622in}}{\pgfqpoint{3.696000in}{3.696000in}}%
\pgfusepath{clip}%
\pgfsetrectcap%
\pgfsetroundjoin%
\pgfsetlinewidth{1.505625pt}%
\definecolor{currentstroke}{rgb}{1.000000,0.000000,0.000000}%
\pgfsetstrokecolor{currentstroke}%
\pgfsetdash{}{0pt}%
\pgfpathmoveto{\pgfqpoint{0.864588in}{1.445694in}}%
\pgfpathlineto{\pgfqpoint{0.883791in}{1.438814in}}%
\pgfusepath{stroke}%
\end{pgfscope}%
\begin{pgfscope}%
\pgfpathrectangle{\pgfqpoint{0.100000in}{0.212622in}}{\pgfqpoint{3.696000in}{3.696000in}}%
\pgfusepath{clip}%
\pgfsetrectcap%
\pgfsetroundjoin%
\pgfsetlinewidth{1.505625pt}%
\definecolor{currentstroke}{rgb}{1.000000,0.000000,0.000000}%
\pgfsetstrokecolor{currentstroke}%
\pgfsetdash{}{0pt}%
\pgfpathmoveto{\pgfqpoint{0.864588in}{1.445694in}}%
\pgfpathlineto{\pgfqpoint{0.883791in}{1.438814in}}%
\pgfusepath{stroke}%
\end{pgfscope}%
\begin{pgfscope}%
\pgfpathrectangle{\pgfqpoint{0.100000in}{0.212622in}}{\pgfqpoint{3.696000in}{3.696000in}}%
\pgfusepath{clip}%
\pgfsetrectcap%
\pgfsetroundjoin%
\pgfsetlinewidth{1.505625pt}%
\definecolor{currentstroke}{rgb}{1.000000,0.000000,0.000000}%
\pgfsetstrokecolor{currentstroke}%
\pgfsetdash{}{0pt}%
\pgfpathmoveto{\pgfqpoint{0.864588in}{1.445694in}}%
\pgfpathlineto{\pgfqpoint{0.883791in}{1.438814in}}%
\pgfusepath{stroke}%
\end{pgfscope}%
\begin{pgfscope}%
\pgfpathrectangle{\pgfqpoint{0.100000in}{0.212622in}}{\pgfqpoint{3.696000in}{3.696000in}}%
\pgfusepath{clip}%
\pgfsetrectcap%
\pgfsetroundjoin%
\pgfsetlinewidth{1.505625pt}%
\definecolor{currentstroke}{rgb}{1.000000,0.000000,0.000000}%
\pgfsetstrokecolor{currentstroke}%
\pgfsetdash{}{0pt}%
\pgfpathmoveto{\pgfqpoint{0.864588in}{1.445694in}}%
\pgfpathlineto{\pgfqpoint{0.883791in}{1.438814in}}%
\pgfusepath{stroke}%
\end{pgfscope}%
\begin{pgfscope}%
\pgfpathrectangle{\pgfqpoint{0.100000in}{0.212622in}}{\pgfqpoint{3.696000in}{3.696000in}}%
\pgfusepath{clip}%
\pgfsetrectcap%
\pgfsetroundjoin%
\pgfsetlinewidth{1.505625pt}%
\definecolor{currentstroke}{rgb}{1.000000,0.000000,0.000000}%
\pgfsetstrokecolor{currentstroke}%
\pgfsetdash{}{0pt}%
\pgfpathmoveto{\pgfqpoint{0.864588in}{1.445694in}}%
\pgfpathlineto{\pgfqpoint{0.883791in}{1.438814in}}%
\pgfusepath{stroke}%
\end{pgfscope}%
\begin{pgfscope}%
\pgfpathrectangle{\pgfqpoint{0.100000in}{0.212622in}}{\pgfqpoint{3.696000in}{3.696000in}}%
\pgfusepath{clip}%
\pgfsetrectcap%
\pgfsetroundjoin%
\pgfsetlinewidth{1.505625pt}%
\definecolor{currentstroke}{rgb}{1.000000,0.000000,0.000000}%
\pgfsetstrokecolor{currentstroke}%
\pgfsetdash{}{0pt}%
\pgfpathmoveto{\pgfqpoint{0.864588in}{1.445694in}}%
\pgfpathlineto{\pgfqpoint{0.883791in}{1.438814in}}%
\pgfusepath{stroke}%
\end{pgfscope}%
\begin{pgfscope}%
\pgfpathrectangle{\pgfqpoint{0.100000in}{0.212622in}}{\pgfqpoint{3.696000in}{3.696000in}}%
\pgfusepath{clip}%
\pgfsetrectcap%
\pgfsetroundjoin%
\pgfsetlinewidth{1.505625pt}%
\definecolor{currentstroke}{rgb}{1.000000,0.000000,0.000000}%
\pgfsetstrokecolor{currentstroke}%
\pgfsetdash{}{0pt}%
\pgfpathmoveto{\pgfqpoint{0.864588in}{1.445694in}}%
\pgfpathlineto{\pgfqpoint{0.883791in}{1.438814in}}%
\pgfusepath{stroke}%
\end{pgfscope}%
\begin{pgfscope}%
\pgfpathrectangle{\pgfqpoint{0.100000in}{0.212622in}}{\pgfqpoint{3.696000in}{3.696000in}}%
\pgfusepath{clip}%
\pgfsetrectcap%
\pgfsetroundjoin%
\pgfsetlinewidth{1.505625pt}%
\definecolor{currentstroke}{rgb}{1.000000,0.000000,0.000000}%
\pgfsetstrokecolor{currentstroke}%
\pgfsetdash{}{0pt}%
\pgfpathmoveto{\pgfqpoint{0.864588in}{1.445694in}}%
\pgfpathlineto{\pgfqpoint{0.883791in}{1.438814in}}%
\pgfusepath{stroke}%
\end{pgfscope}%
\begin{pgfscope}%
\pgfpathrectangle{\pgfqpoint{0.100000in}{0.212622in}}{\pgfqpoint{3.696000in}{3.696000in}}%
\pgfusepath{clip}%
\pgfsetrectcap%
\pgfsetroundjoin%
\pgfsetlinewidth{1.505625pt}%
\definecolor{currentstroke}{rgb}{1.000000,0.000000,0.000000}%
\pgfsetstrokecolor{currentstroke}%
\pgfsetdash{}{0pt}%
\pgfpathmoveto{\pgfqpoint{0.864588in}{1.445694in}}%
\pgfpathlineto{\pgfqpoint{0.883791in}{1.438814in}}%
\pgfusepath{stroke}%
\end{pgfscope}%
\begin{pgfscope}%
\pgfpathrectangle{\pgfqpoint{0.100000in}{0.212622in}}{\pgfqpoint{3.696000in}{3.696000in}}%
\pgfusepath{clip}%
\pgfsetrectcap%
\pgfsetroundjoin%
\pgfsetlinewidth{1.505625pt}%
\definecolor{currentstroke}{rgb}{1.000000,0.000000,0.000000}%
\pgfsetstrokecolor{currentstroke}%
\pgfsetdash{}{0pt}%
\pgfpathmoveto{\pgfqpoint{0.864588in}{1.445694in}}%
\pgfpathlineto{\pgfqpoint{0.883791in}{1.438814in}}%
\pgfusepath{stroke}%
\end{pgfscope}%
\begin{pgfscope}%
\pgfpathrectangle{\pgfqpoint{0.100000in}{0.212622in}}{\pgfqpoint{3.696000in}{3.696000in}}%
\pgfusepath{clip}%
\pgfsetrectcap%
\pgfsetroundjoin%
\pgfsetlinewidth{1.505625pt}%
\definecolor{currentstroke}{rgb}{1.000000,0.000000,0.000000}%
\pgfsetstrokecolor{currentstroke}%
\pgfsetdash{}{0pt}%
\pgfpathmoveto{\pgfqpoint{0.864588in}{1.445694in}}%
\pgfpathlineto{\pgfqpoint{0.883791in}{1.438814in}}%
\pgfusepath{stroke}%
\end{pgfscope}%
\begin{pgfscope}%
\pgfpathrectangle{\pgfqpoint{0.100000in}{0.212622in}}{\pgfqpoint{3.696000in}{3.696000in}}%
\pgfusepath{clip}%
\pgfsetrectcap%
\pgfsetroundjoin%
\pgfsetlinewidth{1.505625pt}%
\definecolor{currentstroke}{rgb}{1.000000,0.000000,0.000000}%
\pgfsetstrokecolor{currentstroke}%
\pgfsetdash{}{0pt}%
\pgfpathmoveto{\pgfqpoint{0.864588in}{1.445694in}}%
\pgfpathlineto{\pgfqpoint{0.883791in}{1.438814in}}%
\pgfusepath{stroke}%
\end{pgfscope}%
\begin{pgfscope}%
\pgfpathrectangle{\pgfqpoint{0.100000in}{0.212622in}}{\pgfqpoint{3.696000in}{3.696000in}}%
\pgfusepath{clip}%
\pgfsetrectcap%
\pgfsetroundjoin%
\pgfsetlinewidth{1.505625pt}%
\definecolor{currentstroke}{rgb}{1.000000,0.000000,0.000000}%
\pgfsetstrokecolor{currentstroke}%
\pgfsetdash{}{0pt}%
\pgfpathmoveto{\pgfqpoint{0.864588in}{1.445694in}}%
\pgfpathlineto{\pgfqpoint{0.883791in}{1.438814in}}%
\pgfusepath{stroke}%
\end{pgfscope}%
\begin{pgfscope}%
\pgfpathrectangle{\pgfqpoint{0.100000in}{0.212622in}}{\pgfqpoint{3.696000in}{3.696000in}}%
\pgfusepath{clip}%
\pgfsetrectcap%
\pgfsetroundjoin%
\pgfsetlinewidth{1.505625pt}%
\definecolor{currentstroke}{rgb}{1.000000,0.000000,0.000000}%
\pgfsetstrokecolor{currentstroke}%
\pgfsetdash{}{0pt}%
\pgfpathmoveto{\pgfqpoint{0.864588in}{1.445694in}}%
\pgfpathlineto{\pgfqpoint{0.883791in}{1.438814in}}%
\pgfusepath{stroke}%
\end{pgfscope}%
\begin{pgfscope}%
\pgfpathrectangle{\pgfqpoint{0.100000in}{0.212622in}}{\pgfqpoint{3.696000in}{3.696000in}}%
\pgfusepath{clip}%
\pgfsetrectcap%
\pgfsetroundjoin%
\pgfsetlinewidth{1.505625pt}%
\definecolor{currentstroke}{rgb}{1.000000,0.000000,0.000000}%
\pgfsetstrokecolor{currentstroke}%
\pgfsetdash{}{0pt}%
\pgfpathmoveto{\pgfqpoint{0.864588in}{1.445694in}}%
\pgfpathlineto{\pgfqpoint{0.883791in}{1.438814in}}%
\pgfusepath{stroke}%
\end{pgfscope}%
\begin{pgfscope}%
\pgfpathrectangle{\pgfqpoint{0.100000in}{0.212622in}}{\pgfqpoint{3.696000in}{3.696000in}}%
\pgfusepath{clip}%
\pgfsetrectcap%
\pgfsetroundjoin%
\pgfsetlinewidth{1.505625pt}%
\definecolor{currentstroke}{rgb}{1.000000,0.000000,0.000000}%
\pgfsetstrokecolor{currentstroke}%
\pgfsetdash{}{0pt}%
\pgfpathmoveto{\pgfqpoint{0.864588in}{1.445694in}}%
\pgfpathlineto{\pgfqpoint{0.883791in}{1.438814in}}%
\pgfusepath{stroke}%
\end{pgfscope}%
\begin{pgfscope}%
\pgfpathrectangle{\pgfqpoint{0.100000in}{0.212622in}}{\pgfqpoint{3.696000in}{3.696000in}}%
\pgfusepath{clip}%
\pgfsetrectcap%
\pgfsetroundjoin%
\pgfsetlinewidth{1.505625pt}%
\definecolor{currentstroke}{rgb}{1.000000,0.000000,0.000000}%
\pgfsetstrokecolor{currentstroke}%
\pgfsetdash{}{0pt}%
\pgfpathmoveto{\pgfqpoint{0.864588in}{1.445694in}}%
\pgfpathlineto{\pgfqpoint{0.883791in}{1.438814in}}%
\pgfusepath{stroke}%
\end{pgfscope}%
\begin{pgfscope}%
\pgfpathrectangle{\pgfqpoint{0.100000in}{0.212622in}}{\pgfqpoint{3.696000in}{3.696000in}}%
\pgfusepath{clip}%
\pgfsetrectcap%
\pgfsetroundjoin%
\pgfsetlinewidth{1.505625pt}%
\definecolor{currentstroke}{rgb}{1.000000,0.000000,0.000000}%
\pgfsetstrokecolor{currentstroke}%
\pgfsetdash{}{0pt}%
\pgfpathmoveto{\pgfqpoint{0.864588in}{1.445694in}}%
\pgfpathlineto{\pgfqpoint{0.883791in}{1.438814in}}%
\pgfusepath{stroke}%
\end{pgfscope}%
\begin{pgfscope}%
\pgfpathrectangle{\pgfqpoint{0.100000in}{0.212622in}}{\pgfqpoint{3.696000in}{3.696000in}}%
\pgfusepath{clip}%
\pgfsetrectcap%
\pgfsetroundjoin%
\pgfsetlinewidth{1.505625pt}%
\definecolor{currentstroke}{rgb}{1.000000,0.000000,0.000000}%
\pgfsetstrokecolor{currentstroke}%
\pgfsetdash{}{0pt}%
\pgfpathmoveto{\pgfqpoint{0.864588in}{1.445694in}}%
\pgfpathlineto{\pgfqpoint{0.883791in}{1.438814in}}%
\pgfusepath{stroke}%
\end{pgfscope}%
\begin{pgfscope}%
\pgfpathrectangle{\pgfqpoint{0.100000in}{0.212622in}}{\pgfqpoint{3.696000in}{3.696000in}}%
\pgfusepath{clip}%
\pgfsetrectcap%
\pgfsetroundjoin%
\pgfsetlinewidth{1.505625pt}%
\definecolor{currentstroke}{rgb}{1.000000,0.000000,0.000000}%
\pgfsetstrokecolor{currentstroke}%
\pgfsetdash{}{0pt}%
\pgfpathmoveto{\pgfqpoint{0.864588in}{1.445694in}}%
\pgfpathlineto{\pgfqpoint{0.883791in}{1.438814in}}%
\pgfusepath{stroke}%
\end{pgfscope}%
\begin{pgfscope}%
\pgfpathrectangle{\pgfqpoint{0.100000in}{0.212622in}}{\pgfqpoint{3.696000in}{3.696000in}}%
\pgfusepath{clip}%
\pgfsetrectcap%
\pgfsetroundjoin%
\pgfsetlinewidth{1.505625pt}%
\definecolor{currentstroke}{rgb}{1.000000,0.000000,0.000000}%
\pgfsetstrokecolor{currentstroke}%
\pgfsetdash{}{0pt}%
\pgfpathmoveto{\pgfqpoint{0.864588in}{1.445694in}}%
\pgfpathlineto{\pgfqpoint{0.883791in}{1.438814in}}%
\pgfusepath{stroke}%
\end{pgfscope}%
\begin{pgfscope}%
\pgfpathrectangle{\pgfqpoint{0.100000in}{0.212622in}}{\pgfqpoint{3.696000in}{3.696000in}}%
\pgfusepath{clip}%
\pgfsetrectcap%
\pgfsetroundjoin%
\pgfsetlinewidth{1.505625pt}%
\definecolor{currentstroke}{rgb}{1.000000,0.000000,0.000000}%
\pgfsetstrokecolor{currentstroke}%
\pgfsetdash{}{0pt}%
\pgfpathmoveto{\pgfqpoint{0.864588in}{1.445694in}}%
\pgfpathlineto{\pgfqpoint{0.883791in}{1.438814in}}%
\pgfusepath{stroke}%
\end{pgfscope}%
\begin{pgfscope}%
\pgfpathrectangle{\pgfqpoint{0.100000in}{0.212622in}}{\pgfqpoint{3.696000in}{3.696000in}}%
\pgfusepath{clip}%
\pgfsetrectcap%
\pgfsetroundjoin%
\pgfsetlinewidth{1.505625pt}%
\definecolor{currentstroke}{rgb}{1.000000,0.000000,0.000000}%
\pgfsetstrokecolor{currentstroke}%
\pgfsetdash{}{0pt}%
\pgfpathmoveto{\pgfqpoint{0.864588in}{1.445694in}}%
\pgfpathlineto{\pgfqpoint{0.883791in}{1.438814in}}%
\pgfusepath{stroke}%
\end{pgfscope}%
\begin{pgfscope}%
\pgfpathrectangle{\pgfqpoint{0.100000in}{0.212622in}}{\pgfqpoint{3.696000in}{3.696000in}}%
\pgfusepath{clip}%
\pgfsetrectcap%
\pgfsetroundjoin%
\pgfsetlinewidth{1.505625pt}%
\definecolor{currentstroke}{rgb}{1.000000,0.000000,0.000000}%
\pgfsetstrokecolor{currentstroke}%
\pgfsetdash{}{0pt}%
\pgfpathmoveto{\pgfqpoint{0.864588in}{1.445694in}}%
\pgfpathlineto{\pgfqpoint{0.883791in}{1.438814in}}%
\pgfusepath{stroke}%
\end{pgfscope}%
\begin{pgfscope}%
\pgfpathrectangle{\pgfqpoint{0.100000in}{0.212622in}}{\pgfqpoint{3.696000in}{3.696000in}}%
\pgfusepath{clip}%
\pgfsetrectcap%
\pgfsetroundjoin%
\pgfsetlinewidth{1.505625pt}%
\definecolor{currentstroke}{rgb}{1.000000,0.000000,0.000000}%
\pgfsetstrokecolor{currentstroke}%
\pgfsetdash{}{0pt}%
\pgfpathmoveto{\pgfqpoint{0.864588in}{1.445694in}}%
\pgfpathlineto{\pgfqpoint{0.883791in}{1.438814in}}%
\pgfusepath{stroke}%
\end{pgfscope}%
\begin{pgfscope}%
\pgfpathrectangle{\pgfqpoint{0.100000in}{0.212622in}}{\pgfqpoint{3.696000in}{3.696000in}}%
\pgfusepath{clip}%
\pgfsetrectcap%
\pgfsetroundjoin%
\pgfsetlinewidth{1.505625pt}%
\definecolor{currentstroke}{rgb}{1.000000,0.000000,0.000000}%
\pgfsetstrokecolor{currentstroke}%
\pgfsetdash{}{0pt}%
\pgfpathmoveto{\pgfqpoint{0.864588in}{1.445694in}}%
\pgfpathlineto{\pgfqpoint{0.883791in}{1.438814in}}%
\pgfusepath{stroke}%
\end{pgfscope}%
\begin{pgfscope}%
\pgfpathrectangle{\pgfqpoint{0.100000in}{0.212622in}}{\pgfqpoint{3.696000in}{3.696000in}}%
\pgfusepath{clip}%
\pgfsetrectcap%
\pgfsetroundjoin%
\pgfsetlinewidth{1.505625pt}%
\definecolor{currentstroke}{rgb}{1.000000,0.000000,0.000000}%
\pgfsetstrokecolor{currentstroke}%
\pgfsetdash{}{0pt}%
\pgfpathmoveto{\pgfqpoint{0.864588in}{1.445694in}}%
\pgfpathlineto{\pgfqpoint{0.883791in}{1.438814in}}%
\pgfusepath{stroke}%
\end{pgfscope}%
\begin{pgfscope}%
\pgfpathrectangle{\pgfqpoint{0.100000in}{0.212622in}}{\pgfqpoint{3.696000in}{3.696000in}}%
\pgfusepath{clip}%
\pgfsetrectcap%
\pgfsetroundjoin%
\pgfsetlinewidth{1.505625pt}%
\definecolor{currentstroke}{rgb}{1.000000,0.000000,0.000000}%
\pgfsetstrokecolor{currentstroke}%
\pgfsetdash{}{0pt}%
\pgfpathmoveto{\pgfqpoint{0.864588in}{1.445694in}}%
\pgfpathlineto{\pgfqpoint{0.883791in}{1.438814in}}%
\pgfusepath{stroke}%
\end{pgfscope}%
\begin{pgfscope}%
\pgfpathrectangle{\pgfqpoint{0.100000in}{0.212622in}}{\pgfqpoint{3.696000in}{3.696000in}}%
\pgfusepath{clip}%
\pgfsetrectcap%
\pgfsetroundjoin%
\pgfsetlinewidth{1.505625pt}%
\definecolor{currentstroke}{rgb}{1.000000,0.000000,0.000000}%
\pgfsetstrokecolor{currentstroke}%
\pgfsetdash{}{0pt}%
\pgfpathmoveto{\pgfqpoint{0.864588in}{1.445694in}}%
\pgfpathlineto{\pgfqpoint{0.883791in}{1.438814in}}%
\pgfusepath{stroke}%
\end{pgfscope}%
\begin{pgfscope}%
\pgfpathrectangle{\pgfqpoint{0.100000in}{0.212622in}}{\pgfqpoint{3.696000in}{3.696000in}}%
\pgfusepath{clip}%
\pgfsetrectcap%
\pgfsetroundjoin%
\pgfsetlinewidth{1.505625pt}%
\definecolor{currentstroke}{rgb}{1.000000,0.000000,0.000000}%
\pgfsetstrokecolor{currentstroke}%
\pgfsetdash{}{0pt}%
\pgfpathmoveto{\pgfqpoint{0.864588in}{1.445694in}}%
\pgfpathlineto{\pgfqpoint{0.883791in}{1.438814in}}%
\pgfusepath{stroke}%
\end{pgfscope}%
\begin{pgfscope}%
\pgfpathrectangle{\pgfqpoint{0.100000in}{0.212622in}}{\pgfqpoint{3.696000in}{3.696000in}}%
\pgfusepath{clip}%
\pgfsetrectcap%
\pgfsetroundjoin%
\pgfsetlinewidth{1.505625pt}%
\definecolor{currentstroke}{rgb}{1.000000,0.000000,0.000000}%
\pgfsetstrokecolor{currentstroke}%
\pgfsetdash{}{0pt}%
\pgfpathmoveto{\pgfqpoint{0.864588in}{1.445694in}}%
\pgfpathlineto{\pgfqpoint{0.883791in}{1.438814in}}%
\pgfusepath{stroke}%
\end{pgfscope}%
\begin{pgfscope}%
\pgfpathrectangle{\pgfqpoint{0.100000in}{0.212622in}}{\pgfqpoint{3.696000in}{3.696000in}}%
\pgfusepath{clip}%
\pgfsetrectcap%
\pgfsetroundjoin%
\pgfsetlinewidth{1.505625pt}%
\definecolor{currentstroke}{rgb}{1.000000,0.000000,0.000000}%
\pgfsetstrokecolor{currentstroke}%
\pgfsetdash{}{0pt}%
\pgfpathmoveto{\pgfqpoint{0.864588in}{1.445694in}}%
\pgfpathlineto{\pgfqpoint{0.883791in}{1.438814in}}%
\pgfusepath{stroke}%
\end{pgfscope}%
\begin{pgfscope}%
\pgfpathrectangle{\pgfqpoint{0.100000in}{0.212622in}}{\pgfqpoint{3.696000in}{3.696000in}}%
\pgfusepath{clip}%
\pgfsetrectcap%
\pgfsetroundjoin%
\pgfsetlinewidth{1.505625pt}%
\definecolor{currentstroke}{rgb}{1.000000,0.000000,0.000000}%
\pgfsetstrokecolor{currentstroke}%
\pgfsetdash{}{0pt}%
\pgfpathmoveto{\pgfqpoint{0.864588in}{1.445694in}}%
\pgfpathlineto{\pgfqpoint{0.883791in}{1.438814in}}%
\pgfusepath{stroke}%
\end{pgfscope}%
\begin{pgfscope}%
\pgfpathrectangle{\pgfqpoint{0.100000in}{0.212622in}}{\pgfqpoint{3.696000in}{3.696000in}}%
\pgfusepath{clip}%
\pgfsetrectcap%
\pgfsetroundjoin%
\pgfsetlinewidth{1.505625pt}%
\definecolor{currentstroke}{rgb}{1.000000,0.000000,0.000000}%
\pgfsetstrokecolor{currentstroke}%
\pgfsetdash{}{0pt}%
\pgfpathmoveto{\pgfqpoint{0.864588in}{1.445694in}}%
\pgfpathlineto{\pgfqpoint{0.883791in}{1.438814in}}%
\pgfusepath{stroke}%
\end{pgfscope}%
\begin{pgfscope}%
\pgfpathrectangle{\pgfqpoint{0.100000in}{0.212622in}}{\pgfqpoint{3.696000in}{3.696000in}}%
\pgfusepath{clip}%
\pgfsetrectcap%
\pgfsetroundjoin%
\pgfsetlinewidth{1.505625pt}%
\definecolor{currentstroke}{rgb}{1.000000,0.000000,0.000000}%
\pgfsetstrokecolor{currentstroke}%
\pgfsetdash{}{0pt}%
\pgfpathmoveto{\pgfqpoint{0.864588in}{1.445694in}}%
\pgfpathlineto{\pgfqpoint{0.883791in}{1.438814in}}%
\pgfusepath{stroke}%
\end{pgfscope}%
\begin{pgfscope}%
\pgfpathrectangle{\pgfqpoint{0.100000in}{0.212622in}}{\pgfqpoint{3.696000in}{3.696000in}}%
\pgfusepath{clip}%
\pgfsetrectcap%
\pgfsetroundjoin%
\pgfsetlinewidth{1.505625pt}%
\definecolor{currentstroke}{rgb}{1.000000,0.000000,0.000000}%
\pgfsetstrokecolor{currentstroke}%
\pgfsetdash{}{0pt}%
\pgfpathmoveto{\pgfqpoint{0.864588in}{1.445694in}}%
\pgfpathlineto{\pgfqpoint{0.883791in}{1.438814in}}%
\pgfusepath{stroke}%
\end{pgfscope}%
\begin{pgfscope}%
\pgfpathrectangle{\pgfqpoint{0.100000in}{0.212622in}}{\pgfqpoint{3.696000in}{3.696000in}}%
\pgfusepath{clip}%
\pgfsetrectcap%
\pgfsetroundjoin%
\pgfsetlinewidth{1.505625pt}%
\definecolor{currentstroke}{rgb}{1.000000,0.000000,0.000000}%
\pgfsetstrokecolor{currentstroke}%
\pgfsetdash{}{0pt}%
\pgfpathmoveto{\pgfqpoint{0.864588in}{1.445694in}}%
\pgfpathlineto{\pgfqpoint{0.883791in}{1.438814in}}%
\pgfusepath{stroke}%
\end{pgfscope}%
\begin{pgfscope}%
\pgfpathrectangle{\pgfqpoint{0.100000in}{0.212622in}}{\pgfqpoint{3.696000in}{3.696000in}}%
\pgfusepath{clip}%
\pgfsetrectcap%
\pgfsetroundjoin%
\pgfsetlinewidth{1.505625pt}%
\definecolor{currentstroke}{rgb}{1.000000,0.000000,0.000000}%
\pgfsetstrokecolor{currentstroke}%
\pgfsetdash{}{0pt}%
\pgfpathmoveto{\pgfqpoint{0.864588in}{1.445694in}}%
\pgfpathlineto{\pgfqpoint{0.883791in}{1.438814in}}%
\pgfusepath{stroke}%
\end{pgfscope}%
\begin{pgfscope}%
\pgfpathrectangle{\pgfqpoint{0.100000in}{0.212622in}}{\pgfqpoint{3.696000in}{3.696000in}}%
\pgfusepath{clip}%
\pgfsetrectcap%
\pgfsetroundjoin%
\pgfsetlinewidth{1.505625pt}%
\definecolor{currentstroke}{rgb}{1.000000,0.000000,0.000000}%
\pgfsetstrokecolor{currentstroke}%
\pgfsetdash{}{0pt}%
\pgfpathmoveto{\pgfqpoint{0.864588in}{1.445694in}}%
\pgfpathlineto{\pgfqpoint{0.883791in}{1.438814in}}%
\pgfusepath{stroke}%
\end{pgfscope}%
\begin{pgfscope}%
\pgfpathrectangle{\pgfqpoint{0.100000in}{0.212622in}}{\pgfqpoint{3.696000in}{3.696000in}}%
\pgfusepath{clip}%
\pgfsetrectcap%
\pgfsetroundjoin%
\pgfsetlinewidth{1.505625pt}%
\definecolor{currentstroke}{rgb}{1.000000,0.000000,0.000000}%
\pgfsetstrokecolor{currentstroke}%
\pgfsetdash{}{0pt}%
\pgfpathmoveto{\pgfqpoint{0.864588in}{1.445694in}}%
\pgfpathlineto{\pgfqpoint{0.883791in}{1.438814in}}%
\pgfusepath{stroke}%
\end{pgfscope}%
\begin{pgfscope}%
\pgfpathrectangle{\pgfqpoint{0.100000in}{0.212622in}}{\pgfqpoint{3.696000in}{3.696000in}}%
\pgfusepath{clip}%
\pgfsetrectcap%
\pgfsetroundjoin%
\pgfsetlinewidth{1.505625pt}%
\definecolor{currentstroke}{rgb}{1.000000,0.000000,0.000000}%
\pgfsetstrokecolor{currentstroke}%
\pgfsetdash{}{0pt}%
\pgfpathmoveto{\pgfqpoint{0.864588in}{1.445694in}}%
\pgfpathlineto{\pgfqpoint{0.883791in}{1.438814in}}%
\pgfusepath{stroke}%
\end{pgfscope}%
\begin{pgfscope}%
\pgfpathrectangle{\pgfqpoint{0.100000in}{0.212622in}}{\pgfqpoint{3.696000in}{3.696000in}}%
\pgfusepath{clip}%
\pgfsetrectcap%
\pgfsetroundjoin%
\pgfsetlinewidth{1.505625pt}%
\definecolor{currentstroke}{rgb}{1.000000,0.000000,0.000000}%
\pgfsetstrokecolor{currentstroke}%
\pgfsetdash{}{0pt}%
\pgfpathmoveto{\pgfqpoint{0.864588in}{1.445694in}}%
\pgfpathlineto{\pgfqpoint{0.883791in}{1.438814in}}%
\pgfusepath{stroke}%
\end{pgfscope}%
\begin{pgfscope}%
\pgfpathrectangle{\pgfqpoint{0.100000in}{0.212622in}}{\pgfqpoint{3.696000in}{3.696000in}}%
\pgfusepath{clip}%
\pgfsetrectcap%
\pgfsetroundjoin%
\pgfsetlinewidth{1.505625pt}%
\definecolor{currentstroke}{rgb}{1.000000,0.000000,0.000000}%
\pgfsetstrokecolor{currentstroke}%
\pgfsetdash{}{0pt}%
\pgfpathmoveto{\pgfqpoint{0.864588in}{1.445694in}}%
\pgfpathlineto{\pgfqpoint{0.883791in}{1.438814in}}%
\pgfusepath{stroke}%
\end{pgfscope}%
\begin{pgfscope}%
\pgfpathrectangle{\pgfqpoint{0.100000in}{0.212622in}}{\pgfqpoint{3.696000in}{3.696000in}}%
\pgfusepath{clip}%
\pgfsetrectcap%
\pgfsetroundjoin%
\pgfsetlinewidth{1.505625pt}%
\definecolor{currentstroke}{rgb}{1.000000,0.000000,0.000000}%
\pgfsetstrokecolor{currentstroke}%
\pgfsetdash{}{0pt}%
\pgfpathmoveto{\pgfqpoint{0.864588in}{1.445694in}}%
\pgfpathlineto{\pgfqpoint{0.883791in}{1.438814in}}%
\pgfusepath{stroke}%
\end{pgfscope}%
\begin{pgfscope}%
\pgfpathrectangle{\pgfqpoint{0.100000in}{0.212622in}}{\pgfqpoint{3.696000in}{3.696000in}}%
\pgfusepath{clip}%
\pgfsetrectcap%
\pgfsetroundjoin%
\pgfsetlinewidth{1.505625pt}%
\definecolor{currentstroke}{rgb}{1.000000,0.000000,0.000000}%
\pgfsetstrokecolor{currentstroke}%
\pgfsetdash{}{0pt}%
\pgfpathmoveto{\pgfqpoint{0.864588in}{1.445694in}}%
\pgfpathlineto{\pgfqpoint{0.883791in}{1.438814in}}%
\pgfusepath{stroke}%
\end{pgfscope}%
\begin{pgfscope}%
\pgfpathrectangle{\pgfqpoint{0.100000in}{0.212622in}}{\pgfqpoint{3.696000in}{3.696000in}}%
\pgfusepath{clip}%
\pgfsetrectcap%
\pgfsetroundjoin%
\pgfsetlinewidth{1.505625pt}%
\definecolor{currentstroke}{rgb}{1.000000,0.000000,0.000000}%
\pgfsetstrokecolor{currentstroke}%
\pgfsetdash{}{0pt}%
\pgfpathmoveto{\pgfqpoint{0.864588in}{1.445694in}}%
\pgfpathlineto{\pgfqpoint{0.883791in}{1.438814in}}%
\pgfusepath{stroke}%
\end{pgfscope}%
\begin{pgfscope}%
\pgfpathrectangle{\pgfqpoint{0.100000in}{0.212622in}}{\pgfqpoint{3.696000in}{3.696000in}}%
\pgfusepath{clip}%
\pgfsetrectcap%
\pgfsetroundjoin%
\pgfsetlinewidth{1.505625pt}%
\definecolor{currentstroke}{rgb}{1.000000,0.000000,0.000000}%
\pgfsetstrokecolor{currentstroke}%
\pgfsetdash{}{0pt}%
\pgfpathmoveto{\pgfqpoint{0.864588in}{1.445694in}}%
\pgfpathlineto{\pgfqpoint{0.883791in}{1.438814in}}%
\pgfusepath{stroke}%
\end{pgfscope}%
\begin{pgfscope}%
\pgfpathrectangle{\pgfqpoint{0.100000in}{0.212622in}}{\pgfqpoint{3.696000in}{3.696000in}}%
\pgfusepath{clip}%
\pgfsetrectcap%
\pgfsetroundjoin%
\pgfsetlinewidth{1.505625pt}%
\definecolor{currentstroke}{rgb}{1.000000,0.000000,0.000000}%
\pgfsetstrokecolor{currentstroke}%
\pgfsetdash{}{0pt}%
\pgfpathmoveto{\pgfqpoint{0.864588in}{1.445694in}}%
\pgfpathlineto{\pgfqpoint{0.883791in}{1.438814in}}%
\pgfusepath{stroke}%
\end{pgfscope}%
\begin{pgfscope}%
\pgfpathrectangle{\pgfqpoint{0.100000in}{0.212622in}}{\pgfqpoint{3.696000in}{3.696000in}}%
\pgfusepath{clip}%
\pgfsetrectcap%
\pgfsetroundjoin%
\pgfsetlinewidth{1.505625pt}%
\definecolor{currentstroke}{rgb}{1.000000,0.000000,0.000000}%
\pgfsetstrokecolor{currentstroke}%
\pgfsetdash{}{0pt}%
\pgfpathmoveto{\pgfqpoint{0.864588in}{1.445694in}}%
\pgfpathlineto{\pgfqpoint{0.883791in}{1.438814in}}%
\pgfusepath{stroke}%
\end{pgfscope}%
\begin{pgfscope}%
\pgfpathrectangle{\pgfqpoint{0.100000in}{0.212622in}}{\pgfqpoint{3.696000in}{3.696000in}}%
\pgfusepath{clip}%
\pgfsetrectcap%
\pgfsetroundjoin%
\pgfsetlinewidth{1.505625pt}%
\definecolor{currentstroke}{rgb}{1.000000,0.000000,0.000000}%
\pgfsetstrokecolor{currentstroke}%
\pgfsetdash{}{0pt}%
\pgfpathmoveto{\pgfqpoint{0.864588in}{1.445694in}}%
\pgfpathlineto{\pgfqpoint{0.883791in}{1.438814in}}%
\pgfusepath{stroke}%
\end{pgfscope}%
\begin{pgfscope}%
\pgfpathrectangle{\pgfqpoint{0.100000in}{0.212622in}}{\pgfqpoint{3.696000in}{3.696000in}}%
\pgfusepath{clip}%
\pgfsetrectcap%
\pgfsetroundjoin%
\pgfsetlinewidth{1.505625pt}%
\definecolor{currentstroke}{rgb}{1.000000,0.000000,0.000000}%
\pgfsetstrokecolor{currentstroke}%
\pgfsetdash{}{0pt}%
\pgfpathmoveto{\pgfqpoint{0.864588in}{1.445694in}}%
\pgfpathlineto{\pgfqpoint{0.883791in}{1.438814in}}%
\pgfusepath{stroke}%
\end{pgfscope}%
\begin{pgfscope}%
\pgfpathrectangle{\pgfqpoint{0.100000in}{0.212622in}}{\pgfqpoint{3.696000in}{3.696000in}}%
\pgfusepath{clip}%
\pgfsetrectcap%
\pgfsetroundjoin%
\pgfsetlinewidth{1.505625pt}%
\definecolor{currentstroke}{rgb}{1.000000,0.000000,0.000000}%
\pgfsetstrokecolor{currentstroke}%
\pgfsetdash{}{0pt}%
\pgfpathmoveto{\pgfqpoint{0.864588in}{1.445694in}}%
\pgfpathlineto{\pgfqpoint{0.883791in}{1.438814in}}%
\pgfusepath{stroke}%
\end{pgfscope}%
\begin{pgfscope}%
\pgfpathrectangle{\pgfqpoint{0.100000in}{0.212622in}}{\pgfqpoint{3.696000in}{3.696000in}}%
\pgfusepath{clip}%
\pgfsetrectcap%
\pgfsetroundjoin%
\pgfsetlinewidth{1.505625pt}%
\definecolor{currentstroke}{rgb}{1.000000,0.000000,0.000000}%
\pgfsetstrokecolor{currentstroke}%
\pgfsetdash{}{0pt}%
\pgfpathmoveto{\pgfqpoint{0.864588in}{1.445694in}}%
\pgfpathlineto{\pgfqpoint{0.883791in}{1.438814in}}%
\pgfusepath{stroke}%
\end{pgfscope}%
\begin{pgfscope}%
\pgfpathrectangle{\pgfqpoint{0.100000in}{0.212622in}}{\pgfqpoint{3.696000in}{3.696000in}}%
\pgfusepath{clip}%
\pgfsetrectcap%
\pgfsetroundjoin%
\pgfsetlinewidth{1.505625pt}%
\definecolor{currentstroke}{rgb}{1.000000,0.000000,0.000000}%
\pgfsetstrokecolor{currentstroke}%
\pgfsetdash{}{0pt}%
\pgfpathmoveto{\pgfqpoint{0.864588in}{1.445694in}}%
\pgfpathlineto{\pgfqpoint{0.883791in}{1.438814in}}%
\pgfusepath{stroke}%
\end{pgfscope}%
\begin{pgfscope}%
\pgfpathrectangle{\pgfqpoint{0.100000in}{0.212622in}}{\pgfqpoint{3.696000in}{3.696000in}}%
\pgfusepath{clip}%
\pgfsetrectcap%
\pgfsetroundjoin%
\pgfsetlinewidth{1.505625pt}%
\definecolor{currentstroke}{rgb}{1.000000,0.000000,0.000000}%
\pgfsetstrokecolor{currentstroke}%
\pgfsetdash{}{0pt}%
\pgfpathmoveto{\pgfqpoint{0.864588in}{1.445694in}}%
\pgfpathlineto{\pgfqpoint{0.883791in}{1.438814in}}%
\pgfusepath{stroke}%
\end{pgfscope}%
\begin{pgfscope}%
\pgfpathrectangle{\pgfqpoint{0.100000in}{0.212622in}}{\pgfqpoint{3.696000in}{3.696000in}}%
\pgfusepath{clip}%
\pgfsetrectcap%
\pgfsetroundjoin%
\pgfsetlinewidth{1.505625pt}%
\definecolor{currentstroke}{rgb}{1.000000,0.000000,0.000000}%
\pgfsetstrokecolor{currentstroke}%
\pgfsetdash{}{0pt}%
\pgfpathmoveto{\pgfqpoint{0.864588in}{1.445694in}}%
\pgfpathlineto{\pgfqpoint{0.883791in}{1.438814in}}%
\pgfusepath{stroke}%
\end{pgfscope}%
\begin{pgfscope}%
\pgfpathrectangle{\pgfqpoint{0.100000in}{0.212622in}}{\pgfqpoint{3.696000in}{3.696000in}}%
\pgfusepath{clip}%
\pgfsetrectcap%
\pgfsetroundjoin%
\pgfsetlinewidth{1.505625pt}%
\definecolor{currentstroke}{rgb}{1.000000,0.000000,0.000000}%
\pgfsetstrokecolor{currentstroke}%
\pgfsetdash{}{0pt}%
\pgfpathmoveto{\pgfqpoint{0.864588in}{1.445694in}}%
\pgfpathlineto{\pgfqpoint{0.883791in}{1.438814in}}%
\pgfusepath{stroke}%
\end{pgfscope}%
\begin{pgfscope}%
\pgfpathrectangle{\pgfqpoint{0.100000in}{0.212622in}}{\pgfqpoint{3.696000in}{3.696000in}}%
\pgfusepath{clip}%
\pgfsetrectcap%
\pgfsetroundjoin%
\pgfsetlinewidth{1.505625pt}%
\definecolor{currentstroke}{rgb}{1.000000,0.000000,0.000000}%
\pgfsetstrokecolor{currentstroke}%
\pgfsetdash{}{0pt}%
\pgfpathmoveto{\pgfqpoint{0.864035in}{1.445514in}}%
\pgfpathlineto{\pgfqpoint{0.883791in}{1.438814in}}%
\pgfusepath{stroke}%
\end{pgfscope}%
\begin{pgfscope}%
\pgfpathrectangle{\pgfqpoint{0.100000in}{0.212622in}}{\pgfqpoint{3.696000in}{3.696000in}}%
\pgfusepath{clip}%
\pgfsetrectcap%
\pgfsetroundjoin%
\pgfsetlinewidth{1.505625pt}%
\definecolor{currentstroke}{rgb}{1.000000,0.000000,0.000000}%
\pgfsetstrokecolor{currentstroke}%
\pgfsetdash{}{0pt}%
\pgfpathmoveto{\pgfqpoint{0.862639in}{1.445271in}}%
\pgfpathlineto{\pgfqpoint{0.883791in}{1.438814in}}%
\pgfusepath{stroke}%
\end{pgfscope}%
\begin{pgfscope}%
\pgfpathrectangle{\pgfqpoint{0.100000in}{0.212622in}}{\pgfqpoint{3.696000in}{3.696000in}}%
\pgfusepath{clip}%
\pgfsetrectcap%
\pgfsetroundjoin%
\pgfsetlinewidth{1.505625pt}%
\definecolor{currentstroke}{rgb}{1.000000,0.000000,0.000000}%
\pgfsetstrokecolor{currentstroke}%
\pgfsetdash{}{0pt}%
\pgfpathmoveto{\pgfqpoint{0.860032in}{1.445174in}}%
\pgfpathlineto{\pgfqpoint{0.883791in}{1.438814in}}%
\pgfusepath{stroke}%
\end{pgfscope}%
\begin{pgfscope}%
\pgfpathrectangle{\pgfqpoint{0.100000in}{0.212622in}}{\pgfqpoint{3.696000in}{3.696000in}}%
\pgfusepath{clip}%
\pgfsetrectcap%
\pgfsetroundjoin%
\pgfsetlinewidth{1.505625pt}%
\definecolor{currentstroke}{rgb}{1.000000,0.000000,0.000000}%
\pgfsetstrokecolor{currentstroke}%
\pgfsetdash{}{0pt}%
\pgfpathmoveto{\pgfqpoint{0.855887in}{1.445750in}}%
\pgfpathlineto{\pgfqpoint{0.883791in}{1.438814in}}%
\pgfusepath{stroke}%
\end{pgfscope}%
\begin{pgfscope}%
\pgfpathrectangle{\pgfqpoint{0.100000in}{0.212622in}}{\pgfqpoint{3.696000in}{3.696000in}}%
\pgfusepath{clip}%
\pgfsetrectcap%
\pgfsetroundjoin%
\pgfsetlinewidth{1.505625pt}%
\definecolor{currentstroke}{rgb}{1.000000,0.000000,0.000000}%
\pgfsetstrokecolor{currentstroke}%
\pgfsetdash{}{0pt}%
\pgfpathmoveto{\pgfqpoint{0.851847in}{1.447656in}}%
\pgfpathlineto{\pgfqpoint{0.883791in}{1.438814in}}%
\pgfusepath{stroke}%
\end{pgfscope}%
\begin{pgfscope}%
\pgfpathrectangle{\pgfqpoint{0.100000in}{0.212622in}}{\pgfqpoint{3.696000in}{3.696000in}}%
\pgfusepath{clip}%
\pgfsetrectcap%
\pgfsetroundjoin%
\pgfsetlinewidth{1.505625pt}%
\definecolor{currentstroke}{rgb}{1.000000,0.000000,0.000000}%
\pgfsetstrokecolor{currentstroke}%
\pgfsetdash{}{0pt}%
\pgfpathmoveto{\pgfqpoint{0.848760in}{1.454383in}}%
\pgfpathlineto{\pgfqpoint{0.883791in}{1.438814in}}%
\pgfusepath{stroke}%
\end{pgfscope}%
\begin{pgfscope}%
\pgfpathrectangle{\pgfqpoint{0.100000in}{0.212622in}}{\pgfqpoint{3.696000in}{3.696000in}}%
\pgfusepath{clip}%
\pgfsetrectcap%
\pgfsetroundjoin%
\pgfsetlinewidth{1.505625pt}%
\definecolor{currentstroke}{rgb}{1.000000,0.000000,0.000000}%
\pgfsetstrokecolor{currentstroke}%
\pgfsetdash{}{0pt}%
\pgfpathmoveto{\pgfqpoint{0.849354in}{1.461283in}}%
\pgfpathlineto{\pgfqpoint{0.883791in}{1.438814in}}%
\pgfusepath{stroke}%
\end{pgfscope}%
\begin{pgfscope}%
\pgfpathrectangle{\pgfqpoint{0.100000in}{0.212622in}}{\pgfqpoint{3.696000in}{3.696000in}}%
\pgfusepath{clip}%
\pgfsetrectcap%
\pgfsetroundjoin%
\pgfsetlinewidth{1.505625pt}%
\definecolor{currentstroke}{rgb}{1.000000,0.000000,0.000000}%
\pgfsetstrokecolor{currentstroke}%
\pgfsetdash{}{0pt}%
\pgfpathmoveto{\pgfqpoint{0.851099in}{1.465063in}}%
\pgfpathlineto{\pgfqpoint{0.883791in}{1.438814in}}%
\pgfusepath{stroke}%
\end{pgfscope}%
\begin{pgfscope}%
\pgfpathrectangle{\pgfqpoint{0.100000in}{0.212622in}}{\pgfqpoint{3.696000in}{3.696000in}}%
\pgfusepath{clip}%
\pgfsetrectcap%
\pgfsetroundjoin%
\pgfsetlinewidth{1.505625pt}%
\definecolor{currentstroke}{rgb}{1.000000,0.000000,0.000000}%
\pgfsetstrokecolor{currentstroke}%
\pgfsetdash{}{0pt}%
\pgfpathmoveto{\pgfqpoint{0.852889in}{1.465412in}}%
\pgfpathlineto{\pgfqpoint{0.883791in}{1.438814in}}%
\pgfusepath{stroke}%
\end{pgfscope}%
\begin{pgfscope}%
\pgfpathrectangle{\pgfqpoint{0.100000in}{0.212622in}}{\pgfqpoint{3.696000in}{3.696000in}}%
\pgfusepath{clip}%
\pgfsetrectcap%
\pgfsetroundjoin%
\pgfsetlinewidth{1.505625pt}%
\definecolor{currentstroke}{rgb}{1.000000,0.000000,0.000000}%
\pgfsetstrokecolor{currentstroke}%
\pgfsetdash{}{0pt}%
\pgfpathmoveto{\pgfqpoint{0.855668in}{1.466693in}}%
\pgfpathlineto{\pgfqpoint{0.883791in}{1.438814in}}%
\pgfusepath{stroke}%
\end{pgfscope}%
\begin{pgfscope}%
\pgfpathrectangle{\pgfqpoint{0.100000in}{0.212622in}}{\pgfqpoint{3.696000in}{3.696000in}}%
\pgfusepath{clip}%
\pgfsetrectcap%
\pgfsetroundjoin%
\pgfsetlinewidth{1.505625pt}%
\definecolor{currentstroke}{rgb}{1.000000,0.000000,0.000000}%
\pgfsetstrokecolor{currentstroke}%
\pgfsetdash{}{0pt}%
\pgfpathmoveto{\pgfqpoint{0.859232in}{1.468039in}}%
\pgfpathlineto{\pgfqpoint{0.883791in}{1.438814in}}%
\pgfusepath{stroke}%
\end{pgfscope}%
\begin{pgfscope}%
\pgfpathrectangle{\pgfqpoint{0.100000in}{0.212622in}}{\pgfqpoint{3.696000in}{3.696000in}}%
\pgfusepath{clip}%
\pgfsetrectcap%
\pgfsetroundjoin%
\pgfsetlinewidth{1.505625pt}%
\definecolor{currentstroke}{rgb}{1.000000,0.000000,0.000000}%
\pgfsetstrokecolor{currentstroke}%
\pgfsetdash{}{0pt}%
\pgfpathmoveto{\pgfqpoint{0.863184in}{1.472241in}}%
\pgfpathlineto{\pgfqpoint{0.883791in}{1.438814in}}%
\pgfusepath{stroke}%
\end{pgfscope}%
\begin{pgfscope}%
\pgfpathrectangle{\pgfqpoint{0.100000in}{0.212622in}}{\pgfqpoint{3.696000in}{3.696000in}}%
\pgfusepath{clip}%
\pgfsetrectcap%
\pgfsetroundjoin%
\pgfsetlinewidth{1.505625pt}%
\definecolor{currentstroke}{rgb}{1.000000,0.000000,0.000000}%
\pgfsetstrokecolor{currentstroke}%
\pgfsetdash{}{0pt}%
\pgfpathmoveto{\pgfqpoint{0.868006in}{1.473390in}}%
\pgfpathlineto{\pgfqpoint{0.883791in}{1.438814in}}%
\pgfusepath{stroke}%
\end{pgfscope}%
\begin{pgfscope}%
\pgfpathrectangle{\pgfqpoint{0.100000in}{0.212622in}}{\pgfqpoint{3.696000in}{3.696000in}}%
\pgfusepath{clip}%
\pgfsetrectcap%
\pgfsetroundjoin%
\pgfsetlinewidth{1.505625pt}%
\definecolor{currentstroke}{rgb}{1.000000,0.000000,0.000000}%
\pgfsetstrokecolor{currentstroke}%
\pgfsetdash{}{0pt}%
\pgfpathmoveto{\pgfqpoint{0.870640in}{1.475084in}}%
\pgfpathlineto{\pgfqpoint{0.893342in}{1.446892in}}%
\pgfusepath{stroke}%
\end{pgfscope}%
\begin{pgfscope}%
\pgfpathrectangle{\pgfqpoint{0.100000in}{0.212622in}}{\pgfqpoint{3.696000in}{3.696000in}}%
\pgfusepath{clip}%
\pgfsetrectcap%
\pgfsetroundjoin%
\pgfsetlinewidth{1.505625pt}%
\definecolor{currentstroke}{rgb}{1.000000,0.000000,0.000000}%
\pgfsetstrokecolor{currentstroke}%
\pgfsetdash{}{0pt}%
\pgfpathmoveto{\pgfqpoint{0.871974in}{1.476213in}}%
\pgfpathlineto{\pgfqpoint{0.893342in}{1.446892in}}%
\pgfusepath{stroke}%
\end{pgfscope}%
\begin{pgfscope}%
\pgfpathrectangle{\pgfqpoint{0.100000in}{0.212622in}}{\pgfqpoint{3.696000in}{3.696000in}}%
\pgfusepath{clip}%
\pgfsetrectcap%
\pgfsetroundjoin%
\pgfsetlinewidth{1.505625pt}%
\definecolor{currentstroke}{rgb}{1.000000,0.000000,0.000000}%
\pgfsetstrokecolor{currentstroke}%
\pgfsetdash{}{0pt}%
\pgfpathmoveto{\pgfqpoint{0.872717in}{1.476639in}}%
\pgfpathlineto{\pgfqpoint{0.893342in}{1.446892in}}%
\pgfusepath{stroke}%
\end{pgfscope}%
\begin{pgfscope}%
\pgfpathrectangle{\pgfqpoint{0.100000in}{0.212622in}}{\pgfqpoint{3.696000in}{3.696000in}}%
\pgfusepath{clip}%
\pgfsetrectcap%
\pgfsetroundjoin%
\pgfsetlinewidth{1.505625pt}%
\definecolor{currentstroke}{rgb}{1.000000,0.000000,0.000000}%
\pgfsetstrokecolor{currentstroke}%
\pgfsetdash{}{0pt}%
\pgfpathmoveto{\pgfqpoint{0.873152in}{1.476950in}}%
\pgfpathlineto{\pgfqpoint{0.893342in}{1.446892in}}%
\pgfusepath{stroke}%
\end{pgfscope}%
\begin{pgfscope}%
\pgfpathrectangle{\pgfqpoint{0.100000in}{0.212622in}}{\pgfqpoint{3.696000in}{3.696000in}}%
\pgfusepath{clip}%
\pgfsetrectcap%
\pgfsetroundjoin%
\pgfsetlinewidth{1.505625pt}%
\definecolor{currentstroke}{rgb}{1.000000,0.000000,0.000000}%
\pgfsetstrokecolor{currentstroke}%
\pgfsetdash{}{0pt}%
\pgfpathmoveto{\pgfqpoint{0.873380in}{1.477080in}}%
\pgfpathlineto{\pgfqpoint{0.893342in}{1.446892in}}%
\pgfusepath{stroke}%
\end{pgfscope}%
\begin{pgfscope}%
\pgfpathrectangle{\pgfqpoint{0.100000in}{0.212622in}}{\pgfqpoint{3.696000in}{3.696000in}}%
\pgfusepath{clip}%
\pgfsetrectcap%
\pgfsetroundjoin%
\pgfsetlinewidth{1.505625pt}%
\definecolor{currentstroke}{rgb}{1.000000,0.000000,0.000000}%
\pgfsetstrokecolor{currentstroke}%
\pgfsetdash{}{0pt}%
\pgfpathmoveto{\pgfqpoint{0.873504in}{1.477204in}}%
\pgfpathlineto{\pgfqpoint{0.893342in}{1.446892in}}%
\pgfusepath{stroke}%
\end{pgfscope}%
\begin{pgfscope}%
\pgfpathrectangle{\pgfqpoint{0.100000in}{0.212622in}}{\pgfqpoint{3.696000in}{3.696000in}}%
\pgfusepath{clip}%
\pgfsetrectcap%
\pgfsetroundjoin%
\pgfsetlinewidth{1.505625pt}%
\definecolor{currentstroke}{rgb}{1.000000,0.000000,0.000000}%
\pgfsetstrokecolor{currentstroke}%
\pgfsetdash{}{0pt}%
\pgfpathmoveto{\pgfqpoint{0.873588in}{1.477209in}}%
\pgfpathlineto{\pgfqpoint{0.893342in}{1.446892in}}%
\pgfusepath{stroke}%
\end{pgfscope}%
\begin{pgfscope}%
\pgfpathrectangle{\pgfqpoint{0.100000in}{0.212622in}}{\pgfqpoint{3.696000in}{3.696000in}}%
\pgfusepath{clip}%
\pgfsetrectcap%
\pgfsetroundjoin%
\pgfsetlinewidth{1.505625pt}%
\definecolor{currentstroke}{rgb}{1.000000,0.000000,0.000000}%
\pgfsetstrokecolor{currentstroke}%
\pgfsetdash{}{0pt}%
\pgfpathmoveto{\pgfqpoint{0.873623in}{1.477230in}}%
\pgfpathlineto{\pgfqpoint{0.893342in}{1.446892in}}%
\pgfusepath{stroke}%
\end{pgfscope}%
\begin{pgfscope}%
\pgfpathrectangle{\pgfqpoint{0.100000in}{0.212622in}}{\pgfqpoint{3.696000in}{3.696000in}}%
\pgfusepath{clip}%
\pgfsetrectcap%
\pgfsetroundjoin%
\pgfsetlinewidth{1.505625pt}%
\definecolor{currentstroke}{rgb}{1.000000,0.000000,0.000000}%
\pgfsetstrokecolor{currentstroke}%
\pgfsetdash{}{0pt}%
\pgfpathmoveto{\pgfqpoint{0.873645in}{1.477245in}}%
\pgfpathlineto{\pgfqpoint{0.893342in}{1.446892in}}%
\pgfusepath{stroke}%
\end{pgfscope}%
\begin{pgfscope}%
\pgfpathrectangle{\pgfqpoint{0.100000in}{0.212622in}}{\pgfqpoint{3.696000in}{3.696000in}}%
\pgfusepath{clip}%
\pgfsetrectcap%
\pgfsetroundjoin%
\pgfsetlinewidth{1.505625pt}%
\definecolor{currentstroke}{rgb}{1.000000,0.000000,0.000000}%
\pgfsetstrokecolor{currentstroke}%
\pgfsetdash{}{0pt}%
\pgfpathmoveto{\pgfqpoint{0.873657in}{1.477250in}}%
\pgfpathlineto{\pgfqpoint{0.893342in}{1.446892in}}%
\pgfusepath{stroke}%
\end{pgfscope}%
\begin{pgfscope}%
\pgfpathrectangle{\pgfqpoint{0.100000in}{0.212622in}}{\pgfqpoint{3.696000in}{3.696000in}}%
\pgfusepath{clip}%
\pgfsetrectcap%
\pgfsetroundjoin%
\pgfsetlinewidth{1.505625pt}%
\definecolor{currentstroke}{rgb}{1.000000,0.000000,0.000000}%
\pgfsetstrokecolor{currentstroke}%
\pgfsetdash{}{0pt}%
\pgfpathmoveto{\pgfqpoint{0.873663in}{1.477256in}}%
\pgfpathlineto{\pgfqpoint{0.893342in}{1.446892in}}%
\pgfusepath{stroke}%
\end{pgfscope}%
\begin{pgfscope}%
\pgfpathrectangle{\pgfqpoint{0.100000in}{0.212622in}}{\pgfqpoint{3.696000in}{3.696000in}}%
\pgfusepath{clip}%
\pgfsetrectcap%
\pgfsetroundjoin%
\pgfsetlinewidth{1.505625pt}%
\definecolor{currentstroke}{rgb}{1.000000,0.000000,0.000000}%
\pgfsetstrokecolor{currentstroke}%
\pgfsetdash{}{0pt}%
\pgfpathmoveto{\pgfqpoint{0.873667in}{1.477258in}}%
\pgfpathlineto{\pgfqpoint{0.893342in}{1.446892in}}%
\pgfusepath{stroke}%
\end{pgfscope}%
\begin{pgfscope}%
\pgfpathrectangle{\pgfqpoint{0.100000in}{0.212622in}}{\pgfqpoint{3.696000in}{3.696000in}}%
\pgfusepath{clip}%
\pgfsetrectcap%
\pgfsetroundjoin%
\pgfsetlinewidth{1.505625pt}%
\definecolor{currentstroke}{rgb}{1.000000,0.000000,0.000000}%
\pgfsetstrokecolor{currentstroke}%
\pgfsetdash{}{0pt}%
\pgfpathmoveto{\pgfqpoint{0.873669in}{1.477260in}}%
\pgfpathlineto{\pgfqpoint{0.893342in}{1.446892in}}%
\pgfusepath{stroke}%
\end{pgfscope}%
\begin{pgfscope}%
\pgfpathrectangle{\pgfqpoint{0.100000in}{0.212622in}}{\pgfqpoint{3.696000in}{3.696000in}}%
\pgfusepath{clip}%
\pgfsetrectcap%
\pgfsetroundjoin%
\pgfsetlinewidth{1.505625pt}%
\definecolor{currentstroke}{rgb}{1.000000,0.000000,0.000000}%
\pgfsetstrokecolor{currentstroke}%
\pgfsetdash{}{0pt}%
\pgfpathmoveto{\pgfqpoint{0.873670in}{1.477260in}}%
\pgfpathlineto{\pgfqpoint{0.893342in}{1.446892in}}%
\pgfusepath{stroke}%
\end{pgfscope}%
\begin{pgfscope}%
\pgfpathrectangle{\pgfqpoint{0.100000in}{0.212622in}}{\pgfqpoint{3.696000in}{3.696000in}}%
\pgfusepath{clip}%
\pgfsetrectcap%
\pgfsetroundjoin%
\pgfsetlinewidth{1.505625pt}%
\definecolor{currentstroke}{rgb}{1.000000,0.000000,0.000000}%
\pgfsetstrokecolor{currentstroke}%
\pgfsetdash{}{0pt}%
\pgfpathmoveto{\pgfqpoint{0.873671in}{1.477260in}}%
\pgfpathlineto{\pgfqpoint{0.893342in}{1.446892in}}%
\pgfusepath{stroke}%
\end{pgfscope}%
\begin{pgfscope}%
\pgfpathrectangle{\pgfqpoint{0.100000in}{0.212622in}}{\pgfqpoint{3.696000in}{3.696000in}}%
\pgfusepath{clip}%
\pgfsetrectcap%
\pgfsetroundjoin%
\pgfsetlinewidth{1.505625pt}%
\definecolor{currentstroke}{rgb}{1.000000,0.000000,0.000000}%
\pgfsetstrokecolor{currentstroke}%
\pgfsetdash{}{0pt}%
\pgfpathmoveto{\pgfqpoint{0.873671in}{1.477261in}}%
\pgfpathlineto{\pgfqpoint{0.893342in}{1.446892in}}%
\pgfusepath{stroke}%
\end{pgfscope}%
\begin{pgfscope}%
\pgfpathrectangle{\pgfqpoint{0.100000in}{0.212622in}}{\pgfqpoint{3.696000in}{3.696000in}}%
\pgfusepath{clip}%
\pgfsetrectcap%
\pgfsetroundjoin%
\pgfsetlinewidth{1.505625pt}%
\definecolor{currentstroke}{rgb}{1.000000,0.000000,0.000000}%
\pgfsetstrokecolor{currentstroke}%
\pgfsetdash{}{0pt}%
\pgfpathmoveto{\pgfqpoint{0.873671in}{1.477261in}}%
\pgfpathlineto{\pgfqpoint{0.893342in}{1.446892in}}%
\pgfusepath{stroke}%
\end{pgfscope}%
\begin{pgfscope}%
\pgfpathrectangle{\pgfqpoint{0.100000in}{0.212622in}}{\pgfqpoint{3.696000in}{3.696000in}}%
\pgfusepath{clip}%
\pgfsetrectcap%
\pgfsetroundjoin%
\pgfsetlinewidth{1.505625pt}%
\definecolor{currentstroke}{rgb}{1.000000,0.000000,0.000000}%
\pgfsetstrokecolor{currentstroke}%
\pgfsetdash{}{0pt}%
\pgfpathmoveto{\pgfqpoint{0.873671in}{1.477261in}}%
\pgfpathlineto{\pgfqpoint{0.893342in}{1.446892in}}%
\pgfusepath{stroke}%
\end{pgfscope}%
\begin{pgfscope}%
\pgfpathrectangle{\pgfqpoint{0.100000in}{0.212622in}}{\pgfqpoint{3.696000in}{3.696000in}}%
\pgfusepath{clip}%
\pgfsetrectcap%
\pgfsetroundjoin%
\pgfsetlinewidth{1.505625pt}%
\definecolor{currentstroke}{rgb}{1.000000,0.000000,0.000000}%
\pgfsetstrokecolor{currentstroke}%
\pgfsetdash{}{0pt}%
\pgfpathmoveto{\pgfqpoint{0.873671in}{1.477261in}}%
\pgfpathlineto{\pgfqpoint{0.893342in}{1.446892in}}%
\pgfusepath{stroke}%
\end{pgfscope}%
\begin{pgfscope}%
\pgfpathrectangle{\pgfqpoint{0.100000in}{0.212622in}}{\pgfqpoint{3.696000in}{3.696000in}}%
\pgfusepath{clip}%
\pgfsetrectcap%
\pgfsetroundjoin%
\pgfsetlinewidth{1.505625pt}%
\definecolor{currentstroke}{rgb}{1.000000,0.000000,0.000000}%
\pgfsetstrokecolor{currentstroke}%
\pgfsetdash{}{0pt}%
\pgfpathmoveto{\pgfqpoint{0.874753in}{1.476722in}}%
\pgfpathlineto{\pgfqpoint{0.893342in}{1.446892in}}%
\pgfusepath{stroke}%
\end{pgfscope}%
\begin{pgfscope}%
\pgfpathrectangle{\pgfqpoint{0.100000in}{0.212622in}}{\pgfqpoint{3.696000in}{3.696000in}}%
\pgfusepath{clip}%
\pgfsetrectcap%
\pgfsetroundjoin%
\pgfsetlinewidth{1.505625pt}%
\definecolor{currentstroke}{rgb}{1.000000,0.000000,0.000000}%
\pgfsetstrokecolor{currentstroke}%
\pgfsetdash{}{0pt}%
\pgfpathmoveto{\pgfqpoint{0.878188in}{1.476788in}}%
\pgfpathlineto{\pgfqpoint{0.893342in}{1.446892in}}%
\pgfusepath{stroke}%
\end{pgfscope}%
\begin{pgfscope}%
\pgfpathrectangle{\pgfqpoint{0.100000in}{0.212622in}}{\pgfqpoint{3.696000in}{3.696000in}}%
\pgfusepath{clip}%
\pgfsetrectcap%
\pgfsetroundjoin%
\pgfsetlinewidth{1.505625pt}%
\definecolor{currentstroke}{rgb}{1.000000,0.000000,0.000000}%
\pgfsetstrokecolor{currentstroke}%
\pgfsetdash{}{0pt}%
\pgfpathmoveto{\pgfqpoint{0.881627in}{1.477764in}}%
\pgfpathlineto{\pgfqpoint{0.902881in}{1.454959in}}%
\pgfusepath{stroke}%
\end{pgfscope}%
\begin{pgfscope}%
\pgfpathrectangle{\pgfqpoint{0.100000in}{0.212622in}}{\pgfqpoint{3.696000in}{3.696000in}}%
\pgfusepath{clip}%
\pgfsetrectcap%
\pgfsetroundjoin%
\pgfsetlinewidth{1.505625pt}%
\definecolor{currentstroke}{rgb}{1.000000,0.000000,0.000000}%
\pgfsetstrokecolor{currentstroke}%
\pgfsetdash{}{0pt}%
\pgfpathmoveto{\pgfqpoint{0.890019in}{1.478277in}}%
\pgfpathlineto{\pgfqpoint{0.902881in}{1.454959in}}%
\pgfusepath{stroke}%
\end{pgfscope}%
\begin{pgfscope}%
\pgfpathrectangle{\pgfqpoint{0.100000in}{0.212622in}}{\pgfqpoint{3.696000in}{3.696000in}}%
\pgfusepath{clip}%
\pgfsetrectcap%
\pgfsetroundjoin%
\pgfsetlinewidth{1.505625pt}%
\definecolor{currentstroke}{rgb}{1.000000,0.000000,0.000000}%
\pgfsetstrokecolor{currentstroke}%
\pgfsetdash{}{0pt}%
\pgfpathmoveto{\pgfqpoint{0.896622in}{1.485591in}}%
\pgfpathlineto{\pgfqpoint{0.912407in}{1.463016in}}%
\pgfusepath{stroke}%
\end{pgfscope}%
\begin{pgfscope}%
\pgfpathrectangle{\pgfqpoint{0.100000in}{0.212622in}}{\pgfqpoint{3.696000in}{3.696000in}}%
\pgfusepath{clip}%
\pgfsetrectcap%
\pgfsetroundjoin%
\pgfsetlinewidth{1.505625pt}%
\definecolor{currentstroke}{rgb}{1.000000,0.000000,0.000000}%
\pgfsetstrokecolor{currentstroke}%
\pgfsetdash{}{0pt}%
\pgfpathmoveto{\pgfqpoint{0.900798in}{1.488331in}}%
\pgfpathlineto{\pgfqpoint{0.921921in}{1.471061in}}%
\pgfusepath{stroke}%
\end{pgfscope}%
\begin{pgfscope}%
\pgfpathrectangle{\pgfqpoint{0.100000in}{0.212622in}}{\pgfqpoint{3.696000in}{3.696000in}}%
\pgfusepath{clip}%
\pgfsetrectcap%
\pgfsetroundjoin%
\pgfsetlinewidth{1.505625pt}%
\definecolor{currentstroke}{rgb}{1.000000,0.000000,0.000000}%
\pgfsetstrokecolor{currentstroke}%
\pgfsetdash{}{0pt}%
\pgfpathmoveto{\pgfqpoint{0.903552in}{1.490817in}}%
\pgfpathlineto{\pgfqpoint{0.921921in}{1.471061in}}%
\pgfusepath{stroke}%
\end{pgfscope}%
\begin{pgfscope}%
\pgfpathrectangle{\pgfqpoint{0.100000in}{0.212622in}}{\pgfqpoint{3.696000in}{3.696000in}}%
\pgfusepath{clip}%
\pgfsetrectcap%
\pgfsetroundjoin%
\pgfsetlinewidth{1.505625pt}%
\definecolor{currentstroke}{rgb}{1.000000,0.000000,0.000000}%
\pgfsetstrokecolor{currentstroke}%
\pgfsetdash{}{0pt}%
\pgfpathmoveto{\pgfqpoint{0.906328in}{1.489967in}}%
\pgfpathlineto{\pgfqpoint{0.921921in}{1.471061in}}%
\pgfusepath{stroke}%
\end{pgfscope}%
\begin{pgfscope}%
\pgfpathrectangle{\pgfqpoint{0.100000in}{0.212622in}}{\pgfqpoint{3.696000in}{3.696000in}}%
\pgfusepath{clip}%
\pgfsetrectcap%
\pgfsetroundjoin%
\pgfsetlinewidth{1.505625pt}%
\definecolor{currentstroke}{rgb}{1.000000,0.000000,0.000000}%
\pgfsetstrokecolor{currentstroke}%
\pgfsetdash{}{0pt}%
\pgfpathmoveto{\pgfqpoint{0.912270in}{1.491427in}}%
\pgfpathlineto{\pgfqpoint{0.931422in}{1.479097in}}%
\pgfusepath{stroke}%
\end{pgfscope}%
\begin{pgfscope}%
\pgfpathrectangle{\pgfqpoint{0.100000in}{0.212622in}}{\pgfqpoint{3.696000in}{3.696000in}}%
\pgfusepath{clip}%
\pgfsetrectcap%
\pgfsetroundjoin%
\pgfsetlinewidth{1.505625pt}%
\definecolor{currentstroke}{rgb}{1.000000,0.000000,0.000000}%
\pgfsetstrokecolor{currentstroke}%
\pgfsetdash{}{0pt}%
\pgfpathmoveto{\pgfqpoint{0.917662in}{1.493851in}}%
\pgfpathlineto{\pgfqpoint{0.940911in}{1.487121in}}%
\pgfusepath{stroke}%
\end{pgfscope}%
\begin{pgfscope}%
\pgfpathrectangle{\pgfqpoint{0.100000in}{0.212622in}}{\pgfqpoint{3.696000in}{3.696000in}}%
\pgfusepath{clip}%
\pgfsetrectcap%
\pgfsetroundjoin%
\pgfsetlinewidth{1.505625pt}%
\definecolor{currentstroke}{rgb}{1.000000,0.000000,0.000000}%
\pgfsetstrokecolor{currentstroke}%
\pgfsetdash{}{0pt}%
\pgfpathmoveto{\pgfqpoint{0.927578in}{1.495554in}}%
\pgfpathlineto{\pgfqpoint{0.940911in}{1.487121in}}%
\pgfusepath{stroke}%
\end{pgfscope}%
\begin{pgfscope}%
\pgfpathrectangle{\pgfqpoint{0.100000in}{0.212622in}}{\pgfqpoint{3.696000in}{3.696000in}}%
\pgfusepath{clip}%
\pgfsetrectcap%
\pgfsetroundjoin%
\pgfsetlinewidth{1.505625pt}%
\definecolor{currentstroke}{rgb}{1.000000,0.000000,0.000000}%
\pgfsetstrokecolor{currentstroke}%
\pgfsetdash{}{0pt}%
\pgfpathmoveto{\pgfqpoint{0.935004in}{1.505759in}}%
\pgfpathlineto{\pgfqpoint{0.950387in}{1.495135in}}%
\pgfusepath{stroke}%
\end{pgfscope}%
\begin{pgfscope}%
\pgfpathrectangle{\pgfqpoint{0.100000in}{0.212622in}}{\pgfqpoint{3.696000in}{3.696000in}}%
\pgfusepath{clip}%
\pgfsetrectcap%
\pgfsetroundjoin%
\pgfsetlinewidth{1.505625pt}%
\definecolor{currentstroke}{rgb}{1.000000,0.000000,0.000000}%
\pgfsetstrokecolor{currentstroke}%
\pgfsetdash{}{0pt}%
\pgfpathmoveto{\pgfqpoint{0.939828in}{1.508427in}}%
\pgfpathlineto{\pgfqpoint{0.959850in}{1.503139in}}%
\pgfusepath{stroke}%
\end{pgfscope}%
\begin{pgfscope}%
\pgfpathrectangle{\pgfqpoint{0.100000in}{0.212622in}}{\pgfqpoint{3.696000in}{3.696000in}}%
\pgfusepath{clip}%
\pgfsetrectcap%
\pgfsetroundjoin%
\pgfsetlinewidth{1.505625pt}%
\definecolor{currentstroke}{rgb}{1.000000,0.000000,0.000000}%
\pgfsetstrokecolor{currentstroke}%
\pgfsetdash{}{0pt}%
\pgfpathmoveto{\pgfqpoint{0.942860in}{1.511525in}}%
\pgfpathlineto{\pgfqpoint{0.959850in}{1.503139in}}%
\pgfusepath{stroke}%
\end{pgfscope}%
\begin{pgfscope}%
\pgfpathrectangle{\pgfqpoint{0.100000in}{0.212622in}}{\pgfqpoint{3.696000in}{3.696000in}}%
\pgfusepath{clip}%
\pgfsetrectcap%
\pgfsetroundjoin%
\pgfsetlinewidth{1.505625pt}%
\definecolor{currentstroke}{rgb}{1.000000,0.000000,0.000000}%
\pgfsetstrokecolor{currentstroke}%
\pgfsetdash{}{0pt}%
\pgfpathmoveto{\pgfqpoint{0.945090in}{1.511298in}}%
\pgfpathlineto{\pgfqpoint{0.959850in}{1.503139in}}%
\pgfusepath{stroke}%
\end{pgfscope}%
\begin{pgfscope}%
\pgfpathrectangle{\pgfqpoint{0.100000in}{0.212622in}}{\pgfqpoint{3.696000in}{3.696000in}}%
\pgfusepath{clip}%
\pgfsetrectcap%
\pgfsetroundjoin%
\pgfsetlinewidth{1.505625pt}%
\definecolor{currentstroke}{rgb}{1.000000,0.000000,0.000000}%
\pgfsetstrokecolor{currentstroke}%
\pgfsetdash{}{0pt}%
\pgfpathmoveto{\pgfqpoint{0.951508in}{1.512180in}}%
\pgfpathlineto{\pgfqpoint{0.969302in}{1.511132in}}%
\pgfusepath{stroke}%
\end{pgfscope}%
\begin{pgfscope}%
\pgfpathrectangle{\pgfqpoint{0.100000in}{0.212622in}}{\pgfqpoint{3.696000in}{3.696000in}}%
\pgfusepath{clip}%
\pgfsetrectcap%
\pgfsetroundjoin%
\pgfsetlinewidth{1.505625pt}%
\definecolor{currentstroke}{rgb}{1.000000,0.000000,0.000000}%
\pgfsetstrokecolor{currentstroke}%
\pgfsetdash{}{0pt}%
\pgfpathmoveto{\pgfqpoint{0.956692in}{1.515328in}}%
\pgfpathlineto{\pgfqpoint{0.978740in}{1.519114in}}%
\pgfusepath{stroke}%
\end{pgfscope}%
\begin{pgfscope}%
\pgfpathrectangle{\pgfqpoint{0.100000in}{0.212622in}}{\pgfqpoint{3.696000in}{3.696000in}}%
\pgfusepath{clip}%
\pgfsetrectcap%
\pgfsetroundjoin%
\pgfsetlinewidth{1.505625pt}%
\definecolor{currentstroke}{rgb}{1.000000,0.000000,0.000000}%
\pgfsetstrokecolor{currentstroke}%
\pgfsetdash{}{0pt}%
\pgfpathmoveto{\pgfqpoint{0.965392in}{1.521781in}}%
\pgfpathlineto{\pgfqpoint{0.978740in}{1.519114in}}%
\pgfusepath{stroke}%
\end{pgfscope}%
\begin{pgfscope}%
\pgfpathrectangle{\pgfqpoint{0.100000in}{0.212622in}}{\pgfqpoint{3.696000in}{3.696000in}}%
\pgfusepath{clip}%
\pgfsetrectcap%
\pgfsetroundjoin%
\pgfsetlinewidth{1.505625pt}%
\definecolor{currentstroke}{rgb}{1.000000,0.000000,0.000000}%
\pgfsetstrokecolor{currentstroke}%
\pgfsetdash{}{0pt}%
\pgfpathmoveto{\pgfqpoint{0.974112in}{1.534386in}}%
\pgfpathlineto{\pgfqpoint{0.988167in}{1.527086in}}%
\pgfusepath{stroke}%
\end{pgfscope}%
\begin{pgfscope}%
\pgfpathrectangle{\pgfqpoint{0.100000in}{0.212622in}}{\pgfqpoint{3.696000in}{3.696000in}}%
\pgfusepath{clip}%
\pgfsetrectcap%
\pgfsetroundjoin%
\pgfsetlinewidth{1.505625pt}%
\definecolor{currentstroke}{rgb}{1.000000,0.000000,0.000000}%
\pgfsetstrokecolor{currentstroke}%
\pgfsetdash{}{0pt}%
\pgfpathmoveto{\pgfqpoint{0.982120in}{1.537389in}}%
\pgfpathlineto{\pgfqpoint{0.997580in}{1.535048in}}%
\pgfusepath{stroke}%
\end{pgfscope}%
\begin{pgfscope}%
\pgfpathrectangle{\pgfqpoint{0.100000in}{0.212622in}}{\pgfqpoint{3.696000in}{3.696000in}}%
\pgfusepath{clip}%
\pgfsetrectcap%
\pgfsetroundjoin%
\pgfsetlinewidth{1.505625pt}%
\definecolor{currentstroke}{rgb}{1.000000,0.000000,0.000000}%
\pgfsetstrokecolor{currentstroke}%
\pgfsetdash{}{0pt}%
\pgfpathmoveto{\pgfqpoint{0.995565in}{1.544747in}}%
\pgfpathlineto{\pgfqpoint{1.006982in}{1.542999in}}%
\pgfusepath{stroke}%
\end{pgfscope}%
\begin{pgfscope}%
\pgfpathrectangle{\pgfqpoint{0.100000in}{0.212622in}}{\pgfqpoint{3.696000in}{3.696000in}}%
\pgfusepath{clip}%
\pgfsetrectcap%
\pgfsetroundjoin%
\pgfsetlinewidth{1.505625pt}%
\definecolor{currentstroke}{rgb}{1.000000,0.000000,0.000000}%
\pgfsetstrokecolor{currentstroke}%
\pgfsetdash{}{0pt}%
\pgfpathmoveto{\pgfqpoint{1.007838in}{1.552334in}}%
\pgfpathlineto{\pgfqpoint{1.016371in}{1.550940in}}%
\pgfusepath{stroke}%
\end{pgfscope}%
\begin{pgfscope}%
\pgfpathrectangle{\pgfqpoint{0.100000in}{0.212622in}}{\pgfqpoint{3.696000in}{3.696000in}}%
\pgfusepath{clip}%
\pgfsetrectcap%
\pgfsetroundjoin%
\pgfsetlinewidth{1.505625pt}%
\definecolor{currentstroke}{rgb}{1.000000,0.000000,0.000000}%
\pgfsetstrokecolor{currentstroke}%
\pgfsetdash{}{0pt}%
\pgfpathmoveto{\pgfqpoint{1.014355in}{1.556858in}}%
\pgfpathlineto{\pgfqpoint{1.025748in}{1.558870in}}%
\pgfusepath{stroke}%
\end{pgfscope}%
\begin{pgfscope}%
\pgfpathrectangle{\pgfqpoint{0.100000in}{0.212622in}}{\pgfqpoint{3.696000in}{3.696000in}}%
\pgfusepath{clip}%
\pgfsetrectcap%
\pgfsetroundjoin%
\pgfsetlinewidth{1.505625pt}%
\definecolor{currentstroke}{rgb}{1.000000,0.000000,0.000000}%
\pgfsetstrokecolor{currentstroke}%
\pgfsetdash{}{0pt}%
\pgfpathmoveto{\pgfqpoint{1.022715in}{1.564010in}}%
\pgfpathlineto{\pgfqpoint{1.035113in}{1.566790in}}%
\pgfusepath{stroke}%
\end{pgfscope}%
\begin{pgfscope}%
\pgfpathrectangle{\pgfqpoint{0.100000in}{0.212622in}}{\pgfqpoint{3.696000in}{3.696000in}}%
\pgfusepath{clip}%
\pgfsetrectcap%
\pgfsetroundjoin%
\pgfsetlinewidth{1.505625pt}%
\definecolor{currentstroke}{rgb}{1.000000,0.000000,0.000000}%
\pgfsetstrokecolor{currentstroke}%
\pgfsetdash{}{0pt}%
\pgfpathmoveto{\pgfqpoint{1.030863in}{1.568559in}}%
\pgfpathlineto{\pgfqpoint{1.044465in}{1.574699in}}%
\pgfusepath{stroke}%
\end{pgfscope}%
\begin{pgfscope}%
\pgfpathrectangle{\pgfqpoint{0.100000in}{0.212622in}}{\pgfqpoint{3.696000in}{3.696000in}}%
\pgfusepath{clip}%
\pgfsetrectcap%
\pgfsetroundjoin%
\pgfsetlinewidth{1.505625pt}%
\definecolor{currentstroke}{rgb}{1.000000,0.000000,0.000000}%
\pgfsetstrokecolor{currentstroke}%
\pgfsetdash{}{0pt}%
\pgfpathmoveto{\pgfqpoint{1.041499in}{1.577662in}}%
\pgfpathlineto{\pgfqpoint{1.053805in}{1.582598in}}%
\pgfusepath{stroke}%
\end{pgfscope}%
\begin{pgfscope}%
\pgfpathrectangle{\pgfqpoint{0.100000in}{0.212622in}}{\pgfqpoint{3.696000in}{3.696000in}}%
\pgfusepath{clip}%
\pgfsetrectcap%
\pgfsetroundjoin%
\pgfsetlinewidth{1.505625pt}%
\definecolor{currentstroke}{rgb}{1.000000,0.000000,0.000000}%
\pgfsetstrokecolor{currentstroke}%
\pgfsetdash{}{0pt}%
\pgfpathmoveto{\pgfqpoint{1.052356in}{1.591388in}}%
\pgfpathlineto{\pgfqpoint{1.063133in}{1.590487in}}%
\pgfusepath{stroke}%
\end{pgfscope}%
\begin{pgfscope}%
\pgfpathrectangle{\pgfqpoint{0.100000in}{0.212622in}}{\pgfqpoint{3.696000in}{3.696000in}}%
\pgfusepath{clip}%
\pgfsetrectcap%
\pgfsetroundjoin%
\pgfsetlinewidth{1.505625pt}%
\definecolor{currentstroke}{rgb}{1.000000,0.000000,0.000000}%
\pgfsetstrokecolor{currentstroke}%
\pgfsetdash{}{0pt}%
\pgfpathmoveto{\pgfqpoint{1.062834in}{1.597818in}}%
\pgfpathlineto{\pgfqpoint{1.072449in}{1.598365in}}%
\pgfusepath{stroke}%
\end{pgfscope}%
\begin{pgfscope}%
\pgfpathrectangle{\pgfqpoint{0.100000in}{0.212622in}}{\pgfqpoint{3.696000in}{3.696000in}}%
\pgfusepath{clip}%
\pgfsetrectcap%
\pgfsetroundjoin%
\pgfsetlinewidth{1.505625pt}%
\definecolor{currentstroke}{rgb}{1.000000,0.000000,0.000000}%
\pgfsetstrokecolor{currentstroke}%
\pgfsetdash{}{0pt}%
\pgfpathmoveto{\pgfqpoint{1.070147in}{1.606013in}}%
\pgfpathlineto{\pgfqpoint{1.081752in}{1.606233in}}%
\pgfusepath{stroke}%
\end{pgfscope}%
\begin{pgfscope}%
\pgfpathrectangle{\pgfqpoint{0.100000in}{0.212622in}}{\pgfqpoint{3.696000in}{3.696000in}}%
\pgfusepath{clip}%
\pgfsetrectcap%
\pgfsetroundjoin%
\pgfsetlinewidth{1.505625pt}%
\definecolor{currentstroke}{rgb}{1.000000,0.000000,0.000000}%
\pgfsetstrokecolor{currentstroke}%
\pgfsetdash{}{0pt}%
\pgfpathmoveto{\pgfqpoint{1.074572in}{1.610432in}}%
\pgfpathlineto{\pgfqpoint{1.081752in}{1.606233in}}%
\pgfusepath{stroke}%
\end{pgfscope}%
\begin{pgfscope}%
\pgfpathrectangle{\pgfqpoint{0.100000in}{0.212622in}}{\pgfqpoint{3.696000in}{3.696000in}}%
\pgfusepath{clip}%
\pgfsetrectcap%
\pgfsetroundjoin%
\pgfsetlinewidth{1.505625pt}%
\definecolor{currentstroke}{rgb}{1.000000,0.000000,0.000000}%
\pgfsetstrokecolor{currentstroke}%
\pgfsetdash{}{0pt}%
\pgfpathmoveto{\pgfqpoint{1.079664in}{1.612620in}}%
\pgfpathlineto{\pgfqpoint{1.091044in}{1.614091in}}%
\pgfusepath{stroke}%
\end{pgfscope}%
\begin{pgfscope}%
\pgfpathrectangle{\pgfqpoint{0.100000in}{0.212622in}}{\pgfqpoint{3.696000in}{3.696000in}}%
\pgfusepath{clip}%
\pgfsetrectcap%
\pgfsetroundjoin%
\pgfsetlinewidth{1.505625pt}%
\definecolor{currentstroke}{rgb}{1.000000,0.000000,0.000000}%
\pgfsetstrokecolor{currentstroke}%
\pgfsetdash{}{0pt}%
\pgfpathmoveto{\pgfqpoint{1.087622in}{1.625463in}}%
\pgfpathlineto{\pgfqpoint{1.091044in}{1.614091in}}%
\pgfusepath{stroke}%
\end{pgfscope}%
\begin{pgfscope}%
\pgfpathrectangle{\pgfqpoint{0.100000in}{0.212622in}}{\pgfqpoint{3.696000in}{3.696000in}}%
\pgfusepath{clip}%
\pgfsetrectcap%
\pgfsetroundjoin%
\pgfsetlinewidth{1.505625pt}%
\definecolor{currentstroke}{rgb}{1.000000,0.000000,0.000000}%
\pgfsetstrokecolor{currentstroke}%
\pgfsetdash{}{0pt}%
\pgfpathmoveto{\pgfqpoint{1.096668in}{1.631916in}}%
\pgfpathlineto{\pgfqpoint{1.100323in}{1.621939in}}%
\pgfusepath{stroke}%
\end{pgfscope}%
\begin{pgfscope}%
\pgfpathrectangle{\pgfqpoint{0.100000in}{0.212622in}}{\pgfqpoint{3.696000in}{3.696000in}}%
\pgfusepath{clip}%
\pgfsetrectcap%
\pgfsetroundjoin%
\pgfsetlinewidth{1.505625pt}%
\definecolor{currentstroke}{rgb}{1.000000,0.000000,0.000000}%
\pgfsetstrokecolor{currentstroke}%
\pgfsetdash{}{0pt}%
\pgfpathmoveto{\pgfqpoint{1.104023in}{1.632000in}}%
\pgfpathlineto{\pgfqpoint{1.109590in}{1.629776in}}%
\pgfusepath{stroke}%
\end{pgfscope}%
\begin{pgfscope}%
\pgfpathrectangle{\pgfqpoint{0.100000in}{0.212622in}}{\pgfqpoint{3.696000in}{3.696000in}}%
\pgfusepath{clip}%
\pgfsetrectcap%
\pgfsetroundjoin%
\pgfsetlinewidth{1.505625pt}%
\definecolor{currentstroke}{rgb}{1.000000,0.000000,0.000000}%
\pgfsetstrokecolor{currentstroke}%
\pgfsetdash{}{0pt}%
\pgfpathmoveto{\pgfqpoint{1.113944in}{1.641119in}}%
\pgfpathlineto{\pgfqpoint{1.118845in}{1.637603in}}%
\pgfusepath{stroke}%
\end{pgfscope}%
\begin{pgfscope}%
\pgfpathrectangle{\pgfqpoint{0.100000in}{0.212622in}}{\pgfqpoint{3.696000in}{3.696000in}}%
\pgfusepath{clip}%
\pgfsetrectcap%
\pgfsetroundjoin%
\pgfsetlinewidth{1.505625pt}%
\definecolor{currentstroke}{rgb}{1.000000,0.000000,0.000000}%
\pgfsetstrokecolor{currentstroke}%
\pgfsetdash{}{0pt}%
\pgfpathmoveto{\pgfqpoint{1.120203in}{1.646013in}}%
\pgfpathlineto{\pgfqpoint{1.118845in}{1.637603in}}%
\pgfusepath{stroke}%
\end{pgfscope}%
\begin{pgfscope}%
\pgfpathrectangle{\pgfqpoint{0.100000in}{0.212622in}}{\pgfqpoint{3.696000in}{3.696000in}}%
\pgfusepath{clip}%
\pgfsetrectcap%
\pgfsetroundjoin%
\pgfsetlinewidth{1.505625pt}%
\definecolor{currentstroke}{rgb}{1.000000,0.000000,0.000000}%
\pgfsetstrokecolor{currentstroke}%
\pgfsetdash{}{0pt}%
\pgfpathmoveto{\pgfqpoint{1.124330in}{1.643866in}}%
\pgfpathlineto{\pgfqpoint{1.128088in}{1.645420in}}%
\pgfusepath{stroke}%
\end{pgfscope}%
\begin{pgfscope}%
\pgfpathrectangle{\pgfqpoint{0.100000in}{0.212622in}}{\pgfqpoint{3.696000in}{3.696000in}}%
\pgfusepath{clip}%
\pgfsetrectcap%
\pgfsetroundjoin%
\pgfsetlinewidth{1.505625pt}%
\definecolor{currentstroke}{rgb}{1.000000,0.000000,0.000000}%
\pgfsetstrokecolor{currentstroke}%
\pgfsetdash{}{0pt}%
\pgfpathmoveto{\pgfqpoint{1.132686in}{1.648254in}}%
\pgfpathlineto{\pgfqpoint{1.128088in}{1.645420in}}%
\pgfusepath{stroke}%
\end{pgfscope}%
\begin{pgfscope}%
\pgfpathrectangle{\pgfqpoint{0.100000in}{0.212622in}}{\pgfqpoint{3.696000in}{3.696000in}}%
\pgfusepath{clip}%
\pgfsetrectcap%
\pgfsetroundjoin%
\pgfsetlinewidth{1.505625pt}%
\definecolor{currentstroke}{rgb}{1.000000,0.000000,0.000000}%
\pgfsetstrokecolor{currentstroke}%
\pgfsetdash{}{0pt}%
\pgfpathmoveto{\pgfqpoint{1.140665in}{1.655076in}}%
\pgfpathlineto{\pgfqpoint{1.137319in}{1.653227in}}%
\pgfusepath{stroke}%
\end{pgfscope}%
\begin{pgfscope}%
\pgfpathrectangle{\pgfqpoint{0.100000in}{0.212622in}}{\pgfqpoint{3.696000in}{3.696000in}}%
\pgfusepath{clip}%
\pgfsetrectcap%
\pgfsetroundjoin%
\pgfsetlinewidth{1.505625pt}%
\definecolor{currentstroke}{rgb}{1.000000,0.000000,0.000000}%
\pgfsetstrokecolor{currentstroke}%
\pgfsetdash{}{0pt}%
\pgfpathmoveto{\pgfqpoint{1.145164in}{1.659277in}}%
\pgfpathlineto{\pgfqpoint{1.146538in}{1.661024in}}%
\pgfusepath{stroke}%
\end{pgfscope}%
\begin{pgfscope}%
\pgfpathrectangle{\pgfqpoint{0.100000in}{0.212622in}}{\pgfqpoint{3.696000in}{3.696000in}}%
\pgfusepath{clip}%
\pgfsetrectcap%
\pgfsetroundjoin%
\pgfsetlinewidth{1.505625pt}%
\definecolor{currentstroke}{rgb}{1.000000,0.000000,0.000000}%
\pgfsetstrokecolor{currentstroke}%
\pgfsetdash{}{0pt}%
\pgfpathmoveto{\pgfqpoint{1.154185in}{1.662210in}}%
\pgfpathlineto{\pgfqpoint{1.146538in}{1.661024in}}%
\pgfusepath{stroke}%
\end{pgfscope}%
\begin{pgfscope}%
\pgfpathrectangle{\pgfqpoint{0.100000in}{0.212622in}}{\pgfqpoint{3.696000in}{3.696000in}}%
\pgfusepath{clip}%
\pgfsetrectcap%
\pgfsetroundjoin%
\pgfsetlinewidth{1.505625pt}%
\definecolor{currentstroke}{rgb}{1.000000,0.000000,0.000000}%
\pgfsetstrokecolor{currentstroke}%
\pgfsetdash{}{0pt}%
\pgfpathmoveto{\pgfqpoint{1.163162in}{1.664178in}}%
\pgfpathlineto{\pgfqpoint{1.155745in}{1.668811in}}%
\pgfusepath{stroke}%
\end{pgfscope}%
\begin{pgfscope}%
\pgfpathrectangle{\pgfqpoint{0.100000in}{0.212622in}}{\pgfqpoint{3.696000in}{3.696000in}}%
\pgfusepath{clip}%
\pgfsetrectcap%
\pgfsetroundjoin%
\pgfsetlinewidth{1.505625pt}%
\definecolor{currentstroke}{rgb}{1.000000,0.000000,0.000000}%
\pgfsetstrokecolor{currentstroke}%
\pgfsetdash{}{0pt}%
\pgfpathmoveto{\pgfqpoint{1.175789in}{1.682351in}}%
\pgfpathlineto{\pgfqpoint{1.164941in}{1.676587in}}%
\pgfusepath{stroke}%
\end{pgfscope}%
\begin{pgfscope}%
\pgfpathrectangle{\pgfqpoint{0.100000in}{0.212622in}}{\pgfqpoint{3.696000in}{3.696000in}}%
\pgfusepath{clip}%
\pgfsetrectcap%
\pgfsetroundjoin%
\pgfsetlinewidth{1.505625pt}%
\definecolor{currentstroke}{rgb}{1.000000,0.000000,0.000000}%
\pgfsetstrokecolor{currentstroke}%
\pgfsetdash{}{0pt}%
\pgfpathmoveto{\pgfqpoint{1.183241in}{1.686666in}}%
\pgfpathlineto{\pgfqpoint{1.174124in}{1.684353in}}%
\pgfusepath{stroke}%
\end{pgfscope}%
\begin{pgfscope}%
\pgfpathrectangle{\pgfqpoint{0.100000in}{0.212622in}}{\pgfqpoint{3.696000in}{3.696000in}}%
\pgfusepath{clip}%
\pgfsetrectcap%
\pgfsetroundjoin%
\pgfsetlinewidth{1.505625pt}%
\definecolor{currentstroke}{rgb}{1.000000,0.000000,0.000000}%
\pgfsetstrokecolor{currentstroke}%
\pgfsetdash{}{0pt}%
\pgfpathmoveto{\pgfqpoint{1.185873in}{1.686171in}}%
\pgfpathlineto{\pgfqpoint{1.174124in}{1.684353in}}%
\pgfusepath{stroke}%
\end{pgfscope}%
\begin{pgfscope}%
\pgfpathrectangle{\pgfqpoint{0.100000in}{0.212622in}}{\pgfqpoint{3.696000in}{3.696000in}}%
\pgfusepath{clip}%
\pgfsetrectcap%
\pgfsetroundjoin%
\pgfsetlinewidth{1.505625pt}%
\definecolor{currentstroke}{rgb}{1.000000,0.000000,0.000000}%
\pgfsetstrokecolor{currentstroke}%
\pgfsetdash{}{0pt}%
\pgfpathmoveto{\pgfqpoint{1.190662in}{1.690595in}}%
\pgfpathlineto{\pgfqpoint{1.183295in}{1.692110in}}%
\pgfusepath{stroke}%
\end{pgfscope}%
\begin{pgfscope}%
\pgfpathrectangle{\pgfqpoint{0.100000in}{0.212622in}}{\pgfqpoint{3.696000in}{3.696000in}}%
\pgfusepath{clip}%
\pgfsetrectcap%
\pgfsetroundjoin%
\pgfsetlinewidth{1.505625pt}%
\definecolor{currentstroke}{rgb}{1.000000,0.000000,0.000000}%
\pgfsetstrokecolor{currentstroke}%
\pgfsetdash{}{0pt}%
\pgfpathmoveto{\pgfqpoint{1.193685in}{1.692881in}}%
\pgfpathlineto{\pgfqpoint{1.183295in}{1.692110in}}%
\pgfusepath{stroke}%
\end{pgfscope}%
\begin{pgfscope}%
\pgfpathrectangle{\pgfqpoint{0.100000in}{0.212622in}}{\pgfqpoint{3.696000in}{3.696000in}}%
\pgfusepath{clip}%
\pgfsetrectcap%
\pgfsetroundjoin%
\pgfsetlinewidth{1.505625pt}%
\definecolor{currentstroke}{rgb}{1.000000,0.000000,0.000000}%
\pgfsetstrokecolor{currentstroke}%
\pgfsetdash{}{0pt}%
\pgfpathmoveto{\pgfqpoint{1.195053in}{1.692955in}}%
\pgfpathlineto{\pgfqpoint{1.183295in}{1.692110in}}%
\pgfusepath{stroke}%
\end{pgfscope}%
\begin{pgfscope}%
\pgfpathrectangle{\pgfqpoint{0.100000in}{0.212622in}}{\pgfqpoint{3.696000in}{3.696000in}}%
\pgfusepath{clip}%
\pgfsetrectcap%
\pgfsetroundjoin%
\pgfsetlinewidth{1.505625pt}%
\definecolor{currentstroke}{rgb}{1.000000,0.000000,0.000000}%
\pgfsetstrokecolor{currentstroke}%
\pgfsetdash{}{0pt}%
\pgfpathmoveto{\pgfqpoint{1.198879in}{1.698159in}}%
\pgfpathlineto{\pgfqpoint{1.183295in}{1.692110in}}%
\pgfusepath{stroke}%
\end{pgfscope}%
\begin{pgfscope}%
\pgfpathrectangle{\pgfqpoint{0.100000in}{0.212622in}}{\pgfqpoint{3.696000in}{3.696000in}}%
\pgfusepath{clip}%
\pgfsetrectcap%
\pgfsetroundjoin%
\pgfsetlinewidth{1.505625pt}%
\definecolor{currentstroke}{rgb}{1.000000,0.000000,0.000000}%
\pgfsetstrokecolor{currentstroke}%
\pgfsetdash{}{0pt}%
\pgfpathmoveto{\pgfqpoint{1.201039in}{1.700028in}}%
\pgfpathlineto{\pgfqpoint{1.183295in}{1.692110in}}%
\pgfusepath{stroke}%
\end{pgfscope}%
\begin{pgfscope}%
\pgfpathrectangle{\pgfqpoint{0.100000in}{0.212622in}}{\pgfqpoint{3.696000in}{3.696000in}}%
\pgfusepath{clip}%
\pgfsetrectcap%
\pgfsetroundjoin%
\pgfsetlinewidth{1.505625pt}%
\definecolor{currentstroke}{rgb}{1.000000,0.000000,0.000000}%
\pgfsetstrokecolor{currentstroke}%
\pgfsetdash{}{0pt}%
\pgfpathmoveto{\pgfqpoint{1.203488in}{1.701052in}}%
\pgfpathlineto{\pgfqpoint{1.192454in}{1.699856in}}%
\pgfusepath{stroke}%
\end{pgfscope}%
\begin{pgfscope}%
\pgfpathrectangle{\pgfqpoint{0.100000in}{0.212622in}}{\pgfqpoint{3.696000in}{3.696000in}}%
\pgfusepath{clip}%
\pgfsetrectcap%
\pgfsetroundjoin%
\pgfsetlinewidth{1.505625pt}%
\definecolor{currentstroke}{rgb}{1.000000,0.000000,0.000000}%
\pgfsetstrokecolor{currentstroke}%
\pgfsetdash{}{0pt}%
\pgfpathmoveto{\pgfqpoint{1.209360in}{1.708620in}}%
\pgfpathlineto{\pgfqpoint{1.192454in}{1.699856in}}%
\pgfusepath{stroke}%
\end{pgfscope}%
\begin{pgfscope}%
\pgfpathrectangle{\pgfqpoint{0.100000in}{0.212622in}}{\pgfqpoint{3.696000in}{3.696000in}}%
\pgfusepath{clip}%
\pgfsetrectcap%
\pgfsetroundjoin%
\pgfsetlinewidth{1.505625pt}%
\definecolor{currentstroke}{rgb}{1.000000,0.000000,0.000000}%
\pgfsetstrokecolor{currentstroke}%
\pgfsetdash{}{0pt}%
\pgfpathmoveto{\pgfqpoint{1.212305in}{1.712150in}}%
\pgfpathlineto{\pgfqpoint{1.192454in}{1.699856in}}%
\pgfusepath{stroke}%
\end{pgfscope}%
\begin{pgfscope}%
\pgfpathrectangle{\pgfqpoint{0.100000in}{0.212622in}}{\pgfqpoint{3.696000in}{3.696000in}}%
\pgfusepath{clip}%
\pgfsetrectcap%
\pgfsetroundjoin%
\pgfsetlinewidth{1.505625pt}%
\definecolor{currentstroke}{rgb}{1.000000,0.000000,0.000000}%
\pgfsetstrokecolor{currentstroke}%
\pgfsetdash{}{0pt}%
\pgfpathmoveto{\pgfqpoint{1.216164in}{1.714531in}}%
\pgfpathlineto{\pgfqpoint{1.201602in}{1.707592in}}%
\pgfusepath{stroke}%
\end{pgfscope}%
\begin{pgfscope}%
\pgfpathrectangle{\pgfqpoint{0.100000in}{0.212622in}}{\pgfqpoint{3.696000in}{3.696000in}}%
\pgfusepath{clip}%
\pgfsetrectcap%
\pgfsetroundjoin%
\pgfsetlinewidth{1.505625pt}%
\definecolor{currentstroke}{rgb}{1.000000,0.000000,0.000000}%
\pgfsetstrokecolor{currentstroke}%
\pgfsetdash{}{0pt}%
\pgfpathmoveto{\pgfqpoint{1.220841in}{1.718182in}}%
\pgfpathlineto{\pgfqpoint{1.201602in}{1.707592in}}%
\pgfusepath{stroke}%
\end{pgfscope}%
\begin{pgfscope}%
\pgfpathrectangle{\pgfqpoint{0.100000in}{0.212622in}}{\pgfqpoint{3.696000in}{3.696000in}}%
\pgfusepath{clip}%
\pgfsetrectcap%
\pgfsetroundjoin%
\pgfsetlinewidth{1.505625pt}%
\definecolor{currentstroke}{rgb}{1.000000,0.000000,0.000000}%
\pgfsetstrokecolor{currentstroke}%
\pgfsetdash{}{0pt}%
\pgfpathmoveto{\pgfqpoint{1.228427in}{1.724434in}}%
\pgfpathlineto{\pgfqpoint{1.210737in}{1.715318in}}%
\pgfusepath{stroke}%
\end{pgfscope}%
\begin{pgfscope}%
\pgfpathrectangle{\pgfqpoint{0.100000in}{0.212622in}}{\pgfqpoint{3.696000in}{3.696000in}}%
\pgfusepath{clip}%
\pgfsetrectcap%
\pgfsetroundjoin%
\pgfsetlinewidth{1.505625pt}%
\definecolor{currentstroke}{rgb}{1.000000,0.000000,0.000000}%
\pgfsetstrokecolor{currentstroke}%
\pgfsetdash{}{0pt}%
\pgfpathmoveto{\pgfqpoint{1.237089in}{1.727399in}}%
\pgfpathlineto{\pgfqpoint{1.219861in}{1.723034in}}%
\pgfusepath{stroke}%
\end{pgfscope}%
\begin{pgfscope}%
\pgfpathrectangle{\pgfqpoint{0.100000in}{0.212622in}}{\pgfqpoint{3.696000in}{3.696000in}}%
\pgfusepath{clip}%
\pgfsetrectcap%
\pgfsetroundjoin%
\pgfsetlinewidth{1.505625pt}%
\definecolor{currentstroke}{rgb}{1.000000,0.000000,0.000000}%
\pgfsetstrokecolor{currentstroke}%
\pgfsetdash{}{0pt}%
\pgfpathmoveto{\pgfqpoint{1.247559in}{1.733370in}}%
\pgfpathlineto{\pgfqpoint{1.228973in}{1.730740in}}%
\pgfusepath{stroke}%
\end{pgfscope}%
\begin{pgfscope}%
\pgfpathrectangle{\pgfqpoint{0.100000in}{0.212622in}}{\pgfqpoint{3.696000in}{3.696000in}}%
\pgfusepath{clip}%
\pgfsetrectcap%
\pgfsetroundjoin%
\pgfsetlinewidth{1.505625pt}%
\definecolor{currentstroke}{rgb}{1.000000,0.000000,0.000000}%
\pgfsetstrokecolor{currentstroke}%
\pgfsetdash{}{0pt}%
\pgfpathmoveto{\pgfqpoint{1.261178in}{1.740050in}}%
\pgfpathlineto{\pgfqpoint{1.238074in}{1.738437in}}%
\pgfusepath{stroke}%
\end{pgfscope}%
\begin{pgfscope}%
\pgfpathrectangle{\pgfqpoint{0.100000in}{0.212622in}}{\pgfqpoint{3.696000in}{3.696000in}}%
\pgfusepath{clip}%
\pgfsetrectcap%
\pgfsetroundjoin%
\pgfsetlinewidth{1.505625pt}%
\definecolor{currentstroke}{rgb}{1.000000,0.000000,0.000000}%
\pgfsetstrokecolor{currentstroke}%
\pgfsetdash{}{0pt}%
\pgfpathmoveto{\pgfqpoint{1.274407in}{1.744502in}}%
\pgfpathlineto{\pgfqpoint{1.247162in}{1.746123in}}%
\pgfusepath{stroke}%
\end{pgfscope}%
\begin{pgfscope}%
\pgfpathrectangle{\pgfqpoint{0.100000in}{0.212622in}}{\pgfqpoint{3.696000in}{3.696000in}}%
\pgfusepath{clip}%
\pgfsetrectcap%
\pgfsetroundjoin%
\pgfsetlinewidth{1.505625pt}%
\definecolor{currentstroke}{rgb}{1.000000,0.000000,0.000000}%
\pgfsetstrokecolor{currentstroke}%
\pgfsetdash{}{0pt}%
\pgfpathmoveto{\pgfqpoint{1.283576in}{1.748438in}}%
\pgfpathlineto{\pgfqpoint{1.256239in}{1.753799in}}%
\pgfusepath{stroke}%
\end{pgfscope}%
\begin{pgfscope}%
\pgfpathrectangle{\pgfqpoint{0.100000in}{0.212622in}}{\pgfqpoint{3.696000in}{3.696000in}}%
\pgfusepath{clip}%
\pgfsetrectcap%
\pgfsetroundjoin%
\pgfsetlinewidth{1.505625pt}%
\definecolor{currentstroke}{rgb}{1.000000,0.000000,0.000000}%
\pgfsetstrokecolor{currentstroke}%
\pgfsetdash{}{0pt}%
\pgfpathmoveto{\pgfqpoint{1.295085in}{1.756287in}}%
\pgfpathlineto{\pgfqpoint{1.265304in}{1.761466in}}%
\pgfusepath{stroke}%
\end{pgfscope}%
\begin{pgfscope}%
\pgfpathrectangle{\pgfqpoint{0.100000in}{0.212622in}}{\pgfqpoint{3.696000in}{3.696000in}}%
\pgfusepath{clip}%
\pgfsetrectcap%
\pgfsetroundjoin%
\pgfsetlinewidth{1.505625pt}%
\definecolor{currentstroke}{rgb}{1.000000,0.000000,0.000000}%
\pgfsetstrokecolor{currentstroke}%
\pgfsetdash{}{0pt}%
\pgfpathmoveto{\pgfqpoint{1.306094in}{1.761862in}}%
\pgfpathlineto{\pgfqpoint{1.274357in}{1.769122in}}%
\pgfusepath{stroke}%
\end{pgfscope}%
\begin{pgfscope}%
\pgfpathrectangle{\pgfqpoint{0.100000in}{0.212622in}}{\pgfqpoint{3.696000in}{3.696000in}}%
\pgfusepath{clip}%
\pgfsetrectcap%
\pgfsetroundjoin%
\pgfsetlinewidth{1.505625pt}%
\definecolor{currentstroke}{rgb}{1.000000,0.000000,0.000000}%
\pgfsetstrokecolor{currentstroke}%
\pgfsetdash{}{0pt}%
\pgfpathmoveto{\pgfqpoint{1.312772in}{1.763983in}}%
\pgfpathlineto{\pgfqpoint{1.283399in}{1.776769in}}%
\pgfusepath{stroke}%
\end{pgfscope}%
\begin{pgfscope}%
\pgfpathrectangle{\pgfqpoint{0.100000in}{0.212622in}}{\pgfqpoint{3.696000in}{3.696000in}}%
\pgfusepath{clip}%
\pgfsetrectcap%
\pgfsetroundjoin%
\pgfsetlinewidth{1.505625pt}%
\definecolor{currentstroke}{rgb}{1.000000,0.000000,0.000000}%
\pgfsetstrokecolor{currentstroke}%
\pgfsetdash{}{0pt}%
\pgfpathmoveto{\pgfqpoint{1.316653in}{1.766777in}}%
\pgfpathlineto{\pgfqpoint{1.283399in}{1.776769in}}%
\pgfusepath{stroke}%
\end{pgfscope}%
\begin{pgfscope}%
\pgfpathrectangle{\pgfqpoint{0.100000in}{0.212622in}}{\pgfqpoint{3.696000in}{3.696000in}}%
\pgfusepath{clip}%
\pgfsetrectcap%
\pgfsetroundjoin%
\pgfsetlinewidth{1.505625pt}%
\definecolor{currentstroke}{rgb}{1.000000,0.000000,0.000000}%
\pgfsetstrokecolor{currentstroke}%
\pgfsetdash{}{0pt}%
\pgfpathmoveto{\pgfqpoint{1.322089in}{1.768317in}}%
\pgfpathlineto{\pgfqpoint{1.283399in}{1.776769in}}%
\pgfusepath{stroke}%
\end{pgfscope}%
\begin{pgfscope}%
\pgfpathrectangle{\pgfqpoint{0.100000in}{0.212622in}}{\pgfqpoint{3.696000in}{3.696000in}}%
\pgfusepath{clip}%
\pgfsetrectcap%
\pgfsetroundjoin%
\pgfsetlinewidth{1.505625pt}%
\definecolor{currentstroke}{rgb}{1.000000,0.000000,0.000000}%
\pgfsetstrokecolor{currentstroke}%
\pgfsetdash{}{0pt}%
\pgfpathmoveto{\pgfqpoint{1.325218in}{1.770945in}}%
\pgfpathlineto{\pgfqpoint{1.292429in}{1.784406in}}%
\pgfusepath{stroke}%
\end{pgfscope}%
\begin{pgfscope}%
\pgfpathrectangle{\pgfqpoint{0.100000in}{0.212622in}}{\pgfqpoint{3.696000in}{3.696000in}}%
\pgfusepath{clip}%
\pgfsetrectcap%
\pgfsetroundjoin%
\pgfsetlinewidth{1.505625pt}%
\definecolor{currentstroke}{rgb}{1.000000,0.000000,0.000000}%
\pgfsetstrokecolor{currentstroke}%
\pgfsetdash{}{0pt}%
\pgfpathmoveto{\pgfqpoint{1.331351in}{1.774134in}}%
\pgfpathlineto{\pgfqpoint{1.292429in}{1.784406in}}%
\pgfusepath{stroke}%
\end{pgfscope}%
\begin{pgfscope}%
\pgfpathrectangle{\pgfqpoint{0.100000in}{0.212622in}}{\pgfqpoint{3.696000in}{3.696000in}}%
\pgfusepath{clip}%
\pgfsetrectcap%
\pgfsetroundjoin%
\pgfsetlinewidth{1.505625pt}%
\definecolor{currentstroke}{rgb}{1.000000,0.000000,0.000000}%
\pgfsetstrokecolor{currentstroke}%
\pgfsetdash{}{0pt}%
\pgfpathmoveto{\pgfqpoint{1.337973in}{1.775553in}}%
\pgfpathlineto{\pgfqpoint{1.301448in}{1.792033in}}%
\pgfusepath{stroke}%
\end{pgfscope}%
\begin{pgfscope}%
\pgfpathrectangle{\pgfqpoint{0.100000in}{0.212622in}}{\pgfqpoint{3.696000in}{3.696000in}}%
\pgfusepath{clip}%
\pgfsetrectcap%
\pgfsetroundjoin%
\pgfsetlinewidth{1.505625pt}%
\definecolor{currentstroke}{rgb}{1.000000,0.000000,0.000000}%
\pgfsetstrokecolor{currentstroke}%
\pgfsetdash{}{0pt}%
\pgfpathmoveto{\pgfqpoint{1.342213in}{1.779767in}}%
\pgfpathlineto{\pgfqpoint{1.301448in}{1.792033in}}%
\pgfusepath{stroke}%
\end{pgfscope}%
\begin{pgfscope}%
\pgfpathrectangle{\pgfqpoint{0.100000in}{0.212622in}}{\pgfqpoint{3.696000in}{3.696000in}}%
\pgfusepath{clip}%
\pgfsetrectcap%
\pgfsetroundjoin%
\pgfsetlinewidth{1.505625pt}%
\definecolor{currentstroke}{rgb}{1.000000,0.000000,0.000000}%
\pgfsetstrokecolor{currentstroke}%
\pgfsetdash{}{0pt}%
\pgfpathmoveto{\pgfqpoint{1.348715in}{1.781966in}}%
\pgfpathlineto{\pgfqpoint{1.310455in}{1.799650in}}%
\pgfusepath{stroke}%
\end{pgfscope}%
\begin{pgfscope}%
\pgfpathrectangle{\pgfqpoint{0.100000in}{0.212622in}}{\pgfqpoint{3.696000in}{3.696000in}}%
\pgfusepath{clip}%
\pgfsetrectcap%
\pgfsetroundjoin%
\pgfsetlinewidth{1.505625pt}%
\definecolor{currentstroke}{rgb}{1.000000,0.000000,0.000000}%
\pgfsetstrokecolor{currentstroke}%
\pgfsetdash{}{0pt}%
\pgfpathmoveto{\pgfqpoint{1.356464in}{1.784646in}}%
\pgfpathlineto{\pgfqpoint{1.319450in}{1.807258in}}%
\pgfusepath{stroke}%
\end{pgfscope}%
\begin{pgfscope}%
\pgfpathrectangle{\pgfqpoint{0.100000in}{0.212622in}}{\pgfqpoint{3.696000in}{3.696000in}}%
\pgfusepath{clip}%
\pgfsetrectcap%
\pgfsetroundjoin%
\pgfsetlinewidth{1.505625pt}%
\definecolor{currentstroke}{rgb}{1.000000,0.000000,0.000000}%
\pgfsetstrokecolor{currentstroke}%
\pgfsetdash{}{0pt}%
\pgfpathmoveto{\pgfqpoint{1.361612in}{1.787769in}}%
\pgfpathlineto{\pgfqpoint{1.319450in}{1.807258in}}%
\pgfusepath{stroke}%
\end{pgfscope}%
\begin{pgfscope}%
\pgfpathrectangle{\pgfqpoint{0.100000in}{0.212622in}}{\pgfqpoint{3.696000in}{3.696000in}}%
\pgfusepath{clip}%
\pgfsetrectcap%
\pgfsetroundjoin%
\pgfsetlinewidth{1.505625pt}%
\definecolor{currentstroke}{rgb}{1.000000,0.000000,0.000000}%
\pgfsetstrokecolor{currentstroke}%
\pgfsetdash{}{0pt}%
\pgfpathmoveto{\pgfqpoint{1.364436in}{1.790048in}}%
\pgfpathlineto{\pgfqpoint{1.319450in}{1.807258in}}%
\pgfusepath{stroke}%
\end{pgfscope}%
\begin{pgfscope}%
\pgfpathrectangle{\pgfqpoint{0.100000in}{0.212622in}}{\pgfqpoint{3.696000in}{3.696000in}}%
\pgfusepath{clip}%
\pgfsetrectcap%
\pgfsetroundjoin%
\pgfsetlinewidth{1.505625pt}%
\definecolor{currentstroke}{rgb}{1.000000,0.000000,0.000000}%
\pgfsetstrokecolor{currentstroke}%
\pgfsetdash{}{0pt}%
\pgfpathmoveto{\pgfqpoint{1.367126in}{1.791160in}}%
\pgfpathlineto{\pgfqpoint{1.328434in}{1.814856in}}%
\pgfusepath{stroke}%
\end{pgfscope}%
\begin{pgfscope}%
\pgfpathrectangle{\pgfqpoint{0.100000in}{0.212622in}}{\pgfqpoint{3.696000in}{3.696000in}}%
\pgfusepath{clip}%
\pgfsetrectcap%
\pgfsetroundjoin%
\pgfsetlinewidth{1.505625pt}%
\definecolor{currentstroke}{rgb}{1.000000,0.000000,0.000000}%
\pgfsetstrokecolor{currentstroke}%
\pgfsetdash{}{0pt}%
\pgfpathmoveto{\pgfqpoint{1.371037in}{1.794850in}}%
\pgfpathlineto{\pgfqpoint{1.328434in}{1.814856in}}%
\pgfusepath{stroke}%
\end{pgfscope}%
\begin{pgfscope}%
\pgfpathrectangle{\pgfqpoint{0.100000in}{0.212622in}}{\pgfqpoint{3.696000in}{3.696000in}}%
\pgfusepath{clip}%
\pgfsetrectcap%
\pgfsetroundjoin%
\pgfsetlinewidth{1.505625pt}%
\definecolor{currentstroke}{rgb}{1.000000,0.000000,0.000000}%
\pgfsetstrokecolor{currentstroke}%
\pgfsetdash{}{0pt}%
\pgfpathmoveto{\pgfqpoint{1.376489in}{1.798469in}}%
\pgfpathlineto{\pgfqpoint{1.337406in}{1.822444in}}%
\pgfusepath{stroke}%
\end{pgfscope}%
\begin{pgfscope}%
\pgfpathrectangle{\pgfqpoint{0.100000in}{0.212622in}}{\pgfqpoint{3.696000in}{3.696000in}}%
\pgfusepath{clip}%
\pgfsetrectcap%
\pgfsetroundjoin%
\pgfsetlinewidth{1.505625pt}%
\definecolor{currentstroke}{rgb}{1.000000,0.000000,0.000000}%
\pgfsetstrokecolor{currentstroke}%
\pgfsetdash{}{0pt}%
\pgfpathmoveto{\pgfqpoint{1.383144in}{1.802038in}}%
\pgfpathlineto{\pgfqpoint{1.337406in}{1.822444in}}%
\pgfusepath{stroke}%
\end{pgfscope}%
\begin{pgfscope}%
\pgfpathrectangle{\pgfqpoint{0.100000in}{0.212622in}}{\pgfqpoint{3.696000in}{3.696000in}}%
\pgfusepath{clip}%
\pgfsetrectcap%
\pgfsetroundjoin%
\pgfsetlinewidth{1.505625pt}%
\definecolor{currentstroke}{rgb}{1.000000,0.000000,0.000000}%
\pgfsetstrokecolor{currentstroke}%
\pgfsetdash{}{0pt}%
\pgfpathmoveto{\pgfqpoint{1.387627in}{1.804753in}}%
\pgfpathlineto{\pgfqpoint{1.346367in}{1.830022in}}%
\pgfusepath{stroke}%
\end{pgfscope}%
\begin{pgfscope}%
\pgfpathrectangle{\pgfqpoint{0.100000in}{0.212622in}}{\pgfqpoint{3.696000in}{3.696000in}}%
\pgfusepath{clip}%
\pgfsetrectcap%
\pgfsetroundjoin%
\pgfsetlinewidth{1.505625pt}%
\definecolor{currentstroke}{rgb}{1.000000,0.000000,0.000000}%
\pgfsetstrokecolor{currentstroke}%
\pgfsetdash{}{0pt}%
\pgfpathmoveto{\pgfqpoint{1.392960in}{1.811727in}}%
\pgfpathlineto{\pgfqpoint{1.346367in}{1.830022in}}%
\pgfusepath{stroke}%
\end{pgfscope}%
\begin{pgfscope}%
\pgfpathrectangle{\pgfqpoint{0.100000in}{0.212622in}}{\pgfqpoint{3.696000in}{3.696000in}}%
\pgfusepath{clip}%
\pgfsetrectcap%
\pgfsetroundjoin%
\pgfsetlinewidth{1.505625pt}%
\definecolor{currentstroke}{rgb}{1.000000,0.000000,0.000000}%
\pgfsetstrokecolor{currentstroke}%
\pgfsetdash{}{0pt}%
\pgfpathmoveto{\pgfqpoint{1.400928in}{1.815750in}}%
\pgfpathlineto{\pgfqpoint{1.355316in}{1.837590in}}%
\pgfusepath{stroke}%
\end{pgfscope}%
\begin{pgfscope}%
\pgfpathrectangle{\pgfqpoint{0.100000in}{0.212622in}}{\pgfqpoint{3.696000in}{3.696000in}}%
\pgfusepath{clip}%
\pgfsetrectcap%
\pgfsetroundjoin%
\pgfsetlinewidth{1.505625pt}%
\definecolor{currentstroke}{rgb}{1.000000,0.000000,0.000000}%
\pgfsetstrokecolor{currentstroke}%
\pgfsetdash{}{0pt}%
\pgfpathmoveto{\pgfqpoint{1.405181in}{1.819013in}}%
\pgfpathlineto{\pgfqpoint{1.355316in}{1.837590in}}%
\pgfusepath{stroke}%
\end{pgfscope}%
\begin{pgfscope}%
\pgfpathrectangle{\pgfqpoint{0.100000in}{0.212622in}}{\pgfqpoint{3.696000in}{3.696000in}}%
\pgfusepath{clip}%
\pgfsetrectcap%
\pgfsetroundjoin%
\pgfsetlinewidth{1.505625pt}%
\definecolor{currentstroke}{rgb}{1.000000,0.000000,0.000000}%
\pgfsetstrokecolor{currentstroke}%
\pgfsetdash{}{0pt}%
\pgfpathmoveto{\pgfqpoint{1.410496in}{1.822854in}}%
\pgfpathlineto{\pgfqpoint{1.364254in}{1.845149in}}%
\pgfusepath{stroke}%
\end{pgfscope}%
\begin{pgfscope}%
\pgfpathrectangle{\pgfqpoint{0.100000in}{0.212622in}}{\pgfqpoint{3.696000in}{3.696000in}}%
\pgfusepath{clip}%
\pgfsetrectcap%
\pgfsetroundjoin%
\pgfsetlinewidth{1.505625pt}%
\definecolor{currentstroke}{rgb}{1.000000,0.000000,0.000000}%
\pgfsetstrokecolor{currentstroke}%
\pgfsetdash{}{0pt}%
\pgfpathmoveto{\pgfqpoint{1.416246in}{1.824299in}}%
\pgfpathlineto{\pgfqpoint{1.373180in}{1.852698in}}%
\pgfusepath{stroke}%
\end{pgfscope}%
\begin{pgfscope}%
\pgfpathrectangle{\pgfqpoint{0.100000in}{0.212622in}}{\pgfqpoint{3.696000in}{3.696000in}}%
\pgfusepath{clip}%
\pgfsetrectcap%
\pgfsetroundjoin%
\pgfsetlinewidth{1.505625pt}%
\definecolor{currentstroke}{rgb}{1.000000,0.000000,0.000000}%
\pgfsetstrokecolor{currentstroke}%
\pgfsetdash{}{0pt}%
\pgfpathmoveto{\pgfqpoint{1.423673in}{1.828792in}}%
\pgfpathlineto{\pgfqpoint{1.373180in}{1.852698in}}%
\pgfusepath{stroke}%
\end{pgfscope}%
\begin{pgfscope}%
\pgfpathrectangle{\pgfqpoint{0.100000in}{0.212622in}}{\pgfqpoint{3.696000in}{3.696000in}}%
\pgfusepath{clip}%
\pgfsetrectcap%
\pgfsetroundjoin%
\pgfsetlinewidth{1.505625pt}%
\definecolor{currentstroke}{rgb}{1.000000,0.000000,0.000000}%
\pgfsetstrokecolor{currentstroke}%
\pgfsetdash{}{0pt}%
\pgfpathmoveto{\pgfqpoint{1.428003in}{1.833149in}}%
\pgfpathlineto{\pgfqpoint{1.382095in}{1.860238in}}%
\pgfusepath{stroke}%
\end{pgfscope}%
\begin{pgfscope}%
\pgfpathrectangle{\pgfqpoint{0.100000in}{0.212622in}}{\pgfqpoint{3.696000in}{3.696000in}}%
\pgfusepath{clip}%
\pgfsetrectcap%
\pgfsetroundjoin%
\pgfsetlinewidth{1.505625pt}%
\definecolor{currentstroke}{rgb}{1.000000,0.000000,0.000000}%
\pgfsetstrokecolor{currentstroke}%
\pgfsetdash{}{0pt}%
\pgfpathmoveto{\pgfqpoint{1.432886in}{1.835238in}}%
\pgfpathlineto{\pgfqpoint{1.382095in}{1.860238in}}%
\pgfusepath{stroke}%
\end{pgfscope}%
\begin{pgfscope}%
\pgfpathrectangle{\pgfqpoint{0.100000in}{0.212622in}}{\pgfqpoint{3.696000in}{3.696000in}}%
\pgfusepath{clip}%
\pgfsetrectcap%
\pgfsetroundjoin%
\pgfsetlinewidth{1.505625pt}%
\definecolor{currentstroke}{rgb}{1.000000,0.000000,0.000000}%
\pgfsetstrokecolor{currentstroke}%
\pgfsetdash{}{0pt}%
\pgfpathmoveto{\pgfqpoint{1.438737in}{1.839065in}}%
\pgfpathlineto{\pgfqpoint{1.390999in}{1.867768in}}%
\pgfusepath{stroke}%
\end{pgfscope}%
\begin{pgfscope}%
\pgfpathrectangle{\pgfqpoint{0.100000in}{0.212622in}}{\pgfqpoint{3.696000in}{3.696000in}}%
\pgfusepath{clip}%
\pgfsetrectcap%
\pgfsetroundjoin%
\pgfsetlinewidth{1.505625pt}%
\definecolor{currentstroke}{rgb}{1.000000,0.000000,0.000000}%
\pgfsetstrokecolor{currentstroke}%
\pgfsetdash{}{0pt}%
\pgfpathmoveto{\pgfqpoint{1.445456in}{1.844151in}}%
\pgfpathlineto{\pgfqpoint{1.390999in}{1.867768in}}%
\pgfusepath{stroke}%
\end{pgfscope}%
\begin{pgfscope}%
\pgfpathrectangle{\pgfqpoint{0.100000in}{0.212622in}}{\pgfqpoint{3.696000in}{3.696000in}}%
\pgfusepath{clip}%
\pgfsetrectcap%
\pgfsetroundjoin%
\pgfsetlinewidth{1.505625pt}%
\definecolor{currentstroke}{rgb}{1.000000,0.000000,0.000000}%
\pgfsetstrokecolor{currentstroke}%
\pgfsetdash{}{0pt}%
\pgfpathmoveto{\pgfqpoint{1.451956in}{1.846552in}}%
\pgfpathlineto{\pgfqpoint{1.399891in}{1.875288in}}%
\pgfusepath{stroke}%
\end{pgfscope}%
\begin{pgfscope}%
\pgfpathrectangle{\pgfqpoint{0.100000in}{0.212622in}}{\pgfqpoint{3.696000in}{3.696000in}}%
\pgfusepath{clip}%
\pgfsetrectcap%
\pgfsetroundjoin%
\pgfsetlinewidth{1.505625pt}%
\definecolor{currentstroke}{rgb}{1.000000,0.000000,0.000000}%
\pgfsetstrokecolor{currentstroke}%
\pgfsetdash{}{0pt}%
\pgfpathmoveto{\pgfqpoint{1.458261in}{1.830322in}}%
\pgfpathlineto{\pgfqpoint{1.408772in}{1.882799in}}%
\pgfusepath{stroke}%
\end{pgfscope}%
\begin{pgfscope}%
\pgfpathrectangle{\pgfqpoint{0.100000in}{0.212622in}}{\pgfqpoint{3.696000in}{3.696000in}}%
\pgfusepath{clip}%
\pgfsetrectcap%
\pgfsetroundjoin%
\pgfsetlinewidth{1.505625pt}%
\definecolor{currentstroke}{rgb}{1.000000,0.000000,0.000000}%
\pgfsetstrokecolor{currentstroke}%
\pgfsetdash{}{0pt}%
\pgfpathmoveto{\pgfqpoint{1.462021in}{1.833897in}}%
\pgfpathlineto{\pgfqpoint{1.408772in}{1.882799in}}%
\pgfusepath{stroke}%
\end{pgfscope}%
\begin{pgfscope}%
\pgfpathrectangle{\pgfqpoint{0.100000in}{0.212622in}}{\pgfqpoint{3.696000in}{3.696000in}}%
\pgfusepath{clip}%
\pgfsetrectcap%
\pgfsetroundjoin%
\pgfsetlinewidth{1.505625pt}%
\definecolor{currentstroke}{rgb}{1.000000,0.000000,0.000000}%
\pgfsetstrokecolor{currentstroke}%
\pgfsetdash{}{0pt}%
\pgfpathmoveto{\pgfqpoint{1.466766in}{1.834360in}}%
\pgfpathlineto{\pgfqpoint{1.417641in}{1.890300in}}%
\pgfusepath{stroke}%
\end{pgfscope}%
\begin{pgfscope}%
\pgfpathrectangle{\pgfqpoint{0.100000in}{0.212622in}}{\pgfqpoint{3.696000in}{3.696000in}}%
\pgfusepath{clip}%
\pgfsetrectcap%
\pgfsetroundjoin%
\pgfsetlinewidth{1.505625pt}%
\definecolor{currentstroke}{rgb}{1.000000,0.000000,0.000000}%
\pgfsetstrokecolor{currentstroke}%
\pgfsetdash{}{0pt}%
\pgfpathmoveto{\pgfqpoint{1.473660in}{1.829150in}}%
\pgfpathlineto{\pgfqpoint{1.417641in}{1.890300in}}%
\pgfusepath{stroke}%
\end{pgfscope}%
\begin{pgfscope}%
\pgfpathrectangle{\pgfqpoint{0.100000in}{0.212622in}}{\pgfqpoint{3.696000in}{3.696000in}}%
\pgfusepath{clip}%
\pgfsetrectcap%
\pgfsetroundjoin%
\pgfsetlinewidth{1.505625pt}%
\definecolor{currentstroke}{rgb}{1.000000,0.000000,0.000000}%
\pgfsetstrokecolor{currentstroke}%
\pgfsetdash{}{0pt}%
\pgfpathmoveto{\pgfqpoint{1.477105in}{1.833723in}}%
\pgfpathlineto{\pgfqpoint{1.426500in}{1.897791in}}%
\pgfusepath{stroke}%
\end{pgfscope}%
\begin{pgfscope}%
\pgfpathrectangle{\pgfqpoint{0.100000in}{0.212622in}}{\pgfqpoint{3.696000in}{3.696000in}}%
\pgfusepath{clip}%
\pgfsetrectcap%
\pgfsetroundjoin%
\pgfsetlinewidth{1.505625pt}%
\definecolor{currentstroke}{rgb}{1.000000,0.000000,0.000000}%
\pgfsetstrokecolor{currentstroke}%
\pgfsetdash{}{0pt}%
\pgfpathmoveto{\pgfqpoint{1.480823in}{1.832368in}}%
\pgfpathlineto{\pgfqpoint{1.426500in}{1.897791in}}%
\pgfusepath{stroke}%
\end{pgfscope}%
\begin{pgfscope}%
\pgfpathrectangle{\pgfqpoint{0.100000in}{0.212622in}}{\pgfqpoint{3.696000in}{3.696000in}}%
\pgfusepath{clip}%
\pgfsetrectcap%
\pgfsetroundjoin%
\pgfsetlinewidth{1.505625pt}%
\definecolor{currentstroke}{rgb}{1.000000,0.000000,0.000000}%
\pgfsetstrokecolor{currentstroke}%
\pgfsetdash{}{0pt}%
\pgfpathmoveto{\pgfqpoint{1.485682in}{1.837390in}}%
\pgfpathlineto{\pgfqpoint{1.435346in}{1.905273in}}%
\pgfusepath{stroke}%
\end{pgfscope}%
\begin{pgfscope}%
\pgfpathrectangle{\pgfqpoint{0.100000in}{0.212622in}}{\pgfqpoint{3.696000in}{3.696000in}}%
\pgfusepath{clip}%
\pgfsetrectcap%
\pgfsetroundjoin%
\pgfsetlinewidth{1.505625pt}%
\definecolor{currentstroke}{rgb}{1.000000,0.000000,0.000000}%
\pgfsetstrokecolor{currentstroke}%
\pgfsetdash{}{0pt}%
\pgfpathmoveto{\pgfqpoint{1.488511in}{1.838615in}}%
\pgfpathlineto{\pgfqpoint{1.435346in}{1.905273in}}%
\pgfusepath{stroke}%
\end{pgfscope}%
\begin{pgfscope}%
\pgfpathrectangle{\pgfqpoint{0.100000in}{0.212622in}}{\pgfqpoint{3.696000in}{3.696000in}}%
\pgfusepath{clip}%
\pgfsetrectcap%
\pgfsetroundjoin%
\pgfsetlinewidth{1.505625pt}%
\definecolor{currentstroke}{rgb}{1.000000,0.000000,0.000000}%
\pgfsetstrokecolor{currentstroke}%
\pgfsetdash{}{0pt}%
\pgfpathmoveto{\pgfqpoint{1.489927in}{1.838852in}}%
\pgfpathlineto{\pgfqpoint{1.435346in}{1.905273in}}%
\pgfusepath{stroke}%
\end{pgfscope}%
\begin{pgfscope}%
\pgfpathrectangle{\pgfqpoint{0.100000in}{0.212622in}}{\pgfqpoint{3.696000in}{3.696000in}}%
\pgfusepath{clip}%
\pgfsetrectcap%
\pgfsetroundjoin%
\pgfsetlinewidth{1.505625pt}%
\definecolor{currentstroke}{rgb}{1.000000,0.000000,0.000000}%
\pgfsetstrokecolor{currentstroke}%
\pgfsetdash{}{0pt}%
\pgfpathmoveto{\pgfqpoint{1.492317in}{1.840384in}}%
\pgfpathlineto{\pgfqpoint{1.435346in}{1.905273in}}%
\pgfusepath{stroke}%
\end{pgfscope}%
\begin{pgfscope}%
\pgfpathrectangle{\pgfqpoint{0.100000in}{0.212622in}}{\pgfqpoint{3.696000in}{3.696000in}}%
\pgfusepath{clip}%
\pgfsetrectcap%
\pgfsetroundjoin%
\pgfsetlinewidth{1.505625pt}%
\definecolor{currentstroke}{rgb}{1.000000,0.000000,0.000000}%
\pgfsetstrokecolor{currentstroke}%
\pgfsetdash{}{0pt}%
\pgfpathmoveto{\pgfqpoint{1.495338in}{1.841644in}}%
\pgfpathlineto{\pgfqpoint{1.444182in}{1.912746in}}%
\pgfusepath{stroke}%
\end{pgfscope}%
\begin{pgfscope}%
\pgfpathrectangle{\pgfqpoint{0.100000in}{0.212622in}}{\pgfqpoint{3.696000in}{3.696000in}}%
\pgfusepath{clip}%
\pgfsetrectcap%
\pgfsetroundjoin%
\pgfsetlinewidth{1.505625pt}%
\definecolor{currentstroke}{rgb}{1.000000,0.000000,0.000000}%
\pgfsetstrokecolor{currentstroke}%
\pgfsetdash{}{0pt}%
\pgfpathmoveto{\pgfqpoint{1.499083in}{1.843925in}}%
\pgfpathlineto{\pgfqpoint{1.444182in}{1.912746in}}%
\pgfusepath{stroke}%
\end{pgfscope}%
\begin{pgfscope}%
\pgfpathrectangle{\pgfqpoint{0.100000in}{0.212622in}}{\pgfqpoint{3.696000in}{3.696000in}}%
\pgfusepath{clip}%
\pgfsetrectcap%
\pgfsetroundjoin%
\pgfsetlinewidth{1.505625pt}%
\definecolor{currentstroke}{rgb}{1.000000,0.000000,0.000000}%
\pgfsetstrokecolor{currentstroke}%
\pgfsetdash{}{0pt}%
\pgfpathmoveto{\pgfqpoint{1.504335in}{1.846956in}}%
\pgfpathlineto{\pgfqpoint{1.444182in}{1.912746in}}%
\pgfusepath{stroke}%
\end{pgfscope}%
\begin{pgfscope}%
\pgfpathrectangle{\pgfqpoint{0.100000in}{0.212622in}}{\pgfqpoint{3.696000in}{3.696000in}}%
\pgfusepath{clip}%
\pgfsetrectcap%
\pgfsetroundjoin%
\pgfsetlinewidth{1.505625pt}%
\definecolor{currentstroke}{rgb}{1.000000,0.000000,0.000000}%
\pgfsetstrokecolor{currentstroke}%
\pgfsetdash{}{0pt}%
\pgfpathmoveto{\pgfqpoint{1.509741in}{1.847221in}}%
\pgfpathlineto{\pgfqpoint{1.453007in}{1.920209in}}%
\pgfusepath{stroke}%
\end{pgfscope}%
\begin{pgfscope}%
\pgfpathrectangle{\pgfqpoint{0.100000in}{0.212622in}}{\pgfqpoint{3.696000in}{3.696000in}}%
\pgfusepath{clip}%
\pgfsetrectcap%
\pgfsetroundjoin%
\pgfsetlinewidth{1.505625pt}%
\definecolor{currentstroke}{rgb}{1.000000,0.000000,0.000000}%
\pgfsetstrokecolor{currentstroke}%
\pgfsetdash{}{0pt}%
\pgfpathmoveto{\pgfqpoint{1.513151in}{1.848355in}}%
\pgfpathlineto{\pgfqpoint{1.453007in}{1.920209in}}%
\pgfusepath{stroke}%
\end{pgfscope}%
\begin{pgfscope}%
\pgfpathrectangle{\pgfqpoint{0.100000in}{0.212622in}}{\pgfqpoint{3.696000in}{3.696000in}}%
\pgfusepath{clip}%
\pgfsetrectcap%
\pgfsetroundjoin%
\pgfsetlinewidth{1.505625pt}%
\definecolor{currentstroke}{rgb}{1.000000,0.000000,0.000000}%
\pgfsetstrokecolor{currentstroke}%
\pgfsetdash{}{0pt}%
\pgfpathmoveto{\pgfqpoint{1.515105in}{1.850115in}}%
\pgfpathlineto{\pgfqpoint{1.453007in}{1.920209in}}%
\pgfusepath{stroke}%
\end{pgfscope}%
\begin{pgfscope}%
\pgfpathrectangle{\pgfqpoint{0.100000in}{0.212622in}}{\pgfqpoint{3.696000in}{3.696000in}}%
\pgfusepath{clip}%
\pgfsetrectcap%
\pgfsetroundjoin%
\pgfsetlinewidth{1.505625pt}%
\definecolor{currentstroke}{rgb}{1.000000,0.000000,0.000000}%
\pgfsetstrokecolor{currentstroke}%
\pgfsetdash{}{0pt}%
\pgfpathmoveto{\pgfqpoint{1.518726in}{1.850602in}}%
\pgfpathlineto{\pgfqpoint{1.461820in}{1.927662in}}%
\pgfusepath{stroke}%
\end{pgfscope}%
\begin{pgfscope}%
\pgfpathrectangle{\pgfqpoint{0.100000in}{0.212622in}}{\pgfqpoint{3.696000in}{3.696000in}}%
\pgfusepath{clip}%
\pgfsetrectcap%
\pgfsetroundjoin%
\pgfsetlinewidth{1.505625pt}%
\definecolor{currentstroke}{rgb}{1.000000,0.000000,0.000000}%
\pgfsetstrokecolor{currentstroke}%
\pgfsetdash{}{0pt}%
\pgfpathmoveto{\pgfqpoint{1.521007in}{1.853247in}}%
\pgfpathlineto{\pgfqpoint{1.461820in}{1.927662in}}%
\pgfusepath{stroke}%
\end{pgfscope}%
\begin{pgfscope}%
\pgfpathrectangle{\pgfqpoint{0.100000in}{0.212622in}}{\pgfqpoint{3.696000in}{3.696000in}}%
\pgfusepath{clip}%
\pgfsetrectcap%
\pgfsetroundjoin%
\pgfsetlinewidth{1.505625pt}%
\definecolor{currentstroke}{rgb}{1.000000,0.000000,0.000000}%
\pgfsetstrokecolor{currentstroke}%
\pgfsetdash{}{0pt}%
\pgfpathmoveto{\pgfqpoint{1.522436in}{1.853794in}}%
\pgfpathlineto{\pgfqpoint{1.461820in}{1.927662in}}%
\pgfusepath{stroke}%
\end{pgfscope}%
\begin{pgfscope}%
\pgfpathrectangle{\pgfqpoint{0.100000in}{0.212622in}}{\pgfqpoint{3.696000in}{3.696000in}}%
\pgfusepath{clip}%
\pgfsetrectcap%
\pgfsetroundjoin%
\pgfsetlinewidth{1.505625pt}%
\definecolor{currentstroke}{rgb}{1.000000,0.000000,0.000000}%
\pgfsetstrokecolor{currentstroke}%
\pgfsetdash{}{0pt}%
\pgfpathmoveto{\pgfqpoint{1.525417in}{1.854013in}}%
\pgfpathlineto{\pgfqpoint{1.461820in}{1.927662in}}%
\pgfusepath{stroke}%
\end{pgfscope}%
\begin{pgfscope}%
\pgfpathrectangle{\pgfqpoint{0.100000in}{0.212622in}}{\pgfqpoint{3.696000in}{3.696000in}}%
\pgfusepath{clip}%
\pgfsetrectcap%
\pgfsetroundjoin%
\pgfsetlinewidth{1.505625pt}%
\definecolor{currentstroke}{rgb}{1.000000,0.000000,0.000000}%
\pgfsetstrokecolor{currentstroke}%
\pgfsetdash{}{0pt}%
\pgfpathmoveto{\pgfqpoint{1.527285in}{1.854627in}}%
\pgfpathlineto{\pgfqpoint{1.470622in}{1.935106in}}%
\pgfusepath{stroke}%
\end{pgfscope}%
\begin{pgfscope}%
\pgfpathrectangle{\pgfqpoint{0.100000in}{0.212622in}}{\pgfqpoint{3.696000in}{3.696000in}}%
\pgfusepath{clip}%
\pgfsetrectcap%
\pgfsetroundjoin%
\pgfsetlinewidth{1.505625pt}%
\definecolor{currentstroke}{rgb}{1.000000,0.000000,0.000000}%
\pgfsetstrokecolor{currentstroke}%
\pgfsetdash{}{0pt}%
\pgfpathmoveto{\pgfqpoint{1.528426in}{1.855346in}}%
\pgfpathlineto{\pgfqpoint{1.470622in}{1.935106in}}%
\pgfusepath{stroke}%
\end{pgfscope}%
\begin{pgfscope}%
\pgfpathrectangle{\pgfqpoint{0.100000in}{0.212622in}}{\pgfqpoint{3.696000in}{3.696000in}}%
\pgfusepath{clip}%
\pgfsetrectcap%
\pgfsetroundjoin%
\pgfsetlinewidth{1.505625pt}%
\definecolor{currentstroke}{rgb}{1.000000,0.000000,0.000000}%
\pgfsetstrokecolor{currentstroke}%
\pgfsetdash{}{0pt}%
\pgfpathmoveto{\pgfqpoint{1.530796in}{1.855666in}}%
\pgfpathlineto{\pgfqpoint{1.470622in}{1.935106in}}%
\pgfusepath{stroke}%
\end{pgfscope}%
\begin{pgfscope}%
\pgfpathrectangle{\pgfqpoint{0.100000in}{0.212622in}}{\pgfqpoint{3.696000in}{3.696000in}}%
\pgfusepath{clip}%
\pgfsetrectcap%
\pgfsetroundjoin%
\pgfsetlinewidth{1.505625pt}%
\definecolor{currentstroke}{rgb}{1.000000,0.000000,0.000000}%
\pgfsetstrokecolor{currentstroke}%
\pgfsetdash{}{0pt}%
\pgfpathmoveto{\pgfqpoint{1.532297in}{1.855941in}}%
\pgfpathlineto{\pgfqpoint{1.470622in}{1.935106in}}%
\pgfusepath{stroke}%
\end{pgfscope}%
\begin{pgfscope}%
\pgfpathrectangle{\pgfqpoint{0.100000in}{0.212622in}}{\pgfqpoint{3.696000in}{3.696000in}}%
\pgfusepath{clip}%
\pgfsetrectcap%
\pgfsetroundjoin%
\pgfsetlinewidth{1.505625pt}%
\definecolor{currentstroke}{rgb}{1.000000,0.000000,0.000000}%
\pgfsetstrokecolor{currentstroke}%
\pgfsetdash{}{0pt}%
\pgfpathmoveto{\pgfqpoint{1.534550in}{1.857256in}}%
\pgfpathlineto{\pgfqpoint{1.470622in}{1.935106in}}%
\pgfusepath{stroke}%
\end{pgfscope}%
\begin{pgfscope}%
\pgfpathrectangle{\pgfqpoint{0.100000in}{0.212622in}}{\pgfqpoint{3.696000in}{3.696000in}}%
\pgfusepath{clip}%
\pgfsetrectcap%
\pgfsetroundjoin%
\pgfsetlinewidth{1.505625pt}%
\definecolor{currentstroke}{rgb}{1.000000,0.000000,0.000000}%
\pgfsetstrokecolor{currentstroke}%
\pgfsetdash{}{0pt}%
\pgfpathmoveto{\pgfqpoint{1.538025in}{1.859071in}}%
\pgfpathlineto{\pgfqpoint{1.479413in}{1.942541in}}%
\pgfusepath{stroke}%
\end{pgfscope}%
\begin{pgfscope}%
\pgfpathrectangle{\pgfqpoint{0.100000in}{0.212622in}}{\pgfqpoint{3.696000in}{3.696000in}}%
\pgfusepath{clip}%
\pgfsetrectcap%
\pgfsetroundjoin%
\pgfsetlinewidth{1.505625pt}%
\definecolor{currentstroke}{rgb}{1.000000,0.000000,0.000000}%
\pgfsetstrokecolor{currentstroke}%
\pgfsetdash{}{0pt}%
\pgfpathmoveto{\pgfqpoint{1.540130in}{1.859329in}}%
\pgfpathlineto{\pgfqpoint{1.479413in}{1.942541in}}%
\pgfusepath{stroke}%
\end{pgfscope}%
\begin{pgfscope}%
\pgfpathrectangle{\pgfqpoint{0.100000in}{0.212622in}}{\pgfqpoint{3.696000in}{3.696000in}}%
\pgfusepath{clip}%
\pgfsetrectcap%
\pgfsetroundjoin%
\pgfsetlinewidth{1.505625pt}%
\definecolor{currentstroke}{rgb}{1.000000,0.000000,0.000000}%
\pgfsetstrokecolor{currentstroke}%
\pgfsetdash{}{0pt}%
\pgfpathmoveto{\pgfqpoint{1.543274in}{1.860713in}}%
\pgfpathlineto{\pgfqpoint{1.479413in}{1.942541in}}%
\pgfusepath{stroke}%
\end{pgfscope}%
\begin{pgfscope}%
\pgfpathrectangle{\pgfqpoint{0.100000in}{0.212622in}}{\pgfqpoint{3.696000in}{3.696000in}}%
\pgfusepath{clip}%
\pgfsetrectcap%
\pgfsetroundjoin%
\pgfsetlinewidth{1.505625pt}%
\definecolor{currentstroke}{rgb}{1.000000,0.000000,0.000000}%
\pgfsetstrokecolor{currentstroke}%
\pgfsetdash{}{0pt}%
\pgfpathmoveto{\pgfqpoint{1.547186in}{1.861826in}}%
\pgfpathlineto{\pgfqpoint{1.488192in}{1.949966in}}%
\pgfusepath{stroke}%
\end{pgfscope}%
\begin{pgfscope}%
\pgfpathrectangle{\pgfqpoint{0.100000in}{0.212622in}}{\pgfqpoint{3.696000in}{3.696000in}}%
\pgfusepath{clip}%
\pgfsetrectcap%
\pgfsetroundjoin%
\pgfsetlinewidth{1.505625pt}%
\definecolor{currentstroke}{rgb}{1.000000,0.000000,0.000000}%
\pgfsetstrokecolor{currentstroke}%
\pgfsetdash{}{0pt}%
\pgfpathmoveto{\pgfqpoint{1.549667in}{1.862401in}}%
\pgfpathlineto{\pgfqpoint{1.488192in}{1.949966in}}%
\pgfusepath{stroke}%
\end{pgfscope}%
\begin{pgfscope}%
\pgfpathrectangle{\pgfqpoint{0.100000in}{0.212622in}}{\pgfqpoint{3.696000in}{3.696000in}}%
\pgfusepath{clip}%
\pgfsetrectcap%
\pgfsetroundjoin%
\pgfsetlinewidth{1.505625pt}%
\definecolor{currentstroke}{rgb}{1.000000,0.000000,0.000000}%
\pgfsetstrokecolor{currentstroke}%
\pgfsetdash{}{0pt}%
\pgfpathmoveto{\pgfqpoint{1.552667in}{1.864129in}}%
\pgfpathlineto{\pgfqpoint{1.488192in}{1.949966in}}%
\pgfusepath{stroke}%
\end{pgfscope}%
\begin{pgfscope}%
\pgfpathrectangle{\pgfqpoint{0.100000in}{0.212622in}}{\pgfqpoint{3.696000in}{3.696000in}}%
\pgfusepath{clip}%
\pgfsetrectcap%
\pgfsetroundjoin%
\pgfsetlinewidth{1.505625pt}%
\definecolor{currentstroke}{rgb}{1.000000,0.000000,0.000000}%
\pgfsetstrokecolor{currentstroke}%
\pgfsetdash{}{0pt}%
\pgfpathmoveto{\pgfqpoint{1.557232in}{1.864332in}}%
\pgfpathlineto{\pgfqpoint{1.496961in}{1.957382in}}%
\pgfusepath{stroke}%
\end{pgfscope}%
\begin{pgfscope}%
\pgfpathrectangle{\pgfqpoint{0.100000in}{0.212622in}}{\pgfqpoint{3.696000in}{3.696000in}}%
\pgfusepath{clip}%
\pgfsetrectcap%
\pgfsetroundjoin%
\pgfsetlinewidth{1.505625pt}%
\definecolor{currentstroke}{rgb}{1.000000,0.000000,0.000000}%
\pgfsetstrokecolor{currentstroke}%
\pgfsetdash{}{0pt}%
\pgfpathmoveto{\pgfqpoint{1.560704in}{1.868060in}}%
\pgfpathlineto{\pgfqpoint{1.496961in}{1.957382in}}%
\pgfusepath{stroke}%
\end{pgfscope}%
\begin{pgfscope}%
\pgfpathrectangle{\pgfqpoint{0.100000in}{0.212622in}}{\pgfqpoint{3.696000in}{3.696000in}}%
\pgfusepath{clip}%
\pgfsetrectcap%
\pgfsetroundjoin%
\pgfsetlinewidth{1.505625pt}%
\definecolor{currentstroke}{rgb}{1.000000,0.000000,0.000000}%
\pgfsetstrokecolor{currentstroke}%
\pgfsetdash{}{0pt}%
\pgfpathmoveto{\pgfqpoint{1.567635in}{1.872079in}}%
\pgfpathlineto{\pgfqpoint{1.496961in}{1.957382in}}%
\pgfusepath{stroke}%
\end{pgfscope}%
\begin{pgfscope}%
\pgfpathrectangle{\pgfqpoint{0.100000in}{0.212622in}}{\pgfqpoint{3.696000in}{3.696000in}}%
\pgfusepath{clip}%
\pgfsetrectcap%
\pgfsetroundjoin%
\pgfsetlinewidth{1.505625pt}%
\definecolor{currentstroke}{rgb}{1.000000,0.000000,0.000000}%
\pgfsetstrokecolor{currentstroke}%
\pgfsetdash{}{0pt}%
\pgfpathmoveto{\pgfqpoint{1.575400in}{1.870353in}}%
\pgfpathlineto{\pgfqpoint{1.505718in}{1.964788in}}%
\pgfusepath{stroke}%
\end{pgfscope}%
\begin{pgfscope}%
\pgfpathrectangle{\pgfqpoint{0.100000in}{0.212622in}}{\pgfqpoint{3.696000in}{3.696000in}}%
\pgfusepath{clip}%
\pgfsetrectcap%
\pgfsetroundjoin%
\pgfsetlinewidth{1.505625pt}%
\definecolor{currentstroke}{rgb}{1.000000,0.000000,0.000000}%
\pgfsetstrokecolor{currentstroke}%
\pgfsetdash{}{0pt}%
\pgfpathmoveto{\pgfqpoint{1.580538in}{1.875856in}}%
\pgfpathlineto{\pgfqpoint{1.514465in}{1.972185in}}%
\pgfusepath{stroke}%
\end{pgfscope}%
\begin{pgfscope}%
\pgfpathrectangle{\pgfqpoint{0.100000in}{0.212622in}}{\pgfqpoint{3.696000in}{3.696000in}}%
\pgfusepath{clip}%
\pgfsetrectcap%
\pgfsetroundjoin%
\pgfsetlinewidth{1.505625pt}%
\definecolor{currentstroke}{rgb}{1.000000,0.000000,0.000000}%
\pgfsetstrokecolor{currentstroke}%
\pgfsetdash{}{0pt}%
\pgfpathmoveto{\pgfqpoint{1.589172in}{1.879860in}}%
\pgfpathlineto{\pgfqpoint{1.514465in}{1.972185in}}%
\pgfusepath{stroke}%
\end{pgfscope}%
\begin{pgfscope}%
\pgfpathrectangle{\pgfqpoint{0.100000in}{0.212622in}}{\pgfqpoint{3.696000in}{3.696000in}}%
\pgfusepath{clip}%
\pgfsetrectcap%
\pgfsetroundjoin%
\pgfsetlinewidth{1.505625pt}%
\definecolor{currentstroke}{rgb}{1.000000,0.000000,0.000000}%
\pgfsetstrokecolor{currentstroke}%
\pgfsetdash{}{0pt}%
\pgfpathmoveto{\pgfqpoint{1.598023in}{1.878779in}}%
\pgfpathlineto{\pgfqpoint{1.523200in}{1.979573in}}%
\pgfusepath{stroke}%
\end{pgfscope}%
\begin{pgfscope}%
\pgfpathrectangle{\pgfqpoint{0.100000in}{0.212622in}}{\pgfqpoint{3.696000in}{3.696000in}}%
\pgfusepath{clip}%
\pgfsetrectcap%
\pgfsetroundjoin%
\pgfsetlinewidth{1.505625pt}%
\definecolor{currentstroke}{rgb}{1.000000,0.000000,0.000000}%
\pgfsetstrokecolor{currentstroke}%
\pgfsetdash{}{0pt}%
\pgfpathmoveto{\pgfqpoint{1.604664in}{1.882889in}}%
\pgfpathlineto{\pgfqpoint{1.531924in}{1.986951in}}%
\pgfusepath{stroke}%
\end{pgfscope}%
\begin{pgfscope}%
\pgfpathrectangle{\pgfqpoint{0.100000in}{0.212622in}}{\pgfqpoint{3.696000in}{3.696000in}}%
\pgfusepath{clip}%
\pgfsetrectcap%
\pgfsetroundjoin%
\pgfsetlinewidth{1.505625pt}%
\definecolor{currentstroke}{rgb}{1.000000,0.000000,0.000000}%
\pgfsetstrokecolor{currentstroke}%
\pgfsetdash{}{0pt}%
\pgfpathmoveto{\pgfqpoint{1.614589in}{1.889948in}}%
\pgfpathlineto{\pgfqpoint{1.540637in}{1.994320in}}%
\pgfusepath{stroke}%
\end{pgfscope}%
\begin{pgfscope}%
\pgfpathrectangle{\pgfqpoint{0.100000in}{0.212622in}}{\pgfqpoint{3.696000in}{3.696000in}}%
\pgfusepath{clip}%
\pgfsetrectcap%
\pgfsetroundjoin%
\pgfsetlinewidth{1.505625pt}%
\definecolor{currentstroke}{rgb}{1.000000,0.000000,0.000000}%
\pgfsetstrokecolor{currentstroke}%
\pgfsetdash{}{0pt}%
\pgfpathmoveto{\pgfqpoint{1.625033in}{1.888353in}}%
\pgfpathlineto{\pgfqpoint{1.549340in}{2.001679in}}%
\pgfusepath{stroke}%
\end{pgfscope}%
\begin{pgfscope}%
\pgfpathrectangle{\pgfqpoint{0.100000in}{0.212622in}}{\pgfqpoint{3.696000in}{3.696000in}}%
\pgfusepath{clip}%
\pgfsetrectcap%
\pgfsetroundjoin%
\pgfsetlinewidth{1.505625pt}%
\definecolor{currentstroke}{rgb}{1.000000,0.000000,0.000000}%
\pgfsetstrokecolor{currentstroke}%
\pgfsetdash{}{0pt}%
\pgfpathmoveto{\pgfqpoint{1.631726in}{1.892763in}}%
\pgfpathlineto{\pgfqpoint{1.558031in}{2.009029in}}%
\pgfusepath{stroke}%
\end{pgfscope}%
\begin{pgfscope}%
\pgfpathrectangle{\pgfqpoint{0.100000in}{0.212622in}}{\pgfqpoint{3.696000in}{3.696000in}}%
\pgfusepath{clip}%
\pgfsetrectcap%
\pgfsetroundjoin%
\pgfsetlinewidth{1.505625pt}%
\definecolor{currentstroke}{rgb}{1.000000,0.000000,0.000000}%
\pgfsetstrokecolor{currentstroke}%
\pgfsetdash{}{0pt}%
\pgfpathmoveto{\pgfqpoint{1.635309in}{1.896075in}}%
\pgfpathlineto{\pgfqpoint{1.558031in}{2.009029in}}%
\pgfusepath{stroke}%
\end{pgfscope}%
\begin{pgfscope}%
\pgfpathrectangle{\pgfqpoint{0.100000in}{0.212622in}}{\pgfqpoint{3.696000in}{3.696000in}}%
\pgfusepath{clip}%
\pgfsetrectcap%
\pgfsetroundjoin%
\pgfsetlinewidth{1.505625pt}%
\definecolor{currentstroke}{rgb}{1.000000,0.000000,0.000000}%
\pgfsetstrokecolor{currentstroke}%
\pgfsetdash{}{0pt}%
\pgfpathmoveto{\pgfqpoint{1.641416in}{1.899854in}}%
\pgfpathlineto{\pgfqpoint{1.566711in}{2.016370in}}%
\pgfusepath{stroke}%
\end{pgfscope}%
\begin{pgfscope}%
\pgfpathrectangle{\pgfqpoint{0.100000in}{0.212622in}}{\pgfqpoint{3.696000in}{3.696000in}}%
\pgfusepath{clip}%
\pgfsetrectcap%
\pgfsetroundjoin%
\pgfsetlinewidth{1.505625pt}%
\definecolor{currentstroke}{rgb}{1.000000,0.000000,0.000000}%
\pgfsetstrokecolor{currentstroke}%
\pgfsetdash{}{0pt}%
\pgfpathmoveto{\pgfqpoint{1.645501in}{1.900333in}}%
\pgfpathlineto{\pgfqpoint{1.566711in}{2.016370in}}%
\pgfusepath{stroke}%
\end{pgfscope}%
\begin{pgfscope}%
\pgfpathrectangle{\pgfqpoint{0.100000in}{0.212622in}}{\pgfqpoint{3.696000in}{3.696000in}}%
\pgfusepath{clip}%
\pgfsetrectcap%
\pgfsetroundjoin%
\pgfsetlinewidth{1.505625pt}%
\definecolor{currentstroke}{rgb}{1.000000,0.000000,0.000000}%
\pgfsetstrokecolor{currentstroke}%
\pgfsetdash{}{0pt}%
\pgfpathmoveto{\pgfqpoint{1.650340in}{1.904397in}}%
\pgfpathlineto{\pgfqpoint{1.566711in}{2.016370in}}%
\pgfusepath{stroke}%
\end{pgfscope}%
\begin{pgfscope}%
\pgfpathrectangle{\pgfqpoint{0.100000in}{0.212622in}}{\pgfqpoint{3.696000in}{3.696000in}}%
\pgfusepath{clip}%
\pgfsetrectcap%
\pgfsetroundjoin%
\pgfsetlinewidth{1.505625pt}%
\definecolor{currentstroke}{rgb}{1.000000,0.000000,0.000000}%
\pgfsetstrokecolor{currentstroke}%
\pgfsetdash{}{0pt}%
\pgfpathmoveto{\pgfqpoint{1.656350in}{1.905563in}}%
\pgfpathlineto{\pgfqpoint{1.575380in}{2.023702in}}%
\pgfusepath{stroke}%
\end{pgfscope}%
\begin{pgfscope}%
\pgfpathrectangle{\pgfqpoint{0.100000in}{0.212622in}}{\pgfqpoint{3.696000in}{3.696000in}}%
\pgfusepath{clip}%
\pgfsetrectcap%
\pgfsetroundjoin%
\pgfsetlinewidth{1.505625pt}%
\definecolor{currentstroke}{rgb}{1.000000,0.000000,0.000000}%
\pgfsetstrokecolor{currentstroke}%
\pgfsetdash{}{0pt}%
\pgfpathmoveto{\pgfqpoint{1.659197in}{1.905738in}}%
\pgfpathlineto{\pgfqpoint{1.575380in}{2.023702in}}%
\pgfusepath{stroke}%
\end{pgfscope}%
\begin{pgfscope}%
\pgfpathrectangle{\pgfqpoint{0.100000in}{0.212622in}}{\pgfqpoint{3.696000in}{3.696000in}}%
\pgfusepath{clip}%
\pgfsetrectcap%
\pgfsetroundjoin%
\pgfsetlinewidth{1.505625pt}%
\definecolor{currentstroke}{rgb}{1.000000,0.000000,0.000000}%
\pgfsetstrokecolor{currentstroke}%
\pgfsetdash{}{0pt}%
\pgfpathmoveto{\pgfqpoint{1.664162in}{1.908546in}}%
\pgfpathlineto{\pgfqpoint{1.584039in}{2.031025in}}%
\pgfusepath{stroke}%
\end{pgfscope}%
\begin{pgfscope}%
\pgfpathrectangle{\pgfqpoint{0.100000in}{0.212622in}}{\pgfqpoint{3.696000in}{3.696000in}}%
\pgfusepath{clip}%
\pgfsetrectcap%
\pgfsetroundjoin%
\pgfsetlinewidth{1.505625pt}%
\definecolor{currentstroke}{rgb}{1.000000,0.000000,0.000000}%
\pgfsetstrokecolor{currentstroke}%
\pgfsetdash{}{0pt}%
\pgfpathmoveto{\pgfqpoint{1.669587in}{1.912991in}}%
\pgfpathlineto{\pgfqpoint{1.584039in}{2.031025in}}%
\pgfusepath{stroke}%
\end{pgfscope}%
\begin{pgfscope}%
\pgfpathrectangle{\pgfqpoint{0.100000in}{0.212622in}}{\pgfqpoint{3.696000in}{3.696000in}}%
\pgfusepath{clip}%
\pgfsetrectcap%
\pgfsetroundjoin%
\pgfsetlinewidth{1.505625pt}%
\definecolor{currentstroke}{rgb}{1.000000,0.000000,0.000000}%
\pgfsetstrokecolor{currentstroke}%
\pgfsetdash{}{0pt}%
\pgfpathmoveto{\pgfqpoint{1.673323in}{1.915943in}}%
\pgfpathlineto{\pgfqpoint{1.592686in}{2.038338in}}%
\pgfusepath{stroke}%
\end{pgfscope}%
\begin{pgfscope}%
\pgfpathrectangle{\pgfqpoint{0.100000in}{0.212622in}}{\pgfqpoint{3.696000in}{3.696000in}}%
\pgfusepath{clip}%
\pgfsetrectcap%
\pgfsetroundjoin%
\pgfsetlinewidth{1.505625pt}%
\definecolor{currentstroke}{rgb}{1.000000,0.000000,0.000000}%
\pgfsetstrokecolor{currentstroke}%
\pgfsetdash{}{0pt}%
\pgfpathmoveto{\pgfqpoint{1.675596in}{1.917431in}}%
\pgfpathlineto{\pgfqpoint{1.592686in}{2.038338in}}%
\pgfusepath{stroke}%
\end{pgfscope}%
\begin{pgfscope}%
\pgfpathrectangle{\pgfqpoint{0.100000in}{0.212622in}}{\pgfqpoint{3.696000in}{3.696000in}}%
\pgfusepath{clip}%
\pgfsetrectcap%
\pgfsetroundjoin%
\pgfsetlinewidth{1.505625pt}%
\definecolor{currentstroke}{rgb}{1.000000,0.000000,0.000000}%
\pgfsetstrokecolor{currentstroke}%
\pgfsetdash{}{0pt}%
\pgfpathmoveto{\pgfqpoint{1.679244in}{1.918304in}}%
\pgfpathlineto{\pgfqpoint{1.592686in}{2.038338in}}%
\pgfusepath{stroke}%
\end{pgfscope}%
\begin{pgfscope}%
\pgfpathrectangle{\pgfqpoint{0.100000in}{0.212622in}}{\pgfqpoint{3.696000in}{3.696000in}}%
\pgfusepath{clip}%
\pgfsetrectcap%
\pgfsetroundjoin%
\pgfsetlinewidth{1.505625pt}%
\definecolor{currentstroke}{rgb}{1.000000,0.000000,0.000000}%
\pgfsetstrokecolor{currentstroke}%
\pgfsetdash{}{0pt}%
\pgfpathmoveto{\pgfqpoint{1.681783in}{1.920524in}}%
\pgfpathlineto{\pgfqpoint{1.601323in}{2.045642in}}%
\pgfusepath{stroke}%
\end{pgfscope}%
\begin{pgfscope}%
\pgfpathrectangle{\pgfqpoint{0.100000in}{0.212622in}}{\pgfqpoint{3.696000in}{3.696000in}}%
\pgfusepath{clip}%
\pgfsetrectcap%
\pgfsetroundjoin%
\pgfsetlinewidth{1.505625pt}%
\definecolor{currentstroke}{rgb}{1.000000,0.000000,0.000000}%
\pgfsetstrokecolor{currentstroke}%
\pgfsetdash{}{0pt}%
\pgfpathmoveto{\pgfqpoint{1.685058in}{1.923750in}}%
\pgfpathlineto{\pgfqpoint{1.601323in}{2.045642in}}%
\pgfusepath{stroke}%
\end{pgfscope}%
\begin{pgfscope}%
\pgfpathrectangle{\pgfqpoint{0.100000in}{0.212622in}}{\pgfqpoint{3.696000in}{3.696000in}}%
\pgfusepath{clip}%
\pgfsetrectcap%
\pgfsetroundjoin%
\pgfsetlinewidth{1.505625pt}%
\definecolor{currentstroke}{rgb}{1.000000,0.000000,0.000000}%
\pgfsetstrokecolor{currentstroke}%
\pgfsetdash{}{0pt}%
\pgfpathmoveto{\pgfqpoint{1.689212in}{1.924188in}}%
\pgfpathlineto{\pgfqpoint{1.601323in}{2.045642in}}%
\pgfusepath{stroke}%
\end{pgfscope}%
\begin{pgfscope}%
\pgfpathrectangle{\pgfqpoint{0.100000in}{0.212622in}}{\pgfqpoint{3.696000in}{3.696000in}}%
\pgfusepath{clip}%
\pgfsetrectcap%
\pgfsetroundjoin%
\pgfsetlinewidth{1.505625pt}%
\definecolor{currentstroke}{rgb}{1.000000,0.000000,0.000000}%
\pgfsetstrokecolor{currentstroke}%
\pgfsetdash{}{0pt}%
\pgfpathmoveto{\pgfqpoint{1.695785in}{1.927854in}}%
\pgfpathlineto{\pgfqpoint{1.609948in}{2.052937in}}%
\pgfusepath{stroke}%
\end{pgfscope}%
\begin{pgfscope}%
\pgfpathrectangle{\pgfqpoint{0.100000in}{0.212622in}}{\pgfqpoint{3.696000in}{3.696000in}}%
\pgfusepath{clip}%
\pgfsetrectcap%
\pgfsetroundjoin%
\pgfsetlinewidth{1.505625pt}%
\definecolor{currentstroke}{rgb}{1.000000,0.000000,0.000000}%
\pgfsetstrokecolor{currentstroke}%
\pgfsetdash{}{0pt}%
\pgfpathmoveto{\pgfqpoint{1.702842in}{1.934417in}}%
\pgfpathlineto{\pgfqpoint{1.618563in}{2.060223in}}%
\pgfusepath{stroke}%
\end{pgfscope}%
\begin{pgfscope}%
\pgfpathrectangle{\pgfqpoint{0.100000in}{0.212622in}}{\pgfqpoint{3.696000in}{3.696000in}}%
\pgfusepath{clip}%
\pgfsetrectcap%
\pgfsetroundjoin%
\pgfsetlinewidth{1.505625pt}%
\definecolor{currentstroke}{rgb}{1.000000,0.000000,0.000000}%
\pgfsetstrokecolor{currentstroke}%
\pgfsetdash{}{0pt}%
\pgfpathmoveto{\pgfqpoint{1.708694in}{1.932382in}}%
\pgfpathlineto{\pgfqpoint{1.618563in}{2.060223in}}%
\pgfusepath{stroke}%
\end{pgfscope}%
\begin{pgfscope}%
\pgfpathrectangle{\pgfqpoint{0.100000in}{0.212622in}}{\pgfqpoint{3.696000in}{3.696000in}}%
\pgfusepath{clip}%
\pgfsetrectcap%
\pgfsetroundjoin%
\pgfsetlinewidth{1.505625pt}%
\definecolor{currentstroke}{rgb}{1.000000,0.000000,0.000000}%
\pgfsetstrokecolor{currentstroke}%
\pgfsetdash{}{0pt}%
\pgfpathmoveto{\pgfqpoint{1.717553in}{1.938420in}}%
\pgfpathlineto{\pgfqpoint{1.627167in}{2.067499in}}%
\pgfusepath{stroke}%
\end{pgfscope}%
\begin{pgfscope}%
\pgfpathrectangle{\pgfqpoint{0.100000in}{0.212622in}}{\pgfqpoint{3.696000in}{3.696000in}}%
\pgfusepath{clip}%
\pgfsetrectcap%
\pgfsetroundjoin%
\pgfsetlinewidth{1.505625pt}%
\definecolor{currentstroke}{rgb}{1.000000,0.000000,0.000000}%
\pgfsetstrokecolor{currentstroke}%
\pgfsetdash{}{0pt}%
\pgfpathmoveto{\pgfqpoint{1.725939in}{1.941772in}}%
\pgfpathlineto{\pgfqpoint{1.635761in}{2.074767in}}%
\pgfusepath{stroke}%
\end{pgfscope}%
\begin{pgfscope}%
\pgfpathrectangle{\pgfqpoint{0.100000in}{0.212622in}}{\pgfqpoint{3.696000in}{3.696000in}}%
\pgfusepath{clip}%
\pgfsetrectcap%
\pgfsetroundjoin%
\pgfsetlinewidth{1.505625pt}%
\definecolor{currentstroke}{rgb}{1.000000,0.000000,0.000000}%
\pgfsetstrokecolor{currentstroke}%
\pgfsetdash{}{0pt}%
\pgfpathmoveto{\pgfqpoint{1.730406in}{1.944079in}}%
\pgfpathlineto{\pgfqpoint{1.644343in}{2.082025in}}%
\pgfusepath{stroke}%
\end{pgfscope}%
\begin{pgfscope}%
\pgfpathrectangle{\pgfqpoint{0.100000in}{0.212622in}}{\pgfqpoint{3.696000in}{3.696000in}}%
\pgfusepath{clip}%
\pgfsetrectcap%
\pgfsetroundjoin%
\pgfsetlinewidth{1.505625pt}%
\definecolor{currentstroke}{rgb}{1.000000,0.000000,0.000000}%
\pgfsetstrokecolor{currentstroke}%
\pgfsetdash{}{0pt}%
\pgfpathmoveto{\pgfqpoint{1.736954in}{1.945815in}}%
\pgfpathlineto{\pgfqpoint{1.644343in}{2.082025in}}%
\pgfusepath{stroke}%
\end{pgfscope}%
\begin{pgfscope}%
\pgfpathrectangle{\pgfqpoint{0.100000in}{0.212622in}}{\pgfqpoint{3.696000in}{3.696000in}}%
\pgfusepath{clip}%
\pgfsetrectcap%
\pgfsetroundjoin%
\pgfsetlinewidth{1.505625pt}%
\definecolor{currentstroke}{rgb}{1.000000,0.000000,0.000000}%
\pgfsetstrokecolor{currentstroke}%
\pgfsetdash{}{0pt}%
\pgfpathmoveto{\pgfqpoint{1.744053in}{1.948926in}}%
\pgfpathlineto{\pgfqpoint{1.652915in}{2.089274in}}%
\pgfusepath{stroke}%
\end{pgfscope}%
\begin{pgfscope}%
\pgfpathrectangle{\pgfqpoint{0.100000in}{0.212622in}}{\pgfqpoint{3.696000in}{3.696000in}}%
\pgfusepath{clip}%
\pgfsetrectcap%
\pgfsetroundjoin%
\pgfsetlinewidth{1.505625pt}%
\definecolor{currentstroke}{rgb}{1.000000,0.000000,0.000000}%
\pgfsetstrokecolor{currentstroke}%
\pgfsetdash{}{0pt}%
\pgfpathmoveto{\pgfqpoint{1.748243in}{1.950782in}}%
\pgfpathlineto{\pgfqpoint{1.652915in}{2.089274in}}%
\pgfusepath{stroke}%
\end{pgfscope}%
\begin{pgfscope}%
\pgfpathrectangle{\pgfqpoint{0.100000in}{0.212622in}}{\pgfqpoint{3.696000in}{3.696000in}}%
\pgfusepath{clip}%
\pgfsetrectcap%
\pgfsetroundjoin%
\pgfsetlinewidth{1.505625pt}%
\definecolor{currentstroke}{rgb}{1.000000,0.000000,0.000000}%
\pgfsetstrokecolor{currentstroke}%
\pgfsetdash{}{0pt}%
\pgfpathmoveto{\pgfqpoint{1.753328in}{1.953613in}}%
\pgfpathlineto{\pgfqpoint{1.661476in}{2.096515in}}%
\pgfusepath{stroke}%
\end{pgfscope}%
\begin{pgfscope}%
\pgfpathrectangle{\pgfqpoint{0.100000in}{0.212622in}}{\pgfqpoint{3.696000in}{3.696000in}}%
\pgfusepath{clip}%
\pgfsetrectcap%
\pgfsetroundjoin%
\pgfsetlinewidth{1.505625pt}%
\definecolor{currentstroke}{rgb}{1.000000,0.000000,0.000000}%
\pgfsetstrokecolor{currentstroke}%
\pgfsetdash{}{0pt}%
\pgfpathmoveto{\pgfqpoint{1.759566in}{1.955339in}}%
\pgfpathlineto{\pgfqpoint{1.661476in}{2.096515in}}%
\pgfusepath{stroke}%
\end{pgfscope}%
\begin{pgfscope}%
\pgfpathrectangle{\pgfqpoint{0.100000in}{0.212622in}}{\pgfqpoint{3.696000in}{3.696000in}}%
\pgfusepath{clip}%
\pgfsetrectcap%
\pgfsetroundjoin%
\pgfsetlinewidth{1.505625pt}%
\definecolor{currentstroke}{rgb}{1.000000,0.000000,0.000000}%
\pgfsetstrokecolor{currentstroke}%
\pgfsetdash{}{0pt}%
\pgfpathmoveto{\pgfqpoint{1.767351in}{1.963406in}}%
\pgfpathlineto{\pgfqpoint{1.670026in}{2.103746in}}%
\pgfusepath{stroke}%
\end{pgfscope}%
\begin{pgfscope}%
\pgfpathrectangle{\pgfqpoint{0.100000in}{0.212622in}}{\pgfqpoint{3.696000in}{3.696000in}}%
\pgfusepath{clip}%
\pgfsetrectcap%
\pgfsetroundjoin%
\pgfsetlinewidth{1.505625pt}%
\definecolor{currentstroke}{rgb}{1.000000,0.000000,0.000000}%
\pgfsetstrokecolor{currentstroke}%
\pgfsetdash{}{0pt}%
\pgfpathmoveto{\pgfqpoint{1.771818in}{1.966762in}}%
\pgfpathlineto{\pgfqpoint{1.678566in}{2.110968in}}%
\pgfusepath{stroke}%
\end{pgfscope}%
\begin{pgfscope}%
\pgfpathrectangle{\pgfqpoint{0.100000in}{0.212622in}}{\pgfqpoint{3.696000in}{3.696000in}}%
\pgfusepath{clip}%
\pgfsetrectcap%
\pgfsetroundjoin%
\pgfsetlinewidth{1.505625pt}%
\definecolor{currentstroke}{rgb}{1.000000,0.000000,0.000000}%
\pgfsetstrokecolor{currentstroke}%
\pgfsetdash{}{0pt}%
\pgfpathmoveto{\pgfqpoint{1.775567in}{1.966503in}}%
\pgfpathlineto{\pgfqpoint{1.678566in}{2.110968in}}%
\pgfusepath{stroke}%
\end{pgfscope}%
\begin{pgfscope}%
\pgfpathrectangle{\pgfqpoint{0.100000in}{0.212622in}}{\pgfqpoint{3.696000in}{3.696000in}}%
\pgfusepath{clip}%
\pgfsetrectcap%
\pgfsetroundjoin%
\pgfsetlinewidth{1.505625pt}%
\definecolor{currentstroke}{rgb}{1.000000,0.000000,0.000000}%
\pgfsetstrokecolor{currentstroke}%
\pgfsetdash{}{0pt}%
\pgfpathmoveto{\pgfqpoint{1.783806in}{1.977571in}}%
\pgfpathlineto{\pgfqpoint{1.687094in}{2.118181in}}%
\pgfusepath{stroke}%
\end{pgfscope}%
\begin{pgfscope}%
\pgfpathrectangle{\pgfqpoint{0.100000in}{0.212622in}}{\pgfqpoint{3.696000in}{3.696000in}}%
\pgfusepath{clip}%
\pgfsetrectcap%
\pgfsetroundjoin%
\pgfsetlinewidth{1.505625pt}%
\definecolor{currentstroke}{rgb}{1.000000,0.000000,0.000000}%
\pgfsetstrokecolor{currentstroke}%
\pgfsetdash{}{0pt}%
\pgfpathmoveto{\pgfqpoint{1.788171in}{1.980281in}}%
\pgfpathlineto{\pgfqpoint{1.687094in}{2.118181in}}%
\pgfusepath{stroke}%
\end{pgfscope}%
\begin{pgfscope}%
\pgfpathrectangle{\pgfqpoint{0.100000in}{0.212622in}}{\pgfqpoint{3.696000in}{3.696000in}}%
\pgfusepath{clip}%
\pgfsetrectcap%
\pgfsetroundjoin%
\pgfsetlinewidth{1.505625pt}%
\definecolor{currentstroke}{rgb}{1.000000,0.000000,0.000000}%
\pgfsetstrokecolor{currentstroke}%
\pgfsetdash{}{0pt}%
\pgfpathmoveto{\pgfqpoint{1.789883in}{1.981783in}}%
\pgfpathlineto{\pgfqpoint{1.687094in}{2.118181in}}%
\pgfusepath{stroke}%
\end{pgfscope}%
\begin{pgfscope}%
\pgfpathrectangle{\pgfqpoint{0.100000in}{0.212622in}}{\pgfqpoint{3.696000in}{3.696000in}}%
\pgfusepath{clip}%
\pgfsetrectcap%
\pgfsetroundjoin%
\pgfsetlinewidth{1.505625pt}%
\definecolor{currentstroke}{rgb}{1.000000,0.000000,0.000000}%
\pgfsetstrokecolor{currentstroke}%
\pgfsetdash{}{0pt}%
\pgfpathmoveto{\pgfqpoint{1.793173in}{1.983499in}}%
\pgfpathlineto{\pgfqpoint{1.687094in}{2.118181in}}%
\pgfusepath{stroke}%
\end{pgfscope}%
\begin{pgfscope}%
\pgfpathrectangle{\pgfqpoint{0.100000in}{0.212622in}}{\pgfqpoint{3.696000in}{3.696000in}}%
\pgfusepath{clip}%
\pgfsetrectcap%
\pgfsetroundjoin%
\pgfsetlinewidth{1.505625pt}%
\definecolor{currentstroke}{rgb}{1.000000,0.000000,0.000000}%
\pgfsetstrokecolor{currentstroke}%
\pgfsetdash{}{0pt}%
\pgfpathmoveto{\pgfqpoint{1.799501in}{1.991081in}}%
\pgfpathlineto{\pgfqpoint{1.695613in}{2.125385in}}%
\pgfusepath{stroke}%
\end{pgfscope}%
\begin{pgfscope}%
\pgfpathrectangle{\pgfqpoint{0.100000in}{0.212622in}}{\pgfqpoint{3.696000in}{3.696000in}}%
\pgfusepath{clip}%
\pgfsetrectcap%
\pgfsetroundjoin%
\pgfsetlinewidth{1.505625pt}%
\definecolor{currentstroke}{rgb}{1.000000,0.000000,0.000000}%
\pgfsetstrokecolor{currentstroke}%
\pgfsetdash{}{0pt}%
\pgfpathmoveto{\pgfqpoint{1.807253in}{1.994763in}}%
\pgfpathlineto{\pgfqpoint{1.704120in}{2.132580in}}%
\pgfusepath{stroke}%
\end{pgfscope}%
\begin{pgfscope}%
\pgfpathrectangle{\pgfqpoint{0.100000in}{0.212622in}}{\pgfqpoint{3.696000in}{3.696000in}}%
\pgfusepath{clip}%
\pgfsetrectcap%
\pgfsetroundjoin%
\pgfsetlinewidth{1.505625pt}%
\definecolor{currentstroke}{rgb}{1.000000,0.000000,0.000000}%
\pgfsetstrokecolor{currentstroke}%
\pgfsetdash{}{0pt}%
\pgfpathmoveto{\pgfqpoint{1.812166in}{1.996860in}}%
\pgfpathlineto{\pgfqpoint{1.876050in}{2.159760in}}%
\pgfusepath{stroke}%
\end{pgfscope}%
\begin{pgfscope}%
\pgfpathrectangle{\pgfqpoint{0.100000in}{0.212622in}}{\pgfqpoint{3.696000in}{3.696000in}}%
\pgfusepath{clip}%
\pgfsetrectcap%
\pgfsetroundjoin%
\pgfsetlinewidth{1.505625pt}%
\definecolor{currentstroke}{rgb}{1.000000,0.000000,0.000000}%
\pgfsetstrokecolor{currentstroke}%
\pgfsetdash{}{0pt}%
\pgfpathmoveto{\pgfqpoint{1.815151in}{1.998821in}}%
\pgfpathlineto{\pgfqpoint{1.876050in}{2.159760in}}%
\pgfusepath{stroke}%
\end{pgfscope}%
\begin{pgfscope}%
\pgfpathrectangle{\pgfqpoint{0.100000in}{0.212622in}}{\pgfqpoint{3.696000in}{3.696000in}}%
\pgfusepath{clip}%
\pgfsetrectcap%
\pgfsetroundjoin%
\pgfsetlinewidth{1.505625pt}%
\definecolor{currentstroke}{rgb}{1.000000,0.000000,0.000000}%
\pgfsetstrokecolor{currentstroke}%
\pgfsetdash{}{0pt}%
\pgfpathmoveto{\pgfqpoint{1.819047in}{2.000774in}}%
\pgfpathlineto{\pgfqpoint{1.876050in}{2.159760in}}%
\pgfusepath{stroke}%
\end{pgfscope}%
\begin{pgfscope}%
\pgfpathrectangle{\pgfqpoint{0.100000in}{0.212622in}}{\pgfqpoint{3.696000in}{3.696000in}}%
\pgfusepath{clip}%
\pgfsetrectcap%
\pgfsetroundjoin%
\pgfsetlinewidth{1.505625pt}%
\definecolor{currentstroke}{rgb}{1.000000,0.000000,0.000000}%
\pgfsetstrokecolor{currentstroke}%
\pgfsetdash{}{0pt}%
\pgfpathmoveto{\pgfqpoint{1.821664in}{2.002774in}}%
\pgfpathlineto{\pgfqpoint{1.876050in}{2.159760in}}%
\pgfusepath{stroke}%
\end{pgfscope}%
\begin{pgfscope}%
\pgfpathrectangle{\pgfqpoint{0.100000in}{0.212622in}}{\pgfqpoint{3.696000in}{3.696000in}}%
\pgfusepath{clip}%
\pgfsetrectcap%
\pgfsetroundjoin%
\pgfsetlinewidth{1.505625pt}%
\definecolor{currentstroke}{rgb}{1.000000,0.000000,0.000000}%
\pgfsetstrokecolor{currentstroke}%
\pgfsetdash{}{0pt}%
\pgfpathmoveto{\pgfqpoint{1.825639in}{2.005245in}}%
\pgfpathlineto{\pgfqpoint{1.876050in}{2.159760in}}%
\pgfusepath{stroke}%
\end{pgfscope}%
\begin{pgfscope}%
\pgfpathrectangle{\pgfqpoint{0.100000in}{0.212622in}}{\pgfqpoint{3.696000in}{3.696000in}}%
\pgfusepath{clip}%
\pgfsetrectcap%
\pgfsetroundjoin%
\pgfsetlinewidth{1.505625pt}%
\definecolor{currentstroke}{rgb}{1.000000,0.000000,0.000000}%
\pgfsetstrokecolor{currentstroke}%
\pgfsetdash{}{0pt}%
\pgfpathmoveto{\pgfqpoint{1.829734in}{2.005730in}}%
\pgfpathlineto{\pgfqpoint{1.876050in}{2.159760in}}%
\pgfusepath{stroke}%
\end{pgfscope}%
\begin{pgfscope}%
\pgfpathrectangle{\pgfqpoint{0.100000in}{0.212622in}}{\pgfqpoint{3.696000in}{3.696000in}}%
\pgfusepath{clip}%
\pgfsetrectcap%
\pgfsetroundjoin%
\pgfsetlinewidth{1.505625pt}%
\definecolor{currentstroke}{rgb}{1.000000,0.000000,0.000000}%
\pgfsetstrokecolor{currentstroke}%
\pgfsetdash{}{0pt}%
\pgfpathmoveto{\pgfqpoint{1.832781in}{2.006993in}}%
\pgfpathlineto{\pgfqpoint{1.876050in}{2.159760in}}%
\pgfusepath{stroke}%
\end{pgfscope}%
\begin{pgfscope}%
\pgfpathrectangle{\pgfqpoint{0.100000in}{0.212622in}}{\pgfqpoint{3.696000in}{3.696000in}}%
\pgfusepath{clip}%
\pgfsetrectcap%
\pgfsetroundjoin%
\pgfsetlinewidth{1.505625pt}%
\definecolor{currentstroke}{rgb}{1.000000,0.000000,0.000000}%
\pgfsetstrokecolor{currentstroke}%
\pgfsetdash{}{0pt}%
\pgfpathmoveto{\pgfqpoint{1.836431in}{2.009936in}}%
\pgfpathlineto{\pgfqpoint{1.876050in}{2.159760in}}%
\pgfusepath{stroke}%
\end{pgfscope}%
\begin{pgfscope}%
\pgfpathrectangle{\pgfqpoint{0.100000in}{0.212622in}}{\pgfqpoint{3.696000in}{3.696000in}}%
\pgfusepath{clip}%
\pgfsetrectcap%
\pgfsetroundjoin%
\pgfsetlinewidth{1.505625pt}%
\definecolor{currentstroke}{rgb}{1.000000,0.000000,0.000000}%
\pgfsetstrokecolor{currentstroke}%
\pgfsetdash{}{0pt}%
\pgfpathmoveto{\pgfqpoint{1.840946in}{2.010152in}}%
\pgfpathlineto{\pgfqpoint{1.876050in}{2.159760in}}%
\pgfusepath{stroke}%
\end{pgfscope}%
\begin{pgfscope}%
\pgfpathrectangle{\pgfqpoint{0.100000in}{0.212622in}}{\pgfqpoint{3.696000in}{3.696000in}}%
\pgfusepath{clip}%
\pgfsetrectcap%
\pgfsetroundjoin%
\pgfsetlinewidth{1.505625pt}%
\definecolor{currentstroke}{rgb}{1.000000,0.000000,0.000000}%
\pgfsetstrokecolor{currentstroke}%
\pgfsetdash{}{0pt}%
\pgfpathmoveto{\pgfqpoint{1.843704in}{2.012321in}}%
\pgfpathlineto{\pgfqpoint{1.876050in}{2.159760in}}%
\pgfusepath{stroke}%
\end{pgfscope}%
\begin{pgfscope}%
\pgfpathrectangle{\pgfqpoint{0.100000in}{0.212622in}}{\pgfqpoint{3.696000in}{3.696000in}}%
\pgfusepath{clip}%
\pgfsetrectcap%
\pgfsetroundjoin%
\pgfsetlinewidth{1.505625pt}%
\definecolor{currentstroke}{rgb}{1.000000,0.000000,0.000000}%
\pgfsetstrokecolor{currentstroke}%
\pgfsetdash{}{0pt}%
\pgfpathmoveto{\pgfqpoint{1.847604in}{2.013092in}}%
\pgfpathlineto{\pgfqpoint{1.876050in}{2.159760in}}%
\pgfusepath{stroke}%
\end{pgfscope}%
\begin{pgfscope}%
\pgfpathrectangle{\pgfqpoint{0.100000in}{0.212622in}}{\pgfqpoint{3.696000in}{3.696000in}}%
\pgfusepath{clip}%
\pgfsetrectcap%
\pgfsetroundjoin%
\pgfsetlinewidth{1.505625pt}%
\definecolor{currentstroke}{rgb}{1.000000,0.000000,0.000000}%
\pgfsetstrokecolor{currentstroke}%
\pgfsetdash{}{0pt}%
\pgfpathmoveto{\pgfqpoint{1.852696in}{2.015018in}}%
\pgfpathlineto{\pgfqpoint{1.876050in}{2.159760in}}%
\pgfusepath{stroke}%
\end{pgfscope}%
\begin{pgfscope}%
\pgfpathrectangle{\pgfqpoint{0.100000in}{0.212622in}}{\pgfqpoint{3.696000in}{3.696000in}}%
\pgfusepath{clip}%
\pgfsetrectcap%
\pgfsetroundjoin%
\pgfsetlinewidth{1.505625pt}%
\definecolor{currentstroke}{rgb}{1.000000,0.000000,0.000000}%
\pgfsetstrokecolor{currentstroke}%
\pgfsetdash{}{0pt}%
\pgfpathmoveto{\pgfqpoint{1.855854in}{2.017621in}}%
\pgfpathlineto{\pgfqpoint{1.876050in}{2.159760in}}%
\pgfusepath{stroke}%
\end{pgfscope}%
\begin{pgfscope}%
\pgfpathrectangle{\pgfqpoint{0.100000in}{0.212622in}}{\pgfqpoint{3.696000in}{3.696000in}}%
\pgfusepath{clip}%
\pgfsetrectcap%
\pgfsetroundjoin%
\pgfsetlinewidth{1.505625pt}%
\definecolor{currentstroke}{rgb}{1.000000,0.000000,0.000000}%
\pgfsetstrokecolor{currentstroke}%
\pgfsetdash{}{0pt}%
\pgfpathmoveto{\pgfqpoint{1.862630in}{2.017563in}}%
\pgfpathlineto{\pgfqpoint{1.889119in}{2.155992in}}%
\pgfusepath{stroke}%
\end{pgfscope}%
\begin{pgfscope}%
\pgfpathrectangle{\pgfqpoint{0.100000in}{0.212622in}}{\pgfqpoint{3.696000in}{3.696000in}}%
\pgfusepath{clip}%
\pgfsetrectcap%
\pgfsetroundjoin%
\pgfsetlinewidth{1.505625pt}%
\definecolor{currentstroke}{rgb}{1.000000,0.000000,0.000000}%
\pgfsetstrokecolor{currentstroke}%
\pgfsetdash{}{0pt}%
\pgfpathmoveto{\pgfqpoint{1.869277in}{2.019401in}}%
\pgfpathlineto{\pgfqpoint{1.889119in}{2.155992in}}%
\pgfusepath{stroke}%
\end{pgfscope}%
\begin{pgfscope}%
\pgfpathrectangle{\pgfqpoint{0.100000in}{0.212622in}}{\pgfqpoint{3.696000in}{3.696000in}}%
\pgfusepath{clip}%
\pgfsetrectcap%
\pgfsetroundjoin%
\pgfsetlinewidth{1.505625pt}%
\definecolor{currentstroke}{rgb}{1.000000,0.000000,0.000000}%
\pgfsetstrokecolor{currentstroke}%
\pgfsetdash{}{0pt}%
\pgfpathmoveto{\pgfqpoint{1.873954in}{2.022437in}}%
\pgfpathlineto{\pgfqpoint{1.889119in}{2.155992in}}%
\pgfusepath{stroke}%
\end{pgfscope}%
\begin{pgfscope}%
\pgfpathrectangle{\pgfqpoint{0.100000in}{0.212622in}}{\pgfqpoint{3.696000in}{3.696000in}}%
\pgfusepath{clip}%
\pgfsetrectcap%
\pgfsetroundjoin%
\pgfsetlinewidth{1.505625pt}%
\definecolor{currentstroke}{rgb}{1.000000,0.000000,0.000000}%
\pgfsetstrokecolor{currentstroke}%
\pgfsetdash{}{0pt}%
\pgfpathmoveto{\pgfqpoint{1.879478in}{2.026228in}}%
\pgfpathlineto{\pgfqpoint{1.889119in}{2.155992in}}%
\pgfusepath{stroke}%
\end{pgfscope}%
\begin{pgfscope}%
\pgfpathrectangle{\pgfqpoint{0.100000in}{0.212622in}}{\pgfqpoint{3.696000in}{3.696000in}}%
\pgfusepath{clip}%
\pgfsetrectcap%
\pgfsetroundjoin%
\pgfsetlinewidth{1.505625pt}%
\definecolor{currentstroke}{rgb}{1.000000,0.000000,0.000000}%
\pgfsetstrokecolor{currentstroke}%
\pgfsetdash{}{0pt}%
\pgfpathmoveto{\pgfqpoint{1.885657in}{2.029848in}}%
\pgfpathlineto{\pgfqpoint{1.889119in}{2.155992in}}%
\pgfusepath{stroke}%
\end{pgfscope}%
\begin{pgfscope}%
\pgfpathrectangle{\pgfqpoint{0.100000in}{0.212622in}}{\pgfqpoint{3.696000in}{3.696000in}}%
\pgfusepath{clip}%
\pgfsetrectcap%
\pgfsetroundjoin%
\pgfsetlinewidth{1.505625pt}%
\definecolor{currentstroke}{rgb}{1.000000,0.000000,0.000000}%
\pgfsetstrokecolor{currentstroke}%
\pgfsetdash{}{0pt}%
\pgfpathmoveto{\pgfqpoint{1.889775in}{2.031765in}}%
\pgfpathlineto{\pgfqpoint{1.889119in}{2.155992in}}%
\pgfusepath{stroke}%
\end{pgfscope}%
\begin{pgfscope}%
\pgfpathrectangle{\pgfqpoint{0.100000in}{0.212622in}}{\pgfqpoint{3.696000in}{3.696000in}}%
\pgfusepath{clip}%
\pgfsetrectcap%
\pgfsetroundjoin%
\pgfsetlinewidth{1.505625pt}%
\definecolor{currentstroke}{rgb}{1.000000,0.000000,0.000000}%
\pgfsetstrokecolor{currentstroke}%
\pgfsetdash{}{0pt}%
\pgfpathmoveto{\pgfqpoint{1.894540in}{2.035492in}}%
\pgfpathlineto{\pgfqpoint{1.889119in}{2.155992in}}%
\pgfusepath{stroke}%
\end{pgfscope}%
\begin{pgfscope}%
\pgfpathrectangle{\pgfqpoint{0.100000in}{0.212622in}}{\pgfqpoint{3.696000in}{3.696000in}}%
\pgfusepath{clip}%
\pgfsetrectcap%
\pgfsetroundjoin%
\pgfsetlinewidth{1.505625pt}%
\definecolor{currentstroke}{rgb}{1.000000,0.000000,0.000000}%
\pgfsetstrokecolor{currentstroke}%
\pgfsetdash{}{0pt}%
\pgfpathmoveto{\pgfqpoint{1.900587in}{2.039562in}}%
\pgfpathlineto{\pgfqpoint{1.889119in}{2.155992in}}%
\pgfusepath{stroke}%
\end{pgfscope}%
\begin{pgfscope}%
\pgfpathrectangle{\pgfqpoint{0.100000in}{0.212622in}}{\pgfqpoint{3.696000in}{3.696000in}}%
\pgfusepath{clip}%
\pgfsetrectcap%
\pgfsetroundjoin%
\pgfsetlinewidth{1.505625pt}%
\definecolor{currentstroke}{rgb}{1.000000,0.000000,0.000000}%
\pgfsetstrokecolor{currentstroke}%
\pgfsetdash{}{0pt}%
\pgfpathmoveto{\pgfqpoint{1.904493in}{2.042538in}}%
\pgfpathlineto{\pgfqpoint{1.889119in}{2.155992in}}%
\pgfusepath{stroke}%
\end{pgfscope}%
\begin{pgfscope}%
\pgfpathrectangle{\pgfqpoint{0.100000in}{0.212622in}}{\pgfqpoint{3.696000in}{3.696000in}}%
\pgfusepath{clip}%
\pgfsetrectcap%
\pgfsetroundjoin%
\pgfsetlinewidth{1.505625pt}%
\definecolor{currentstroke}{rgb}{1.000000,0.000000,0.000000}%
\pgfsetstrokecolor{currentstroke}%
\pgfsetdash{}{0pt}%
\pgfpathmoveto{\pgfqpoint{1.909042in}{2.045112in}}%
\pgfpathlineto{\pgfqpoint{1.889119in}{2.155992in}}%
\pgfusepath{stroke}%
\end{pgfscope}%
\begin{pgfscope}%
\pgfpathrectangle{\pgfqpoint{0.100000in}{0.212622in}}{\pgfqpoint{3.696000in}{3.696000in}}%
\pgfusepath{clip}%
\pgfsetrectcap%
\pgfsetroundjoin%
\pgfsetlinewidth{1.505625pt}%
\definecolor{currentstroke}{rgb}{1.000000,0.000000,0.000000}%
\pgfsetstrokecolor{currentstroke}%
\pgfsetdash{}{0pt}%
\pgfpathmoveto{\pgfqpoint{1.915862in}{2.046057in}}%
\pgfpathlineto{\pgfqpoint{1.889119in}{2.155992in}}%
\pgfusepath{stroke}%
\end{pgfscope}%
\begin{pgfscope}%
\pgfpathrectangle{\pgfqpoint{0.100000in}{0.212622in}}{\pgfqpoint{3.696000in}{3.696000in}}%
\pgfusepath{clip}%
\pgfsetrectcap%
\pgfsetroundjoin%
\pgfsetlinewidth{1.505625pt}%
\definecolor{currentstroke}{rgb}{1.000000,0.000000,0.000000}%
\pgfsetstrokecolor{currentstroke}%
\pgfsetdash{}{0pt}%
\pgfpathmoveto{\pgfqpoint{1.920009in}{2.048495in}}%
\pgfpathlineto{\pgfqpoint{1.889119in}{2.155992in}}%
\pgfusepath{stroke}%
\end{pgfscope}%
\begin{pgfscope}%
\pgfpathrectangle{\pgfqpoint{0.100000in}{0.212622in}}{\pgfqpoint{3.696000in}{3.696000in}}%
\pgfusepath{clip}%
\pgfsetrectcap%
\pgfsetroundjoin%
\pgfsetlinewidth{1.505625pt}%
\definecolor{currentstroke}{rgb}{1.000000,0.000000,0.000000}%
\pgfsetstrokecolor{currentstroke}%
\pgfsetdash{}{0pt}%
\pgfpathmoveto{\pgfqpoint{1.922368in}{2.049691in}}%
\pgfpathlineto{\pgfqpoint{1.889119in}{2.155992in}}%
\pgfusepath{stroke}%
\end{pgfscope}%
\begin{pgfscope}%
\pgfpathrectangle{\pgfqpoint{0.100000in}{0.212622in}}{\pgfqpoint{3.696000in}{3.696000in}}%
\pgfusepath{clip}%
\pgfsetrectcap%
\pgfsetroundjoin%
\pgfsetlinewidth{1.505625pt}%
\definecolor{currentstroke}{rgb}{1.000000,0.000000,0.000000}%
\pgfsetstrokecolor{currentstroke}%
\pgfsetdash{}{0pt}%
\pgfpathmoveto{\pgfqpoint{1.926896in}{2.051456in}}%
\pgfpathlineto{\pgfqpoint{1.889119in}{2.155992in}}%
\pgfusepath{stroke}%
\end{pgfscope}%
\begin{pgfscope}%
\pgfpathrectangle{\pgfqpoint{0.100000in}{0.212622in}}{\pgfqpoint{3.696000in}{3.696000in}}%
\pgfusepath{clip}%
\pgfsetrectcap%
\pgfsetroundjoin%
\pgfsetlinewidth{1.505625pt}%
\definecolor{currentstroke}{rgb}{1.000000,0.000000,0.000000}%
\pgfsetstrokecolor{currentstroke}%
\pgfsetdash{}{0pt}%
\pgfpathmoveto{\pgfqpoint{1.931388in}{2.052091in}}%
\pgfpathlineto{\pgfqpoint{1.902198in}{2.152221in}}%
\pgfusepath{stroke}%
\end{pgfscope}%
\begin{pgfscope}%
\pgfpathrectangle{\pgfqpoint{0.100000in}{0.212622in}}{\pgfqpoint{3.696000in}{3.696000in}}%
\pgfusepath{clip}%
\pgfsetrectcap%
\pgfsetroundjoin%
\pgfsetlinewidth{1.505625pt}%
\definecolor{currentstroke}{rgb}{1.000000,0.000000,0.000000}%
\pgfsetstrokecolor{currentstroke}%
\pgfsetdash{}{0pt}%
\pgfpathmoveto{\pgfqpoint{1.937012in}{2.053242in}}%
\pgfpathlineto{\pgfqpoint{1.902198in}{2.152221in}}%
\pgfusepath{stroke}%
\end{pgfscope}%
\begin{pgfscope}%
\pgfpathrectangle{\pgfqpoint{0.100000in}{0.212622in}}{\pgfqpoint{3.696000in}{3.696000in}}%
\pgfusepath{clip}%
\pgfsetrectcap%
\pgfsetroundjoin%
\pgfsetlinewidth{1.505625pt}%
\definecolor{currentstroke}{rgb}{1.000000,0.000000,0.000000}%
\pgfsetstrokecolor{currentstroke}%
\pgfsetdash{}{0pt}%
\pgfpathmoveto{\pgfqpoint{1.940183in}{2.052892in}}%
\pgfpathlineto{\pgfqpoint{1.902198in}{2.152221in}}%
\pgfusepath{stroke}%
\end{pgfscope}%
\begin{pgfscope}%
\pgfpathrectangle{\pgfqpoint{0.100000in}{0.212622in}}{\pgfqpoint{3.696000in}{3.696000in}}%
\pgfusepath{clip}%
\pgfsetrectcap%
\pgfsetroundjoin%
\pgfsetlinewidth{1.505625pt}%
\definecolor{currentstroke}{rgb}{1.000000,0.000000,0.000000}%
\pgfsetstrokecolor{currentstroke}%
\pgfsetdash{}{0pt}%
\pgfpathmoveto{\pgfqpoint{1.941953in}{2.053989in}}%
\pgfpathlineto{\pgfqpoint{1.902198in}{2.152221in}}%
\pgfusepath{stroke}%
\end{pgfscope}%
\begin{pgfscope}%
\pgfpathrectangle{\pgfqpoint{0.100000in}{0.212622in}}{\pgfqpoint{3.696000in}{3.696000in}}%
\pgfusepath{clip}%
\pgfsetrectcap%
\pgfsetroundjoin%
\pgfsetlinewidth{1.505625pt}%
\definecolor{currentstroke}{rgb}{1.000000,0.000000,0.000000}%
\pgfsetstrokecolor{currentstroke}%
\pgfsetdash{}{0pt}%
\pgfpathmoveto{\pgfqpoint{1.942972in}{2.054285in}}%
\pgfpathlineto{\pgfqpoint{1.902198in}{2.152221in}}%
\pgfusepath{stroke}%
\end{pgfscope}%
\begin{pgfscope}%
\pgfpathrectangle{\pgfqpoint{0.100000in}{0.212622in}}{\pgfqpoint{3.696000in}{3.696000in}}%
\pgfusepath{clip}%
\pgfsetrectcap%
\pgfsetroundjoin%
\pgfsetlinewidth{1.505625pt}%
\definecolor{currentstroke}{rgb}{1.000000,0.000000,0.000000}%
\pgfsetstrokecolor{currentstroke}%
\pgfsetdash{}{0pt}%
\pgfpathmoveto{\pgfqpoint{1.944486in}{2.054630in}}%
\pgfpathlineto{\pgfqpoint{1.902198in}{2.152221in}}%
\pgfusepath{stroke}%
\end{pgfscope}%
\begin{pgfscope}%
\pgfpathrectangle{\pgfqpoint{0.100000in}{0.212622in}}{\pgfqpoint{3.696000in}{3.696000in}}%
\pgfusepath{clip}%
\pgfsetrectcap%
\pgfsetroundjoin%
\pgfsetlinewidth{1.505625pt}%
\definecolor{currentstroke}{rgb}{1.000000,0.000000,0.000000}%
\pgfsetstrokecolor{currentstroke}%
\pgfsetdash{}{0pt}%
\pgfpathmoveto{\pgfqpoint{1.946313in}{2.055124in}}%
\pgfpathlineto{\pgfqpoint{1.902198in}{2.152221in}}%
\pgfusepath{stroke}%
\end{pgfscope}%
\begin{pgfscope}%
\pgfpathrectangle{\pgfqpoint{0.100000in}{0.212622in}}{\pgfqpoint{3.696000in}{3.696000in}}%
\pgfusepath{clip}%
\pgfsetrectcap%
\pgfsetroundjoin%
\pgfsetlinewidth{1.505625pt}%
\definecolor{currentstroke}{rgb}{1.000000,0.000000,0.000000}%
\pgfsetstrokecolor{currentstroke}%
\pgfsetdash{}{0pt}%
\pgfpathmoveto{\pgfqpoint{1.947408in}{2.055396in}}%
\pgfpathlineto{\pgfqpoint{1.902198in}{2.152221in}}%
\pgfusepath{stroke}%
\end{pgfscope}%
\begin{pgfscope}%
\pgfpathrectangle{\pgfqpoint{0.100000in}{0.212622in}}{\pgfqpoint{3.696000in}{3.696000in}}%
\pgfusepath{clip}%
\pgfsetrectcap%
\pgfsetroundjoin%
\pgfsetlinewidth{1.505625pt}%
\definecolor{currentstroke}{rgb}{1.000000,0.000000,0.000000}%
\pgfsetstrokecolor{currentstroke}%
\pgfsetdash{}{0pt}%
\pgfpathmoveto{\pgfqpoint{1.947990in}{2.055575in}}%
\pgfpathlineto{\pgfqpoint{1.902198in}{2.152221in}}%
\pgfusepath{stroke}%
\end{pgfscope}%
\begin{pgfscope}%
\pgfpathrectangle{\pgfqpoint{0.100000in}{0.212622in}}{\pgfqpoint{3.696000in}{3.696000in}}%
\pgfusepath{clip}%
\pgfsetrectcap%
\pgfsetroundjoin%
\pgfsetlinewidth{1.505625pt}%
\definecolor{currentstroke}{rgb}{1.000000,0.000000,0.000000}%
\pgfsetstrokecolor{currentstroke}%
\pgfsetdash{}{0pt}%
\pgfpathmoveto{\pgfqpoint{1.948313in}{2.055670in}}%
\pgfpathlineto{\pgfqpoint{1.902198in}{2.152221in}}%
\pgfusepath{stroke}%
\end{pgfscope}%
\begin{pgfscope}%
\pgfpathrectangle{\pgfqpoint{0.100000in}{0.212622in}}{\pgfqpoint{3.696000in}{3.696000in}}%
\pgfusepath{clip}%
\pgfsetrectcap%
\pgfsetroundjoin%
\pgfsetlinewidth{1.505625pt}%
\definecolor{currentstroke}{rgb}{1.000000,0.000000,0.000000}%
\pgfsetstrokecolor{currentstroke}%
\pgfsetdash{}{0pt}%
\pgfpathmoveto{\pgfqpoint{1.948492in}{2.055713in}}%
\pgfpathlineto{\pgfqpoint{1.902198in}{2.152221in}}%
\pgfusepath{stroke}%
\end{pgfscope}%
\begin{pgfscope}%
\pgfpathrectangle{\pgfqpoint{0.100000in}{0.212622in}}{\pgfqpoint{3.696000in}{3.696000in}}%
\pgfusepath{clip}%
\pgfsetrectcap%
\pgfsetroundjoin%
\pgfsetlinewidth{1.505625pt}%
\definecolor{currentstroke}{rgb}{1.000000,0.000000,0.000000}%
\pgfsetstrokecolor{currentstroke}%
\pgfsetdash{}{0pt}%
\pgfpathmoveto{\pgfqpoint{1.948583in}{2.055742in}}%
\pgfpathlineto{\pgfqpoint{1.902198in}{2.152221in}}%
\pgfusepath{stroke}%
\end{pgfscope}%
\begin{pgfscope}%
\pgfpathrectangle{\pgfqpoint{0.100000in}{0.212622in}}{\pgfqpoint{3.696000in}{3.696000in}}%
\pgfusepath{clip}%
\pgfsetrectcap%
\pgfsetroundjoin%
\pgfsetlinewidth{1.505625pt}%
\definecolor{currentstroke}{rgb}{1.000000,0.000000,0.000000}%
\pgfsetstrokecolor{currentstroke}%
\pgfsetdash{}{0pt}%
\pgfpathmoveto{\pgfqpoint{1.948642in}{2.055733in}}%
\pgfpathlineto{\pgfqpoint{1.902198in}{2.152221in}}%
\pgfusepath{stroke}%
\end{pgfscope}%
\begin{pgfscope}%
\pgfpathrectangle{\pgfqpoint{0.100000in}{0.212622in}}{\pgfqpoint{3.696000in}{3.696000in}}%
\pgfusepath{clip}%
\pgfsetrectcap%
\pgfsetroundjoin%
\pgfsetlinewidth{1.505625pt}%
\definecolor{currentstroke}{rgb}{1.000000,0.000000,0.000000}%
\pgfsetstrokecolor{currentstroke}%
\pgfsetdash{}{0pt}%
\pgfpathmoveto{\pgfqpoint{1.949252in}{2.055049in}}%
\pgfpathlineto{\pgfqpoint{1.902198in}{2.152221in}}%
\pgfusepath{stroke}%
\end{pgfscope}%
\begin{pgfscope}%
\pgfpathrectangle{\pgfqpoint{0.100000in}{0.212622in}}{\pgfqpoint{3.696000in}{3.696000in}}%
\pgfusepath{clip}%
\pgfsetrectcap%
\pgfsetroundjoin%
\pgfsetlinewidth{1.505625pt}%
\definecolor{currentstroke}{rgb}{1.000000,0.000000,0.000000}%
\pgfsetstrokecolor{currentstroke}%
\pgfsetdash{}{0pt}%
\pgfpathmoveto{\pgfqpoint{1.949612in}{2.055079in}}%
\pgfpathlineto{\pgfqpoint{1.902198in}{2.152221in}}%
\pgfusepath{stroke}%
\end{pgfscope}%
\begin{pgfscope}%
\pgfpathrectangle{\pgfqpoint{0.100000in}{0.212622in}}{\pgfqpoint{3.696000in}{3.696000in}}%
\pgfusepath{clip}%
\pgfsetrectcap%
\pgfsetroundjoin%
\pgfsetlinewidth{1.505625pt}%
\definecolor{currentstroke}{rgb}{1.000000,0.000000,0.000000}%
\pgfsetstrokecolor{currentstroke}%
\pgfsetdash{}{0pt}%
\pgfpathmoveto{\pgfqpoint{1.949818in}{2.054935in}}%
\pgfpathlineto{\pgfqpoint{1.902198in}{2.152221in}}%
\pgfusepath{stroke}%
\end{pgfscope}%
\begin{pgfscope}%
\pgfpathrectangle{\pgfqpoint{0.100000in}{0.212622in}}{\pgfqpoint{3.696000in}{3.696000in}}%
\pgfusepath{clip}%
\pgfsetrectcap%
\pgfsetroundjoin%
\pgfsetlinewidth{1.505625pt}%
\definecolor{currentstroke}{rgb}{1.000000,0.000000,0.000000}%
\pgfsetstrokecolor{currentstroke}%
\pgfsetdash{}{0pt}%
\pgfpathmoveto{\pgfqpoint{1.950563in}{2.054701in}}%
\pgfpathlineto{\pgfqpoint{1.902198in}{2.152221in}}%
\pgfusepath{stroke}%
\end{pgfscope}%
\begin{pgfscope}%
\pgfpathrectangle{\pgfqpoint{0.100000in}{0.212622in}}{\pgfqpoint{3.696000in}{3.696000in}}%
\pgfusepath{clip}%
\pgfsetrectcap%
\pgfsetroundjoin%
\pgfsetlinewidth{1.505625pt}%
\definecolor{currentstroke}{rgb}{1.000000,0.000000,0.000000}%
\pgfsetstrokecolor{currentstroke}%
\pgfsetdash{}{0pt}%
\pgfpathmoveto{\pgfqpoint{1.950916in}{2.054321in}}%
\pgfpathlineto{\pgfqpoint{1.902198in}{2.152221in}}%
\pgfusepath{stroke}%
\end{pgfscope}%
\begin{pgfscope}%
\pgfpathrectangle{\pgfqpoint{0.100000in}{0.212622in}}{\pgfqpoint{3.696000in}{3.696000in}}%
\pgfusepath{clip}%
\pgfsetrectcap%
\pgfsetroundjoin%
\pgfsetlinewidth{1.505625pt}%
\definecolor{currentstroke}{rgb}{1.000000,0.000000,0.000000}%
\pgfsetstrokecolor{currentstroke}%
\pgfsetdash{}{0pt}%
\pgfpathmoveto{\pgfqpoint{1.951094in}{2.054262in}}%
\pgfpathlineto{\pgfqpoint{1.902198in}{2.152221in}}%
\pgfusepath{stroke}%
\end{pgfscope}%
\begin{pgfscope}%
\pgfpathrectangle{\pgfqpoint{0.100000in}{0.212622in}}{\pgfqpoint{3.696000in}{3.696000in}}%
\pgfusepath{clip}%
\pgfsetrectcap%
\pgfsetroundjoin%
\pgfsetlinewidth{1.505625pt}%
\definecolor{currentstroke}{rgb}{1.000000,0.000000,0.000000}%
\pgfsetstrokecolor{currentstroke}%
\pgfsetdash{}{0pt}%
\pgfpathmoveto{\pgfqpoint{1.951168in}{2.054267in}}%
\pgfpathlineto{\pgfqpoint{1.902198in}{2.152221in}}%
\pgfusepath{stroke}%
\end{pgfscope}%
\begin{pgfscope}%
\pgfpathrectangle{\pgfqpoint{0.100000in}{0.212622in}}{\pgfqpoint{3.696000in}{3.696000in}}%
\pgfusepath{clip}%
\pgfsetrectcap%
\pgfsetroundjoin%
\pgfsetlinewidth{1.505625pt}%
\definecolor{currentstroke}{rgb}{1.000000,0.000000,0.000000}%
\pgfsetstrokecolor{currentstroke}%
\pgfsetdash{}{0pt}%
\pgfpathmoveto{\pgfqpoint{1.951195in}{2.054246in}}%
\pgfpathlineto{\pgfqpoint{1.902198in}{2.152221in}}%
\pgfusepath{stroke}%
\end{pgfscope}%
\begin{pgfscope}%
\pgfpathrectangle{\pgfqpoint{0.100000in}{0.212622in}}{\pgfqpoint{3.696000in}{3.696000in}}%
\pgfusepath{clip}%
\pgfsetrectcap%
\pgfsetroundjoin%
\pgfsetlinewidth{1.505625pt}%
\definecolor{currentstroke}{rgb}{1.000000,0.000000,0.000000}%
\pgfsetstrokecolor{currentstroke}%
\pgfsetdash{}{0pt}%
\pgfpathmoveto{\pgfqpoint{1.951206in}{2.054240in}}%
\pgfpathlineto{\pgfqpoint{1.902198in}{2.152221in}}%
\pgfusepath{stroke}%
\end{pgfscope}%
\begin{pgfscope}%
\pgfpathrectangle{\pgfqpoint{0.100000in}{0.212622in}}{\pgfqpoint{3.696000in}{3.696000in}}%
\pgfusepath{clip}%
\pgfsetrectcap%
\pgfsetroundjoin%
\pgfsetlinewidth{1.505625pt}%
\definecolor{currentstroke}{rgb}{1.000000,0.000000,0.000000}%
\pgfsetstrokecolor{currentstroke}%
\pgfsetdash{}{0pt}%
\pgfpathmoveto{\pgfqpoint{1.951269in}{2.053964in}}%
\pgfpathlineto{\pgfqpoint{1.902198in}{2.152221in}}%
\pgfusepath{stroke}%
\end{pgfscope}%
\begin{pgfscope}%
\pgfpathrectangle{\pgfqpoint{0.100000in}{0.212622in}}{\pgfqpoint{3.696000in}{3.696000in}}%
\pgfusepath{clip}%
\pgfsetrectcap%
\pgfsetroundjoin%
\pgfsetlinewidth{1.505625pt}%
\definecolor{currentstroke}{rgb}{1.000000,0.000000,0.000000}%
\pgfsetstrokecolor{currentstroke}%
\pgfsetdash{}{0pt}%
\pgfpathmoveto{\pgfqpoint{1.951334in}{2.054283in}}%
\pgfpathlineto{\pgfqpoint{1.902198in}{2.152221in}}%
\pgfusepath{stroke}%
\end{pgfscope}%
\begin{pgfscope}%
\pgfpathrectangle{\pgfqpoint{0.100000in}{0.212622in}}{\pgfqpoint{3.696000in}{3.696000in}}%
\pgfusepath{clip}%
\pgfsetrectcap%
\pgfsetroundjoin%
\pgfsetlinewidth{1.505625pt}%
\definecolor{currentstroke}{rgb}{1.000000,0.000000,0.000000}%
\pgfsetstrokecolor{currentstroke}%
\pgfsetdash{}{0pt}%
\pgfpathmoveto{\pgfqpoint{1.951334in}{2.054280in}}%
\pgfpathlineto{\pgfqpoint{1.902198in}{2.152221in}}%
\pgfusepath{stroke}%
\end{pgfscope}%
\begin{pgfscope}%
\pgfpathrectangle{\pgfqpoint{0.100000in}{0.212622in}}{\pgfqpoint{3.696000in}{3.696000in}}%
\pgfusepath{clip}%
\pgfsetrectcap%
\pgfsetroundjoin%
\pgfsetlinewidth{1.505625pt}%
\definecolor{currentstroke}{rgb}{1.000000,0.000000,0.000000}%
\pgfsetstrokecolor{currentstroke}%
\pgfsetdash{}{0pt}%
\pgfpathmoveto{\pgfqpoint{1.951315in}{2.054134in}}%
\pgfpathlineto{\pgfqpoint{1.902198in}{2.152221in}}%
\pgfusepath{stroke}%
\end{pgfscope}%
\begin{pgfscope}%
\pgfpathrectangle{\pgfqpoint{0.100000in}{0.212622in}}{\pgfqpoint{3.696000in}{3.696000in}}%
\pgfusepath{clip}%
\pgfsetrectcap%
\pgfsetroundjoin%
\pgfsetlinewidth{1.505625pt}%
\definecolor{currentstroke}{rgb}{1.000000,0.000000,0.000000}%
\pgfsetstrokecolor{currentstroke}%
\pgfsetdash{}{0pt}%
\pgfpathmoveto{\pgfqpoint{1.951257in}{2.054059in}}%
\pgfpathlineto{\pgfqpoint{1.902198in}{2.152221in}}%
\pgfusepath{stroke}%
\end{pgfscope}%
\begin{pgfscope}%
\pgfpathrectangle{\pgfqpoint{0.100000in}{0.212622in}}{\pgfqpoint{3.696000in}{3.696000in}}%
\pgfusepath{clip}%
\pgfsetrectcap%
\pgfsetroundjoin%
\pgfsetlinewidth{1.505625pt}%
\definecolor{currentstroke}{rgb}{1.000000,0.000000,0.000000}%
\pgfsetstrokecolor{currentstroke}%
\pgfsetdash{}{0pt}%
\pgfpathmoveto{\pgfqpoint{1.951178in}{2.053946in}}%
\pgfpathlineto{\pgfqpoint{1.902198in}{2.152221in}}%
\pgfusepath{stroke}%
\end{pgfscope}%
\begin{pgfscope}%
\pgfpathrectangle{\pgfqpoint{0.100000in}{0.212622in}}{\pgfqpoint{3.696000in}{3.696000in}}%
\pgfusepath{clip}%
\pgfsetrectcap%
\pgfsetroundjoin%
\pgfsetlinewidth{1.505625pt}%
\definecolor{currentstroke}{rgb}{1.000000,0.000000,0.000000}%
\pgfsetstrokecolor{currentstroke}%
\pgfsetdash{}{0pt}%
\pgfpathmoveto{\pgfqpoint{1.951026in}{2.054485in}}%
\pgfpathlineto{\pgfqpoint{1.902198in}{2.152221in}}%
\pgfusepath{stroke}%
\end{pgfscope}%
\begin{pgfscope}%
\pgfpathrectangle{\pgfqpoint{0.100000in}{0.212622in}}{\pgfqpoint{3.696000in}{3.696000in}}%
\pgfusepath{clip}%
\pgfsetrectcap%
\pgfsetroundjoin%
\pgfsetlinewidth{1.505625pt}%
\definecolor{currentstroke}{rgb}{1.000000,0.000000,0.000000}%
\pgfsetstrokecolor{currentstroke}%
\pgfsetdash{}{0pt}%
\pgfpathmoveto{\pgfqpoint{1.951405in}{2.051892in}}%
\pgfpathlineto{\pgfqpoint{1.902198in}{2.152221in}}%
\pgfusepath{stroke}%
\end{pgfscope}%
\begin{pgfscope}%
\pgfpathrectangle{\pgfqpoint{0.100000in}{0.212622in}}{\pgfqpoint{3.696000in}{3.696000in}}%
\pgfusepath{clip}%
\pgfsetrectcap%
\pgfsetroundjoin%
\pgfsetlinewidth{1.505625pt}%
\definecolor{currentstroke}{rgb}{1.000000,0.000000,0.000000}%
\pgfsetstrokecolor{currentstroke}%
\pgfsetdash{}{0pt}%
\pgfpathmoveto{\pgfqpoint{1.952205in}{2.049369in}}%
\pgfpathlineto{\pgfqpoint{1.915284in}{2.148447in}}%
\pgfusepath{stroke}%
\end{pgfscope}%
\begin{pgfscope}%
\pgfpathrectangle{\pgfqpoint{0.100000in}{0.212622in}}{\pgfqpoint{3.696000in}{3.696000in}}%
\pgfusepath{clip}%
\pgfsetrectcap%
\pgfsetroundjoin%
\pgfsetlinewidth{1.505625pt}%
\definecolor{currentstroke}{rgb}{1.000000,0.000000,0.000000}%
\pgfsetstrokecolor{currentstroke}%
\pgfsetdash{}{0pt}%
\pgfpathmoveto{\pgfqpoint{1.953631in}{2.051890in}}%
\pgfpathlineto{\pgfqpoint{1.915284in}{2.148447in}}%
\pgfusepath{stroke}%
\end{pgfscope}%
\begin{pgfscope}%
\pgfpathrectangle{\pgfqpoint{0.100000in}{0.212622in}}{\pgfqpoint{3.696000in}{3.696000in}}%
\pgfusepath{clip}%
\pgfsetrectcap%
\pgfsetroundjoin%
\pgfsetlinewidth{1.505625pt}%
\definecolor{currentstroke}{rgb}{1.000000,0.000000,0.000000}%
\pgfsetstrokecolor{currentstroke}%
\pgfsetdash{}{0pt}%
\pgfpathmoveto{\pgfqpoint{1.954148in}{2.048182in}}%
\pgfpathlineto{\pgfqpoint{1.915284in}{2.148447in}}%
\pgfusepath{stroke}%
\end{pgfscope}%
\begin{pgfscope}%
\pgfpathrectangle{\pgfqpoint{0.100000in}{0.212622in}}{\pgfqpoint{3.696000in}{3.696000in}}%
\pgfusepath{clip}%
\pgfsetrectcap%
\pgfsetroundjoin%
\pgfsetlinewidth{1.505625pt}%
\definecolor{currentstroke}{rgb}{1.000000,0.000000,0.000000}%
\pgfsetstrokecolor{currentstroke}%
\pgfsetdash{}{0pt}%
\pgfpathmoveto{\pgfqpoint{1.956741in}{2.044008in}}%
\pgfpathlineto{\pgfqpoint{1.928380in}{2.144672in}}%
\pgfusepath{stroke}%
\end{pgfscope}%
\begin{pgfscope}%
\pgfpathrectangle{\pgfqpoint{0.100000in}{0.212622in}}{\pgfqpoint{3.696000in}{3.696000in}}%
\pgfusepath{clip}%
\pgfsetrectcap%
\pgfsetroundjoin%
\pgfsetlinewidth{1.505625pt}%
\definecolor{currentstroke}{rgb}{1.000000,0.000000,0.000000}%
\pgfsetstrokecolor{currentstroke}%
\pgfsetdash{}{0pt}%
\pgfpathmoveto{\pgfqpoint{1.958161in}{2.044808in}}%
\pgfpathlineto{\pgfqpoint{1.928380in}{2.144672in}}%
\pgfusepath{stroke}%
\end{pgfscope}%
\begin{pgfscope}%
\pgfpathrectangle{\pgfqpoint{0.100000in}{0.212622in}}{\pgfqpoint{3.696000in}{3.696000in}}%
\pgfusepath{clip}%
\pgfsetrectcap%
\pgfsetroundjoin%
\pgfsetlinewidth{1.505625pt}%
\definecolor{currentstroke}{rgb}{1.000000,0.000000,0.000000}%
\pgfsetstrokecolor{currentstroke}%
\pgfsetdash{}{0pt}%
\pgfpathmoveto{\pgfqpoint{1.958574in}{2.041629in}}%
\pgfpathlineto{\pgfqpoint{1.928380in}{2.144672in}}%
\pgfusepath{stroke}%
\end{pgfscope}%
\begin{pgfscope}%
\pgfpathrectangle{\pgfqpoint{0.100000in}{0.212622in}}{\pgfqpoint{3.696000in}{3.696000in}}%
\pgfusepath{clip}%
\pgfsetrectcap%
\pgfsetroundjoin%
\pgfsetlinewidth{1.505625pt}%
\definecolor{currentstroke}{rgb}{1.000000,0.000000,0.000000}%
\pgfsetstrokecolor{currentstroke}%
\pgfsetdash{}{0pt}%
\pgfpathmoveto{\pgfqpoint{1.959619in}{2.040870in}}%
\pgfpathlineto{\pgfqpoint{1.928380in}{2.144672in}}%
\pgfusepath{stroke}%
\end{pgfscope}%
\begin{pgfscope}%
\pgfpathrectangle{\pgfqpoint{0.100000in}{0.212622in}}{\pgfqpoint{3.696000in}{3.696000in}}%
\pgfusepath{clip}%
\pgfsetrectcap%
\pgfsetroundjoin%
\pgfsetlinewidth{1.505625pt}%
\definecolor{currentstroke}{rgb}{1.000000,0.000000,0.000000}%
\pgfsetstrokecolor{currentstroke}%
\pgfsetdash{}{0pt}%
\pgfpathmoveto{\pgfqpoint{1.961435in}{2.042463in}}%
\pgfpathlineto{\pgfqpoint{1.941484in}{2.140894in}}%
\pgfusepath{stroke}%
\end{pgfscope}%
\begin{pgfscope}%
\pgfpathrectangle{\pgfqpoint{0.100000in}{0.212622in}}{\pgfqpoint{3.696000in}{3.696000in}}%
\pgfusepath{clip}%
\pgfsetrectcap%
\pgfsetroundjoin%
\pgfsetlinewidth{1.505625pt}%
\definecolor{currentstroke}{rgb}{1.000000,0.000000,0.000000}%
\pgfsetstrokecolor{currentstroke}%
\pgfsetdash{}{0pt}%
\pgfpathmoveto{\pgfqpoint{1.961618in}{2.041646in}}%
\pgfpathlineto{\pgfqpoint{1.941484in}{2.140894in}}%
\pgfusepath{stroke}%
\end{pgfscope}%
\begin{pgfscope}%
\pgfpathrectangle{\pgfqpoint{0.100000in}{0.212622in}}{\pgfqpoint{3.696000in}{3.696000in}}%
\pgfusepath{clip}%
\pgfsetrectcap%
\pgfsetroundjoin%
\pgfsetlinewidth{1.505625pt}%
\definecolor{currentstroke}{rgb}{1.000000,0.000000,0.000000}%
\pgfsetstrokecolor{currentstroke}%
\pgfsetdash{}{0pt}%
\pgfpathmoveto{\pgfqpoint{1.963072in}{2.040729in}}%
\pgfpathlineto{\pgfqpoint{1.941484in}{2.140894in}}%
\pgfusepath{stroke}%
\end{pgfscope}%
\begin{pgfscope}%
\pgfpathrectangle{\pgfqpoint{0.100000in}{0.212622in}}{\pgfqpoint{3.696000in}{3.696000in}}%
\pgfusepath{clip}%
\pgfsetrectcap%
\pgfsetroundjoin%
\pgfsetlinewidth{1.505625pt}%
\definecolor{currentstroke}{rgb}{1.000000,0.000000,0.000000}%
\pgfsetstrokecolor{currentstroke}%
\pgfsetdash{}{0pt}%
\pgfpathmoveto{\pgfqpoint{1.964962in}{2.040893in}}%
\pgfpathlineto{\pgfqpoint{1.941484in}{2.140894in}}%
\pgfusepath{stroke}%
\end{pgfscope}%
\begin{pgfscope}%
\pgfpathrectangle{\pgfqpoint{0.100000in}{0.212622in}}{\pgfqpoint{3.696000in}{3.696000in}}%
\pgfusepath{clip}%
\pgfsetrectcap%
\pgfsetroundjoin%
\pgfsetlinewidth{1.505625pt}%
\definecolor{currentstroke}{rgb}{1.000000,0.000000,0.000000}%
\pgfsetstrokecolor{currentstroke}%
\pgfsetdash{}{0pt}%
\pgfpathmoveto{\pgfqpoint{1.965670in}{2.037582in}}%
\pgfpathlineto{\pgfqpoint{1.954597in}{2.137113in}}%
\pgfusepath{stroke}%
\end{pgfscope}%
\begin{pgfscope}%
\pgfpathrectangle{\pgfqpoint{0.100000in}{0.212622in}}{\pgfqpoint{3.696000in}{3.696000in}}%
\pgfusepath{clip}%
\pgfsetrectcap%
\pgfsetroundjoin%
\pgfsetlinewidth{1.505625pt}%
\definecolor{currentstroke}{rgb}{1.000000,0.000000,0.000000}%
\pgfsetstrokecolor{currentstroke}%
\pgfsetdash{}{0pt}%
\pgfpathmoveto{\pgfqpoint{1.967987in}{2.037365in}}%
\pgfpathlineto{\pgfqpoint{1.954597in}{2.137113in}}%
\pgfusepath{stroke}%
\end{pgfscope}%
\begin{pgfscope}%
\pgfpathrectangle{\pgfqpoint{0.100000in}{0.212622in}}{\pgfqpoint{3.696000in}{3.696000in}}%
\pgfusepath{clip}%
\pgfsetrectcap%
\pgfsetroundjoin%
\pgfsetlinewidth{1.505625pt}%
\definecolor{currentstroke}{rgb}{1.000000,0.000000,0.000000}%
\pgfsetstrokecolor{currentstroke}%
\pgfsetdash{}{0pt}%
\pgfpathmoveto{\pgfqpoint{1.970741in}{2.038854in}}%
\pgfpathlineto{\pgfqpoint{1.967718in}{2.133330in}}%
\pgfusepath{stroke}%
\end{pgfscope}%
\begin{pgfscope}%
\pgfpathrectangle{\pgfqpoint{0.100000in}{0.212622in}}{\pgfqpoint{3.696000in}{3.696000in}}%
\pgfusepath{clip}%
\pgfsetrectcap%
\pgfsetroundjoin%
\pgfsetlinewidth{1.505625pt}%
\definecolor{currentstroke}{rgb}{1.000000,0.000000,0.000000}%
\pgfsetstrokecolor{currentstroke}%
\pgfsetdash{}{0pt}%
\pgfpathmoveto{\pgfqpoint{1.971625in}{2.037867in}}%
\pgfpathlineto{\pgfqpoint{1.967718in}{2.133330in}}%
\pgfusepath{stroke}%
\end{pgfscope}%
\begin{pgfscope}%
\pgfpathrectangle{\pgfqpoint{0.100000in}{0.212622in}}{\pgfqpoint{3.696000in}{3.696000in}}%
\pgfusepath{clip}%
\pgfsetrectcap%
\pgfsetroundjoin%
\pgfsetlinewidth{1.505625pt}%
\definecolor{currentstroke}{rgb}{1.000000,0.000000,0.000000}%
\pgfsetstrokecolor{currentstroke}%
\pgfsetdash{}{0pt}%
\pgfpathmoveto{\pgfqpoint{1.973457in}{2.037221in}}%
\pgfpathlineto{\pgfqpoint{1.967718in}{2.133330in}}%
\pgfusepath{stroke}%
\end{pgfscope}%
\begin{pgfscope}%
\pgfpathrectangle{\pgfqpoint{0.100000in}{0.212622in}}{\pgfqpoint{3.696000in}{3.696000in}}%
\pgfusepath{clip}%
\pgfsetrectcap%
\pgfsetroundjoin%
\pgfsetlinewidth{1.505625pt}%
\definecolor{currentstroke}{rgb}{1.000000,0.000000,0.000000}%
\pgfsetstrokecolor{currentstroke}%
\pgfsetdash{}{0pt}%
\pgfpathmoveto{\pgfqpoint{1.976093in}{2.038900in}}%
\pgfpathlineto{\pgfqpoint{1.980848in}{2.129544in}}%
\pgfusepath{stroke}%
\end{pgfscope}%
\begin{pgfscope}%
\pgfpathrectangle{\pgfqpoint{0.100000in}{0.212622in}}{\pgfqpoint{3.696000in}{3.696000in}}%
\pgfusepath{clip}%
\pgfsetrectcap%
\pgfsetroundjoin%
\pgfsetlinewidth{1.505625pt}%
\definecolor{currentstroke}{rgb}{1.000000,0.000000,0.000000}%
\pgfsetstrokecolor{currentstroke}%
\pgfsetdash{}{0pt}%
\pgfpathmoveto{\pgfqpoint{1.977742in}{2.036837in}}%
\pgfpathlineto{\pgfqpoint{1.980848in}{2.129544in}}%
\pgfusepath{stroke}%
\end{pgfscope}%
\begin{pgfscope}%
\pgfpathrectangle{\pgfqpoint{0.100000in}{0.212622in}}{\pgfqpoint{3.696000in}{3.696000in}}%
\pgfusepath{clip}%
\pgfsetrectcap%
\pgfsetroundjoin%
\pgfsetlinewidth{1.505625pt}%
\definecolor{currentstroke}{rgb}{1.000000,0.000000,0.000000}%
\pgfsetstrokecolor{currentstroke}%
\pgfsetdash{}{0pt}%
\pgfpathmoveto{\pgfqpoint{1.980262in}{2.031542in}}%
\pgfpathlineto{\pgfqpoint{1.993986in}{2.125756in}}%
\pgfusepath{stroke}%
\end{pgfscope}%
\begin{pgfscope}%
\pgfpathrectangle{\pgfqpoint{0.100000in}{0.212622in}}{\pgfqpoint{3.696000in}{3.696000in}}%
\pgfusepath{clip}%
\pgfsetrectcap%
\pgfsetroundjoin%
\pgfsetlinewidth{1.505625pt}%
\definecolor{currentstroke}{rgb}{1.000000,0.000000,0.000000}%
\pgfsetstrokecolor{currentstroke}%
\pgfsetdash{}{0pt}%
\pgfpathmoveto{\pgfqpoint{1.984160in}{2.033384in}}%
\pgfpathlineto{\pgfqpoint{1.993986in}{2.125756in}}%
\pgfusepath{stroke}%
\end{pgfscope}%
\begin{pgfscope}%
\pgfpathrectangle{\pgfqpoint{0.100000in}{0.212622in}}{\pgfqpoint{3.696000in}{3.696000in}}%
\pgfusepath{clip}%
\pgfsetrectcap%
\pgfsetroundjoin%
\pgfsetlinewidth{1.505625pt}%
\definecolor{currentstroke}{rgb}{1.000000,0.000000,0.000000}%
\pgfsetstrokecolor{currentstroke}%
\pgfsetdash{}{0pt}%
\pgfpathmoveto{\pgfqpoint{1.988033in}{2.021883in}}%
\pgfpathlineto{\pgfqpoint{2.007134in}{2.121965in}}%
\pgfusepath{stroke}%
\end{pgfscope}%
\begin{pgfscope}%
\pgfpathrectangle{\pgfqpoint{0.100000in}{0.212622in}}{\pgfqpoint{3.696000in}{3.696000in}}%
\pgfusepath{clip}%
\pgfsetrectcap%
\pgfsetroundjoin%
\pgfsetlinewidth{1.505625pt}%
\definecolor{currentstroke}{rgb}{1.000000,0.000000,0.000000}%
\pgfsetstrokecolor{currentstroke}%
\pgfsetdash{}{0pt}%
\pgfpathmoveto{\pgfqpoint{1.989544in}{2.016996in}}%
\pgfpathlineto{\pgfqpoint{2.020289in}{2.118172in}}%
\pgfusepath{stroke}%
\end{pgfscope}%
\begin{pgfscope}%
\pgfpathrectangle{\pgfqpoint{0.100000in}{0.212622in}}{\pgfqpoint{3.696000in}{3.696000in}}%
\pgfusepath{clip}%
\pgfsetrectcap%
\pgfsetroundjoin%
\pgfsetlinewidth{1.505625pt}%
\definecolor{currentstroke}{rgb}{1.000000,0.000000,0.000000}%
\pgfsetstrokecolor{currentstroke}%
\pgfsetdash{}{0pt}%
\pgfpathmoveto{\pgfqpoint{1.993204in}{2.013751in}}%
\pgfpathlineto{\pgfqpoint{2.033454in}{2.114376in}}%
\pgfusepath{stroke}%
\end{pgfscope}%
\begin{pgfscope}%
\pgfpathrectangle{\pgfqpoint{0.100000in}{0.212622in}}{\pgfqpoint{3.696000in}{3.696000in}}%
\pgfusepath{clip}%
\pgfsetrectcap%
\pgfsetroundjoin%
\pgfsetlinewidth{1.505625pt}%
\definecolor{currentstroke}{rgb}{1.000000,0.000000,0.000000}%
\pgfsetstrokecolor{currentstroke}%
\pgfsetdash{}{0pt}%
\pgfpathmoveto{\pgfqpoint{1.997161in}{2.009337in}}%
\pgfpathlineto{\pgfqpoint{2.046627in}{2.110578in}}%
\pgfusepath{stroke}%
\end{pgfscope}%
\begin{pgfscope}%
\pgfpathrectangle{\pgfqpoint{0.100000in}{0.212622in}}{\pgfqpoint{3.696000in}{3.696000in}}%
\pgfusepath{clip}%
\pgfsetrectcap%
\pgfsetroundjoin%
\pgfsetlinewidth{1.505625pt}%
\definecolor{currentstroke}{rgb}{1.000000,0.000000,0.000000}%
\pgfsetstrokecolor{currentstroke}%
\pgfsetdash{}{0pt}%
\pgfpathmoveto{\pgfqpoint{1.997183in}{2.010809in}}%
\pgfpathlineto{\pgfqpoint{2.059809in}{2.106778in}}%
\pgfusepath{stroke}%
\end{pgfscope}%
\begin{pgfscope}%
\pgfpathrectangle{\pgfqpoint{0.100000in}{0.212622in}}{\pgfqpoint{3.696000in}{3.696000in}}%
\pgfusepath{clip}%
\pgfsetrectcap%
\pgfsetroundjoin%
\pgfsetlinewidth{1.505625pt}%
\definecolor{currentstroke}{rgb}{1.000000,0.000000,0.000000}%
\pgfsetstrokecolor{currentstroke}%
\pgfsetdash{}{0pt}%
\pgfpathmoveto{\pgfqpoint{1.998458in}{2.000644in}}%
\pgfpathlineto{\pgfqpoint{2.059809in}{2.106778in}}%
\pgfusepath{stroke}%
\end{pgfscope}%
\begin{pgfscope}%
\pgfpathrectangle{\pgfqpoint{0.100000in}{0.212622in}}{\pgfqpoint{3.696000in}{3.696000in}}%
\pgfusepath{clip}%
\pgfsetrectcap%
\pgfsetroundjoin%
\pgfsetlinewidth{1.505625pt}%
\definecolor{currentstroke}{rgb}{1.000000,0.000000,0.000000}%
\pgfsetstrokecolor{currentstroke}%
\pgfsetdash{}{0pt}%
\pgfpathmoveto{\pgfqpoint{2.000258in}{1.992874in}}%
\pgfpathlineto{\pgfqpoint{2.073000in}{2.102974in}}%
\pgfusepath{stroke}%
\end{pgfscope}%
\begin{pgfscope}%
\pgfpathrectangle{\pgfqpoint{0.100000in}{0.212622in}}{\pgfqpoint{3.696000in}{3.696000in}}%
\pgfusepath{clip}%
\pgfsetrectcap%
\pgfsetroundjoin%
\pgfsetlinewidth{1.505625pt}%
\definecolor{currentstroke}{rgb}{1.000000,0.000000,0.000000}%
\pgfsetstrokecolor{currentstroke}%
\pgfsetdash{}{0pt}%
\pgfpathmoveto{\pgfqpoint{1.999286in}{1.984826in}}%
\pgfpathlineto{\pgfqpoint{2.086199in}{2.099169in}}%
\pgfusepath{stroke}%
\end{pgfscope}%
\begin{pgfscope}%
\pgfpathrectangle{\pgfqpoint{0.100000in}{0.212622in}}{\pgfqpoint{3.696000in}{3.696000in}}%
\pgfusepath{clip}%
\pgfsetrectcap%
\pgfsetroundjoin%
\pgfsetlinewidth{1.505625pt}%
\definecolor{currentstroke}{rgb}{1.000000,0.000000,0.000000}%
\pgfsetstrokecolor{currentstroke}%
\pgfsetdash{}{0pt}%
\pgfpathmoveto{\pgfqpoint{2.003842in}{1.981830in}}%
\pgfpathlineto{\pgfqpoint{2.112624in}{2.091550in}}%
\pgfusepath{stroke}%
\end{pgfscope}%
\begin{pgfscope}%
\pgfpathrectangle{\pgfqpoint{0.100000in}{0.212622in}}{\pgfqpoint{3.696000in}{3.696000in}}%
\pgfusepath{clip}%
\pgfsetrectcap%
\pgfsetroundjoin%
\pgfsetlinewidth{1.505625pt}%
\definecolor{currentstroke}{rgb}{1.000000,0.000000,0.000000}%
\pgfsetstrokecolor{currentstroke}%
\pgfsetdash{}{0pt}%
\pgfpathmoveto{\pgfqpoint{2.005560in}{1.978398in}}%
\pgfpathlineto{\pgfqpoint{2.112624in}{2.091550in}}%
\pgfusepath{stroke}%
\end{pgfscope}%
\begin{pgfscope}%
\pgfpathrectangle{\pgfqpoint{0.100000in}{0.212622in}}{\pgfqpoint{3.696000in}{3.696000in}}%
\pgfusepath{clip}%
\pgfsetrectcap%
\pgfsetroundjoin%
\pgfsetlinewidth{1.505625pt}%
\definecolor{currentstroke}{rgb}{1.000000,0.000000,0.000000}%
\pgfsetstrokecolor{currentstroke}%
\pgfsetdash{}{0pt}%
\pgfpathmoveto{\pgfqpoint{2.005466in}{1.973596in}}%
\pgfpathlineto{\pgfqpoint{2.125849in}{2.087737in}}%
\pgfusepath{stroke}%
\end{pgfscope}%
\begin{pgfscope}%
\pgfpathrectangle{\pgfqpoint{0.100000in}{0.212622in}}{\pgfqpoint{3.696000in}{3.696000in}}%
\pgfusepath{clip}%
\pgfsetrectcap%
\pgfsetroundjoin%
\pgfsetlinewidth{1.505625pt}%
\definecolor{currentstroke}{rgb}{1.000000,0.000000,0.000000}%
\pgfsetstrokecolor{currentstroke}%
\pgfsetdash{}{0pt}%
\pgfpathmoveto{\pgfqpoint{2.008475in}{1.970405in}}%
\pgfpathlineto{\pgfqpoint{2.139083in}{2.083921in}}%
\pgfusepath{stroke}%
\end{pgfscope}%
\begin{pgfscope}%
\pgfpathrectangle{\pgfqpoint{0.100000in}{0.212622in}}{\pgfqpoint{3.696000in}{3.696000in}}%
\pgfusepath{clip}%
\pgfsetrectcap%
\pgfsetroundjoin%
\pgfsetlinewidth{1.505625pt}%
\definecolor{currentstroke}{rgb}{1.000000,0.000000,0.000000}%
\pgfsetstrokecolor{currentstroke}%
\pgfsetdash{}{0pt}%
\pgfpathmoveto{\pgfqpoint{2.013129in}{1.967883in}}%
\pgfpathlineto{\pgfqpoint{2.152326in}{2.080103in}}%
\pgfusepath{stroke}%
\end{pgfscope}%
\begin{pgfscope}%
\pgfpathrectangle{\pgfqpoint{0.100000in}{0.212622in}}{\pgfqpoint{3.696000in}{3.696000in}}%
\pgfusepath{clip}%
\pgfsetrectcap%
\pgfsetroundjoin%
\pgfsetlinewidth{1.505625pt}%
\definecolor{currentstroke}{rgb}{1.000000,0.000000,0.000000}%
\pgfsetstrokecolor{currentstroke}%
\pgfsetdash{}{0pt}%
\pgfpathmoveto{\pgfqpoint{2.013597in}{1.964145in}}%
\pgfpathlineto{\pgfqpoint{2.152326in}{2.080103in}}%
\pgfusepath{stroke}%
\end{pgfscope}%
\begin{pgfscope}%
\pgfpathrectangle{\pgfqpoint{0.100000in}{0.212622in}}{\pgfqpoint{3.696000in}{3.696000in}}%
\pgfusepath{clip}%
\pgfsetrectcap%
\pgfsetroundjoin%
\pgfsetlinewidth{1.505625pt}%
\definecolor{currentstroke}{rgb}{1.000000,0.000000,0.000000}%
\pgfsetstrokecolor{currentstroke}%
\pgfsetdash{}{0pt}%
\pgfpathmoveto{\pgfqpoint{2.014848in}{1.962489in}}%
\pgfpathlineto{\pgfqpoint{2.152326in}{2.080103in}}%
\pgfusepath{stroke}%
\end{pgfscope}%
\begin{pgfscope}%
\pgfpathrectangle{\pgfqpoint{0.100000in}{0.212622in}}{\pgfqpoint{3.696000in}{3.696000in}}%
\pgfusepath{clip}%
\pgfsetrectcap%
\pgfsetroundjoin%
\pgfsetlinewidth{1.505625pt}%
\definecolor{currentstroke}{rgb}{1.000000,0.000000,0.000000}%
\pgfsetstrokecolor{currentstroke}%
\pgfsetdash{}{0pt}%
\pgfpathmoveto{\pgfqpoint{2.016367in}{1.961434in}}%
\pgfpathlineto{\pgfqpoint{2.165578in}{2.076282in}}%
\pgfusepath{stroke}%
\end{pgfscope}%
\begin{pgfscope}%
\pgfpathrectangle{\pgfqpoint{0.100000in}{0.212622in}}{\pgfqpoint{3.696000in}{3.696000in}}%
\pgfusepath{clip}%
\pgfsetrectcap%
\pgfsetroundjoin%
\pgfsetlinewidth{1.505625pt}%
\definecolor{currentstroke}{rgb}{1.000000,0.000000,0.000000}%
\pgfsetstrokecolor{currentstroke}%
\pgfsetdash{}{0pt}%
\pgfpathmoveto{\pgfqpoint{2.017032in}{1.956839in}}%
\pgfpathlineto{\pgfqpoint{2.165578in}{2.076282in}}%
\pgfusepath{stroke}%
\end{pgfscope}%
\begin{pgfscope}%
\pgfpathrectangle{\pgfqpoint{0.100000in}{0.212622in}}{\pgfqpoint{3.696000in}{3.696000in}}%
\pgfusepath{clip}%
\pgfsetrectcap%
\pgfsetroundjoin%
\pgfsetlinewidth{1.505625pt}%
\definecolor{currentstroke}{rgb}{1.000000,0.000000,0.000000}%
\pgfsetstrokecolor{currentstroke}%
\pgfsetdash{}{0pt}%
\pgfpathmoveto{\pgfqpoint{2.018394in}{1.953066in}}%
\pgfpathlineto{\pgfqpoint{2.165578in}{2.076282in}}%
\pgfusepath{stroke}%
\end{pgfscope}%
\begin{pgfscope}%
\pgfpathrectangle{\pgfqpoint{0.100000in}{0.212622in}}{\pgfqpoint{3.696000in}{3.696000in}}%
\pgfusepath{clip}%
\pgfsetrectcap%
\pgfsetroundjoin%
\pgfsetlinewidth{1.505625pt}%
\definecolor{currentstroke}{rgb}{1.000000,0.000000,0.000000}%
\pgfsetstrokecolor{currentstroke}%
\pgfsetdash{}{0pt}%
\pgfpathmoveto{\pgfqpoint{2.021019in}{1.949857in}}%
\pgfpathlineto{\pgfqpoint{2.178838in}{2.072459in}}%
\pgfusepath{stroke}%
\end{pgfscope}%
\begin{pgfscope}%
\pgfpathrectangle{\pgfqpoint{0.100000in}{0.212622in}}{\pgfqpoint{3.696000in}{3.696000in}}%
\pgfusepath{clip}%
\pgfsetrectcap%
\pgfsetroundjoin%
\pgfsetlinewidth{1.505625pt}%
\definecolor{currentstroke}{rgb}{1.000000,0.000000,0.000000}%
\pgfsetstrokecolor{currentstroke}%
\pgfsetdash{}{0pt}%
\pgfpathmoveto{\pgfqpoint{2.022096in}{1.950074in}}%
\pgfpathlineto{\pgfqpoint{2.178838in}{2.072459in}}%
\pgfusepath{stroke}%
\end{pgfscope}%
\begin{pgfscope}%
\pgfpathrectangle{\pgfqpoint{0.100000in}{0.212622in}}{\pgfqpoint{3.696000in}{3.696000in}}%
\pgfusepath{clip}%
\pgfsetrectcap%
\pgfsetroundjoin%
\pgfsetlinewidth{1.505625pt}%
\definecolor{currentstroke}{rgb}{1.000000,0.000000,0.000000}%
\pgfsetstrokecolor{currentstroke}%
\pgfsetdash{}{0pt}%
\pgfpathmoveto{\pgfqpoint{2.022231in}{1.948636in}}%
\pgfpathlineto{\pgfqpoint{2.192107in}{2.068633in}}%
\pgfusepath{stroke}%
\end{pgfscope}%
\begin{pgfscope}%
\pgfpathrectangle{\pgfqpoint{0.100000in}{0.212622in}}{\pgfqpoint{3.696000in}{3.696000in}}%
\pgfusepath{clip}%
\pgfsetrectcap%
\pgfsetroundjoin%
\pgfsetlinewidth{1.505625pt}%
\definecolor{currentstroke}{rgb}{1.000000,0.000000,0.000000}%
\pgfsetstrokecolor{currentstroke}%
\pgfsetdash{}{0pt}%
\pgfpathmoveto{\pgfqpoint{2.022693in}{1.945831in}}%
\pgfpathlineto{\pgfqpoint{2.192107in}{2.068633in}}%
\pgfusepath{stroke}%
\end{pgfscope}%
\begin{pgfscope}%
\pgfpathrectangle{\pgfqpoint{0.100000in}{0.212622in}}{\pgfqpoint{3.696000in}{3.696000in}}%
\pgfusepath{clip}%
\pgfsetrectcap%
\pgfsetroundjoin%
\pgfsetlinewidth{1.505625pt}%
\definecolor{currentstroke}{rgb}{1.000000,0.000000,0.000000}%
\pgfsetstrokecolor{currentstroke}%
\pgfsetdash{}{0pt}%
\pgfpathmoveto{\pgfqpoint{2.023742in}{1.945397in}}%
\pgfpathlineto{\pgfqpoint{2.192107in}{2.068633in}}%
\pgfusepath{stroke}%
\end{pgfscope}%
\begin{pgfscope}%
\pgfpathrectangle{\pgfqpoint{0.100000in}{0.212622in}}{\pgfqpoint{3.696000in}{3.696000in}}%
\pgfusepath{clip}%
\pgfsetrectcap%
\pgfsetroundjoin%
\pgfsetlinewidth{1.505625pt}%
\definecolor{currentstroke}{rgb}{1.000000,0.000000,0.000000}%
\pgfsetstrokecolor{currentstroke}%
\pgfsetdash{}{0pt}%
\pgfpathmoveto{\pgfqpoint{2.024469in}{1.943671in}}%
\pgfpathlineto{\pgfqpoint{2.192107in}{2.068633in}}%
\pgfusepath{stroke}%
\end{pgfscope}%
\begin{pgfscope}%
\pgfpathrectangle{\pgfqpoint{0.100000in}{0.212622in}}{\pgfqpoint{3.696000in}{3.696000in}}%
\pgfusepath{clip}%
\pgfsetrectcap%
\pgfsetroundjoin%
\pgfsetlinewidth{1.505625pt}%
\definecolor{currentstroke}{rgb}{1.000000,0.000000,0.000000}%
\pgfsetstrokecolor{currentstroke}%
\pgfsetdash{}{0pt}%
\pgfpathmoveto{\pgfqpoint{2.024819in}{1.942715in}}%
\pgfpathlineto{\pgfqpoint{2.205385in}{2.064805in}}%
\pgfusepath{stroke}%
\end{pgfscope}%
\begin{pgfscope}%
\pgfpathrectangle{\pgfqpoint{0.100000in}{0.212622in}}{\pgfqpoint{3.696000in}{3.696000in}}%
\pgfusepath{clip}%
\pgfsetrectcap%
\pgfsetroundjoin%
\pgfsetlinewidth{1.505625pt}%
\definecolor{currentstroke}{rgb}{1.000000,0.000000,0.000000}%
\pgfsetstrokecolor{currentstroke}%
\pgfsetdash{}{0pt}%
\pgfpathmoveto{\pgfqpoint{2.025201in}{1.942200in}}%
\pgfpathlineto{\pgfqpoint{2.205385in}{2.064805in}}%
\pgfusepath{stroke}%
\end{pgfscope}%
\begin{pgfscope}%
\pgfpathrectangle{\pgfqpoint{0.100000in}{0.212622in}}{\pgfqpoint{3.696000in}{3.696000in}}%
\pgfusepath{clip}%
\pgfsetrectcap%
\pgfsetroundjoin%
\pgfsetlinewidth{1.505625pt}%
\definecolor{currentstroke}{rgb}{1.000000,0.000000,0.000000}%
\pgfsetstrokecolor{currentstroke}%
\pgfsetdash{}{0pt}%
\pgfpathmoveto{\pgfqpoint{2.025367in}{1.942368in}}%
\pgfpathlineto{\pgfqpoint{2.205385in}{2.064805in}}%
\pgfusepath{stroke}%
\end{pgfscope}%
\begin{pgfscope}%
\pgfpathrectangle{\pgfqpoint{0.100000in}{0.212622in}}{\pgfqpoint{3.696000in}{3.696000in}}%
\pgfusepath{clip}%
\pgfsetrectcap%
\pgfsetroundjoin%
\pgfsetlinewidth{1.505625pt}%
\definecolor{currentstroke}{rgb}{1.000000,0.000000,0.000000}%
\pgfsetstrokecolor{currentstroke}%
\pgfsetdash{}{0pt}%
\pgfpathmoveto{\pgfqpoint{2.025424in}{1.942155in}}%
\pgfpathlineto{\pgfqpoint{2.205385in}{2.064805in}}%
\pgfusepath{stroke}%
\end{pgfscope}%
\begin{pgfscope}%
\pgfpathrectangle{\pgfqpoint{0.100000in}{0.212622in}}{\pgfqpoint{3.696000in}{3.696000in}}%
\pgfusepath{clip}%
\pgfsetrectcap%
\pgfsetroundjoin%
\pgfsetlinewidth{1.505625pt}%
\definecolor{currentstroke}{rgb}{1.000000,0.000000,0.000000}%
\pgfsetstrokecolor{currentstroke}%
\pgfsetdash{}{0pt}%
\pgfpathmoveto{\pgfqpoint{2.025830in}{1.941100in}}%
\pgfpathlineto{\pgfqpoint{2.205385in}{2.064805in}}%
\pgfusepath{stroke}%
\end{pgfscope}%
\begin{pgfscope}%
\pgfpathrectangle{\pgfqpoint{0.100000in}{0.212622in}}{\pgfqpoint{3.696000in}{3.696000in}}%
\pgfusepath{clip}%
\pgfsetrectcap%
\pgfsetroundjoin%
\pgfsetlinewidth{1.505625pt}%
\definecolor{currentstroke}{rgb}{1.000000,0.000000,0.000000}%
\pgfsetstrokecolor{currentstroke}%
\pgfsetdash{}{0pt}%
\pgfpathmoveto{\pgfqpoint{2.026151in}{1.941120in}}%
\pgfpathlineto{\pgfqpoint{2.205385in}{2.064805in}}%
\pgfusepath{stroke}%
\end{pgfscope}%
\begin{pgfscope}%
\pgfpathrectangle{\pgfqpoint{0.100000in}{0.212622in}}{\pgfqpoint{3.696000in}{3.696000in}}%
\pgfusepath{clip}%
\pgfsetrectcap%
\pgfsetroundjoin%
\pgfsetlinewidth{1.505625pt}%
\definecolor{currentstroke}{rgb}{1.000000,0.000000,0.000000}%
\pgfsetstrokecolor{currentstroke}%
\pgfsetdash{}{0pt}%
\pgfpathmoveto{\pgfqpoint{2.026606in}{1.940563in}}%
\pgfpathlineto{\pgfqpoint{2.205385in}{2.064805in}}%
\pgfusepath{stroke}%
\end{pgfscope}%
\begin{pgfscope}%
\pgfpathrectangle{\pgfqpoint{0.100000in}{0.212622in}}{\pgfqpoint{3.696000in}{3.696000in}}%
\pgfusepath{clip}%
\pgfsetrectcap%
\pgfsetroundjoin%
\pgfsetlinewidth{1.505625pt}%
\definecolor{currentstroke}{rgb}{1.000000,0.000000,0.000000}%
\pgfsetstrokecolor{currentstroke}%
\pgfsetdash{}{0pt}%
\pgfpathmoveto{\pgfqpoint{2.027046in}{1.939654in}}%
\pgfpathlineto{\pgfqpoint{2.205385in}{2.064805in}}%
\pgfusepath{stroke}%
\end{pgfscope}%
\begin{pgfscope}%
\pgfpathrectangle{\pgfqpoint{0.100000in}{0.212622in}}{\pgfqpoint{3.696000in}{3.696000in}}%
\pgfusepath{clip}%
\pgfsetrectcap%
\pgfsetroundjoin%
\pgfsetlinewidth{1.505625pt}%
\definecolor{currentstroke}{rgb}{1.000000,0.000000,0.000000}%
\pgfsetstrokecolor{currentstroke}%
\pgfsetdash{}{0pt}%
\pgfpathmoveto{\pgfqpoint{2.027649in}{1.938129in}}%
\pgfpathlineto{\pgfqpoint{2.205385in}{2.064805in}}%
\pgfusepath{stroke}%
\end{pgfscope}%
\begin{pgfscope}%
\pgfpathrectangle{\pgfqpoint{0.100000in}{0.212622in}}{\pgfqpoint{3.696000in}{3.696000in}}%
\pgfusepath{clip}%
\pgfsetrectcap%
\pgfsetroundjoin%
\pgfsetlinewidth{1.505625pt}%
\definecolor{currentstroke}{rgb}{1.000000,0.000000,0.000000}%
\pgfsetstrokecolor{currentstroke}%
\pgfsetdash{}{0pt}%
\pgfpathmoveto{\pgfqpoint{2.029074in}{1.937759in}}%
\pgfpathlineto{\pgfqpoint{2.218672in}{2.060974in}}%
\pgfusepath{stroke}%
\end{pgfscope}%
\begin{pgfscope}%
\pgfpathrectangle{\pgfqpoint{0.100000in}{0.212622in}}{\pgfqpoint{3.696000in}{3.696000in}}%
\pgfusepath{clip}%
\pgfsetrectcap%
\pgfsetroundjoin%
\pgfsetlinewidth{1.505625pt}%
\definecolor{currentstroke}{rgb}{1.000000,0.000000,0.000000}%
\pgfsetstrokecolor{currentstroke}%
\pgfsetdash{}{0pt}%
\pgfpathmoveto{\pgfqpoint{2.029495in}{1.935292in}}%
\pgfpathlineto{\pgfqpoint{2.218672in}{2.060974in}}%
\pgfusepath{stroke}%
\end{pgfscope}%
\begin{pgfscope}%
\pgfpathrectangle{\pgfqpoint{0.100000in}{0.212622in}}{\pgfqpoint{3.696000in}{3.696000in}}%
\pgfusepath{clip}%
\pgfsetrectcap%
\pgfsetroundjoin%
\pgfsetlinewidth{1.505625pt}%
\definecolor{currentstroke}{rgb}{1.000000,0.000000,0.000000}%
\pgfsetstrokecolor{currentstroke}%
\pgfsetdash{}{0pt}%
\pgfpathmoveto{\pgfqpoint{2.030248in}{1.936640in}}%
\pgfpathlineto{\pgfqpoint{2.218672in}{2.060974in}}%
\pgfusepath{stroke}%
\end{pgfscope}%
\begin{pgfscope}%
\pgfpathrectangle{\pgfqpoint{0.100000in}{0.212622in}}{\pgfqpoint{3.696000in}{3.696000in}}%
\pgfusepath{clip}%
\pgfsetrectcap%
\pgfsetroundjoin%
\pgfsetlinewidth{1.505625pt}%
\definecolor{currentstroke}{rgb}{1.000000,0.000000,0.000000}%
\pgfsetstrokecolor{currentstroke}%
\pgfsetdash{}{0pt}%
\pgfpathmoveto{\pgfqpoint{2.031504in}{1.935734in}}%
\pgfpathlineto{\pgfqpoint{2.231967in}{2.057140in}}%
\pgfusepath{stroke}%
\end{pgfscope}%
\begin{pgfscope}%
\pgfpathrectangle{\pgfqpoint{0.100000in}{0.212622in}}{\pgfqpoint{3.696000in}{3.696000in}}%
\pgfusepath{clip}%
\pgfsetrectcap%
\pgfsetroundjoin%
\pgfsetlinewidth{1.505625pt}%
\definecolor{currentstroke}{rgb}{1.000000,0.000000,0.000000}%
\pgfsetstrokecolor{currentstroke}%
\pgfsetdash{}{0pt}%
\pgfpathmoveto{\pgfqpoint{2.032441in}{1.935160in}}%
\pgfpathlineto{\pgfqpoint{2.231967in}{2.057140in}}%
\pgfusepath{stroke}%
\end{pgfscope}%
\begin{pgfscope}%
\pgfpathrectangle{\pgfqpoint{0.100000in}{0.212622in}}{\pgfqpoint{3.696000in}{3.696000in}}%
\pgfusepath{clip}%
\pgfsetrectcap%
\pgfsetroundjoin%
\pgfsetlinewidth{1.505625pt}%
\definecolor{currentstroke}{rgb}{1.000000,0.000000,0.000000}%
\pgfsetstrokecolor{currentstroke}%
\pgfsetdash{}{0pt}%
\pgfpathmoveto{\pgfqpoint{2.032780in}{1.934071in}}%
\pgfpathlineto{\pgfqpoint{2.231967in}{2.057140in}}%
\pgfusepath{stroke}%
\end{pgfscope}%
\begin{pgfscope}%
\pgfpathrectangle{\pgfqpoint{0.100000in}{0.212622in}}{\pgfqpoint{3.696000in}{3.696000in}}%
\pgfusepath{clip}%
\pgfsetrectcap%
\pgfsetroundjoin%
\pgfsetlinewidth{1.505625pt}%
\definecolor{currentstroke}{rgb}{1.000000,0.000000,0.000000}%
\pgfsetstrokecolor{currentstroke}%
\pgfsetdash{}{0pt}%
\pgfpathmoveto{\pgfqpoint{2.033392in}{1.933423in}}%
\pgfpathlineto{\pgfqpoint{2.231967in}{2.057140in}}%
\pgfusepath{stroke}%
\end{pgfscope}%
\begin{pgfscope}%
\pgfpathrectangle{\pgfqpoint{0.100000in}{0.212622in}}{\pgfqpoint{3.696000in}{3.696000in}}%
\pgfusepath{clip}%
\pgfsetrectcap%
\pgfsetroundjoin%
\pgfsetlinewidth{1.505625pt}%
\definecolor{currentstroke}{rgb}{1.000000,0.000000,0.000000}%
\pgfsetstrokecolor{currentstroke}%
\pgfsetdash{}{0pt}%
\pgfpathmoveto{\pgfqpoint{2.033933in}{1.933675in}}%
\pgfpathlineto{\pgfqpoint{2.231967in}{2.057140in}}%
\pgfusepath{stroke}%
\end{pgfscope}%
\begin{pgfscope}%
\pgfpathrectangle{\pgfqpoint{0.100000in}{0.212622in}}{\pgfqpoint{3.696000in}{3.696000in}}%
\pgfusepath{clip}%
\pgfsetrectcap%
\pgfsetroundjoin%
\pgfsetlinewidth{1.505625pt}%
\definecolor{currentstroke}{rgb}{1.000000,0.000000,0.000000}%
\pgfsetstrokecolor{currentstroke}%
\pgfsetdash{}{0pt}%
\pgfpathmoveto{\pgfqpoint{2.033824in}{1.933049in}}%
\pgfpathlineto{\pgfqpoint{2.231967in}{2.057140in}}%
\pgfusepath{stroke}%
\end{pgfscope}%
\begin{pgfscope}%
\pgfpathrectangle{\pgfqpoint{0.100000in}{0.212622in}}{\pgfqpoint{3.696000in}{3.696000in}}%
\pgfusepath{clip}%
\pgfsetrectcap%
\pgfsetroundjoin%
\pgfsetlinewidth{1.505625pt}%
\definecolor{currentstroke}{rgb}{1.000000,0.000000,0.000000}%
\pgfsetstrokecolor{currentstroke}%
\pgfsetdash{}{0pt}%
\pgfpathmoveto{\pgfqpoint{2.034928in}{1.930606in}}%
\pgfpathlineto{\pgfqpoint{2.245272in}{2.053304in}}%
\pgfusepath{stroke}%
\end{pgfscope}%
\begin{pgfscope}%
\pgfpathrectangle{\pgfqpoint{0.100000in}{0.212622in}}{\pgfqpoint{3.696000in}{3.696000in}}%
\pgfusepath{clip}%
\pgfsetrectcap%
\pgfsetroundjoin%
\pgfsetlinewidth{1.505625pt}%
\definecolor{currentstroke}{rgb}{1.000000,0.000000,0.000000}%
\pgfsetstrokecolor{currentstroke}%
\pgfsetdash{}{0pt}%
\pgfpathmoveto{\pgfqpoint{2.035517in}{1.930621in}}%
\pgfpathlineto{\pgfqpoint{2.245272in}{2.053304in}}%
\pgfusepath{stroke}%
\end{pgfscope}%
\begin{pgfscope}%
\pgfpathrectangle{\pgfqpoint{0.100000in}{0.212622in}}{\pgfqpoint{3.696000in}{3.696000in}}%
\pgfusepath{clip}%
\pgfsetrectcap%
\pgfsetroundjoin%
\pgfsetlinewidth{1.505625pt}%
\definecolor{currentstroke}{rgb}{1.000000,0.000000,0.000000}%
\pgfsetstrokecolor{currentstroke}%
\pgfsetdash{}{0pt}%
\pgfpathmoveto{\pgfqpoint{2.035876in}{1.928938in}}%
\pgfpathlineto{\pgfqpoint{2.245272in}{2.053304in}}%
\pgfusepath{stroke}%
\end{pgfscope}%
\begin{pgfscope}%
\pgfpathrectangle{\pgfqpoint{0.100000in}{0.212622in}}{\pgfqpoint{3.696000in}{3.696000in}}%
\pgfusepath{clip}%
\pgfsetrectcap%
\pgfsetroundjoin%
\pgfsetlinewidth{1.505625pt}%
\definecolor{currentstroke}{rgb}{1.000000,0.000000,0.000000}%
\pgfsetstrokecolor{currentstroke}%
\pgfsetdash{}{0pt}%
\pgfpathmoveto{\pgfqpoint{2.036493in}{1.928542in}}%
\pgfpathlineto{\pgfqpoint{2.245272in}{2.053304in}}%
\pgfusepath{stroke}%
\end{pgfscope}%
\begin{pgfscope}%
\pgfpathrectangle{\pgfqpoint{0.100000in}{0.212622in}}{\pgfqpoint{3.696000in}{3.696000in}}%
\pgfusepath{clip}%
\pgfsetrectcap%
\pgfsetroundjoin%
\pgfsetlinewidth{1.505625pt}%
\definecolor{currentstroke}{rgb}{1.000000,0.000000,0.000000}%
\pgfsetstrokecolor{currentstroke}%
\pgfsetdash{}{0pt}%
\pgfpathmoveto{\pgfqpoint{2.037402in}{1.926207in}}%
\pgfpathlineto{\pgfqpoint{2.258585in}{2.049466in}}%
\pgfusepath{stroke}%
\end{pgfscope}%
\begin{pgfscope}%
\pgfpathrectangle{\pgfqpoint{0.100000in}{0.212622in}}{\pgfqpoint{3.696000in}{3.696000in}}%
\pgfusepath{clip}%
\pgfsetrectcap%
\pgfsetroundjoin%
\pgfsetlinewidth{1.505625pt}%
\definecolor{currentstroke}{rgb}{1.000000,0.000000,0.000000}%
\pgfsetstrokecolor{currentstroke}%
\pgfsetdash{}{0pt}%
\pgfpathmoveto{\pgfqpoint{2.038903in}{1.924626in}}%
\pgfpathlineto{\pgfqpoint{2.258585in}{2.049466in}}%
\pgfusepath{stroke}%
\end{pgfscope}%
\begin{pgfscope}%
\pgfpathrectangle{\pgfqpoint{0.100000in}{0.212622in}}{\pgfqpoint{3.696000in}{3.696000in}}%
\pgfusepath{clip}%
\pgfsetrectcap%
\pgfsetroundjoin%
\pgfsetlinewidth{1.505625pt}%
\definecolor{currentstroke}{rgb}{1.000000,0.000000,0.000000}%
\pgfsetstrokecolor{currentstroke}%
\pgfsetdash{}{0pt}%
\pgfpathmoveto{\pgfqpoint{2.040740in}{1.919882in}}%
\pgfpathlineto{\pgfqpoint{2.258585in}{2.049466in}}%
\pgfusepath{stroke}%
\end{pgfscope}%
\begin{pgfscope}%
\pgfpathrectangle{\pgfqpoint{0.100000in}{0.212622in}}{\pgfqpoint{3.696000in}{3.696000in}}%
\pgfusepath{clip}%
\pgfsetrectcap%
\pgfsetroundjoin%
\pgfsetlinewidth{1.505625pt}%
\definecolor{currentstroke}{rgb}{1.000000,0.000000,0.000000}%
\pgfsetstrokecolor{currentstroke}%
\pgfsetdash{}{0pt}%
\pgfpathmoveto{\pgfqpoint{2.041325in}{1.919722in}}%
\pgfpathlineto{\pgfqpoint{2.271907in}{2.045625in}}%
\pgfusepath{stroke}%
\end{pgfscope}%
\begin{pgfscope}%
\pgfpathrectangle{\pgfqpoint{0.100000in}{0.212622in}}{\pgfqpoint{3.696000in}{3.696000in}}%
\pgfusepath{clip}%
\pgfsetrectcap%
\pgfsetroundjoin%
\pgfsetlinewidth{1.505625pt}%
\definecolor{currentstroke}{rgb}{1.000000,0.000000,0.000000}%
\pgfsetstrokecolor{currentstroke}%
\pgfsetdash{}{0pt}%
\pgfpathmoveto{\pgfqpoint{2.042365in}{1.917426in}}%
\pgfpathlineto{\pgfqpoint{2.271907in}{2.045625in}}%
\pgfusepath{stroke}%
\end{pgfscope}%
\begin{pgfscope}%
\pgfpathrectangle{\pgfqpoint{0.100000in}{0.212622in}}{\pgfqpoint{3.696000in}{3.696000in}}%
\pgfusepath{clip}%
\pgfsetrectcap%
\pgfsetroundjoin%
\pgfsetlinewidth{1.505625pt}%
\definecolor{currentstroke}{rgb}{1.000000,0.000000,0.000000}%
\pgfsetstrokecolor{currentstroke}%
\pgfsetdash{}{0pt}%
\pgfpathmoveto{\pgfqpoint{2.044323in}{1.916845in}}%
\pgfpathlineto{\pgfqpoint{2.271907in}{2.045625in}}%
\pgfusepath{stroke}%
\end{pgfscope}%
\begin{pgfscope}%
\pgfpathrectangle{\pgfqpoint{0.100000in}{0.212622in}}{\pgfqpoint{3.696000in}{3.696000in}}%
\pgfusepath{clip}%
\pgfsetrectcap%
\pgfsetroundjoin%
\pgfsetlinewidth{1.505625pt}%
\definecolor{currentstroke}{rgb}{1.000000,0.000000,0.000000}%
\pgfsetstrokecolor{currentstroke}%
\pgfsetdash{}{0pt}%
\pgfpathmoveto{\pgfqpoint{2.044834in}{1.917400in}}%
\pgfpathlineto{\pgfqpoint{2.285238in}{2.041781in}}%
\pgfusepath{stroke}%
\end{pgfscope}%
\begin{pgfscope}%
\pgfpathrectangle{\pgfqpoint{0.100000in}{0.212622in}}{\pgfqpoint{3.696000in}{3.696000in}}%
\pgfusepath{clip}%
\pgfsetrectcap%
\pgfsetroundjoin%
\pgfsetlinewidth{1.505625pt}%
\definecolor{currentstroke}{rgb}{1.000000,0.000000,0.000000}%
\pgfsetstrokecolor{currentstroke}%
\pgfsetdash{}{0pt}%
\pgfpathmoveto{\pgfqpoint{2.047003in}{1.912614in}}%
\pgfpathlineto{\pgfqpoint{2.285238in}{2.041781in}}%
\pgfusepath{stroke}%
\end{pgfscope}%
\begin{pgfscope}%
\pgfpathrectangle{\pgfqpoint{0.100000in}{0.212622in}}{\pgfqpoint{3.696000in}{3.696000in}}%
\pgfusepath{clip}%
\pgfsetrectcap%
\pgfsetroundjoin%
\pgfsetlinewidth{1.505625pt}%
\definecolor{currentstroke}{rgb}{1.000000,0.000000,0.000000}%
\pgfsetstrokecolor{currentstroke}%
\pgfsetdash{}{0pt}%
\pgfpathmoveto{\pgfqpoint{2.049123in}{1.908639in}}%
\pgfpathlineto{\pgfqpoint{2.298577in}{2.037935in}}%
\pgfusepath{stroke}%
\end{pgfscope}%
\begin{pgfscope}%
\pgfpathrectangle{\pgfqpoint{0.100000in}{0.212622in}}{\pgfqpoint{3.696000in}{3.696000in}}%
\pgfusepath{clip}%
\pgfsetrectcap%
\pgfsetroundjoin%
\pgfsetlinewidth{1.505625pt}%
\definecolor{currentstroke}{rgb}{1.000000,0.000000,0.000000}%
\pgfsetstrokecolor{currentstroke}%
\pgfsetdash{}{0pt}%
\pgfpathmoveto{\pgfqpoint{2.049857in}{1.903399in}}%
\pgfpathlineto{\pgfqpoint{2.298577in}{2.037935in}}%
\pgfusepath{stroke}%
\end{pgfscope}%
\begin{pgfscope}%
\pgfpathrectangle{\pgfqpoint{0.100000in}{0.212622in}}{\pgfqpoint{3.696000in}{3.696000in}}%
\pgfusepath{clip}%
\pgfsetrectcap%
\pgfsetroundjoin%
\pgfsetlinewidth{1.505625pt}%
\definecolor{currentstroke}{rgb}{1.000000,0.000000,0.000000}%
\pgfsetstrokecolor{currentstroke}%
\pgfsetdash{}{0pt}%
\pgfpathmoveto{\pgfqpoint{2.052617in}{1.897274in}}%
\pgfpathlineto{\pgfqpoint{2.311926in}{2.034086in}}%
\pgfusepath{stroke}%
\end{pgfscope}%
\begin{pgfscope}%
\pgfpathrectangle{\pgfqpoint{0.100000in}{0.212622in}}{\pgfqpoint{3.696000in}{3.696000in}}%
\pgfusepath{clip}%
\pgfsetrectcap%
\pgfsetroundjoin%
\pgfsetlinewidth{1.505625pt}%
\definecolor{currentstroke}{rgb}{1.000000,0.000000,0.000000}%
\pgfsetstrokecolor{currentstroke}%
\pgfsetdash{}{0pt}%
\pgfpathmoveto{\pgfqpoint{2.053904in}{1.895412in}}%
\pgfpathlineto{\pgfqpoint{2.325283in}{2.030235in}}%
\pgfusepath{stroke}%
\end{pgfscope}%
\begin{pgfscope}%
\pgfpathrectangle{\pgfqpoint{0.100000in}{0.212622in}}{\pgfqpoint{3.696000in}{3.696000in}}%
\pgfusepath{clip}%
\pgfsetrectcap%
\pgfsetroundjoin%
\pgfsetlinewidth{1.505625pt}%
\definecolor{currentstroke}{rgb}{1.000000,0.000000,0.000000}%
\pgfsetstrokecolor{currentstroke}%
\pgfsetdash{}{0pt}%
\pgfpathmoveto{\pgfqpoint{2.056221in}{1.894303in}}%
\pgfpathlineto{\pgfqpoint{2.325283in}{2.030235in}}%
\pgfusepath{stroke}%
\end{pgfscope}%
\begin{pgfscope}%
\pgfpathrectangle{\pgfqpoint{0.100000in}{0.212622in}}{\pgfqpoint{3.696000in}{3.696000in}}%
\pgfusepath{clip}%
\pgfsetrectcap%
\pgfsetroundjoin%
\pgfsetlinewidth{1.505625pt}%
\definecolor{currentstroke}{rgb}{1.000000,0.000000,0.000000}%
\pgfsetstrokecolor{currentstroke}%
\pgfsetdash{}{0pt}%
\pgfpathmoveto{\pgfqpoint{2.058053in}{1.891877in}}%
\pgfpathlineto{\pgfqpoint{2.338649in}{2.026381in}}%
\pgfusepath{stroke}%
\end{pgfscope}%
\begin{pgfscope}%
\pgfpathrectangle{\pgfqpoint{0.100000in}{0.212622in}}{\pgfqpoint{3.696000in}{3.696000in}}%
\pgfusepath{clip}%
\pgfsetrectcap%
\pgfsetroundjoin%
\pgfsetlinewidth{1.505625pt}%
\definecolor{currentstroke}{rgb}{1.000000,0.000000,0.000000}%
\pgfsetstrokecolor{currentstroke}%
\pgfsetdash{}{0pt}%
\pgfpathmoveto{\pgfqpoint{2.059874in}{1.890347in}}%
\pgfpathlineto{\pgfqpoint{2.338649in}{2.026381in}}%
\pgfusepath{stroke}%
\end{pgfscope}%
\begin{pgfscope}%
\pgfpathrectangle{\pgfqpoint{0.100000in}{0.212622in}}{\pgfqpoint{3.696000in}{3.696000in}}%
\pgfusepath{clip}%
\pgfsetrectcap%
\pgfsetroundjoin%
\pgfsetlinewidth{1.505625pt}%
\definecolor{currentstroke}{rgb}{1.000000,0.000000,0.000000}%
\pgfsetstrokecolor{currentstroke}%
\pgfsetdash{}{0pt}%
\pgfpathmoveto{\pgfqpoint{2.064514in}{1.890493in}}%
\pgfpathlineto{\pgfqpoint{2.352024in}{2.022525in}}%
\pgfusepath{stroke}%
\end{pgfscope}%
\begin{pgfscope}%
\pgfpathrectangle{\pgfqpoint{0.100000in}{0.212622in}}{\pgfqpoint{3.696000in}{3.696000in}}%
\pgfusepath{clip}%
\pgfsetrectcap%
\pgfsetroundjoin%
\pgfsetlinewidth{1.505625pt}%
\definecolor{currentstroke}{rgb}{1.000000,0.000000,0.000000}%
\pgfsetstrokecolor{currentstroke}%
\pgfsetdash{}{0pt}%
\pgfpathmoveto{\pgfqpoint{2.065242in}{1.884998in}}%
\pgfpathlineto{\pgfqpoint{2.365408in}{2.018666in}}%
\pgfusepath{stroke}%
\end{pgfscope}%
\begin{pgfscope}%
\pgfpathrectangle{\pgfqpoint{0.100000in}{0.212622in}}{\pgfqpoint{3.696000in}{3.696000in}}%
\pgfusepath{clip}%
\pgfsetrectcap%
\pgfsetroundjoin%
\pgfsetlinewidth{1.505625pt}%
\definecolor{currentstroke}{rgb}{1.000000,0.000000,0.000000}%
\pgfsetstrokecolor{currentstroke}%
\pgfsetdash{}{0pt}%
\pgfpathmoveto{\pgfqpoint{2.067658in}{1.877660in}}%
\pgfpathlineto{\pgfqpoint{2.378801in}{2.014805in}}%
\pgfusepath{stroke}%
\end{pgfscope}%
\begin{pgfscope}%
\pgfpathrectangle{\pgfqpoint{0.100000in}{0.212622in}}{\pgfqpoint{3.696000in}{3.696000in}}%
\pgfusepath{clip}%
\pgfsetrectcap%
\pgfsetroundjoin%
\pgfsetlinewidth{1.505625pt}%
\definecolor{currentstroke}{rgb}{1.000000,0.000000,0.000000}%
\pgfsetstrokecolor{currentstroke}%
\pgfsetdash{}{0pt}%
\pgfpathmoveto{\pgfqpoint{2.070187in}{1.877087in}}%
\pgfpathlineto{\pgfqpoint{2.378801in}{2.014805in}}%
\pgfusepath{stroke}%
\end{pgfscope}%
\begin{pgfscope}%
\pgfpathrectangle{\pgfqpoint{0.100000in}{0.212622in}}{\pgfqpoint{3.696000in}{3.696000in}}%
\pgfusepath{clip}%
\pgfsetrectcap%
\pgfsetroundjoin%
\pgfsetlinewidth{1.505625pt}%
\definecolor{currentstroke}{rgb}{1.000000,0.000000,0.000000}%
\pgfsetstrokecolor{currentstroke}%
\pgfsetdash{}{0pt}%
\pgfpathmoveto{\pgfqpoint{2.072852in}{1.874989in}}%
\pgfpathlineto{\pgfqpoint{2.392203in}{2.010941in}}%
\pgfusepath{stroke}%
\end{pgfscope}%
\begin{pgfscope}%
\pgfpathrectangle{\pgfqpoint{0.100000in}{0.212622in}}{\pgfqpoint{3.696000in}{3.696000in}}%
\pgfusepath{clip}%
\pgfsetrectcap%
\pgfsetroundjoin%
\pgfsetlinewidth{1.505625pt}%
\definecolor{currentstroke}{rgb}{1.000000,0.000000,0.000000}%
\pgfsetstrokecolor{currentstroke}%
\pgfsetdash{}{0pt}%
\pgfpathmoveto{\pgfqpoint{2.073935in}{1.870774in}}%
\pgfpathlineto{\pgfqpoint{2.392203in}{2.010941in}}%
\pgfusepath{stroke}%
\end{pgfscope}%
\begin{pgfscope}%
\pgfpathrectangle{\pgfqpoint{0.100000in}{0.212622in}}{\pgfqpoint{3.696000in}{3.696000in}}%
\pgfusepath{clip}%
\pgfsetrectcap%
\pgfsetroundjoin%
\pgfsetlinewidth{1.505625pt}%
\definecolor{currentstroke}{rgb}{1.000000,0.000000,0.000000}%
\pgfsetstrokecolor{currentstroke}%
\pgfsetdash{}{0pt}%
\pgfpathmoveto{\pgfqpoint{2.076355in}{1.863492in}}%
\pgfpathlineto{\pgfqpoint{2.405613in}{2.007074in}}%
\pgfusepath{stroke}%
\end{pgfscope}%
\begin{pgfscope}%
\pgfpathrectangle{\pgfqpoint{0.100000in}{0.212622in}}{\pgfqpoint{3.696000in}{3.696000in}}%
\pgfusepath{clip}%
\pgfsetrectcap%
\pgfsetroundjoin%
\pgfsetlinewidth{1.505625pt}%
\definecolor{currentstroke}{rgb}{1.000000,0.000000,0.000000}%
\pgfsetstrokecolor{currentstroke}%
\pgfsetdash{}{0pt}%
\pgfpathmoveto{\pgfqpoint{2.079652in}{1.861686in}}%
\pgfpathlineto{\pgfqpoint{2.405613in}{2.007074in}}%
\pgfusepath{stroke}%
\end{pgfscope}%
\begin{pgfscope}%
\pgfpathrectangle{\pgfqpoint{0.100000in}{0.212622in}}{\pgfqpoint{3.696000in}{3.696000in}}%
\pgfusepath{clip}%
\pgfsetrectcap%
\pgfsetroundjoin%
\pgfsetlinewidth{1.505625pt}%
\definecolor{currentstroke}{rgb}{1.000000,0.000000,0.000000}%
\pgfsetstrokecolor{currentstroke}%
\pgfsetdash{}{0pt}%
\pgfpathmoveto{\pgfqpoint{2.083238in}{1.858277in}}%
\pgfpathlineto{\pgfqpoint{2.419033in}{2.003205in}}%
\pgfusepath{stroke}%
\end{pgfscope}%
\begin{pgfscope}%
\pgfpathrectangle{\pgfqpoint{0.100000in}{0.212622in}}{\pgfqpoint{3.696000in}{3.696000in}}%
\pgfusepath{clip}%
\pgfsetrectcap%
\pgfsetroundjoin%
\pgfsetlinewidth{1.505625pt}%
\definecolor{currentstroke}{rgb}{1.000000,0.000000,0.000000}%
\pgfsetstrokecolor{currentstroke}%
\pgfsetdash{}{0pt}%
\pgfpathmoveto{\pgfqpoint{2.084304in}{1.854915in}}%
\pgfpathlineto{\pgfqpoint{2.432461in}{1.999333in}}%
\pgfusepath{stroke}%
\end{pgfscope}%
\begin{pgfscope}%
\pgfpathrectangle{\pgfqpoint{0.100000in}{0.212622in}}{\pgfqpoint{3.696000in}{3.696000in}}%
\pgfusepath{clip}%
\pgfsetrectcap%
\pgfsetroundjoin%
\pgfsetlinewidth{1.505625pt}%
\definecolor{currentstroke}{rgb}{1.000000,0.000000,0.000000}%
\pgfsetstrokecolor{currentstroke}%
\pgfsetdash{}{0pt}%
\pgfpathmoveto{\pgfqpoint{2.088749in}{1.850172in}}%
\pgfpathlineto{\pgfqpoint{2.445899in}{1.995459in}}%
\pgfusepath{stroke}%
\end{pgfscope}%
\begin{pgfscope}%
\pgfpathrectangle{\pgfqpoint{0.100000in}{0.212622in}}{\pgfqpoint{3.696000in}{3.696000in}}%
\pgfusepath{clip}%
\pgfsetrectcap%
\pgfsetroundjoin%
\pgfsetlinewidth{1.505625pt}%
\definecolor{currentstroke}{rgb}{1.000000,0.000000,0.000000}%
\pgfsetstrokecolor{currentstroke}%
\pgfsetdash{}{0pt}%
\pgfpathmoveto{\pgfqpoint{2.090641in}{1.847124in}}%
\pgfpathlineto{\pgfqpoint{2.445899in}{1.995459in}}%
\pgfusepath{stroke}%
\end{pgfscope}%
\begin{pgfscope}%
\pgfpathrectangle{\pgfqpoint{0.100000in}{0.212622in}}{\pgfqpoint{3.696000in}{3.696000in}}%
\pgfusepath{clip}%
\pgfsetrectcap%
\pgfsetroundjoin%
\pgfsetlinewidth{1.505625pt}%
\definecolor{currentstroke}{rgb}{1.000000,0.000000,0.000000}%
\pgfsetstrokecolor{currentstroke}%
\pgfsetdash{}{0pt}%
\pgfpathmoveto{\pgfqpoint{2.093075in}{1.844885in}}%
\pgfpathlineto{\pgfqpoint{2.459345in}{1.991582in}}%
\pgfusepath{stroke}%
\end{pgfscope}%
\begin{pgfscope}%
\pgfpathrectangle{\pgfqpoint{0.100000in}{0.212622in}}{\pgfqpoint{3.696000in}{3.696000in}}%
\pgfusepath{clip}%
\pgfsetrectcap%
\pgfsetroundjoin%
\pgfsetlinewidth{1.505625pt}%
\definecolor{currentstroke}{rgb}{1.000000,0.000000,0.000000}%
\pgfsetstrokecolor{currentstroke}%
\pgfsetdash{}{0pt}%
\pgfpathmoveto{\pgfqpoint{2.094220in}{1.842951in}}%
\pgfpathlineto{\pgfqpoint{2.459345in}{1.991582in}}%
\pgfusepath{stroke}%
\end{pgfscope}%
\begin{pgfscope}%
\pgfpathrectangle{\pgfqpoint{0.100000in}{0.212622in}}{\pgfqpoint{3.696000in}{3.696000in}}%
\pgfusepath{clip}%
\pgfsetrectcap%
\pgfsetroundjoin%
\pgfsetlinewidth{1.505625pt}%
\definecolor{currentstroke}{rgb}{1.000000,0.000000,0.000000}%
\pgfsetstrokecolor{currentstroke}%
\pgfsetdash{}{0pt}%
\pgfpathmoveto{\pgfqpoint{2.094828in}{1.841532in}}%
\pgfpathlineto{\pgfqpoint{2.459345in}{1.991582in}}%
\pgfusepath{stroke}%
\end{pgfscope}%
\begin{pgfscope}%
\pgfpathrectangle{\pgfqpoint{0.100000in}{0.212622in}}{\pgfqpoint{3.696000in}{3.696000in}}%
\pgfusepath{clip}%
\pgfsetrectcap%
\pgfsetroundjoin%
\pgfsetlinewidth{1.505625pt}%
\definecolor{currentstroke}{rgb}{1.000000,0.000000,0.000000}%
\pgfsetstrokecolor{currentstroke}%
\pgfsetdash{}{0pt}%
\pgfpathmoveto{\pgfqpoint{2.096143in}{1.840183in}}%
\pgfpathlineto{\pgfqpoint{2.459345in}{1.991582in}}%
\pgfusepath{stroke}%
\end{pgfscope}%
\begin{pgfscope}%
\pgfpathrectangle{\pgfqpoint{0.100000in}{0.212622in}}{\pgfqpoint{3.696000in}{3.696000in}}%
\pgfusepath{clip}%
\pgfsetrectcap%
\pgfsetroundjoin%
\pgfsetlinewidth{1.505625pt}%
\definecolor{currentstroke}{rgb}{1.000000,0.000000,0.000000}%
\pgfsetstrokecolor{currentstroke}%
\pgfsetdash{}{0pt}%
\pgfpathmoveto{\pgfqpoint{2.096511in}{1.838434in}}%
\pgfpathlineto{\pgfqpoint{2.472800in}{1.987702in}}%
\pgfusepath{stroke}%
\end{pgfscope}%
\begin{pgfscope}%
\pgfpathrectangle{\pgfqpoint{0.100000in}{0.212622in}}{\pgfqpoint{3.696000in}{3.696000in}}%
\pgfusepath{clip}%
\pgfsetrectcap%
\pgfsetroundjoin%
\pgfsetlinewidth{1.505625pt}%
\definecolor{currentstroke}{rgb}{1.000000,0.000000,0.000000}%
\pgfsetstrokecolor{currentstroke}%
\pgfsetdash{}{0pt}%
\pgfpathmoveto{\pgfqpoint{2.098445in}{1.835967in}}%
\pgfpathlineto{\pgfqpoint{2.472800in}{1.987702in}}%
\pgfusepath{stroke}%
\end{pgfscope}%
\begin{pgfscope}%
\pgfpathrectangle{\pgfqpoint{0.100000in}{0.212622in}}{\pgfqpoint{3.696000in}{3.696000in}}%
\pgfusepath{clip}%
\pgfsetrectcap%
\pgfsetroundjoin%
\pgfsetlinewidth{1.505625pt}%
\definecolor{currentstroke}{rgb}{1.000000,0.000000,0.000000}%
\pgfsetstrokecolor{currentstroke}%
\pgfsetdash{}{0pt}%
\pgfpathmoveto{\pgfqpoint{2.100254in}{1.834052in}}%
\pgfpathlineto{\pgfqpoint{2.738021in}{1.454061in}}%
\pgfusepath{stroke}%
\end{pgfscope}%
\begin{pgfscope}%
\pgfpathrectangle{\pgfqpoint{0.100000in}{0.212622in}}{\pgfqpoint{3.696000in}{3.696000in}}%
\pgfusepath{clip}%
\pgfsetrectcap%
\pgfsetroundjoin%
\pgfsetlinewidth{1.505625pt}%
\definecolor{currentstroke}{rgb}{1.000000,0.000000,0.000000}%
\pgfsetstrokecolor{currentstroke}%
\pgfsetdash{}{0pt}%
\pgfpathmoveto{\pgfqpoint{2.102170in}{1.829902in}}%
\pgfpathlineto{\pgfqpoint{2.729696in}{1.445992in}}%
\pgfusepath{stroke}%
\end{pgfscope}%
\begin{pgfscope}%
\pgfpathrectangle{\pgfqpoint{0.100000in}{0.212622in}}{\pgfqpoint{3.696000in}{3.696000in}}%
\pgfusepath{clip}%
\pgfsetrectcap%
\pgfsetroundjoin%
\pgfsetlinewidth{1.505625pt}%
\definecolor{currentstroke}{rgb}{1.000000,0.000000,0.000000}%
\pgfsetstrokecolor{currentstroke}%
\pgfsetdash{}{0pt}%
\pgfpathmoveto{\pgfqpoint{2.104591in}{1.825286in}}%
\pgfpathlineto{\pgfqpoint{2.721360in}{1.437913in}}%
\pgfusepath{stroke}%
\end{pgfscope}%
\begin{pgfscope}%
\pgfpathrectangle{\pgfqpoint{0.100000in}{0.212622in}}{\pgfqpoint{3.696000in}{3.696000in}}%
\pgfusepath{clip}%
\pgfsetrectcap%
\pgfsetroundjoin%
\pgfsetlinewidth{1.505625pt}%
\definecolor{currentstroke}{rgb}{1.000000,0.000000,0.000000}%
\pgfsetstrokecolor{currentstroke}%
\pgfsetdash{}{0pt}%
\pgfpathmoveto{\pgfqpoint{2.107115in}{1.819035in}}%
\pgfpathlineto{\pgfqpoint{2.721360in}{1.437913in}}%
\pgfusepath{stroke}%
\end{pgfscope}%
\begin{pgfscope}%
\pgfpathrectangle{\pgfqpoint{0.100000in}{0.212622in}}{\pgfqpoint{3.696000in}{3.696000in}}%
\pgfusepath{clip}%
\pgfsetrectcap%
\pgfsetroundjoin%
\pgfsetlinewidth{1.505625pt}%
\definecolor{currentstroke}{rgb}{1.000000,0.000000,0.000000}%
\pgfsetstrokecolor{currentstroke}%
\pgfsetdash{}{0pt}%
\pgfpathmoveto{\pgfqpoint{2.112626in}{1.818410in}}%
\pgfpathlineto{\pgfqpoint{2.713013in}{1.429823in}}%
\pgfusepath{stroke}%
\end{pgfscope}%
\begin{pgfscope}%
\pgfpathrectangle{\pgfqpoint{0.100000in}{0.212622in}}{\pgfqpoint{3.696000in}{3.696000in}}%
\pgfusepath{clip}%
\pgfsetrectcap%
\pgfsetroundjoin%
\pgfsetlinewidth{1.505625pt}%
\definecolor{currentstroke}{rgb}{1.000000,0.000000,0.000000}%
\pgfsetstrokecolor{currentstroke}%
\pgfsetdash{}{0pt}%
\pgfpathmoveto{\pgfqpoint{2.113001in}{1.815817in}}%
\pgfpathlineto{\pgfqpoint{2.704654in}{1.421723in}}%
\pgfusepath{stroke}%
\end{pgfscope}%
\begin{pgfscope}%
\pgfpathrectangle{\pgfqpoint{0.100000in}{0.212622in}}{\pgfqpoint{3.696000in}{3.696000in}}%
\pgfusepath{clip}%
\pgfsetrectcap%
\pgfsetroundjoin%
\pgfsetlinewidth{1.505625pt}%
\definecolor{currentstroke}{rgb}{1.000000,0.000000,0.000000}%
\pgfsetstrokecolor{currentstroke}%
\pgfsetdash{}{0pt}%
\pgfpathmoveto{\pgfqpoint{2.115098in}{1.812260in}}%
\pgfpathlineto{\pgfqpoint{2.704654in}{1.421723in}}%
\pgfusepath{stroke}%
\end{pgfscope}%
\begin{pgfscope}%
\pgfpathrectangle{\pgfqpoint{0.100000in}{0.212622in}}{\pgfqpoint{3.696000in}{3.696000in}}%
\pgfusepath{clip}%
\pgfsetrectcap%
\pgfsetroundjoin%
\pgfsetlinewidth{1.505625pt}%
\definecolor{currentstroke}{rgb}{1.000000,0.000000,0.000000}%
\pgfsetstrokecolor{currentstroke}%
\pgfsetdash{}{0pt}%
\pgfpathmoveto{\pgfqpoint{2.117839in}{1.808681in}}%
\pgfpathlineto{\pgfqpoint{2.696285in}{1.413612in}}%
\pgfusepath{stroke}%
\end{pgfscope}%
\begin{pgfscope}%
\pgfpathrectangle{\pgfqpoint{0.100000in}{0.212622in}}{\pgfqpoint{3.696000in}{3.696000in}}%
\pgfusepath{clip}%
\pgfsetrectcap%
\pgfsetroundjoin%
\pgfsetlinewidth{1.505625pt}%
\definecolor{currentstroke}{rgb}{1.000000,0.000000,0.000000}%
\pgfsetstrokecolor{currentstroke}%
\pgfsetdash{}{0pt}%
\pgfpathmoveto{\pgfqpoint{2.119775in}{1.805800in}}%
\pgfpathlineto{\pgfqpoint{2.696285in}{1.413612in}}%
\pgfusepath{stroke}%
\end{pgfscope}%
\begin{pgfscope}%
\pgfpathrectangle{\pgfqpoint{0.100000in}{0.212622in}}{\pgfqpoint{3.696000in}{3.696000in}}%
\pgfusepath{clip}%
\pgfsetrectcap%
\pgfsetroundjoin%
\pgfsetlinewidth{1.505625pt}%
\definecolor{currentstroke}{rgb}{1.000000,0.000000,0.000000}%
\pgfsetstrokecolor{currentstroke}%
\pgfsetdash{}{0pt}%
\pgfpathmoveto{\pgfqpoint{2.120884in}{1.802594in}}%
\pgfpathlineto{\pgfqpoint{2.687905in}{1.405490in}}%
\pgfusepath{stroke}%
\end{pgfscope}%
\begin{pgfscope}%
\pgfpathrectangle{\pgfqpoint{0.100000in}{0.212622in}}{\pgfqpoint{3.696000in}{3.696000in}}%
\pgfusepath{clip}%
\pgfsetrectcap%
\pgfsetroundjoin%
\pgfsetlinewidth{1.505625pt}%
\definecolor{currentstroke}{rgb}{1.000000,0.000000,0.000000}%
\pgfsetstrokecolor{currentstroke}%
\pgfsetdash{}{0pt}%
\pgfpathmoveto{\pgfqpoint{2.124490in}{1.800992in}}%
\pgfpathlineto{\pgfqpoint{2.687905in}{1.405490in}}%
\pgfusepath{stroke}%
\end{pgfscope}%
\begin{pgfscope}%
\pgfpathrectangle{\pgfqpoint{0.100000in}{0.212622in}}{\pgfqpoint{3.696000in}{3.696000in}}%
\pgfusepath{clip}%
\pgfsetrectcap%
\pgfsetroundjoin%
\pgfsetlinewidth{1.505625pt}%
\definecolor{currentstroke}{rgb}{1.000000,0.000000,0.000000}%
\pgfsetstrokecolor{currentstroke}%
\pgfsetdash{}{0pt}%
\pgfpathmoveto{\pgfqpoint{2.127846in}{1.797948in}}%
\pgfpathlineto{\pgfqpoint{2.679513in}{1.397357in}}%
\pgfusepath{stroke}%
\end{pgfscope}%
\begin{pgfscope}%
\pgfpathrectangle{\pgfqpoint{0.100000in}{0.212622in}}{\pgfqpoint{3.696000in}{3.696000in}}%
\pgfusepath{clip}%
\pgfsetrectcap%
\pgfsetroundjoin%
\pgfsetlinewidth{1.505625pt}%
\definecolor{currentstroke}{rgb}{1.000000,0.000000,0.000000}%
\pgfsetstrokecolor{currentstroke}%
\pgfsetdash{}{0pt}%
\pgfpathmoveto{\pgfqpoint{2.128955in}{1.794745in}}%
\pgfpathlineto{\pgfqpoint{2.671110in}{1.389213in}}%
\pgfusepath{stroke}%
\end{pgfscope}%
\begin{pgfscope}%
\pgfpathrectangle{\pgfqpoint{0.100000in}{0.212622in}}{\pgfqpoint{3.696000in}{3.696000in}}%
\pgfusepath{clip}%
\pgfsetrectcap%
\pgfsetroundjoin%
\pgfsetlinewidth{1.505625pt}%
\definecolor{currentstroke}{rgb}{1.000000,0.000000,0.000000}%
\pgfsetstrokecolor{currentstroke}%
\pgfsetdash{}{0pt}%
\pgfpathmoveto{\pgfqpoint{2.129900in}{1.792875in}}%
\pgfpathlineto{\pgfqpoint{2.671110in}{1.389213in}}%
\pgfusepath{stroke}%
\end{pgfscope}%
\begin{pgfscope}%
\pgfpathrectangle{\pgfqpoint{0.100000in}{0.212622in}}{\pgfqpoint{3.696000in}{3.696000in}}%
\pgfusepath{clip}%
\pgfsetrectcap%
\pgfsetroundjoin%
\pgfsetlinewidth{1.505625pt}%
\definecolor{currentstroke}{rgb}{1.000000,0.000000,0.000000}%
\pgfsetstrokecolor{currentstroke}%
\pgfsetdash{}{0pt}%
\pgfpathmoveto{\pgfqpoint{2.130922in}{1.791530in}}%
\pgfpathlineto{\pgfqpoint{2.671110in}{1.389213in}}%
\pgfusepath{stroke}%
\end{pgfscope}%
\begin{pgfscope}%
\pgfpathrectangle{\pgfqpoint{0.100000in}{0.212622in}}{\pgfqpoint{3.696000in}{3.696000in}}%
\pgfusepath{clip}%
\pgfsetrectcap%
\pgfsetroundjoin%
\pgfsetlinewidth{1.505625pt}%
\definecolor{currentstroke}{rgb}{1.000000,0.000000,0.000000}%
\pgfsetstrokecolor{currentstroke}%
\pgfsetdash{}{0pt}%
\pgfpathmoveto{\pgfqpoint{2.132402in}{1.790703in}}%
\pgfpathlineto{\pgfqpoint{2.671110in}{1.389213in}}%
\pgfusepath{stroke}%
\end{pgfscope}%
\begin{pgfscope}%
\pgfpathrectangle{\pgfqpoint{0.100000in}{0.212622in}}{\pgfqpoint{3.696000in}{3.696000in}}%
\pgfusepath{clip}%
\pgfsetrectcap%
\pgfsetroundjoin%
\pgfsetlinewidth{1.505625pt}%
\definecolor{currentstroke}{rgb}{1.000000,0.000000,0.000000}%
\pgfsetstrokecolor{currentstroke}%
\pgfsetdash{}{0pt}%
\pgfpathmoveto{\pgfqpoint{2.133065in}{1.789332in}}%
\pgfpathlineto{\pgfqpoint{2.662697in}{1.381059in}}%
\pgfusepath{stroke}%
\end{pgfscope}%
\begin{pgfscope}%
\pgfpathrectangle{\pgfqpoint{0.100000in}{0.212622in}}{\pgfqpoint{3.696000in}{3.696000in}}%
\pgfusepath{clip}%
\pgfsetrectcap%
\pgfsetroundjoin%
\pgfsetlinewidth{1.505625pt}%
\definecolor{currentstroke}{rgb}{1.000000,0.000000,0.000000}%
\pgfsetstrokecolor{currentstroke}%
\pgfsetdash{}{0pt}%
\pgfpathmoveto{\pgfqpoint{2.133907in}{1.787373in}}%
\pgfpathlineto{\pgfqpoint{2.662697in}{1.381059in}}%
\pgfusepath{stroke}%
\end{pgfscope}%
\begin{pgfscope}%
\pgfpathrectangle{\pgfqpoint{0.100000in}{0.212622in}}{\pgfqpoint{3.696000in}{3.696000in}}%
\pgfusepath{clip}%
\pgfsetrectcap%
\pgfsetroundjoin%
\pgfsetlinewidth{1.505625pt}%
\definecolor{currentstroke}{rgb}{1.000000,0.000000,0.000000}%
\pgfsetstrokecolor{currentstroke}%
\pgfsetdash{}{0pt}%
\pgfpathmoveto{\pgfqpoint{2.135227in}{1.785957in}}%
\pgfpathlineto{\pgfqpoint{2.662697in}{1.381059in}}%
\pgfusepath{stroke}%
\end{pgfscope}%
\begin{pgfscope}%
\pgfpathrectangle{\pgfqpoint{0.100000in}{0.212622in}}{\pgfqpoint{3.696000in}{3.696000in}}%
\pgfusepath{clip}%
\pgfsetrectcap%
\pgfsetroundjoin%
\pgfsetlinewidth{1.505625pt}%
\definecolor{currentstroke}{rgb}{1.000000,0.000000,0.000000}%
\pgfsetstrokecolor{currentstroke}%
\pgfsetdash{}{0pt}%
\pgfpathmoveto{\pgfqpoint{2.137615in}{1.786142in}}%
\pgfpathlineto{\pgfqpoint{2.662697in}{1.381059in}}%
\pgfusepath{stroke}%
\end{pgfscope}%
\begin{pgfscope}%
\pgfpathrectangle{\pgfqpoint{0.100000in}{0.212622in}}{\pgfqpoint{3.696000in}{3.696000in}}%
\pgfusepath{clip}%
\pgfsetrectcap%
\pgfsetroundjoin%
\pgfsetlinewidth{1.505625pt}%
\definecolor{currentstroke}{rgb}{1.000000,0.000000,0.000000}%
\pgfsetstrokecolor{currentstroke}%
\pgfsetdash{}{0pt}%
\pgfpathmoveto{\pgfqpoint{2.138653in}{1.783001in}}%
\pgfpathlineto{\pgfqpoint{2.654272in}{1.372893in}}%
\pgfusepath{stroke}%
\end{pgfscope}%
\begin{pgfscope}%
\pgfpathrectangle{\pgfqpoint{0.100000in}{0.212622in}}{\pgfqpoint{3.696000in}{3.696000in}}%
\pgfusepath{clip}%
\pgfsetrectcap%
\pgfsetroundjoin%
\pgfsetlinewidth{1.505625pt}%
\definecolor{currentstroke}{rgb}{1.000000,0.000000,0.000000}%
\pgfsetstrokecolor{currentstroke}%
\pgfsetdash{}{0pt}%
\pgfpathmoveto{\pgfqpoint{2.141081in}{1.777145in}}%
\pgfpathlineto{\pgfqpoint{2.654272in}{1.372893in}}%
\pgfusepath{stroke}%
\end{pgfscope}%
\begin{pgfscope}%
\pgfpathrectangle{\pgfqpoint{0.100000in}{0.212622in}}{\pgfqpoint{3.696000in}{3.696000in}}%
\pgfusepath{clip}%
\pgfsetrectcap%
\pgfsetroundjoin%
\pgfsetlinewidth{1.505625pt}%
\definecolor{currentstroke}{rgb}{1.000000,0.000000,0.000000}%
\pgfsetstrokecolor{currentstroke}%
\pgfsetdash{}{0pt}%
\pgfpathmoveto{\pgfqpoint{2.143094in}{1.774973in}}%
\pgfpathlineto{\pgfqpoint{2.645835in}{1.364717in}}%
\pgfusepath{stroke}%
\end{pgfscope}%
\begin{pgfscope}%
\pgfpathrectangle{\pgfqpoint{0.100000in}{0.212622in}}{\pgfqpoint{3.696000in}{3.696000in}}%
\pgfusepath{clip}%
\pgfsetrectcap%
\pgfsetroundjoin%
\pgfsetlinewidth{1.505625pt}%
\definecolor{currentstroke}{rgb}{1.000000,0.000000,0.000000}%
\pgfsetstrokecolor{currentstroke}%
\pgfsetdash{}{0pt}%
\pgfpathmoveto{\pgfqpoint{2.145438in}{1.773477in}}%
\pgfpathlineto{\pgfqpoint{2.637388in}{1.356530in}}%
\pgfusepath{stroke}%
\end{pgfscope}%
\begin{pgfscope}%
\pgfpathrectangle{\pgfqpoint{0.100000in}{0.212622in}}{\pgfqpoint{3.696000in}{3.696000in}}%
\pgfusepath{clip}%
\pgfsetrectcap%
\pgfsetroundjoin%
\pgfsetlinewidth{1.505625pt}%
\definecolor{currentstroke}{rgb}{1.000000,0.000000,0.000000}%
\pgfsetstrokecolor{currentstroke}%
\pgfsetdash{}{0pt}%
\pgfpathmoveto{\pgfqpoint{2.147696in}{1.768802in}}%
\pgfpathlineto{\pgfqpoint{2.628929in}{1.348332in}}%
\pgfusepath{stroke}%
\end{pgfscope}%
\begin{pgfscope}%
\pgfpathrectangle{\pgfqpoint{0.100000in}{0.212622in}}{\pgfqpoint{3.696000in}{3.696000in}}%
\pgfusepath{clip}%
\pgfsetrectcap%
\pgfsetroundjoin%
\pgfsetlinewidth{1.505625pt}%
\definecolor{currentstroke}{rgb}{1.000000,0.000000,0.000000}%
\pgfsetstrokecolor{currentstroke}%
\pgfsetdash{}{0pt}%
\pgfpathmoveto{\pgfqpoint{2.151038in}{1.761711in}}%
\pgfpathlineto{\pgfqpoint{2.628929in}{1.348332in}}%
\pgfusepath{stroke}%
\end{pgfscope}%
\begin{pgfscope}%
\pgfpathrectangle{\pgfqpoint{0.100000in}{0.212622in}}{\pgfqpoint{3.696000in}{3.696000in}}%
\pgfusepath{clip}%
\pgfsetrectcap%
\pgfsetroundjoin%
\pgfsetlinewidth{1.505625pt}%
\definecolor{currentstroke}{rgb}{1.000000,0.000000,0.000000}%
\pgfsetstrokecolor{currentstroke}%
\pgfsetdash{}{0pt}%
\pgfpathmoveto{\pgfqpoint{2.157478in}{1.759494in}}%
\pgfpathlineto{\pgfqpoint{2.620459in}{1.340123in}}%
\pgfusepath{stroke}%
\end{pgfscope}%
\begin{pgfscope}%
\pgfpathrectangle{\pgfqpoint{0.100000in}{0.212622in}}{\pgfqpoint{3.696000in}{3.696000in}}%
\pgfusepath{clip}%
\pgfsetrectcap%
\pgfsetroundjoin%
\pgfsetlinewidth{1.505625pt}%
\definecolor{currentstroke}{rgb}{1.000000,0.000000,0.000000}%
\pgfsetstrokecolor{currentstroke}%
\pgfsetdash{}{0pt}%
\pgfpathmoveto{\pgfqpoint{2.160594in}{1.752405in}}%
\pgfpathlineto{\pgfqpoint{2.611977in}{1.331903in}}%
\pgfusepath{stroke}%
\end{pgfscope}%
\begin{pgfscope}%
\pgfpathrectangle{\pgfqpoint{0.100000in}{0.212622in}}{\pgfqpoint{3.696000in}{3.696000in}}%
\pgfusepath{clip}%
\pgfsetrectcap%
\pgfsetroundjoin%
\pgfsetlinewidth{1.505625pt}%
\definecolor{currentstroke}{rgb}{1.000000,0.000000,0.000000}%
\pgfsetstrokecolor{currentstroke}%
\pgfsetdash{}{0pt}%
\pgfpathmoveto{\pgfqpoint{2.163139in}{1.740430in}}%
\pgfpathlineto{\pgfqpoint{2.603485in}{1.323672in}}%
\pgfusepath{stroke}%
\end{pgfscope}%
\begin{pgfscope}%
\pgfpathrectangle{\pgfqpoint{0.100000in}{0.212622in}}{\pgfqpoint{3.696000in}{3.696000in}}%
\pgfusepath{clip}%
\pgfsetrectcap%
\pgfsetroundjoin%
\pgfsetlinewidth{1.505625pt}%
\definecolor{currentstroke}{rgb}{1.000000,0.000000,0.000000}%
\pgfsetstrokecolor{currentstroke}%
\pgfsetdash{}{0pt}%
\pgfpathmoveto{\pgfqpoint{2.167626in}{1.728008in}}%
\pgfpathlineto{\pgfqpoint{2.586465in}{1.307177in}}%
\pgfusepath{stroke}%
\end{pgfscope}%
\begin{pgfscope}%
\pgfpathrectangle{\pgfqpoint{0.100000in}{0.212622in}}{\pgfqpoint{3.696000in}{3.696000in}}%
\pgfusepath{clip}%
\pgfsetrectcap%
\pgfsetroundjoin%
\pgfsetlinewidth{1.505625pt}%
\definecolor{currentstroke}{rgb}{1.000000,0.000000,0.000000}%
\pgfsetstrokecolor{currentstroke}%
\pgfsetdash{}{0pt}%
\pgfpathmoveto{\pgfqpoint{2.175747in}{1.728813in}}%
\pgfpathlineto{\pgfqpoint{2.577938in}{1.298913in}}%
\pgfusepath{stroke}%
\end{pgfscope}%
\begin{pgfscope}%
\pgfpathrectangle{\pgfqpoint{0.100000in}{0.212622in}}{\pgfqpoint{3.696000in}{3.696000in}}%
\pgfusepath{clip}%
\pgfsetrectcap%
\pgfsetroundjoin%
\pgfsetlinewidth{1.505625pt}%
\definecolor{currentstroke}{rgb}{1.000000,0.000000,0.000000}%
\pgfsetstrokecolor{currentstroke}%
\pgfsetdash{}{0pt}%
\pgfpathmoveto{\pgfqpoint{2.178160in}{1.724169in}}%
\pgfpathlineto{\pgfqpoint{2.577938in}{1.298913in}}%
\pgfusepath{stroke}%
\end{pgfscope}%
\begin{pgfscope}%
\pgfpathrectangle{\pgfqpoint{0.100000in}{0.212622in}}{\pgfqpoint{3.696000in}{3.696000in}}%
\pgfusepath{clip}%
\pgfsetrectcap%
\pgfsetroundjoin%
\pgfsetlinewidth{1.505625pt}%
\definecolor{currentstroke}{rgb}{1.000000,0.000000,0.000000}%
\pgfsetstrokecolor{currentstroke}%
\pgfsetdash{}{0pt}%
\pgfpathmoveto{\pgfqpoint{2.181012in}{1.714719in}}%
\pgfpathlineto{\pgfqpoint{2.569400in}{1.290638in}}%
\pgfusepath{stroke}%
\end{pgfscope}%
\begin{pgfscope}%
\pgfpathrectangle{\pgfqpoint{0.100000in}{0.212622in}}{\pgfqpoint{3.696000in}{3.696000in}}%
\pgfusepath{clip}%
\pgfsetrectcap%
\pgfsetroundjoin%
\pgfsetlinewidth{1.505625pt}%
\definecolor{currentstroke}{rgb}{1.000000,0.000000,0.000000}%
\pgfsetstrokecolor{currentstroke}%
\pgfsetdash{}{0pt}%
\pgfpathmoveto{\pgfqpoint{2.183147in}{1.711430in}}%
\pgfpathlineto{\pgfqpoint{2.569400in}{1.290638in}}%
\pgfusepath{stroke}%
\end{pgfscope}%
\begin{pgfscope}%
\pgfpathrectangle{\pgfqpoint{0.100000in}{0.212622in}}{\pgfqpoint{3.696000in}{3.696000in}}%
\pgfusepath{clip}%
\pgfsetrectcap%
\pgfsetroundjoin%
\pgfsetlinewidth{1.505625pt}%
\definecolor{currentstroke}{rgb}{1.000000,0.000000,0.000000}%
\pgfsetstrokecolor{currentstroke}%
\pgfsetdash{}{0pt}%
\pgfpathmoveto{\pgfqpoint{2.184324in}{1.711034in}}%
\pgfpathlineto{\pgfqpoint{2.560850in}{1.282352in}}%
\pgfusepath{stroke}%
\end{pgfscope}%
\begin{pgfscope}%
\pgfpathrectangle{\pgfqpoint{0.100000in}{0.212622in}}{\pgfqpoint{3.696000in}{3.696000in}}%
\pgfusepath{clip}%
\pgfsetrectcap%
\pgfsetroundjoin%
\pgfsetlinewidth{1.505625pt}%
\definecolor{currentstroke}{rgb}{1.000000,0.000000,0.000000}%
\pgfsetstrokecolor{currentstroke}%
\pgfsetdash{}{0pt}%
\pgfpathmoveto{\pgfqpoint{2.184590in}{1.709724in}}%
\pgfpathlineto{\pgfqpoint{2.560850in}{1.282352in}}%
\pgfusepath{stroke}%
\end{pgfscope}%
\begin{pgfscope}%
\pgfpathrectangle{\pgfqpoint{0.100000in}{0.212622in}}{\pgfqpoint{3.696000in}{3.696000in}}%
\pgfusepath{clip}%
\pgfsetrectcap%
\pgfsetroundjoin%
\pgfsetlinewidth{1.505625pt}%
\definecolor{currentstroke}{rgb}{1.000000,0.000000,0.000000}%
\pgfsetstrokecolor{currentstroke}%
\pgfsetdash{}{0pt}%
\pgfpathmoveto{\pgfqpoint{2.185240in}{1.707679in}}%
\pgfpathlineto{\pgfqpoint{2.560850in}{1.282352in}}%
\pgfusepath{stroke}%
\end{pgfscope}%
\begin{pgfscope}%
\pgfpathrectangle{\pgfqpoint{0.100000in}{0.212622in}}{\pgfqpoint{3.696000in}{3.696000in}}%
\pgfusepath{clip}%
\pgfsetrectcap%
\pgfsetroundjoin%
\pgfsetlinewidth{1.505625pt}%
\definecolor{currentstroke}{rgb}{1.000000,0.000000,0.000000}%
\pgfsetstrokecolor{currentstroke}%
\pgfsetdash{}{0pt}%
\pgfpathmoveto{\pgfqpoint{2.185626in}{1.706902in}}%
\pgfpathlineto{\pgfqpoint{2.560850in}{1.282352in}}%
\pgfusepath{stroke}%
\end{pgfscope}%
\begin{pgfscope}%
\pgfpathrectangle{\pgfqpoint{0.100000in}{0.212622in}}{\pgfqpoint{3.696000in}{3.696000in}}%
\pgfusepath{clip}%
\pgfsetrectcap%
\pgfsetroundjoin%
\pgfsetlinewidth{1.505625pt}%
\definecolor{currentstroke}{rgb}{1.000000,0.000000,0.000000}%
\pgfsetstrokecolor{currentstroke}%
\pgfsetdash{}{0pt}%
\pgfpathmoveto{\pgfqpoint{2.186279in}{1.706977in}}%
\pgfpathlineto{\pgfqpoint{2.560850in}{1.282352in}}%
\pgfusepath{stroke}%
\end{pgfscope}%
\begin{pgfscope}%
\pgfpathrectangle{\pgfqpoint{0.100000in}{0.212622in}}{\pgfqpoint{3.696000in}{3.696000in}}%
\pgfusepath{clip}%
\pgfsetrectcap%
\pgfsetroundjoin%
\pgfsetlinewidth{1.505625pt}%
\definecolor{currentstroke}{rgb}{1.000000,0.000000,0.000000}%
\pgfsetstrokecolor{currentstroke}%
\pgfsetdash{}{0pt}%
\pgfpathmoveto{\pgfqpoint{2.186446in}{1.706043in}}%
\pgfpathlineto{\pgfqpoint{2.560850in}{1.282352in}}%
\pgfusepath{stroke}%
\end{pgfscope}%
\begin{pgfscope}%
\pgfpathrectangle{\pgfqpoint{0.100000in}{0.212622in}}{\pgfqpoint{3.696000in}{3.696000in}}%
\pgfusepath{clip}%
\pgfsetrectcap%
\pgfsetroundjoin%
\pgfsetlinewidth{1.505625pt}%
\definecolor{currentstroke}{rgb}{1.000000,0.000000,0.000000}%
\pgfsetstrokecolor{currentstroke}%
\pgfsetdash{}{0pt}%
\pgfpathmoveto{\pgfqpoint{2.187341in}{1.704212in}}%
\pgfpathlineto{\pgfqpoint{2.560850in}{1.282352in}}%
\pgfusepath{stroke}%
\end{pgfscope}%
\begin{pgfscope}%
\pgfpathrectangle{\pgfqpoint{0.100000in}{0.212622in}}{\pgfqpoint{3.696000in}{3.696000in}}%
\pgfusepath{clip}%
\pgfsetrectcap%
\pgfsetroundjoin%
\pgfsetlinewidth{1.505625pt}%
\definecolor{currentstroke}{rgb}{1.000000,0.000000,0.000000}%
\pgfsetstrokecolor{currentstroke}%
\pgfsetdash{}{0pt}%
\pgfpathmoveto{\pgfqpoint{2.187629in}{1.703783in}}%
\pgfpathlineto{\pgfqpoint{2.560850in}{1.282352in}}%
\pgfusepath{stroke}%
\end{pgfscope}%
\begin{pgfscope}%
\pgfpathrectangle{\pgfqpoint{0.100000in}{0.212622in}}{\pgfqpoint{3.696000in}{3.696000in}}%
\pgfusepath{clip}%
\pgfsetrectcap%
\pgfsetroundjoin%
\pgfsetlinewidth{1.505625pt}%
\definecolor{currentstroke}{rgb}{1.000000,0.000000,0.000000}%
\pgfsetstrokecolor{currentstroke}%
\pgfsetdash{}{0pt}%
\pgfpathmoveto{\pgfqpoint{2.188565in}{1.704065in}}%
\pgfpathlineto{\pgfqpoint{2.560850in}{1.282352in}}%
\pgfusepath{stroke}%
\end{pgfscope}%
\begin{pgfscope}%
\pgfpathrectangle{\pgfqpoint{0.100000in}{0.212622in}}{\pgfqpoint{3.696000in}{3.696000in}}%
\pgfusepath{clip}%
\pgfsetrectcap%
\pgfsetroundjoin%
\pgfsetlinewidth{1.505625pt}%
\definecolor{currentstroke}{rgb}{1.000000,0.000000,0.000000}%
\pgfsetstrokecolor{currentstroke}%
\pgfsetdash{}{0pt}%
\pgfpathmoveto{\pgfqpoint{2.189278in}{1.702106in}}%
\pgfpathlineto{\pgfqpoint{2.552288in}{1.274054in}}%
\pgfusepath{stroke}%
\end{pgfscope}%
\begin{pgfscope}%
\pgfpathrectangle{\pgfqpoint{0.100000in}{0.212622in}}{\pgfqpoint{3.696000in}{3.696000in}}%
\pgfusepath{clip}%
\pgfsetrectcap%
\pgfsetroundjoin%
\pgfsetlinewidth{1.505625pt}%
\definecolor{currentstroke}{rgb}{1.000000,0.000000,0.000000}%
\pgfsetstrokecolor{currentstroke}%
\pgfsetdash{}{0pt}%
\pgfpathmoveto{\pgfqpoint{2.190480in}{1.698848in}}%
\pgfpathlineto{\pgfqpoint{2.552288in}{1.274054in}}%
\pgfusepath{stroke}%
\end{pgfscope}%
\begin{pgfscope}%
\pgfpathrectangle{\pgfqpoint{0.100000in}{0.212622in}}{\pgfqpoint{3.696000in}{3.696000in}}%
\pgfusepath{clip}%
\pgfsetrectcap%
\pgfsetroundjoin%
\pgfsetlinewidth{1.505625pt}%
\definecolor{currentstroke}{rgb}{1.000000,0.000000,0.000000}%
\pgfsetstrokecolor{currentstroke}%
\pgfsetdash{}{0pt}%
\pgfpathmoveto{\pgfqpoint{2.191485in}{1.699199in}}%
\pgfpathlineto{\pgfqpoint{2.552288in}{1.274054in}}%
\pgfusepath{stroke}%
\end{pgfscope}%
\begin{pgfscope}%
\pgfpathrectangle{\pgfqpoint{0.100000in}{0.212622in}}{\pgfqpoint{3.696000in}{3.696000in}}%
\pgfusepath{clip}%
\pgfsetrectcap%
\pgfsetroundjoin%
\pgfsetlinewidth{1.505625pt}%
\definecolor{currentstroke}{rgb}{1.000000,0.000000,0.000000}%
\pgfsetstrokecolor{currentstroke}%
\pgfsetdash{}{0pt}%
\pgfpathmoveto{\pgfqpoint{2.192849in}{1.697482in}}%
\pgfpathlineto{\pgfqpoint{2.552288in}{1.274054in}}%
\pgfusepath{stroke}%
\end{pgfscope}%
\begin{pgfscope}%
\pgfpathrectangle{\pgfqpoint{0.100000in}{0.212622in}}{\pgfqpoint{3.696000in}{3.696000in}}%
\pgfusepath{clip}%
\pgfsetrectcap%
\pgfsetroundjoin%
\pgfsetlinewidth{1.505625pt}%
\definecolor{currentstroke}{rgb}{1.000000,0.000000,0.000000}%
\pgfsetstrokecolor{currentstroke}%
\pgfsetdash{}{0pt}%
\pgfpathmoveto{\pgfqpoint{2.194345in}{1.692579in}}%
\pgfpathlineto{\pgfqpoint{2.543716in}{1.265746in}}%
\pgfusepath{stroke}%
\end{pgfscope}%
\begin{pgfscope}%
\pgfpathrectangle{\pgfqpoint{0.100000in}{0.212622in}}{\pgfqpoint{3.696000in}{3.696000in}}%
\pgfusepath{clip}%
\pgfsetrectcap%
\pgfsetroundjoin%
\pgfsetlinewidth{1.505625pt}%
\definecolor{currentstroke}{rgb}{1.000000,0.000000,0.000000}%
\pgfsetstrokecolor{currentstroke}%
\pgfsetdash{}{0pt}%
\pgfpathmoveto{\pgfqpoint{2.196074in}{1.687384in}}%
\pgfpathlineto{\pgfqpoint{2.543716in}{1.265746in}}%
\pgfusepath{stroke}%
\end{pgfscope}%
\begin{pgfscope}%
\pgfpathrectangle{\pgfqpoint{0.100000in}{0.212622in}}{\pgfqpoint{3.696000in}{3.696000in}}%
\pgfusepath{clip}%
\pgfsetrectcap%
\pgfsetroundjoin%
\pgfsetlinewidth{1.505625pt}%
\definecolor{currentstroke}{rgb}{1.000000,0.000000,0.000000}%
\pgfsetstrokecolor{currentstroke}%
\pgfsetdash{}{0pt}%
\pgfpathmoveto{\pgfqpoint{2.199476in}{1.689836in}}%
\pgfpathlineto{\pgfqpoint{2.535131in}{1.257426in}}%
\pgfusepath{stroke}%
\end{pgfscope}%
\begin{pgfscope}%
\pgfpathrectangle{\pgfqpoint{0.100000in}{0.212622in}}{\pgfqpoint{3.696000in}{3.696000in}}%
\pgfusepath{clip}%
\pgfsetrectcap%
\pgfsetroundjoin%
\pgfsetlinewidth{1.505625pt}%
\definecolor{currentstroke}{rgb}{1.000000,0.000000,0.000000}%
\pgfsetstrokecolor{currentstroke}%
\pgfsetdash{}{0pt}%
\pgfpathmoveto{\pgfqpoint{2.199916in}{1.687463in}}%
\pgfpathlineto{\pgfqpoint{2.535131in}{1.257426in}}%
\pgfusepath{stroke}%
\end{pgfscope}%
\begin{pgfscope}%
\pgfpathrectangle{\pgfqpoint{0.100000in}{0.212622in}}{\pgfqpoint{3.696000in}{3.696000in}}%
\pgfusepath{clip}%
\pgfsetrectcap%
\pgfsetroundjoin%
\pgfsetlinewidth{1.505625pt}%
\definecolor{currentstroke}{rgb}{1.000000,0.000000,0.000000}%
\pgfsetstrokecolor{currentstroke}%
\pgfsetdash{}{0pt}%
\pgfpathmoveto{\pgfqpoint{2.201375in}{1.683033in}}%
\pgfpathlineto{\pgfqpoint{2.535131in}{1.257426in}}%
\pgfusepath{stroke}%
\end{pgfscope}%
\begin{pgfscope}%
\pgfpathrectangle{\pgfqpoint{0.100000in}{0.212622in}}{\pgfqpoint{3.696000in}{3.696000in}}%
\pgfusepath{clip}%
\pgfsetrectcap%
\pgfsetroundjoin%
\pgfsetlinewidth{1.505625pt}%
\definecolor{currentstroke}{rgb}{1.000000,0.000000,0.000000}%
\pgfsetstrokecolor{currentstroke}%
\pgfsetdash{}{0pt}%
\pgfpathmoveto{\pgfqpoint{2.202028in}{1.681859in}}%
\pgfpathlineto{\pgfqpoint{2.535131in}{1.257426in}}%
\pgfusepath{stroke}%
\end{pgfscope}%
\begin{pgfscope}%
\pgfpathrectangle{\pgfqpoint{0.100000in}{0.212622in}}{\pgfqpoint{3.696000in}{3.696000in}}%
\pgfusepath{clip}%
\pgfsetrectcap%
\pgfsetroundjoin%
\pgfsetlinewidth{1.505625pt}%
\definecolor{currentstroke}{rgb}{1.000000,0.000000,0.000000}%
\pgfsetstrokecolor{currentstroke}%
\pgfsetdash{}{0pt}%
\pgfpathmoveto{\pgfqpoint{2.203514in}{1.682789in}}%
\pgfpathlineto{\pgfqpoint{2.526536in}{1.249096in}}%
\pgfusepath{stroke}%
\end{pgfscope}%
\begin{pgfscope}%
\pgfpathrectangle{\pgfqpoint{0.100000in}{0.212622in}}{\pgfqpoint{3.696000in}{3.696000in}}%
\pgfusepath{clip}%
\pgfsetrectcap%
\pgfsetroundjoin%
\pgfsetlinewidth{1.505625pt}%
\definecolor{currentstroke}{rgb}{1.000000,0.000000,0.000000}%
\pgfsetstrokecolor{currentstroke}%
\pgfsetdash{}{0pt}%
\pgfpathmoveto{\pgfqpoint{2.203769in}{1.681374in}}%
\pgfpathlineto{\pgfqpoint{2.526536in}{1.249096in}}%
\pgfusepath{stroke}%
\end{pgfscope}%
\begin{pgfscope}%
\pgfpathrectangle{\pgfqpoint{0.100000in}{0.212622in}}{\pgfqpoint{3.696000in}{3.696000in}}%
\pgfusepath{clip}%
\pgfsetrectcap%
\pgfsetroundjoin%
\pgfsetlinewidth{1.505625pt}%
\definecolor{currentstroke}{rgb}{1.000000,0.000000,0.000000}%
\pgfsetstrokecolor{currentstroke}%
\pgfsetdash{}{0pt}%
\pgfpathmoveto{\pgfqpoint{2.205052in}{1.678788in}}%
\pgfpathlineto{\pgfqpoint{2.526536in}{1.249096in}}%
\pgfusepath{stroke}%
\end{pgfscope}%
\begin{pgfscope}%
\pgfpathrectangle{\pgfqpoint{0.100000in}{0.212622in}}{\pgfqpoint{3.696000in}{3.696000in}}%
\pgfusepath{clip}%
\pgfsetrectcap%
\pgfsetroundjoin%
\pgfsetlinewidth{1.505625pt}%
\definecolor{currentstroke}{rgb}{1.000000,0.000000,0.000000}%
\pgfsetstrokecolor{currentstroke}%
\pgfsetdash{}{0pt}%
\pgfpathmoveto{\pgfqpoint{2.206081in}{1.675221in}}%
\pgfpathlineto{\pgfqpoint{2.526536in}{1.249096in}}%
\pgfusepath{stroke}%
\end{pgfscope}%
\begin{pgfscope}%
\pgfpathrectangle{\pgfqpoint{0.100000in}{0.212622in}}{\pgfqpoint{3.696000in}{3.696000in}}%
\pgfusepath{clip}%
\pgfsetrectcap%
\pgfsetroundjoin%
\pgfsetlinewidth{1.505625pt}%
\definecolor{currentstroke}{rgb}{1.000000,0.000000,0.000000}%
\pgfsetstrokecolor{currentstroke}%
\pgfsetdash{}{0pt}%
\pgfpathmoveto{\pgfqpoint{2.208383in}{1.676853in}}%
\pgfpathlineto{\pgfqpoint{2.517928in}{1.240753in}}%
\pgfusepath{stroke}%
\end{pgfscope}%
\begin{pgfscope}%
\pgfpathrectangle{\pgfqpoint{0.100000in}{0.212622in}}{\pgfqpoint{3.696000in}{3.696000in}}%
\pgfusepath{clip}%
\pgfsetrectcap%
\pgfsetroundjoin%
\pgfsetlinewidth{1.505625pt}%
\definecolor{currentstroke}{rgb}{1.000000,0.000000,0.000000}%
\pgfsetstrokecolor{currentstroke}%
\pgfsetdash{}{0pt}%
\pgfpathmoveto{\pgfqpoint{2.209400in}{1.675622in}}%
\pgfpathlineto{\pgfqpoint{2.517928in}{1.240753in}}%
\pgfusepath{stroke}%
\end{pgfscope}%
\begin{pgfscope}%
\pgfpathrectangle{\pgfqpoint{0.100000in}{0.212622in}}{\pgfqpoint{3.696000in}{3.696000in}}%
\pgfusepath{clip}%
\pgfsetrectcap%
\pgfsetroundjoin%
\pgfsetlinewidth{1.505625pt}%
\definecolor{currentstroke}{rgb}{1.000000,0.000000,0.000000}%
\pgfsetstrokecolor{currentstroke}%
\pgfsetdash{}{0pt}%
\pgfpathmoveto{\pgfqpoint{2.209915in}{1.674641in}}%
\pgfpathlineto{\pgfqpoint{2.517928in}{1.240753in}}%
\pgfusepath{stroke}%
\end{pgfscope}%
\begin{pgfscope}%
\pgfpathrectangle{\pgfqpoint{0.100000in}{0.212622in}}{\pgfqpoint{3.696000in}{3.696000in}}%
\pgfusepath{clip}%
\pgfsetrectcap%
\pgfsetroundjoin%
\pgfsetlinewidth{1.505625pt}%
\definecolor{currentstroke}{rgb}{1.000000,0.000000,0.000000}%
\pgfsetstrokecolor{currentstroke}%
\pgfsetdash{}{0pt}%
\pgfpathmoveto{\pgfqpoint{2.210584in}{1.673073in}}%
\pgfpathlineto{\pgfqpoint{2.517928in}{1.240753in}}%
\pgfusepath{stroke}%
\end{pgfscope}%
\begin{pgfscope}%
\pgfpathrectangle{\pgfqpoint{0.100000in}{0.212622in}}{\pgfqpoint{3.696000in}{3.696000in}}%
\pgfusepath{clip}%
\pgfsetrectcap%
\pgfsetroundjoin%
\pgfsetlinewidth{1.505625pt}%
\definecolor{currentstroke}{rgb}{1.000000,0.000000,0.000000}%
\pgfsetstrokecolor{currentstroke}%
\pgfsetdash{}{0pt}%
\pgfpathmoveto{\pgfqpoint{2.211616in}{1.673478in}}%
\pgfpathlineto{\pgfqpoint{2.517928in}{1.240753in}}%
\pgfusepath{stroke}%
\end{pgfscope}%
\begin{pgfscope}%
\pgfpathrectangle{\pgfqpoint{0.100000in}{0.212622in}}{\pgfqpoint{3.696000in}{3.696000in}}%
\pgfusepath{clip}%
\pgfsetrectcap%
\pgfsetroundjoin%
\pgfsetlinewidth{1.505625pt}%
\definecolor{currentstroke}{rgb}{1.000000,0.000000,0.000000}%
\pgfsetstrokecolor{currentstroke}%
\pgfsetdash{}{0pt}%
\pgfpathmoveto{\pgfqpoint{2.211869in}{1.672337in}}%
\pgfpathlineto{\pgfqpoint{2.517928in}{1.240753in}}%
\pgfusepath{stroke}%
\end{pgfscope}%
\begin{pgfscope}%
\pgfpathrectangle{\pgfqpoint{0.100000in}{0.212622in}}{\pgfqpoint{3.696000in}{3.696000in}}%
\pgfusepath{clip}%
\pgfsetrectcap%
\pgfsetroundjoin%
\pgfsetlinewidth{1.505625pt}%
\definecolor{currentstroke}{rgb}{1.000000,0.000000,0.000000}%
\pgfsetstrokecolor{currentstroke}%
\pgfsetdash{}{0pt}%
\pgfpathmoveto{\pgfqpoint{2.212511in}{1.671647in}}%
\pgfpathlineto{\pgfqpoint{2.517928in}{1.240753in}}%
\pgfusepath{stroke}%
\end{pgfscope}%
\begin{pgfscope}%
\pgfpathrectangle{\pgfqpoint{0.100000in}{0.212622in}}{\pgfqpoint{3.696000in}{3.696000in}}%
\pgfusepath{clip}%
\pgfsetrectcap%
\pgfsetroundjoin%
\pgfsetlinewidth{1.505625pt}%
\definecolor{currentstroke}{rgb}{1.000000,0.000000,0.000000}%
\pgfsetstrokecolor{currentstroke}%
\pgfsetdash{}{0pt}%
\pgfpathmoveto{\pgfqpoint{2.212753in}{1.671511in}}%
\pgfpathlineto{\pgfqpoint{2.517928in}{1.240753in}}%
\pgfusepath{stroke}%
\end{pgfscope}%
\begin{pgfscope}%
\pgfpathrectangle{\pgfqpoint{0.100000in}{0.212622in}}{\pgfqpoint{3.696000in}{3.696000in}}%
\pgfusepath{clip}%
\pgfsetrectcap%
\pgfsetroundjoin%
\pgfsetlinewidth{1.505625pt}%
\definecolor{currentstroke}{rgb}{1.000000,0.000000,0.000000}%
\pgfsetstrokecolor{currentstroke}%
\pgfsetdash{}{0pt}%
\pgfpathmoveto{\pgfqpoint{2.213606in}{1.671452in}}%
\pgfpathlineto{\pgfqpoint{2.517928in}{1.240753in}}%
\pgfusepath{stroke}%
\end{pgfscope}%
\begin{pgfscope}%
\pgfpathrectangle{\pgfqpoint{0.100000in}{0.212622in}}{\pgfqpoint{3.696000in}{3.696000in}}%
\pgfusepath{clip}%
\pgfsetrectcap%
\pgfsetroundjoin%
\pgfsetlinewidth{1.505625pt}%
\definecolor{currentstroke}{rgb}{1.000000,0.000000,0.000000}%
\pgfsetstrokecolor{currentstroke}%
\pgfsetdash{}{0pt}%
\pgfpathmoveto{\pgfqpoint{2.213985in}{1.670498in}}%
\pgfpathlineto{\pgfqpoint{2.509309in}{1.232400in}}%
\pgfusepath{stroke}%
\end{pgfscope}%
\begin{pgfscope}%
\pgfpathrectangle{\pgfqpoint{0.100000in}{0.212622in}}{\pgfqpoint{3.696000in}{3.696000in}}%
\pgfusepath{clip}%
\pgfsetrectcap%
\pgfsetroundjoin%
\pgfsetlinewidth{1.505625pt}%
\definecolor{currentstroke}{rgb}{1.000000,0.000000,0.000000}%
\pgfsetstrokecolor{currentstroke}%
\pgfsetdash{}{0pt}%
\pgfpathmoveto{\pgfqpoint{2.215392in}{1.666281in}}%
\pgfpathlineto{\pgfqpoint{2.509309in}{1.232400in}}%
\pgfusepath{stroke}%
\end{pgfscope}%
\begin{pgfscope}%
\pgfpathrectangle{\pgfqpoint{0.100000in}{0.212622in}}{\pgfqpoint{3.696000in}{3.696000in}}%
\pgfusepath{clip}%
\pgfsetrectcap%
\pgfsetroundjoin%
\pgfsetlinewidth{1.505625pt}%
\definecolor{currentstroke}{rgb}{1.000000,0.000000,0.000000}%
\pgfsetstrokecolor{currentstroke}%
\pgfsetdash{}{0pt}%
\pgfpathmoveto{\pgfqpoint{2.217742in}{1.666599in}}%
\pgfpathlineto{\pgfqpoint{2.509309in}{1.232400in}}%
\pgfusepath{stroke}%
\end{pgfscope}%
\begin{pgfscope}%
\pgfpathrectangle{\pgfqpoint{0.100000in}{0.212622in}}{\pgfqpoint{3.696000in}{3.696000in}}%
\pgfusepath{clip}%
\pgfsetrectcap%
\pgfsetroundjoin%
\pgfsetlinewidth{1.505625pt}%
\definecolor{currentstroke}{rgb}{1.000000,0.000000,0.000000}%
\pgfsetstrokecolor{currentstroke}%
\pgfsetdash{}{0pt}%
\pgfpathmoveto{\pgfqpoint{2.219308in}{1.664528in}}%
\pgfpathlineto{\pgfqpoint{2.500678in}{1.224036in}}%
\pgfusepath{stroke}%
\end{pgfscope}%
\begin{pgfscope}%
\pgfpathrectangle{\pgfqpoint{0.100000in}{0.212622in}}{\pgfqpoint{3.696000in}{3.696000in}}%
\pgfusepath{clip}%
\pgfsetrectcap%
\pgfsetroundjoin%
\pgfsetlinewidth{1.505625pt}%
\definecolor{currentstroke}{rgb}{1.000000,0.000000,0.000000}%
\pgfsetstrokecolor{currentstroke}%
\pgfsetdash{}{0pt}%
\pgfpathmoveto{\pgfqpoint{2.220832in}{1.661938in}}%
\pgfpathlineto{\pgfqpoint{2.500678in}{1.224036in}}%
\pgfusepath{stroke}%
\end{pgfscope}%
\begin{pgfscope}%
\pgfpathrectangle{\pgfqpoint{0.100000in}{0.212622in}}{\pgfqpoint{3.696000in}{3.696000in}}%
\pgfusepath{clip}%
\pgfsetrectcap%
\pgfsetroundjoin%
\pgfsetlinewidth{1.505625pt}%
\definecolor{currentstroke}{rgb}{1.000000,0.000000,0.000000}%
\pgfsetstrokecolor{currentstroke}%
\pgfsetdash{}{0pt}%
\pgfpathmoveto{\pgfqpoint{2.223806in}{1.656381in}}%
\pgfpathlineto{\pgfqpoint{2.500678in}{1.224036in}}%
\pgfusepath{stroke}%
\end{pgfscope}%
\begin{pgfscope}%
\pgfpathrectangle{\pgfqpoint{0.100000in}{0.212622in}}{\pgfqpoint{3.696000in}{3.696000in}}%
\pgfusepath{clip}%
\pgfsetrectcap%
\pgfsetroundjoin%
\pgfsetlinewidth{1.505625pt}%
\definecolor{currentstroke}{rgb}{1.000000,0.000000,0.000000}%
\pgfsetstrokecolor{currentstroke}%
\pgfsetdash{}{0pt}%
\pgfpathmoveto{\pgfqpoint{2.227786in}{1.657172in}}%
\pgfpathlineto{\pgfqpoint{2.492036in}{1.215660in}}%
\pgfusepath{stroke}%
\end{pgfscope}%
\begin{pgfscope}%
\pgfpathrectangle{\pgfqpoint{0.100000in}{0.212622in}}{\pgfqpoint{3.696000in}{3.696000in}}%
\pgfusepath{clip}%
\pgfsetrectcap%
\pgfsetroundjoin%
\pgfsetlinewidth{1.505625pt}%
\definecolor{currentstroke}{rgb}{1.000000,0.000000,0.000000}%
\pgfsetstrokecolor{currentstroke}%
\pgfsetdash{}{0pt}%
\pgfpathmoveto{\pgfqpoint{2.230498in}{1.652079in}}%
\pgfpathlineto{\pgfqpoint{2.483382in}{1.207273in}}%
\pgfusepath{stroke}%
\end{pgfscope}%
\begin{pgfscope}%
\pgfpathrectangle{\pgfqpoint{0.100000in}{0.212622in}}{\pgfqpoint{3.696000in}{3.696000in}}%
\pgfusepath{clip}%
\pgfsetrectcap%
\pgfsetroundjoin%
\pgfsetlinewidth{1.505625pt}%
\definecolor{currentstroke}{rgb}{1.000000,0.000000,0.000000}%
\pgfsetstrokecolor{currentstroke}%
\pgfsetdash{}{0pt}%
\pgfpathmoveto{\pgfqpoint{2.232320in}{1.648077in}}%
\pgfpathlineto{\pgfqpoint{2.483382in}{1.207273in}}%
\pgfusepath{stroke}%
\end{pgfscope}%
\begin{pgfscope}%
\pgfpathrectangle{\pgfqpoint{0.100000in}{0.212622in}}{\pgfqpoint{3.696000in}{3.696000in}}%
\pgfusepath{clip}%
\pgfsetrectcap%
\pgfsetroundjoin%
\pgfsetlinewidth{1.505625pt}%
\definecolor{currentstroke}{rgb}{1.000000,0.000000,0.000000}%
\pgfsetstrokecolor{currentstroke}%
\pgfsetdash{}{0pt}%
\pgfpathmoveto{\pgfqpoint{2.234178in}{1.645143in}}%
\pgfpathlineto{\pgfqpoint{2.483382in}{1.207273in}}%
\pgfusepath{stroke}%
\end{pgfscope}%
\begin{pgfscope}%
\pgfpathrectangle{\pgfqpoint{0.100000in}{0.212622in}}{\pgfqpoint{3.696000in}{3.696000in}}%
\pgfusepath{clip}%
\pgfsetrectcap%
\pgfsetroundjoin%
\pgfsetlinewidth{1.505625pt}%
\definecolor{currentstroke}{rgb}{1.000000,0.000000,0.000000}%
\pgfsetstrokecolor{currentstroke}%
\pgfsetdash{}{0pt}%
\pgfpathmoveto{\pgfqpoint{2.235518in}{1.645037in}}%
\pgfpathlineto{\pgfqpoint{2.483382in}{1.207273in}}%
\pgfusepath{stroke}%
\end{pgfscope}%
\begin{pgfscope}%
\pgfpathrectangle{\pgfqpoint{0.100000in}{0.212622in}}{\pgfqpoint{3.696000in}{3.696000in}}%
\pgfusepath{clip}%
\pgfsetrectcap%
\pgfsetroundjoin%
\pgfsetlinewidth{1.505625pt}%
\definecolor{currentstroke}{rgb}{1.000000,0.000000,0.000000}%
\pgfsetstrokecolor{currentstroke}%
\pgfsetdash{}{0pt}%
\pgfpathmoveto{\pgfqpoint{2.235853in}{1.643558in}}%
\pgfpathlineto{\pgfqpoint{2.474717in}{1.198874in}}%
\pgfusepath{stroke}%
\end{pgfscope}%
\begin{pgfscope}%
\pgfpathrectangle{\pgfqpoint{0.100000in}{0.212622in}}{\pgfqpoint{3.696000in}{3.696000in}}%
\pgfusepath{clip}%
\pgfsetrectcap%
\pgfsetroundjoin%
\pgfsetlinewidth{1.505625pt}%
\definecolor{currentstroke}{rgb}{1.000000,0.000000,0.000000}%
\pgfsetstrokecolor{currentstroke}%
\pgfsetdash{}{0pt}%
\pgfpathmoveto{\pgfqpoint{2.237022in}{1.641105in}}%
\pgfpathlineto{\pgfqpoint{2.474717in}{1.198874in}}%
\pgfusepath{stroke}%
\end{pgfscope}%
\begin{pgfscope}%
\pgfpathrectangle{\pgfqpoint{0.100000in}{0.212622in}}{\pgfqpoint{3.696000in}{3.696000in}}%
\pgfusepath{clip}%
\pgfsetrectcap%
\pgfsetroundjoin%
\pgfsetlinewidth{1.505625pt}%
\definecolor{currentstroke}{rgb}{1.000000,0.000000,0.000000}%
\pgfsetstrokecolor{currentstroke}%
\pgfsetdash{}{0pt}%
\pgfpathmoveto{\pgfqpoint{2.237494in}{1.640234in}}%
\pgfpathlineto{\pgfqpoint{2.474717in}{1.198874in}}%
\pgfusepath{stroke}%
\end{pgfscope}%
\begin{pgfscope}%
\pgfpathrectangle{\pgfqpoint{0.100000in}{0.212622in}}{\pgfqpoint{3.696000in}{3.696000in}}%
\pgfusepath{clip}%
\pgfsetrectcap%
\pgfsetroundjoin%
\pgfsetlinewidth{1.505625pt}%
\definecolor{currentstroke}{rgb}{1.000000,0.000000,0.000000}%
\pgfsetstrokecolor{currentstroke}%
\pgfsetdash{}{0pt}%
\pgfpathmoveto{\pgfqpoint{2.238420in}{1.640075in}}%
\pgfpathlineto{\pgfqpoint{2.474717in}{1.198874in}}%
\pgfusepath{stroke}%
\end{pgfscope}%
\begin{pgfscope}%
\pgfpathrectangle{\pgfqpoint{0.100000in}{0.212622in}}{\pgfqpoint{3.696000in}{3.696000in}}%
\pgfusepath{clip}%
\pgfsetrectcap%
\pgfsetroundjoin%
\pgfsetlinewidth{1.505625pt}%
\definecolor{currentstroke}{rgb}{1.000000,0.000000,0.000000}%
\pgfsetstrokecolor{currentstroke}%
\pgfsetdash{}{0pt}%
\pgfpathmoveto{\pgfqpoint{2.238742in}{1.639351in}}%
\pgfpathlineto{\pgfqpoint{2.474717in}{1.198874in}}%
\pgfusepath{stroke}%
\end{pgfscope}%
\begin{pgfscope}%
\pgfpathrectangle{\pgfqpoint{0.100000in}{0.212622in}}{\pgfqpoint{3.696000in}{3.696000in}}%
\pgfusepath{clip}%
\pgfsetrectcap%
\pgfsetroundjoin%
\pgfsetlinewidth{1.505625pt}%
\definecolor{currentstroke}{rgb}{1.000000,0.000000,0.000000}%
\pgfsetstrokecolor{currentstroke}%
\pgfsetdash{}{0pt}%
\pgfpathmoveto{\pgfqpoint{2.239690in}{1.636462in}}%
\pgfpathlineto{\pgfqpoint{2.474717in}{1.198874in}}%
\pgfusepath{stroke}%
\end{pgfscope}%
\begin{pgfscope}%
\pgfpathrectangle{\pgfqpoint{0.100000in}{0.212622in}}{\pgfqpoint{3.696000in}{3.696000in}}%
\pgfusepath{clip}%
\pgfsetrectcap%
\pgfsetroundjoin%
\pgfsetlinewidth{1.505625pt}%
\definecolor{currentstroke}{rgb}{1.000000,0.000000,0.000000}%
\pgfsetstrokecolor{currentstroke}%
\pgfsetdash{}{0pt}%
\pgfpathmoveto{\pgfqpoint{2.240744in}{1.634927in}}%
\pgfpathlineto{\pgfqpoint{2.474717in}{1.198874in}}%
\pgfusepath{stroke}%
\end{pgfscope}%
\begin{pgfscope}%
\pgfpathrectangle{\pgfqpoint{0.100000in}{0.212622in}}{\pgfqpoint{3.696000in}{3.696000in}}%
\pgfusepath{clip}%
\pgfsetrectcap%
\pgfsetroundjoin%
\pgfsetlinewidth{1.505625pt}%
\definecolor{currentstroke}{rgb}{1.000000,0.000000,0.000000}%
\pgfsetstrokecolor{currentstroke}%
\pgfsetdash{}{0pt}%
\pgfpathmoveto{\pgfqpoint{2.243162in}{1.633690in}}%
\pgfpathlineto{\pgfqpoint{2.466039in}{1.190464in}}%
\pgfusepath{stroke}%
\end{pgfscope}%
\begin{pgfscope}%
\pgfpathrectangle{\pgfqpoint{0.100000in}{0.212622in}}{\pgfqpoint{3.696000in}{3.696000in}}%
\pgfusepath{clip}%
\pgfsetrectcap%
\pgfsetroundjoin%
\pgfsetlinewidth{1.505625pt}%
\definecolor{currentstroke}{rgb}{1.000000,0.000000,0.000000}%
\pgfsetstrokecolor{currentstroke}%
\pgfsetdash{}{0pt}%
\pgfpathmoveto{\pgfqpoint{2.243441in}{1.632043in}}%
\pgfpathlineto{\pgfqpoint{2.466039in}{1.190464in}}%
\pgfusepath{stroke}%
\end{pgfscope}%
\begin{pgfscope}%
\pgfpathrectangle{\pgfqpoint{0.100000in}{0.212622in}}{\pgfqpoint{3.696000in}{3.696000in}}%
\pgfusepath{clip}%
\pgfsetrectcap%
\pgfsetroundjoin%
\pgfsetlinewidth{1.505625pt}%
\definecolor{currentstroke}{rgb}{1.000000,0.000000,0.000000}%
\pgfsetstrokecolor{currentstroke}%
\pgfsetdash{}{0pt}%
\pgfpathmoveto{\pgfqpoint{2.244831in}{1.626633in}}%
\pgfpathlineto{\pgfqpoint{2.466039in}{1.190464in}}%
\pgfusepath{stroke}%
\end{pgfscope}%
\begin{pgfscope}%
\pgfpathrectangle{\pgfqpoint{0.100000in}{0.212622in}}{\pgfqpoint{3.696000in}{3.696000in}}%
\pgfusepath{clip}%
\pgfsetrectcap%
\pgfsetroundjoin%
\pgfsetlinewidth{1.505625pt}%
\definecolor{currentstroke}{rgb}{1.000000,0.000000,0.000000}%
\pgfsetstrokecolor{currentstroke}%
\pgfsetdash{}{0pt}%
\pgfpathmoveto{\pgfqpoint{2.245780in}{1.625958in}}%
\pgfpathlineto{\pgfqpoint{2.466039in}{1.190464in}}%
\pgfusepath{stroke}%
\end{pgfscope}%
\begin{pgfscope}%
\pgfpathrectangle{\pgfqpoint{0.100000in}{0.212622in}}{\pgfqpoint{3.696000in}{3.696000in}}%
\pgfusepath{clip}%
\pgfsetrectcap%
\pgfsetroundjoin%
\pgfsetlinewidth{1.505625pt}%
\definecolor{currentstroke}{rgb}{1.000000,0.000000,0.000000}%
\pgfsetstrokecolor{currentstroke}%
\pgfsetdash{}{0pt}%
\pgfpathmoveto{\pgfqpoint{2.246790in}{1.625170in}}%
\pgfpathlineto{\pgfqpoint{2.466039in}{1.190464in}}%
\pgfusepath{stroke}%
\end{pgfscope}%
\begin{pgfscope}%
\pgfpathrectangle{\pgfqpoint{0.100000in}{0.212622in}}{\pgfqpoint{3.696000in}{3.696000in}}%
\pgfusepath{clip}%
\pgfsetrectcap%
\pgfsetroundjoin%
\pgfsetlinewidth{1.505625pt}%
\definecolor{currentstroke}{rgb}{1.000000,0.000000,0.000000}%
\pgfsetstrokecolor{currentstroke}%
\pgfsetdash{}{0pt}%
\pgfpathmoveto{\pgfqpoint{2.247088in}{1.624445in}}%
\pgfpathlineto{\pgfqpoint{2.457350in}{1.182043in}}%
\pgfusepath{stroke}%
\end{pgfscope}%
\begin{pgfscope}%
\pgfpathrectangle{\pgfqpoint{0.100000in}{0.212622in}}{\pgfqpoint{3.696000in}{3.696000in}}%
\pgfusepath{clip}%
\pgfsetrectcap%
\pgfsetroundjoin%
\pgfsetlinewidth{1.505625pt}%
\definecolor{currentstroke}{rgb}{1.000000,0.000000,0.000000}%
\pgfsetstrokecolor{currentstroke}%
\pgfsetdash{}{0pt}%
\pgfpathmoveto{\pgfqpoint{2.248118in}{1.621549in}}%
\pgfpathlineto{\pgfqpoint{2.457350in}{1.182043in}}%
\pgfusepath{stroke}%
\end{pgfscope}%
\begin{pgfscope}%
\pgfpathrectangle{\pgfqpoint{0.100000in}{0.212622in}}{\pgfqpoint{3.696000in}{3.696000in}}%
\pgfusepath{clip}%
\pgfsetrectcap%
\pgfsetroundjoin%
\pgfsetlinewidth{1.505625pt}%
\definecolor{currentstroke}{rgb}{1.000000,0.000000,0.000000}%
\pgfsetstrokecolor{currentstroke}%
\pgfsetdash{}{0pt}%
\pgfpathmoveto{\pgfqpoint{2.248782in}{1.620852in}}%
\pgfpathlineto{\pgfqpoint{2.457350in}{1.182043in}}%
\pgfusepath{stroke}%
\end{pgfscope}%
\begin{pgfscope}%
\pgfpathrectangle{\pgfqpoint{0.100000in}{0.212622in}}{\pgfqpoint{3.696000in}{3.696000in}}%
\pgfusepath{clip}%
\pgfsetrectcap%
\pgfsetroundjoin%
\pgfsetlinewidth{1.505625pt}%
\definecolor{currentstroke}{rgb}{1.000000,0.000000,0.000000}%
\pgfsetstrokecolor{currentstroke}%
\pgfsetdash{}{0pt}%
\pgfpathmoveto{\pgfqpoint{2.249587in}{1.620738in}}%
\pgfpathlineto{\pgfqpoint{2.457350in}{1.182043in}}%
\pgfusepath{stroke}%
\end{pgfscope}%
\begin{pgfscope}%
\pgfpathrectangle{\pgfqpoint{0.100000in}{0.212622in}}{\pgfqpoint{3.696000in}{3.696000in}}%
\pgfusepath{clip}%
\pgfsetrectcap%
\pgfsetroundjoin%
\pgfsetlinewidth{1.505625pt}%
\definecolor{currentstroke}{rgb}{1.000000,0.000000,0.000000}%
\pgfsetstrokecolor{currentstroke}%
\pgfsetdash{}{0pt}%
\pgfpathmoveto{\pgfqpoint{2.249830in}{1.620127in}}%
\pgfpathlineto{\pgfqpoint{2.457350in}{1.182043in}}%
\pgfusepath{stroke}%
\end{pgfscope}%
\begin{pgfscope}%
\pgfpathrectangle{\pgfqpoint{0.100000in}{0.212622in}}{\pgfqpoint{3.696000in}{3.696000in}}%
\pgfusepath{clip}%
\pgfsetrectcap%
\pgfsetroundjoin%
\pgfsetlinewidth{1.505625pt}%
\definecolor{currentstroke}{rgb}{1.000000,0.000000,0.000000}%
\pgfsetstrokecolor{currentstroke}%
\pgfsetdash{}{0pt}%
\pgfpathmoveto{\pgfqpoint{2.250458in}{1.619103in}}%
\pgfpathlineto{\pgfqpoint{2.457350in}{1.182043in}}%
\pgfusepath{stroke}%
\end{pgfscope}%
\begin{pgfscope}%
\pgfpathrectangle{\pgfqpoint{0.100000in}{0.212622in}}{\pgfqpoint{3.696000in}{3.696000in}}%
\pgfusepath{clip}%
\pgfsetrectcap%
\pgfsetroundjoin%
\pgfsetlinewidth{1.505625pt}%
\definecolor{currentstroke}{rgb}{1.000000,0.000000,0.000000}%
\pgfsetstrokecolor{currentstroke}%
\pgfsetdash{}{0pt}%
\pgfpathmoveto{\pgfqpoint{2.251459in}{1.616269in}}%
\pgfpathlineto{\pgfqpoint{2.457350in}{1.182043in}}%
\pgfusepath{stroke}%
\end{pgfscope}%
\begin{pgfscope}%
\pgfpathrectangle{\pgfqpoint{0.100000in}{0.212622in}}{\pgfqpoint{3.696000in}{3.696000in}}%
\pgfusepath{clip}%
\pgfsetrectcap%
\pgfsetroundjoin%
\pgfsetlinewidth{1.505625pt}%
\definecolor{currentstroke}{rgb}{1.000000,0.000000,0.000000}%
\pgfsetstrokecolor{currentstroke}%
\pgfsetdash{}{0pt}%
\pgfpathmoveto{\pgfqpoint{2.252061in}{1.616020in}}%
\pgfpathlineto{\pgfqpoint{2.457350in}{1.182043in}}%
\pgfusepath{stroke}%
\end{pgfscope}%
\begin{pgfscope}%
\pgfpathrectangle{\pgfqpoint{0.100000in}{0.212622in}}{\pgfqpoint{3.696000in}{3.696000in}}%
\pgfusepath{clip}%
\pgfsetrectcap%
\pgfsetroundjoin%
\pgfsetlinewidth{1.505625pt}%
\definecolor{currentstroke}{rgb}{1.000000,0.000000,0.000000}%
\pgfsetstrokecolor{currentstroke}%
\pgfsetdash{}{0pt}%
\pgfpathmoveto{\pgfqpoint{2.252438in}{1.615609in}}%
\pgfpathlineto{\pgfqpoint{2.457350in}{1.182043in}}%
\pgfusepath{stroke}%
\end{pgfscope}%
\begin{pgfscope}%
\pgfpathrectangle{\pgfqpoint{0.100000in}{0.212622in}}{\pgfqpoint{3.696000in}{3.696000in}}%
\pgfusepath{clip}%
\pgfsetrectcap%
\pgfsetroundjoin%
\pgfsetlinewidth{1.505625pt}%
\definecolor{currentstroke}{rgb}{1.000000,0.000000,0.000000}%
\pgfsetstrokecolor{currentstroke}%
\pgfsetdash{}{0pt}%
\pgfpathmoveto{\pgfqpoint{2.253016in}{1.614050in}}%
\pgfpathlineto{\pgfqpoint{2.448649in}{1.173611in}}%
\pgfusepath{stroke}%
\end{pgfscope}%
\begin{pgfscope}%
\pgfpathrectangle{\pgfqpoint{0.100000in}{0.212622in}}{\pgfqpoint{3.696000in}{3.696000in}}%
\pgfusepath{clip}%
\pgfsetrectcap%
\pgfsetroundjoin%
\pgfsetlinewidth{1.505625pt}%
\definecolor{currentstroke}{rgb}{1.000000,0.000000,0.000000}%
\pgfsetstrokecolor{currentstroke}%
\pgfsetdash{}{0pt}%
\pgfpathmoveto{\pgfqpoint{2.254201in}{1.610168in}}%
\pgfpathlineto{\pgfqpoint{2.448649in}{1.173611in}}%
\pgfusepath{stroke}%
\end{pgfscope}%
\begin{pgfscope}%
\pgfpathrectangle{\pgfqpoint{0.100000in}{0.212622in}}{\pgfqpoint{3.696000in}{3.696000in}}%
\pgfusepath{clip}%
\pgfsetrectcap%
\pgfsetroundjoin%
\pgfsetlinewidth{1.505625pt}%
\definecolor{currentstroke}{rgb}{1.000000,0.000000,0.000000}%
\pgfsetstrokecolor{currentstroke}%
\pgfsetdash{}{0pt}%
\pgfpathmoveto{\pgfqpoint{2.255289in}{1.610010in}}%
\pgfpathlineto{\pgfqpoint{2.448649in}{1.173611in}}%
\pgfusepath{stroke}%
\end{pgfscope}%
\begin{pgfscope}%
\pgfpathrectangle{\pgfqpoint{0.100000in}{0.212622in}}{\pgfqpoint{3.696000in}{3.696000in}}%
\pgfusepath{clip}%
\pgfsetrectcap%
\pgfsetroundjoin%
\pgfsetlinewidth{1.505625pt}%
\definecolor{currentstroke}{rgb}{1.000000,0.000000,0.000000}%
\pgfsetstrokecolor{currentstroke}%
\pgfsetdash{}{0pt}%
\pgfpathmoveto{\pgfqpoint{2.256377in}{1.609172in}}%
\pgfpathlineto{\pgfqpoint{2.448649in}{1.173611in}}%
\pgfusepath{stroke}%
\end{pgfscope}%
\begin{pgfscope}%
\pgfpathrectangle{\pgfqpoint{0.100000in}{0.212622in}}{\pgfqpoint{3.696000in}{3.696000in}}%
\pgfusepath{clip}%
\pgfsetrectcap%
\pgfsetroundjoin%
\pgfsetlinewidth{1.505625pt}%
\definecolor{currentstroke}{rgb}{1.000000,0.000000,0.000000}%
\pgfsetstrokecolor{currentstroke}%
\pgfsetdash{}{0pt}%
\pgfpathmoveto{\pgfqpoint{2.257240in}{1.605966in}}%
\pgfpathlineto{\pgfqpoint{2.439937in}{1.165167in}}%
\pgfusepath{stroke}%
\end{pgfscope}%
\begin{pgfscope}%
\pgfpathrectangle{\pgfqpoint{0.100000in}{0.212622in}}{\pgfqpoint{3.696000in}{3.696000in}}%
\pgfusepath{clip}%
\pgfsetrectcap%
\pgfsetroundjoin%
\pgfsetlinewidth{1.505625pt}%
\definecolor{currentstroke}{rgb}{1.000000,0.000000,0.000000}%
\pgfsetstrokecolor{currentstroke}%
\pgfsetdash{}{0pt}%
\pgfpathmoveto{\pgfqpoint{2.259598in}{1.598988in}}%
\pgfpathlineto{\pgfqpoint{2.439937in}{1.165167in}}%
\pgfusepath{stroke}%
\end{pgfscope}%
\begin{pgfscope}%
\pgfpathrectangle{\pgfqpoint{0.100000in}{0.212622in}}{\pgfqpoint{3.696000in}{3.696000in}}%
\pgfusepath{clip}%
\pgfsetrectcap%
\pgfsetroundjoin%
\pgfsetlinewidth{1.505625pt}%
\definecolor{currentstroke}{rgb}{1.000000,0.000000,0.000000}%
\pgfsetstrokecolor{currentstroke}%
\pgfsetdash{}{0pt}%
\pgfpathmoveto{\pgfqpoint{2.261588in}{1.593756in}}%
\pgfpathlineto{\pgfqpoint{2.439937in}{1.165167in}}%
\pgfusepath{stroke}%
\end{pgfscope}%
\begin{pgfscope}%
\pgfpathrectangle{\pgfqpoint{0.100000in}{0.212622in}}{\pgfqpoint{3.696000in}{3.696000in}}%
\pgfusepath{clip}%
\pgfsetrectcap%
\pgfsetroundjoin%
\pgfsetlinewidth{1.505625pt}%
\definecolor{currentstroke}{rgb}{1.000000,0.000000,0.000000}%
\pgfsetstrokecolor{currentstroke}%
\pgfsetdash{}{0pt}%
\pgfpathmoveto{\pgfqpoint{2.265258in}{1.595416in}}%
\pgfpathlineto{\pgfqpoint{2.431212in}{1.156711in}}%
\pgfusepath{stroke}%
\end{pgfscope}%
\begin{pgfscope}%
\pgfpathrectangle{\pgfqpoint{0.100000in}{0.212622in}}{\pgfqpoint{3.696000in}{3.696000in}}%
\pgfusepath{clip}%
\pgfsetrectcap%
\pgfsetroundjoin%
\pgfsetlinewidth{1.505625pt}%
\definecolor{currentstroke}{rgb}{1.000000,0.000000,0.000000}%
\pgfsetstrokecolor{currentstroke}%
\pgfsetdash{}{0pt}%
\pgfpathmoveto{\pgfqpoint{2.265793in}{1.592601in}}%
\pgfpathlineto{\pgfqpoint{2.431212in}{1.156711in}}%
\pgfusepath{stroke}%
\end{pgfscope}%
\begin{pgfscope}%
\pgfpathrectangle{\pgfqpoint{0.100000in}{0.212622in}}{\pgfqpoint{3.696000in}{3.696000in}}%
\pgfusepath{clip}%
\pgfsetrectcap%
\pgfsetroundjoin%
\pgfsetlinewidth{1.505625pt}%
\definecolor{currentstroke}{rgb}{1.000000,0.000000,0.000000}%
\pgfsetstrokecolor{currentstroke}%
\pgfsetdash{}{0pt}%
\pgfpathmoveto{\pgfqpoint{2.268032in}{1.588368in}}%
\pgfpathlineto{\pgfqpoint{2.431212in}{1.156711in}}%
\pgfusepath{stroke}%
\end{pgfscope}%
\begin{pgfscope}%
\pgfpathrectangle{\pgfqpoint{0.100000in}{0.212622in}}{\pgfqpoint{3.696000in}{3.696000in}}%
\pgfusepath{clip}%
\pgfsetrectcap%
\pgfsetroundjoin%
\pgfsetlinewidth{1.505625pt}%
\definecolor{currentstroke}{rgb}{1.000000,0.000000,0.000000}%
\pgfsetstrokecolor{currentstroke}%
\pgfsetdash{}{0pt}%
\pgfpathmoveto{\pgfqpoint{2.269340in}{1.583248in}}%
\pgfpathlineto{\pgfqpoint{2.422476in}{1.148244in}}%
\pgfusepath{stroke}%
\end{pgfscope}%
\begin{pgfscope}%
\pgfpathrectangle{\pgfqpoint{0.100000in}{0.212622in}}{\pgfqpoint{3.696000in}{3.696000in}}%
\pgfusepath{clip}%
\pgfsetrectcap%
\pgfsetroundjoin%
\pgfsetlinewidth{1.505625pt}%
\definecolor{currentstroke}{rgb}{1.000000,0.000000,0.000000}%
\pgfsetstrokecolor{currentstroke}%
\pgfsetdash{}{0pt}%
\pgfpathmoveto{\pgfqpoint{2.270736in}{1.583355in}}%
\pgfpathlineto{\pgfqpoint{2.422476in}{1.148244in}}%
\pgfusepath{stroke}%
\end{pgfscope}%
\begin{pgfscope}%
\pgfpathrectangle{\pgfqpoint{0.100000in}{0.212622in}}{\pgfqpoint{3.696000in}{3.696000in}}%
\pgfusepath{clip}%
\pgfsetrectcap%
\pgfsetroundjoin%
\pgfsetlinewidth{1.505625pt}%
\definecolor{currentstroke}{rgb}{1.000000,0.000000,0.000000}%
\pgfsetstrokecolor{currentstroke}%
\pgfsetdash{}{0pt}%
\pgfpathmoveto{\pgfqpoint{2.271106in}{1.582869in}}%
\pgfpathlineto{\pgfqpoint{2.422476in}{1.148244in}}%
\pgfusepath{stroke}%
\end{pgfscope}%
\begin{pgfscope}%
\pgfpathrectangle{\pgfqpoint{0.100000in}{0.212622in}}{\pgfqpoint{3.696000in}{3.696000in}}%
\pgfusepath{clip}%
\pgfsetrectcap%
\pgfsetroundjoin%
\pgfsetlinewidth{1.505625pt}%
\definecolor{currentstroke}{rgb}{1.000000,0.000000,0.000000}%
\pgfsetstrokecolor{currentstroke}%
\pgfsetdash{}{0pt}%
\pgfpathmoveto{\pgfqpoint{2.271930in}{1.581240in}}%
\pgfpathlineto{\pgfqpoint{2.422476in}{1.148244in}}%
\pgfusepath{stroke}%
\end{pgfscope}%
\begin{pgfscope}%
\pgfpathrectangle{\pgfqpoint{0.100000in}{0.212622in}}{\pgfqpoint{3.696000in}{3.696000in}}%
\pgfusepath{clip}%
\pgfsetrectcap%
\pgfsetroundjoin%
\pgfsetlinewidth{1.505625pt}%
\definecolor{currentstroke}{rgb}{1.000000,0.000000,0.000000}%
\pgfsetstrokecolor{currentstroke}%
\pgfsetdash{}{0pt}%
\pgfpathmoveto{\pgfqpoint{2.272762in}{1.578321in}}%
\pgfpathlineto{\pgfqpoint{2.422476in}{1.148244in}}%
\pgfusepath{stroke}%
\end{pgfscope}%
\begin{pgfscope}%
\pgfpathrectangle{\pgfqpoint{0.100000in}{0.212622in}}{\pgfqpoint{3.696000in}{3.696000in}}%
\pgfusepath{clip}%
\pgfsetrectcap%
\pgfsetroundjoin%
\pgfsetlinewidth{1.505625pt}%
\definecolor{currentstroke}{rgb}{1.000000,0.000000,0.000000}%
\pgfsetstrokecolor{currentstroke}%
\pgfsetdash{}{0pt}%
\pgfpathmoveto{\pgfqpoint{2.274403in}{1.577213in}}%
\pgfpathlineto{\pgfqpoint{2.422476in}{1.148244in}}%
\pgfusepath{stroke}%
\end{pgfscope}%
\begin{pgfscope}%
\pgfpathrectangle{\pgfqpoint{0.100000in}{0.212622in}}{\pgfqpoint{3.696000in}{3.696000in}}%
\pgfusepath{clip}%
\pgfsetrectcap%
\pgfsetroundjoin%
\pgfsetlinewidth{1.505625pt}%
\definecolor{currentstroke}{rgb}{1.000000,0.000000,0.000000}%
\pgfsetstrokecolor{currentstroke}%
\pgfsetdash{}{0pt}%
\pgfpathmoveto{\pgfqpoint{2.274976in}{1.576665in}}%
\pgfpathlineto{\pgfqpoint{2.413728in}{1.139766in}}%
\pgfusepath{stroke}%
\end{pgfscope}%
\begin{pgfscope}%
\pgfpathrectangle{\pgfqpoint{0.100000in}{0.212622in}}{\pgfqpoint{3.696000in}{3.696000in}}%
\pgfusepath{clip}%
\pgfsetrectcap%
\pgfsetroundjoin%
\pgfsetlinewidth{1.505625pt}%
\definecolor{currentstroke}{rgb}{1.000000,0.000000,0.000000}%
\pgfsetstrokecolor{currentstroke}%
\pgfsetdash{}{0pt}%
\pgfpathmoveto{\pgfqpoint{2.275368in}{1.575090in}}%
\pgfpathlineto{\pgfqpoint{2.413728in}{1.139766in}}%
\pgfusepath{stroke}%
\end{pgfscope}%
\begin{pgfscope}%
\pgfpathrectangle{\pgfqpoint{0.100000in}{0.212622in}}{\pgfqpoint{3.696000in}{3.696000in}}%
\pgfusepath{clip}%
\pgfsetrectcap%
\pgfsetroundjoin%
\pgfsetlinewidth{1.505625pt}%
\definecolor{currentstroke}{rgb}{1.000000,0.000000,0.000000}%
\pgfsetstrokecolor{currentstroke}%
\pgfsetdash{}{0pt}%
\pgfpathmoveto{\pgfqpoint{2.276454in}{1.569427in}}%
\pgfpathlineto{\pgfqpoint{2.413728in}{1.139766in}}%
\pgfusepath{stroke}%
\end{pgfscope}%
\begin{pgfscope}%
\pgfpathrectangle{\pgfqpoint{0.100000in}{0.212622in}}{\pgfqpoint{3.696000in}{3.696000in}}%
\pgfusepath{clip}%
\pgfsetrectcap%
\pgfsetroundjoin%
\pgfsetlinewidth{1.505625pt}%
\definecolor{currentstroke}{rgb}{1.000000,0.000000,0.000000}%
\pgfsetstrokecolor{currentstroke}%
\pgfsetdash{}{0pt}%
\pgfpathmoveto{\pgfqpoint{2.277374in}{1.568458in}}%
\pgfpathlineto{\pgfqpoint{2.413728in}{1.139766in}}%
\pgfusepath{stroke}%
\end{pgfscope}%
\begin{pgfscope}%
\pgfpathrectangle{\pgfqpoint{0.100000in}{0.212622in}}{\pgfqpoint{3.696000in}{3.696000in}}%
\pgfusepath{clip}%
\pgfsetrectcap%
\pgfsetroundjoin%
\pgfsetlinewidth{1.505625pt}%
\definecolor{currentstroke}{rgb}{1.000000,0.000000,0.000000}%
\pgfsetstrokecolor{currentstroke}%
\pgfsetdash{}{0pt}%
\pgfpathmoveto{\pgfqpoint{2.278694in}{1.568457in}}%
\pgfpathlineto{\pgfqpoint{2.413728in}{1.139766in}}%
\pgfusepath{stroke}%
\end{pgfscope}%
\begin{pgfscope}%
\pgfpathrectangle{\pgfqpoint{0.100000in}{0.212622in}}{\pgfqpoint{3.696000in}{3.696000in}}%
\pgfusepath{clip}%
\pgfsetrectcap%
\pgfsetroundjoin%
\pgfsetlinewidth{1.505625pt}%
\definecolor{currentstroke}{rgb}{1.000000,0.000000,0.000000}%
\pgfsetstrokecolor{currentstroke}%
\pgfsetdash{}{0pt}%
\pgfpathmoveto{\pgfqpoint{2.279011in}{1.567247in}}%
\pgfpathlineto{\pgfqpoint{2.404968in}{1.131276in}}%
\pgfusepath{stroke}%
\end{pgfscope}%
\begin{pgfscope}%
\pgfpathrectangle{\pgfqpoint{0.100000in}{0.212622in}}{\pgfqpoint{3.696000in}{3.696000in}}%
\pgfusepath{clip}%
\pgfsetrectcap%
\pgfsetroundjoin%
\pgfsetlinewidth{1.505625pt}%
\definecolor{currentstroke}{rgb}{1.000000,0.000000,0.000000}%
\pgfsetstrokecolor{currentstroke}%
\pgfsetdash{}{0pt}%
\pgfpathmoveto{\pgfqpoint{2.279444in}{1.566648in}}%
\pgfpathlineto{\pgfqpoint{2.404968in}{1.131276in}}%
\pgfusepath{stroke}%
\end{pgfscope}%
\begin{pgfscope}%
\pgfpathrectangle{\pgfqpoint{0.100000in}{0.212622in}}{\pgfqpoint{3.696000in}{3.696000in}}%
\pgfusepath{clip}%
\pgfsetrectcap%
\pgfsetroundjoin%
\pgfsetlinewidth{1.505625pt}%
\definecolor{currentstroke}{rgb}{1.000000,0.000000,0.000000}%
\pgfsetstrokecolor{currentstroke}%
\pgfsetdash{}{0pt}%
\pgfpathmoveto{\pgfqpoint{2.280062in}{1.565631in}}%
\pgfpathlineto{\pgfqpoint{2.404968in}{1.131276in}}%
\pgfusepath{stroke}%
\end{pgfscope}%
\begin{pgfscope}%
\pgfpathrectangle{\pgfqpoint{0.100000in}{0.212622in}}{\pgfqpoint{3.696000in}{3.696000in}}%
\pgfusepath{clip}%
\pgfsetrectcap%
\pgfsetroundjoin%
\pgfsetlinewidth{1.505625pt}%
\definecolor{currentstroke}{rgb}{1.000000,0.000000,0.000000}%
\pgfsetstrokecolor{currentstroke}%
\pgfsetdash{}{0pt}%
\pgfpathmoveto{\pgfqpoint{2.281044in}{1.565964in}}%
\pgfpathlineto{\pgfqpoint{2.404968in}{1.131276in}}%
\pgfusepath{stroke}%
\end{pgfscope}%
\begin{pgfscope}%
\pgfpathrectangle{\pgfqpoint{0.100000in}{0.212622in}}{\pgfqpoint{3.696000in}{3.696000in}}%
\pgfusepath{clip}%
\pgfsetrectcap%
\pgfsetroundjoin%
\pgfsetlinewidth{1.505625pt}%
\definecolor{currentstroke}{rgb}{1.000000,0.000000,0.000000}%
\pgfsetstrokecolor{currentstroke}%
\pgfsetdash{}{0pt}%
\pgfpathmoveto{\pgfqpoint{2.281249in}{1.565536in}}%
\pgfpathlineto{\pgfqpoint{2.404968in}{1.131276in}}%
\pgfusepath{stroke}%
\end{pgfscope}%
\begin{pgfscope}%
\pgfpathrectangle{\pgfqpoint{0.100000in}{0.212622in}}{\pgfqpoint{3.696000in}{3.696000in}}%
\pgfusepath{clip}%
\pgfsetrectcap%
\pgfsetroundjoin%
\pgfsetlinewidth{1.505625pt}%
\definecolor{currentstroke}{rgb}{1.000000,0.000000,0.000000}%
\pgfsetstrokecolor{currentstroke}%
\pgfsetdash{}{0pt}%
\pgfpathmoveto{\pgfqpoint{2.281684in}{1.563986in}}%
\pgfpathlineto{\pgfqpoint{2.404968in}{1.131276in}}%
\pgfusepath{stroke}%
\end{pgfscope}%
\begin{pgfscope}%
\pgfpathrectangle{\pgfqpoint{0.100000in}{0.212622in}}{\pgfqpoint{3.696000in}{3.696000in}}%
\pgfusepath{clip}%
\pgfsetrectcap%
\pgfsetroundjoin%
\pgfsetlinewidth{1.505625pt}%
\definecolor{currentstroke}{rgb}{1.000000,0.000000,0.000000}%
\pgfsetstrokecolor{currentstroke}%
\pgfsetdash{}{0pt}%
\pgfpathmoveto{\pgfqpoint{2.282685in}{1.561243in}}%
\pgfpathlineto{\pgfqpoint{2.404968in}{1.131276in}}%
\pgfusepath{stroke}%
\end{pgfscope}%
\begin{pgfscope}%
\pgfpathrectangle{\pgfqpoint{0.100000in}{0.212622in}}{\pgfqpoint{3.696000in}{3.696000in}}%
\pgfusepath{clip}%
\pgfsetrectcap%
\pgfsetroundjoin%
\pgfsetlinewidth{1.505625pt}%
\definecolor{currentstroke}{rgb}{1.000000,0.000000,0.000000}%
\pgfsetstrokecolor{currentstroke}%
\pgfsetdash{}{0pt}%
\pgfpathmoveto{\pgfqpoint{2.283736in}{1.559838in}}%
\pgfpathlineto{\pgfqpoint{2.404968in}{1.131276in}}%
\pgfusepath{stroke}%
\end{pgfscope}%
\begin{pgfscope}%
\pgfpathrectangle{\pgfqpoint{0.100000in}{0.212622in}}{\pgfqpoint{3.696000in}{3.696000in}}%
\pgfusepath{clip}%
\pgfsetrectcap%
\pgfsetroundjoin%
\pgfsetlinewidth{1.505625pt}%
\definecolor{currentstroke}{rgb}{1.000000,0.000000,0.000000}%
\pgfsetstrokecolor{currentstroke}%
\pgfsetdash{}{0pt}%
\pgfpathmoveto{\pgfqpoint{2.284900in}{1.557987in}}%
\pgfpathlineto{\pgfqpoint{2.396196in}{1.122775in}}%
\pgfusepath{stroke}%
\end{pgfscope}%
\begin{pgfscope}%
\pgfpathrectangle{\pgfqpoint{0.100000in}{0.212622in}}{\pgfqpoint{3.696000in}{3.696000in}}%
\pgfusepath{clip}%
\pgfsetrectcap%
\pgfsetroundjoin%
\pgfsetlinewidth{1.505625pt}%
\definecolor{currentstroke}{rgb}{1.000000,0.000000,0.000000}%
\pgfsetstrokecolor{currentstroke}%
\pgfsetdash{}{0pt}%
\pgfpathmoveto{\pgfqpoint{2.287026in}{1.555377in}}%
\pgfpathlineto{\pgfqpoint{2.396196in}{1.122775in}}%
\pgfusepath{stroke}%
\end{pgfscope}%
\begin{pgfscope}%
\pgfpathrectangle{\pgfqpoint{0.100000in}{0.212622in}}{\pgfqpoint{3.696000in}{3.696000in}}%
\pgfusepath{clip}%
\pgfsetrectcap%
\pgfsetroundjoin%
\pgfsetlinewidth{1.505625pt}%
\definecolor{currentstroke}{rgb}{1.000000,0.000000,0.000000}%
\pgfsetstrokecolor{currentstroke}%
\pgfsetdash{}{0pt}%
\pgfpathmoveto{\pgfqpoint{2.287872in}{1.553553in}}%
\pgfpathlineto{\pgfqpoint{2.396196in}{1.122775in}}%
\pgfusepath{stroke}%
\end{pgfscope}%
\begin{pgfscope}%
\pgfpathrectangle{\pgfqpoint{0.100000in}{0.212622in}}{\pgfqpoint{3.696000in}{3.696000in}}%
\pgfusepath{clip}%
\pgfsetrectcap%
\pgfsetroundjoin%
\pgfsetlinewidth{1.505625pt}%
\definecolor{currentstroke}{rgb}{1.000000,0.000000,0.000000}%
\pgfsetstrokecolor{currentstroke}%
\pgfsetdash{}{0pt}%
\pgfpathmoveto{\pgfqpoint{2.288891in}{1.549773in}}%
\pgfpathlineto{\pgfqpoint{2.387412in}{1.114262in}}%
\pgfusepath{stroke}%
\end{pgfscope}%
\begin{pgfscope}%
\pgfpathrectangle{\pgfqpoint{0.100000in}{0.212622in}}{\pgfqpoint{3.696000in}{3.696000in}}%
\pgfusepath{clip}%
\pgfsetrectcap%
\pgfsetroundjoin%
\pgfsetlinewidth{1.505625pt}%
\definecolor{currentstroke}{rgb}{1.000000,0.000000,0.000000}%
\pgfsetstrokecolor{currentstroke}%
\pgfsetdash{}{0pt}%
\pgfpathmoveto{\pgfqpoint{2.289574in}{1.549614in}}%
\pgfpathlineto{\pgfqpoint{2.387412in}{1.114262in}}%
\pgfusepath{stroke}%
\end{pgfscope}%
\begin{pgfscope}%
\pgfpathrectangle{\pgfqpoint{0.100000in}{0.212622in}}{\pgfqpoint{3.696000in}{3.696000in}}%
\pgfusepath{clip}%
\pgfsetrectcap%
\pgfsetroundjoin%
\pgfsetlinewidth{1.505625pt}%
\definecolor{currentstroke}{rgb}{1.000000,0.000000,0.000000}%
\pgfsetstrokecolor{currentstroke}%
\pgfsetdash{}{0pt}%
\pgfpathmoveto{\pgfqpoint{2.290350in}{1.548503in}}%
\pgfpathlineto{\pgfqpoint{2.387412in}{1.114262in}}%
\pgfusepath{stroke}%
\end{pgfscope}%
\begin{pgfscope}%
\pgfpathrectangle{\pgfqpoint{0.100000in}{0.212622in}}{\pgfqpoint{3.696000in}{3.696000in}}%
\pgfusepath{clip}%
\pgfsetrectcap%
\pgfsetroundjoin%
\pgfsetlinewidth{1.505625pt}%
\definecolor{currentstroke}{rgb}{1.000000,0.000000,0.000000}%
\pgfsetstrokecolor{currentstroke}%
\pgfsetdash{}{0pt}%
\pgfpathmoveto{\pgfqpoint{2.290615in}{1.547237in}}%
\pgfpathlineto{\pgfqpoint{2.387412in}{1.114262in}}%
\pgfusepath{stroke}%
\end{pgfscope}%
\begin{pgfscope}%
\pgfpathrectangle{\pgfqpoint{0.100000in}{0.212622in}}{\pgfqpoint{3.696000in}{3.696000in}}%
\pgfusepath{clip}%
\pgfsetrectcap%
\pgfsetroundjoin%
\pgfsetlinewidth{1.505625pt}%
\definecolor{currentstroke}{rgb}{1.000000,0.000000,0.000000}%
\pgfsetstrokecolor{currentstroke}%
\pgfsetdash{}{0pt}%
\pgfpathmoveto{\pgfqpoint{2.291312in}{1.544571in}}%
\pgfpathlineto{\pgfqpoint{2.387412in}{1.114262in}}%
\pgfusepath{stroke}%
\end{pgfscope}%
\begin{pgfscope}%
\pgfpathrectangle{\pgfqpoint{0.100000in}{0.212622in}}{\pgfqpoint{3.696000in}{3.696000in}}%
\pgfusepath{clip}%
\pgfsetrectcap%
\pgfsetroundjoin%
\pgfsetlinewidth{1.505625pt}%
\definecolor{currentstroke}{rgb}{1.000000,0.000000,0.000000}%
\pgfsetstrokecolor{currentstroke}%
\pgfsetdash{}{0pt}%
\pgfpathmoveto{\pgfqpoint{2.292506in}{1.544572in}}%
\pgfpathlineto{\pgfqpoint{2.387412in}{1.114262in}}%
\pgfusepath{stroke}%
\end{pgfscope}%
\begin{pgfscope}%
\pgfpathrectangle{\pgfqpoint{0.100000in}{0.212622in}}{\pgfqpoint{3.696000in}{3.696000in}}%
\pgfusepath{clip}%
\pgfsetrectcap%
\pgfsetroundjoin%
\pgfsetlinewidth{1.505625pt}%
\definecolor{currentstroke}{rgb}{1.000000,0.000000,0.000000}%
\pgfsetstrokecolor{currentstroke}%
\pgfsetdash{}{0pt}%
\pgfpathmoveto{\pgfqpoint{2.294080in}{1.543816in}}%
\pgfpathlineto{\pgfqpoint{2.378617in}{1.105737in}}%
\pgfusepath{stroke}%
\end{pgfscope}%
\begin{pgfscope}%
\pgfpathrectangle{\pgfqpoint{0.100000in}{0.212622in}}{\pgfqpoint{3.696000in}{3.696000in}}%
\pgfusepath{clip}%
\pgfsetrectcap%
\pgfsetroundjoin%
\pgfsetlinewidth{1.505625pt}%
\definecolor{currentstroke}{rgb}{1.000000,0.000000,0.000000}%
\pgfsetstrokecolor{currentstroke}%
\pgfsetdash{}{0pt}%
\pgfpathmoveto{\pgfqpoint{2.294766in}{1.540539in}}%
\pgfpathlineto{\pgfqpoint{2.378617in}{1.105737in}}%
\pgfusepath{stroke}%
\end{pgfscope}%
\begin{pgfscope}%
\pgfpathrectangle{\pgfqpoint{0.100000in}{0.212622in}}{\pgfqpoint{3.696000in}{3.696000in}}%
\pgfusepath{clip}%
\pgfsetrectcap%
\pgfsetroundjoin%
\pgfsetlinewidth{1.505625pt}%
\definecolor{currentstroke}{rgb}{1.000000,0.000000,0.000000}%
\pgfsetstrokecolor{currentstroke}%
\pgfsetdash{}{0pt}%
\pgfpathmoveto{\pgfqpoint{2.296969in}{1.537237in}}%
\pgfpathlineto{\pgfqpoint{2.369809in}{1.097201in}}%
\pgfusepath{stroke}%
\end{pgfscope}%
\begin{pgfscope}%
\pgfpathrectangle{\pgfqpoint{0.100000in}{0.212622in}}{\pgfqpoint{3.696000in}{3.696000in}}%
\pgfusepath{clip}%
\pgfsetrectcap%
\pgfsetroundjoin%
\pgfsetlinewidth{1.505625pt}%
\definecolor{currentstroke}{rgb}{1.000000,0.000000,0.000000}%
\pgfsetstrokecolor{currentstroke}%
\pgfsetdash{}{0pt}%
\pgfpathmoveto{\pgfqpoint{2.298970in}{1.532836in}}%
\pgfpathlineto{\pgfqpoint{2.369809in}{1.097201in}}%
\pgfusepath{stroke}%
\end{pgfscope}%
\begin{pgfscope}%
\pgfpathrectangle{\pgfqpoint{0.100000in}{0.212622in}}{\pgfqpoint{3.696000in}{3.696000in}}%
\pgfusepath{clip}%
\pgfsetrectcap%
\pgfsetroundjoin%
\pgfsetlinewidth{1.505625pt}%
\definecolor{currentstroke}{rgb}{1.000000,0.000000,0.000000}%
\pgfsetstrokecolor{currentstroke}%
\pgfsetdash{}{0pt}%
\pgfpathmoveto{\pgfqpoint{2.302895in}{1.535634in}}%
\pgfpathlineto{\pgfqpoint{2.369809in}{1.097201in}}%
\pgfusepath{stroke}%
\end{pgfscope}%
\begin{pgfscope}%
\pgfpathrectangle{\pgfqpoint{0.100000in}{0.212622in}}{\pgfqpoint{3.696000in}{3.696000in}}%
\pgfusepath{clip}%
\pgfsetrectcap%
\pgfsetroundjoin%
\pgfsetlinewidth{1.505625pt}%
\definecolor{currentstroke}{rgb}{1.000000,0.000000,0.000000}%
\pgfsetstrokecolor{currentstroke}%
\pgfsetdash{}{0pt}%
\pgfpathmoveto{\pgfqpoint{2.304367in}{1.531622in}}%
\pgfpathlineto{\pgfqpoint{2.360989in}{1.088653in}}%
\pgfusepath{stroke}%
\end{pgfscope}%
\begin{pgfscope}%
\pgfpathrectangle{\pgfqpoint{0.100000in}{0.212622in}}{\pgfqpoint{3.696000in}{3.696000in}}%
\pgfusepath{clip}%
\pgfsetrectcap%
\pgfsetroundjoin%
\pgfsetlinewidth{1.505625pt}%
\definecolor{currentstroke}{rgb}{1.000000,0.000000,0.000000}%
\pgfsetstrokecolor{currentstroke}%
\pgfsetdash{}{0pt}%
\pgfpathmoveto{\pgfqpoint{2.307500in}{1.527701in}}%
\pgfpathlineto{\pgfqpoint{2.352157in}{1.080094in}}%
\pgfusepath{stroke}%
\end{pgfscope}%
\begin{pgfscope}%
\pgfpathrectangle{\pgfqpoint{0.100000in}{0.212622in}}{\pgfqpoint{3.696000in}{3.696000in}}%
\pgfusepath{clip}%
\pgfsetrectcap%
\pgfsetroundjoin%
\pgfsetlinewidth{1.505625pt}%
\definecolor{currentstroke}{rgb}{1.000000,0.000000,0.000000}%
\pgfsetstrokecolor{currentstroke}%
\pgfsetdash{}{0pt}%
\pgfpathmoveto{\pgfqpoint{2.310315in}{1.514868in}}%
\pgfpathlineto{\pgfqpoint{2.352157in}{1.080094in}}%
\pgfusepath{stroke}%
\end{pgfscope}%
\begin{pgfscope}%
\pgfpathrectangle{\pgfqpoint{0.100000in}{0.212622in}}{\pgfqpoint{3.696000in}{3.696000in}}%
\pgfusepath{clip}%
\pgfsetrectcap%
\pgfsetroundjoin%
\pgfsetlinewidth{1.505625pt}%
\definecolor{currentstroke}{rgb}{1.000000,0.000000,0.000000}%
\pgfsetstrokecolor{currentstroke}%
\pgfsetdash{}{0pt}%
\pgfpathmoveto{\pgfqpoint{2.312279in}{1.513611in}}%
\pgfpathlineto{\pgfqpoint{2.352157in}{1.080094in}}%
\pgfusepath{stroke}%
\end{pgfscope}%
\begin{pgfscope}%
\pgfpathrectangle{\pgfqpoint{0.100000in}{0.212622in}}{\pgfqpoint{3.696000in}{3.696000in}}%
\pgfusepath{clip}%
\pgfsetrectcap%
\pgfsetroundjoin%
\pgfsetlinewidth{1.505625pt}%
\definecolor{currentstroke}{rgb}{1.000000,0.000000,0.000000}%
\pgfsetstrokecolor{currentstroke}%
\pgfsetdash{}{0pt}%
\pgfpathmoveto{\pgfqpoint{2.314949in}{1.512886in}}%
\pgfpathlineto{\pgfqpoint{2.343313in}{1.071522in}}%
\pgfusepath{stroke}%
\end{pgfscope}%
\begin{pgfscope}%
\pgfpathrectangle{\pgfqpoint{0.100000in}{0.212622in}}{\pgfqpoint{3.696000in}{3.696000in}}%
\pgfusepath{clip}%
\pgfsetrectcap%
\pgfsetroundjoin%
\pgfsetlinewidth{1.505625pt}%
\definecolor{currentstroke}{rgb}{1.000000,0.000000,0.000000}%
\pgfsetstrokecolor{currentstroke}%
\pgfsetdash{}{0pt}%
\pgfpathmoveto{\pgfqpoint{2.316292in}{1.509013in}}%
\pgfpathlineto{\pgfqpoint{2.343313in}{1.071522in}}%
\pgfusepath{stroke}%
\end{pgfscope}%
\begin{pgfscope}%
\pgfpathrectangle{\pgfqpoint{0.100000in}{0.212622in}}{\pgfqpoint{3.696000in}{3.696000in}}%
\pgfusepath{clip}%
\pgfsetrectcap%
\pgfsetroundjoin%
\pgfsetlinewidth{1.505625pt}%
\definecolor{currentstroke}{rgb}{1.000000,0.000000,0.000000}%
\pgfsetstrokecolor{currentstroke}%
\pgfsetdash{}{0pt}%
\pgfpathmoveto{\pgfqpoint{2.317302in}{1.507790in}}%
\pgfpathlineto{\pgfqpoint{2.343313in}{1.071522in}}%
\pgfusepath{stroke}%
\end{pgfscope}%
\begin{pgfscope}%
\pgfpathrectangle{\pgfqpoint{0.100000in}{0.212622in}}{\pgfqpoint{3.696000in}{3.696000in}}%
\pgfusepath{clip}%
\pgfsetrectcap%
\pgfsetroundjoin%
\pgfsetlinewidth{1.505625pt}%
\definecolor{currentstroke}{rgb}{1.000000,0.000000,0.000000}%
\pgfsetstrokecolor{currentstroke}%
\pgfsetdash{}{0pt}%
\pgfpathmoveto{\pgfqpoint{2.317976in}{1.505542in}}%
\pgfpathlineto{\pgfqpoint{2.334458in}{1.062940in}}%
\pgfusepath{stroke}%
\end{pgfscope}%
\begin{pgfscope}%
\pgfpathrectangle{\pgfqpoint{0.100000in}{0.212622in}}{\pgfqpoint{3.696000in}{3.696000in}}%
\pgfusepath{clip}%
\pgfsetrectcap%
\pgfsetroundjoin%
\pgfsetlinewidth{1.505625pt}%
\definecolor{currentstroke}{rgb}{1.000000,0.000000,0.000000}%
\pgfsetstrokecolor{currentstroke}%
\pgfsetdash{}{0pt}%
\pgfpathmoveto{\pgfqpoint{2.318333in}{1.505296in}}%
\pgfpathlineto{\pgfqpoint{2.334458in}{1.062940in}}%
\pgfusepath{stroke}%
\end{pgfscope}%
\begin{pgfscope}%
\pgfpathrectangle{\pgfqpoint{0.100000in}{0.212622in}}{\pgfqpoint{3.696000in}{3.696000in}}%
\pgfusepath{clip}%
\pgfsetrectcap%
\pgfsetroundjoin%
\pgfsetlinewidth{1.505625pt}%
\definecolor{currentstroke}{rgb}{1.000000,0.000000,0.000000}%
\pgfsetstrokecolor{currentstroke}%
\pgfsetdash{}{0pt}%
\pgfpathmoveto{\pgfqpoint{2.319073in}{1.504892in}}%
\pgfpathlineto{\pgfqpoint{2.334458in}{1.062940in}}%
\pgfusepath{stroke}%
\end{pgfscope}%
\begin{pgfscope}%
\pgfpathrectangle{\pgfqpoint{0.100000in}{0.212622in}}{\pgfqpoint{3.696000in}{3.696000in}}%
\pgfusepath{clip}%
\pgfsetrectcap%
\pgfsetroundjoin%
\pgfsetlinewidth{1.505625pt}%
\definecolor{currentstroke}{rgb}{1.000000,0.000000,0.000000}%
\pgfsetstrokecolor{currentstroke}%
\pgfsetdash{}{0pt}%
\pgfpathmoveto{\pgfqpoint{2.319857in}{1.502298in}}%
\pgfpathlineto{\pgfqpoint{2.334458in}{1.062940in}}%
\pgfusepath{stroke}%
\end{pgfscope}%
\begin{pgfscope}%
\pgfpathrectangle{\pgfqpoint{0.100000in}{0.212622in}}{\pgfqpoint{3.696000in}{3.696000in}}%
\pgfusepath{clip}%
\pgfsetrectcap%
\pgfsetroundjoin%
\pgfsetlinewidth{1.505625pt}%
\definecolor{currentstroke}{rgb}{1.000000,0.000000,0.000000}%
\pgfsetstrokecolor{currentstroke}%
\pgfsetdash{}{0pt}%
\pgfpathmoveto{\pgfqpoint{2.321087in}{1.498715in}}%
\pgfpathlineto{\pgfqpoint{2.334458in}{1.062940in}}%
\pgfusepath{stroke}%
\end{pgfscope}%
\begin{pgfscope}%
\pgfpathrectangle{\pgfqpoint{0.100000in}{0.212622in}}{\pgfqpoint{3.696000in}{3.696000in}}%
\pgfusepath{clip}%
\pgfsetrectcap%
\pgfsetroundjoin%
\pgfsetlinewidth{1.505625pt}%
\definecolor{currentstroke}{rgb}{1.000000,0.000000,0.000000}%
\pgfsetstrokecolor{currentstroke}%
\pgfsetdash{}{0pt}%
\pgfpathmoveto{\pgfqpoint{2.321834in}{1.497247in}}%
\pgfpathlineto{\pgfqpoint{2.325590in}{1.054345in}}%
\pgfusepath{stroke}%
\end{pgfscope}%
\begin{pgfscope}%
\pgfpathrectangle{\pgfqpoint{0.100000in}{0.212622in}}{\pgfqpoint{3.696000in}{3.696000in}}%
\pgfusepath{clip}%
\pgfsetrectcap%
\pgfsetroundjoin%
\pgfsetlinewidth{1.505625pt}%
\definecolor{currentstroke}{rgb}{1.000000,0.000000,0.000000}%
\pgfsetstrokecolor{currentstroke}%
\pgfsetdash{}{0pt}%
\pgfpathmoveto{\pgfqpoint{2.322369in}{1.496529in}}%
\pgfpathlineto{\pgfqpoint{2.325590in}{1.054345in}}%
\pgfusepath{stroke}%
\end{pgfscope}%
\begin{pgfscope}%
\pgfpathrectangle{\pgfqpoint{0.100000in}{0.212622in}}{\pgfqpoint{3.696000in}{3.696000in}}%
\pgfusepath{clip}%
\pgfsetrectcap%
\pgfsetroundjoin%
\pgfsetlinewidth{1.505625pt}%
\definecolor{currentstroke}{rgb}{1.000000,0.000000,0.000000}%
\pgfsetstrokecolor{currentstroke}%
\pgfsetdash{}{0pt}%
\pgfpathmoveto{\pgfqpoint{2.322606in}{1.496024in}}%
\pgfpathlineto{\pgfqpoint{2.325590in}{1.054345in}}%
\pgfusepath{stroke}%
\end{pgfscope}%
\begin{pgfscope}%
\pgfpathrectangle{\pgfqpoint{0.100000in}{0.212622in}}{\pgfqpoint{3.696000in}{3.696000in}}%
\pgfusepath{clip}%
\pgfsetrectcap%
\pgfsetroundjoin%
\pgfsetlinewidth{1.505625pt}%
\definecolor{currentstroke}{rgb}{1.000000,0.000000,0.000000}%
\pgfsetstrokecolor{currentstroke}%
\pgfsetdash{}{0pt}%
\pgfpathmoveto{\pgfqpoint{2.323297in}{1.495421in}}%
\pgfpathlineto{\pgfqpoint{2.325590in}{1.054345in}}%
\pgfusepath{stroke}%
\end{pgfscope}%
\begin{pgfscope}%
\pgfpathrectangle{\pgfqpoint{0.100000in}{0.212622in}}{\pgfqpoint{3.696000in}{3.696000in}}%
\pgfusepath{clip}%
\pgfsetrectcap%
\pgfsetroundjoin%
\pgfsetlinewidth{1.505625pt}%
\definecolor{currentstroke}{rgb}{1.000000,0.000000,0.000000}%
\pgfsetstrokecolor{currentstroke}%
\pgfsetdash{}{0pt}%
\pgfpathmoveto{\pgfqpoint{2.323487in}{1.494504in}}%
\pgfpathlineto{\pgfqpoint{2.325590in}{1.054345in}}%
\pgfusepath{stroke}%
\end{pgfscope}%
\begin{pgfscope}%
\pgfpathrectangle{\pgfqpoint{0.100000in}{0.212622in}}{\pgfqpoint{3.696000in}{3.696000in}}%
\pgfusepath{clip}%
\pgfsetrectcap%
\pgfsetroundjoin%
\pgfsetlinewidth{1.505625pt}%
\definecolor{currentstroke}{rgb}{1.000000,0.000000,0.000000}%
\pgfsetstrokecolor{currentstroke}%
\pgfsetdash{}{0pt}%
\pgfpathmoveto{\pgfqpoint{2.324079in}{1.494417in}}%
\pgfpathlineto{\pgfqpoint{2.325590in}{1.054345in}}%
\pgfusepath{stroke}%
\end{pgfscope}%
\begin{pgfscope}%
\pgfpathrectangle{\pgfqpoint{0.100000in}{0.212622in}}{\pgfqpoint{3.696000in}{3.696000in}}%
\pgfusepath{clip}%
\pgfsetrectcap%
\pgfsetroundjoin%
\pgfsetlinewidth{1.505625pt}%
\definecolor{currentstroke}{rgb}{1.000000,0.000000,0.000000}%
\pgfsetstrokecolor{currentstroke}%
\pgfsetdash{}{0pt}%
\pgfpathmoveto{\pgfqpoint{2.324347in}{1.494153in}}%
\pgfpathlineto{\pgfqpoint{2.325590in}{1.054345in}}%
\pgfusepath{stroke}%
\end{pgfscope}%
\begin{pgfscope}%
\pgfpathrectangle{\pgfqpoint{0.100000in}{0.212622in}}{\pgfqpoint{3.696000in}{3.696000in}}%
\pgfusepath{clip}%
\pgfsetrectcap%
\pgfsetroundjoin%
\pgfsetlinewidth{1.505625pt}%
\definecolor{currentstroke}{rgb}{1.000000,0.000000,0.000000}%
\pgfsetstrokecolor{currentstroke}%
\pgfsetdash{}{0pt}%
\pgfpathmoveto{\pgfqpoint{2.324409in}{1.493758in}}%
\pgfpathlineto{\pgfqpoint{2.325590in}{1.054345in}}%
\pgfusepath{stroke}%
\end{pgfscope}%
\begin{pgfscope}%
\pgfpathrectangle{\pgfqpoint{0.100000in}{0.212622in}}{\pgfqpoint{3.696000in}{3.696000in}}%
\pgfusepath{clip}%
\pgfsetrectcap%
\pgfsetroundjoin%
\pgfsetlinewidth{1.505625pt}%
\definecolor{currentstroke}{rgb}{1.000000,0.000000,0.000000}%
\pgfsetstrokecolor{currentstroke}%
\pgfsetdash{}{0pt}%
\pgfpathmoveto{\pgfqpoint{2.324998in}{1.491814in}}%
\pgfpathlineto{\pgfqpoint{2.325590in}{1.054345in}}%
\pgfusepath{stroke}%
\end{pgfscope}%
\begin{pgfscope}%
\pgfpathrectangle{\pgfqpoint{0.100000in}{0.212622in}}{\pgfqpoint{3.696000in}{3.696000in}}%
\pgfusepath{clip}%
\pgfsetrectcap%
\pgfsetroundjoin%
\pgfsetlinewidth{1.505625pt}%
\definecolor{currentstroke}{rgb}{1.000000,0.000000,0.000000}%
\pgfsetstrokecolor{currentstroke}%
\pgfsetdash{}{0pt}%
\pgfpathmoveto{\pgfqpoint{2.325615in}{1.490931in}}%
\pgfpathlineto{\pgfqpoint{2.325590in}{1.054345in}}%
\pgfusepath{stroke}%
\end{pgfscope}%
\begin{pgfscope}%
\pgfpathrectangle{\pgfqpoint{0.100000in}{0.212622in}}{\pgfqpoint{3.696000in}{3.696000in}}%
\pgfusepath{clip}%
\pgfsetrectcap%
\pgfsetroundjoin%
\pgfsetlinewidth{1.505625pt}%
\definecolor{currentstroke}{rgb}{1.000000,0.000000,0.000000}%
\pgfsetstrokecolor{currentstroke}%
\pgfsetdash{}{0pt}%
\pgfpathmoveto{\pgfqpoint{2.326783in}{1.489833in}}%
\pgfpathlineto{\pgfqpoint{2.325590in}{1.054345in}}%
\pgfusepath{stroke}%
\end{pgfscope}%
\begin{pgfscope}%
\pgfpathrectangle{\pgfqpoint{0.100000in}{0.212622in}}{\pgfqpoint{3.696000in}{3.696000in}}%
\pgfusepath{clip}%
\pgfsetrectcap%
\pgfsetroundjoin%
\pgfsetlinewidth{1.505625pt}%
\definecolor{currentstroke}{rgb}{1.000000,0.000000,0.000000}%
\pgfsetstrokecolor{currentstroke}%
\pgfsetdash{}{0pt}%
\pgfpathmoveto{\pgfqpoint{2.327377in}{1.488676in}}%
\pgfpathlineto{\pgfqpoint{2.316710in}{1.045739in}}%
\pgfusepath{stroke}%
\end{pgfscope}%
\begin{pgfscope}%
\pgfpathrectangle{\pgfqpoint{0.100000in}{0.212622in}}{\pgfqpoint{3.696000in}{3.696000in}}%
\pgfusepath{clip}%
\pgfsetrectcap%
\pgfsetroundjoin%
\pgfsetlinewidth{1.505625pt}%
\definecolor{currentstroke}{rgb}{1.000000,0.000000,0.000000}%
\pgfsetstrokecolor{currentstroke}%
\pgfsetdash{}{0pt}%
\pgfpathmoveto{\pgfqpoint{2.328035in}{1.487575in}}%
\pgfpathlineto{\pgfqpoint{2.316710in}{1.045739in}}%
\pgfusepath{stroke}%
\end{pgfscope}%
\begin{pgfscope}%
\pgfpathrectangle{\pgfqpoint{0.100000in}{0.212622in}}{\pgfqpoint{3.696000in}{3.696000in}}%
\pgfusepath{clip}%
\pgfsetrectcap%
\pgfsetroundjoin%
\pgfsetlinewidth{1.505625pt}%
\definecolor{currentstroke}{rgb}{1.000000,0.000000,0.000000}%
\pgfsetstrokecolor{currentstroke}%
\pgfsetdash{}{0pt}%
\pgfpathmoveto{\pgfqpoint{2.329131in}{1.485429in}}%
\pgfpathlineto{\pgfqpoint{2.316710in}{1.045739in}}%
\pgfusepath{stroke}%
\end{pgfscope}%
\begin{pgfscope}%
\pgfpathrectangle{\pgfqpoint{0.100000in}{0.212622in}}{\pgfqpoint{3.696000in}{3.696000in}}%
\pgfusepath{clip}%
\pgfsetrectcap%
\pgfsetroundjoin%
\pgfsetlinewidth{1.505625pt}%
\definecolor{currentstroke}{rgb}{1.000000,0.000000,0.000000}%
\pgfsetstrokecolor{currentstroke}%
\pgfsetdash{}{0pt}%
\pgfpathmoveto{\pgfqpoint{2.329724in}{1.485477in}}%
\pgfpathlineto{\pgfqpoint{2.316710in}{1.045739in}}%
\pgfusepath{stroke}%
\end{pgfscope}%
\begin{pgfscope}%
\pgfpathrectangle{\pgfqpoint{0.100000in}{0.212622in}}{\pgfqpoint{3.696000in}{3.696000in}}%
\pgfusepath{clip}%
\pgfsetrectcap%
\pgfsetroundjoin%
\pgfsetlinewidth{1.505625pt}%
\definecolor{currentstroke}{rgb}{1.000000,0.000000,0.000000}%
\pgfsetstrokecolor{currentstroke}%
\pgfsetdash{}{0pt}%
\pgfpathmoveto{\pgfqpoint{2.330262in}{1.485284in}}%
\pgfpathlineto{\pgfqpoint{2.316710in}{1.045739in}}%
\pgfusepath{stroke}%
\end{pgfscope}%
\begin{pgfscope}%
\pgfpathrectangle{\pgfqpoint{0.100000in}{0.212622in}}{\pgfqpoint{3.696000in}{3.696000in}}%
\pgfusepath{clip}%
\pgfsetrectcap%
\pgfsetroundjoin%
\pgfsetlinewidth{1.505625pt}%
\definecolor{currentstroke}{rgb}{1.000000,0.000000,0.000000}%
\pgfsetstrokecolor{currentstroke}%
\pgfsetdash{}{0pt}%
\pgfpathmoveto{\pgfqpoint{2.330660in}{1.484206in}}%
\pgfpathlineto{\pgfqpoint{2.316710in}{1.045739in}}%
\pgfusepath{stroke}%
\end{pgfscope}%
\begin{pgfscope}%
\pgfpathrectangle{\pgfqpoint{0.100000in}{0.212622in}}{\pgfqpoint{3.696000in}{3.696000in}}%
\pgfusepath{clip}%
\pgfsetrectcap%
\pgfsetroundjoin%
\pgfsetlinewidth{1.505625pt}%
\definecolor{currentstroke}{rgb}{1.000000,0.000000,0.000000}%
\pgfsetstrokecolor{currentstroke}%
\pgfsetdash{}{0pt}%
\pgfpathmoveto{\pgfqpoint{2.331350in}{1.482224in}}%
\pgfpathlineto{\pgfqpoint{2.307817in}{1.037121in}}%
\pgfusepath{stroke}%
\end{pgfscope}%
\begin{pgfscope}%
\pgfpathrectangle{\pgfqpoint{0.100000in}{0.212622in}}{\pgfqpoint{3.696000in}{3.696000in}}%
\pgfusepath{clip}%
\pgfsetrectcap%
\pgfsetroundjoin%
\pgfsetlinewidth{1.505625pt}%
\definecolor{currentstroke}{rgb}{1.000000,0.000000,0.000000}%
\pgfsetstrokecolor{currentstroke}%
\pgfsetdash{}{0pt}%
\pgfpathmoveto{\pgfqpoint{2.332179in}{1.481704in}}%
\pgfpathlineto{\pgfqpoint{2.307817in}{1.037121in}}%
\pgfusepath{stroke}%
\end{pgfscope}%
\begin{pgfscope}%
\pgfpathrectangle{\pgfqpoint{0.100000in}{0.212622in}}{\pgfqpoint{3.696000in}{3.696000in}}%
\pgfusepath{clip}%
\pgfsetrectcap%
\pgfsetroundjoin%
\pgfsetlinewidth{1.505625pt}%
\definecolor{currentstroke}{rgb}{1.000000,0.000000,0.000000}%
\pgfsetstrokecolor{currentstroke}%
\pgfsetdash{}{0pt}%
\pgfpathmoveto{\pgfqpoint{2.334592in}{1.481038in}}%
\pgfpathlineto{\pgfqpoint{2.307817in}{1.037121in}}%
\pgfusepath{stroke}%
\end{pgfscope}%
\begin{pgfscope}%
\pgfpathrectangle{\pgfqpoint{0.100000in}{0.212622in}}{\pgfqpoint{3.696000in}{3.696000in}}%
\pgfusepath{clip}%
\pgfsetrectcap%
\pgfsetroundjoin%
\pgfsetlinewidth{1.505625pt}%
\definecolor{currentstroke}{rgb}{1.000000,0.000000,0.000000}%
\pgfsetstrokecolor{currentstroke}%
\pgfsetdash{}{0pt}%
\pgfpathmoveto{\pgfqpoint{2.335717in}{1.477894in}}%
\pgfpathlineto{\pgfqpoint{2.307817in}{1.037121in}}%
\pgfusepath{stroke}%
\end{pgfscope}%
\begin{pgfscope}%
\pgfpathrectangle{\pgfqpoint{0.100000in}{0.212622in}}{\pgfqpoint{3.696000in}{3.696000in}}%
\pgfusepath{clip}%
\pgfsetrectcap%
\pgfsetroundjoin%
\pgfsetlinewidth{1.505625pt}%
\definecolor{currentstroke}{rgb}{1.000000,0.000000,0.000000}%
\pgfsetstrokecolor{currentstroke}%
\pgfsetdash{}{0pt}%
\pgfpathmoveto{\pgfqpoint{2.337674in}{1.473274in}}%
\pgfpathlineto{\pgfqpoint{2.298913in}{1.028491in}}%
\pgfusepath{stroke}%
\end{pgfscope}%
\begin{pgfscope}%
\pgfpathrectangle{\pgfqpoint{0.100000in}{0.212622in}}{\pgfqpoint{3.696000in}{3.696000in}}%
\pgfusepath{clip}%
\pgfsetrectcap%
\pgfsetroundjoin%
\pgfsetlinewidth{1.505625pt}%
\definecolor{currentstroke}{rgb}{1.000000,0.000000,0.000000}%
\pgfsetstrokecolor{currentstroke}%
\pgfsetdash{}{0pt}%
\pgfpathmoveto{\pgfqpoint{2.339564in}{1.469634in}}%
\pgfpathlineto{\pgfqpoint{2.298913in}{1.028491in}}%
\pgfusepath{stroke}%
\end{pgfscope}%
\begin{pgfscope}%
\pgfpathrectangle{\pgfqpoint{0.100000in}{0.212622in}}{\pgfqpoint{3.696000in}{3.696000in}}%
\pgfusepath{clip}%
\pgfsetrectcap%
\pgfsetroundjoin%
\pgfsetlinewidth{1.505625pt}%
\definecolor{currentstroke}{rgb}{1.000000,0.000000,0.000000}%
\pgfsetstrokecolor{currentstroke}%
\pgfsetdash{}{0pt}%
\pgfpathmoveto{\pgfqpoint{2.341170in}{1.469174in}}%
\pgfpathlineto{\pgfqpoint{2.298913in}{1.028491in}}%
\pgfusepath{stroke}%
\end{pgfscope}%
\begin{pgfscope}%
\pgfpathrectangle{\pgfqpoint{0.100000in}{0.212622in}}{\pgfqpoint{3.696000in}{3.696000in}}%
\pgfusepath{clip}%
\pgfsetrectcap%
\pgfsetroundjoin%
\pgfsetlinewidth{1.505625pt}%
\definecolor{currentstroke}{rgb}{1.000000,0.000000,0.000000}%
\pgfsetstrokecolor{currentstroke}%
\pgfsetdash{}{0pt}%
\pgfpathmoveto{\pgfqpoint{2.342353in}{1.466997in}}%
\pgfpathlineto{\pgfqpoint{2.298913in}{1.028491in}}%
\pgfusepath{stroke}%
\end{pgfscope}%
\begin{pgfscope}%
\pgfpathrectangle{\pgfqpoint{0.100000in}{0.212622in}}{\pgfqpoint{3.696000in}{3.696000in}}%
\pgfusepath{clip}%
\pgfsetrectcap%
\pgfsetroundjoin%
\pgfsetlinewidth{1.505625pt}%
\definecolor{currentstroke}{rgb}{1.000000,0.000000,0.000000}%
\pgfsetstrokecolor{currentstroke}%
\pgfsetdash{}{0pt}%
\pgfpathmoveto{\pgfqpoint{2.343943in}{1.463655in}}%
\pgfpathlineto{\pgfqpoint{2.289996in}{1.019849in}}%
\pgfusepath{stroke}%
\end{pgfscope}%
\begin{pgfscope}%
\pgfpathrectangle{\pgfqpoint{0.100000in}{0.212622in}}{\pgfqpoint{3.696000in}{3.696000in}}%
\pgfusepath{clip}%
\pgfsetrectcap%
\pgfsetroundjoin%
\pgfsetlinewidth{1.505625pt}%
\definecolor{currentstroke}{rgb}{1.000000,0.000000,0.000000}%
\pgfsetstrokecolor{currentstroke}%
\pgfsetdash{}{0pt}%
\pgfpathmoveto{\pgfqpoint{2.345791in}{1.457677in}}%
\pgfpathlineto{\pgfqpoint{2.289996in}{1.019849in}}%
\pgfusepath{stroke}%
\end{pgfscope}%
\begin{pgfscope}%
\pgfpathrectangle{\pgfqpoint{0.100000in}{0.212622in}}{\pgfqpoint{3.696000in}{3.696000in}}%
\pgfusepath{clip}%
\pgfsetrectcap%
\pgfsetroundjoin%
\pgfsetlinewidth{1.505625pt}%
\definecolor{currentstroke}{rgb}{1.000000,0.000000,0.000000}%
\pgfsetstrokecolor{currentstroke}%
\pgfsetdash{}{0pt}%
\pgfpathmoveto{\pgfqpoint{2.347844in}{1.455593in}}%
\pgfpathlineto{\pgfqpoint{2.281067in}{1.011195in}}%
\pgfusepath{stroke}%
\end{pgfscope}%
\begin{pgfscope}%
\pgfpathrectangle{\pgfqpoint{0.100000in}{0.212622in}}{\pgfqpoint{3.696000in}{3.696000in}}%
\pgfusepath{clip}%
\pgfsetrectcap%
\pgfsetroundjoin%
\pgfsetlinewidth{1.505625pt}%
\definecolor{currentstroke}{rgb}{1.000000,0.000000,0.000000}%
\pgfsetstrokecolor{currentstroke}%
\pgfsetdash{}{0pt}%
\pgfpathmoveto{\pgfqpoint{2.351536in}{1.453094in}}%
\pgfpathlineto{\pgfqpoint{2.281067in}{1.011195in}}%
\pgfusepath{stroke}%
\end{pgfscope}%
\begin{pgfscope}%
\pgfpathrectangle{\pgfqpoint{0.100000in}{0.212622in}}{\pgfqpoint{3.696000in}{3.696000in}}%
\pgfusepath{clip}%
\pgfsetrectcap%
\pgfsetroundjoin%
\pgfsetlinewidth{1.505625pt}%
\definecolor{currentstroke}{rgb}{1.000000,0.000000,0.000000}%
\pgfsetstrokecolor{currentstroke}%
\pgfsetdash{}{0pt}%
\pgfpathmoveto{\pgfqpoint{2.354394in}{1.450336in}}%
\pgfpathlineto{\pgfqpoint{2.272126in}{1.002530in}}%
\pgfusepath{stroke}%
\end{pgfscope}%
\begin{pgfscope}%
\pgfpathrectangle{\pgfqpoint{0.100000in}{0.212622in}}{\pgfqpoint{3.696000in}{3.696000in}}%
\pgfusepath{clip}%
\pgfsetrectcap%
\pgfsetroundjoin%
\pgfsetlinewidth{1.505625pt}%
\definecolor{currentstroke}{rgb}{1.000000,0.000000,0.000000}%
\pgfsetstrokecolor{currentstroke}%
\pgfsetdash{}{0pt}%
\pgfpathmoveto{\pgfqpoint{2.357660in}{1.444174in}}%
\pgfpathlineto{\pgfqpoint{2.272126in}{1.002530in}}%
\pgfusepath{stroke}%
\end{pgfscope}%
\begin{pgfscope}%
\pgfpathrectangle{\pgfqpoint{0.100000in}{0.212622in}}{\pgfqpoint{3.696000in}{3.696000in}}%
\pgfusepath{clip}%
\pgfsetrectcap%
\pgfsetroundjoin%
\pgfsetlinewidth{1.505625pt}%
\definecolor{currentstroke}{rgb}{1.000000,0.000000,0.000000}%
\pgfsetstrokecolor{currentstroke}%
\pgfsetdash{}{0pt}%
\pgfpathmoveto{\pgfqpoint{2.359937in}{1.432943in}}%
\pgfpathlineto{\pgfqpoint{2.272126in}{1.002530in}}%
\pgfusepath{stroke}%
\end{pgfscope}%
\begin{pgfscope}%
\pgfpathrectangle{\pgfqpoint{0.100000in}{0.212622in}}{\pgfqpoint{3.696000in}{3.696000in}}%
\pgfusepath{clip}%
\pgfsetrectcap%
\pgfsetroundjoin%
\pgfsetlinewidth{1.505625pt}%
\definecolor{currentstroke}{rgb}{1.000000,0.000000,0.000000}%
\pgfsetstrokecolor{currentstroke}%
\pgfsetdash{}{0pt}%
\pgfpathmoveto{\pgfqpoint{2.364362in}{1.427543in}}%
\pgfpathlineto{\pgfqpoint{2.272126in}{1.002530in}}%
\pgfusepath{stroke}%
\end{pgfscope}%
\begin{pgfscope}%
\pgfpathrectangle{\pgfqpoint{0.100000in}{0.212622in}}{\pgfqpoint{3.696000in}{3.696000in}}%
\pgfusepath{clip}%
\pgfsetrectcap%
\pgfsetroundjoin%
\pgfsetlinewidth{1.505625pt}%
\definecolor{currentstroke}{rgb}{1.000000,0.000000,0.000000}%
\pgfsetstrokecolor{currentstroke}%
\pgfsetdash{}{0pt}%
\pgfpathmoveto{\pgfqpoint{2.368911in}{1.422276in}}%
\pgfpathlineto{\pgfqpoint{2.272126in}{1.002530in}}%
\pgfusepath{stroke}%
\end{pgfscope}%
\begin{pgfscope}%
\pgfpathrectangle{\pgfqpoint{0.100000in}{0.212622in}}{\pgfqpoint{3.696000in}{3.696000in}}%
\pgfusepath{clip}%
\pgfsetrectcap%
\pgfsetroundjoin%
\pgfsetlinewidth{1.505625pt}%
\definecolor{currentstroke}{rgb}{1.000000,0.000000,0.000000}%
\pgfsetstrokecolor{currentstroke}%
\pgfsetdash{}{0pt}%
\pgfpathmoveto{\pgfqpoint{2.373013in}{1.420679in}}%
\pgfpathlineto{\pgfqpoint{2.272126in}{1.002530in}}%
\pgfusepath{stroke}%
\end{pgfscope}%
\begin{pgfscope}%
\pgfpathrectangle{\pgfqpoint{0.100000in}{0.212622in}}{\pgfqpoint{3.696000in}{3.696000in}}%
\pgfusepath{clip}%
\pgfsetrectcap%
\pgfsetroundjoin%
\pgfsetlinewidth{1.505625pt}%
\definecolor{currentstroke}{rgb}{1.000000,0.000000,0.000000}%
\pgfsetstrokecolor{currentstroke}%
\pgfsetdash{}{0pt}%
\pgfpathmoveto{\pgfqpoint{2.376083in}{1.409886in}}%
\pgfpathlineto{\pgfqpoint{2.272126in}{1.002530in}}%
\pgfusepath{stroke}%
\end{pgfscope}%
\begin{pgfscope}%
\pgfpathrectangle{\pgfqpoint{0.100000in}{0.212622in}}{\pgfqpoint{3.696000in}{3.696000in}}%
\pgfusepath{clip}%
\pgfsetrectcap%
\pgfsetroundjoin%
\pgfsetlinewidth{1.505625pt}%
\definecolor{currentstroke}{rgb}{1.000000,0.000000,0.000000}%
\pgfsetstrokecolor{currentstroke}%
\pgfsetdash{}{0pt}%
\pgfpathmoveto{\pgfqpoint{2.380541in}{1.400068in}}%
\pgfpathlineto{\pgfqpoint{2.272126in}{1.002530in}}%
\pgfusepath{stroke}%
\end{pgfscope}%
\begin{pgfscope}%
\pgfpathrectangle{\pgfqpoint{0.100000in}{0.212622in}}{\pgfqpoint{3.696000in}{3.696000in}}%
\pgfusepath{clip}%
\pgfsetrectcap%
\pgfsetroundjoin%
\pgfsetlinewidth{1.505625pt}%
\definecolor{currentstroke}{rgb}{1.000000,0.000000,0.000000}%
\pgfsetstrokecolor{currentstroke}%
\pgfsetdash{}{0pt}%
\pgfpathmoveto{\pgfqpoint{2.385647in}{1.385550in}}%
\pgfpathlineto{\pgfqpoint{2.272126in}{1.002530in}}%
\pgfusepath{stroke}%
\end{pgfscope}%
\begin{pgfscope}%
\pgfpathrectangle{\pgfqpoint{0.100000in}{0.212622in}}{\pgfqpoint{3.696000in}{3.696000in}}%
\pgfusepath{clip}%
\pgfsetrectcap%
\pgfsetroundjoin%
\pgfsetlinewidth{1.505625pt}%
\definecolor{currentstroke}{rgb}{1.000000,0.000000,0.000000}%
\pgfsetstrokecolor{currentstroke}%
\pgfsetdash{}{0pt}%
\pgfpathmoveto{\pgfqpoint{2.388194in}{1.379461in}}%
\pgfpathlineto{\pgfqpoint{2.272126in}{1.002530in}}%
\pgfusepath{stroke}%
\end{pgfscope}%
\begin{pgfscope}%
\pgfpathrectangle{\pgfqpoint{0.100000in}{0.212622in}}{\pgfqpoint{3.696000in}{3.696000in}}%
\pgfusepath{clip}%
\pgfsetrectcap%
\pgfsetroundjoin%
\pgfsetlinewidth{1.505625pt}%
\definecolor{currentstroke}{rgb}{1.000000,0.000000,0.000000}%
\pgfsetstrokecolor{currentstroke}%
\pgfsetdash{}{0pt}%
\pgfpathmoveto{\pgfqpoint{2.391498in}{1.373881in}}%
\pgfpathlineto{\pgfqpoint{2.272126in}{1.002530in}}%
\pgfusepath{stroke}%
\end{pgfscope}%
\begin{pgfscope}%
\pgfpathrectangle{\pgfqpoint{0.100000in}{0.212622in}}{\pgfqpoint{3.696000in}{3.696000in}}%
\pgfusepath{clip}%
\pgfsetrectcap%
\pgfsetroundjoin%
\pgfsetlinewidth{1.505625pt}%
\definecolor{currentstroke}{rgb}{1.000000,0.000000,0.000000}%
\pgfsetstrokecolor{currentstroke}%
\pgfsetdash{}{0pt}%
\pgfpathmoveto{\pgfqpoint{2.394064in}{1.368894in}}%
\pgfpathlineto{\pgfqpoint{2.272126in}{1.002530in}}%
\pgfusepath{stroke}%
\end{pgfscope}%
\begin{pgfscope}%
\pgfpathrectangle{\pgfqpoint{0.100000in}{0.212622in}}{\pgfqpoint{3.696000in}{3.696000in}}%
\pgfusepath{clip}%
\pgfsetrectcap%
\pgfsetroundjoin%
\pgfsetlinewidth{1.505625pt}%
\definecolor{currentstroke}{rgb}{1.000000,0.000000,0.000000}%
\pgfsetstrokecolor{currentstroke}%
\pgfsetdash{}{0pt}%
\pgfpathmoveto{\pgfqpoint{2.396921in}{1.362350in}}%
\pgfpathlineto{\pgfqpoint{2.272126in}{1.002530in}}%
\pgfusepath{stroke}%
\end{pgfscope}%
\begin{pgfscope}%
\pgfpathrectangle{\pgfqpoint{0.100000in}{0.212622in}}{\pgfqpoint{3.696000in}{3.696000in}}%
\pgfusepath{clip}%
\pgfsetrectcap%
\pgfsetroundjoin%
\pgfsetlinewidth{1.505625pt}%
\definecolor{currentstroke}{rgb}{1.000000,0.000000,0.000000}%
\pgfsetstrokecolor{currentstroke}%
\pgfsetdash{}{0pt}%
\pgfpathmoveto{\pgfqpoint{2.400941in}{1.358628in}}%
\pgfpathlineto{\pgfqpoint{2.272126in}{1.002530in}}%
\pgfusepath{stroke}%
\end{pgfscope}%
\begin{pgfscope}%
\pgfpathrectangle{\pgfqpoint{0.100000in}{0.212622in}}{\pgfqpoint{3.696000in}{3.696000in}}%
\pgfusepath{clip}%
\pgfsetrectcap%
\pgfsetroundjoin%
\pgfsetlinewidth{1.505625pt}%
\definecolor{currentstroke}{rgb}{1.000000,0.000000,0.000000}%
\pgfsetstrokecolor{currentstroke}%
\pgfsetdash{}{0pt}%
\pgfpathmoveto{\pgfqpoint{2.404185in}{1.353086in}}%
\pgfpathlineto{\pgfqpoint{2.272126in}{1.002530in}}%
\pgfusepath{stroke}%
\end{pgfscope}%
\begin{pgfscope}%
\pgfpathrectangle{\pgfqpoint{0.100000in}{0.212622in}}{\pgfqpoint{3.696000in}{3.696000in}}%
\pgfusepath{clip}%
\pgfsetrectcap%
\pgfsetroundjoin%
\pgfsetlinewidth{1.505625pt}%
\definecolor{currentstroke}{rgb}{1.000000,0.000000,0.000000}%
\pgfsetstrokecolor{currentstroke}%
\pgfsetdash{}{0pt}%
\pgfpathmoveto{\pgfqpoint{2.407531in}{1.343772in}}%
\pgfpathlineto{\pgfqpoint{2.272126in}{1.002530in}}%
\pgfusepath{stroke}%
\end{pgfscope}%
\begin{pgfscope}%
\pgfpathrectangle{\pgfqpoint{0.100000in}{0.212622in}}{\pgfqpoint{3.696000in}{3.696000in}}%
\pgfusepath{clip}%
\pgfsetrectcap%
\pgfsetroundjoin%
\pgfsetlinewidth{1.505625pt}%
\definecolor{currentstroke}{rgb}{1.000000,0.000000,0.000000}%
\pgfsetstrokecolor{currentstroke}%
\pgfsetdash{}{0pt}%
\pgfpathmoveto{\pgfqpoint{2.409055in}{1.339870in}}%
\pgfpathlineto{\pgfqpoint{2.272126in}{1.002530in}}%
\pgfusepath{stroke}%
\end{pgfscope}%
\begin{pgfscope}%
\pgfpathrectangle{\pgfqpoint{0.100000in}{0.212622in}}{\pgfqpoint{3.696000in}{3.696000in}}%
\pgfusepath{clip}%
\pgfsetrectcap%
\pgfsetroundjoin%
\pgfsetlinewidth{1.505625pt}%
\definecolor{currentstroke}{rgb}{1.000000,0.000000,0.000000}%
\pgfsetstrokecolor{currentstroke}%
\pgfsetdash{}{0pt}%
\pgfpathmoveto{\pgfqpoint{2.408486in}{1.332502in}}%
\pgfpathlineto{\pgfqpoint{2.272126in}{1.002530in}}%
\pgfusepath{stroke}%
\end{pgfscope}%
\begin{pgfscope}%
\pgfpathrectangle{\pgfqpoint{0.100000in}{0.212622in}}{\pgfqpoint{3.696000in}{3.696000in}}%
\pgfusepath{clip}%
\pgfsetrectcap%
\pgfsetroundjoin%
\pgfsetlinewidth{1.505625pt}%
\definecolor{currentstroke}{rgb}{1.000000,0.000000,0.000000}%
\pgfsetstrokecolor{currentstroke}%
\pgfsetdash{}{0pt}%
\pgfpathmoveto{\pgfqpoint{2.407468in}{1.331027in}}%
\pgfpathlineto{\pgfqpoint{2.272126in}{1.002530in}}%
\pgfusepath{stroke}%
\end{pgfscope}%
\begin{pgfscope}%
\pgfpathrectangle{\pgfqpoint{0.100000in}{0.212622in}}{\pgfqpoint{3.696000in}{3.696000in}}%
\pgfusepath{clip}%
\pgfsetrectcap%
\pgfsetroundjoin%
\pgfsetlinewidth{1.505625pt}%
\definecolor{currentstroke}{rgb}{1.000000,0.000000,0.000000}%
\pgfsetstrokecolor{currentstroke}%
\pgfsetdash{}{0pt}%
\pgfpathmoveto{\pgfqpoint{2.405766in}{1.329209in}}%
\pgfpathlineto{\pgfqpoint{2.272126in}{1.002530in}}%
\pgfusepath{stroke}%
\end{pgfscope}%
\begin{pgfscope}%
\pgfpathrectangle{\pgfqpoint{0.100000in}{0.212622in}}{\pgfqpoint{3.696000in}{3.696000in}}%
\pgfusepath{clip}%
\pgfsetrectcap%
\pgfsetroundjoin%
\pgfsetlinewidth{1.505625pt}%
\definecolor{currentstroke}{rgb}{1.000000,0.000000,0.000000}%
\pgfsetstrokecolor{currentstroke}%
\pgfsetdash{}{0pt}%
\pgfpathmoveto{\pgfqpoint{2.404577in}{1.329328in}}%
\pgfpathlineto{\pgfqpoint{2.272126in}{1.002530in}}%
\pgfusepath{stroke}%
\end{pgfscope}%
\begin{pgfscope}%
\pgfpathrectangle{\pgfqpoint{0.100000in}{0.212622in}}{\pgfqpoint{3.696000in}{3.696000in}}%
\pgfusepath{clip}%
\pgfsetrectcap%
\pgfsetroundjoin%
\pgfsetlinewidth{1.505625pt}%
\definecolor{currentstroke}{rgb}{1.000000,0.000000,0.000000}%
\pgfsetstrokecolor{currentstroke}%
\pgfsetdash{}{0pt}%
\pgfpathmoveto{\pgfqpoint{2.402369in}{1.328491in}}%
\pgfpathlineto{\pgfqpoint{2.272126in}{1.002530in}}%
\pgfusepath{stroke}%
\end{pgfscope}%
\begin{pgfscope}%
\pgfpathrectangle{\pgfqpoint{0.100000in}{0.212622in}}{\pgfqpoint{3.696000in}{3.696000in}}%
\pgfusepath{clip}%
\pgfsetrectcap%
\pgfsetroundjoin%
\pgfsetlinewidth{1.505625pt}%
\definecolor{currentstroke}{rgb}{1.000000,0.000000,0.000000}%
\pgfsetstrokecolor{currentstroke}%
\pgfsetdash{}{0pt}%
\pgfpathmoveto{\pgfqpoint{2.399048in}{1.327720in}}%
\pgfpathlineto{\pgfqpoint{2.272126in}{1.002530in}}%
\pgfusepath{stroke}%
\end{pgfscope}%
\begin{pgfscope}%
\pgfpathrectangle{\pgfqpoint{0.100000in}{0.212622in}}{\pgfqpoint{3.696000in}{3.696000in}}%
\pgfusepath{clip}%
\pgfsetrectcap%
\pgfsetroundjoin%
\pgfsetlinewidth{1.505625pt}%
\definecolor{currentstroke}{rgb}{1.000000,0.000000,0.000000}%
\pgfsetstrokecolor{currentstroke}%
\pgfsetdash{}{0pt}%
\pgfpathmoveto{\pgfqpoint{2.395083in}{1.331316in}}%
\pgfpathlineto{\pgfqpoint{2.272126in}{1.002530in}}%
\pgfusepath{stroke}%
\end{pgfscope}%
\begin{pgfscope}%
\pgfpathrectangle{\pgfqpoint{0.100000in}{0.212622in}}{\pgfqpoint{3.696000in}{3.696000in}}%
\pgfusepath{clip}%
\pgfsetrectcap%
\pgfsetroundjoin%
\pgfsetlinewidth{1.505625pt}%
\definecolor{currentstroke}{rgb}{1.000000,0.000000,0.000000}%
\pgfsetstrokecolor{currentstroke}%
\pgfsetdash{}{0pt}%
\pgfpathmoveto{\pgfqpoint{2.393174in}{1.332212in}}%
\pgfpathlineto{\pgfqpoint{2.272126in}{1.002530in}}%
\pgfusepath{stroke}%
\end{pgfscope}%
\begin{pgfscope}%
\pgfpathrectangle{\pgfqpoint{0.100000in}{0.212622in}}{\pgfqpoint{3.696000in}{3.696000in}}%
\pgfusepath{clip}%
\pgfsetrectcap%
\pgfsetroundjoin%
\pgfsetlinewidth{1.505625pt}%
\definecolor{currentstroke}{rgb}{1.000000,0.000000,0.000000}%
\pgfsetstrokecolor{currentstroke}%
\pgfsetdash{}{0pt}%
\pgfpathmoveto{\pgfqpoint{2.388983in}{1.335342in}}%
\pgfpathlineto{\pgfqpoint{2.272126in}{1.002530in}}%
\pgfusepath{stroke}%
\end{pgfscope}%
\begin{pgfscope}%
\pgfpathrectangle{\pgfqpoint{0.100000in}{0.212622in}}{\pgfqpoint{3.696000in}{3.696000in}}%
\pgfusepath{clip}%
\pgfsetrectcap%
\pgfsetroundjoin%
\pgfsetlinewidth{1.505625pt}%
\definecolor{currentstroke}{rgb}{1.000000,0.000000,0.000000}%
\pgfsetstrokecolor{currentstroke}%
\pgfsetdash{}{0pt}%
\pgfpathmoveto{\pgfqpoint{2.382776in}{1.336123in}}%
\pgfpathlineto{\pgfqpoint{2.272126in}{1.002530in}}%
\pgfusepath{stroke}%
\end{pgfscope}%
\begin{pgfscope}%
\pgfpathrectangle{\pgfqpoint{0.100000in}{0.212622in}}{\pgfqpoint{3.696000in}{3.696000in}}%
\pgfusepath{clip}%
\pgfsetrectcap%
\pgfsetroundjoin%
\pgfsetlinewidth{1.505625pt}%
\definecolor{currentstroke}{rgb}{1.000000,0.000000,0.000000}%
\pgfsetstrokecolor{currentstroke}%
\pgfsetdash{}{0pt}%
\pgfpathmoveto{\pgfqpoint{2.375710in}{1.345298in}}%
\pgfpathlineto{\pgfqpoint{2.272126in}{1.002530in}}%
\pgfusepath{stroke}%
\end{pgfscope}%
\begin{pgfscope}%
\pgfpathrectangle{\pgfqpoint{0.100000in}{0.212622in}}{\pgfqpoint{3.696000in}{3.696000in}}%
\pgfusepath{clip}%
\pgfsetrectcap%
\pgfsetroundjoin%
\pgfsetlinewidth{1.505625pt}%
\definecolor{currentstroke}{rgb}{1.000000,0.000000,0.000000}%
\pgfsetstrokecolor{currentstroke}%
\pgfsetdash{}{0pt}%
\pgfpathmoveto{\pgfqpoint{2.368341in}{1.350803in}}%
\pgfpathlineto{\pgfqpoint{2.272126in}{1.002530in}}%
\pgfusepath{stroke}%
\end{pgfscope}%
\begin{pgfscope}%
\pgfpathrectangle{\pgfqpoint{0.100000in}{0.212622in}}{\pgfqpoint{3.696000in}{3.696000in}}%
\pgfusepath{clip}%
\pgfsetrectcap%
\pgfsetroundjoin%
\pgfsetlinewidth{1.505625pt}%
\definecolor{currentstroke}{rgb}{1.000000,0.000000,0.000000}%
\pgfsetstrokecolor{currentstroke}%
\pgfsetdash{}{0pt}%
\pgfpathmoveto{\pgfqpoint{2.360146in}{1.357936in}}%
\pgfpathlineto{\pgfqpoint{2.272126in}{1.002530in}}%
\pgfusepath{stroke}%
\end{pgfscope}%
\begin{pgfscope}%
\pgfpathrectangle{\pgfqpoint{0.100000in}{0.212622in}}{\pgfqpoint{3.696000in}{3.696000in}}%
\pgfusepath{clip}%
\pgfsetrectcap%
\pgfsetroundjoin%
\pgfsetlinewidth{1.505625pt}%
\definecolor{currentstroke}{rgb}{1.000000,0.000000,0.000000}%
\pgfsetstrokecolor{currentstroke}%
\pgfsetdash{}{0pt}%
\pgfpathmoveto{\pgfqpoint{2.355736in}{1.361337in}}%
\pgfpathlineto{\pgfqpoint{2.272126in}{1.002530in}}%
\pgfusepath{stroke}%
\end{pgfscope}%
\begin{pgfscope}%
\pgfpathrectangle{\pgfqpoint{0.100000in}{0.212622in}}{\pgfqpoint{3.696000in}{3.696000in}}%
\pgfusepath{clip}%
\pgfsetrectcap%
\pgfsetroundjoin%
\pgfsetlinewidth{1.505625pt}%
\definecolor{currentstroke}{rgb}{1.000000,0.000000,0.000000}%
\pgfsetstrokecolor{currentstroke}%
\pgfsetdash{}{0pt}%
\pgfpathmoveto{\pgfqpoint{2.353301in}{1.363344in}}%
\pgfpathlineto{\pgfqpoint{2.272126in}{1.002530in}}%
\pgfusepath{stroke}%
\end{pgfscope}%
\begin{pgfscope}%
\pgfpathrectangle{\pgfqpoint{0.100000in}{0.212622in}}{\pgfqpoint{3.696000in}{3.696000in}}%
\pgfusepath{clip}%
\pgfsetrectcap%
\pgfsetroundjoin%
\pgfsetlinewidth{1.505625pt}%
\definecolor{currentstroke}{rgb}{1.000000,0.000000,0.000000}%
\pgfsetstrokecolor{currentstroke}%
\pgfsetdash{}{0pt}%
\pgfpathmoveto{\pgfqpoint{2.351981in}{1.364359in}}%
\pgfpathlineto{\pgfqpoint{2.272126in}{1.002530in}}%
\pgfusepath{stroke}%
\end{pgfscope}%
\begin{pgfscope}%
\pgfpathrectangle{\pgfqpoint{0.100000in}{0.212622in}}{\pgfqpoint{3.696000in}{3.696000in}}%
\pgfusepath{clip}%
\pgfsetrectcap%
\pgfsetroundjoin%
\pgfsetlinewidth{1.505625pt}%
\definecolor{currentstroke}{rgb}{1.000000,0.000000,0.000000}%
\pgfsetstrokecolor{currentstroke}%
\pgfsetdash{}{0pt}%
\pgfpathmoveto{\pgfqpoint{2.351243in}{1.364850in}}%
\pgfpathlineto{\pgfqpoint{2.272126in}{1.002530in}}%
\pgfusepath{stroke}%
\end{pgfscope}%
\begin{pgfscope}%
\pgfpathrectangle{\pgfqpoint{0.100000in}{0.212622in}}{\pgfqpoint{3.696000in}{3.696000in}}%
\pgfusepath{clip}%
\pgfsetrectcap%
\pgfsetroundjoin%
\pgfsetlinewidth{1.505625pt}%
\definecolor{currentstroke}{rgb}{1.000000,0.000000,0.000000}%
\pgfsetstrokecolor{currentstroke}%
\pgfsetdash{}{0pt}%
\pgfpathmoveto{\pgfqpoint{2.350822in}{1.365116in}}%
\pgfpathlineto{\pgfqpoint{2.272126in}{1.002530in}}%
\pgfusepath{stroke}%
\end{pgfscope}%
\begin{pgfscope}%
\pgfpathrectangle{\pgfqpoint{0.100000in}{0.212622in}}{\pgfqpoint{3.696000in}{3.696000in}}%
\pgfusepath{clip}%
\pgfsetrectcap%
\pgfsetroundjoin%
\pgfsetlinewidth{1.505625pt}%
\definecolor{currentstroke}{rgb}{1.000000,0.000000,0.000000}%
\pgfsetstrokecolor{currentstroke}%
\pgfsetdash{}{0pt}%
\pgfpathmoveto{\pgfqpoint{2.350586in}{1.365273in}}%
\pgfpathlineto{\pgfqpoint{2.272126in}{1.002530in}}%
\pgfusepath{stroke}%
\end{pgfscope}%
\begin{pgfscope}%
\pgfpathrectangle{\pgfqpoint{0.100000in}{0.212622in}}{\pgfqpoint{3.696000in}{3.696000in}}%
\pgfusepath{clip}%
\pgfsetrectcap%
\pgfsetroundjoin%
\pgfsetlinewidth{1.505625pt}%
\definecolor{currentstroke}{rgb}{1.000000,0.000000,0.000000}%
\pgfsetstrokecolor{currentstroke}%
\pgfsetdash{}{0pt}%
\pgfpathmoveto{\pgfqpoint{2.349447in}{1.365884in}}%
\pgfpathlineto{\pgfqpoint{2.272126in}{1.002530in}}%
\pgfusepath{stroke}%
\end{pgfscope}%
\begin{pgfscope}%
\pgfpathrectangle{\pgfqpoint{0.100000in}{0.212622in}}{\pgfqpoint{3.696000in}{3.696000in}}%
\pgfusepath{clip}%
\pgfsetrectcap%
\pgfsetroundjoin%
\pgfsetlinewidth{1.505625pt}%
\definecolor{currentstroke}{rgb}{1.000000,0.000000,0.000000}%
\pgfsetstrokecolor{currentstroke}%
\pgfsetdash{}{0pt}%
\pgfpathmoveto{\pgfqpoint{2.346199in}{1.367554in}}%
\pgfpathlineto{\pgfqpoint{2.272126in}{1.002530in}}%
\pgfusepath{stroke}%
\end{pgfscope}%
\begin{pgfscope}%
\pgfpathrectangle{\pgfqpoint{0.100000in}{0.212622in}}{\pgfqpoint{3.696000in}{3.696000in}}%
\pgfusepath{clip}%
\pgfsetrectcap%
\pgfsetroundjoin%
\pgfsetlinewidth{1.505625pt}%
\definecolor{currentstroke}{rgb}{1.000000,0.000000,0.000000}%
\pgfsetstrokecolor{currentstroke}%
\pgfsetdash{}{0pt}%
\pgfpathmoveto{\pgfqpoint{2.344405in}{1.368882in}}%
\pgfpathlineto{\pgfqpoint{2.272126in}{1.002530in}}%
\pgfusepath{stroke}%
\end{pgfscope}%
\begin{pgfscope}%
\pgfpathrectangle{\pgfqpoint{0.100000in}{0.212622in}}{\pgfqpoint{3.696000in}{3.696000in}}%
\pgfusepath{clip}%
\pgfsetrectcap%
\pgfsetroundjoin%
\pgfsetlinewidth{1.505625pt}%
\definecolor{currentstroke}{rgb}{1.000000,0.000000,0.000000}%
\pgfsetstrokecolor{currentstroke}%
\pgfsetdash{}{0pt}%
\pgfpathmoveto{\pgfqpoint{2.340560in}{1.373025in}}%
\pgfpathlineto{\pgfqpoint{2.272126in}{1.002530in}}%
\pgfusepath{stroke}%
\end{pgfscope}%
\begin{pgfscope}%
\pgfpathrectangle{\pgfqpoint{0.100000in}{0.212622in}}{\pgfqpoint{3.696000in}{3.696000in}}%
\pgfusepath{clip}%
\pgfsetrectcap%
\pgfsetroundjoin%
\pgfsetlinewidth{1.505625pt}%
\definecolor{currentstroke}{rgb}{1.000000,0.000000,0.000000}%
\pgfsetstrokecolor{currentstroke}%
\pgfsetdash{}{0pt}%
\pgfpathmoveto{\pgfqpoint{2.335668in}{1.374395in}}%
\pgfpathlineto{\pgfqpoint{2.272126in}{1.002530in}}%
\pgfusepath{stroke}%
\end{pgfscope}%
\begin{pgfscope}%
\pgfpathrectangle{\pgfqpoint{0.100000in}{0.212622in}}{\pgfqpoint{3.696000in}{3.696000in}}%
\pgfusepath{clip}%
\pgfsetrectcap%
\pgfsetroundjoin%
\pgfsetlinewidth{1.505625pt}%
\definecolor{currentstroke}{rgb}{1.000000,0.000000,0.000000}%
\pgfsetstrokecolor{currentstroke}%
\pgfsetdash{}{0pt}%
\pgfpathmoveto{\pgfqpoint{2.327316in}{1.379721in}}%
\pgfpathlineto{\pgfqpoint{2.272126in}{1.002530in}}%
\pgfusepath{stroke}%
\end{pgfscope}%
\begin{pgfscope}%
\pgfpathrectangle{\pgfqpoint{0.100000in}{0.212622in}}{\pgfqpoint{3.696000in}{3.696000in}}%
\pgfusepath{clip}%
\pgfsetrectcap%
\pgfsetroundjoin%
\pgfsetlinewidth{1.505625pt}%
\definecolor{currentstroke}{rgb}{1.000000,0.000000,0.000000}%
\pgfsetstrokecolor{currentstroke}%
\pgfsetdash{}{0pt}%
\pgfpathmoveto{\pgfqpoint{2.322884in}{1.382719in}}%
\pgfpathlineto{\pgfqpoint{2.272126in}{1.002530in}}%
\pgfusepath{stroke}%
\end{pgfscope}%
\begin{pgfscope}%
\pgfpathrectangle{\pgfqpoint{0.100000in}{0.212622in}}{\pgfqpoint{3.696000in}{3.696000in}}%
\pgfusepath{clip}%
\pgfsetrectcap%
\pgfsetroundjoin%
\pgfsetlinewidth{1.505625pt}%
\definecolor{currentstroke}{rgb}{1.000000,0.000000,0.000000}%
\pgfsetstrokecolor{currentstroke}%
\pgfsetdash{}{0pt}%
\pgfpathmoveto{\pgfqpoint{2.316989in}{1.386658in}}%
\pgfpathlineto{\pgfqpoint{2.272126in}{1.002530in}}%
\pgfusepath{stroke}%
\end{pgfscope}%
\begin{pgfscope}%
\pgfpathrectangle{\pgfqpoint{0.100000in}{0.212622in}}{\pgfqpoint{3.696000in}{3.696000in}}%
\pgfusepath{clip}%
\pgfsetrectcap%
\pgfsetroundjoin%
\pgfsetlinewidth{1.505625pt}%
\definecolor{currentstroke}{rgb}{1.000000,0.000000,0.000000}%
\pgfsetstrokecolor{currentstroke}%
\pgfsetdash{}{0pt}%
\pgfpathmoveto{\pgfqpoint{2.310330in}{1.390506in}}%
\pgfpathlineto{\pgfqpoint{2.272126in}{1.002530in}}%
\pgfusepath{stroke}%
\end{pgfscope}%
\begin{pgfscope}%
\pgfpathrectangle{\pgfqpoint{0.100000in}{0.212622in}}{\pgfqpoint{3.696000in}{3.696000in}}%
\pgfusepath{clip}%
\pgfsetrectcap%
\pgfsetroundjoin%
\pgfsetlinewidth{1.505625pt}%
\definecolor{currentstroke}{rgb}{1.000000,0.000000,0.000000}%
\pgfsetstrokecolor{currentstroke}%
\pgfsetdash{}{0pt}%
\pgfpathmoveto{\pgfqpoint{2.300540in}{1.396150in}}%
\pgfpathlineto{\pgfqpoint{2.272126in}{1.002530in}}%
\pgfusepath{stroke}%
\end{pgfscope}%
\begin{pgfscope}%
\pgfpathrectangle{\pgfqpoint{0.100000in}{0.212622in}}{\pgfqpoint{3.696000in}{3.696000in}}%
\pgfusepath{clip}%
\pgfsetrectcap%
\pgfsetroundjoin%
\pgfsetlinewidth{1.505625pt}%
\definecolor{currentstroke}{rgb}{1.000000,0.000000,0.000000}%
\pgfsetstrokecolor{currentstroke}%
\pgfsetdash{}{0pt}%
\pgfpathmoveto{\pgfqpoint{2.289982in}{1.408509in}}%
\pgfpathlineto{\pgfqpoint{2.272126in}{1.002530in}}%
\pgfusepath{stroke}%
\end{pgfscope}%
\begin{pgfscope}%
\pgfpathrectangle{\pgfqpoint{0.100000in}{0.212622in}}{\pgfqpoint{3.696000in}{3.696000in}}%
\pgfusepath{clip}%
\pgfsetrectcap%
\pgfsetroundjoin%
\pgfsetlinewidth{1.505625pt}%
\definecolor{currentstroke}{rgb}{1.000000,0.000000,0.000000}%
\pgfsetstrokecolor{currentstroke}%
\pgfsetdash{}{0pt}%
\pgfpathmoveto{\pgfqpoint{2.277415in}{1.413587in}}%
\pgfpathlineto{\pgfqpoint{2.272126in}{1.002530in}}%
\pgfusepath{stroke}%
\end{pgfscope}%
\begin{pgfscope}%
\pgfpathrectangle{\pgfqpoint{0.100000in}{0.212622in}}{\pgfqpoint{3.696000in}{3.696000in}}%
\pgfusepath{clip}%
\pgfsetrectcap%
\pgfsetroundjoin%
\pgfsetlinewidth{1.505625pt}%
\definecolor{currentstroke}{rgb}{1.000000,0.000000,0.000000}%
\pgfsetstrokecolor{currentstroke}%
\pgfsetdash{}{0pt}%
\pgfpathmoveto{\pgfqpoint{2.263921in}{1.418549in}}%
\pgfpathlineto{\pgfqpoint{2.272126in}{1.002530in}}%
\pgfusepath{stroke}%
\end{pgfscope}%
\begin{pgfscope}%
\pgfpathrectangle{\pgfqpoint{0.100000in}{0.212622in}}{\pgfqpoint{3.696000in}{3.696000in}}%
\pgfusepath{clip}%
\pgfsetrectcap%
\pgfsetroundjoin%
\pgfsetlinewidth{1.505625pt}%
\definecolor{currentstroke}{rgb}{1.000000,0.000000,0.000000}%
\pgfsetstrokecolor{currentstroke}%
\pgfsetdash{}{0pt}%
\pgfpathmoveto{\pgfqpoint{2.246719in}{1.432913in}}%
\pgfpathlineto{\pgfqpoint{2.272126in}{1.002530in}}%
\pgfusepath{stroke}%
\end{pgfscope}%
\begin{pgfscope}%
\pgfpathrectangle{\pgfqpoint{0.100000in}{0.212622in}}{\pgfqpoint{3.696000in}{3.696000in}}%
\pgfusepath{clip}%
\pgfsetrectcap%
\pgfsetroundjoin%
\pgfsetlinewidth{1.505625pt}%
\definecolor{currentstroke}{rgb}{1.000000,0.000000,0.000000}%
\pgfsetstrokecolor{currentstroke}%
\pgfsetdash{}{0pt}%
\pgfpathmoveto{\pgfqpoint{2.231249in}{1.444419in}}%
\pgfpathlineto{\pgfqpoint{2.272126in}{1.002530in}}%
\pgfusepath{stroke}%
\end{pgfscope}%
\begin{pgfscope}%
\pgfpathrectangle{\pgfqpoint{0.100000in}{0.212622in}}{\pgfqpoint{3.696000in}{3.696000in}}%
\pgfusepath{clip}%
\pgfsetrectcap%
\pgfsetroundjoin%
\pgfsetlinewidth{1.505625pt}%
\definecolor{currentstroke}{rgb}{1.000000,0.000000,0.000000}%
\pgfsetstrokecolor{currentstroke}%
\pgfsetdash{}{0pt}%
\pgfpathmoveto{\pgfqpoint{2.214061in}{1.460840in}}%
\pgfpathlineto{\pgfqpoint{2.272126in}{1.002530in}}%
\pgfusepath{stroke}%
\end{pgfscope}%
\begin{pgfscope}%
\pgfpathrectangle{\pgfqpoint{0.100000in}{0.212622in}}{\pgfqpoint{3.696000in}{3.696000in}}%
\pgfusepath{clip}%
\pgfsetrectcap%
\pgfsetroundjoin%
\pgfsetlinewidth{1.505625pt}%
\definecolor{currentstroke}{rgb}{1.000000,0.000000,0.000000}%
\pgfsetstrokecolor{currentstroke}%
\pgfsetdash{}{0pt}%
\pgfpathmoveto{\pgfqpoint{2.203375in}{1.464151in}}%
\pgfpathlineto{\pgfqpoint{2.272126in}{1.002530in}}%
\pgfusepath{stroke}%
\end{pgfscope}%
\begin{pgfscope}%
\pgfpathrectangle{\pgfqpoint{0.100000in}{0.212622in}}{\pgfqpoint{3.696000in}{3.696000in}}%
\pgfusepath{clip}%
\pgfsetrectcap%
\pgfsetroundjoin%
\pgfsetlinewidth{1.505625pt}%
\definecolor{currentstroke}{rgb}{1.000000,0.000000,0.000000}%
\pgfsetstrokecolor{currentstroke}%
\pgfsetdash{}{0pt}%
\pgfpathmoveto{\pgfqpoint{2.191706in}{1.471693in}}%
\pgfpathlineto{\pgfqpoint{2.272126in}{1.002530in}}%
\pgfusepath{stroke}%
\end{pgfscope}%
\begin{pgfscope}%
\pgfpathrectangle{\pgfqpoint{0.100000in}{0.212622in}}{\pgfqpoint{3.696000in}{3.696000in}}%
\pgfusepath{clip}%
\pgfsetrectcap%
\pgfsetroundjoin%
\pgfsetlinewidth{1.505625pt}%
\definecolor{currentstroke}{rgb}{1.000000,0.000000,0.000000}%
\pgfsetstrokecolor{currentstroke}%
\pgfsetdash{}{0pt}%
\pgfpathmoveto{\pgfqpoint{2.177430in}{1.478757in}}%
\pgfpathlineto{\pgfqpoint{2.272126in}{1.002530in}}%
\pgfusepath{stroke}%
\end{pgfscope}%
\begin{pgfscope}%
\pgfpathrectangle{\pgfqpoint{0.100000in}{0.212622in}}{\pgfqpoint{3.696000in}{3.696000in}}%
\pgfusepath{clip}%
\pgfsetrectcap%
\pgfsetroundjoin%
\pgfsetlinewidth{1.505625pt}%
\definecolor{currentstroke}{rgb}{1.000000,0.000000,0.000000}%
\pgfsetstrokecolor{currentstroke}%
\pgfsetdash{}{0pt}%
\pgfpathmoveto{\pgfqpoint{2.162068in}{1.489383in}}%
\pgfpathlineto{\pgfqpoint{2.257607in}{1.007093in}}%
\pgfusepath{stroke}%
\end{pgfscope}%
\begin{pgfscope}%
\pgfpathrectangle{\pgfqpoint{0.100000in}{0.212622in}}{\pgfqpoint{3.696000in}{3.696000in}}%
\pgfusepath{clip}%
\pgfsetrectcap%
\pgfsetroundjoin%
\pgfsetlinewidth{1.505625pt}%
\definecolor{currentstroke}{rgb}{1.000000,0.000000,0.000000}%
\pgfsetstrokecolor{currentstroke}%
\pgfsetdash{}{0pt}%
\pgfpathmoveto{\pgfqpoint{2.145904in}{1.503840in}}%
\pgfpathlineto{\pgfqpoint{2.243098in}{1.011652in}}%
\pgfusepath{stroke}%
\end{pgfscope}%
\begin{pgfscope}%
\pgfpathrectangle{\pgfqpoint{0.100000in}{0.212622in}}{\pgfqpoint{3.696000in}{3.696000in}}%
\pgfusepath{clip}%
\pgfsetrectcap%
\pgfsetroundjoin%
\pgfsetlinewidth{1.505625pt}%
\definecolor{currentstroke}{rgb}{1.000000,0.000000,0.000000}%
\pgfsetstrokecolor{currentstroke}%
\pgfsetdash{}{0pt}%
\pgfpathmoveto{\pgfqpoint{2.127992in}{1.516224in}}%
\pgfpathlineto{\pgfqpoint{2.228599in}{1.016208in}}%
\pgfusepath{stroke}%
\end{pgfscope}%
\begin{pgfscope}%
\pgfpathrectangle{\pgfqpoint{0.100000in}{0.212622in}}{\pgfqpoint{3.696000in}{3.696000in}}%
\pgfusepath{clip}%
\pgfsetrectcap%
\pgfsetroundjoin%
\pgfsetlinewidth{1.505625pt}%
\definecolor{currentstroke}{rgb}{1.000000,0.000000,0.000000}%
\pgfsetstrokecolor{currentstroke}%
\pgfsetdash{}{0pt}%
\pgfpathmoveto{\pgfqpoint{2.106788in}{1.537615in}}%
\pgfpathlineto{\pgfqpoint{2.199633in}{1.025311in}}%
\pgfusepath{stroke}%
\end{pgfscope}%
\begin{pgfscope}%
\pgfpathrectangle{\pgfqpoint{0.100000in}{0.212622in}}{\pgfqpoint{3.696000in}{3.696000in}}%
\pgfusepath{clip}%
\pgfsetrectcap%
\pgfsetroundjoin%
\pgfsetlinewidth{1.505625pt}%
\definecolor{currentstroke}{rgb}{1.000000,0.000000,0.000000}%
\pgfsetstrokecolor{currentstroke}%
\pgfsetdash{}{0pt}%
\pgfpathmoveto{\pgfqpoint{2.095085in}{1.543936in}}%
\pgfpathlineto{\pgfqpoint{2.199633in}{1.025311in}}%
\pgfusepath{stroke}%
\end{pgfscope}%
\begin{pgfscope}%
\pgfpathrectangle{\pgfqpoint{0.100000in}{0.212622in}}{\pgfqpoint{3.696000in}{3.696000in}}%
\pgfusepath{clip}%
\pgfsetrectcap%
\pgfsetroundjoin%
\pgfsetlinewidth{1.505625pt}%
\definecolor{currentstroke}{rgb}{1.000000,0.000000,0.000000}%
\pgfsetstrokecolor{currentstroke}%
\pgfsetdash{}{0pt}%
\pgfpathmoveto{\pgfqpoint{2.080911in}{1.556576in}}%
\pgfpathlineto{\pgfqpoint{2.185166in}{1.029857in}}%
\pgfusepath{stroke}%
\end{pgfscope}%
\begin{pgfscope}%
\pgfpathrectangle{\pgfqpoint{0.100000in}{0.212622in}}{\pgfqpoint{3.696000in}{3.696000in}}%
\pgfusepath{clip}%
\pgfsetrectcap%
\pgfsetroundjoin%
\pgfsetlinewidth{1.505625pt}%
\definecolor{currentstroke}{rgb}{1.000000,0.000000,0.000000}%
\pgfsetstrokecolor{currentstroke}%
\pgfsetdash{}{0pt}%
\pgfpathmoveto{\pgfqpoint{2.072460in}{1.559399in}}%
\pgfpathlineto{\pgfqpoint{2.170709in}{1.034400in}}%
\pgfusepath{stroke}%
\end{pgfscope}%
\begin{pgfscope}%
\pgfpathrectangle{\pgfqpoint{0.100000in}{0.212622in}}{\pgfqpoint{3.696000in}{3.696000in}}%
\pgfusepath{clip}%
\pgfsetrectcap%
\pgfsetroundjoin%
\pgfsetlinewidth{1.505625pt}%
\definecolor{currentstroke}{rgb}{1.000000,0.000000,0.000000}%
\pgfsetstrokecolor{currentstroke}%
\pgfsetdash{}{0pt}%
\pgfpathmoveto{\pgfqpoint{2.060408in}{1.566714in}}%
\pgfpathlineto{\pgfqpoint{2.156262in}{1.038940in}}%
\pgfusepath{stroke}%
\end{pgfscope}%
\begin{pgfscope}%
\pgfpathrectangle{\pgfqpoint{0.100000in}{0.212622in}}{\pgfqpoint{3.696000in}{3.696000in}}%
\pgfusepath{clip}%
\pgfsetrectcap%
\pgfsetroundjoin%
\pgfsetlinewidth{1.505625pt}%
\definecolor{currentstroke}{rgb}{1.000000,0.000000,0.000000}%
\pgfsetstrokecolor{currentstroke}%
\pgfsetdash{}{0pt}%
\pgfpathmoveto{\pgfqpoint{2.053984in}{1.572360in}}%
\pgfpathlineto{\pgfqpoint{2.156262in}{1.038940in}}%
\pgfusepath{stroke}%
\end{pgfscope}%
\begin{pgfscope}%
\pgfpathrectangle{\pgfqpoint{0.100000in}{0.212622in}}{\pgfqpoint{3.696000in}{3.696000in}}%
\pgfusepath{clip}%
\pgfsetrectcap%
\pgfsetroundjoin%
\pgfsetlinewidth{1.505625pt}%
\definecolor{currentstroke}{rgb}{1.000000,0.000000,0.000000}%
\pgfsetstrokecolor{currentstroke}%
\pgfsetdash{}{0pt}%
\pgfpathmoveto{\pgfqpoint{2.045275in}{1.577058in}}%
\pgfpathlineto{\pgfqpoint{2.141826in}{1.043477in}}%
\pgfusepath{stroke}%
\end{pgfscope}%
\begin{pgfscope}%
\pgfpathrectangle{\pgfqpoint{0.100000in}{0.212622in}}{\pgfqpoint{3.696000in}{3.696000in}}%
\pgfusepath{clip}%
\pgfsetrectcap%
\pgfsetroundjoin%
\pgfsetlinewidth{1.505625pt}%
\definecolor{currentstroke}{rgb}{1.000000,0.000000,0.000000}%
\pgfsetstrokecolor{currentstroke}%
\pgfsetdash{}{0pt}%
\pgfpathmoveto{\pgfqpoint{2.035031in}{1.584816in}}%
\pgfpathlineto{\pgfqpoint{2.127400in}{1.048010in}}%
\pgfusepath{stroke}%
\end{pgfscope}%
\begin{pgfscope}%
\pgfpathrectangle{\pgfqpoint{0.100000in}{0.212622in}}{\pgfqpoint{3.696000in}{3.696000in}}%
\pgfusepath{clip}%
\pgfsetrectcap%
\pgfsetroundjoin%
\pgfsetlinewidth{1.505625pt}%
\definecolor{currentstroke}{rgb}{1.000000,0.000000,0.000000}%
\pgfsetstrokecolor{currentstroke}%
\pgfsetdash{}{0pt}%
\pgfpathmoveto{\pgfqpoint{2.022758in}{1.593781in}}%
\pgfpathlineto{\pgfqpoint{2.112985in}{1.052540in}}%
\pgfusepath{stroke}%
\end{pgfscope}%
\begin{pgfscope}%
\pgfpathrectangle{\pgfqpoint{0.100000in}{0.212622in}}{\pgfqpoint{3.696000in}{3.696000in}}%
\pgfusepath{clip}%
\pgfsetrectcap%
\pgfsetroundjoin%
\pgfsetlinewidth{1.505625pt}%
\definecolor{currentstroke}{rgb}{1.000000,0.000000,0.000000}%
\pgfsetstrokecolor{currentstroke}%
\pgfsetdash{}{0pt}%
\pgfpathmoveto{\pgfqpoint{2.009034in}{1.603833in}}%
\pgfpathlineto{\pgfqpoint{2.112985in}{1.052540in}}%
\pgfusepath{stroke}%
\end{pgfscope}%
\begin{pgfscope}%
\pgfpathrectangle{\pgfqpoint{0.100000in}{0.212622in}}{\pgfqpoint{3.696000in}{3.696000in}}%
\pgfusepath{clip}%
\pgfsetrectcap%
\pgfsetroundjoin%
\pgfsetlinewidth{1.505625pt}%
\definecolor{currentstroke}{rgb}{1.000000,0.000000,0.000000}%
\pgfsetstrokecolor{currentstroke}%
\pgfsetdash{}{0pt}%
\pgfpathmoveto{\pgfqpoint{1.992573in}{1.615227in}}%
\pgfpathlineto{\pgfqpoint{2.084186in}{1.061590in}}%
\pgfusepath{stroke}%
\end{pgfscope}%
\begin{pgfscope}%
\pgfpathrectangle{\pgfqpoint{0.100000in}{0.212622in}}{\pgfqpoint{3.696000in}{3.696000in}}%
\pgfusepath{clip}%
\pgfsetrectcap%
\pgfsetroundjoin%
\pgfsetlinewidth{1.505625pt}%
\definecolor{currentstroke}{rgb}{1.000000,0.000000,0.000000}%
\pgfsetstrokecolor{currentstroke}%
\pgfsetdash{}{0pt}%
\pgfpathmoveto{\pgfqpoint{1.974749in}{1.624811in}}%
\pgfpathlineto{\pgfqpoint{2.069801in}{1.066110in}}%
\pgfusepath{stroke}%
\end{pgfscope}%
\begin{pgfscope}%
\pgfpathrectangle{\pgfqpoint{0.100000in}{0.212622in}}{\pgfqpoint{3.696000in}{3.696000in}}%
\pgfusepath{clip}%
\pgfsetrectcap%
\pgfsetroundjoin%
\pgfsetlinewidth{1.505625pt}%
\definecolor{currentstroke}{rgb}{1.000000,0.000000,0.000000}%
\pgfsetstrokecolor{currentstroke}%
\pgfsetdash{}{0pt}%
\pgfpathmoveto{\pgfqpoint{1.954393in}{1.636167in}}%
\pgfpathlineto{\pgfqpoint{2.055427in}{1.070627in}}%
\pgfusepath{stroke}%
\end{pgfscope}%
\begin{pgfscope}%
\pgfpathrectangle{\pgfqpoint{0.100000in}{0.212622in}}{\pgfqpoint{3.696000in}{3.696000in}}%
\pgfusepath{clip}%
\pgfsetrectcap%
\pgfsetroundjoin%
\pgfsetlinewidth{1.505625pt}%
\definecolor{currentstroke}{rgb}{1.000000,0.000000,0.000000}%
\pgfsetstrokecolor{currentstroke}%
\pgfsetdash{}{0pt}%
\pgfpathmoveto{\pgfqpoint{1.930932in}{1.662148in}}%
\pgfpathlineto{\pgfqpoint{2.026710in}{1.079652in}}%
\pgfusepath{stroke}%
\end{pgfscope}%
\begin{pgfscope}%
\pgfpathrectangle{\pgfqpoint{0.100000in}{0.212622in}}{\pgfqpoint{3.696000in}{3.696000in}}%
\pgfusepath{clip}%
\pgfsetrectcap%
\pgfsetroundjoin%
\pgfsetlinewidth{1.505625pt}%
\definecolor{currentstroke}{rgb}{1.000000,0.000000,0.000000}%
\pgfsetstrokecolor{currentstroke}%
\pgfsetdash{}{0pt}%
\pgfpathmoveto{\pgfqpoint{1.903096in}{1.676524in}}%
\pgfpathlineto{\pgfqpoint{1.998035in}{1.088663in}}%
\pgfusepath{stroke}%
\end{pgfscope}%
\begin{pgfscope}%
\pgfpathrectangle{\pgfqpoint{0.100000in}{0.212622in}}{\pgfqpoint{3.696000in}{3.696000in}}%
\pgfusepath{clip}%
\pgfsetrectcap%
\pgfsetroundjoin%
\pgfsetlinewidth{1.505625pt}%
\definecolor{currentstroke}{rgb}{1.000000,0.000000,0.000000}%
\pgfsetstrokecolor{currentstroke}%
\pgfsetdash{}{0pt}%
\pgfpathmoveto{\pgfqpoint{1.874279in}{1.691100in}}%
\pgfpathlineto{\pgfqpoint{1.969400in}{1.097662in}}%
\pgfusepath{stroke}%
\end{pgfscope}%
\begin{pgfscope}%
\pgfpathrectangle{\pgfqpoint{0.100000in}{0.212622in}}{\pgfqpoint{3.696000in}{3.696000in}}%
\pgfusepath{clip}%
\pgfsetrectcap%
\pgfsetroundjoin%
\pgfsetlinewidth{1.505625pt}%
\definecolor{currentstroke}{rgb}{1.000000,0.000000,0.000000}%
\pgfsetstrokecolor{currentstroke}%
\pgfsetdash{}{0pt}%
\pgfpathmoveto{\pgfqpoint{1.842136in}{1.715252in}}%
\pgfpathlineto{\pgfqpoint{1.940806in}{1.106647in}}%
\pgfusepath{stroke}%
\end{pgfscope}%
\begin{pgfscope}%
\pgfpathrectangle{\pgfqpoint{0.100000in}{0.212622in}}{\pgfqpoint{3.696000in}{3.696000in}}%
\pgfusepath{clip}%
\pgfsetrectcap%
\pgfsetroundjoin%
\pgfsetlinewidth{1.505625pt}%
\definecolor{currentstroke}{rgb}{1.000000,0.000000,0.000000}%
\pgfsetstrokecolor{currentstroke}%
\pgfsetdash{}{0pt}%
\pgfpathmoveto{\pgfqpoint{1.808439in}{1.746641in}}%
\pgfpathlineto{\pgfqpoint{1.912254in}{1.115620in}}%
\pgfusepath{stroke}%
\end{pgfscope}%
\begin{pgfscope}%
\pgfpathrectangle{\pgfqpoint{0.100000in}{0.212622in}}{\pgfqpoint{3.696000in}{3.696000in}}%
\pgfusepath{clip}%
\pgfsetrectcap%
\pgfsetroundjoin%
\pgfsetlinewidth{1.505625pt}%
\definecolor{currentstroke}{rgb}{1.000000,0.000000,0.000000}%
\pgfsetstrokecolor{currentstroke}%
\pgfsetdash{}{0pt}%
\pgfpathmoveto{\pgfqpoint{1.771125in}{1.765866in}}%
\pgfpathlineto{\pgfqpoint{1.869501in}{1.129055in}}%
\pgfusepath{stroke}%
\end{pgfscope}%
\begin{pgfscope}%
\pgfpathrectangle{\pgfqpoint{0.100000in}{0.212622in}}{\pgfqpoint{3.696000in}{3.696000in}}%
\pgfusepath{clip}%
\pgfsetrectcap%
\pgfsetroundjoin%
\pgfsetlinewidth{1.505625pt}%
\definecolor{currentstroke}{rgb}{1.000000,0.000000,0.000000}%
\pgfsetstrokecolor{currentstroke}%
\pgfsetdash{}{0pt}%
\pgfpathmoveto{\pgfqpoint{1.732638in}{1.789149in}}%
\pgfpathlineto{\pgfqpoint{1.826840in}{1.142461in}}%
\pgfusepath{stroke}%
\end{pgfscope}%
\begin{pgfscope}%
\pgfpathrectangle{\pgfqpoint{0.100000in}{0.212622in}}{\pgfqpoint{3.696000in}{3.696000in}}%
\pgfusepath{clip}%
\pgfsetrectcap%
\pgfsetroundjoin%
\pgfsetlinewidth{1.505625pt}%
\definecolor{currentstroke}{rgb}{1.000000,0.000000,0.000000}%
\pgfsetstrokecolor{currentstroke}%
\pgfsetdash{}{0pt}%
\pgfpathmoveto{\pgfqpoint{1.693756in}{1.808903in}}%
\pgfpathlineto{\pgfqpoint{1.798450in}{1.151382in}}%
\pgfusepath{stroke}%
\end{pgfscope}%
\begin{pgfscope}%
\pgfpathrectangle{\pgfqpoint{0.100000in}{0.212622in}}{\pgfqpoint{3.696000in}{3.696000in}}%
\pgfusepath{clip}%
\pgfsetrectcap%
\pgfsetroundjoin%
\pgfsetlinewidth{1.505625pt}%
\definecolor{currentstroke}{rgb}{1.000000,0.000000,0.000000}%
\pgfsetstrokecolor{currentstroke}%
\pgfsetdash{}{0pt}%
\pgfpathmoveto{\pgfqpoint{1.672804in}{1.827765in}}%
\pgfpathlineto{\pgfqpoint{1.770101in}{1.160291in}}%
\pgfusepath{stroke}%
\end{pgfscope}%
\begin{pgfscope}%
\pgfpathrectangle{\pgfqpoint{0.100000in}{0.212622in}}{\pgfqpoint{3.696000in}{3.696000in}}%
\pgfusepath{clip}%
\pgfsetrectcap%
\pgfsetroundjoin%
\pgfsetlinewidth{1.505625pt}%
\definecolor{currentstroke}{rgb}{1.000000,0.000000,0.000000}%
\pgfsetstrokecolor{currentstroke}%
\pgfsetdash{}{0pt}%
\pgfpathmoveto{\pgfqpoint{1.648667in}{1.851815in}}%
\pgfpathlineto{\pgfqpoint{1.741792in}{1.169187in}}%
\pgfusepath{stroke}%
\end{pgfscope}%
\begin{pgfscope}%
\pgfpathrectangle{\pgfqpoint{0.100000in}{0.212622in}}{\pgfqpoint{3.696000in}{3.696000in}}%
\pgfusepath{clip}%
\pgfsetrectcap%
\pgfsetroundjoin%
\pgfsetlinewidth{1.505625pt}%
\definecolor{currentstroke}{rgb}{1.000000,0.000000,0.000000}%
\pgfsetstrokecolor{currentstroke}%
\pgfsetdash{}{0pt}%
\pgfpathmoveto{\pgfqpoint{1.623167in}{1.866364in}}%
\pgfpathlineto{\pgfqpoint{1.713523in}{1.178071in}}%
\pgfusepath{stroke}%
\end{pgfscope}%
\begin{pgfscope}%
\pgfpathrectangle{\pgfqpoint{0.100000in}{0.212622in}}{\pgfqpoint{3.696000in}{3.696000in}}%
\pgfusepath{clip}%
\pgfsetrectcap%
\pgfsetroundjoin%
\pgfsetlinewidth{1.505625pt}%
\definecolor{currentstroke}{rgb}{1.000000,0.000000,0.000000}%
\pgfsetstrokecolor{currentstroke}%
\pgfsetdash{}{0pt}%
\pgfpathmoveto{\pgfqpoint{1.590742in}{1.899921in}}%
\pgfpathlineto{\pgfqpoint{1.685295in}{1.186942in}}%
\pgfusepath{stroke}%
\end{pgfscope}%
\begin{pgfscope}%
\pgfpathrectangle{\pgfqpoint{0.100000in}{0.212622in}}{\pgfqpoint{3.696000in}{3.696000in}}%
\pgfusepath{clip}%
\pgfsetrectcap%
\pgfsetroundjoin%
\pgfsetlinewidth{1.505625pt}%
\definecolor{currentstroke}{rgb}{1.000000,0.000000,0.000000}%
\pgfsetstrokecolor{currentstroke}%
\pgfsetdash{}{0pt}%
\pgfpathmoveto{\pgfqpoint{1.559772in}{1.916371in}}%
\pgfpathlineto{\pgfqpoint{1.657106in}{1.195800in}}%
\pgfusepath{stroke}%
\end{pgfscope}%
\begin{pgfscope}%
\pgfpathrectangle{\pgfqpoint{0.100000in}{0.212622in}}{\pgfqpoint{3.696000in}{3.696000in}}%
\pgfusepath{clip}%
\pgfsetrectcap%
\pgfsetroundjoin%
\pgfsetlinewidth{1.505625pt}%
\definecolor{currentstroke}{rgb}{1.000000,0.000000,0.000000}%
\pgfsetstrokecolor{currentstroke}%
\pgfsetdash{}{0pt}%
\pgfpathmoveto{\pgfqpoint{1.524892in}{1.937607in}}%
\pgfpathlineto{\pgfqpoint{1.628958in}{1.204645in}}%
\pgfusepath{stroke}%
\end{pgfscope}%
\begin{pgfscope}%
\pgfpathrectangle{\pgfqpoint{0.100000in}{0.212622in}}{\pgfqpoint{3.696000in}{3.696000in}}%
\pgfusepath{clip}%
\pgfsetrectcap%
\pgfsetroundjoin%
\pgfsetlinewidth{1.505625pt}%
\definecolor{currentstroke}{rgb}{1.000000,0.000000,0.000000}%
\pgfsetstrokecolor{currentstroke}%
\pgfsetdash{}{0pt}%
\pgfpathmoveto{\pgfqpoint{1.487633in}{1.960489in}}%
\pgfpathlineto{\pgfqpoint{1.586811in}{1.217890in}}%
\pgfusepath{stroke}%
\end{pgfscope}%
\begin{pgfscope}%
\pgfpathrectangle{\pgfqpoint{0.100000in}{0.212622in}}{\pgfqpoint{3.696000in}{3.696000in}}%
\pgfusepath{clip}%
\pgfsetrectcap%
\pgfsetroundjoin%
\pgfsetlinewidth{1.505625pt}%
\definecolor{currentstroke}{rgb}{1.000000,0.000000,0.000000}%
\pgfsetstrokecolor{currentstroke}%
\pgfsetdash{}{0pt}%
\pgfpathmoveto{\pgfqpoint{1.449382in}{1.993667in}}%
\pgfpathlineto{\pgfqpoint{1.544753in}{1.231107in}}%
\pgfusepath{stroke}%
\end{pgfscope}%
\begin{pgfscope}%
\pgfpathrectangle{\pgfqpoint{0.100000in}{0.212622in}}{\pgfqpoint{3.696000in}{3.696000in}}%
\pgfusepath{clip}%
\pgfsetrectcap%
\pgfsetroundjoin%
\pgfsetlinewidth{1.505625pt}%
\definecolor{currentstroke}{rgb}{1.000000,0.000000,0.000000}%
\pgfsetstrokecolor{currentstroke}%
\pgfsetdash{}{0pt}%
\pgfpathmoveto{\pgfqpoint{1.409481in}{2.013731in}}%
\pgfpathlineto{\pgfqpoint{1.502784in}{1.244295in}}%
\pgfusepath{stroke}%
\end{pgfscope}%
\begin{pgfscope}%
\pgfpathrectangle{\pgfqpoint{0.100000in}{0.212622in}}{\pgfqpoint{3.696000in}{3.696000in}}%
\pgfusepath{clip}%
\pgfsetrectcap%
\pgfsetroundjoin%
\pgfsetlinewidth{1.505625pt}%
\definecolor{currentstroke}{rgb}{1.000000,0.000000,0.000000}%
\pgfsetstrokecolor{currentstroke}%
\pgfsetdash{}{0pt}%
\pgfpathmoveto{\pgfqpoint{1.370390in}{2.047426in}}%
\pgfpathlineto{\pgfqpoint{1.460905in}{1.257456in}}%
\pgfusepath{stroke}%
\end{pgfscope}%
\begin{pgfscope}%
\pgfpathrectangle{\pgfqpoint{0.100000in}{0.212622in}}{\pgfqpoint{3.696000in}{3.696000in}}%
\pgfusepath{clip}%
\pgfsetrectcap%
\pgfsetroundjoin%
\pgfsetlinewidth{1.505625pt}%
\definecolor{currentstroke}{rgb}{1.000000,0.000000,0.000000}%
\pgfsetstrokecolor{currentstroke}%
\pgfsetdash{}{0pt}%
\pgfpathmoveto{\pgfqpoint{1.325194in}{2.081841in}}%
\pgfpathlineto{\pgfqpoint{1.419114in}{1.270589in}}%
\pgfusepath{stroke}%
\end{pgfscope}%
\begin{pgfscope}%
\pgfpathrectangle{\pgfqpoint{0.100000in}{0.212622in}}{\pgfqpoint{3.696000in}{3.696000in}}%
\pgfusepath{clip}%
\pgfsetrectcap%
\pgfsetroundjoin%
\pgfsetlinewidth{1.505625pt}%
\definecolor{currentstroke}{rgb}{1.000000,0.000000,0.000000}%
\pgfsetstrokecolor{currentstroke}%
\pgfsetdash{}{0pt}%
\pgfpathmoveto{\pgfqpoint{1.280557in}{2.113778in}}%
\pgfpathlineto{\pgfqpoint{1.377412in}{1.283693in}}%
\pgfusepath{stroke}%
\end{pgfscope}%
\begin{pgfscope}%
\pgfpathrectangle{\pgfqpoint{0.100000in}{0.212622in}}{\pgfqpoint{3.696000in}{3.696000in}}%
\pgfusepath{clip}%
\pgfsetrectcap%
\pgfsetroundjoin%
\pgfsetlinewidth{1.505625pt}%
\definecolor{currentstroke}{rgb}{1.000000,0.000000,0.000000}%
\pgfsetstrokecolor{currentstroke}%
\pgfsetdash{}{0pt}%
\pgfpathmoveto{\pgfqpoint{1.228095in}{2.152616in}}%
\pgfpathlineto{\pgfqpoint{1.321947in}{1.301124in}}%
\pgfusepath{stroke}%
\end{pgfscope}%
\begin{pgfscope}%
\pgfpathrectangle{\pgfqpoint{0.100000in}{0.212622in}}{\pgfqpoint{3.696000in}{3.696000in}}%
\pgfusepath{clip}%
\pgfsetrectcap%
\pgfsetroundjoin%
\pgfsetlinewidth{1.505625pt}%
\definecolor{currentstroke}{rgb}{1.000000,0.000000,0.000000}%
\pgfsetstrokecolor{currentstroke}%
\pgfsetdash{}{0pt}%
\pgfpathmoveto{\pgfqpoint{1.178449in}{2.182162in}}%
\pgfpathlineto{\pgfqpoint{1.280450in}{1.314164in}}%
\pgfusepath{stroke}%
\end{pgfscope}%
\begin{pgfscope}%
\pgfpathrectangle{\pgfqpoint{0.100000in}{0.212622in}}{\pgfqpoint{3.696000in}{3.696000in}}%
\pgfusepath{clip}%
\pgfsetrectcap%
\pgfsetroundjoin%
\pgfsetlinewidth{1.505625pt}%
\definecolor{currentstroke}{rgb}{1.000000,0.000000,0.000000}%
\pgfsetstrokecolor{currentstroke}%
\pgfsetdash{}{0pt}%
\pgfpathmoveto{\pgfqpoint{1.129135in}{2.223494in}}%
\pgfpathlineto{\pgfqpoint{1.225258in}{1.331508in}}%
\pgfusepath{stroke}%
\end{pgfscope}%
\begin{pgfscope}%
\pgfpathrectangle{\pgfqpoint{0.100000in}{0.212622in}}{\pgfqpoint{3.696000in}{3.696000in}}%
\pgfusepath{clip}%
\pgfsetrectcap%
\pgfsetroundjoin%
\pgfsetlinewidth{1.505625pt}%
\definecolor{currentstroke}{rgb}{1.000000,0.000000,0.000000}%
\pgfsetstrokecolor{currentstroke}%
\pgfsetdash{}{0pt}%
\pgfpathmoveto{\pgfqpoint{1.076500in}{2.266952in}}%
\pgfpathlineto{\pgfqpoint{1.170220in}{1.348804in}}%
\pgfusepath{stroke}%
\end{pgfscope}%
\begin{pgfscope}%
\pgfpathrectangle{\pgfqpoint{0.100000in}{0.212622in}}{\pgfqpoint{3.696000in}{3.696000in}}%
\pgfusepath{clip}%
\pgfsetrectcap%
\pgfsetroundjoin%
\pgfsetlinewidth{1.505625pt}%
\definecolor{currentstroke}{rgb}{1.000000,0.000000,0.000000}%
\pgfsetstrokecolor{currentstroke}%
\pgfsetdash{}{0pt}%
\pgfpathmoveto{\pgfqpoint{1.022447in}{2.312981in}}%
\pgfpathlineto{\pgfqpoint{1.115337in}{1.366051in}}%
\pgfusepath{stroke}%
\end{pgfscope}%
\begin{pgfscope}%
\pgfpathrectangle{\pgfqpoint{0.100000in}{0.212622in}}{\pgfqpoint{3.696000in}{3.696000in}}%
\pgfusepath{clip}%
\pgfsetrectcap%
\pgfsetroundjoin%
\pgfsetlinewidth{1.505625pt}%
\definecolor{currentstroke}{rgb}{1.000000,0.000000,0.000000}%
\pgfsetstrokecolor{currentstroke}%
\pgfsetdash{}{0pt}%
\pgfpathmoveto{\pgfqpoint{0.963315in}{2.368112in}}%
\pgfpathlineto{\pgfqpoint{1.060607in}{1.383249in}}%
\pgfusepath{stroke}%
\end{pgfscope}%
\begin{pgfscope}%
\pgfpathrectangle{\pgfqpoint{0.100000in}{0.212622in}}{\pgfqpoint{3.696000in}{3.696000in}}%
\pgfusepath{clip}%
\pgfsetrectcap%
\pgfsetroundjoin%
\pgfsetlinewidth{1.505625pt}%
\definecolor{currentstroke}{rgb}{1.000000,0.000000,0.000000}%
\pgfsetstrokecolor{currentstroke}%
\pgfsetdash{}{0pt}%
\pgfpathmoveto{\pgfqpoint{0.931121in}{2.389254in}}%
\pgfpathlineto{\pgfqpoint{1.019661in}{1.396117in}}%
\pgfusepath{stroke}%
\end{pgfscope}%
\begin{pgfscope}%
\pgfpathrectangle{\pgfqpoint{0.100000in}{0.212622in}}{\pgfqpoint{3.696000in}{3.696000in}}%
\pgfusepath{clip}%
\pgfsetrectcap%
\pgfsetroundjoin%
\pgfsetlinewidth{1.505625pt}%
\definecolor{currentstroke}{rgb}{1.000000,0.000000,0.000000}%
\pgfsetstrokecolor{currentstroke}%
\pgfsetdash{}{0pt}%
\pgfpathmoveto{\pgfqpoint{0.900341in}{2.431981in}}%
\pgfpathlineto{\pgfqpoint{0.992411in}{1.404680in}}%
\pgfusepath{stroke}%
\end{pgfscope}%
\begin{pgfscope}%
\pgfpathrectangle{\pgfqpoint{0.100000in}{0.212622in}}{\pgfqpoint{3.696000in}{3.696000in}}%
\pgfusepath{clip}%
\pgfsetrectcap%
\pgfsetroundjoin%
\pgfsetlinewidth{1.505625pt}%
\definecolor{currentstroke}{rgb}{1.000000,0.000000,0.000000}%
\pgfsetstrokecolor{currentstroke}%
\pgfsetdash{}{0pt}%
\pgfpathmoveto{\pgfqpoint{0.863686in}{2.449176in}}%
\pgfpathlineto{\pgfqpoint{0.951607in}{1.417503in}}%
\pgfusepath{stroke}%
\end{pgfscope}%
\begin{pgfscope}%
\pgfpathrectangle{\pgfqpoint{0.100000in}{0.212622in}}{\pgfqpoint{3.696000in}{3.696000in}}%
\pgfusepath{clip}%
\pgfsetrectcap%
\pgfsetroundjoin%
\pgfsetlinewidth{1.505625pt}%
\definecolor{currentstroke}{rgb}{1.000000,0.000000,0.000000}%
\pgfsetstrokecolor{currentstroke}%
\pgfsetdash{}{0pt}%
\pgfpathmoveto{\pgfqpoint{0.824082in}{2.475544in}}%
\pgfpathlineto{\pgfqpoint{0.910889in}{1.430298in}}%
\pgfusepath{stroke}%
\end{pgfscope}%
\begin{pgfscope}%
\pgfpathrectangle{\pgfqpoint{0.100000in}{0.212622in}}{\pgfqpoint{3.696000in}{3.696000in}}%
\pgfusepath{clip}%
\pgfsetrectcap%
\pgfsetroundjoin%
\pgfsetlinewidth{1.505625pt}%
\definecolor{currentstroke}{rgb}{1.000000,0.000000,0.000000}%
\pgfsetstrokecolor{currentstroke}%
\pgfsetdash{}{0pt}%
\pgfpathmoveto{\pgfqpoint{0.785957in}{2.507259in}}%
\pgfpathlineto{\pgfqpoint{0.883791in}{1.438814in}}%
\pgfusepath{stroke}%
\end{pgfscope}%
\begin{pgfscope}%
\pgfpathrectangle{\pgfqpoint{0.100000in}{0.212622in}}{\pgfqpoint{3.696000in}{3.696000in}}%
\pgfusepath{clip}%
\pgfsetrectcap%
\pgfsetroundjoin%
\pgfsetlinewidth{1.505625pt}%
\definecolor{currentstroke}{rgb}{1.000000,0.000000,0.000000}%
\pgfsetstrokecolor{currentstroke}%
\pgfsetdash{}{0pt}%
\pgfpathmoveto{\pgfqpoint{0.744444in}{2.552359in}}%
\pgfpathlineto{\pgfqpoint{0.883791in}{1.438814in}}%
\pgfusepath{stroke}%
\end{pgfscope}%
\begin{pgfscope}%
\pgfpathrectangle{\pgfqpoint{0.100000in}{0.212622in}}{\pgfqpoint{3.696000in}{3.696000in}}%
\pgfusepath{clip}%
\pgfsetrectcap%
\pgfsetroundjoin%
\pgfsetlinewidth{1.505625pt}%
\definecolor{currentstroke}{rgb}{1.000000,0.000000,0.000000}%
\pgfsetstrokecolor{currentstroke}%
\pgfsetdash{}{0pt}%
\pgfpathmoveto{\pgfqpoint{0.703063in}{2.594398in}}%
\pgfpathlineto{\pgfqpoint{0.883791in}{1.438814in}}%
\pgfusepath{stroke}%
\end{pgfscope}%
\begin{pgfscope}%
\pgfpathrectangle{\pgfqpoint{0.100000in}{0.212622in}}{\pgfqpoint{3.696000in}{3.696000in}}%
\pgfusepath{clip}%
\pgfsetrectcap%
\pgfsetroundjoin%
\pgfsetlinewidth{1.505625pt}%
\definecolor{currentstroke}{rgb}{1.000000,0.000000,0.000000}%
\pgfsetstrokecolor{currentstroke}%
\pgfsetdash{}{0pt}%
\pgfpathmoveto{\pgfqpoint{0.682250in}{2.601387in}}%
\pgfpathlineto{\pgfqpoint{0.883791in}{1.438814in}}%
\pgfusepath{stroke}%
\end{pgfscope}%
\begin{pgfscope}%
\pgfpathrectangle{\pgfqpoint{0.100000in}{0.212622in}}{\pgfqpoint{3.696000in}{3.696000in}}%
\pgfusepath{clip}%
\pgfsetrectcap%
\pgfsetroundjoin%
\pgfsetlinewidth{1.505625pt}%
\definecolor{currentstroke}{rgb}{1.000000,0.000000,0.000000}%
\pgfsetstrokecolor{currentstroke}%
\pgfsetdash{}{0pt}%
\pgfpathmoveto{\pgfqpoint{0.669862in}{2.614819in}}%
\pgfpathlineto{\pgfqpoint{0.883791in}{1.438814in}}%
\pgfusepath{stroke}%
\end{pgfscope}%
\begin{pgfscope}%
\pgfpathrectangle{\pgfqpoint{0.100000in}{0.212622in}}{\pgfqpoint{3.696000in}{3.696000in}}%
\pgfusepath{clip}%
\pgfsetrectcap%
\pgfsetroundjoin%
\pgfsetlinewidth{1.505625pt}%
\definecolor{currentstroke}{rgb}{1.000000,0.000000,0.000000}%
\pgfsetstrokecolor{currentstroke}%
\pgfsetdash{}{0pt}%
\pgfpathmoveto{\pgfqpoint{0.658315in}{2.620223in}}%
\pgfpathlineto{\pgfqpoint{0.883791in}{1.438814in}}%
\pgfusepath{stroke}%
\end{pgfscope}%
\begin{pgfscope}%
\pgfpathrectangle{\pgfqpoint{0.100000in}{0.212622in}}{\pgfqpoint{3.696000in}{3.696000in}}%
\pgfusepath{clip}%
\pgfsetrectcap%
\pgfsetroundjoin%
\pgfsetlinewidth{1.505625pt}%
\definecolor{currentstroke}{rgb}{1.000000,0.000000,0.000000}%
\pgfsetstrokecolor{currentstroke}%
\pgfsetdash{}{0pt}%
\pgfpathmoveto{\pgfqpoint{0.652050in}{2.627440in}}%
\pgfpathlineto{\pgfqpoint{0.883791in}{1.438814in}}%
\pgfusepath{stroke}%
\end{pgfscope}%
\begin{pgfscope}%
\pgfpathrectangle{\pgfqpoint{0.100000in}{0.212622in}}{\pgfqpoint{3.696000in}{3.696000in}}%
\pgfusepath{clip}%
\pgfsetrectcap%
\pgfsetroundjoin%
\pgfsetlinewidth{1.505625pt}%
\definecolor{currentstroke}{rgb}{1.000000,0.000000,0.000000}%
\pgfsetstrokecolor{currentstroke}%
\pgfsetdash{}{0pt}%
\pgfpathmoveto{\pgfqpoint{0.645732in}{2.629057in}}%
\pgfpathlineto{\pgfqpoint{0.883791in}{1.438814in}}%
\pgfusepath{stroke}%
\end{pgfscope}%
\begin{pgfscope}%
\pgfpathrectangle{\pgfqpoint{0.100000in}{0.212622in}}{\pgfqpoint{3.696000in}{3.696000in}}%
\pgfusepath{clip}%
\pgfsetrectcap%
\pgfsetroundjoin%
\pgfsetlinewidth{1.505625pt}%
\definecolor{currentstroke}{rgb}{1.000000,0.000000,0.000000}%
\pgfsetstrokecolor{currentstroke}%
\pgfsetdash{}{0pt}%
\pgfpathmoveto{\pgfqpoint{0.635681in}{2.647875in}}%
\pgfpathlineto{\pgfqpoint{0.883791in}{1.438814in}}%
\pgfusepath{stroke}%
\end{pgfscope}%
\begin{pgfscope}%
\pgfpathrectangle{\pgfqpoint{0.100000in}{0.212622in}}{\pgfqpoint{3.696000in}{3.696000in}}%
\pgfusepath{clip}%
\pgfsetrectcap%
\pgfsetroundjoin%
\pgfsetlinewidth{1.505625pt}%
\definecolor{currentstroke}{rgb}{1.000000,0.000000,0.000000}%
\pgfsetstrokecolor{currentstroke}%
\pgfsetdash{}{0pt}%
\pgfpathmoveto{\pgfqpoint{0.624273in}{2.653869in}}%
\pgfpathlineto{\pgfqpoint{0.883791in}{1.438814in}}%
\pgfusepath{stroke}%
\end{pgfscope}%
\begin{pgfscope}%
\pgfpathrectangle{\pgfqpoint{0.100000in}{0.212622in}}{\pgfqpoint{3.696000in}{3.696000in}}%
\pgfusepath{clip}%
\pgfsetrectcap%
\pgfsetroundjoin%
\pgfsetlinewidth{1.505625pt}%
\definecolor{currentstroke}{rgb}{1.000000,0.000000,0.000000}%
\pgfsetstrokecolor{currentstroke}%
\pgfsetdash{}{0pt}%
\pgfpathmoveto{\pgfqpoint{0.617087in}{2.670583in}}%
\pgfpathlineto{\pgfqpoint{0.883791in}{1.438814in}}%
\pgfusepath{stroke}%
\end{pgfscope}%
\begin{pgfscope}%
\pgfpathrectangle{\pgfqpoint{0.100000in}{0.212622in}}{\pgfqpoint{3.696000in}{3.696000in}}%
\pgfusepath{clip}%
\pgfsetrectcap%
\pgfsetroundjoin%
\pgfsetlinewidth{1.505625pt}%
\definecolor{currentstroke}{rgb}{1.000000,0.000000,0.000000}%
\pgfsetstrokecolor{currentstroke}%
\pgfsetdash{}{0pt}%
\pgfpathmoveto{\pgfqpoint{0.614009in}{2.669780in}}%
\pgfpathlineto{\pgfqpoint{0.883791in}{1.438814in}}%
\pgfusepath{stroke}%
\end{pgfscope}%
\begin{pgfscope}%
\pgfpathrectangle{\pgfqpoint{0.100000in}{0.212622in}}{\pgfqpoint{3.696000in}{3.696000in}}%
\pgfusepath{clip}%
\pgfsetrectcap%
\pgfsetroundjoin%
\pgfsetlinewidth{1.505625pt}%
\definecolor{currentstroke}{rgb}{1.000000,0.000000,0.000000}%
\pgfsetstrokecolor{currentstroke}%
\pgfsetdash{}{0pt}%
\pgfpathmoveto{\pgfqpoint{0.611889in}{2.672268in}}%
\pgfpathlineto{\pgfqpoint{0.883791in}{1.438814in}}%
\pgfusepath{stroke}%
\end{pgfscope}%
\begin{pgfscope}%
\pgfpathrectangle{\pgfqpoint{0.100000in}{0.212622in}}{\pgfqpoint{3.696000in}{3.696000in}}%
\pgfusepath{clip}%
\pgfsetrectcap%
\pgfsetroundjoin%
\pgfsetlinewidth{1.505625pt}%
\definecolor{currentstroke}{rgb}{1.000000,0.000000,0.000000}%
\pgfsetstrokecolor{currentstroke}%
\pgfsetdash{}{0pt}%
\pgfpathmoveto{\pgfqpoint{0.610810in}{2.671855in}}%
\pgfpathlineto{\pgfqpoint{0.883791in}{1.438814in}}%
\pgfusepath{stroke}%
\end{pgfscope}%
\begin{pgfscope}%
\pgfpathrectangle{\pgfqpoint{0.100000in}{0.212622in}}{\pgfqpoint{3.696000in}{3.696000in}}%
\pgfusepath{clip}%
\pgfsetrectcap%
\pgfsetroundjoin%
\pgfsetlinewidth{1.505625pt}%
\definecolor{currentstroke}{rgb}{1.000000,0.000000,0.000000}%
\pgfsetstrokecolor{currentstroke}%
\pgfsetdash{}{0pt}%
\pgfpathmoveto{\pgfqpoint{0.610257in}{2.671864in}}%
\pgfpathlineto{\pgfqpoint{0.883791in}{1.438814in}}%
\pgfusepath{stroke}%
\end{pgfscope}%
\begin{pgfscope}%
\pgfpathrectangle{\pgfqpoint{0.100000in}{0.212622in}}{\pgfqpoint{3.696000in}{3.696000in}}%
\pgfusepath{clip}%
\pgfsetrectcap%
\pgfsetroundjoin%
\pgfsetlinewidth{1.505625pt}%
\definecolor{currentstroke}{rgb}{1.000000,0.000000,0.000000}%
\pgfsetstrokecolor{currentstroke}%
\pgfsetdash{}{0pt}%
\pgfpathmoveto{\pgfqpoint{0.607750in}{2.670124in}}%
\pgfpathlineto{\pgfqpoint{0.883791in}{1.438814in}}%
\pgfusepath{stroke}%
\end{pgfscope}%
\begin{pgfscope}%
\pgfpathrectangle{\pgfqpoint{0.100000in}{0.212622in}}{\pgfqpoint{3.696000in}{3.696000in}}%
\pgfusepath{clip}%
\pgfsetrectcap%
\pgfsetroundjoin%
\pgfsetlinewidth{1.505625pt}%
\definecolor{currentstroke}{rgb}{1.000000,0.000000,0.000000}%
\pgfsetstrokecolor{currentstroke}%
\pgfsetdash{}{0pt}%
\pgfpathmoveto{\pgfqpoint{0.603932in}{2.667474in}}%
\pgfpathlineto{\pgfqpoint{0.883791in}{1.438814in}}%
\pgfusepath{stroke}%
\end{pgfscope}%
\begin{pgfscope}%
\pgfpathrectangle{\pgfqpoint{0.100000in}{0.212622in}}{\pgfqpoint{3.696000in}{3.696000in}}%
\pgfusepath{clip}%
\pgfsetbuttcap%
\pgfsetroundjoin%
\definecolor{currentfill}{rgb}{0.121569,0.466667,0.705882}%
\pgfsetfillcolor{currentfill}%
\pgfsetfillopacity{0.300000}%
\pgfsetlinewidth{1.003750pt}%
\definecolor{currentstroke}{rgb}{0.121569,0.466667,0.705882}%
\pgfsetstrokecolor{currentstroke}%
\pgfsetstrokeopacity{0.300000}%
\pgfsetdash{}{0pt}%
\pgfpathmoveto{\pgfqpoint{1.950563in}{2.023644in}}%
\pgfpathcurveto{\pgfqpoint{1.958799in}{2.023644in}}{\pgfqpoint{1.966699in}{2.026917in}}{\pgfqpoint{1.972523in}{2.032740in}}%
\pgfpathcurveto{\pgfqpoint{1.978347in}{2.038564in}}{\pgfqpoint{1.981619in}{2.046464in}}{\pgfqpoint{1.981619in}{2.054701in}}%
\pgfpathcurveto{\pgfqpoint{1.981619in}{2.062937in}}{\pgfqpoint{1.978347in}{2.070837in}}{\pgfqpoint{1.972523in}{2.076661in}}%
\pgfpathcurveto{\pgfqpoint{1.966699in}{2.082485in}}{\pgfqpoint{1.958799in}{2.085757in}}{\pgfqpoint{1.950563in}{2.085757in}}%
\pgfpathcurveto{\pgfqpoint{1.942326in}{2.085757in}}{\pgfqpoint{1.934426in}{2.082485in}}{\pgfqpoint{1.928603in}{2.076661in}}%
\pgfpathcurveto{\pgfqpoint{1.922779in}{2.070837in}}{\pgfqpoint{1.919506in}{2.062937in}}{\pgfqpoint{1.919506in}{2.054701in}}%
\pgfpathcurveto{\pgfqpoint{1.919506in}{2.046464in}}{\pgfqpoint{1.922779in}{2.038564in}}{\pgfqpoint{1.928603in}{2.032740in}}%
\pgfpathcurveto{\pgfqpoint{1.934426in}{2.026917in}}{\pgfqpoint{1.942326in}{2.023644in}}{\pgfqpoint{1.950563in}{2.023644in}}%
\pgfpathclose%
\pgfusepath{stroke,fill}%
\end{pgfscope}%
\begin{pgfscope}%
\pgfpathrectangle{\pgfqpoint{0.100000in}{0.212622in}}{\pgfqpoint{3.696000in}{3.696000in}}%
\pgfusepath{clip}%
\pgfsetbuttcap%
\pgfsetroundjoin%
\definecolor{currentfill}{rgb}{0.121569,0.466667,0.705882}%
\pgfsetfillcolor{currentfill}%
\pgfsetfillopacity{0.300003}%
\pgfsetlinewidth{1.003750pt}%
\definecolor{currentstroke}{rgb}{0.121569,0.466667,0.705882}%
\pgfsetstrokecolor{currentstroke}%
\pgfsetstrokeopacity{0.300003}%
\pgfsetdash{}{0pt}%
\pgfpathmoveto{\pgfqpoint{1.950916in}{2.023264in}}%
\pgfpathcurveto{\pgfqpoint{1.959152in}{2.023264in}}{\pgfqpoint{1.967052in}{2.026537in}}{\pgfqpoint{1.972876in}{2.032361in}}%
\pgfpathcurveto{\pgfqpoint{1.978700in}{2.038185in}}{\pgfqpoint{1.981972in}{2.046085in}}{\pgfqpoint{1.981972in}{2.054321in}}%
\pgfpathcurveto{\pgfqpoint{1.981972in}{2.062557in}}{\pgfqpoint{1.978700in}{2.070457in}}{\pgfqpoint{1.972876in}{2.076281in}}%
\pgfpathcurveto{\pgfqpoint{1.967052in}{2.082105in}}{\pgfqpoint{1.959152in}{2.085377in}}{\pgfqpoint{1.950916in}{2.085377in}}%
\pgfpathcurveto{\pgfqpoint{1.942679in}{2.085377in}}{\pgfqpoint{1.934779in}{2.082105in}}{\pgfqpoint{1.928955in}{2.076281in}}%
\pgfpathcurveto{\pgfqpoint{1.923132in}{2.070457in}}{\pgfqpoint{1.919859in}{2.062557in}}{\pgfqpoint{1.919859in}{2.054321in}}%
\pgfpathcurveto{\pgfqpoint{1.919859in}{2.046085in}}{\pgfqpoint{1.923132in}{2.038185in}}{\pgfqpoint{1.928955in}{2.032361in}}%
\pgfpathcurveto{\pgfqpoint{1.934779in}{2.026537in}}{\pgfqpoint{1.942679in}{2.023264in}}{\pgfqpoint{1.950916in}{2.023264in}}%
\pgfpathclose%
\pgfusepath{stroke,fill}%
\end{pgfscope}%
\begin{pgfscope}%
\pgfpathrectangle{\pgfqpoint{0.100000in}{0.212622in}}{\pgfqpoint{3.696000in}{3.696000in}}%
\pgfusepath{clip}%
\pgfsetbuttcap%
\pgfsetroundjoin%
\definecolor{currentfill}{rgb}{0.121569,0.466667,0.705882}%
\pgfsetfillcolor{currentfill}%
\pgfsetfillopacity{0.300012}%
\pgfsetlinewidth{1.003750pt}%
\definecolor{currentstroke}{rgb}{0.121569,0.466667,0.705882}%
\pgfsetstrokecolor{currentstroke}%
\pgfsetstrokeopacity{0.300012}%
\pgfsetdash{}{0pt}%
\pgfpathmoveto{\pgfqpoint{1.949818in}{2.023878in}}%
\pgfpathcurveto{\pgfqpoint{1.958055in}{2.023878in}}{\pgfqpoint{1.965955in}{2.027151in}}{\pgfqpoint{1.971779in}{2.032975in}}%
\pgfpathcurveto{\pgfqpoint{1.977603in}{2.038799in}}{\pgfqpoint{1.980875in}{2.046699in}}{\pgfqpoint{1.980875in}{2.054935in}}%
\pgfpathcurveto{\pgfqpoint{1.980875in}{2.063171in}}{\pgfqpoint{1.977603in}{2.071071in}}{\pgfqpoint{1.971779in}{2.076895in}}%
\pgfpathcurveto{\pgfqpoint{1.965955in}{2.082719in}}{\pgfqpoint{1.958055in}{2.085991in}}{\pgfqpoint{1.949818in}{2.085991in}}%
\pgfpathcurveto{\pgfqpoint{1.941582in}{2.085991in}}{\pgfqpoint{1.933682in}{2.082719in}}{\pgfqpoint{1.927858in}{2.076895in}}%
\pgfpathcurveto{\pgfqpoint{1.922034in}{2.071071in}}{\pgfqpoint{1.918762in}{2.063171in}}{\pgfqpoint{1.918762in}{2.054935in}}%
\pgfpathcurveto{\pgfqpoint{1.918762in}{2.046699in}}{\pgfqpoint{1.922034in}{2.038799in}}{\pgfqpoint{1.927858in}{2.032975in}}%
\pgfpathcurveto{\pgfqpoint{1.933682in}{2.027151in}}{\pgfqpoint{1.941582in}{2.023878in}}{\pgfqpoint{1.949818in}{2.023878in}}%
\pgfpathclose%
\pgfusepath{stroke,fill}%
\end{pgfscope}%
\begin{pgfscope}%
\pgfpathrectangle{\pgfqpoint{0.100000in}{0.212622in}}{\pgfqpoint{3.696000in}{3.696000in}}%
\pgfusepath{clip}%
\pgfsetbuttcap%
\pgfsetroundjoin%
\definecolor{currentfill}{rgb}{0.121569,0.466667,0.705882}%
\pgfsetfillcolor{currentfill}%
\pgfsetfillopacity{0.300033}%
\pgfsetlinewidth{1.003750pt}%
\definecolor{currentstroke}{rgb}{0.121569,0.466667,0.705882}%
\pgfsetstrokecolor{currentstroke}%
\pgfsetstrokeopacity{0.300033}%
\pgfsetdash{}{0pt}%
\pgfpathmoveto{\pgfqpoint{1.951094in}{2.023206in}}%
\pgfpathcurveto{\pgfqpoint{1.959330in}{2.023206in}}{\pgfqpoint{1.967230in}{2.026478in}}{\pgfqpoint{1.973054in}{2.032302in}}%
\pgfpathcurveto{\pgfqpoint{1.978878in}{2.038126in}}{\pgfqpoint{1.982150in}{2.046026in}}{\pgfqpoint{1.982150in}{2.054262in}}%
\pgfpathcurveto{\pgfqpoint{1.982150in}{2.062499in}}{\pgfqpoint{1.978878in}{2.070399in}}{\pgfqpoint{1.973054in}{2.076223in}}%
\pgfpathcurveto{\pgfqpoint{1.967230in}{2.082047in}}{\pgfqpoint{1.959330in}{2.085319in}}{\pgfqpoint{1.951094in}{2.085319in}}%
\pgfpathcurveto{\pgfqpoint{1.942858in}{2.085319in}}{\pgfqpoint{1.934958in}{2.082047in}}{\pgfqpoint{1.929134in}{2.076223in}}%
\pgfpathcurveto{\pgfqpoint{1.923310in}{2.070399in}}{\pgfqpoint{1.920037in}{2.062499in}}{\pgfqpoint{1.920037in}{2.054262in}}%
\pgfpathcurveto{\pgfqpoint{1.920037in}{2.046026in}}{\pgfqpoint{1.923310in}{2.038126in}}{\pgfqpoint{1.929134in}{2.032302in}}%
\pgfpathcurveto{\pgfqpoint{1.934958in}{2.026478in}}{\pgfqpoint{1.942858in}{2.023206in}}{\pgfqpoint{1.951094in}{2.023206in}}%
\pgfpathclose%
\pgfusepath{stroke,fill}%
\end{pgfscope}%
\begin{pgfscope}%
\pgfpathrectangle{\pgfqpoint{0.100000in}{0.212622in}}{\pgfqpoint{3.696000in}{3.696000in}}%
\pgfusepath{clip}%
\pgfsetbuttcap%
\pgfsetroundjoin%
\definecolor{currentfill}{rgb}{0.121569,0.466667,0.705882}%
\pgfsetfillcolor{currentfill}%
\pgfsetfillopacity{0.300037}%
\pgfsetlinewidth{1.003750pt}%
\definecolor{currentstroke}{rgb}{0.121569,0.466667,0.705882}%
\pgfsetstrokecolor{currentstroke}%
\pgfsetstrokeopacity{0.300037}%
\pgfsetdash{}{0pt}%
\pgfpathmoveto{\pgfqpoint{1.949612in}{2.024022in}}%
\pgfpathcurveto{\pgfqpoint{1.957849in}{2.024022in}}{\pgfqpoint{1.965749in}{2.027295in}}{\pgfqpoint{1.971573in}{2.033119in}}%
\pgfpathcurveto{\pgfqpoint{1.977397in}{2.038943in}}{\pgfqpoint{1.980669in}{2.046843in}}{\pgfqpoint{1.980669in}{2.055079in}}%
\pgfpathcurveto{\pgfqpoint{1.980669in}{2.063315in}}{\pgfqpoint{1.977397in}{2.071215in}}{\pgfqpoint{1.971573in}{2.077039in}}%
\pgfpathcurveto{\pgfqpoint{1.965749in}{2.082863in}}{\pgfqpoint{1.957849in}{2.086135in}}{\pgfqpoint{1.949612in}{2.086135in}}%
\pgfpathcurveto{\pgfqpoint{1.941376in}{2.086135in}}{\pgfqpoint{1.933476in}{2.082863in}}{\pgfqpoint{1.927652in}{2.077039in}}%
\pgfpathcurveto{\pgfqpoint{1.921828in}{2.071215in}}{\pgfqpoint{1.918556in}{2.063315in}}{\pgfqpoint{1.918556in}{2.055079in}}%
\pgfpathcurveto{\pgfqpoint{1.918556in}{2.046843in}}{\pgfqpoint{1.921828in}{2.038943in}}{\pgfqpoint{1.927652in}{2.033119in}}%
\pgfpathcurveto{\pgfqpoint{1.933476in}{2.027295in}}{\pgfqpoint{1.941376in}{2.024022in}}{\pgfqpoint{1.949612in}{2.024022in}}%
\pgfpathclose%
\pgfusepath{stroke,fill}%
\end{pgfscope}%
\begin{pgfscope}%
\pgfpathrectangle{\pgfqpoint{0.100000in}{0.212622in}}{\pgfqpoint{3.696000in}{3.696000in}}%
\pgfusepath{clip}%
\pgfsetbuttcap%
\pgfsetroundjoin%
\definecolor{currentfill}{rgb}{0.121569,0.466667,0.705882}%
\pgfsetfillcolor{currentfill}%
\pgfsetfillopacity{0.300062}%
\pgfsetlinewidth{1.003750pt}%
\definecolor{currentstroke}{rgb}{0.121569,0.466667,0.705882}%
\pgfsetstrokecolor{currentstroke}%
\pgfsetstrokeopacity{0.300062}%
\pgfsetdash{}{0pt}%
\pgfpathmoveto{\pgfqpoint{1.951168in}{2.023210in}}%
\pgfpathcurveto{\pgfqpoint{1.959404in}{2.023210in}}{\pgfqpoint{1.967304in}{2.026482in}}{\pgfqpoint{1.973128in}{2.032306in}}%
\pgfpathcurveto{\pgfqpoint{1.978952in}{2.038130in}}{\pgfqpoint{1.982224in}{2.046030in}}{\pgfqpoint{1.982224in}{2.054267in}}%
\pgfpathcurveto{\pgfqpoint{1.982224in}{2.062503in}}{\pgfqpoint{1.978952in}{2.070403in}}{\pgfqpoint{1.973128in}{2.076227in}}%
\pgfpathcurveto{\pgfqpoint{1.967304in}{2.082051in}}{\pgfqpoint{1.959404in}{2.085323in}}{\pgfqpoint{1.951168in}{2.085323in}}%
\pgfpathcurveto{\pgfqpoint{1.942932in}{2.085323in}}{\pgfqpoint{1.935032in}{2.082051in}}{\pgfqpoint{1.929208in}{2.076227in}}%
\pgfpathcurveto{\pgfqpoint{1.923384in}{2.070403in}}{\pgfqpoint{1.920111in}{2.062503in}}{\pgfqpoint{1.920111in}{2.054267in}}%
\pgfpathcurveto{\pgfqpoint{1.920111in}{2.046030in}}{\pgfqpoint{1.923384in}{2.038130in}}{\pgfqpoint{1.929208in}{2.032306in}}%
\pgfpathcurveto{\pgfqpoint{1.935032in}{2.026482in}}{\pgfqpoint{1.942932in}{2.023210in}}{\pgfqpoint{1.951168in}{2.023210in}}%
\pgfpathclose%
\pgfusepath{stroke,fill}%
\end{pgfscope}%
\begin{pgfscope}%
\pgfpathrectangle{\pgfqpoint{0.100000in}{0.212622in}}{\pgfqpoint{3.696000in}{3.696000in}}%
\pgfusepath{clip}%
\pgfsetbuttcap%
\pgfsetroundjoin%
\definecolor{currentfill}{rgb}{0.121569,0.466667,0.705882}%
\pgfsetfillcolor{currentfill}%
\pgfsetfillopacity{0.300071}%
\pgfsetlinewidth{1.003750pt}%
\definecolor{currentstroke}{rgb}{0.121569,0.466667,0.705882}%
\pgfsetstrokecolor{currentstroke}%
\pgfsetstrokeopacity{0.300071}%
\pgfsetdash{}{0pt}%
\pgfpathmoveto{\pgfqpoint{1.949252in}{2.023993in}}%
\pgfpathcurveto{\pgfqpoint{1.957488in}{2.023993in}}{\pgfqpoint{1.965388in}{2.027265in}}{\pgfqpoint{1.971212in}{2.033089in}}%
\pgfpathcurveto{\pgfqpoint{1.977036in}{2.038913in}}{\pgfqpoint{1.980308in}{2.046813in}}{\pgfqpoint{1.980308in}{2.055049in}}%
\pgfpathcurveto{\pgfqpoint{1.980308in}{2.063286in}}{\pgfqpoint{1.977036in}{2.071186in}}{\pgfqpoint{1.971212in}{2.077010in}}%
\pgfpathcurveto{\pgfqpoint{1.965388in}{2.082834in}}{\pgfqpoint{1.957488in}{2.086106in}}{\pgfqpoint{1.949252in}{2.086106in}}%
\pgfpathcurveto{\pgfqpoint{1.941015in}{2.086106in}}{\pgfqpoint{1.933115in}{2.082834in}}{\pgfqpoint{1.927291in}{2.077010in}}%
\pgfpathcurveto{\pgfqpoint{1.921468in}{2.071186in}}{\pgfqpoint{1.918195in}{2.063286in}}{\pgfqpoint{1.918195in}{2.055049in}}%
\pgfpathcurveto{\pgfqpoint{1.918195in}{2.046813in}}{\pgfqpoint{1.921468in}{2.038913in}}{\pgfqpoint{1.927291in}{2.033089in}}%
\pgfpathcurveto{\pgfqpoint{1.933115in}{2.027265in}}{\pgfqpoint{1.941015in}{2.023993in}}{\pgfqpoint{1.949252in}{2.023993in}}%
\pgfpathclose%
\pgfusepath{stroke,fill}%
\end{pgfscope}%
\begin{pgfscope}%
\pgfpathrectangle{\pgfqpoint{0.100000in}{0.212622in}}{\pgfqpoint{3.696000in}{3.696000in}}%
\pgfusepath{clip}%
\pgfsetbuttcap%
\pgfsetroundjoin%
\definecolor{currentfill}{rgb}{0.121569,0.466667,0.705882}%
\pgfsetfillcolor{currentfill}%
\pgfsetfillopacity{0.300077}%
\pgfsetlinewidth{1.003750pt}%
\definecolor{currentstroke}{rgb}{0.121569,0.466667,0.705882}%
\pgfsetstrokecolor{currentstroke}%
\pgfsetstrokeopacity{0.300077}%
\pgfsetdash{}{0pt}%
\pgfpathmoveto{\pgfqpoint{1.951195in}{2.023189in}}%
\pgfpathcurveto{\pgfqpoint{1.959431in}{2.023189in}}{\pgfqpoint{1.967331in}{2.026461in}}{\pgfqpoint{1.973155in}{2.032285in}}%
\pgfpathcurveto{\pgfqpoint{1.978979in}{2.038109in}}{\pgfqpoint{1.982251in}{2.046009in}}{\pgfqpoint{1.982251in}{2.054246in}}%
\pgfpathcurveto{\pgfqpoint{1.982251in}{2.062482in}}{\pgfqpoint{1.978979in}{2.070382in}}{\pgfqpoint{1.973155in}{2.076206in}}%
\pgfpathcurveto{\pgfqpoint{1.967331in}{2.082030in}}{\pgfqpoint{1.959431in}{2.085302in}}{\pgfqpoint{1.951195in}{2.085302in}}%
\pgfpathcurveto{\pgfqpoint{1.942958in}{2.085302in}}{\pgfqpoint{1.935058in}{2.082030in}}{\pgfqpoint{1.929234in}{2.076206in}}%
\pgfpathcurveto{\pgfqpoint{1.923410in}{2.070382in}}{\pgfqpoint{1.920138in}{2.062482in}}{\pgfqpoint{1.920138in}{2.054246in}}%
\pgfpathcurveto{\pgfqpoint{1.920138in}{2.046009in}}{\pgfqpoint{1.923410in}{2.038109in}}{\pgfqpoint{1.929234in}{2.032285in}}%
\pgfpathcurveto{\pgfqpoint{1.935058in}{2.026461in}}{\pgfqpoint{1.942958in}{2.023189in}}{\pgfqpoint{1.951195in}{2.023189in}}%
\pgfpathclose%
\pgfusepath{stroke,fill}%
\end{pgfscope}%
\begin{pgfscope}%
\pgfpathrectangle{\pgfqpoint{0.100000in}{0.212622in}}{\pgfqpoint{3.696000in}{3.696000in}}%
\pgfusepath{clip}%
\pgfsetbuttcap%
\pgfsetroundjoin%
\definecolor{currentfill}{rgb}{0.121569,0.466667,0.705882}%
\pgfsetfillcolor{currentfill}%
\pgfsetfillopacity{0.300087}%
\pgfsetlinewidth{1.003750pt}%
\definecolor{currentstroke}{rgb}{0.121569,0.466667,0.705882}%
\pgfsetstrokecolor{currentstroke}%
\pgfsetstrokeopacity{0.300087}%
\pgfsetdash{}{0pt}%
\pgfpathmoveto{\pgfqpoint{1.951206in}{2.023184in}}%
\pgfpathcurveto{\pgfqpoint{1.959442in}{2.023184in}}{\pgfqpoint{1.967342in}{2.026456in}}{\pgfqpoint{1.973166in}{2.032280in}}%
\pgfpathcurveto{\pgfqpoint{1.978990in}{2.038104in}}{\pgfqpoint{1.982262in}{2.046004in}}{\pgfqpoint{1.982262in}{2.054240in}}%
\pgfpathcurveto{\pgfqpoint{1.982262in}{2.062477in}}{\pgfqpoint{1.978990in}{2.070377in}}{\pgfqpoint{1.973166in}{2.076201in}}%
\pgfpathcurveto{\pgfqpoint{1.967342in}{2.082025in}}{\pgfqpoint{1.959442in}{2.085297in}}{\pgfqpoint{1.951206in}{2.085297in}}%
\pgfpathcurveto{\pgfqpoint{1.942970in}{2.085297in}}{\pgfqpoint{1.935070in}{2.082025in}}{\pgfqpoint{1.929246in}{2.076201in}}%
\pgfpathcurveto{\pgfqpoint{1.923422in}{2.070377in}}{\pgfqpoint{1.920149in}{2.062477in}}{\pgfqpoint{1.920149in}{2.054240in}}%
\pgfpathcurveto{\pgfqpoint{1.920149in}{2.046004in}}{\pgfqpoint{1.923422in}{2.038104in}}{\pgfqpoint{1.929246in}{2.032280in}}%
\pgfpathcurveto{\pgfqpoint{1.935070in}{2.026456in}}{\pgfqpoint{1.942970in}{2.023184in}}{\pgfqpoint{1.951206in}{2.023184in}}%
\pgfpathclose%
\pgfusepath{stroke,fill}%
\end{pgfscope}%
\begin{pgfscope}%
\pgfpathrectangle{\pgfqpoint{0.100000in}{0.212622in}}{\pgfqpoint{3.696000in}{3.696000in}}%
\pgfusepath{clip}%
\pgfsetbuttcap%
\pgfsetroundjoin%
\definecolor{currentfill}{rgb}{0.121569,0.466667,0.705882}%
\pgfsetfillcolor{currentfill}%
\pgfsetfillopacity{0.300233}%
\pgfsetlinewidth{1.003750pt}%
\definecolor{currentstroke}{rgb}{0.121569,0.466667,0.705882}%
\pgfsetstrokecolor{currentstroke}%
\pgfsetstrokeopacity{0.300233}%
\pgfsetdash{}{0pt}%
\pgfpathmoveto{\pgfqpoint{1.951269in}{2.022907in}}%
\pgfpathcurveto{\pgfqpoint{1.959505in}{2.022907in}}{\pgfqpoint{1.967405in}{2.026180in}}{\pgfqpoint{1.973229in}{2.032004in}}%
\pgfpathcurveto{\pgfqpoint{1.979053in}{2.037827in}}{\pgfqpoint{1.982326in}{2.045728in}}{\pgfqpoint{1.982326in}{2.053964in}}%
\pgfpathcurveto{\pgfqpoint{1.982326in}{2.062200in}}{\pgfqpoint{1.979053in}{2.070100in}}{\pgfqpoint{1.973229in}{2.075924in}}%
\pgfpathcurveto{\pgfqpoint{1.967405in}{2.081748in}}{\pgfqpoint{1.959505in}{2.085020in}}{\pgfqpoint{1.951269in}{2.085020in}}%
\pgfpathcurveto{\pgfqpoint{1.943033in}{2.085020in}}{\pgfqpoint{1.935133in}{2.081748in}}{\pgfqpoint{1.929309in}{2.075924in}}%
\pgfpathcurveto{\pgfqpoint{1.923485in}{2.070100in}}{\pgfqpoint{1.920213in}{2.062200in}}{\pgfqpoint{1.920213in}{2.053964in}}%
\pgfpathcurveto{\pgfqpoint{1.920213in}{2.045728in}}{\pgfqpoint{1.923485in}{2.037827in}}{\pgfqpoint{1.929309in}{2.032004in}}%
\pgfpathcurveto{\pgfqpoint{1.935133in}{2.026180in}}{\pgfqpoint{1.943033in}{2.022907in}}{\pgfqpoint{1.951269in}{2.022907in}}%
\pgfpathclose%
\pgfusepath{stroke,fill}%
\end{pgfscope}%
\begin{pgfscope}%
\pgfpathrectangle{\pgfqpoint{0.100000in}{0.212622in}}{\pgfqpoint{3.696000in}{3.696000in}}%
\pgfusepath{clip}%
\pgfsetbuttcap%
\pgfsetroundjoin%
\definecolor{currentfill}{rgb}{0.121569,0.466667,0.705882}%
\pgfsetfillcolor{currentfill}%
\pgfsetfillopacity{0.300275}%
\pgfsetlinewidth{1.003750pt}%
\definecolor{currentstroke}{rgb}{0.121569,0.466667,0.705882}%
\pgfsetstrokecolor{currentstroke}%
\pgfsetstrokeopacity{0.300275}%
\pgfsetdash{}{0pt}%
\pgfpathmoveto{\pgfqpoint{1.948642in}{2.024677in}}%
\pgfpathcurveto{\pgfqpoint{1.956879in}{2.024677in}}{\pgfqpoint{1.964779in}{2.027949in}}{\pgfqpoint{1.970603in}{2.033773in}}%
\pgfpathcurveto{\pgfqpoint{1.976426in}{2.039597in}}{\pgfqpoint{1.979699in}{2.047497in}}{\pgfqpoint{1.979699in}{2.055733in}}%
\pgfpathcurveto{\pgfqpoint{1.979699in}{2.063970in}}{\pgfqpoint{1.976426in}{2.071870in}}{\pgfqpoint{1.970603in}{2.077694in}}%
\pgfpathcurveto{\pgfqpoint{1.964779in}{2.083518in}}{\pgfqpoint{1.956879in}{2.086790in}}{\pgfqpoint{1.948642in}{2.086790in}}%
\pgfpathcurveto{\pgfqpoint{1.940406in}{2.086790in}}{\pgfqpoint{1.932506in}{2.083518in}}{\pgfqpoint{1.926682in}{2.077694in}}%
\pgfpathcurveto{\pgfqpoint{1.920858in}{2.071870in}}{\pgfqpoint{1.917586in}{2.063970in}}{\pgfqpoint{1.917586in}{2.055733in}}%
\pgfpathcurveto{\pgfqpoint{1.917586in}{2.047497in}}{\pgfqpoint{1.920858in}{2.039597in}}{\pgfqpoint{1.926682in}{2.033773in}}%
\pgfpathcurveto{\pgfqpoint{1.932506in}{2.027949in}}{\pgfqpoint{1.940406in}{2.024677in}}{\pgfqpoint{1.948642in}{2.024677in}}%
\pgfpathclose%
\pgfusepath{stroke,fill}%
\end{pgfscope}%
\begin{pgfscope}%
\pgfpathrectangle{\pgfqpoint{0.100000in}{0.212622in}}{\pgfqpoint{3.696000in}{3.696000in}}%
\pgfusepath{clip}%
\pgfsetbuttcap%
\pgfsetroundjoin%
\definecolor{currentfill}{rgb}{0.121569,0.466667,0.705882}%
\pgfsetfillcolor{currentfill}%
\pgfsetfillopacity{0.300289}%
\pgfsetlinewidth{1.003750pt}%
\definecolor{currentstroke}{rgb}{0.121569,0.466667,0.705882}%
\pgfsetstrokecolor{currentstroke}%
\pgfsetstrokeopacity{0.300289}%
\pgfsetdash{}{0pt}%
\pgfpathmoveto{\pgfqpoint{1.948583in}{2.024686in}}%
\pgfpathcurveto{\pgfqpoint{1.956819in}{2.024686in}}{\pgfqpoint{1.964719in}{2.027958in}}{\pgfqpoint{1.970543in}{2.033782in}}%
\pgfpathcurveto{\pgfqpoint{1.976367in}{2.039606in}}{\pgfqpoint{1.979639in}{2.047506in}}{\pgfqpoint{1.979639in}{2.055742in}}%
\pgfpathcurveto{\pgfqpoint{1.979639in}{2.063979in}}{\pgfqpoint{1.976367in}{2.071879in}}{\pgfqpoint{1.970543in}{2.077703in}}%
\pgfpathcurveto{\pgfqpoint{1.964719in}{2.083527in}}{\pgfqpoint{1.956819in}{2.086799in}}{\pgfqpoint{1.948583in}{2.086799in}}%
\pgfpathcurveto{\pgfqpoint{1.940346in}{2.086799in}}{\pgfqpoint{1.932446in}{2.083527in}}{\pgfqpoint{1.926622in}{2.077703in}}%
\pgfpathcurveto{\pgfqpoint{1.920799in}{2.071879in}}{\pgfqpoint{1.917526in}{2.063979in}}{\pgfqpoint{1.917526in}{2.055742in}}%
\pgfpathcurveto{\pgfqpoint{1.917526in}{2.047506in}}{\pgfqpoint{1.920799in}{2.039606in}}{\pgfqpoint{1.926622in}{2.033782in}}%
\pgfpathcurveto{\pgfqpoint{1.932446in}{2.027958in}}{\pgfqpoint{1.940346in}{2.024686in}}{\pgfqpoint{1.948583in}{2.024686in}}%
\pgfpathclose%
\pgfusepath{stroke,fill}%
\end{pgfscope}%
\begin{pgfscope}%
\pgfpathrectangle{\pgfqpoint{0.100000in}{0.212622in}}{\pgfqpoint{3.696000in}{3.696000in}}%
\pgfusepath{clip}%
\pgfsetbuttcap%
\pgfsetroundjoin%
\definecolor{currentfill}{rgb}{0.121569,0.466667,0.705882}%
\pgfsetfillcolor{currentfill}%
\pgfsetfillopacity{0.300314}%
\pgfsetlinewidth{1.003750pt}%
\definecolor{currentstroke}{rgb}{0.121569,0.466667,0.705882}%
\pgfsetstrokecolor{currentstroke}%
\pgfsetstrokeopacity{0.300314}%
\pgfsetdash{}{0pt}%
\pgfpathmoveto{\pgfqpoint{1.948492in}{2.024656in}}%
\pgfpathcurveto{\pgfqpoint{1.956728in}{2.024656in}}{\pgfqpoint{1.964628in}{2.027929in}}{\pgfqpoint{1.970452in}{2.033752in}}%
\pgfpathcurveto{\pgfqpoint{1.976276in}{2.039576in}}{\pgfqpoint{1.979548in}{2.047476in}}{\pgfqpoint{1.979548in}{2.055713in}}%
\pgfpathcurveto{\pgfqpoint{1.979548in}{2.063949in}}{\pgfqpoint{1.976276in}{2.071849in}}{\pgfqpoint{1.970452in}{2.077673in}}%
\pgfpathcurveto{\pgfqpoint{1.964628in}{2.083497in}}{\pgfqpoint{1.956728in}{2.086769in}}{\pgfqpoint{1.948492in}{2.086769in}}%
\pgfpathcurveto{\pgfqpoint{1.940255in}{2.086769in}}{\pgfqpoint{1.932355in}{2.083497in}}{\pgfqpoint{1.926531in}{2.077673in}}%
\pgfpathcurveto{\pgfqpoint{1.920707in}{2.071849in}}{\pgfqpoint{1.917435in}{2.063949in}}{\pgfqpoint{1.917435in}{2.055713in}}%
\pgfpathcurveto{\pgfqpoint{1.917435in}{2.047476in}}{\pgfqpoint{1.920707in}{2.039576in}}{\pgfqpoint{1.926531in}{2.033752in}}%
\pgfpathcurveto{\pgfqpoint{1.932355in}{2.027929in}}{\pgfqpoint{1.940255in}{2.024656in}}{\pgfqpoint{1.948492in}{2.024656in}}%
\pgfpathclose%
\pgfusepath{stroke,fill}%
\end{pgfscope}%
\begin{pgfscope}%
\pgfpathrectangle{\pgfqpoint{0.100000in}{0.212622in}}{\pgfqpoint{3.696000in}{3.696000in}}%
\pgfusepath{clip}%
\pgfsetbuttcap%
\pgfsetroundjoin%
\definecolor{currentfill}{rgb}{0.121569,0.466667,0.705882}%
\pgfsetfillcolor{currentfill}%
\pgfsetfillopacity{0.300357}%
\pgfsetlinewidth{1.003750pt}%
\definecolor{currentstroke}{rgb}{0.121569,0.466667,0.705882}%
\pgfsetstrokecolor{currentstroke}%
\pgfsetstrokeopacity{0.300357}%
\pgfsetdash{}{0pt}%
\pgfpathmoveto{\pgfqpoint{1.948313in}{2.024613in}}%
\pgfpathcurveto{\pgfqpoint{1.956549in}{2.024613in}}{\pgfqpoint{1.964449in}{2.027885in}}{\pgfqpoint{1.970273in}{2.033709in}}%
\pgfpathcurveto{\pgfqpoint{1.976097in}{2.039533in}}{\pgfqpoint{1.979370in}{2.047433in}}{\pgfqpoint{1.979370in}{2.055670in}}%
\pgfpathcurveto{\pgfqpoint{1.979370in}{2.063906in}}{\pgfqpoint{1.976097in}{2.071806in}}{\pgfqpoint{1.970273in}{2.077630in}}%
\pgfpathcurveto{\pgfqpoint{1.964449in}{2.083454in}}{\pgfqpoint{1.956549in}{2.086726in}}{\pgfqpoint{1.948313in}{2.086726in}}%
\pgfpathcurveto{\pgfqpoint{1.940077in}{2.086726in}}{\pgfqpoint{1.932177in}{2.083454in}}{\pgfqpoint{1.926353in}{2.077630in}}%
\pgfpathcurveto{\pgfqpoint{1.920529in}{2.071806in}}{\pgfqpoint{1.917257in}{2.063906in}}{\pgfqpoint{1.917257in}{2.055670in}}%
\pgfpathcurveto{\pgfqpoint{1.917257in}{2.047433in}}{\pgfqpoint{1.920529in}{2.039533in}}{\pgfqpoint{1.926353in}{2.033709in}}%
\pgfpathcurveto{\pgfqpoint{1.932177in}{2.027885in}}{\pgfqpoint{1.940077in}{2.024613in}}{\pgfqpoint{1.948313in}{2.024613in}}%
\pgfpathclose%
\pgfusepath{stroke,fill}%
\end{pgfscope}%
\begin{pgfscope}%
\pgfpathrectangle{\pgfqpoint{0.100000in}{0.212622in}}{\pgfqpoint{3.696000in}{3.696000in}}%
\pgfusepath{clip}%
\pgfsetbuttcap%
\pgfsetroundjoin%
\definecolor{currentfill}{rgb}{0.121569,0.466667,0.705882}%
\pgfsetfillcolor{currentfill}%
\pgfsetfillopacity{0.300434}%
\pgfsetlinewidth{1.003750pt}%
\definecolor{currentstroke}{rgb}{0.121569,0.466667,0.705882}%
\pgfsetstrokecolor{currentstroke}%
\pgfsetstrokeopacity{0.300434}%
\pgfsetdash{}{0pt}%
\pgfpathmoveto{\pgfqpoint{1.947990in}{2.024518in}}%
\pgfpathcurveto{\pgfqpoint{1.956226in}{2.024518in}}{\pgfqpoint{1.964126in}{2.027791in}}{\pgfqpoint{1.969950in}{2.033614in}}%
\pgfpathcurveto{\pgfqpoint{1.975774in}{2.039438in}}{\pgfqpoint{1.979047in}{2.047338in}}{\pgfqpoint{1.979047in}{2.055575in}}%
\pgfpathcurveto{\pgfqpoint{1.979047in}{2.063811in}}{\pgfqpoint{1.975774in}{2.071711in}}{\pgfqpoint{1.969950in}{2.077535in}}%
\pgfpathcurveto{\pgfqpoint{1.964126in}{2.083359in}}{\pgfqpoint{1.956226in}{2.086631in}}{\pgfqpoint{1.947990in}{2.086631in}}%
\pgfpathcurveto{\pgfqpoint{1.939754in}{2.086631in}}{\pgfqpoint{1.931854in}{2.083359in}}{\pgfqpoint{1.926030in}{2.077535in}}%
\pgfpathcurveto{\pgfqpoint{1.920206in}{2.071711in}}{\pgfqpoint{1.916934in}{2.063811in}}{\pgfqpoint{1.916934in}{2.055575in}}%
\pgfpathcurveto{\pgfqpoint{1.916934in}{2.047338in}}{\pgfqpoint{1.920206in}{2.039438in}}{\pgfqpoint{1.926030in}{2.033614in}}%
\pgfpathcurveto{\pgfqpoint{1.931854in}{2.027791in}}{\pgfqpoint{1.939754in}{2.024518in}}{\pgfqpoint{1.947990in}{2.024518in}}%
\pgfpathclose%
\pgfusepath{stroke,fill}%
\end{pgfscope}%
\begin{pgfscope}%
\pgfpathrectangle{\pgfqpoint{0.100000in}{0.212622in}}{\pgfqpoint{3.696000in}{3.696000in}}%
\pgfusepath{clip}%
\pgfsetbuttcap%
\pgfsetroundjoin%
\definecolor{currentfill}{rgb}{0.121569,0.466667,0.705882}%
\pgfsetfillcolor{currentfill}%
\pgfsetfillopacity{0.300575}%
\pgfsetlinewidth{1.003750pt}%
\definecolor{currentstroke}{rgb}{0.121569,0.466667,0.705882}%
\pgfsetstrokecolor{currentstroke}%
\pgfsetstrokeopacity{0.300575}%
\pgfsetdash{}{0pt}%
\pgfpathmoveto{\pgfqpoint{1.947408in}{2.024340in}}%
\pgfpathcurveto{\pgfqpoint{1.955644in}{2.024340in}}{\pgfqpoint{1.963544in}{2.027612in}}{\pgfqpoint{1.969368in}{2.033436in}}%
\pgfpathcurveto{\pgfqpoint{1.975192in}{2.039260in}}{\pgfqpoint{1.978464in}{2.047160in}}{\pgfqpoint{1.978464in}{2.055396in}}%
\pgfpathcurveto{\pgfqpoint{1.978464in}{2.063633in}}{\pgfqpoint{1.975192in}{2.071533in}}{\pgfqpoint{1.969368in}{2.077357in}}%
\pgfpathcurveto{\pgfqpoint{1.963544in}{2.083181in}}{\pgfqpoint{1.955644in}{2.086453in}}{\pgfqpoint{1.947408in}{2.086453in}}%
\pgfpathcurveto{\pgfqpoint{1.939171in}{2.086453in}}{\pgfqpoint{1.931271in}{2.083181in}}{\pgfqpoint{1.925447in}{2.077357in}}%
\pgfpathcurveto{\pgfqpoint{1.919623in}{2.071533in}}{\pgfqpoint{1.916351in}{2.063633in}}{\pgfqpoint{1.916351in}{2.055396in}}%
\pgfpathcurveto{\pgfqpoint{1.916351in}{2.047160in}}{\pgfqpoint{1.919623in}{2.039260in}}{\pgfqpoint{1.925447in}{2.033436in}}%
\pgfpathcurveto{\pgfqpoint{1.931271in}{2.027612in}}{\pgfqpoint{1.939171in}{2.024340in}}{\pgfqpoint{1.947408in}{2.024340in}}%
\pgfpathclose%
\pgfusepath{stroke,fill}%
\end{pgfscope}%
\begin{pgfscope}%
\pgfpathrectangle{\pgfqpoint{0.100000in}{0.212622in}}{\pgfqpoint{3.696000in}{3.696000in}}%
\pgfusepath{clip}%
\pgfsetbuttcap%
\pgfsetroundjoin%
\definecolor{currentfill}{rgb}{0.121569,0.466667,0.705882}%
\pgfsetfillcolor{currentfill}%
\pgfsetfillopacity{0.300691}%
\pgfsetlinewidth{1.003750pt}%
\definecolor{currentstroke}{rgb}{0.121569,0.466667,0.705882}%
\pgfsetstrokecolor{currentstroke}%
\pgfsetstrokeopacity{0.300691}%
\pgfsetdash{}{0pt}%
\pgfpathmoveto{\pgfqpoint{1.951334in}{2.023226in}}%
\pgfpathcurveto{\pgfqpoint{1.959570in}{2.023226in}}{\pgfqpoint{1.967470in}{2.026499in}}{\pgfqpoint{1.973294in}{2.032322in}}%
\pgfpathcurveto{\pgfqpoint{1.979118in}{2.038146in}}{\pgfqpoint{1.982390in}{2.046046in}}{\pgfqpoint{1.982390in}{2.054283in}}%
\pgfpathcurveto{\pgfqpoint{1.982390in}{2.062519in}}{\pgfqpoint{1.979118in}{2.070419in}}{\pgfqpoint{1.973294in}{2.076243in}}%
\pgfpathcurveto{\pgfqpoint{1.967470in}{2.082067in}}{\pgfqpoint{1.959570in}{2.085339in}}{\pgfqpoint{1.951334in}{2.085339in}}%
\pgfpathcurveto{\pgfqpoint{1.943097in}{2.085339in}}{\pgfqpoint{1.935197in}{2.082067in}}{\pgfqpoint{1.929373in}{2.076243in}}%
\pgfpathcurveto{\pgfqpoint{1.923549in}{2.070419in}}{\pgfqpoint{1.920277in}{2.062519in}}{\pgfqpoint{1.920277in}{2.054283in}}%
\pgfpathcurveto{\pgfqpoint{1.920277in}{2.046046in}}{\pgfqpoint{1.923549in}{2.038146in}}{\pgfqpoint{1.929373in}{2.032322in}}%
\pgfpathcurveto{\pgfqpoint{1.935197in}{2.026499in}}{\pgfqpoint{1.943097in}{2.023226in}}{\pgfqpoint{1.951334in}{2.023226in}}%
\pgfpathclose%
\pgfusepath{stroke,fill}%
\end{pgfscope}%
\begin{pgfscope}%
\pgfpathrectangle{\pgfqpoint{0.100000in}{0.212622in}}{\pgfqpoint{3.696000in}{3.696000in}}%
\pgfusepath{clip}%
\pgfsetbuttcap%
\pgfsetroundjoin%
\definecolor{currentfill}{rgb}{0.121569,0.466667,0.705882}%
\pgfsetfillcolor{currentfill}%
\pgfsetfillopacity{0.300826}%
\pgfsetlinewidth{1.003750pt}%
\definecolor{currentstroke}{rgb}{0.121569,0.466667,0.705882}%
\pgfsetstrokecolor{currentstroke}%
\pgfsetstrokeopacity{0.300826}%
\pgfsetdash{}{0pt}%
\pgfpathmoveto{\pgfqpoint{1.946313in}{2.024067in}}%
\pgfpathcurveto{\pgfqpoint{1.954550in}{2.024067in}}{\pgfqpoint{1.962450in}{2.027340in}}{\pgfqpoint{1.968274in}{2.033164in}}%
\pgfpathcurveto{\pgfqpoint{1.974098in}{2.038988in}}{\pgfqpoint{1.977370in}{2.046888in}}{\pgfqpoint{1.977370in}{2.055124in}}%
\pgfpathcurveto{\pgfqpoint{1.977370in}{2.063360in}}{\pgfqpoint{1.974098in}{2.071260in}}{\pgfqpoint{1.968274in}{2.077084in}}%
\pgfpathcurveto{\pgfqpoint{1.962450in}{2.082908in}}{\pgfqpoint{1.954550in}{2.086180in}}{\pgfqpoint{1.946313in}{2.086180in}}%
\pgfpathcurveto{\pgfqpoint{1.938077in}{2.086180in}}{\pgfqpoint{1.930177in}{2.082908in}}{\pgfqpoint{1.924353in}{2.077084in}}%
\pgfpathcurveto{\pgfqpoint{1.918529in}{2.071260in}}{\pgfqpoint{1.915257in}{2.063360in}}{\pgfqpoint{1.915257in}{2.055124in}}%
\pgfpathcurveto{\pgfqpoint{1.915257in}{2.046888in}}{\pgfqpoint{1.918529in}{2.038988in}}{\pgfqpoint{1.924353in}{2.033164in}}%
\pgfpathcurveto{\pgfqpoint{1.930177in}{2.027340in}}{\pgfqpoint{1.938077in}{2.024067in}}{\pgfqpoint{1.946313in}{2.024067in}}%
\pgfpathclose%
\pgfusepath{stroke,fill}%
\end{pgfscope}%
\begin{pgfscope}%
\pgfpathrectangle{\pgfqpoint{0.100000in}{0.212622in}}{\pgfqpoint{3.696000in}{3.696000in}}%
\pgfusepath{clip}%
\pgfsetbuttcap%
\pgfsetroundjoin%
\definecolor{currentfill}{rgb}{0.121569,0.466667,0.705882}%
\pgfsetfillcolor{currentfill}%
\pgfsetfillopacity{0.300919}%
\pgfsetlinewidth{1.003750pt}%
\definecolor{currentstroke}{rgb}{0.121569,0.466667,0.705882}%
\pgfsetstrokecolor{currentstroke}%
\pgfsetstrokeopacity{0.300919}%
\pgfsetdash{}{0pt}%
\pgfpathmoveto{\pgfqpoint{1.951334in}{2.023223in}}%
\pgfpathcurveto{\pgfqpoint{1.959570in}{2.023223in}}{\pgfqpoint{1.967470in}{2.026495in}}{\pgfqpoint{1.973294in}{2.032319in}}%
\pgfpathcurveto{\pgfqpoint{1.979118in}{2.038143in}}{\pgfqpoint{1.982390in}{2.046043in}}{\pgfqpoint{1.982390in}{2.054280in}}%
\pgfpathcurveto{\pgfqpoint{1.982390in}{2.062516in}}{\pgfqpoint{1.979118in}{2.070416in}}{\pgfqpoint{1.973294in}{2.076240in}}%
\pgfpathcurveto{\pgfqpoint{1.967470in}{2.082064in}}{\pgfqpoint{1.959570in}{2.085336in}}{\pgfqpoint{1.951334in}{2.085336in}}%
\pgfpathcurveto{\pgfqpoint{1.943097in}{2.085336in}}{\pgfqpoint{1.935197in}{2.082064in}}{\pgfqpoint{1.929373in}{2.076240in}}%
\pgfpathcurveto{\pgfqpoint{1.923549in}{2.070416in}}{\pgfqpoint{1.920277in}{2.062516in}}{\pgfqpoint{1.920277in}{2.054280in}}%
\pgfpathcurveto{\pgfqpoint{1.920277in}{2.046043in}}{\pgfqpoint{1.923549in}{2.038143in}}{\pgfqpoint{1.929373in}{2.032319in}}%
\pgfpathcurveto{\pgfqpoint{1.935197in}{2.026495in}}{\pgfqpoint{1.943097in}{2.023223in}}{\pgfqpoint{1.951334in}{2.023223in}}%
\pgfpathclose%
\pgfusepath{stroke,fill}%
\end{pgfscope}%
\begin{pgfscope}%
\pgfpathrectangle{\pgfqpoint{0.100000in}{0.212622in}}{\pgfqpoint{3.696000in}{3.696000in}}%
\pgfusepath{clip}%
\pgfsetbuttcap%
\pgfsetroundjoin%
\definecolor{currentfill}{rgb}{0.121569,0.466667,0.705882}%
\pgfsetfillcolor{currentfill}%
\pgfsetfillopacity{0.301024}%
\pgfsetlinewidth{1.003750pt}%
\definecolor{currentstroke}{rgb}{0.121569,0.466667,0.705882}%
\pgfsetstrokecolor{currentstroke}%
\pgfsetstrokeopacity{0.301024}%
\pgfsetdash{}{0pt}%
\pgfpathmoveto{\pgfqpoint{1.951315in}{2.023077in}}%
\pgfpathcurveto{\pgfqpoint{1.959551in}{2.023077in}}{\pgfqpoint{1.967451in}{2.026350in}}{\pgfqpoint{1.973275in}{2.032173in}}%
\pgfpathcurveto{\pgfqpoint{1.979099in}{2.037997in}}{\pgfqpoint{1.982371in}{2.045897in}}{\pgfqpoint{1.982371in}{2.054134in}}%
\pgfpathcurveto{\pgfqpoint{1.982371in}{2.062370in}}{\pgfqpoint{1.979099in}{2.070270in}}{\pgfqpoint{1.973275in}{2.076094in}}%
\pgfpathcurveto{\pgfqpoint{1.967451in}{2.081918in}}{\pgfqpoint{1.959551in}{2.085190in}}{\pgfqpoint{1.951315in}{2.085190in}}%
\pgfpathcurveto{\pgfqpoint{1.943078in}{2.085190in}}{\pgfqpoint{1.935178in}{2.081918in}}{\pgfqpoint{1.929354in}{2.076094in}}%
\pgfpathcurveto{\pgfqpoint{1.923530in}{2.070270in}}{\pgfqpoint{1.920258in}{2.062370in}}{\pgfqpoint{1.920258in}{2.054134in}}%
\pgfpathcurveto{\pgfqpoint{1.920258in}{2.045897in}}{\pgfqpoint{1.923530in}{2.037997in}}{\pgfqpoint{1.929354in}{2.032173in}}%
\pgfpathcurveto{\pgfqpoint{1.935178in}{2.026350in}}{\pgfqpoint{1.943078in}{2.023077in}}{\pgfqpoint{1.951315in}{2.023077in}}%
\pgfpathclose%
\pgfusepath{stroke,fill}%
\end{pgfscope}%
\begin{pgfscope}%
\pgfpathrectangle{\pgfqpoint{0.100000in}{0.212622in}}{\pgfqpoint{3.696000in}{3.696000in}}%
\pgfusepath{clip}%
\pgfsetbuttcap%
\pgfsetroundjoin%
\definecolor{currentfill}{rgb}{0.121569,0.466667,0.705882}%
\pgfsetfillcolor{currentfill}%
\pgfsetfillopacity{0.301334}%
\pgfsetlinewidth{1.003750pt}%
\definecolor{currentstroke}{rgb}{0.121569,0.466667,0.705882}%
\pgfsetstrokecolor{currentstroke}%
\pgfsetstrokeopacity{0.301334}%
\pgfsetdash{}{0pt}%
\pgfpathmoveto{\pgfqpoint{1.944486in}{2.023573in}}%
\pgfpathcurveto{\pgfqpoint{1.952722in}{2.023573in}}{\pgfqpoint{1.960622in}{2.026846in}}{\pgfqpoint{1.966446in}{2.032670in}}%
\pgfpathcurveto{\pgfqpoint{1.972270in}{2.038494in}}{\pgfqpoint{1.975542in}{2.046394in}}{\pgfqpoint{1.975542in}{2.054630in}}%
\pgfpathcurveto{\pgfqpoint{1.975542in}{2.062866in}}{\pgfqpoint{1.972270in}{2.070766in}}{\pgfqpoint{1.966446in}{2.076590in}}%
\pgfpathcurveto{\pgfqpoint{1.960622in}{2.082414in}}{\pgfqpoint{1.952722in}{2.085686in}}{\pgfqpoint{1.944486in}{2.085686in}}%
\pgfpathcurveto{\pgfqpoint{1.936249in}{2.085686in}}{\pgfqpoint{1.928349in}{2.082414in}}{\pgfqpoint{1.922525in}{2.076590in}}%
\pgfpathcurveto{\pgfqpoint{1.916701in}{2.070766in}}{\pgfqpoint{1.913429in}{2.062866in}}{\pgfqpoint{1.913429in}{2.054630in}}%
\pgfpathcurveto{\pgfqpoint{1.913429in}{2.046394in}}{\pgfqpoint{1.916701in}{2.038494in}}{\pgfqpoint{1.922525in}{2.032670in}}%
\pgfpathcurveto{\pgfqpoint{1.928349in}{2.026846in}}{\pgfqpoint{1.936249in}{2.023573in}}{\pgfqpoint{1.944486in}{2.023573in}}%
\pgfpathclose%
\pgfusepath{stroke,fill}%
\end{pgfscope}%
\begin{pgfscope}%
\pgfpathrectangle{\pgfqpoint{0.100000in}{0.212622in}}{\pgfqpoint{3.696000in}{3.696000in}}%
\pgfusepath{clip}%
\pgfsetbuttcap%
\pgfsetroundjoin%
\definecolor{currentfill}{rgb}{0.121569,0.466667,0.705882}%
\pgfsetfillcolor{currentfill}%
\pgfsetfillopacity{0.301363}%
\pgfsetlinewidth{1.003750pt}%
\definecolor{currentstroke}{rgb}{0.121569,0.466667,0.705882}%
\pgfsetstrokecolor{currentstroke}%
\pgfsetstrokeopacity{0.301363}%
\pgfsetdash{}{0pt}%
\pgfpathmoveto{\pgfqpoint{1.951257in}{2.023002in}}%
\pgfpathcurveto{\pgfqpoint{1.959493in}{2.023002in}}{\pgfqpoint{1.967393in}{2.026275in}}{\pgfqpoint{1.973217in}{2.032098in}}%
\pgfpathcurveto{\pgfqpoint{1.979041in}{2.037922in}}{\pgfqpoint{1.982313in}{2.045822in}}{\pgfqpoint{1.982313in}{2.054059in}}%
\pgfpathcurveto{\pgfqpoint{1.982313in}{2.062295in}}{\pgfqpoint{1.979041in}{2.070195in}}{\pgfqpoint{1.973217in}{2.076019in}}%
\pgfpathcurveto{\pgfqpoint{1.967393in}{2.081843in}}{\pgfqpoint{1.959493in}{2.085115in}}{\pgfqpoint{1.951257in}{2.085115in}}%
\pgfpathcurveto{\pgfqpoint{1.943020in}{2.085115in}}{\pgfqpoint{1.935120in}{2.081843in}}{\pgfqpoint{1.929296in}{2.076019in}}%
\pgfpathcurveto{\pgfqpoint{1.923472in}{2.070195in}}{\pgfqpoint{1.920200in}{2.062295in}}{\pgfqpoint{1.920200in}{2.054059in}}%
\pgfpathcurveto{\pgfqpoint{1.920200in}{2.045822in}}{\pgfqpoint{1.923472in}{2.037922in}}{\pgfqpoint{1.929296in}{2.032098in}}%
\pgfpathcurveto{\pgfqpoint{1.935120in}{2.026275in}}{\pgfqpoint{1.943020in}{2.023002in}}{\pgfqpoint{1.951257in}{2.023002in}}%
\pgfpathclose%
\pgfusepath{stroke,fill}%
\end{pgfscope}%
\begin{pgfscope}%
\pgfpathrectangle{\pgfqpoint{0.100000in}{0.212622in}}{\pgfqpoint{3.696000in}{3.696000in}}%
\pgfusepath{clip}%
\pgfsetbuttcap%
\pgfsetroundjoin%
\definecolor{currentfill}{rgb}{0.121569,0.466667,0.705882}%
\pgfsetfillcolor{currentfill}%
\pgfsetfillopacity{0.301539}%
\pgfsetlinewidth{1.003750pt}%
\definecolor{currentstroke}{rgb}{0.121569,0.466667,0.705882}%
\pgfsetstrokecolor{currentstroke}%
\pgfsetstrokeopacity{0.301539}%
\pgfsetdash{}{0pt}%
\pgfpathmoveto{\pgfqpoint{1.951178in}{2.022890in}}%
\pgfpathcurveto{\pgfqpoint{1.959415in}{2.022890in}}{\pgfqpoint{1.967315in}{2.026162in}}{\pgfqpoint{1.973139in}{2.031986in}}%
\pgfpathcurveto{\pgfqpoint{1.978963in}{2.037810in}}{\pgfqpoint{1.982235in}{2.045710in}}{\pgfqpoint{1.982235in}{2.053946in}}%
\pgfpathcurveto{\pgfqpoint{1.982235in}{2.062183in}}{\pgfqpoint{1.978963in}{2.070083in}}{\pgfqpoint{1.973139in}{2.075907in}}%
\pgfpathcurveto{\pgfqpoint{1.967315in}{2.081731in}}{\pgfqpoint{1.959415in}{2.085003in}}{\pgfqpoint{1.951178in}{2.085003in}}%
\pgfpathcurveto{\pgfqpoint{1.942942in}{2.085003in}}{\pgfqpoint{1.935042in}{2.081731in}}{\pgfqpoint{1.929218in}{2.075907in}}%
\pgfpathcurveto{\pgfqpoint{1.923394in}{2.070083in}}{\pgfqpoint{1.920122in}{2.062183in}}{\pgfqpoint{1.920122in}{2.053946in}}%
\pgfpathcurveto{\pgfqpoint{1.920122in}{2.045710in}}{\pgfqpoint{1.923394in}{2.037810in}}{\pgfqpoint{1.929218in}{2.031986in}}%
\pgfpathcurveto{\pgfqpoint{1.935042in}{2.026162in}}{\pgfqpoint{1.942942in}{2.022890in}}{\pgfqpoint{1.951178in}{2.022890in}}%
\pgfpathclose%
\pgfusepath{stroke,fill}%
\end{pgfscope}%
\begin{pgfscope}%
\pgfpathrectangle{\pgfqpoint{0.100000in}{0.212622in}}{\pgfqpoint{3.696000in}{3.696000in}}%
\pgfusepath{clip}%
\pgfsetbuttcap%
\pgfsetroundjoin%
\definecolor{currentfill}{rgb}{0.121569,0.466667,0.705882}%
\pgfsetfillcolor{currentfill}%
\pgfsetfillopacity{0.301726}%
\pgfsetlinewidth{1.003750pt}%
\definecolor{currentstroke}{rgb}{0.121569,0.466667,0.705882}%
\pgfsetstrokecolor{currentstroke}%
\pgfsetstrokeopacity{0.301726}%
\pgfsetdash{}{0pt}%
\pgfpathmoveto{\pgfqpoint{1.942972in}{2.023229in}}%
\pgfpathcurveto{\pgfqpoint{1.951208in}{2.023229in}}{\pgfqpoint{1.959108in}{2.026501in}}{\pgfqpoint{1.964932in}{2.032325in}}%
\pgfpathcurveto{\pgfqpoint{1.970756in}{2.038149in}}{\pgfqpoint{1.974028in}{2.046049in}}{\pgfqpoint{1.974028in}{2.054285in}}%
\pgfpathcurveto{\pgfqpoint{1.974028in}{2.062522in}}{\pgfqpoint{1.970756in}{2.070422in}}{\pgfqpoint{1.964932in}{2.076246in}}%
\pgfpathcurveto{\pgfqpoint{1.959108in}{2.082070in}}{\pgfqpoint{1.951208in}{2.085342in}}{\pgfqpoint{1.942972in}{2.085342in}}%
\pgfpathcurveto{\pgfqpoint{1.934736in}{2.085342in}}{\pgfqpoint{1.926836in}{2.082070in}}{\pgfqpoint{1.921012in}{2.076246in}}%
\pgfpathcurveto{\pgfqpoint{1.915188in}{2.070422in}}{\pgfqpoint{1.911915in}{2.062522in}}{\pgfqpoint{1.911915in}{2.054285in}}%
\pgfpathcurveto{\pgfqpoint{1.911915in}{2.046049in}}{\pgfqpoint{1.915188in}{2.038149in}}{\pgfqpoint{1.921012in}{2.032325in}}%
\pgfpathcurveto{\pgfqpoint{1.926836in}{2.026501in}}{\pgfqpoint{1.934736in}{2.023229in}}{\pgfqpoint{1.942972in}{2.023229in}}%
\pgfpathclose%
\pgfusepath{stroke,fill}%
\end{pgfscope}%
\begin{pgfscope}%
\pgfpathrectangle{\pgfqpoint{0.100000in}{0.212622in}}{\pgfqpoint{3.696000in}{3.696000in}}%
\pgfusepath{clip}%
\pgfsetbuttcap%
\pgfsetroundjoin%
\definecolor{currentfill}{rgb}{0.121569,0.466667,0.705882}%
\pgfsetfillcolor{currentfill}%
\pgfsetfillopacity{0.301953}%
\pgfsetlinewidth{1.003750pt}%
\definecolor{currentstroke}{rgb}{0.121569,0.466667,0.705882}%
\pgfsetstrokecolor{currentstroke}%
\pgfsetstrokeopacity{0.301953}%
\pgfsetdash{}{0pt}%
\pgfpathmoveto{\pgfqpoint{1.951026in}{2.023429in}}%
\pgfpathcurveto{\pgfqpoint{1.959262in}{2.023429in}}{\pgfqpoint{1.967162in}{2.026701in}}{\pgfqpoint{1.972986in}{2.032525in}}%
\pgfpathcurveto{\pgfqpoint{1.978810in}{2.038349in}}{\pgfqpoint{1.982082in}{2.046249in}}{\pgfqpoint{1.982082in}{2.054485in}}%
\pgfpathcurveto{\pgfqpoint{1.982082in}{2.062722in}}{\pgfqpoint{1.978810in}{2.070622in}}{\pgfqpoint{1.972986in}{2.076446in}}%
\pgfpathcurveto{\pgfqpoint{1.967162in}{2.082270in}}{\pgfqpoint{1.959262in}{2.085542in}}{\pgfqpoint{1.951026in}{2.085542in}}%
\pgfpathcurveto{\pgfqpoint{1.942790in}{2.085542in}}{\pgfqpoint{1.934890in}{2.082270in}}{\pgfqpoint{1.929066in}{2.076446in}}%
\pgfpathcurveto{\pgfqpoint{1.923242in}{2.070622in}}{\pgfqpoint{1.919969in}{2.062722in}}{\pgfqpoint{1.919969in}{2.054485in}}%
\pgfpathcurveto{\pgfqpoint{1.919969in}{2.046249in}}{\pgfqpoint{1.923242in}{2.038349in}}{\pgfqpoint{1.929066in}{2.032525in}}%
\pgfpathcurveto{\pgfqpoint{1.934890in}{2.026701in}}{\pgfqpoint{1.942790in}{2.023429in}}{\pgfqpoint{1.951026in}{2.023429in}}%
\pgfpathclose%
\pgfusepath{stroke,fill}%
\end{pgfscope}%
\begin{pgfscope}%
\pgfpathrectangle{\pgfqpoint{0.100000in}{0.212622in}}{\pgfqpoint{3.696000in}{3.696000in}}%
\pgfusepath{clip}%
\pgfsetbuttcap%
\pgfsetroundjoin%
\definecolor{currentfill}{rgb}{0.121569,0.466667,0.705882}%
\pgfsetfillcolor{currentfill}%
\pgfsetfillopacity{0.301984}%
\pgfsetlinewidth{1.003750pt}%
\definecolor{currentstroke}{rgb}{0.121569,0.466667,0.705882}%
\pgfsetstrokecolor{currentstroke}%
\pgfsetstrokeopacity{0.301984}%
\pgfsetdash{}{0pt}%
\pgfpathmoveto{\pgfqpoint{1.941953in}{2.022932in}}%
\pgfpathcurveto{\pgfqpoint{1.950189in}{2.022932in}}{\pgfqpoint{1.958089in}{2.026204in}}{\pgfqpoint{1.963913in}{2.032028in}}%
\pgfpathcurveto{\pgfqpoint{1.969737in}{2.037852in}}{\pgfqpoint{1.973009in}{2.045752in}}{\pgfqpoint{1.973009in}{2.053989in}}%
\pgfpathcurveto{\pgfqpoint{1.973009in}{2.062225in}}{\pgfqpoint{1.969737in}{2.070125in}}{\pgfqpoint{1.963913in}{2.075949in}}%
\pgfpathcurveto{\pgfqpoint{1.958089in}{2.081773in}}{\pgfqpoint{1.950189in}{2.085045in}}{\pgfqpoint{1.941953in}{2.085045in}}%
\pgfpathcurveto{\pgfqpoint{1.933716in}{2.085045in}}{\pgfqpoint{1.925816in}{2.081773in}}{\pgfqpoint{1.919992in}{2.075949in}}%
\pgfpathcurveto{\pgfqpoint{1.914169in}{2.070125in}}{\pgfqpoint{1.910896in}{2.062225in}}{\pgfqpoint{1.910896in}{2.053989in}}%
\pgfpathcurveto{\pgfqpoint{1.910896in}{2.045752in}}{\pgfqpoint{1.914169in}{2.037852in}}{\pgfqpoint{1.919992in}{2.032028in}}%
\pgfpathcurveto{\pgfqpoint{1.925816in}{2.026204in}}{\pgfqpoint{1.933716in}{2.022932in}}{\pgfqpoint{1.941953in}{2.022932in}}%
\pgfpathclose%
\pgfusepath{stroke,fill}%
\end{pgfscope}%
\begin{pgfscope}%
\pgfpathrectangle{\pgfqpoint{0.100000in}{0.212622in}}{\pgfqpoint{3.696000in}{3.696000in}}%
\pgfusepath{clip}%
\pgfsetbuttcap%
\pgfsetroundjoin%
\definecolor{currentfill}{rgb}{0.121569,0.466667,0.705882}%
\pgfsetfillcolor{currentfill}%
\pgfsetfillopacity{0.302399}%
\pgfsetlinewidth{1.003750pt}%
\definecolor{currentstroke}{rgb}{0.121569,0.466667,0.705882}%
\pgfsetstrokecolor{currentstroke}%
\pgfsetstrokeopacity{0.302399}%
\pgfsetdash{}{0pt}%
\pgfpathmoveto{\pgfqpoint{1.940183in}{2.021835in}}%
\pgfpathcurveto{\pgfqpoint{1.948419in}{2.021835in}}{\pgfqpoint{1.956319in}{2.025108in}}{\pgfqpoint{1.962143in}{2.030932in}}%
\pgfpathcurveto{\pgfqpoint{1.967967in}{2.036755in}}{\pgfqpoint{1.971239in}{2.044656in}}{\pgfqpoint{1.971239in}{2.052892in}}%
\pgfpathcurveto{\pgfqpoint{1.971239in}{2.061128in}}{\pgfqpoint{1.967967in}{2.069028in}}{\pgfqpoint{1.962143in}{2.074852in}}%
\pgfpathcurveto{\pgfqpoint{1.956319in}{2.080676in}}{\pgfqpoint{1.948419in}{2.083948in}}{\pgfqpoint{1.940183in}{2.083948in}}%
\pgfpathcurveto{\pgfqpoint{1.931946in}{2.083948in}}{\pgfqpoint{1.924046in}{2.080676in}}{\pgfqpoint{1.918222in}{2.074852in}}%
\pgfpathcurveto{\pgfqpoint{1.912398in}{2.069028in}}{\pgfqpoint{1.909126in}{2.061128in}}{\pgfqpoint{1.909126in}{2.052892in}}%
\pgfpathcurveto{\pgfqpoint{1.909126in}{2.044656in}}{\pgfqpoint{1.912398in}{2.036755in}}{\pgfqpoint{1.918222in}{2.030932in}}%
\pgfpathcurveto{\pgfqpoint{1.924046in}{2.025108in}}{\pgfqpoint{1.931946in}{2.021835in}}{\pgfqpoint{1.940183in}{2.021835in}}%
\pgfpathclose%
\pgfusepath{stroke,fill}%
\end{pgfscope}%
\begin{pgfscope}%
\pgfpathrectangle{\pgfqpoint{0.100000in}{0.212622in}}{\pgfqpoint{3.696000in}{3.696000in}}%
\pgfusepath{clip}%
\pgfsetbuttcap%
\pgfsetroundjoin%
\definecolor{currentfill}{rgb}{0.121569,0.466667,0.705882}%
\pgfsetfillcolor{currentfill}%
\pgfsetfillopacity{0.302518}%
\pgfsetlinewidth{1.003750pt}%
\definecolor{currentstroke}{rgb}{0.121569,0.466667,0.705882}%
\pgfsetstrokecolor{currentstroke}%
\pgfsetstrokeopacity{0.302518}%
\pgfsetdash{}{0pt}%
\pgfpathmoveto{\pgfqpoint{1.951405in}{2.020835in}}%
\pgfpathcurveto{\pgfqpoint{1.959641in}{2.020835in}}{\pgfqpoint{1.967541in}{2.024107in}}{\pgfqpoint{1.973365in}{2.029931in}}%
\pgfpathcurveto{\pgfqpoint{1.979189in}{2.035755in}}{\pgfqpoint{1.982461in}{2.043655in}}{\pgfqpoint{1.982461in}{2.051892in}}%
\pgfpathcurveto{\pgfqpoint{1.982461in}{2.060128in}}{\pgfqpoint{1.979189in}{2.068028in}}{\pgfqpoint{1.973365in}{2.073852in}}%
\pgfpathcurveto{\pgfqpoint{1.967541in}{2.079676in}}{\pgfqpoint{1.959641in}{2.082948in}}{\pgfqpoint{1.951405in}{2.082948in}}%
\pgfpathcurveto{\pgfqpoint{1.943168in}{2.082948in}}{\pgfqpoint{1.935268in}{2.079676in}}{\pgfqpoint{1.929444in}{2.073852in}}%
\pgfpathcurveto{\pgfqpoint{1.923620in}{2.068028in}}{\pgfqpoint{1.920348in}{2.060128in}}{\pgfqpoint{1.920348in}{2.051892in}}%
\pgfpathcurveto{\pgfqpoint{1.920348in}{2.043655in}}{\pgfqpoint{1.923620in}{2.035755in}}{\pgfqpoint{1.929444in}{2.029931in}}%
\pgfpathcurveto{\pgfqpoint{1.935268in}{2.024107in}}{\pgfqpoint{1.943168in}{2.020835in}}{\pgfqpoint{1.951405in}{2.020835in}}%
\pgfpathclose%
\pgfusepath{stroke,fill}%
\end{pgfscope}%
\begin{pgfscope}%
\pgfpathrectangle{\pgfqpoint{0.100000in}{0.212622in}}{\pgfqpoint{3.696000in}{3.696000in}}%
\pgfusepath{clip}%
\pgfsetbuttcap%
\pgfsetroundjoin%
\definecolor{currentfill}{rgb}{0.121569,0.466667,0.705882}%
\pgfsetfillcolor{currentfill}%
\pgfsetfillopacity{0.303493}%
\pgfsetlinewidth{1.003750pt}%
\definecolor{currentstroke}{rgb}{0.121569,0.466667,0.705882}%
\pgfsetstrokecolor{currentstroke}%
\pgfsetstrokeopacity{0.303493}%
\pgfsetdash{}{0pt}%
\pgfpathmoveto{\pgfqpoint{1.952205in}{2.018312in}}%
\pgfpathcurveto{\pgfqpoint{1.960442in}{2.018312in}}{\pgfqpoint{1.968342in}{2.021585in}}{\pgfqpoint{1.974166in}{2.027409in}}%
\pgfpathcurveto{\pgfqpoint{1.979990in}{2.033233in}}{\pgfqpoint{1.983262in}{2.041133in}}{\pgfqpoint{1.983262in}{2.049369in}}%
\pgfpathcurveto{\pgfqpoint{1.983262in}{2.057605in}}{\pgfqpoint{1.979990in}{2.065505in}}{\pgfqpoint{1.974166in}{2.071329in}}%
\pgfpathcurveto{\pgfqpoint{1.968342in}{2.077153in}}{\pgfqpoint{1.960442in}{2.080425in}}{\pgfqpoint{1.952205in}{2.080425in}}%
\pgfpathcurveto{\pgfqpoint{1.943969in}{2.080425in}}{\pgfqpoint{1.936069in}{2.077153in}}{\pgfqpoint{1.930245in}{2.071329in}}%
\pgfpathcurveto{\pgfqpoint{1.924421in}{2.065505in}}{\pgfqpoint{1.921149in}{2.057605in}}{\pgfqpoint{1.921149in}{2.049369in}}%
\pgfpathcurveto{\pgfqpoint{1.921149in}{2.041133in}}{\pgfqpoint{1.924421in}{2.033233in}}{\pgfqpoint{1.930245in}{2.027409in}}%
\pgfpathcurveto{\pgfqpoint{1.936069in}{2.021585in}}{\pgfqpoint{1.943969in}{2.018312in}}{\pgfqpoint{1.952205in}{2.018312in}}%
\pgfpathclose%
\pgfusepath{stroke,fill}%
\end{pgfscope}%
\begin{pgfscope}%
\pgfpathrectangle{\pgfqpoint{0.100000in}{0.212622in}}{\pgfqpoint{3.696000in}{3.696000in}}%
\pgfusepath{clip}%
\pgfsetbuttcap%
\pgfsetroundjoin%
\definecolor{currentfill}{rgb}{0.121569,0.466667,0.705882}%
\pgfsetfillcolor{currentfill}%
\pgfsetfillopacity{0.303502}%
\pgfsetlinewidth{1.003750pt}%
\definecolor{currentstroke}{rgb}{0.121569,0.466667,0.705882}%
\pgfsetstrokecolor{currentstroke}%
\pgfsetstrokeopacity{0.303502}%
\pgfsetdash{}{0pt}%
\pgfpathmoveto{\pgfqpoint{1.937012in}{2.022185in}}%
\pgfpathcurveto{\pgfqpoint{1.945248in}{2.022185in}}{\pgfqpoint{1.953149in}{2.025457in}}{\pgfqpoint{1.958972in}{2.031281in}}%
\pgfpathcurveto{\pgfqpoint{1.964796in}{2.037105in}}{\pgfqpoint{1.968069in}{2.045005in}}{\pgfqpoint{1.968069in}{2.053242in}}%
\pgfpathcurveto{\pgfqpoint{1.968069in}{2.061478in}}{\pgfqpoint{1.964796in}{2.069378in}}{\pgfqpoint{1.958972in}{2.075202in}}%
\pgfpathcurveto{\pgfqpoint{1.953149in}{2.081026in}}{\pgfqpoint{1.945248in}{2.084298in}}{\pgfqpoint{1.937012in}{2.084298in}}%
\pgfpathcurveto{\pgfqpoint{1.928776in}{2.084298in}}{\pgfqpoint{1.920876in}{2.081026in}}{\pgfqpoint{1.915052in}{2.075202in}}%
\pgfpathcurveto{\pgfqpoint{1.909228in}{2.069378in}}{\pgfqpoint{1.905956in}{2.061478in}}{\pgfqpoint{1.905956in}{2.053242in}}%
\pgfpathcurveto{\pgfqpoint{1.905956in}{2.045005in}}{\pgfqpoint{1.909228in}{2.037105in}}{\pgfqpoint{1.915052in}{2.031281in}}%
\pgfpathcurveto{\pgfqpoint{1.920876in}{2.025457in}}{\pgfqpoint{1.928776in}{2.022185in}}{\pgfqpoint{1.937012in}{2.022185in}}%
\pgfpathclose%
\pgfusepath{stroke,fill}%
\end{pgfscope}%
\begin{pgfscope}%
\pgfpathrectangle{\pgfqpoint{0.100000in}{0.212622in}}{\pgfqpoint{3.696000in}{3.696000in}}%
\pgfusepath{clip}%
\pgfsetbuttcap%
\pgfsetroundjoin%
\definecolor{currentfill}{rgb}{0.121569,0.466667,0.705882}%
\pgfsetfillcolor{currentfill}%
\pgfsetfillopacity{0.305289}%
\pgfsetlinewidth{1.003750pt}%
\definecolor{currentstroke}{rgb}{0.121569,0.466667,0.705882}%
\pgfsetstrokecolor{currentstroke}%
\pgfsetstrokeopacity{0.305289}%
\pgfsetdash{}{0pt}%
\pgfpathmoveto{\pgfqpoint{1.931388in}{2.021034in}}%
\pgfpathcurveto{\pgfqpoint{1.939624in}{2.021034in}}{\pgfqpoint{1.947524in}{2.024307in}}{\pgfqpoint{1.953348in}{2.030131in}}%
\pgfpathcurveto{\pgfqpoint{1.959172in}{2.035955in}}{\pgfqpoint{1.962444in}{2.043855in}}{\pgfqpoint{1.962444in}{2.052091in}}%
\pgfpathcurveto{\pgfqpoint{1.962444in}{2.060327in}}{\pgfqpoint{1.959172in}{2.068227in}}{\pgfqpoint{1.953348in}{2.074051in}}%
\pgfpathcurveto{\pgfqpoint{1.947524in}{2.079875in}}{\pgfqpoint{1.939624in}{2.083147in}}{\pgfqpoint{1.931388in}{2.083147in}}%
\pgfpathcurveto{\pgfqpoint{1.923152in}{2.083147in}}{\pgfqpoint{1.915252in}{2.079875in}}{\pgfqpoint{1.909428in}{2.074051in}}%
\pgfpathcurveto{\pgfqpoint{1.903604in}{2.068227in}}{\pgfqpoint{1.900331in}{2.060327in}}{\pgfqpoint{1.900331in}{2.052091in}}%
\pgfpathcurveto{\pgfqpoint{1.900331in}{2.043855in}}{\pgfqpoint{1.903604in}{2.035955in}}{\pgfqpoint{1.909428in}{2.030131in}}%
\pgfpathcurveto{\pgfqpoint{1.915252in}{2.024307in}}{\pgfqpoint{1.923152in}{2.021034in}}{\pgfqpoint{1.931388in}{2.021034in}}%
\pgfpathclose%
\pgfusepath{stroke,fill}%
\end{pgfscope}%
\begin{pgfscope}%
\pgfpathrectangle{\pgfqpoint{0.100000in}{0.212622in}}{\pgfqpoint{3.696000in}{3.696000in}}%
\pgfusepath{clip}%
\pgfsetbuttcap%
\pgfsetroundjoin%
\definecolor{currentfill}{rgb}{0.121569,0.466667,0.705882}%
\pgfsetfillcolor{currentfill}%
\pgfsetfillopacity{0.305323}%
\pgfsetlinewidth{1.003750pt}%
\definecolor{currentstroke}{rgb}{0.121569,0.466667,0.705882}%
\pgfsetstrokecolor{currentstroke}%
\pgfsetstrokeopacity{0.305323}%
\pgfsetdash{}{0pt}%
\pgfpathmoveto{\pgfqpoint{1.953631in}{2.020833in}}%
\pgfpathcurveto{\pgfqpoint{1.961867in}{2.020833in}}{\pgfqpoint{1.969767in}{2.024106in}}{\pgfqpoint{1.975591in}{2.029930in}}%
\pgfpathcurveto{\pgfqpoint{1.981415in}{2.035754in}}{\pgfqpoint{1.984687in}{2.043654in}}{\pgfqpoint{1.984687in}{2.051890in}}%
\pgfpathcurveto{\pgfqpoint{1.984687in}{2.060126in}}{\pgfqpoint{1.981415in}{2.068026in}}{\pgfqpoint{1.975591in}{2.073850in}}%
\pgfpathcurveto{\pgfqpoint{1.969767in}{2.079674in}}{\pgfqpoint{1.961867in}{2.082946in}}{\pgfqpoint{1.953631in}{2.082946in}}%
\pgfpathcurveto{\pgfqpoint{1.945394in}{2.082946in}}{\pgfqpoint{1.937494in}{2.079674in}}{\pgfqpoint{1.931670in}{2.073850in}}%
\pgfpathcurveto{\pgfqpoint{1.925847in}{2.068026in}}{\pgfqpoint{1.922574in}{2.060126in}}{\pgfqpoint{1.922574in}{2.051890in}}%
\pgfpathcurveto{\pgfqpoint{1.922574in}{2.043654in}}{\pgfqpoint{1.925847in}{2.035754in}}{\pgfqpoint{1.931670in}{2.029930in}}%
\pgfpathcurveto{\pgfqpoint{1.937494in}{2.024106in}}{\pgfqpoint{1.945394in}{2.020833in}}{\pgfqpoint{1.953631in}{2.020833in}}%
\pgfpathclose%
\pgfusepath{stroke,fill}%
\end{pgfscope}%
\begin{pgfscope}%
\pgfpathrectangle{\pgfqpoint{0.100000in}{0.212622in}}{\pgfqpoint{3.696000in}{3.696000in}}%
\pgfusepath{clip}%
\pgfsetbuttcap%
\pgfsetroundjoin%
\definecolor{currentfill}{rgb}{0.121569,0.466667,0.705882}%
\pgfsetfillcolor{currentfill}%
\pgfsetfillopacity{0.307011}%
\pgfsetlinewidth{1.003750pt}%
\definecolor{currentstroke}{rgb}{0.121569,0.466667,0.705882}%
\pgfsetstrokecolor{currentstroke}%
\pgfsetstrokeopacity{0.307011}%
\pgfsetdash{}{0pt}%
\pgfpathmoveto{\pgfqpoint{1.926896in}{2.020400in}}%
\pgfpathcurveto{\pgfqpoint{1.935132in}{2.020400in}}{\pgfqpoint{1.943032in}{2.023672in}}{\pgfqpoint{1.948856in}{2.029496in}}%
\pgfpathcurveto{\pgfqpoint{1.954680in}{2.035320in}}{\pgfqpoint{1.957952in}{2.043220in}}{\pgfqpoint{1.957952in}{2.051456in}}%
\pgfpathcurveto{\pgfqpoint{1.957952in}{2.059693in}}{\pgfqpoint{1.954680in}{2.067593in}}{\pgfqpoint{1.948856in}{2.073417in}}%
\pgfpathcurveto{\pgfqpoint{1.943032in}{2.079240in}}{\pgfqpoint{1.935132in}{2.082513in}}{\pgfqpoint{1.926896in}{2.082513in}}%
\pgfpathcurveto{\pgfqpoint{1.918659in}{2.082513in}}{\pgfqpoint{1.910759in}{2.079240in}}{\pgfqpoint{1.904935in}{2.073417in}}%
\pgfpathcurveto{\pgfqpoint{1.899111in}{2.067593in}}{\pgfqpoint{1.895839in}{2.059693in}}{\pgfqpoint{1.895839in}{2.051456in}}%
\pgfpathcurveto{\pgfqpoint{1.895839in}{2.043220in}}{\pgfqpoint{1.899111in}{2.035320in}}{\pgfqpoint{1.904935in}{2.029496in}}%
\pgfpathcurveto{\pgfqpoint{1.910759in}{2.023672in}}{\pgfqpoint{1.918659in}{2.020400in}}{\pgfqpoint{1.926896in}{2.020400in}}%
\pgfpathclose%
\pgfusepath{stroke,fill}%
\end{pgfscope}%
\begin{pgfscope}%
\pgfpathrectangle{\pgfqpoint{0.100000in}{0.212622in}}{\pgfqpoint{3.696000in}{3.696000in}}%
\pgfusepath{clip}%
\pgfsetbuttcap%
\pgfsetroundjoin%
\definecolor{currentfill}{rgb}{0.121569,0.466667,0.705882}%
\pgfsetfillcolor{currentfill}%
\pgfsetfillopacity{0.307582}%
\pgfsetlinewidth{1.003750pt}%
\definecolor{currentstroke}{rgb}{0.121569,0.466667,0.705882}%
\pgfsetstrokecolor{currentstroke}%
\pgfsetstrokeopacity{0.307582}%
\pgfsetdash{}{0pt}%
\pgfpathmoveto{\pgfqpoint{1.954148in}{2.017126in}}%
\pgfpathcurveto{\pgfqpoint{1.962384in}{2.017126in}}{\pgfqpoint{1.970284in}{2.020398in}}{\pgfqpoint{1.976108in}{2.026222in}}%
\pgfpathcurveto{\pgfqpoint{1.981932in}{2.032046in}}{\pgfqpoint{1.985205in}{2.039946in}}{\pgfqpoint{1.985205in}{2.048182in}}%
\pgfpathcurveto{\pgfqpoint{1.985205in}{2.056418in}}{\pgfqpoint{1.981932in}{2.064319in}}{\pgfqpoint{1.976108in}{2.070142in}}%
\pgfpathcurveto{\pgfqpoint{1.970284in}{2.075966in}}{\pgfqpoint{1.962384in}{2.079239in}}{\pgfqpoint{1.954148in}{2.079239in}}%
\pgfpathcurveto{\pgfqpoint{1.945912in}{2.079239in}}{\pgfqpoint{1.938012in}{2.075966in}}{\pgfqpoint{1.932188in}{2.070142in}}%
\pgfpathcurveto{\pgfqpoint{1.926364in}{2.064319in}}{\pgfqpoint{1.923092in}{2.056418in}}{\pgfqpoint{1.923092in}{2.048182in}}%
\pgfpathcurveto{\pgfqpoint{1.923092in}{2.039946in}}{\pgfqpoint{1.926364in}{2.032046in}}{\pgfqpoint{1.932188in}{2.026222in}}%
\pgfpathcurveto{\pgfqpoint{1.938012in}{2.020398in}}{\pgfqpoint{1.945912in}{2.017126in}}{\pgfqpoint{1.954148in}{2.017126in}}%
\pgfpathclose%
\pgfusepath{stroke,fill}%
\end{pgfscope}%
\begin{pgfscope}%
\pgfpathrectangle{\pgfqpoint{0.100000in}{0.212622in}}{\pgfqpoint{3.696000in}{3.696000in}}%
\pgfusepath{clip}%
\pgfsetbuttcap%
\pgfsetroundjoin%
\definecolor{currentfill}{rgb}{0.121569,0.466667,0.705882}%
\pgfsetfillcolor{currentfill}%
\pgfsetfillopacity{0.308306}%
\pgfsetlinewidth{1.003750pt}%
\definecolor{currentstroke}{rgb}{0.121569,0.466667,0.705882}%
\pgfsetstrokecolor{currentstroke}%
\pgfsetstrokeopacity{0.308306}%
\pgfsetdash{}{0pt}%
\pgfpathmoveto{\pgfqpoint{1.922368in}{2.018635in}}%
\pgfpathcurveto{\pgfqpoint{1.930604in}{2.018635in}}{\pgfqpoint{1.938504in}{2.021907in}}{\pgfqpoint{1.944328in}{2.027731in}}%
\pgfpathcurveto{\pgfqpoint{1.950152in}{2.033555in}}{\pgfqpoint{1.953424in}{2.041455in}}{\pgfqpoint{1.953424in}{2.049691in}}%
\pgfpathcurveto{\pgfqpoint{1.953424in}{2.057928in}}{\pgfqpoint{1.950152in}{2.065828in}}{\pgfqpoint{1.944328in}{2.071651in}}%
\pgfpathcurveto{\pgfqpoint{1.938504in}{2.077475in}}{\pgfqpoint{1.930604in}{2.080748in}}{\pgfqpoint{1.922368in}{2.080748in}}%
\pgfpathcurveto{\pgfqpoint{1.914131in}{2.080748in}}{\pgfqpoint{1.906231in}{2.077475in}}{\pgfqpoint{1.900407in}{2.071651in}}%
\pgfpathcurveto{\pgfqpoint{1.894584in}{2.065828in}}{\pgfqpoint{1.891311in}{2.057928in}}{\pgfqpoint{1.891311in}{2.049691in}}%
\pgfpathcurveto{\pgfqpoint{1.891311in}{2.041455in}}{\pgfqpoint{1.894584in}{2.033555in}}{\pgfqpoint{1.900407in}{2.027731in}}%
\pgfpathcurveto{\pgfqpoint{1.906231in}{2.021907in}}{\pgfqpoint{1.914131in}{2.018635in}}{\pgfqpoint{1.922368in}{2.018635in}}%
\pgfpathclose%
\pgfusepath{stroke,fill}%
\end{pgfscope}%
\begin{pgfscope}%
\pgfpathrectangle{\pgfqpoint{0.100000in}{0.212622in}}{\pgfqpoint{3.696000in}{3.696000in}}%
\pgfusepath{clip}%
\pgfsetbuttcap%
\pgfsetroundjoin%
\definecolor{currentfill}{rgb}{0.121569,0.466667,0.705882}%
\pgfsetfillcolor{currentfill}%
\pgfsetfillopacity{0.308879}%
\pgfsetlinewidth{1.003750pt}%
\definecolor{currentstroke}{rgb}{0.121569,0.466667,0.705882}%
\pgfsetstrokecolor{currentstroke}%
\pgfsetstrokeopacity{0.308879}%
\pgfsetdash{}{0pt}%
\pgfpathmoveto{\pgfqpoint{1.920009in}{2.017439in}}%
\pgfpathcurveto{\pgfqpoint{1.928246in}{2.017439in}}{\pgfqpoint{1.936146in}{2.020711in}}{\pgfqpoint{1.941970in}{2.026535in}}%
\pgfpathcurveto{\pgfqpoint{1.947794in}{2.032359in}}{\pgfqpoint{1.951066in}{2.040259in}}{\pgfqpoint{1.951066in}{2.048495in}}%
\pgfpathcurveto{\pgfqpoint{1.951066in}{2.056732in}}{\pgfqpoint{1.947794in}{2.064632in}}{\pgfqpoint{1.941970in}{2.070456in}}%
\pgfpathcurveto{\pgfqpoint{1.936146in}{2.076280in}}{\pgfqpoint{1.928246in}{2.079552in}}{\pgfqpoint{1.920009in}{2.079552in}}%
\pgfpathcurveto{\pgfqpoint{1.911773in}{2.079552in}}{\pgfqpoint{1.903873in}{2.076280in}}{\pgfqpoint{1.898049in}{2.070456in}}%
\pgfpathcurveto{\pgfqpoint{1.892225in}{2.064632in}}{\pgfqpoint{1.888953in}{2.056732in}}{\pgfqpoint{1.888953in}{2.048495in}}%
\pgfpathcurveto{\pgfqpoint{1.888953in}{2.040259in}}{\pgfqpoint{1.892225in}{2.032359in}}{\pgfqpoint{1.898049in}{2.026535in}}%
\pgfpathcurveto{\pgfqpoint{1.903873in}{2.020711in}}{\pgfqpoint{1.911773in}{2.017439in}}{\pgfqpoint{1.920009in}{2.017439in}}%
\pgfpathclose%
\pgfusepath{stroke,fill}%
\end{pgfscope}%
\begin{pgfscope}%
\pgfpathrectangle{\pgfqpoint{0.100000in}{0.212622in}}{\pgfqpoint{3.696000in}{3.696000in}}%
\pgfusepath{clip}%
\pgfsetbuttcap%
\pgfsetroundjoin%
\definecolor{currentfill}{rgb}{0.121569,0.466667,0.705882}%
\pgfsetfillcolor{currentfill}%
\pgfsetfillopacity{0.309921}%
\pgfsetlinewidth{1.003750pt}%
\definecolor{currentstroke}{rgb}{0.121569,0.466667,0.705882}%
\pgfsetstrokecolor{currentstroke}%
\pgfsetstrokeopacity{0.309921}%
\pgfsetdash{}{0pt}%
\pgfpathmoveto{\pgfqpoint{1.915862in}{2.015000in}}%
\pgfpathcurveto{\pgfqpoint{1.924098in}{2.015000in}}{\pgfqpoint{1.931998in}{2.018273in}}{\pgfqpoint{1.937822in}{2.024096in}}%
\pgfpathcurveto{\pgfqpoint{1.943646in}{2.029920in}}{\pgfqpoint{1.946919in}{2.037820in}}{\pgfqpoint{1.946919in}{2.046057in}}%
\pgfpathcurveto{\pgfqpoint{1.946919in}{2.054293in}}{\pgfqpoint{1.943646in}{2.062193in}}{\pgfqpoint{1.937822in}{2.068017in}}%
\pgfpathcurveto{\pgfqpoint{1.931998in}{2.073841in}}{\pgfqpoint{1.924098in}{2.077113in}}{\pgfqpoint{1.915862in}{2.077113in}}%
\pgfpathcurveto{\pgfqpoint{1.907626in}{2.077113in}}{\pgfqpoint{1.899726in}{2.073841in}}{\pgfqpoint{1.893902in}{2.068017in}}%
\pgfpathcurveto{\pgfqpoint{1.888078in}{2.062193in}}{\pgfqpoint{1.884806in}{2.054293in}}{\pgfqpoint{1.884806in}{2.046057in}}%
\pgfpathcurveto{\pgfqpoint{1.884806in}{2.037820in}}{\pgfqpoint{1.888078in}{2.029920in}}{\pgfqpoint{1.893902in}{2.024096in}}%
\pgfpathcurveto{\pgfqpoint{1.899726in}{2.018273in}}{\pgfqpoint{1.907626in}{2.015000in}}{\pgfqpoint{1.915862in}{2.015000in}}%
\pgfpathclose%
\pgfusepath{stroke,fill}%
\end{pgfscope}%
\begin{pgfscope}%
\pgfpathrectangle{\pgfqpoint{0.100000in}{0.212622in}}{\pgfqpoint{3.696000in}{3.696000in}}%
\pgfusepath{clip}%
\pgfsetbuttcap%
\pgfsetroundjoin%
\definecolor{currentfill}{rgb}{0.121569,0.466667,0.705882}%
\pgfsetfillcolor{currentfill}%
\pgfsetfillopacity{0.310025}%
\pgfsetlinewidth{1.003750pt}%
\definecolor{currentstroke}{rgb}{0.121569,0.466667,0.705882}%
\pgfsetstrokecolor{currentstroke}%
\pgfsetstrokeopacity{0.310025}%
\pgfsetdash{}{0pt}%
\pgfpathmoveto{\pgfqpoint{1.956741in}{2.012951in}}%
\pgfpathcurveto{\pgfqpoint{1.964977in}{2.012951in}}{\pgfqpoint{1.972877in}{2.016223in}}{\pgfqpoint{1.978701in}{2.022047in}}%
\pgfpathcurveto{\pgfqpoint{1.984525in}{2.027871in}}{\pgfqpoint{1.987797in}{2.035771in}}{\pgfqpoint{1.987797in}{2.044008in}}%
\pgfpathcurveto{\pgfqpoint{1.987797in}{2.052244in}}{\pgfqpoint{1.984525in}{2.060144in}}{\pgfqpoint{1.978701in}{2.065968in}}%
\pgfpathcurveto{\pgfqpoint{1.972877in}{2.071792in}}{\pgfqpoint{1.964977in}{2.075064in}}{\pgfqpoint{1.956741in}{2.075064in}}%
\pgfpathcurveto{\pgfqpoint{1.948504in}{2.075064in}}{\pgfqpoint{1.940604in}{2.071792in}}{\pgfqpoint{1.934780in}{2.065968in}}%
\pgfpathcurveto{\pgfqpoint{1.928957in}{2.060144in}}{\pgfqpoint{1.925684in}{2.052244in}}{\pgfqpoint{1.925684in}{2.044008in}}%
\pgfpathcurveto{\pgfqpoint{1.925684in}{2.035771in}}{\pgfqpoint{1.928957in}{2.027871in}}{\pgfqpoint{1.934780in}{2.022047in}}%
\pgfpathcurveto{\pgfqpoint{1.940604in}{2.016223in}}{\pgfqpoint{1.948504in}{2.012951in}}{\pgfqpoint{1.956741in}{2.012951in}}%
\pgfpathclose%
\pgfusepath{stroke,fill}%
\end{pgfscope}%
\begin{pgfscope}%
\pgfpathrectangle{\pgfqpoint{0.100000in}{0.212622in}}{\pgfqpoint{3.696000in}{3.696000in}}%
\pgfusepath{clip}%
\pgfsetbuttcap%
\pgfsetroundjoin%
\definecolor{currentfill}{rgb}{0.121569,0.466667,0.705882}%
\pgfsetfillcolor{currentfill}%
\pgfsetfillopacity{0.311814}%
\pgfsetlinewidth{1.003750pt}%
\definecolor{currentstroke}{rgb}{0.121569,0.466667,0.705882}%
\pgfsetstrokecolor{currentstroke}%
\pgfsetstrokeopacity{0.311814}%
\pgfsetdash{}{0pt}%
\pgfpathmoveto{\pgfqpoint{1.958161in}{2.013751in}}%
\pgfpathcurveto{\pgfqpoint{1.966398in}{2.013751in}}{\pgfqpoint{1.974298in}{2.017024in}}{\pgfqpoint{1.980122in}{2.022848in}}%
\pgfpathcurveto{\pgfqpoint{1.985945in}{2.028672in}}{\pgfqpoint{1.989218in}{2.036572in}}{\pgfqpoint{1.989218in}{2.044808in}}%
\pgfpathcurveto{\pgfqpoint{1.989218in}{2.053044in}}{\pgfqpoint{1.985945in}{2.060944in}}{\pgfqpoint{1.980122in}{2.066768in}}%
\pgfpathcurveto{\pgfqpoint{1.974298in}{2.072592in}}{\pgfqpoint{1.966398in}{2.075864in}}{\pgfqpoint{1.958161in}{2.075864in}}%
\pgfpathcurveto{\pgfqpoint{1.949925in}{2.075864in}}{\pgfqpoint{1.942025in}{2.072592in}}{\pgfqpoint{1.936201in}{2.066768in}}%
\pgfpathcurveto{\pgfqpoint{1.930377in}{2.060944in}}{\pgfqpoint{1.927105in}{2.053044in}}{\pgfqpoint{1.927105in}{2.044808in}}%
\pgfpathcurveto{\pgfqpoint{1.927105in}{2.036572in}}{\pgfqpoint{1.930377in}{2.028672in}}{\pgfqpoint{1.936201in}{2.022848in}}%
\pgfpathcurveto{\pgfqpoint{1.942025in}{2.017024in}}{\pgfqpoint{1.949925in}{2.013751in}}{\pgfqpoint{1.958161in}{2.013751in}}%
\pgfpathclose%
\pgfusepath{stroke,fill}%
\end{pgfscope}%
\begin{pgfscope}%
\pgfpathrectangle{\pgfqpoint{0.100000in}{0.212622in}}{\pgfqpoint{3.696000in}{3.696000in}}%
\pgfusepath{clip}%
\pgfsetbuttcap%
\pgfsetroundjoin%
\definecolor{currentfill}{rgb}{0.121569,0.466667,0.705882}%
\pgfsetfillcolor{currentfill}%
\pgfsetfillopacity{0.312492}%
\pgfsetlinewidth{1.003750pt}%
\definecolor{currentstroke}{rgb}{0.121569,0.466667,0.705882}%
\pgfsetstrokecolor{currentstroke}%
\pgfsetstrokeopacity{0.312492}%
\pgfsetdash{}{0pt}%
\pgfpathmoveto{\pgfqpoint{1.909042in}{2.014056in}}%
\pgfpathcurveto{\pgfqpoint{1.917279in}{2.014056in}}{\pgfqpoint{1.925179in}{2.017328in}}{\pgfqpoint{1.931003in}{2.023152in}}%
\pgfpathcurveto{\pgfqpoint{1.936827in}{2.028976in}}{\pgfqpoint{1.940099in}{2.036876in}}{\pgfqpoint{1.940099in}{2.045112in}}%
\pgfpathcurveto{\pgfqpoint{1.940099in}{2.053348in}}{\pgfqpoint{1.936827in}{2.061248in}}{\pgfqpoint{1.931003in}{2.067072in}}%
\pgfpathcurveto{\pgfqpoint{1.925179in}{2.072896in}}{\pgfqpoint{1.917279in}{2.076169in}}{\pgfqpoint{1.909042in}{2.076169in}}%
\pgfpathcurveto{\pgfqpoint{1.900806in}{2.076169in}}{\pgfqpoint{1.892906in}{2.072896in}}{\pgfqpoint{1.887082in}{2.067072in}}%
\pgfpathcurveto{\pgfqpoint{1.881258in}{2.061248in}}{\pgfqpoint{1.877986in}{2.053348in}}{\pgfqpoint{1.877986in}{2.045112in}}%
\pgfpathcurveto{\pgfqpoint{1.877986in}{2.036876in}}{\pgfqpoint{1.881258in}{2.028976in}}{\pgfqpoint{1.887082in}{2.023152in}}%
\pgfpathcurveto{\pgfqpoint{1.892906in}{2.017328in}}{\pgfqpoint{1.900806in}{2.014056in}}{\pgfqpoint{1.909042in}{2.014056in}}%
\pgfpathclose%
\pgfusepath{stroke,fill}%
\end{pgfscope}%
\begin{pgfscope}%
\pgfpathrectangle{\pgfqpoint{0.100000in}{0.212622in}}{\pgfqpoint{3.696000in}{3.696000in}}%
\pgfusepath{clip}%
\pgfsetbuttcap%
\pgfsetroundjoin%
\definecolor{currentfill}{rgb}{0.121569,0.466667,0.705882}%
\pgfsetfillcolor{currentfill}%
\pgfsetfillopacity{0.313530}%
\pgfsetlinewidth{1.003750pt}%
\definecolor{currentstroke}{rgb}{0.121569,0.466667,0.705882}%
\pgfsetstrokecolor{currentstroke}%
\pgfsetstrokeopacity{0.313530}%
\pgfsetdash{}{0pt}%
\pgfpathmoveto{\pgfqpoint{1.904493in}{2.011481in}}%
\pgfpathcurveto{\pgfqpoint{1.912729in}{2.011481in}}{\pgfqpoint{1.920630in}{2.014754in}}{\pgfqpoint{1.926453in}{2.020578in}}%
\pgfpathcurveto{\pgfqpoint{1.932277in}{2.026402in}}{\pgfqpoint{1.935550in}{2.034302in}}{\pgfqpoint{1.935550in}{2.042538in}}%
\pgfpathcurveto{\pgfqpoint{1.935550in}{2.050774in}}{\pgfqpoint{1.932277in}{2.058674in}}{\pgfqpoint{1.926453in}{2.064498in}}%
\pgfpathcurveto{\pgfqpoint{1.920630in}{2.070322in}}{\pgfqpoint{1.912729in}{2.073594in}}{\pgfqpoint{1.904493in}{2.073594in}}%
\pgfpathcurveto{\pgfqpoint{1.896257in}{2.073594in}}{\pgfqpoint{1.888357in}{2.070322in}}{\pgfqpoint{1.882533in}{2.064498in}}%
\pgfpathcurveto{\pgfqpoint{1.876709in}{2.058674in}}{\pgfqpoint{1.873437in}{2.050774in}}{\pgfqpoint{1.873437in}{2.042538in}}%
\pgfpathcurveto{\pgfqpoint{1.873437in}{2.034302in}}{\pgfqpoint{1.876709in}{2.026402in}}{\pgfqpoint{1.882533in}{2.020578in}}%
\pgfpathcurveto{\pgfqpoint{1.888357in}{2.014754in}}{\pgfqpoint{1.896257in}{2.011481in}}{\pgfqpoint{1.904493in}{2.011481in}}%
\pgfpathclose%
\pgfusepath{stroke,fill}%
\end{pgfscope}%
\begin{pgfscope}%
\pgfpathrectangle{\pgfqpoint{0.100000in}{0.212622in}}{\pgfqpoint{3.696000in}{3.696000in}}%
\pgfusepath{clip}%
\pgfsetbuttcap%
\pgfsetroundjoin%
\definecolor{currentfill}{rgb}{0.121569,0.466667,0.705882}%
\pgfsetfillcolor{currentfill}%
\pgfsetfillopacity{0.313800}%
\pgfsetlinewidth{1.003750pt}%
\definecolor{currentstroke}{rgb}{0.121569,0.466667,0.705882}%
\pgfsetstrokecolor{currentstroke}%
\pgfsetstrokeopacity{0.313800}%
\pgfsetdash{}{0pt}%
\pgfpathmoveto{\pgfqpoint{1.958574in}{2.010573in}}%
\pgfpathcurveto{\pgfqpoint{1.966810in}{2.010573in}}{\pgfqpoint{1.974710in}{2.013845in}}{\pgfqpoint{1.980534in}{2.019669in}}%
\pgfpathcurveto{\pgfqpoint{1.986358in}{2.025493in}}{\pgfqpoint{1.989630in}{2.033393in}}{\pgfqpoint{1.989630in}{2.041629in}}%
\pgfpathcurveto{\pgfqpoint{1.989630in}{2.049865in}}{\pgfqpoint{1.986358in}{2.057765in}}{\pgfqpoint{1.980534in}{2.063589in}}%
\pgfpathcurveto{\pgfqpoint{1.974710in}{2.069413in}}{\pgfqpoint{1.966810in}{2.072686in}}{\pgfqpoint{1.958574in}{2.072686in}}%
\pgfpathcurveto{\pgfqpoint{1.950338in}{2.072686in}}{\pgfqpoint{1.942437in}{2.069413in}}{\pgfqpoint{1.936614in}{2.063589in}}%
\pgfpathcurveto{\pgfqpoint{1.930790in}{2.057765in}}{\pgfqpoint{1.927517in}{2.049865in}}{\pgfqpoint{1.927517in}{2.041629in}}%
\pgfpathcurveto{\pgfqpoint{1.927517in}{2.033393in}}{\pgfqpoint{1.930790in}{2.025493in}}{\pgfqpoint{1.936614in}{2.019669in}}%
\pgfpathcurveto{\pgfqpoint{1.942437in}{2.013845in}}{\pgfqpoint{1.950338in}{2.010573in}}{\pgfqpoint{1.958574in}{2.010573in}}%
\pgfpathclose%
\pgfusepath{stroke,fill}%
\end{pgfscope}%
\begin{pgfscope}%
\pgfpathrectangle{\pgfqpoint{0.100000in}{0.212622in}}{\pgfqpoint{3.696000in}{3.696000in}}%
\pgfusepath{clip}%
\pgfsetbuttcap%
\pgfsetroundjoin%
\definecolor{currentfill}{rgb}{0.121569,0.466667,0.705882}%
\pgfsetfillcolor{currentfill}%
\pgfsetfillopacity{0.314431}%
\pgfsetlinewidth{1.003750pt}%
\definecolor{currentstroke}{rgb}{0.121569,0.466667,0.705882}%
\pgfsetstrokecolor{currentstroke}%
\pgfsetstrokeopacity{0.314431}%
\pgfsetdash{}{0pt}%
\pgfpathmoveto{\pgfqpoint{1.900587in}{2.008506in}}%
\pgfpathcurveto{\pgfqpoint{1.908823in}{2.008506in}}{\pgfqpoint{1.916723in}{2.011778in}}{\pgfqpoint{1.922547in}{2.017602in}}%
\pgfpathcurveto{\pgfqpoint{1.928371in}{2.023426in}}{\pgfqpoint{1.931643in}{2.031326in}}{\pgfqpoint{1.931643in}{2.039562in}}%
\pgfpathcurveto{\pgfqpoint{1.931643in}{2.047799in}}{\pgfqpoint{1.928371in}{2.055699in}}{\pgfqpoint{1.922547in}{2.061523in}}%
\pgfpathcurveto{\pgfqpoint{1.916723in}{2.067347in}}{\pgfqpoint{1.908823in}{2.070619in}}{\pgfqpoint{1.900587in}{2.070619in}}%
\pgfpathcurveto{\pgfqpoint{1.892351in}{2.070619in}}{\pgfqpoint{1.884451in}{2.067347in}}{\pgfqpoint{1.878627in}{2.061523in}}%
\pgfpathcurveto{\pgfqpoint{1.872803in}{2.055699in}}{\pgfqpoint{1.869530in}{2.047799in}}{\pgfqpoint{1.869530in}{2.039562in}}%
\pgfpathcurveto{\pgfqpoint{1.869530in}{2.031326in}}{\pgfqpoint{1.872803in}{2.023426in}}{\pgfqpoint{1.878627in}{2.017602in}}%
\pgfpathcurveto{\pgfqpoint{1.884451in}{2.011778in}}{\pgfqpoint{1.892351in}{2.008506in}}{\pgfqpoint{1.900587in}{2.008506in}}%
\pgfpathclose%
\pgfusepath{stroke,fill}%
\end{pgfscope}%
\begin{pgfscope}%
\pgfpathrectangle{\pgfqpoint{0.100000in}{0.212622in}}{\pgfqpoint{3.696000in}{3.696000in}}%
\pgfusepath{clip}%
\pgfsetbuttcap%
\pgfsetroundjoin%
\definecolor{currentfill}{rgb}{0.121569,0.466667,0.705882}%
\pgfsetfillcolor{currentfill}%
\pgfsetfillopacity{0.314979}%
\pgfsetlinewidth{1.003750pt}%
\definecolor{currentstroke}{rgb}{0.121569,0.466667,0.705882}%
\pgfsetstrokecolor{currentstroke}%
\pgfsetstrokeopacity{0.314979}%
\pgfsetdash{}{0pt}%
\pgfpathmoveto{\pgfqpoint{1.959619in}{2.009814in}}%
\pgfpathcurveto{\pgfqpoint{1.967855in}{2.009814in}}{\pgfqpoint{1.975755in}{2.013086in}}{\pgfqpoint{1.981579in}{2.018910in}}%
\pgfpathcurveto{\pgfqpoint{1.987403in}{2.024734in}}{\pgfqpoint{1.990675in}{2.032634in}}{\pgfqpoint{1.990675in}{2.040870in}}%
\pgfpathcurveto{\pgfqpoint{1.990675in}{2.049107in}}{\pgfqpoint{1.987403in}{2.057007in}}{\pgfqpoint{1.981579in}{2.062831in}}%
\pgfpathcurveto{\pgfqpoint{1.975755in}{2.068654in}}{\pgfqpoint{1.967855in}{2.071927in}}{\pgfqpoint{1.959619in}{2.071927in}}%
\pgfpathcurveto{\pgfqpoint{1.951383in}{2.071927in}}{\pgfqpoint{1.943483in}{2.068654in}}{\pgfqpoint{1.937659in}{2.062831in}}%
\pgfpathcurveto{\pgfqpoint{1.931835in}{2.057007in}}{\pgfqpoint{1.928562in}{2.049107in}}{\pgfqpoint{1.928562in}{2.040870in}}%
\pgfpathcurveto{\pgfqpoint{1.928562in}{2.032634in}}{\pgfqpoint{1.931835in}{2.024734in}}{\pgfqpoint{1.937659in}{2.018910in}}%
\pgfpathcurveto{\pgfqpoint{1.943483in}{2.013086in}}{\pgfqpoint{1.951383in}{2.009814in}}{\pgfqpoint{1.959619in}{2.009814in}}%
\pgfpathclose%
\pgfusepath{stroke,fill}%
\end{pgfscope}%
\begin{pgfscope}%
\pgfpathrectangle{\pgfqpoint{0.100000in}{0.212622in}}{\pgfqpoint{3.696000in}{3.696000in}}%
\pgfusepath{clip}%
\pgfsetbuttcap%
\pgfsetroundjoin%
\definecolor{currentfill}{rgb}{0.121569,0.466667,0.705882}%
\pgfsetfillcolor{currentfill}%
\pgfsetfillopacity{0.316500}%
\pgfsetlinewidth{1.003750pt}%
\definecolor{currentstroke}{rgb}{0.121569,0.466667,0.705882}%
\pgfsetstrokecolor{currentstroke}%
\pgfsetstrokeopacity{0.316500}%
\pgfsetdash{}{0pt}%
\pgfpathmoveto{\pgfqpoint{1.894540in}{2.004435in}}%
\pgfpathcurveto{\pgfqpoint{1.902776in}{2.004435in}}{\pgfqpoint{1.910676in}{2.007707in}}{\pgfqpoint{1.916500in}{2.013531in}}%
\pgfpathcurveto{\pgfqpoint{1.922324in}{2.019355in}}{\pgfqpoint{1.925597in}{2.027255in}}{\pgfqpoint{1.925597in}{2.035492in}}%
\pgfpathcurveto{\pgfqpoint{1.925597in}{2.043728in}}{\pgfqpoint{1.922324in}{2.051628in}}{\pgfqpoint{1.916500in}{2.057452in}}%
\pgfpathcurveto{\pgfqpoint{1.910676in}{2.063276in}}{\pgfqpoint{1.902776in}{2.066548in}}{\pgfqpoint{1.894540in}{2.066548in}}%
\pgfpathcurveto{\pgfqpoint{1.886304in}{2.066548in}}{\pgfqpoint{1.878404in}{2.063276in}}{\pgfqpoint{1.872580in}{2.057452in}}%
\pgfpathcurveto{\pgfqpoint{1.866756in}{2.051628in}}{\pgfqpoint{1.863484in}{2.043728in}}{\pgfqpoint{1.863484in}{2.035492in}}%
\pgfpathcurveto{\pgfqpoint{1.863484in}{2.027255in}}{\pgfqpoint{1.866756in}{2.019355in}}{\pgfqpoint{1.872580in}{2.013531in}}%
\pgfpathcurveto{\pgfqpoint{1.878404in}{2.007707in}}{\pgfqpoint{1.886304in}{2.004435in}}{\pgfqpoint{1.894540in}{2.004435in}}%
\pgfpathclose%
\pgfusepath{stroke,fill}%
\end{pgfscope}%
\begin{pgfscope}%
\pgfpathrectangle{\pgfqpoint{0.100000in}{0.212622in}}{\pgfqpoint{3.696000in}{3.696000in}}%
\pgfusepath{clip}%
\pgfsetbuttcap%
\pgfsetroundjoin%
\definecolor{currentfill}{rgb}{0.121569,0.466667,0.705882}%
\pgfsetfillcolor{currentfill}%
\pgfsetfillopacity{0.316882}%
\pgfsetlinewidth{1.003750pt}%
\definecolor{currentstroke}{rgb}{0.121569,0.466667,0.705882}%
\pgfsetstrokecolor{currentstroke}%
\pgfsetstrokeopacity{0.316882}%
\pgfsetdash{}{0pt}%
\pgfpathmoveto{\pgfqpoint{1.961435in}{2.011407in}}%
\pgfpathcurveto{\pgfqpoint{1.969671in}{2.011407in}}{\pgfqpoint{1.977571in}{2.014679in}}{\pgfqpoint{1.983395in}{2.020503in}}%
\pgfpathcurveto{\pgfqpoint{1.989219in}{2.026327in}}{\pgfqpoint{1.992491in}{2.034227in}}{\pgfqpoint{1.992491in}{2.042463in}}%
\pgfpathcurveto{\pgfqpoint{1.992491in}{2.050700in}}{\pgfqpoint{1.989219in}{2.058600in}}{\pgfqpoint{1.983395in}{2.064424in}}%
\pgfpathcurveto{\pgfqpoint{1.977571in}{2.070248in}}{\pgfqpoint{1.969671in}{2.073520in}}{\pgfqpoint{1.961435in}{2.073520in}}%
\pgfpathcurveto{\pgfqpoint{1.953199in}{2.073520in}}{\pgfqpoint{1.945299in}{2.070248in}}{\pgfqpoint{1.939475in}{2.064424in}}%
\pgfpathcurveto{\pgfqpoint{1.933651in}{2.058600in}}{\pgfqpoint{1.930378in}{2.050700in}}{\pgfqpoint{1.930378in}{2.042463in}}%
\pgfpathcurveto{\pgfqpoint{1.930378in}{2.034227in}}{\pgfqpoint{1.933651in}{2.026327in}}{\pgfqpoint{1.939475in}{2.020503in}}%
\pgfpathcurveto{\pgfqpoint{1.945299in}{2.014679in}}{\pgfqpoint{1.953199in}{2.011407in}}{\pgfqpoint{1.961435in}{2.011407in}}%
\pgfpathclose%
\pgfusepath{stroke,fill}%
\end{pgfscope}%
\begin{pgfscope}%
\pgfpathrectangle{\pgfqpoint{0.100000in}{0.212622in}}{\pgfqpoint{3.696000in}{3.696000in}}%
\pgfusepath{clip}%
\pgfsetbuttcap%
\pgfsetroundjoin%
\definecolor{currentfill}{rgb}{0.121569,0.466667,0.705882}%
\pgfsetfillcolor{currentfill}%
\pgfsetfillopacity{0.317565}%
\pgfsetlinewidth{1.003750pt}%
\definecolor{currentstroke}{rgb}{0.121569,0.466667,0.705882}%
\pgfsetstrokecolor{currentstroke}%
\pgfsetstrokeopacity{0.317565}%
\pgfsetdash{}{0pt}%
\pgfpathmoveto{\pgfqpoint{1.889775in}{2.000709in}}%
\pgfpathcurveto{\pgfqpoint{1.898011in}{2.000709in}}{\pgfqpoint{1.905911in}{2.003981in}}{\pgfqpoint{1.911735in}{2.009805in}}%
\pgfpathcurveto{\pgfqpoint{1.917559in}{2.015629in}}{\pgfqpoint{1.920832in}{2.023529in}}{\pgfqpoint{1.920832in}{2.031765in}}%
\pgfpathcurveto{\pgfqpoint{1.920832in}{2.040002in}}{\pgfqpoint{1.917559in}{2.047902in}}{\pgfqpoint{1.911735in}{2.053726in}}%
\pgfpathcurveto{\pgfqpoint{1.905911in}{2.059550in}}{\pgfqpoint{1.898011in}{2.062822in}}{\pgfqpoint{1.889775in}{2.062822in}}%
\pgfpathcurveto{\pgfqpoint{1.881539in}{2.062822in}}{\pgfqpoint{1.873639in}{2.059550in}}{\pgfqpoint{1.867815in}{2.053726in}}%
\pgfpathcurveto{\pgfqpoint{1.861991in}{2.047902in}}{\pgfqpoint{1.858719in}{2.040002in}}{\pgfqpoint{1.858719in}{2.031765in}}%
\pgfpathcurveto{\pgfqpoint{1.858719in}{2.023529in}}{\pgfqpoint{1.861991in}{2.015629in}}{\pgfqpoint{1.867815in}{2.009805in}}%
\pgfpathcurveto{\pgfqpoint{1.873639in}{2.003981in}}{\pgfqpoint{1.881539in}{2.000709in}}{\pgfqpoint{1.889775in}{2.000709in}}%
\pgfpathclose%
\pgfusepath{stroke,fill}%
\end{pgfscope}%
\begin{pgfscope}%
\pgfpathrectangle{\pgfqpoint{0.100000in}{0.212622in}}{\pgfqpoint{3.696000in}{3.696000in}}%
\pgfusepath{clip}%
\pgfsetbuttcap%
\pgfsetroundjoin%
\definecolor{currentfill}{rgb}{0.121569,0.466667,0.705882}%
\pgfsetfillcolor{currentfill}%
\pgfsetfillopacity{0.317750}%
\pgfsetlinewidth{1.003750pt}%
\definecolor{currentstroke}{rgb}{0.121569,0.466667,0.705882}%
\pgfsetstrokecolor{currentstroke}%
\pgfsetstrokeopacity{0.317750}%
\pgfsetdash{}{0pt}%
\pgfpathmoveto{\pgfqpoint{1.961618in}{2.010589in}}%
\pgfpathcurveto{\pgfqpoint{1.969855in}{2.010589in}}{\pgfqpoint{1.977755in}{2.013861in}}{\pgfqpoint{1.983579in}{2.019685in}}%
\pgfpathcurveto{\pgfqpoint{1.989403in}{2.025509in}}{\pgfqpoint{1.992675in}{2.033409in}}{\pgfqpoint{1.992675in}{2.041646in}}%
\pgfpathcurveto{\pgfqpoint{1.992675in}{2.049882in}}{\pgfqpoint{1.989403in}{2.057782in}}{\pgfqpoint{1.983579in}{2.063606in}}%
\pgfpathcurveto{\pgfqpoint{1.977755in}{2.069430in}}{\pgfqpoint{1.969855in}{2.072702in}}{\pgfqpoint{1.961618in}{2.072702in}}%
\pgfpathcurveto{\pgfqpoint{1.953382in}{2.072702in}}{\pgfqpoint{1.945482in}{2.069430in}}{\pgfqpoint{1.939658in}{2.063606in}}%
\pgfpathcurveto{\pgfqpoint{1.933834in}{2.057782in}}{\pgfqpoint{1.930562in}{2.049882in}}{\pgfqpoint{1.930562in}{2.041646in}}%
\pgfpathcurveto{\pgfqpoint{1.930562in}{2.033409in}}{\pgfqpoint{1.933834in}{2.025509in}}{\pgfqpoint{1.939658in}{2.019685in}}%
\pgfpathcurveto{\pgfqpoint{1.945482in}{2.013861in}}{\pgfqpoint{1.953382in}{2.010589in}}{\pgfqpoint{1.961618in}{2.010589in}}%
\pgfpathclose%
\pgfusepath{stroke,fill}%
\end{pgfscope}%
\begin{pgfscope}%
\pgfpathrectangle{\pgfqpoint{0.100000in}{0.212622in}}{\pgfqpoint{3.696000in}{3.696000in}}%
\pgfusepath{clip}%
\pgfsetbuttcap%
\pgfsetroundjoin%
\definecolor{currentfill}{rgb}{0.121569,0.466667,0.705882}%
\pgfsetfillcolor{currentfill}%
\pgfsetfillopacity{0.318847}%
\pgfsetlinewidth{1.003750pt}%
\definecolor{currentstroke}{rgb}{0.121569,0.466667,0.705882}%
\pgfsetstrokecolor{currentstroke}%
\pgfsetstrokeopacity{0.318847}%
\pgfsetdash{}{0pt}%
\pgfpathmoveto{\pgfqpoint{1.885657in}{1.998792in}}%
\pgfpathcurveto{\pgfqpoint{1.893894in}{1.998792in}}{\pgfqpoint{1.901794in}{2.002064in}}{\pgfqpoint{1.907618in}{2.007888in}}%
\pgfpathcurveto{\pgfqpoint{1.913442in}{2.013712in}}{\pgfqpoint{1.916714in}{2.021612in}}{\pgfqpoint{1.916714in}{2.029848in}}%
\pgfpathcurveto{\pgfqpoint{1.916714in}{2.038084in}}{\pgfqpoint{1.913442in}{2.045984in}}{\pgfqpoint{1.907618in}{2.051808in}}%
\pgfpathcurveto{\pgfqpoint{1.901794in}{2.057632in}}{\pgfqpoint{1.893894in}{2.060905in}}{\pgfqpoint{1.885657in}{2.060905in}}%
\pgfpathcurveto{\pgfqpoint{1.877421in}{2.060905in}}{\pgfqpoint{1.869521in}{2.057632in}}{\pgfqpoint{1.863697in}{2.051808in}}%
\pgfpathcurveto{\pgfqpoint{1.857873in}{2.045984in}}{\pgfqpoint{1.854601in}{2.038084in}}{\pgfqpoint{1.854601in}{2.029848in}}%
\pgfpathcurveto{\pgfqpoint{1.854601in}{2.021612in}}{\pgfqpoint{1.857873in}{2.013712in}}{\pgfqpoint{1.863697in}{2.007888in}}%
\pgfpathcurveto{\pgfqpoint{1.869521in}{2.002064in}}{\pgfqpoint{1.877421in}{1.998792in}}{\pgfqpoint{1.885657in}{1.998792in}}%
\pgfpathclose%
\pgfusepath{stroke,fill}%
\end{pgfscope}%
\begin{pgfscope}%
\pgfpathrectangle{\pgfqpoint{0.100000in}{0.212622in}}{\pgfqpoint{3.696000in}{3.696000in}}%
\pgfusepath{clip}%
\pgfsetbuttcap%
\pgfsetroundjoin%
\definecolor{currentfill}{rgb}{0.121569,0.466667,0.705882}%
\pgfsetfillcolor{currentfill}%
\pgfsetfillopacity{0.319293}%
\pgfsetlinewidth{1.003750pt}%
\definecolor{currentstroke}{rgb}{0.121569,0.466667,0.705882}%
\pgfsetstrokecolor{currentstroke}%
\pgfsetstrokeopacity{0.319293}%
\pgfsetdash{}{0pt}%
\pgfpathmoveto{\pgfqpoint{1.963072in}{2.009672in}}%
\pgfpathcurveto{\pgfqpoint{1.971308in}{2.009672in}}{\pgfqpoint{1.979208in}{2.012944in}}{\pgfqpoint{1.985032in}{2.018768in}}%
\pgfpathcurveto{\pgfqpoint{1.990856in}{2.024592in}}{\pgfqpoint{1.994128in}{2.032492in}}{\pgfqpoint{1.994128in}{2.040729in}}%
\pgfpathcurveto{\pgfqpoint{1.994128in}{2.048965in}}{\pgfqpoint{1.990856in}{2.056865in}}{\pgfqpoint{1.985032in}{2.062689in}}%
\pgfpathcurveto{\pgfqpoint{1.979208in}{2.068513in}}{\pgfqpoint{1.971308in}{2.071785in}}{\pgfqpoint{1.963072in}{2.071785in}}%
\pgfpathcurveto{\pgfqpoint{1.954836in}{2.071785in}}{\pgfqpoint{1.946936in}{2.068513in}}{\pgfqpoint{1.941112in}{2.062689in}}%
\pgfpathcurveto{\pgfqpoint{1.935288in}{2.056865in}}{\pgfqpoint{1.932015in}{2.048965in}}{\pgfqpoint{1.932015in}{2.040729in}}%
\pgfpathcurveto{\pgfqpoint{1.932015in}{2.032492in}}{\pgfqpoint{1.935288in}{2.024592in}}{\pgfqpoint{1.941112in}{2.018768in}}%
\pgfpathcurveto{\pgfqpoint{1.946936in}{2.012944in}}{\pgfqpoint{1.954836in}{2.009672in}}{\pgfqpoint{1.963072in}{2.009672in}}%
\pgfpathclose%
\pgfusepath{stroke,fill}%
\end{pgfscope}%
\begin{pgfscope}%
\pgfpathrectangle{\pgfqpoint{0.100000in}{0.212622in}}{\pgfqpoint{3.696000in}{3.696000in}}%
\pgfusepath{clip}%
\pgfsetbuttcap%
\pgfsetroundjoin%
\definecolor{currentfill}{rgb}{0.121569,0.466667,0.705882}%
\pgfsetfillcolor{currentfill}%
\pgfsetfillopacity{0.321364}%
\pgfsetlinewidth{1.003750pt}%
\definecolor{currentstroke}{rgb}{0.121569,0.466667,0.705882}%
\pgfsetstrokecolor{currentstroke}%
\pgfsetstrokeopacity{0.321364}%
\pgfsetdash{}{0pt}%
\pgfpathmoveto{\pgfqpoint{1.964962in}{2.009837in}}%
\pgfpathcurveto{\pgfqpoint{1.973198in}{2.009837in}}{\pgfqpoint{1.981098in}{2.013109in}}{\pgfqpoint{1.986922in}{2.018933in}}%
\pgfpathcurveto{\pgfqpoint{1.992746in}{2.024757in}}{\pgfqpoint{1.996018in}{2.032657in}}{\pgfqpoint{1.996018in}{2.040893in}}%
\pgfpathcurveto{\pgfqpoint{1.996018in}{2.049129in}}{\pgfqpoint{1.992746in}{2.057029in}}{\pgfqpoint{1.986922in}{2.062853in}}%
\pgfpathcurveto{\pgfqpoint{1.981098in}{2.068677in}}{\pgfqpoint{1.973198in}{2.071950in}}{\pgfqpoint{1.964962in}{2.071950in}}%
\pgfpathcurveto{\pgfqpoint{1.956725in}{2.071950in}}{\pgfqpoint{1.948825in}{2.068677in}}{\pgfqpoint{1.943001in}{2.062853in}}%
\pgfpathcurveto{\pgfqpoint{1.937177in}{2.057029in}}{\pgfqpoint{1.933905in}{2.049129in}}{\pgfqpoint{1.933905in}{2.040893in}}%
\pgfpathcurveto{\pgfqpoint{1.933905in}{2.032657in}}{\pgfqpoint{1.937177in}{2.024757in}}{\pgfqpoint{1.943001in}{2.018933in}}%
\pgfpathcurveto{\pgfqpoint{1.948825in}{2.013109in}}{\pgfqpoint{1.956725in}{2.009837in}}{\pgfqpoint{1.964962in}{2.009837in}}%
\pgfpathclose%
\pgfusepath{stroke,fill}%
\end{pgfscope}%
\begin{pgfscope}%
\pgfpathrectangle{\pgfqpoint{0.100000in}{0.212622in}}{\pgfqpoint{3.696000in}{3.696000in}}%
\pgfusepath{clip}%
\pgfsetbuttcap%
\pgfsetroundjoin%
\definecolor{currentfill}{rgb}{0.121569,0.466667,0.705882}%
\pgfsetfillcolor{currentfill}%
\pgfsetfillopacity{0.321416}%
\pgfsetlinewidth{1.003750pt}%
\definecolor{currentstroke}{rgb}{0.121569,0.466667,0.705882}%
\pgfsetstrokecolor{currentstroke}%
\pgfsetstrokeopacity{0.321416}%
\pgfsetdash{}{0pt}%
\pgfpathmoveto{\pgfqpoint{1.879478in}{1.995172in}}%
\pgfpathcurveto{\pgfqpoint{1.887714in}{1.995172in}}{\pgfqpoint{1.895614in}{1.998444in}}{\pgfqpoint{1.901438in}{2.004268in}}%
\pgfpathcurveto{\pgfqpoint{1.907262in}{2.010092in}}{\pgfqpoint{1.910534in}{2.017992in}}{\pgfqpoint{1.910534in}{2.026228in}}%
\pgfpathcurveto{\pgfqpoint{1.910534in}{2.034465in}}{\pgfqpoint{1.907262in}{2.042365in}}{\pgfqpoint{1.901438in}{2.048189in}}%
\pgfpathcurveto{\pgfqpoint{1.895614in}{2.054012in}}{\pgfqpoint{1.887714in}{2.057285in}}{\pgfqpoint{1.879478in}{2.057285in}}%
\pgfpathcurveto{\pgfqpoint{1.871242in}{2.057285in}}{\pgfqpoint{1.863341in}{2.054012in}}{\pgfqpoint{1.857518in}{2.048189in}}%
\pgfpathcurveto{\pgfqpoint{1.851694in}{2.042365in}}{\pgfqpoint{1.848421in}{2.034465in}}{\pgfqpoint{1.848421in}{2.026228in}}%
\pgfpathcurveto{\pgfqpoint{1.848421in}{2.017992in}}{\pgfqpoint{1.851694in}{2.010092in}}{\pgfqpoint{1.857518in}{2.004268in}}%
\pgfpathcurveto{\pgfqpoint{1.863341in}{1.998444in}}{\pgfqpoint{1.871242in}{1.995172in}}{\pgfqpoint{1.879478in}{1.995172in}}%
\pgfpathclose%
\pgfusepath{stroke,fill}%
\end{pgfscope}%
\begin{pgfscope}%
\pgfpathrectangle{\pgfqpoint{0.100000in}{0.212622in}}{\pgfqpoint{3.696000in}{3.696000in}}%
\pgfusepath{clip}%
\pgfsetbuttcap%
\pgfsetroundjoin%
\definecolor{currentfill}{rgb}{0.121569,0.466667,0.705882}%
\pgfsetfillcolor{currentfill}%
\pgfsetfillopacity{0.322566}%
\pgfsetlinewidth{1.003750pt}%
\definecolor{currentstroke}{rgb}{0.121569,0.466667,0.705882}%
\pgfsetstrokecolor{currentstroke}%
\pgfsetstrokeopacity{0.322566}%
\pgfsetdash{}{0pt}%
\pgfpathmoveto{\pgfqpoint{1.873954in}{1.991381in}}%
\pgfpathcurveto{\pgfqpoint{1.882191in}{1.991381in}}{\pgfqpoint{1.890091in}{1.994653in}}{\pgfqpoint{1.895915in}{2.000477in}}%
\pgfpathcurveto{\pgfqpoint{1.901739in}{2.006301in}}{\pgfqpoint{1.905011in}{2.014201in}}{\pgfqpoint{1.905011in}{2.022437in}}%
\pgfpathcurveto{\pgfqpoint{1.905011in}{2.030673in}}{\pgfqpoint{1.901739in}{2.038573in}}{\pgfqpoint{1.895915in}{2.044397in}}%
\pgfpathcurveto{\pgfqpoint{1.890091in}{2.050221in}}{\pgfqpoint{1.882191in}{2.053494in}}{\pgfqpoint{1.873954in}{2.053494in}}%
\pgfpathcurveto{\pgfqpoint{1.865718in}{2.053494in}}{\pgfqpoint{1.857818in}{2.050221in}}{\pgfqpoint{1.851994in}{2.044397in}}%
\pgfpathcurveto{\pgfqpoint{1.846170in}{2.038573in}}{\pgfqpoint{1.842898in}{2.030673in}}{\pgfqpoint{1.842898in}{2.022437in}}%
\pgfpathcurveto{\pgfqpoint{1.842898in}{2.014201in}}{\pgfqpoint{1.846170in}{2.006301in}}{\pgfqpoint{1.851994in}{2.000477in}}%
\pgfpathcurveto{\pgfqpoint{1.857818in}{1.994653in}}{\pgfqpoint{1.865718in}{1.991381in}}{\pgfqpoint{1.873954in}{1.991381in}}%
\pgfpathclose%
\pgfusepath{stroke,fill}%
\end{pgfscope}%
\begin{pgfscope}%
\pgfpathrectangle{\pgfqpoint{0.100000in}{0.212622in}}{\pgfqpoint{3.696000in}{3.696000in}}%
\pgfusepath{clip}%
\pgfsetbuttcap%
\pgfsetroundjoin%
\definecolor{currentfill}{rgb}{0.121569,0.466667,0.705882}%
\pgfsetfillcolor{currentfill}%
\pgfsetfillopacity{0.323499}%
\pgfsetlinewidth{1.003750pt}%
\definecolor{currentstroke}{rgb}{0.121569,0.466667,0.705882}%
\pgfsetstrokecolor{currentstroke}%
\pgfsetstrokeopacity{0.323499}%
\pgfsetdash{}{0pt}%
\pgfpathmoveto{\pgfqpoint{1.965670in}{2.006526in}}%
\pgfpathcurveto{\pgfqpoint{1.973906in}{2.006526in}}{\pgfqpoint{1.981807in}{2.009798in}}{\pgfqpoint{1.987630in}{2.015622in}}%
\pgfpathcurveto{\pgfqpoint{1.993454in}{2.021446in}}{\pgfqpoint{1.996727in}{2.029346in}}{\pgfqpoint{1.996727in}{2.037582in}}%
\pgfpathcurveto{\pgfqpoint{1.996727in}{2.045819in}}{\pgfqpoint{1.993454in}{2.053719in}}{\pgfqpoint{1.987630in}{2.059543in}}%
\pgfpathcurveto{\pgfqpoint{1.981807in}{2.065367in}}{\pgfqpoint{1.973906in}{2.068639in}}{\pgfqpoint{1.965670in}{2.068639in}}%
\pgfpathcurveto{\pgfqpoint{1.957434in}{2.068639in}}{\pgfqpoint{1.949534in}{2.065367in}}{\pgfqpoint{1.943710in}{2.059543in}}%
\pgfpathcurveto{\pgfqpoint{1.937886in}{2.053719in}}{\pgfqpoint{1.934614in}{2.045819in}}{\pgfqpoint{1.934614in}{2.037582in}}%
\pgfpathcurveto{\pgfqpoint{1.934614in}{2.029346in}}{\pgfqpoint{1.937886in}{2.021446in}}{\pgfqpoint{1.943710in}{2.015622in}}%
\pgfpathcurveto{\pgfqpoint{1.949534in}{2.009798in}}{\pgfqpoint{1.957434in}{2.006526in}}{\pgfqpoint{1.965670in}{2.006526in}}%
\pgfpathclose%
\pgfusepath{stroke,fill}%
\end{pgfscope}%
\begin{pgfscope}%
\pgfpathrectangle{\pgfqpoint{0.100000in}{0.212622in}}{\pgfqpoint{3.696000in}{3.696000in}}%
\pgfusepath{clip}%
\pgfsetbuttcap%
\pgfsetroundjoin%
\definecolor{currentfill}{rgb}{0.121569,0.466667,0.705882}%
\pgfsetfillcolor{currentfill}%
\pgfsetfillopacity{0.323746}%
\pgfsetlinewidth{1.003750pt}%
\definecolor{currentstroke}{rgb}{0.121569,0.466667,0.705882}%
\pgfsetstrokecolor{currentstroke}%
\pgfsetstrokeopacity{0.323746}%
\pgfsetdash{}{0pt}%
\pgfpathmoveto{\pgfqpoint{1.869277in}{1.988344in}}%
\pgfpathcurveto{\pgfqpoint{1.877513in}{1.988344in}}{\pgfqpoint{1.885413in}{1.991616in}}{\pgfqpoint{1.891237in}{1.997440in}}%
\pgfpathcurveto{\pgfqpoint{1.897061in}{2.003264in}}{\pgfqpoint{1.900333in}{2.011164in}}{\pgfqpoint{1.900333in}{2.019401in}}%
\pgfpathcurveto{\pgfqpoint{1.900333in}{2.027637in}}{\pgfqpoint{1.897061in}{2.035537in}}{\pgfqpoint{1.891237in}{2.041361in}}%
\pgfpathcurveto{\pgfqpoint{1.885413in}{2.047185in}}{\pgfqpoint{1.877513in}{2.050457in}}{\pgfqpoint{1.869277in}{2.050457in}}%
\pgfpathcurveto{\pgfqpoint{1.861040in}{2.050457in}}{\pgfqpoint{1.853140in}{2.047185in}}{\pgfqpoint{1.847316in}{2.041361in}}%
\pgfpathcurveto{\pgfqpoint{1.841492in}{2.035537in}}{\pgfqpoint{1.838220in}{2.027637in}}{\pgfqpoint{1.838220in}{2.019401in}}%
\pgfpathcurveto{\pgfqpoint{1.838220in}{2.011164in}}{\pgfqpoint{1.841492in}{2.003264in}}{\pgfqpoint{1.847316in}{1.997440in}}%
\pgfpathcurveto{\pgfqpoint{1.853140in}{1.991616in}}{\pgfqpoint{1.861040in}{1.988344in}}{\pgfqpoint{1.869277in}{1.988344in}}%
\pgfpathclose%
\pgfusepath{stroke,fill}%
\end{pgfscope}%
\begin{pgfscope}%
\pgfpathrectangle{\pgfqpoint{0.100000in}{0.212622in}}{\pgfqpoint{3.696000in}{3.696000in}}%
\pgfusepath{clip}%
\pgfsetbuttcap%
\pgfsetroundjoin%
\definecolor{currentfill}{rgb}{0.121569,0.466667,0.705882}%
\pgfsetfillcolor{currentfill}%
\pgfsetfillopacity{0.326144}%
\pgfsetlinewidth{1.003750pt}%
\definecolor{currentstroke}{rgb}{0.121569,0.466667,0.705882}%
\pgfsetstrokecolor{currentstroke}%
\pgfsetstrokeopacity{0.326144}%
\pgfsetdash{}{0pt}%
\pgfpathmoveto{\pgfqpoint{1.967987in}{2.006309in}}%
\pgfpathcurveto{\pgfqpoint{1.976224in}{2.006309in}}{\pgfqpoint{1.984124in}{2.009581in}}{\pgfqpoint{1.989948in}{2.015405in}}%
\pgfpathcurveto{\pgfqpoint{1.995772in}{2.021229in}}{\pgfqpoint{1.999044in}{2.029129in}}{\pgfqpoint{1.999044in}{2.037365in}}%
\pgfpathcurveto{\pgfqpoint{1.999044in}{2.045601in}}{\pgfqpoint{1.995772in}{2.053502in}}{\pgfqpoint{1.989948in}{2.059325in}}%
\pgfpathcurveto{\pgfqpoint{1.984124in}{2.065149in}}{\pgfqpoint{1.976224in}{2.068422in}}{\pgfqpoint{1.967987in}{2.068422in}}%
\pgfpathcurveto{\pgfqpoint{1.959751in}{2.068422in}}{\pgfqpoint{1.951851in}{2.065149in}}{\pgfqpoint{1.946027in}{2.059325in}}%
\pgfpathcurveto{\pgfqpoint{1.940203in}{2.053502in}}{\pgfqpoint{1.936931in}{2.045601in}}{\pgfqpoint{1.936931in}{2.037365in}}%
\pgfpathcurveto{\pgfqpoint{1.936931in}{2.029129in}}{\pgfqpoint{1.940203in}{2.021229in}}{\pgfqpoint{1.946027in}{2.015405in}}%
\pgfpathcurveto{\pgfqpoint{1.951851in}{2.009581in}}{\pgfqpoint{1.959751in}{2.006309in}}{\pgfqpoint{1.967987in}{2.006309in}}%
\pgfpathclose%
\pgfusepath{stroke,fill}%
\end{pgfscope}%
\begin{pgfscope}%
\pgfpathrectangle{\pgfqpoint{0.100000in}{0.212622in}}{\pgfqpoint{3.696000in}{3.696000in}}%
\pgfusepath{clip}%
\pgfsetbuttcap%
\pgfsetroundjoin%
\definecolor{currentfill}{rgb}{0.121569,0.466667,0.705882}%
\pgfsetfillcolor{currentfill}%
\pgfsetfillopacity{0.326811}%
\pgfsetlinewidth{1.003750pt}%
\definecolor{currentstroke}{rgb}{0.121569,0.466667,0.705882}%
\pgfsetstrokecolor{currentstroke}%
\pgfsetstrokeopacity{0.326811}%
\pgfsetdash{}{0pt}%
\pgfpathmoveto{\pgfqpoint{1.862630in}{1.986506in}}%
\pgfpathcurveto{\pgfqpoint{1.870866in}{1.986506in}}{\pgfqpoint{1.878766in}{1.989778in}}{\pgfqpoint{1.884590in}{1.995602in}}%
\pgfpathcurveto{\pgfqpoint{1.890414in}{2.001426in}}{\pgfqpoint{1.893686in}{2.009326in}}{\pgfqpoint{1.893686in}{2.017563in}}%
\pgfpathcurveto{\pgfqpoint{1.893686in}{2.025799in}}{\pgfqpoint{1.890414in}{2.033699in}}{\pgfqpoint{1.884590in}{2.039523in}}%
\pgfpathcurveto{\pgfqpoint{1.878766in}{2.045347in}}{\pgfqpoint{1.870866in}{2.048619in}}{\pgfqpoint{1.862630in}{2.048619in}}%
\pgfpathcurveto{\pgfqpoint{1.854394in}{2.048619in}}{\pgfqpoint{1.846494in}{2.045347in}}{\pgfqpoint{1.840670in}{2.039523in}}%
\pgfpathcurveto{\pgfqpoint{1.834846in}{2.033699in}}{\pgfqpoint{1.831573in}{2.025799in}}{\pgfqpoint{1.831573in}{2.017563in}}%
\pgfpathcurveto{\pgfqpoint{1.831573in}{2.009326in}}{\pgfqpoint{1.834846in}{2.001426in}}{\pgfqpoint{1.840670in}{1.995602in}}%
\pgfpathcurveto{\pgfqpoint{1.846494in}{1.989778in}}{\pgfqpoint{1.854394in}{1.986506in}}{\pgfqpoint{1.862630in}{1.986506in}}%
\pgfpathclose%
\pgfusepath{stroke,fill}%
\end{pgfscope}%
\begin{pgfscope}%
\pgfpathrectangle{\pgfqpoint{0.100000in}{0.212622in}}{\pgfqpoint{3.696000in}{3.696000in}}%
\pgfusepath{clip}%
\pgfsetbuttcap%
\pgfsetroundjoin%
\definecolor{currentfill}{rgb}{0.121569,0.466667,0.705882}%
\pgfsetfillcolor{currentfill}%
\pgfsetfillopacity{0.328393}%
\pgfsetlinewidth{1.003750pt}%
\definecolor{currentstroke}{rgb}{0.121569,0.466667,0.705882}%
\pgfsetstrokecolor{currentstroke}%
\pgfsetstrokeopacity{0.328393}%
\pgfsetdash{}{0pt}%
\pgfpathmoveto{\pgfqpoint{1.855854in}{1.986565in}}%
\pgfpathcurveto{\pgfqpoint{1.864090in}{1.986565in}}{\pgfqpoint{1.871991in}{1.989837in}}{\pgfqpoint{1.877814in}{1.995661in}}%
\pgfpathcurveto{\pgfqpoint{1.883638in}{2.001485in}}{\pgfqpoint{1.886911in}{2.009385in}}{\pgfqpoint{1.886911in}{2.017621in}}%
\pgfpathcurveto{\pgfqpoint{1.886911in}{2.025858in}}{\pgfqpoint{1.883638in}{2.033758in}}{\pgfqpoint{1.877814in}{2.039582in}}%
\pgfpathcurveto{\pgfqpoint{1.871991in}{2.045406in}}{\pgfqpoint{1.864090in}{2.048678in}}{\pgfqpoint{1.855854in}{2.048678in}}%
\pgfpathcurveto{\pgfqpoint{1.847618in}{2.048678in}}{\pgfqpoint{1.839718in}{2.045406in}}{\pgfqpoint{1.833894in}{2.039582in}}%
\pgfpathcurveto{\pgfqpoint{1.828070in}{2.033758in}}{\pgfqpoint{1.824798in}{2.025858in}}{\pgfqpoint{1.824798in}{2.017621in}}%
\pgfpathcurveto{\pgfqpoint{1.824798in}{2.009385in}}{\pgfqpoint{1.828070in}{2.001485in}}{\pgfqpoint{1.833894in}{1.995661in}}%
\pgfpathcurveto{\pgfqpoint{1.839718in}{1.989837in}}{\pgfqpoint{1.847618in}{1.986565in}}{\pgfqpoint{1.855854in}{1.986565in}}%
\pgfpathclose%
\pgfusepath{stroke,fill}%
\end{pgfscope}%
\begin{pgfscope}%
\pgfpathrectangle{\pgfqpoint{0.100000in}{0.212622in}}{\pgfqpoint{3.696000in}{3.696000in}}%
\pgfusepath{clip}%
\pgfsetbuttcap%
\pgfsetroundjoin%
\definecolor{currentfill}{rgb}{0.121569,0.466667,0.705882}%
\pgfsetfillcolor{currentfill}%
\pgfsetfillopacity{0.329185}%
\pgfsetlinewidth{1.003750pt}%
\definecolor{currentstroke}{rgb}{0.121569,0.466667,0.705882}%
\pgfsetstrokecolor{currentstroke}%
\pgfsetstrokeopacity{0.329185}%
\pgfsetdash{}{0pt}%
\pgfpathmoveto{\pgfqpoint{1.852696in}{1.983961in}}%
\pgfpathcurveto{\pgfqpoint{1.860932in}{1.983961in}}{\pgfqpoint{1.868832in}{1.987234in}}{\pgfqpoint{1.874656in}{1.993058in}}%
\pgfpathcurveto{\pgfqpoint{1.880480in}{1.998881in}}{\pgfqpoint{1.883753in}{2.006781in}}{\pgfqpoint{1.883753in}{2.015018in}}%
\pgfpathcurveto{\pgfqpoint{1.883753in}{2.023254in}}{\pgfqpoint{1.880480in}{2.031154in}}{\pgfqpoint{1.874656in}{2.036978in}}%
\pgfpathcurveto{\pgfqpoint{1.868832in}{2.042802in}}{\pgfqpoint{1.860932in}{2.046074in}}{\pgfqpoint{1.852696in}{2.046074in}}%
\pgfpathcurveto{\pgfqpoint{1.844460in}{2.046074in}}{\pgfqpoint{1.836560in}{2.042802in}}{\pgfqpoint{1.830736in}{2.036978in}}%
\pgfpathcurveto{\pgfqpoint{1.824912in}{2.031154in}}{\pgfqpoint{1.821640in}{2.023254in}}{\pgfqpoint{1.821640in}{2.015018in}}%
\pgfpathcurveto{\pgfqpoint{1.821640in}{2.006781in}}{\pgfqpoint{1.824912in}{1.998881in}}{\pgfqpoint{1.830736in}{1.993058in}}%
\pgfpathcurveto{\pgfqpoint{1.836560in}{1.987234in}}{\pgfqpoint{1.844460in}{1.983961in}}{\pgfqpoint{1.852696in}{1.983961in}}%
\pgfpathclose%
\pgfusepath{stroke,fill}%
\end{pgfscope}%
\begin{pgfscope}%
\pgfpathrectangle{\pgfqpoint{0.100000in}{0.212622in}}{\pgfqpoint{3.696000in}{3.696000in}}%
\pgfusepath{clip}%
\pgfsetbuttcap%
\pgfsetroundjoin%
\definecolor{currentfill}{rgb}{0.121569,0.466667,0.705882}%
\pgfsetfillcolor{currentfill}%
\pgfsetfillopacity{0.329238}%
\pgfsetlinewidth{1.003750pt}%
\definecolor{currentstroke}{rgb}{0.121569,0.466667,0.705882}%
\pgfsetstrokecolor{currentstroke}%
\pgfsetstrokeopacity{0.329238}%
\pgfsetdash{}{0pt}%
\pgfpathmoveto{\pgfqpoint{1.970741in}{2.007798in}}%
\pgfpathcurveto{\pgfqpoint{1.978977in}{2.007798in}}{\pgfqpoint{1.986877in}{2.011070in}}{\pgfqpoint{1.992701in}{2.016894in}}%
\pgfpathcurveto{\pgfqpoint{1.998525in}{2.022718in}}{\pgfqpoint{2.001797in}{2.030618in}}{\pgfqpoint{2.001797in}{2.038854in}}%
\pgfpathcurveto{\pgfqpoint{2.001797in}{2.047091in}}{\pgfqpoint{1.998525in}{2.054991in}}{\pgfqpoint{1.992701in}{2.060815in}}%
\pgfpathcurveto{\pgfqpoint{1.986877in}{2.066639in}}{\pgfqpoint{1.978977in}{2.069911in}}{\pgfqpoint{1.970741in}{2.069911in}}%
\pgfpathcurveto{\pgfqpoint{1.962505in}{2.069911in}}{\pgfqpoint{1.954605in}{2.066639in}}{\pgfqpoint{1.948781in}{2.060815in}}%
\pgfpathcurveto{\pgfqpoint{1.942957in}{2.054991in}}{\pgfqpoint{1.939684in}{2.047091in}}{\pgfqpoint{1.939684in}{2.038854in}}%
\pgfpathcurveto{\pgfqpoint{1.939684in}{2.030618in}}{\pgfqpoint{1.942957in}{2.022718in}}{\pgfqpoint{1.948781in}{2.016894in}}%
\pgfpathcurveto{\pgfqpoint{1.954605in}{2.011070in}}{\pgfqpoint{1.962505in}{2.007798in}}{\pgfqpoint{1.970741in}{2.007798in}}%
\pgfpathclose%
\pgfusepath{stroke,fill}%
\end{pgfscope}%
\begin{pgfscope}%
\pgfpathrectangle{\pgfqpoint{0.100000in}{0.212622in}}{\pgfqpoint{3.696000in}{3.696000in}}%
\pgfusepath{clip}%
\pgfsetbuttcap%
\pgfsetroundjoin%
\definecolor{currentfill}{rgb}{0.121569,0.466667,0.705882}%
\pgfsetfillcolor{currentfill}%
\pgfsetfillopacity{0.330738}%
\pgfsetlinewidth{1.003750pt}%
\definecolor{currentstroke}{rgb}{0.121569,0.466667,0.705882}%
\pgfsetstrokecolor{currentstroke}%
\pgfsetstrokeopacity{0.330738}%
\pgfsetdash{}{0pt}%
\pgfpathmoveto{\pgfqpoint{1.971625in}{2.006811in}}%
\pgfpathcurveto{\pgfqpoint{1.979861in}{2.006811in}}{\pgfqpoint{1.987761in}{2.010083in}}{\pgfqpoint{1.993585in}{2.015907in}}%
\pgfpathcurveto{\pgfqpoint{1.999409in}{2.021731in}}{\pgfqpoint{2.002681in}{2.029631in}}{\pgfqpoint{2.002681in}{2.037867in}}%
\pgfpathcurveto{\pgfqpoint{2.002681in}{2.046103in}}{\pgfqpoint{1.999409in}{2.054004in}}{\pgfqpoint{1.993585in}{2.059827in}}%
\pgfpathcurveto{\pgfqpoint{1.987761in}{2.065651in}}{\pgfqpoint{1.979861in}{2.068924in}}{\pgfqpoint{1.971625in}{2.068924in}}%
\pgfpathcurveto{\pgfqpoint{1.963388in}{2.068924in}}{\pgfqpoint{1.955488in}{2.065651in}}{\pgfqpoint{1.949664in}{2.059827in}}%
\pgfpathcurveto{\pgfqpoint{1.943840in}{2.054004in}}{\pgfqpoint{1.940568in}{2.046103in}}{\pgfqpoint{1.940568in}{2.037867in}}%
\pgfpathcurveto{\pgfqpoint{1.940568in}{2.029631in}}{\pgfqpoint{1.943840in}{2.021731in}}{\pgfqpoint{1.949664in}{2.015907in}}%
\pgfpathcurveto{\pgfqpoint{1.955488in}{2.010083in}}{\pgfqpoint{1.963388in}{2.006811in}}{\pgfqpoint{1.971625in}{2.006811in}}%
\pgfpathclose%
\pgfusepath{stroke,fill}%
\end{pgfscope}%
\begin{pgfscope}%
\pgfpathrectangle{\pgfqpoint{0.100000in}{0.212622in}}{\pgfqpoint{3.696000in}{3.696000in}}%
\pgfusepath{clip}%
\pgfsetbuttcap%
\pgfsetroundjoin%
\definecolor{currentfill}{rgb}{0.121569,0.466667,0.705882}%
\pgfsetfillcolor{currentfill}%
\pgfsetfillopacity{0.331172}%
\pgfsetlinewidth{1.003750pt}%
\definecolor{currentstroke}{rgb}{0.121569,0.466667,0.705882}%
\pgfsetstrokecolor{currentstroke}%
\pgfsetstrokeopacity{0.331172}%
\pgfsetdash{}{0pt}%
\pgfpathmoveto{\pgfqpoint{1.847604in}{1.982036in}}%
\pgfpathcurveto{\pgfqpoint{1.855840in}{1.982036in}}{\pgfqpoint{1.863740in}{1.985308in}}{\pgfqpoint{1.869564in}{1.991132in}}%
\pgfpathcurveto{\pgfqpoint{1.875388in}{1.996956in}}{\pgfqpoint{1.878660in}{2.004856in}}{\pgfqpoint{1.878660in}{2.013092in}}%
\pgfpathcurveto{\pgfqpoint{1.878660in}{2.021329in}}{\pgfqpoint{1.875388in}{2.029229in}}{\pgfqpoint{1.869564in}{2.035053in}}%
\pgfpathcurveto{\pgfqpoint{1.863740in}{2.040876in}}{\pgfqpoint{1.855840in}{2.044149in}}{\pgfqpoint{1.847604in}{2.044149in}}%
\pgfpathcurveto{\pgfqpoint{1.839367in}{2.044149in}}{\pgfqpoint{1.831467in}{2.040876in}}{\pgfqpoint{1.825643in}{2.035053in}}%
\pgfpathcurveto{\pgfqpoint{1.819819in}{2.029229in}}{\pgfqpoint{1.816547in}{2.021329in}}{\pgfqpoint{1.816547in}{2.013092in}}%
\pgfpathcurveto{\pgfqpoint{1.816547in}{2.004856in}}{\pgfqpoint{1.819819in}{1.996956in}}{\pgfqpoint{1.825643in}{1.991132in}}%
\pgfpathcurveto{\pgfqpoint{1.831467in}{1.985308in}}{\pgfqpoint{1.839367in}{1.982036in}}{\pgfqpoint{1.847604in}{1.982036in}}%
\pgfpathclose%
\pgfusepath{stroke,fill}%
\end{pgfscope}%
\begin{pgfscope}%
\pgfpathrectangle{\pgfqpoint{0.100000in}{0.212622in}}{\pgfqpoint{3.696000in}{3.696000in}}%
\pgfusepath{clip}%
\pgfsetbuttcap%
\pgfsetroundjoin%
\definecolor{currentfill}{rgb}{0.121569,0.466667,0.705882}%
\pgfsetfillcolor{currentfill}%
\pgfsetfillopacity{0.332147}%
\pgfsetlinewidth{1.003750pt}%
\definecolor{currentstroke}{rgb}{0.121569,0.466667,0.705882}%
\pgfsetstrokecolor{currentstroke}%
\pgfsetstrokeopacity{0.332147}%
\pgfsetdash{}{0pt}%
\pgfpathmoveto{\pgfqpoint{1.843704in}{1.981265in}}%
\pgfpathcurveto{\pgfqpoint{1.851940in}{1.981265in}}{\pgfqpoint{1.859840in}{1.984537in}}{\pgfqpoint{1.865664in}{1.990361in}}%
\pgfpathcurveto{\pgfqpoint{1.871488in}{1.996185in}}{\pgfqpoint{1.874760in}{2.004085in}}{\pgfqpoint{1.874760in}{2.012321in}}%
\pgfpathcurveto{\pgfqpoint{1.874760in}{2.020557in}}{\pgfqpoint{1.871488in}{2.028457in}}{\pgfqpoint{1.865664in}{2.034281in}}%
\pgfpathcurveto{\pgfqpoint{1.859840in}{2.040105in}}{\pgfqpoint{1.851940in}{2.043378in}}{\pgfqpoint{1.843704in}{2.043378in}}%
\pgfpathcurveto{\pgfqpoint{1.835467in}{2.043378in}}{\pgfqpoint{1.827567in}{2.040105in}}{\pgfqpoint{1.821743in}{2.034281in}}%
\pgfpathcurveto{\pgfqpoint{1.815920in}{2.028457in}}{\pgfqpoint{1.812647in}{2.020557in}}{\pgfqpoint{1.812647in}{2.012321in}}%
\pgfpathcurveto{\pgfqpoint{1.812647in}{2.004085in}}{\pgfqpoint{1.815920in}{1.996185in}}{\pgfqpoint{1.821743in}{1.990361in}}%
\pgfpathcurveto{\pgfqpoint{1.827567in}{1.984537in}}{\pgfqpoint{1.835467in}{1.981265in}}{\pgfqpoint{1.843704in}{1.981265in}}%
\pgfpathclose%
\pgfusepath{stroke,fill}%
\end{pgfscope}%
\begin{pgfscope}%
\pgfpathrectangle{\pgfqpoint{0.100000in}{0.212622in}}{\pgfqpoint{3.696000in}{3.696000in}}%
\pgfusepath{clip}%
\pgfsetbuttcap%
\pgfsetroundjoin%
\definecolor{currentfill}{rgb}{0.121569,0.466667,0.705882}%
\pgfsetfillcolor{currentfill}%
\pgfsetfillopacity{0.332633}%
\pgfsetlinewidth{1.003750pt}%
\definecolor{currentstroke}{rgb}{0.121569,0.466667,0.705882}%
\pgfsetstrokecolor{currentstroke}%
\pgfsetstrokeopacity{0.332633}%
\pgfsetdash{}{0pt}%
\pgfpathmoveto{\pgfqpoint{1.973457in}{2.006164in}}%
\pgfpathcurveto{\pgfqpoint{1.981693in}{2.006164in}}{\pgfqpoint{1.989593in}{2.009437in}}{\pgfqpoint{1.995417in}{2.015261in}}%
\pgfpathcurveto{\pgfqpoint{2.001241in}{2.021084in}}{\pgfqpoint{2.004513in}{2.028985in}}{\pgfqpoint{2.004513in}{2.037221in}}%
\pgfpathcurveto{\pgfqpoint{2.004513in}{2.045457in}}{\pgfqpoint{2.001241in}{2.053357in}}{\pgfqpoint{1.995417in}{2.059181in}}%
\pgfpathcurveto{\pgfqpoint{1.989593in}{2.065005in}}{\pgfqpoint{1.981693in}{2.068277in}}{\pgfqpoint{1.973457in}{2.068277in}}%
\pgfpathcurveto{\pgfqpoint{1.965220in}{2.068277in}}{\pgfqpoint{1.957320in}{2.065005in}}{\pgfqpoint{1.951496in}{2.059181in}}%
\pgfpathcurveto{\pgfqpoint{1.945672in}{2.053357in}}{\pgfqpoint{1.942400in}{2.045457in}}{\pgfqpoint{1.942400in}{2.037221in}}%
\pgfpathcurveto{\pgfqpoint{1.942400in}{2.028985in}}{\pgfqpoint{1.945672in}{2.021084in}}{\pgfqpoint{1.951496in}{2.015261in}}%
\pgfpathcurveto{\pgfqpoint{1.957320in}{2.009437in}}{\pgfqpoint{1.965220in}{2.006164in}}{\pgfqpoint{1.973457in}{2.006164in}}%
\pgfpathclose%
\pgfusepath{stroke,fill}%
\end{pgfscope}%
\begin{pgfscope}%
\pgfpathrectangle{\pgfqpoint{0.100000in}{0.212622in}}{\pgfqpoint{3.696000in}{3.696000in}}%
\pgfusepath{clip}%
\pgfsetbuttcap%
\pgfsetroundjoin%
\definecolor{currentfill}{rgb}{0.121569,0.466667,0.705882}%
\pgfsetfillcolor{currentfill}%
\pgfsetfillopacity{0.332899}%
\pgfsetlinewidth{1.003750pt}%
\definecolor{currentstroke}{rgb}{0.121569,0.466667,0.705882}%
\pgfsetstrokecolor{currentstroke}%
\pgfsetstrokeopacity{0.332899}%
\pgfsetdash{}{0pt}%
\pgfpathmoveto{\pgfqpoint{1.840946in}{1.979095in}}%
\pgfpathcurveto{\pgfqpoint{1.849182in}{1.979095in}}{\pgfqpoint{1.857082in}{1.982367in}}{\pgfqpoint{1.862906in}{1.988191in}}%
\pgfpathcurveto{\pgfqpoint{1.868730in}{1.994015in}}{\pgfqpoint{1.872002in}{2.001915in}}{\pgfqpoint{1.872002in}{2.010152in}}%
\pgfpathcurveto{\pgfqpoint{1.872002in}{2.018388in}}{\pgfqpoint{1.868730in}{2.026288in}}{\pgfqpoint{1.862906in}{2.032112in}}%
\pgfpathcurveto{\pgfqpoint{1.857082in}{2.037936in}}{\pgfqpoint{1.849182in}{2.041208in}}{\pgfqpoint{1.840946in}{2.041208in}}%
\pgfpathcurveto{\pgfqpoint{1.832710in}{2.041208in}}{\pgfqpoint{1.824810in}{2.037936in}}{\pgfqpoint{1.818986in}{2.032112in}}%
\pgfpathcurveto{\pgfqpoint{1.813162in}{2.026288in}}{\pgfqpoint{1.809889in}{2.018388in}}{\pgfqpoint{1.809889in}{2.010152in}}%
\pgfpathcurveto{\pgfqpoint{1.809889in}{2.001915in}}{\pgfqpoint{1.813162in}{1.994015in}}{\pgfqpoint{1.818986in}{1.988191in}}%
\pgfpathcurveto{\pgfqpoint{1.824810in}{1.982367in}}{\pgfqpoint{1.832710in}{1.979095in}}{\pgfqpoint{1.840946in}{1.979095in}}%
\pgfpathclose%
\pgfusepath{stroke,fill}%
\end{pgfscope}%
\begin{pgfscope}%
\pgfpathrectangle{\pgfqpoint{0.100000in}{0.212622in}}{\pgfqpoint{3.696000in}{3.696000in}}%
\pgfusepath{clip}%
\pgfsetbuttcap%
\pgfsetroundjoin%
\definecolor{currentfill}{rgb}{0.121569,0.466667,0.705882}%
\pgfsetfillcolor{currentfill}%
\pgfsetfillopacity{0.334909}%
\pgfsetlinewidth{1.003750pt}%
\definecolor{currentstroke}{rgb}{0.121569,0.466667,0.705882}%
\pgfsetstrokecolor{currentstroke}%
\pgfsetstrokeopacity{0.334909}%
\pgfsetdash{}{0pt}%
\pgfpathmoveto{\pgfqpoint{1.836431in}{1.978880in}}%
\pgfpathcurveto{\pgfqpoint{1.844667in}{1.978880in}}{\pgfqpoint{1.852567in}{1.982152in}}{\pgfqpoint{1.858391in}{1.987976in}}%
\pgfpathcurveto{\pgfqpoint{1.864215in}{1.993800in}}{\pgfqpoint{1.867487in}{2.001700in}}{\pgfqpoint{1.867487in}{2.009936in}}%
\pgfpathcurveto{\pgfqpoint{1.867487in}{2.018173in}}{\pgfqpoint{1.864215in}{2.026073in}}{\pgfqpoint{1.858391in}{2.031897in}}%
\pgfpathcurveto{\pgfqpoint{1.852567in}{2.037721in}}{\pgfqpoint{1.844667in}{2.040993in}}{\pgfqpoint{1.836431in}{2.040993in}}%
\pgfpathcurveto{\pgfqpoint{1.828194in}{2.040993in}}{\pgfqpoint{1.820294in}{2.037721in}}{\pgfqpoint{1.814470in}{2.031897in}}%
\pgfpathcurveto{\pgfqpoint{1.808646in}{2.026073in}}{\pgfqpoint{1.805374in}{2.018173in}}{\pgfqpoint{1.805374in}{2.009936in}}%
\pgfpathcurveto{\pgfqpoint{1.805374in}{2.001700in}}{\pgfqpoint{1.808646in}{1.993800in}}{\pgfqpoint{1.814470in}{1.987976in}}%
\pgfpathcurveto{\pgfqpoint{1.820294in}{1.982152in}}{\pgfqpoint{1.828194in}{1.978880in}}{\pgfqpoint{1.836431in}{1.978880in}}%
\pgfpathclose%
\pgfusepath{stroke,fill}%
\end{pgfscope}%
\begin{pgfscope}%
\pgfpathrectangle{\pgfqpoint{0.100000in}{0.212622in}}{\pgfqpoint{3.696000in}{3.696000in}}%
\pgfusepath{clip}%
\pgfsetbuttcap%
\pgfsetroundjoin%
\definecolor{currentfill}{rgb}{0.121569,0.466667,0.705882}%
\pgfsetfillcolor{currentfill}%
\pgfsetfillopacity{0.335299}%
\pgfsetlinewidth{1.003750pt}%
\definecolor{currentstroke}{rgb}{0.121569,0.466667,0.705882}%
\pgfsetstrokecolor{currentstroke}%
\pgfsetstrokeopacity{0.335299}%
\pgfsetdash{}{0pt}%
\pgfpathmoveto{\pgfqpoint{1.976093in}{2.007844in}}%
\pgfpathcurveto{\pgfqpoint{1.984330in}{2.007844in}}{\pgfqpoint{1.992230in}{2.011116in}}{\pgfqpoint{1.998054in}{2.016940in}}%
\pgfpathcurveto{\pgfqpoint{2.003878in}{2.022764in}}{\pgfqpoint{2.007150in}{2.030664in}}{\pgfqpoint{2.007150in}{2.038900in}}%
\pgfpathcurveto{\pgfqpoint{2.007150in}{2.047136in}}{\pgfqpoint{2.003878in}{2.055036in}}{\pgfqpoint{1.998054in}{2.060860in}}%
\pgfpathcurveto{\pgfqpoint{1.992230in}{2.066684in}}{\pgfqpoint{1.984330in}{2.069957in}}{\pgfqpoint{1.976093in}{2.069957in}}%
\pgfpathcurveto{\pgfqpoint{1.967857in}{2.069957in}}{\pgfqpoint{1.959957in}{2.066684in}}{\pgfqpoint{1.954133in}{2.060860in}}%
\pgfpathcurveto{\pgfqpoint{1.948309in}{2.055036in}}{\pgfqpoint{1.945037in}{2.047136in}}{\pgfqpoint{1.945037in}{2.038900in}}%
\pgfpathcurveto{\pgfqpoint{1.945037in}{2.030664in}}{\pgfqpoint{1.948309in}{2.022764in}}{\pgfqpoint{1.954133in}{2.016940in}}%
\pgfpathcurveto{\pgfqpoint{1.959957in}{2.011116in}}{\pgfqpoint{1.967857in}{2.007844in}}{\pgfqpoint{1.976093in}{2.007844in}}%
\pgfpathclose%
\pgfusepath{stroke,fill}%
\end{pgfscope}%
\begin{pgfscope}%
\pgfpathrectangle{\pgfqpoint{0.100000in}{0.212622in}}{\pgfqpoint{3.696000in}{3.696000in}}%
\pgfusepath{clip}%
\pgfsetbuttcap%
\pgfsetroundjoin%
\definecolor{currentfill}{rgb}{0.121569,0.466667,0.705882}%
\pgfsetfillcolor{currentfill}%
\pgfsetfillopacity{0.335752}%
\pgfsetlinewidth{1.003750pt}%
\definecolor{currentstroke}{rgb}{0.121569,0.466667,0.705882}%
\pgfsetstrokecolor{currentstroke}%
\pgfsetstrokeopacity{0.335752}%
\pgfsetdash{}{0pt}%
\pgfpathmoveto{\pgfqpoint{1.832781in}{1.975937in}}%
\pgfpathcurveto{\pgfqpoint{1.841017in}{1.975937in}}{\pgfqpoint{1.848917in}{1.979209in}}{\pgfqpoint{1.854741in}{1.985033in}}%
\pgfpathcurveto{\pgfqpoint{1.860565in}{1.990857in}}{\pgfqpoint{1.863837in}{1.998757in}}{\pgfqpoint{1.863837in}{2.006993in}}%
\pgfpathcurveto{\pgfqpoint{1.863837in}{2.015229in}}{\pgfqpoint{1.860565in}{2.023129in}}{\pgfqpoint{1.854741in}{2.028953in}}%
\pgfpathcurveto{\pgfqpoint{1.848917in}{2.034777in}}{\pgfqpoint{1.841017in}{2.038050in}}{\pgfqpoint{1.832781in}{2.038050in}}%
\pgfpathcurveto{\pgfqpoint{1.824544in}{2.038050in}}{\pgfqpoint{1.816644in}{2.034777in}}{\pgfqpoint{1.810820in}{2.028953in}}%
\pgfpathcurveto{\pgfqpoint{1.804996in}{2.023129in}}{\pgfqpoint{1.801724in}{2.015229in}}{\pgfqpoint{1.801724in}{2.006993in}}%
\pgfpathcurveto{\pgfqpoint{1.801724in}{1.998757in}}{\pgfqpoint{1.804996in}{1.990857in}}{\pgfqpoint{1.810820in}{1.985033in}}%
\pgfpathcurveto{\pgfqpoint{1.816644in}{1.979209in}}{\pgfqpoint{1.824544in}{1.975937in}}{\pgfqpoint{1.832781in}{1.975937in}}%
\pgfpathclose%
\pgfusepath{stroke,fill}%
\end{pgfscope}%
\begin{pgfscope}%
\pgfpathrectangle{\pgfqpoint{0.100000in}{0.212622in}}{\pgfqpoint{3.696000in}{3.696000in}}%
\pgfusepath{clip}%
\pgfsetbuttcap%
\pgfsetroundjoin%
\definecolor{currentfill}{rgb}{0.121569,0.466667,0.705882}%
\pgfsetfillcolor{currentfill}%
\pgfsetfillopacity{0.336583}%
\pgfsetlinewidth{1.003750pt}%
\definecolor{currentstroke}{rgb}{0.121569,0.466667,0.705882}%
\pgfsetstrokecolor{currentstroke}%
\pgfsetstrokeopacity{0.336583}%
\pgfsetdash{}{0pt}%
\pgfpathmoveto{\pgfqpoint{1.829734in}{1.974674in}}%
\pgfpathcurveto{\pgfqpoint{1.837971in}{1.974674in}}{\pgfqpoint{1.845871in}{1.977946in}}{\pgfqpoint{1.851695in}{1.983770in}}%
\pgfpathcurveto{\pgfqpoint{1.857519in}{1.989594in}}{\pgfqpoint{1.860791in}{1.997494in}}{\pgfqpoint{1.860791in}{2.005730in}}%
\pgfpathcurveto{\pgfqpoint{1.860791in}{2.013967in}}{\pgfqpoint{1.857519in}{2.021867in}}{\pgfqpoint{1.851695in}{2.027691in}}%
\pgfpathcurveto{\pgfqpoint{1.845871in}{2.033514in}}{\pgfqpoint{1.837971in}{2.036787in}}{\pgfqpoint{1.829734in}{2.036787in}}%
\pgfpathcurveto{\pgfqpoint{1.821498in}{2.036787in}}{\pgfqpoint{1.813598in}{2.033514in}}{\pgfqpoint{1.807774in}{2.027691in}}%
\pgfpathcurveto{\pgfqpoint{1.801950in}{2.021867in}}{\pgfqpoint{1.798678in}{2.013967in}}{\pgfqpoint{1.798678in}{2.005730in}}%
\pgfpathcurveto{\pgfqpoint{1.798678in}{1.997494in}}{\pgfqpoint{1.801950in}{1.989594in}}{\pgfqpoint{1.807774in}{1.983770in}}%
\pgfpathcurveto{\pgfqpoint{1.813598in}{1.977946in}}{\pgfqpoint{1.821498in}{1.974674in}}{\pgfqpoint{1.829734in}{1.974674in}}%
\pgfpathclose%
\pgfusepath{stroke,fill}%
\end{pgfscope}%
\begin{pgfscope}%
\pgfpathrectangle{\pgfqpoint{0.100000in}{0.212622in}}{\pgfqpoint{3.696000in}{3.696000in}}%
\pgfusepath{clip}%
\pgfsetbuttcap%
\pgfsetroundjoin%
\definecolor{currentfill}{rgb}{0.121569,0.466667,0.705882}%
\pgfsetfillcolor{currentfill}%
\pgfsetfillopacity{0.338336}%
\pgfsetlinewidth{1.003750pt}%
\definecolor{currentstroke}{rgb}{0.121569,0.466667,0.705882}%
\pgfsetstrokecolor{currentstroke}%
\pgfsetstrokeopacity{0.338336}%
\pgfsetdash{}{0pt}%
\pgfpathmoveto{\pgfqpoint{1.977742in}{2.005781in}}%
\pgfpathcurveto{\pgfqpoint{1.985978in}{2.005781in}}{\pgfqpoint{1.993878in}{2.009053in}}{\pgfqpoint{1.999702in}{2.014877in}}%
\pgfpathcurveto{\pgfqpoint{2.005526in}{2.020701in}}{\pgfqpoint{2.008798in}{2.028601in}}{\pgfqpoint{2.008798in}{2.036837in}}%
\pgfpathcurveto{\pgfqpoint{2.008798in}{2.045073in}}{\pgfqpoint{2.005526in}{2.052973in}}{\pgfqpoint{1.999702in}{2.058797in}}%
\pgfpathcurveto{\pgfqpoint{1.993878in}{2.064621in}}{\pgfqpoint{1.985978in}{2.067894in}}{\pgfqpoint{1.977742in}{2.067894in}}%
\pgfpathcurveto{\pgfqpoint{1.969506in}{2.067894in}}{\pgfqpoint{1.961606in}{2.064621in}}{\pgfqpoint{1.955782in}{2.058797in}}%
\pgfpathcurveto{\pgfqpoint{1.949958in}{2.052973in}}{\pgfqpoint{1.946685in}{2.045073in}}{\pgfqpoint{1.946685in}{2.036837in}}%
\pgfpathcurveto{\pgfqpoint{1.946685in}{2.028601in}}{\pgfqpoint{1.949958in}{2.020701in}}{\pgfqpoint{1.955782in}{2.014877in}}%
\pgfpathcurveto{\pgfqpoint{1.961606in}{2.009053in}}{\pgfqpoint{1.969506in}{2.005781in}}{\pgfqpoint{1.977742in}{2.005781in}}%
\pgfpathclose%
\pgfusepath{stroke,fill}%
\end{pgfscope}%
\begin{pgfscope}%
\pgfpathrectangle{\pgfqpoint{0.100000in}{0.212622in}}{\pgfqpoint{3.696000in}{3.696000in}}%
\pgfusepath{clip}%
\pgfsetbuttcap%
\pgfsetroundjoin%
\definecolor{currentfill}{rgb}{0.121569,0.466667,0.705882}%
\pgfsetfillcolor{currentfill}%
\pgfsetfillopacity{0.338656}%
\pgfsetlinewidth{1.003750pt}%
\definecolor{currentstroke}{rgb}{0.121569,0.466667,0.705882}%
\pgfsetstrokecolor{currentstroke}%
\pgfsetstrokeopacity{0.338656}%
\pgfsetdash{}{0pt}%
\pgfpathmoveto{\pgfqpoint{1.825639in}{1.974189in}}%
\pgfpathcurveto{\pgfqpoint{1.833876in}{1.974189in}}{\pgfqpoint{1.841776in}{1.977461in}}{\pgfqpoint{1.847600in}{1.983285in}}%
\pgfpathcurveto{\pgfqpoint{1.853423in}{1.989109in}}{\pgfqpoint{1.856696in}{1.997009in}}{\pgfqpoint{1.856696in}{2.005245in}}%
\pgfpathcurveto{\pgfqpoint{1.856696in}{2.013482in}}{\pgfqpoint{1.853423in}{2.021382in}}{\pgfqpoint{1.847600in}{2.027206in}}%
\pgfpathcurveto{\pgfqpoint{1.841776in}{2.033030in}}{\pgfqpoint{1.833876in}{2.036302in}}{\pgfqpoint{1.825639in}{2.036302in}}%
\pgfpathcurveto{\pgfqpoint{1.817403in}{2.036302in}}{\pgfqpoint{1.809503in}{2.033030in}}{\pgfqpoint{1.803679in}{2.027206in}}%
\pgfpathcurveto{\pgfqpoint{1.797855in}{2.021382in}}{\pgfqpoint{1.794583in}{2.013482in}}{\pgfqpoint{1.794583in}{2.005245in}}%
\pgfpathcurveto{\pgfqpoint{1.794583in}{1.997009in}}{\pgfqpoint{1.797855in}{1.989109in}}{\pgfqpoint{1.803679in}{1.983285in}}%
\pgfpathcurveto{\pgfqpoint{1.809503in}{1.977461in}}{\pgfqpoint{1.817403in}{1.974189in}}{\pgfqpoint{1.825639in}{1.974189in}}%
\pgfpathclose%
\pgfusepath{stroke,fill}%
\end{pgfscope}%
\begin{pgfscope}%
\pgfpathrectangle{\pgfqpoint{0.100000in}{0.212622in}}{\pgfqpoint{3.696000in}{3.696000in}}%
\pgfusepath{clip}%
\pgfsetbuttcap%
\pgfsetroundjoin%
\definecolor{currentfill}{rgb}{0.121569,0.466667,0.705882}%
\pgfsetfillcolor{currentfill}%
\pgfsetfillopacity{0.339524}%
\pgfsetlinewidth{1.003750pt}%
\definecolor{currentstroke}{rgb}{0.121569,0.466667,0.705882}%
\pgfsetstrokecolor{currentstroke}%
\pgfsetstrokeopacity{0.339524}%
\pgfsetdash{}{0pt}%
\pgfpathmoveto{\pgfqpoint{1.821664in}{1.971717in}}%
\pgfpathcurveto{\pgfqpoint{1.829900in}{1.971717in}}{\pgfqpoint{1.837800in}{1.974990in}}{\pgfqpoint{1.843624in}{1.980814in}}%
\pgfpathcurveto{\pgfqpoint{1.849448in}{1.986637in}}{\pgfqpoint{1.852720in}{1.994537in}}{\pgfqpoint{1.852720in}{2.002774in}}%
\pgfpathcurveto{\pgfqpoint{1.852720in}{2.011010in}}{\pgfqpoint{1.849448in}{2.018910in}}{\pgfqpoint{1.843624in}{2.024734in}}%
\pgfpathcurveto{\pgfqpoint{1.837800in}{2.030558in}}{\pgfqpoint{1.829900in}{2.033830in}}{\pgfqpoint{1.821664in}{2.033830in}}%
\pgfpathcurveto{\pgfqpoint{1.813428in}{2.033830in}}{\pgfqpoint{1.805528in}{2.030558in}}{\pgfqpoint{1.799704in}{2.024734in}}%
\pgfpathcurveto{\pgfqpoint{1.793880in}{2.018910in}}{\pgfqpoint{1.790607in}{2.011010in}}{\pgfqpoint{1.790607in}{2.002774in}}%
\pgfpathcurveto{\pgfqpoint{1.790607in}{1.994537in}}{\pgfqpoint{1.793880in}{1.986637in}}{\pgfqpoint{1.799704in}{1.980814in}}%
\pgfpathcurveto{\pgfqpoint{1.805528in}{1.974990in}}{\pgfqpoint{1.813428in}{1.971717in}}{\pgfqpoint{1.821664in}{1.971717in}}%
\pgfpathclose%
\pgfusepath{stroke,fill}%
\end{pgfscope}%
\begin{pgfscope}%
\pgfpathrectangle{\pgfqpoint{0.100000in}{0.212622in}}{\pgfqpoint{3.696000in}{3.696000in}}%
\pgfusepath{clip}%
\pgfsetbuttcap%
\pgfsetroundjoin%
\definecolor{currentfill}{rgb}{0.121569,0.466667,0.705882}%
\pgfsetfillcolor{currentfill}%
\pgfsetfillopacity{0.340246}%
\pgfsetlinewidth{1.003750pt}%
\definecolor{currentstroke}{rgb}{0.121569,0.466667,0.705882}%
\pgfsetstrokecolor{currentstroke}%
\pgfsetstrokeopacity{0.340246}%
\pgfsetdash{}{0pt}%
\pgfpathmoveto{\pgfqpoint{1.819047in}{1.969717in}}%
\pgfpathcurveto{\pgfqpoint{1.827283in}{1.969717in}}{\pgfqpoint{1.835183in}{1.972990in}}{\pgfqpoint{1.841007in}{1.978814in}}%
\pgfpathcurveto{\pgfqpoint{1.846831in}{1.984637in}}{\pgfqpoint{1.850103in}{1.992538in}}{\pgfqpoint{1.850103in}{2.000774in}}%
\pgfpathcurveto{\pgfqpoint{1.850103in}{2.009010in}}{\pgfqpoint{1.846831in}{2.016910in}}{\pgfqpoint{1.841007in}{2.022734in}}%
\pgfpathcurveto{\pgfqpoint{1.835183in}{2.028558in}}{\pgfqpoint{1.827283in}{2.031830in}}{\pgfqpoint{1.819047in}{2.031830in}}%
\pgfpathcurveto{\pgfqpoint{1.810810in}{2.031830in}}{\pgfqpoint{1.802910in}{2.028558in}}{\pgfqpoint{1.797086in}{2.022734in}}%
\pgfpathcurveto{\pgfqpoint{1.791262in}{2.016910in}}{\pgfqpoint{1.787990in}{2.009010in}}{\pgfqpoint{1.787990in}{2.000774in}}%
\pgfpathcurveto{\pgfqpoint{1.787990in}{1.992538in}}{\pgfqpoint{1.791262in}{1.984637in}}{\pgfqpoint{1.797086in}{1.978814in}}%
\pgfpathcurveto{\pgfqpoint{1.802910in}{1.972990in}}{\pgfqpoint{1.810810in}{1.969717in}}{\pgfqpoint{1.819047in}{1.969717in}}%
\pgfpathclose%
\pgfusepath{stroke,fill}%
\end{pgfscope}%
\begin{pgfscope}%
\pgfpathrectangle{\pgfqpoint{0.100000in}{0.212622in}}{\pgfqpoint{3.696000in}{3.696000in}}%
\pgfusepath{clip}%
\pgfsetbuttcap%
\pgfsetroundjoin%
\definecolor{currentfill}{rgb}{0.121569,0.466667,0.705882}%
\pgfsetfillcolor{currentfill}%
\pgfsetfillopacity{0.341165}%
\pgfsetlinewidth{1.003750pt}%
\definecolor{currentstroke}{rgb}{0.121569,0.466667,0.705882}%
\pgfsetstrokecolor{currentstroke}%
\pgfsetstrokeopacity{0.341165}%
\pgfsetdash{}{0pt}%
\pgfpathmoveto{\pgfqpoint{1.980262in}{2.000485in}}%
\pgfpathcurveto{\pgfqpoint{1.988498in}{2.000485in}}{\pgfqpoint{1.996398in}{2.003758in}}{\pgfqpoint{2.002222in}{2.009581in}}%
\pgfpathcurveto{\pgfqpoint{2.008046in}{2.015405in}}{\pgfqpoint{2.011318in}{2.023305in}}{\pgfqpoint{2.011318in}{2.031542in}}%
\pgfpathcurveto{\pgfqpoint{2.011318in}{2.039778in}}{\pgfqpoint{2.008046in}{2.047678in}}{\pgfqpoint{2.002222in}{2.053502in}}%
\pgfpathcurveto{\pgfqpoint{1.996398in}{2.059326in}}{\pgfqpoint{1.988498in}{2.062598in}}{\pgfqpoint{1.980262in}{2.062598in}}%
\pgfpathcurveto{\pgfqpoint{1.972026in}{2.062598in}}{\pgfqpoint{1.964125in}{2.059326in}}{\pgfqpoint{1.958302in}{2.053502in}}%
\pgfpathcurveto{\pgfqpoint{1.952478in}{2.047678in}}{\pgfqpoint{1.949205in}{2.039778in}}{\pgfqpoint{1.949205in}{2.031542in}}%
\pgfpathcurveto{\pgfqpoint{1.949205in}{2.023305in}}{\pgfqpoint{1.952478in}{2.015405in}}{\pgfqpoint{1.958302in}{2.009581in}}%
\pgfpathcurveto{\pgfqpoint{1.964125in}{2.003758in}}{\pgfqpoint{1.972026in}{2.000485in}}{\pgfqpoint{1.980262in}{2.000485in}}%
\pgfpathclose%
\pgfusepath{stroke,fill}%
\end{pgfscope}%
\begin{pgfscope}%
\pgfpathrectangle{\pgfqpoint{0.100000in}{0.212622in}}{\pgfqpoint{3.696000in}{3.696000in}}%
\pgfusepath{clip}%
\pgfsetbuttcap%
\pgfsetroundjoin%
\definecolor{currentfill}{rgb}{0.121569,0.466667,0.705882}%
\pgfsetfillcolor{currentfill}%
\pgfsetfillopacity{0.341968}%
\pgfsetlinewidth{1.003750pt}%
\definecolor{currentstroke}{rgb}{0.121569,0.466667,0.705882}%
\pgfsetstrokecolor{currentstroke}%
\pgfsetstrokeopacity{0.341968}%
\pgfsetdash{}{0pt}%
\pgfpathmoveto{\pgfqpoint{1.815151in}{1.967764in}}%
\pgfpathcurveto{\pgfqpoint{1.823387in}{1.967764in}}{\pgfqpoint{1.831287in}{1.971037in}}{\pgfqpoint{1.837111in}{1.976861in}}%
\pgfpathcurveto{\pgfqpoint{1.842935in}{1.982685in}}{\pgfqpoint{1.846207in}{1.990585in}}{\pgfqpoint{1.846207in}{1.998821in}}%
\pgfpathcurveto{\pgfqpoint{1.846207in}{2.007057in}}{\pgfqpoint{1.842935in}{2.014957in}}{\pgfqpoint{1.837111in}{2.020781in}}%
\pgfpathcurveto{\pgfqpoint{1.831287in}{2.026605in}}{\pgfqpoint{1.823387in}{2.029877in}}{\pgfqpoint{1.815151in}{2.029877in}}%
\pgfpathcurveto{\pgfqpoint{1.806915in}{2.029877in}}{\pgfqpoint{1.799015in}{2.026605in}}{\pgfqpoint{1.793191in}{2.020781in}}%
\pgfpathcurveto{\pgfqpoint{1.787367in}{2.014957in}}{\pgfqpoint{1.784094in}{2.007057in}}{\pgfqpoint{1.784094in}{1.998821in}}%
\pgfpathcurveto{\pgfqpoint{1.784094in}{1.990585in}}{\pgfqpoint{1.787367in}{1.982685in}}{\pgfqpoint{1.793191in}{1.976861in}}%
\pgfpathcurveto{\pgfqpoint{1.799015in}{1.971037in}}{\pgfqpoint{1.806915in}{1.967764in}}{\pgfqpoint{1.815151in}{1.967764in}}%
\pgfpathclose%
\pgfusepath{stroke,fill}%
\end{pgfscope}%
\begin{pgfscope}%
\pgfpathrectangle{\pgfqpoint{0.100000in}{0.212622in}}{\pgfqpoint{3.696000in}{3.696000in}}%
\pgfusepath{clip}%
\pgfsetbuttcap%
\pgfsetroundjoin%
\definecolor{currentfill}{rgb}{0.121569,0.466667,0.705882}%
\pgfsetfillcolor{currentfill}%
\pgfsetfillopacity{0.342666}%
\pgfsetlinewidth{1.003750pt}%
\definecolor{currentstroke}{rgb}{0.121569,0.466667,0.705882}%
\pgfsetstrokecolor{currentstroke}%
\pgfsetstrokeopacity{0.342666}%
\pgfsetdash{}{0pt}%
\pgfpathmoveto{\pgfqpoint{1.812166in}{1.965804in}}%
\pgfpathcurveto{\pgfqpoint{1.820402in}{1.965804in}}{\pgfqpoint{1.828302in}{1.969076in}}{\pgfqpoint{1.834126in}{1.974900in}}%
\pgfpathcurveto{\pgfqpoint{1.839950in}{1.980724in}}{\pgfqpoint{1.843223in}{1.988624in}}{\pgfqpoint{1.843223in}{1.996860in}}%
\pgfpathcurveto{\pgfqpoint{1.843223in}{2.005097in}}{\pgfqpoint{1.839950in}{2.012997in}}{\pgfqpoint{1.834126in}{2.018821in}}%
\pgfpathcurveto{\pgfqpoint{1.828302in}{2.024644in}}{\pgfqpoint{1.820402in}{2.027917in}}{\pgfqpoint{1.812166in}{2.027917in}}%
\pgfpathcurveto{\pgfqpoint{1.803930in}{2.027917in}}{\pgfqpoint{1.796030in}{2.024644in}}{\pgfqpoint{1.790206in}{2.018821in}}%
\pgfpathcurveto{\pgfqpoint{1.784382in}{2.012997in}}{\pgfqpoint{1.781110in}{2.005097in}}{\pgfqpoint{1.781110in}{1.996860in}}%
\pgfpathcurveto{\pgfqpoint{1.781110in}{1.988624in}}{\pgfqpoint{1.784382in}{1.980724in}}{\pgfqpoint{1.790206in}{1.974900in}}%
\pgfpathcurveto{\pgfqpoint{1.796030in}{1.969076in}}{\pgfqpoint{1.803930in}{1.965804in}}{\pgfqpoint{1.812166in}{1.965804in}}%
\pgfpathclose%
\pgfusepath{stroke,fill}%
\end{pgfscope}%
\begin{pgfscope}%
\pgfpathrectangle{\pgfqpoint{0.100000in}{0.212622in}}{\pgfqpoint{3.696000in}{3.696000in}}%
\pgfusepath{clip}%
\pgfsetbuttcap%
\pgfsetroundjoin%
\definecolor{currentfill}{rgb}{0.121569,0.466667,0.705882}%
\pgfsetfillcolor{currentfill}%
\pgfsetfillopacity{0.344276}%
\pgfsetlinewidth{1.003750pt}%
\definecolor{currentstroke}{rgb}{0.121569,0.466667,0.705882}%
\pgfsetstrokecolor{currentstroke}%
\pgfsetstrokeopacity{0.344276}%
\pgfsetdash{}{0pt}%
\pgfpathmoveto{\pgfqpoint{1.807253in}{1.963706in}}%
\pgfpathcurveto{\pgfqpoint{1.815489in}{1.963706in}}{\pgfqpoint{1.823389in}{1.966979in}}{\pgfqpoint{1.829213in}{1.972802in}}%
\pgfpathcurveto{\pgfqpoint{1.835037in}{1.978626in}}{\pgfqpoint{1.838310in}{1.986526in}}{\pgfqpoint{1.838310in}{1.994763in}}%
\pgfpathcurveto{\pgfqpoint{1.838310in}{2.002999in}}{\pgfqpoint{1.835037in}{2.010899in}}{\pgfqpoint{1.829213in}{2.016723in}}%
\pgfpathcurveto{\pgfqpoint{1.823389in}{2.022547in}}{\pgfqpoint{1.815489in}{2.025819in}}{\pgfqpoint{1.807253in}{2.025819in}}%
\pgfpathcurveto{\pgfqpoint{1.799017in}{2.025819in}}{\pgfqpoint{1.791117in}{2.022547in}}{\pgfqpoint{1.785293in}{2.016723in}}%
\pgfpathcurveto{\pgfqpoint{1.779469in}{2.010899in}}{\pgfqpoint{1.776197in}{2.002999in}}{\pgfqpoint{1.776197in}{1.994763in}}%
\pgfpathcurveto{\pgfqpoint{1.776197in}{1.986526in}}{\pgfqpoint{1.779469in}{1.978626in}}{\pgfqpoint{1.785293in}{1.972802in}}%
\pgfpathcurveto{\pgfqpoint{1.791117in}{1.966979in}}{\pgfqpoint{1.799017in}{1.963706in}}{\pgfqpoint{1.807253in}{1.963706in}}%
\pgfpathclose%
\pgfusepath{stroke,fill}%
\end{pgfscope}%
\begin{pgfscope}%
\pgfpathrectangle{\pgfqpoint{0.100000in}{0.212622in}}{\pgfqpoint{3.696000in}{3.696000in}}%
\pgfusepath{clip}%
\pgfsetbuttcap%
\pgfsetroundjoin%
\definecolor{currentfill}{rgb}{0.121569,0.466667,0.705882}%
\pgfsetfillcolor{currentfill}%
\pgfsetfillopacity{0.345437}%
\pgfsetlinewidth{1.003750pt}%
\definecolor{currentstroke}{rgb}{0.121569,0.466667,0.705882}%
\pgfsetstrokecolor{currentstroke}%
\pgfsetstrokeopacity{0.345437}%
\pgfsetdash{}{0pt}%
\pgfpathmoveto{\pgfqpoint{1.984160in}{2.002328in}}%
\pgfpathcurveto{\pgfqpoint{1.992396in}{2.002328in}}{\pgfqpoint{2.000296in}{2.005600in}}{\pgfqpoint{2.006120in}{2.011424in}}%
\pgfpathcurveto{\pgfqpoint{2.011944in}{2.017248in}}{\pgfqpoint{2.015217in}{2.025148in}}{\pgfqpoint{2.015217in}{2.033384in}}%
\pgfpathcurveto{\pgfqpoint{2.015217in}{2.041621in}}{\pgfqpoint{2.011944in}{2.049521in}}{\pgfqpoint{2.006120in}{2.055345in}}%
\pgfpathcurveto{\pgfqpoint{2.000296in}{2.061169in}}{\pgfqpoint{1.992396in}{2.064441in}}{\pgfqpoint{1.984160in}{2.064441in}}%
\pgfpathcurveto{\pgfqpoint{1.975924in}{2.064441in}}{\pgfqpoint{1.968024in}{2.061169in}}{\pgfqpoint{1.962200in}{2.055345in}}%
\pgfpathcurveto{\pgfqpoint{1.956376in}{2.049521in}}{\pgfqpoint{1.953104in}{2.041621in}}{\pgfqpoint{1.953104in}{2.033384in}}%
\pgfpathcurveto{\pgfqpoint{1.953104in}{2.025148in}}{\pgfqpoint{1.956376in}{2.017248in}}{\pgfqpoint{1.962200in}{2.011424in}}%
\pgfpathcurveto{\pgfqpoint{1.968024in}{2.005600in}}{\pgfqpoint{1.975924in}{2.002328in}}{\pgfqpoint{1.984160in}{2.002328in}}%
\pgfpathclose%
\pgfusepath{stroke,fill}%
\end{pgfscope}%
\begin{pgfscope}%
\pgfpathrectangle{\pgfqpoint{0.100000in}{0.212622in}}{\pgfqpoint{3.696000in}{3.696000in}}%
\pgfusepath{clip}%
\pgfsetbuttcap%
\pgfsetroundjoin%
\definecolor{currentfill}{rgb}{0.121569,0.466667,0.705882}%
\pgfsetfillcolor{currentfill}%
\pgfsetfillopacity{0.347462}%
\pgfsetlinewidth{1.003750pt}%
\definecolor{currentstroke}{rgb}{0.121569,0.466667,0.705882}%
\pgfsetstrokecolor{currentstroke}%
\pgfsetstrokeopacity{0.347462}%
\pgfsetdash{}{0pt}%
\pgfpathmoveto{\pgfqpoint{1.799501in}{1.960024in}}%
\pgfpathcurveto{\pgfqpoint{1.807737in}{1.960024in}}{\pgfqpoint{1.815637in}{1.963296in}}{\pgfqpoint{1.821461in}{1.969120in}}%
\pgfpathcurveto{\pgfqpoint{1.827285in}{1.974944in}}{\pgfqpoint{1.830558in}{1.982844in}}{\pgfqpoint{1.830558in}{1.991081in}}%
\pgfpathcurveto{\pgfqpoint{1.830558in}{1.999317in}}{\pgfqpoint{1.827285in}{2.007217in}}{\pgfqpoint{1.821461in}{2.013041in}}%
\pgfpathcurveto{\pgfqpoint{1.815637in}{2.018865in}}{\pgfqpoint{1.807737in}{2.022137in}}{\pgfqpoint{1.799501in}{2.022137in}}%
\pgfpathcurveto{\pgfqpoint{1.791265in}{2.022137in}}{\pgfqpoint{1.783365in}{2.018865in}}{\pgfqpoint{1.777541in}{2.013041in}}%
\pgfpathcurveto{\pgfqpoint{1.771717in}{2.007217in}}{\pgfqpoint{1.768445in}{1.999317in}}{\pgfqpoint{1.768445in}{1.991081in}}%
\pgfpathcurveto{\pgfqpoint{1.768445in}{1.982844in}}{\pgfqpoint{1.771717in}{1.974944in}}{\pgfqpoint{1.777541in}{1.969120in}}%
\pgfpathcurveto{\pgfqpoint{1.783365in}{1.963296in}}{\pgfqpoint{1.791265in}{1.960024in}}{\pgfqpoint{1.799501in}{1.960024in}}%
\pgfpathclose%
\pgfusepath{stroke,fill}%
\end{pgfscope}%
\begin{pgfscope}%
\pgfpathrectangle{\pgfqpoint{0.100000in}{0.212622in}}{\pgfqpoint{3.696000in}{3.696000in}}%
\pgfusepath{clip}%
\pgfsetbuttcap%
\pgfsetroundjoin%
\definecolor{currentfill}{rgb}{0.121569,0.466667,0.705882}%
\pgfsetfillcolor{currentfill}%
\pgfsetfillopacity{0.348706}%
\pgfsetlinewidth{1.003750pt}%
\definecolor{currentstroke}{rgb}{0.121569,0.466667,0.705882}%
\pgfsetstrokecolor{currentstroke}%
\pgfsetstrokeopacity{0.348706}%
\pgfsetdash{}{0pt}%
\pgfpathmoveto{\pgfqpoint{1.988033in}{1.990827in}}%
\pgfpathcurveto{\pgfqpoint{1.996269in}{1.990827in}}{\pgfqpoint{2.004169in}{1.994099in}}{\pgfqpoint{2.009993in}{1.999923in}}%
\pgfpathcurveto{\pgfqpoint{2.015817in}{2.005747in}}{\pgfqpoint{2.019089in}{2.013647in}}{\pgfqpoint{2.019089in}{2.021883in}}%
\pgfpathcurveto{\pgfqpoint{2.019089in}{2.030120in}}{\pgfqpoint{2.015817in}{2.038020in}}{\pgfqpoint{2.009993in}{2.043844in}}%
\pgfpathcurveto{\pgfqpoint{2.004169in}{2.049668in}}{\pgfqpoint{1.996269in}{2.052940in}}{\pgfqpoint{1.988033in}{2.052940in}}%
\pgfpathcurveto{\pgfqpoint{1.979797in}{2.052940in}}{\pgfqpoint{1.971897in}{2.049668in}}{\pgfqpoint{1.966073in}{2.043844in}}%
\pgfpathcurveto{\pgfqpoint{1.960249in}{2.038020in}}{\pgfqpoint{1.956976in}{2.030120in}}{\pgfqpoint{1.956976in}{2.021883in}}%
\pgfpathcurveto{\pgfqpoint{1.956976in}{2.013647in}}{\pgfqpoint{1.960249in}{2.005747in}}{\pgfqpoint{1.966073in}{1.999923in}}%
\pgfpathcurveto{\pgfqpoint{1.971897in}{1.994099in}}{\pgfqpoint{1.979797in}{1.990827in}}{\pgfqpoint{1.988033in}{1.990827in}}%
\pgfpathclose%
\pgfusepath{stroke,fill}%
\end{pgfscope}%
\begin{pgfscope}%
\pgfpathrectangle{\pgfqpoint{0.100000in}{0.212622in}}{\pgfqpoint{3.696000in}{3.696000in}}%
\pgfusepath{clip}%
\pgfsetbuttcap%
\pgfsetroundjoin%
\definecolor{currentfill}{rgb}{0.121569,0.466667,0.705882}%
\pgfsetfillcolor{currentfill}%
\pgfsetfillopacity{0.348957}%
\pgfsetlinewidth{1.003750pt}%
\definecolor{currentstroke}{rgb}{0.121569,0.466667,0.705882}%
\pgfsetstrokecolor{currentstroke}%
\pgfsetstrokeopacity{0.348957}%
\pgfsetdash{}{0pt}%
\pgfpathmoveto{\pgfqpoint{1.793173in}{1.952442in}}%
\pgfpathcurveto{\pgfqpoint{1.801410in}{1.952442in}}{\pgfqpoint{1.809310in}{1.955715in}}{\pgfqpoint{1.815134in}{1.961538in}}%
\pgfpathcurveto{\pgfqpoint{1.820958in}{1.967362in}}{\pgfqpoint{1.824230in}{1.975262in}}{\pgfqpoint{1.824230in}{1.983499in}}%
\pgfpathcurveto{\pgfqpoint{1.824230in}{1.991735in}}{\pgfqpoint{1.820958in}{1.999635in}}{\pgfqpoint{1.815134in}{2.005459in}}%
\pgfpathcurveto{\pgfqpoint{1.809310in}{2.011283in}}{\pgfqpoint{1.801410in}{2.014555in}}{\pgfqpoint{1.793173in}{2.014555in}}%
\pgfpathcurveto{\pgfqpoint{1.784937in}{2.014555in}}{\pgfqpoint{1.777037in}{2.011283in}}{\pgfqpoint{1.771213in}{2.005459in}}%
\pgfpathcurveto{\pgfqpoint{1.765389in}{1.999635in}}{\pgfqpoint{1.762117in}{1.991735in}}{\pgfqpoint{1.762117in}{1.983499in}}%
\pgfpathcurveto{\pgfqpoint{1.762117in}{1.975262in}}{\pgfqpoint{1.765389in}{1.967362in}}{\pgfqpoint{1.771213in}{1.961538in}}%
\pgfpathcurveto{\pgfqpoint{1.777037in}{1.955715in}}{\pgfqpoint{1.784937in}{1.952442in}}{\pgfqpoint{1.793173in}{1.952442in}}%
\pgfpathclose%
\pgfusepath{stroke,fill}%
\end{pgfscope}%
\begin{pgfscope}%
\pgfpathrectangle{\pgfqpoint{0.100000in}{0.212622in}}{\pgfqpoint{3.696000in}{3.696000in}}%
\pgfusepath{clip}%
\pgfsetbuttcap%
\pgfsetroundjoin%
\definecolor{currentfill}{rgb}{0.121569,0.466667,0.705882}%
\pgfsetfillcolor{currentfill}%
\pgfsetfillopacity{0.350005}%
\pgfsetlinewidth{1.003750pt}%
\definecolor{currentstroke}{rgb}{0.121569,0.466667,0.705882}%
\pgfsetstrokecolor{currentstroke}%
\pgfsetstrokeopacity{0.350005}%
\pgfsetdash{}{0pt}%
\pgfpathmoveto{\pgfqpoint{1.789883in}{1.950727in}}%
\pgfpathcurveto{\pgfqpoint{1.798119in}{1.950727in}}{\pgfqpoint{1.806019in}{1.953999in}}{\pgfqpoint{1.811843in}{1.959823in}}%
\pgfpathcurveto{\pgfqpoint{1.817667in}{1.965647in}}{\pgfqpoint{1.820939in}{1.973547in}}{\pgfqpoint{1.820939in}{1.981783in}}%
\pgfpathcurveto{\pgfqpoint{1.820939in}{1.990020in}}{\pgfqpoint{1.817667in}{1.997920in}}{\pgfqpoint{1.811843in}{2.003744in}}%
\pgfpathcurveto{\pgfqpoint{1.806019in}{2.009567in}}{\pgfqpoint{1.798119in}{2.012840in}}{\pgfqpoint{1.789883in}{2.012840in}}%
\pgfpathcurveto{\pgfqpoint{1.781646in}{2.012840in}}{\pgfqpoint{1.773746in}{2.009567in}}{\pgfqpoint{1.767922in}{2.003744in}}%
\pgfpathcurveto{\pgfqpoint{1.762098in}{1.997920in}}{\pgfqpoint{1.758826in}{1.990020in}}{\pgfqpoint{1.758826in}{1.981783in}}%
\pgfpathcurveto{\pgfqpoint{1.758826in}{1.973547in}}{\pgfqpoint{1.762098in}{1.965647in}}{\pgfqpoint{1.767922in}{1.959823in}}%
\pgfpathcurveto{\pgfqpoint{1.773746in}{1.953999in}}{\pgfqpoint{1.781646in}{1.950727in}}{\pgfqpoint{1.789883in}{1.950727in}}%
\pgfpathclose%
\pgfusepath{stroke,fill}%
\end{pgfscope}%
\begin{pgfscope}%
\pgfpathrectangle{\pgfqpoint{0.100000in}{0.212622in}}{\pgfqpoint{3.696000in}{3.696000in}}%
\pgfusepath{clip}%
\pgfsetbuttcap%
\pgfsetroundjoin%
\definecolor{currentfill}{rgb}{0.121569,0.466667,0.705882}%
\pgfsetfillcolor{currentfill}%
\pgfsetfillopacity{0.350514}%
\pgfsetlinewidth{1.003750pt}%
\definecolor{currentstroke}{rgb}{0.121569,0.466667,0.705882}%
\pgfsetstrokecolor{currentstroke}%
\pgfsetstrokeopacity{0.350514}%
\pgfsetdash{}{0pt}%
\pgfpathmoveto{\pgfqpoint{1.788171in}{1.949224in}}%
\pgfpathcurveto{\pgfqpoint{1.796408in}{1.949224in}}{\pgfqpoint{1.804308in}{1.952497in}}{\pgfqpoint{1.810132in}{1.958321in}}%
\pgfpathcurveto{\pgfqpoint{1.815956in}{1.964145in}}{\pgfqpoint{1.819228in}{1.972045in}}{\pgfqpoint{1.819228in}{1.980281in}}%
\pgfpathcurveto{\pgfqpoint{1.819228in}{1.988517in}}{\pgfqpoint{1.815956in}{1.996417in}}{\pgfqpoint{1.810132in}{2.002241in}}%
\pgfpathcurveto{\pgfqpoint{1.804308in}{2.008065in}}{\pgfqpoint{1.796408in}{2.011337in}}{\pgfqpoint{1.788171in}{2.011337in}}%
\pgfpathcurveto{\pgfqpoint{1.779935in}{2.011337in}}{\pgfqpoint{1.772035in}{2.008065in}}{\pgfqpoint{1.766211in}{2.002241in}}%
\pgfpathcurveto{\pgfqpoint{1.760387in}{1.996417in}}{\pgfqpoint{1.757115in}{1.988517in}}{\pgfqpoint{1.757115in}{1.980281in}}%
\pgfpathcurveto{\pgfqpoint{1.757115in}{1.972045in}}{\pgfqpoint{1.760387in}{1.964145in}}{\pgfqpoint{1.766211in}{1.958321in}}%
\pgfpathcurveto{\pgfqpoint{1.772035in}{1.952497in}}{\pgfqpoint{1.779935in}{1.949224in}}{\pgfqpoint{1.788171in}{1.949224in}}%
\pgfpathclose%
\pgfusepath{stroke,fill}%
\end{pgfscope}%
\begin{pgfscope}%
\pgfpathrectangle{\pgfqpoint{0.100000in}{0.212622in}}{\pgfqpoint{3.696000in}{3.696000in}}%
\pgfusepath{clip}%
\pgfsetbuttcap%
\pgfsetroundjoin%
\definecolor{currentfill}{rgb}{0.121569,0.466667,0.705882}%
\pgfsetfillcolor{currentfill}%
\pgfsetfillopacity{0.351000}%
\pgfsetlinewidth{1.003750pt}%
\definecolor{currentstroke}{rgb}{0.121569,0.466667,0.705882}%
\pgfsetstrokecolor{currentstroke}%
\pgfsetstrokeopacity{0.351000}%
\pgfsetdash{}{0pt}%
\pgfpathmoveto{\pgfqpoint{1.775567in}{1.935446in}}%
\pgfpathcurveto{\pgfqpoint{1.783803in}{1.935446in}}{\pgfqpoint{1.791703in}{1.938718in}}{\pgfqpoint{1.797527in}{1.944542in}}%
\pgfpathcurveto{\pgfqpoint{1.803351in}{1.950366in}}{\pgfqpoint{1.806623in}{1.958266in}}{\pgfqpoint{1.806623in}{1.966503in}}%
\pgfpathcurveto{\pgfqpoint{1.806623in}{1.974739in}}{\pgfqpoint{1.803351in}{1.982639in}}{\pgfqpoint{1.797527in}{1.988463in}}%
\pgfpathcurveto{\pgfqpoint{1.791703in}{1.994287in}}{\pgfqpoint{1.783803in}{1.997559in}}{\pgfqpoint{1.775567in}{1.997559in}}%
\pgfpathcurveto{\pgfqpoint{1.767331in}{1.997559in}}{\pgfqpoint{1.759431in}{1.994287in}}{\pgfqpoint{1.753607in}{1.988463in}}%
\pgfpathcurveto{\pgfqpoint{1.747783in}{1.982639in}}{\pgfqpoint{1.744510in}{1.974739in}}{\pgfqpoint{1.744510in}{1.966503in}}%
\pgfpathcurveto{\pgfqpoint{1.744510in}{1.958266in}}{\pgfqpoint{1.747783in}{1.950366in}}{\pgfqpoint{1.753607in}{1.944542in}}%
\pgfpathcurveto{\pgfqpoint{1.759431in}{1.938718in}}{\pgfqpoint{1.767331in}{1.935446in}}{\pgfqpoint{1.775567in}{1.935446in}}%
\pgfpathclose%
\pgfusepath{stroke,fill}%
\end{pgfscope}%
\begin{pgfscope}%
\pgfpathrectangle{\pgfqpoint{0.100000in}{0.212622in}}{\pgfqpoint{3.696000in}{3.696000in}}%
\pgfusepath{clip}%
\pgfsetbuttcap%
\pgfsetroundjoin%
\definecolor{currentfill}{rgb}{0.121569,0.466667,0.705882}%
\pgfsetfillcolor{currentfill}%
\pgfsetfillopacity{0.351062}%
\pgfsetlinewidth{1.003750pt}%
\definecolor{currentstroke}{rgb}{0.121569,0.466667,0.705882}%
\pgfsetstrokecolor{currentstroke}%
\pgfsetstrokeopacity{0.351062}%
\pgfsetdash{}{0pt}%
\pgfpathmoveto{\pgfqpoint{1.783806in}{1.946515in}}%
\pgfpathcurveto{\pgfqpoint{1.792042in}{1.946515in}}{\pgfqpoint{1.799942in}{1.949787in}}{\pgfqpoint{1.805766in}{1.955611in}}%
\pgfpathcurveto{\pgfqpoint{1.811590in}{1.961435in}}{\pgfqpoint{1.814862in}{1.969335in}}{\pgfqpoint{1.814862in}{1.977571in}}%
\pgfpathcurveto{\pgfqpoint{1.814862in}{1.985807in}}{\pgfqpoint{1.811590in}{1.993707in}}{\pgfqpoint{1.805766in}{1.999531in}}%
\pgfpathcurveto{\pgfqpoint{1.799942in}{2.005355in}}{\pgfqpoint{1.792042in}{2.008628in}}{\pgfqpoint{1.783806in}{2.008628in}}%
\pgfpathcurveto{\pgfqpoint{1.775570in}{2.008628in}}{\pgfqpoint{1.767669in}{2.005355in}}{\pgfqpoint{1.761846in}{1.999531in}}%
\pgfpathcurveto{\pgfqpoint{1.756022in}{1.993707in}}{\pgfqpoint{1.752749in}{1.985807in}}{\pgfqpoint{1.752749in}{1.977571in}}%
\pgfpathcurveto{\pgfqpoint{1.752749in}{1.969335in}}{\pgfqpoint{1.756022in}{1.961435in}}{\pgfqpoint{1.761846in}{1.955611in}}%
\pgfpathcurveto{\pgfqpoint{1.767669in}{1.949787in}}{\pgfqpoint{1.775570in}{1.946515in}}{\pgfqpoint{1.783806in}{1.946515in}}%
\pgfpathclose%
\pgfusepath{stroke,fill}%
\end{pgfscope}%
\begin{pgfscope}%
\pgfpathrectangle{\pgfqpoint{0.100000in}{0.212622in}}{\pgfqpoint{3.696000in}{3.696000in}}%
\pgfusepath{clip}%
\pgfsetbuttcap%
\pgfsetroundjoin%
\definecolor{currentfill}{rgb}{0.121569,0.466667,0.705882}%
\pgfsetfillcolor{currentfill}%
\pgfsetfillopacity{0.353073}%
\pgfsetlinewidth{1.003750pt}%
\definecolor{currentstroke}{rgb}{0.121569,0.466667,0.705882}%
\pgfsetstrokecolor{currentstroke}%
\pgfsetstrokeopacity{0.353073}%
\pgfsetdash{}{0pt}%
\pgfpathmoveto{\pgfqpoint{1.771818in}{1.935705in}}%
\pgfpathcurveto{\pgfqpoint{1.780054in}{1.935705in}}{\pgfqpoint{1.787954in}{1.938978in}}{\pgfqpoint{1.793778in}{1.944802in}}%
\pgfpathcurveto{\pgfqpoint{1.799602in}{1.950626in}}{\pgfqpoint{1.802875in}{1.958526in}}{\pgfqpoint{1.802875in}{1.966762in}}%
\pgfpathcurveto{\pgfqpoint{1.802875in}{1.974998in}}{\pgfqpoint{1.799602in}{1.982898in}}{\pgfqpoint{1.793778in}{1.988722in}}%
\pgfpathcurveto{\pgfqpoint{1.787954in}{1.994546in}}{\pgfqpoint{1.780054in}{1.997818in}}{\pgfqpoint{1.771818in}{1.997818in}}%
\pgfpathcurveto{\pgfqpoint{1.763582in}{1.997818in}}{\pgfqpoint{1.755682in}{1.994546in}}{\pgfqpoint{1.749858in}{1.988722in}}%
\pgfpathcurveto{\pgfqpoint{1.744034in}{1.982898in}}{\pgfqpoint{1.740762in}{1.974998in}}{\pgfqpoint{1.740762in}{1.966762in}}%
\pgfpathcurveto{\pgfqpoint{1.740762in}{1.958526in}}{\pgfqpoint{1.744034in}{1.950626in}}{\pgfqpoint{1.749858in}{1.944802in}}%
\pgfpathcurveto{\pgfqpoint{1.755682in}{1.938978in}}{\pgfqpoint{1.763582in}{1.935705in}}{\pgfqpoint{1.771818in}{1.935705in}}%
\pgfpathclose%
\pgfusepath{stroke,fill}%
\end{pgfscope}%
\begin{pgfscope}%
\pgfpathrectangle{\pgfqpoint{0.100000in}{0.212622in}}{\pgfqpoint{3.696000in}{3.696000in}}%
\pgfusepath{clip}%
\pgfsetbuttcap%
\pgfsetroundjoin%
\definecolor{currentfill}{rgb}{0.121569,0.466667,0.705882}%
\pgfsetfillcolor{currentfill}%
\pgfsetfillopacity{0.353390}%
\pgfsetlinewidth{1.003750pt}%
\definecolor{currentstroke}{rgb}{0.121569,0.466667,0.705882}%
\pgfsetstrokecolor{currentstroke}%
\pgfsetstrokeopacity{0.353390}%
\pgfsetdash{}{0pt}%
\pgfpathmoveto{\pgfqpoint{1.989544in}{1.985939in}}%
\pgfpathcurveto{\pgfqpoint{1.997780in}{1.985939in}}{\pgfqpoint{2.005680in}{1.989212in}}{\pgfqpoint{2.011504in}{1.995036in}}%
\pgfpathcurveto{\pgfqpoint{2.017328in}{2.000860in}}{\pgfqpoint{2.020600in}{2.008760in}}{\pgfqpoint{2.020600in}{2.016996in}}%
\pgfpathcurveto{\pgfqpoint{2.020600in}{2.025232in}}{\pgfqpoint{2.017328in}{2.033132in}}{\pgfqpoint{2.011504in}{2.038956in}}%
\pgfpathcurveto{\pgfqpoint{2.005680in}{2.044780in}}{\pgfqpoint{1.997780in}{2.048052in}}{\pgfqpoint{1.989544in}{2.048052in}}%
\pgfpathcurveto{\pgfqpoint{1.981308in}{2.048052in}}{\pgfqpoint{1.973408in}{2.044780in}}{\pgfqpoint{1.967584in}{2.038956in}}%
\pgfpathcurveto{\pgfqpoint{1.961760in}{2.033132in}}{\pgfqpoint{1.958487in}{2.025232in}}{\pgfqpoint{1.958487in}{2.016996in}}%
\pgfpathcurveto{\pgfqpoint{1.958487in}{2.008760in}}{\pgfqpoint{1.961760in}{2.000860in}}{\pgfqpoint{1.967584in}{1.995036in}}%
\pgfpathcurveto{\pgfqpoint{1.973408in}{1.989212in}}{\pgfqpoint{1.981308in}{1.985939in}}{\pgfqpoint{1.989544in}{1.985939in}}%
\pgfpathclose%
\pgfusepath{stroke,fill}%
\end{pgfscope}%
\begin{pgfscope}%
\pgfpathrectangle{\pgfqpoint{0.100000in}{0.212622in}}{\pgfqpoint{3.696000in}{3.696000in}}%
\pgfusepath{clip}%
\pgfsetbuttcap%
\pgfsetroundjoin%
\definecolor{currentfill}{rgb}{0.121569,0.466667,0.705882}%
\pgfsetfillcolor{currentfill}%
\pgfsetfillopacity{0.354044}%
\pgfsetlinewidth{1.003750pt}%
\definecolor{currentstroke}{rgb}{0.121569,0.466667,0.705882}%
\pgfsetstrokecolor{currentstroke}%
\pgfsetstrokeopacity{0.354044}%
\pgfsetdash{}{0pt}%
\pgfpathmoveto{\pgfqpoint{1.767351in}{1.932349in}}%
\pgfpathcurveto{\pgfqpoint{1.775587in}{1.932349in}}{\pgfqpoint{1.783487in}{1.935622in}}{\pgfqpoint{1.789311in}{1.941445in}}%
\pgfpathcurveto{\pgfqpoint{1.795135in}{1.947269in}}{\pgfqpoint{1.798407in}{1.955169in}}{\pgfqpoint{1.798407in}{1.963406in}}%
\pgfpathcurveto{\pgfqpoint{1.798407in}{1.971642in}}{\pgfqpoint{1.795135in}{1.979542in}}{\pgfqpoint{1.789311in}{1.985366in}}%
\pgfpathcurveto{\pgfqpoint{1.783487in}{1.991190in}}{\pgfqpoint{1.775587in}{1.994462in}}{\pgfqpoint{1.767351in}{1.994462in}}%
\pgfpathcurveto{\pgfqpoint{1.759115in}{1.994462in}}{\pgfqpoint{1.751215in}{1.991190in}}{\pgfqpoint{1.745391in}{1.985366in}}%
\pgfpathcurveto{\pgfqpoint{1.739567in}{1.979542in}}{\pgfqpoint{1.736294in}{1.971642in}}{\pgfqpoint{1.736294in}{1.963406in}}%
\pgfpathcurveto{\pgfqpoint{1.736294in}{1.955169in}}{\pgfqpoint{1.739567in}{1.947269in}}{\pgfqpoint{1.745391in}{1.941445in}}%
\pgfpathcurveto{\pgfqpoint{1.751215in}{1.935622in}}{\pgfqpoint{1.759115in}{1.932349in}}{\pgfqpoint{1.767351in}{1.932349in}}%
\pgfpathclose%
\pgfusepath{stroke,fill}%
\end{pgfscope}%
\begin{pgfscope}%
\pgfpathrectangle{\pgfqpoint{0.100000in}{0.212622in}}{\pgfqpoint{3.696000in}{3.696000in}}%
\pgfusepath{clip}%
\pgfsetbuttcap%
\pgfsetroundjoin%
\definecolor{currentfill}{rgb}{0.121569,0.466667,0.705882}%
\pgfsetfillcolor{currentfill}%
\pgfsetfillopacity{0.355615}%
\pgfsetlinewidth{1.003750pt}%
\definecolor{currentstroke}{rgb}{0.121569,0.466667,0.705882}%
\pgfsetstrokecolor{currentstroke}%
\pgfsetstrokeopacity{0.355615}%
\pgfsetdash{}{0pt}%
\pgfpathmoveto{\pgfqpoint{1.759566in}{1.924283in}}%
\pgfpathcurveto{\pgfqpoint{1.767802in}{1.924283in}}{\pgfqpoint{1.775702in}{1.927555in}}{\pgfqpoint{1.781526in}{1.933379in}}%
\pgfpathcurveto{\pgfqpoint{1.787350in}{1.939203in}}{\pgfqpoint{1.790622in}{1.947103in}}{\pgfqpoint{1.790622in}{1.955339in}}%
\pgfpathcurveto{\pgfqpoint{1.790622in}{1.963576in}}{\pgfqpoint{1.787350in}{1.971476in}}{\pgfqpoint{1.781526in}{1.977300in}}%
\pgfpathcurveto{\pgfqpoint{1.775702in}{1.983124in}}{\pgfqpoint{1.767802in}{1.986396in}}{\pgfqpoint{1.759566in}{1.986396in}}%
\pgfpathcurveto{\pgfqpoint{1.751329in}{1.986396in}}{\pgfqpoint{1.743429in}{1.983124in}}{\pgfqpoint{1.737606in}{1.977300in}}%
\pgfpathcurveto{\pgfqpoint{1.731782in}{1.971476in}}{\pgfqpoint{1.728509in}{1.963576in}}{\pgfqpoint{1.728509in}{1.955339in}}%
\pgfpathcurveto{\pgfqpoint{1.728509in}{1.947103in}}{\pgfqpoint{1.731782in}{1.939203in}}{\pgfqpoint{1.737606in}{1.933379in}}%
\pgfpathcurveto{\pgfqpoint{1.743429in}{1.927555in}}{\pgfqpoint{1.751329in}{1.924283in}}{\pgfqpoint{1.759566in}{1.924283in}}%
\pgfpathclose%
\pgfusepath{stroke,fill}%
\end{pgfscope}%
\begin{pgfscope}%
\pgfpathrectangle{\pgfqpoint{0.100000in}{0.212622in}}{\pgfqpoint{3.696000in}{3.696000in}}%
\pgfusepath{clip}%
\pgfsetbuttcap%
\pgfsetroundjoin%
\definecolor{currentfill}{rgb}{0.121569,0.466667,0.705882}%
\pgfsetfillcolor{currentfill}%
\pgfsetfillopacity{0.358258}%
\pgfsetlinewidth{1.003750pt}%
\definecolor{currentstroke}{rgb}{0.121569,0.466667,0.705882}%
\pgfsetstrokecolor{currentstroke}%
\pgfsetstrokeopacity{0.358258}%
\pgfsetdash{}{0pt}%
\pgfpathmoveto{\pgfqpoint{1.753328in}{1.922556in}}%
\pgfpathcurveto{\pgfqpoint{1.761565in}{1.922556in}}{\pgfqpoint{1.769465in}{1.925829in}}{\pgfqpoint{1.775289in}{1.931653in}}%
\pgfpathcurveto{\pgfqpoint{1.781112in}{1.937477in}}{\pgfqpoint{1.784385in}{1.945377in}}{\pgfqpoint{1.784385in}{1.953613in}}%
\pgfpathcurveto{\pgfqpoint{1.784385in}{1.961849in}}{\pgfqpoint{1.781112in}{1.969749in}}{\pgfqpoint{1.775289in}{1.975573in}}%
\pgfpathcurveto{\pgfqpoint{1.769465in}{1.981397in}}{\pgfqpoint{1.761565in}{1.984669in}}{\pgfqpoint{1.753328in}{1.984669in}}%
\pgfpathcurveto{\pgfqpoint{1.745092in}{1.984669in}}{\pgfqpoint{1.737192in}{1.981397in}}{\pgfqpoint{1.731368in}{1.975573in}}%
\pgfpathcurveto{\pgfqpoint{1.725544in}{1.969749in}}{\pgfqpoint{1.722272in}{1.961849in}}{\pgfqpoint{1.722272in}{1.953613in}}%
\pgfpathcurveto{\pgfqpoint{1.722272in}{1.945377in}}{\pgfqpoint{1.725544in}{1.937477in}}{\pgfqpoint{1.731368in}{1.931653in}}%
\pgfpathcurveto{\pgfqpoint{1.737192in}{1.925829in}}{\pgfqpoint{1.745092in}{1.922556in}}{\pgfqpoint{1.753328in}{1.922556in}}%
\pgfpathclose%
\pgfusepath{stroke,fill}%
\end{pgfscope}%
\begin{pgfscope}%
\pgfpathrectangle{\pgfqpoint{0.100000in}{0.212622in}}{\pgfqpoint{3.696000in}{3.696000in}}%
\pgfusepath{clip}%
\pgfsetbuttcap%
\pgfsetroundjoin%
\definecolor{currentfill}{rgb}{0.121569,0.466667,0.705882}%
\pgfsetfillcolor{currentfill}%
\pgfsetfillopacity{0.358625}%
\pgfsetlinewidth{1.003750pt}%
\definecolor{currentstroke}{rgb}{0.121569,0.466667,0.705882}%
\pgfsetstrokecolor{currentstroke}%
\pgfsetstrokeopacity{0.358625}%
\pgfsetdash{}{0pt}%
\pgfpathmoveto{\pgfqpoint{1.993204in}{1.982695in}}%
\pgfpathcurveto{\pgfqpoint{2.001440in}{1.982695in}}{\pgfqpoint{2.009340in}{1.985967in}}{\pgfqpoint{2.015164in}{1.991791in}}%
\pgfpathcurveto{\pgfqpoint{2.020988in}{1.997615in}}{\pgfqpoint{2.024261in}{2.005515in}}{\pgfqpoint{2.024261in}{2.013751in}}%
\pgfpathcurveto{\pgfqpoint{2.024261in}{2.021987in}}{\pgfqpoint{2.020988in}{2.029888in}}{\pgfqpoint{2.015164in}{2.035711in}}%
\pgfpathcurveto{\pgfqpoint{2.009340in}{2.041535in}}{\pgfqpoint{2.001440in}{2.044808in}}{\pgfqpoint{1.993204in}{2.044808in}}%
\pgfpathcurveto{\pgfqpoint{1.984968in}{2.044808in}}{\pgfqpoint{1.977068in}{2.041535in}}{\pgfqpoint{1.971244in}{2.035711in}}%
\pgfpathcurveto{\pgfqpoint{1.965420in}{2.029888in}}{\pgfqpoint{1.962148in}{2.021987in}}{\pgfqpoint{1.962148in}{2.013751in}}%
\pgfpathcurveto{\pgfqpoint{1.962148in}{2.005515in}}{\pgfqpoint{1.965420in}{1.997615in}}{\pgfqpoint{1.971244in}{1.991791in}}%
\pgfpathcurveto{\pgfqpoint{1.977068in}{1.985967in}}{\pgfqpoint{1.984968in}{1.982695in}}{\pgfqpoint{1.993204in}{1.982695in}}%
\pgfpathclose%
\pgfusepath{stroke,fill}%
\end{pgfscope}%
\begin{pgfscope}%
\pgfpathrectangle{\pgfqpoint{0.100000in}{0.212622in}}{\pgfqpoint{3.696000in}{3.696000in}}%
\pgfusepath{clip}%
\pgfsetbuttcap%
\pgfsetroundjoin%
\definecolor{currentfill}{rgb}{0.121569,0.466667,0.705882}%
\pgfsetfillcolor{currentfill}%
\pgfsetfillopacity{0.359564}%
\pgfsetlinewidth{1.003750pt}%
\definecolor{currentstroke}{rgb}{0.121569,0.466667,0.705882}%
\pgfsetstrokecolor{currentstroke}%
\pgfsetstrokeopacity{0.359564}%
\pgfsetdash{}{0pt}%
\pgfpathmoveto{\pgfqpoint{1.748243in}{1.919726in}}%
\pgfpathcurveto{\pgfqpoint{1.756480in}{1.919726in}}{\pgfqpoint{1.764380in}{1.922998in}}{\pgfqpoint{1.770204in}{1.928822in}}%
\pgfpathcurveto{\pgfqpoint{1.776028in}{1.934646in}}{\pgfqpoint{1.779300in}{1.942546in}}{\pgfqpoint{1.779300in}{1.950782in}}%
\pgfpathcurveto{\pgfqpoint{1.779300in}{1.959019in}}{\pgfqpoint{1.776028in}{1.966919in}}{\pgfqpoint{1.770204in}{1.972743in}}%
\pgfpathcurveto{\pgfqpoint{1.764380in}{1.978567in}}{\pgfqpoint{1.756480in}{1.981839in}}{\pgfqpoint{1.748243in}{1.981839in}}%
\pgfpathcurveto{\pgfqpoint{1.740007in}{1.981839in}}{\pgfqpoint{1.732107in}{1.978567in}}{\pgfqpoint{1.726283in}{1.972743in}}%
\pgfpathcurveto{\pgfqpoint{1.720459in}{1.966919in}}{\pgfqpoint{1.717187in}{1.959019in}}{\pgfqpoint{1.717187in}{1.950782in}}%
\pgfpathcurveto{\pgfqpoint{1.717187in}{1.942546in}}{\pgfqpoint{1.720459in}{1.934646in}}{\pgfqpoint{1.726283in}{1.928822in}}%
\pgfpathcurveto{\pgfqpoint{1.732107in}{1.922998in}}{\pgfqpoint{1.740007in}{1.919726in}}{\pgfqpoint{1.748243in}{1.919726in}}%
\pgfpathclose%
\pgfusepath{stroke,fill}%
\end{pgfscope}%
\begin{pgfscope}%
\pgfpathrectangle{\pgfqpoint{0.100000in}{0.212622in}}{\pgfqpoint{3.696000in}{3.696000in}}%
\pgfusepath{clip}%
\pgfsetbuttcap%
\pgfsetroundjoin%
\definecolor{currentfill}{rgb}{0.121569,0.466667,0.705882}%
\pgfsetfillcolor{currentfill}%
\pgfsetfillopacity{0.360928}%
\pgfsetlinewidth{1.003750pt}%
\definecolor{currentstroke}{rgb}{0.121569,0.466667,0.705882}%
\pgfsetstrokecolor{currentstroke}%
\pgfsetstrokeopacity{0.360928}%
\pgfsetdash{}{0pt}%
\pgfpathmoveto{\pgfqpoint{1.744053in}{1.917869in}}%
\pgfpathcurveto{\pgfqpoint{1.752289in}{1.917869in}}{\pgfqpoint{1.760189in}{1.921141in}}{\pgfqpoint{1.766013in}{1.926965in}}%
\pgfpathcurveto{\pgfqpoint{1.771837in}{1.932789in}}{\pgfqpoint{1.775109in}{1.940689in}}{\pgfqpoint{1.775109in}{1.948926in}}%
\pgfpathcurveto{\pgfqpoint{1.775109in}{1.957162in}}{\pgfqpoint{1.771837in}{1.965062in}}{\pgfqpoint{1.766013in}{1.970886in}}%
\pgfpathcurveto{\pgfqpoint{1.760189in}{1.976710in}}{\pgfqpoint{1.752289in}{1.979982in}}{\pgfqpoint{1.744053in}{1.979982in}}%
\pgfpathcurveto{\pgfqpoint{1.735816in}{1.979982in}}{\pgfqpoint{1.727916in}{1.976710in}}{\pgfqpoint{1.722092in}{1.970886in}}%
\pgfpathcurveto{\pgfqpoint{1.716268in}{1.965062in}}{\pgfqpoint{1.712996in}{1.957162in}}{\pgfqpoint{1.712996in}{1.948926in}}%
\pgfpathcurveto{\pgfqpoint{1.712996in}{1.940689in}}{\pgfqpoint{1.716268in}{1.932789in}}{\pgfqpoint{1.722092in}{1.926965in}}%
\pgfpathcurveto{\pgfqpoint{1.727916in}{1.921141in}}{\pgfqpoint{1.735816in}{1.917869in}}{\pgfqpoint{1.744053in}{1.917869in}}%
\pgfpathclose%
\pgfusepath{stroke,fill}%
\end{pgfscope}%
\begin{pgfscope}%
\pgfpathrectangle{\pgfqpoint{0.100000in}{0.212622in}}{\pgfqpoint{3.696000in}{3.696000in}}%
\pgfusepath{clip}%
\pgfsetbuttcap%
\pgfsetroundjoin%
\definecolor{currentfill}{rgb}{0.121569,0.466667,0.705882}%
\pgfsetfillcolor{currentfill}%
\pgfsetfillopacity{0.363559}%
\pgfsetlinewidth{1.003750pt}%
\definecolor{currentstroke}{rgb}{0.121569,0.466667,0.705882}%
\pgfsetstrokecolor{currentstroke}%
\pgfsetstrokeopacity{0.363559}%
\pgfsetdash{}{0pt}%
\pgfpathmoveto{\pgfqpoint{1.736954in}{1.914758in}}%
\pgfpathcurveto{\pgfqpoint{1.745190in}{1.914758in}}{\pgfqpoint{1.753090in}{1.918031in}}{\pgfqpoint{1.758914in}{1.923855in}}%
\pgfpathcurveto{\pgfqpoint{1.764738in}{1.929679in}}{\pgfqpoint{1.768010in}{1.937579in}}{\pgfqpoint{1.768010in}{1.945815in}}%
\pgfpathcurveto{\pgfqpoint{1.768010in}{1.954051in}}{\pgfqpoint{1.764738in}{1.961951in}}{\pgfqpoint{1.758914in}{1.967775in}}%
\pgfpathcurveto{\pgfqpoint{1.753090in}{1.973599in}}{\pgfqpoint{1.745190in}{1.976871in}}{\pgfqpoint{1.736954in}{1.976871in}}%
\pgfpathcurveto{\pgfqpoint{1.728717in}{1.976871in}}{\pgfqpoint{1.720817in}{1.973599in}}{\pgfqpoint{1.714993in}{1.967775in}}%
\pgfpathcurveto{\pgfqpoint{1.709169in}{1.961951in}}{\pgfqpoint{1.705897in}{1.954051in}}{\pgfqpoint{1.705897in}{1.945815in}}%
\pgfpathcurveto{\pgfqpoint{1.705897in}{1.937579in}}{\pgfqpoint{1.709169in}{1.929679in}}{\pgfqpoint{1.714993in}{1.923855in}}%
\pgfpathcurveto{\pgfqpoint{1.720817in}{1.918031in}}{\pgfqpoint{1.728717in}{1.914758in}}{\pgfqpoint{1.736954in}{1.914758in}}%
\pgfpathclose%
\pgfusepath{stroke,fill}%
\end{pgfscope}%
\begin{pgfscope}%
\pgfpathrectangle{\pgfqpoint{0.100000in}{0.212622in}}{\pgfqpoint{3.696000in}{3.696000in}}%
\pgfusepath{clip}%
\pgfsetbuttcap%
\pgfsetroundjoin%
\definecolor{currentfill}{rgb}{0.121569,0.466667,0.705882}%
\pgfsetfillcolor{currentfill}%
\pgfsetfillopacity{0.364230}%
\pgfsetlinewidth{1.003750pt}%
\definecolor{currentstroke}{rgb}{0.121569,0.466667,0.705882}%
\pgfsetstrokecolor{currentstroke}%
\pgfsetstrokeopacity{0.364230}%
\pgfsetdash{}{0pt}%
\pgfpathmoveto{\pgfqpoint{1.997161in}{1.978281in}}%
\pgfpathcurveto{\pgfqpoint{2.005397in}{1.978281in}}{\pgfqpoint{2.013297in}{1.981553in}}{\pgfqpoint{2.019121in}{1.987377in}}%
\pgfpathcurveto{\pgfqpoint{2.024945in}{1.993201in}}{\pgfqpoint{2.028217in}{2.001101in}}{\pgfqpoint{2.028217in}{2.009337in}}%
\pgfpathcurveto{\pgfqpoint{2.028217in}{2.017574in}}{\pgfqpoint{2.024945in}{2.025474in}}{\pgfqpoint{2.019121in}{2.031298in}}%
\pgfpathcurveto{\pgfqpoint{2.013297in}{2.037122in}}{\pgfqpoint{2.005397in}{2.040394in}}{\pgfqpoint{1.997161in}{2.040394in}}%
\pgfpathcurveto{\pgfqpoint{1.988924in}{2.040394in}}{\pgfqpoint{1.981024in}{2.037122in}}{\pgfqpoint{1.975200in}{2.031298in}}%
\pgfpathcurveto{\pgfqpoint{1.969377in}{2.025474in}}{\pgfqpoint{1.966104in}{2.017574in}}{\pgfqpoint{1.966104in}{2.009337in}}%
\pgfpathcurveto{\pgfqpoint{1.966104in}{2.001101in}}{\pgfqpoint{1.969377in}{1.993201in}}{\pgfqpoint{1.975200in}{1.987377in}}%
\pgfpathcurveto{\pgfqpoint{1.981024in}{1.981553in}}{\pgfqpoint{1.988924in}{1.978281in}}{\pgfqpoint{1.997161in}{1.978281in}}%
\pgfpathclose%
\pgfusepath{stroke,fill}%
\end{pgfscope}%
\begin{pgfscope}%
\pgfpathrectangle{\pgfqpoint{0.100000in}{0.212622in}}{\pgfqpoint{3.696000in}{3.696000in}}%
\pgfusepath{clip}%
\pgfsetbuttcap%
\pgfsetroundjoin%
\definecolor{currentfill}{rgb}{0.121569,0.466667,0.705882}%
\pgfsetfillcolor{currentfill}%
\pgfsetfillopacity{0.365231}%
\pgfsetlinewidth{1.003750pt}%
\definecolor{currentstroke}{rgb}{0.121569,0.466667,0.705882}%
\pgfsetstrokecolor{currentstroke}%
\pgfsetstrokeopacity{0.365231}%
\pgfsetdash{}{0pt}%
\pgfpathmoveto{\pgfqpoint{1.730406in}{1.913023in}}%
\pgfpathcurveto{\pgfqpoint{1.738642in}{1.913023in}}{\pgfqpoint{1.746542in}{1.916295in}}{\pgfqpoint{1.752366in}{1.922119in}}%
\pgfpathcurveto{\pgfqpoint{1.758190in}{1.927943in}}{\pgfqpoint{1.761462in}{1.935843in}}{\pgfqpoint{1.761462in}{1.944079in}}%
\pgfpathcurveto{\pgfqpoint{1.761462in}{1.952315in}}{\pgfqpoint{1.758190in}{1.960215in}}{\pgfqpoint{1.752366in}{1.966039in}}%
\pgfpathcurveto{\pgfqpoint{1.746542in}{1.971863in}}{\pgfqpoint{1.738642in}{1.975136in}}{\pgfqpoint{1.730406in}{1.975136in}}%
\pgfpathcurveto{\pgfqpoint{1.722170in}{1.975136in}}{\pgfqpoint{1.714269in}{1.971863in}}{\pgfqpoint{1.708446in}{1.966039in}}%
\pgfpathcurveto{\pgfqpoint{1.702622in}{1.960215in}}{\pgfqpoint{1.699349in}{1.952315in}}{\pgfqpoint{1.699349in}{1.944079in}}%
\pgfpathcurveto{\pgfqpoint{1.699349in}{1.935843in}}{\pgfqpoint{1.702622in}{1.927943in}}{\pgfqpoint{1.708446in}{1.922119in}}%
\pgfpathcurveto{\pgfqpoint{1.714269in}{1.916295in}}{\pgfqpoint{1.722170in}{1.913023in}}{\pgfqpoint{1.730406in}{1.913023in}}%
\pgfpathclose%
\pgfusepath{stroke,fill}%
\end{pgfscope}%
\begin{pgfscope}%
\pgfpathrectangle{\pgfqpoint{0.100000in}{0.212622in}}{\pgfqpoint{3.696000in}{3.696000in}}%
\pgfusepath{clip}%
\pgfsetbuttcap%
\pgfsetroundjoin%
\definecolor{currentfill}{rgb}{0.121569,0.466667,0.705882}%
\pgfsetfillcolor{currentfill}%
\pgfsetfillopacity{0.366875}%
\pgfsetlinewidth{1.003750pt}%
\definecolor{currentstroke}{rgb}{0.121569,0.466667,0.705882}%
\pgfsetstrokecolor{currentstroke}%
\pgfsetstrokeopacity{0.366875}%
\pgfsetdash{}{0pt}%
\pgfpathmoveto{\pgfqpoint{1.725939in}{1.910716in}}%
\pgfpathcurveto{\pgfqpoint{1.734176in}{1.910716in}}{\pgfqpoint{1.742076in}{1.913988in}}{\pgfqpoint{1.747900in}{1.919812in}}%
\pgfpathcurveto{\pgfqpoint{1.753724in}{1.925636in}}{\pgfqpoint{1.756996in}{1.933536in}}{\pgfqpoint{1.756996in}{1.941772in}}%
\pgfpathcurveto{\pgfqpoint{1.756996in}{1.950008in}}{\pgfqpoint{1.753724in}{1.957908in}}{\pgfqpoint{1.747900in}{1.963732in}}%
\pgfpathcurveto{\pgfqpoint{1.742076in}{1.969556in}}{\pgfqpoint{1.734176in}{1.972829in}}{\pgfqpoint{1.725939in}{1.972829in}}%
\pgfpathcurveto{\pgfqpoint{1.717703in}{1.972829in}}{\pgfqpoint{1.709803in}{1.969556in}}{\pgfqpoint{1.703979in}{1.963732in}}%
\pgfpathcurveto{\pgfqpoint{1.698155in}{1.957908in}}{\pgfqpoint{1.694883in}{1.950008in}}{\pgfqpoint{1.694883in}{1.941772in}}%
\pgfpathcurveto{\pgfqpoint{1.694883in}{1.933536in}}{\pgfqpoint{1.698155in}{1.925636in}}{\pgfqpoint{1.703979in}{1.919812in}}%
\pgfpathcurveto{\pgfqpoint{1.709803in}{1.913988in}}{\pgfqpoint{1.717703in}{1.910716in}}{\pgfqpoint{1.725939in}{1.910716in}}%
\pgfpathclose%
\pgfusepath{stroke,fill}%
\end{pgfscope}%
\begin{pgfscope}%
\pgfpathrectangle{\pgfqpoint{0.100000in}{0.212622in}}{\pgfqpoint{3.696000in}{3.696000in}}%
\pgfusepath{clip}%
\pgfsetbuttcap%
\pgfsetroundjoin%
\definecolor{currentfill}{rgb}{0.121569,0.466667,0.705882}%
\pgfsetfillcolor{currentfill}%
\pgfsetfillopacity{0.369939}%
\pgfsetlinewidth{1.003750pt}%
\definecolor{currentstroke}{rgb}{0.121569,0.466667,0.705882}%
\pgfsetstrokecolor{currentstroke}%
\pgfsetstrokeopacity{0.369939}%
\pgfsetdash{}{0pt}%
\pgfpathmoveto{\pgfqpoint{1.717553in}{1.907363in}}%
\pgfpathcurveto{\pgfqpoint{1.725789in}{1.907363in}}{\pgfqpoint{1.733689in}{1.910635in}}{\pgfqpoint{1.739513in}{1.916459in}}%
\pgfpathcurveto{\pgfqpoint{1.745337in}{1.922283in}}{\pgfqpoint{1.748609in}{1.930183in}}{\pgfqpoint{1.748609in}{1.938420in}}%
\pgfpathcurveto{\pgfqpoint{1.748609in}{1.946656in}}{\pgfqpoint{1.745337in}{1.954556in}}{\pgfqpoint{1.739513in}{1.960380in}}%
\pgfpathcurveto{\pgfqpoint{1.733689in}{1.966204in}}{\pgfqpoint{1.725789in}{1.969476in}}{\pgfqpoint{1.717553in}{1.969476in}}%
\pgfpathcurveto{\pgfqpoint{1.709316in}{1.969476in}}{\pgfqpoint{1.701416in}{1.966204in}}{\pgfqpoint{1.695592in}{1.960380in}}%
\pgfpathcurveto{\pgfqpoint{1.689769in}{1.954556in}}{\pgfqpoint{1.686496in}{1.946656in}}{\pgfqpoint{1.686496in}{1.938420in}}%
\pgfpathcurveto{\pgfqpoint{1.686496in}{1.930183in}}{\pgfqpoint{1.689769in}{1.922283in}}{\pgfqpoint{1.695592in}{1.916459in}}%
\pgfpathcurveto{\pgfqpoint{1.701416in}{1.910635in}}{\pgfqpoint{1.709316in}{1.907363in}}{\pgfqpoint{1.717553in}{1.907363in}}%
\pgfpathclose%
\pgfusepath{stroke,fill}%
\end{pgfscope}%
\begin{pgfscope}%
\pgfpathrectangle{\pgfqpoint{0.100000in}{0.212622in}}{\pgfqpoint{3.696000in}{3.696000in}}%
\pgfusepath{clip}%
\pgfsetbuttcap%
\pgfsetroundjoin%
\definecolor{currentfill}{rgb}{0.121569,0.466667,0.705882}%
\pgfsetfillcolor{currentfill}%
\pgfsetfillopacity{0.371230}%
\pgfsetlinewidth{1.003750pt}%
\definecolor{currentstroke}{rgb}{0.121569,0.466667,0.705882}%
\pgfsetstrokecolor{currentstroke}%
\pgfsetstrokeopacity{0.371230}%
\pgfsetdash{}{0pt}%
\pgfpathmoveto{\pgfqpoint{1.997183in}{1.979752in}}%
\pgfpathcurveto{\pgfqpoint{2.005419in}{1.979752in}}{\pgfqpoint{2.013319in}{1.983024in}}{\pgfqpoint{2.019143in}{1.988848in}}%
\pgfpathcurveto{\pgfqpoint{2.024967in}{1.994672in}}{\pgfqpoint{2.028239in}{2.002572in}}{\pgfqpoint{2.028239in}{2.010809in}}%
\pgfpathcurveto{\pgfqpoint{2.028239in}{2.019045in}}{\pgfqpoint{2.024967in}{2.026945in}}{\pgfqpoint{2.019143in}{2.032769in}}%
\pgfpathcurveto{\pgfqpoint{2.013319in}{2.038593in}}{\pgfqpoint{2.005419in}{2.041865in}}{\pgfqpoint{1.997183in}{2.041865in}}%
\pgfpathcurveto{\pgfqpoint{1.988947in}{2.041865in}}{\pgfqpoint{1.981047in}{2.038593in}}{\pgfqpoint{1.975223in}{2.032769in}}%
\pgfpathcurveto{\pgfqpoint{1.969399in}{2.026945in}}{\pgfqpoint{1.966126in}{2.019045in}}{\pgfqpoint{1.966126in}{2.010809in}}%
\pgfpathcurveto{\pgfqpoint{1.966126in}{2.002572in}}{\pgfqpoint{1.969399in}{1.994672in}}{\pgfqpoint{1.975223in}{1.988848in}}%
\pgfpathcurveto{\pgfqpoint{1.981047in}{1.983024in}}{\pgfqpoint{1.988947in}{1.979752in}}{\pgfqpoint{1.997183in}{1.979752in}}%
\pgfpathclose%
\pgfusepath{stroke,fill}%
\end{pgfscope}%
\begin{pgfscope}%
\pgfpathrectangle{\pgfqpoint{0.100000in}{0.212622in}}{\pgfqpoint{3.696000in}{3.696000in}}%
\pgfusepath{clip}%
\pgfsetbuttcap%
\pgfsetroundjoin%
\definecolor{currentfill}{rgb}{0.121569,0.466667,0.705882}%
\pgfsetfillcolor{currentfill}%
\pgfsetfillopacity{0.372168}%
\pgfsetlinewidth{1.003750pt}%
\definecolor{currentstroke}{rgb}{0.121569,0.466667,0.705882}%
\pgfsetstrokecolor{currentstroke}%
\pgfsetstrokeopacity{0.372168}%
\pgfsetdash{}{0pt}%
\pgfpathmoveto{\pgfqpoint{1.708694in}{1.901326in}}%
\pgfpathcurveto{\pgfqpoint{1.716930in}{1.901326in}}{\pgfqpoint{1.724830in}{1.904598in}}{\pgfqpoint{1.730654in}{1.910422in}}%
\pgfpathcurveto{\pgfqpoint{1.736478in}{1.916246in}}{\pgfqpoint{1.739750in}{1.924146in}}{\pgfqpoint{1.739750in}{1.932382in}}%
\pgfpathcurveto{\pgfqpoint{1.739750in}{1.940619in}}{\pgfqpoint{1.736478in}{1.948519in}}{\pgfqpoint{1.730654in}{1.954343in}}%
\pgfpathcurveto{\pgfqpoint{1.724830in}{1.960167in}}{\pgfqpoint{1.716930in}{1.963439in}}{\pgfqpoint{1.708694in}{1.963439in}}%
\pgfpathcurveto{\pgfqpoint{1.700458in}{1.963439in}}{\pgfqpoint{1.692557in}{1.960167in}}{\pgfqpoint{1.686734in}{1.954343in}}%
\pgfpathcurveto{\pgfqpoint{1.680910in}{1.948519in}}{\pgfqpoint{1.677637in}{1.940619in}}{\pgfqpoint{1.677637in}{1.932382in}}%
\pgfpathcurveto{\pgfqpoint{1.677637in}{1.924146in}}{\pgfqpoint{1.680910in}{1.916246in}}{\pgfqpoint{1.686734in}{1.910422in}}%
\pgfpathcurveto{\pgfqpoint{1.692557in}{1.904598in}}{\pgfqpoint{1.700458in}{1.901326in}}{\pgfqpoint{1.708694in}{1.901326in}}%
\pgfpathclose%
\pgfusepath{stroke,fill}%
\end{pgfscope}%
\begin{pgfscope}%
\pgfpathrectangle{\pgfqpoint{0.100000in}{0.212622in}}{\pgfqpoint{3.696000in}{3.696000in}}%
\pgfusepath{clip}%
\pgfsetbuttcap%
\pgfsetroundjoin%
\definecolor{currentfill}{rgb}{0.121569,0.466667,0.705882}%
\pgfsetfillcolor{currentfill}%
\pgfsetfillopacity{0.375488}%
\pgfsetlinewidth{1.003750pt}%
\definecolor{currentstroke}{rgb}{0.121569,0.466667,0.705882}%
\pgfsetstrokecolor{currentstroke}%
\pgfsetstrokeopacity{0.375488}%
\pgfsetdash{}{0pt}%
\pgfpathmoveto{\pgfqpoint{1.702842in}{1.903361in}}%
\pgfpathcurveto{\pgfqpoint{1.711078in}{1.903361in}}{\pgfqpoint{1.718979in}{1.906633in}}{\pgfqpoint{1.724802in}{1.912457in}}%
\pgfpathcurveto{\pgfqpoint{1.730626in}{1.918281in}}{\pgfqpoint{1.733899in}{1.926181in}}{\pgfqpoint{1.733899in}{1.934417in}}%
\pgfpathcurveto{\pgfqpoint{1.733899in}{1.942653in}}{\pgfqpoint{1.730626in}{1.950554in}}{\pgfqpoint{1.724802in}{1.956377in}}%
\pgfpathcurveto{\pgfqpoint{1.718979in}{1.962201in}}{\pgfqpoint{1.711078in}{1.965474in}}{\pgfqpoint{1.702842in}{1.965474in}}%
\pgfpathcurveto{\pgfqpoint{1.694606in}{1.965474in}}{\pgfqpoint{1.686706in}{1.962201in}}{\pgfqpoint{1.680882in}{1.956377in}}%
\pgfpathcurveto{\pgfqpoint{1.675058in}{1.950554in}}{\pgfqpoint{1.671786in}{1.942653in}}{\pgfqpoint{1.671786in}{1.934417in}}%
\pgfpathcurveto{\pgfqpoint{1.671786in}{1.926181in}}{\pgfqpoint{1.675058in}{1.918281in}}{\pgfqpoint{1.680882in}{1.912457in}}%
\pgfpathcurveto{\pgfqpoint{1.686706in}{1.906633in}}{\pgfqpoint{1.694606in}{1.903361in}}{\pgfqpoint{1.702842in}{1.903361in}}%
\pgfpathclose%
\pgfusepath{stroke,fill}%
\end{pgfscope}%
\begin{pgfscope}%
\pgfpathrectangle{\pgfqpoint{0.100000in}{0.212622in}}{\pgfqpoint{3.696000in}{3.696000in}}%
\pgfusepath{clip}%
\pgfsetbuttcap%
\pgfsetroundjoin%
\definecolor{currentfill}{rgb}{0.121569,0.466667,0.705882}%
\pgfsetfillcolor{currentfill}%
\pgfsetfillopacity{0.376725}%
\pgfsetlinewidth{1.003750pt}%
\definecolor{currentstroke}{rgb}{0.121569,0.466667,0.705882}%
\pgfsetstrokecolor{currentstroke}%
\pgfsetstrokeopacity{0.376725}%
\pgfsetdash{}{0pt}%
\pgfpathmoveto{\pgfqpoint{1.998458in}{1.969587in}}%
\pgfpathcurveto{\pgfqpoint{2.006694in}{1.969587in}}{\pgfqpoint{2.014594in}{1.972860in}}{\pgfqpoint{2.020418in}{1.978684in}}%
\pgfpathcurveto{\pgfqpoint{2.026242in}{1.984508in}}{\pgfqpoint{2.029514in}{1.992408in}}{\pgfqpoint{2.029514in}{2.000644in}}%
\pgfpathcurveto{\pgfqpoint{2.029514in}{2.008880in}}{\pgfqpoint{2.026242in}{2.016780in}}{\pgfqpoint{2.020418in}{2.022604in}}%
\pgfpathcurveto{\pgfqpoint{2.014594in}{2.028428in}}{\pgfqpoint{2.006694in}{2.031700in}}{\pgfqpoint{1.998458in}{2.031700in}}%
\pgfpathcurveto{\pgfqpoint{1.990221in}{2.031700in}}{\pgfqpoint{1.982321in}{2.028428in}}{\pgfqpoint{1.976497in}{2.022604in}}%
\pgfpathcurveto{\pgfqpoint{1.970674in}{2.016780in}}{\pgfqpoint{1.967401in}{2.008880in}}{\pgfqpoint{1.967401in}{2.000644in}}%
\pgfpathcurveto{\pgfqpoint{1.967401in}{1.992408in}}{\pgfqpoint{1.970674in}{1.984508in}}{\pgfqpoint{1.976497in}{1.978684in}}%
\pgfpathcurveto{\pgfqpoint{1.982321in}{1.972860in}}{\pgfqpoint{1.990221in}{1.969587in}}{\pgfqpoint{1.998458in}{1.969587in}}%
\pgfpathclose%
\pgfusepath{stroke,fill}%
\end{pgfscope}%
\begin{pgfscope}%
\pgfpathrectangle{\pgfqpoint{0.100000in}{0.212622in}}{\pgfqpoint{3.696000in}{3.696000in}}%
\pgfusepath{clip}%
\pgfsetbuttcap%
\pgfsetroundjoin%
\definecolor{currentfill}{rgb}{0.121569,0.466667,0.705882}%
\pgfsetfillcolor{currentfill}%
\pgfsetfillopacity{0.377108}%
\pgfsetlinewidth{1.003750pt}%
\definecolor{currentstroke}{rgb}{0.121569,0.466667,0.705882}%
\pgfsetstrokecolor{currentstroke}%
\pgfsetstrokeopacity{0.377108}%
\pgfsetdash{}{0pt}%
\pgfpathmoveto{\pgfqpoint{1.695785in}{1.896798in}}%
\pgfpathcurveto{\pgfqpoint{1.704022in}{1.896798in}}{\pgfqpoint{1.711922in}{1.900070in}}{\pgfqpoint{1.717746in}{1.905894in}}%
\pgfpathcurveto{\pgfqpoint{1.723570in}{1.911718in}}{\pgfqpoint{1.726842in}{1.919618in}}{\pgfqpoint{1.726842in}{1.927854in}}%
\pgfpathcurveto{\pgfqpoint{1.726842in}{1.936091in}}{\pgfqpoint{1.723570in}{1.943991in}}{\pgfqpoint{1.717746in}{1.949815in}}%
\pgfpathcurveto{\pgfqpoint{1.711922in}{1.955639in}}{\pgfqpoint{1.704022in}{1.958911in}}{\pgfqpoint{1.695785in}{1.958911in}}%
\pgfpathcurveto{\pgfqpoint{1.687549in}{1.958911in}}{\pgfqpoint{1.679649in}{1.955639in}}{\pgfqpoint{1.673825in}{1.949815in}}%
\pgfpathcurveto{\pgfqpoint{1.668001in}{1.943991in}}{\pgfqpoint{1.664729in}{1.936091in}}{\pgfqpoint{1.664729in}{1.927854in}}%
\pgfpathcurveto{\pgfqpoint{1.664729in}{1.919618in}}{\pgfqpoint{1.668001in}{1.911718in}}{\pgfqpoint{1.673825in}{1.905894in}}%
\pgfpathcurveto{\pgfqpoint{1.679649in}{1.900070in}}{\pgfqpoint{1.687549in}{1.896798in}}{\pgfqpoint{1.695785in}{1.896798in}}%
\pgfpathclose%
\pgfusepath{stroke,fill}%
\end{pgfscope}%
\begin{pgfscope}%
\pgfpathrectangle{\pgfqpoint{0.100000in}{0.212622in}}{\pgfqpoint{3.696000in}{3.696000in}}%
\pgfusepath{clip}%
\pgfsetbuttcap%
\pgfsetroundjoin%
\definecolor{currentfill}{rgb}{0.121569,0.466667,0.705882}%
\pgfsetfillcolor{currentfill}%
\pgfsetfillopacity{0.378796}%
\pgfsetlinewidth{1.003750pt}%
\definecolor{currentstroke}{rgb}{0.121569,0.466667,0.705882}%
\pgfsetstrokecolor{currentstroke}%
\pgfsetstrokeopacity{0.378796}%
\pgfsetdash{}{0pt}%
\pgfpathmoveto{\pgfqpoint{1.689212in}{1.893132in}}%
\pgfpathcurveto{\pgfqpoint{1.697448in}{1.893132in}}{\pgfqpoint{1.705348in}{1.896404in}}{\pgfqpoint{1.711172in}{1.902228in}}%
\pgfpathcurveto{\pgfqpoint{1.716996in}{1.908052in}}{\pgfqpoint{1.720268in}{1.915952in}}{\pgfqpoint{1.720268in}{1.924188in}}%
\pgfpathcurveto{\pgfqpoint{1.720268in}{1.932425in}}{\pgfqpoint{1.716996in}{1.940325in}}{\pgfqpoint{1.711172in}{1.946149in}}%
\pgfpathcurveto{\pgfqpoint{1.705348in}{1.951972in}}{\pgfqpoint{1.697448in}{1.955245in}}{\pgfqpoint{1.689212in}{1.955245in}}%
\pgfpathcurveto{\pgfqpoint{1.680976in}{1.955245in}}{\pgfqpoint{1.673075in}{1.951972in}}{\pgfqpoint{1.667252in}{1.946149in}}%
\pgfpathcurveto{\pgfqpoint{1.661428in}{1.940325in}}{\pgfqpoint{1.658155in}{1.932425in}}{\pgfqpoint{1.658155in}{1.924188in}}%
\pgfpathcurveto{\pgfqpoint{1.658155in}{1.915952in}}{\pgfqpoint{1.661428in}{1.908052in}}{\pgfqpoint{1.667252in}{1.902228in}}%
\pgfpathcurveto{\pgfqpoint{1.673075in}{1.896404in}}{\pgfqpoint{1.680976in}{1.893132in}}{\pgfqpoint{1.689212in}{1.893132in}}%
\pgfpathclose%
\pgfusepath{stroke,fill}%
\end{pgfscope}%
\begin{pgfscope}%
\pgfpathrectangle{\pgfqpoint{0.100000in}{0.212622in}}{\pgfqpoint{3.696000in}{3.696000in}}%
\pgfusepath{clip}%
\pgfsetbuttcap%
\pgfsetroundjoin%
\definecolor{currentfill}{rgb}{0.121569,0.466667,0.705882}%
\pgfsetfillcolor{currentfill}%
\pgfsetfillopacity{0.380995}%
\pgfsetlinewidth{1.003750pt}%
\definecolor{currentstroke}{rgb}{0.121569,0.466667,0.705882}%
\pgfsetstrokecolor{currentstroke}%
\pgfsetstrokeopacity{0.380995}%
\pgfsetdash{}{0pt}%
\pgfpathmoveto{\pgfqpoint{1.685058in}{1.892693in}}%
\pgfpathcurveto{\pgfqpoint{1.693294in}{1.892693in}}{\pgfqpoint{1.701194in}{1.895966in}}{\pgfqpoint{1.707018in}{1.901790in}}%
\pgfpathcurveto{\pgfqpoint{1.712842in}{1.907614in}}{\pgfqpoint{1.716114in}{1.915514in}}{\pgfqpoint{1.716114in}{1.923750in}}%
\pgfpathcurveto{\pgfqpoint{1.716114in}{1.931986in}}{\pgfqpoint{1.712842in}{1.939886in}}{\pgfqpoint{1.707018in}{1.945710in}}%
\pgfpathcurveto{\pgfqpoint{1.701194in}{1.951534in}}{\pgfqpoint{1.693294in}{1.954806in}}{\pgfqpoint{1.685058in}{1.954806in}}%
\pgfpathcurveto{\pgfqpoint{1.676821in}{1.954806in}}{\pgfqpoint{1.668921in}{1.951534in}}{\pgfqpoint{1.663097in}{1.945710in}}%
\pgfpathcurveto{\pgfqpoint{1.657273in}{1.939886in}}{\pgfqpoint{1.654001in}{1.931986in}}{\pgfqpoint{1.654001in}{1.923750in}}%
\pgfpathcurveto{\pgfqpoint{1.654001in}{1.915514in}}{\pgfqpoint{1.657273in}{1.907614in}}{\pgfqpoint{1.663097in}{1.901790in}}%
\pgfpathcurveto{\pgfqpoint{1.668921in}{1.895966in}}{\pgfqpoint{1.676821in}{1.892693in}}{\pgfqpoint{1.685058in}{1.892693in}}%
\pgfpathclose%
\pgfusepath{stroke,fill}%
\end{pgfscope}%
\begin{pgfscope}%
\pgfpathrectangle{\pgfqpoint{0.100000in}{0.212622in}}{\pgfqpoint{3.696000in}{3.696000in}}%
\pgfusepath{clip}%
\pgfsetbuttcap%
\pgfsetroundjoin%
\definecolor{currentfill}{rgb}{0.121569,0.466667,0.705882}%
\pgfsetfillcolor{currentfill}%
\pgfsetfillopacity{0.381659}%
\pgfsetlinewidth{1.003750pt}%
\definecolor{currentstroke}{rgb}{0.121569,0.466667,0.705882}%
\pgfsetstrokecolor{currentstroke}%
\pgfsetstrokeopacity{0.381659}%
\pgfsetdash{}{0pt}%
\pgfpathmoveto{\pgfqpoint{1.681783in}{1.889467in}}%
\pgfpathcurveto{\pgfqpoint{1.690020in}{1.889467in}}{\pgfqpoint{1.697920in}{1.892739in}}{\pgfqpoint{1.703744in}{1.898563in}}%
\pgfpathcurveto{\pgfqpoint{1.709567in}{1.904387in}}{\pgfqpoint{1.712840in}{1.912287in}}{\pgfqpoint{1.712840in}{1.920524in}}%
\pgfpathcurveto{\pgfqpoint{1.712840in}{1.928760in}}{\pgfqpoint{1.709567in}{1.936660in}}{\pgfqpoint{1.703744in}{1.942484in}}%
\pgfpathcurveto{\pgfqpoint{1.697920in}{1.948308in}}{\pgfqpoint{1.690020in}{1.951580in}}{\pgfqpoint{1.681783in}{1.951580in}}%
\pgfpathcurveto{\pgfqpoint{1.673547in}{1.951580in}}{\pgfqpoint{1.665647in}{1.948308in}}{\pgfqpoint{1.659823in}{1.942484in}}%
\pgfpathcurveto{\pgfqpoint{1.653999in}{1.936660in}}{\pgfqpoint{1.650727in}{1.928760in}}{\pgfqpoint{1.650727in}{1.920524in}}%
\pgfpathcurveto{\pgfqpoint{1.650727in}{1.912287in}}{\pgfqpoint{1.653999in}{1.904387in}}{\pgfqpoint{1.659823in}{1.898563in}}%
\pgfpathcurveto{\pgfqpoint{1.665647in}{1.892739in}}{\pgfqpoint{1.673547in}{1.889467in}}{\pgfqpoint{1.681783in}{1.889467in}}%
\pgfpathclose%
\pgfusepath{stroke,fill}%
\end{pgfscope}%
\begin{pgfscope}%
\pgfpathrectangle{\pgfqpoint{0.100000in}{0.212622in}}{\pgfqpoint{3.696000in}{3.696000in}}%
\pgfusepath{clip}%
\pgfsetbuttcap%
\pgfsetroundjoin%
\definecolor{currentfill}{rgb}{0.121569,0.466667,0.705882}%
\pgfsetfillcolor{currentfill}%
\pgfsetfillopacity{0.382313}%
\pgfsetlinewidth{1.003750pt}%
\definecolor{currentstroke}{rgb}{0.121569,0.466667,0.705882}%
\pgfsetstrokecolor{currentstroke}%
\pgfsetstrokeopacity{0.382313}%
\pgfsetdash{}{0pt}%
\pgfpathmoveto{\pgfqpoint{1.679244in}{1.887247in}}%
\pgfpathcurveto{\pgfqpoint{1.687480in}{1.887247in}}{\pgfqpoint{1.695380in}{1.890520in}}{\pgfqpoint{1.701204in}{1.896344in}}%
\pgfpathcurveto{\pgfqpoint{1.707028in}{1.902168in}}{\pgfqpoint{1.710300in}{1.910068in}}{\pgfqpoint{1.710300in}{1.918304in}}%
\pgfpathcurveto{\pgfqpoint{1.710300in}{1.926540in}}{\pgfqpoint{1.707028in}{1.934440in}}{\pgfqpoint{1.701204in}{1.940264in}}%
\pgfpathcurveto{\pgfqpoint{1.695380in}{1.946088in}}{\pgfqpoint{1.687480in}{1.949360in}}{\pgfqpoint{1.679244in}{1.949360in}}%
\pgfpathcurveto{\pgfqpoint{1.671007in}{1.949360in}}{\pgfqpoint{1.663107in}{1.946088in}}{\pgfqpoint{1.657283in}{1.940264in}}%
\pgfpathcurveto{\pgfqpoint{1.651460in}{1.934440in}}{\pgfqpoint{1.648187in}{1.926540in}}{\pgfqpoint{1.648187in}{1.918304in}}%
\pgfpathcurveto{\pgfqpoint{1.648187in}{1.910068in}}{\pgfqpoint{1.651460in}{1.902168in}}{\pgfqpoint{1.657283in}{1.896344in}}%
\pgfpathcurveto{\pgfqpoint{1.663107in}{1.890520in}}{\pgfqpoint{1.671007in}{1.887247in}}{\pgfqpoint{1.679244in}{1.887247in}}%
\pgfpathclose%
\pgfusepath{stroke,fill}%
\end{pgfscope}%
\begin{pgfscope}%
\pgfpathrectangle{\pgfqpoint{0.100000in}{0.212622in}}{\pgfqpoint{3.696000in}{3.696000in}}%
\pgfusepath{clip}%
\pgfsetbuttcap%
\pgfsetroundjoin%
\definecolor{currentfill}{rgb}{0.121569,0.466667,0.705882}%
\pgfsetfillcolor{currentfill}%
\pgfsetfillopacity{0.383481}%
\pgfsetlinewidth{1.003750pt}%
\definecolor{currentstroke}{rgb}{0.121569,0.466667,0.705882}%
\pgfsetstrokecolor{currentstroke}%
\pgfsetstrokeopacity{0.383481}%
\pgfsetdash{}{0pt}%
\pgfpathmoveto{\pgfqpoint{2.000258in}{1.961818in}}%
\pgfpathcurveto{\pgfqpoint{2.008494in}{1.961818in}}{\pgfqpoint{2.016394in}{1.965090in}}{\pgfqpoint{2.022218in}{1.970914in}}%
\pgfpathcurveto{\pgfqpoint{2.028042in}{1.976738in}}{\pgfqpoint{2.031314in}{1.984638in}}{\pgfqpoint{2.031314in}{1.992874in}}%
\pgfpathcurveto{\pgfqpoint{2.031314in}{2.001111in}}{\pgfqpoint{2.028042in}{2.009011in}}{\pgfqpoint{2.022218in}{2.014835in}}%
\pgfpathcurveto{\pgfqpoint{2.016394in}{2.020659in}}{\pgfqpoint{2.008494in}{2.023931in}}{\pgfqpoint{2.000258in}{2.023931in}}%
\pgfpathcurveto{\pgfqpoint{1.992022in}{2.023931in}}{\pgfqpoint{1.984122in}{2.020659in}}{\pgfqpoint{1.978298in}{2.014835in}}%
\pgfpathcurveto{\pgfqpoint{1.972474in}{2.009011in}}{\pgfqpoint{1.969201in}{2.001111in}}{\pgfqpoint{1.969201in}{1.992874in}}%
\pgfpathcurveto{\pgfqpoint{1.969201in}{1.984638in}}{\pgfqpoint{1.972474in}{1.976738in}}{\pgfqpoint{1.978298in}{1.970914in}}%
\pgfpathcurveto{\pgfqpoint{1.984122in}{1.965090in}}{\pgfqpoint{1.992022in}{1.961818in}}{\pgfqpoint{2.000258in}{1.961818in}}%
\pgfpathclose%
\pgfusepath{stroke,fill}%
\end{pgfscope}%
\begin{pgfscope}%
\pgfpathrectangle{\pgfqpoint{0.100000in}{0.212622in}}{\pgfqpoint{3.696000in}{3.696000in}}%
\pgfusepath{clip}%
\pgfsetbuttcap%
\pgfsetroundjoin%
\definecolor{currentfill}{rgb}{0.121569,0.466667,0.705882}%
\pgfsetfillcolor{currentfill}%
\pgfsetfillopacity{0.384153}%
\pgfsetlinewidth{1.003750pt}%
\definecolor{currentstroke}{rgb}{0.121569,0.466667,0.705882}%
\pgfsetstrokecolor{currentstroke}%
\pgfsetstrokeopacity{0.384153}%
\pgfsetdash{}{0pt}%
\pgfpathmoveto{\pgfqpoint{1.675596in}{1.886375in}}%
\pgfpathcurveto{\pgfqpoint{1.683832in}{1.886375in}}{\pgfqpoint{1.691732in}{1.889647in}}{\pgfqpoint{1.697556in}{1.895471in}}%
\pgfpathcurveto{\pgfqpoint{1.703380in}{1.901295in}}{\pgfqpoint{1.706652in}{1.909195in}}{\pgfqpoint{1.706652in}{1.917431in}}%
\pgfpathcurveto{\pgfqpoint{1.706652in}{1.925667in}}{\pgfqpoint{1.703380in}{1.933567in}}{\pgfqpoint{1.697556in}{1.939391in}}%
\pgfpathcurveto{\pgfqpoint{1.691732in}{1.945215in}}{\pgfqpoint{1.683832in}{1.948488in}}{\pgfqpoint{1.675596in}{1.948488in}}%
\pgfpathcurveto{\pgfqpoint{1.667359in}{1.948488in}}{\pgfqpoint{1.659459in}{1.945215in}}{\pgfqpoint{1.653635in}{1.939391in}}%
\pgfpathcurveto{\pgfqpoint{1.647811in}{1.933567in}}{\pgfqpoint{1.644539in}{1.925667in}}{\pgfqpoint{1.644539in}{1.917431in}}%
\pgfpathcurveto{\pgfqpoint{1.644539in}{1.909195in}}{\pgfqpoint{1.647811in}{1.901295in}}{\pgfqpoint{1.653635in}{1.895471in}}%
\pgfpathcurveto{\pgfqpoint{1.659459in}{1.889647in}}{\pgfqpoint{1.667359in}{1.886375in}}{\pgfqpoint{1.675596in}{1.886375in}}%
\pgfpathclose%
\pgfusepath{stroke,fill}%
\end{pgfscope}%
\begin{pgfscope}%
\pgfpathrectangle{\pgfqpoint{0.100000in}{0.212622in}}{\pgfqpoint{3.696000in}{3.696000in}}%
\pgfusepath{clip}%
\pgfsetbuttcap%
\pgfsetroundjoin%
\definecolor{currentfill}{rgb}{0.121569,0.466667,0.705882}%
\pgfsetfillcolor{currentfill}%
\pgfsetfillopacity{0.384588}%
\pgfsetlinewidth{1.003750pt}%
\definecolor{currentstroke}{rgb}{0.121569,0.466667,0.705882}%
\pgfsetstrokecolor{currentstroke}%
\pgfsetstrokeopacity{0.384588}%
\pgfsetdash{}{0pt}%
\pgfpathmoveto{\pgfqpoint{1.673323in}{1.884886in}}%
\pgfpathcurveto{\pgfqpoint{1.681559in}{1.884886in}}{\pgfqpoint{1.689459in}{1.888159in}}{\pgfqpoint{1.695283in}{1.893983in}}%
\pgfpathcurveto{\pgfqpoint{1.701107in}{1.899807in}}{\pgfqpoint{1.704379in}{1.907707in}}{\pgfqpoint{1.704379in}{1.915943in}}%
\pgfpathcurveto{\pgfqpoint{1.704379in}{1.924179in}}{\pgfqpoint{1.701107in}{1.932079in}}{\pgfqpoint{1.695283in}{1.937903in}}%
\pgfpathcurveto{\pgfqpoint{1.689459in}{1.943727in}}{\pgfqpoint{1.681559in}{1.946999in}}{\pgfqpoint{1.673323in}{1.946999in}}%
\pgfpathcurveto{\pgfqpoint{1.665086in}{1.946999in}}{\pgfqpoint{1.657186in}{1.943727in}}{\pgfqpoint{1.651362in}{1.937903in}}%
\pgfpathcurveto{\pgfqpoint{1.645538in}{1.932079in}}{\pgfqpoint{1.642266in}{1.924179in}}{\pgfqpoint{1.642266in}{1.915943in}}%
\pgfpathcurveto{\pgfqpoint{1.642266in}{1.907707in}}{\pgfqpoint{1.645538in}{1.899807in}}{\pgfqpoint{1.651362in}{1.893983in}}%
\pgfpathcurveto{\pgfqpoint{1.657186in}{1.888159in}}{\pgfqpoint{1.665086in}{1.884886in}}{\pgfqpoint{1.673323in}{1.884886in}}%
\pgfpathclose%
\pgfusepath{stroke,fill}%
\end{pgfscope}%
\begin{pgfscope}%
\pgfpathrectangle{\pgfqpoint{0.100000in}{0.212622in}}{\pgfqpoint{3.696000in}{3.696000in}}%
\pgfusepath{clip}%
\pgfsetbuttcap%
\pgfsetroundjoin%
\definecolor{currentfill}{rgb}{0.121569,0.466667,0.705882}%
\pgfsetfillcolor{currentfill}%
\pgfsetfillopacity{0.385460}%
\pgfsetlinewidth{1.003750pt}%
\definecolor{currentstroke}{rgb}{0.121569,0.466667,0.705882}%
\pgfsetstrokecolor{currentstroke}%
\pgfsetstrokeopacity{0.385460}%
\pgfsetdash{}{0pt}%
\pgfpathmoveto{\pgfqpoint{1.669587in}{1.881934in}}%
\pgfpathcurveto{\pgfqpoint{1.677824in}{1.881934in}}{\pgfqpoint{1.685724in}{1.885207in}}{\pgfqpoint{1.691548in}{1.891031in}}%
\pgfpathcurveto{\pgfqpoint{1.697372in}{1.896855in}}{\pgfqpoint{1.700644in}{1.904755in}}{\pgfqpoint{1.700644in}{1.912991in}}%
\pgfpathcurveto{\pgfqpoint{1.700644in}{1.921227in}}{\pgfqpoint{1.697372in}{1.929127in}}{\pgfqpoint{1.691548in}{1.934951in}}%
\pgfpathcurveto{\pgfqpoint{1.685724in}{1.940775in}}{\pgfqpoint{1.677824in}{1.944047in}}{\pgfqpoint{1.669587in}{1.944047in}}%
\pgfpathcurveto{\pgfqpoint{1.661351in}{1.944047in}}{\pgfqpoint{1.653451in}{1.940775in}}{\pgfqpoint{1.647627in}{1.934951in}}%
\pgfpathcurveto{\pgfqpoint{1.641803in}{1.929127in}}{\pgfqpoint{1.638531in}{1.921227in}}{\pgfqpoint{1.638531in}{1.912991in}}%
\pgfpathcurveto{\pgfqpoint{1.638531in}{1.904755in}}{\pgfqpoint{1.641803in}{1.896855in}}{\pgfqpoint{1.647627in}{1.891031in}}%
\pgfpathcurveto{\pgfqpoint{1.653451in}{1.885207in}}{\pgfqpoint{1.661351in}{1.881934in}}{\pgfqpoint{1.669587in}{1.881934in}}%
\pgfpathclose%
\pgfusepath{stroke,fill}%
\end{pgfscope}%
\begin{pgfscope}%
\pgfpathrectangle{\pgfqpoint{0.100000in}{0.212622in}}{\pgfqpoint{3.696000in}{3.696000in}}%
\pgfusepath{clip}%
\pgfsetbuttcap%
\pgfsetroundjoin%
\definecolor{currentfill}{rgb}{0.121569,0.466667,0.705882}%
\pgfsetfillcolor{currentfill}%
\pgfsetfillopacity{0.387474}%
\pgfsetlinewidth{1.003750pt}%
\definecolor{currentstroke}{rgb}{0.121569,0.466667,0.705882}%
\pgfsetstrokecolor{currentstroke}%
\pgfsetstrokeopacity{0.387474}%
\pgfsetdash{}{0pt}%
\pgfpathmoveto{\pgfqpoint{1.664162in}{1.877490in}}%
\pgfpathcurveto{\pgfqpoint{1.672398in}{1.877490in}}{\pgfqpoint{1.680298in}{1.880762in}}{\pgfqpoint{1.686122in}{1.886586in}}%
\pgfpathcurveto{\pgfqpoint{1.691946in}{1.892410in}}{\pgfqpoint{1.695219in}{1.900310in}}{\pgfqpoint{1.695219in}{1.908546in}}%
\pgfpathcurveto{\pgfqpoint{1.695219in}{1.916782in}}{\pgfqpoint{1.691946in}{1.924682in}}{\pgfqpoint{1.686122in}{1.930506in}}%
\pgfpathcurveto{\pgfqpoint{1.680298in}{1.936330in}}{\pgfqpoint{1.672398in}{1.939603in}}{\pgfqpoint{1.664162in}{1.939603in}}%
\pgfpathcurveto{\pgfqpoint{1.655926in}{1.939603in}}{\pgfqpoint{1.648026in}{1.936330in}}{\pgfqpoint{1.642202in}{1.930506in}}%
\pgfpathcurveto{\pgfqpoint{1.636378in}{1.924682in}}{\pgfqpoint{1.633106in}{1.916782in}}{\pgfqpoint{1.633106in}{1.908546in}}%
\pgfpathcurveto{\pgfqpoint{1.633106in}{1.900310in}}{\pgfqpoint{1.636378in}{1.892410in}}{\pgfqpoint{1.642202in}{1.886586in}}%
\pgfpathcurveto{\pgfqpoint{1.648026in}{1.880762in}}{\pgfqpoint{1.655926in}{1.877490in}}{\pgfqpoint{1.664162in}{1.877490in}}%
\pgfpathclose%
\pgfusepath{stroke,fill}%
\end{pgfscope}%
\begin{pgfscope}%
\pgfpathrectangle{\pgfqpoint{0.100000in}{0.212622in}}{\pgfqpoint{3.696000in}{3.696000in}}%
\pgfusepath{clip}%
\pgfsetbuttcap%
\pgfsetroundjoin%
\definecolor{currentfill}{rgb}{0.121569,0.466667,0.705882}%
\pgfsetfillcolor{currentfill}%
\pgfsetfillopacity{0.388537}%
\pgfsetlinewidth{1.003750pt}%
\definecolor{currentstroke}{rgb}{0.121569,0.466667,0.705882}%
\pgfsetstrokecolor{currentstroke}%
\pgfsetstrokeopacity{0.388537}%
\pgfsetdash{}{0pt}%
\pgfpathmoveto{\pgfqpoint{1.659197in}{1.874682in}}%
\pgfpathcurveto{\pgfqpoint{1.667433in}{1.874682in}}{\pgfqpoint{1.675333in}{1.877954in}}{\pgfqpoint{1.681157in}{1.883778in}}%
\pgfpathcurveto{\pgfqpoint{1.686981in}{1.889602in}}{\pgfqpoint{1.690253in}{1.897502in}}{\pgfqpoint{1.690253in}{1.905738in}}%
\pgfpathcurveto{\pgfqpoint{1.690253in}{1.913975in}}{\pgfqpoint{1.686981in}{1.921875in}}{\pgfqpoint{1.681157in}{1.927699in}}%
\pgfpathcurveto{\pgfqpoint{1.675333in}{1.933523in}}{\pgfqpoint{1.667433in}{1.936795in}}{\pgfqpoint{1.659197in}{1.936795in}}%
\pgfpathcurveto{\pgfqpoint{1.650961in}{1.936795in}}{\pgfqpoint{1.643061in}{1.933523in}}{\pgfqpoint{1.637237in}{1.927699in}}%
\pgfpathcurveto{\pgfqpoint{1.631413in}{1.921875in}}{\pgfqpoint{1.628140in}{1.913975in}}{\pgfqpoint{1.628140in}{1.905738in}}%
\pgfpathcurveto{\pgfqpoint{1.628140in}{1.897502in}}{\pgfqpoint{1.631413in}{1.889602in}}{\pgfqpoint{1.637237in}{1.883778in}}%
\pgfpathcurveto{\pgfqpoint{1.643061in}{1.877954in}}{\pgfqpoint{1.650961in}{1.874682in}}{\pgfqpoint{1.659197in}{1.874682in}}%
\pgfpathclose%
\pgfusepath{stroke,fill}%
\end{pgfscope}%
\begin{pgfscope}%
\pgfpathrectangle{\pgfqpoint{0.100000in}{0.212622in}}{\pgfqpoint{3.696000in}{3.696000in}}%
\pgfusepath{clip}%
\pgfsetbuttcap%
\pgfsetroundjoin%
\definecolor{currentfill}{rgb}{0.121569,0.466667,0.705882}%
\pgfsetfillcolor{currentfill}%
\pgfsetfillopacity{0.389859}%
\pgfsetlinewidth{1.003750pt}%
\definecolor{currentstroke}{rgb}{0.121569,0.466667,0.705882}%
\pgfsetstrokecolor{currentstroke}%
\pgfsetstrokeopacity{0.389859}%
\pgfsetdash{}{0pt}%
\pgfpathmoveto{\pgfqpoint{1.656350in}{1.874506in}}%
\pgfpathcurveto{\pgfqpoint{1.664586in}{1.874506in}}{\pgfqpoint{1.672486in}{1.877778in}}{\pgfqpoint{1.678310in}{1.883602in}}%
\pgfpathcurveto{\pgfqpoint{1.684134in}{1.889426in}}{\pgfqpoint{1.687406in}{1.897326in}}{\pgfqpoint{1.687406in}{1.905563in}}%
\pgfpathcurveto{\pgfqpoint{1.687406in}{1.913799in}}{\pgfqpoint{1.684134in}{1.921699in}}{\pgfqpoint{1.678310in}{1.927523in}}%
\pgfpathcurveto{\pgfqpoint{1.672486in}{1.933347in}}{\pgfqpoint{1.664586in}{1.936619in}}{\pgfqpoint{1.656350in}{1.936619in}}%
\pgfpathcurveto{\pgfqpoint{1.648113in}{1.936619in}}{\pgfqpoint{1.640213in}{1.933347in}}{\pgfqpoint{1.634389in}{1.927523in}}%
\pgfpathcurveto{\pgfqpoint{1.628566in}{1.921699in}}{\pgfqpoint{1.625293in}{1.913799in}}{\pgfqpoint{1.625293in}{1.905563in}}%
\pgfpathcurveto{\pgfqpoint{1.625293in}{1.897326in}}{\pgfqpoint{1.628566in}{1.889426in}}{\pgfqpoint{1.634389in}{1.883602in}}%
\pgfpathcurveto{\pgfqpoint{1.640213in}{1.877778in}}{\pgfqpoint{1.648113in}{1.874506in}}{\pgfqpoint{1.656350in}{1.874506in}}%
\pgfpathclose%
\pgfusepath{stroke,fill}%
\end{pgfscope}%
\begin{pgfscope}%
\pgfpathrectangle{\pgfqpoint{0.100000in}{0.212622in}}{\pgfqpoint{3.696000in}{3.696000in}}%
\pgfusepath{clip}%
\pgfsetbuttcap%
\pgfsetroundjoin%
\definecolor{currentfill}{rgb}{0.121569,0.466667,0.705882}%
\pgfsetfillcolor{currentfill}%
\pgfsetfillopacity{0.390637}%
\pgfsetlinewidth{1.003750pt}%
\definecolor{currentstroke}{rgb}{0.121569,0.466667,0.705882}%
\pgfsetstrokecolor{currentstroke}%
\pgfsetstrokeopacity{0.390637}%
\pgfsetdash{}{0pt}%
\pgfpathmoveto{\pgfqpoint{1.999286in}{1.953770in}}%
\pgfpathcurveto{\pgfqpoint{2.007522in}{1.953770in}}{\pgfqpoint{2.015422in}{1.957042in}}{\pgfqpoint{2.021246in}{1.962866in}}%
\pgfpathcurveto{\pgfqpoint{2.027070in}{1.968690in}}{\pgfqpoint{2.030342in}{1.976590in}}{\pgfqpoint{2.030342in}{1.984826in}}%
\pgfpathcurveto{\pgfqpoint{2.030342in}{1.993062in}}{\pgfqpoint{2.027070in}{2.000963in}}{\pgfqpoint{2.021246in}{2.006786in}}%
\pgfpathcurveto{\pgfqpoint{2.015422in}{2.012610in}}{\pgfqpoint{2.007522in}{2.015883in}}{\pgfqpoint{1.999286in}{2.015883in}}%
\pgfpathcurveto{\pgfqpoint{1.991050in}{2.015883in}}{\pgfqpoint{1.983150in}{2.012610in}}{\pgfqpoint{1.977326in}{2.006786in}}%
\pgfpathcurveto{\pgfqpoint{1.971502in}{2.000963in}}{\pgfqpoint{1.968229in}{1.993062in}}{\pgfqpoint{1.968229in}{1.984826in}}%
\pgfpathcurveto{\pgfqpoint{1.968229in}{1.976590in}}{\pgfqpoint{1.971502in}{1.968690in}}{\pgfqpoint{1.977326in}{1.962866in}}%
\pgfpathcurveto{\pgfqpoint{1.983150in}{1.957042in}}{\pgfqpoint{1.991050in}{1.953770in}}{\pgfqpoint{1.999286in}{1.953770in}}%
\pgfpathclose%
\pgfusepath{stroke,fill}%
\end{pgfscope}%
\begin{pgfscope}%
\pgfpathrectangle{\pgfqpoint{0.100000in}{0.212622in}}{\pgfqpoint{3.696000in}{3.696000in}}%
\pgfusepath{clip}%
\pgfsetbuttcap%
\pgfsetroundjoin%
\definecolor{currentfill}{rgb}{0.121569,0.466667,0.705882}%
\pgfsetfillcolor{currentfill}%
\pgfsetfillopacity{0.391968}%
\pgfsetlinewidth{1.003750pt}%
\definecolor{currentstroke}{rgb}{0.121569,0.466667,0.705882}%
\pgfsetstrokecolor{currentstroke}%
\pgfsetstrokeopacity{0.391968}%
\pgfsetdash{}{0pt}%
\pgfpathmoveto{\pgfqpoint{1.650340in}{1.873341in}}%
\pgfpathcurveto{\pgfqpoint{1.658576in}{1.873341in}}{\pgfqpoint{1.666476in}{1.876613in}}{\pgfqpoint{1.672300in}{1.882437in}}%
\pgfpathcurveto{\pgfqpoint{1.678124in}{1.888261in}}{\pgfqpoint{1.681396in}{1.896161in}}{\pgfqpoint{1.681396in}{1.904397in}}%
\pgfpathcurveto{\pgfqpoint{1.681396in}{1.912634in}}{\pgfqpoint{1.678124in}{1.920534in}}{\pgfqpoint{1.672300in}{1.926358in}}%
\pgfpathcurveto{\pgfqpoint{1.666476in}{1.932182in}}{\pgfqpoint{1.658576in}{1.935454in}}{\pgfqpoint{1.650340in}{1.935454in}}%
\pgfpathcurveto{\pgfqpoint{1.642104in}{1.935454in}}{\pgfqpoint{1.634204in}{1.932182in}}{\pgfqpoint{1.628380in}{1.926358in}}%
\pgfpathcurveto{\pgfqpoint{1.622556in}{1.920534in}}{\pgfqpoint{1.619283in}{1.912634in}}{\pgfqpoint{1.619283in}{1.904397in}}%
\pgfpathcurveto{\pgfqpoint{1.619283in}{1.896161in}}{\pgfqpoint{1.622556in}{1.888261in}}{\pgfqpoint{1.628380in}{1.882437in}}%
\pgfpathcurveto{\pgfqpoint{1.634204in}{1.876613in}}{\pgfqpoint{1.642104in}{1.873341in}}{\pgfqpoint{1.650340in}{1.873341in}}%
\pgfpathclose%
\pgfusepath{stroke,fill}%
\end{pgfscope}%
\begin{pgfscope}%
\pgfpathrectangle{\pgfqpoint{0.100000in}{0.212622in}}{\pgfqpoint{3.696000in}{3.696000in}}%
\pgfusepath{clip}%
\pgfsetbuttcap%
\pgfsetroundjoin%
\definecolor{currentfill}{rgb}{0.121569,0.466667,0.705882}%
\pgfsetfillcolor{currentfill}%
\pgfsetfillopacity{0.393095}%
\pgfsetlinewidth{1.003750pt}%
\definecolor{currentstroke}{rgb}{0.121569,0.466667,0.705882}%
\pgfsetstrokecolor{currentstroke}%
\pgfsetstrokeopacity{0.393095}%
\pgfsetdash{}{0pt}%
\pgfpathmoveto{\pgfqpoint{1.645501in}{1.869277in}}%
\pgfpathcurveto{\pgfqpoint{1.653738in}{1.869277in}}{\pgfqpoint{1.661638in}{1.872549in}}{\pgfqpoint{1.667462in}{1.878373in}}%
\pgfpathcurveto{\pgfqpoint{1.673286in}{1.884197in}}{\pgfqpoint{1.676558in}{1.892097in}}{\pgfqpoint{1.676558in}{1.900333in}}%
\pgfpathcurveto{\pgfqpoint{1.676558in}{1.908570in}}{\pgfqpoint{1.673286in}{1.916470in}}{\pgfqpoint{1.667462in}{1.922294in}}%
\pgfpathcurveto{\pgfqpoint{1.661638in}{1.928117in}}{\pgfqpoint{1.653738in}{1.931390in}}{\pgfqpoint{1.645501in}{1.931390in}}%
\pgfpathcurveto{\pgfqpoint{1.637265in}{1.931390in}}{\pgfqpoint{1.629365in}{1.928117in}}{\pgfqpoint{1.623541in}{1.922294in}}%
\pgfpathcurveto{\pgfqpoint{1.617717in}{1.916470in}}{\pgfqpoint{1.614445in}{1.908570in}}{\pgfqpoint{1.614445in}{1.900333in}}%
\pgfpathcurveto{\pgfqpoint{1.614445in}{1.892097in}}{\pgfqpoint{1.617717in}{1.884197in}}{\pgfqpoint{1.623541in}{1.878373in}}%
\pgfpathcurveto{\pgfqpoint{1.629365in}{1.872549in}}{\pgfqpoint{1.637265in}{1.869277in}}{\pgfqpoint{1.645501in}{1.869277in}}%
\pgfpathclose%
\pgfusepath{stroke,fill}%
\end{pgfscope}%
\begin{pgfscope}%
\pgfpathrectangle{\pgfqpoint{0.100000in}{0.212622in}}{\pgfqpoint{3.696000in}{3.696000in}}%
\pgfusepath{clip}%
\pgfsetbuttcap%
\pgfsetroundjoin%
\definecolor{currentfill}{rgb}{0.121569,0.466667,0.705882}%
\pgfsetfillcolor{currentfill}%
\pgfsetfillopacity{0.394457}%
\pgfsetlinewidth{1.003750pt}%
\definecolor{currentstroke}{rgb}{0.121569,0.466667,0.705882}%
\pgfsetstrokecolor{currentstroke}%
\pgfsetstrokeopacity{0.394457}%
\pgfsetdash{}{0pt}%
\pgfpathmoveto{\pgfqpoint{1.641416in}{1.868797in}}%
\pgfpathcurveto{\pgfqpoint{1.649652in}{1.868797in}}{\pgfqpoint{1.657552in}{1.872070in}}{\pgfqpoint{1.663376in}{1.877894in}}%
\pgfpathcurveto{\pgfqpoint{1.669200in}{1.883717in}}{\pgfqpoint{1.672472in}{1.891617in}}{\pgfqpoint{1.672472in}{1.899854in}}%
\pgfpathcurveto{\pgfqpoint{1.672472in}{1.908090in}}{\pgfqpoint{1.669200in}{1.915990in}}{\pgfqpoint{1.663376in}{1.921814in}}%
\pgfpathcurveto{\pgfqpoint{1.657552in}{1.927638in}}{\pgfqpoint{1.649652in}{1.930910in}}{\pgfqpoint{1.641416in}{1.930910in}}%
\pgfpathcurveto{\pgfqpoint{1.633179in}{1.930910in}}{\pgfqpoint{1.625279in}{1.927638in}}{\pgfqpoint{1.619455in}{1.921814in}}%
\pgfpathcurveto{\pgfqpoint{1.613631in}{1.915990in}}{\pgfqpoint{1.610359in}{1.908090in}}{\pgfqpoint{1.610359in}{1.899854in}}%
\pgfpathcurveto{\pgfqpoint{1.610359in}{1.891617in}}{\pgfqpoint{1.613631in}{1.883717in}}{\pgfqpoint{1.619455in}{1.877894in}}%
\pgfpathcurveto{\pgfqpoint{1.625279in}{1.872070in}}{\pgfqpoint{1.633179in}{1.868797in}}{\pgfqpoint{1.641416in}{1.868797in}}%
\pgfpathclose%
\pgfusepath{stroke,fill}%
\end{pgfscope}%
\begin{pgfscope}%
\pgfpathrectangle{\pgfqpoint{0.100000in}{0.212622in}}{\pgfqpoint{3.696000in}{3.696000in}}%
\pgfusepath{clip}%
\pgfsetbuttcap%
\pgfsetroundjoin%
\definecolor{currentfill}{rgb}{0.121569,0.466667,0.705882}%
\pgfsetfillcolor{currentfill}%
\pgfsetfillopacity{0.396791}%
\pgfsetlinewidth{1.003750pt}%
\definecolor{currentstroke}{rgb}{0.121569,0.466667,0.705882}%
\pgfsetstrokecolor{currentstroke}%
\pgfsetstrokeopacity{0.396791}%
\pgfsetdash{}{0pt}%
\pgfpathmoveto{\pgfqpoint{1.635309in}{1.865018in}}%
\pgfpathcurveto{\pgfqpoint{1.643545in}{1.865018in}}{\pgfqpoint{1.651445in}{1.868291in}}{\pgfqpoint{1.657269in}{1.874115in}}%
\pgfpathcurveto{\pgfqpoint{1.663093in}{1.879939in}}{\pgfqpoint{1.666365in}{1.887839in}}{\pgfqpoint{1.666365in}{1.896075in}}%
\pgfpathcurveto{\pgfqpoint{1.666365in}{1.904311in}}{\pgfqpoint{1.663093in}{1.912211in}}{\pgfqpoint{1.657269in}{1.918035in}}%
\pgfpathcurveto{\pgfqpoint{1.651445in}{1.923859in}}{\pgfqpoint{1.643545in}{1.927131in}}{\pgfqpoint{1.635309in}{1.927131in}}%
\pgfpathcurveto{\pgfqpoint{1.627073in}{1.927131in}}{\pgfqpoint{1.619173in}{1.923859in}}{\pgfqpoint{1.613349in}{1.918035in}}%
\pgfpathcurveto{\pgfqpoint{1.607525in}{1.912211in}}{\pgfqpoint{1.604252in}{1.904311in}}{\pgfqpoint{1.604252in}{1.896075in}}%
\pgfpathcurveto{\pgfqpoint{1.604252in}{1.887839in}}{\pgfqpoint{1.607525in}{1.879939in}}{\pgfqpoint{1.613349in}{1.874115in}}%
\pgfpathcurveto{\pgfqpoint{1.619173in}{1.868291in}}{\pgfqpoint{1.627073in}{1.865018in}}{\pgfqpoint{1.635309in}{1.865018in}}%
\pgfpathclose%
\pgfusepath{stroke,fill}%
\end{pgfscope}%
\begin{pgfscope}%
\pgfpathrectangle{\pgfqpoint{0.100000in}{0.212622in}}{\pgfqpoint{3.696000in}{3.696000in}}%
\pgfusepath{clip}%
\pgfsetbuttcap%
\pgfsetroundjoin%
\definecolor{currentfill}{rgb}{0.121569,0.466667,0.705882}%
\pgfsetfillcolor{currentfill}%
\pgfsetfillopacity{0.397646}%
\pgfsetlinewidth{1.003750pt}%
\definecolor{currentstroke}{rgb}{0.121569,0.466667,0.705882}%
\pgfsetstrokecolor{currentstroke}%
\pgfsetstrokeopacity{0.397646}%
\pgfsetdash{}{0pt}%
\pgfpathmoveto{\pgfqpoint{1.631726in}{1.861706in}}%
\pgfpathcurveto{\pgfqpoint{1.639963in}{1.861706in}}{\pgfqpoint{1.647863in}{1.864979in}}{\pgfqpoint{1.653687in}{1.870802in}}%
\pgfpathcurveto{\pgfqpoint{1.659511in}{1.876626in}}{\pgfqpoint{1.662783in}{1.884526in}}{\pgfqpoint{1.662783in}{1.892763in}}%
\pgfpathcurveto{\pgfqpoint{1.662783in}{1.900999in}}{\pgfqpoint{1.659511in}{1.908899in}}{\pgfqpoint{1.653687in}{1.914723in}}%
\pgfpathcurveto{\pgfqpoint{1.647863in}{1.920547in}}{\pgfqpoint{1.639963in}{1.923819in}}{\pgfqpoint{1.631726in}{1.923819in}}%
\pgfpathcurveto{\pgfqpoint{1.623490in}{1.923819in}}{\pgfqpoint{1.615590in}{1.920547in}}{\pgfqpoint{1.609766in}{1.914723in}}%
\pgfpathcurveto{\pgfqpoint{1.603942in}{1.908899in}}{\pgfqpoint{1.600670in}{1.900999in}}{\pgfqpoint{1.600670in}{1.892763in}}%
\pgfpathcurveto{\pgfqpoint{1.600670in}{1.884526in}}{\pgfqpoint{1.603942in}{1.876626in}}{\pgfqpoint{1.609766in}{1.870802in}}%
\pgfpathcurveto{\pgfqpoint{1.615590in}{1.864979in}}{\pgfqpoint{1.623490in}{1.861706in}}{\pgfqpoint{1.631726in}{1.861706in}}%
\pgfpathclose%
\pgfusepath{stroke,fill}%
\end{pgfscope}%
\begin{pgfscope}%
\pgfpathrectangle{\pgfqpoint{0.100000in}{0.212622in}}{\pgfqpoint{3.696000in}{3.696000in}}%
\pgfusepath{clip}%
\pgfsetbuttcap%
\pgfsetroundjoin%
\definecolor{currentfill}{rgb}{0.121569,0.466667,0.705882}%
\pgfsetfillcolor{currentfill}%
\pgfsetfillopacity{0.399017}%
\pgfsetlinewidth{1.003750pt}%
\definecolor{currentstroke}{rgb}{0.121569,0.466667,0.705882}%
\pgfsetstrokecolor{currentstroke}%
\pgfsetstrokeopacity{0.399017}%
\pgfsetdash{}{0pt}%
\pgfpathmoveto{\pgfqpoint{2.003842in}{1.950774in}}%
\pgfpathcurveto{\pgfqpoint{2.012078in}{1.950774in}}{\pgfqpoint{2.019978in}{1.954046in}}{\pgfqpoint{2.025802in}{1.959870in}}%
\pgfpathcurveto{\pgfqpoint{2.031626in}{1.965694in}}{\pgfqpoint{2.034898in}{1.973594in}}{\pgfqpoint{2.034898in}{1.981830in}}%
\pgfpathcurveto{\pgfqpoint{2.034898in}{1.990067in}}{\pgfqpoint{2.031626in}{1.997967in}}{\pgfqpoint{2.025802in}{2.003790in}}%
\pgfpathcurveto{\pgfqpoint{2.019978in}{2.009614in}}{\pgfqpoint{2.012078in}{2.012887in}}{\pgfqpoint{2.003842in}{2.012887in}}%
\pgfpathcurveto{\pgfqpoint{1.995606in}{2.012887in}}{\pgfqpoint{1.987706in}{2.009614in}}{\pgfqpoint{1.981882in}{2.003790in}}%
\pgfpathcurveto{\pgfqpoint{1.976058in}{1.997967in}}{\pgfqpoint{1.972785in}{1.990067in}}{\pgfqpoint{1.972785in}{1.981830in}}%
\pgfpathcurveto{\pgfqpoint{1.972785in}{1.973594in}}{\pgfqpoint{1.976058in}{1.965694in}}{\pgfqpoint{1.981882in}{1.959870in}}%
\pgfpathcurveto{\pgfqpoint{1.987706in}{1.954046in}}{\pgfqpoint{1.995606in}{1.950774in}}{\pgfqpoint{2.003842in}{1.950774in}}%
\pgfpathclose%
\pgfusepath{stroke,fill}%
\end{pgfscope}%
\begin{pgfscope}%
\pgfpathrectangle{\pgfqpoint{0.100000in}{0.212622in}}{\pgfqpoint{3.696000in}{3.696000in}}%
\pgfusepath{clip}%
\pgfsetbuttcap%
\pgfsetroundjoin%
\definecolor{currentfill}{rgb}{0.121569,0.466667,0.705882}%
\pgfsetfillcolor{currentfill}%
\pgfsetfillopacity{0.399400}%
\pgfsetlinewidth{1.003750pt}%
\definecolor{currentstroke}{rgb}{0.121569,0.466667,0.705882}%
\pgfsetstrokecolor{currentstroke}%
\pgfsetstrokeopacity{0.399400}%
\pgfsetdash{}{0pt}%
\pgfpathmoveto{\pgfqpoint{1.625033in}{1.857296in}}%
\pgfpathcurveto{\pgfqpoint{1.633269in}{1.857296in}}{\pgfqpoint{1.641169in}{1.860569in}}{\pgfqpoint{1.646993in}{1.866392in}}%
\pgfpathcurveto{\pgfqpoint{1.652817in}{1.872216in}}{\pgfqpoint{1.656090in}{1.880116in}}{\pgfqpoint{1.656090in}{1.888353in}}%
\pgfpathcurveto{\pgfqpoint{1.656090in}{1.896589in}}{\pgfqpoint{1.652817in}{1.904489in}}{\pgfqpoint{1.646993in}{1.910313in}}%
\pgfpathcurveto{\pgfqpoint{1.641169in}{1.916137in}}{\pgfqpoint{1.633269in}{1.919409in}}{\pgfqpoint{1.625033in}{1.919409in}}%
\pgfpathcurveto{\pgfqpoint{1.616797in}{1.919409in}}{\pgfqpoint{1.608897in}{1.916137in}}{\pgfqpoint{1.603073in}{1.910313in}}%
\pgfpathcurveto{\pgfqpoint{1.597249in}{1.904489in}}{\pgfqpoint{1.593977in}{1.896589in}}{\pgfqpoint{1.593977in}{1.888353in}}%
\pgfpathcurveto{\pgfqpoint{1.593977in}{1.880116in}}{\pgfqpoint{1.597249in}{1.872216in}}{\pgfqpoint{1.603073in}{1.866392in}}%
\pgfpathcurveto{\pgfqpoint{1.608897in}{1.860569in}}{\pgfqpoint{1.616797in}{1.857296in}}{\pgfqpoint{1.625033in}{1.857296in}}%
\pgfpathclose%
\pgfusepath{stroke,fill}%
\end{pgfscope}%
\begin{pgfscope}%
\pgfpathrectangle{\pgfqpoint{0.100000in}{0.212622in}}{\pgfqpoint{3.696000in}{3.696000in}}%
\pgfusepath{clip}%
\pgfsetbuttcap%
\pgfsetroundjoin%
\definecolor{currentfill}{rgb}{0.121569,0.466667,0.705882}%
\pgfsetfillcolor{currentfill}%
\pgfsetfillopacity{0.403428}%
\pgfsetlinewidth{1.003750pt}%
\definecolor{currentstroke}{rgb}{0.121569,0.466667,0.705882}%
\pgfsetstrokecolor{currentstroke}%
\pgfsetstrokeopacity{0.403428}%
\pgfsetdash{}{0pt}%
\pgfpathmoveto{\pgfqpoint{2.005560in}{1.947342in}}%
\pgfpathcurveto{\pgfqpoint{2.013796in}{1.947342in}}{\pgfqpoint{2.021696in}{1.950614in}}{\pgfqpoint{2.027520in}{1.956438in}}%
\pgfpathcurveto{\pgfqpoint{2.033344in}{1.962262in}}{\pgfqpoint{2.036617in}{1.970162in}}{\pgfqpoint{2.036617in}{1.978398in}}%
\pgfpathcurveto{\pgfqpoint{2.036617in}{1.986635in}}{\pgfqpoint{2.033344in}{1.994535in}}{\pgfqpoint{2.027520in}{2.000359in}}%
\pgfpathcurveto{\pgfqpoint{2.021696in}{2.006182in}}{\pgfqpoint{2.013796in}{2.009455in}}{\pgfqpoint{2.005560in}{2.009455in}}%
\pgfpathcurveto{\pgfqpoint{1.997324in}{2.009455in}}{\pgfqpoint{1.989424in}{2.006182in}}{\pgfqpoint{1.983600in}{2.000359in}}%
\pgfpathcurveto{\pgfqpoint{1.977776in}{1.994535in}}{\pgfqpoint{1.974504in}{1.986635in}}{\pgfqpoint{1.974504in}{1.978398in}}%
\pgfpathcurveto{\pgfqpoint{1.974504in}{1.970162in}}{\pgfqpoint{1.977776in}{1.962262in}}{\pgfqpoint{1.983600in}{1.956438in}}%
\pgfpathcurveto{\pgfqpoint{1.989424in}{1.950614in}}{\pgfqpoint{1.997324in}{1.947342in}}{\pgfqpoint{2.005560in}{1.947342in}}%
\pgfpathclose%
\pgfusepath{stroke,fill}%
\end{pgfscope}%
\begin{pgfscope}%
\pgfpathrectangle{\pgfqpoint{0.100000in}{0.212622in}}{\pgfqpoint{3.696000in}{3.696000in}}%
\pgfusepath{clip}%
\pgfsetbuttcap%
\pgfsetroundjoin%
\definecolor{currentfill}{rgb}{0.121569,0.466667,0.705882}%
\pgfsetfillcolor{currentfill}%
\pgfsetfillopacity{0.404405}%
\pgfsetlinewidth{1.003750pt}%
\definecolor{currentstroke}{rgb}{0.121569,0.466667,0.705882}%
\pgfsetstrokecolor{currentstroke}%
\pgfsetstrokeopacity{0.404405}%
\pgfsetdash{}{0pt}%
\pgfpathmoveto{\pgfqpoint{1.614589in}{1.858892in}}%
\pgfpathcurveto{\pgfqpoint{1.622825in}{1.858892in}}{\pgfqpoint{1.630725in}{1.862164in}}{\pgfqpoint{1.636549in}{1.867988in}}%
\pgfpathcurveto{\pgfqpoint{1.642373in}{1.873812in}}{\pgfqpoint{1.645645in}{1.881712in}}{\pgfqpoint{1.645645in}{1.889948in}}%
\pgfpathcurveto{\pgfqpoint{1.645645in}{1.898184in}}{\pgfqpoint{1.642373in}{1.906084in}}{\pgfqpoint{1.636549in}{1.911908in}}%
\pgfpathcurveto{\pgfqpoint{1.630725in}{1.917732in}}{\pgfqpoint{1.622825in}{1.921005in}}{\pgfqpoint{1.614589in}{1.921005in}}%
\pgfpathcurveto{\pgfqpoint{1.606352in}{1.921005in}}{\pgfqpoint{1.598452in}{1.917732in}}{\pgfqpoint{1.592628in}{1.911908in}}%
\pgfpathcurveto{\pgfqpoint{1.586804in}{1.906084in}}{\pgfqpoint{1.583532in}{1.898184in}}{\pgfqpoint{1.583532in}{1.889948in}}%
\pgfpathcurveto{\pgfqpoint{1.583532in}{1.881712in}}{\pgfqpoint{1.586804in}{1.873812in}}{\pgfqpoint{1.592628in}{1.867988in}}%
\pgfpathcurveto{\pgfqpoint{1.598452in}{1.862164in}}{\pgfqpoint{1.606352in}{1.858892in}}{\pgfqpoint{1.614589in}{1.858892in}}%
\pgfpathclose%
\pgfusepath{stroke,fill}%
\end{pgfscope}%
\begin{pgfscope}%
\pgfpathrectangle{\pgfqpoint{0.100000in}{0.212622in}}{\pgfqpoint{3.696000in}{3.696000in}}%
\pgfusepath{clip}%
\pgfsetbuttcap%
\pgfsetroundjoin%
\definecolor{currentfill}{rgb}{0.121569,0.466667,0.705882}%
\pgfsetfillcolor{currentfill}%
\pgfsetfillopacity{0.406612}%
\pgfsetlinewidth{1.003750pt}%
\definecolor{currentstroke}{rgb}{0.121569,0.466667,0.705882}%
\pgfsetstrokecolor{currentstroke}%
\pgfsetstrokeopacity{0.406612}%
\pgfsetdash{}{0pt}%
\pgfpathmoveto{\pgfqpoint{1.604664in}{1.851833in}}%
\pgfpathcurveto{\pgfqpoint{1.612900in}{1.851833in}}{\pgfqpoint{1.620801in}{1.855105in}}{\pgfqpoint{1.626624in}{1.860929in}}%
\pgfpathcurveto{\pgfqpoint{1.632448in}{1.866753in}}{\pgfqpoint{1.635721in}{1.874653in}}{\pgfqpoint{1.635721in}{1.882889in}}%
\pgfpathcurveto{\pgfqpoint{1.635721in}{1.891126in}}{\pgfqpoint{1.632448in}{1.899026in}}{\pgfqpoint{1.626624in}{1.904849in}}%
\pgfpathcurveto{\pgfqpoint{1.620801in}{1.910673in}}{\pgfqpoint{1.612900in}{1.913946in}}{\pgfqpoint{1.604664in}{1.913946in}}%
\pgfpathcurveto{\pgfqpoint{1.596428in}{1.913946in}}{\pgfqpoint{1.588528in}{1.910673in}}{\pgfqpoint{1.582704in}{1.904849in}}%
\pgfpathcurveto{\pgfqpoint{1.576880in}{1.899026in}}{\pgfqpoint{1.573608in}{1.891126in}}{\pgfqpoint{1.573608in}{1.882889in}}%
\pgfpathcurveto{\pgfqpoint{1.573608in}{1.874653in}}{\pgfqpoint{1.576880in}{1.866753in}}{\pgfqpoint{1.582704in}{1.860929in}}%
\pgfpathcurveto{\pgfqpoint{1.588528in}{1.855105in}}{\pgfqpoint{1.596428in}{1.851833in}}{\pgfqpoint{1.604664in}{1.851833in}}%
\pgfpathclose%
\pgfusepath{stroke,fill}%
\end{pgfscope}%
\begin{pgfscope}%
\pgfpathrectangle{\pgfqpoint{0.100000in}{0.212622in}}{\pgfqpoint{3.696000in}{3.696000in}}%
\pgfusepath{clip}%
\pgfsetbuttcap%
\pgfsetroundjoin%
\definecolor{currentfill}{rgb}{0.121569,0.466667,0.705882}%
\pgfsetfillcolor{currentfill}%
\pgfsetfillopacity{0.408016}%
\pgfsetlinewidth{1.003750pt}%
\definecolor{currentstroke}{rgb}{0.121569,0.466667,0.705882}%
\pgfsetstrokecolor{currentstroke}%
\pgfsetstrokeopacity{0.408016}%
\pgfsetdash{}{0pt}%
\pgfpathmoveto{\pgfqpoint{1.598023in}{1.847722in}}%
\pgfpathcurveto{\pgfqpoint{1.606259in}{1.847722in}}{\pgfqpoint{1.614159in}{1.850995in}}{\pgfqpoint{1.619983in}{1.856818in}}%
\pgfpathcurveto{\pgfqpoint{1.625807in}{1.862642in}}{\pgfqpoint{1.629079in}{1.870542in}}{\pgfqpoint{1.629079in}{1.878779in}}%
\pgfpathcurveto{\pgfqpoint{1.629079in}{1.887015in}}{\pgfqpoint{1.625807in}{1.894915in}}{\pgfqpoint{1.619983in}{1.900739in}}%
\pgfpathcurveto{\pgfqpoint{1.614159in}{1.906563in}}{\pgfqpoint{1.606259in}{1.909835in}}{\pgfqpoint{1.598023in}{1.909835in}}%
\pgfpathcurveto{\pgfqpoint{1.589786in}{1.909835in}}{\pgfqpoint{1.581886in}{1.906563in}}{\pgfqpoint{1.576062in}{1.900739in}}%
\pgfpathcurveto{\pgfqpoint{1.570238in}{1.894915in}}{\pgfqpoint{1.566966in}{1.887015in}}{\pgfqpoint{1.566966in}{1.878779in}}%
\pgfpathcurveto{\pgfqpoint{1.566966in}{1.870542in}}{\pgfqpoint{1.570238in}{1.862642in}}{\pgfqpoint{1.576062in}{1.856818in}}%
\pgfpathcurveto{\pgfqpoint{1.581886in}{1.850995in}}{\pgfqpoint{1.589786in}{1.847722in}}{\pgfqpoint{1.598023in}{1.847722in}}%
\pgfpathclose%
\pgfusepath{stroke,fill}%
\end{pgfscope}%
\begin{pgfscope}%
\pgfpathrectangle{\pgfqpoint{0.100000in}{0.212622in}}{\pgfqpoint{3.696000in}{3.696000in}}%
\pgfusepath{clip}%
\pgfsetbuttcap%
\pgfsetroundjoin%
\definecolor{currentfill}{rgb}{0.121569,0.466667,0.705882}%
\pgfsetfillcolor{currentfill}%
\pgfsetfillopacity{0.408143}%
\pgfsetlinewidth{1.003750pt}%
\definecolor{currentstroke}{rgb}{0.121569,0.466667,0.705882}%
\pgfsetstrokecolor{currentstroke}%
\pgfsetstrokeopacity{0.408143}%
\pgfsetdash{}{0pt}%
\pgfpathmoveto{\pgfqpoint{2.005466in}{1.942539in}}%
\pgfpathcurveto{\pgfqpoint{2.013702in}{1.942539in}}{\pgfqpoint{2.021602in}{1.945812in}}{\pgfqpoint{2.027426in}{1.951636in}}%
\pgfpathcurveto{\pgfqpoint{2.033250in}{1.957460in}}{\pgfqpoint{2.036523in}{1.965360in}}{\pgfqpoint{2.036523in}{1.973596in}}%
\pgfpathcurveto{\pgfqpoint{2.036523in}{1.981832in}}{\pgfqpoint{2.033250in}{1.989732in}}{\pgfqpoint{2.027426in}{1.995556in}}%
\pgfpathcurveto{\pgfqpoint{2.021602in}{2.001380in}}{\pgfqpoint{2.013702in}{2.004652in}}{\pgfqpoint{2.005466in}{2.004652in}}%
\pgfpathcurveto{\pgfqpoint{1.997230in}{2.004652in}}{\pgfqpoint{1.989330in}{2.001380in}}{\pgfqpoint{1.983506in}{1.995556in}}%
\pgfpathcurveto{\pgfqpoint{1.977682in}{1.989732in}}{\pgfqpoint{1.974410in}{1.981832in}}{\pgfqpoint{1.974410in}{1.973596in}}%
\pgfpathcurveto{\pgfqpoint{1.974410in}{1.965360in}}{\pgfqpoint{1.977682in}{1.957460in}}{\pgfqpoint{1.983506in}{1.951636in}}%
\pgfpathcurveto{\pgfqpoint{1.989330in}{1.945812in}}{\pgfqpoint{1.997230in}{1.942539in}}{\pgfqpoint{2.005466in}{1.942539in}}%
\pgfpathclose%
\pgfusepath{stroke,fill}%
\end{pgfscope}%
\begin{pgfscope}%
\pgfpathrectangle{\pgfqpoint{0.100000in}{0.212622in}}{\pgfqpoint{3.696000in}{3.696000in}}%
\pgfusepath{clip}%
\pgfsetbuttcap%
\pgfsetroundjoin%
\definecolor{currentfill}{rgb}{0.121569,0.466667,0.705882}%
\pgfsetfillcolor{currentfill}%
\pgfsetfillopacity{0.412603}%
\pgfsetlinewidth{1.003750pt}%
\definecolor{currentstroke}{rgb}{0.121569,0.466667,0.705882}%
\pgfsetstrokecolor{currentstroke}%
\pgfsetstrokeopacity{0.412603}%
\pgfsetdash{}{0pt}%
\pgfpathmoveto{\pgfqpoint{1.589172in}{1.848803in}}%
\pgfpathcurveto{\pgfqpoint{1.597409in}{1.848803in}}{\pgfqpoint{1.605309in}{1.852076in}}{\pgfqpoint{1.611133in}{1.857900in}}%
\pgfpathcurveto{\pgfqpoint{1.616956in}{1.863724in}}{\pgfqpoint{1.620229in}{1.871624in}}{\pgfqpoint{1.620229in}{1.879860in}}%
\pgfpathcurveto{\pgfqpoint{1.620229in}{1.888096in}}{\pgfqpoint{1.616956in}{1.895996in}}{\pgfqpoint{1.611133in}{1.901820in}}%
\pgfpathcurveto{\pgfqpoint{1.605309in}{1.907644in}}{\pgfqpoint{1.597409in}{1.910916in}}{\pgfqpoint{1.589172in}{1.910916in}}%
\pgfpathcurveto{\pgfqpoint{1.580936in}{1.910916in}}{\pgfqpoint{1.573036in}{1.907644in}}{\pgfqpoint{1.567212in}{1.901820in}}%
\pgfpathcurveto{\pgfqpoint{1.561388in}{1.895996in}}{\pgfqpoint{1.558116in}{1.888096in}}{\pgfqpoint{1.558116in}{1.879860in}}%
\pgfpathcurveto{\pgfqpoint{1.558116in}{1.871624in}}{\pgfqpoint{1.561388in}{1.863724in}}{\pgfqpoint{1.567212in}{1.857900in}}%
\pgfpathcurveto{\pgfqpoint{1.573036in}{1.852076in}}{\pgfqpoint{1.580936in}{1.848803in}}{\pgfqpoint{1.589172in}{1.848803in}}%
\pgfpathclose%
\pgfusepath{stroke,fill}%
\end{pgfscope}%
\begin{pgfscope}%
\pgfpathrectangle{\pgfqpoint{0.100000in}{0.212622in}}{\pgfqpoint{3.696000in}{3.696000in}}%
\pgfusepath{clip}%
\pgfsetbuttcap%
\pgfsetroundjoin%
\definecolor{currentfill}{rgb}{0.121569,0.466667,0.705882}%
\pgfsetfillcolor{currentfill}%
\pgfsetfillopacity{0.413679}%
\pgfsetlinewidth{1.003750pt}%
\definecolor{currentstroke}{rgb}{0.121569,0.466667,0.705882}%
\pgfsetstrokecolor{currentstroke}%
\pgfsetstrokeopacity{0.413679}%
\pgfsetdash{}{0pt}%
\pgfpathmoveto{\pgfqpoint{2.008475in}{1.939349in}}%
\pgfpathcurveto{\pgfqpoint{2.016711in}{1.939349in}}{\pgfqpoint{2.024612in}{1.942621in}}{\pgfqpoint{2.030435in}{1.948445in}}%
\pgfpathcurveto{\pgfqpoint{2.036259in}{1.954269in}}{\pgfqpoint{2.039532in}{1.962169in}}{\pgfqpoint{2.039532in}{1.970405in}}%
\pgfpathcurveto{\pgfqpoint{2.039532in}{1.978641in}}{\pgfqpoint{2.036259in}{1.986541in}}{\pgfqpoint{2.030435in}{1.992365in}}%
\pgfpathcurveto{\pgfqpoint{2.024612in}{1.998189in}}{\pgfqpoint{2.016711in}{2.001462in}}{\pgfqpoint{2.008475in}{2.001462in}}%
\pgfpathcurveto{\pgfqpoint{2.000239in}{2.001462in}}{\pgfqpoint{1.992339in}{1.998189in}}{\pgfqpoint{1.986515in}{1.992365in}}%
\pgfpathcurveto{\pgfqpoint{1.980691in}{1.986541in}}{\pgfqpoint{1.977419in}{1.978641in}}{\pgfqpoint{1.977419in}{1.970405in}}%
\pgfpathcurveto{\pgfqpoint{1.977419in}{1.962169in}}{\pgfqpoint{1.980691in}{1.954269in}}{\pgfqpoint{1.986515in}{1.948445in}}%
\pgfpathcurveto{\pgfqpoint{1.992339in}{1.942621in}}{\pgfqpoint{2.000239in}{1.939349in}}{\pgfqpoint{2.008475in}{1.939349in}}%
\pgfpathclose%
\pgfusepath{stroke,fill}%
\end{pgfscope}%
\begin{pgfscope}%
\pgfpathrectangle{\pgfqpoint{0.100000in}{0.212622in}}{\pgfqpoint{3.696000in}{3.696000in}}%
\pgfusepath{clip}%
\pgfsetbuttcap%
\pgfsetroundjoin%
\definecolor{currentfill}{rgb}{0.121569,0.466667,0.705882}%
\pgfsetfillcolor{currentfill}%
\pgfsetfillopacity{0.414631}%
\pgfsetlinewidth{1.003750pt}%
\definecolor{currentstroke}{rgb}{0.121569,0.466667,0.705882}%
\pgfsetstrokecolor{currentstroke}%
\pgfsetstrokeopacity{0.414631}%
\pgfsetdash{}{0pt}%
\pgfpathmoveto{\pgfqpoint{1.580538in}{1.844800in}}%
\pgfpathcurveto{\pgfqpoint{1.588774in}{1.844800in}}{\pgfqpoint{1.596674in}{1.848072in}}{\pgfqpoint{1.602498in}{1.853896in}}%
\pgfpathcurveto{\pgfqpoint{1.608322in}{1.859720in}}{\pgfqpoint{1.611594in}{1.867620in}}{\pgfqpoint{1.611594in}{1.875856in}}%
\pgfpathcurveto{\pgfqpoint{1.611594in}{1.884092in}}{\pgfqpoint{1.608322in}{1.891992in}}{\pgfqpoint{1.602498in}{1.897816in}}%
\pgfpathcurveto{\pgfqpoint{1.596674in}{1.903640in}}{\pgfqpoint{1.588774in}{1.906913in}}{\pgfqpoint{1.580538in}{1.906913in}}%
\pgfpathcurveto{\pgfqpoint{1.572301in}{1.906913in}}{\pgfqpoint{1.564401in}{1.903640in}}{\pgfqpoint{1.558578in}{1.897816in}}%
\pgfpathcurveto{\pgfqpoint{1.552754in}{1.891992in}}{\pgfqpoint{1.549481in}{1.884092in}}{\pgfqpoint{1.549481in}{1.875856in}}%
\pgfpathcurveto{\pgfqpoint{1.549481in}{1.867620in}}{\pgfqpoint{1.552754in}{1.859720in}}{\pgfqpoint{1.558578in}{1.853896in}}%
\pgfpathcurveto{\pgfqpoint{1.564401in}{1.848072in}}{\pgfqpoint{1.572301in}{1.844800in}}{\pgfqpoint{1.580538in}{1.844800in}}%
\pgfpathclose%
\pgfusepath{stroke,fill}%
\end{pgfscope}%
\begin{pgfscope}%
\pgfpathrectangle{\pgfqpoint{0.100000in}{0.212622in}}{\pgfqpoint{3.696000in}{3.696000in}}%
\pgfusepath{clip}%
\pgfsetbuttcap%
\pgfsetroundjoin%
\definecolor{currentfill}{rgb}{0.121569,0.466667,0.705882}%
\pgfsetfillcolor{currentfill}%
\pgfsetfillopacity{0.415564}%
\pgfsetlinewidth{1.003750pt}%
\definecolor{currentstroke}{rgb}{0.121569,0.466667,0.705882}%
\pgfsetstrokecolor{currentstroke}%
\pgfsetstrokeopacity{0.415564}%
\pgfsetdash{}{0pt}%
\pgfpathmoveto{\pgfqpoint{1.575400in}{1.839296in}}%
\pgfpathcurveto{\pgfqpoint{1.583636in}{1.839296in}}{\pgfqpoint{1.591536in}{1.842569in}}{\pgfqpoint{1.597360in}{1.848393in}}%
\pgfpathcurveto{\pgfqpoint{1.603184in}{1.854217in}}{\pgfqpoint{1.606456in}{1.862117in}}{\pgfqpoint{1.606456in}{1.870353in}}%
\pgfpathcurveto{\pgfqpoint{1.606456in}{1.878589in}}{\pgfqpoint{1.603184in}{1.886489in}}{\pgfqpoint{1.597360in}{1.892313in}}%
\pgfpathcurveto{\pgfqpoint{1.591536in}{1.898137in}}{\pgfqpoint{1.583636in}{1.901409in}}{\pgfqpoint{1.575400in}{1.901409in}}%
\pgfpathcurveto{\pgfqpoint{1.567164in}{1.901409in}}{\pgfqpoint{1.559263in}{1.898137in}}{\pgfqpoint{1.553440in}{1.892313in}}%
\pgfpathcurveto{\pgfqpoint{1.547616in}{1.886489in}}{\pgfqpoint{1.544343in}{1.878589in}}{\pgfqpoint{1.544343in}{1.870353in}}%
\pgfpathcurveto{\pgfqpoint{1.544343in}{1.862117in}}{\pgfqpoint{1.547616in}{1.854217in}}{\pgfqpoint{1.553440in}{1.848393in}}%
\pgfpathcurveto{\pgfqpoint{1.559263in}{1.842569in}}{\pgfqpoint{1.567164in}{1.839296in}}{\pgfqpoint{1.575400in}{1.839296in}}%
\pgfpathclose%
\pgfusepath{stroke,fill}%
\end{pgfscope}%
\begin{pgfscope}%
\pgfpathrectangle{\pgfqpoint{0.100000in}{0.212622in}}{\pgfqpoint{3.696000in}{3.696000in}}%
\pgfusepath{clip}%
\pgfsetbuttcap%
\pgfsetroundjoin%
\definecolor{currentfill}{rgb}{0.121569,0.466667,0.705882}%
\pgfsetfillcolor{currentfill}%
\pgfsetfillopacity{0.419382}%
\pgfsetlinewidth{1.003750pt}%
\definecolor{currentstroke}{rgb}{0.121569,0.466667,0.705882}%
\pgfsetstrokecolor{currentstroke}%
\pgfsetstrokeopacity{0.419382}%
\pgfsetdash{}{0pt}%
\pgfpathmoveto{\pgfqpoint{1.567635in}{1.841022in}}%
\pgfpathcurveto{\pgfqpoint{1.575871in}{1.841022in}}{\pgfqpoint{1.583771in}{1.844295in}}{\pgfqpoint{1.589595in}{1.850119in}}%
\pgfpathcurveto{\pgfqpoint{1.595419in}{1.855943in}}{\pgfqpoint{1.598691in}{1.863843in}}{\pgfqpoint{1.598691in}{1.872079in}}%
\pgfpathcurveto{\pgfqpoint{1.598691in}{1.880315in}}{\pgfqpoint{1.595419in}{1.888215in}}{\pgfqpoint{1.589595in}{1.894039in}}%
\pgfpathcurveto{\pgfqpoint{1.583771in}{1.899863in}}{\pgfqpoint{1.575871in}{1.903135in}}{\pgfqpoint{1.567635in}{1.903135in}}%
\pgfpathcurveto{\pgfqpoint{1.559398in}{1.903135in}}{\pgfqpoint{1.551498in}{1.899863in}}{\pgfqpoint{1.545674in}{1.894039in}}%
\pgfpathcurveto{\pgfqpoint{1.539851in}{1.888215in}}{\pgfqpoint{1.536578in}{1.880315in}}{\pgfqpoint{1.536578in}{1.872079in}}%
\pgfpathcurveto{\pgfqpoint{1.536578in}{1.863843in}}{\pgfqpoint{1.539851in}{1.855943in}}{\pgfqpoint{1.545674in}{1.850119in}}%
\pgfpathcurveto{\pgfqpoint{1.551498in}{1.844295in}}{\pgfqpoint{1.559398in}{1.841022in}}{\pgfqpoint{1.567635in}{1.841022in}}%
\pgfpathclose%
\pgfusepath{stroke,fill}%
\end{pgfscope}%
\begin{pgfscope}%
\pgfpathrectangle{\pgfqpoint{0.100000in}{0.212622in}}{\pgfqpoint{3.696000in}{3.696000in}}%
\pgfusepath{clip}%
\pgfsetbuttcap%
\pgfsetroundjoin%
\definecolor{currentfill}{rgb}{0.121569,0.466667,0.705882}%
\pgfsetfillcolor{currentfill}%
\pgfsetfillopacity{0.419431}%
\pgfsetlinewidth{1.003750pt}%
\definecolor{currentstroke}{rgb}{0.121569,0.466667,0.705882}%
\pgfsetstrokecolor{currentstroke}%
\pgfsetstrokeopacity{0.419431}%
\pgfsetdash{}{0pt}%
\pgfpathmoveto{\pgfqpoint{2.013129in}{1.936827in}}%
\pgfpathcurveto{\pgfqpoint{2.021365in}{1.936827in}}{\pgfqpoint{2.029265in}{1.940099in}}{\pgfqpoint{2.035089in}{1.945923in}}%
\pgfpathcurveto{\pgfqpoint{2.040913in}{1.951747in}}{\pgfqpoint{2.044185in}{1.959647in}}{\pgfqpoint{2.044185in}{1.967883in}}%
\pgfpathcurveto{\pgfqpoint{2.044185in}{1.976120in}}{\pgfqpoint{2.040913in}{1.984020in}}{\pgfqpoint{2.035089in}{1.989844in}}%
\pgfpathcurveto{\pgfqpoint{2.029265in}{1.995668in}}{\pgfqpoint{2.021365in}{1.998940in}}{\pgfqpoint{2.013129in}{1.998940in}}%
\pgfpathcurveto{\pgfqpoint{2.004892in}{1.998940in}}{\pgfqpoint{1.996992in}{1.995668in}}{\pgfqpoint{1.991168in}{1.989844in}}%
\pgfpathcurveto{\pgfqpoint{1.985344in}{1.984020in}}{\pgfqpoint{1.982072in}{1.976120in}}{\pgfqpoint{1.982072in}{1.967883in}}%
\pgfpathcurveto{\pgfqpoint{1.982072in}{1.959647in}}{\pgfqpoint{1.985344in}{1.951747in}}{\pgfqpoint{1.991168in}{1.945923in}}%
\pgfpathcurveto{\pgfqpoint{1.996992in}{1.940099in}}{\pgfqpoint{2.004892in}{1.936827in}}{\pgfqpoint{2.013129in}{1.936827in}}%
\pgfpathclose%
\pgfusepath{stroke,fill}%
\end{pgfscope}%
\begin{pgfscope}%
\pgfpathrectangle{\pgfqpoint{0.100000in}{0.212622in}}{\pgfqpoint{3.696000in}{3.696000in}}%
\pgfusepath{clip}%
\pgfsetbuttcap%
\pgfsetroundjoin%
\definecolor{currentfill}{rgb}{0.121569,0.466667,0.705882}%
\pgfsetfillcolor{currentfill}%
\pgfsetfillopacity{0.420660}%
\pgfsetlinewidth{1.003750pt}%
\definecolor{currentstroke}{rgb}{0.121569,0.466667,0.705882}%
\pgfsetstrokecolor{currentstroke}%
\pgfsetstrokeopacity{0.420660}%
\pgfsetdash{}{0pt}%
\pgfpathmoveto{\pgfqpoint{1.560704in}{1.837003in}}%
\pgfpathcurveto{\pgfqpoint{1.568941in}{1.837003in}}{\pgfqpoint{1.576841in}{1.840275in}}{\pgfqpoint{1.582665in}{1.846099in}}%
\pgfpathcurveto{\pgfqpoint{1.588488in}{1.851923in}}{\pgfqpoint{1.591761in}{1.859823in}}{\pgfqpoint{1.591761in}{1.868060in}}%
\pgfpathcurveto{\pgfqpoint{1.591761in}{1.876296in}}{\pgfqpoint{1.588488in}{1.884196in}}{\pgfqpoint{1.582665in}{1.890020in}}%
\pgfpathcurveto{\pgfqpoint{1.576841in}{1.895844in}}{\pgfqpoint{1.568941in}{1.899116in}}{\pgfqpoint{1.560704in}{1.899116in}}%
\pgfpathcurveto{\pgfqpoint{1.552468in}{1.899116in}}{\pgfqpoint{1.544568in}{1.895844in}}{\pgfqpoint{1.538744in}{1.890020in}}%
\pgfpathcurveto{\pgfqpoint{1.532920in}{1.884196in}}{\pgfqpoint{1.529648in}{1.876296in}}{\pgfqpoint{1.529648in}{1.868060in}}%
\pgfpathcurveto{\pgfqpoint{1.529648in}{1.859823in}}{\pgfqpoint{1.532920in}{1.851923in}}{\pgfqpoint{1.538744in}{1.846099in}}%
\pgfpathcurveto{\pgfqpoint{1.544568in}{1.840275in}}{\pgfqpoint{1.552468in}{1.837003in}}{\pgfqpoint{1.560704in}{1.837003in}}%
\pgfpathclose%
\pgfusepath{stroke,fill}%
\end{pgfscope}%
\begin{pgfscope}%
\pgfpathrectangle{\pgfqpoint{0.100000in}{0.212622in}}{\pgfqpoint{3.696000in}{3.696000in}}%
\pgfusepath{clip}%
\pgfsetbuttcap%
\pgfsetroundjoin%
\definecolor{currentfill}{rgb}{0.121569,0.466667,0.705882}%
\pgfsetfillcolor{currentfill}%
\pgfsetfillopacity{0.421128}%
\pgfsetlinewidth{1.003750pt}%
\definecolor{currentstroke}{rgb}{0.121569,0.466667,0.705882}%
\pgfsetstrokecolor{currentstroke}%
\pgfsetstrokeopacity{0.421128}%
\pgfsetdash{}{0pt}%
\pgfpathmoveto{\pgfqpoint{1.557232in}{1.833275in}}%
\pgfpathcurveto{\pgfqpoint{1.565468in}{1.833275in}}{\pgfqpoint{1.573368in}{1.836548in}}{\pgfqpoint{1.579192in}{1.842371in}}%
\pgfpathcurveto{\pgfqpoint{1.585016in}{1.848195in}}{\pgfqpoint{1.588288in}{1.856095in}}{\pgfqpoint{1.588288in}{1.864332in}}%
\pgfpathcurveto{\pgfqpoint{1.588288in}{1.872568in}}{\pgfqpoint{1.585016in}{1.880468in}}{\pgfqpoint{1.579192in}{1.886292in}}%
\pgfpathcurveto{\pgfqpoint{1.573368in}{1.892116in}}{\pgfqpoint{1.565468in}{1.895388in}}{\pgfqpoint{1.557232in}{1.895388in}}%
\pgfpathcurveto{\pgfqpoint{1.548996in}{1.895388in}}{\pgfqpoint{1.541096in}{1.892116in}}{\pgfqpoint{1.535272in}{1.886292in}}%
\pgfpathcurveto{\pgfqpoint{1.529448in}{1.880468in}}{\pgfqpoint{1.526175in}{1.872568in}}{\pgfqpoint{1.526175in}{1.864332in}}%
\pgfpathcurveto{\pgfqpoint{1.526175in}{1.856095in}}{\pgfqpoint{1.529448in}{1.848195in}}{\pgfqpoint{1.535272in}{1.842371in}}%
\pgfpathcurveto{\pgfqpoint{1.541096in}{1.836548in}}{\pgfqpoint{1.548996in}{1.833275in}}{\pgfqpoint{1.557232in}{1.833275in}}%
\pgfpathclose%
\pgfusepath{stroke,fill}%
\end{pgfscope}%
\begin{pgfscope}%
\pgfpathrectangle{\pgfqpoint{0.100000in}{0.212622in}}{\pgfqpoint{3.696000in}{3.696000in}}%
\pgfusepath{clip}%
\pgfsetbuttcap%
\pgfsetroundjoin%
\definecolor{currentfill}{rgb}{0.121569,0.466667,0.705882}%
\pgfsetfillcolor{currentfill}%
\pgfsetfillopacity{0.422392}%
\pgfsetlinewidth{1.003750pt}%
\definecolor{currentstroke}{rgb}{0.121569,0.466667,0.705882}%
\pgfsetstrokecolor{currentstroke}%
\pgfsetstrokeopacity{0.422392}%
\pgfsetdash{}{0pt}%
\pgfpathmoveto{\pgfqpoint{2.013597in}{1.933088in}}%
\pgfpathcurveto{\pgfqpoint{2.021833in}{1.933088in}}{\pgfqpoint{2.029733in}{1.936360in}}{\pgfqpoint{2.035557in}{1.942184in}}%
\pgfpathcurveto{\pgfqpoint{2.041381in}{1.948008in}}{\pgfqpoint{2.044653in}{1.955908in}}{\pgfqpoint{2.044653in}{1.964145in}}%
\pgfpathcurveto{\pgfqpoint{2.044653in}{1.972381in}}{\pgfqpoint{2.041381in}{1.980281in}}{\pgfqpoint{2.035557in}{1.986105in}}%
\pgfpathcurveto{\pgfqpoint{2.029733in}{1.991929in}}{\pgfqpoint{2.021833in}{1.995201in}}{\pgfqpoint{2.013597in}{1.995201in}}%
\pgfpathcurveto{\pgfqpoint{2.005360in}{1.995201in}}{\pgfqpoint{1.997460in}{1.991929in}}{\pgfqpoint{1.991636in}{1.986105in}}%
\pgfpathcurveto{\pgfqpoint{1.985812in}{1.980281in}}{\pgfqpoint{1.982540in}{1.972381in}}{\pgfqpoint{1.982540in}{1.964145in}}%
\pgfpathcurveto{\pgfqpoint{1.982540in}{1.955908in}}{\pgfqpoint{1.985812in}{1.948008in}}{\pgfqpoint{1.991636in}{1.942184in}}%
\pgfpathcurveto{\pgfqpoint{1.997460in}{1.936360in}}{\pgfqpoint{2.005360in}{1.933088in}}{\pgfqpoint{2.013597in}{1.933088in}}%
\pgfpathclose%
\pgfusepath{stroke,fill}%
\end{pgfscope}%
\begin{pgfscope}%
\pgfpathrectangle{\pgfqpoint{0.100000in}{0.212622in}}{\pgfqpoint{3.696000in}{3.696000in}}%
\pgfusepath{clip}%
\pgfsetbuttcap%
\pgfsetroundjoin%
\definecolor{currentfill}{rgb}{0.121569,0.466667,0.705882}%
\pgfsetfillcolor{currentfill}%
\pgfsetfillopacity{0.423387}%
\pgfsetlinewidth{1.003750pt}%
\definecolor{currentstroke}{rgb}{0.121569,0.466667,0.705882}%
\pgfsetstrokecolor{currentstroke}%
\pgfsetstrokeopacity{0.423387}%
\pgfsetdash{}{0pt}%
\pgfpathmoveto{\pgfqpoint{1.552667in}{1.833072in}}%
\pgfpathcurveto{\pgfqpoint{1.560903in}{1.833072in}}{\pgfqpoint{1.568803in}{1.836344in}}{\pgfqpoint{1.574627in}{1.842168in}}%
\pgfpathcurveto{\pgfqpoint{1.580451in}{1.847992in}}{\pgfqpoint{1.583723in}{1.855892in}}{\pgfqpoint{1.583723in}{1.864129in}}%
\pgfpathcurveto{\pgfqpoint{1.583723in}{1.872365in}}{\pgfqpoint{1.580451in}{1.880265in}}{\pgfqpoint{1.574627in}{1.886089in}}%
\pgfpathcurveto{\pgfqpoint{1.568803in}{1.891913in}}{\pgfqpoint{1.560903in}{1.895185in}}{\pgfqpoint{1.552667in}{1.895185in}}%
\pgfpathcurveto{\pgfqpoint{1.544430in}{1.895185in}}{\pgfqpoint{1.536530in}{1.891913in}}{\pgfqpoint{1.530706in}{1.886089in}}%
\pgfpathcurveto{\pgfqpoint{1.524882in}{1.880265in}}{\pgfqpoint{1.521610in}{1.872365in}}{\pgfqpoint{1.521610in}{1.864129in}}%
\pgfpathcurveto{\pgfqpoint{1.521610in}{1.855892in}}{\pgfqpoint{1.524882in}{1.847992in}}{\pgfqpoint{1.530706in}{1.842168in}}%
\pgfpathcurveto{\pgfqpoint{1.536530in}{1.836344in}}{\pgfqpoint{1.544430in}{1.833072in}}{\pgfqpoint{1.552667in}{1.833072in}}%
\pgfpathclose%
\pgfusepath{stroke,fill}%
\end{pgfscope}%
\begin{pgfscope}%
\pgfpathrectangle{\pgfqpoint{0.100000in}{0.212622in}}{\pgfqpoint{3.696000in}{3.696000in}}%
\pgfusepath{clip}%
\pgfsetbuttcap%
\pgfsetroundjoin%
\definecolor{currentfill}{rgb}{0.121569,0.466667,0.705882}%
\pgfsetfillcolor{currentfill}%
\pgfsetfillopacity{0.424020}%
\pgfsetlinewidth{1.003750pt}%
\definecolor{currentstroke}{rgb}{0.121569,0.466667,0.705882}%
\pgfsetstrokecolor{currentstroke}%
\pgfsetstrokeopacity{0.424020}%
\pgfsetdash{}{0pt}%
\pgfpathmoveto{\pgfqpoint{2.014848in}{1.931432in}}%
\pgfpathcurveto{\pgfqpoint{2.023084in}{1.931432in}}{\pgfqpoint{2.030984in}{1.934705in}}{\pgfqpoint{2.036808in}{1.940529in}}%
\pgfpathcurveto{\pgfqpoint{2.042632in}{1.946353in}}{\pgfqpoint{2.045904in}{1.954253in}}{\pgfqpoint{2.045904in}{1.962489in}}%
\pgfpathcurveto{\pgfqpoint{2.045904in}{1.970725in}}{\pgfqpoint{2.042632in}{1.978625in}}{\pgfqpoint{2.036808in}{1.984449in}}%
\pgfpathcurveto{\pgfqpoint{2.030984in}{1.990273in}}{\pgfqpoint{2.023084in}{1.993545in}}{\pgfqpoint{2.014848in}{1.993545in}}%
\pgfpathcurveto{\pgfqpoint{2.006612in}{1.993545in}}{\pgfqpoint{1.998712in}{1.990273in}}{\pgfqpoint{1.992888in}{1.984449in}}%
\pgfpathcurveto{\pgfqpoint{1.987064in}{1.978625in}}{\pgfqpoint{1.983791in}{1.970725in}}{\pgfqpoint{1.983791in}{1.962489in}}%
\pgfpathcurveto{\pgfqpoint{1.983791in}{1.954253in}}{\pgfqpoint{1.987064in}{1.946353in}}{\pgfqpoint{1.992888in}{1.940529in}}%
\pgfpathcurveto{\pgfqpoint{1.998712in}{1.934705in}}{\pgfqpoint{2.006612in}{1.931432in}}{\pgfqpoint{2.014848in}{1.931432in}}%
\pgfpathclose%
\pgfusepath{stroke,fill}%
\end{pgfscope}%
\begin{pgfscope}%
\pgfpathrectangle{\pgfqpoint{0.100000in}{0.212622in}}{\pgfqpoint{3.696000in}{3.696000in}}%
\pgfusepath{clip}%
\pgfsetbuttcap%
\pgfsetroundjoin%
\definecolor{currentfill}{rgb}{0.121569,0.466667,0.705882}%
\pgfsetfillcolor{currentfill}%
\pgfsetfillopacity{0.424105}%
\pgfsetlinewidth{1.003750pt}%
\definecolor{currentstroke}{rgb}{0.121569,0.466667,0.705882}%
\pgfsetstrokecolor{currentstroke}%
\pgfsetstrokeopacity{0.424105}%
\pgfsetdash{}{0pt}%
\pgfpathmoveto{\pgfqpoint{1.549667in}{1.831344in}}%
\pgfpathcurveto{\pgfqpoint{1.557903in}{1.831344in}}{\pgfqpoint{1.565803in}{1.834616in}}{\pgfqpoint{1.571627in}{1.840440in}}%
\pgfpathcurveto{\pgfqpoint{1.577451in}{1.846264in}}{\pgfqpoint{1.580723in}{1.854164in}}{\pgfqpoint{1.580723in}{1.862401in}}%
\pgfpathcurveto{\pgfqpoint{1.580723in}{1.870637in}}{\pgfqpoint{1.577451in}{1.878537in}}{\pgfqpoint{1.571627in}{1.884361in}}%
\pgfpathcurveto{\pgfqpoint{1.565803in}{1.890185in}}{\pgfqpoint{1.557903in}{1.893457in}}{\pgfqpoint{1.549667in}{1.893457in}}%
\pgfpathcurveto{\pgfqpoint{1.541430in}{1.893457in}}{\pgfqpoint{1.533530in}{1.890185in}}{\pgfqpoint{1.527706in}{1.884361in}}%
\pgfpathcurveto{\pgfqpoint{1.521882in}{1.878537in}}{\pgfqpoint{1.518610in}{1.870637in}}{\pgfqpoint{1.518610in}{1.862401in}}%
\pgfpathcurveto{\pgfqpoint{1.518610in}{1.854164in}}{\pgfqpoint{1.521882in}{1.846264in}}{\pgfqpoint{1.527706in}{1.840440in}}%
\pgfpathcurveto{\pgfqpoint{1.533530in}{1.834616in}}{\pgfqpoint{1.541430in}{1.831344in}}{\pgfqpoint{1.549667in}{1.831344in}}%
\pgfpathclose%
\pgfusepath{stroke,fill}%
\end{pgfscope}%
\begin{pgfscope}%
\pgfpathrectangle{\pgfqpoint{0.100000in}{0.212622in}}{\pgfqpoint{3.696000in}{3.696000in}}%
\pgfusepath{clip}%
\pgfsetbuttcap%
\pgfsetroundjoin%
\definecolor{currentfill}{rgb}{0.121569,0.466667,0.705882}%
\pgfsetfillcolor{currentfill}%
\pgfsetfillopacity{0.424910}%
\pgfsetlinewidth{1.003750pt}%
\definecolor{currentstroke}{rgb}{0.121569,0.466667,0.705882}%
\pgfsetstrokecolor{currentstroke}%
\pgfsetstrokeopacity{0.424910}%
\pgfsetdash{}{0pt}%
\pgfpathmoveto{\pgfqpoint{1.547186in}{1.830770in}}%
\pgfpathcurveto{\pgfqpoint{1.555422in}{1.830770in}}{\pgfqpoint{1.563322in}{1.834042in}}{\pgfqpoint{1.569146in}{1.839866in}}%
\pgfpathcurveto{\pgfqpoint{1.574970in}{1.845690in}}{\pgfqpoint{1.578242in}{1.853590in}}{\pgfqpoint{1.578242in}{1.861826in}}%
\pgfpathcurveto{\pgfqpoint{1.578242in}{1.870063in}}{\pgfqpoint{1.574970in}{1.877963in}}{\pgfqpoint{1.569146in}{1.883787in}}%
\pgfpathcurveto{\pgfqpoint{1.563322in}{1.889611in}}{\pgfqpoint{1.555422in}{1.892883in}}{\pgfqpoint{1.547186in}{1.892883in}}%
\pgfpathcurveto{\pgfqpoint{1.538950in}{1.892883in}}{\pgfqpoint{1.531049in}{1.889611in}}{\pgfqpoint{1.525226in}{1.883787in}}%
\pgfpathcurveto{\pgfqpoint{1.519402in}{1.877963in}}{\pgfqpoint{1.516129in}{1.870063in}}{\pgfqpoint{1.516129in}{1.861826in}}%
\pgfpathcurveto{\pgfqpoint{1.516129in}{1.853590in}}{\pgfqpoint{1.519402in}{1.845690in}}{\pgfqpoint{1.525226in}{1.839866in}}%
\pgfpathcurveto{\pgfqpoint{1.531049in}{1.834042in}}{\pgfqpoint{1.538950in}{1.830770in}}{\pgfqpoint{1.547186in}{1.830770in}}%
\pgfpathclose%
\pgfusepath{stroke,fill}%
\end{pgfscope}%
\begin{pgfscope}%
\pgfpathrectangle{\pgfqpoint{0.100000in}{0.212622in}}{\pgfqpoint{3.696000in}{3.696000in}}%
\pgfusepath{clip}%
\pgfsetbuttcap%
\pgfsetroundjoin%
\definecolor{currentfill}{rgb}{0.121569,0.466667,0.705882}%
\pgfsetfillcolor{currentfill}%
\pgfsetfillopacity{0.426102}%
\pgfsetlinewidth{1.003750pt}%
\definecolor{currentstroke}{rgb}{0.121569,0.466667,0.705882}%
\pgfsetstrokecolor{currentstroke}%
\pgfsetstrokeopacity{0.426102}%
\pgfsetdash{}{0pt}%
\pgfpathmoveto{\pgfqpoint{2.016367in}{1.930377in}}%
\pgfpathcurveto{\pgfqpoint{2.024604in}{1.930377in}}{\pgfqpoint{2.032504in}{1.933650in}}{\pgfqpoint{2.038328in}{1.939473in}}%
\pgfpathcurveto{\pgfqpoint{2.044152in}{1.945297in}}{\pgfqpoint{2.047424in}{1.953197in}}{\pgfqpoint{2.047424in}{1.961434in}}%
\pgfpathcurveto{\pgfqpoint{2.047424in}{1.969670in}}{\pgfqpoint{2.044152in}{1.977570in}}{\pgfqpoint{2.038328in}{1.983394in}}%
\pgfpathcurveto{\pgfqpoint{2.032504in}{1.989218in}}{\pgfqpoint{2.024604in}{1.992490in}}{\pgfqpoint{2.016367in}{1.992490in}}%
\pgfpathcurveto{\pgfqpoint{2.008131in}{1.992490in}}{\pgfqpoint{2.000231in}{1.989218in}}{\pgfqpoint{1.994407in}{1.983394in}}%
\pgfpathcurveto{\pgfqpoint{1.988583in}{1.977570in}}{\pgfqpoint{1.985311in}{1.969670in}}{\pgfqpoint{1.985311in}{1.961434in}}%
\pgfpathcurveto{\pgfqpoint{1.985311in}{1.953197in}}{\pgfqpoint{1.988583in}{1.945297in}}{\pgfqpoint{1.994407in}{1.939473in}}%
\pgfpathcurveto{\pgfqpoint{2.000231in}{1.933650in}}{\pgfqpoint{2.008131in}{1.930377in}}{\pgfqpoint{2.016367in}{1.930377in}}%
\pgfpathclose%
\pgfusepath{stroke,fill}%
\end{pgfscope}%
\begin{pgfscope}%
\pgfpathrectangle{\pgfqpoint{0.100000in}{0.212622in}}{\pgfqpoint{3.696000in}{3.696000in}}%
\pgfusepath{clip}%
\pgfsetbuttcap%
\pgfsetroundjoin%
\definecolor{currentfill}{rgb}{0.121569,0.466667,0.705882}%
\pgfsetfillcolor{currentfill}%
\pgfsetfillopacity{0.426497}%
\pgfsetlinewidth{1.003750pt}%
\definecolor{currentstroke}{rgb}{0.121569,0.466667,0.705882}%
\pgfsetstrokecolor{currentstroke}%
\pgfsetstrokeopacity{0.426497}%
\pgfsetdash{}{0pt}%
\pgfpathmoveto{\pgfqpoint{1.543274in}{1.829657in}}%
\pgfpathcurveto{\pgfqpoint{1.551510in}{1.829657in}}{\pgfqpoint{1.559410in}{1.832929in}}{\pgfqpoint{1.565234in}{1.838753in}}%
\pgfpathcurveto{\pgfqpoint{1.571058in}{1.844577in}}{\pgfqpoint{1.574330in}{1.852477in}}{\pgfqpoint{1.574330in}{1.860713in}}%
\pgfpathcurveto{\pgfqpoint{1.574330in}{1.868950in}}{\pgfqpoint{1.571058in}{1.876850in}}{\pgfqpoint{1.565234in}{1.882674in}}%
\pgfpathcurveto{\pgfqpoint{1.559410in}{1.888498in}}{\pgfqpoint{1.551510in}{1.891770in}}{\pgfqpoint{1.543274in}{1.891770in}}%
\pgfpathcurveto{\pgfqpoint{1.535037in}{1.891770in}}{\pgfqpoint{1.527137in}{1.888498in}}{\pgfqpoint{1.521313in}{1.882674in}}%
\pgfpathcurveto{\pgfqpoint{1.515489in}{1.876850in}}{\pgfqpoint{1.512217in}{1.868950in}}{\pgfqpoint{1.512217in}{1.860713in}}%
\pgfpathcurveto{\pgfqpoint{1.512217in}{1.852477in}}{\pgfqpoint{1.515489in}{1.844577in}}{\pgfqpoint{1.521313in}{1.838753in}}%
\pgfpathcurveto{\pgfqpoint{1.527137in}{1.832929in}}{\pgfqpoint{1.535037in}{1.829657in}}{\pgfqpoint{1.543274in}{1.829657in}}%
\pgfpathclose%
\pgfusepath{stroke,fill}%
\end{pgfscope}%
\begin{pgfscope}%
\pgfpathrectangle{\pgfqpoint{0.100000in}{0.212622in}}{\pgfqpoint{3.696000in}{3.696000in}}%
\pgfusepath{clip}%
\pgfsetbuttcap%
\pgfsetroundjoin%
\definecolor{currentfill}{rgb}{0.121569,0.466667,0.705882}%
\pgfsetfillcolor{currentfill}%
\pgfsetfillopacity{0.427283}%
\pgfsetlinewidth{1.003750pt}%
\definecolor{currentstroke}{rgb}{0.121569,0.466667,0.705882}%
\pgfsetstrokecolor{currentstroke}%
\pgfsetstrokeopacity{0.427283}%
\pgfsetdash{}{0pt}%
\pgfpathmoveto{\pgfqpoint{1.540130in}{1.828272in}}%
\pgfpathcurveto{\pgfqpoint{1.548367in}{1.828272in}}{\pgfqpoint{1.556267in}{1.831545in}}{\pgfqpoint{1.562091in}{1.837369in}}%
\pgfpathcurveto{\pgfqpoint{1.567915in}{1.843192in}}{\pgfqpoint{1.571187in}{1.851092in}}{\pgfqpoint{1.571187in}{1.859329in}}%
\pgfpathcurveto{\pgfqpoint{1.571187in}{1.867565in}}{\pgfqpoint{1.567915in}{1.875465in}}{\pgfqpoint{1.562091in}{1.881289in}}%
\pgfpathcurveto{\pgfqpoint{1.556267in}{1.887113in}}{\pgfqpoint{1.548367in}{1.890385in}}{\pgfqpoint{1.540130in}{1.890385in}}%
\pgfpathcurveto{\pgfqpoint{1.531894in}{1.890385in}}{\pgfqpoint{1.523994in}{1.887113in}}{\pgfqpoint{1.518170in}{1.881289in}}%
\pgfpathcurveto{\pgfqpoint{1.512346in}{1.875465in}}{\pgfqpoint{1.509074in}{1.867565in}}{\pgfqpoint{1.509074in}{1.859329in}}%
\pgfpathcurveto{\pgfqpoint{1.509074in}{1.851092in}}{\pgfqpoint{1.512346in}{1.843192in}}{\pgfqpoint{1.518170in}{1.837369in}}%
\pgfpathcurveto{\pgfqpoint{1.523994in}{1.831545in}}{\pgfqpoint{1.531894in}{1.828272in}}{\pgfqpoint{1.540130in}{1.828272in}}%
\pgfpathclose%
\pgfusepath{stroke,fill}%
\end{pgfscope}%
\begin{pgfscope}%
\pgfpathrectangle{\pgfqpoint{0.100000in}{0.212622in}}{\pgfqpoint{3.696000in}{3.696000in}}%
\pgfusepath{clip}%
\pgfsetbuttcap%
\pgfsetroundjoin%
\definecolor{currentfill}{rgb}{0.121569,0.466667,0.705882}%
\pgfsetfillcolor{currentfill}%
\pgfsetfillopacity{0.428124}%
\pgfsetlinewidth{1.003750pt}%
\definecolor{currentstroke}{rgb}{0.121569,0.466667,0.705882}%
\pgfsetstrokecolor{currentstroke}%
\pgfsetstrokeopacity{0.428124}%
\pgfsetdash{}{0pt}%
\pgfpathmoveto{\pgfqpoint{1.538025in}{1.828015in}}%
\pgfpathcurveto{\pgfqpoint{1.546261in}{1.828015in}}{\pgfqpoint{1.554161in}{1.831287in}}{\pgfqpoint{1.559985in}{1.837111in}}%
\pgfpathcurveto{\pgfqpoint{1.565809in}{1.842935in}}{\pgfqpoint{1.569082in}{1.850835in}}{\pgfqpoint{1.569082in}{1.859071in}}%
\pgfpathcurveto{\pgfqpoint{1.569082in}{1.867308in}}{\pgfqpoint{1.565809in}{1.875208in}}{\pgfqpoint{1.559985in}{1.881032in}}%
\pgfpathcurveto{\pgfqpoint{1.554161in}{1.886855in}}{\pgfqpoint{1.546261in}{1.890128in}}{\pgfqpoint{1.538025in}{1.890128in}}%
\pgfpathcurveto{\pgfqpoint{1.529789in}{1.890128in}}{\pgfqpoint{1.521889in}{1.886855in}}{\pgfqpoint{1.516065in}{1.881032in}}%
\pgfpathcurveto{\pgfqpoint{1.510241in}{1.875208in}}{\pgfqpoint{1.506969in}{1.867308in}}{\pgfqpoint{1.506969in}{1.859071in}}%
\pgfpathcurveto{\pgfqpoint{1.506969in}{1.850835in}}{\pgfqpoint{1.510241in}{1.842935in}}{\pgfqpoint{1.516065in}{1.837111in}}%
\pgfpathcurveto{\pgfqpoint{1.521889in}{1.831287in}}{\pgfqpoint{1.529789in}{1.828015in}}{\pgfqpoint{1.538025in}{1.828015in}}%
\pgfpathclose%
\pgfusepath{stroke,fill}%
\end{pgfscope}%
\begin{pgfscope}%
\pgfpathrectangle{\pgfqpoint{0.100000in}{0.212622in}}{\pgfqpoint{3.696000in}{3.696000in}}%
\pgfusepath{clip}%
\pgfsetbuttcap%
\pgfsetroundjoin%
\definecolor{currentfill}{rgb}{0.121569,0.466667,0.705882}%
\pgfsetfillcolor{currentfill}%
\pgfsetfillopacity{0.428606}%
\pgfsetlinewidth{1.003750pt}%
\definecolor{currentstroke}{rgb}{0.121569,0.466667,0.705882}%
\pgfsetstrokecolor{currentstroke}%
\pgfsetstrokeopacity{0.428606}%
\pgfsetdash{}{0pt}%
\pgfpathmoveto{\pgfqpoint{2.017032in}{1.925783in}}%
\pgfpathcurveto{\pgfqpoint{2.025269in}{1.925783in}}{\pgfqpoint{2.033169in}{1.929055in}}{\pgfqpoint{2.038993in}{1.934879in}}%
\pgfpathcurveto{\pgfqpoint{2.044817in}{1.940703in}}{\pgfqpoint{2.048089in}{1.948603in}}{\pgfqpoint{2.048089in}{1.956839in}}%
\pgfpathcurveto{\pgfqpoint{2.048089in}{1.965075in}}{\pgfqpoint{2.044817in}{1.972976in}}{\pgfqpoint{2.038993in}{1.978799in}}%
\pgfpathcurveto{\pgfqpoint{2.033169in}{1.984623in}}{\pgfqpoint{2.025269in}{1.987896in}}{\pgfqpoint{2.017032in}{1.987896in}}%
\pgfpathcurveto{\pgfqpoint{2.008796in}{1.987896in}}{\pgfqpoint{2.000896in}{1.984623in}}{\pgfqpoint{1.995072in}{1.978799in}}%
\pgfpathcurveto{\pgfqpoint{1.989248in}{1.972976in}}{\pgfqpoint{1.985976in}{1.965075in}}{\pgfqpoint{1.985976in}{1.956839in}}%
\pgfpathcurveto{\pgfqpoint{1.985976in}{1.948603in}}{\pgfqpoint{1.989248in}{1.940703in}}{\pgfqpoint{1.995072in}{1.934879in}}%
\pgfpathcurveto{\pgfqpoint{2.000896in}{1.929055in}}{\pgfqpoint{2.008796in}{1.925783in}}{\pgfqpoint{2.017032in}{1.925783in}}%
\pgfpathclose%
\pgfusepath{stroke,fill}%
\end{pgfscope}%
\begin{pgfscope}%
\pgfpathrectangle{\pgfqpoint{0.100000in}{0.212622in}}{\pgfqpoint{3.696000in}{3.696000in}}%
\pgfusepath{clip}%
\pgfsetbuttcap%
\pgfsetroundjoin%
\definecolor{currentfill}{rgb}{0.121569,0.466667,0.705882}%
\pgfsetfillcolor{currentfill}%
\pgfsetfillopacity{0.429520}%
\pgfsetlinewidth{1.003750pt}%
\definecolor{currentstroke}{rgb}{0.121569,0.466667,0.705882}%
\pgfsetstrokecolor{currentstroke}%
\pgfsetstrokeopacity{0.429520}%
\pgfsetdash{}{0pt}%
\pgfpathmoveto{\pgfqpoint{1.534550in}{1.826200in}}%
\pgfpathcurveto{\pgfqpoint{1.542787in}{1.826200in}}{\pgfqpoint{1.550687in}{1.829472in}}{\pgfqpoint{1.556511in}{1.835296in}}%
\pgfpathcurveto{\pgfqpoint{1.562335in}{1.841120in}}{\pgfqpoint{1.565607in}{1.849020in}}{\pgfqpoint{1.565607in}{1.857256in}}%
\pgfpathcurveto{\pgfqpoint{1.565607in}{1.865492in}}{\pgfqpoint{1.562335in}{1.873392in}}{\pgfqpoint{1.556511in}{1.879216in}}%
\pgfpathcurveto{\pgfqpoint{1.550687in}{1.885040in}}{\pgfqpoint{1.542787in}{1.888313in}}{\pgfqpoint{1.534550in}{1.888313in}}%
\pgfpathcurveto{\pgfqpoint{1.526314in}{1.888313in}}{\pgfqpoint{1.518414in}{1.885040in}}{\pgfqpoint{1.512590in}{1.879216in}}%
\pgfpathcurveto{\pgfqpoint{1.506766in}{1.873392in}}{\pgfqpoint{1.503494in}{1.865492in}}{\pgfqpoint{1.503494in}{1.857256in}}%
\pgfpathcurveto{\pgfqpoint{1.503494in}{1.849020in}}{\pgfqpoint{1.506766in}{1.841120in}}{\pgfqpoint{1.512590in}{1.835296in}}%
\pgfpathcurveto{\pgfqpoint{1.518414in}{1.829472in}}{\pgfqpoint{1.526314in}{1.826200in}}{\pgfqpoint{1.534550in}{1.826200in}}%
\pgfpathclose%
\pgfusepath{stroke,fill}%
\end{pgfscope}%
\begin{pgfscope}%
\pgfpathrectangle{\pgfqpoint{0.100000in}{0.212622in}}{\pgfqpoint{3.696000in}{3.696000in}}%
\pgfusepath{clip}%
\pgfsetbuttcap%
\pgfsetroundjoin%
\definecolor{currentfill}{rgb}{0.121569,0.466667,0.705882}%
\pgfsetfillcolor{currentfill}%
\pgfsetfillopacity{0.430036}%
\pgfsetlinewidth{1.003750pt}%
\definecolor{currentstroke}{rgb}{0.121569,0.466667,0.705882}%
\pgfsetstrokecolor{currentstroke}%
\pgfsetstrokeopacity{0.430036}%
\pgfsetdash{}{0pt}%
\pgfpathmoveto{\pgfqpoint{1.532297in}{1.824885in}}%
\pgfpathcurveto{\pgfqpoint{1.540533in}{1.824885in}}{\pgfqpoint{1.548434in}{1.828157in}}{\pgfqpoint{1.554257in}{1.833981in}}%
\pgfpathcurveto{\pgfqpoint{1.560081in}{1.839805in}}{\pgfqpoint{1.563354in}{1.847705in}}{\pgfqpoint{1.563354in}{1.855941in}}%
\pgfpathcurveto{\pgfqpoint{1.563354in}{1.864178in}}{\pgfqpoint{1.560081in}{1.872078in}}{\pgfqpoint{1.554257in}{1.877902in}}%
\pgfpathcurveto{\pgfqpoint{1.548434in}{1.883726in}}{\pgfqpoint{1.540533in}{1.886998in}}{\pgfqpoint{1.532297in}{1.886998in}}%
\pgfpathcurveto{\pgfqpoint{1.524061in}{1.886998in}}{\pgfqpoint{1.516161in}{1.883726in}}{\pgfqpoint{1.510337in}{1.877902in}}%
\pgfpathcurveto{\pgfqpoint{1.504513in}{1.872078in}}{\pgfqpoint{1.501241in}{1.864178in}}{\pgfqpoint{1.501241in}{1.855941in}}%
\pgfpathcurveto{\pgfqpoint{1.501241in}{1.847705in}}{\pgfqpoint{1.504513in}{1.839805in}}{\pgfqpoint{1.510337in}{1.833981in}}%
\pgfpathcurveto{\pgfqpoint{1.516161in}{1.828157in}}{\pgfqpoint{1.524061in}{1.824885in}}{\pgfqpoint{1.532297in}{1.824885in}}%
\pgfpathclose%
\pgfusepath{stroke,fill}%
\end{pgfscope}%
\begin{pgfscope}%
\pgfpathrectangle{\pgfqpoint{0.100000in}{0.212622in}}{\pgfqpoint{3.696000in}{3.696000in}}%
\pgfusepath{clip}%
\pgfsetbuttcap%
\pgfsetroundjoin%
\definecolor{currentfill}{rgb}{0.121569,0.466667,0.705882}%
\pgfsetfillcolor{currentfill}%
\pgfsetfillopacity{0.430588}%
\pgfsetlinewidth{1.003750pt}%
\definecolor{currentstroke}{rgb}{0.121569,0.466667,0.705882}%
\pgfsetstrokecolor{currentstroke}%
\pgfsetstrokeopacity{0.430588}%
\pgfsetdash{}{0pt}%
\pgfpathmoveto{\pgfqpoint{1.530796in}{1.824609in}}%
\pgfpathcurveto{\pgfqpoint{1.539032in}{1.824609in}}{\pgfqpoint{1.546933in}{1.827882in}}{\pgfqpoint{1.552756in}{1.833706in}}%
\pgfpathcurveto{\pgfqpoint{1.558580in}{1.839529in}}{\pgfqpoint{1.561853in}{1.847430in}}{\pgfqpoint{1.561853in}{1.855666in}}%
\pgfpathcurveto{\pgfqpoint{1.561853in}{1.863902in}}{\pgfqpoint{1.558580in}{1.871802in}}{\pgfqpoint{1.552756in}{1.877626in}}%
\pgfpathcurveto{\pgfqpoint{1.546933in}{1.883450in}}{\pgfqpoint{1.539032in}{1.886722in}}{\pgfqpoint{1.530796in}{1.886722in}}%
\pgfpathcurveto{\pgfqpoint{1.522560in}{1.886722in}}{\pgfqpoint{1.514660in}{1.883450in}}{\pgfqpoint{1.508836in}{1.877626in}}%
\pgfpathcurveto{\pgfqpoint{1.503012in}{1.871802in}}{\pgfqpoint{1.499740in}{1.863902in}}{\pgfqpoint{1.499740in}{1.855666in}}%
\pgfpathcurveto{\pgfqpoint{1.499740in}{1.847430in}}{\pgfqpoint{1.503012in}{1.839529in}}{\pgfqpoint{1.508836in}{1.833706in}}%
\pgfpathcurveto{\pgfqpoint{1.514660in}{1.827882in}}{\pgfqpoint{1.522560in}{1.824609in}}{\pgfqpoint{1.530796in}{1.824609in}}%
\pgfpathclose%
\pgfusepath{stroke,fill}%
\end{pgfscope}%
\begin{pgfscope}%
\pgfpathrectangle{\pgfqpoint{0.100000in}{0.212622in}}{\pgfqpoint{3.696000in}{3.696000in}}%
\pgfusepath{clip}%
\pgfsetbuttcap%
\pgfsetroundjoin%
\definecolor{currentfill}{rgb}{0.121569,0.466667,0.705882}%
\pgfsetfillcolor{currentfill}%
\pgfsetfillopacity{0.431486}%
\pgfsetlinewidth{1.003750pt}%
\definecolor{currentstroke}{rgb}{0.121569,0.466667,0.705882}%
\pgfsetstrokecolor{currentstroke}%
\pgfsetstrokeopacity{0.431486}%
\pgfsetdash{}{0pt}%
\pgfpathmoveto{\pgfqpoint{2.018394in}{1.922010in}}%
\pgfpathcurveto{\pgfqpoint{2.026630in}{1.922010in}}{\pgfqpoint{2.034530in}{1.925282in}}{\pgfqpoint{2.040354in}{1.931106in}}%
\pgfpathcurveto{\pgfqpoint{2.046178in}{1.936930in}}{\pgfqpoint{2.049450in}{1.944830in}}{\pgfqpoint{2.049450in}{1.953066in}}%
\pgfpathcurveto{\pgfqpoint{2.049450in}{1.961302in}}{\pgfqpoint{2.046178in}{1.969203in}}{\pgfqpoint{2.040354in}{1.975026in}}%
\pgfpathcurveto{\pgfqpoint{2.034530in}{1.980850in}}{\pgfqpoint{2.026630in}{1.984123in}}{\pgfqpoint{2.018394in}{1.984123in}}%
\pgfpathcurveto{\pgfqpoint{2.010157in}{1.984123in}}{\pgfqpoint{2.002257in}{1.980850in}}{\pgfqpoint{1.996433in}{1.975026in}}%
\pgfpathcurveto{\pgfqpoint{1.990609in}{1.969203in}}{\pgfqpoint{1.987337in}{1.961302in}}{\pgfqpoint{1.987337in}{1.953066in}}%
\pgfpathcurveto{\pgfqpoint{1.987337in}{1.944830in}}{\pgfqpoint{1.990609in}{1.936930in}}{\pgfqpoint{1.996433in}{1.931106in}}%
\pgfpathcurveto{\pgfqpoint{2.002257in}{1.925282in}}{\pgfqpoint{2.010157in}{1.922010in}}{\pgfqpoint{2.018394in}{1.922010in}}%
\pgfpathclose%
\pgfusepath{stroke,fill}%
\end{pgfscope}%
\begin{pgfscope}%
\pgfpathrectangle{\pgfqpoint{0.100000in}{0.212622in}}{\pgfqpoint{3.696000in}{3.696000in}}%
\pgfusepath{clip}%
\pgfsetbuttcap%
\pgfsetroundjoin%
\definecolor{currentfill}{rgb}{0.121569,0.466667,0.705882}%
\pgfsetfillcolor{currentfill}%
\pgfsetfillopacity{0.431692}%
\pgfsetlinewidth{1.003750pt}%
\definecolor{currentstroke}{rgb}{0.121569,0.466667,0.705882}%
\pgfsetstrokecolor{currentstroke}%
\pgfsetstrokeopacity{0.431692}%
\pgfsetdash{}{0pt}%
\pgfpathmoveto{\pgfqpoint{1.528426in}{1.824290in}}%
\pgfpathcurveto{\pgfqpoint{1.536662in}{1.824290in}}{\pgfqpoint{1.544562in}{1.827562in}}{\pgfqpoint{1.550386in}{1.833386in}}%
\pgfpathcurveto{\pgfqpoint{1.556210in}{1.839210in}}{\pgfqpoint{1.559482in}{1.847110in}}{\pgfqpoint{1.559482in}{1.855346in}}%
\pgfpathcurveto{\pgfqpoint{1.559482in}{1.863583in}}{\pgfqpoint{1.556210in}{1.871483in}}{\pgfqpoint{1.550386in}{1.877307in}}%
\pgfpathcurveto{\pgfqpoint{1.544562in}{1.883131in}}{\pgfqpoint{1.536662in}{1.886403in}}{\pgfqpoint{1.528426in}{1.886403in}}%
\pgfpathcurveto{\pgfqpoint{1.520189in}{1.886403in}}{\pgfqpoint{1.512289in}{1.883131in}}{\pgfqpoint{1.506465in}{1.877307in}}%
\pgfpathcurveto{\pgfqpoint{1.500641in}{1.871483in}}{\pgfqpoint{1.497369in}{1.863583in}}{\pgfqpoint{1.497369in}{1.855346in}}%
\pgfpathcurveto{\pgfqpoint{1.497369in}{1.847110in}}{\pgfqpoint{1.500641in}{1.839210in}}{\pgfqpoint{1.506465in}{1.833386in}}%
\pgfpathcurveto{\pgfqpoint{1.512289in}{1.827562in}}{\pgfqpoint{1.520189in}{1.824290in}}{\pgfqpoint{1.528426in}{1.824290in}}%
\pgfpathclose%
\pgfusepath{stroke,fill}%
\end{pgfscope}%
\begin{pgfscope}%
\pgfpathrectangle{\pgfqpoint{0.100000in}{0.212622in}}{\pgfqpoint{3.696000in}{3.696000in}}%
\pgfusepath{clip}%
\pgfsetbuttcap%
\pgfsetroundjoin%
\definecolor{currentfill}{rgb}{0.121569,0.466667,0.705882}%
\pgfsetfillcolor{currentfill}%
\pgfsetfillopacity{0.431955}%
\pgfsetlinewidth{1.003750pt}%
\definecolor{currentstroke}{rgb}{0.121569,0.466667,0.705882}%
\pgfsetstrokecolor{currentstroke}%
\pgfsetstrokeopacity{0.431955}%
\pgfsetdash{}{0pt}%
\pgfpathmoveto{\pgfqpoint{1.527285in}{1.823570in}}%
\pgfpathcurveto{\pgfqpoint{1.535521in}{1.823570in}}{\pgfqpoint{1.543421in}{1.826842in}}{\pgfqpoint{1.549245in}{1.832666in}}%
\pgfpathcurveto{\pgfqpoint{1.555069in}{1.838490in}}{\pgfqpoint{1.558341in}{1.846390in}}{\pgfqpoint{1.558341in}{1.854627in}}%
\pgfpathcurveto{\pgfqpoint{1.558341in}{1.862863in}}{\pgfqpoint{1.555069in}{1.870763in}}{\pgfqpoint{1.549245in}{1.876587in}}%
\pgfpathcurveto{\pgfqpoint{1.543421in}{1.882411in}}{\pgfqpoint{1.535521in}{1.885683in}}{\pgfqpoint{1.527285in}{1.885683in}}%
\pgfpathcurveto{\pgfqpoint{1.519049in}{1.885683in}}{\pgfqpoint{1.511148in}{1.882411in}}{\pgfqpoint{1.505325in}{1.876587in}}%
\pgfpathcurveto{\pgfqpoint{1.499501in}{1.870763in}}{\pgfqpoint{1.496228in}{1.862863in}}{\pgfqpoint{1.496228in}{1.854627in}}%
\pgfpathcurveto{\pgfqpoint{1.496228in}{1.846390in}}{\pgfqpoint{1.499501in}{1.838490in}}{\pgfqpoint{1.505325in}{1.832666in}}%
\pgfpathcurveto{\pgfqpoint{1.511148in}{1.826842in}}{\pgfqpoint{1.519049in}{1.823570in}}{\pgfqpoint{1.527285in}{1.823570in}}%
\pgfpathclose%
\pgfusepath{stroke,fill}%
\end{pgfscope}%
\begin{pgfscope}%
\pgfpathrectangle{\pgfqpoint{0.100000in}{0.212622in}}{\pgfqpoint{3.696000in}{3.696000in}}%
\pgfusepath{clip}%
\pgfsetbuttcap%
\pgfsetroundjoin%
\definecolor{currentfill}{rgb}{0.121569,0.466667,0.705882}%
\pgfsetfillcolor{currentfill}%
\pgfsetfillopacity{0.432592}%
\pgfsetlinewidth{1.003750pt}%
\definecolor{currentstroke}{rgb}{0.121569,0.466667,0.705882}%
\pgfsetstrokecolor{currentstroke}%
\pgfsetstrokeopacity{0.432592}%
\pgfsetdash{}{0pt}%
\pgfpathmoveto{\pgfqpoint{1.525417in}{1.822957in}}%
\pgfpathcurveto{\pgfqpoint{1.533653in}{1.822957in}}{\pgfqpoint{1.541553in}{1.826229in}}{\pgfqpoint{1.547377in}{1.832053in}}%
\pgfpathcurveto{\pgfqpoint{1.553201in}{1.837877in}}{\pgfqpoint{1.556473in}{1.845777in}}{\pgfqpoint{1.556473in}{1.854013in}}%
\pgfpathcurveto{\pgfqpoint{1.556473in}{1.862249in}}{\pgfqpoint{1.553201in}{1.870150in}}{\pgfqpoint{1.547377in}{1.875973in}}%
\pgfpathcurveto{\pgfqpoint{1.541553in}{1.881797in}}{\pgfqpoint{1.533653in}{1.885070in}}{\pgfqpoint{1.525417in}{1.885070in}}%
\pgfpathcurveto{\pgfqpoint{1.517180in}{1.885070in}}{\pgfqpoint{1.509280in}{1.881797in}}{\pgfqpoint{1.503456in}{1.875973in}}%
\pgfpathcurveto{\pgfqpoint{1.497632in}{1.870150in}}{\pgfqpoint{1.494360in}{1.862249in}}{\pgfqpoint{1.494360in}{1.854013in}}%
\pgfpathcurveto{\pgfqpoint{1.494360in}{1.845777in}}{\pgfqpoint{1.497632in}{1.837877in}}{\pgfqpoint{1.503456in}{1.832053in}}%
\pgfpathcurveto{\pgfqpoint{1.509280in}{1.826229in}}{\pgfqpoint{1.517180in}{1.822957in}}{\pgfqpoint{1.525417in}{1.822957in}}%
\pgfpathclose%
\pgfusepath{stroke,fill}%
\end{pgfscope}%
\begin{pgfscope}%
\pgfpathrectangle{\pgfqpoint{0.100000in}{0.212622in}}{\pgfqpoint{3.696000in}{3.696000in}}%
\pgfusepath{clip}%
\pgfsetbuttcap%
\pgfsetroundjoin%
\definecolor{currentfill}{rgb}{0.121569,0.466667,0.705882}%
\pgfsetfillcolor{currentfill}%
\pgfsetfillopacity{0.433970}%
\pgfsetlinewidth{1.003750pt}%
\definecolor{currentstroke}{rgb}{0.121569,0.466667,0.705882}%
\pgfsetstrokecolor{currentstroke}%
\pgfsetstrokeopacity{0.433970}%
\pgfsetdash{}{0pt}%
\pgfpathmoveto{\pgfqpoint{1.522436in}{1.822737in}}%
\pgfpathcurveto{\pgfqpoint{1.530672in}{1.822737in}}{\pgfqpoint{1.538572in}{1.826010in}}{\pgfqpoint{1.544396in}{1.831834in}}%
\pgfpathcurveto{\pgfqpoint{1.550220in}{1.837658in}}{\pgfqpoint{1.553492in}{1.845558in}}{\pgfqpoint{1.553492in}{1.853794in}}%
\pgfpathcurveto{\pgfqpoint{1.553492in}{1.862030in}}{\pgfqpoint{1.550220in}{1.869930in}}{\pgfqpoint{1.544396in}{1.875754in}}%
\pgfpathcurveto{\pgfqpoint{1.538572in}{1.881578in}}{\pgfqpoint{1.530672in}{1.884850in}}{\pgfqpoint{1.522436in}{1.884850in}}%
\pgfpathcurveto{\pgfqpoint{1.514199in}{1.884850in}}{\pgfqpoint{1.506299in}{1.881578in}}{\pgfqpoint{1.500475in}{1.875754in}}%
\pgfpathcurveto{\pgfqpoint{1.494652in}{1.869930in}}{\pgfqpoint{1.491379in}{1.862030in}}{\pgfqpoint{1.491379in}{1.853794in}}%
\pgfpathcurveto{\pgfqpoint{1.491379in}{1.845558in}}{\pgfqpoint{1.494652in}{1.837658in}}{\pgfqpoint{1.500475in}{1.831834in}}%
\pgfpathcurveto{\pgfqpoint{1.506299in}{1.826010in}}{\pgfqpoint{1.514199in}{1.822737in}}{\pgfqpoint{1.522436in}{1.822737in}}%
\pgfpathclose%
\pgfusepath{stroke,fill}%
\end{pgfscope}%
\begin{pgfscope}%
\pgfpathrectangle{\pgfqpoint{0.100000in}{0.212622in}}{\pgfqpoint{3.696000in}{3.696000in}}%
\pgfusepath{clip}%
\pgfsetbuttcap%
\pgfsetroundjoin%
\definecolor{currentfill}{rgb}{0.121569,0.466667,0.705882}%
\pgfsetfillcolor{currentfill}%
\pgfsetfillopacity{0.434273}%
\pgfsetlinewidth{1.003750pt}%
\definecolor{currentstroke}{rgb}{0.121569,0.466667,0.705882}%
\pgfsetstrokecolor{currentstroke}%
\pgfsetstrokeopacity{0.434273}%
\pgfsetdash{}{0pt}%
\pgfpathmoveto{\pgfqpoint{1.521007in}{1.822191in}}%
\pgfpathcurveto{\pgfqpoint{1.529243in}{1.822191in}}{\pgfqpoint{1.537143in}{1.825463in}}{\pgfqpoint{1.542967in}{1.831287in}}%
\pgfpathcurveto{\pgfqpoint{1.548791in}{1.837111in}}{\pgfqpoint{1.552063in}{1.845011in}}{\pgfqpoint{1.552063in}{1.853247in}}%
\pgfpathcurveto{\pgfqpoint{1.552063in}{1.861483in}}{\pgfqpoint{1.548791in}{1.869384in}}{\pgfqpoint{1.542967in}{1.875207in}}%
\pgfpathcurveto{\pgfqpoint{1.537143in}{1.881031in}}{\pgfqpoint{1.529243in}{1.884304in}}{\pgfqpoint{1.521007in}{1.884304in}}%
\pgfpathcurveto{\pgfqpoint{1.512771in}{1.884304in}}{\pgfqpoint{1.504871in}{1.881031in}}{\pgfqpoint{1.499047in}{1.875207in}}%
\pgfpathcurveto{\pgfqpoint{1.493223in}{1.869384in}}{\pgfqpoint{1.489950in}{1.861483in}}{\pgfqpoint{1.489950in}{1.853247in}}%
\pgfpathcurveto{\pgfqpoint{1.489950in}{1.845011in}}{\pgfqpoint{1.493223in}{1.837111in}}{\pgfqpoint{1.499047in}{1.831287in}}%
\pgfpathcurveto{\pgfqpoint{1.504871in}{1.825463in}}{\pgfqpoint{1.512771in}{1.822191in}}{\pgfqpoint{1.521007in}{1.822191in}}%
\pgfpathclose%
\pgfusepath{stroke,fill}%
\end{pgfscope}%
\begin{pgfscope}%
\pgfpathrectangle{\pgfqpoint{0.100000in}{0.212622in}}{\pgfqpoint{3.696000in}{3.696000in}}%
\pgfusepath{clip}%
\pgfsetbuttcap%
\pgfsetroundjoin%
\definecolor{currentfill}{rgb}{0.121569,0.466667,0.705882}%
\pgfsetfillcolor{currentfill}%
\pgfsetfillopacity{0.434672}%
\pgfsetlinewidth{1.003750pt}%
\definecolor{currentstroke}{rgb}{0.121569,0.466667,0.705882}%
\pgfsetstrokecolor{currentstroke}%
\pgfsetstrokeopacity{0.434672}%
\pgfsetdash{}{0pt}%
\pgfpathmoveto{\pgfqpoint{1.518726in}{1.819545in}}%
\pgfpathcurveto{\pgfqpoint{1.526962in}{1.819545in}}{\pgfqpoint{1.534862in}{1.822817in}}{\pgfqpoint{1.540686in}{1.828641in}}%
\pgfpathcurveto{\pgfqpoint{1.546510in}{1.834465in}}{\pgfqpoint{1.549782in}{1.842365in}}{\pgfqpoint{1.549782in}{1.850602in}}%
\pgfpathcurveto{\pgfqpoint{1.549782in}{1.858838in}}{\pgfqpoint{1.546510in}{1.866738in}}{\pgfqpoint{1.540686in}{1.872562in}}%
\pgfpathcurveto{\pgfqpoint{1.534862in}{1.878386in}}{\pgfqpoint{1.526962in}{1.881658in}}{\pgfqpoint{1.518726in}{1.881658in}}%
\pgfpathcurveto{\pgfqpoint{1.510489in}{1.881658in}}{\pgfqpoint{1.502589in}{1.878386in}}{\pgfqpoint{1.496765in}{1.872562in}}%
\pgfpathcurveto{\pgfqpoint{1.490942in}{1.866738in}}{\pgfqpoint{1.487669in}{1.858838in}}{\pgfqpoint{1.487669in}{1.850602in}}%
\pgfpathcurveto{\pgfqpoint{1.487669in}{1.842365in}}{\pgfqpoint{1.490942in}{1.834465in}}{\pgfqpoint{1.496765in}{1.828641in}}%
\pgfpathcurveto{\pgfqpoint{1.502589in}{1.822817in}}{\pgfqpoint{1.510489in}{1.819545in}}{\pgfqpoint{1.518726in}{1.819545in}}%
\pgfpathclose%
\pgfusepath{stroke,fill}%
\end{pgfscope}%
\begin{pgfscope}%
\pgfpathrectangle{\pgfqpoint{0.100000in}{0.212622in}}{\pgfqpoint{3.696000in}{3.696000in}}%
\pgfusepath{clip}%
\pgfsetbuttcap%
\pgfsetroundjoin%
\definecolor{currentfill}{rgb}{0.121569,0.466667,0.705882}%
\pgfsetfillcolor{currentfill}%
\pgfsetfillopacity{0.434755}%
\pgfsetlinewidth{1.003750pt}%
\definecolor{currentstroke}{rgb}{0.121569,0.466667,0.705882}%
\pgfsetstrokecolor{currentstroke}%
\pgfsetstrokeopacity{0.434755}%
\pgfsetdash{}{0pt}%
\pgfpathmoveto{\pgfqpoint{2.021019in}{1.918801in}}%
\pgfpathcurveto{\pgfqpoint{2.029255in}{1.918801in}}{\pgfqpoint{2.037155in}{1.922073in}}{\pgfqpoint{2.042979in}{1.927897in}}%
\pgfpathcurveto{\pgfqpoint{2.048803in}{1.933721in}}{\pgfqpoint{2.052075in}{1.941621in}}{\pgfqpoint{2.052075in}{1.949857in}}%
\pgfpathcurveto{\pgfqpoint{2.052075in}{1.958094in}}{\pgfqpoint{2.048803in}{1.965994in}}{\pgfqpoint{2.042979in}{1.971818in}}%
\pgfpathcurveto{\pgfqpoint{2.037155in}{1.977642in}}{\pgfqpoint{2.029255in}{1.980914in}}{\pgfqpoint{2.021019in}{1.980914in}}%
\pgfpathcurveto{\pgfqpoint{2.012782in}{1.980914in}}{\pgfqpoint{2.004882in}{1.977642in}}{\pgfqpoint{1.999058in}{1.971818in}}%
\pgfpathcurveto{\pgfqpoint{1.993235in}{1.965994in}}{\pgfqpoint{1.989962in}{1.958094in}}{\pgfqpoint{1.989962in}{1.949857in}}%
\pgfpathcurveto{\pgfqpoint{1.989962in}{1.941621in}}{\pgfqpoint{1.993235in}{1.933721in}}{\pgfqpoint{1.999058in}{1.927897in}}%
\pgfpathcurveto{\pgfqpoint{2.004882in}{1.922073in}}{\pgfqpoint{2.012782in}{1.918801in}}{\pgfqpoint{2.021019in}{1.918801in}}%
\pgfpathclose%
\pgfusepath{stroke,fill}%
\end{pgfscope}%
\begin{pgfscope}%
\pgfpathrectangle{\pgfqpoint{0.100000in}{0.212622in}}{\pgfqpoint{3.696000in}{3.696000in}}%
\pgfusepath{clip}%
\pgfsetbuttcap%
\pgfsetroundjoin%
\definecolor{currentfill}{rgb}{0.121569,0.466667,0.705882}%
\pgfsetfillcolor{currentfill}%
\pgfsetfillopacity{0.436174}%
\pgfsetlinewidth{1.003750pt}%
\definecolor{currentstroke}{rgb}{0.121569,0.466667,0.705882}%
\pgfsetstrokecolor{currentstroke}%
\pgfsetstrokeopacity{0.436174}%
\pgfsetdash{}{0pt}%
\pgfpathmoveto{\pgfqpoint{1.515105in}{1.819058in}}%
\pgfpathcurveto{\pgfqpoint{1.523342in}{1.819058in}}{\pgfqpoint{1.531242in}{1.822331in}}{\pgfqpoint{1.537066in}{1.828155in}}%
\pgfpathcurveto{\pgfqpoint{1.542890in}{1.833979in}}{\pgfqpoint{1.546162in}{1.841879in}}{\pgfqpoint{1.546162in}{1.850115in}}%
\pgfpathcurveto{\pgfqpoint{1.546162in}{1.858351in}}{\pgfqpoint{1.542890in}{1.866251in}}{\pgfqpoint{1.537066in}{1.872075in}}%
\pgfpathcurveto{\pgfqpoint{1.531242in}{1.877899in}}{\pgfqpoint{1.523342in}{1.881171in}}{\pgfqpoint{1.515105in}{1.881171in}}%
\pgfpathcurveto{\pgfqpoint{1.506869in}{1.881171in}}{\pgfqpoint{1.498969in}{1.877899in}}{\pgfqpoint{1.493145in}{1.872075in}}%
\pgfpathcurveto{\pgfqpoint{1.487321in}{1.866251in}}{\pgfqpoint{1.484049in}{1.858351in}}{\pgfqpoint{1.484049in}{1.850115in}}%
\pgfpathcurveto{\pgfqpoint{1.484049in}{1.841879in}}{\pgfqpoint{1.487321in}{1.833979in}}{\pgfqpoint{1.493145in}{1.828155in}}%
\pgfpathcurveto{\pgfqpoint{1.498969in}{1.822331in}}{\pgfqpoint{1.506869in}{1.819058in}}{\pgfqpoint{1.515105in}{1.819058in}}%
\pgfpathclose%
\pgfusepath{stroke,fill}%
\end{pgfscope}%
\begin{pgfscope}%
\pgfpathrectangle{\pgfqpoint{0.100000in}{0.212622in}}{\pgfqpoint{3.696000in}{3.696000in}}%
\pgfusepath{clip}%
\pgfsetbuttcap%
\pgfsetroundjoin%
\definecolor{currentfill}{rgb}{0.121569,0.466667,0.705882}%
\pgfsetfillcolor{currentfill}%
\pgfsetfillopacity{0.436592}%
\pgfsetlinewidth{1.003750pt}%
\definecolor{currentstroke}{rgb}{0.121569,0.466667,0.705882}%
\pgfsetstrokecolor{currentstroke}%
\pgfsetstrokeopacity{0.436592}%
\pgfsetdash{}{0pt}%
\pgfpathmoveto{\pgfqpoint{1.513151in}{1.817299in}}%
\pgfpathcurveto{\pgfqpoint{1.521388in}{1.817299in}}{\pgfqpoint{1.529288in}{1.820571in}}{\pgfqpoint{1.535112in}{1.826395in}}%
\pgfpathcurveto{\pgfqpoint{1.540936in}{1.832219in}}{\pgfqpoint{1.544208in}{1.840119in}}{\pgfqpoint{1.544208in}{1.848355in}}%
\pgfpathcurveto{\pgfqpoint{1.544208in}{1.856592in}}{\pgfqpoint{1.540936in}{1.864492in}}{\pgfqpoint{1.535112in}{1.870316in}}%
\pgfpathcurveto{\pgfqpoint{1.529288in}{1.876139in}}{\pgfqpoint{1.521388in}{1.879412in}}{\pgfqpoint{1.513151in}{1.879412in}}%
\pgfpathcurveto{\pgfqpoint{1.504915in}{1.879412in}}{\pgfqpoint{1.497015in}{1.876139in}}{\pgfqpoint{1.491191in}{1.870316in}}%
\pgfpathcurveto{\pgfqpoint{1.485367in}{1.864492in}}{\pgfqpoint{1.482095in}{1.856592in}}{\pgfqpoint{1.482095in}{1.848355in}}%
\pgfpathcurveto{\pgfqpoint{1.482095in}{1.840119in}}{\pgfqpoint{1.485367in}{1.832219in}}{\pgfqpoint{1.491191in}{1.826395in}}%
\pgfpathcurveto{\pgfqpoint{1.497015in}{1.820571in}}{\pgfqpoint{1.504915in}{1.817299in}}{\pgfqpoint{1.513151in}{1.817299in}}%
\pgfpathclose%
\pgfusepath{stroke,fill}%
\end{pgfscope}%
\begin{pgfscope}%
\pgfpathrectangle{\pgfqpoint{0.100000in}{0.212622in}}{\pgfqpoint{3.696000in}{3.696000in}}%
\pgfusepath{clip}%
\pgfsetbuttcap%
\pgfsetroundjoin%
\definecolor{currentfill}{rgb}{0.121569,0.466667,0.705882}%
\pgfsetfillcolor{currentfill}%
\pgfsetfillopacity{0.436878}%
\pgfsetlinewidth{1.003750pt}%
\definecolor{currentstroke}{rgb}{0.121569,0.466667,0.705882}%
\pgfsetstrokecolor{currentstroke}%
\pgfsetstrokeopacity{0.436878}%
\pgfsetdash{}{0pt}%
\pgfpathmoveto{\pgfqpoint{2.022096in}{1.919018in}}%
\pgfpathcurveto{\pgfqpoint{2.030332in}{1.919018in}}{\pgfqpoint{2.038232in}{1.922290in}}{\pgfqpoint{2.044056in}{1.928114in}}%
\pgfpathcurveto{\pgfqpoint{2.049880in}{1.933938in}}{\pgfqpoint{2.053152in}{1.941838in}}{\pgfqpoint{2.053152in}{1.950074in}}%
\pgfpathcurveto{\pgfqpoint{2.053152in}{1.958310in}}{\pgfqpoint{2.049880in}{1.966210in}}{\pgfqpoint{2.044056in}{1.972034in}}%
\pgfpathcurveto{\pgfqpoint{2.038232in}{1.977858in}}{\pgfqpoint{2.030332in}{1.981131in}}{\pgfqpoint{2.022096in}{1.981131in}}%
\pgfpathcurveto{\pgfqpoint{2.013859in}{1.981131in}}{\pgfqpoint{2.005959in}{1.977858in}}{\pgfqpoint{2.000135in}{1.972034in}}%
\pgfpathcurveto{\pgfqpoint{1.994312in}{1.966210in}}{\pgfqpoint{1.991039in}{1.958310in}}{\pgfqpoint{1.991039in}{1.950074in}}%
\pgfpathcurveto{\pgfqpoint{1.991039in}{1.941838in}}{\pgfqpoint{1.994312in}{1.933938in}}{\pgfqpoint{2.000135in}{1.928114in}}%
\pgfpathcurveto{\pgfqpoint{2.005959in}{1.922290in}}{\pgfqpoint{2.013859in}{1.919018in}}{\pgfqpoint{2.022096in}{1.919018in}}%
\pgfpathclose%
\pgfusepath{stroke,fill}%
\end{pgfscope}%
\begin{pgfscope}%
\pgfpathrectangle{\pgfqpoint{0.100000in}{0.212622in}}{\pgfqpoint{3.696000in}{3.696000in}}%
\pgfusepath{clip}%
\pgfsetbuttcap%
\pgfsetroundjoin%
\definecolor{currentfill}{rgb}{0.121569,0.466667,0.705882}%
\pgfsetfillcolor{currentfill}%
\pgfsetfillopacity{0.437702}%
\pgfsetlinewidth{1.003750pt}%
\definecolor{currentstroke}{rgb}{0.121569,0.466667,0.705882}%
\pgfsetstrokecolor{currentstroke}%
\pgfsetstrokeopacity{0.437702}%
\pgfsetdash{}{0pt}%
\pgfpathmoveto{\pgfqpoint{1.509741in}{1.816165in}}%
\pgfpathcurveto{\pgfqpoint{1.517978in}{1.816165in}}{\pgfqpoint{1.525878in}{1.819437in}}{\pgfqpoint{1.531702in}{1.825261in}}%
\pgfpathcurveto{\pgfqpoint{1.537526in}{1.831085in}}{\pgfqpoint{1.540798in}{1.838985in}}{\pgfqpoint{1.540798in}{1.847221in}}%
\pgfpathcurveto{\pgfqpoint{1.540798in}{1.855457in}}{\pgfqpoint{1.537526in}{1.863357in}}{\pgfqpoint{1.531702in}{1.869181in}}%
\pgfpathcurveto{\pgfqpoint{1.525878in}{1.875005in}}{\pgfqpoint{1.517978in}{1.878278in}}{\pgfqpoint{1.509741in}{1.878278in}}%
\pgfpathcurveto{\pgfqpoint{1.501505in}{1.878278in}}{\pgfqpoint{1.493605in}{1.875005in}}{\pgfqpoint{1.487781in}{1.869181in}}%
\pgfpathcurveto{\pgfqpoint{1.481957in}{1.863357in}}{\pgfqpoint{1.478685in}{1.855457in}}{\pgfqpoint{1.478685in}{1.847221in}}%
\pgfpathcurveto{\pgfqpoint{1.478685in}{1.838985in}}{\pgfqpoint{1.481957in}{1.831085in}}{\pgfqpoint{1.487781in}{1.825261in}}%
\pgfpathcurveto{\pgfqpoint{1.493605in}{1.819437in}}{\pgfqpoint{1.501505in}{1.816165in}}{\pgfqpoint{1.509741in}{1.816165in}}%
\pgfpathclose%
\pgfusepath{stroke,fill}%
\end{pgfscope}%
\begin{pgfscope}%
\pgfpathrectangle{\pgfqpoint{0.100000in}{0.212622in}}{\pgfqpoint{3.696000in}{3.696000in}}%
\pgfusepath{clip}%
\pgfsetbuttcap%
\pgfsetroundjoin%
\definecolor{currentfill}{rgb}{0.121569,0.466667,0.705882}%
\pgfsetfillcolor{currentfill}%
\pgfsetfillopacity{0.439182}%
\pgfsetlinewidth{1.003750pt}%
\definecolor{currentstroke}{rgb}{0.121569,0.466667,0.705882}%
\pgfsetstrokecolor{currentstroke}%
\pgfsetstrokeopacity{0.439182}%
\pgfsetdash{}{0pt}%
\pgfpathmoveto{\pgfqpoint{2.022231in}{1.917579in}}%
\pgfpathcurveto{\pgfqpoint{2.030467in}{1.917579in}}{\pgfqpoint{2.038367in}{1.920851in}}{\pgfqpoint{2.044191in}{1.926675in}}%
\pgfpathcurveto{\pgfqpoint{2.050015in}{1.932499in}}{\pgfqpoint{2.053287in}{1.940399in}}{\pgfqpoint{2.053287in}{1.948636in}}%
\pgfpathcurveto{\pgfqpoint{2.053287in}{1.956872in}}{\pgfqpoint{2.050015in}{1.964772in}}{\pgfqpoint{2.044191in}{1.970596in}}%
\pgfpathcurveto{\pgfqpoint{2.038367in}{1.976420in}}{\pgfqpoint{2.030467in}{1.979692in}}{\pgfqpoint{2.022231in}{1.979692in}}%
\pgfpathcurveto{\pgfqpoint{2.013994in}{1.979692in}}{\pgfqpoint{2.006094in}{1.976420in}}{\pgfqpoint{2.000270in}{1.970596in}}%
\pgfpathcurveto{\pgfqpoint{1.994446in}{1.964772in}}{\pgfqpoint{1.991174in}{1.956872in}}{\pgfqpoint{1.991174in}{1.948636in}}%
\pgfpathcurveto{\pgfqpoint{1.991174in}{1.940399in}}{\pgfqpoint{1.994446in}{1.932499in}}{\pgfqpoint{2.000270in}{1.926675in}}%
\pgfpathcurveto{\pgfqpoint{2.006094in}{1.920851in}}{\pgfqpoint{2.013994in}{1.917579in}}{\pgfqpoint{2.022231in}{1.917579in}}%
\pgfpathclose%
\pgfusepath{stroke,fill}%
\end{pgfscope}%
\begin{pgfscope}%
\pgfpathrectangle{\pgfqpoint{0.100000in}{0.212622in}}{\pgfqpoint{3.696000in}{3.696000in}}%
\pgfusepath{clip}%
\pgfsetbuttcap%
\pgfsetroundjoin%
\definecolor{currentfill}{rgb}{0.121569,0.466667,0.705882}%
\pgfsetfillcolor{currentfill}%
\pgfsetfillopacity{0.440163}%
\pgfsetlinewidth{1.003750pt}%
\definecolor{currentstroke}{rgb}{0.121569,0.466667,0.705882}%
\pgfsetstrokecolor{currentstroke}%
\pgfsetstrokeopacity{0.440163}%
\pgfsetdash{}{0pt}%
\pgfpathmoveto{\pgfqpoint{1.504335in}{1.815899in}}%
\pgfpathcurveto{\pgfqpoint{1.512571in}{1.815899in}}{\pgfqpoint{1.520471in}{1.819172in}}{\pgfqpoint{1.526295in}{1.824996in}}%
\pgfpathcurveto{\pgfqpoint{1.532119in}{1.830820in}}{\pgfqpoint{1.535391in}{1.838720in}}{\pgfqpoint{1.535391in}{1.846956in}}%
\pgfpathcurveto{\pgfqpoint{1.535391in}{1.855192in}}{\pgfqpoint{1.532119in}{1.863092in}}{\pgfqpoint{1.526295in}{1.868916in}}%
\pgfpathcurveto{\pgfqpoint{1.520471in}{1.874740in}}{\pgfqpoint{1.512571in}{1.878012in}}{\pgfqpoint{1.504335in}{1.878012in}}%
\pgfpathcurveto{\pgfqpoint{1.496099in}{1.878012in}}{\pgfqpoint{1.488199in}{1.874740in}}{\pgfqpoint{1.482375in}{1.868916in}}%
\pgfpathcurveto{\pgfqpoint{1.476551in}{1.863092in}}{\pgfqpoint{1.473278in}{1.855192in}}{\pgfqpoint{1.473278in}{1.846956in}}%
\pgfpathcurveto{\pgfqpoint{1.473278in}{1.838720in}}{\pgfqpoint{1.476551in}{1.830820in}}{\pgfqpoint{1.482375in}{1.824996in}}%
\pgfpathcurveto{\pgfqpoint{1.488199in}{1.819172in}}{\pgfqpoint{1.496099in}{1.815899in}}{\pgfqpoint{1.504335in}{1.815899in}}%
\pgfpathclose%
\pgfusepath{stroke,fill}%
\end{pgfscope}%
\begin{pgfscope}%
\pgfpathrectangle{\pgfqpoint{0.100000in}{0.212622in}}{\pgfqpoint{3.696000in}{3.696000in}}%
\pgfusepath{clip}%
\pgfsetbuttcap%
\pgfsetroundjoin%
\definecolor{currentfill}{rgb}{0.121569,0.466667,0.705882}%
\pgfsetfillcolor{currentfill}%
\pgfsetfillopacity{0.441321}%
\pgfsetlinewidth{1.003750pt}%
\definecolor{currentstroke}{rgb}{0.121569,0.466667,0.705882}%
\pgfsetstrokecolor{currentstroke}%
\pgfsetstrokeopacity{0.441321}%
\pgfsetdash{}{0pt}%
\pgfpathmoveto{\pgfqpoint{1.499083in}{1.812868in}}%
\pgfpathcurveto{\pgfqpoint{1.507319in}{1.812868in}}{\pgfqpoint{1.515219in}{1.816140in}}{\pgfqpoint{1.521043in}{1.821964in}}%
\pgfpathcurveto{\pgfqpoint{1.526867in}{1.827788in}}{\pgfqpoint{1.530139in}{1.835688in}}{\pgfqpoint{1.530139in}{1.843925in}}%
\pgfpathcurveto{\pgfqpoint{1.530139in}{1.852161in}}{\pgfqpoint{1.526867in}{1.860061in}}{\pgfqpoint{1.521043in}{1.865885in}}%
\pgfpathcurveto{\pgfqpoint{1.515219in}{1.871709in}}{\pgfqpoint{1.507319in}{1.874981in}}{\pgfqpoint{1.499083in}{1.874981in}}%
\pgfpathcurveto{\pgfqpoint{1.490846in}{1.874981in}}{\pgfqpoint{1.482946in}{1.871709in}}{\pgfqpoint{1.477122in}{1.865885in}}%
\pgfpathcurveto{\pgfqpoint{1.471299in}{1.860061in}}{\pgfqpoint{1.468026in}{1.852161in}}{\pgfqpoint{1.468026in}{1.843925in}}%
\pgfpathcurveto{\pgfqpoint{1.468026in}{1.835688in}}{\pgfqpoint{1.471299in}{1.827788in}}{\pgfqpoint{1.477122in}{1.821964in}}%
\pgfpathcurveto{\pgfqpoint{1.482946in}{1.816140in}}{\pgfqpoint{1.490846in}{1.812868in}}{\pgfqpoint{1.499083in}{1.812868in}}%
\pgfpathclose%
\pgfusepath{stroke,fill}%
\end{pgfscope}%
\begin{pgfscope}%
\pgfpathrectangle{\pgfqpoint{0.100000in}{0.212622in}}{\pgfqpoint{3.696000in}{3.696000in}}%
\pgfusepath{clip}%
\pgfsetbuttcap%
\pgfsetroundjoin%
\definecolor{currentfill}{rgb}{0.121569,0.466667,0.705882}%
\pgfsetfillcolor{currentfill}%
\pgfsetfillopacity{0.441800}%
\pgfsetlinewidth{1.003750pt}%
\definecolor{currentstroke}{rgb}{0.121569,0.466667,0.705882}%
\pgfsetstrokecolor{currentstroke}%
\pgfsetstrokeopacity{0.441800}%
\pgfsetdash{}{0pt}%
\pgfpathmoveto{\pgfqpoint{2.022693in}{1.914775in}}%
\pgfpathcurveto{\pgfqpoint{2.030929in}{1.914775in}}{\pgfqpoint{2.038829in}{1.918047in}}{\pgfqpoint{2.044653in}{1.923871in}}%
\pgfpathcurveto{\pgfqpoint{2.050477in}{1.929695in}}{\pgfqpoint{2.053749in}{1.937595in}}{\pgfqpoint{2.053749in}{1.945831in}}%
\pgfpathcurveto{\pgfqpoint{2.053749in}{1.954068in}}{\pgfqpoint{2.050477in}{1.961968in}}{\pgfqpoint{2.044653in}{1.967792in}}%
\pgfpathcurveto{\pgfqpoint{2.038829in}{1.973615in}}{\pgfqpoint{2.030929in}{1.976888in}}{\pgfqpoint{2.022693in}{1.976888in}}%
\pgfpathcurveto{\pgfqpoint{2.014457in}{1.976888in}}{\pgfqpoint{2.006557in}{1.973615in}}{\pgfqpoint{2.000733in}{1.967792in}}%
\pgfpathcurveto{\pgfqpoint{1.994909in}{1.961968in}}{\pgfqpoint{1.991636in}{1.954068in}}{\pgfqpoint{1.991636in}{1.945831in}}%
\pgfpathcurveto{\pgfqpoint{1.991636in}{1.937595in}}{\pgfqpoint{1.994909in}{1.929695in}}{\pgfqpoint{2.000733in}{1.923871in}}%
\pgfpathcurveto{\pgfqpoint{2.006557in}{1.918047in}}{\pgfqpoint{2.014457in}{1.914775in}}{\pgfqpoint{2.022693in}{1.914775in}}%
\pgfpathclose%
\pgfusepath{stroke,fill}%
\end{pgfscope}%
\begin{pgfscope}%
\pgfpathrectangle{\pgfqpoint{0.100000in}{0.212622in}}{\pgfqpoint{3.696000in}{3.696000in}}%
\pgfusepath{clip}%
\pgfsetbuttcap%
\pgfsetroundjoin%
\definecolor{currentfill}{rgb}{0.121569,0.466667,0.705882}%
\pgfsetfillcolor{currentfill}%
\pgfsetfillopacity{0.442433}%
\pgfsetlinewidth{1.003750pt}%
\definecolor{currentstroke}{rgb}{0.121569,0.466667,0.705882}%
\pgfsetstrokecolor{currentstroke}%
\pgfsetstrokeopacity{0.442433}%
\pgfsetdash{}{0pt}%
\pgfpathmoveto{\pgfqpoint{1.495338in}{1.810587in}}%
\pgfpathcurveto{\pgfqpoint{1.503574in}{1.810587in}}{\pgfqpoint{1.511474in}{1.813860in}}{\pgfqpoint{1.517298in}{1.819684in}}%
\pgfpathcurveto{\pgfqpoint{1.523122in}{1.825508in}}{\pgfqpoint{1.526395in}{1.833408in}}{\pgfqpoint{1.526395in}{1.841644in}}%
\pgfpathcurveto{\pgfqpoint{1.526395in}{1.849880in}}{\pgfqpoint{1.523122in}{1.857780in}}{\pgfqpoint{1.517298in}{1.863604in}}%
\pgfpathcurveto{\pgfqpoint{1.511474in}{1.869428in}}{\pgfqpoint{1.503574in}{1.872700in}}{\pgfqpoint{1.495338in}{1.872700in}}%
\pgfpathcurveto{\pgfqpoint{1.487102in}{1.872700in}}{\pgfqpoint{1.479202in}{1.869428in}}{\pgfqpoint{1.473378in}{1.863604in}}%
\pgfpathcurveto{\pgfqpoint{1.467554in}{1.857780in}}{\pgfqpoint{1.464282in}{1.849880in}}{\pgfqpoint{1.464282in}{1.841644in}}%
\pgfpathcurveto{\pgfqpoint{1.464282in}{1.833408in}}{\pgfqpoint{1.467554in}{1.825508in}}{\pgfqpoint{1.473378in}{1.819684in}}%
\pgfpathcurveto{\pgfqpoint{1.479202in}{1.813860in}}{\pgfqpoint{1.487102in}{1.810587in}}{\pgfqpoint{1.495338in}{1.810587in}}%
\pgfpathclose%
\pgfusepath{stroke,fill}%
\end{pgfscope}%
\begin{pgfscope}%
\pgfpathrectangle{\pgfqpoint{0.100000in}{0.212622in}}{\pgfqpoint{3.696000in}{3.696000in}}%
\pgfusepath{clip}%
\pgfsetbuttcap%
\pgfsetroundjoin%
\definecolor{currentfill}{rgb}{0.121569,0.466667,0.705882}%
\pgfsetfillcolor{currentfill}%
\pgfsetfillopacity{0.443355}%
\pgfsetlinewidth{1.003750pt}%
\definecolor{currentstroke}{rgb}{0.121569,0.466667,0.705882}%
\pgfsetstrokecolor{currentstroke}%
\pgfsetstrokeopacity{0.443355}%
\pgfsetdash{}{0pt}%
\pgfpathmoveto{\pgfqpoint{2.023742in}{1.914340in}}%
\pgfpathcurveto{\pgfqpoint{2.031979in}{1.914340in}}{\pgfqpoint{2.039879in}{1.917612in}}{\pgfqpoint{2.045703in}{1.923436in}}%
\pgfpathcurveto{\pgfqpoint{2.051526in}{1.929260in}}{\pgfqpoint{2.054799in}{1.937160in}}{\pgfqpoint{2.054799in}{1.945397in}}%
\pgfpathcurveto{\pgfqpoint{2.054799in}{1.953633in}}{\pgfqpoint{2.051526in}{1.961533in}}{\pgfqpoint{2.045703in}{1.967357in}}%
\pgfpathcurveto{\pgfqpoint{2.039879in}{1.973181in}}{\pgfqpoint{2.031979in}{1.976453in}}{\pgfqpoint{2.023742in}{1.976453in}}%
\pgfpathcurveto{\pgfqpoint{2.015506in}{1.976453in}}{\pgfqpoint{2.007606in}{1.973181in}}{\pgfqpoint{2.001782in}{1.967357in}}%
\pgfpathcurveto{\pgfqpoint{1.995958in}{1.961533in}}{\pgfqpoint{1.992686in}{1.953633in}}{\pgfqpoint{1.992686in}{1.945397in}}%
\pgfpathcurveto{\pgfqpoint{1.992686in}{1.937160in}}{\pgfqpoint{1.995958in}{1.929260in}}{\pgfqpoint{2.001782in}{1.923436in}}%
\pgfpathcurveto{\pgfqpoint{2.007606in}{1.917612in}}{\pgfqpoint{2.015506in}{1.914340in}}{\pgfqpoint{2.023742in}{1.914340in}}%
\pgfpathclose%
\pgfusepath{stroke,fill}%
\end{pgfscope}%
\begin{pgfscope}%
\pgfpathrectangle{\pgfqpoint{0.100000in}{0.212622in}}{\pgfqpoint{3.696000in}{3.696000in}}%
\pgfusepath{clip}%
\pgfsetbuttcap%
\pgfsetroundjoin%
\definecolor{currentfill}{rgb}{0.121569,0.466667,0.705882}%
\pgfsetfillcolor{currentfill}%
\pgfsetfillopacity{0.443630}%
\pgfsetlinewidth{1.003750pt}%
\definecolor{currentstroke}{rgb}{0.121569,0.466667,0.705882}%
\pgfsetstrokecolor{currentstroke}%
\pgfsetstrokeopacity{0.443630}%
\pgfsetdash{}{0pt}%
\pgfpathmoveto{\pgfqpoint{1.492317in}{1.809327in}}%
\pgfpathcurveto{\pgfqpoint{1.500553in}{1.809327in}}{\pgfqpoint{1.508453in}{1.812599in}}{\pgfqpoint{1.514277in}{1.818423in}}%
\pgfpathcurveto{\pgfqpoint{1.520101in}{1.824247in}}{\pgfqpoint{1.523373in}{1.832147in}}{\pgfqpoint{1.523373in}{1.840384in}}%
\pgfpathcurveto{\pgfqpoint{1.523373in}{1.848620in}}{\pgfqpoint{1.520101in}{1.856520in}}{\pgfqpoint{1.514277in}{1.862344in}}%
\pgfpathcurveto{\pgfqpoint{1.508453in}{1.868168in}}{\pgfqpoint{1.500553in}{1.871440in}}{\pgfqpoint{1.492317in}{1.871440in}}%
\pgfpathcurveto{\pgfqpoint{1.484080in}{1.871440in}}{\pgfqpoint{1.476180in}{1.868168in}}{\pgfqpoint{1.470356in}{1.862344in}}%
\pgfpathcurveto{\pgfqpoint{1.464532in}{1.856520in}}{\pgfqpoint{1.461260in}{1.848620in}}{\pgfqpoint{1.461260in}{1.840384in}}%
\pgfpathcurveto{\pgfqpoint{1.461260in}{1.832147in}}{\pgfqpoint{1.464532in}{1.824247in}}{\pgfqpoint{1.470356in}{1.818423in}}%
\pgfpathcurveto{\pgfqpoint{1.476180in}{1.812599in}}{\pgfqpoint{1.484080in}{1.809327in}}{\pgfqpoint{1.492317in}{1.809327in}}%
\pgfpathclose%
\pgfusepath{stroke,fill}%
\end{pgfscope}%
\begin{pgfscope}%
\pgfpathrectangle{\pgfqpoint{0.100000in}{0.212622in}}{\pgfqpoint{3.696000in}{3.696000in}}%
\pgfusepath{clip}%
\pgfsetbuttcap%
\pgfsetroundjoin%
\definecolor{currentfill}{rgb}{0.121569,0.466667,0.705882}%
\pgfsetfillcolor{currentfill}%
\pgfsetfillopacity{0.444238}%
\pgfsetlinewidth{1.003750pt}%
\definecolor{currentstroke}{rgb}{0.121569,0.466667,0.705882}%
\pgfsetstrokecolor{currentstroke}%
\pgfsetstrokeopacity{0.444238}%
\pgfsetdash{}{0pt}%
\pgfpathmoveto{\pgfqpoint{1.489927in}{1.807796in}}%
\pgfpathcurveto{\pgfqpoint{1.498163in}{1.807796in}}{\pgfqpoint{1.506063in}{1.811068in}}{\pgfqpoint{1.511887in}{1.816892in}}%
\pgfpathcurveto{\pgfqpoint{1.517711in}{1.822716in}}{\pgfqpoint{1.520983in}{1.830616in}}{\pgfqpoint{1.520983in}{1.838852in}}%
\pgfpathcurveto{\pgfqpoint{1.520983in}{1.847088in}}{\pgfqpoint{1.517711in}{1.854989in}}{\pgfqpoint{1.511887in}{1.860812in}}%
\pgfpathcurveto{\pgfqpoint{1.506063in}{1.866636in}}{\pgfqpoint{1.498163in}{1.869909in}}{\pgfqpoint{1.489927in}{1.869909in}}%
\pgfpathcurveto{\pgfqpoint{1.481691in}{1.869909in}}{\pgfqpoint{1.473791in}{1.866636in}}{\pgfqpoint{1.467967in}{1.860812in}}%
\pgfpathcurveto{\pgfqpoint{1.462143in}{1.854989in}}{\pgfqpoint{1.458870in}{1.847088in}}{\pgfqpoint{1.458870in}{1.838852in}}%
\pgfpathcurveto{\pgfqpoint{1.458870in}{1.830616in}}{\pgfqpoint{1.462143in}{1.822716in}}{\pgfqpoint{1.467967in}{1.816892in}}%
\pgfpathcurveto{\pgfqpoint{1.473791in}{1.811068in}}{\pgfqpoint{1.481691in}{1.807796in}}{\pgfqpoint{1.489927in}{1.807796in}}%
\pgfpathclose%
\pgfusepath{stroke,fill}%
\end{pgfscope}%
\begin{pgfscope}%
\pgfpathrectangle{\pgfqpoint{0.100000in}{0.212622in}}{\pgfqpoint{3.696000in}{3.696000in}}%
\pgfusepath{clip}%
\pgfsetbuttcap%
\pgfsetroundjoin%
\definecolor{currentfill}{rgb}{0.121569,0.466667,0.705882}%
\pgfsetfillcolor{currentfill}%
\pgfsetfillopacity{0.444842}%
\pgfsetlinewidth{1.003750pt}%
\definecolor{currentstroke}{rgb}{0.121569,0.466667,0.705882}%
\pgfsetstrokecolor{currentstroke}%
\pgfsetstrokeopacity{0.444842}%
\pgfsetdash{}{0pt}%
\pgfpathmoveto{\pgfqpoint{1.488511in}{1.807559in}}%
\pgfpathcurveto{\pgfqpoint{1.496747in}{1.807559in}}{\pgfqpoint{1.504647in}{1.810831in}}{\pgfqpoint{1.510471in}{1.816655in}}%
\pgfpathcurveto{\pgfqpoint{1.516295in}{1.822479in}}{\pgfqpoint{1.519568in}{1.830379in}}{\pgfqpoint{1.519568in}{1.838615in}}%
\pgfpathcurveto{\pgfqpoint{1.519568in}{1.846851in}}{\pgfqpoint{1.516295in}{1.854752in}}{\pgfqpoint{1.510471in}{1.860575in}}%
\pgfpathcurveto{\pgfqpoint{1.504647in}{1.866399in}}{\pgfqpoint{1.496747in}{1.869672in}}{\pgfqpoint{1.488511in}{1.869672in}}%
\pgfpathcurveto{\pgfqpoint{1.480275in}{1.869672in}}{\pgfqpoint{1.472375in}{1.866399in}}{\pgfqpoint{1.466551in}{1.860575in}}%
\pgfpathcurveto{\pgfqpoint{1.460727in}{1.854752in}}{\pgfqpoint{1.457455in}{1.846851in}}{\pgfqpoint{1.457455in}{1.838615in}}%
\pgfpathcurveto{\pgfqpoint{1.457455in}{1.830379in}}{\pgfqpoint{1.460727in}{1.822479in}}{\pgfqpoint{1.466551in}{1.816655in}}%
\pgfpathcurveto{\pgfqpoint{1.472375in}{1.810831in}}{\pgfqpoint{1.480275in}{1.807559in}}{\pgfqpoint{1.488511in}{1.807559in}}%
\pgfpathclose%
\pgfusepath{stroke,fill}%
\end{pgfscope}%
\begin{pgfscope}%
\pgfpathrectangle{\pgfqpoint{0.100000in}{0.212622in}}{\pgfqpoint{3.696000in}{3.696000in}}%
\pgfusepath{clip}%
\pgfsetbuttcap%
\pgfsetroundjoin%
\definecolor{currentfill}{rgb}{0.121569,0.466667,0.705882}%
\pgfsetfillcolor{currentfill}%
\pgfsetfillopacity{0.445433}%
\pgfsetlinewidth{1.003750pt}%
\definecolor{currentstroke}{rgb}{0.121569,0.466667,0.705882}%
\pgfsetstrokecolor{currentstroke}%
\pgfsetstrokeopacity{0.445433}%
\pgfsetdash{}{0pt}%
\pgfpathmoveto{\pgfqpoint{2.024469in}{1.912615in}}%
\pgfpathcurveto{\pgfqpoint{2.032705in}{1.912615in}}{\pgfqpoint{2.040605in}{1.915887in}}{\pgfqpoint{2.046429in}{1.921711in}}%
\pgfpathcurveto{\pgfqpoint{2.052253in}{1.927535in}}{\pgfqpoint{2.055526in}{1.935435in}}{\pgfqpoint{2.055526in}{1.943671in}}%
\pgfpathcurveto{\pgfqpoint{2.055526in}{1.951908in}}{\pgfqpoint{2.052253in}{1.959808in}}{\pgfqpoint{2.046429in}{1.965632in}}%
\pgfpathcurveto{\pgfqpoint{2.040605in}{1.971455in}}{\pgfqpoint{2.032705in}{1.974728in}}{\pgfqpoint{2.024469in}{1.974728in}}%
\pgfpathcurveto{\pgfqpoint{2.016233in}{1.974728in}}{\pgfqpoint{2.008333in}{1.971455in}}{\pgfqpoint{2.002509in}{1.965632in}}%
\pgfpathcurveto{\pgfqpoint{1.996685in}{1.959808in}}{\pgfqpoint{1.993413in}{1.951908in}}{\pgfqpoint{1.993413in}{1.943671in}}%
\pgfpathcurveto{\pgfqpoint{1.993413in}{1.935435in}}{\pgfqpoint{1.996685in}{1.927535in}}{\pgfqpoint{2.002509in}{1.921711in}}%
\pgfpathcurveto{\pgfqpoint{2.008333in}{1.915887in}}{\pgfqpoint{2.016233in}{1.912615in}}{\pgfqpoint{2.024469in}{1.912615in}}%
\pgfpathclose%
\pgfusepath{stroke,fill}%
\end{pgfscope}%
\begin{pgfscope}%
\pgfpathrectangle{\pgfqpoint{0.100000in}{0.212622in}}{\pgfqpoint{3.696000in}{3.696000in}}%
\pgfusepath{clip}%
\pgfsetbuttcap%
\pgfsetroundjoin%
\definecolor{currentfill}{rgb}{0.121569,0.466667,0.705882}%
\pgfsetfillcolor{currentfill}%
\pgfsetfillopacity{0.445768}%
\pgfsetlinewidth{1.003750pt}%
\definecolor{currentstroke}{rgb}{0.121569,0.466667,0.705882}%
\pgfsetstrokecolor{currentstroke}%
\pgfsetstrokeopacity{0.445768}%
\pgfsetdash{}{0pt}%
\pgfpathmoveto{\pgfqpoint{1.485682in}{1.806333in}}%
\pgfpathcurveto{\pgfqpoint{1.493918in}{1.806333in}}{\pgfqpoint{1.501818in}{1.809606in}}{\pgfqpoint{1.507642in}{1.815430in}}%
\pgfpathcurveto{\pgfqpoint{1.513466in}{1.821253in}}{\pgfqpoint{1.516738in}{1.829154in}}{\pgfqpoint{1.516738in}{1.837390in}}%
\pgfpathcurveto{\pgfqpoint{1.516738in}{1.845626in}}{\pgfqpoint{1.513466in}{1.853526in}}{\pgfqpoint{1.507642in}{1.859350in}}%
\pgfpathcurveto{\pgfqpoint{1.501818in}{1.865174in}}{\pgfqpoint{1.493918in}{1.868446in}}{\pgfqpoint{1.485682in}{1.868446in}}%
\pgfpathcurveto{\pgfqpoint{1.477446in}{1.868446in}}{\pgfqpoint{1.469545in}{1.865174in}}{\pgfqpoint{1.463722in}{1.859350in}}%
\pgfpathcurveto{\pgfqpoint{1.457898in}{1.853526in}}{\pgfqpoint{1.454625in}{1.845626in}}{\pgfqpoint{1.454625in}{1.837390in}}%
\pgfpathcurveto{\pgfqpoint{1.454625in}{1.829154in}}{\pgfqpoint{1.457898in}{1.821253in}}{\pgfqpoint{1.463722in}{1.815430in}}%
\pgfpathcurveto{\pgfqpoint{1.469545in}{1.809606in}}{\pgfqpoint{1.477446in}{1.806333in}}{\pgfqpoint{1.485682in}{1.806333in}}%
\pgfpathclose%
\pgfusepath{stroke,fill}%
\end{pgfscope}%
\begin{pgfscope}%
\pgfpathrectangle{\pgfqpoint{0.100000in}{0.212622in}}{\pgfqpoint{3.696000in}{3.696000in}}%
\pgfusepath{clip}%
\pgfsetbuttcap%
\pgfsetroundjoin%
\definecolor{currentfill}{rgb}{0.121569,0.466667,0.705882}%
\pgfsetfillcolor{currentfill}%
\pgfsetfillopacity{0.446578}%
\pgfsetlinewidth{1.003750pt}%
\definecolor{currentstroke}{rgb}{0.121569,0.466667,0.705882}%
\pgfsetstrokecolor{currentstroke}%
\pgfsetstrokeopacity{0.446578}%
\pgfsetdash{}{0pt}%
\pgfpathmoveto{\pgfqpoint{2.024819in}{1.911658in}}%
\pgfpathcurveto{\pgfqpoint{2.033056in}{1.911658in}}{\pgfqpoint{2.040956in}{1.914931in}}{\pgfqpoint{2.046780in}{1.920754in}}%
\pgfpathcurveto{\pgfqpoint{2.052604in}{1.926578in}}{\pgfqpoint{2.055876in}{1.934478in}}{\pgfqpoint{2.055876in}{1.942715in}}%
\pgfpathcurveto{\pgfqpoint{2.055876in}{1.950951in}}{\pgfqpoint{2.052604in}{1.958851in}}{\pgfqpoint{2.046780in}{1.964675in}}%
\pgfpathcurveto{\pgfqpoint{2.040956in}{1.970499in}}{\pgfqpoint{2.033056in}{1.973771in}}{\pgfqpoint{2.024819in}{1.973771in}}%
\pgfpathcurveto{\pgfqpoint{2.016583in}{1.973771in}}{\pgfqpoint{2.008683in}{1.970499in}}{\pgfqpoint{2.002859in}{1.964675in}}%
\pgfpathcurveto{\pgfqpoint{1.997035in}{1.958851in}}{\pgfqpoint{1.993763in}{1.950951in}}{\pgfqpoint{1.993763in}{1.942715in}}%
\pgfpathcurveto{\pgfqpoint{1.993763in}{1.934478in}}{\pgfqpoint{1.997035in}{1.926578in}}{\pgfqpoint{2.002859in}{1.920754in}}%
\pgfpathcurveto{\pgfqpoint{2.008683in}{1.914931in}}{\pgfqpoint{2.016583in}{1.911658in}}{\pgfqpoint{2.024819in}{1.911658in}}%
\pgfpathclose%
\pgfusepath{stroke,fill}%
\end{pgfscope}%
\begin{pgfscope}%
\pgfpathrectangle{\pgfqpoint{0.100000in}{0.212622in}}{\pgfqpoint{3.696000in}{3.696000in}}%
\pgfusepath{clip}%
\pgfsetbuttcap%
\pgfsetroundjoin%
\definecolor{currentfill}{rgb}{0.121569,0.466667,0.705882}%
\pgfsetfillcolor{currentfill}%
\pgfsetfillopacity{0.447093}%
\pgfsetlinewidth{1.003750pt}%
\definecolor{currentstroke}{rgb}{0.121569,0.466667,0.705882}%
\pgfsetstrokecolor{currentstroke}%
\pgfsetstrokeopacity{0.447093}%
\pgfsetdash{}{0pt}%
\pgfpathmoveto{\pgfqpoint{1.480823in}{1.801312in}}%
\pgfpathcurveto{\pgfqpoint{1.489059in}{1.801312in}}{\pgfqpoint{1.496959in}{1.804584in}}{\pgfqpoint{1.502783in}{1.810408in}}%
\pgfpathcurveto{\pgfqpoint{1.508607in}{1.816232in}}{\pgfqpoint{1.511880in}{1.824132in}}{\pgfqpoint{1.511880in}{1.832368in}}%
\pgfpathcurveto{\pgfqpoint{1.511880in}{1.840604in}}{\pgfqpoint{1.508607in}{1.848504in}}{\pgfqpoint{1.502783in}{1.854328in}}%
\pgfpathcurveto{\pgfqpoint{1.496959in}{1.860152in}}{\pgfqpoint{1.489059in}{1.863425in}}{\pgfqpoint{1.480823in}{1.863425in}}%
\pgfpathcurveto{\pgfqpoint{1.472587in}{1.863425in}}{\pgfqpoint{1.464687in}{1.860152in}}{\pgfqpoint{1.458863in}{1.854328in}}%
\pgfpathcurveto{\pgfqpoint{1.453039in}{1.848504in}}{\pgfqpoint{1.449767in}{1.840604in}}{\pgfqpoint{1.449767in}{1.832368in}}%
\pgfpathcurveto{\pgfqpoint{1.449767in}{1.824132in}}{\pgfqpoint{1.453039in}{1.816232in}}{\pgfqpoint{1.458863in}{1.810408in}}%
\pgfpathcurveto{\pgfqpoint{1.464687in}{1.804584in}}{\pgfqpoint{1.472587in}{1.801312in}}{\pgfqpoint{1.480823in}{1.801312in}}%
\pgfpathclose%
\pgfusepath{stroke,fill}%
\end{pgfscope}%
\begin{pgfscope}%
\pgfpathrectangle{\pgfqpoint{0.100000in}{0.212622in}}{\pgfqpoint{3.696000in}{3.696000in}}%
\pgfusepath{clip}%
\pgfsetbuttcap%
\pgfsetroundjoin%
\definecolor{currentfill}{rgb}{0.121569,0.466667,0.705882}%
\pgfsetfillcolor{currentfill}%
\pgfsetfillopacity{0.447198}%
\pgfsetlinewidth{1.003750pt}%
\definecolor{currentstroke}{rgb}{0.121569,0.466667,0.705882}%
\pgfsetstrokecolor{currentstroke}%
\pgfsetstrokeopacity{0.447198}%
\pgfsetdash{}{0pt}%
\pgfpathmoveto{\pgfqpoint{2.025201in}{1.911143in}}%
\pgfpathcurveto{\pgfqpoint{2.033438in}{1.911143in}}{\pgfqpoint{2.041338in}{1.914415in}}{\pgfqpoint{2.047162in}{1.920239in}}%
\pgfpathcurveto{\pgfqpoint{2.052986in}{1.926063in}}{\pgfqpoint{2.056258in}{1.933963in}}{\pgfqpoint{2.056258in}{1.942200in}}%
\pgfpathcurveto{\pgfqpoint{2.056258in}{1.950436in}}{\pgfqpoint{2.052986in}{1.958336in}}{\pgfqpoint{2.047162in}{1.964160in}}%
\pgfpathcurveto{\pgfqpoint{2.041338in}{1.969984in}}{\pgfqpoint{2.033438in}{1.973256in}}{\pgfqpoint{2.025201in}{1.973256in}}%
\pgfpathcurveto{\pgfqpoint{2.016965in}{1.973256in}}{\pgfqpoint{2.009065in}{1.969984in}}{\pgfqpoint{2.003241in}{1.964160in}}%
\pgfpathcurveto{\pgfqpoint{1.997417in}{1.958336in}}{\pgfqpoint{1.994145in}{1.950436in}}{\pgfqpoint{1.994145in}{1.942200in}}%
\pgfpathcurveto{\pgfqpoint{1.994145in}{1.933963in}}{\pgfqpoint{1.997417in}{1.926063in}}{\pgfqpoint{2.003241in}{1.920239in}}%
\pgfpathcurveto{\pgfqpoint{2.009065in}{1.914415in}}{\pgfqpoint{2.016965in}{1.911143in}}{\pgfqpoint{2.025201in}{1.911143in}}%
\pgfpathclose%
\pgfusepath{stroke,fill}%
\end{pgfscope}%
\begin{pgfscope}%
\pgfpathrectangle{\pgfqpoint{0.100000in}{0.212622in}}{\pgfqpoint{3.696000in}{3.696000in}}%
\pgfusepath{clip}%
\pgfsetbuttcap%
\pgfsetroundjoin%
\definecolor{currentfill}{rgb}{0.121569,0.466667,0.705882}%
\pgfsetfillcolor{currentfill}%
\pgfsetfillopacity{0.447609}%
\pgfsetlinewidth{1.003750pt}%
\definecolor{currentstroke}{rgb}{0.121569,0.466667,0.705882}%
\pgfsetstrokecolor{currentstroke}%
\pgfsetstrokeopacity{0.447609}%
\pgfsetdash{}{0pt}%
\pgfpathmoveto{\pgfqpoint{2.025367in}{1.911311in}}%
\pgfpathcurveto{\pgfqpoint{2.033603in}{1.911311in}}{\pgfqpoint{2.041503in}{1.914584in}}{\pgfqpoint{2.047327in}{1.920408in}}%
\pgfpathcurveto{\pgfqpoint{2.053151in}{1.926231in}}{\pgfqpoint{2.056423in}{1.934132in}}{\pgfqpoint{2.056423in}{1.942368in}}%
\pgfpathcurveto{\pgfqpoint{2.056423in}{1.950604in}}{\pgfqpoint{2.053151in}{1.958504in}}{\pgfqpoint{2.047327in}{1.964328in}}%
\pgfpathcurveto{\pgfqpoint{2.041503in}{1.970152in}}{\pgfqpoint{2.033603in}{1.973424in}}{\pgfqpoint{2.025367in}{1.973424in}}%
\pgfpathcurveto{\pgfqpoint{2.017131in}{1.973424in}}{\pgfqpoint{2.009230in}{1.970152in}}{\pgfqpoint{2.003407in}{1.964328in}}%
\pgfpathcurveto{\pgfqpoint{1.997583in}{1.958504in}}{\pgfqpoint{1.994310in}{1.950604in}}{\pgfqpoint{1.994310in}{1.942368in}}%
\pgfpathcurveto{\pgfqpoint{1.994310in}{1.934132in}}{\pgfqpoint{1.997583in}{1.926231in}}{\pgfqpoint{2.003407in}{1.920408in}}%
\pgfpathcurveto{\pgfqpoint{2.009230in}{1.914584in}}{\pgfqpoint{2.017131in}{1.911311in}}{\pgfqpoint{2.025367in}{1.911311in}}%
\pgfpathclose%
\pgfusepath{stroke,fill}%
\end{pgfscope}%
\begin{pgfscope}%
\pgfpathrectangle{\pgfqpoint{0.100000in}{0.212622in}}{\pgfqpoint{3.696000in}{3.696000in}}%
\pgfusepath{clip}%
\pgfsetbuttcap%
\pgfsetroundjoin%
\definecolor{currentfill}{rgb}{0.121569,0.466667,0.705882}%
\pgfsetfillcolor{currentfill}%
\pgfsetfillopacity{0.447791}%
\pgfsetlinewidth{1.003750pt}%
\definecolor{currentstroke}{rgb}{0.121569,0.466667,0.705882}%
\pgfsetstrokecolor{currentstroke}%
\pgfsetstrokeopacity{0.447791}%
\pgfsetdash{}{0pt}%
\pgfpathmoveto{\pgfqpoint{2.025424in}{1.911098in}}%
\pgfpathcurveto{\pgfqpoint{2.033660in}{1.911098in}}{\pgfqpoint{2.041560in}{1.914371in}}{\pgfqpoint{2.047384in}{1.920195in}}%
\pgfpathcurveto{\pgfqpoint{2.053208in}{1.926018in}}{\pgfqpoint{2.056481in}{1.933919in}}{\pgfqpoint{2.056481in}{1.942155in}}%
\pgfpathcurveto{\pgfqpoint{2.056481in}{1.950391in}}{\pgfqpoint{2.053208in}{1.958291in}}{\pgfqpoint{2.047384in}{1.964115in}}%
\pgfpathcurveto{\pgfqpoint{2.041560in}{1.969939in}}{\pgfqpoint{2.033660in}{1.973211in}}{\pgfqpoint{2.025424in}{1.973211in}}%
\pgfpathcurveto{\pgfqpoint{2.017188in}{1.973211in}}{\pgfqpoint{2.009288in}{1.969939in}}{\pgfqpoint{2.003464in}{1.964115in}}%
\pgfpathcurveto{\pgfqpoint{1.997640in}{1.958291in}}{\pgfqpoint{1.994368in}{1.950391in}}{\pgfqpoint{1.994368in}{1.942155in}}%
\pgfpathcurveto{\pgfqpoint{1.994368in}{1.933919in}}{\pgfqpoint{1.997640in}{1.926018in}}{\pgfqpoint{2.003464in}{1.920195in}}%
\pgfpathcurveto{\pgfqpoint{2.009288in}{1.914371in}}{\pgfqpoint{2.017188in}{1.911098in}}{\pgfqpoint{2.025424in}{1.911098in}}%
\pgfpathclose%
\pgfusepath{stroke,fill}%
\end{pgfscope}%
\begin{pgfscope}%
\pgfpathrectangle{\pgfqpoint{0.100000in}{0.212622in}}{\pgfqpoint{3.696000in}{3.696000in}}%
\pgfusepath{clip}%
\pgfsetbuttcap%
\pgfsetroundjoin%
\definecolor{currentfill}{rgb}{0.121569,0.466667,0.705882}%
\pgfsetfillcolor{currentfill}%
\pgfsetfillopacity{0.448318}%
\pgfsetlinewidth{1.003750pt}%
\definecolor{currentstroke}{rgb}{0.121569,0.466667,0.705882}%
\pgfsetstrokecolor{currentstroke}%
\pgfsetstrokeopacity{0.448318}%
\pgfsetdash{}{0pt}%
\pgfpathmoveto{\pgfqpoint{2.025830in}{1.910044in}}%
\pgfpathcurveto{\pgfqpoint{2.034066in}{1.910044in}}{\pgfqpoint{2.041966in}{1.913316in}}{\pgfqpoint{2.047790in}{1.919140in}}%
\pgfpathcurveto{\pgfqpoint{2.053614in}{1.924964in}}{\pgfqpoint{2.056886in}{1.932864in}}{\pgfqpoint{2.056886in}{1.941100in}}%
\pgfpathcurveto{\pgfqpoint{2.056886in}{1.949337in}}{\pgfqpoint{2.053614in}{1.957237in}}{\pgfqpoint{2.047790in}{1.963061in}}%
\pgfpathcurveto{\pgfqpoint{2.041966in}{1.968884in}}{\pgfqpoint{2.034066in}{1.972157in}}{\pgfqpoint{2.025830in}{1.972157in}}%
\pgfpathcurveto{\pgfqpoint{2.017594in}{1.972157in}}{\pgfqpoint{2.009694in}{1.968884in}}{\pgfqpoint{2.003870in}{1.963061in}}%
\pgfpathcurveto{\pgfqpoint{1.998046in}{1.957237in}}{\pgfqpoint{1.994773in}{1.949337in}}{\pgfqpoint{1.994773in}{1.941100in}}%
\pgfpathcurveto{\pgfqpoint{1.994773in}{1.932864in}}{\pgfqpoint{1.998046in}{1.924964in}}{\pgfqpoint{2.003870in}{1.919140in}}%
\pgfpathcurveto{\pgfqpoint{2.009694in}{1.913316in}}{\pgfqpoint{2.017594in}{1.910044in}}{\pgfqpoint{2.025830in}{1.910044in}}%
\pgfpathclose%
\pgfusepath{stroke,fill}%
\end{pgfscope}%
\begin{pgfscope}%
\pgfpathrectangle{\pgfqpoint{0.100000in}{0.212622in}}{\pgfqpoint{3.696000in}{3.696000in}}%
\pgfusepath{clip}%
\pgfsetbuttcap%
\pgfsetroundjoin%
\definecolor{currentfill}{rgb}{0.121569,0.466667,0.705882}%
\pgfsetfillcolor{currentfill}%
\pgfsetfillopacity{0.448687}%
\pgfsetlinewidth{1.003750pt}%
\definecolor{currentstroke}{rgb}{0.121569,0.466667,0.705882}%
\pgfsetstrokecolor{currentstroke}%
\pgfsetstrokeopacity{0.448687}%
\pgfsetdash{}{0pt}%
\pgfpathmoveto{\pgfqpoint{2.026151in}{1.910064in}}%
\pgfpathcurveto{\pgfqpoint{2.034387in}{1.910064in}}{\pgfqpoint{2.042287in}{1.913336in}}{\pgfqpoint{2.048111in}{1.919160in}}%
\pgfpathcurveto{\pgfqpoint{2.053935in}{1.924984in}}{\pgfqpoint{2.057208in}{1.932884in}}{\pgfqpoint{2.057208in}{1.941120in}}%
\pgfpathcurveto{\pgfqpoint{2.057208in}{1.949357in}}{\pgfqpoint{2.053935in}{1.957257in}}{\pgfqpoint{2.048111in}{1.963081in}}%
\pgfpathcurveto{\pgfqpoint{2.042287in}{1.968905in}}{\pgfqpoint{2.034387in}{1.972177in}}{\pgfqpoint{2.026151in}{1.972177in}}%
\pgfpathcurveto{\pgfqpoint{2.017915in}{1.972177in}}{\pgfqpoint{2.010015in}{1.968905in}}{\pgfqpoint{2.004191in}{1.963081in}}%
\pgfpathcurveto{\pgfqpoint{1.998367in}{1.957257in}}{\pgfqpoint{1.995095in}{1.949357in}}{\pgfqpoint{1.995095in}{1.941120in}}%
\pgfpathcurveto{\pgfqpoint{1.995095in}{1.932884in}}{\pgfqpoint{1.998367in}{1.924984in}}{\pgfqpoint{2.004191in}{1.919160in}}%
\pgfpathcurveto{\pgfqpoint{2.010015in}{1.913336in}}{\pgfqpoint{2.017915in}{1.910064in}}{\pgfqpoint{2.026151in}{1.910064in}}%
\pgfpathclose%
\pgfusepath{stroke,fill}%
\end{pgfscope}%
\begin{pgfscope}%
\pgfpathrectangle{\pgfqpoint{0.100000in}{0.212622in}}{\pgfqpoint{3.696000in}{3.696000in}}%
\pgfusepath{clip}%
\pgfsetbuttcap%
\pgfsetroundjoin%
\definecolor{currentfill}{rgb}{0.121569,0.466667,0.705882}%
\pgfsetfillcolor{currentfill}%
\pgfsetfillopacity{0.449097}%
\pgfsetlinewidth{1.003750pt}%
\definecolor{currentstroke}{rgb}{0.121569,0.466667,0.705882}%
\pgfsetstrokecolor{currentstroke}%
\pgfsetstrokeopacity{0.449097}%
\pgfsetdash{}{0pt}%
\pgfpathmoveto{\pgfqpoint{1.477105in}{1.802667in}}%
\pgfpathcurveto{\pgfqpoint{1.485341in}{1.802667in}}{\pgfqpoint{1.493241in}{1.805939in}}{\pgfqpoint{1.499065in}{1.811763in}}%
\pgfpathcurveto{\pgfqpoint{1.504889in}{1.817587in}}{\pgfqpoint{1.508161in}{1.825487in}}{\pgfqpoint{1.508161in}{1.833723in}}%
\pgfpathcurveto{\pgfqpoint{1.508161in}{1.841960in}}{\pgfqpoint{1.504889in}{1.849860in}}{\pgfqpoint{1.499065in}{1.855684in}}%
\pgfpathcurveto{\pgfqpoint{1.493241in}{1.861507in}}{\pgfqpoint{1.485341in}{1.864780in}}{\pgfqpoint{1.477105in}{1.864780in}}%
\pgfpathcurveto{\pgfqpoint{1.468868in}{1.864780in}}{\pgfqpoint{1.460968in}{1.861507in}}{\pgfqpoint{1.455144in}{1.855684in}}%
\pgfpathcurveto{\pgfqpoint{1.449320in}{1.849860in}}{\pgfqpoint{1.446048in}{1.841960in}}{\pgfqpoint{1.446048in}{1.833723in}}%
\pgfpathcurveto{\pgfqpoint{1.446048in}{1.825487in}}{\pgfqpoint{1.449320in}{1.817587in}}{\pgfqpoint{1.455144in}{1.811763in}}%
\pgfpathcurveto{\pgfqpoint{1.460968in}{1.805939in}}{\pgfqpoint{1.468868in}{1.802667in}}{\pgfqpoint{1.477105in}{1.802667in}}%
\pgfpathclose%
\pgfusepath{stroke,fill}%
\end{pgfscope}%
\begin{pgfscope}%
\pgfpathrectangle{\pgfqpoint{0.100000in}{0.212622in}}{\pgfqpoint{3.696000in}{3.696000in}}%
\pgfusepath{clip}%
\pgfsetbuttcap%
\pgfsetroundjoin%
\definecolor{currentfill}{rgb}{0.121569,0.466667,0.705882}%
\pgfsetfillcolor{currentfill}%
\pgfsetfillopacity{0.449692}%
\pgfsetlinewidth{1.003750pt}%
\definecolor{currentstroke}{rgb}{0.121569,0.466667,0.705882}%
\pgfsetstrokecolor{currentstroke}%
\pgfsetstrokeopacity{0.449692}%
\pgfsetdash{}{0pt}%
\pgfpathmoveto{\pgfqpoint{2.026606in}{1.909507in}}%
\pgfpathcurveto{\pgfqpoint{2.034843in}{1.909507in}}{\pgfqpoint{2.042743in}{1.912779in}}{\pgfqpoint{2.048567in}{1.918603in}}%
\pgfpathcurveto{\pgfqpoint{2.054391in}{1.924427in}}{\pgfqpoint{2.057663in}{1.932327in}}{\pgfqpoint{2.057663in}{1.940563in}}%
\pgfpathcurveto{\pgfqpoint{2.057663in}{1.948799in}}{\pgfqpoint{2.054391in}{1.956699in}}{\pgfqpoint{2.048567in}{1.962523in}}%
\pgfpathcurveto{\pgfqpoint{2.042743in}{1.968347in}}{\pgfqpoint{2.034843in}{1.971620in}}{\pgfqpoint{2.026606in}{1.971620in}}%
\pgfpathcurveto{\pgfqpoint{2.018370in}{1.971620in}}{\pgfqpoint{2.010470in}{1.968347in}}{\pgfqpoint{2.004646in}{1.962523in}}%
\pgfpathcurveto{\pgfqpoint{1.998822in}{1.956699in}}{\pgfqpoint{1.995550in}{1.948799in}}{\pgfqpoint{1.995550in}{1.940563in}}%
\pgfpathcurveto{\pgfqpoint{1.995550in}{1.932327in}}{\pgfqpoint{1.998822in}{1.924427in}}{\pgfqpoint{2.004646in}{1.918603in}}%
\pgfpathcurveto{\pgfqpoint{2.010470in}{1.912779in}}{\pgfqpoint{2.018370in}{1.909507in}}{\pgfqpoint{2.026606in}{1.909507in}}%
\pgfpathclose%
\pgfusepath{stroke,fill}%
\end{pgfscope}%
\begin{pgfscope}%
\pgfpathrectangle{\pgfqpoint{0.100000in}{0.212622in}}{\pgfqpoint{3.696000in}{3.696000in}}%
\pgfusepath{clip}%
\pgfsetbuttcap%
\pgfsetroundjoin%
\definecolor{currentfill}{rgb}{0.121569,0.466667,0.705882}%
\pgfsetfillcolor{currentfill}%
\pgfsetfillopacity{0.449862}%
\pgfsetlinewidth{1.003750pt}%
\definecolor{currentstroke}{rgb}{0.121569,0.466667,0.705882}%
\pgfsetstrokecolor{currentstroke}%
\pgfsetstrokeopacity{0.449862}%
\pgfsetdash{}{0pt}%
\pgfpathmoveto{\pgfqpoint{1.473660in}{1.798094in}}%
\pgfpathcurveto{\pgfqpoint{1.481896in}{1.798094in}}{\pgfqpoint{1.489796in}{1.801366in}}{\pgfqpoint{1.495620in}{1.807190in}}%
\pgfpathcurveto{\pgfqpoint{1.501444in}{1.813014in}}{\pgfqpoint{1.504717in}{1.820914in}}{\pgfqpoint{1.504717in}{1.829150in}}%
\pgfpathcurveto{\pgfqpoint{1.504717in}{1.837386in}}{\pgfqpoint{1.501444in}{1.845287in}}{\pgfqpoint{1.495620in}{1.851110in}}%
\pgfpathcurveto{\pgfqpoint{1.489796in}{1.856934in}}{\pgfqpoint{1.481896in}{1.860207in}}{\pgfqpoint{1.473660in}{1.860207in}}%
\pgfpathcurveto{\pgfqpoint{1.465424in}{1.860207in}}{\pgfqpoint{1.457524in}{1.856934in}}{\pgfqpoint{1.451700in}{1.851110in}}%
\pgfpathcurveto{\pgfqpoint{1.445876in}{1.845287in}}{\pgfqpoint{1.442604in}{1.837386in}}{\pgfqpoint{1.442604in}{1.829150in}}%
\pgfpathcurveto{\pgfqpoint{1.442604in}{1.820914in}}{\pgfqpoint{1.445876in}{1.813014in}}{\pgfqpoint{1.451700in}{1.807190in}}%
\pgfpathcurveto{\pgfqpoint{1.457524in}{1.801366in}}{\pgfqpoint{1.465424in}{1.798094in}}{\pgfqpoint{1.473660in}{1.798094in}}%
\pgfpathclose%
\pgfusepath{stroke,fill}%
\end{pgfscope}%
\begin{pgfscope}%
\pgfpathrectangle{\pgfqpoint{0.100000in}{0.212622in}}{\pgfqpoint{3.696000in}{3.696000in}}%
\pgfusepath{clip}%
\pgfsetbuttcap%
\pgfsetroundjoin%
\definecolor{currentfill}{rgb}{0.121569,0.466667,0.705882}%
\pgfsetfillcolor{currentfill}%
\pgfsetfillopacity{0.450866}%
\pgfsetlinewidth{1.003750pt}%
\definecolor{currentstroke}{rgb}{0.121569,0.466667,0.705882}%
\pgfsetstrokecolor{currentstroke}%
\pgfsetstrokeopacity{0.450866}%
\pgfsetdash{}{0pt}%
\pgfpathmoveto{\pgfqpoint{2.027046in}{1.908597in}}%
\pgfpathcurveto{\pgfqpoint{2.035282in}{1.908597in}}{\pgfqpoint{2.043182in}{1.911870in}}{\pgfqpoint{2.049006in}{1.917694in}}%
\pgfpathcurveto{\pgfqpoint{2.054830in}{1.923518in}}{\pgfqpoint{2.058103in}{1.931418in}}{\pgfqpoint{2.058103in}{1.939654in}}%
\pgfpathcurveto{\pgfqpoint{2.058103in}{1.947890in}}{\pgfqpoint{2.054830in}{1.955790in}}{\pgfqpoint{2.049006in}{1.961614in}}%
\pgfpathcurveto{\pgfqpoint{2.043182in}{1.967438in}}{\pgfqpoint{2.035282in}{1.970710in}}{\pgfqpoint{2.027046in}{1.970710in}}%
\pgfpathcurveto{\pgfqpoint{2.018810in}{1.970710in}}{\pgfqpoint{2.010910in}{1.967438in}}{\pgfqpoint{2.005086in}{1.961614in}}%
\pgfpathcurveto{\pgfqpoint{1.999262in}{1.955790in}}{\pgfqpoint{1.995990in}{1.947890in}}{\pgfqpoint{1.995990in}{1.939654in}}%
\pgfpathcurveto{\pgfqpoint{1.995990in}{1.931418in}}{\pgfqpoint{1.999262in}{1.923518in}}{\pgfqpoint{2.005086in}{1.917694in}}%
\pgfpathcurveto{\pgfqpoint{2.010910in}{1.911870in}}{\pgfqpoint{2.018810in}{1.908597in}}{\pgfqpoint{2.027046in}{1.908597in}}%
\pgfpathclose%
\pgfusepath{stroke,fill}%
\end{pgfscope}%
\begin{pgfscope}%
\pgfpathrectangle{\pgfqpoint{0.100000in}{0.212622in}}{\pgfqpoint{3.696000in}{3.696000in}}%
\pgfusepath{clip}%
\pgfsetbuttcap%
\pgfsetroundjoin%
\definecolor{currentfill}{rgb}{0.121569,0.466667,0.705882}%
\pgfsetfillcolor{currentfill}%
\pgfsetfillopacity{0.452276}%
\pgfsetlinewidth{1.003750pt}%
\definecolor{currentstroke}{rgb}{0.121569,0.466667,0.705882}%
\pgfsetstrokecolor{currentstroke}%
\pgfsetstrokeopacity{0.452276}%
\pgfsetdash{}{0pt}%
\pgfpathmoveto{\pgfqpoint{2.027649in}{1.907072in}}%
\pgfpathcurveto{\pgfqpoint{2.035885in}{1.907072in}}{\pgfqpoint{2.043785in}{1.910344in}}{\pgfqpoint{2.049609in}{1.916168in}}%
\pgfpathcurveto{\pgfqpoint{2.055433in}{1.921992in}}{\pgfqpoint{2.058705in}{1.929892in}}{\pgfqpoint{2.058705in}{1.938129in}}%
\pgfpathcurveto{\pgfqpoint{2.058705in}{1.946365in}}{\pgfqpoint{2.055433in}{1.954265in}}{\pgfqpoint{2.049609in}{1.960089in}}%
\pgfpathcurveto{\pgfqpoint{2.043785in}{1.965913in}}{\pgfqpoint{2.035885in}{1.969185in}}{\pgfqpoint{2.027649in}{1.969185in}}%
\pgfpathcurveto{\pgfqpoint{2.019413in}{1.969185in}}{\pgfqpoint{2.011512in}{1.965913in}}{\pgfqpoint{2.005689in}{1.960089in}}%
\pgfpathcurveto{\pgfqpoint{1.999865in}{1.954265in}}{\pgfqpoint{1.996592in}{1.946365in}}{\pgfqpoint{1.996592in}{1.938129in}}%
\pgfpathcurveto{\pgfqpoint{1.996592in}{1.929892in}}{\pgfqpoint{1.999865in}{1.921992in}}{\pgfqpoint{2.005689in}{1.916168in}}%
\pgfpathcurveto{\pgfqpoint{2.011512in}{1.910344in}}{\pgfqpoint{2.019413in}{1.907072in}}{\pgfqpoint{2.027649in}{1.907072in}}%
\pgfpathclose%
\pgfusepath{stroke,fill}%
\end{pgfscope}%
\begin{pgfscope}%
\pgfpathrectangle{\pgfqpoint{0.100000in}{0.212622in}}{\pgfqpoint{3.696000in}{3.696000in}}%
\pgfusepath{clip}%
\pgfsetbuttcap%
\pgfsetroundjoin%
\definecolor{currentfill}{rgb}{0.121569,0.466667,0.705882}%
\pgfsetfillcolor{currentfill}%
\pgfsetfillopacity{0.453162}%
\pgfsetlinewidth{1.003750pt}%
\definecolor{currentstroke}{rgb}{0.121569,0.466667,0.705882}%
\pgfsetstrokecolor{currentstroke}%
\pgfsetstrokeopacity{0.453162}%
\pgfsetdash{}{0pt}%
\pgfpathmoveto{\pgfqpoint{1.466766in}{1.803303in}}%
\pgfpathcurveto{\pgfqpoint{1.475003in}{1.803303in}}{\pgfqpoint{1.482903in}{1.806576in}}{\pgfqpoint{1.488727in}{1.812400in}}%
\pgfpathcurveto{\pgfqpoint{1.494550in}{1.818224in}}{\pgfqpoint{1.497823in}{1.826124in}}{\pgfqpoint{1.497823in}{1.834360in}}%
\pgfpathcurveto{\pgfqpoint{1.497823in}{1.842596in}}{\pgfqpoint{1.494550in}{1.850496in}}{\pgfqpoint{1.488727in}{1.856320in}}%
\pgfpathcurveto{\pgfqpoint{1.482903in}{1.862144in}}{\pgfqpoint{1.475003in}{1.865416in}}{\pgfqpoint{1.466766in}{1.865416in}}%
\pgfpathcurveto{\pgfqpoint{1.458530in}{1.865416in}}{\pgfqpoint{1.450630in}{1.862144in}}{\pgfqpoint{1.444806in}{1.856320in}}%
\pgfpathcurveto{\pgfqpoint{1.438982in}{1.850496in}}{\pgfqpoint{1.435710in}{1.842596in}}{\pgfqpoint{1.435710in}{1.834360in}}%
\pgfpathcurveto{\pgfqpoint{1.435710in}{1.826124in}}{\pgfqpoint{1.438982in}{1.818224in}}{\pgfqpoint{1.444806in}{1.812400in}}%
\pgfpathcurveto{\pgfqpoint{1.450630in}{1.806576in}}{\pgfqpoint{1.458530in}{1.803303in}}{\pgfqpoint{1.466766in}{1.803303in}}%
\pgfpathclose%
\pgfusepath{stroke,fill}%
\end{pgfscope}%
\begin{pgfscope}%
\pgfpathrectangle{\pgfqpoint{0.100000in}{0.212622in}}{\pgfqpoint{3.696000in}{3.696000in}}%
\pgfusepath{clip}%
\pgfsetbuttcap%
\pgfsetroundjoin%
\definecolor{currentfill}{rgb}{0.121569,0.466667,0.705882}%
\pgfsetfillcolor{currentfill}%
\pgfsetfillopacity{0.454169}%
\pgfsetlinewidth{1.003750pt}%
\definecolor{currentstroke}{rgb}{0.121569,0.466667,0.705882}%
\pgfsetstrokecolor{currentstroke}%
\pgfsetstrokeopacity{0.454169}%
\pgfsetdash{}{0pt}%
\pgfpathmoveto{\pgfqpoint{2.029074in}{1.906702in}}%
\pgfpathcurveto{\pgfqpoint{2.037311in}{1.906702in}}{\pgfqpoint{2.045211in}{1.909975in}}{\pgfqpoint{2.051035in}{1.915799in}}%
\pgfpathcurveto{\pgfqpoint{2.056859in}{1.921623in}}{\pgfqpoint{2.060131in}{1.929523in}}{\pgfqpoint{2.060131in}{1.937759in}}%
\pgfpathcurveto{\pgfqpoint{2.060131in}{1.945995in}}{\pgfqpoint{2.056859in}{1.953895in}}{\pgfqpoint{2.051035in}{1.959719in}}%
\pgfpathcurveto{\pgfqpoint{2.045211in}{1.965543in}}{\pgfqpoint{2.037311in}{1.968815in}}{\pgfqpoint{2.029074in}{1.968815in}}%
\pgfpathcurveto{\pgfqpoint{2.020838in}{1.968815in}}{\pgfqpoint{2.012938in}{1.965543in}}{\pgfqpoint{2.007114in}{1.959719in}}%
\pgfpathcurveto{\pgfqpoint{2.001290in}{1.953895in}}{\pgfqpoint{1.998018in}{1.945995in}}{\pgfqpoint{1.998018in}{1.937759in}}%
\pgfpathcurveto{\pgfqpoint{1.998018in}{1.929523in}}{\pgfqpoint{2.001290in}{1.921623in}}{\pgfqpoint{2.007114in}{1.915799in}}%
\pgfpathcurveto{\pgfqpoint{2.012938in}{1.909975in}}{\pgfqpoint{2.020838in}{1.906702in}}{\pgfqpoint{2.029074in}{1.906702in}}%
\pgfpathclose%
\pgfusepath{stroke,fill}%
\end{pgfscope}%
\begin{pgfscope}%
\pgfpathrectangle{\pgfqpoint{0.100000in}{0.212622in}}{\pgfqpoint{3.696000in}{3.696000in}}%
\pgfusepath{clip}%
\pgfsetbuttcap%
\pgfsetroundjoin%
\definecolor{currentfill}{rgb}{0.121569,0.466667,0.705882}%
\pgfsetfillcolor{currentfill}%
\pgfsetfillopacity{0.455092}%
\pgfsetlinewidth{1.003750pt}%
\definecolor{currentstroke}{rgb}{0.121569,0.466667,0.705882}%
\pgfsetstrokecolor{currentstroke}%
\pgfsetstrokeopacity{0.455092}%
\pgfsetdash{}{0pt}%
\pgfpathmoveto{\pgfqpoint{1.462021in}{1.802841in}}%
\pgfpathcurveto{\pgfqpoint{1.470257in}{1.802841in}}{\pgfqpoint{1.478157in}{1.806113in}}{\pgfqpoint{1.483981in}{1.811937in}}%
\pgfpathcurveto{\pgfqpoint{1.489805in}{1.817761in}}{\pgfqpoint{1.493078in}{1.825661in}}{\pgfqpoint{1.493078in}{1.833897in}}%
\pgfpathcurveto{\pgfqpoint{1.493078in}{1.842133in}}{\pgfqpoint{1.489805in}{1.850033in}}{\pgfqpoint{1.483981in}{1.855857in}}%
\pgfpathcurveto{\pgfqpoint{1.478157in}{1.861681in}}{\pgfqpoint{1.470257in}{1.864954in}}{\pgfqpoint{1.462021in}{1.864954in}}%
\pgfpathcurveto{\pgfqpoint{1.453785in}{1.864954in}}{\pgfqpoint{1.445885in}{1.861681in}}{\pgfqpoint{1.440061in}{1.855857in}}%
\pgfpathcurveto{\pgfqpoint{1.434237in}{1.850033in}}{\pgfqpoint{1.430965in}{1.842133in}}{\pgfqpoint{1.430965in}{1.833897in}}%
\pgfpathcurveto{\pgfqpoint{1.430965in}{1.825661in}}{\pgfqpoint{1.434237in}{1.817761in}}{\pgfqpoint{1.440061in}{1.811937in}}%
\pgfpathcurveto{\pgfqpoint{1.445885in}{1.806113in}}{\pgfqpoint{1.453785in}{1.802841in}}{\pgfqpoint{1.462021in}{1.802841in}}%
\pgfpathclose%
\pgfusepath{stroke,fill}%
\end{pgfscope}%
\begin{pgfscope}%
\pgfpathrectangle{\pgfqpoint{0.100000in}{0.212622in}}{\pgfqpoint{3.696000in}{3.696000in}}%
\pgfusepath{clip}%
\pgfsetbuttcap%
\pgfsetroundjoin%
\definecolor{currentfill}{rgb}{0.121569,0.466667,0.705882}%
\pgfsetfillcolor{currentfill}%
\pgfsetfillopacity{0.456165}%
\pgfsetlinewidth{1.003750pt}%
\definecolor{currentstroke}{rgb}{0.121569,0.466667,0.705882}%
\pgfsetstrokecolor{currentstroke}%
\pgfsetstrokeopacity{0.456165}%
\pgfsetdash{}{0pt}%
\pgfpathmoveto{\pgfqpoint{2.029495in}{1.904235in}}%
\pgfpathcurveto{\pgfqpoint{2.037731in}{1.904235in}}{\pgfqpoint{2.045631in}{1.907508in}}{\pgfqpoint{2.051455in}{1.913332in}}%
\pgfpathcurveto{\pgfqpoint{2.057279in}{1.919156in}}{\pgfqpoint{2.060551in}{1.927056in}}{\pgfqpoint{2.060551in}{1.935292in}}%
\pgfpathcurveto{\pgfqpoint{2.060551in}{1.943528in}}{\pgfqpoint{2.057279in}{1.951428in}}{\pgfqpoint{2.051455in}{1.957252in}}%
\pgfpathcurveto{\pgfqpoint{2.045631in}{1.963076in}}{\pgfqpoint{2.037731in}{1.966348in}}{\pgfqpoint{2.029495in}{1.966348in}}%
\pgfpathcurveto{\pgfqpoint{2.021258in}{1.966348in}}{\pgfqpoint{2.013358in}{1.963076in}}{\pgfqpoint{2.007534in}{1.957252in}}%
\pgfpathcurveto{\pgfqpoint{2.001710in}{1.951428in}}{\pgfqpoint{1.998438in}{1.943528in}}{\pgfqpoint{1.998438in}{1.935292in}}%
\pgfpathcurveto{\pgfqpoint{1.998438in}{1.927056in}}{\pgfqpoint{2.001710in}{1.919156in}}{\pgfqpoint{2.007534in}{1.913332in}}%
\pgfpathcurveto{\pgfqpoint{2.013358in}{1.907508in}}{\pgfqpoint{2.021258in}{1.904235in}}{\pgfqpoint{2.029495in}{1.904235in}}%
\pgfpathclose%
\pgfusepath{stroke,fill}%
\end{pgfscope}%
\begin{pgfscope}%
\pgfpathrectangle{\pgfqpoint{0.100000in}{0.212622in}}{\pgfqpoint{3.696000in}{3.696000in}}%
\pgfusepath{clip}%
\pgfsetbuttcap%
\pgfsetroundjoin%
\definecolor{currentfill}{rgb}{0.121569,0.466667,0.705882}%
\pgfsetfillcolor{currentfill}%
\pgfsetfillopacity{0.456534}%
\pgfsetlinewidth{1.003750pt}%
\definecolor{currentstroke}{rgb}{0.121569,0.466667,0.705882}%
\pgfsetstrokecolor{currentstroke}%
\pgfsetstrokeopacity{0.456534}%
\pgfsetdash{}{0pt}%
\pgfpathmoveto{\pgfqpoint{1.458261in}{1.799266in}}%
\pgfpathcurveto{\pgfqpoint{1.466497in}{1.799266in}}{\pgfqpoint{1.474397in}{1.802538in}}{\pgfqpoint{1.480221in}{1.808362in}}%
\pgfpathcurveto{\pgfqpoint{1.486045in}{1.814186in}}{\pgfqpoint{1.489318in}{1.822086in}}{\pgfqpoint{1.489318in}{1.830322in}}%
\pgfpathcurveto{\pgfqpoint{1.489318in}{1.838559in}}{\pgfqpoint{1.486045in}{1.846459in}}{\pgfqpoint{1.480221in}{1.852283in}}%
\pgfpathcurveto{\pgfqpoint{1.474397in}{1.858107in}}{\pgfqpoint{1.466497in}{1.861379in}}{\pgfqpoint{1.458261in}{1.861379in}}%
\pgfpathcurveto{\pgfqpoint{1.450025in}{1.861379in}}{\pgfqpoint{1.442125in}{1.858107in}}{\pgfqpoint{1.436301in}{1.852283in}}%
\pgfpathcurveto{\pgfqpoint{1.430477in}{1.846459in}}{\pgfqpoint{1.427205in}{1.838559in}}{\pgfqpoint{1.427205in}{1.830322in}}%
\pgfpathcurveto{\pgfqpoint{1.427205in}{1.822086in}}{\pgfqpoint{1.430477in}{1.814186in}}{\pgfqpoint{1.436301in}{1.808362in}}%
\pgfpathcurveto{\pgfqpoint{1.442125in}{1.802538in}}{\pgfqpoint{1.450025in}{1.799266in}}{\pgfqpoint{1.458261in}{1.799266in}}%
\pgfpathclose%
\pgfusepath{stroke,fill}%
\end{pgfscope}%
\begin{pgfscope}%
\pgfpathrectangle{\pgfqpoint{0.100000in}{0.212622in}}{\pgfqpoint{3.696000in}{3.696000in}}%
\pgfusepath{clip}%
\pgfsetbuttcap%
\pgfsetroundjoin%
\definecolor{currentfill}{rgb}{0.121569,0.466667,0.705882}%
\pgfsetfillcolor{currentfill}%
\pgfsetfillopacity{0.459270}%
\pgfsetlinewidth{1.003750pt}%
\definecolor{currentstroke}{rgb}{0.121569,0.466667,0.705882}%
\pgfsetstrokecolor{currentstroke}%
\pgfsetstrokeopacity{0.459270}%
\pgfsetdash{}{0pt}%
\pgfpathmoveto{\pgfqpoint{2.030248in}{1.905584in}}%
\pgfpathcurveto{\pgfqpoint{2.038485in}{1.905584in}}{\pgfqpoint{2.046385in}{1.908856in}}{\pgfqpoint{2.052209in}{1.914680in}}%
\pgfpathcurveto{\pgfqpoint{2.058033in}{1.920504in}}{\pgfqpoint{2.061305in}{1.928404in}}{\pgfqpoint{2.061305in}{1.936640in}}%
\pgfpathcurveto{\pgfqpoint{2.061305in}{1.944877in}}{\pgfqpoint{2.058033in}{1.952777in}}{\pgfqpoint{2.052209in}{1.958601in}}%
\pgfpathcurveto{\pgfqpoint{2.046385in}{1.964425in}}{\pgfqpoint{2.038485in}{1.967697in}}{\pgfqpoint{2.030248in}{1.967697in}}%
\pgfpathcurveto{\pgfqpoint{2.022012in}{1.967697in}}{\pgfqpoint{2.014112in}{1.964425in}}{\pgfqpoint{2.008288in}{1.958601in}}%
\pgfpathcurveto{\pgfqpoint{2.002464in}{1.952777in}}{\pgfqpoint{1.999192in}{1.944877in}}{\pgfqpoint{1.999192in}{1.936640in}}%
\pgfpathcurveto{\pgfqpoint{1.999192in}{1.928404in}}{\pgfqpoint{2.002464in}{1.920504in}}{\pgfqpoint{2.008288in}{1.914680in}}%
\pgfpathcurveto{\pgfqpoint{2.014112in}{1.908856in}}{\pgfqpoint{2.022012in}{1.905584in}}{\pgfqpoint{2.030248in}{1.905584in}}%
\pgfpathclose%
\pgfusepath{stroke,fill}%
\end{pgfscope}%
\begin{pgfscope}%
\pgfpathrectangle{\pgfqpoint{0.100000in}{0.212622in}}{\pgfqpoint{3.696000in}{3.696000in}}%
\pgfusepath{clip}%
\pgfsetbuttcap%
\pgfsetroundjoin%
\definecolor{currentfill}{rgb}{0.121569,0.466667,0.705882}%
\pgfsetfillcolor{currentfill}%
\pgfsetfillopacity{0.460673}%
\pgfsetlinewidth{1.003750pt}%
\definecolor{currentstroke}{rgb}{0.121569,0.466667,0.705882}%
\pgfsetstrokecolor{currentstroke}%
\pgfsetstrokeopacity{0.460673}%
\pgfsetdash{}{0pt}%
\pgfpathmoveto{\pgfqpoint{2.031504in}{1.904678in}}%
\pgfpathcurveto{\pgfqpoint{2.039740in}{1.904678in}}{\pgfqpoint{2.047640in}{1.907950in}}{\pgfqpoint{2.053464in}{1.913774in}}%
\pgfpathcurveto{\pgfqpoint{2.059288in}{1.919598in}}{\pgfqpoint{2.062561in}{1.927498in}}{\pgfqpoint{2.062561in}{1.935734in}}%
\pgfpathcurveto{\pgfqpoint{2.062561in}{1.943970in}}{\pgfqpoint{2.059288in}{1.951871in}}{\pgfqpoint{2.053464in}{1.957694in}}%
\pgfpathcurveto{\pgfqpoint{2.047640in}{1.963518in}}{\pgfqpoint{2.039740in}{1.966791in}}{\pgfqpoint{2.031504in}{1.966791in}}%
\pgfpathcurveto{\pgfqpoint{2.023268in}{1.966791in}}{\pgfqpoint{2.015368in}{1.963518in}}{\pgfqpoint{2.009544in}{1.957694in}}%
\pgfpathcurveto{\pgfqpoint{2.003720in}{1.951871in}}{\pgfqpoint{2.000448in}{1.943970in}}{\pgfqpoint{2.000448in}{1.935734in}}%
\pgfpathcurveto{\pgfqpoint{2.000448in}{1.927498in}}{\pgfqpoint{2.003720in}{1.919598in}}{\pgfqpoint{2.009544in}{1.913774in}}%
\pgfpathcurveto{\pgfqpoint{2.015368in}{1.907950in}}{\pgfqpoint{2.023268in}{1.904678in}}{\pgfqpoint{2.031504in}{1.904678in}}%
\pgfpathclose%
\pgfusepath{stroke,fill}%
\end{pgfscope}%
\begin{pgfscope}%
\pgfpathrectangle{\pgfqpoint{0.100000in}{0.212622in}}{\pgfqpoint{3.696000in}{3.696000in}}%
\pgfusepath{clip}%
\pgfsetbuttcap%
\pgfsetroundjoin%
\definecolor{currentfill}{rgb}{0.121569,0.466667,0.705882}%
\pgfsetfillcolor{currentfill}%
\pgfsetfillopacity{0.462663}%
\pgfsetlinewidth{1.003750pt}%
\definecolor{currentstroke}{rgb}{0.121569,0.466667,0.705882}%
\pgfsetstrokecolor{currentstroke}%
\pgfsetstrokeopacity{0.462663}%
\pgfsetdash{}{0pt}%
\pgfpathmoveto{\pgfqpoint{1.451956in}{1.815495in}}%
\pgfpathcurveto{\pgfqpoint{1.460192in}{1.815495in}}{\pgfqpoint{1.468092in}{1.818768in}}{\pgfqpoint{1.473916in}{1.824592in}}%
\pgfpathcurveto{\pgfqpoint{1.479740in}{1.830415in}}{\pgfqpoint{1.483012in}{1.838316in}}{\pgfqpoint{1.483012in}{1.846552in}}%
\pgfpathcurveto{\pgfqpoint{1.483012in}{1.854788in}}{\pgfqpoint{1.479740in}{1.862688in}}{\pgfqpoint{1.473916in}{1.868512in}}%
\pgfpathcurveto{\pgfqpoint{1.468092in}{1.874336in}}{\pgfqpoint{1.460192in}{1.877608in}}{\pgfqpoint{1.451956in}{1.877608in}}%
\pgfpathcurveto{\pgfqpoint{1.443720in}{1.877608in}}{\pgfqpoint{1.435820in}{1.874336in}}{\pgfqpoint{1.429996in}{1.868512in}}%
\pgfpathcurveto{\pgfqpoint{1.424172in}{1.862688in}}{\pgfqpoint{1.420899in}{1.854788in}}{\pgfqpoint{1.420899in}{1.846552in}}%
\pgfpathcurveto{\pgfqpoint{1.420899in}{1.838316in}}{\pgfqpoint{1.424172in}{1.830415in}}{\pgfqpoint{1.429996in}{1.824592in}}%
\pgfpathcurveto{\pgfqpoint{1.435820in}{1.818768in}}{\pgfqpoint{1.443720in}{1.815495in}}{\pgfqpoint{1.451956in}{1.815495in}}%
\pgfpathclose%
\pgfusepath{stroke,fill}%
\end{pgfscope}%
\begin{pgfscope}%
\pgfpathrectangle{\pgfqpoint{0.100000in}{0.212622in}}{\pgfqpoint{3.696000in}{3.696000in}}%
\pgfusepath{clip}%
\pgfsetbuttcap%
\pgfsetroundjoin%
\definecolor{currentfill}{rgb}{0.121569,0.466667,0.705882}%
\pgfsetfillcolor{currentfill}%
\pgfsetfillopacity{0.462735}%
\pgfsetlinewidth{1.003750pt}%
\definecolor{currentstroke}{rgb}{0.121569,0.466667,0.705882}%
\pgfsetstrokecolor{currentstroke}%
\pgfsetstrokeopacity{0.462735}%
\pgfsetdash{}{0pt}%
\pgfpathmoveto{\pgfqpoint{2.032441in}{1.904103in}}%
\pgfpathcurveto{\pgfqpoint{2.040677in}{1.904103in}}{\pgfqpoint{2.048577in}{1.907375in}}{\pgfqpoint{2.054401in}{1.913199in}}%
\pgfpathcurveto{\pgfqpoint{2.060225in}{1.919023in}}{\pgfqpoint{2.063497in}{1.926923in}}{\pgfqpoint{2.063497in}{1.935160in}}%
\pgfpathcurveto{\pgfqpoint{2.063497in}{1.943396in}}{\pgfqpoint{2.060225in}{1.951296in}}{\pgfqpoint{2.054401in}{1.957120in}}%
\pgfpathcurveto{\pgfqpoint{2.048577in}{1.962944in}}{\pgfqpoint{2.040677in}{1.966216in}}{\pgfqpoint{2.032441in}{1.966216in}}%
\pgfpathcurveto{\pgfqpoint{2.024204in}{1.966216in}}{\pgfqpoint{2.016304in}{1.962944in}}{\pgfqpoint{2.010480in}{1.957120in}}%
\pgfpathcurveto{\pgfqpoint{2.004656in}{1.951296in}}{\pgfqpoint{2.001384in}{1.943396in}}{\pgfqpoint{2.001384in}{1.935160in}}%
\pgfpathcurveto{\pgfqpoint{2.001384in}{1.926923in}}{\pgfqpoint{2.004656in}{1.919023in}}{\pgfqpoint{2.010480in}{1.913199in}}%
\pgfpathcurveto{\pgfqpoint{2.016304in}{1.907375in}}{\pgfqpoint{2.024204in}{1.904103in}}{\pgfqpoint{2.032441in}{1.904103in}}%
\pgfpathclose%
\pgfusepath{stroke,fill}%
\end{pgfscope}%
\begin{pgfscope}%
\pgfpathrectangle{\pgfqpoint{0.100000in}{0.212622in}}{\pgfqpoint{3.696000in}{3.696000in}}%
\pgfusepath{clip}%
\pgfsetbuttcap%
\pgfsetroundjoin%
\definecolor{currentfill}{rgb}{0.121569,0.466667,0.705882}%
\pgfsetfillcolor{currentfill}%
\pgfsetfillopacity{0.463765}%
\pgfsetlinewidth{1.003750pt}%
\definecolor{currentstroke}{rgb}{0.121569,0.466667,0.705882}%
\pgfsetstrokecolor{currentstroke}%
\pgfsetstrokeopacity{0.463765}%
\pgfsetdash{}{0pt}%
\pgfpathmoveto{\pgfqpoint{2.032780in}{1.903014in}}%
\pgfpathcurveto{\pgfqpoint{2.041016in}{1.903014in}}{\pgfqpoint{2.048916in}{1.906287in}}{\pgfqpoint{2.054740in}{1.912110in}}%
\pgfpathcurveto{\pgfqpoint{2.060564in}{1.917934in}}{\pgfqpoint{2.063836in}{1.925834in}}{\pgfqpoint{2.063836in}{1.934071in}}%
\pgfpathcurveto{\pgfqpoint{2.063836in}{1.942307in}}{\pgfqpoint{2.060564in}{1.950207in}}{\pgfqpoint{2.054740in}{1.956031in}}%
\pgfpathcurveto{\pgfqpoint{2.048916in}{1.961855in}}{\pgfqpoint{2.041016in}{1.965127in}}{\pgfqpoint{2.032780in}{1.965127in}}%
\pgfpathcurveto{\pgfqpoint{2.024544in}{1.965127in}}{\pgfqpoint{2.016644in}{1.961855in}}{\pgfqpoint{2.010820in}{1.956031in}}%
\pgfpathcurveto{\pgfqpoint{2.004996in}{1.950207in}}{\pgfqpoint{2.001723in}{1.942307in}}{\pgfqpoint{2.001723in}{1.934071in}}%
\pgfpathcurveto{\pgfqpoint{2.001723in}{1.925834in}}{\pgfqpoint{2.004996in}{1.917934in}}{\pgfqpoint{2.010820in}{1.912110in}}%
\pgfpathcurveto{\pgfqpoint{2.016644in}{1.906287in}}{\pgfqpoint{2.024544in}{1.903014in}}{\pgfqpoint{2.032780in}{1.903014in}}%
\pgfpathclose%
\pgfusepath{stroke,fill}%
\end{pgfscope}%
\begin{pgfscope}%
\pgfpathrectangle{\pgfqpoint{0.100000in}{0.212622in}}{\pgfqpoint{3.696000in}{3.696000in}}%
\pgfusepath{clip}%
\pgfsetbuttcap%
\pgfsetroundjoin%
\definecolor{currentfill}{rgb}{0.121569,0.466667,0.705882}%
\pgfsetfillcolor{currentfill}%
\pgfsetfillopacity{0.465001}%
\pgfsetlinewidth{1.003750pt}%
\definecolor{currentstroke}{rgb}{0.121569,0.466667,0.705882}%
\pgfsetstrokecolor{currentstroke}%
\pgfsetstrokeopacity{0.465001}%
\pgfsetdash{}{0pt}%
\pgfpathmoveto{\pgfqpoint{2.033392in}{1.902367in}}%
\pgfpathcurveto{\pgfqpoint{2.041628in}{1.902367in}}{\pgfqpoint{2.049528in}{1.905639in}}{\pgfqpoint{2.055352in}{1.911463in}}%
\pgfpathcurveto{\pgfqpoint{2.061176in}{1.917287in}}{\pgfqpoint{2.064448in}{1.925187in}}{\pgfqpoint{2.064448in}{1.933423in}}%
\pgfpathcurveto{\pgfqpoint{2.064448in}{1.941660in}}{\pgfqpoint{2.061176in}{1.949560in}}{\pgfqpoint{2.055352in}{1.955384in}}%
\pgfpathcurveto{\pgfqpoint{2.049528in}{1.961207in}}{\pgfqpoint{2.041628in}{1.964480in}}{\pgfqpoint{2.033392in}{1.964480in}}%
\pgfpathcurveto{\pgfqpoint{2.025155in}{1.964480in}}{\pgfqpoint{2.017255in}{1.961207in}}{\pgfqpoint{2.011432in}{1.955384in}}%
\pgfpathcurveto{\pgfqpoint{2.005608in}{1.949560in}}{\pgfqpoint{2.002335in}{1.941660in}}{\pgfqpoint{2.002335in}{1.933423in}}%
\pgfpathcurveto{\pgfqpoint{2.002335in}{1.925187in}}{\pgfqpoint{2.005608in}{1.917287in}}{\pgfqpoint{2.011432in}{1.911463in}}%
\pgfpathcurveto{\pgfqpoint{2.017255in}{1.905639in}}{\pgfqpoint{2.025155in}{1.902367in}}{\pgfqpoint{2.033392in}{1.902367in}}%
\pgfpathclose%
\pgfusepath{stroke,fill}%
\end{pgfscope}%
\begin{pgfscope}%
\pgfpathrectangle{\pgfqpoint{0.100000in}{0.212622in}}{\pgfqpoint{3.696000in}{3.696000in}}%
\pgfusepath{clip}%
\pgfsetbuttcap%
\pgfsetroundjoin%
\definecolor{currentfill}{rgb}{0.121569,0.466667,0.705882}%
\pgfsetfillcolor{currentfill}%
\pgfsetfillopacity{0.465532}%
\pgfsetlinewidth{1.003750pt}%
\definecolor{currentstroke}{rgb}{0.121569,0.466667,0.705882}%
\pgfsetstrokecolor{currentstroke}%
\pgfsetstrokeopacity{0.465532}%
\pgfsetdash{}{0pt}%
\pgfpathmoveto{\pgfqpoint{1.445456in}{1.813094in}}%
\pgfpathcurveto{\pgfqpoint{1.453693in}{1.813094in}}{\pgfqpoint{1.461593in}{1.816367in}}{\pgfqpoint{1.467417in}{1.822191in}}%
\pgfpathcurveto{\pgfqpoint{1.473241in}{1.828015in}}{\pgfqpoint{1.476513in}{1.835915in}}{\pgfqpoint{1.476513in}{1.844151in}}%
\pgfpathcurveto{\pgfqpoint{1.476513in}{1.852387in}}{\pgfqpoint{1.473241in}{1.860287in}}{\pgfqpoint{1.467417in}{1.866111in}}%
\pgfpathcurveto{\pgfqpoint{1.461593in}{1.871935in}}{\pgfqpoint{1.453693in}{1.875207in}}{\pgfqpoint{1.445456in}{1.875207in}}%
\pgfpathcurveto{\pgfqpoint{1.437220in}{1.875207in}}{\pgfqpoint{1.429320in}{1.871935in}}{\pgfqpoint{1.423496in}{1.866111in}}%
\pgfpathcurveto{\pgfqpoint{1.417672in}{1.860287in}}{\pgfqpoint{1.414400in}{1.852387in}}{\pgfqpoint{1.414400in}{1.844151in}}%
\pgfpathcurveto{\pgfqpoint{1.414400in}{1.835915in}}{\pgfqpoint{1.417672in}{1.828015in}}{\pgfqpoint{1.423496in}{1.822191in}}%
\pgfpathcurveto{\pgfqpoint{1.429320in}{1.816367in}}{\pgfqpoint{1.437220in}{1.813094in}}{\pgfqpoint{1.445456in}{1.813094in}}%
\pgfpathclose%
\pgfusepath{stroke,fill}%
\end{pgfscope}%
\begin{pgfscope}%
\pgfpathrectangle{\pgfqpoint{0.100000in}{0.212622in}}{\pgfqpoint{3.696000in}{3.696000in}}%
\pgfusepath{clip}%
\pgfsetbuttcap%
\pgfsetroundjoin%
\definecolor{currentfill}{rgb}{0.121569,0.466667,0.705882}%
\pgfsetfillcolor{currentfill}%
\pgfsetfillopacity{0.465754}%
\pgfsetlinewidth{1.003750pt}%
\definecolor{currentstroke}{rgb}{0.121569,0.466667,0.705882}%
\pgfsetstrokecolor{currentstroke}%
\pgfsetstrokeopacity{0.465754}%
\pgfsetdash{}{0pt}%
\pgfpathmoveto{\pgfqpoint{2.033933in}{1.902619in}}%
\pgfpathcurveto{\pgfqpoint{2.042170in}{1.902619in}}{\pgfqpoint{2.050070in}{1.905891in}}{\pgfqpoint{2.055894in}{1.911715in}}%
\pgfpathcurveto{\pgfqpoint{2.061717in}{1.917539in}}{\pgfqpoint{2.064990in}{1.925439in}}{\pgfqpoint{2.064990in}{1.933675in}}%
\pgfpathcurveto{\pgfqpoint{2.064990in}{1.941911in}}{\pgfqpoint{2.061717in}{1.949811in}}{\pgfqpoint{2.055894in}{1.955635in}}%
\pgfpathcurveto{\pgfqpoint{2.050070in}{1.961459in}}{\pgfqpoint{2.042170in}{1.964732in}}{\pgfqpoint{2.033933in}{1.964732in}}%
\pgfpathcurveto{\pgfqpoint{2.025697in}{1.964732in}}{\pgfqpoint{2.017797in}{1.961459in}}{\pgfqpoint{2.011973in}{1.955635in}}%
\pgfpathcurveto{\pgfqpoint{2.006149in}{1.949811in}}{\pgfqpoint{2.002877in}{1.941911in}}{\pgfqpoint{2.002877in}{1.933675in}}%
\pgfpathcurveto{\pgfqpoint{2.002877in}{1.925439in}}{\pgfqpoint{2.006149in}{1.917539in}}{\pgfqpoint{2.011973in}{1.911715in}}%
\pgfpathcurveto{\pgfqpoint{2.017797in}{1.905891in}}{\pgfqpoint{2.025697in}{1.902619in}}{\pgfqpoint{2.033933in}{1.902619in}}%
\pgfpathclose%
\pgfusepath{stroke,fill}%
\end{pgfscope}%
\begin{pgfscope}%
\pgfpathrectangle{\pgfqpoint{0.100000in}{0.212622in}}{\pgfqpoint{3.696000in}{3.696000in}}%
\pgfusepath{clip}%
\pgfsetbuttcap%
\pgfsetroundjoin%
\definecolor{currentfill}{rgb}{0.121569,0.466667,0.705882}%
\pgfsetfillcolor{currentfill}%
\pgfsetfillopacity{0.466630}%
\pgfsetlinewidth{1.003750pt}%
\definecolor{currentstroke}{rgb}{0.121569,0.466667,0.705882}%
\pgfsetstrokecolor{currentstroke}%
\pgfsetstrokeopacity{0.466630}%
\pgfsetdash{}{0pt}%
\pgfpathmoveto{\pgfqpoint{2.033824in}{1.901992in}}%
\pgfpathcurveto{\pgfqpoint{2.042061in}{1.901992in}}{\pgfqpoint{2.049961in}{1.905265in}}{\pgfqpoint{2.055785in}{1.911089in}}%
\pgfpathcurveto{\pgfqpoint{2.061609in}{1.916913in}}{\pgfqpoint{2.064881in}{1.924813in}}{\pgfqpoint{2.064881in}{1.933049in}}%
\pgfpathcurveto{\pgfqpoint{2.064881in}{1.941285in}}{\pgfqpoint{2.061609in}{1.949185in}}{\pgfqpoint{2.055785in}{1.955009in}}%
\pgfpathcurveto{\pgfqpoint{2.049961in}{1.960833in}}{\pgfqpoint{2.042061in}{1.964105in}}{\pgfqpoint{2.033824in}{1.964105in}}%
\pgfpathcurveto{\pgfqpoint{2.025588in}{1.964105in}}{\pgfqpoint{2.017688in}{1.960833in}}{\pgfqpoint{2.011864in}{1.955009in}}%
\pgfpathcurveto{\pgfqpoint{2.006040in}{1.949185in}}{\pgfqpoint{2.002768in}{1.941285in}}{\pgfqpoint{2.002768in}{1.933049in}}%
\pgfpathcurveto{\pgfqpoint{2.002768in}{1.924813in}}{\pgfqpoint{2.006040in}{1.916913in}}{\pgfqpoint{2.011864in}{1.911089in}}%
\pgfpathcurveto{\pgfqpoint{2.017688in}{1.905265in}}{\pgfqpoint{2.025588in}{1.901992in}}{\pgfqpoint{2.033824in}{1.901992in}}%
\pgfpathclose%
\pgfusepath{stroke,fill}%
\end{pgfscope}%
\begin{pgfscope}%
\pgfpathrectangle{\pgfqpoint{0.100000in}{0.212622in}}{\pgfqpoint{3.696000in}{3.696000in}}%
\pgfusepath{clip}%
\pgfsetbuttcap%
\pgfsetroundjoin%
\definecolor{currentfill}{rgb}{0.121569,0.466667,0.705882}%
\pgfsetfillcolor{currentfill}%
\pgfsetfillopacity{0.467041}%
\pgfsetlinewidth{1.003750pt}%
\definecolor{currentstroke}{rgb}{0.121569,0.466667,0.705882}%
\pgfsetstrokecolor{currentstroke}%
\pgfsetstrokeopacity{0.467041}%
\pgfsetdash{}{0pt}%
\pgfpathmoveto{\pgfqpoint{1.438737in}{1.808008in}}%
\pgfpathcurveto{\pgfqpoint{1.446973in}{1.808008in}}{\pgfqpoint{1.454873in}{1.811281in}}{\pgfqpoint{1.460697in}{1.817105in}}%
\pgfpathcurveto{\pgfqpoint{1.466521in}{1.822928in}}{\pgfqpoint{1.469794in}{1.830828in}}{\pgfqpoint{1.469794in}{1.839065in}}%
\pgfpathcurveto{\pgfqpoint{1.469794in}{1.847301in}}{\pgfqpoint{1.466521in}{1.855201in}}{\pgfqpoint{1.460697in}{1.861025in}}%
\pgfpathcurveto{\pgfqpoint{1.454873in}{1.866849in}}{\pgfqpoint{1.446973in}{1.870121in}}{\pgfqpoint{1.438737in}{1.870121in}}%
\pgfpathcurveto{\pgfqpoint{1.430501in}{1.870121in}}{\pgfqpoint{1.422601in}{1.866849in}}{\pgfqpoint{1.416777in}{1.861025in}}%
\pgfpathcurveto{\pgfqpoint{1.410953in}{1.855201in}}{\pgfqpoint{1.407681in}{1.847301in}}{\pgfqpoint{1.407681in}{1.839065in}}%
\pgfpathcurveto{\pgfqpoint{1.407681in}{1.830828in}}{\pgfqpoint{1.410953in}{1.822928in}}{\pgfqpoint{1.416777in}{1.817105in}}%
\pgfpathcurveto{\pgfqpoint{1.422601in}{1.811281in}}{\pgfqpoint{1.430501in}{1.808008in}}{\pgfqpoint{1.438737in}{1.808008in}}%
\pgfpathclose%
\pgfusepath{stroke,fill}%
\end{pgfscope}%
\begin{pgfscope}%
\pgfpathrectangle{\pgfqpoint{0.100000in}{0.212622in}}{\pgfqpoint{3.696000in}{3.696000in}}%
\pgfusepath{clip}%
\pgfsetbuttcap%
\pgfsetroundjoin%
\definecolor{currentfill}{rgb}{0.121569,0.466667,0.705882}%
\pgfsetfillcolor{currentfill}%
\pgfsetfillopacity{0.467913}%
\pgfsetlinewidth{1.003750pt}%
\definecolor{currentstroke}{rgb}{0.121569,0.466667,0.705882}%
\pgfsetstrokecolor{currentstroke}%
\pgfsetstrokeopacity{0.467913}%
\pgfsetdash{}{0pt}%
\pgfpathmoveto{\pgfqpoint{2.034928in}{1.899550in}}%
\pgfpathcurveto{\pgfqpoint{2.043165in}{1.899550in}}{\pgfqpoint{2.051065in}{1.902822in}}{\pgfqpoint{2.056889in}{1.908646in}}%
\pgfpathcurveto{\pgfqpoint{2.062713in}{1.914470in}}{\pgfqpoint{2.065985in}{1.922370in}}{\pgfqpoint{2.065985in}{1.930606in}}%
\pgfpathcurveto{\pgfqpoint{2.065985in}{1.938843in}}{\pgfqpoint{2.062713in}{1.946743in}}{\pgfqpoint{2.056889in}{1.952567in}}%
\pgfpathcurveto{\pgfqpoint{2.051065in}{1.958390in}}{\pgfqpoint{2.043165in}{1.961663in}}{\pgfqpoint{2.034928in}{1.961663in}}%
\pgfpathcurveto{\pgfqpoint{2.026692in}{1.961663in}}{\pgfqpoint{2.018792in}{1.958390in}}{\pgfqpoint{2.012968in}{1.952567in}}%
\pgfpathcurveto{\pgfqpoint{2.007144in}{1.946743in}}{\pgfqpoint{2.003872in}{1.938843in}}{\pgfqpoint{2.003872in}{1.930606in}}%
\pgfpathcurveto{\pgfqpoint{2.003872in}{1.922370in}}{\pgfqpoint{2.007144in}{1.914470in}}{\pgfqpoint{2.012968in}{1.908646in}}%
\pgfpathcurveto{\pgfqpoint{2.018792in}{1.902822in}}{\pgfqpoint{2.026692in}{1.899550in}}{\pgfqpoint{2.034928in}{1.899550in}}%
\pgfpathclose%
\pgfusepath{stroke,fill}%
\end{pgfscope}%
\begin{pgfscope}%
\pgfpathrectangle{\pgfqpoint{0.100000in}{0.212622in}}{\pgfqpoint{3.696000in}{3.696000in}}%
\pgfusepath{clip}%
\pgfsetbuttcap%
\pgfsetroundjoin%
\definecolor{currentfill}{rgb}{0.121569,0.466667,0.705882}%
\pgfsetfillcolor{currentfill}%
\pgfsetfillopacity{0.468740}%
\pgfsetlinewidth{1.003750pt}%
\definecolor{currentstroke}{rgb}{0.121569,0.466667,0.705882}%
\pgfsetstrokecolor{currentstroke}%
\pgfsetstrokeopacity{0.468740}%
\pgfsetdash{}{0pt}%
\pgfpathmoveto{\pgfqpoint{1.432886in}{1.804182in}}%
\pgfpathcurveto{\pgfqpoint{1.441122in}{1.804182in}}{\pgfqpoint{1.449022in}{1.807454in}}{\pgfqpoint{1.454846in}{1.813278in}}%
\pgfpathcurveto{\pgfqpoint{1.460670in}{1.819102in}}{\pgfqpoint{1.463942in}{1.827002in}}{\pgfqpoint{1.463942in}{1.835238in}}%
\pgfpathcurveto{\pgfqpoint{1.463942in}{1.843474in}}{\pgfqpoint{1.460670in}{1.851374in}}{\pgfqpoint{1.454846in}{1.857198in}}%
\pgfpathcurveto{\pgfqpoint{1.449022in}{1.863022in}}{\pgfqpoint{1.441122in}{1.866295in}}{\pgfqpoint{1.432886in}{1.866295in}}%
\pgfpathcurveto{\pgfqpoint{1.424649in}{1.866295in}}{\pgfqpoint{1.416749in}{1.863022in}}{\pgfqpoint{1.410925in}{1.857198in}}%
\pgfpathcurveto{\pgfqpoint{1.405101in}{1.851374in}}{\pgfqpoint{1.401829in}{1.843474in}}{\pgfqpoint{1.401829in}{1.835238in}}%
\pgfpathcurveto{\pgfqpoint{1.401829in}{1.827002in}}{\pgfqpoint{1.405101in}{1.819102in}}{\pgfqpoint{1.410925in}{1.813278in}}%
\pgfpathcurveto{\pgfqpoint{1.416749in}{1.807454in}}{\pgfqpoint{1.424649in}{1.804182in}}{\pgfqpoint{1.432886in}{1.804182in}}%
\pgfpathclose%
\pgfusepath{stroke,fill}%
\end{pgfscope}%
\begin{pgfscope}%
\pgfpathrectangle{\pgfqpoint{0.100000in}{0.212622in}}{\pgfqpoint{3.696000in}{3.696000in}}%
\pgfusepath{clip}%
\pgfsetbuttcap%
\pgfsetroundjoin%
\definecolor{currentfill}{rgb}{0.121569,0.466667,0.705882}%
\pgfsetfillcolor{currentfill}%
\pgfsetfillopacity{0.468822}%
\pgfsetlinewidth{1.003750pt}%
\definecolor{currentstroke}{rgb}{0.121569,0.466667,0.705882}%
\pgfsetstrokecolor{currentstroke}%
\pgfsetstrokeopacity{0.468822}%
\pgfsetdash{}{0pt}%
\pgfpathmoveto{\pgfqpoint{2.035517in}{1.899565in}}%
\pgfpathcurveto{\pgfqpoint{2.043753in}{1.899565in}}{\pgfqpoint{2.051653in}{1.902837in}}{\pgfqpoint{2.057477in}{1.908661in}}%
\pgfpathcurveto{\pgfqpoint{2.063301in}{1.914485in}}{\pgfqpoint{2.066573in}{1.922385in}}{\pgfqpoint{2.066573in}{1.930621in}}%
\pgfpathcurveto{\pgfqpoint{2.066573in}{1.938857in}}{\pgfqpoint{2.063301in}{1.946758in}}{\pgfqpoint{2.057477in}{1.952581in}}%
\pgfpathcurveto{\pgfqpoint{2.051653in}{1.958405in}}{\pgfqpoint{2.043753in}{1.961678in}}{\pgfqpoint{2.035517in}{1.961678in}}%
\pgfpathcurveto{\pgfqpoint{2.027280in}{1.961678in}}{\pgfqpoint{2.019380in}{1.958405in}}{\pgfqpoint{2.013556in}{1.952581in}}%
\pgfpathcurveto{\pgfqpoint{2.007732in}{1.946758in}}{\pgfqpoint{2.004460in}{1.938857in}}{\pgfqpoint{2.004460in}{1.930621in}}%
\pgfpathcurveto{\pgfqpoint{2.004460in}{1.922385in}}{\pgfqpoint{2.007732in}{1.914485in}}{\pgfqpoint{2.013556in}{1.908661in}}%
\pgfpathcurveto{\pgfqpoint{2.019380in}{1.902837in}}{\pgfqpoint{2.027280in}{1.899565in}}{\pgfqpoint{2.035517in}{1.899565in}}%
\pgfpathclose%
\pgfusepath{stroke,fill}%
\end{pgfscope}%
\begin{pgfscope}%
\pgfpathrectangle{\pgfqpoint{0.100000in}{0.212622in}}{\pgfqpoint{3.696000in}{3.696000in}}%
\pgfusepath{clip}%
\pgfsetbuttcap%
\pgfsetroundjoin%
\definecolor{currentfill}{rgb}{0.121569,0.466667,0.705882}%
\pgfsetfillcolor{currentfill}%
\pgfsetfillopacity{0.470135}%
\pgfsetlinewidth{1.003750pt}%
\definecolor{currentstroke}{rgb}{0.121569,0.466667,0.705882}%
\pgfsetstrokecolor{currentstroke}%
\pgfsetstrokeopacity{0.470135}%
\pgfsetdash{}{0pt}%
\pgfpathmoveto{\pgfqpoint{2.035876in}{1.897881in}}%
\pgfpathcurveto{\pgfqpoint{2.044112in}{1.897881in}}{\pgfqpoint{2.052013in}{1.901154in}}{\pgfqpoint{2.057836in}{1.906978in}}%
\pgfpathcurveto{\pgfqpoint{2.063660in}{1.912802in}}{\pgfqpoint{2.066933in}{1.920702in}}{\pgfqpoint{2.066933in}{1.928938in}}%
\pgfpathcurveto{\pgfqpoint{2.066933in}{1.937174in}}{\pgfqpoint{2.063660in}{1.945074in}}{\pgfqpoint{2.057836in}{1.950898in}}%
\pgfpathcurveto{\pgfqpoint{2.052013in}{1.956722in}}{\pgfqpoint{2.044112in}{1.959994in}}{\pgfqpoint{2.035876in}{1.959994in}}%
\pgfpathcurveto{\pgfqpoint{2.027640in}{1.959994in}}{\pgfqpoint{2.019740in}{1.956722in}}{\pgfqpoint{2.013916in}{1.950898in}}%
\pgfpathcurveto{\pgfqpoint{2.008092in}{1.945074in}}{\pgfqpoint{2.004820in}{1.937174in}}{\pgfqpoint{2.004820in}{1.928938in}}%
\pgfpathcurveto{\pgfqpoint{2.004820in}{1.920702in}}{\pgfqpoint{2.008092in}{1.912802in}}{\pgfqpoint{2.013916in}{1.906978in}}%
\pgfpathcurveto{\pgfqpoint{2.019740in}{1.901154in}}{\pgfqpoint{2.027640in}{1.897881in}}{\pgfqpoint{2.035876in}{1.897881in}}%
\pgfpathclose%
\pgfusepath{stroke,fill}%
\end{pgfscope}%
\begin{pgfscope}%
\pgfpathrectangle{\pgfqpoint{0.100000in}{0.212622in}}{\pgfqpoint{3.696000in}{3.696000in}}%
\pgfusepath{clip}%
\pgfsetbuttcap%
\pgfsetroundjoin%
\definecolor{currentfill}{rgb}{0.121569,0.466667,0.705882}%
\pgfsetfillcolor{currentfill}%
\pgfsetfillopacity{0.470627}%
\pgfsetlinewidth{1.003750pt}%
\definecolor{currentstroke}{rgb}{0.121569,0.466667,0.705882}%
\pgfsetstrokecolor{currentstroke}%
\pgfsetstrokeopacity{0.470627}%
\pgfsetdash{}{0pt}%
\pgfpathmoveto{\pgfqpoint{1.428003in}{1.802092in}}%
\pgfpathcurveto{\pgfqpoint{1.436240in}{1.802092in}}{\pgfqpoint{1.444140in}{1.805365in}}{\pgfqpoint{1.449964in}{1.811188in}}%
\pgfpathcurveto{\pgfqpoint{1.455788in}{1.817012in}}{\pgfqpoint{1.459060in}{1.824912in}}{\pgfqpoint{1.459060in}{1.833149in}}%
\pgfpathcurveto{\pgfqpoint{1.459060in}{1.841385in}}{\pgfqpoint{1.455788in}{1.849285in}}{\pgfqpoint{1.449964in}{1.855109in}}%
\pgfpathcurveto{\pgfqpoint{1.444140in}{1.860933in}}{\pgfqpoint{1.436240in}{1.864205in}}{\pgfqpoint{1.428003in}{1.864205in}}%
\pgfpathcurveto{\pgfqpoint{1.419767in}{1.864205in}}{\pgfqpoint{1.411867in}{1.860933in}}{\pgfqpoint{1.406043in}{1.855109in}}%
\pgfpathcurveto{\pgfqpoint{1.400219in}{1.849285in}}{\pgfqpoint{1.396947in}{1.841385in}}{\pgfqpoint{1.396947in}{1.833149in}}%
\pgfpathcurveto{\pgfqpoint{1.396947in}{1.824912in}}{\pgfqpoint{1.400219in}{1.817012in}}{\pgfqpoint{1.406043in}{1.811188in}}%
\pgfpathcurveto{\pgfqpoint{1.411867in}{1.805365in}}{\pgfqpoint{1.419767in}{1.802092in}}{\pgfqpoint{1.428003in}{1.802092in}}%
\pgfpathclose%
\pgfusepath{stroke,fill}%
\end{pgfscope}%
\begin{pgfscope}%
\pgfpathrectangle{\pgfqpoint{0.100000in}{0.212622in}}{\pgfqpoint{3.696000in}{3.696000in}}%
\pgfusepath{clip}%
\pgfsetbuttcap%
\pgfsetroundjoin%
\definecolor{currentfill}{rgb}{0.121569,0.466667,0.705882}%
\pgfsetfillcolor{currentfill}%
\pgfsetfillopacity{0.471569}%
\pgfsetlinewidth{1.003750pt}%
\definecolor{currentstroke}{rgb}{0.121569,0.466667,0.705882}%
\pgfsetstrokecolor{currentstroke}%
\pgfsetstrokeopacity{0.471569}%
\pgfsetdash{}{0pt}%
\pgfpathmoveto{\pgfqpoint{1.423673in}{1.797736in}}%
\pgfpathcurveto{\pgfqpoint{1.431910in}{1.797736in}}{\pgfqpoint{1.439810in}{1.801008in}}{\pgfqpoint{1.445633in}{1.806832in}}%
\pgfpathcurveto{\pgfqpoint{1.451457in}{1.812656in}}{\pgfqpoint{1.454730in}{1.820556in}}{\pgfqpoint{1.454730in}{1.828792in}}%
\pgfpathcurveto{\pgfqpoint{1.454730in}{1.837029in}}{\pgfqpoint{1.451457in}{1.844929in}}{\pgfqpoint{1.445633in}{1.850753in}}%
\pgfpathcurveto{\pgfqpoint{1.439810in}{1.856576in}}{\pgfqpoint{1.431910in}{1.859849in}}{\pgfqpoint{1.423673in}{1.859849in}}%
\pgfpathcurveto{\pgfqpoint{1.415437in}{1.859849in}}{\pgfqpoint{1.407537in}{1.856576in}}{\pgfqpoint{1.401713in}{1.850753in}}%
\pgfpathcurveto{\pgfqpoint{1.395889in}{1.844929in}}{\pgfqpoint{1.392617in}{1.837029in}}{\pgfqpoint{1.392617in}{1.828792in}}%
\pgfpathcurveto{\pgfqpoint{1.392617in}{1.820556in}}{\pgfqpoint{1.395889in}{1.812656in}}{\pgfqpoint{1.401713in}{1.806832in}}%
\pgfpathcurveto{\pgfqpoint{1.407537in}{1.801008in}}{\pgfqpoint{1.415437in}{1.797736in}}{\pgfqpoint{1.423673in}{1.797736in}}%
\pgfpathclose%
\pgfusepath{stroke,fill}%
\end{pgfscope}%
\begin{pgfscope}%
\pgfpathrectangle{\pgfqpoint{0.100000in}{0.212622in}}{\pgfqpoint{3.696000in}{3.696000in}}%
\pgfusepath{clip}%
\pgfsetbuttcap%
\pgfsetroundjoin%
\definecolor{currentfill}{rgb}{0.121569,0.466667,0.705882}%
\pgfsetfillcolor{currentfill}%
\pgfsetfillopacity{0.472153}%
\pgfsetlinewidth{1.003750pt}%
\definecolor{currentstroke}{rgb}{0.121569,0.466667,0.705882}%
\pgfsetstrokecolor{currentstroke}%
\pgfsetstrokeopacity{0.472153}%
\pgfsetdash{}{0pt}%
\pgfpathmoveto{\pgfqpoint{2.036493in}{1.897485in}}%
\pgfpathcurveto{\pgfqpoint{2.044729in}{1.897485in}}{\pgfqpoint{2.052629in}{1.900758in}}{\pgfqpoint{2.058453in}{1.906582in}}%
\pgfpathcurveto{\pgfqpoint{2.064277in}{1.912405in}}{\pgfqpoint{2.067549in}{1.920306in}}{\pgfqpoint{2.067549in}{1.928542in}}%
\pgfpathcurveto{\pgfqpoint{2.067549in}{1.936778in}}{\pgfqpoint{2.064277in}{1.944678in}}{\pgfqpoint{2.058453in}{1.950502in}}%
\pgfpathcurveto{\pgfqpoint{2.052629in}{1.956326in}}{\pgfqpoint{2.044729in}{1.959598in}}{\pgfqpoint{2.036493in}{1.959598in}}%
\pgfpathcurveto{\pgfqpoint{2.028257in}{1.959598in}}{\pgfqpoint{2.020357in}{1.956326in}}{\pgfqpoint{2.014533in}{1.950502in}}%
\pgfpathcurveto{\pgfqpoint{2.008709in}{1.944678in}}{\pgfqpoint{2.005436in}{1.936778in}}{\pgfqpoint{2.005436in}{1.928542in}}%
\pgfpathcurveto{\pgfqpoint{2.005436in}{1.920306in}}{\pgfqpoint{2.008709in}{1.912405in}}{\pgfqpoint{2.014533in}{1.906582in}}%
\pgfpathcurveto{\pgfqpoint{2.020357in}{1.900758in}}{\pgfqpoint{2.028257in}{1.897485in}}{\pgfqpoint{2.036493in}{1.897485in}}%
\pgfpathclose%
\pgfusepath{stroke,fill}%
\end{pgfscope}%
\begin{pgfscope}%
\pgfpathrectangle{\pgfqpoint{0.100000in}{0.212622in}}{\pgfqpoint{3.696000in}{3.696000in}}%
\pgfusepath{clip}%
\pgfsetbuttcap%
\pgfsetroundjoin%
\definecolor{currentfill}{rgb}{0.121569,0.466667,0.705882}%
\pgfsetfillcolor{currentfill}%
\pgfsetfillopacity{0.473914}%
\pgfsetlinewidth{1.003750pt}%
\definecolor{currentstroke}{rgb}{0.121569,0.466667,0.705882}%
\pgfsetstrokecolor{currentstroke}%
\pgfsetstrokeopacity{0.473914}%
\pgfsetdash{}{0pt}%
\pgfpathmoveto{\pgfqpoint{1.416246in}{1.793242in}}%
\pgfpathcurveto{\pgfqpoint{1.424483in}{1.793242in}}{\pgfqpoint{1.432383in}{1.796515in}}{\pgfqpoint{1.438206in}{1.802339in}}%
\pgfpathcurveto{\pgfqpoint{1.444030in}{1.808163in}}{\pgfqpoint{1.447303in}{1.816063in}}{\pgfqpoint{1.447303in}{1.824299in}}%
\pgfpathcurveto{\pgfqpoint{1.447303in}{1.832535in}}{\pgfqpoint{1.444030in}{1.840435in}}{\pgfqpoint{1.438206in}{1.846259in}}%
\pgfpathcurveto{\pgfqpoint{1.432383in}{1.852083in}}{\pgfqpoint{1.424483in}{1.855355in}}{\pgfqpoint{1.416246in}{1.855355in}}%
\pgfpathcurveto{\pgfqpoint{1.408010in}{1.855355in}}{\pgfqpoint{1.400110in}{1.852083in}}{\pgfqpoint{1.394286in}{1.846259in}}%
\pgfpathcurveto{\pgfqpoint{1.388462in}{1.840435in}}{\pgfqpoint{1.385190in}{1.832535in}}{\pgfqpoint{1.385190in}{1.824299in}}%
\pgfpathcurveto{\pgfqpoint{1.385190in}{1.816063in}}{\pgfqpoint{1.388462in}{1.808163in}}{\pgfqpoint{1.394286in}{1.802339in}}%
\pgfpathcurveto{\pgfqpoint{1.400110in}{1.796515in}}{\pgfqpoint{1.408010in}{1.793242in}}{\pgfqpoint{1.416246in}{1.793242in}}%
\pgfpathclose%
\pgfusepath{stroke,fill}%
\end{pgfscope}%
\begin{pgfscope}%
\pgfpathrectangle{\pgfqpoint{0.100000in}{0.212622in}}{\pgfqpoint{3.696000in}{3.696000in}}%
\pgfusepath{clip}%
\pgfsetbuttcap%
\pgfsetroundjoin%
\definecolor{currentfill}{rgb}{0.121569,0.466667,0.705882}%
\pgfsetfillcolor{currentfill}%
\pgfsetfillopacity{0.474297}%
\pgfsetlinewidth{1.003750pt}%
\definecolor{currentstroke}{rgb}{0.121569,0.466667,0.705882}%
\pgfsetstrokecolor{currentstroke}%
\pgfsetstrokeopacity{0.474297}%
\pgfsetdash{}{0pt}%
\pgfpathmoveto{\pgfqpoint{2.037402in}{1.895151in}}%
\pgfpathcurveto{\pgfqpoint{2.045639in}{1.895151in}}{\pgfqpoint{2.053539in}{1.898423in}}{\pgfqpoint{2.059363in}{1.904247in}}%
\pgfpathcurveto{\pgfqpoint{2.065187in}{1.910071in}}{\pgfqpoint{2.068459in}{1.917971in}}{\pgfqpoint{2.068459in}{1.926207in}}%
\pgfpathcurveto{\pgfqpoint{2.068459in}{1.934444in}}{\pgfqpoint{2.065187in}{1.942344in}}{\pgfqpoint{2.059363in}{1.948168in}}%
\pgfpathcurveto{\pgfqpoint{2.053539in}{1.953992in}}{\pgfqpoint{2.045639in}{1.957264in}}{\pgfqpoint{2.037402in}{1.957264in}}%
\pgfpathcurveto{\pgfqpoint{2.029166in}{1.957264in}}{\pgfqpoint{2.021266in}{1.953992in}}{\pgfqpoint{2.015442in}{1.948168in}}%
\pgfpathcurveto{\pgfqpoint{2.009618in}{1.942344in}}{\pgfqpoint{2.006346in}{1.934444in}}{\pgfqpoint{2.006346in}{1.926207in}}%
\pgfpathcurveto{\pgfqpoint{2.006346in}{1.917971in}}{\pgfqpoint{2.009618in}{1.910071in}}{\pgfqpoint{2.015442in}{1.904247in}}%
\pgfpathcurveto{\pgfqpoint{2.021266in}{1.898423in}}{\pgfqpoint{2.029166in}{1.895151in}}{\pgfqpoint{2.037402in}{1.895151in}}%
\pgfpathclose%
\pgfusepath{stroke,fill}%
\end{pgfscope}%
\begin{pgfscope}%
\pgfpathrectangle{\pgfqpoint{0.100000in}{0.212622in}}{\pgfqpoint{3.696000in}{3.696000in}}%
\pgfusepath{clip}%
\pgfsetbuttcap%
\pgfsetroundjoin%
\definecolor{currentfill}{rgb}{0.121569,0.466667,0.705882}%
\pgfsetfillcolor{currentfill}%
\pgfsetfillopacity{0.476500}%
\pgfsetlinewidth{1.003750pt}%
\definecolor{currentstroke}{rgb}{0.121569,0.466667,0.705882}%
\pgfsetstrokecolor{currentstroke}%
\pgfsetstrokeopacity{0.476500}%
\pgfsetdash{}{0pt}%
\pgfpathmoveto{\pgfqpoint{1.410496in}{1.791797in}}%
\pgfpathcurveto{\pgfqpoint{1.418733in}{1.791797in}}{\pgfqpoint{1.426633in}{1.795069in}}{\pgfqpoint{1.432457in}{1.800893in}}%
\pgfpathcurveto{\pgfqpoint{1.438281in}{1.806717in}}{\pgfqpoint{1.441553in}{1.814617in}}{\pgfqpoint{1.441553in}{1.822854in}}%
\pgfpathcurveto{\pgfqpoint{1.441553in}{1.831090in}}{\pgfqpoint{1.438281in}{1.838990in}}{\pgfqpoint{1.432457in}{1.844814in}}%
\pgfpathcurveto{\pgfqpoint{1.426633in}{1.850638in}}{\pgfqpoint{1.418733in}{1.853910in}}{\pgfqpoint{1.410496in}{1.853910in}}%
\pgfpathcurveto{\pgfqpoint{1.402260in}{1.853910in}}{\pgfqpoint{1.394360in}{1.850638in}}{\pgfqpoint{1.388536in}{1.844814in}}%
\pgfpathcurveto{\pgfqpoint{1.382712in}{1.838990in}}{\pgfqpoint{1.379440in}{1.831090in}}{\pgfqpoint{1.379440in}{1.822854in}}%
\pgfpathcurveto{\pgfqpoint{1.379440in}{1.814617in}}{\pgfqpoint{1.382712in}{1.806717in}}{\pgfqpoint{1.388536in}{1.800893in}}%
\pgfpathcurveto{\pgfqpoint{1.394360in}{1.795069in}}{\pgfqpoint{1.402260in}{1.791797in}}{\pgfqpoint{1.410496in}{1.791797in}}%
\pgfpathclose%
\pgfusepath{stroke,fill}%
\end{pgfscope}%
\begin{pgfscope}%
\pgfpathrectangle{\pgfqpoint{0.100000in}{0.212622in}}{\pgfqpoint{3.696000in}{3.696000in}}%
\pgfusepath{clip}%
\pgfsetbuttcap%
\pgfsetroundjoin%
\definecolor{currentfill}{rgb}{0.121569,0.466667,0.705882}%
\pgfsetfillcolor{currentfill}%
\pgfsetfillopacity{0.476826}%
\pgfsetlinewidth{1.003750pt}%
\definecolor{currentstroke}{rgb}{0.121569,0.466667,0.705882}%
\pgfsetstrokecolor{currentstroke}%
\pgfsetstrokeopacity{0.476826}%
\pgfsetdash{}{0pt}%
\pgfpathmoveto{\pgfqpoint{2.038903in}{1.893570in}}%
\pgfpathcurveto{\pgfqpoint{2.047139in}{1.893570in}}{\pgfqpoint{2.055039in}{1.896842in}}{\pgfqpoint{2.060863in}{1.902666in}}%
\pgfpathcurveto{\pgfqpoint{2.066687in}{1.908490in}}{\pgfqpoint{2.069960in}{1.916390in}}{\pgfqpoint{2.069960in}{1.924626in}}%
\pgfpathcurveto{\pgfqpoint{2.069960in}{1.932863in}}{\pgfqpoint{2.066687in}{1.940763in}}{\pgfqpoint{2.060863in}{1.946587in}}%
\pgfpathcurveto{\pgfqpoint{2.055039in}{1.952411in}}{\pgfqpoint{2.047139in}{1.955683in}}{\pgfqpoint{2.038903in}{1.955683in}}%
\pgfpathcurveto{\pgfqpoint{2.030667in}{1.955683in}}{\pgfqpoint{2.022767in}{1.952411in}}{\pgfqpoint{2.016943in}{1.946587in}}%
\pgfpathcurveto{\pgfqpoint{2.011119in}{1.940763in}}{\pgfqpoint{2.007847in}{1.932863in}}{\pgfqpoint{2.007847in}{1.924626in}}%
\pgfpathcurveto{\pgfqpoint{2.007847in}{1.916390in}}{\pgfqpoint{2.011119in}{1.908490in}}{\pgfqpoint{2.016943in}{1.902666in}}%
\pgfpathcurveto{\pgfqpoint{2.022767in}{1.896842in}}{\pgfqpoint{2.030667in}{1.893570in}}{\pgfqpoint{2.038903in}{1.893570in}}%
\pgfpathclose%
\pgfusepath{stroke,fill}%
\end{pgfscope}%
\begin{pgfscope}%
\pgfpathrectangle{\pgfqpoint{0.100000in}{0.212622in}}{\pgfqpoint{3.696000in}{3.696000in}}%
\pgfusepath{clip}%
\pgfsetbuttcap%
\pgfsetroundjoin%
\definecolor{currentfill}{rgb}{0.121569,0.466667,0.705882}%
\pgfsetfillcolor{currentfill}%
\pgfsetfillopacity{0.477753}%
\pgfsetlinewidth{1.003750pt}%
\definecolor{currentstroke}{rgb}{0.121569,0.466667,0.705882}%
\pgfsetstrokecolor{currentstroke}%
\pgfsetstrokeopacity{0.477753}%
\pgfsetdash{}{0pt}%
\pgfpathmoveto{\pgfqpoint{1.405181in}{1.787956in}}%
\pgfpathcurveto{\pgfqpoint{1.413418in}{1.787956in}}{\pgfqpoint{1.421318in}{1.791228in}}{\pgfqpoint{1.427142in}{1.797052in}}%
\pgfpathcurveto{\pgfqpoint{1.432966in}{1.802876in}}{\pgfqpoint{1.436238in}{1.810776in}}{\pgfqpoint{1.436238in}{1.819013in}}%
\pgfpathcurveto{\pgfqpoint{1.436238in}{1.827249in}}{\pgfqpoint{1.432966in}{1.835149in}}{\pgfqpoint{1.427142in}{1.840973in}}%
\pgfpathcurveto{\pgfqpoint{1.421318in}{1.846797in}}{\pgfqpoint{1.413418in}{1.850069in}}{\pgfqpoint{1.405181in}{1.850069in}}%
\pgfpathcurveto{\pgfqpoint{1.396945in}{1.850069in}}{\pgfqpoint{1.389045in}{1.846797in}}{\pgfqpoint{1.383221in}{1.840973in}}%
\pgfpathcurveto{\pgfqpoint{1.377397in}{1.835149in}}{\pgfqpoint{1.374125in}{1.827249in}}{\pgfqpoint{1.374125in}{1.819013in}}%
\pgfpathcurveto{\pgfqpoint{1.374125in}{1.810776in}}{\pgfqpoint{1.377397in}{1.802876in}}{\pgfqpoint{1.383221in}{1.797052in}}%
\pgfpathcurveto{\pgfqpoint{1.389045in}{1.791228in}}{\pgfqpoint{1.396945in}{1.787956in}}{\pgfqpoint{1.405181in}{1.787956in}}%
\pgfpathclose%
\pgfusepath{stroke,fill}%
\end{pgfscope}%
\begin{pgfscope}%
\pgfpathrectangle{\pgfqpoint{0.100000in}{0.212622in}}{\pgfqpoint{3.696000in}{3.696000in}}%
\pgfusepath{clip}%
\pgfsetbuttcap%
\pgfsetroundjoin%
\definecolor{currentfill}{rgb}{0.121569,0.466667,0.705882}%
\pgfsetfillcolor{currentfill}%
\pgfsetfillopacity{0.479136}%
\pgfsetlinewidth{1.003750pt}%
\definecolor{currentstroke}{rgb}{0.121569,0.466667,0.705882}%
\pgfsetstrokecolor{currentstroke}%
\pgfsetstrokeopacity{0.479136}%
\pgfsetdash{}{0pt}%
\pgfpathmoveto{\pgfqpoint{1.400928in}{1.784694in}}%
\pgfpathcurveto{\pgfqpoint{1.409164in}{1.784694in}}{\pgfqpoint{1.417064in}{1.787966in}}{\pgfqpoint{1.422888in}{1.793790in}}%
\pgfpathcurveto{\pgfqpoint{1.428712in}{1.799614in}}{\pgfqpoint{1.431984in}{1.807514in}}{\pgfqpoint{1.431984in}{1.815750in}}%
\pgfpathcurveto{\pgfqpoint{1.431984in}{1.823987in}}{\pgfqpoint{1.428712in}{1.831887in}}{\pgfqpoint{1.422888in}{1.837711in}}%
\pgfpathcurveto{\pgfqpoint{1.417064in}{1.843535in}}{\pgfqpoint{1.409164in}{1.846807in}}{\pgfqpoint{1.400928in}{1.846807in}}%
\pgfpathcurveto{\pgfqpoint{1.392692in}{1.846807in}}{\pgfqpoint{1.384791in}{1.843535in}}{\pgfqpoint{1.378968in}{1.837711in}}%
\pgfpathcurveto{\pgfqpoint{1.373144in}{1.831887in}}{\pgfqpoint{1.369871in}{1.823987in}}{\pgfqpoint{1.369871in}{1.815750in}}%
\pgfpathcurveto{\pgfqpoint{1.369871in}{1.807514in}}{\pgfqpoint{1.373144in}{1.799614in}}{\pgfqpoint{1.378968in}{1.793790in}}%
\pgfpathcurveto{\pgfqpoint{1.384791in}{1.787966in}}{\pgfqpoint{1.392692in}{1.784694in}}{\pgfqpoint{1.400928in}{1.784694in}}%
\pgfpathclose%
\pgfusepath{stroke,fill}%
\end{pgfscope}%
\begin{pgfscope}%
\pgfpathrectangle{\pgfqpoint{0.100000in}{0.212622in}}{\pgfqpoint{3.696000in}{3.696000in}}%
\pgfusepath{clip}%
\pgfsetbuttcap%
\pgfsetroundjoin%
\definecolor{currentfill}{rgb}{0.121569,0.466667,0.705882}%
\pgfsetfillcolor{currentfill}%
\pgfsetfillopacity{0.479269}%
\pgfsetlinewidth{1.003750pt}%
\definecolor{currentstroke}{rgb}{0.121569,0.466667,0.705882}%
\pgfsetstrokecolor{currentstroke}%
\pgfsetstrokeopacity{0.479269}%
\pgfsetdash{}{0pt}%
\pgfpathmoveto{\pgfqpoint{2.040740in}{1.888825in}}%
\pgfpathcurveto{\pgfqpoint{2.048977in}{1.888825in}}{\pgfqpoint{2.056877in}{1.892097in}}{\pgfqpoint{2.062701in}{1.897921in}}%
\pgfpathcurveto{\pgfqpoint{2.068525in}{1.903745in}}{\pgfqpoint{2.071797in}{1.911645in}}{\pgfqpoint{2.071797in}{1.919882in}}%
\pgfpathcurveto{\pgfqpoint{2.071797in}{1.928118in}}{\pgfqpoint{2.068525in}{1.936018in}}{\pgfqpoint{2.062701in}{1.941842in}}%
\pgfpathcurveto{\pgfqpoint{2.056877in}{1.947666in}}{\pgfqpoint{2.048977in}{1.950938in}}{\pgfqpoint{2.040740in}{1.950938in}}%
\pgfpathcurveto{\pgfqpoint{2.032504in}{1.950938in}}{\pgfqpoint{2.024604in}{1.947666in}}{\pgfqpoint{2.018780in}{1.941842in}}%
\pgfpathcurveto{\pgfqpoint{2.012956in}{1.936018in}}{\pgfqpoint{2.009684in}{1.928118in}}{\pgfqpoint{2.009684in}{1.919882in}}%
\pgfpathcurveto{\pgfqpoint{2.009684in}{1.911645in}}{\pgfqpoint{2.012956in}{1.903745in}}{\pgfqpoint{2.018780in}{1.897921in}}%
\pgfpathcurveto{\pgfqpoint{2.024604in}{1.892097in}}{\pgfqpoint{2.032504in}{1.888825in}}{\pgfqpoint{2.040740in}{1.888825in}}%
\pgfpathclose%
\pgfusepath{stroke,fill}%
\end{pgfscope}%
\begin{pgfscope}%
\pgfpathrectangle{\pgfqpoint{0.100000in}{0.212622in}}{\pgfqpoint{3.696000in}{3.696000in}}%
\pgfusepath{clip}%
\pgfsetbuttcap%
\pgfsetroundjoin%
\definecolor{currentfill}{rgb}{0.121569,0.466667,0.705882}%
\pgfsetfillcolor{currentfill}%
\pgfsetfillopacity{0.481008}%
\pgfsetlinewidth{1.003750pt}%
\definecolor{currentstroke}{rgb}{0.121569,0.466667,0.705882}%
\pgfsetstrokecolor{currentstroke}%
\pgfsetstrokeopacity{0.481008}%
\pgfsetdash{}{0pt}%
\pgfpathmoveto{\pgfqpoint{2.041325in}{1.888665in}}%
\pgfpathcurveto{\pgfqpoint{2.049561in}{1.888665in}}{\pgfqpoint{2.057461in}{1.891938in}}{\pgfqpoint{2.063285in}{1.897762in}}%
\pgfpathcurveto{\pgfqpoint{2.069109in}{1.903586in}}{\pgfqpoint{2.072381in}{1.911486in}}{\pgfqpoint{2.072381in}{1.919722in}}%
\pgfpathcurveto{\pgfqpoint{2.072381in}{1.927958in}}{\pgfqpoint{2.069109in}{1.935858in}}{\pgfqpoint{2.063285in}{1.941682in}}%
\pgfpathcurveto{\pgfqpoint{2.057461in}{1.947506in}}{\pgfqpoint{2.049561in}{1.950778in}}{\pgfqpoint{2.041325in}{1.950778in}}%
\pgfpathcurveto{\pgfqpoint{2.033088in}{1.950778in}}{\pgfqpoint{2.025188in}{1.947506in}}{\pgfqpoint{2.019365in}{1.941682in}}%
\pgfpathcurveto{\pgfqpoint{2.013541in}{1.935858in}}{\pgfqpoint{2.010268in}{1.927958in}}{\pgfqpoint{2.010268in}{1.919722in}}%
\pgfpathcurveto{\pgfqpoint{2.010268in}{1.911486in}}{\pgfqpoint{2.013541in}{1.903586in}}{\pgfqpoint{2.019365in}{1.897762in}}%
\pgfpathcurveto{\pgfqpoint{2.025188in}{1.891938in}}{\pgfqpoint{2.033088in}{1.888665in}}{\pgfqpoint{2.041325in}{1.888665in}}%
\pgfpathclose%
\pgfusepath{stroke,fill}%
\end{pgfscope}%
\begin{pgfscope}%
\pgfpathrectangle{\pgfqpoint{0.100000in}{0.212622in}}{\pgfqpoint{3.696000in}{3.696000in}}%
\pgfusepath{clip}%
\pgfsetbuttcap%
\pgfsetroundjoin%
\definecolor{currentfill}{rgb}{0.121569,0.466667,0.705882}%
\pgfsetfillcolor{currentfill}%
\pgfsetfillopacity{0.481895}%
\pgfsetlinewidth{1.003750pt}%
\definecolor{currentstroke}{rgb}{0.121569,0.466667,0.705882}%
\pgfsetstrokecolor{currentstroke}%
\pgfsetstrokeopacity{0.481895}%
\pgfsetdash{}{0pt}%
\pgfpathmoveto{\pgfqpoint{1.392960in}{1.780670in}}%
\pgfpathcurveto{\pgfqpoint{1.401197in}{1.780670in}}{\pgfqpoint{1.409097in}{1.783942in}}{\pgfqpoint{1.414921in}{1.789766in}}%
\pgfpathcurveto{\pgfqpoint{1.420745in}{1.795590in}}{\pgfqpoint{1.424017in}{1.803490in}}{\pgfqpoint{1.424017in}{1.811727in}}%
\pgfpathcurveto{\pgfqpoint{1.424017in}{1.819963in}}{\pgfqpoint{1.420745in}{1.827863in}}{\pgfqpoint{1.414921in}{1.833687in}}%
\pgfpathcurveto{\pgfqpoint{1.409097in}{1.839511in}}{\pgfqpoint{1.401197in}{1.842783in}}{\pgfqpoint{1.392960in}{1.842783in}}%
\pgfpathcurveto{\pgfqpoint{1.384724in}{1.842783in}}{\pgfqpoint{1.376824in}{1.839511in}}{\pgfqpoint{1.371000in}{1.833687in}}%
\pgfpathcurveto{\pgfqpoint{1.365176in}{1.827863in}}{\pgfqpoint{1.361904in}{1.819963in}}{\pgfqpoint{1.361904in}{1.811727in}}%
\pgfpathcurveto{\pgfqpoint{1.361904in}{1.803490in}}{\pgfqpoint{1.365176in}{1.795590in}}{\pgfqpoint{1.371000in}{1.789766in}}%
\pgfpathcurveto{\pgfqpoint{1.376824in}{1.783942in}}{\pgfqpoint{1.384724in}{1.780670in}}{\pgfqpoint{1.392960in}{1.780670in}}%
\pgfpathclose%
\pgfusepath{stroke,fill}%
\end{pgfscope}%
\begin{pgfscope}%
\pgfpathrectangle{\pgfqpoint{0.100000in}{0.212622in}}{\pgfqpoint{3.696000in}{3.696000in}}%
\pgfusepath{clip}%
\pgfsetbuttcap%
\pgfsetroundjoin%
\definecolor{currentfill}{rgb}{0.121569,0.466667,0.705882}%
\pgfsetfillcolor{currentfill}%
\pgfsetfillopacity{0.482828}%
\pgfsetlinewidth{1.003750pt}%
\definecolor{currentstroke}{rgb}{0.121569,0.466667,0.705882}%
\pgfsetstrokecolor{currentstroke}%
\pgfsetstrokeopacity{0.482828}%
\pgfsetdash{}{0pt}%
\pgfpathmoveto{\pgfqpoint{2.042365in}{1.886370in}}%
\pgfpathcurveto{\pgfqpoint{2.050602in}{1.886370in}}{\pgfqpoint{2.058502in}{1.889642in}}{\pgfqpoint{2.064326in}{1.895466in}}%
\pgfpathcurveto{\pgfqpoint{2.070149in}{1.901290in}}{\pgfqpoint{2.073422in}{1.909190in}}{\pgfqpoint{2.073422in}{1.917426in}}%
\pgfpathcurveto{\pgfqpoint{2.073422in}{1.925662in}}{\pgfqpoint{2.070149in}{1.933562in}}{\pgfqpoint{2.064326in}{1.939386in}}%
\pgfpathcurveto{\pgfqpoint{2.058502in}{1.945210in}}{\pgfqpoint{2.050602in}{1.948483in}}{\pgfqpoint{2.042365in}{1.948483in}}%
\pgfpathcurveto{\pgfqpoint{2.034129in}{1.948483in}}{\pgfqpoint{2.026229in}{1.945210in}}{\pgfqpoint{2.020405in}{1.939386in}}%
\pgfpathcurveto{\pgfqpoint{2.014581in}{1.933562in}}{\pgfqpoint{2.011309in}{1.925662in}}{\pgfqpoint{2.011309in}{1.917426in}}%
\pgfpathcurveto{\pgfqpoint{2.011309in}{1.909190in}}{\pgfqpoint{2.014581in}{1.901290in}}{\pgfqpoint{2.020405in}{1.895466in}}%
\pgfpathcurveto{\pgfqpoint{2.026229in}{1.889642in}}{\pgfqpoint{2.034129in}{1.886370in}}{\pgfqpoint{2.042365in}{1.886370in}}%
\pgfpathclose%
\pgfusepath{stroke,fill}%
\end{pgfscope}%
\begin{pgfscope}%
\pgfpathrectangle{\pgfqpoint{0.100000in}{0.212622in}}{\pgfqpoint{3.696000in}{3.696000in}}%
\pgfusepath{clip}%
\pgfsetbuttcap%
\pgfsetroundjoin%
\definecolor{currentfill}{rgb}{0.121569,0.466667,0.705882}%
\pgfsetfillcolor{currentfill}%
\pgfsetfillopacity{0.483154}%
\pgfsetlinewidth{1.003750pt}%
\definecolor{currentstroke}{rgb}{0.121569,0.466667,0.705882}%
\pgfsetstrokecolor{currentstroke}%
\pgfsetstrokeopacity{0.483154}%
\pgfsetdash{}{0pt}%
\pgfpathmoveto{\pgfqpoint{1.387627in}{1.773697in}}%
\pgfpathcurveto{\pgfqpoint{1.395863in}{1.773697in}}{\pgfqpoint{1.403763in}{1.776969in}}{\pgfqpoint{1.409587in}{1.782793in}}%
\pgfpathcurveto{\pgfqpoint{1.415411in}{1.788617in}}{\pgfqpoint{1.418683in}{1.796517in}}{\pgfqpoint{1.418683in}{1.804753in}}%
\pgfpathcurveto{\pgfqpoint{1.418683in}{1.812990in}}{\pgfqpoint{1.415411in}{1.820890in}}{\pgfqpoint{1.409587in}{1.826714in}}%
\pgfpathcurveto{\pgfqpoint{1.403763in}{1.832538in}}{\pgfqpoint{1.395863in}{1.835810in}}{\pgfqpoint{1.387627in}{1.835810in}}%
\pgfpathcurveto{\pgfqpoint{1.379391in}{1.835810in}}{\pgfqpoint{1.371491in}{1.832538in}}{\pgfqpoint{1.365667in}{1.826714in}}%
\pgfpathcurveto{\pgfqpoint{1.359843in}{1.820890in}}{\pgfqpoint{1.356570in}{1.812990in}}{\pgfqpoint{1.356570in}{1.804753in}}%
\pgfpathcurveto{\pgfqpoint{1.356570in}{1.796517in}}{\pgfqpoint{1.359843in}{1.788617in}}{\pgfqpoint{1.365667in}{1.782793in}}%
\pgfpathcurveto{\pgfqpoint{1.371491in}{1.776969in}}{\pgfqpoint{1.379391in}{1.773697in}}{\pgfqpoint{1.387627in}{1.773697in}}%
\pgfpathclose%
\pgfusepath{stroke,fill}%
\end{pgfscope}%
\begin{pgfscope}%
\pgfpathrectangle{\pgfqpoint{0.100000in}{0.212622in}}{\pgfqpoint{3.696000in}{3.696000in}}%
\pgfusepath{clip}%
\pgfsetbuttcap%
\pgfsetroundjoin%
\definecolor{currentfill}{rgb}{0.121569,0.466667,0.705882}%
\pgfsetfillcolor{currentfill}%
\pgfsetfillopacity{0.484246}%
\pgfsetlinewidth{1.003750pt}%
\definecolor{currentstroke}{rgb}{0.121569,0.466667,0.705882}%
\pgfsetstrokecolor{currentstroke}%
\pgfsetstrokeopacity{0.484246}%
\pgfsetdash{}{0pt}%
\pgfpathmoveto{\pgfqpoint{1.383144in}{1.770982in}}%
\pgfpathcurveto{\pgfqpoint{1.391380in}{1.770982in}}{\pgfqpoint{1.399280in}{1.774254in}}{\pgfqpoint{1.405104in}{1.780078in}}%
\pgfpathcurveto{\pgfqpoint{1.410928in}{1.785902in}}{\pgfqpoint{1.414200in}{1.793802in}}{\pgfqpoint{1.414200in}{1.802038in}}%
\pgfpathcurveto{\pgfqpoint{1.414200in}{1.810275in}}{\pgfqpoint{1.410928in}{1.818175in}}{\pgfqpoint{1.405104in}{1.823999in}}%
\pgfpathcurveto{\pgfqpoint{1.399280in}{1.829823in}}{\pgfqpoint{1.391380in}{1.833095in}}{\pgfqpoint{1.383144in}{1.833095in}}%
\pgfpathcurveto{\pgfqpoint{1.374907in}{1.833095in}}{\pgfqpoint{1.367007in}{1.829823in}}{\pgfqpoint{1.361183in}{1.823999in}}%
\pgfpathcurveto{\pgfqpoint{1.355359in}{1.818175in}}{\pgfqpoint{1.352087in}{1.810275in}}{\pgfqpoint{1.352087in}{1.802038in}}%
\pgfpathcurveto{\pgfqpoint{1.352087in}{1.793802in}}{\pgfqpoint{1.355359in}{1.785902in}}{\pgfqpoint{1.361183in}{1.780078in}}%
\pgfpathcurveto{\pgfqpoint{1.367007in}{1.774254in}}{\pgfqpoint{1.374907in}{1.770982in}}{\pgfqpoint{1.383144in}{1.770982in}}%
\pgfpathclose%
\pgfusepath{stroke,fill}%
\end{pgfscope}%
\begin{pgfscope}%
\pgfpathrectangle{\pgfqpoint{0.100000in}{0.212622in}}{\pgfqpoint{3.696000in}{3.696000in}}%
\pgfusepath{clip}%
\pgfsetbuttcap%
\pgfsetroundjoin%
\definecolor{currentfill}{rgb}{0.121569,0.466667,0.705882}%
\pgfsetfillcolor{currentfill}%
\pgfsetfillopacity{0.485692}%
\pgfsetlinewidth{1.003750pt}%
\definecolor{currentstroke}{rgb}{0.121569,0.466667,0.705882}%
\pgfsetstrokecolor{currentstroke}%
\pgfsetstrokeopacity{0.485692}%
\pgfsetdash{}{0pt}%
\pgfpathmoveto{\pgfqpoint{2.044323in}{1.885789in}}%
\pgfpathcurveto{\pgfqpoint{2.052560in}{1.885789in}}{\pgfqpoint{2.060460in}{1.889061in}}{\pgfqpoint{2.066284in}{1.894885in}}%
\pgfpathcurveto{\pgfqpoint{2.072108in}{1.900709in}}{\pgfqpoint{2.075380in}{1.908609in}}{\pgfqpoint{2.075380in}{1.916845in}}%
\pgfpathcurveto{\pgfqpoint{2.075380in}{1.925082in}}{\pgfqpoint{2.072108in}{1.932982in}}{\pgfqpoint{2.066284in}{1.938806in}}%
\pgfpathcurveto{\pgfqpoint{2.060460in}{1.944630in}}{\pgfqpoint{2.052560in}{1.947902in}}{\pgfqpoint{2.044323in}{1.947902in}}%
\pgfpathcurveto{\pgfqpoint{2.036087in}{1.947902in}}{\pgfqpoint{2.028187in}{1.944630in}}{\pgfqpoint{2.022363in}{1.938806in}}%
\pgfpathcurveto{\pgfqpoint{2.016539in}{1.932982in}}{\pgfqpoint{2.013267in}{1.925082in}}{\pgfqpoint{2.013267in}{1.916845in}}%
\pgfpathcurveto{\pgfqpoint{2.013267in}{1.908609in}}{\pgfqpoint{2.016539in}{1.900709in}}{\pgfqpoint{2.022363in}{1.894885in}}%
\pgfpathcurveto{\pgfqpoint{2.028187in}{1.889061in}}{\pgfqpoint{2.036087in}{1.885789in}}{\pgfqpoint{2.044323in}{1.885789in}}%
\pgfpathclose%
\pgfusepath{stroke,fill}%
\end{pgfscope}%
\begin{pgfscope}%
\pgfpathrectangle{\pgfqpoint{0.100000in}{0.212622in}}{\pgfqpoint{3.696000in}{3.696000in}}%
\pgfusepath{clip}%
\pgfsetbuttcap%
\pgfsetroundjoin%
\definecolor{currentfill}{rgb}{0.121569,0.466667,0.705882}%
\pgfsetfillcolor{currentfill}%
\pgfsetfillopacity{0.486797}%
\pgfsetlinewidth{1.003750pt}%
\definecolor{currentstroke}{rgb}{0.121569,0.466667,0.705882}%
\pgfsetstrokecolor{currentstroke}%
\pgfsetstrokeopacity{0.486797}%
\pgfsetdash{}{0pt}%
\pgfpathmoveto{\pgfqpoint{1.376489in}{1.767413in}}%
\pgfpathcurveto{\pgfqpoint{1.384725in}{1.767413in}}{\pgfqpoint{1.392625in}{1.770685in}}{\pgfqpoint{1.398449in}{1.776509in}}%
\pgfpathcurveto{\pgfqpoint{1.404273in}{1.782333in}}{\pgfqpoint{1.407545in}{1.790233in}}{\pgfqpoint{1.407545in}{1.798469in}}%
\pgfpathcurveto{\pgfqpoint{1.407545in}{1.806706in}}{\pgfqpoint{1.404273in}{1.814606in}}{\pgfqpoint{1.398449in}{1.820430in}}%
\pgfpathcurveto{\pgfqpoint{1.392625in}{1.826253in}}{\pgfqpoint{1.384725in}{1.829526in}}{\pgfqpoint{1.376489in}{1.829526in}}%
\pgfpathcurveto{\pgfqpoint{1.368252in}{1.829526in}}{\pgfqpoint{1.360352in}{1.826253in}}{\pgfqpoint{1.354528in}{1.820430in}}%
\pgfpathcurveto{\pgfqpoint{1.348705in}{1.814606in}}{\pgfqpoint{1.345432in}{1.806706in}}{\pgfqpoint{1.345432in}{1.798469in}}%
\pgfpathcurveto{\pgfqpoint{1.345432in}{1.790233in}}{\pgfqpoint{1.348705in}{1.782333in}}{\pgfqpoint{1.354528in}{1.776509in}}%
\pgfpathcurveto{\pgfqpoint{1.360352in}{1.770685in}}{\pgfqpoint{1.368252in}{1.767413in}}{\pgfqpoint{1.376489in}{1.767413in}}%
\pgfpathclose%
\pgfusepath{stroke,fill}%
\end{pgfscope}%
\begin{pgfscope}%
\pgfpathrectangle{\pgfqpoint{0.100000in}{0.212622in}}{\pgfqpoint{3.696000in}{3.696000in}}%
\pgfusepath{clip}%
\pgfsetbuttcap%
\pgfsetroundjoin%
\definecolor{currentfill}{rgb}{0.121569,0.466667,0.705882}%
\pgfsetfillcolor{currentfill}%
\pgfsetfillopacity{0.488233}%
\pgfsetlinewidth{1.003750pt}%
\definecolor{currentstroke}{rgb}{0.121569,0.466667,0.705882}%
\pgfsetstrokecolor{currentstroke}%
\pgfsetstrokeopacity{0.488233}%
\pgfsetdash{}{0pt}%
\pgfpathmoveto{\pgfqpoint{1.371037in}{1.763793in}}%
\pgfpathcurveto{\pgfqpoint{1.379274in}{1.763793in}}{\pgfqpoint{1.387174in}{1.767065in}}{\pgfqpoint{1.392998in}{1.772889in}}%
\pgfpathcurveto{\pgfqpoint{1.398822in}{1.778713in}}{\pgfqpoint{1.402094in}{1.786613in}}{\pgfqpoint{1.402094in}{1.794850in}}%
\pgfpathcurveto{\pgfqpoint{1.402094in}{1.803086in}}{\pgfqpoint{1.398822in}{1.810986in}}{\pgfqpoint{1.392998in}{1.816810in}}%
\pgfpathcurveto{\pgfqpoint{1.387174in}{1.822634in}}{\pgfqpoint{1.379274in}{1.825906in}}{\pgfqpoint{1.371037in}{1.825906in}}%
\pgfpathcurveto{\pgfqpoint{1.362801in}{1.825906in}}{\pgfqpoint{1.354901in}{1.822634in}}{\pgfqpoint{1.349077in}{1.816810in}}%
\pgfpathcurveto{\pgfqpoint{1.343253in}{1.810986in}}{\pgfqpoint{1.339981in}{1.803086in}}{\pgfqpoint{1.339981in}{1.794850in}}%
\pgfpathcurveto{\pgfqpoint{1.339981in}{1.786613in}}{\pgfqpoint{1.343253in}{1.778713in}}{\pgfqpoint{1.349077in}{1.772889in}}%
\pgfpathcurveto{\pgfqpoint{1.354901in}{1.767065in}}{\pgfqpoint{1.362801in}{1.763793in}}{\pgfqpoint{1.371037in}{1.763793in}}%
\pgfpathclose%
\pgfusepath{stroke,fill}%
\end{pgfscope}%
\begin{pgfscope}%
\pgfpathrectangle{\pgfqpoint{0.100000in}{0.212622in}}{\pgfqpoint{3.696000in}{3.696000in}}%
\pgfusepath{clip}%
\pgfsetbuttcap%
\pgfsetroundjoin%
\definecolor{currentfill}{rgb}{0.121569,0.466667,0.705882}%
\pgfsetfillcolor{currentfill}%
\pgfsetfillopacity{0.489123}%
\pgfsetlinewidth{1.003750pt}%
\definecolor{currentstroke}{rgb}{0.121569,0.466667,0.705882}%
\pgfsetstrokecolor{currentstroke}%
\pgfsetstrokeopacity{0.489123}%
\pgfsetdash{}{0pt}%
\pgfpathmoveto{\pgfqpoint{2.044834in}{1.886344in}}%
\pgfpathcurveto{\pgfqpoint{2.053070in}{1.886344in}}{\pgfqpoint{2.060970in}{1.889616in}}{\pgfqpoint{2.066794in}{1.895440in}}%
\pgfpathcurveto{\pgfqpoint{2.072618in}{1.901264in}}{\pgfqpoint{2.075890in}{1.909164in}}{\pgfqpoint{2.075890in}{1.917400in}}%
\pgfpathcurveto{\pgfqpoint{2.075890in}{1.925636in}}{\pgfqpoint{2.072618in}{1.933536in}}{\pgfqpoint{2.066794in}{1.939360in}}%
\pgfpathcurveto{\pgfqpoint{2.060970in}{1.945184in}}{\pgfqpoint{2.053070in}{1.948457in}}{\pgfqpoint{2.044834in}{1.948457in}}%
\pgfpathcurveto{\pgfqpoint{2.036598in}{1.948457in}}{\pgfqpoint{2.028697in}{1.945184in}}{\pgfqpoint{2.022874in}{1.939360in}}%
\pgfpathcurveto{\pgfqpoint{2.017050in}{1.933536in}}{\pgfqpoint{2.013777in}{1.925636in}}{\pgfqpoint{2.013777in}{1.917400in}}%
\pgfpathcurveto{\pgfqpoint{2.013777in}{1.909164in}}{\pgfqpoint{2.017050in}{1.901264in}}{\pgfqpoint{2.022874in}{1.895440in}}%
\pgfpathcurveto{\pgfqpoint{2.028697in}{1.889616in}}{\pgfqpoint{2.036598in}{1.886344in}}{\pgfqpoint{2.044834in}{1.886344in}}%
\pgfpathclose%
\pgfusepath{stroke,fill}%
\end{pgfscope}%
\begin{pgfscope}%
\pgfpathrectangle{\pgfqpoint{0.100000in}{0.212622in}}{\pgfqpoint{3.696000in}{3.696000in}}%
\pgfusepath{clip}%
\pgfsetbuttcap%
\pgfsetroundjoin%
\definecolor{currentfill}{rgb}{0.121569,0.466667,0.705882}%
\pgfsetfillcolor{currentfill}%
\pgfsetfillopacity{0.489201}%
\pgfsetlinewidth{1.003750pt}%
\definecolor{currentstroke}{rgb}{0.121569,0.466667,0.705882}%
\pgfsetstrokecolor{currentstroke}%
\pgfsetstrokeopacity{0.489201}%
\pgfsetdash{}{0pt}%
\pgfpathmoveto{\pgfqpoint{1.367126in}{1.760103in}}%
\pgfpathcurveto{\pgfqpoint{1.375362in}{1.760103in}}{\pgfqpoint{1.383263in}{1.763376in}}{\pgfqpoint{1.389086in}{1.769200in}}%
\pgfpathcurveto{\pgfqpoint{1.394910in}{1.775023in}}{\pgfqpoint{1.398183in}{1.782923in}}{\pgfqpoint{1.398183in}{1.791160in}}%
\pgfpathcurveto{\pgfqpoint{1.398183in}{1.799396in}}{\pgfqpoint{1.394910in}{1.807296in}}{\pgfqpoint{1.389086in}{1.813120in}}%
\pgfpathcurveto{\pgfqpoint{1.383263in}{1.818944in}}{\pgfqpoint{1.375362in}{1.822216in}}{\pgfqpoint{1.367126in}{1.822216in}}%
\pgfpathcurveto{\pgfqpoint{1.358890in}{1.822216in}}{\pgfqpoint{1.350990in}{1.818944in}}{\pgfqpoint{1.345166in}{1.813120in}}%
\pgfpathcurveto{\pgfqpoint{1.339342in}{1.807296in}}{\pgfqpoint{1.336070in}{1.799396in}}{\pgfqpoint{1.336070in}{1.791160in}}%
\pgfpathcurveto{\pgfqpoint{1.336070in}{1.782923in}}{\pgfqpoint{1.339342in}{1.775023in}}{\pgfqpoint{1.345166in}{1.769200in}}%
\pgfpathcurveto{\pgfqpoint{1.350990in}{1.763376in}}{\pgfqpoint{1.358890in}{1.760103in}}{\pgfqpoint{1.367126in}{1.760103in}}%
\pgfpathclose%
\pgfusepath{stroke,fill}%
\end{pgfscope}%
\begin{pgfscope}%
\pgfpathrectangle{\pgfqpoint{0.100000in}{0.212622in}}{\pgfqpoint{3.696000in}{3.696000in}}%
\pgfusepath{clip}%
\pgfsetbuttcap%
\pgfsetroundjoin%
\definecolor{currentfill}{rgb}{0.121569,0.466667,0.705882}%
\pgfsetfillcolor{currentfill}%
\pgfsetfillopacity{0.490319}%
\pgfsetlinewidth{1.003750pt}%
\definecolor{currentstroke}{rgb}{0.121569,0.466667,0.705882}%
\pgfsetstrokecolor{currentstroke}%
\pgfsetstrokeopacity{0.490319}%
\pgfsetdash{}{0pt}%
\pgfpathmoveto{\pgfqpoint{1.364436in}{1.758991in}}%
\pgfpathcurveto{\pgfqpoint{1.372673in}{1.758991in}}{\pgfqpoint{1.380573in}{1.762263in}}{\pgfqpoint{1.386397in}{1.768087in}}%
\pgfpathcurveto{\pgfqpoint{1.392221in}{1.773911in}}{\pgfqpoint{1.395493in}{1.781811in}}{\pgfqpoint{1.395493in}{1.790048in}}%
\pgfpathcurveto{\pgfqpoint{1.395493in}{1.798284in}}{\pgfqpoint{1.392221in}{1.806184in}}{\pgfqpoint{1.386397in}{1.812008in}}%
\pgfpathcurveto{\pgfqpoint{1.380573in}{1.817832in}}{\pgfqpoint{1.372673in}{1.821104in}}{\pgfqpoint{1.364436in}{1.821104in}}%
\pgfpathcurveto{\pgfqpoint{1.356200in}{1.821104in}}{\pgfqpoint{1.348300in}{1.817832in}}{\pgfqpoint{1.342476in}{1.812008in}}%
\pgfpathcurveto{\pgfqpoint{1.336652in}{1.806184in}}{\pgfqpoint{1.333380in}{1.798284in}}{\pgfqpoint{1.333380in}{1.790048in}}%
\pgfpathcurveto{\pgfqpoint{1.333380in}{1.781811in}}{\pgfqpoint{1.336652in}{1.773911in}}{\pgfqpoint{1.342476in}{1.768087in}}%
\pgfpathcurveto{\pgfqpoint{1.348300in}{1.762263in}}{\pgfqpoint{1.356200in}{1.758991in}}{\pgfqpoint{1.364436in}{1.758991in}}%
\pgfpathclose%
\pgfusepath{stroke,fill}%
\end{pgfscope}%
\begin{pgfscope}%
\pgfpathrectangle{\pgfqpoint{0.100000in}{0.212622in}}{\pgfqpoint{3.696000in}{3.696000in}}%
\pgfusepath{clip}%
\pgfsetbuttcap%
\pgfsetroundjoin%
\definecolor{currentfill}{rgb}{0.121569,0.466667,0.705882}%
\pgfsetfillcolor{currentfill}%
\pgfsetfillopacity{0.490951}%
\pgfsetlinewidth{1.003750pt}%
\definecolor{currentstroke}{rgb}{0.121569,0.466667,0.705882}%
\pgfsetstrokecolor{currentstroke}%
\pgfsetstrokeopacity{0.490951}%
\pgfsetdash{}{0pt}%
\pgfpathmoveto{\pgfqpoint{1.361612in}{1.756713in}}%
\pgfpathcurveto{\pgfqpoint{1.369848in}{1.756713in}}{\pgfqpoint{1.377748in}{1.759985in}}{\pgfqpoint{1.383572in}{1.765809in}}%
\pgfpathcurveto{\pgfqpoint{1.389396in}{1.771633in}}{\pgfqpoint{1.392668in}{1.779533in}}{\pgfqpoint{1.392668in}{1.787769in}}%
\pgfpathcurveto{\pgfqpoint{1.392668in}{1.796005in}}{\pgfqpoint{1.389396in}{1.803905in}}{\pgfqpoint{1.383572in}{1.809729in}}%
\pgfpathcurveto{\pgfqpoint{1.377748in}{1.815553in}}{\pgfqpoint{1.369848in}{1.818826in}}{\pgfqpoint{1.361612in}{1.818826in}}%
\pgfpathcurveto{\pgfqpoint{1.353376in}{1.818826in}}{\pgfqpoint{1.345476in}{1.815553in}}{\pgfqpoint{1.339652in}{1.809729in}}%
\pgfpathcurveto{\pgfqpoint{1.333828in}{1.803905in}}{\pgfqpoint{1.330555in}{1.796005in}}{\pgfqpoint{1.330555in}{1.787769in}}%
\pgfpathcurveto{\pgfqpoint{1.330555in}{1.779533in}}{\pgfqpoint{1.333828in}{1.771633in}}{\pgfqpoint{1.339652in}{1.765809in}}%
\pgfpathcurveto{\pgfqpoint{1.345476in}{1.759985in}}{\pgfqpoint{1.353376in}{1.756713in}}{\pgfqpoint{1.361612in}{1.756713in}}%
\pgfpathclose%
\pgfusepath{stroke,fill}%
\end{pgfscope}%
\begin{pgfscope}%
\pgfpathrectangle{\pgfqpoint{0.100000in}{0.212622in}}{\pgfqpoint{3.696000in}{3.696000in}}%
\pgfusepath{clip}%
\pgfsetbuttcap%
\pgfsetroundjoin%
\definecolor{currentfill}{rgb}{0.121569,0.466667,0.705882}%
\pgfsetfillcolor{currentfill}%
\pgfsetfillopacity{0.492255}%
\pgfsetlinewidth{1.003750pt}%
\definecolor{currentstroke}{rgb}{0.121569,0.466667,0.705882}%
\pgfsetstrokecolor{currentstroke}%
\pgfsetstrokeopacity{0.492255}%
\pgfsetdash{}{0pt}%
\pgfpathmoveto{\pgfqpoint{1.356464in}{1.753589in}}%
\pgfpathcurveto{\pgfqpoint{1.364701in}{1.753589in}}{\pgfqpoint{1.372601in}{1.756862in}}{\pgfqpoint{1.378425in}{1.762685in}}%
\pgfpathcurveto{\pgfqpoint{1.384249in}{1.768509in}}{\pgfqpoint{1.387521in}{1.776409in}}{\pgfqpoint{1.387521in}{1.784646in}}%
\pgfpathcurveto{\pgfqpoint{1.387521in}{1.792882in}}{\pgfqpoint{1.384249in}{1.800782in}}{\pgfqpoint{1.378425in}{1.806606in}}%
\pgfpathcurveto{\pgfqpoint{1.372601in}{1.812430in}}{\pgfqpoint{1.364701in}{1.815702in}}{\pgfqpoint{1.356464in}{1.815702in}}%
\pgfpathcurveto{\pgfqpoint{1.348228in}{1.815702in}}{\pgfqpoint{1.340328in}{1.812430in}}{\pgfqpoint{1.334504in}{1.806606in}}%
\pgfpathcurveto{\pgfqpoint{1.328680in}{1.800782in}}{\pgfqpoint{1.325408in}{1.792882in}}{\pgfqpoint{1.325408in}{1.784646in}}%
\pgfpathcurveto{\pgfqpoint{1.325408in}{1.776409in}}{\pgfqpoint{1.328680in}{1.768509in}}{\pgfqpoint{1.334504in}{1.762685in}}%
\pgfpathcurveto{\pgfqpoint{1.340328in}{1.756862in}}{\pgfqpoint{1.348228in}{1.753589in}}{\pgfqpoint{1.356464in}{1.753589in}}%
\pgfpathclose%
\pgfusepath{stroke,fill}%
\end{pgfscope}%
\begin{pgfscope}%
\pgfpathrectangle{\pgfqpoint{0.100000in}{0.212622in}}{\pgfqpoint{3.696000in}{3.696000in}}%
\pgfusepath{clip}%
\pgfsetbuttcap%
\pgfsetroundjoin%
\definecolor{currentfill}{rgb}{0.121569,0.466667,0.705882}%
\pgfsetfillcolor{currentfill}%
\pgfsetfillopacity{0.492328}%
\pgfsetlinewidth{1.003750pt}%
\definecolor{currentstroke}{rgb}{0.121569,0.466667,0.705882}%
\pgfsetstrokecolor{currentstroke}%
\pgfsetstrokeopacity{0.492328}%
\pgfsetdash{}{0pt}%
\pgfpathmoveto{\pgfqpoint{2.047003in}{1.881557in}}%
\pgfpathcurveto{\pgfqpoint{2.055239in}{1.881557in}}{\pgfqpoint{2.063139in}{1.884830in}}{\pgfqpoint{2.068963in}{1.890654in}}%
\pgfpathcurveto{\pgfqpoint{2.074787in}{1.896478in}}{\pgfqpoint{2.078059in}{1.904378in}}{\pgfqpoint{2.078059in}{1.912614in}}%
\pgfpathcurveto{\pgfqpoint{2.078059in}{1.920850in}}{\pgfqpoint{2.074787in}{1.928750in}}{\pgfqpoint{2.068963in}{1.934574in}}%
\pgfpathcurveto{\pgfqpoint{2.063139in}{1.940398in}}{\pgfqpoint{2.055239in}{1.943670in}}{\pgfqpoint{2.047003in}{1.943670in}}%
\pgfpathcurveto{\pgfqpoint{2.038766in}{1.943670in}}{\pgfqpoint{2.030866in}{1.940398in}}{\pgfqpoint{2.025042in}{1.934574in}}%
\pgfpathcurveto{\pgfqpoint{2.019218in}{1.928750in}}{\pgfqpoint{2.015946in}{1.920850in}}{\pgfqpoint{2.015946in}{1.912614in}}%
\pgfpathcurveto{\pgfqpoint{2.015946in}{1.904378in}}{\pgfqpoint{2.019218in}{1.896478in}}{\pgfqpoint{2.025042in}{1.890654in}}%
\pgfpathcurveto{\pgfqpoint{2.030866in}{1.884830in}}{\pgfqpoint{2.038766in}{1.881557in}}{\pgfqpoint{2.047003in}{1.881557in}}%
\pgfpathclose%
\pgfusepath{stroke,fill}%
\end{pgfscope}%
\begin{pgfscope}%
\pgfpathrectangle{\pgfqpoint{0.100000in}{0.212622in}}{\pgfqpoint{3.696000in}{3.696000in}}%
\pgfusepath{clip}%
\pgfsetbuttcap%
\pgfsetroundjoin%
\definecolor{currentfill}{rgb}{0.121569,0.466667,0.705882}%
\pgfsetfillcolor{currentfill}%
\pgfsetfillopacity{0.495464}%
\pgfsetlinewidth{1.003750pt}%
\definecolor{currentstroke}{rgb}{0.121569,0.466667,0.705882}%
\pgfsetstrokecolor{currentstroke}%
\pgfsetstrokeopacity{0.495464}%
\pgfsetdash{}{0pt}%
\pgfpathmoveto{\pgfqpoint{1.348715in}{1.750909in}}%
\pgfpathcurveto{\pgfqpoint{1.356951in}{1.750909in}}{\pgfqpoint{1.364851in}{1.754182in}}{\pgfqpoint{1.370675in}{1.760006in}}%
\pgfpathcurveto{\pgfqpoint{1.376499in}{1.765830in}}{\pgfqpoint{1.379771in}{1.773730in}}{\pgfqpoint{1.379771in}{1.781966in}}%
\pgfpathcurveto{\pgfqpoint{1.379771in}{1.790202in}}{\pgfqpoint{1.376499in}{1.798102in}}{\pgfqpoint{1.370675in}{1.803926in}}%
\pgfpathcurveto{\pgfqpoint{1.364851in}{1.809750in}}{\pgfqpoint{1.356951in}{1.813022in}}{\pgfqpoint{1.348715in}{1.813022in}}%
\pgfpathcurveto{\pgfqpoint{1.340478in}{1.813022in}}{\pgfqpoint{1.332578in}{1.809750in}}{\pgfqpoint{1.326754in}{1.803926in}}%
\pgfpathcurveto{\pgfqpoint{1.320930in}{1.798102in}}{\pgfqpoint{1.317658in}{1.790202in}}{\pgfqpoint{1.317658in}{1.781966in}}%
\pgfpathcurveto{\pgfqpoint{1.317658in}{1.773730in}}{\pgfqpoint{1.320930in}{1.765830in}}{\pgfqpoint{1.326754in}{1.760006in}}%
\pgfpathcurveto{\pgfqpoint{1.332578in}{1.754182in}}{\pgfqpoint{1.340478in}{1.750909in}}{\pgfqpoint{1.348715in}{1.750909in}}%
\pgfpathclose%
\pgfusepath{stroke,fill}%
\end{pgfscope}%
\begin{pgfscope}%
\pgfpathrectangle{\pgfqpoint{0.100000in}{0.212622in}}{\pgfqpoint{3.696000in}{3.696000in}}%
\pgfusepath{clip}%
\pgfsetbuttcap%
\pgfsetroundjoin%
\definecolor{currentfill}{rgb}{0.121569,0.466667,0.705882}%
\pgfsetfillcolor{currentfill}%
\pgfsetfillopacity{0.496141}%
\pgfsetlinewidth{1.003750pt}%
\definecolor{currentstroke}{rgb}{0.121569,0.466667,0.705882}%
\pgfsetstrokecolor{currentstroke}%
\pgfsetstrokeopacity{0.496141}%
\pgfsetdash{}{0pt}%
\pgfpathmoveto{\pgfqpoint{2.049123in}{1.877582in}}%
\pgfpathcurveto{\pgfqpoint{2.057360in}{1.877582in}}{\pgfqpoint{2.065260in}{1.880855in}}{\pgfqpoint{2.071084in}{1.886679in}}%
\pgfpathcurveto{\pgfqpoint{2.076908in}{1.892502in}}{\pgfqpoint{2.080180in}{1.900403in}}{\pgfqpoint{2.080180in}{1.908639in}}%
\pgfpathcurveto{\pgfqpoint{2.080180in}{1.916875in}}{\pgfqpoint{2.076908in}{1.924775in}}{\pgfqpoint{2.071084in}{1.930599in}}%
\pgfpathcurveto{\pgfqpoint{2.065260in}{1.936423in}}{\pgfqpoint{2.057360in}{1.939695in}}{\pgfqpoint{2.049123in}{1.939695in}}%
\pgfpathcurveto{\pgfqpoint{2.040887in}{1.939695in}}{\pgfqpoint{2.032987in}{1.936423in}}{\pgfqpoint{2.027163in}{1.930599in}}%
\pgfpathcurveto{\pgfqpoint{2.021339in}{1.924775in}}{\pgfqpoint{2.018067in}{1.916875in}}{\pgfqpoint{2.018067in}{1.908639in}}%
\pgfpathcurveto{\pgfqpoint{2.018067in}{1.900403in}}{\pgfqpoint{2.021339in}{1.892502in}}{\pgfqpoint{2.027163in}{1.886679in}}%
\pgfpathcurveto{\pgfqpoint{2.032987in}{1.880855in}}{\pgfqpoint{2.040887in}{1.877582in}}{\pgfqpoint{2.049123in}{1.877582in}}%
\pgfpathclose%
\pgfusepath{stroke,fill}%
\end{pgfscope}%
\begin{pgfscope}%
\pgfpathrectangle{\pgfqpoint{0.100000in}{0.212622in}}{\pgfqpoint{3.696000in}{3.696000in}}%
\pgfusepath{clip}%
\pgfsetbuttcap%
\pgfsetroundjoin%
\definecolor{currentfill}{rgb}{0.121569,0.466667,0.705882}%
\pgfsetfillcolor{currentfill}%
\pgfsetfillopacity{0.497070}%
\pgfsetlinewidth{1.003750pt}%
\definecolor{currentstroke}{rgb}{0.121569,0.466667,0.705882}%
\pgfsetstrokecolor{currentstroke}%
\pgfsetstrokeopacity{0.497070}%
\pgfsetdash{}{0pt}%
\pgfpathmoveto{\pgfqpoint{1.342213in}{1.748710in}}%
\pgfpathcurveto{\pgfqpoint{1.350450in}{1.748710in}}{\pgfqpoint{1.358350in}{1.751983in}}{\pgfqpoint{1.364174in}{1.757807in}}%
\pgfpathcurveto{\pgfqpoint{1.369997in}{1.763631in}}{\pgfqpoint{1.373270in}{1.771531in}}{\pgfqpoint{1.373270in}{1.779767in}}%
\pgfpathcurveto{\pgfqpoint{1.373270in}{1.788003in}}{\pgfqpoint{1.369997in}{1.795903in}}{\pgfqpoint{1.364174in}{1.801727in}}%
\pgfpathcurveto{\pgfqpoint{1.358350in}{1.807551in}}{\pgfqpoint{1.350450in}{1.810823in}}{\pgfqpoint{1.342213in}{1.810823in}}%
\pgfpathcurveto{\pgfqpoint{1.333977in}{1.810823in}}{\pgfqpoint{1.326077in}{1.807551in}}{\pgfqpoint{1.320253in}{1.801727in}}%
\pgfpathcurveto{\pgfqpoint{1.314429in}{1.795903in}}{\pgfqpoint{1.311157in}{1.788003in}}{\pgfqpoint{1.311157in}{1.779767in}}%
\pgfpathcurveto{\pgfqpoint{1.311157in}{1.771531in}}{\pgfqpoint{1.314429in}{1.763631in}}{\pgfqpoint{1.320253in}{1.757807in}}%
\pgfpathcurveto{\pgfqpoint{1.326077in}{1.751983in}}{\pgfqpoint{1.333977in}{1.748710in}}{\pgfqpoint{1.342213in}{1.748710in}}%
\pgfpathclose%
\pgfusepath{stroke,fill}%
\end{pgfscope}%
\begin{pgfscope}%
\pgfpathrectangle{\pgfqpoint{0.100000in}{0.212622in}}{\pgfqpoint{3.696000in}{3.696000in}}%
\pgfusepath{clip}%
\pgfsetbuttcap%
\pgfsetroundjoin%
\definecolor{currentfill}{rgb}{0.121569,0.466667,0.705882}%
\pgfsetfillcolor{currentfill}%
\pgfsetfillopacity{0.498111}%
\pgfsetlinewidth{1.003750pt}%
\definecolor{currentstroke}{rgb}{0.121569,0.466667,0.705882}%
\pgfsetstrokecolor{currentstroke}%
\pgfsetstrokeopacity{0.498111}%
\pgfsetdash{}{0pt}%
\pgfpathmoveto{\pgfqpoint{1.337973in}{1.744497in}}%
\pgfpathcurveto{\pgfqpoint{1.346210in}{1.744497in}}{\pgfqpoint{1.354110in}{1.747769in}}{\pgfqpoint{1.359934in}{1.753593in}}%
\pgfpathcurveto{\pgfqpoint{1.365758in}{1.759417in}}{\pgfqpoint{1.369030in}{1.767317in}}{\pgfqpoint{1.369030in}{1.775553in}}%
\pgfpathcurveto{\pgfqpoint{1.369030in}{1.783790in}}{\pgfqpoint{1.365758in}{1.791690in}}{\pgfqpoint{1.359934in}{1.797514in}}%
\pgfpathcurveto{\pgfqpoint{1.354110in}{1.803338in}}{\pgfqpoint{1.346210in}{1.806610in}}{\pgfqpoint{1.337973in}{1.806610in}}%
\pgfpathcurveto{\pgfqpoint{1.329737in}{1.806610in}}{\pgfqpoint{1.321837in}{1.803338in}}{\pgfqpoint{1.316013in}{1.797514in}}%
\pgfpathcurveto{\pgfqpoint{1.310189in}{1.791690in}}{\pgfqpoint{1.306917in}{1.783790in}}{\pgfqpoint{1.306917in}{1.775553in}}%
\pgfpathcurveto{\pgfqpoint{1.306917in}{1.767317in}}{\pgfqpoint{1.310189in}{1.759417in}}{\pgfqpoint{1.316013in}{1.753593in}}%
\pgfpathcurveto{\pgfqpoint{1.321837in}{1.747769in}}{\pgfqpoint{1.329737in}{1.744497in}}{\pgfqpoint{1.337973in}{1.744497in}}%
\pgfpathclose%
\pgfusepath{stroke,fill}%
\end{pgfscope}%
\begin{pgfscope}%
\pgfpathrectangle{\pgfqpoint{0.100000in}{0.212622in}}{\pgfqpoint{3.696000in}{3.696000in}}%
\pgfusepath{clip}%
\pgfsetbuttcap%
\pgfsetroundjoin%
\definecolor{currentfill}{rgb}{0.121569,0.466667,0.705882}%
\pgfsetfillcolor{currentfill}%
\pgfsetfillopacity{0.500391}%
\pgfsetlinewidth{1.003750pt}%
\definecolor{currentstroke}{rgb}{0.121569,0.466667,0.705882}%
\pgfsetstrokecolor{currentstroke}%
\pgfsetstrokeopacity{0.500391}%
\pgfsetdash{}{0pt}%
\pgfpathmoveto{\pgfqpoint{2.049857in}{1.872343in}}%
\pgfpathcurveto{\pgfqpoint{2.058094in}{1.872343in}}{\pgfqpoint{2.065994in}{1.875615in}}{\pgfqpoint{2.071818in}{1.881439in}}%
\pgfpathcurveto{\pgfqpoint{2.077642in}{1.887263in}}{\pgfqpoint{2.080914in}{1.895163in}}{\pgfqpoint{2.080914in}{1.903399in}}%
\pgfpathcurveto{\pgfqpoint{2.080914in}{1.911636in}}{\pgfqpoint{2.077642in}{1.919536in}}{\pgfqpoint{2.071818in}{1.925360in}}%
\pgfpathcurveto{\pgfqpoint{2.065994in}{1.931184in}}{\pgfqpoint{2.058094in}{1.934456in}}{\pgfqpoint{2.049857in}{1.934456in}}%
\pgfpathcurveto{\pgfqpoint{2.041621in}{1.934456in}}{\pgfqpoint{2.033721in}{1.931184in}}{\pgfqpoint{2.027897in}{1.925360in}}%
\pgfpathcurveto{\pgfqpoint{2.022073in}{1.919536in}}{\pgfqpoint{2.018801in}{1.911636in}}{\pgfqpoint{2.018801in}{1.903399in}}%
\pgfpathcurveto{\pgfqpoint{2.018801in}{1.895163in}}{\pgfqpoint{2.022073in}{1.887263in}}{\pgfqpoint{2.027897in}{1.881439in}}%
\pgfpathcurveto{\pgfqpoint{2.033721in}{1.875615in}}{\pgfqpoint{2.041621in}{1.872343in}}{\pgfqpoint{2.049857in}{1.872343in}}%
\pgfpathclose%
\pgfusepath{stroke,fill}%
\end{pgfscope}%
\begin{pgfscope}%
\pgfpathrectangle{\pgfqpoint{0.100000in}{0.212622in}}{\pgfqpoint{3.696000in}{3.696000in}}%
\pgfusepath{clip}%
\pgfsetbuttcap%
\pgfsetroundjoin%
\definecolor{currentfill}{rgb}{0.121569,0.466667,0.705882}%
\pgfsetfillcolor{currentfill}%
\pgfsetfillopacity{0.501188}%
\pgfsetlinewidth{1.003750pt}%
\definecolor{currentstroke}{rgb}{0.121569,0.466667,0.705882}%
\pgfsetstrokecolor{currentstroke}%
\pgfsetstrokeopacity{0.501188}%
\pgfsetdash{}{0pt}%
\pgfpathmoveto{\pgfqpoint{1.331351in}{1.743078in}}%
\pgfpathcurveto{\pgfqpoint{1.339587in}{1.743078in}}{\pgfqpoint{1.347487in}{1.746350in}}{\pgfqpoint{1.353311in}{1.752174in}}%
\pgfpathcurveto{\pgfqpoint{1.359135in}{1.757998in}}{\pgfqpoint{1.362407in}{1.765898in}}{\pgfqpoint{1.362407in}{1.774134in}}%
\pgfpathcurveto{\pgfqpoint{1.362407in}{1.782370in}}{\pgfqpoint{1.359135in}{1.790270in}}{\pgfqpoint{1.353311in}{1.796094in}}%
\pgfpathcurveto{\pgfqpoint{1.347487in}{1.801918in}}{\pgfqpoint{1.339587in}{1.805191in}}{\pgfqpoint{1.331351in}{1.805191in}}%
\pgfpathcurveto{\pgfqpoint{1.323115in}{1.805191in}}{\pgfqpoint{1.315214in}{1.801918in}}{\pgfqpoint{1.309391in}{1.796094in}}%
\pgfpathcurveto{\pgfqpoint{1.303567in}{1.790270in}}{\pgfqpoint{1.300294in}{1.782370in}}{\pgfqpoint{1.300294in}{1.774134in}}%
\pgfpathcurveto{\pgfqpoint{1.300294in}{1.765898in}}{\pgfqpoint{1.303567in}{1.757998in}}{\pgfqpoint{1.309391in}{1.752174in}}%
\pgfpathcurveto{\pgfqpoint{1.315214in}{1.746350in}}{\pgfqpoint{1.323115in}{1.743078in}}{\pgfqpoint{1.331351in}{1.743078in}}%
\pgfpathclose%
\pgfusepath{stroke,fill}%
\end{pgfscope}%
\begin{pgfscope}%
\pgfpathrectangle{\pgfqpoint{0.100000in}{0.212622in}}{\pgfqpoint{3.696000in}{3.696000in}}%
\pgfusepath{clip}%
\pgfsetbuttcap%
\pgfsetroundjoin%
\definecolor{currentfill}{rgb}{0.121569,0.466667,0.705882}%
\pgfsetfillcolor{currentfill}%
\pgfsetfillopacity{0.502594}%
\pgfsetlinewidth{1.003750pt}%
\definecolor{currentstroke}{rgb}{0.121569,0.466667,0.705882}%
\pgfsetstrokecolor{currentstroke}%
\pgfsetstrokeopacity{0.502594}%
\pgfsetdash{}{0pt}%
\pgfpathmoveto{\pgfqpoint{1.325218in}{1.739888in}}%
\pgfpathcurveto{\pgfqpoint{1.333455in}{1.739888in}}{\pgfqpoint{1.341355in}{1.743160in}}{\pgfqpoint{1.347179in}{1.748984in}}%
\pgfpathcurveto{\pgfqpoint{1.353003in}{1.754808in}}{\pgfqpoint{1.356275in}{1.762708in}}{\pgfqpoint{1.356275in}{1.770945in}}%
\pgfpathcurveto{\pgfqpoint{1.356275in}{1.779181in}}{\pgfqpoint{1.353003in}{1.787081in}}{\pgfqpoint{1.347179in}{1.792905in}}%
\pgfpathcurveto{\pgfqpoint{1.341355in}{1.798729in}}{\pgfqpoint{1.333455in}{1.802001in}}{\pgfqpoint{1.325218in}{1.802001in}}%
\pgfpathcurveto{\pgfqpoint{1.316982in}{1.802001in}}{\pgfqpoint{1.309082in}{1.798729in}}{\pgfqpoint{1.303258in}{1.792905in}}%
\pgfpathcurveto{\pgfqpoint{1.297434in}{1.787081in}}{\pgfqpoint{1.294162in}{1.779181in}}{\pgfqpoint{1.294162in}{1.770945in}}%
\pgfpathcurveto{\pgfqpoint{1.294162in}{1.762708in}}{\pgfqpoint{1.297434in}{1.754808in}}{\pgfqpoint{1.303258in}{1.748984in}}%
\pgfpathcurveto{\pgfqpoint{1.309082in}{1.743160in}}{\pgfqpoint{1.316982in}{1.739888in}}{\pgfqpoint{1.325218in}{1.739888in}}%
\pgfpathclose%
\pgfusepath{stroke,fill}%
\end{pgfscope}%
\begin{pgfscope}%
\pgfpathrectangle{\pgfqpoint{0.100000in}{0.212622in}}{\pgfqpoint{3.696000in}{3.696000in}}%
\pgfusepath{clip}%
\pgfsetbuttcap%
\pgfsetroundjoin%
\definecolor{currentfill}{rgb}{0.121569,0.466667,0.705882}%
\pgfsetfillcolor{currentfill}%
\pgfsetfillopacity{0.503388}%
\pgfsetlinewidth{1.003750pt}%
\definecolor{currentstroke}{rgb}{0.121569,0.466667,0.705882}%
\pgfsetstrokecolor{currentstroke}%
\pgfsetstrokeopacity{0.503388}%
\pgfsetdash{}{0pt}%
\pgfpathmoveto{\pgfqpoint{1.322089in}{1.737260in}}%
\pgfpathcurveto{\pgfqpoint{1.330326in}{1.737260in}}{\pgfqpoint{1.338226in}{1.740533in}}{\pgfqpoint{1.344050in}{1.746356in}}%
\pgfpathcurveto{\pgfqpoint{1.349874in}{1.752180in}}{\pgfqpoint{1.353146in}{1.760080in}}{\pgfqpoint{1.353146in}{1.768317in}}%
\pgfpathcurveto{\pgfqpoint{1.353146in}{1.776553in}}{\pgfqpoint{1.349874in}{1.784453in}}{\pgfqpoint{1.344050in}{1.790277in}}%
\pgfpathcurveto{\pgfqpoint{1.338226in}{1.796101in}}{\pgfqpoint{1.330326in}{1.799373in}}{\pgfqpoint{1.322089in}{1.799373in}}%
\pgfpathcurveto{\pgfqpoint{1.313853in}{1.799373in}}{\pgfqpoint{1.305953in}{1.796101in}}{\pgfqpoint{1.300129in}{1.790277in}}%
\pgfpathcurveto{\pgfqpoint{1.294305in}{1.784453in}}{\pgfqpoint{1.291033in}{1.776553in}}{\pgfqpoint{1.291033in}{1.768317in}}%
\pgfpathcurveto{\pgfqpoint{1.291033in}{1.760080in}}{\pgfqpoint{1.294305in}{1.752180in}}{\pgfqpoint{1.300129in}{1.746356in}}%
\pgfpathcurveto{\pgfqpoint{1.305953in}{1.740533in}}{\pgfqpoint{1.313853in}{1.737260in}}{\pgfqpoint{1.322089in}{1.737260in}}%
\pgfpathclose%
\pgfusepath{stroke,fill}%
\end{pgfscope}%
\begin{pgfscope}%
\pgfpathrectangle{\pgfqpoint{0.100000in}{0.212622in}}{\pgfqpoint{3.696000in}{3.696000in}}%
\pgfusepath{clip}%
\pgfsetbuttcap%
\pgfsetroundjoin%
\definecolor{currentfill}{rgb}{0.121569,0.466667,0.705882}%
\pgfsetfillcolor{currentfill}%
\pgfsetfillopacity{0.505373}%
\pgfsetlinewidth{1.003750pt}%
\definecolor{currentstroke}{rgb}{0.121569,0.466667,0.705882}%
\pgfsetstrokecolor{currentstroke}%
\pgfsetstrokeopacity{0.505373}%
\pgfsetdash{}{0pt}%
\pgfpathmoveto{\pgfqpoint{2.052617in}{1.866217in}}%
\pgfpathcurveto{\pgfqpoint{2.060853in}{1.866217in}}{\pgfqpoint{2.068753in}{1.869490in}}{\pgfqpoint{2.074577in}{1.875314in}}%
\pgfpathcurveto{\pgfqpoint{2.080401in}{1.881138in}}{\pgfqpoint{2.083673in}{1.889038in}}{\pgfqpoint{2.083673in}{1.897274in}}%
\pgfpathcurveto{\pgfqpoint{2.083673in}{1.905510in}}{\pgfqpoint{2.080401in}{1.913410in}}{\pgfqpoint{2.074577in}{1.919234in}}%
\pgfpathcurveto{\pgfqpoint{2.068753in}{1.925058in}}{\pgfqpoint{2.060853in}{1.928330in}}{\pgfqpoint{2.052617in}{1.928330in}}%
\pgfpathcurveto{\pgfqpoint{2.044380in}{1.928330in}}{\pgfqpoint{2.036480in}{1.925058in}}{\pgfqpoint{2.030656in}{1.919234in}}%
\pgfpathcurveto{\pgfqpoint{2.024832in}{1.913410in}}{\pgfqpoint{2.021560in}{1.905510in}}{\pgfqpoint{2.021560in}{1.897274in}}%
\pgfpathcurveto{\pgfqpoint{2.021560in}{1.889038in}}{\pgfqpoint{2.024832in}{1.881138in}}{\pgfqpoint{2.030656in}{1.875314in}}%
\pgfpathcurveto{\pgfqpoint{2.036480in}{1.869490in}}{\pgfqpoint{2.044380in}{1.866217in}}{\pgfqpoint{2.052617in}{1.866217in}}%
\pgfpathclose%
\pgfusepath{stroke,fill}%
\end{pgfscope}%
\begin{pgfscope}%
\pgfpathrectangle{\pgfqpoint{0.100000in}{0.212622in}}{\pgfqpoint{3.696000in}{3.696000in}}%
\pgfusepath{clip}%
\pgfsetbuttcap%
\pgfsetroundjoin%
\definecolor{currentfill}{rgb}{0.121569,0.466667,0.705882}%
\pgfsetfillcolor{currentfill}%
\pgfsetfillopacity{0.505394}%
\pgfsetlinewidth{1.003750pt}%
\definecolor{currentstroke}{rgb}{0.121569,0.466667,0.705882}%
\pgfsetstrokecolor{currentstroke}%
\pgfsetstrokeopacity{0.505394}%
\pgfsetdash{}{0pt}%
\pgfpathmoveto{\pgfqpoint{1.316653in}{1.735720in}}%
\pgfpathcurveto{\pgfqpoint{1.324889in}{1.735720in}}{\pgfqpoint{1.332789in}{1.738992in}}{\pgfqpoint{1.338613in}{1.744816in}}%
\pgfpathcurveto{\pgfqpoint{1.344437in}{1.750640in}}{\pgfqpoint{1.347709in}{1.758540in}}{\pgfqpoint{1.347709in}{1.766777in}}%
\pgfpathcurveto{\pgfqpoint{1.347709in}{1.775013in}}{\pgfqpoint{1.344437in}{1.782913in}}{\pgfqpoint{1.338613in}{1.788737in}}%
\pgfpathcurveto{\pgfqpoint{1.332789in}{1.794561in}}{\pgfqpoint{1.324889in}{1.797833in}}{\pgfqpoint{1.316653in}{1.797833in}}%
\pgfpathcurveto{\pgfqpoint{1.308417in}{1.797833in}}{\pgfqpoint{1.300517in}{1.794561in}}{\pgfqpoint{1.294693in}{1.788737in}}%
\pgfpathcurveto{\pgfqpoint{1.288869in}{1.782913in}}{\pgfqpoint{1.285596in}{1.775013in}}{\pgfqpoint{1.285596in}{1.766777in}}%
\pgfpathcurveto{\pgfqpoint{1.285596in}{1.758540in}}{\pgfqpoint{1.288869in}{1.750640in}}{\pgfqpoint{1.294693in}{1.744816in}}%
\pgfpathcurveto{\pgfqpoint{1.300517in}{1.738992in}}{\pgfqpoint{1.308417in}{1.735720in}}{\pgfqpoint{1.316653in}{1.735720in}}%
\pgfpathclose%
\pgfusepath{stroke,fill}%
\end{pgfscope}%
\begin{pgfscope}%
\pgfpathrectangle{\pgfqpoint{0.100000in}{0.212622in}}{\pgfqpoint{3.696000in}{3.696000in}}%
\pgfusepath{clip}%
\pgfsetbuttcap%
\pgfsetroundjoin%
\definecolor{currentfill}{rgb}{0.121569,0.466667,0.705882}%
\pgfsetfillcolor{currentfill}%
\pgfsetfillopacity{0.506388}%
\pgfsetlinewidth{1.003750pt}%
\definecolor{currentstroke}{rgb}{0.121569,0.466667,0.705882}%
\pgfsetstrokecolor{currentstroke}%
\pgfsetstrokeopacity{0.506388}%
\pgfsetdash{}{0pt}%
\pgfpathmoveto{\pgfqpoint{1.312772in}{1.732927in}}%
\pgfpathcurveto{\pgfqpoint{1.321009in}{1.732927in}}{\pgfqpoint{1.328909in}{1.736199in}}{\pgfqpoint{1.334733in}{1.742023in}}%
\pgfpathcurveto{\pgfqpoint{1.340557in}{1.747847in}}{\pgfqpoint{1.343829in}{1.755747in}}{\pgfqpoint{1.343829in}{1.763983in}}%
\pgfpathcurveto{\pgfqpoint{1.343829in}{1.772219in}}{\pgfqpoint{1.340557in}{1.780119in}}{\pgfqpoint{1.334733in}{1.785943in}}%
\pgfpathcurveto{\pgfqpoint{1.328909in}{1.791767in}}{\pgfqpoint{1.321009in}{1.795040in}}{\pgfqpoint{1.312772in}{1.795040in}}%
\pgfpathcurveto{\pgfqpoint{1.304536in}{1.795040in}}{\pgfqpoint{1.296636in}{1.791767in}}{\pgfqpoint{1.290812in}{1.785943in}}%
\pgfpathcurveto{\pgfqpoint{1.284988in}{1.780119in}}{\pgfqpoint{1.281716in}{1.772219in}}{\pgfqpoint{1.281716in}{1.763983in}}%
\pgfpathcurveto{\pgfqpoint{1.281716in}{1.755747in}}{\pgfqpoint{1.284988in}{1.747847in}}{\pgfqpoint{1.290812in}{1.742023in}}%
\pgfpathcurveto{\pgfqpoint{1.296636in}{1.736199in}}{\pgfqpoint{1.304536in}{1.732927in}}{\pgfqpoint{1.312772in}{1.732927in}}%
\pgfpathclose%
\pgfusepath{stroke,fill}%
\end{pgfscope}%
\begin{pgfscope}%
\pgfpathrectangle{\pgfqpoint{0.100000in}{0.212622in}}{\pgfqpoint{3.696000in}{3.696000in}}%
\pgfusepath{clip}%
\pgfsetbuttcap%
\pgfsetroundjoin%
\definecolor{currentfill}{rgb}{0.121569,0.466667,0.705882}%
\pgfsetfillcolor{currentfill}%
\pgfsetfillopacity{0.508360}%
\pgfsetlinewidth{1.003750pt}%
\definecolor{currentstroke}{rgb}{0.121569,0.466667,0.705882}%
\pgfsetstrokecolor{currentstroke}%
\pgfsetstrokeopacity{0.508360}%
\pgfsetdash{}{0pt}%
\pgfpathmoveto{\pgfqpoint{2.053904in}{1.864355in}}%
\pgfpathcurveto{\pgfqpoint{2.062140in}{1.864355in}}{\pgfqpoint{2.070040in}{1.867627in}}{\pgfqpoint{2.075864in}{1.873451in}}%
\pgfpathcurveto{\pgfqpoint{2.081688in}{1.879275in}}{\pgfqpoint{2.084960in}{1.887175in}}{\pgfqpoint{2.084960in}{1.895412in}}%
\pgfpathcurveto{\pgfqpoint{2.084960in}{1.903648in}}{\pgfqpoint{2.081688in}{1.911548in}}{\pgfqpoint{2.075864in}{1.917372in}}%
\pgfpathcurveto{\pgfqpoint{2.070040in}{1.923196in}}{\pgfqpoint{2.062140in}{1.926468in}}{\pgfqpoint{2.053904in}{1.926468in}}%
\pgfpathcurveto{\pgfqpoint{2.045667in}{1.926468in}}{\pgfqpoint{2.037767in}{1.923196in}}{\pgfqpoint{2.031943in}{1.917372in}}%
\pgfpathcurveto{\pgfqpoint{2.026119in}{1.911548in}}{\pgfqpoint{2.022847in}{1.903648in}}{\pgfqpoint{2.022847in}{1.895412in}}%
\pgfpathcurveto{\pgfqpoint{2.022847in}{1.887175in}}{\pgfqpoint{2.026119in}{1.879275in}}{\pgfqpoint{2.031943in}{1.873451in}}%
\pgfpathcurveto{\pgfqpoint{2.037767in}{1.867627in}}{\pgfqpoint{2.045667in}{1.864355in}}{\pgfqpoint{2.053904in}{1.864355in}}%
\pgfpathclose%
\pgfusepath{stroke,fill}%
\end{pgfscope}%
\begin{pgfscope}%
\pgfpathrectangle{\pgfqpoint{0.100000in}{0.212622in}}{\pgfqpoint{3.696000in}{3.696000in}}%
\pgfusepath{clip}%
\pgfsetbuttcap%
\pgfsetroundjoin%
\definecolor{currentfill}{rgb}{0.121569,0.466667,0.705882}%
\pgfsetfillcolor{currentfill}%
\pgfsetfillopacity{0.508749}%
\pgfsetlinewidth{1.003750pt}%
\definecolor{currentstroke}{rgb}{0.121569,0.466667,0.705882}%
\pgfsetstrokecolor{currentstroke}%
\pgfsetstrokeopacity{0.508749}%
\pgfsetdash{}{0pt}%
\pgfpathmoveto{\pgfqpoint{1.306094in}{1.730805in}}%
\pgfpathcurveto{\pgfqpoint{1.314330in}{1.730805in}}{\pgfqpoint{1.322230in}{1.734078in}}{\pgfqpoint{1.328054in}{1.739902in}}%
\pgfpathcurveto{\pgfqpoint{1.333878in}{1.745725in}}{\pgfqpoint{1.337150in}{1.753625in}}{\pgfqpoint{1.337150in}{1.761862in}}%
\pgfpathcurveto{\pgfqpoint{1.337150in}{1.770098in}}{\pgfqpoint{1.333878in}{1.777998in}}{\pgfqpoint{1.328054in}{1.783822in}}%
\pgfpathcurveto{\pgfqpoint{1.322230in}{1.789646in}}{\pgfqpoint{1.314330in}{1.792918in}}{\pgfqpoint{1.306094in}{1.792918in}}%
\pgfpathcurveto{\pgfqpoint{1.297858in}{1.792918in}}{\pgfqpoint{1.289958in}{1.789646in}}{\pgfqpoint{1.284134in}{1.783822in}}%
\pgfpathcurveto{\pgfqpoint{1.278310in}{1.777998in}}{\pgfqpoint{1.275037in}{1.770098in}}{\pgfqpoint{1.275037in}{1.761862in}}%
\pgfpathcurveto{\pgfqpoint{1.275037in}{1.753625in}}{\pgfqpoint{1.278310in}{1.745725in}}{\pgfqpoint{1.284134in}{1.739902in}}%
\pgfpathcurveto{\pgfqpoint{1.289958in}{1.734078in}}{\pgfqpoint{1.297858in}{1.730805in}}{\pgfqpoint{1.306094in}{1.730805in}}%
\pgfpathclose%
\pgfusepath{stroke,fill}%
\end{pgfscope}%
\begin{pgfscope}%
\pgfpathrectangle{\pgfqpoint{0.100000in}{0.212622in}}{\pgfqpoint{3.696000in}{3.696000in}}%
\pgfusepath{clip}%
\pgfsetbuttcap%
\pgfsetroundjoin%
\definecolor{currentfill}{rgb}{0.121569,0.466667,0.705882}%
\pgfsetfillcolor{currentfill}%
\pgfsetfillopacity{0.512072}%
\pgfsetlinewidth{1.003750pt}%
\definecolor{currentstroke}{rgb}{0.121569,0.466667,0.705882}%
\pgfsetstrokecolor{currentstroke}%
\pgfsetstrokeopacity{0.512072}%
\pgfsetdash{}{0pt}%
\pgfpathmoveto{\pgfqpoint{2.056221in}{1.863247in}}%
\pgfpathcurveto{\pgfqpoint{2.064457in}{1.863247in}}{\pgfqpoint{2.072357in}{1.866519in}}{\pgfqpoint{2.078181in}{1.872343in}}%
\pgfpathcurveto{\pgfqpoint{2.084005in}{1.878167in}}{\pgfqpoint{2.087278in}{1.886067in}}{\pgfqpoint{2.087278in}{1.894303in}}%
\pgfpathcurveto{\pgfqpoint{2.087278in}{1.902540in}}{\pgfqpoint{2.084005in}{1.910440in}}{\pgfqpoint{2.078181in}{1.916264in}}%
\pgfpathcurveto{\pgfqpoint{2.072357in}{1.922088in}}{\pgfqpoint{2.064457in}{1.925360in}}{\pgfqpoint{2.056221in}{1.925360in}}%
\pgfpathcurveto{\pgfqpoint{2.047985in}{1.925360in}}{\pgfqpoint{2.040085in}{1.922088in}}{\pgfqpoint{2.034261in}{1.916264in}}%
\pgfpathcurveto{\pgfqpoint{2.028437in}{1.910440in}}{\pgfqpoint{2.025165in}{1.902540in}}{\pgfqpoint{2.025165in}{1.894303in}}%
\pgfpathcurveto{\pgfqpoint{2.025165in}{1.886067in}}{\pgfqpoint{2.028437in}{1.878167in}}{\pgfqpoint{2.034261in}{1.872343in}}%
\pgfpathcurveto{\pgfqpoint{2.040085in}{1.866519in}}{\pgfqpoint{2.047985in}{1.863247in}}{\pgfqpoint{2.056221in}{1.863247in}}%
\pgfpathclose%
\pgfusepath{stroke,fill}%
\end{pgfscope}%
\begin{pgfscope}%
\pgfpathrectangle{\pgfqpoint{0.100000in}{0.212622in}}{\pgfqpoint{3.696000in}{3.696000in}}%
\pgfusepath{clip}%
\pgfsetbuttcap%
\pgfsetroundjoin%
\definecolor{currentfill}{rgb}{0.121569,0.466667,0.705882}%
\pgfsetfillcolor{currentfill}%
\pgfsetfillopacity{0.513023}%
\pgfsetlinewidth{1.003750pt}%
\definecolor{currentstroke}{rgb}{0.121569,0.466667,0.705882}%
\pgfsetstrokecolor{currentstroke}%
\pgfsetstrokeopacity{0.513023}%
\pgfsetdash{}{0pt}%
\pgfpathmoveto{\pgfqpoint{1.295085in}{1.725231in}}%
\pgfpathcurveto{\pgfqpoint{1.303321in}{1.725231in}}{\pgfqpoint{1.311222in}{1.728503in}}{\pgfqpoint{1.317045in}{1.734327in}}%
\pgfpathcurveto{\pgfqpoint{1.322869in}{1.740151in}}{\pgfqpoint{1.326142in}{1.748051in}}{\pgfqpoint{1.326142in}{1.756287in}}%
\pgfpathcurveto{\pgfqpoint{1.326142in}{1.764523in}}{\pgfqpoint{1.322869in}{1.772424in}}{\pgfqpoint{1.317045in}{1.778247in}}%
\pgfpathcurveto{\pgfqpoint{1.311222in}{1.784071in}}{\pgfqpoint{1.303321in}{1.787344in}}{\pgfqpoint{1.295085in}{1.787344in}}%
\pgfpathcurveto{\pgfqpoint{1.286849in}{1.787344in}}{\pgfqpoint{1.278949in}{1.784071in}}{\pgfqpoint{1.273125in}{1.778247in}}%
\pgfpathcurveto{\pgfqpoint{1.267301in}{1.772424in}}{\pgfqpoint{1.264029in}{1.764523in}}{\pgfqpoint{1.264029in}{1.756287in}}%
\pgfpathcurveto{\pgfqpoint{1.264029in}{1.748051in}}{\pgfqpoint{1.267301in}{1.740151in}}{\pgfqpoint{1.273125in}{1.734327in}}%
\pgfpathcurveto{\pgfqpoint{1.278949in}{1.728503in}}{\pgfqpoint{1.286849in}{1.725231in}}{\pgfqpoint{1.295085in}{1.725231in}}%
\pgfpathclose%
\pgfusepath{stroke,fill}%
\end{pgfscope}%
\begin{pgfscope}%
\pgfpathrectangle{\pgfqpoint{0.100000in}{0.212622in}}{\pgfqpoint{3.696000in}{3.696000in}}%
\pgfusepath{clip}%
\pgfsetbuttcap%
\pgfsetroundjoin%
\definecolor{currentfill}{rgb}{0.121569,0.466667,0.705882}%
\pgfsetfillcolor{currentfill}%
\pgfsetfillopacity{0.515477}%
\pgfsetlinewidth{1.003750pt}%
\definecolor{currentstroke}{rgb}{0.121569,0.466667,0.705882}%
\pgfsetstrokecolor{currentstroke}%
\pgfsetstrokeopacity{0.515477}%
\pgfsetdash{}{0pt}%
\pgfpathmoveto{\pgfqpoint{1.283576in}{1.717382in}}%
\pgfpathcurveto{\pgfqpoint{1.291812in}{1.717382in}}{\pgfqpoint{1.299712in}{1.720654in}}{\pgfqpoint{1.305536in}{1.726478in}}%
\pgfpathcurveto{\pgfqpoint{1.311360in}{1.732302in}}{\pgfqpoint{1.314632in}{1.740202in}}{\pgfqpoint{1.314632in}{1.748438in}}%
\pgfpathcurveto{\pgfqpoint{1.314632in}{1.756674in}}{\pgfqpoint{1.311360in}{1.764574in}}{\pgfqpoint{1.305536in}{1.770398in}}%
\pgfpathcurveto{\pgfqpoint{1.299712in}{1.776222in}}{\pgfqpoint{1.291812in}{1.779495in}}{\pgfqpoint{1.283576in}{1.779495in}}%
\pgfpathcurveto{\pgfqpoint{1.275340in}{1.779495in}}{\pgfqpoint{1.267440in}{1.776222in}}{\pgfqpoint{1.261616in}{1.770398in}}%
\pgfpathcurveto{\pgfqpoint{1.255792in}{1.764574in}}{\pgfqpoint{1.252519in}{1.756674in}}{\pgfqpoint{1.252519in}{1.748438in}}%
\pgfpathcurveto{\pgfqpoint{1.252519in}{1.740202in}}{\pgfqpoint{1.255792in}{1.732302in}}{\pgfqpoint{1.261616in}{1.726478in}}%
\pgfpathcurveto{\pgfqpoint{1.267440in}{1.720654in}}{\pgfqpoint{1.275340in}{1.717382in}}{\pgfqpoint{1.283576in}{1.717382in}}%
\pgfpathclose%
\pgfusepath{stroke,fill}%
\end{pgfscope}%
\begin{pgfscope}%
\pgfpathrectangle{\pgfqpoint{0.100000in}{0.212622in}}{\pgfqpoint{3.696000in}{3.696000in}}%
\pgfusepath{clip}%
\pgfsetbuttcap%
\pgfsetroundjoin%
\definecolor{currentfill}{rgb}{0.121569,0.466667,0.705882}%
\pgfsetfillcolor{currentfill}%
\pgfsetfillopacity{0.515967}%
\pgfsetlinewidth{1.003750pt}%
\definecolor{currentstroke}{rgb}{0.121569,0.466667,0.705882}%
\pgfsetstrokecolor{currentstroke}%
\pgfsetstrokeopacity{0.515967}%
\pgfsetdash{}{0pt}%
\pgfpathmoveto{\pgfqpoint{2.058053in}{1.860821in}}%
\pgfpathcurveto{\pgfqpoint{2.066289in}{1.860821in}}{\pgfqpoint{2.074189in}{1.864093in}}{\pgfqpoint{2.080013in}{1.869917in}}%
\pgfpathcurveto{\pgfqpoint{2.085837in}{1.875741in}}{\pgfqpoint{2.089109in}{1.883641in}}{\pgfqpoint{2.089109in}{1.891877in}}%
\pgfpathcurveto{\pgfqpoint{2.089109in}{1.900113in}}{\pgfqpoint{2.085837in}{1.908013in}}{\pgfqpoint{2.080013in}{1.913837in}}%
\pgfpathcurveto{\pgfqpoint{2.074189in}{1.919661in}}{\pgfqpoint{2.066289in}{1.922934in}}{\pgfqpoint{2.058053in}{1.922934in}}%
\pgfpathcurveto{\pgfqpoint{2.049817in}{1.922934in}}{\pgfqpoint{2.041917in}{1.919661in}}{\pgfqpoint{2.036093in}{1.913837in}}%
\pgfpathcurveto{\pgfqpoint{2.030269in}{1.908013in}}{\pgfqpoint{2.026996in}{1.900113in}}{\pgfqpoint{2.026996in}{1.891877in}}%
\pgfpathcurveto{\pgfqpoint{2.026996in}{1.883641in}}{\pgfqpoint{2.030269in}{1.875741in}}{\pgfqpoint{2.036093in}{1.869917in}}%
\pgfpathcurveto{\pgfqpoint{2.041917in}{1.864093in}}{\pgfqpoint{2.049817in}{1.860821in}}{\pgfqpoint{2.058053in}{1.860821in}}%
\pgfpathclose%
\pgfusepath{stroke,fill}%
\end{pgfscope}%
\begin{pgfscope}%
\pgfpathrectangle{\pgfqpoint{0.100000in}{0.212622in}}{\pgfqpoint{3.696000in}{3.696000in}}%
\pgfusepath{clip}%
\pgfsetbuttcap%
\pgfsetroundjoin%
\definecolor{currentfill}{rgb}{0.121569,0.466667,0.705882}%
\pgfsetfillcolor{currentfill}%
\pgfsetfillopacity{0.517959}%
\pgfsetlinewidth{1.003750pt}%
\definecolor{currentstroke}{rgb}{0.121569,0.466667,0.705882}%
\pgfsetstrokecolor{currentstroke}%
\pgfsetstrokeopacity{0.517959}%
\pgfsetdash{}{0pt}%
\pgfpathmoveto{\pgfqpoint{1.274407in}{1.713445in}}%
\pgfpathcurveto{\pgfqpoint{1.282644in}{1.713445in}}{\pgfqpoint{1.290544in}{1.716718in}}{\pgfqpoint{1.296367in}{1.722542in}}%
\pgfpathcurveto{\pgfqpoint{1.302191in}{1.728365in}}{\pgfqpoint{1.305464in}{1.736265in}}{\pgfqpoint{1.305464in}{1.744502in}}%
\pgfpathcurveto{\pgfqpoint{1.305464in}{1.752738in}}{\pgfqpoint{1.302191in}{1.760638in}}{\pgfqpoint{1.296367in}{1.766462in}}%
\pgfpathcurveto{\pgfqpoint{1.290544in}{1.772286in}}{\pgfqpoint{1.282644in}{1.775558in}}{\pgfqpoint{1.274407in}{1.775558in}}%
\pgfpathcurveto{\pgfqpoint{1.266171in}{1.775558in}}{\pgfqpoint{1.258271in}{1.772286in}}{\pgfqpoint{1.252447in}{1.766462in}}%
\pgfpathcurveto{\pgfqpoint{1.246623in}{1.760638in}}{\pgfqpoint{1.243351in}{1.752738in}}{\pgfqpoint{1.243351in}{1.744502in}}%
\pgfpathcurveto{\pgfqpoint{1.243351in}{1.736265in}}{\pgfqpoint{1.246623in}{1.728365in}}{\pgfqpoint{1.252447in}{1.722542in}}%
\pgfpathcurveto{\pgfqpoint{1.258271in}{1.716718in}}{\pgfqpoint{1.266171in}{1.713445in}}{\pgfqpoint{1.274407in}{1.713445in}}%
\pgfpathclose%
\pgfusepath{stroke,fill}%
\end{pgfscope}%
\begin{pgfscope}%
\pgfpathrectangle{\pgfqpoint{0.100000in}{0.212622in}}{\pgfqpoint{3.696000in}{3.696000in}}%
\pgfusepath{clip}%
\pgfsetbuttcap%
\pgfsetroundjoin%
\definecolor{currentfill}{rgb}{0.121569,0.466667,0.705882}%
\pgfsetfillcolor{currentfill}%
\pgfsetfillopacity{0.520606}%
\pgfsetlinewidth{1.003750pt}%
\definecolor{currentstroke}{rgb}{0.121569,0.466667,0.705882}%
\pgfsetstrokecolor{currentstroke}%
\pgfsetstrokeopacity{0.520606}%
\pgfsetdash{}{0pt}%
\pgfpathmoveto{\pgfqpoint{2.059874in}{1.859291in}}%
\pgfpathcurveto{\pgfqpoint{2.068110in}{1.859291in}}{\pgfqpoint{2.076010in}{1.862563in}}{\pgfqpoint{2.081834in}{1.868387in}}%
\pgfpathcurveto{\pgfqpoint{2.087658in}{1.874211in}}{\pgfqpoint{2.090930in}{1.882111in}}{\pgfqpoint{2.090930in}{1.890347in}}%
\pgfpathcurveto{\pgfqpoint{2.090930in}{1.898583in}}{\pgfqpoint{2.087658in}{1.906483in}}{\pgfqpoint{2.081834in}{1.912307in}}%
\pgfpathcurveto{\pgfqpoint{2.076010in}{1.918131in}}{\pgfqpoint{2.068110in}{1.921404in}}{\pgfqpoint{2.059874in}{1.921404in}}%
\pgfpathcurveto{\pgfqpoint{2.051638in}{1.921404in}}{\pgfqpoint{2.043737in}{1.918131in}}{\pgfqpoint{2.037914in}{1.912307in}}%
\pgfpathcurveto{\pgfqpoint{2.032090in}{1.906483in}}{\pgfqpoint{2.028817in}{1.898583in}}{\pgfqpoint{2.028817in}{1.890347in}}%
\pgfpathcurveto{\pgfqpoint{2.028817in}{1.882111in}}{\pgfqpoint{2.032090in}{1.874211in}}{\pgfqpoint{2.037914in}{1.868387in}}%
\pgfpathcurveto{\pgfqpoint{2.043737in}{1.862563in}}{\pgfqpoint{2.051638in}{1.859291in}}{\pgfqpoint{2.059874in}{1.859291in}}%
\pgfpathclose%
\pgfusepath{stroke,fill}%
\end{pgfscope}%
\begin{pgfscope}%
\pgfpathrectangle{\pgfqpoint{0.100000in}{0.212622in}}{\pgfqpoint{3.696000in}{3.696000in}}%
\pgfusepath{clip}%
\pgfsetbuttcap%
\pgfsetroundjoin%
\definecolor{currentfill}{rgb}{0.121569,0.466667,0.705882}%
\pgfsetfillcolor{currentfill}%
\pgfsetfillopacity{0.523704}%
\pgfsetlinewidth{1.003750pt}%
\definecolor{currentstroke}{rgb}{0.121569,0.466667,0.705882}%
\pgfsetstrokecolor{currentstroke}%
\pgfsetstrokeopacity{0.523704}%
\pgfsetdash{}{0pt}%
\pgfpathmoveto{\pgfqpoint{1.261178in}{1.708994in}}%
\pgfpathcurveto{\pgfqpoint{1.269414in}{1.708994in}}{\pgfqpoint{1.277314in}{1.712266in}}{\pgfqpoint{1.283138in}{1.718090in}}%
\pgfpathcurveto{\pgfqpoint{1.288962in}{1.723914in}}{\pgfqpoint{1.292235in}{1.731814in}}{\pgfqpoint{1.292235in}{1.740050in}}%
\pgfpathcurveto{\pgfqpoint{1.292235in}{1.748286in}}{\pgfqpoint{1.288962in}{1.756186in}}{\pgfqpoint{1.283138in}{1.762010in}}%
\pgfpathcurveto{\pgfqpoint{1.277314in}{1.767834in}}{\pgfqpoint{1.269414in}{1.771107in}}{\pgfqpoint{1.261178in}{1.771107in}}%
\pgfpathcurveto{\pgfqpoint{1.252942in}{1.771107in}}{\pgfqpoint{1.245042in}{1.767834in}}{\pgfqpoint{1.239218in}{1.762010in}}%
\pgfpathcurveto{\pgfqpoint{1.233394in}{1.756186in}}{\pgfqpoint{1.230122in}{1.748286in}}{\pgfqpoint{1.230122in}{1.740050in}}%
\pgfpathcurveto{\pgfqpoint{1.230122in}{1.731814in}}{\pgfqpoint{1.233394in}{1.723914in}}{\pgfqpoint{1.239218in}{1.718090in}}%
\pgfpathcurveto{\pgfqpoint{1.245042in}{1.712266in}}{\pgfqpoint{1.252942in}{1.708994in}}{\pgfqpoint{1.261178in}{1.708994in}}%
\pgfpathclose%
\pgfusepath{stroke,fill}%
\end{pgfscope}%
\begin{pgfscope}%
\pgfpathrectangle{\pgfqpoint{0.100000in}{0.212622in}}{\pgfqpoint{3.696000in}{3.696000in}}%
\pgfusepath{clip}%
\pgfsetbuttcap%
\pgfsetroundjoin%
\definecolor{currentfill}{rgb}{0.121569,0.466667,0.705882}%
\pgfsetfillcolor{currentfill}%
\pgfsetfillopacity{0.525670}%
\pgfsetlinewidth{1.003750pt}%
\definecolor{currentstroke}{rgb}{0.121569,0.466667,0.705882}%
\pgfsetstrokecolor{currentstroke}%
\pgfsetstrokeopacity{0.525670}%
\pgfsetdash{}{0pt}%
\pgfpathmoveto{\pgfqpoint{2.064514in}{1.859436in}}%
\pgfpathcurveto{\pgfqpoint{2.072750in}{1.859436in}}{\pgfqpoint{2.080650in}{1.862709in}}{\pgfqpoint{2.086474in}{1.868532in}}%
\pgfpathcurveto{\pgfqpoint{2.092298in}{1.874356in}}{\pgfqpoint{2.095570in}{1.882256in}}{\pgfqpoint{2.095570in}{1.890493in}}%
\pgfpathcurveto{\pgfqpoint{2.095570in}{1.898729in}}{\pgfqpoint{2.092298in}{1.906629in}}{\pgfqpoint{2.086474in}{1.912453in}}%
\pgfpathcurveto{\pgfqpoint{2.080650in}{1.918277in}}{\pgfqpoint{2.072750in}{1.921549in}}{\pgfqpoint{2.064514in}{1.921549in}}%
\pgfpathcurveto{\pgfqpoint{2.056278in}{1.921549in}}{\pgfqpoint{2.048377in}{1.918277in}}{\pgfqpoint{2.042554in}{1.912453in}}%
\pgfpathcurveto{\pgfqpoint{2.036730in}{1.906629in}}{\pgfqpoint{2.033457in}{1.898729in}}{\pgfqpoint{2.033457in}{1.890493in}}%
\pgfpathcurveto{\pgfqpoint{2.033457in}{1.882256in}}{\pgfqpoint{2.036730in}{1.874356in}}{\pgfqpoint{2.042554in}{1.868532in}}%
\pgfpathcurveto{\pgfqpoint{2.048377in}{1.862709in}}{\pgfqpoint{2.056278in}{1.859436in}}{\pgfqpoint{2.064514in}{1.859436in}}%
\pgfpathclose%
\pgfusepath{stroke,fill}%
\end{pgfscope}%
\begin{pgfscope}%
\pgfpathrectangle{\pgfqpoint{0.100000in}{0.212622in}}{\pgfqpoint{3.696000in}{3.696000in}}%
\pgfusepath{clip}%
\pgfsetbuttcap%
\pgfsetroundjoin%
\definecolor{currentfill}{rgb}{0.121569,0.466667,0.705882}%
\pgfsetfillcolor{currentfill}%
\pgfsetfillopacity{0.526959}%
\pgfsetlinewidth{1.003750pt}%
\definecolor{currentstroke}{rgb}{0.121569,0.466667,0.705882}%
\pgfsetstrokecolor{currentstroke}%
\pgfsetstrokeopacity{0.526959}%
\pgfsetdash{}{0pt}%
\pgfpathmoveto{\pgfqpoint{1.247559in}{1.702313in}}%
\pgfpathcurveto{\pgfqpoint{1.255795in}{1.702313in}}{\pgfqpoint{1.263695in}{1.705586in}}{\pgfqpoint{1.269519in}{1.711409in}}%
\pgfpathcurveto{\pgfqpoint{1.275343in}{1.717233in}}{\pgfqpoint{1.278615in}{1.725133in}}{\pgfqpoint{1.278615in}{1.733370in}}%
\pgfpathcurveto{\pgfqpoint{1.278615in}{1.741606in}}{\pgfqpoint{1.275343in}{1.749506in}}{\pgfqpoint{1.269519in}{1.755330in}}%
\pgfpathcurveto{\pgfqpoint{1.263695in}{1.761154in}}{\pgfqpoint{1.255795in}{1.764426in}}{\pgfqpoint{1.247559in}{1.764426in}}%
\pgfpathcurveto{\pgfqpoint{1.239322in}{1.764426in}}{\pgfqpoint{1.231422in}{1.761154in}}{\pgfqpoint{1.225598in}{1.755330in}}%
\pgfpathcurveto{\pgfqpoint{1.219774in}{1.749506in}}{\pgfqpoint{1.216502in}{1.741606in}}{\pgfqpoint{1.216502in}{1.733370in}}%
\pgfpathcurveto{\pgfqpoint{1.216502in}{1.725133in}}{\pgfqpoint{1.219774in}{1.717233in}}{\pgfqpoint{1.225598in}{1.711409in}}%
\pgfpathcurveto{\pgfqpoint{1.231422in}{1.705586in}}{\pgfqpoint{1.239322in}{1.702313in}}{\pgfqpoint{1.247559in}{1.702313in}}%
\pgfpathclose%
\pgfusepath{stroke,fill}%
\end{pgfscope}%
\begin{pgfscope}%
\pgfpathrectangle{\pgfqpoint{0.100000in}{0.212622in}}{\pgfqpoint{3.696000in}{3.696000in}}%
\pgfusepath{clip}%
\pgfsetbuttcap%
\pgfsetroundjoin%
\definecolor{currentfill}{rgb}{0.121569,0.466667,0.705882}%
\pgfsetfillcolor{currentfill}%
\pgfsetfillopacity{0.530050}%
\pgfsetlinewidth{1.003750pt}%
\definecolor{currentstroke}{rgb}{0.121569,0.466667,0.705882}%
\pgfsetstrokecolor{currentstroke}%
\pgfsetstrokeopacity{0.530050}%
\pgfsetdash{}{0pt}%
\pgfpathmoveto{\pgfqpoint{1.237089in}{1.696342in}}%
\pgfpathcurveto{\pgfqpoint{1.245325in}{1.696342in}}{\pgfqpoint{1.253225in}{1.699615in}}{\pgfqpoint{1.259049in}{1.705438in}}%
\pgfpathcurveto{\pgfqpoint{1.264873in}{1.711262in}}{\pgfqpoint{1.268146in}{1.719162in}}{\pgfqpoint{1.268146in}{1.727399in}}%
\pgfpathcurveto{\pgfqpoint{1.268146in}{1.735635in}}{\pgfqpoint{1.264873in}{1.743535in}}{\pgfqpoint{1.259049in}{1.749359in}}%
\pgfpathcurveto{\pgfqpoint{1.253225in}{1.755183in}}{\pgfqpoint{1.245325in}{1.758455in}}{\pgfqpoint{1.237089in}{1.758455in}}%
\pgfpathcurveto{\pgfqpoint{1.228853in}{1.758455in}}{\pgfqpoint{1.220953in}{1.755183in}}{\pgfqpoint{1.215129in}{1.749359in}}%
\pgfpathcurveto{\pgfqpoint{1.209305in}{1.743535in}}{\pgfqpoint{1.206033in}{1.735635in}}{\pgfqpoint{1.206033in}{1.727399in}}%
\pgfpathcurveto{\pgfqpoint{1.206033in}{1.719162in}}{\pgfqpoint{1.209305in}{1.711262in}}{\pgfqpoint{1.215129in}{1.705438in}}%
\pgfpathcurveto{\pgfqpoint{1.220953in}{1.699615in}}{\pgfqpoint{1.228853in}{1.696342in}}{\pgfqpoint{1.237089in}{1.696342in}}%
\pgfpathclose%
\pgfusepath{stroke,fill}%
\end{pgfscope}%
\begin{pgfscope}%
\pgfpathrectangle{\pgfqpoint{0.100000in}{0.212622in}}{\pgfqpoint{3.696000in}{3.696000in}}%
\pgfusepath{clip}%
\pgfsetbuttcap%
\pgfsetroundjoin%
\definecolor{currentfill}{rgb}{0.121569,0.466667,0.705882}%
\pgfsetfillcolor{currentfill}%
\pgfsetfillopacity{0.530598}%
\pgfsetlinewidth{1.003750pt}%
\definecolor{currentstroke}{rgb}{0.121569,0.466667,0.705882}%
\pgfsetstrokecolor{currentstroke}%
\pgfsetstrokeopacity{0.530598}%
\pgfsetdash{}{0pt}%
\pgfpathmoveto{\pgfqpoint{2.065242in}{1.853941in}}%
\pgfpathcurveto{\pgfqpoint{2.073478in}{1.853941in}}{\pgfqpoint{2.081378in}{1.857214in}}{\pgfqpoint{2.087202in}{1.863038in}}%
\pgfpathcurveto{\pgfqpoint{2.093026in}{1.868862in}}{\pgfqpoint{2.096298in}{1.876762in}}{\pgfqpoint{2.096298in}{1.884998in}}%
\pgfpathcurveto{\pgfqpoint{2.096298in}{1.893234in}}{\pgfqpoint{2.093026in}{1.901134in}}{\pgfqpoint{2.087202in}{1.906958in}}%
\pgfpathcurveto{\pgfqpoint{2.081378in}{1.912782in}}{\pgfqpoint{2.073478in}{1.916054in}}{\pgfqpoint{2.065242in}{1.916054in}}%
\pgfpathcurveto{\pgfqpoint{2.057005in}{1.916054in}}{\pgfqpoint{2.049105in}{1.912782in}}{\pgfqpoint{2.043281in}{1.906958in}}%
\pgfpathcurveto{\pgfqpoint{2.037458in}{1.901134in}}{\pgfqpoint{2.034185in}{1.893234in}}{\pgfqpoint{2.034185in}{1.884998in}}%
\pgfpathcurveto{\pgfqpoint{2.034185in}{1.876762in}}{\pgfqpoint{2.037458in}{1.868862in}}{\pgfqpoint{2.043281in}{1.863038in}}%
\pgfpathcurveto{\pgfqpoint{2.049105in}{1.857214in}}{\pgfqpoint{2.057005in}{1.853941in}}{\pgfqpoint{2.065242in}{1.853941in}}%
\pgfpathclose%
\pgfusepath{stroke,fill}%
\end{pgfscope}%
\begin{pgfscope}%
\pgfpathrectangle{\pgfqpoint{0.100000in}{0.212622in}}{\pgfqpoint{3.696000in}{3.696000in}}%
\pgfusepath{clip}%
\pgfsetbuttcap%
\pgfsetroundjoin%
\definecolor{currentfill}{rgb}{0.121569,0.466667,0.705882}%
\pgfsetfillcolor{currentfill}%
\pgfsetfillopacity{0.533797}%
\pgfsetlinewidth{1.003750pt}%
\definecolor{currentstroke}{rgb}{0.121569,0.466667,0.705882}%
\pgfsetstrokecolor{currentstroke}%
\pgfsetstrokeopacity{0.533797}%
\pgfsetdash{}{0pt}%
\pgfpathmoveto{\pgfqpoint{1.228427in}{1.693377in}}%
\pgfpathcurveto{\pgfqpoint{1.236663in}{1.693377in}}{\pgfqpoint{1.244563in}{1.696650in}}{\pgfqpoint{1.250387in}{1.702474in}}%
\pgfpathcurveto{\pgfqpoint{1.256211in}{1.708297in}}{\pgfqpoint{1.259483in}{1.716197in}}{\pgfqpoint{1.259483in}{1.724434in}}%
\pgfpathcurveto{\pgfqpoint{1.259483in}{1.732670in}}{\pgfqpoint{1.256211in}{1.740570in}}{\pgfqpoint{1.250387in}{1.746394in}}%
\pgfpathcurveto{\pgfqpoint{1.244563in}{1.752218in}}{\pgfqpoint{1.236663in}{1.755490in}}{\pgfqpoint{1.228427in}{1.755490in}}%
\pgfpathcurveto{\pgfqpoint{1.220191in}{1.755490in}}{\pgfqpoint{1.212290in}{1.752218in}}{\pgfqpoint{1.206467in}{1.746394in}}%
\pgfpathcurveto{\pgfqpoint{1.200643in}{1.740570in}}{\pgfqpoint{1.197370in}{1.732670in}}{\pgfqpoint{1.197370in}{1.724434in}}%
\pgfpathcurveto{\pgfqpoint{1.197370in}{1.716197in}}{\pgfqpoint{1.200643in}{1.708297in}}{\pgfqpoint{1.206467in}{1.702474in}}%
\pgfpathcurveto{\pgfqpoint{1.212290in}{1.696650in}}{\pgfqpoint{1.220191in}{1.693377in}}{\pgfqpoint{1.228427in}{1.693377in}}%
\pgfpathclose%
\pgfusepath{stroke,fill}%
\end{pgfscope}%
\begin{pgfscope}%
\pgfpathrectangle{\pgfqpoint{0.100000in}{0.212622in}}{\pgfqpoint{3.696000in}{3.696000in}}%
\pgfusepath{clip}%
\pgfsetbuttcap%
\pgfsetroundjoin%
\definecolor{currentfill}{rgb}{0.121569,0.466667,0.705882}%
\pgfsetfillcolor{currentfill}%
\pgfsetfillopacity{0.535401}%
\pgfsetlinewidth{1.003750pt}%
\definecolor{currentstroke}{rgb}{0.121569,0.466667,0.705882}%
\pgfsetstrokecolor{currentstroke}%
\pgfsetstrokeopacity{0.535401}%
\pgfsetdash{}{0pt}%
\pgfpathmoveto{\pgfqpoint{1.220841in}{1.687125in}}%
\pgfpathcurveto{\pgfqpoint{1.229078in}{1.687125in}}{\pgfqpoint{1.236978in}{1.690398in}}{\pgfqpoint{1.242802in}{1.696222in}}%
\pgfpathcurveto{\pgfqpoint{1.248625in}{1.702046in}}{\pgfqpoint{1.251898in}{1.709946in}}{\pgfqpoint{1.251898in}{1.718182in}}%
\pgfpathcurveto{\pgfqpoint{1.251898in}{1.726418in}}{\pgfqpoint{1.248625in}{1.734318in}}{\pgfqpoint{1.242802in}{1.740142in}}%
\pgfpathcurveto{\pgfqpoint{1.236978in}{1.745966in}}{\pgfqpoint{1.229078in}{1.749238in}}{\pgfqpoint{1.220841in}{1.749238in}}%
\pgfpathcurveto{\pgfqpoint{1.212605in}{1.749238in}}{\pgfqpoint{1.204705in}{1.745966in}}{\pgfqpoint{1.198881in}{1.740142in}}%
\pgfpathcurveto{\pgfqpoint{1.193057in}{1.734318in}}{\pgfqpoint{1.189785in}{1.726418in}}{\pgfqpoint{1.189785in}{1.718182in}}%
\pgfpathcurveto{\pgfqpoint{1.189785in}{1.709946in}}{\pgfqpoint{1.193057in}{1.702046in}}{\pgfqpoint{1.198881in}{1.696222in}}%
\pgfpathcurveto{\pgfqpoint{1.204705in}{1.690398in}}{\pgfqpoint{1.212605in}{1.687125in}}{\pgfqpoint{1.220841in}{1.687125in}}%
\pgfpathclose%
\pgfusepath{stroke,fill}%
\end{pgfscope}%
\begin{pgfscope}%
\pgfpathrectangle{\pgfqpoint{0.100000in}{0.212622in}}{\pgfqpoint{3.696000in}{3.696000in}}%
\pgfusepath{clip}%
\pgfsetbuttcap%
\pgfsetroundjoin%
\definecolor{currentfill}{rgb}{0.121569,0.466667,0.705882}%
\pgfsetfillcolor{currentfill}%
\pgfsetfillopacity{0.535713}%
\pgfsetlinewidth{1.003750pt}%
\definecolor{currentstroke}{rgb}{0.121569,0.466667,0.705882}%
\pgfsetstrokecolor{currentstroke}%
\pgfsetstrokeopacity{0.535713}%
\pgfsetdash{}{0pt}%
\pgfpathmoveto{\pgfqpoint{2.067658in}{1.846604in}}%
\pgfpathcurveto{\pgfqpoint{2.075894in}{1.846604in}}{\pgfqpoint{2.083794in}{1.849876in}}{\pgfqpoint{2.089618in}{1.855700in}}%
\pgfpathcurveto{\pgfqpoint{2.095442in}{1.861524in}}{\pgfqpoint{2.098714in}{1.869424in}}{\pgfqpoint{2.098714in}{1.877660in}}%
\pgfpathcurveto{\pgfqpoint{2.098714in}{1.885896in}}{\pgfqpoint{2.095442in}{1.893796in}}{\pgfqpoint{2.089618in}{1.899620in}}%
\pgfpathcurveto{\pgfqpoint{2.083794in}{1.905444in}}{\pgfqpoint{2.075894in}{1.908717in}}{\pgfqpoint{2.067658in}{1.908717in}}%
\pgfpathcurveto{\pgfqpoint{2.059421in}{1.908717in}}{\pgfqpoint{2.051521in}{1.905444in}}{\pgfqpoint{2.045697in}{1.899620in}}%
\pgfpathcurveto{\pgfqpoint{2.039873in}{1.893796in}}{\pgfqpoint{2.036601in}{1.885896in}}{\pgfqpoint{2.036601in}{1.877660in}}%
\pgfpathcurveto{\pgfqpoint{2.036601in}{1.869424in}}{\pgfqpoint{2.039873in}{1.861524in}}{\pgfqpoint{2.045697in}{1.855700in}}%
\pgfpathcurveto{\pgfqpoint{2.051521in}{1.849876in}}{\pgfqpoint{2.059421in}{1.846604in}}{\pgfqpoint{2.067658in}{1.846604in}}%
\pgfpathclose%
\pgfusepath{stroke,fill}%
\end{pgfscope}%
\begin{pgfscope}%
\pgfpathrectangle{\pgfqpoint{0.100000in}{0.212622in}}{\pgfqpoint{3.696000in}{3.696000in}}%
\pgfusepath{clip}%
\pgfsetbuttcap%
\pgfsetroundjoin%
\definecolor{currentfill}{rgb}{0.121569,0.466667,0.705882}%
\pgfsetfillcolor{currentfill}%
\pgfsetfillopacity{0.536617}%
\pgfsetlinewidth{1.003750pt}%
\definecolor{currentstroke}{rgb}{0.121569,0.466667,0.705882}%
\pgfsetstrokecolor{currentstroke}%
\pgfsetstrokeopacity{0.536617}%
\pgfsetdash{}{0pt}%
\pgfpathmoveto{\pgfqpoint{1.216164in}{1.683474in}}%
\pgfpathcurveto{\pgfqpoint{1.224401in}{1.683474in}}{\pgfqpoint{1.232301in}{1.686747in}}{\pgfqpoint{1.238125in}{1.692570in}}%
\pgfpathcurveto{\pgfqpoint{1.243948in}{1.698394in}}{\pgfqpoint{1.247221in}{1.706294in}}{\pgfqpoint{1.247221in}{1.714531in}}%
\pgfpathcurveto{\pgfqpoint{1.247221in}{1.722767in}}{\pgfqpoint{1.243948in}{1.730667in}}{\pgfqpoint{1.238125in}{1.736491in}}%
\pgfpathcurveto{\pgfqpoint{1.232301in}{1.742315in}}{\pgfqpoint{1.224401in}{1.745587in}}{\pgfqpoint{1.216164in}{1.745587in}}%
\pgfpathcurveto{\pgfqpoint{1.207928in}{1.745587in}}{\pgfqpoint{1.200028in}{1.742315in}}{\pgfqpoint{1.194204in}{1.736491in}}%
\pgfpathcurveto{\pgfqpoint{1.188380in}{1.730667in}}{\pgfqpoint{1.185108in}{1.722767in}}{\pgfqpoint{1.185108in}{1.714531in}}%
\pgfpathcurveto{\pgfqpoint{1.185108in}{1.706294in}}{\pgfqpoint{1.188380in}{1.698394in}}{\pgfqpoint{1.194204in}{1.692570in}}%
\pgfpathcurveto{\pgfqpoint{1.200028in}{1.686747in}}{\pgfqpoint{1.207928in}{1.683474in}}{\pgfqpoint{1.216164in}{1.683474in}}%
\pgfpathclose%
\pgfusepath{stroke,fill}%
\end{pgfscope}%
\begin{pgfscope}%
\pgfpathrectangle{\pgfqpoint{0.100000in}{0.212622in}}{\pgfqpoint{3.696000in}{3.696000in}}%
\pgfusepath{clip}%
\pgfsetbuttcap%
\pgfsetroundjoin%
\definecolor{currentfill}{rgb}{0.121569,0.466667,0.705882}%
\pgfsetfillcolor{currentfill}%
\pgfsetfillopacity{0.537908}%
\pgfsetlinewidth{1.003750pt}%
\definecolor{currentstroke}{rgb}{0.121569,0.466667,0.705882}%
\pgfsetstrokecolor{currentstroke}%
\pgfsetstrokeopacity{0.537908}%
\pgfsetdash{}{0pt}%
\pgfpathmoveto{\pgfqpoint{1.212305in}{1.681094in}}%
\pgfpathcurveto{\pgfqpoint{1.220541in}{1.681094in}}{\pgfqpoint{1.228441in}{1.684366in}}{\pgfqpoint{1.234265in}{1.690190in}}%
\pgfpathcurveto{\pgfqpoint{1.240089in}{1.696014in}}{\pgfqpoint{1.243361in}{1.703914in}}{\pgfqpoint{1.243361in}{1.712150in}}%
\pgfpathcurveto{\pgfqpoint{1.243361in}{1.720386in}}{\pgfqpoint{1.240089in}{1.728286in}}{\pgfqpoint{1.234265in}{1.734110in}}%
\pgfpathcurveto{\pgfqpoint{1.228441in}{1.739934in}}{\pgfqpoint{1.220541in}{1.743207in}}{\pgfqpoint{1.212305in}{1.743207in}}%
\pgfpathcurveto{\pgfqpoint{1.204068in}{1.743207in}}{\pgfqpoint{1.196168in}{1.739934in}}{\pgfqpoint{1.190344in}{1.734110in}}%
\pgfpathcurveto{\pgfqpoint{1.184520in}{1.728286in}}{\pgfqpoint{1.181248in}{1.720386in}}{\pgfqpoint{1.181248in}{1.712150in}}%
\pgfpathcurveto{\pgfqpoint{1.181248in}{1.703914in}}{\pgfqpoint{1.184520in}{1.696014in}}{\pgfqpoint{1.190344in}{1.690190in}}%
\pgfpathcurveto{\pgfqpoint{1.196168in}{1.684366in}}{\pgfqpoint{1.204068in}{1.681094in}}{\pgfqpoint{1.212305in}{1.681094in}}%
\pgfpathclose%
\pgfusepath{stroke,fill}%
\end{pgfscope}%
\begin{pgfscope}%
\pgfpathrectangle{\pgfqpoint{0.100000in}{0.212622in}}{\pgfqpoint{3.696000in}{3.696000in}}%
\pgfusepath{clip}%
\pgfsetbuttcap%
\pgfsetroundjoin%
\definecolor{currentfill}{rgb}{0.121569,0.466667,0.705882}%
\pgfsetfillcolor{currentfill}%
\pgfsetfillopacity{0.538167}%
\pgfsetlinewidth{1.003750pt}%
\definecolor{currentstroke}{rgb}{0.121569,0.466667,0.705882}%
\pgfsetstrokecolor{currentstroke}%
\pgfsetstrokeopacity{0.538167}%
\pgfsetdash{}{0pt}%
\pgfpathmoveto{\pgfqpoint{1.209360in}{1.677563in}}%
\pgfpathcurveto{\pgfqpoint{1.217596in}{1.677563in}}{\pgfqpoint{1.225496in}{1.680835in}}{\pgfqpoint{1.231320in}{1.686659in}}%
\pgfpathcurveto{\pgfqpoint{1.237144in}{1.692483in}}{\pgfqpoint{1.240416in}{1.700383in}}{\pgfqpoint{1.240416in}{1.708620in}}%
\pgfpathcurveto{\pgfqpoint{1.240416in}{1.716856in}}{\pgfqpoint{1.237144in}{1.724756in}}{\pgfqpoint{1.231320in}{1.730580in}}%
\pgfpathcurveto{\pgfqpoint{1.225496in}{1.736404in}}{\pgfqpoint{1.217596in}{1.739676in}}{\pgfqpoint{1.209360in}{1.739676in}}%
\pgfpathcurveto{\pgfqpoint{1.201124in}{1.739676in}}{\pgfqpoint{1.193224in}{1.736404in}}{\pgfqpoint{1.187400in}{1.730580in}}%
\pgfpathcurveto{\pgfqpoint{1.181576in}{1.724756in}}{\pgfqpoint{1.178303in}{1.716856in}}{\pgfqpoint{1.178303in}{1.708620in}}%
\pgfpathcurveto{\pgfqpoint{1.178303in}{1.700383in}}{\pgfqpoint{1.181576in}{1.692483in}}{\pgfqpoint{1.187400in}{1.686659in}}%
\pgfpathcurveto{\pgfqpoint{1.193224in}{1.680835in}}{\pgfqpoint{1.201124in}{1.677563in}}{\pgfqpoint{1.209360in}{1.677563in}}%
\pgfpathclose%
\pgfusepath{stroke,fill}%
\end{pgfscope}%
\begin{pgfscope}%
\pgfpathrectangle{\pgfqpoint{0.100000in}{0.212622in}}{\pgfqpoint{3.696000in}{3.696000in}}%
\pgfusepath{clip}%
\pgfsetbuttcap%
\pgfsetroundjoin%
\definecolor{currentfill}{rgb}{0.121569,0.466667,0.705882}%
\pgfsetfillcolor{currentfill}%
\pgfsetfillopacity{0.538202}%
\pgfsetlinewidth{1.003750pt}%
\definecolor{currentstroke}{rgb}{0.121569,0.466667,0.705882}%
\pgfsetstrokecolor{currentstroke}%
\pgfsetstrokeopacity{0.538202}%
\pgfsetdash{}{0pt}%
\pgfpathmoveto{\pgfqpoint{1.203488in}{1.669996in}}%
\pgfpathcurveto{\pgfqpoint{1.211724in}{1.669996in}}{\pgfqpoint{1.219624in}{1.673268in}}{\pgfqpoint{1.225448in}{1.679092in}}%
\pgfpathcurveto{\pgfqpoint{1.231272in}{1.684916in}}{\pgfqpoint{1.234544in}{1.692816in}}{\pgfqpoint{1.234544in}{1.701052in}}%
\pgfpathcurveto{\pgfqpoint{1.234544in}{1.709289in}}{\pgfqpoint{1.231272in}{1.717189in}}{\pgfqpoint{1.225448in}{1.723013in}}%
\pgfpathcurveto{\pgfqpoint{1.219624in}{1.728836in}}{\pgfqpoint{1.211724in}{1.732109in}}{\pgfqpoint{1.203488in}{1.732109in}}%
\pgfpathcurveto{\pgfqpoint{1.195251in}{1.732109in}}{\pgfqpoint{1.187351in}{1.728836in}}{\pgfqpoint{1.181527in}{1.723013in}}%
\pgfpathcurveto{\pgfqpoint{1.175703in}{1.717189in}}{\pgfqpoint{1.172431in}{1.709289in}}{\pgfqpoint{1.172431in}{1.701052in}}%
\pgfpathcurveto{\pgfqpoint{1.172431in}{1.692816in}}{\pgfqpoint{1.175703in}{1.684916in}}{\pgfqpoint{1.181527in}{1.679092in}}%
\pgfpathcurveto{\pgfqpoint{1.187351in}{1.673268in}}{\pgfqpoint{1.195251in}{1.669996in}}{\pgfqpoint{1.203488in}{1.669996in}}%
\pgfpathclose%
\pgfusepath{stroke,fill}%
\end{pgfscope}%
\begin{pgfscope}%
\pgfpathrectangle{\pgfqpoint{0.100000in}{0.212622in}}{\pgfqpoint{3.696000in}{3.696000in}}%
\pgfusepath{clip}%
\pgfsetbuttcap%
\pgfsetroundjoin%
\definecolor{currentfill}{rgb}{0.121569,0.466667,0.705882}%
\pgfsetfillcolor{currentfill}%
\pgfsetfillopacity{0.538958}%
\pgfsetlinewidth{1.003750pt}%
\definecolor{currentstroke}{rgb}{0.121569,0.466667,0.705882}%
\pgfsetstrokecolor{currentstroke}%
\pgfsetstrokeopacity{0.538958}%
\pgfsetdash{}{0pt}%
\pgfpathmoveto{\pgfqpoint{2.070187in}{1.846030in}}%
\pgfpathcurveto{\pgfqpoint{2.078423in}{1.846030in}}{\pgfqpoint{2.086323in}{1.849303in}}{\pgfqpoint{2.092147in}{1.855127in}}%
\pgfpathcurveto{\pgfqpoint{2.097971in}{1.860951in}}{\pgfqpoint{2.101244in}{1.868851in}}{\pgfqpoint{2.101244in}{1.877087in}}%
\pgfpathcurveto{\pgfqpoint{2.101244in}{1.885323in}}{\pgfqpoint{2.097971in}{1.893223in}}{\pgfqpoint{2.092147in}{1.899047in}}%
\pgfpathcurveto{\pgfqpoint{2.086323in}{1.904871in}}{\pgfqpoint{2.078423in}{1.908143in}}{\pgfqpoint{2.070187in}{1.908143in}}%
\pgfpathcurveto{\pgfqpoint{2.061951in}{1.908143in}}{\pgfqpoint{2.054051in}{1.904871in}}{\pgfqpoint{2.048227in}{1.899047in}}%
\pgfpathcurveto{\pgfqpoint{2.042403in}{1.893223in}}{\pgfqpoint{2.039131in}{1.885323in}}{\pgfqpoint{2.039131in}{1.877087in}}%
\pgfpathcurveto{\pgfqpoint{2.039131in}{1.868851in}}{\pgfqpoint{2.042403in}{1.860951in}}{\pgfqpoint{2.048227in}{1.855127in}}%
\pgfpathcurveto{\pgfqpoint{2.054051in}{1.849303in}}{\pgfqpoint{2.061951in}{1.846030in}}{\pgfqpoint{2.070187in}{1.846030in}}%
\pgfpathclose%
\pgfusepath{stroke,fill}%
\end{pgfscope}%
\begin{pgfscope}%
\pgfpathrectangle{\pgfqpoint{0.100000in}{0.212622in}}{\pgfqpoint{3.696000in}{3.696000in}}%
\pgfusepath{clip}%
\pgfsetbuttcap%
\pgfsetroundjoin%
\definecolor{currentfill}{rgb}{0.121569,0.466667,0.705882}%
\pgfsetfillcolor{currentfill}%
\pgfsetfillopacity{0.539297}%
\pgfsetlinewidth{1.003750pt}%
\definecolor{currentstroke}{rgb}{0.121569,0.466667,0.705882}%
\pgfsetstrokecolor{currentstroke}%
\pgfsetstrokeopacity{0.539297}%
\pgfsetdash{}{0pt}%
\pgfpathmoveto{\pgfqpoint{1.201039in}{1.668972in}}%
\pgfpathcurveto{\pgfqpoint{1.209275in}{1.668972in}}{\pgfqpoint{1.217175in}{1.672244in}}{\pgfqpoint{1.222999in}{1.678068in}}%
\pgfpathcurveto{\pgfqpoint{1.228823in}{1.683892in}}{\pgfqpoint{1.232096in}{1.691792in}}{\pgfqpoint{1.232096in}{1.700028in}}%
\pgfpathcurveto{\pgfqpoint{1.232096in}{1.708264in}}{\pgfqpoint{1.228823in}{1.716164in}}{\pgfqpoint{1.222999in}{1.721988in}}%
\pgfpathcurveto{\pgfqpoint{1.217175in}{1.727812in}}{\pgfqpoint{1.209275in}{1.731085in}}{\pgfqpoint{1.201039in}{1.731085in}}%
\pgfpathcurveto{\pgfqpoint{1.192803in}{1.731085in}}{\pgfqpoint{1.184903in}{1.727812in}}{\pgfqpoint{1.179079in}{1.721988in}}%
\pgfpathcurveto{\pgfqpoint{1.173255in}{1.716164in}}{\pgfqpoint{1.169983in}{1.708264in}}{\pgfqpoint{1.169983in}{1.700028in}}%
\pgfpathcurveto{\pgfqpoint{1.169983in}{1.691792in}}{\pgfqpoint{1.173255in}{1.683892in}}{\pgfqpoint{1.179079in}{1.678068in}}%
\pgfpathcurveto{\pgfqpoint{1.184903in}{1.672244in}}{\pgfqpoint{1.192803in}{1.668972in}}{\pgfqpoint{1.201039in}{1.668972in}}%
\pgfpathclose%
\pgfusepath{stroke,fill}%
\end{pgfscope}%
\begin{pgfscope}%
\pgfpathrectangle{\pgfqpoint{0.100000in}{0.212622in}}{\pgfqpoint{3.696000in}{3.696000in}}%
\pgfusepath{clip}%
\pgfsetbuttcap%
\pgfsetroundjoin%
\definecolor{currentfill}{rgb}{0.121569,0.466667,0.705882}%
\pgfsetfillcolor{currentfill}%
\pgfsetfillopacity{0.539587}%
\pgfsetlinewidth{1.003750pt}%
\definecolor{currentstroke}{rgb}{0.121569,0.466667,0.705882}%
\pgfsetstrokecolor{currentstroke}%
\pgfsetstrokeopacity{0.539587}%
\pgfsetdash{}{0pt}%
\pgfpathmoveto{\pgfqpoint{1.198879in}{1.667103in}}%
\pgfpathcurveto{\pgfqpoint{1.207115in}{1.667103in}}{\pgfqpoint{1.215015in}{1.670375in}}{\pgfqpoint{1.220839in}{1.676199in}}%
\pgfpathcurveto{\pgfqpoint{1.226663in}{1.682023in}}{\pgfqpoint{1.229935in}{1.689923in}}{\pgfqpoint{1.229935in}{1.698159in}}%
\pgfpathcurveto{\pgfqpoint{1.229935in}{1.706395in}}{\pgfqpoint{1.226663in}{1.714295in}}{\pgfqpoint{1.220839in}{1.720119in}}%
\pgfpathcurveto{\pgfqpoint{1.215015in}{1.725943in}}{\pgfqpoint{1.207115in}{1.729216in}}{\pgfqpoint{1.198879in}{1.729216in}}%
\pgfpathcurveto{\pgfqpoint{1.190642in}{1.729216in}}{\pgfqpoint{1.182742in}{1.725943in}}{\pgfqpoint{1.176918in}{1.720119in}}%
\pgfpathcurveto{\pgfqpoint{1.171094in}{1.714295in}}{\pgfqpoint{1.167822in}{1.706395in}}{\pgfqpoint{1.167822in}{1.698159in}}%
\pgfpathcurveto{\pgfqpoint{1.167822in}{1.689923in}}{\pgfqpoint{1.171094in}{1.682023in}}{\pgfqpoint{1.176918in}{1.676199in}}%
\pgfpathcurveto{\pgfqpoint{1.182742in}{1.670375in}}{\pgfqpoint{1.190642in}{1.667103in}}{\pgfqpoint{1.198879in}{1.667103in}}%
\pgfpathclose%
\pgfusepath{stroke,fill}%
\end{pgfscope}%
\begin{pgfscope}%
\pgfpathrectangle{\pgfqpoint{0.100000in}{0.212622in}}{\pgfqpoint{3.696000in}{3.696000in}}%
\pgfusepath{clip}%
\pgfsetbuttcap%
\pgfsetroundjoin%
\definecolor{currentfill}{rgb}{0.121569,0.466667,0.705882}%
\pgfsetfillcolor{currentfill}%
\pgfsetfillopacity{0.539864}%
\pgfsetlinewidth{1.003750pt}%
\definecolor{currentstroke}{rgb}{0.121569,0.466667,0.705882}%
\pgfsetstrokecolor{currentstroke}%
\pgfsetstrokeopacity{0.539864}%
\pgfsetdash{}{0pt}%
\pgfpathmoveto{\pgfqpoint{1.195053in}{1.661899in}}%
\pgfpathcurveto{\pgfqpoint{1.203290in}{1.661899in}}{\pgfqpoint{1.211190in}{1.665171in}}{\pgfqpoint{1.217014in}{1.670995in}}%
\pgfpathcurveto{\pgfqpoint{1.222838in}{1.676819in}}{\pgfqpoint{1.226110in}{1.684719in}}{\pgfqpoint{1.226110in}{1.692955in}}%
\pgfpathcurveto{\pgfqpoint{1.226110in}{1.701192in}}{\pgfqpoint{1.222838in}{1.709092in}}{\pgfqpoint{1.217014in}{1.714916in}}%
\pgfpathcurveto{\pgfqpoint{1.211190in}{1.720740in}}{\pgfqpoint{1.203290in}{1.724012in}}{\pgfqpoint{1.195053in}{1.724012in}}%
\pgfpathcurveto{\pgfqpoint{1.186817in}{1.724012in}}{\pgfqpoint{1.178917in}{1.720740in}}{\pgfqpoint{1.173093in}{1.714916in}}%
\pgfpathcurveto{\pgfqpoint{1.167269in}{1.709092in}}{\pgfqpoint{1.163997in}{1.701192in}}{\pgfqpoint{1.163997in}{1.692955in}}%
\pgfpathcurveto{\pgfqpoint{1.163997in}{1.684719in}}{\pgfqpoint{1.167269in}{1.676819in}}{\pgfqpoint{1.173093in}{1.670995in}}%
\pgfpathcurveto{\pgfqpoint{1.178917in}{1.665171in}}{\pgfqpoint{1.186817in}{1.661899in}}{\pgfqpoint{1.195053in}{1.661899in}}%
\pgfpathclose%
\pgfusepath{stroke,fill}%
\end{pgfscope}%
\begin{pgfscope}%
\pgfpathrectangle{\pgfqpoint{0.100000in}{0.212622in}}{\pgfqpoint{3.696000in}{3.696000in}}%
\pgfusepath{clip}%
\pgfsetbuttcap%
\pgfsetroundjoin%
\definecolor{currentfill}{rgb}{0.121569,0.466667,0.705882}%
\pgfsetfillcolor{currentfill}%
\pgfsetfillopacity{0.540494}%
\pgfsetlinewidth{1.003750pt}%
\definecolor{currentstroke}{rgb}{0.121569,0.466667,0.705882}%
\pgfsetstrokecolor{currentstroke}%
\pgfsetstrokeopacity{0.540494}%
\pgfsetdash{}{0pt}%
\pgfpathmoveto{\pgfqpoint{1.193685in}{1.661825in}}%
\pgfpathcurveto{\pgfqpoint{1.201921in}{1.661825in}}{\pgfqpoint{1.209822in}{1.665097in}}{\pgfqpoint{1.215645in}{1.670921in}}%
\pgfpathcurveto{\pgfqpoint{1.221469in}{1.676745in}}{\pgfqpoint{1.224742in}{1.684645in}}{\pgfqpoint{1.224742in}{1.692881in}}%
\pgfpathcurveto{\pgfqpoint{1.224742in}{1.701117in}}{\pgfqpoint{1.221469in}{1.709017in}}{\pgfqpoint{1.215645in}{1.714841in}}%
\pgfpathcurveto{\pgfqpoint{1.209822in}{1.720665in}}{\pgfqpoint{1.201921in}{1.723938in}}{\pgfqpoint{1.193685in}{1.723938in}}%
\pgfpathcurveto{\pgfqpoint{1.185449in}{1.723938in}}{\pgfqpoint{1.177549in}{1.720665in}}{\pgfqpoint{1.171725in}{1.714841in}}%
\pgfpathcurveto{\pgfqpoint{1.165901in}{1.709017in}}{\pgfqpoint{1.162629in}{1.701117in}}{\pgfqpoint{1.162629in}{1.692881in}}%
\pgfpathcurveto{\pgfqpoint{1.162629in}{1.684645in}}{\pgfqpoint{1.165901in}{1.676745in}}{\pgfqpoint{1.171725in}{1.670921in}}%
\pgfpathcurveto{\pgfqpoint{1.177549in}{1.665097in}}{\pgfqpoint{1.185449in}{1.661825in}}{\pgfqpoint{1.193685in}{1.661825in}}%
\pgfpathclose%
\pgfusepath{stroke,fill}%
\end{pgfscope}%
\begin{pgfscope}%
\pgfpathrectangle{\pgfqpoint{0.100000in}{0.212622in}}{\pgfqpoint{3.696000in}{3.696000in}}%
\pgfusepath{clip}%
\pgfsetbuttcap%
\pgfsetroundjoin%
\definecolor{currentfill}{rgb}{0.121569,0.466667,0.705882}%
\pgfsetfillcolor{currentfill}%
\pgfsetfillopacity{0.541176}%
\pgfsetlinewidth{1.003750pt}%
\definecolor{currentstroke}{rgb}{0.121569,0.466667,0.705882}%
\pgfsetstrokecolor{currentstroke}%
\pgfsetstrokeopacity{0.541176}%
\pgfsetdash{}{0pt}%
\pgfpathmoveto{\pgfqpoint{1.190662in}{1.659539in}}%
\pgfpathcurveto{\pgfqpoint{1.198898in}{1.659539in}}{\pgfqpoint{1.206798in}{1.662811in}}{\pgfqpoint{1.212622in}{1.668635in}}%
\pgfpathcurveto{\pgfqpoint{1.218446in}{1.674459in}}{\pgfqpoint{1.221718in}{1.682359in}}{\pgfqpoint{1.221718in}{1.690595in}}%
\pgfpathcurveto{\pgfqpoint{1.221718in}{1.698832in}}{\pgfqpoint{1.218446in}{1.706732in}}{\pgfqpoint{1.212622in}{1.712556in}}%
\pgfpathcurveto{\pgfqpoint{1.206798in}{1.718380in}}{\pgfqpoint{1.198898in}{1.721652in}}{\pgfqpoint{1.190662in}{1.721652in}}%
\pgfpathcurveto{\pgfqpoint{1.182426in}{1.721652in}}{\pgfqpoint{1.174525in}{1.718380in}}{\pgfqpoint{1.168702in}{1.712556in}}%
\pgfpathcurveto{\pgfqpoint{1.162878in}{1.706732in}}{\pgfqpoint{1.159605in}{1.698832in}}{\pgfqpoint{1.159605in}{1.690595in}}%
\pgfpathcurveto{\pgfqpoint{1.159605in}{1.682359in}}{\pgfqpoint{1.162878in}{1.674459in}}{\pgfqpoint{1.168702in}{1.668635in}}%
\pgfpathcurveto{\pgfqpoint{1.174525in}{1.662811in}}{\pgfqpoint{1.182426in}{1.659539in}}{\pgfqpoint{1.190662in}{1.659539in}}%
\pgfpathclose%
\pgfusepath{stroke,fill}%
\end{pgfscope}%
\begin{pgfscope}%
\pgfpathrectangle{\pgfqpoint{0.100000in}{0.212622in}}{\pgfqpoint{3.696000in}{3.696000in}}%
\pgfusepath{clip}%
\pgfsetbuttcap%
\pgfsetroundjoin%
\definecolor{currentfill}{rgb}{0.121569,0.466667,0.705882}%
\pgfsetfillcolor{currentfill}%
\pgfsetfillopacity{0.542254}%
\pgfsetlinewidth{1.003750pt}%
\definecolor{currentstroke}{rgb}{0.121569,0.466667,0.705882}%
\pgfsetstrokecolor{currentstroke}%
\pgfsetstrokeopacity{0.542254}%
\pgfsetdash{}{0pt}%
\pgfpathmoveto{\pgfqpoint{2.072852in}{1.843933in}}%
\pgfpathcurveto{\pgfqpoint{2.081089in}{1.843933in}}{\pgfqpoint{2.088989in}{1.847205in}}{\pgfqpoint{2.094813in}{1.853029in}}%
\pgfpathcurveto{\pgfqpoint{2.100637in}{1.858853in}}{\pgfqpoint{2.103909in}{1.866753in}}{\pgfqpoint{2.103909in}{1.874989in}}%
\pgfpathcurveto{\pgfqpoint{2.103909in}{1.883226in}}{\pgfqpoint{2.100637in}{1.891126in}}{\pgfqpoint{2.094813in}{1.896950in}}%
\pgfpathcurveto{\pgfqpoint{2.088989in}{1.902774in}}{\pgfqpoint{2.081089in}{1.906046in}}{\pgfqpoint{2.072852in}{1.906046in}}%
\pgfpathcurveto{\pgfqpoint{2.064616in}{1.906046in}}{\pgfqpoint{2.056716in}{1.902774in}}{\pgfqpoint{2.050892in}{1.896950in}}%
\pgfpathcurveto{\pgfqpoint{2.045068in}{1.891126in}}{\pgfqpoint{2.041796in}{1.883226in}}{\pgfqpoint{2.041796in}{1.874989in}}%
\pgfpathcurveto{\pgfqpoint{2.041796in}{1.866753in}}{\pgfqpoint{2.045068in}{1.858853in}}{\pgfqpoint{2.050892in}{1.853029in}}%
\pgfpathcurveto{\pgfqpoint{2.056716in}{1.847205in}}{\pgfqpoint{2.064616in}{1.843933in}}{\pgfqpoint{2.072852in}{1.843933in}}%
\pgfpathclose%
\pgfusepath{stroke,fill}%
\end{pgfscope}%
\begin{pgfscope}%
\pgfpathrectangle{\pgfqpoint{0.100000in}{0.212622in}}{\pgfqpoint{3.696000in}{3.696000in}}%
\pgfusepath{clip}%
\pgfsetbuttcap%
\pgfsetroundjoin%
\definecolor{currentfill}{rgb}{0.121569,0.466667,0.705882}%
\pgfsetfillcolor{currentfill}%
\pgfsetfillopacity{0.542554}%
\pgfsetlinewidth{1.003750pt}%
\definecolor{currentstroke}{rgb}{0.121569,0.466667,0.705882}%
\pgfsetstrokecolor{currentstroke}%
\pgfsetstrokeopacity{0.542554}%
\pgfsetdash{}{0pt}%
\pgfpathmoveto{\pgfqpoint{1.185873in}{1.655114in}}%
\pgfpathcurveto{\pgfqpoint{1.194109in}{1.655114in}}{\pgfqpoint{1.202009in}{1.658387in}}{\pgfqpoint{1.207833in}{1.664211in}}%
\pgfpathcurveto{\pgfqpoint{1.213657in}{1.670034in}}{\pgfqpoint{1.216929in}{1.677935in}}{\pgfqpoint{1.216929in}{1.686171in}}%
\pgfpathcurveto{\pgfqpoint{1.216929in}{1.694407in}}{\pgfqpoint{1.213657in}{1.702307in}}{\pgfqpoint{1.207833in}{1.708131in}}%
\pgfpathcurveto{\pgfqpoint{1.202009in}{1.713955in}}{\pgfqpoint{1.194109in}{1.717227in}}{\pgfqpoint{1.185873in}{1.717227in}}%
\pgfpathcurveto{\pgfqpoint{1.177637in}{1.717227in}}{\pgfqpoint{1.169737in}{1.713955in}}{\pgfqpoint{1.163913in}{1.708131in}}%
\pgfpathcurveto{\pgfqpoint{1.158089in}{1.702307in}}{\pgfqpoint{1.154816in}{1.694407in}}{\pgfqpoint{1.154816in}{1.686171in}}%
\pgfpathcurveto{\pgfqpoint{1.154816in}{1.677935in}}{\pgfqpoint{1.158089in}{1.670034in}}{\pgfqpoint{1.163913in}{1.664211in}}%
\pgfpathcurveto{\pgfqpoint{1.169737in}{1.658387in}}{\pgfqpoint{1.177637in}{1.655114in}}{\pgfqpoint{1.185873in}{1.655114in}}%
\pgfpathclose%
\pgfusepath{stroke,fill}%
\end{pgfscope}%
\begin{pgfscope}%
\pgfpathrectangle{\pgfqpoint{0.100000in}{0.212622in}}{\pgfqpoint{3.696000in}{3.696000in}}%
\pgfusepath{clip}%
\pgfsetbuttcap%
\pgfsetroundjoin%
\definecolor{currentfill}{rgb}{0.121569,0.466667,0.705882}%
\pgfsetfillcolor{currentfill}%
\pgfsetfillopacity{0.544032}%
\pgfsetlinewidth{1.003750pt}%
\definecolor{currentstroke}{rgb}{0.121569,0.466667,0.705882}%
\pgfsetstrokecolor{currentstroke}%
\pgfsetstrokeopacity{0.544032}%
\pgfsetdash{}{0pt}%
\pgfpathmoveto{\pgfqpoint{1.183241in}{1.655610in}}%
\pgfpathcurveto{\pgfqpoint{1.191478in}{1.655610in}}{\pgfqpoint{1.199378in}{1.658882in}}{\pgfqpoint{1.205202in}{1.664706in}}%
\pgfpathcurveto{\pgfqpoint{1.211025in}{1.670530in}}{\pgfqpoint{1.214298in}{1.678430in}}{\pgfqpoint{1.214298in}{1.686666in}}%
\pgfpathcurveto{\pgfqpoint{1.214298in}{1.694903in}}{\pgfqpoint{1.211025in}{1.702803in}}{\pgfqpoint{1.205202in}{1.708627in}}%
\pgfpathcurveto{\pgfqpoint{1.199378in}{1.714451in}}{\pgfqpoint{1.191478in}{1.717723in}}{\pgfqpoint{1.183241in}{1.717723in}}%
\pgfpathcurveto{\pgfqpoint{1.175005in}{1.717723in}}{\pgfqpoint{1.167105in}{1.714451in}}{\pgfqpoint{1.161281in}{1.708627in}}%
\pgfpathcurveto{\pgfqpoint{1.155457in}{1.702803in}}{\pgfqpoint{1.152185in}{1.694903in}}{\pgfqpoint{1.152185in}{1.686666in}}%
\pgfpathcurveto{\pgfqpoint{1.152185in}{1.678430in}}{\pgfqpoint{1.155457in}{1.670530in}}{\pgfqpoint{1.161281in}{1.664706in}}%
\pgfpathcurveto{\pgfqpoint{1.167105in}{1.658882in}}{\pgfqpoint{1.175005in}{1.655610in}}{\pgfqpoint{1.183241in}{1.655610in}}%
\pgfpathclose%
\pgfusepath{stroke,fill}%
\end{pgfscope}%
\begin{pgfscope}%
\pgfpathrectangle{\pgfqpoint{0.100000in}{0.212622in}}{\pgfqpoint{3.696000in}{3.696000in}}%
\pgfusepath{clip}%
\pgfsetbuttcap%
\pgfsetroundjoin%
\definecolor{currentfill}{rgb}{0.121569,0.466667,0.705882}%
\pgfsetfillcolor{currentfill}%
\pgfsetfillopacity{0.545165}%
\pgfsetlinewidth{1.003750pt}%
\definecolor{currentstroke}{rgb}{0.121569,0.466667,0.705882}%
\pgfsetstrokecolor{currentstroke}%
\pgfsetstrokeopacity{0.545165}%
\pgfsetdash{}{0pt}%
\pgfpathmoveto{\pgfqpoint{1.175789in}{1.651294in}}%
\pgfpathcurveto{\pgfqpoint{1.184026in}{1.651294in}}{\pgfqpoint{1.191926in}{1.654567in}}{\pgfqpoint{1.197750in}{1.660390in}}%
\pgfpathcurveto{\pgfqpoint{1.203573in}{1.666214in}}{\pgfqpoint{1.206846in}{1.674114in}}{\pgfqpoint{1.206846in}{1.682351in}}%
\pgfpathcurveto{\pgfqpoint{1.206846in}{1.690587in}}{\pgfqpoint{1.203573in}{1.698487in}}{\pgfqpoint{1.197750in}{1.704311in}}%
\pgfpathcurveto{\pgfqpoint{1.191926in}{1.710135in}}{\pgfqpoint{1.184026in}{1.713407in}}{\pgfqpoint{1.175789in}{1.713407in}}%
\pgfpathcurveto{\pgfqpoint{1.167553in}{1.713407in}}{\pgfqpoint{1.159653in}{1.710135in}}{\pgfqpoint{1.153829in}{1.704311in}}%
\pgfpathcurveto{\pgfqpoint{1.148005in}{1.698487in}}{\pgfqpoint{1.144733in}{1.690587in}}{\pgfqpoint{1.144733in}{1.682351in}}%
\pgfpathcurveto{\pgfqpoint{1.144733in}{1.674114in}}{\pgfqpoint{1.148005in}{1.666214in}}{\pgfqpoint{1.153829in}{1.660390in}}%
\pgfpathcurveto{\pgfqpoint{1.159653in}{1.654567in}}{\pgfqpoint{1.167553in}{1.651294in}}{\pgfqpoint{1.175789in}{1.651294in}}%
\pgfpathclose%
\pgfusepath{stroke,fill}%
\end{pgfscope}%
\begin{pgfscope}%
\pgfpathrectangle{\pgfqpoint{0.100000in}{0.212622in}}{\pgfqpoint{3.696000in}{3.696000in}}%
\pgfusepath{clip}%
\pgfsetbuttcap%
\pgfsetroundjoin%
\definecolor{currentfill}{rgb}{0.121569,0.466667,0.705882}%
\pgfsetfillcolor{currentfill}%
\pgfsetfillopacity{0.545771}%
\pgfsetlinewidth{1.003750pt}%
\definecolor{currentstroke}{rgb}{0.121569,0.466667,0.705882}%
\pgfsetstrokecolor{currentstroke}%
\pgfsetstrokeopacity{0.545771}%
\pgfsetdash{}{0pt}%
\pgfpathmoveto{\pgfqpoint{2.073935in}{1.839718in}}%
\pgfpathcurveto{\pgfqpoint{2.082171in}{1.839718in}}{\pgfqpoint{2.090071in}{1.842990in}}{\pgfqpoint{2.095895in}{1.848814in}}%
\pgfpathcurveto{\pgfqpoint{2.101719in}{1.854638in}}{\pgfqpoint{2.104991in}{1.862538in}}{\pgfqpoint{2.104991in}{1.870774in}}%
\pgfpathcurveto{\pgfqpoint{2.104991in}{1.879010in}}{\pgfqpoint{2.101719in}{1.886911in}}{\pgfqpoint{2.095895in}{1.892734in}}%
\pgfpathcurveto{\pgfqpoint{2.090071in}{1.898558in}}{\pgfqpoint{2.082171in}{1.901831in}}{\pgfqpoint{2.073935in}{1.901831in}}%
\pgfpathcurveto{\pgfqpoint{2.065698in}{1.901831in}}{\pgfqpoint{2.057798in}{1.898558in}}{\pgfqpoint{2.051974in}{1.892734in}}%
\pgfpathcurveto{\pgfqpoint{2.046150in}{1.886911in}}{\pgfqpoint{2.042878in}{1.879010in}}{\pgfqpoint{2.042878in}{1.870774in}}%
\pgfpathcurveto{\pgfqpoint{2.042878in}{1.862538in}}{\pgfqpoint{2.046150in}{1.854638in}}{\pgfqpoint{2.051974in}{1.848814in}}%
\pgfpathcurveto{\pgfqpoint{2.057798in}{1.842990in}}{\pgfqpoint{2.065698in}{1.839718in}}{\pgfqpoint{2.073935in}{1.839718in}}%
\pgfpathclose%
\pgfusepath{stroke,fill}%
\end{pgfscope}%
\begin{pgfscope}%
\pgfpathrectangle{\pgfqpoint{0.100000in}{0.212622in}}{\pgfqpoint{3.696000in}{3.696000in}}%
\pgfusepath{clip}%
\pgfsetbuttcap%
\pgfsetroundjoin%
\definecolor{currentfill}{rgb}{0.121569,0.466667,0.705882}%
\pgfsetfillcolor{currentfill}%
\pgfsetfillopacity{0.545948}%
\pgfsetlinewidth{1.003750pt}%
\definecolor{currentstroke}{rgb}{0.121569,0.466667,0.705882}%
\pgfsetstrokecolor{currentstroke}%
\pgfsetstrokeopacity{0.545948}%
\pgfsetdash{}{0pt}%
\pgfpathmoveto{\pgfqpoint{1.163162in}{1.633121in}}%
\pgfpathcurveto{\pgfqpoint{1.171399in}{1.633121in}}{\pgfqpoint{1.179299in}{1.636394in}}{\pgfqpoint{1.185123in}{1.642218in}}%
\pgfpathcurveto{\pgfqpoint{1.190947in}{1.648042in}}{\pgfqpoint{1.194219in}{1.655942in}}{\pgfqpoint{1.194219in}{1.664178in}}%
\pgfpathcurveto{\pgfqpoint{1.194219in}{1.672414in}}{\pgfqpoint{1.190947in}{1.680314in}}{\pgfqpoint{1.185123in}{1.686138in}}%
\pgfpathcurveto{\pgfqpoint{1.179299in}{1.691962in}}{\pgfqpoint{1.171399in}{1.695234in}}{\pgfqpoint{1.163162in}{1.695234in}}%
\pgfpathcurveto{\pgfqpoint{1.154926in}{1.695234in}}{\pgfqpoint{1.147026in}{1.691962in}}{\pgfqpoint{1.141202in}{1.686138in}}%
\pgfpathcurveto{\pgfqpoint{1.135378in}{1.680314in}}{\pgfqpoint{1.132106in}{1.672414in}}{\pgfqpoint{1.132106in}{1.664178in}}%
\pgfpathcurveto{\pgfqpoint{1.132106in}{1.655942in}}{\pgfqpoint{1.135378in}{1.648042in}}{\pgfqpoint{1.141202in}{1.642218in}}%
\pgfpathcurveto{\pgfqpoint{1.147026in}{1.636394in}}{\pgfqpoint{1.154926in}{1.633121in}}{\pgfqpoint{1.163162in}{1.633121in}}%
\pgfpathclose%
\pgfusepath{stroke,fill}%
\end{pgfscope}%
\begin{pgfscope}%
\pgfpathrectangle{\pgfqpoint{0.100000in}{0.212622in}}{\pgfqpoint{3.696000in}{3.696000in}}%
\pgfusepath{clip}%
\pgfsetbuttcap%
\pgfsetroundjoin%
\definecolor{currentfill}{rgb}{0.121569,0.466667,0.705882}%
\pgfsetfillcolor{currentfill}%
\pgfsetfillopacity{0.549399}%
\pgfsetlinewidth{1.003750pt}%
\definecolor{currentstroke}{rgb}{0.121569,0.466667,0.705882}%
\pgfsetstrokecolor{currentstroke}%
\pgfsetstrokeopacity{0.549399}%
\pgfsetdash{}{0pt}%
\pgfpathmoveto{\pgfqpoint{2.076355in}{1.832436in}}%
\pgfpathcurveto{\pgfqpoint{2.084591in}{1.832436in}}{\pgfqpoint{2.092491in}{1.835708in}}{\pgfqpoint{2.098315in}{1.841532in}}%
\pgfpathcurveto{\pgfqpoint{2.104139in}{1.847356in}}{\pgfqpoint{2.107411in}{1.855256in}}{\pgfqpoint{2.107411in}{1.863492in}}%
\pgfpathcurveto{\pgfqpoint{2.107411in}{1.871729in}}{\pgfqpoint{2.104139in}{1.879629in}}{\pgfqpoint{2.098315in}{1.885453in}}%
\pgfpathcurveto{\pgfqpoint{2.092491in}{1.891276in}}{\pgfqpoint{2.084591in}{1.894549in}}{\pgfqpoint{2.076355in}{1.894549in}}%
\pgfpathcurveto{\pgfqpoint{2.068118in}{1.894549in}}{\pgfqpoint{2.060218in}{1.891276in}}{\pgfqpoint{2.054394in}{1.885453in}}%
\pgfpathcurveto{\pgfqpoint{2.048570in}{1.879629in}}{\pgfqpoint{2.045298in}{1.871729in}}{\pgfqpoint{2.045298in}{1.863492in}}%
\pgfpathcurveto{\pgfqpoint{2.045298in}{1.855256in}}{\pgfqpoint{2.048570in}{1.847356in}}{\pgfqpoint{2.054394in}{1.841532in}}%
\pgfpathcurveto{\pgfqpoint{2.060218in}{1.835708in}}{\pgfqpoint{2.068118in}{1.832436in}}{\pgfqpoint{2.076355in}{1.832436in}}%
\pgfpathclose%
\pgfusepath{stroke,fill}%
\end{pgfscope}%
\begin{pgfscope}%
\pgfpathrectangle{\pgfqpoint{0.100000in}{0.212622in}}{\pgfqpoint{3.696000in}{3.696000in}}%
\pgfusepath{clip}%
\pgfsetbuttcap%
\pgfsetroundjoin%
\definecolor{currentfill}{rgb}{0.121569,0.466667,0.705882}%
\pgfsetfillcolor{currentfill}%
\pgfsetfillopacity{0.550051}%
\pgfsetlinewidth{1.003750pt}%
\definecolor{currentstroke}{rgb}{0.121569,0.466667,0.705882}%
\pgfsetstrokecolor{currentstroke}%
\pgfsetstrokeopacity{0.550051}%
\pgfsetdash{}{0pt}%
\pgfpathmoveto{\pgfqpoint{1.154185in}{1.631154in}}%
\pgfpathcurveto{\pgfqpoint{1.162421in}{1.631154in}}{\pgfqpoint{1.170321in}{1.634426in}}{\pgfqpoint{1.176145in}{1.640250in}}%
\pgfpathcurveto{\pgfqpoint{1.181969in}{1.646074in}}{\pgfqpoint{1.185241in}{1.653974in}}{\pgfqpoint{1.185241in}{1.662210in}}%
\pgfpathcurveto{\pgfqpoint{1.185241in}{1.670446in}}{\pgfqpoint{1.181969in}{1.678346in}}{\pgfqpoint{1.176145in}{1.684170in}}%
\pgfpathcurveto{\pgfqpoint{1.170321in}{1.689994in}}{\pgfqpoint{1.162421in}{1.693267in}}{\pgfqpoint{1.154185in}{1.693267in}}%
\pgfpathcurveto{\pgfqpoint{1.145948in}{1.693267in}}{\pgfqpoint{1.138048in}{1.689994in}}{\pgfqpoint{1.132224in}{1.684170in}}%
\pgfpathcurveto{\pgfqpoint{1.126400in}{1.678346in}}{\pgfqpoint{1.123128in}{1.670446in}}{\pgfqpoint{1.123128in}{1.662210in}}%
\pgfpathcurveto{\pgfqpoint{1.123128in}{1.653974in}}{\pgfqpoint{1.126400in}{1.646074in}}{\pgfqpoint{1.132224in}{1.640250in}}%
\pgfpathcurveto{\pgfqpoint{1.138048in}{1.634426in}}{\pgfqpoint{1.145948in}{1.631154in}}{\pgfqpoint{1.154185in}{1.631154in}}%
\pgfpathclose%
\pgfusepath{stroke,fill}%
\end{pgfscope}%
\begin{pgfscope}%
\pgfpathrectangle{\pgfqpoint{0.100000in}{0.212622in}}{\pgfqpoint{3.696000in}{3.696000in}}%
\pgfusepath{clip}%
\pgfsetbuttcap%
\pgfsetroundjoin%
\definecolor{currentfill}{rgb}{0.121569,0.466667,0.705882}%
\pgfsetfillcolor{currentfill}%
\pgfsetfillopacity{0.552338}%
\pgfsetlinewidth{1.003750pt}%
\definecolor{currentstroke}{rgb}{0.121569,0.466667,0.705882}%
\pgfsetstrokecolor{currentstroke}%
\pgfsetstrokeopacity{0.552338}%
\pgfsetdash{}{0pt}%
\pgfpathmoveto{\pgfqpoint{1.145164in}{1.628220in}}%
\pgfpathcurveto{\pgfqpoint{1.153400in}{1.628220in}}{\pgfqpoint{1.161300in}{1.631493in}}{\pgfqpoint{1.167124in}{1.637317in}}%
\pgfpathcurveto{\pgfqpoint{1.172948in}{1.643140in}}{\pgfqpoint{1.176220in}{1.651041in}}{\pgfqpoint{1.176220in}{1.659277in}}%
\pgfpathcurveto{\pgfqpoint{1.176220in}{1.667513in}}{\pgfqpoint{1.172948in}{1.675413in}}{\pgfqpoint{1.167124in}{1.681237in}}%
\pgfpathcurveto{\pgfqpoint{1.161300in}{1.687061in}}{\pgfqpoint{1.153400in}{1.690333in}}{\pgfqpoint{1.145164in}{1.690333in}}%
\pgfpathcurveto{\pgfqpoint{1.136928in}{1.690333in}}{\pgfqpoint{1.129028in}{1.687061in}}{\pgfqpoint{1.123204in}{1.681237in}}%
\pgfpathcurveto{\pgfqpoint{1.117380in}{1.675413in}}{\pgfqpoint{1.114107in}{1.667513in}}{\pgfqpoint{1.114107in}{1.659277in}}%
\pgfpathcurveto{\pgfqpoint{1.114107in}{1.651041in}}{\pgfqpoint{1.117380in}{1.643140in}}{\pgfqpoint{1.123204in}{1.637317in}}%
\pgfpathcurveto{\pgfqpoint{1.129028in}{1.631493in}}{\pgfqpoint{1.136928in}{1.628220in}}{\pgfqpoint{1.145164in}{1.628220in}}%
\pgfpathclose%
\pgfusepath{stroke,fill}%
\end{pgfscope}%
\begin{pgfscope}%
\pgfpathrectangle{\pgfqpoint{0.100000in}{0.212622in}}{\pgfqpoint{3.696000in}{3.696000in}}%
\pgfusepath{clip}%
\pgfsetbuttcap%
\pgfsetroundjoin%
\definecolor{currentfill}{rgb}{0.121569,0.466667,0.705882}%
\pgfsetfillcolor{currentfill}%
\pgfsetfillopacity{0.553716}%
\pgfsetlinewidth{1.003750pt}%
\definecolor{currentstroke}{rgb}{0.121569,0.466667,0.705882}%
\pgfsetstrokecolor{currentstroke}%
\pgfsetstrokeopacity{0.553716}%
\pgfsetdash{}{0pt}%
\pgfpathmoveto{\pgfqpoint{1.140665in}{1.624019in}}%
\pgfpathcurveto{\pgfqpoint{1.148901in}{1.624019in}}{\pgfqpoint{1.156801in}{1.627292in}}{\pgfqpoint{1.162625in}{1.633116in}}%
\pgfpathcurveto{\pgfqpoint{1.168449in}{1.638940in}}{\pgfqpoint{1.171721in}{1.646840in}}{\pgfqpoint{1.171721in}{1.655076in}}%
\pgfpathcurveto{\pgfqpoint{1.171721in}{1.663312in}}{\pgfqpoint{1.168449in}{1.671212in}}{\pgfqpoint{1.162625in}{1.677036in}}%
\pgfpathcurveto{\pgfqpoint{1.156801in}{1.682860in}}{\pgfqpoint{1.148901in}{1.686132in}}{\pgfqpoint{1.140665in}{1.686132in}}%
\pgfpathcurveto{\pgfqpoint{1.132429in}{1.686132in}}{\pgfqpoint{1.124528in}{1.682860in}}{\pgfqpoint{1.118705in}{1.677036in}}%
\pgfpathcurveto{\pgfqpoint{1.112881in}{1.671212in}}{\pgfqpoint{1.109608in}{1.663312in}}{\pgfqpoint{1.109608in}{1.655076in}}%
\pgfpathcurveto{\pgfqpoint{1.109608in}{1.646840in}}{\pgfqpoint{1.112881in}{1.638940in}}{\pgfqpoint{1.118705in}{1.633116in}}%
\pgfpathcurveto{\pgfqpoint{1.124528in}{1.627292in}}{\pgfqpoint{1.132429in}{1.624019in}}{\pgfqpoint{1.140665in}{1.624019in}}%
\pgfpathclose%
\pgfusepath{stroke,fill}%
\end{pgfscope}%
\begin{pgfscope}%
\pgfpathrectangle{\pgfqpoint{0.100000in}{0.212622in}}{\pgfqpoint{3.696000in}{3.696000in}}%
\pgfusepath{clip}%
\pgfsetbuttcap%
\pgfsetroundjoin%
\definecolor{currentfill}{rgb}{0.121569,0.466667,0.705882}%
\pgfsetfillcolor{currentfill}%
\pgfsetfillopacity{0.554084}%
\pgfsetlinewidth{1.003750pt}%
\definecolor{currentstroke}{rgb}{0.121569,0.466667,0.705882}%
\pgfsetstrokecolor{currentstroke}%
\pgfsetstrokeopacity{0.554084}%
\pgfsetdash{}{0pt}%
\pgfpathmoveto{\pgfqpoint{2.079652in}{1.830630in}}%
\pgfpathcurveto{\pgfqpoint{2.087888in}{1.830630in}}{\pgfqpoint{2.095788in}{1.833902in}}{\pgfqpoint{2.101612in}{1.839726in}}%
\pgfpathcurveto{\pgfqpoint{2.107436in}{1.845550in}}{\pgfqpoint{2.110708in}{1.853450in}}{\pgfqpoint{2.110708in}{1.861686in}}%
\pgfpathcurveto{\pgfqpoint{2.110708in}{1.869922in}}{\pgfqpoint{2.107436in}{1.877822in}}{\pgfqpoint{2.101612in}{1.883646in}}%
\pgfpathcurveto{\pgfqpoint{2.095788in}{1.889470in}}{\pgfqpoint{2.087888in}{1.892743in}}{\pgfqpoint{2.079652in}{1.892743in}}%
\pgfpathcurveto{\pgfqpoint{2.071415in}{1.892743in}}{\pgfqpoint{2.063515in}{1.889470in}}{\pgfqpoint{2.057691in}{1.883646in}}%
\pgfpathcurveto{\pgfqpoint{2.051867in}{1.877822in}}{\pgfqpoint{2.048595in}{1.869922in}}{\pgfqpoint{2.048595in}{1.861686in}}%
\pgfpathcurveto{\pgfqpoint{2.048595in}{1.853450in}}{\pgfqpoint{2.051867in}{1.845550in}}{\pgfqpoint{2.057691in}{1.839726in}}%
\pgfpathcurveto{\pgfqpoint{2.063515in}{1.833902in}}{\pgfqpoint{2.071415in}{1.830630in}}{\pgfqpoint{2.079652in}{1.830630in}}%
\pgfpathclose%
\pgfusepath{stroke,fill}%
\end{pgfscope}%
\begin{pgfscope}%
\pgfpathrectangle{\pgfqpoint{0.100000in}{0.212622in}}{\pgfqpoint{3.696000in}{3.696000in}}%
\pgfusepath{clip}%
\pgfsetbuttcap%
\pgfsetroundjoin%
\definecolor{currentfill}{rgb}{0.121569,0.466667,0.705882}%
\pgfsetfillcolor{currentfill}%
\pgfsetfillopacity{0.556394}%
\pgfsetlinewidth{1.003750pt}%
\definecolor{currentstroke}{rgb}{0.121569,0.466667,0.705882}%
\pgfsetstrokecolor{currentstroke}%
\pgfsetstrokeopacity{0.556394}%
\pgfsetdash{}{0pt}%
\pgfpathmoveto{\pgfqpoint{1.132686in}{1.617198in}}%
\pgfpathcurveto{\pgfqpoint{1.140922in}{1.617198in}}{\pgfqpoint{1.148822in}{1.620470in}}{\pgfqpoint{1.154646in}{1.626294in}}%
\pgfpathcurveto{\pgfqpoint{1.160470in}{1.632118in}}{\pgfqpoint{1.163743in}{1.640018in}}{\pgfqpoint{1.163743in}{1.648254in}}%
\pgfpathcurveto{\pgfqpoint{1.163743in}{1.656490in}}{\pgfqpoint{1.160470in}{1.664390in}}{\pgfqpoint{1.154646in}{1.670214in}}%
\pgfpathcurveto{\pgfqpoint{1.148822in}{1.676038in}}{\pgfqpoint{1.140922in}{1.679311in}}{\pgfqpoint{1.132686in}{1.679311in}}%
\pgfpathcurveto{\pgfqpoint{1.124450in}{1.679311in}}{\pgfqpoint{1.116550in}{1.676038in}}{\pgfqpoint{1.110726in}{1.670214in}}%
\pgfpathcurveto{\pgfqpoint{1.104902in}{1.664390in}}{\pgfqpoint{1.101630in}{1.656490in}}{\pgfqpoint{1.101630in}{1.648254in}}%
\pgfpathcurveto{\pgfqpoint{1.101630in}{1.640018in}}{\pgfqpoint{1.104902in}{1.632118in}}{\pgfqpoint{1.110726in}{1.626294in}}%
\pgfpathcurveto{\pgfqpoint{1.116550in}{1.620470in}}{\pgfqpoint{1.124450in}{1.617198in}}{\pgfqpoint{1.132686in}{1.617198in}}%
\pgfpathclose%
\pgfusepath{stroke,fill}%
\end{pgfscope}%
\begin{pgfscope}%
\pgfpathrectangle{\pgfqpoint{0.100000in}{0.212622in}}{\pgfqpoint{3.696000in}{3.696000in}}%
\pgfusepath{clip}%
\pgfsetbuttcap%
\pgfsetroundjoin%
\definecolor{currentfill}{rgb}{0.121569,0.466667,0.705882}%
\pgfsetfillcolor{currentfill}%
\pgfsetfillopacity{0.558664}%
\pgfsetlinewidth{1.003750pt}%
\definecolor{currentstroke}{rgb}{0.121569,0.466667,0.705882}%
\pgfsetstrokecolor{currentstroke}%
\pgfsetstrokeopacity{0.558664}%
\pgfsetdash{}{0pt}%
\pgfpathmoveto{\pgfqpoint{1.124330in}{1.612809in}}%
\pgfpathcurveto{\pgfqpoint{1.132566in}{1.612809in}}{\pgfqpoint{1.140466in}{1.616081in}}{\pgfqpoint{1.146290in}{1.621905in}}%
\pgfpathcurveto{\pgfqpoint{1.152114in}{1.627729in}}{\pgfqpoint{1.155387in}{1.635629in}}{\pgfqpoint{1.155387in}{1.643866in}}%
\pgfpathcurveto{\pgfqpoint{1.155387in}{1.652102in}}{\pgfqpoint{1.152114in}{1.660002in}}{\pgfqpoint{1.146290in}{1.665826in}}%
\pgfpathcurveto{\pgfqpoint{1.140466in}{1.671650in}}{\pgfqpoint{1.132566in}{1.674922in}}{\pgfqpoint{1.124330in}{1.674922in}}%
\pgfpathcurveto{\pgfqpoint{1.116094in}{1.674922in}}{\pgfqpoint{1.108194in}{1.671650in}}{\pgfqpoint{1.102370in}{1.665826in}}%
\pgfpathcurveto{\pgfqpoint{1.096546in}{1.660002in}}{\pgfqpoint{1.093274in}{1.652102in}}{\pgfqpoint{1.093274in}{1.643866in}}%
\pgfpathcurveto{\pgfqpoint{1.093274in}{1.635629in}}{\pgfqpoint{1.096546in}{1.627729in}}{\pgfqpoint{1.102370in}{1.621905in}}%
\pgfpathcurveto{\pgfqpoint{1.108194in}{1.616081in}}{\pgfqpoint{1.116094in}{1.612809in}}{\pgfqpoint{1.124330in}{1.612809in}}%
\pgfpathclose%
\pgfusepath{stroke,fill}%
\end{pgfscope}%
\begin{pgfscope}%
\pgfpathrectangle{\pgfqpoint{0.100000in}{0.212622in}}{\pgfqpoint{3.696000in}{3.696000in}}%
\pgfusepath{clip}%
\pgfsetbuttcap%
\pgfsetroundjoin%
\definecolor{currentfill}{rgb}{0.121569,0.466667,0.705882}%
\pgfsetfillcolor{currentfill}%
\pgfsetfillopacity{0.558853}%
\pgfsetlinewidth{1.003750pt}%
\definecolor{currentstroke}{rgb}{0.121569,0.466667,0.705882}%
\pgfsetstrokecolor{currentstroke}%
\pgfsetstrokeopacity{0.558853}%
\pgfsetdash{}{0pt}%
\pgfpathmoveto{\pgfqpoint{2.083238in}{1.827220in}}%
\pgfpathcurveto{\pgfqpoint{2.091474in}{1.827220in}}{\pgfqpoint{2.099374in}{1.830492in}}{\pgfqpoint{2.105198in}{1.836316in}}%
\pgfpathcurveto{\pgfqpoint{2.111022in}{1.842140in}}{\pgfqpoint{2.114294in}{1.850040in}}{\pgfqpoint{2.114294in}{1.858277in}}%
\pgfpathcurveto{\pgfqpoint{2.114294in}{1.866513in}}{\pgfqpoint{2.111022in}{1.874413in}}{\pgfqpoint{2.105198in}{1.880237in}}%
\pgfpathcurveto{\pgfqpoint{2.099374in}{1.886061in}}{\pgfqpoint{2.091474in}{1.889333in}}{\pgfqpoint{2.083238in}{1.889333in}}%
\pgfpathcurveto{\pgfqpoint{2.075002in}{1.889333in}}{\pgfqpoint{2.067102in}{1.886061in}}{\pgfqpoint{2.061278in}{1.880237in}}%
\pgfpathcurveto{\pgfqpoint{2.055454in}{1.874413in}}{\pgfqpoint{2.052181in}{1.866513in}}{\pgfqpoint{2.052181in}{1.858277in}}%
\pgfpathcurveto{\pgfqpoint{2.052181in}{1.850040in}}{\pgfqpoint{2.055454in}{1.842140in}}{\pgfqpoint{2.061278in}{1.836316in}}%
\pgfpathcurveto{\pgfqpoint{2.067102in}{1.830492in}}{\pgfqpoint{2.075002in}{1.827220in}}{\pgfqpoint{2.083238in}{1.827220in}}%
\pgfpathclose%
\pgfusepath{stroke,fill}%
\end{pgfscope}%
\begin{pgfscope}%
\pgfpathrectangle{\pgfqpoint{0.100000in}{0.212622in}}{\pgfqpoint{3.696000in}{3.696000in}}%
\pgfusepath{clip}%
\pgfsetbuttcap%
\pgfsetroundjoin%
\definecolor{currentfill}{rgb}{0.121569,0.466667,0.705882}%
\pgfsetfillcolor{currentfill}%
\pgfsetfillopacity{0.561683}%
\pgfsetlinewidth{1.003750pt}%
\definecolor{currentstroke}{rgb}{0.121569,0.466667,0.705882}%
\pgfsetstrokecolor{currentstroke}%
\pgfsetstrokeopacity{0.561683}%
\pgfsetdash{}{0pt}%
\pgfpathmoveto{\pgfqpoint{1.120203in}{1.614956in}}%
\pgfpathcurveto{\pgfqpoint{1.128439in}{1.614956in}}{\pgfqpoint{1.136339in}{1.618229in}}{\pgfqpoint{1.142163in}{1.624053in}}%
\pgfpathcurveto{\pgfqpoint{1.147987in}{1.629877in}}{\pgfqpoint{1.151260in}{1.637777in}}{\pgfqpoint{1.151260in}{1.646013in}}%
\pgfpathcurveto{\pgfqpoint{1.151260in}{1.654249in}}{\pgfqpoint{1.147987in}{1.662149in}}{\pgfqpoint{1.142163in}{1.667973in}}%
\pgfpathcurveto{\pgfqpoint{1.136339in}{1.673797in}}{\pgfqpoint{1.128439in}{1.677069in}}{\pgfqpoint{1.120203in}{1.677069in}}%
\pgfpathcurveto{\pgfqpoint{1.111967in}{1.677069in}}{\pgfqpoint{1.104067in}{1.673797in}}{\pgfqpoint{1.098243in}{1.667973in}}%
\pgfpathcurveto{\pgfqpoint{1.092419in}{1.662149in}}{\pgfqpoint{1.089147in}{1.654249in}}{\pgfqpoint{1.089147in}{1.646013in}}%
\pgfpathcurveto{\pgfqpoint{1.089147in}{1.637777in}}{\pgfqpoint{1.092419in}{1.629877in}}{\pgfqpoint{1.098243in}{1.624053in}}%
\pgfpathcurveto{\pgfqpoint{1.104067in}{1.618229in}}{\pgfqpoint{1.111967in}{1.614956in}}{\pgfqpoint{1.120203in}{1.614956in}}%
\pgfpathclose%
\pgfusepath{stroke,fill}%
\end{pgfscope}%
\begin{pgfscope}%
\pgfpathrectangle{\pgfqpoint{0.100000in}{0.212622in}}{\pgfqpoint{3.696000in}{3.696000in}}%
\pgfusepath{clip}%
\pgfsetbuttcap%
\pgfsetroundjoin%
\definecolor{currentfill}{rgb}{0.121569,0.466667,0.705882}%
\pgfsetfillcolor{currentfill}%
\pgfsetfillopacity{0.562942}%
\pgfsetlinewidth{1.003750pt}%
\definecolor{currentstroke}{rgb}{0.121569,0.466667,0.705882}%
\pgfsetstrokecolor{currentstroke}%
\pgfsetstrokeopacity{0.562942}%
\pgfsetdash{}{0pt}%
\pgfpathmoveto{\pgfqpoint{1.113944in}{1.610062in}}%
\pgfpathcurveto{\pgfqpoint{1.122180in}{1.610062in}}{\pgfqpoint{1.130080in}{1.613335in}}{\pgfqpoint{1.135904in}{1.619159in}}%
\pgfpathcurveto{\pgfqpoint{1.141728in}{1.624983in}}{\pgfqpoint{1.145001in}{1.632883in}}{\pgfqpoint{1.145001in}{1.641119in}}%
\pgfpathcurveto{\pgfqpoint{1.145001in}{1.649355in}}{\pgfqpoint{1.141728in}{1.657255in}}{\pgfqpoint{1.135904in}{1.663079in}}%
\pgfpathcurveto{\pgfqpoint{1.130080in}{1.668903in}}{\pgfqpoint{1.122180in}{1.672175in}}{\pgfqpoint{1.113944in}{1.672175in}}%
\pgfpathcurveto{\pgfqpoint{1.105708in}{1.672175in}}{\pgfqpoint{1.097808in}{1.668903in}}{\pgfqpoint{1.091984in}{1.663079in}}%
\pgfpathcurveto{\pgfqpoint{1.086160in}{1.657255in}}{\pgfqpoint{1.082888in}{1.649355in}}{\pgfqpoint{1.082888in}{1.641119in}}%
\pgfpathcurveto{\pgfqpoint{1.082888in}{1.632883in}}{\pgfqpoint{1.086160in}{1.624983in}}{\pgfqpoint{1.091984in}{1.619159in}}%
\pgfpathcurveto{\pgfqpoint{1.097808in}{1.613335in}}{\pgfqpoint{1.105708in}{1.610062in}}{\pgfqpoint{1.113944in}{1.610062in}}%
\pgfpathclose%
\pgfusepath{stroke,fill}%
\end{pgfscope}%
\begin{pgfscope}%
\pgfpathrectangle{\pgfqpoint{0.100000in}{0.212622in}}{\pgfqpoint{3.696000in}{3.696000in}}%
\pgfusepath{clip}%
\pgfsetbuttcap%
\pgfsetroundjoin%
\definecolor{currentfill}{rgb}{0.121569,0.466667,0.705882}%
\pgfsetfillcolor{currentfill}%
\pgfsetfillopacity{0.564165}%
\pgfsetlinewidth{1.003750pt}%
\definecolor{currentstroke}{rgb}{0.121569,0.466667,0.705882}%
\pgfsetstrokecolor{currentstroke}%
\pgfsetstrokeopacity{0.564165}%
\pgfsetdash{}{0pt}%
\pgfpathmoveto{\pgfqpoint{2.084304in}{1.823858in}}%
\pgfpathcurveto{\pgfqpoint{2.092541in}{1.823858in}}{\pgfqpoint{2.100441in}{1.827131in}}{\pgfqpoint{2.106265in}{1.832955in}}%
\pgfpathcurveto{\pgfqpoint{2.112089in}{1.838779in}}{\pgfqpoint{2.115361in}{1.846679in}}{\pgfqpoint{2.115361in}{1.854915in}}%
\pgfpathcurveto{\pgfqpoint{2.115361in}{1.863151in}}{\pgfqpoint{2.112089in}{1.871051in}}{\pgfqpoint{2.106265in}{1.876875in}}%
\pgfpathcurveto{\pgfqpoint{2.100441in}{1.882699in}}{\pgfqpoint{2.092541in}{1.885971in}}{\pgfqpoint{2.084304in}{1.885971in}}%
\pgfpathcurveto{\pgfqpoint{2.076068in}{1.885971in}}{\pgfqpoint{2.068168in}{1.882699in}}{\pgfqpoint{2.062344in}{1.876875in}}%
\pgfpathcurveto{\pgfqpoint{2.056520in}{1.871051in}}{\pgfqpoint{2.053248in}{1.863151in}}{\pgfqpoint{2.053248in}{1.854915in}}%
\pgfpathcurveto{\pgfqpoint{2.053248in}{1.846679in}}{\pgfqpoint{2.056520in}{1.838779in}}{\pgfqpoint{2.062344in}{1.832955in}}%
\pgfpathcurveto{\pgfqpoint{2.068168in}{1.827131in}}{\pgfqpoint{2.076068in}{1.823858in}}{\pgfqpoint{2.084304in}{1.823858in}}%
\pgfpathclose%
\pgfusepath{stroke,fill}%
\end{pgfscope}%
\begin{pgfscope}%
\pgfpathrectangle{\pgfqpoint{0.100000in}{0.212622in}}{\pgfqpoint{3.696000in}{3.696000in}}%
\pgfusepath{clip}%
\pgfsetbuttcap%
\pgfsetroundjoin%
\definecolor{currentfill}{rgb}{0.121569,0.466667,0.705882}%
\pgfsetfillcolor{currentfill}%
\pgfsetfillopacity{0.565604}%
\pgfsetlinewidth{1.003750pt}%
\definecolor{currentstroke}{rgb}{0.121569,0.466667,0.705882}%
\pgfsetstrokecolor{currentstroke}%
\pgfsetstrokeopacity{0.565604}%
\pgfsetdash{}{0pt}%
\pgfpathmoveto{\pgfqpoint{1.104023in}{1.600944in}}%
\pgfpathcurveto{\pgfqpoint{1.112259in}{1.600944in}}{\pgfqpoint{1.120159in}{1.604216in}}{\pgfqpoint{1.125983in}{1.610040in}}%
\pgfpathcurveto{\pgfqpoint{1.131807in}{1.615864in}}{\pgfqpoint{1.135079in}{1.623764in}}{\pgfqpoint{1.135079in}{1.632000in}}%
\pgfpathcurveto{\pgfqpoint{1.135079in}{1.640237in}}{\pgfqpoint{1.131807in}{1.648137in}}{\pgfqpoint{1.125983in}{1.653961in}}%
\pgfpathcurveto{\pgfqpoint{1.120159in}{1.659784in}}{\pgfqpoint{1.112259in}{1.663057in}}{\pgfqpoint{1.104023in}{1.663057in}}%
\pgfpathcurveto{\pgfqpoint{1.095787in}{1.663057in}}{\pgfqpoint{1.087887in}{1.659784in}}{\pgfqpoint{1.082063in}{1.653961in}}%
\pgfpathcurveto{\pgfqpoint{1.076239in}{1.648137in}}{\pgfqpoint{1.072966in}{1.640237in}}{\pgfqpoint{1.072966in}{1.632000in}}%
\pgfpathcurveto{\pgfqpoint{1.072966in}{1.623764in}}{\pgfqpoint{1.076239in}{1.615864in}}{\pgfqpoint{1.082063in}{1.610040in}}%
\pgfpathcurveto{\pgfqpoint{1.087887in}{1.604216in}}{\pgfqpoint{1.095787in}{1.600944in}}{\pgfqpoint{1.104023in}{1.600944in}}%
\pgfpathclose%
\pgfusepath{stroke,fill}%
\end{pgfscope}%
\begin{pgfscope}%
\pgfpathrectangle{\pgfqpoint{0.100000in}{0.212622in}}{\pgfqpoint{3.696000in}{3.696000in}}%
\pgfusepath{clip}%
\pgfsetbuttcap%
\pgfsetroundjoin%
\definecolor{currentfill}{rgb}{0.121569,0.466667,0.705882}%
\pgfsetfillcolor{currentfill}%
\pgfsetfillopacity{0.569427}%
\pgfsetlinewidth{1.003750pt}%
\definecolor{currentstroke}{rgb}{0.121569,0.466667,0.705882}%
\pgfsetstrokecolor{currentstroke}%
\pgfsetstrokeopacity{0.569427}%
\pgfsetdash{}{0pt}%
\pgfpathmoveto{\pgfqpoint{1.096668in}{1.600859in}}%
\pgfpathcurveto{\pgfqpoint{1.104904in}{1.600859in}}{\pgfqpoint{1.112804in}{1.604132in}}{\pgfqpoint{1.118628in}{1.609955in}}%
\pgfpathcurveto{\pgfqpoint{1.124452in}{1.615779in}}{\pgfqpoint{1.127725in}{1.623679in}}{\pgfqpoint{1.127725in}{1.631916in}}%
\pgfpathcurveto{\pgfqpoint{1.127725in}{1.640152in}}{\pgfqpoint{1.124452in}{1.648052in}}{\pgfqpoint{1.118628in}{1.653876in}}%
\pgfpathcurveto{\pgfqpoint{1.112804in}{1.659700in}}{\pgfqpoint{1.104904in}{1.662972in}}{\pgfqpoint{1.096668in}{1.662972in}}%
\pgfpathcurveto{\pgfqpoint{1.088432in}{1.662972in}}{\pgfqpoint{1.080532in}{1.659700in}}{\pgfqpoint{1.074708in}{1.653876in}}%
\pgfpathcurveto{\pgfqpoint{1.068884in}{1.648052in}}{\pgfqpoint{1.065612in}{1.640152in}}{\pgfqpoint{1.065612in}{1.631916in}}%
\pgfpathcurveto{\pgfqpoint{1.065612in}{1.623679in}}{\pgfqpoint{1.068884in}{1.615779in}}{\pgfqpoint{1.074708in}{1.609955in}}%
\pgfpathcurveto{\pgfqpoint{1.080532in}{1.604132in}}{\pgfqpoint{1.088432in}{1.600859in}}{\pgfqpoint{1.096668in}{1.600859in}}%
\pgfpathclose%
\pgfusepath{stroke,fill}%
\end{pgfscope}%
\begin{pgfscope}%
\pgfpathrectangle{\pgfqpoint{0.100000in}{0.212622in}}{\pgfqpoint{3.696000in}{3.696000in}}%
\pgfusepath{clip}%
\pgfsetbuttcap%
\pgfsetroundjoin%
\definecolor{currentfill}{rgb}{0.121569,0.466667,0.705882}%
\pgfsetfillcolor{currentfill}%
\pgfsetfillopacity{0.569998}%
\pgfsetlinewidth{1.003750pt}%
\definecolor{currentstroke}{rgb}{0.121569,0.466667,0.705882}%
\pgfsetstrokecolor{currentstroke}%
\pgfsetstrokeopacity{0.569998}%
\pgfsetdash{}{0pt}%
\pgfpathmoveto{\pgfqpoint{2.088749in}{1.819116in}}%
\pgfpathcurveto{\pgfqpoint{2.096985in}{1.819116in}}{\pgfqpoint{2.104885in}{1.822388in}}{\pgfqpoint{2.110709in}{1.828212in}}%
\pgfpathcurveto{\pgfqpoint{2.116533in}{1.834036in}}{\pgfqpoint{2.119806in}{1.841936in}}{\pgfqpoint{2.119806in}{1.850172in}}%
\pgfpathcurveto{\pgfqpoint{2.119806in}{1.858408in}}{\pgfqpoint{2.116533in}{1.866308in}}{\pgfqpoint{2.110709in}{1.872132in}}%
\pgfpathcurveto{\pgfqpoint{2.104885in}{1.877956in}}{\pgfqpoint{2.096985in}{1.881229in}}{\pgfqpoint{2.088749in}{1.881229in}}%
\pgfpathcurveto{\pgfqpoint{2.080513in}{1.881229in}}{\pgfqpoint{2.072613in}{1.877956in}}{\pgfqpoint{2.066789in}{1.872132in}}%
\pgfpathcurveto{\pgfqpoint{2.060965in}{1.866308in}}{\pgfqpoint{2.057693in}{1.858408in}}{\pgfqpoint{2.057693in}{1.850172in}}%
\pgfpathcurveto{\pgfqpoint{2.057693in}{1.841936in}}{\pgfqpoint{2.060965in}{1.834036in}}{\pgfqpoint{2.066789in}{1.828212in}}%
\pgfpathcurveto{\pgfqpoint{2.072613in}{1.822388in}}{\pgfqpoint{2.080513in}{1.819116in}}{\pgfqpoint{2.088749in}{1.819116in}}%
\pgfpathclose%
\pgfusepath{stroke,fill}%
\end{pgfscope}%
\begin{pgfscope}%
\pgfpathrectangle{\pgfqpoint{0.100000in}{0.212622in}}{\pgfqpoint{3.696000in}{3.696000in}}%
\pgfusepath{clip}%
\pgfsetbuttcap%
\pgfsetroundjoin%
\definecolor{currentfill}{rgb}{0.121569,0.466667,0.705882}%
\pgfsetfillcolor{currentfill}%
\pgfsetfillopacity{0.570652}%
\pgfsetlinewidth{1.003750pt}%
\definecolor{currentstroke}{rgb}{0.121569,0.466667,0.705882}%
\pgfsetstrokecolor{currentstroke}%
\pgfsetstrokeopacity{0.570652}%
\pgfsetdash{}{0pt}%
\pgfpathmoveto{\pgfqpoint{1.087622in}{1.594406in}}%
\pgfpathcurveto{\pgfqpoint{1.095858in}{1.594406in}}{\pgfqpoint{1.103758in}{1.597679in}}{\pgfqpoint{1.109582in}{1.603503in}}%
\pgfpathcurveto{\pgfqpoint{1.115406in}{1.609327in}}{\pgfqpoint{1.118678in}{1.617227in}}{\pgfqpoint{1.118678in}{1.625463in}}%
\pgfpathcurveto{\pgfqpoint{1.118678in}{1.633699in}}{\pgfqpoint{1.115406in}{1.641599in}}{\pgfqpoint{1.109582in}{1.647423in}}%
\pgfpathcurveto{\pgfqpoint{1.103758in}{1.653247in}}{\pgfqpoint{1.095858in}{1.656519in}}{\pgfqpoint{1.087622in}{1.656519in}}%
\pgfpathcurveto{\pgfqpoint{1.079385in}{1.656519in}}{\pgfqpoint{1.071485in}{1.653247in}}{\pgfqpoint{1.065661in}{1.647423in}}%
\pgfpathcurveto{\pgfqpoint{1.059837in}{1.641599in}}{\pgfqpoint{1.056565in}{1.633699in}}{\pgfqpoint{1.056565in}{1.625463in}}%
\pgfpathcurveto{\pgfqpoint{1.056565in}{1.617227in}}{\pgfqpoint{1.059837in}{1.609327in}}{\pgfqpoint{1.065661in}{1.603503in}}%
\pgfpathcurveto{\pgfqpoint{1.071485in}{1.597679in}}{\pgfqpoint{1.079385in}{1.594406in}}{\pgfqpoint{1.087622in}{1.594406in}}%
\pgfpathclose%
\pgfusepath{stroke,fill}%
\end{pgfscope}%
\begin{pgfscope}%
\pgfpathrectangle{\pgfqpoint{0.100000in}{0.212622in}}{\pgfqpoint{3.696000in}{3.696000in}}%
\pgfusepath{clip}%
\pgfsetbuttcap%
\pgfsetroundjoin%
\definecolor{currentfill}{rgb}{0.121569,0.466667,0.705882}%
\pgfsetfillcolor{currentfill}%
\pgfsetfillopacity{0.570978}%
\pgfsetlinewidth{1.003750pt}%
\definecolor{currentstroke}{rgb}{0.121569,0.466667,0.705882}%
\pgfsetstrokecolor{currentstroke}%
\pgfsetstrokeopacity{0.570978}%
\pgfsetdash{}{0pt}%
\pgfpathmoveto{\pgfqpoint{1.079664in}{1.581564in}}%
\pgfpathcurveto{\pgfqpoint{1.087901in}{1.581564in}}{\pgfqpoint{1.095801in}{1.584836in}}{\pgfqpoint{1.101625in}{1.590660in}}%
\pgfpathcurveto{\pgfqpoint{1.107449in}{1.596484in}}{\pgfqpoint{1.110721in}{1.604384in}}{\pgfqpoint{1.110721in}{1.612620in}}%
\pgfpathcurveto{\pgfqpoint{1.110721in}{1.620857in}}{\pgfqpoint{1.107449in}{1.628757in}}{\pgfqpoint{1.101625in}{1.634581in}}%
\pgfpathcurveto{\pgfqpoint{1.095801in}{1.640405in}}{\pgfqpoint{1.087901in}{1.643677in}}{\pgfqpoint{1.079664in}{1.643677in}}%
\pgfpathcurveto{\pgfqpoint{1.071428in}{1.643677in}}{\pgfqpoint{1.063528in}{1.640405in}}{\pgfqpoint{1.057704in}{1.634581in}}%
\pgfpathcurveto{\pgfqpoint{1.051880in}{1.628757in}}{\pgfqpoint{1.048608in}{1.620857in}}{\pgfqpoint{1.048608in}{1.612620in}}%
\pgfpathcurveto{\pgfqpoint{1.048608in}{1.604384in}}{\pgfqpoint{1.051880in}{1.596484in}}{\pgfqpoint{1.057704in}{1.590660in}}%
\pgfpathcurveto{\pgfqpoint{1.063528in}{1.584836in}}{\pgfqpoint{1.071428in}{1.581564in}}{\pgfqpoint{1.079664in}{1.581564in}}%
\pgfpathclose%
\pgfusepath{stroke,fill}%
\end{pgfscope}%
\begin{pgfscope}%
\pgfpathrectangle{\pgfqpoint{0.100000in}{0.212622in}}{\pgfqpoint{3.696000in}{3.696000in}}%
\pgfusepath{clip}%
\pgfsetbuttcap%
\pgfsetroundjoin%
\definecolor{currentfill}{rgb}{0.121569,0.466667,0.705882}%
\pgfsetfillcolor{currentfill}%
\pgfsetfillopacity{0.573191}%
\pgfsetlinewidth{1.003750pt}%
\definecolor{currentstroke}{rgb}{0.121569,0.466667,0.705882}%
\pgfsetstrokecolor{currentstroke}%
\pgfsetstrokeopacity{0.573191}%
\pgfsetdash{}{0pt}%
\pgfpathmoveto{\pgfqpoint{2.090641in}{1.816067in}}%
\pgfpathcurveto{\pgfqpoint{2.098877in}{1.816067in}}{\pgfqpoint{2.106777in}{1.819340in}}{\pgfqpoint{2.112601in}{1.825164in}}%
\pgfpathcurveto{\pgfqpoint{2.118425in}{1.830988in}}{\pgfqpoint{2.121697in}{1.838888in}}{\pgfqpoint{2.121697in}{1.847124in}}%
\pgfpathcurveto{\pgfqpoint{2.121697in}{1.855360in}}{\pgfqpoint{2.118425in}{1.863260in}}{\pgfqpoint{2.112601in}{1.869084in}}%
\pgfpathcurveto{\pgfqpoint{2.106777in}{1.874908in}}{\pgfqpoint{2.098877in}{1.878180in}}{\pgfqpoint{2.090641in}{1.878180in}}%
\pgfpathcurveto{\pgfqpoint{2.082405in}{1.878180in}}{\pgfqpoint{2.074505in}{1.874908in}}{\pgfqpoint{2.068681in}{1.869084in}}%
\pgfpathcurveto{\pgfqpoint{2.062857in}{1.863260in}}{\pgfqpoint{2.059584in}{1.855360in}}{\pgfqpoint{2.059584in}{1.847124in}}%
\pgfpathcurveto{\pgfqpoint{2.059584in}{1.838888in}}{\pgfqpoint{2.062857in}{1.830988in}}{\pgfqpoint{2.068681in}{1.825164in}}%
\pgfpathcurveto{\pgfqpoint{2.074505in}{1.819340in}}{\pgfqpoint{2.082405in}{1.816067in}}{\pgfqpoint{2.090641in}{1.816067in}}%
\pgfpathclose%
\pgfusepath{stroke,fill}%
\end{pgfscope}%
\begin{pgfscope}%
\pgfpathrectangle{\pgfqpoint{0.100000in}{0.212622in}}{\pgfqpoint{3.696000in}{3.696000in}}%
\pgfusepath{clip}%
\pgfsetbuttcap%
\pgfsetroundjoin%
\definecolor{currentfill}{rgb}{0.121569,0.466667,0.705882}%
\pgfsetfillcolor{currentfill}%
\pgfsetfillopacity{0.573281}%
\pgfsetlinewidth{1.003750pt}%
\definecolor{currentstroke}{rgb}{0.121569,0.466667,0.705882}%
\pgfsetstrokecolor{currentstroke}%
\pgfsetstrokeopacity{0.573281}%
\pgfsetdash{}{0pt}%
\pgfpathmoveto{\pgfqpoint{1.074572in}{1.579376in}}%
\pgfpathcurveto{\pgfqpoint{1.082808in}{1.579376in}}{\pgfqpoint{1.090708in}{1.582648in}}{\pgfqpoint{1.096532in}{1.588472in}}%
\pgfpathcurveto{\pgfqpoint{1.102356in}{1.594296in}}{\pgfqpoint{1.105628in}{1.602196in}}{\pgfqpoint{1.105628in}{1.610432in}}%
\pgfpathcurveto{\pgfqpoint{1.105628in}{1.618669in}}{\pgfqpoint{1.102356in}{1.626569in}}{\pgfqpoint{1.096532in}{1.632393in}}%
\pgfpathcurveto{\pgfqpoint{1.090708in}{1.638216in}}{\pgfqpoint{1.082808in}{1.641489in}}{\pgfqpoint{1.074572in}{1.641489in}}%
\pgfpathcurveto{\pgfqpoint{1.066335in}{1.641489in}}{\pgfqpoint{1.058435in}{1.638216in}}{\pgfqpoint{1.052611in}{1.632393in}}%
\pgfpathcurveto{\pgfqpoint{1.046787in}{1.626569in}}{\pgfqpoint{1.043515in}{1.618669in}}{\pgfqpoint{1.043515in}{1.610432in}}%
\pgfpathcurveto{\pgfqpoint{1.043515in}{1.602196in}}{\pgfqpoint{1.046787in}{1.594296in}}{\pgfqpoint{1.052611in}{1.588472in}}%
\pgfpathcurveto{\pgfqpoint{1.058435in}{1.582648in}}{\pgfqpoint{1.066335in}{1.579376in}}{\pgfqpoint{1.074572in}{1.579376in}}%
\pgfpathclose%
\pgfusepath{stroke,fill}%
\end{pgfscope}%
\begin{pgfscope}%
\pgfpathrectangle{\pgfqpoint{0.100000in}{0.212622in}}{\pgfqpoint{3.696000in}{3.696000in}}%
\pgfusepath{clip}%
\pgfsetbuttcap%
\pgfsetroundjoin%
\definecolor{currentfill}{rgb}{0.121569,0.466667,0.705882}%
\pgfsetfillcolor{currentfill}%
\pgfsetfillopacity{0.574192}%
\pgfsetlinewidth{1.003750pt}%
\definecolor{currentstroke}{rgb}{0.121569,0.466667,0.705882}%
\pgfsetstrokecolor{currentstroke}%
\pgfsetstrokeopacity{0.574192}%
\pgfsetdash{}{0pt}%
\pgfpathmoveto{\pgfqpoint{1.070147in}{1.574957in}}%
\pgfpathcurveto{\pgfqpoint{1.078383in}{1.574957in}}{\pgfqpoint{1.086283in}{1.578229in}}{\pgfqpoint{1.092107in}{1.584053in}}%
\pgfpathcurveto{\pgfqpoint{1.097931in}{1.589877in}}{\pgfqpoint{1.101203in}{1.597777in}}{\pgfqpoint{1.101203in}{1.606013in}}%
\pgfpathcurveto{\pgfqpoint{1.101203in}{1.614249in}}{\pgfqpoint{1.097931in}{1.622149in}}{\pgfqpoint{1.092107in}{1.627973in}}%
\pgfpathcurveto{\pgfqpoint{1.086283in}{1.633797in}}{\pgfqpoint{1.078383in}{1.637070in}}{\pgfqpoint{1.070147in}{1.637070in}}%
\pgfpathcurveto{\pgfqpoint{1.061911in}{1.637070in}}{\pgfqpoint{1.054010in}{1.633797in}}{\pgfqpoint{1.048187in}{1.627973in}}%
\pgfpathcurveto{\pgfqpoint{1.042363in}{1.622149in}}{\pgfqpoint{1.039090in}{1.614249in}}{\pgfqpoint{1.039090in}{1.606013in}}%
\pgfpathcurveto{\pgfqpoint{1.039090in}{1.597777in}}{\pgfqpoint{1.042363in}{1.589877in}}{\pgfqpoint{1.048187in}{1.584053in}}%
\pgfpathcurveto{\pgfqpoint{1.054010in}{1.578229in}}{\pgfqpoint{1.061911in}{1.574957in}}{\pgfqpoint{1.070147in}{1.574957in}}%
\pgfpathclose%
\pgfusepath{stroke,fill}%
\end{pgfscope}%
\begin{pgfscope}%
\pgfpathrectangle{\pgfqpoint{0.100000in}{0.212622in}}{\pgfqpoint{3.696000in}{3.696000in}}%
\pgfusepath{clip}%
\pgfsetbuttcap%
\pgfsetroundjoin%
\definecolor{currentfill}{rgb}{0.121569,0.466667,0.705882}%
\pgfsetfillcolor{currentfill}%
\pgfsetfillopacity{0.576011}%
\pgfsetlinewidth{1.003750pt}%
\definecolor{currentstroke}{rgb}{0.121569,0.466667,0.705882}%
\pgfsetstrokecolor{currentstroke}%
\pgfsetstrokeopacity{0.576011}%
\pgfsetdash{}{0pt}%
\pgfpathmoveto{\pgfqpoint{1.062834in}{1.566761in}}%
\pgfpathcurveto{\pgfqpoint{1.071071in}{1.566761in}}{\pgfqpoint{1.078971in}{1.570033in}}{\pgfqpoint{1.084794in}{1.575857in}}%
\pgfpathcurveto{\pgfqpoint{1.090618in}{1.581681in}}{\pgfqpoint{1.093891in}{1.589581in}}{\pgfqpoint{1.093891in}{1.597818in}}%
\pgfpathcurveto{\pgfqpoint{1.093891in}{1.606054in}}{\pgfqpoint{1.090618in}{1.613954in}}{\pgfqpoint{1.084794in}{1.619778in}}%
\pgfpathcurveto{\pgfqpoint{1.078971in}{1.625602in}}{\pgfqpoint{1.071071in}{1.628874in}}{\pgfqpoint{1.062834in}{1.628874in}}%
\pgfpathcurveto{\pgfqpoint{1.054598in}{1.628874in}}{\pgfqpoint{1.046698in}{1.625602in}}{\pgfqpoint{1.040874in}{1.619778in}}%
\pgfpathcurveto{\pgfqpoint{1.035050in}{1.613954in}}{\pgfqpoint{1.031778in}{1.606054in}}{\pgfqpoint{1.031778in}{1.597818in}}%
\pgfpathcurveto{\pgfqpoint{1.031778in}{1.589581in}}{\pgfqpoint{1.035050in}{1.581681in}}{\pgfqpoint{1.040874in}{1.575857in}}%
\pgfpathcurveto{\pgfqpoint{1.046698in}{1.570033in}}{\pgfqpoint{1.054598in}{1.566761in}}{\pgfqpoint{1.062834in}{1.566761in}}%
\pgfpathclose%
\pgfusepath{stroke,fill}%
\end{pgfscope}%
\begin{pgfscope}%
\pgfpathrectangle{\pgfqpoint{0.100000in}{0.212622in}}{\pgfqpoint{3.696000in}{3.696000in}}%
\pgfusepath{clip}%
\pgfsetbuttcap%
\pgfsetroundjoin%
\definecolor{currentfill}{rgb}{0.121569,0.466667,0.705882}%
\pgfsetfillcolor{currentfill}%
\pgfsetfillopacity{0.577196}%
\pgfsetlinewidth{1.003750pt}%
\definecolor{currentstroke}{rgb}{0.121569,0.466667,0.705882}%
\pgfsetstrokecolor{currentstroke}%
\pgfsetstrokeopacity{0.577196}%
\pgfsetdash{}{0pt}%
\pgfpathmoveto{\pgfqpoint{2.093075in}{1.813828in}}%
\pgfpathcurveto{\pgfqpoint{2.101312in}{1.813828in}}{\pgfqpoint{2.109212in}{1.817100in}}{\pgfqpoint{2.115036in}{1.822924in}}%
\pgfpathcurveto{\pgfqpoint{2.120859in}{1.828748in}}{\pgfqpoint{2.124132in}{1.836648in}}{\pgfqpoint{2.124132in}{1.844885in}}%
\pgfpathcurveto{\pgfqpoint{2.124132in}{1.853121in}}{\pgfqpoint{2.120859in}{1.861021in}}{\pgfqpoint{2.115036in}{1.866845in}}%
\pgfpathcurveto{\pgfqpoint{2.109212in}{1.872669in}}{\pgfqpoint{2.101312in}{1.875941in}}{\pgfqpoint{2.093075in}{1.875941in}}%
\pgfpathcurveto{\pgfqpoint{2.084839in}{1.875941in}}{\pgfqpoint{2.076939in}{1.872669in}}{\pgfqpoint{2.071115in}{1.866845in}}%
\pgfpathcurveto{\pgfqpoint{2.065291in}{1.861021in}}{\pgfqpoint{2.062019in}{1.853121in}}{\pgfqpoint{2.062019in}{1.844885in}}%
\pgfpathcurveto{\pgfqpoint{2.062019in}{1.836648in}}{\pgfqpoint{2.065291in}{1.828748in}}{\pgfqpoint{2.071115in}{1.822924in}}%
\pgfpathcurveto{\pgfqpoint{2.076939in}{1.817100in}}{\pgfqpoint{2.084839in}{1.813828in}}{\pgfqpoint{2.093075in}{1.813828in}}%
\pgfpathclose%
\pgfusepath{stroke,fill}%
\end{pgfscope}%
\begin{pgfscope}%
\pgfpathrectangle{\pgfqpoint{0.100000in}{0.212622in}}{\pgfqpoint{3.696000in}{3.696000in}}%
\pgfusepath{clip}%
\pgfsetbuttcap%
\pgfsetroundjoin%
\definecolor{currentfill}{rgb}{0.121569,0.466667,0.705882}%
\pgfsetfillcolor{currentfill}%
\pgfsetfillopacity{0.579310}%
\pgfsetlinewidth{1.003750pt}%
\definecolor{currentstroke}{rgb}{0.121569,0.466667,0.705882}%
\pgfsetstrokecolor{currentstroke}%
\pgfsetstrokeopacity{0.579310}%
\pgfsetdash{}{0pt}%
\pgfpathmoveto{\pgfqpoint{2.094220in}{1.811894in}}%
\pgfpathcurveto{\pgfqpoint{2.102457in}{1.811894in}}{\pgfqpoint{2.110357in}{1.815166in}}{\pgfqpoint{2.116181in}{1.820990in}}%
\pgfpathcurveto{\pgfqpoint{2.122005in}{1.826814in}}{\pgfqpoint{2.125277in}{1.834714in}}{\pgfqpoint{2.125277in}{1.842951in}}%
\pgfpathcurveto{\pgfqpoint{2.125277in}{1.851187in}}{\pgfqpoint{2.122005in}{1.859087in}}{\pgfqpoint{2.116181in}{1.864911in}}%
\pgfpathcurveto{\pgfqpoint{2.110357in}{1.870735in}}{\pgfqpoint{2.102457in}{1.874007in}}{\pgfqpoint{2.094220in}{1.874007in}}%
\pgfpathcurveto{\pgfqpoint{2.085984in}{1.874007in}}{\pgfqpoint{2.078084in}{1.870735in}}{\pgfqpoint{2.072260in}{1.864911in}}%
\pgfpathcurveto{\pgfqpoint{2.066436in}{1.859087in}}{\pgfqpoint{2.063164in}{1.851187in}}{\pgfqpoint{2.063164in}{1.842951in}}%
\pgfpathcurveto{\pgfqpoint{2.063164in}{1.834714in}}{\pgfqpoint{2.066436in}{1.826814in}}{\pgfqpoint{2.072260in}{1.820990in}}%
\pgfpathcurveto{\pgfqpoint{2.078084in}{1.815166in}}{\pgfqpoint{2.085984in}{1.811894in}}{\pgfqpoint{2.094220in}{1.811894in}}%
\pgfpathclose%
\pgfusepath{stroke,fill}%
\end{pgfscope}%
\begin{pgfscope}%
\pgfpathrectangle{\pgfqpoint{0.100000in}{0.212622in}}{\pgfqpoint{3.696000in}{3.696000in}}%
\pgfusepath{clip}%
\pgfsetbuttcap%
\pgfsetroundjoin%
\definecolor{currentfill}{rgb}{0.121569,0.466667,0.705882}%
\pgfsetfillcolor{currentfill}%
\pgfsetfillopacity{0.580420}%
\pgfsetlinewidth{1.003750pt}%
\definecolor{currentstroke}{rgb}{0.121569,0.466667,0.705882}%
\pgfsetstrokecolor{currentstroke}%
\pgfsetstrokeopacity{0.580420}%
\pgfsetdash{}{0pt}%
\pgfpathmoveto{\pgfqpoint{2.094828in}{1.810475in}}%
\pgfpathcurveto{\pgfqpoint{2.103064in}{1.810475in}}{\pgfqpoint{2.110964in}{1.813748in}}{\pgfqpoint{2.116788in}{1.819571in}}%
\pgfpathcurveto{\pgfqpoint{2.122612in}{1.825395in}}{\pgfqpoint{2.125884in}{1.833295in}}{\pgfqpoint{2.125884in}{1.841532in}}%
\pgfpathcurveto{\pgfqpoint{2.125884in}{1.849768in}}{\pgfqpoint{2.122612in}{1.857668in}}{\pgfqpoint{2.116788in}{1.863492in}}%
\pgfpathcurveto{\pgfqpoint{2.110964in}{1.869316in}}{\pgfqpoint{2.103064in}{1.872588in}}{\pgfqpoint{2.094828in}{1.872588in}}%
\pgfpathcurveto{\pgfqpoint{2.086591in}{1.872588in}}{\pgfqpoint{2.078691in}{1.869316in}}{\pgfqpoint{2.072867in}{1.863492in}}%
\pgfpathcurveto{\pgfqpoint{2.067043in}{1.857668in}}{\pgfqpoint{2.063771in}{1.849768in}}{\pgfqpoint{2.063771in}{1.841532in}}%
\pgfpathcurveto{\pgfqpoint{2.063771in}{1.833295in}}{\pgfqpoint{2.067043in}{1.825395in}}{\pgfqpoint{2.072867in}{1.819571in}}%
\pgfpathcurveto{\pgfqpoint{2.078691in}{1.813748in}}{\pgfqpoint{2.086591in}{1.810475in}}{\pgfqpoint{2.094828in}{1.810475in}}%
\pgfpathclose%
\pgfusepath{stroke,fill}%
\end{pgfscope}%
\begin{pgfscope}%
\pgfpathrectangle{\pgfqpoint{0.100000in}{0.212622in}}{\pgfqpoint{3.696000in}{3.696000in}}%
\pgfusepath{clip}%
\pgfsetbuttcap%
\pgfsetroundjoin%
\definecolor{currentfill}{rgb}{0.121569,0.466667,0.705882}%
\pgfsetfillcolor{currentfill}%
\pgfsetfillopacity{0.581183}%
\pgfsetlinewidth{1.003750pt}%
\definecolor{currentstroke}{rgb}{0.121569,0.466667,0.705882}%
\pgfsetstrokecolor{currentstroke}%
\pgfsetstrokeopacity{0.581183}%
\pgfsetdash{}{0pt}%
\pgfpathmoveto{\pgfqpoint{1.052356in}{1.560331in}}%
\pgfpathcurveto{\pgfqpoint{1.060592in}{1.560331in}}{\pgfqpoint{1.068492in}{1.563604in}}{\pgfqpoint{1.074316in}{1.569427in}}%
\pgfpathcurveto{\pgfqpoint{1.080140in}{1.575251in}}{\pgfqpoint{1.083412in}{1.583151in}}{\pgfqpoint{1.083412in}{1.591388in}}%
\pgfpathcurveto{\pgfqpoint{1.083412in}{1.599624in}}{\pgfqpoint{1.080140in}{1.607524in}}{\pgfqpoint{1.074316in}{1.613348in}}%
\pgfpathcurveto{\pgfqpoint{1.068492in}{1.619172in}}{\pgfqpoint{1.060592in}{1.622444in}}{\pgfqpoint{1.052356in}{1.622444in}}%
\pgfpathcurveto{\pgfqpoint{1.044120in}{1.622444in}}{\pgfqpoint{1.036219in}{1.619172in}}{\pgfqpoint{1.030396in}{1.613348in}}%
\pgfpathcurveto{\pgfqpoint{1.024572in}{1.607524in}}{\pgfqpoint{1.021299in}{1.599624in}}{\pgfqpoint{1.021299in}{1.591388in}}%
\pgfpathcurveto{\pgfqpoint{1.021299in}{1.583151in}}{\pgfqpoint{1.024572in}{1.575251in}}{\pgfqpoint{1.030396in}{1.569427in}}%
\pgfpathcurveto{\pgfqpoint{1.036219in}{1.563604in}}{\pgfqpoint{1.044120in}{1.560331in}}{\pgfqpoint{1.052356in}{1.560331in}}%
\pgfpathclose%
\pgfusepath{stroke,fill}%
\end{pgfscope}%
\begin{pgfscope}%
\pgfpathrectangle{\pgfqpoint{0.100000in}{0.212622in}}{\pgfqpoint{3.696000in}{3.696000in}}%
\pgfusepath{clip}%
\pgfsetbuttcap%
\pgfsetroundjoin%
\definecolor{currentfill}{rgb}{0.121569,0.466667,0.705882}%
\pgfsetfillcolor{currentfill}%
\pgfsetfillopacity{0.581717}%
\pgfsetlinewidth{1.003750pt}%
\definecolor{currentstroke}{rgb}{0.121569,0.466667,0.705882}%
\pgfsetstrokecolor{currentstroke}%
\pgfsetstrokeopacity{0.581717}%
\pgfsetdash{}{0pt}%
\pgfpathmoveto{\pgfqpoint{2.096143in}{1.809126in}}%
\pgfpathcurveto{\pgfqpoint{2.104379in}{1.809126in}}{\pgfqpoint{2.112279in}{1.812399in}}{\pgfqpoint{2.118103in}{1.818223in}}%
\pgfpathcurveto{\pgfqpoint{2.123927in}{1.824046in}}{\pgfqpoint{2.127199in}{1.831946in}}{\pgfqpoint{2.127199in}{1.840183in}}%
\pgfpathcurveto{\pgfqpoint{2.127199in}{1.848419in}}{\pgfqpoint{2.123927in}{1.856319in}}{\pgfqpoint{2.118103in}{1.862143in}}%
\pgfpathcurveto{\pgfqpoint{2.112279in}{1.867967in}}{\pgfqpoint{2.104379in}{1.871239in}}{\pgfqpoint{2.096143in}{1.871239in}}%
\pgfpathcurveto{\pgfqpoint{2.087906in}{1.871239in}}{\pgfqpoint{2.080006in}{1.867967in}}{\pgfqpoint{2.074182in}{1.862143in}}%
\pgfpathcurveto{\pgfqpoint{2.068358in}{1.856319in}}{\pgfqpoint{2.065086in}{1.848419in}}{\pgfqpoint{2.065086in}{1.840183in}}%
\pgfpathcurveto{\pgfqpoint{2.065086in}{1.831946in}}{\pgfqpoint{2.068358in}{1.824046in}}{\pgfqpoint{2.074182in}{1.818223in}}%
\pgfpathcurveto{\pgfqpoint{2.080006in}{1.812399in}}{\pgfqpoint{2.087906in}{1.809126in}}{\pgfqpoint{2.096143in}{1.809126in}}%
\pgfpathclose%
\pgfusepath{stroke,fill}%
\end{pgfscope}%
\begin{pgfscope}%
\pgfpathrectangle{\pgfqpoint{0.100000in}{0.212622in}}{\pgfqpoint{3.696000in}{3.696000in}}%
\pgfusepath{clip}%
\pgfsetbuttcap%
\pgfsetroundjoin%
\definecolor{currentfill}{rgb}{0.121569,0.466667,0.705882}%
\pgfsetfillcolor{currentfill}%
\pgfsetfillopacity{0.583595}%
\pgfsetlinewidth{1.003750pt}%
\definecolor{currentstroke}{rgb}{0.121569,0.466667,0.705882}%
\pgfsetstrokecolor{currentstroke}%
\pgfsetstrokeopacity{0.583595}%
\pgfsetdash{}{0pt}%
\pgfpathmoveto{\pgfqpoint{2.096511in}{1.807377in}}%
\pgfpathcurveto{\pgfqpoint{2.104747in}{1.807377in}}{\pgfqpoint{2.112647in}{1.810649in}}{\pgfqpoint{2.118471in}{1.816473in}}%
\pgfpathcurveto{\pgfqpoint{2.124295in}{1.822297in}}{\pgfqpoint{2.127567in}{1.830197in}}{\pgfqpoint{2.127567in}{1.838434in}}%
\pgfpathcurveto{\pgfqpoint{2.127567in}{1.846670in}}{\pgfqpoint{2.124295in}{1.854570in}}{\pgfqpoint{2.118471in}{1.860394in}}%
\pgfpathcurveto{\pgfqpoint{2.112647in}{1.866218in}}{\pgfqpoint{2.104747in}{1.869490in}}{\pgfqpoint{2.096511in}{1.869490in}}%
\pgfpathcurveto{\pgfqpoint{2.088274in}{1.869490in}}{\pgfqpoint{2.080374in}{1.866218in}}{\pgfqpoint{2.074550in}{1.860394in}}%
\pgfpathcurveto{\pgfqpoint{2.068726in}{1.854570in}}{\pgfqpoint{2.065454in}{1.846670in}}{\pgfqpoint{2.065454in}{1.838434in}}%
\pgfpathcurveto{\pgfqpoint{2.065454in}{1.830197in}}{\pgfqpoint{2.068726in}{1.822297in}}{\pgfqpoint{2.074550in}{1.816473in}}%
\pgfpathcurveto{\pgfqpoint{2.080374in}{1.810649in}}{\pgfqpoint{2.088274in}{1.807377in}}{\pgfqpoint{2.096511in}{1.807377in}}%
\pgfpathclose%
\pgfusepath{stroke,fill}%
\end{pgfscope}%
\begin{pgfscope}%
\pgfpathrectangle{\pgfqpoint{0.100000in}{0.212622in}}{\pgfqpoint{3.696000in}{3.696000in}}%
\pgfusepath{clip}%
\pgfsetbuttcap%
\pgfsetroundjoin%
\definecolor{currentfill}{rgb}{0.121569,0.466667,0.705882}%
\pgfsetfillcolor{currentfill}%
\pgfsetfillopacity{0.583614}%
\pgfsetlinewidth{1.003750pt}%
\definecolor{currentstroke}{rgb}{0.121569,0.466667,0.705882}%
\pgfsetstrokecolor{currentstroke}%
\pgfsetstrokeopacity{0.583614}%
\pgfsetdash{}{0pt}%
\pgfpathmoveto{\pgfqpoint{1.041499in}{1.546605in}}%
\pgfpathcurveto{\pgfqpoint{1.049735in}{1.546605in}}{\pgfqpoint{1.057635in}{1.549878in}}{\pgfqpoint{1.063459in}{1.555702in}}%
\pgfpathcurveto{\pgfqpoint{1.069283in}{1.561526in}}{\pgfqpoint{1.072556in}{1.569426in}}{\pgfqpoint{1.072556in}{1.577662in}}%
\pgfpathcurveto{\pgfqpoint{1.072556in}{1.585898in}}{\pgfqpoint{1.069283in}{1.593798in}}{\pgfqpoint{1.063459in}{1.599622in}}%
\pgfpathcurveto{\pgfqpoint{1.057635in}{1.605446in}}{\pgfqpoint{1.049735in}{1.608718in}}{\pgfqpoint{1.041499in}{1.608718in}}%
\pgfpathcurveto{\pgfqpoint{1.033263in}{1.608718in}}{\pgfqpoint{1.025363in}{1.605446in}}{\pgfqpoint{1.019539in}{1.599622in}}%
\pgfpathcurveto{\pgfqpoint{1.013715in}{1.593798in}}{\pgfqpoint{1.010443in}{1.585898in}}{\pgfqpoint{1.010443in}{1.577662in}}%
\pgfpathcurveto{\pgfqpoint{1.010443in}{1.569426in}}{\pgfqpoint{1.013715in}{1.561526in}}{\pgfqpoint{1.019539in}{1.555702in}}%
\pgfpathcurveto{\pgfqpoint{1.025363in}{1.549878in}}{\pgfqpoint{1.033263in}{1.546605in}}{\pgfqpoint{1.041499in}{1.546605in}}%
\pgfpathclose%
\pgfusepath{stroke,fill}%
\end{pgfscope}%
\begin{pgfscope}%
\pgfpathrectangle{\pgfqpoint{0.100000in}{0.212622in}}{\pgfqpoint{3.696000in}{3.696000in}}%
\pgfusepath{clip}%
\pgfsetbuttcap%
\pgfsetroundjoin%
\definecolor{currentfill}{rgb}{0.121569,0.466667,0.705882}%
\pgfsetfillcolor{currentfill}%
\pgfsetfillopacity{0.586137}%
\pgfsetlinewidth{1.003750pt}%
\definecolor{currentstroke}{rgb}{0.121569,0.466667,0.705882}%
\pgfsetstrokecolor{currentstroke}%
\pgfsetstrokeopacity{0.586137}%
\pgfsetdash{}{0pt}%
\pgfpathmoveto{\pgfqpoint{2.098445in}{1.804911in}}%
\pgfpathcurveto{\pgfqpoint{2.106681in}{1.804911in}}{\pgfqpoint{2.114581in}{1.808183in}}{\pgfqpoint{2.120405in}{1.814007in}}%
\pgfpathcurveto{\pgfqpoint{2.126229in}{1.819831in}}{\pgfqpoint{2.129502in}{1.827731in}}{\pgfqpoint{2.129502in}{1.835967in}}%
\pgfpathcurveto{\pgfqpoint{2.129502in}{1.844204in}}{\pgfqpoint{2.126229in}{1.852104in}}{\pgfqpoint{2.120405in}{1.857928in}}%
\pgfpathcurveto{\pgfqpoint{2.114581in}{1.863752in}}{\pgfqpoint{2.106681in}{1.867024in}}{\pgfqpoint{2.098445in}{1.867024in}}%
\pgfpathcurveto{\pgfqpoint{2.090209in}{1.867024in}}{\pgfqpoint{2.082309in}{1.863752in}}{\pgfqpoint{2.076485in}{1.857928in}}%
\pgfpathcurveto{\pgfqpoint{2.070661in}{1.852104in}}{\pgfqpoint{2.067389in}{1.844204in}}{\pgfqpoint{2.067389in}{1.835967in}}%
\pgfpathcurveto{\pgfqpoint{2.067389in}{1.827731in}}{\pgfqpoint{2.070661in}{1.819831in}}{\pgfqpoint{2.076485in}{1.814007in}}%
\pgfpathcurveto{\pgfqpoint{2.082309in}{1.808183in}}{\pgfqpoint{2.090209in}{1.804911in}}{\pgfqpoint{2.098445in}{1.804911in}}%
\pgfpathclose%
\pgfusepath{stroke,fill}%
\end{pgfscope}%
\begin{pgfscope}%
\pgfpathrectangle{\pgfqpoint{0.100000in}{0.212622in}}{\pgfqpoint{3.696000in}{3.696000in}}%
\pgfusepath{clip}%
\pgfsetbuttcap%
\pgfsetroundjoin%
\definecolor{currentfill}{rgb}{0.121569,0.466667,0.705882}%
\pgfsetfillcolor{currentfill}%
\pgfsetfillopacity{0.586210}%
\pgfsetlinewidth{1.003750pt}%
\definecolor{currentstroke}{rgb}{0.121569,0.466667,0.705882}%
\pgfsetstrokecolor{currentstroke}%
\pgfsetstrokeopacity{0.586210}%
\pgfsetdash{}{0pt}%
\pgfpathmoveto{\pgfqpoint{1.030863in}{1.537502in}}%
\pgfpathcurveto{\pgfqpoint{1.039100in}{1.537502in}}{\pgfqpoint{1.047000in}{1.540775in}}{\pgfqpoint{1.052824in}{1.546599in}}%
\pgfpathcurveto{\pgfqpoint{1.058648in}{1.552423in}}{\pgfqpoint{1.061920in}{1.560323in}}{\pgfqpoint{1.061920in}{1.568559in}}%
\pgfpathcurveto{\pgfqpoint{1.061920in}{1.576795in}}{\pgfqpoint{1.058648in}{1.584695in}}{\pgfqpoint{1.052824in}{1.590519in}}%
\pgfpathcurveto{\pgfqpoint{1.047000in}{1.596343in}}{\pgfqpoint{1.039100in}{1.599615in}}{\pgfqpoint{1.030863in}{1.599615in}}%
\pgfpathcurveto{\pgfqpoint{1.022627in}{1.599615in}}{\pgfqpoint{1.014727in}{1.596343in}}{\pgfqpoint{1.008903in}{1.590519in}}%
\pgfpathcurveto{\pgfqpoint{1.003079in}{1.584695in}}{\pgfqpoint{0.999807in}{1.576795in}}{\pgfqpoint{0.999807in}{1.568559in}}%
\pgfpathcurveto{\pgfqpoint{0.999807in}{1.560323in}}{\pgfqpoint{1.003079in}{1.552423in}}{\pgfqpoint{1.008903in}{1.546599in}}%
\pgfpathcurveto{\pgfqpoint{1.014727in}{1.540775in}}{\pgfqpoint{1.022627in}{1.537502in}}{\pgfqpoint{1.030863in}{1.537502in}}%
\pgfpathclose%
\pgfusepath{stroke,fill}%
\end{pgfscope}%
\begin{pgfscope}%
\pgfpathrectangle{\pgfqpoint{0.100000in}{0.212622in}}{\pgfqpoint{3.696000in}{3.696000in}}%
\pgfusepath{clip}%
\pgfsetbuttcap%
\pgfsetroundjoin%
\definecolor{currentfill}{rgb}{0.121569,0.466667,0.705882}%
\pgfsetfillcolor{currentfill}%
\pgfsetfillopacity{0.589278}%
\pgfsetlinewidth{1.003750pt}%
\definecolor{currentstroke}{rgb}{0.121569,0.466667,0.705882}%
\pgfsetstrokecolor{currentstroke}%
\pgfsetstrokeopacity{0.589278}%
\pgfsetdash{}{0pt}%
\pgfpathmoveto{\pgfqpoint{2.100254in}{1.802995in}}%
\pgfpathcurveto{\pgfqpoint{2.108491in}{1.802995in}}{\pgfqpoint{2.116391in}{1.806268in}}{\pgfqpoint{2.122215in}{1.812092in}}%
\pgfpathcurveto{\pgfqpoint{2.128038in}{1.817916in}}{\pgfqpoint{2.131311in}{1.825816in}}{\pgfqpoint{2.131311in}{1.834052in}}%
\pgfpathcurveto{\pgfqpoint{2.131311in}{1.842288in}}{\pgfqpoint{2.128038in}{1.850188in}}{\pgfqpoint{2.122215in}{1.856012in}}%
\pgfpathcurveto{\pgfqpoint{2.116391in}{1.861836in}}{\pgfqpoint{2.108491in}{1.865108in}}{\pgfqpoint{2.100254in}{1.865108in}}%
\pgfpathcurveto{\pgfqpoint{2.092018in}{1.865108in}}{\pgfqpoint{2.084118in}{1.861836in}}{\pgfqpoint{2.078294in}{1.856012in}}%
\pgfpathcurveto{\pgfqpoint{2.072470in}{1.850188in}}{\pgfqpoint{2.069198in}{1.842288in}}{\pgfqpoint{2.069198in}{1.834052in}}%
\pgfpathcurveto{\pgfqpoint{2.069198in}{1.825816in}}{\pgfqpoint{2.072470in}{1.817916in}}{\pgfqpoint{2.078294in}{1.812092in}}%
\pgfpathcurveto{\pgfqpoint{2.084118in}{1.806268in}}{\pgfqpoint{2.092018in}{1.802995in}}{\pgfqpoint{2.100254in}{1.802995in}}%
\pgfpathclose%
\pgfusepath{stroke,fill}%
\end{pgfscope}%
\begin{pgfscope}%
\pgfpathrectangle{\pgfqpoint{0.100000in}{0.212622in}}{\pgfqpoint{3.696000in}{3.696000in}}%
\pgfusepath{clip}%
\pgfsetbuttcap%
\pgfsetroundjoin%
\definecolor{currentfill}{rgb}{0.121569,0.466667,0.705882}%
\pgfsetfillcolor{currentfill}%
\pgfsetfillopacity{0.589828}%
\pgfsetlinewidth{1.003750pt}%
\definecolor{currentstroke}{rgb}{0.121569,0.466667,0.705882}%
\pgfsetstrokecolor{currentstroke}%
\pgfsetstrokeopacity{0.589828}%
\pgfsetdash{}{0pt}%
\pgfpathmoveto{\pgfqpoint{1.022715in}{1.532953in}}%
\pgfpathcurveto{\pgfqpoint{1.030951in}{1.532953in}}{\pgfqpoint{1.038851in}{1.536226in}}{\pgfqpoint{1.044675in}{1.542050in}}%
\pgfpathcurveto{\pgfqpoint{1.050499in}{1.547874in}}{\pgfqpoint{1.053771in}{1.555774in}}{\pgfqpoint{1.053771in}{1.564010in}}%
\pgfpathcurveto{\pgfqpoint{1.053771in}{1.572246in}}{\pgfqpoint{1.050499in}{1.580146in}}{\pgfqpoint{1.044675in}{1.585970in}}%
\pgfpathcurveto{\pgfqpoint{1.038851in}{1.591794in}}{\pgfqpoint{1.030951in}{1.595066in}}{\pgfqpoint{1.022715in}{1.595066in}}%
\pgfpathcurveto{\pgfqpoint{1.014478in}{1.595066in}}{\pgfqpoint{1.006578in}{1.591794in}}{\pgfqpoint{1.000754in}{1.585970in}}%
\pgfpathcurveto{\pgfqpoint{0.994930in}{1.580146in}}{\pgfqpoint{0.991658in}{1.572246in}}{\pgfqpoint{0.991658in}{1.564010in}}%
\pgfpathcurveto{\pgfqpoint{0.991658in}{1.555774in}}{\pgfqpoint{0.994930in}{1.547874in}}{\pgfqpoint{1.000754in}{1.542050in}}%
\pgfpathcurveto{\pgfqpoint{1.006578in}{1.536226in}}{\pgfqpoint{1.014478in}{1.532953in}}{\pgfqpoint{1.022715in}{1.532953in}}%
\pgfpathclose%
\pgfusepath{stroke,fill}%
\end{pgfscope}%
\begin{pgfscope}%
\pgfpathrectangle{\pgfqpoint{0.100000in}{0.212622in}}{\pgfqpoint{3.696000in}{3.696000in}}%
\pgfusepath{clip}%
\pgfsetbuttcap%
\pgfsetroundjoin%
\definecolor{currentfill}{rgb}{0.121569,0.466667,0.705882}%
\pgfsetfillcolor{currentfill}%
\pgfsetfillopacity{0.591613}%
\pgfsetlinewidth{1.003750pt}%
\definecolor{currentstroke}{rgb}{0.121569,0.466667,0.705882}%
\pgfsetstrokecolor{currentstroke}%
\pgfsetstrokeopacity{0.591613}%
\pgfsetdash{}{0pt}%
\pgfpathmoveto{\pgfqpoint{1.014355in}{1.525801in}}%
\pgfpathcurveto{\pgfqpoint{1.022591in}{1.525801in}}{\pgfqpoint{1.030491in}{1.529073in}}{\pgfqpoint{1.036315in}{1.534897in}}%
\pgfpathcurveto{\pgfqpoint{1.042139in}{1.540721in}}{\pgfqpoint{1.045412in}{1.548621in}}{\pgfqpoint{1.045412in}{1.556858in}}%
\pgfpathcurveto{\pgfqpoint{1.045412in}{1.565094in}}{\pgfqpoint{1.042139in}{1.572994in}}{\pgfqpoint{1.036315in}{1.578818in}}%
\pgfpathcurveto{\pgfqpoint{1.030491in}{1.584642in}}{\pgfqpoint{1.022591in}{1.587914in}}{\pgfqpoint{1.014355in}{1.587914in}}%
\pgfpathcurveto{\pgfqpoint{1.006119in}{1.587914in}}{\pgfqpoint{0.998219in}{1.584642in}}{\pgfqpoint{0.992395in}{1.578818in}}%
\pgfpathcurveto{\pgfqpoint{0.986571in}{1.572994in}}{\pgfqpoint{0.983299in}{1.565094in}}{\pgfqpoint{0.983299in}{1.556858in}}%
\pgfpathcurveto{\pgfqpoint{0.983299in}{1.548621in}}{\pgfqpoint{0.986571in}{1.540721in}}{\pgfqpoint{0.992395in}{1.534897in}}%
\pgfpathcurveto{\pgfqpoint{0.998219in}{1.529073in}}{\pgfqpoint{1.006119in}{1.525801in}}{\pgfqpoint{1.014355in}{1.525801in}}%
\pgfpathclose%
\pgfusepath{stroke,fill}%
\end{pgfscope}%
\begin{pgfscope}%
\pgfpathrectangle{\pgfqpoint{0.100000in}{0.212622in}}{\pgfqpoint{3.696000in}{3.696000in}}%
\pgfusepath{clip}%
\pgfsetbuttcap%
\pgfsetroundjoin%
\definecolor{currentfill}{rgb}{0.121569,0.466667,0.705882}%
\pgfsetfillcolor{currentfill}%
\pgfsetfillopacity{0.592673}%
\pgfsetlinewidth{1.003750pt}%
\definecolor{currentstroke}{rgb}{0.121569,0.466667,0.705882}%
\pgfsetstrokecolor{currentstroke}%
\pgfsetstrokeopacity{0.592673}%
\pgfsetdash{}{0pt}%
\pgfpathmoveto{\pgfqpoint{2.102170in}{1.798845in}}%
\pgfpathcurveto{\pgfqpoint{2.110407in}{1.798845in}}{\pgfqpoint{2.118307in}{1.802118in}}{\pgfqpoint{2.124131in}{1.807942in}}%
\pgfpathcurveto{\pgfqpoint{2.129955in}{1.813766in}}{\pgfqpoint{2.133227in}{1.821666in}}{\pgfqpoint{2.133227in}{1.829902in}}%
\pgfpathcurveto{\pgfqpoint{2.133227in}{1.838138in}}{\pgfqpoint{2.129955in}{1.846038in}}{\pgfqpoint{2.124131in}{1.851862in}}%
\pgfpathcurveto{\pgfqpoint{2.118307in}{1.857686in}}{\pgfqpoint{2.110407in}{1.860958in}}{\pgfqpoint{2.102170in}{1.860958in}}%
\pgfpathcurveto{\pgfqpoint{2.093934in}{1.860958in}}{\pgfqpoint{2.086034in}{1.857686in}}{\pgfqpoint{2.080210in}{1.851862in}}%
\pgfpathcurveto{\pgfqpoint{2.074386in}{1.846038in}}{\pgfqpoint{2.071114in}{1.838138in}}{\pgfqpoint{2.071114in}{1.829902in}}%
\pgfpathcurveto{\pgfqpoint{2.071114in}{1.821666in}}{\pgfqpoint{2.074386in}{1.813766in}}{\pgfqpoint{2.080210in}{1.807942in}}%
\pgfpathcurveto{\pgfqpoint{2.086034in}{1.802118in}}{\pgfqpoint{2.093934in}{1.798845in}}{\pgfqpoint{2.102170in}{1.798845in}}%
\pgfpathclose%
\pgfusepath{stroke,fill}%
\end{pgfscope}%
\begin{pgfscope}%
\pgfpathrectangle{\pgfqpoint{0.100000in}{0.212622in}}{\pgfqpoint{3.696000in}{3.696000in}}%
\pgfusepath{clip}%
\pgfsetbuttcap%
\pgfsetroundjoin%
\definecolor{currentfill}{rgb}{0.121569,0.466667,0.705882}%
\pgfsetfillcolor{currentfill}%
\pgfsetfillopacity{0.593853}%
\pgfsetlinewidth{1.003750pt}%
\definecolor{currentstroke}{rgb}{0.121569,0.466667,0.705882}%
\pgfsetstrokecolor{currentstroke}%
\pgfsetstrokeopacity{0.593853}%
\pgfsetdash{}{0pt}%
\pgfpathmoveto{\pgfqpoint{1.007838in}{1.521277in}}%
\pgfpathcurveto{\pgfqpoint{1.016074in}{1.521277in}}{\pgfqpoint{1.023974in}{1.524550in}}{\pgfqpoint{1.029798in}{1.530374in}}%
\pgfpathcurveto{\pgfqpoint{1.035622in}{1.536197in}}{\pgfqpoint{1.038894in}{1.544098in}}{\pgfqpoint{1.038894in}{1.552334in}}%
\pgfpathcurveto{\pgfqpoint{1.038894in}{1.560570in}}{\pgfqpoint{1.035622in}{1.568470in}}{\pgfqpoint{1.029798in}{1.574294in}}%
\pgfpathcurveto{\pgfqpoint{1.023974in}{1.580118in}}{\pgfqpoint{1.016074in}{1.583390in}}{\pgfqpoint{1.007838in}{1.583390in}}%
\pgfpathcurveto{\pgfqpoint{0.999602in}{1.583390in}}{\pgfqpoint{0.991702in}{1.580118in}}{\pgfqpoint{0.985878in}{1.574294in}}%
\pgfpathcurveto{\pgfqpoint{0.980054in}{1.568470in}}{\pgfqpoint{0.976781in}{1.560570in}}{\pgfqpoint{0.976781in}{1.552334in}}%
\pgfpathcurveto{\pgfqpoint{0.976781in}{1.544098in}}{\pgfqpoint{0.980054in}{1.536197in}}{\pgfqpoint{0.985878in}{1.530374in}}%
\pgfpathcurveto{\pgfqpoint{0.991702in}{1.524550in}}{\pgfqpoint{0.999602in}{1.521277in}}{\pgfqpoint{1.007838in}{1.521277in}}%
\pgfpathclose%
\pgfusepath{stroke,fill}%
\end{pgfscope}%
\begin{pgfscope}%
\pgfpathrectangle{\pgfqpoint{0.100000in}{0.212622in}}{\pgfqpoint{3.696000in}{3.696000in}}%
\pgfusepath{clip}%
\pgfsetbuttcap%
\pgfsetroundjoin%
\definecolor{currentfill}{rgb}{0.121569,0.466667,0.705882}%
\pgfsetfillcolor{currentfill}%
\pgfsetfillopacity{0.596214}%
\pgfsetlinewidth{1.003750pt}%
\definecolor{currentstroke}{rgb}{0.121569,0.466667,0.705882}%
\pgfsetstrokecolor{currentstroke}%
\pgfsetstrokeopacity{0.596214}%
\pgfsetdash{}{0pt}%
\pgfpathmoveto{\pgfqpoint{2.104591in}{1.794229in}}%
\pgfpathcurveto{\pgfqpoint{2.112827in}{1.794229in}}{\pgfqpoint{2.120727in}{1.797502in}}{\pgfqpoint{2.126551in}{1.803326in}}%
\pgfpathcurveto{\pgfqpoint{2.132375in}{1.809150in}}{\pgfqpoint{2.135647in}{1.817050in}}{\pgfqpoint{2.135647in}{1.825286in}}%
\pgfpathcurveto{\pgfqpoint{2.135647in}{1.833522in}}{\pgfqpoint{2.132375in}{1.841422in}}{\pgfqpoint{2.126551in}{1.847246in}}%
\pgfpathcurveto{\pgfqpoint{2.120727in}{1.853070in}}{\pgfqpoint{2.112827in}{1.856342in}}{\pgfqpoint{2.104591in}{1.856342in}}%
\pgfpathcurveto{\pgfqpoint{2.096354in}{1.856342in}}{\pgfqpoint{2.088454in}{1.853070in}}{\pgfqpoint{2.082631in}{1.847246in}}%
\pgfpathcurveto{\pgfqpoint{2.076807in}{1.841422in}}{\pgfqpoint{2.073534in}{1.833522in}}{\pgfqpoint{2.073534in}{1.825286in}}%
\pgfpathcurveto{\pgfqpoint{2.073534in}{1.817050in}}{\pgfqpoint{2.076807in}{1.809150in}}{\pgfqpoint{2.082631in}{1.803326in}}%
\pgfpathcurveto{\pgfqpoint{2.088454in}{1.797502in}}{\pgfqpoint{2.096354in}{1.794229in}}{\pgfqpoint{2.104591in}{1.794229in}}%
\pgfpathclose%
\pgfusepath{stroke,fill}%
\end{pgfscope}%
\begin{pgfscope}%
\pgfpathrectangle{\pgfqpoint{0.100000in}{0.212622in}}{\pgfqpoint{3.696000in}{3.696000in}}%
\pgfusepath{clip}%
\pgfsetbuttcap%
\pgfsetroundjoin%
\definecolor{currentfill}{rgb}{0.121569,0.466667,0.705882}%
\pgfsetfillcolor{currentfill}%
\pgfsetfillopacity{0.597946}%
\pgfsetlinewidth{1.003750pt}%
\definecolor{currentstroke}{rgb}{0.121569,0.466667,0.705882}%
\pgfsetstrokecolor{currentstroke}%
\pgfsetstrokeopacity{0.597946}%
\pgfsetdash{}{0pt}%
\pgfpathmoveto{\pgfqpoint{0.995565in}{1.513690in}}%
\pgfpathcurveto{\pgfqpoint{1.003801in}{1.513690in}}{\pgfqpoint{1.011701in}{1.516962in}}{\pgfqpoint{1.017525in}{1.522786in}}%
\pgfpathcurveto{\pgfqpoint{1.023349in}{1.528610in}}{\pgfqpoint{1.026622in}{1.536510in}}{\pgfqpoint{1.026622in}{1.544747in}}%
\pgfpathcurveto{\pgfqpoint{1.026622in}{1.552983in}}{\pgfqpoint{1.023349in}{1.560883in}}{\pgfqpoint{1.017525in}{1.566707in}}%
\pgfpathcurveto{\pgfqpoint{1.011701in}{1.572531in}}{\pgfqpoint{1.003801in}{1.575803in}}{\pgfqpoint{0.995565in}{1.575803in}}%
\pgfpathcurveto{\pgfqpoint{0.987329in}{1.575803in}}{\pgfqpoint{0.979429in}{1.572531in}}{\pgfqpoint{0.973605in}{1.566707in}}%
\pgfpathcurveto{\pgfqpoint{0.967781in}{1.560883in}}{\pgfqpoint{0.964509in}{1.552983in}}{\pgfqpoint{0.964509in}{1.544747in}}%
\pgfpathcurveto{\pgfqpoint{0.964509in}{1.536510in}}{\pgfqpoint{0.967781in}{1.528610in}}{\pgfqpoint{0.973605in}{1.522786in}}%
\pgfpathcurveto{\pgfqpoint{0.979429in}{1.516962in}}{\pgfqpoint{0.987329in}{1.513690in}}{\pgfqpoint{0.995565in}{1.513690in}}%
\pgfpathclose%
\pgfusepath{stroke,fill}%
\end{pgfscope}%
\begin{pgfscope}%
\pgfpathrectangle{\pgfqpoint{0.100000in}{0.212622in}}{\pgfqpoint{3.696000in}{3.696000in}}%
\pgfusepath{clip}%
\pgfsetbuttcap%
\pgfsetroundjoin%
\definecolor{currentfill}{rgb}{0.121569,0.466667,0.705882}%
\pgfsetfillcolor{currentfill}%
\pgfsetfillopacity{0.600019}%
\pgfsetlinewidth{1.003750pt}%
\definecolor{currentstroke}{rgb}{0.121569,0.466667,0.705882}%
\pgfsetstrokecolor{currentstroke}%
\pgfsetstrokeopacity{0.600019}%
\pgfsetdash{}{0pt}%
\pgfpathmoveto{\pgfqpoint{2.107115in}{1.787979in}}%
\pgfpathcurveto{\pgfqpoint{2.115351in}{1.787979in}}{\pgfqpoint{2.123251in}{1.791251in}}{\pgfqpoint{2.129075in}{1.797075in}}%
\pgfpathcurveto{\pgfqpoint{2.134899in}{1.802899in}}{\pgfqpoint{2.138171in}{1.810799in}}{\pgfqpoint{2.138171in}{1.819035in}}%
\pgfpathcurveto{\pgfqpoint{2.138171in}{1.827272in}}{\pgfqpoint{2.134899in}{1.835172in}}{\pgfqpoint{2.129075in}{1.840996in}}%
\pgfpathcurveto{\pgfqpoint{2.123251in}{1.846820in}}{\pgfqpoint{2.115351in}{1.850092in}}{\pgfqpoint{2.107115in}{1.850092in}}%
\pgfpathcurveto{\pgfqpoint{2.098878in}{1.850092in}}{\pgfqpoint{2.090978in}{1.846820in}}{\pgfqpoint{2.085154in}{1.840996in}}%
\pgfpathcurveto{\pgfqpoint{2.079330in}{1.835172in}}{\pgfqpoint{2.076058in}{1.827272in}}{\pgfqpoint{2.076058in}{1.819035in}}%
\pgfpathcurveto{\pgfqpoint{2.076058in}{1.810799in}}{\pgfqpoint{2.079330in}{1.802899in}}{\pgfqpoint{2.085154in}{1.797075in}}%
\pgfpathcurveto{\pgfqpoint{2.090978in}{1.791251in}}{\pgfqpoint{2.098878in}{1.787979in}}{\pgfqpoint{2.107115in}{1.787979in}}%
\pgfpathclose%
\pgfusepath{stroke,fill}%
\end{pgfscope}%
\begin{pgfscope}%
\pgfpathrectangle{\pgfqpoint{0.100000in}{0.212622in}}{\pgfqpoint{3.696000in}{3.696000in}}%
\pgfusepath{clip}%
\pgfsetbuttcap%
\pgfsetroundjoin%
\definecolor{currentfill}{rgb}{0.121569,0.466667,0.705882}%
\pgfsetfillcolor{currentfill}%
\pgfsetfillopacity{0.601477}%
\pgfsetlinewidth{1.003750pt}%
\definecolor{currentstroke}{rgb}{0.121569,0.466667,0.705882}%
\pgfsetstrokecolor{currentstroke}%
\pgfsetstrokeopacity{0.601477}%
\pgfsetdash{}{0pt}%
\pgfpathmoveto{\pgfqpoint{0.982120in}{1.506332in}}%
\pgfpathcurveto{\pgfqpoint{0.990356in}{1.506332in}}{\pgfqpoint{0.998256in}{1.509605in}}{\pgfqpoint{1.004080in}{1.515429in}}%
\pgfpathcurveto{\pgfqpoint{1.009904in}{1.521252in}}{\pgfqpoint{1.013176in}{1.529153in}}{\pgfqpoint{1.013176in}{1.537389in}}%
\pgfpathcurveto{\pgfqpoint{1.013176in}{1.545625in}}{\pgfqpoint{1.009904in}{1.553525in}}{\pgfqpoint{1.004080in}{1.559349in}}%
\pgfpathcurveto{\pgfqpoint{0.998256in}{1.565173in}}{\pgfqpoint{0.990356in}{1.568445in}}{\pgfqpoint{0.982120in}{1.568445in}}%
\pgfpathcurveto{\pgfqpoint{0.973883in}{1.568445in}}{\pgfqpoint{0.965983in}{1.565173in}}{\pgfqpoint{0.960159in}{1.559349in}}%
\pgfpathcurveto{\pgfqpoint{0.954336in}{1.553525in}}{\pgfqpoint{0.951063in}{1.545625in}}{\pgfqpoint{0.951063in}{1.537389in}}%
\pgfpathcurveto{\pgfqpoint{0.951063in}{1.529153in}}{\pgfqpoint{0.954336in}{1.521252in}}{\pgfqpoint{0.960159in}{1.515429in}}%
\pgfpathcurveto{\pgfqpoint{0.965983in}{1.509605in}}{\pgfqpoint{0.973883in}{1.506332in}}{\pgfqpoint{0.982120in}{1.506332in}}%
\pgfpathclose%
\pgfusepath{stroke,fill}%
\end{pgfscope}%
\begin{pgfscope}%
\pgfpathrectangle{\pgfqpoint{0.100000in}{0.212622in}}{\pgfqpoint{3.696000in}{3.696000in}}%
\pgfusepath{clip}%
\pgfsetbuttcap%
\pgfsetroundjoin%
\definecolor{currentfill}{rgb}{0.121569,0.466667,0.705882}%
\pgfsetfillcolor{currentfill}%
\pgfsetfillopacity{0.604976}%
\pgfsetlinewidth{1.003750pt}%
\definecolor{currentstroke}{rgb}{0.121569,0.466667,0.705882}%
\pgfsetstrokecolor{currentstroke}%
\pgfsetstrokeopacity{0.604976}%
\pgfsetdash{}{0pt}%
\pgfpathmoveto{\pgfqpoint{2.112626in}{1.787354in}}%
\pgfpathcurveto{\pgfqpoint{2.120863in}{1.787354in}}{\pgfqpoint{2.128763in}{1.790626in}}{\pgfqpoint{2.134587in}{1.796450in}}%
\pgfpathcurveto{\pgfqpoint{2.140411in}{1.802274in}}{\pgfqpoint{2.143683in}{1.810174in}}{\pgfqpoint{2.143683in}{1.818410in}}%
\pgfpathcurveto{\pgfqpoint{2.143683in}{1.826647in}}{\pgfqpoint{2.140411in}{1.834547in}}{\pgfqpoint{2.134587in}{1.840371in}}%
\pgfpathcurveto{\pgfqpoint{2.128763in}{1.846195in}}{\pgfqpoint{2.120863in}{1.849467in}}{\pgfqpoint{2.112626in}{1.849467in}}%
\pgfpathcurveto{\pgfqpoint{2.104390in}{1.849467in}}{\pgfqpoint{2.096490in}{1.846195in}}{\pgfqpoint{2.090666in}{1.840371in}}%
\pgfpathcurveto{\pgfqpoint{2.084842in}{1.834547in}}{\pgfqpoint{2.081570in}{1.826647in}}{\pgfqpoint{2.081570in}{1.818410in}}%
\pgfpathcurveto{\pgfqpoint{2.081570in}{1.810174in}}{\pgfqpoint{2.084842in}{1.802274in}}{\pgfqpoint{2.090666in}{1.796450in}}%
\pgfpathcurveto{\pgfqpoint{2.096490in}{1.790626in}}{\pgfqpoint{2.104390in}{1.787354in}}{\pgfqpoint{2.112626in}{1.787354in}}%
\pgfpathclose%
\pgfusepath{stroke,fill}%
\end{pgfscope}%
\begin{pgfscope}%
\pgfpathrectangle{\pgfqpoint{0.100000in}{0.212622in}}{\pgfqpoint{3.696000in}{3.696000in}}%
\pgfusepath{clip}%
\pgfsetbuttcap%
\pgfsetroundjoin%
\definecolor{currentfill}{rgb}{0.121569,0.466667,0.705882}%
\pgfsetfillcolor{currentfill}%
\pgfsetfillopacity{0.605483}%
\pgfsetlinewidth{1.003750pt}%
\definecolor{currentstroke}{rgb}{0.121569,0.466667,0.705882}%
\pgfsetstrokecolor{currentstroke}%
\pgfsetstrokeopacity{0.605483}%
\pgfsetdash{}{0pt}%
\pgfpathmoveto{\pgfqpoint{0.974112in}{1.503329in}}%
\pgfpathcurveto{\pgfqpoint{0.982349in}{1.503329in}}{\pgfqpoint{0.990249in}{1.506601in}}{\pgfqpoint{0.996073in}{1.512425in}}%
\pgfpathcurveto{\pgfqpoint{1.001897in}{1.518249in}}{\pgfqpoint{1.005169in}{1.526149in}}{\pgfqpoint{1.005169in}{1.534386in}}%
\pgfpathcurveto{\pgfqpoint{1.005169in}{1.542622in}}{\pgfqpoint{1.001897in}{1.550522in}}{\pgfqpoint{0.996073in}{1.556346in}}%
\pgfpathcurveto{\pgfqpoint{0.990249in}{1.562170in}}{\pgfqpoint{0.982349in}{1.565442in}}{\pgfqpoint{0.974112in}{1.565442in}}%
\pgfpathcurveto{\pgfqpoint{0.965876in}{1.565442in}}{\pgfqpoint{0.957976in}{1.562170in}}{\pgfqpoint{0.952152in}{1.556346in}}%
\pgfpathcurveto{\pgfqpoint{0.946328in}{1.550522in}}{\pgfqpoint{0.943056in}{1.542622in}}{\pgfqpoint{0.943056in}{1.534386in}}%
\pgfpathcurveto{\pgfqpoint{0.943056in}{1.526149in}}{\pgfqpoint{0.946328in}{1.518249in}}{\pgfqpoint{0.952152in}{1.512425in}}%
\pgfpathcurveto{\pgfqpoint{0.957976in}{1.506601in}}{\pgfqpoint{0.965876in}{1.503329in}}{\pgfqpoint{0.974112in}{1.503329in}}%
\pgfpathclose%
\pgfusepath{stroke,fill}%
\end{pgfscope}%
\begin{pgfscope}%
\pgfpathrectangle{\pgfqpoint{0.100000in}{0.212622in}}{\pgfqpoint{3.696000in}{3.696000in}}%
\pgfusepath{clip}%
\pgfsetbuttcap%
\pgfsetroundjoin%
\definecolor{currentfill}{rgb}{0.121569,0.466667,0.705882}%
\pgfsetfillcolor{currentfill}%
\pgfsetfillopacity{0.607575}%
\pgfsetlinewidth{1.003750pt}%
\definecolor{currentstroke}{rgb}{0.121569,0.466667,0.705882}%
\pgfsetstrokecolor{currentstroke}%
\pgfsetstrokeopacity{0.607575}%
\pgfsetdash{}{0pt}%
\pgfpathmoveto{\pgfqpoint{0.965392in}{1.490725in}}%
\pgfpathcurveto{\pgfqpoint{0.973629in}{1.490725in}}{\pgfqpoint{0.981529in}{1.493997in}}{\pgfqpoint{0.987353in}{1.499821in}}%
\pgfpathcurveto{\pgfqpoint{0.993177in}{1.505645in}}{\pgfqpoint{0.996449in}{1.513545in}}{\pgfqpoint{0.996449in}{1.521781in}}%
\pgfpathcurveto{\pgfqpoint{0.996449in}{1.530018in}}{\pgfqpoint{0.993177in}{1.537918in}}{\pgfqpoint{0.987353in}{1.543742in}}%
\pgfpathcurveto{\pgfqpoint{0.981529in}{1.549565in}}{\pgfqpoint{0.973629in}{1.552838in}}{\pgfqpoint{0.965392in}{1.552838in}}%
\pgfpathcurveto{\pgfqpoint{0.957156in}{1.552838in}}{\pgfqpoint{0.949256in}{1.549565in}}{\pgfqpoint{0.943432in}{1.543742in}}%
\pgfpathcurveto{\pgfqpoint{0.937608in}{1.537918in}}{\pgfqpoint{0.934336in}{1.530018in}}{\pgfqpoint{0.934336in}{1.521781in}}%
\pgfpathcurveto{\pgfqpoint{0.934336in}{1.513545in}}{\pgfqpoint{0.937608in}{1.505645in}}{\pgfqpoint{0.943432in}{1.499821in}}%
\pgfpathcurveto{\pgfqpoint{0.949256in}{1.493997in}}{\pgfqpoint{0.957156in}{1.490725in}}{\pgfqpoint{0.965392in}{1.490725in}}%
\pgfpathclose%
\pgfusepath{stroke,fill}%
\end{pgfscope}%
\begin{pgfscope}%
\pgfpathrectangle{\pgfqpoint{0.100000in}{0.212622in}}{\pgfqpoint{3.696000in}{3.696000in}}%
\pgfusepath{clip}%
\pgfsetbuttcap%
\pgfsetroundjoin%
\definecolor{currentfill}{rgb}{0.121569,0.466667,0.705882}%
\pgfsetfillcolor{currentfill}%
\pgfsetfillopacity{0.607582}%
\pgfsetlinewidth{1.003750pt}%
\definecolor{currentstroke}{rgb}{0.121569,0.466667,0.705882}%
\pgfsetstrokecolor{currentstroke}%
\pgfsetstrokeopacity{0.607582}%
\pgfsetdash{}{0pt}%
\pgfpathmoveto{\pgfqpoint{2.113001in}{1.784760in}}%
\pgfpathcurveto{\pgfqpoint{2.121237in}{1.784760in}}{\pgfqpoint{2.129138in}{1.788032in}}{\pgfqpoint{2.134961in}{1.793856in}}%
\pgfpathcurveto{\pgfqpoint{2.140785in}{1.799680in}}{\pgfqpoint{2.144058in}{1.807580in}}{\pgfqpoint{2.144058in}{1.815817in}}%
\pgfpathcurveto{\pgfqpoint{2.144058in}{1.824053in}}{\pgfqpoint{2.140785in}{1.831953in}}{\pgfqpoint{2.134961in}{1.837777in}}%
\pgfpathcurveto{\pgfqpoint{2.129138in}{1.843601in}}{\pgfqpoint{2.121237in}{1.846873in}}{\pgfqpoint{2.113001in}{1.846873in}}%
\pgfpathcurveto{\pgfqpoint{2.104765in}{1.846873in}}{\pgfqpoint{2.096865in}{1.843601in}}{\pgfqpoint{2.091041in}{1.837777in}}%
\pgfpathcurveto{\pgfqpoint{2.085217in}{1.831953in}}{\pgfqpoint{2.081945in}{1.824053in}}{\pgfqpoint{2.081945in}{1.815817in}}%
\pgfpathcurveto{\pgfqpoint{2.081945in}{1.807580in}}{\pgfqpoint{2.085217in}{1.799680in}}{\pgfqpoint{2.091041in}{1.793856in}}%
\pgfpathcurveto{\pgfqpoint{2.096865in}{1.788032in}}{\pgfqpoint{2.104765in}{1.784760in}}{\pgfqpoint{2.113001in}{1.784760in}}%
\pgfpathclose%
\pgfusepath{stroke,fill}%
\end{pgfscope}%
\begin{pgfscope}%
\pgfpathrectangle{\pgfqpoint{0.100000in}{0.212622in}}{\pgfqpoint{3.696000in}{3.696000in}}%
\pgfusepath{clip}%
\pgfsetbuttcap%
\pgfsetroundjoin%
\definecolor{currentfill}{rgb}{0.121569,0.466667,0.705882}%
\pgfsetfillcolor{currentfill}%
\pgfsetfillopacity{0.610265}%
\pgfsetlinewidth{1.003750pt}%
\definecolor{currentstroke}{rgb}{0.121569,0.466667,0.705882}%
\pgfsetstrokecolor{currentstroke}%
\pgfsetstrokeopacity{0.610265}%
\pgfsetdash{}{0pt}%
\pgfpathmoveto{\pgfqpoint{0.956692in}{1.484272in}}%
\pgfpathcurveto{\pgfqpoint{0.964929in}{1.484272in}}{\pgfqpoint{0.972829in}{1.487544in}}{\pgfqpoint{0.978653in}{1.493368in}}%
\pgfpathcurveto{\pgfqpoint{0.984477in}{1.499192in}}{\pgfqpoint{0.987749in}{1.507092in}}{\pgfqpoint{0.987749in}{1.515328in}}%
\pgfpathcurveto{\pgfqpoint{0.987749in}{1.523564in}}{\pgfqpoint{0.984477in}{1.531464in}}{\pgfqpoint{0.978653in}{1.537288in}}%
\pgfpathcurveto{\pgfqpoint{0.972829in}{1.543112in}}{\pgfqpoint{0.964929in}{1.546385in}}{\pgfqpoint{0.956692in}{1.546385in}}%
\pgfpathcurveto{\pgfqpoint{0.948456in}{1.546385in}}{\pgfqpoint{0.940556in}{1.543112in}}{\pgfqpoint{0.934732in}{1.537288in}}%
\pgfpathcurveto{\pgfqpoint{0.928908in}{1.531464in}}{\pgfqpoint{0.925636in}{1.523564in}}{\pgfqpoint{0.925636in}{1.515328in}}%
\pgfpathcurveto{\pgfqpoint{0.925636in}{1.507092in}}{\pgfqpoint{0.928908in}{1.499192in}}{\pgfqpoint{0.934732in}{1.493368in}}%
\pgfpathcurveto{\pgfqpoint{0.940556in}{1.487544in}}{\pgfqpoint{0.948456in}{1.484272in}}{\pgfqpoint{0.956692in}{1.484272in}}%
\pgfpathclose%
\pgfusepath{stroke,fill}%
\end{pgfscope}%
\begin{pgfscope}%
\pgfpathrectangle{\pgfqpoint{0.100000in}{0.212622in}}{\pgfqpoint{3.696000in}{3.696000in}}%
\pgfusepath{clip}%
\pgfsetbuttcap%
\pgfsetroundjoin%
\definecolor{currentfill}{rgb}{0.121569,0.466667,0.705882}%
\pgfsetfillcolor{currentfill}%
\pgfsetfillopacity{0.610284}%
\pgfsetlinewidth{1.003750pt}%
\definecolor{currentstroke}{rgb}{0.121569,0.466667,0.705882}%
\pgfsetstrokecolor{currentstroke}%
\pgfsetstrokeopacity{0.610284}%
\pgfsetdash{}{0pt}%
\pgfpathmoveto{\pgfqpoint{2.115098in}{1.781204in}}%
\pgfpathcurveto{\pgfqpoint{2.123334in}{1.781204in}}{\pgfqpoint{2.131234in}{1.784476in}}{\pgfqpoint{2.137058in}{1.790300in}}%
\pgfpathcurveto{\pgfqpoint{2.142882in}{1.796124in}}{\pgfqpoint{2.146155in}{1.804024in}}{\pgfqpoint{2.146155in}{1.812260in}}%
\pgfpathcurveto{\pgfqpoint{2.146155in}{1.820497in}}{\pgfqpoint{2.142882in}{1.828397in}}{\pgfqpoint{2.137058in}{1.834221in}}%
\pgfpathcurveto{\pgfqpoint{2.131234in}{1.840045in}}{\pgfqpoint{2.123334in}{1.843317in}}{\pgfqpoint{2.115098in}{1.843317in}}%
\pgfpathcurveto{\pgfqpoint{2.106862in}{1.843317in}}{\pgfqpoint{2.098962in}{1.840045in}}{\pgfqpoint{2.093138in}{1.834221in}}%
\pgfpathcurveto{\pgfqpoint{2.087314in}{1.828397in}}{\pgfqpoint{2.084042in}{1.820497in}}{\pgfqpoint{2.084042in}{1.812260in}}%
\pgfpathcurveto{\pgfqpoint{2.084042in}{1.804024in}}{\pgfqpoint{2.087314in}{1.796124in}}{\pgfqpoint{2.093138in}{1.790300in}}%
\pgfpathcurveto{\pgfqpoint{2.098962in}{1.784476in}}{\pgfqpoint{2.106862in}{1.781204in}}{\pgfqpoint{2.115098in}{1.781204in}}%
\pgfpathclose%
\pgfusepath{stroke,fill}%
\end{pgfscope}%
\begin{pgfscope}%
\pgfpathrectangle{\pgfqpoint{0.100000in}{0.212622in}}{\pgfqpoint{3.696000in}{3.696000in}}%
\pgfusepath{clip}%
\pgfsetbuttcap%
\pgfsetroundjoin%
\definecolor{currentfill}{rgb}{0.121569,0.466667,0.705882}%
\pgfsetfillcolor{currentfill}%
\pgfsetfillopacity{0.613265}%
\pgfsetlinewidth{1.003750pt}%
\definecolor{currentstroke}{rgb}{0.121569,0.466667,0.705882}%
\pgfsetstrokecolor{currentstroke}%
\pgfsetstrokeopacity{0.613265}%
\pgfsetdash{}{0pt}%
\pgfpathmoveto{\pgfqpoint{2.117839in}{1.777624in}}%
\pgfpathcurveto{\pgfqpoint{2.126075in}{1.777624in}}{\pgfqpoint{2.133975in}{1.780897in}}{\pgfqpoint{2.139799in}{1.786721in}}%
\pgfpathcurveto{\pgfqpoint{2.145623in}{1.792545in}}{\pgfqpoint{2.148895in}{1.800445in}}{\pgfqpoint{2.148895in}{1.808681in}}%
\pgfpathcurveto{\pgfqpoint{2.148895in}{1.816917in}}{\pgfqpoint{2.145623in}{1.824817in}}{\pgfqpoint{2.139799in}{1.830641in}}%
\pgfpathcurveto{\pgfqpoint{2.133975in}{1.836465in}}{\pgfqpoint{2.126075in}{1.839737in}}{\pgfqpoint{2.117839in}{1.839737in}}%
\pgfpathcurveto{\pgfqpoint{2.109602in}{1.839737in}}{\pgfqpoint{2.101702in}{1.836465in}}{\pgfqpoint{2.095878in}{1.830641in}}%
\pgfpathcurveto{\pgfqpoint{2.090054in}{1.824817in}}{\pgfqpoint{2.086782in}{1.816917in}}{\pgfqpoint{2.086782in}{1.808681in}}%
\pgfpathcurveto{\pgfqpoint{2.086782in}{1.800445in}}{\pgfqpoint{2.090054in}{1.792545in}}{\pgfqpoint{2.095878in}{1.786721in}}%
\pgfpathcurveto{\pgfqpoint{2.101702in}{1.780897in}}{\pgfqpoint{2.109602in}{1.777624in}}{\pgfqpoint{2.117839in}{1.777624in}}%
\pgfpathclose%
\pgfusepath{stroke,fill}%
\end{pgfscope}%
\begin{pgfscope}%
\pgfpathrectangle{\pgfqpoint{0.100000in}{0.212622in}}{\pgfqpoint{3.696000in}{3.696000in}}%
\pgfusepath{clip}%
\pgfsetbuttcap%
\pgfsetroundjoin%
\definecolor{currentfill}{rgb}{0.121569,0.466667,0.705882}%
\pgfsetfillcolor{currentfill}%
\pgfsetfillopacity{0.613459}%
\pgfsetlinewidth{1.003750pt}%
\definecolor{currentstroke}{rgb}{0.121569,0.466667,0.705882}%
\pgfsetstrokecolor{currentstroke}%
\pgfsetstrokeopacity{0.613459}%
\pgfsetdash{}{0pt}%
\pgfpathmoveto{\pgfqpoint{0.951508in}{1.481123in}}%
\pgfpathcurveto{\pgfqpoint{0.959744in}{1.481123in}}{\pgfqpoint{0.967645in}{1.484396in}}{\pgfqpoint{0.973468in}{1.490219in}}%
\pgfpathcurveto{\pgfqpoint{0.979292in}{1.496043in}}{\pgfqpoint{0.982565in}{1.503943in}}{\pgfqpoint{0.982565in}{1.512180in}}%
\pgfpathcurveto{\pgfqpoint{0.982565in}{1.520416in}}{\pgfqpoint{0.979292in}{1.528316in}}{\pgfqpoint{0.973468in}{1.534140in}}%
\pgfpathcurveto{\pgfqpoint{0.967645in}{1.539964in}}{\pgfqpoint{0.959744in}{1.543236in}}{\pgfqpoint{0.951508in}{1.543236in}}%
\pgfpathcurveto{\pgfqpoint{0.943272in}{1.543236in}}{\pgfqpoint{0.935372in}{1.539964in}}{\pgfqpoint{0.929548in}{1.534140in}}%
\pgfpathcurveto{\pgfqpoint{0.923724in}{1.528316in}}{\pgfqpoint{0.920452in}{1.520416in}}{\pgfqpoint{0.920452in}{1.512180in}}%
\pgfpathcurveto{\pgfqpoint{0.920452in}{1.503943in}}{\pgfqpoint{0.923724in}{1.496043in}}{\pgfqpoint{0.929548in}{1.490219in}}%
\pgfpathcurveto{\pgfqpoint{0.935372in}{1.484396in}}{\pgfqpoint{0.943272in}{1.481123in}}{\pgfqpoint{0.951508in}{1.481123in}}%
\pgfpathclose%
\pgfusepath{stroke,fill}%
\end{pgfscope}%
\begin{pgfscope}%
\pgfpathrectangle{\pgfqpoint{0.100000in}{0.212622in}}{\pgfqpoint{3.696000in}{3.696000in}}%
\pgfusepath{clip}%
\pgfsetbuttcap%
\pgfsetroundjoin%
\definecolor{currentfill}{rgb}{0.121569,0.466667,0.705882}%
\pgfsetfillcolor{currentfill}%
\pgfsetfillopacity{0.615466}%
\pgfsetlinewidth{1.003750pt}%
\definecolor{currentstroke}{rgb}{0.121569,0.466667,0.705882}%
\pgfsetstrokecolor{currentstroke}%
\pgfsetstrokeopacity{0.615466}%
\pgfsetdash{}{0pt}%
\pgfpathmoveto{\pgfqpoint{0.945090in}{1.480241in}}%
\pgfpathcurveto{\pgfqpoint{0.953326in}{1.480241in}}{\pgfqpoint{0.961226in}{1.483514in}}{\pgfqpoint{0.967050in}{1.489338in}}%
\pgfpathcurveto{\pgfqpoint{0.972874in}{1.495161in}}{\pgfqpoint{0.976147in}{1.503062in}}{\pgfqpoint{0.976147in}{1.511298in}}%
\pgfpathcurveto{\pgfqpoint{0.976147in}{1.519534in}}{\pgfqpoint{0.972874in}{1.527434in}}{\pgfqpoint{0.967050in}{1.533258in}}%
\pgfpathcurveto{\pgfqpoint{0.961226in}{1.539082in}}{\pgfqpoint{0.953326in}{1.542354in}}{\pgfqpoint{0.945090in}{1.542354in}}%
\pgfpathcurveto{\pgfqpoint{0.936854in}{1.542354in}}{\pgfqpoint{0.928954in}{1.539082in}}{\pgfqpoint{0.923130in}{1.533258in}}%
\pgfpathcurveto{\pgfqpoint{0.917306in}{1.527434in}}{\pgfqpoint{0.914034in}{1.519534in}}{\pgfqpoint{0.914034in}{1.511298in}}%
\pgfpathcurveto{\pgfqpoint{0.914034in}{1.503062in}}{\pgfqpoint{0.917306in}{1.495161in}}{\pgfqpoint{0.923130in}{1.489338in}}%
\pgfpathcurveto{\pgfqpoint{0.928954in}{1.483514in}}{\pgfqpoint{0.936854in}{1.480241in}}{\pgfqpoint{0.945090in}{1.480241in}}%
\pgfpathclose%
\pgfusepath{stroke,fill}%
\end{pgfscope}%
\begin{pgfscope}%
\pgfpathrectangle{\pgfqpoint{0.100000in}{0.212622in}}{\pgfqpoint{3.696000in}{3.696000in}}%
\pgfusepath{clip}%
\pgfsetbuttcap%
\pgfsetroundjoin%
\definecolor{currentfill}{rgb}{0.121569,0.466667,0.705882}%
\pgfsetfillcolor{currentfill}%
\pgfsetfillopacity{0.616906}%
\pgfsetlinewidth{1.003750pt}%
\definecolor{currentstroke}{rgb}{0.121569,0.466667,0.705882}%
\pgfsetstrokecolor{currentstroke}%
\pgfsetstrokeopacity{0.616906}%
\pgfsetdash{}{0pt}%
\pgfpathmoveto{\pgfqpoint{2.119775in}{1.774744in}}%
\pgfpathcurveto{\pgfqpoint{2.128012in}{1.774744in}}{\pgfqpoint{2.135912in}{1.778016in}}{\pgfqpoint{2.141736in}{1.783840in}}%
\pgfpathcurveto{\pgfqpoint{2.147560in}{1.789664in}}{\pgfqpoint{2.150832in}{1.797564in}}{\pgfqpoint{2.150832in}{1.805800in}}%
\pgfpathcurveto{\pgfqpoint{2.150832in}{1.814037in}}{\pgfqpoint{2.147560in}{1.821937in}}{\pgfqpoint{2.141736in}{1.827761in}}%
\pgfpathcurveto{\pgfqpoint{2.135912in}{1.833585in}}{\pgfqpoint{2.128012in}{1.836857in}}{\pgfqpoint{2.119775in}{1.836857in}}%
\pgfpathcurveto{\pgfqpoint{2.111539in}{1.836857in}}{\pgfqpoint{2.103639in}{1.833585in}}{\pgfqpoint{2.097815in}{1.827761in}}%
\pgfpathcurveto{\pgfqpoint{2.091991in}{1.821937in}}{\pgfqpoint{2.088719in}{1.814037in}}{\pgfqpoint{2.088719in}{1.805800in}}%
\pgfpathcurveto{\pgfqpoint{2.088719in}{1.797564in}}{\pgfqpoint{2.091991in}{1.789664in}}{\pgfqpoint{2.097815in}{1.783840in}}%
\pgfpathcurveto{\pgfqpoint{2.103639in}{1.778016in}}{\pgfqpoint{2.111539in}{1.774744in}}{\pgfqpoint{2.119775in}{1.774744in}}%
\pgfpathclose%
\pgfusepath{stroke,fill}%
\end{pgfscope}%
\begin{pgfscope}%
\pgfpathrectangle{\pgfqpoint{0.100000in}{0.212622in}}{\pgfqpoint{3.696000in}{3.696000in}}%
\pgfusepath{clip}%
\pgfsetbuttcap%
\pgfsetroundjoin%
\definecolor{currentfill}{rgb}{0.121569,0.466667,0.705882}%
\pgfsetfillcolor{currentfill}%
\pgfsetfillopacity{0.617418}%
\pgfsetlinewidth{1.003750pt}%
\definecolor{currentstroke}{rgb}{0.121569,0.466667,0.705882}%
\pgfsetstrokecolor{currentstroke}%
\pgfsetstrokeopacity{0.617418}%
\pgfsetdash{}{0pt}%
\pgfpathmoveto{\pgfqpoint{0.942860in}{1.480469in}}%
\pgfpathcurveto{\pgfqpoint{0.951097in}{1.480469in}}{\pgfqpoint{0.958997in}{1.483741in}}{\pgfqpoint{0.964821in}{1.489565in}}%
\pgfpathcurveto{\pgfqpoint{0.970644in}{1.495389in}}{\pgfqpoint{0.973917in}{1.503289in}}{\pgfqpoint{0.973917in}{1.511525in}}%
\pgfpathcurveto{\pgfqpoint{0.973917in}{1.519761in}}{\pgfqpoint{0.970644in}{1.527661in}}{\pgfqpoint{0.964821in}{1.533485in}}%
\pgfpathcurveto{\pgfqpoint{0.958997in}{1.539309in}}{\pgfqpoint{0.951097in}{1.542582in}}{\pgfqpoint{0.942860in}{1.542582in}}%
\pgfpathcurveto{\pgfqpoint{0.934624in}{1.542582in}}{\pgfqpoint{0.926724in}{1.539309in}}{\pgfqpoint{0.920900in}{1.533485in}}%
\pgfpathcurveto{\pgfqpoint{0.915076in}{1.527661in}}{\pgfqpoint{0.911804in}{1.519761in}}{\pgfqpoint{0.911804in}{1.511525in}}%
\pgfpathcurveto{\pgfqpoint{0.911804in}{1.503289in}}{\pgfqpoint{0.915076in}{1.495389in}}{\pgfqpoint{0.920900in}{1.489565in}}%
\pgfpathcurveto{\pgfqpoint{0.926724in}{1.483741in}}{\pgfqpoint{0.934624in}{1.480469in}}{\pgfqpoint{0.942860in}{1.480469in}}%
\pgfpathclose%
\pgfusepath{stroke,fill}%
\end{pgfscope}%
\begin{pgfscope}%
\pgfpathrectangle{\pgfqpoint{0.100000in}{0.212622in}}{\pgfqpoint{3.696000in}{3.696000in}}%
\pgfusepath{clip}%
\pgfsetbuttcap%
\pgfsetroundjoin%
\definecolor{currentfill}{rgb}{0.121569,0.466667,0.705882}%
\pgfsetfillcolor{currentfill}%
\pgfsetfillopacity{0.618133}%
\pgfsetlinewidth{1.003750pt}%
\definecolor{currentstroke}{rgb}{0.121569,0.466667,0.705882}%
\pgfsetstrokecolor{currentstroke}%
\pgfsetstrokeopacity{0.618133}%
\pgfsetdash{}{0pt}%
\pgfpathmoveto{\pgfqpoint{0.939828in}{1.477371in}}%
\pgfpathcurveto{\pgfqpoint{0.948064in}{1.477371in}}{\pgfqpoint{0.955964in}{1.480643in}}{\pgfqpoint{0.961788in}{1.486467in}}%
\pgfpathcurveto{\pgfqpoint{0.967612in}{1.492291in}}{\pgfqpoint{0.970884in}{1.500191in}}{\pgfqpoint{0.970884in}{1.508427in}}%
\pgfpathcurveto{\pgfqpoint{0.970884in}{1.516663in}}{\pgfqpoint{0.967612in}{1.524563in}}{\pgfqpoint{0.961788in}{1.530387in}}%
\pgfpathcurveto{\pgfqpoint{0.955964in}{1.536211in}}{\pgfqpoint{0.948064in}{1.539484in}}{\pgfqpoint{0.939828in}{1.539484in}}%
\pgfpathcurveto{\pgfqpoint{0.931591in}{1.539484in}}{\pgfqpoint{0.923691in}{1.536211in}}{\pgfqpoint{0.917867in}{1.530387in}}%
\pgfpathcurveto{\pgfqpoint{0.912043in}{1.524563in}}{\pgfqpoint{0.908771in}{1.516663in}}{\pgfqpoint{0.908771in}{1.508427in}}%
\pgfpathcurveto{\pgfqpoint{0.908771in}{1.500191in}}{\pgfqpoint{0.912043in}{1.492291in}}{\pgfqpoint{0.917867in}{1.486467in}}%
\pgfpathcurveto{\pgfqpoint{0.923691in}{1.480643in}}{\pgfqpoint{0.931591in}{1.477371in}}{\pgfqpoint{0.939828in}{1.477371in}}%
\pgfpathclose%
\pgfusepath{stroke,fill}%
\end{pgfscope}%
\begin{pgfscope}%
\pgfpathrectangle{\pgfqpoint{0.100000in}{0.212622in}}{\pgfqpoint{3.696000in}{3.696000in}}%
\pgfusepath{clip}%
\pgfsetbuttcap%
\pgfsetroundjoin%
\definecolor{currentfill}{rgb}{0.121569,0.466667,0.705882}%
\pgfsetfillcolor{currentfill}%
\pgfsetfillopacity{0.620064}%
\pgfsetlinewidth{1.003750pt}%
\definecolor{currentstroke}{rgb}{0.121569,0.466667,0.705882}%
\pgfsetstrokecolor{currentstroke}%
\pgfsetstrokeopacity{0.620064}%
\pgfsetdash{}{0pt}%
\pgfpathmoveto{\pgfqpoint{0.935004in}{1.474703in}}%
\pgfpathcurveto{\pgfqpoint{0.943240in}{1.474703in}}{\pgfqpoint{0.951140in}{1.477975in}}{\pgfqpoint{0.956964in}{1.483799in}}%
\pgfpathcurveto{\pgfqpoint{0.962788in}{1.489623in}}{\pgfqpoint{0.966061in}{1.497523in}}{\pgfqpoint{0.966061in}{1.505759in}}%
\pgfpathcurveto{\pgfqpoint{0.966061in}{1.513996in}}{\pgfqpoint{0.962788in}{1.521896in}}{\pgfqpoint{0.956964in}{1.527720in}}%
\pgfpathcurveto{\pgfqpoint{0.951140in}{1.533544in}}{\pgfqpoint{0.943240in}{1.536816in}}{\pgfqpoint{0.935004in}{1.536816in}}%
\pgfpathcurveto{\pgfqpoint{0.926768in}{1.536816in}}{\pgfqpoint{0.918868in}{1.533544in}}{\pgfqpoint{0.913044in}{1.527720in}}%
\pgfpathcurveto{\pgfqpoint{0.907220in}{1.521896in}}{\pgfqpoint{0.903948in}{1.513996in}}{\pgfqpoint{0.903948in}{1.505759in}}%
\pgfpathcurveto{\pgfqpoint{0.903948in}{1.497523in}}{\pgfqpoint{0.907220in}{1.489623in}}{\pgfqpoint{0.913044in}{1.483799in}}%
\pgfpathcurveto{\pgfqpoint{0.918868in}{1.477975in}}{\pgfqpoint{0.926768in}{1.474703in}}{\pgfqpoint{0.935004in}{1.474703in}}%
\pgfpathclose%
\pgfusepath{stroke,fill}%
\end{pgfscope}%
\begin{pgfscope}%
\pgfpathrectangle{\pgfqpoint{0.100000in}{0.212622in}}{\pgfqpoint{3.696000in}{3.696000in}}%
\pgfusepath{clip}%
\pgfsetbuttcap%
\pgfsetroundjoin%
\definecolor{currentfill}{rgb}{0.121569,0.466667,0.705882}%
\pgfsetfillcolor{currentfill}%
\pgfsetfillopacity{0.620949}%
\pgfsetlinewidth{1.003750pt}%
\definecolor{currentstroke}{rgb}{0.121569,0.466667,0.705882}%
\pgfsetstrokecolor{currentstroke}%
\pgfsetstrokeopacity{0.620949}%
\pgfsetdash{}{0pt}%
\pgfpathmoveto{\pgfqpoint{2.120884in}{1.771538in}}%
\pgfpathcurveto{\pgfqpoint{2.129120in}{1.771538in}}{\pgfqpoint{2.137020in}{1.774810in}}{\pgfqpoint{2.142844in}{1.780634in}}%
\pgfpathcurveto{\pgfqpoint{2.148668in}{1.786458in}}{\pgfqpoint{2.151940in}{1.794358in}}{\pgfqpoint{2.151940in}{1.802594in}}%
\pgfpathcurveto{\pgfqpoint{2.151940in}{1.810831in}}{\pgfqpoint{2.148668in}{1.818731in}}{\pgfqpoint{2.142844in}{1.824555in}}%
\pgfpathcurveto{\pgfqpoint{2.137020in}{1.830379in}}{\pgfqpoint{2.129120in}{1.833651in}}{\pgfqpoint{2.120884in}{1.833651in}}%
\pgfpathcurveto{\pgfqpoint{2.112648in}{1.833651in}}{\pgfqpoint{2.104748in}{1.830379in}}{\pgfqpoint{2.098924in}{1.824555in}}%
\pgfpathcurveto{\pgfqpoint{2.093100in}{1.818731in}}{\pgfqpoint{2.089827in}{1.810831in}}{\pgfqpoint{2.089827in}{1.802594in}}%
\pgfpathcurveto{\pgfqpoint{2.089827in}{1.794358in}}{\pgfqpoint{2.093100in}{1.786458in}}{\pgfqpoint{2.098924in}{1.780634in}}%
\pgfpathcurveto{\pgfqpoint{2.104748in}{1.774810in}}{\pgfqpoint{2.112648in}{1.771538in}}{\pgfqpoint{2.120884in}{1.771538in}}%
\pgfpathclose%
\pgfusepath{stroke,fill}%
\end{pgfscope}%
\begin{pgfscope}%
\pgfpathrectangle{\pgfqpoint{0.100000in}{0.212622in}}{\pgfqpoint{3.696000in}{3.696000in}}%
\pgfusepath{clip}%
\pgfsetbuttcap%
\pgfsetroundjoin%
\definecolor{currentfill}{rgb}{0.121569,0.466667,0.705882}%
\pgfsetfillcolor{currentfill}%
\pgfsetfillopacity{0.622949}%
\pgfsetlinewidth{1.003750pt}%
\definecolor{currentstroke}{rgb}{0.121569,0.466667,0.705882}%
\pgfsetstrokecolor{currentstroke}%
\pgfsetstrokeopacity{0.622949}%
\pgfsetdash{}{0pt}%
\pgfpathmoveto{\pgfqpoint{0.927578in}{1.464498in}}%
\pgfpathcurveto{\pgfqpoint{0.935814in}{1.464498in}}{\pgfqpoint{0.943714in}{1.467770in}}{\pgfqpoint{0.949538in}{1.473594in}}%
\pgfpathcurveto{\pgfqpoint{0.955362in}{1.479418in}}{\pgfqpoint{0.958635in}{1.487318in}}{\pgfqpoint{0.958635in}{1.495554in}}%
\pgfpathcurveto{\pgfqpoint{0.958635in}{1.503791in}}{\pgfqpoint{0.955362in}{1.511691in}}{\pgfqpoint{0.949538in}{1.517515in}}%
\pgfpathcurveto{\pgfqpoint{0.943714in}{1.523338in}}{\pgfqpoint{0.935814in}{1.526611in}}{\pgfqpoint{0.927578in}{1.526611in}}%
\pgfpathcurveto{\pgfqpoint{0.919342in}{1.526611in}}{\pgfqpoint{0.911442in}{1.523338in}}{\pgfqpoint{0.905618in}{1.517515in}}%
\pgfpathcurveto{\pgfqpoint{0.899794in}{1.511691in}}{\pgfqpoint{0.896522in}{1.503791in}}{\pgfqpoint{0.896522in}{1.495554in}}%
\pgfpathcurveto{\pgfqpoint{0.896522in}{1.487318in}}{\pgfqpoint{0.899794in}{1.479418in}}{\pgfqpoint{0.905618in}{1.473594in}}%
\pgfpathcurveto{\pgfqpoint{0.911442in}{1.467770in}}{\pgfqpoint{0.919342in}{1.464498in}}{\pgfqpoint{0.927578in}{1.464498in}}%
\pgfpathclose%
\pgfusepath{stroke,fill}%
\end{pgfscope}%
\begin{pgfscope}%
\pgfpathrectangle{\pgfqpoint{0.100000in}{0.212622in}}{\pgfqpoint{3.696000in}{3.696000in}}%
\pgfusepath{clip}%
\pgfsetbuttcap%
\pgfsetroundjoin%
\definecolor{currentfill}{rgb}{0.121569,0.466667,0.705882}%
\pgfsetfillcolor{currentfill}%
\pgfsetfillopacity{0.625268}%
\pgfsetlinewidth{1.003750pt}%
\definecolor{currentstroke}{rgb}{0.121569,0.466667,0.705882}%
\pgfsetstrokecolor{currentstroke}%
\pgfsetstrokeopacity{0.625268}%
\pgfsetdash{}{0pt}%
\pgfpathmoveto{\pgfqpoint{2.124490in}{1.769936in}}%
\pgfpathcurveto{\pgfqpoint{2.132727in}{1.769936in}}{\pgfqpoint{2.140627in}{1.773208in}}{\pgfqpoint{2.146451in}{1.779032in}}%
\pgfpathcurveto{\pgfqpoint{2.152275in}{1.784856in}}{\pgfqpoint{2.155547in}{1.792756in}}{\pgfqpoint{2.155547in}{1.800992in}}%
\pgfpathcurveto{\pgfqpoint{2.155547in}{1.809229in}}{\pgfqpoint{2.152275in}{1.817129in}}{\pgfqpoint{2.146451in}{1.822953in}}%
\pgfpathcurveto{\pgfqpoint{2.140627in}{1.828777in}}{\pgfqpoint{2.132727in}{1.832049in}}{\pgfqpoint{2.124490in}{1.832049in}}%
\pgfpathcurveto{\pgfqpoint{2.116254in}{1.832049in}}{\pgfqpoint{2.108354in}{1.828777in}}{\pgfqpoint{2.102530in}{1.822953in}}%
\pgfpathcurveto{\pgfqpoint{2.096706in}{1.817129in}}{\pgfqpoint{2.093434in}{1.809229in}}{\pgfqpoint{2.093434in}{1.800992in}}%
\pgfpathcurveto{\pgfqpoint{2.093434in}{1.792756in}}{\pgfqpoint{2.096706in}{1.784856in}}{\pgfqpoint{2.102530in}{1.779032in}}%
\pgfpathcurveto{\pgfqpoint{2.108354in}{1.773208in}}{\pgfqpoint{2.116254in}{1.769936in}}{\pgfqpoint{2.124490in}{1.769936in}}%
\pgfpathclose%
\pgfusepath{stroke,fill}%
\end{pgfscope}%
\begin{pgfscope}%
\pgfpathrectangle{\pgfqpoint{0.100000in}{0.212622in}}{\pgfqpoint{3.696000in}{3.696000in}}%
\pgfusepath{clip}%
\pgfsetbuttcap%
\pgfsetroundjoin%
\definecolor{currentfill}{rgb}{0.121569,0.466667,0.705882}%
\pgfsetfillcolor{currentfill}%
\pgfsetfillopacity{0.625909}%
\pgfsetlinewidth{1.003750pt}%
\definecolor{currentstroke}{rgb}{0.121569,0.466667,0.705882}%
\pgfsetstrokecolor{currentstroke}%
\pgfsetstrokeopacity{0.625909}%
\pgfsetdash{}{0pt}%
\pgfpathmoveto{\pgfqpoint{0.917662in}{1.462795in}}%
\pgfpathcurveto{\pgfqpoint{0.925898in}{1.462795in}}{\pgfqpoint{0.933798in}{1.466067in}}{\pgfqpoint{0.939622in}{1.471891in}}%
\pgfpathcurveto{\pgfqpoint{0.945446in}{1.477715in}}{\pgfqpoint{0.948718in}{1.485615in}}{\pgfqpoint{0.948718in}{1.493851in}}%
\pgfpathcurveto{\pgfqpoint{0.948718in}{1.502087in}}{\pgfqpoint{0.945446in}{1.509987in}}{\pgfqpoint{0.939622in}{1.515811in}}%
\pgfpathcurveto{\pgfqpoint{0.933798in}{1.521635in}}{\pgfqpoint{0.925898in}{1.524908in}}{\pgfqpoint{0.917662in}{1.524908in}}%
\pgfpathcurveto{\pgfqpoint{0.909425in}{1.524908in}}{\pgfqpoint{0.901525in}{1.521635in}}{\pgfqpoint{0.895701in}{1.515811in}}%
\pgfpathcurveto{\pgfqpoint{0.889877in}{1.509987in}}{\pgfqpoint{0.886605in}{1.502087in}}{\pgfqpoint{0.886605in}{1.493851in}}%
\pgfpathcurveto{\pgfqpoint{0.886605in}{1.485615in}}{\pgfqpoint{0.889877in}{1.477715in}}{\pgfqpoint{0.895701in}{1.471891in}}%
\pgfpathcurveto{\pgfqpoint{0.901525in}{1.466067in}}{\pgfqpoint{0.909425in}{1.462795in}}{\pgfqpoint{0.917662in}{1.462795in}}%
\pgfpathclose%
\pgfusepath{stroke,fill}%
\end{pgfscope}%
\begin{pgfscope}%
\pgfpathrectangle{\pgfqpoint{0.100000in}{0.212622in}}{\pgfqpoint{3.696000in}{3.696000in}}%
\pgfusepath{clip}%
\pgfsetbuttcap%
\pgfsetroundjoin%
\definecolor{currentfill}{rgb}{0.121569,0.466667,0.705882}%
\pgfsetfillcolor{currentfill}%
\pgfsetfillopacity{0.629325}%
\pgfsetlinewidth{1.003750pt}%
\definecolor{currentstroke}{rgb}{0.121569,0.466667,0.705882}%
\pgfsetstrokecolor{currentstroke}%
\pgfsetstrokeopacity{0.629325}%
\pgfsetdash{}{0pt}%
\pgfpathmoveto{\pgfqpoint{0.912270in}{1.460371in}}%
\pgfpathcurveto{\pgfqpoint{0.920507in}{1.460371in}}{\pgfqpoint{0.928407in}{1.463643in}}{\pgfqpoint{0.934231in}{1.469467in}}%
\pgfpathcurveto{\pgfqpoint{0.940055in}{1.475291in}}{\pgfqpoint{0.943327in}{1.483191in}}{\pgfqpoint{0.943327in}{1.491427in}}%
\pgfpathcurveto{\pgfqpoint{0.943327in}{1.499664in}}{\pgfqpoint{0.940055in}{1.507564in}}{\pgfqpoint{0.934231in}{1.513388in}}%
\pgfpathcurveto{\pgfqpoint{0.928407in}{1.519212in}}{\pgfqpoint{0.920507in}{1.522484in}}{\pgfqpoint{0.912270in}{1.522484in}}%
\pgfpathcurveto{\pgfqpoint{0.904034in}{1.522484in}}{\pgfqpoint{0.896134in}{1.519212in}}{\pgfqpoint{0.890310in}{1.513388in}}%
\pgfpathcurveto{\pgfqpoint{0.884486in}{1.507564in}}{\pgfqpoint{0.881214in}{1.499664in}}{\pgfqpoint{0.881214in}{1.491427in}}%
\pgfpathcurveto{\pgfqpoint{0.881214in}{1.483191in}}{\pgfqpoint{0.884486in}{1.475291in}}{\pgfqpoint{0.890310in}{1.469467in}}%
\pgfpathcurveto{\pgfqpoint{0.896134in}{1.463643in}}{\pgfqpoint{0.904034in}{1.460371in}}{\pgfqpoint{0.912270in}{1.460371in}}%
\pgfpathclose%
\pgfusepath{stroke,fill}%
\end{pgfscope}%
\begin{pgfscope}%
\pgfpathrectangle{\pgfqpoint{0.100000in}{0.212622in}}{\pgfqpoint{3.696000in}{3.696000in}}%
\pgfusepath{clip}%
\pgfsetbuttcap%
\pgfsetroundjoin%
\definecolor{currentfill}{rgb}{0.121569,0.466667,0.705882}%
\pgfsetfillcolor{currentfill}%
\pgfsetfillopacity{0.629707}%
\pgfsetlinewidth{1.003750pt}%
\definecolor{currentstroke}{rgb}{0.121569,0.466667,0.705882}%
\pgfsetstrokecolor{currentstroke}%
\pgfsetstrokeopacity{0.629707}%
\pgfsetdash{}{0pt}%
\pgfpathmoveto{\pgfqpoint{2.127846in}{1.766891in}}%
\pgfpathcurveto{\pgfqpoint{2.136082in}{1.766891in}}{\pgfqpoint{2.143982in}{1.770164in}}{\pgfqpoint{2.149806in}{1.775987in}}%
\pgfpathcurveto{\pgfqpoint{2.155630in}{1.781811in}}{\pgfqpoint{2.158903in}{1.789711in}}{\pgfqpoint{2.158903in}{1.797948in}}%
\pgfpathcurveto{\pgfqpoint{2.158903in}{1.806184in}}{\pgfqpoint{2.155630in}{1.814084in}}{\pgfqpoint{2.149806in}{1.819908in}}%
\pgfpathcurveto{\pgfqpoint{2.143982in}{1.825732in}}{\pgfqpoint{2.136082in}{1.829004in}}{\pgfqpoint{2.127846in}{1.829004in}}%
\pgfpathcurveto{\pgfqpoint{2.119610in}{1.829004in}}{\pgfqpoint{2.111710in}{1.825732in}}{\pgfqpoint{2.105886in}{1.819908in}}%
\pgfpathcurveto{\pgfqpoint{2.100062in}{1.814084in}}{\pgfqpoint{2.096790in}{1.806184in}}{\pgfqpoint{2.096790in}{1.797948in}}%
\pgfpathcurveto{\pgfqpoint{2.096790in}{1.789711in}}{\pgfqpoint{2.100062in}{1.781811in}}{\pgfqpoint{2.105886in}{1.775987in}}%
\pgfpathcurveto{\pgfqpoint{2.111710in}{1.770164in}}{\pgfqpoint{2.119610in}{1.766891in}}{\pgfqpoint{2.127846in}{1.766891in}}%
\pgfpathclose%
\pgfusepath{stroke,fill}%
\end{pgfscope}%
\begin{pgfscope}%
\pgfpathrectangle{\pgfqpoint{0.100000in}{0.212622in}}{\pgfqpoint{3.696000in}{3.696000in}}%
\pgfusepath{clip}%
\pgfsetbuttcap%
\pgfsetroundjoin%
\definecolor{currentfill}{rgb}{0.121569,0.466667,0.705882}%
\pgfsetfillcolor{currentfill}%
\pgfsetfillopacity{0.631256}%
\pgfsetlinewidth{1.003750pt}%
\definecolor{currentstroke}{rgb}{0.121569,0.466667,0.705882}%
\pgfsetstrokecolor{currentstroke}%
\pgfsetstrokeopacity{0.631256}%
\pgfsetdash{}{0pt}%
\pgfpathmoveto{\pgfqpoint{0.906328in}{1.458911in}}%
\pgfpathcurveto{\pgfqpoint{0.914565in}{1.458911in}}{\pgfqpoint{0.922465in}{1.462183in}}{\pgfqpoint{0.928289in}{1.468007in}}%
\pgfpathcurveto{\pgfqpoint{0.934113in}{1.473831in}}{\pgfqpoint{0.937385in}{1.481731in}}{\pgfqpoint{0.937385in}{1.489967in}}%
\pgfpathcurveto{\pgfqpoint{0.937385in}{1.498203in}}{\pgfqpoint{0.934113in}{1.506103in}}{\pgfqpoint{0.928289in}{1.511927in}}%
\pgfpathcurveto{\pgfqpoint{0.922465in}{1.517751in}}{\pgfqpoint{0.914565in}{1.521024in}}{\pgfqpoint{0.906328in}{1.521024in}}%
\pgfpathcurveto{\pgfqpoint{0.898092in}{1.521024in}}{\pgfqpoint{0.890192in}{1.517751in}}{\pgfqpoint{0.884368in}{1.511927in}}%
\pgfpathcurveto{\pgfqpoint{0.878544in}{1.506103in}}{\pgfqpoint{0.875272in}{1.498203in}}{\pgfqpoint{0.875272in}{1.489967in}}%
\pgfpathcurveto{\pgfqpoint{0.875272in}{1.481731in}}{\pgfqpoint{0.878544in}{1.473831in}}{\pgfqpoint{0.884368in}{1.468007in}}%
\pgfpathcurveto{\pgfqpoint{0.890192in}{1.462183in}}{\pgfqpoint{0.898092in}{1.458911in}}{\pgfqpoint{0.906328in}{1.458911in}}%
\pgfpathclose%
\pgfusepath{stroke,fill}%
\end{pgfscope}%
\begin{pgfscope}%
\pgfpathrectangle{\pgfqpoint{0.100000in}{0.212622in}}{\pgfqpoint{3.696000in}{3.696000in}}%
\pgfusepath{clip}%
\pgfsetbuttcap%
\pgfsetroundjoin%
\definecolor{currentfill}{rgb}{0.121569,0.466667,0.705882}%
\pgfsetfillcolor{currentfill}%
\pgfsetfillopacity{0.631975}%
\pgfsetlinewidth{1.003750pt}%
\definecolor{currentstroke}{rgb}{0.121569,0.466667,0.705882}%
\pgfsetstrokecolor{currentstroke}%
\pgfsetstrokeopacity{0.631975}%
\pgfsetdash{}{0pt}%
\pgfpathmoveto{\pgfqpoint{2.128955in}{1.763689in}}%
\pgfpathcurveto{\pgfqpoint{2.137192in}{1.763689in}}{\pgfqpoint{2.145092in}{1.766961in}}{\pgfqpoint{2.150916in}{1.772785in}}%
\pgfpathcurveto{\pgfqpoint{2.156740in}{1.778609in}}{\pgfqpoint{2.160012in}{1.786509in}}{\pgfqpoint{2.160012in}{1.794745in}}%
\pgfpathcurveto{\pgfqpoint{2.160012in}{1.802982in}}{\pgfqpoint{2.156740in}{1.810882in}}{\pgfqpoint{2.150916in}{1.816706in}}%
\pgfpathcurveto{\pgfqpoint{2.145092in}{1.822530in}}{\pgfqpoint{2.137192in}{1.825802in}}{\pgfqpoint{2.128955in}{1.825802in}}%
\pgfpathcurveto{\pgfqpoint{2.120719in}{1.825802in}}{\pgfqpoint{2.112819in}{1.822530in}}{\pgfqpoint{2.106995in}{1.816706in}}%
\pgfpathcurveto{\pgfqpoint{2.101171in}{1.810882in}}{\pgfqpoint{2.097899in}{1.802982in}}{\pgfqpoint{2.097899in}{1.794745in}}%
\pgfpathcurveto{\pgfqpoint{2.097899in}{1.786509in}}{\pgfqpoint{2.101171in}{1.778609in}}{\pgfqpoint{2.106995in}{1.772785in}}%
\pgfpathcurveto{\pgfqpoint{2.112819in}{1.766961in}}{\pgfqpoint{2.120719in}{1.763689in}}{\pgfqpoint{2.128955in}{1.763689in}}%
\pgfpathclose%
\pgfusepath{stroke,fill}%
\end{pgfscope}%
\begin{pgfscope}%
\pgfpathrectangle{\pgfqpoint{0.100000in}{0.212622in}}{\pgfqpoint{3.696000in}{3.696000in}}%
\pgfusepath{clip}%
\pgfsetbuttcap%
\pgfsetroundjoin%
\definecolor{currentfill}{rgb}{0.121569,0.466667,0.705882}%
\pgfsetfillcolor{currentfill}%
\pgfsetfillopacity{0.633182}%
\pgfsetlinewidth{1.003750pt}%
\definecolor{currentstroke}{rgb}{0.121569,0.466667,0.705882}%
\pgfsetstrokecolor{currentstroke}%
\pgfsetstrokeopacity{0.633182}%
\pgfsetdash{}{0pt}%
\pgfpathmoveto{\pgfqpoint{2.129900in}{1.761819in}}%
\pgfpathcurveto{\pgfqpoint{2.138136in}{1.761819in}}{\pgfqpoint{2.146036in}{1.765091in}}{\pgfqpoint{2.151860in}{1.770915in}}%
\pgfpathcurveto{\pgfqpoint{2.157684in}{1.776739in}}{\pgfqpoint{2.160956in}{1.784639in}}{\pgfqpoint{2.160956in}{1.792875in}}%
\pgfpathcurveto{\pgfqpoint{2.160956in}{1.801112in}}{\pgfqpoint{2.157684in}{1.809012in}}{\pgfqpoint{2.151860in}{1.814836in}}%
\pgfpathcurveto{\pgfqpoint{2.146036in}{1.820660in}}{\pgfqpoint{2.138136in}{1.823932in}}{\pgfqpoint{2.129900in}{1.823932in}}%
\pgfpathcurveto{\pgfqpoint{2.121663in}{1.823932in}}{\pgfqpoint{2.113763in}{1.820660in}}{\pgfqpoint{2.107939in}{1.814836in}}%
\pgfpathcurveto{\pgfqpoint{2.102115in}{1.809012in}}{\pgfqpoint{2.098843in}{1.801112in}}{\pgfqpoint{2.098843in}{1.792875in}}%
\pgfpathcurveto{\pgfqpoint{2.098843in}{1.784639in}}{\pgfqpoint{2.102115in}{1.776739in}}{\pgfqpoint{2.107939in}{1.770915in}}%
\pgfpathcurveto{\pgfqpoint{2.113763in}{1.765091in}}{\pgfqpoint{2.121663in}{1.761819in}}{\pgfqpoint{2.129900in}{1.761819in}}%
\pgfpathclose%
\pgfusepath{stroke,fill}%
\end{pgfscope}%
\begin{pgfscope}%
\pgfpathrectangle{\pgfqpoint{0.100000in}{0.212622in}}{\pgfqpoint{3.696000in}{3.696000in}}%
\pgfusepath{clip}%
\pgfsetbuttcap%
\pgfsetroundjoin%
\definecolor{currentfill}{rgb}{0.121569,0.466667,0.705882}%
\pgfsetfillcolor{currentfill}%
\pgfsetfillopacity{0.633815}%
\pgfsetlinewidth{1.003750pt}%
\definecolor{currentstroke}{rgb}{0.121569,0.466667,0.705882}%
\pgfsetstrokecolor{currentstroke}%
\pgfsetstrokeopacity{0.633815}%
\pgfsetdash{}{0pt}%
\pgfpathmoveto{\pgfqpoint{0.903552in}{1.459761in}}%
\pgfpathcurveto{\pgfqpoint{0.911788in}{1.459761in}}{\pgfqpoint{0.919688in}{1.463033in}}{\pgfqpoint{0.925512in}{1.468857in}}%
\pgfpathcurveto{\pgfqpoint{0.931336in}{1.474681in}}{\pgfqpoint{0.934608in}{1.482581in}}{\pgfqpoint{0.934608in}{1.490817in}}%
\pgfpathcurveto{\pgfqpoint{0.934608in}{1.499054in}}{\pgfqpoint{0.931336in}{1.506954in}}{\pgfqpoint{0.925512in}{1.512778in}}%
\pgfpathcurveto{\pgfqpoint{0.919688in}{1.518602in}}{\pgfqpoint{0.911788in}{1.521874in}}{\pgfqpoint{0.903552in}{1.521874in}}%
\pgfpathcurveto{\pgfqpoint{0.895315in}{1.521874in}}{\pgfqpoint{0.887415in}{1.518602in}}{\pgfqpoint{0.881591in}{1.512778in}}%
\pgfpathcurveto{\pgfqpoint{0.875767in}{1.506954in}}{\pgfqpoint{0.872495in}{1.499054in}}{\pgfqpoint{0.872495in}{1.490817in}}%
\pgfpathcurveto{\pgfqpoint{0.872495in}{1.482581in}}{\pgfqpoint{0.875767in}{1.474681in}}{\pgfqpoint{0.881591in}{1.468857in}}%
\pgfpathcurveto{\pgfqpoint{0.887415in}{1.463033in}}{\pgfqpoint{0.895315in}{1.459761in}}{\pgfqpoint{0.903552in}{1.459761in}}%
\pgfpathclose%
\pgfusepath{stroke,fill}%
\end{pgfscope}%
\begin{pgfscope}%
\pgfpathrectangle{\pgfqpoint{0.100000in}{0.212622in}}{\pgfqpoint{3.696000in}{3.696000in}}%
\pgfusepath{clip}%
\pgfsetbuttcap%
\pgfsetroundjoin%
\definecolor{currentfill}{rgb}{0.121569,0.466667,0.705882}%
\pgfsetfillcolor{currentfill}%
\pgfsetfillopacity{0.634476}%
\pgfsetlinewidth{1.003750pt}%
\definecolor{currentstroke}{rgb}{0.121569,0.466667,0.705882}%
\pgfsetstrokecolor{currentstroke}%
\pgfsetstrokeopacity{0.634476}%
\pgfsetdash{}{0pt}%
\pgfpathmoveto{\pgfqpoint{0.900798in}{1.457275in}}%
\pgfpathcurveto{\pgfqpoint{0.909035in}{1.457275in}}{\pgfqpoint{0.916935in}{1.460547in}}{\pgfqpoint{0.922759in}{1.466371in}}%
\pgfpathcurveto{\pgfqpoint{0.928583in}{1.472195in}}{\pgfqpoint{0.931855in}{1.480095in}}{\pgfqpoint{0.931855in}{1.488331in}}%
\pgfpathcurveto{\pgfqpoint{0.931855in}{1.496567in}}{\pgfqpoint{0.928583in}{1.504467in}}{\pgfqpoint{0.922759in}{1.510291in}}%
\pgfpathcurveto{\pgfqpoint{0.916935in}{1.516115in}}{\pgfqpoint{0.909035in}{1.519388in}}{\pgfqpoint{0.900798in}{1.519388in}}%
\pgfpathcurveto{\pgfqpoint{0.892562in}{1.519388in}}{\pgfqpoint{0.884662in}{1.516115in}}{\pgfqpoint{0.878838in}{1.510291in}}%
\pgfpathcurveto{\pgfqpoint{0.873014in}{1.504467in}}{\pgfqpoint{0.869742in}{1.496567in}}{\pgfqpoint{0.869742in}{1.488331in}}%
\pgfpathcurveto{\pgfqpoint{0.869742in}{1.480095in}}{\pgfqpoint{0.873014in}{1.472195in}}{\pgfqpoint{0.878838in}{1.466371in}}%
\pgfpathcurveto{\pgfqpoint{0.884662in}{1.460547in}}{\pgfqpoint{0.892562in}{1.457275in}}{\pgfqpoint{0.900798in}{1.457275in}}%
\pgfpathclose%
\pgfusepath{stroke,fill}%
\end{pgfscope}%
\begin{pgfscope}%
\pgfpathrectangle{\pgfqpoint{0.100000in}{0.212622in}}{\pgfqpoint{3.696000in}{3.696000in}}%
\pgfusepath{clip}%
\pgfsetbuttcap%
\pgfsetroundjoin%
\definecolor{currentfill}{rgb}{0.121569,0.466667,0.705882}%
\pgfsetfillcolor{currentfill}%
\pgfsetfillopacity{0.634718}%
\pgfsetlinewidth{1.003750pt}%
\definecolor{currentstroke}{rgb}{0.121569,0.466667,0.705882}%
\pgfsetstrokecolor{currentstroke}%
\pgfsetstrokeopacity{0.634718}%
\pgfsetdash{}{0pt}%
\pgfpathmoveto{\pgfqpoint{2.130922in}{1.760473in}}%
\pgfpathcurveto{\pgfqpoint{2.139158in}{1.760473in}}{\pgfqpoint{2.147058in}{1.763745in}}{\pgfqpoint{2.152882in}{1.769569in}}%
\pgfpathcurveto{\pgfqpoint{2.158706in}{1.775393in}}{\pgfqpoint{2.161978in}{1.783293in}}{\pgfqpoint{2.161978in}{1.791530in}}%
\pgfpathcurveto{\pgfqpoint{2.161978in}{1.799766in}}{\pgfqpoint{2.158706in}{1.807666in}}{\pgfqpoint{2.152882in}{1.813490in}}%
\pgfpathcurveto{\pgfqpoint{2.147058in}{1.819314in}}{\pgfqpoint{2.139158in}{1.822586in}}{\pgfqpoint{2.130922in}{1.822586in}}%
\pgfpathcurveto{\pgfqpoint{2.122686in}{1.822586in}}{\pgfqpoint{2.114785in}{1.819314in}}{\pgfqpoint{2.108962in}{1.813490in}}%
\pgfpathcurveto{\pgfqpoint{2.103138in}{1.807666in}}{\pgfqpoint{2.099865in}{1.799766in}}{\pgfqpoint{2.099865in}{1.791530in}}%
\pgfpathcurveto{\pgfqpoint{2.099865in}{1.783293in}}{\pgfqpoint{2.103138in}{1.775393in}}{\pgfqpoint{2.108962in}{1.769569in}}%
\pgfpathcurveto{\pgfqpoint{2.114785in}{1.763745in}}{\pgfqpoint{2.122686in}{1.760473in}}{\pgfqpoint{2.130922in}{1.760473in}}%
\pgfpathclose%
\pgfusepath{stroke,fill}%
\end{pgfscope}%
\begin{pgfscope}%
\pgfpathrectangle{\pgfqpoint{0.100000in}{0.212622in}}{\pgfqpoint{3.696000in}{3.696000in}}%
\pgfusepath{clip}%
\pgfsetbuttcap%
\pgfsetroundjoin%
\definecolor{currentfill}{rgb}{0.121569,0.466667,0.705882}%
\pgfsetfillcolor{currentfill}%
\pgfsetfillopacity{0.636151}%
\pgfsetlinewidth{1.003750pt}%
\definecolor{currentstroke}{rgb}{0.121569,0.466667,0.705882}%
\pgfsetstrokecolor{currentstroke}%
\pgfsetstrokeopacity{0.636151}%
\pgfsetdash{}{0pt}%
\pgfpathmoveto{\pgfqpoint{0.896622in}{1.454535in}}%
\pgfpathcurveto{\pgfqpoint{0.904858in}{1.454535in}}{\pgfqpoint{0.912758in}{1.457807in}}{\pgfqpoint{0.918582in}{1.463631in}}%
\pgfpathcurveto{\pgfqpoint{0.924406in}{1.469455in}}{\pgfqpoint{0.927678in}{1.477355in}}{\pgfqpoint{0.927678in}{1.485591in}}%
\pgfpathcurveto{\pgfqpoint{0.927678in}{1.493827in}}{\pgfqpoint{0.924406in}{1.501727in}}{\pgfqpoint{0.918582in}{1.507551in}}%
\pgfpathcurveto{\pgfqpoint{0.912758in}{1.513375in}}{\pgfqpoint{0.904858in}{1.516648in}}{\pgfqpoint{0.896622in}{1.516648in}}%
\pgfpathcurveto{\pgfqpoint{0.888385in}{1.516648in}}{\pgfqpoint{0.880485in}{1.513375in}}{\pgfqpoint{0.874661in}{1.507551in}}%
\pgfpathcurveto{\pgfqpoint{0.868838in}{1.501727in}}{\pgfqpoint{0.865565in}{1.493827in}}{\pgfqpoint{0.865565in}{1.485591in}}%
\pgfpathcurveto{\pgfqpoint{0.865565in}{1.477355in}}{\pgfqpoint{0.868838in}{1.469455in}}{\pgfqpoint{0.874661in}{1.463631in}}%
\pgfpathcurveto{\pgfqpoint{0.880485in}{1.457807in}}{\pgfqpoint{0.888385in}{1.454535in}}{\pgfqpoint{0.896622in}{1.454535in}}%
\pgfpathclose%
\pgfusepath{stroke,fill}%
\end{pgfscope}%
\begin{pgfscope}%
\pgfpathrectangle{\pgfqpoint{0.100000in}{0.212622in}}{\pgfqpoint{3.696000in}{3.696000in}}%
\pgfusepath{clip}%
\pgfsetbuttcap%
\pgfsetroundjoin%
\definecolor{currentfill}{rgb}{0.121569,0.466667,0.705882}%
\pgfsetfillcolor{currentfill}%
\pgfsetfillopacity{0.636672}%
\pgfsetlinewidth{1.003750pt}%
\definecolor{currentstroke}{rgb}{0.121569,0.466667,0.705882}%
\pgfsetstrokecolor{currentstroke}%
\pgfsetstrokeopacity{0.636672}%
\pgfsetdash{}{0pt}%
\pgfpathmoveto{\pgfqpoint{2.132402in}{1.759646in}}%
\pgfpathcurveto{\pgfqpoint{2.140638in}{1.759646in}}{\pgfqpoint{2.148538in}{1.762919in}}{\pgfqpoint{2.154362in}{1.768743in}}%
\pgfpathcurveto{\pgfqpoint{2.160186in}{1.774566in}}{\pgfqpoint{2.163458in}{1.782466in}}{\pgfqpoint{2.163458in}{1.790703in}}%
\pgfpathcurveto{\pgfqpoint{2.163458in}{1.798939in}}{\pgfqpoint{2.160186in}{1.806839in}}{\pgfqpoint{2.154362in}{1.812663in}}%
\pgfpathcurveto{\pgfqpoint{2.148538in}{1.818487in}}{\pgfqpoint{2.140638in}{1.821759in}}{\pgfqpoint{2.132402in}{1.821759in}}%
\pgfpathcurveto{\pgfqpoint{2.124165in}{1.821759in}}{\pgfqpoint{2.116265in}{1.818487in}}{\pgfqpoint{2.110441in}{1.812663in}}%
\pgfpathcurveto{\pgfqpoint{2.104617in}{1.806839in}}{\pgfqpoint{2.101345in}{1.798939in}}{\pgfqpoint{2.101345in}{1.790703in}}%
\pgfpathcurveto{\pgfqpoint{2.101345in}{1.782466in}}{\pgfqpoint{2.104617in}{1.774566in}}{\pgfqpoint{2.110441in}{1.768743in}}%
\pgfpathcurveto{\pgfqpoint{2.116265in}{1.762919in}}{\pgfqpoint{2.124165in}{1.759646in}}{\pgfqpoint{2.132402in}{1.759646in}}%
\pgfpathclose%
\pgfusepath{stroke,fill}%
\end{pgfscope}%
\begin{pgfscope}%
\pgfpathrectangle{\pgfqpoint{0.100000in}{0.212622in}}{\pgfqpoint{3.696000in}{3.696000in}}%
\pgfusepath{clip}%
\pgfsetbuttcap%
\pgfsetroundjoin%
\definecolor{currentfill}{rgb}{0.121569,0.466667,0.705882}%
\pgfsetfillcolor{currentfill}%
\pgfsetfillopacity{0.637620}%
\pgfsetlinewidth{1.003750pt}%
\definecolor{currentstroke}{rgb}{0.121569,0.466667,0.705882}%
\pgfsetstrokecolor{currentstroke}%
\pgfsetstrokeopacity{0.637620}%
\pgfsetdash{}{0pt}%
\pgfpathmoveto{\pgfqpoint{2.133065in}{1.758276in}}%
\pgfpathcurveto{\pgfqpoint{2.141301in}{1.758276in}}{\pgfqpoint{2.149201in}{1.761548in}}{\pgfqpoint{2.155025in}{1.767372in}}%
\pgfpathcurveto{\pgfqpoint{2.160849in}{1.773196in}}{\pgfqpoint{2.164121in}{1.781096in}}{\pgfqpoint{2.164121in}{1.789332in}}%
\pgfpathcurveto{\pgfqpoint{2.164121in}{1.797569in}}{\pgfqpoint{2.160849in}{1.805469in}}{\pgfqpoint{2.155025in}{1.811293in}}%
\pgfpathcurveto{\pgfqpoint{2.149201in}{1.817117in}}{\pgfqpoint{2.141301in}{1.820389in}}{\pgfqpoint{2.133065in}{1.820389in}}%
\pgfpathcurveto{\pgfqpoint{2.124828in}{1.820389in}}{\pgfqpoint{2.116928in}{1.817117in}}{\pgfqpoint{2.111104in}{1.811293in}}%
\pgfpathcurveto{\pgfqpoint{2.105280in}{1.805469in}}{\pgfqpoint{2.102008in}{1.797569in}}{\pgfqpoint{2.102008in}{1.789332in}}%
\pgfpathcurveto{\pgfqpoint{2.102008in}{1.781096in}}{\pgfqpoint{2.105280in}{1.773196in}}{\pgfqpoint{2.111104in}{1.767372in}}%
\pgfpathcurveto{\pgfqpoint{2.116928in}{1.761548in}}{\pgfqpoint{2.124828in}{1.758276in}}{\pgfqpoint{2.133065in}{1.758276in}}%
\pgfpathclose%
\pgfusepath{stroke,fill}%
\end{pgfscope}%
\begin{pgfscope}%
\pgfpathrectangle{\pgfqpoint{0.100000in}{0.212622in}}{\pgfqpoint{3.696000in}{3.696000in}}%
\pgfusepath{clip}%
\pgfsetbuttcap%
\pgfsetroundjoin%
\definecolor{currentfill}{rgb}{0.121569,0.466667,0.705882}%
\pgfsetfillcolor{currentfill}%
\pgfsetfillopacity{0.638988}%
\pgfsetlinewidth{1.003750pt}%
\definecolor{currentstroke}{rgb}{0.121569,0.466667,0.705882}%
\pgfsetstrokecolor{currentstroke}%
\pgfsetstrokeopacity{0.638988}%
\pgfsetdash{}{0pt}%
\pgfpathmoveto{\pgfqpoint{0.890019in}{1.447221in}}%
\pgfpathcurveto{\pgfqpoint{0.898255in}{1.447221in}}{\pgfqpoint{0.906155in}{1.450493in}}{\pgfqpoint{0.911979in}{1.456317in}}%
\pgfpathcurveto{\pgfqpoint{0.917803in}{1.462141in}}{\pgfqpoint{0.921075in}{1.470041in}}{\pgfqpoint{0.921075in}{1.478277in}}%
\pgfpathcurveto{\pgfqpoint{0.921075in}{1.486514in}}{\pgfqpoint{0.917803in}{1.494414in}}{\pgfqpoint{0.911979in}{1.500238in}}%
\pgfpathcurveto{\pgfqpoint{0.906155in}{1.506062in}}{\pgfqpoint{0.898255in}{1.509334in}}{\pgfqpoint{0.890019in}{1.509334in}}%
\pgfpathcurveto{\pgfqpoint{0.881783in}{1.509334in}}{\pgfqpoint{0.873883in}{1.506062in}}{\pgfqpoint{0.868059in}{1.500238in}}%
\pgfpathcurveto{\pgfqpoint{0.862235in}{1.494414in}}{\pgfqpoint{0.858962in}{1.486514in}}{\pgfqpoint{0.858962in}{1.478277in}}%
\pgfpathcurveto{\pgfqpoint{0.858962in}{1.470041in}}{\pgfqpoint{0.862235in}{1.462141in}}{\pgfqpoint{0.868059in}{1.456317in}}%
\pgfpathcurveto{\pgfqpoint{0.873883in}{1.450493in}}{\pgfqpoint{0.881783in}{1.447221in}}{\pgfqpoint{0.890019in}{1.447221in}}%
\pgfpathclose%
\pgfusepath{stroke,fill}%
\end{pgfscope}%
\begin{pgfscope}%
\pgfpathrectangle{\pgfqpoint{0.100000in}{0.212622in}}{\pgfqpoint{3.696000in}{3.696000in}}%
\pgfusepath{clip}%
\pgfsetbuttcap%
\pgfsetroundjoin%
\definecolor{currentfill}{rgb}{0.121569,0.466667,0.705882}%
\pgfsetfillcolor{currentfill}%
\pgfsetfillopacity{0.639074}%
\pgfsetlinewidth{1.003750pt}%
\definecolor{currentstroke}{rgb}{0.121569,0.466667,0.705882}%
\pgfsetstrokecolor{currentstroke}%
\pgfsetstrokeopacity{0.639074}%
\pgfsetdash{}{0pt}%
\pgfpathmoveto{\pgfqpoint{2.133907in}{1.756317in}}%
\pgfpathcurveto{\pgfqpoint{2.142143in}{1.756317in}}{\pgfqpoint{2.150043in}{1.759589in}}{\pgfqpoint{2.155867in}{1.765413in}}%
\pgfpathcurveto{\pgfqpoint{2.161691in}{1.771237in}}{\pgfqpoint{2.164964in}{1.779137in}}{\pgfqpoint{2.164964in}{1.787373in}}%
\pgfpathcurveto{\pgfqpoint{2.164964in}{1.795609in}}{\pgfqpoint{2.161691in}{1.803510in}}{\pgfqpoint{2.155867in}{1.809333in}}%
\pgfpathcurveto{\pgfqpoint{2.150043in}{1.815157in}}{\pgfqpoint{2.142143in}{1.818430in}}{\pgfqpoint{2.133907in}{1.818430in}}%
\pgfpathcurveto{\pgfqpoint{2.125671in}{1.818430in}}{\pgfqpoint{2.117771in}{1.815157in}}{\pgfqpoint{2.111947in}{1.809333in}}%
\pgfpathcurveto{\pgfqpoint{2.106123in}{1.803510in}}{\pgfqpoint{2.102851in}{1.795609in}}{\pgfqpoint{2.102851in}{1.787373in}}%
\pgfpathcurveto{\pgfqpoint{2.102851in}{1.779137in}}{\pgfqpoint{2.106123in}{1.771237in}}{\pgfqpoint{2.111947in}{1.765413in}}%
\pgfpathcurveto{\pgfqpoint{2.117771in}{1.759589in}}{\pgfqpoint{2.125671in}{1.756317in}}{\pgfqpoint{2.133907in}{1.756317in}}%
\pgfpathclose%
\pgfusepath{stroke,fill}%
\end{pgfscope}%
\begin{pgfscope}%
\pgfpathrectangle{\pgfqpoint{0.100000in}{0.212622in}}{\pgfqpoint{3.696000in}{3.696000in}}%
\pgfusepath{clip}%
\pgfsetbuttcap%
\pgfsetroundjoin%
\definecolor{currentfill}{rgb}{0.121569,0.466667,0.705882}%
\pgfsetfillcolor{currentfill}%
\pgfsetfillopacity{0.640975}%
\pgfsetlinewidth{1.003750pt}%
\definecolor{currentstroke}{rgb}{0.121569,0.466667,0.705882}%
\pgfsetstrokecolor{currentstroke}%
\pgfsetstrokeopacity{0.640975}%
\pgfsetdash{}{0pt}%
\pgfpathmoveto{\pgfqpoint{2.135227in}{1.754900in}}%
\pgfpathcurveto{\pgfqpoint{2.143463in}{1.754900in}}{\pgfqpoint{2.151363in}{1.758173in}}{\pgfqpoint{2.157187in}{1.763996in}}%
\pgfpathcurveto{\pgfqpoint{2.163011in}{1.769820in}}{\pgfqpoint{2.166283in}{1.777720in}}{\pgfqpoint{2.166283in}{1.785957in}}%
\pgfpathcurveto{\pgfqpoint{2.166283in}{1.794193in}}{\pgfqpoint{2.163011in}{1.802093in}}{\pgfqpoint{2.157187in}{1.807917in}}%
\pgfpathcurveto{\pgfqpoint{2.151363in}{1.813741in}}{\pgfqpoint{2.143463in}{1.817013in}}{\pgfqpoint{2.135227in}{1.817013in}}%
\pgfpathcurveto{\pgfqpoint{2.126990in}{1.817013in}}{\pgfqpoint{2.119090in}{1.813741in}}{\pgfqpoint{2.113266in}{1.807917in}}%
\pgfpathcurveto{\pgfqpoint{2.107442in}{1.802093in}}{\pgfqpoint{2.104170in}{1.794193in}}{\pgfqpoint{2.104170in}{1.785957in}}%
\pgfpathcurveto{\pgfqpoint{2.104170in}{1.777720in}}{\pgfqpoint{2.107442in}{1.769820in}}{\pgfqpoint{2.113266in}{1.763996in}}%
\pgfpathcurveto{\pgfqpoint{2.119090in}{1.758173in}}{\pgfqpoint{2.126990in}{1.754900in}}{\pgfqpoint{2.135227in}{1.754900in}}%
\pgfpathclose%
\pgfusepath{stroke,fill}%
\end{pgfscope}%
\begin{pgfscope}%
\pgfpathrectangle{\pgfqpoint{0.100000in}{0.212622in}}{\pgfqpoint{3.696000in}{3.696000in}}%
\pgfusepath{clip}%
\pgfsetbuttcap%
\pgfsetroundjoin%
\definecolor{currentfill}{rgb}{0.121569,0.466667,0.705882}%
\pgfsetfillcolor{currentfill}%
\pgfsetfillopacity{0.641499}%
\pgfsetlinewidth{1.003750pt}%
\definecolor{currentstroke}{rgb}{0.121569,0.466667,0.705882}%
\pgfsetstrokecolor{currentstroke}%
\pgfsetstrokeopacity{0.641499}%
\pgfsetdash{}{0pt}%
\pgfpathmoveto{\pgfqpoint{0.881627in}{1.446707in}}%
\pgfpathcurveto{\pgfqpoint{0.889863in}{1.446707in}}{\pgfqpoint{0.897763in}{1.449980in}}{\pgfqpoint{0.903587in}{1.455804in}}%
\pgfpathcurveto{\pgfqpoint{0.909411in}{1.461628in}}{\pgfqpoint{0.912683in}{1.469528in}}{\pgfqpoint{0.912683in}{1.477764in}}%
\pgfpathcurveto{\pgfqpoint{0.912683in}{1.486000in}}{\pgfqpoint{0.909411in}{1.493900in}}{\pgfqpoint{0.903587in}{1.499724in}}%
\pgfpathcurveto{\pgfqpoint{0.897763in}{1.505548in}}{\pgfqpoint{0.889863in}{1.508820in}}{\pgfqpoint{0.881627in}{1.508820in}}%
\pgfpathcurveto{\pgfqpoint{0.873390in}{1.508820in}}{\pgfqpoint{0.865490in}{1.505548in}}{\pgfqpoint{0.859666in}{1.499724in}}%
\pgfpathcurveto{\pgfqpoint{0.853842in}{1.493900in}}{\pgfqpoint{0.850570in}{1.486000in}}{\pgfqpoint{0.850570in}{1.477764in}}%
\pgfpathcurveto{\pgfqpoint{0.850570in}{1.469528in}}{\pgfqpoint{0.853842in}{1.461628in}}{\pgfqpoint{0.859666in}{1.455804in}}%
\pgfpathcurveto{\pgfqpoint{0.865490in}{1.449980in}}{\pgfqpoint{0.873390in}{1.446707in}}{\pgfqpoint{0.881627in}{1.446707in}}%
\pgfpathclose%
\pgfusepath{stroke,fill}%
\end{pgfscope}%
\begin{pgfscope}%
\pgfpathrectangle{\pgfqpoint{0.100000in}{0.212622in}}{\pgfqpoint{3.696000in}{3.696000in}}%
\pgfusepath{clip}%
\pgfsetbuttcap%
\pgfsetroundjoin%
\definecolor{currentfill}{rgb}{0.121569,0.466667,0.705882}%
\pgfsetfillcolor{currentfill}%
\pgfsetfillopacity{0.643765}%
\pgfsetlinewidth{1.003750pt}%
\definecolor{currentstroke}{rgb}{0.121569,0.466667,0.705882}%
\pgfsetstrokecolor{currentstroke}%
\pgfsetstrokeopacity{0.643765}%
\pgfsetdash{}{0pt}%
\pgfpathmoveto{\pgfqpoint{2.137615in}{1.755086in}}%
\pgfpathcurveto{\pgfqpoint{2.145851in}{1.755086in}}{\pgfqpoint{2.153751in}{1.758358in}}{\pgfqpoint{2.159575in}{1.764182in}}%
\pgfpathcurveto{\pgfqpoint{2.165399in}{1.770006in}}{\pgfqpoint{2.168671in}{1.777906in}}{\pgfqpoint{2.168671in}{1.786142in}}%
\pgfpathcurveto{\pgfqpoint{2.168671in}{1.794378in}}{\pgfqpoint{2.165399in}{1.802278in}}{\pgfqpoint{2.159575in}{1.808102in}}%
\pgfpathcurveto{\pgfqpoint{2.153751in}{1.813926in}}{\pgfqpoint{2.145851in}{1.817199in}}{\pgfqpoint{2.137615in}{1.817199in}}%
\pgfpathcurveto{\pgfqpoint{2.129378in}{1.817199in}}{\pgfqpoint{2.121478in}{1.813926in}}{\pgfqpoint{2.115654in}{1.808102in}}%
\pgfpathcurveto{\pgfqpoint{2.109831in}{1.802278in}}{\pgfqpoint{2.106558in}{1.794378in}}{\pgfqpoint{2.106558in}{1.786142in}}%
\pgfpathcurveto{\pgfqpoint{2.106558in}{1.777906in}}{\pgfqpoint{2.109831in}{1.770006in}}{\pgfqpoint{2.115654in}{1.764182in}}%
\pgfpathcurveto{\pgfqpoint{2.121478in}{1.758358in}}{\pgfqpoint{2.129378in}{1.755086in}}{\pgfqpoint{2.137615in}{1.755086in}}%
\pgfpathclose%
\pgfusepath{stroke,fill}%
\end{pgfscope}%
\begin{pgfscope}%
\pgfpathrectangle{\pgfqpoint{0.100000in}{0.212622in}}{\pgfqpoint{3.696000in}{3.696000in}}%
\pgfusepath{clip}%
\pgfsetbuttcap%
\pgfsetroundjoin%
\definecolor{currentfill}{rgb}{0.121569,0.466667,0.705882}%
\pgfsetfillcolor{currentfill}%
\pgfsetfillopacity{0.643826}%
\pgfsetlinewidth{1.003750pt}%
\definecolor{currentstroke}{rgb}{0.121569,0.466667,0.705882}%
\pgfsetstrokecolor{currentstroke}%
\pgfsetstrokeopacity{0.643826}%
\pgfsetdash{}{0pt}%
\pgfpathmoveto{\pgfqpoint{0.878188in}{1.445732in}}%
\pgfpathcurveto{\pgfqpoint{0.886424in}{1.445732in}}{\pgfqpoint{0.894324in}{1.449004in}}{\pgfqpoint{0.900148in}{1.454828in}}%
\pgfpathcurveto{\pgfqpoint{0.905972in}{1.460652in}}{\pgfqpoint{0.909244in}{1.468552in}}{\pgfqpoint{0.909244in}{1.476788in}}%
\pgfpathcurveto{\pgfqpoint{0.909244in}{1.485025in}}{\pgfqpoint{0.905972in}{1.492925in}}{\pgfqpoint{0.900148in}{1.498749in}}%
\pgfpathcurveto{\pgfqpoint{0.894324in}{1.504573in}}{\pgfqpoint{0.886424in}{1.507845in}}{\pgfqpoint{0.878188in}{1.507845in}}%
\pgfpathcurveto{\pgfqpoint{0.869951in}{1.507845in}}{\pgfqpoint{0.862051in}{1.504573in}}{\pgfqpoint{0.856227in}{1.498749in}}%
\pgfpathcurveto{\pgfqpoint{0.850404in}{1.492925in}}{\pgfqpoint{0.847131in}{1.485025in}}{\pgfqpoint{0.847131in}{1.476788in}}%
\pgfpathcurveto{\pgfqpoint{0.847131in}{1.468552in}}{\pgfqpoint{0.850404in}{1.460652in}}{\pgfqpoint{0.856227in}{1.454828in}}%
\pgfpathcurveto{\pgfqpoint{0.862051in}{1.449004in}}{\pgfqpoint{0.869951in}{1.445732in}}{\pgfqpoint{0.878188in}{1.445732in}}%
\pgfpathclose%
\pgfusepath{stroke,fill}%
\end{pgfscope}%
\begin{pgfscope}%
\pgfpathrectangle{\pgfqpoint{0.100000in}{0.212622in}}{\pgfqpoint{3.696000in}{3.696000in}}%
\pgfusepath{clip}%
\pgfsetbuttcap%
\pgfsetroundjoin%
\definecolor{currentfill}{rgb}{0.121569,0.466667,0.705882}%
\pgfsetfillcolor{currentfill}%
\pgfsetfillopacity{0.644912}%
\pgfsetlinewidth{1.003750pt}%
\definecolor{currentstroke}{rgb}{0.121569,0.466667,0.705882}%
\pgfsetstrokecolor{currentstroke}%
\pgfsetstrokeopacity{0.644912}%
\pgfsetdash{}{0pt}%
\pgfpathmoveto{\pgfqpoint{0.874753in}{1.445666in}}%
\pgfpathcurveto{\pgfqpoint{0.882989in}{1.445666in}}{\pgfqpoint{0.890889in}{1.448938in}}{\pgfqpoint{0.896713in}{1.454762in}}%
\pgfpathcurveto{\pgfqpoint{0.902537in}{1.460586in}}{\pgfqpoint{0.905810in}{1.468486in}}{\pgfqpoint{0.905810in}{1.476722in}}%
\pgfpathcurveto{\pgfqpoint{0.905810in}{1.484959in}}{\pgfqpoint{0.902537in}{1.492859in}}{\pgfqpoint{0.896713in}{1.498683in}}%
\pgfpathcurveto{\pgfqpoint{0.890889in}{1.504507in}}{\pgfqpoint{0.882989in}{1.507779in}}{\pgfqpoint{0.874753in}{1.507779in}}%
\pgfpathcurveto{\pgfqpoint{0.866517in}{1.507779in}}{\pgfqpoint{0.858617in}{1.504507in}}{\pgfqpoint{0.852793in}{1.498683in}}%
\pgfpathcurveto{\pgfqpoint{0.846969in}{1.492859in}}{\pgfqpoint{0.843697in}{1.484959in}}{\pgfqpoint{0.843697in}{1.476722in}}%
\pgfpathcurveto{\pgfqpoint{0.843697in}{1.468486in}}{\pgfqpoint{0.846969in}{1.460586in}}{\pgfqpoint{0.852793in}{1.454762in}}%
\pgfpathcurveto{\pgfqpoint{0.858617in}{1.448938in}}{\pgfqpoint{0.866517in}{1.445666in}}{\pgfqpoint{0.874753in}{1.445666in}}%
\pgfpathclose%
\pgfusepath{stroke,fill}%
\end{pgfscope}%
\begin{pgfscope}%
\pgfpathrectangle{\pgfqpoint{0.100000in}{0.212622in}}{\pgfqpoint{3.696000in}{3.696000in}}%
\pgfusepath{clip}%
\pgfsetbuttcap%
\pgfsetroundjoin%
\definecolor{currentfill}{rgb}{0.121569,0.466667,0.705882}%
\pgfsetfillcolor{currentfill}%
\pgfsetfillopacity{0.645676}%
\pgfsetlinewidth{1.003750pt}%
\definecolor{currentstroke}{rgb}{0.121569,0.466667,0.705882}%
\pgfsetstrokecolor{currentstroke}%
\pgfsetstrokeopacity{0.645676}%
\pgfsetdash{}{0pt}%
\pgfpathmoveto{\pgfqpoint{0.873671in}{1.446204in}}%
\pgfpathcurveto{\pgfqpoint{0.881907in}{1.446204in}}{\pgfqpoint{0.889808in}{1.449477in}}{\pgfqpoint{0.895631in}{1.455301in}}%
\pgfpathcurveto{\pgfqpoint{0.901455in}{1.461124in}}{\pgfqpoint{0.904728in}{1.469024in}}{\pgfqpoint{0.904728in}{1.477261in}}%
\pgfpathcurveto{\pgfqpoint{0.904728in}{1.485497in}}{\pgfqpoint{0.901455in}{1.493397in}}{\pgfqpoint{0.895631in}{1.499221in}}%
\pgfpathcurveto{\pgfqpoint{0.889808in}{1.505045in}}{\pgfqpoint{0.881907in}{1.508317in}}{\pgfqpoint{0.873671in}{1.508317in}}%
\pgfpathcurveto{\pgfqpoint{0.865435in}{1.508317in}}{\pgfqpoint{0.857535in}{1.505045in}}{\pgfqpoint{0.851711in}{1.499221in}}%
\pgfpathcurveto{\pgfqpoint{0.845887in}{1.493397in}}{\pgfqpoint{0.842615in}{1.485497in}}{\pgfqpoint{0.842615in}{1.477261in}}%
\pgfpathcurveto{\pgfqpoint{0.842615in}{1.469024in}}{\pgfqpoint{0.845887in}{1.461124in}}{\pgfqpoint{0.851711in}{1.455301in}}%
\pgfpathcurveto{\pgfqpoint{0.857535in}{1.449477in}}{\pgfqpoint{0.865435in}{1.446204in}}{\pgfqpoint{0.873671in}{1.446204in}}%
\pgfpathclose%
\pgfusepath{stroke,fill}%
\end{pgfscope}%
\begin{pgfscope}%
\pgfpathrectangle{\pgfqpoint{0.100000in}{0.212622in}}{\pgfqpoint{3.696000in}{3.696000in}}%
\pgfusepath{clip}%
\pgfsetbuttcap%
\pgfsetroundjoin%
\definecolor{currentfill}{rgb}{0.121569,0.466667,0.705882}%
\pgfsetfillcolor{currentfill}%
\pgfsetfillopacity{0.645676}%
\pgfsetlinewidth{1.003750pt}%
\definecolor{currentstroke}{rgb}{0.121569,0.466667,0.705882}%
\pgfsetstrokecolor{currentstroke}%
\pgfsetstrokeopacity{0.645676}%
\pgfsetdash{}{0pt}%
\pgfpathmoveto{\pgfqpoint{0.873671in}{1.446204in}}%
\pgfpathcurveto{\pgfqpoint{0.881907in}{1.446204in}}{\pgfqpoint{0.889807in}{1.449477in}}{\pgfqpoint{0.895631in}{1.455300in}}%
\pgfpathcurveto{\pgfqpoint{0.901455in}{1.461124in}}{\pgfqpoint{0.904728in}{1.469024in}}{\pgfqpoint{0.904728in}{1.477261in}}%
\pgfpathcurveto{\pgfqpoint{0.904728in}{1.485497in}}{\pgfqpoint{0.901455in}{1.493397in}}{\pgfqpoint{0.895631in}{1.499221in}}%
\pgfpathcurveto{\pgfqpoint{0.889807in}{1.505045in}}{\pgfqpoint{0.881907in}{1.508317in}}{\pgfqpoint{0.873671in}{1.508317in}}%
\pgfpathcurveto{\pgfqpoint{0.865435in}{1.508317in}}{\pgfqpoint{0.857535in}{1.505045in}}{\pgfqpoint{0.851711in}{1.499221in}}%
\pgfpathcurveto{\pgfqpoint{0.845887in}{1.493397in}}{\pgfqpoint{0.842615in}{1.485497in}}{\pgfqpoint{0.842615in}{1.477261in}}%
\pgfpathcurveto{\pgfqpoint{0.842615in}{1.469024in}}{\pgfqpoint{0.845887in}{1.461124in}}{\pgfqpoint{0.851711in}{1.455300in}}%
\pgfpathcurveto{\pgfqpoint{0.857535in}{1.449477in}}{\pgfqpoint{0.865435in}{1.446204in}}{\pgfqpoint{0.873671in}{1.446204in}}%
\pgfpathclose%
\pgfusepath{stroke,fill}%
\end{pgfscope}%
\begin{pgfscope}%
\pgfpathrectangle{\pgfqpoint{0.100000in}{0.212622in}}{\pgfqpoint{3.696000in}{3.696000in}}%
\pgfusepath{clip}%
\pgfsetbuttcap%
\pgfsetroundjoin%
\definecolor{currentfill}{rgb}{0.121569,0.466667,0.705882}%
\pgfsetfillcolor{currentfill}%
\pgfsetfillopacity{0.645676}%
\pgfsetlinewidth{1.003750pt}%
\definecolor{currentstroke}{rgb}{0.121569,0.466667,0.705882}%
\pgfsetstrokecolor{currentstroke}%
\pgfsetstrokeopacity{0.645676}%
\pgfsetdash{}{0pt}%
\pgfpathmoveto{\pgfqpoint{0.873671in}{1.446204in}}%
\pgfpathcurveto{\pgfqpoint{0.881907in}{1.446204in}}{\pgfqpoint{0.889807in}{1.449476in}}{\pgfqpoint{0.895631in}{1.455300in}}%
\pgfpathcurveto{\pgfqpoint{0.901455in}{1.461124in}}{\pgfqpoint{0.904728in}{1.469024in}}{\pgfqpoint{0.904728in}{1.477261in}}%
\pgfpathcurveto{\pgfqpoint{0.904728in}{1.485497in}}{\pgfqpoint{0.901455in}{1.493397in}}{\pgfqpoint{0.895631in}{1.499221in}}%
\pgfpathcurveto{\pgfqpoint{0.889807in}{1.505045in}}{\pgfqpoint{0.881907in}{1.508317in}}{\pgfqpoint{0.873671in}{1.508317in}}%
\pgfpathcurveto{\pgfqpoint{0.865435in}{1.508317in}}{\pgfqpoint{0.857535in}{1.505045in}}{\pgfqpoint{0.851711in}{1.499221in}}%
\pgfpathcurveto{\pgfqpoint{0.845887in}{1.493397in}}{\pgfqpoint{0.842615in}{1.485497in}}{\pgfqpoint{0.842615in}{1.477261in}}%
\pgfpathcurveto{\pgfqpoint{0.842615in}{1.469024in}}{\pgfqpoint{0.845887in}{1.461124in}}{\pgfqpoint{0.851711in}{1.455300in}}%
\pgfpathcurveto{\pgfqpoint{0.857535in}{1.449476in}}{\pgfqpoint{0.865435in}{1.446204in}}{\pgfqpoint{0.873671in}{1.446204in}}%
\pgfpathclose%
\pgfusepath{stroke,fill}%
\end{pgfscope}%
\begin{pgfscope}%
\pgfpathrectangle{\pgfqpoint{0.100000in}{0.212622in}}{\pgfqpoint{3.696000in}{3.696000in}}%
\pgfusepath{clip}%
\pgfsetbuttcap%
\pgfsetroundjoin%
\definecolor{currentfill}{rgb}{0.121569,0.466667,0.705882}%
\pgfsetfillcolor{currentfill}%
\pgfsetfillopacity{0.645676}%
\pgfsetlinewidth{1.003750pt}%
\definecolor{currentstroke}{rgb}{0.121569,0.466667,0.705882}%
\pgfsetstrokecolor{currentstroke}%
\pgfsetstrokeopacity{0.645676}%
\pgfsetdash{}{0pt}%
\pgfpathmoveto{\pgfqpoint{0.873671in}{1.446204in}}%
\pgfpathcurveto{\pgfqpoint{0.881907in}{1.446204in}}{\pgfqpoint{0.889807in}{1.449476in}}{\pgfqpoint{0.895631in}{1.455300in}}%
\pgfpathcurveto{\pgfqpoint{0.901455in}{1.461124in}}{\pgfqpoint{0.904727in}{1.469024in}}{\pgfqpoint{0.904727in}{1.477261in}}%
\pgfpathcurveto{\pgfqpoint{0.904727in}{1.485497in}}{\pgfqpoint{0.901455in}{1.493397in}}{\pgfqpoint{0.895631in}{1.499221in}}%
\pgfpathcurveto{\pgfqpoint{0.889807in}{1.505045in}}{\pgfqpoint{0.881907in}{1.508317in}}{\pgfqpoint{0.873671in}{1.508317in}}%
\pgfpathcurveto{\pgfqpoint{0.865435in}{1.508317in}}{\pgfqpoint{0.857535in}{1.505045in}}{\pgfqpoint{0.851711in}{1.499221in}}%
\pgfpathcurveto{\pgfqpoint{0.845887in}{1.493397in}}{\pgfqpoint{0.842614in}{1.485497in}}{\pgfqpoint{0.842614in}{1.477261in}}%
\pgfpathcurveto{\pgfqpoint{0.842614in}{1.469024in}}{\pgfqpoint{0.845887in}{1.461124in}}{\pgfqpoint{0.851711in}{1.455300in}}%
\pgfpathcurveto{\pgfqpoint{0.857535in}{1.449476in}}{\pgfqpoint{0.865435in}{1.446204in}}{\pgfqpoint{0.873671in}{1.446204in}}%
\pgfpathclose%
\pgfusepath{stroke,fill}%
\end{pgfscope}%
\begin{pgfscope}%
\pgfpathrectangle{\pgfqpoint{0.100000in}{0.212622in}}{\pgfqpoint{3.696000in}{3.696000in}}%
\pgfusepath{clip}%
\pgfsetbuttcap%
\pgfsetroundjoin%
\definecolor{currentfill}{rgb}{0.121569,0.466667,0.705882}%
\pgfsetfillcolor{currentfill}%
\pgfsetfillopacity{0.645677}%
\pgfsetlinewidth{1.003750pt}%
\definecolor{currentstroke}{rgb}{0.121569,0.466667,0.705882}%
\pgfsetstrokecolor{currentstroke}%
\pgfsetstrokeopacity{0.645677}%
\pgfsetdash{}{0pt}%
\pgfpathmoveto{\pgfqpoint{0.873671in}{1.446204in}}%
\pgfpathcurveto{\pgfqpoint{0.881907in}{1.446204in}}{\pgfqpoint{0.889807in}{1.449476in}}{\pgfqpoint{0.895631in}{1.455300in}}%
\pgfpathcurveto{\pgfqpoint{0.901455in}{1.461124in}}{\pgfqpoint{0.904727in}{1.469024in}}{\pgfqpoint{0.904727in}{1.477260in}}%
\pgfpathcurveto{\pgfqpoint{0.904727in}{1.485497in}}{\pgfqpoint{0.901455in}{1.493397in}}{\pgfqpoint{0.895631in}{1.499221in}}%
\pgfpathcurveto{\pgfqpoint{0.889807in}{1.505045in}}{\pgfqpoint{0.881907in}{1.508317in}}{\pgfqpoint{0.873671in}{1.508317in}}%
\pgfpathcurveto{\pgfqpoint{0.865434in}{1.508317in}}{\pgfqpoint{0.857534in}{1.505045in}}{\pgfqpoint{0.851710in}{1.499221in}}%
\pgfpathcurveto{\pgfqpoint{0.845886in}{1.493397in}}{\pgfqpoint{0.842614in}{1.485497in}}{\pgfqpoint{0.842614in}{1.477260in}}%
\pgfpathcurveto{\pgfqpoint{0.842614in}{1.469024in}}{\pgfqpoint{0.845886in}{1.461124in}}{\pgfqpoint{0.851710in}{1.455300in}}%
\pgfpathcurveto{\pgfqpoint{0.857534in}{1.449476in}}{\pgfqpoint{0.865434in}{1.446204in}}{\pgfqpoint{0.873671in}{1.446204in}}%
\pgfpathclose%
\pgfusepath{stroke,fill}%
\end{pgfscope}%
\begin{pgfscope}%
\pgfpathrectangle{\pgfqpoint{0.100000in}{0.212622in}}{\pgfqpoint{3.696000in}{3.696000in}}%
\pgfusepath{clip}%
\pgfsetbuttcap%
\pgfsetroundjoin%
\definecolor{currentfill}{rgb}{0.121569,0.466667,0.705882}%
\pgfsetfillcolor{currentfill}%
\pgfsetfillopacity{0.645677}%
\pgfsetlinewidth{1.003750pt}%
\definecolor{currentstroke}{rgb}{0.121569,0.466667,0.705882}%
\pgfsetstrokecolor{currentstroke}%
\pgfsetstrokeopacity{0.645677}%
\pgfsetdash{}{0pt}%
\pgfpathmoveto{\pgfqpoint{0.873670in}{1.446203in}}%
\pgfpathcurveto{\pgfqpoint{0.881906in}{1.446203in}}{\pgfqpoint{0.889806in}{1.449476in}}{\pgfqpoint{0.895630in}{1.455300in}}%
\pgfpathcurveto{\pgfqpoint{0.901454in}{1.461124in}}{\pgfqpoint{0.904726in}{1.469024in}}{\pgfqpoint{0.904726in}{1.477260in}}%
\pgfpathcurveto{\pgfqpoint{0.904726in}{1.485496in}}{\pgfqpoint{0.901454in}{1.493396in}}{\pgfqpoint{0.895630in}{1.499220in}}%
\pgfpathcurveto{\pgfqpoint{0.889806in}{1.505044in}}{\pgfqpoint{0.881906in}{1.508316in}}{\pgfqpoint{0.873670in}{1.508316in}}%
\pgfpathcurveto{\pgfqpoint{0.865434in}{1.508316in}}{\pgfqpoint{0.857534in}{1.505044in}}{\pgfqpoint{0.851710in}{1.499220in}}%
\pgfpathcurveto{\pgfqpoint{0.845886in}{1.493396in}}{\pgfqpoint{0.842613in}{1.485496in}}{\pgfqpoint{0.842613in}{1.477260in}}%
\pgfpathcurveto{\pgfqpoint{0.842613in}{1.469024in}}{\pgfqpoint{0.845886in}{1.461124in}}{\pgfqpoint{0.851710in}{1.455300in}}%
\pgfpathcurveto{\pgfqpoint{0.857534in}{1.449476in}}{\pgfqpoint{0.865434in}{1.446203in}}{\pgfqpoint{0.873670in}{1.446203in}}%
\pgfpathclose%
\pgfusepath{stroke,fill}%
\end{pgfscope}%
\begin{pgfscope}%
\pgfpathrectangle{\pgfqpoint{0.100000in}{0.212622in}}{\pgfqpoint{3.696000in}{3.696000in}}%
\pgfusepath{clip}%
\pgfsetbuttcap%
\pgfsetroundjoin%
\definecolor{currentfill}{rgb}{0.121569,0.466667,0.705882}%
\pgfsetfillcolor{currentfill}%
\pgfsetfillopacity{0.645677}%
\pgfsetlinewidth{1.003750pt}%
\definecolor{currentstroke}{rgb}{0.121569,0.466667,0.705882}%
\pgfsetstrokecolor{currentstroke}%
\pgfsetstrokeopacity{0.645677}%
\pgfsetdash{}{0pt}%
\pgfpathmoveto{\pgfqpoint{0.873669in}{1.446203in}}%
\pgfpathcurveto{\pgfqpoint{0.881905in}{1.446203in}}{\pgfqpoint{0.889805in}{1.449475in}}{\pgfqpoint{0.895629in}{1.455299in}}%
\pgfpathcurveto{\pgfqpoint{0.901453in}{1.461123in}}{\pgfqpoint{0.904725in}{1.469023in}}{\pgfqpoint{0.904725in}{1.477260in}}%
\pgfpathcurveto{\pgfqpoint{0.904725in}{1.485496in}}{\pgfqpoint{0.901453in}{1.493396in}}{\pgfqpoint{0.895629in}{1.499220in}}%
\pgfpathcurveto{\pgfqpoint{0.889805in}{1.505044in}}{\pgfqpoint{0.881905in}{1.508316in}}{\pgfqpoint{0.873669in}{1.508316in}}%
\pgfpathcurveto{\pgfqpoint{0.865433in}{1.508316in}}{\pgfqpoint{0.857533in}{1.505044in}}{\pgfqpoint{0.851709in}{1.499220in}}%
\pgfpathcurveto{\pgfqpoint{0.845885in}{1.493396in}}{\pgfqpoint{0.842612in}{1.485496in}}{\pgfqpoint{0.842612in}{1.477260in}}%
\pgfpathcurveto{\pgfqpoint{0.842612in}{1.469023in}}{\pgfqpoint{0.845885in}{1.461123in}}{\pgfqpoint{0.851709in}{1.455299in}}%
\pgfpathcurveto{\pgfqpoint{0.857533in}{1.449475in}}{\pgfqpoint{0.865433in}{1.446203in}}{\pgfqpoint{0.873669in}{1.446203in}}%
\pgfpathclose%
\pgfusepath{stroke,fill}%
\end{pgfscope}%
\begin{pgfscope}%
\pgfpathrectangle{\pgfqpoint{0.100000in}{0.212622in}}{\pgfqpoint{3.696000in}{3.696000in}}%
\pgfusepath{clip}%
\pgfsetbuttcap%
\pgfsetroundjoin%
\definecolor{currentfill}{rgb}{0.121569,0.466667,0.705882}%
\pgfsetfillcolor{currentfill}%
\pgfsetfillopacity{0.645678}%
\pgfsetlinewidth{1.003750pt}%
\definecolor{currentstroke}{rgb}{0.121569,0.466667,0.705882}%
\pgfsetstrokecolor{currentstroke}%
\pgfsetstrokeopacity{0.645678}%
\pgfsetdash{}{0pt}%
\pgfpathmoveto{\pgfqpoint{0.873667in}{1.446202in}}%
\pgfpathcurveto{\pgfqpoint{0.881903in}{1.446202in}}{\pgfqpoint{0.889803in}{1.449474in}}{\pgfqpoint{0.895627in}{1.455298in}}%
\pgfpathcurveto{\pgfqpoint{0.901451in}{1.461122in}}{\pgfqpoint{0.904724in}{1.469022in}}{\pgfqpoint{0.904724in}{1.477258in}}%
\pgfpathcurveto{\pgfqpoint{0.904724in}{1.485495in}}{\pgfqpoint{0.901451in}{1.493395in}}{\pgfqpoint{0.895627in}{1.499219in}}%
\pgfpathcurveto{\pgfqpoint{0.889803in}{1.505043in}}{\pgfqpoint{0.881903in}{1.508315in}}{\pgfqpoint{0.873667in}{1.508315in}}%
\pgfpathcurveto{\pgfqpoint{0.865431in}{1.508315in}}{\pgfqpoint{0.857531in}{1.505043in}}{\pgfqpoint{0.851707in}{1.499219in}}%
\pgfpathcurveto{\pgfqpoint{0.845883in}{1.493395in}}{\pgfqpoint{0.842611in}{1.485495in}}{\pgfqpoint{0.842611in}{1.477258in}}%
\pgfpathcurveto{\pgfqpoint{0.842611in}{1.469022in}}{\pgfqpoint{0.845883in}{1.461122in}}{\pgfqpoint{0.851707in}{1.455298in}}%
\pgfpathcurveto{\pgfqpoint{0.857531in}{1.449474in}}{\pgfqpoint{0.865431in}{1.446202in}}{\pgfqpoint{0.873667in}{1.446202in}}%
\pgfpathclose%
\pgfusepath{stroke,fill}%
\end{pgfscope}%
\begin{pgfscope}%
\pgfpathrectangle{\pgfqpoint{0.100000in}{0.212622in}}{\pgfqpoint{3.696000in}{3.696000in}}%
\pgfusepath{clip}%
\pgfsetbuttcap%
\pgfsetroundjoin%
\definecolor{currentfill}{rgb}{0.121569,0.466667,0.705882}%
\pgfsetfillcolor{currentfill}%
\pgfsetfillopacity{0.645679}%
\pgfsetlinewidth{1.003750pt}%
\definecolor{currentstroke}{rgb}{0.121569,0.466667,0.705882}%
\pgfsetstrokecolor{currentstroke}%
\pgfsetstrokeopacity{0.645679}%
\pgfsetdash{}{0pt}%
\pgfpathmoveto{\pgfqpoint{0.873663in}{1.446199in}}%
\pgfpathcurveto{\pgfqpoint{0.881900in}{1.446199in}}{\pgfqpoint{0.889800in}{1.449472in}}{\pgfqpoint{0.895624in}{1.455296in}}%
\pgfpathcurveto{\pgfqpoint{0.901448in}{1.461120in}}{\pgfqpoint{0.904720in}{1.469020in}}{\pgfqpoint{0.904720in}{1.477256in}}%
\pgfpathcurveto{\pgfqpoint{0.904720in}{1.485492in}}{\pgfqpoint{0.901448in}{1.493392in}}{\pgfqpoint{0.895624in}{1.499216in}}%
\pgfpathcurveto{\pgfqpoint{0.889800in}{1.505040in}}{\pgfqpoint{0.881900in}{1.508312in}}{\pgfqpoint{0.873663in}{1.508312in}}%
\pgfpathcurveto{\pgfqpoint{0.865427in}{1.508312in}}{\pgfqpoint{0.857527in}{1.505040in}}{\pgfqpoint{0.851703in}{1.499216in}}%
\pgfpathcurveto{\pgfqpoint{0.845879in}{1.493392in}}{\pgfqpoint{0.842607in}{1.485492in}}{\pgfqpoint{0.842607in}{1.477256in}}%
\pgfpathcurveto{\pgfqpoint{0.842607in}{1.469020in}}{\pgfqpoint{0.845879in}{1.461120in}}{\pgfqpoint{0.851703in}{1.455296in}}%
\pgfpathcurveto{\pgfqpoint{0.857527in}{1.449472in}}{\pgfqpoint{0.865427in}{1.446199in}}{\pgfqpoint{0.873663in}{1.446199in}}%
\pgfpathclose%
\pgfusepath{stroke,fill}%
\end{pgfscope}%
\begin{pgfscope}%
\pgfpathrectangle{\pgfqpoint{0.100000in}{0.212622in}}{\pgfqpoint{3.696000in}{3.696000in}}%
\pgfusepath{clip}%
\pgfsetbuttcap%
\pgfsetroundjoin%
\definecolor{currentfill}{rgb}{0.121569,0.466667,0.705882}%
\pgfsetfillcolor{currentfill}%
\pgfsetfillopacity{0.645681}%
\pgfsetlinewidth{1.003750pt}%
\definecolor{currentstroke}{rgb}{0.121569,0.466667,0.705882}%
\pgfsetstrokecolor{currentstroke}%
\pgfsetstrokeopacity{0.645681}%
\pgfsetdash{}{0pt}%
\pgfpathmoveto{\pgfqpoint{0.873657in}{1.446194in}}%
\pgfpathcurveto{\pgfqpoint{0.881893in}{1.446194in}}{\pgfqpoint{0.889794in}{1.449466in}}{\pgfqpoint{0.895617in}{1.455290in}}%
\pgfpathcurveto{\pgfqpoint{0.901441in}{1.461114in}}{\pgfqpoint{0.904714in}{1.469014in}}{\pgfqpoint{0.904714in}{1.477250in}}%
\pgfpathcurveto{\pgfqpoint{0.904714in}{1.485487in}}{\pgfqpoint{0.901441in}{1.493387in}}{\pgfqpoint{0.895617in}{1.499211in}}%
\pgfpathcurveto{\pgfqpoint{0.889794in}{1.505035in}}{\pgfqpoint{0.881893in}{1.508307in}}{\pgfqpoint{0.873657in}{1.508307in}}%
\pgfpathcurveto{\pgfqpoint{0.865421in}{1.508307in}}{\pgfqpoint{0.857521in}{1.505035in}}{\pgfqpoint{0.851697in}{1.499211in}}%
\pgfpathcurveto{\pgfqpoint{0.845873in}{1.493387in}}{\pgfqpoint{0.842601in}{1.485487in}}{\pgfqpoint{0.842601in}{1.477250in}}%
\pgfpathcurveto{\pgfqpoint{0.842601in}{1.469014in}}{\pgfqpoint{0.845873in}{1.461114in}}{\pgfqpoint{0.851697in}{1.455290in}}%
\pgfpathcurveto{\pgfqpoint{0.857521in}{1.449466in}}{\pgfqpoint{0.865421in}{1.446194in}}{\pgfqpoint{0.873657in}{1.446194in}}%
\pgfpathclose%
\pgfusepath{stroke,fill}%
\end{pgfscope}%
\begin{pgfscope}%
\pgfpathrectangle{\pgfqpoint{0.100000in}{0.212622in}}{\pgfqpoint{3.696000in}{3.696000in}}%
\pgfusepath{clip}%
\pgfsetbuttcap%
\pgfsetroundjoin%
\definecolor{currentfill}{rgb}{0.121569,0.466667,0.705882}%
\pgfsetfillcolor{currentfill}%
\pgfsetfillopacity{0.645685}%
\pgfsetlinewidth{1.003750pt}%
\definecolor{currentstroke}{rgb}{0.121569,0.466667,0.705882}%
\pgfsetstrokecolor{currentstroke}%
\pgfsetstrokeopacity{0.645685}%
\pgfsetdash{}{0pt}%
\pgfpathmoveto{\pgfqpoint{0.873645in}{1.446189in}}%
\pgfpathcurveto{\pgfqpoint{0.881881in}{1.446189in}}{\pgfqpoint{0.889782in}{1.449461in}}{\pgfqpoint{0.895605in}{1.455285in}}%
\pgfpathcurveto{\pgfqpoint{0.901429in}{1.461109in}}{\pgfqpoint{0.904702in}{1.469009in}}{\pgfqpoint{0.904702in}{1.477245in}}%
\pgfpathcurveto{\pgfqpoint{0.904702in}{1.485482in}}{\pgfqpoint{0.901429in}{1.493382in}}{\pgfqpoint{0.895605in}{1.499206in}}%
\pgfpathcurveto{\pgfqpoint{0.889782in}{1.505030in}}{\pgfqpoint{0.881881in}{1.508302in}}{\pgfqpoint{0.873645in}{1.508302in}}%
\pgfpathcurveto{\pgfqpoint{0.865409in}{1.508302in}}{\pgfqpoint{0.857509in}{1.505030in}}{\pgfqpoint{0.851685in}{1.499206in}}%
\pgfpathcurveto{\pgfqpoint{0.845861in}{1.493382in}}{\pgfqpoint{0.842589in}{1.485482in}}{\pgfqpoint{0.842589in}{1.477245in}}%
\pgfpathcurveto{\pgfqpoint{0.842589in}{1.469009in}}{\pgfqpoint{0.845861in}{1.461109in}}{\pgfqpoint{0.851685in}{1.455285in}}%
\pgfpathcurveto{\pgfqpoint{0.857509in}{1.449461in}}{\pgfqpoint{0.865409in}{1.446189in}}{\pgfqpoint{0.873645in}{1.446189in}}%
\pgfpathclose%
\pgfusepath{stroke,fill}%
\end{pgfscope}%
\begin{pgfscope}%
\pgfpathrectangle{\pgfqpoint{0.100000in}{0.212622in}}{\pgfqpoint{3.696000in}{3.696000in}}%
\pgfusepath{clip}%
\pgfsetbuttcap%
\pgfsetroundjoin%
\definecolor{currentfill}{rgb}{0.121569,0.466667,0.705882}%
\pgfsetfillcolor{currentfill}%
\pgfsetfillopacity{0.645691}%
\pgfsetlinewidth{1.003750pt}%
\definecolor{currentstroke}{rgb}{0.121569,0.466667,0.705882}%
\pgfsetstrokecolor{currentstroke}%
\pgfsetstrokeopacity{0.645691}%
\pgfsetdash{}{0pt}%
\pgfpathmoveto{\pgfqpoint{0.873623in}{1.446173in}}%
\pgfpathcurveto{\pgfqpoint{0.881859in}{1.446173in}}{\pgfqpoint{0.889759in}{1.449446in}}{\pgfqpoint{0.895583in}{1.455270in}}%
\pgfpathcurveto{\pgfqpoint{0.901407in}{1.461094in}}{\pgfqpoint{0.904679in}{1.468994in}}{\pgfqpoint{0.904679in}{1.477230in}}%
\pgfpathcurveto{\pgfqpoint{0.904679in}{1.485466in}}{\pgfqpoint{0.901407in}{1.493366in}}{\pgfqpoint{0.895583in}{1.499190in}}%
\pgfpathcurveto{\pgfqpoint{0.889759in}{1.505014in}}{\pgfqpoint{0.881859in}{1.508286in}}{\pgfqpoint{0.873623in}{1.508286in}}%
\pgfpathcurveto{\pgfqpoint{0.865387in}{1.508286in}}{\pgfqpoint{0.857486in}{1.505014in}}{\pgfqpoint{0.851663in}{1.499190in}}%
\pgfpathcurveto{\pgfqpoint{0.845839in}{1.493366in}}{\pgfqpoint{0.842566in}{1.485466in}}{\pgfqpoint{0.842566in}{1.477230in}}%
\pgfpathcurveto{\pgfqpoint{0.842566in}{1.468994in}}{\pgfqpoint{0.845839in}{1.461094in}}{\pgfqpoint{0.851663in}{1.455270in}}%
\pgfpathcurveto{\pgfqpoint{0.857486in}{1.449446in}}{\pgfqpoint{0.865387in}{1.446173in}}{\pgfqpoint{0.873623in}{1.446173in}}%
\pgfpathclose%
\pgfusepath{stroke,fill}%
\end{pgfscope}%
\begin{pgfscope}%
\pgfpathrectangle{\pgfqpoint{0.100000in}{0.212622in}}{\pgfqpoint{3.696000in}{3.696000in}}%
\pgfusepath{clip}%
\pgfsetbuttcap%
\pgfsetroundjoin%
\definecolor{currentfill}{rgb}{0.121569,0.466667,0.705882}%
\pgfsetfillcolor{currentfill}%
\pgfsetfillopacity{0.645705}%
\pgfsetlinewidth{1.003750pt}%
\definecolor{currentstroke}{rgb}{0.121569,0.466667,0.705882}%
\pgfsetstrokecolor{currentstroke}%
\pgfsetstrokeopacity{0.645705}%
\pgfsetdash{}{0pt}%
\pgfpathmoveto{\pgfqpoint{0.873588in}{1.446152in}}%
\pgfpathcurveto{\pgfqpoint{0.881824in}{1.446152in}}{\pgfqpoint{0.889724in}{1.449425in}}{\pgfqpoint{0.895548in}{1.455248in}}%
\pgfpathcurveto{\pgfqpoint{0.901372in}{1.461072in}}{\pgfqpoint{0.904645in}{1.468972in}}{\pgfqpoint{0.904645in}{1.477209in}}%
\pgfpathcurveto{\pgfqpoint{0.904645in}{1.485445in}}{\pgfqpoint{0.901372in}{1.493345in}}{\pgfqpoint{0.895548in}{1.499169in}}%
\pgfpathcurveto{\pgfqpoint{0.889724in}{1.504993in}}{\pgfqpoint{0.881824in}{1.508265in}}{\pgfqpoint{0.873588in}{1.508265in}}%
\pgfpathcurveto{\pgfqpoint{0.865352in}{1.508265in}}{\pgfqpoint{0.857452in}{1.504993in}}{\pgfqpoint{0.851628in}{1.499169in}}%
\pgfpathcurveto{\pgfqpoint{0.845804in}{1.493345in}}{\pgfqpoint{0.842532in}{1.485445in}}{\pgfqpoint{0.842532in}{1.477209in}}%
\pgfpathcurveto{\pgfqpoint{0.842532in}{1.468972in}}{\pgfqpoint{0.845804in}{1.461072in}}{\pgfqpoint{0.851628in}{1.455248in}}%
\pgfpathcurveto{\pgfqpoint{0.857452in}{1.449425in}}{\pgfqpoint{0.865352in}{1.446152in}}{\pgfqpoint{0.873588in}{1.446152in}}%
\pgfpathclose%
\pgfusepath{stroke,fill}%
\end{pgfscope}%
\begin{pgfscope}%
\pgfpathrectangle{\pgfqpoint{0.100000in}{0.212622in}}{\pgfqpoint{3.696000in}{3.696000in}}%
\pgfusepath{clip}%
\pgfsetbuttcap%
\pgfsetroundjoin%
\definecolor{currentfill}{rgb}{0.121569,0.466667,0.705882}%
\pgfsetfillcolor{currentfill}%
\pgfsetfillopacity{0.645730}%
\pgfsetlinewidth{1.003750pt}%
\definecolor{currentstroke}{rgb}{0.121569,0.466667,0.705882}%
\pgfsetstrokecolor{currentstroke}%
\pgfsetstrokeopacity{0.645730}%
\pgfsetdash{}{0pt}%
\pgfpathmoveto{\pgfqpoint{0.873504in}{1.446148in}}%
\pgfpathcurveto{\pgfqpoint{0.881741in}{1.446148in}}{\pgfqpoint{0.889641in}{1.449420in}}{\pgfqpoint{0.895465in}{1.455244in}}%
\pgfpathcurveto{\pgfqpoint{0.901288in}{1.461068in}}{\pgfqpoint{0.904561in}{1.468968in}}{\pgfqpoint{0.904561in}{1.477204in}}%
\pgfpathcurveto{\pgfqpoint{0.904561in}{1.485440in}}{\pgfqpoint{0.901288in}{1.493340in}}{\pgfqpoint{0.895465in}{1.499164in}}%
\pgfpathcurveto{\pgfqpoint{0.889641in}{1.504988in}}{\pgfqpoint{0.881741in}{1.508261in}}{\pgfqpoint{0.873504in}{1.508261in}}%
\pgfpathcurveto{\pgfqpoint{0.865268in}{1.508261in}}{\pgfqpoint{0.857368in}{1.504988in}}{\pgfqpoint{0.851544in}{1.499164in}}%
\pgfpathcurveto{\pgfqpoint{0.845720in}{1.493340in}}{\pgfqpoint{0.842448in}{1.485440in}}{\pgfqpoint{0.842448in}{1.477204in}}%
\pgfpathcurveto{\pgfqpoint{0.842448in}{1.468968in}}{\pgfqpoint{0.845720in}{1.461068in}}{\pgfqpoint{0.851544in}{1.455244in}}%
\pgfpathcurveto{\pgfqpoint{0.857368in}{1.449420in}}{\pgfqpoint{0.865268in}{1.446148in}}{\pgfqpoint{0.873504in}{1.446148in}}%
\pgfpathclose%
\pgfusepath{stroke,fill}%
\end{pgfscope}%
\begin{pgfscope}%
\pgfpathrectangle{\pgfqpoint{0.100000in}{0.212622in}}{\pgfqpoint{3.696000in}{3.696000in}}%
\pgfusepath{clip}%
\pgfsetbuttcap%
\pgfsetroundjoin%
\definecolor{currentfill}{rgb}{0.121569,0.466667,0.705882}%
\pgfsetfillcolor{currentfill}%
\pgfsetfillopacity{0.645764}%
\pgfsetlinewidth{1.003750pt}%
\definecolor{currentstroke}{rgb}{0.121569,0.466667,0.705882}%
\pgfsetstrokecolor{currentstroke}%
\pgfsetstrokeopacity{0.645764}%
\pgfsetdash{}{0pt}%
\pgfpathmoveto{\pgfqpoint{0.873380in}{1.446023in}}%
\pgfpathcurveto{\pgfqpoint{0.881616in}{1.446023in}}{\pgfqpoint{0.889516in}{1.449295in}}{\pgfqpoint{0.895340in}{1.455119in}}%
\pgfpathcurveto{\pgfqpoint{0.901164in}{1.460943in}}{\pgfqpoint{0.904437in}{1.468843in}}{\pgfqpoint{0.904437in}{1.477080in}}%
\pgfpathcurveto{\pgfqpoint{0.904437in}{1.485316in}}{\pgfqpoint{0.901164in}{1.493216in}}{\pgfqpoint{0.895340in}{1.499040in}}%
\pgfpathcurveto{\pgfqpoint{0.889516in}{1.504864in}}{\pgfqpoint{0.881616in}{1.508136in}}{\pgfqpoint{0.873380in}{1.508136in}}%
\pgfpathcurveto{\pgfqpoint{0.865144in}{1.508136in}}{\pgfqpoint{0.857244in}{1.504864in}}{\pgfqpoint{0.851420in}{1.499040in}}%
\pgfpathcurveto{\pgfqpoint{0.845596in}{1.493216in}}{\pgfqpoint{0.842324in}{1.485316in}}{\pgfqpoint{0.842324in}{1.477080in}}%
\pgfpathcurveto{\pgfqpoint{0.842324in}{1.468843in}}{\pgfqpoint{0.845596in}{1.460943in}}{\pgfqpoint{0.851420in}{1.455119in}}%
\pgfpathcurveto{\pgfqpoint{0.857244in}{1.449295in}}{\pgfqpoint{0.865144in}{1.446023in}}{\pgfqpoint{0.873380in}{1.446023in}}%
\pgfpathclose%
\pgfusepath{stroke,fill}%
\end{pgfscope}%
\begin{pgfscope}%
\pgfpathrectangle{\pgfqpoint{0.100000in}{0.212622in}}{\pgfqpoint{3.696000in}{3.696000in}}%
\pgfusepath{clip}%
\pgfsetbuttcap%
\pgfsetroundjoin%
\definecolor{currentfill}{rgb}{0.121569,0.466667,0.705882}%
\pgfsetfillcolor{currentfill}%
\pgfsetfillopacity{0.645842}%
\pgfsetlinewidth{1.003750pt}%
\definecolor{currentstroke}{rgb}{0.121569,0.466667,0.705882}%
\pgfsetstrokecolor{currentstroke}%
\pgfsetstrokeopacity{0.645842}%
\pgfsetdash{}{0pt}%
\pgfpathmoveto{\pgfqpoint{0.873152in}{1.445894in}}%
\pgfpathcurveto{\pgfqpoint{0.881388in}{1.445894in}}{\pgfqpoint{0.889288in}{1.449166in}}{\pgfqpoint{0.895112in}{1.454990in}}%
\pgfpathcurveto{\pgfqpoint{0.900936in}{1.460814in}}{\pgfqpoint{0.904208in}{1.468714in}}{\pgfqpoint{0.904208in}{1.476950in}}%
\pgfpathcurveto{\pgfqpoint{0.904208in}{1.485186in}}{\pgfqpoint{0.900936in}{1.493086in}}{\pgfqpoint{0.895112in}{1.498910in}}%
\pgfpathcurveto{\pgfqpoint{0.889288in}{1.504734in}}{\pgfqpoint{0.881388in}{1.508007in}}{\pgfqpoint{0.873152in}{1.508007in}}%
\pgfpathcurveto{\pgfqpoint{0.864916in}{1.508007in}}{\pgfqpoint{0.857016in}{1.504734in}}{\pgfqpoint{0.851192in}{1.498910in}}%
\pgfpathcurveto{\pgfqpoint{0.845368in}{1.493086in}}{\pgfqpoint{0.842095in}{1.485186in}}{\pgfqpoint{0.842095in}{1.476950in}}%
\pgfpathcurveto{\pgfqpoint{0.842095in}{1.468714in}}{\pgfqpoint{0.845368in}{1.460814in}}{\pgfqpoint{0.851192in}{1.454990in}}%
\pgfpathcurveto{\pgfqpoint{0.857016in}{1.449166in}}{\pgfqpoint{0.864916in}{1.445894in}}{\pgfqpoint{0.873152in}{1.445894in}}%
\pgfpathclose%
\pgfusepath{stroke,fill}%
\end{pgfscope}%
\begin{pgfscope}%
\pgfpathrectangle{\pgfqpoint{0.100000in}{0.212622in}}{\pgfqpoint{3.696000in}{3.696000in}}%
\pgfusepath{clip}%
\pgfsetbuttcap%
\pgfsetroundjoin%
\definecolor{currentfill}{rgb}{0.121569,0.466667,0.705882}%
\pgfsetfillcolor{currentfill}%
\pgfsetfillopacity{0.645967}%
\pgfsetlinewidth{1.003750pt}%
\definecolor{currentstroke}{rgb}{0.121569,0.466667,0.705882}%
\pgfsetstrokecolor{currentstroke}%
\pgfsetstrokeopacity{0.645967}%
\pgfsetdash{}{0pt}%
\pgfpathmoveto{\pgfqpoint{0.872717in}{1.445583in}}%
\pgfpathcurveto{\pgfqpoint{0.880953in}{1.445583in}}{\pgfqpoint{0.888853in}{1.448855in}}{\pgfqpoint{0.894677in}{1.454679in}}%
\pgfpathcurveto{\pgfqpoint{0.900501in}{1.460503in}}{\pgfqpoint{0.903773in}{1.468403in}}{\pgfqpoint{0.903773in}{1.476639in}}%
\pgfpathcurveto{\pgfqpoint{0.903773in}{1.484876in}}{\pgfqpoint{0.900501in}{1.492776in}}{\pgfqpoint{0.894677in}{1.498600in}}%
\pgfpathcurveto{\pgfqpoint{0.888853in}{1.504424in}}{\pgfqpoint{0.880953in}{1.507696in}}{\pgfqpoint{0.872717in}{1.507696in}}%
\pgfpathcurveto{\pgfqpoint{0.864480in}{1.507696in}}{\pgfqpoint{0.856580in}{1.504424in}}{\pgfqpoint{0.850756in}{1.498600in}}%
\pgfpathcurveto{\pgfqpoint{0.844932in}{1.492776in}}{\pgfqpoint{0.841660in}{1.484876in}}{\pgfqpoint{0.841660in}{1.476639in}}%
\pgfpathcurveto{\pgfqpoint{0.841660in}{1.468403in}}{\pgfqpoint{0.844932in}{1.460503in}}{\pgfqpoint{0.850756in}{1.454679in}}%
\pgfpathcurveto{\pgfqpoint{0.856580in}{1.448855in}}{\pgfqpoint{0.864480in}{1.445583in}}{\pgfqpoint{0.872717in}{1.445583in}}%
\pgfpathclose%
\pgfusepath{stroke,fill}%
\end{pgfscope}%
\begin{pgfscope}%
\pgfpathrectangle{\pgfqpoint{0.100000in}{0.212622in}}{\pgfqpoint{3.696000in}{3.696000in}}%
\pgfusepath{clip}%
\pgfsetbuttcap%
\pgfsetroundjoin%
\definecolor{currentfill}{rgb}{0.121569,0.466667,0.705882}%
\pgfsetfillcolor{currentfill}%
\pgfsetfillopacity{0.646227}%
\pgfsetlinewidth{1.003750pt}%
\definecolor{currentstroke}{rgb}{0.121569,0.466667,0.705882}%
\pgfsetstrokecolor{currentstroke}%
\pgfsetstrokeopacity{0.646227}%
\pgfsetdash{}{0pt}%
\pgfpathmoveto{\pgfqpoint{0.871974in}{1.445157in}}%
\pgfpathcurveto{\pgfqpoint{0.880211in}{1.445157in}}{\pgfqpoint{0.888111in}{1.448429in}}{\pgfqpoint{0.893935in}{1.454253in}}%
\pgfpathcurveto{\pgfqpoint{0.899759in}{1.460077in}}{\pgfqpoint{0.903031in}{1.467977in}}{\pgfqpoint{0.903031in}{1.476213in}}%
\pgfpathcurveto{\pgfqpoint{0.903031in}{1.484450in}}{\pgfqpoint{0.899759in}{1.492350in}}{\pgfqpoint{0.893935in}{1.498174in}}%
\pgfpathcurveto{\pgfqpoint{0.888111in}{1.503998in}}{\pgfqpoint{0.880211in}{1.507270in}}{\pgfqpoint{0.871974in}{1.507270in}}%
\pgfpathcurveto{\pgfqpoint{0.863738in}{1.507270in}}{\pgfqpoint{0.855838in}{1.503998in}}{\pgfqpoint{0.850014in}{1.498174in}}%
\pgfpathcurveto{\pgfqpoint{0.844190in}{1.492350in}}{\pgfqpoint{0.840918in}{1.484450in}}{\pgfqpoint{0.840918in}{1.476213in}}%
\pgfpathcurveto{\pgfqpoint{0.840918in}{1.467977in}}{\pgfqpoint{0.844190in}{1.460077in}}{\pgfqpoint{0.850014in}{1.454253in}}%
\pgfpathcurveto{\pgfqpoint{0.855838in}{1.448429in}}{\pgfqpoint{0.863738in}{1.445157in}}{\pgfqpoint{0.871974in}{1.445157in}}%
\pgfpathclose%
\pgfusepath{stroke,fill}%
\end{pgfscope}%
\begin{pgfscope}%
\pgfpathrectangle{\pgfqpoint{0.100000in}{0.212622in}}{\pgfqpoint{3.696000in}{3.696000in}}%
\pgfusepath{clip}%
\pgfsetbuttcap%
\pgfsetroundjoin%
\definecolor{currentfill}{rgb}{0.121569,0.466667,0.705882}%
\pgfsetfillcolor{currentfill}%
\pgfsetfillopacity{0.646406}%
\pgfsetlinewidth{1.003750pt}%
\definecolor{currentstroke}{rgb}{0.121569,0.466667,0.705882}%
\pgfsetstrokecolor{currentstroke}%
\pgfsetstrokeopacity{0.646406}%
\pgfsetdash{}{0pt}%
\pgfpathmoveto{\pgfqpoint{2.138653in}{1.751945in}}%
\pgfpathcurveto{\pgfqpoint{2.146889in}{1.751945in}}{\pgfqpoint{2.154789in}{1.755217in}}{\pgfqpoint{2.160613in}{1.761041in}}%
\pgfpathcurveto{\pgfqpoint{2.166437in}{1.766865in}}{\pgfqpoint{2.169709in}{1.774765in}}{\pgfqpoint{2.169709in}{1.783001in}}%
\pgfpathcurveto{\pgfqpoint{2.169709in}{1.791238in}}{\pgfqpoint{2.166437in}{1.799138in}}{\pgfqpoint{2.160613in}{1.804962in}}%
\pgfpathcurveto{\pgfqpoint{2.154789in}{1.810785in}}{\pgfqpoint{2.146889in}{1.814058in}}{\pgfqpoint{2.138653in}{1.814058in}}%
\pgfpathcurveto{\pgfqpoint{2.130416in}{1.814058in}}{\pgfqpoint{2.122516in}{1.810785in}}{\pgfqpoint{2.116692in}{1.804962in}}%
\pgfpathcurveto{\pgfqpoint{2.110868in}{1.799138in}}{\pgfqpoint{2.107596in}{1.791238in}}{\pgfqpoint{2.107596in}{1.783001in}}%
\pgfpathcurveto{\pgfqpoint{2.107596in}{1.774765in}}{\pgfqpoint{2.110868in}{1.766865in}}{\pgfqpoint{2.116692in}{1.761041in}}%
\pgfpathcurveto{\pgfqpoint{2.122516in}{1.755217in}}{\pgfqpoint{2.130416in}{1.751945in}}{\pgfqpoint{2.138653in}{1.751945in}}%
\pgfpathclose%
\pgfusepath{stroke,fill}%
\end{pgfscope}%
\begin{pgfscope}%
\pgfpathrectangle{\pgfqpoint{0.100000in}{0.212622in}}{\pgfqpoint{3.696000in}{3.696000in}}%
\pgfusepath{clip}%
\pgfsetbuttcap%
\pgfsetroundjoin%
\definecolor{currentfill}{rgb}{0.121569,0.466667,0.705882}%
\pgfsetfillcolor{currentfill}%
\pgfsetfillopacity{0.646646}%
\pgfsetlinewidth{1.003750pt}%
\definecolor{currentstroke}{rgb}{0.121569,0.466667,0.705882}%
\pgfsetstrokecolor{currentstroke}%
\pgfsetstrokeopacity{0.646646}%
\pgfsetdash{}{0pt}%
\pgfpathmoveto{\pgfqpoint{0.870640in}{1.444027in}}%
\pgfpathcurveto{\pgfqpoint{0.878876in}{1.444027in}}{\pgfqpoint{0.886776in}{1.447300in}}{\pgfqpoint{0.892600in}{1.453124in}}%
\pgfpathcurveto{\pgfqpoint{0.898424in}{1.458947in}}{\pgfqpoint{0.901696in}{1.466847in}}{\pgfqpoint{0.901696in}{1.475084in}}%
\pgfpathcurveto{\pgfqpoint{0.901696in}{1.483320in}}{\pgfqpoint{0.898424in}{1.491220in}}{\pgfqpoint{0.892600in}{1.497044in}}%
\pgfpathcurveto{\pgfqpoint{0.886776in}{1.502868in}}{\pgfqpoint{0.878876in}{1.506140in}}{\pgfqpoint{0.870640in}{1.506140in}}%
\pgfpathcurveto{\pgfqpoint{0.862403in}{1.506140in}}{\pgfqpoint{0.854503in}{1.502868in}}{\pgfqpoint{0.848680in}{1.497044in}}%
\pgfpathcurveto{\pgfqpoint{0.842856in}{1.491220in}}{\pgfqpoint{0.839583in}{1.483320in}}{\pgfqpoint{0.839583in}{1.475084in}}%
\pgfpathcurveto{\pgfqpoint{0.839583in}{1.466847in}}{\pgfqpoint{0.842856in}{1.458947in}}{\pgfqpoint{0.848680in}{1.453124in}}%
\pgfpathcurveto{\pgfqpoint{0.854503in}{1.447300in}}{\pgfqpoint{0.862403in}{1.444027in}}{\pgfqpoint{0.870640in}{1.444027in}}%
\pgfpathclose%
\pgfusepath{stroke,fill}%
\end{pgfscope}%
\begin{pgfscope}%
\pgfpathrectangle{\pgfqpoint{0.100000in}{0.212622in}}{\pgfqpoint{3.696000in}{3.696000in}}%
\pgfusepath{clip}%
\pgfsetbuttcap%
\pgfsetroundjoin%
\definecolor{currentfill}{rgb}{0.121569,0.466667,0.705882}%
\pgfsetfillcolor{currentfill}%
\pgfsetfillopacity{0.646928}%
\pgfsetlinewidth{1.003750pt}%
\definecolor{currentstroke}{rgb}{0.121569,0.466667,0.705882}%
\pgfsetstrokecolor{currentstroke}%
\pgfsetstrokeopacity{0.646928}%
\pgfsetdash{}{0pt}%
\pgfpathmoveto{\pgfqpoint{0.883216in}{1.407339in}}%
\pgfpathcurveto{\pgfqpoint{0.891452in}{1.407339in}}{\pgfqpoint{0.899352in}{1.410612in}}{\pgfqpoint{0.905176in}{1.416436in}}%
\pgfpathcurveto{\pgfqpoint{0.911000in}{1.422260in}}{\pgfqpoint{0.914272in}{1.430160in}}{\pgfqpoint{0.914272in}{1.438396in}}%
\pgfpathcurveto{\pgfqpoint{0.914272in}{1.446632in}}{\pgfqpoint{0.911000in}{1.454532in}}{\pgfqpoint{0.905176in}{1.460356in}}%
\pgfpathcurveto{\pgfqpoint{0.899352in}{1.466180in}}{\pgfqpoint{0.891452in}{1.469452in}}{\pgfqpoint{0.883216in}{1.469452in}}%
\pgfpathcurveto{\pgfqpoint{0.874980in}{1.469452in}}{\pgfqpoint{0.867080in}{1.466180in}}{\pgfqpoint{0.861256in}{1.460356in}}%
\pgfpathcurveto{\pgfqpoint{0.855432in}{1.454532in}}{\pgfqpoint{0.852159in}{1.446632in}}{\pgfqpoint{0.852159in}{1.438396in}}%
\pgfpathcurveto{\pgfqpoint{0.852159in}{1.430160in}}{\pgfqpoint{0.855432in}{1.422260in}}{\pgfqpoint{0.861256in}{1.416436in}}%
\pgfpathcurveto{\pgfqpoint{0.867080in}{1.410612in}}{\pgfqpoint{0.874980in}{1.407339in}}{\pgfqpoint{0.883216in}{1.407339in}}%
\pgfpathclose%
\pgfusepath{stroke,fill}%
\end{pgfscope}%
\begin{pgfscope}%
\pgfpathrectangle{\pgfqpoint{0.100000in}{0.212622in}}{\pgfqpoint{3.696000in}{3.696000in}}%
\pgfusepath{clip}%
\pgfsetbuttcap%
\pgfsetroundjoin%
\definecolor{currentfill}{rgb}{0.121569,0.466667,0.705882}%
\pgfsetfillcolor{currentfill}%
\pgfsetfillopacity{0.647072}%
\pgfsetlinewidth{1.003750pt}%
\definecolor{currentstroke}{rgb}{0.121569,0.466667,0.705882}%
\pgfsetstrokecolor{currentstroke}%
\pgfsetstrokeopacity{0.647072}%
\pgfsetdash{}{0pt}%
\pgfpathmoveto{\pgfqpoint{0.883030in}{1.407534in}}%
\pgfpathcurveto{\pgfqpoint{0.891266in}{1.407534in}}{\pgfqpoint{0.899166in}{1.410806in}}{\pgfqpoint{0.904990in}{1.416630in}}%
\pgfpathcurveto{\pgfqpoint{0.910814in}{1.422454in}}{\pgfqpoint{0.914086in}{1.430354in}}{\pgfqpoint{0.914086in}{1.438590in}}%
\pgfpathcurveto{\pgfqpoint{0.914086in}{1.446827in}}{\pgfqpoint{0.910814in}{1.454727in}}{\pgfqpoint{0.904990in}{1.460551in}}%
\pgfpathcurveto{\pgfqpoint{0.899166in}{1.466375in}}{\pgfqpoint{0.891266in}{1.469647in}}{\pgfqpoint{0.883030in}{1.469647in}}%
\pgfpathcurveto{\pgfqpoint{0.874794in}{1.469647in}}{\pgfqpoint{0.866894in}{1.466375in}}{\pgfqpoint{0.861070in}{1.460551in}}%
\pgfpathcurveto{\pgfqpoint{0.855246in}{1.454727in}}{\pgfqpoint{0.851973in}{1.446827in}}{\pgfqpoint{0.851973in}{1.438590in}}%
\pgfpathcurveto{\pgfqpoint{0.851973in}{1.430354in}}{\pgfqpoint{0.855246in}{1.422454in}}{\pgfqpoint{0.861070in}{1.416630in}}%
\pgfpathcurveto{\pgfqpoint{0.866894in}{1.410806in}}{\pgfqpoint{0.874794in}{1.407534in}}{\pgfqpoint{0.883030in}{1.407534in}}%
\pgfpathclose%
\pgfusepath{stroke,fill}%
\end{pgfscope}%
\begin{pgfscope}%
\pgfpathrectangle{\pgfqpoint{0.100000in}{0.212622in}}{\pgfqpoint{3.696000in}{3.696000in}}%
\pgfusepath{clip}%
\pgfsetbuttcap%
\pgfsetroundjoin%
\definecolor{currentfill}{rgb}{0.121569,0.466667,0.705882}%
\pgfsetfillcolor{currentfill}%
\pgfsetfillopacity{0.647400}%
\pgfsetlinewidth{1.003750pt}%
\definecolor{currentstroke}{rgb}{0.121569,0.466667,0.705882}%
\pgfsetstrokecolor{currentstroke}%
\pgfsetstrokeopacity{0.647400}%
\pgfsetdash{}{0pt}%
\pgfpathmoveto{\pgfqpoint{0.882371in}{1.407742in}}%
\pgfpathcurveto{\pgfqpoint{0.890608in}{1.407742in}}{\pgfqpoint{0.898508in}{1.411014in}}{\pgfqpoint{0.904332in}{1.416838in}}%
\pgfpathcurveto{\pgfqpoint{0.910156in}{1.422662in}}{\pgfqpoint{0.913428in}{1.430562in}}{\pgfqpoint{0.913428in}{1.438798in}}%
\pgfpathcurveto{\pgfqpoint{0.913428in}{1.447035in}}{\pgfqpoint{0.910156in}{1.454935in}}{\pgfqpoint{0.904332in}{1.460759in}}%
\pgfpathcurveto{\pgfqpoint{0.898508in}{1.466583in}}{\pgfqpoint{0.890608in}{1.469855in}}{\pgfqpoint{0.882371in}{1.469855in}}%
\pgfpathcurveto{\pgfqpoint{0.874135in}{1.469855in}}{\pgfqpoint{0.866235in}{1.466583in}}{\pgfqpoint{0.860411in}{1.460759in}}%
\pgfpathcurveto{\pgfqpoint{0.854587in}{1.454935in}}{\pgfqpoint{0.851315in}{1.447035in}}{\pgfqpoint{0.851315in}{1.438798in}}%
\pgfpathcurveto{\pgfqpoint{0.851315in}{1.430562in}}{\pgfqpoint{0.854587in}{1.422662in}}{\pgfqpoint{0.860411in}{1.416838in}}%
\pgfpathcurveto{\pgfqpoint{0.866235in}{1.411014in}}{\pgfqpoint{0.874135in}{1.407742in}}{\pgfqpoint{0.882371in}{1.407742in}}%
\pgfpathclose%
\pgfusepath{stroke,fill}%
\end{pgfscope}%
\begin{pgfscope}%
\pgfpathrectangle{\pgfqpoint{0.100000in}{0.212622in}}{\pgfqpoint{3.696000in}{3.696000in}}%
\pgfusepath{clip}%
\pgfsetbuttcap%
\pgfsetroundjoin%
\definecolor{currentfill}{rgb}{0.121569,0.466667,0.705882}%
\pgfsetfillcolor{currentfill}%
\pgfsetfillopacity{0.647419}%
\pgfsetlinewidth{1.003750pt}%
\definecolor{currentstroke}{rgb}{0.121569,0.466667,0.705882}%
\pgfsetstrokecolor{currentstroke}%
\pgfsetstrokeopacity{0.647419}%
\pgfsetdash{}{0pt}%
\pgfpathmoveto{\pgfqpoint{0.868006in}{1.442334in}}%
\pgfpathcurveto{\pgfqpoint{0.876242in}{1.442334in}}{\pgfqpoint{0.884142in}{1.445606in}}{\pgfqpoint{0.889966in}{1.451430in}}%
\pgfpathcurveto{\pgfqpoint{0.895790in}{1.457254in}}{\pgfqpoint{0.899062in}{1.465154in}}{\pgfqpoint{0.899062in}{1.473390in}}%
\pgfpathcurveto{\pgfqpoint{0.899062in}{1.481626in}}{\pgfqpoint{0.895790in}{1.489526in}}{\pgfqpoint{0.889966in}{1.495350in}}%
\pgfpathcurveto{\pgfqpoint{0.884142in}{1.501174in}}{\pgfqpoint{0.876242in}{1.504447in}}{\pgfqpoint{0.868006in}{1.504447in}}%
\pgfpathcurveto{\pgfqpoint{0.859769in}{1.504447in}}{\pgfqpoint{0.851869in}{1.501174in}}{\pgfqpoint{0.846045in}{1.495350in}}%
\pgfpathcurveto{\pgfqpoint{0.840221in}{1.489526in}}{\pgfqpoint{0.836949in}{1.481626in}}{\pgfqpoint{0.836949in}{1.473390in}}%
\pgfpathcurveto{\pgfqpoint{0.836949in}{1.465154in}}{\pgfqpoint{0.840221in}{1.457254in}}{\pgfqpoint{0.846045in}{1.451430in}}%
\pgfpathcurveto{\pgfqpoint{0.851869in}{1.445606in}}{\pgfqpoint{0.859769in}{1.442334in}}{\pgfqpoint{0.868006in}{1.442334in}}%
\pgfpathclose%
\pgfusepath{stroke,fill}%
\end{pgfscope}%
\begin{pgfscope}%
\pgfpathrectangle{\pgfqpoint{0.100000in}{0.212622in}}{\pgfqpoint{3.696000in}{3.696000in}}%
\pgfusepath{clip}%
\pgfsetbuttcap%
\pgfsetroundjoin%
\definecolor{currentfill}{rgb}{0.121569,0.466667,0.705882}%
\pgfsetfillcolor{currentfill}%
\pgfsetfillopacity{0.648138}%
\pgfsetlinewidth{1.003750pt}%
\definecolor{currentstroke}{rgb}{0.121569,0.466667,0.705882}%
\pgfsetstrokecolor{currentstroke}%
\pgfsetstrokeopacity{0.648138}%
\pgfsetdash{}{0pt}%
\pgfpathmoveto{\pgfqpoint{0.881305in}{1.408179in}}%
\pgfpathcurveto{\pgfqpoint{0.889541in}{1.408179in}}{\pgfqpoint{0.897441in}{1.411451in}}{\pgfqpoint{0.903265in}{1.417275in}}%
\pgfpathcurveto{\pgfqpoint{0.909089in}{1.423099in}}{\pgfqpoint{0.912361in}{1.430999in}}{\pgfqpoint{0.912361in}{1.439235in}}%
\pgfpathcurveto{\pgfqpoint{0.912361in}{1.447472in}}{\pgfqpoint{0.909089in}{1.455372in}}{\pgfqpoint{0.903265in}{1.461196in}}%
\pgfpathcurveto{\pgfqpoint{0.897441in}{1.467020in}}{\pgfqpoint{0.889541in}{1.470292in}}{\pgfqpoint{0.881305in}{1.470292in}}%
\pgfpathcurveto{\pgfqpoint{0.873068in}{1.470292in}}{\pgfqpoint{0.865168in}{1.467020in}}{\pgfqpoint{0.859344in}{1.461196in}}%
\pgfpathcurveto{\pgfqpoint{0.853520in}{1.455372in}}{\pgfqpoint{0.850248in}{1.447472in}}{\pgfqpoint{0.850248in}{1.439235in}}%
\pgfpathcurveto{\pgfqpoint{0.850248in}{1.430999in}}{\pgfqpoint{0.853520in}{1.423099in}}{\pgfqpoint{0.859344in}{1.417275in}}%
\pgfpathcurveto{\pgfqpoint{0.865168in}{1.411451in}}{\pgfqpoint{0.873068in}{1.408179in}}{\pgfqpoint{0.881305in}{1.408179in}}%
\pgfpathclose%
\pgfusepath{stroke,fill}%
\end{pgfscope}%
\begin{pgfscope}%
\pgfpathrectangle{\pgfqpoint{0.100000in}{0.212622in}}{\pgfqpoint{3.696000in}{3.696000in}}%
\pgfusepath{clip}%
\pgfsetbuttcap%
\pgfsetroundjoin%
\definecolor{currentfill}{rgb}{0.121569,0.466667,0.705882}%
\pgfsetfillcolor{currentfill}%
\pgfsetfillopacity{0.649134}%
\pgfsetlinewidth{1.003750pt}%
\definecolor{currentstroke}{rgb}{0.121569,0.466667,0.705882}%
\pgfsetstrokecolor{currentstroke}%
\pgfsetstrokeopacity{0.649134}%
\pgfsetdash{}{0pt}%
\pgfpathmoveto{\pgfqpoint{0.863184in}{1.441185in}}%
\pgfpathcurveto{\pgfqpoint{0.871421in}{1.441185in}}{\pgfqpoint{0.879321in}{1.444457in}}{\pgfqpoint{0.885145in}{1.450281in}}%
\pgfpathcurveto{\pgfqpoint{0.890969in}{1.456105in}}{\pgfqpoint{0.894241in}{1.464005in}}{\pgfqpoint{0.894241in}{1.472241in}}%
\pgfpathcurveto{\pgfqpoint{0.894241in}{1.480478in}}{\pgfqpoint{0.890969in}{1.488378in}}{\pgfqpoint{0.885145in}{1.494202in}}%
\pgfpathcurveto{\pgfqpoint{0.879321in}{1.500026in}}{\pgfqpoint{0.871421in}{1.503298in}}{\pgfqpoint{0.863184in}{1.503298in}}%
\pgfpathcurveto{\pgfqpoint{0.854948in}{1.503298in}}{\pgfqpoint{0.847048in}{1.500026in}}{\pgfqpoint{0.841224in}{1.494202in}}%
\pgfpathcurveto{\pgfqpoint{0.835400in}{1.488378in}}{\pgfqpoint{0.832128in}{1.480478in}}{\pgfqpoint{0.832128in}{1.472241in}}%
\pgfpathcurveto{\pgfqpoint{0.832128in}{1.464005in}}{\pgfqpoint{0.835400in}{1.456105in}}{\pgfqpoint{0.841224in}{1.450281in}}%
\pgfpathcurveto{\pgfqpoint{0.847048in}{1.444457in}}{\pgfqpoint{0.854948in}{1.441185in}}{\pgfqpoint{0.863184in}{1.441185in}}%
\pgfpathclose%
\pgfusepath{stroke,fill}%
\end{pgfscope}%
\begin{pgfscope}%
\pgfpathrectangle{\pgfqpoint{0.100000in}{0.212622in}}{\pgfqpoint{3.696000in}{3.696000in}}%
\pgfusepath{clip}%
\pgfsetbuttcap%
\pgfsetroundjoin%
\definecolor{currentfill}{rgb}{0.121569,0.466667,0.705882}%
\pgfsetfillcolor{currentfill}%
\pgfsetfillopacity{0.649209}%
\pgfsetlinewidth{1.003750pt}%
\definecolor{currentstroke}{rgb}{0.121569,0.466667,0.705882}%
\pgfsetstrokecolor{currentstroke}%
\pgfsetstrokeopacity{0.649209}%
\pgfsetdash{}{0pt}%
\pgfpathmoveto{\pgfqpoint{2.141081in}{1.746088in}}%
\pgfpathcurveto{\pgfqpoint{2.149317in}{1.746088in}}{\pgfqpoint{2.157217in}{1.749360in}}{\pgfqpoint{2.163041in}{1.755184in}}%
\pgfpathcurveto{\pgfqpoint{2.168865in}{1.761008in}}{\pgfqpoint{2.172137in}{1.768908in}}{\pgfqpoint{2.172137in}{1.777145in}}%
\pgfpathcurveto{\pgfqpoint{2.172137in}{1.785381in}}{\pgfqpoint{2.168865in}{1.793281in}}{\pgfqpoint{2.163041in}{1.799105in}}%
\pgfpathcurveto{\pgfqpoint{2.157217in}{1.804929in}}{\pgfqpoint{2.149317in}{1.808201in}}{\pgfqpoint{2.141081in}{1.808201in}}%
\pgfpathcurveto{\pgfqpoint{2.132844in}{1.808201in}}{\pgfqpoint{2.124944in}{1.804929in}}{\pgfqpoint{2.119120in}{1.799105in}}%
\pgfpathcurveto{\pgfqpoint{2.113296in}{1.793281in}}{\pgfqpoint{2.110024in}{1.785381in}}{\pgfqpoint{2.110024in}{1.777145in}}%
\pgfpathcurveto{\pgfqpoint{2.110024in}{1.768908in}}{\pgfqpoint{2.113296in}{1.761008in}}{\pgfqpoint{2.119120in}{1.755184in}}%
\pgfpathcurveto{\pgfqpoint{2.124944in}{1.749360in}}{\pgfqpoint{2.132844in}{1.746088in}}{\pgfqpoint{2.141081in}{1.746088in}}%
\pgfpathclose%
\pgfusepath{stroke,fill}%
\end{pgfscope}%
\begin{pgfscope}%
\pgfpathrectangle{\pgfqpoint{0.100000in}{0.212622in}}{\pgfqpoint{3.696000in}{3.696000in}}%
\pgfusepath{clip}%
\pgfsetbuttcap%
\pgfsetroundjoin%
\definecolor{currentfill}{rgb}{0.121569,0.466667,0.705882}%
\pgfsetfillcolor{currentfill}%
\pgfsetfillopacity{0.649249}%
\pgfsetlinewidth{1.003750pt}%
\definecolor{currentstroke}{rgb}{0.121569,0.466667,0.705882}%
\pgfsetstrokecolor{currentstroke}%
\pgfsetstrokeopacity{0.649249}%
\pgfsetdash{}{0pt}%
\pgfpathmoveto{\pgfqpoint{0.879009in}{1.408994in}}%
\pgfpathcurveto{\pgfqpoint{0.887245in}{1.408994in}}{\pgfqpoint{0.895145in}{1.412266in}}{\pgfqpoint{0.900969in}{1.418090in}}%
\pgfpathcurveto{\pgfqpoint{0.906793in}{1.423914in}}{\pgfqpoint{0.910065in}{1.431814in}}{\pgfqpoint{0.910065in}{1.440050in}}%
\pgfpathcurveto{\pgfqpoint{0.910065in}{1.448286in}}{\pgfqpoint{0.906793in}{1.456186in}}{\pgfqpoint{0.900969in}{1.462010in}}%
\pgfpathcurveto{\pgfqpoint{0.895145in}{1.467834in}}{\pgfqpoint{0.887245in}{1.471107in}}{\pgfqpoint{0.879009in}{1.471107in}}%
\pgfpathcurveto{\pgfqpoint{0.870773in}{1.471107in}}{\pgfqpoint{0.862873in}{1.467834in}}{\pgfqpoint{0.857049in}{1.462010in}}%
\pgfpathcurveto{\pgfqpoint{0.851225in}{1.456186in}}{\pgfqpoint{0.847952in}{1.448286in}}{\pgfqpoint{0.847952in}{1.440050in}}%
\pgfpathcurveto{\pgfqpoint{0.847952in}{1.431814in}}{\pgfqpoint{0.851225in}{1.423914in}}{\pgfqpoint{0.857049in}{1.418090in}}%
\pgfpathcurveto{\pgfqpoint{0.862873in}{1.412266in}}{\pgfqpoint{0.870773in}{1.408994in}}{\pgfqpoint{0.879009in}{1.408994in}}%
\pgfpathclose%
\pgfusepath{stroke,fill}%
\end{pgfscope}%
\begin{pgfscope}%
\pgfpathrectangle{\pgfqpoint{0.100000in}{0.212622in}}{\pgfqpoint{3.696000in}{3.696000in}}%
\pgfusepath{clip}%
\pgfsetbuttcap%
\pgfsetroundjoin%
\definecolor{currentfill}{rgb}{0.121569,0.466667,0.705882}%
\pgfsetfillcolor{currentfill}%
\pgfsetfillopacity{0.650125}%
\pgfsetlinewidth{1.003750pt}%
\definecolor{currentstroke}{rgb}{0.121569,0.466667,0.705882}%
\pgfsetstrokecolor{currentstroke}%
\pgfsetstrokeopacity{0.650125}%
\pgfsetdash{}{0pt}%
\pgfpathmoveto{\pgfqpoint{0.859232in}{1.436982in}}%
\pgfpathcurveto{\pgfqpoint{0.867469in}{1.436982in}}{\pgfqpoint{0.875369in}{1.440255in}}{\pgfqpoint{0.881193in}{1.446078in}}%
\pgfpathcurveto{\pgfqpoint{0.887017in}{1.451902in}}{\pgfqpoint{0.890289in}{1.459802in}}{\pgfqpoint{0.890289in}{1.468039in}}%
\pgfpathcurveto{\pgfqpoint{0.890289in}{1.476275in}}{\pgfqpoint{0.887017in}{1.484175in}}{\pgfqpoint{0.881193in}{1.489999in}}%
\pgfpathcurveto{\pgfqpoint{0.875369in}{1.495823in}}{\pgfqpoint{0.867469in}{1.499095in}}{\pgfqpoint{0.859232in}{1.499095in}}%
\pgfpathcurveto{\pgfqpoint{0.850996in}{1.499095in}}{\pgfqpoint{0.843096in}{1.495823in}}{\pgfqpoint{0.837272in}{1.489999in}}%
\pgfpathcurveto{\pgfqpoint{0.831448in}{1.484175in}}{\pgfqpoint{0.828176in}{1.476275in}}{\pgfqpoint{0.828176in}{1.468039in}}%
\pgfpathcurveto{\pgfqpoint{0.828176in}{1.459802in}}{\pgfqpoint{0.831448in}{1.451902in}}{\pgfqpoint{0.837272in}{1.446078in}}%
\pgfpathcurveto{\pgfqpoint{0.843096in}{1.440255in}}{\pgfqpoint{0.850996in}{1.436982in}}{\pgfqpoint{0.859232in}{1.436982in}}%
\pgfpathclose%
\pgfusepath{stroke,fill}%
\end{pgfscope}%
\begin{pgfscope}%
\pgfpathrectangle{\pgfqpoint{0.100000in}{0.212622in}}{\pgfqpoint{3.696000in}{3.696000in}}%
\pgfusepath{clip}%
\pgfsetbuttcap%
\pgfsetroundjoin%
\definecolor{currentfill}{rgb}{0.121569,0.466667,0.705882}%
\pgfsetfillcolor{currentfill}%
\pgfsetfillopacity{0.650857}%
\pgfsetlinewidth{1.003750pt}%
\definecolor{currentstroke}{rgb}{0.121569,0.466667,0.705882}%
\pgfsetstrokecolor{currentstroke}%
\pgfsetstrokeopacity{0.650857}%
\pgfsetdash{}{0pt}%
\pgfpathmoveto{\pgfqpoint{0.876546in}{1.410033in}}%
\pgfpathcurveto{\pgfqpoint{0.884783in}{1.410033in}}{\pgfqpoint{0.892683in}{1.413306in}}{\pgfqpoint{0.898507in}{1.419130in}}%
\pgfpathcurveto{\pgfqpoint{0.904331in}{1.424954in}}{\pgfqpoint{0.907603in}{1.432854in}}{\pgfqpoint{0.907603in}{1.441090in}}%
\pgfpathcurveto{\pgfqpoint{0.907603in}{1.449326in}}{\pgfqpoint{0.904331in}{1.457226in}}{\pgfqpoint{0.898507in}{1.463050in}}%
\pgfpathcurveto{\pgfqpoint{0.892683in}{1.468874in}}{\pgfqpoint{0.884783in}{1.472146in}}{\pgfqpoint{0.876546in}{1.472146in}}%
\pgfpathcurveto{\pgfqpoint{0.868310in}{1.472146in}}{\pgfqpoint{0.860410in}{1.468874in}}{\pgfqpoint{0.854586in}{1.463050in}}%
\pgfpathcurveto{\pgfqpoint{0.848762in}{1.457226in}}{\pgfqpoint{0.845490in}{1.449326in}}{\pgfqpoint{0.845490in}{1.441090in}}%
\pgfpathcurveto{\pgfqpoint{0.845490in}{1.432854in}}{\pgfqpoint{0.848762in}{1.424954in}}{\pgfqpoint{0.854586in}{1.419130in}}%
\pgfpathcurveto{\pgfqpoint{0.860410in}{1.413306in}}{\pgfqpoint{0.868310in}{1.410033in}}{\pgfqpoint{0.876546in}{1.410033in}}%
\pgfpathclose%
\pgfusepath{stroke,fill}%
\end{pgfscope}%
\begin{pgfscope}%
\pgfpathrectangle{\pgfqpoint{0.100000in}{0.212622in}}{\pgfqpoint{3.696000in}{3.696000in}}%
\pgfusepath{clip}%
\pgfsetbuttcap%
\pgfsetroundjoin%
\definecolor{currentfill}{rgb}{0.121569,0.466667,0.705882}%
\pgfsetfillcolor{currentfill}%
\pgfsetfillopacity{0.651230}%
\pgfsetlinewidth{1.003750pt}%
\definecolor{currentstroke}{rgb}{0.121569,0.466667,0.705882}%
\pgfsetstrokecolor{currentstroke}%
\pgfsetstrokeopacity{0.651230}%
\pgfsetdash{}{0pt}%
\pgfpathmoveto{\pgfqpoint{0.855668in}{1.435636in}}%
\pgfpathcurveto{\pgfqpoint{0.863904in}{1.435636in}}{\pgfqpoint{0.871804in}{1.438909in}}{\pgfqpoint{0.877628in}{1.444732in}}%
\pgfpathcurveto{\pgfqpoint{0.883452in}{1.450556in}}{\pgfqpoint{0.886724in}{1.458456in}}{\pgfqpoint{0.886724in}{1.466693in}}%
\pgfpathcurveto{\pgfqpoint{0.886724in}{1.474929in}}{\pgfqpoint{0.883452in}{1.482829in}}{\pgfqpoint{0.877628in}{1.488653in}}%
\pgfpathcurveto{\pgfqpoint{0.871804in}{1.494477in}}{\pgfqpoint{0.863904in}{1.497749in}}{\pgfqpoint{0.855668in}{1.497749in}}%
\pgfpathcurveto{\pgfqpoint{0.847431in}{1.497749in}}{\pgfqpoint{0.839531in}{1.494477in}}{\pgfqpoint{0.833708in}{1.488653in}}%
\pgfpathcurveto{\pgfqpoint{0.827884in}{1.482829in}}{\pgfqpoint{0.824611in}{1.474929in}}{\pgfqpoint{0.824611in}{1.466693in}}%
\pgfpathcurveto{\pgfqpoint{0.824611in}{1.458456in}}{\pgfqpoint{0.827884in}{1.450556in}}{\pgfqpoint{0.833708in}{1.444732in}}%
\pgfpathcurveto{\pgfqpoint{0.839531in}{1.438909in}}{\pgfqpoint{0.847431in}{1.435636in}}{\pgfqpoint{0.855668in}{1.435636in}}%
\pgfpathclose%
\pgfusepath{stroke,fill}%
\end{pgfscope}%
\begin{pgfscope}%
\pgfpathrectangle{\pgfqpoint{0.100000in}{0.212622in}}{\pgfqpoint{3.696000in}{3.696000in}}%
\pgfusepath{clip}%
\pgfsetbuttcap%
\pgfsetroundjoin%
\definecolor{currentfill}{rgb}{0.121569,0.466667,0.705882}%
\pgfsetfillcolor{currentfill}%
\pgfsetfillopacity{0.652170}%
\pgfsetlinewidth{1.003750pt}%
\definecolor{currentstroke}{rgb}{0.121569,0.466667,0.705882}%
\pgfsetstrokecolor{currentstroke}%
\pgfsetstrokeopacity{0.652170}%
\pgfsetdash{}{0pt}%
\pgfpathmoveto{\pgfqpoint{0.852889in}{1.434355in}}%
\pgfpathcurveto{\pgfqpoint{0.861125in}{1.434355in}}{\pgfqpoint{0.869025in}{1.437628in}}{\pgfqpoint{0.874849in}{1.443452in}}%
\pgfpathcurveto{\pgfqpoint{0.880673in}{1.449275in}}{\pgfqpoint{0.883945in}{1.457176in}}{\pgfqpoint{0.883945in}{1.465412in}}%
\pgfpathcurveto{\pgfqpoint{0.883945in}{1.473648in}}{\pgfqpoint{0.880673in}{1.481548in}}{\pgfqpoint{0.874849in}{1.487372in}}%
\pgfpathcurveto{\pgfqpoint{0.869025in}{1.493196in}}{\pgfqpoint{0.861125in}{1.496468in}}{\pgfqpoint{0.852889in}{1.496468in}}%
\pgfpathcurveto{\pgfqpoint{0.844652in}{1.496468in}}{\pgfqpoint{0.836752in}{1.493196in}}{\pgfqpoint{0.830929in}{1.487372in}}%
\pgfpathcurveto{\pgfqpoint{0.825105in}{1.481548in}}{\pgfqpoint{0.821832in}{1.473648in}}{\pgfqpoint{0.821832in}{1.465412in}}%
\pgfpathcurveto{\pgfqpoint{0.821832in}{1.457176in}}{\pgfqpoint{0.825105in}{1.449275in}}{\pgfqpoint{0.830929in}{1.443452in}}%
\pgfpathcurveto{\pgfqpoint{0.836752in}{1.437628in}}{\pgfqpoint{0.844652in}{1.434355in}}{\pgfqpoint{0.852889in}{1.434355in}}%
\pgfpathclose%
\pgfusepath{stroke,fill}%
\end{pgfscope}%
\begin{pgfscope}%
\pgfpathrectangle{\pgfqpoint{0.100000in}{0.212622in}}{\pgfqpoint{3.696000in}{3.696000in}}%
\pgfusepath{clip}%
\pgfsetbuttcap%
\pgfsetroundjoin%
\definecolor{currentfill}{rgb}{0.121569,0.466667,0.705882}%
\pgfsetfillcolor{currentfill}%
\pgfsetfillopacity{0.652723}%
\pgfsetlinewidth{1.003750pt}%
\definecolor{currentstroke}{rgb}{0.121569,0.466667,0.705882}%
\pgfsetstrokecolor{currentstroke}%
\pgfsetstrokeopacity{0.652723}%
\pgfsetdash{}{0pt}%
\pgfpathmoveto{\pgfqpoint{0.872935in}{1.411467in}}%
\pgfpathcurveto{\pgfqpoint{0.881172in}{1.411467in}}{\pgfqpoint{0.889072in}{1.414739in}}{\pgfqpoint{0.894896in}{1.420563in}}%
\pgfpathcurveto{\pgfqpoint{0.900720in}{1.426387in}}{\pgfqpoint{0.903992in}{1.434287in}}{\pgfqpoint{0.903992in}{1.442523in}}%
\pgfpathcurveto{\pgfqpoint{0.903992in}{1.450759in}}{\pgfqpoint{0.900720in}{1.458659in}}{\pgfqpoint{0.894896in}{1.464483in}}%
\pgfpathcurveto{\pgfqpoint{0.889072in}{1.470307in}}{\pgfqpoint{0.881172in}{1.473580in}}{\pgfqpoint{0.872935in}{1.473580in}}%
\pgfpathcurveto{\pgfqpoint{0.864699in}{1.473580in}}{\pgfqpoint{0.856799in}{1.470307in}}{\pgfqpoint{0.850975in}{1.464483in}}%
\pgfpathcurveto{\pgfqpoint{0.845151in}{1.458659in}}{\pgfqpoint{0.841879in}{1.450759in}}{\pgfqpoint{0.841879in}{1.442523in}}%
\pgfpathcurveto{\pgfqpoint{0.841879in}{1.434287in}}{\pgfqpoint{0.845151in}{1.426387in}}{\pgfqpoint{0.850975in}{1.420563in}}%
\pgfpathcurveto{\pgfqpoint{0.856799in}{1.414739in}}{\pgfqpoint{0.864699in}{1.411467in}}{\pgfqpoint{0.872935in}{1.411467in}}%
\pgfpathclose%
\pgfusepath{stroke,fill}%
\end{pgfscope}%
\begin{pgfscope}%
\pgfpathrectangle{\pgfqpoint{0.100000in}{0.212622in}}{\pgfqpoint{3.696000in}{3.696000in}}%
\pgfusepath{clip}%
\pgfsetbuttcap%
\pgfsetroundjoin%
\definecolor{currentfill}{rgb}{0.121569,0.466667,0.705882}%
\pgfsetfillcolor{currentfill}%
\pgfsetfillopacity{0.652826}%
\pgfsetlinewidth{1.003750pt}%
\definecolor{currentstroke}{rgb}{0.121569,0.466667,0.705882}%
\pgfsetstrokecolor{currentstroke}%
\pgfsetstrokeopacity{0.652826}%
\pgfsetdash{}{0pt}%
\pgfpathmoveto{\pgfqpoint{0.851099in}{1.434006in}}%
\pgfpathcurveto{\pgfqpoint{0.859335in}{1.434006in}}{\pgfqpoint{0.867236in}{1.437278in}}{\pgfqpoint{0.873059in}{1.443102in}}%
\pgfpathcurveto{\pgfqpoint{0.878883in}{1.448926in}}{\pgfqpoint{0.882156in}{1.456826in}}{\pgfqpoint{0.882156in}{1.465063in}}%
\pgfpathcurveto{\pgfqpoint{0.882156in}{1.473299in}}{\pgfqpoint{0.878883in}{1.481199in}}{\pgfqpoint{0.873059in}{1.487023in}}%
\pgfpathcurveto{\pgfqpoint{0.867236in}{1.492847in}}{\pgfqpoint{0.859335in}{1.496119in}}{\pgfqpoint{0.851099in}{1.496119in}}%
\pgfpathcurveto{\pgfqpoint{0.842863in}{1.496119in}}{\pgfqpoint{0.834963in}{1.492847in}}{\pgfqpoint{0.829139in}{1.487023in}}%
\pgfpathcurveto{\pgfqpoint{0.823315in}{1.481199in}}{\pgfqpoint{0.820043in}{1.473299in}}{\pgfqpoint{0.820043in}{1.465063in}}%
\pgfpathcurveto{\pgfqpoint{0.820043in}{1.456826in}}{\pgfqpoint{0.823315in}{1.448926in}}{\pgfqpoint{0.829139in}{1.443102in}}%
\pgfpathcurveto{\pgfqpoint{0.834963in}{1.437278in}}{\pgfqpoint{0.842863in}{1.434006in}}{\pgfqpoint{0.851099in}{1.434006in}}%
\pgfpathclose%
\pgfusepath{stroke,fill}%
\end{pgfscope}%
\begin{pgfscope}%
\pgfpathrectangle{\pgfqpoint{0.100000in}{0.212622in}}{\pgfqpoint{3.696000in}{3.696000in}}%
\pgfusepath{clip}%
\pgfsetbuttcap%
\pgfsetroundjoin%
\definecolor{currentfill}{rgb}{0.121569,0.466667,0.705882}%
\pgfsetfillcolor{currentfill}%
\pgfsetfillopacity{0.652868}%
\pgfsetlinewidth{1.003750pt}%
\definecolor{currentstroke}{rgb}{0.121569,0.466667,0.705882}%
\pgfsetstrokecolor{currentstroke}%
\pgfsetstrokeopacity{0.652868}%
\pgfsetdash{}{0pt}%
\pgfpathmoveto{\pgfqpoint{2.143094in}{1.743917in}}%
\pgfpathcurveto{\pgfqpoint{2.151330in}{1.743917in}}{\pgfqpoint{2.159230in}{1.747189in}}{\pgfqpoint{2.165054in}{1.753013in}}%
\pgfpathcurveto{\pgfqpoint{2.170878in}{1.758837in}}{\pgfqpoint{2.174150in}{1.766737in}}{\pgfqpoint{2.174150in}{1.774973in}}%
\pgfpathcurveto{\pgfqpoint{2.174150in}{1.783209in}}{\pgfqpoint{2.170878in}{1.791109in}}{\pgfqpoint{2.165054in}{1.796933in}}%
\pgfpathcurveto{\pgfqpoint{2.159230in}{1.802757in}}{\pgfqpoint{2.151330in}{1.806030in}}{\pgfqpoint{2.143094in}{1.806030in}}%
\pgfpathcurveto{\pgfqpoint{2.134858in}{1.806030in}}{\pgfqpoint{2.126958in}{1.802757in}}{\pgfqpoint{2.121134in}{1.796933in}}%
\pgfpathcurveto{\pgfqpoint{2.115310in}{1.791109in}}{\pgfqpoint{2.112037in}{1.783209in}}{\pgfqpoint{2.112037in}{1.774973in}}%
\pgfpathcurveto{\pgfqpoint{2.112037in}{1.766737in}}{\pgfqpoint{2.115310in}{1.758837in}}{\pgfqpoint{2.121134in}{1.753013in}}%
\pgfpathcurveto{\pgfqpoint{2.126958in}{1.747189in}}{\pgfqpoint{2.134858in}{1.743917in}}{\pgfqpoint{2.143094in}{1.743917in}}%
\pgfpathclose%
\pgfusepath{stroke,fill}%
\end{pgfscope}%
\begin{pgfscope}%
\pgfpathrectangle{\pgfqpoint{0.100000in}{0.212622in}}{\pgfqpoint{3.696000in}{3.696000in}}%
\pgfusepath{clip}%
\pgfsetbuttcap%
\pgfsetroundjoin%
\definecolor{currentfill}{rgb}{0.121569,0.466667,0.705882}%
\pgfsetfillcolor{currentfill}%
\pgfsetfillopacity{0.653751}%
\pgfsetlinewidth{1.003750pt}%
\definecolor{currentstroke}{rgb}{0.121569,0.466667,0.705882}%
\pgfsetstrokecolor{currentstroke}%
\pgfsetstrokeopacity{0.653751}%
\pgfsetdash{}{0pt}%
\pgfpathmoveto{\pgfqpoint{0.849354in}{1.430227in}}%
\pgfpathcurveto{\pgfqpoint{0.857590in}{1.430227in}}{\pgfqpoint{0.865490in}{1.433499in}}{\pgfqpoint{0.871314in}{1.439323in}}%
\pgfpathcurveto{\pgfqpoint{0.877138in}{1.445147in}}{\pgfqpoint{0.880410in}{1.453047in}}{\pgfqpoint{0.880410in}{1.461283in}}%
\pgfpathcurveto{\pgfqpoint{0.880410in}{1.469520in}}{\pgfqpoint{0.877138in}{1.477420in}}{\pgfqpoint{0.871314in}{1.483244in}}%
\pgfpathcurveto{\pgfqpoint{0.865490in}{1.489068in}}{\pgfqpoint{0.857590in}{1.492340in}}{\pgfqpoint{0.849354in}{1.492340in}}%
\pgfpathcurveto{\pgfqpoint{0.841117in}{1.492340in}}{\pgfqpoint{0.833217in}{1.489068in}}{\pgfqpoint{0.827393in}{1.483244in}}%
\pgfpathcurveto{\pgfqpoint{0.821569in}{1.477420in}}{\pgfqpoint{0.818297in}{1.469520in}}{\pgfqpoint{0.818297in}{1.461283in}}%
\pgfpathcurveto{\pgfqpoint{0.818297in}{1.453047in}}{\pgfqpoint{0.821569in}{1.445147in}}{\pgfqpoint{0.827393in}{1.439323in}}%
\pgfpathcurveto{\pgfqpoint{0.833217in}{1.433499in}}{\pgfqpoint{0.841117in}{1.430227in}}{\pgfqpoint{0.849354in}{1.430227in}}%
\pgfpathclose%
\pgfusepath{stroke,fill}%
\end{pgfscope}%
\begin{pgfscope}%
\pgfpathrectangle{\pgfqpoint{0.100000in}{0.212622in}}{\pgfqpoint{3.696000in}{3.696000in}}%
\pgfusepath{clip}%
\pgfsetbuttcap%
\pgfsetroundjoin%
\definecolor{currentfill}{rgb}{0.121569,0.466667,0.705882}%
\pgfsetfillcolor{currentfill}%
\pgfsetfillopacity{0.654773}%
\pgfsetlinewidth{1.003750pt}%
\definecolor{currentstroke}{rgb}{0.121569,0.466667,0.705882}%
\pgfsetstrokecolor{currentstroke}%
\pgfsetstrokeopacity{0.654773}%
\pgfsetdash{}{0pt}%
\pgfpathmoveto{\pgfqpoint{0.869345in}{1.412531in}}%
\pgfpathcurveto{\pgfqpoint{0.877581in}{1.412531in}}{\pgfqpoint{0.885481in}{1.415804in}}{\pgfqpoint{0.891305in}{1.421628in}}%
\pgfpathcurveto{\pgfqpoint{0.897129in}{1.427452in}}{\pgfqpoint{0.900401in}{1.435352in}}{\pgfqpoint{0.900401in}{1.443588in}}%
\pgfpathcurveto{\pgfqpoint{0.900401in}{1.451824in}}{\pgfqpoint{0.897129in}{1.459724in}}{\pgfqpoint{0.891305in}{1.465548in}}%
\pgfpathcurveto{\pgfqpoint{0.885481in}{1.471372in}}{\pgfqpoint{0.877581in}{1.474644in}}{\pgfqpoint{0.869345in}{1.474644in}}%
\pgfpathcurveto{\pgfqpoint{0.861109in}{1.474644in}}{\pgfqpoint{0.853209in}{1.471372in}}{\pgfqpoint{0.847385in}{1.465548in}}%
\pgfpathcurveto{\pgfqpoint{0.841561in}{1.459724in}}{\pgfqpoint{0.838288in}{1.451824in}}{\pgfqpoint{0.838288in}{1.443588in}}%
\pgfpathcurveto{\pgfqpoint{0.838288in}{1.435352in}}{\pgfqpoint{0.841561in}{1.427452in}}{\pgfqpoint{0.847385in}{1.421628in}}%
\pgfpathcurveto{\pgfqpoint{0.853209in}{1.415804in}}{\pgfqpoint{0.861109in}{1.412531in}}{\pgfqpoint{0.869345in}{1.412531in}}%
\pgfpathclose%
\pgfusepath{stroke,fill}%
\end{pgfscope}%
\begin{pgfscope}%
\pgfpathrectangle{\pgfqpoint{0.100000in}{0.212622in}}{\pgfqpoint{3.696000in}{3.696000in}}%
\pgfusepath{clip}%
\pgfsetbuttcap%
\pgfsetroundjoin%
\definecolor{currentfill}{rgb}{0.121569,0.466667,0.705882}%
\pgfsetfillcolor{currentfill}%
\pgfsetfillopacity{0.655513}%
\pgfsetlinewidth{1.003750pt}%
\definecolor{currentstroke}{rgb}{0.121569,0.466667,0.705882}%
\pgfsetstrokecolor{currentstroke}%
\pgfsetstrokeopacity{0.655513}%
\pgfsetdash{}{0pt}%
\pgfpathmoveto{\pgfqpoint{0.848760in}{1.423327in}}%
\pgfpathcurveto{\pgfqpoint{0.856997in}{1.423327in}}{\pgfqpoint{0.864897in}{1.426599in}}{\pgfqpoint{0.870721in}{1.432423in}}%
\pgfpathcurveto{\pgfqpoint{0.876545in}{1.438247in}}{\pgfqpoint{0.879817in}{1.446147in}}{\pgfqpoint{0.879817in}{1.454383in}}%
\pgfpathcurveto{\pgfqpoint{0.879817in}{1.462619in}}{\pgfqpoint{0.876545in}{1.470519in}}{\pgfqpoint{0.870721in}{1.476343in}}%
\pgfpathcurveto{\pgfqpoint{0.864897in}{1.482167in}}{\pgfqpoint{0.856997in}{1.485440in}}{\pgfqpoint{0.848760in}{1.485440in}}%
\pgfpathcurveto{\pgfqpoint{0.840524in}{1.485440in}}{\pgfqpoint{0.832624in}{1.482167in}}{\pgfqpoint{0.826800in}{1.476343in}}%
\pgfpathcurveto{\pgfqpoint{0.820976in}{1.470519in}}{\pgfqpoint{0.817704in}{1.462619in}}{\pgfqpoint{0.817704in}{1.454383in}}%
\pgfpathcurveto{\pgfqpoint{0.817704in}{1.446147in}}{\pgfqpoint{0.820976in}{1.438247in}}{\pgfqpoint{0.826800in}{1.432423in}}%
\pgfpathcurveto{\pgfqpoint{0.832624in}{1.426599in}}{\pgfqpoint{0.840524in}{1.423327in}}{\pgfqpoint{0.848760in}{1.423327in}}%
\pgfpathclose%
\pgfusepath{stroke,fill}%
\end{pgfscope}%
\begin{pgfscope}%
\pgfpathrectangle{\pgfqpoint{0.100000in}{0.212622in}}{\pgfqpoint{3.696000in}{3.696000in}}%
\pgfusepath{clip}%
\pgfsetbuttcap%
\pgfsetroundjoin%
\definecolor{currentfill}{rgb}{0.121569,0.466667,0.705882}%
\pgfsetfillcolor{currentfill}%
\pgfsetfillopacity{0.655917}%
\pgfsetlinewidth{1.003750pt}%
\definecolor{currentstroke}{rgb}{0.121569,0.466667,0.705882}%
\pgfsetstrokecolor{currentstroke}%
\pgfsetstrokeopacity{0.655917}%
\pgfsetdash{}{0pt}%
\pgfpathmoveto{\pgfqpoint{0.867054in}{1.413593in}}%
\pgfpathcurveto{\pgfqpoint{0.875290in}{1.413593in}}{\pgfqpoint{0.883190in}{1.416865in}}{\pgfqpoint{0.889014in}{1.422689in}}%
\pgfpathcurveto{\pgfqpoint{0.894838in}{1.428513in}}{\pgfqpoint{0.898110in}{1.436413in}}{\pgfqpoint{0.898110in}{1.444649in}}%
\pgfpathcurveto{\pgfqpoint{0.898110in}{1.452885in}}{\pgfqpoint{0.894838in}{1.460786in}}{\pgfqpoint{0.889014in}{1.466609in}}%
\pgfpathcurveto{\pgfqpoint{0.883190in}{1.472433in}}{\pgfqpoint{0.875290in}{1.475706in}}{\pgfqpoint{0.867054in}{1.475706in}}%
\pgfpathcurveto{\pgfqpoint{0.858817in}{1.475706in}}{\pgfqpoint{0.850917in}{1.472433in}}{\pgfqpoint{0.845093in}{1.466609in}}%
\pgfpathcurveto{\pgfqpoint{0.839270in}{1.460786in}}{\pgfqpoint{0.835997in}{1.452885in}}{\pgfqpoint{0.835997in}{1.444649in}}%
\pgfpathcurveto{\pgfqpoint{0.835997in}{1.436413in}}{\pgfqpoint{0.839270in}{1.428513in}}{\pgfqpoint{0.845093in}{1.422689in}}%
\pgfpathcurveto{\pgfqpoint{0.850917in}{1.416865in}}{\pgfqpoint{0.858817in}{1.413593in}}{\pgfqpoint{0.867054in}{1.413593in}}%
\pgfpathclose%
\pgfusepath{stroke,fill}%
\end{pgfscope}%
\begin{pgfscope}%
\pgfpathrectangle{\pgfqpoint{0.100000in}{0.212622in}}{\pgfqpoint{3.696000in}{3.696000in}}%
\pgfusepath{clip}%
\pgfsetbuttcap%
\pgfsetroundjoin%
\definecolor{currentfill}{rgb}{0.121569,0.466667,0.705882}%
\pgfsetfillcolor{currentfill}%
\pgfsetfillopacity{0.656559}%
\pgfsetlinewidth{1.003750pt}%
\definecolor{currentstroke}{rgb}{0.121569,0.466667,0.705882}%
\pgfsetstrokecolor{currentstroke}%
\pgfsetstrokeopacity{0.656559}%
\pgfsetdash{}{0pt}%
\pgfpathmoveto{\pgfqpoint{0.866012in}{1.414001in}}%
\pgfpathcurveto{\pgfqpoint{0.874248in}{1.414001in}}{\pgfqpoint{0.882148in}{1.417274in}}{\pgfqpoint{0.887972in}{1.423098in}}%
\pgfpathcurveto{\pgfqpoint{0.893796in}{1.428922in}}{\pgfqpoint{0.897069in}{1.436822in}}{\pgfqpoint{0.897069in}{1.445058in}}%
\pgfpathcurveto{\pgfqpoint{0.897069in}{1.453294in}}{\pgfqpoint{0.893796in}{1.461194in}}{\pgfqpoint{0.887972in}{1.467018in}}%
\pgfpathcurveto{\pgfqpoint{0.882148in}{1.472842in}}{\pgfqpoint{0.874248in}{1.476114in}}{\pgfqpoint{0.866012in}{1.476114in}}%
\pgfpathcurveto{\pgfqpoint{0.857776in}{1.476114in}}{\pgfqpoint{0.849876in}{1.472842in}}{\pgfqpoint{0.844052in}{1.467018in}}%
\pgfpathcurveto{\pgfqpoint{0.838228in}{1.461194in}}{\pgfqpoint{0.834956in}{1.453294in}}{\pgfqpoint{0.834956in}{1.445058in}}%
\pgfpathcurveto{\pgfqpoint{0.834956in}{1.436822in}}{\pgfqpoint{0.838228in}{1.428922in}}{\pgfqpoint{0.844052in}{1.423098in}}%
\pgfpathcurveto{\pgfqpoint{0.849876in}{1.417274in}}{\pgfqpoint{0.857776in}{1.414001in}}{\pgfqpoint{0.866012in}{1.414001in}}%
\pgfpathclose%
\pgfusepath{stroke,fill}%
\end{pgfscope}%
\begin{pgfscope}%
\pgfpathrectangle{\pgfqpoint{0.100000in}{0.212622in}}{\pgfqpoint{3.696000in}{3.696000in}}%
\pgfusepath{clip}%
\pgfsetbuttcap%
\pgfsetroundjoin%
\definecolor{currentfill}{rgb}{0.121569,0.466667,0.705882}%
\pgfsetfillcolor{currentfill}%
\pgfsetfillopacity{0.656695}%
\pgfsetlinewidth{1.003750pt}%
\definecolor{currentstroke}{rgb}{0.121569,0.466667,0.705882}%
\pgfsetstrokecolor{currentstroke}%
\pgfsetstrokeopacity{0.656695}%
\pgfsetdash{}{0pt}%
\pgfpathmoveto{\pgfqpoint{0.851847in}{1.416599in}}%
\pgfpathcurveto{\pgfqpoint{0.860083in}{1.416599in}}{\pgfqpoint{0.867983in}{1.419872in}}{\pgfqpoint{0.873807in}{1.425695in}}%
\pgfpathcurveto{\pgfqpoint{0.879631in}{1.431519in}}{\pgfqpoint{0.882903in}{1.439419in}}{\pgfqpoint{0.882903in}{1.447656in}}%
\pgfpathcurveto{\pgfqpoint{0.882903in}{1.455892in}}{\pgfqpoint{0.879631in}{1.463792in}}{\pgfqpoint{0.873807in}{1.469616in}}%
\pgfpathcurveto{\pgfqpoint{0.867983in}{1.475440in}}{\pgfqpoint{0.860083in}{1.478712in}}{\pgfqpoint{0.851847in}{1.478712in}}%
\pgfpathcurveto{\pgfqpoint{0.843610in}{1.478712in}}{\pgfqpoint{0.835710in}{1.475440in}}{\pgfqpoint{0.829886in}{1.469616in}}%
\pgfpathcurveto{\pgfqpoint{0.824063in}{1.463792in}}{\pgfqpoint{0.820790in}{1.455892in}}{\pgfqpoint{0.820790in}{1.447656in}}%
\pgfpathcurveto{\pgfqpoint{0.820790in}{1.439419in}}{\pgfqpoint{0.824063in}{1.431519in}}{\pgfqpoint{0.829886in}{1.425695in}}%
\pgfpathcurveto{\pgfqpoint{0.835710in}{1.419872in}}{\pgfqpoint{0.843610in}{1.416599in}}{\pgfqpoint{0.851847in}{1.416599in}}%
\pgfpathclose%
\pgfusepath{stroke,fill}%
\end{pgfscope}%
\begin{pgfscope}%
\pgfpathrectangle{\pgfqpoint{0.100000in}{0.212622in}}{\pgfqpoint{3.696000in}{3.696000in}}%
\pgfusepath{clip}%
\pgfsetbuttcap%
\pgfsetroundjoin%
\definecolor{currentfill}{rgb}{0.121569,0.466667,0.705882}%
\pgfsetfillcolor{currentfill}%
\pgfsetfillopacity{0.656913}%
\pgfsetlinewidth{1.003750pt}%
\definecolor{currentstroke}{rgb}{0.121569,0.466667,0.705882}%
\pgfsetstrokecolor{currentstroke}%
\pgfsetstrokeopacity{0.656913}%
\pgfsetdash{}{0pt}%
\pgfpathmoveto{\pgfqpoint{0.865352in}{1.414323in}}%
\pgfpathcurveto{\pgfqpoint{0.873589in}{1.414323in}}{\pgfqpoint{0.881489in}{1.417595in}}{\pgfqpoint{0.887313in}{1.423419in}}%
\pgfpathcurveto{\pgfqpoint{0.893136in}{1.429243in}}{\pgfqpoint{0.896409in}{1.437143in}}{\pgfqpoint{0.896409in}{1.445379in}}%
\pgfpathcurveto{\pgfqpoint{0.896409in}{1.453616in}}{\pgfqpoint{0.893136in}{1.461516in}}{\pgfqpoint{0.887313in}{1.467340in}}%
\pgfpathcurveto{\pgfqpoint{0.881489in}{1.473163in}}{\pgfqpoint{0.873589in}{1.476436in}}{\pgfqpoint{0.865352in}{1.476436in}}%
\pgfpathcurveto{\pgfqpoint{0.857116in}{1.476436in}}{\pgfqpoint{0.849216in}{1.473163in}}{\pgfqpoint{0.843392in}{1.467340in}}%
\pgfpathcurveto{\pgfqpoint{0.837568in}{1.461516in}}{\pgfqpoint{0.834296in}{1.453616in}}{\pgfqpoint{0.834296in}{1.445379in}}%
\pgfpathcurveto{\pgfqpoint{0.834296in}{1.437143in}}{\pgfqpoint{0.837568in}{1.429243in}}{\pgfqpoint{0.843392in}{1.423419in}}%
\pgfpathcurveto{\pgfqpoint{0.849216in}{1.417595in}}{\pgfqpoint{0.857116in}{1.414323in}}{\pgfqpoint{0.865352in}{1.414323in}}%
\pgfpathclose%
\pgfusepath{stroke,fill}%
\end{pgfscope}%
\begin{pgfscope}%
\pgfpathrectangle{\pgfqpoint{0.100000in}{0.212622in}}{\pgfqpoint{3.696000in}{3.696000in}}%
\pgfusepath{clip}%
\pgfsetbuttcap%
\pgfsetroundjoin%
\definecolor{currentfill}{rgb}{0.121569,0.466667,0.705882}%
\pgfsetfillcolor{currentfill}%
\pgfsetfillopacity{0.657099}%
\pgfsetlinewidth{1.003750pt}%
\definecolor{currentstroke}{rgb}{0.121569,0.466667,0.705882}%
\pgfsetstrokecolor{currentstroke}%
\pgfsetstrokeopacity{0.657099}%
\pgfsetdash{}{0pt}%
\pgfpathmoveto{\pgfqpoint{0.865009in}{1.414430in}}%
\pgfpathcurveto{\pgfqpoint{0.873245in}{1.414430in}}{\pgfqpoint{0.881145in}{1.417702in}}{\pgfqpoint{0.886969in}{1.423526in}}%
\pgfpathcurveto{\pgfqpoint{0.892793in}{1.429350in}}{\pgfqpoint{0.896065in}{1.437250in}}{\pgfqpoint{0.896065in}{1.445486in}}%
\pgfpathcurveto{\pgfqpoint{0.896065in}{1.453723in}}{\pgfqpoint{0.892793in}{1.461623in}}{\pgfqpoint{0.886969in}{1.467447in}}%
\pgfpathcurveto{\pgfqpoint{0.881145in}{1.473271in}}{\pgfqpoint{0.873245in}{1.476543in}}{\pgfqpoint{0.865009in}{1.476543in}}%
\pgfpathcurveto{\pgfqpoint{0.856773in}{1.476543in}}{\pgfqpoint{0.848873in}{1.473271in}}{\pgfqpoint{0.843049in}{1.467447in}}%
\pgfpathcurveto{\pgfqpoint{0.837225in}{1.461623in}}{\pgfqpoint{0.833952in}{1.453723in}}{\pgfqpoint{0.833952in}{1.445486in}}%
\pgfpathcurveto{\pgfqpoint{0.833952in}{1.437250in}}{\pgfqpoint{0.837225in}{1.429350in}}{\pgfqpoint{0.843049in}{1.423526in}}%
\pgfpathcurveto{\pgfqpoint{0.848873in}{1.417702in}}{\pgfqpoint{0.856773in}{1.414430in}}{\pgfqpoint{0.865009in}{1.414430in}}%
\pgfpathclose%
\pgfusepath{stroke,fill}%
\end{pgfscope}%
\begin{pgfscope}%
\pgfpathrectangle{\pgfqpoint{0.100000in}{0.212622in}}{\pgfqpoint{3.696000in}{3.696000in}}%
\pgfusepath{clip}%
\pgfsetbuttcap%
\pgfsetroundjoin%
\definecolor{currentfill}{rgb}{0.121569,0.466667,0.705882}%
\pgfsetfillcolor{currentfill}%
\pgfsetfillopacity{0.657209}%
\pgfsetlinewidth{1.003750pt}%
\definecolor{currentstroke}{rgb}{0.121569,0.466667,0.705882}%
\pgfsetstrokecolor{currentstroke}%
\pgfsetstrokeopacity{0.657209}%
\pgfsetdash{}{0pt}%
\pgfpathmoveto{\pgfqpoint{0.864820in}{1.414533in}}%
\pgfpathcurveto{\pgfqpoint{0.873056in}{1.414533in}}{\pgfqpoint{0.880956in}{1.417806in}}{\pgfqpoint{0.886780in}{1.423629in}}%
\pgfpathcurveto{\pgfqpoint{0.892604in}{1.429453in}}{\pgfqpoint{0.895876in}{1.437353in}}{\pgfqpoint{0.895876in}{1.445590in}}%
\pgfpathcurveto{\pgfqpoint{0.895876in}{1.453826in}}{\pgfqpoint{0.892604in}{1.461726in}}{\pgfqpoint{0.886780in}{1.467550in}}%
\pgfpathcurveto{\pgfqpoint{0.880956in}{1.473374in}}{\pgfqpoint{0.873056in}{1.476646in}}{\pgfqpoint{0.864820in}{1.476646in}}%
\pgfpathcurveto{\pgfqpoint{0.856583in}{1.476646in}}{\pgfqpoint{0.848683in}{1.473374in}}{\pgfqpoint{0.842859in}{1.467550in}}%
\pgfpathcurveto{\pgfqpoint{0.837035in}{1.461726in}}{\pgfqpoint{0.833763in}{1.453826in}}{\pgfqpoint{0.833763in}{1.445590in}}%
\pgfpathcurveto{\pgfqpoint{0.833763in}{1.437353in}}{\pgfqpoint{0.837035in}{1.429453in}}{\pgfqpoint{0.842859in}{1.423629in}}%
\pgfpathcurveto{\pgfqpoint{0.848683in}{1.417806in}}{\pgfqpoint{0.856583in}{1.414533in}}{\pgfqpoint{0.864820in}{1.414533in}}%
\pgfpathclose%
\pgfusepath{stroke,fill}%
\end{pgfscope}%
\begin{pgfscope}%
\pgfpathrectangle{\pgfqpoint{0.100000in}{0.212622in}}{\pgfqpoint{3.696000in}{3.696000in}}%
\pgfusepath{clip}%
\pgfsetbuttcap%
\pgfsetroundjoin%
\definecolor{currentfill}{rgb}{0.121569,0.466667,0.705882}%
\pgfsetfillcolor{currentfill}%
\pgfsetfillopacity{0.657267}%
\pgfsetlinewidth{1.003750pt}%
\definecolor{currentstroke}{rgb}{0.121569,0.466667,0.705882}%
\pgfsetstrokecolor{currentstroke}%
\pgfsetstrokeopacity{0.657267}%
\pgfsetdash{}{0pt}%
\pgfpathmoveto{\pgfqpoint{0.864719in}{1.414570in}}%
\pgfpathcurveto{\pgfqpoint{0.872955in}{1.414570in}}{\pgfqpoint{0.880855in}{1.417842in}}{\pgfqpoint{0.886679in}{1.423666in}}%
\pgfpathcurveto{\pgfqpoint{0.892503in}{1.429490in}}{\pgfqpoint{0.895775in}{1.437390in}}{\pgfqpoint{0.895775in}{1.445626in}}%
\pgfpathcurveto{\pgfqpoint{0.895775in}{1.453863in}}{\pgfqpoint{0.892503in}{1.461763in}}{\pgfqpoint{0.886679in}{1.467587in}}%
\pgfpathcurveto{\pgfqpoint{0.880855in}{1.473411in}}{\pgfqpoint{0.872955in}{1.476683in}}{\pgfqpoint{0.864719in}{1.476683in}}%
\pgfpathcurveto{\pgfqpoint{0.856483in}{1.476683in}}{\pgfqpoint{0.848583in}{1.473411in}}{\pgfqpoint{0.842759in}{1.467587in}}%
\pgfpathcurveto{\pgfqpoint{0.836935in}{1.461763in}}{\pgfqpoint{0.833662in}{1.453863in}}{\pgfqpoint{0.833662in}{1.445626in}}%
\pgfpathcurveto{\pgfqpoint{0.833662in}{1.437390in}}{\pgfqpoint{0.836935in}{1.429490in}}{\pgfqpoint{0.842759in}{1.423666in}}%
\pgfpathcurveto{\pgfqpoint{0.848583in}{1.417842in}}{\pgfqpoint{0.856483in}{1.414570in}}{\pgfqpoint{0.864719in}{1.414570in}}%
\pgfpathclose%
\pgfusepath{stroke,fill}%
\end{pgfscope}%
\begin{pgfscope}%
\pgfpathrectangle{\pgfqpoint{0.100000in}{0.212622in}}{\pgfqpoint{3.696000in}{3.696000in}}%
\pgfusepath{clip}%
\pgfsetbuttcap%
\pgfsetroundjoin%
\definecolor{currentfill}{rgb}{0.121569,0.466667,0.705882}%
\pgfsetfillcolor{currentfill}%
\pgfsetfillopacity{0.657300}%
\pgfsetlinewidth{1.003750pt}%
\definecolor{currentstroke}{rgb}{0.121569,0.466667,0.705882}%
\pgfsetstrokecolor{currentstroke}%
\pgfsetstrokeopacity{0.657300}%
\pgfsetdash{}{0pt}%
\pgfpathmoveto{\pgfqpoint{0.864659in}{1.414601in}}%
\pgfpathcurveto{\pgfqpoint{0.872895in}{1.414601in}}{\pgfqpoint{0.880795in}{1.417873in}}{\pgfqpoint{0.886619in}{1.423697in}}%
\pgfpathcurveto{\pgfqpoint{0.892443in}{1.429521in}}{\pgfqpoint{0.895715in}{1.437421in}}{\pgfqpoint{0.895715in}{1.445657in}}%
\pgfpathcurveto{\pgfqpoint{0.895715in}{1.453893in}}{\pgfqpoint{0.892443in}{1.461793in}}{\pgfqpoint{0.886619in}{1.467617in}}%
\pgfpathcurveto{\pgfqpoint{0.880795in}{1.473441in}}{\pgfqpoint{0.872895in}{1.476714in}}{\pgfqpoint{0.864659in}{1.476714in}}%
\pgfpathcurveto{\pgfqpoint{0.856422in}{1.476714in}}{\pgfqpoint{0.848522in}{1.473441in}}{\pgfqpoint{0.842698in}{1.467617in}}%
\pgfpathcurveto{\pgfqpoint{0.836874in}{1.461793in}}{\pgfqpoint{0.833602in}{1.453893in}}{\pgfqpoint{0.833602in}{1.445657in}}%
\pgfpathcurveto{\pgfqpoint{0.833602in}{1.437421in}}{\pgfqpoint{0.836874in}{1.429521in}}{\pgfqpoint{0.842698in}{1.423697in}}%
\pgfpathcurveto{\pgfqpoint{0.848522in}{1.417873in}}{\pgfqpoint{0.856422in}{1.414601in}}{\pgfqpoint{0.864659in}{1.414601in}}%
\pgfpathclose%
\pgfusepath{stroke,fill}%
\end{pgfscope}%
\begin{pgfscope}%
\pgfpathrectangle{\pgfqpoint{0.100000in}{0.212622in}}{\pgfqpoint{3.696000in}{3.696000in}}%
\pgfusepath{clip}%
\pgfsetbuttcap%
\pgfsetroundjoin%
\definecolor{currentfill}{rgb}{0.121569,0.466667,0.705882}%
\pgfsetfillcolor{currentfill}%
\pgfsetfillopacity{0.657318}%
\pgfsetlinewidth{1.003750pt}%
\definecolor{currentstroke}{rgb}{0.121569,0.466667,0.705882}%
\pgfsetstrokecolor{currentstroke}%
\pgfsetstrokeopacity{0.657318}%
\pgfsetdash{}{0pt}%
\pgfpathmoveto{\pgfqpoint{0.864628in}{1.414617in}}%
\pgfpathcurveto{\pgfqpoint{0.872865in}{1.414617in}}{\pgfqpoint{0.880765in}{1.417889in}}{\pgfqpoint{0.886589in}{1.423713in}}%
\pgfpathcurveto{\pgfqpoint{0.892412in}{1.429537in}}{\pgfqpoint{0.895685in}{1.437437in}}{\pgfqpoint{0.895685in}{1.445673in}}%
\pgfpathcurveto{\pgfqpoint{0.895685in}{1.453909in}}{\pgfqpoint{0.892412in}{1.461809in}}{\pgfqpoint{0.886589in}{1.467633in}}%
\pgfpathcurveto{\pgfqpoint{0.880765in}{1.473457in}}{\pgfqpoint{0.872865in}{1.476730in}}{\pgfqpoint{0.864628in}{1.476730in}}%
\pgfpathcurveto{\pgfqpoint{0.856392in}{1.476730in}}{\pgfqpoint{0.848492in}{1.473457in}}{\pgfqpoint{0.842668in}{1.467633in}}%
\pgfpathcurveto{\pgfqpoint{0.836844in}{1.461809in}}{\pgfqpoint{0.833572in}{1.453909in}}{\pgfqpoint{0.833572in}{1.445673in}}%
\pgfpathcurveto{\pgfqpoint{0.833572in}{1.437437in}}{\pgfqpoint{0.836844in}{1.429537in}}{\pgfqpoint{0.842668in}{1.423713in}}%
\pgfpathcurveto{\pgfqpoint{0.848492in}{1.417889in}}{\pgfqpoint{0.856392in}{1.414617in}}{\pgfqpoint{0.864628in}{1.414617in}}%
\pgfpathclose%
\pgfusepath{stroke,fill}%
\end{pgfscope}%
\begin{pgfscope}%
\pgfpathrectangle{\pgfqpoint{0.100000in}{0.212622in}}{\pgfqpoint{3.696000in}{3.696000in}}%
\pgfusepath{clip}%
\pgfsetbuttcap%
\pgfsetroundjoin%
\definecolor{currentfill}{rgb}{0.121569,0.466667,0.705882}%
\pgfsetfillcolor{currentfill}%
\pgfsetfillopacity{0.657328}%
\pgfsetlinewidth{1.003750pt}%
\definecolor{currentstroke}{rgb}{0.121569,0.466667,0.705882}%
\pgfsetstrokecolor{currentstroke}%
\pgfsetstrokeopacity{0.657328}%
\pgfsetdash{}{0pt}%
\pgfpathmoveto{\pgfqpoint{0.864610in}{1.414628in}}%
\pgfpathcurveto{\pgfqpoint{0.872846in}{1.414628in}}{\pgfqpoint{0.880746in}{1.417900in}}{\pgfqpoint{0.886570in}{1.423724in}}%
\pgfpathcurveto{\pgfqpoint{0.892394in}{1.429548in}}{\pgfqpoint{0.895666in}{1.437448in}}{\pgfqpoint{0.895666in}{1.445684in}}%
\pgfpathcurveto{\pgfqpoint{0.895666in}{1.453921in}}{\pgfqpoint{0.892394in}{1.461821in}}{\pgfqpoint{0.886570in}{1.467645in}}%
\pgfpathcurveto{\pgfqpoint{0.880746in}{1.473469in}}{\pgfqpoint{0.872846in}{1.476741in}}{\pgfqpoint{0.864610in}{1.476741in}}%
\pgfpathcurveto{\pgfqpoint{0.856374in}{1.476741in}}{\pgfqpoint{0.848474in}{1.473469in}}{\pgfqpoint{0.842650in}{1.467645in}}%
\pgfpathcurveto{\pgfqpoint{0.836826in}{1.461821in}}{\pgfqpoint{0.833553in}{1.453921in}}{\pgfqpoint{0.833553in}{1.445684in}}%
\pgfpathcurveto{\pgfqpoint{0.833553in}{1.437448in}}{\pgfqpoint{0.836826in}{1.429548in}}{\pgfqpoint{0.842650in}{1.423724in}}%
\pgfpathcurveto{\pgfqpoint{0.848474in}{1.417900in}}{\pgfqpoint{0.856374in}{1.414628in}}{\pgfqpoint{0.864610in}{1.414628in}}%
\pgfpathclose%
\pgfusepath{stroke,fill}%
\end{pgfscope}%
\begin{pgfscope}%
\pgfpathrectangle{\pgfqpoint{0.100000in}{0.212622in}}{\pgfqpoint{3.696000in}{3.696000in}}%
\pgfusepath{clip}%
\pgfsetbuttcap%
\pgfsetroundjoin%
\definecolor{currentfill}{rgb}{0.121569,0.466667,0.705882}%
\pgfsetfillcolor{currentfill}%
\pgfsetfillopacity{0.657333}%
\pgfsetlinewidth{1.003750pt}%
\definecolor{currentstroke}{rgb}{0.121569,0.466667,0.705882}%
\pgfsetstrokecolor{currentstroke}%
\pgfsetstrokeopacity{0.657333}%
\pgfsetdash{}{0pt}%
\pgfpathmoveto{\pgfqpoint{0.864600in}{1.414632in}}%
\pgfpathcurveto{\pgfqpoint{0.872837in}{1.414632in}}{\pgfqpoint{0.880737in}{1.417904in}}{\pgfqpoint{0.886561in}{1.423728in}}%
\pgfpathcurveto{\pgfqpoint{0.892385in}{1.429552in}}{\pgfqpoint{0.895657in}{1.437452in}}{\pgfqpoint{0.895657in}{1.445688in}}%
\pgfpathcurveto{\pgfqpoint{0.895657in}{1.453925in}}{\pgfqpoint{0.892385in}{1.461825in}}{\pgfqpoint{0.886561in}{1.467648in}}%
\pgfpathcurveto{\pgfqpoint{0.880737in}{1.473472in}}{\pgfqpoint{0.872837in}{1.476745in}}{\pgfqpoint{0.864600in}{1.476745in}}%
\pgfpathcurveto{\pgfqpoint{0.856364in}{1.476745in}}{\pgfqpoint{0.848464in}{1.473472in}}{\pgfqpoint{0.842640in}{1.467648in}}%
\pgfpathcurveto{\pgfqpoint{0.836816in}{1.461825in}}{\pgfqpoint{0.833544in}{1.453925in}}{\pgfqpoint{0.833544in}{1.445688in}}%
\pgfpathcurveto{\pgfqpoint{0.833544in}{1.437452in}}{\pgfqpoint{0.836816in}{1.429552in}}{\pgfqpoint{0.842640in}{1.423728in}}%
\pgfpathcurveto{\pgfqpoint{0.848464in}{1.417904in}}{\pgfqpoint{0.856364in}{1.414632in}}{\pgfqpoint{0.864600in}{1.414632in}}%
\pgfpathclose%
\pgfusepath{stroke,fill}%
\end{pgfscope}%
\begin{pgfscope}%
\pgfpathrectangle{\pgfqpoint{0.100000in}{0.212622in}}{\pgfqpoint{3.696000in}{3.696000in}}%
\pgfusepath{clip}%
\pgfsetbuttcap%
\pgfsetroundjoin%
\definecolor{currentfill}{rgb}{0.121569,0.466667,0.705882}%
\pgfsetfillcolor{currentfill}%
\pgfsetfillopacity{0.657336}%
\pgfsetlinewidth{1.003750pt}%
\definecolor{currentstroke}{rgb}{0.121569,0.466667,0.705882}%
\pgfsetstrokecolor{currentstroke}%
\pgfsetstrokeopacity{0.657336}%
\pgfsetdash{}{0pt}%
\pgfpathmoveto{\pgfqpoint{0.864595in}{1.414635in}}%
\pgfpathcurveto{\pgfqpoint{0.872831in}{1.414635in}}{\pgfqpoint{0.880731in}{1.417907in}}{\pgfqpoint{0.886555in}{1.423731in}}%
\pgfpathcurveto{\pgfqpoint{0.892379in}{1.429555in}}{\pgfqpoint{0.895651in}{1.437455in}}{\pgfqpoint{0.895651in}{1.445692in}}%
\pgfpathcurveto{\pgfqpoint{0.895651in}{1.453928in}}{\pgfqpoint{0.892379in}{1.461828in}}{\pgfqpoint{0.886555in}{1.467652in}}%
\pgfpathcurveto{\pgfqpoint{0.880731in}{1.473476in}}{\pgfqpoint{0.872831in}{1.476748in}}{\pgfqpoint{0.864595in}{1.476748in}}%
\pgfpathcurveto{\pgfqpoint{0.856358in}{1.476748in}}{\pgfqpoint{0.848458in}{1.473476in}}{\pgfqpoint{0.842634in}{1.467652in}}%
\pgfpathcurveto{\pgfqpoint{0.836810in}{1.461828in}}{\pgfqpoint{0.833538in}{1.453928in}}{\pgfqpoint{0.833538in}{1.445692in}}%
\pgfpathcurveto{\pgfqpoint{0.833538in}{1.437455in}}{\pgfqpoint{0.836810in}{1.429555in}}{\pgfqpoint{0.842634in}{1.423731in}}%
\pgfpathcurveto{\pgfqpoint{0.848458in}{1.417907in}}{\pgfqpoint{0.856358in}{1.414635in}}{\pgfqpoint{0.864595in}{1.414635in}}%
\pgfpathclose%
\pgfusepath{stroke,fill}%
\end{pgfscope}%
\begin{pgfscope}%
\pgfpathrectangle{\pgfqpoint{0.100000in}{0.212622in}}{\pgfqpoint{3.696000in}{3.696000in}}%
\pgfusepath{clip}%
\pgfsetbuttcap%
\pgfsetroundjoin%
\definecolor{currentfill}{rgb}{0.121569,0.466667,0.705882}%
\pgfsetfillcolor{currentfill}%
\pgfsetfillopacity{0.657338}%
\pgfsetlinewidth{1.003750pt}%
\definecolor{currentstroke}{rgb}{0.121569,0.466667,0.705882}%
\pgfsetstrokecolor{currentstroke}%
\pgfsetstrokeopacity{0.657338}%
\pgfsetdash{}{0pt}%
\pgfpathmoveto{\pgfqpoint{0.864592in}{1.414636in}}%
\pgfpathcurveto{\pgfqpoint{0.872828in}{1.414636in}}{\pgfqpoint{0.880728in}{1.417908in}}{\pgfqpoint{0.886552in}{1.423732in}}%
\pgfpathcurveto{\pgfqpoint{0.892376in}{1.429556in}}{\pgfqpoint{0.895648in}{1.437456in}}{\pgfqpoint{0.895648in}{1.445693in}}%
\pgfpathcurveto{\pgfqpoint{0.895648in}{1.453929in}}{\pgfqpoint{0.892376in}{1.461829in}}{\pgfqpoint{0.886552in}{1.467653in}}%
\pgfpathcurveto{\pgfqpoint{0.880728in}{1.473477in}}{\pgfqpoint{0.872828in}{1.476749in}}{\pgfqpoint{0.864592in}{1.476749in}}%
\pgfpathcurveto{\pgfqpoint{0.856355in}{1.476749in}}{\pgfqpoint{0.848455in}{1.473477in}}{\pgfqpoint{0.842632in}{1.467653in}}%
\pgfpathcurveto{\pgfqpoint{0.836808in}{1.461829in}}{\pgfqpoint{0.833535in}{1.453929in}}{\pgfqpoint{0.833535in}{1.445693in}}%
\pgfpathcurveto{\pgfqpoint{0.833535in}{1.437456in}}{\pgfqpoint{0.836808in}{1.429556in}}{\pgfqpoint{0.842632in}{1.423732in}}%
\pgfpathcurveto{\pgfqpoint{0.848455in}{1.417908in}}{\pgfqpoint{0.856355in}{1.414636in}}{\pgfqpoint{0.864592in}{1.414636in}}%
\pgfpathclose%
\pgfusepath{stroke,fill}%
\end{pgfscope}%
\begin{pgfscope}%
\pgfpathrectangle{\pgfqpoint{0.100000in}{0.212622in}}{\pgfqpoint{3.696000in}{3.696000in}}%
\pgfusepath{clip}%
\pgfsetbuttcap%
\pgfsetroundjoin%
\definecolor{currentfill}{rgb}{0.121569,0.466667,0.705882}%
\pgfsetfillcolor{currentfill}%
\pgfsetfillopacity{0.657339}%
\pgfsetlinewidth{1.003750pt}%
\definecolor{currentstroke}{rgb}{0.121569,0.466667,0.705882}%
\pgfsetstrokecolor{currentstroke}%
\pgfsetstrokeopacity{0.657339}%
\pgfsetdash{}{0pt}%
\pgfpathmoveto{\pgfqpoint{0.864590in}{1.414637in}}%
\pgfpathcurveto{\pgfqpoint{0.872826in}{1.414637in}}{\pgfqpoint{0.880727in}{1.417909in}}{\pgfqpoint{0.886550in}{1.423733in}}%
\pgfpathcurveto{\pgfqpoint{0.892374in}{1.429557in}}{\pgfqpoint{0.895647in}{1.437457in}}{\pgfqpoint{0.895647in}{1.445693in}}%
\pgfpathcurveto{\pgfqpoint{0.895647in}{1.453929in}}{\pgfqpoint{0.892374in}{1.461830in}}{\pgfqpoint{0.886550in}{1.467653in}}%
\pgfpathcurveto{\pgfqpoint{0.880727in}{1.473477in}}{\pgfqpoint{0.872826in}{1.476750in}}{\pgfqpoint{0.864590in}{1.476750in}}%
\pgfpathcurveto{\pgfqpoint{0.856354in}{1.476750in}}{\pgfqpoint{0.848454in}{1.473477in}}{\pgfqpoint{0.842630in}{1.467653in}}%
\pgfpathcurveto{\pgfqpoint{0.836806in}{1.461830in}}{\pgfqpoint{0.833534in}{1.453929in}}{\pgfqpoint{0.833534in}{1.445693in}}%
\pgfpathcurveto{\pgfqpoint{0.833534in}{1.437457in}}{\pgfqpoint{0.836806in}{1.429557in}}{\pgfqpoint{0.842630in}{1.423733in}}%
\pgfpathcurveto{\pgfqpoint{0.848454in}{1.417909in}}{\pgfqpoint{0.856354in}{1.414637in}}{\pgfqpoint{0.864590in}{1.414637in}}%
\pgfpathclose%
\pgfusepath{stroke,fill}%
\end{pgfscope}%
\begin{pgfscope}%
\pgfpathrectangle{\pgfqpoint{0.100000in}{0.212622in}}{\pgfqpoint{3.696000in}{3.696000in}}%
\pgfusepath{clip}%
\pgfsetbuttcap%
\pgfsetroundjoin%
\definecolor{currentfill}{rgb}{0.121569,0.466667,0.705882}%
\pgfsetfillcolor{currentfill}%
\pgfsetfillopacity{0.657339}%
\pgfsetlinewidth{1.003750pt}%
\definecolor{currentstroke}{rgb}{0.121569,0.466667,0.705882}%
\pgfsetstrokecolor{currentstroke}%
\pgfsetstrokeopacity{0.657339}%
\pgfsetdash{}{0pt}%
\pgfpathmoveto{\pgfqpoint{0.864589in}{1.414637in}}%
\pgfpathcurveto{\pgfqpoint{0.872826in}{1.414637in}}{\pgfqpoint{0.880726in}{1.417909in}}{\pgfqpoint{0.886550in}{1.423733in}}%
\pgfpathcurveto{\pgfqpoint{0.892373in}{1.429557in}}{\pgfqpoint{0.895646in}{1.437457in}}{\pgfqpoint{0.895646in}{1.445694in}}%
\pgfpathcurveto{\pgfqpoint{0.895646in}{1.453930in}}{\pgfqpoint{0.892373in}{1.461830in}}{\pgfqpoint{0.886550in}{1.467654in}}%
\pgfpathcurveto{\pgfqpoint{0.880726in}{1.473478in}}{\pgfqpoint{0.872826in}{1.476750in}}{\pgfqpoint{0.864589in}{1.476750in}}%
\pgfpathcurveto{\pgfqpoint{0.856353in}{1.476750in}}{\pgfqpoint{0.848453in}{1.473478in}}{\pgfqpoint{0.842629in}{1.467654in}}%
\pgfpathcurveto{\pgfqpoint{0.836805in}{1.461830in}}{\pgfqpoint{0.833533in}{1.453930in}}{\pgfqpoint{0.833533in}{1.445694in}}%
\pgfpathcurveto{\pgfqpoint{0.833533in}{1.437457in}}{\pgfqpoint{0.836805in}{1.429557in}}{\pgfqpoint{0.842629in}{1.423733in}}%
\pgfpathcurveto{\pgfqpoint{0.848453in}{1.417909in}}{\pgfqpoint{0.856353in}{1.414637in}}{\pgfqpoint{0.864589in}{1.414637in}}%
\pgfpathclose%
\pgfusepath{stroke,fill}%
\end{pgfscope}%
\begin{pgfscope}%
\pgfpathrectangle{\pgfqpoint{0.100000in}{0.212622in}}{\pgfqpoint{3.696000in}{3.696000in}}%
\pgfusepath{clip}%
\pgfsetbuttcap%
\pgfsetroundjoin%
\definecolor{currentfill}{rgb}{0.121569,0.466667,0.705882}%
\pgfsetfillcolor{currentfill}%
\pgfsetfillopacity{0.657340}%
\pgfsetlinewidth{1.003750pt}%
\definecolor{currentstroke}{rgb}{0.121569,0.466667,0.705882}%
\pgfsetstrokecolor{currentstroke}%
\pgfsetstrokeopacity{0.657340}%
\pgfsetdash{}{0pt}%
\pgfpathmoveto{\pgfqpoint{0.864589in}{1.414637in}}%
\pgfpathcurveto{\pgfqpoint{0.872825in}{1.414637in}}{\pgfqpoint{0.880725in}{1.417909in}}{\pgfqpoint{0.886549in}{1.423733in}}%
\pgfpathcurveto{\pgfqpoint{0.892373in}{1.429557in}}{\pgfqpoint{0.895645in}{1.437457in}}{\pgfqpoint{0.895645in}{1.445694in}}%
\pgfpathcurveto{\pgfqpoint{0.895645in}{1.453930in}}{\pgfqpoint{0.892373in}{1.461830in}}{\pgfqpoint{0.886549in}{1.467654in}}%
\pgfpathcurveto{\pgfqpoint{0.880725in}{1.473478in}}{\pgfqpoint{0.872825in}{1.476750in}}{\pgfqpoint{0.864589in}{1.476750in}}%
\pgfpathcurveto{\pgfqpoint{0.856353in}{1.476750in}}{\pgfqpoint{0.848452in}{1.473478in}}{\pgfqpoint{0.842629in}{1.467654in}}%
\pgfpathcurveto{\pgfqpoint{0.836805in}{1.461830in}}{\pgfqpoint{0.833532in}{1.453930in}}{\pgfqpoint{0.833532in}{1.445694in}}%
\pgfpathcurveto{\pgfqpoint{0.833532in}{1.437457in}}{\pgfqpoint{0.836805in}{1.429557in}}{\pgfqpoint{0.842629in}{1.423733in}}%
\pgfpathcurveto{\pgfqpoint{0.848452in}{1.417909in}}{\pgfqpoint{0.856353in}{1.414637in}}{\pgfqpoint{0.864589in}{1.414637in}}%
\pgfpathclose%
\pgfusepath{stroke,fill}%
\end{pgfscope}%
\begin{pgfscope}%
\pgfpathrectangle{\pgfqpoint{0.100000in}{0.212622in}}{\pgfqpoint{3.696000in}{3.696000in}}%
\pgfusepath{clip}%
\pgfsetbuttcap%
\pgfsetroundjoin%
\definecolor{currentfill}{rgb}{0.121569,0.466667,0.705882}%
\pgfsetfillcolor{currentfill}%
\pgfsetfillopacity{0.657340}%
\pgfsetlinewidth{1.003750pt}%
\definecolor{currentstroke}{rgb}{0.121569,0.466667,0.705882}%
\pgfsetstrokecolor{currentstroke}%
\pgfsetstrokeopacity{0.657340}%
\pgfsetdash{}{0pt}%
\pgfpathmoveto{\pgfqpoint{0.864589in}{1.414637in}}%
\pgfpathcurveto{\pgfqpoint{0.872825in}{1.414637in}}{\pgfqpoint{0.880725in}{1.417909in}}{\pgfqpoint{0.886549in}{1.423733in}}%
\pgfpathcurveto{\pgfqpoint{0.892373in}{1.429557in}}{\pgfqpoint{0.895645in}{1.437457in}}{\pgfqpoint{0.895645in}{1.445694in}}%
\pgfpathcurveto{\pgfqpoint{0.895645in}{1.453930in}}{\pgfqpoint{0.892373in}{1.461830in}}{\pgfqpoint{0.886549in}{1.467654in}}%
\pgfpathcurveto{\pgfqpoint{0.880725in}{1.473478in}}{\pgfqpoint{0.872825in}{1.476750in}}{\pgfqpoint{0.864589in}{1.476750in}}%
\pgfpathcurveto{\pgfqpoint{0.856352in}{1.476750in}}{\pgfqpoint{0.848452in}{1.473478in}}{\pgfqpoint{0.842628in}{1.467654in}}%
\pgfpathcurveto{\pgfqpoint{0.836804in}{1.461830in}}{\pgfqpoint{0.833532in}{1.453930in}}{\pgfqpoint{0.833532in}{1.445694in}}%
\pgfpathcurveto{\pgfqpoint{0.833532in}{1.437457in}}{\pgfqpoint{0.836804in}{1.429557in}}{\pgfqpoint{0.842628in}{1.423733in}}%
\pgfpathcurveto{\pgfqpoint{0.848452in}{1.417909in}}{\pgfqpoint{0.856352in}{1.414637in}}{\pgfqpoint{0.864589in}{1.414637in}}%
\pgfpathclose%
\pgfusepath{stroke,fill}%
\end{pgfscope}%
\begin{pgfscope}%
\pgfpathrectangle{\pgfqpoint{0.100000in}{0.212622in}}{\pgfqpoint{3.696000in}{3.696000in}}%
\pgfusepath{clip}%
\pgfsetbuttcap%
\pgfsetroundjoin%
\definecolor{currentfill}{rgb}{0.121569,0.466667,0.705882}%
\pgfsetfillcolor{currentfill}%
\pgfsetfillopacity{0.657340}%
\pgfsetlinewidth{1.003750pt}%
\definecolor{currentstroke}{rgb}{0.121569,0.466667,0.705882}%
\pgfsetstrokecolor{currentstroke}%
\pgfsetstrokeopacity{0.657340}%
\pgfsetdash{}{0pt}%
\pgfpathmoveto{\pgfqpoint{0.864588in}{1.414637in}}%
\pgfpathcurveto{\pgfqpoint{0.872825in}{1.414637in}}{\pgfqpoint{0.880725in}{1.417909in}}{\pgfqpoint{0.886549in}{1.423733in}}%
\pgfpathcurveto{\pgfqpoint{0.892373in}{1.429557in}}{\pgfqpoint{0.895645in}{1.437457in}}{\pgfqpoint{0.895645in}{1.445694in}}%
\pgfpathcurveto{\pgfqpoint{0.895645in}{1.453930in}}{\pgfqpoint{0.892373in}{1.461830in}}{\pgfqpoint{0.886549in}{1.467654in}}%
\pgfpathcurveto{\pgfqpoint{0.880725in}{1.473478in}}{\pgfqpoint{0.872825in}{1.476750in}}{\pgfqpoint{0.864588in}{1.476750in}}%
\pgfpathcurveto{\pgfqpoint{0.856352in}{1.476750in}}{\pgfqpoint{0.848452in}{1.473478in}}{\pgfqpoint{0.842628in}{1.467654in}}%
\pgfpathcurveto{\pgfqpoint{0.836804in}{1.461830in}}{\pgfqpoint{0.833532in}{1.453930in}}{\pgfqpoint{0.833532in}{1.445694in}}%
\pgfpathcurveto{\pgfqpoint{0.833532in}{1.437457in}}{\pgfqpoint{0.836804in}{1.429557in}}{\pgfqpoint{0.842628in}{1.423733in}}%
\pgfpathcurveto{\pgfqpoint{0.848452in}{1.417909in}}{\pgfqpoint{0.856352in}{1.414637in}}{\pgfqpoint{0.864588in}{1.414637in}}%
\pgfpathclose%
\pgfusepath{stroke,fill}%
\end{pgfscope}%
\begin{pgfscope}%
\pgfpathrectangle{\pgfqpoint{0.100000in}{0.212622in}}{\pgfqpoint{3.696000in}{3.696000in}}%
\pgfusepath{clip}%
\pgfsetbuttcap%
\pgfsetroundjoin%
\definecolor{currentfill}{rgb}{0.121569,0.466667,0.705882}%
\pgfsetfillcolor{currentfill}%
\pgfsetfillopacity{0.657340}%
\pgfsetlinewidth{1.003750pt}%
\definecolor{currentstroke}{rgb}{0.121569,0.466667,0.705882}%
\pgfsetstrokecolor{currentstroke}%
\pgfsetstrokeopacity{0.657340}%
\pgfsetdash{}{0pt}%
\pgfpathmoveto{\pgfqpoint{0.864588in}{1.414637in}}%
\pgfpathcurveto{\pgfqpoint{0.872825in}{1.414637in}}{\pgfqpoint{0.880725in}{1.417909in}}{\pgfqpoint{0.886549in}{1.423733in}}%
\pgfpathcurveto{\pgfqpoint{0.892373in}{1.429557in}}{\pgfqpoint{0.895645in}{1.437457in}}{\pgfqpoint{0.895645in}{1.445694in}}%
\pgfpathcurveto{\pgfqpoint{0.895645in}{1.453930in}}{\pgfqpoint{0.892373in}{1.461830in}}{\pgfqpoint{0.886549in}{1.467654in}}%
\pgfpathcurveto{\pgfqpoint{0.880725in}{1.473478in}}{\pgfqpoint{0.872825in}{1.476750in}}{\pgfqpoint{0.864588in}{1.476750in}}%
\pgfpathcurveto{\pgfqpoint{0.856352in}{1.476750in}}{\pgfqpoint{0.848452in}{1.473478in}}{\pgfqpoint{0.842628in}{1.467654in}}%
\pgfpathcurveto{\pgfqpoint{0.836804in}{1.461830in}}{\pgfqpoint{0.833532in}{1.453930in}}{\pgfqpoint{0.833532in}{1.445694in}}%
\pgfpathcurveto{\pgfqpoint{0.833532in}{1.437457in}}{\pgfqpoint{0.836804in}{1.429557in}}{\pgfqpoint{0.842628in}{1.423733in}}%
\pgfpathcurveto{\pgfqpoint{0.848452in}{1.417909in}}{\pgfqpoint{0.856352in}{1.414637in}}{\pgfqpoint{0.864588in}{1.414637in}}%
\pgfpathclose%
\pgfusepath{stroke,fill}%
\end{pgfscope}%
\begin{pgfscope}%
\pgfpathrectangle{\pgfqpoint{0.100000in}{0.212622in}}{\pgfqpoint{3.696000in}{3.696000in}}%
\pgfusepath{clip}%
\pgfsetbuttcap%
\pgfsetroundjoin%
\definecolor{currentfill}{rgb}{0.121569,0.466667,0.705882}%
\pgfsetfillcolor{currentfill}%
\pgfsetfillopacity{0.657340}%
\pgfsetlinewidth{1.003750pt}%
\definecolor{currentstroke}{rgb}{0.121569,0.466667,0.705882}%
\pgfsetstrokecolor{currentstroke}%
\pgfsetstrokeopacity{0.657340}%
\pgfsetdash{}{0pt}%
\pgfpathmoveto{\pgfqpoint{0.864588in}{1.414637in}}%
\pgfpathcurveto{\pgfqpoint{0.872825in}{1.414637in}}{\pgfqpoint{0.880725in}{1.417909in}}{\pgfqpoint{0.886549in}{1.423733in}}%
\pgfpathcurveto{\pgfqpoint{0.892372in}{1.429557in}}{\pgfqpoint{0.895645in}{1.437457in}}{\pgfqpoint{0.895645in}{1.445694in}}%
\pgfpathcurveto{\pgfqpoint{0.895645in}{1.453930in}}{\pgfqpoint{0.892372in}{1.461830in}}{\pgfqpoint{0.886549in}{1.467654in}}%
\pgfpathcurveto{\pgfqpoint{0.880725in}{1.473478in}}{\pgfqpoint{0.872825in}{1.476750in}}{\pgfqpoint{0.864588in}{1.476750in}}%
\pgfpathcurveto{\pgfqpoint{0.856352in}{1.476750in}}{\pgfqpoint{0.848452in}{1.473478in}}{\pgfqpoint{0.842628in}{1.467654in}}%
\pgfpathcurveto{\pgfqpoint{0.836804in}{1.461830in}}{\pgfqpoint{0.833532in}{1.453930in}}{\pgfqpoint{0.833532in}{1.445694in}}%
\pgfpathcurveto{\pgfqpoint{0.833532in}{1.437457in}}{\pgfqpoint{0.836804in}{1.429557in}}{\pgfqpoint{0.842628in}{1.423733in}}%
\pgfpathcurveto{\pgfqpoint{0.848452in}{1.417909in}}{\pgfqpoint{0.856352in}{1.414637in}}{\pgfqpoint{0.864588in}{1.414637in}}%
\pgfpathclose%
\pgfusepath{stroke,fill}%
\end{pgfscope}%
\begin{pgfscope}%
\pgfpathrectangle{\pgfqpoint{0.100000in}{0.212622in}}{\pgfqpoint{3.696000in}{3.696000in}}%
\pgfusepath{clip}%
\pgfsetbuttcap%
\pgfsetroundjoin%
\definecolor{currentfill}{rgb}{0.121569,0.466667,0.705882}%
\pgfsetfillcolor{currentfill}%
\pgfsetfillopacity{0.657340}%
\pgfsetlinewidth{1.003750pt}%
\definecolor{currentstroke}{rgb}{0.121569,0.466667,0.705882}%
\pgfsetstrokecolor{currentstroke}%
\pgfsetstrokeopacity{0.657340}%
\pgfsetdash{}{0pt}%
\pgfpathmoveto{\pgfqpoint{0.864588in}{1.414637in}}%
\pgfpathcurveto{\pgfqpoint{0.872825in}{1.414637in}}{\pgfqpoint{0.880725in}{1.417909in}}{\pgfqpoint{0.886549in}{1.423733in}}%
\pgfpathcurveto{\pgfqpoint{0.892372in}{1.429557in}}{\pgfqpoint{0.895645in}{1.437457in}}{\pgfqpoint{0.895645in}{1.445694in}}%
\pgfpathcurveto{\pgfqpoint{0.895645in}{1.453930in}}{\pgfqpoint{0.892372in}{1.461830in}}{\pgfqpoint{0.886549in}{1.467654in}}%
\pgfpathcurveto{\pgfqpoint{0.880725in}{1.473478in}}{\pgfqpoint{0.872825in}{1.476750in}}{\pgfqpoint{0.864588in}{1.476750in}}%
\pgfpathcurveto{\pgfqpoint{0.856352in}{1.476750in}}{\pgfqpoint{0.848452in}{1.473478in}}{\pgfqpoint{0.842628in}{1.467654in}}%
\pgfpathcurveto{\pgfqpoint{0.836804in}{1.461830in}}{\pgfqpoint{0.833532in}{1.453930in}}{\pgfqpoint{0.833532in}{1.445694in}}%
\pgfpathcurveto{\pgfqpoint{0.833532in}{1.437457in}}{\pgfqpoint{0.836804in}{1.429557in}}{\pgfqpoint{0.842628in}{1.423733in}}%
\pgfpathcurveto{\pgfqpoint{0.848452in}{1.417909in}}{\pgfqpoint{0.856352in}{1.414637in}}{\pgfqpoint{0.864588in}{1.414637in}}%
\pgfpathclose%
\pgfusepath{stroke,fill}%
\end{pgfscope}%
\begin{pgfscope}%
\pgfpathrectangle{\pgfqpoint{0.100000in}{0.212622in}}{\pgfqpoint{3.696000in}{3.696000in}}%
\pgfusepath{clip}%
\pgfsetbuttcap%
\pgfsetroundjoin%
\definecolor{currentfill}{rgb}{0.121569,0.466667,0.705882}%
\pgfsetfillcolor{currentfill}%
\pgfsetfillopacity{0.657340}%
\pgfsetlinewidth{1.003750pt}%
\definecolor{currentstroke}{rgb}{0.121569,0.466667,0.705882}%
\pgfsetstrokecolor{currentstroke}%
\pgfsetstrokeopacity{0.657340}%
\pgfsetdash{}{0pt}%
\pgfpathmoveto{\pgfqpoint{0.864588in}{1.414637in}}%
\pgfpathcurveto{\pgfqpoint{0.872825in}{1.414637in}}{\pgfqpoint{0.880725in}{1.417909in}}{\pgfqpoint{0.886549in}{1.423733in}}%
\pgfpathcurveto{\pgfqpoint{0.892372in}{1.429557in}}{\pgfqpoint{0.895645in}{1.437457in}}{\pgfqpoint{0.895645in}{1.445694in}}%
\pgfpathcurveto{\pgfqpoint{0.895645in}{1.453930in}}{\pgfqpoint{0.892372in}{1.461830in}}{\pgfqpoint{0.886549in}{1.467654in}}%
\pgfpathcurveto{\pgfqpoint{0.880725in}{1.473478in}}{\pgfqpoint{0.872825in}{1.476750in}}{\pgfqpoint{0.864588in}{1.476750in}}%
\pgfpathcurveto{\pgfqpoint{0.856352in}{1.476750in}}{\pgfqpoint{0.848452in}{1.473478in}}{\pgfqpoint{0.842628in}{1.467654in}}%
\pgfpathcurveto{\pgfqpoint{0.836804in}{1.461830in}}{\pgfqpoint{0.833532in}{1.453930in}}{\pgfqpoint{0.833532in}{1.445694in}}%
\pgfpathcurveto{\pgfqpoint{0.833532in}{1.437457in}}{\pgfqpoint{0.836804in}{1.429557in}}{\pgfqpoint{0.842628in}{1.423733in}}%
\pgfpathcurveto{\pgfqpoint{0.848452in}{1.417909in}}{\pgfqpoint{0.856352in}{1.414637in}}{\pgfqpoint{0.864588in}{1.414637in}}%
\pgfpathclose%
\pgfusepath{stroke,fill}%
\end{pgfscope}%
\begin{pgfscope}%
\pgfpathrectangle{\pgfqpoint{0.100000in}{0.212622in}}{\pgfqpoint{3.696000in}{3.696000in}}%
\pgfusepath{clip}%
\pgfsetbuttcap%
\pgfsetroundjoin%
\definecolor{currentfill}{rgb}{0.121569,0.466667,0.705882}%
\pgfsetfillcolor{currentfill}%
\pgfsetfillopacity{0.657340}%
\pgfsetlinewidth{1.003750pt}%
\definecolor{currentstroke}{rgb}{0.121569,0.466667,0.705882}%
\pgfsetstrokecolor{currentstroke}%
\pgfsetstrokeopacity{0.657340}%
\pgfsetdash{}{0pt}%
\pgfpathmoveto{\pgfqpoint{0.864588in}{1.414637in}}%
\pgfpathcurveto{\pgfqpoint{0.872825in}{1.414637in}}{\pgfqpoint{0.880725in}{1.417909in}}{\pgfqpoint{0.886548in}{1.423733in}}%
\pgfpathcurveto{\pgfqpoint{0.892372in}{1.429557in}}{\pgfqpoint{0.895645in}{1.437457in}}{\pgfqpoint{0.895645in}{1.445694in}}%
\pgfpathcurveto{\pgfqpoint{0.895645in}{1.453930in}}{\pgfqpoint{0.892372in}{1.461830in}}{\pgfqpoint{0.886548in}{1.467654in}}%
\pgfpathcurveto{\pgfqpoint{0.880725in}{1.473478in}}{\pgfqpoint{0.872825in}{1.476750in}}{\pgfqpoint{0.864588in}{1.476750in}}%
\pgfpathcurveto{\pgfqpoint{0.856352in}{1.476750in}}{\pgfqpoint{0.848452in}{1.473478in}}{\pgfqpoint{0.842628in}{1.467654in}}%
\pgfpathcurveto{\pgfqpoint{0.836804in}{1.461830in}}{\pgfqpoint{0.833532in}{1.453930in}}{\pgfqpoint{0.833532in}{1.445694in}}%
\pgfpathcurveto{\pgfqpoint{0.833532in}{1.437457in}}{\pgfqpoint{0.836804in}{1.429557in}}{\pgfqpoint{0.842628in}{1.423733in}}%
\pgfpathcurveto{\pgfqpoint{0.848452in}{1.417909in}}{\pgfqpoint{0.856352in}{1.414637in}}{\pgfqpoint{0.864588in}{1.414637in}}%
\pgfpathclose%
\pgfusepath{stroke,fill}%
\end{pgfscope}%
\begin{pgfscope}%
\pgfpathrectangle{\pgfqpoint{0.100000in}{0.212622in}}{\pgfqpoint{3.696000in}{3.696000in}}%
\pgfusepath{clip}%
\pgfsetbuttcap%
\pgfsetroundjoin%
\definecolor{currentfill}{rgb}{0.121569,0.466667,0.705882}%
\pgfsetfillcolor{currentfill}%
\pgfsetfillopacity{0.657340}%
\pgfsetlinewidth{1.003750pt}%
\definecolor{currentstroke}{rgb}{0.121569,0.466667,0.705882}%
\pgfsetstrokecolor{currentstroke}%
\pgfsetstrokeopacity{0.657340}%
\pgfsetdash{}{0pt}%
\pgfpathmoveto{\pgfqpoint{0.864588in}{1.414637in}}%
\pgfpathcurveto{\pgfqpoint{0.872825in}{1.414637in}}{\pgfqpoint{0.880725in}{1.417909in}}{\pgfqpoint{0.886548in}{1.423733in}}%
\pgfpathcurveto{\pgfqpoint{0.892372in}{1.429557in}}{\pgfqpoint{0.895645in}{1.437457in}}{\pgfqpoint{0.895645in}{1.445694in}}%
\pgfpathcurveto{\pgfqpoint{0.895645in}{1.453930in}}{\pgfqpoint{0.892372in}{1.461830in}}{\pgfqpoint{0.886548in}{1.467654in}}%
\pgfpathcurveto{\pgfqpoint{0.880725in}{1.473478in}}{\pgfqpoint{0.872825in}{1.476750in}}{\pgfqpoint{0.864588in}{1.476750in}}%
\pgfpathcurveto{\pgfqpoint{0.856352in}{1.476750in}}{\pgfqpoint{0.848452in}{1.473478in}}{\pgfqpoint{0.842628in}{1.467654in}}%
\pgfpathcurveto{\pgfqpoint{0.836804in}{1.461830in}}{\pgfqpoint{0.833532in}{1.453930in}}{\pgfqpoint{0.833532in}{1.445694in}}%
\pgfpathcurveto{\pgfqpoint{0.833532in}{1.437457in}}{\pgfqpoint{0.836804in}{1.429557in}}{\pgfqpoint{0.842628in}{1.423733in}}%
\pgfpathcurveto{\pgfqpoint{0.848452in}{1.417909in}}{\pgfqpoint{0.856352in}{1.414637in}}{\pgfqpoint{0.864588in}{1.414637in}}%
\pgfpathclose%
\pgfusepath{stroke,fill}%
\end{pgfscope}%
\begin{pgfscope}%
\pgfpathrectangle{\pgfqpoint{0.100000in}{0.212622in}}{\pgfqpoint{3.696000in}{3.696000in}}%
\pgfusepath{clip}%
\pgfsetbuttcap%
\pgfsetroundjoin%
\definecolor{currentfill}{rgb}{0.121569,0.466667,0.705882}%
\pgfsetfillcolor{currentfill}%
\pgfsetfillopacity{0.657340}%
\pgfsetlinewidth{1.003750pt}%
\definecolor{currentstroke}{rgb}{0.121569,0.466667,0.705882}%
\pgfsetstrokecolor{currentstroke}%
\pgfsetstrokeopacity{0.657340}%
\pgfsetdash{}{0pt}%
\pgfpathmoveto{\pgfqpoint{0.864588in}{1.414637in}}%
\pgfpathcurveto{\pgfqpoint{0.872825in}{1.414637in}}{\pgfqpoint{0.880725in}{1.417909in}}{\pgfqpoint{0.886548in}{1.423733in}}%
\pgfpathcurveto{\pgfqpoint{0.892372in}{1.429557in}}{\pgfqpoint{0.895645in}{1.437457in}}{\pgfqpoint{0.895645in}{1.445694in}}%
\pgfpathcurveto{\pgfqpoint{0.895645in}{1.453930in}}{\pgfqpoint{0.892372in}{1.461830in}}{\pgfqpoint{0.886548in}{1.467654in}}%
\pgfpathcurveto{\pgfqpoint{0.880725in}{1.473478in}}{\pgfqpoint{0.872825in}{1.476750in}}{\pgfqpoint{0.864588in}{1.476750in}}%
\pgfpathcurveto{\pgfqpoint{0.856352in}{1.476750in}}{\pgfqpoint{0.848452in}{1.473478in}}{\pgfqpoint{0.842628in}{1.467654in}}%
\pgfpathcurveto{\pgfqpoint{0.836804in}{1.461830in}}{\pgfqpoint{0.833532in}{1.453930in}}{\pgfqpoint{0.833532in}{1.445694in}}%
\pgfpathcurveto{\pgfqpoint{0.833532in}{1.437457in}}{\pgfqpoint{0.836804in}{1.429557in}}{\pgfqpoint{0.842628in}{1.423733in}}%
\pgfpathcurveto{\pgfqpoint{0.848452in}{1.417909in}}{\pgfqpoint{0.856352in}{1.414637in}}{\pgfqpoint{0.864588in}{1.414637in}}%
\pgfpathclose%
\pgfusepath{stroke,fill}%
\end{pgfscope}%
\begin{pgfscope}%
\pgfpathrectangle{\pgfqpoint{0.100000in}{0.212622in}}{\pgfqpoint{3.696000in}{3.696000in}}%
\pgfusepath{clip}%
\pgfsetbuttcap%
\pgfsetroundjoin%
\definecolor{currentfill}{rgb}{0.121569,0.466667,0.705882}%
\pgfsetfillcolor{currentfill}%
\pgfsetfillopacity{0.657340}%
\pgfsetlinewidth{1.003750pt}%
\definecolor{currentstroke}{rgb}{0.121569,0.466667,0.705882}%
\pgfsetstrokecolor{currentstroke}%
\pgfsetstrokeopacity{0.657340}%
\pgfsetdash{}{0pt}%
\pgfpathmoveto{\pgfqpoint{0.864588in}{1.414637in}}%
\pgfpathcurveto{\pgfqpoint{0.872825in}{1.414637in}}{\pgfqpoint{0.880725in}{1.417909in}}{\pgfqpoint{0.886548in}{1.423733in}}%
\pgfpathcurveto{\pgfqpoint{0.892372in}{1.429557in}}{\pgfqpoint{0.895645in}{1.437457in}}{\pgfqpoint{0.895645in}{1.445694in}}%
\pgfpathcurveto{\pgfqpoint{0.895645in}{1.453930in}}{\pgfqpoint{0.892372in}{1.461830in}}{\pgfqpoint{0.886548in}{1.467654in}}%
\pgfpathcurveto{\pgfqpoint{0.880725in}{1.473478in}}{\pgfqpoint{0.872825in}{1.476750in}}{\pgfqpoint{0.864588in}{1.476750in}}%
\pgfpathcurveto{\pgfqpoint{0.856352in}{1.476750in}}{\pgfqpoint{0.848452in}{1.473478in}}{\pgfqpoint{0.842628in}{1.467654in}}%
\pgfpathcurveto{\pgfqpoint{0.836804in}{1.461830in}}{\pgfqpoint{0.833532in}{1.453930in}}{\pgfqpoint{0.833532in}{1.445694in}}%
\pgfpathcurveto{\pgfqpoint{0.833532in}{1.437457in}}{\pgfqpoint{0.836804in}{1.429557in}}{\pgfqpoint{0.842628in}{1.423733in}}%
\pgfpathcurveto{\pgfqpoint{0.848452in}{1.417909in}}{\pgfqpoint{0.856352in}{1.414637in}}{\pgfqpoint{0.864588in}{1.414637in}}%
\pgfpathclose%
\pgfusepath{stroke,fill}%
\end{pgfscope}%
\begin{pgfscope}%
\pgfpathrectangle{\pgfqpoint{0.100000in}{0.212622in}}{\pgfqpoint{3.696000in}{3.696000in}}%
\pgfusepath{clip}%
\pgfsetbuttcap%
\pgfsetroundjoin%
\definecolor{currentfill}{rgb}{0.121569,0.466667,0.705882}%
\pgfsetfillcolor{currentfill}%
\pgfsetfillopacity{0.657340}%
\pgfsetlinewidth{1.003750pt}%
\definecolor{currentstroke}{rgb}{0.121569,0.466667,0.705882}%
\pgfsetstrokecolor{currentstroke}%
\pgfsetstrokeopacity{0.657340}%
\pgfsetdash{}{0pt}%
\pgfpathmoveto{\pgfqpoint{0.864588in}{1.414637in}}%
\pgfpathcurveto{\pgfqpoint{0.872825in}{1.414637in}}{\pgfqpoint{0.880725in}{1.417909in}}{\pgfqpoint{0.886548in}{1.423733in}}%
\pgfpathcurveto{\pgfqpoint{0.892372in}{1.429557in}}{\pgfqpoint{0.895645in}{1.437457in}}{\pgfqpoint{0.895645in}{1.445694in}}%
\pgfpathcurveto{\pgfqpoint{0.895645in}{1.453930in}}{\pgfqpoint{0.892372in}{1.461830in}}{\pgfqpoint{0.886548in}{1.467654in}}%
\pgfpathcurveto{\pgfqpoint{0.880725in}{1.473478in}}{\pgfqpoint{0.872825in}{1.476750in}}{\pgfqpoint{0.864588in}{1.476750in}}%
\pgfpathcurveto{\pgfqpoint{0.856352in}{1.476750in}}{\pgfqpoint{0.848452in}{1.473478in}}{\pgfqpoint{0.842628in}{1.467654in}}%
\pgfpathcurveto{\pgfqpoint{0.836804in}{1.461830in}}{\pgfqpoint{0.833532in}{1.453930in}}{\pgfqpoint{0.833532in}{1.445694in}}%
\pgfpathcurveto{\pgfqpoint{0.833532in}{1.437457in}}{\pgfqpoint{0.836804in}{1.429557in}}{\pgfqpoint{0.842628in}{1.423733in}}%
\pgfpathcurveto{\pgfqpoint{0.848452in}{1.417909in}}{\pgfqpoint{0.856352in}{1.414637in}}{\pgfqpoint{0.864588in}{1.414637in}}%
\pgfpathclose%
\pgfusepath{stroke,fill}%
\end{pgfscope}%
\begin{pgfscope}%
\pgfpathrectangle{\pgfqpoint{0.100000in}{0.212622in}}{\pgfqpoint{3.696000in}{3.696000in}}%
\pgfusepath{clip}%
\pgfsetbuttcap%
\pgfsetroundjoin%
\definecolor{currentfill}{rgb}{0.121569,0.466667,0.705882}%
\pgfsetfillcolor{currentfill}%
\pgfsetfillopacity{0.657340}%
\pgfsetlinewidth{1.003750pt}%
\definecolor{currentstroke}{rgb}{0.121569,0.466667,0.705882}%
\pgfsetstrokecolor{currentstroke}%
\pgfsetstrokeopacity{0.657340}%
\pgfsetdash{}{0pt}%
\pgfpathmoveto{\pgfqpoint{0.864588in}{1.414637in}}%
\pgfpathcurveto{\pgfqpoint{0.872825in}{1.414637in}}{\pgfqpoint{0.880725in}{1.417909in}}{\pgfqpoint{0.886548in}{1.423733in}}%
\pgfpathcurveto{\pgfqpoint{0.892372in}{1.429557in}}{\pgfqpoint{0.895645in}{1.437457in}}{\pgfqpoint{0.895645in}{1.445694in}}%
\pgfpathcurveto{\pgfqpoint{0.895645in}{1.453930in}}{\pgfqpoint{0.892372in}{1.461830in}}{\pgfqpoint{0.886548in}{1.467654in}}%
\pgfpathcurveto{\pgfqpoint{0.880725in}{1.473478in}}{\pgfqpoint{0.872825in}{1.476750in}}{\pgfqpoint{0.864588in}{1.476750in}}%
\pgfpathcurveto{\pgfqpoint{0.856352in}{1.476750in}}{\pgfqpoint{0.848452in}{1.473478in}}{\pgfqpoint{0.842628in}{1.467654in}}%
\pgfpathcurveto{\pgfqpoint{0.836804in}{1.461830in}}{\pgfqpoint{0.833532in}{1.453930in}}{\pgfqpoint{0.833532in}{1.445694in}}%
\pgfpathcurveto{\pgfqpoint{0.833532in}{1.437457in}}{\pgfqpoint{0.836804in}{1.429557in}}{\pgfqpoint{0.842628in}{1.423733in}}%
\pgfpathcurveto{\pgfqpoint{0.848452in}{1.417909in}}{\pgfqpoint{0.856352in}{1.414637in}}{\pgfqpoint{0.864588in}{1.414637in}}%
\pgfpathclose%
\pgfusepath{stroke,fill}%
\end{pgfscope}%
\begin{pgfscope}%
\pgfpathrectangle{\pgfqpoint{0.100000in}{0.212622in}}{\pgfqpoint{3.696000in}{3.696000in}}%
\pgfusepath{clip}%
\pgfsetbuttcap%
\pgfsetroundjoin%
\definecolor{currentfill}{rgb}{0.121569,0.466667,0.705882}%
\pgfsetfillcolor{currentfill}%
\pgfsetfillopacity{0.657340}%
\pgfsetlinewidth{1.003750pt}%
\definecolor{currentstroke}{rgb}{0.121569,0.466667,0.705882}%
\pgfsetstrokecolor{currentstroke}%
\pgfsetstrokeopacity{0.657340}%
\pgfsetdash{}{0pt}%
\pgfpathmoveto{\pgfqpoint{0.864588in}{1.414637in}}%
\pgfpathcurveto{\pgfqpoint{0.872825in}{1.414637in}}{\pgfqpoint{0.880725in}{1.417909in}}{\pgfqpoint{0.886548in}{1.423733in}}%
\pgfpathcurveto{\pgfqpoint{0.892372in}{1.429557in}}{\pgfqpoint{0.895645in}{1.437457in}}{\pgfqpoint{0.895645in}{1.445694in}}%
\pgfpathcurveto{\pgfqpoint{0.895645in}{1.453930in}}{\pgfqpoint{0.892372in}{1.461830in}}{\pgfqpoint{0.886548in}{1.467654in}}%
\pgfpathcurveto{\pgfqpoint{0.880725in}{1.473478in}}{\pgfqpoint{0.872825in}{1.476750in}}{\pgfqpoint{0.864588in}{1.476750in}}%
\pgfpathcurveto{\pgfqpoint{0.856352in}{1.476750in}}{\pgfqpoint{0.848452in}{1.473478in}}{\pgfqpoint{0.842628in}{1.467654in}}%
\pgfpathcurveto{\pgfqpoint{0.836804in}{1.461830in}}{\pgfqpoint{0.833532in}{1.453930in}}{\pgfqpoint{0.833532in}{1.445694in}}%
\pgfpathcurveto{\pgfqpoint{0.833532in}{1.437457in}}{\pgfqpoint{0.836804in}{1.429557in}}{\pgfqpoint{0.842628in}{1.423733in}}%
\pgfpathcurveto{\pgfqpoint{0.848452in}{1.417909in}}{\pgfqpoint{0.856352in}{1.414637in}}{\pgfqpoint{0.864588in}{1.414637in}}%
\pgfpathclose%
\pgfusepath{stroke,fill}%
\end{pgfscope}%
\begin{pgfscope}%
\pgfpathrectangle{\pgfqpoint{0.100000in}{0.212622in}}{\pgfqpoint{3.696000in}{3.696000in}}%
\pgfusepath{clip}%
\pgfsetbuttcap%
\pgfsetroundjoin%
\definecolor{currentfill}{rgb}{0.121569,0.466667,0.705882}%
\pgfsetfillcolor{currentfill}%
\pgfsetfillopacity{0.657340}%
\pgfsetlinewidth{1.003750pt}%
\definecolor{currentstroke}{rgb}{0.121569,0.466667,0.705882}%
\pgfsetstrokecolor{currentstroke}%
\pgfsetstrokeopacity{0.657340}%
\pgfsetdash{}{0pt}%
\pgfpathmoveto{\pgfqpoint{0.864588in}{1.414637in}}%
\pgfpathcurveto{\pgfqpoint{0.872825in}{1.414637in}}{\pgfqpoint{0.880725in}{1.417909in}}{\pgfqpoint{0.886548in}{1.423733in}}%
\pgfpathcurveto{\pgfqpoint{0.892372in}{1.429557in}}{\pgfqpoint{0.895645in}{1.437457in}}{\pgfqpoint{0.895645in}{1.445694in}}%
\pgfpathcurveto{\pgfqpoint{0.895645in}{1.453930in}}{\pgfqpoint{0.892372in}{1.461830in}}{\pgfqpoint{0.886548in}{1.467654in}}%
\pgfpathcurveto{\pgfqpoint{0.880725in}{1.473478in}}{\pgfqpoint{0.872825in}{1.476750in}}{\pgfqpoint{0.864588in}{1.476750in}}%
\pgfpathcurveto{\pgfqpoint{0.856352in}{1.476750in}}{\pgfqpoint{0.848452in}{1.473478in}}{\pgfqpoint{0.842628in}{1.467654in}}%
\pgfpathcurveto{\pgfqpoint{0.836804in}{1.461830in}}{\pgfqpoint{0.833532in}{1.453930in}}{\pgfqpoint{0.833532in}{1.445694in}}%
\pgfpathcurveto{\pgfqpoint{0.833532in}{1.437457in}}{\pgfqpoint{0.836804in}{1.429557in}}{\pgfqpoint{0.842628in}{1.423733in}}%
\pgfpathcurveto{\pgfqpoint{0.848452in}{1.417909in}}{\pgfqpoint{0.856352in}{1.414637in}}{\pgfqpoint{0.864588in}{1.414637in}}%
\pgfpathclose%
\pgfusepath{stroke,fill}%
\end{pgfscope}%
\begin{pgfscope}%
\pgfpathrectangle{\pgfqpoint{0.100000in}{0.212622in}}{\pgfqpoint{3.696000in}{3.696000in}}%
\pgfusepath{clip}%
\pgfsetbuttcap%
\pgfsetroundjoin%
\definecolor{currentfill}{rgb}{0.121569,0.466667,0.705882}%
\pgfsetfillcolor{currentfill}%
\pgfsetfillopacity{0.657340}%
\pgfsetlinewidth{1.003750pt}%
\definecolor{currentstroke}{rgb}{0.121569,0.466667,0.705882}%
\pgfsetstrokecolor{currentstroke}%
\pgfsetstrokeopacity{0.657340}%
\pgfsetdash{}{0pt}%
\pgfpathmoveto{\pgfqpoint{0.864588in}{1.414637in}}%
\pgfpathcurveto{\pgfqpoint{0.872825in}{1.414637in}}{\pgfqpoint{0.880725in}{1.417909in}}{\pgfqpoint{0.886548in}{1.423733in}}%
\pgfpathcurveto{\pgfqpoint{0.892372in}{1.429557in}}{\pgfqpoint{0.895645in}{1.437457in}}{\pgfqpoint{0.895645in}{1.445694in}}%
\pgfpathcurveto{\pgfqpoint{0.895645in}{1.453930in}}{\pgfqpoint{0.892372in}{1.461830in}}{\pgfqpoint{0.886548in}{1.467654in}}%
\pgfpathcurveto{\pgfqpoint{0.880725in}{1.473478in}}{\pgfqpoint{0.872825in}{1.476750in}}{\pgfqpoint{0.864588in}{1.476750in}}%
\pgfpathcurveto{\pgfqpoint{0.856352in}{1.476750in}}{\pgfqpoint{0.848452in}{1.473478in}}{\pgfqpoint{0.842628in}{1.467654in}}%
\pgfpathcurveto{\pgfqpoint{0.836804in}{1.461830in}}{\pgfqpoint{0.833532in}{1.453930in}}{\pgfqpoint{0.833532in}{1.445694in}}%
\pgfpathcurveto{\pgfqpoint{0.833532in}{1.437457in}}{\pgfqpoint{0.836804in}{1.429557in}}{\pgfqpoint{0.842628in}{1.423733in}}%
\pgfpathcurveto{\pgfqpoint{0.848452in}{1.417909in}}{\pgfqpoint{0.856352in}{1.414637in}}{\pgfqpoint{0.864588in}{1.414637in}}%
\pgfpathclose%
\pgfusepath{stroke,fill}%
\end{pgfscope}%
\begin{pgfscope}%
\pgfpathrectangle{\pgfqpoint{0.100000in}{0.212622in}}{\pgfqpoint{3.696000in}{3.696000in}}%
\pgfusepath{clip}%
\pgfsetbuttcap%
\pgfsetroundjoin%
\definecolor{currentfill}{rgb}{0.121569,0.466667,0.705882}%
\pgfsetfillcolor{currentfill}%
\pgfsetfillopacity{0.657340}%
\pgfsetlinewidth{1.003750pt}%
\definecolor{currentstroke}{rgb}{0.121569,0.466667,0.705882}%
\pgfsetstrokecolor{currentstroke}%
\pgfsetstrokeopacity{0.657340}%
\pgfsetdash{}{0pt}%
\pgfpathmoveto{\pgfqpoint{0.864588in}{1.414637in}}%
\pgfpathcurveto{\pgfqpoint{0.872825in}{1.414637in}}{\pgfqpoint{0.880725in}{1.417909in}}{\pgfqpoint{0.886548in}{1.423733in}}%
\pgfpathcurveto{\pgfqpoint{0.892372in}{1.429557in}}{\pgfqpoint{0.895645in}{1.437457in}}{\pgfqpoint{0.895645in}{1.445694in}}%
\pgfpathcurveto{\pgfqpoint{0.895645in}{1.453930in}}{\pgfqpoint{0.892372in}{1.461830in}}{\pgfqpoint{0.886548in}{1.467654in}}%
\pgfpathcurveto{\pgfqpoint{0.880725in}{1.473478in}}{\pgfqpoint{0.872825in}{1.476750in}}{\pgfqpoint{0.864588in}{1.476750in}}%
\pgfpathcurveto{\pgfqpoint{0.856352in}{1.476750in}}{\pgfqpoint{0.848452in}{1.473478in}}{\pgfqpoint{0.842628in}{1.467654in}}%
\pgfpathcurveto{\pgfqpoint{0.836804in}{1.461830in}}{\pgfqpoint{0.833532in}{1.453930in}}{\pgfqpoint{0.833532in}{1.445694in}}%
\pgfpathcurveto{\pgfqpoint{0.833532in}{1.437457in}}{\pgfqpoint{0.836804in}{1.429557in}}{\pgfqpoint{0.842628in}{1.423733in}}%
\pgfpathcurveto{\pgfqpoint{0.848452in}{1.417909in}}{\pgfqpoint{0.856352in}{1.414637in}}{\pgfqpoint{0.864588in}{1.414637in}}%
\pgfpathclose%
\pgfusepath{stroke,fill}%
\end{pgfscope}%
\begin{pgfscope}%
\pgfpathrectangle{\pgfqpoint{0.100000in}{0.212622in}}{\pgfqpoint{3.696000in}{3.696000in}}%
\pgfusepath{clip}%
\pgfsetbuttcap%
\pgfsetroundjoin%
\definecolor{currentfill}{rgb}{0.121569,0.466667,0.705882}%
\pgfsetfillcolor{currentfill}%
\pgfsetfillopacity{0.657340}%
\pgfsetlinewidth{1.003750pt}%
\definecolor{currentstroke}{rgb}{0.121569,0.466667,0.705882}%
\pgfsetstrokecolor{currentstroke}%
\pgfsetstrokeopacity{0.657340}%
\pgfsetdash{}{0pt}%
\pgfpathmoveto{\pgfqpoint{0.864588in}{1.414637in}}%
\pgfpathcurveto{\pgfqpoint{0.872825in}{1.414637in}}{\pgfqpoint{0.880725in}{1.417909in}}{\pgfqpoint{0.886548in}{1.423733in}}%
\pgfpathcurveto{\pgfqpoint{0.892372in}{1.429557in}}{\pgfqpoint{0.895645in}{1.437457in}}{\pgfqpoint{0.895645in}{1.445694in}}%
\pgfpathcurveto{\pgfqpoint{0.895645in}{1.453930in}}{\pgfqpoint{0.892372in}{1.461830in}}{\pgfqpoint{0.886548in}{1.467654in}}%
\pgfpathcurveto{\pgfqpoint{0.880725in}{1.473478in}}{\pgfqpoint{0.872825in}{1.476750in}}{\pgfqpoint{0.864588in}{1.476750in}}%
\pgfpathcurveto{\pgfqpoint{0.856352in}{1.476750in}}{\pgfqpoint{0.848452in}{1.473478in}}{\pgfqpoint{0.842628in}{1.467654in}}%
\pgfpathcurveto{\pgfqpoint{0.836804in}{1.461830in}}{\pgfqpoint{0.833532in}{1.453930in}}{\pgfqpoint{0.833532in}{1.445694in}}%
\pgfpathcurveto{\pgfqpoint{0.833532in}{1.437457in}}{\pgfqpoint{0.836804in}{1.429557in}}{\pgfqpoint{0.842628in}{1.423733in}}%
\pgfpathcurveto{\pgfqpoint{0.848452in}{1.417909in}}{\pgfqpoint{0.856352in}{1.414637in}}{\pgfqpoint{0.864588in}{1.414637in}}%
\pgfpathclose%
\pgfusepath{stroke,fill}%
\end{pgfscope}%
\begin{pgfscope}%
\pgfpathrectangle{\pgfqpoint{0.100000in}{0.212622in}}{\pgfqpoint{3.696000in}{3.696000in}}%
\pgfusepath{clip}%
\pgfsetbuttcap%
\pgfsetroundjoin%
\definecolor{currentfill}{rgb}{0.121569,0.466667,0.705882}%
\pgfsetfillcolor{currentfill}%
\pgfsetfillopacity{0.657340}%
\pgfsetlinewidth{1.003750pt}%
\definecolor{currentstroke}{rgb}{0.121569,0.466667,0.705882}%
\pgfsetstrokecolor{currentstroke}%
\pgfsetstrokeopacity{0.657340}%
\pgfsetdash{}{0pt}%
\pgfpathmoveto{\pgfqpoint{0.864588in}{1.414637in}}%
\pgfpathcurveto{\pgfqpoint{0.872825in}{1.414637in}}{\pgfqpoint{0.880725in}{1.417909in}}{\pgfqpoint{0.886548in}{1.423733in}}%
\pgfpathcurveto{\pgfqpoint{0.892372in}{1.429557in}}{\pgfqpoint{0.895645in}{1.437457in}}{\pgfqpoint{0.895645in}{1.445694in}}%
\pgfpathcurveto{\pgfqpoint{0.895645in}{1.453930in}}{\pgfqpoint{0.892372in}{1.461830in}}{\pgfqpoint{0.886548in}{1.467654in}}%
\pgfpathcurveto{\pgfqpoint{0.880725in}{1.473478in}}{\pgfqpoint{0.872825in}{1.476750in}}{\pgfqpoint{0.864588in}{1.476750in}}%
\pgfpathcurveto{\pgfqpoint{0.856352in}{1.476750in}}{\pgfqpoint{0.848452in}{1.473478in}}{\pgfqpoint{0.842628in}{1.467654in}}%
\pgfpathcurveto{\pgfqpoint{0.836804in}{1.461830in}}{\pgfqpoint{0.833532in}{1.453930in}}{\pgfqpoint{0.833532in}{1.445694in}}%
\pgfpathcurveto{\pgfqpoint{0.833532in}{1.437457in}}{\pgfqpoint{0.836804in}{1.429557in}}{\pgfqpoint{0.842628in}{1.423733in}}%
\pgfpathcurveto{\pgfqpoint{0.848452in}{1.417909in}}{\pgfqpoint{0.856352in}{1.414637in}}{\pgfqpoint{0.864588in}{1.414637in}}%
\pgfpathclose%
\pgfusepath{stroke,fill}%
\end{pgfscope}%
\begin{pgfscope}%
\pgfpathrectangle{\pgfqpoint{0.100000in}{0.212622in}}{\pgfqpoint{3.696000in}{3.696000in}}%
\pgfusepath{clip}%
\pgfsetbuttcap%
\pgfsetroundjoin%
\definecolor{currentfill}{rgb}{0.121569,0.466667,0.705882}%
\pgfsetfillcolor{currentfill}%
\pgfsetfillopacity{0.657340}%
\pgfsetlinewidth{1.003750pt}%
\definecolor{currentstroke}{rgb}{0.121569,0.466667,0.705882}%
\pgfsetstrokecolor{currentstroke}%
\pgfsetstrokeopacity{0.657340}%
\pgfsetdash{}{0pt}%
\pgfpathmoveto{\pgfqpoint{0.864588in}{1.414637in}}%
\pgfpathcurveto{\pgfqpoint{0.872825in}{1.414637in}}{\pgfqpoint{0.880725in}{1.417909in}}{\pgfqpoint{0.886548in}{1.423733in}}%
\pgfpathcurveto{\pgfqpoint{0.892372in}{1.429557in}}{\pgfqpoint{0.895645in}{1.437457in}}{\pgfqpoint{0.895645in}{1.445694in}}%
\pgfpathcurveto{\pgfqpoint{0.895645in}{1.453930in}}{\pgfqpoint{0.892372in}{1.461830in}}{\pgfqpoint{0.886548in}{1.467654in}}%
\pgfpathcurveto{\pgfqpoint{0.880725in}{1.473478in}}{\pgfqpoint{0.872825in}{1.476750in}}{\pgfqpoint{0.864588in}{1.476750in}}%
\pgfpathcurveto{\pgfqpoint{0.856352in}{1.476750in}}{\pgfqpoint{0.848452in}{1.473478in}}{\pgfqpoint{0.842628in}{1.467654in}}%
\pgfpathcurveto{\pgfqpoint{0.836804in}{1.461830in}}{\pgfqpoint{0.833532in}{1.453930in}}{\pgfqpoint{0.833532in}{1.445694in}}%
\pgfpathcurveto{\pgfqpoint{0.833532in}{1.437457in}}{\pgfqpoint{0.836804in}{1.429557in}}{\pgfqpoint{0.842628in}{1.423733in}}%
\pgfpathcurveto{\pgfqpoint{0.848452in}{1.417909in}}{\pgfqpoint{0.856352in}{1.414637in}}{\pgfqpoint{0.864588in}{1.414637in}}%
\pgfpathclose%
\pgfusepath{stroke,fill}%
\end{pgfscope}%
\begin{pgfscope}%
\pgfpathrectangle{\pgfqpoint{0.100000in}{0.212622in}}{\pgfqpoint{3.696000in}{3.696000in}}%
\pgfusepath{clip}%
\pgfsetbuttcap%
\pgfsetroundjoin%
\definecolor{currentfill}{rgb}{0.121569,0.466667,0.705882}%
\pgfsetfillcolor{currentfill}%
\pgfsetfillopacity{0.657340}%
\pgfsetlinewidth{1.003750pt}%
\definecolor{currentstroke}{rgb}{0.121569,0.466667,0.705882}%
\pgfsetstrokecolor{currentstroke}%
\pgfsetstrokeopacity{0.657340}%
\pgfsetdash{}{0pt}%
\pgfpathmoveto{\pgfqpoint{0.864588in}{1.414637in}}%
\pgfpathcurveto{\pgfqpoint{0.872825in}{1.414637in}}{\pgfqpoint{0.880725in}{1.417909in}}{\pgfqpoint{0.886548in}{1.423733in}}%
\pgfpathcurveto{\pgfqpoint{0.892372in}{1.429557in}}{\pgfqpoint{0.895645in}{1.437457in}}{\pgfqpoint{0.895645in}{1.445694in}}%
\pgfpathcurveto{\pgfqpoint{0.895645in}{1.453930in}}{\pgfqpoint{0.892372in}{1.461830in}}{\pgfqpoint{0.886548in}{1.467654in}}%
\pgfpathcurveto{\pgfqpoint{0.880725in}{1.473478in}}{\pgfqpoint{0.872825in}{1.476750in}}{\pgfqpoint{0.864588in}{1.476750in}}%
\pgfpathcurveto{\pgfqpoint{0.856352in}{1.476750in}}{\pgfqpoint{0.848452in}{1.473478in}}{\pgfqpoint{0.842628in}{1.467654in}}%
\pgfpathcurveto{\pgfqpoint{0.836804in}{1.461830in}}{\pgfqpoint{0.833532in}{1.453930in}}{\pgfqpoint{0.833532in}{1.445694in}}%
\pgfpathcurveto{\pgfqpoint{0.833532in}{1.437457in}}{\pgfqpoint{0.836804in}{1.429557in}}{\pgfqpoint{0.842628in}{1.423733in}}%
\pgfpathcurveto{\pgfqpoint{0.848452in}{1.417909in}}{\pgfqpoint{0.856352in}{1.414637in}}{\pgfqpoint{0.864588in}{1.414637in}}%
\pgfpathclose%
\pgfusepath{stroke,fill}%
\end{pgfscope}%
\begin{pgfscope}%
\pgfpathrectangle{\pgfqpoint{0.100000in}{0.212622in}}{\pgfqpoint{3.696000in}{3.696000in}}%
\pgfusepath{clip}%
\pgfsetbuttcap%
\pgfsetroundjoin%
\definecolor{currentfill}{rgb}{0.121569,0.466667,0.705882}%
\pgfsetfillcolor{currentfill}%
\pgfsetfillopacity{0.657340}%
\pgfsetlinewidth{1.003750pt}%
\definecolor{currentstroke}{rgb}{0.121569,0.466667,0.705882}%
\pgfsetstrokecolor{currentstroke}%
\pgfsetstrokeopacity{0.657340}%
\pgfsetdash{}{0pt}%
\pgfpathmoveto{\pgfqpoint{0.864588in}{1.414637in}}%
\pgfpathcurveto{\pgfqpoint{0.872825in}{1.414637in}}{\pgfqpoint{0.880725in}{1.417909in}}{\pgfqpoint{0.886548in}{1.423733in}}%
\pgfpathcurveto{\pgfqpoint{0.892372in}{1.429557in}}{\pgfqpoint{0.895645in}{1.437457in}}{\pgfqpoint{0.895645in}{1.445694in}}%
\pgfpathcurveto{\pgfqpoint{0.895645in}{1.453930in}}{\pgfqpoint{0.892372in}{1.461830in}}{\pgfqpoint{0.886548in}{1.467654in}}%
\pgfpathcurveto{\pgfqpoint{0.880725in}{1.473478in}}{\pgfqpoint{0.872825in}{1.476750in}}{\pgfqpoint{0.864588in}{1.476750in}}%
\pgfpathcurveto{\pgfqpoint{0.856352in}{1.476750in}}{\pgfqpoint{0.848452in}{1.473478in}}{\pgfqpoint{0.842628in}{1.467654in}}%
\pgfpathcurveto{\pgfqpoint{0.836804in}{1.461830in}}{\pgfqpoint{0.833532in}{1.453930in}}{\pgfqpoint{0.833532in}{1.445694in}}%
\pgfpathcurveto{\pgfqpoint{0.833532in}{1.437457in}}{\pgfqpoint{0.836804in}{1.429557in}}{\pgfqpoint{0.842628in}{1.423733in}}%
\pgfpathcurveto{\pgfqpoint{0.848452in}{1.417909in}}{\pgfqpoint{0.856352in}{1.414637in}}{\pgfqpoint{0.864588in}{1.414637in}}%
\pgfpathclose%
\pgfusepath{stroke,fill}%
\end{pgfscope}%
\begin{pgfscope}%
\pgfpathrectangle{\pgfqpoint{0.100000in}{0.212622in}}{\pgfqpoint{3.696000in}{3.696000in}}%
\pgfusepath{clip}%
\pgfsetbuttcap%
\pgfsetroundjoin%
\definecolor{currentfill}{rgb}{0.121569,0.466667,0.705882}%
\pgfsetfillcolor{currentfill}%
\pgfsetfillopacity{0.657340}%
\pgfsetlinewidth{1.003750pt}%
\definecolor{currentstroke}{rgb}{0.121569,0.466667,0.705882}%
\pgfsetstrokecolor{currentstroke}%
\pgfsetstrokeopacity{0.657340}%
\pgfsetdash{}{0pt}%
\pgfpathmoveto{\pgfqpoint{0.864588in}{1.414637in}}%
\pgfpathcurveto{\pgfqpoint{0.872825in}{1.414637in}}{\pgfqpoint{0.880725in}{1.417909in}}{\pgfqpoint{0.886548in}{1.423733in}}%
\pgfpathcurveto{\pgfqpoint{0.892372in}{1.429557in}}{\pgfqpoint{0.895645in}{1.437457in}}{\pgfqpoint{0.895645in}{1.445694in}}%
\pgfpathcurveto{\pgfqpoint{0.895645in}{1.453930in}}{\pgfqpoint{0.892372in}{1.461830in}}{\pgfqpoint{0.886548in}{1.467654in}}%
\pgfpathcurveto{\pgfqpoint{0.880725in}{1.473478in}}{\pgfqpoint{0.872825in}{1.476750in}}{\pgfqpoint{0.864588in}{1.476750in}}%
\pgfpathcurveto{\pgfqpoint{0.856352in}{1.476750in}}{\pgfqpoint{0.848452in}{1.473478in}}{\pgfqpoint{0.842628in}{1.467654in}}%
\pgfpathcurveto{\pgfqpoint{0.836804in}{1.461830in}}{\pgfqpoint{0.833532in}{1.453930in}}{\pgfqpoint{0.833532in}{1.445694in}}%
\pgfpathcurveto{\pgfqpoint{0.833532in}{1.437457in}}{\pgfqpoint{0.836804in}{1.429557in}}{\pgfqpoint{0.842628in}{1.423733in}}%
\pgfpathcurveto{\pgfqpoint{0.848452in}{1.417909in}}{\pgfqpoint{0.856352in}{1.414637in}}{\pgfqpoint{0.864588in}{1.414637in}}%
\pgfpathclose%
\pgfusepath{stroke,fill}%
\end{pgfscope}%
\begin{pgfscope}%
\pgfpathrectangle{\pgfqpoint{0.100000in}{0.212622in}}{\pgfqpoint{3.696000in}{3.696000in}}%
\pgfusepath{clip}%
\pgfsetbuttcap%
\pgfsetroundjoin%
\definecolor{currentfill}{rgb}{0.121569,0.466667,0.705882}%
\pgfsetfillcolor{currentfill}%
\pgfsetfillopacity{0.657340}%
\pgfsetlinewidth{1.003750pt}%
\definecolor{currentstroke}{rgb}{0.121569,0.466667,0.705882}%
\pgfsetstrokecolor{currentstroke}%
\pgfsetstrokeopacity{0.657340}%
\pgfsetdash{}{0pt}%
\pgfpathmoveto{\pgfqpoint{0.864588in}{1.414637in}}%
\pgfpathcurveto{\pgfqpoint{0.872825in}{1.414637in}}{\pgfqpoint{0.880725in}{1.417909in}}{\pgfqpoint{0.886548in}{1.423733in}}%
\pgfpathcurveto{\pgfqpoint{0.892372in}{1.429557in}}{\pgfqpoint{0.895645in}{1.437457in}}{\pgfqpoint{0.895645in}{1.445694in}}%
\pgfpathcurveto{\pgfqpoint{0.895645in}{1.453930in}}{\pgfqpoint{0.892372in}{1.461830in}}{\pgfqpoint{0.886548in}{1.467654in}}%
\pgfpathcurveto{\pgfqpoint{0.880725in}{1.473478in}}{\pgfqpoint{0.872825in}{1.476750in}}{\pgfqpoint{0.864588in}{1.476750in}}%
\pgfpathcurveto{\pgfqpoint{0.856352in}{1.476750in}}{\pgfqpoint{0.848452in}{1.473478in}}{\pgfqpoint{0.842628in}{1.467654in}}%
\pgfpathcurveto{\pgfqpoint{0.836804in}{1.461830in}}{\pgfqpoint{0.833532in}{1.453930in}}{\pgfqpoint{0.833532in}{1.445694in}}%
\pgfpathcurveto{\pgfqpoint{0.833532in}{1.437457in}}{\pgfqpoint{0.836804in}{1.429557in}}{\pgfqpoint{0.842628in}{1.423733in}}%
\pgfpathcurveto{\pgfqpoint{0.848452in}{1.417909in}}{\pgfqpoint{0.856352in}{1.414637in}}{\pgfqpoint{0.864588in}{1.414637in}}%
\pgfpathclose%
\pgfusepath{stroke,fill}%
\end{pgfscope}%
\begin{pgfscope}%
\pgfpathrectangle{\pgfqpoint{0.100000in}{0.212622in}}{\pgfqpoint{3.696000in}{3.696000in}}%
\pgfusepath{clip}%
\pgfsetbuttcap%
\pgfsetroundjoin%
\definecolor{currentfill}{rgb}{0.121569,0.466667,0.705882}%
\pgfsetfillcolor{currentfill}%
\pgfsetfillopacity{0.657340}%
\pgfsetlinewidth{1.003750pt}%
\definecolor{currentstroke}{rgb}{0.121569,0.466667,0.705882}%
\pgfsetstrokecolor{currentstroke}%
\pgfsetstrokeopacity{0.657340}%
\pgfsetdash{}{0pt}%
\pgfpathmoveto{\pgfqpoint{0.864588in}{1.414637in}}%
\pgfpathcurveto{\pgfqpoint{0.872825in}{1.414637in}}{\pgfqpoint{0.880725in}{1.417909in}}{\pgfqpoint{0.886548in}{1.423733in}}%
\pgfpathcurveto{\pgfqpoint{0.892372in}{1.429557in}}{\pgfqpoint{0.895645in}{1.437457in}}{\pgfqpoint{0.895645in}{1.445694in}}%
\pgfpathcurveto{\pgfqpoint{0.895645in}{1.453930in}}{\pgfqpoint{0.892372in}{1.461830in}}{\pgfqpoint{0.886548in}{1.467654in}}%
\pgfpathcurveto{\pgfqpoint{0.880725in}{1.473478in}}{\pgfqpoint{0.872825in}{1.476750in}}{\pgfqpoint{0.864588in}{1.476750in}}%
\pgfpathcurveto{\pgfqpoint{0.856352in}{1.476750in}}{\pgfqpoint{0.848452in}{1.473478in}}{\pgfqpoint{0.842628in}{1.467654in}}%
\pgfpathcurveto{\pgfqpoint{0.836804in}{1.461830in}}{\pgfqpoint{0.833532in}{1.453930in}}{\pgfqpoint{0.833532in}{1.445694in}}%
\pgfpathcurveto{\pgfqpoint{0.833532in}{1.437457in}}{\pgfqpoint{0.836804in}{1.429557in}}{\pgfqpoint{0.842628in}{1.423733in}}%
\pgfpathcurveto{\pgfqpoint{0.848452in}{1.417909in}}{\pgfqpoint{0.856352in}{1.414637in}}{\pgfqpoint{0.864588in}{1.414637in}}%
\pgfpathclose%
\pgfusepath{stroke,fill}%
\end{pgfscope}%
\begin{pgfscope}%
\pgfpathrectangle{\pgfqpoint{0.100000in}{0.212622in}}{\pgfqpoint{3.696000in}{3.696000in}}%
\pgfusepath{clip}%
\pgfsetbuttcap%
\pgfsetroundjoin%
\definecolor{currentfill}{rgb}{0.121569,0.466667,0.705882}%
\pgfsetfillcolor{currentfill}%
\pgfsetfillopacity{0.657340}%
\pgfsetlinewidth{1.003750pt}%
\definecolor{currentstroke}{rgb}{0.121569,0.466667,0.705882}%
\pgfsetstrokecolor{currentstroke}%
\pgfsetstrokeopacity{0.657340}%
\pgfsetdash{}{0pt}%
\pgfpathmoveto{\pgfqpoint{0.864588in}{1.414637in}}%
\pgfpathcurveto{\pgfqpoint{0.872825in}{1.414637in}}{\pgfqpoint{0.880725in}{1.417909in}}{\pgfqpoint{0.886548in}{1.423733in}}%
\pgfpathcurveto{\pgfqpoint{0.892372in}{1.429557in}}{\pgfqpoint{0.895645in}{1.437457in}}{\pgfqpoint{0.895645in}{1.445694in}}%
\pgfpathcurveto{\pgfqpoint{0.895645in}{1.453930in}}{\pgfqpoint{0.892372in}{1.461830in}}{\pgfqpoint{0.886548in}{1.467654in}}%
\pgfpathcurveto{\pgfqpoint{0.880725in}{1.473478in}}{\pgfqpoint{0.872825in}{1.476750in}}{\pgfqpoint{0.864588in}{1.476750in}}%
\pgfpathcurveto{\pgfqpoint{0.856352in}{1.476750in}}{\pgfqpoint{0.848452in}{1.473478in}}{\pgfqpoint{0.842628in}{1.467654in}}%
\pgfpathcurveto{\pgfqpoint{0.836804in}{1.461830in}}{\pgfqpoint{0.833532in}{1.453930in}}{\pgfqpoint{0.833532in}{1.445694in}}%
\pgfpathcurveto{\pgfqpoint{0.833532in}{1.437457in}}{\pgfqpoint{0.836804in}{1.429557in}}{\pgfqpoint{0.842628in}{1.423733in}}%
\pgfpathcurveto{\pgfqpoint{0.848452in}{1.417909in}}{\pgfqpoint{0.856352in}{1.414637in}}{\pgfqpoint{0.864588in}{1.414637in}}%
\pgfpathclose%
\pgfusepath{stroke,fill}%
\end{pgfscope}%
\begin{pgfscope}%
\pgfpathrectangle{\pgfqpoint{0.100000in}{0.212622in}}{\pgfqpoint{3.696000in}{3.696000in}}%
\pgfusepath{clip}%
\pgfsetbuttcap%
\pgfsetroundjoin%
\definecolor{currentfill}{rgb}{0.121569,0.466667,0.705882}%
\pgfsetfillcolor{currentfill}%
\pgfsetfillopacity{0.657340}%
\pgfsetlinewidth{1.003750pt}%
\definecolor{currentstroke}{rgb}{0.121569,0.466667,0.705882}%
\pgfsetstrokecolor{currentstroke}%
\pgfsetstrokeopacity{0.657340}%
\pgfsetdash{}{0pt}%
\pgfpathmoveto{\pgfqpoint{0.864588in}{1.414637in}}%
\pgfpathcurveto{\pgfqpoint{0.872825in}{1.414637in}}{\pgfqpoint{0.880725in}{1.417909in}}{\pgfqpoint{0.886548in}{1.423733in}}%
\pgfpathcurveto{\pgfqpoint{0.892372in}{1.429557in}}{\pgfqpoint{0.895645in}{1.437457in}}{\pgfqpoint{0.895645in}{1.445694in}}%
\pgfpathcurveto{\pgfqpoint{0.895645in}{1.453930in}}{\pgfqpoint{0.892372in}{1.461830in}}{\pgfqpoint{0.886548in}{1.467654in}}%
\pgfpathcurveto{\pgfqpoint{0.880725in}{1.473478in}}{\pgfqpoint{0.872825in}{1.476750in}}{\pgfqpoint{0.864588in}{1.476750in}}%
\pgfpathcurveto{\pgfqpoint{0.856352in}{1.476750in}}{\pgfqpoint{0.848452in}{1.473478in}}{\pgfqpoint{0.842628in}{1.467654in}}%
\pgfpathcurveto{\pgfqpoint{0.836804in}{1.461830in}}{\pgfqpoint{0.833532in}{1.453930in}}{\pgfqpoint{0.833532in}{1.445694in}}%
\pgfpathcurveto{\pgfqpoint{0.833532in}{1.437457in}}{\pgfqpoint{0.836804in}{1.429557in}}{\pgfqpoint{0.842628in}{1.423733in}}%
\pgfpathcurveto{\pgfqpoint{0.848452in}{1.417909in}}{\pgfqpoint{0.856352in}{1.414637in}}{\pgfqpoint{0.864588in}{1.414637in}}%
\pgfpathclose%
\pgfusepath{stroke,fill}%
\end{pgfscope}%
\begin{pgfscope}%
\pgfpathrectangle{\pgfqpoint{0.100000in}{0.212622in}}{\pgfqpoint{3.696000in}{3.696000in}}%
\pgfusepath{clip}%
\pgfsetbuttcap%
\pgfsetroundjoin%
\definecolor{currentfill}{rgb}{0.121569,0.466667,0.705882}%
\pgfsetfillcolor{currentfill}%
\pgfsetfillopacity{0.657340}%
\pgfsetlinewidth{1.003750pt}%
\definecolor{currentstroke}{rgb}{0.121569,0.466667,0.705882}%
\pgfsetstrokecolor{currentstroke}%
\pgfsetstrokeopacity{0.657340}%
\pgfsetdash{}{0pt}%
\pgfpathmoveto{\pgfqpoint{0.864588in}{1.414637in}}%
\pgfpathcurveto{\pgfqpoint{0.872825in}{1.414637in}}{\pgfqpoint{0.880725in}{1.417909in}}{\pgfqpoint{0.886548in}{1.423733in}}%
\pgfpathcurveto{\pgfqpoint{0.892372in}{1.429557in}}{\pgfqpoint{0.895645in}{1.437457in}}{\pgfqpoint{0.895645in}{1.445694in}}%
\pgfpathcurveto{\pgfqpoint{0.895645in}{1.453930in}}{\pgfqpoint{0.892372in}{1.461830in}}{\pgfqpoint{0.886548in}{1.467654in}}%
\pgfpathcurveto{\pgfqpoint{0.880725in}{1.473478in}}{\pgfqpoint{0.872825in}{1.476750in}}{\pgfqpoint{0.864588in}{1.476750in}}%
\pgfpathcurveto{\pgfqpoint{0.856352in}{1.476750in}}{\pgfqpoint{0.848452in}{1.473478in}}{\pgfqpoint{0.842628in}{1.467654in}}%
\pgfpathcurveto{\pgfqpoint{0.836804in}{1.461830in}}{\pgfqpoint{0.833532in}{1.453930in}}{\pgfqpoint{0.833532in}{1.445694in}}%
\pgfpathcurveto{\pgfqpoint{0.833532in}{1.437457in}}{\pgfqpoint{0.836804in}{1.429557in}}{\pgfqpoint{0.842628in}{1.423733in}}%
\pgfpathcurveto{\pgfqpoint{0.848452in}{1.417909in}}{\pgfqpoint{0.856352in}{1.414637in}}{\pgfqpoint{0.864588in}{1.414637in}}%
\pgfpathclose%
\pgfusepath{stroke,fill}%
\end{pgfscope}%
\begin{pgfscope}%
\pgfpathrectangle{\pgfqpoint{0.100000in}{0.212622in}}{\pgfqpoint{3.696000in}{3.696000in}}%
\pgfusepath{clip}%
\pgfsetbuttcap%
\pgfsetroundjoin%
\definecolor{currentfill}{rgb}{0.121569,0.466667,0.705882}%
\pgfsetfillcolor{currentfill}%
\pgfsetfillopacity{0.657340}%
\pgfsetlinewidth{1.003750pt}%
\definecolor{currentstroke}{rgb}{0.121569,0.466667,0.705882}%
\pgfsetstrokecolor{currentstroke}%
\pgfsetstrokeopacity{0.657340}%
\pgfsetdash{}{0pt}%
\pgfpathmoveto{\pgfqpoint{0.864588in}{1.414637in}}%
\pgfpathcurveto{\pgfqpoint{0.872825in}{1.414637in}}{\pgfqpoint{0.880725in}{1.417909in}}{\pgfqpoint{0.886548in}{1.423733in}}%
\pgfpathcurveto{\pgfqpoint{0.892372in}{1.429557in}}{\pgfqpoint{0.895645in}{1.437457in}}{\pgfqpoint{0.895645in}{1.445694in}}%
\pgfpathcurveto{\pgfqpoint{0.895645in}{1.453930in}}{\pgfqpoint{0.892372in}{1.461830in}}{\pgfqpoint{0.886548in}{1.467654in}}%
\pgfpathcurveto{\pgfqpoint{0.880725in}{1.473478in}}{\pgfqpoint{0.872825in}{1.476750in}}{\pgfqpoint{0.864588in}{1.476750in}}%
\pgfpathcurveto{\pgfqpoint{0.856352in}{1.476750in}}{\pgfqpoint{0.848452in}{1.473478in}}{\pgfqpoint{0.842628in}{1.467654in}}%
\pgfpathcurveto{\pgfqpoint{0.836804in}{1.461830in}}{\pgfqpoint{0.833532in}{1.453930in}}{\pgfqpoint{0.833532in}{1.445694in}}%
\pgfpathcurveto{\pgfqpoint{0.833532in}{1.437457in}}{\pgfqpoint{0.836804in}{1.429557in}}{\pgfqpoint{0.842628in}{1.423733in}}%
\pgfpathcurveto{\pgfqpoint{0.848452in}{1.417909in}}{\pgfqpoint{0.856352in}{1.414637in}}{\pgfqpoint{0.864588in}{1.414637in}}%
\pgfpathclose%
\pgfusepath{stroke,fill}%
\end{pgfscope}%
\begin{pgfscope}%
\pgfpathrectangle{\pgfqpoint{0.100000in}{0.212622in}}{\pgfqpoint{3.696000in}{3.696000in}}%
\pgfusepath{clip}%
\pgfsetbuttcap%
\pgfsetroundjoin%
\definecolor{currentfill}{rgb}{0.121569,0.466667,0.705882}%
\pgfsetfillcolor{currentfill}%
\pgfsetfillopacity{0.657340}%
\pgfsetlinewidth{1.003750pt}%
\definecolor{currentstroke}{rgb}{0.121569,0.466667,0.705882}%
\pgfsetstrokecolor{currentstroke}%
\pgfsetstrokeopacity{0.657340}%
\pgfsetdash{}{0pt}%
\pgfpathmoveto{\pgfqpoint{0.864588in}{1.414637in}}%
\pgfpathcurveto{\pgfqpoint{0.872825in}{1.414637in}}{\pgfqpoint{0.880725in}{1.417909in}}{\pgfqpoint{0.886548in}{1.423733in}}%
\pgfpathcurveto{\pgfqpoint{0.892372in}{1.429557in}}{\pgfqpoint{0.895645in}{1.437457in}}{\pgfqpoint{0.895645in}{1.445694in}}%
\pgfpathcurveto{\pgfqpoint{0.895645in}{1.453930in}}{\pgfqpoint{0.892372in}{1.461830in}}{\pgfqpoint{0.886548in}{1.467654in}}%
\pgfpathcurveto{\pgfqpoint{0.880725in}{1.473478in}}{\pgfqpoint{0.872825in}{1.476750in}}{\pgfqpoint{0.864588in}{1.476750in}}%
\pgfpathcurveto{\pgfqpoint{0.856352in}{1.476750in}}{\pgfqpoint{0.848452in}{1.473478in}}{\pgfqpoint{0.842628in}{1.467654in}}%
\pgfpathcurveto{\pgfqpoint{0.836804in}{1.461830in}}{\pgfqpoint{0.833532in}{1.453930in}}{\pgfqpoint{0.833532in}{1.445694in}}%
\pgfpathcurveto{\pgfqpoint{0.833532in}{1.437457in}}{\pgfqpoint{0.836804in}{1.429557in}}{\pgfqpoint{0.842628in}{1.423733in}}%
\pgfpathcurveto{\pgfqpoint{0.848452in}{1.417909in}}{\pgfqpoint{0.856352in}{1.414637in}}{\pgfqpoint{0.864588in}{1.414637in}}%
\pgfpathclose%
\pgfusepath{stroke,fill}%
\end{pgfscope}%
\begin{pgfscope}%
\pgfpathrectangle{\pgfqpoint{0.100000in}{0.212622in}}{\pgfqpoint{3.696000in}{3.696000in}}%
\pgfusepath{clip}%
\pgfsetbuttcap%
\pgfsetroundjoin%
\definecolor{currentfill}{rgb}{0.121569,0.466667,0.705882}%
\pgfsetfillcolor{currentfill}%
\pgfsetfillopacity{0.657340}%
\pgfsetlinewidth{1.003750pt}%
\definecolor{currentstroke}{rgb}{0.121569,0.466667,0.705882}%
\pgfsetstrokecolor{currentstroke}%
\pgfsetstrokeopacity{0.657340}%
\pgfsetdash{}{0pt}%
\pgfpathmoveto{\pgfqpoint{0.864588in}{1.414637in}}%
\pgfpathcurveto{\pgfqpoint{0.872825in}{1.414637in}}{\pgfqpoint{0.880725in}{1.417909in}}{\pgfqpoint{0.886548in}{1.423733in}}%
\pgfpathcurveto{\pgfqpoint{0.892372in}{1.429557in}}{\pgfqpoint{0.895645in}{1.437457in}}{\pgfqpoint{0.895645in}{1.445694in}}%
\pgfpathcurveto{\pgfqpoint{0.895645in}{1.453930in}}{\pgfqpoint{0.892372in}{1.461830in}}{\pgfqpoint{0.886548in}{1.467654in}}%
\pgfpathcurveto{\pgfqpoint{0.880725in}{1.473478in}}{\pgfqpoint{0.872825in}{1.476750in}}{\pgfqpoint{0.864588in}{1.476750in}}%
\pgfpathcurveto{\pgfqpoint{0.856352in}{1.476750in}}{\pgfqpoint{0.848452in}{1.473478in}}{\pgfqpoint{0.842628in}{1.467654in}}%
\pgfpathcurveto{\pgfqpoint{0.836804in}{1.461830in}}{\pgfqpoint{0.833532in}{1.453930in}}{\pgfqpoint{0.833532in}{1.445694in}}%
\pgfpathcurveto{\pgfqpoint{0.833532in}{1.437457in}}{\pgfqpoint{0.836804in}{1.429557in}}{\pgfqpoint{0.842628in}{1.423733in}}%
\pgfpathcurveto{\pgfqpoint{0.848452in}{1.417909in}}{\pgfqpoint{0.856352in}{1.414637in}}{\pgfqpoint{0.864588in}{1.414637in}}%
\pgfpathclose%
\pgfusepath{stroke,fill}%
\end{pgfscope}%
\begin{pgfscope}%
\pgfpathrectangle{\pgfqpoint{0.100000in}{0.212622in}}{\pgfqpoint{3.696000in}{3.696000in}}%
\pgfusepath{clip}%
\pgfsetbuttcap%
\pgfsetroundjoin%
\definecolor{currentfill}{rgb}{0.121569,0.466667,0.705882}%
\pgfsetfillcolor{currentfill}%
\pgfsetfillopacity{0.657340}%
\pgfsetlinewidth{1.003750pt}%
\definecolor{currentstroke}{rgb}{0.121569,0.466667,0.705882}%
\pgfsetstrokecolor{currentstroke}%
\pgfsetstrokeopacity{0.657340}%
\pgfsetdash{}{0pt}%
\pgfpathmoveto{\pgfqpoint{0.864588in}{1.414637in}}%
\pgfpathcurveto{\pgfqpoint{0.872825in}{1.414637in}}{\pgfqpoint{0.880725in}{1.417909in}}{\pgfqpoint{0.886548in}{1.423733in}}%
\pgfpathcurveto{\pgfqpoint{0.892372in}{1.429557in}}{\pgfqpoint{0.895645in}{1.437457in}}{\pgfqpoint{0.895645in}{1.445694in}}%
\pgfpathcurveto{\pgfqpoint{0.895645in}{1.453930in}}{\pgfqpoint{0.892372in}{1.461830in}}{\pgfqpoint{0.886548in}{1.467654in}}%
\pgfpathcurveto{\pgfqpoint{0.880725in}{1.473478in}}{\pgfqpoint{0.872825in}{1.476750in}}{\pgfqpoint{0.864588in}{1.476750in}}%
\pgfpathcurveto{\pgfqpoint{0.856352in}{1.476750in}}{\pgfqpoint{0.848452in}{1.473478in}}{\pgfqpoint{0.842628in}{1.467654in}}%
\pgfpathcurveto{\pgfqpoint{0.836804in}{1.461830in}}{\pgfqpoint{0.833532in}{1.453930in}}{\pgfqpoint{0.833532in}{1.445694in}}%
\pgfpathcurveto{\pgfqpoint{0.833532in}{1.437457in}}{\pgfqpoint{0.836804in}{1.429557in}}{\pgfqpoint{0.842628in}{1.423733in}}%
\pgfpathcurveto{\pgfqpoint{0.848452in}{1.417909in}}{\pgfqpoint{0.856352in}{1.414637in}}{\pgfqpoint{0.864588in}{1.414637in}}%
\pgfpathclose%
\pgfusepath{stroke,fill}%
\end{pgfscope}%
\begin{pgfscope}%
\pgfpathrectangle{\pgfqpoint{0.100000in}{0.212622in}}{\pgfqpoint{3.696000in}{3.696000in}}%
\pgfusepath{clip}%
\pgfsetbuttcap%
\pgfsetroundjoin%
\definecolor{currentfill}{rgb}{0.121569,0.466667,0.705882}%
\pgfsetfillcolor{currentfill}%
\pgfsetfillopacity{0.657340}%
\pgfsetlinewidth{1.003750pt}%
\definecolor{currentstroke}{rgb}{0.121569,0.466667,0.705882}%
\pgfsetstrokecolor{currentstroke}%
\pgfsetstrokeopacity{0.657340}%
\pgfsetdash{}{0pt}%
\pgfpathmoveto{\pgfqpoint{0.864588in}{1.414637in}}%
\pgfpathcurveto{\pgfqpoint{0.872825in}{1.414637in}}{\pgfqpoint{0.880725in}{1.417909in}}{\pgfqpoint{0.886548in}{1.423733in}}%
\pgfpathcurveto{\pgfqpoint{0.892372in}{1.429557in}}{\pgfqpoint{0.895645in}{1.437457in}}{\pgfqpoint{0.895645in}{1.445694in}}%
\pgfpathcurveto{\pgfqpoint{0.895645in}{1.453930in}}{\pgfqpoint{0.892372in}{1.461830in}}{\pgfqpoint{0.886548in}{1.467654in}}%
\pgfpathcurveto{\pgfqpoint{0.880725in}{1.473478in}}{\pgfqpoint{0.872825in}{1.476750in}}{\pgfqpoint{0.864588in}{1.476750in}}%
\pgfpathcurveto{\pgfqpoint{0.856352in}{1.476750in}}{\pgfqpoint{0.848452in}{1.473478in}}{\pgfqpoint{0.842628in}{1.467654in}}%
\pgfpathcurveto{\pgfqpoint{0.836804in}{1.461830in}}{\pgfqpoint{0.833532in}{1.453930in}}{\pgfqpoint{0.833532in}{1.445694in}}%
\pgfpathcurveto{\pgfqpoint{0.833532in}{1.437457in}}{\pgfqpoint{0.836804in}{1.429557in}}{\pgfqpoint{0.842628in}{1.423733in}}%
\pgfpathcurveto{\pgfqpoint{0.848452in}{1.417909in}}{\pgfqpoint{0.856352in}{1.414637in}}{\pgfqpoint{0.864588in}{1.414637in}}%
\pgfpathclose%
\pgfusepath{stroke,fill}%
\end{pgfscope}%
\begin{pgfscope}%
\pgfpathrectangle{\pgfqpoint{0.100000in}{0.212622in}}{\pgfqpoint{3.696000in}{3.696000in}}%
\pgfusepath{clip}%
\pgfsetbuttcap%
\pgfsetroundjoin%
\definecolor{currentfill}{rgb}{0.121569,0.466667,0.705882}%
\pgfsetfillcolor{currentfill}%
\pgfsetfillopacity{0.657340}%
\pgfsetlinewidth{1.003750pt}%
\definecolor{currentstroke}{rgb}{0.121569,0.466667,0.705882}%
\pgfsetstrokecolor{currentstroke}%
\pgfsetstrokeopacity{0.657340}%
\pgfsetdash{}{0pt}%
\pgfpathmoveto{\pgfqpoint{0.864588in}{1.414637in}}%
\pgfpathcurveto{\pgfqpoint{0.872825in}{1.414637in}}{\pgfqpoint{0.880725in}{1.417909in}}{\pgfqpoint{0.886548in}{1.423733in}}%
\pgfpathcurveto{\pgfqpoint{0.892372in}{1.429557in}}{\pgfqpoint{0.895645in}{1.437457in}}{\pgfqpoint{0.895645in}{1.445694in}}%
\pgfpathcurveto{\pgfqpoint{0.895645in}{1.453930in}}{\pgfqpoint{0.892372in}{1.461830in}}{\pgfqpoint{0.886548in}{1.467654in}}%
\pgfpathcurveto{\pgfqpoint{0.880725in}{1.473478in}}{\pgfqpoint{0.872825in}{1.476750in}}{\pgfqpoint{0.864588in}{1.476750in}}%
\pgfpathcurveto{\pgfqpoint{0.856352in}{1.476750in}}{\pgfqpoint{0.848452in}{1.473478in}}{\pgfqpoint{0.842628in}{1.467654in}}%
\pgfpathcurveto{\pgfqpoint{0.836804in}{1.461830in}}{\pgfqpoint{0.833532in}{1.453930in}}{\pgfqpoint{0.833532in}{1.445694in}}%
\pgfpathcurveto{\pgfqpoint{0.833532in}{1.437457in}}{\pgfqpoint{0.836804in}{1.429557in}}{\pgfqpoint{0.842628in}{1.423733in}}%
\pgfpathcurveto{\pgfqpoint{0.848452in}{1.417909in}}{\pgfqpoint{0.856352in}{1.414637in}}{\pgfqpoint{0.864588in}{1.414637in}}%
\pgfpathclose%
\pgfusepath{stroke,fill}%
\end{pgfscope}%
\begin{pgfscope}%
\pgfpathrectangle{\pgfqpoint{0.100000in}{0.212622in}}{\pgfqpoint{3.696000in}{3.696000in}}%
\pgfusepath{clip}%
\pgfsetbuttcap%
\pgfsetroundjoin%
\definecolor{currentfill}{rgb}{0.121569,0.466667,0.705882}%
\pgfsetfillcolor{currentfill}%
\pgfsetfillopacity{0.657340}%
\pgfsetlinewidth{1.003750pt}%
\definecolor{currentstroke}{rgb}{0.121569,0.466667,0.705882}%
\pgfsetstrokecolor{currentstroke}%
\pgfsetstrokeopacity{0.657340}%
\pgfsetdash{}{0pt}%
\pgfpathmoveto{\pgfqpoint{0.864588in}{1.414637in}}%
\pgfpathcurveto{\pgfqpoint{0.872825in}{1.414637in}}{\pgfqpoint{0.880725in}{1.417909in}}{\pgfqpoint{0.886548in}{1.423733in}}%
\pgfpathcurveto{\pgfqpoint{0.892372in}{1.429557in}}{\pgfqpoint{0.895645in}{1.437457in}}{\pgfqpoint{0.895645in}{1.445694in}}%
\pgfpathcurveto{\pgfqpoint{0.895645in}{1.453930in}}{\pgfqpoint{0.892372in}{1.461830in}}{\pgfqpoint{0.886548in}{1.467654in}}%
\pgfpathcurveto{\pgfqpoint{0.880725in}{1.473478in}}{\pgfqpoint{0.872825in}{1.476750in}}{\pgfqpoint{0.864588in}{1.476750in}}%
\pgfpathcurveto{\pgfqpoint{0.856352in}{1.476750in}}{\pgfqpoint{0.848452in}{1.473478in}}{\pgfqpoint{0.842628in}{1.467654in}}%
\pgfpathcurveto{\pgfqpoint{0.836804in}{1.461830in}}{\pgfqpoint{0.833532in}{1.453930in}}{\pgfqpoint{0.833532in}{1.445694in}}%
\pgfpathcurveto{\pgfqpoint{0.833532in}{1.437457in}}{\pgfqpoint{0.836804in}{1.429557in}}{\pgfqpoint{0.842628in}{1.423733in}}%
\pgfpathcurveto{\pgfqpoint{0.848452in}{1.417909in}}{\pgfqpoint{0.856352in}{1.414637in}}{\pgfqpoint{0.864588in}{1.414637in}}%
\pgfpathclose%
\pgfusepath{stroke,fill}%
\end{pgfscope}%
\begin{pgfscope}%
\pgfpathrectangle{\pgfqpoint{0.100000in}{0.212622in}}{\pgfqpoint{3.696000in}{3.696000in}}%
\pgfusepath{clip}%
\pgfsetbuttcap%
\pgfsetroundjoin%
\definecolor{currentfill}{rgb}{0.121569,0.466667,0.705882}%
\pgfsetfillcolor{currentfill}%
\pgfsetfillopacity{0.657340}%
\pgfsetlinewidth{1.003750pt}%
\definecolor{currentstroke}{rgb}{0.121569,0.466667,0.705882}%
\pgfsetstrokecolor{currentstroke}%
\pgfsetstrokeopacity{0.657340}%
\pgfsetdash{}{0pt}%
\pgfpathmoveto{\pgfqpoint{0.864588in}{1.414637in}}%
\pgfpathcurveto{\pgfqpoint{0.872825in}{1.414637in}}{\pgfqpoint{0.880725in}{1.417909in}}{\pgfqpoint{0.886548in}{1.423733in}}%
\pgfpathcurveto{\pgfqpoint{0.892372in}{1.429557in}}{\pgfqpoint{0.895645in}{1.437457in}}{\pgfqpoint{0.895645in}{1.445694in}}%
\pgfpathcurveto{\pgfqpoint{0.895645in}{1.453930in}}{\pgfqpoint{0.892372in}{1.461830in}}{\pgfqpoint{0.886548in}{1.467654in}}%
\pgfpathcurveto{\pgfqpoint{0.880725in}{1.473478in}}{\pgfqpoint{0.872825in}{1.476750in}}{\pgfqpoint{0.864588in}{1.476750in}}%
\pgfpathcurveto{\pgfqpoint{0.856352in}{1.476750in}}{\pgfqpoint{0.848452in}{1.473478in}}{\pgfqpoint{0.842628in}{1.467654in}}%
\pgfpathcurveto{\pgfqpoint{0.836804in}{1.461830in}}{\pgfqpoint{0.833532in}{1.453930in}}{\pgfqpoint{0.833532in}{1.445694in}}%
\pgfpathcurveto{\pgfqpoint{0.833532in}{1.437457in}}{\pgfqpoint{0.836804in}{1.429557in}}{\pgfqpoint{0.842628in}{1.423733in}}%
\pgfpathcurveto{\pgfqpoint{0.848452in}{1.417909in}}{\pgfqpoint{0.856352in}{1.414637in}}{\pgfqpoint{0.864588in}{1.414637in}}%
\pgfpathclose%
\pgfusepath{stroke,fill}%
\end{pgfscope}%
\begin{pgfscope}%
\pgfpathrectangle{\pgfqpoint{0.100000in}{0.212622in}}{\pgfqpoint{3.696000in}{3.696000in}}%
\pgfusepath{clip}%
\pgfsetbuttcap%
\pgfsetroundjoin%
\definecolor{currentfill}{rgb}{0.121569,0.466667,0.705882}%
\pgfsetfillcolor{currentfill}%
\pgfsetfillopacity{0.657340}%
\pgfsetlinewidth{1.003750pt}%
\definecolor{currentstroke}{rgb}{0.121569,0.466667,0.705882}%
\pgfsetstrokecolor{currentstroke}%
\pgfsetstrokeopacity{0.657340}%
\pgfsetdash{}{0pt}%
\pgfpathmoveto{\pgfqpoint{0.864588in}{1.414637in}}%
\pgfpathcurveto{\pgfqpoint{0.872825in}{1.414637in}}{\pgfqpoint{0.880725in}{1.417909in}}{\pgfqpoint{0.886548in}{1.423733in}}%
\pgfpathcurveto{\pgfqpoint{0.892372in}{1.429557in}}{\pgfqpoint{0.895645in}{1.437457in}}{\pgfqpoint{0.895645in}{1.445694in}}%
\pgfpathcurveto{\pgfqpoint{0.895645in}{1.453930in}}{\pgfqpoint{0.892372in}{1.461830in}}{\pgfqpoint{0.886548in}{1.467654in}}%
\pgfpathcurveto{\pgfqpoint{0.880725in}{1.473478in}}{\pgfqpoint{0.872825in}{1.476750in}}{\pgfqpoint{0.864588in}{1.476750in}}%
\pgfpathcurveto{\pgfqpoint{0.856352in}{1.476750in}}{\pgfqpoint{0.848452in}{1.473478in}}{\pgfqpoint{0.842628in}{1.467654in}}%
\pgfpathcurveto{\pgfqpoint{0.836804in}{1.461830in}}{\pgfqpoint{0.833532in}{1.453930in}}{\pgfqpoint{0.833532in}{1.445694in}}%
\pgfpathcurveto{\pgfqpoint{0.833532in}{1.437457in}}{\pgfqpoint{0.836804in}{1.429557in}}{\pgfqpoint{0.842628in}{1.423733in}}%
\pgfpathcurveto{\pgfqpoint{0.848452in}{1.417909in}}{\pgfqpoint{0.856352in}{1.414637in}}{\pgfqpoint{0.864588in}{1.414637in}}%
\pgfpathclose%
\pgfusepath{stroke,fill}%
\end{pgfscope}%
\begin{pgfscope}%
\pgfpathrectangle{\pgfqpoint{0.100000in}{0.212622in}}{\pgfqpoint{3.696000in}{3.696000in}}%
\pgfusepath{clip}%
\pgfsetbuttcap%
\pgfsetroundjoin%
\definecolor{currentfill}{rgb}{0.121569,0.466667,0.705882}%
\pgfsetfillcolor{currentfill}%
\pgfsetfillopacity{0.657340}%
\pgfsetlinewidth{1.003750pt}%
\definecolor{currentstroke}{rgb}{0.121569,0.466667,0.705882}%
\pgfsetstrokecolor{currentstroke}%
\pgfsetstrokeopacity{0.657340}%
\pgfsetdash{}{0pt}%
\pgfpathmoveto{\pgfqpoint{0.864588in}{1.414637in}}%
\pgfpathcurveto{\pgfqpoint{0.872825in}{1.414637in}}{\pgfqpoint{0.880725in}{1.417909in}}{\pgfqpoint{0.886548in}{1.423733in}}%
\pgfpathcurveto{\pgfqpoint{0.892372in}{1.429557in}}{\pgfqpoint{0.895645in}{1.437457in}}{\pgfqpoint{0.895645in}{1.445694in}}%
\pgfpathcurveto{\pgfqpoint{0.895645in}{1.453930in}}{\pgfqpoint{0.892372in}{1.461830in}}{\pgfqpoint{0.886548in}{1.467654in}}%
\pgfpathcurveto{\pgfqpoint{0.880725in}{1.473478in}}{\pgfqpoint{0.872825in}{1.476750in}}{\pgfqpoint{0.864588in}{1.476750in}}%
\pgfpathcurveto{\pgfqpoint{0.856352in}{1.476750in}}{\pgfqpoint{0.848452in}{1.473478in}}{\pgfqpoint{0.842628in}{1.467654in}}%
\pgfpathcurveto{\pgfqpoint{0.836804in}{1.461830in}}{\pgfqpoint{0.833532in}{1.453930in}}{\pgfqpoint{0.833532in}{1.445694in}}%
\pgfpathcurveto{\pgfqpoint{0.833532in}{1.437457in}}{\pgfqpoint{0.836804in}{1.429557in}}{\pgfqpoint{0.842628in}{1.423733in}}%
\pgfpathcurveto{\pgfqpoint{0.848452in}{1.417909in}}{\pgfqpoint{0.856352in}{1.414637in}}{\pgfqpoint{0.864588in}{1.414637in}}%
\pgfpathclose%
\pgfusepath{stroke,fill}%
\end{pgfscope}%
\begin{pgfscope}%
\pgfpathrectangle{\pgfqpoint{0.100000in}{0.212622in}}{\pgfqpoint{3.696000in}{3.696000in}}%
\pgfusepath{clip}%
\pgfsetbuttcap%
\pgfsetroundjoin%
\definecolor{currentfill}{rgb}{0.121569,0.466667,0.705882}%
\pgfsetfillcolor{currentfill}%
\pgfsetfillopacity{0.657340}%
\pgfsetlinewidth{1.003750pt}%
\definecolor{currentstroke}{rgb}{0.121569,0.466667,0.705882}%
\pgfsetstrokecolor{currentstroke}%
\pgfsetstrokeopacity{0.657340}%
\pgfsetdash{}{0pt}%
\pgfpathmoveto{\pgfqpoint{0.864588in}{1.414637in}}%
\pgfpathcurveto{\pgfqpoint{0.872825in}{1.414637in}}{\pgfqpoint{0.880725in}{1.417909in}}{\pgfqpoint{0.886548in}{1.423733in}}%
\pgfpathcurveto{\pgfqpoint{0.892372in}{1.429557in}}{\pgfqpoint{0.895645in}{1.437457in}}{\pgfqpoint{0.895645in}{1.445694in}}%
\pgfpathcurveto{\pgfqpoint{0.895645in}{1.453930in}}{\pgfqpoint{0.892372in}{1.461830in}}{\pgfqpoint{0.886548in}{1.467654in}}%
\pgfpathcurveto{\pgfqpoint{0.880725in}{1.473478in}}{\pgfqpoint{0.872825in}{1.476750in}}{\pgfqpoint{0.864588in}{1.476750in}}%
\pgfpathcurveto{\pgfqpoint{0.856352in}{1.476750in}}{\pgfqpoint{0.848452in}{1.473478in}}{\pgfqpoint{0.842628in}{1.467654in}}%
\pgfpathcurveto{\pgfqpoint{0.836804in}{1.461830in}}{\pgfqpoint{0.833532in}{1.453930in}}{\pgfqpoint{0.833532in}{1.445694in}}%
\pgfpathcurveto{\pgfqpoint{0.833532in}{1.437457in}}{\pgfqpoint{0.836804in}{1.429557in}}{\pgfqpoint{0.842628in}{1.423733in}}%
\pgfpathcurveto{\pgfqpoint{0.848452in}{1.417909in}}{\pgfqpoint{0.856352in}{1.414637in}}{\pgfqpoint{0.864588in}{1.414637in}}%
\pgfpathclose%
\pgfusepath{stroke,fill}%
\end{pgfscope}%
\begin{pgfscope}%
\pgfpathrectangle{\pgfqpoint{0.100000in}{0.212622in}}{\pgfqpoint{3.696000in}{3.696000in}}%
\pgfusepath{clip}%
\pgfsetbuttcap%
\pgfsetroundjoin%
\definecolor{currentfill}{rgb}{0.121569,0.466667,0.705882}%
\pgfsetfillcolor{currentfill}%
\pgfsetfillopacity{0.657340}%
\pgfsetlinewidth{1.003750pt}%
\definecolor{currentstroke}{rgb}{0.121569,0.466667,0.705882}%
\pgfsetstrokecolor{currentstroke}%
\pgfsetstrokeopacity{0.657340}%
\pgfsetdash{}{0pt}%
\pgfpathmoveto{\pgfqpoint{0.864588in}{1.414637in}}%
\pgfpathcurveto{\pgfqpoint{0.872825in}{1.414637in}}{\pgfqpoint{0.880725in}{1.417909in}}{\pgfqpoint{0.886548in}{1.423733in}}%
\pgfpathcurveto{\pgfqpoint{0.892372in}{1.429557in}}{\pgfqpoint{0.895645in}{1.437457in}}{\pgfqpoint{0.895645in}{1.445694in}}%
\pgfpathcurveto{\pgfqpoint{0.895645in}{1.453930in}}{\pgfqpoint{0.892372in}{1.461830in}}{\pgfqpoint{0.886548in}{1.467654in}}%
\pgfpathcurveto{\pgfqpoint{0.880725in}{1.473478in}}{\pgfqpoint{0.872825in}{1.476750in}}{\pgfqpoint{0.864588in}{1.476750in}}%
\pgfpathcurveto{\pgfqpoint{0.856352in}{1.476750in}}{\pgfqpoint{0.848452in}{1.473478in}}{\pgfqpoint{0.842628in}{1.467654in}}%
\pgfpathcurveto{\pgfqpoint{0.836804in}{1.461830in}}{\pgfqpoint{0.833532in}{1.453930in}}{\pgfqpoint{0.833532in}{1.445694in}}%
\pgfpathcurveto{\pgfqpoint{0.833532in}{1.437457in}}{\pgfqpoint{0.836804in}{1.429557in}}{\pgfqpoint{0.842628in}{1.423733in}}%
\pgfpathcurveto{\pgfqpoint{0.848452in}{1.417909in}}{\pgfqpoint{0.856352in}{1.414637in}}{\pgfqpoint{0.864588in}{1.414637in}}%
\pgfpathclose%
\pgfusepath{stroke,fill}%
\end{pgfscope}%
\begin{pgfscope}%
\pgfpathrectangle{\pgfqpoint{0.100000in}{0.212622in}}{\pgfqpoint{3.696000in}{3.696000in}}%
\pgfusepath{clip}%
\pgfsetbuttcap%
\pgfsetroundjoin%
\definecolor{currentfill}{rgb}{0.121569,0.466667,0.705882}%
\pgfsetfillcolor{currentfill}%
\pgfsetfillopacity{0.657340}%
\pgfsetlinewidth{1.003750pt}%
\definecolor{currentstroke}{rgb}{0.121569,0.466667,0.705882}%
\pgfsetstrokecolor{currentstroke}%
\pgfsetstrokeopacity{0.657340}%
\pgfsetdash{}{0pt}%
\pgfpathmoveto{\pgfqpoint{0.864588in}{1.414637in}}%
\pgfpathcurveto{\pgfqpoint{0.872825in}{1.414637in}}{\pgfqpoint{0.880725in}{1.417909in}}{\pgfqpoint{0.886548in}{1.423733in}}%
\pgfpathcurveto{\pgfqpoint{0.892372in}{1.429557in}}{\pgfqpoint{0.895645in}{1.437457in}}{\pgfqpoint{0.895645in}{1.445694in}}%
\pgfpathcurveto{\pgfqpoint{0.895645in}{1.453930in}}{\pgfqpoint{0.892372in}{1.461830in}}{\pgfqpoint{0.886548in}{1.467654in}}%
\pgfpathcurveto{\pgfqpoint{0.880725in}{1.473478in}}{\pgfqpoint{0.872825in}{1.476750in}}{\pgfqpoint{0.864588in}{1.476750in}}%
\pgfpathcurveto{\pgfqpoint{0.856352in}{1.476750in}}{\pgfqpoint{0.848452in}{1.473478in}}{\pgfqpoint{0.842628in}{1.467654in}}%
\pgfpathcurveto{\pgfqpoint{0.836804in}{1.461830in}}{\pgfqpoint{0.833532in}{1.453930in}}{\pgfqpoint{0.833532in}{1.445694in}}%
\pgfpathcurveto{\pgfqpoint{0.833532in}{1.437457in}}{\pgfqpoint{0.836804in}{1.429557in}}{\pgfqpoint{0.842628in}{1.423733in}}%
\pgfpathcurveto{\pgfqpoint{0.848452in}{1.417909in}}{\pgfqpoint{0.856352in}{1.414637in}}{\pgfqpoint{0.864588in}{1.414637in}}%
\pgfpathclose%
\pgfusepath{stroke,fill}%
\end{pgfscope}%
\begin{pgfscope}%
\pgfpathrectangle{\pgfqpoint{0.100000in}{0.212622in}}{\pgfqpoint{3.696000in}{3.696000in}}%
\pgfusepath{clip}%
\pgfsetbuttcap%
\pgfsetroundjoin%
\definecolor{currentfill}{rgb}{0.121569,0.466667,0.705882}%
\pgfsetfillcolor{currentfill}%
\pgfsetfillopacity{0.657340}%
\pgfsetlinewidth{1.003750pt}%
\definecolor{currentstroke}{rgb}{0.121569,0.466667,0.705882}%
\pgfsetstrokecolor{currentstroke}%
\pgfsetstrokeopacity{0.657340}%
\pgfsetdash{}{0pt}%
\pgfpathmoveto{\pgfqpoint{0.864588in}{1.414637in}}%
\pgfpathcurveto{\pgfqpoint{0.872825in}{1.414637in}}{\pgfqpoint{0.880725in}{1.417909in}}{\pgfqpoint{0.886548in}{1.423733in}}%
\pgfpathcurveto{\pgfqpoint{0.892372in}{1.429557in}}{\pgfqpoint{0.895645in}{1.437457in}}{\pgfqpoint{0.895645in}{1.445694in}}%
\pgfpathcurveto{\pgfqpoint{0.895645in}{1.453930in}}{\pgfqpoint{0.892372in}{1.461830in}}{\pgfqpoint{0.886548in}{1.467654in}}%
\pgfpathcurveto{\pgfqpoint{0.880725in}{1.473478in}}{\pgfqpoint{0.872825in}{1.476750in}}{\pgfqpoint{0.864588in}{1.476750in}}%
\pgfpathcurveto{\pgfqpoint{0.856352in}{1.476750in}}{\pgfqpoint{0.848452in}{1.473478in}}{\pgfqpoint{0.842628in}{1.467654in}}%
\pgfpathcurveto{\pgfqpoint{0.836804in}{1.461830in}}{\pgfqpoint{0.833532in}{1.453930in}}{\pgfqpoint{0.833532in}{1.445694in}}%
\pgfpathcurveto{\pgfqpoint{0.833532in}{1.437457in}}{\pgfqpoint{0.836804in}{1.429557in}}{\pgfqpoint{0.842628in}{1.423733in}}%
\pgfpathcurveto{\pgfqpoint{0.848452in}{1.417909in}}{\pgfqpoint{0.856352in}{1.414637in}}{\pgfqpoint{0.864588in}{1.414637in}}%
\pgfpathclose%
\pgfusepath{stroke,fill}%
\end{pgfscope}%
\begin{pgfscope}%
\pgfpathrectangle{\pgfqpoint{0.100000in}{0.212622in}}{\pgfqpoint{3.696000in}{3.696000in}}%
\pgfusepath{clip}%
\pgfsetbuttcap%
\pgfsetroundjoin%
\definecolor{currentfill}{rgb}{0.121569,0.466667,0.705882}%
\pgfsetfillcolor{currentfill}%
\pgfsetfillopacity{0.657340}%
\pgfsetlinewidth{1.003750pt}%
\definecolor{currentstroke}{rgb}{0.121569,0.466667,0.705882}%
\pgfsetstrokecolor{currentstroke}%
\pgfsetstrokeopacity{0.657340}%
\pgfsetdash{}{0pt}%
\pgfpathmoveto{\pgfqpoint{0.864588in}{1.414637in}}%
\pgfpathcurveto{\pgfqpoint{0.872825in}{1.414637in}}{\pgfqpoint{0.880725in}{1.417909in}}{\pgfqpoint{0.886548in}{1.423733in}}%
\pgfpathcurveto{\pgfqpoint{0.892372in}{1.429557in}}{\pgfqpoint{0.895645in}{1.437457in}}{\pgfqpoint{0.895645in}{1.445694in}}%
\pgfpathcurveto{\pgfqpoint{0.895645in}{1.453930in}}{\pgfqpoint{0.892372in}{1.461830in}}{\pgfqpoint{0.886548in}{1.467654in}}%
\pgfpathcurveto{\pgfqpoint{0.880725in}{1.473478in}}{\pgfqpoint{0.872825in}{1.476750in}}{\pgfqpoint{0.864588in}{1.476750in}}%
\pgfpathcurveto{\pgfqpoint{0.856352in}{1.476750in}}{\pgfqpoint{0.848452in}{1.473478in}}{\pgfqpoint{0.842628in}{1.467654in}}%
\pgfpathcurveto{\pgfqpoint{0.836804in}{1.461830in}}{\pgfqpoint{0.833532in}{1.453930in}}{\pgfqpoint{0.833532in}{1.445694in}}%
\pgfpathcurveto{\pgfqpoint{0.833532in}{1.437457in}}{\pgfqpoint{0.836804in}{1.429557in}}{\pgfqpoint{0.842628in}{1.423733in}}%
\pgfpathcurveto{\pgfqpoint{0.848452in}{1.417909in}}{\pgfqpoint{0.856352in}{1.414637in}}{\pgfqpoint{0.864588in}{1.414637in}}%
\pgfpathclose%
\pgfusepath{stroke,fill}%
\end{pgfscope}%
\begin{pgfscope}%
\pgfpathrectangle{\pgfqpoint{0.100000in}{0.212622in}}{\pgfqpoint{3.696000in}{3.696000in}}%
\pgfusepath{clip}%
\pgfsetbuttcap%
\pgfsetroundjoin%
\definecolor{currentfill}{rgb}{0.121569,0.466667,0.705882}%
\pgfsetfillcolor{currentfill}%
\pgfsetfillopacity{0.657340}%
\pgfsetlinewidth{1.003750pt}%
\definecolor{currentstroke}{rgb}{0.121569,0.466667,0.705882}%
\pgfsetstrokecolor{currentstroke}%
\pgfsetstrokeopacity{0.657340}%
\pgfsetdash{}{0pt}%
\pgfpathmoveto{\pgfqpoint{0.864588in}{1.414637in}}%
\pgfpathcurveto{\pgfqpoint{0.872825in}{1.414637in}}{\pgfqpoint{0.880725in}{1.417909in}}{\pgfqpoint{0.886548in}{1.423733in}}%
\pgfpathcurveto{\pgfqpoint{0.892372in}{1.429557in}}{\pgfqpoint{0.895645in}{1.437457in}}{\pgfqpoint{0.895645in}{1.445694in}}%
\pgfpathcurveto{\pgfqpoint{0.895645in}{1.453930in}}{\pgfqpoint{0.892372in}{1.461830in}}{\pgfqpoint{0.886548in}{1.467654in}}%
\pgfpathcurveto{\pgfqpoint{0.880725in}{1.473478in}}{\pgfqpoint{0.872825in}{1.476750in}}{\pgfqpoint{0.864588in}{1.476750in}}%
\pgfpathcurveto{\pgfqpoint{0.856352in}{1.476750in}}{\pgfqpoint{0.848452in}{1.473478in}}{\pgfqpoint{0.842628in}{1.467654in}}%
\pgfpathcurveto{\pgfqpoint{0.836804in}{1.461830in}}{\pgfqpoint{0.833532in}{1.453930in}}{\pgfqpoint{0.833532in}{1.445694in}}%
\pgfpathcurveto{\pgfqpoint{0.833532in}{1.437457in}}{\pgfqpoint{0.836804in}{1.429557in}}{\pgfqpoint{0.842628in}{1.423733in}}%
\pgfpathcurveto{\pgfqpoint{0.848452in}{1.417909in}}{\pgfqpoint{0.856352in}{1.414637in}}{\pgfqpoint{0.864588in}{1.414637in}}%
\pgfpathclose%
\pgfusepath{stroke,fill}%
\end{pgfscope}%
\begin{pgfscope}%
\pgfpathrectangle{\pgfqpoint{0.100000in}{0.212622in}}{\pgfqpoint{3.696000in}{3.696000in}}%
\pgfusepath{clip}%
\pgfsetbuttcap%
\pgfsetroundjoin%
\definecolor{currentfill}{rgb}{0.121569,0.466667,0.705882}%
\pgfsetfillcolor{currentfill}%
\pgfsetfillopacity{0.657340}%
\pgfsetlinewidth{1.003750pt}%
\definecolor{currentstroke}{rgb}{0.121569,0.466667,0.705882}%
\pgfsetstrokecolor{currentstroke}%
\pgfsetstrokeopacity{0.657340}%
\pgfsetdash{}{0pt}%
\pgfpathmoveto{\pgfqpoint{0.864588in}{1.414637in}}%
\pgfpathcurveto{\pgfqpoint{0.872825in}{1.414637in}}{\pgfqpoint{0.880725in}{1.417909in}}{\pgfqpoint{0.886548in}{1.423733in}}%
\pgfpathcurveto{\pgfqpoint{0.892372in}{1.429557in}}{\pgfqpoint{0.895645in}{1.437457in}}{\pgfqpoint{0.895645in}{1.445694in}}%
\pgfpathcurveto{\pgfqpoint{0.895645in}{1.453930in}}{\pgfqpoint{0.892372in}{1.461830in}}{\pgfqpoint{0.886548in}{1.467654in}}%
\pgfpathcurveto{\pgfqpoint{0.880725in}{1.473478in}}{\pgfqpoint{0.872825in}{1.476750in}}{\pgfqpoint{0.864588in}{1.476750in}}%
\pgfpathcurveto{\pgfqpoint{0.856352in}{1.476750in}}{\pgfqpoint{0.848452in}{1.473478in}}{\pgfqpoint{0.842628in}{1.467654in}}%
\pgfpathcurveto{\pgfqpoint{0.836804in}{1.461830in}}{\pgfqpoint{0.833532in}{1.453930in}}{\pgfqpoint{0.833532in}{1.445694in}}%
\pgfpathcurveto{\pgfqpoint{0.833532in}{1.437457in}}{\pgfqpoint{0.836804in}{1.429557in}}{\pgfqpoint{0.842628in}{1.423733in}}%
\pgfpathcurveto{\pgfqpoint{0.848452in}{1.417909in}}{\pgfqpoint{0.856352in}{1.414637in}}{\pgfqpoint{0.864588in}{1.414637in}}%
\pgfpathclose%
\pgfusepath{stroke,fill}%
\end{pgfscope}%
\begin{pgfscope}%
\pgfpathrectangle{\pgfqpoint{0.100000in}{0.212622in}}{\pgfqpoint{3.696000in}{3.696000in}}%
\pgfusepath{clip}%
\pgfsetbuttcap%
\pgfsetroundjoin%
\definecolor{currentfill}{rgb}{0.121569,0.466667,0.705882}%
\pgfsetfillcolor{currentfill}%
\pgfsetfillopacity{0.657340}%
\pgfsetlinewidth{1.003750pt}%
\definecolor{currentstroke}{rgb}{0.121569,0.466667,0.705882}%
\pgfsetstrokecolor{currentstroke}%
\pgfsetstrokeopacity{0.657340}%
\pgfsetdash{}{0pt}%
\pgfpathmoveto{\pgfqpoint{0.864588in}{1.414637in}}%
\pgfpathcurveto{\pgfqpoint{0.872825in}{1.414637in}}{\pgfqpoint{0.880725in}{1.417909in}}{\pgfqpoint{0.886548in}{1.423733in}}%
\pgfpathcurveto{\pgfqpoint{0.892372in}{1.429557in}}{\pgfqpoint{0.895645in}{1.437457in}}{\pgfqpoint{0.895645in}{1.445694in}}%
\pgfpathcurveto{\pgfqpoint{0.895645in}{1.453930in}}{\pgfqpoint{0.892372in}{1.461830in}}{\pgfqpoint{0.886548in}{1.467654in}}%
\pgfpathcurveto{\pgfqpoint{0.880725in}{1.473478in}}{\pgfqpoint{0.872825in}{1.476750in}}{\pgfqpoint{0.864588in}{1.476750in}}%
\pgfpathcurveto{\pgfqpoint{0.856352in}{1.476750in}}{\pgfqpoint{0.848452in}{1.473478in}}{\pgfqpoint{0.842628in}{1.467654in}}%
\pgfpathcurveto{\pgfqpoint{0.836804in}{1.461830in}}{\pgfqpoint{0.833532in}{1.453930in}}{\pgfqpoint{0.833532in}{1.445694in}}%
\pgfpathcurveto{\pgfqpoint{0.833532in}{1.437457in}}{\pgfqpoint{0.836804in}{1.429557in}}{\pgfqpoint{0.842628in}{1.423733in}}%
\pgfpathcurveto{\pgfqpoint{0.848452in}{1.417909in}}{\pgfqpoint{0.856352in}{1.414637in}}{\pgfqpoint{0.864588in}{1.414637in}}%
\pgfpathclose%
\pgfusepath{stroke,fill}%
\end{pgfscope}%
\begin{pgfscope}%
\pgfpathrectangle{\pgfqpoint{0.100000in}{0.212622in}}{\pgfqpoint{3.696000in}{3.696000in}}%
\pgfusepath{clip}%
\pgfsetbuttcap%
\pgfsetroundjoin%
\definecolor{currentfill}{rgb}{0.121569,0.466667,0.705882}%
\pgfsetfillcolor{currentfill}%
\pgfsetfillopacity{0.657340}%
\pgfsetlinewidth{1.003750pt}%
\definecolor{currentstroke}{rgb}{0.121569,0.466667,0.705882}%
\pgfsetstrokecolor{currentstroke}%
\pgfsetstrokeopacity{0.657340}%
\pgfsetdash{}{0pt}%
\pgfpathmoveto{\pgfqpoint{0.864588in}{1.414637in}}%
\pgfpathcurveto{\pgfqpoint{0.872825in}{1.414637in}}{\pgfqpoint{0.880725in}{1.417909in}}{\pgfqpoint{0.886548in}{1.423733in}}%
\pgfpathcurveto{\pgfqpoint{0.892372in}{1.429557in}}{\pgfqpoint{0.895645in}{1.437457in}}{\pgfqpoint{0.895645in}{1.445694in}}%
\pgfpathcurveto{\pgfqpoint{0.895645in}{1.453930in}}{\pgfqpoint{0.892372in}{1.461830in}}{\pgfqpoint{0.886548in}{1.467654in}}%
\pgfpathcurveto{\pgfqpoint{0.880725in}{1.473478in}}{\pgfqpoint{0.872825in}{1.476750in}}{\pgfqpoint{0.864588in}{1.476750in}}%
\pgfpathcurveto{\pgfqpoint{0.856352in}{1.476750in}}{\pgfqpoint{0.848452in}{1.473478in}}{\pgfqpoint{0.842628in}{1.467654in}}%
\pgfpathcurveto{\pgfqpoint{0.836804in}{1.461830in}}{\pgfqpoint{0.833532in}{1.453930in}}{\pgfqpoint{0.833532in}{1.445694in}}%
\pgfpathcurveto{\pgfqpoint{0.833532in}{1.437457in}}{\pgfqpoint{0.836804in}{1.429557in}}{\pgfqpoint{0.842628in}{1.423733in}}%
\pgfpathcurveto{\pgfqpoint{0.848452in}{1.417909in}}{\pgfqpoint{0.856352in}{1.414637in}}{\pgfqpoint{0.864588in}{1.414637in}}%
\pgfpathclose%
\pgfusepath{stroke,fill}%
\end{pgfscope}%
\begin{pgfscope}%
\pgfpathrectangle{\pgfqpoint{0.100000in}{0.212622in}}{\pgfqpoint{3.696000in}{3.696000in}}%
\pgfusepath{clip}%
\pgfsetbuttcap%
\pgfsetroundjoin%
\definecolor{currentfill}{rgb}{0.121569,0.466667,0.705882}%
\pgfsetfillcolor{currentfill}%
\pgfsetfillopacity{0.657340}%
\pgfsetlinewidth{1.003750pt}%
\definecolor{currentstroke}{rgb}{0.121569,0.466667,0.705882}%
\pgfsetstrokecolor{currentstroke}%
\pgfsetstrokeopacity{0.657340}%
\pgfsetdash{}{0pt}%
\pgfpathmoveto{\pgfqpoint{0.864588in}{1.414637in}}%
\pgfpathcurveto{\pgfqpoint{0.872825in}{1.414637in}}{\pgfqpoint{0.880725in}{1.417909in}}{\pgfqpoint{0.886548in}{1.423733in}}%
\pgfpathcurveto{\pgfqpoint{0.892372in}{1.429557in}}{\pgfqpoint{0.895645in}{1.437457in}}{\pgfqpoint{0.895645in}{1.445694in}}%
\pgfpathcurveto{\pgfqpoint{0.895645in}{1.453930in}}{\pgfqpoint{0.892372in}{1.461830in}}{\pgfqpoint{0.886548in}{1.467654in}}%
\pgfpathcurveto{\pgfqpoint{0.880725in}{1.473478in}}{\pgfqpoint{0.872825in}{1.476750in}}{\pgfqpoint{0.864588in}{1.476750in}}%
\pgfpathcurveto{\pgfqpoint{0.856352in}{1.476750in}}{\pgfqpoint{0.848452in}{1.473478in}}{\pgfqpoint{0.842628in}{1.467654in}}%
\pgfpathcurveto{\pgfqpoint{0.836804in}{1.461830in}}{\pgfqpoint{0.833532in}{1.453930in}}{\pgfqpoint{0.833532in}{1.445694in}}%
\pgfpathcurveto{\pgfqpoint{0.833532in}{1.437457in}}{\pgfqpoint{0.836804in}{1.429557in}}{\pgfqpoint{0.842628in}{1.423733in}}%
\pgfpathcurveto{\pgfqpoint{0.848452in}{1.417909in}}{\pgfqpoint{0.856352in}{1.414637in}}{\pgfqpoint{0.864588in}{1.414637in}}%
\pgfpathclose%
\pgfusepath{stroke,fill}%
\end{pgfscope}%
\begin{pgfscope}%
\pgfpathrectangle{\pgfqpoint{0.100000in}{0.212622in}}{\pgfqpoint{3.696000in}{3.696000in}}%
\pgfusepath{clip}%
\pgfsetbuttcap%
\pgfsetroundjoin%
\definecolor{currentfill}{rgb}{0.121569,0.466667,0.705882}%
\pgfsetfillcolor{currentfill}%
\pgfsetfillopacity{0.657340}%
\pgfsetlinewidth{1.003750pt}%
\definecolor{currentstroke}{rgb}{0.121569,0.466667,0.705882}%
\pgfsetstrokecolor{currentstroke}%
\pgfsetstrokeopacity{0.657340}%
\pgfsetdash{}{0pt}%
\pgfpathmoveto{\pgfqpoint{0.864588in}{1.414637in}}%
\pgfpathcurveto{\pgfqpoint{0.872825in}{1.414637in}}{\pgfqpoint{0.880725in}{1.417909in}}{\pgfqpoint{0.886548in}{1.423733in}}%
\pgfpathcurveto{\pgfqpoint{0.892372in}{1.429557in}}{\pgfqpoint{0.895645in}{1.437457in}}{\pgfqpoint{0.895645in}{1.445694in}}%
\pgfpathcurveto{\pgfqpoint{0.895645in}{1.453930in}}{\pgfqpoint{0.892372in}{1.461830in}}{\pgfqpoint{0.886548in}{1.467654in}}%
\pgfpathcurveto{\pgfqpoint{0.880725in}{1.473478in}}{\pgfqpoint{0.872825in}{1.476750in}}{\pgfqpoint{0.864588in}{1.476750in}}%
\pgfpathcurveto{\pgfqpoint{0.856352in}{1.476750in}}{\pgfqpoint{0.848452in}{1.473478in}}{\pgfqpoint{0.842628in}{1.467654in}}%
\pgfpathcurveto{\pgfqpoint{0.836804in}{1.461830in}}{\pgfqpoint{0.833532in}{1.453930in}}{\pgfqpoint{0.833532in}{1.445694in}}%
\pgfpathcurveto{\pgfqpoint{0.833532in}{1.437457in}}{\pgfqpoint{0.836804in}{1.429557in}}{\pgfqpoint{0.842628in}{1.423733in}}%
\pgfpathcurveto{\pgfqpoint{0.848452in}{1.417909in}}{\pgfqpoint{0.856352in}{1.414637in}}{\pgfqpoint{0.864588in}{1.414637in}}%
\pgfpathclose%
\pgfusepath{stroke,fill}%
\end{pgfscope}%
\begin{pgfscope}%
\pgfpathrectangle{\pgfqpoint{0.100000in}{0.212622in}}{\pgfqpoint{3.696000in}{3.696000in}}%
\pgfusepath{clip}%
\pgfsetbuttcap%
\pgfsetroundjoin%
\definecolor{currentfill}{rgb}{0.121569,0.466667,0.705882}%
\pgfsetfillcolor{currentfill}%
\pgfsetfillopacity{0.657340}%
\pgfsetlinewidth{1.003750pt}%
\definecolor{currentstroke}{rgb}{0.121569,0.466667,0.705882}%
\pgfsetstrokecolor{currentstroke}%
\pgfsetstrokeopacity{0.657340}%
\pgfsetdash{}{0pt}%
\pgfpathmoveto{\pgfqpoint{0.864588in}{1.414637in}}%
\pgfpathcurveto{\pgfqpoint{0.872825in}{1.414637in}}{\pgfqpoint{0.880725in}{1.417909in}}{\pgfqpoint{0.886548in}{1.423733in}}%
\pgfpathcurveto{\pgfqpoint{0.892372in}{1.429557in}}{\pgfqpoint{0.895645in}{1.437457in}}{\pgfqpoint{0.895645in}{1.445694in}}%
\pgfpathcurveto{\pgfqpoint{0.895645in}{1.453930in}}{\pgfqpoint{0.892372in}{1.461830in}}{\pgfqpoint{0.886548in}{1.467654in}}%
\pgfpathcurveto{\pgfqpoint{0.880725in}{1.473478in}}{\pgfqpoint{0.872825in}{1.476750in}}{\pgfqpoint{0.864588in}{1.476750in}}%
\pgfpathcurveto{\pgfqpoint{0.856352in}{1.476750in}}{\pgfqpoint{0.848452in}{1.473478in}}{\pgfqpoint{0.842628in}{1.467654in}}%
\pgfpathcurveto{\pgfqpoint{0.836804in}{1.461830in}}{\pgfqpoint{0.833532in}{1.453930in}}{\pgfqpoint{0.833532in}{1.445694in}}%
\pgfpathcurveto{\pgfqpoint{0.833532in}{1.437457in}}{\pgfqpoint{0.836804in}{1.429557in}}{\pgfqpoint{0.842628in}{1.423733in}}%
\pgfpathcurveto{\pgfqpoint{0.848452in}{1.417909in}}{\pgfqpoint{0.856352in}{1.414637in}}{\pgfqpoint{0.864588in}{1.414637in}}%
\pgfpathclose%
\pgfusepath{stroke,fill}%
\end{pgfscope}%
\begin{pgfscope}%
\pgfpathrectangle{\pgfqpoint{0.100000in}{0.212622in}}{\pgfqpoint{3.696000in}{3.696000in}}%
\pgfusepath{clip}%
\pgfsetbuttcap%
\pgfsetroundjoin%
\definecolor{currentfill}{rgb}{0.121569,0.466667,0.705882}%
\pgfsetfillcolor{currentfill}%
\pgfsetfillopacity{0.657340}%
\pgfsetlinewidth{1.003750pt}%
\definecolor{currentstroke}{rgb}{0.121569,0.466667,0.705882}%
\pgfsetstrokecolor{currentstroke}%
\pgfsetstrokeopacity{0.657340}%
\pgfsetdash{}{0pt}%
\pgfpathmoveto{\pgfqpoint{0.864588in}{1.414637in}}%
\pgfpathcurveto{\pgfqpoint{0.872825in}{1.414637in}}{\pgfqpoint{0.880725in}{1.417909in}}{\pgfqpoint{0.886548in}{1.423733in}}%
\pgfpathcurveto{\pgfqpoint{0.892372in}{1.429557in}}{\pgfqpoint{0.895645in}{1.437457in}}{\pgfqpoint{0.895645in}{1.445694in}}%
\pgfpathcurveto{\pgfqpoint{0.895645in}{1.453930in}}{\pgfqpoint{0.892372in}{1.461830in}}{\pgfqpoint{0.886548in}{1.467654in}}%
\pgfpathcurveto{\pgfqpoint{0.880725in}{1.473478in}}{\pgfqpoint{0.872825in}{1.476750in}}{\pgfqpoint{0.864588in}{1.476750in}}%
\pgfpathcurveto{\pgfqpoint{0.856352in}{1.476750in}}{\pgfqpoint{0.848452in}{1.473478in}}{\pgfqpoint{0.842628in}{1.467654in}}%
\pgfpathcurveto{\pgfqpoint{0.836804in}{1.461830in}}{\pgfqpoint{0.833532in}{1.453930in}}{\pgfqpoint{0.833532in}{1.445694in}}%
\pgfpathcurveto{\pgfqpoint{0.833532in}{1.437457in}}{\pgfqpoint{0.836804in}{1.429557in}}{\pgfqpoint{0.842628in}{1.423733in}}%
\pgfpathcurveto{\pgfqpoint{0.848452in}{1.417909in}}{\pgfqpoint{0.856352in}{1.414637in}}{\pgfqpoint{0.864588in}{1.414637in}}%
\pgfpathclose%
\pgfusepath{stroke,fill}%
\end{pgfscope}%
\begin{pgfscope}%
\pgfpathrectangle{\pgfqpoint{0.100000in}{0.212622in}}{\pgfqpoint{3.696000in}{3.696000in}}%
\pgfusepath{clip}%
\pgfsetbuttcap%
\pgfsetroundjoin%
\definecolor{currentfill}{rgb}{0.121569,0.466667,0.705882}%
\pgfsetfillcolor{currentfill}%
\pgfsetfillopacity{0.657340}%
\pgfsetlinewidth{1.003750pt}%
\definecolor{currentstroke}{rgb}{0.121569,0.466667,0.705882}%
\pgfsetstrokecolor{currentstroke}%
\pgfsetstrokeopacity{0.657340}%
\pgfsetdash{}{0pt}%
\pgfpathmoveto{\pgfqpoint{0.864588in}{1.414637in}}%
\pgfpathcurveto{\pgfqpoint{0.872825in}{1.414637in}}{\pgfqpoint{0.880725in}{1.417909in}}{\pgfqpoint{0.886548in}{1.423733in}}%
\pgfpathcurveto{\pgfqpoint{0.892372in}{1.429557in}}{\pgfqpoint{0.895645in}{1.437457in}}{\pgfqpoint{0.895645in}{1.445694in}}%
\pgfpathcurveto{\pgfqpoint{0.895645in}{1.453930in}}{\pgfqpoint{0.892372in}{1.461830in}}{\pgfqpoint{0.886548in}{1.467654in}}%
\pgfpathcurveto{\pgfqpoint{0.880725in}{1.473478in}}{\pgfqpoint{0.872825in}{1.476750in}}{\pgfqpoint{0.864588in}{1.476750in}}%
\pgfpathcurveto{\pgfqpoint{0.856352in}{1.476750in}}{\pgfqpoint{0.848452in}{1.473478in}}{\pgfqpoint{0.842628in}{1.467654in}}%
\pgfpathcurveto{\pgfqpoint{0.836804in}{1.461830in}}{\pgfqpoint{0.833532in}{1.453930in}}{\pgfqpoint{0.833532in}{1.445694in}}%
\pgfpathcurveto{\pgfqpoint{0.833532in}{1.437457in}}{\pgfqpoint{0.836804in}{1.429557in}}{\pgfqpoint{0.842628in}{1.423733in}}%
\pgfpathcurveto{\pgfqpoint{0.848452in}{1.417909in}}{\pgfqpoint{0.856352in}{1.414637in}}{\pgfqpoint{0.864588in}{1.414637in}}%
\pgfpathclose%
\pgfusepath{stroke,fill}%
\end{pgfscope}%
\begin{pgfscope}%
\pgfpathrectangle{\pgfqpoint{0.100000in}{0.212622in}}{\pgfqpoint{3.696000in}{3.696000in}}%
\pgfusepath{clip}%
\pgfsetbuttcap%
\pgfsetroundjoin%
\definecolor{currentfill}{rgb}{0.121569,0.466667,0.705882}%
\pgfsetfillcolor{currentfill}%
\pgfsetfillopacity{0.657340}%
\pgfsetlinewidth{1.003750pt}%
\definecolor{currentstroke}{rgb}{0.121569,0.466667,0.705882}%
\pgfsetstrokecolor{currentstroke}%
\pgfsetstrokeopacity{0.657340}%
\pgfsetdash{}{0pt}%
\pgfpathmoveto{\pgfqpoint{0.864588in}{1.414637in}}%
\pgfpathcurveto{\pgfqpoint{0.872825in}{1.414637in}}{\pgfqpoint{0.880725in}{1.417909in}}{\pgfqpoint{0.886548in}{1.423733in}}%
\pgfpathcurveto{\pgfqpoint{0.892372in}{1.429557in}}{\pgfqpoint{0.895645in}{1.437457in}}{\pgfqpoint{0.895645in}{1.445694in}}%
\pgfpathcurveto{\pgfqpoint{0.895645in}{1.453930in}}{\pgfqpoint{0.892372in}{1.461830in}}{\pgfqpoint{0.886548in}{1.467654in}}%
\pgfpathcurveto{\pgfqpoint{0.880725in}{1.473478in}}{\pgfqpoint{0.872825in}{1.476750in}}{\pgfqpoint{0.864588in}{1.476750in}}%
\pgfpathcurveto{\pgfqpoint{0.856352in}{1.476750in}}{\pgfqpoint{0.848452in}{1.473478in}}{\pgfqpoint{0.842628in}{1.467654in}}%
\pgfpathcurveto{\pgfqpoint{0.836804in}{1.461830in}}{\pgfqpoint{0.833532in}{1.453930in}}{\pgfqpoint{0.833532in}{1.445694in}}%
\pgfpathcurveto{\pgfqpoint{0.833532in}{1.437457in}}{\pgfqpoint{0.836804in}{1.429557in}}{\pgfqpoint{0.842628in}{1.423733in}}%
\pgfpathcurveto{\pgfqpoint{0.848452in}{1.417909in}}{\pgfqpoint{0.856352in}{1.414637in}}{\pgfqpoint{0.864588in}{1.414637in}}%
\pgfpathclose%
\pgfusepath{stroke,fill}%
\end{pgfscope}%
\begin{pgfscope}%
\pgfpathrectangle{\pgfqpoint{0.100000in}{0.212622in}}{\pgfqpoint{3.696000in}{3.696000in}}%
\pgfusepath{clip}%
\pgfsetbuttcap%
\pgfsetroundjoin%
\definecolor{currentfill}{rgb}{0.121569,0.466667,0.705882}%
\pgfsetfillcolor{currentfill}%
\pgfsetfillopacity{0.657340}%
\pgfsetlinewidth{1.003750pt}%
\definecolor{currentstroke}{rgb}{0.121569,0.466667,0.705882}%
\pgfsetstrokecolor{currentstroke}%
\pgfsetstrokeopacity{0.657340}%
\pgfsetdash{}{0pt}%
\pgfpathmoveto{\pgfqpoint{0.864588in}{1.414637in}}%
\pgfpathcurveto{\pgfqpoint{0.872825in}{1.414637in}}{\pgfqpoint{0.880725in}{1.417909in}}{\pgfqpoint{0.886548in}{1.423733in}}%
\pgfpathcurveto{\pgfqpoint{0.892372in}{1.429557in}}{\pgfqpoint{0.895645in}{1.437457in}}{\pgfqpoint{0.895645in}{1.445694in}}%
\pgfpathcurveto{\pgfqpoint{0.895645in}{1.453930in}}{\pgfqpoint{0.892372in}{1.461830in}}{\pgfqpoint{0.886548in}{1.467654in}}%
\pgfpathcurveto{\pgfqpoint{0.880725in}{1.473478in}}{\pgfqpoint{0.872825in}{1.476750in}}{\pgfqpoint{0.864588in}{1.476750in}}%
\pgfpathcurveto{\pgfqpoint{0.856352in}{1.476750in}}{\pgfqpoint{0.848452in}{1.473478in}}{\pgfqpoint{0.842628in}{1.467654in}}%
\pgfpathcurveto{\pgfqpoint{0.836804in}{1.461830in}}{\pgfqpoint{0.833532in}{1.453930in}}{\pgfqpoint{0.833532in}{1.445694in}}%
\pgfpathcurveto{\pgfqpoint{0.833532in}{1.437457in}}{\pgfqpoint{0.836804in}{1.429557in}}{\pgfqpoint{0.842628in}{1.423733in}}%
\pgfpathcurveto{\pgfqpoint{0.848452in}{1.417909in}}{\pgfqpoint{0.856352in}{1.414637in}}{\pgfqpoint{0.864588in}{1.414637in}}%
\pgfpathclose%
\pgfusepath{stroke,fill}%
\end{pgfscope}%
\begin{pgfscope}%
\pgfpathrectangle{\pgfqpoint{0.100000in}{0.212622in}}{\pgfqpoint{3.696000in}{3.696000in}}%
\pgfusepath{clip}%
\pgfsetbuttcap%
\pgfsetroundjoin%
\definecolor{currentfill}{rgb}{0.121569,0.466667,0.705882}%
\pgfsetfillcolor{currentfill}%
\pgfsetfillopacity{0.657340}%
\pgfsetlinewidth{1.003750pt}%
\definecolor{currentstroke}{rgb}{0.121569,0.466667,0.705882}%
\pgfsetstrokecolor{currentstroke}%
\pgfsetstrokeopacity{0.657340}%
\pgfsetdash{}{0pt}%
\pgfpathmoveto{\pgfqpoint{0.864588in}{1.414637in}}%
\pgfpathcurveto{\pgfqpoint{0.872825in}{1.414637in}}{\pgfqpoint{0.880725in}{1.417909in}}{\pgfqpoint{0.886548in}{1.423733in}}%
\pgfpathcurveto{\pgfqpoint{0.892372in}{1.429557in}}{\pgfqpoint{0.895645in}{1.437457in}}{\pgfqpoint{0.895645in}{1.445694in}}%
\pgfpathcurveto{\pgfqpoint{0.895645in}{1.453930in}}{\pgfqpoint{0.892372in}{1.461830in}}{\pgfqpoint{0.886548in}{1.467654in}}%
\pgfpathcurveto{\pgfqpoint{0.880725in}{1.473478in}}{\pgfqpoint{0.872825in}{1.476750in}}{\pgfqpoint{0.864588in}{1.476750in}}%
\pgfpathcurveto{\pgfqpoint{0.856352in}{1.476750in}}{\pgfqpoint{0.848452in}{1.473478in}}{\pgfqpoint{0.842628in}{1.467654in}}%
\pgfpathcurveto{\pgfqpoint{0.836804in}{1.461830in}}{\pgfqpoint{0.833532in}{1.453930in}}{\pgfqpoint{0.833532in}{1.445694in}}%
\pgfpathcurveto{\pgfqpoint{0.833532in}{1.437457in}}{\pgfqpoint{0.836804in}{1.429557in}}{\pgfqpoint{0.842628in}{1.423733in}}%
\pgfpathcurveto{\pgfqpoint{0.848452in}{1.417909in}}{\pgfqpoint{0.856352in}{1.414637in}}{\pgfqpoint{0.864588in}{1.414637in}}%
\pgfpathclose%
\pgfusepath{stroke,fill}%
\end{pgfscope}%
\begin{pgfscope}%
\pgfpathrectangle{\pgfqpoint{0.100000in}{0.212622in}}{\pgfqpoint{3.696000in}{3.696000in}}%
\pgfusepath{clip}%
\pgfsetbuttcap%
\pgfsetroundjoin%
\definecolor{currentfill}{rgb}{0.121569,0.466667,0.705882}%
\pgfsetfillcolor{currentfill}%
\pgfsetfillopacity{0.657340}%
\pgfsetlinewidth{1.003750pt}%
\definecolor{currentstroke}{rgb}{0.121569,0.466667,0.705882}%
\pgfsetstrokecolor{currentstroke}%
\pgfsetstrokeopacity{0.657340}%
\pgfsetdash{}{0pt}%
\pgfpathmoveto{\pgfqpoint{0.864588in}{1.414637in}}%
\pgfpathcurveto{\pgfqpoint{0.872825in}{1.414637in}}{\pgfqpoint{0.880725in}{1.417909in}}{\pgfqpoint{0.886548in}{1.423733in}}%
\pgfpathcurveto{\pgfqpoint{0.892372in}{1.429557in}}{\pgfqpoint{0.895645in}{1.437457in}}{\pgfqpoint{0.895645in}{1.445694in}}%
\pgfpathcurveto{\pgfqpoint{0.895645in}{1.453930in}}{\pgfqpoint{0.892372in}{1.461830in}}{\pgfqpoint{0.886548in}{1.467654in}}%
\pgfpathcurveto{\pgfqpoint{0.880725in}{1.473478in}}{\pgfqpoint{0.872825in}{1.476750in}}{\pgfqpoint{0.864588in}{1.476750in}}%
\pgfpathcurveto{\pgfqpoint{0.856352in}{1.476750in}}{\pgfqpoint{0.848452in}{1.473478in}}{\pgfqpoint{0.842628in}{1.467654in}}%
\pgfpathcurveto{\pgfqpoint{0.836804in}{1.461830in}}{\pgfqpoint{0.833532in}{1.453930in}}{\pgfqpoint{0.833532in}{1.445694in}}%
\pgfpathcurveto{\pgfqpoint{0.833532in}{1.437457in}}{\pgfqpoint{0.836804in}{1.429557in}}{\pgfqpoint{0.842628in}{1.423733in}}%
\pgfpathcurveto{\pgfqpoint{0.848452in}{1.417909in}}{\pgfqpoint{0.856352in}{1.414637in}}{\pgfqpoint{0.864588in}{1.414637in}}%
\pgfpathclose%
\pgfusepath{stroke,fill}%
\end{pgfscope}%
\begin{pgfscope}%
\pgfpathrectangle{\pgfqpoint{0.100000in}{0.212622in}}{\pgfqpoint{3.696000in}{3.696000in}}%
\pgfusepath{clip}%
\pgfsetbuttcap%
\pgfsetroundjoin%
\definecolor{currentfill}{rgb}{0.121569,0.466667,0.705882}%
\pgfsetfillcolor{currentfill}%
\pgfsetfillopacity{0.657340}%
\pgfsetlinewidth{1.003750pt}%
\definecolor{currentstroke}{rgb}{0.121569,0.466667,0.705882}%
\pgfsetstrokecolor{currentstroke}%
\pgfsetstrokeopacity{0.657340}%
\pgfsetdash{}{0pt}%
\pgfpathmoveto{\pgfqpoint{0.864588in}{1.414637in}}%
\pgfpathcurveto{\pgfqpoint{0.872825in}{1.414637in}}{\pgfqpoint{0.880725in}{1.417909in}}{\pgfqpoint{0.886548in}{1.423733in}}%
\pgfpathcurveto{\pgfqpoint{0.892372in}{1.429557in}}{\pgfqpoint{0.895645in}{1.437457in}}{\pgfqpoint{0.895645in}{1.445694in}}%
\pgfpathcurveto{\pgfqpoint{0.895645in}{1.453930in}}{\pgfqpoint{0.892372in}{1.461830in}}{\pgfqpoint{0.886548in}{1.467654in}}%
\pgfpathcurveto{\pgfqpoint{0.880725in}{1.473478in}}{\pgfqpoint{0.872825in}{1.476750in}}{\pgfqpoint{0.864588in}{1.476750in}}%
\pgfpathcurveto{\pgfqpoint{0.856352in}{1.476750in}}{\pgfqpoint{0.848452in}{1.473478in}}{\pgfqpoint{0.842628in}{1.467654in}}%
\pgfpathcurveto{\pgfqpoint{0.836804in}{1.461830in}}{\pgfqpoint{0.833532in}{1.453930in}}{\pgfqpoint{0.833532in}{1.445694in}}%
\pgfpathcurveto{\pgfqpoint{0.833532in}{1.437457in}}{\pgfqpoint{0.836804in}{1.429557in}}{\pgfqpoint{0.842628in}{1.423733in}}%
\pgfpathcurveto{\pgfqpoint{0.848452in}{1.417909in}}{\pgfqpoint{0.856352in}{1.414637in}}{\pgfqpoint{0.864588in}{1.414637in}}%
\pgfpathclose%
\pgfusepath{stroke,fill}%
\end{pgfscope}%
\begin{pgfscope}%
\pgfpathrectangle{\pgfqpoint{0.100000in}{0.212622in}}{\pgfqpoint{3.696000in}{3.696000in}}%
\pgfusepath{clip}%
\pgfsetbuttcap%
\pgfsetroundjoin%
\definecolor{currentfill}{rgb}{0.121569,0.466667,0.705882}%
\pgfsetfillcolor{currentfill}%
\pgfsetfillopacity{0.657340}%
\pgfsetlinewidth{1.003750pt}%
\definecolor{currentstroke}{rgb}{0.121569,0.466667,0.705882}%
\pgfsetstrokecolor{currentstroke}%
\pgfsetstrokeopacity{0.657340}%
\pgfsetdash{}{0pt}%
\pgfpathmoveto{\pgfqpoint{0.864588in}{1.414637in}}%
\pgfpathcurveto{\pgfqpoint{0.872825in}{1.414637in}}{\pgfqpoint{0.880725in}{1.417909in}}{\pgfqpoint{0.886548in}{1.423733in}}%
\pgfpathcurveto{\pgfqpoint{0.892372in}{1.429557in}}{\pgfqpoint{0.895645in}{1.437457in}}{\pgfqpoint{0.895645in}{1.445694in}}%
\pgfpathcurveto{\pgfqpoint{0.895645in}{1.453930in}}{\pgfqpoint{0.892372in}{1.461830in}}{\pgfqpoint{0.886548in}{1.467654in}}%
\pgfpathcurveto{\pgfqpoint{0.880725in}{1.473478in}}{\pgfqpoint{0.872825in}{1.476750in}}{\pgfqpoint{0.864588in}{1.476750in}}%
\pgfpathcurveto{\pgfqpoint{0.856352in}{1.476750in}}{\pgfqpoint{0.848452in}{1.473478in}}{\pgfqpoint{0.842628in}{1.467654in}}%
\pgfpathcurveto{\pgfqpoint{0.836804in}{1.461830in}}{\pgfqpoint{0.833532in}{1.453930in}}{\pgfqpoint{0.833532in}{1.445694in}}%
\pgfpathcurveto{\pgfqpoint{0.833532in}{1.437457in}}{\pgfqpoint{0.836804in}{1.429557in}}{\pgfqpoint{0.842628in}{1.423733in}}%
\pgfpathcurveto{\pgfqpoint{0.848452in}{1.417909in}}{\pgfqpoint{0.856352in}{1.414637in}}{\pgfqpoint{0.864588in}{1.414637in}}%
\pgfpathclose%
\pgfusepath{stroke,fill}%
\end{pgfscope}%
\begin{pgfscope}%
\pgfpathrectangle{\pgfqpoint{0.100000in}{0.212622in}}{\pgfqpoint{3.696000in}{3.696000in}}%
\pgfusepath{clip}%
\pgfsetbuttcap%
\pgfsetroundjoin%
\definecolor{currentfill}{rgb}{0.121569,0.466667,0.705882}%
\pgfsetfillcolor{currentfill}%
\pgfsetfillopacity{0.657340}%
\pgfsetlinewidth{1.003750pt}%
\definecolor{currentstroke}{rgb}{0.121569,0.466667,0.705882}%
\pgfsetstrokecolor{currentstroke}%
\pgfsetstrokeopacity{0.657340}%
\pgfsetdash{}{0pt}%
\pgfpathmoveto{\pgfqpoint{0.864588in}{1.414637in}}%
\pgfpathcurveto{\pgfqpoint{0.872825in}{1.414637in}}{\pgfqpoint{0.880725in}{1.417909in}}{\pgfqpoint{0.886548in}{1.423733in}}%
\pgfpathcurveto{\pgfqpoint{0.892372in}{1.429557in}}{\pgfqpoint{0.895645in}{1.437457in}}{\pgfqpoint{0.895645in}{1.445694in}}%
\pgfpathcurveto{\pgfqpoint{0.895645in}{1.453930in}}{\pgfqpoint{0.892372in}{1.461830in}}{\pgfqpoint{0.886548in}{1.467654in}}%
\pgfpathcurveto{\pgfqpoint{0.880725in}{1.473478in}}{\pgfqpoint{0.872825in}{1.476750in}}{\pgfqpoint{0.864588in}{1.476750in}}%
\pgfpathcurveto{\pgfqpoint{0.856352in}{1.476750in}}{\pgfqpoint{0.848452in}{1.473478in}}{\pgfqpoint{0.842628in}{1.467654in}}%
\pgfpathcurveto{\pgfqpoint{0.836804in}{1.461830in}}{\pgfqpoint{0.833532in}{1.453930in}}{\pgfqpoint{0.833532in}{1.445694in}}%
\pgfpathcurveto{\pgfqpoint{0.833532in}{1.437457in}}{\pgfqpoint{0.836804in}{1.429557in}}{\pgfqpoint{0.842628in}{1.423733in}}%
\pgfpathcurveto{\pgfqpoint{0.848452in}{1.417909in}}{\pgfqpoint{0.856352in}{1.414637in}}{\pgfqpoint{0.864588in}{1.414637in}}%
\pgfpathclose%
\pgfusepath{stroke,fill}%
\end{pgfscope}%
\begin{pgfscope}%
\pgfpathrectangle{\pgfqpoint{0.100000in}{0.212622in}}{\pgfqpoint{3.696000in}{3.696000in}}%
\pgfusepath{clip}%
\pgfsetbuttcap%
\pgfsetroundjoin%
\definecolor{currentfill}{rgb}{0.121569,0.466667,0.705882}%
\pgfsetfillcolor{currentfill}%
\pgfsetfillopacity{0.657340}%
\pgfsetlinewidth{1.003750pt}%
\definecolor{currentstroke}{rgb}{0.121569,0.466667,0.705882}%
\pgfsetstrokecolor{currentstroke}%
\pgfsetstrokeopacity{0.657340}%
\pgfsetdash{}{0pt}%
\pgfpathmoveto{\pgfqpoint{0.864588in}{1.414637in}}%
\pgfpathcurveto{\pgfqpoint{0.872825in}{1.414637in}}{\pgfqpoint{0.880725in}{1.417909in}}{\pgfqpoint{0.886548in}{1.423733in}}%
\pgfpathcurveto{\pgfqpoint{0.892372in}{1.429557in}}{\pgfqpoint{0.895645in}{1.437457in}}{\pgfqpoint{0.895645in}{1.445694in}}%
\pgfpathcurveto{\pgfqpoint{0.895645in}{1.453930in}}{\pgfqpoint{0.892372in}{1.461830in}}{\pgfqpoint{0.886548in}{1.467654in}}%
\pgfpathcurveto{\pgfqpoint{0.880725in}{1.473478in}}{\pgfqpoint{0.872825in}{1.476750in}}{\pgfqpoint{0.864588in}{1.476750in}}%
\pgfpathcurveto{\pgfqpoint{0.856352in}{1.476750in}}{\pgfqpoint{0.848452in}{1.473478in}}{\pgfqpoint{0.842628in}{1.467654in}}%
\pgfpathcurveto{\pgfqpoint{0.836804in}{1.461830in}}{\pgfqpoint{0.833532in}{1.453930in}}{\pgfqpoint{0.833532in}{1.445694in}}%
\pgfpathcurveto{\pgfqpoint{0.833532in}{1.437457in}}{\pgfqpoint{0.836804in}{1.429557in}}{\pgfqpoint{0.842628in}{1.423733in}}%
\pgfpathcurveto{\pgfqpoint{0.848452in}{1.417909in}}{\pgfqpoint{0.856352in}{1.414637in}}{\pgfqpoint{0.864588in}{1.414637in}}%
\pgfpathclose%
\pgfusepath{stroke,fill}%
\end{pgfscope}%
\begin{pgfscope}%
\pgfpathrectangle{\pgfqpoint{0.100000in}{0.212622in}}{\pgfqpoint{3.696000in}{3.696000in}}%
\pgfusepath{clip}%
\pgfsetbuttcap%
\pgfsetroundjoin%
\definecolor{currentfill}{rgb}{0.121569,0.466667,0.705882}%
\pgfsetfillcolor{currentfill}%
\pgfsetfillopacity{0.657340}%
\pgfsetlinewidth{1.003750pt}%
\definecolor{currentstroke}{rgb}{0.121569,0.466667,0.705882}%
\pgfsetstrokecolor{currentstroke}%
\pgfsetstrokeopacity{0.657340}%
\pgfsetdash{}{0pt}%
\pgfpathmoveto{\pgfqpoint{0.864588in}{1.414637in}}%
\pgfpathcurveto{\pgfqpoint{0.872825in}{1.414637in}}{\pgfqpoint{0.880725in}{1.417909in}}{\pgfqpoint{0.886548in}{1.423733in}}%
\pgfpathcurveto{\pgfqpoint{0.892372in}{1.429557in}}{\pgfqpoint{0.895645in}{1.437457in}}{\pgfqpoint{0.895645in}{1.445694in}}%
\pgfpathcurveto{\pgfqpoint{0.895645in}{1.453930in}}{\pgfqpoint{0.892372in}{1.461830in}}{\pgfqpoint{0.886548in}{1.467654in}}%
\pgfpathcurveto{\pgfqpoint{0.880725in}{1.473478in}}{\pgfqpoint{0.872825in}{1.476750in}}{\pgfqpoint{0.864588in}{1.476750in}}%
\pgfpathcurveto{\pgfqpoint{0.856352in}{1.476750in}}{\pgfqpoint{0.848452in}{1.473478in}}{\pgfqpoint{0.842628in}{1.467654in}}%
\pgfpathcurveto{\pgfqpoint{0.836804in}{1.461830in}}{\pgfqpoint{0.833532in}{1.453930in}}{\pgfqpoint{0.833532in}{1.445694in}}%
\pgfpathcurveto{\pgfqpoint{0.833532in}{1.437457in}}{\pgfqpoint{0.836804in}{1.429557in}}{\pgfqpoint{0.842628in}{1.423733in}}%
\pgfpathcurveto{\pgfqpoint{0.848452in}{1.417909in}}{\pgfqpoint{0.856352in}{1.414637in}}{\pgfqpoint{0.864588in}{1.414637in}}%
\pgfpathclose%
\pgfusepath{stroke,fill}%
\end{pgfscope}%
\begin{pgfscope}%
\pgfpathrectangle{\pgfqpoint{0.100000in}{0.212622in}}{\pgfqpoint{3.696000in}{3.696000in}}%
\pgfusepath{clip}%
\pgfsetbuttcap%
\pgfsetroundjoin%
\definecolor{currentfill}{rgb}{0.121569,0.466667,0.705882}%
\pgfsetfillcolor{currentfill}%
\pgfsetfillopacity{0.657340}%
\pgfsetlinewidth{1.003750pt}%
\definecolor{currentstroke}{rgb}{0.121569,0.466667,0.705882}%
\pgfsetstrokecolor{currentstroke}%
\pgfsetstrokeopacity{0.657340}%
\pgfsetdash{}{0pt}%
\pgfpathmoveto{\pgfqpoint{0.864588in}{1.414637in}}%
\pgfpathcurveto{\pgfqpoint{0.872825in}{1.414637in}}{\pgfqpoint{0.880725in}{1.417909in}}{\pgfqpoint{0.886548in}{1.423733in}}%
\pgfpathcurveto{\pgfqpoint{0.892372in}{1.429557in}}{\pgfqpoint{0.895645in}{1.437457in}}{\pgfqpoint{0.895645in}{1.445694in}}%
\pgfpathcurveto{\pgfqpoint{0.895645in}{1.453930in}}{\pgfqpoint{0.892372in}{1.461830in}}{\pgfqpoint{0.886548in}{1.467654in}}%
\pgfpathcurveto{\pgfqpoint{0.880725in}{1.473478in}}{\pgfqpoint{0.872825in}{1.476750in}}{\pgfqpoint{0.864588in}{1.476750in}}%
\pgfpathcurveto{\pgfqpoint{0.856352in}{1.476750in}}{\pgfqpoint{0.848452in}{1.473478in}}{\pgfqpoint{0.842628in}{1.467654in}}%
\pgfpathcurveto{\pgfqpoint{0.836804in}{1.461830in}}{\pgfqpoint{0.833532in}{1.453930in}}{\pgfqpoint{0.833532in}{1.445694in}}%
\pgfpathcurveto{\pgfqpoint{0.833532in}{1.437457in}}{\pgfqpoint{0.836804in}{1.429557in}}{\pgfqpoint{0.842628in}{1.423733in}}%
\pgfpathcurveto{\pgfqpoint{0.848452in}{1.417909in}}{\pgfqpoint{0.856352in}{1.414637in}}{\pgfqpoint{0.864588in}{1.414637in}}%
\pgfpathclose%
\pgfusepath{stroke,fill}%
\end{pgfscope}%
\begin{pgfscope}%
\pgfpathrectangle{\pgfqpoint{0.100000in}{0.212622in}}{\pgfqpoint{3.696000in}{3.696000in}}%
\pgfusepath{clip}%
\pgfsetbuttcap%
\pgfsetroundjoin%
\definecolor{currentfill}{rgb}{0.121569,0.466667,0.705882}%
\pgfsetfillcolor{currentfill}%
\pgfsetfillopacity{0.657340}%
\pgfsetlinewidth{1.003750pt}%
\definecolor{currentstroke}{rgb}{0.121569,0.466667,0.705882}%
\pgfsetstrokecolor{currentstroke}%
\pgfsetstrokeopacity{0.657340}%
\pgfsetdash{}{0pt}%
\pgfpathmoveto{\pgfqpoint{0.864588in}{1.414637in}}%
\pgfpathcurveto{\pgfqpoint{0.872825in}{1.414637in}}{\pgfqpoint{0.880725in}{1.417909in}}{\pgfqpoint{0.886548in}{1.423733in}}%
\pgfpathcurveto{\pgfqpoint{0.892372in}{1.429557in}}{\pgfqpoint{0.895645in}{1.437457in}}{\pgfqpoint{0.895645in}{1.445694in}}%
\pgfpathcurveto{\pgfqpoint{0.895645in}{1.453930in}}{\pgfqpoint{0.892372in}{1.461830in}}{\pgfqpoint{0.886548in}{1.467654in}}%
\pgfpathcurveto{\pgfqpoint{0.880725in}{1.473478in}}{\pgfqpoint{0.872825in}{1.476750in}}{\pgfqpoint{0.864588in}{1.476750in}}%
\pgfpathcurveto{\pgfqpoint{0.856352in}{1.476750in}}{\pgfqpoint{0.848452in}{1.473478in}}{\pgfqpoint{0.842628in}{1.467654in}}%
\pgfpathcurveto{\pgfqpoint{0.836804in}{1.461830in}}{\pgfqpoint{0.833532in}{1.453930in}}{\pgfqpoint{0.833532in}{1.445694in}}%
\pgfpathcurveto{\pgfqpoint{0.833532in}{1.437457in}}{\pgfqpoint{0.836804in}{1.429557in}}{\pgfqpoint{0.842628in}{1.423733in}}%
\pgfpathcurveto{\pgfqpoint{0.848452in}{1.417909in}}{\pgfqpoint{0.856352in}{1.414637in}}{\pgfqpoint{0.864588in}{1.414637in}}%
\pgfpathclose%
\pgfusepath{stroke,fill}%
\end{pgfscope}%
\begin{pgfscope}%
\pgfpathrectangle{\pgfqpoint{0.100000in}{0.212622in}}{\pgfqpoint{3.696000in}{3.696000in}}%
\pgfusepath{clip}%
\pgfsetbuttcap%
\pgfsetroundjoin%
\definecolor{currentfill}{rgb}{0.121569,0.466667,0.705882}%
\pgfsetfillcolor{currentfill}%
\pgfsetfillopacity{0.657340}%
\pgfsetlinewidth{1.003750pt}%
\definecolor{currentstroke}{rgb}{0.121569,0.466667,0.705882}%
\pgfsetstrokecolor{currentstroke}%
\pgfsetstrokeopacity{0.657340}%
\pgfsetdash{}{0pt}%
\pgfpathmoveto{\pgfqpoint{0.864588in}{1.414637in}}%
\pgfpathcurveto{\pgfqpoint{0.872825in}{1.414637in}}{\pgfqpoint{0.880725in}{1.417909in}}{\pgfqpoint{0.886548in}{1.423733in}}%
\pgfpathcurveto{\pgfqpoint{0.892372in}{1.429557in}}{\pgfqpoint{0.895645in}{1.437457in}}{\pgfqpoint{0.895645in}{1.445694in}}%
\pgfpathcurveto{\pgfqpoint{0.895645in}{1.453930in}}{\pgfqpoint{0.892372in}{1.461830in}}{\pgfqpoint{0.886548in}{1.467654in}}%
\pgfpathcurveto{\pgfqpoint{0.880725in}{1.473478in}}{\pgfqpoint{0.872825in}{1.476750in}}{\pgfqpoint{0.864588in}{1.476750in}}%
\pgfpathcurveto{\pgfqpoint{0.856352in}{1.476750in}}{\pgfqpoint{0.848452in}{1.473478in}}{\pgfqpoint{0.842628in}{1.467654in}}%
\pgfpathcurveto{\pgfqpoint{0.836804in}{1.461830in}}{\pgfqpoint{0.833532in}{1.453930in}}{\pgfqpoint{0.833532in}{1.445694in}}%
\pgfpathcurveto{\pgfqpoint{0.833532in}{1.437457in}}{\pgfqpoint{0.836804in}{1.429557in}}{\pgfqpoint{0.842628in}{1.423733in}}%
\pgfpathcurveto{\pgfqpoint{0.848452in}{1.417909in}}{\pgfqpoint{0.856352in}{1.414637in}}{\pgfqpoint{0.864588in}{1.414637in}}%
\pgfpathclose%
\pgfusepath{stroke,fill}%
\end{pgfscope}%
\begin{pgfscope}%
\pgfpathrectangle{\pgfqpoint{0.100000in}{0.212622in}}{\pgfqpoint{3.696000in}{3.696000in}}%
\pgfusepath{clip}%
\pgfsetbuttcap%
\pgfsetroundjoin%
\definecolor{currentfill}{rgb}{0.121569,0.466667,0.705882}%
\pgfsetfillcolor{currentfill}%
\pgfsetfillopacity{0.657340}%
\pgfsetlinewidth{1.003750pt}%
\definecolor{currentstroke}{rgb}{0.121569,0.466667,0.705882}%
\pgfsetstrokecolor{currentstroke}%
\pgfsetstrokeopacity{0.657340}%
\pgfsetdash{}{0pt}%
\pgfpathmoveto{\pgfqpoint{0.864588in}{1.414637in}}%
\pgfpathcurveto{\pgfqpoint{0.872825in}{1.414637in}}{\pgfqpoint{0.880725in}{1.417909in}}{\pgfqpoint{0.886548in}{1.423733in}}%
\pgfpathcurveto{\pgfqpoint{0.892372in}{1.429557in}}{\pgfqpoint{0.895645in}{1.437457in}}{\pgfqpoint{0.895645in}{1.445694in}}%
\pgfpathcurveto{\pgfqpoint{0.895645in}{1.453930in}}{\pgfqpoint{0.892372in}{1.461830in}}{\pgfqpoint{0.886548in}{1.467654in}}%
\pgfpathcurveto{\pgfqpoint{0.880725in}{1.473478in}}{\pgfqpoint{0.872825in}{1.476750in}}{\pgfqpoint{0.864588in}{1.476750in}}%
\pgfpathcurveto{\pgfqpoint{0.856352in}{1.476750in}}{\pgfqpoint{0.848452in}{1.473478in}}{\pgfqpoint{0.842628in}{1.467654in}}%
\pgfpathcurveto{\pgfqpoint{0.836804in}{1.461830in}}{\pgfqpoint{0.833532in}{1.453930in}}{\pgfqpoint{0.833532in}{1.445694in}}%
\pgfpathcurveto{\pgfqpoint{0.833532in}{1.437457in}}{\pgfqpoint{0.836804in}{1.429557in}}{\pgfqpoint{0.842628in}{1.423733in}}%
\pgfpathcurveto{\pgfqpoint{0.848452in}{1.417909in}}{\pgfqpoint{0.856352in}{1.414637in}}{\pgfqpoint{0.864588in}{1.414637in}}%
\pgfpathclose%
\pgfusepath{stroke,fill}%
\end{pgfscope}%
\begin{pgfscope}%
\pgfpathrectangle{\pgfqpoint{0.100000in}{0.212622in}}{\pgfqpoint{3.696000in}{3.696000in}}%
\pgfusepath{clip}%
\pgfsetbuttcap%
\pgfsetroundjoin%
\definecolor{currentfill}{rgb}{0.121569,0.466667,0.705882}%
\pgfsetfillcolor{currentfill}%
\pgfsetfillopacity{0.657340}%
\pgfsetlinewidth{1.003750pt}%
\definecolor{currentstroke}{rgb}{0.121569,0.466667,0.705882}%
\pgfsetstrokecolor{currentstroke}%
\pgfsetstrokeopacity{0.657340}%
\pgfsetdash{}{0pt}%
\pgfpathmoveto{\pgfqpoint{0.864588in}{1.414637in}}%
\pgfpathcurveto{\pgfqpoint{0.872825in}{1.414637in}}{\pgfqpoint{0.880725in}{1.417909in}}{\pgfqpoint{0.886548in}{1.423733in}}%
\pgfpathcurveto{\pgfqpoint{0.892372in}{1.429557in}}{\pgfqpoint{0.895645in}{1.437457in}}{\pgfqpoint{0.895645in}{1.445694in}}%
\pgfpathcurveto{\pgfqpoint{0.895645in}{1.453930in}}{\pgfqpoint{0.892372in}{1.461830in}}{\pgfqpoint{0.886548in}{1.467654in}}%
\pgfpathcurveto{\pgfqpoint{0.880725in}{1.473478in}}{\pgfqpoint{0.872825in}{1.476750in}}{\pgfqpoint{0.864588in}{1.476750in}}%
\pgfpathcurveto{\pgfqpoint{0.856352in}{1.476750in}}{\pgfqpoint{0.848452in}{1.473478in}}{\pgfqpoint{0.842628in}{1.467654in}}%
\pgfpathcurveto{\pgfqpoint{0.836804in}{1.461830in}}{\pgfqpoint{0.833532in}{1.453930in}}{\pgfqpoint{0.833532in}{1.445694in}}%
\pgfpathcurveto{\pgfqpoint{0.833532in}{1.437457in}}{\pgfqpoint{0.836804in}{1.429557in}}{\pgfqpoint{0.842628in}{1.423733in}}%
\pgfpathcurveto{\pgfqpoint{0.848452in}{1.417909in}}{\pgfqpoint{0.856352in}{1.414637in}}{\pgfqpoint{0.864588in}{1.414637in}}%
\pgfpathclose%
\pgfusepath{stroke,fill}%
\end{pgfscope}%
\begin{pgfscope}%
\pgfpathrectangle{\pgfqpoint{0.100000in}{0.212622in}}{\pgfqpoint{3.696000in}{3.696000in}}%
\pgfusepath{clip}%
\pgfsetbuttcap%
\pgfsetroundjoin%
\definecolor{currentfill}{rgb}{0.121569,0.466667,0.705882}%
\pgfsetfillcolor{currentfill}%
\pgfsetfillopacity{0.657340}%
\pgfsetlinewidth{1.003750pt}%
\definecolor{currentstroke}{rgb}{0.121569,0.466667,0.705882}%
\pgfsetstrokecolor{currentstroke}%
\pgfsetstrokeopacity{0.657340}%
\pgfsetdash{}{0pt}%
\pgfpathmoveto{\pgfqpoint{0.864588in}{1.414637in}}%
\pgfpathcurveto{\pgfqpoint{0.872825in}{1.414637in}}{\pgfqpoint{0.880725in}{1.417909in}}{\pgfqpoint{0.886548in}{1.423733in}}%
\pgfpathcurveto{\pgfqpoint{0.892372in}{1.429557in}}{\pgfqpoint{0.895645in}{1.437457in}}{\pgfqpoint{0.895645in}{1.445694in}}%
\pgfpathcurveto{\pgfqpoint{0.895645in}{1.453930in}}{\pgfqpoint{0.892372in}{1.461830in}}{\pgfqpoint{0.886548in}{1.467654in}}%
\pgfpathcurveto{\pgfqpoint{0.880725in}{1.473478in}}{\pgfqpoint{0.872825in}{1.476750in}}{\pgfqpoint{0.864588in}{1.476750in}}%
\pgfpathcurveto{\pgfqpoint{0.856352in}{1.476750in}}{\pgfqpoint{0.848452in}{1.473478in}}{\pgfqpoint{0.842628in}{1.467654in}}%
\pgfpathcurveto{\pgfqpoint{0.836804in}{1.461830in}}{\pgfqpoint{0.833532in}{1.453930in}}{\pgfqpoint{0.833532in}{1.445694in}}%
\pgfpathcurveto{\pgfqpoint{0.833532in}{1.437457in}}{\pgfqpoint{0.836804in}{1.429557in}}{\pgfqpoint{0.842628in}{1.423733in}}%
\pgfpathcurveto{\pgfqpoint{0.848452in}{1.417909in}}{\pgfqpoint{0.856352in}{1.414637in}}{\pgfqpoint{0.864588in}{1.414637in}}%
\pgfpathclose%
\pgfusepath{stroke,fill}%
\end{pgfscope}%
\begin{pgfscope}%
\pgfpathrectangle{\pgfqpoint{0.100000in}{0.212622in}}{\pgfqpoint{3.696000in}{3.696000in}}%
\pgfusepath{clip}%
\pgfsetbuttcap%
\pgfsetroundjoin%
\definecolor{currentfill}{rgb}{0.121569,0.466667,0.705882}%
\pgfsetfillcolor{currentfill}%
\pgfsetfillopacity{0.657340}%
\pgfsetlinewidth{1.003750pt}%
\definecolor{currentstroke}{rgb}{0.121569,0.466667,0.705882}%
\pgfsetstrokecolor{currentstroke}%
\pgfsetstrokeopacity{0.657340}%
\pgfsetdash{}{0pt}%
\pgfpathmoveto{\pgfqpoint{0.864588in}{1.414637in}}%
\pgfpathcurveto{\pgfqpoint{0.872825in}{1.414637in}}{\pgfqpoint{0.880725in}{1.417909in}}{\pgfqpoint{0.886548in}{1.423733in}}%
\pgfpathcurveto{\pgfqpoint{0.892372in}{1.429557in}}{\pgfqpoint{0.895645in}{1.437457in}}{\pgfqpoint{0.895645in}{1.445694in}}%
\pgfpathcurveto{\pgfqpoint{0.895645in}{1.453930in}}{\pgfqpoint{0.892372in}{1.461830in}}{\pgfqpoint{0.886548in}{1.467654in}}%
\pgfpathcurveto{\pgfqpoint{0.880725in}{1.473478in}}{\pgfqpoint{0.872825in}{1.476750in}}{\pgfqpoint{0.864588in}{1.476750in}}%
\pgfpathcurveto{\pgfqpoint{0.856352in}{1.476750in}}{\pgfqpoint{0.848452in}{1.473478in}}{\pgfqpoint{0.842628in}{1.467654in}}%
\pgfpathcurveto{\pgfqpoint{0.836804in}{1.461830in}}{\pgfqpoint{0.833532in}{1.453930in}}{\pgfqpoint{0.833532in}{1.445694in}}%
\pgfpathcurveto{\pgfqpoint{0.833532in}{1.437457in}}{\pgfqpoint{0.836804in}{1.429557in}}{\pgfqpoint{0.842628in}{1.423733in}}%
\pgfpathcurveto{\pgfqpoint{0.848452in}{1.417909in}}{\pgfqpoint{0.856352in}{1.414637in}}{\pgfqpoint{0.864588in}{1.414637in}}%
\pgfpathclose%
\pgfusepath{stroke,fill}%
\end{pgfscope}%
\begin{pgfscope}%
\pgfpathrectangle{\pgfqpoint{0.100000in}{0.212622in}}{\pgfqpoint{3.696000in}{3.696000in}}%
\pgfusepath{clip}%
\pgfsetbuttcap%
\pgfsetroundjoin%
\definecolor{currentfill}{rgb}{0.121569,0.466667,0.705882}%
\pgfsetfillcolor{currentfill}%
\pgfsetfillopacity{0.657340}%
\pgfsetlinewidth{1.003750pt}%
\definecolor{currentstroke}{rgb}{0.121569,0.466667,0.705882}%
\pgfsetstrokecolor{currentstroke}%
\pgfsetstrokeopacity{0.657340}%
\pgfsetdash{}{0pt}%
\pgfpathmoveto{\pgfqpoint{0.864588in}{1.414637in}}%
\pgfpathcurveto{\pgfqpoint{0.872825in}{1.414637in}}{\pgfqpoint{0.880725in}{1.417909in}}{\pgfqpoint{0.886548in}{1.423733in}}%
\pgfpathcurveto{\pgfqpoint{0.892372in}{1.429557in}}{\pgfqpoint{0.895645in}{1.437457in}}{\pgfqpoint{0.895645in}{1.445694in}}%
\pgfpathcurveto{\pgfqpoint{0.895645in}{1.453930in}}{\pgfqpoint{0.892372in}{1.461830in}}{\pgfqpoint{0.886548in}{1.467654in}}%
\pgfpathcurveto{\pgfqpoint{0.880725in}{1.473478in}}{\pgfqpoint{0.872825in}{1.476750in}}{\pgfqpoint{0.864588in}{1.476750in}}%
\pgfpathcurveto{\pgfqpoint{0.856352in}{1.476750in}}{\pgfqpoint{0.848452in}{1.473478in}}{\pgfqpoint{0.842628in}{1.467654in}}%
\pgfpathcurveto{\pgfqpoint{0.836804in}{1.461830in}}{\pgfqpoint{0.833532in}{1.453930in}}{\pgfqpoint{0.833532in}{1.445694in}}%
\pgfpathcurveto{\pgfqpoint{0.833532in}{1.437457in}}{\pgfqpoint{0.836804in}{1.429557in}}{\pgfqpoint{0.842628in}{1.423733in}}%
\pgfpathcurveto{\pgfqpoint{0.848452in}{1.417909in}}{\pgfqpoint{0.856352in}{1.414637in}}{\pgfqpoint{0.864588in}{1.414637in}}%
\pgfpathclose%
\pgfusepath{stroke,fill}%
\end{pgfscope}%
\begin{pgfscope}%
\pgfpathrectangle{\pgfqpoint{0.100000in}{0.212622in}}{\pgfqpoint{3.696000in}{3.696000in}}%
\pgfusepath{clip}%
\pgfsetbuttcap%
\pgfsetroundjoin%
\definecolor{currentfill}{rgb}{0.121569,0.466667,0.705882}%
\pgfsetfillcolor{currentfill}%
\pgfsetfillopacity{0.657340}%
\pgfsetlinewidth{1.003750pt}%
\definecolor{currentstroke}{rgb}{0.121569,0.466667,0.705882}%
\pgfsetstrokecolor{currentstroke}%
\pgfsetstrokeopacity{0.657340}%
\pgfsetdash{}{0pt}%
\pgfpathmoveto{\pgfqpoint{0.864588in}{1.414637in}}%
\pgfpathcurveto{\pgfqpoint{0.872825in}{1.414637in}}{\pgfqpoint{0.880725in}{1.417909in}}{\pgfqpoint{0.886548in}{1.423733in}}%
\pgfpathcurveto{\pgfqpoint{0.892372in}{1.429557in}}{\pgfqpoint{0.895645in}{1.437457in}}{\pgfqpoint{0.895645in}{1.445694in}}%
\pgfpathcurveto{\pgfqpoint{0.895645in}{1.453930in}}{\pgfqpoint{0.892372in}{1.461830in}}{\pgfqpoint{0.886548in}{1.467654in}}%
\pgfpathcurveto{\pgfqpoint{0.880725in}{1.473478in}}{\pgfqpoint{0.872825in}{1.476750in}}{\pgfqpoint{0.864588in}{1.476750in}}%
\pgfpathcurveto{\pgfqpoint{0.856352in}{1.476750in}}{\pgfqpoint{0.848452in}{1.473478in}}{\pgfqpoint{0.842628in}{1.467654in}}%
\pgfpathcurveto{\pgfqpoint{0.836804in}{1.461830in}}{\pgfqpoint{0.833532in}{1.453930in}}{\pgfqpoint{0.833532in}{1.445694in}}%
\pgfpathcurveto{\pgfqpoint{0.833532in}{1.437457in}}{\pgfqpoint{0.836804in}{1.429557in}}{\pgfqpoint{0.842628in}{1.423733in}}%
\pgfpathcurveto{\pgfqpoint{0.848452in}{1.417909in}}{\pgfqpoint{0.856352in}{1.414637in}}{\pgfqpoint{0.864588in}{1.414637in}}%
\pgfpathclose%
\pgfusepath{stroke,fill}%
\end{pgfscope}%
\begin{pgfscope}%
\pgfpathrectangle{\pgfqpoint{0.100000in}{0.212622in}}{\pgfqpoint{3.696000in}{3.696000in}}%
\pgfusepath{clip}%
\pgfsetbuttcap%
\pgfsetroundjoin%
\definecolor{currentfill}{rgb}{0.121569,0.466667,0.705882}%
\pgfsetfillcolor{currentfill}%
\pgfsetfillopacity{0.657340}%
\pgfsetlinewidth{1.003750pt}%
\definecolor{currentstroke}{rgb}{0.121569,0.466667,0.705882}%
\pgfsetstrokecolor{currentstroke}%
\pgfsetstrokeopacity{0.657340}%
\pgfsetdash{}{0pt}%
\pgfpathmoveto{\pgfqpoint{0.864588in}{1.414637in}}%
\pgfpathcurveto{\pgfqpoint{0.872825in}{1.414637in}}{\pgfqpoint{0.880725in}{1.417909in}}{\pgfqpoint{0.886548in}{1.423733in}}%
\pgfpathcurveto{\pgfqpoint{0.892372in}{1.429557in}}{\pgfqpoint{0.895645in}{1.437457in}}{\pgfqpoint{0.895645in}{1.445694in}}%
\pgfpathcurveto{\pgfqpoint{0.895645in}{1.453930in}}{\pgfqpoint{0.892372in}{1.461830in}}{\pgfqpoint{0.886548in}{1.467654in}}%
\pgfpathcurveto{\pgfqpoint{0.880725in}{1.473478in}}{\pgfqpoint{0.872825in}{1.476750in}}{\pgfqpoint{0.864588in}{1.476750in}}%
\pgfpathcurveto{\pgfqpoint{0.856352in}{1.476750in}}{\pgfqpoint{0.848452in}{1.473478in}}{\pgfqpoint{0.842628in}{1.467654in}}%
\pgfpathcurveto{\pgfqpoint{0.836804in}{1.461830in}}{\pgfqpoint{0.833532in}{1.453930in}}{\pgfqpoint{0.833532in}{1.445694in}}%
\pgfpathcurveto{\pgfqpoint{0.833532in}{1.437457in}}{\pgfqpoint{0.836804in}{1.429557in}}{\pgfqpoint{0.842628in}{1.423733in}}%
\pgfpathcurveto{\pgfqpoint{0.848452in}{1.417909in}}{\pgfqpoint{0.856352in}{1.414637in}}{\pgfqpoint{0.864588in}{1.414637in}}%
\pgfpathclose%
\pgfusepath{stroke,fill}%
\end{pgfscope}%
\begin{pgfscope}%
\pgfpathrectangle{\pgfqpoint{0.100000in}{0.212622in}}{\pgfqpoint{3.696000in}{3.696000in}}%
\pgfusepath{clip}%
\pgfsetbuttcap%
\pgfsetroundjoin%
\definecolor{currentfill}{rgb}{0.121569,0.466667,0.705882}%
\pgfsetfillcolor{currentfill}%
\pgfsetfillopacity{0.657340}%
\pgfsetlinewidth{1.003750pt}%
\definecolor{currentstroke}{rgb}{0.121569,0.466667,0.705882}%
\pgfsetstrokecolor{currentstroke}%
\pgfsetstrokeopacity{0.657340}%
\pgfsetdash{}{0pt}%
\pgfpathmoveto{\pgfqpoint{0.864588in}{1.414637in}}%
\pgfpathcurveto{\pgfqpoint{0.872825in}{1.414637in}}{\pgfqpoint{0.880725in}{1.417909in}}{\pgfqpoint{0.886548in}{1.423733in}}%
\pgfpathcurveto{\pgfqpoint{0.892372in}{1.429557in}}{\pgfqpoint{0.895645in}{1.437457in}}{\pgfqpoint{0.895645in}{1.445694in}}%
\pgfpathcurveto{\pgfqpoint{0.895645in}{1.453930in}}{\pgfqpoint{0.892372in}{1.461830in}}{\pgfqpoint{0.886548in}{1.467654in}}%
\pgfpathcurveto{\pgfqpoint{0.880725in}{1.473478in}}{\pgfqpoint{0.872825in}{1.476750in}}{\pgfqpoint{0.864588in}{1.476750in}}%
\pgfpathcurveto{\pgfqpoint{0.856352in}{1.476750in}}{\pgfqpoint{0.848452in}{1.473478in}}{\pgfqpoint{0.842628in}{1.467654in}}%
\pgfpathcurveto{\pgfqpoint{0.836804in}{1.461830in}}{\pgfqpoint{0.833532in}{1.453930in}}{\pgfqpoint{0.833532in}{1.445694in}}%
\pgfpathcurveto{\pgfqpoint{0.833532in}{1.437457in}}{\pgfqpoint{0.836804in}{1.429557in}}{\pgfqpoint{0.842628in}{1.423733in}}%
\pgfpathcurveto{\pgfqpoint{0.848452in}{1.417909in}}{\pgfqpoint{0.856352in}{1.414637in}}{\pgfqpoint{0.864588in}{1.414637in}}%
\pgfpathclose%
\pgfusepath{stroke,fill}%
\end{pgfscope}%
\begin{pgfscope}%
\pgfpathrectangle{\pgfqpoint{0.100000in}{0.212622in}}{\pgfqpoint{3.696000in}{3.696000in}}%
\pgfusepath{clip}%
\pgfsetbuttcap%
\pgfsetroundjoin%
\definecolor{currentfill}{rgb}{0.121569,0.466667,0.705882}%
\pgfsetfillcolor{currentfill}%
\pgfsetfillopacity{0.657340}%
\pgfsetlinewidth{1.003750pt}%
\definecolor{currentstroke}{rgb}{0.121569,0.466667,0.705882}%
\pgfsetstrokecolor{currentstroke}%
\pgfsetstrokeopacity{0.657340}%
\pgfsetdash{}{0pt}%
\pgfpathmoveto{\pgfqpoint{0.864588in}{1.414637in}}%
\pgfpathcurveto{\pgfqpoint{0.872825in}{1.414637in}}{\pgfqpoint{0.880725in}{1.417909in}}{\pgfqpoint{0.886548in}{1.423733in}}%
\pgfpathcurveto{\pgfqpoint{0.892372in}{1.429557in}}{\pgfqpoint{0.895645in}{1.437457in}}{\pgfqpoint{0.895645in}{1.445694in}}%
\pgfpathcurveto{\pgfqpoint{0.895645in}{1.453930in}}{\pgfqpoint{0.892372in}{1.461830in}}{\pgfqpoint{0.886548in}{1.467654in}}%
\pgfpathcurveto{\pgfqpoint{0.880725in}{1.473478in}}{\pgfqpoint{0.872825in}{1.476750in}}{\pgfqpoint{0.864588in}{1.476750in}}%
\pgfpathcurveto{\pgfqpoint{0.856352in}{1.476750in}}{\pgfqpoint{0.848452in}{1.473478in}}{\pgfqpoint{0.842628in}{1.467654in}}%
\pgfpathcurveto{\pgfqpoint{0.836804in}{1.461830in}}{\pgfqpoint{0.833532in}{1.453930in}}{\pgfqpoint{0.833532in}{1.445694in}}%
\pgfpathcurveto{\pgfqpoint{0.833532in}{1.437457in}}{\pgfqpoint{0.836804in}{1.429557in}}{\pgfqpoint{0.842628in}{1.423733in}}%
\pgfpathcurveto{\pgfqpoint{0.848452in}{1.417909in}}{\pgfqpoint{0.856352in}{1.414637in}}{\pgfqpoint{0.864588in}{1.414637in}}%
\pgfpathclose%
\pgfusepath{stroke,fill}%
\end{pgfscope}%
\begin{pgfscope}%
\pgfpathrectangle{\pgfqpoint{0.100000in}{0.212622in}}{\pgfqpoint{3.696000in}{3.696000in}}%
\pgfusepath{clip}%
\pgfsetbuttcap%
\pgfsetroundjoin%
\definecolor{currentfill}{rgb}{0.121569,0.466667,0.705882}%
\pgfsetfillcolor{currentfill}%
\pgfsetfillopacity{0.657340}%
\pgfsetlinewidth{1.003750pt}%
\definecolor{currentstroke}{rgb}{0.121569,0.466667,0.705882}%
\pgfsetstrokecolor{currentstroke}%
\pgfsetstrokeopacity{0.657340}%
\pgfsetdash{}{0pt}%
\pgfpathmoveto{\pgfqpoint{0.864588in}{1.414637in}}%
\pgfpathcurveto{\pgfqpoint{0.872825in}{1.414637in}}{\pgfqpoint{0.880725in}{1.417909in}}{\pgfqpoint{0.886548in}{1.423733in}}%
\pgfpathcurveto{\pgfqpoint{0.892372in}{1.429557in}}{\pgfqpoint{0.895645in}{1.437457in}}{\pgfqpoint{0.895645in}{1.445694in}}%
\pgfpathcurveto{\pgfqpoint{0.895645in}{1.453930in}}{\pgfqpoint{0.892372in}{1.461830in}}{\pgfqpoint{0.886548in}{1.467654in}}%
\pgfpathcurveto{\pgfqpoint{0.880725in}{1.473478in}}{\pgfqpoint{0.872825in}{1.476750in}}{\pgfqpoint{0.864588in}{1.476750in}}%
\pgfpathcurveto{\pgfqpoint{0.856352in}{1.476750in}}{\pgfqpoint{0.848452in}{1.473478in}}{\pgfqpoint{0.842628in}{1.467654in}}%
\pgfpathcurveto{\pgfqpoint{0.836804in}{1.461830in}}{\pgfqpoint{0.833532in}{1.453930in}}{\pgfqpoint{0.833532in}{1.445694in}}%
\pgfpathcurveto{\pgfqpoint{0.833532in}{1.437457in}}{\pgfqpoint{0.836804in}{1.429557in}}{\pgfqpoint{0.842628in}{1.423733in}}%
\pgfpathcurveto{\pgfqpoint{0.848452in}{1.417909in}}{\pgfqpoint{0.856352in}{1.414637in}}{\pgfqpoint{0.864588in}{1.414637in}}%
\pgfpathclose%
\pgfusepath{stroke,fill}%
\end{pgfscope}%
\begin{pgfscope}%
\pgfpathrectangle{\pgfqpoint{0.100000in}{0.212622in}}{\pgfqpoint{3.696000in}{3.696000in}}%
\pgfusepath{clip}%
\pgfsetbuttcap%
\pgfsetroundjoin%
\definecolor{currentfill}{rgb}{0.121569,0.466667,0.705882}%
\pgfsetfillcolor{currentfill}%
\pgfsetfillopacity{0.657340}%
\pgfsetlinewidth{1.003750pt}%
\definecolor{currentstroke}{rgb}{0.121569,0.466667,0.705882}%
\pgfsetstrokecolor{currentstroke}%
\pgfsetstrokeopacity{0.657340}%
\pgfsetdash{}{0pt}%
\pgfpathmoveto{\pgfqpoint{0.864588in}{1.414637in}}%
\pgfpathcurveto{\pgfqpoint{0.872825in}{1.414637in}}{\pgfqpoint{0.880725in}{1.417909in}}{\pgfqpoint{0.886548in}{1.423733in}}%
\pgfpathcurveto{\pgfqpoint{0.892372in}{1.429557in}}{\pgfqpoint{0.895645in}{1.437457in}}{\pgfqpoint{0.895645in}{1.445694in}}%
\pgfpathcurveto{\pgfqpoint{0.895645in}{1.453930in}}{\pgfqpoint{0.892372in}{1.461830in}}{\pgfqpoint{0.886548in}{1.467654in}}%
\pgfpathcurveto{\pgfqpoint{0.880725in}{1.473478in}}{\pgfqpoint{0.872825in}{1.476750in}}{\pgfqpoint{0.864588in}{1.476750in}}%
\pgfpathcurveto{\pgfqpoint{0.856352in}{1.476750in}}{\pgfqpoint{0.848452in}{1.473478in}}{\pgfqpoint{0.842628in}{1.467654in}}%
\pgfpathcurveto{\pgfqpoint{0.836804in}{1.461830in}}{\pgfqpoint{0.833532in}{1.453930in}}{\pgfqpoint{0.833532in}{1.445694in}}%
\pgfpathcurveto{\pgfqpoint{0.833532in}{1.437457in}}{\pgfqpoint{0.836804in}{1.429557in}}{\pgfqpoint{0.842628in}{1.423733in}}%
\pgfpathcurveto{\pgfqpoint{0.848452in}{1.417909in}}{\pgfqpoint{0.856352in}{1.414637in}}{\pgfqpoint{0.864588in}{1.414637in}}%
\pgfpathclose%
\pgfusepath{stroke,fill}%
\end{pgfscope}%
\begin{pgfscope}%
\pgfpathrectangle{\pgfqpoint{0.100000in}{0.212622in}}{\pgfqpoint{3.696000in}{3.696000in}}%
\pgfusepath{clip}%
\pgfsetbuttcap%
\pgfsetroundjoin%
\definecolor{currentfill}{rgb}{0.121569,0.466667,0.705882}%
\pgfsetfillcolor{currentfill}%
\pgfsetfillopacity{0.657340}%
\pgfsetlinewidth{1.003750pt}%
\definecolor{currentstroke}{rgb}{0.121569,0.466667,0.705882}%
\pgfsetstrokecolor{currentstroke}%
\pgfsetstrokeopacity{0.657340}%
\pgfsetdash{}{0pt}%
\pgfpathmoveto{\pgfqpoint{0.864588in}{1.414637in}}%
\pgfpathcurveto{\pgfqpoint{0.872825in}{1.414637in}}{\pgfqpoint{0.880725in}{1.417909in}}{\pgfqpoint{0.886548in}{1.423733in}}%
\pgfpathcurveto{\pgfqpoint{0.892372in}{1.429557in}}{\pgfqpoint{0.895645in}{1.437457in}}{\pgfqpoint{0.895645in}{1.445694in}}%
\pgfpathcurveto{\pgfqpoint{0.895645in}{1.453930in}}{\pgfqpoint{0.892372in}{1.461830in}}{\pgfqpoint{0.886548in}{1.467654in}}%
\pgfpathcurveto{\pgfqpoint{0.880725in}{1.473478in}}{\pgfqpoint{0.872825in}{1.476750in}}{\pgfqpoint{0.864588in}{1.476750in}}%
\pgfpathcurveto{\pgfqpoint{0.856352in}{1.476750in}}{\pgfqpoint{0.848452in}{1.473478in}}{\pgfqpoint{0.842628in}{1.467654in}}%
\pgfpathcurveto{\pgfqpoint{0.836804in}{1.461830in}}{\pgfqpoint{0.833532in}{1.453930in}}{\pgfqpoint{0.833532in}{1.445694in}}%
\pgfpathcurveto{\pgfqpoint{0.833532in}{1.437457in}}{\pgfqpoint{0.836804in}{1.429557in}}{\pgfqpoint{0.842628in}{1.423733in}}%
\pgfpathcurveto{\pgfqpoint{0.848452in}{1.417909in}}{\pgfqpoint{0.856352in}{1.414637in}}{\pgfqpoint{0.864588in}{1.414637in}}%
\pgfpathclose%
\pgfusepath{stroke,fill}%
\end{pgfscope}%
\begin{pgfscope}%
\pgfpathrectangle{\pgfqpoint{0.100000in}{0.212622in}}{\pgfqpoint{3.696000in}{3.696000in}}%
\pgfusepath{clip}%
\pgfsetbuttcap%
\pgfsetroundjoin%
\definecolor{currentfill}{rgb}{0.121569,0.466667,0.705882}%
\pgfsetfillcolor{currentfill}%
\pgfsetfillopacity{0.657452}%
\pgfsetlinewidth{1.003750pt}%
\definecolor{currentstroke}{rgb}{0.121569,0.466667,0.705882}%
\pgfsetstrokecolor{currentstroke}%
\pgfsetstrokeopacity{0.657452}%
\pgfsetdash{}{0pt}%
\pgfpathmoveto{\pgfqpoint{0.864035in}{1.414458in}}%
\pgfpathcurveto{\pgfqpoint{0.872271in}{1.414458in}}{\pgfqpoint{0.880171in}{1.417730in}}{\pgfqpoint{0.885995in}{1.423554in}}%
\pgfpathcurveto{\pgfqpoint{0.891819in}{1.429378in}}{\pgfqpoint{0.895091in}{1.437278in}}{\pgfqpoint{0.895091in}{1.445514in}}%
\pgfpathcurveto{\pgfqpoint{0.895091in}{1.453751in}}{\pgfqpoint{0.891819in}{1.461651in}}{\pgfqpoint{0.885995in}{1.467475in}}%
\pgfpathcurveto{\pgfqpoint{0.880171in}{1.473299in}}{\pgfqpoint{0.872271in}{1.476571in}}{\pgfqpoint{0.864035in}{1.476571in}}%
\pgfpathcurveto{\pgfqpoint{0.855798in}{1.476571in}}{\pgfqpoint{0.847898in}{1.473299in}}{\pgfqpoint{0.842074in}{1.467475in}}%
\pgfpathcurveto{\pgfqpoint{0.836250in}{1.461651in}}{\pgfqpoint{0.832978in}{1.453751in}}{\pgfqpoint{0.832978in}{1.445514in}}%
\pgfpathcurveto{\pgfqpoint{0.832978in}{1.437278in}}{\pgfqpoint{0.836250in}{1.429378in}}{\pgfqpoint{0.842074in}{1.423554in}}%
\pgfpathcurveto{\pgfqpoint{0.847898in}{1.417730in}}{\pgfqpoint{0.855798in}{1.414458in}}{\pgfqpoint{0.864035in}{1.414458in}}%
\pgfpathclose%
\pgfusepath{stroke,fill}%
\end{pgfscope}%
\begin{pgfscope}%
\pgfpathrectangle{\pgfqpoint{0.100000in}{0.212622in}}{\pgfqpoint{3.696000in}{3.696000in}}%
\pgfusepath{clip}%
\pgfsetbuttcap%
\pgfsetroundjoin%
\definecolor{currentfill}{rgb}{0.121569,0.466667,0.705882}%
\pgfsetfillcolor{currentfill}%
\pgfsetfillopacity{0.657530}%
\pgfsetlinewidth{1.003750pt}%
\definecolor{currentstroke}{rgb}{0.121569,0.466667,0.705882}%
\pgfsetstrokecolor{currentstroke}%
\pgfsetstrokeopacity{0.657530}%
\pgfsetdash{}{0pt}%
\pgfpathmoveto{\pgfqpoint{2.145438in}{1.742421in}}%
\pgfpathcurveto{\pgfqpoint{2.153674in}{1.742421in}}{\pgfqpoint{2.161574in}{1.745693in}}{\pgfqpoint{2.167398in}{1.751517in}}%
\pgfpathcurveto{\pgfqpoint{2.173222in}{1.757341in}}{\pgfqpoint{2.176495in}{1.765241in}}{\pgfqpoint{2.176495in}{1.773477in}}%
\pgfpathcurveto{\pgfqpoint{2.176495in}{1.781714in}}{\pgfqpoint{2.173222in}{1.789614in}}{\pgfqpoint{2.167398in}{1.795438in}}%
\pgfpathcurveto{\pgfqpoint{2.161574in}{1.801262in}}{\pgfqpoint{2.153674in}{1.804534in}}{\pgfqpoint{2.145438in}{1.804534in}}%
\pgfpathcurveto{\pgfqpoint{2.137202in}{1.804534in}}{\pgfqpoint{2.129302in}{1.801262in}}{\pgfqpoint{2.123478in}{1.795438in}}%
\pgfpathcurveto{\pgfqpoint{2.117654in}{1.789614in}}{\pgfqpoint{2.114382in}{1.781714in}}{\pgfqpoint{2.114382in}{1.773477in}}%
\pgfpathcurveto{\pgfqpoint{2.114382in}{1.765241in}}{\pgfqpoint{2.117654in}{1.757341in}}{\pgfqpoint{2.123478in}{1.751517in}}%
\pgfpathcurveto{\pgfqpoint{2.129302in}{1.745693in}}{\pgfqpoint{2.137202in}{1.742421in}}{\pgfqpoint{2.145438in}{1.742421in}}%
\pgfpathclose%
\pgfusepath{stroke,fill}%
\end{pgfscope}%
\begin{pgfscope}%
\pgfpathrectangle{\pgfqpoint{0.100000in}{0.212622in}}{\pgfqpoint{3.696000in}{3.696000in}}%
\pgfusepath{clip}%
\pgfsetbuttcap%
\pgfsetroundjoin%
\definecolor{currentfill}{rgb}{0.121569,0.466667,0.705882}%
\pgfsetfillcolor{currentfill}%
\pgfsetfillopacity{0.657567}%
\pgfsetlinewidth{1.003750pt}%
\definecolor{currentstroke}{rgb}{0.121569,0.466667,0.705882}%
\pgfsetstrokecolor{currentstroke}%
\pgfsetstrokeopacity{0.657567}%
\pgfsetdash{}{0pt}%
\pgfpathmoveto{\pgfqpoint{0.855887in}{1.414694in}}%
\pgfpathcurveto{\pgfqpoint{0.864123in}{1.414694in}}{\pgfqpoint{0.872023in}{1.417966in}}{\pgfqpoint{0.877847in}{1.423790in}}%
\pgfpathcurveto{\pgfqpoint{0.883671in}{1.429614in}}{\pgfqpoint{0.886943in}{1.437514in}}{\pgfqpoint{0.886943in}{1.445750in}}%
\pgfpathcurveto{\pgfqpoint{0.886943in}{1.453986in}}{\pgfqpoint{0.883671in}{1.461886in}}{\pgfqpoint{0.877847in}{1.467710in}}%
\pgfpathcurveto{\pgfqpoint{0.872023in}{1.473534in}}{\pgfqpoint{0.864123in}{1.476807in}}{\pgfqpoint{0.855887in}{1.476807in}}%
\pgfpathcurveto{\pgfqpoint{0.847651in}{1.476807in}}{\pgfqpoint{0.839750in}{1.473534in}}{\pgfqpoint{0.833927in}{1.467710in}}%
\pgfpathcurveto{\pgfqpoint{0.828103in}{1.461886in}}{\pgfqpoint{0.824830in}{1.453986in}}{\pgfqpoint{0.824830in}{1.445750in}}%
\pgfpathcurveto{\pgfqpoint{0.824830in}{1.437514in}}{\pgfqpoint{0.828103in}{1.429614in}}{\pgfqpoint{0.833927in}{1.423790in}}%
\pgfpathcurveto{\pgfqpoint{0.839750in}{1.417966in}}{\pgfqpoint{0.847651in}{1.414694in}}{\pgfqpoint{0.855887in}{1.414694in}}%
\pgfpathclose%
\pgfusepath{stroke,fill}%
\end{pgfscope}%
\begin{pgfscope}%
\pgfpathrectangle{\pgfqpoint{0.100000in}{0.212622in}}{\pgfqpoint{3.696000in}{3.696000in}}%
\pgfusepath{clip}%
\pgfsetbuttcap%
\pgfsetroundjoin%
\definecolor{currentfill}{rgb}{0.121569,0.466667,0.705882}%
\pgfsetfillcolor{currentfill}%
\pgfsetfillopacity{0.657636}%
\pgfsetlinewidth{1.003750pt}%
\definecolor{currentstroke}{rgb}{0.121569,0.466667,0.705882}%
\pgfsetstrokecolor{currentstroke}%
\pgfsetstrokeopacity{0.657636}%
\pgfsetdash{}{0pt}%
\pgfpathmoveto{\pgfqpoint{0.862639in}{1.414214in}}%
\pgfpathcurveto{\pgfqpoint{0.870875in}{1.414214in}}{\pgfqpoint{0.878775in}{1.417487in}}{\pgfqpoint{0.884599in}{1.423311in}}%
\pgfpathcurveto{\pgfqpoint{0.890423in}{1.429135in}}{\pgfqpoint{0.893695in}{1.437035in}}{\pgfqpoint{0.893695in}{1.445271in}}%
\pgfpathcurveto{\pgfqpoint{0.893695in}{1.453507in}}{\pgfqpoint{0.890423in}{1.461407in}}{\pgfqpoint{0.884599in}{1.467231in}}%
\pgfpathcurveto{\pgfqpoint{0.878775in}{1.473055in}}{\pgfqpoint{0.870875in}{1.476327in}}{\pgfqpoint{0.862639in}{1.476327in}}%
\pgfpathcurveto{\pgfqpoint{0.854402in}{1.476327in}}{\pgfqpoint{0.846502in}{1.473055in}}{\pgfqpoint{0.840678in}{1.467231in}}%
\pgfpathcurveto{\pgfqpoint{0.834855in}{1.461407in}}{\pgfqpoint{0.831582in}{1.453507in}}{\pgfqpoint{0.831582in}{1.445271in}}%
\pgfpathcurveto{\pgfqpoint{0.831582in}{1.437035in}}{\pgfqpoint{0.834855in}{1.429135in}}{\pgfqpoint{0.840678in}{1.423311in}}%
\pgfpathcurveto{\pgfqpoint{0.846502in}{1.417487in}}{\pgfqpoint{0.854402in}{1.414214in}}{\pgfqpoint{0.862639in}{1.414214in}}%
\pgfpathclose%
\pgfusepath{stroke,fill}%
\end{pgfscope}%
\begin{pgfscope}%
\pgfpathrectangle{\pgfqpoint{0.100000in}{0.212622in}}{\pgfqpoint{3.696000in}{3.696000in}}%
\pgfusepath{clip}%
\pgfsetbuttcap%
\pgfsetroundjoin%
\definecolor{currentfill}{rgb}{0.121569,0.466667,0.705882}%
\pgfsetfillcolor{currentfill}%
\pgfsetfillopacity{0.657796}%
\pgfsetlinewidth{1.003750pt}%
\definecolor{currentstroke}{rgb}{0.121569,0.466667,0.705882}%
\pgfsetstrokecolor{currentstroke}%
\pgfsetstrokeopacity{0.657796}%
\pgfsetdash{}{0pt}%
\pgfpathmoveto{\pgfqpoint{0.860032in}{1.414118in}}%
\pgfpathcurveto{\pgfqpoint{0.868268in}{1.414118in}}{\pgfqpoint{0.876168in}{1.417390in}}{\pgfqpoint{0.881992in}{1.423214in}}%
\pgfpathcurveto{\pgfqpoint{0.887816in}{1.429038in}}{\pgfqpoint{0.891089in}{1.436938in}}{\pgfqpoint{0.891089in}{1.445174in}}%
\pgfpathcurveto{\pgfqpoint{0.891089in}{1.453410in}}{\pgfqpoint{0.887816in}{1.461311in}}{\pgfqpoint{0.881992in}{1.467134in}}%
\pgfpathcurveto{\pgfqpoint{0.876168in}{1.472958in}}{\pgfqpoint{0.868268in}{1.476231in}}{\pgfqpoint{0.860032in}{1.476231in}}%
\pgfpathcurveto{\pgfqpoint{0.851796in}{1.476231in}}{\pgfqpoint{0.843896in}{1.472958in}}{\pgfqpoint{0.838072in}{1.467134in}}%
\pgfpathcurveto{\pgfqpoint{0.832248in}{1.461311in}}{\pgfqpoint{0.828976in}{1.453410in}}{\pgfqpoint{0.828976in}{1.445174in}}%
\pgfpathcurveto{\pgfqpoint{0.828976in}{1.436938in}}{\pgfqpoint{0.832248in}{1.429038in}}{\pgfqpoint{0.838072in}{1.423214in}}%
\pgfpathcurveto{\pgfqpoint{0.843896in}{1.417390in}}{\pgfqpoint{0.851796in}{1.414118in}}{\pgfqpoint{0.860032in}{1.414118in}}%
\pgfpathclose%
\pgfusepath{stroke,fill}%
\end{pgfscope}%
\begin{pgfscope}%
\pgfpathrectangle{\pgfqpoint{0.100000in}{0.212622in}}{\pgfqpoint{3.696000in}{3.696000in}}%
\pgfusepath{clip}%
\pgfsetbuttcap%
\pgfsetroundjoin%
\definecolor{currentfill}{rgb}{0.121569,0.466667,0.705882}%
\pgfsetfillcolor{currentfill}%
\pgfsetfillopacity{0.662081}%
\pgfsetlinewidth{1.003750pt}%
\definecolor{currentstroke}{rgb}{0.121569,0.466667,0.705882}%
\pgfsetstrokecolor{currentstroke}%
\pgfsetstrokeopacity{0.662081}%
\pgfsetdash{}{0pt}%
\pgfpathmoveto{\pgfqpoint{2.147696in}{1.737746in}}%
\pgfpathcurveto{\pgfqpoint{2.155933in}{1.737746in}}{\pgfqpoint{2.163833in}{1.741018in}}{\pgfqpoint{2.169657in}{1.746842in}}%
\pgfpathcurveto{\pgfqpoint{2.175481in}{1.752666in}}{\pgfqpoint{2.178753in}{1.760566in}}{\pgfqpoint{2.178753in}{1.768802in}}%
\pgfpathcurveto{\pgfqpoint{2.178753in}{1.777039in}}{\pgfqpoint{2.175481in}{1.784939in}}{\pgfqpoint{2.169657in}{1.790763in}}%
\pgfpathcurveto{\pgfqpoint{2.163833in}{1.796587in}}{\pgfqpoint{2.155933in}{1.799859in}}{\pgfqpoint{2.147696in}{1.799859in}}%
\pgfpathcurveto{\pgfqpoint{2.139460in}{1.799859in}}{\pgfqpoint{2.131560in}{1.796587in}}{\pgfqpoint{2.125736in}{1.790763in}}%
\pgfpathcurveto{\pgfqpoint{2.119912in}{1.784939in}}{\pgfqpoint{2.116640in}{1.777039in}}{\pgfqpoint{2.116640in}{1.768802in}}%
\pgfpathcurveto{\pgfqpoint{2.116640in}{1.760566in}}{\pgfqpoint{2.119912in}{1.752666in}}{\pgfqpoint{2.125736in}{1.746842in}}%
\pgfpathcurveto{\pgfqpoint{2.131560in}{1.741018in}}{\pgfqpoint{2.139460in}{1.737746in}}{\pgfqpoint{2.147696in}{1.737746in}}%
\pgfpathclose%
\pgfusepath{stroke,fill}%
\end{pgfscope}%
\begin{pgfscope}%
\pgfpathrectangle{\pgfqpoint{0.100000in}{0.212622in}}{\pgfqpoint{3.696000in}{3.696000in}}%
\pgfusepath{clip}%
\pgfsetbuttcap%
\pgfsetroundjoin%
\definecolor{currentfill}{rgb}{0.121569,0.466667,0.705882}%
\pgfsetfillcolor{currentfill}%
\pgfsetfillopacity{0.666939}%
\pgfsetlinewidth{1.003750pt}%
\definecolor{currentstroke}{rgb}{0.121569,0.466667,0.705882}%
\pgfsetstrokecolor{currentstroke}%
\pgfsetstrokeopacity{0.666939}%
\pgfsetdash{}{0pt}%
\pgfpathmoveto{\pgfqpoint{2.151038in}{1.730655in}}%
\pgfpathcurveto{\pgfqpoint{2.159274in}{1.730655in}}{\pgfqpoint{2.167174in}{1.733927in}}{\pgfqpoint{2.172998in}{1.739751in}}%
\pgfpathcurveto{\pgfqpoint{2.178822in}{1.745575in}}{\pgfqpoint{2.182095in}{1.753475in}}{\pgfqpoint{2.182095in}{1.761711in}}%
\pgfpathcurveto{\pgfqpoint{2.182095in}{1.769947in}}{\pgfqpoint{2.178822in}{1.777847in}}{\pgfqpoint{2.172998in}{1.783671in}}%
\pgfpathcurveto{\pgfqpoint{2.167174in}{1.789495in}}{\pgfqpoint{2.159274in}{1.792768in}}{\pgfqpoint{2.151038in}{1.792768in}}%
\pgfpathcurveto{\pgfqpoint{2.142802in}{1.792768in}}{\pgfqpoint{2.134902in}{1.789495in}}{\pgfqpoint{2.129078in}{1.783671in}}%
\pgfpathcurveto{\pgfqpoint{2.123254in}{1.777847in}}{\pgfqpoint{2.119982in}{1.769947in}}{\pgfqpoint{2.119982in}{1.761711in}}%
\pgfpathcurveto{\pgfqpoint{2.119982in}{1.753475in}}{\pgfqpoint{2.123254in}{1.745575in}}{\pgfqpoint{2.129078in}{1.739751in}}%
\pgfpathcurveto{\pgfqpoint{2.134902in}{1.733927in}}{\pgfqpoint{2.142802in}{1.730655in}}{\pgfqpoint{2.151038in}{1.730655in}}%
\pgfpathclose%
\pgfusepath{stroke,fill}%
\end{pgfscope}%
\begin{pgfscope}%
\pgfpathrectangle{\pgfqpoint{0.100000in}{0.212622in}}{\pgfqpoint{3.696000in}{3.696000in}}%
\pgfusepath{clip}%
\pgfsetbuttcap%
\pgfsetroundjoin%
\definecolor{currentfill}{rgb}{0.121569,0.466667,0.705882}%
\pgfsetfillcolor{currentfill}%
\pgfsetfillopacity{0.672707}%
\pgfsetlinewidth{1.003750pt}%
\definecolor{currentstroke}{rgb}{0.121569,0.466667,0.705882}%
\pgfsetstrokecolor{currentstroke}%
\pgfsetstrokeopacity{0.672707}%
\pgfsetdash{}{0pt}%
\pgfpathmoveto{\pgfqpoint{2.157478in}{1.728438in}}%
\pgfpathcurveto{\pgfqpoint{2.165715in}{1.728438in}}{\pgfqpoint{2.173615in}{1.731710in}}{\pgfqpoint{2.179439in}{1.737534in}}%
\pgfpathcurveto{\pgfqpoint{2.185262in}{1.743358in}}{\pgfqpoint{2.188535in}{1.751258in}}{\pgfqpoint{2.188535in}{1.759494in}}%
\pgfpathcurveto{\pgfqpoint{2.188535in}{1.767731in}}{\pgfqpoint{2.185262in}{1.775631in}}{\pgfqpoint{2.179439in}{1.781455in}}%
\pgfpathcurveto{\pgfqpoint{2.173615in}{1.787278in}}{\pgfqpoint{2.165715in}{1.790551in}}{\pgfqpoint{2.157478in}{1.790551in}}%
\pgfpathcurveto{\pgfqpoint{2.149242in}{1.790551in}}{\pgfqpoint{2.141342in}{1.787278in}}{\pgfqpoint{2.135518in}{1.781455in}}%
\pgfpathcurveto{\pgfqpoint{2.129694in}{1.775631in}}{\pgfqpoint{2.126422in}{1.767731in}}{\pgfqpoint{2.126422in}{1.759494in}}%
\pgfpathcurveto{\pgfqpoint{2.126422in}{1.751258in}}{\pgfqpoint{2.129694in}{1.743358in}}{\pgfqpoint{2.135518in}{1.737534in}}%
\pgfpathcurveto{\pgfqpoint{2.141342in}{1.731710in}}{\pgfqpoint{2.149242in}{1.728438in}}{\pgfqpoint{2.157478in}{1.728438in}}%
\pgfpathclose%
\pgfusepath{stroke,fill}%
\end{pgfscope}%
\begin{pgfscope}%
\pgfpathrectangle{\pgfqpoint{0.100000in}{0.212622in}}{\pgfqpoint{3.696000in}{3.696000in}}%
\pgfusepath{clip}%
\pgfsetbuttcap%
\pgfsetroundjoin%
\definecolor{currentfill}{rgb}{0.121569,0.466667,0.705882}%
\pgfsetfillcolor{currentfill}%
\pgfsetfillopacity{0.678967}%
\pgfsetlinewidth{1.003750pt}%
\definecolor{currentstroke}{rgb}{0.121569,0.466667,0.705882}%
\pgfsetstrokecolor{currentstroke}%
\pgfsetstrokeopacity{0.678967}%
\pgfsetdash{}{0pt}%
\pgfpathmoveto{\pgfqpoint{2.160594in}{1.721348in}}%
\pgfpathcurveto{\pgfqpoint{2.168831in}{1.721348in}}{\pgfqpoint{2.176731in}{1.724621in}}{\pgfqpoint{2.182555in}{1.730445in}}%
\pgfpathcurveto{\pgfqpoint{2.188378in}{1.736269in}}{\pgfqpoint{2.191651in}{1.744169in}}{\pgfqpoint{2.191651in}{1.752405in}}%
\pgfpathcurveto{\pgfqpoint{2.191651in}{1.760641in}}{\pgfqpoint{2.188378in}{1.768541in}}{\pgfqpoint{2.182555in}{1.774365in}}%
\pgfpathcurveto{\pgfqpoint{2.176731in}{1.780189in}}{\pgfqpoint{2.168831in}{1.783461in}}{\pgfqpoint{2.160594in}{1.783461in}}%
\pgfpathcurveto{\pgfqpoint{2.152358in}{1.783461in}}{\pgfqpoint{2.144458in}{1.780189in}}{\pgfqpoint{2.138634in}{1.774365in}}%
\pgfpathcurveto{\pgfqpoint{2.132810in}{1.768541in}}{\pgfqpoint{2.129538in}{1.760641in}}{\pgfqpoint{2.129538in}{1.752405in}}%
\pgfpathcurveto{\pgfqpoint{2.129538in}{1.744169in}}{\pgfqpoint{2.132810in}{1.736269in}}{\pgfqpoint{2.138634in}{1.730445in}}%
\pgfpathcurveto{\pgfqpoint{2.144458in}{1.724621in}}{\pgfqpoint{2.152358in}{1.721348in}}{\pgfqpoint{2.160594in}{1.721348in}}%
\pgfpathclose%
\pgfusepath{stroke,fill}%
\end{pgfscope}%
\begin{pgfscope}%
\pgfpathrectangle{\pgfqpoint{0.100000in}{0.212622in}}{\pgfqpoint{3.696000in}{3.696000in}}%
\pgfusepath{clip}%
\pgfsetbuttcap%
\pgfsetroundjoin%
\definecolor{currentfill}{rgb}{0.121569,0.466667,0.705882}%
\pgfsetfillcolor{currentfill}%
\pgfsetfillopacity{0.684971}%
\pgfsetlinewidth{1.003750pt}%
\definecolor{currentstroke}{rgb}{0.121569,0.466667,0.705882}%
\pgfsetstrokecolor{currentstroke}%
\pgfsetstrokeopacity{0.684971}%
\pgfsetdash{}{0pt}%
\pgfpathmoveto{\pgfqpoint{2.163139in}{1.709374in}}%
\pgfpathcurveto{\pgfqpoint{2.171375in}{1.709374in}}{\pgfqpoint{2.179275in}{1.712646in}}{\pgfqpoint{2.185099in}{1.718470in}}%
\pgfpathcurveto{\pgfqpoint{2.190923in}{1.724294in}}{\pgfqpoint{2.194195in}{1.732194in}}{\pgfqpoint{2.194195in}{1.740430in}}%
\pgfpathcurveto{\pgfqpoint{2.194195in}{1.748667in}}{\pgfqpoint{2.190923in}{1.756567in}}{\pgfqpoint{2.185099in}{1.762391in}}%
\pgfpathcurveto{\pgfqpoint{2.179275in}{1.768215in}}{\pgfqpoint{2.171375in}{1.771487in}}{\pgfqpoint{2.163139in}{1.771487in}}%
\pgfpathcurveto{\pgfqpoint{2.154903in}{1.771487in}}{\pgfqpoint{2.147002in}{1.768215in}}{\pgfqpoint{2.141179in}{1.762391in}}%
\pgfpathcurveto{\pgfqpoint{2.135355in}{1.756567in}}{\pgfqpoint{2.132082in}{1.748667in}}{\pgfqpoint{2.132082in}{1.740430in}}%
\pgfpathcurveto{\pgfqpoint{2.132082in}{1.732194in}}{\pgfqpoint{2.135355in}{1.724294in}}{\pgfqpoint{2.141179in}{1.718470in}}%
\pgfpathcurveto{\pgfqpoint{2.147002in}{1.712646in}}{\pgfqpoint{2.154903in}{1.709374in}}{\pgfqpoint{2.163139in}{1.709374in}}%
\pgfpathclose%
\pgfusepath{stroke,fill}%
\end{pgfscope}%
\begin{pgfscope}%
\pgfpathrectangle{\pgfqpoint{0.100000in}{0.212622in}}{\pgfqpoint{3.696000in}{3.696000in}}%
\pgfusepath{clip}%
\pgfsetbuttcap%
\pgfsetroundjoin%
\definecolor{currentfill}{rgb}{0.121569,0.466667,0.705882}%
\pgfsetfillcolor{currentfill}%
\pgfsetfillopacity{0.691385}%
\pgfsetlinewidth{1.003750pt}%
\definecolor{currentstroke}{rgb}{0.121569,0.466667,0.705882}%
\pgfsetstrokecolor{currentstroke}%
\pgfsetstrokeopacity{0.691385}%
\pgfsetdash{}{0pt}%
\pgfpathmoveto{\pgfqpoint{2.167626in}{1.696952in}}%
\pgfpathcurveto{\pgfqpoint{2.175863in}{1.696952in}}{\pgfqpoint{2.183763in}{1.700224in}}{\pgfqpoint{2.189587in}{1.706048in}}%
\pgfpathcurveto{\pgfqpoint{2.195411in}{1.711872in}}{\pgfqpoint{2.198683in}{1.719772in}}{\pgfqpoint{2.198683in}{1.728008in}}%
\pgfpathcurveto{\pgfqpoint{2.198683in}{1.736245in}}{\pgfqpoint{2.195411in}{1.744145in}}{\pgfqpoint{2.189587in}{1.749969in}}%
\pgfpathcurveto{\pgfqpoint{2.183763in}{1.755793in}}{\pgfqpoint{2.175863in}{1.759065in}}{\pgfqpoint{2.167626in}{1.759065in}}%
\pgfpathcurveto{\pgfqpoint{2.159390in}{1.759065in}}{\pgfqpoint{2.151490in}{1.755793in}}{\pgfqpoint{2.145666in}{1.749969in}}%
\pgfpathcurveto{\pgfqpoint{2.139842in}{1.744145in}}{\pgfqpoint{2.136570in}{1.736245in}}{\pgfqpoint{2.136570in}{1.728008in}}%
\pgfpathcurveto{\pgfqpoint{2.136570in}{1.719772in}}{\pgfqpoint{2.139842in}{1.711872in}}{\pgfqpoint{2.145666in}{1.706048in}}%
\pgfpathcurveto{\pgfqpoint{2.151490in}{1.700224in}}{\pgfqpoint{2.159390in}{1.696952in}}{\pgfqpoint{2.167626in}{1.696952in}}%
\pgfpathclose%
\pgfusepath{stroke,fill}%
\end{pgfscope}%
\begin{pgfscope}%
\pgfpathrectangle{\pgfqpoint{0.100000in}{0.212622in}}{\pgfqpoint{3.696000in}{3.696000in}}%
\pgfusepath{clip}%
\pgfsetbuttcap%
\pgfsetroundjoin%
\definecolor{currentfill}{rgb}{0.121569,0.466667,0.705882}%
\pgfsetfillcolor{currentfill}%
\pgfsetfillopacity{0.699703}%
\pgfsetlinewidth{1.003750pt}%
\definecolor{currentstroke}{rgb}{0.121569,0.466667,0.705882}%
\pgfsetstrokecolor{currentstroke}%
\pgfsetstrokeopacity{0.699703}%
\pgfsetdash{}{0pt}%
\pgfpathmoveto{\pgfqpoint{2.175747in}{1.697756in}}%
\pgfpathcurveto{\pgfqpoint{2.183984in}{1.697756in}}{\pgfqpoint{2.191884in}{1.701029in}}{\pgfqpoint{2.197708in}{1.706852in}}%
\pgfpathcurveto{\pgfqpoint{2.203532in}{1.712676in}}{\pgfqpoint{2.206804in}{1.720576in}}{\pgfqpoint{2.206804in}{1.728813in}}%
\pgfpathcurveto{\pgfqpoint{2.206804in}{1.737049in}}{\pgfqpoint{2.203532in}{1.744949in}}{\pgfqpoint{2.197708in}{1.750773in}}%
\pgfpathcurveto{\pgfqpoint{2.191884in}{1.756597in}}{\pgfqpoint{2.183984in}{1.759869in}}{\pgfqpoint{2.175747in}{1.759869in}}%
\pgfpathcurveto{\pgfqpoint{2.167511in}{1.759869in}}{\pgfqpoint{2.159611in}{1.756597in}}{\pgfqpoint{2.153787in}{1.750773in}}%
\pgfpathcurveto{\pgfqpoint{2.147963in}{1.744949in}}{\pgfqpoint{2.144691in}{1.737049in}}{\pgfqpoint{2.144691in}{1.728813in}}%
\pgfpathcurveto{\pgfqpoint{2.144691in}{1.720576in}}{\pgfqpoint{2.147963in}{1.712676in}}{\pgfqpoint{2.153787in}{1.706852in}}%
\pgfpathcurveto{\pgfqpoint{2.159611in}{1.701029in}}{\pgfqpoint{2.167511in}{1.697756in}}{\pgfqpoint{2.175747in}{1.697756in}}%
\pgfpathclose%
\pgfusepath{stroke,fill}%
\end{pgfscope}%
\begin{pgfscope}%
\pgfpathrectangle{\pgfqpoint{0.100000in}{0.212622in}}{\pgfqpoint{3.696000in}{3.696000in}}%
\pgfusepath{clip}%
\pgfsetbuttcap%
\pgfsetroundjoin%
\definecolor{currentfill}{rgb}{0.121569,0.466667,0.705882}%
\pgfsetfillcolor{currentfill}%
\pgfsetfillopacity{0.703700}%
\pgfsetlinewidth{1.003750pt}%
\definecolor{currentstroke}{rgb}{0.121569,0.466667,0.705882}%
\pgfsetstrokecolor{currentstroke}%
\pgfsetstrokeopacity{0.703700}%
\pgfsetdash{}{0pt}%
\pgfpathmoveto{\pgfqpoint{2.178160in}{1.693113in}}%
\pgfpathcurveto{\pgfqpoint{2.186396in}{1.693113in}}{\pgfqpoint{2.194296in}{1.696385in}}{\pgfqpoint{2.200120in}{1.702209in}}%
\pgfpathcurveto{\pgfqpoint{2.205944in}{1.708033in}}{\pgfqpoint{2.209217in}{1.715933in}}{\pgfqpoint{2.209217in}{1.724169in}}%
\pgfpathcurveto{\pgfqpoint{2.209217in}{1.732405in}}{\pgfqpoint{2.205944in}{1.740305in}}{\pgfqpoint{2.200120in}{1.746129in}}%
\pgfpathcurveto{\pgfqpoint{2.194296in}{1.751953in}}{\pgfqpoint{2.186396in}{1.755226in}}{\pgfqpoint{2.178160in}{1.755226in}}%
\pgfpathcurveto{\pgfqpoint{2.169924in}{1.755226in}}{\pgfqpoint{2.162024in}{1.751953in}}{\pgfqpoint{2.156200in}{1.746129in}}%
\pgfpathcurveto{\pgfqpoint{2.150376in}{1.740305in}}{\pgfqpoint{2.147104in}{1.732405in}}{\pgfqpoint{2.147104in}{1.724169in}}%
\pgfpathcurveto{\pgfqpoint{2.147104in}{1.715933in}}{\pgfqpoint{2.150376in}{1.708033in}}{\pgfqpoint{2.156200in}{1.702209in}}%
\pgfpathcurveto{\pgfqpoint{2.162024in}{1.696385in}}{\pgfqpoint{2.169924in}{1.693113in}}{\pgfqpoint{2.178160in}{1.693113in}}%
\pgfpathclose%
\pgfusepath{stroke,fill}%
\end{pgfscope}%
\begin{pgfscope}%
\pgfpathrectangle{\pgfqpoint{0.100000in}{0.212622in}}{\pgfqpoint{3.696000in}{3.696000in}}%
\pgfusepath{clip}%
\pgfsetbuttcap%
\pgfsetroundjoin%
\definecolor{currentfill}{rgb}{0.121569,0.466667,0.705882}%
\pgfsetfillcolor{currentfill}%
\pgfsetfillopacity{0.707091}%
\pgfsetlinewidth{1.003750pt}%
\definecolor{currentstroke}{rgb}{0.121569,0.466667,0.705882}%
\pgfsetstrokecolor{currentstroke}%
\pgfsetstrokeopacity{0.707091}%
\pgfsetdash{}{0pt}%
\pgfpathmoveto{\pgfqpoint{2.181012in}{1.683663in}}%
\pgfpathcurveto{\pgfqpoint{2.189248in}{1.683663in}}{\pgfqpoint{2.197148in}{1.686935in}}{\pgfqpoint{2.202972in}{1.692759in}}%
\pgfpathcurveto{\pgfqpoint{2.208796in}{1.698583in}}{\pgfqpoint{2.212068in}{1.706483in}}{\pgfqpoint{2.212068in}{1.714719in}}%
\pgfpathcurveto{\pgfqpoint{2.212068in}{1.722955in}}{\pgfqpoint{2.208796in}{1.730855in}}{\pgfqpoint{2.202972in}{1.736679in}}%
\pgfpathcurveto{\pgfqpoint{2.197148in}{1.742503in}}{\pgfqpoint{2.189248in}{1.745776in}}{\pgfqpoint{2.181012in}{1.745776in}}%
\pgfpathcurveto{\pgfqpoint{2.172776in}{1.745776in}}{\pgfqpoint{2.164875in}{1.742503in}}{\pgfqpoint{2.159052in}{1.736679in}}%
\pgfpathcurveto{\pgfqpoint{2.153228in}{1.730855in}}{\pgfqpoint{2.149955in}{1.722955in}}{\pgfqpoint{2.149955in}{1.714719in}}%
\pgfpathcurveto{\pgfqpoint{2.149955in}{1.706483in}}{\pgfqpoint{2.153228in}{1.698583in}}{\pgfqpoint{2.159052in}{1.692759in}}%
\pgfpathcurveto{\pgfqpoint{2.164875in}{1.686935in}}{\pgfqpoint{2.172776in}{1.683663in}}{\pgfqpoint{2.181012in}{1.683663in}}%
\pgfpathclose%
\pgfusepath{stroke,fill}%
\end{pgfscope}%
\begin{pgfscope}%
\pgfpathrectangle{\pgfqpoint{0.100000in}{0.212622in}}{\pgfqpoint{3.696000in}{3.696000in}}%
\pgfusepath{clip}%
\pgfsetbuttcap%
\pgfsetroundjoin%
\definecolor{currentfill}{rgb}{0.121569,0.466667,0.705882}%
\pgfsetfillcolor{currentfill}%
\pgfsetfillopacity{0.709203}%
\pgfsetlinewidth{1.003750pt}%
\definecolor{currentstroke}{rgb}{0.121569,0.466667,0.705882}%
\pgfsetstrokecolor{currentstroke}%
\pgfsetstrokeopacity{0.709203}%
\pgfsetdash{}{0pt}%
\pgfpathmoveto{\pgfqpoint{2.183147in}{1.680373in}}%
\pgfpathcurveto{\pgfqpoint{2.191384in}{1.680373in}}{\pgfqpoint{2.199284in}{1.683646in}}{\pgfqpoint{2.205108in}{1.689470in}}%
\pgfpathcurveto{\pgfqpoint{2.210931in}{1.695294in}}{\pgfqpoint{2.214204in}{1.703194in}}{\pgfqpoint{2.214204in}{1.711430in}}%
\pgfpathcurveto{\pgfqpoint{2.214204in}{1.719666in}}{\pgfqpoint{2.210931in}{1.727566in}}{\pgfqpoint{2.205108in}{1.733390in}}%
\pgfpathcurveto{\pgfqpoint{2.199284in}{1.739214in}}{\pgfqpoint{2.191384in}{1.742486in}}{\pgfqpoint{2.183147in}{1.742486in}}%
\pgfpathcurveto{\pgfqpoint{2.174911in}{1.742486in}}{\pgfqpoint{2.167011in}{1.739214in}}{\pgfqpoint{2.161187in}{1.733390in}}%
\pgfpathcurveto{\pgfqpoint{2.155363in}{1.727566in}}{\pgfqpoint{2.152091in}{1.719666in}}{\pgfqpoint{2.152091in}{1.711430in}}%
\pgfpathcurveto{\pgfqpoint{2.152091in}{1.703194in}}{\pgfqpoint{2.155363in}{1.695294in}}{\pgfqpoint{2.161187in}{1.689470in}}%
\pgfpathcurveto{\pgfqpoint{2.167011in}{1.683646in}}{\pgfqpoint{2.174911in}{1.680373in}}{\pgfqpoint{2.183147in}{1.680373in}}%
\pgfpathclose%
\pgfusepath{stroke,fill}%
\end{pgfscope}%
\begin{pgfscope}%
\pgfpathrectangle{\pgfqpoint{0.100000in}{0.212622in}}{\pgfqpoint{3.696000in}{3.696000in}}%
\pgfusepath{clip}%
\pgfsetbuttcap%
\pgfsetroundjoin%
\definecolor{currentfill}{rgb}{0.121569,0.466667,0.705882}%
\pgfsetfillcolor{currentfill}%
\pgfsetfillopacity{0.710587}%
\pgfsetlinewidth{1.003750pt}%
\definecolor{currentstroke}{rgb}{0.121569,0.466667,0.705882}%
\pgfsetstrokecolor{currentstroke}%
\pgfsetstrokeopacity{0.710587}%
\pgfsetdash{}{0pt}%
\pgfpathmoveto{\pgfqpoint{2.184324in}{1.679977in}}%
\pgfpathcurveto{\pgfqpoint{2.192560in}{1.679977in}}{\pgfqpoint{2.200460in}{1.683250in}}{\pgfqpoint{2.206284in}{1.689074in}}%
\pgfpathcurveto{\pgfqpoint{2.212108in}{1.694898in}}{\pgfqpoint{2.215380in}{1.702798in}}{\pgfqpoint{2.215380in}{1.711034in}}%
\pgfpathcurveto{\pgfqpoint{2.215380in}{1.719270in}}{\pgfqpoint{2.212108in}{1.727170in}}{\pgfqpoint{2.206284in}{1.732994in}}%
\pgfpathcurveto{\pgfqpoint{2.200460in}{1.738818in}}{\pgfqpoint{2.192560in}{1.742090in}}{\pgfqpoint{2.184324in}{1.742090in}}%
\pgfpathcurveto{\pgfqpoint{2.176088in}{1.742090in}}{\pgfqpoint{2.168188in}{1.738818in}}{\pgfqpoint{2.162364in}{1.732994in}}%
\pgfpathcurveto{\pgfqpoint{2.156540in}{1.727170in}}{\pgfqpoint{2.153267in}{1.719270in}}{\pgfqpoint{2.153267in}{1.711034in}}%
\pgfpathcurveto{\pgfqpoint{2.153267in}{1.702798in}}{\pgfqpoint{2.156540in}{1.694898in}}{\pgfqpoint{2.162364in}{1.689074in}}%
\pgfpathcurveto{\pgfqpoint{2.168188in}{1.683250in}}{\pgfqpoint{2.176088in}{1.679977in}}{\pgfqpoint{2.184324in}{1.679977in}}%
\pgfpathclose%
\pgfusepath{stroke,fill}%
\end{pgfscope}%
\begin{pgfscope}%
\pgfpathrectangle{\pgfqpoint{0.100000in}{0.212622in}}{\pgfqpoint{3.696000in}{3.696000in}}%
\pgfusepath{clip}%
\pgfsetbuttcap%
\pgfsetroundjoin%
\definecolor{currentfill}{rgb}{0.121569,0.466667,0.705882}%
\pgfsetfillcolor{currentfill}%
\pgfsetfillopacity{0.711208}%
\pgfsetlinewidth{1.003750pt}%
\definecolor{currentstroke}{rgb}{0.121569,0.466667,0.705882}%
\pgfsetstrokecolor{currentstroke}%
\pgfsetstrokeopacity{0.711208}%
\pgfsetdash{}{0pt}%
\pgfpathmoveto{\pgfqpoint{2.184590in}{1.678667in}}%
\pgfpathcurveto{\pgfqpoint{2.192827in}{1.678667in}}{\pgfqpoint{2.200727in}{1.681939in}}{\pgfqpoint{2.206551in}{1.687763in}}%
\pgfpathcurveto{\pgfqpoint{2.212375in}{1.693587in}}{\pgfqpoint{2.215647in}{1.701487in}}{\pgfqpoint{2.215647in}{1.709724in}}%
\pgfpathcurveto{\pgfqpoint{2.215647in}{1.717960in}}{\pgfqpoint{2.212375in}{1.725860in}}{\pgfqpoint{2.206551in}{1.731684in}}%
\pgfpathcurveto{\pgfqpoint{2.200727in}{1.737508in}}{\pgfqpoint{2.192827in}{1.740780in}}{\pgfqpoint{2.184590in}{1.740780in}}%
\pgfpathcurveto{\pgfqpoint{2.176354in}{1.740780in}}{\pgfqpoint{2.168454in}{1.737508in}}{\pgfqpoint{2.162630in}{1.731684in}}%
\pgfpathcurveto{\pgfqpoint{2.156806in}{1.725860in}}{\pgfqpoint{2.153534in}{1.717960in}}{\pgfqpoint{2.153534in}{1.709724in}}%
\pgfpathcurveto{\pgfqpoint{2.153534in}{1.701487in}}{\pgfqpoint{2.156806in}{1.693587in}}{\pgfqpoint{2.162630in}{1.687763in}}%
\pgfpathcurveto{\pgfqpoint{2.168454in}{1.681939in}}{\pgfqpoint{2.176354in}{1.678667in}}{\pgfqpoint{2.184590in}{1.678667in}}%
\pgfpathclose%
\pgfusepath{stroke,fill}%
\end{pgfscope}%
\begin{pgfscope}%
\pgfpathrectangle{\pgfqpoint{0.100000in}{0.212622in}}{\pgfqpoint{3.696000in}{3.696000in}}%
\pgfusepath{clip}%
\pgfsetbuttcap%
\pgfsetroundjoin%
\definecolor{currentfill}{rgb}{0.121569,0.466667,0.705882}%
\pgfsetfillcolor{currentfill}%
\pgfsetfillopacity{0.712032}%
\pgfsetlinewidth{1.003750pt}%
\definecolor{currentstroke}{rgb}{0.121569,0.466667,0.705882}%
\pgfsetstrokecolor{currentstroke}%
\pgfsetstrokeopacity{0.712032}%
\pgfsetdash{}{0pt}%
\pgfpathmoveto{\pgfqpoint{2.185240in}{1.676623in}}%
\pgfpathcurveto{\pgfqpoint{2.193476in}{1.676623in}}{\pgfqpoint{2.201376in}{1.679895in}}{\pgfqpoint{2.207200in}{1.685719in}}%
\pgfpathcurveto{\pgfqpoint{2.213024in}{1.691543in}}{\pgfqpoint{2.216296in}{1.699443in}}{\pgfqpoint{2.216296in}{1.707679in}}%
\pgfpathcurveto{\pgfqpoint{2.216296in}{1.715915in}}{\pgfqpoint{2.213024in}{1.723815in}}{\pgfqpoint{2.207200in}{1.729639in}}%
\pgfpathcurveto{\pgfqpoint{2.201376in}{1.735463in}}{\pgfqpoint{2.193476in}{1.738736in}}{\pgfqpoint{2.185240in}{1.738736in}}%
\pgfpathcurveto{\pgfqpoint{2.177004in}{1.738736in}}{\pgfqpoint{2.169104in}{1.735463in}}{\pgfqpoint{2.163280in}{1.729639in}}%
\pgfpathcurveto{\pgfqpoint{2.157456in}{1.723815in}}{\pgfqpoint{2.154183in}{1.715915in}}{\pgfqpoint{2.154183in}{1.707679in}}%
\pgfpathcurveto{\pgfqpoint{2.154183in}{1.699443in}}{\pgfqpoint{2.157456in}{1.691543in}}{\pgfqpoint{2.163280in}{1.685719in}}%
\pgfpathcurveto{\pgfqpoint{2.169104in}{1.679895in}}{\pgfqpoint{2.177004in}{1.676623in}}{\pgfqpoint{2.185240in}{1.676623in}}%
\pgfpathclose%
\pgfusepath{stroke,fill}%
\end{pgfscope}%
\begin{pgfscope}%
\pgfpathrectangle{\pgfqpoint{0.100000in}{0.212622in}}{\pgfqpoint{3.696000in}{3.696000in}}%
\pgfusepath{clip}%
\pgfsetbuttcap%
\pgfsetroundjoin%
\definecolor{currentfill}{rgb}{0.121569,0.466667,0.705882}%
\pgfsetfillcolor{currentfill}%
\pgfsetfillopacity{0.712537}%
\pgfsetlinewidth{1.003750pt}%
\definecolor{currentstroke}{rgb}{0.121569,0.466667,0.705882}%
\pgfsetstrokecolor{currentstroke}%
\pgfsetstrokeopacity{0.712537}%
\pgfsetdash{}{0pt}%
\pgfpathmoveto{\pgfqpoint{2.185626in}{1.675846in}}%
\pgfpathcurveto{\pgfqpoint{2.193862in}{1.675846in}}{\pgfqpoint{2.201762in}{1.679118in}}{\pgfqpoint{2.207586in}{1.684942in}}%
\pgfpathcurveto{\pgfqpoint{2.213410in}{1.690766in}}{\pgfqpoint{2.216683in}{1.698666in}}{\pgfqpoint{2.216683in}{1.706902in}}%
\pgfpathcurveto{\pgfqpoint{2.216683in}{1.715139in}}{\pgfqpoint{2.213410in}{1.723039in}}{\pgfqpoint{2.207586in}{1.728863in}}%
\pgfpathcurveto{\pgfqpoint{2.201762in}{1.734686in}}{\pgfqpoint{2.193862in}{1.737959in}}{\pgfqpoint{2.185626in}{1.737959in}}%
\pgfpathcurveto{\pgfqpoint{2.177390in}{1.737959in}}{\pgfqpoint{2.169490in}{1.734686in}}{\pgfqpoint{2.163666in}{1.728863in}}%
\pgfpathcurveto{\pgfqpoint{2.157842in}{1.723039in}}{\pgfqpoint{2.154570in}{1.715139in}}{\pgfqpoint{2.154570in}{1.706902in}}%
\pgfpathcurveto{\pgfqpoint{2.154570in}{1.698666in}}{\pgfqpoint{2.157842in}{1.690766in}}{\pgfqpoint{2.163666in}{1.684942in}}%
\pgfpathcurveto{\pgfqpoint{2.169490in}{1.679118in}}{\pgfqpoint{2.177390in}{1.675846in}}{\pgfqpoint{2.185626in}{1.675846in}}%
\pgfpathclose%
\pgfusepath{stroke,fill}%
\end{pgfscope}%
\begin{pgfscope}%
\pgfpathrectangle{\pgfqpoint{0.100000in}{0.212622in}}{\pgfqpoint{3.696000in}{3.696000in}}%
\pgfusepath{clip}%
\pgfsetbuttcap%
\pgfsetroundjoin%
\definecolor{currentfill}{rgb}{0.121569,0.466667,0.705882}%
\pgfsetfillcolor{currentfill}%
\pgfsetfillopacity{0.713405}%
\pgfsetlinewidth{1.003750pt}%
\definecolor{currentstroke}{rgb}{0.121569,0.466667,0.705882}%
\pgfsetstrokecolor{currentstroke}%
\pgfsetstrokeopacity{0.713405}%
\pgfsetdash{}{0pt}%
\pgfpathmoveto{\pgfqpoint{2.186279in}{1.675920in}}%
\pgfpathcurveto{\pgfqpoint{2.194515in}{1.675920in}}{\pgfqpoint{2.202416in}{1.679192in}}{\pgfqpoint{2.208239in}{1.685016in}}%
\pgfpathcurveto{\pgfqpoint{2.214063in}{1.690840in}}{\pgfqpoint{2.217336in}{1.698740in}}{\pgfqpoint{2.217336in}{1.706977in}}%
\pgfpathcurveto{\pgfqpoint{2.217336in}{1.715213in}}{\pgfqpoint{2.214063in}{1.723113in}}{\pgfqpoint{2.208239in}{1.728937in}}%
\pgfpathcurveto{\pgfqpoint{2.202416in}{1.734761in}}{\pgfqpoint{2.194515in}{1.738033in}}{\pgfqpoint{2.186279in}{1.738033in}}%
\pgfpathcurveto{\pgfqpoint{2.178043in}{1.738033in}}{\pgfqpoint{2.170143in}{1.734761in}}{\pgfqpoint{2.164319in}{1.728937in}}%
\pgfpathcurveto{\pgfqpoint{2.158495in}{1.723113in}}{\pgfqpoint{2.155223in}{1.715213in}}{\pgfqpoint{2.155223in}{1.706977in}}%
\pgfpathcurveto{\pgfqpoint{2.155223in}{1.698740in}}{\pgfqpoint{2.158495in}{1.690840in}}{\pgfqpoint{2.164319in}{1.685016in}}%
\pgfpathcurveto{\pgfqpoint{2.170143in}{1.679192in}}{\pgfqpoint{2.178043in}{1.675920in}}{\pgfqpoint{2.186279in}{1.675920in}}%
\pgfpathclose%
\pgfusepath{stroke,fill}%
\end{pgfscope}%
\begin{pgfscope}%
\pgfpathrectangle{\pgfqpoint{0.100000in}{0.212622in}}{\pgfqpoint{3.696000in}{3.696000in}}%
\pgfusepath{clip}%
\pgfsetbuttcap%
\pgfsetroundjoin%
\definecolor{currentfill}{rgb}{0.121569,0.466667,0.705882}%
\pgfsetfillcolor{currentfill}%
\pgfsetfillopacity{0.713745}%
\pgfsetlinewidth{1.003750pt}%
\definecolor{currentstroke}{rgb}{0.121569,0.466667,0.705882}%
\pgfsetstrokecolor{currentstroke}%
\pgfsetstrokeopacity{0.713745}%
\pgfsetdash{}{0pt}%
\pgfpathmoveto{\pgfqpoint{2.186446in}{1.674986in}}%
\pgfpathcurveto{\pgfqpoint{2.194683in}{1.674986in}}{\pgfqpoint{2.202583in}{1.678259in}}{\pgfqpoint{2.208407in}{1.684082in}}%
\pgfpathcurveto{\pgfqpoint{2.214231in}{1.689906in}}{\pgfqpoint{2.217503in}{1.697806in}}{\pgfqpoint{2.217503in}{1.706043in}}%
\pgfpathcurveto{\pgfqpoint{2.217503in}{1.714279in}}{\pgfqpoint{2.214231in}{1.722179in}}{\pgfqpoint{2.208407in}{1.728003in}}%
\pgfpathcurveto{\pgfqpoint{2.202583in}{1.733827in}}{\pgfqpoint{2.194683in}{1.737099in}}{\pgfqpoint{2.186446in}{1.737099in}}%
\pgfpathcurveto{\pgfqpoint{2.178210in}{1.737099in}}{\pgfqpoint{2.170310in}{1.733827in}}{\pgfqpoint{2.164486in}{1.728003in}}%
\pgfpathcurveto{\pgfqpoint{2.158662in}{1.722179in}}{\pgfqpoint{2.155390in}{1.714279in}}{\pgfqpoint{2.155390in}{1.706043in}}%
\pgfpathcurveto{\pgfqpoint{2.155390in}{1.697806in}}{\pgfqpoint{2.158662in}{1.689906in}}{\pgfqpoint{2.164486in}{1.684082in}}%
\pgfpathcurveto{\pgfqpoint{2.170310in}{1.678259in}}{\pgfqpoint{2.178210in}{1.674986in}}{\pgfqpoint{2.186446in}{1.674986in}}%
\pgfpathclose%
\pgfusepath{stroke,fill}%
\end{pgfscope}%
\begin{pgfscope}%
\pgfpathrectangle{\pgfqpoint{0.100000in}{0.212622in}}{\pgfqpoint{3.696000in}{3.696000in}}%
\pgfusepath{clip}%
\pgfsetbuttcap%
\pgfsetroundjoin%
\definecolor{currentfill}{rgb}{0.121569,0.466667,0.705882}%
\pgfsetfillcolor{currentfill}%
\pgfsetfillopacity{0.714393}%
\pgfsetlinewidth{1.003750pt}%
\definecolor{currentstroke}{rgb}{0.121569,0.466667,0.705882}%
\pgfsetstrokecolor{currentstroke}%
\pgfsetstrokeopacity{0.714393}%
\pgfsetdash{}{0pt}%
\pgfpathmoveto{\pgfqpoint{2.187341in}{1.673155in}}%
\pgfpathcurveto{\pgfqpoint{2.195578in}{1.673155in}}{\pgfqpoint{2.203478in}{1.676428in}}{\pgfqpoint{2.209302in}{1.682252in}}%
\pgfpathcurveto{\pgfqpoint{2.215126in}{1.688076in}}{\pgfqpoint{2.218398in}{1.695976in}}{\pgfqpoint{2.218398in}{1.704212in}}%
\pgfpathcurveto{\pgfqpoint{2.218398in}{1.712448in}}{\pgfqpoint{2.215126in}{1.720348in}}{\pgfqpoint{2.209302in}{1.726172in}}%
\pgfpathcurveto{\pgfqpoint{2.203478in}{1.731996in}}{\pgfqpoint{2.195578in}{1.735268in}}{\pgfqpoint{2.187341in}{1.735268in}}%
\pgfpathcurveto{\pgfqpoint{2.179105in}{1.735268in}}{\pgfqpoint{2.171205in}{1.731996in}}{\pgfqpoint{2.165381in}{1.726172in}}%
\pgfpathcurveto{\pgfqpoint{2.159557in}{1.720348in}}{\pgfqpoint{2.156285in}{1.712448in}}{\pgfqpoint{2.156285in}{1.704212in}}%
\pgfpathcurveto{\pgfqpoint{2.156285in}{1.695976in}}{\pgfqpoint{2.159557in}{1.688076in}}{\pgfqpoint{2.165381in}{1.682252in}}%
\pgfpathcurveto{\pgfqpoint{2.171205in}{1.676428in}}{\pgfqpoint{2.179105in}{1.673155in}}{\pgfqpoint{2.187341in}{1.673155in}}%
\pgfpathclose%
\pgfusepath{stroke,fill}%
\end{pgfscope}%
\begin{pgfscope}%
\pgfpathrectangle{\pgfqpoint{0.100000in}{0.212622in}}{\pgfqpoint{3.696000in}{3.696000in}}%
\pgfusepath{clip}%
\pgfsetbuttcap%
\pgfsetroundjoin%
\definecolor{currentfill}{rgb}{0.121569,0.466667,0.705882}%
\pgfsetfillcolor{currentfill}%
\pgfsetfillopacity{0.714861}%
\pgfsetlinewidth{1.003750pt}%
\definecolor{currentstroke}{rgb}{0.121569,0.466667,0.705882}%
\pgfsetstrokecolor{currentstroke}%
\pgfsetstrokeopacity{0.714861}%
\pgfsetdash{}{0pt}%
\pgfpathmoveto{\pgfqpoint{2.187629in}{1.672727in}}%
\pgfpathcurveto{\pgfqpoint{2.195865in}{1.672727in}}{\pgfqpoint{2.203765in}{1.675999in}}{\pgfqpoint{2.209589in}{1.681823in}}%
\pgfpathcurveto{\pgfqpoint{2.215413in}{1.687647in}}{\pgfqpoint{2.218685in}{1.695547in}}{\pgfqpoint{2.218685in}{1.703783in}}%
\pgfpathcurveto{\pgfqpoint{2.218685in}{1.712020in}}{\pgfqpoint{2.215413in}{1.719920in}}{\pgfqpoint{2.209589in}{1.725744in}}%
\pgfpathcurveto{\pgfqpoint{2.203765in}{1.731568in}}{\pgfqpoint{2.195865in}{1.734840in}}{\pgfqpoint{2.187629in}{1.734840in}}%
\pgfpathcurveto{\pgfqpoint{2.179392in}{1.734840in}}{\pgfqpoint{2.171492in}{1.731568in}}{\pgfqpoint{2.165668in}{1.725744in}}%
\pgfpathcurveto{\pgfqpoint{2.159844in}{1.719920in}}{\pgfqpoint{2.156572in}{1.712020in}}{\pgfqpoint{2.156572in}{1.703783in}}%
\pgfpathcurveto{\pgfqpoint{2.156572in}{1.695547in}}{\pgfqpoint{2.159844in}{1.687647in}}{\pgfqpoint{2.165668in}{1.681823in}}%
\pgfpathcurveto{\pgfqpoint{2.171492in}{1.675999in}}{\pgfqpoint{2.179392in}{1.672727in}}{\pgfqpoint{2.187629in}{1.672727in}}%
\pgfpathclose%
\pgfusepath{stroke,fill}%
\end{pgfscope}%
\begin{pgfscope}%
\pgfpathrectangle{\pgfqpoint{0.100000in}{0.212622in}}{\pgfqpoint{3.696000in}{3.696000in}}%
\pgfusepath{clip}%
\pgfsetbuttcap%
\pgfsetroundjoin%
\definecolor{currentfill}{rgb}{0.121569,0.466667,0.705882}%
\pgfsetfillcolor{currentfill}%
\pgfsetfillopacity{0.716031}%
\pgfsetlinewidth{1.003750pt}%
\definecolor{currentstroke}{rgb}{0.121569,0.466667,0.705882}%
\pgfsetstrokecolor{currentstroke}%
\pgfsetstrokeopacity{0.716031}%
\pgfsetdash{}{0pt}%
\pgfpathmoveto{\pgfqpoint{2.188565in}{1.673009in}}%
\pgfpathcurveto{\pgfqpoint{2.196801in}{1.673009in}}{\pgfqpoint{2.204701in}{1.676281in}}{\pgfqpoint{2.210525in}{1.682105in}}%
\pgfpathcurveto{\pgfqpoint{2.216349in}{1.687929in}}{\pgfqpoint{2.219622in}{1.695829in}}{\pgfqpoint{2.219622in}{1.704065in}}%
\pgfpathcurveto{\pgfqpoint{2.219622in}{1.712301in}}{\pgfqpoint{2.216349in}{1.720202in}}{\pgfqpoint{2.210525in}{1.726025in}}%
\pgfpathcurveto{\pgfqpoint{2.204701in}{1.731849in}}{\pgfqpoint{2.196801in}{1.735122in}}{\pgfqpoint{2.188565in}{1.735122in}}%
\pgfpathcurveto{\pgfqpoint{2.180329in}{1.735122in}}{\pgfqpoint{2.172429in}{1.731849in}}{\pgfqpoint{2.166605in}{1.726025in}}%
\pgfpathcurveto{\pgfqpoint{2.160781in}{1.720202in}}{\pgfqpoint{2.157509in}{1.712301in}}{\pgfqpoint{2.157509in}{1.704065in}}%
\pgfpathcurveto{\pgfqpoint{2.157509in}{1.695829in}}{\pgfqpoint{2.160781in}{1.687929in}}{\pgfqpoint{2.166605in}{1.682105in}}%
\pgfpathcurveto{\pgfqpoint{2.172429in}{1.676281in}}{\pgfqpoint{2.180329in}{1.673009in}}{\pgfqpoint{2.188565in}{1.673009in}}%
\pgfpathclose%
\pgfusepath{stroke,fill}%
\end{pgfscope}%
\begin{pgfscope}%
\pgfpathrectangle{\pgfqpoint{0.100000in}{0.212622in}}{\pgfqpoint{3.696000in}{3.696000in}}%
\pgfusepath{clip}%
\pgfsetbuttcap%
\pgfsetroundjoin%
\definecolor{currentfill}{rgb}{0.121569,0.466667,0.705882}%
\pgfsetfillcolor{currentfill}%
\pgfsetfillopacity{0.717080}%
\pgfsetlinewidth{1.003750pt}%
\definecolor{currentstroke}{rgb}{0.121569,0.466667,0.705882}%
\pgfsetstrokecolor{currentstroke}%
\pgfsetstrokeopacity{0.717080}%
\pgfsetdash{}{0pt}%
\pgfpathmoveto{\pgfqpoint{2.189278in}{1.671050in}}%
\pgfpathcurveto{\pgfqpoint{2.197514in}{1.671050in}}{\pgfqpoint{2.205414in}{1.674322in}}{\pgfqpoint{2.211238in}{1.680146in}}%
\pgfpathcurveto{\pgfqpoint{2.217062in}{1.685970in}}{\pgfqpoint{2.220334in}{1.693870in}}{\pgfqpoint{2.220334in}{1.702106in}}%
\pgfpathcurveto{\pgfqpoint{2.220334in}{1.710342in}}{\pgfqpoint{2.217062in}{1.718242in}}{\pgfqpoint{2.211238in}{1.724066in}}%
\pgfpathcurveto{\pgfqpoint{2.205414in}{1.729890in}}{\pgfqpoint{2.197514in}{1.733163in}}{\pgfqpoint{2.189278in}{1.733163in}}%
\pgfpathcurveto{\pgfqpoint{2.181041in}{1.733163in}}{\pgfqpoint{2.173141in}{1.729890in}}{\pgfqpoint{2.167317in}{1.724066in}}%
\pgfpathcurveto{\pgfqpoint{2.161493in}{1.718242in}}{\pgfqpoint{2.158221in}{1.710342in}}{\pgfqpoint{2.158221in}{1.702106in}}%
\pgfpathcurveto{\pgfqpoint{2.158221in}{1.693870in}}{\pgfqpoint{2.161493in}{1.685970in}}{\pgfqpoint{2.167317in}{1.680146in}}%
\pgfpathcurveto{\pgfqpoint{2.173141in}{1.674322in}}{\pgfqpoint{2.181041in}{1.671050in}}{\pgfqpoint{2.189278in}{1.671050in}}%
\pgfpathclose%
\pgfusepath{stroke,fill}%
\end{pgfscope}%
\begin{pgfscope}%
\pgfpathrectangle{\pgfqpoint{0.100000in}{0.212622in}}{\pgfqpoint{3.696000in}{3.696000in}}%
\pgfusepath{clip}%
\pgfsetbuttcap%
\pgfsetroundjoin%
\definecolor{currentfill}{rgb}{0.121569,0.466667,0.705882}%
\pgfsetfillcolor{currentfill}%
\pgfsetfillopacity{0.718945}%
\pgfsetlinewidth{1.003750pt}%
\definecolor{currentstroke}{rgb}{0.121569,0.466667,0.705882}%
\pgfsetstrokecolor{currentstroke}%
\pgfsetstrokeopacity{0.718945}%
\pgfsetdash{}{0pt}%
\pgfpathmoveto{\pgfqpoint{2.190480in}{1.667791in}}%
\pgfpathcurveto{\pgfqpoint{2.198716in}{1.667791in}}{\pgfqpoint{2.206616in}{1.671063in}}{\pgfqpoint{2.212440in}{1.676887in}}%
\pgfpathcurveto{\pgfqpoint{2.218264in}{1.682711in}}{\pgfqpoint{2.221537in}{1.690611in}}{\pgfqpoint{2.221537in}{1.698848in}}%
\pgfpathcurveto{\pgfqpoint{2.221537in}{1.707084in}}{\pgfqpoint{2.218264in}{1.714984in}}{\pgfqpoint{2.212440in}{1.720808in}}%
\pgfpathcurveto{\pgfqpoint{2.206616in}{1.726632in}}{\pgfqpoint{2.198716in}{1.729904in}}{\pgfqpoint{2.190480in}{1.729904in}}%
\pgfpathcurveto{\pgfqpoint{2.182244in}{1.729904in}}{\pgfqpoint{2.174344in}{1.726632in}}{\pgfqpoint{2.168520in}{1.720808in}}%
\pgfpathcurveto{\pgfqpoint{2.162696in}{1.714984in}}{\pgfqpoint{2.159424in}{1.707084in}}{\pgfqpoint{2.159424in}{1.698848in}}%
\pgfpathcurveto{\pgfqpoint{2.159424in}{1.690611in}}{\pgfqpoint{2.162696in}{1.682711in}}{\pgfqpoint{2.168520in}{1.676887in}}%
\pgfpathcurveto{\pgfqpoint{2.174344in}{1.671063in}}{\pgfqpoint{2.182244in}{1.667791in}}{\pgfqpoint{2.190480in}{1.667791in}}%
\pgfpathclose%
\pgfusepath{stroke,fill}%
\end{pgfscope}%
\begin{pgfscope}%
\pgfpathrectangle{\pgfqpoint{0.100000in}{0.212622in}}{\pgfqpoint{3.696000in}{3.696000in}}%
\pgfusepath{clip}%
\pgfsetbuttcap%
\pgfsetroundjoin%
\definecolor{currentfill}{rgb}{0.121569,0.466667,0.705882}%
\pgfsetfillcolor{currentfill}%
\pgfsetfillopacity{0.720278}%
\pgfsetlinewidth{1.003750pt}%
\definecolor{currentstroke}{rgb}{0.121569,0.466667,0.705882}%
\pgfsetstrokecolor{currentstroke}%
\pgfsetstrokeopacity{0.720278}%
\pgfsetdash{}{0pt}%
\pgfpathmoveto{\pgfqpoint{2.191485in}{1.668142in}}%
\pgfpathcurveto{\pgfqpoint{2.199721in}{1.668142in}}{\pgfqpoint{2.207621in}{1.671415in}}{\pgfqpoint{2.213445in}{1.677239in}}%
\pgfpathcurveto{\pgfqpoint{2.219269in}{1.683062in}}{\pgfqpoint{2.222541in}{1.690962in}}{\pgfqpoint{2.222541in}{1.699199in}}%
\pgfpathcurveto{\pgfqpoint{2.222541in}{1.707435in}}{\pgfqpoint{2.219269in}{1.715335in}}{\pgfqpoint{2.213445in}{1.721159in}}%
\pgfpathcurveto{\pgfqpoint{2.207621in}{1.726983in}}{\pgfqpoint{2.199721in}{1.730255in}}{\pgfqpoint{2.191485in}{1.730255in}}%
\pgfpathcurveto{\pgfqpoint{2.183249in}{1.730255in}}{\pgfqpoint{2.175349in}{1.726983in}}{\pgfqpoint{2.169525in}{1.721159in}}%
\pgfpathcurveto{\pgfqpoint{2.163701in}{1.715335in}}{\pgfqpoint{2.160428in}{1.707435in}}{\pgfqpoint{2.160428in}{1.699199in}}%
\pgfpathcurveto{\pgfqpoint{2.160428in}{1.690962in}}{\pgfqpoint{2.163701in}{1.683062in}}{\pgfqpoint{2.169525in}{1.677239in}}%
\pgfpathcurveto{\pgfqpoint{2.175349in}{1.671415in}}{\pgfqpoint{2.183249in}{1.668142in}}{\pgfqpoint{2.191485in}{1.668142in}}%
\pgfpathclose%
\pgfusepath{stroke,fill}%
\end{pgfscope}%
\begin{pgfscope}%
\pgfpathrectangle{\pgfqpoint{0.100000in}{0.212622in}}{\pgfqpoint{3.696000in}{3.696000in}}%
\pgfusepath{clip}%
\pgfsetbuttcap%
\pgfsetroundjoin%
\definecolor{currentfill}{rgb}{0.121569,0.466667,0.705882}%
\pgfsetfillcolor{currentfill}%
\pgfsetfillopacity{0.722271}%
\pgfsetlinewidth{1.003750pt}%
\definecolor{currentstroke}{rgb}{0.121569,0.466667,0.705882}%
\pgfsetstrokecolor{currentstroke}%
\pgfsetstrokeopacity{0.722271}%
\pgfsetdash{}{0pt}%
\pgfpathmoveto{\pgfqpoint{2.192849in}{1.666426in}}%
\pgfpathcurveto{\pgfqpoint{2.201085in}{1.666426in}}{\pgfqpoint{2.208985in}{1.669698in}}{\pgfqpoint{2.214809in}{1.675522in}}%
\pgfpathcurveto{\pgfqpoint{2.220633in}{1.681346in}}{\pgfqpoint{2.223905in}{1.689246in}}{\pgfqpoint{2.223905in}{1.697482in}}%
\pgfpathcurveto{\pgfqpoint{2.223905in}{1.705719in}}{\pgfqpoint{2.220633in}{1.713619in}}{\pgfqpoint{2.214809in}{1.719443in}}%
\pgfpathcurveto{\pgfqpoint{2.208985in}{1.725267in}}{\pgfqpoint{2.201085in}{1.728539in}}{\pgfqpoint{2.192849in}{1.728539in}}%
\pgfpathcurveto{\pgfqpoint{2.184612in}{1.728539in}}{\pgfqpoint{2.176712in}{1.725267in}}{\pgfqpoint{2.170889in}{1.719443in}}%
\pgfpathcurveto{\pgfqpoint{2.165065in}{1.713619in}}{\pgfqpoint{2.161792in}{1.705719in}}{\pgfqpoint{2.161792in}{1.697482in}}%
\pgfpathcurveto{\pgfqpoint{2.161792in}{1.689246in}}{\pgfqpoint{2.165065in}{1.681346in}}{\pgfqpoint{2.170889in}{1.675522in}}%
\pgfpathcurveto{\pgfqpoint{2.176712in}{1.669698in}}{\pgfqpoint{2.184612in}{1.666426in}}{\pgfqpoint{2.192849in}{1.666426in}}%
\pgfpathclose%
\pgfusepath{stroke,fill}%
\end{pgfscope}%
\begin{pgfscope}%
\pgfpathrectangle{\pgfqpoint{0.100000in}{0.212622in}}{\pgfqpoint{3.696000in}{3.696000in}}%
\pgfusepath{clip}%
\pgfsetbuttcap%
\pgfsetroundjoin%
\definecolor{currentfill}{rgb}{0.121569,0.466667,0.705882}%
\pgfsetfillcolor{currentfill}%
\pgfsetfillopacity{0.724234}%
\pgfsetlinewidth{1.003750pt}%
\definecolor{currentstroke}{rgb}{0.121569,0.466667,0.705882}%
\pgfsetstrokecolor{currentstroke}%
\pgfsetstrokeopacity{0.724234}%
\pgfsetdash{}{0pt}%
\pgfpathmoveto{\pgfqpoint{2.194345in}{1.661522in}}%
\pgfpathcurveto{\pgfqpoint{2.202582in}{1.661522in}}{\pgfqpoint{2.210482in}{1.664795in}}{\pgfqpoint{2.216306in}{1.670619in}}%
\pgfpathcurveto{\pgfqpoint{2.222129in}{1.676443in}}{\pgfqpoint{2.225402in}{1.684343in}}{\pgfqpoint{2.225402in}{1.692579in}}%
\pgfpathcurveto{\pgfqpoint{2.225402in}{1.700815in}}{\pgfqpoint{2.222129in}{1.708715in}}{\pgfqpoint{2.216306in}{1.714539in}}%
\pgfpathcurveto{\pgfqpoint{2.210482in}{1.720363in}}{\pgfqpoint{2.202582in}{1.723635in}}{\pgfqpoint{2.194345in}{1.723635in}}%
\pgfpathcurveto{\pgfqpoint{2.186109in}{1.723635in}}{\pgfqpoint{2.178209in}{1.720363in}}{\pgfqpoint{2.172385in}{1.714539in}}%
\pgfpathcurveto{\pgfqpoint{2.166561in}{1.708715in}}{\pgfqpoint{2.163289in}{1.700815in}}{\pgfqpoint{2.163289in}{1.692579in}}%
\pgfpathcurveto{\pgfqpoint{2.163289in}{1.684343in}}{\pgfqpoint{2.166561in}{1.676443in}}{\pgfqpoint{2.172385in}{1.670619in}}%
\pgfpathcurveto{\pgfqpoint{2.178209in}{1.664795in}}{\pgfqpoint{2.186109in}{1.661522in}}{\pgfqpoint{2.194345in}{1.661522in}}%
\pgfpathclose%
\pgfusepath{stroke,fill}%
\end{pgfscope}%
\begin{pgfscope}%
\pgfpathrectangle{\pgfqpoint{0.100000in}{0.212622in}}{\pgfqpoint{3.696000in}{3.696000in}}%
\pgfusepath{clip}%
\pgfsetbuttcap%
\pgfsetroundjoin%
\definecolor{currentfill}{rgb}{0.121569,0.466667,0.705882}%
\pgfsetfillcolor{currentfill}%
\pgfsetfillopacity{0.726763}%
\pgfsetlinewidth{1.003750pt}%
\definecolor{currentstroke}{rgb}{0.121569,0.466667,0.705882}%
\pgfsetstrokecolor{currentstroke}%
\pgfsetstrokeopacity{0.726763}%
\pgfsetdash{}{0pt}%
\pgfpathmoveto{\pgfqpoint{2.196074in}{1.656327in}}%
\pgfpathcurveto{\pgfqpoint{2.204311in}{1.656327in}}{\pgfqpoint{2.212211in}{1.659599in}}{\pgfqpoint{2.218035in}{1.665423in}}%
\pgfpathcurveto{\pgfqpoint{2.223859in}{1.671247in}}{\pgfqpoint{2.227131in}{1.679147in}}{\pgfqpoint{2.227131in}{1.687384in}}%
\pgfpathcurveto{\pgfqpoint{2.227131in}{1.695620in}}{\pgfqpoint{2.223859in}{1.703520in}}{\pgfqpoint{2.218035in}{1.709344in}}%
\pgfpathcurveto{\pgfqpoint{2.212211in}{1.715168in}}{\pgfqpoint{2.204311in}{1.718440in}}{\pgfqpoint{2.196074in}{1.718440in}}%
\pgfpathcurveto{\pgfqpoint{2.187838in}{1.718440in}}{\pgfqpoint{2.179938in}{1.715168in}}{\pgfqpoint{2.174114in}{1.709344in}}%
\pgfpathcurveto{\pgfqpoint{2.168290in}{1.703520in}}{\pgfqpoint{2.165018in}{1.695620in}}{\pgfqpoint{2.165018in}{1.687384in}}%
\pgfpathcurveto{\pgfqpoint{2.165018in}{1.679147in}}{\pgfqpoint{2.168290in}{1.671247in}}{\pgfqpoint{2.174114in}{1.665423in}}%
\pgfpathcurveto{\pgfqpoint{2.179938in}{1.659599in}}{\pgfqpoint{2.187838in}{1.656327in}}{\pgfqpoint{2.196074in}{1.656327in}}%
\pgfpathclose%
\pgfusepath{stroke,fill}%
\end{pgfscope}%
\begin{pgfscope}%
\pgfpathrectangle{\pgfqpoint{0.100000in}{0.212622in}}{\pgfqpoint{3.696000in}{3.696000in}}%
\pgfusepath{clip}%
\pgfsetbuttcap%
\pgfsetroundjoin%
\definecolor{currentfill}{rgb}{0.121569,0.466667,0.705882}%
\pgfsetfillcolor{currentfill}%
\pgfsetfillopacity{0.730509}%
\pgfsetlinewidth{1.003750pt}%
\definecolor{currentstroke}{rgb}{0.121569,0.466667,0.705882}%
\pgfsetstrokecolor{currentstroke}%
\pgfsetstrokeopacity{0.730509}%
\pgfsetdash{}{0pt}%
\pgfpathmoveto{\pgfqpoint{2.199476in}{1.658780in}}%
\pgfpathcurveto{\pgfqpoint{2.207713in}{1.658780in}}{\pgfqpoint{2.215613in}{1.662052in}}{\pgfqpoint{2.221437in}{1.667876in}}%
\pgfpathcurveto{\pgfqpoint{2.227261in}{1.673700in}}{\pgfqpoint{2.230533in}{1.681600in}}{\pgfqpoint{2.230533in}{1.689836in}}%
\pgfpathcurveto{\pgfqpoint{2.230533in}{1.698072in}}{\pgfqpoint{2.227261in}{1.705973in}}{\pgfqpoint{2.221437in}{1.711796in}}%
\pgfpathcurveto{\pgfqpoint{2.215613in}{1.717620in}}{\pgfqpoint{2.207713in}{1.720893in}}{\pgfqpoint{2.199476in}{1.720893in}}%
\pgfpathcurveto{\pgfqpoint{2.191240in}{1.720893in}}{\pgfqpoint{2.183340in}{1.717620in}}{\pgfqpoint{2.177516in}{1.711796in}}%
\pgfpathcurveto{\pgfqpoint{2.171692in}{1.705973in}}{\pgfqpoint{2.168420in}{1.698072in}}{\pgfqpoint{2.168420in}{1.689836in}}%
\pgfpathcurveto{\pgfqpoint{2.168420in}{1.681600in}}{\pgfqpoint{2.171692in}{1.673700in}}{\pgfqpoint{2.177516in}{1.667876in}}%
\pgfpathcurveto{\pgfqpoint{2.183340in}{1.662052in}}{\pgfqpoint{2.191240in}{1.658780in}}{\pgfqpoint{2.199476in}{1.658780in}}%
\pgfpathclose%
\pgfusepath{stroke,fill}%
\end{pgfscope}%
\begin{pgfscope}%
\pgfpathrectangle{\pgfqpoint{0.100000in}{0.212622in}}{\pgfqpoint{3.696000in}{3.696000in}}%
\pgfusepath{clip}%
\pgfsetbuttcap%
\pgfsetroundjoin%
\definecolor{currentfill}{rgb}{0.121569,0.466667,0.705882}%
\pgfsetfillcolor{currentfill}%
\pgfsetfillopacity{0.732109}%
\pgfsetlinewidth{1.003750pt}%
\definecolor{currentstroke}{rgb}{0.121569,0.466667,0.705882}%
\pgfsetstrokecolor{currentstroke}%
\pgfsetstrokeopacity{0.732109}%
\pgfsetdash{}{0pt}%
\pgfpathmoveto{\pgfqpoint{2.199916in}{1.656406in}}%
\pgfpathcurveto{\pgfqpoint{2.208153in}{1.656406in}}{\pgfqpoint{2.216053in}{1.659679in}}{\pgfqpoint{2.221877in}{1.665503in}}%
\pgfpathcurveto{\pgfqpoint{2.227701in}{1.671327in}}{\pgfqpoint{2.230973in}{1.679227in}}{\pgfqpoint{2.230973in}{1.687463in}}%
\pgfpathcurveto{\pgfqpoint{2.230973in}{1.695699in}}{\pgfqpoint{2.227701in}{1.703599in}}{\pgfqpoint{2.221877in}{1.709423in}}%
\pgfpathcurveto{\pgfqpoint{2.216053in}{1.715247in}}{\pgfqpoint{2.208153in}{1.718519in}}{\pgfqpoint{2.199916in}{1.718519in}}%
\pgfpathcurveto{\pgfqpoint{2.191680in}{1.718519in}}{\pgfqpoint{2.183780in}{1.715247in}}{\pgfqpoint{2.177956in}{1.709423in}}%
\pgfpathcurveto{\pgfqpoint{2.172132in}{1.703599in}}{\pgfqpoint{2.168860in}{1.695699in}}{\pgfqpoint{2.168860in}{1.687463in}}%
\pgfpathcurveto{\pgfqpoint{2.168860in}{1.679227in}}{\pgfqpoint{2.172132in}{1.671327in}}{\pgfqpoint{2.177956in}{1.665503in}}%
\pgfpathcurveto{\pgfqpoint{2.183780in}{1.659679in}}{\pgfqpoint{2.191680in}{1.656406in}}{\pgfqpoint{2.199916in}{1.656406in}}%
\pgfpathclose%
\pgfusepath{stroke,fill}%
\end{pgfscope}%
\begin{pgfscope}%
\pgfpathrectangle{\pgfqpoint{0.100000in}{0.212622in}}{\pgfqpoint{3.696000in}{3.696000in}}%
\pgfusepath{clip}%
\pgfsetbuttcap%
\pgfsetroundjoin%
\definecolor{currentfill}{rgb}{0.121569,0.466667,0.705882}%
\pgfsetfillcolor{currentfill}%
\pgfsetfillopacity{0.733694}%
\pgfsetlinewidth{1.003750pt}%
\definecolor{currentstroke}{rgb}{0.121569,0.466667,0.705882}%
\pgfsetstrokecolor{currentstroke}%
\pgfsetstrokeopacity{0.733694}%
\pgfsetdash{}{0pt}%
\pgfpathmoveto{\pgfqpoint{2.201375in}{1.651976in}}%
\pgfpathcurveto{\pgfqpoint{2.209611in}{1.651976in}}{\pgfqpoint{2.217511in}{1.655248in}}{\pgfqpoint{2.223335in}{1.661072in}}%
\pgfpathcurveto{\pgfqpoint{2.229159in}{1.666896in}}{\pgfqpoint{2.232431in}{1.674796in}}{\pgfqpoint{2.232431in}{1.683033in}}%
\pgfpathcurveto{\pgfqpoint{2.232431in}{1.691269in}}{\pgfqpoint{2.229159in}{1.699169in}}{\pgfqpoint{2.223335in}{1.704993in}}%
\pgfpathcurveto{\pgfqpoint{2.217511in}{1.710817in}}{\pgfqpoint{2.209611in}{1.714089in}}{\pgfqpoint{2.201375in}{1.714089in}}%
\pgfpathcurveto{\pgfqpoint{2.193138in}{1.714089in}}{\pgfqpoint{2.185238in}{1.710817in}}{\pgfqpoint{2.179414in}{1.704993in}}%
\pgfpathcurveto{\pgfqpoint{2.173590in}{1.699169in}}{\pgfqpoint{2.170318in}{1.691269in}}{\pgfqpoint{2.170318in}{1.683033in}}%
\pgfpathcurveto{\pgfqpoint{2.170318in}{1.674796in}}{\pgfqpoint{2.173590in}{1.666896in}}{\pgfqpoint{2.179414in}{1.661072in}}%
\pgfpathcurveto{\pgfqpoint{2.185238in}{1.655248in}}{\pgfqpoint{2.193138in}{1.651976in}}{\pgfqpoint{2.201375in}{1.651976in}}%
\pgfpathclose%
\pgfusepath{stroke,fill}%
\end{pgfscope}%
\begin{pgfscope}%
\pgfpathrectangle{\pgfqpoint{0.100000in}{0.212622in}}{\pgfqpoint{3.696000in}{3.696000in}}%
\pgfusepath{clip}%
\pgfsetbuttcap%
\pgfsetroundjoin%
\definecolor{currentfill}{rgb}{0.121569,0.466667,0.705882}%
\pgfsetfillcolor{currentfill}%
\pgfsetfillopacity{0.734779}%
\pgfsetlinewidth{1.003750pt}%
\definecolor{currentstroke}{rgb}{0.121569,0.466667,0.705882}%
\pgfsetstrokecolor{currentstroke}%
\pgfsetstrokeopacity{0.734779}%
\pgfsetdash{}{0pt}%
\pgfpathmoveto{\pgfqpoint{2.202028in}{1.650802in}}%
\pgfpathcurveto{\pgfqpoint{2.210264in}{1.650802in}}{\pgfqpoint{2.218164in}{1.654074in}}{\pgfqpoint{2.223988in}{1.659898in}}%
\pgfpathcurveto{\pgfqpoint{2.229812in}{1.665722in}}{\pgfqpoint{2.233084in}{1.673622in}}{\pgfqpoint{2.233084in}{1.681859in}}%
\pgfpathcurveto{\pgfqpoint{2.233084in}{1.690095in}}{\pgfqpoint{2.229812in}{1.697995in}}{\pgfqpoint{2.223988in}{1.703819in}}%
\pgfpathcurveto{\pgfqpoint{2.218164in}{1.709643in}}{\pgfqpoint{2.210264in}{1.712915in}}{\pgfqpoint{2.202028in}{1.712915in}}%
\pgfpathcurveto{\pgfqpoint{2.193791in}{1.712915in}}{\pgfqpoint{2.185891in}{1.709643in}}{\pgfqpoint{2.180067in}{1.703819in}}%
\pgfpathcurveto{\pgfqpoint{2.174244in}{1.697995in}}{\pgfqpoint{2.170971in}{1.690095in}}{\pgfqpoint{2.170971in}{1.681859in}}%
\pgfpathcurveto{\pgfqpoint{2.170971in}{1.673622in}}{\pgfqpoint{2.174244in}{1.665722in}}{\pgfqpoint{2.180067in}{1.659898in}}%
\pgfpathcurveto{\pgfqpoint{2.185891in}{1.654074in}}{\pgfqpoint{2.193791in}{1.650802in}}{\pgfqpoint{2.202028in}{1.650802in}}%
\pgfpathclose%
\pgfusepath{stroke,fill}%
\end{pgfscope}%
\begin{pgfscope}%
\pgfpathrectangle{\pgfqpoint{0.100000in}{0.212622in}}{\pgfqpoint{3.696000in}{3.696000in}}%
\pgfusepath{clip}%
\pgfsetbuttcap%
\pgfsetroundjoin%
\definecolor{currentfill}{rgb}{0.121569,0.466667,0.705882}%
\pgfsetfillcolor{currentfill}%
\pgfsetfillopacity{0.736498}%
\pgfsetlinewidth{1.003750pt}%
\definecolor{currentstroke}{rgb}{0.121569,0.466667,0.705882}%
\pgfsetstrokecolor{currentstroke}%
\pgfsetstrokeopacity{0.736498}%
\pgfsetdash{}{0pt}%
\pgfpathmoveto{\pgfqpoint{2.203514in}{1.651732in}}%
\pgfpathcurveto{\pgfqpoint{2.211750in}{1.651732in}}{\pgfqpoint{2.219650in}{1.655004in}}{\pgfqpoint{2.225474in}{1.660828in}}%
\pgfpathcurveto{\pgfqpoint{2.231298in}{1.666652in}}{\pgfqpoint{2.234571in}{1.674552in}}{\pgfqpoint{2.234571in}{1.682789in}}%
\pgfpathcurveto{\pgfqpoint{2.234571in}{1.691025in}}{\pgfqpoint{2.231298in}{1.698925in}}{\pgfqpoint{2.225474in}{1.704749in}}%
\pgfpathcurveto{\pgfqpoint{2.219650in}{1.710573in}}{\pgfqpoint{2.211750in}{1.713845in}}{\pgfqpoint{2.203514in}{1.713845in}}%
\pgfpathcurveto{\pgfqpoint{2.195278in}{1.713845in}}{\pgfqpoint{2.187378in}{1.710573in}}{\pgfqpoint{2.181554in}{1.704749in}}%
\pgfpathcurveto{\pgfqpoint{2.175730in}{1.698925in}}{\pgfqpoint{2.172458in}{1.691025in}}{\pgfqpoint{2.172458in}{1.682789in}}%
\pgfpathcurveto{\pgfqpoint{2.172458in}{1.674552in}}{\pgfqpoint{2.175730in}{1.666652in}}{\pgfqpoint{2.181554in}{1.660828in}}%
\pgfpathcurveto{\pgfqpoint{2.187378in}{1.655004in}}{\pgfqpoint{2.195278in}{1.651732in}}{\pgfqpoint{2.203514in}{1.651732in}}%
\pgfpathclose%
\pgfusepath{stroke,fill}%
\end{pgfscope}%
\begin{pgfscope}%
\pgfpathrectangle{\pgfqpoint{0.100000in}{0.212622in}}{\pgfqpoint{3.696000in}{3.696000in}}%
\pgfusepath{clip}%
\pgfsetbuttcap%
\pgfsetroundjoin%
\definecolor{currentfill}{rgb}{0.121569,0.466667,0.705882}%
\pgfsetfillcolor{currentfill}%
\pgfsetfillopacity{0.737189}%
\pgfsetlinewidth{1.003750pt}%
\definecolor{currentstroke}{rgb}{0.121569,0.466667,0.705882}%
\pgfsetstrokecolor{currentstroke}%
\pgfsetstrokeopacity{0.737189}%
\pgfsetdash{}{0pt}%
\pgfpathmoveto{\pgfqpoint{2.203769in}{1.650317in}}%
\pgfpathcurveto{\pgfqpoint{2.212005in}{1.650317in}}{\pgfqpoint{2.219905in}{1.653590in}}{\pgfqpoint{2.225729in}{1.659414in}}%
\pgfpathcurveto{\pgfqpoint{2.231553in}{1.665238in}}{\pgfqpoint{2.234825in}{1.673138in}}{\pgfqpoint{2.234825in}{1.681374in}}%
\pgfpathcurveto{\pgfqpoint{2.234825in}{1.689610in}}{\pgfqpoint{2.231553in}{1.697510in}}{\pgfqpoint{2.225729in}{1.703334in}}%
\pgfpathcurveto{\pgfqpoint{2.219905in}{1.709158in}}{\pgfqpoint{2.212005in}{1.712430in}}{\pgfqpoint{2.203769in}{1.712430in}}%
\pgfpathcurveto{\pgfqpoint{2.195532in}{1.712430in}}{\pgfqpoint{2.187632in}{1.709158in}}{\pgfqpoint{2.181808in}{1.703334in}}%
\pgfpathcurveto{\pgfqpoint{2.175984in}{1.697510in}}{\pgfqpoint{2.172712in}{1.689610in}}{\pgfqpoint{2.172712in}{1.681374in}}%
\pgfpathcurveto{\pgfqpoint{2.172712in}{1.673138in}}{\pgfqpoint{2.175984in}{1.665238in}}{\pgfqpoint{2.181808in}{1.659414in}}%
\pgfpathcurveto{\pgfqpoint{2.187632in}{1.653590in}}{\pgfqpoint{2.195532in}{1.650317in}}{\pgfqpoint{2.203769in}{1.650317in}}%
\pgfpathclose%
\pgfusepath{stroke,fill}%
\end{pgfscope}%
\begin{pgfscope}%
\pgfpathrectangle{\pgfqpoint{0.100000in}{0.212622in}}{\pgfqpoint{3.696000in}{3.696000in}}%
\pgfusepath{clip}%
\pgfsetbuttcap%
\pgfsetroundjoin%
\definecolor{currentfill}{rgb}{0.121569,0.466667,0.705882}%
\pgfsetfillcolor{currentfill}%
\pgfsetfillopacity{0.738179}%
\pgfsetlinewidth{1.003750pt}%
\definecolor{currentstroke}{rgb}{0.121569,0.466667,0.705882}%
\pgfsetstrokecolor{currentstroke}%
\pgfsetstrokeopacity{0.738179}%
\pgfsetdash{}{0pt}%
\pgfpathmoveto{\pgfqpoint{2.205052in}{1.647732in}}%
\pgfpathcurveto{\pgfqpoint{2.213288in}{1.647732in}}{\pgfqpoint{2.221188in}{1.651004in}}{\pgfqpoint{2.227012in}{1.656828in}}%
\pgfpathcurveto{\pgfqpoint{2.232836in}{1.662652in}}{\pgfqpoint{2.236108in}{1.670552in}}{\pgfqpoint{2.236108in}{1.678788in}}%
\pgfpathcurveto{\pgfqpoint{2.236108in}{1.687025in}}{\pgfqpoint{2.232836in}{1.694925in}}{\pgfqpoint{2.227012in}{1.700749in}}%
\pgfpathcurveto{\pgfqpoint{2.221188in}{1.706573in}}{\pgfqpoint{2.213288in}{1.709845in}}{\pgfqpoint{2.205052in}{1.709845in}}%
\pgfpathcurveto{\pgfqpoint{2.196815in}{1.709845in}}{\pgfqpoint{2.188915in}{1.706573in}}{\pgfqpoint{2.183091in}{1.700749in}}%
\pgfpathcurveto{\pgfqpoint{2.177267in}{1.694925in}}{\pgfqpoint{2.173995in}{1.687025in}}{\pgfqpoint{2.173995in}{1.678788in}}%
\pgfpathcurveto{\pgfqpoint{2.173995in}{1.670552in}}{\pgfqpoint{2.177267in}{1.662652in}}{\pgfqpoint{2.183091in}{1.656828in}}%
\pgfpathcurveto{\pgfqpoint{2.188915in}{1.651004in}}{\pgfqpoint{2.196815in}{1.647732in}}{\pgfqpoint{2.205052in}{1.647732in}}%
\pgfpathclose%
\pgfusepath{stroke,fill}%
\end{pgfscope}%
\begin{pgfscope}%
\pgfpathrectangle{\pgfqpoint{0.100000in}{0.212622in}}{\pgfqpoint{3.696000in}{3.696000in}}%
\pgfusepath{clip}%
\pgfsetbuttcap%
\pgfsetroundjoin%
\definecolor{currentfill}{rgb}{0.121569,0.466667,0.705882}%
\pgfsetfillcolor{currentfill}%
\pgfsetfillopacity{0.739291}%
\pgfsetlinewidth{1.003750pt}%
\definecolor{currentstroke}{rgb}{0.121569,0.466667,0.705882}%
\pgfsetstrokecolor{currentstroke}%
\pgfsetstrokeopacity{0.739291}%
\pgfsetdash{}{0pt}%
\pgfpathmoveto{\pgfqpoint{2.206081in}{1.644164in}}%
\pgfpathcurveto{\pgfqpoint{2.214317in}{1.644164in}}{\pgfqpoint{2.222217in}{1.647437in}}{\pgfqpoint{2.228041in}{1.653261in}}%
\pgfpathcurveto{\pgfqpoint{2.233865in}{1.659085in}}{\pgfqpoint{2.237138in}{1.666985in}}{\pgfqpoint{2.237138in}{1.675221in}}%
\pgfpathcurveto{\pgfqpoint{2.237138in}{1.683457in}}{\pgfqpoint{2.233865in}{1.691357in}}{\pgfqpoint{2.228041in}{1.697181in}}%
\pgfpathcurveto{\pgfqpoint{2.222217in}{1.703005in}}{\pgfqpoint{2.214317in}{1.706277in}}{\pgfqpoint{2.206081in}{1.706277in}}%
\pgfpathcurveto{\pgfqpoint{2.197845in}{1.706277in}}{\pgfqpoint{2.189945in}{1.703005in}}{\pgfqpoint{2.184121in}{1.697181in}}%
\pgfpathcurveto{\pgfqpoint{2.178297in}{1.691357in}}{\pgfqpoint{2.175025in}{1.683457in}}{\pgfqpoint{2.175025in}{1.675221in}}%
\pgfpathcurveto{\pgfqpoint{2.175025in}{1.666985in}}{\pgfqpoint{2.178297in}{1.659085in}}{\pgfqpoint{2.184121in}{1.653261in}}%
\pgfpathcurveto{\pgfqpoint{2.189945in}{1.647437in}}{\pgfqpoint{2.197845in}{1.644164in}}{\pgfqpoint{2.206081in}{1.644164in}}%
\pgfpathclose%
\pgfusepath{stroke,fill}%
\end{pgfscope}%
\begin{pgfscope}%
\pgfpathrectangle{\pgfqpoint{0.100000in}{0.212622in}}{\pgfqpoint{3.696000in}{3.696000in}}%
\pgfusepath{clip}%
\pgfsetbuttcap%
\pgfsetroundjoin%
\definecolor{currentfill}{rgb}{0.121569,0.466667,0.705882}%
\pgfsetfillcolor{currentfill}%
\pgfsetfillopacity{0.741647}%
\pgfsetlinewidth{1.003750pt}%
\definecolor{currentstroke}{rgb}{0.121569,0.466667,0.705882}%
\pgfsetstrokecolor{currentstroke}%
\pgfsetstrokeopacity{0.741647}%
\pgfsetdash{}{0pt}%
\pgfpathmoveto{\pgfqpoint{2.208383in}{1.645796in}}%
\pgfpathcurveto{\pgfqpoint{2.216619in}{1.645796in}}{\pgfqpoint{2.224519in}{1.649069in}}{\pgfqpoint{2.230343in}{1.654893in}}%
\pgfpathcurveto{\pgfqpoint{2.236167in}{1.660717in}}{\pgfqpoint{2.239439in}{1.668617in}}{\pgfqpoint{2.239439in}{1.676853in}}%
\pgfpathcurveto{\pgfqpoint{2.239439in}{1.685089in}}{\pgfqpoint{2.236167in}{1.692989in}}{\pgfqpoint{2.230343in}{1.698813in}}%
\pgfpathcurveto{\pgfqpoint{2.224519in}{1.704637in}}{\pgfqpoint{2.216619in}{1.707909in}}{\pgfqpoint{2.208383in}{1.707909in}}%
\pgfpathcurveto{\pgfqpoint{2.200146in}{1.707909in}}{\pgfqpoint{2.192246in}{1.704637in}}{\pgfqpoint{2.186422in}{1.698813in}}%
\pgfpathcurveto{\pgfqpoint{2.180598in}{1.692989in}}{\pgfqpoint{2.177326in}{1.685089in}}{\pgfqpoint{2.177326in}{1.676853in}}%
\pgfpathcurveto{\pgfqpoint{2.177326in}{1.668617in}}{\pgfqpoint{2.180598in}{1.660717in}}{\pgfqpoint{2.186422in}{1.654893in}}%
\pgfpathcurveto{\pgfqpoint{2.192246in}{1.649069in}}{\pgfqpoint{2.200146in}{1.645796in}}{\pgfqpoint{2.208383in}{1.645796in}}%
\pgfpathclose%
\pgfusepath{stroke,fill}%
\end{pgfscope}%
\begin{pgfscope}%
\pgfpathrectangle{\pgfqpoint{0.100000in}{0.212622in}}{\pgfqpoint{3.696000in}{3.696000in}}%
\pgfusepath{clip}%
\pgfsetbuttcap%
\pgfsetroundjoin%
\definecolor{currentfill}{rgb}{0.121569,0.466667,0.705882}%
\pgfsetfillcolor{currentfill}%
\pgfsetfillopacity{0.742639}%
\pgfsetlinewidth{1.003750pt}%
\definecolor{currentstroke}{rgb}{0.121569,0.466667,0.705882}%
\pgfsetstrokecolor{currentstroke}%
\pgfsetstrokeopacity{0.742639}%
\pgfsetdash{}{0pt}%
\pgfpathmoveto{\pgfqpoint{2.209400in}{1.644565in}}%
\pgfpathcurveto{\pgfqpoint{2.217637in}{1.644565in}}{\pgfqpoint{2.225537in}{1.647837in}}{\pgfqpoint{2.231361in}{1.653661in}}%
\pgfpathcurveto{\pgfqpoint{2.237185in}{1.659485in}}{\pgfqpoint{2.240457in}{1.667385in}}{\pgfqpoint{2.240457in}{1.675622in}}%
\pgfpathcurveto{\pgfqpoint{2.240457in}{1.683858in}}{\pgfqpoint{2.237185in}{1.691758in}}{\pgfqpoint{2.231361in}{1.697582in}}%
\pgfpathcurveto{\pgfqpoint{2.225537in}{1.703406in}}{\pgfqpoint{2.217637in}{1.706678in}}{\pgfqpoint{2.209400in}{1.706678in}}%
\pgfpathcurveto{\pgfqpoint{2.201164in}{1.706678in}}{\pgfqpoint{2.193264in}{1.703406in}}{\pgfqpoint{2.187440in}{1.697582in}}%
\pgfpathcurveto{\pgfqpoint{2.181616in}{1.691758in}}{\pgfqpoint{2.178344in}{1.683858in}}{\pgfqpoint{2.178344in}{1.675622in}}%
\pgfpathcurveto{\pgfqpoint{2.178344in}{1.667385in}}{\pgfqpoint{2.181616in}{1.659485in}}{\pgfqpoint{2.187440in}{1.653661in}}%
\pgfpathcurveto{\pgfqpoint{2.193264in}{1.647837in}}{\pgfqpoint{2.201164in}{1.644565in}}{\pgfqpoint{2.209400in}{1.644565in}}%
\pgfpathclose%
\pgfusepath{stroke,fill}%
\end{pgfscope}%
\begin{pgfscope}%
\pgfpathrectangle{\pgfqpoint{0.100000in}{0.212622in}}{\pgfqpoint{3.696000in}{3.696000in}}%
\pgfusepath{clip}%
\pgfsetbuttcap%
\pgfsetroundjoin%
\definecolor{currentfill}{rgb}{0.121569,0.466667,0.705882}%
\pgfsetfillcolor{currentfill}%
\pgfsetfillopacity{0.744040}%
\pgfsetlinewidth{1.003750pt}%
\definecolor{currentstroke}{rgb}{0.121569,0.466667,0.705882}%
\pgfsetstrokecolor{currentstroke}%
\pgfsetstrokeopacity{0.744040}%
\pgfsetdash{}{0pt}%
\pgfpathmoveto{\pgfqpoint{2.209915in}{1.643584in}}%
\pgfpathcurveto{\pgfqpoint{2.218151in}{1.643584in}}{\pgfqpoint{2.226051in}{1.646857in}}{\pgfqpoint{2.231875in}{1.652681in}}%
\pgfpathcurveto{\pgfqpoint{2.237699in}{1.658505in}}{\pgfqpoint{2.240971in}{1.666405in}}{\pgfqpoint{2.240971in}{1.674641in}}%
\pgfpathcurveto{\pgfqpoint{2.240971in}{1.682877in}}{\pgfqpoint{2.237699in}{1.690777in}}{\pgfqpoint{2.231875in}{1.696601in}}%
\pgfpathcurveto{\pgfqpoint{2.226051in}{1.702425in}}{\pgfqpoint{2.218151in}{1.705697in}}{\pgfqpoint{2.209915in}{1.705697in}}%
\pgfpathcurveto{\pgfqpoint{2.201678in}{1.705697in}}{\pgfqpoint{2.193778in}{1.702425in}}{\pgfqpoint{2.187954in}{1.696601in}}%
\pgfpathcurveto{\pgfqpoint{2.182130in}{1.690777in}}{\pgfqpoint{2.178858in}{1.682877in}}{\pgfqpoint{2.178858in}{1.674641in}}%
\pgfpathcurveto{\pgfqpoint{2.178858in}{1.666405in}}{\pgfqpoint{2.182130in}{1.658505in}}{\pgfqpoint{2.187954in}{1.652681in}}%
\pgfpathcurveto{\pgfqpoint{2.193778in}{1.646857in}}{\pgfqpoint{2.201678in}{1.643584in}}{\pgfqpoint{2.209915in}{1.643584in}}%
\pgfpathclose%
\pgfusepath{stroke,fill}%
\end{pgfscope}%
\begin{pgfscope}%
\pgfpathrectangle{\pgfqpoint{0.100000in}{0.212622in}}{\pgfqpoint{3.696000in}{3.696000in}}%
\pgfusepath{clip}%
\pgfsetbuttcap%
\pgfsetroundjoin%
\definecolor{currentfill}{rgb}{0.121569,0.466667,0.705882}%
\pgfsetfillcolor{currentfill}%
\pgfsetfillopacity{0.744616}%
\pgfsetlinewidth{1.003750pt}%
\definecolor{currentstroke}{rgb}{0.121569,0.466667,0.705882}%
\pgfsetstrokecolor{currentstroke}%
\pgfsetstrokeopacity{0.744616}%
\pgfsetdash{}{0pt}%
\pgfpathmoveto{\pgfqpoint{2.210584in}{1.642017in}}%
\pgfpathcurveto{\pgfqpoint{2.218821in}{1.642017in}}{\pgfqpoint{2.226721in}{1.645289in}}{\pgfqpoint{2.232545in}{1.651113in}}%
\pgfpathcurveto{\pgfqpoint{2.238368in}{1.656937in}}{\pgfqpoint{2.241641in}{1.664837in}}{\pgfqpoint{2.241641in}{1.673073in}}%
\pgfpathcurveto{\pgfqpoint{2.241641in}{1.681310in}}{\pgfqpoint{2.238368in}{1.689210in}}{\pgfqpoint{2.232545in}{1.695034in}}%
\pgfpathcurveto{\pgfqpoint{2.226721in}{1.700858in}}{\pgfqpoint{2.218821in}{1.704130in}}{\pgfqpoint{2.210584in}{1.704130in}}%
\pgfpathcurveto{\pgfqpoint{2.202348in}{1.704130in}}{\pgfqpoint{2.194448in}{1.700858in}}{\pgfqpoint{2.188624in}{1.695034in}}%
\pgfpathcurveto{\pgfqpoint{2.182800in}{1.689210in}}{\pgfqpoint{2.179528in}{1.681310in}}{\pgfqpoint{2.179528in}{1.673073in}}%
\pgfpathcurveto{\pgfqpoint{2.179528in}{1.664837in}}{\pgfqpoint{2.182800in}{1.656937in}}{\pgfqpoint{2.188624in}{1.651113in}}%
\pgfpathcurveto{\pgfqpoint{2.194448in}{1.645289in}}{\pgfqpoint{2.202348in}{1.642017in}}{\pgfqpoint{2.210584in}{1.642017in}}%
\pgfpathclose%
\pgfusepath{stroke,fill}%
\end{pgfscope}%
\begin{pgfscope}%
\pgfpathrectangle{\pgfqpoint{0.100000in}{0.212622in}}{\pgfqpoint{3.696000in}{3.696000in}}%
\pgfusepath{clip}%
\pgfsetbuttcap%
\pgfsetroundjoin%
\definecolor{currentfill}{rgb}{0.121569,0.466667,0.705882}%
\pgfsetfillcolor{currentfill}%
\pgfsetfillopacity{0.745641}%
\pgfsetlinewidth{1.003750pt}%
\definecolor{currentstroke}{rgb}{0.121569,0.466667,0.705882}%
\pgfsetstrokecolor{currentstroke}%
\pgfsetstrokeopacity{0.745641}%
\pgfsetdash{}{0pt}%
\pgfpathmoveto{\pgfqpoint{2.211616in}{1.642421in}}%
\pgfpathcurveto{\pgfqpoint{2.219852in}{1.642421in}}{\pgfqpoint{2.227752in}{1.645694in}}{\pgfqpoint{2.233576in}{1.651518in}}%
\pgfpathcurveto{\pgfqpoint{2.239400in}{1.657342in}}{\pgfqpoint{2.242672in}{1.665242in}}{\pgfqpoint{2.242672in}{1.673478in}}%
\pgfpathcurveto{\pgfqpoint{2.242672in}{1.681714in}}{\pgfqpoint{2.239400in}{1.689614in}}{\pgfqpoint{2.233576in}{1.695438in}}%
\pgfpathcurveto{\pgfqpoint{2.227752in}{1.701262in}}{\pgfqpoint{2.219852in}{1.704534in}}{\pgfqpoint{2.211616in}{1.704534in}}%
\pgfpathcurveto{\pgfqpoint{2.203380in}{1.704534in}}{\pgfqpoint{2.195480in}{1.701262in}}{\pgfqpoint{2.189656in}{1.695438in}}%
\pgfpathcurveto{\pgfqpoint{2.183832in}{1.689614in}}{\pgfqpoint{2.180559in}{1.681714in}}{\pgfqpoint{2.180559in}{1.673478in}}%
\pgfpathcurveto{\pgfqpoint{2.180559in}{1.665242in}}{\pgfqpoint{2.183832in}{1.657342in}}{\pgfqpoint{2.189656in}{1.651518in}}%
\pgfpathcurveto{\pgfqpoint{2.195480in}{1.645694in}}{\pgfqpoint{2.203380in}{1.642421in}}{\pgfqpoint{2.211616in}{1.642421in}}%
\pgfpathclose%
\pgfusepath{stroke,fill}%
\end{pgfscope}%
\begin{pgfscope}%
\pgfpathrectangle{\pgfqpoint{0.100000in}{0.212622in}}{\pgfqpoint{3.696000in}{3.696000in}}%
\pgfusepath{clip}%
\pgfsetbuttcap%
\pgfsetroundjoin%
\definecolor{currentfill}{rgb}{0.121569,0.466667,0.705882}%
\pgfsetfillcolor{currentfill}%
\pgfsetfillopacity{0.746021}%
\pgfsetlinewidth{1.003750pt}%
\definecolor{currentstroke}{rgb}{0.121569,0.466667,0.705882}%
\pgfsetstrokecolor{currentstroke}%
\pgfsetstrokeopacity{0.746021}%
\pgfsetdash{}{0pt}%
\pgfpathmoveto{\pgfqpoint{2.211869in}{1.641280in}}%
\pgfpathcurveto{\pgfqpoint{2.220105in}{1.641280in}}{\pgfqpoint{2.228005in}{1.644553in}}{\pgfqpoint{2.233829in}{1.650377in}}%
\pgfpathcurveto{\pgfqpoint{2.239653in}{1.656200in}}{\pgfqpoint{2.242925in}{1.664101in}}{\pgfqpoint{2.242925in}{1.672337in}}%
\pgfpathcurveto{\pgfqpoint{2.242925in}{1.680573in}}{\pgfqpoint{2.239653in}{1.688473in}}{\pgfqpoint{2.233829in}{1.694297in}}%
\pgfpathcurveto{\pgfqpoint{2.228005in}{1.700121in}}{\pgfqpoint{2.220105in}{1.703393in}}{\pgfqpoint{2.211869in}{1.703393in}}%
\pgfpathcurveto{\pgfqpoint{2.203632in}{1.703393in}}{\pgfqpoint{2.195732in}{1.700121in}}{\pgfqpoint{2.189909in}{1.694297in}}%
\pgfpathcurveto{\pgfqpoint{2.184085in}{1.688473in}}{\pgfqpoint{2.180812in}{1.680573in}}{\pgfqpoint{2.180812in}{1.672337in}}%
\pgfpathcurveto{\pgfqpoint{2.180812in}{1.664101in}}{\pgfqpoint{2.184085in}{1.656200in}}{\pgfqpoint{2.189909in}{1.650377in}}%
\pgfpathcurveto{\pgfqpoint{2.195732in}{1.644553in}}{\pgfqpoint{2.203632in}{1.641280in}}{\pgfqpoint{2.211869in}{1.641280in}}%
\pgfpathclose%
\pgfusepath{stroke,fill}%
\end{pgfscope}%
\begin{pgfscope}%
\pgfpathrectangle{\pgfqpoint{0.100000in}{0.212622in}}{\pgfqpoint{3.696000in}{3.696000in}}%
\pgfusepath{clip}%
\pgfsetbuttcap%
\pgfsetroundjoin%
\definecolor{currentfill}{rgb}{0.121569,0.466667,0.705882}%
\pgfsetfillcolor{currentfill}%
\pgfsetfillopacity{0.746759}%
\pgfsetlinewidth{1.003750pt}%
\definecolor{currentstroke}{rgb}{0.121569,0.466667,0.705882}%
\pgfsetstrokecolor{currentstroke}%
\pgfsetstrokeopacity{0.746759}%
\pgfsetdash{}{0pt}%
\pgfpathmoveto{\pgfqpoint{2.212511in}{1.640590in}}%
\pgfpathcurveto{\pgfqpoint{2.220747in}{1.640590in}}{\pgfqpoint{2.228647in}{1.643863in}}{\pgfqpoint{2.234471in}{1.649687in}}%
\pgfpathcurveto{\pgfqpoint{2.240295in}{1.655511in}}{\pgfqpoint{2.243568in}{1.663411in}}{\pgfqpoint{2.243568in}{1.671647in}}%
\pgfpathcurveto{\pgfqpoint{2.243568in}{1.679883in}}{\pgfqpoint{2.240295in}{1.687783in}}{\pgfqpoint{2.234471in}{1.693607in}}%
\pgfpathcurveto{\pgfqpoint{2.228647in}{1.699431in}}{\pgfqpoint{2.220747in}{1.702703in}}{\pgfqpoint{2.212511in}{1.702703in}}%
\pgfpathcurveto{\pgfqpoint{2.204275in}{1.702703in}}{\pgfqpoint{2.196375in}{1.699431in}}{\pgfqpoint{2.190551in}{1.693607in}}%
\pgfpathcurveto{\pgfqpoint{2.184727in}{1.687783in}}{\pgfqpoint{2.181455in}{1.679883in}}{\pgfqpoint{2.181455in}{1.671647in}}%
\pgfpathcurveto{\pgfqpoint{2.181455in}{1.663411in}}{\pgfqpoint{2.184727in}{1.655511in}}{\pgfqpoint{2.190551in}{1.649687in}}%
\pgfpathcurveto{\pgfqpoint{2.196375in}{1.643863in}}{\pgfqpoint{2.204275in}{1.640590in}}{\pgfqpoint{2.212511in}{1.640590in}}%
\pgfpathclose%
\pgfusepath{stroke,fill}%
\end{pgfscope}%
\begin{pgfscope}%
\pgfpathrectangle{\pgfqpoint{0.100000in}{0.212622in}}{\pgfqpoint{3.696000in}{3.696000in}}%
\pgfusepath{clip}%
\pgfsetbuttcap%
\pgfsetroundjoin%
\definecolor{currentfill}{rgb}{0.121569,0.466667,0.705882}%
\pgfsetfillcolor{currentfill}%
\pgfsetfillopacity{0.747214}%
\pgfsetlinewidth{1.003750pt}%
\definecolor{currentstroke}{rgb}{0.121569,0.466667,0.705882}%
\pgfsetstrokecolor{currentstroke}%
\pgfsetstrokeopacity{0.747214}%
\pgfsetdash{}{0pt}%
\pgfpathmoveto{\pgfqpoint{2.212753in}{1.640455in}}%
\pgfpathcurveto{\pgfqpoint{2.220989in}{1.640455in}}{\pgfqpoint{2.228889in}{1.643727in}}{\pgfqpoint{2.234713in}{1.649551in}}%
\pgfpathcurveto{\pgfqpoint{2.240537in}{1.655375in}}{\pgfqpoint{2.243810in}{1.663275in}}{\pgfqpoint{2.243810in}{1.671511in}}%
\pgfpathcurveto{\pgfqpoint{2.243810in}{1.679748in}}{\pgfqpoint{2.240537in}{1.687648in}}{\pgfqpoint{2.234713in}{1.693472in}}%
\pgfpathcurveto{\pgfqpoint{2.228889in}{1.699295in}}{\pgfqpoint{2.220989in}{1.702568in}}{\pgfqpoint{2.212753in}{1.702568in}}%
\pgfpathcurveto{\pgfqpoint{2.204517in}{1.702568in}}{\pgfqpoint{2.196617in}{1.699295in}}{\pgfqpoint{2.190793in}{1.693472in}}%
\pgfpathcurveto{\pgfqpoint{2.184969in}{1.687648in}}{\pgfqpoint{2.181697in}{1.679748in}}{\pgfqpoint{2.181697in}{1.671511in}}%
\pgfpathcurveto{\pgfqpoint{2.181697in}{1.663275in}}{\pgfqpoint{2.184969in}{1.655375in}}{\pgfqpoint{2.190793in}{1.649551in}}%
\pgfpathcurveto{\pgfqpoint{2.196617in}{1.643727in}}{\pgfqpoint{2.204517in}{1.640455in}}{\pgfqpoint{2.212753in}{1.640455in}}%
\pgfpathclose%
\pgfusepath{stroke,fill}%
\end{pgfscope}%
\begin{pgfscope}%
\pgfpathrectangle{\pgfqpoint{0.100000in}{0.212622in}}{\pgfqpoint{3.696000in}{3.696000in}}%
\pgfusepath{clip}%
\pgfsetbuttcap%
\pgfsetroundjoin%
\definecolor{currentfill}{rgb}{0.121569,0.466667,0.705882}%
\pgfsetfillcolor{currentfill}%
\pgfsetfillopacity{0.747973}%
\pgfsetlinewidth{1.003750pt}%
\definecolor{currentstroke}{rgb}{0.121569,0.466667,0.705882}%
\pgfsetstrokecolor{currentstroke}%
\pgfsetstrokeopacity{0.747973}%
\pgfsetdash{}{0pt}%
\pgfpathmoveto{\pgfqpoint{2.213606in}{1.640395in}}%
\pgfpathcurveto{\pgfqpoint{2.221842in}{1.640395in}}{\pgfqpoint{2.229742in}{1.643667in}}{\pgfqpoint{2.235566in}{1.649491in}}%
\pgfpathcurveto{\pgfqpoint{2.241390in}{1.655315in}}{\pgfqpoint{2.244663in}{1.663215in}}{\pgfqpoint{2.244663in}{1.671452in}}%
\pgfpathcurveto{\pgfqpoint{2.244663in}{1.679688in}}{\pgfqpoint{2.241390in}{1.687588in}}{\pgfqpoint{2.235566in}{1.693412in}}%
\pgfpathcurveto{\pgfqpoint{2.229742in}{1.699236in}}{\pgfqpoint{2.221842in}{1.702508in}}{\pgfqpoint{2.213606in}{1.702508in}}%
\pgfpathcurveto{\pgfqpoint{2.205370in}{1.702508in}}{\pgfqpoint{2.197470in}{1.699236in}}{\pgfqpoint{2.191646in}{1.693412in}}%
\pgfpathcurveto{\pgfqpoint{2.185822in}{1.687588in}}{\pgfqpoint{2.182550in}{1.679688in}}{\pgfqpoint{2.182550in}{1.671452in}}%
\pgfpathcurveto{\pgfqpoint{2.182550in}{1.663215in}}{\pgfqpoint{2.185822in}{1.655315in}}{\pgfqpoint{2.191646in}{1.649491in}}%
\pgfpathcurveto{\pgfqpoint{2.197470in}{1.643667in}}{\pgfqpoint{2.205370in}{1.640395in}}{\pgfqpoint{2.213606in}{1.640395in}}%
\pgfpathclose%
\pgfusepath{stroke,fill}%
\end{pgfscope}%
\begin{pgfscope}%
\pgfpathrectangle{\pgfqpoint{0.100000in}{0.212622in}}{\pgfqpoint{3.696000in}{3.696000in}}%
\pgfusepath{clip}%
\pgfsetbuttcap%
\pgfsetroundjoin%
\definecolor{currentfill}{rgb}{0.121569,0.466667,0.705882}%
\pgfsetfillcolor{currentfill}%
\pgfsetfillopacity{0.748869}%
\pgfsetlinewidth{1.003750pt}%
\definecolor{currentstroke}{rgb}{0.121569,0.466667,0.705882}%
\pgfsetstrokecolor{currentstroke}%
\pgfsetstrokeopacity{0.748869}%
\pgfsetdash{}{0pt}%
\pgfpathmoveto{\pgfqpoint{2.213985in}{1.639442in}}%
\pgfpathcurveto{\pgfqpoint{2.222222in}{1.639442in}}{\pgfqpoint{2.230122in}{1.642714in}}{\pgfqpoint{2.235946in}{1.648538in}}%
\pgfpathcurveto{\pgfqpoint{2.241770in}{1.654362in}}{\pgfqpoint{2.245042in}{1.662262in}}{\pgfqpoint{2.245042in}{1.670498in}}%
\pgfpathcurveto{\pgfqpoint{2.245042in}{1.678735in}}{\pgfqpoint{2.241770in}{1.686635in}}{\pgfqpoint{2.235946in}{1.692459in}}%
\pgfpathcurveto{\pgfqpoint{2.230122in}{1.698283in}}{\pgfqpoint{2.222222in}{1.701555in}}{\pgfqpoint{2.213985in}{1.701555in}}%
\pgfpathcurveto{\pgfqpoint{2.205749in}{1.701555in}}{\pgfqpoint{2.197849in}{1.698283in}}{\pgfqpoint{2.192025in}{1.692459in}}%
\pgfpathcurveto{\pgfqpoint{2.186201in}{1.686635in}}{\pgfqpoint{2.182929in}{1.678735in}}{\pgfqpoint{2.182929in}{1.670498in}}%
\pgfpathcurveto{\pgfqpoint{2.182929in}{1.662262in}}{\pgfqpoint{2.186201in}{1.654362in}}{\pgfqpoint{2.192025in}{1.648538in}}%
\pgfpathcurveto{\pgfqpoint{2.197849in}{1.642714in}}{\pgfqpoint{2.205749in}{1.639442in}}{\pgfqpoint{2.213985in}{1.639442in}}%
\pgfpathclose%
\pgfusepath{stroke,fill}%
\end{pgfscope}%
\begin{pgfscope}%
\pgfpathrectangle{\pgfqpoint{0.100000in}{0.212622in}}{\pgfqpoint{3.696000in}{3.696000in}}%
\pgfusepath{clip}%
\pgfsetbuttcap%
\pgfsetroundjoin%
\definecolor{currentfill}{rgb}{0.121569,0.466667,0.705882}%
\pgfsetfillcolor{currentfill}%
\pgfsetfillopacity{0.750056}%
\pgfsetlinewidth{1.003750pt}%
\definecolor{currentstroke}{rgb}{0.121569,0.466667,0.705882}%
\pgfsetstrokecolor{currentstroke}%
\pgfsetstrokeopacity{0.750056}%
\pgfsetdash{}{0pt}%
\pgfpathmoveto{\pgfqpoint{2.215392in}{1.635224in}}%
\pgfpathcurveto{\pgfqpoint{2.223628in}{1.635224in}}{\pgfqpoint{2.231528in}{1.638496in}}{\pgfqpoint{2.237352in}{1.644320in}}%
\pgfpathcurveto{\pgfqpoint{2.243176in}{1.650144in}}{\pgfqpoint{2.246448in}{1.658044in}}{\pgfqpoint{2.246448in}{1.666281in}}%
\pgfpathcurveto{\pgfqpoint{2.246448in}{1.674517in}}{\pgfqpoint{2.243176in}{1.682417in}}{\pgfqpoint{2.237352in}{1.688241in}}%
\pgfpathcurveto{\pgfqpoint{2.231528in}{1.694065in}}{\pgfqpoint{2.223628in}{1.697337in}}{\pgfqpoint{2.215392in}{1.697337in}}%
\pgfpathcurveto{\pgfqpoint{2.207156in}{1.697337in}}{\pgfqpoint{2.199256in}{1.694065in}}{\pgfqpoint{2.193432in}{1.688241in}}%
\pgfpathcurveto{\pgfqpoint{2.187608in}{1.682417in}}{\pgfqpoint{2.184335in}{1.674517in}}{\pgfqpoint{2.184335in}{1.666281in}}%
\pgfpathcurveto{\pgfqpoint{2.184335in}{1.658044in}}{\pgfqpoint{2.187608in}{1.650144in}}{\pgfqpoint{2.193432in}{1.644320in}}%
\pgfpathcurveto{\pgfqpoint{2.199256in}{1.638496in}}{\pgfqpoint{2.207156in}{1.635224in}}{\pgfqpoint{2.215392in}{1.635224in}}%
\pgfpathclose%
\pgfusepath{stroke,fill}%
\end{pgfscope}%
\begin{pgfscope}%
\pgfpathrectangle{\pgfqpoint{0.100000in}{0.212622in}}{\pgfqpoint{3.696000in}{3.696000in}}%
\pgfusepath{clip}%
\pgfsetbuttcap%
\pgfsetroundjoin%
\definecolor{currentfill}{rgb}{0.121569,0.466667,0.705882}%
\pgfsetfillcolor{currentfill}%
\pgfsetfillopacity{0.752204}%
\pgfsetlinewidth{1.003750pt}%
\definecolor{currentstroke}{rgb}{0.121569,0.466667,0.705882}%
\pgfsetstrokecolor{currentstroke}%
\pgfsetstrokeopacity{0.752204}%
\pgfsetdash{}{0pt}%
\pgfpathmoveto{\pgfqpoint{2.217742in}{1.635543in}}%
\pgfpathcurveto{\pgfqpoint{2.225979in}{1.635543in}}{\pgfqpoint{2.233879in}{1.638815in}}{\pgfqpoint{2.239703in}{1.644639in}}%
\pgfpathcurveto{\pgfqpoint{2.245526in}{1.650463in}}{\pgfqpoint{2.248799in}{1.658363in}}{\pgfqpoint{2.248799in}{1.666599in}}%
\pgfpathcurveto{\pgfqpoint{2.248799in}{1.674835in}}{\pgfqpoint{2.245526in}{1.682735in}}{\pgfqpoint{2.239703in}{1.688559in}}%
\pgfpathcurveto{\pgfqpoint{2.233879in}{1.694383in}}{\pgfqpoint{2.225979in}{1.697656in}}{\pgfqpoint{2.217742in}{1.697656in}}%
\pgfpathcurveto{\pgfqpoint{2.209506in}{1.697656in}}{\pgfqpoint{2.201606in}{1.694383in}}{\pgfqpoint{2.195782in}{1.688559in}}%
\pgfpathcurveto{\pgfqpoint{2.189958in}{1.682735in}}{\pgfqpoint{2.186686in}{1.674835in}}{\pgfqpoint{2.186686in}{1.666599in}}%
\pgfpathcurveto{\pgfqpoint{2.186686in}{1.658363in}}{\pgfqpoint{2.189958in}{1.650463in}}{\pgfqpoint{2.195782in}{1.644639in}}%
\pgfpathcurveto{\pgfqpoint{2.201606in}{1.638815in}}{\pgfqpoint{2.209506in}{1.635543in}}{\pgfqpoint{2.217742in}{1.635543in}}%
\pgfpathclose%
\pgfusepath{stroke,fill}%
\end{pgfscope}%
\begin{pgfscope}%
\pgfpathrectangle{\pgfqpoint{0.100000in}{0.212622in}}{\pgfqpoint{3.696000in}{3.696000in}}%
\pgfusepath{clip}%
\pgfsetbuttcap%
\pgfsetroundjoin%
\definecolor{currentfill}{rgb}{0.121569,0.466667,0.705882}%
\pgfsetfillcolor{currentfill}%
\pgfsetfillopacity{0.754740}%
\pgfsetlinewidth{1.003750pt}%
\definecolor{currentstroke}{rgb}{0.121569,0.466667,0.705882}%
\pgfsetstrokecolor{currentstroke}%
\pgfsetstrokeopacity{0.754740}%
\pgfsetdash{}{0pt}%
\pgfpathmoveto{\pgfqpoint{2.219308in}{1.633471in}}%
\pgfpathcurveto{\pgfqpoint{2.227544in}{1.633471in}}{\pgfqpoint{2.235444in}{1.636744in}}{\pgfqpoint{2.241268in}{1.642568in}}%
\pgfpathcurveto{\pgfqpoint{2.247092in}{1.648392in}}{\pgfqpoint{2.250364in}{1.656292in}}{\pgfqpoint{2.250364in}{1.664528in}}%
\pgfpathcurveto{\pgfqpoint{2.250364in}{1.672764in}}{\pgfqpoint{2.247092in}{1.680664in}}{\pgfqpoint{2.241268in}{1.686488in}}%
\pgfpathcurveto{\pgfqpoint{2.235444in}{1.692312in}}{\pgfqpoint{2.227544in}{1.695584in}}{\pgfqpoint{2.219308in}{1.695584in}}%
\pgfpathcurveto{\pgfqpoint{2.211071in}{1.695584in}}{\pgfqpoint{2.203171in}{1.692312in}}{\pgfqpoint{2.197347in}{1.686488in}}%
\pgfpathcurveto{\pgfqpoint{2.191524in}{1.680664in}}{\pgfqpoint{2.188251in}{1.672764in}}{\pgfqpoint{2.188251in}{1.664528in}}%
\pgfpathcurveto{\pgfqpoint{2.188251in}{1.656292in}}{\pgfqpoint{2.191524in}{1.648392in}}{\pgfqpoint{2.197347in}{1.642568in}}%
\pgfpathcurveto{\pgfqpoint{2.203171in}{1.636744in}}{\pgfqpoint{2.211071in}{1.633471in}}{\pgfqpoint{2.219308in}{1.633471in}}%
\pgfpathclose%
\pgfusepath{stroke,fill}%
\end{pgfscope}%
\begin{pgfscope}%
\pgfpathrectangle{\pgfqpoint{0.100000in}{0.212622in}}{\pgfqpoint{3.696000in}{3.696000in}}%
\pgfusepath{clip}%
\pgfsetbuttcap%
\pgfsetroundjoin%
\definecolor{currentfill}{rgb}{0.121569,0.466667,0.705882}%
\pgfsetfillcolor{currentfill}%
\pgfsetfillopacity{0.757746}%
\pgfsetlinewidth{1.003750pt}%
\definecolor{currentstroke}{rgb}{0.121569,0.466667,0.705882}%
\pgfsetstrokecolor{currentstroke}%
\pgfsetstrokeopacity{0.757746}%
\pgfsetdash{}{0pt}%
\pgfpathmoveto{\pgfqpoint{2.220832in}{1.630881in}}%
\pgfpathcurveto{\pgfqpoint{2.229068in}{1.630881in}}{\pgfqpoint{2.236968in}{1.634154in}}{\pgfqpoint{2.242792in}{1.639978in}}%
\pgfpathcurveto{\pgfqpoint{2.248616in}{1.645802in}}{\pgfqpoint{2.251888in}{1.653702in}}{\pgfqpoint{2.251888in}{1.661938in}}%
\pgfpathcurveto{\pgfqpoint{2.251888in}{1.670174in}}{\pgfqpoint{2.248616in}{1.678074in}}{\pgfqpoint{2.242792in}{1.683898in}}%
\pgfpathcurveto{\pgfqpoint{2.236968in}{1.689722in}}{\pgfqpoint{2.229068in}{1.692994in}}{\pgfqpoint{2.220832in}{1.692994in}}%
\pgfpathcurveto{\pgfqpoint{2.212595in}{1.692994in}}{\pgfqpoint{2.204695in}{1.689722in}}{\pgfqpoint{2.198871in}{1.683898in}}%
\pgfpathcurveto{\pgfqpoint{2.193048in}{1.678074in}}{\pgfqpoint{2.189775in}{1.670174in}}{\pgfqpoint{2.189775in}{1.661938in}}%
\pgfpathcurveto{\pgfqpoint{2.189775in}{1.653702in}}{\pgfqpoint{2.193048in}{1.645802in}}{\pgfqpoint{2.198871in}{1.639978in}}%
\pgfpathcurveto{\pgfqpoint{2.204695in}{1.634154in}}{\pgfqpoint{2.212595in}{1.630881in}}{\pgfqpoint{2.220832in}{1.630881in}}%
\pgfpathclose%
\pgfusepath{stroke,fill}%
\end{pgfscope}%
\begin{pgfscope}%
\pgfpathrectangle{\pgfqpoint{0.100000in}{0.212622in}}{\pgfqpoint{3.696000in}{3.696000in}}%
\pgfusepath{clip}%
\pgfsetbuttcap%
\pgfsetroundjoin%
\definecolor{currentfill}{rgb}{0.121569,0.466667,0.705882}%
\pgfsetfillcolor{currentfill}%
\pgfsetfillopacity{0.760660}%
\pgfsetlinewidth{1.003750pt}%
\definecolor{currentstroke}{rgb}{0.121569,0.466667,0.705882}%
\pgfsetstrokecolor{currentstroke}%
\pgfsetstrokeopacity{0.760660}%
\pgfsetdash{}{0pt}%
\pgfpathmoveto{\pgfqpoint{2.223806in}{1.625324in}}%
\pgfpathcurveto{\pgfqpoint{2.232043in}{1.625324in}}{\pgfqpoint{2.239943in}{1.628596in}}{\pgfqpoint{2.245767in}{1.634420in}}%
\pgfpathcurveto{\pgfqpoint{2.251591in}{1.640244in}}{\pgfqpoint{2.254863in}{1.648144in}}{\pgfqpoint{2.254863in}{1.656381in}}%
\pgfpathcurveto{\pgfqpoint{2.254863in}{1.664617in}}{\pgfqpoint{2.251591in}{1.672517in}}{\pgfqpoint{2.245767in}{1.678341in}}%
\pgfpathcurveto{\pgfqpoint{2.239943in}{1.684165in}}{\pgfqpoint{2.232043in}{1.687437in}}{\pgfqpoint{2.223806in}{1.687437in}}%
\pgfpathcurveto{\pgfqpoint{2.215570in}{1.687437in}}{\pgfqpoint{2.207670in}{1.684165in}}{\pgfqpoint{2.201846in}{1.678341in}}%
\pgfpathcurveto{\pgfqpoint{2.196022in}{1.672517in}}{\pgfqpoint{2.192750in}{1.664617in}}{\pgfqpoint{2.192750in}{1.656381in}}%
\pgfpathcurveto{\pgfqpoint{2.192750in}{1.648144in}}{\pgfqpoint{2.196022in}{1.640244in}}{\pgfqpoint{2.201846in}{1.634420in}}%
\pgfpathcurveto{\pgfqpoint{2.207670in}{1.628596in}}{\pgfqpoint{2.215570in}{1.625324in}}{\pgfqpoint{2.223806in}{1.625324in}}%
\pgfpathclose%
\pgfusepath{stroke,fill}%
\end{pgfscope}%
\begin{pgfscope}%
\pgfpathrectangle{\pgfqpoint{0.100000in}{0.212622in}}{\pgfqpoint{3.696000in}{3.696000in}}%
\pgfusepath{clip}%
\pgfsetbuttcap%
\pgfsetroundjoin%
\definecolor{currentfill}{rgb}{0.121569,0.466667,0.705882}%
\pgfsetfillcolor{currentfill}%
\pgfsetfillopacity{0.764679}%
\pgfsetlinewidth{1.003750pt}%
\definecolor{currentstroke}{rgb}{0.121569,0.466667,0.705882}%
\pgfsetstrokecolor{currentstroke}%
\pgfsetstrokeopacity{0.764679}%
\pgfsetdash{}{0pt}%
\pgfpathmoveto{\pgfqpoint{2.227786in}{1.626116in}}%
\pgfpathcurveto{\pgfqpoint{2.236022in}{1.626116in}}{\pgfqpoint{2.243922in}{1.629388in}}{\pgfqpoint{2.249746in}{1.635212in}}%
\pgfpathcurveto{\pgfqpoint{2.255570in}{1.641036in}}{\pgfqpoint{2.258843in}{1.648936in}}{\pgfqpoint{2.258843in}{1.657172in}}%
\pgfpathcurveto{\pgfqpoint{2.258843in}{1.665408in}}{\pgfqpoint{2.255570in}{1.673309in}}{\pgfqpoint{2.249746in}{1.679132in}}%
\pgfpathcurveto{\pgfqpoint{2.243922in}{1.684956in}}{\pgfqpoint{2.236022in}{1.688229in}}{\pgfqpoint{2.227786in}{1.688229in}}%
\pgfpathcurveto{\pgfqpoint{2.219550in}{1.688229in}}{\pgfqpoint{2.211650in}{1.684956in}}{\pgfqpoint{2.205826in}{1.679132in}}%
\pgfpathcurveto{\pgfqpoint{2.200002in}{1.673309in}}{\pgfqpoint{2.196730in}{1.665408in}}{\pgfqpoint{2.196730in}{1.657172in}}%
\pgfpathcurveto{\pgfqpoint{2.196730in}{1.648936in}}{\pgfqpoint{2.200002in}{1.641036in}}{\pgfqpoint{2.205826in}{1.635212in}}%
\pgfpathcurveto{\pgfqpoint{2.211650in}{1.629388in}}{\pgfqpoint{2.219550in}{1.626116in}}{\pgfqpoint{2.227786in}{1.626116in}}%
\pgfpathclose%
\pgfusepath{stroke,fill}%
\end{pgfscope}%
\begin{pgfscope}%
\pgfpathrectangle{\pgfqpoint{0.100000in}{0.212622in}}{\pgfqpoint{3.696000in}{3.696000in}}%
\pgfusepath{clip}%
\pgfsetbuttcap%
\pgfsetroundjoin%
\definecolor{currentfill}{rgb}{0.121569,0.466667,0.705882}%
\pgfsetfillcolor{currentfill}%
\pgfsetfillopacity{0.768229}%
\pgfsetlinewidth{1.003750pt}%
\definecolor{currentstroke}{rgb}{0.121569,0.466667,0.705882}%
\pgfsetstrokecolor{currentstroke}%
\pgfsetstrokeopacity{0.768229}%
\pgfsetdash{}{0pt}%
\pgfpathmoveto{\pgfqpoint{2.230498in}{1.621022in}}%
\pgfpathcurveto{\pgfqpoint{2.238734in}{1.621022in}}{\pgfqpoint{2.246634in}{1.624295in}}{\pgfqpoint{2.252458in}{1.630118in}}%
\pgfpathcurveto{\pgfqpoint{2.258282in}{1.635942in}}{\pgfqpoint{2.261554in}{1.643842in}}{\pgfqpoint{2.261554in}{1.652079in}}%
\pgfpathcurveto{\pgfqpoint{2.261554in}{1.660315in}}{\pgfqpoint{2.258282in}{1.668215in}}{\pgfqpoint{2.252458in}{1.674039in}}%
\pgfpathcurveto{\pgfqpoint{2.246634in}{1.679863in}}{\pgfqpoint{2.238734in}{1.683135in}}{\pgfqpoint{2.230498in}{1.683135in}}%
\pgfpathcurveto{\pgfqpoint{2.222261in}{1.683135in}}{\pgfqpoint{2.214361in}{1.679863in}}{\pgfqpoint{2.208537in}{1.674039in}}%
\pgfpathcurveto{\pgfqpoint{2.202713in}{1.668215in}}{\pgfqpoint{2.199441in}{1.660315in}}{\pgfqpoint{2.199441in}{1.652079in}}%
\pgfpathcurveto{\pgfqpoint{2.199441in}{1.643842in}}{\pgfqpoint{2.202713in}{1.635942in}}{\pgfqpoint{2.208537in}{1.630118in}}%
\pgfpathcurveto{\pgfqpoint{2.214361in}{1.624295in}}{\pgfqpoint{2.222261in}{1.621022in}}{\pgfqpoint{2.230498in}{1.621022in}}%
\pgfpathclose%
\pgfusepath{stroke,fill}%
\end{pgfscope}%
\begin{pgfscope}%
\pgfpathrectangle{\pgfqpoint{0.100000in}{0.212622in}}{\pgfqpoint{3.696000in}{3.696000in}}%
\pgfusepath{clip}%
\pgfsetbuttcap%
\pgfsetroundjoin%
\definecolor{currentfill}{rgb}{0.121569,0.466667,0.705882}%
\pgfsetfillcolor{currentfill}%
\pgfsetfillopacity{0.769960}%
\pgfsetlinewidth{1.003750pt}%
\definecolor{currentstroke}{rgb}{0.121569,0.466667,0.705882}%
\pgfsetstrokecolor{currentstroke}%
\pgfsetstrokeopacity{0.769960}%
\pgfsetdash{}{0pt}%
\pgfpathmoveto{\pgfqpoint{2.232320in}{1.617021in}}%
\pgfpathcurveto{\pgfqpoint{2.240557in}{1.617021in}}{\pgfqpoint{2.248457in}{1.620293in}}{\pgfqpoint{2.254281in}{1.626117in}}%
\pgfpathcurveto{\pgfqpoint{2.260104in}{1.631941in}}{\pgfqpoint{2.263377in}{1.639841in}}{\pgfqpoint{2.263377in}{1.648077in}}%
\pgfpathcurveto{\pgfqpoint{2.263377in}{1.656314in}}{\pgfqpoint{2.260104in}{1.664214in}}{\pgfqpoint{2.254281in}{1.670038in}}%
\pgfpathcurveto{\pgfqpoint{2.248457in}{1.675862in}}{\pgfqpoint{2.240557in}{1.679134in}}{\pgfqpoint{2.232320in}{1.679134in}}%
\pgfpathcurveto{\pgfqpoint{2.224084in}{1.679134in}}{\pgfqpoint{2.216184in}{1.675862in}}{\pgfqpoint{2.210360in}{1.670038in}}%
\pgfpathcurveto{\pgfqpoint{2.204536in}{1.664214in}}{\pgfqpoint{2.201264in}{1.656314in}}{\pgfqpoint{2.201264in}{1.648077in}}%
\pgfpathcurveto{\pgfqpoint{2.201264in}{1.639841in}}{\pgfqpoint{2.204536in}{1.631941in}}{\pgfqpoint{2.210360in}{1.626117in}}%
\pgfpathcurveto{\pgfqpoint{2.216184in}{1.620293in}}{\pgfqpoint{2.224084in}{1.617021in}}{\pgfqpoint{2.232320in}{1.617021in}}%
\pgfpathclose%
\pgfusepath{stroke,fill}%
\end{pgfscope}%
\begin{pgfscope}%
\pgfpathrectangle{\pgfqpoint{0.100000in}{0.212622in}}{\pgfqpoint{3.696000in}{3.696000in}}%
\pgfusepath{clip}%
\pgfsetbuttcap%
\pgfsetroundjoin%
\definecolor{currentfill}{rgb}{0.121569,0.466667,0.705882}%
\pgfsetfillcolor{currentfill}%
\pgfsetfillopacity{0.772193}%
\pgfsetlinewidth{1.003750pt}%
\definecolor{currentstroke}{rgb}{0.121569,0.466667,0.705882}%
\pgfsetstrokecolor{currentstroke}%
\pgfsetstrokeopacity{0.772193}%
\pgfsetdash{}{0pt}%
\pgfpathmoveto{\pgfqpoint{2.234178in}{1.614087in}}%
\pgfpathcurveto{\pgfqpoint{2.242414in}{1.614087in}}{\pgfqpoint{2.250314in}{1.617359in}}{\pgfqpoint{2.256138in}{1.623183in}}%
\pgfpathcurveto{\pgfqpoint{2.261962in}{1.629007in}}{\pgfqpoint{2.265234in}{1.636907in}}{\pgfqpoint{2.265234in}{1.645143in}}%
\pgfpathcurveto{\pgfqpoint{2.265234in}{1.653379in}}{\pgfqpoint{2.261962in}{1.661280in}}{\pgfqpoint{2.256138in}{1.667103in}}%
\pgfpathcurveto{\pgfqpoint{2.250314in}{1.672927in}}{\pgfqpoint{2.242414in}{1.676200in}}{\pgfqpoint{2.234178in}{1.676200in}}%
\pgfpathcurveto{\pgfqpoint{2.225941in}{1.676200in}}{\pgfqpoint{2.218041in}{1.672927in}}{\pgfqpoint{2.212217in}{1.667103in}}%
\pgfpathcurveto{\pgfqpoint{2.206393in}{1.661280in}}{\pgfqpoint{2.203121in}{1.653379in}}{\pgfqpoint{2.203121in}{1.645143in}}%
\pgfpathcurveto{\pgfqpoint{2.203121in}{1.636907in}}{\pgfqpoint{2.206393in}{1.629007in}}{\pgfqpoint{2.212217in}{1.623183in}}%
\pgfpathcurveto{\pgfqpoint{2.218041in}{1.617359in}}{\pgfqpoint{2.225941in}{1.614087in}}{\pgfqpoint{2.234178in}{1.614087in}}%
\pgfpathclose%
\pgfusepath{stroke,fill}%
\end{pgfscope}%
\begin{pgfscope}%
\pgfpathrectangle{\pgfqpoint{0.100000in}{0.212622in}}{\pgfqpoint{3.696000in}{3.696000in}}%
\pgfusepath{clip}%
\pgfsetbuttcap%
\pgfsetroundjoin%
\definecolor{currentfill}{rgb}{0.121569,0.466667,0.705882}%
\pgfsetfillcolor{currentfill}%
\pgfsetfillopacity{0.773627}%
\pgfsetlinewidth{1.003750pt}%
\definecolor{currentstroke}{rgb}{0.121569,0.466667,0.705882}%
\pgfsetstrokecolor{currentstroke}%
\pgfsetstrokeopacity{0.773627}%
\pgfsetdash{}{0pt}%
\pgfpathmoveto{\pgfqpoint{2.235518in}{1.613981in}}%
\pgfpathcurveto{\pgfqpoint{2.243754in}{1.613981in}}{\pgfqpoint{2.251654in}{1.617253in}}{\pgfqpoint{2.257478in}{1.623077in}}%
\pgfpathcurveto{\pgfqpoint{2.263302in}{1.628901in}}{\pgfqpoint{2.266575in}{1.636801in}}{\pgfqpoint{2.266575in}{1.645037in}}%
\pgfpathcurveto{\pgfqpoint{2.266575in}{1.653273in}}{\pgfqpoint{2.263302in}{1.661173in}}{\pgfqpoint{2.257478in}{1.666997in}}%
\pgfpathcurveto{\pgfqpoint{2.251654in}{1.672821in}}{\pgfqpoint{2.243754in}{1.676094in}}{\pgfqpoint{2.235518in}{1.676094in}}%
\pgfpathcurveto{\pgfqpoint{2.227282in}{1.676094in}}{\pgfqpoint{2.219382in}{1.672821in}}{\pgfqpoint{2.213558in}{1.666997in}}%
\pgfpathcurveto{\pgfqpoint{2.207734in}{1.661173in}}{\pgfqpoint{2.204462in}{1.653273in}}{\pgfqpoint{2.204462in}{1.645037in}}%
\pgfpathcurveto{\pgfqpoint{2.204462in}{1.636801in}}{\pgfqpoint{2.207734in}{1.628901in}}{\pgfqpoint{2.213558in}{1.623077in}}%
\pgfpathcurveto{\pgfqpoint{2.219382in}{1.617253in}}{\pgfqpoint{2.227282in}{1.613981in}}{\pgfqpoint{2.235518in}{1.613981in}}%
\pgfpathclose%
\pgfusepath{stroke,fill}%
\end{pgfscope}%
\begin{pgfscope}%
\pgfpathrectangle{\pgfqpoint{0.100000in}{0.212622in}}{\pgfqpoint{3.696000in}{3.696000in}}%
\pgfusepath{clip}%
\pgfsetbuttcap%
\pgfsetroundjoin%
\definecolor{currentfill}{rgb}{0.121569,0.466667,0.705882}%
\pgfsetfillcolor{currentfill}%
\pgfsetfillopacity{0.774228}%
\pgfsetlinewidth{1.003750pt}%
\definecolor{currentstroke}{rgb}{0.121569,0.466667,0.705882}%
\pgfsetstrokecolor{currentstroke}%
\pgfsetstrokeopacity{0.774228}%
\pgfsetdash{}{0pt}%
\pgfpathmoveto{\pgfqpoint{2.235853in}{1.612501in}}%
\pgfpathcurveto{\pgfqpoint{2.244089in}{1.612501in}}{\pgfqpoint{2.251989in}{1.615773in}}{\pgfqpoint{2.257813in}{1.621597in}}%
\pgfpathcurveto{\pgfqpoint{2.263637in}{1.627421in}}{\pgfqpoint{2.266909in}{1.635321in}}{\pgfqpoint{2.266909in}{1.643558in}}%
\pgfpathcurveto{\pgfqpoint{2.266909in}{1.651794in}}{\pgfqpoint{2.263637in}{1.659694in}}{\pgfqpoint{2.257813in}{1.665518in}}%
\pgfpathcurveto{\pgfqpoint{2.251989in}{1.671342in}}{\pgfqpoint{2.244089in}{1.674614in}}{\pgfqpoint{2.235853in}{1.674614in}}%
\pgfpathcurveto{\pgfqpoint{2.227616in}{1.674614in}}{\pgfqpoint{2.219716in}{1.671342in}}{\pgfqpoint{2.213892in}{1.665518in}}%
\pgfpathcurveto{\pgfqpoint{2.208068in}{1.659694in}}{\pgfqpoint{2.204796in}{1.651794in}}{\pgfqpoint{2.204796in}{1.643558in}}%
\pgfpathcurveto{\pgfqpoint{2.204796in}{1.635321in}}{\pgfqpoint{2.208068in}{1.627421in}}{\pgfqpoint{2.213892in}{1.621597in}}%
\pgfpathcurveto{\pgfqpoint{2.219716in}{1.615773in}}{\pgfqpoint{2.227616in}{1.612501in}}{\pgfqpoint{2.235853in}{1.612501in}}%
\pgfpathclose%
\pgfusepath{stroke,fill}%
\end{pgfscope}%
\begin{pgfscope}%
\pgfpathrectangle{\pgfqpoint{0.100000in}{0.212622in}}{\pgfqpoint{3.696000in}{3.696000in}}%
\pgfusepath{clip}%
\pgfsetbuttcap%
\pgfsetroundjoin%
\definecolor{currentfill}{rgb}{0.121569,0.466667,0.705882}%
\pgfsetfillcolor{currentfill}%
\pgfsetfillopacity{0.775115}%
\pgfsetlinewidth{1.003750pt}%
\definecolor{currentstroke}{rgb}{0.121569,0.466667,0.705882}%
\pgfsetstrokecolor{currentstroke}%
\pgfsetstrokeopacity{0.775115}%
\pgfsetdash{}{0pt}%
\pgfpathmoveto{\pgfqpoint{2.237022in}{1.610049in}}%
\pgfpathcurveto{\pgfqpoint{2.245258in}{1.610049in}}{\pgfqpoint{2.253158in}{1.613321in}}{\pgfqpoint{2.258982in}{1.619145in}}%
\pgfpathcurveto{\pgfqpoint{2.264806in}{1.624969in}}{\pgfqpoint{2.268078in}{1.632869in}}{\pgfqpoint{2.268078in}{1.641105in}}%
\pgfpathcurveto{\pgfqpoint{2.268078in}{1.649342in}}{\pgfqpoint{2.264806in}{1.657242in}}{\pgfqpoint{2.258982in}{1.663066in}}%
\pgfpathcurveto{\pgfqpoint{2.253158in}{1.668890in}}{\pgfqpoint{2.245258in}{1.672162in}}{\pgfqpoint{2.237022in}{1.672162in}}%
\pgfpathcurveto{\pgfqpoint{2.228786in}{1.672162in}}{\pgfqpoint{2.220885in}{1.668890in}}{\pgfqpoint{2.215062in}{1.663066in}}%
\pgfpathcurveto{\pgfqpoint{2.209238in}{1.657242in}}{\pgfqpoint{2.205965in}{1.649342in}}{\pgfqpoint{2.205965in}{1.641105in}}%
\pgfpathcurveto{\pgfqpoint{2.205965in}{1.632869in}}{\pgfqpoint{2.209238in}{1.624969in}}{\pgfqpoint{2.215062in}{1.619145in}}%
\pgfpathcurveto{\pgfqpoint{2.220885in}{1.613321in}}{\pgfqpoint{2.228786in}{1.610049in}}{\pgfqpoint{2.237022in}{1.610049in}}%
\pgfpathclose%
\pgfusepath{stroke,fill}%
\end{pgfscope}%
\begin{pgfscope}%
\pgfpathrectangle{\pgfqpoint{0.100000in}{0.212622in}}{\pgfqpoint{3.696000in}{3.696000in}}%
\pgfusepath{clip}%
\pgfsetbuttcap%
\pgfsetroundjoin%
\definecolor{currentfill}{rgb}{0.121569,0.466667,0.705882}%
\pgfsetfillcolor{currentfill}%
\pgfsetfillopacity{0.775697}%
\pgfsetlinewidth{1.003750pt}%
\definecolor{currentstroke}{rgb}{0.121569,0.466667,0.705882}%
\pgfsetstrokecolor{currentstroke}%
\pgfsetstrokeopacity{0.775697}%
\pgfsetdash{}{0pt}%
\pgfpathmoveto{\pgfqpoint{2.237494in}{1.609178in}}%
\pgfpathcurveto{\pgfqpoint{2.245730in}{1.609178in}}{\pgfqpoint{2.253630in}{1.612450in}}{\pgfqpoint{2.259454in}{1.618274in}}%
\pgfpathcurveto{\pgfqpoint{2.265278in}{1.624098in}}{\pgfqpoint{2.268550in}{1.631998in}}{\pgfqpoint{2.268550in}{1.640234in}}%
\pgfpathcurveto{\pgfqpoint{2.268550in}{1.648470in}}{\pgfqpoint{2.265278in}{1.656370in}}{\pgfqpoint{2.259454in}{1.662194in}}%
\pgfpathcurveto{\pgfqpoint{2.253630in}{1.668018in}}{\pgfqpoint{2.245730in}{1.671291in}}{\pgfqpoint{2.237494in}{1.671291in}}%
\pgfpathcurveto{\pgfqpoint{2.229258in}{1.671291in}}{\pgfqpoint{2.221357in}{1.668018in}}{\pgfqpoint{2.215534in}{1.662194in}}%
\pgfpathcurveto{\pgfqpoint{2.209710in}{1.656370in}}{\pgfqpoint{2.206437in}{1.648470in}}{\pgfqpoint{2.206437in}{1.640234in}}%
\pgfpathcurveto{\pgfqpoint{2.206437in}{1.631998in}}{\pgfqpoint{2.209710in}{1.624098in}}{\pgfqpoint{2.215534in}{1.618274in}}%
\pgfpathcurveto{\pgfqpoint{2.221357in}{1.612450in}}{\pgfqpoint{2.229258in}{1.609178in}}{\pgfqpoint{2.237494in}{1.609178in}}%
\pgfpathclose%
\pgfusepath{stroke,fill}%
\end{pgfscope}%
\begin{pgfscope}%
\pgfpathrectangle{\pgfqpoint{0.100000in}{0.212622in}}{\pgfqpoint{3.696000in}{3.696000in}}%
\pgfusepath{clip}%
\pgfsetbuttcap%
\pgfsetroundjoin%
\definecolor{currentfill}{rgb}{0.121569,0.466667,0.705882}%
\pgfsetfillcolor{currentfill}%
\pgfsetfillopacity{0.776582}%
\pgfsetlinewidth{1.003750pt}%
\definecolor{currentstroke}{rgb}{0.121569,0.466667,0.705882}%
\pgfsetstrokecolor{currentstroke}%
\pgfsetstrokeopacity{0.776582}%
\pgfsetdash{}{0pt}%
\pgfpathmoveto{\pgfqpoint{2.238420in}{1.609019in}}%
\pgfpathcurveto{\pgfqpoint{2.246657in}{1.609019in}}{\pgfqpoint{2.254557in}{1.612291in}}{\pgfqpoint{2.260381in}{1.618115in}}%
\pgfpathcurveto{\pgfqpoint{2.266205in}{1.623939in}}{\pgfqpoint{2.269477in}{1.631839in}}{\pgfqpoint{2.269477in}{1.640075in}}%
\pgfpathcurveto{\pgfqpoint{2.269477in}{1.648312in}}{\pgfqpoint{2.266205in}{1.656212in}}{\pgfqpoint{2.260381in}{1.662036in}}%
\pgfpathcurveto{\pgfqpoint{2.254557in}{1.667860in}}{\pgfqpoint{2.246657in}{1.671132in}}{\pgfqpoint{2.238420in}{1.671132in}}%
\pgfpathcurveto{\pgfqpoint{2.230184in}{1.671132in}}{\pgfqpoint{2.222284in}{1.667860in}}{\pgfqpoint{2.216460in}{1.662036in}}%
\pgfpathcurveto{\pgfqpoint{2.210636in}{1.656212in}}{\pgfqpoint{2.207364in}{1.648312in}}{\pgfqpoint{2.207364in}{1.640075in}}%
\pgfpathcurveto{\pgfqpoint{2.207364in}{1.631839in}}{\pgfqpoint{2.210636in}{1.623939in}}{\pgfqpoint{2.216460in}{1.618115in}}%
\pgfpathcurveto{\pgfqpoint{2.222284in}{1.612291in}}{\pgfqpoint{2.230184in}{1.609019in}}{\pgfqpoint{2.238420in}{1.609019in}}%
\pgfpathclose%
\pgfusepath{stroke,fill}%
\end{pgfscope}%
\begin{pgfscope}%
\pgfpathrectangle{\pgfqpoint{0.100000in}{0.212622in}}{\pgfqpoint{3.696000in}{3.696000in}}%
\pgfusepath{clip}%
\pgfsetbuttcap%
\pgfsetroundjoin%
\definecolor{currentfill}{rgb}{0.121569,0.466667,0.705882}%
\pgfsetfillcolor{currentfill}%
\pgfsetfillopacity{0.776989}%
\pgfsetlinewidth{1.003750pt}%
\definecolor{currentstroke}{rgb}{0.121569,0.466667,0.705882}%
\pgfsetstrokecolor{currentstroke}%
\pgfsetstrokeopacity{0.776989}%
\pgfsetdash{}{0pt}%
\pgfpathmoveto{\pgfqpoint{2.238742in}{1.608295in}}%
\pgfpathcurveto{\pgfqpoint{2.246979in}{1.608295in}}{\pgfqpoint{2.254879in}{1.611567in}}{\pgfqpoint{2.260703in}{1.617391in}}%
\pgfpathcurveto{\pgfqpoint{2.266526in}{1.623215in}}{\pgfqpoint{2.269799in}{1.631115in}}{\pgfqpoint{2.269799in}{1.639351in}}%
\pgfpathcurveto{\pgfqpoint{2.269799in}{1.647588in}}{\pgfqpoint{2.266526in}{1.655488in}}{\pgfqpoint{2.260703in}{1.661312in}}%
\pgfpathcurveto{\pgfqpoint{2.254879in}{1.667136in}}{\pgfqpoint{2.246979in}{1.670408in}}{\pgfqpoint{2.238742in}{1.670408in}}%
\pgfpathcurveto{\pgfqpoint{2.230506in}{1.670408in}}{\pgfqpoint{2.222606in}{1.667136in}}{\pgfqpoint{2.216782in}{1.661312in}}%
\pgfpathcurveto{\pgfqpoint{2.210958in}{1.655488in}}{\pgfqpoint{2.207686in}{1.647588in}}{\pgfqpoint{2.207686in}{1.639351in}}%
\pgfpathcurveto{\pgfqpoint{2.207686in}{1.631115in}}{\pgfqpoint{2.210958in}{1.623215in}}{\pgfqpoint{2.216782in}{1.617391in}}%
\pgfpathcurveto{\pgfqpoint{2.222606in}{1.611567in}}{\pgfqpoint{2.230506in}{1.608295in}}{\pgfqpoint{2.238742in}{1.608295in}}%
\pgfpathclose%
\pgfusepath{stroke,fill}%
\end{pgfscope}%
\begin{pgfscope}%
\pgfpathrectangle{\pgfqpoint{0.100000in}{0.212622in}}{\pgfqpoint{3.696000in}{3.696000in}}%
\pgfusepath{clip}%
\pgfsetbuttcap%
\pgfsetroundjoin%
\definecolor{currentfill}{rgb}{0.121569,0.466667,0.705882}%
\pgfsetfillcolor{currentfill}%
\pgfsetfillopacity{0.777868}%
\pgfsetlinewidth{1.003750pt}%
\definecolor{currentstroke}{rgb}{0.121569,0.466667,0.705882}%
\pgfsetstrokecolor{currentstroke}%
\pgfsetstrokeopacity{0.777868}%
\pgfsetdash{}{0pt}%
\pgfpathmoveto{\pgfqpoint{2.239690in}{1.605405in}}%
\pgfpathcurveto{\pgfqpoint{2.247926in}{1.605405in}}{\pgfqpoint{2.255826in}{1.608678in}}{\pgfqpoint{2.261650in}{1.614502in}}%
\pgfpathcurveto{\pgfqpoint{2.267474in}{1.620326in}}{\pgfqpoint{2.270746in}{1.628226in}}{\pgfqpoint{2.270746in}{1.636462in}}%
\pgfpathcurveto{\pgfqpoint{2.270746in}{1.644698in}}{\pgfqpoint{2.267474in}{1.652598in}}{\pgfqpoint{2.261650in}{1.658422in}}%
\pgfpathcurveto{\pgfqpoint{2.255826in}{1.664246in}}{\pgfqpoint{2.247926in}{1.667518in}}{\pgfqpoint{2.239690in}{1.667518in}}%
\pgfpathcurveto{\pgfqpoint{2.231453in}{1.667518in}}{\pgfqpoint{2.223553in}{1.664246in}}{\pgfqpoint{2.217729in}{1.658422in}}%
\pgfpathcurveto{\pgfqpoint{2.211905in}{1.652598in}}{\pgfqpoint{2.208633in}{1.644698in}}{\pgfqpoint{2.208633in}{1.636462in}}%
\pgfpathcurveto{\pgfqpoint{2.208633in}{1.628226in}}{\pgfqpoint{2.211905in}{1.620326in}}{\pgfqpoint{2.217729in}{1.614502in}}%
\pgfpathcurveto{\pgfqpoint{2.223553in}{1.608678in}}{\pgfqpoint{2.231453in}{1.605405in}}{\pgfqpoint{2.239690in}{1.605405in}}%
\pgfpathclose%
\pgfusepath{stroke,fill}%
\end{pgfscope}%
\begin{pgfscope}%
\pgfpathrectangle{\pgfqpoint{0.100000in}{0.212622in}}{\pgfqpoint{3.696000in}{3.696000in}}%
\pgfusepath{clip}%
\pgfsetbuttcap%
\pgfsetroundjoin%
\definecolor{currentfill}{rgb}{0.121569,0.466667,0.705882}%
\pgfsetfillcolor{currentfill}%
\pgfsetfillopacity{0.779172}%
\pgfsetlinewidth{1.003750pt}%
\definecolor{currentstroke}{rgb}{0.121569,0.466667,0.705882}%
\pgfsetstrokecolor{currentstroke}%
\pgfsetstrokeopacity{0.779172}%
\pgfsetdash{}{0pt}%
\pgfpathmoveto{\pgfqpoint{2.240744in}{1.603870in}}%
\pgfpathcurveto{\pgfqpoint{2.248980in}{1.603870in}}{\pgfqpoint{2.256880in}{1.607143in}}{\pgfqpoint{2.262704in}{1.612967in}}%
\pgfpathcurveto{\pgfqpoint{2.268528in}{1.618791in}}{\pgfqpoint{2.271801in}{1.626691in}}{\pgfqpoint{2.271801in}{1.634927in}}%
\pgfpathcurveto{\pgfqpoint{2.271801in}{1.643163in}}{\pgfqpoint{2.268528in}{1.651063in}}{\pgfqpoint{2.262704in}{1.656887in}}%
\pgfpathcurveto{\pgfqpoint{2.256880in}{1.662711in}}{\pgfqpoint{2.248980in}{1.665983in}}{\pgfqpoint{2.240744in}{1.665983in}}%
\pgfpathcurveto{\pgfqpoint{2.232508in}{1.665983in}}{\pgfqpoint{2.224608in}{1.662711in}}{\pgfqpoint{2.218784in}{1.656887in}}%
\pgfpathcurveto{\pgfqpoint{2.212960in}{1.651063in}}{\pgfqpoint{2.209688in}{1.643163in}}{\pgfqpoint{2.209688in}{1.634927in}}%
\pgfpathcurveto{\pgfqpoint{2.209688in}{1.626691in}}{\pgfqpoint{2.212960in}{1.618791in}}{\pgfqpoint{2.218784in}{1.612967in}}%
\pgfpathcurveto{\pgfqpoint{2.224608in}{1.607143in}}{\pgfqpoint{2.232508in}{1.603870in}}{\pgfqpoint{2.240744in}{1.603870in}}%
\pgfpathclose%
\pgfusepath{stroke,fill}%
\end{pgfscope}%
\begin{pgfscope}%
\pgfpathrectangle{\pgfqpoint{0.100000in}{0.212622in}}{\pgfqpoint{3.696000in}{3.696000in}}%
\pgfusepath{clip}%
\pgfsetbuttcap%
\pgfsetroundjoin%
\definecolor{currentfill}{rgb}{0.121569,0.466667,0.705882}%
\pgfsetfillcolor{currentfill}%
\pgfsetfillopacity{0.781119}%
\pgfsetlinewidth{1.003750pt}%
\definecolor{currentstroke}{rgb}{0.121569,0.466667,0.705882}%
\pgfsetstrokecolor{currentstroke}%
\pgfsetstrokeopacity{0.781119}%
\pgfsetdash{}{0pt}%
\pgfpathmoveto{\pgfqpoint{2.243162in}{1.602633in}}%
\pgfpathcurveto{\pgfqpoint{2.251399in}{1.602633in}}{\pgfqpoint{2.259299in}{1.605906in}}{\pgfqpoint{2.265123in}{1.611730in}}%
\pgfpathcurveto{\pgfqpoint{2.270947in}{1.617554in}}{\pgfqpoint{2.274219in}{1.625454in}}{\pgfqpoint{2.274219in}{1.633690in}}%
\pgfpathcurveto{\pgfqpoint{2.274219in}{1.641926in}}{\pgfqpoint{2.270947in}{1.649826in}}{\pgfqpoint{2.265123in}{1.655650in}}%
\pgfpathcurveto{\pgfqpoint{2.259299in}{1.661474in}}{\pgfqpoint{2.251399in}{1.664746in}}{\pgfqpoint{2.243162in}{1.664746in}}%
\pgfpathcurveto{\pgfqpoint{2.234926in}{1.664746in}}{\pgfqpoint{2.227026in}{1.661474in}}{\pgfqpoint{2.221202in}{1.655650in}}%
\pgfpathcurveto{\pgfqpoint{2.215378in}{1.649826in}}{\pgfqpoint{2.212106in}{1.641926in}}{\pgfqpoint{2.212106in}{1.633690in}}%
\pgfpathcurveto{\pgfqpoint{2.212106in}{1.625454in}}{\pgfqpoint{2.215378in}{1.617554in}}{\pgfqpoint{2.221202in}{1.611730in}}%
\pgfpathcurveto{\pgfqpoint{2.227026in}{1.605906in}}{\pgfqpoint{2.234926in}{1.602633in}}{\pgfqpoint{2.243162in}{1.602633in}}%
\pgfpathclose%
\pgfusepath{stroke,fill}%
\end{pgfscope}%
\begin{pgfscope}%
\pgfpathrectangle{\pgfqpoint{0.100000in}{0.212622in}}{\pgfqpoint{3.696000in}{3.696000in}}%
\pgfusepath{clip}%
\pgfsetbuttcap%
\pgfsetroundjoin%
\definecolor{currentfill}{rgb}{0.121569,0.466667,0.705882}%
\pgfsetfillcolor{currentfill}%
\pgfsetfillopacity{0.782136}%
\pgfsetlinewidth{1.003750pt}%
\definecolor{currentstroke}{rgb}{0.121569,0.466667,0.705882}%
\pgfsetstrokecolor{currentstroke}%
\pgfsetstrokeopacity{0.782136}%
\pgfsetdash{}{0pt}%
\pgfpathmoveto{\pgfqpoint{2.243441in}{1.600986in}}%
\pgfpathcurveto{\pgfqpoint{2.251678in}{1.600986in}}{\pgfqpoint{2.259578in}{1.604258in}}{\pgfqpoint{2.265402in}{1.610082in}}%
\pgfpathcurveto{\pgfqpoint{2.271226in}{1.615906in}}{\pgfqpoint{2.274498in}{1.623806in}}{\pgfqpoint{2.274498in}{1.632043in}}%
\pgfpathcurveto{\pgfqpoint{2.274498in}{1.640279in}}{\pgfqpoint{2.271226in}{1.648179in}}{\pgfqpoint{2.265402in}{1.654003in}}%
\pgfpathcurveto{\pgfqpoint{2.259578in}{1.659827in}}{\pgfqpoint{2.251678in}{1.663099in}}{\pgfqpoint{2.243441in}{1.663099in}}%
\pgfpathcurveto{\pgfqpoint{2.235205in}{1.663099in}}{\pgfqpoint{2.227305in}{1.659827in}}{\pgfqpoint{2.221481in}{1.654003in}}%
\pgfpathcurveto{\pgfqpoint{2.215657in}{1.648179in}}{\pgfqpoint{2.212385in}{1.640279in}}{\pgfqpoint{2.212385in}{1.632043in}}%
\pgfpathcurveto{\pgfqpoint{2.212385in}{1.623806in}}{\pgfqpoint{2.215657in}{1.615906in}}{\pgfqpoint{2.221481in}{1.610082in}}%
\pgfpathcurveto{\pgfqpoint{2.227305in}{1.604258in}}{\pgfqpoint{2.235205in}{1.600986in}}{\pgfqpoint{2.243441in}{1.600986in}}%
\pgfpathclose%
\pgfusepath{stroke,fill}%
\end{pgfscope}%
\begin{pgfscope}%
\pgfpathrectangle{\pgfqpoint{0.100000in}{0.212622in}}{\pgfqpoint{3.696000in}{3.696000in}}%
\pgfusepath{clip}%
\pgfsetbuttcap%
\pgfsetroundjoin%
\definecolor{currentfill}{rgb}{0.121569,0.466667,0.705882}%
\pgfsetfillcolor{currentfill}%
\pgfsetfillopacity{0.783312}%
\pgfsetlinewidth{1.003750pt}%
\definecolor{currentstroke}{rgb}{0.121569,0.466667,0.705882}%
\pgfsetstrokecolor{currentstroke}%
\pgfsetstrokeopacity{0.783312}%
\pgfsetdash{}{0pt}%
\pgfpathmoveto{\pgfqpoint{2.244831in}{1.595577in}}%
\pgfpathcurveto{\pgfqpoint{2.253067in}{1.595577in}}{\pgfqpoint{2.260967in}{1.598849in}}{\pgfqpoint{2.266791in}{1.604673in}}%
\pgfpathcurveto{\pgfqpoint{2.272615in}{1.610497in}}{\pgfqpoint{2.275887in}{1.618397in}}{\pgfqpoint{2.275887in}{1.626633in}}%
\pgfpathcurveto{\pgfqpoint{2.275887in}{1.634869in}}{\pgfqpoint{2.272615in}{1.642769in}}{\pgfqpoint{2.266791in}{1.648593in}}%
\pgfpathcurveto{\pgfqpoint{2.260967in}{1.654417in}}{\pgfqpoint{2.253067in}{1.657690in}}{\pgfqpoint{2.244831in}{1.657690in}}%
\pgfpathcurveto{\pgfqpoint{2.236594in}{1.657690in}}{\pgfqpoint{2.228694in}{1.654417in}}{\pgfqpoint{2.222871in}{1.648593in}}%
\pgfpathcurveto{\pgfqpoint{2.217047in}{1.642769in}}{\pgfqpoint{2.213774in}{1.634869in}}{\pgfqpoint{2.213774in}{1.626633in}}%
\pgfpathcurveto{\pgfqpoint{2.213774in}{1.618397in}}{\pgfqpoint{2.217047in}{1.610497in}}{\pgfqpoint{2.222871in}{1.604673in}}%
\pgfpathcurveto{\pgfqpoint{2.228694in}{1.598849in}}{\pgfqpoint{2.236594in}{1.595577in}}{\pgfqpoint{2.244831in}{1.595577in}}%
\pgfpathclose%
\pgfusepath{stroke,fill}%
\end{pgfscope}%
\begin{pgfscope}%
\pgfpathrectangle{\pgfqpoint{0.100000in}{0.212622in}}{\pgfqpoint{3.696000in}{3.696000in}}%
\pgfusepath{clip}%
\pgfsetbuttcap%
\pgfsetroundjoin%
\definecolor{currentfill}{rgb}{0.121569,0.466667,0.705882}%
\pgfsetfillcolor{currentfill}%
\pgfsetfillopacity{0.784307}%
\pgfsetlinewidth{1.003750pt}%
\definecolor{currentstroke}{rgb}{0.121569,0.466667,0.705882}%
\pgfsetstrokecolor{currentstroke}%
\pgfsetstrokeopacity{0.784307}%
\pgfsetdash{}{0pt}%
\pgfpathmoveto{\pgfqpoint{2.245780in}{1.594901in}}%
\pgfpathcurveto{\pgfqpoint{2.254017in}{1.594901in}}{\pgfqpoint{2.261917in}{1.598173in}}{\pgfqpoint{2.267740in}{1.603997in}}%
\pgfpathcurveto{\pgfqpoint{2.273564in}{1.609821in}}{\pgfqpoint{2.276837in}{1.617721in}}{\pgfqpoint{2.276837in}{1.625958in}}%
\pgfpathcurveto{\pgfqpoint{2.276837in}{1.634194in}}{\pgfqpoint{2.273564in}{1.642094in}}{\pgfqpoint{2.267740in}{1.647918in}}%
\pgfpathcurveto{\pgfqpoint{2.261917in}{1.653742in}}{\pgfqpoint{2.254017in}{1.657014in}}{\pgfqpoint{2.245780in}{1.657014in}}%
\pgfpathcurveto{\pgfqpoint{2.237544in}{1.657014in}}{\pgfqpoint{2.229644in}{1.653742in}}{\pgfqpoint{2.223820in}{1.647918in}}%
\pgfpathcurveto{\pgfqpoint{2.217996in}{1.642094in}}{\pgfqpoint{2.214724in}{1.634194in}}{\pgfqpoint{2.214724in}{1.625958in}}%
\pgfpathcurveto{\pgfqpoint{2.214724in}{1.617721in}}{\pgfqpoint{2.217996in}{1.609821in}}{\pgfqpoint{2.223820in}{1.603997in}}%
\pgfpathcurveto{\pgfqpoint{2.229644in}{1.598173in}}{\pgfqpoint{2.237544in}{1.594901in}}{\pgfqpoint{2.245780in}{1.594901in}}%
\pgfpathclose%
\pgfusepath{stroke,fill}%
\end{pgfscope}%
\begin{pgfscope}%
\pgfpathrectangle{\pgfqpoint{0.100000in}{0.212622in}}{\pgfqpoint{3.696000in}{3.696000in}}%
\pgfusepath{clip}%
\pgfsetbuttcap%
\pgfsetroundjoin%
\definecolor{currentfill}{rgb}{0.121569,0.466667,0.705882}%
\pgfsetfillcolor{currentfill}%
\pgfsetfillopacity{0.785557}%
\pgfsetlinewidth{1.003750pt}%
\definecolor{currentstroke}{rgb}{0.121569,0.466667,0.705882}%
\pgfsetstrokecolor{currentstroke}%
\pgfsetstrokeopacity{0.785557}%
\pgfsetdash{}{0pt}%
\pgfpathmoveto{\pgfqpoint{2.246790in}{1.594113in}}%
\pgfpathcurveto{\pgfqpoint{2.255026in}{1.594113in}}{\pgfqpoint{2.262926in}{1.597386in}}{\pgfqpoint{2.268750in}{1.603209in}}%
\pgfpathcurveto{\pgfqpoint{2.274574in}{1.609033in}}{\pgfqpoint{2.277846in}{1.616933in}}{\pgfqpoint{2.277846in}{1.625170in}}%
\pgfpathcurveto{\pgfqpoint{2.277846in}{1.633406in}}{\pgfqpoint{2.274574in}{1.641306in}}{\pgfqpoint{2.268750in}{1.647130in}}%
\pgfpathcurveto{\pgfqpoint{2.262926in}{1.652954in}}{\pgfqpoint{2.255026in}{1.656226in}}{\pgfqpoint{2.246790in}{1.656226in}}%
\pgfpathcurveto{\pgfqpoint{2.238553in}{1.656226in}}{\pgfqpoint{2.230653in}{1.652954in}}{\pgfqpoint{2.224829in}{1.647130in}}%
\pgfpathcurveto{\pgfqpoint{2.219005in}{1.641306in}}{\pgfqpoint{2.215733in}{1.633406in}}{\pgfqpoint{2.215733in}{1.625170in}}%
\pgfpathcurveto{\pgfqpoint{2.215733in}{1.616933in}}{\pgfqpoint{2.219005in}{1.609033in}}{\pgfqpoint{2.224829in}{1.603209in}}%
\pgfpathcurveto{\pgfqpoint{2.230653in}{1.597386in}}{\pgfqpoint{2.238553in}{1.594113in}}{\pgfqpoint{2.246790in}{1.594113in}}%
\pgfpathclose%
\pgfusepath{stroke,fill}%
\end{pgfscope}%
\begin{pgfscope}%
\pgfpathrectangle{\pgfqpoint{0.100000in}{0.212622in}}{\pgfqpoint{3.696000in}{3.696000in}}%
\pgfusepath{clip}%
\pgfsetbuttcap%
\pgfsetroundjoin%
\definecolor{currentfill}{rgb}{0.121569,0.466667,0.705882}%
\pgfsetfillcolor{currentfill}%
\pgfsetfillopacity{0.786220}%
\pgfsetlinewidth{1.003750pt}%
\definecolor{currentstroke}{rgb}{0.121569,0.466667,0.705882}%
\pgfsetstrokecolor{currentstroke}%
\pgfsetstrokeopacity{0.786220}%
\pgfsetdash{}{0pt}%
\pgfpathmoveto{\pgfqpoint{2.247088in}{1.593389in}}%
\pgfpathcurveto{\pgfqpoint{2.255324in}{1.593389in}}{\pgfqpoint{2.263224in}{1.596661in}}{\pgfqpoint{2.269048in}{1.602485in}}%
\pgfpathcurveto{\pgfqpoint{2.274872in}{1.608309in}}{\pgfqpoint{2.278144in}{1.616209in}}{\pgfqpoint{2.278144in}{1.624445in}}%
\pgfpathcurveto{\pgfqpoint{2.278144in}{1.632682in}}{\pgfqpoint{2.274872in}{1.640582in}}{\pgfqpoint{2.269048in}{1.646406in}}%
\pgfpathcurveto{\pgfqpoint{2.263224in}{1.652229in}}{\pgfqpoint{2.255324in}{1.655502in}}{\pgfqpoint{2.247088in}{1.655502in}}%
\pgfpathcurveto{\pgfqpoint{2.238851in}{1.655502in}}{\pgfqpoint{2.230951in}{1.652229in}}{\pgfqpoint{2.225127in}{1.646406in}}%
\pgfpathcurveto{\pgfqpoint{2.219304in}{1.640582in}}{\pgfqpoint{2.216031in}{1.632682in}}{\pgfqpoint{2.216031in}{1.624445in}}%
\pgfpathcurveto{\pgfqpoint{2.216031in}{1.616209in}}{\pgfqpoint{2.219304in}{1.608309in}}{\pgfqpoint{2.225127in}{1.602485in}}%
\pgfpathcurveto{\pgfqpoint{2.230951in}{1.596661in}}{\pgfqpoint{2.238851in}{1.593389in}}{\pgfqpoint{2.247088in}{1.593389in}}%
\pgfpathclose%
\pgfusepath{stroke,fill}%
\end{pgfscope}%
\begin{pgfscope}%
\pgfpathrectangle{\pgfqpoint{0.100000in}{0.212622in}}{\pgfqpoint{3.696000in}{3.696000in}}%
\pgfusepath{clip}%
\pgfsetbuttcap%
\pgfsetroundjoin%
\definecolor{currentfill}{rgb}{0.121569,0.466667,0.705882}%
\pgfsetfillcolor{currentfill}%
\pgfsetfillopacity{0.787132}%
\pgfsetlinewidth{1.003750pt}%
\definecolor{currentstroke}{rgb}{0.121569,0.466667,0.705882}%
\pgfsetstrokecolor{currentstroke}%
\pgfsetstrokeopacity{0.787132}%
\pgfsetdash{}{0pt}%
\pgfpathmoveto{\pgfqpoint{2.248118in}{1.590493in}}%
\pgfpathcurveto{\pgfqpoint{2.256355in}{1.590493in}}{\pgfqpoint{2.264255in}{1.593765in}}{\pgfqpoint{2.270079in}{1.599589in}}%
\pgfpathcurveto{\pgfqpoint{2.275903in}{1.605413in}}{\pgfqpoint{2.279175in}{1.613313in}}{\pgfqpoint{2.279175in}{1.621549in}}%
\pgfpathcurveto{\pgfqpoint{2.279175in}{1.629785in}}{\pgfqpoint{2.275903in}{1.637685in}}{\pgfqpoint{2.270079in}{1.643509in}}%
\pgfpathcurveto{\pgfqpoint{2.264255in}{1.649333in}}{\pgfqpoint{2.256355in}{1.652606in}}{\pgfqpoint{2.248118in}{1.652606in}}%
\pgfpathcurveto{\pgfqpoint{2.239882in}{1.652606in}}{\pgfqpoint{2.231982in}{1.649333in}}{\pgfqpoint{2.226158in}{1.643509in}}%
\pgfpathcurveto{\pgfqpoint{2.220334in}{1.637685in}}{\pgfqpoint{2.217062in}{1.629785in}}{\pgfqpoint{2.217062in}{1.621549in}}%
\pgfpathcurveto{\pgfqpoint{2.217062in}{1.613313in}}{\pgfqpoint{2.220334in}{1.605413in}}{\pgfqpoint{2.226158in}{1.599589in}}%
\pgfpathcurveto{\pgfqpoint{2.231982in}{1.593765in}}{\pgfqpoint{2.239882in}{1.590493in}}{\pgfqpoint{2.248118in}{1.590493in}}%
\pgfpathclose%
\pgfusepath{stroke,fill}%
\end{pgfscope}%
\begin{pgfscope}%
\pgfpathrectangle{\pgfqpoint{0.100000in}{0.212622in}}{\pgfqpoint{3.696000in}{3.696000in}}%
\pgfusepath{clip}%
\pgfsetbuttcap%
\pgfsetroundjoin%
\definecolor{currentfill}{rgb}{0.121569,0.466667,0.705882}%
\pgfsetfillcolor{currentfill}%
\pgfsetfillopacity{0.787766}%
\pgfsetlinewidth{1.003750pt}%
\definecolor{currentstroke}{rgb}{0.121569,0.466667,0.705882}%
\pgfsetstrokecolor{currentstroke}%
\pgfsetstrokeopacity{0.787766}%
\pgfsetdash{}{0pt}%
\pgfpathmoveto{\pgfqpoint{2.248782in}{1.589796in}}%
\pgfpathcurveto{\pgfqpoint{2.257018in}{1.589796in}}{\pgfqpoint{2.264918in}{1.593068in}}{\pgfqpoint{2.270742in}{1.598892in}}%
\pgfpathcurveto{\pgfqpoint{2.276566in}{1.604716in}}{\pgfqpoint{2.279838in}{1.612616in}}{\pgfqpoint{2.279838in}{1.620852in}}%
\pgfpathcurveto{\pgfqpoint{2.279838in}{1.629089in}}{\pgfqpoint{2.276566in}{1.636989in}}{\pgfqpoint{2.270742in}{1.642813in}}%
\pgfpathcurveto{\pgfqpoint{2.264918in}{1.648637in}}{\pgfqpoint{2.257018in}{1.651909in}}{\pgfqpoint{2.248782in}{1.651909in}}%
\pgfpathcurveto{\pgfqpoint{2.240545in}{1.651909in}}{\pgfqpoint{2.232645in}{1.648637in}}{\pgfqpoint{2.226821in}{1.642813in}}%
\pgfpathcurveto{\pgfqpoint{2.220998in}{1.636989in}}{\pgfqpoint{2.217725in}{1.629089in}}{\pgfqpoint{2.217725in}{1.620852in}}%
\pgfpathcurveto{\pgfqpoint{2.217725in}{1.612616in}}{\pgfqpoint{2.220998in}{1.604716in}}{\pgfqpoint{2.226821in}{1.598892in}}%
\pgfpathcurveto{\pgfqpoint{2.232645in}{1.593068in}}{\pgfqpoint{2.240545in}{1.589796in}}{\pgfqpoint{2.248782in}{1.589796in}}%
\pgfpathclose%
\pgfusepath{stroke,fill}%
\end{pgfscope}%
\begin{pgfscope}%
\pgfpathrectangle{\pgfqpoint{0.100000in}{0.212622in}}{\pgfqpoint{3.696000in}{3.696000in}}%
\pgfusepath{clip}%
\pgfsetbuttcap%
\pgfsetroundjoin%
\definecolor{currentfill}{rgb}{0.121569,0.466667,0.705882}%
\pgfsetfillcolor{currentfill}%
\pgfsetfillopacity{0.788699}%
\pgfsetlinewidth{1.003750pt}%
\definecolor{currentstroke}{rgb}{0.121569,0.466667,0.705882}%
\pgfsetstrokecolor{currentstroke}%
\pgfsetstrokeopacity{0.788699}%
\pgfsetdash{}{0pt}%
\pgfpathmoveto{\pgfqpoint{2.249587in}{1.589681in}}%
\pgfpathcurveto{\pgfqpoint{2.257823in}{1.589681in}}{\pgfqpoint{2.265723in}{1.592954in}}{\pgfqpoint{2.271547in}{1.598777in}}%
\pgfpathcurveto{\pgfqpoint{2.277371in}{1.604601in}}{\pgfqpoint{2.280644in}{1.612501in}}{\pgfqpoint{2.280644in}{1.620738in}}%
\pgfpathcurveto{\pgfqpoint{2.280644in}{1.628974in}}{\pgfqpoint{2.277371in}{1.636874in}}{\pgfqpoint{2.271547in}{1.642698in}}%
\pgfpathcurveto{\pgfqpoint{2.265723in}{1.648522in}}{\pgfqpoint{2.257823in}{1.651794in}}{\pgfqpoint{2.249587in}{1.651794in}}%
\pgfpathcurveto{\pgfqpoint{2.241351in}{1.651794in}}{\pgfqpoint{2.233451in}{1.648522in}}{\pgfqpoint{2.227627in}{1.642698in}}%
\pgfpathcurveto{\pgfqpoint{2.221803in}{1.636874in}}{\pgfqpoint{2.218531in}{1.628974in}}{\pgfqpoint{2.218531in}{1.620738in}}%
\pgfpathcurveto{\pgfqpoint{2.218531in}{1.612501in}}{\pgfqpoint{2.221803in}{1.604601in}}{\pgfqpoint{2.227627in}{1.598777in}}%
\pgfpathcurveto{\pgfqpoint{2.233451in}{1.592954in}}{\pgfqpoint{2.241351in}{1.589681in}}{\pgfqpoint{2.249587in}{1.589681in}}%
\pgfpathclose%
\pgfusepath{stroke,fill}%
\end{pgfscope}%
\begin{pgfscope}%
\pgfpathrectangle{\pgfqpoint{0.100000in}{0.212622in}}{\pgfqpoint{3.696000in}{3.696000in}}%
\pgfusepath{clip}%
\pgfsetbuttcap%
\pgfsetroundjoin%
\definecolor{currentfill}{rgb}{0.121569,0.466667,0.705882}%
\pgfsetfillcolor{currentfill}%
\pgfsetfillopacity{0.789143}%
\pgfsetlinewidth{1.003750pt}%
\definecolor{currentstroke}{rgb}{0.121569,0.466667,0.705882}%
\pgfsetstrokecolor{currentstroke}%
\pgfsetstrokeopacity{0.789143}%
\pgfsetdash{}{0pt}%
\pgfpathmoveto{\pgfqpoint{2.249830in}{1.589071in}}%
\pgfpathcurveto{\pgfqpoint{2.258067in}{1.589071in}}{\pgfqpoint{2.265967in}{1.592343in}}{\pgfqpoint{2.271791in}{1.598167in}}%
\pgfpathcurveto{\pgfqpoint{2.277615in}{1.603991in}}{\pgfqpoint{2.280887in}{1.611891in}}{\pgfqpoint{2.280887in}{1.620127in}}%
\pgfpathcurveto{\pgfqpoint{2.280887in}{1.628364in}}{\pgfqpoint{2.277615in}{1.636264in}}{\pgfqpoint{2.271791in}{1.642088in}}%
\pgfpathcurveto{\pgfqpoint{2.265967in}{1.647911in}}{\pgfqpoint{2.258067in}{1.651184in}}{\pgfqpoint{2.249830in}{1.651184in}}%
\pgfpathcurveto{\pgfqpoint{2.241594in}{1.651184in}}{\pgfqpoint{2.233694in}{1.647911in}}{\pgfqpoint{2.227870in}{1.642088in}}%
\pgfpathcurveto{\pgfqpoint{2.222046in}{1.636264in}}{\pgfqpoint{2.218774in}{1.628364in}}{\pgfqpoint{2.218774in}{1.620127in}}%
\pgfpathcurveto{\pgfqpoint{2.218774in}{1.611891in}}{\pgfqpoint{2.222046in}{1.603991in}}{\pgfqpoint{2.227870in}{1.598167in}}%
\pgfpathcurveto{\pgfqpoint{2.233694in}{1.592343in}}{\pgfqpoint{2.241594in}{1.589071in}}{\pgfqpoint{2.249830in}{1.589071in}}%
\pgfpathclose%
\pgfusepath{stroke,fill}%
\end{pgfscope}%
\begin{pgfscope}%
\pgfpathrectangle{\pgfqpoint{0.100000in}{0.212622in}}{\pgfqpoint{3.696000in}{3.696000in}}%
\pgfusepath{clip}%
\pgfsetbuttcap%
\pgfsetroundjoin%
\definecolor{currentfill}{rgb}{0.121569,0.466667,0.705882}%
\pgfsetfillcolor{currentfill}%
\pgfsetfillopacity{0.789873}%
\pgfsetlinewidth{1.003750pt}%
\definecolor{currentstroke}{rgb}{0.121569,0.466667,0.705882}%
\pgfsetstrokecolor{currentstroke}%
\pgfsetstrokeopacity{0.789873}%
\pgfsetdash{}{0pt}%
\pgfpathmoveto{\pgfqpoint{2.250458in}{1.588046in}}%
\pgfpathcurveto{\pgfqpoint{2.258694in}{1.588046in}}{\pgfqpoint{2.266594in}{1.591319in}}{\pgfqpoint{2.272418in}{1.597142in}}%
\pgfpathcurveto{\pgfqpoint{2.278242in}{1.602966in}}{\pgfqpoint{2.281515in}{1.610866in}}{\pgfqpoint{2.281515in}{1.619103in}}%
\pgfpathcurveto{\pgfqpoint{2.281515in}{1.627339in}}{\pgfqpoint{2.278242in}{1.635239in}}{\pgfqpoint{2.272418in}{1.641063in}}%
\pgfpathcurveto{\pgfqpoint{2.266594in}{1.646887in}}{\pgfqpoint{2.258694in}{1.650159in}}{\pgfqpoint{2.250458in}{1.650159in}}%
\pgfpathcurveto{\pgfqpoint{2.242222in}{1.650159in}}{\pgfqpoint{2.234322in}{1.646887in}}{\pgfqpoint{2.228498in}{1.641063in}}%
\pgfpathcurveto{\pgfqpoint{2.222674in}{1.635239in}}{\pgfqpoint{2.219402in}{1.627339in}}{\pgfqpoint{2.219402in}{1.619103in}}%
\pgfpathcurveto{\pgfqpoint{2.219402in}{1.610866in}}{\pgfqpoint{2.222674in}{1.602966in}}{\pgfqpoint{2.228498in}{1.597142in}}%
\pgfpathcurveto{\pgfqpoint{2.234322in}{1.591319in}}{\pgfqpoint{2.242222in}{1.588046in}}{\pgfqpoint{2.250458in}{1.588046in}}%
\pgfpathclose%
\pgfusepath{stroke,fill}%
\end{pgfscope}%
\begin{pgfscope}%
\pgfpathrectangle{\pgfqpoint{0.100000in}{0.212622in}}{\pgfqpoint{3.696000in}{3.696000in}}%
\pgfusepath{clip}%
\pgfsetbuttcap%
\pgfsetroundjoin%
\definecolor{currentfill}{rgb}{0.121569,0.466667,0.705882}%
\pgfsetfillcolor{currentfill}%
\pgfsetfillopacity{0.790464}%
\pgfsetlinewidth{1.003750pt}%
\definecolor{currentstroke}{rgb}{0.121569,0.466667,0.705882}%
\pgfsetstrokecolor{currentstroke}%
\pgfsetstrokeopacity{0.790464}%
\pgfsetdash{}{0pt}%
\pgfpathmoveto{\pgfqpoint{2.251459in}{1.585213in}}%
\pgfpathcurveto{\pgfqpoint{2.259695in}{1.585213in}}{\pgfqpoint{2.267595in}{1.588485in}}{\pgfqpoint{2.273419in}{1.594309in}}%
\pgfpathcurveto{\pgfqpoint{2.279243in}{1.600133in}}{\pgfqpoint{2.282515in}{1.608033in}}{\pgfqpoint{2.282515in}{1.616269in}}%
\pgfpathcurveto{\pgfqpoint{2.282515in}{1.624505in}}{\pgfqpoint{2.279243in}{1.632405in}}{\pgfqpoint{2.273419in}{1.638229in}}%
\pgfpathcurveto{\pgfqpoint{2.267595in}{1.644053in}}{\pgfqpoint{2.259695in}{1.647326in}}{\pgfqpoint{2.251459in}{1.647326in}}%
\pgfpathcurveto{\pgfqpoint{2.243222in}{1.647326in}}{\pgfqpoint{2.235322in}{1.644053in}}{\pgfqpoint{2.229498in}{1.638229in}}%
\pgfpathcurveto{\pgfqpoint{2.223675in}{1.632405in}}{\pgfqpoint{2.220402in}{1.624505in}}{\pgfqpoint{2.220402in}{1.616269in}}%
\pgfpathcurveto{\pgfqpoint{2.220402in}{1.608033in}}{\pgfqpoint{2.223675in}{1.600133in}}{\pgfqpoint{2.229498in}{1.594309in}}%
\pgfpathcurveto{\pgfqpoint{2.235322in}{1.588485in}}{\pgfqpoint{2.243222in}{1.585213in}}{\pgfqpoint{2.251459in}{1.585213in}}%
\pgfpathclose%
\pgfusepath{stroke,fill}%
\end{pgfscope}%
\begin{pgfscope}%
\pgfpathrectangle{\pgfqpoint{0.100000in}{0.212622in}}{\pgfqpoint{3.696000in}{3.696000in}}%
\pgfusepath{clip}%
\pgfsetbuttcap%
\pgfsetroundjoin%
\definecolor{currentfill}{rgb}{0.121569,0.466667,0.705882}%
\pgfsetfillcolor{currentfill}%
\pgfsetfillopacity{0.790992}%
\pgfsetlinewidth{1.003750pt}%
\definecolor{currentstroke}{rgb}{0.121569,0.466667,0.705882}%
\pgfsetstrokecolor{currentstroke}%
\pgfsetstrokeopacity{0.790992}%
\pgfsetdash{}{0pt}%
\pgfpathmoveto{\pgfqpoint{2.252061in}{1.584964in}}%
\pgfpathcurveto{\pgfqpoint{2.260298in}{1.584964in}}{\pgfqpoint{2.268198in}{1.588236in}}{\pgfqpoint{2.274022in}{1.594060in}}%
\pgfpathcurveto{\pgfqpoint{2.279846in}{1.599884in}}{\pgfqpoint{2.283118in}{1.607784in}}{\pgfqpoint{2.283118in}{1.616020in}}%
\pgfpathcurveto{\pgfqpoint{2.283118in}{1.624256in}}{\pgfqpoint{2.279846in}{1.632156in}}{\pgfqpoint{2.274022in}{1.637980in}}%
\pgfpathcurveto{\pgfqpoint{2.268198in}{1.643804in}}{\pgfqpoint{2.260298in}{1.647077in}}{\pgfqpoint{2.252061in}{1.647077in}}%
\pgfpathcurveto{\pgfqpoint{2.243825in}{1.647077in}}{\pgfqpoint{2.235925in}{1.643804in}}{\pgfqpoint{2.230101in}{1.637980in}}%
\pgfpathcurveto{\pgfqpoint{2.224277in}{1.632156in}}{\pgfqpoint{2.221005in}{1.624256in}}{\pgfqpoint{2.221005in}{1.616020in}}%
\pgfpathcurveto{\pgfqpoint{2.221005in}{1.607784in}}{\pgfqpoint{2.224277in}{1.599884in}}{\pgfqpoint{2.230101in}{1.594060in}}%
\pgfpathcurveto{\pgfqpoint{2.235925in}{1.588236in}}{\pgfqpoint{2.243825in}{1.584964in}}{\pgfqpoint{2.252061in}{1.584964in}}%
\pgfpathclose%
\pgfusepath{stroke,fill}%
\end{pgfscope}%
\begin{pgfscope}%
\pgfpathrectangle{\pgfqpoint{0.100000in}{0.212622in}}{\pgfqpoint{3.696000in}{3.696000in}}%
\pgfusepath{clip}%
\pgfsetbuttcap%
\pgfsetroundjoin%
\definecolor{currentfill}{rgb}{0.121569,0.466667,0.705882}%
\pgfsetfillcolor{currentfill}%
\pgfsetfillopacity{0.791790}%
\pgfsetlinewidth{1.003750pt}%
\definecolor{currentstroke}{rgb}{0.121569,0.466667,0.705882}%
\pgfsetstrokecolor{currentstroke}%
\pgfsetstrokeopacity{0.791790}%
\pgfsetdash{}{0pt}%
\pgfpathmoveto{\pgfqpoint{2.252438in}{1.584552in}}%
\pgfpathcurveto{\pgfqpoint{2.260674in}{1.584552in}}{\pgfqpoint{2.268575in}{1.587824in}}{\pgfqpoint{2.274398in}{1.593648in}}%
\pgfpathcurveto{\pgfqpoint{2.280222in}{1.599472in}}{\pgfqpoint{2.283495in}{1.607372in}}{\pgfqpoint{2.283495in}{1.615609in}}%
\pgfpathcurveto{\pgfqpoint{2.283495in}{1.623845in}}{\pgfqpoint{2.280222in}{1.631745in}}{\pgfqpoint{2.274398in}{1.637569in}}%
\pgfpathcurveto{\pgfqpoint{2.268575in}{1.643393in}}{\pgfqpoint{2.260674in}{1.646665in}}{\pgfqpoint{2.252438in}{1.646665in}}%
\pgfpathcurveto{\pgfqpoint{2.244202in}{1.646665in}}{\pgfqpoint{2.236302in}{1.643393in}}{\pgfqpoint{2.230478in}{1.637569in}}%
\pgfpathcurveto{\pgfqpoint{2.224654in}{1.631745in}}{\pgfqpoint{2.221382in}{1.623845in}}{\pgfqpoint{2.221382in}{1.615609in}}%
\pgfpathcurveto{\pgfqpoint{2.221382in}{1.607372in}}{\pgfqpoint{2.224654in}{1.599472in}}{\pgfqpoint{2.230478in}{1.593648in}}%
\pgfpathcurveto{\pgfqpoint{2.236302in}{1.587824in}}{\pgfqpoint{2.244202in}{1.584552in}}{\pgfqpoint{2.252438in}{1.584552in}}%
\pgfpathclose%
\pgfusepath{stroke,fill}%
\end{pgfscope}%
\begin{pgfscope}%
\pgfpathrectangle{\pgfqpoint{0.100000in}{0.212622in}}{\pgfqpoint{3.696000in}{3.696000in}}%
\pgfusepath{clip}%
\pgfsetbuttcap%
\pgfsetroundjoin%
\definecolor{currentfill}{rgb}{0.121569,0.466667,0.705882}%
\pgfsetfillcolor{currentfill}%
\pgfsetfillopacity{0.792933}%
\pgfsetlinewidth{1.003750pt}%
\definecolor{currentstroke}{rgb}{0.121569,0.466667,0.705882}%
\pgfsetstrokecolor{currentstroke}%
\pgfsetstrokeopacity{0.792933}%
\pgfsetdash{}{0pt}%
\pgfpathmoveto{\pgfqpoint{2.253016in}{1.582994in}}%
\pgfpathcurveto{\pgfqpoint{2.261252in}{1.582994in}}{\pgfqpoint{2.269152in}{1.586266in}}{\pgfqpoint{2.274976in}{1.592090in}}%
\pgfpathcurveto{\pgfqpoint{2.280800in}{1.597914in}}{\pgfqpoint{2.284072in}{1.605814in}}{\pgfqpoint{2.284072in}{1.614050in}}%
\pgfpathcurveto{\pgfqpoint{2.284072in}{1.622287in}}{\pgfqpoint{2.280800in}{1.630187in}}{\pgfqpoint{2.274976in}{1.636011in}}%
\pgfpathcurveto{\pgfqpoint{2.269152in}{1.641835in}}{\pgfqpoint{2.261252in}{1.645107in}}{\pgfqpoint{2.253016in}{1.645107in}}%
\pgfpathcurveto{\pgfqpoint{2.244780in}{1.645107in}}{\pgfqpoint{2.236879in}{1.641835in}}{\pgfqpoint{2.231056in}{1.636011in}}%
\pgfpathcurveto{\pgfqpoint{2.225232in}{1.630187in}}{\pgfqpoint{2.221959in}{1.622287in}}{\pgfqpoint{2.221959in}{1.614050in}}%
\pgfpathcurveto{\pgfqpoint{2.221959in}{1.605814in}}{\pgfqpoint{2.225232in}{1.597914in}}{\pgfqpoint{2.231056in}{1.592090in}}%
\pgfpathcurveto{\pgfqpoint{2.236879in}{1.586266in}}{\pgfqpoint{2.244780in}{1.582994in}}{\pgfqpoint{2.253016in}{1.582994in}}%
\pgfpathclose%
\pgfusepath{stroke,fill}%
\end{pgfscope}%
\begin{pgfscope}%
\pgfpathrectangle{\pgfqpoint{0.100000in}{0.212622in}}{\pgfqpoint{3.696000in}{3.696000in}}%
\pgfusepath{clip}%
\pgfsetbuttcap%
\pgfsetroundjoin%
\definecolor{currentfill}{rgb}{0.121569,0.466667,0.705882}%
\pgfsetfillcolor{currentfill}%
\pgfsetfillopacity{0.794606}%
\pgfsetlinewidth{1.003750pt}%
\definecolor{currentstroke}{rgb}{0.121569,0.466667,0.705882}%
\pgfsetstrokecolor{currentstroke}%
\pgfsetstrokeopacity{0.794606}%
\pgfsetdash{}{0pt}%
\pgfpathmoveto{\pgfqpoint{2.254201in}{1.579112in}}%
\pgfpathcurveto{\pgfqpoint{2.262437in}{1.579112in}}{\pgfqpoint{2.270337in}{1.582384in}}{\pgfqpoint{2.276161in}{1.588208in}}%
\pgfpathcurveto{\pgfqpoint{2.281985in}{1.594032in}}{\pgfqpoint{2.285257in}{1.601932in}}{\pgfqpoint{2.285257in}{1.610168in}}%
\pgfpathcurveto{\pgfqpoint{2.285257in}{1.618404in}}{\pgfqpoint{2.281985in}{1.626304in}}{\pgfqpoint{2.276161in}{1.632128in}}%
\pgfpathcurveto{\pgfqpoint{2.270337in}{1.637952in}}{\pgfqpoint{2.262437in}{1.641225in}}{\pgfqpoint{2.254201in}{1.641225in}}%
\pgfpathcurveto{\pgfqpoint{2.245965in}{1.641225in}}{\pgfqpoint{2.238065in}{1.637952in}}{\pgfqpoint{2.232241in}{1.632128in}}%
\pgfpathcurveto{\pgfqpoint{2.226417in}{1.626304in}}{\pgfqpoint{2.223144in}{1.618404in}}{\pgfqpoint{2.223144in}{1.610168in}}%
\pgfpathcurveto{\pgfqpoint{2.223144in}{1.601932in}}{\pgfqpoint{2.226417in}{1.594032in}}{\pgfqpoint{2.232241in}{1.588208in}}%
\pgfpathcurveto{\pgfqpoint{2.238065in}{1.582384in}}{\pgfqpoint{2.245965in}{1.579112in}}{\pgfqpoint{2.254201in}{1.579112in}}%
\pgfpathclose%
\pgfusepath{stroke,fill}%
\end{pgfscope}%
\begin{pgfscope}%
\pgfpathrectangle{\pgfqpoint{0.100000in}{0.212622in}}{\pgfqpoint{3.696000in}{3.696000in}}%
\pgfusepath{clip}%
\pgfsetbuttcap%
\pgfsetroundjoin%
\definecolor{currentfill}{rgb}{0.121569,0.466667,0.705882}%
\pgfsetfillcolor{currentfill}%
\pgfsetfillopacity{0.795800}%
\pgfsetlinewidth{1.003750pt}%
\definecolor{currentstroke}{rgb}{0.121569,0.466667,0.705882}%
\pgfsetstrokecolor{currentstroke}%
\pgfsetstrokeopacity{0.795800}%
\pgfsetdash{}{0pt}%
\pgfpathmoveto{\pgfqpoint{2.255289in}{1.578953in}}%
\pgfpathcurveto{\pgfqpoint{2.263525in}{1.578953in}}{\pgfqpoint{2.271425in}{1.582226in}}{\pgfqpoint{2.277249in}{1.588050in}}%
\pgfpathcurveto{\pgfqpoint{2.283073in}{1.593874in}}{\pgfqpoint{2.286345in}{1.601774in}}{\pgfqpoint{2.286345in}{1.610010in}}%
\pgfpathcurveto{\pgfqpoint{2.286345in}{1.618246in}}{\pgfqpoint{2.283073in}{1.626146in}}{\pgfqpoint{2.277249in}{1.631970in}}%
\pgfpathcurveto{\pgfqpoint{2.271425in}{1.637794in}}{\pgfqpoint{2.263525in}{1.641066in}}{\pgfqpoint{2.255289in}{1.641066in}}%
\pgfpathcurveto{\pgfqpoint{2.247052in}{1.641066in}}{\pgfqpoint{2.239152in}{1.637794in}}{\pgfqpoint{2.233328in}{1.631970in}}%
\pgfpathcurveto{\pgfqpoint{2.227504in}{1.626146in}}{\pgfqpoint{2.224232in}{1.618246in}}{\pgfqpoint{2.224232in}{1.610010in}}%
\pgfpathcurveto{\pgfqpoint{2.224232in}{1.601774in}}{\pgfqpoint{2.227504in}{1.593874in}}{\pgfqpoint{2.233328in}{1.588050in}}%
\pgfpathcurveto{\pgfqpoint{2.239152in}{1.582226in}}{\pgfqpoint{2.247052in}{1.578953in}}{\pgfqpoint{2.255289in}{1.578953in}}%
\pgfpathclose%
\pgfusepath{stroke,fill}%
\end{pgfscope}%
\begin{pgfscope}%
\pgfpathrectangle{\pgfqpoint{0.100000in}{0.212622in}}{\pgfqpoint{3.696000in}{3.696000in}}%
\pgfusepath{clip}%
\pgfsetbuttcap%
\pgfsetroundjoin%
\definecolor{currentfill}{rgb}{0.121569,0.466667,0.705882}%
\pgfsetfillcolor{currentfill}%
\pgfsetfillopacity{0.797302}%
\pgfsetlinewidth{1.003750pt}%
\definecolor{currentstroke}{rgb}{0.121569,0.466667,0.705882}%
\pgfsetstrokecolor{currentstroke}%
\pgfsetstrokeopacity{0.797302}%
\pgfsetdash{}{0pt}%
\pgfpathmoveto{\pgfqpoint{2.256377in}{1.578115in}}%
\pgfpathcurveto{\pgfqpoint{2.264613in}{1.578115in}}{\pgfqpoint{2.272513in}{1.581388in}}{\pgfqpoint{2.278337in}{1.587211in}}%
\pgfpathcurveto{\pgfqpoint{2.284161in}{1.593035in}}{\pgfqpoint{2.287433in}{1.600935in}}{\pgfqpoint{2.287433in}{1.609172in}}%
\pgfpathcurveto{\pgfqpoint{2.287433in}{1.617408in}}{\pgfqpoint{2.284161in}{1.625308in}}{\pgfqpoint{2.278337in}{1.631132in}}%
\pgfpathcurveto{\pgfqpoint{2.272513in}{1.636956in}}{\pgfqpoint{2.264613in}{1.640228in}}{\pgfqpoint{2.256377in}{1.640228in}}%
\pgfpathcurveto{\pgfqpoint{2.248141in}{1.640228in}}{\pgfqpoint{2.240241in}{1.636956in}}{\pgfqpoint{2.234417in}{1.631132in}}%
\pgfpathcurveto{\pgfqpoint{2.228593in}{1.625308in}}{\pgfqpoint{2.225320in}{1.617408in}}{\pgfqpoint{2.225320in}{1.609172in}}%
\pgfpathcurveto{\pgfqpoint{2.225320in}{1.600935in}}{\pgfqpoint{2.228593in}{1.593035in}}{\pgfqpoint{2.234417in}{1.587211in}}%
\pgfpathcurveto{\pgfqpoint{2.240241in}{1.581388in}}{\pgfqpoint{2.248141in}{1.578115in}}{\pgfqpoint{2.256377in}{1.578115in}}%
\pgfpathclose%
\pgfusepath{stroke,fill}%
\end{pgfscope}%
\begin{pgfscope}%
\pgfpathrectangle{\pgfqpoint{0.100000in}{0.212622in}}{\pgfqpoint{3.696000in}{3.696000in}}%
\pgfusepath{clip}%
\pgfsetbuttcap%
\pgfsetroundjoin%
\definecolor{currentfill}{rgb}{0.121569,0.466667,0.705882}%
\pgfsetfillcolor{currentfill}%
\pgfsetfillopacity{0.798740}%
\pgfsetlinewidth{1.003750pt}%
\definecolor{currentstroke}{rgb}{0.121569,0.466667,0.705882}%
\pgfsetstrokecolor{currentstroke}%
\pgfsetstrokeopacity{0.798740}%
\pgfsetdash{}{0pt}%
\pgfpathmoveto{\pgfqpoint{2.257240in}{1.574910in}}%
\pgfpathcurveto{\pgfqpoint{2.265476in}{1.574910in}}{\pgfqpoint{2.273376in}{1.578182in}}{\pgfqpoint{2.279200in}{1.584006in}}%
\pgfpathcurveto{\pgfqpoint{2.285024in}{1.589830in}}{\pgfqpoint{2.288296in}{1.597730in}}{\pgfqpoint{2.288296in}{1.605966in}}%
\pgfpathcurveto{\pgfqpoint{2.288296in}{1.614203in}}{\pgfqpoint{2.285024in}{1.622103in}}{\pgfqpoint{2.279200in}{1.627927in}}%
\pgfpathcurveto{\pgfqpoint{2.273376in}{1.633751in}}{\pgfqpoint{2.265476in}{1.637023in}}{\pgfqpoint{2.257240in}{1.637023in}}%
\pgfpathcurveto{\pgfqpoint{2.249003in}{1.637023in}}{\pgfqpoint{2.241103in}{1.633751in}}{\pgfqpoint{2.235279in}{1.627927in}}%
\pgfpathcurveto{\pgfqpoint{2.229455in}{1.622103in}}{\pgfqpoint{2.226183in}{1.614203in}}{\pgfqpoint{2.226183in}{1.605966in}}%
\pgfpathcurveto{\pgfqpoint{2.226183in}{1.597730in}}{\pgfqpoint{2.229455in}{1.589830in}}{\pgfqpoint{2.235279in}{1.584006in}}%
\pgfpathcurveto{\pgfqpoint{2.241103in}{1.578182in}}{\pgfqpoint{2.249003in}{1.574910in}}{\pgfqpoint{2.257240in}{1.574910in}}%
\pgfpathclose%
\pgfusepath{stroke,fill}%
\end{pgfscope}%
\begin{pgfscope}%
\pgfpathrectangle{\pgfqpoint{0.100000in}{0.212622in}}{\pgfqpoint{3.696000in}{3.696000in}}%
\pgfusepath{clip}%
\pgfsetbuttcap%
\pgfsetroundjoin%
\definecolor{currentfill}{rgb}{0.121569,0.466667,0.705882}%
\pgfsetfillcolor{currentfill}%
\pgfsetfillopacity{0.800235}%
\pgfsetlinewidth{1.003750pt}%
\definecolor{currentstroke}{rgb}{0.121569,0.466667,0.705882}%
\pgfsetstrokecolor{currentstroke}%
\pgfsetstrokeopacity{0.800235}%
\pgfsetdash{}{0pt}%
\pgfpathmoveto{\pgfqpoint{2.259598in}{1.567932in}}%
\pgfpathcurveto{\pgfqpoint{2.267835in}{1.567932in}}{\pgfqpoint{2.275735in}{1.571204in}}{\pgfqpoint{2.281559in}{1.577028in}}%
\pgfpathcurveto{\pgfqpoint{2.287383in}{1.582852in}}{\pgfqpoint{2.290655in}{1.590752in}}{\pgfqpoint{2.290655in}{1.598988in}}%
\pgfpathcurveto{\pgfqpoint{2.290655in}{1.607225in}}{\pgfqpoint{2.287383in}{1.615125in}}{\pgfqpoint{2.281559in}{1.620949in}}%
\pgfpathcurveto{\pgfqpoint{2.275735in}{1.626773in}}{\pgfqpoint{2.267835in}{1.630045in}}{\pgfqpoint{2.259598in}{1.630045in}}%
\pgfpathcurveto{\pgfqpoint{2.251362in}{1.630045in}}{\pgfqpoint{2.243462in}{1.626773in}}{\pgfqpoint{2.237638in}{1.620949in}}%
\pgfpathcurveto{\pgfqpoint{2.231814in}{1.615125in}}{\pgfqpoint{2.228542in}{1.607225in}}{\pgfqpoint{2.228542in}{1.598988in}}%
\pgfpathcurveto{\pgfqpoint{2.228542in}{1.590752in}}{\pgfqpoint{2.231814in}{1.582852in}}{\pgfqpoint{2.237638in}{1.577028in}}%
\pgfpathcurveto{\pgfqpoint{2.243462in}{1.571204in}}{\pgfqpoint{2.251362in}{1.567932in}}{\pgfqpoint{2.259598in}{1.567932in}}%
\pgfpathclose%
\pgfusepath{stroke,fill}%
\end{pgfscope}%
\begin{pgfscope}%
\pgfpathrectangle{\pgfqpoint{0.100000in}{0.212622in}}{\pgfqpoint{3.696000in}{3.696000in}}%
\pgfusepath{clip}%
\pgfsetbuttcap%
\pgfsetroundjoin%
\definecolor{currentfill}{rgb}{0.121569,0.466667,0.705882}%
\pgfsetfillcolor{currentfill}%
\pgfsetfillopacity{0.802436}%
\pgfsetlinewidth{1.003750pt}%
\definecolor{currentstroke}{rgb}{0.121569,0.466667,0.705882}%
\pgfsetstrokecolor{currentstroke}%
\pgfsetstrokeopacity{0.802436}%
\pgfsetdash{}{0pt}%
\pgfpathmoveto{\pgfqpoint{2.261588in}{1.562700in}}%
\pgfpathcurveto{\pgfqpoint{2.269825in}{1.562700in}}{\pgfqpoint{2.277725in}{1.565972in}}{\pgfqpoint{2.283549in}{1.571796in}}%
\pgfpathcurveto{\pgfqpoint{2.289372in}{1.577620in}}{\pgfqpoint{2.292645in}{1.585520in}}{\pgfqpoint{2.292645in}{1.593756in}}%
\pgfpathcurveto{\pgfqpoint{2.292645in}{1.601993in}}{\pgfqpoint{2.289372in}{1.609893in}}{\pgfqpoint{2.283549in}{1.615717in}}%
\pgfpathcurveto{\pgfqpoint{2.277725in}{1.621540in}}{\pgfqpoint{2.269825in}{1.624813in}}{\pgfqpoint{2.261588in}{1.624813in}}%
\pgfpathcurveto{\pgfqpoint{2.253352in}{1.624813in}}{\pgfqpoint{2.245452in}{1.621540in}}{\pgfqpoint{2.239628in}{1.615717in}}%
\pgfpathcurveto{\pgfqpoint{2.233804in}{1.609893in}}{\pgfqpoint{2.230532in}{1.601993in}}{\pgfqpoint{2.230532in}{1.593756in}}%
\pgfpathcurveto{\pgfqpoint{2.230532in}{1.585520in}}{\pgfqpoint{2.233804in}{1.577620in}}{\pgfqpoint{2.239628in}{1.571796in}}%
\pgfpathcurveto{\pgfqpoint{2.245452in}{1.565972in}}{\pgfqpoint{2.253352in}{1.562700in}}{\pgfqpoint{2.261588in}{1.562700in}}%
\pgfpathclose%
\pgfusepath{stroke,fill}%
\end{pgfscope}%
\begin{pgfscope}%
\pgfpathrectangle{\pgfqpoint{0.100000in}{0.212622in}}{\pgfqpoint{3.696000in}{3.696000in}}%
\pgfusepath{clip}%
\pgfsetbuttcap%
\pgfsetroundjoin%
\definecolor{currentfill}{rgb}{0.121569,0.466667,0.705882}%
\pgfsetfillcolor{currentfill}%
\pgfsetfillopacity{0.805785}%
\pgfsetlinewidth{1.003750pt}%
\definecolor{currentstroke}{rgb}{0.121569,0.466667,0.705882}%
\pgfsetstrokecolor{currentstroke}%
\pgfsetstrokeopacity{0.805785}%
\pgfsetdash{}{0pt}%
\pgfpathmoveto{\pgfqpoint{2.265258in}{1.564360in}}%
\pgfpathcurveto{\pgfqpoint{2.273494in}{1.564360in}}{\pgfqpoint{2.281394in}{1.567632in}}{\pgfqpoint{2.287218in}{1.573456in}}%
\pgfpathcurveto{\pgfqpoint{2.293042in}{1.579280in}}{\pgfqpoint{2.296314in}{1.587180in}}{\pgfqpoint{2.296314in}{1.595416in}}%
\pgfpathcurveto{\pgfqpoint{2.296314in}{1.603652in}}{\pgfqpoint{2.293042in}{1.611552in}}{\pgfqpoint{2.287218in}{1.617376in}}%
\pgfpathcurveto{\pgfqpoint{2.281394in}{1.623200in}}{\pgfqpoint{2.273494in}{1.626473in}}{\pgfqpoint{2.265258in}{1.626473in}}%
\pgfpathcurveto{\pgfqpoint{2.257022in}{1.626473in}}{\pgfqpoint{2.249122in}{1.623200in}}{\pgfqpoint{2.243298in}{1.617376in}}%
\pgfpathcurveto{\pgfqpoint{2.237474in}{1.611552in}}{\pgfqpoint{2.234201in}{1.603652in}}{\pgfqpoint{2.234201in}{1.595416in}}%
\pgfpathcurveto{\pgfqpoint{2.234201in}{1.587180in}}{\pgfqpoint{2.237474in}{1.579280in}}{\pgfqpoint{2.243298in}{1.573456in}}%
\pgfpathcurveto{\pgfqpoint{2.249122in}{1.567632in}}{\pgfqpoint{2.257022in}{1.564360in}}{\pgfqpoint{2.265258in}{1.564360in}}%
\pgfpathclose%
\pgfusepath{stroke,fill}%
\end{pgfscope}%
\begin{pgfscope}%
\pgfpathrectangle{\pgfqpoint{0.100000in}{0.212622in}}{\pgfqpoint{3.696000in}{3.696000in}}%
\pgfusepath{clip}%
\pgfsetbuttcap%
\pgfsetroundjoin%
\definecolor{currentfill}{rgb}{0.121569,0.466667,0.705882}%
\pgfsetfillcolor{currentfill}%
\pgfsetfillopacity{0.807180}%
\pgfsetlinewidth{1.003750pt}%
\definecolor{currentstroke}{rgb}{0.121569,0.466667,0.705882}%
\pgfsetstrokecolor{currentstroke}%
\pgfsetstrokeopacity{0.807180}%
\pgfsetdash{}{0pt}%
\pgfpathmoveto{\pgfqpoint{2.265793in}{1.561545in}}%
\pgfpathcurveto{\pgfqpoint{2.274030in}{1.561545in}}{\pgfqpoint{2.281930in}{1.564817in}}{\pgfqpoint{2.287753in}{1.570641in}}%
\pgfpathcurveto{\pgfqpoint{2.293577in}{1.576465in}}{\pgfqpoint{2.296850in}{1.584365in}}{\pgfqpoint{2.296850in}{1.592601in}}%
\pgfpathcurveto{\pgfqpoint{2.296850in}{1.600837in}}{\pgfqpoint{2.293577in}{1.608737in}}{\pgfqpoint{2.287753in}{1.614561in}}%
\pgfpathcurveto{\pgfqpoint{2.281930in}{1.620385in}}{\pgfqpoint{2.274030in}{1.623658in}}{\pgfqpoint{2.265793in}{1.623658in}}%
\pgfpathcurveto{\pgfqpoint{2.257557in}{1.623658in}}{\pgfqpoint{2.249657in}{1.620385in}}{\pgfqpoint{2.243833in}{1.614561in}}%
\pgfpathcurveto{\pgfqpoint{2.238009in}{1.608737in}}{\pgfqpoint{2.234737in}{1.600837in}}{\pgfqpoint{2.234737in}{1.592601in}}%
\pgfpathcurveto{\pgfqpoint{2.234737in}{1.584365in}}{\pgfqpoint{2.238009in}{1.576465in}}{\pgfqpoint{2.243833in}{1.570641in}}%
\pgfpathcurveto{\pgfqpoint{2.249657in}{1.564817in}}{\pgfqpoint{2.257557in}{1.561545in}}{\pgfqpoint{2.265793in}{1.561545in}}%
\pgfpathclose%
\pgfusepath{stroke,fill}%
\end{pgfscope}%
\begin{pgfscope}%
\pgfpathrectangle{\pgfqpoint{0.100000in}{0.212622in}}{\pgfqpoint{3.696000in}{3.696000in}}%
\pgfusepath{clip}%
\pgfsetbuttcap%
\pgfsetroundjoin%
\definecolor{currentfill}{rgb}{0.121569,0.466667,0.705882}%
\pgfsetfillcolor{currentfill}%
\pgfsetfillopacity{0.808917}%
\pgfsetlinewidth{1.003750pt}%
\definecolor{currentstroke}{rgb}{0.121569,0.466667,0.705882}%
\pgfsetstrokecolor{currentstroke}%
\pgfsetstrokeopacity{0.808917}%
\pgfsetdash{}{0pt}%
\pgfpathmoveto{\pgfqpoint{2.268032in}{1.557311in}}%
\pgfpathcurveto{\pgfqpoint{2.276268in}{1.557311in}}{\pgfqpoint{2.284168in}{1.560583in}}{\pgfqpoint{2.289992in}{1.566407in}}%
\pgfpathcurveto{\pgfqpoint{2.295816in}{1.572231in}}{\pgfqpoint{2.299088in}{1.580131in}}{\pgfqpoint{2.299088in}{1.588368in}}%
\pgfpathcurveto{\pgfqpoint{2.299088in}{1.596604in}}{\pgfqpoint{2.295816in}{1.604504in}}{\pgfqpoint{2.289992in}{1.610328in}}%
\pgfpathcurveto{\pgfqpoint{2.284168in}{1.616152in}}{\pgfqpoint{2.276268in}{1.619424in}}{\pgfqpoint{2.268032in}{1.619424in}}%
\pgfpathcurveto{\pgfqpoint{2.259795in}{1.619424in}}{\pgfqpoint{2.251895in}{1.616152in}}{\pgfqpoint{2.246072in}{1.610328in}}%
\pgfpathcurveto{\pgfqpoint{2.240248in}{1.604504in}}{\pgfqpoint{2.236975in}{1.596604in}}{\pgfqpoint{2.236975in}{1.588368in}}%
\pgfpathcurveto{\pgfqpoint{2.236975in}{1.580131in}}{\pgfqpoint{2.240248in}{1.572231in}}{\pgfqpoint{2.246072in}{1.566407in}}%
\pgfpathcurveto{\pgfqpoint{2.251895in}{1.560583in}}{\pgfqpoint{2.259795in}{1.557311in}}{\pgfqpoint{2.268032in}{1.557311in}}%
\pgfpathclose%
\pgfusepath{stroke,fill}%
\end{pgfscope}%
\begin{pgfscope}%
\pgfpathrectangle{\pgfqpoint{0.100000in}{0.212622in}}{\pgfqpoint{3.696000in}{3.696000in}}%
\pgfusepath{clip}%
\pgfsetbuttcap%
\pgfsetroundjoin%
\definecolor{currentfill}{rgb}{0.121569,0.466667,0.705882}%
\pgfsetfillcolor{currentfill}%
\pgfsetfillopacity{0.810847}%
\pgfsetlinewidth{1.003750pt}%
\definecolor{currentstroke}{rgb}{0.121569,0.466667,0.705882}%
\pgfsetstrokecolor{currentstroke}%
\pgfsetstrokeopacity{0.810847}%
\pgfsetdash{}{0pt}%
\pgfpathmoveto{\pgfqpoint{2.269340in}{1.552191in}}%
\pgfpathcurveto{\pgfqpoint{2.277577in}{1.552191in}}{\pgfqpoint{2.285477in}{1.555464in}}{\pgfqpoint{2.291301in}{1.561288in}}%
\pgfpathcurveto{\pgfqpoint{2.297125in}{1.567111in}}{\pgfqpoint{2.300397in}{1.575012in}}{\pgfqpoint{2.300397in}{1.583248in}}%
\pgfpathcurveto{\pgfqpoint{2.300397in}{1.591484in}}{\pgfqpoint{2.297125in}{1.599384in}}{\pgfqpoint{2.291301in}{1.605208in}}%
\pgfpathcurveto{\pgfqpoint{2.285477in}{1.611032in}}{\pgfqpoint{2.277577in}{1.614304in}}{\pgfqpoint{2.269340in}{1.614304in}}%
\pgfpathcurveto{\pgfqpoint{2.261104in}{1.614304in}}{\pgfqpoint{2.253204in}{1.611032in}}{\pgfqpoint{2.247380in}{1.605208in}}%
\pgfpathcurveto{\pgfqpoint{2.241556in}{1.599384in}}{\pgfqpoint{2.238284in}{1.591484in}}{\pgfqpoint{2.238284in}{1.583248in}}%
\pgfpathcurveto{\pgfqpoint{2.238284in}{1.575012in}}{\pgfqpoint{2.241556in}{1.567111in}}{\pgfqpoint{2.247380in}{1.561288in}}%
\pgfpathcurveto{\pgfqpoint{2.253204in}{1.555464in}}{\pgfqpoint{2.261104in}{1.552191in}}{\pgfqpoint{2.269340in}{1.552191in}}%
\pgfpathclose%
\pgfusepath{stroke,fill}%
\end{pgfscope}%
\begin{pgfscope}%
\pgfpathrectangle{\pgfqpoint{0.100000in}{0.212622in}}{\pgfqpoint{3.696000in}{3.696000in}}%
\pgfusepath{clip}%
\pgfsetbuttcap%
\pgfsetroundjoin%
\definecolor{currentfill}{rgb}{0.121569,0.466667,0.705882}%
\pgfsetfillcolor{currentfill}%
\pgfsetfillopacity{0.812310}%
\pgfsetlinewidth{1.003750pt}%
\definecolor{currentstroke}{rgb}{0.121569,0.466667,0.705882}%
\pgfsetstrokecolor{currentstroke}%
\pgfsetstrokeopacity{0.812310}%
\pgfsetdash{}{0pt}%
\pgfpathmoveto{\pgfqpoint{2.270736in}{1.552298in}}%
\pgfpathcurveto{\pgfqpoint{2.278972in}{1.552298in}}{\pgfqpoint{2.286872in}{1.555571in}}{\pgfqpoint{2.292696in}{1.561395in}}%
\pgfpathcurveto{\pgfqpoint{2.298520in}{1.567218in}}{\pgfqpoint{2.301792in}{1.575119in}}{\pgfqpoint{2.301792in}{1.583355in}}%
\pgfpathcurveto{\pgfqpoint{2.301792in}{1.591591in}}{\pgfqpoint{2.298520in}{1.599491in}}{\pgfqpoint{2.292696in}{1.605315in}}%
\pgfpathcurveto{\pgfqpoint{2.286872in}{1.611139in}}{\pgfqpoint{2.278972in}{1.614411in}}{\pgfqpoint{2.270736in}{1.614411in}}%
\pgfpathcurveto{\pgfqpoint{2.262499in}{1.614411in}}{\pgfqpoint{2.254599in}{1.611139in}}{\pgfqpoint{2.248775in}{1.605315in}}%
\pgfpathcurveto{\pgfqpoint{2.242951in}{1.599491in}}{\pgfqpoint{2.239679in}{1.591591in}}{\pgfqpoint{2.239679in}{1.583355in}}%
\pgfpathcurveto{\pgfqpoint{2.239679in}{1.575119in}}{\pgfqpoint{2.242951in}{1.567218in}}{\pgfqpoint{2.248775in}{1.561395in}}%
\pgfpathcurveto{\pgfqpoint{2.254599in}{1.555571in}}{\pgfqpoint{2.262499in}{1.552298in}}{\pgfqpoint{2.270736in}{1.552298in}}%
\pgfpathclose%
\pgfusepath{stroke,fill}%
\end{pgfscope}%
\begin{pgfscope}%
\pgfpathrectangle{\pgfqpoint{0.100000in}{0.212622in}}{\pgfqpoint{3.696000in}{3.696000in}}%
\pgfusepath{clip}%
\pgfsetbuttcap%
\pgfsetroundjoin%
\definecolor{currentfill}{rgb}{0.121569,0.466667,0.705882}%
\pgfsetfillcolor{currentfill}%
\pgfsetfillopacity{0.813067}%
\pgfsetlinewidth{1.003750pt}%
\definecolor{currentstroke}{rgb}{0.121569,0.466667,0.705882}%
\pgfsetstrokecolor{currentstroke}%
\pgfsetstrokeopacity{0.813067}%
\pgfsetdash{}{0pt}%
\pgfpathmoveto{\pgfqpoint{2.271106in}{1.551813in}}%
\pgfpathcurveto{\pgfqpoint{2.279343in}{1.551813in}}{\pgfqpoint{2.287243in}{1.555085in}}{\pgfqpoint{2.293067in}{1.560909in}}%
\pgfpathcurveto{\pgfqpoint{2.298890in}{1.566733in}}{\pgfqpoint{2.302163in}{1.574633in}}{\pgfqpoint{2.302163in}{1.582869in}}%
\pgfpathcurveto{\pgfqpoint{2.302163in}{1.591106in}}{\pgfqpoint{2.298890in}{1.599006in}}{\pgfqpoint{2.293067in}{1.604830in}}%
\pgfpathcurveto{\pgfqpoint{2.287243in}{1.610654in}}{\pgfqpoint{2.279343in}{1.613926in}}{\pgfqpoint{2.271106in}{1.613926in}}%
\pgfpathcurveto{\pgfqpoint{2.262870in}{1.613926in}}{\pgfqpoint{2.254970in}{1.610654in}}{\pgfqpoint{2.249146in}{1.604830in}}%
\pgfpathcurveto{\pgfqpoint{2.243322in}{1.599006in}}{\pgfqpoint{2.240050in}{1.591106in}}{\pgfqpoint{2.240050in}{1.582869in}}%
\pgfpathcurveto{\pgfqpoint{2.240050in}{1.574633in}}{\pgfqpoint{2.243322in}{1.566733in}}{\pgfqpoint{2.249146in}{1.560909in}}%
\pgfpathcurveto{\pgfqpoint{2.254970in}{1.555085in}}{\pgfqpoint{2.262870in}{1.551813in}}{\pgfqpoint{2.271106in}{1.551813in}}%
\pgfpathclose%
\pgfusepath{stroke,fill}%
\end{pgfscope}%
\begin{pgfscope}%
\pgfpathrectangle{\pgfqpoint{0.100000in}{0.212622in}}{\pgfqpoint{3.696000in}{3.696000in}}%
\pgfusepath{clip}%
\pgfsetbuttcap%
\pgfsetroundjoin%
\definecolor{currentfill}{rgb}{0.121569,0.466667,0.705882}%
\pgfsetfillcolor{currentfill}%
\pgfsetfillopacity{0.814058}%
\pgfsetlinewidth{1.003750pt}%
\definecolor{currentstroke}{rgb}{0.121569,0.466667,0.705882}%
\pgfsetstrokecolor{currentstroke}%
\pgfsetstrokeopacity{0.814058}%
\pgfsetdash{}{0pt}%
\pgfpathmoveto{\pgfqpoint{2.271930in}{1.550183in}}%
\pgfpathcurveto{\pgfqpoint{2.280167in}{1.550183in}}{\pgfqpoint{2.288067in}{1.553455in}}{\pgfqpoint{2.293891in}{1.559279in}}%
\pgfpathcurveto{\pgfqpoint{2.299715in}{1.565103in}}{\pgfqpoint{2.302987in}{1.573003in}}{\pgfqpoint{2.302987in}{1.581240in}}%
\pgfpathcurveto{\pgfqpoint{2.302987in}{1.589476in}}{\pgfqpoint{2.299715in}{1.597376in}}{\pgfqpoint{2.293891in}{1.603200in}}%
\pgfpathcurveto{\pgfqpoint{2.288067in}{1.609024in}}{\pgfqpoint{2.280167in}{1.612296in}}{\pgfqpoint{2.271930in}{1.612296in}}%
\pgfpathcurveto{\pgfqpoint{2.263694in}{1.612296in}}{\pgfqpoint{2.255794in}{1.609024in}}{\pgfqpoint{2.249970in}{1.603200in}}%
\pgfpathcurveto{\pgfqpoint{2.244146in}{1.597376in}}{\pgfqpoint{2.240874in}{1.589476in}}{\pgfqpoint{2.240874in}{1.581240in}}%
\pgfpathcurveto{\pgfqpoint{2.240874in}{1.573003in}}{\pgfqpoint{2.244146in}{1.565103in}}{\pgfqpoint{2.249970in}{1.559279in}}%
\pgfpathcurveto{\pgfqpoint{2.255794in}{1.553455in}}{\pgfqpoint{2.263694in}{1.550183in}}{\pgfqpoint{2.271930in}{1.550183in}}%
\pgfpathclose%
\pgfusepath{stroke,fill}%
\end{pgfscope}%
\begin{pgfscope}%
\pgfpathrectangle{\pgfqpoint{0.100000in}{0.212622in}}{\pgfqpoint{3.696000in}{3.696000in}}%
\pgfusepath{clip}%
\pgfsetbuttcap%
\pgfsetroundjoin%
\definecolor{currentfill}{rgb}{0.121569,0.466667,0.705882}%
\pgfsetfillcolor{currentfill}%
\pgfsetfillopacity{0.815050}%
\pgfsetlinewidth{1.003750pt}%
\definecolor{currentstroke}{rgb}{0.121569,0.466667,0.705882}%
\pgfsetstrokecolor{currentstroke}%
\pgfsetstrokeopacity{0.815050}%
\pgfsetdash{}{0pt}%
\pgfpathmoveto{\pgfqpoint{2.272762in}{1.547264in}}%
\pgfpathcurveto{\pgfqpoint{2.280998in}{1.547264in}}{\pgfqpoint{2.288899in}{1.550537in}}{\pgfqpoint{2.294722in}{1.556360in}}%
\pgfpathcurveto{\pgfqpoint{2.300546in}{1.562184in}}{\pgfqpoint{2.303819in}{1.570084in}}{\pgfqpoint{2.303819in}{1.578321in}}%
\pgfpathcurveto{\pgfqpoint{2.303819in}{1.586557in}}{\pgfqpoint{2.300546in}{1.594457in}}{\pgfqpoint{2.294722in}{1.600281in}}%
\pgfpathcurveto{\pgfqpoint{2.288899in}{1.606105in}}{\pgfqpoint{2.280998in}{1.609377in}}{\pgfqpoint{2.272762in}{1.609377in}}%
\pgfpathcurveto{\pgfqpoint{2.264526in}{1.609377in}}{\pgfqpoint{2.256626in}{1.606105in}}{\pgfqpoint{2.250802in}{1.600281in}}%
\pgfpathcurveto{\pgfqpoint{2.244978in}{1.594457in}}{\pgfqpoint{2.241706in}{1.586557in}}{\pgfqpoint{2.241706in}{1.578321in}}%
\pgfpathcurveto{\pgfqpoint{2.241706in}{1.570084in}}{\pgfqpoint{2.244978in}{1.562184in}}{\pgfqpoint{2.250802in}{1.556360in}}%
\pgfpathcurveto{\pgfqpoint{2.256626in}{1.550537in}}{\pgfqpoint{2.264526in}{1.547264in}}{\pgfqpoint{2.272762in}{1.547264in}}%
\pgfpathclose%
\pgfusepath{stroke,fill}%
\end{pgfscope}%
\begin{pgfscope}%
\pgfpathrectangle{\pgfqpoint{0.100000in}{0.212622in}}{\pgfqpoint{3.696000in}{3.696000in}}%
\pgfusepath{clip}%
\pgfsetbuttcap%
\pgfsetroundjoin%
\definecolor{currentfill}{rgb}{0.121569,0.466667,0.705882}%
\pgfsetfillcolor{currentfill}%
\pgfsetfillopacity{0.816513}%
\pgfsetlinewidth{1.003750pt}%
\definecolor{currentstroke}{rgb}{0.121569,0.466667,0.705882}%
\pgfsetstrokecolor{currentstroke}%
\pgfsetstrokeopacity{0.816513}%
\pgfsetdash{}{0pt}%
\pgfpathmoveto{\pgfqpoint{2.274403in}{1.546157in}}%
\pgfpathcurveto{\pgfqpoint{2.282639in}{1.546157in}}{\pgfqpoint{2.290539in}{1.549429in}}{\pgfqpoint{2.296363in}{1.555253in}}%
\pgfpathcurveto{\pgfqpoint{2.302187in}{1.561077in}}{\pgfqpoint{2.305459in}{1.568977in}}{\pgfqpoint{2.305459in}{1.577213in}}%
\pgfpathcurveto{\pgfqpoint{2.305459in}{1.585450in}}{\pgfqpoint{2.302187in}{1.593350in}}{\pgfqpoint{2.296363in}{1.599174in}}%
\pgfpathcurveto{\pgfqpoint{2.290539in}{1.604998in}}{\pgfqpoint{2.282639in}{1.608270in}}{\pgfqpoint{2.274403in}{1.608270in}}%
\pgfpathcurveto{\pgfqpoint{2.266167in}{1.608270in}}{\pgfqpoint{2.258267in}{1.604998in}}{\pgfqpoint{2.252443in}{1.599174in}}%
\pgfpathcurveto{\pgfqpoint{2.246619in}{1.593350in}}{\pgfqpoint{2.243346in}{1.585450in}}{\pgfqpoint{2.243346in}{1.577213in}}%
\pgfpathcurveto{\pgfqpoint{2.243346in}{1.568977in}}{\pgfqpoint{2.246619in}{1.561077in}}{\pgfqpoint{2.252443in}{1.555253in}}%
\pgfpathcurveto{\pgfqpoint{2.258267in}{1.549429in}}{\pgfqpoint{2.266167in}{1.546157in}}{\pgfqpoint{2.274403in}{1.546157in}}%
\pgfpathclose%
\pgfusepath{stroke,fill}%
\end{pgfscope}%
\begin{pgfscope}%
\pgfpathrectangle{\pgfqpoint{0.100000in}{0.212622in}}{\pgfqpoint{3.696000in}{3.696000in}}%
\pgfusepath{clip}%
\pgfsetbuttcap%
\pgfsetroundjoin%
\definecolor{currentfill}{rgb}{0.121569,0.466667,0.705882}%
\pgfsetfillcolor{currentfill}%
\pgfsetfillopacity{0.817365}%
\pgfsetlinewidth{1.003750pt}%
\definecolor{currentstroke}{rgb}{0.121569,0.466667,0.705882}%
\pgfsetstrokecolor{currentstroke}%
\pgfsetstrokeopacity{0.817365}%
\pgfsetdash{}{0pt}%
\pgfpathmoveto{\pgfqpoint{2.274976in}{1.545608in}}%
\pgfpathcurveto{\pgfqpoint{2.283212in}{1.545608in}}{\pgfqpoint{2.291112in}{1.548881in}}{\pgfqpoint{2.296936in}{1.554705in}}%
\pgfpathcurveto{\pgfqpoint{2.302760in}{1.560529in}}{\pgfqpoint{2.306032in}{1.568429in}}{\pgfqpoint{2.306032in}{1.576665in}}%
\pgfpathcurveto{\pgfqpoint{2.306032in}{1.584901in}}{\pgfqpoint{2.302760in}{1.592801in}}{\pgfqpoint{2.296936in}{1.598625in}}%
\pgfpathcurveto{\pgfqpoint{2.291112in}{1.604449in}}{\pgfqpoint{2.283212in}{1.607721in}}{\pgfqpoint{2.274976in}{1.607721in}}%
\pgfpathcurveto{\pgfqpoint{2.266740in}{1.607721in}}{\pgfqpoint{2.258839in}{1.604449in}}{\pgfqpoint{2.253016in}{1.598625in}}%
\pgfpathcurveto{\pgfqpoint{2.247192in}{1.592801in}}{\pgfqpoint{2.243919in}{1.584901in}}{\pgfqpoint{2.243919in}{1.576665in}}%
\pgfpathcurveto{\pgfqpoint{2.243919in}{1.568429in}}{\pgfqpoint{2.247192in}{1.560529in}}{\pgfqpoint{2.253016in}{1.554705in}}%
\pgfpathcurveto{\pgfqpoint{2.258839in}{1.548881in}}{\pgfqpoint{2.266740in}{1.545608in}}{\pgfqpoint{2.274976in}{1.545608in}}%
\pgfpathclose%
\pgfusepath{stroke,fill}%
\end{pgfscope}%
\begin{pgfscope}%
\pgfpathrectangle{\pgfqpoint{0.100000in}{0.212622in}}{\pgfqpoint{3.696000in}{3.696000in}}%
\pgfusepath{clip}%
\pgfsetbuttcap%
\pgfsetroundjoin%
\definecolor{currentfill}{rgb}{0.121569,0.466667,0.705882}%
\pgfsetfillcolor{currentfill}%
\pgfsetfillopacity{0.818432}%
\pgfsetlinewidth{1.003750pt}%
\definecolor{currentstroke}{rgb}{0.121569,0.466667,0.705882}%
\pgfsetstrokecolor{currentstroke}%
\pgfsetstrokeopacity{0.818432}%
\pgfsetdash{}{0pt}%
\pgfpathmoveto{\pgfqpoint{2.275368in}{1.544033in}}%
\pgfpathcurveto{\pgfqpoint{2.283604in}{1.544033in}}{\pgfqpoint{2.291504in}{1.547306in}}{\pgfqpoint{2.297328in}{1.553130in}}%
\pgfpathcurveto{\pgfqpoint{2.303152in}{1.558954in}}{\pgfqpoint{2.306425in}{1.566854in}}{\pgfqpoint{2.306425in}{1.575090in}}%
\pgfpathcurveto{\pgfqpoint{2.306425in}{1.583326in}}{\pgfqpoint{2.303152in}{1.591226in}}{\pgfqpoint{2.297328in}{1.597050in}}%
\pgfpathcurveto{\pgfqpoint{2.291504in}{1.602874in}}{\pgfqpoint{2.283604in}{1.606146in}}{\pgfqpoint{2.275368in}{1.606146in}}%
\pgfpathcurveto{\pgfqpoint{2.267132in}{1.606146in}}{\pgfqpoint{2.259232in}{1.602874in}}{\pgfqpoint{2.253408in}{1.597050in}}%
\pgfpathcurveto{\pgfqpoint{2.247584in}{1.591226in}}{\pgfqpoint{2.244312in}{1.583326in}}{\pgfqpoint{2.244312in}{1.575090in}}%
\pgfpathcurveto{\pgfqpoint{2.244312in}{1.566854in}}{\pgfqpoint{2.247584in}{1.558954in}}{\pgfqpoint{2.253408in}{1.553130in}}%
\pgfpathcurveto{\pgfqpoint{2.259232in}{1.547306in}}{\pgfqpoint{2.267132in}{1.544033in}}{\pgfqpoint{2.275368in}{1.544033in}}%
\pgfpathclose%
\pgfusepath{stroke,fill}%
\end{pgfscope}%
\begin{pgfscope}%
\pgfpathrectangle{\pgfqpoint{0.100000in}{0.212622in}}{\pgfqpoint{3.696000in}{3.696000in}}%
\pgfusepath{clip}%
\pgfsetbuttcap%
\pgfsetroundjoin%
\definecolor{currentfill}{rgb}{0.121569,0.466667,0.705882}%
\pgfsetfillcolor{currentfill}%
\pgfsetfillopacity{0.819694}%
\pgfsetlinewidth{1.003750pt}%
\definecolor{currentstroke}{rgb}{0.121569,0.466667,0.705882}%
\pgfsetstrokecolor{currentstroke}%
\pgfsetstrokeopacity{0.819694}%
\pgfsetdash{}{0pt}%
\pgfpathmoveto{\pgfqpoint{2.276454in}{1.538370in}}%
\pgfpathcurveto{\pgfqpoint{2.284691in}{1.538370in}}{\pgfqpoint{2.292591in}{1.541642in}}{\pgfqpoint{2.298415in}{1.547466in}}%
\pgfpathcurveto{\pgfqpoint{2.304239in}{1.553290in}}{\pgfqpoint{2.307511in}{1.561190in}}{\pgfqpoint{2.307511in}{1.569427in}}%
\pgfpathcurveto{\pgfqpoint{2.307511in}{1.577663in}}{\pgfqpoint{2.304239in}{1.585563in}}{\pgfqpoint{2.298415in}{1.591387in}}%
\pgfpathcurveto{\pgfqpoint{2.292591in}{1.597211in}}{\pgfqpoint{2.284691in}{1.600483in}}{\pgfqpoint{2.276454in}{1.600483in}}%
\pgfpathcurveto{\pgfqpoint{2.268218in}{1.600483in}}{\pgfqpoint{2.260318in}{1.597211in}}{\pgfqpoint{2.254494in}{1.591387in}}%
\pgfpathcurveto{\pgfqpoint{2.248670in}{1.585563in}}{\pgfqpoint{2.245398in}{1.577663in}}{\pgfqpoint{2.245398in}{1.569427in}}%
\pgfpathcurveto{\pgfqpoint{2.245398in}{1.561190in}}{\pgfqpoint{2.248670in}{1.553290in}}{\pgfqpoint{2.254494in}{1.547466in}}%
\pgfpathcurveto{\pgfqpoint{2.260318in}{1.541642in}}{\pgfqpoint{2.268218in}{1.538370in}}{\pgfqpoint{2.276454in}{1.538370in}}%
\pgfpathclose%
\pgfusepath{stroke,fill}%
\end{pgfscope}%
\begin{pgfscope}%
\pgfpathrectangle{\pgfqpoint{0.100000in}{0.212622in}}{\pgfqpoint{3.696000in}{3.696000in}}%
\pgfusepath{clip}%
\pgfsetbuttcap%
\pgfsetroundjoin%
\definecolor{currentfill}{rgb}{0.121569,0.466667,0.705882}%
\pgfsetfillcolor{currentfill}%
\pgfsetfillopacity{0.820705}%
\pgfsetlinewidth{1.003750pt}%
\definecolor{currentstroke}{rgb}{0.121569,0.466667,0.705882}%
\pgfsetstrokecolor{currentstroke}%
\pgfsetstrokeopacity{0.820705}%
\pgfsetdash{}{0pt}%
\pgfpathmoveto{\pgfqpoint{2.277374in}{1.537402in}}%
\pgfpathcurveto{\pgfqpoint{2.285610in}{1.537402in}}{\pgfqpoint{2.293510in}{1.540674in}}{\pgfqpoint{2.299334in}{1.546498in}}%
\pgfpathcurveto{\pgfqpoint{2.305158in}{1.552322in}}{\pgfqpoint{2.308431in}{1.560222in}}{\pgfqpoint{2.308431in}{1.568458in}}%
\pgfpathcurveto{\pgfqpoint{2.308431in}{1.576694in}}{\pgfqpoint{2.305158in}{1.584595in}}{\pgfqpoint{2.299334in}{1.590418in}}%
\pgfpathcurveto{\pgfqpoint{2.293510in}{1.596242in}}{\pgfqpoint{2.285610in}{1.599515in}}{\pgfqpoint{2.277374in}{1.599515in}}%
\pgfpathcurveto{\pgfqpoint{2.269138in}{1.599515in}}{\pgfqpoint{2.261238in}{1.596242in}}{\pgfqpoint{2.255414in}{1.590418in}}%
\pgfpathcurveto{\pgfqpoint{2.249590in}{1.584595in}}{\pgfqpoint{2.246318in}{1.576694in}}{\pgfqpoint{2.246318in}{1.568458in}}%
\pgfpathcurveto{\pgfqpoint{2.246318in}{1.560222in}}{\pgfqpoint{2.249590in}{1.552322in}}{\pgfqpoint{2.255414in}{1.546498in}}%
\pgfpathcurveto{\pgfqpoint{2.261238in}{1.540674in}}{\pgfqpoint{2.269138in}{1.537402in}}{\pgfqpoint{2.277374in}{1.537402in}}%
\pgfpathclose%
\pgfusepath{stroke,fill}%
\end{pgfscope}%
\begin{pgfscope}%
\pgfpathrectangle{\pgfqpoint{0.100000in}{0.212622in}}{\pgfqpoint{3.696000in}{3.696000in}}%
\pgfusepath{clip}%
\pgfsetbuttcap%
\pgfsetroundjoin%
\definecolor{currentfill}{rgb}{0.121569,0.466667,0.705882}%
\pgfsetfillcolor{currentfill}%
\pgfsetfillopacity{0.822141}%
\pgfsetlinewidth{1.003750pt}%
\definecolor{currentstroke}{rgb}{0.121569,0.466667,0.705882}%
\pgfsetstrokecolor{currentstroke}%
\pgfsetstrokeopacity{0.822141}%
\pgfsetdash{}{0pt}%
\pgfpathmoveto{\pgfqpoint{2.278694in}{1.537400in}}%
\pgfpathcurveto{\pgfqpoint{2.286931in}{1.537400in}}{\pgfqpoint{2.294831in}{1.540673in}}{\pgfqpoint{2.300655in}{1.546497in}}%
\pgfpathcurveto{\pgfqpoint{2.306479in}{1.552320in}}{\pgfqpoint{2.309751in}{1.560221in}}{\pgfqpoint{2.309751in}{1.568457in}}%
\pgfpathcurveto{\pgfqpoint{2.309751in}{1.576693in}}{\pgfqpoint{2.306479in}{1.584593in}}{\pgfqpoint{2.300655in}{1.590417in}}%
\pgfpathcurveto{\pgfqpoint{2.294831in}{1.596241in}}{\pgfqpoint{2.286931in}{1.599513in}}{\pgfqpoint{2.278694in}{1.599513in}}%
\pgfpathcurveto{\pgfqpoint{2.270458in}{1.599513in}}{\pgfqpoint{2.262558in}{1.596241in}}{\pgfqpoint{2.256734in}{1.590417in}}%
\pgfpathcurveto{\pgfqpoint{2.250910in}{1.584593in}}{\pgfqpoint{2.247638in}{1.576693in}}{\pgfqpoint{2.247638in}{1.568457in}}%
\pgfpathcurveto{\pgfqpoint{2.247638in}{1.560221in}}{\pgfqpoint{2.250910in}{1.552320in}}{\pgfqpoint{2.256734in}{1.546497in}}%
\pgfpathcurveto{\pgfqpoint{2.262558in}{1.540673in}}{\pgfqpoint{2.270458in}{1.537400in}}{\pgfqpoint{2.278694in}{1.537400in}}%
\pgfpathclose%
\pgfusepath{stroke,fill}%
\end{pgfscope}%
\begin{pgfscope}%
\pgfpathrectangle{\pgfqpoint{0.100000in}{0.212622in}}{\pgfqpoint{3.696000in}{3.696000in}}%
\pgfusepath{clip}%
\pgfsetbuttcap%
\pgfsetroundjoin%
\definecolor{currentfill}{rgb}{0.121569,0.466667,0.705882}%
\pgfsetfillcolor{currentfill}%
\pgfsetfillopacity{0.822775}%
\pgfsetlinewidth{1.003750pt}%
\definecolor{currentstroke}{rgb}{0.121569,0.466667,0.705882}%
\pgfsetstrokecolor{currentstroke}%
\pgfsetstrokeopacity{0.822775}%
\pgfsetdash{}{0pt}%
\pgfpathmoveto{\pgfqpoint{2.279011in}{1.536190in}}%
\pgfpathcurveto{\pgfqpoint{2.287247in}{1.536190in}}{\pgfqpoint{2.295147in}{1.539463in}}{\pgfqpoint{2.300971in}{1.545286in}}%
\pgfpathcurveto{\pgfqpoint{2.306795in}{1.551110in}}{\pgfqpoint{2.310067in}{1.559010in}}{\pgfqpoint{2.310067in}{1.567247in}}%
\pgfpathcurveto{\pgfqpoint{2.310067in}{1.575483in}}{\pgfqpoint{2.306795in}{1.583383in}}{\pgfqpoint{2.300971in}{1.589207in}}%
\pgfpathcurveto{\pgfqpoint{2.295147in}{1.595031in}}{\pgfqpoint{2.287247in}{1.598303in}}{\pgfqpoint{2.279011in}{1.598303in}}%
\pgfpathcurveto{\pgfqpoint{2.270775in}{1.598303in}}{\pgfqpoint{2.262875in}{1.595031in}}{\pgfqpoint{2.257051in}{1.589207in}}%
\pgfpathcurveto{\pgfqpoint{2.251227in}{1.583383in}}{\pgfqpoint{2.247954in}{1.575483in}}{\pgfqpoint{2.247954in}{1.567247in}}%
\pgfpathcurveto{\pgfqpoint{2.247954in}{1.559010in}}{\pgfqpoint{2.251227in}{1.551110in}}{\pgfqpoint{2.257051in}{1.545286in}}%
\pgfpathcurveto{\pgfqpoint{2.262875in}{1.539463in}}{\pgfqpoint{2.270775in}{1.536190in}}{\pgfqpoint{2.279011in}{1.536190in}}%
\pgfpathclose%
\pgfusepath{stroke,fill}%
\end{pgfscope}%
\begin{pgfscope}%
\pgfpathrectangle{\pgfqpoint{0.100000in}{0.212622in}}{\pgfqpoint{3.696000in}{3.696000in}}%
\pgfusepath{clip}%
\pgfsetbuttcap%
\pgfsetroundjoin%
\definecolor{currentfill}{rgb}{0.121569,0.466667,0.705882}%
\pgfsetfillcolor{currentfill}%
\pgfsetfillopacity{0.822844}%
\pgfsetlinewidth{1.003750pt}%
\definecolor{currentstroke}{rgb}{0.121569,0.466667,0.705882}%
\pgfsetstrokecolor{currentstroke}%
\pgfsetstrokeopacity{0.822844}%
\pgfsetdash{}{0pt}%
\pgfpathmoveto{\pgfqpoint{0.645732in}{2.598001in}}%
\pgfpathcurveto{\pgfqpoint{0.653968in}{2.598001in}}{\pgfqpoint{0.661868in}{2.601273in}}{\pgfqpoint{0.667692in}{2.607097in}}%
\pgfpathcurveto{\pgfqpoint{0.673516in}{2.612921in}}{\pgfqpoint{0.676788in}{2.620821in}}{\pgfqpoint{0.676788in}{2.629057in}}%
\pgfpathcurveto{\pgfqpoint{0.676788in}{2.637294in}}{\pgfqpoint{0.673516in}{2.645194in}}{\pgfqpoint{0.667692in}{2.651018in}}%
\pgfpathcurveto{\pgfqpoint{0.661868in}{2.656841in}}{\pgfqpoint{0.653968in}{2.660114in}}{\pgfqpoint{0.645732in}{2.660114in}}%
\pgfpathcurveto{\pgfqpoint{0.637495in}{2.660114in}}{\pgfqpoint{0.629595in}{2.656841in}}{\pgfqpoint{0.623771in}{2.651018in}}%
\pgfpathcurveto{\pgfqpoint{0.617947in}{2.645194in}}{\pgfqpoint{0.614675in}{2.637294in}}{\pgfqpoint{0.614675in}{2.629057in}}%
\pgfpathcurveto{\pgfqpoint{0.614675in}{2.620821in}}{\pgfqpoint{0.617947in}{2.612921in}}{\pgfqpoint{0.623771in}{2.607097in}}%
\pgfpathcurveto{\pgfqpoint{0.629595in}{2.601273in}}{\pgfqpoint{0.637495in}{2.598001in}}{\pgfqpoint{0.645732in}{2.598001in}}%
\pgfpathclose%
\pgfusepath{stroke,fill}%
\end{pgfscope}%
\begin{pgfscope}%
\pgfpathrectangle{\pgfqpoint{0.100000in}{0.212622in}}{\pgfqpoint{3.696000in}{3.696000in}}%
\pgfusepath{clip}%
\pgfsetbuttcap%
\pgfsetroundjoin%
\definecolor{currentfill}{rgb}{0.121569,0.466667,0.705882}%
\pgfsetfillcolor{currentfill}%
\pgfsetfillopacity{0.823951}%
\pgfsetlinewidth{1.003750pt}%
\definecolor{currentstroke}{rgb}{0.121569,0.466667,0.705882}%
\pgfsetstrokecolor{currentstroke}%
\pgfsetstrokeopacity{0.823951}%
\pgfsetdash{}{0pt}%
\pgfpathmoveto{\pgfqpoint{2.279444in}{1.535591in}}%
\pgfpathcurveto{\pgfqpoint{2.287681in}{1.535591in}}{\pgfqpoint{2.295581in}{1.538863in}}{\pgfqpoint{2.301405in}{1.544687in}}%
\pgfpathcurveto{\pgfqpoint{2.307229in}{1.550511in}}{\pgfqpoint{2.310501in}{1.558411in}}{\pgfqpoint{2.310501in}{1.566648in}}%
\pgfpathcurveto{\pgfqpoint{2.310501in}{1.574884in}}{\pgfqpoint{2.307229in}{1.582784in}}{\pgfqpoint{2.301405in}{1.588608in}}%
\pgfpathcurveto{\pgfqpoint{2.295581in}{1.594432in}}{\pgfqpoint{2.287681in}{1.597704in}}{\pgfqpoint{2.279444in}{1.597704in}}%
\pgfpathcurveto{\pgfqpoint{2.271208in}{1.597704in}}{\pgfqpoint{2.263308in}{1.594432in}}{\pgfqpoint{2.257484in}{1.588608in}}%
\pgfpathcurveto{\pgfqpoint{2.251660in}{1.582784in}}{\pgfqpoint{2.248388in}{1.574884in}}{\pgfqpoint{2.248388in}{1.566648in}}%
\pgfpathcurveto{\pgfqpoint{2.248388in}{1.558411in}}{\pgfqpoint{2.251660in}{1.550511in}}{\pgfqpoint{2.257484in}{1.544687in}}%
\pgfpathcurveto{\pgfqpoint{2.263308in}{1.538863in}}{\pgfqpoint{2.271208in}{1.535591in}}{\pgfqpoint{2.279444in}{1.535591in}}%
\pgfpathclose%
\pgfusepath{stroke,fill}%
\end{pgfscope}%
\begin{pgfscope}%
\pgfpathrectangle{\pgfqpoint{0.100000in}{0.212622in}}{\pgfqpoint{3.696000in}{3.696000in}}%
\pgfusepath{clip}%
\pgfsetbuttcap%
\pgfsetroundjoin%
\definecolor{currentfill}{rgb}{0.121569,0.466667,0.705882}%
\pgfsetfillcolor{currentfill}%
\pgfsetfillopacity{0.824140}%
\pgfsetlinewidth{1.003750pt}%
\definecolor{currentstroke}{rgb}{0.121569,0.466667,0.705882}%
\pgfsetstrokecolor{currentstroke}%
\pgfsetstrokeopacity{0.824140}%
\pgfsetdash{}{0pt}%
\pgfpathmoveto{\pgfqpoint{0.624273in}{2.622812in}}%
\pgfpathcurveto{\pgfqpoint{0.632509in}{2.622812in}}{\pgfqpoint{0.640409in}{2.626084in}}{\pgfqpoint{0.646233in}{2.631908in}}%
\pgfpathcurveto{\pgfqpoint{0.652057in}{2.637732in}}{\pgfqpoint{0.655329in}{2.645632in}}{\pgfqpoint{0.655329in}{2.653869in}}%
\pgfpathcurveto{\pgfqpoint{0.655329in}{2.662105in}}{\pgfqpoint{0.652057in}{2.670005in}}{\pgfqpoint{0.646233in}{2.675829in}}%
\pgfpathcurveto{\pgfqpoint{0.640409in}{2.681653in}}{\pgfqpoint{0.632509in}{2.684925in}}{\pgfqpoint{0.624273in}{2.684925in}}%
\pgfpathcurveto{\pgfqpoint{0.616036in}{2.684925in}}{\pgfqpoint{0.608136in}{2.681653in}}{\pgfqpoint{0.602312in}{2.675829in}}%
\pgfpathcurveto{\pgfqpoint{0.596489in}{2.670005in}}{\pgfqpoint{0.593216in}{2.662105in}}{\pgfqpoint{0.593216in}{2.653869in}}%
\pgfpathcurveto{\pgfqpoint{0.593216in}{2.645632in}}{\pgfqpoint{0.596489in}{2.637732in}}{\pgfqpoint{0.602312in}{2.631908in}}%
\pgfpathcurveto{\pgfqpoint{0.608136in}{2.626084in}}{\pgfqpoint{0.616036in}{2.622812in}}{\pgfqpoint{0.624273in}{2.622812in}}%
\pgfpathclose%
\pgfusepath{stroke,fill}%
\end{pgfscope}%
\begin{pgfscope}%
\pgfpathrectangle{\pgfqpoint{0.100000in}{0.212622in}}{\pgfqpoint{3.696000in}{3.696000in}}%
\pgfusepath{clip}%
\pgfsetbuttcap%
\pgfsetroundjoin%
\definecolor{currentfill}{rgb}{0.121569,0.466667,0.705882}%
\pgfsetfillcolor{currentfill}%
\pgfsetfillopacity{0.824147}%
\pgfsetlinewidth{1.003750pt}%
\definecolor{currentstroke}{rgb}{0.121569,0.466667,0.705882}%
\pgfsetstrokecolor{currentstroke}%
\pgfsetstrokeopacity{0.824147}%
\pgfsetdash{}{0pt}%
\pgfpathmoveto{\pgfqpoint{0.635681in}{2.616818in}}%
\pgfpathcurveto{\pgfqpoint{0.643917in}{2.616818in}}{\pgfqpoint{0.651817in}{2.620091in}}{\pgfqpoint{0.657641in}{2.625915in}}%
\pgfpathcurveto{\pgfqpoint{0.663465in}{2.631739in}}{\pgfqpoint{0.666737in}{2.639639in}}{\pgfqpoint{0.666737in}{2.647875in}}%
\pgfpathcurveto{\pgfqpoint{0.666737in}{2.656111in}}{\pgfqpoint{0.663465in}{2.664011in}}{\pgfqpoint{0.657641in}{2.669835in}}%
\pgfpathcurveto{\pgfqpoint{0.651817in}{2.675659in}}{\pgfqpoint{0.643917in}{2.678931in}}{\pgfqpoint{0.635681in}{2.678931in}}%
\pgfpathcurveto{\pgfqpoint{0.627444in}{2.678931in}}{\pgfqpoint{0.619544in}{2.675659in}}{\pgfqpoint{0.613720in}{2.669835in}}%
\pgfpathcurveto{\pgfqpoint{0.607897in}{2.664011in}}{\pgfqpoint{0.604624in}{2.656111in}}{\pgfqpoint{0.604624in}{2.647875in}}%
\pgfpathcurveto{\pgfqpoint{0.604624in}{2.639639in}}{\pgfqpoint{0.607897in}{2.631739in}}{\pgfqpoint{0.613720in}{2.625915in}}%
\pgfpathcurveto{\pgfqpoint{0.619544in}{2.620091in}}{\pgfqpoint{0.627444in}{2.616818in}}{\pgfqpoint{0.635681in}{2.616818in}}%
\pgfpathclose%
\pgfusepath{stroke,fill}%
\end{pgfscope}%
\begin{pgfscope}%
\pgfpathrectangle{\pgfqpoint{0.100000in}{0.212622in}}{\pgfqpoint{3.696000in}{3.696000in}}%
\pgfusepath{clip}%
\pgfsetbuttcap%
\pgfsetroundjoin%
\definecolor{currentfill}{rgb}{0.121569,0.466667,0.705882}%
\pgfsetfillcolor{currentfill}%
\pgfsetfillopacity{0.824452}%
\pgfsetlinewidth{1.003750pt}%
\definecolor{currentstroke}{rgb}{0.121569,0.466667,0.705882}%
\pgfsetstrokecolor{currentstroke}%
\pgfsetstrokeopacity{0.824452}%
\pgfsetdash{}{0pt}%
\pgfpathmoveto{\pgfqpoint{2.280062in}{1.534574in}}%
\pgfpathcurveto{\pgfqpoint{2.288298in}{1.534574in}}{\pgfqpoint{2.296199in}{1.537847in}}{\pgfqpoint{2.302022in}{1.543671in}}%
\pgfpathcurveto{\pgfqpoint{2.307846in}{1.549495in}}{\pgfqpoint{2.311119in}{1.557395in}}{\pgfqpoint{2.311119in}{1.565631in}}%
\pgfpathcurveto{\pgfqpoint{2.311119in}{1.573867in}}{\pgfqpoint{2.307846in}{1.581767in}}{\pgfqpoint{2.302022in}{1.587591in}}%
\pgfpathcurveto{\pgfqpoint{2.296199in}{1.593415in}}{\pgfqpoint{2.288298in}{1.596687in}}{\pgfqpoint{2.280062in}{1.596687in}}%
\pgfpathcurveto{\pgfqpoint{2.271826in}{1.596687in}}{\pgfqpoint{2.263926in}{1.593415in}}{\pgfqpoint{2.258102in}{1.587591in}}%
\pgfpathcurveto{\pgfqpoint{2.252278in}{1.581767in}}{\pgfqpoint{2.249006in}{1.573867in}}{\pgfqpoint{2.249006in}{1.565631in}}%
\pgfpathcurveto{\pgfqpoint{2.249006in}{1.557395in}}{\pgfqpoint{2.252278in}{1.549495in}}{\pgfqpoint{2.258102in}{1.543671in}}%
\pgfpathcurveto{\pgfqpoint{2.263926in}{1.537847in}}{\pgfqpoint{2.271826in}{1.534574in}}{\pgfqpoint{2.280062in}{1.534574in}}%
\pgfpathclose%
\pgfusepath{stroke,fill}%
\end{pgfscope}%
\begin{pgfscope}%
\pgfpathrectangle{\pgfqpoint{0.100000in}{0.212622in}}{\pgfqpoint{3.696000in}{3.696000in}}%
\pgfusepath{clip}%
\pgfsetbuttcap%
\pgfsetroundjoin%
\definecolor{currentfill}{rgb}{0.121569,0.466667,0.705882}%
\pgfsetfillcolor{currentfill}%
\pgfsetfillopacity{0.824916}%
\pgfsetlinewidth{1.003750pt}%
\definecolor{currentstroke}{rgb}{0.121569,0.466667,0.705882}%
\pgfsetstrokecolor{currentstroke}%
\pgfsetstrokeopacity{0.824916}%
\pgfsetdash{}{0pt}%
\pgfpathmoveto{\pgfqpoint{0.652050in}{2.596384in}}%
\pgfpathcurveto{\pgfqpoint{0.660286in}{2.596384in}}{\pgfqpoint{0.668186in}{2.599656in}}{\pgfqpoint{0.674010in}{2.605480in}}%
\pgfpathcurveto{\pgfqpoint{0.679834in}{2.611304in}}{\pgfqpoint{0.683106in}{2.619204in}}{\pgfqpoint{0.683106in}{2.627440in}}%
\pgfpathcurveto{\pgfqpoint{0.683106in}{2.635676in}}{\pgfqpoint{0.679834in}{2.643576in}}{\pgfqpoint{0.674010in}{2.649400in}}%
\pgfpathcurveto{\pgfqpoint{0.668186in}{2.655224in}}{\pgfqpoint{0.660286in}{2.658497in}}{\pgfqpoint{0.652050in}{2.658497in}}%
\pgfpathcurveto{\pgfqpoint{0.643814in}{2.658497in}}{\pgfqpoint{0.635914in}{2.655224in}}{\pgfqpoint{0.630090in}{2.649400in}}%
\pgfpathcurveto{\pgfqpoint{0.624266in}{2.643576in}}{\pgfqpoint{0.620993in}{2.635676in}}{\pgfqpoint{0.620993in}{2.627440in}}%
\pgfpathcurveto{\pgfqpoint{0.620993in}{2.619204in}}{\pgfqpoint{0.624266in}{2.611304in}}{\pgfqpoint{0.630090in}{2.605480in}}%
\pgfpathcurveto{\pgfqpoint{0.635914in}{2.599656in}}{\pgfqpoint{0.643814in}{2.596384in}}{\pgfqpoint{0.652050in}{2.596384in}}%
\pgfpathclose%
\pgfusepath{stroke,fill}%
\end{pgfscope}%
\begin{pgfscope}%
\pgfpathrectangle{\pgfqpoint{0.100000in}{0.212622in}}{\pgfqpoint{3.696000in}{3.696000in}}%
\pgfusepath{clip}%
\pgfsetbuttcap%
\pgfsetroundjoin%
\definecolor{currentfill}{rgb}{0.121569,0.466667,0.705882}%
\pgfsetfillcolor{currentfill}%
\pgfsetfillopacity{0.825443}%
\pgfsetlinewidth{1.003750pt}%
\definecolor{currentstroke}{rgb}{0.121569,0.466667,0.705882}%
\pgfsetstrokecolor{currentstroke}%
\pgfsetstrokeopacity{0.825443}%
\pgfsetdash{}{0pt}%
\pgfpathmoveto{\pgfqpoint{2.281044in}{1.534908in}}%
\pgfpathcurveto{\pgfqpoint{2.289280in}{1.534908in}}{\pgfqpoint{2.297180in}{1.538180in}}{\pgfqpoint{2.303004in}{1.544004in}}%
\pgfpathcurveto{\pgfqpoint{2.308828in}{1.549828in}}{\pgfqpoint{2.312100in}{1.557728in}}{\pgfqpoint{2.312100in}{1.565964in}}%
\pgfpathcurveto{\pgfqpoint{2.312100in}{1.574200in}}{\pgfqpoint{2.308828in}{1.582100in}}{\pgfqpoint{2.303004in}{1.587924in}}%
\pgfpathcurveto{\pgfqpoint{2.297180in}{1.593748in}}{\pgfqpoint{2.289280in}{1.597021in}}{\pgfqpoint{2.281044in}{1.597021in}}%
\pgfpathcurveto{\pgfqpoint{2.272807in}{1.597021in}}{\pgfqpoint{2.264907in}{1.593748in}}{\pgfqpoint{2.259083in}{1.587924in}}%
\pgfpathcurveto{\pgfqpoint{2.253259in}{1.582100in}}{\pgfqpoint{2.249987in}{1.574200in}}{\pgfqpoint{2.249987in}{1.565964in}}%
\pgfpathcurveto{\pgfqpoint{2.249987in}{1.557728in}}{\pgfqpoint{2.253259in}{1.549828in}}{\pgfqpoint{2.259083in}{1.544004in}}%
\pgfpathcurveto{\pgfqpoint{2.264907in}{1.538180in}}{\pgfqpoint{2.272807in}{1.534908in}}{\pgfqpoint{2.281044in}{1.534908in}}%
\pgfpathclose%
\pgfusepath{stroke,fill}%
\end{pgfscope}%
\begin{pgfscope}%
\pgfpathrectangle{\pgfqpoint{0.100000in}{0.212622in}}{\pgfqpoint{3.696000in}{3.696000in}}%
\pgfusepath{clip}%
\pgfsetbuttcap%
\pgfsetroundjoin%
\definecolor{currentfill}{rgb}{0.121569,0.466667,0.705882}%
\pgfsetfillcolor{currentfill}%
\pgfsetfillopacity{0.825734}%
\pgfsetlinewidth{1.003750pt}%
\definecolor{currentstroke}{rgb}{0.121569,0.466667,0.705882}%
\pgfsetstrokecolor{currentstroke}%
\pgfsetstrokeopacity{0.825734}%
\pgfsetdash{}{0pt}%
\pgfpathmoveto{\pgfqpoint{0.658315in}{2.589167in}}%
\pgfpathcurveto{\pgfqpoint{0.666551in}{2.589167in}}{\pgfqpoint{0.674451in}{2.592439in}}{\pgfqpoint{0.680275in}{2.598263in}}%
\pgfpathcurveto{\pgfqpoint{0.686099in}{2.604087in}}{\pgfqpoint{0.689371in}{2.611987in}}{\pgfqpoint{0.689371in}{2.620223in}}%
\pgfpathcurveto{\pgfqpoint{0.689371in}{2.628459in}}{\pgfqpoint{0.686099in}{2.636359in}}{\pgfqpoint{0.680275in}{2.642183in}}%
\pgfpathcurveto{\pgfqpoint{0.674451in}{2.648007in}}{\pgfqpoint{0.666551in}{2.651280in}}{\pgfqpoint{0.658315in}{2.651280in}}%
\pgfpathcurveto{\pgfqpoint{0.650078in}{2.651280in}}{\pgfqpoint{0.642178in}{2.648007in}}{\pgfqpoint{0.636354in}{2.642183in}}%
\pgfpathcurveto{\pgfqpoint{0.630530in}{2.636359in}}{\pgfqpoint{0.627258in}{2.628459in}}{\pgfqpoint{0.627258in}{2.620223in}}%
\pgfpathcurveto{\pgfqpoint{0.627258in}{2.611987in}}{\pgfqpoint{0.630530in}{2.604087in}}{\pgfqpoint{0.636354in}{2.598263in}}%
\pgfpathcurveto{\pgfqpoint{0.642178in}{2.592439in}}{\pgfqpoint{0.650078in}{2.589167in}}{\pgfqpoint{0.658315in}{2.589167in}}%
\pgfpathclose%
\pgfusepath{stroke,fill}%
\end{pgfscope}%
\begin{pgfscope}%
\pgfpathrectangle{\pgfqpoint{0.100000in}{0.212622in}}{\pgfqpoint{3.696000in}{3.696000in}}%
\pgfusepath{clip}%
\pgfsetbuttcap%
\pgfsetroundjoin%
\definecolor{currentfill}{rgb}{0.121569,0.466667,0.705882}%
\pgfsetfillcolor{currentfill}%
\pgfsetfillopacity{0.825922}%
\pgfsetlinewidth{1.003750pt}%
\definecolor{currentstroke}{rgb}{0.121569,0.466667,0.705882}%
\pgfsetstrokecolor{currentstroke}%
\pgfsetstrokeopacity{0.825922}%
\pgfsetdash{}{0pt}%
\pgfpathmoveto{\pgfqpoint{2.281249in}{1.534479in}}%
\pgfpathcurveto{\pgfqpoint{2.289486in}{1.534479in}}{\pgfqpoint{2.297386in}{1.537751in}}{\pgfqpoint{2.303210in}{1.543575in}}%
\pgfpathcurveto{\pgfqpoint{2.309034in}{1.549399in}}{\pgfqpoint{2.312306in}{1.557299in}}{\pgfqpoint{2.312306in}{1.565536in}}%
\pgfpathcurveto{\pgfqpoint{2.312306in}{1.573772in}}{\pgfqpoint{2.309034in}{1.581672in}}{\pgfqpoint{2.303210in}{1.587496in}}%
\pgfpathcurveto{\pgfqpoint{2.297386in}{1.593320in}}{\pgfqpoint{2.289486in}{1.596592in}}{\pgfqpoint{2.281249in}{1.596592in}}%
\pgfpathcurveto{\pgfqpoint{2.273013in}{1.596592in}}{\pgfqpoint{2.265113in}{1.593320in}}{\pgfqpoint{2.259289in}{1.587496in}}%
\pgfpathcurveto{\pgfqpoint{2.253465in}{1.581672in}}{\pgfqpoint{2.250193in}{1.573772in}}{\pgfqpoint{2.250193in}{1.565536in}}%
\pgfpathcurveto{\pgfqpoint{2.250193in}{1.557299in}}{\pgfqpoint{2.253465in}{1.549399in}}{\pgfqpoint{2.259289in}{1.543575in}}%
\pgfpathcurveto{\pgfqpoint{2.265113in}{1.537751in}}{\pgfqpoint{2.273013in}{1.534479in}}{\pgfqpoint{2.281249in}{1.534479in}}%
\pgfpathclose%
\pgfusepath{stroke,fill}%
\end{pgfscope}%
\begin{pgfscope}%
\pgfpathrectangle{\pgfqpoint{0.100000in}{0.212622in}}{\pgfqpoint{3.696000in}{3.696000in}}%
\pgfusepath{clip}%
\pgfsetbuttcap%
\pgfsetroundjoin%
\definecolor{currentfill}{rgb}{0.121569,0.466667,0.705882}%
\pgfsetfillcolor{currentfill}%
\pgfsetfillopacity{0.826037}%
\pgfsetlinewidth{1.003750pt}%
\definecolor{currentstroke}{rgb}{0.121569,0.466667,0.705882}%
\pgfsetstrokecolor{currentstroke}%
\pgfsetstrokeopacity{0.826037}%
\pgfsetdash{}{0pt}%
\pgfpathmoveto{\pgfqpoint{0.614009in}{2.638724in}}%
\pgfpathcurveto{\pgfqpoint{0.622245in}{2.638724in}}{\pgfqpoint{0.630145in}{2.641996in}}{\pgfqpoint{0.635969in}{2.647820in}}%
\pgfpathcurveto{\pgfqpoint{0.641793in}{2.653644in}}{\pgfqpoint{0.645065in}{2.661544in}}{\pgfqpoint{0.645065in}{2.669780in}}%
\pgfpathcurveto{\pgfqpoint{0.645065in}{2.678016in}}{\pgfqpoint{0.641793in}{2.685917in}}{\pgfqpoint{0.635969in}{2.691740in}}%
\pgfpathcurveto{\pgfqpoint{0.630145in}{2.697564in}}{\pgfqpoint{0.622245in}{2.700837in}}{\pgfqpoint{0.614009in}{2.700837in}}%
\pgfpathcurveto{\pgfqpoint{0.605772in}{2.700837in}}{\pgfqpoint{0.597872in}{2.697564in}}{\pgfqpoint{0.592048in}{2.691740in}}%
\pgfpathcurveto{\pgfqpoint{0.586224in}{2.685917in}}{\pgfqpoint{0.582952in}{2.678016in}}{\pgfqpoint{0.582952in}{2.669780in}}%
\pgfpathcurveto{\pgfqpoint{0.582952in}{2.661544in}}{\pgfqpoint{0.586224in}{2.653644in}}{\pgfqpoint{0.592048in}{2.647820in}}%
\pgfpathcurveto{\pgfqpoint{0.597872in}{2.641996in}}{\pgfqpoint{0.605772in}{2.638724in}}{\pgfqpoint{0.614009in}{2.638724in}}%
\pgfpathclose%
\pgfusepath{stroke,fill}%
\end{pgfscope}%
\begin{pgfscope}%
\pgfpathrectangle{\pgfqpoint{0.100000in}{0.212622in}}{\pgfqpoint{3.696000in}{3.696000in}}%
\pgfusepath{clip}%
\pgfsetbuttcap%
\pgfsetroundjoin%
\definecolor{currentfill}{rgb}{0.121569,0.466667,0.705882}%
\pgfsetfillcolor{currentfill}%
\pgfsetfillopacity{0.826463}%
\pgfsetlinewidth{1.003750pt}%
\definecolor{currentstroke}{rgb}{0.121569,0.466667,0.705882}%
\pgfsetstrokecolor{currentstroke}%
\pgfsetstrokeopacity{0.826463}%
\pgfsetdash{}{0pt}%
\pgfpathmoveto{\pgfqpoint{0.610810in}{2.640799in}}%
\pgfpathcurveto{\pgfqpoint{0.619047in}{2.640799in}}{\pgfqpoint{0.626947in}{2.644071in}}{\pgfqpoint{0.632771in}{2.649895in}}%
\pgfpathcurveto{\pgfqpoint{0.638595in}{2.655719in}}{\pgfqpoint{0.641867in}{2.663619in}}{\pgfqpoint{0.641867in}{2.671855in}}%
\pgfpathcurveto{\pgfqpoint{0.641867in}{2.680092in}}{\pgfqpoint{0.638595in}{2.687992in}}{\pgfqpoint{0.632771in}{2.693816in}}%
\pgfpathcurveto{\pgfqpoint{0.626947in}{2.699640in}}{\pgfqpoint{0.619047in}{2.702912in}}{\pgfqpoint{0.610810in}{2.702912in}}%
\pgfpathcurveto{\pgfqpoint{0.602574in}{2.702912in}}{\pgfqpoint{0.594674in}{2.699640in}}{\pgfqpoint{0.588850in}{2.693816in}}%
\pgfpathcurveto{\pgfqpoint{0.583026in}{2.687992in}}{\pgfqpoint{0.579754in}{2.680092in}}{\pgfqpoint{0.579754in}{2.671855in}}%
\pgfpathcurveto{\pgfqpoint{0.579754in}{2.663619in}}{\pgfqpoint{0.583026in}{2.655719in}}{\pgfqpoint{0.588850in}{2.649895in}}%
\pgfpathcurveto{\pgfqpoint{0.594674in}{2.644071in}}{\pgfqpoint{0.602574in}{2.640799in}}{\pgfqpoint{0.610810in}{2.640799in}}%
\pgfpathclose%
\pgfusepath{stroke,fill}%
\end{pgfscope}%
\begin{pgfscope}%
\pgfpathrectangle{\pgfqpoint{0.100000in}{0.212622in}}{\pgfqpoint{3.696000in}{3.696000in}}%
\pgfusepath{clip}%
\pgfsetbuttcap%
\pgfsetroundjoin%
\definecolor{currentfill}{rgb}{0.121569,0.466667,0.705882}%
\pgfsetfillcolor{currentfill}%
\pgfsetfillopacity{0.826533}%
\pgfsetlinewidth{1.003750pt}%
\definecolor{currentstroke}{rgb}{0.121569,0.466667,0.705882}%
\pgfsetstrokecolor{currentstroke}%
\pgfsetstrokeopacity{0.826533}%
\pgfsetdash{}{0pt}%
\pgfpathmoveto{\pgfqpoint{0.611889in}{2.641212in}}%
\pgfpathcurveto{\pgfqpoint{0.620125in}{2.641212in}}{\pgfqpoint{0.628025in}{2.644484in}}{\pgfqpoint{0.633849in}{2.650308in}}%
\pgfpathcurveto{\pgfqpoint{0.639673in}{2.656132in}}{\pgfqpoint{0.642945in}{2.664032in}}{\pgfqpoint{0.642945in}{2.672268in}}%
\pgfpathcurveto{\pgfqpoint{0.642945in}{2.680505in}}{\pgfqpoint{0.639673in}{2.688405in}}{\pgfqpoint{0.633849in}{2.694229in}}%
\pgfpathcurveto{\pgfqpoint{0.628025in}{2.700053in}}{\pgfqpoint{0.620125in}{2.703325in}}{\pgfqpoint{0.611889in}{2.703325in}}%
\pgfpathcurveto{\pgfqpoint{0.603652in}{2.703325in}}{\pgfqpoint{0.595752in}{2.700053in}}{\pgfqpoint{0.589928in}{2.694229in}}%
\pgfpathcurveto{\pgfqpoint{0.584105in}{2.688405in}}{\pgfqpoint{0.580832in}{2.680505in}}{\pgfqpoint{0.580832in}{2.672268in}}%
\pgfpathcurveto{\pgfqpoint{0.580832in}{2.664032in}}{\pgfqpoint{0.584105in}{2.656132in}}{\pgfqpoint{0.589928in}{2.650308in}}%
\pgfpathcurveto{\pgfqpoint{0.595752in}{2.644484in}}{\pgfqpoint{0.603652in}{2.641212in}}{\pgfqpoint{0.611889in}{2.641212in}}%
\pgfpathclose%
\pgfusepath{stroke,fill}%
\end{pgfscope}%
\begin{pgfscope}%
\pgfpathrectangle{\pgfqpoint{0.100000in}{0.212622in}}{\pgfqpoint{3.696000in}{3.696000in}}%
\pgfusepath{clip}%
\pgfsetbuttcap%
\pgfsetroundjoin%
\definecolor{currentfill}{rgb}{0.121569,0.466667,0.705882}%
\pgfsetfillcolor{currentfill}%
\pgfsetfillopacity{0.826568}%
\pgfsetlinewidth{1.003750pt}%
\definecolor{currentstroke}{rgb}{0.121569,0.466667,0.705882}%
\pgfsetstrokecolor{currentstroke}%
\pgfsetstrokeopacity{0.826568}%
\pgfsetdash{}{0pt}%
\pgfpathmoveto{\pgfqpoint{2.281684in}{1.532930in}}%
\pgfpathcurveto{\pgfqpoint{2.289921in}{1.532930in}}{\pgfqpoint{2.297821in}{1.536202in}}{\pgfqpoint{2.303645in}{1.542026in}}%
\pgfpathcurveto{\pgfqpoint{2.309469in}{1.547850in}}{\pgfqpoint{2.312741in}{1.555750in}}{\pgfqpoint{2.312741in}{1.563986in}}%
\pgfpathcurveto{\pgfqpoint{2.312741in}{1.572222in}}{\pgfqpoint{2.309469in}{1.580122in}}{\pgfqpoint{2.303645in}{1.585946in}}%
\pgfpathcurveto{\pgfqpoint{2.297821in}{1.591770in}}{\pgfqpoint{2.289921in}{1.595043in}}{\pgfqpoint{2.281684in}{1.595043in}}%
\pgfpathcurveto{\pgfqpoint{2.273448in}{1.595043in}}{\pgfqpoint{2.265548in}{1.591770in}}{\pgfqpoint{2.259724in}{1.585946in}}%
\pgfpathcurveto{\pgfqpoint{2.253900in}{1.580122in}}{\pgfqpoint{2.250628in}{1.572222in}}{\pgfqpoint{2.250628in}{1.563986in}}%
\pgfpathcurveto{\pgfqpoint{2.250628in}{1.555750in}}{\pgfqpoint{2.253900in}{1.547850in}}{\pgfqpoint{2.259724in}{1.542026in}}%
\pgfpathcurveto{\pgfqpoint{2.265548in}{1.536202in}}{\pgfqpoint{2.273448in}{1.532930in}}{\pgfqpoint{2.281684in}{1.532930in}}%
\pgfpathclose%
\pgfusepath{stroke,fill}%
\end{pgfscope}%
\begin{pgfscope}%
\pgfpathrectangle{\pgfqpoint{0.100000in}{0.212622in}}{\pgfqpoint{3.696000in}{3.696000in}}%
\pgfusepath{clip}%
\pgfsetbuttcap%
\pgfsetroundjoin%
\definecolor{currentfill}{rgb}{0.121569,0.466667,0.705882}%
\pgfsetfillcolor{currentfill}%
\pgfsetfillopacity{0.826576}%
\pgfsetlinewidth{1.003750pt}%
\definecolor{currentstroke}{rgb}{0.121569,0.466667,0.705882}%
\pgfsetstrokecolor{currentstroke}%
\pgfsetstrokeopacity{0.826576}%
\pgfsetdash{}{0pt}%
\pgfpathmoveto{\pgfqpoint{0.610257in}{2.640808in}}%
\pgfpathcurveto{\pgfqpoint{0.618494in}{2.640808in}}{\pgfqpoint{0.626394in}{2.644080in}}{\pgfqpoint{0.632218in}{2.649904in}}%
\pgfpathcurveto{\pgfqpoint{0.638042in}{2.655728in}}{\pgfqpoint{0.641314in}{2.663628in}}{\pgfqpoint{0.641314in}{2.671864in}}%
\pgfpathcurveto{\pgfqpoint{0.641314in}{2.680100in}}{\pgfqpoint{0.638042in}{2.688001in}}{\pgfqpoint{0.632218in}{2.693824in}}%
\pgfpathcurveto{\pgfqpoint{0.626394in}{2.699648in}}{\pgfqpoint{0.618494in}{2.702921in}}{\pgfqpoint{0.610257in}{2.702921in}}%
\pgfpathcurveto{\pgfqpoint{0.602021in}{2.702921in}}{\pgfqpoint{0.594121in}{2.699648in}}{\pgfqpoint{0.588297in}{2.693824in}}%
\pgfpathcurveto{\pgfqpoint{0.582473in}{2.688001in}}{\pgfqpoint{0.579201in}{2.680100in}}{\pgfqpoint{0.579201in}{2.671864in}}%
\pgfpathcurveto{\pgfqpoint{0.579201in}{2.663628in}}{\pgfqpoint{0.582473in}{2.655728in}}{\pgfqpoint{0.588297in}{2.649904in}}%
\pgfpathcurveto{\pgfqpoint{0.594121in}{2.644080in}}{\pgfqpoint{0.602021in}{2.640808in}}{\pgfqpoint{0.610257in}{2.640808in}}%
\pgfpathclose%
\pgfusepath{stroke,fill}%
\end{pgfscope}%
\begin{pgfscope}%
\pgfpathrectangle{\pgfqpoint{0.100000in}{0.212622in}}{\pgfqpoint{3.696000in}{3.696000in}}%
\pgfusepath{clip}%
\pgfsetbuttcap%
\pgfsetroundjoin%
\definecolor{currentfill}{rgb}{0.121569,0.466667,0.705882}%
\pgfsetfillcolor{currentfill}%
\pgfsetfillopacity{0.826624}%
\pgfsetlinewidth{1.003750pt}%
\definecolor{currentstroke}{rgb}{0.121569,0.466667,0.705882}%
\pgfsetstrokecolor{currentstroke}%
\pgfsetstrokeopacity{0.826624}%
\pgfsetdash{}{0pt}%
\pgfpathmoveto{\pgfqpoint{0.617087in}{2.639527in}}%
\pgfpathcurveto{\pgfqpoint{0.625323in}{2.639527in}}{\pgfqpoint{0.633223in}{2.642799in}}{\pgfqpoint{0.639047in}{2.648623in}}%
\pgfpathcurveto{\pgfqpoint{0.644871in}{2.654447in}}{\pgfqpoint{0.648143in}{2.662347in}}{\pgfqpoint{0.648143in}{2.670583in}}%
\pgfpathcurveto{\pgfqpoint{0.648143in}{2.678820in}}{\pgfqpoint{0.644871in}{2.686720in}}{\pgfqpoint{0.639047in}{2.692543in}}%
\pgfpathcurveto{\pgfqpoint{0.633223in}{2.698367in}}{\pgfqpoint{0.625323in}{2.701640in}}{\pgfqpoint{0.617087in}{2.701640in}}%
\pgfpathcurveto{\pgfqpoint{0.608850in}{2.701640in}}{\pgfqpoint{0.600950in}{2.698367in}}{\pgfqpoint{0.595126in}{2.692543in}}%
\pgfpathcurveto{\pgfqpoint{0.589302in}{2.686720in}}{\pgfqpoint{0.586030in}{2.678820in}}{\pgfqpoint{0.586030in}{2.670583in}}%
\pgfpathcurveto{\pgfqpoint{0.586030in}{2.662347in}}{\pgfqpoint{0.589302in}{2.654447in}}{\pgfqpoint{0.595126in}{2.648623in}}%
\pgfpathcurveto{\pgfqpoint{0.600950in}{2.642799in}}{\pgfqpoint{0.608850in}{2.639527in}}{\pgfqpoint{0.617087in}{2.639527in}}%
\pgfpathclose%
\pgfusepath{stroke,fill}%
\end{pgfscope}%
\begin{pgfscope}%
\pgfpathrectangle{\pgfqpoint{0.100000in}{0.212622in}}{\pgfqpoint{3.696000in}{3.696000in}}%
\pgfusepath{clip}%
\pgfsetbuttcap%
\pgfsetroundjoin%
\definecolor{currentfill}{rgb}{0.121569,0.466667,0.705882}%
\pgfsetfillcolor{currentfill}%
\pgfsetfillopacity{0.826897}%
\pgfsetlinewidth{1.003750pt}%
\definecolor{currentstroke}{rgb}{0.121569,0.466667,0.705882}%
\pgfsetstrokecolor{currentstroke}%
\pgfsetstrokeopacity{0.826897}%
\pgfsetdash{}{0pt}%
\pgfpathmoveto{\pgfqpoint{0.607750in}{2.639068in}}%
\pgfpathcurveto{\pgfqpoint{0.615987in}{2.639068in}}{\pgfqpoint{0.623887in}{2.642340in}}{\pgfqpoint{0.629711in}{2.648164in}}%
\pgfpathcurveto{\pgfqpoint{0.635535in}{2.653988in}}{\pgfqpoint{0.638807in}{2.661888in}}{\pgfqpoint{0.638807in}{2.670124in}}%
\pgfpathcurveto{\pgfqpoint{0.638807in}{2.678361in}}{\pgfqpoint{0.635535in}{2.686261in}}{\pgfqpoint{0.629711in}{2.692085in}}%
\pgfpathcurveto{\pgfqpoint{0.623887in}{2.697909in}}{\pgfqpoint{0.615987in}{2.701181in}}{\pgfqpoint{0.607750in}{2.701181in}}%
\pgfpathcurveto{\pgfqpoint{0.599514in}{2.701181in}}{\pgfqpoint{0.591614in}{2.697909in}}{\pgfqpoint{0.585790in}{2.692085in}}%
\pgfpathcurveto{\pgfqpoint{0.579966in}{2.686261in}}{\pgfqpoint{0.576694in}{2.678361in}}{\pgfqpoint{0.576694in}{2.670124in}}%
\pgfpathcurveto{\pgfqpoint{0.576694in}{2.661888in}}{\pgfqpoint{0.579966in}{2.653988in}}{\pgfqpoint{0.585790in}{2.648164in}}%
\pgfpathcurveto{\pgfqpoint{0.591614in}{2.642340in}}{\pgfqpoint{0.599514in}{2.639068in}}{\pgfqpoint{0.607750in}{2.639068in}}%
\pgfpathclose%
\pgfusepath{stroke,fill}%
\end{pgfscope}%
\begin{pgfscope}%
\pgfpathrectangle{\pgfqpoint{0.100000in}{0.212622in}}{\pgfqpoint{3.696000in}{3.696000in}}%
\pgfusepath{clip}%
\pgfsetbuttcap%
\pgfsetroundjoin%
\definecolor{currentfill}{rgb}{0.121569,0.466667,0.705882}%
\pgfsetfillcolor{currentfill}%
\pgfsetfillopacity{0.827162}%
\pgfsetlinewidth{1.003750pt}%
\definecolor{currentstroke}{rgb}{0.121569,0.466667,0.705882}%
\pgfsetstrokecolor{currentstroke}%
\pgfsetstrokeopacity{0.827162}%
\pgfsetdash{}{0pt}%
\pgfpathmoveto{\pgfqpoint{2.282685in}{1.530186in}}%
\pgfpathcurveto{\pgfqpoint{2.290921in}{1.530186in}}{\pgfqpoint{2.298821in}{1.533458in}}{\pgfqpoint{2.304645in}{1.539282in}}%
\pgfpathcurveto{\pgfqpoint{2.310469in}{1.545106in}}{\pgfqpoint{2.313742in}{1.553006in}}{\pgfqpoint{2.313742in}{1.561243in}}%
\pgfpathcurveto{\pgfqpoint{2.313742in}{1.569479in}}{\pgfqpoint{2.310469in}{1.577379in}}{\pgfqpoint{2.304645in}{1.583203in}}%
\pgfpathcurveto{\pgfqpoint{2.298821in}{1.589027in}}{\pgfqpoint{2.290921in}{1.592299in}}{\pgfqpoint{2.282685in}{1.592299in}}%
\pgfpathcurveto{\pgfqpoint{2.274449in}{1.592299in}}{\pgfqpoint{2.266549in}{1.589027in}}{\pgfqpoint{2.260725in}{1.583203in}}%
\pgfpathcurveto{\pgfqpoint{2.254901in}{1.577379in}}{\pgfqpoint{2.251629in}{1.569479in}}{\pgfqpoint{2.251629in}{1.561243in}}%
\pgfpathcurveto{\pgfqpoint{2.251629in}{1.553006in}}{\pgfqpoint{2.254901in}{1.545106in}}{\pgfqpoint{2.260725in}{1.539282in}}%
\pgfpathcurveto{\pgfqpoint{2.266549in}{1.533458in}}{\pgfqpoint{2.274449in}{1.530186in}}{\pgfqpoint{2.282685in}{1.530186in}}%
\pgfpathclose%
\pgfusepath{stroke,fill}%
\end{pgfscope}%
\begin{pgfscope}%
\pgfpathrectangle{\pgfqpoint{0.100000in}{0.212622in}}{\pgfqpoint{3.696000in}{3.696000in}}%
\pgfusepath{clip}%
\pgfsetbuttcap%
\pgfsetroundjoin%
\definecolor{currentfill}{rgb}{0.121569,0.466667,0.705882}%
\pgfsetfillcolor{currentfill}%
\pgfsetfillopacity{0.827386}%
\pgfsetlinewidth{1.003750pt}%
\definecolor{currentstroke}{rgb}{0.121569,0.466667,0.705882}%
\pgfsetstrokecolor{currentstroke}%
\pgfsetstrokeopacity{0.827386}%
\pgfsetdash{}{0pt}%
\pgfpathmoveto{\pgfqpoint{0.603932in}{2.636418in}}%
\pgfpathcurveto{\pgfqpoint{0.612168in}{2.636418in}}{\pgfqpoint{0.620068in}{2.639690in}}{\pgfqpoint{0.625892in}{2.645514in}}%
\pgfpathcurveto{\pgfqpoint{0.631716in}{2.651338in}}{\pgfqpoint{0.634988in}{2.659238in}}{\pgfqpoint{0.634988in}{2.667474in}}%
\pgfpathcurveto{\pgfqpoint{0.634988in}{2.675711in}}{\pgfqpoint{0.631716in}{2.683611in}}{\pgfqpoint{0.625892in}{2.689435in}}%
\pgfpathcurveto{\pgfqpoint{0.620068in}{2.695258in}}{\pgfqpoint{0.612168in}{2.698531in}}{\pgfqpoint{0.603932in}{2.698531in}}%
\pgfpathcurveto{\pgfqpoint{0.595695in}{2.698531in}}{\pgfqpoint{0.587795in}{2.695258in}}{\pgfqpoint{0.581971in}{2.689435in}}%
\pgfpathcurveto{\pgfqpoint{0.576147in}{2.683611in}}{\pgfqpoint{0.572875in}{2.675711in}}{\pgfqpoint{0.572875in}{2.667474in}}%
\pgfpathcurveto{\pgfqpoint{0.572875in}{2.659238in}}{\pgfqpoint{0.576147in}{2.651338in}}{\pgfqpoint{0.581971in}{2.645514in}}%
\pgfpathcurveto{\pgfqpoint{0.587795in}{2.639690in}}{\pgfqpoint{0.595695in}{2.636418in}}{\pgfqpoint{0.603932in}{2.636418in}}%
\pgfpathclose%
\pgfusepath{stroke,fill}%
\end{pgfscope}%
\begin{pgfscope}%
\pgfpathrectangle{\pgfqpoint{0.100000in}{0.212622in}}{\pgfqpoint{3.696000in}{3.696000in}}%
\pgfusepath{clip}%
\pgfsetbuttcap%
\pgfsetroundjoin%
\definecolor{currentfill}{rgb}{0.121569,0.466667,0.705882}%
\pgfsetfillcolor{currentfill}%
\pgfsetfillopacity{0.828276}%
\pgfsetlinewidth{1.003750pt}%
\definecolor{currentstroke}{rgb}{0.121569,0.466667,0.705882}%
\pgfsetstrokecolor{currentstroke}%
\pgfsetstrokeopacity{0.828276}%
\pgfsetdash{}{0pt}%
\pgfpathmoveto{\pgfqpoint{0.669862in}{2.583763in}}%
\pgfpathcurveto{\pgfqpoint{0.678098in}{2.583763in}}{\pgfqpoint{0.685998in}{2.587035in}}{\pgfqpoint{0.691822in}{2.592859in}}%
\pgfpathcurveto{\pgfqpoint{0.697646in}{2.598683in}}{\pgfqpoint{0.700919in}{2.606583in}}{\pgfqpoint{0.700919in}{2.614819in}}%
\pgfpathcurveto{\pgfqpoint{0.700919in}{2.623056in}}{\pgfqpoint{0.697646in}{2.630956in}}{\pgfqpoint{0.691822in}{2.636780in}}%
\pgfpathcurveto{\pgfqpoint{0.685998in}{2.642603in}}{\pgfqpoint{0.678098in}{2.645876in}}{\pgfqpoint{0.669862in}{2.645876in}}%
\pgfpathcurveto{\pgfqpoint{0.661626in}{2.645876in}}{\pgfqpoint{0.653726in}{2.642603in}}{\pgfqpoint{0.647902in}{2.636780in}}%
\pgfpathcurveto{\pgfqpoint{0.642078in}{2.630956in}}{\pgfqpoint{0.638806in}{2.623056in}}{\pgfqpoint{0.638806in}{2.614819in}}%
\pgfpathcurveto{\pgfqpoint{0.638806in}{2.606583in}}{\pgfqpoint{0.642078in}{2.598683in}}{\pgfqpoint{0.647902in}{2.592859in}}%
\pgfpathcurveto{\pgfqpoint{0.653726in}{2.587035in}}{\pgfqpoint{0.661626in}{2.583763in}}{\pgfqpoint{0.669862in}{2.583763in}}%
\pgfpathclose%
\pgfusepath{stroke,fill}%
\end{pgfscope}%
\begin{pgfscope}%
\pgfpathrectangle{\pgfqpoint{0.100000in}{0.212622in}}{\pgfqpoint{3.696000in}{3.696000in}}%
\pgfusepath{clip}%
\pgfsetbuttcap%
\pgfsetroundjoin%
\definecolor{currentfill}{rgb}{0.121569,0.466667,0.705882}%
\pgfsetfillcolor{currentfill}%
\pgfsetfillopacity{0.828617}%
\pgfsetlinewidth{1.003750pt}%
\definecolor{currentstroke}{rgb}{0.121569,0.466667,0.705882}%
\pgfsetstrokecolor{currentstroke}%
\pgfsetstrokeopacity{0.828617}%
\pgfsetdash{}{0pt}%
\pgfpathmoveto{\pgfqpoint{2.283736in}{1.528782in}}%
\pgfpathcurveto{\pgfqpoint{2.291973in}{1.528782in}}{\pgfqpoint{2.299873in}{1.532054in}}{\pgfqpoint{2.305697in}{1.537878in}}%
\pgfpathcurveto{\pgfqpoint{2.311521in}{1.543702in}}{\pgfqpoint{2.314793in}{1.551602in}}{\pgfqpoint{2.314793in}{1.559838in}}%
\pgfpathcurveto{\pgfqpoint{2.314793in}{1.568074in}}{\pgfqpoint{2.311521in}{1.575974in}}{\pgfqpoint{2.305697in}{1.581798in}}%
\pgfpathcurveto{\pgfqpoint{2.299873in}{1.587622in}}{\pgfqpoint{2.291973in}{1.590895in}}{\pgfqpoint{2.283736in}{1.590895in}}%
\pgfpathcurveto{\pgfqpoint{2.275500in}{1.590895in}}{\pgfqpoint{2.267600in}{1.587622in}}{\pgfqpoint{2.261776in}{1.581798in}}%
\pgfpathcurveto{\pgfqpoint{2.255952in}{1.575974in}}{\pgfqpoint{2.252680in}{1.568074in}}{\pgfqpoint{2.252680in}{1.559838in}}%
\pgfpathcurveto{\pgfqpoint{2.252680in}{1.551602in}}{\pgfqpoint{2.255952in}{1.543702in}}{\pgfqpoint{2.261776in}{1.537878in}}%
\pgfpathcurveto{\pgfqpoint{2.267600in}{1.532054in}}{\pgfqpoint{2.275500in}{1.528782in}}{\pgfqpoint{2.283736in}{1.528782in}}%
\pgfpathclose%
\pgfusepath{stroke,fill}%
\end{pgfscope}%
\begin{pgfscope}%
\pgfpathrectangle{\pgfqpoint{0.100000in}{0.212622in}}{\pgfqpoint{3.696000in}{3.696000in}}%
\pgfusepath{clip}%
\pgfsetbuttcap%
\pgfsetroundjoin%
\definecolor{currentfill}{rgb}{0.121569,0.466667,0.705882}%
\pgfsetfillcolor{currentfill}%
\pgfsetfillopacity{0.829106}%
\pgfsetlinewidth{1.003750pt}%
\definecolor{currentstroke}{rgb}{0.121569,0.466667,0.705882}%
\pgfsetstrokecolor{currentstroke}%
\pgfsetstrokeopacity{0.829106}%
\pgfsetdash{}{0pt}%
\pgfpathmoveto{\pgfqpoint{0.682250in}{2.570331in}}%
\pgfpathcurveto{\pgfqpoint{0.690487in}{2.570331in}}{\pgfqpoint{0.698387in}{2.573603in}}{\pgfqpoint{0.704211in}{2.579427in}}%
\pgfpathcurveto{\pgfqpoint{0.710035in}{2.585251in}}{\pgfqpoint{0.713307in}{2.593151in}}{\pgfqpoint{0.713307in}{2.601387in}}%
\pgfpathcurveto{\pgfqpoint{0.713307in}{2.609623in}}{\pgfqpoint{0.710035in}{2.617524in}}{\pgfqpoint{0.704211in}{2.623347in}}%
\pgfpathcurveto{\pgfqpoint{0.698387in}{2.629171in}}{\pgfqpoint{0.690487in}{2.632444in}}{\pgfqpoint{0.682250in}{2.632444in}}%
\pgfpathcurveto{\pgfqpoint{0.674014in}{2.632444in}}{\pgfqpoint{0.666114in}{2.629171in}}{\pgfqpoint{0.660290in}{2.623347in}}%
\pgfpathcurveto{\pgfqpoint{0.654466in}{2.617524in}}{\pgfqpoint{0.651194in}{2.609623in}}{\pgfqpoint{0.651194in}{2.601387in}}%
\pgfpathcurveto{\pgfqpoint{0.651194in}{2.593151in}}{\pgfqpoint{0.654466in}{2.585251in}}{\pgfqpoint{0.660290in}{2.579427in}}%
\pgfpathcurveto{\pgfqpoint{0.666114in}{2.573603in}}{\pgfqpoint{0.674014in}{2.570331in}}{\pgfqpoint{0.682250in}{2.570331in}}%
\pgfpathclose%
\pgfusepath{stroke,fill}%
\end{pgfscope}%
\begin{pgfscope}%
\pgfpathrectangle{\pgfqpoint{0.100000in}{0.212622in}}{\pgfqpoint{3.696000in}{3.696000in}}%
\pgfusepath{clip}%
\pgfsetbuttcap%
\pgfsetroundjoin%
\definecolor{currentfill}{rgb}{0.121569,0.466667,0.705882}%
\pgfsetfillcolor{currentfill}%
\pgfsetfillopacity{0.830864}%
\pgfsetlinewidth{1.003750pt}%
\definecolor{currentstroke}{rgb}{0.121569,0.466667,0.705882}%
\pgfsetstrokecolor{currentstroke}%
\pgfsetstrokeopacity{0.830864}%
\pgfsetdash{}{0pt}%
\pgfpathmoveto{\pgfqpoint{2.284900in}{1.526930in}}%
\pgfpathcurveto{\pgfqpoint{2.293136in}{1.526930in}}{\pgfqpoint{2.301036in}{1.530203in}}{\pgfqpoint{2.306860in}{1.536027in}}%
\pgfpathcurveto{\pgfqpoint{2.312684in}{1.541850in}}{\pgfqpoint{2.315956in}{1.549750in}}{\pgfqpoint{2.315956in}{1.557987in}}%
\pgfpathcurveto{\pgfqpoint{2.315956in}{1.566223in}}{\pgfqpoint{2.312684in}{1.574123in}}{\pgfqpoint{2.306860in}{1.579947in}}%
\pgfpathcurveto{\pgfqpoint{2.301036in}{1.585771in}}{\pgfqpoint{2.293136in}{1.589043in}}{\pgfqpoint{2.284900in}{1.589043in}}%
\pgfpathcurveto{\pgfqpoint{2.276663in}{1.589043in}}{\pgfqpoint{2.268763in}{1.585771in}}{\pgfqpoint{2.262939in}{1.579947in}}%
\pgfpathcurveto{\pgfqpoint{2.257115in}{1.574123in}}{\pgfqpoint{2.253843in}{1.566223in}}{\pgfqpoint{2.253843in}{1.557987in}}%
\pgfpathcurveto{\pgfqpoint{2.253843in}{1.549750in}}{\pgfqpoint{2.257115in}{1.541850in}}{\pgfqpoint{2.262939in}{1.536027in}}%
\pgfpathcurveto{\pgfqpoint{2.268763in}{1.530203in}}{\pgfqpoint{2.276663in}{1.526930in}}{\pgfqpoint{2.284900in}{1.526930in}}%
\pgfpathclose%
\pgfusepath{stroke,fill}%
\end{pgfscope}%
\begin{pgfscope}%
\pgfpathrectangle{\pgfqpoint{0.100000in}{0.212622in}}{\pgfqpoint{3.696000in}{3.696000in}}%
\pgfusepath{clip}%
\pgfsetbuttcap%
\pgfsetroundjoin%
\definecolor{currentfill}{rgb}{0.121569,0.466667,0.705882}%
\pgfsetfillcolor{currentfill}%
\pgfsetfillopacity{0.833497}%
\pgfsetlinewidth{1.003750pt}%
\definecolor{currentstroke}{rgb}{0.121569,0.466667,0.705882}%
\pgfsetstrokecolor{currentstroke}%
\pgfsetstrokeopacity{0.833497}%
\pgfsetdash{}{0pt}%
\pgfpathmoveto{\pgfqpoint{2.287026in}{1.524320in}}%
\pgfpathcurveto{\pgfqpoint{2.295262in}{1.524320in}}{\pgfqpoint{2.303162in}{1.527593in}}{\pgfqpoint{2.308986in}{1.533417in}}%
\pgfpathcurveto{\pgfqpoint{2.314810in}{1.539240in}}{\pgfqpoint{2.318082in}{1.547141in}}{\pgfqpoint{2.318082in}{1.555377in}}%
\pgfpathcurveto{\pgfqpoint{2.318082in}{1.563613in}}{\pgfqpoint{2.314810in}{1.571513in}}{\pgfqpoint{2.308986in}{1.577337in}}%
\pgfpathcurveto{\pgfqpoint{2.303162in}{1.583161in}}{\pgfqpoint{2.295262in}{1.586433in}}{\pgfqpoint{2.287026in}{1.586433in}}%
\pgfpathcurveto{\pgfqpoint{2.278789in}{1.586433in}}{\pgfqpoint{2.270889in}{1.583161in}}{\pgfqpoint{2.265065in}{1.577337in}}%
\pgfpathcurveto{\pgfqpoint{2.259241in}{1.571513in}}{\pgfqpoint{2.255969in}{1.563613in}}{\pgfqpoint{2.255969in}{1.555377in}}%
\pgfpathcurveto{\pgfqpoint{2.255969in}{1.547141in}}{\pgfqpoint{2.259241in}{1.539240in}}{\pgfqpoint{2.265065in}{1.533417in}}%
\pgfpathcurveto{\pgfqpoint{2.270889in}{1.527593in}}{\pgfqpoint{2.278789in}{1.524320in}}{\pgfqpoint{2.287026in}{1.524320in}}%
\pgfpathclose%
\pgfusepath{stroke,fill}%
\end{pgfscope}%
\begin{pgfscope}%
\pgfpathrectangle{\pgfqpoint{0.100000in}{0.212622in}}{\pgfqpoint{3.696000in}{3.696000in}}%
\pgfusepath{clip}%
\pgfsetbuttcap%
\pgfsetroundjoin%
\definecolor{currentfill}{rgb}{0.121569,0.466667,0.705882}%
\pgfsetfillcolor{currentfill}%
\pgfsetfillopacity{0.833741}%
\pgfsetlinewidth{1.003750pt}%
\definecolor{currentstroke}{rgb}{0.121569,0.466667,0.705882}%
\pgfsetstrokecolor{currentstroke}%
\pgfsetstrokeopacity{0.833741}%
\pgfsetdash{}{0pt}%
\pgfpathmoveto{\pgfqpoint{0.703063in}{2.563342in}}%
\pgfpathcurveto{\pgfqpoint{0.711299in}{2.563342in}}{\pgfqpoint{0.719199in}{2.566614in}}{\pgfqpoint{0.725023in}{2.572438in}}%
\pgfpathcurveto{\pgfqpoint{0.730847in}{2.578262in}}{\pgfqpoint{0.734120in}{2.586162in}}{\pgfqpoint{0.734120in}{2.594398in}}%
\pgfpathcurveto{\pgfqpoint{0.734120in}{2.602634in}}{\pgfqpoint{0.730847in}{2.610535in}}{\pgfqpoint{0.725023in}{2.616358in}}%
\pgfpathcurveto{\pgfqpoint{0.719199in}{2.622182in}}{\pgfqpoint{0.711299in}{2.625455in}}{\pgfqpoint{0.703063in}{2.625455in}}%
\pgfpathcurveto{\pgfqpoint{0.694827in}{2.625455in}}{\pgfqpoint{0.686927in}{2.622182in}}{\pgfqpoint{0.681103in}{2.616358in}}%
\pgfpathcurveto{\pgfqpoint{0.675279in}{2.610535in}}{\pgfqpoint{0.672007in}{2.602634in}}{\pgfqpoint{0.672007in}{2.594398in}}%
\pgfpathcurveto{\pgfqpoint{0.672007in}{2.586162in}}{\pgfqpoint{0.675279in}{2.578262in}}{\pgfqpoint{0.681103in}{2.572438in}}%
\pgfpathcurveto{\pgfqpoint{0.686927in}{2.566614in}}{\pgfqpoint{0.694827in}{2.563342in}}{\pgfqpoint{0.703063in}{2.563342in}}%
\pgfpathclose%
\pgfusepath{stroke,fill}%
\end{pgfscope}%
\begin{pgfscope}%
\pgfpathrectangle{\pgfqpoint{0.100000in}{0.212622in}}{\pgfqpoint{3.696000in}{3.696000in}}%
\pgfusepath{clip}%
\pgfsetbuttcap%
\pgfsetroundjoin%
\definecolor{currentfill}{rgb}{0.121569,0.466667,0.705882}%
\pgfsetfillcolor{currentfill}%
\pgfsetfillopacity{0.834912}%
\pgfsetlinewidth{1.003750pt}%
\definecolor{currentstroke}{rgb}{0.121569,0.466667,0.705882}%
\pgfsetstrokecolor{currentstroke}%
\pgfsetstrokeopacity{0.834912}%
\pgfsetdash{}{0pt}%
\pgfpathmoveto{\pgfqpoint{2.287872in}{1.522497in}}%
\pgfpathcurveto{\pgfqpoint{2.296108in}{1.522497in}}{\pgfqpoint{2.304008in}{1.525769in}}{\pgfqpoint{2.309832in}{1.531593in}}%
\pgfpathcurveto{\pgfqpoint{2.315656in}{1.537417in}}{\pgfqpoint{2.318928in}{1.545317in}}{\pgfqpoint{2.318928in}{1.553553in}}%
\pgfpathcurveto{\pgfqpoint{2.318928in}{1.561789in}}{\pgfqpoint{2.315656in}{1.569690in}}{\pgfqpoint{2.309832in}{1.575513in}}%
\pgfpathcurveto{\pgfqpoint{2.304008in}{1.581337in}}{\pgfqpoint{2.296108in}{1.584610in}}{\pgfqpoint{2.287872in}{1.584610in}}%
\pgfpathcurveto{\pgfqpoint{2.279635in}{1.584610in}}{\pgfqpoint{2.271735in}{1.581337in}}{\pgfqpoint{2.265911in}{1.575513in}}%
\pgfpathcurveto{\pgfqpoint{2.260088in}{1.569690in}}{\pgfqpoint{2.256815in}{1.561789in}}{\pgfqpoint{2.256815in}{1.553553in}}%
\pgfpathcurveto{\pgfqpoint{2.256815in}{1.545317in}}{\pgfqpoint{2.260088in}{1.537417in}}{\pgfqpoint{2.265911in}{1.531593in}}%
\pgfpathcurveto{\pgfqpoint{2.271735in}{1.525769in}}{\pgfqpoint{2.279635in}{1.522497in}}{\pgfqpoint{2.287872in}{1.522497in}}%
\pgfpathclose%
\pgfusepath{stroke,fill}%
\end{pgfscope}%
\begin{pgfscope}%
\pgfpathrectangle{\pgfqpoint{0.100000in}{0.212622in}}{\pgfqpoint{3.696000in}{3.696000in}}%
\pgfusepath{clip}%
\pgfsetbuttcap%
\pgfsetroundjoin%
\definecolor{currentfill}{rgb}{0.121569,0.466667,0.705882}%
\pgfsetfillcolor{currentfill}%
\pgfsetfillopacity{0.836434}%
\pgfsetlinewidth{1.003750pt}%
\definecolor{currentstroke}{rgb}{0.121569,0.466667,0.705882}%
\pgfsetstrokecolor{currentstroke}%
\pgfsetstrokeopacity{0.836434}%
\pgfsetdash{}{0pt}%
\pgfpathmoveto{\pgfqpoint{2.288891in}{1.518716in}}%
\pgfpathcurveto{\pgfqpoint{2.297127in}{1.518716in}}{\pgfqpoint{2.305027in}{1.521989in}}{\pgfqpoint{2.310851in}{1.527813in}}%
\pgfpathcurveto{\pgfqpoint{2.316675in}{1.533637in}}{\pgfqpoint{2.319947in}{1.541537in}}{\pgfqpoint{2.319947in}{1.549773in}}%
\pgfpathcurveto{\pgfqpoint{2.319947in}{1.558009in}}{\pgfqpoint{2.316675in}{1.565909in}}{\pgfqpoint{2.310851in}{1.571733in}}%
\pgfpathcurveto{\pgfqpoint{2.305027in}{1.577557in}}{\pgfqpoint{2.297127in}{1.580829in}}{\pgfqpoint{2.288891in}{1.580829in}}%
\pgfpathcurveto{\pgfqpoint{2.280655in}{1.580829in}}{\pgfqpoint{2.272755in}{1.577557in}}{\pgfqpoint{2.266931in}{1.571733in}}%
\pgfpathcurveto{\pgfqpoint{2.261107in}{1.565909in}}{\pgfqpoint{2.257834in}{1.558009in}}{\pgfqpoint{2.257834in}{1.549773in}}%
\pgfpathcurveto{\pgfqpoint{2.257834in}{1.541537in}}{\pgfqpoint{2.261107in}{1.533637in}}{\pgfqpoint{2.266931in}{1.527813in}}%
\pgfpathcurveto{\pgfqpoint{2.272755in}{1.521989in}}{\pgfqpoint{2.280655in}{1.518716in}}{\pgfqpoint{2.288891in}{1.518716in}}%
\pgfpathclose%
\pgfusepath{stroke,fill}%
\end{pgfscope}%
\begin{pgfscope}%
\pgfpathrectangle{\pgfqpoint{0.100000in}{0.212622in}}{\pgfqpoint{3.696000in}{3.696000in}}%
\pgfusepath{clip}%
\pgfsetbuttcap%
\pgfsetroundjoin%
\definecolor{currentfill}{rgb}{0.121569,0.466667,0.705882}%
\pgfsetfillcolor{currentfill}%
\pgfsetfillopacity{0.836693}%
\pgfsetlinewidth{1.003750pt}%
\definecolor{currentstroke}{rgb}{0.121569,0.466667,0.705882}%
\pgfsetstrokecolor{currentstroke}%
\pgfsetstrokeopacity{0.836693}%
\pgfsetdash{}{0pt}%
\pgfpathmoveto{\pgfqpoint{0.744444in}{2.521302in}}%
\pgfpathcurveto{\pgfqpoint{0.752681in}{2.521302in}}{\pgfqpoint{0.760581in}{2.524575in}}{\pgfqpoint{0.766404in}{2.530398in}}%
\pgfpathcurveto{\pgfqpoint{0.772228in}{2.536222in}}{\pgfqpoint{0.775501in}{2.544122in}}{\pgfqpoint{0.775501in}{2.552359in}}%
\pgfpathcurveto{\pgfqpoint{0.775501in}{2.560595in}}{\pgfqpoint{0.772228in}{2.568495in}}{\pgfqpoint{0.766404in}{2.574319in}}%
\pgfpathcurveto{\pgfqpoint{0.760581in}{2.580143in}}{\pgfqpoint{0.752681in}{2.583415in}}{\pgfqpoint{0.744444in}{2.583415in}}%
\pgfpathcurveto{\pgfqpoint{0.736208in}{2.583415in}}{\pgfqpoint{0.728308in}{2.580143in}}{\pgfqpoint{0.722484in}{2.574319in}}%
\pgfpathcurveto{\pgfqpoint{0.716660in}{2.568495in}}{\pgfqpoint{0.713388in}{2.560595in}}{\pgfqpoint{0.713388in}{2.552359in}}%
\pgfpathcurveto{\pgfqpoint{0.713388in}{2.544122in}}{\pgfqpoint{0.716660in}{2.536222in}}{\pgfqpoint{0.722484in}{2.530398in}}%
\pgfpathcurveto{\pgfqpoint{0.728308in}{2.524575in}}{\pgfqpoint{0.736208in}{2.521302in}}{\pgfqpoint{0.744444in}{2.521302in}}%
\pgfpathclose%
\pgfusepath{stroke,fill}%
\end{pgfscope}%
\begin{pgfscope}%
\pgfpathrectangle{\pgfqpoint{0.100000in}{0.212622in}}{\pgfqpoint{3.696000in}{3.696000in}}%
\pgfusepath{clip}%
\pgfsetbuttcap%
\pgfsetroundjoin%
\definecolor{currentfill}{rgb}{0.121569,0.466667,0.705882}%
\pgfsetfillcolor{currentfill}%
\pgfsetfillopacity{0.837571}%
\pgfsetlinewidth{1.003750pt}%
\definecolor{currentstroke}{rgb}{0.121569,0.466667,0.705882}%
\pgfsetstrokecolor{currentstroke}%
\pgfsetstrokeopacity{0.837571}%
\pgfsetdash{}{0pt}%
\pgfpathmoveto{\pgfqpoint{2.289574in}{1.518557in}}%
\pgfpathcurveto{\pgfqpoint{2.297810in}{1.518557in}}{\pgfqpoint{2.305710in}{1.521829in}}{\pgfqpoint{2.311534in}{1.527653in}}%
\pgfpathcurveto{\pgfqpoint{2.317358in}{1.533477in}}{\pgfqpoint{2.320631in}{1.541377in}}{\pgfqpoint{2.320631in}{1.549614in}}%
\pgfpathcurveto{\pgfqpoint{2.320631in}{1.557850in}}{\pgfqpoint{2.317358in}{1.565750in}}{\pgfqpoint{2.311534in}{1.571574in}}%
\pgfpathcurveto{\pgfqpoint{2.305710in}{1.577398in}}{\pgfqpoint{2.297810in}{1.580670in}}{\pgfqpoint{2.289574in}{1.580670in}}%
\pgfpathcurveto{\pgfqpoint{2.281338in}{1.580670in}}{\pgfqpoint{2.273438in}{1.577398in}}{\pgfqpoint{2.267614in}{1.571574in}}%
\pgfpathcurveto{\pgfqpoint{2.261790in}{1.565750in}}{\pgfqpoint{2.258518in}{1.557850in}}{\pgfqpoint{2.258518in}{1.549614in}}%
\pgfpathcurveto{\pgfqpoint{2.258518in}{1.541377in}}{\pgfqpoint{2.261790in}{1.533477in}}{\pgfqpoint{2.267614in}{1.527653in}}%
\pgfpathcurveto{\pgfqpoint{2.273438in}{1.521829in}}{\pgfqpoint{2.281338in}{1.518557in}}{\pgfqpoint{2.289574in}{1.518557in}}%
\pgfpathclose%
\pgfusepath{stroke,fill}%
\end{pgfscope}%
\begin{pgfscope}%
\pgfpathrectangle{\pgfqpoint{0.100000in}{0.212622in}}{\pgfqpoint{3.696000in}{3.696000in}}%
\pgfusepath{clip}%
\pgfsetbuttcap%
\pgfsetroundjoin%
\definecolor{currentfill}{rgb}{0.121569,0.466667,0.705882}%
\pgfsetfillcolor{currentfill}%
\pgfsetfillopacity{0.838813}%
\pgfsetlinewidth{1.003750pt}%
\definecolor{currentstroke}{rgb}{0.121569,0.466667,0.705882}%
\pgfsetstrokecolor{currentstroke}%
\pgfsetstrokeopacity{0.838813}%
\pgfsetdash{}{0pt}%
\pgfpathmoveto{\pgfqpoint{2.290350in}{1.517446in}}%
\pgfpathcurveto{\pgfqpoint{2.298586in}{1.517446in}}{\pgfqpoint{2.306486in}{1.520718in}}{\pgfqpoint{2.312310in}{1.526542in}}%
\pgfpathcurveto{\pgfqpoint{2.318134in}{1.532366in}}{\pgfqpoint{2.321406in}{1.540266in}}{\pgfqpoint{2.321406in}{1.548503in}}%
\pgfpathcurveto{\pgfqpoint{2.321406in}{1.556739in}}{\pgfqpoint{2.318134in}{1.564639in}}{\pgfqpoint{2.312310in}{1.570463in}}%
\pgfpathcurveto{\pgfqpoint{2.306486in}{1.576287in}}{\pgfqpoint{2.298586in}{1.579559in}}{\pgfqpoint{2.290350in}{1.579559in}}%
\pgfpathcurveto{\pgfqpoint{2.282113in}{1.579559in}}{\pgfqpoint{2.274213in}{1.576287in}}{\pgfqpoint{2.268389in}{1.570463in}}%
\pgfpathcurveto{\pgfqpoint{2.262565in}{1.564639in}}{\pgfqpoint{2.259293in}{1.556739in}}{\pgfqpoint{2.259293in}{1.548503in}}%
\pgfpathcurveto{\pgfqpoint{2.259293in}{1.540266in}}{\pgfqpoint{2.262565in}{1.532366in}}{\pgfqpoint{2.268389in}{1.526542in}}%
\pgfpathcurveto{\pgfqpoint{2.274213in}{1.520718in}}{\pgfqpoint{2.282113in}{1.517446in}}{\pgfqpoint{2.290350in}{1.517446in}}%
\pgfpathclose%
\pgfusepath{stroke,fill}%
\end{pgfscope}%
\begin{pgfscope}%
\pgfpathrectangle{\pgfqpoint{0.100000in}{0.212622in}}{\pgfqpoint{3.696000in}{3.696000in}}%
\pgfusepath{clip}%
\pgfsetbuttcap%
\pgfsetroundjoin%
\definecolor{currentfill}{rgb}{0.121569,0.466667,0.705882}%
\pgfsetfillcolor{currentfill}%
\pgfsetfillopacity{0.838889}%
\pgfsetlinewidth{1.003750pt}%
\definecolor{currentstroke}{rgb}{0.121569,0.466667,0.705882}%
\pgfsetstrokecolor{currentstroke}%
\pgfsetstrokeopacity{0.838889}%
\pgfsetdash{}{0pt}%
\pgfpathmoveto{\pgfqpoint{0.785957in}{2.476203in}}%
\pgfpathcurveto{\pgfqpoint{0.794193in}{2.476203in}}{\pgfqpoint{0.802093in}{2.479475in}}{\pgfqpoint{0.807917in}{2.485299in}}%
\pgfpathcurveto{\pgfqpoint{0.813741in}{2.491123in}}{\pgfqpoint{0.817013in}{2.499023in}}{\pgfqpoint{0.817013in}{2.507259in}}%
\pgfpathcurveto{\pgfqpoint{0.817013in}{2.515496in}}{\pgfqpoint{0.813741in}{2.523396in}}{\pgfqpoint{0.807917in}{2.529220in}}%
\pgfpathcurveto{\pgfqpoint{0.802093in}{2.535044in}}{\pgfqpoint{0.794193in}{2.538316in}}{\pgfqpoint{0.785957in}{2.538316in}}%
\pgfpathcurveto{\pgfqpoint{0.777720in}{2.538316in}}{\pgfqpoint{0.769820in}{2.535044in}}{\pgfqpoint{0.763996in}{2.529220in}}%
\pgfpathcurveto{\pgfqpoint{0.758172in}{2.523396in}}{\pgfqpoint{0.754900in}{2.515496in}}{\pgfqpoint{0.754900in}{2.507259in}}%
\pgfpathcurveto{\pgfqpoint{0.754900in}{2.499023in}}{\pgfqpoint{0.758172in}{2.491123in}}{\pgfqpoint{0.763996in}{2.485299in}}%
\pgfpathcurveto{\pgfqpoint{0.769820in}{2.479475in}}{\pgfqpoint{0.777720in}{2.476203in}}{\pgfqpoint{0.785957in}{2.476203in}}%
\pgfpathclose%
\pgfusepath{stroke,fill}%
\end{pgfscope}%
\begin{pgfscope}%
\pgfpathrectangle{\pgfqpoint{0.100000in}{0.212622in}}{\pgfqpoint{3.696000in}{3.696000in}}%
\pgfusepath{clip}%
\pgfsetbuttcap%
\pgfsetroundjoin%
\definecolor{currentfill}{rgb}{0.121569,0.466667,0.705882}%
\pgfsetfillcolor{currentfill}%
\pgfsetfillopacity{0.839402}%
\pgfsetlinewidth{1.003750pt}%
\definecolor{currentstroke}{rgb}{0.121569,0.466667,0.705882}%
\pgfsetstrokecolor{currentstroke}%
\pgfsetstrokeopacity{0.839402}%
\pgfsetdash{}{0pt}%
\pgfpathmoveto{\pgfqpoint{2.290615in}{1.516181in}}%
\pgfpathcurveto{\pgfqpoint{2.298851in}{1.516181in}}{\pgfqpoint{2.306751in}{1.519453in}}{\pgfqpoint{2.312575in}{1.525277in}}%
\pgfpathcurveto{\pgfqpoint{2.318399in}{1.531101in}}{\pgfqpoint{2.321671in}{1.539001in}}{\pgfqpoint{2.321671in}{1.547237in}}%
\pgfpathcurveto{\pgfqpoint{2.321671in}{1.555474in}}{\pgfqpoint{2.318399in}{1.563374in}}{\pgfqpoint{2.312575in}{1.569198in}}%
\pgfpathcurveto{\pgfqpoint{2.306751in}{1.575022in}}{\pgfqpoint{2.298851in}{1.578294in}}{\pgfqpoint{2.290615in}{1.578294in}}%
\pgfpathcurveto{\pgfqpoint{2.282379in}{1.578294in}}{\pgfqpoint{2.274479in}{1.575022in}}{\pgfqpoint{2.268655in}{1.569198in}}%
\pgfpathcurveto{\pgfqpoint{2.262831in}{1.563374in}}{\pgfqpoint{2.259558in}{1.555474in}}{\pgfqpoint{2.259558in}{1.547237in}}%
\pgfpathcurveto{\pgfqpoint{2.259558in}{1.539001in}}{\pgfqpoint{2.262831in}{1.531101in}}{\pgfqpoint{2.268655in}{1.525277in}}%
\pgfpathcurveto{\pgfqpoint{2.274479in}{1.519453in}}{\pgfqpoint{2.282379in}{1.516181in}}{\pgfqpoint{2.290615in}{1.516181in}}%
\pgfpathclose%
\pgfusepath{stroke,fill}%
\end{pgfscope}%
\begin{pgfscope}%
\pgfpathrectangle{\pgfqpoint{0.100000in}{0.212622in}}{\pgfqpoint{3.696000in}{3.696000in}}%
\pgfusepath{clip}%
\pgfsetbuttcap%
\pgfsetroundjoin%
\definecolor{currentfill}{rgb}{0.121569,0.466667,0.705882}%
\pgfsetfillcolor{currentfill}%
\pgfsetfillopacity{0.840074}%
\pgfsetlinewidth{1.003750pt}%
\definecolor{currentstroke}{rgb}{0.121569,0.466667,0.705882}%
\pgfsetstrokecolor{currentstroke}%
\pgfsetstrokeopacity{0.840074}%
\pgfsetdash{}{0pt}%
\pgfpathmoveto{\pgfqpoint{2.291312in}{1.513514in}}%
\pgfpathcurveto{\pgfqpoint{2.299548in}{1.513514in}}{\pgfqpoint{2.307448in}{1.516787in}}{\pgfqpoint{2.313272in}{1.522611in}}%
\pgfpathcurveto{\pgfqpoint{2.319096in}{1.528434in}}{\pgfqpoint{2.322368in}{1.536335in}}{\pgfqpoint{2.322368in}{1.544571in}}%
\pgfpathcurveto{\pgfqpoint{2.322368in}{1.552807in}}{\pgfqpoint{2.319096in}{1.560707in}}{\pgfqpoint{2.313272in}{1.566531in}}%
\pgfpathcurveto{\pgfqpoint{2.307448in}{1.572355in}}{\pgfqpoint{2.299548in}{1.575627in}}{\pgfqpoint{2.291312in}{1.575627in}}%
\pgfpathcurveto{\pgfqpoint{2.283076in}{1.575627in}}{\pgfqpoint{2.275175in}{1.572355in}}{\pgfqpoint{2.269352in}{1.566531in}}%
\pgfpathcurveto{\pgfqpoint{2.263528in}{1.560707in}}{\pgfqpoint{2.260255in}{1.552807in}}{\pgfqpoint{2.260255in}{1.544571in}}%
\pgfpathcurveto{\pgfqpoint{2.260255in}{1.536335in}}{\pgfqpoint{2.263528in}{1.528434in}}{\pgfqpoint{2.269352in}{1.522611in}}%
\pgfpathcurveto{\pgfqpoint{2.275175in}{1.516787in}}{\pgfqpoint{2.283076in}{1.513514in}}{\pgfqpoint{2.291312in}{1.513514in}}%
\pgfpathclose%
\pgfusepath{stroke,fill}%
\end{pgfscope}%
\begin{pgfscope}%
\pgfpathrectangle{\pgfqpoint{0.100000in}{0.212622in}}{\pgfqpoint{3.696000in}{3.696000in}}%
\pgfusepath{clip}%
\pgfsetbuttcap%
\pgfsetroundjoin%
\definecolor{currentfill}{rgb}{0.121569,0.466667,0.705882}%
\pgfsetfillcolor{currentfill}%
\pgfsetfillopacity{0.841581}%
\pgfsetlinewidth{1.003750pt}%
\definecolor{currentstroke}{rgb}{0.121569,0.466667,0.705882}%
\pgfsetstrokecolor{currentstroke}%
\pgfsetstrokeopacity{0.841581}%
\pgfsetdash{}{0pt}%
\pgfpathmoveto{\pgfqpoint{2.292506in}{1.513516in}}%
\pgfpathcurveto{\pgfqpoint{2.300743in}{1.513516in}}{\pgfqpoint{2.308643in}{1.516788in}}{\pgfqpoint{2.314467in}{1.522612in}}%
\pgfpathcurveto{\pgfqpoint{2.320290in}{1.528436in}}{\pgfqpoint{2.323563in}{1.536336in}}{\pgfqpoint{2.323563in}{1.544572in}}%
\pgfpathcurveto{\pgfqpoint{2.323563in}{1.552808in}}{\pgfqpoint{2.320290in}{1.560708in}}{\pgfqpoint{2.314467in}{1.566532in}}%
\pgfpathcurveto{\pgfqpoint{2.308643in}{1.572356in}}{\pgfqpoint{2.300743in}{1.575629in}}{\pgfqpoint{2.292506in}{1.575629in}}%
\pgfpathcurveto{\pgfqpoint{2.284270in}{1.575629in}}{\pgfqpoint{2.276370in}{1.572356in}}{\pgfqpoint{2.270546in}{1.566532in}}%
\pgfpathcurveto{\pgfqpoint{2.264722in}{1.560708in}}{\pgfqpoint{2.261450in}{1.552808in}}{\pgfqpoint{2.261450in}{1.544572in}}%
\pgfpathcurveto{\pgfqpoint{2.261450in}{1.536336in}}{\pgfqpoint{2.264722in}{1.528436in}}{\pgfqpoint{2.270546in}{1.522612in}}%
\pgfpathcurveto{\pgfqpoint{2.276370in}{1.516788in}}{\pgfqpoint{2.284270in}{1.513516in}}{\pgfqpoint{2.292506in}{1.513516in}}%
\pgfpathclose%
\pgfusepath{stroke,fill}%
\end{pgfscope}%
\begin{pgfscope}%
\pgfpathrectangle{\pgfqpoint{0.100000in}{0.212622in}}{\pgfqpoint{3.696000in}{3.696000in}}%
\pgfusepath{clip}%
\pgfsetbuttcap%
\pgfsetroundjoin%
\definecolor{currentfill}{rgb}{0.121569,0.466667,0.705882}%
\pgfsetfillcolor{currentfill}%
\pgfsetfillopacity{0.843849}%
\pgfsetlinewidth{1.003750pt}%
\definecolor{currentstroke}{rgb}{0.121569,0.466667,0.705882}%
\pgfsetstrokecolor{currentstroke}%
\pgfsetstrokeopacity{0.843849}%
\pgfsetdash{}{0pt}%
\pgfpathmoveto{\pgfqpoint{0.824082in}{2.444487in}}%
\pgfpathcurveto{\pgfqpoint{0.832318in}{2.444487in}}{\pgfqpoint{0.840218in}{2.447759in}}{\pgfqpoint{0.846042in}{2.453583in}}%
\pgfpathcurveto{\pgfqpoint{0.851866in}{2.459407in}}{\pgfqpoint{0.855138in}{2.467307in}}{\pgfqpoint{0.855138in}{2.475544in}}%
\pgfpathcurveto{\pgfqpoint{0.855138in}{2.483780in}}{\pgfqpoint{0.851866in}{2.491680in}}{\pgfqpoint{0.846042in}{2.497504in}}%
\pgfpathcurveto{\pgfqpoint{0.840218in}{2.503328in}}{\pgfqpoint{0.832318in}{2.506600in}}{\pgfqpoint{0.824082in}{2.506600in}}%
\pgfpathcurveto{\pgfqpoint{0.815846in}{2.506600in}}{\pgfqpoint{0.807946in}{2.503328in}}{\pgfqpoint{0.802122in}{2.497504in}}%
\pgfpathcurveto{\pgfqpoint{0.796298in}{2.491680in}}{\pgfqpoint{0.793025in}{2.483780in}}{\pgfqpoint{0.793025in}{2.475544in}}%
\pgfpathcurveto{\pgfqpoint{0.793025in}{2.467307in}}{\pgfqpoint{0.796298in}{2.459407in}}{\pgfqpoint{0.802122in}{2.453583in}}%
\pgfpathcurveto{\pgfqpoint{0.807946in}{2.447759in}}{\pgfqpoint{0.815846in}{2.444487in}}{\pgfqpoint{0.824082in}{2.444487in}}%
\pgfpathclose%
\pgfusepath{stroke,fill}%
\end{pgfscope}%
\begin{pgfscope}%
\pgfpathrectangle{\pgfqpoint{0.100000in}{0.212622in}}{\pgfqpoint{3.696000in}{3.696000in}}%
\pgfusepath{clip}%
\pgfsetbuttcap%
\pgfsetroundjoin%
\definecolor{currentfill}{rgb}{0.121569,0.466667,0.705882}%
\pgfsetfillcolor{currentfill}%
\pgfsetfillopacity{0.843932}%
\pgfsetlinewidth{1.003750pt}%
\definecolor{currentstroke}{rgb}{0.121569,0.466667,0.705882}%
\pgfsetstrokecolor{currentstroke}%
\pgfsetstrokeopacity{0.843932}%
\pgfsetdash{}{0pt}%
\pgfpathmoveto{\pgfqpoint{2.294080in}{1.512759in}}%
\pgfpathcurveto{\pgfqpoint{2.302316in}{1.512759in}}{\pgfqpoint{2.310216in}{1.516031in}}{\pgfqpoint{2.316040in}{1.521855in}}%
\pgfpathcurveto{\pgfqpoint{2.321864in}{1.527679in}}{\pgfqpoint{2.325137in}{1.535579in}}{\pgfqpoint{2.325137in}{1.543816in}}%
\pgfpathcurveto{\pgfqpoint{2.325137in}{1.552052in}}{\pgfqpoint{2.321864in}{1.559952in}}{\pgfqpoint{2.316040in}{1.565776in}}%
\pgfpathcurveto{\pgfqpoint{2.310216in}{1.571600in}}{\pgfqpoint{2.302316in}{1.574872in}}{\pgfqpoint{2.294080in}{1.574872in}}%
\pgfpathcurveto{\pgfqpoint{2.285844in}{1.574872in}}{\pgfqpoint{2.277944in}{1.571600in}}{\pgfqpoint{2.272120in}{1.565776in}}%
\pgfpathcurveto{\pgfqpoint{2.266296in}{1.559952in}}{\pgfqpoint{2.263024in}{1.552052in}}{\pgfqpoint{2.263024in}{1.543816in}}%
\pgfpathcurveto{\pgfqpoint{2.263024in}{1.535579in}}{\pgfqpoint{2.266296in}{1.527679in}}{\pgfqpoint{2.272120in}{1.521855in}}%
\pgfpathcurveto{\pgfqpoint{2.277944in}{1.516031in}}{\pgfqpoint{2.285844in}{1.512759in}}{\pgfqpoint{2.294080in}{1.512759in}}%
\pgfpathclose%
\pgfusepath{stroke,fill}%
\end{pgfscope}%
\begin{pgfscope}%
\pgfpathrectangle{\pgfqpoint{0.100000in}{0.212622in}}{\pgfqpoint{3.696000in}{3.696000in}}%
\pgfusepath{clip}%
\pgfsetbuttcap%
\pgfsetroundjoin%
\definecolor{currentfill}{rgb}{0.121569,0.466667,0.705882}%
\pgfsetfillcolor{currentfill}%
\pgfsetfillopacity{0.846256}%
\pgfsetlinewidth{1.003750pt}%
\definecolor{currentstroke}{rgb}{0.121569,0.466667,0.705882}%
\pgfsetstrokecolor{currentstroke}%
\pgfsetstrokeopacity{0.846256}%
\pgfsetdash{}{0pt}%
\pgfpathmoveto{\pgfqpoint{2.294766in}{1.509482in}}%
\pgfpathcurveto{\pgfqpoint{2.303002in}{1.509482in}}{\pgfqpoint{2.310902in}{1.512754in}}{\pgfqpoint{2.316726in}{1.518578in}}%
\pgfpathcurveto{\pgfqpoint{2.322550in}{1.524402in}}{\pgfqpoint{2.325823in}{1.532302in}}{\pgfqpoint{2.325823in}{1.540539in}}%
\pgfpathcurveto{\pgfqpoint{2.325823in}{1.548775in}}{\pgfqpoint{2.322550in}{1.556675in}}{\pgfqpoint{2.316726in}{1.562499in}}%
\pgfpathcurveto{\pgfqpoint{2.310902in}{1.568323in}}{\pgfqpoint{2.303002in}{1.571595in}}{\pgfqpoint{2.294766in}{1.571595in}}%
\pgfpathcurveto{\pgfqpoint{2.286530in}{1.571595in}}{\pgfqpoint{2.278630in}{1.568323in}}{\pgfqpoint{2.272806in}{1.562499in}}%
\pgfpathcurveto{\pgfqpoint{2.266982in}{1.556675in}}{\pgfqpoint{2.263710in}{1.548775in}}{\pgfqpoint{2.263710in}{1.540539in}}%
\pgfpathcurveto{\pgfqpoint{2.263710in}{1.532302in}}{\pgfqpoint{2.266982in}{1.524402in}}{\pgfqpoint{2.272806in}{1.518578in}}%
\pgfpathcurveto{\pgfqpoint{2.278630in}{1.512754in}}{\pgfqpoint{2.286530in}{1.509482in}}{\pgfqpoint{2.294766in}{1.509482in}}%
\pgfpathclose%
\pgfusepath{stroke,fill}%
\end{pgfscope}%
\begin{pgfscope}%
\pgfpathrectangle{\pgfqpoint{0.100000in}{0.212622in}}{\pgfqpoint{3.696000in}{3.696000in}}%
\pgfusepath{clip}%
\pgfsetbuttcap%
\pgfsetroundjoin%
\definecolor{currentfill}{rgb}{0.121569,0.466667,0.705882}%
\pgfsetfillcolor{currentfill}%
\pgfsetfillopacity{0.847633}%
\pgfsetlinewidth{1.003750pt}%
\definecolor{currentstroke}{rgb}{0.121569,0.466667,0.705882}%
\pgfsetstrokecolor{currentstroke}%
\pgfsetstrokeopacity{0.847633}%
\pgfsetdash{}{0pt}%
\pgfpathmoveto{\pgfqpoint{0.863686in}{2.418120in}}%
\pgfpathcurveto{\pgfqpoint{0.871922in}{2.418120in}}{\pgfqpoint{0.879822in}{2.421392in}}{\pgfqpoint{0.885646in}{2.427216in}}%
\pgfpathcurveto{\pgfqpoint{0.891470in}{2.433040in}}{\pgfqpoint{0.894742in}{2.440940in}}{\pgfqpoint{0.894742in}{2.449176in}}%
\pgfpathcurveto{\pgfqpoint{0.894742in}{2.457413in}}{\pgfqpoint{0.891470in}{2.465313in}}{\pgfqpoint{0.885646in}{2.471137in}}%
\pgfpathcurveto{\pgfqpoint{0.879822in}{2.476961in}}{\pgfqpoint{0.871922in}{2.480233in}}{\pgfqpoint{0.863686in}{2.480233in}}%
\pgfpathcurveto{\pgfqpoint{0.855450in}{2.480233in}}{\pgfqpoint{0.847550in}{2.476961in}}{\pgfqpoint{0.841726in}{2.471137in}}%
\pgfpathcurveto{\pgfqpoint{0.835902in}{2.465313in}}{\pgfqpoint{0.832629in}{2.457413in}}{\pgfqpoint{0.832629in}{2.449176in}}%
\pgfpathcurveto{\pgfqpoint{0.832629in}{2.440940in}}{\pgfqpoint{0.835902in}{2.433040in}}{\pgfqpoint{0.841726in}{2.427216in}}%
\pgfpathcurveto{\pgfqpoint{0.847550in}{2.421392in}}{\pgfqpoint{0.855450in}{2.418120in}}{\pgfqpoint{0.863686in}{2.418120in}}%
\pgfpathclose%
\pgfusepath{stroke,fill}%
\end{pgfscope}%
\begin{pgfscope}%
\pgfpathrectangle{\pgfqpoint{0.100000in}{0.212622in}}{\pgfqpoint{3.696000in}{3.696000in}}%
\pgfusepath{clip}%
\pgfsetbuttcap%
\pgfsetroundjoin%
\definecolor{currentfill}{rgb}{0.121569,0.466667,0.705882}%
\pgfsetfillcolor{currentfill}%
\pgfsetfillopacity{0.848654}%
\pgfsetlinewidth{1.003750pt}%
\definecolor{currentstroke}{rgb}{0.121569,0.466667,0.705882}%
\pgfsetstrokecolor{currentstroke}%
\pgfsetstrokeopacity{0.848654}%
\pgfsetdash{}{0pt}%
\pgfpathmoveto{\pgfqpoint{2.296969in}{1.506180in}}%
\pgfpathcurveto{\pgfqpoint{2.305205in}{1.506180in}}{\pgfqpoint{2.313105in}{1.509452in}}{\pgfqpoint{2.318929in}{1.515276in}}%
\pgfpathcurveto{\pgfqpoint{2.324753in}{1.521100in}}{\pgfqpoint{2.328025in}{1.529000in}}{\pgfqpoint{2.328025in}{1.537237in}}%
\pgfpathcurveto{\pgfqpoint{2.328025in}{1.545473in}}{\pgfqpoint{2.324753in}{1.553373in}}{\pgfqpoint{2.318929in}{1.559197in}}%
\pgfpathcurveto{\pgfqpoint{2.313105in}{1.565021in}}{\pgfqpoint{2.305205in}{1.568293in}}{\pgfqpoint{2.296969in}{1.568293in}}%
\pgfpathcurveto{\pgfqpoint{2.288732in}{1.568293in}}{\pgfqpoint{2.280832in}{1.565021in}}{\pgfqpoint{2.275009in}{1.559197in}}%
\pgfpathcurveto{\pgfqpoint{2.269185in}{1.553373in}}{\pgfqpoint{2.265912in}{1.545473in}}{\pgfqpoint{2.265912in}{1.537237in}}%
\pgfpathcurveto{\pgfqpoint{2.265912in}{1.529000in}}{\pgfqpoint{2.269185in}{1.521100in}}{\pgfqpoint{2.275009in}{1.515276in}}%
\pgfpathcurveto{\pgfqpoint{2.280832in}{1.509452in}}{\pgfqpoint{2.288732in}{1.506180in}}{\pgfqpoint{2.296969in}{1.506180in}}%
\pgfpathclose%
\pgfusepath{stroke,fill}%
\end{pgfscope}%
\begin{pgfscope}%
\pgfpathrectangle{\pgfqpoint{0.100000in}{0.212622in}}{\pgfqpoint{3.696000in}{3.696000in}}%
\pgfusepath{clip}%
\pgfsetbuttcap%
\pgfsetroundjoin%
\definecolor{currentfill}{rgb}{0.121569,0.466667,0.705882}%
\pgfsetfillcolor{currentfill}%
\pgfsetfillopacity{0.851207}%
\pgfsetlinewidth{1.003750pt}%
\definecolor{currentstroke}{rgb}{0.121569,0.466667,0.705882}%
\pgfsetstrokecolor{currentstroke}%
\pgfsetstrokeopacity{0.851207}%
\pgfsetdash{}{0pt}%
\pgfpathmoveto{\pgfqpoint{2.298970in}{1.501780in}}%
\pgfpathcurveto{\pgfqpoint{2.307207in}{1.501780in}}{\pgfqpoint{2.315107in}{1.505052in}}{\pgfqpoint{2.320931in}{1.510876in}}%
\pgfpathcurveto{\pgfqpoint{2.326755in}{1.516700in}}{\pgfqpoint{2.330027in}{1.524600in}}{\pgfqpoint{2.330027in}{1.532836in}}%
\pgfpathcurveto{\pgfqpoint{2.330027in}{1.541072in}}{\pgfqpoint{2.326755in}{1.548972in}}{\pgfqpoint{2.320931in}{1.554796in}}%
\pgfpathcurveto{\pgfqpoint{2.315107in}{1.560620in}}{\pgfqpoint{2.307207in}{1.563893in}}{\pgfqpoint{2.298970in}{1.563893in}}%
\pgfpathcurveto{\pgfqpoint{2.290734in}{1.563893in}}{\pgfqpoint{2.282834in}{1.560620in}}{\pgfqpoint{2.277010in}{1.554796in}}%
\pgfpathcurveto{\pgfqpoint{2.271186in}{1.548972in}}{\pgfqpoint{2.267914in}{1.541072in}}{\pgfqpoint{2.267914in}{1.532836in}}%
\pgfpathcurveto{\pgfqpoint{2.267914in}{1.524600in}}{\pgfqpoint{2.271186in}{1.516700in}}{\pgfqpoint{2.277010in}{1.510876in}}%
\pgfpathcurveto{\pgfqpoint{2.282834in}{1.505052in}}{\pgfqpoint{2.290734in}{1.501780in}}{\pgfqpoint{2.298970in}{1.501780in}}%
\pgfpathclose%
\pgfusepath{stroke,fill}%
\end{pgfscope}%
\begin{pgfscope}%
\pgfpathrectangle{\pgfqpoint{0.100000in}{0.212622in}}{\pgfqpoint{3.696000in}{3.696000in}}%
\pgfusepath{clip}%
\pgfsetbuttcap%
\pgfsetroundjoin%
\definecolor{currentfill}{rgb}{0.121569,0.466667,0.705882}%
\pgfsetfillcolor{currentfill}%
\pgfsetfillopacity{0.852093}%
\pgfsetlinewidth{1.003750pt}%
\definecolor{currentstroke}{rgb}{0.121569,0.466667,0.705882}%
\pgfsetstrokecolor{currentstroke}%
\pgfsetstrokeopacity{0.852093}%
\pgfsetdash{}{0pt}%
\pgfpathmoveto{\pgfqpoint{0.900341in}{2.400925in}}%
\pgfpathcurveto{\pgfqpoint{0.908577in}{2.400925in}}{\pgfqpoint{0.916477in}{2.404197in}}{\pgfqpoint{0.922301in}{2.410021in}}%
\pgfpathcurveto{\pgfqpoint{0.928125in}{2.415845in}}{\pgfqpoint{0.931397in}{2.423745in}}{\pgfqpoint{0.931397in}{2.431981in}}%
\pgfpathcurveto{\pgfqpoint{0.931397in}{2.440218in}}{\pgfqpoint{0.928125in}{2.448118in}}{\pgfqpoint{0.922301in}{2.453942in}}%
\pgfpathcurveto{\pgfqpoint{0.916477in}{2.459765in}}{\pgfqpoint{0.908577in}{2.463038in}}{\pgfqpoint{0.900341in}{2.463038in}}%
\pgfpathcurveto{\pgfqpoint{0.892105in}{2.463038in}}{\pgfqpoint{0.884204in}{2.459765in}}{\pgfqpoint{0.878381in}{2.453942in}}%
\pgfpathcurveto{\pgfqpoint{0.872557in}{2.448118in}}{\pgfqpoint{0.869284in}{2.440218in}}{\pgfqpoint{0.869284in}{2.431981in}}%
\pgfpathcurveto{\pgfqpoint{0.869284in}{2.423745in}}{\pgfqpoint{0.872557in}{2.415845in}}{\pgfqpoint{0.878381in}{2.410021in}}%
\pgfpathcurveto{\pgfqpoint{0.884204in}{2.404197in}}{\pgfqpoint{0.892105in}{2.400925in}}{\pgfqpoint{0.900341in}{2.400925in}}%
\pgfpathclose%
\pgfusepath{stroke,fill}%
\end{pgfscope}%
\begin{pgfscope}%
\pgfpathrectangle{\pgfqpoint{0.100000in}{0.212622in}}{\pgfqpoint{3.696000in}{3.696000in}}%
\pgfusepath{clip}%
\pgfsetbuttcap%
\pgfsetroundjoin%
\definecolor{currentfill}{rgb}{0.121569,0.466667,0.705882}%
\pgfsetfillcolor{currentfill}%
\pgfsetfillopacity{0.854743}%
\pgfsetlinewidth{1.003750pt}%
\definecolor{currentstroke}{rgb}{0.121569,0.466667,0.705882}%
\pgfsetstrokecolor{currentstroke}%
\pgfsetstrokeopacity{0.854743}%
\pgfsetdash{}{0pt}%
\pgfpathmoveto{\pgfqpoint{0.931121in}{2.358197in}}%
\pgfpathcurveto{\pgfqpoint{0.939357in}{2.358197in}}{\pgfqpoint{0.947257in}{2.361469in}}{\pgfqpoint{0.953081in}{2.367293in}}%
\pgfpathcurveto{\pgfqpoint{0.958905in}{2.373117in}}{\pgfqpoint{0.962177in}{2.381017in}}{\pgfqpoint{0.962177in}{2.389254in}}%
\pgfpathcurveto{\pgfqpoint{0.962177in}{2.397490in}}{\pgfqpoint{0.958905in}{2.405390in}}{\pgfqpoint{0.953081in}{2.411214in}}%
\pgfpathcurveto{\pgfqpoint{0.947257in}{2.417038in}}{\pgfqpoint{0.939357in}{2.420310in}}{\pgfqpoint{0.931121in}{2.420310in}}%
\pgfpathcurveto{\pgfqpoint{0.922884in}{2.420310in}}{\pgfqpoint{0.914984in}{2.417038in}}{\pgfqpoint{0.909160in}{2.411214in}}%
\pgfpathcurveto{\pgfqpoint{0.903336in}{2.405390in}}{\pgfqpoint{0.900064in}{2.397490in}}{\pgfqpoint{0.900064in}{2.389254in}}%
\pgfpathcurveto{\pgfqpoint{0.900064in}{2.381017in}}{\pgfqpoint{0.903336in}{2.373117in}}{\pgfqpoint{0.909160in}{2.367293in}}%
\pgfpathcurveto{\pgfqpoint{0.914984in}{2.361469in}}{\pgfqpoint{0.922884in}{2.358197in}}{\pgfqpoint{0.931121in}{2.358197in}}%
\pgfpathclose%
\pgfusepath{stroke,fill}%
\end{pgfscope}%
\begin{pgfscope}%
\pgfpathrectangle{\pgfqpoint{0.100000in}{0.212622in}}{\pgfqpoint{3.696000in}{3.696000in}}%
\pgfusepath{clip}%
\pgfsetbuttcap%
\pgfsetroundjoin%
\definecolor{currentfill}{rgb}{0.121569,0.466667,0.705882}%
\pgfsetfillcolor{currentfill}%
\pgfsetfillopacity{0.854974}%
\pgfsetlinewidth{1.003750pt}%
\definecolor{currentstroke}{rgb}{0.121569,0.466667,0.705882}%
\pgfsetstrokecolor{currentstroke}%
\pgfsetstrokeopacity{0.854974}%
\pgfsetdash{}{0pt}%
\pgfpathmoveto{\pgfqpoint{2.302895in}{1.504578in}}%
\pgfpathcurveto{\pgfqpoint{2.311131in}{1.504578in}}{\pgfqpoint{2.319031in}{1.507850in}}{\pgfqpoint{2.324855in}{1.513674in}}%
\pgfpathcurveto{\pgfqpoint{2.330679in}{1.519498in}}{\pgfqpoint{2.333951in}{1.527398in}}{\pgfqpoint{2.333951in}{1.535634in}}%
\pgfpathcurveto{\pgfqpoint{2.333951in}{1.543870in}}{\pgfqpoint{2.330679in}{1.551771in}}{\pgfqpoint{2.324855in}{1.557594in}}%
\pgfpathcurveto{\pgfqpoint{2.319031in}{1.563418in}}{\pgfqpoint{2.311131in}{1.566691in}}{\pgfqpoint{2.302895in}{1.566691in}}%
\pgfpathcurveto{\pgfqpoint{2.294658in}{1.566691in}}{\pgfqpoint{2.286758in}{1.563418in}}{\pgfqpoint{2.280934in}{1.557594in}}%
\pgfpathcurveto{\pgfqpoint{2.275110in}{1.551771in}}{\pgfqpoint{2.271838in}{1.543870in}}{\pgfqpoint{2.271838in}{1.535634in}}%
\pgfpathcurveto{\pgfqpoint{2.271838in}{1.527398in}}{\pgfqpoint{2.275110in}{1.519498in}}{\pgfqpoint{2.280934in}{1.513674in}}%
\pgfpathcurveto{\pgfqpoint{2.286758in}{1.507850in}}{\pgfqpoint{2.294658in}{1.504578in}}{\pgfqpoint{2.302895in}{1.504578in}}%
\pgfpathclose%
\pgfusepath{stroke,fill}%
\end{pgfscope}%
\begin{pgfscope}%
\pgfpathrectangle{\pgfqpoint{0.100000in}{0.212622in}}{\pgfqpoint{3.696000in}{3.696000in}}%
\pgfusepath{clip}%
\pgfsetbuttcap%
\pgfsetroundjoin%
\definecolor{currentfill}{rgb}{0.121569,0.466667,0.705882}%
\pgfsetfillcolor{currentfill}%
\pgfsetfillopacity{0.858065}%
\pgfsetlinewidth{1.003750pt}%
\definecolor{currentstroke}{rgb}{0.121569,0.466667,0.705882}%
\pgfsetstrokecolor{currentstroke}%
\pgfsetstrokeopacity{0.858065}%
\pgfsetdash{}{0pt}%
\pgfpathmoveto{\pgfqpoint{0.963315in}{2.337055in}}%
\pgfpathcurveto{\pgfqpoint{0.971551in}{2.337055in}}{\pgfqpoint{0.979451in}{2.340328in}}{\pgfqpoint{0.985275in}{2.346152in}}%
\pgfpathcurveto{\pgfqpoint{0.991099in}{2.351975in}}{\pgfqpoint{0.994371in}{2.359876in}}{\pgfqpoint{0.994371in}{2.368112in}}%
\pgfpathcurveto{\pgfqpoint{0.994371in}{2.376348in}}{\pgfqpoint{0.991099in}{2.384248in}}{\pgfqpoint{0.985275in}{2.390072in}}%
\pgfpathcurveto{\pgfqpoint{0.979451in}{2.395896in}}{\pgfqpoint{0.971551in}{2.399168in}}{\pgfqpoint{0.963315in}{2.399168in}}%
\pgfpathcurveto{\pgfqpoint{0.955078in}{2.399168in}}{\pgfqpoint{0.947178in}{2.395896in}}{\pgfqpoint{0.941354in}{2.390072in}}%
\pgfpathcurveto{\pgfqpoint{0.935531in}{2.384248in}}{\pgfqpoint{0.932258in}{2.376348in}}{\pgfqpoint{0.932258in}{2.368112in}}%
\pgfpathcurveto{\pgfqpoint{0.932258in}{2.359876in}}{\pgfqpoint{0.935531in}{2.351975in}}{\pgfqpoint{0.941354in}{2.346152in}}%
\pgfpathcurveto{\pgfqpoint{0.947178in}{2.340328in}}{\pgfqpoint{0.955078in}{2.337055in}}{\pgfqpoint{0.963315in}{2.337055in}}%
\pgfpathclose%
\pgfusepath{stroke,fill}%
\end{pgfscope}%
\begin{pgfscope}%
\pgfpathrectangle{\pgfqpoint{0.100000in}{0.212622in}}{\pgfqpoint{3.696000in}{3.696000in}}%
\pgfusepath{clip}%
\pgfsetbuttcap%
\pgfsetroundjoin%
\definecolor{currentfill}{rgb}{0.121569,0.466667,0.705882}%
\pgfsetfillcolor{currentfill}%
\pgfsetfillopacity{0.858390}%
\pgfsetlinewidth{1.003750pt}%
\definecolor{currentstroke}{rgb}{0.121569,0.466667,0.705882}%
\pgfsetstrokecolor{currentstroke}%
\pgfsetstrokeopacity{0.858390}%
\pgfsetdash{}{0pt}%
\pgfpathmoveto{\pgfqpoint{2.304367in}{1.500565in}}%
\pgfpathcurveto{\pgfqpoint{2.312604in}{1.500565in}}{\pgfqpoint{2.320504in}{1.503837in}}{\pgfqpoint{2.326327in}{1.509661in}}%
\pgfpathcurveto{\pgfqpoint{2.332151in}{1.515485in}}{\pgfqpoint{2.335424in}{1.523385in}}{\pgfqpoint{2.335424in}{1.531622in}}%
\pgfpathcurveto{\pgfqpoint{2.335424in}{1.539858in}}{\pgfqpoint{2.332151in}{1.547758in}}{\pgfqpoint{2.326327in}{1.553582in}}%
\pgfpathcurveto{\pgfqpoint{2.320504in}{1.559406in}}{\pgfqpoint{2.312604in}{1.562678in}}{\pgfqpoint{2.304367in}{1.562678in}}%
\pgfpathcurveto{\pgfqpoint{2.296131in}{1.562678in}}{\pgfqpoint{2.288231in}{1.559406in}}{\pgfqpoint{2.282407in}{1.553582in}}%
\pgfpathcurveto{\pgfqpoint{2.276583in}{1.547758in}}{\pgfqpoint{2.273311in}{1.539858in}}{\pgfqpoint{2.273311in}{1.531622in}}%
\pgfpathcurveto{\pgfqpoint{2.273311in}{1.523385in}}{\pgfqpoint{2.276583in}{1.515485in}}{\pgfqpoint{2.282407in}{1.509661in}}%
\pgfpathcurveto{\pgfqpoint{2.288231in}{1.503837in}}{\pgfqpoint{2.296131in}{1.500565in}}{\pgfqpoint{2.304367in}{1.500565in}}%
\pgfpathclose%
\pgfusepath{stroke,fill}%
\end{pgfscope}%
\begin{pgfscope}%
\pgfpathrectangle{\pgfqpoint{0.100000in}{0.212622in}}{\pgfqpoint{3.696000in}{3.696000in}}%
\pgfusepath{clip}%
\pgfsetbuttcap%
\pgfsetroundjoin%
\definecolor{currentfill}{rgb}{0.121569,0.466667,0.705882}%
\pgfsetfillcolor{currentfill}%
\pgfsetfillopacity{0.861477}%
\pgfsetlinewidth{1.003750pt}%
\definecolor{currentstroke}{rgb}{0.121569,0.466667,0.705882}%
\pgfsetstrokecolor{currentstroke}%
\pgfsetstrokeopacity{0.861477}%
\pgfsetdash{}{0pt}%
\pgfpathmoveto{\pgfqpoint{1.022447in}{2.281925in}}%
\pgfpathcurveto{\pgfqpoint{1.030683in}{2.281925in}}{\pgfqpoint{1.038583in}{2.285197in}}{\pgfqpoint{1.044407in}{2.291021in}}%
\pgfpathcurveto{\pgfqpoint{1.050231in}{2.296845in}}{\pgfqpoint{1.053503in}{2.304745in}}{\pgfqpoint{1.053503in}{2.312981in}}%
\pgfpathcurveto{\pgfqpoint{1.053503in}{2.321217in}}{\pgfqpoint{1.050231in}{2.329117in}}{\pgfqpoint{1.044407in}{2.334941in}}%
\pgfpathcurveto{\pgfqpoint{1.038583in}{2.340765in}}{\pgfqpoint{1.030683in}{2.344038in}}{\pgfqpoint{1.022447in}{2.344038in}}%
\pgfpathcurveto{\pgfqpoint{1.014210in}{2.344038in}}{\pgfqpoint{1.006310in}{2.340765in}}{\pgfqpoint{1.000486in}{2.334941in}}%
\pgfpathcurveto{\pgfqpoint{0.994662in}{2.329117in}}{\pgfqpoint{0.991390in}{2.321217in}}{\pgfqpoint{0.991390in}{2.312981in}}%
\pgfpathcurveto{\pgfqpoint{0.991390in}{2.304745in}}{\pgfqpoint{0.994662in}{2.296845in}}{\pgfqpoint{1.000486in}{2.291021in}}%
\pgfpathcurveto{\pgfqpoint{1.006310in}{2.285197in}}{\pgfqpoint{1.014210in}{2.281925in}}{\pgfqpoint{1.022447in}{2.281925in}}%
\pgfpathclose%
\pgfusepath{stroke,fill}%
\end{pgfscope}%
\begin{pgfscope}%
\pgfpathrectangle{\pgfqpoint{0.100000in}{0.212622in}}{\pgfqpoint{3.696000in}{3.696000in}}%
\pgfusepath{clip}%
\pgfsetbuttcap%
\pgfsetroundjoin%
\definecolor{currentfill}{rgb}{0.121569,0.466667,0.705882}%
\pgfsetfillcolor{currentfill}%
\pgfsetfillopacity{0.862067}%
\pgfsetlinewidth{1.003750pt}%
\definecolor{currentstroke}{rgb}{0.121569,0.466667,0.705882}%
\pgfsetstrokecolor{currentstroke}%
\pgfsetstrokeopacity{0.862067}%
\pgfsetdash{}{0pt}%
\pgfpathmoveto{\pgfqpoint{2.307500in}{1.496645in}}%
\pgfpathcurveto{\pgfqpoint{2.315737in}{1.496645in}}{\pgfqpoint{2.323637in}{1.499917in}}{\pgfqpoint{2.329461in}{1.505741in}}%
\pgfpathcurveto{\pgfqpoint{2.335284in}{1.511565in}}{\pgfqpoint{2.338557in}{1.519465in}}{\pgfqpoint{2.338557in}{1.527701in}}%
\pgfpathcurveto{\pgfqpoint{2.338557in}{1.535937in}}{\pgfqpoint{2.335284in}{1.543837in}}{\pgfqpoint{2.329461in}{1.549661in}}%
\pgfpathcurveto{\pgfqpoint{2.323637in}{1.555485in}}{\pgfqpoint{2.315737in}{1.558758in}}{\pgfqpoint{2.307500in}{1.558758in}}%
\pgfpathcurveto{\pgfqpoint{2.299264in}{1.558758in}}{\pgfqpoint{2.291364in}{1.555485in}}{\pgfqpoint{2.285540in}{1.549661in}}%
\pgfpathcurveto{\pgfqpoint{2.279716in}{1.543837in}}{\pgfqpoint{2.276444in}{1.535937in}}{\pgfqpoint{2.276444in}{1.527701in}}%
\pgfpathcurveto{\pgfqpoint{2.276444in}{1.519465in}}{\pgfqpoint{2.279716in}{1.511565in}}{\pgfqpoint{2.285540in}{1.505741in}}%
\pgfpathcurveto{\pgfqpoint{2.291364in}{1.499917in}}{\pgfqpoint{2.299264in}{1.496645in}}{\pgfqpoint{2.307500in}{1.496645in}}%
\pgfpathclose%
\pgfusepath{stroke,fill}%
\end{pgfscope}%
\begin{pgfscope}%
\pgfpathrectangle{\pgfqpoint{0.100000in}{0.212622in}}{\pgfqpoint{3.696000in}{3.696000in}}%
\pgfusepath{clip}%
\pgfsetbuttcap%
\pgfsetroundjoin%
\definecolor{currentfill}{rgb}{0.121569,0.466667,0.705882}%
\pgfsetfillcolor{currentfill}%
\pgfsetfillopacity{0.864669}%
\pgfsetlinewidth{1.003750pt}%
\definecolor{currentstroke}{rgb}{0.121569,0.466667,0.705882}%
\pgfsetstrokecolor{currentstroke}%
\pgfsetstrokeopacity{0.864669}%
\pgfsetdash{}{0pt}%
\pgfpathmoveto{\pgfqpoint{2.310315in}{1.483812in}}%
\pgfpathcurveto{\pgfqpoint{2.318551in}{1.483812in}}{\pgfqpoint{2.326451in}{1.487084in}}{\pgfqpoint{2.332275in}{1.492908in}}%
\pgfpathcurveto{\pgfqpoint{2.338099in}{1.498732in}}{\pgfqpoint{2.341371in}{1.506632in}}{\pgfqpoint{2.341371in}{1.514868in}}%
\pgfpathcurveto{\pgfqpoint{2.341371in}{1.523105in}}{\pgfqpoint{2.338099in}{1.531005in}}{\pgfqpoint{2.332275in}{1.536829in}}%
\pgfpathcurveto{\pgfqpoint{2.326451in}{1.542653in}}{\pgfqpoint{2.318551in}{1.545925in}}{\pgfqpoint{2.310315in}{1.545925in}}%
\pgfpathcurveto{\pgfqpoint{2.302078in}{1.545925in}}{\pgfqpoint{2.294178in}{1.542653in}}{\pgfqpoint{2.288354in}{1.536829in}}%
\pgfpathcurveto{\pgfqpoint{2.282530in}{1.531005in}}{\pgfqpoint{2.279258in}{1.523105in}}{\pgfqpoint{2.279258in}{1.514868in}}%
\pgfpathcurveto{\pgfqpoint{2.279258in}{1.506632in}}{\pgfqpoint{2.282530in}{1.498732in}}{\pgfqpoint{2.288354in}{1.492908in}}%
\pgfpathcurveto{\pgfqpoint{2.294178in}{1.487084in}}{\pgfqpoint{2.302078in}{1.483812in}}{\pgfqpoint{2.310315in}{1.483812in}}%
\pgfpathclose%
\pgfusepath{stroke,fill}%
\end{pgfscope}%
\begin{pgfscope}%
\pgfpathrectangle{\pgfqpoint{0.100000in}{0.212622in}}{\pgfqpoint{3.696000in}{3.696000in}}%
\pgfusepath{clip}%
\pgfsetbuttcap%
\pgfsetroundjoin%
\definecolor{currentfill}{rgb}{0.121569,0.466667,0.705882}%
\pgfsetfillcolor{currentfill}%
\pgfsetfillopacity{0.866186}%
\pgfsetlinewidth{1.003750pt}%
\definecolor{currentstroke}{rgb}{0.121569,0.466667,0.705882}%
\pgfsetstrokecolor{currentstroke}%
\pgfsetstrokeopacity{0.866186}%
\pgfsetdash{}{0pt}%
\pgfpathmoveto{\pgfqpoint{1.076500in}{2.235895in}}%
\pgfpathcurveto{\pgfqpoint{1.084737in}{2.235895in}}{\pgfqpoint{1.092637in}{2.239168in}}{\pgfqpoint{1.098461in}{2.244992in}}%
\pgfpathcurveto{\pgfqpoint{1.104285in}{2.250815in}}{\pgfqpoint{1.107557in}{2.258715in}}{\pgfqpoint{1.107557in}{2.266952in}}%
\pgfpathcurveto{\pgfqpoint{1.107557in}{2.275188in}}{\pgfqpoint{1.104285in}{2.283088in}}{\pgfqpoint{1.098461in}{2.288912in}}%
\pgfpathcurveto{\pgfqpoint{1.092637in}{2.294736in}}{\pgfqpoint{1.084737in}{2.298008in}}{\pgfqpoint{1.076500in}{2.298008in}}%
\pgfpathcurveto{\pgfqpoint{1.068264in}{2.298008in}}{\pgfqpoint{1.060364in}{2.294736in}}{\pgfqpoint{1.054540in}{2.288912in}}%
\pgfpathcurveto{\pgfqpoint{1.048716in}{2.283088in}}{\pgfqpoint{1.045444in}{2.275188in}}{\pgfqpoint{1.045444in}{2.266952in}}%
\pgfpathcurveto{\pgfqpoint{1.045444in}{2.258715in}}{\pgfqpoint{1.048716in}{2.250815in}}{\pgfqpoint{1.054540in}{2.244992in}}%
\pgfpathcurveto{\pgfqpoint{1.060364in}{2.239168in}}{\pgfqpoint{1.068264in}{2.235895in}}{\pgfqpoint{1.076500in}{2.235895in}}%
\pgfpathclose%
\pgfusepath{stroke,fill}%
\end{pgfscope}%
\begin{pgfscope}%
\pgfpathrectangle{\pgfqpoint{0.100000in}{0.212622in}}{\pgfqpoint{3.696000in}{3.696000in}}%
\pgfusepath{clip}%
\pgfsetbuttcap%
\pgfsetroundjoin%
\definecolor{currentfill}{rgb}{0.121569,0.466667,0.705882}%
\pgfsetfillcolor{currentfill}%
\pgfsetfillopacity{0.867005}%
\pgfsetlinewidth{1.003750pt}%
\definecolor{currentstroke}{rgb}{0.121569,0.466667,0.705882}%
\pgfsetstrokecolor{currentstroke}%
\pgfsetstrokeopacity{0.867005}%
\pgfsetdash{}{0pt}%
\pgfpathmoveto{\pgfqpoint{2.312279in}{1.482554in}}%
\pgfpathcurveto{\pgfqpoint{2.320515in}{1.482554in}}{\pgfqpoint{2.328415in}{1.485827in}}{\pgfqpoint{2.334239in}{1.491651in}}%
\pgfpathcurveto{\pgfqpoint{2.340063in}{1.497475in}}{\pgfqpoint{2.343335in}{1.505375in}}{\pgfqpoint{2.343335in}{1.513611in}}%
\pgfpathcurveto{\pgfqpoint{2.343335in}{1.521847in}}{\pgfqpoint{2.340063in}{1.529747in}}{\pgfqpoint{2.334239in}{1.535571in}}%
\pgfpathcurveto{\pgfqpoint{2.328415in}{1.541395in}}{\pgfqpoint{2.320515in}{1.544667in}}{\pgfqpoint{2.312279in}{1.544667in}}%
\pgfpathcurveto{\pgfqpoint{2.304043in}{1.544667in}}{\pgfqpoint{2.296142in}{1.541395in}}{\pgfqpoint{2.290319in}{1.535571in}}%
\pgfpathcurveto{\pgfqpoint{2.284495in}{1.529747in}}{\pgfqpoint{2.281222in}{1.521847in}}{\pgfqpoint{2.281222in}{1.513611in}}%
\pgfpathcurveto{\pgfqpoint{2.281222in}{1.505375in}}{\pgfqpoint{2.284495in}{1.497475in}}{\pgfqpoint{2.290319in}{1.491651in}}%
\pgfpathcurveto{\pgfqpoint{2.296142in}{1.485827in}}{\pgfqpoint{2.304043in}{1.482554in}}{\pgfqpoint{2.312279in}{1.482554in}}%
\pgfpathclose%
\pgfusepath{stroke,fill}%
\end{pgfscope}%
\begin{pgfscope}%
\pgfpathrectangle{\pgfqpoint{0.100000in}{0.212622in}}{\pgfqpoint{3.696000in}{3.696000in}}%
\pgfusepath{clip}%
\pgfsetbuttcap%
\pgfsetroundjoin%
\definecolor{currentfill}{rgb}{0.121569,0.466667,0.705882}%
\pgfsetfillcolor{currentfill}%
\pgfsetfillopacity{0.869702}%
\pgfsetlinewidth{1.003750pt}%
\definecolor{currentstroke}{rgb}{0.121569,0.466667,0.705882}%
\pgfsetstrokecolor{currentstroke}%
\pgfsetstrokeopacity{0.869702}%
\pgfsetdash{}{0pt}%
\pgfpathmoveto{\pgfqpoint{2.314949in}{1.481829in}}%
\pgfpathcurveto{\pgfqpoint{2.323186in}{1.481829in}}{\pgfqpoint{2.331086in}{1.485102in}}{\pgfqpoint{2.336910in}{1.490926in}}%
\pgfpathcurveto{\pgfqpoint{2.342733in}{1.496750in}}{\pgfqpoint{2.346006in}{1.504650in}}{\pgfqpoint{2.346006in}{1.512886in}}%
\pgfpathcurveto{\pgfqpoint{2.346006in}{1.521122in}}{\pgfqpoint{2.342733in}{1.529022in}}{\pgfqpoint{2.336910in}{1.534846in}}%
\pgfpathcurveto{\pgfqpoint{2.331086in}{1.540670in}}{\pgfqpoint{2.323186in}{1.543942in}}{\pgfqpoint{2.314949in}{1.543942in}}%
\pgfpathcurveto{\pgfqpoint{2.306713in}{1.543942in}}{\pgfqpoint{2.298813in}{1.540670in}}{\pgfqpoint{2.292989in}{1.534846in}}%
\pgfpathcurveto{\pgfqpoint{2.287165in}{1.529022in}}{\pgfqpoint{2.283893in}{1.521122in}}{\pgfqpoint{2.283893in}{1.512886in}}%
\pgfpathcurveto{\pgfqpoint{2.283893in}{1.504650in}}{\pgfqpoint{2.287165in}{1.496750in}}{\pgfqpoint{2.292989in}{1.490926in}}%
\pgfpathcurveto{\pgfqpoint{2.298813in}{1.485102in}}{\pgfqpoint{2.306713in}{1.481829in}}{\pgfqpoint{2.314949in}{1.481829in}}%
\pgfpathclose%
\pgfusepath{stroke,fill}%
\end{pgfscope}%
\begin{pgfscope}%
\pgfpathrectangle{\pgfqpoint{0.100000in}{0.212622in}}{\pgfqpoint{3.696000in}{3.696000in}}%
\pgfusepath{clip}%
\pgfsetbuttcap%
\pgfsetroundjoin%
\definecolor{currentfill}{rgb}{0.121569,0.466667,0.705882}%
\pgfsetfillcolor{currentfill}%
\pgfsetfillopacity{0.871632}%
\pgfsetlinewidth{1.003750pt}%
\definecolor{currentstroke}{rgb}{0.121569,0.466667,0.705882}%
\pgfsetstrokecolor{currentstroke}%
\pgfsetstrokeopacity{0.871632}%
\pgfsetdash{}{0pt}%
\pgfpathmoveto{\pgfqpoint{1.129135in}{2.192437in}}%
\pgfpathcurveto{\pgfqpoint{1.137371in}{2.192437in}}{\pgfqpoint{1.145271in}{2.195709in}}{\pgfqpoint{1.151095in}{2.201533in}}%
\pgfpathcurveto{\pgfqpoint{1.156919in}{2.207357in}}{\pgfqpoint{1.160191in}{2.215257in}}{\pgfqpoint{1.160191in}{2.223494in}}%
\pgfpathcurveto{\pgfqpoint{1.160191in}{2.231730in}}{\pgfqpoint{1.156919in}{2.239630in}}{\pgfqpoint{1.151095in}{2.245454in}}%
\pgfpathcurveto{\pgfqpoint{1.145271in}{2.251278in}}{\pgfqpoint{1.137371in}{2.254550in}}{\pgfqpoint{1.129135in}{2.254550in}}%
\pgfpathcurveto{\pgfqpoint{1.120898in}{2.254550in}}{\pgfqpoint{1.112998in}{2.251278in}}{\pgfqpoint{1.107174in}{2.245454in}}%
\pgfpathcurveto{\pgfqpoint{1.101350in}{2.239630in}}{\pgfqpoint{1.098078in}{2.231730in}}{\pgfqpoint{1.098078in}{2.223494in}}%
\pgfpathcurveto{\pgfqpoint{1.098078in}{2.215257in}}{\pgfqpoint{1.101350in}{2.207357in}}{\pgfqpoint{1.107174in}{2.201533in}}%
\pgfpathcurveto{\pgfqpoint{1.112998in}{2.195709in}}{\pgfqpoint{1.120898in}{2.192437in}}{\pgfqpoint{1.129135in}{2.192437in}}%
\pgfpathclose%
\pgfusepath{stroke,fill}%
\end{pgfscope}%
\begin{pgfscope}%
\pgfpathrectangle{\pgfqpoint{0.100000in}{0.212622in}}{\pgfqpoint{3.696000in}{3.696000in}}%
\pgfusepath{clip}%
\pgfsetbuttcap%
\pgfsetroundjoin%
\definecolor{currentfill}{rgb}{0.121569,0.466667,0.705882}%
\pgfsetfillcolor{currentfill}%
\pgfsetfillopacity{0.872210}%
\pgfsetlinewidth{1.003750pt}%
\definecolor{currentstroke}{rgb}{0.121569,0.466667,0.705882}%
\pgfsetstrokecolor{currentstroke}%
\pgfsetstrokeopacity{0.872210}%
\pgfsetdash{}{0pt}%
\pgfpathmoveto{\pgfqpoint{2.316292in}{1.477957in}}%
\pgfpathcurveto{\pgfqpoint{2.324528in}{1.477957in}}{\pgfqpoint{2.332428in}{1.481229in}}{\pgfqpoint{2.338252in}{1.487053in}}%
\pgfpathcurveto{\pgfqpoint{2.344076in}{1.492877in}}{\pgfqpoint{2.347348in}{1.500777in}}{\pgfqpoint{2.347348in}{1.509013in}}%
\pgfpathcurveto{\pgfqpoint{2.347348in}{1.517249in}}{\pgfqpoint{2.344076in}{1.525149in}}{\pgfqpoint{2.338252in}{1.530973in}}%
\pgfpathcurveto{\pgfqpoint{2.332428in}{1.536797in}}{\pgfqpoint{2.324528in}{1.540070in}}{\pgfqpoint{2.316292in}{1.540070in}}%
\pgfpathcurveto{\pgfqpoint{2.308055in}{1.540070in}}{\pgfqpoint{2.300155in}{1.536797in}}{\pgfqpoint{2.294331in}{1.530973in}}%
\pgfpathcurveto{\pgfqpoint{2.288507in}{1.525149in}}{\pgfqpoint{2.285235in}{1.517249in}}{\pgfqpoint{2.285235in}{1.509013in}}%
\pgfpathcurveto{\pgfqpoint{2.285235in}{1.500777in}}{\pgfqpoint{2.288507in}{1.492877in}}{\pgfqpoint{2.294331in}{1.487053in}}%
\pgfpathcurveto{\pgfqpoint{2.300155in}{1.481229in}}{\pgfqpoint{2.308055in}{1.477957in}}{\pgfqpoint{2.316292in}{1.477957in}}%
\pgfpathclose%
\pgfusepath{stroke,fill}%
\end{pgfscope}%
\begin{pgfscope}%
\pgfpathrectangle{\pgfqpoint{0.100000in}{0.212622in}}{\pgfqpoint{3.696000in}{3.696000in}}%
\pgfusepath{clip}%
\pgfsetbuttcap%
\pgfsetroundjoin%
\definecolor{currentfill}{rgb}{0.121569,0.466667,0.705882}%
\pgfsetfillcolor{currentfill}%
\pgfsetfillopacity{0.873717}%
\pgfsetlinewidth{1.003750pt}%
\definecolor{currentstroke}{rgb}{0.121569,0.466667,0.705882}%
\pgfsetstrokecolor{currentstroke}%
\pgfsetstrokeopacity{0.873717}%
\pgfsetdash{}{0pt}%
\pgfpathmoveto{\pgfqpoint{2.317302in}{1.476733in}}%
\pgfpathcurveto{\pgfqpoint{2.325539in}{1.476733in}}{\pgfqpoint{2.333439in}{1.480006in}}{\pgfqpoint{2.339263in}{1.485830in}}%
\pgfpathcurveto{\pgfqpoint{2.345087in}{1.491654in}}{\pgfqpoint{2.348359in}{1.499554in}}{\pgfqpoint{2.348359in}{1.507790in}}%
\pgfpathcurveto{\pgfqpoint{2.348359in}{1.516026in}}{\pgfqpoint{2.345087in}{1.523926in}}{\pgfqpoint{2.339263in}{1.529750in}}%
\pgfpathcurveto{\pgfqpoint{2.333439in}{1.535574in}}{\pgfqpoint{2.325539in}{1.538846in}}{\pgfqpoint{2.317302in}{1.538846in}}%
\pgfpathcurveto{\pgfqpoint{2.309066in}{1.538846in}}{\pgfqpoint{2.301166in}{1.535574in}}{\pgfqpoint{2.295342in}{1.529750in}}%
\pgfpathcurveto{\pgfqpoint{2.289518in}{1.523926in}}{\pgfqpoint{2.286246in}{1.516026in}}{\pgfqpoint{2.286246in}{1.507790in}}%
\pgfpathcurveto{\pgfqpoint{2.286246in}{1.499554in}}{\pgfqpoint{2.289518in}{1.491654in}}{\pgfqpoint{2.295342in}{1.485830in}}%
\pgfpathcurveto{\pgfqpoint{2.301166in}{1.480006in}}{\pgfqpoint{2.309066in}{1.476733in}}{\pgfqpoint{2.317302in}{1.476733in}}%
\pgfpathclose%
\pgfusepath{stroke,fill}%
\end{pgfscope}%
\begin{pgfscope}%
\pgfpathrectangle{\pgfqpoint{0.100000in}{0.212622in}}{\pgfqpoint{3.696000in}{3.696000in}}%
\pgfusepath{clip}%
\pgfsetbuttcap%
\pgfsetroundjoin%
\definecolor{currentfill}{rgb}{0.121569,0.466667,0.705882}%
\pgfsetfillcolor{currentfill}%
\pgfsetfillopacity{0.874278}%
\pgfsetlinewidth{1.003750pt}%
\definecolor{currentstroke}{rgb}{0.121569,0.466667,0.705882}%
\pgfsetstrokecolor{currentstroke}%
\pgfsetstrokeopacity{0.874278}%
\pgfsetdash{}{0pt}%
\pgfpathmoveto{\pgfqpoint{2.317976in}{1.474486in}}%
\pgfpathcurveto{\pgfqpoint{2.326212in}{1.474486in}}{\pgfqpoint{2.334112in}{1.477758in}}{\pgfqpoint{2.339936in}{1.483582in}}%
\pgfpathcurveto{\pgfqpoint{2.345760in}{1.489406in}}{\pgfqpoint{2.349032in}{1.497306in}}{\pgfqpoint{2.349032in}{1.505542in}}%
\pgfpathcurveto{\pgfqpoint{2.349032in}{1.513779in}}{\pgfqpoint{2.345760in}{1.521679in}}{\pgfqpoint{2.339936in}{1.527503in}}%
\pgfpathcurveto{\pgfqpoint{2.334112in}{1.533327in}}{\pgfqpoint{2.326212in}{1.536599in}}{\pgfqpoint{2.317976in}{1.536599in}}%
\pgfpathcurveto{\pgfqpoint{2.309739in}{1.536599in}}{\pgfqpoint{2.301839in}{1.533327in}}{\pgfqpoint{2.296015in}{1.527503in}}%
\pgfpathcurveto{\pgfqpoint{2.290191in}{1.521679in}}{\pgfqpoint{2.286919in}{1.513779in}}{\pgfqpoint{2.286919in}{1.505542in}}%
\pgfpathcurveto{\pgfqpoint{2.286919in}{1.497306in}}{\pgfqpoint{2.290191in}{1.489406in}}{\pgfqpoint{2.296015in}{1.483582in}}%
\pgfpathcurveto{\pgfqpoint{2.301839in}{1.477758in}}{\pgfqpoint{2.309739in}{1.474486in}}{\pgfqpoint{2.317976in}{1.474486in}}%
\pgfpathclose%
\pgfusepath{stroke,fill}%
\end{pgfscope}%
\begin{pgfscope}%
\pgfpathrectangle{\pgfqpoint{0.100000in}{0.212622in}}{\pgfqpoint{3.696000in}{3.696000in}}%
\pgfusepath{clip}%
\pgfsetbuttcap%
\pgfsetroundjoin%
\definecolor{currentfill}{rgb}{0.121569,0.466667,0.705882}%
\pgfsetfillcolor{currentfill}%
\pgfsetfillopacity{0.874749}%
\pgfsetlinewidth{1.003750pt}%
\definecolor{currentstroke}{rgb}{0.121569,0.466667,0.705882}%
\pgfsetstrokecolor{currentstroke}%
\pgfsetstrokeopacity{0.874749}%
\pgfsetdash{}{0pt}%
\pgfpathmoveto{\pgfqpoint{2.318333in}{1.474239in}}%
\pgfpathcurveto{\pgfqpoint{2.326569in}{1.474239in}}{\pgfqpoint{2.334469in}{1.477512in}}{\pgfqpoint{2.340293in}{1.483336in}}%
\pgfpathcurveto{\pgfqpoint{2.346117in}{1.489160in}}{\pgfqpoint{2.349390in}{1.497060in}}{\pgfqpoint{2.349390in}{1.505296in}}%
\pgfpathcurveto{\pgfqpoint{2.349390in}{1.513532in}}{\pgfqpoint{2.346117in}{1.521432in}}{\pgfqpoint{2.340293in}{1.527256in}}%
\pgfpathcurveto{\pgfqpoint{2.334469in}{1.533080in}}{\pgfqpoint{2.326569in}{1.536352in}}{\pgfqpoint{2.318333in}{1.536352in}}%
\pgfpathcurveto{\pgfqpoint{2.310097in}{1.536352in}}{\pgfqpoint{2.302197in}{1.533080in}}{\pgfqpoint{2.296373in}{1.527256in}}%
\pgfpathcurveto{\pgfqpoint{2.290549in}{1.521432in}}{\pgfqpoint{2.287277in}{1.513532in}}{\pgfqpoint{2.287277in}{1.505296in}}%
\pgfpathcurveto{\pgfqpoint{2.287277in}{1.497060in}}{\pgfqpoint{2.290549in}{1.489160in}}{\pgfqpoint{2.296373in}{1.483336in}}%
\pgfpathcurveto{\pgfqpoint{2.302197in}{1.477512in}}{\pgfqpoint{2.310097in}{1.474239in}}{\pgfqpoint{2.318333in}{1.474239in}}%
\pgfpathclose%
\pgfusepath{stroke,fill}%
\end{pgfscope}%
\begin{pgfscope}%
\pgfpathrectangle{\pgfqpoint{0.100000in}{0.212622in}}{\pgfqpoint{3.696000in}{3.696000in}}%
\pgfusepath{clip}%
\pgfsetbuttcap%
\pgfsetroundjoin%
\definecolor{currentfill}{rgb}{0.121569,0.466667,0.705882}%
\pgfsetfillcolor{currentfill}%
\pgfsetfillopacity{0.875785}%
\pgfsetlinewidth{1.003750pt}%
\definecolor{currentstroke}{rgb}{0.121569,0.466667,0.705882}%
\pgfsetstrokecolor{currentstroke}%
\pgfsetstrokeopacity{0.875785}%
\pgfsetdash{}{0pt}%
\pgfpathmoveto{\pgfqpoint{2.319073in}{1.473835in}}%
\pgfpathcurveto{\pgfqpoint{2.327310in}{1.473835in}}{\pgfqpoint{2.335210in}{1.477107in}}{\pgfqpoint{2.341034in}{1.482931in}}%
\pgfpathcurveto{\pgfqpoint{2.346857in}{1.488755in}}{\pgfqpoint{2.350130in}{1.496655in}}{\pgfqpoint{2.350130in}{1.504892in}}%
\pgfpathcurveto{\pgfqpoint{2.350130in}{1.513128in}}{\pgfqpoint{2.346857in}{1.521028in}}{\pgfqpoint{2.341034in}{1.526852in}}%
\pgfpathcurveto{\pgfqpoint{2.335210in}{1.532676in}}{\pgfqpoint{2.327310in}{1.535948in}}{\pgfqpoint{2.319073in}{1.535948in}}%
\pgfpathcurveto{\pgfqpoint{2.310837in}{1.535948in}}{\pgfqpoint{2.302937in}{1.532676in}}{\pgfqpoint{2.297113in}{1.526852in}}%
\pgfpathcurveto{\pgfqpoint{2.291289in}{1.521028in}}{\pgfqpoint{2.288017in}{1.513128in}}{\pgfqpoint{2.288017in}{1.504892in}}%
\pgfpathcurveto{\pgfqpoint{2.288017in}{1.496655in}}{\pgfqpoint{2.291289in}{1.488755in}}{\pgfqpoint{2.297113in}{1.482931in}}%
\pgfpathcurveto{\pgfqpoint{2.302937in}{1.477107in}}{\pgfqpoint{2.310837in}{1.473835in}}{\pgfqpoint{2.319073in}{1.473835in}}%
\pgfpathclose%
\pgfusepath{stroke,fill}%
\end{pgfscope}%
\begin{pgfscope}%
\pgfpathrectangle{\pgfqpoint{0.100000in}{0.212622in}}{\pgfqpoint{3.696000in}{3.696000in}}%
\pgfusepath{clip}%
\pgfsetbuttcap%
\pgfsetroundjoin%
\definecolor{currentfill}{rgb}{0.121569,0.466667,0.705882}%
\pgfsetfillcolor{currentfill}%
\pgfsetfillopacity{0.877117}%
\pgfsetlinewidth{1.003750pt}%
\definecolor{currentstroke}{rgb}{0.121569,0.466667,0.705882}%
\pgfsetstrokecolor{currentstroke}%
\pgfsetstrokeopacity{0.877117}%
\pgfsetdash{}{0pt}%
\pgfpathmoveto{\pgfqpoint{2.319857in}{1.471242in}}%
\pgfpathcurveto{\pgfqpoint{2.328093in}{1.471242in}}{\pgfqpoint{2.335993in}{1.474514in}}{\pgfqpoint{2.341817in}{1.480338in}}%
\pgfpathcurveto{\pgfqpoint{2.347641in}{1.486162in}}{\pgfqpoint{2.350914in}{1.494062in}}{\pgfqpoint{2.350914in}{1.502298in}}%
\pgfpathcurveto{\pgfqpoint{2.350914in}{1.510535in}}{\pgfqpoint{2.347641in}{1.518435in}}{\pgfqpoint{2.341817in}{1.524259in}}%
\pgfpathcurveto{\pgfqpoint{2.335993in}{1.530082in}}{\pgfqpoint{2.328093in}{1.533355in}}{\pgfqpoint{2.319857in}{1.533355in}}%
\pgfpathcurveto{\pgfqpoint{2.311621in}{1.533355in}}{\pgfqpoint{2.303721in}{1.530082in}}{\pgfqpoint{2.297897in}{1.524259in}}%
\pgfpathcurveto{\pgfqpoint{2.292073in}{1.518435in}}{\pgfqpoint{2.288801in}{1.510535in}}{\pgfqpoint{2.288801in}{1.502298in}}%
\pgfpathcurveto{\pgfqpoint{2.288801in}{1.494062in}}{\pgfqpoint{2.292073in}{1.486162in}}{\pgfqpoint{2.297897in}{1.480338in}}%
\pgfpathcurveto{\pgfqpoint{2.303721in}{1.474514in}}{\pgfqpoint{2.311621in}{1.471242in}}{\pgfqpoint{2.319857in}{1.471242in}}%
\pgfpathclose%
\pgfusepath{stroke,fill}%
\end{pgfscope}%
\begin{pgfscope}%
\pgfpathrectangle{\pgfqpoint{0.100000in}{0.212622in}}{\pgfqpoint{3.696000in}{3.696000in}}%
\pgfusepath{clip}%
\pgfsetbuttcap%
\pgfsetroundjoin%
\definecolor{currentfill}{rgb}{0.121569,0.466667,0.705882}%
\pgfsetfillcolor{currentfill}%
\pgfsetfillopacity{0.877365}%
\pgfsetlinewidth{1.003750pt}%
\definecolor{currentstroke}{rgb}{0.121569,0.466667,0.705882}%
\pgfsetstrokecolor{currentstroke}%
\pgfsetstrokeopacity{0.877365}%
\pgfsetdash{}{0pt}%
\pgfpathmoveto{\pgfqpoint{1.178449in}{2.151105in}}%
\pgfpathcurveto{\pgfqpoint{1.186686in}{2.151105in}}{\pgfqpoint{1.194586in}{2.154378in}}{\pgfqpoint{1.200410in}{2.160202in}}%
\pgfpathcurveto{\pgfqpoint{1.206234in}{2.166025in}}{\pgfqpoint{1.209506in}{2.173925in}}{\pgfqpoint{1.209506in}{2.182162in}}%
\pgfpathcurveto{\pgfqpoint{1.209506in}{2.190398in}}{\pgfqpoint{1.206234in}{2.198298in}}{\pgfqpoint{1.200410in}{2.204122in}}%
\pgfpathcurveto{\pgfqpoint{1.194586in}{2.209946in}}{\pgfqpoint{1.186686in}{2.213218in}}{\pgfqpoint{1.178449in}{2.213218in}}%
\pgfpathcurveto{\pgfqpoint{1.170213in}{2.213218in}}{\pgfqpoint{1.162313in}{2.209946in}}{\pgfqpoint{1.156489in}{2.204122in}}%
\pgfpathcurveto{\pgfqpoint{1.150665in}{2.198298in}}{\pgfqpoint{1.147393in}{2.190398in}}{\pgfqpoint{1.147393in}{2.182162in}}%
\pgfpathcurveto{\pgfqpoint{1.147393in}{2.173925in}}{\pgfqpoint{1.150665in}{2.166025in}}{\pgfqpoint{1.156489in}{2.160202in}}%
\pgfpathcurveto{\pgfqpoint{1.162313in}{2.154378in}}{\pgfqpoint{1.170213in}{2.151105in}}{\pgfqpoint{1.178449in}{2.151105in}}%
\pgfpathclose%
\pgfusepath{stroke,fill}%
\end{pgfscope}%
\begin{pgfscope}%
\pgfpathrectangle{\pgfqpoint{0.100000in}{0.212622in}}{\pgfqpoint{3.696000in}{3.696000in}}%
\pgfusepath{clip}%
\pgfsetbuttcap%
\pgfsetroundjoin%
\definecolor{currentfill}{rgb}{0.121569,0.466667,0.705882}%
\pgfsetfillcolor{currentfill}%
\pgfsetfillopacity{0.879190}%
\pgfsetlinewidth{1.003750pt}%
\definecolor{currentstroke}{rgb}{0.121569,0.466667,0.705882}%
\pgfsetstrokecolor{currentstroke}%
\pgfsetstrokeopacity{0.879190}%
\pgfsetdash{}{0pt}%
\pgfpathmoveto{\pgfqpoint{2.321087in}{1.467658in}}%
\pgfpathcurveto{\pgfqpoint{2.329323in}{1.467658in}}{\pgfqpoint{2.337223in}{1.470930in}}{\pgfqpoint{2.343047in}{1.476754in}}%
\pgfpathcurveto{\pgfqpoint{2.348871in}{1.482578in}}{\pgfqpoint{2.352144in}{1.490478in}}{\pgfqpoint{2.352144in}{1.498715in}}%
\pgfpathcurveto{\pgfqpoint{2.352144in}{1.506951in}}{\pgfqpoint{2.348871in}{1.514851in}}{\pgfqpoint{2.343047in}{1.520675in}}%
\pgfpathcurveto{\pgfqpoint{2.337223in}{1.526499in}}{\pgfqpoint{2.329323in}{1.529771in}}{\pgfqpoint{2.321087in}{1.529771in}}%
\pgfpathcurveto{\pgfqpoint{2.312851in}{1.529771in}}{\pgfqpoint{2.304951in}{1.526499in}}{\pgfqpoint{2.299127in}{1.520675in}}%
\pgfpathcurveto{\pgfqpoint{2.293303in}{1.514851in}}{\pgfqpoint{2.290031in}{1.506951in}}{\pgfqpoint{2.290031in}{1.498715in}}%
\pgfpathcurveto{\pgfqpoint{2.290031in}{1.490478in}}{\pgfqpoint{2.293303in}{1.482578in}}{\pgfqpoint{2.299127in}{1.476754in}}%
\pgfpathcurveto{\pgfqpoint{2.304951in}{1.470930in}}{\pgfqpoint{2.312851in}{1.467658in}}{\pgfqpoint{2.321087in}{1.467658in}}%
\pgfpathclose%
\pgfusepath{stroke,fill}%
\end{pgfscope}%
\begin{pgfscope}%
\pgfpathrectangle{\pgfqpoint{0.100000in}{0.212622in}}{\pgfqpoint{3.696000in}{3.696000in}}%
\pgfusepath{clip}%
\pgfsetbuttcap%
\pgfsetroundjoin%
\definecolor{currentfill}{rgb}{0.121569,0.466667,0.705882}%
\pgfsetfillcolor{currentfill}%
\pgfsetfillopacity{0.880408}%
\pgfsetlinewidth{1.003750pt}%
\definecolor{currentstroke}{rgb}{0.121569,0.466667,0.705882}%
\pgfsetstrokecolor{currentstroke}%
\pgfsetstrokeopacity{0.880408}%
\pgfsetdash{}{0pt}%
\pgfpathmoveto{\pgfqpoint{2.321834in}{1.466191in}}%
\pgfpathcurveto{\pgfqpoint{2.330070in}{1.466191in}}{\pgfqpoint{2.337970in}{1.469463in}}{\pgfqpoint{2.343794in}{1.475287in}}%
\pgfpathcurveto{\pgfqpoint{2.349618in}{1.481111in}}{\pgfqpoint{2.352891in}{1.489011in}}{\pgfqpoint{2.352891in}{1.497247in}}%
\pgfpathcurveto{\pgfqpoint{2.352891in}{1.505484in}}{\pgfqpoint{2.349618in}{1.513384in}}{\pgfqpoint{2.343794in}{1.519208in}}%
\pgfpathcurveto{\pgfqpoint{2.337970in}{1.525031in}}{\pgfqpoint{2.330070in}{1.528304in}}{\pgfqpoint{2.321834in}{1.528304in}}%
\pgfpathcurveto{\pgfqpoint{2.313598in}{1.528304in}}{\pgfqpoint{2.305698in}{1.525031in}}{\pgfqpoint{2.299874in}{1.519208in}}%
\pgfpathcurveto{\pgfqpoint{2.294050in}{1.513384in}}{\pgfqpoint{2.290778in}{1.505484in}}{\pgfqpoint{2.290778in}{1.497247in}}%
\pgfpathcurveto{\pgfqpoint{2.290778in}{1.489011in}}{\pgfqpoint{2.294050in}{1.481111in}}{\pgfqpoint{2.299874in}{1.475287in}}%
\pgfpathcurveto{\pgfqpoint{2.305698in}{1.469463in}}{\pgfqpoint{2.313598in}{1.466191in}}{\pgfqpoint{2.321834in}{1.466191in}}%
\pgfpathclose%
\pgfusepath{stroke,fill}%
\end{pgfscope}%
\begin{pgfscope}%
\pgfpathrectangle{\pgfqpoint{0.100000in}{0.212622in}}{\pgfqpoint{3.696000in}{3.696000in}}%
\pgfusepath{clip}%
\pgfsetbuttcap%
\pgfsetroundjoin%
\definecolor{currentfill}{rgb}{0.121569,0.466667,0.705882}%
\pgfsetfillcolor{currentfill}%
\pgfsetfillopacity{0.881081}%
\pgfsetlinewidth{1.003750pt}%
\definecolor{currentstroke}{rgb}{0.121569,0.466667,0.705882}%
\pgfsetstrokecolor{currentstroke}%
\pgfsetstrokeopacity{0.881081}%
\pgfsetdash{}{0pt}%
\pgfpathmoveto{\pgfqpoint{2.322369in}{1.465473in}}%
\pgfpathcurveto{\pgfqpoint{2.330605in}{1.465473in}}{\pgfqpoint{2.338505in}{1.468745in}}{\pgfqpoint{2.344329in}{1.474569in}}%
\pgfpathcurveto{\pgfqpoint{2.350153in}{1.480393in}}{\pgfqpoint{2.353425in}{1.488293in}}{\pgfqpoint{2.353425in}{1.496529in}}%
\pgfpathcurveto{\pgfqpoint{2.353425in}{1.504765in}}{\pgfqpoint{2.350153in}{1.512665in}}{\pgfqpoint{2.344329in}{1.518489in}}%
\pgfpathcurveto{\pgfqpoint{2.338505in}{1.524313in}}{\pgfqpoint{2.330605in}{1.527586in}}{\pgfqpoint{2.322369in}{1.527586in}}%
\pgfpathcurveto{\pgfqpoint{2.314133in}{1.527586in}}{\pgfqpoint{2.306232in}{1.524313in}}{\pgfqpoint{2.300409in}{1.518489in}}%
\pgfpathcurveto{\pgfqpoint{2.294585in}{1.512665in}}{\pgfqpoint{2.291312in}{1.504765in}}{\pgfqpoint{2.291312in}{1.496529in}}%
\pgfpathcurveto{\pgfqpoint{2.291312in}{1.488293in}}{\pgfqpoint{2.294585in}{1.480393in}}{\pgfqpoint{2.300409in}{1.474569in}}%
\pgfpathcurveto{\pgfqpoint{2.306232in}{1.468745in}}{\pgfqpoint{2.314133in}{1.465473in}}{\pgfqpoint{2.322369in}{1.465473in}}%
\pgfpathclose%
\pgfusepath{stroke,fill}%
\end{pgfscope}%
\begin{pgfscope}%
\pgfpathrectangle{\pgfqpoint{0.100000in}{0.212622in}}{\pgfqpoint{3.696000in}{3.696000in}}%
\pgfusepath{clip}%
\pgfsetbuttcap%
\pgfsetroundjoin%
\definecolor{currentfill}{rgb}{0.121569,0.466667,0.705882}%
\pgfsetfillcolor{currentfill}%
\pgfsetfillopacity{0.881439}%
\pgfsetlinewidth{1.003750pt}%
\definecolor{currentstroke}{rgb}{0.121569,0.466667,0.705882}%
\pgfsetstrokecolor{currentstroke}%
\pgfsetstrokeopacity{0.881439}%
\pgfsetdash{}{0pt}%
\pgfpathmoveto{\pgfqpoint{2.322606in}{1.464967in}}%
\pgfpathcurveto{\pgfqpoint{2.330842in}{1.464967in}}{\pgfqpoint{2.338742in}{1.468240in}}{\pgfqpoint{2.344566in}{1.474064in}}%
\pgfpathcurveto{\pgfqpoint{2.350390in}{1.479888in}}{\pgfqpoint{2.353662in}{1.487788in}}{\pgfqpoint{2.353662in}{1.496024in}}%
\pgfpathcurveto{\pgfqpoint{2.353662in}{1.504260in}}{\pgfqpoint{2.350390in}{1.512160in}}{\pgfqpoint{2.344566in}{1.517984in}}%
\pgfpathcurveto{\pgfqpoint{2.338742in}{1.523808in}}{\pgfqpoint{2.330842in}{1.527080in}}{\pgfqpoint{2.322606in}{1.527080in}}%
\pgfpathcurveto{\pgfqpoint{2.314369in}{1.527080in}}{\pgfqpoint{2.306469in}{1.523808in}}{\pgfqpoint{2.300645in}{1.517984in}}%
\pgfpathcurveto{\pgfqpoint{2.294822in}{1.512160in}}{\pgfqpoint{2.291549in}{1.504260in}}{\pgfqpoint{2.291549in}{1.496024in}}%
\pgfpathcurveto{\pgfqpoint{2.291549in}{1.487788in}}{\pgfqpoint{2.294822in}{1.479888in}}{\pgfqpoint{2.300645in}{1.474064in}}%
\pgfpathcurveto{\pgfqpoint{2.306469in}{1.468240in}}{\pgfqpoint{2.314369in}{1.464967in}}{\pgfqpoint{2.322606in}{1.464967in}}%
\pgfpathclose%
\pgfusepath{stroke,fill}%
\end{pgfscope}%
\begin{pgfscope}%
\pgfpathrectangle{\pgfqpoint{0.100000in}{0.212622in}}{\pgfqpoint{3.696000in}{3.696000in}}%
\pgfusepath{clip}%
\pgfsetbuttcap%
\pgfsetroundjoin%
\definecolor{currentfill}{rgb}{0.121569,0.466667,0.705882}%
\pgfsetfillcolor{currentfill}%
\pgfsetfillopacity{0.882153}%
\pgfsetlinewidth{1.003750pt}%
\definecolor{currentstroke}{rgb}{0.121569,0.466667,0.705882}%
\pgfsetstrokecolor{currentstroke}%
\pgfsetstrokeopacity{0.882153}%
\pgfsetdash{}{0pt}%
\pgfpathmoveto{\pgfqpoint{2.323297in}{1.464364in}}%
\pgfpathcurveto{\pgfqpoint{2.331533in}{1.464364in}}{\pgfqpoint{2.339434in}{1.467637in}}{\pgfqpoint{2.345257in}{1.473461in}}%
\pgfpathcurveto{\pgfqpoint{2.351081in}{1.479284in}}{\pgfqpoint{2.354354in}{1.487185in}}{\pgfqpoint{2.354354in}{1.495421in}}%
\pgfpathcurveto{\pgfqpoint{2.354354in}{1.503657in}}{\pgfqpoint{2.351081in}{1.511557in}}{\pgfqpoint{2.345257in}{1.517381in}}%
\pgfpathcurveto{\pgfqpoint{2.339434in}{1.523205in}}{\pgfqpoint{2.331533in}{1.526477in}}{\pgfqpoint{2.323297in}{1.526477in}}%
\pgfpathcurveto{\pgfqpoint{2.315061in}{1.526477in}}{\pgfqpoint{2.307161in}{1.523205in}}{\pgfqpoint{2.301337in}{1.517381in}}%
\pgfpathcurveto{\pgfqpoint{2.295513in}{1.511557in}}{\pgfqpoint{2.292241in}{1.503657in}}{\pgfqpoint{2.292241in}{1.495421in}}%
\pgfpathcurveto{\pgfqpoint{2.292241in}{1.487185in}}{\pgfqpoint{2.295513in}{1.479284in}}{\pgfqpoint{2.301337in}{1.473461in}}%
\pgfpathcurveto{\pgfqpoint{2.307161in}{1.467637in}}{\pgfqpoint{2.315061in}{1.464364in}}{\pgfqpoint{2.323297in}{1.464364in}}%
\pgfpathclose%
\pgfusepath{stroke,fill}%
\end{pgfscope}%
\begin{pgfscope}%
\pgfpathrectangle{\pgfqpoint{0.100000in}{0.212622in}}{\pgfqpoint{3.696000in}{3.696000in}}%
\pgfusepath{clip}%
\pgfsetbuttcap%
\pgfsetroundjoin%
\definecolor{currentfill}{rgb}{0.121569,0.466667,0.705882}%
\pgfsetfillcolor{currentfill}%
\pgfsetfillopacity{0.882468}%
\pgfsetlinewidth{1.003750pt}%
\definecolor{currentstroke}{rgb}{0.121569,0.466667,0.705882}%
\pgfsetstrokecolor{currentstroke}%
\pgfsetstrokeopacity{0.882468}%
\pgfsetdash{}{0pt}%
\pgfpathmoveto{\pgfqpoint{2.323487in}{1.463447in}}%
\pgfpathcurveto{\pgfqpoint{2.331723in}{1.463447in}}{\pgfqpoint{2.339623in}{1.466719in}}{\pgfqpoint{2.345447in}{1.472543in}}%
\pgfpathcurveto{\pgfqpoint{2.351271in}{1.478367in}}{\pgfqpoint{2.354543in}{1.486267in}}{\pgfqpoint{2.354543in}{1.494504in}}%
\pgfpathcurveto{\pgfqpoint{2.354543in}{1.502740in}}{\pgfqpoint{2.351271in}{1.510640in}}{\pgfqpoint{2.345447in}{1.516464in}}%
\pgfpathcurveto{\pgfqpoint{2.339623in}{1.522288in}}{\pgfqpoint{2.331723in}{1.525560in}}{\pgfqpoint{2.323487in}{1.525560in}}%
\pgfpathcurveto{\pgfqpoint{2.315250in}{1.525560in}}{\pgfqpoint{2.307350in}{1.522288in}}{\pgfqpoint{2.301526in}{1.516464in}}%
\pgfpathcurveto{\pgfqpoint{2.295703in}{1.510640in}}{\pgfqpoint{2.292430in}{1.502740in}}{\pgfqpoint{2.292430in}{1.494504in}}%
\pgfpathcurveto{\pgfqpoint{2.292430in}{1.486267in}}{\pgfqpoint{2.295703in}{1.478367in}}{\pgfqpoint{2.301526in}{1.472543in}}%
\pgfpathcurveto{\pgfqpoint{2.307350in}{1.466719in}}{\pgfqpoint{2.315250in}{1.463447in}}{\pgfqpoint{2.323487in}{1.463447in}}%
\pgfpathclose%
\pgfusepath{stroke,fill}%
\end{pgfscope}%
\begin{pgfscope}%
\pgfpathrectangle{\pgfqpoint{0.100000in}{0.212622in}}{\pgfqpoint{3.696000in}{3.696000in}}%
\pgfusepath{clip}%
\pgfsetbuttcap%
\pgfsetroundjoin%
\definecolor{currentfill}{rgb}{0.121569,0.466667,0.705882}%
\pgfsetfillcolor{currentfill}%
\pgfsetfillopacity{0.883185}%
\pgfsetlinewidth{1.003750pt}%
\definecolor{currentstroke}{rgb}{0.121569,0.466667,0.705882}%
\pgfsetstrokecolor{currentstroke}%
\pgfsetstrokeopacity{0.883185}%
\pgfsetdash{}{0pt}%
\pgfpathmoveto{\pgfqpoint{2.324079in}{1.463361in}}%
\pgfpathcurveto{\pgfqpoint{2.332316in}{1.463361in}}{\pgfqpoint{2.340216in}{1.466633in}}{\pgfqpoint{2.346040in}{1.472457in}}%
\pgfpathcurveto{\pgfqpoint{2.351864in}{1.478281in}}{\pgfqpoint{2.355136in}{1.486181in}}{\pgfqpoint{2.355136in}{1.494417in}}%
\pgfpathcurveto{\pgfqpoint{2.355136in}{1.502653in}}{\pgfqpoint{2.351864in}{1.510553in}}{\pgfqpoint{2.346040in}{1.516377in}}%
\pgfpathcurveto{\pgfqpoint{2.340216in}{1.522201in}}{\pgfqpoint{2.332316in}{1.525474in}}{\pgfqpoint{2.324079in}{1.525474in}}%
\pgfpathcurveto{\pgfqpoint{2.315843in}{1.525474in}}{\pgfqpoint{2.307943in}{1.522201in}}{\pgfqpoint{2.302119in}{1.516377in}}%
\pgfpathcurveto{\pgfqpoint{2.296295in}{1.510553in}}{\pgfqpoint{2.293023in}{1.502653in}}{\pgfqpoint{2.293023in}{1.494417in}}%
\pgfpathcurveto{\pgfqpoint{2.293023in}{1.486181in}}{\pgfqpoint{2.296295in}{1.478281in}}{\pgfqpoint{2.302119in}{1.472457in}}%
\pgfpathcurveto{\pgfqpoint{2.307943in}{1.466633in}}{\pgfqpoint{2.315843in}{1.463361in}}{\pgfqpoint{2.324079in}{1.463361in}}%
\pgfpathclose%
\pgfusepath{stroke,fill}%
\end{pgfscope}%
\begin{pgfscope}%
\pgfpathrectangle{\pgfqpoint{0.100000in}{0.212622in}}{\pgfqpoint{3.696000in}{3.696000in}}%
\pgfusepath{clip}%
\pgfsetbuttcap%
\pgfsetroundjoin%
\definecolor{currentfill}{rgb}{0.121569,0.466667,0.705882}%
\pgfsetfillcolor{currentfill}%
\pgfsetfillopacity{0.883551}%
\pgfsetlinewidth{1.003750pt}%
\definecolor{currentstroke}{rgb}{0.121569,0.466667,0.705882}%
\pgfsetstrokecolor{currentstroke}%
\pgfsetstrokeopacity{0.883551}%
\pgfsetdash{}{0pt}%
\pgfpathmoveto{\pgfqpoint{2.324347in}{1.463096in}}%
\pgfpathcurveto{\pgfqpoint{2.332584in}{1.463096in}}{\pgfqpoint{2.340484in}{1.466368in}}{\pgfqpoint{2.346307in}{1.472192in}}%
\pgfpathcurveto{\pgfqpoint{2.352131in}{1.478016in}}{\pgfqpoint{2.355404in}{1.485916in}}{\pgfqpoint{2.355404in}{1.494153in}}%
\pgfpathcurveto{\pgfqpoint{2.355404in}{1.502389in}}{\pgfqpoint{2.352131in}{1.510289in}}{\pgfqpoint{2.346307in}{1.516113in}}%
\pgfpathcurveto{\pgfqpoint{2.340484in}{1.521937in}}{\pgfqpoint{2.332584in}{1.525209in}}{\pgfqpoint{2.324347in}{1.525209in}}%
\pgfpathcurveto{\pgfqpoint{2.316111in}{1.525209in}}{\pgfqpoint{2.308211in}{1.521937in}}{\pgfqpoint{2.302387in}{1.516113in}}%
\pgfpathcurveto{\pgfqpoint{2.296563in}{1.510289in}}{\pgfqpoint{2.293291in}{1.502389in}}{\pgfqpoint{2.293291in}{1.494153in}}%
\pgfpathcurveto{\pgfqpoint{2.293291in}{1.485916in}}{\pgfqpoint{2.296563in}{1.478016in}}{\pgfqpoint{2.302387in}{1.472192in}}%
\pgfpathcurveto{\pgfqpoint{2.308211in}{1.466368in}}{\pgfqpoint{2.316111in}{1.463096in}}{\pgfqpoint{2.324347in}{1.463096in}}%
\pgfpathclose%
\pgfusepath{stroke,fill}%
\end{pgfscope}%
\begin{pgfscope}%
\pgfpathrectangle{\pgfqpoint{0.100000in}{0.212622in}}{\pgfqpoint{3.696000in}{3.696000in}}%
\pgfusepath{clip}%
\pgfsetbuttcap%
\pgfsetroundjoin%
\definecolor{currentfill}{rgb}{0.121569,0.466667,0.705882}%
\pgfsetfillcolor{currentfill}%
\pgfsetfillopacity{0.883717}%
\pgfsetlinewidth{1.003750pt}%
\definecolor{currentstroke}{rgb}{0.121569,0.466667,0.705882}%
\pgfsetstrokecolor{currentstroke}%
\pgfsetstrokeopacity{0.883717}%
\pgfsetdash{}{0pt}%
\pgfpathmoveto{\pgfqpoint{2.324409in}{1.462701in}}%
\pgfpathcurveto{\pgfqpoint{2.332645in}{1.462701in}}{\pgfqpoint{2.340545in}{1.465973in}}{\pgfqpoint{2.346369in}{1.471797in}}%
\pgfpathcurveto{\pgfqpoint{2.352193in}{1.477621in}}{\pgfqpoint{2.355466in}{1.485521in}}{\pgfqpoint{2.355466in}{1.493758in}}%
\pgfpathcurveto{\pgfqpoint{2.355466in}{1.501994in}}{\pgfqpoint{2.352193in}{1.509894in}}{\pgfqpoint{2.346369in}{1.515718in}}%
\pgfpathcurveto{\pgfqpoint{2.340545in}{1.521542in}}{\pgfqpoint{2.332645in}{1.524814in}}{\pgfqpoint{2.324409in}{1.524814in}}%
\pgfpathcurveto{\pgfqpoint{2.316173in}{1.524814in}}{\pgfqpoint{2.308273in}{1.521542in}}{\pgfqpoint{2.302449in}{1.515718in}}%
\pgfpathcurveto{\pgfqpoint{2.296625in}{1.509894in}}{\pgfqpoint{2.293353in}{1.501994in}}{\pgfqpoint{2.293353in}{1.493758in}}%
\pgfpathcurveto{\pgfqpoint{2.293353in}{1.485521in}}{\pgfqpoint{2.296625in}{1.477621in}}{\pgfqpoint{2.302449in}{1.471797in}}%
\pgfpathcurveto{\pgfqpoint{2.308273in}{1.465973in}}{\pgfqpoint{2.316173in}{1.462701in}}{\pgfqpoint{2.324409in}{1.462701in}}%
\pgfpathclose%
\pgfusepath{stroke,fill}%
\end{pgfscope}%
\begin{pgfscope}%
\pgfpathrectangle{\pgfqpoint{0.100000in}{0.212622in}}{\pgfqpoint{3.696000in}{3.696000in}}%
\pgfusepath{clip}%
\pgfsetbuttcap%
\pgfsetroundjoin%
\definecolor{currentfill}{rgb}{0.121569,0.466667,0.705882}%
\pgfsetfillcolor{currentfill}%
\pgfsetfillopacity{0.883866}%
\pgfsetlinewidth{1.003750pt}%
\definecolor{currentstroke}{rgb}{0.121569,0.466667,0.705882}%
\pgfsetstrokecolor{currentstroke}%
\pgfsetstrokeopacity{0.883866}%
\pgfsetdash{}{0pt}%
\pgfpathmoveto{\pgfqpoint{1.228095in}{2.121560in}}%
\pgfpathcurveto{\pgfqpoint{1.236331in}{2.121560in}}{\pgfqpoint{1.244231in}{2.124832in}}{\pgfqpoint{1.250055in}{2.130656in}}%
\pgfpathcurveto{\pgfqpoint{1.255879in}{2.136480in}}{\pgfqpoint{1.259151in}{2.144380in}}{\pgfqpoint{1.259151in}{2.152616in}}%
\pgfpathcurveto{\pgfqpoint{1.259151in}{2.160853in}}{\pgfqpoint{1.255879in}{2.168753in}}{\pgfqpoint{1.250055in}{2.174577in}}%
\pgfpathcurveto{\pgfqpoint{1.244231in}{2.180401in}}{\pgfqpoint{1.236331in}{2.183673in}}{\pgfqpoint{1.228095in}{2.183673in}}%
\pgfpathcurveto{\pgfqpoint{1.219858in}{2.183673in}}{\pgfqpoint{1.211958in}{2.180401in}}{\pgfqpoint{1.206134in}{2.174577in}}%
\pgfpathcurveto{\pgfqpoint{1.200310in}{2.168753in}}{\pgfqpoint{1.197038in}{2.160853in}}{\pgfqpoint{1.197038in}{2.152616in}}%
\pgfpathcurveto{\pgfqpoint{1.197038in}{2.144380in}}{\pgfqpoint{1.200310in}{2.136480in}}{\pgfqpoint{1.206134in}{2.130656in}}%
\pgfpathcurveto{\pgfqpoint{1.211958in}{2.124832in}}{\pgfqpoint{1.219858in}{2.121560in}}{\pgfqpoint{1.228095in}{2.121560in}}%
\pgfpathclose%
\pgfusepath{stroke,fill}%
\end{pgfscope}%
\begin{pgfscope}%
\pgfpathrectangle{\pgfqpoint{0.100000in}{0.212622in}}{\pgfqpoint{3.696000in}{3.696000in}}%
\pgfusepath{clip}%
\pgfsetbuttcap%
\pgfsetroundjoin%
\definecolor{currentfill}{rgb}{0.121569,0.466667,0.705882}%
\pgfsetfillcolor{currentfill}%
\pgfsetfillopacity{0.884243}%
\pgfsetlinewidth{1.003750pt}%
\definecolor{currentstroke}{rgb}{0.121569,0.466667,0.705882}%
\pgfsetstrokecolor{currentstroke}%
\pgfsetstrokeopacity{0.884243}%
\pgfsetdash{}{0pt}%
\pgfpathmoveto{\pgfqpoint{2.324998in}{1.460757in}}%
\pgfpathcurveto{\pgfqpoint{2.333234in}{1.460757in}}{\pgfqpoint{2.341135in}{1.464029in}}{\pgfqpoint{2.346958in}{1.469853in}}%
\pgfpathcurveto{\pgfqpoint{2.352782in}{1.475677in}}{\pgfqpoint{2.356055in}{1.483577in}}{\pgfqpoint{2.356055in}{1.491814in}}%
\pgfpathcurveto{\pgfqpoint{2.356055in}{1.500050in}}{\pgfqpoint{2.352782in}{1.507950in}}{\pgfqpoint{2.346958in}{1.513774in}}%
\pgfpathcurveto{\pgfqpoint{2.341135in}{1.519598in}}{\pgfqpoint{2.333234in}{1.522870in}}{\pgfqpoint{2.324998in}{1.522870in}}%
\pgfpathcurveto{\pgfqpoint{2.316762in}{1.522870in}}{\pgfqpoint{2.308862in}{1.519598in}}{\pgfqpoint{2.303038in}{1.513774in}}%
\pgfpathcurveto{\pgfqpoint{2.297214in}{1.507950in}}{\pgfqpoint{2.293942in}{1.500050in}}{\pgfqpoint{2.293942in}{1.491814in}}%
\pgfpathcurveto{\pgfqpoint{2.293942in}{1.483577in}}{\pgfqpoint{2.297214in}{1.475677in}}{\pgfqpoint{2.303038in}{1.469853in}}%
\pgfpathcurveto{\pgfqpoint{2.308862in}{1.464029in}}{\pgfqpoint{2.316762in}{1.460757in}}{\pgfqpoint{2.324998in}{1.460757in}}%
\pgfpathclose%
\pgfusepath{stroke,fill}%
\end{pgfscope}%
\begin{pgfscope}%
\pgfpathrectangle{\pgfqpoint{0.100000in}{0.212622in}}{\pgfqpoint{3.696000in}{3.696000in}}%
\pgfusepath{clip}%
\pgfsetbuttcap%
\pgfsetroundjoin%
\definecolor{currentfill}{rgb}{0.121569,0.466667,0.705882}%
\pgfsetfillcolor{currentfill}%
\pgfsetfillopacity{0.885236}%
\pgfsetlinewidth{1.003750pt}%
\definecolor{currentstroke}{rgb}{0.121569,0.466667,0.705882}%
\pgfsetstrokecolor{currentstroke}%
\pgfsetstrokeopacity{0.885236}%
\pgfsetdash{}{0pt}%
\pgfpathmoveto{\pgfqpoint{2.325615in}{1.459875in}}%
\pgfpathcurveto{\pgfqpoint{2.333851in}{1.459875in}}{\pgfqpoint{2.341751in}{1.463147in}}{\pgfqpoint{2.347575in}{1.468971in}}%
\pgfpathcurveto{\pgfqpoint{2.353399in}{1.474795in}}{\pgfqpoint{2.356671in}{1.482695in}}{\pgfqpoint{2.356671in}{1.490931in}}%
\pgfpathcurveto{\pgfqpoint{2.356671in}{1.499167in}}{\pgfqpoint{2.353399in}{1.507067in}}{\pgfqpoint{2.347575in}{1.512891in}}%
\pgfpathcurveto{\pgfqpoint{2.341751in}{1.518715in}}{\pgfqpoint{2.333851in}{1.521988in}}{\pgfqpoint{2.325615in}{1.521988in}}%
\pgfpathcurveto{\pgfqpoint{2.317378in}{1.521988in}}{\pgfqpoint{2.309478in}{1.518715in}}{\pgfqpoint{2.303654in}{1.512891in}}%
\pgfpathcurveto{\pgfqpoint{2.297830in}{1.507067in}}{\pgfqpoint{2.294558in}{1.499167in}}{\pgfqpoint{2.294558in}{1.490931in}}%
\pgfpathcurveto{\pgfqpoint{2.294558in}{1.482695in}}{\pgfqpoint{2.297830in}{1.474795in}}{\pgfqpoint{2.303654in}{1.468971in}}%
\pgfpathcurveto{\pgfqpoint{2.309478in}{1.463147in}}{\pgfqpoint{2.317378in}{1.459875in}}{\pgfqpoint{2.325615in}{1.459875in}}%
\pgfpathclose%
\pgfusepath{stroke,fill}%
\end{pgfscope}%
\begin{pgfscope}%
\pgfpathrectangle{\pgfqpoint{0.100000in}{0.212622in}}{\pgfqpoint{3.696000in}{3.696000in}}%
\pgfusepath{clip}%
\pgfsetbuttcap%
\pgfsetroundjoin%
\definecolor{currentfill}{rgb}{0.121569,0.466667,0.705882}%
\pgfsetfillcolor{currentfill}%
\pgfsetfillopacity{0.886274}%
\pgfsetlinewidth{1.003750pt}%
\definecolor{currentstroke}{rgb}{0.121569,0.466667,0.705882}%
\pgfsetstrokecolor{currentstroke}%
\pgfsetstrokeopacity{0.886274}%
\pgfsetdash{}{0pt}%
\pgfpathmoveto{\pgfqpoint{1.280557in}{2.082721in}}%
\pgfpathcurveto{\pgfqpoint{1.288793in}{2.082721in}}{\pgfqpoint{1.296693in}{2.085994in}}{\pgfqpoint{1.302517in}{2.091818in}}%
\pgfpathcurveto{\pgfqpoint{1.308341in}{2.097642in}}{\pgfqpoint{1.311613in}{2.105542in}}{\pgfqpoint{1.311613in}{2.113778in}}%
\pgfpathcurveto{\pgfqpoint{1.311613in}{2.122014in}}{\pgfqpoint{1.308341in}{2.129914in}}{\pgfqpoint{1.302517in}{2.135738in}}%
\pgfpathcurveto{\pgfqpoint{1.296693in}{2.141562in}}{\pgfqpoint{1.288793in}{2.144834in}}{\pgfqpoint{1.280557in}{2.144834in}}%
\pgfpathcurveto{\pgfqpoint{1.272321in}{2.144834in}}{\pgfqpoint{1.264421in}{2.141562in}}{\pgfqpoint{1.258597in}{2.135738in}}%
\pgfpathcurveto{\pgfqpoint{1.252773in}{2.129914in}}{\pgfqpoint{1.249500in}{2.122014in}}{\pgfqpoint{1.249500in}{2.113778in}}%
\pgfpathcurveto{\pgfqpoint{1.249500in}{2.105542in}}{\pgfqpoint{1.252773in}{2.097642in}}{\pgfqpoint{1.258597in}{2.091818in}}%
\pgfpathcurveto{\pgfqpoint{1.264421in}{2.085994in}}{\pgfqpoint{1.272321in}{2.082721in}}{\pgfqpoint{1.280557in}{2.082721in}}%
\pgfpathclose%
\pgfusepath{stroke,fill}%
\end{pgfscope}%
\begin{pgfscope}%
\pgfpathrectangle{\pgfqpoint{0.100000in}{0.212622in}}{\pgfqpoint{3.696000in}{3.696000in}}%
\pgfusepath{clip}%
\pgfsetbuttcap%
\pgfsetroundjoin%
\definecolor{currentfill}{rgb}{0.121569,0.466667,0.705882}%
\pgfsetfillcolor{currentfill}%
\pgfsetfillopacity{0.886770}%
\pgfsetlinewidth{1.003750pt}%
\definecolor{currentstroke}{rgb}{0.121569,0.466667,0.705882}%
\pgfsetstrokecolor{currentstroke}%
\pgfsetstrokeopacity{0.886770}%
\pgfsetdash{}{0pt}%
\pgfpathmoveto{\pgfqpoint{2.326783in}{1.458776in}}%
\pgfpathcurveto{\pgfqpoint{2.335019in}{1.458776in}}{\pgfqpoint{2.342919in}{1.462049in}}{\pgfqpoint{2.348743in}{1.467873in}}%
\pgfpathcurveto{\pgfqpoint{2.354567in}{1.473697in}}{\pgfqpoint{2.357839in}{1.481597in}}{\pgfqpoint{2.357839in}{1.489833in}}%
\pgfpathcurveto{\pgfqpoint{2.357839in}{1.498069in}}{\pgfqpoint{2.354567in}{1.505969in}}{\pgfqpoint{2.348743in}{1.511793in}}%
\pgfpathcurveto{\pgfqpoint{2.342919in}{1.517617in}}{\pgfqpoint{2.335019in}{1.520889in}}{\pgfqpoint{2.326783in}{1.520889in}}%
\pgfpathcurveto{\pgfqpoint{2.318546in}{1.520889in}}{\pgfqpoint{2.310646in}{1.517617in}}{\pgfqpoint{2.304822in}{1.511793in}}%
\pgfpathcurveto{\pgfqpoint{2.298998in}{1.505969in}}{\pgfqpoint{2.295726in}{1.498069in}}{\pgfqpoint{2.295726in}{1.489833in}}%
\pgfpathcurveto{\pgfqpoint{2.295726in}{1.481597in}}{\pgfqpoint{2.298998in}{1.473697in}}{\pgfqpoint{2.304822in}{1.467873in}}%
\pgfpathcurveto{\pgfqpoint{2.310646in}{1.462049in}}{\pgfqpoint{2.318546in}{1.458776in}}{\pgfqpoint{2.326783in}{1.458776in}}%
\pgfpathclose%
\pgfusepath{stroke,fill}%
\end{pgfscope}%
\begin{pgfscope}%
\pgfpathrectangle{\pgfqpoint{0.100000in}{0.212622in}}{\pgfqpoint{3.696000in}{3.696000in}}%
\pgfusepath{clip}%
\pgfsetbuttcap%
\pgfsetroundjoin%
\definecolor{currentfill}{rgb}{0.121569,0.466667,0.705882}%
\pgfsetfillcolor{currentfill}%
\pgfsetfillopacity{0.888638}%
\pgfsetlinewidth{1.003750pt}%
\definecolor{currentstroke}{rgb}{0.121569,0.466667,0.705882}%
\pgfsetstrokecolor{currentstroke}%
\pgfsetstrokeopacity{0.888638}%
\pgfsetdash{}{0pt}%
\pgfpathmoveto{\pgfqpoint{2.327377in}{1.457620in}}%
\pgfpathcurveto{\pgfqpoint{2.335613in}{1.457620in}}{\pgfqpoint{2.343513in}{1.460892in}}{\pgfqpoint{2.349337in}{1.466716in}}%
\pgfpathcurveto{\pgfqpoint{2.355161in}{1.472540in}}{\pgfqpoint{2.358433in}{1.480440in}}{\pgfqpoint{2.358433in}{1.488676in}}%
\pgfpathcurveto{\pgfqpoint{2.358433in}{1.496913in}}{\pgfqpoint{2.355161in}{1.504813in}}{\pgfqpoint{2.349337in}{1.510637in}}%
\pgfpathcurveto{\pgfqpoint{2.343513in}{1.516461in}}{\pgfqpoint{2.335613in}{1.519733in}}{\pgfqpoint{2.327377in}{1.519733in}}%
\pgfpathcurveto{\pgfqpoint{2.319141in}{1.519733in}}{\pgfqpoint{2.311241in}{1.516461in}}{\pgfqpoint{2.305417in}{1.510637in}}%
\pgfpathcurveto{\pgfqpoint{2.299593in}{1.504813in}}{\pgfqpoint{2.296320in}{1.496913in}}{\pgfqpoint{2.296320in}{1.488676in}}%
\pgfpathcurveto{\pgfqpoint{2.296320in}{1.480440in}}{\pgfqpoint{2.299593in}{1.472540in}}{\pgfqpoint{2.305417in}{1.466716in}}%
\pgfpathcurveto{\pgfqpoint{2.311241in}{1.460892in}}{\pgfqpoint{2.319141in}{1.457620in}}{\pgfqpoint{2.327377in}{1.457620in}}%
\pgfpathclose%
\pgfusepath{stroke,fill}%
\end{pgfscope}%
\begin{pgfscope}%
\pgfpathrectangle{\pgfqpoint{0.100000in}{0.212622in}}{\pgfqpoint{3.696000in}{3.696000in}}%
\pgfusepath{clip}%
\pgfsetbuttcap%
\pgfsetroundjoin%
\definecolor{currentfill}{rgb}{0.121569,0.466667,0.705882}%
\pgfsetfillcolor{currentfill}%
\pgfsetfillopacity{0.889566}%
\pgfsetlinewidth{1.003750pt}%
\definecolor{currentstroke}{rgb}{0.121569,0.466667,0.705882}%
\pgfsetstrokecolor{currentstroke}%
\pgfsetstrokeopacity{0.889566}%
\pgfsetdash{}{0pt}%
\pgfpathmoveto{\pgfqpoint{2.328035in}{1.456519in}}%
\pgfpathcurveto{\pgfqpoint{2.336272in}{1.456519in}}{\pgfqpoint{2.344172in}{1.459791in}}{\pgfqpoint{2.349996in}{1.465615in}}%
\pgfpathcurveto{\pgfqpoint{2.355820in}{1.471439in}}{\pgfqpoint{2.359092in}{1.479339in}}{\pgfqpoint{2.359092in}{1.487575in}}%
\pgfpathcurveto{\pgfqpoint{2.359092in}{1.495811in}}{\pgfqpoint{2.355820in}{1.503711in}}{\pgfqpoint{2.349996in}{1.509535in}}%
\pgfpathcurveto{\pgfqpoint{2.344172in}{1.515359in}}{\pgfqpoint{2.336272in}{1.518632in}}{\pgfqpoint{2.328035in}{1.518632in}}%
\pgfpathcurveto{\pgfqpoint{2.319799in}{1.518632in}}{\pgfqpoint{2.311899in}{1.515359in}}{\pgfqpoint{2.306075in}{1.509535in}}%
\pgfpathcurveto{\pgfqpoint{2.300251in}{1.503711in}}{\pgfqpoint{2.296979in}{1.495811in}}{\pgfqpoint{2.296979in}{1.487575in}}%
\pgfpathcurveto{\pgfqpoint{2.296979in}{1.479339in}}{\pgfqpoint{2.300251in}{1.471439in}}{\pgfqpoint{2.306075in}{1.465615in}}%
\pgfpathcurveto{\pgfqpoint{2.311899in}{1.459791in}}{\pgfqpoint{2.319799in}{1.456519in}}{\pgfqpoint{2.328035in}{1.456519in}}%
\pgfpathclose%
\pgfusepath{stroke,fill}%
\end{pgfscope}%
\begin{pgfscope}%
\pgfpathrectangle{\pgfqpoint{0.100000in}{0.212622in}}{\pgfqpoint{3.696000in}{3.696000in}}%
\pgfusepath{clip}%
\pgfsetbuttcap%
\pgfsetroundjoin%
\definecolor{currentfill}{rgb}{0.121569,0.466667,0.705882}%
\pgfsetfillcolor{currentfill}%
\pgfsetfillopacity{0.890567}%
\pgfsetlinewidth{1.003750pt}%
\definecolor{currentstroke}{rgb}{0.121569,0.466667,0.705882}%
\pgfsetstrokecolor{currentstroke}%
\pgfsetstrokeopacity{0.890567}%
\pgfsetdash{}{0pt}%
\pgfpathmoveto{\pgfqpoint{2.329131in}{1.454372in}}%
\pgfpathcurveto{\pgfqpoint{2.337368in}{1.454372in}}{\pgfqpoint{2.345268in}{1.457645in}}{\pgfqpoint{2.351092in}{1.463469in}}%
\pgfpathcurveto{\pgfqpoint{2.356916in}{1.469292in}}{\pgfqpoint{2.360188in}{1.477193in}}{\pgfqpoint{2.360188in}{1.485429in}}%
\pgfpathcurveto{\pgfqpoint{2.360188in}{1.493665in}}{\pgfqpoint{2.356916in}{1.501565in}}{\pgfqpoint{2.351092in}{1.507389in}}%
\pgfpathcurveto{\pgfqpoint{2.345268in}{1.513213in}}{\pgfqpoint{2.337368in}{1.516485in}}{\pgfqpoint{2.329131in}{1.516485in}}%
\pgfpathcurveto{\pgfqpoint{2.320895in}{1.516485in}}{\pgfqpoint{2.312995in}{1.513213in}}{\pgfqpoint{2.307171in}{1.507389in}}%
\pgfpathcurveto{\pgfqpoint{2.301347in}{1.501565in}}{\pgfqpoint{2.298075in}{1.493665in}}{\pgfqpoint{2.298075in}{1.485429in}}%
\pgfpathcurveto{\pgfqpoint{2.298075in}{1.477193in}}{\pgfqpoint{2.301347in}{1.469292in}}{\pgfqpoint{2.307171in}{1.463469in}}%
\pgfpathcurveto{\pgfqpoint{2.312995in}{1.457645in}}{\pgfqpoint{2.320895in}{1.454372in}}{\pgfqpoint{2.329131in}{1.454372in}}%
\pgfpathclose%
\pgfusepath{stroke,fill}%
\end{pgfscope}%
\begin{pgfscope}%
\pgfpathrectangle{\pgfqpoint{0.100000in}{0.212622in}}{\pgfqpoint{3.696000in}{3.696000in}}%
\pgfusepath{clip}%
\pgfsetbuttcap%
\pgfsetroundjoin%
\definecolor{currentfill}{rgb}{0.121569,0.466667,0.705882}%
\pgfsetfillcolor{currentfill}%
\pgfsetfillopacity{0.891319}%
\pgfsetlinewidth{1.003750pt}%
\definecolor{currentstroke}{rgb}{0.121569,0.466667,0.705882}%
\pgfsetstrokecolor{currentstroke}%
\pgfsetstrokeopacity{0.891319}%
\pgfsetdash{}{0pt}%
\pgfpathmoveto{\pgfqpoint{2.329724in}{1.454421in}}%
\pgfpathcurveto{\pgfqpoint{2.337960in}{1.454421in}}{\pgfqpoint{2.345860in}{1.457693in}}{\pgfqpoint{2.351684in}{1.463517in}}%
\pgfpathcurveto{\pgfqpoint{2.357508in}{1.469341in}}{\pgfqpoint{2.360781in}{1.477241in}}{\pgfqpoint{2.360781in}{1.485477in}}%
\pgfpathcurveto{\pgfqpoint{2.360781in}{1.493714in}}{\pgfqpoint{2.357508in}{1.501614in}}{\pgfqpoint{2.351684in}{1.507438in}}%
\pgfpathcurveto{\pgfqpoint{2.345860in}{1.513262in}}{\pgfqpoint{2.337960in}{1.516534in}}{\pgfqpoint{2.329724in}{1.516534in}}%
\pgfpathcurveto{\pgfqpoint{2.321488in}{1.516534in}}{\pgfqpoint{2.313588in}{1.513262in}}{\pgfqpoint{2.307764in}{1.507438in}}%
\pgfpathcurveto{\pgfqpoint{2.301940in}{1.501614in}}{\pgfqpoint{2.298668in}{1.493714in}}{\pgfqpoint{2.298668in}{1.485477in}}%
\pgfpathcurveto{\pgfqpoint{2.298668in}{1.477241in}}{\pgfqpoint{2.301940in}{1.469341in}}{\pgfqpoint{2.307764in}{1.463517in}}%
\pgfpathcurveto{\pgfqpoint{2.313588in}{1.457693in}}{\pgfqpoint{2.321488in}{1.454421in}}{\pgfqpoint{2.329724in}{1.454421in}}%
\pgfpathclose%
\pgfusepath{stroke,fill}%
\end{pgfscope}%
\begin{pgfscope}%
\pgfpathrectangle{\pgfqpoint{0.100000in}{0.212622in}}{\pgfqpoint{3.696000in}{3.696000in}}%
\pgfusepath{clip}%
\pgfsetbuttcap%
\pgfsetroundjoin%
\definecolor{currentfill}{rgb}{0.121569,0.466667,0.705882}%
\pgfsetfillcolor{currentfill}%
\pgfsetfillopacity{0.891823}%
\pgfsetlinewidth{1.003750pt}%
\definecolor{currentstroke}{rgb}{0.121569,0.466667,0.705882}%
\pgfsetstrokecolor{currentstroke}%
\pgfsetstrokeopacity{0.891823}%
\pgfsetdash{}{0pt}%
\pgfpathmoveto{\pgfqpoint{1.325194in}{2.050785in}}%
\pgfpathcurveto{\pgfqpoint{1.333430in}{2.050785in}}{\pgfqpoint{1.341330in}{2.054057in}}{\pgfqpoint{1.347154in}{2.059881in}}%
\pgfpathcurveto{\pgfqpoint{1.352978in}{2.065705in}}{\pgfqpoint{1.356250in}{2.073605in}}{\pgfqpoint{1.356250in}{2.081841in}}%
\pgfpathcurveto{\pgfqpoint{1.356250in}{2.090077in}}{\pgfqpoint{1.352978in}{2.097977in}}{\pgfqpoint{1.347154in}{2.103801in}}%
\pgfpathcurveto{\pgfqpoint{1.341330in}{2.109625in}}{\pgfqpoint{1.333430in}{2.112898in}}{\pgfqpoint{1.325194in}{2.112898in}}%
\pgfpathcurveto{\pgfqpoint{1.316957in}{2.112898in}}{\pgfqpoint{1.309057in}{2.109625in}}{\pgfqpoint{1.303233in}{2.103801in}}%
\pgfpathcurveto{\pgfqpoint{1.297409in}{2.097977in}}{\pgfqpoint{1.294137in}{2.090077in}}{\pgfqpoint{1.294137in}{2.081841in}}%
\pgfpathcurveto{\pgfqpoint{1.294137in}{2.073605in}}{\pgfqpoint{1.297409in}{2.065705in}}{\pgfqpoint{1.303233in}{2.059881in}}%
\pgfpathcurveto{\pgfqpoint{1.309057in}{2.054057in}}{\pgfqpoint{1.316957in}{2.050785in}}{\pgfqpoint{1.325194in}{2.050785in}}%
\pgfpathclose%
\pgfusepath{stroke,fill}%
\end{pgfscope}%
\begin{pgfscope}%
\pgfpathrectangle{\pgfqpoint{0.100000in}{0.212622in}}{\pgfqpoint{3.696000in}{3.696000in}}%
\pgfusepath{clip}%
\pgfsetbuttcap%
\pgfsetroundjoin%
\definecolor{currentfill}{rgb}{0.121569,0.466667,0.705882}%
\pgfsetfillcolor{currentfill}%
\pgfsetfillopacity{0.892399}%
\pgfsetlinewidth{1.003750pt}%
\definecolor{currentstroke}{rgb}{0.121569,0.466667,0.705882}%
\pgfsetstrokecolor{currentstroke}%
\pgfsetstrokeopacity{0.892399}%
\pgfsetdash{}{0pt}%
\pgfpathmoveto{\pgfqpoint{2.330262in}{1.454227in}}%
\pgfpathcurveto{\pgfqpoint{2.338499in}{1.454227in}}{\pgfqpoint{2.346399in}{1.457500in}}{\pgfqpoint{2.352222in}{1.463323in}}%
\pgfpathcurveto{\pgfqpoint{2.358046in}{1.469147in}}{\pgfqpoint{2.361319in}{1.477047in}}{\pgfqpoint{2.361319in}{1.485284in}}%
\pgfpathcurveto{\pgfqpoint{2.361319in}{1.493520in}}{\pgfqpoint{2.358046in}{1.501420in}}{\pgfqpoint{2.352222in}{1.507244in}}%
\pgfpathcurveto{\pgfqpoint{2.346399in}{1.513068in}}{\pgfqpoint{2.338499in}{1.516340in}}{\pgfqpoint{2.330262in}{1.516340in}}%
\pgfpathcurveto{\pgfqpoint{2.322026in}{1.516340in}}{\pgfqpoint{2.314126in}{1.513068in}}{\pgfqpoint{2.308302in}{1.507244in}}%
\pgfpathcurveto{\pgfqpoint{2.302478in}{1.501420in}}{\pgfqpoint{2.299206in}{1.493520in}}{\pgfqpoint{2.299206in}{1.485284in}}%
\pgfpathcurveto{\pgfqpoint{2.299206in}{1.477047in}}{\pgfqpoint{2.302478in}{1.469147in}}{\pgfqpoint{2.308302in}{1.463323in}}%
\pgfpathcurveto{\pgfqpoint{2.314126in}{1.457500in}}{\pgfqpoint{2.322026in}{1.454227in}}{\pgfqpoint{2.330262in}{1.454227in}}%
\pgfpathclose%
\pgfusepath{stroke,fill}%
\end{pgfscope}%
\begin{pgfscope}%
\pgfpathrectangle{\pgfqpoint{0.100000in}{0.212622in}}{\pgfqpoint{3.696000in}{3.696000in}}%
\pgfusepath{clip}%
\pgfsetbuttcap%
\pgfsetroundjoin%
\definecolor{currentfill}{rgb}{0.121569,0.466667,0.705882}%
\pgfsetfillcolor{currentfill}%
\pgfsetfillopacity{0.892824}%
\pgfsetlinewidth{1.003750pt}%
\definecolor{currentstroke}{rgb}{0.121569,0.466667,0.705882}%
\pgfsetstrokecolor{currentstroke}%
\pgfsetstrokeopacity{0.892824}%
\pgfsetdash{}{0pt}%
\pgfpathmoveto{\pgfqpoint{2.330660in}{1.453149in}}%
\pgfpathcurveto{\pgfqpoint{2.338897in}{1.453149in}}{\pgfqpoint{2.346797in}{1.456421in}}{\pgfqpoint{2.352621in}{1.462245in}}%
\pgfpathcurveto{\pgfqpoint{2.358445in}{1.468069in}}{\pgfqpoint{2.361717in}{1.475969in}}{\pgfqpoint{2.361717in}{1.484206in}}%
\pgfpathcurveto{\pgfqpoint{2.361717in}{1.492442in}}{\pgfqpoint{2.358445in}{1.500342in}}{\pgfqpoint{2.352621in}{1.506166in}}%
\pgfpathcurveto{\pgfqpoint{2.346797in}{1.511990in}}{\pgfqpoint{2.338897in}{1.515262in}}{\pgfqpoint{2.330660in}{1.515262in}}%
\pgfpathcurveto{\pgfqpoint{2.322424in}{1.515262in}}{\pgfqpoint{2.314524in}{1.511990in}}{\pgfqpoint{2.308700in}{1.506166in}}%
\pgfpathcurveto{\pgfqpoint{2.302876in}{1.500342in}}{\pgfqpoint{2.299604in}{1.492442in}}{\pgfqpoint{2.299604in}{1.484206in}}%
\pgfpathcurveto{\pgfqpoint{2.299604in}{1.475969in}}{\pgfqpoint{2.302876in}{1.468069in}}{\pgfqpoint{2.308700in}{1.462245in}}%
\pgfpathcurveto{\pgfqpoint{2.314524in}{1.456421in}}{\pgfqpoint{2.322424in}{1.453149in}}{\pgfqpoint{2.330660in}{1.453149in}}%
\pgfpathclose%
\pgfusepath{stroke,fill}%
\end{pgfscope}%
\begin{pgfscope}%
\pgfpathrectangle{\pgfqpoint{0.100000in}{0.212622in}}{\pgfqpoint{3.696000in}{3.696000in}}%
\pgfusepath{clip}%
\pgfsetbuttcap%
\pgfsetroundjoin%
\definecolor{currentfill}{rgb}{0.121569,0.466667,0.705882}%
\pgfsetfillcolor{currentfill}%
\pgfsetfillopacity{0.893286}%
\pgfsetlinewidth{1.003750pt}%
\definecolor{currentstroke}{rgb}{0.121569,0.466667,0.705882}%
\pgfsetstrokecolor{currentstroke}%
\pgfsetstrokeopacity{0.893286}%
\pgfsetdash{}{0pt}%
\pgfpathmoveto{\pgfqpoint{2.331350in}{1.451167in}}%
\pgfpathcurveto{\pgfqpoint{2.339586in}{1.451167in}}{\pgfqpoint{2.347486in}{1.454440in}}{\pgfqpoint{2.353310in}{1.460264in}}%
\pgfpathcurveto{\pgfqpoint{2.359134in}{1.466088in}}{\pgfqpoint{2.362406in}{1.473988in}}{\pgfqpoint{2.362406in}{1.482224in}}%
\pgfpathcurveto{\pgfqpoint{2.362406in}{1.490460in}}{\pgfqpoint{2.359134in}{1.498360in}}{\pgfqpoint{2.353310in}{1.504184in}}%
\pgfpathcurveto{\pgfqpoint{2.347486in}{1.510008in}}{\pgfqpoint{2.339586in}{1.513280in}}{\pgfqpoint{2.331350in}{1.513280in}}%
\pgfpathcurveto{\pgfqpoint{2.323114in}{1.513280in}}{\pgfqpoint{2.315214in}{1.510008in}}{\pgfqpoint{2.309390in}{1.504184in}}%
\pgfpathcurveto{\pgfqpoint{2.303566in}{1.498360in}}{\pgfqpoint{2.300293in}{1.490460in}}{\pgfqpoint{2.300293in}{1.482224in}}%
\pgfpathcurveto{\pgfqpoint{2.300293in}{1.473988in}}{\pgfqpoint{2.303566in}{1.466088in}}{\pgfqpoint{2.309390in}{1.460264in}}%
\pgfpathcurveto{\pgfqpoint{2.315214in}{1.454440in}}{\pgfqpoint{2.323114in}{1.451167in}}{\pgfqpoint{2.331350in}{1.451167in}}%
\pgfpathclose%
\pgfusepath{stroke,fill}%
\end{pgfscope}%
\begin{pgfscope}%
\pgfpathrectangle{\pgfqpoint{0.100000in}{0.212622in}}{\pgfqpoint{3.696000in}{3.696000in}}%
\pgfusepath{clip}%
\pgfsetbuttcap%
\pgfsetroundjoin%
\definecolor{currentfill}{rgb}{0.121569,0.466667,0.705882}%
\pgfsetfillcolor{currentfill}%
\pgfsetfillopacity{0.894560}%
\pgfsetlinewidth{1.003750pt}%
\definecolor{currentstroke}{rgb}{0.121569,0.466667,0.705882}%
\pgfsetstrokecolor{currentstroke}%
\pgfsetstrokeopacity{0.894560}%
\pgfsetdash{}{0pt}%
\pgfpathmoveto{\pgfqpoint{2.332179in}{1.450648in}}%
\pgfpathcurveto{\pgfqpoint{2.340415in}{1.450648in}}{\pgfqpoint{2.348315in}{1.453920in}}{\pgfqpoint{2.354139in}{1.459744in}}%
\pgfpathcurveto{\pgfqpoint{2.359963in}{1.465568in}}{\pgfqpoint{2.363235in}{1.473468in}}{\pgfqpoint{2.363235in}{1.481704in}}%
\pgfpathcurveto{\pgfqpoint{2.363235in}{1.489941in}}{\pgfqpoint{2.359963in}{1.497841in}}{\pgfqpoint{2.354139in}{1.503665in}}%
\pgfpathcurveto{\pgfqpoint{2.348315in}{1.509489in}}{\pgfqpoint{2.340415in}{1.512761in}}{\pgfqpoint{2.332179in}{1.512761in}}%
\pgfpathcurveto{\pgfqpoint{2.323942in}{1.512761in}}{\pgfqpoint{2.316042in}{1.509489in}}{\pgfqpoint{2.310218in}{1.503665in}}%
\pgfpathcurveto{\pgfqpoint{2.304394in}{1.497841in}}{\pgfqpoint{2.301122in}{1.489941in}}{\pgfqpoint{2.301122in}{1.481704in}}%
\pgfpathcurveto{\pgfqpoint{2.301122in}{1.473468in}}{\pgfqpoint{2.304394in}{1.465568in}}{\pgfqpoint{2.310218in}{1.459744in}}%
\pgfpathcurveto{\pgfqpoint{2.316042in}{1.453920in}}{\pgfqpoint{2.323942in}{1.450648in}}{\pgfqpoint{2.332179in}{1.450648in}}%
\pgfpathclose%
\pgfusepath{stroke,fill}%
\end{pgfscope}%
\begin{pgfscope}%
\pgfpathrectangle{\pgfqpoint{0.100000in}{0.212622in}}{\pgfqpoint{3.696000in}{3.696000in}}%
\pgfusepath{clip}%
\pgfsetbuttcap%
\pgfsetroundjoin%
\definecolor{currentfill}{rgb}{0.121569,0.466667,0.705882}%
\pgfsetfillcolor{currentfill}%
\pgfsetfillopacity{0.895705}%
\pgfsetlinewidth{1.003750pt}%
\definecolor{currentstroke}{rgb}{0.121569,0.466667,0.705882}%
\pgfsetstrokecolor{currentstroke}%
\pgfsetstrokeopacity{0.895705}%
\pgfsetdash{}{0pt}%
\pgfpathmoveto{\pgfqpoint{1.370390in}{2.016370in}}%
\pgfpathcurveto{\pgfqpoint{1.378626in}{2.016370in}}{\pgfqpoint{1.386526in}{2.019642in}}{\pgfqpoint{1.392350in}{2.025466in}}%
\pgfpathcurveto{\pgfqpoint{1.398174in}{2.031290in}}{\pgfqpoint{1.401446in}{2.039190in}}{\pgfqpoint{1.401446in}{2.047426in}}%
\pgfpathcurveto{\pgfqpoint{1.401446in}{2.055663in}}{\pgfqpoint{1.398174in}{2.063563in}}{\pgfqpoint{1.392350in}{2.069387in}}%
\pgfpathcurveto{\pgfqpoint{1.386526in}{2.075211in}}{\pgfqpoint{1.378626in}{2.078483in}}{\pgfqpoint{1.370390in}{2.078483in}}%
\pgfpathcurveto{\pgfqpoint{1.362154in}{2.078483in}}{\pgfqpoint{1.354254in}{2.075211in}}{\pgfqpoint{1.348430in}{2.069387in}}%
\pgfpathcurveto{\pgfqpoint{1.342606in}{2.063563in}}{\pgfqpoint{1.339333in}{2.055663in}}{\pgfqpoint{1.339333in}{2.047426in}}%
\pgfpathcurveto{\pgfqpoint{1.339333in}{2.039190in}}{\pgfqpoint{1.342606in}{2.031290in}}{\pgfqpoint{1.348430in}{2.025466in}}%
\pgfpathcurveto{\pgfqpoint{1.354254in}{2.019642in}}{\pgfqpoint{1.362154in}{2.016370in}}{\pgfqpoint{1.370390in}{2.016370in}}%
\pgfpathclose%
\pgfusepath{stroke,fill}%
\end{pgfscope}%
\begin{pgfscope}%
\pgfpathrectangle{\pgfqpoint{0.100000in}{0.212622in}}{\pgfqpoint{3.696000in}{3.696000in}}%
\pgfusepath{clip}%
\pgfsetbuttcap%
\pgfsetroundjoin%
\definecolor{currentfill}{rgb}{0.121569,0.466667,0.705882}%
\pgfsetfillcolor{currentfill}%
\pgfsetfillopacity{0.896648}%
\pgfsetlinewidth{1.003750pt}%
\definecolor{currentstroke}{rgb}{0.121569,0.466667,0.705882}%
\pgfsetstrokecolor{currentstroke}%
\pgfsetstrokeopacity{0.896648}%
\pgfsetdash{}{0pt}%
\pgfpathmoveto{\pgfqpoint{2.334592in}{1.449981in}}%
\pgfpathcurveto{\pgfqpoint{2.342829in}{1.449981in}}{\pgfqpoint{2.350729in}{1.453253in}}{\pgfqpoint{2.356553in}{1.459077in}}%
\pgfpathcurveto{\pgfqpoint{2.362376in}{1.464901in}}{\pgfqpoint{2.365649in}{1.472801in}}{\pgfqpoint{2.365649in}{1.481038in}}%
\pgfpathcurveto{\pgfqpoint{2.365649in}{1.489274in}}{\pgfqpoint{2.362376in}{1.497174in}}{\pgfqpoint{2.356553in}{1.502998in}}%
\pgfpathcurveto{\pgfqpoint{2.350729in}{1.508822in}}{\pgfqpoint{2.342829in}{1.512094in}}{\pgfqpoint{2.334592in}{1.512094in}}%
\pgfpathcurveto{\pgfqpoint{2.326356in}{1.512094in}}{\pgfqpoint{2.318456in}{1.508822in}}{\pgfqpoint{2.312632in}{1.502998in}}%
\pgfpathcurveto{\pgfqpoint{2.306808in}{1.497174in}}{\pgfqpoint{2.303536in}{1.489274in}}{\pgfqpoint{2.303536in}{1.481038in}}%
\pgfpathcurveto{\pgfqpoint{2.303536in}{1.472801in}}{\pgfqpoint{2.306808in}{1.464901in}}{\pgfqpoint{2.312632in}{1.459077in}}%
\pgfpathcurveto{\pgfqpoint{2.318456in}{1.453253in}}{\pgfqpoint{2.326356in}{1.449981in}}{\pgfqpoint{2.334592in}{1.449981in}}%
\pgfpathclose%
\pgfusepath{stroke,fill}%
\end{pgfscope}%
\begin{pgfscope}%
\pgfpathrectangle{\pgfqpoint{0.100000in}{0.212622in}}{\pgfqpoint{3.696000in}{3.696000in}}%
\pgfusepath{clip}%
\pgfsetbuttcap%
\pgfsetroundjoin%
\definecolor{currentfill}{rgb}{0.121569,0.466667,0.705882}%
\pgfsetfillcolor{currentfill}%
\pgfsetfillopacity{0.898659}%
\pgfsetlinewidth{1.003750pt}%
\definecolor{currentstroke}{rgb}{0.121569,0.466667,0.705882}%
\pgfsetstrokecolor{currentstroke}%
\pgfsetstrokeopacity{0.898659}%
\pgfsetdash{}{0pt}%
\pgfpathmoveto{\pgfqpoint{2.335717in}{1.446838in}}%
\pgfpathcurveto{\pgfqpoint{2.343954in}{1.446838in}}{\pgfqpoint{2.351854in}{1.450110in}}{\pgfqpoint{2.357678in}{1.455934in}}%
\pgfpathcurveto{\pgfqpoint{2.363502in}{1.461758in}}{\pgfqpoint{2.366774in}{1.469658in}}{\pgfqpoint{2.366774in}{1.477894in}}%
\pgfpathcurveto{\pgfqpoint{2.366774in}{1.486130in}}{\pgfqpoint{2.363502in}{1.494030in}}{\pgfqpoint{2.357678in}{1.499854in}}%
\pgfpathcurveto{\pgfqpoint{2.351854in}{1.505678in}}{\pgfqpoint{2.343954in}{1.508951in}}{\pgfqpoint{2.335717in}{1.508951in}}%
\pgfpathcurveto{\pgfqpoint{2.327481in}{1.508951in}}{\pgfqpoint{2.319581in}{1.505678in}}{\pgfqpoint{2.313757in}{1.499854in}}%
\pgfpathcurveto{\pgfqpoint{2.307933in}{1.494030in}}{\pgfqpoint{2.304661in}{1.486130in}}{\pgfqpoint{2.304661in}{1.477894in}}%
\pgfpathcurveto{\pgfqpoint{2.304661in}{1.469658in}}{\pgfqpoint{2.307933in}{1.461758in}}{\pgfqpoint{2.313757in}{1.455934in}}%
\pgfpathcurveto{\pgfqpoint{2.319581in}{1.450110in}}{\pgfqpoint{2.327481in}{1.446838in}}{\pgfqpoint{2.335717in}{1.446838in}}%
\pgfpathclose%
\pgfusepath{stroke,fill}%
\end{pgfscope}%
\begin{pgfscope}%
\pgfpathrectangle{\pgfqpoint{0.100000in}{0.212622in}}{\pgfqpoint{3.696000in}{3.696000in}}%
\pgfusepath{clip}%
\pgfsetbuttcap%
\pgfsetroundjoin%
\definecolor{currentfill}{rgb}{0.121569,0.466667,0.705882}%
\pgfsetfillcolor{currentfill}%
\pgfsetfillopacity{0.900178}%
\pgfsetlinewidth{1.003750pt}%
\definecolor{currentstroke}{rgb}{0.121569,0.466667,0.705882}%
\pgfsetstrokecolor{currentstroke}%
\pgfsetstrokeopacity{0.900178}%
\pgfsetdash{}{0pt}%
\pgfpathmoveto{\pgfqpoint{1.409481in}{1.982674in}}%
\pgfpathcurveto{\pgfqpoint{1.417717in}{1.982674in}}{\pgfqpoint{1.425617in}{1.985946in}}{\pgfqpoint{1.431441in}{1.991770in}}%
\pgfpathcurveto{\pgfqpoint{1.437265in}{1.997594in}}{\pgfqpoint{1.440537in}{2.005494in}}{\pgfqpoint{1.440537in}{2.013731in}}%
\pgfpathcurveto{\pgfqpoint{1.440537in}{2.021967in}}{\pgfqpoint{1.437265in}{2.029867in}}{\pgfqpoint{1.431441in}{2.035691in}}%
\pgfpathcurveto{\pgfqpoint{1.425617in}{2.041515in}}{\pgfqpoint{1.417717in}{2.044787in}}{\pgfqpoint{1.409481in}{2.044787in}}%
\pgfpathcurveto{\pgfqpoint{1.401245in}{2.044787in}}{\pgfqpoint{1.393345in}{2.041515in}}{\pgfqpoint{1.387521in}{2.035691in}}%
\pgfpathcurveto{\pgfqpoint{1.381697in}{2.029867in}}{\pgfqpoint{1.378424in}{2.021967in}}{\pgfqpoint{1.378424in}{2.013731in}}%
\pgfpathcurveto{\pgfqpoint{1.378424in}{2.005494in}}{\pgfqpoint{1.381697in}{1.997594in}}{\pgfqpoint{1.387521in}{1.991770in}}%
\pgfpathcurveto{\pgfqpoint{1.393345in}{1.985946in}}{\pgfqpoint{1.401245in}{1.982674in}}{\pgfqpoint{1.409481in}{1.982674in}}%
\pgfpathclose%
\pgfusepath{stroke,fill}%
\end{pgfscope}%
\begin{pgfscope}%
\pgfpathrectangle{\pgfqpoint{0.100000in}{0.212622in}}{\pgfqpoint{3.696000in}{3.696000in}}%
\pgfusepath{clip}%
\pgfsetbuttcap%
\pgfsetroundjoin%
\definecolor{currentfill}{rgb}{0.121569,0.466667,0.705882}%
\pgfsetfillcolor{currentfill}%
\pgfsetfillopacity{0.900764}%
\pgfsetlinewidth{1.003750pt}%
\definecolor{currentstroke}{rgb}{0.121569,0.466667,0.705882}%
\pgfsetstrokecolor{currentstroke}%
\pgfsetstrokeopacity{0.900764}%
\pgfsetdash{}{0pt}%
\pgfpathmoveto{\pgfqpoint{2.337674in}{1.442217in}}%
\pgfpathcurveto{\pgfqpoint{2.345911in}{1.442217in}}{\pgfqpoint{2.353811in}{1.445489in}}{\pgfqpoint{2.359635in}{1.451313in}}%
\pgfpathcurveto{\pgfqpoint{2.365458in}{1.457137in}}{\pgfqpoint{2.368731in}{1.465037in}}{\pgfqpoint{2.368731in}{1.473274in}}%
\pgfpathcurveto{\pgfqpoint{2.368731in}{1.481510in}}{\pgfqpoint{2.365458in}{1.489410in}}{\pgfqpoint{2.359635in}{1.495234in}}%
\pgfpathcurveto{\pgfqpoint{2.353811in}{1.501058in}}{\pgfqpoint{2.345911in}{1.504330in}}{\pgfqpoint{2.337674in}{1.504330in}}%
\pgfpathcurveto{\pgfqpoint{2.329438in}{1.504330in}}{\pgfqpoint{2.321538in}{1.501058in}}{\pgfqpoint{2.315714in}{1.495234in}}%
\pgfpathcurveto{\pgfqpoint{2.309890in}{1.489410in}}{\pgfqpoint{2.306618in}{1.481510in}}{\pgfqpoint{2.306618in}{1.473274in}}%
\pgfpathcurveto{\pgfqpoint{2.306618in}{1.465037in}}{\pgfqpoint{2.309890in}{1.457137in}}{\pgfqpoint{2.315714in}{1.451313in}}%
\pgfpathcurveto{\pgfqpoint{2.321538in}{1.445489in}}{\pgfqpoint{2.329438in}{1.442217in}}{\pgfqpoint{2.337674in}{1.442217in}}%
\pgfpathclose%
\pgfusepath{stroke,fill}%
\end{pgfscope}%
\begin{pgfscope}%
\pgfpathrectangle{\pgfqpoint{0.100000in}{0.212622in}}{\pgfqpoint{3.696000in}{3.696000in}}%
\pgfusepath{clip}%
\pgfsetbuttcap%
\pgfsetroundjoin%
\definecolor{currentfill}{rgb}{0.121569,0.466667,0.705882}%
\pgfsetfillcolor{currentfill}%
\pgfsetfillopacity{0.903326}%
\pgfsetlinewidth{1.003750pt}%
\definecolor{currentstroke}{rgb}{0.121569,0.466667,0.705882}%
\pgfsetstrokecolor{currentstroke}%
\pgfsetstrokeopacity{0.903326}%
\pgfsetdash{}{0pt}%
\pgfpathmoveto{\pgfqpoint{2.339564in}{1.438577in}}%
\pgfpathcurveto{\pgfqpoint{2.347801in}{1.438577in}}{\pgfqpoint{2.355701in}{1.441849in}}{\pgfqpoint{2.361525in}{1.447673in}}%
\pgfpathcurveto{\pgfqpoint{2.367349in}{1.453497in}}{\pgfqpoint{2.370621in}{1.461397in}}{\pgfqpoint{2.370621in}{1.469634in}}%
\pgfpathcurveto{\pgfqpoint{2.370621in}{1.477870in}}{\pgfqpoint{2.367349in}{1.485770in}}{\pgfqpoint{2.361525in}{1.491594in}}%
\pgfpathcurveto{\pgfqpoint{2.355701in}{1.497418in}}{\pgfqpoint{2.347801in}{1.500690in}}{\pgfqpoint{2.339564in}{1.500690in}}%
\pgfpathcurveto{\pgfqpoint{2.331328in}{1.500690in}}{\pgfqpoint{2.323428in}{1.497418in}}{\pgfqpoint{2.317604in}{1.491594in}}%
\pgfpathcurveto{\pgfqpoint{2.311780in}{1.485770in}}{\pgfqpoint{2.308508in}{1.477870in}}{\pgfqpoint{2.308508in}{1.469634in}}%
\pgfpathcurveto{\pgfqpoint{2.308508in}{1.461397in}}{\pgfqpoint{2.311780in}{1.453497in}}{\pgfqpoint{2.317604in}{1.447673in}}%
\pgfpathcurveto{\pgfqpoint{2.323428in}{1.441849in}}{\pgfqpoint{2.331328in}{1.438577in}}{\pgfqpoint{2.339564in}{1.438577in}}%
\pgfpathclose%
\pgfusepath{stroke,fill}%
\end{pgfscope}%
\begin{pgfscope}%
\pgfpathrectangle{\pgfqpoint{0.100000in}{0.212622in}}{\pgfqpoint{3.696000in}{3.696000in}}%
\pgfusepath{clip}%
\pgfsetbuttcap%
\pgfsetroundjoin%
\definecolor{currentfill}{rgb}{0.121569,0.466667,0.705882}%
\pgfsetfillcolor{currentfill}%
\pgfsetfillopacity{0.904927}%
\pgfsetlinewidth{1.003750pt}%
\definecolor{currentstroke}{rgb}{0.121569,0.466667,0.705882}%
\pgfsetstrokecolor{currentstroke}%
\pgfsetstrokeopacity{0.904927}%
\pgfsetdash{}{0pt}%
\pgfpathmoveto{\pgfqpoint{2.341170in}{1.438117in}}%
\pgfpathcurveto{\pgfqpoint{2.349407in}{1.438117in}}{\pgfqpoint{2.357307in}{1.441389in}}{\pgfqpoint{2.363131in}{1.447213in}}%
\pgfpathcurveto{\pgfqpoint{2.368955in}{1.453037in}}{\pgfqpoint{2.372227in}{1.460937in}}{\pgfqpoint{2.372227in}{1.469174in}}%
\pgfpathcurveto{\pgfqpoint{2.372227in}{1.477410in}}{\pgfqpoint{2.368955in}{1.485310in}}{\pgfqpoint{2.363131in}{1.491134in}}%
\pgfpathcurveto{\pgfqpoint{2.357307in}{1.496958in}}{\pgfqpoint{2.349407in}{1.500230in}}{\pgfqpoint{2.341170in}{1.500230in}}%
\pgfpathcurveto{\pgfqpoint{2.332934in}{1.500230in}}{\pgfqpoint{2.325034in}{1.496958in}}{\pgfqpoint{2.319210in}{1.491134in}}%
\pgfpathcurveto{\pgfqpoint{2.313386in}{1.485310in}}{\pgfqpoint{2.310114in}{1.477410in}}{\pgfqpoint{2.310114in}{1.469174in}}%
\pgfpathcurveto{\pgfqpoint{2.310114in}{1.460937in}}{\pgfqpoint{2.313386in}{1.453037in}}{\pgfqpoint{2.319210in}{1.447213in}}%
\pgfpathcurveto{\pgfqpoint{2.325034in}{1.441389in}}{\pgfqpoint{2.332934in}{1.438117in}}{\pgfqpoint{2.341170in}{1.438117in}}%
\pgfpathclose%
\pgfusepath{stroke,fill}%
\end{pgfscope}%
\begin{pgfscope}%
\pgfpathrectangle{\pgfqpoint{0.100000in}{0.212622in}}{\pgfqpoint{3.696000in}{3.696000in}}%
\pgfusepath{clip}%
\pgfsetbuttcap%
\pgfsetroundjoin%
\definecolor{currentfill}{rgb}{0.121569,0.466667,0.705882}%
\pgfsetfillcolor{currentfill}%
\pgfsetfillopacity{0.906076}%
\pgfsetlinewidth{1.003750pt}%
\definecolor{currentstroke}{rgb}{0.121569,0.466667,0.705882}%
\pgfsetstrokecolor{currentstroke}%
\pgfsetstrokeopacity{0.906076}%
\pgfsetdash{}{0pt}%
\pgfpathmoveto{\pgfqpoint{1.449382in}{1.962611in}}%
\pgfpathcurveto{\pgfqpoint{1.457618in}{1.962611in}}{\pgfqpoint{1.465518in}{1.965883in}}{\pgfqpoint{1.471342in}{1.971707in}}%
\pgfpathcurveto{\pgfqpoint{1.477166in}{1.977531in}}{\pgfqpoint{1.480439in}{1.985431in}}{\pgfqpoint{1.480439in}{1.993667in}}%
\pgfpathcurveto{\pgfqpoint{1.480439in}{2.001903in}}{\pgfqpoint{1.477166in}{2.009803in}}{\pgfqpoint{1.471342in}{2.015627in}}%
\pgfpathcurveto{\pgfqpoint{1.465518in}{2.021451in}}{\pgfqpoint{1.457618in}{2.024724in}}{\pgfqpoint{1.449382in}{2.024724in}}%
\pgfpathcurveto{\pgfqpoint{1.441146in}{2.024724in}}{\pgfqpoint{1.433246in}{2.021451in}}{\pgfqpoint{1.427422in}{2.015627in}}%
\pgfpathcurveto{\pgfqpoint{1.421598in}{2.009803in}}{\pgfqpoint{1.418326in}{2.001903in}}{\pgfqpoint{1.418326in}{1.993667in}}%
\pgfpathcurveto{\pgfqpoint{1.418326in}{1.985431in}}{\pgfqpoint{1.421598in}{1.977531in}}{\pgfqpoint{1.427422in}{1.971707in}}%
\pgfpathcurveto{\pgfqpoint{1.433246in}{1.965883in}}{\pgfqpoint{1.441146in}{1.962611in}}{\pgfqpoint{1.449382in}{1.962611in}}%
\pgfpathclose%
\pgfusepath{stroke,fill}%
\end{pgfscope}%
\begin{pgfscope}%
\pgfpathrectangle{\pgfqpoint{0.100000in}{0.212622in}}{\pgfqpoint{3.696000in}{3.696000in}}%
\pgfusepath{clip}%
\pgfsetbuttcap%
\pgfsetroundjoin%
\definecolor{currentfill}{rgb}{0.121569,0.466667,0.705882}%
\pgfsetfillcolor{currentfill}%
\pgfsetfillopacity{0.906609}%
\pgfsetlinewidth{1.003750pt}%
\definecolor{currentstroke}{rgb}{0.121569,0.466667,0.705882}%
\pgfsetstrokecolor{currentstroke}%
\pgfsetstrokeopacity{0.906609}%
\pgfsetdash{}{0pt}%
\pgfpathmoveto{\pgfqpoint{2.342353in}{1.435940in}}%
\pgfpathcurveto{\pgfqpoint{2.350589in}{1.435940in}}{\pgfqpoint{2.358489in}{1.439212in}}{\pgfqpoint{2.364313in}{1.445036in}}%
\pgfpathcurveto{\pgfqpoint{2.370137in}{1.450860in}}{\pgfqpoint{2.373409in}{1.458760in}}{\pgfqpoint{2.373409in}{1.466997in}}%
\pgfpathcurveto{\pgfqpoint{2.373409in}{1.475233in}}{\pgfqpoint{2.370137in}{1.483133in}}{\pgfqpoint{2.364313in}{1.488957in}}%
\pgfpathcurveto{\pgfqpoint{2.358489in}{1.494781in}}{\pgfqpoint{2.350589in}{1.498053in}}{\pgfqpoint{2.342353in}{1.498053in}}%
\pgfpathcurveto{\pgfqpoint{2.334117in}{1.498053in}}{\pgfqpoint{2.326217in}{1.494781in}}{\pgfqpoint{2.320393in}{1.488957in}}%
\pgfpathcurveto{\pgfqpoint{2.314569in}{1.483133in}}{\pgfqpoint{2.311296in}{1.475233in}}{\pgfqpoint{2.311296in}{1.466997in}}%
\pgfpathcurveto{\pgfqpoint{2.311296in}{1.458760in}}{\pgfqpoint{2.314569in}{1.450860in}}{\pgfqpoint{2.320393in}{1.445036in}}%
\pgfpathcurveto{\pgfqpoint{2.326217in}{1.439212in}}{\pgfqpoint{2.334117in}{1.435940in}}{\pgfqpoint{2.342353in}{1.435940in}}%
\pgfpathclose%
\pgfusepath{stroke,fill}%
\end{pgfscope}%
\begin{pgfscope}%
\pgfpathrectangle{\pgfqpoint{0.100000in}{0.212622in}}{\pgfqpoint{3.696000in}{3.696000in}}%
\pgfusepath{clip}%
\pgfsetbuttcap%
\pgfsetroundjoin%
\definecolor{currentfill}{rgb}{0.121569,0.466667,0.705882}%
\pgfsetfillcolor{currentfill}%
\pgfsetfillopacity{0.908232}%
\pgfsetlinewidth{1.003750pt}%
\definecolor{currentstroke}{rgb}{0.121569,0.466667,0.705882}%
\pgfsetstrokecolor{currentstroke}%
\pgfsetstrokeopacity{0.908232}%
\pgfsetdash{}{0pt}%
\pgfpathmoveto{\pgfqpoint{1.487633in}{1.929432in}}%
\pgfpathcurveto{\pgfqpoint{1.495869in}{1.929432in}}{\pgfqpoint{1.503769in}{1.932705in}}{\pgfqpoint{1.509593in}{1.938529in}}%
\pgfpathcurveto{\pgfqpoint{1.515417in}{1.944353in}}{\pgfqpoint{1.518689in}{1.952253in}}{\pgfqpoint{1.518689in}{1.960489in}}%
\pgfpathcurveto{\pgfqpoint{1.518689in}{1.968725in}}{\pgfqpoint{1.515417in}{1.976625in}}{\pgfqpoint{1.509593in}{1.982449in}}%
\pgfpathcurveto{\pgfqpoint{1.503769in}{1.988273in}}{\pgfqpoint{1.495869in}{1.991545in}}{\pgfqpoint{1.487633in}{1.991545in}}%
\pgfpathcurveto{\pgfqpoint{1.479396in}{1.991545in}}{\pgfqpoint{1.471496in}{1.988273in}}{\pgfqpoint{1.465672in}{1.982449in}}%
\pgfpathcurveto{\pgfqpoint{1.459848in}{1.976625in}}{\pgfqpoint{1.456576in}{1.968725in}}{\pgfqpoint{1.456576in}{1.960489in}}%
\pgfpathcurveto{\pgfqpoint{1.456576in}{1.952253in}}{\pgfqpoint{1.459848in}{1.944353in}}{\pgfqpoint{1.465672in}{1.938529in}}%
\pgfpathcurveto{\pgfqpoint{1.471496in}{1.932705in}}{\pgfqpoint{1.479396in}{1.929432in}}{\pgfqpoint{1.487633in}{1.929432in}}%
\pgfpathclose%
\pgfusepath{stroke,fill}%
\end{pgfscope}%
\begin{pgfscope}%
\pgfpathrectangle{\pgfqpoint{0.100000in}{0.212622in}}{\pgfqpoint{3.696000in}{3.696000in}}%
\pgfusepath{clip}%
\pgfsetbuttcap%
\pgfsetroundjoin%
\definecolor{currentfill}{rgb}{0.121569,0.466667,0.705882}%
\pgfsetfillcolor{currentfill}%
\pgfsetfillopacity{0.908379}%
\pgfsetlinewidth{1.003750pt}%
\definecolor{currentstroke}{rgb}{0.121569,0.466667,0.705882}%
\pgfsetstrokecolor{currentstroke}%
\pgfsetstrokeopacity{0.908379}%
\pgfsetdash{}{0pt}%
\pgfpathmoveto{\pgfqpoint{2.343943in}{1.432599in}}%
\pgfpathcurveto{\pgfqpoint{2.352179in}{1.432599in}}{\pgfqpoint{2.360079in}{1.435871in}}{\pgfqpoint{2.365903in}{1.441695in}}%
\pgfpathcurveto{\pgfqpoint{2.371727in}{1.447519in}}{\pgfqpoint{2.374999in}{1.455419in}}{\pgfqpoint{2.374999in}{1.463655in}}%
\pgfpathcurveto{\pgfqpoint{2.374999in}{1.471892in}}{\pgfqpoint{2.371727in}{1.479792in}}{\pgfqpoint{2.365903in}{1.485616in}}%
\pgfpathcurveto{\pgfqpoint{2.360079in}{1.491439in}}{\pgfqpoint{2.352179in}{1.494712in}}{\pgfqpoint{2.343943in}{1.494712in}}%
\pgfpathcurveto{\pgfqpoint{2.335707in}{1.494712in}}{\pgfqpoint{2.327807in}{1.491439in}}{\pgfqpoint{2.321983in}{1.485616in}}%
\pgfpathcurveto{\pgfqpoint{2.316159in}{1.479792in}}{\pgfqpoint{2.312886in}{1.471892in}}{\pgfqpoint{2.312886in}{1.463655in}}%
\pgfpathcurveto{\pgfqpoint{2.312886in}{1.455419in}}{\pgfqpoint{2.316159in}{1.447519in}}{\pgfqpoint{2.321983in}{1.441695in}}%
\pgfpathcurveto{\pgfqpoint{2.327807in}{1.435871in}}{\pgfqpoint{2.335707in}{1.432599in}}{\pgfqpoint{2.343943in}{1.432599in}}%
\pgfpathclose%
\pgfusepath{stroke,fill}%
\end{pgfscope}%
\begin{pgfscope}%
\pgfpathrectangle{\pgfqpoint{0.100000in}{0.212622in}}{\pgfqpoint{3.696000in}{3.696000in}}%
\pgfusepath{clip}%
\pgfsetbuttcap%
\pgfsetroundjoin%
\definecolor{currentfill}{rgb}{0.121569,0.466667,0.705882}%
\pgfsetfillcolor{currentfill}%
\pgfsetfillopacity{0.910015}%
\pgfsetlinewidth{1.003750pt}%
\definecolor{currentstroke}{rgb}{0.121569,0.466667,0.705882}%
\pgfsetstrokecolor{currentstroke}%
\pgfsetstrokeopacity{0.910015}%
\pgfsetdash{}{0pt}%
\pgfpathmoveto{\pgfqpoint{2.345791in}{1.426621in}}%
\pgfpathcurveto{\pgfqpoint{2.354027in}{1.426621in}}{\pgfqpoint{2.361927in}{1.429893in}}{\pgfqpoint{2.367751in}{1.435717in}}%
\pgfpathcurveto{\pgfqpoint{2.373575in}{1.441541in}}{\pgfqpoint{2.376847in}{1.449441in}}{\pgfqpoint{2.376847in}{1.457677in}}%
\pgfpathcurveto{\pgfqpoint{2.376847in}{1.465914in}}{\pgfqpoint{2.373575in}{1.473814in}}{\pgfqpoint{2.367751in}{1.479638in}}%
\pgfpathcurveto{\pgfqpoint{2.361927in}{1.485462in}}{\pgfqpoint{2.354027in}{1.488734in}}{\pgfqpoint{2.345791in}{1.488734in}}%
\pgfpathcurveto{\pgfqpoint{2.337555in}{1.488734in}}{\pgfqpoint{2.329654in}{1.485462in}}{\pgfqpoint{2.323831in}{1.479638in}}%
\pgfpathcurveto{\pgfqpoint{2.318007in}{1.473814in}}{\pgfqpoint{2.314734in}{1.465914in}}{\pgfqpoint{2.314734in}{1.457677in}}%
\pgfpathcurveto{\pgfqpoint{2.314734in}{1.449441in}}{\pgfqpoint{2.318007in}{1.441541in}}{\pgfqpoint{2.323831in}{1.435717in}}%
\pgfpathcurveto{\pgfqpoint{2.329654in}{1.429893in}}{\pgfqpoint{2.337555in}{1.426621in}}{\pgfqpoint{2.345791in}{1.426621in}}%
\pgfpathclose%
\pgfusepath{stroke,fill}%
\end{pgfscope}%
\begin{pgfscope}%
\pgfpathrectangle{\pgfqpoint{0.100000in}{0.212622in}}{\pgfqpoint{3.696000in}{3.696000in}}%
\pgfusepath{clip}%
\pgfsetbuttcap%
\pgfsetroundjoin%
\definecolor{currentfill}{rgb}{0.121569,0.466667,0.705882}%
\pgfsetfillcolor{currentfill}%
\pgfsetfillopacity{0.911842}%
\pgfsetlinewidth{1.003750pt}%
\definecolor{currentstroke}{rgb}{0.121569,0.466667,0.705882}%
\pgfsetstrokecolor{currentstroke}%
\pgfsetstrokeopacity{0.911842}%
\pgfsetdash{}{0pt}%
\pgfpathmoveto{\pgfqpoint{1.524892in}{1.906550in}}%
\pgfpathcurveto{\pgfqpoint{1.533128in}{1.906550in}}{\pgfqpoint{1.541028in}{1.909823in}}{\pgfqpoint{1.546852in}{1.915647in}}%
\pgfpathcurveto{\pgfqpoint{1.552676in}{1.921471in}}{\pgfqpoint{1.555948in}{1.929371in}}{\pgfqpoint{1.555948in}{1.937607in}}%
\pgfpathcurveto{\pgfqpoint{1.555948in}{1.945843in}}{\pgfqpoint{1.552676in}{1.953743in}}{\pgfqpoint{1.546852in}{1.959567in}}%
\pgfpathcurveto{\pgfqpoint{1.541028in}{1.965391in}}{\pgfqpoint{1.533128in}{1.968663in}}{\pgfqpoint{1.524892in}{1.968663in}}%
\pgfpathcurveto{\pgfqpoint{1.516655in}{1.968663in}}{\pgfqpoint{1.508755in}{1.965391in}}{\pgfqpoint{1.502931in}{1.959567in}}%
\pgfpathcurveto{\pgfqpoint{1.497107in}{1.953743in}}{\pgfqpoint{1.493835in}{1.945843in}}{\pgfqpoint{1.493835in}{1.937607in}}%
\pgfpathcurveto{\pgfqpoint{1.493835in}{1.929371in}}{\pgfqpoint{1.497107in}{1.921471in}}{\pgfqpoint{1.502931in}{1.915647in}}%
\pgfpathcurveto{\pgfqpoint{1.508755in}{1.909823in}}{\pgfqpoint{1.516655in}{1.906550in}}{\pgfqpoint{1.524892in}{1.906550in}}%
\pgfpathclose%
\pgfusepath{stroke,fill}%
\end{pgfscope}%
\begin{pgfscope}%
\pgfpathrectangle{\pgfqpoint{0.100000in}{0.212622in}}{\pgfqpoint{3.696000in}{3.696000in}}%
\pgfusepath{clip}%
\pgfsetbuttcap%
\pgfsetroundjoin%
\definecolor{currentfill}{rgb}{0.121569,0.466667,0.705882}%
\pgfsetfillcolor{currentfill}%
\pgfsetfillopacity{0.913077}%
\pgfsetlinewidth{1.003750pt}%
\definecolor{currentstroke}{rgb}{0.121569,0.466667,0.705882}%
\pgfsetstrokecolor{currentstroke}%
\pgfsetstrokeopacity{0.913077}%
\pgfsetdash{}{0pt}%
\pgfpathmoveto{\pgfqpoint{2.347844in}{1.424537in}}%
\pgfpathcurveto{\pgfqpoint{2.356080in}{1.424537in}}{\pgfqpoint{2.363980in}{1.427809in}}{\pgfqpoint{2.369804in}{1.433633in}}%
\pgfpathcurveto{\pgfqpoint{2.375628in}{1.439457in}}{\pgfqpoint{2.378901in}{1.447357in}}{\pgfqpoint{2.378901in}{1.455593in}}%
\pgfpathcurveto{\pgfqpoint{2.378901in}{1.463829in}}{\pgfqpoint{2.375628in}{1.471729in}}{\pgfqpoint{2.369804in}{1.477553in}}%
\pgfpathcurveto{\pgfqpoint{2.363980in}{1.483377in}}{\pgfqpoint{2.356080in}{1.486650in}}{\pgfqpoint{2.347844in}{1.486650in}}%
\pgfpathcurveto{\pgfqpoint{2.339608in}{1.486650in}}{\pgfqpoint{2.331708in}{1.483377in}}{\pgfqpoint{2.325884in}{1.477553in}}%
\pgfpathcurveto{\pgfqpoint{2.320060in}{1.471729in}}{\pgfqpoint{2.316788in}{1.463829in}}{\pgfqpoint{2.316788in}{1.455593in}}%
\pgfpathcurveto{\pgfqpoint{2.316788in}{1.447357in}}{\pgfqpoint{2.320060in}{1.439457in}}{\pgfqpoint{2.325884in}{1.433633in}}%
\pgfpathcurveto{\pgfqpoint{2.331708in}{1.427809in}}{\pgfqpoint{2.339608in}{1.424537in}}{\pgfqpoint{2.347844in}{1.424537in}}%
\pgfpathclose%
\pgfusepath{stroke,fill}%
\end{pgfscope}%
\begin{pgfscope}%
\pgfpathrectangle{\pgfqpoint{0.100000in}{0.212622in}}{\pgfqpoint{3.696000in}{3.696000in}}%
\pgfusepath{clip}%
\pgfsetbuttcap%
\pgfsetroundjoin%
\definecolor{currentfill}{rgb}{0.121569,0.466667,0.705882}%
\pgfsetfillcolor{currentfill}%
\pgfsetfillopacity{0.915397}%
\pgfsetlinewidth{1.003750pt}%
\definecolor{currentstroke}{rgb}{0.121569,0.466667,0.705882}%
\pgfsetstrokecolor{currentstroke}%
\pgfsetstrokeopacity{0.915397}%
\pgfsetdash{}{0pt}%
\pgfpathmoveto{\pgfqpoint{1.559772in}{1.885314in}}%
\pgfpathcurveto{\pgfqpoint{1.568009in}{1.885314in}}{\pgfqpoint{1.575909in}{1.888587in}}{\pgfqpoint{1.581733in}{1.894411in}}%
\pgfpathcurveto{\pgfqpoint{1.587557in}{1.900234in}}{\pgfqpoint{1.590829in}{1.908135in}}{\pgfqpoint{1.590829in}{1.916371in}}%
\pgfpathcurveto{\pgfqpoint{1.590829in}{1.924607in}}{\pgfqpoint{1.587557in}{1.932507in}}{\pgfqpoint{1.581733in}{1.938331in}}%
\pgfpathcurveto{\pgfqpoint{1.575909in}{1.944155in}}{\pgfqpoint{1.568009in}{1.947427in}}{\pgfqpoint{1.559772in}{1.947427in}}%
\pgfpathcurveto{\pgfqpoint{1.551536in}{1.947427in}}{\pgfqpoint{1.543636in}{1.944155in}}{\pgfqpoint{1.537812in}{1.938331in}}%
\pgfpathcurveto{\pgfqpoint{1.531988in}{1.932507in}}{\pgfqpoint{1.528716in}{1.924607in}}{\pgfqpoint{1.528716in}{1.916371in}}%
\pgfpathcurveto{\pgfqpoint{1.528716in}{1.908135in}}{\pgfqpoint{1.531988in}{1.900234in}}{\pgfqpoint{1.537812in}{1.894411in}}%
\pgfpathcurveto{\pgfqpoint{1.543636in}{1.888587in}}{\pgfqpoint{1.551536in}{1.885314in}}{\pgfqpoint{1.559772in}{1.885314in}}%
\pgfpathclose%
\pgfusepath{stroke,fill}%
\end{pgfscope}%
\begin{pgfscope}%
\pgfpathrectangle{\pgfqpoint{0.100000in}{0.212622in}}{\pgfqpoint{3.696000in}{3.696000in}}%
\pgfusepath{clip}%
\pgfsetbuttcap%
\pgfsetroundjoin%
\definecolor{currentfill}{rgb}{0.121569,0.466667,0.705882}%
\pgfsetfillcolor{currentfill}%
\pgfsetfillopacity{0.916743}%
\pgfsetlinewidth{1.003750pt}%
\definecolor{currentstroke}{rgb}{0.121569,0.466667,0.705882}%
\pgfsetstrokecolor{currentstroke}%
\pgfsetstrokeopacity{0.916743}%
\pgfsetdash{}{0pt}%
\pgfpathmoveto{\pgfqpoint{2.351536in}{1.422038in}}%
\pgfpathcurveto{\pgfqpoint{2.359772in}{1.422038in}}{\pgfqpoint{2.367672in}{1.425310in}}{\pgfqpoint{2.373496in}{1.431134in}}%
\pgfpathcurveto{\pgfqpoint{2.379320in}{1.436958in}}{\pgfqpoint{2.382592in}{1.444858in}}{\pgfqpoint{2.382592in}{1.453094in}}%
\pgfpathcurveto{\pgfqpoint{2.382592in}{1.461331in}}{\pgfqpoint{2.379320in}{1.469231in}}{\pgfqpoint{2.373496in}{1.475055in}}%
\pgfpathcurveto{\pgfqpoint{2.367672in}{1.480878in}}{\pgfqpoint{2.359772in}{1.484151in}}{\pgfqpoint{2.351536in}{1.484151in}}%
\pgfpathcurveto{\pgfqpoint{2.343300in}{1.484151in}}{\pgfqpoint{2.335399in}{1.480878in}}{\pgfqpoint{2.329576in}{1.475055in}}%
\pgfpathcurveto{\pgfqpoint{2.323752in}{1.469231in}}{\pgfqpoint{2.320479in}{1.461331in}}{\pgfqpoint{2.320479in}{1.453094in}}%
\pgfpathcurveto{\pgfqpoint{2.320479in}{1.444858in}}{\pgfqpoint{2.323752in}{1.436958in}}{\pgfqpoint{2.329576in}{1.431134in}}%
\pgfpathcurveto{\pgfqpoint{2.335399in}{1.425310in}}{\pgfqpoint{2.343300in}{1.422038in}}{\pgfqpoint{2.351536in}{1.422038in}}%
\pgfpathclose%
\pgfusepath{stroke,fill}%
\end{pgfscope}%
\begin{pgfscope}%
\pgfpathrectangle{\pgfqpoint{0.100000in}{0.212622in}}{\pgfqpoint{3.696000in}{3.696000in}}%
\pgfusepath{clip}%
\pgfsetbuttcap%
\pgfsetroundjoin%
\definecolor{currentfill}{rgb}{0.121569,0.466667,0.705882}%
\pgfsetfillcolor{currentfill}%
\pgfsetfillopacity{0.919707}%
\pgfsetlinewidth{1.003750pt}%
\definecolor{currentstroke}{rgb}{0.121569,0.466667,0.705882}%
\pgfsetstrokecolor{currentstroke}%
\pgfsetstrokeopacity{0.919707}%
\pgfsetdash{}{0pt}%
\pgfpathmoveto{\pgfqpoint{1.623167in}{1.835308in}}%
\pgfpathcurveto{\pgfqpoint{1.631403in}{1.835308in}}{\pgfqpoint{1.639303in}{1.838580in}}{\pgfqpoint{1.645127in}{1.844404in}}%
\pgfpathcurveto{\pgfqpoint{1.650951in}{1.850228in}}{\pgfqpoint{1.654223in}{1.858128in}}{\pgfqpoint{1.654223in}{1.866364in}}%
\pgfpathcurveto{\pgfqpoint{1.654223in}{1.874601in}}{\pgfqpoint{1.650951in}{1.882501in}}{\pgfqpoint{1.645127in}{1.888325in}}%
\pgfpathcurveto{\pgfqpoint{1.639303in}{1.894149in}}{\pgfqpoint{1.631403in}{1.897421in}}{\pgfqpoint{1.623167in}{1.897421in}}%
\pgfpathcurveto{\pgfqpoint{1.614930in}{1.897421in}}{\pgfqpoint{1.607030in}{1.894149in}}{\pgfqpoint{1.601207in}{1.888325in}}%
\pgfpathcurveto{\pgfqpoint{1.595383in}{1.882501in}}{\pgfqpoint{1.592110in}{1.874601in}}{\pgfqpoint{1.592110in}{1.866364in}}%
\pgfpathcurveto{\pgfqpoint{1.592110in}{1.858128in}}{\pgfqpoint{1.595383in}{1.850228in}}{\pgfqpoint{1.601207in}{1.844404in}}%
\pgfpathcurveto{\pgfqpoint{1.607030in}{1.838580in}}{\pgfqpoint{1.614930in}{1.835308in}}{\pgfqpoint{1.623167in}{1.835308in}}%
\pgfpathclose%
\pgfusepath{stroke,fill}%
\end{pgfscope}%
\begin{pgfscope}%
\pgfpathrectangle{\pgfqpoint{0.100000in}{0.212622in}}{\pgfqpoint{3.696000in}{3.696000in}}%
\pgfusepath{clip}%
\pgfsetbuttcap%
\pgfsetroundjoin%
\definecolor{currentfill}{rgb}{0.121569,0.466667,0.705882}%
\pgfsetfillcolor{currentfill}%
\pgfsetfillopacity{0.919933}%
\pgfsetlinewidth{1.003750pt}%
\definecolor{currentstroke}{rgb}{0.121569,0.466667,0.705882}%
\pgfsetstrokecolor{currentstroke}%
\pgfsetstrokeopacity{0.919933}%
\pgfsetdash{}{0pt}%
\pgfpathmoveto{\pgfqpoint{1.590742in}{1.868864in}}%
\pgfpathcurveto{\pgfqpoint{1.598978in}{1.868864in}}{\pgfqpoint{1.606878in}{1.872136in}}{\pgfqpoint{1.612702in}{1.877960in}}%
\pgfpathcurveto{\pgfqpoint{1.618526in}{1.883784in}}{\pgfqpoint{1.621798in}{1.891684in}}{\pgfqpoint{1.621798in}{1.899921in}}%
\pgfpathcurveto{\pgfqpoint{1.621798in}{1.908157in}}{\pgfqpoint{1.618526in}{1.916057in}}{\pgfqpoint{1.612702in}{1.921881in}}%
\pgfpathcurveto{\pgfqpoint{1.606878in}{1.927705in}}{\pgfqpoint{1.598978in}{1.930977in}}{\pgfqpoint{1.590742in}{1.930977in}}%
\pgfpathcurveto{\pgfqpoint{1.582505in}{1.930977in}}{\pgfqpoint{1.574605in}{1.927705in}}{\pgfqpoint{1.568781in}{1.921881in}}%
\pgfpathcurveto{\pgfqpoint{1.562957in}{1.916057in}}{\pgfqpoint{1.559685in}{1.908157in}}{\pgfqpoint{1.559685in}{1.899921in}}%
\pgfpathcurveto{\pgfqpoint{1.559685in}{1.891684in}}{\pgfqpoint{1.562957in}{1.883784in}}{\pgfqpoint{1.568781in}{1.877960in}}%
\pgfpathcurveto{\pgfqpoint{1.574605in}{1.872136in}}{\pgfqpoint{1.582505in}{1.868864in}}{\pgfqpoint{1.590742in}{1.868864in}}%
\pgfpathclose%
\pgfusepath{stroke,fill}%
\end{pgfscope}%
\begin{pgfscope}%
\pgfpathrectangle{\pgfqpoint{0.100000in}{0.212622in}}{\pgfqpoint{3.696000in}{3.696000in}}%
\pgfusepath{clip}%
\pgfsetbuttcap%
\pgfsetroundjoin%
\definecolor{currentfill}{rgb}{0.121569,0.466667,0.705882}%
\pgfsetfillcolor{currentfill}%
\pgfsetfillopacity{0.921009}%
\pgfsetlinewidth{1.003750pt}%
\definecolor{currentstroke}{rgb}{0.121569,0.466667,0.705882}%
\pgfsetstrokecolor{currentstroke}%
\pgfsetstrokeopacity{0.921009}%
\pgfsetdash{}{0pt}%
\pgfpathmoveto{\pgfqpoint{2.354394in}{1.419279in}}%
\pgfpathcurveto{\pgfqpoint{2.362631in}{1.419279in}}{\pgfqpoint{2.370531in}{1.422551in}}{\pgfqpoint{2.376354in}{1.428375in}}%
\pgfpathcurveto{\pgfqpoint{2.382178in}{1.434199in}}{\pgfqpoint{2.385451in}{1.442099in}}{\pgfqpoint{2.385451in}{1.450336in}}%
\pgfpathcurveto{\pgfqpoint{2.385451in}{1.458572in}}{\pgfqpoint{2.382178in}{1.466472in}}{\pgfqpoint{2.376354in}{1.472296in}}%
\pgfpathcurveto{\pgfqpoint{2.370531in}{1.478120in}}{\pgfqpoint{2.362631in}{1.481392in}}{\pgfqpoint{2.354394in}{1.481392in}}%
\pgfpathcurveto{\pgfqpoint{2.346158in}{1.481392in}}{\pgfqpoint{2.338258in}{1.478120in}}{\pgfqpoint{2.332434in}{1.472296in}}%
\pgfpathcurveto{\pgfqpoint{2.326610in}{1.466472in}}{\pgfqpoint{2.323338in}{1.458572in}}{\pgfqpoint{2.323338in}{1.450336in}}%
\pgfpathcurveto{\pgfqpoint{2.323338in}{1.442099in}}{\pgfqpoint{2.326610in}{1.434199in}}{\pgfqpoint{2.332434in}{1.428375in}}%
\pgfpathcurveto{\pgfqpoint{2.338258in}{1.422551in}}{\pgfqpoint{2.346158in}{1.419279in}}{\pgfqpoint{2.354394in}{1.419279in}}%
\pgfpathclose%
\pgfusepath{stroke,fill}%
\end{pgfscope}%
\begin{pgfscope}%
\pgfpathrectangle{\pgfqpoint{0.100000in}{0.212622in}}{\pgfqpoint{3.696000in}{3.696000in}}%
\pgfusepath{clip}%
\pgfsetbuttcap%
\pgfsetroundjoin%
\definecolor{currentfill}{rgb}{0.121569,0.466667,0.705882}%
\pgfsetfillcolor{currentfill}%
\pgfsetfillopacity{0.923434}%
\pgfsetlinewidth{1.003750pt}%
\definecolor{currentstroke}{rgb}{0.121569,0.466667,0.705882}%
\pgfsetstrokecolor{currentstroke}%
\pgfsetstrokeopacity{0.923434}%
\pgfsetdash{}{0pt}%
\pgfpathmoveto{\pgfqpoint{1.648667in}{1.820758in}}%
\pgfpathcurveto{\pgfqpoint{1.656903in}{1.820758in}}{\pgfqpoint{1.664803in}{1.824031in}}{\pgfqpoint{1.670627in}{1.829855in}}%
\pgfpathcurveto{\pgfqpoint{1.676451in}{1.835678in}}{\pgfqpoint{1.679723in}{1.843579in}}{\pgfqpoint{1.679723in}{1.851815in}}%
\pgfpathcurveto{\pgfqpoint{1.679723in}{1.860051in}}{\pgfqpoint{1.676451in}{1.867951in}}{\pgfqpoint{1.670627in}{1.873775in}}%
\pgfpathcurveto{\pgfqpoint{1.664803in}{1.879599in}}{\pgfqpoint{1.656903in}{1.882871in}}{\pgfqpoint{1.648667in}{1.882871in}}%
\pgfpathcurveto{\pgfqpoint{1.640430in}{1.882871in}}{\pgfqpoint{1.632530in}{1.879599in}}{\pgfqpoint{1.626706in}{1.873775in}}%
\pgfpathcurveto{\pgfqpoint{1.620882in}{1.867951in}}{\pgfqpoint{1.617610in}{1.860051in}}{\pgfqpoint{1.617610in}{1.851815in}}%
\pgfpathcurveto{\pgfqpoint{1.617610in}{1.843579in}}{\pgfqpoint{1.620882in}{1.835678in}}{\pgfqpoint{1.626706in}{1.829855in}}%
\pgfpathcurveto{\pgfqpoint{1.632530in}{1.824031in}}{\pgfqpoint{1.640430in}{1.820758in}}{\pgfqpoint{1.648667in}{1.820758in}}%
\pgfpathclose%
\pgfusepath{stroke,fill}%
\end{pgfscope}%
\begin{pgfscope}%
\pgfpathrectangle{\pgfqpoint{0.100000in}{0.212622in}}{\pgfqpoint{3.696000in}{3.696000in}}%
\pgfusepath{clip}%
\pgfsetbuttcap%
\pgfsetroundjoin%
\definecolor{currentfill}{rgb}{0.121569,0.466667,0.705882}%
\pgfsetfillcolor{currentfill}%
\pgfsetfillopacity{0.925079}%
\pgfsetlinewidth{1.003750pt}%
\definecolor{currentstroke}{rgb}{0.121569,0.466667,0.705882}%
\pgfsetstrokecolor{currentstroke}%
\pgfsetstrokeopacity{0.925079}%
\pgfsetdash{}{0pt}%
\pgfpathmoveto{\pgfqpoint{2.357660in}{1.413118in}}%
\pgfpathcurveto{\pgfqpoint{2.365896in}{1.413118in}}{\pgfqpoint{2.373796in}{1.416390in}}{\pgfqpoint{2.379620in}{1.422214in}}%
\pgfpathcurveto{\pgfqpoint{2.385444in}{1.428038in}}{\pgfqpoint{2.388716in}{1.435938in}}{\pgfqpoint{2.388716in}{1.444174in}}%
\pgfpathcurveto{\pgfqpoint{2.388716in}{1.452411in}}{\pgfqpoint{2.385444in}{1.460311in}}{\pgfqpoint{2.379620in}{1.466135in}}%
\pgfpathcurveto{\pgfqpoint{2.373796in}{1.471959in}}{\pgfqpoint{2.365896in}{1.475231in}}{\pgfqpoint{2.357660in}{1.475231in}}%
\pgfpathcurveto{\pgfqpoint{2.349423in}{1.475231in}}{\pgfqpoint{2.341523in}{1.471959in}}{\pgfqpoint{2.335699in}{1.466135in}}%
\pgfpathcurveto{\pgfqpoint{2.329875in}{1.460311in}}{\pgfqpoint{2.326603in}{1.452411in}}{\pgfqpoint{2.326603in}{1.444174in}}%
\pgfpathcurveto{\pgfqpoint{2.326603in}{1.435938in}}{\pgfqpoint{2.329875in}{1.428038in}}{\pgfqpoint{2.335699in}{1.422214in}}%
\pgfpathcurveto{\pgfqpoint{2.341523in}{1.416390in}}{\pgfqpoint{2.349423in}{1.413118in}}{\pgfqpoint{2.357660in}{1.413118in}}%
\pgfpathclose%
\pgfusepath{stroke,fill}%
\end{pgfscope}%
\begin{pgfscope}%
\pgfpathrectangle{\pgfqpoint{0.100000in}{0.212622in}}{\pgfqpoint{3.696000in}{3.696000in}}%
\pgfusepath{clip}%
\pgfsetbuttcap%
\pgfsetroundjoin%
\definecolor{currentfill}{rgb}{0.121569,0.466667,0.705882}%
\pgfsetfillcolor{currentfill}%
\pgfsetfillopacity{0.925631}%
\pgfsetlinewidth{1.003750pt}%
\definecolor{currentstroke}{rgb}{0.121569,0.466667,0.705882}%
\pgfsetstrokecolor{currentstroke}%
\pgfsetstrokeopacity{0.925631}%
\pgfsetdash{}{0pt}%
\pgfpathmoveto{\pgfqpoint{1.672804in}{1.796708in}}%
\pgfpathcurveto{\pgfqpoint{1.681040in}{1.796708in}}{\pgfqpoint{1.688940in}{1.799980in}}{\pgfqpoint{1.694764in}{1.805804in}}%
\pgfpathcurveto{\pgfqpoint{1.700588in}{1.811628in}}{\pgfqpoint{1.703860in}{1.819528in}}{\pgfqpoint{1.703860in}{1.827765in}}%
\pgfpathcurveto{\pgfqpoint{1.703860in}{1.836001in}}{\pgfqpoint{1.700588in}{1.843901in}}{\pgfqpoint{1.694764in}{1.849725in}}%
\pgfpathcurveto{\pgfqpoint{1.688940in}{1.855549in}}{\pgfqpoint{1.681040in}{1.858821in}}{\pgfqpoint{1.672804in}{1.858821in}}%
\pgfpathcurveto{\pgfqpoint{1.664568in}{1.858821in}}{\pgfqpoint{1.656668in}{1.855549in}}{\pgfqpoint{1.650844in}{1.849725in}}%
\pgfpathcurveto{\pgfqpoint{1.645020in}{1.843901in}}{\pgfqpoint{1.641747in}{1.836001in}}{\pgfqpoint{1.641747in}{1.827765in}}%
\pgfpathcurveto{\pgfqpoint{1.641747in}{1.819528in}}{\pgfqpoint{1.645020in}{1.811628in}}{\pgfqpoint{1.650844in}{1.805804in}}%
\pgfpathcurveto{\pgfqpoint{1.656668in}{1.799980in}}{\pgfqpoint{1.664568in}{1.796708in}}{\pgfqpoint{1.672804in}{1.796708in}}%
\pgfpathclose%
\pgfusepath{stroke,fill}%
\end{pgfscope}%
\begin{pgfscope}%
\pgfpathrectangle{\pgfqpoint{0.100000in}{0.212622in}}{\pgfqpoint{3.696000in}{3.696000in}}%
\pgfusepath{clip}%
\pgfsetbuttcap%
\pgfsetroundjoin%
\definecolor{currentfill}{rgb}{0.121569,0.466667,0.705882}%
\pgfsetfillcolor{currentfill}%
\pgfsetfillopacity{0.927874}%
\pgfsetlinewidth{1.003750pt}%
\definecolor{currentstroke}{rgb}{0.121569,0.466667,0.705882}%
\pgfsetstrokecolor{currentstroke}%
\pgfsetstrokeopacity{0.927874}%
\pgfsetdash{}{0pt}%
\pgfpathmoveto{\pgfqpoint{1.693756in}{1.777846in}}%
\pgfpathcurveto{\pgfqpoint{1.701992in}{1.777846in}}{\pgfqpoint{1.709892in}{1.781118in}}{\pgfqpoint{1.715716in}{1.786942in}}%
\pgfpathcurveto{\pgfqpoint{1.721540in}{1.792766in}}{\pgfqpoint{1.724812in}{1.800666in}}{\pgfqpoint{1.724812in}{1.808903in}}%
\pgfpathcurveto{\pgfqpoint{1.724812in}{1.817139in}}{\pgfqpoint{1.721540in}{1.825039in}}{\pgfqpoint{1.715716in}{1.830863in}}%
\pgfpathcurveto{\pgfqpoint{1.709892in}{1.836687in}}{\pgfqpoint{1.701992in}{1.839959in}}{\pgfqpoint{1.693756in}{1.839959in}}%
\pgfpathcurveto{\pgfqpoint{1.685519in}{1.839959in}}{\pgfqpoint{1.677619in}{1.836687in}}{\pgfqpoint{1.671795in}{1.830863in}}%
\pgfpathcurveto{\pgfqpoint{1.665971in}{1.825039in}}{\pgfqpoint{1.662699in}{1.817139in}}{\pgfqpoint{1.662699in}{1.808903in}}%
\pgfpathcurveto{\pgfqpoint{1.662699in}{1.800666in}}{\pgfqpoint{1.665971in}{1.792766in}}{\pgfqpoint{1.671795in}{1.786942in}}%
\pgfpathcurveto{\pgfqpoint{1.677619in}{1.781118in}}{\pgfqpoint{1.685519in}{1.777846in}}{\pgfqpoint{1.693756in}{1.777846in}}%
\pgfpathclose%
\pgfusepath{stroke,fill}%
\end{pgfscope}%
\begin{pgfscope}%
\pgfpathrectangle{\pgfqpoint{0.100000in}{0.212622in}}{\pgfqpoint{3.696000in}{3.696000in}}%
\pgfusepath{clip}%
\pgfsetbuttcap%
\pgfsetroundjoin%
\definecolor{currentfill}{rgb}{0.121569,0.466667,0.705882}%
\pgfsetfillcolor{currentfill}%
\pgfsetfillopacity{0.928633}%
\pgfsetlinewidth{1.003750pt}%
\definecolor{currentstroke}{rgb}{0.121569,0.466667,0.705882}%
\pgfsetstrokecolor{currentstroke}%
\pgfsetstrokeopacity{0.928633}%
\pgfsetdash{}{0pt}%
\pgfpathmoveto{\pgfqpoint{2.359937in}{1.401887in}}%
\pgfpathcurveto{\pgfqpoint{2.368173in}{1.401887in}}{\pgfqpoint{2.376073in}{1.405159in}}{\pgfqpoint{2.381897in}{1.410983in}}%
\pgfpathcurveto{\pgfqpoint{2.387721in}{1.416807in}}{\pgfqpoint{2.390993in}{1.424707in}}{\pgfqpoint{2.390993in}{1.432943in}}%
\pgfpathcurveto{\pgfqpoint{2.390993in}{1.441180in}}{\pgfqpoint{2.387721in}{1.449080in}}{\pgfqpoint{2.381897in}{1.454904in}}%
\pgfpathcurveto{\pgfqpoint{2.376073in}{1.460728in}}{\pgfqpoint{2.368173in}{1.464000in}}{\pgfqpoint{2.359937in}{1.464000in}}%
\pgfpathcurveto{\pgfqpoint{2.351700in}{1.464000in}}{\pgfqpoint{2.343800in}{1.460728in}}{\pgfqpoint{2.337976in}{1.454904in}}%
\pgfpathcurveto{\pgfqpoint{2.332152in}{1.449080in}}{\pgfqpoint{2.328880in}{1.441180in}}{\pgfqpoint{2.328880in}{1.432943in}}%
\pgfpathcurveto{\pgfqpoint{2.328880in}{1.424707in}}{\pgfqpoint{2.332152in}{1.416807in}}{\pgfqpoint{2.337976in}{1.410983in}}%
\pgfpathcurveto{\pgfqpoint{2.343800in}{1.405159in}}{\pgfqpoint{2.351700in}{1.401887in}}{\pgfqpoint{2.359937in}{1.401887in}}%
\pgfpathclose%
\pgfusepath{stroke,fill}%
\end{pgfscope}%
\begin{pgfscope}%
\pgfpathrectangle{\pgfqpoint{0.100000in}{0.212622in}}{\pgfqpoint{3.696000in}{3.696000in}}%
\pgfusepath{clip}%
\pgfsetbuttcap%
\pgfsetroundjoin%
\definecolor{currentfill}{rgb}{0.121569,0.466667,0.705882}%
\pgfsetfillcolor{currentfill}%
\pgfsetfillopacity{0.933392}%
\pgfsetlinewidth{1.003750pt}%
\definecolor{currentstroke}{rgb}{0.121569,0.466667,0.705882}%
\pgfsetstrokecolor{currentstroke}%
\pgfsetstrokeopacity{0.933392}%
\pgfsetdash{}{0pt}%
\pgfpathmoveto{\pgfqpoint{2.364362in}{1.396486in}}%
\pgfpathcurveto{\pgfqpoint{2.372598in}{1.396486in}}{\pgfqpoint{2.380498in}{1.399759in}}{\pgfqpoint{2.386322in}{1.405583in}}%
\pgfpathcurveto{\pgfqpoint{2.392146in}{1.411406in}}{\pgfqpoint{2.395418in}{1.419307in}}{\pgfqpoint{2.395418in}{1.427543in}}%
\pgfpathcurveto{\pgfqpoint{2.395418in}{1.435779in}}{\pgfqpoint{2.392146in}{1.443679in}}{\pgfqpoint{2.386322in}{1.449503in}}%
\pgfpathcurveto{\pgfqpoint{2.380498in}{1.455327in}}{\pgfqpoint{2.372598in}{1.458599in}}{\pgfqpoint{2.364362in}{1.458599in}}%
\pgfpathcurveto{\pgfqpoint{2.356125in}{1.458599in}}{\pgfqpoint{2.348225in}{1.455327in}}{\pgfqpoint{2.342401in}{1.449503in}}%
\pgfpathcurveto{\pgfqpoint{2.336577in}{1.443679in}}{\pgfqpoint{2.333305in}{1.435779in}}{\pgfqpoint{2.333305in}{1.427543in}}%
\pgfpathcurveto{\pgfqpoint{2.333305in}{1.419307in}}{\pgfqpoint{2.336577in}{1.411406in}}{\pgfqpoint{2.342401in}{1.405583in}}%
\pgfpathcurveto{\pgfqpoint{2.348225in}{1.399759in}}{\pgfqpoint{2.356125in}{1.396486in}}{\pgfqpoint{2.364362in}{1.396486in}}%
\pgfpathclose%
\pgfusepath{stroke,fill}%
\end{pgfscope}%
\begin{pgfscope}%
\pgfpathrectangle{\pgfqpoint{0.100000in}{0.212622in}}{\pgfqpoint{3.696000in}{3.696000in}}%
\pgfusepath{clip}%
\pgfsetbuttcap%
\pgfsetroundjoin%
\definecolor{currentfill}{rgb}{0.121569,0.466667,0.705882}%
\pgfsetfillcolor{currentfill}%
\pgfsetfillopacity{0.933946}%
\pgfsetlinewidth{1.003750pt}%
\definecolor{currentstroke}{rgb}{0.121569,0.466667,0.705882}%
\pgfsetstrokecolor{currentstroke}%
\pgfsetstrokeopacity{0.933946}%
\pgfsetdash{}{0pt}%
\pgfpathmoveto{\pgfqpoint{1.732638in}{1.758092in}}%
\pgfpathcurveto{\pgfqpoint{1.740875in}{1.758092in}}{\pgfqpoint{1.748775in}{1.761365in}}{\pgfqpoint{1.754599in}{1.767189in}}%
\pgfpathcurveto{\pgfqpoint{1.760422in}{1.773013in}}{\pgfqpoint{1.763695in}{1.780913in}}{\pgfqpoint{1.763695in}{1.789149in}}%
\pgfpathcurveto{\pgfqpoint{1.763695in}{1.797385in}}{\pgfqpoint{1.760422in}{1.805285in}}{\pgfqpoint{1.754599in}{1.811109in}}%
\pgfpathcurveto{\pgfqpoint{1.748775in}{1.816933in}}{\pgfqpoint{1.740875in}{1.820205in}}{\pgfqpoint{1.732638in}{1.820205in}}%
\pgfpathcurveto{\pgfqpoint{1.724402in}{1.820205in}}{\pgfqpoint{1.716502in}{1.816933in}}{\pgfqpoint{1.710678in}{1.811109in}}%
\pgfpathcurveto{\pgfqpoint{1.704854in}{1.805285in}}{\pgfqpoint{1.701582in}{1.797385in}}{\pgfqpoint{1.701582in}{1.789149in}}%
\pgfpathcurveto{\pgfqpoint{1.701582in}{1.780913in}}{\pgfqpoint{1.704854in}{1.773013in}}{\pgfqpoint{1.710678in}{1.767189in}}%
\pgfpathcurveto{\pgfqpoint{1.716502in}{1.761365in}}{\pgfqpoint{1.724402in}{1.758092in}}{\pgfqpoint{1.732638in}{1.758092in}}%
\pgfpathclose%
\pgfusepath{stroke,fill}%
\end{pgfscope}%
\begin{pgfscope}%
\pgfpathrectangle{\pgfqpoint{0.100000in}{0.212622in}}{\pgfqpoint{3.696000in}{3.696000in}}%
\pgfusepath{clip}%
\pgfsetbuttcap%
\pgfsetroundjoin%
\definecolor{currentfill}{rgb}{0.121569,0.466667,0.705882}%
\pgfsetfillcolor{currentfill}%
\pgfsetfillopacity{0.937502}%
\pgfsetlinewidth{1.003750pt}%
\definecolor{currentstroke}{rgb}{0.121569,0.466667,0.705882}%
\pgfsetstrokecolor{currentstroke}%
\pgfsetstrokeopacity{0.937502}%
\pgfsetdash{}{0pt}%
\pgfpathmoveto{\pgfqpoint{1.771125in}{1.734809in}}%
\pgfpathcurveto{\pgfqpoint{1.779361in}{1.734809in}}{\pgfqpoint{1.787261in}{1.738082in}}{\pgfqpoint{1.793085in}{1.743905in}}%
\pgfpathcurveto{\pgfqpoint{1.798909in}{1.749729in}}{\pgfqpoint{1.802181in}{1.757629in}}{\pgfqpoint{1.802181in}{1.765866in}}%
\pgfpathcurveto{\pgfqpoint{1.802181in}{1.774102in}}{\pgfqpoint{1.798909in}{1.782002in}}{\pgfqpoint{1.793085in}{1.787826in}}%
\pgfpathcurveto{\pgfqpoint{1.787261in}{1.793650in}}{\pgfqpoint{1.779361in}{1.796922in}}{\pgfqpoint{1.771125in}{1.796922in}}%
\pgfpathcurveto{\pgfqpoint{1.762889in}{1.796922in}}{\pgfqpoint{1.754989in}{1.793650in}}{\pgfqpoint{1.749165in}{1.787826in}}%
\pgfpathcurveto{\pgfqpoint{1.743341in}{1.782002in}}{\pgfqpoint{1.740068in}{1.774102in}}{\pgfqpoint{1.740068in}{1.765866in}}%
\pgfpathcurveto{\pgfqpoint{1.740068in}{1.757629in}}{\pgfqpoint{1.743341in}{1.749729in}}{\pgfqpoint{1.749165in}{1.743905in}}%
\pgfpathcurveto{\pgfqpoint{1.754989in}{1.738082in}}{\pgfqpoint{1.762889in}{1.734809in}}{\pgfqpoint{1.771125in}{1.734809in}}%
\pgfpathclose%
\pgfusepath{stroke,fill}%
\end{pgfscope}%
\begin{pgfscope}%
\pgfpathrectangle{\pgfqpoint{0.100000in}{0.212622in}}{\pgfqpoint{3.696000in}{3.696000in}}%
\pgfusepath{clip}%
\pgfsetbuttcap%
\pgfsetroundjoin%
\definecolor{currentfill}{rgb}{0.121569,0.466667,0.705882}%
\pgfsetfillcolor{currentfill}%
\pgfsetfillopacity{0.938570}%
\pgfsetlinewidth{1.003750pt}%
\definecolor{currentstroke}{rgb}{0.121569,0.466667,0.705882}%
\pgfsetstrokecolor{currentstroke}%
\pgfsetstrokeopacity{0.938570}%
\pgfsetdash{}{0pt}%
\pgfpathmoveto{\pgfqpoint{2.368911in}{1.391219in}}%
\pgfpathcurveto{\pgfqpoint{2.377148in}{1.391219in}}{\pgfqpoint{2.385048in}{1.394491in}}{\pgfqpoint{2.390872in}{1.400315in}}%
\pgfpathcurveto{\pgfqpoint{2.396696in}{1.406139in}}{\pgfqpoint{2.399968in}{1.414039in}}{\pgfqpoint{2.399968in}{1.422276in}}%
\pgfpathcurveto{\pgfqpoint{2.399968in}{1.430512in}}{\pgfqpoint{2.396696in}{1.438412in}}{\pgfqpoint{2.390872in}{1.444236in}}%
\pgfpathcurveto{\pgfqpoint{2.385048in}{1.450060in}}{\pgfqpoint{2.377148in}{1.453332in}}{\pgfqpoint{2.368911in}{1.453332in}}%
\pgfpathcurveto{\pgfqpoint{2.360675in}{1.453332in}}{\pgfqpoint{2.352775in}{1.450060in}}{\pgfqpoint{2.346951in}{1.444236in}}%
\pgfpathcurveto{\pgfqpoint{2.341127in}{1.438412in}}{\pgfqpoint{2.337855in}{1.430512in}}{\pgfqpoint{2.337855in}{1.422276in}}%
\pgfpathcurveto{\pgfqpoint{2.337855in}{1.414039in}}{\pgfqpoint{2.341127in}{1.406139in}}{\pgfqpoint{2.346951in}{1.400315in}}%
\pgfpathcurveto{\pgfqpoint{2.352775in}{1.394491in}}{\pgfqpoint{2.360675in}{1.391219in}}{\pgfqpoint{2.368911in}{1.391219in}}%
\pgfpathclose%
\pgfusepath{stroke,fill}%
\end{pgfscope}%
\begin{pgfscope}%
\pgfpathrectangle{\pgfqpoint{0.100000in}{0.212622in}}{\pgfqpoint{3.696000in}{3.696000in}}%
\pgfusepath{clip}%
\pgfsetbuttcap%
\pgfsetroundjoin%
\definecolor{currentfill}{rgb}{0.121569,0.466667,0.705882}%
\pgfsetfillcolor{currentfill}%
\pgfsetfillopacity{0.941524}%
\pgfsetlinewidth{1.003750pt}%
\definecolor{currentstroke}{rgb}{0.121569,0.466667,0.705882}%
\pgfsetstrokecolor{currentstroke}%
\pgfsetstrokeopacity{0.941524}%
\pgfsetdash{}{0pt}%
\pgfpathmoveto{\pgfqpoint{1.808439in}{1.715585in}}%
\pgfpathcurveto{\pgfqpoint{1.816675in}{1.715585in}}{\pgfqpoint{1.824575in}{1.718857in}}{\pgfqpoint{1.830399in}{1.724681in}}%
\pgfpathcurveto{\pgfqpoint{1.836223in}{1.730505in}}{\pgfqpoint{1.839495in}{1.738405in}}{\pgfqpoint{1.839495in}{1.746641in}}%
\pgfpathcurveto{\pgfqpoint{1.839495in}{1.754878in}}{\pgfqpoint{1.836223in}{1.762778in}}{\pgfqpoint{1.830399in}{1.768602in}}%
\pgfpathcurveto{\pgfqpoint{1.824575in}{1.774426in}}{\pgfqpoint{1.816675in}{1.777698in}}{\pgfqpoint{1.808439in}{1.777698in}}%
\pgfpathcurveto{\pgfqpoint{1.800203in}{1.777698in}}{\pgfqpoint{1.792303in}{1.774426in}}{\pgfqpoint{1.786479in}{1.768602in}}%
\pgfpathcurveto{\pgfqpoint{1.780655in}{1.762778in}}{\pgfqpoint{1.777382in}{1.754878in}}{\pgfqpoint{1.777382in}{1.746641in}}%
\pgfpathcurveto{\pgfqpoint{1.777382in}{1.738405in}}{\pgfqpoint{1.780655in}{1.730505in}}{\pgfqpoint{1.786479in}{1.724681in}}%
\pgfpathcurveto{\pgfqpoint{1.792303in}{1.718857in}}{\pgfqpoint{1.800203in}{1.715585in}}{\pgfqpoint{1.808439in}{1.715585in}}%
\pgfpathclose%
\pgfusepath{stroke,fill}%
\end{pgfscope}%
\begin{pgfscope}%
\pgfpathrectangle{\pgfqpoint{0.100000in}{0.212622in}}{\pgfqpoint{3.696000in}{3.696000in}}%
\pgfusepath{clip}%
\pgfsetbuttcap%
\pgfsetroundjoin%
\definecolor{currentfill}{rgb}{0.121569,0.466667,0.705882}%
\pgfsetfillcolor{currentfill}%
\pgfsetfillopacity{0.943614}%
\pgfsetlinewidth{1.003750pt}%
\definecolor{currentstroke}{rgb}{0.121569,0.466667,0.705882}%
\pgfsetstrokecolor{currentstroke}%
\pgfsetstrokeopacity{0.943614}%
\pgfsetdash{}{0pt}%
\pgfpathmoveto{\pgfqpoint{1.842136in}{1.684196in}}%
\pgfpathcurveto{\pgfqpoint{1.850372in}{1.684196in}}{\pgfqpoint{1.858272in}{1.687468in}}{\pgfqpoint{1.864096in}{1.693292in}}%
\pgfpathcurveto{\pgfqpoint{1.869920in}{1.699116in}}{\pgfqpoint{1.873192in}{1.707016in}}{\pgfqpoint{1.873192in}{1.715252in}}%
\pgfpathcurveto{\pgfqpoint{1.873192in}{1.723489in}}{\pgfqpoint{1.869920in}{1.731389in}}{\pgfqpoint{1.864096in}{1.737213in}}%
\pgfpathcurveto{\pgfqpoint{1.858272in}{1.743036in}}{\pgfqpoint{1.850372in}{1.746309in}}{\pgfqpoint{1.842136in}{1.746309in}}%
\pgfpathcurveto{\pgfqpoint{1.833899in}{1.746309in}}{\pgfqpoint{1.825999in}{1.743036in}}{\pgfqpoint{1.820175in}{1.737213in}}%
\pgfpathcurveto{\pgfqpoint{1.814351in}{1.731389in}}{\pgfqpoint{1.811079in}{1.723489in}}{\pgfqpoint{1.811079in}{1.715252in}}%
\pgfpathcurveto{\pgfqpoint{1.811079in}{1.707016in}}{\pgfqpoint{1.814351in}{1.699116in}}{\pgfqpoint{1.820175in}{1.693292in}}%
\pgfpathcurveto{\pgfqpoint{1.825999in}{1.687468in}}{\pgfqpoint{1.833899in}{1.684196in}}{\pgfqpoint{1.842136in}{1.684196in}}%
\pgfpathclose%
\pgfusepath{stroke,fill}%
\end{pgfscope}%
\begin{pgfscope}%
\pgfpathrectangle{\pgfqpoint{0.100000in}{0.212622in}}{\pgfqpoint{3.696000in}{3.696000in}}%
\pgfusepath{clip}%
\pgfsetbuttcap%
\pgfsetroundjoin%
\definecolor{currentfill}{rgb}{0.121569,0.466667,0.705882}%
\pgfsetfillcolor{currentfill}%
\pgfsetfillopacity{0.944658}%
\pgfsetlinewidth{1.003750pt}%
\definecolor{currentstroke}{rgb}{0.121569,0.466667,0.705882}%
\pgfsetstrokecolor{currentstroke}%
\pgfsetstrokeopacity{0.944658}%
\pgfsetdash{}{0pt}%
\pgfpathmoveto{\pgfqpoint{2.373013in}{1.389622in}}%
\pgfpathcurveto{\pgfqpoint{2.381249in}{1.389622in}}{\pgfqpoint{2.389149in}{1.392895in}}{\pgfqpoint{2.394973in}{1.398719in}}%
\pgfpathcurveto{\pgfqpoint{2.400797in}{1.404543in}}{\pgfqpoint{2.404069in}{1.412443in}}{\pgfqpoint{2.404069in}{1.420679in}}%
\pgfpathcurveto{\pgfqpoint{2.404069in}{1.428915in}}{\pgfqpoint{2.400797in}{1.436815in}}{\pgfqpoint{2.394973in}{1.442639in}}%
\pgfpathcurveto{\pgfqpoint{2.389149in}{1.448463in}}{\pgfqpoint{2.381249in}{1.451735in}}{\pgfqpoint{2.373013in}{1.451735in}}%
\pgfpathcurveto{\pgfqpoint{2.364776in}{1.451735in}}{\pgfqpoint{2.356876in}{1.448463in}}{\pgfqpoint{2.351052in}{1.442639in}}%
\pgfpathcurveto{\pgfqpoint{2.345228in}{1.436815in}}{\pgfqpoint{2.341956in}{1.428915in}}{\pgfqpoint{2.341956in}{1.420679in}}%
\pgfpathcurveto{\pgfqpoint{2.341956in}{1.412443in}}{\pgfqpoint{2.345228in}{1.404543in}}{\pgfqpoint{2.351052in}{1.398719in}}%
\pgfpathcurveto{\pgfqpoint{2.356876in}{1.392895in}}{\pgfqpoint{2.364776in}{1.389622in}}{\pgfqpoint{2.373013in}{1.389622in}}%
\pgfpathclose%
\pgfusepath{stroke,fill}%
\end{pgfscope}%
\begin{pgfscope}%
\pgfpathrectangle{\pgfqpoint{0.100000in}{0.212622in}}{\pgfqpoint{3.696000in}{3.696000in}}%
\pgfusepath{clip}%
\pgfsetbuttcap%
\pgfsetroundjoin%
\definecolor{currentfill}{rgb}{0.121569,0.466667,0.705882}%
\pgfsetfillcolor{currentfill}%
\pgfsetfillopacity{0.946181}%
\pgfsetlinewidth{1.003750pt}%
\definecolor{currentstroke}{rgb}{0.121569,0.466667,0.705882}%
\pgfsetstrokecolor{currentstroke}%
\pgfsetstrokeopacity{0.946181}%
\pgfsetdash{}{0pt}%
\pgfpathmoveto{\pgfqpoint{1.874279in}{1.660043in}}%
\pgfpathcurveto{\pgfqpoint{1.882515in}{1.660043in}}{\pgfqpoint{1.890415in}{1.663316in}}{\pgfqpoint{1.896239in}{1.669140in}}%
\pgfpathcurveto{\pgfqpoint{1.902063in}{1.674964in}}{\pgfqpoint{1.905335in}{1.682864in}}{\pgfqpoint{1.905335in}{1.691100in}}%
\pgfpathcurveto{\pgfqpoint{1.905335in}{1.699336in}}{\pgfqpoint{1.902063in}{1.707236in}}{\pgfqpoint{1.896239in}{1.713060in}}%
\pgfpathcurveto{\pgfqpoint{1.890415in}{1.718884in}}{\pgfqpoint{1.882515in}{1.722156in}}{\pgfqpoint{1.874279in}{1.722156in}}%
\pgfpathcurveto{\pgfqpoint{1.866042in}{1.722156in}}{\pgfqpoint{1.858142in}{1.718884in}}{\pgfqpoint{1.852318in}{1.713060in}}%
\pgfpathcurveto{\pgfqpoint{1.846494in}{1.707236in}}{\pgfqpoint{1.843222in}{1.699336in}}{\pgfqpoint{1.843222in}{1.691100in}}%
\pgfpathcurveto{\pgfqpoint{1.843222in}{1.682864in}}{\pgfqpoint{1.846494in}{1.674964in}}{\pgfqpoint{1.852318in}{1.669140in}}%
\pgfpathcurveto{\pgfqpoint{1.858142in}{1.663316in}}{\pgfqpoint{1.866042in}{1.660043in}}{\pgfqpoint{1.874279in}{1.660043in}}%
\pgfpathclose%
\pgfusepath{stroke,fill}%
\end{pgfscope}%
\begin{pgfscope}%
\pgfpathrectangle{\pgfqpoint{0.100000in}{0.212622in}}{\pgfqpoint{3.696000in}{3.696000in}}%
\pgfusepath{clip}%
\pgfsetbuttcap%
\pgfsetroundjoin%
\definecolor{currentfill}{rgb}{0.121569,0.466667,0.705882}%
\pgfsetfillcolor{currentfill}%
\pgfsetfillopacity{0.949818}%
\pgfsetlinewidth{1.003750pt}%
\definecolor{currentstroke}{rgb}{0.121569,0.466667,0.705882}%
\pgfsetstrokecolor{currentstroke}%
\pgfsetstrokeopacity{0.949818}%
\pgfsetdash{}{0pt}%
\pgfpathmoveto{\pgfqpoint{2.376083in}{1.378830in}}%
\pgfpathcurveto{\pgfqpoint{2.384319in}{1.378830in}}{\pgfqpoint{2.392219in}{1.382102in}}{\pgfqpoint{2.398043in}{1.387926in}}%
\pgfpathcurveto{\pgfqpoint{2.403867in}{1.393750in}}{\pgfqpoint{2.407139in}{1.401650in}}{\pgfqpoint{2.407139in}{1.409886in}}%
\pgfpathcurveto{\pgfqpoint{2.407139in}{1.418123in}}{\pgfqpoint{2.403867in}{1.426023in}}{\pgfqpoint{2.398043in}{1.431847in}}%
\pgfpathcurveto{\pgfqpoint{2.392219in}{1.437671in}}{\pgfqpoint{2.384319in}{1.440943in}}{\pgfqpoint{2.376083in}{1.440943in}}%
\pgfpathcurveto{\pgfqpoint{2.367846in}{1.440943in}}{\pgfqpoint{2.359946in}{1.437671in}}{\pgfqpoint{2.354122in}{1.431847in}}%
\pgfpathcurveto{\pgfqpoint{2.348298in}{1.426023in}}{\pgfqpoint{2.345026in}{1.418123in}}{\pgfqpoint{2.345026in}{1.409886in}}%
\pgfpathcurveto{\pgfqpoint{2.345026in}{1.401650in}}{\pgfqpoint{2.348298in}{1.393750in}}{\pgfqpoint{2.354122in}{1.387926in}}%
\pgfpathcurveto{\pgfqpoint{2.359946in}{1.382102in}}{\pgfqpoint{2.367846in}{1.378830in}}{\pgfqpoint{2.376083in}{1.378830in}}%
\pgfpathclose%
\pgfusepath{stroke,fill}%
\end{pgfscope}%
\begin{pgfscope}%
\pgfpathrectangle{\pgfqpoint{0.100000in}{0.212622in}}{\pgfqpoint{3.696000in}{3.696000in}}%
\pgfusepath{clip}%
\pgfsetbuttcap%
\pgfsetroundjoin%
\definecolor{currentfill}{rgb}{0.121569,0.466667,0.705882}%
\pgfsetfillcolor{currentfill}%
\pgfsetfillopacity{0.950228}%
\pgfsetlinewidth{1.003750pt}%
\definecolor{currentstroke}{rgb}{0.121569,0.466667,0.705882}%
\pgfsetstrokecolor{currentstroke}%
\pgfsetstrokeopacity{0.950228}%
\pgfsetdash{}{0pt}%
\pgfpathmoveto{\pgfqpoint{1.903096in}{1.645467in}}%
\pgfpathcurveto{\pgfqpoint{1.911332in}{1.645467in}}{\pgfqpoint{1.919232in}{1.648740in}}{\pgfqpoint{1.925056in}{1.654563in}}%
\pgfpathcurveto{\pgfqpoint{1.930880in}{1.660387in}}{\pgfqpoint{1.934153in}{1.668287in}}{\pgfqpoint{1.934153in}{1.676524in}}%
\pgfpathcurveto{\pgfqpoint{1.934153in}{1.684760in}}{\pgfqpoint{1.930880in}{1.692660in}}{\pgfqpoint{1.925056in}{1.698484in}}%
\pgfpathcurveto{\pgfqpoint{1.919232in}{1.704308in}}{\pgfqpoint{1.911332in}{1.707580in}}{\pgfqpoint{1.903096in}{1.707580in}}%
\pgfpathcurveto{\pgfqpoint{1.894860in}{1.707580in}}{\pgfqpoint{1.886960in}{1.704308in}}{\pgfqpoint{1.881136in}{1.698484in}}%
\pgfpathcurveto{\pgfqpoint{1.875312in}{1.692660in}}{\pgfqpoint{1.872040in}{1.684760in}}{\pgfqpoint{1.872040in}{1.676524in}}%
\pgfpathcurveto{\pgfqpoint{1.872040in}{1.668287in}}{\pgfqpoint{1.875312in}{1.660387in}}{\pgfqpoint{1.881136in}{1.654563in}}%
\pgfpathcurveto{\pgfqpoint{1.886960in}{1.648740in}}{\pgfqpoint{1.894860in}{1.645467in}}{\pgfqpoint{1.903096in}{1.645467in}}%
\pgfpathclose%
\pgfusepath{stroke,fill}%
\end{pgfscope}%
\begin{pgfscope}%
\pgfpathrectangle{\pgfqpoint{0.100000in}{0.212622in}}{\pgfqpoint{3.696000in}{3.696000in}}%
\pgfusepath{clip}%
\pgfsetbuttcap%
\pgfsetroundjoin%
\definecolor{currentfill}{rgb}{0.121569,0.466667,0.705882}%
\pgfsetfillcolor{currentfill}%
\pgfsetfillopacity{0.953008}%
\pgfsetlinewidth{1.003750pt}%
\definecolor{currentstroke}{rgb}{0.121569,0.466667,0.705882}%
\pgfsetstrokecolor{currentstroke}%
\pgfsetstrokeopacity{0.953008}%
\pgfsetdash{}{0pt}%
\pgfpathmoveto{\pgfqpoint{1.930932in}{1.631091in}}%
\pgfpathcurveto{\pgfqpoint{1.939169in}{1.631091in}}{\pgfqpoint{1.947069in}{1.634364in}}{\pgfqpoint{1.952892in}{1.640188in}}%
\pgfpathcurveto{\pgfqpoint{1.958716in}{1.646012in}}{\pgfqpoint{1.961989in}{1.653912in}}{\pgfqpoint{1.961989in}{1.662148in}}%
\pgfpathcurveto{\pgfqpoint{1.961989in}{1.670384in}}{\pgfqpoint{1.958716in}{1.678284in}}{\pgfqpoint{1.952892in}{1.684108in}}%
\pgfpathcurveto{\pgfqpoint{1.947069in}{1.689932in}}{\pgfqpoint{1.939169in}{1.693204in}}{\pgfqpoint{1.930932in}{1.693204in}}%
\pgfpathcurveto{\pgfqpoint{1.922696in}{1.693204in}}{\pgfqpoint{1.914796in}{1.689932in}}{\pgfqpoint{1.908972in}{1.684108in}}%
\pgfpathcurveto{\pgfqpoint{1.903148in}{1.678284in}}{\pgfqpoint{1.899876in}{1.670384in}}{\pgfqpoint{1.899876in}{1.662148in}}%
\pgfpathcurveto{\pgfqpoint{1.899876in}{1.653912in}}{\pgfqpoint{1.903148in}{1.646012in}}{\pgfqpoint{1.908972in}{1.640188in}}%
\pgfpathcurveto{\pgfqpoint{1.914796in}{1.634364in}}{\pgfqpoint{1.922696in}{1.631091in}}{\pgfqpoint{1.930932in}{1.631091in}}%
\pgfpathclose%
\pgfusepath{stroke,fill}%
\end{pgfscope}%
\begin{pgfscope}%
\pgfpathrectangle{\pgfqpoint{0.100000in}{0.212622in}}{\pgfqpoint{3.696000in}{3.696000in}}%
\pgfusepath{clip}%
\pgfsetbuttcap%
\pgfsetroundjoin%
\definecolor{currentfill}{rgb}{0.121569,0.466667,0.705882}%
\pgfsetfillcolor{currentfill}%
\pgfsetfillopacity{0.954108}%
\pgfsetlinewidth{1.003750pt}%
\definecolor{currentstroke}{rgb}{0.121569,0.466667,0.705882}%
\pgfsetstrokecolor{currentstroke}%
\pgfsetstrokeopacity{0.954108}%
\pgfsetdash{}{0pt}%
\pgfpathmoveto{\pgfqpoint{1.954393in}{1.605111in}}%
\pgfpathcurveto{\pgfqpoint{1.962629in}{1.605111in}}{\pgfqpoint{1.970529in}{1.608383in}}{\pgfqpoint{1.976353in}{1.614207in}}%
\pgfpathcurveto{\pgfqpoint{1.982177in}{1.620031in}}{\pgfqpoint{1.985449in}{1.627931in}}{\pgfqpoint{1.985449in}{1.636167in}}%
\pgfpathcurveto{\pgfqpoint{1.985449in}{1.644404in}}{\pgfqpoint{1.982177in}{1.652304in}}{\pgfqpoint{1.976353in}{1.658128in}}%
\pgfpathcurveto{\pgfqpoint{1.970529in}{1.663952in}}{\pgfqpoint{1.962629in}{1.667224in}}{\pgfqpoint{1.954393in}{1.667224in}}%
\pgfpathcurveto{\pgfqpoint{1.946156in}{1.667224in}}{\pgfqpoint{1.938256in}{1.663952in}}{\pgfqpoint{1.932432in}{1.658128in}}%
\pgfpathcurveto{\pgfqpoint{1.926609in}{1.652304in}}{\pgfqpoint{1.923336in}{1.644404in}}{\pgfqpoint{1.923336in}{1.636167in}}%
\pgfpathcurveto{\pgfqpoint{1.923336in}{1.627931in}}{\pgfqpoint{1.926609in}{1.620031in}}{\pgfqpoint{1.932432in}{1.614207in}}%
\pgfpathcurveto{\pgfqpoint{1.938256in}{1.608383in}}{\pgfqpoint{1.946156in}{1.605111in}}{\pgfqpoint{1.954393in}{1.605111in}}%
\pgfpathclose%
\pgfusepath{stroke,fill}%
\end{pgfscope}%
\begin{pgfscope}%
\pgfpathrectangle{\pgfqpoint{0.100000in}{0.212622in}}{\pgfqpoint{3.696000in}{3.696000in}}%
\pgfusepath{clip}%
\pgfsetbuttcap%
\pgfsetroundjoin%
\definecolor{currentfill}{rgb}{0.121569,0.466667,0.705882}%
\pgfsetfillcolor{currentfill}%
\pgfsetfillopacity{0.955449}%
\pgfsetlinewidth{1.003750pt}%
\definecolor{currentstroke}{rgb}{0.121569,0.466667,0.705882}%
\pgfsetstrokecolor{currentstroke}%
\pgfsetstrokeopacity{0.955449}%
\pgfsetdash{}{0pt}%
\pgfpathmoveto{\pgfqpoint{2.380541in}{1.369011in}}%
\pgfpathcurveto{\pgfqpoint{2.388777in}{1.369011in}}{\pgfqpoint{2.396677in}{1.372284in}}{\pgfqpoint{2.402501in}{1.378108in}}%
\pgfpathcurveto{\pgfqpoint{2.408325in}{1.383932in}}{\pgfqpoint{2.411597in}{1.391832in}}{\pgfqpoint{2.411597in}{1.400068in}}%
\pgfpathcurveto{\pgfqpoint{2.411597in}{1.408304in}}{\pgfqpoint{2.408325in}{1.416204in}}{\pgfqpoint{2.402501in}{1.422028in}}%
\pgfpathcurveto{\pgfqpoint{2.396677in}{1.427852in}}{\pgfqpoint{2.388777in}{1.431124in}}{\pgfqpoint{2.380541in}{1.431124in}}%
\pgfpathcurveto{\pgfqpoint{2.372304in}{1.431124in}}{\pgfqpoint{2.364404in}{1.427852in}}{\pgfqpoint{2.358580in}{1.422028in}}%
\pgfpathcurveto{\pgfqpoint{2.352756in}{1.416204in}}{\pgfqpoint{2.349484in}{1.408304in}}{\pgfqpoint{2.349484in}{1.400068in}}%
\pgfpathcurveto{\pgfqpoint{2.349484in}{1.391832in}}{\pgfqpoint{2.352756in}{1.383932in}}{\pgfqpoint{2.358580in}{1.378108in}}%
\pgfpathcurveto{\pgfqpoint{2.364404in}{1.372284in}}{\pgfqpoint{2.372304in}{1.369011in}}{\pgfqpoint{2.380541in}{1.369011in}}%
\pgfpathclose%
\pgfusepath{stroke,fill}%
\end{pgfscope}%
\begin{pgfscope}%
\pgfpathrectangle{\pgfqpoint{0.100000in}{0.212622in}}{\pgfqpoint{3.696000in}{3.696000in}}%
\pgfusepath{clip}%
\pgfsetbuttcap%
\pgfsetroundjoin%
\definecolor{currentfill}{rgb}{0.121569,0.466667,0.705882}%
\pgfsetfillcolor{currentfill}%
\pgfsetfillopacity{0.955762}%
\pgfsetlinewidth{1.003750pt}%
\definecolor{currentstroke}{rgb}{0.121569,0.466667,0.705882}%
\pgfsetstrokecolor{currentstroke}%
\pgfsetstrokeopacity{0.955762}%
\pgfsetdash{}{0pt}%
\pgfpathmoveto{\pgfqpoint{1.974749in}{1.593755in}}%
\pgfpathcurveto{\pgfqpoint{1.982986in}{1.593755in}}{\pgfqpoint{1.990886in}{1.597027in}}{\pgfqpoint{1.996710in}{1.602851in}}%
\pgfpathcurveto{\pgfqpoint{2.002533in}{1.608675in}}{\pgfqpoint{2.005806in}{1.616575in}}{\pgfqpoint{2.005806in}{1.624811in}}%
\pgfpathcurveto{\pgfqpoint{2.005806in}{1.633048in}}{\pgfqpoint{2.002533in}{1.640948in}}{\pgfqpoint{1.996710in}{1.646772in}}%
\pgfpathcurveto{\pgfqpoint{1.990886in}{1.652595in}}{\pgfqpoint{1.982986in}{1.655868in}}{\pgfqpoint{1.974749in}{1.655868in}}%
\pgfpathcurveto{\pgfqpoint{1.966513in}{1.655868in}}{\pgfqpoint{1.958613in}{1.652595in}}{\pgfqpoint{1.952789in}{1.646772in}}%
\pgfpathcurveto{\pgfqpoint{1.946965in}{1.640948in}}{\pgfqpoint{1.943693in}{1.633048in}}{\pgfqpoint{1.943693in}{1.624811in}}%
\pgfpathcurveto{\pgfqpoint{1.943693in}{1.616575in}}{\pgfqpoint{1.946965in}{1.608675in}}{\pgfqpoint{1.952789in}{1.602851in}}%
\pgfpathcurveto{\pgfqpoint{1.958613in}{1.597027in}}{\pgfqpoint{1.966513in}{1.593755in}}{\pgfqpoint{1.974749in}{1.593755in}}%
\pgfpathclose%
\pgfusepath{stroke,fill}%
\end{pgfscope}%
\begin{pgfscope}%
\pgfpathrectangle{\pgfqpoint{0.100000in}{0.212622in}}{\pgfqpoint{3.696000in}{3.696000in}}%
\pgfusepath{clip}%
\pgfsetbuttcap%
\pgfsetroundjoin%
\definecolor{currentfill}{rgb}{0.121569,0.466667,0.705882}%
\pgfsetfillcolor{currentfill}%
\pgfsetfillopacity{0.957715}%
\pgfsetlinewidth{1.003750pt}%
\definecolor{currentstroke}{rgb}{0.121569,0.466667,0.705882}%
\pgfsetstrokecolor{currentstroke}%
\pgfsetstrokeopacity{0.957715}%
\pgfsetdash{}{0pt}%
\pgfpathmoveto{\pgfqpoint{1.992573in}{1.584171in}}%
\pgfpathcurveto{\pgfqpoint{2.000810in}{1.584171in}}{\pgfqpoint{2.008710in}{1.587443in}}{\pgfqpoint{2.014534in}{1.593267in}}%
\pgfpathcurveto{\pgfqpoint{2.020358in}{1.599091in}}{\pgfqpoint{2.023630in}{1.606991in}}{\pgfqpoint{2.023630in}{1.615227in}}%
\pgfpathcurveto{\pgfqpoint{2.023630in}{1.623463in}}{\pgfqpoint{2.020358in}{1.631363in}}{\pgfqpoint{2.014534in}{1.637187in}}%
\pgfpathcurveto{\pgfqpoint{2.008710in}{1.643011in}}{\pgfqpoint{2.000810in}{1.646284in}}{\pgfqpoint{1.992573in}{1.646284in}}%
\pgfpathcurveto{\pgfqpoint{1.984337in}{1.646284in}}{\pgfqpoint{1.976437in}{1.643011in}}{\pgfqpoint{1.970613in}{1.637187in}}%
\pgfpathcurveto{\pgfqpoint{1.964789in}{1.631363in}}{\pgfqpoint{1.961517in}{1.623463in}}{\pgfqpoint{1.961517in}{1.615227in}}%
\pgfpathcurveto{\pgfqpoint{1.961517in}{1.606991in}}{\pgfqpoint{1.964789in}{1.599091in}}{\pgfqpoint{1.970613in}{1.593267in}}%
\pgfpathcurveto{\pgfqpoint{1.976437in}{1.587443in}}{\pgfqpoint{1.984337in}{1.584171in}}{\pgfqpoint{1.992573in}{1.584171in}}%
\pgfpathclose%
\pgfusepath{stroke,fill}%
\end{pgfscope}%
\begin{pgfscope}%
\pgfpathrectangle{\pgfqpoint{0.100000in}{0.212622in}}{\pgfqpoint{3.696000in}{3.696000in}}%
\pgfusepath{clip}%
\pgfsetbuttcap%
\pgfsetroundjoin%
\definecolor{currentfill}{rgb}{0.121569,0.466667,0.705882}%
\pgfsetfillcolor{currentfill}%
\pgfsetfillopacity{0.958750}%
\pgfsetlinewidth{1.003750pt}%
\definecolor{currentstroke}{rgb}{0.121569,0.466667,0.705882}%
\pgfsetstrokecolor{currentstroke}%
\pgfsetstrokeopacity{0.958750}%
\pgfsetdash{}{0pt}%
\pgfpathmoveto{\pgfqpoint{2.009034in}{1.572776in}}%
\pgfpathcurveto{\pgfqpoint{2.017270in}{1.572776in}}{\pgfqpoint{2.025170in}{1.576049in}}{\pgfqpoint{2.030994in}{1.581872in}}%
\pgfpathcurveto{\pgfqpoint{2.036818in}{1.587696in}}{\pgfqpoint{2.040091in}{1.595596in}}{\pgfqpoint{2.040091in}{1.603833in}}%
\pgfpathcurveto{\pgfqpoint{2.040091in}{1.612069in}}{\pgfqpoint{2.036818in}{1.619969in}}{\pgfqpoint{2.030994in}{1.625793in}}%
\pgfpathcurveto{\pgfqpoint{2.025170in}{1.631617in}}{\pgfqpoint{2.017270in}{1.634889in}}{\pgfqpoint{2.009034in}{1.634889in}}%
\pgfpathcurveto{\pgfqpoint{2.000798in}{1.634889in}}{\pgfqpoint{1.992898in}{1.631617in}}{\pgfqpoint{1.987074in}{1.625793in}}%
\pgfpathcurveto{\pgfqpoint{1.981250in}{1.619969in}}{\pgfqpoint{1.977978in}{1.612069in}}{\pgfqpoint{1.977978in}{1.603833in}}%
\pgfpathcurveto{\pgfqpoint{1.977978in}{1.595596in}}{\pgfqpoint{1.981250in}{1.587696in}}{\pgfqpoint{1.987074in}{1.581872in}}%
\pgfpathcurveto{\pgfqpoint{1.992898in}{1.576049in}}{\pgfqpoint{2.000798in}{1.572776in}}{\pgfqpoint{2.009034in}{1.572776in}}%
\pgfpathclose%
\pgfusepath{stroke,fill}%
\end{pgfscope}%
\begin{pgfscope}%
\pgfpathrectangle{\pgfqpoint{0.100000in}{0.212622in}}{\pgfqpoint{3.696000in}{3.696000in}}%
\pgfusepath{clip}%
\pgfsetbuttcap%
\pgfsetroundjoin%
\definecolor{currentfill}{rgb}{0.121569,0.466667,0.705882}%
\pgfsetfillcolor{currentfill}%
\pgfsetfillopacity{0.959863}%
\pgfsetlinewidth{1.003750pt}%
\definecolor{currentstroke}{rgb}{0.121569,0.466667,0.705882}%
\pgfsetstrokecolor{currentstroke}%
\pgfsetstrokeopacity{0.959863}%
\pgfsetdash{}{0pt}%
\pgfpathmoveto{\pgfqpoint{2.022758in}{1.562725in}}%
\pgfpathcurveto{\pgfqpoint{2.030995in}{1.562725in}}{\pgfqpoint{2.038895in}{1.565997in}}{\pgfqpoint{2.044719in}{1.571821in}}%
\pgfpathcurveto{\pgfqpoint{2.050543in}{1.577645in}}{\pgfqpoint{2.053815in}{1.585545in}}{\pgfqpoint{2.053815in}{1.593781in}}%
\pgfpathcurveto{\pgfqpoint{2.053815in}{1.602017in}}{\pgfqpoint{2.050543in}{1.609917in}}{\pgfqpoint{2.044719in}{1.615741in}}%
\pgfpathcurveto{\pgfqpoint{2.038895in}{1.621565in}}{\pgfqpoint{2.030995in}{1.624838in}}{\pgfqpoint{2.022758in}{1.624838in}}%
\pgfpathcurveto{\pgfqpoint{2.014522in}{1.624838in}}{\pgfqpoint{2.006622in}{1.621565in}}{\pgfqpoint{2.000798in}{1.615741in}}%
\pgfpathcurveto{\pgfqpoint{1.994974in}{1.609917in}}{\pgfqpoint{1.991702in}{1.602017in}}{\pgfqpoint{1.991702in}{1.593781in}}%
\pgfpathcurveto{\pgfqpoint{1.991702in}{1.585545in}}{\pgfqpoint{1.994974in}{1.577645in}}{\pgfqpoint{2.000798in}{1.571821in}}%
\pgfpathcurveto{\pgfqpoint{2.006622in}{1.565997in}}{\pgfqpoint{2.014522in}{1.562725in}}{\pgfqpoint{2.022758in}{1.562725in}}%
\pgfpathclose%
\pgfusepath{stroke,fill}%
\end{pgfscope}%
\begin{pgfscope}%
\pgfpathrectangle{\pgfqpoint{0.100000in}{0.212622in}}{\pgfqpoint{3.696000in}{3.696000in}}%
\pgfusepath{clip}%
\pgfsetbuttcap%
\pgfsetroundjoin%
\definecolor{currentfill}{rgb}{0.121569,0.466667,0.705882}%
\pgfsetfillcolor{currentfill}%
\pgfsetfillopacity{0.960602}%
\pgfsetlinewidth{1.003750pt}%
\definecolor{currentstroke}{rgb}{0.121569,0.466667,0.705882}%
\pgfsetstrokecolor{currentstroke}%
\pgfsetstrokeopacity{0.960602}%
\pgfsetdash{}{0pt}%
\pgfpathmoveto{\pgfqpoint{2.385647in}{1.354493in}}%
\pgfpathcurveto{\pgfqpoint{2.393883in}{1.354493in}}{\pgfqpoint{2.401783in}{1.357766in}}{\pgfqpoint{2.407607in}{1.363589in}}%
\pgfpathcurveto{\pgfqpoint{2.413431in}{1.369413in}}{\pgfqpoint{2.416704in}{1.377313in}}{\pgfqpoint{2.416704in}{1.385550in}}%
\pgfpathcurveto{\pgfqpoint{2.416704in}{1.393786in}}{\pgfqpoint{2.413431in}{1.401686in}}{\pgfqpoint{2.407607in}{1.407510in}}%
\pgfpathcurveto{\pgfqpoint{2.401783in}{1.413334in}}{\pgfqpoint{2.393883in}{1.416606in}}{\pgfqpoint{2.385647in}{1.416606in}}%
\pgfpathcurveto{\pgfqpoint{2.377411in}{1.416606in}}{\pgfqpoint{2.369511in}{1.413334in}}{\pgfqpoint{2.363687in}{1.407510in}}%
\pgfpathcurveto{\pgfqpoint{2.357863in}{1.401686in}}{\pgfqpoint{2.354591in}{1.393786in}}{\pgfqpoint{2.354591in}{1.385550in}}%
\pgfpathcurveto{\pgfqpoint{2.354591in}{1.377313in}}{\pgfqpoint{2.357863in}{1.369413in}}{\pgfqpoint{2.363687in}{1.363589in}}%
\pgfpathcurveto{\pgfqpoint{2.369511in}{1.357766in}}{\pgfqpoint{2.377411in}{1.354493in}}{\pgfqpoint{2.385647in}{1.354493in}}%
\pgfpathclose%
\pgfusepath{stroke,fill}%
\end{pgfscope}%
\begin{pgfscope}%
\pgfpathrectangle{\pgfqpoint{0.100000in}{0.212622in}}{\pgfqpoint{3.696000in}{3.696000in}}%
\pgfusepath{clip}%
\pgfsetbuttcap%
\pgfsetroundjoin%
\definecolor{currentfill}{rgb}{0.121569,0.466667,0.705882}%
\pgfsetfillcolor{currentfill}%
\pgfsetfillopacity{0.961343}%
\pgfsetlinewidth{1.003750pt}%
\definecolor{currentstroke}{rgb}{0.121569,0.466667,0.705882}%
\pgfsetstrokecolor{currentstroke}%
\pgfsetstrokeopacity{0.961343}%
\pgfsetdash{}{0pt}%
\pgfpathmoveto{\pgfqpoint{2.035031in}{1.553759in}}%
\pgfpathcurveto{\pgfqpoint{2.043267in}{1.553759in}}{\pgfqpoint{2.051167in}{1.557032in}}{\pgfqpoint{2.056991in}{1.562855in}}%
\pgfpathcurveto{\pgfqpoint{2.062815in}{1.568679in}}{\pgfqpoint{2.066087in}{1.576579in}}{\pgfqpoint{2.066087in}{1.584816in}}%
\pgfpathcurveto{\pgfqpoint{2.066087in}{1.593052in}}{\pgfqpoint{2.062815in}{1.600952in}}{\pgfqpoint{2.056991in}{1.606776in}}%
\pgfpathcurveto{\pgfqpoint{2.051167in}{1.612600in}}{\pgfqpoint{2.043267in}{1.615872in}}{\pgfqpoint{2.035031in}{1.615872in}}%
\pgfpathcurveto{\pgfqpoint{2.026794in}{1.615872in}}{\pgfqpoint{2.018894in}{1.612600in}}{\pgfqpoint{2.013070in}{1.606776in}}%
\pgfpathcurveto{\pgfqpoint{2.007247in}{1.600952in}}{\pgfqpoint{2.003974in}{1.593052in}}{\pgfqpoint{2.003974in}{1.584816in}}%
\pgfpathcurveto{\pgfqpoint{2.003974in}{1.576579in}}{\pgfqpoint{2.007247in}{1.568679in}}{\pgfqpoint{2.013070in}{1.562855in}}%
\pgfpathcurveto{\pgfqpoint{2.018894in}{1.557032in}}{\pgfqpoint{2.026794in}{1.553759in}}{\pgfqpoint{2.035031in}{1.553759in}}%
\pgfpathclose%
\pgfusepath{stroke,fill}%
\end{pgfscope}%
\begin{pgfscope}%
\pgfpathrectangle{\pgfqpoint{0.100000in}{0.212622in}}{\pgfqpoint{3.696000in}{3.696000in}}%
\pgfusepath{clip}%
\pgfsetbuttcap%
\pgfsetroundjoin%
\definecolor{currentfill}{rgb}{0.121569,0.466667,0.705882}%
\pgfsetfillcolor{currentfill}%
\pgfsetfillopacity{0.962019}%
\pgfsetlinewidth{1.003750pt}%
\definecolor{currentstroke}{rgb}{0.121569,0.466667,0.705882}%
\pgfsetstrokecolor{currentstroke}%
\pgfsetstrokeopacity{0.962019}%
\pgfsetdash{}{0pt}%
\pgfpathmoveto{\pgfqpoint{2.045275in}{1.546002in}}%
\pgfpathcurveto{\pgfqpoint{2.053511in}{1.546002in}}{\pgfqpoint{2.061411in}{1.549274in}}{\pgfqpoint{2.067235in}{1.555098in}}%
\pgfpathcurveto{\pgfqpoint{2.073059in}{1.560922in}}{\pgfqpoint{2.076331in}{1.568822in}}{\pgfqpoint{2.076331in}{1.577058in}}%
\pgfpathcurveto{\pgfqpoint{2.076331in}{1.585294in}}{\pgfqpoint{2.073059in}{1.593194in}}{\pgfqpoint{2.067235in}{1.599018in}}%
\pgfpathcurveto{\pgfqpoint{2.061411in}{1.604842in}}{\pgfqpoint{2.053511in}{1.608115in}}{\pgfqpoint{2.045275in}{1.608115in}}%
\pgfpathcurveto{\pgfqpoint{2.037038in}{1.608115in}}{\pgfqpoint{2.029138in}{1.604842in}}{\pgfqpoint{2.023314in}{1.599018in}}%
\pgfpathcurveto{\pgfqpoint{2.017490in}{1.593194in}}{\pgfqpoint{2.014218in}{1.585294in}}{\pgfqpoint{2.014218in}{1.577058in}}%
\pgfpathcurveto{\pgfqpoint{2.014218in}{1.568822in}}{\pgfqpoint{2.017490in}{1.560922in}}{\pgfqpoint{2.023314in}{1.555098in}}%
\pgfpathcurveto{\pgfqpoint{2.029138in}{1.549274in}}{\pgfqpoint{2.037038in}{1.546002in}}{\pgfqpoint{2.045275in}{1.546002in}}%
\pgfpathclose%
\pgfusepath{stroke,fill}%
\end{pgfscope}%
\begin{pgfscope}%
\pgfpathrectangle{\pgfqpoint{0.100000in}{0.212622in}}{\pgfqpoint{3.696000in}{3.696000in}}%
\pgfusepath{clip}%
\pgfsetbuttcap%
\pgfsetroundjoin%
\definecolor{currentfill}{rgb}{0.121569,0.466667,0.705882}%
\pgfsetfillcolor{currentfill}%
\pgfsetfillopacity{0.963072}%
\pgfsetlinewidth{1.003750pt}%
\definecolor{currentstroke}{rgb}{0.121569,0.466667,0.705882}%
\pgfsetstrokecolor{currentstroke}%
\pgfsetstrokeopacity{0.963072}%
\pgfsetdash{}{0pt}%
\pgfpathmoveto{\pgfqpoint{2.053984in}{1.541303in}}%
\pgfpathcurveto{\pgfqpoint{2.062220in}{1.541303in}}{\pgfqpoint{2.070121in}{1.544575in}}{\pgfqpoint{2.075944in}{1.550399in}}%
\pgfpathcurveto{\pgfqpoint{2.081768in}{1.556223in}}{\pgfqpoint{2.085041in}{1.564123in}}{\pgfqpoint{2.085041in}{1.572360in}}%
\pgfpathcurveto{\pgfqpoint{2.085041in}{1.580596in}}{\pgfqpoint{2.081768in}{1.588496in}}{\pgfqpoint{2.075944in}{1.594320in}}%
\pgfpathcurveto{\pgfqpoint{2.070121in}{1.600144in}}{\pgfqpoint{2.062220in}{1.603416in}}{\pgfqpoint{2.053984in}{1.603416in}}%
\pgfpathcurveto{\pgfqpoint{2.045748in}{1.603416in}}{\pgfqpoint{2.037848in}{1.600144in}}{\pgfqpoint{2.032024in}{1.594320in}}%
\pgfpathcurveto{\pgfqpoint{2.026200in}{1.588496in}}{\pgfqpoint{2.022928in}{1.580596in}}{\pgfqpoint{2.022928in}{1.572360in}}%
\pgfpathcurveto{\pgfqpoint{2.022928in}{1.564123in}}{\pgfqpoint{2.026200in}{1.556223in}}{\pgfqpoint{2.032024in}{1.550399in}}%
\pgfpathcurveto{\pgfqpoint{2.037848in}{1.544575in}}{\pgfqpoint{2.045748in}{1.541303in}}{\pgfqpoint{2.053984in}{1.541303in}}%
\pgfpathclose%
\pgfusepath{stroke,fill}%
\end{pgfscope}%
\begin{pgfscope}%
\pgfpathrectangle{\pgfqpoint{0.100000in}{0.212622in}}{\pgfqpoint{3.696000in}{3.696000in}}%
\pgfusepath{clip}%
\pgfsetbuttcap%
\pgfsetroundjoin%
\definecolor{currentfill}{rgb}{0.121569,0.466667,0.705882}%
\pgfsetfillcolor{currentfill}%
\pgfsetfillopacity{0.963535}%
\pgfsetlinewidth{1.003750pt}%
\definecolor{currentstroke}{rgb}{0.121569,0.466667,0.705882}%
\pgfsetstrokecolor{currentstroke}%
\pgfsetstrokeopacity{0.963535}%
\pgfsetdash{}{0pt}%
\pgfpathmoveto{\pgfqpoint{2.060408in}{1.535658in}}%
\pgfpathcurveto{\pgfqpoint{2.068644in}{1.535658in}}{\pgfqpoint{2.076545in}{1.538930in}}{\pgfqpoint{2.082368in}{1.544754in}}%
\pgfpathcurveto{\pgfqpoint{2.088192in}{1.550578in}}{\pgfqpoint{2.091465in}{1.558478in}}{\pgfqpoint{2.091465in}{1.566714in}}%
\pgfpathcurveto{\pgfqpoint{2.091465in}{1.574950in}}{\pgfqpoint{2.088192in}{1.582850in}}{\pgfqpoint{2.082368in}{1.588674in}}%
\pgfpathcurveto{\pgfqpoint{2.076545in}{1.594498in}}{\pgfqpoint{2.068644in}{1.597771in}}{\pgfqpoint{2.060408in}{1.597771in}}%
\pgfpathcurveto{\pgfqpoint{2.052172in}{1.597771in}}{\pgfqpoint{2.044272in}{1.594498in}}{\pgfqpoint{2.038448in}{1.588674in}}%
\pgfpathcurveto{\pgfqpoint{2.032624in}{1.582850in}}{\pgfqpoint{2.029352in}{1.574950in}}{\pgfqpoint{2.029352in}{1.566714in}}%
\pgfpathcurveto{\pgfqpoint{2.029352in}{1.558478in}}{\pgfqpoint{2.032624in}{1.550578in}}{\pgfqpoint{2.038448in}{1.544754in}}%
\pgfpathcurveto{\pgfqpoint{2.044272in}{1.538930in}}{\pgfqpoint{2.052172in}{1.535658in}}{\pgfqpoint{2.060408in}{1.535658in}}%
\pgfpathclose%
\pgfusepath{stroke,fill}%
\end{pgfscope}%
\begin{pgfscope}%
\pgfpathrectangle{\pgfqpoint{0.100000in}{0.212622in}}{\pgfqpoint{3.696000in}{3.696000in}}%
\pgfusepath{clip}%
\pgfsetbuttcap%
\pgfsetroundjoin%
\definecolor{currentfill}{rgb}{0.121569,0.466667,0.705882}%
\pgfsetfillcolor{currentfill}%
\pgfsetfillopacity{0.963784}%
\pgfsetlinewidth{1.003750pt}%
\definecolor{currentstroke}{rgb}{0.121569,0.466667,0.705882}%
\pgfsetstrokecolor{currentstroke}%
\pgfsetstrokeopacity{0.963784}%
\pgfsetdash{}{0pt}%
\pgfpathmoveto{\pgfqpoint{2.388194in}{1.348404in}}%
\pgfpathcurveto{\pgfqpoint{2.396430in}{1.348404in}}{\pgfqpoint{2.404330in}{1.351677in}}{\pgfqpoint{2.410154in}{1.357501in}}%
\pgfpathcurveto{\pgfqpoint{2.415978in}{1.363325in}}{\pgfqpoint{2.419250in}{1.371225in}}{\pgfqpoint{2.419250in}{1.379461in}}%
\pgfpathcurveto{\pgfqpoint{2.419250in}{1.387697in}}{\pgfqpoint{2.415978in}{1.395597in}}{\pgfqpoint{2.410154in}{1.401421in}}%
\pgfpathcurveto{\pgfqpoint{2.404330in}{1.407245in}}{\pgfqpoint{2.396430in}{1.410517in}}{\pgfqpoint{2.388194in}{1.410517in}}%
\pgfpathcurveto{\pgfqpoint{2.379957in}{1.410517in}}{\pgfqpoint{2.372057in}{1.407245in}}{\pgfqpoint{2.366233in}{1.401421in}}%
\pgfpathcurveto{\pgfqpoint{2.360410in}{1.395597in}}{\pgfqpoint{2.357137in}{1.387697in}}{\pgfqpoint{2.357137in}{1.379461in}}%
\pgfpathcurveto{\pgfqpoint{2.357137in}{1.371225in}}{\pgfqpoint{2.360410in}{1.363325in}}{\pgfqpoint{2.366233in}{1.357501in}}%
\pgfpathcurveto{\pgfqpoint{2.372057in}{1.351677in}}{\pgfqpoint{2.379957in}{1.348404in}}{\pgfqpoint{2.388194in}{1.348404in}}%
\pgfpathclose%
\pgfusepath{stroke,fill}%
\end{pgfscope}%
\begin{pgfscope}%
\pgfpathrectangle{\pgfqpoint{0.100000in}{0.212622in}}{\pgfqpoint{3.696000in}{3.696000in}}%
\pgfusepath{clip}%
\pgfsetbuttcap%
\pgfsetroundjoin%
\definecolor{currentfill}{rgb}{0.121569,0.466667,0.705882}%
\pgfsetfillcolor{currentfill}%
\pgfsetfillopacity{0.964666}%
\pgfsetlinewidth{1.003750pt}%
\definecolor{currentstroke}{rgb}{0.121569,0.466667,0.705882}%
\pgfsetstrokecolor{currentstroke}%
\pgfsetstrokeopacity{0.964666}%
\pgfsetdash{}{0pt}%
\pgfpathmoveto{\pgfqpoint{2.072460in}{1.528342in}}%
\pgfpathcurveto{\pgfqpoint{2.080697in}{1.528342in}}{\pgfqpoint{2.088597in}{1.531614in}}{\pgfqpoint{2.094420in}{1.537438in}}%
\pgfpathcurveto{\pgfqpoint{2.100244in}{1.543262in}}{\pgfqpoint{2.103517in}{1.551162in}}{\pgfqpoint{2.103517in}{1.559399in}}%
\pgfpathcurveto{\pgfqpoint{2.103517in}{1.567635in}}{\pgfqpoint{2.100244in}{1.575535in}}{\pgfqpoint{2.094420in}{1.581359in}}%
\pgfpathcurveto{\pgfqpoint{2.088597in}{1.587183in}}{\pgfqpoint{2.080697in}{1.590455in}}{\pgfqpoint{2.072460in}{1.590455in}}%
\pgfpathcurveto{\pgfqpoint{2.064224in}{1.590455in}}{\pgfqpoint{2.056324in}{1.587183in}}{\pgfqpoint{2.050500in}{1.581359in}}%
\pgfpathcurveto{\pgfqpoint{2.044676in}{1.575535in}}{\pgfqpoint{2.041404in}{1.567635in}}{\pgfqpoint{2.041404in}{1.559399in}}%
\pgfpathcurveto{\pgfqpoint{2.041404in}{1.551162in}}{\pgfqpoint{2.044676in}{1.543262in}}{\pgfqpoint{2.050500in}{1.537438in}}%
\pgfpathcurveto{\pgfqpoint{2.056324in}{1.531614in}}{\pgfqpoint{2.064224in}{1.528342in}}{\pgfqpoint{2.072460in}{1.528342in}}%
\pgfpathclose%
\pgfusepath{stroke,fill}%
\end{pgfscope}%
\begin{pgfscope}%
\pgfpathrectangle{\pgfqpoint{0.100000in}{0.212622in}}{\pgfqpoint{3.696000in}{3.696000in}}%
\pgfusepath{clip}%
\pgfsetbuttcap%
\pgfsetroundjoin%
\definecolor{currentfill}{rgb}{0.121569,0.466667,0.705882}%
\pgfsetfillcolor{currentfill}%
\pgfsetfillopacity{0.965659}%
\pgfsetlinewidth{1.003750pt}%
\definecolor{currentstroke}{rgb}{0.121569,0.466667,0.705882}%
\pgfsetstrokecolor{currentstroke}%
\pgfsetstrokeopacity{0.965659}%
\pgfsetdash{}{0pt}%
\pgfpathmoveto{\pgfqpoint{2.080911in}{1.525520in}}%
\pgfpathcurveto{\pgfqpoint{2.089147in}{1.525520in}}{\pgfqpoint{2.097047in}{1.528792in}}{\pgfqpoint{2.102871in}{1.534616in}}%
\pgfpathcurveto{\pgfqpoint{2.108695in}{1.540440in}}{\pgfqpoint{2.111967in}{1.548340in}}{\pgfqpoint{2.111967in}{1.556576in}}%
\pgfpathcurveto{\pgfqpoint{2.111967in}{1.564813in}}{\pgfqpoint{2.108695in}{1.572713in}}{\pgfqpoint{2.102871in}{1.578537in}}%
\pgfpathcurveto{\pgfqpoint{2.097047in}{1.584361in}}{\pgfqpoint{2.089147in}{1.587633in}}{\pgfqpoint{2.080911in}{1.587633in}}%
\pgfpathcurveto{\pgfqpoint{2.072674in}{1.587633in}}{\pgfqpoint{2.064774in}{1.584361in}}{\pgfqpoint{2.058950in}{1.578537in}}%
\pgfpathcurveto{\pgfqpoint{2.053127in}{1.572713in}}{\pgfqpoint{2.049854in}{1.564813in}}{\pgfqpoint{2.049854in}{1.556576in}}%
\pgfpathcurveto{\pgfqpoint{2.049854in}{1.548340in}}{\pgfqpoint{2.053127in}{1.540440in}}{\pgfqpoint{2.058950in}{1.534616in}}%
\pgfpathcurveto{\pgfqpoint{2.064774in}{1.528792in}}{\pgfqpoint{2.072674in}{1.525520in}}{\pgfqpoint{2.080911in}{1.525520in}}%
\pgfpathclose%
\pgfusepath{stroke,fill}%
\end{pgfscope}%
\begin{pgfscope}%
\pgfpathrectangle{\pgfqpoint{0.100000in}{0.212622in}}{\pgfqpoint{3.696000in}{3.696000in}}%
\pgfusepath{clip}%
\pgfsetbuttcap%
\pgfsetroundjoin%
\definecolor{currentfill}{rgb}{0.121569,0.466667,0.705882}%
\pgfsetfillcolor{currentfill}%
\pgfsetfillopacity{0.966874}%
\pgfsetlinewidth{1.003750pt}%
\definecolor{currentstroke}{rgb}{0.121569,0.466667,0.705882}%
\pgfsetstrokecolor{currentstroke}%
\pgfsetstrokeopacity{0.966874}%
\pgfsetdash{}{0pt}%
\pgfpathmoveto{\pgfqpoint{2.095085in}{1.512879in}}%
\pgfpathcurveto{\pgfqpoint{2.103321in}{1.512879in}}{\pgfqpoint{2.111221in}{1.516151in}}{\pgfqpoint{2.117045in}{1.521975in}}%
\pgfpathcurveto{\pgfqpoint{2.122869in}{1.527799in}}{\pgfqpoint{2.126142in}{1.535699in}}{\pgfqpoint{2.126142in}{1.543936in}}%
\pgfpathcurveto{\pgfqpoint{2.126142in}{1.552172in}}{\pgfqpoint{2.122869in}{1.560072in}}{\pgfqpoint{2.117045in}{1.565896in}}%
\pgfpathcurveto{\pgfqpoint{2.111221in}{1.571720in}}{\pgfqpoint{2.103321in}{1.574992in}}{\pgfqpoint{2.095085in}{1.574992in}}%
\pgfpathcurveto{\pgfqpoint{2.086849in}{1.574992in}}{\pgfqpoint{2.078949in}{1.571720in}}{\pgfqpoint{2.073125in}{1.565896in}}%
\pgfpathcurveto{\pgfqpoint{2.067301in}{1.560072in}}{\pgfqpoint{2.064029in}{1.552172in}}{\pgfqpoint{2.064029in}{1.543936in}}%
\pgfpathcurveto{\pgfqpoint{2.064029in}{1.535699in}}{\pgfqpoint{2.067301in}{1.527799in}}{\pgfqpoint{2.073125in}{1.521975in}}%
\pgfpathcurveto{\pgfqpoint{2.078949in}{1.516151in}}{\pgfqpoint{2.086849in}{1.512879in}}{\pgfqpoint{2.095085in}{1.512879in}}%
\pgfpathclose%
\pgfusepath{stroke,fill}%
\end{pgfscope}%
\begin{pgfscope}%
\pgfpathrectangle{\pgfqpoint{0.100000in}{0.212622in}}{\pgfqpoint{3.696000in}{3.696000in}}%
\pgfusepath{clip}%
\pgfsetbuttcap%
\pgfsetroundjoin%
\definecolor{currentfill}{rgb}{0.121569,0.466667,0.705882}%
\pgfsetfillcolor{currentfill}%
\pgfsetfillopacity{0.967235}%
\pgfsetlinewidth{1.003750pt}%
\definecolor{currentstroke}{rgb}{0.121569,0.466667,0.705882}%
\pgfsetstrokecolor{currentstroke}%
\pgfsetstrokeopacity{0.967235}%
\pgfsetdash{}{0pt}%
\pgfpathmoveto{\pgfqpoint{2.391498in}{1.342824in}}%
\pgfpathcurveto{\pgfqpoint{2.399735in}{1.342824in}}{\pgfqpoint{2.407635in}{1.346096in}}{\pgfqpoint{2.413459in}{1.351920in}}%
\pgfpathcurveto{\pgfqpoint{2.419283in}{1.357744in}}{\pgfqpoint{2.422555in}{1.365644in}}{\pgfqpoint{2.422555in}{1.373881in}}%
\pgfpathcurveto{\pgfqpoint{2.422555in}{1.382117in}}{\pgfqpoint{2.419283in}{1.390017in}}{\pgfqpoint{2.413459in}{1.395841in}}%
\pgfpathcurveto{\pgfqpoint{2.407635in}{1.401665in}}{\pgfqpoint{2.399735in}{1.404937in}}{\pgfqpoint{2.391498in}{1.404937in}}%
\pgfpathcurveto{\pgfqpoint{2.383262in}{1.404937in}}{\pgfqpoint{2.375362in}{1.401665in}}{\pgfqpoint{2.369538in}{1.395841in}}%
\pgfpathcurveto{\pgfqpoint{2.363714in}{1.390017in}}{\pgfqpoint{2.360442in}{1.382117in}}{\pgfqpoint{2.360442in}{1.373881in}}%
\pgfpathcurveto{\pgfqpoint{2.360442in}{1.365644in}}{\pgfqpoint{2.363714in}{1.357744in}}{\pgfqpoint{2.369538in}{1.351920in}}%
\pgfpathcurveto{\pgfqpoint{2.375362in}{1.346096in}}{\pgfqpoint{2.383262in}{1.342824in}}{\pgfqpoint{2.391498in}{1.342824in}}%
\pgfpathclose%
\pgfusepath{stroke,fill}%
\end{pgfscope}%
\begin{pgfscope}%
\pgfpathrectangle{\pgfqpoint{0.100000in}{0.212622in}}{\pgfqpoint{3.696000in}{3.696000in}}%
\pgfusepath{clip}%
\pgfsetbuttcap%
\pgfsetroundjoin%
\definecolor{currentfill}{rgb}{0.121569,0.466667,0.705882}%
\pgfsetfillcolor{currentfill}%
\pgfsetfillopacity{0.968408}%
\pgfsetlinewidth{1.003750pt}%
\definecolor{currentstroke}{rgb}{0.121569,0.466667,0.705882}%
\pgfsetstrokecolor{currentstroke}%
\pgfsetstrokeopacity{0.968408}%
\pgfsetdash{}{0pt}%
\pgfpathmoveto{\pgfqpoint{2.106788in}{1.506559in}}%
\pgfpathcurveto{\pgfqpoint{2.115024in}{1.506559in}}{\pgfqpoint{2.122924in}{1.509831in}}{\pgfqpoint{2.128748in}{1.515655in}}%
\pgfpathcurveto{\pgfqpoint{2.134572in}{1.521479in}}{\pgfqpoint{2.137845in}{1.529379in}}{\pgfqpoint{2.137845in}{1.537615in}}%
\pgfpathcurveto{\pgfqpoint{2.137845in}{1.545851in}}{\pgfqpoint{2.134572in}{1.553751in}}{\pgfqpoint{2.128748in}{1.559575in}}%
\pgfpathcurveto{\pgfqpoint{2.122924in}{1.565399in}}{\pgfqpoint{2.115024in}{1.568672in}}{\pgfqpoint{2.106788in}{1.568672in}}%
\pgfpathcurveto{\pgfqpoint{2.098552in}{1.568672in}}{\pgfqpoint{2.090652in}{1.565399in}}{\pgfqpoint{2.084828in}{1.559575in}}%
\pgfpathcurveto{\pgfqpoint{2.079004in}{1.553751in}}{\pgfqpoint{2.075732in}{1.545851in}}{\pgfqpoint{2.075732in}{1.537615in}}%
\pgfpathcurveto{\pgfqpoint{2.075732in}{1.529379in}}{\pgfqpoint{2.079004in}{1.521479in}}{\pgfqpoint{2.084828in}{1.515655in}}%
\pgfpathcurveto{\pgfqpoint{2.090652in}{1.509831in}}{\pgfqpoint{2.098552in}{1.506559in}}{\pgfqpoint{2.106788in}{1.506559in}}%
\pgfpathclose%
\pgfusepath{stroke,fill}%
\end{pgfscope}%
\begin{pgfscope}%
\pgfpathrectangle{\pgfqpoint{0.100000in}{0.212622in}}{\pgfqpoint{3.696000in}{3.696000in}}%
\pgfusepath{clip}%
\pgfsetbuttcap%
\pgfsetroundjoin%
\definecolor{currentfill}{rgb}{0.121569,0.466667,0.705882}%
\pgfsetfillcolor{currentfill}%
\pgfsetfillopacity{0.969598}%
\pgfsetlinewidth{1.003750pt}%
\definecolor{currentstroke}{rgb}{0.121569,0.466667,0.705882}%
\pgfsetstrokecolor{currentstroke}%
\pgfsetstrokeopacity{0.969598}%
\pgfsetdash{}{0pt}%
\pgfpathmoveto{\pgfqpoint{2.127992in}{1.485167in}}%
\pgfpathcurveto{\pgfqpoint{2.136228in}{1.485167in}}{\pgfqpoint{2.144128in}{1.488440in}}{\pgfqpoint{2.149952in}{1.494263in}}%
\pgfpathcurveto{\pgfqpoint{2.155776in}{1.500087in}}{\pgfqpoint{2.159048in}{1.507987in}}{\pgfqpoint{2.159048in}{1.516224in}}%
\pgfpathcurveto{\pgfqpoint{2.159048in}{1.524460in}}{\pgfqpoint{2.155776in}{1.532360in}}{\pgfqpoint{2.149952in}{1.538184in}}%
\pgfpathcurveto{\pgfqpoint{2.144128in}{1.544008in}}{\pgfqpoint{2.136228in}{1.547280in}}{\pgfqpoint{2.127992in}{1.547280in}}%
\pgfpathcurveto{\pgfqpoint{2.119755in}{1.547280in}}{\pgfqpoint{2.111855in}{1.544008in}}{\pgfqpoint{2.106031in}{1.538184in}}%
\pgfpathcurveto{\pgfqpoint{2.100207in}{1.532360in}}{\pgfqpoint{2.096935in}{1.524460in}}{\pgfqpoint{2.096935in}{1.516224in}}%
\pgfpathcurveto{\pgfqpoint{2.096935in}{1.507987in}}{\pgfqpoint{2.100207in}{1.500087in}}{\pgfqpoint{2.106031in}{1.494263in}}%
\pgfpathcurveto{\pgfqpoint{2.111855in}{1.488440in}}{\pgfqpoint{2.119755in}{1.485167in}}{\pgfqpoint{2.127992in}{1.485167in}}%
\pgfpathclose%
\pgfusepath{stroke,fill}%
\end{pgfscope}%
\begin{pgfscope}%
\pgfpathrectangle{\pgfqpoint{0.100000in}{0.212622in}}{\pgfqpoint{3.696000in}{3.696000in}}%
\pgfusepath{clip}%
\pgfsetbuttcap%
\pgfsetroundjoin%
\definecolor{currentfill}{rgb}{0.121569,0.466667,0.705882}%
\pgfsetfillcolor{currentfill}%
\pgfsetfillopacity{0.971384}%
\pgfsetlinewidth{1.003750pt}%
\definecolor{currentstroke}{rgb}{0.121569,0.466667,0.705882}%
\pgfsetstrokecolor{currentstroke}%
\pgfsetstrokeopacity{0.971384}%
\pgfsetdash{}{0pt}%
\pgfpathmoveto{\pgfqpoint{2.145904in}{1.472783in}}%
\pgfpathcurveto{\pgfqpoint{2.154141in}{1.472783in}}{\pgfqpoint{2.162041in}{1.476056in}}{\pgfqpoint{2.167865in}{1.481880in}}%
\pgfpathcurveto{\pgfqpoint{2.173689in}{1.487703in}}{\pgfqpoint{2.176961in}{1.495604in}}{\pgfqpoint{2.176961in}{1.503840in}}%
\pgfpathcurveto{\pgfqpoint{2.176961in}{1.512076in}}{\pgfqpoint{2.173689in}{1.519976in}}{\pgfqpoint{2.167865in}{1.525800in}}%
\pgfpathcurveto{\pgfqpoint{2.162041in}{1.531624in}}{\pgfqpoint{2.154141in}{1.534896in}}{\pgfqpoint{2.145904in}{1.534896in}}%
\pgfpathcurveto{\pgfqpoint{2.137668in}{1.534896in}}{\pgfqpoint{2.129768in}{1.531624in}}{\pgfqpoint{2.123944in}{1.525800in}}%
\pgfpathcurveto{\pgfqpoint{2.118120in}{1.519976in}}{\pgfqpoint{2.114848in}{1.512076in}}{\pgfqpoint{2.114848in}{1.503840in}}%
\pgfpathcurveto{\pgfqpoint{2.114848in}{1.495604in}}{\pgfqpoint{2.118120in}{1.487703in}}{\pgfqpoint{2.123944in}{1.481880in}}%
\pgfpathcurveto{\pgfqpoint{2.129768in}{1.476056in}}{\pgfqpoint{2.137668in}{1.472783in}}{\pgfqpoint{2.145904in}{1.472783in}}%
\pgfpathclose%
\pgfusepath{stroke,fill}%
\end{pgfscope}%
\begin{pgfscope}%
\pgfpathrectangle{\pgfqpoint{0.100000in}{0.212622in}}{\pgfqpoint{3.696000in}{3.696000in}}%
\pgfusepath{clip}%
\pgfsetbuttcap%
\pgfsetroundjoin%
\definecolor{currentfill}{rgb}{0.121569,0.466667,0.705882}%
\pgfsetfillcolor{currentfill}%
\pgfsetfillopacity{0.971521}%
\pgfsetlinewidth{1.003750pt}%
\definecolor{currentstroke}{rgb}{0.121569,0.466667,0.705882}%
\pgfsetstrokecolor{currentstroke}%
\pgfsetstrokeopacity{0.971521}%
\pgfsetdash{}{0pt}%
\pgfpathmoveto{\pgfqpoint{2.394064in}{1.337837in}}%
\pgfpathcurveto{\pgfqpoint{2.402300in}{1.337837in}}{\pgfqpoint{2.410200in}{1.341109in}}{\pgfqpoint{2.416024in}{1.346933in}}%
\pgfpathcurveto{\pgfqpoint{2.421848in}{1.352757in}}{\pgfqpoint{2.425120in}{1.360657in}}{\pgfqpoint{2.425120in}{1.368894in}}%
\pgfpathcurveto{\pgfqpoint{2.425120in}{1.377130in}}{\pgfqpoint{2.421848in}{1.385030in}}{\pgfqpoint{2.416024in}{1.390854in}}%
\pgfpathcurveto{\pgfqpoint{2.410200in}{1.396678in}}{\pgfqpoint{2.402300in}{1.399950in}}{\pgfqpoint{2.394064in}{1.399950in}}%
\pgfpathcurveto{\pgfqpoint{2.385827in}{1.399950in}}{\pgfqpoint{2.377927in}{1.396678in}}{\pgfqpoint{2.372103in}{1.390854in}}%
\pgfpathcurveto{\pgfqpoint{2.366279in}{1.385030in}}{\pgfqpoint{2.363007in}{1.377130in}}{\pgfqpoint{2.363007in}{1.368894in}}%
\pgfpathcurveto{\pgfqpoint{2.363007in}{1.360657in}}{\pgfqpoint{2.366279in}{1.352757in}}{\pgfqpoint{2.372103in}{1.346933in}}%
\pgfpathcurveto{\pgfqpoint{2.377927in}{1.341109in}}{\pgfqpoint{2.385827in}{1.337837in}}{\pgfqpoint{2.394064in}{1.337837in}}%
\pgfpathclose%
\pgfusepath{stroke,fill}%
\end{pgfscope}%
\begin{pgfscope}%
\pgfpathrectangle{\pgfqpoint{0.100000in}{0.212622in}}{\pgfqpoint{3.696000in}{3.696000in}}%
\pgfusepath{clip}%
\pgfsetbuttcap%
\pgfsetroundjoin%
\definecolor{currentfill}{rgb}{0.121569,0.466667,0.705882}%
\pgfsetfillcolor{currentfill}%
\pgfsetfillopacity{0.973347}%
\pgfsetlinewidth{1.003750pt}%
\definecolor{currentstroke}{rgb}{0.121569,0.466667,0.705882}%
\pgfsetstrokecolor{currentstroke}%
\pgfsetstrokeopacity{0.973347}%
\pgfsetdash{}{0pt}%
\pgfpathmoveto{\pgfqpoint{2.162068in}{1.458326in}}%
\pgfpathcurveto{\pgfqpoint{2.170304in}{1.458326in}}{\pgfqpoint{2.178204in}{1.461599in}}{\pgfqpoint{2.184028in}{1.467422in}}%
\pgfpathcurveto{\pgfqpoint{2.189852in}{1.473246in}}{\pgfqpoint{2.193124in}{1.481146in}}{\pgfqpoint{2.193124in}{1.489383in}}%
\pgfpathcurveto{\pgfqpoint{2.193124in}{1.497619in}}{\pgfqpoint{2.189852in}{1.505519in}}{\pgfqpoint{2.184028in}{1.511343in}}%
\pgfpathcurveto{\pgfqpoint{2.178204in}{1.517167in}}{\pgfqpoint{2.170304in}{1.520439in}}{\pgfqpoint{2.162068in}{1.520439in}}%
\pgfpathcurveto{\pgfqpoint{2.153831in}{1.520439in}}{\pgfqpoint{2.145931in}{1.517167in}}{\pgfqpoint{2.140107in}{1.511343in}}%
\pgfpathcurveto{\pgfqpoint{2.134283in}{1.505519in}}{\pgfqpoint{2.131011in}{1.497619in}}{\pgfqpoint{2.131011in}{1.489383in}}%
\pgfpathcurveto{\pgfqpoint{2.131011in}{1.481146in}}{\pgfqpoint{2.134283in}{1.473246in}}{\pgfqpoint{2.140107in}{1.467422in}}%
\pgfpathcurveto{\pgfqpoint{2.145931in}{1.461599in}}{\pgfqpoint{2.153831in}{1.458326in}}{\pgfqpoint{2.162068in}{1.458326in}}%
\pgfpathclose%
\pgfusepath{stroke,fill}%
\end{pgfscope}%
\begin{pgfscope}%
\pgfpathrectangle{\pgfqpoint{0.100000in}{0.212622in}}{\pgfqpoint{3.696000in}{3.696000in}}%
\pgfusepath{clip}%
\pgfsetbuttcap%
\pgfsetroundjoin%
\definecolor{currentfill}{rgb}{0.121569,0.466667,0.705882}%
\pgfsetfillcolor{currentfill}%
\pgfsetfillopacity{0.974915}%
\pgfsetlinewidth{1.003750pt}%
\definecolor{currentstroke}{rgb}{0.121569,0.466667,0.705882}%
\pgfsetstrokecolor{currentstroke}%
\pgfsetstrokeopacity{0.974915}%
\pgfsetdash{}{0pt}%
\pgfpathmoveto{\pgfqpoint{2.177430in}{1.447700in}}%
\pgfpathcurveto{\pgfqpoint{2.185667in}{1.447700in}}{\pgfqpoint{2.193567in}{1.450973in}}{\pgfqpoint{2.199391in}{1.456797in}}%
\pgfpathcurveto{\pgfqpoint{2.205214in}{1.462621in}}{\pgfqpoint{2.208487in}{1.470521in}}{\pgfqpoint{2.208487in}{1.478757in}}%
\pgfpathcurveto{\pgfqpoint{2.208487in}{1.486993in}}{\pgfqpoint{2.205214in}{1.494893in}}{\pgfqpoint{2.199391in}{1.500717in}}%
\pgfpathcurveto{\pgfqpoint{2.193567in}{1.506541in}}{\pgfqpoint{2.185667in}{1.509813in}}{\pgfqpoint{2.177430in}{1.509813in}}%
\pgfpathcurveto{\pgfqpoint{2.169194in}{1.509813in}}{\pgfqpoint{2.161294in}{1.506541in}}{\pgfqpoint{2.155470in}{1.500717in}}%
\pgfpathcurveto{\pgfqpoint{2.149646in}{1.494893in}}{\pgfqpoint{2.146374in}{1.486993in}}{\pgfqpoint{2.146374in}{1.478757in}}%
\pgfpathcurveto{\pgfqpoint{2.146374in}{1.470521in}}{\pgfqpoint{2.149646in}{1.462621in}}{\pgfqpoint{2.155470in}{1.456797in}}%
\pgfpathcurveto{\pgfqpoint{2.161294in}{1.450973in}}{\pgfqpoint{2.169194in}{1.447700in}}{\pgfqpoint{2.177430in}{1.447700in}}%
\pgfpathclose%
\pgfusepath{stroke,fill}%
\end{pgfscope}%
\begin{pgfscope}%
\pgfpathrectangle{\pgfqpoint{0.100000in}{0.212622in}}{\pgfqpoint{3.696000in}{3.696000in}}%
\pgfusepath{clip}%
\pgfsetbuttcap%
\pgfsetroundjoin%
\definecolor{currentfill}{rgb}{0.121569,0.466667,0.705882}%
\pgfsetfillcolor{currentfill}%
\pgfsetfillopacity{0.975890}%
\pgfsetlinewidth{1.003750pt}%
\definecolor{currentstroke}{rgb}{0.121569,0.466667,0.705882}%
\pgfsetstrokecolor{currentstroke}%
\pgfsetstrokeopacity{0.975890}%
\pgfsetdash{}{0pt}%
\pgfpathmoveto{\pgfqpoint{2.396921in}{1.331294in}}%
\pgfpathcurveto{\pgfqpoint{2.405157in}{1.331294in}}{\pgfqpoint{2.413057in}{1.334566in}}{\pgfqpoint{2.418881in}{1.340390in}}%
\pgfpathcurveto{\pgfqpoint{2.424705in}{1.346214in}}{\pgfqpoint{2.427977in}{1.354114in}}{\pgfqpoint{2.427977in}{1.362350in}}%
\pgfpathcurveto{\pgfqpoint{2.427977in}{1.370587in}}{\pgfqpoint{2.424705in}{1.378487in}}{\pgfqpoint{2.418881in}{1.384311in}}%
\pgfpathcurveto{\pgfqpoint{2.413057in}{1.390135in}}{\pgfqpoint{2.405157in}{1.393407in}}{\pgfqpoint{2.396921in}{1.393407in}}%
\pgfpathcurveto{\pgfqpoint{2.388685in}{1.393407in}}{\pgfqpoint{2.380785in}{1.390135in}}{\pgfqpoint{2.374961in}{1.384311in}}%
\pgfpathcurveto{\pgfqpoint{2.369137in}{1.378487in}}{\pgfqpoint{2.365864in}{1.370587in}}{\pgfqpoint{2.365864in}{1.362350in}}%
\pgfpathcurveto{\pgfqpoint{2.365864in}{1.354114in}}{\pgfqpoint{2.369137in}{1.346214in}}{\pgfqpoint{2.374961in}{1.340390in}}%
\pgfpathcurveto{\pgfqpoint{2.380785in}{1.334566in}}{\pgfqpoint{2.388685in}{1.331294in}}{\pgfqpoint{2.396921in}{1.331294in}}%
\pgfpathclose%
\pgfusepath{stroke,fill}%
\end{pgfscope}%
\begin{pgfscope}%
\pgfpathrectangle{\pgfqpoint{0.100000in}{0.212622in}}{\pgfqpoint{3.696000in}{3.696000in}}%
\pgfusepath{clip}%
\pgfsetbuttcap%
\pgfsetroundjoin%
\definecolor{currentfill}{rgb}{0.121569,0.466667,0.705882}%
\pgfsetfillcolor{currentfill}%
\pgfsetfillopacity{0.977245}%
\pgfsetlinewidth{1.003750pt}%
\definecolor{currentstroke}{rgb}{0.121569,0.466667,0.705882}%
\pgfsetstrokecolor{currentstroke}%
\pgfsetstrokeopacity{0.977245}%
\pgfsetdash{}{0pt}%
\pgfpathmoveto{\pgfqpoint{2.191706in}{1.440636in}}%
\pgfpathcurveto{\pgfqpoint{2.199942in}{1.440636in}}{\pgfqpoint{2.207843in}{1.443909in}}{\pgfqpoint{2.213666in}{1.449733in}}%
\pgfpathcurveto{\pgfqpoint{2.219490in}{1.455557in}}{\pgfqpoint{2.222763in}{1.463457in}}{\pgfqpoint{2.222763in}{1.471693in}}%
\pgfpathcurveto{\pgfqpoint{2.222763in}{1.479929in}}{\pgfqpoint{2.219490in}{1.487829in}}{\pgfqpoint{2.213666in}{1.493653in}}%
\pgfpathcurveto{\pgfqpoint{2.207843in}{1.499477in}}{\pgfqpoint{2.199942in}{1.502749in}}{\pgfqpoint{2.191706in}{1.502749in}}%
\pgfpathcurveto{\pgfqpoint{2.183470in}{1.502749in}}{\pgfqpoint{2.175570in}{1.499477in}}{\pgfqpoint{2.169746in}{1.493653in}}%
\pgfpathcurveto{\pgfqpoint{2.163922in}{1.487829in}}{\pgfqpoint{2.160650in}{1.479929in}}{\pgfqpoint{2.160650in}{1.471693in}}%
\pgfpathcurveto{\pgfqpoint{2.160650in}{1.463457in}}{\pgfqpoint{2.163922in}{1.455557in}}{\pgfqpoint{2.169746in}{1.449733in}}%
\pgfpathcurveto{\pgfqpoint{2.175570in}{1.443909in}}{\pgfqpoint{2.183470in}{1.440636in}}{\pgfqpoint{2.191706in}{1.440636in}}%
\pgfpathclose%
\pgfusepath{stroke,fill}%
\end{pgfscope}%
\begin{pgfscope}%
\pgfpathrectangle{\pgfqpoint{0.100000in}{0.212622in}}{\pgfqpoint{3.696000in}{3.696000in}}%
\pgfusepath{clip}%
\pgfsetbuttcap%
\pgfsetroundjoin%
\definecolor{currentfill}{rgb}{0.121569,0.466667,0.705882}%
\pgfsetfillcolor{currentfill}%
\pgfsetfillopacity{0.978786}%
\pgfsetlinewidth{1.003750pt}%
\definecolor{currentstroke}{rgb}{0.121569,0.466667,0.705882}%
\pgfsetstrokecolor{currentstroke}%
\pgfsetstrokeopacity{0.978786}%
\pgfsetdash{}{0pt}%
\pgfpathmoveto{\pgfqpoint{2.203375in}{1.433094in}}%
\pgfpathcurveto{\pgfqpoint{2.211611in}{1.433094in}}{\pgfqpoint{2.219511in}{1.436366in}}{\pgfqpoint{2.225335in}{1.442190in}}%
\pgfpathcurveto{\pgfqpoint{2.231159in}{1.448014in}}{\pgfqpoint{2.234432in}{1.455914in}}{\pgfqpoint{2.234432in}{1.464151in}}%
\pgfpathcurveto{\pgfqpoint{2.234432in}{1.472387in}}{\pgfqpoint{2.231159in}{1.480287in}}{\pgfqpoint{2.225335in}{1.486111in}}%
\pgfpathcurveto{\pgfqpoint{2.219511in}{1.491935in}}{\pgfqpoint{2.211611in}{1.495207in}}{\pgfqpoint{2.203375in}{1.495207in}}%
\pgfpathcurveto{\pgfqpoint{2.195139in}{1.495207in}}{\pgfqpoint{2.187239in}{1.491935in}}{\pgfqpoint{2.181415in}{1.486111in}}%
\pgfpathcurveto{\pgfqpoint{2.175591in}{1.480287in}}{\pgfqpoint{2.172319in}{1.472387in}}{\pgfqpoint{2.172319in}{1.464151in}}%
\pgfpathcurveto{\pgfqpoint{2.172319in}{1.455914in}}{\pgfqpoint{2.175591in}{1.448014in}}{\pgfqpoint{2.181415in}{1.442190in}}%
\pgfpathcurveto{\pgfqpoint{2.187239in}{1.436366in}}{\pgfqpoint{2.195139in}{1.433094in}}{\pgfqpoint{2.203375in}{1.433094in}}%
\pgfpathclose%
\pgfusepath{stroke,fill}%
\end{pgfscope}%
\begin{pgfscope}%
\pgfpathrectangle{\pgfqpoint{0.100000in}{0.212622in}}{\pgfqpoint{3.696000in}{3.696000in}}%
\pgfusepath{clip}%
\pgfsetbuttcap%
\pgfsetroundjoin%
\definecolor{currentfill}{rgb}{0.121569,0.466667,0.705882}%
\pgfsetfillcolor{currentfill}%
\pgfsetfillopacity{0.980167}%
\pgfsetlinewidth{1.003750pt}%
\definecolor{currentstroke}{rgb}{0.121569,0.466667,0.705882}%
\pgfsetstrokecolor{currentstroke}%
\pgfsetstrokeopacity{0.980167}%
\pgfsetdash{}{0pt}%
\pgfpathmoveto{\pgfqpoint{2.214061in}{1.429783in}}%
\pgfpathcurveto{\pgfqpoint{2.222298in}{1.429783in}}{\pgfqpoint{2.230198in}{1.433055in}}{\pgfqpoint{2.236022in}{1.438879in}}%
\pgfpathcurveto{\pgfqpoint{2.241846in}{1.444703in}}{\pgfqpoint{2.245118in}{1.452603in}}{\pgfqpoint{2.245118in}{1.460840in}}%
\pgfpathcurveto{\pgfqpoint{2.245118in}{1.469076in}}{\pgfqpoint{2.241846in}{1.476976in}}{\pgfqpoint{2.236022in}{1.482800in}}%
\pgfpathcurveto{\pgfqpoint{2.230198in}{1.488624in}}{\pgfqpoint{2.222298in}{1.491896in}}{\pgfqpoint{2.214061in}{1.491896in}}%
\pgfpathcurveto{\pgfqpoint{2.205825in}{1.491896in}}{\pgfqpoint{2.197925in}{1.488624in}}{\pgfqpoint{2.192101in}{1.482800in}}%
\pgfpathcurveto{\pgfqpoint{2.186277in}{1.476976in}}{\pgfqpoint{2.183005in}{1.469076in}}{\pgfqpoint{2.183005in}{1.460840in}}%
\pgfpathcurveto{\pgfqpoint{2.183005in}{1.452603in}}{\pgfqpoint{2.186277in}{1.444703in}}{\pgfqpoint{2.192101in}{1.438879in}}%
\pgfpathcurveto{\pgfqpoint{2.197925in}{1.433055in}}{\pgfqpoint{2.205825in}{1.429783in}}{\pgfqpoint{2.214061in}{1.429783in}}%
\pgfpathclose%
\pgfusepath{stroke,fill}%
\end{pgfscope}%
\begin{pgfscope}%
\pgfpathrectangle{\pgfqpoint{0.100000in}{0.212622in}}{\pgfqpoint{3.696000in}{3.696000in}}%
\pgfusepath{clip}%
\pgfsetbuttcap%
\pgfsetroundjoin%
\definecolor{currentfill}{rgb}{0.121569,0.466667,0.705882}%
\pgfsetfillcolor{currentfill}%
\pgfsetfillopacity{0.980893}%
\pgfsetlinewidth{1.003750pt}%
\definecolor{currentstroke}{rgb}{0.121569,0.466667,0.705882}%
\pgfsetstrokecolor{currentstroke}%
\pgfsetstrokeopacity{0.980893}%
\pgfsetdash{}{0pt}%
\pgfpathmoveto{\pgfqpoint{2.400941in}{1.327572in}}%
\pgfpathcurveto{\pgfqpoint{2.409177in}{1.327572in}}{\pgfqpoint{2.417077in}{1.330844in}}{\pgfqpoint{2.422901in}{1.336668in}}%
\pgfpathcurveto{\pgfqpoint{2.428725in}{1.342492in}}{\pgfqpoint{2.431998in}{1.350392in}}{\pgfqpoint{2.431998in}{1.358628in}}%
\pgfpathcurveto{\pgfqpoint{2.431998in}{1.366864in}}{\pgfqpoint{2.428725in}{1.374765in}}{\pgfqpoint{2.422901in}{1.380588in}}%
\pgfpathcurveto{\pgfqpoint{2.417077in}{1.386412in}}{\pgfqpoint{2.409177in}{1.389685in}}{\pgfqpoint{2.400941in}{1.389685in}}%
\pgfpathcurveto{\pgfqpoint{2.392705in}{1.389685in}}{\pgfqpoint{2.384805in}{1.386412in}}{\pgfqpoint{2.378981in}{1.380588in}}%
\pgfpathcurveto{\pgfqpoint{2.373157in}{1.374765in}}{\pgfqpoint{2.369885in}{1.366864in}}{\pgfqpoint{2.369885in}{1.358628in}}%
\pgfpathcurveto{\pgfqpoint{2.369885in}{1.350392in}}{\pgfqpoint{2.373157in}{1.342492in}}{\pgfqpoint{2.378981in}{1.336668in}}%
\pgfpathcurveto{\pgfqpoint{2.384805in}{1.330844in}}{\pgfqpoint{2.392705in}{1.327572in}}{\pgfqpoint{2.400941in}{1.327572in}}%
\pgfpathclose%
\pgfusepath{stroke,fill}%
\end{pgfscope}%
\begin{pgfscope}%
\pgfpathrectangle{\pgfqpoint{0.100000in}{0.212622in}}{\pgfqpoint{3.696000in}{3.696000in}}%
\pgfusepath{clip}%
\pgfsetbuttcap%
\pgfsetroundjoin%
\definecolor{currentfill}{rgb}{0.121569,0.466667,0.705882}%
\pgfsetfillcolor{currentfill}%
\pgfsetfillopacity{0.982081}%
\pgfsetlinewidth{1.003750pt}%
\definecolor{currentstroke}{rgb}{0.121569,0.466667,0.705882}%
\pgfsetstrokecolor{currentstroke}%
\pgfsetstrokeopacity{0.982081}%
\pgfsetdash{}{0pt}%
\pgfpathmoveto{\pgfqpoint{2.231249in}{1.413363in}}%
\pgfpathcurveto{\pgfqpoint{2.239485in}{1.413363in}}{\pgfqpoint{2.247385in}{1.416635in}}{\pgfqpoint{2.253209in}{1.422459in}}%
\pgfpathcurveto{\pgfqpoint{2.259033in}{1.428283in}}{\pgfqpoint{2.262305in}{1.436183in}}{\pgfqpoint{2.262305in}{1.444419in}}%
\pgfpathcurveto{\pgfqpoint{2.262305in}{1.452655in}}{\pgfqpoint{2.259033in}{1.460555in}}{\pgfqpoint{2.253209in}{1.466379in}}%
\pgfpathcurveto{\pgfqpoint{2.247385in}{1.472203in}}{\pgfqpoint{2.239485in}{1.475476in}}{\pgfqpoint{2.231249in}{1.475476in}}%
\pgfpathcurveto{\pgfqpoint{2.223013in}{1.475476in}}{\pgfqpoint{2.215113in}{1.472203in}}{\pgfqpoint{2.209289in}{1.466379in}}%
\pgfpathcurveto{\pgfqpoint{2.203465in}{1.460555in}}{\pgfqpoint{2.200192in}{1.452655in}}{\pgfqpoint{2.200192in}{1.444419in}}%
\pgfpathcurveto{\pgfqpoint{2.200192in}{1.436183in}}{\pgfqpoint{2.203465in}{1.428283in}}{\pgfqpoint{2.209289in}{1.422459in}}%
\pgfpathcurveto{\pgfqpoint{2.215113in}{1.416635in}}{\pgfqpoint{2.223013in}{1.413363in}}{\pgfqpoint{2.231249in}{1.413363in}}%
\pgfpathclose%
\pgfusepath{stroke,fill}%
\end{pgfscope}%
\begin{pgfscope}%
\pgfpathrectangle{\pgfqpoint{0.100000in}{0.212622in}}{\pgfqpoint{3.696000in}{3.696000in}}%
\pgfusepath{clip}%
\pgfsetbuttcap%
\pgfsetroundjoin%
\definecolor{currentfill}{rgb}{0.121569,0.466667,0.705882}%
\pgfsetfillcolor{currentfill}%
\pgfsetfillopacity{0.984701}%
\pgfsetlinewidth{1.003750pt}%
\definecolor{currentstroke}{rgb}{0.121569,0.466667,0.705882}%
\pgfsetstrokecolor{currentstroke}%
\pgfsetstrokeopacity{0.984701}%
\pgfsetdash{}{0pt}%
\pgfpathmoveto{\pgfqpoint{2.246719in}{1.401856in}}%
\pgfpathcurveto{\pgfqpoint{2.254955in}{1.401856in}}{\pgfqpoint{2.262855in}{1.405128in}}{\pgfqpoint{2.268679in}{1.410952in}}%
\pgfpathcurveto{\pgfqpoint{2.274503in}{1.416776in}}{\pgfqpoint{2.277776in}{1.424676in}}{\pgfqpoint{2.277776in}{1.432913in}}%
\pgfpathcurveto{\pgfqpoint{2.277776in}{1.441149in}}{\pgfqpoint{2.274503in}{1.449049in}}{\pgfqpoint{2.268679in}{1.454873in}}%
\pgfpathcurveto{\pgfqpoint{2.262855in}{1.460697in}}{\pgfqpoint{2.254955in}{1.463969in}}{\pgfqpoint{2.246719in}{1.463969in}}%
\pgfpathcurveto{\pgfqpoint{2.238483in}{1.463969in}}{\pgfqpoint{2.230583in}{1.460697in}}{\pgfqpoint{2.224759in}{1.454873in}}%
\pgfpathcurveto{\pgfqpoint{2.218935in}{1.449049in}}{\pgfqpoint{2.215663in}{1.441149in}}{\pgfqpoint{2.215663in}{1.432913in}}%
\pgfpathcurveto{\pgfqpoint{2.215663in}{1.424676in}}{\pgfqpoint{2.218935in}{1.416776in}}{\pgfqpoint{2.224759in}{1.410952in}}%
\pgfpathcurveto{\pgfqpoint{2.230583in}{1.405128in}}{\pgfqpoint{2.238483in}{1.401856in}}{\pgfqpoint{2.246719in}{1.401856in}}%
\pgfpathclose%
\pgfusepath{stroke,fill}%
\end{pgfscope}%
\begin{pgfscope}%
\pgfpathrectangle{\pgfqpoint{0.100000in}{0.212622in}}{\pgfqpoint{3.696000in}{3.696000in}}%
\pgfusepath{clip}%
\pgfsetbuttcap%
\pgfsetroundjoin%
\definecolor{currentfill}{rgb}{0.121569,0.466667,0.705882}%
\pgfsetfillcolor{currentfill}%
\pgfsetfillopacity{0.985182}%
\pgfsetlinewidth{1.003750pt}%
\definecolor{currentstroke}{rgb}{0.121569,0.466667,0.705882}%
\pgfsetstrokecolor{currentstroke}%
\pgfsetstrokeopacity{0.985182}%
\pgfsetdash{}{0pt}%
\pgfpathmoveto{\pgfqpoint{2.263921in}{1.387493in}}%
\pgfpathcurveto{\pgfqpoint{2.272157in}{1.387493in}}{\pgfqpoint{2.280057in}{1.390765in}}{\pgfqpoint{2.285881in}{1.396589in}}%
\pgfpathcurveto{\pgfqpoint{2.291705in}{1.402413in}}{\pgfqpoint{2.294978in}{1.410313in}}{\pgfqpoint{2.294978in}{1.418549in}}%
\pgfpathcurveto{\pgfqpoint{2.294978in}{1.426786in}}{\pgfqpoint{2.291705in}{1.434686in}}{\pgfqpoint{2.285881in}{1.440510in}}%
\pgfpathcurveto{\pgfqpoint{2.280057in}{1.446333in}}{\pgfqpoint{2.272157in}{1.449606in}}{\pgfqpoint{2.263921in}{1.449606in}}%
\pgfpathcurveto{\pgfqpoint{2.255685in}{1.449606in}}{\pgfqpoint{2.247785in}{1.446333in}}{\pgfqpoint{2.241961in}{1.440510in}}%
\pgfpathcurveto{\pgfqpoint{2.236137in}{1.434686in}}{\pgfqpoint{2.232865in}{1.426786in}}{\pgfqpoint{2.232865in}{1.418549in}}%
\pgfpathcurveto{\pgfqpoint{2.232865in}{1.410313in}}{\pgfqpoint{2.236137in}{1.402413in}}{\pgfqpoint{2.241961in}{1.396589in}}%
\pgfpathcurveto{\pgfqpoint{2.247785in}{1.390765in}}{\pgfqpoint{2.255685in}{1.387493in}}{\pgfqpoint{2.263921in}{1.387493in}}%
\pgfpathclose%
\pgfusepath{stroke,fill}%
\end{pgfscope}%
\begin{pgfscope}%
\pgfpathrectangle{\pgfqpoint{0.100000in}{0.212622in}}{\pgfqpoint{3.696000in}{3.696000in}}%
\pgfusepath{clip}%
\pgfsetbuttcap%
\pgfsetroundjoin%
\definecolor{currentfill}{rgb}{0.121569,0.466667,0.705882}%
\pgfsetfillcolor{currentfill}%
\pgfsetfillopacity{0.986030}%
\pgfsetlinewidth{1.003750pt}%
\definecolor{currentstroke}{rgb}{0.121569,0.466667,0.705882}%
\pgfsetstrokecolor{currentstroke}%
\pgfsetstrokeopacity{0.986030}%
\pgfsetdash{}{0pt}%
\pgfpathmoveto{\pgfqpoint{2.404185in}{1.322029in}}%
\pgfpathcurveto{\pgfqpoint{2.412421in}{1.322029in}}{\pgfqpoint{2.420321in}{1.325301in}}{\pgfqpoint{2.426145in}{1.331125in}}%
\pgfpathcurveto{\pgfqpoint{2.431969in}{1.336949in}}{\pgfqpoint{2.435242in}{1.344849in}}{\pgfqpoint{2.435242in}{1.353086in}}%
\pgfpathcurveto{\pgfqpoint{2.435242in}{1.361322in}}{\pgfqpoint{2.431969in}{1.369222in}}{\pgfqpoint{2.426145in}{1.375046in}}%
\pgfpathcurveto{\pgfqpoint{2.420321in}{1.380870in}}{\pgfqpoint{2.412421in}{1.384142in}}{\pgfqpoint{2.404185in}{1.384142in}}%
\pgfpathcurveto{\pgfqpoint{2.395949in}{1.384142in}}{\pgfqpoint{2.388049in}{1.380870in}}{\pgfqpoint{2.382225in}{1.375046in}}%
\pgfpathcurveto{\pgfqpoint{2.376401in}{1.369222in}}{\pgfqpoint{2.373129in}{1.361322in}}{\pgfqpoint{2.373129in}{1.353086in}}%
\pgfpathcurveto{\pgfqpoint{2.373129in}{1.344849in}}{\pgfqpoint{2.376401in}{1.336949in}}{\pgfqpoint{2.382225in}{1.331125in}}%
\pgfpathcurveto{\pgfqpoint{2.388049in}{1.325301in}}{\pgfqpoint{2.395949in}{1.322029in}}{\pgfqpoint{2.404185in}{1.322029in}}%
\pgfpathclose%
\pgfusepath{stroke,fill}%
\end{pgfscope}%
\begin{pgfscope}%
\pgfpathrectangle{\pgfqpoint{0.100000in}{0.212622in}}{\pgfqpoint{3.696000in}{3.696000in}}%
\pgfusepath{clip}%
\pgfsetbuttcap%
\pgfsetroundjoin%
\definecolor{currentfill}{rgb}{0.121569,0.466667,0.705882}%
\pgfsetfillcolor{currentfill}%
\pgfsetfillopacity{0.987416}%
\pgfsetlinewidth{1.003750pt}%
\definecolor{currentstroke}{rgb}{0.121569,0.466667,0.705882}%
\pgfsetstrokecolor{currentstroke}%
\pgfsetstrokeopacity{0.987416}%
\pgfsetdash{}{0pt}%
\pgfpathmoveto{\pgfqpoint{2.277415in}{1.382530in}}%
\pgfpathcurveto{\pgfqpoint{2.285652in}{1.382530in}}{\pgfqpoint{2.293552in}{1.385803in}}{\pgfqpoint{2.299376in}{1.391626in}}%
\pgfpathcurveto{\pgfqpoint{2.305200in}{1.397450in}}{\pgfqpoint{2.308472in}{1.405350in}}{\pgfqpoint{2.308472in}{1.413587in}}%
\pgfpathcurveto{\pgfqpoint{2.308472in}{1.421823in}}{\pgfqpoint{2.305200in}{1.429723in}}{\pgfqpoint{2.299376in}{1.435547in}}%
\pgfpathcurveto{\pgfqpoint{2.293552in}{1.441371in}}{\pgfqpoint{2.285652in}{1.444643in}}{\pgfqpoint{2.277415in}{1.444643in}}%
\pgfpathcurveto{\pgfqpoint{2.269179in}{1.444643in}}{\pgfqpoint{2.261279in}{1.441371in}}{\pgfqpoint{2.255455in}{1.435547in}}%
\pgfpathcurveto{\pgfqpoint{2.249631in}{1.429723in}}{\pgfqpoint{2.246359in}{1.421823in}}{\pgfqpoint{2.246359in}{1.413587in}}%
\pgfpathcurveto{\pgfqpoint{2.246359in}{1.405350in}}{\pgfqpoint{2.249631in}{1.397450in}}{\pgfqpoint{2.255455in}{1.391626in}}%
\pgfpathcurveto{\pgfqpoint{2.261279in}{1.385803in}}{\pgfqpoint{2.269179in}{1.382530in}}{\pgfqpoint{2.277415in}{1.382530in}}%
\pgfpathclose%
\pgfusepath{stroke,fill}%
\end{pgfscope}%
\begin{pgfscope}%
\pgfpathrectangle{\pgfqpoint{0.100000in}{0.212622in}}{\pgfqpoint{3.696000in}{3.696000in}}%
\pgfusepath{clip}%
\pgfsetbuttcap%
\pgfsetroundjoin%
\definecolor{currentfill}{rgb}{0.121569,0.466667,0.705882}%
\pgfsetfillcolor{currentfill}%
\pgfsetfillopacity{0.989174}%
\pgfsetlinewidth{1.003750pt}%
\definecolor{currentstroke}{rgb}{0.121569,0.466667,0.705882}%
\pgfsetstrokecolor{currentstroke}%
\pgfsetstrokeopacity{0.989174}%
\pgfsetdash{}{0pt}%
\pgfpathmoveto{\pgfqpoint{2.289982in}{1.377452in}}%
\pgfpathcurveto{\pgfqpoint{2.298218in}{1.377452in}}{\pgfqpoint{2.306118in}{1.380724in}}{\pgfqpoint{2.311942in}{1.386548in}}%
\pgfpathcurveto{\pgfqpoint{2.317766in}{1.392372in}}{\pgfqpoint{2.321038in}{1.400272in}}{\pgfqpoint{2.321038in}{1.408509in}}%
\pgfpathcurveto{\pgfqpoint{2.321038in}{1.416745in}}{\pgfqpoint{2.317766in}{1.424645in}}{\pgfqpoint{2.311942in}{1.430469in}}%
\pgfpathcurveto{\pgfqpoint{2.306118in}{1.436293in}}{\pgfqpoint{2.298218in}{1.439565in}}{\pgfqpoint{2.289982in}{1.439565in}}%
\pgfpathcurveto{\pgfqpoint{2.281746in}{1.439565in}}{\pgfqpoint{2.273846in}{1.436293in}}{\pgfqpoint{2.268022in}{1.430469in}}%
\pgfpathcurveto{\pgfqpoint{2.262198in}{1.424645in}}{\pgfqpoint{2.258925in}{1.416745in}}{\pgfqpoint{2.258925in}{1.408509in}}%
\pgfpathcurveto{\pgfqpoint{2.258925in}{1.400272in}}{\pgfqpoint{2.262198in}{1.392372in}}{\pgfqpoint{2.268022in}{1.386548in}}%
\pgfpathcurveto{\pgfqpoint{2.273846in}{1.380724in}}{\pgfqpoint{2.281746in}{1.377452in}}{\pgfqpoint{2.289982in}{1.377452in}}%
\pgfpathclose%
\pgfusepath{stroke,fill}%
\end{pgfscope}%
\begin{pgfscope}%
\pgfpathrectangle{\pgfqpoint{0.100000in}{0.212622in}}{\pgfqpoint{3.696000in}{3.696000in}}%
\pgfusepath{clip}%
\pgfsetbuttcap%
\pgfsetroundjoin%
\definecolor{currentfill}{rgb}{0.121569,0.466667,0.705882}%
\pgfsetfillcolor{currentfill}%
\pgfsetfillopacity{0.989933}%
\pgfsetlinewidth{1.003750pt}%
\definecolor{currentstroke}{rgb}{0.121569,0.466667,0.705882}%
\pgfsetstrokecolor{currentstroke}%
\pgfsetstrokeopacity{0.989933}%
\pgfsetdash{}{0pt}%
\pgfpathmoveto{\pgfqpoint{2.300540in}{1.365094in}}%
\pgfpathcurveto{\pgfqpoint{2.308776in}{1.365094in}}{\pgfqpoint{2.316676in}{1.368366in}}{\pgfqpoint{2.322500in}{1.374190in}}%
\pgfpathcurveto{\pgfqpoint{2.328324in}{1.380014in}}{\pgfqpoint{2.331596in}{1.387914in}}{\pgfqpoint{2.331596in}{1.396150in}}%
\pgfpathcurveto{\pgfqpoint{2.331596in}{1.404387in}}{\pgfqpoint{2.328324in}{1.412287in}}{\pgfqpoint{2.322500in}{1.418110in}}%
\pgfpathcurveto{\pgfqpoint{2.316676in}{1.423934in}}{\pgfqpoint{2.308776in}{1.427207in}}{\pgfqpoint{2.300540in}{1.427207in}}%
\pgfpathcurveto{\pgfqpoint{2.292303in}{1.427207in}}{\pgfqpoint{2.284403in}{1.423934in}}{\pgfqpoint{2.278579in}{1.418110in}}%
\pgfpathcurveto{\pgfqpoint{2.272755in}{1.412287in}}{\pgfqpoint{2.269483in}{1.404387in}}{\pgfqpoint{2.269483in}{1.396150in}}%
\pgfpathcurveto{\pgfqpoint{2.269483in}{1.387914in}}{\pgfqpoint{2.272755in}{1.380014in}}{\pgfqpoint{2.278579in}{1.374190in}}%
\pgfpathcurveto{\pgfqpoint{2.284403in}{1.368366in}}{\pgfqpoint{2.292303in}{1.365094in}}{\pgfqpoint{2.300540in}{1.365094in}}%
\pgfpathclose%
\pgfusepath{stroke,fill}%
\end{pgfscope}%
\begin{pgfscope}%
\pgfpathrectangle{\pgfqpoint{0.100000in}{0.212622in}}{\pgfqpoint{3.696000in}{3.696000in}}%
\pgfusepath{clip}%
\pgfsetbuttcap%
\pgfsetroundjoin%
\definecolor{currentfill}{rgb}{0.121569,0.466667,0.705882}%
\pgfsetfillcolor{currentfill}%
\pgfsetfillopacity{0.990875}%
\pgfsetlinewidth{1.003750pt}%
\definecolor{currentstroke}{rgb}{0.121569,0.466667,0.705882}%
\pgfsetstrokecolor{currentstroke}%
\pgfsetstrokeopacity{0.990875}%
\pgfsetdash{}{0pt}%
\pgfpathmoveto{\pgfqpoint{2.407531in}{1.312715in}}%
\pgfpathcurveto{\pgfqpoint{2.415767in}{1.312715in}}{\pgfqpoint{2.423667in}{1.315987in}}{\pgfqpoint{2.429491in}{1.321811in}}%
\pgfpathcurveto{\pgfqpoint{2.435315in}{1.327635in}}{\pgfqpoint{2.438587in}{1.335535in}}{\pgfqpoint{2.438587in}{1.343772in}}%
\pgfpathcurveto{\pgfqpoint{2.438587in}{1.352008in}}{\pgfqpoint{2.435315in}{1.359908in}}{\pgfqpoint{2.429491in}{1.365732in}}%
\pgfpathcurveto{\pgfqpoint{2.423667in}{1.371556in}}{\pgfqpoint{2.415767in}{1.374828in}}{\pgfqpoint{2.407531in}{1.374828in}}%
\pgfpathcurveto{\pgfqpoint{2.399294in}{1.374828in}}{\pgfqpoint{2.391394in}{1.371556in}}{\pgfqpoint{2.385570in}{1.365732in}}%
\pgfpathcurveto{\pgfqpoint{2.379746in}{1.359908in}}{\pgfqpoint{2.376474in}{1.352008in}}{\pgfqpoint{2.376474in}{1.343772in}}%
\pgfpathcurveto{\pgfqpoint{2.376474in}{1.335535in}}{\pgfqpoint{2.379746in}{1.327635in}}{\pgfqpoint{2.385570in}{1.321811in}}%
\pgfpathcurveto{\pgfqpoint{2.391394in}{1.315987in}}{\pgfqpoint{2.399294in}{1.312715in}}{\pgfqpoint{2.407531in}{1.312715in}}%
\pgfpathclose%
\pgfusepath{stroke,fill}%
\end{pgfscope}%
\begin{pgfscope}%
\pgfpathrectangle{\pgfqpoint{0.100000in}{0.212622in}}{\pgfqpoint{3.696000in}{3.696000in}}%
\pgfusepath{clip}%
\pgfsetbuttcap%
\pgfsetroundjoin%
\definecolor{currentfill}{rgb}{0.121569,0.466667,0.705882}%
\pgfsetfillcolor{currentfill}%
\pgfsetfillopacity{0.991099}%
\pgfsetlinewidth{1.003750pt}%
\definecolor{currentstroke}{rgb}{0.121569,0.466667,0.705882}%
\pgfsetstrokecolor{currentstroke}%
\pgfsetstrokeopacity{0.991099}%
\pgfsetdash{}{0pt}%
\pgfpathmoveto{\pgfqpoint{2.310330in}{1.359450in}}%
\pgfpathcurveto{\pgfqpoint{2.318567in}{1.359450in}}{\pgfqpoint{2.326467in}{1.362722in}}{\pgfqpoint{2.332290in}{1.368546in}}%
\pgfpathcurveto{\pgfqpoint{2.338114in}{1.374370in}}{\pgfqpoint{2.341387in}{1.382270in}}{\pgfqpoint{2.341387in}{1.390506in}}%
\pgfpathcurveto{\pgfqpoint{2.341387in}{1.398743in}}{\pgfqpoint{2.338114in}{1.406643in}}{\pgfqpoint{2.332290in}{1.412467in}}%
\pgfpathcurveto{\pgfqpoint{2.326467in}{1.418291in}}{\pgfqpoint{2.318567in}{1.421563in}}{\pgfqpoint{2.310330in}{1.421563in}}%
\pgfpathcurveto{\pgfqpoint{2.302094in}{1.421563in}}{\pgfqpoint{2.294194in}{1.418291in}}{\pgfqpoint{2.288370in}{1.412467in}}%
\pgfpathcurveto{\pgfqpoint{2.282546in}{1.406643in}}{\pgfqpoint{2.279274in}{1.398743in}}{\pgfqpoint{2.279274in}{1.390506in}}%
\pgfpathcurveto{\pgfqpoint{2.279274in}{1.382270in}}{\pgfqpoint{2.282546in}{1.374370in}}{\pgfqpoint{2.288370in}{1.368546in}}%
\pgfpathcurveto{\pgfqpoint{2.294194in}{1.362722in}}{\pgfqpoint{2.302094in}{1.359450in}}{\pgfqpoint{2.310330in}{1.359450in}}%
\pgfpathclose%
\pgfusepath{stroke,fill}%
\end{pgfscope}%
\begin{pgfscope}%
\pgfpathrectangle{\pgfqpoint{0.100000in}{0.212622in}}{\pgfqpoint{3.696000in}{3.696000in}}%
\pgfusepath{clip}%
\pgfsetbuttcap%
\pgfsetroundjoin%
\definecolor{currentfill}{rgb}{0.121569,0.466667,0.705882}%
\pgfsetfillcolor{currentfill}%
\pgfsetfillopacity{0.991910}%
\pgfsetlinewidth{1.003750pt}%
\definecolor{currentstroke}{rgb}{0.121569,0.466667,0.705882}%
\pgfsetstrokecolor{currentstroke}%
\pgfsetstrokeopacity{0.991910}%
\pgfsetdash{}{0pt}%
\pgfpathmoveto{\pgfqpoint{2.316989in}{1.355602in}}%
\pgfpathcurveto{\pgfqpoint{2.325225in}{1.355602in}}{\pgfqpoint{2.333125in}{1.358874in}}{\pgfqpoint{2.338949in}{1.364698in}}%
\pgfpathcurveto{\pgfqpoint{2.344773in}{1.370522in}}{\pgfqpoint{2.348046in}{1.378422in}}{\pgfqpoint{2.348046in}{1.386658in}}%
\pgfpathcurveto{\pgfqpoint{2.348046in}{1.394895in}}{\pgfqpoint{2.344773in}{1.402795in}}{\pgfqpoint{2.338949in}{1.408619in}}%
\pgfpathcurveto{\pgfqpoint{2.333125in}{1.414443in}}{\pgfqpoint{2.325225in}{1.417715in}}{\pgfqpoint{2.316989in}{1.417715in}}%
\pgfpathcurveto{\pgfqpoint{2.308753in}{1.417715in}}{\pgfqpoint{2.300853in}{1.414443in}}{\pgfqpoint{2.295029in}{1.408619in}}%
\pgfpathcurveto{\pgfqpoint{2.289205in}{1.402795in}}{\pgfqpoint{2.285933in}{1.394895in}}{\pgfqpoint{2.285933in}{1.386658in}}%
\pgfpathcurveto{\pgfqpoint{2.285933in}{1.378422in}}{\pgfqpoint{2.289205in}{1.370522in}}{\pgfqpoint{2.295029in}{1.364698in}}%
\pgfpathcurveto{\pgfqpoint{2.300853in}{1.358874in}}{\pgfqpoint{2.308753in}{1.355602in}}{\pgfqpoint{2.316989in}{1.355602in}}%
\pgfpathclose%
\pgfusepath{stroke,fill}%
\end{pgfscope}%
\begin{pgfscope}%
\pgfpathrectangle{\pgfqpoint{0.100000in}{0.212622in}}{\pgfqpoint{3.696000in}{3.696000in}}%
\pgfusepath{clip}%
\pgfsetbuttcap%
\pgfsetroundjoin%
\definecolor{currentfill}{rgb}{0.121569,0.466667,0.705882}%
\pgfsetfillcolor{currentfill}%
\pgfsetfillopacity{0.992568}%
\pgfsetlinewidth{1.003750pt}%
\definecolor{currentstroke}{rgb}{0.121569,0.466667,0.705882}%
\pgfsetstrokecolor{currentstroke}%
\pgfsetstrokeopacity{0.992568}%
\pgfsetdash{}{0pt}%
\pgfpathmoveto{\pgfqpoint{2.322884in}{1.351663in}}%
\pgfpathcurveto{\pgfqpoint{2.331121in}{1.351663in}}{\pgfqpoint{2.339021in}{1.354935in}}{\pgfqpoint{2.344845in}{1.360759in}}%
\pgfpathcurveto{\pgfqpoint{2.350669in}{1.366583in}}{\pgfqpoint{2.353941in}{1.374483in}}{\pgfqpoint{2.353941in}{1.382719in}}%
\pgfpathcurveto{\pgfqpoint{2.353941in}{1.390956in}}{\pgfqpoint{2.350669in}{1.398856in}}{\pgfqpoint{2.344845in}{1.404680in}}%
\pgfpathcurveto{\pgfqpoint{2.339021in}{1.410504in}}{\pgfqpoint{2.331121in}{1.413776in}}{\pgfqpoint{2.322884in}{1.413776in}}%
\pgfpathcurveto{\pgfqpoint{2.314648in}{1.413776in}}{\pgfqpoint{2.306748in}{1.410504in}}{\pgfqpoint{2.300924in}{1.404680in}}%
\pgfpathcurveto{\pgfqpoint{2.295100in}{1.398856in}}{\pgfqpoint{2.291828in}{1.390956in}}{\pgfqpoint{2.291828in}{1.382719in}}%
\pgfpathcurveto{\pgfqpoint{2.291828in}{1.374483in}}{\pgfqpoint{2.295100in}{1.366583in}}{\pgfqpoint{2.300924in}{1.360759in}}%
\pgfpathcurveto{\pgfqpoint{2.306748in}{1.354935in}}{\pgfqpoint{2.314648in}{1.351663in}}{\pgfqpoint{2.322884in}{1.351663in}}%
\pgfpathclose%
\pgfusepath{stroke,fill}%
\end{pgfscope}%
\begin{pgfscope}%
\pgfpathrectangle{\pgfqpoint{0.100000in}{0.212622in}}{\pgfqpoint{3.696000in}{3.696000in}}%
\pgfusepath{clip}%
\pgfsetbuttcap%
\pgfsetroundjoin%
\definecolor{currentfill}{rgb}{0.121569,0.466667,0.705882}%
\pgfsetfillcolor{currentfill}%
\pgfsetfillopacity{0.993013}%
\pgfsetlinewidth{1.003750pt}%
\definecolor{currentstroke}{rgb}{0.121569,0.466667,0.705882}%
\pgfsetstrokecolor{currentstroke}%
\pgfsetstrokeopacity{0.993013}%
\pgfsetdash{}{0pt}%
\pgfpathmoveto{\pgfqpoint{2.327316in}{1.348664in}}%
\pgfpathcurveto{\pgfqpoint{2.335553in}{1.348664in}}{\pgfqpoint{2.343453in}{1.351936in}}{\pgfqpoint{2.349277in}{1.357760in}}%
\pgfpathcurveto{\pgfqpoint{2.355100in}{1.363584in}}{\pgfqpoint{2.358373in}{1.371484in}}{\pgfqpoint{2.358373in}{1.379721in}}%
\pgfpathcurveto{\pgfqpoint{2.358373in}{1.387957in}}{\pgfqpoint{2.355100in}{1.395857in}}{\pgfqpoint{2.349277in}{1.401681in}}%
\pgfpathcurveto{\pgfqpoint{2.343453in}{1.407505in}}{\pgfqpoint{2.335553in}{1.410777in}}{\pgfqpoint{2.327316in}{1.410777in}}%
\pgfpathcurveto{\pgfqpoint{2.319080in}{1.410777in}}{\pgfqpoint{2.311180in}{1.407505in}}{\pgfqpoint{2.305356in}{1.401681in}}%
\pgfpathcurveto{\pgfqpoint{2.299532in}{1.395857in}}{\pgfqpoint{2.296260in}{1.387957in}}{\pgfqpoint{2.296260in}{1.379721in}}%
\pgfpathcurveto{\pgfqpoint{2.296260in}{1.371484in}}{\pgfqpoint{2.299532in}{1.363584in}}{\pgfqpoint{2.305356in}{1.357760in}}%
\pgfpathcurveto{\pgfqpoint{2.311180in}{1.351936in}}{\pgfqpoint{2.319080in}{1.348664in}}{\pgfqpoint{2.327316in}{1.348664in}}%
\pgfpathclose%
\pgfusepath{stroke,fill}%
\end{pgfscope}%
\begin{pgfscope}%
\pgfpathrectangle{\pgfqpoint{0.100000in}{0.212622in}}{\pgfqpoint{3.696000in}{3.696000in}}%
\pgfusepath{clip}%
\pgfsetbuttcap%
\pgfsetroundjoin%
\definecolor{currentfill}{rgb}{0.121569,0.466667,0.705882}%
\pgfsetfillcolor{currentfill}%
\pgfsetfillopacity{0.993675}%
\pgfsetlinewidth{1.003750pt}%
\definecolor{currentstroke}{rgb}{0.121569,0.466667,0.705882}%
\pgfsetstrokecolor{currentstroke}%
\pgfsetstrokeopacity{0.993675}%
\pgfsetdash{}{0pt}%
\pgfpathmoveto{\pgfqpoint{2.335668in}{1.343338in}}%
\pgfpathcurveto{\pgfqpoint{2.343905in}{1.343338in}}{\pgfqpoint{2.351805in}{1.346611in}}{\pgfqpoint{2.357628in}{1.352435in}}%
\pgfpathcurveto{\pgfqpoint{2.363452in}{1.358259in}}{\pgfqpoint{2.366725in}{1.366159in}}{\pgfqpoint{2.366725in}{1.374395in}}%
\pgfpathcurveto{\pgfqpoint{2.366725in}{1.382631in}}{\pgfqpoint{2.363452in}{1.390531in}}{\pgfqpoint{2.357628in}{1.396355in}}%
\pgfpathcurveto{\pgfqpoint{2.351805in}{1.402179in}}{\pgfqpoint{2.343905in}{1.405451in}}{\pgfqpoint{2.335668in}{1.405451in}}%
\pgfpathcurveto{\pgfqpoint{2.327432in}{1.405451in}}{\pgfqpoint{2.319532in}{1.402179in}}{\pgfqpoint{2.313708in}{1.396355in}}%
\pgfpathcurveto{\pgfqpoint{2.307884in}{1.390531in}}{\pgfqpoint{2.304612in}{1.382631in}}{\pgfqpoint{2.304612in}{1.374395in}}%
\pgfpathcurveto{\pgfqpoint{2.304612in}{1.366159in}}{\pgfqpoint{2.307884in}{1.358259in}}{\pgfqpoint{2.313708in}{1.352435in}}%
\pgfpathcurveto{\pgfqpoint{2.319532in}{1.346611in}}{\pgfqpoint{2.327432in}{1.343338in}}{\pgfqpoint{2.335668in}{1.343338in}}%
\pgfpathclose%
\pgfusepath{stroke,fill}%
\end{pgfscope}%
\begin{pgfscope}%
\pgfpathrectangle{\pgfqpoint{0.100000in}{0.212622in}}{\pgfqpoint{3.696000in}{3.696000in}}%
\pgfusepath{clip}%
\pgfsetbuttcap%
\pgfsetroundjoin%
\definecolor{currentfill}{rgb}{0.121569,0.466667,0.705882}%
\pgfsetfillcolor{currentfill}%
\pgfsetfillopacity{0.993774}%
\pgfsetlinewidth{1.003750pt}%
\definecolor{currentstroke}{rgb}{0.121569,0.466667,0.705882}%
\pgfsetstrokecolor{currentstroke}%
\pgfsetstrokeopacity{0.993774}%
\pgfsetdash{}{0pt}%
\pgfpathmoveto{\pgfqpoint{2.409055in}{1.308814in}}%
\pgfpathcurveto{\pgfqpoint{2.417291in}{1.308814in}}{\pgfqpoint{2.425191in}{1.312086in}}{\pgfqpoint{2.431015in}{1.317910in}}%
\pgfpathcurveto{\pgfqpoint{2.436839in}{1.323734in}}{\pgfqpoint{2.440111in}{1.331634in}}{\pgfqpoint{2.440111in}{1.339870in}}%
\pgfpathcurveto{\pgfqpoint{2.440111in}{1.348106in}}{\pgfqpoint{2.436839in}{1.356006in}}{\pgfqpoint{2.431015in}{1.361830in}}%
\pgfpathcurveto{\pgfqpoint{2.425191in}{1.367654in}}{\pgfqpoint{2.417291in}{1.370927in}}{\pgfqpoint{2.409055in}{1.370927in}}%
\pgfpathcurveto{\pgfqpoint{2.400819in}{1.370927in}}{\pgfqpoint{2.392919in}{1.367654in}}{\pgfqpoint{2.387095in}{1.361830in}}%
\pgfpathcurveto{\pgfqpoint{2.381271in}{1.356006in}}{\pgfqpoint{2.377998in}{1.348106in}}{\pgfqpoint{2.377998in}{1.339870in}}%
\pgfpathcurveto{\pgfqpoint{2.377998in}{1.331634in}}{\pgfqpoint{2.381271in}{1.323734in}}{\pgfqpoint{2.387095in}{1.317910in}}%
\pgfpathcurveto{\pgfqpoint{2.392919in}{1.312086in}}{\pgfqpoint{2.400819in}{1.308814in}}{\pgfqpoint{2.409055in}{1.308814in}}%
\pgfpathclose%
\pgfusepath{stroke,fill}%
\end{pgfscope}%
\begin{pgfscope}%
\pgfpathrectangle{\pgfqpoint{0.100000in}{0.212622in}}{\pgfqpoint{3.696000in}{3.696000in}}%
\pgfusepath{clip}%
\pgfsetbuttcap%
\pgfsetroundjoin%
\definecolor{currentfill}{rgb}{0.121569,0.466667,0.705882}%
\pgfsetfillcolor{currentfill}%
\pgfsetfillopacity{0.994332}%
\pgfsetlinewidth{1.003750pt}%
\definecolor{currentstroke}{rgb}{0.121569,0.466667,0.705882}%
\pgfsetstrokecolor{currentstroke}%
\pgfsetstrokeopacity{0.994332}%
\pgfsetdash{}{0pt}%
\pgfpathmoveto{\pgfqpoint{2.340560in}{1.341969in}}%
\pgfpathcurveto{\pgfqpoint{2.348796in}{1.341969in}}{\pgfqpoint{2.356696in}{1.345241in}}{\pgfqpoint{2.362520in}{1.351065in}}%
\pgfpathcurveto{\pgfqpoint{2.368344in}{1.356889in}}{\pgfqpoint{2.371616in}{1.364789in}}{\pgfqpoint{2.371616in}{1.373025in}}%
\pgfpathcurveto{\pgfqpoint{2.371616in}{1.381262in}}{\pgfqpoint{2.368344in}{1.389162in}}{\pgfqpoint{2.362520in}{1.394986in}}%
\pgfpathcurveto{\pgfqpoint{2.356696in}{1.400810in}}{\pgfqpoint{2.348796in}{1.404082in}}{\pgfqpoint{2.340560in}{1.404082in}}%
\pgfpathcurveto{\pgfqpoint{2.332324in}{1.404082in}}{\pgfqpoint{2.324424in}{1.400810in}}{\pgfqpoint{2.318600in}{1.394986in}}%
\pgfpathcurveto{\pgfqpoint{2.312776in}{1.389162in}}{\pgfqpoint{2.309503in}{1.381262in}}{\pgfqpoint{2.309503in}{1.373025in}}%
\pgfpathcurveto{\pgfqpoint{2.309503in}{1.364789in}}{\pgfqpoint{2.312776in}{1.356889in}}{\pgfqpoint{2.318600in}{1.351065in}}%
\pgfpathcurveto{\pgfqpoint{2.324424in}{1.345241in}}{\pgfqpoint{2.332324in}{1.341969in}}{\pgfqpoint{2.340560in}{1.341969in}}%
\pgfpathclose%
\pgfusepath{stroke,fill}%
\end{pgfscope}%
\begin{pgfscope}%
\pgfpathrectangle{\pgfqpoint{0.100000in}{0.212622in}}{\pgfqpoint{3.696000in}{3.696000in}}%
\pgfusepath{clip}%
\pgfsetbuttcap%
\pgfsetroundjoin%
\definecolor{currentfill}{rgb}{0.121569,0.466667,0.705882}%
\pgfsetfillcolor{currentfill}%
\pgfsetfillopacity{0.994624}%
\pgfsetlinewidth{1.003750pt}%
\definecolor{currentstroke}{rgb}{0.121569,0.466667,0.705882}%
\pgfsetstrokecolor{currentstroke}%
\pgfsetstrokeopacity{0.994624}%
\pgfsetdash{}{0pt}%
\pgfpathmoveto{\pgfqpoint{2.344405in}{1.337826in}}%
\pgfpathcurveto{\pgfqpoint{2.352642in}{1.337826in}}{\pgfqpoint{2.360542in}{1.341098in}}{\pgfqpoint{2.366366in}{1.346922in}}%
\pgfpathcurveto{\pgfqpoint{2.372190in}{1.352746in}}{\pgfqpoint{2.375462in}{1.360646in}}{\pgfqpoint{2.375462in}{1.368882in}}%
\pgfpathcurveto{\pgfqpoint{2.375462in}{1.377118in}}{\pgfqpoint{2.372190in}{1.385018in}}{\pgfqpoint{2.366366in}{1.390842in}}%
\pgfpathcurveto{\pgfqpoint{2.360542in}{1.396666in}}{\pgfqpoint{2.352642in}{1.399939in}}{\pgfqpoint{2.344405in}{1.399939in}}%
\pgfpathcurveto{\pgfqpoint{2.336169in}{1.399939in}}{\pgfqpoint{2.328269in}{1.396666in}}{\pgfqpoint{2.322445in}{1.390842in}}%
\pgfpathcurveto{\pgfqpoint{2.316621in}{1.385018in}}{\pgfqpoint{2.313349in}{1.377118in}}{\pgfqpoint{2.313349in}{1.368882in}}%
\pgfpathcurveto{\pgfqpoint{2.313349in}{1.360646in}}{\pgfqpoint{2.316621in}{1.352746in}}{\pgfqpoint{2.322445in}{1.346922in}}%
\pgfpathcurveto{\pgfqpoint{2.328269in}{1.341098in}}{\pgfqpoint{2.336169in}{1.337826in}}{\pgfqpoint{2.344405in}{1.337826in}}%
\pgfpathclose%
\pgfusepath{stroke,fill}%
\end{pgfscope}%
\begin{pgfscope}%
\pgfpathrectangle{\pgfqpoint{0.100000in}{0.212622in}}{\pgfqpoint{3.696000in}{3.696000in}}%
\pgfusepath{clip}%
\pgfsetbuttcap%
\pgfsetroundjoin%
\definecolor{currentfill}{rgb}{0.121569,0.466667,0.705882}%
\pgfsetfillcolor{currentfill}%
\pgfsetfillopacity{0.994741}%
\pgfsetlinewidth{1.003750pt}%
\definecolor{currentstroke}{rgb}{0.121569,0.466667,0.705882}%
\pgfsetstrokecolor{currentstroke}%
\pgfsetstrokeopacity{0.994741}%
\pgfsetdash{}{0pt}%
\pgfpathmoveto{\pgfqpoint{2.346199in}{1.336498in}}%
\pgfpathcurveto{\pgfqpoint{2.354435in}{1.336498in}}{\pgfqpoint{2.362335in}{1.339770in}}{\pgfqpoint{2.368159in}{1.345594in}}%
\pgfpathcurveto{\pgfqpoint{2.373983in}{1.351418in}}{\pgfqpoint{2.377255in}{1.359318in}}{\pgfqpoint{2.377255in}{1.367554in}}%
\pgfpathcurveto{\pgfqpoint{2.377255in}{1.375791in}}{\pgfqpoint{2.373983in}{1.383691in}}{\pgfqpoint{2.368159in}{1.389515in}}%
\pgfpathcurveto{\pgfqpoint{2.362335in}{1.395338in}}{\pgfqpoint{2.354435in}{1.398611in}}{\pgfqpoint{2.346199in}{1.398611in}}%
\pgfpathcurveto{\pgfqpoint{2.337963in}{1.398611in}}{\pgfqpoint{2.330063in}{1.395338in}}{\pgfqpoint{2.324239in}{1.389515in}}%
\pgfpathcurveto{\pgfqpoint{2.318415in}{1.383691in}}{\pgfqpoint{2.315142in}{1.375791in}}{\pgfqpoint{2.315142in}{1.367554in}}%
\pgfpathcurveto{\pgfqpoint{2.315142in}{1.359318in}}{\pgfqpoint{2.318415in}{1.351418in}}{\pgfqpoint{2.324239in}{1.345594in}}%
\pgfpathcurveto{\pgfqpoint{2.330063in}{1.339770in}}{\pgfqpoint{2.337963in}{1.336498in}}{\pgfqpoint{2.346199in}{1.336498in}}%
\pgfpathclose%
\pgfusepath{stroke,fill}%
\end{pgfscope}%
\begin{pgfscope}%
\pgfpathrectangle{\pgfqpoint{0.100000in}{0.212622in}}{\pgfqpoint{3.696000in}{3.696000in}}%
\pgfusepath{clip}%
\pgfsetbuttcap%
\pgfsetroundjoin%
\definecolor{currentfill}{rgb}{0.121569,0.466667,0.705882}%
\pgfsetfillcolor{currentfill}%
\pgfsetfillopacity{0.995088}%
\pgfsetlinewidth{1.003750pt}%
\definecolor{currentstroke}{rgb}{0.121569,0.466667,0.705882}%
\pgfsetstrokecolor{currentstroke}%
\pgfsetstrokeopacity{0.995088}%
\pgfsetdash{}{0pt}%
\pgfpathmoveto{\pgfqpoint{2.349447in}{1.334827in}}%
\pgfpathcurveto{\pgfqpoint{2.357683in}{1.334827in}}{\pgfqpoint{2.365584in}{1.338100in}}{\pgfqpoint{2.371407in}{1.343923in}}%
\pgfpathcurveto{\pgfqpoint{2.377231in}{1.349747in}}{\pgfqpoint{2.380504in}{1.357647in}}{\pgfqpoint{2.380504in}{1.365884in}}%
\pgfpathcurveto{\pgfqpoint{2.380504in}{1.374120in}}{\pgfqpoint{2.377231in}{1.382020in}}{\pgfqpoint{2.371407in}{1.387844in}}%
\pgfpathcurveto{\pgfqpoint{2.365584in}{1.393668in}}{\pgfqpoint{2.357683in}{1.396940in}}{\pgfqpoint{2.349447in}{1.396940in}}%
\pgfpathcurveto{\pgfqpoint{2.341211in}{1.396940in}}{\pgfqpoint{2.333311in}{1.393668in}}{\pgfqpoint{2.327487in}{1.387844in}}%
\pgfpathcurveto{\pgfqpoint{2.321663in}{1.382020in}}{\pgfqpoint{2.318391in}{1.374120in}}{\pgfqpoint{2.318391in}{1.365884in}}%
\pgfpathcurveto{\pgfqpoint{2.318391in}{1.357647in}}{\pgfqpoint{2.321663in}{1.349747in}}{\pgfqpoint{2.327487in}{1.343923in}}%
\pgfpathcurveto{\pgfqpoint{2.333311in}{1.338100in}}{\pgfqpoint{2.341211in}{1.334827in}}{\pgfqpoint{2.349447in}{1.334827in}}%
\pgfpathclose%
\pgfusepath{stroke,fill}%
\end{pgfscope}%
\begin{pgfscope}%
\pgfpathrectangle{\pgfqpoint{0.100000in}{0.212622in}}{\pgfqpoint{3.696000in}{3.696000in}}%
\pgfusepath{clip}%
\pgfsetbuttcap%
\pgfsetroundjoin%
\definecolor{currentfill}{rgb}{0.121569,0.466667,0.705882}%
\pgfsetfillcolor{currentfill}%
\pgfsetfillopacity{0.995214}%
\pgfsetlinewidth{1.003750pt}%
\definecolor{currentstroke}{rgb}{0.121569,0.466667,0.705882}%
\pgfsetstrokecolor{currentstroke}%
\pgfsetstrokeopacity{0.995214}%
\pgfsetdash{}{0pt}%
\pgfpathmoveto{\pgfqpoint{2.350586in}{1.334217in}}%
\pgfpathcurveto{\pgfqpoint{2.358823in}{1.334217in}}{\pgfqpoint{2.366723in}{1.337489in}}{\pgfqpoint{2.372547in}{1.343313in}}%
\pgfpathcurveto{\pgfqpoint{2.378371in}{1.349137in}}{\pgfqpoint{2.381643in}{1.357037in}}{\pgfqpoint{2.381643in}{1.365273in}}%
\pgfpathcurveto{\pgfqpoint{2.381643in}{1.373510in}}{\pgfqpoint{2.378371in}{1.381410in}}{\pgfqpoint{2.372547in}{1.387234in}}%
\pgfpathcurveto{\pgfqpoint{2.366723in}{1.393058in}}{\pgfqpoint{2.358823in}{1.396330in}}{\pgfqpoint{2.350586in}{1.396330in}}%
\pgfpathcurveto{\pgfqpoint{2.342350in}{1.396330in}}{\pgfqpoint{2.334450in}{1.393058in}}{\pgfqpoint{2.328626in}{1.387234in}}%
\pgfpathcurveto{\pgfqpoint{2.322802in}{1.381410in}}{\pgfqpoint{2.319530in}{1.373510in}}{\pgfqpoint{2.319530in}{1.365273in}}%
\pgfpathcurveto{\pgfqpoint{2.319530in}{1.357037in}}{\pgfqpoint{2.322802in}{1.349137in}}{\pgfqpoint{2.328626in}{1.343313in}}%
\pgfpathcurveto{\pgfqpoint{2.334450in}{1.337489in}}{\pgfqpoint{2.342350in}{1.334217in}}{\pgfqpoint{2.350586in}{1.334217in}}%
\pgfpathclose%
\pgfusepath{stroke,fill}%
\end{pgfscope}%
\begin{pgfscope}%
\pgfpathrectangle{\pgfqpoint{0.100000in}{0.212622in}}{\pgfqpoint{3.696000in}{3.696000in}}%
\pgfusepath{clip}%
\pgfsetbuttcap%
\pgfsetroundjoin%
\definecolor{currentfill}{rgb}{0.121569,0.466667,0.705882}%
\pgfsetfillcolor{currentfill}%
\pgfsetfillopacity{0.995236}%
\pgfsetlinewidth{1.003750pt}%
\definecolor{currentstroke}{rgb}{0.121569,0.466667,0.705882}%
\pgfsetstrokecolor{currentstroke}%
\pgfsetstrokeopacity{0.995236}%
\pgfsetdash{}{0pt}%
\pgfpathmoveto{\pgfqpoint{2.350822in}{1.334060in}}%
\pgfpathcurveto{\pgfqpoint{2.359058in}{1.334060in}}{\pgfqpoint{2.366958in}{1.337332in}}{\pgfqpoint{2.372782in}{1.343156in}}%
\pgfpathcurveto{\pgfqpoint{2.378606in}{1.348980in}}{\pgfqpoint{2.381878in}{1.356880in}}{\pgfqpoint{2.381878in}{1.365116in}}%
\pgfpathcurveto{\pgfqpoint{2.381878in}{1.373352in}}{\pgfqpoint{2.378606in}{1.381253in}}{\pgfqpoint{2.372782in}{1.387076in}}%
\pgfpathcurveto{\pgfqpoint{2.366958in}{1.392900in}}{\pgfqpoint{2.359058in}{1.396173in}}{\pgfqpoint{2.350822in}{1.396173in}}%
\pgfpathcurveto{\pgfqpoint{2.342586in}{1.396173in}}{\pgfqpoint{2.334685in}{1.392900in}}{\pgfqpoint{2.328862in}{1.387076in}}%
\pgfpathcurveto{\pgfqpoint{2.323038in}{1.381253in}}{\pgfqpoint{2.319765in}{1.373352in}}{\pgfqpoint{2.319765in}{1.365116in}}%
\pgfpathcurveto{\pgfqpoint{2.319765in}{1.356880in}}{\pgfqpoint{2.323038in}{1.348980in}}{\pgfqpoint{2.328862in}{1.343156in}}%
\pgfpathcurveto{\pgfqpoint{2.334685in}{1.337332in}}{\pgfqpoint{2.342586in}{1.334060in}}{\pgfqpoint{2.350822in}{1.334060in}}%
\pgfpathclose%
\pgfusepath{stroke,fill}%
\end{pgfscope}%
\begin{pgfscope}%
\pgfpathrectangle{\pgfqpoint{0.100000in}{0.212622in}}{\pgfqpoint{3.696000in}{3.696000in}}%
\pgfusepath{clip}%
\pgfsetbuttcap%
\pgfsetroundjoin%
\definecolor{currentfill}{rgb}{0.121569,0.466667,0.705882}%
\pgfsetfillcolor{currentfill}%
\pgfsetfillopacity{0.995284}%
\pgfsetlinewidth{1.003750pt}%
\definecolor{currentstroke}{rgb}{0.121569,0.466667,0.705882}%
\pgfsetstrokecolor{currentstroke}%
\pgfsetstrokeopacity{0.995284}%
\pgfsetdash{}{0pt}%
\pgfpathmoveto{\pgfqpoint{2.351243in}{1.333794in}}%
\pgfpathcurveto{\pgfqpoint{2.359479in}{1.333794in}}{\pgfqpoint{2.367379in}{1.337066in}}{\pgfqpoint{2.373203in}{1.342890in}}%
\pgfpathcurveto{\pgfqpoint{2.379027in}{1.348714in}}{\pgfqpoint{2.382300in}{1.356614in}}{\pgfqpoint{2.382300in}{1.364850in}}%
\pgfpathcurveto{\pgfqpoint{2.382300in}{1.373087in}}{\pgfqpoint{2.379027in}{1.380987in}}{\pgfqpoint{2.373203in}{1.386811in}}%
\pgfpathcurveto{\pgfqpoint{2.367379in}{1.392635in}}{\pgfqpoint{2.359479in}{1.395907in}}{\pgfqpoint{2.351243in}{1.395907in}}%
\pgfpathcurveto{\pgfqpoint{2.343007in}{1.395907in}}{\pgfqpoint{2.335107in}{1.392635in}}{\pgfqpoint{2.329283in}{1.386811in}}%
\pgfpathcurveto{\pgfqpoint{2.323459in}{1.380987in}}{\pgfqpoint{2.320187in}{1.373087in}}{\pgfqpoint{2.320187in}{1.364850in}}%
\pgfpathcurveto{\pgfqpoint{2.320187in}{1.356614in}}{\pgfqpoint{2.323459in}{1.348714in}}{\pgfqpoint{2.329283in}{1.342890in}}%
\pgfpathcurveto{\pgfqpoint{2.335107in}{1.337066in}}{\pgfqpoint{2.343007in}{1.333794in}}{\pgfqpoint{2.351243in}{1.333794in}}%
\pgfpathclose%
\pgfusepath{stroke,fill}%
\end{pgfscope}%
\begin{pgfscope}%
\pgfpathrectangle{\pgfqpoint{0.100000in}{0.212622in}}{\pgfqpoint{3.696000in}{3.696000in}}%
\pgfusepath{clip}%
\pgfsetbuttcap%
\pgfsetroundjoin%
\definecolor{currentfill}{rgb}{0.121569,0.466667,0.705882}%
\pgfsetfillcolor{currentfill}%
\pgfsetfillopacity{0.995384}%
\pgfsetlinewidth{1.003750pt}%
\definecolor{currentstroke}{rgb}{0.121569,0.466667,0.705882}%
\pgfsetstrokecolor{currentstroke}%
\pgfsetstrokeopacity{0.995384}%
\pgfsetdash{}{0pt}%
\pgfpathmoveto{\pgfqpoint{2.351981in}{1.333302in}}%
\pgfpathcurveto{\pgfqpoint{2.360218in}{1.333302in}}{\pgfqpoint{2.368118in}{1.336575in}}{\pgfqpoint{2.373942in}{1.342398in}}%
\pgfpathcurveto{\pgfqpoint{2.379766in}{1.348222in}}{\pgfqpoint{2.383038in}{1.356122in}}{\pgfqpoint{2.383038in}{1.364359in}}%
\pgfpathcurveto{\pgfqpoint{2.383038in}{1.372595in}}{\pgfqpoint{2.379766in}{1.380495in}}{\pgfqpoint{2.373942in}{1.386319in}}%
\pgfpathcurveto{\pgfqpoint{2.368118in}{1.392143in}}{\pgfqpoint{2.360218in}{1.395415in}}{\pgfqpoint{2.351981in}{1.395415in}}%
\pgfpathcurveto{\pgfqpoint{2.343745in}{1.395415in}}{\pgfqpoint{2.335845in}{1.392143in}}{\pgfqpoint{2.330021in}{1.386319in}}%
\pgfpathcurveto{\pgfqpoint{2.324197in}{1.380495in}}{\pgfqpoint{2.320925in}{1.372595in}}{\pgfqpoint{2.320925in}{1.364359in}}%
\pgfpathcurveto{\pgfqpoint{2.320925in}{1.356122in}}{\pgfqpoint{2.324197in}{1.348222in}}{\pgfqpoint{2.330021in}{1.342398in}}%
\pgfpathcurveto{\pgfqpoint{2.335845in}{1.336575in}}{\pgfqpoint{2.343745in}{1.333302in}}{\pgfqpoint{2.351981in}{1.333302in}}%
\pgfpathclose%
\pgfusepath{stroke,fill}%
\end{pgfscope}%
\begin{pgfscope}%
\pgfpathrectangle{\pgfqpoint{0.100000in}{0.212622in}}{\pgfqpoint{3.696000in}{3.696000in}}%
\pgfusepath{clip}%
\pgfsetbuttcap%
\pgfsetroundjoin%
\definecolor{currentfill}{rgb}{0.121569,0.466667,0.705882}%
\pgfsetfillcolor{currentfill}%
\pgfsetfillopacity{0.995556}%
\pgfsetlinewidth{1.003750pt}%
\definecolor{currentstroke}{rgb}{0.121569,0.466667,0.705882}%
\pgfsetstrokecolor{currentstroke}%
\pgfsetstrokeopacity{0.995556}%
\pgfsetdash{}{0pt}%
\pgfpathmoveto{\pgfqpoint{2.353301in}{1.332287in}}%
\pgfpathcurveto{\pgfqpoint{2.361537in}{1.332287in}}{\pgfqpoint{2.369437in}{1.335560in}}{\pgfqpoint{2.375261in}{1.341384in}}%
\pgfpathcurveto{\pgfqpoint{2.381085in}{1.347208in}}{\pgfqpoint{2.384357in}{1.355108in}}{\pgfqpoint{2.384357in}{1.363344in}}%
\pgfpathcurveto{\pgfqpoint{2.384357in}{1.371580in}}{\pgfqpoint{2.381085in}{1.379480in}}{\pgfqpoint{2.375261in}{1.385304in}}%
\pgfpathcurveto{\pgfqpoint{2.369437in}{1.391128in}}{\pgfqpoint{2.361537in}{1.394400in}}{\pgfqpoint{2.353301in}{1.394400in}}%
\pgfpathcurveto{\pgfqpoint{2.345064in}{1.394400in}}{\pgfqpoint{2.337164in}{1.391128in}}{\pgfqpoint{2.331340in}{1.385304in}}%
\pgfpathcurveto{\pgfqpoint{2.325516in}{1.379480in}}{\pgfqpoint{2.322244in}{1.371580in}}{\pgfqpoint{2.322244in}{1.363344in}}%
\pgfpathcurveto{\pgfqpoint{2.322244in}{1.355108in}}{\pgfqpoint{2.325516in}{1.347208in}}{\pgfqpoint{2.331340in}{1.341384in}}%
\pgfpathcurveto{\pgfqpoint{2.337164in}{1.335560in}}{\pgfqpoint{2.345064in}{1.332287in}}{\pgfqpoint{2.353301in}{1.332287in}}%
\pgfpathclose%
\pgfusepath{stroke,fill}%
\end{pgfscope}%
\begin{pgfscope}%
\pgfpathrectangle{\pgfqpoint{0.100000in}{0.212622in}}{\pgfqpoint{3.696000in}{3.696000in}}%
\pgfusepath{clip}%
\pgfsetbuttcap%
\pgfsetroundjoin%
\definecolor{currentfill}{rgb}{0.121569,0.466667,0.705882}%
\pgfsetfillcolor{currentfill}%
\pgfsetfillopacity{0.995823}%
\pgfsetlinewidth{1.003750pt}%
\definecolor{currentstroke}{rgb}{0.121569,0.466667,0.705882}%
\pgfsetstrokecolor{currentstroke}%
\pgfsetstrokeopacity{0.995823}%
\pgfsetdash{}{0pt}%
\pgfpathmoveto{\pgfqpoint{2.355736in}{1.330280in}}%
\pgfpathcurveto{\pgfqpoint{2.363972in}{1.330280in}}{\pgfqpoint{2.371872in}{1.333553in}}{\pgfqpoint{2.377696in}{1.339377in}}%
\pgfpathcurveto{\pgfqpoint{2.383520in}{1.345201in}}{\pgfqpoint{2.386792in}{1.353101in}}{\pgfqpoint{2.386792in}{1.361337in}}%
\pgfpathcurveto{\pgfqpoint{2.386792in}{1.369573in}}{\pgfqpoint{2.383520in}{1.377473in}}{\pgfqpoint{2.377696in}{1.383297in}}%
\pgfpathcurveto{\pgfqpoint{2.371872in}{1.389121in}}{\pgfqpoint{2.363972in}{1.392393in}}{\pgfqpoint{2.355736in}{1.392393in}}%
\pgfpathcurveto{\pgfqpoint{2.347499in}{1.392393in}}{\pgfqpoint{2.339599in}{1.389121in}}{\pgfqpoint{2.333775in}{1.383297in}}%
\pgfpathcurveto{\pgfqpoint{2.327951in}{1.377473in}}{\pgfqpoint{2.324679in}{1.369573in}}{\pgfqpoint{2.324679in}{1.361337in}}%
\pgfpathcurveto{\pgfqpoint{2.324679in}{1.353101in}}{\pgfqpoint{2.327951in}{1.345201in}}{\pgfqpoint{2.333775in}{1.339377in}}%
\pgfpathcurveto{\pgfqpoint{2.339599in}{1.333553in}}{\pgfqpoint{2.347499in}{1.330280in}}{\pgfqpoint{2.355736in}{1.330280in}}%
\pgfpathclose%
\pgfusepath{stroke,fill}%
\end{pgfscope}%
\begin{pgfscope}%
\pgfpathrectangle{\pgfqpoint{0.100000in}{0.212622in}}{\pgfqpoint{3.696000in}{3.696000in}}%
\pgfusepath{clip}%
\pgfsetbuttcap%
\pgfsetroundjoin%
\definecolor{currentfill}{rgb}{0.121569,0.466667,0.705882}%
\pgfsetfillcolor{currentfill}%
\pgfsetfillopacity{0.996360}%
\pgfsetlinewidth{1.003750pt}%
\definecolor{currentstroke}{rgb}{0.121569,0.466667,0.705882}%
\pgfsetstrokecolor{currentstroke}%
\pgfsetstrokeopacity{0.996360}%
\pgfsetdash{}{0pt}%
\pgfpathmoveto{\pgfqpoint{2.360146in}{1.326880in}}%
\pgfpathcurveto{\pgfqpoint{2.368382in}{1.326880in}}{\pgfqpoint{2.376282in}{1.330152in}}{\pgfqpoint{2.382106in}{1.335976in}}%
\pgfpathcurveto{\pgfqpoint{2.387930in}{1.341800in}}{\pgfqpoint{2.391202in}{1.349700in}}{\pgfqpoint{2.391202in}{1.357936in}}%
\pgfpathcurveto{\pgfqpoint{2.391202in}{1.366172in}}{\pgfqpoint{2.387930in}{1.374073in}}{\pgfqpoint{2.382106in}{1.379896in}}%
\pgfpathcurveto{\pgfqpoint{2.376282in}{1.385720in}}{\pgfqpoint{2.368382in}{1.388993in}}{\pgfqpoint{2.360146in}{1.388993in}}%
\pgfpathcurveto{\pgfqpoint{2.351909in}{1.388993in}}{\pgfqpoint{2.344009in}{1.385720in}}{\pgfqpoint{2.338185in}{1.379896in}}%
\pgfpathcurveto{\pgfqpoint{2.332361in}{1.374073in}}{\pgfqpoint{2.329089in}{1.366172in}}{\pgfqpoint{2.329089in}{1.357936in}}%
\pgfpathcurveto{\pgfqpoint{2.329089in}{1.349700in}}{\pgfqpoint{2.332361in}{1.341800in}}{\pgfqpoint{2.338185in}{1.335976in}}%
\pgfpathcurveto{\pgfqpoint{2.344009in}{1.330152in}}{\pgfqpoint{2.351909in}{1.326880in}}{\pgfqpoint{2.360146in}{1.326880in}}%
\pgfpathclose%
\pgfusepath{stroke,fill}%
\end{pgfscope}%
\begin{pgfscope}%
\pgfpathrectangle{\pgfqpoint{0.100000in}{0.212622in}}{\pgfqpoint{3.696000in}{3.696000in}}%
\pgfusepath{clip}%
\pgfsetbuttcap%
\pgfsetroundjoin%
\definecolor{currentfill}{rgb}{0.121569,0.466667,0.705882}%
\pgfsetfillcolor{currentfill}%
\pgfsetfillopacity{0.996402}%
\pgfsetlinewidth{1.003750pt}%
\definecolor{currentstroke}{rgb}{0.121569,0.466667,0.705882}%
\pgfsetstrokecolor{currentstroke}%
\pgfsetstrokeopacity{0.996402}%
\pgfsetdash{}{0pt}%
\pgfpathmoveto{\pgfqpoint{2.408486in}{1.301446in}}%
\pgfpathcurveto{\pgfqpoint{2.416722in}{1.301446in}}{\pgfqpoint{2.424622in}{1.304718in}}{\pgfqpoint{2.430446in}{1.310542in}}%
\pgfpathcurveto{\pgfqpoint{2.436270in}{1.316366in}}{\pgfqpoint{2.439542in}{1.324266in}}{\pgfqpoint{2.439542in}{1.332502in}}%
\pgfpathcurveto{\pgfqpoint{2.439542in}{1.340739in}}{\pgfqpoint{2.436270in}{1.348639in}}{\pgfqpoint{2.430446in}{1.354463in}}%
\pgfpathcurveto{\pgfqpoint{2.424622in}{1.360287in}}{\pgfqpoint{2.416722in}{1.363559in}}{\pgfqpoint{2.408486in}{1.363559in}}%
\pgfpathcurveto{\pgfqpoint{2.400250in}{1.363559in}}{\pgfqpoint{2.392349in}{1.360287in}}{\pgfqpoint{2.386526in}{1.354463in}}%
\pgfpathcurveto{\pgfqpoint{2.380702in}{1.348639in}}{\pgfqpoint{2.377429in}{1.340739in}}{\pgfqpoint{2.377429in}{1.332502in}}%
\pgfpathcurveto{\pgfqpoint{2.377429in}{1.324266in}}{\pgfqpoint{2.380702in}{1.316366in}}{\pgfqpoint{2.386526in}{1.310542in}}%
\pgfpathcurveto{\pgfqpoint{2.392349in}{1.304718in}}{\pgfqpoint{2.400250in}{1.301446in}}{\pgfqpoint{2.408486in}{1.301446in}}%
\pgfpathclose%
\pgfusepath{stroke,fill}%
\end{pgfscope}%
\begin{pgfscope}%
\pgfpathrectangle{\pgfqpoint{0.100000in}{0.212622in}}{\pgfqpoint{3.696000in}{3.696000in}}%
\pgfusepath{clip}%
\pgfsetbuttcap%
\pgfsetroundjoin%
\definecolor{currentfill}{rgb}{0.121569,0.466667,0.705882}%
\pgfsetfillcolor{currentfill}%
\pgfsetfillopacity{0.997089}%
\pgfsetlinewidth{1.003750pt}%
\definecolor{currentstroke}{rgb}{0.121569,0.466667,0.705882}%
\pgfsetstrokecolor{currentstroke}%
\pgfsetstrokeopacity{0.997089}%
\pgfsetdash{}{0pt}%
\pgfpathmoveto{\pgfqpoint{2.368341in}{1.319747in}}%
\pgfpathcurveto{\pgfqpoint{2.376577in}{1.319747in}}{\pgfqpoint{2.384477in}{1.323019in}}{\pgfqpoint{2.390301in}{1.328843in}}%
\pgfpathcurveto{\pgfqpoint{2.396125in}{1.334667in}}{\pgfqpoint{2.399397in}{1.342567in}}{\pgfqpoint{2.399397in}{1.350803in}}%
\pgfpathcurveto{\pgfqpoint{2.399397in}{1.359040in}}{\pgfqpoint{2.396125in}{1.366940in}}{\pgfqpoint{2.390301in}{1.372764in}}%
\pgfpathcurveto{\pgfqpoint{2.384477in}{1.378587in}}{\pgfqpoint{2.376577in}{1.381860in}}{\pgfqpoint{2.368341in}{1.381860in}}%
\pgfpathcurveto{\pgfqpoint{2.360105in}{1.381860in}}{\pgfqpoint{2.352204in}{1.378587in}}{\pgfqpoint{2.346381in}{1.372764in}}%
\pgfpathcurveto{\pgfqpoint{2.340557in}{1.366940in}}{\pgfqpoint{2.337284in}{1.359040in}}{\pgfqpoint{2.337284in}{1.350803in}}%
\pgfpathcurveto{\pgfqpoint{2.337284in}{1.342567in}}{\pgfqpoint{2.340557in}{1.334667in}}{\pgfqpoint{2.346381in}{1.328843in}}%
\pgfpathcurveto{\pgfqpoint{2.352204in}{1.323019in}}{\pgfqpoint{2.360105in}{1.319747in}}{\pgfqpoint{2.368341in}{1.319747in}}%
\pgfpathclose%
\pgfusepath{stroke,fill}%
\end{pgfscope}%
\begin{pgfscope}%
\pgfpathrectangle{\pgfqpoint{0.100000in}{0.212622in}}{\pgfqpoint{3.696000in}{3.696000in}}%
\pgfusepath{clip}%
\pgfsetbuttcap%
\pgfsetroundjoin%
\definecolor{currentfill}{rgb}{0.121569,0.466667,0.705882}%
\pgfsetfillcolor{currentfill}%
\pgfsetfillopacity{0.998141}%
\pgfsetlinewidth{1.003750pt}%
\definecolor{currentstroke}{rgb}{0.121569,0.466667,0.705882}%
\pgfsetstrokecolor{currentstroke}%
\pgfsetstrokeopacity{0.998141}%
\pgfsetdash{}{0pt}%
\pgfpathmoveto{\pgfqpoint{2.375710in}{1.314242in}}%
\pgfpathcurveto{\pgfqpoint{2.383946in}{1.314242in}}{\pgfqpoint{2.391846in}{1.317514in}}{\pgfqpoint{2.397670in}{1.323338in}}%
\pgfpathcurveto{\pgfqpoint{2.403494in}{1.329162in}}{\pgfqpoint{2.406767in}{1.337062in}}{\pgfqpoint{2.406767in}{1.345298in}}%
\pgfpathcurveto{\pgfqpoint{2.406767in}{1.353535in}}{\pgfqpoint{2.403494in}{1.361435in}}{\pgfqpoint{2.397670in}{1.367259in}}%
\pgfpathcurveto{\pgfqpoint{2.391846in}{1.373083in}}{\pgfqpoint{2.383946in}{1.376355in}}{\pgfqpoint{2.375710in}{1.376355in}}%
\pgfpathcurveto{\pgfqpoint{2.367474in}{1.376355in}}{\pgfqpoint{2.359574in}{1.373083in}}{\pgfqpoint{2.353750in}{1.367259in}}%
\pgfpathcurveto{\pgfqpoint{2.347926in}{1.361435in}}{\pgfqpoint{2.344654in}{1.353535in}}{\pgfqpoint{2.344654in}{1.345298in}}%
\pgfpathcurveto{\pgfqpoint{2.344654in}{1.337062in}}{\pgfqpoint{2.347926in}{1.329162in}}{\pgfqpoint{2.353750in}{1.323338in}}%
\pgfpathcurveto{\pgfqpoint{2.359574in}{1.317514in}}{\pgfqpoint{2.367474in}{1.314242in}}{\pgfqpoint{2.375710in}{1.314242in}}%
\pgfpathclose%
\pgfusepath{stroke,fill}%
\end{pgfscope}%
\begin{pgfscope}%
\pgfpathrectangle{\pgfqpoint{0.100000in}{0.212622in}}{\pgfqpoint{3.696000in}{3.696000in}}%
\pgfusepath{clip}%
\pgfsetbuttcap%
\pgfsetroundjoin%
\definecolor{currentfill}{rgb}{0.121569,0.466667,0.705882}%
\pgfsetfillcolor{currentfill}%
\pgfsetfillopacity{0.998213}%
\pgfsetlinewidth{1.003750pt}%
\definecolor{currentstroke}{rgb}{0.121569,0.466667,0.705882}%
\pgfsetstrokecolor{currentstroke}%
\pgfsetstrokeopacity{0.998213}%
\pgfsetdash{}{0pt}%
\pgfpathmoveto{\pgfqpoint{2.382776in}{1.305067in}}%
\pgfpathcurveto{\pgfqpoint{2.391012in}{1.305067in}}{\pgfqpoint{2.398912in}{1.308339in}}{\pgfqpoint{2.404736in}{1.314163in}}%
\pgfpathcurveto{\pgfqpoint{2.410560in}{1.319987in}}{\pgfqpoint{2.413832in}{1.327887in}}{\pgfqpoint{2.413832in}{1.336123in}}%
\pgfpathcurveto{\pgfqpoint{2.413832in}{1.344359in}}{\pgfqpoint{2.410560in}{1.352259in}}{\pgfqpoint{2.404736in}{1.358083in}}%
\pgfpathcurveto{\pgfqpoint{2.398912in}{1.363907in}}{\pgfqpoint{2.391012in}{1.367180in}}{\pgfqpoint{2.382776in}{1.367180in}}%
\pgfpathcurveto{\pgfqpoint{2.374539in}{1.367180in}}{\pgfqpoint{2.366639in}{1.363907in}}{\pgfqpoint{2.360816in}{1.358083in}}%
\pgfpathcurveto{\pgfqpoint{2.354992in}{1.352259in}}{\pgfqpoint{2.351719in}{1.344359in}}{\pgfqpoint{2.351719in}{1.336123in}}%
\pgfpathcurveto{\pgfqpoint{2.351719in}{1.327887in}}{\pgfqpoint{2.354992in}{1.319987in}}{\pgfqpoint{2.360816in}{1.314163in}}%
\pgfpathcurveto{\pgfqpoint{2.366639in}{1.308339in}}{\pgfqpoint{2.374539in}{1.305067in}}{\pgfqpoint{2.382776in}{1.305067in}}%
\pgfpathclose%
\pgfusepath{stroke,fill}%
\end{pgfscope}%
\begin{pgfscope}%
\pgfpathrectangle{\pgfqpoint{0.100000in}{0.212622in}}{\pgfqpoint{3.696000in}{3.696000in}}%
\pgfusepath{clip}%
\pgfsetbuttcap%
\pgfsetroundjoin%
\definecolor{currentfill}{rgb}{0.121569,0.466667,0.705882}%
\pgfsetfillcolor{currentfill}%
\pgfsetfillopacity{0.998265}%
\pgfsetlinewidth{1.003750pt}%
\definecolor{currentstroke}{rgb}{0.121569,0.466667,0.705882}%
\pgfsetstrokecolor{currentstroke}%
\pgfsetstrokeopacity{0.998265}%
\pgfsetdash{}{0pt}%
\pgfpathmoveto{\pgfqpoint{2.407468in}{1.299971in}}%
\pgfpathcurveto{\pgfqpoint{2.415704in}{1.299971in}}{\pgfqpoint{2.423604in}{1.303243in}}{\pgfqpoint{2.429428in}{1.309067in}}%
\pgfpathcurveto{\pgfqpoint{2.435252in}{1.314891in}}{\pgfqpoint{2.438525in}{1.322791in}}{\pgfqpoint{2.438525in}{1.331027in}}%
\pgfpathcurveto{\pgfqpoint{2.438525in}{1.339263in}}{\pgfqpoint{2.435252in}{1.347163in}}{\pgfqpoint{2.429428in}{1.352987in}}%
\pgfpathcurveto{\pgfqpoint{2.423604in}{1.358811in}}{\pgfqpoint{2.415704in}{1.362084in}}{\pgfqpoint{2.407468in}{1.362084in}}%
\pgfpathcurveto{\pgfqpoint{2.399232in}{1.362084in}}{\pgfqpoint{2.391332in}{1.358811in}}{\pgfqpoint{2.385508in}{1.352987in}}%
\pgfpathcurveto{\pgfqpoint{2.379684in}{1.347163in}}{\pgfqpoint{2.376412in}{1.339263in}}{\pgfqpoint{2.376412in}{1.331027in}}%
\pgfpathcurveto{\pgfqpoint{2.376412in}{1.322791in}}{\pgfqpoint{2.379684in}{1.314891in}}{\pgfqpoint{2.385508in}{1.309067in}}%
\pgfpathcurveto{\pgfqpoint{2.391332in}{1.303243in}}{\pgfqpoint{2.399232in}{1.299971in}}{\pgfqpoint{2.407468in}{1.299971in}}%
\pgfpathclose%
\pgfusepath{stroke,fill}%
\end{pgfscope}%
\begin{pgfscope}%
\pgfpathrectangle{\pgfqpoint{0.100000in}{0.212622in}}{\pgfqpoint{3.696000in}{3.696000in}}%
\pgfusepath{clip}%
\pgfsetbuttcap%
\pgfsetroundjoin%
\definecolor{currentfill}{rgb}{0.121569,0.466667,0.705882}%
\pgfsetfillcolor{currentfill}%
\pgfsetfillopacity{0.999002}%
\pgfsetlinewidth{1.003750pt}%
\definecolor{currentstroke}{rgb}{0.121569,0.466667,0.705882}%
\pgfsetstrokecolor{currentstroke}%
\pgfsetstrokeopacity{0.999002}%
\pgfsetdash{}{0pt}%
\pgfpathmoveto{\pgfqpoint{2.405766in}{1.298153in}}%
\pgfpathcurveto{\pgfqpoint{2.414003in}{1.298153in}}{\pgfqpoint{2.421903in}{1.301425in}}{\pgfqpoint{2.427727in}{1.307249in}}%
\pgfpathcurveto{\pgfqpoint{2.433550in}{1.313073in}}{\pgfqpoint{2.436823in}{1.320973in}}{\pgfqpoint{2.436823in}{1.329209in}}%
\pgfpathcurveto{\pgfqpoint{2.436823in}{1.337445in}}{\pgfqpoint{2.433550in}{1.345346in}}{\pgfqpoint{2.427727in}{1.351169in}}%
\pgfpathcurveto{\pgfqpoint{2.421903in}{1.356993in}}{\pgfqpoint{2.414003in}{1.360266in}}{\pgfqpoint{2.405766in}{1.360266in}}%
\pgfpathcurveto{\pgfqpoint{2.397530in}{1.360266in}}{\pgfqpoint{2.389630in}{1.356993in}}{\pgfqpoint{2.383806in}{1.351169in}}%
\pgfpathcurveto{\pgfqpoint{2.377982in}{1.345346in}}{\pgfqpoint{2.374710in}{1.337445in}}{\pgfqpoint{2.374710in}{1.329209in}}%
\pgfpathcurveto{\pgfqpoint{2.374710in}{1.320973in}}{\pgfqpoint{2.377982in}{1.313073in}}{\pgfqpoint{2.383806in}{1.307249in}}%
\pgfpathcurveto{\pgfqpoint{2.389630in}{1.301425in}}{\pgfqpoint{2.397530in}{1.298153in}}{\pgfqpoint{2.405766in}{1.298153in}}%
\pgfpathclose%
\pgfusepath{stroke,fill}%
\end{pgfscope}%
\begin{pgfscope}%
\pgfpathrectangle{\pgfqpoint{0.100000in}{0.212622in}}{\pgfqpoint{3.696000in}{3.696000in}}%
\pgfusepath{clip}%
\pgfsetbuttcap%
\pgfsetroundjoin%
\definecolor{currentfill}{rgb}{0.121569,0.466667,0.705882}%
\pgfsetfillcolor{currentfill}%
\pgfsetfillopacity{0.999486}%
\pgfsetlinewidth{1.003750pt}%
\definecolor{currentstroke}{rgb}{0.121569,0.466667,0.705882}%
\pgfsetstrokecolor{currentstroke}%
\pgfsetstrokeopacity{0.999486}%
\pgfsetdash{}{0pt}%
\pgfpathmoveto{\pgfqpoint{2.388983in}{1.304286in}}%
\pgfpathcurveto{\pgfqpoint{2.397219in}{1.304286in}}{\pgfqpoint{2.405119in}{1.307558in}}{\pgfqpoint{2.410943in}{1.313382in}}%
\pgfpathcurveto{\pgfqpoint{2.416767in}{1.319206in}}{\pgfqpoint{2.420039in}{1.327106in}}{\pgfqpoint{2.420039in}{1.335342in}}%
\pgfpathcurveto{\pgfqpoint{2.420039in}{1.343579in}}{\pgfqpoint{2.416767in}{1.351479in}}{\pgfqpoint{2.410943in}{1.357303in}}%
\pgfpathcurveto{\pgfqpoint{2.405119in}{1.363127in}}{\pgfqpoint{2.397219in}{1.366399in}}{\pgfqpoint{2.388983in}{1.366399in}}%
\pgfpathcurveto{\pgfqpoint{2.380746in}{1.366399in}}{\pgfqpoint{2.372846in}{1.363127in}}{\pgfqpoint{2.367022in}{1.357303in}}%
\pgfpathcurveto{\pgfqpoint{2.361198in}{1.351479in}}{\pgfqpoint{2.357926in}{1.343579in}}{\pgfqpoint{2.357926in}{1.335342in}}%
\pgfpathcurveto{\pgfqpoint{2.357926in}{1.327106in}}{\pgfqpoint{2.361198in}{1.319206in}}{\pgfqpoint{2.367022in}{1.313382in}}%
\pgfpathcurveto{\pgfqpoint{2.372846in}{1.307558in}}{\pgfqpoint{2.380746in}{1.304286in}}{\pgfqpoint{2.388983in}{1.304286in}}%
\pgfpathclose%
\pgfusepath{stroke,fill}%
\end{pgfscope}%
\begin{pgfscope}%
\pgfpathrectangle{\pgfqpoint{0.100000in}{0.212622in}}{\pgfqpoint{3.696000in}{3.696000in}}%
\pgfusepath{clip}%
\pgfsetbuttcap%
\pgfsetroundjoin%
\definecolor{currentfill}{rgb}{0.121569,0.466667,0.705882}%
\pgfsetfillcolor{currentfill}%
\pgfsetfillopacity{0.999533}%
\pgfsetlinewidth{1.003750pt}%
\definecolor{currentstroke}{rgb}{0.121569,0.466667,0.705882}%
\pgfsetstrokecolor{currentstroke}%
\pgfsetstrokeopacity{0.999533}%
\pgfsetdash{}{0pt}%
\pgfpathmoveto{\pgfqpoint{2.404577in}{1.298272in}}%
\pgfpathcurveto{\pgfqpoint{2.412814in}{1.298272in}}{\pgfqpoint{2.420714in}{1.301544in}}{\pgfqpoint{2.426538in}{1.307368in}}%
\pgfpathcurveto{\pgfqpoint{2.432362in}{1.313192in}}{\pgfqpoint{2.435634in}{1.321092in}}{\pgfqpoint{2.435634in}{1.329328in}}%
\pgfpathcurveto{\pgfqpoint{2.435634in}{1.337565in}}{\pgfqpoint{2.432362in}{1.345465in}}{\pgfqpoint{2.426538in}{1.351289in}}%
\pgfpathcurveto{\pgfqpoint{2.420714in}{1.357113in}}{\pgfqpoint{2.412814in}{1.360385in}}{\pgfqpoint{2.404577in}{1.360385in}}%
\pgfpathcurveto{\pgfqpoint{2.396341in}{1.360385in}}{\pgfqpoint{2.388441in}{1.357113in}}{\pgfqpoint{2.382617in}{1.351289in}}%
\pgfpathcurveto{\pgfqpoint{2.376793in}{1.345465in}}{\pgfqpoint{2.373521in}{1.337565in}}{\pgfqpoint{2.373521in}{1.329328in}}%
\pgfpathcurveto{\pgfqpoint{2.373521in}{1.321092in}}{\pgfqpoint{2.376793in}{1.313192in}}{\pgfqpoint{2.382617in}{1.307368in}}%
\pgfpathcurveto{\pgfqpoint{2.388441in}{1.301544in}}{\pgfqpoint{2.396341in}{1.298272in}}{\pgfqpoint{2.404577in}{1.298272in}}%
\pgfpathclose%
\pgfusepath{stroke,fill}%
\end{pgfscope}%
\begin{pgfscope}%
\pgfpathrectangle{\pgfqpoint{0.100000in}{0.212622in}}{\pgfqpoint{3.696000in}{3.696000in}}%
\pgfusepath{clip}%
\pgfsetbuttcap%
\pgfsetroundjoin%
\definecolor{currentfill}{rgb}{0.121569,0.466667,0.705882}%
\pgfsetfillcolor{currentfill}%
\pgfsetfillopacity{0.999636}%
\pgfsetlinewidth{1.003750pt}%
\definecolor{currentstroke}{rgb}{0.121569,0.466667,0.705882}%
\pgfsetstrokecolor{currentstroke}%
\pgfsetstrokeopacity{0.999636}%
\pgfsetdash{}{0pt}%
\pgfpathmoveto{\pgfqpoint{2.399048in}{1.296663in}}%
\pgfpathcurveto{\pgfqpoint{2.407284in}{1.296663in}}{\pgfqpoint{2.415184in}{1.299936in}}{\pgfqpoint{2.421008in}{1.305760in}}%
\pgfpathcurveto{\pgfqpoint{2.426832in}{1.311583in}}{\pgfqpoint{2.430104in}{1.319484in}}{\pgfqpoint{2.430104in}{1.327720in}}%
\pgfpathcurveto{\pgfqpoint{2.430104in}{1.335956in}}{\pgfqpoint{2.426832in}{1.343856in}}{\pgfqpoint{2.421008in}{1.349680in}}%
\pgfpathcurveto{\pgfqpoint{2.415184in}{1.355504in}}{\pgfqpoint{2.407284in}{1.358776in}}{\pgfqpoint{2.399048in}{1.358776in}}%
\pgfpathcurveto{\pgfqpoint{2.390812in}{1.358776in}}{\pgfqpoint{2.382912in}{1.355504in}}{\pgfqpoint{2.377088in}{1.349680in}}%
\pgfpathcurveto{\pgfqpoint{2.371264in}{1.343856in}}{\pgfqpoint{2.367991in}{1.335956in}}{\pgfqpoint{2.367991in}{1.327720in}}%
\pgfpathcurveto{\pgfqpoint{2.367991in}{1.319484in}}{\pgfqpoint{2.371264in}{1.311583in}}{\pgfqpoint{2.377088in}{1.305760in}}%
\pgfpathcurveto{\pgfqpoint{2.382912in}{1.299936in}}{\pgfqpoint{2.390812in}{1.296663in}}{\pgfqpoint{2.399048in}{1.296663in}}%
\pgfpathclose%
\pgfusepath{stroke,fill}%
\end{pgfscope}%
\begin{pgfscope}%
\pgfpathrectangle{\pgfqpoint{0.100000in}{0.212622in}}{\pgfqpoint{3.696000in}{3.696000in}}%
\pgfusepath{clip}%
\pgfsetbuttcap%
\pgfsetroundjoin%
\definecolor{currentfill}{rgb}{0.121569,0.466667,0.705882}%
\pgfsetfillcolor{currentfill}%
\pgfsetfillopacity{0.999771}%
\pgfsetlinewidth{1.003750pt}%
\definecolor{currentstroke}{rgb}{0.121569,0.466667,0.705882}%
\pgfsetstrokecolor{currentstroke}%
\pgfsetstrokeopacity{0.999771}%
\pgfsetdash{}{0pt}%
\pgfpathmoveto{\pgfqpoint{2.393174in}{1.301156in}}%
\pgfpathcurveto{\pgfqpoint{2.401410in}{1.301156in}}{\pgfqpoint{2.409311in}{1.304428in}}{\pgfqpoint{2.415134in}{1.310252in}}%
\pgfpathcurveto{\pgfqpoint{2.420958in}{1.316076in}}{\pgfqpoint{2.424231in}{1.323976in}}{\pgfqpoint{2.424231in}{1.332212in}}%
\pgfpathcurveto{\pgfqpoint{2.424231in}{1.340449in}}{\pgfqpoint{2.420958in}{1.348349in}}{\pgfqpoint{2.415134in}{1.354173in}}%
\pgfpathcurveto{\pgfqpoint{2.409311in}{1.359997in}}{\pgfqpoint{2.401410in}{1.363269in}}{\pgfqpoint{2.393174in}{1.363269in}}%
\pgfpathcurveto{\pgfqpoint{2.384938in}{1.363269in}}{\pgfqpoint{2.377038in}{1.359997in}}{\pgfqpoint{2.371214in}{1.354173in}}%
\pgfpathcurveto{\pgfqpoint{2.365390in}{1.348349in}}{\pgfqpoint{2.362118in}{1.340449in}}{\pgfqpoint{2.362118in}{1.332212in}}%
\pgfpathcurveto{\pgfqpoint{2.362118in}{1.323976in}}{\pgfqpoint{2.365390in}{1.316076in}}{\pgfqpoint{2.371214in}{1.310252in}}%
\pgfpathcurveto{\pgfqpoint{2.377038in}{1.304428in}}{\pgfqpoint{2.384938in}{1.301156in}}{\pgfqpoint{2.393174in}{1.301156in}}%
\pgfpathclose%
\pgfusepath{stroke,fill}%
\end{pgfscope}%
\begin{pgfscope}%
\pgfpathrectangle{\pgfqpoint{0.100000in}{0.212622in}}{\pgfqpoint{3.696000in}{3.696000in}}%
\pgfusepath{clip}%
\pgfsetbuttcap%
\pgfsetroundjoin%
\definecolor{currentfill}{rgb}{0.121569,0.466667,0.705882}%
\pgfsetfillcolor{currentfill}%
\pgfsetfillopacity{0.999925}%
\pgfsetlinewidth{1.003750pt}%
\definecolor{currentstroke}{rgb}{0.121569,0.466667,0.705882}%
\pgfsetstrokecolor{currentstroke}%
\pgfsetstrokeopacity{0.999925}%
\pgfsetdash{}{0pt}%
\pgfpathmoveto{\pgfqpoint{2.402369in}{1.297434in}}%
\pgfpathcurveto{\pgfqpoint{2.410605in}{1.297434in}}{\pgfqpoint{2.418505in}{1.300707in}}{\pgfqpoint{2.424329in}{1.306531in}}%
\pgfpathcurveto{\pgfqpoint{2.430153in}{1.312354in}}{\pgfqpoint{2.433425in}{1.320255in}}{\pgfqpoint{2.433425in}{1.328491in}}%
\pgfpathcurveto{\pgfqpoint{2.433425in}{1.336727in}}{\pgfqpoint{2.430153in}{1.344627in}}{\pgfqpoint{2.424329in}{1.350451in}}%
\pgfpathcurveto{\pgfqpoint{2.418505in}{1.356275in}}{\pgfqpoint{2.410605in}{1.359547in}}{\pgfqpoint{2.402369in}{1.359547in}}%
\pgfpathcurveto{\pgfqpoint{2.394133in}{1.359547in}}{\pgfqpoint{2.386233in}{1.356275in}}{\pgfqpoint{2.380409in}{1.350451in}}%
\pgfpathcurveto{\pgfqpoint{2.374585in}{1.344627in}}{\pgfqpoint{2.371312in}{1.336727in}}{\pgfqpoint{2.371312in}{1.328491in}}%
\pgfpathcurveto{\pgfqpoint{2.371312in}{1.320255in}}{\pgfqpoint{2.374585in}{1.312354in}}{\pgfqpoint{2.380409in}{1.306531in}}%
\pgfpathcurveto{\pgfqpoint{2.386233in}{1.300707in}}{\pgfqpoint{2.394133in}{1.297434in}}{\pgfqpoint{2.402369in}{1.297434in}}%
\pgfpathclose%
\pgfusepath{stroke,fill}%
\end{pgfscope}%
\begin{pgfscope}%
\pgfpathrectangle{\pgfqpoint{0.100000in}{0.212622in}}{\pgfqpoint{3.696000in}{3.696000in}}%
\pgfusepath{clip}%
\pgfsetbuttcap%
\pgfsetroundjoin%
\definecolor{currentfill}{rgb}{0.121569,0.466667,0.705882}%
\pgfsetfillcolor{currentfill}%
\pgfsetlinewidth{1.003750pt}%
\definecolor{currentstroke}{rgb}{0.121569,0.466667,0.705882}%
\pgfsetstrokecolor{currentstroke}%
\pgfsetdash{}{0pt}%
\pgfpathmoveto{\pgfqpoint{2.395083in}{1.300260in}}%
\pgfpathcurveto{\pgfqpoint{2.403320in}{1.300260in}}{\pgfqpoint{2.411220in}{1.303532in}}{\pgfqpoint{2.417044in}{1.309356in}}%
\pgfpathcurveto{\pgfqpoint{2.422868in}{1.315180in}}{\pgfqpoint{2.426140in}{1.323080in}}{\pgfqpoint{2.426140in}{1.331316in}}%
\pgfpathcurveto{\pgfqpoint{2.426140in}{1.339553in}}{\pgfqpoint{2.422868in}{1.347453in}}{\pgfqpoint{2.417044in}{1.353277in}}%
\pgfpathcurveto{\pgfqpoint{2.411220in}{1.359101in}}{\pgfqpoint{2.403320in}{1.362373in}}{\pgfqpoint{2.395083in}{1.362373in}}%
\pgfpathcurveto{\pgfqpoint{2.386847in}{1.362373in}}{\pgfqpoint{2.378947in}{1.359101in}}{\pgfqpoint{2.373123in}{1.353277in}}%
\pgfpathcurveto{\pgfqpoint{2.367299in}{1.347453in}}{\pgfqpoint{2.364027in}{1.339553in}}{\pgfqpoint{2.364027in}{1.331316in}}%
\pgfpathcurveto{\pgfqpoint{2.364027in}{1.323080in}}{\pgfqpoint{2.367299in}{1.315180in}}{\pgfqpoint{2.373123in}{1.309356in}}%
\pgfpathcurveto{\pgfqpoint{2.378947in}{1.303532in}}{\pgfqpoint{2.386847in}{1.300260in}}{\pgfqpoint{2.395083in}{1.300260in}}%
\pgfpathclose%
\pgfusepath{stroke,fill}%
\end{pgfscope}%
\begin{pgfscope}%
\definecolor{textcolor}{rgb}{0.000000,0.000000,0.000000}%
\pgfsetstrokecolor{textcolor}%
\pgfsetfillcolor{textcolor}%
\pgftext[x=1.948000in,y=3.991956in,,base]{\color{textcolor}\rmfamily\fontsize{12.000000}{14.400000}\selectfont Mahony}%
\end{pgfscope}%
\begin{pgfscope}%
\pgfsetbuttcap%
\pgfsetmiterjoin%
\definecolor{currentfill}{rgb}{1.000000,1.000000,1.000000}%
\pgfsetfillcolor{currentfill}%
\pgfsetfillopacity{0.800000}%
\pgfsetlinewidth{1.003750pt}%
\definecolor{currentstroke}{rgb}{0.800000,0.800000,0.800000}%
\pgfsetstrokecolor{currentstroke}%
\pgfsetstrokeopacity{0.800000}%
\pgfsetdash{}{0pt}%
\pgfpathmoveto{\pgfqpoint{2.104889in}{3.410289in}}%
\pgfpathlineto{\pgfqpoint{3.698778in}{3.410289in}}%
\pgfpathquadraticcurveto{\pgfqpoint{3.726556in}{3.410289in}}{\pgfqpoint{3.726556in}{3.438067in}}%
\pgfpathlineto{\pgfqpoint{3.726556in}{3.811400in}}%
\pgfpathquadraticcurveto{\pgfqpoint{3.726556in}{3.839178in}}{\pgfqpoint{3.698778in}{3.839178in}}%
\pgfpathlineto{\pgfqpoint{2.104889in}{3.839178in}}%
\pgfpathquadraticcurveto{\pgfqpoint{2.077111in}{3.839178in}}{\pgfqpoint{2.077111in}{3.811400in}}%
\pgfpathlineto{\pgfqpoint{2.077111in}{3.438067in}}%
\pgfpathquadraticcurveto{\pgfqpoint{2.077111in}{3.410289in}}{\pgfqpoint{2.104889in}{3.410289in}}%
\pgfpathclose%
\pgfusepath{stroke,fill}%
\end{pgfscope}%
\begin{pgfscope}%
\pgfsetrectcap%
\pgfsetroundjoin%
\pgfsetlinewidth{1.505625pt}%
\definecolor{currentstroke}{rgb}{0.121569,0.466667,0.705882}%
\pgfsetstrokecolor{currentstroke}%
\pgfsetdash{}{0pt}%
\pgfpathmoveto{\pgfqpoint{2.132667in}{3.735011in}}%
\pgfpathlineto{\pgfqpoint{2.410444in}{3.735011in}}%
\pgfusepath{stroke}%
\end{pgfscope}%
\begin{pgfscope}%
\definecolor{textcolor}{rgb}{0.000000,0.000000,0.000000}%
\pgfsetstrokecolor{textcolor}%
\pgfsetfillcolor{textcolor}%
\pgftext[x=2.521555in,y=3.686400in,left,base]{\color{textcolor}\rmfamily\fontsize{10.000000}{12.000000}\selectfont Ground truth}%
\end{pgfscope}%
\begin{pgfscope}%
\pgfsetbuttcap%
\pgfsetroundjoin%
\definecolor{currentfill}{rgb}{0.121569,0.466667,0.705882}%
\pgfsetfillcolor{currentfill}%
\pgfsetlinewidth{1.003750pt}%
\definecolor{currentstroke}{rgb}{0.121569,0.466667,0.705882}%
\pgfsetstrokecolor{currentstroke}%
\pgfsetdash{}{0pt}%
\pgfsys@defobject{currentmarker}{\pgfqpoint{-0.031056in}{-0.031056in}}{\pgfqpoint{0.031056in}{0.031056in}}{%
\pgfpathmoveto{\pgfqpoint{0.000000in}{-0.031056in}}%
\pgfpathcurveto{\pgfqpoint{0.008236in}{-0.031056in}}{\pgfqpoint{0.016136in}{-0.027784in}}{\pgfqpoint{0.021960in}{-0.021960in}}%
\pgfpathcurveto{\pgfqpoint{0.027784in}{-0.016136in}}{\pgfqpoint{0.031056in}{-0.008236in}}{\pgfqpoint{0.031056in}{0.000000in}}%
\pgfpathcurveto{\pgfqpoint{0.031056in}{0.008236in}}{\pgfqpoint{0.027784in}{0.016136in}}{\pgfqpoint{0.021960in}{0.021960in}}%
\pgfpathcurveto{\pgfqpoint{0.016136in}{0.027784in}}{\pgfqpoint{0.008236in}{0.031056in}}{\pgfqpoint{0.000000in}{0.031056in}}%
\pgfpathcurveto{\pgfqpoint{-0.008236in}{0.031056in}}{\pgfqpoint{-0.016136in}{0.027784in}}{\pgfqpoint{-0.021960in}{0.021960in}}%
\pgfpathcurveto{\pgfqpoint{-0.027784in}{0.016136in}}{\pgfqpoint{-0.031056in}{0.008236in}}{\pgfqpoint{-0.031056in}{0.000000in}}%
\pgfpathcurveto{\pgfqpoint{-0.031056in}{-0.008236in}}{\pgfqpoint{-0.027784in}{-0.016136in}}{\pgfqpoint{-0.021960in}{-0.021960in}}%
\pgfpathcurveto{\pgfqpoint{-0.016136in}{-0.027784in}}{\pgfqpoint{-0.008236in}{-0.031056in}}{\pgfqpoint{0.000000in}{-0.031056in}}%
\pgfpathclose%
\pgfusepath{stroke,fill}%
}%
\begin{pgfscope}%
\pgfsys@transformshift{2.271555in}{3.529248in}%
\pgfsys@useobject{currentmarker}{}%
\end{pgfscope}%
\end{pgfscope}%
\begin{pgfscope}%
\definecolor{textcolor}{rgb}{0.000000,0.000000,0.000000}%
\pgfsetstrokecolor{textcolor}%
\pgfsetfillcolor{textcolor}%
\pgftext[x=2.521555in,y=3.492789in,left,base]{\color{textcolor}\rmfamily\fontsize{10.000000}{12.000000}\selectfont Estimated position}%
\end{pgfscope}%
\end{pgfpicture}%
\makeatother%
\endgroup%
}
%         \caption{INS Hardware}
%         \label{fig:triangle16_2D}
%     \end{subfigure}
%     \begin{subfigure}{0.49\textwidth}
%         \centering
%         \resizebox{1\linewidth}{!}{%% Creator: Matplotlib, PGF backend
%%
%% To include the figure in your LaTeX document, write
%%   \input{<filename>.pgf}
%%
%% Make sure the required packages are loaded in your preamble
%%   \usepackage{pgf}
%%
%% and, on pdftex
%%   \usepackage[utf8]{inputenc}\DeclareUnicodeCharacter{2212}{-}
%%
%% or, on luatex and xetex
%%   \usepackage{unicode-math}
%%
%% Figures using additional raster images can only be included by \input if
%% they are in the same directory as the main LaTeX file. For loading figures
%% from other directories you can use the `import` package
%%   \usepackage{import}
%%
%% and then include the figures with
%%   \import{<path to file>}{<filename>.pgf}
%%
%% Matplotlib used the following preamble
%%   \usepackage{fontspec}
%%
\begingroup%
\makeatletter%
\begin{pgfpicture}%
\pgfpathrectangle{\pgfpointorigin}{\pgfqpoint{4.342355in}{4.207622in}}%
\pgfusepath{use as bounding box, clip}%
\begin{pgfscope}%
\pgfsetbuttcap%
\pgfsetmiterjoin%
\definecolor{currentfill}{rgb}{1.000000,1.000000,1.000000}%
\pgfsetfillcolor{currentfill}%
\pgfsetlinewidth{0.000000pt}%
\definecolor{currentstroke}{rgb}{1.000000,1.000000,1.000000}%
\pgfsetstrokecolor{currentstroke}%
\pgfsetdash{}{0pt}%
\pgfpathmoveto{\pgfqpoint{0.000000in}{-0.000000in}}%
\pgfpathlineto{\pgfqpoint{4.342355in}{-0.000000in}}%
\pgfpathlineto{\pgfqpoint{4.342355in}{4.207622in}}%
\pgfpathlineto{\pgfqpoint{0.000000in}{4.207622in}}%
\pgfpathclose%
\pgfusepath{fill}%
\end{pgfscope}%
\begin{pgfscope}%
\pgfsetbuttcap%
\pgfsetmiterjoin%
\definecolor{currentfill}{rgb}{1.000000,1.000000,1.000000}%
\pgfsetfillcolor{currentfill}%
\pgfsetlinewidth{0.000000pt}%
\definecolor{currentstroke}{rgb}{0.000000,0.000000,0.000000}%
\pgfsetstrokecolor{currentstroke}%
\pgfsetstrokeopacity{0.000000}%
\pgfsetdash{}{0pt}%
\pgfpathmoveto{\pgfqpoint{0.100000in}{0.212622in}}%
\pgfpathlineto{\pgfqpoint{3.796000in}{0.212622in}}%
\pgfpathlineto{\pgfqpoint{3.796000in}{3.908622in}}%
\pgfpathlineto{\pgfqpoint{0.100000in}{3.908622in}}%
\pgfpathclose%
\pgfusepath{fill}%
\end{pgfscope}%
\begin{pgfscope}%
\pgfsetbuttcap%
\pgfsetmiterjoin%
\definecolor{currentfill}{rgb}{0.950000,0.950000,0.950000}%
\pgfsetfillcolor{currentfill}%
\pgfsetfillopacity{0.500000}%
\pgfsetlinewidth{1.003750pt}%
\definecolor{currentstroke}{rgb}{0.950000,0.950000,0.950000}%
\pgfsetstrokecolor{currentstroke}%
\pgfsetstrokeopacity{0.500000}%
\pgfsetdash{}{0pt}%
\pgfpathmoveto{\pgfqpoint{0.379073in}{1.123938in}}%
\pgfpathlineto{\pgfqpoint{1.599613in}{2.147018in}}%
\pgfpathlineto{\pgfqpoint{1.582647in}{3.622484in}}%
\pgfpathlineto{\pgfqpoint{0.303698in}{2.689165in}}%
\pgfusepath{stroke,fill}%
\end{pgfscope}%
\begin{pgfscope}%
\pgfsetbuttcap%
\pgfsetmiterjoin%
\definecolor{currentfill}{rgb}{0.900000,0.900000,0.900000}%
\pgfsetfillcolor{currentfill}%
\pgfsetfillopacity{0.500000}%
\pgfsetlinewidth{1.003750pt}%
\definecolor{currentstroke}{rgb}{0.900000,0.900000,0.900000}%
\pgfsetstrokecolor{currentstroke}%
\pgfsetstrokeopacity{0.500000}%
\pgfsetdash{}{0pt}%
\pgfpathmoveto{\pgfqpoint{1.599613in}{2.147018in}}%
\pgfpathlineto{\pgfqpoint{3.558144in}{1.577751in}}%
\pgfpathlineto{\pgfqpoint{3.628038in}{3.104037in}}%
\pgfpathlineto{\pgfqpoint{1.582647in}{3.622484in}}%
\pgfusepath{stroke,fill}%
\end{pgfscope}%
\begin{pgfscope}%
\pgfsetbuttcap%
\pgfsetmiterjoin%
\definecolor{currentfill}{rgb}{0.925000,0.925000,0.925000}%
\pgfsetfillcolor{currentfill}%
\pgfsetfillopacity{0.500000}%
\pgfsetlinewidth{1.003750pt}%
\definecolor{currentstroke}{rgb}{0.925000,0.925000,0.925000}%
\pgfsetstrokecolor{currentstroke}%
\pgfsetstrokeopacity{0.500000}%
\pgfsetdash{}{0pt}%
\pgfpathmoveto{\pgfqpoint{0.379073in}{1.123938in}}%
\pgfpathlineto{\pgfqpoint{2.455212in}{0.445871in}}%
\pgfpathlineto{\pgfqpoint{3.558144in}{1.577751in}}%
\pgfpathlineto{\pgfqpoint{1.599613in}{2.147018in}}%
\pgfusepath{stroke,fill}%
\end{pgfscope}%
\begin{pgfscope}%
\pgfsetrectcap%
\pgfsetroundjoin%
\pgfsetlinewidth{0.803000pt}%
\definecolor{currentstroke}{rgb}{0.000000,0.000000,0.000000}%
\pgfsetstrokecolor{currentstroke}%
\pgfsetdash{}{0pt}%
\pgfpathmoveto{\pgfqpoint{0.379073in}{1.123938in}}%
\pgfpathlineto{\pgfqpoint{2.455212in}{0.445871in}}%
\pgfusepath{stroke}%
\end{pgfscope}%
\begin{pgfscope}%
\definecolor{textcolor}{rgb}{0.000000,0.000000,0.000000}%
\pgfsetstrokecolor{textcolor}%
\pgfsetfillcolor{textcolor}%
\pgftext[x=0.730374in, y=0.408886in, left, base,rotate=341.912962]{\color{textcolor}\rmfamily\fontsize{10.000000}{12.000000}\selectfont Position X [\(\displaystyle m\)]}%
\end{pgfscope}%
\begin{pgfscope}%
\pgfsetbuttcap%
\pgfsetroundjoin%
\pgfsetlinewidth{0.803000pt}%
\definecolor{currentstroke}{rgb}{0.690196,0.690196,0.690196}%
\pgfsetstrokecolor{currentstroke}%
\pgfsetdash{}{0pt}%
\pgfpathmoveto{\pgfqpoint{0.713978in}{1.014558in}}%
\pgfpathlineto{\pgfqpoint{1.916718in}{2.054849in}}%
\pgfpathlineto{\pgfqpoint{1.913229in}{3.538691in}}%
\pgfusepath{stroke}%
\end{pgfscope}%
\begin{pgfscope}%
\pgfsetbuttcap%
\pgfsetroundjoin%
\pgfsetlinewidth{0.803000pt}%
\definecolor{currentstroke}{rgb}{0.690196,0.690196,0.690196}%
\pgfsetstrokecolor{currentstroke}%
\pgfsetdash{}{0pt}%
\pgfpathmoveto{\pgfqpoint{1.138855in}{0.875793in}}%
\pgfpathlineto{\pgfqpoint{2.318363in}{1.938106in}}%
\pgfpathlineto{\pgfqpoint{2.332270in}{3.432476in}}%
\pgfusepath{stroke}%
\end{pgfscope}%
\begin{pgfscope}%
\pgfsetbuttcap%
\pgfsetroundjoin%
\pgfsetlinewidth{0.803000pt}%
\definecolor{currentstroke}{rgb}{0.690196,0.690196,0.690196}%
\pgfsetstrokecolor{currentstroke}%
\pgfsetdash{}{0pt}%
\pgfpathmoveto{\pgfqpoint{1.573172in}{0.733945in}}%
\pgfpathlineto{\pgfqpoint{2.728182in}{1.818988in}}%
\pgfpathlineto{\pgfqpoint{2.760212in}{3.324005in}}%
\pgfusepath{stroke}%
\end{pgfscope}%
\begin{pgfscope}%
\pgfsetbuttcap%
\pgfsetroundjoin%
\pgfsetlinewidth{0.803000pt}%
\definecolor{currentstroke}{rgb}{0.690196,0.690196,0.690196}%
\pgfsetstrokecolor{currentstroke}%
\pgfsetdash{}{0pt}%
\pgfpathmoveto{\pgfqpoint{2.017247in}{0.588910in}}%
\pgfpathlineto{\pgfqpoint{3.146426in}{1.697421in}}%
\pgfpathlineto{\pgfqpoint{3.197342in}{3.213205in}}%
\pgfusepath{stroke}%
\end{pgfscope}%
\begin{pgfscope}%
\pgfsetrectcap%
\pgfsetroundjoin%
\pgfsetlinewidth{0.803000pt}%
\definecolor{currentstroke}{rgb}{0.000000,0.000000,0.000000}%
\pgfsetstrokecolor{currentstroke}%
\pgfsetdash{}{0pt}%
\pgfpathmoveto{\pgfqpoint{0.724456in}{1.023621in}}%
\pgfpathlineto{\pgfqpoint{0.692977in}{0.996393in}}%
\pgfusepath{stroke}%
\end{pgfscope}%
\begin{pgfscope}%
\definecolor{textcolor}{rgb}{0.000000,0.000000,0.000000}%
\pgfsetstrokecolor{textcolor}%
\pgfsetfillcolor{textcolor}%
\pgftext[x=0.609619in,y=0.794907in,,top]{\color{textcolor}\rmfamily\fontsize{10.000000}{12.000000}\selectfont \(\displaystyle {0}\)}%
\end{pgfscope}%
\begin{pgfscope}%
\pgfsetrectcap%
\pgfsetroundjoin%
\pgfsetlinewidth{0.803000pt}%
\definecolor{currentstroke}{rgb}{0.000000,0.000000,0.000000}%
\pgfsetstrokecolor{currentstroke}%
\pgfsetdash{}{0pt}%
\pgfpathmoveto{\pgfqpoint{1.149140in}{0.885056in}}%
\pgfpathlineto{\pgfqpoint{1.118241in}{0.857227in}}%
\pgfusepath{stroke}%
\end{pgfscope}%
\begin{pgfscope}%
\definecolor{textcolor}{rgb}{0.000000,0.000000,0.000000}%
\pgfsetstrokecolor{textcolor}%
\pgfsetfillcolor{textcolor}%
\pgftext[x=1.034951in,y=0.653195in,,top]{\color{textcolor}\rmfamily\fontsize{10.000000}{12.000000}\selectfont \(\displaystyle {5}\)}%
\end{pgfscope}%
\begin{pgfscope}%
\pgfsetrectcap%
\pgfsetroundjoin%
\pgfsetlinewidth{0.803000pt}%
\definecolor{currentstroke}{rgb}{0.000000,0.000000,0.000000}%
\pgfsetstrokecolor{currentstroke}%
\pgfsetdash{}{0pt}%
\pgfpathmoveto{\pgfqpoint{1.583253in}{0.743415in}}%
\pgfpathlineto{\pgfqpoint{1.552967in}{0.714964in}}%
\pgfusepath{stroke}%
\end{pgfscope}%
\begin{pgfscope}%
\definecolor{textcolor}{rgb}{0.000000,0.000000,0.000000}%
\pgfsetstrokecolor{textcolor}%
\pgfsetfillcolor{textcolor}%
\pgftext[x=1.469769in,y=0.508323in,,top]{\color{textcolor}\rmfamily\fontsize{10.000000}{12.000000}\selectfont \(\displaystyle {10}\)}%
\end{pgfscope}%
\begin{pgfscope}%
\pgfsetrectcap%
\pgfsetroundjoin%
\pgfsetlinewidth{0.803000pt}%
\definecolor{currentstroke}{rgb}{0.000000,0.000000,0.000000}%
\pgfsetstrokecolor{currentstroke}%
\pgfsetdash{}{0pt}%
\pgfpathmoveto{\pgfqpoint{2.027111in}{0.598594in}}%
\pgfpathlineto{\pgfqpoint{1.997474in}{0.569500in}}%
\pgfusepath{stroke}%
\end{pgfscope}%
\begin{pgfscope}%
\definecolor{textcolor}{rgb}{0.000000,0.000000,0.000000}%
\pgfsetstrokecolor{textcolor}%
\pgfsetfillcolor{textcolor}%
\pgftext[x=1.914393in,y=0.360184in,,top]{\color{textcolor}\rmfamily\fontsize{10.000000}{12.000000}\selectfont \(\displaystyle {15}\)}%
\end{pgfscope}%
\begin{pgfscope}%
\pgfsetrectcap%
\pgfsetroundjoin%
\pgfsetlinewidth{0.803000pt}%
\definecolor{currentstroke}{rgb}{0.000000,0.000000,0.000000}%
\pgfsetstrokecolor{currentstroke}%
\pgfsetdash{}{0pt}%
\pgfpathmoveto{\pgfqpoint{3.558144in}{1.577751in}}%
\pgfpathlineto{\pgfqpoint{2.455212in}{0.445871in}}%
\pgfusepath{stroke}%
\end{pgfscope}%
\begin{pgfscope}%
\definecolor{textcolor}{rgb}{0.000000,0.000000,0.000000}%
\pgfsetstrokecolor{textcolor}%
\pgfsetfillcolor{textcolor}%
\pgftext[x=3.120747in, y=0.305657in, left, base,rotate=45.742112]{\color{textcolor}\rmfamily\fontsize{10.000000}{12.000000}\selectfont Position Y [\(\displaystyle m\)]}%
\end{pgfscope}%
\begin{pgfscope}%
\pgfsetbuttcap%
\pgfsetroundjoin%
\pgfsetlinewidth{0.803000pt}%
\definecolor{currentstroke}{rgb}{0.690196,0.690196,0.690196}%
\pgfsetstrokecolor{currentstroke}%
\pgfsetdash{}{0pt}%
\pgfpathmoveto{\pgfqpoint{0.520509in}{2.847384in}}%
\pgfpathlineto{\pgfqpoint{0.585332in}{1.296827in}}%
\pgfpathlineto{\pgfqpoint{2.642281in}{0.637849in}}%
\pgfusepath{stroke}%
\end{pgfscope}%
\begin{pgfscope}%
\pgfsetbuttcap%
\pgfsetroundjoin%
\pgfsetlinewidth{0.803000pt}%
\definecolor{currentstroke}{rgb}{0.690196,0.690196,0.690196}%
\pgfsetstrokecolor{currentstroke}%
\pgfsetdash{}{0pt}%
\pgfpathmoveto{\pgfqpoint{0.825985in}{3.070307in}}%
\pgfpathlineto{\pgfqpoint{0.876389in}{1.540797in}}%
\pgfpathlineto{\pgfqpoint{2.905784in}{0.908268in}}%
\pgfusepath{stroke}%
\end{pgfscope}%
\begin{pgfscope}%
\pgfsetbuttcap%
\pgfsetroundjoin%
\pgfsetlinewidth{0.803000pt}%
\definecolor{currentstroke}{rgb}{0.690196,0.690196,0.690196}%
\pgfsetstrokecolor{currentstroke}%
\pgfsetdash{}{0pt}%
\pgfpathmoveto{\pgfqpoint{1.119103in}{3.284211in}}%
\pgfpathlineto{\pgfqpoint{1.156168in}{1.775314in}}%
\pgfpathlineto{\pgfqpoint{3.158554in}{1.167673in}}%
\pgfusepath{stroke}%
\end{pgfscope}%
\begin{pgfscope}%
\pgfsetbuttcap%
\pgfsetroundjoin%
\pgfsetlinewidth{0.803000pt}%
\definecolor{currentstroke}{rgb}{0.690196,0.690196,0.690196}%
\pgfsetstrokecolor{currentstroke}%
\pgfsetdash{}{0pt}%
\pgfpathmoveto{\pgfqpoint{1.400598in}{3.489633in}}%
\pgfpathlineto{\pgfqpoint{1.425312in}{2.000915in}}%
\pgfpathlineto{\pgfqpoint{3.401233in}{1.416721in}}%
\pgfusepath{stroke}%
\end{pgfscope}%
\begin{pgfscope}%
\pgfsetrectcap%
\pgfsetroundjoin%
\pgfsetlinewidth{0.803000pt}%
\definecolor{currentstroke}{rgb}{0.000000,0.000000,0.000000}%
\pgfsetstrokecolor{currentstroke}%
\pgfsetdash{}{0pt}%
\pgfpathmoveto{\pgfqpoint{2.624954in}{0.643400in}}%
\pgfpathlineto{\pgfqpoint{2.676978in}{0.626733in}}%
\pgfusepath{stroke}%
\end{pgfscope}%
\begin{pgfscope}%
\definecolor{textcolor}{rgb}{0.000000,0.000000,0.000000}%
\pgfsetstrokecolor{textcolor}%
\pgfsetfillcolor{textcolor}%
\pgftext[x=2.819693in,y=0.453098in,,top]{\color{textcolor}\rmfamily\fontsize{10.000000}{12.000000}\selectfont \(\displaystyle {0}\)}%
\end{pgfscope}%
\begin{pgfscope}%
\pgfsetrectcap%
\pgfsetroundjoin%
\pgfsetlinewidth{0.803000pt}%
\definecolor{currentstroke}{rgb}{0.000000,0.000000,0.000000}%
\pgfsetstrokecolor{currentstroke}%
\pgfsetdash{}{0pt}%
\pgfpathmoveto{\pgfqpoint{2.888708in}{0.913591in}}%
\pgfpathlineto{\pgfqpoint{2.939980in}{0.897610in}}%
\pgfusepath{stroke}%
\end{pgfscope}%
\begin{pgfscope}%
\definecolor{textcolor}{rgb}{0.000000,0.000000,0.000000}%
\pgfsetstrokecolor{textcolor}%
\pgfsetfillcolor{textcolor}%
\pgftext[x=3.079659in,y=0.727517in,,top]{\color{textcolor}\rmfamily\fontsize{10.000000}{12.000000}\selectfont \(\displaystyle {5}\)}%
\end{pgfscope}%
\begin{pgfscope}%
\pgfsetrectcap%
\pgfsetroundjoin%
\pgfsetlinewidth{0.803000pt}%
\definecolor{currentstroke}{rgb}{0.000000,0.000000,0.000000}%
\pgfsetstrokecolor{currentstroke}%
\pgfsetdash{}{0pt}%
\pgfpathmoveto{\pgfqpoint{3.141722in}{1.172780in}}%
\pgfpathlineto{\pgfqpoint{3.192260in}{1.157444in}}%
\pgfusepath{stroke}%
\end{pgfscope}%
\begin{pgfscope}%
\definecolor{textcolor}{rgb}{0.000000,0.000000,0.000000}%
\pgfsetstrokecolor{textcolor}%
\pgfsetfillcolor{textcolor}%
\pgftext[x=3.329030in,y=0.990753in,,top]{\color{textcolor}\rmfamily\fontsize{10.000000}{12.000000}\selectfont \(\displaystyle {10}\)}%
\end{pgfscope}%
\begin{pgfscope}%
\pgfsetrectcap%
\pgfsetroundjoin%
\pgfsetlinewidth{0.803000pt}%
\definecolor{currentstroke}{rgb}{0.000000,0.000000,0.000000}%
\pgfsetstrokecolor{currentstroke}%
\pgfsetdash{}{0pt}%
\pgfpathmoveto{\pgfqpoint{3.384640in}{1.421627in}}%
\pgfpathlineto{\pgfqpoint{3.434461in}{1.406897in}}%
\pgfusepath{stroke}%
\end{pgfscope}%
\begin{pgfscope}%
\definecolor{textcolor}{rgb}{0.000000,0.000000,0.000000}%
\pgfsetstrokecolor{textcolor}%
\pgfsetfillcolor{textcolor}%
\pgftext[x=3.568440in,y=1.243474in,,top]{\color{textcolor}\rmfamily\fontsize{10.000000}{12.000000}\selectfont \(\displaystyle {15}\)}%
\end{pgfscope}%
\begin{pgfscope}%
\pgfsetrectcap%
\pgfsetroundjoin%
\pgfsetlinewidth{0.803000pt}%
\definecolor{currentstroke}{rgb}{0.000000,0.000000,0.000000}%
\pgfsetstrokecolor{currentstroke}%
\pgfsetdash{}{0pt}%
\pgfpathmoveto{\pgfqpoint{3.558144in}{1.577751in}}%
\pgfpathlineto{\pgfqpoint{3.628038in}{3.104037in}}%
\pgfusepath{stroke}%
\end{pgfscope}%
\begin{pgfscope}%
\definecolor{textcolor}{rgb}{0.000000,0.000000,0.000000}%
\pgfsetstrokecolor{textcolor}%
\pgfsetfillcolor{textcolor}%
\pgftext[x=4.167903in, y=1.963517in, left, base,rotate=87.378092]{\color{textcolor}\rmfamily\fontsize{10.000000}{12.000000}\selectfont Position Z [\(\displaystyle m\)]}%
\end{pgfscope}%
\begin{pgfscope}%
\pgfsetbuttcap%
\pgfsetroundjoin%
\pgfsetlinewidth{0.803000pt}%
\definecolor{currentstroke}{rgb}{0.690196,0.690196,0.690196}%
\pgfsetstrokecolor{currentstroke}%
\pgfsetdash{}{0pt}%
\pgfpathmoveto{\pgfqpoint{3.562420in}{1.671128in}}%
\pgfpathlineto{\pgfqpoint{1.598573in}{2.237465in}}%
\pgfpathlineto{\pgfqpoint{0.374469in}{1.219546in}}%
\pgfusepath{stroke}%
\end{pgfscope}%
\begin{pgfscope}%
\pgfsetbuttcap%
\pgfsetroundjoin%
\pgfsetlinewidth{0.803000pt}%
\definecolor{currentstroke}{rgb}{0.690196,0.690196,0.690196}%
\pgfsetstrokecolor{currentstroke}%
\pgfsetdash{}{0pt}%
\pgfpathmoveto{\pgfqpoint{3.575840in}{1.964176in}}%
\pgfpathlineto{\pgfqpoint{1.595311in}{2.521166in}}%
\pgfpathlineto{\pgfqpoint{0.360014in}{1.519723in}}%
\pgfusepath{stroke}%
\end{pgfscope}%
\begin{pgfscope}%
\pgfsetbuttcap%
\pgfsetroundjoin%
\pgfsetlinewidth{0.803000pt}%
\definecolor{currentstroke}{rgb}{0.690196,0.690196,0.690196}%
\pgfsetstrokecolor{currentstroke}%
\pgfsetdash{}{0pt}%
\pgfpathmoveto{\pgfqpoint{3.589492in}{2.262295in}}%
\pgfpathlineto{\pgfqpoint{1.591995in}{2.809540in}}%
\pgfpathlineto{\pgfqpoint{0.345299in}{1.825295in}}%
\pgfusepath{stroke}%
\end{pgfscope}%
\begin{pgfscope}%
\pgfsetbuttcap%
\pgfsetroundjoin%
\pgfsetlinewidth{0.803000pt}%
\definecolor{currentstroke}{rgb}{0.690196,0.690196,0.690196}%
\pgfsetstrokecolor{currentstroke}%
\pgfsetdash{}{0pt}%
\pgfpathmoveto{\pgfqpoint{3.603382in}{2.565618in}}%
\pgfpathlineto{\pgfqpoint{1.588624in}{3.102704in}}%
\pgfpathlineto{\pgfqpoint{0.330317in}{2.136406in}}%
\pgfusepath{stroke}%
\end{pgfscope}%
\begin{pgfscope}%
\pgfsetbuttcap%
\pgfsetroundjoin%
\pgfsetlinewidth{0.803000pt}%
\definecolor{currentstroke}{rgb}{0.690196,0.690196,0.690196}%
\pgfsetstrokecolor{currentstroke}%
\pgfsetdash{}{0pt}%
\pgfpathmoveto{\pgfqpoint{3.617516in}{2.874282in}}%
\pgfpathlineto{\pgfqpoint{1.585196in}{3.400777in}}%
\pgfpathlineto{\pgfqpoint{0.315060in}{2.453210in}}%
\pgfusepath{stroke}%
\end{pgfscope}%
\begin{pgfscope}%
\pgfsetrectcap%
\pgfsetroundjoin%
\pgfsetlinewidth{0.803000pt}%
\definecolor{currentstroke}{rgb}{0.000000,0.000000,0.000000}%
\pgfsetstrokecolor{currentstroke}%
\pgfsetdash{}{0pt}%
\pgfpathmoveto{\pgfqpoint{3.545936in}{1.675882in}}%
\pgfpathlineto{\pgfqpoint{3.595428in}{1.661609in}}%
\pgfusepath{stroke}%
\end{pgfscope}%
\begin{pgfscope}%
\definecolor{textcolor}{rgb}{0.000000,0.000000,0.000000}%
\pgfsetstrokecolor{textcolor}%
\pgfsetfillcolor{textcolor}%
\pgftext[x=3.816553in,y=1.707127in,,top]{\color{textcolor}\rmfamily\fontsize{10.000000}{12.000000}\selectfont \(\displaystyle {0}\)}%
\end{pgfscope}%
\begin{pgfscope}%
\pgfsetrectcap%
\pgfsetroundjoin%
\pgfsetlinewidth{0.803000pt}%
\definecolor{currentstroke}{rgb}{0.000000,0.000000,0.000000}%
\pgfsetstrokecolor{currentstroke}%
\pgfsetdash{}{0pt}%
\pgfpathmoveto{\pgfqpoint{3.559209in}{1.968853in}}%
\pgfpathlineto{\pgfqpoint{3.609142in}{1.954811in}}%
\pgfusepath{stroke}%
\end{pgfscope}%
\begin{pgfscope}%
\definecolor{textcolor}{rgb}{0.000000,0.000000,0.000000}%
\pgfsetstrokecolor{textcolor}%
\pgfsetfillcolor{textcolor}%
\pgftext[x=3.832106in,y=1.999594in,,top]{\color{textcolor}\rmfamily\fontsize{10.000000}{12.000000}\selectfont \(\displaystyle {1}\)}%
\end{pgfscope}%
\begin{pgfscope}%
\pgfsetrectcap%
\pgfsetroundjoin%
\pgfsetlinewidth{0.803000pt}%
\definecolor{currentstroke}{rgb}{0.000000,0.000000,0.000000}%
\pgfsetstrokecolor{currentstroke}%
\pgfsetdash{}{0pt}%
\pgfpathmoveto{\pgfqpoint{3.572711in}{2.266892in}}%
\pgfpathlineto{\pgfqpoint{3.623093in}{2.253090in}}%
\pgfusepath{stroke}%
\end{pgfscope}%
\begin{pgfscope}%
\definecolor{textcolor}{rgb}{0.000000,0.000000,0.000000}%
\pgfsetstrokecolor{textcolor}%
\pgfsetfillcolor{textcolor}%
\pgftext[x=3.847926in,y=2.297107in,,top]{\color{textcolor}\rmfamily\fontsize{10.000000}{12.000000}\selectfont \(\displaystyle {2}\)}%
\end{pgfscope}%
\begin{pgfscope}%
\pgfsetrectcap%
\pgfsetroundjoin%
\pgfsetlinewidth{0.803000pt}%
\definecolor{currentstroke}{rgb}{0.000000,0.000000,0.000000}%
\pgfsetstrokecolor{currentstroke}%
\pgfsetdash{}{0pt}%
\pgfpathmoveto{\pgfqpoint{3.586449in}{2.570132in}}%
\pgfpathlineto{\pgfqpoint{3.637288in}{2.556579in}}%
\pgfusepath{stroke}%
\end{pgfscope}%
\begin{pgfscope}%
\definecolor{textcolor}{rgb}{0.000000,0.000000,0.000000}%
\pgfsetstrokecolor{textcolor}%
\pgfsetfillcolor{textcolor}%
\pgftext[x=3.864022in,y=2.599797in,,top]{\color{textcolor}\rmfamily\fontsize{10.000000}{12.000000}\selectfont \(\displaystyle {3}\)}%
\end{pgfscope}%
\begin{pgfscope}%
\pgfsetrectcap%
\pgfsetroundjoin%
\pgfsetlinewidth{0.803000pt}%
\definecolor{currentstroke}{rgb}{0.000000,0.000000,0.000000}%
\pgfsetstrokecolor{currentstroke}%
\pgfsetdash{}{0pt}%
\pgfpathmoveto{\pgfqpoint{3.600429in}{2.878708in}}%
\pgfpathlineto{\pgfqpoint{3.651733in}{2.865418in}}%
\pgfusepath{stroke}%
\end{pgfscope}%
\begin{pgfscope}%
\definecolor{textcolor}{rgb}{0.000000,0.000000,0.000000}%
\pgfsetstrokecolor{textcolor}%
\pgfsetfillcolor{textcolor}%
\pgftext[x=3.880400in,y=2.907800in,,top]{\color{textcolor}\rmfamily\fontsize{10.000000}{12.000000}\selectfont \(\displaystyle {4}\)}%
\end{pgfscope}%
\begin{pgfscope}%
\pgfpathrectangle{\pgfqpoint{0.100000in}{0.212622in}}{\pgfqpoint{3.696000in}{3.696000in}}%
\pgfusepath{clip}%
\pgfsetrectcap%
\pgfsetroundjoin%
\pgfsetlinewidth{1.505625pt}%
\definecolor{currentstroke}{rgb}{0.121569,0.466667,0.705882}%
\pgfsetstrokecolor{currentstroke}%
\pgfsetdash{}{0pt}%
\pgfpathmoveto{\pgfqpoint{0.914292in}{1.285747in}}%
\pgfpathlineto{\pgfqpoint{1.796340in}{2.042663in}}%
\pgfpathlineto{\pgfqpoint{2.298639in}{0.845607in}}%
\pgfpathlineto{\pgfqpoint{0.914292in}{1.285747in}}%
\pgfusepath{stroke}%
\end{pgfscope}%
\begin{pgfscope}%
\pgfpathrectangle{\pgfqpoint{0.100000in}{0.212622in}}{\pgfqpoint{3.696000in}{3.696000in}}%
\pgfusepath{clip}%
\pgfsetrectcap%
\pgfsetroundjoin%
\pgfsetlinewidth{1.505625pt}%
\definecolor{currentstroke}{rgb}{1.000000,0.000000,0.000000}%
\pgfsetstrokecolor{currentstroke}%
\pgfsetdash{}{0pt}%
\pgfpathmoveto{\pgfqpoint{0.913757in}{1.285322in}}%
\pgfpathlineto{\pgfqpoint{0.914292in}{1.285747in}}%
\pgfusepath{stroke}%
\end{pgfscope}%
\begin{pgfscope}%
\pgfpathrectangle{\pgfqpoint{0.100000in}{0.212622in}}{\pgfqpoint{3.696000in}{3.696000in}}%
\pgfusepath{clip}%
\pgfsetrectcap%
\pgfsetroundjoin%
\pgfsetlinewidth{1.505625pt}%
\definecolor{currentstroke}{rgb}{1.000000,0.000000,0.000000}%
\pgfsetstrokecolor{currentstroke}%
\pgfsetdash{}{0pt}%
\pgfpathmoveto{\pgfqpoint{1.093551in}{1.527512in}}%
\pgfpathlineto{\pgfqpoint{0.914292in}{1.285747in}}%
\pgfusepath{stroke}%
\end{pgfscope}%
\begin{pgfscope}%
\pgfpathrectangle{\pgfqpoint{0.100000in}{0.212622in}}{\pgfqpoint{3.696000in}{3.696000in}}%
\pgfusepath{clip}%
\pgfsetrectcap%
\pgfsetroundjoin%
\pgfsetlinewidth{1.505625pt}%
\definecolor{currentstroke}{rgb}{1.000000,0.000000,0.000000}%
\pgfsetstrokecolor{currentstroke}%
\pgfsetdash{}{0pt}%
\pgfpathmoveto{\pgfqpoint{1.943659in}{2.109762in}}%
\pgfpathlineto{\pgfqpoint{1.796340in}{2.042663in}}%
\pgfusepath{stroke}%
\end{pgfscope}%
\begin{pgfscope}%
\pgfpathrectangle{\pgfqpoint{0.100000in}{0.212622in}}{\pgfqpoint{3.696000in}{3.696000in}}%
\pgfusepath{clip}%
\pgfsetrectcap%
\pgfsetroundjoin%
\pgfsetlinewidth{1.505625pt}%
\definecolor{currentstroke}{rgb}{1.000000,0.000000,0.000000}%
\pgfsetstrokecolor{currentstroke}%
\pgfsetdash{}{0pt}%
\pgfpathmoveto{\pgfqpoint{2.404987in}{1.351665in}}%
\pgfpathlineto{\pgfqpoint{2.298639in}{0.845607in}}%
\pgfusepath{stroke}%
\end{pgfscope}%
\begin{pgfscope}%
\pgfpathrectangle{\pgfqpoint{0.100000in}{0.212622in}}{\pgfqpoint{3.696000in}{3.696000in}}%
\pgfusepath{clip}%
\pgfsetrectcap%
\pgfsetroundjoin%
\pgfsetlinewidth{1.505625pt}%
\definecolor{currentstroke}{rgb}{1.000000,0.000000,0.000000}%
\pgfsetstrokecolor{currentstroke}%
\pgfsetdash{}{0pt}%
\pgfpathmoveto{\pgfqpoint{0.604784in}{2.669036in}}%
\pgfpathlineto{\pgfqpoint{0.914292in}{1.285747in}}%
\pgfusepath{stroke}%
\end{pgfscope}%
\begin{pgfscope}%
\pgfpathrectangle{\pgfqpoint{0.100000in}{0.212622in}}{\pgfqpoint{3.696000in}{3.696000in}}%
\pgfusepath{clip}%
\pgfsetbuttcap%
\pgfsetroundjoin%
\definecolor{currentfill}{rgb}{1.000000,0.498039,0.054902}%
\pgfsetfillcolor{currentfill}%
\pgfsetfillopacity{0.300000}%
\pgfsetlinewidth{1.003750pt}%
\definecolor{currentstroke}{rgb}{1.000000,0.498039,0.054902}%
\pgfsetstrokecolor{currentstroke}%
\pgfsetstrokeopacity{0.300000}%
\pgfsetdash{}{0pt}%
\pgfpathmoveto{\pgfqpoint{1.943659in}{2.078705in}}%
\pgfpathcurveto{\pgfqpoint{1.951895in}{2.078705in}}{\pgfqpoint{1.959795in}{2.081978in}}{\pgfqpoint{1.965619in}{2.087802in}}%
\pgfpathcurveto{\pgfqpoint{1.971443in}{2.093625in}}{\pgfqpoint{1.974715in}{2.101525in}}{\pgfqpoint{1.974715in}{2.109762in}}%
\pgfpathcurveto{\pgfqpoint{1.974715in}{2.117998in}}{\pgfqpoint{1.971443in}{2.125898in}}{\pgfqpoint{1.965619in}{2.131722in}}%
\pgfpathcurveto{\pgfqpoint{1.959795in}{2.137546in}}{\pgfqpoint{1.951895in}{2.140818in}}{\pgfqpoint{1.943659in}{2.140818in}}%
\pgfpathcurveto{\pgfqpoint{1.935422in}{2.140818in}}{\pgfqpoint{1.927522in}{2.137546in}}{\pgfqpoint{1.921698in}{2.131722in}}%
\pgfpathcurveto{\pgfqpoint{1.915874in}{2.125898in}}{\pgfqpoint{1.912602in}{2.117998in}}{\pgfqpoint{1.912602in}{2.109762in}}%
\pgfpathcurveto{\pgfqpoint{1.912602in}{2.101525in}}{\pgfqpoint{1.915874in}{2.093625in}}{\pgfqpoint{1.921698in}{2.087802in}}%
\pgfpathcurveto{\pgfqpoint{1.927522in}{2.081978in}}{\pgfqpoint{1.935422in}{2.078705in}}{\pgfqpoint{1.943659in}{2.078705in}}%
\pgfpathclose%
\pgfusepath{stroke,fill}%
\end{pgfscope}%
\begin{pgfscope}%
\pgfpathrectangle{\pgfqpoint{0.100000in}{0.212622in}}{\pgfqpoint{3.696000in}{3.696000in}}%
\pgfusepath{clip}%
\pgfsetbuttcap%
\pgfsetroundjoin%
\definecolor{currentfill}{rgb}{1.000000,0.498039,0.054902}%
\pgfsetfillcolor{currentfill}%
\pgfsetfillopacity{0.558330}%
\pgfsetlinewidth{1.003750pt}%
\definecolor{currentstroke}{rgb}{1.000000,0.498039,0.054902}%
\pgfsetstrokecolor{currentstroke}%
\pgfsetstrokeopacity{0.558330}%
\pgfsetdash{}{0pt}%
\pgfpathmoveto{\pgfqpoint{1.093551in}{1.496456in}}%
\pgfpathcurveto{\pgfqpoint{1.101787in}{1.496456in}}{\pgfqpoint{1.109687in}{1.499728in}}{\pgfqpoint{1.115511in}{1.505552in}}%
\pgfpathcurveto{\pgfqpoint{1.121335in}{1.511376in}}{\pgfqpoint{1.124607in}{1.519276in}}{\pgfqpoint{1.124607in}{1.527512in}}%
\pgfpathcurveto{\pgfqpoint{1.124607in}{1.535749in}}{\pgfqpoint{1.121335in}{1.543649in}}{\pgfqpoint{1.115511in}{1.549473in}}%
\pgfpathcurveto{\pgfqpoint{1.109687in}{1.555296in}}{\pgfqpoint{1.101787in}{1.558569in}}{\pgfqpoint{1.093551in}{1.558569in}}%
\pgfpathcurveto{\pgfqpoint{1.085315in}{1.558569in}}{\pgfqpoint{1.077414in}{1.555296in}}{\pgfqpoint{1.071591in}{1.549473in}}%
\pgfpathcurveto{\pgfqpoint{1.065767in}{1.543649in}}{\pgfqpoint{1.062494in}{1.535749in}}{\pgfqpoint{1.062494in}{1.527512in}}%
\pgfpathcurveto{\pgfqpoint{1.062494in}{1.519276in}}{\pgfqpoint{1.065767in}{1.511376in}}{\pgfqpoint{1.071591in}{1.505552in}}%
\pgfpathcurveto{\pgfqpoint{1.077414in}{1.499728in}}{\pgfqpoint{1.085315in}{1.496456in}}{\pgfqpoint{1.093551in}{1.496456in}}%
\pgfpathclose%
\pgfusepath{stroke,fill}%
\end{pgfscope}%
\begin{pgfscope}%
\pgfpathrectangle{\pgfqpoint{0.100000in}{0.212622in}}{\pgfqpoint{3.696000in}{3.696000in}}%
\pgfusepath{clip}%
\pgfsetbuttcap%
\pgfsetroundjoin%
\definecolor{currentfill}{rgb}{1.000000,0.498039,0.054902}%
\pgfsetfillcolor{currentfill}%
\pgfsetfillopacity{0.622372}%
\pgfsetlinewidth{1.003750pt}%
\definecolor{currentstroke}{rgb}{1.000000,0.498039,0.054902}%
\pgfsetstrokecolor{currentstroke}%
\pgfsetstrokeopacity{0.622372}%
\pgfsetdash{}{0pt}%
\pgfpathmoveto{\pgfqpoint{0.913757in}{1.254265in}}%
\pgfpathcurveto{\pgfqpoint{0.921993in}{1.254265in}}{\pgfqpoint{0.929893in}{1.257538in}}{\pgfqpoint{0.935717in}{1.263362in}}%
\pgfpathcurveto{\pgfqpoint{0.941541in}{1.269186in}}{\pgfqpoint{0.944813in}{1.277086in}}{\pgfqpoint{0.944813in}{1.285322in}}%
\pgfpathcurveto{\pgfqpoint{0.944813in}{1.293558in}}{\pgfqpoint{0.941541in}{1.301458in}}{\pgfqpoint{0.935717in}{1.307282in}}%
\pgfpathcurveto{\pgfqpoint{0.929893in}{1.313106in}}{\pgfqpoint{0.921993in}{1.316378in}}{\pgfqpoint{0.913757in}{1.316378in}}%
\pgfpathcurveto{\pgfqpoint{0.905520in}{1.316378in}}{\pgfqpoint{0.897620in}{1.313106in}}{\pgfqpoint{0.891796in}{1.307282in}}%
\pgfpathcurveto{\pgfqpoint{0.885972in}{1.301458in}}{\pgfqpoint{0.882700in}{1.293558in}}{\pgfqpoint{0.882700in}{1.285322in}}%
\pgfpathcurveto{\pgfqpoint{0.882700in}{1.277086in}}{\pgfqpoint{0.885972in}{1.269186in}}{\pgfqpoint{0.891796in}{1.263362in}}%
\pgfpathcurveto{\pgfqpoint{0.897620in}{1.257538in}}{\pgfqpoint{0.905520in}{1.254265in}}{\pgfqpoint{0.913757in}{1.254265in}}%
\pgfpathclose%
\pgfusepath{stroke,fill}%
\end{pgfscope}%
\begin{pgfscope}%
\pgfpathrectangle{\pgfqpoint{0.100000in}{0.212622in}}{\pgfqpoint{3.696000in}{3.696000in}}%
\pgfusepath{clip}%
\pgfsetbuttcap%
\pgfsetroundjoin%
\definecolor{currentfill}{rgb}{1.000000,0.498039,0.054902}%
\pgfsetfillcolor{currentfill}%
\pgfsetfillopacity{0.823184}%
\pgfsetlinewidth{1.003750pt}%
\definecolor{currentstroke}{rgb}{1.000000,0.498039,0.054902}%
\pgfsetstrokecolor{currentstroke}%
\pgfsetstrokeopacity{0.823184}%
\pgfsetdash{}{0pt}%
\pgfpathmoveto{\pgfqpoint{0.604784in}{2.637979in}}%
\pgfpathcurveto{\pgfqpoint{0.613021in}{2.637979in}}{\pgfqpoint{0.620921in}{2.641251in}}{\pgfqpoint{0.626745in}{2.647075in}}%
\pgfpathcurveto{\pgfqpoint{0.632569in}{2.652899in}}{\pgfqpoint{0.635841in}{2.660799in}}{\pgfqpoint{0.635841in}{2.669036in}}%
\pgfpathcurveto{\pgfqpoint{0.635841in}{2.677272in}}{\pgfqpoint{0.632569in}{2.685172in}}{\pgfqpoint{0.626745in}{2.690996in}}%
\pgfpathcurveto{\pgfqpoint{0.620921in}{2.696820in}}{\pgfqpoint{0.613021in}{2.700092in}}{\pgfqpoint{0.604784in}{2.700092in}}%
\pgfpathcurveto{\pgfqpoint{0.596548in}{2.700092in}}{\pgfqpoint{0.588648in}{2.696820in}}{\pgfqpoint{0.582824in}{2.690996in}}%
\pgfpathcurveto{\pgfqpoint{0.577000in}{2.685172in}}{\pgfqpoint{0.573728in}{2.677272in}}{\pgfqpoint{0.573728in}{2.669036in}}%
\pgfpathcurveto{\pgfqpoint{0.573728in}{2.660799in}}{\pgfqpoint{0.577000in}{2.652899in}}{\pgfqpoint{0.582824in}{2.647075in}}%
\pgfpathcurveto{\pgfqpoint{0.588648in}{2.641251in}}{\pgfqpoint{0.596548in}{2.637979in}}{\pgfqpoint{0.604784in}{2.637979in}}%
\pgfpathclose%
\pgfusepath{stroke,fill}%
\end{pgfscope}%
\begin{pgfscope}%
\pgfpathrectangle{\pgfqpoint{0.100000in}{0.212622in}}{\pgfqpoint{3.696000in}{3.696000in}}%
\pgfusepath{clip}%
\pgfsetbuttcap%
\pgfsetroundjoin%
\definecolor{currentfill}{rgb}{1.000000,0.498039,0.054902}%
\pgfsetfillcolor{currentfill}%
\pgfsetlinewidth{1.003750pt}%
\definecolor{currentstroke}{rgb}{1.000000,0.498039,0.054902}%
\pgfsetstrokecolor{currentstroke}%
\pgfsetdash{}{0pt}%
\pgfpathmoveto{\pgfqpoint{2.404987in}{1.320609in}}%
\pgfpathcurveto{\pgfqpoint{2.413223in}{1.320609in}}{\pgfqpoint{2.421123in}{1.323881in}}{\pgfqpoint{2.426947in}{1.329705in}}%
\pgfpathcurveto{\pgfqpoint{2.432771in}{1.335529in}}{\pgfqpoint{2.436043in}{1.343429in}}{\pgfqpoint{2.436043in}{1.351665in}}%
\pgfpathcurveto{\pgfqpoint{2.436043in}{1.359901in}}{\pgfqpoint{2.432771in}{1.367802in}}{\pgfqpoint{2.426947in}{1.373625in}}%
\pgfpathcurveto{\pgfqpoint{2.421123in}{1.379449in}}{\pgfqpoint{2.413223in}{1.382722in}}{\pgfqpoint{2.404987in}{1.382722in}}%
\pgfpathcurveto{\pgfqpoint{2.396751in}{1.382722in}}{\pgfqpoint{2.388851in}{1.379449in}}{\pgfqpoint{2.383027in}{1.373625in}}%
\pgfpathcurveto{\pgfqpoint{2.377203in}{1.367802in}}{\pgfqpoint{2.373930in}{1.359901in}}{\pgfqpoint{2.373930in}{1.351665in}}%
\pgfpathcurveto{\pgfqpoint{2.373930in}{1.343429in}}{\pgfqpoint{2.377203in}{1.335529in}}{\pgfqpoint{2.383027in}{1.329705in}}%
\pgfpathcurveto{\pgfqpoint{2.388851in}{1.323881in}}{\pgfqpoint{2.396751in}{1.320609in}}{\pgfqpoint{2.404987in}{1.320609in}}%
\pgfpathclose%
\pgfusepath{stroke,fill}%
\end{pgfscope}%
\begin{pgfscope}%
\definecolor{textcolor}{rgb}{0.000000,0.000000,0.000000}%
\pgfsetstrokecolor{textcolor}%
\pgfsetfillcolor{textcolor}%
\pgftext[x=1.948000in,y=3.991956in,,base]{\color{textcolor}\rmfamily\fontsize{12.000000}{14.400000}\selectfont Tilt}%
\end{pgfscope}%
\begin{pgfscope}%
\pgfpathrectangle{\pgfqpoint{0.100000in}{0.212622in}}{\pgfqpoint{3.696000in}{3.696000in}}%
\pgfusepath{clip}%
\pgfsetbuttcap%
\pgfsetroundjoin%
\definecolor{currentfill}{rgb}{0.121569,0.466667,0.705882}%
\pgfsetfillcolor{currentfill}%
\pgfsetfillopacity{0.300000}%
\pgfsetlinewidth{1.003750pt}%
\definecolor{currentstroke}{rgb}{0.121569,0.466667,0.705882}%
\pgfsetstrokecolor{currentstroke}%
\pgfsetstrokeopacity{0.300000}%
\pgfsetdash{}{0pt}%
\pgfpathmoveto{\pgfqpoint{1.943052in}{2.078993in}}%
\pgfpathcurveto{\pgfqpoint{1.951288in}{2.078993in}}{\pgfqpoint{1.959188in}{2.082266in}}{\pgfqpoint{1.965012in}{2.088090in}}%
\pgfpathcurveto{\pgfqpoint{1.970836in}{2.093914in}}{\pgfqpoint{1.974108in}{2.101814in}}{\pgfqpoint{1.974108in}{2.110050in}}%
\pgfpathcurveto{\pgfqpoint{1.974108in}{2.118286in}}{\pgfqpoint{1.970836in}{2.126186in}}{\pgfqpoint{1.965012in}{2.132010in}}%
\pgfpathcurveto{\pgfqpoint{1.959188in}{2.137834in}}{\pgfqpoint{1.951288in}{2.141106in}}{\pgfqpoint{1.943052in}{2.141106in}}%
\pgfpathcurveto{\pgfqpoint{1.934816in}{2.141106in}}{\pgfqpoint{1.926916in}{2.137834in}}{\pgfqpoint{1.921092in}{2.132010in}}%
\pgfpathcurveto{\pgfqpoint{1.915268in}{2.126186in}}{\pgfqpoint{1.911995in}{2.118286in}}{\pgfqpoint{1.911995in}{2.110050in}}%
\pgfpathcurveto{\pgfqpoint{1.911995in}{2.101814in}}{\pgfqpoint{1.915268in}{2.093914in}}{\pgfqpoint{1.921092in}{2.088090in}}%
\pgfpathcurveto{\pgfqpoint{1.926916in}{2.082266in}}{\pgfqpoint{1.934816in}{2.078993in}}{\pgfqpoint{1.943052in}{2.078993in}}%
\pgfpathclose%
\pgfusepath{stroke,fill}%
\end{pgfscope}%
\begin{pgfscope}%
\pgfpathrectangle{\pgfqpoint{0.100000in}{0.212622in}}{\pgfqpoint{3.696000in}{3.696000in}}%
\pgfusepath{clip}%
\pgfsetbuttcap%
\pgfsetroundjoin%
\definecolor{currentfill}{rgb}{0.121569,0.466667,0.705882}%
\pgfsetfillcolor{currentfill}%
\pgfsetfillopacity{0.300002}%
\pgfsetlinewidth{1.003750pt}%
\definecolor{currentstroke}{rgb}{0.121569,0.466667,0.705882}%
\pgfsetstrokecolor{currentstroke}%
\pgfsetstrokeopacity{0.300002}%
\pgfsetdash{}{0pt}%
\pgfpathmoveto{\pgfqpoint{1.943455in}{2.078852in}}%
\pgfpathcurveto{\pgfqpoint{1.951691in}{2.078852in}}{\pgfqpoint{1.959591in}{2.082125in}}{\pgfqpoint{1.965415in}{2.087949in}}%
\pgfpathcurveto{\pgfqpoint{1.971239in}{2.093772in}}{\pgfqpoint{1.974511in}{2.101673in}}{\pgfqpoint{1.974511in}{2.109909in}}%
\pgfpathcurveto{\pgfqpoint{1.974511in}{2.118145in}}{\pgfqpoint{1.971239in}{2.126045in}}{\pgfqpoint{1.965415in}{2.131869in}}%
\pgfpathcurveto{\pgfqpoint{1.959591in}{2.137693in}}{\pgfqpoint{1.951691in}{2.140965in}}{\pgfqpoint{1.943455in}{2.140965in}}%
\pgfpathcurveto{\pgfqpoint{1.935218in}{2.140965in}}{\pgfqpoint{1.927318in}{2.137693in}}{\pgfqpoint{1.921494in}{2.131869in}}%
\pgfpathcurveto{\pgfqpoint{1.915670in}{2.126045in}}{\pgfqpoint{1.912398in}{2.118145in}}{\pgfqpoint{1.912398in}{2.109909in}}%
\pgfpathcurveto{\pgfqpoint{1.912398in}{2.101673in}}{\pgfqpoint{1.915670in}{2.093772in}}{\pgfqpoint{1.921494in}{2.087949in}}%
\pgfpathcurveto{\pgfqpoint{1.927318in}{2.082125in}}{\pgfqpoint{1.935218in}{2.078852in}}{\pgfqpoint{1.943455in}{2.078852in}}%
\pgfpathclose%
\pgfusepath{stroke,fill}%
\end{pgfscope}%
\begin{pgfscope}%
\pgfpathrectangle{\pgfqpoint{0.100000in}{0.212622in}}{\pgfqpoint{3.696000in}{3.696000in}}%
\pgfusepath{clip}%
\pgfsetbuttcap%
\pgfsetroundjoin%
\definecolor{currentfill}{rgb}{0.121569,0.466667,0.705882}%
\pgfsetfillcolor{currentfill}%
\pgfsetfillopacity{0.300007}%
\pgfsetlinewidth{1.003750pt}%
\definecolor{currentstroke}{rgb}{0.121569,0.466667,0.705882}%
\pgfsetstrokecolor{currentstroke}%
\pgfsetstrokeopacity{0.300007}%
\pgfsetdash{}{0pt}%
\pgfpathmoveto{\pgfqpoint{1.943659in}{2.078705in}}%
\pgfpathcurveto{\pgfqpoint{1.951895in}{2.078705in}}{\pgfqpoint{1.959795in}{2.081978in}}{\pgfqpoint{1.965619in}{2.087802in}}%
\pgfpathcurveto{\pgfqpoint{1.971443in}{2.093625in}}{\pgfqpoint{1.974715in}{2.101525in}}{\pgfqpoint{1.974715in}{2.109762in}}%
\pgfpathcurveto{\pgfqpoint{1.974715in}{2.117998in}}{\pgfqpoint{1.971443in}{2.125898in}}{\pgfqpoint{1.965619in}{2.131722in}}%
\pgfpathcurveto{\pgfqpoint{1.959795in}{2.137546in}}{\pgfqpoint{1.951895in}{2.140818in}}{\pgfqpoint{1.943659in}{2.140818in}}%
\pgfpathcurveto{\pgfqpoint{1.935422in}{2.140818in}}{\pgfqpoint{1.927522in}{2.137546in}}{\pgfqpoint{1.921698in}{2.131722in}}%
\pgfpathcurveto{\pgfqpoint{1.915874in}{2.125898in}}{\pgfqpoint{1.912602in}{2.117998in}}{\pgfqpoint{1.912602in}{2.109762in}}%
\pgfpathcurveto{\pgfqpoint{1.912602in}{2.101525in}}{\pgfqpoint{1.915874in}{2.093625in}}{\pgfqpoint{1.921698in}{2.087802in}}%
\pgfpathcurveto{\pgfqpoint{1.927522in}{2.081978in}}{\pgfqpoint{1.935422in}{2.078705in}}{\pgfqpoint{1.943659in}{2.078705in}}%
\pgfpathclose%
\pgfusepath{stroke,fill}%
\end{pgfscope}%
\begin{pgfscope}%
\pgfpathrectangle{\pgfqpoint{0.100000in}{0.212622in}}{\pgfqpoint{3.696000in}{3.696000in}}%
\pgfusepath{clip}%
\pgfsetbuttcap%
\pgfsetroundjoin%
\definecolor{currentfill}{rgb}{0.121569,0.466667,0.705882}%
\pgfsetfillcolor{currentfill}%
\pgfsetfillopacity{0.300023}%
\pgfsetlinewidth{1.003750pt}%
\definecolor{currentstroke}{rgb}{0.121569,0.466667,0.705882}%
\pgfsetstrokecolor{currentstroke}%
\pgfsetstrokeopacity{0.300023}%
\pgfsetdash{}{0pt}%
\pgfpathmoveto{\pgfqpoint{1.943754in}{2.078664in}}%
\pgfpathcurveto{\pgfqpoint{1.951990in}{2.078664in}}{\pgfqpoint{1.959890in}{2.081937in}}{\pgfqpoint{1.965714in}{2.087761in}}%
\pgfpathcurveto{\pgfqpoint{1.971538in}{2.093584in}}{\pgfqpoint{1.974810in}{2.101485in}}{\pgfqpoint{1.974810in}{2.109721in}}%
\pgfpathcurveto{\pgfqpoint{1.974810in}{2.117957in}}{\pgfqpoint{1.971538in}{2.125857in}}{\pgfqpoint{1.965714in}{2.131681in}}%
\pgfpathcurveto{\pgfqpoint{1.959890in}{2.137505in}}{\pgfqpoint{1.951990in}{2.140777in}}{\pgfqpoint{1.943754in}{2.140777in}}%
\pgfpathcurveto{\pgfqpoint{1.935517in}{2.140777in}}{\pgfqpoint{1.927617in}{2.137505in}}{\pgfqpoint{1.921793in}{2.131681in}}%
\pgfpathcurveto{\pgfqpoint{1.915969in}{2.125857in}}{\pgfqpoint{1.912697in}{2.117957in}}{\pgfqpoint{1.912697in}{2.109721in}}%
\pgfpathcurveto{\pgfqpoint{1.912697in}{2.101485in}}{\pgfqpoint{1.915969in}{2.093584in}}{\pgfqpoint{1.921793in}{2.087761in}}%
\pgfpathcurveto{\pgfqpoint{1.927617in}{2.081937in}}{\pgfqpoint{1.935517in}{2.078664in}}{\pgfqpoint{1.943754in}{2.078664in}}%
\pgfpathclose%
\pgfusepath{stroke,fill}%
\end{pgfscope}%
\begin{pgfscope}%
\pgfpathrectangle{\pgfqpoint{0.100000in}{0.212622in}}{\pgfqpoint{3.696000in}{3.696000in}}%
\pgfusepath{clip}%
\pgfsetbuttcap%
\pgfsetroundjoin%
\definecolor{currentfill}{rgb}{0.121569,0.466667,0.705882}%
\pgfsetfillcolor{currentfill}%
\pgfsetfillopacity{0.300039}%
\pgfsetlinewidth{1.003750pt}%
\definecolor{currentstroke}{rgb}{0.121569,0.466667,0.705882}%
\pgfsetstrokecolor{currentstroke}%
\pgfsetstrokeopacity{0.300039}%
\pgfsetdash{}{0pt}%
\pgfpathmoveto{\pgfqpoint{1.943796in}{2.078664in}}%
\pgfpathcurveto{\pgfqpoint{1.952033in}{2.078664in}}{\pgfqpoint{1.959933in}{2.081937in}}{\pgfqpoint{1.965757in}{2.087761in}}%
\pgfpathcurveto{\pgfqpoint{1.971581in}{2.093584in}}{\pgfqpoint{1.974853in}{2.101485in}}{\pgfqpoint{1.974853in}{2.109721in}}%
\pgfpathcurveto{\pgfqpoint{1.974853in}{2.117957in}}{\pgfqpoint{1.971581in}{2.125857in}}{\pgfqpoint{1.965757in}{2.131681in}}%
\pgfpathcurveto{\pgfqpoint{1.959933in}{2.137505in}}{\pgfqpoint{1.952033in}{2.140777in}}{\pgfqpoint{1.943796in}{2.140777in}}%
\pgfpathcurveto{\pgfqpoint{1.935560in}{2.140777in}}{\pgfqpoint{1.927660in}{2.137505in}}{\pgfqpoint{1.921836in}{2.131681in}}%
\pgfpathcurveto{\pgfqpoint{1.916012in}{2.125857in}}{\pgfqpoint{1.912740in}{2.117957in}}{\pgfqpoint{1.912740in}{2.109721in}}%
\pgfpathcurveto{\pgfqpoint{1.912740in}{2.101485in}}{\pgfqpoint{1.916012in}{2.093584in}}{\pgfqpoint{1.921836in}{2.087761in}}%
\pgfpathcurveto{\pgfqpoint{1.927660in}{2.081937in}}{\pgfqpoint{1.935560in}{2.078664in}}{\pgfqpoint{1.943796in}{2.078664in}}%
\pgfpathclose%
\pgfusepath{stroke,fill}%
\end{pgfscope}%
\begin{pgfscope}%
\pgfpathrectangle{\pgfqpoint{0.100000in}{0.212622in}}{\pgfqpoint{3.696000in}{3.696000in}}%
\pgfusepath{clip}%
\pgfsetbuttcap%
\pgfsetroundjoin%
\definecolor{currentfill}{rgb}{0.121569,0.466667,0.705882}%
\pgfsetfillcolor{currentfill}%
\pgfsetfillopacity{0.300047}%
\pgfsetlinewidth{1.003750pt}%
\definecolor{currentstroke}{rgb}{0.121569,0.466667,0.705882}%
\pgfsetstrokecolor{currentstroke}%
\pgfsetstrokeopacity{0.300047}%
\pgfsetdash{}{0pt}%
\pgfpathmoveto{\pgfqpoint{1.943813in}{2.078655in}}%
\pgfpathcurveto{\pgfqpoint{1.952049in}{2.078655in}}{\pgfqpoint{1.959949in}{2.081927in}}{\pgfqpoint{1.965773in}{2.087751in}}%
\pgfpathcurveto{\pgfqpoint{1.971597in}{2.093575in}}{\pgfqpoint{1.974869in}{2.101475in}}{\pgfqpoint{1.974869in}{2.109711in}}%
\pgfpathcurveto{\pgfqpoint{1.974869in}{2.117948in}}{\pgfqpoint{1.971597in}{2.125848in}}{\pgfqpoint{1.965773in}{2.131672in}}%
\pgfpathcurveto{\pgfqpoint{1.959949in}{2.137496in}}{\pgfqpoint{1.952049in}{2.140768in}}{\pgfqpoint{1.943813in}{2.140768in}}%
\pgfpathcurveto{\pgfqpoint{1.935576in}{2.140768in}}{\pgfqpoint{1.927676in}{2.137496in}}{\pgfqpoint{1.921852in}{2.131672in}}%
\pgfpathcurveto{\pgfqpoint{1.916028in}{2.125848in}}{\pgfqpoint{1.912756in}{2.117948in}}{\pgfqpoint{1.912756in}{2.109711in}}%
\pgfpathcurveto{\pgfqpoint{1.912756in}{2.101475in}}{\pgfqpoint{1.916028in}{2.093575in}}{\pgfqpoint{1.921852in}{2.087751in}}%
\pgfpathcurveto{\pgfqpoint{1.927676in}{2.081927in}}{\pgfqpoint{1.935576in}{2.078655in}}{\pgfqpoint{1.943813in}{2.078655in}}%
\pgfpathclose%
\pgfusepath{stroke,fill}%
\end{pgfscope}%
\begin{pgfscope}%
\pgfpathrectangle{\pgfqpoint{0.100000in}{0.212622in}}{\pgfqpoint{3.696000in}{3.696000in}}%
\pgfusepath{clip}%
\pgfsetbuttcap%
\pgfsetroundjoin%
\definecolor{currentfill}{rgb}{0.121569,0.466667,0.705882}%
\pgfsetfillcolor{currentfill}%
\pgfsetfillopacity{0.300073}%
\pgfsetlinewidth{1.003750pt}%
\definecolor{currentstroke}{rgb}{0.121569,0.466667,0.705882}%
\pgfsetstrokecolor{currentstroke}%
\pgfsetstrokeopacity{0.300073}%
\pgfsetdash{}{0pt}%
\pgfpathmoveto{\pgfqpoint{1.942299in}{2.079206in}}%
\pgfpathcurveto{\pgfqpoint{1.950536in}{2.079206in}}{\pgfqpoint{1.958436in}{2.082479in}}{\pgfqpoint{1.964260in}{2.088303in}}%
\pgfpathcurveto{\pgfqpoint{1.970083in}{2.094126in}}{\pgfqpoint{1.973356in}{2.102026in}}{\pgfqpoint{1.973356in}{2.110263in}}%
\pgfpathcurveto{\pgfqpoint{1.973356in}{2.118499in}}{\pgfqpoint{1.970083in}{2.126399in}}{\pgfqpoint{1.964260in}{2.132223in}}%
\pgfpathcurveto{\pgfqpoint{1.958436in}{2.138047in}}{\pgfqpoint{1.950536in}{2.141319in}}{\pgfqpoint{1.942299in}{2.141319in}}%
\pgfpathcurveto{\pgfqpoint{1.934063in}{2.141319in}}{\pgfqpoint{1.926163in}{2.138047in}}{\pgfqpoint{1.920339in}{2.132223in}}%
\pgfpathcurveto{\pgfqpoint{1.914515in}{2.126399in}}{\pgfqpoint{1.911243in}{2.118499in}}{\pgfqpoint{1.911243in}{2.110263in}}%
\pgfpathcurveto{\pgfqpoint{1.911243in}{2.102026in}}{\pgfqpoint{1.914515in}{2.094126in}}{\pgfqpoint{1.920339in}{2.088303in}}%
\pgfpathcurveto{\pgfqpoint{1.926163in}{2.082479in}}{\pgfqpoint{1.934063in}{2.079206in}}{\pgfqpoint{1.942299in}{2.079206in}}%
\pgfpathclose%
\pgfusepath{stroke,fill}%
\end{pgfscope}%
\begin{pgfscope}%
\pgfpathrectangle{\pgfqpoint{0.100000in}{0.212622in}}{\pgfqpoint{3.696000in}{3.696000in}}%
\pgfusepath{clip}%
\pgfsetbuttcap%
\pgfsetroundjoin%
\definecolor{currentfill}{rgb}{0.121569,0.466667,0.705882}%
\pgfsetfillcolor{currentfill}%
\pgfsetfillopacity{0.300101}%
\pgfsetlinewidth{1.003750pt}%
\definecolor{currentstroke}{rgb}{0.121569,0.466667,0.705882}%
\pgfsetstrokecolor{currentstroke}%
\pgfsetstrokeopacity{0.300101}%
\pgfsetdash{}{0pt}%
\pgfpathmoveto{\pgfqpoint{1.942102in}{2.079242in}}%
\pgfpathcurveto{\pgfqpoint{1.950338in}{2.079242in}}{\pgfqpoint{1.958238in}{2.082515in}}{\pgfqpoint{1.964062in}{2.088339in}}%
\pgfpathcurveto{\pgfqpoint{1.969886in}{2.094163in}}{\pgfqpoint{1.973158in}{2.102063in}}{\pgfqpoint{1.973158in}{2.110299in}}%
\pgfpathcurveto{\pgfqpoint{1.973158in}{2.118535in}}{\pgfqpoint{1.969886in}{2.126435in}}{\pgfqpoint{1.964062in}{2.132259in}}%
\pgfpathcurveto{\pgfqpoint{1.958238in}{2.138083in}}{\pgfqpoint{1.950338in}{2.141355in}}{\pgfqpoint{1.942102in}{2.141355in}}%
\pgfpathcurveto{\pgfqpoint{1.933866in}{2.141355in}}{\pgfqpoint{1.925966in}{2.138083in}}{\pgfqpoint{1.920142in}{2.132259in}}%
\pgfpathcurveto{\pgfqpoint{1.914318in}{2.126435in}}{\pgfqpoint{1.911045in}{2.118535in}}{\pgfqpoint{1.911045in}{2.110299in}}%
\pgfpathcurveto{\pgfqpoint{1.911045in}{2.102063in}}{\pgfqpoint{1.914318in}{2.094163in}}{\pgfqpoint{1.920142in}{2.088339in}}%
\pgfpathcurveto{\pgfqpoint{1.925966in}{2.082515in}}{\pgfqpoint{1.933866in}{2.079242in}}{\pgfqpoint{1.942102in}{2.079242in}}%
\pgfpathclose%
\pgfusepath{stroke,fill}%
\end{pgfscope}%
\begin{pgfscope}%
\pgfpathrectangle{\pgfqpoint{0.100000in}{0.212622in}}{\pgfqpoint{3.696000in}{3.696000in}}%
\pgfusepath{clip}%
\pgfsetbuttcap%
\pgfsetroundjoin%
\definecolor{currentfill}{rgb}{0.121569,0.466667,0.705882}%
\pgfsetfillcolor{currentfill}%
\pgfsetfillopacity{0.300204}%
\pgfsetlinewidth{1.003750pt}%
\definecolor{currentstroke}{rgb}{0.121569,0.466667,0.705882}%
\pgfsetstrokecolor{currentstroke}%
\pgfsetstrokeopacity{0.300204}%
\pgfsetdash{}{0pt}%
\pgfpathmoveto{\pgfqpoint{1.941767in}{2.079547in}}%
\pgfpathcurveto{\pgfqpoint{1.950003in}{2.079547in}}{\pgfqpoint{1.957903in}{2.082819in}}{\pgfqpoint{1.963727in}{2.088643in}}%
\pgfpathcurveto{\pgfqpoint{1.969551in}{2.094467in}}{\pgfqpoint{1.972824in}{2.102367in}}{\pgfqpoint{1.972824in}{2.110603in}}%
\pgfpathcurveto{\pgfqpoint{1.972824in}{2.118839in}}{\pgfqpoint{1.969551in}{2.126739in}}{\pgfqpoint{1.963727in}{2.132563in}}%
\pgfpathcurveto{\pgfqpoint{1.957903in}{2.138387in}}{\pgfqpoint{1.950003in}{2.141660in}}{\pgfqpoint{1.941767in}{2.141660in}}%
\pgfpathcurveto{\pgfqpoint{1.933531in}{2.141660in}}{\pgfqpoint{1.925631in}{2.138387in}}{\pgfqpoint{1.919807in}{2.132563in}}%
\pgfpathcurveto{\pgfqpoint{1.913983in}{2.126739in}}{\pgfqpoint{1.910711in}{2.118839in}}{\pgfqpoint{1.910711in}{2.110603in}}%
\pgfpathcurveto{\pgfqpoint{1.910711in}{2.102367in}}{\pgfqpoint{1.913983in}{2.094467in}}{\pgfqpoint{1.919807in}{2.088643in}}%
\pgfpathcurveto{\pgfqpoint{1.925631in}{2.082819in}}{\pgfqpoint{1.933531in}{2.079547in}}{\pgfqpoint{1.941767in}{2.079547in}}%
\pgfpathclose%
\pgfusepath{stroke,fill}%
\end{pgfscope}%
\begin{pgfscope}%
\pgfpathrectangle{\pgfqpoint{0.100000in}{0.212622in}}{\pgfqpoint{3.696000in}{3.696000in}}%
\pgfusepath{clip}%
\pgfsetbuttcap%
\pgfsetroundjoin%
\definecolor{currentfill}{rgb}{0.121569,0.466667,0.705882}%
\pgfsetfillcolor{currentfill}%
\pgfsetfillopacity{0.300208}%
\pgfsetlinewidth{1.003750pt}%
\definecolor{currentstroke}{rgb}{0.121569,0.466667,0.705882}%
\pgfsetstrokecolor{currentstroke}%
\pgfsetstrokeopacity{0.300208}%
\pgfsetdash{}{0pt}%
\pgfpathmoveto{\pgfqpoint{1.944005in}{2.078557in}}%
\pgfpathcurveto{\pgfqpoint{1.952242in}{2.078557in}}{\pgfqpoint{1.960142in}{2.081829in}}{\pgfqpoint{1.965966in}{2.087653in}}%
\pgfpathcurveto{\pgfqpoint{1.971790in}{2.093477in}}{\pgfqpoint{1.975062in}{2.101377in}}{\pgfqpoint{1.975062in}{2.109613in}}%
\pgfpathcurveto{\pgfqpoint{1.975062in}{2.117850in}}{\pgfqpoint{1.971790in}{2.125750in}}{\pgfqpoint{1.965966in}{2.131574in}}%
\pgfpathcurveto{\pgfqpoint{1.960142in}{2.137397in}}{\pgfqpoint{1.952242in}{2.140670in}}{\pgfqpoint{1.944005in}{2.140670in}}%
\pgfpathcurveto{\pgfqpoint{1.935769in}{2.140670in}}{\pgfqpoint{1.927869in}{2.137397in}}{\pgfqpoint{1.922045in}{2.131574in}}%
\pgfpathcurveto{\pgfqpoint{1.916221in}{2.125750in}}{\pgfqpoint{1.912949in}{2.117850in}}{\pgfqpoint{1.912949in}{2.109613in}}%
\pgfpathcurveto{\pgfqpoint{1.912949in}{2.101377in}}{\pgfqpoint{1.916221in}{2.093477in}}{\pgfqpoint{1.922045in}{2.087653in}}%
\pgfpathcurveto{\pgfqpoint{1.927869in}{2.081829in}}{\pgfqpoint{1.935769in}{2.078557in}}{\pgfqpoint{1.944005in}{2.078557in}}%
\pgfpathclose%
\pgfusepath{stroke,fill}%
\end{pgfscope}%
\begin{pgfscope}%
\pgfpathrectangle{\pgfqpoint{0.100000in}{0.212622in}}{\pgfqpoint{3.696000in}{3.696000in}}%
\pgfusepath{clip}%
\pgfsetbuttcap%
\pgfsetroundjoin%
\definecolor{currentfill}{rgb}{0.121569,0.466667,0.705882}%
\pgfsetfillcolor{currentfill}%
\pgfsetfillopacity{0.300336}%
\pgfsetlinewidth{1.003750pt}%
\definecolor{currentstroke}{rgb}{0.121569,0.466667,0.705882}%
\pgfsetstrokecolor{currentstroke}%
\pgfsetstrokeopacity{0.300336}%
\pgfsetdash{}{0pt}%
\pgfpathmoveto{\pgfqpoint{1.941210in}{2.079556in}}%
\pgfpathcurveto{\pgfqpoint{1.949446in}{2.079556in}}{\pgfqpoint{1.957346in}{2.082828in}}{\pgfqpoint{1.963170in}{2.088652in}}%
\pgfpathcurveto{\pgfqpoint{1.968994in}{2.094476in}}{\pgfqpoint{1.972267in}{2.102376in}}{\pgfqpoint{1.972267in}{2.110613in}}%
\pgfpathcurveto{\pgfqpoint{1.972267in}{2.118849in}}{\pgfqpoint{1.968994in}{2.126749in}}{\pgfqpoint{1.963170in}{2.132573in}}%
\pgfpathcurveto{\pgfqpoint{1.957346in}{2.138397in}}{\pgfqpoint{1.949446in}{2.141669in}}{\pgfqpoint{1.941210in}{2.141669in}}%
\pgfpathcurveto{\pgfqpoint{1.932974in}{2.141669in}}{\pgfqpoint{1.925074in}{2.138397in}}{\pgfqpoint{1.919250in}{2.132573in}}%
\pgfpathcurveto{\pgfqpoint{1.913426in}{2.126749in}}{\pgfqpoint{1.910154in}{2.118849in}}{\pgfqpoint{1.910154in}{2.110613in}}%
\pgfpathcurveto{\pgfqpoint{1.910154in}{2.102376in}}{\pgfqpoint{1.913426in}{2.094476in}}{\pgfqpoint{1.919250in}{2.088652in}}%
\pgfpathcurveto{\pgfqpoint{1.925074in}{2.082828in}}{\pgfqpoint{1.932974in}{2.079556in}}{\pgfqpoint{1.941210in}{2.079556in}}%
\pgfpathclose%
\pgfusepath{stroke,fill}%
\end{pgfscope}%
\begin{pgfscope}%
\pgfpathrectangle{\pgfqpoint{0.100000in}{0.212622in}}{\pgfqpoint{3.696000in}{3.696000in}}%
\pgfusepath{clip}%
\pgfsetbuttcap%
\pgfsetroundjoin%
\definecolor{currentfill}{rgb}{0.121569,0.466667,0.705882}%
\pgfsetfillcolor{currentfill}%
\pgfsetfillopacity{0.300350}%
\pgfsetlinewidth{1.003750pt}%
\definecolor{currentstroke}{rgb}{0.121569,0.466667,0.705882}%
\pgfsetstrokecolor{currentstroke}%
\pgfsetstrokeopacity{0.300350}%
\pgfsetdash{}{0pt}%
\pgfpathmoveto{\pgfqpoint{1.941159in}{2.079540in}}%
\pgfpathcurveto{\pgfqpoint{1.949395in}{2.079540in}}{\pgfqpoint{1.957295in}{2.082813in}}{\pgfqpoint{1.963119in}{2.088637in}}%
\pgfpathcurveto{\pgfqpoint{1.968943in}{2.094460in}}{\pgfqpoint{1.972215in}{2.102361in}}{\pgfqpoint{1.972215in}{2.110597in}}%
\pgfpathcurveto{\pgfqpoint{1.972215in}{2.118833in}}{\pgfqpoint{1.968943in}{2.126733in}}{\pgfqpoint{1.963119in}{2.132557in}}%
\pgfpathcurveto{\pgfqpoint{1.957295in}{2.138381in}}{\pgfqpoint{1.949395in}{2.141653in}}{\pgfqpoint{1.941159in}{2.141653in}}%
\pgfpathcurveto{\pgfqpoint{1.932923in}{2.141653in}}{\pgfqpoint{1.925023in}{2.138381in}}{\pgfqpoint{1.919199in}{2.132557in}}%
\pgfpathcurveto{\pgfqpoint{1.913375in}{2.126733in}}{\pgfqpoint{1.910102in}{2.118833in}}{\pgfqpoint{1.910102in}{2.110597in}}%
\pgfpathcurveto{\pgfqpoint{1.910102in}{2.102361in}}{\pgfqpoint{1.913375in}{2.094460in}}{\pgfqpoint{1.919199in}{2.088637in}}%
\pgfpathcurveto{\pgfqpoint{1.925023in}{2.082813in}}{\pgfqpoint{1.932923in}{2.079540in}}{\pgfqpoint{1.941159in}{2.079540in}}%
\pgfpathclose%
\pgfusepath{stroke,fill}%
\end{pgfscope}%
\begin{pgfscope}%
\pgfpathrectangle{\pgfqpoint{0.100000in}{0.212622in}}{\pgfqpoint{3.696000in}{3.696000in}}%
\pgfusepath{clip}%
\pgfsetbuttcap%
\pgfsetroundjoin%
\definecolor{currentfill}{rgb}{0.121569,0.466667,0.705882}%
\pgfsetfillcolor{currentfill}%
\pgfsetfillopacity{0.300373}%
\pgfsetlinewidth{1.003750pt}%
\definecolor{currentstroke}{rgb}{0.121569,0.466667,0.705882}%
\pgfsetstrokecolor{currentstroke}%
\pgfsetstrokeopacity{0.300373}%
\pgfsetdash{}{0pt}%
\pgfpathmoveto{\pgfqpoint{1.941061in}{2.079515in}}%
\pgfpathcurveto{\pgfqpoint{1.949297in}{2.079515in}}{\pgfqpoint{1.957197in}{2.082788in}}{\pgfqpoint{1.963021in}{2.088612in}}%
\pgfpathcurveto{\pgfqpoint{1.968845in}{2.094436in}}{\pgfqpoint{1.972117in}{2.102336in}}{\pgfqpoint{1.972117in}{2.110572in}}%
\pgfpathcurveto{\pgfqpoint{1.972117in}{2.118808in}}{\pgfqpoint{1.968845in}{2.126708in}}{\pgfqpoint{1.963021in}{2.132532in}}%
\pgfpathcurveto{\pgfqpoint{1.957197in}{2.138356in}}{\pgfqpoint{1.949297in}{2.141628in}}{\pgfqpoint{1.941061in}{2.141628in}}%
\pgfpathcurveto{\pgfqpoint{1.932825in}{2.141628in}}{\pgfqpoint{1.924924in}{2.138356in}}{\pgfqpoint{1.919101in}{2.132532in}}%
\pgfpathcurveto{\pgfqpoint{1.913277in}{2.126708in}}{\pgfqpoint{1.910004in}{2.118808in}}{\pgfqpoint{1.910004in}{2.110572in}}%
\pgfpathcurveto{\pgfqpoint{1.910004in}{2.102336in}}{\pgfqpoint{1.913277in}{2.094436in}}{\pgfqpoint{1.919101in}{2.088612in}}%
\pgfpathcurveto{\pgfqpoint{1.924924in}{2.082788in}}{\pgfqpoint{1.932825in}{2.079515in}}{\pgfqpoint{1.941061in}{2.079515in}}%
\pgfpathclose%
\pgfusepath{stroke,fill}%
\end{pgfscope}%
\begin{pgfscope}%
\pgfpathrectangle{\pgfqpoint{0.100000in}{0.212622in}}{\pgfqpoint{3.696000in}{3.696000in}}%
\pgfusepath{clip}%
\pgfsetbuttcap%
\pgfsetroundjoin%
\definecolor{currentfill}{rgb}{0.121569,0.466667,0.705882}%
\pgfsetfillcolor{currentfill}%
\pgfsetfillopacity{0.300416}%
\pgfsetlinewidth{1.003750pt}%
\definecolor{currentstroke}{rgb}{0.121569,0.466667,0.705882}%
\pgfsetstrokecolor{currentstroke}%
\pgfsetstrokeopacity{0.300416}%
\pgfsetdash{}{0pt}%
\pgfpathmoveto{\pgfqpoint{1.940883in}{2.079462in}}%
\pgfpathcurveto{\pgfqpoint{1.949120in}{2.079462in}}{\pgfqpoint{1.957020in}{2.082735in}}{\pgfqpoint{1.962844in}{2.088559in}}%
\pgfpathcurveto{\pgfqpoint{1.968667in}{2.094383in}}{\pgfqpoint{1.971940in}{2.102283in}}{\pgfqpoint{1.971940in}{2.110519in}}%
\pgfpathcurveto{\pgfqpoint{1.971940in}{2.118755in}}{\pgfqpoint{1.968667in}{2.126655in}}{\pgfqpoint{1.962844in}{2.132479in}}%
\pgfpathcurveto{\pgfqpoint{1.957020in}{2.138303in}}{\pgfqpoint{1.949120in}{2.141575in}}{\pgfqpoint{1.940883in}{2.141575in}}%
\pgfpathcurveto{\pgfqpoint{1.932647in}{2.141575in}}{\pgfqpoint{1.924747in}{2.138303in}}{\pgfqpoint{1.918923in}{2.132479in}}%
\pgfpathcurveto{\pgfqpoint{1.913099in}{2.126655in}}{\pgfqpoint{1.909827in}{2.118755in}}{\pgfqpoint{1.909827in}{2.110519in}}%
\pgfpathcurveto{\pgfqpoint{1.909827in}{2.102283in}}{\pgfqpoint{1.913099in}{2.094383in}}{\pgfqpoint{1.918923in}{2.088559in}}%
\pgfpathcurveto{\pgfqpoint{1.924747in}{2.082735in}}{\pgfqpoint{1.932647in}{2.079462in}}{\pgfqpoint{1.940883in}{2.079462in}}%
\pgfpathclose%
\pgfusepath{stroke,fill}%
\end{pgfscope}%
\begin{pgfscope}%
\pgfpathrectangle{\pgfqpoint{0.100000in}{0.212622in}}{\pgfqpoint{3.696000in}{3.696000in}}%
\pgfusepath{clip}%
\pgfsetbuttcap%
\pgfsetroundjoin%
\definecolor{currentfill}{rgb}{0.121569,0.466667,0.705882}%
\pgfsetfillcolor{currentfill}%
\pgfsetfillopacity{0.300493}%
\pgfsetlinewidth{1.003750pt}%
\definecolor{currentstroke}{rgb}{0.121569,0.466667,0.705882}%
\pgfsetstrokecolor{currentstroke}%
\pgfsetstrokeopacity{0.300493}%
\pgfsetdash{}{0pt}%
\pgfpathmoveto{\pgfqpoint{1.940560in}{2.079369in}}%
\pgfpathcurveto{\pgfqpoint{1.948796in}{2.079369in}}{\pgfqpoint{1.956696in}{2.082641in}}{\pgfqpoint{1.962520in}{2.088465in}}%
\pgfpathcurveto{\pgfqpoint{1.968344in}{2.094289in}}{\pgfqpoint{1.971616in}{2.102189in}}{\pgfqpoint{1.971616in}{2.110426in}}%
\pgfpathcurveto{\pgfqpoint{1.971616in}{2.118662in}}{\pgfqpoint{1.968344in}{2.126562in}}{\pgfqpoint{1.962520in}{2.132386in}}%
\pgfpathcurveto{\pgfqpoint{1.956696in}{2.138210in}}{\pgfqpoint{1.948796in}{2.141482in}}{\pgfqpoint{1.940560in}{2.141482in}}%
\pgfpathcurveto{\pgfqpoint{1.932324in}{2.141482in}}{\pgfqpoint{1.924424in}{2.138210in}}{\pgfqpoint{1.918600in}{2.132386in}}%
\pgfpathcurveto{\pgfqpoint{1.912776in}{2.126562in}}{\pgfqpoint{1.909503in}{2.118662in}}{\pgfqpoint{1.909503in}{2.110426in}}%
\pgfpathcurveto{\pgfqpoint{1.909503in}{2.102189in}}{\pgfqpoint{1.912776in}{2.094289in}}{\pgfqpoint{1.918600in}{2.088465in}}%
\pgfpathcurveto{\pgfqpoint{1.924424in}{2.082641in}}{\pgfqpoint{1.932324in}{2.079369in}}{\pgfqpoint{1.940560in}{2.079369in}}%
\pgfpathclose%
\pgfusepath{stroke,fill}%
\end{pgfscope}%
\begin{pgfscope}%
\pgfpathrectangle{\pgfqpoint{0.100000in}{0.212622in}}{\pgfqpoint{3.696000in}{3.696000in}}%
\pgfusepath{clip}%
\pgfsetbuttcap%
\pgfsetroundjoin%
\definecolor{currentfill}{rgb}{0.121569,0.466667,0.705882}%
\pgfsetfillcolor{currentfill}%
\pgfsetfillopacity{0.300537}%
\pgfsetlinewidth{1.003750pt}%
\definecolor{currentstroke}{rgb}{0.121569,0.466667,0.705882}%
\pgfsetstrokecolor{currentstroke}%
\pgfsetstrokeopacity{0.300537}%
\pgfsetdash{}{0pt}%
\pgfpathmoveto{\pgfqpoint{1.944199in}{2.078003in}}%
\pgfpathcurveto{\pgfqpoint{1.952435in}{2.078003in}}{\pgfqpoint{1.960335in}{2.081275in}}{\pgfqpoint{1.966159in}{2.087099in}}%
\pgfpathcurveto{\pgfqpoint{1.971983in}{2.092923in}}{\pgfqpoint{1.975256in}{2.100823in}}{\pgfqpoint{1.975256in}{2.109059in}}%
\pgfpathcurveto{\pgfqpoint{1.975256in}{2.117295in}}{\pgfqpoint{1.971983in}{2.125195in}}{\pgfqpoint{1.966159in}{2.131019in}}%
\pgfpathcurveto{\pgfqpoint{1.960335in}{2.136843in}}{\pgfqpoint{1.952435in}{2.140116in}}{\pgfqpoint{1.944199in}{2.140116in}}%
\pgfpathcurveto{\pgfqpoint{1.935963in}{2.140116in}}{\pgfqpoint{1.928063in}{2.136843in}}{\pgfqpoint{1.922239in}{2.131019in}}%
\pgfpathcurveto{\pgfqpoint{1.916415in}{2.125195in}}{\pgfqpoint{1.913143in}{2.117295in}}{\pgfqpoint{1.913143in}{2.109059in}}%
\pgfpathcurveto{\pgfqpoint{1.913143in}{2.100823in}}{\pgfqpoint{1.916415in}{2.092923in}}{\pgfqpoint{1.922239in}{2.087099in}}%
\pgfpathcurveto{\pgfqpoint{1.928063in}{2.081275in}}{\pgfqpoint{1.935963in}{2.078003in}}{\pgfqpoint{1.944199in}{2.078003in}}%
\pgfpathclose%
\pgfusepath{stroke,fill}%
\end{pgfscope}%
\begin{pgfscope}%
\pgfpathrectangle{\pgfqpoint{0.100000in}{0.212622in}}{\pgfqpoint{3.696000in}{3.696000in}}%
\pgfusepath{clip}%
\pgfsetbuttcap%
\pgfsetroundjoin%
\definecolor{currentfill}{rgb}{0.121569,0.466667,0.705882}%
\pgfsetfillcolor{currentfill}%
\pgfsetfillopacity{0.300635}%
\pgfsetlinewidth{1.003750pt}%
\definecolor{currentstroke}{rgb}{0.121569,0.466667,0.705882}%
\pgfsetstrokecolor{currentstroke}%
\pgfsetstrokeopacity{0.300635}%
\pgfsetdash{}{0pt}%
\pgfpathmoveto{\pgfqpoint{1.939968in}{2.079214in}}%
\pgfpathcurveto{\pgfqpoint{1.948204in}{2.079214in}}{\pgfqpoint{1.956104in}{2.082487in}}{\pgfqpoint{1.961928in}{2.088311in}}%
\pgfpathcurveto{\pgfqpoint{1.967752in}{2.094134in}}{\pgfqpoint{1.971024in}{2.102035in}}{\pgfqpoint{1.971024in}{2.110271in}}%
\pgfpathcurveto{\pgfqpoint{1.971024in}{2.118507in}}{\pgfqpoint{1.967752in}{2.126407in}}{\pgfqpoint{1.961928in}{2.132231in}}%
\pgfpathcurveto{\pgfqpoint{1.956104in}{2.138055in}}{\pgfqpoint{1.948204in}{2.141327in}}{\pgfqpoint{1.939968in}{2.141327in}}%
\pgfpathcurveto{\pgfqpoint{1.931732in}{2.141327in}}{\pgfqpoint{1.923831in}{2.138055in}}{\pgfqpoint{1.918008in}{2.132231in}}%
\pgfpathcurveto{\pgfqpoint{1.912184in}{2.126407in}}{\pgfqpoint{1.908911in}{2.118507in}}{\pgfqpoint{1.908911in}{2.110271in}}%
\pgfpathcurveto{\pgfqpoint{1.908911in}{2.102035in}}{\pgfqpoint{1.912184in}{2.094134in}}{\pgfqpoint{1.918008in}{2.088311in}}%
\pgfpathcurveto{\pgfqpoint{1.923831in}{2.082487in}}{\pgfqpoint{1.931732in}{2.079214in}}{\pgfqpoint{1.939968in}{2.079214in}}%
\pgfpathclose%
\pgfusepath{stroke,fill}%
\end{pgfscope}%
\begin{pgfscope}%
\pgfpathrectangle{\pgfqpoint{0.100000in}{0.212622in}}{\pgfqpoint{3.696000in}{3.696000in}}%
\pgfusepath{clip}%
\pgfsetbuttcap%
\pgfsetroundjoin%
\definecolor{currentfill}{rgb}{0.121569,0.466667,0.705882}%
\pgfsetfillcolor{currentfill}%
\pgfsetfillopacity{0.300770}%
\pgfsetlinewidth{1.003750pt}%
\definecolor{currentstroke}{rgb}{0.121569,0.466667,0.705882}%
\pgfsetstrokecolor{currentstroke}%
\pgfsetstrokeopacity{0.300770}%
\pgfsetdash{}{0pt}%
\pgfpathmoveto{\pgfqpoint{1.944244in}{2.078045in}}%
\pgfpathcurveto{\pgfqpoint{1.952480in}{2.078045in}}{\pgfqpoint{1.960380in}{2.081317in}}{\pgfqpoint{1.966204in}{2.087141in}}%
\pgfpathcurveto{\pgfqpoint{1.972028in}{2.092965in}}{\pgfqpoint{1.975301in}{2.100865in}}{\pgfqpoint{1.975301in}{2.109101in}}%
\pgfpathcurveto{\pgfqpoint{1.975301in}{2.117338in}}{\pgfqpoint{1.972028in}{2.125238in}}{\pgfqpoint{1.966204in}{2.131062in}}%
\pgfpathcurveto{\pgfqpoint{1.960380in}{2.136886in}}{\pgfqpoint{1.952480in}{2.140158in}}{\pgfqpoint{1.944244in}{2.140158in}}%
\pgfpathcurveto{\pgfqpoint{1.936008in}{2.140158in}}{\pgfqpoint{1.928108in}{2.136886in}}{\pgfqpoint{1.922284in}{2.131062in}}%
\pgfpathcurveto{\pgfqpoint{1.916460in}{2.125238in}}{\pgfqpoint{1.913188in}{2.117338in}}{\pgfqpoint{1.913188in}{2.109101in}}%
\pgfpathcurveto{\pgfqpoint{1.913188in}{2.100865in}}{\pgfqpoint{1.916460in}{2.092965in}}{\pgfqpoint{1.922284in}{2.087141in}}%
\pgfpathcurveto{\pgfqpoint{1.928108in}{2.081317in}}{\pgfqpoint{1.936008in}{2.078045in}}{\pgfqpoint{1.944244in}{2.078045in}}%
\pgfpathclose%
\pgfusepath{stroke,fill}%
\end{pgfscope}%
\begin{pgfscope}%
\pgfpathrectangle{\pgfqpoint{0.100000in}{0.212622in}}{\pgfqpoint{3.696000in}{3.696000in}}%
\pgfusepath{clip}%
\pgfsetbuttcap%
\pgfsetroundjoin%
\definecolor{currentfill}{rgb}{0.121569,0.466667,0.705882}%
\pgfsetfillcolor{currentfill}%
\pgfsetfillopacity{0.300898}%
\pgfsetlinewidth{1.003750pt}%
\definecolor{currentstroke}{rgb}{0.121569,0.466667,0.705882}%
\pgfsetstrokecolor{currentstroke}%
\pgfsetstrokeopacity{0.300898}%
\pgfsetdash{}{0pt}%
\pgfpathmoveto{\pgfqpoint{1.944242in}{2.078054in}}%
\pgfpathcurveto{\pgfqpoint{1.952479in}{2.078054in}}{\pgfqpoint{1.960379in}{2.081326in}}{\pgfqpoint{1.966203in}{2.087150in}}%
\pgfpathcurveto{\pgfqpoint{1.972027in}{2.092974in}}{\pgfqpoint{1.975299in}{2.100874in}}{\pgfqpoint{1.975299in}{2.109110in}}%
\pgfpathcurveto{\pgfqpoint{1.975299in}{2.117347in}}{\pgfqpoint{1.972027in}{2.125247in}}{\pgfqpoint{1.966203in}{2.131071in}}%
\pgfpathcurveto{\pgfqpoint{1.960379in}{2.136894in}}{\pgfqpoint{1.952479in}{2.140167in}}{\pgfqpoint{1.944242in}{2.140167in}}%
\pgfpathcurveto{\pgfqpoint{1.936006in}{2.140167in}}{\pgfqpoint{1.928106in}{2.136894in}}{\pgfqpoint{1.922282in}{2.131071in}}%
\pgfpathcurveto{\pgfqpoint{1.916458in}{2.125247in}}{\pgfqpoint{1.913186in}{2.117347in}}{\pgfqpoint{1.913186in}{2.109110in}}%
\pgfpathcurveto{\pgfqpoint{1.913186in}{2.100874in}}{\pgfqpoint{1.916458in}{2.092974in}}{\pgfqpoint{1.922282in}{2.087150in}}%
\pgfpathcurveto{\pgfqpoint{1.928106in}{2.081326in}}{\pgfqpoint{1.936006in}{2.078054in}}{\pgfqpoint{1.944242in}{2.078054in}}%
\pgfpathclose%
\pgfusepath{stroke,fill}%
\end{pgfscope}%
\begin{pgfscope}%
\pgfpathrectangle{\pgfqpoint{0.100000in}{0.212622in}}{\pgfqpoint{3.696000in}{3.696000in}}%
\pgfusepath{clip}%
\pgfsetbuttcap%
\pgfsetroundjoin%
\definecolor{currentfill}{rgb}{0.121569,0.466667,0.705882}%
\pgfsetfillcolor{currentfill}%
\pgfsetfillopacity{0.300913}%
\pgfsetlinewidth{1.003750pt}%
\definecolor{currentstroke}{rgb}{0.121569,0.466667,0.705882}%
\pgfsetstrokecolor{currentstroke}%
\pgfsetstrokeopacity{0.300913}%
\pgfsetdash{}{0pt}%
\pgfpathmoveto{\pgfqpoint{1.938952in}{2.078945in}}%
\pgfpathcurveto{\pgfqpoint{1.947188in}{2.078945in}}{\pgfqpoint{1.955088in}{2.082217in}}{\pgfqpoint{1.960912in}{2.088041in}}%
\pgfpathcurveto{\pgfqpoint{1.966736in}{2.093865in}}{\pgfqpoint{1.970008in}{2.101765in}}{\pgfqpoint{1.970008in}{2.110001in}}%
\pgfpathcurveto{\pgfqpoint{1.970008in}{2.118238in}}{\pgfqpoint{1.966736in}{2.126138in}}{\pgfqpoint{1.960912in}{2.131962in}}%
\pgfpathcurveto{\pgfqpoint{1.955088in}{2.137786in}}{\pgfqpoint{1.947188in}{2.141058in}}{\pgfqpoint{1.938952in}{2.141058in}}%
\pgfpathcurveto{\pgfqpoint{1.930715in}{2.141058in}}{\pgfqpoint{1.922815in}{2.137786in}}{\pgfqpoint{1.916991in}{2.131962in}}%
\pgfpathcurveto{\pgfqpoint{1.911167in}{2.126138in}}{\pgfqpoint{1.907895in}{2.118238in}}{\pgfqpoint{1.907895in}{2.110001in}}%
\pgfpathcurveto{\pgfqpoint{1.907895in}{2.101765in}}{\pgfqpoint{1.911167in}{2.093865in}}{\pgfqpoint{1.916991in}{2.088041in}}%
\pgfpathcurveto{\pgfqpoint{1.922815in}{2.082217in}}{\pgfqpoint{1.930715in}{2.078945in}}{\pgfqpoint{1.938952in}{2.078945in}}%
\pgfpathclose%
\pgfusepath{stroke,fill}%
\end{pgfscope}%
\begin{pgfscope}%
\pgfpathrectangle{\pgfqpoint{0.100000in}{0.212622in}}{\pgfqpoint{3.696000in}{3.696000in}}%
\pgfusepath{clip}%
\pgfsetbuttcap%
\pgfsetroundjoin%
\definecolor{currentfill}{rgb}{0.121569,0.466667,0.705882}%
\pgfsetfillcolor{currentfill}%
\pgfsetfillopacity{0.301203}%
\pgfsetlinewidth{1.003750pt}%
\definecolor{currentstroke}{rgb}{0.121569,0.466667,0.705882}%
\pgfsetstrokecolor{currentstroke}%
\pgfsetstrokeopacity{0.301203}%
\pgfsetdash{}{0pt}%
\pgfpathmoveto{\pgfqpoint{1.944203in}{2.077733in}}%
\pgfpathcurveto{\pgfqpoint{1.952439in}{2.077733in}}{\pgfqpoint{1.960339in}{2.081005in}}{\pgfqpoint{1.966163in}{2.086829in}}%
\pgfpathcurveto{\pgfqpoint{1.971987in}{2.092653in}}{\pgfqpoint{1.975260in}{2.100553in}}{\pgfqpoint{1.975260in}{2.108790in}}%
\pgfpathcurveto{\pgfqpoint{1.975260in}{2.117026in}}{\pgfqpoint{1.971987in}{2.124926in}}{\pgfqpoint{1.966163in}{2.130750in}}%
\pgfpathcurveto{\pgfqpoint{1.960339in}{2.136574in}}{\pgfqpoint{1.952439in}{2.139846in}}{\pgfqpoint{1.944203in}{2.139846in}}%
\pgfpathcurveto{\pgfqpoint{1.935967in}{2.139846in}}{\pgfqpoint{1.928067in}{2.136574in}}{\pgfqpoint{1.922243in}{2.130750in}}%
\pgfpathcurveto{\pgfqpoint{1.916419in}{2.124926in}}{\pgfqpoint{1.913147in}{2.117026in}}{\pgfqpoint{1.913147in}{2.108790in}}%
\pgfpathcurveto{\pgfqpoint{1.913147in}{2.100553in}}{\pgfqpoint{1.916419in}{2.092653in}}{\pgfqpoint{1.922243in}{2.086829in}}%
\pgfpathcurveto{\pgfqpoint{1.928067in}{2.081005in}}{\pgfqpoint{1.935967in}{2.077733in}}{\pgfqpoint{1.944203in}{2.077733in}}%
\pgfpathclose%
\pgfusepath{stroke,fill}%
\end{pgfscope}%
\begin{pgfscope}%
\pgfpathrectangle{\pgfqpoint{0.100000in}{0.212622in}}{\pgfqpoint{3.696000in}{3.696000in}}%
\pgfusepath{clip}%
\pgfsetbuttcap%
\pgfsetroundjoin%
\definecolor{currentfill}{rgb}{0.121569,0.466667,0.705882}%
\pgfsetfillcolor{currentfill}%
\pgfsetfillopacity{0.301383}%
\pgfsetlinewidth{1.003750pt}%
\definecolor{currentstroke}{rgb}{0.121569,0.466667,0.705882}%
\pgfsetstrokecolor{currentstroke}%
\pgfsetstrokeopacity{0.301383}%
\pgfsetdash{}{0pt}%
\pgfpathmoveto{\pgfqpoint{1.944173in}{2.077642in}}%
\pgfpathcurveto{\pgfqpoint{1.952409in}{2.077642in}}{\pgfqpoint{1.960309in}{2.080914in}}{\pgfqpoint{1.966133in}{2.086738in}}%
\pgfpathcurveto{\pgfqpoint{1.971957in}{2.092562in}}{\pgfqpoint{1.975229in}{2.100462in}}{\pgfqpoint{1.975229in}{2.108698in}}%
\pgfpathcurveto{\pgfqpoint{1.975229in}{2.116934in}}{\pgfqpoint{1.971957in}{2.124834in}}{\pgfqpoint{1.966133in}{2.130658in}}%
\pgfpathcurveto{\pgfqpoint{1.960309in}{2.136482in}}{\pgfqpoint{1.952409in}{2.139755in}}{\pgfqpoint{1.944173in}{2.139755in}}%
\pgfpathcurveto{\pgfqpoint{1.935936in}{2.139755in}}{\pgfqpoint{1.928036in}{2.136482in}}{\pgfqpoint{1.922212in}{2.130658in}}%
\pgfpathcurveto{\pgfqpoint{1.916389in}{2.124834in}}{\pgfqpoint{1.913116in}{2.116934in}}{\pgfqpoint{1.913116in}{2.108698in}}%
\pgfpathcurveto{\pgfqpoint{1.913116in}{2.100462in}}{\pgfqpoint{1.916389in}{2.092562in}}{\pgfqpoint{1.922212in}{2.086738in}}%
\pgfpathcurveto{\pgfqpoint{1.928036in}{2.080914in}}{\pgfqpoint{1.935936in}{2.077642in}}{\pgfqpoint{1.944173in}{2.077642in}}%
\pgfpathclose%
\pgfusepath{stroke,fill}%
\end{pgfscope}%
\begin{pgfscope}%
\pgfpathrectangle{\pgfqpoint{0.100000in}{0.212622in}}{\pgfqpoint{3.696000in}{3.696000in}}%
\pgfusepath{clip}%
\pgfsetbuttcap%
\pgfsetroundjoin%
\definecolor{currentfill}{rgb}{0.121569,0.466667,0.705882}%
\pgfsetfillcolor{currentfill}%
\pgfsetfillopacity{0.301404}%
\pgfsetlinewidth{1.003750pt}%
\definecolor{currentstroke}{rgb}{0.121569,0.466667,0.705882}%
\pgfsetstrokecolor{currentstroke}%
\pgfsetstrokeopacity{0.301404}%
\pgfsetdash{}{0pt}%
\pgfpathmoveto{\pgfqpoint{1.937042in}{2.078489in}}%
\pgfpathcurveto{\pgfqpoint{1.945278in}{2.078489in}}{\pgfqpoint{1.953178in}{2.081761in}}{\pgfqpoint{1.959002in}{2.087585in}}%
\pgfpathcurveto{\pgfqpoint{1.964826in}{2.093409in}}{\pgfqpoint{1.968098in}{2.101309in}}{\pgfqpoint{1.968098in}{2.109546in}}%
\pgfpathcurveto{\pgfqpoint{1.968098in}{2.117782in}}{\pgfqpoint{1.964826in}{2.125682in}}{\pgfqpoint{1.959002in}{2.131506in}}%
\pgfpathcurveto{\pgfqpoint{1.953178in}{2.137330in}}{\pgfqpoint{1.945278in}{2.140602in}}{\pgfqpoint{1.937042in}{2.140602in}}%
\pgfpathcurveto{\pgfqpoint{1.928806in}{2.140602in}}{\pgfqpoint{1.920906in}{2.137330in}}{\pgfqpoint{1.915082in}{2.131506in}}%
\pgfpathcurveto{\pgfqpoint{1.909258in}{2.125682in}}{\pgfqpoint{1.905985in}{2.117782in}}{\pgfqpoint{1.905985in}{2.109546in}}%
\pgfpathcurveto{\pgfqpoint{1.905985in}{2.101309in}}{\pgfqpoint{1.909258in}{2.093409in}}{\pgfqpoint{1.915082in}{2.087585in}}%
\pgfpathcurveto{\pgfqpoint{1.920906in}{2.081761in}}{\pgfqpoint{1.928806in}{2.078489in}}{\pgfqpoint{1.937042in}{2.078489in}}%
\pgfpathclose%
\pgfusepath{stroke,fill}%
\end{pgfscope}%
\begin{pgfscope}%
\pgfpathrectangle{\pgfqpoint{0.100000in}{0.212622in}}{\pgfqpoint{3.696000in}{3.696000in}}%
\pgfusepath{clip}%
\pgfsetbuttcap%
\pgfsetroundjoin%
\definecolor{currentfill}{rgb}{0.121569,0.466667,0.705882}%
\pgfsetfillcolor{currentfill}%
\pgfsetfillopacity{0.301685}%
\pgfsetlinewidth{1.003750pt}%
\definecolor{currentstroke}{rgb}{0.121569,0.466667,0.705882}%
\pgfsetstrokecolor{currentstroke}%
\pgfsetstrokeopacity{0.301685}%
\pgfsetdash{}{0pt}%
\pgfpathmoveto{\pgfqpoint{1.943991in}{2.077399in}}%
\pgfpathcurveto{\pgfqpoint{1.952227in}{2.077399in}}{\pgfqpoint{1.960127in}{2.080671in}}{\pgfqpoint{1.965951in}{2.086495in}}%
\pgfpathcurveto{\pgfqpoint{1.971775in}{2.092319in}}{\pgfqpoint{1.975047in}{2.100219in}}{\pgfqpoint{1.975047in}{2.108456in}}%
\pgfpathcurveto{\pgfqpoint{1.975047in}{2.116692in}}{\pgfqpoint{1.971775in}{2.124592in}}{\pgfqpoint{1.965951in}{2.130416in}}%
\pgfpathcurveto{\pgfqpoint{1.960127in}{2.136240in}}{\pgfqpoint{1.952227in}{2.139512in}}{\pgfqpoint{1.943991in}{2.139512in}}%
\pgfpathcurveto{\pgfqpoint{1.935755in}{2.139512in}}{\pgfqpoint{1.927855in}{2.136240in}}{\pgfqpoint{1.922031in}{2.130416in}}%
\pgfpathcurveto{\pgfqpoint{1.916207in}{2.124592in}}{\pgfqpoint{1.912934in}{2.116692in}}{\pgfqpoint{1.912934in}{2.108456in}}%
\pgfpathcurveto{\pgfqpoint{1.912934in}{2.100219in}}{\pgfqpoint{1.916207in}{2.092319in}}{\pgfqpoint{1.922031in}{2.086495in}}%
\pgfpathcurveto{\pgfqpoint{1.927855in}{2.080671in}}{\pgfqpoint{1.935755in}{2.077399in}}{\pgfqpoint{1.943991in}{2.077399in}}%
\pgfpathclose%
\pgfusepath{stroke,fill}%
\end{pgfscope}%
\begin{pgfscope}%
\pgfpathrectangle{\pgfqpoint{0.100000in}{0.212622in}}{\pgfqpoint{3.696000in}{3.696000in}}%
\pgfusepath{clip}%
\pgfsetbuttcap%
\pgfsetroundjoin%
\definecolor{currentfill}{rgb}{0.121569,0.466667,0.705882}%
\pgfsetfillcolor{currentfill}%
\pgfsetfillopacity{0.301776}%
\pgfsetlinewidth{1.003750pt}%
\definecolor{currentstroke}{rgb}{0.121569,0.466667,0.705882}%
\pgfsetstrokecolor{currentstroke}%
\pgfsetstrokeopacity{0.301776}%
\pgfsetdash{}{0pt}%
\pgfpathmoveto{\pgfqpoint{1.935550in}{2.077945in}}%
\pgfpathcurveto{\pgfqpoint{1.943787in}{2.077945in}}{\pgfqpoint{1.951687in}{2.081217in}}{\pgfqpoint{1.957511in}{2.087041in}}%
\pgfpathcurveto{\pgfqpoint{1.963335in}{2.092865in}}{\pgfqpoint{1.966607in}{2.100765in}}{\pgfqpoint{1.966607in}{2.109001in}}%
\pgfpathcurveto{\pgfqpoint{1.966607in}{2.117237in}}{\pgfqpoint{1.963335in}{2.125137in}}{\pgfqpoint{1.957511in}{2.130961in}}%
\pgfpathcurveto{\pgfqpoint{1.951687in}{2.136785in}}{\pgfqpoint{1.943787in}{2.140058in}}{\pgfqpoint{1.935550in}{2.140058in}}%
\pgfpathcurveto{\pgfqpoint{1.927314in}{2.140058in}}{\pgfqpoint{1.919414in}{2.136785in}}{\pgfqpoint{1.913590in}{2.130961in}}%
\pgfpathcurveto{\pgfqpoint{1.907766in}{2.125137in}}{\pgfqpoint{1.904494in}{2.117237in}}{\pgfqpoint{1.904494in}{2.109001in}}%
\pgfpathcurveto{\pgfqpoint{1.904494in}{2.100765in}}{\pgfqpoint{1.907766in}{2.092865in}}{\pgfqpoint{1.913590in}{2.087041in}}%
\pgfpathcurveto{\pgfqpoint{1.919414in}{2.081217in}}{\pgfqpoint{1.927314in}{2.077945in}}{\pgfqpoint{1.935550in}{2.077945in}}%
\pgfpathclose%
\pgfusepath{stroke,fill}%
\end{pgfscope}%
\begin{pgfscope}%
\pgfpathrectangle{\pgfqpoint{0.100000in}{0.212622in}}{\pgfqpoint{3.696000in}{3.696000in}}%
\pgfusepath{clip}%
\pgfsetbuttcap%
\pgfsetroundjoin%
\definecolor{currentfill}{rgb}{0.121569,0.466667,0.705882}%
\pgfsetfillcolor{currentfill}%
\pgfsetfillopacity{0.302025}%
\pgfsetlinewidth{1.003750pt}%
\definecolor{currentstroke}{rgb}{0.121569,0.466667,0.705882}%
\pgfsetstrokecolor{currentstroke}%
\pgfsetstrokeopacity{0.302025}%
\pgfsetdash{}{0pt}%
\pgfpathmoveto{\pgfqpoint{1.934567in}{2.077497in}}%
\pgfpathcurveto{\pgfqpoint{1.942803in}{2.077497in}}{\pgfqpoint{1.950703in}{2.080770in}}{\pgfqpoint{1.956527in}{2.086594in}}%
\pgfpathcurveto{\pgfqpoint{1.962351in}{2.092417in}}{\pgfqpoint{1.965623in}{2.100318in}}{\pgfqpoint{1.965623in}{2.108554in}}%
\pgfpathcurveto{\pgfqpoint{1.965623in}{2.116790in}}{\pgfqpoint{1.962351in}{2.124690in}}{\pgfqpoint{1.956527in}{2.130514in}}%
\pgfpathcurveto{\pgfqpoint{1.950703in}{2.136338in}}{\pgfqpoint{1.942803in}{2.139610in}}{\pgfqpoint{1.934567in}{2.139610in}}%
\pgfpathcurveto{\pgfqpoint{1.926330in}{2.139610in}}{\pgfqpoint{1.918430in}{2.136338in}}{\pgfqpoint{1.912606in}{2.130514in}}%
\pgfpathcurveto{\pgfqpoint{1.906783in}{2.124690in}}{\pgfqpoint{1.903510in}{2.116790in}}{\pgfqpoint{1.903510in}{2.108554in}}%
\pgfpathcurveto{\pgfqpoint{1.903510in}{2.100318in}}{\pgfqpoint{1.906783in}{2.092417in}}{\pgfqpoint{1.912606in}{2.086594in}}%
\pgfpathcurveto{\pgfqpoint{1.918430in}{2.080770in}}{\pgfqpoint{1.926330in}{2.077497in}}{\pgfqpoint{1.934567in}{2.077497in}}%
\pgfpathclose%
\pgfusepath{stroke,fill}%
\end{pgfscope}%
\begin{pgfscope}%
\pgfpathrectangle{\pgfqpoint{0.100000in}{0.212622in}}{\pgfqpoint{3.696000in}{3.696000in}}%
\pgfusepath{clip}%
\pgfsetbuttcap%
\pgfsetroundjoin%
\definecolor{currentfill}{rgb}{0.121569,0.466667,0.705882}%
\pgfsetfillcolor{currentfill}%
\pgfsetfillopacity{0.302620}%
\pgfsetlinewidth{1.003750pt}%
\definecolor{currentstroke}{rgb}{0.121569,0.466667,0.705882}%
\pgfsetstrokecolor{currentstroke}%
\pgfsetstrokeopacity{0.302620}%
\pgfsetdash{}{0pt}%
\pgfpathmoveto{\pgfqpoint{1.932838in}{2.077564in}}%
\pgfpathcurveto{\pgfqpoint{1.941075in}{2.077564in}}{\pgfqpoint{1.948975in}{2.080837in}}{\pgfqpoint{1.954798in}{2.086660in}}%
\pgfpathcurveto{\pgfqpoint{1.960622in}{2.092484in}}{\pgfqpoint{1.963895in}{2.100384in}}{\pgfqpoint{1.963895in}{2.108621in}}%
\pgfpathcurveto{\pgfqpoint{1.963895in}{2.116857in}}{\pgfqpoint{1.960622in}{2.124757in}}{\pgfqpoint{1.954798in}{2.130581in}}%
\pgfpathcurveto{\pgfqpoint{1.948975in}{2.136405in}}{\pgfqpoint{1.941075in}{2.139677in}}{\pgfqpoint{1.932838in}{2.139677in}}%
\pgfpathcurveto{\pgfqpoint{1.924602in}{2.139677in}}{\pgfqpoint{1.916702in}{2.136405in}}{\pgfqpoint{1.910878in}{2.130581in}}%
\pgfpathcurveto{\pgfqpoint{1.905054in}{2.124757in}}{\pgfqpoint{1.901782in}{2.116857in}}{\pgfqpoint{1.901782in}{2.108621in}}%
\pgfpathcurveto{\pgfqpoint{1.901782in}{2.100384in}}{\pgfqpoint{1.905054in}{2.092484in}}{\pgfqpoint{1.910878in}{2.086660in}}%
\pgfpathcurveto{\pgfqpoint{1.916702in}{2.080837in}}{\pgfqpoint{1.924602in}{2.077564in}}{\pgfqpoint{1.932838in}{2.077564in}}%
\pgfpathclose%
\pgfusepath{stroke,fill}%
\end{pgfscope}%
\begin{pgfscope}%
\pgfpathrectangle{\pgfqpoint{0.100000in}{0.212622in}}{\pgfqpoint{3.696000in}{3.696000in}}%
\pgfusepath{clip}%
\pgfsetbuttcap%
\pgfsetroundjoin%
\definecolor{currentfill}{rgb}{0.121569,0.466667,0.705882}%
\pgfsetfillcolor{currentfill}%
\pgfsetfillopacity{0.302837}%
\pgfsetlinewidth{1.003750pt}%
\definecolor{currentstroke}{rgb}{0.121569,0.466667,0.705882}%
\pgfsetstrokecolor{currentstroke}%
\pgfsetstrokeopacity{0.302837}%
\pgfsetdash{}{0pt}%
\pgfpathmoveto{\pgfqpoint{1.943914in}{2.078748in}}%
\pgfpathcurveto{\pgfqpoint{1.952151in}{2.078748in}}{\pgfqpoint{1.960051in}{2.082020in}}{\pgfqpoint{1.965875in}{2.087844in}}%
\pgfpathcurveto{\pgfqpoint{1.971699in}{2.093668in}}{\pgfqpoint{1.974971in}{2.101568in}}{\pgfqpoint{1.974971in}{2.109804in}}%
\pgfpathcurveto{\pgfqpoint{1.974971in}{2.118041in}}{\pgfqpoint{1.971699in}{2.125941in}}{\pgfqpoint{1.965875in}{2.131765in}}%
\pgfpathcurveto{\pgfqpoint{1.960051in}{2.137589in}}{\pgfqpoint{1.952151in}{2.140861in}}{\pgfqpoint{1.943914in}{2.140861in}}%
\pgfpathcurveto{\pgfqpoint{1.935678in}{2.140861in}}{\pgfqpoint{1.927778in}{2.137589in}}{\pgfqpoint{1.921954in}{2.131765in}}%
\pgfpathcurveto{\pgfqpoint{1.916130in}{2.125941in}}{\pgfqpoint{1.912858in}{2.118041in}}{\pgfqpoint{1.912858in}{2.109804in}}%
\pgfpathcurveto{\pgfqpoint{1.912858in}{2.101568in}}{\pgfqpoint{1.916130in}{2.093668in}}{\pgfqpoint{1.921954in}{2.087844in}}%
\pgfpathcurveto{\pgfqpoint{1.927778in}{2.082020in}}{\pgfqpoint{1.935678in}{2.078748in}}{\pgfqpoint{1.943914in}{2.078748in}}%
\pgfpathclose%
\pgfusepath{stroke,fill}%
\end{pgfscope}%
\begin{pgfscope}%
\pgfpathrectangle{\pgfqpoint{0.100000in}{0.212622in}}{\pgfqpoint{3.696000in}{3.696000in}}%
\pgfusepath{clip}%
\pgfsetbuttcap%
\pgfsetroundjoin%
\definecolor{currentfill}{rgb}{0.121569,0.466667,0.705882}%
\pgfsetfillcolor{currentfill}%
\pgfsetfillopacity{0.303603}%
\pgfsetlinewidth{1.003750pt}%
\definecolor{currentstroke}{rgb}{0.121569,0.466667,0.705882}%
\pgfsetstrokecolor{currentstroke}%
\pgfsetstrokeopacity{0.303603}%
\pgfsetdash{}{0pt}%
\pgfpathmoveto{\pgfqpoint{1.929783in}{2.076841in}}%
\pgfpathcurveto{\pgfqpoint{1.938020in}{2.076841in}}{\pgfqpoint{1.945920in}{2.080113in}}{\pgfqpoint{1.951744in}{2.085937in}}%
\pgfpathcurveto{\pgfqpoint{1.957568in}{2.091761in}}{\pgfqpoint{1.960840in}{2.099661in}}{\pgfqpoint{1.960840in}{2.107897in}}%
\pgfpathcurveto{\pgfqpoint{1.960840in}{2.116133in}}{\pgfqpoint{1.957568in}{2.124033in}}{\pgfqpoint{1.951744in}{2.129857in}}%
\pgfpathcurveto{\pgfqpoint{1.945920in}{2.135681in}}{\pgfqpoint{1.938020in}{2.138954in}}{\pgfqpoint{1.929783in}{2.138954in}}%
\pgfpathcurveto{\pgfqpoint{1.921547in}{2.138954in}}{\pgfqpoint{1.913647in}{2.135681in}}{\pgfqpoint{1.907823in}{2.129857in}}%
\pgfpathcurveto{\pgfqpoint{1.901999in}{2.124033in}}{\pgfqpoint{1.898727in}{2.116133in}}{\pgfqpoint{1.898727in}{2.107897in}}%
\pgfpathcurveto{\pgfqpoint{1.898727in}{2.099661in}}{\pgfqpoint{1.901999in}{2.091761in}}{\pgfqpoint{1.907823in}{2.085937in}}%
\pgfpathcurveto{\pgfqpoint{1.913647in}{2.080113in}}{\pgfqpoint{1.921547in}{2.076841in}}{\pgfqpoint{1.929783in}{2.076841in}}%
\pgfpathclose%
\pgfusepath{stroke,fill}%
\end{pgfscope}%
\begin{pgfscope}%
\pgfpathrectangle{\pgfqpoint{0.100000in}{0.212622in}}{\pgfqpoint{3.696000in}{3.696000in}}%
\pgfusepath{clip}%
\pgfsetbuttcap%
\pgfsetroundjoin%
\definecolor{currentfill}{rgb}{0.121569,0.466667,0.705882}%
\pgfsetfillcolor{currentfill}%
\pgfsetfillopacity{0.303879}%
\pgfsetlinewidth{1.003750pt}%
\definecolor{currentstroke}{rgb}{0.121569,0.466667,0.705882}%
\pgfsetstrokecolor{currentstroke}%
\pgfsetstrokeopacity{0.303879}%
\pgfsetdash{}{0pt}%
\pgfpathmoveto{\pgfqpoint{1.944522in}{2.076566in}}%
\pgfpathcurveto{\pgfqpoint{1.952758in}{2.076566in}}{\pgfqpoint{1.960658in}{2.079839in}}{\pgfqpoint{1.966482in}{2.085663in}}%
\pgfpathcurveto{\pgfqpoint{1.972306in}{2.091487in}}{\pgfqpoint{1.975578in}{2.099387in}}{\pgfqpoint{1.975578in}{2.107623in}}%
\pgfpathcurveto{\pgfqpoint{1.975578in}{2.115859in}}{\pgfqpoint{1.972306in}{2.123759in}}{\pgfqpoint{1.966482in}{2.129583in}}%
\pgfpathcurveto{\pgfqpoint{1.960658in}{2.135407in}}{\pgfqpoint{1.952758in}{2.138679in}}{\pgfqpoint{1.944522in}{2.138679in}}%
\pgfpathcurveto{\pgfqpoint{1.936286in}{2.138679in}}{\pgfqpoint{1.928386in}{2.135407in}}{\pgfqpoint{1.922562in}{2.129583in}}%
\pgfpathcurveto{\pgfqpoint{1.916738in}{2.123759in}}{\pgfqpoint{1.913465in}{2.115859in}}{\pgfqpoint{1.913465in}{2.107623in}}%
\pgfpathcurveto{\pgfqpoint{1.913465in}{2.099387in}}{\pgfqpoint{1.916738in}{2.091487in}}{\pgfqpoint{1.922562in}{2.085663in}}%
\pgfpathcurveto{\pgfqpoint{1.928386in}{2.079839in}}{\pgfqpoint{1.936286in}{2.076566in}}{\pgfqpoint{1.944522in}{2.076566in}}%
\pgfpathclose%
\pgfusepath{stroke,fill}%
\end{pgfscope}%
\begin{pgfscope}%
\pgfpathrectangle{\pgfqpoint{0.100000in}{0.212622in}}{\pgfqpoint{3.696000in}{3.696000in}}%
\pgfusepath{clip}%
\pgfsetbuttcap%
\pgfsetroundjoin%
\definecolor{currentfill}{rgb}{0.121569,0.466667,0.705882}%
\pgfsetfillcolor{currentfill}%
\pgfsetfillopacity{0.304937}%
\pgfsetlinewidth{1.003750pt}%
\definecolor{currentstroke}{rgb}{0.121569,0.466667,0.705882}%
\pgfsetstrokecolor{currentstroke}%
\pgfsetstrokeopacity{0.304937}%
\pgfsetdash{}{0pt}%
\pgfpathmoveto{\pgfqpoint{1.945533in}{2.073362in}}%
\pgfpathcurveto{\pgfqpoint{1.953770in}{2.073362in}}{\pgfqpoint{1.961670in}{2.076635in}}{\pgfqpoint{1.967494in}{2.082459in}}%
\pgfpathcurveto{\pgfqpoint{1.973317in}{2.088283in}}{\pgfqpoint{1.976590in}{2.096183in}}{\pgfqpoint{1.976590in}{2.104419in}}%
\pgfpathcurveto{\pgfqpoint{1.976590in}{2.112655in}}{\pgfqpoint{1.973317in}{2.120555in}}{\pgfqpoint{1.967494in}{2.126379in}}%
\pgfpathcurveto{\pgfqpoint{1.961670in}{2.132203in}}{\pgfqpoint{1.953770in}{2.135475in}}{\pgfqpoint{1.945533in}{2.135475in}}%
\pgfpathcurveto{\pgfqpoint{1.937297in}{2.135475in}}{\pgfqpoint{1.929397in}{2.132203in}}{\pgfqpoint{1.923573in}{2.126379in}}%
\pgfpathcurveto{\pgfqpoint{1.917749in}{2.120555in}}{\pgfqpoint{1.914477in}{2.112655in}}{\pgfqpoint{1.914477in}{2.104419in}}%
\pgfpathcurveto{\pgfqpoint{1.914477in}{2.096183in}}{\pgfqpoint{1.917749in}{2.088283in}}{\pgfqpoint{1.923573in}{2.082459in}}%
\pgfpathcurveto{\pgfqpoint{1.929397in}{2.076635in}}{\pgfqpoint{1.937297in}{2.073362in}}{\pgfqpoint{1.945533in}{2.073362in}}%
\pgfpathclose%
\pgfusepath{stroke,fill}%
\end{pgfscope}%
\begin{pgfscope}%
\pgfpathrectangle{\pgfqpoint{0.100000in}{0.212622in}}{\pgfqpoint{3.696000in}{3.696000in}}%
\pgfusepath{clip}%
\pgfsetbuttcap%
\pgfsetroundjoin%
\definecolor{currentfill}{rgb}{0.121569,0.466667,0.705882}%
\pgfsetfillcolor{currentfill}%
\pgfsetfillopacity{0.305399}%
\pgfsetlinewidth{1.003750pt}%
\definecolor{currentstroke}{rgb}{0.121569,0.466667,0.705882}%
\pgfsetstrokecolor{currentstroke}%
\pgfsetstrokeopacity{0.305399}%
\pgfsetdash{}{0pt}%
\pgfpathmoveto{\pgfqpoint{1.924217in}{2.075585in}}%
\pgfpathcurveto{\pgfqpoint{1.932453in}{2.075585in}}{\pgfqpoint{1.940353in}{2.078857in}}{\pgfqpoint{1.946177in}{2.084681in}}%
\pgfpathcurveto{\pgfqpoint{1.952001in}{2.090505in}}{\pgfqpoint{1.955273in}{2.098405in}}{\pgfqpoint{1.955273in}{2.106641in}}%
\pgfpathcurveto{\pgfqpoint{1.955273in}{2.114878in}}{\pgfqpoint{1.952001in}{2.122778in}}{\pgfqpoint{1.946177in}{2.128602in}}%
\pgfpathcurveto{\pgfqpoint{1.940353in}{2.134425in}}{\pgfqpoint{1.932453in}{2.137698in}}{\pgfqpoint{1.924217in}{2.137698in}}%
\pgfpathcurveto{\pgfqpoint{1.915981in}{2.137698in}}{\pgfqpoint{1.908081in}{2.134425in}}{\pgfqpoint{1.902257in}{2.128602in}}%
\pgfpathcurveto{\pgfqpoint{1.896433in}{2.122778in}}{\pgfqpoint{1.893160in}{2.114878in}}{\pgfqpoint{1.893160in}{2.106641in}}%
\pgfpathcurveto{\pgfqpoint{1.893160in}{2.098405in}}{\pgfqpoint{1.896433in}{2.090505in}}{\pgfqpoint{1.902257in}{2.084681in}}%
\pgfpathcurveto{\pgfqpoint{1.908081in}{2.078857in}}{\pgfqpoint{1.915981in}{2.075585in}}{\pgfqpoint{1.924217in}{2.075585in}}%
\pgfpathclose%
\pgfusepath{stroke,fill}%
\end{pgfscope}%
\begin{pgfscope}%
\pgfpathrectangle{\pgfqpoint{0.100000in}{0.212622in}}{\pgfqpoint{3.696000in}{3.696000in}}%
\pgfusepath{clip}%
\pgfsetbuttcap%
\pgfsetroundjoin%
\definecolor{currentfill}{rgb}{0.121569,0.466667,0.705882}%
\pgfsetfillcolor{currentfill}%
\pgfsetfillopacity{0.306740}%
\pgfsetlinewidth{1.003750pt}%
\definecolor{currentstroke}{rgb}{0.121569,0.466667,0.705882}%
\pgfsetstrokecolor{currentstroke}%
\pgfsetstrokeopacity{0.306740}%
\pgfsetdash{}{0pt}%
\pgfpathmoveto{\pgfqpoint{1.918960in}{2.073568in}}%
\pgfpathcurveto{\pgfqpoint{1.927196in}{2.073568in}}{\pgfqpoint{1.935096in}{2.076840in}}{\pgfqpoint{1.940920in}{2.082664in}}%
\pgfpathcurveto{\pgfqpoint{1.946744in}{2.088488in}}{\pgfqpoint{1.950016in}{2.096388in}}{\pgfqpoint{1.950016in}{2.104624in}}%
\pgfpathcurveto{\pgfqpoint{1.950016in}{2.112860in}}{\pgfqpoint{1.946744in}{2.120761in}}{\pgfqpoint{1.940920in}{2.126584in}}%
\pgfpathcurveto{\pgfqpoint{1.935096in}{2.132408in}}{\pgfqpoint{1.927196in}{2.135681in}}{\pgfqpoint{1.918960in}{2.135681in}}%
\pgfpathcurveto{\pgfqpoint{1.910724in}{2.135681in}}{\pgfqpoint{1.902823in}{2.132408in}}{\pgfqpoint{1.897000in}{2.126584in}}%
\pgfpathcurveto{\pgfqpoint{1.891176in}{2.120761in}}{\pgfqpoint{1.887903in}{2.112860in}}{\pgfqpoint{1.887903in}{2.104624in}}%
\pgfpathcurveto{\pgfqpoint{1.887903in}{2.096388in}}{\pgfqpoint{1.891176in}{2.088488in}}{\pgfqpoint{1.897000in}{2.082664in}}%
\pgfpathcurveto{\pgfqpoint{1.902823in}{2.076840in}}{\pgfqpoint{1.910724in}{2.073568in}}{\pgfqpoint{1.918960in}{2.073568in}}%
\pgfpathclose%
\pgfusepath{stroke,fill}%
\end{pgfscope}%
\begin{pgfscope}%
\pgfpathrectangle{\pgfqpoint{0.100000in}{0.212622in}}{\pgfqpoint{3.696000in}{3.696000in}}%
\pgfusepath{clip}%
\pgfsetbuttcap%
\pgfsetroundjoin%
\definecolor{currentfill}{rgb}{0.121569,0.466667,0.705882}%
\pgfsetfillcolor{currentfill}%
\pgfsetfillopacity{0.308011}%
\pgfsetlinewidth{1.003750pt}%
\definecolor{currentstroke}{rgb}{0.121569,0.466667,0.705882}%
\pgfsetstrokecolor{currentstroke}%
\pgfsetstrokeopacity{0.308011}%
\pgfsetdash{}{0pt}%
\pgfpathmoveto{\pgfqpoint{1.915024in}{2.070669in}}%
\pgfpathcurveto{\pgfqpoint{1.923260in}{2.070669in}}{\pgfqpoint{1.931160in}{2.073941in}}{\pgfqpoint{1.936984in}{2.079765in}}%
\pgfpathcurveto{\pgfqpoint{1.942808in}{2.085589in}}{\pgfqpoint{1.946080in}{2.093489in}}{\pgfqpoint{1.946080in}{2.101726in}}%
\pgfpathcurveto{\pgfqpoint{1.946080in}{2.109962in}}{\pgfqpoint{1.942808in}{2.117862in}}{\pgfqpoint{1.936984in}{2.123686in}}%
\pgfpathcurveto{\pgfqpoint{1.931160in}{2.129510in}}{\pgfqpoint{1.923260in}{2.132782in}}{\pgfqpoint{1.915024in}{2.132782in}}%
\pgfpathcurveto{\pgfqpoint{1.906787in}{2.132782in}}{\pgfqpoint{1.898887in}{2.129510in}}{\pgfqpoint{1.893063in}{2.123686in}}%
\pgfpathcurveto{\pgfqpoint{1.887239in}{2.117862in}}{\pgfqpoint{1.883967in}{2.109962in}}{\pgfqpoint{1.883967in}{2.101726in}}%
\pgfpathcurveto{\pgfqpoint{1.883967in}{2.093489in}}{\pgfqpoint{1.887239in}{2.085589in}}{\pgfqpoint{1.893063in}{2.079765in}}%
\pgfpathcurveto{\pgfqpoint{1.898887in}{2.073941in}}{\pgfqpoint{1.906787in}{2.070669in}}{\pgfqpoint{1.915024in}{2.070669in}}%
\pgfpathclose%
\pgfusepath{stroke,fill}%
\end{pgfscope}%
\begin{pgfscope}%
\pgfpathrectangle{\pgfqpoint{0.100000in}{0.212622in}}{\pgfqpoint{3.696000in}{3.696000in}}%
\pgfusepath{clip}%
\pgfsetbuttcap%
\pgfsetroundjoin%
\definecolor{currentfill}{rgb}{0.121569,0.466667,0.705882}%
\pgfsetfillcolor{currentfill}%
\pgfsetfillopacity{0.308061}%
\pgfsetlinewidth{1.003750pt}%
\definecolor{currentstroke}{rgb}{0.121569,0.466667,0.705882}%
\pgfsetstrokecolor{currentstroke}%
\pgfsetstrokeopacity{0.308061}%
\pgfsetdash{}{0pt}%
\pgfpathmoveto{\pgfqpoint{1.946626in}{2.075810in}}%
\pgfpathcurveto{\pgfqpoint{1.954862in}{2.075810in}}{\pgfqpoint{1.962763in}{2.079082in}}{\pgfqpoint{1.968586in}{2.084906in}}%
\pgfpathcurveto{\pgfqpoint{1.974410in}{2.090730in}}{\pgfqpoint{1.977683in}{2.098630in}}{\pgfqpoint{1.977683in}{2.106867in}}%
\pgfpathcurveto{\pgfqpoint{1.977683in}{2.115103in}}{\pgfqpoint{1.974410in}{2.123003in}}{\pgfqpoint{1.968586in}{2.128827in}}%
\pgfpathcurveto{\pgfqpoint{1.962763in}{2.134651in}}{\pgfqpoint{1.954862in}{2.137923in}}{\pgfqpoint{1.946626in}{2.137923in}}%
\pgfpathcurveto{\pgfqpoint{1.938390in}{2.137923in}}{\pgfqpoint{1.930490in}{2.134651in}}{\pgfqpoint{1.924666in}{2.128827in}}%
\pgfpathcurveto{\pgfqpoint{1.918842in}{2.123003in}}{\pgfqpoint{1.915570in}{2.115103in}}{\pgfqpoint{1.915570in}{2.106867in}}%
\pgfpathcurveto{\pgfqpoint{1.915570in}{2.098630in}}{\pgfqpoint{1.918842in}{2.090730in}}{\pgfqpoint{1.924666in}{2.084906in}}%
\pgfpathcurveto{\pgfqpoint{1.930490in}{2.079082in}}{\pgfqpoint{1.938390in}{2.075810in}}{\pgfqpoint{1.946626in}{2.075810in}}%
\pgfpathclose%
\pgfusepath{stroke,fill}%
\end{pgfscope}%
\begin{pgfscope}%
\pgfpathrectangle{\pgfqpoint{0.100000in}{0.212622in}}{\pgfqpoint{3.696000in}{3.696000in}}%
\pgfusepath{clip}%
\pgfsetbuttcap%
\pgfsetroundjoin%
\definecolor{currentfill}{rgb}{0.121569,0.466667,0.705882}%
\pgfsetfillcolor{currentfill}%
\pgfsetfillopacity{0.308644}%
\pgfsetlinewidth{1.003750pt}%
\definecolor{currentstroke}{rgb}{0.121569,0.466667,0.705882}%
\pgfsetstrokecolor{currentstroke}%
\pgfsetstrokeopacity{0.308644}%
\pgfsetdash{}{0pt}%
\pgfpathmoveto{\pgfqpoint{1.912764in}{2.069684in}}%
\pgfpathcurveto{\pgfqpoint{1.921001in}{2.069684in}}{\pgfqpoint{1.928901in}{2.072956in}}{\pgfqpoint{1.934725in}{2.078780in}}%
\pgfpathcurveto{\pgfqpoint{1.940549in}{2.084604in}}{\pgfqpoint{1.943821in}{2.092504in}}{\pgfqpoint{1.943821in}{2.100740in}}%
\pgfpathcurveto{\pgfqpoint{1.943821in}{2.108976in}}{\pgfqpoint{1.940549in}{2.116876in}}{\pgfqpoint{1.934725in}{2.122700in}}%
\pgfpathcurveto{\pgfqpoint{1.928901in}{2.128524in}}{\pgfqpoint{1.921001in}{2.131797in}}{\pgfqpoint{1.912764in}{2.131797in}}%
\pgfpathcurveto{\pgfqpoint{1.904528in}{2.131797in}}{\pgfqpoint{1.896628in}{2.128524in}}{\pgfqpoint{1.890804in}{2.122700in}}%
\pgfpathcurveto{\pgfqpoint{1.884980in}{2.116876in}}{\pgfqpoint{1.881708in}{2.108976in}}{\pgfqpoint{1.881708in}{2.100740in}}%
\pgfpathcurveto{\pgfqpoint{1.881708in}{2.092504in}}{\pgfqpoint{1.884980in}{2.084604in}}{\pgfqpoint{1.890804in}{2.078780in}}%
\pgfpathcurveto{\pgfqpoint{1.896628in}{2.072956in}}{\pgfqpoint{1.904528in}{2.069684in}}{\pgfqpoint{1.912764in}{2.069684in}}%
\pgfpathclose%
\pgfusepath{stroke,fill}%
\end{pgfscope}%
\begin{pgfscope}%
\pgfpathrectangle{\pgfqpoint{0.100000in}{0.212622in}}{\pgfqpoint{3.696000in}{3.696000in}}%
\pgfusepath{clip}%
\pgfsetbuttcap%
\pgfsetroundjoin%
\definecolor{currentfill}{rgb}{0.121569,0.466667,0.705882}%
\pgfsetfillcolor{currentfill}%
\pgfsetfillopacity{0.309902}%
\pgfsetlinewidth{1.003750pt}%
\definecolor{currentstroke}{rgb}{0.121569,0.466667,0.705882}%
\pgfsetstrokecolor{currentstroke}%
\pgfsetstrokeopacity{0.309902}%
\pgfsetdash{}{0pt}%
\pgfpathmoveto{\pgfqpoint{1.908806in}{2.068375in}}%
\pgfpathcurveto{\pgfqpoint{1.917042in}{2.068375in}}{\pgfqpoint{1.924942in}{2.071648in}}{\pgfqpoint{1.930766in}{2.077472in}}%
\pgfpathcurveto{\pgfqpoint{1.936590in}{2.083295in}}{\pgfqpoint{1.939862in}{2.091195in}}{\pgfqpoint{1.939862in}{2.099432in}}%
\pgfpathcurveto{\pgfqpoint{1.939862in}{2.107668in}}{\pgfqpoint{1.936590in}{2.115568in}}{\pgfqpoint{1.930766in}{2.121392in}}%
\pgfpathcurveto{\pgfqpoint{1.924942in}{2.127216in}}{\pgfqpoint{1.917042in}{2.130488in}}{\pgfqpoint{1.908806in}{2.130488in}}%
\pgfpathcurveto{\pgfqpoint{1.900570in}{2.130488in}}{\pgfqpoint{1.892670in}{2.127216in}}{\pgfqpoint{1.886846in}{2.121392in}}%
\pgfpathcurveto{\pgfqpoint{1.881022in}{2.115568in}}{\pgfqpoint{1.877749in}{2.107668in}}{\pgfqpoint{1.877749in}{2.099432in}}%
\pgfpathcurveto{\pgfqpoint{1.877749in}{2.091195in}}{\pgfqpoint{1.881022in}{2.083295in}}{\pgfqpoint{1.886846in}{2.077472in}}%
\pgfpathcurveto{\pgfqpoint{1.892670in}{2.071648in}}{\pgfqpoint{1.900570in}{2.068375in}}{\pgfqpoint{1.908806in}{2.068375in}}%
\pgfpathclose%
\pgfusepath{stroke,fill}%
\end{pgfscope}%
\begin{pgfscope}%
\pgfpathrectangle{\pgfqpoint{0.100000in}{0.212622in}}{\pgfqpoint{3.696000in}{3.696000in}}%
\pgfusepath{clip}%
\pgfsetbuttcap%
\pgfsetroundjoin%
\definecolor{currentfill}{rgb}{0.121569,0.466667,0.705882}%
\pgfsetfillcolor{currentfill}%
\pgfsetfillopacity{0.310742}%
\pgfsetlinewidth{1.003750pt}%
\definecolor{currentstroke}{rgb}{0.121569,0.466667,0.705882}%
\pgfsetstrokecolor{currentstroke}%
\pgfsetstrokeopacity{0.310742}%
\pgfsetdash{}{0pt}%
\pgfpathmoveto{\pgfqpoint{1.948522in}{2.072769in}}%
\pgfpathcurveto{\pgfqpoint{1.956759in}{2.072769in}}{\pgfqpoint{1.964659in}{2.076041in}}{\pgfqpoint{1.970483in}{2.081865in}}%
\pgfpathcurveto{\pgfqpoint{1.976307in}{2.087689in}}{\pgfqpoint{1.979579in}{2.095589in}}{\pgfqpoint{1.979579in}{2.103825in}}%
\pgfpathcurveto{\pgfqpoint{1.979579in}{2.112061in}}{\pgfqpoint{1.976307in}{2.119962in}}{\pgfqpoint{1.970483in}{2.125785in}}%
\pgfpathcurveto{\pgfqpoint{1.964659in}{2.131609in}}{\pgfqpoint{1.956759in}{2.134882in}}{\pgfqpoint{1.948522in}{2.134882in}}%
\pgfpathcurveto{\pgfqpoint{1.940286in}{2.134882in}}{\pgfqpoint{1.932386in}{2.131609in}}{\pgfqpoint{1.926562in}{2.125785in}}%
\pgfpathcurveto{\pgfqpoint{1.920738in}{2.119962in}}{\pgfqpoint{1.917466in}{2.112061in}}{\pgfqpoint{1.917466in}{2.103825in}}%
\pgfpathcurveto{\pgfqpoint{1.917466in}{2.095589in}}{\pgfqpoint{1.920738in}{2.087689in}}{\pgfqpoint{1.926562in}{2.081865in}}%
\pgfpathcurveto{\pgfqpoint{1.932386in}{2.076041in}}{\pgfqpoint{1.940286in}{2.072769in}}{\pgfqpoint{1.948522in}{2.072769in}}%
\pgfpathclose%
\pgfusepath{stroke,fill}%
\end{pgfscope}%
\begin{pgfscope}%
\pgfpathrectangle{\pgfqpoint{0.100000in}{0.212622in}}{\pgfqpoint{3.696000in}{3.696000in}}%
\pgfusepath{clip}%
\pgfsetbuttcap%
\pgfsetroundjoin%
\definecolor{currentfill}{rgb}{0.121569,0.466667,0.705882}%
\pgfsetfillcolor{currentfill}%
\pgfsetfillopacity{0.311804}%
\pgfsetlinewidth{1.003750pt}%
\definecolor{currentstroke}{rgb}{0.121569,0.466667,0.705882}%
\pgfsetstrokecolor{currentstroke}%
\pgfsetstrokeopacity{0.311804}%
\pgfsetdash{}{0pt}%
\pgfpathmoveto{\pgfqpoint{1.901879in}{2.062877in}}%
\pgfpathcurveto{\pgfqpoint{1.910115in}{2.062877in}}{\pgfqpoint{1.918015in}{2.066149in}}{\pgfqpoint{1.923839in}{2.071973in}}%
\pgfpathcurveto{\pgfqpoint{1.929663in}{2.077797in}}{\pgfqpoint{1.932935in}{2.085697in}}{\pgfqpoint{1.932935in}{2.093934in}}%
\pgfpathcurveto{\pgfqpoint{1.932935in}{2.102170in}}{\pgfqpoint{1.929663in}{2.110070in}}{\pgfqpoint{1.923839in}{2.115894in}}%
\pgfpathcurveto{\pgfqpoint{1.918015in}{2.121718in}}{\pgfqpoint{1.910115in}{2.124990in}}{\pgfqpoint{1.901879in}{2.124990in}}%
\pgfpathcurveto{\pgfqpoint{1.893643in}{2.124990in}}{\pgfqpoint{1.885742in}{2.121718in}}{\pgfqpoint{1.879919in}{2.115894in}}%
\pgfpathcurveto{\pgfqpoint{1.874095in}{2.110070in}}{\pgfqpoint{1.870822in}{2.102170in}}{\pgfqpoint{1.870822in}{2.093934in}}%
\pgfpathcurveto{\pgfqpoint{1.870822in}{2.085697in}}{\pgfqpoint{1.874095in}{2.077797in}}{\pgfqpoint{1.879919in}{2.071973in}}%
\pgfpathcurveto{\pgfqpoint{1.885742in}{2.066149in}}{\pgfqpoint{1.893643in}{2.062877in}}{\pgfqpoint{1.901879in}{2.062877in}}%
\pgfpathclose%
\pgfusepath{stroke,fill}%
\end{pgfscope}%
\begin{pgfscope}%
\pgfpathrectangle{\pgfqpoint{0.100000in}{0.212622in}}{\pgfqpoint{3.696000in}{3.696000in}}%
\pgfusepath{clip}%
\pgfsetbuttcap%
\pgfsetroundjoin%
\definecolor{currentfill}{rgb}{0.121569,0.466667,0.705882}%
\pgfsetfillcolor{currentfill}%
\pgfsetfillopacity{0.312153}%
\pgfsetlinewidth{1.003750pt}%
\definecolor{currentstroke}{rgb}{0.121569,0.466667,0.705882}%
\pgfsetstrokecolor{currentstroke}%
\pgfsetstrokeopacity{0.312153}%
\pgfsetdash{}{0pt}%
\pgfpathmoveto{\pgfqpoint{1.949673in}{2.070710in}}%
\pgfpathcurveto{\pgfqpoint{1.957909in}{2.070710in}}{\pgfqpoint{1.965809in}{2.073983in}}{\pgfqpoint{1.971633in}{2.079807in}}%
\pgfpathcurveto{\pgfqpoint{1.977457in}{2.085631in}}{\pgfqpoint{1.980730in}{2.093531in}}{\pgfqpoint{1.980730in}{2.101767in}}%
\pgfpathcurveto{\pgfqpoint{1.980730in}{2.110003in}}{\pgfqpoint{1.977457in}{2.117903in}}{\pgfqpoint{1.971633in}{2.123727in}}%
\pgfpathcurveto{\pgfqpoint{1.965809in}{2.129551in}}{\pgfqpoint{1.957909in}{2.132823in}}{\pgfqpoint{1.949673in}{2.132823in}}%
\pgfpathcurveto{\pgfqpoint{1.941437in}{2.132823in}}{\pgfqpoint{1.933537in}{2.129551in}}{\pgfqpoint{1.927713in}{2.123727in}}%
\pgfpathcurveto{\pgfqpoint{1.921889in}{2.117903in}}{\pgfqpoint{1.918617in}{2.110003in}}{\pgfqpoint{1.918617in}{2.101767in}}%
\pgfpathcurveto{\pgfqpoint{1.918617in}{2.093531in}}{\pgfqpoint{1.921889in}{2.085631in}}{\pgfqpoint{1.927713in}{2.079807in}}%
\pgfpathcurveto{\pgfqpoint{1.933537in}{2.073983in}}{\pgfqpoint{1.941437in}{2.070710in}}{\pgfqpoint{1.949673in}{2.070710in}}%
\pgfpathclose%
\pgfusepath{stroke,fill}%
\end{pgfscope}%
\begin{pgfscope}%
\pgfpathrectangle{\pgfqpoint{0.100000in}{0.212622in}}{\pgfqpoint{3.696000in}{3.696000in}}%
\pgfusepath{clip}%
\pgfsetbuttcap%
\pgfsetroundjoin%
\definecolor{currentfill}{rgb}{0.121569,0.466667,0.705882}%
\pgfsetfillcolor{currentfill}%
\pgfsetfillopacity{0.312923}%
\pgfsetlinewidth{1.003750pt}%
\definecolor{currentstroke}{rgb}{0.121569,0.466667,0.705882}%
\pgfsetstrokecolor{currentstroke}%
\pgfsetstrokeopacity{0.312923}%
\pgfsetdash{}{0pt}%
\pgfpathmoveto{\pgfqpoint{1.897682in}{2.060179in}}%
\pgfpathcurveto{\pgfqpoint{1.905919in}{2.060179in}}{\pgfqpoint{1.913819in}{2.063452in}}{\pgfqpoint{1.919643in}{2.069276in}}%
\pgfpathcurveto{\pgfqpoint{1.925467in}{2.075100in}}{\pgfqpoint{1.928739in}{2.083000in}}{\pgfqpoint{1.928739in}{2.091236in}}%
\pgfpathcurveto{\pgfqpoint{1.928739in}{2.099472in}}{\pgfqpoint{1.925467in}{2.107372in}}{\pgfqpoint{1.919643in}{2.113196in}}%
\pgfpathcurveto{\pgfqpoint{1.913819in}{2.119020in}}{\pgfqpoint{1.905919in}{2.122292in}}{\pgfqpoint{1.897682in}{2.122292in}}%
\pgfpathcurveto{\pgfqpoint{1.889446in}{2.122292in}}{\pgfqpoint{1.881546in}{2.119020in}}{\pgfqpoint{1.875722in}{2.113196in}}%
\pgfpathcurveto{\pgfqpoint{1.869898in}{2.107372in}}{\pgfqpoint{1.866626in}{2.099472in}}{\pgfqpoint{1.866626in}{2.091236in}}%
\pgfpathcurveto{\pgfqpoint{1.866626in}{2.083000in}}{\pgfqpoint{1.869898in}{2.075100in}}{\pgfqpoint{1.875722in}{2.069276in}}%
\pgfpathcurveto{\pgfqpoint{1.881546in}{2.063452in}}{\pgfqpoint{1.889446in}{2.060179in}}{\pgfqpoint{1.897682in}{2.060179in}}%
\pgfpathclose%
\pgfusepath{stroke,fill}%
\end{pgfscope}%
\begin{pgfscope}%
\pgfpathrectangle{\pgfqpoint{0.100000in}{0.212622in}}{\pgfqpoint{3.696000in}{3.696000in}}%
\pgfusepath{clip}%
\pgfsetbuttcap%
\pgfsetroundjoin%
\definecolor{currentfill}{rgb}{0.121569,0.466667,0.705882}%
\pgfsetfillcolor{currentfill}%
\pgfsetfillopacity{0.313897}%
\pgfsetlinewidth{1.003750pt}%
\definecolor{currentstroke}{rgb}{0.121569,0.466667,0.705882}%
\pgfsetstrokecolor{currentstroke}%
\pgfsetstrokeopacity{0.313897}%
\pgfsetdash{}{0pt}%
\pgfpathmoveto{\pgfqpoint{1.893844in}{2.057564in}}%
\pgfpathcurveto{\pgfqpoint{1.902080in}{2.057564in}}{\pgfqpoint{1.909980in}{2.060837in}}{\pgfqpoint{1.915804in}{2.066661in}}%
\pgfpathcurveto{\pgfqpoint{1.921628in}{2.072485in}}{\pgfqpoint{1.924901in}{2.080385in}}{\pgfqpoint{1.924901in}{2.088621in}}%
\pgfpathcurveto{\pgfqpoint{1.924901in}{2.096857in}}{\pgfqpoint{1.921628in}{2.104757in}}{\pgfqpoint{1.915804in}{2.110581in}}%
\pgfpathcurveto{\pgfqpoint{1.909980in}{2.116405in}}{\pgfqpoint{1.902080in}{2.119677in}}{\pgfqpoint{1.893844in}{2.119677in}}%
\pgfpathcurveto{\pgfqpoint{1.885608in}{2.119677in}}{\pgfqpoint{1.877708in}{2.116405in}}{\pgfqpoint{1.871884in}{2.110581in}}%
\pgfpathcurveto{\pgfqpoint{1.866060in}{2.104757in}}{\pgfqpoint{1.862788in}{2.096857in}}{\pgfqpoint{1.862788in}{2.088621in}}%
\pgfpathcurveto{\pgfqpoint{1.862788in}{2.080385in}}{\pgfqpoint{1.866060in}{2.072485in}}{\pgfqpoint{1.871884in}{2.066661in}}%
\pgfpathcurveto{\pgfqpoint{1.877708in}{2.060837in}}{\pgfqpoint{1.885608in}{2.057564in}}{\pgfqpoint{1.893844in}{2.057564in}}%
\pgfpathclose%
\pgfusepath{stroke,fill}%
\end{pgfscope}%
\begin{pgfscope}%
\pgfpathrectangle{\pgfqpoint{0.100000in}{0.212622in}}{\pgfqpoint{3.696000in}{3.696000in}}%
\pgfusepath{clip}%
\pgfsetbuttcap%
\pgfsetroundjoin%
\definecolor{currentfill}{rgb}{0.121569,0.466667,0.705882}%
\pgfsetfillcolor{currentfill}%
\pgfsetfillopacity{0.314663}%
\pgfsetlinewidth{1.003750pt}%
\definecolor{currentstroke}{rgb}{0.121569,0.466667,0.705882}%
\pgfsetstrokecolor{currentstroke}%
\pgfsetstrokeopacity{0.314663}%
\pgfsetdash{}{0pt}%
\pgfpathmoveto{\pgfqpoint{1.950510in}{2.071246in}}%
\pgfpathcurveto{\pgfqpoint{1.958746in}{2.071246in}}{\pgfqpoint{1.966646in}{2.074518in}}{\pgfqpoint{1.972470in}{2.080342in}}%
\pgfpathcurveto{\pgfqpoint{1.978294in}{2.086166in}}{\pgfqpoint{1.981566in}{2.094066in}}{\pgfqpoint{1.981566in}{2.102302in}}%
\pgfpathcurveto{\pgfqpoint{1.981566in}{2.110539in}}{\pgfqpoint{1.978294in}{2.118439in}}{\pgfqpoint{1.972470in}{2.124263in}}%
\pgfpathcurveto{\pgfqpoint{1.966646in}{2.130087in}}{\pgfqpoint{1.958746in}{2.133359in}}{\pgfqpoint{1.950510in}{2.133359in}}%
\pgfpathcurveto{\pgfqpoint{1.942273in}{2.133359in}}{\pgfqpoint{1.934373in}{2.130087in}}{\pgfqpoint{1.928549in}{2.124263in}}%
\pgfpathcurveto{\pgfqpoint{1.922725in}{2.118439in}}{\pgfqpoint{1.919453in}{2.110539in}}{\pgfqpoint{1.919453in}{2.102302in}}%
\pgfpathcurveto{\pgfqpoint{1.919453in}{2.094066in}}{\pgfqpoint{1.922725in}{2.086166in}}{\pgfqpoint{1.928549in}{2.080342in}}%
\pgfpathcurveto{\pgfqpoint{1.934373in}{2.074518in}}{\pgfqpoint{1.942273in}{2.071246in}}{\pgfqpoint{1.950510in}{2.071246in}}%
\pgfpathclose%
\pgfusepath{stroke,fill}%
\end{pgfscope}%
\begin{pgfscope}%
\pgfpathrectangle{\pgfqpoint{0.100000in}{0.212622in}}{\pgfqpoint{3.696000in}{3.696000in}}%
\pgfusepath{clip}%
\pgfsetbuttcap%
\pgfsetroundjoin%
\definecolor{currentfill}{rgb}{0.121569,0.466667,0.705882}%
\pgfsetfillcolor{currentfill}%
\pgfsetfillopacity{0.315607}%
\pgfsetlinewidth{1.003750pt}%
\definecolor{currentstroke}{rgb}{0.121569,0.466667,0.705882}%
\pgfsetstrokecolor{currentstroke}%
\pgfsetstrokeopacity{0.315607}%
\pgfsetdash{}{0pt}%
\pgfpathmoveto{\pgfqpoint{1.887687in}{2.051090in}}%
\pgfpathcurveto{\pgfqpoint{1.895923in}{2.051090in}}{\pgfqpoint{1.903823in}{2.054362in}}{\pgfqpoint{1.909647in}{2.060186in}}%
\pgfpathcurveto{\pgfqpoint{1.915471in}{2.066010in}}{\pgfqpoint{1.918743in}{2.073910in}}{\pgfqpoint{1.918743in}{2.082146in}}%
\pgfpathcurveto{\pgfqpoint{1.918743in}{2.090383in}}{\pgfqpoint{1.915471in}{2.098283in}}{\pgfqpoint{1.909647in}{2.104107in}}%
\pgfpathcurveto{\pgfqpoint{1.903823in}{2.109931in}}{\pgfqpoint{1.895923in}{2.113203in}}{\pgfqpoint{1.887687in}{2.113203in}}%
\pgfpathcurveto{\pgfqpoint{1.879450in}{2.113203in}}{\pgfqpoint{1.871550in}{2.109931in}}{\pgfqpoint{1.865726in}{2.104107in}}%
\pgfpathcurveto{\pgfqpoint{1.859902in}{2.098283in}}{\pgfqpoint{1.856630in}{2.090383in}}{\pgfqpoint{1.856630in}{2.082146in}}%
\pgfpathcurveto{\pgfqpoint{1.856630in}{2.073910in}}{\pgfqpoint{1.859902in}{2.066010in}}{\pgfqpoint{1.865726in}{2.060186in}}%
\pgfpathcurveto{\pgfqpoint{1.871550in}{2.054362in}}{\pgfqpoint{1.879450in}{2.051090in}}{\pgfqpoint{1.887687in}{2.051090in}}%
\pgfpathclose%
\pgfusepath{stroke,fill}%
\end{pgfscope}%
\begin{pgfscope}%
\pgfpathrectangle{\pgfqpoint{0.100000in}{0.212622in}}{\pgfqpoint{3.696000in}{3.696000in}}%
\pgfusepath{clip}%
\pgfsetbuttcap%
\pgfsetroundjoin%
\definecolor{currentfill}{rgb}{0.121569,0.466667,0.705882}%
\pgfsetfillcolor{currentfill}%
\pgfsetfillopacity{0.315740}%
\pgfsetlinewidth{1.003750pt}%
\definecolor{currentstroke}{rgb}{0.121569,0.466667,0.705882}%
\pgfsetstrokecolor{currentstroke}%
\pgfsetstrokeopacity{0.315740}%
\pgfsetdash{}{0pt}%
\pgfpathmoveto{\pgfqpoint{1.951538in}{2.069721in}}%
\pgfpathcurveto{\pgfqpoint{1.959774in}{2.069721in}}{\pgfqpoint{1.967674in}{2.072993in}}{\pgfqpoint{1.973498in}{2.078817in}}%
\pgfpathcurveto{\pgfqpoint{1.979322in}{2.084641in}}{\pgfqpoint{1.982594in}{2.092541in}}{\pgfqpoint{1.982594in}{2.100777in}}%
\pgfpathcurveto{\pgfqpoint{1.982594in}{2.109014in}}{\pgfqpoint{1.979322in}{2.116914in}}{\pgfqpoint{1.973498in}{2.122738in}}%
\pgfpathcurveto{\pgfqpoint{1.967674in}{2.128562in}}{\pgfqpoint{1.959774in}{2.131834in}}{\pgfqpoint{1.951538in}{2.131834in}}%
\pgfpathcurveto{\pgfqpoint{1.943302in}{2.131834in}}{\pgfqpoint{1.935402in}{2.128562in}}{\pgfqpoint{1.929578in}{2.122738in}}%
\pgfpathcurveto{\pgfqpoint{1.923754in}{2.116914in}}{\pgfqpoint{1.920481in}{2.109014in}}{\pgfqpoint{1.920481in}{2.100777in}}%
\pgfpathcurveto{\pgfqpoint{1.920481in}{2.092541in}}{\pgfqpoint{1.923754in}{2.084641in}}{\pgfqpoint{1.929578in}{2.078817in}}%
\pgfpathcurveto{\pgfqpoint{1.935402in}{2.072993in}}{\pgfqpoint{1.943302in}{2.069721in}}{\pgfqpoint{1.951538in}{2.069721in}}%
\pgfpathclose%
\pgfusepath{stroke,fill}%
\end{pgfscope}%
\begin{pgfscope}%
\pgfpathrectangle{\pgfqpoint{0.100000in}{0.212622in}}{\pgfqpoint{3.696000in}{3.696000in}}%
\pgfusepath{clip}%
\pgfsetbuttcap%
\pgfsetroundjoin%
\definecolor{currentfill}{rgb}{0.121569,0.466667,0.705882}%
\pgfsetfillcolor{currentfill}%
\pgfsetfillopacity{0.317076}%
\pgfsetlinewidth{1.003750pt}%
\definecolor{currentstroke}{rgb}{0.121569,0.466667,0.705882}%
\pgfsetstrokecolor{currentstroke}%
\pgfsetstrokeopacity{0.317076}%
\pgfsetdash{}{0pt}%
\pgfpathmoveto{\pgfqpoint{1.883298in}{2.049492in}}%
\pgfpathcurveto{\pgfqpoint{1.891534in}{2.049492in}}{\pgfqpoint{1.899434in}{2.052764in}}{\pgfqpoint{1.905258in}{2.058588in}}%
\pgfpathcurveto{\pgfqpoint{1.911082in}{2.064412in}}{\pgfqpoint{1.914354in}{2.072312in}}{\pgfqpoint{1.914354in}{2.080548in}}%
\pgfpathcurveto{\pgfqpoint{1.914354in}{2.088785in}}{\pgfqpoint{1.911082in}{2.096685in}}{\pgfqpoint{1.905258in}{2.102509in}}%
\pgfpathcurveto{\pgfqpoint{1.899434in}{2.108332in}}{\pgfqpoint{1.891534in}{2.111605in}}{\pgfqpoint{1.883298in}{2.111605in}}%
\pgfpathcurveto{\pgfqpoint{1.875061in}{2.111605in}}{\pgfqpoint{1.867161in}{2.108332in}}{\pgfqpoint{1.861337in}{2.102509in}}%
\pgfpathcurveto{\pgfqpoint{1.855513in}{2.096685in}}{\pgfqpoint{1.852241in}{2.088785in}}{\pgfqpoint{1.852241in}{2.080548in}}%
\pgfpathcurveto{\pgfqpoint{1.852241in}{2.072312in}}{\pgfqpoint{1.855513in}{2.064412in}}{\pgfqpoint{1.861337in}{2.058588in}}%
\pgfpathcurveto{\pgfqpoint{1.867161in}{2.052764in}}{\pgfqpoint{1.875061in}{2.049492in}}{\pgfqpoint{1.883298in}{2.049492in}}%
\pgfpathclose%
\pgfusepath{stroke,fill}%
\end{pgfscope}%
\begin{pgfscope}%
\pgfpathrectangle{\pgfqpoint{0.100000in}{0.212622in}}{\pgfqpoint{3.696000in}{3.696000in}}%
\pgfusepath{clip}%
\pgfsetbuttcap%
\pgfsetroundjoin%
\definecolor{currentfill}{rgb}{0.121569,0.466667,0.705882}%
\pgfsetfillcolor{currentfill}%
\pgfsetfillopacity{0.317285}%
\pgfsetlinewidth{1.003750pt}%
\definecolor{currentstroke}{rgb}{0.121569,0.466667,0.705882}%
\pgfsetstrokecolor{currentstroke}%
\pgfsetstrokeopacity{0.317285}%
\pgfsetdash{}{0pt}%
\pgfpathmoveto{\pgfqpoint{1.952707in}{2.068305in}}%
\pgfpathcurveto{\pgfqpoint{1.960944in}{2.068305in}}{\pgfqpoint{1.968844in}{2.071577in}}{\pgfqpoint{1.974668in}{2.077401in}}%
\pgfpathcurveto{\pgfqpoint{1.980492in}{2.083225in}}{\pgfqpoint{1.983764in}{2.091125in}}{\pgfqpoint{1.983764in}{2.099361in}}%
\pgfpathcurveto{\pgfqpoint{1.983764in}{2.107598in}}{\pgfqpoint{1.980492in}{2.115498in}}{\pgfqpoint{1.974668in}{2.121322in}}%
\pgfpathcurveto{\pgfqpoint{1.968844in}{2.127146in}}{\pgfqpoint{1.960944in}{2.130418in}}{\pgfqpoint{1.952707in}{2.130418in}}%
\pgfpathcurveto{\pgfqpoint{1.944471in}{2.130418in}}{\pgfqpoint{1.936571in}{2.127146in}}{\pgfqpoint{1.930747in}{2.121322in}}%
\pgfpathcurveto{\pgfqpoint{1.924923in}{2.115498in}}{\pgfqpoint{1.921651in}{2.107598in}}{\pgfqpoint{1.921651in}{2.099361in}}%
\pgfpathcurveto{\pgfqpoint{1.921651in}{2.091125in}}{\pgfqpoint{1.924923in}{2.083225in}}{\pgfqpoint{1.930747in}{2.077401in}}%
\pgfpathcurveto{\pgfqpoint{1.936571in}{2.071577in}}{\pgfqpoint{1.944471in}{2.068305in}}{\pgfqpoint{1.952707in}{2.068305in}}%
\pgfpathclose%
\pgfusepath{stroke,fill}%
\end{pgfscope}%
\begin{pgfscope}%
\pgfpathrectangle{\pgfqpoint{0.100000in}{0.212622in}}{\pgfqpoint{3.696000in}{3.696000in}}%
\pgfusepath{clip}%
\pgfsetbuttcap%
\pgfsetroundjoin%
\definecolor{currentfill}{rgb}{0.121569,0.466667,0.705882}%
\pgfsetfillcolor{currentfill}%
\pgfsetfillopacity{0.318268}%
\pgfsetlinewidth{1.003750pt}%
\definecolor{currentstroke}{rgb}{0.121569,0.466667,0.705882}%
\pgfsetstrokecolor{currentstroke}%
\pgfsetstrokeopacity{0.318268}%
\pgfsetdash{}{0pt}%
\pgfpathmoveto{\pgfqpoint{1.879378in}{2.046613in}}%
\pgfpathcurveto{\pgfqpoint{1.887614in}{2.046613in}}{\pgfqpoint{1.895515in}{2.049886in}}{\pgfqpoint{1.901338in}{2.055709in}}%
\pgfpathcurveto{\pgfqpoint{1.907162in}{2.061533in}}{\pgfqpoint{1.910435in}{2.069433in}}{\pgfqpoint{1.910435in}{2.077670in}}%
\pgfpathcurveto{\pgfqpoint{1.910435in}{2.085906in}}{\pgfqpoint{1.907162in}{2.093806in}}{\pgfqpoint{1.901338in}{2.099630in}}%
\pgfpathcurveto{\pgfqpoint{1.895515in}{2.105454in}}{\pgfqpoint{1.887614in}{2.108726in}}{\pgfqpoint{1.879378in}{2.108726in}}%
\pgfpathcurveto{\pgfqpoint{1.871142in}{2.108726in}}{\pgfqpoint{1.863242in}{2.105454in}}{\pgfqpoint{1.857418in}{2.099630in}}%
\pgfpathcurveto{\pgfqpoint{1.851594in}{2.093806in}}{\pgfqpoint{1.848322in}{2.085906in}}{\pgfqpoint{1.848322in}{2.077670in}}%
\pgfpathcurveto{\pgfqpoint{1.848322in}{2.069433in}}{\pgfqpoint{1.851594in}{2.061533in}}{\pgfqpoint{1.857418in}{2.055709in}}%
\pgfpathcurveto{\pgfqpoint{1.863242in}{2.049886in}}{\pgfqpoint{1.871142in}{2.046613in}}{\pgfqpoint{1.879378in}{2.046613in}}%
\pgfpathclose%
\pgfusepath{stroke,fill}%
\end{pgfscope}%
\begin{pgfscope}%
\pgfpathrectangle{\pgfqpoint{0.100000in}{0.212622in}}{\pgfqpoint{3.696000in}{3.696000in}}%
\pgfusepath{clip}%
\pgfsetbuttcap%
\pgfsetroundjoin%
\definecolor{currentfill}{rgb}{0.121569,0.466667,0.705882}%
\pgfsetfillcolor{currentfill}%
\pgfsetfillopacity{0.318307}%
\pgfsetlinewidth{1.003750pt}%
\definecolor{currentstroke}{rgb}{0.121569,0.466667,0.705882}%
\pgfsetstrokecolor{currentstroke}%
\pgfsetstrokeopacity{0.318307}%
\pgfsetdash{}{0pt}%
\pgfpathmoveto{\pgfqpoint{1.953263in}{2.068676in}}%
\pgfpathcurveto{\pgfqpoint{1.961499in}{2.068676in}}{\pgfqpoint{1.969399in}{2.071948in}}{\pgfqpoint{1.975223in}{2.077772in}}%
\pgfpathcurveto{\pgfqpoint{1.981047in}{2.083596in}}{\pgfqpoint{1.984320in}{2.091496in}}{\pgfqpoint{1.984320in}{2.099733in}}%
\pgfpathcurveto{\pgfqpoint{1.984320in}{2.107969in}}{\pgfqpoint{1.981047in}{2.115869in}}{\pgfqpoint{1.975223in}{2.121693in}}%
\pgfpathcurveto{\pgfqpoint{1.969399in}{2.127517in}}{\pgfqpoint{1.961499in}{2.130789in}}{\pgfqpoint{1.953263in}{2.130789in}}%
\pgfpathcurveto{\pgfqpoint{1.945027in}{2.130789in}}{\pgfqpoint{1.937127in}{2.127517in}}{\pgfqpoint{1.931303in}{2.121693in}}%
\pgfpathcurveto{\pgfqpoint{1.925479in}{2.115869in}}{\pgfqpoint{1.922207in}{2.107969in}}{\pgfqpoint{1.922207in}{2.099733in}}%
\pgfpathcurveto{\pgfqpoint{1.922207in}{2.091496in}}{\pgfqpoint{1.925479in}{2.083596in}}{\pgfqpoint{1.931303in}{2.077772in}}%
\pgfpathcurveto{\pgfqpoint{1.937127in}{2.071948in}}{\pgfqpoint{1.945027in}{2.068676in}}{\pgfqpoint{1.953263in}{2.068676in}}%
\pgfpathclose%
\pgfusepath{stroke,fill}%
\end{pgfscope}%
\begin{pgfscope}%
\pgfpathrectangle{\pgfqpoint{0.100000in}{0.212622in}}{\pgfqpoint{3.696000in}{3.696000in}}%
\pgfusepath{clip}%
\pgfsetbuttcap%
\pgfsetroundjoin%
\definecolor{currentfill}{rgb}{0.121569,0.466667,0.705882}%
\pgfsetfillcolor{currentfill}%
\pgfsetfillopacity{0.319766}%
\pgfsetlinewidth{1.003750pt}%
\definecolor{currentstroke}{rgb}{0.121569,0.466667,0.705882}%
\pgfsetstrokecolor{currentstroke}%
\pgfsetstrokeopacity{0.319766}%
\pgfsetdash{}{0pt}%
\pgfpathmoveto{\pgfqpoint{1.954444in}{2.066949in}}%
\pgfpathcurveto{\pgfqpoint{1.962680in}{2.066949in}}{\pgfqpoint{1.970580in}{2.070222in}}{\pgfqpoint{1.976404in}{2.076045in}}%
\pgfpathcurveto{\pgfqpoint{1.982228in}{2.081869in}}{\pgfqpoint{1.985501in}{2.089769in}}{\pgfqpoint{1.985501in}{2.098006in}}%
\pgfpathcurveto{\pgfqpoint{1.985501in}{2.106242in}}{\pgfqpoint{1.982228in}{2.114142in}}{\pgfqpoint{1.976404in}{2.119966in}}%
\pgfpathcurveto{\pgfqpoint{1.970580in}{2.125790in}}{\pgfqpoint{1.962680in}{2.129062in}}{\pgfqpoint{1.954444in}{2.129062in}}%
\pgfpathcurveto{\pgfqpoint{1.946208in}{2.129062in}}{\pgfqpoint{1.938308in}{2.125790in}}{\pgfqpoint{1.932484in}{2.119966in}}%
\pgfpathcurveto{\pgfqpoint{1.926660in}{2.114142in}}{\pgfqpoint{1.923388in}{2.106242in}}{\pgfqpoint{1.923388in}{2.098006in}}%
\pgfpathcurveto{\pgfqpoint{1.923388in}{2.089769in}}{\pgfqpoint{1.926660in}{2.081869in}}{\pgfqpoint{1.932484in}{2.076045in}}%
\pgfpathcurveto{\pgfqpoint{1.938308in}{2.070222in}}{\pgfqpoint{1.946208in}{2.066949in}}{\pgfqpoint{1.954444in}{2.066949in}}%
\pgfpathclose%
\pgfusepath{stroke,fill}%
\end{pgfscope}%
\begin{pgfscope}%
\pgfpathrectangle{\pgfqpoint{0.100000in}{0.212622in}}{\pgfqpoint{3.696000in}{3.696000in}}%
\pgfusepath{clip}%
\pgfsetbuttcap%
\pgfsetroundjoin%
\definecolor{currentfill}{rgb}{0.121569,0.466667,0.705882}%
\pgfsetfillcolor{currentfill}%
\pgfsetfillopacity{0.320258}%
\pgfsetlinewidth{1.003750pt}%
\definecolor{currentstroke}{rgb}{0.121569,0.466667,0.705882}%
\pgfsetstrokecolor{currentstroke}%
\pgfsetstrokeopacity{0.320258}%
\pgfsetdash{}{0pt}%
\pgfpathmoveto{\pgfqpoint{1.872719in}{2.039500in}}%
\pgfpathcurveto{\pgfqpoint{1.880956in}{2.039500in}}{\pgfqpoint{1.888856in}{2.042772in}}{\pgfqpoint{1.894680in}{2.048596in}}%
\pgfpathcurveto{\pgfqpoint{1.900503in}{2.054420in}}{\pgfqpoint{1.903776in}{2.062320in}}{\pgfqpoint{1.903776in}{2.070556in}}%
\pgfpathcurveto{\pgfqpoint{1.903776in}{2.078793in}}{\pgfqpoint{1.900503in}{2.086693in}}{\pgfqpoint{1.894680in}{2.092517in}}%
\pgfpathcurveto{\pgfqpoint{1.888856in}{2.098341in}}{\pgfqpoint{1.880956in}{2.101613in}}{\pgfqpoint{1.872719in}{2.101613in}}%
\pgfpathcurveto{\pgfqpoint{1.864483in}{2.101613in}}{\pgfqpoint{1.856583in}{2.098341in}}{\pgfqpoint{1.850759in}{2.092517in}}%
\pgfpathcurveto{\pgfqpoint{1.844935in}{2.086693in}}{\pgfqpoint{1.841663in}{2.078793in}}{\pgfqpoint{1.841663in}{2.070556in}}%
\pgfpathcurveto{\pgfqpoint{1.841663in}{2.062320in}}{\pgfqpoint{1.844935in}{2.054420in}}{\pgfqpoint{1.850759in}{2.048596in}}%
\pgfpathcurveto{\pgfqpoint{1.856583in}{2.042772in}}{\pgfqpoint{1.864483in}{2.039500in}}{\pgfqpoint{1.872719in}{2.039500in}}%
\pgfpathclose%
\pgfusepath{stroke,fill}%
\end{pgfscope}%
\begin{pgfscope}%
\pgfpathrectangle{\pgfqpoint{0.100000in}{0.212622in}}{\pgfqpoint{3.696000in}{3.696000in}}%
\pgfusepath{clip}%
\pgfsetbuttcap%
\pgfsetroundjoin%
\definecolor{currentfill}{rgb}{0.121569,0.466667,0.705882}%
\pgfsetfillcolor{currentfill}%
\pgfsetfillopacity{0.321663}%
\pgfsetlinewidth{1.003750pt}%
\definecolor{currentstroke}{rgb}{0.121569,0.466667,0.705882}%
\pgfsetstrokecolor{currentstroke}%
\pgfsetstrokeopacity{0.321663}%
\pgfsetdash{}{0pt}%
\pgfpathmoveto{\pgfqpoint{1.955978in}{2.065593in}}%
\pgfpathcurveto{\pgfqpoint{1.964214in}{2.065593in}}{\pgfqpoint{1.972114in}{2.068866in}}{\pgfqpoint{1.977938in}{2.074690in}}%
\pgfpathcurveto{\pgfqpoint{1.983762in}{2.080514in}}{\pgfqpoint{1.987034in}{2.088414in}}{\pgfqpoint{1.987034in}{2.096650in}}%
\pgfpathcurveto{\pgfqpoint{1.987034in}{2.104886in}}{\pgfqpoint{1.983762in}{2.112786in}}{\pgfqpoint{1.977938in}{2.118610in}}%
\pgfpathcurveto{\pgfqpoint{1.972114in}{2.124434in}}{\pgfqpoint{1.964214in}{2.127706in}}{\pgfqpoint{1.955978in}{2.127706in}}%
\pgfpathcurveto{\pgfqpoint{1.947742in}{2.127706in}}{\pgfqpoint{1.939842in}{2.124434in}}{\pgfqpoint{1.934018in}{2.118610in}}%
\pgfpathcurveto{\pgfqpoint{1.928194in}{2.112786in}}{\pgfqpoint{1.924921in}{2.104886in}}{\pgfqpoint{1.924921in}{2.096650in}}%
\pgfpathcurveto{\pgfqpoint{1.924921in}{2.088414in}}{\pgfqpoint{1.928194in}{2.080514in}}{\pgfqpoint{1.934018in}{2.074690in}}%
\pgfpathcurveto{\pgfqpoint{1.939842in}{2.068866in}}{\pgfqpoint{1.947742in}{2.065593in}}{\pgfqpoint{1.955978in}{2.065593in}}%
\pgfpathclose%
\pgfusepath{stroke,fill}%
\end{pgfscope}%
\begin{pgfscope}%
\pgfpathrectangle{\pgfqpoint{0.100000in}{0.212622in}}{\pgfqpoint{3.696000in}{3.696000in}}%
\pgfusepath{clip}%
\pgfsetbuttcap%
\pgfsetroundjoin%
\definecolor{currentfill}{rgb}{0.121569,0.466667,0.705882}%
\pgfsetfillcolor{currentfill}%
\pgfsetfillopacity{0.321725}%
\pgfsetlinewidth{1.003750pt}%
\definecolor{currentstroke}{rgb}{0.121569,0.466667,0.705882}%
\pgfsetstrokecolor{currentstroke}%
\pgfsetstrokeopacity{0.321725}%
\pgfsetdash{}{0pt}%
\pgfpathmoveto{\pgfqpoint{1.867524in}{2.037232in}}%
\pgfpathcurveto{\pgfqpoint{1.875760in}{2.037232in}}{\pgfqpoint{1.883660in}{2.040504in}}{\pgfqpoint{1.889484in}{2.046328in}}%
\pgfpathcurveto{\pgfqpoint{1.895308in}{2.052152in}}{\pgfqpoint{1.898581in}{2.060052in}}{\pgfqpoint{1.898581in}{2.068289in}}%
\pgfpathcurveto{\pgfqpoint{1.898581in}{2.076525in}}{\pgfqpoint{1.895308in}{2.084425in}}{\pgfqpoint{1.889484in}{2.090249in}}%
\pgfpathcurveto{\pgfqpoint{1.883660in}{2.096073in}}{\pgfqpoint{1.875760in}{2.099345in}}{\pgfqpoint{1.867524in}{2.099345in}}%
\pgfpathcurveto{\pgfqpoint{1.859288in}{2.099345in}}{\pgfqpoint{1.851388in}{2.096073in}}{\pgfqpoint{1.845564in}{2.090249in}}%
\pgfpathcurveto{\pgfqpoint{1.839740in}{2.084425in}}{\pgfqpoint{1.836468in}{2.076525in}}{\pgfqpoint{1.836468in}{2.068289in}}%
\pgfpathcurveto{\pgfqpoint{1.836468in}{2.060052in}}{\pgfqpoint{1.839740in}{2.052152in}}{\pgfqpoint{1.845564in}{2.046328in}}%
\pgfpathcurveto{\pgfqpoint{1.851388in}{2.040504in}}{\pgfqpoint{1.859288in}{2.037232in}}{\pgfqpoint{1.867524in}{2.037232in}}%
\pgfpathclose%
\pgfusepath{stroke,fill}%
\end{pgfscope}%
\begin{pgfscope}%
\pgfpathrectangle{\pgfqpoint{0.100000in}{0.212622in}}{\pgfqpoint{3.696000in}{3.696000in}}%
\pgfusepath{clip}%
\pgfsetbuttcap%
\pgfsetroundjoin%
\definecolor{currentfill}{rgb}{0.121569,0.466667,0.705882}%
\pgfsetfillcolor{currentfill}%
\pgfsetfillopacity{0.323109}%
\pgfsetlinewidth{1.003750pt}%
\definecolor{currentstroke}{rgb}{0.121569,0.466667,0.705882}%
\pgfsetstrokecolor{currentstroke}%
\pgfsetstrokeopacity{0.323109}%
\pgfsetdash{}{0pt}%
\pgfpathmoveto{\pgfqpoint{1.863244in}{2.034919in}}%
\pgfpathcurveto{\pgfqpoint{1.871480in}{2.034919in}}{\pgfqpoint{1.879380in}{2.038191in}}{\pgfqpoint{1.885204in}{2.044015in}}%
\pgfpathcurveto{\pgfqpoint{1.891028in}{2.049839in}}{\pgfqpoint{1.894300in}{2.057739in}}{\pgfqpoint{1.894300in}{2.065975in}}%
\pgfpathcurveto{\pgfqpoint{1.894300in}{2.074211in}}{\pgfqpoint{1.891028in}{2.082112in}}{\pgfqpoint{1.885204in}{2.087935in}}%
\pgfpathcurveto{\pgfqpoint{1.879380in}{2.093759in}}{\pgfqpoint{1.871480in}{2.097032in}}{\pgfqpoint{1.863244in}{2.097032in}}%
\pgfpathcurveto{\pgfqpoint{1.855008in}{2.097032in}}{\pgfqpoint{1.847108in}{2.093759in}}{\pgfqpoint{1.841284in}{2.087935in}}%
\pgfpathcurveto{\pgfqpoint{1.835460in}{2.082112in}}{\pgfqpoint{1.832187in}{2.074211in}}{\pgfqpoint{1.832187in}{2.065975in}}%
\pgfpathcurveto{\pgfqpoint{1.832187in}{2.057739in}}{\pgfqpoint{1.835460in}{2.049839in}}{\pgfqpoint{1.841284in}{2.044015in}}%
\pgfpathcurveto{\pgfqpoint{1.847108in}{2.038191in}}{\pgfqpoint{1.855008in}{2.034919in}}{\pgfqpoint{1.863244in}{2.034919in}}%
\pgfpathclose%
\pgfusepath{stroke,fill}%
\end{pgfscope}%
\begin{pgfscope}%
\pgfpathrectangle{\pgfqpoint{0.100000in}{0.212622in}}{\pgfqpoint{3.696000in}{3.696000in}}%
\pgfusepath{clip}%
\pgfsetbuttcap%
\pgfsetroundjoin%
\definecolor{currentfill}{rgb}{0.121569,0.466667,0.705882}%
\pgfsetfillcolor{currentfill}%
\pgfsetfillopacity{0.324186}%
\pgfsetlinewidth{1.003750pt}%
\definecolor{currentstroke}{rgb}{0.121569,0.466667,0.705882}%
\pgfsetstrokecolor{currentstroke}%
\pgfsetstrokeopacity{0.324186}%
\pgfsetdash{}{0pt}%
\pgfpathmoveto{\pgfqpoint{1.957545in}{2.065281in}}%
\pgfpathcurveto{\pgfqpoint{1.965781in}{2.065281in}}{\pgfqpoint{1.973681in}{2.068553in}}{\pgfqpoint{1.979505in}{2.074377in}}%
\pgfpathcurveto{\pgfqpoint{1.985329in}{2.080201in}}{\pgfqpoint{1.988601in}{2.088101in}}{\pgfqpoint{1.988601in}{2.096337in}}%
\pgfpathcurveto{\pgfqpoint{1.988601in}{2.104574in}}{\pgfqpoint{1.985329in}{2.112474in}}{\pgfqpoint{1.979505in}{2.118298in}}%
\pgfpathcurveto{\pgfqpoint{1.973681in}{2.124122in}}{\pgfqpoint{1.965781in}{2.127394in}}{\pgfqpoint{1.957545in}{2.127394in}}%
\pgfpathcurveto{\pgfqpoint{1.949308in}{2.127394in}}{\pgfqpoint{1.941408in}{2.124122in}}{\pgfqpoint{1.935585in}{2.118298in}}%
\pgfpathcurveto{\pgfqpoint{1.929761in}{2.112474in}}{\pgfqpoint{1.926488in}{2.104574in}}{\pgfqpoint{1.926488in}{2.096337in}}%
\pgfpathcurveto{\pgfqpoint{1.926488in}{2.088101in}}{\pgfqpoint{1.929761in}{2.080201in}}{\pgfqpoint{1.935585in}{2.074377in}}%
\pgfpathcurveto{\pgfqpoint{1.941408in}{2.068553in}}{\pgfqpoint{1.949308in}{2.065281in}}{\pgfqpoint{1.957545in}{2.065281in}}%
\pgfpathclose%
\pgfusepath{stroke,fill}%
\end{pgfscope}%
\begin{pgfscope}%
\pgfpathrectangle{\pgfqpoint{0.100000in}{0.212622in}}{\pgfqpoint{3.696000in}{3.696000in}}%
\pgfusepath{clip}%
\pgfsetbuttcap%
\pgfsetroundjoin%
\definecolor{currentfill}{rgb}{0.121569,0.466667,0.705882}%
\pgfsetfillcolor{currentfill}%
\pgfsetfillopacity{0.325410}%
\pgfsetlinewidth{1.003750pt}%
\definecolor{currentstroke}{rgb}{0.121569,0.466667,0.705882}%
\pgfsetstrokecolor{currentstroke}%
\pgfsetstrokeopacity{0.325410}%
\pgfsetdash{}{0pt}%
\pgfpathmoveto{\pgfqpoint{1.855212in}{2.029577in}}%
\pgfpathcurveto{\pgfqpoint{1.863448in}{2.029577in}}{\pgfqpoint{1.871348in}{2.032849in}}{\pgfqpoint{1.877172in}{2.038673in}}%
\pgfpathcurveto{\pgfqpoint{1.882996in}{2.044497in}}{\pgfqpoint{1.886269in}{2.052397in}}{\pgfqpoint{1.886269in}{2.060633in}}%
\pgfpathcurveto{\pgfqpoint{1.886269in}{2.068870in}}{\pgfqpoint{1.882996in}{2.076770in}}{\pgfqpoint{1.877172in}{2.082594in}}%
\pgfpathcurveto{\pgfqpoint{1.871348in}{2.088418in}}{\pgfqpoint{1.863448in}{2.091690in}}{\pgfqpoint{1.855212in}{2.091690in}}%
\pgfpathcurveto{\pgfqpoint{1.846976in}{2.091690in}}{\pgfqpoint{1.839076in}{2.088418in}}{\pgfqpoint{1.833252in}{2.082594in}}%
\pgfpathcurveto{\pgfqpoint{1.827428in}{2.076770in}}{\pgfqpoint{1.824156in}{2.068870in}}{\pgfqpoint{1.824156in}{2.060633in}}%
\pgfpathcurveto{\pgfqpoint{1.824156in}{2.052397in}}{\pgfqpoint{1.827428in}{2.044497in}}{\pgfqpoint{1.833252in}{2.038673in}}%
\pgfpathcurveto{\pgfqpoint{1.839076in}{2.032849in}}{\pgfqpoint{1.846976in}{2.029577in}}{\pgfqpoint{1.855212in}{2.029577in}}%
\pgfpathclose%
\pgfusepath{stroke,fill}%
\end{pgfscope}%
\begin{pgfscope}%
\pgfpathrectangle{\pgfqpoint{0.100000in}{0.212622in}}{\pgfqpoint{3.696000in}{3.696000in}}%
\pgfusepath{clip}%
\pgfsetbuttcap%
\pgfsetroundjoin%
\definecolor{currentfill}{rgb}{0.121569,0.466667,0.705882}%
\pgfsetfillcolor{currentfill}%
\pgfsetfillopacity{0.326434}%
\pgfsetlinewidth{1.003750pt}%
\definecolor{currentstroke}{rgb}{0.121569,0.466667,0.705882}%
\pgfsetstrokecolor{currentstroke}%
\pgfsetstrokeopacity{0.326434}%
\pgfsetdash{}{0pt}%
\pgfpathmoveto{\pgfqpoint{1.959560in}{2.062018in}}%
\pgfpathcurveto{\pgfqpoint{1.967796in}{2.062018in}}{\pgfqpoint{1.975696in}{2.065290in}}{\pgfqpoint{1.981520in}{2.071114in}}%
\pgfpathcurveto{\pgfqpoint{1.987344in}{2.076938in}}{\pgfqpoint{1.990617in}{2.084838in}}{\pgfqpoint{1.990617in}{2.093074in}}%
\pgfpathcurveto{\pgfqpoint{1.990617in}{2.101310in}}{\pgfqpoint{1.987344in}{2.109210in}}{\pgfqpoint{1.981520in}{2.115034in}}%
\pgfpathcurveto{\pgfqpoint{1.975696in}{2.120858in}}{\pgfqpoint{1.967796in}{2.124131in}}{\pgfqpoint{1.959560in}{2.124131in}}%
\pgfpathcurveto{\pgfqpoint{1.951324in}{2.124131in}}{\pgfqpoint{1.943424in}{2.120858in}}{\pgfqpoint{1.937600in}{2.115034in}}%
\pgfpathcurveto{\pgfqpoint{1.931776in}{2.109210in}}{\pgfqpoint{1.928504in}{2.101310in}}{\pgfqpoint{1.928504in}{2.093074in}}%
\pgfpathcurveto{\pgfqpoint{1.928504in}{2.084838in}}{\pgfqpoint{1.931776in}{2.076938in}}{\pgfqpoint{1.937600in}{2.071114in}}%
\pgfpathcurveto{\pgfqpoint{1.943424in}{2.065290in}}{\pgfqpoint{1.951324in}{2.062018in}}{\pgfqpoint{1.959560in}{2.062018in}}%
\pgfpathclose%
\pgfusepath{stroke,fill}%
\end{pgfscope}%
\begin{pgfscope}%
\pgfpathrectangle{\pgfqpoint{0.100000in}{0.212622in}}{\pgfqpoint{3.696000in}{3.696000in}}%
\pgfusepath{clip}%
\pgfsetbuttcap%
\pgfsetroundjoin%
\definecolor{currentfill}{rgb}{0.121569,0.466667,0.705882}%
\pgfsetfillcolor{currentfill}%
\pgfsetfillopacity{0.326885}%
\pgfsetlinewidth{1.003750pt}%
\definecolor{currentstroke}{rgb}{0.121569,0.466667,0.705882}%
\pgfsetstrokecolor{currentstroke}%
\pgfsetstrokeopacity{0.326885}%
\pgfsetdash{}{0pt}%
\pgfpathmoveto{\pgfqpoint{1.849686in}{2.026120in}}%
\pgfpathcurveto{\pgfqpoint{1.857923in}{2.026120in}}{\pgfqpoint{1.865823in}{2.029392in}}{\pgfqpoint{1.871647in}{2.035216in}}%
\pgfpathcurveto{\pgfqpoint{1.877471in}{2.041040in}}{\pgfqpoint{1.880743in}{2.048940in}}{\pgfqpoint{1.880743in}{2.057177in}}%
\pgfpathcurveto{\pgfqpoint{1.880743in}{2.065413in}}{\pgfqpoint{1.877471in}{2.073313in}}{\pgfqpoint{1.871647in}{2.079137in}}%
\pgfpathcurveto{\pgfqpoint{1.865823in}{2.084961in}}{\pgfqpoint{1.857923in}{2.088233in}}{\pgfqpoint{1.849686in}{2.088233in}}%
\pgfpathcurveto{\pgfqpoint{1.841450in}{2.088233in}}{\pgfqpoint{1.833550in}{2.084961in}}{\pgfqpoint{1.827726in}{2.079137in}}%
\pgfpathcurveto{\pgfqpoint{1.821902in}{2.073313in}}{\pgfqpoint{1.818630in}{2.065413in}}{\pgfqpoint{1.818630in}{2.057177in}}%
\pgfpathcurveto{\pgfqpoint{1.818630in}{2.048940in}}{\pgfqpoint{1.821902in}{2.041040in}}{\pgfqpoint{1.827726in}{2.035216in}}%
\pgfpathcurveto{\pgfqpoint{1.833550in}{2.029392in}}{\pgfqpoint{1.841450in}{2.026120in}}{\pgfqpoint{1.849686in}{2.026120in}}%
\pgfpathclose%
\pgfusepath{stroke,fill}%
\end{pgfscope}%
\begin{pgfscope}%
\pgfpathrectangle{\pgfqpoint{0.100000in}{0.212622in}}{\pgfqpoint{3.696000in}{3.696000in}}%
\pgfusepath{clip}%
\pgfsetbuttcap%
\pgfsetroundjoin%
\definecolor{currentfill}{rgb}{0.121569,0.466667,0.705882}%
\pgfsetfillcolor{currentfill}%
\pgfsetfillopacity{0.327832}%
\pgfsetlinewidth{1.003750pt}%
\definecolor{currentstroke}{rgb}{0.121569,0.466667,0.705882}%
\pgfsetstrokecolor{currentstroke}%
\pgfsetstrokeopacity{0.327832}%
\pgfsetdash{}{0pt}%
\pgfpathmoveto{\pgfqpoint{1.846513in}{2.024583in}}%
\pgfpathcurveto{\pgfqpoint{1.854749in}{2.024583in}}{\pgfqpoint{1.862650in}{2.027855in}}{\pgfqpoint{1.868473in}{2.033679in}}%
\pgfpathcurveto{\pgfqpoint{1.874297in}{2.039503in}}{\pgfqpoint{1.877570in}{2.047403in}}{\pgfqpoint{1.877570in}{2.055640in}}%
\pgfpathcurveto{\pgfqpoint{1.877570in}{2.063876in}}{\pgfqpoint{1.874297in}{2.071776in}}{\pgfqpoint{1.868473in}{2.077600in}}%
\pgfpathcurveto{\pgfqpoint{1.862650in}{2.083424in}}{\pgfqpoint{1.854749in}{2.086696in}}{\pgfqpoint{1.846513in}{2.086696in}}%
\pgfpathcurveto{\pgfqpoint{1.838277in}{2.086696in}}{\pgfqpoint{1.830377in}{2.083424in}}{\pgfqpoint{1.824553in}{2.077600in}}%
\pgfpathcurveto{\pgfqpoint{1.818729in}{2.071776in}}{\pgfqpoint{1.815457in}{2.063876in}}{\pgfqpoint{1.815457in}{2.055640in}}%
\pgfpathcurveto{\pgfqpoint{1.815457in}{2.047403in}}{\pgfqpoint{1.818729in}{2.039503in}}{\pgfqpoint{1.824553in}{2.033679in}}%
\pgfpathcurveto{\pgfqpoint{1.830377in}{2.027855in}}{\pgfqpoint{1.838277in}{2.024583in}}{\pgfqpoint{1.846513in}{2.024583in}}%
\pgfpathclose%
\pgfusepath{stroke,fill}%
\end{pgfscope}%
\begin{pgfscope}%
\pgfpathrectangle{\pgfqpoint{0.100000in}{0.212622in}}{\pgfqpoint{3.696000in}{3.696000in}}%
\pgfusepath{clip}%
\pgfsetbuttcap%
\pgfsetroundjoin%
\definecolor{currentfill}{rgb}{0.121569,0.466667,0.705882}%
\pgfsetfillcolor{currentfill}%
\pgfsetfillopacity{0.329249}%
\pgfsetlinewidth{1.003750pt}%
\definecolor{currentstroke}{rgb}{0.121569,0.466667,0.705882}%
\pgfsetstrokecolor{currentstroke}%
\pgfsetstrokeopacity{0.329249}%
\pgfsetdash{}{0pt}%
\pgfpathmoveto{\pgfqpoint{1.961767in}{2.061086in}}%
\pgfpathcurveto{\pgfqpoint{1.970004in}{2.061086in}}{\pgfqpoint{1.977904in}{2.064359in}}{\pgfqpoint{1.983728in}{2.070183in}}%
\pgfpathcurveto{\pgfqpoint{1.989552in}{2.076006in}}{\pgfqpoint{1.992824in}{2.083907in}}{\pgfqpoint{1.992824in}{2.092143in}}%
\pgfpathcurveto{\pgfqpoint{1.992824in}{2.100379in}}{\pgfqpoint{1.989552in}{2.108279in}}{\pgfqpoint{1.983728in}{2.114103in}}%
\pgfpathcurveto{\pgfqpoint{1.977904in}{2.119927in}}{\pgfqpoint{1.970004in}{2.123199in}}{\pgfqpoint{1.961767in}{2.123199in}}%
\pgfpathcurveto{\pgfqpoint{1.953531in}{2.123199in}}{\pgfqpoint{1.945631in}{2.119927in}}{\pgfqpoint{1.939807in}{2.114103in}}%
\pgfpathcurveto{\pgfqpoint{1.933983in}{2.108279in}}{\pgfqpoint{1.930711in}{2.100379in}}{\pgfqpoint{1.930711in}{2.092143in}}%
\pgfpathcurveto{\pgfqpoint{1.930711in}{2.083907in}}{\pgfqpoint{1.933983in}{2.076006in}}{\pgfqpoint{1.939807in}{2.070183in}}%
\pgfpathcurveto{\pgfqpoint{1.945631in}{2.064359in}}{\pgfqpoint{1.953531in}{2.061086in}}{\pgfqpoint{1.961767in}{2.061086in}}%
\pgfpathclose%
\pgfusepath{stroke,fill}%
\end{pgfscope}%
\begin{pgfscope}%
\pgfpathrectangle{\pgfqpoint{0.100000in}{0.212622in}}{\pgfqpoint{3.696000in}{3.696000in}}%
\pgfusepath{clip}%
\pgfsetbuttcap%
\pgfsetroundjoin%
\definecolor{currentfill}{rgb}{0.121569,0.466667,0.705882}%
\pgfsetfillcolor{currentfill}%
\pgfsetfillopacity{0.329476}%
\pgfsetlinewidth{1.003750pt}%
\definecolor{currentstroke}{rgb}{0.121569,0.466667,0.705882}%
\pgfsetstrokecolor{currentstroke}%
\pgfsetstrokeopacity{0.329476}%
\pgfsetdash{}{0pt}%
\pgfpathmoveto{\pgfqpoint{1.841109in}{2.020676in}}%
\pgfpathcurveto{\pgfqpoint{1.849346in}{2.020676in}}{\pgfqpoint{1.857246in}{2.023948in}}{\pgfqpoint{1.863070in}{2.029772in}}%
\pgfpathcurveto{\pgfqpoint{1.868894in}{2.035596in}}{\pgfqpoint{1.872166in}{2.043496in}}{\pgfqpoint{1.872166in}{2.051732in}}%
\pgfpathcurveto{\pgfqpoint{1.872166in}{2.059969in}}{\pgfqpoint{1.868894in}{2.067869in}}{\pgfqpoint{1.863070in}{2.073693in}}%
\pgfpathcurveto{\pgfqpoint{1.857246in}{2.079517in}}{\pgfqpoint{1.849346in}{2.082789in}}{\pgfqpoint{1.841109in}{2.082789in}}%
\pgfpathcurveto{\pgfqpoint{1.832873in}{2.082789in}}{\pgfqpoint{1.824973in}{2.079517in}}{\pgfqpoint{1.819149in}{2.073693in}}%
\pgfpathcurveto{\pgfqpoint{1.813325in}{2.067869in}}{\pgfqpoint{1.810053in}{2.059969in}}{\pgfqpoint{1.810053in}{2.051732in}}%
\pgfpathcurveto{\pgfqpoint{1.810053in}{2.043496in}}{\pgfqpoint{1.813325in}{2.035596in}}{\pgfqpoint{1.819149in}{2.029772in}}%
\pgfpathcurveto{\pgfqpoint{1.824973in}{2.023948in}}{\pgfqpoint{1.832873in}{2.020676in}}{\pgfqpoint{1.841109in}{2.020676in}}%
\pgfpathclose%
\pgfusepath{stroke,fill}%
\end{pgfscope}%
\begin{pgfscope}%
\pgfpathrectangle{\pgfqpoint{0.100000in}{0.212622in}}{\pgfqpoint{3.696000in}{3.696000in}}%
\pgfusepath{clip}%
\pgfsetbuttcap%
\pgfsetroundjoin%
\definecolor{currentfill}{rgb}{0.121569,0.466667,0.705882}%
\pgfsetfillcolor{currentfill}%
\pgfsetfillopacity{0.330463}%
\pgfsetlinewidth{1.003750pt}%
\definecolor{currentstroke}{rgb}{0.121569,0.466667,0.705882}%
\pgfsetstrokecolor{currentstroke}%
\pgfsetstrokeopacity{0.330463}%
\pgfsetdash{}{0pt}%
\pgfpathmoveto{\pgfqpoint{1.837794in}{2.018838in}}%
\pgfpathcurveto{\pgfqpoint{1.846030in}{2.018838in}}{\pgfqpoint{1.853931in}{2.022110in}}{\pgfqpoint{1.859754in}{2.027934in}}%
\pgfpathcurveto{\pgfqpoint{1.865578in}{2.033758in}}{\pgfqpoint{1.868851in}{2.041658in}}{\pgfqpoint{1.868851in}{2.049894in}}%
\pgfpathcurveto{\pgfqpoint{1.868851in}{2.058130in}}{\pgfqpoint{1.865578in}{2.066030in}}{\pgfqpoint{1.859754in}{2.071854in}}%
\pgfpathcurveto{\pgfqpoint{1.853931in}{2.077678in}}{\pgfqpoint{1.846030in}{2.080951in}}{\pgfqpoint{1.837794in}{2.080951in}}%
\pgfpathcurveto{\pgfqpoint{1.829558in}{2.080951in}}{\pgfqpoint{1.821658in}{2.077678in}}{\pgfqpoint{1.815834in}{2.071854in}}%
\pgfpathcurveto{\pgfqpoint{1.810010in}{2.066030in}}{\pgfqpoint{1.806738in}{2.058130in}}{\pgfqpoint{1.806738in}{2.049894in}}%
\pgfpathcurveto{\pgfqpoint{1.806738in}{2.041658in}}{\pgfqpoint{1.810010in}{2.033758in}}{\pgfqpoint{1.815834in}{2.027934in}}%
\pgfpathcurveto{\pgfqpoint{1.821658in}{2.022110in}}{\pgfqpoint{1.829558in}{2.018838in}}{\pgfqpoint{1.837794in}{2.018838in}}%
\pgfpathclose%
\pgfusepath{stroke,fill}%
\end{pgfscope}%
\begin{pgfscope}%
\pgfpathrectangle{\pgfqpoint{0.100000in}{0.212622in}}{\pgfqpoint{3.696000in}{3.696000in}}%
\pgfusepath{clip}%
\pgfsetbuttcap%
\pgfsetroundjoin%
\definecolor{currentfill}{rgb}{0.121569,0.466667,0.705882}%
\pgfsetfillcolor{currentfill}%
\pgfsetfillopacity{0.330944}%
\pgfsetlinewidth{1.003750pt}%
\definecolor{currentstroke}{rgb}{0.121569,0.466667,0.705882}%
\pgfsetstrokecolor{currentstroke}%
\pgfsetstrokeopacity{0.330944}%
\pgfsetdash{}{0pt}%
\pgfpathmoveto{\pgfqpoint{1.963043in}{2.061626in}}%
\pgfpathcurveto{\pgfqpoint{1.971279in}{2.061626in}}{\pgfqpoint{1.979179in}{2.064898in}}{\pgfqpoint{1.985003in}{2.070722in}}%
\pgfpathcurveto{\pgfqpoint{1.990827in}{2.076546in}}{\pgfqpoint{1.994100in}{2.084446in}}{\pgfqpoint{1.994100in}{2.092682in}}%
\pgfpathcurveto{\pgfqpoint{1.994100in}{2.100918in}}{\pgfqpoint{1.990827in}{2.108819in}}{\pgfqpoint{1.985003in}{2.114642in}}%
\pgfpathcurveto{\pgfqpoint{1.979179in}{2.120466in}}{\pgfqpoint{1.971279in}{2.123739in}}{\pgfqpoint{1.963043in}{2.123739in}}%
\pgfpathcurveto{\pgfqpoint{1.954807in}{2.123739in}}{\pgfqpoint{1.946907in}{2.120466in}}{\pgfqpoint{1.941083in}{2.114642in}}%
\pgfpathcurveto{\pgfqpoint{1.935259in}{2.108819in}}{\pgfqpoint{1.931987in}{2.100918in}}{\pgfqpoint{1.931987in}{2.092682in}}%
\pgfpathcurveto{\pgfqpoint{1.931987in}{2.084446in}}{\pgfqpoint{1.935259in}{2.076546in}}{\pgfqpoint{1.941083in}{2.070722in}}%
\pgfpathcurveto{\pgfqpoint{1.946907in}{2.064898in}}{\pgfqpoint{1.954807in}{2.061626in}}{\pgfqpoint{1.963043in}{2.061626in}}%
\pgfpathclose%
\pgfusepath{stroke,fill}%
\end{pgfscope}%
\begin{pgfscope}%
\pgfpathrectangle{\pgfqpoint{0.100000in}{0.212622in}}{\pgfqpoint{3.696000in}{3.696000in}}%
\pgfusepath{clip}%
\pgfsetbuttcap%
\pgfsetroundjoin%
\definecolor{currentfill}{rgb}{0.121569,0.466667,0.705882}%
\pgfsetfillcolor{currentfill}%
\pgfsetfillopacity{0.331419}%
\pgfsetlinewidth{1.003750pt}%
\definecolor{currentstroke}{rgb}{0.121569,0.466667,0.705882}%
\pgfsetstrokecolor{currentstroke}%
\pgfsetstrokeopacity{0.331419}%
\pgfsetdash{}{0pt}%
\pgfpathmoveto{\pgfqpoint{1.835078in}{2.017988in}}%
\pgfpathcurveto{\pgfqpoint{1.843314in}{2.017988in}}{\pgfqpoint{1.851214in}{2.021260in}}{\pgfqpoint{1.857038in}{2.027084in}}%
\pgfpathcurveto{\pgfqpoint{1.862862in}{2.032908in}}{\pgfqpoint{1.866134in}{2.040808in}}{\pgfqpoint{1.866134in}{2.049045in}}%
\pgfpathcurveto{\pgfqpoint{1.866134in}{2.057281in}}{\pgfqpoint{1.862862in}{2.065181in}}{\pgfqpoint{1.857038in}{2.071005in}}%
\pgfpathcurveto{\pgfqpoint{1.851214in}{2.076829in}}{\pgfqpoint{1.843314in}{2.080101in}}{\pgfqpoint{1.835078in}{2.080101in}}%
\pgfpathcurveto{\pgfqpoint{1.826842in}{2.080101in}}{\pgfqpoint{1.818942in}{2.076829in}}{\pgfqpoint{1.813118in}{2.071005in}}%
\pgfpathcurveto{\pgfqpoint{1.807294in}{2.065181in}}{\pgfqpoint{1.804021in}{2.057281in}}{\pgfqpoint{1.804021in}{2.049045in}}%
\pgfpathcurveto{\pgfqpoint{1.804021in}{2.040808in}}{\pgfqpoint{1.807294in}{2.032908in}}{\pgfqpoint{1.813118in}{2.027084in}}%
\pgfpathcurveto{\pgfqpoint{1.818942in}{2.021260in}}{\pgfqpoint{1.826842in}{2.017988in}}{\pgfqpoint{1.835078in}{2.017988in}}%
\pgfpathclose%
\pgfusepath{stroke,fill}%
\end{pgfscope}%
\begin{pgfscope}%
\pgfpathrectangle{\pgfqpoint{0.100000in}{0.212622in}}{\pgfqpoint{3.696000in}{3.696000in}}%
\pgfusepath{clip}%
\pgfsetbuttcap%
\pgfsetroundjoin%
\definecolor{currentfill}{rgb}{0.121569,0.466667,0.705882}%
\pgfsetfillcolor{currentfill}%
\pgfsetfillopacity{0.332800}%
\pgfsetlinewidth{1.003750pt}%
\definecolor{currentstroke}{rgb}{0.121569,0.466667,0.705882}%
\pgfsetstrokecolor{currentstroke}%
\pgfsetstrokeopacity{0.332800}%
\pgfsetdash{}{0pt}%
\pgfpathmoveto{\pgfqpoint{1.964748in}{2.060557in}}%
\pgfpathcurveto{\pgfqpoint{1.972985in}{2.060557in}}{\pgfqpoint{1.980885in}{2.063829in}}{\pgfqpoint{1.986709in}{2.069653in}}%
\pgfpathcurveto{\pgfqpoint{1.992533in}{2.075477in}}{\pgfqpoint{1.995805in}{2.083377in}}{\pgfqpoint{1.995805in}{2.091613in}}%
\pgfpathcurveto{\pgfqpoint{1.995805in}{2.099850in}}{\pgfqpoint{1.992533in}{2.107750in}}{\pgfqpoint{1.986709in}{2.113574in}}%
\pgfpathcurveto{\pgfqpoint{1.980885in}{2.119397in}}{\pgfqpoint{1.972985in}{2.122670in}}{\pgfqpoint{1.964748in}{2.122670in}}%
\pgfpathcurveto{\pgfqpoint{1.956512in}{2.122670in}}{\pgfqpoint{1.948612in}{2.119397in}}{\pgfqpoint{1.942788in}{2.113574in}}%
\pgfpathcurveto{\pgfqpoint{1.936964in}{2.107750in}}{\pgfqpoint{1.933692in}{2.099850in}}{\pgfqpoint{1.933692in}{2.091613in}}%
\pgfpathcurveto{\pgfqpoint{1.933692in}{2.083377in}}{\pgfqpoint{1.936964in}{2.075477in}}{\pgfqpoint{1.942788in}{2.069653in}}%
\pgfpathcurveto{\pgfqpoint{1.948612in}{2.063829in}}{\pgfqpoint{1.956512in}{2.060557in}}{\pgfqpoint{1.964748in}{2.060557in}}%
\pgfpathclose%
\pgfusepath{stroke,fill}%
\end{pgfscope}%
\begin{pgfscope}%
\pgfpathrectangle{\pgfqpoint{0.100000in}{0.212622in}}{\pgfqpoint{3.696000in}{3.696000in}}%
\pgfusepath{clip}%
\pgfsetbuttcap%
\pgfsetroundjoin%
\definecolor{currentfill}{rgb}{0.121569,0.466667,0.705882}%
\pgfsetfillcolor{currentfill}%
\pgfsetfillopacity{0.332828}%
\pgfsetlinewidth{1.003750pt}%
\definecolor{currentstroke}{rgb}{0.121569,0.466667,0.705882}%
\pgfsetstrokecolor{currentstroke}%
\pgfsetstrokeopacity{0.332828}%
\pgfsetdash{}{0pt}%
\pgfpathmoveto{\pgfqpoint{1.830297in}{2.013941in}}%
\pgfpathcurveto{\pgfqpoint{1.838534in}{2.013941in}}{\pgfqpoint{1.846434in}{2.017213in}}{\pgfqpoint{1.852258in}{2.023037in}}%
\pgfpathcurveto{\pgfqpoint{1.858081in}{2.028861in}}{\pgfqpoint{1.861354in}{2.036761in}}{\pgfqpoint{1.861354in}{2.044997in}}%
\pgfpathcurveto{\pgfqpoint{1.861354in}{2.053234in}}{\pgfqpoint{1.858081in}{2.061134in}}{\pgfqpoint{1.852258in}{2.066958in}}%
\pgfpathcurveto{\pgfqpoint{1.846434in}{2.072782in}}{\pgfqpoint{1.838534in}{2.076054in}}{\pgfqpoint{1.830297in}{2.076054in}}%
\pgfpathcurveto{\pgfqpoint{1.822061in}{2.076054in}}{\pgfqpoint{1.814161in}{2.072782in}}{\pgfqpoint{1.808337in}{2.066958in}}%
\pgfpathcurveto{\pgfqpoint{1.802513in}{2.061134in}}{\pgfqpoint{1.799241in}{2.053234in}}{\pgfqpoint{1.799241in}{2.044997in}}%
\pgfpathcurveto{\pgfqpoint{1.799241in}{2.036761in}}{\pgfqpoint{1.802513in}{2.028861in}}{\pgfqpoint{1.808337in}{2.023037in}}%
\pgfpathcurveto{\pgfqpoint{1.814161in}{2.017213in}}{\pgfqpoint{1.822061in}{2.013941in}}{\pgfqpoint{1.830297in}{2.013941in}}%
\pgfpathclose%
\pgfusepath{stroke,fill}%
\end{pgfscope}%
\begin{pgfscope}%
\pgfpathrectangle{\pgfqpoint{0.100000in}{0.212622in}}{\pgfqpoint{3.696000in}{3.696000in}}%
\pgfusepath{clip}%
\pgfsetbuttcap%
\pgfsetroundjoin%
\definecolor{currentfill}{rgb}{0.121569,0.466667,0.705882}%
\pgfsetfillcolor{currentfill}%
\pgfsetfillopacity{0.333954}%
\pgfsetlinewidth{1.003750pt}%
\definecolor{currentstroke}{rgb}{0.121569,0.466667,0.705882}%
\pgfsetstrokecolor{currentstroke}%
\pgfsetstrokeopacity{0.333954}%
\pgfsetdash{}{0pt}%
\pgfpathmoveto{\pgfqpoint{1.826652in}{2.012904in}}%
\pgfpathcurveto{\pgfqpoint{1.834888in}{2.012904in}}{\pgfqpoint{1.842788in}{2.016176in}}{\pgfqpoint{1.848612in}{2.022000in}}%
\pgfpathcurveto{\pgfqpoint{1.854436in}{2.027824in}}{\pgfqpoint{1.857708in}{2.035724in}}{\pgfqpoint{1.857708in}{2.043960in}}%
\pgfpathcurveto{\pgfqpoint{1.857708in}{2.052197in}}{\pgfqpoint{1.854436in}{2.060097in}}{\pgfqpoint{1.848612in}{2.065920in}}%
\pgfpathcurveto{\pgfqpoint{1.842788in}{2.071744in}}{\pgfqpoint{1.834888in}{2.075017in}}{\pgfqpoint{1.826652in}{2.075017in}}%
\pgfpathcurveto{\pgfqpoint{1.818415in}{2.075017in}}{\pgfqpoint{1.810515in}{2.071744in}}{\pgfqpoint{1.804691in}{2.065920in}}%
\pgfpathcurveto{\pgfqpoint{1.798867in}{2.060097in}}{\pgfqpoint{1.795595in}{2.052197in}}{\pgfqpoint{1.795595in}{2.043960in}}%
\pgfpathcurveto{\pgfqpoint{1.795595in}{2.035724in}}{\pgfqpoint{1.798867in}{2.027824in}}{\pgfqpoint{1.804691in}{2.022000in}}%
\pgfpathcurveto{\pgfqpoint{1.810515in}{2.016176in}}{\pgfqpoint{1.818415in}{2.012904in}}{\pgfqpoint{1.826652in}{2.012904in}}%
\pgfpathclose%
\pgfusepath{stroke,fill}%
\end{pgfscope}%
\begin{pgfscope}%
\pgfpathrectangle{\pgfqpoint{0.100000in}{0.212622in}}{\pgfqpoint{3.696000in}{3.696000in}}%
\pgfusepath{clip}%
\pgfsetbuttcap%
\pgfsetroundjoin%
\definecolor{currentfill}{rgb}{0.121569,0.466667,0.705882}%
\pgfsetfillcolor{currentfill}%
\pgfsetfillopacity{0.334911}%
\pgfsetlinewidth{1.003750pt}%
\definecolor{currentstroke}{rgb}{0.121569,0.466667,0.705882}%
\pgfsetstrokecolor{currentstroke}%
\pgfsetstrokeopacity{0.334911}%
\pgfsetdash{}{0pt}%
\pgfpathmoveto{\pgfqpoint{1.824062in}{2.011765in}}%
\pgfpathcurveto{\pgfqpoint{1.832298in}{2.011765in}}{\pgfqpoint{1.840198in}{2.015037in}}{\pgfqpoint{1.846022in}{2.020861in}}%
\pgfpathcurveto{\pgfqpoint{1.851846in}{2.026685in}}{\pgfqpoint{1.855118in}{2.034585in}}{\pgfqpoint{1.855118in}{2.042822in}}%
\pgfpathcurveto{\pgfqpoint{1.855118in}{2.051058in}}{\pgfqpoint{1.851846in}{2.058958in}}{\pgfqpoint{1.846022in}{2.064782in}}%
\pgfpathcurveto{\pgfqpoint{1.840198in}{2.070606in}}{\pgfqpoint{1.832298in}{2.073878in}}{\pgfqpoint{1.824062in}{2.073878in}}%
\pgfpathcurveto{\pgfqpoint{1.815825in}{2.073878in}}{\pgfqpoint{1.807925in}{2.070606in}}{\pgfqpoint{1.802101in}{2.064782in}}%
\pgfpathcurveto{\pgfqpoint{1.796277in}{2.058958in}}{\pgfqpoint{1.793005in}{2.051058in}}{\pgfqpoint{1.793005in}{2.042822in}}%
\pgfpathcurveto{\pgfqpoint{1.793005in}{2.034585in}}{\pgfqpoint{1.796277in}{2.026685in}}{\pgfqpoint{1.802101in}{2.020861in}}%
\pgfpathcurveto{\pgfqpoint{1.807925in}{2.015037in}}{\pgfqpoint{1.815825in}{2.011765in}}{\pgfqpoint{1.824062in}{2.011765in}}%
\pgfpathclose%
\pgfusepath{stroke,fill}%
\end{pgfscope}%
\begin{pgfscope}%
\pgfpathrectangle{\pgfqpoint{0.100000in}{0.212622in}}{\pgfqpoint{3.696000in}{3.696000in}}%
\pgfusepath{clip}%
\pgfsetbuttcap%
\pgfsetroundjoin%
\definecolor{currentfill}{rgb}{0.121569,0.466667,0.705882}%
\pgfsetfillcolor{currentfill}%
\pgfsetfillopacity{0.335153}%
\pgfsetlinewidth{1.003750pt}%
\definecolor{currentstroke}{rgb}{0.121569,0.466667,0.705882}%
\pgfsetstrokecolor{currentstroke}%
\pgfsetstrokeopacity{0.335153}%
\pgfsetdash{}{0pt}%
\pgfpathmoveto{\pgfqpoint{1.966845in}{2.059560in}}%
\pgfpathcurveto{\pgfqpoint{1.975081in}{2.059560in}}{\pgfqpoint{1.982981in}{2.062833in}}{\pgfqpoint{1.988805in}{2.068657in}}%
\pgfpathcurveto{\pgfqpoint{1.994629in}{2.074481in}}{\pgfqpoint{1.997901in}{2.082381in}}{\pgfqpoint{1.997901in}{2.090617in}}%
\pgfpathcurveto{\pgfqpoint{1.997901in}{2.098853in}}{\pgfqpoint{1.994629in}{2.106753in}}{\pgfqpoint{1.988805in}{2.112577in}}%
\pgfpathcurveto{\pgfqpoint{1.982981in}{2.118401in}}{\pgfqpoint{1.975081in}{2.121673in}}{\pgfqpoint{1.966845in}{2.121673in}}%
\pgfpathcurveto{\pgfqpoint{1.958608in}{2.121673in}}{\pgfqpoint{1.950708in}{2.118401in}}{\pgfqpoint{1.944884in}{2.112577in}}%
\pgfpathcurveto{\pgfqpoint{1.939060in}{2.106753in}}{\pgfqpoint{1.935788in}{2.098853in}}{\pgfqpoint{1.935788in}{2.090617in}}%
\pgfpathcurveto{\pgfqpoint{1.935788in}{2.082381in}}{\pgfqpoint{1.939060in}{2.074481in}}{\pgfqpoint{1.944884in}{2.068657in}}%
\pgfpathcurveto{\pgfqpoint{1.950708in}{2.062833in}}{\pgfqpoint{1.958608in}{2.059560in}}{\pgfqpoint{1.966845in}{2.059560in}}%
\pgfpathclose%
\pgfusepath{stroke,fill}%
\end{pgfscope}%
\begin{pgfscope}%
\pgfpathrectangle{\pgfqpoint{0.100000in}{0.212622in}}{\pgfqpoint{3.696000in}{3.696000in}}%
\pgfusepath{clip}%
\pgfsetbuttcap%
\pgfsetroundjoin%
\definecolor{currentfill}{rgb}{0.121569,0.466667,0.705882}%
\pgfsetfillcolor{currentfill}%
\pgfsetfillopacity{0.336188}%
\pgfsetlinewidth{1.003750pt}%
\definecolor{currentstroke}{rgb}{0.121569,0.466667,0.705882}%
\pgfsetstrokecolor{currentstroke}%
\pgfsetstrokeopacity{0.336188}%
\pgfsetdash{}{0pt}%
\pgfpathmoveto{\pgfqpoint{1.819117in}{2.006827in}}%
\pgfpathcurveto{\pgfqpoint{1.827354in}{2.006827in}}{\pgfqpoint{1.835254in}{2.010100in}}{\pgfqpoint{1.841078in}{2.015924in}}%
\pgfpathcurveto{\pgfqpoint{1.846902in}{2.021747in}}{\pgfqpoint{1.850174in}{2.029647in}}{\pgfqpoint{1.850174in}{2.037884in}}%
\pgfpathcurveto{\pgfqpoint{1.850174in}{2.046120in}}{\pgfqpoint{1.846902in}{2.054020in}}{\pgfqpoint{1.841078in}{2.059844in}}%
\pgfpathcurveto{\pgfqpoint{1.835254in}{2.065668in}}{\pgfqpoint{1.827354in}{2.068940in}}{\pgfqpoint{1.819117in}{2.068940in}}%
\pgfpathcurveto{\pgfqpoint{1.810881in}{2.068940in}}{\pgfqpoint{1.802981in}{2.065668in}}{\pgfqpoint{1.797157in}{2.059844in}}%
\pgfpathcurveto{\pgfqpoint{1.791333in}{2.054020in}}{\pgfqpoint{1.788061in}{2.046120in}}{\pgfqpoint{1.788061in}{2.037884in}}%
\pgfpathcurveto{\pgfqpoint{1.788061in}{2.029647in}}{\pgfqpoint{1.791333in}{2.021747in}}{\pgfqpoint{1.797157in}{2.015924in}}%
\pgfpathcurveto{\pgfqpoint{1.802981in}{2.010100in}}{\pgfqpoint{1.810881in}{2.006827in}}{\pgfqpoint{1.819117in}{2.006827in}}%
\pgfpathclose%
\pgfusepath{stroke,fill}%
\end{pgfscope}%
\begin{pgfscope}%
\pgfpathrectangle{\pgfqpoint{0.100000in}{0.212622in}}{\pgfqpoint{3.696000in}{3.696000in}}%
\pgfusepath{clip}%
\pgfsetbuttcap%
\pgfsetroundjoin%
\definecolor{currentfill}{rgb}{0.121569,0.466667,0.705882}%
\pgfsetfillcolor{currentfill}%
\pgfsetfillopacity{0.337259}%
\pgfsetlinewidth{1.003750pt}%
\definecolor{currentstroke}{rgb}{0.121569,0.466667,0.705882}%
\pgfsetstrokecolor{currentstroke}%
\pgfsetstrokeopacity{0.337259}%
\pgfsetdash{}{0pt}%
\pgfpathmoveto{\pgfqpoint{1.815596in}{2.004913in}}%
\pgfpathcurveto{\pgfqpoint{1.823833in}{2.004913in}}{\pgfqpoint{1.831733in}{2.008185in}}{\pgfqpoint{1.837556in}{2.014009in}}%
\pgfpathcurveto{\pgfqpoint{1.843380in}{2.019833in}}{\pgfqpoint{1.846653in}{2.027733in}}{\pgfqpoint{1.846653in}{2.035969in}}%
\pgfpathcurveto{\pgfqpoint{1.846653in}{2.044205in}}{\pgfqpoint{1.843380in}{2.052105in}}{\pgfqpoint{1.837556in}{2.057929in}}%
\pgfpathcurveto{\pgfqpoint{1.831733in}{2.063753in}}{\pgfqpoint{1.823833in}{2.067026in}}{\pgfqpoint{1.815596in}{2.067026in}}%
\pgfpathcurveto{\pgfqpoint{1.807360in}{2.067026in}}{\pgfqpoint{1.799460in}{2.063753in}}{\pgfqpoint{1.793636in}{2.057929in}}%
\pgfpathcurveto{\pgfqpoint{1.787812in}{2.052105in}}{\pgfqpoint{1.784540in}{2.044205in}}{\pgfqpoint{1.784540in}{2.035969in}}%
\pgfpathcurveto{\pgfqpoint{1.784540in}{2.027733in}}{\pgfqpoint{1.787812in}{2.019833in}}{\pgfqpoint{1.793636in}{2.014009in}}%
\pgfpathcurveto{\pgfqpoint{1.799460in}{2.008185in}}{\pgfqpoint{1.807360in}{2.004913in}}{\pgfqpoint{1.815596in}{2.004913in}}%
\pgfpathclose%
\pgfusepath{stroke,fill}%
\end{pgfscope}%
\begin{pgfscope}%
\pgfpathrectangle{\pgfqpoint{0.100000in}{0.212622in}}{\pgfqpoint{3.696000in}{3.696000in}}%
\pgfusepath{clip}%
\pgfsetbuttcap%
\pgfsetroundjoin%
\definecolor{currentfill}{rgb}{0.121569,0.466667,0.705882}%
\pgfsetfillcolor{currentfill}%
\pgfsetfillopacity{0.338073}%
\pgfsetlinewidth{1.003750pt}%
\definecolor{currentstroke}{rgb}{0.121569,0.466667,0.705882}%
\pgfsetstrokecolor{currentstroke}%
\pgfsetstrokeopacity{0.338073}%
\pgfsetdash{}{0pt}%
\pgfpathmoveto{\pgfqpoint{1.813156in}{2.003267in}}%
\pgfpathcurveto{\pgfqpoint{1.821392in}{2.003267in}}{\pgfqpoint{1.829292in}{2.006539in}}{\pgfqpoint{1.835116in}{2.012363in}}%
\pgfpathcurveto{\pgfqpoint{1.840940in}{2.018187in}}{\pgfqpoint{1.844212in}{2.026087in}}{\pgfqpoint{1.844212in}{2.034324in}}%
\pgfpathcurveto{\pgfqpoint{1.844212in}{2.042560in}}{\pgfqpoint{1.840940in}{2.050460in}}{\pgfqpoint{1.835116in}{2.056284in}}%
\pgfpathcurveto{\pgfqpoint{1.829292in}{2.062108in}}{\pgfqpoint{1.821392in}{2.065380in}}{\pgfqpoint{1.813156in}{2.065380in}}%
\pgfpathcurveto{\pgfqpoint{1.804919in}{2.065380in}}{\pgfqpoint{1.797019in}{2.062108in}}{\pgfqpoint{1.791196in}{2.056284in}}%
\pgfpathcurveto{\pgfqpoint{1.785372in}{2.050460in}}{\pgfqpoint{1.782099in}{2.042560in}}{\pgfqpoint{1.782099in}{2.034324in}}%
\pgfpathcurveto{\pgfqpoint{1.782099in}{2.026087in}}{\pgfqpoint{1.785372in}{2.018187in}}{\pgfqpoint{1.791196in}{2.012363in}}%
\pgfpathcurveto{\pgfqpoint{1.797019in}{2.006539in}}{\pgfqpoint{1.804919in}{2.003267in}}{\pgfqpoint{1.813156in}{2.003267in}}%
\pgfpathclose%
\pgfusepath{stroke,fill}%
\end{pgfscope}%
\begin{pgfscope}%
\pgfpathrectangle{\pgfqpoint{0.100000in}{0.212622in}}{\pgfqpoint{3.696000in}{3.696000in}}%
\pgfusepath{clip}%
\pgfsetbuttcap%
\pgfsetroundjoin%
\definecolor{currentfill}{rgb}{0.121569,0.466667,0.705882}%
\pgfsetfillcolor{currentfill}%
\pgfsetfillopacity{0.338608}%
\pgfsetlinewidth{1.003750pt}%
\definecolor{currentstroke}{rgb}{0.121569,0.466667,0.705882}%
\pgfsetstrokecolor{currentstroke}%
\pgfsetstrokeopacity{0.338608}%
\pgfsetdash{}{0pt}%
\pgfpathmoveto{\pgfqpoint{1.969176in}{2.060662in}}%
\pgfpathcurveto{\pgfqpoint{1.977412in}{2.060662in}}{\pgfqpoint{1.985312in}{2.063935in}}{\pgfqpoint{1.991136in}{2.069759in}}%
\pgfpathcurveto{\pgfqpoint{1.996960in}{2.075583in}}{\pgfqpoint{2.000233in}{2.083483in}}{\pgfqpoint{2.000233in}{2.091719in}}%
\pgfpathcurveto{\pgfqpoint{2.000233in}{2.099955in}}{\pgfqpoint{1.996960in}{2.107855in}}{\pgfqpoint{1.991136in}{2.113679in}}%
\pgfpathcurveto{\pgfqpoint{1.985312in}{2.119503in}}{\pgfqpoint{1.977412in}{2.122775in}}{\pgfqpoint{1.969176in}{2.122775in}}%
\pgfpathcurveto{\pgfqpoint{1.960940in}{2.122775in}}{\pgfqpoint{1.953040in}{2.119503in}}{\pgfqpoint{1.947216in}{2.113679in}}%
\pgfpathcurveto{\pgfqpoint{1.941392in}{2.107855in}}{\pgfqpoint{1.938120in}{2.099955in}}{\pgfqpoint{1.938120in}{2.091719in}}%
\pgfpathcurveto{\pgfqpoint{1.938120in}{2.083483in}}{\pgfqpoint{1.941392in}{2.075583in}}{\pgfqpoint{1.947216in}{2.069759in}}%
\pgfpathcurveto{\pgfqpoint{1.953040in}{2.063935in}}{\pgfqpoint{1.960940in}{2.060662in}}{\pgfqpoint{1.969176in}{2.060662in}}%
\pgfpathclose%
\pgfusepath{stroke,fill}%
\end{pgfscope}%
\begin{pgfscope}%
\pgfpathrectangle{\pgfqpoint{0.100000in}{0.212622in}}{\pgfqpoint{3.696000in}{3.696000in}}%
\pgfusepath{clip}%
\pgfsetbuttcap%
\pgfsetroundjoin%
\definecolor{currentfill}{rgb}{0.121569,0.466667,0.705882}%
\pgfsetfillcolor{currentfill}%
\pgfsetfillopacity{0.339417}%
\pgfsetlinewidth{1.003750pt}%
\definecolor{currentstroke}{rgb}{0.121569,0.466667,0.705882}%
\pgfsetstrokecolor{currentstroke}%
\pgfsetstrokeopacity{0.339417}%
\pgfsetdash{}{0pt}%
\pgfpathmoveto{\pgfqpoint{1.809188in}{1.998757in}}%
\pgfpathcurveto{\pgfqpoint{1.817425in}{1.998757in}}{\pgfqpoint{1.825325in}{2.002030in}}{\pgfqpoint{1.831149in}{2.007854in}}%
\pgfpathcurveto{\pgfqpoint{1.836972in}{2.013677in}}{\pgfqpoint{1.840245in}{2.021578in}}{\pgfqpoint{1.840245in}{2.029814in}}%
\pgfpathcurveto{\pgfqpoint{1.840245in}{2.038050in}}{\pgfqpoint{1.836972in}{2.045950in}}{\pgfqpoint{1.831149in}{2.051774in}}%
\pgfpathcurveto{\pgfqpoint{1.825325in}{2.057598in}}{\pgfqpoint{1.817425in}{2.060870in}}{\pgfqpoint{1.809188in}{2.060870in}}%
\pgfpathcurveto{\pgfqpoint{1.800952in}{2.060870in}}{\pgfqpoint{1.793052in}{2.057598in}}{\pgfqpoint{1.787228in}{2.051774in}}%
\pgfpathcurveto{\pgfqpoint{1.781404in}{2.045950in}}{\pgfqpoint{1.778132in}{2.038050in}}{\pgfqpoint{1.778132in}{2.029814in}}%
\pgfpathcurveto{\pgfqpoint{1.778132in}{2.021578in}}{\pgfqpoint{1.781404in}{2.013677in}}{\pgfqpoint{1.787228in}{2.007854in}}%
\pgfpathcurveto{\pgfqpoint{1.793052in}{2.002030in}}{\pgfqpoint{1.800952in}{1.998757in}}{\pgfqpoint{1.809188in}{1.998757in}}%
\pgfpathclose%
\pgfusepath{stroke,fill}%
\end{pgfscope}%
\begin{pgfscope}%
\pgfpathrectangle{\pgfqpoint{0.100000in}{0.212622in}}{\pgfqpoint{3.696000in}{3.696000in}}%
\pgfusepath{clip}%
\pgfsetbuttcap%
\pgfsetroundjoin%
\definecolor{currentfill}{rgb}{0.121569,0.466667,0.705882}%
\pgfsetfillcolor{currentfill}%
\pgfsetfillopacity{0.340315}%
\pgfsetlinewidth{1.003750pt}%
\definecolor{currentstroke}{rgb}{0.121569,0.466667,0.705882}%
\pgfsetstrokecolor{currentstroke}%
\pgfsetstrokeopacity{0.340315}%
\pgfsetdash{}{0pt}%
\pgfpathmoveto{\pgfqpoint{1.806341in}{1.997900in}}%
\pgfpathcurveto{\pgfqpoint{1.814577in}{1.997900in}}{\pgfqpoint{1.822477in}{2.001173in}}{\pgfqpoint{1.828301in}{2.006996in}}%
\pgfpathcurveto{\pgfqpoint{1.834125in}{2.012820in}}{\pgfqpoint{1.837397in}{2.020720in}}{\pgfqpoint{1.837397in}{2.028957in}}%
\pgfpathcurveto{\pgfqpoint{1.837397in}{2.037193in}}{\pgfqpoint{1.834125in}{2.045093in}}{\pgfqpoint{1.828301in}{2.050917in}}%
\pgfpathcurveto{\pgfqpoint{1.822477in}{2.056741in}}{\pgfqpoint{1.814577in}{2.060013in}}{\pgfqpoint{1.806341in}{2.060013in}}%
\pgfpathcurveto{\pgfqpoint{1.798104in}{2.060013in}}{\pgfqpoint{1.790204in}{2.056741in}}{\pgfqpoint{1.784380in}{2.050917in}}%
\pgfpathcurveto{\pgfqpoint{1.778556in}{2.045093in}}{\pgfqpoint{1.775284in}{2.037193in}}{\pgfqpoint{1.775284in}{2.028957in}}%
\pgfpathcurveto{\pgfqpoint{1.775284in}{2.020720in}}{\pgfqpoint{1.778556in}{2.012820in}}{\pgfqpoint{1.784380in}{2.006996in}}%
\pgfpathcurveto{\pgfqpoint{1.790204in}{2.001173in}}{\pgfqpoint{1.798104in}{1.997900in}}{\pgfqpoint{1.806341in}{1.997900in}}%
\pgfpathclose%
\pgfusepath{stroke,fill}%
\end{pgfscope}%
\begin{pgfscope}%
\pgfpathrectangle{\pgfqpoint{0.100000in}{0.212622in}}{\pgfqpoint{3.696000in}{3.696000in}}%
\pgfusepath{clip}%
\pgfsetbuttcap%
\pgfsetroundjoin%
\definecolor{currentfill}{rgb}{0.121569,0.466667,0.705882}%
\pgfsetfillcolor{currentfill}%
\pgfsetfillopacity{0.341842}%
\pgfsetlinewidth{1.003750pt}%
\definecolor{currentstroke}{rgb}{0.121569,0.466667,0.705882}%
\pgfsetstrokecolor{currentstroke}%
\pgfsetstrokeopacity{0.341842}%
\pgfsetdash{}{0pt}%
\pgfpathmoveto{\pgfqpoint{1.801575in}{1.994972in}}%
\pgfpathcurveto{\pgfqpoint{1.809812in}{1.994972in}}{\pgfqpoint{1.817712in}{1.998244in}}{\pgfqpoint{1.823536in}{2.004068in}}%
\pgfpathcurveto{\pgfqpoint{1.829359in}{2.009892in}}{\pgfqpoint{1.832632in}{2.017792in}}{\pgfqpoint{1.832632in}{2.026028in}}%
\pgfpathcurveto{\pgfqpoint{1.832632in}{2.034264in}}{\pgfqpoint{1.829359in}{2.042164in}}{\pgfqpoint{1.823536in}{2.047988in}}%
\pgfpathcurveto{\pgfqpoint{1.817712in}{2.053812in}}{\pgfqpoint{1.809812in}{2.057085in}}{\pgfqpoint{1.801575in}{2.057085in}}%
\pgfpathcurveto{\pgfqpoint{1.793339in}{2.057085in}}{\pgfqpoint{1.785439in}{2.053812in}}{\pgfqpoint{1.779615in}{2.047988in}}%
\pgfpathcurveto{\pgfqpoint{1.773791in}{2.042164in}}{\pgfqpoint{1.770519in}{2.034264in}}{\pgfqpoint{1.770519in}{2.026028in}}%
\pgfpathcurveto{\pgfqpoint{1.770519in}{2.017792in}}{\pgfqpoint{1.773791in}{2.009892in}}{\pgfqpoint{1.779615in}{2.004068in}}%
\pgfpathcurveto{\pgfqpoint{1.785439in}{1.998244in}}{\pgfqpoint{1.793339in}{1.994972in}}{\pgfqpoint{1.801575in}{1.994972in}}%
\pgfpathclose%
\pgfusepath{stroke,fill}%
\end{pgfscope}%
\begin{pgfscope}%
\pgfpathrectangle{\pgfqpoint{0.100000in}{0.212622in}}{\pgfqpoint{3.696000in}{3.696000in}}%
\pgfusepath{clip}%
\pgfsetbuttcap%
\pgfsetroundjoin%
\definecolor{currentfill}{rgb}{0.121569,0.466667,0.705882}%
\pgfsetfillcolor{currentfill}%
\pgfsetfillopacity{0.341935}%
\pgfsetlinewidth{1.003750pt}%
\definecolor{currentstroke}{rgb}{0.121569,0.466667,0.705882}%
\pgfsetstrokecolor{currentstroke}%
\pgfsetstrokeopacity{0.341935}%
\pgfsetdash{}{0pt}%
\pgfpathmoveto{\pgfqpoint{1.972271in}{2.059086in}}%
\pgfpathcurveto{\pgfqpoint{1.980507in}{2.059086in}}{\pgfqpoint{1.988407in}{2.062359in}}{\pgfqpoint{1.994231in}{2.068183in}}%
\pgfpathcurveto{\pgfqpoint{2.000055in}{2.074007in}}{\pgfqpoint{2.003327in}{2.081907in}}{\pgfqpoint{2.003327in}{2.090143in}}%
\pgfpathcurveto{\pgfqpoint{2.003327in}{2.098379in}}{\pgfqpoint{2.000055in}{2.106279in}}{\pgfqpoint{1.994231in}{2.112103in}}%
\pgfpathcurveto{\pgfqpoint{1.988407in}{2.117927in}}{\pgfqpoint{1.980507in}{2.121199in}}{\pgfqpoint{1.972271in}{2.121199in}}%
\pgfpathcurveto{\pgfqpoint{1.964035in}{2.121199in}}{\pgfqpoint{1.956135in}{2.117927in}}{\pgfqpoint{1.950311in}{2.112103in}}%
\pgfpathcurveto{\pgfqpoint{1.944487in}{2.106279in}}{\pgfqpoint{1.941214in}{2.098379in}}{\pgfqpoint{1.941214in}{2.090143in}}%
\pgfpathcurveto{\pgfqpoint{1.941214in}{2.081907in}}{\pgfqpoint{1.944487in}{2.074007in}}{\pgfqpoint{1.950311in}{2.068183in}}%
\pgfpathcurveto{\pgfqpoint{1.956135in}{2.062359in}}{\pgfqpoint{1.964035in}{2.059086in}}{\pgfqpoint{1.972271in}{2.059086in}}%
\pgfpathclose%
\pgfusepath{stroke,fill}%
\end{pgfscope}%
\begin{pgfscope}%
\pgfpathrectangle{\pgfqpoint{0.100000in}{0.212622in}}{\pgfqpoint{3.696000in}{3.696000in}}%
\pgfusepath{clip}%
\pgfsetbuttcap%
\pgfsetroundjoin%
\definecolor{currentfill}{rgb}{0.121569,0.466667,0.705882}%
\pgfsetfillcolor{currentfill}%
\pgfsetfillopacity{0.344152}%
\pgfsetlinewidth{1.003750pt}%
\definecolor{currentstroke}{rgb}{0.121569,0.466667,0.705882}%
\pgfsetstrokecolor{currentstroke}%
\pgfsetstrokeopacity{0.344152}%
\pgfsetdash{}{0pt}%
\pgfpathmoveto{\pgfqpoint{1.793970in}{1.985044in}}%
\pgfpathcurveto{\pgfqpoint{1.802207in}{1.985044in}}{\pgfqpoint{1.810107in}{1.988317in}}{\pgfqpoint{1.815931in}{1.994140in}}%
\pgfpathcurveto{\pgfqpoint{1.821755in}{1.999964in}}{\pgfqpoint{1.825027in}{2.007864in}}{\pgfqpoint{1.825027in}{2.016101in}}%
\pgfpathcurveto{\pgfqpoint{1.825027in}{2.024337in}}{\pgfqpoint{1.821755in}{2.032237in}}{\pgfqpoint{1.815931in}{2.038061in}}%
\pgfpathcurveto{\pgfqpoint{1.810107in}{2.043885in}}{\pgfqpoint{1.802207in}{2.047157in}}{\pgfqpoint{1.793970in}{2.047157in}}%
\pgfpathcurveto{\pgfqpoint{1.785734in}{2.047157in}}{\pgfqpoint{1.777834in}{2.043885in}}{\pgfqpoint{1.772010in}{2.038061in}}%
\pgfpathcurveto{\pgfqpoint{1.766186in}{2.032237in}}{\pgfqpoint{1.762914in}{2.024337in}}{\pgfqpoint{1.762914in}{2.016101in}}%
\pgfpathcurveto{\pgfqpoint{1.762914in}{2.007864in}}{\pgfqpoint{1.766186in}{1.999964in}}{\pgfqpoint{1.772010in}{1.994140in}}%
\pgfpathcurveto{\pgfqpoint{1.777834in}{1.988317in}}{\pgfqpoint{1.785734in}{1.985044in}}{\pgfqpoint{1.793970in}{1.985044in}}%
\pgfpathclose%
\pgfusepath{stroke,fill}%
\end{pgfscope}%
\begin{pgfscope}%
\pgfpathrectangle{\pgfqpoint{0.100000in}{0.212622in}}{\pgfqpoint{3.696000in}{3.696000in}}%
\pgfusepath{clip}%
\pgfsetbuttcap%
\pgfsetroundjoin%
\definecolor{currentfill}{rgb}{0.121569,0.466667,0.705882}%
\pgfsetfillcolor{currentfill}%
\pgfsetfillopacity{0.345302}%
\pgfsetlinewidth{1.003750pt}%
\definecolor{currentstroke}{rgb}{0.121569,0.466667,0.705882}%
\pgfsetstrokecolor{currentstroke}%
\pgfsetstrokeopacity{0.345302}%
\pgfsetdash{}{0pt}%
\pgfpathmoveto{\pgfqpoint{1.975096in}{2.053802in}}%
\pgfpathcurveto{\pgfqpoint{1.983332in}{2.053802in}}{\pgfqpoint{1.991232in}{2.057074in}}{\pgfqpoint{1.997056in}{2.062898in}}%
\pgfpathcurveto{\pgfqpoint{2.002880in}{2.068722in}}{\pgfqpoint{2.006152in}{2.076622in}}{\pgfqpoint{2.006152in}{2.084858in}}%
\pgfpathcurveto{\pgfqpoint{2.006152in}{2.093094in}}{\pgfqpoint{2.002880in}{2.100994in}}{\pgfqpoint{1.997056in}{2.106818in}}%
\pgfpathcurveto{\pgfqpoint{1.991232in}{2.112642in}}{\pgfqpoint{1.983332in}{2.115915in}}{\pgfqpoint{1.975096in}{2.115915in}}%
\pgfpathcurveto{\pgfqpoint{1.966859in}{2.115915in}}{\pgfqpoint{1.958959in}{2.112642in}}{\pgfqpoint{1.953135in}{2.106818in}}%
\pgfpathcurveto{\pgfqpoint{1.947311in}{2.100994in}}{\pgfqpoint{1.944039in}{2.093094in}}{\pgfqpoint{1.944039in}{2.084858in}}%
\pgfpathcurveto{\pgfqpoint{1.944039in}{2.076622in}}{\pgfqpoint{1.947311in}{2.068722in}}{\pgfqpoint{1.953135in}{2.062898in}}%
\pgfpathcurveto{\pgfqpoint{1.958959in}{2.057074in}}{\pgfqpoint{1.966859in}{2.053802in}}{\pgfqpoint{1.975096in}{2.053802in}}%
\pgfpathclose%
\pgfusepath{stroke,fill}%
\end{pgfscope}%
\begin{pgfscope}%
\pgfpathrectangle{\pgfqpoint{0.100000in}{0.212622in}}{\pgfqpoint{3.696000in}{3.696000in}}%
\pgfusepath{clip}%
\pgfsetbuttcap%
\pgfsetroundjoin%
\definecolor{currentfill}{rgb}{0.121569,0.466667,0.705882}%
\pgfsetfillcolor{currentfill}%
\pgfsetfillopacity{0.346140}%
\pgfsetlinewidth{1.003750pt}%
\definecolor{currentstroke}{rgb}{0.121569,0.466667,0.705882}%
\pgfsetstrokecolor{currentstroke}%
\pgfsetstrokeopacity{0.346140}%
\pgfsetdash{}{0pt}%
\pgfpathmoveto{\pgfqpoint{1.787051in}{1.981723in}}%
\pgfpathcurveto{\pgfqpoint{1.795287in}{1.981723in}}{\pgfqpoint{1.803187in}{1.984996in}}{\pgfqpoint{1.809011in}{1.990820in}}%
\pgfpathcurveto{\pgfqpoint{1.814835in}{1.996644in}}{\pgfqpoint{1.818108in}{2.004544in}}{\pgfqpoint{1.818108in}{2.012780in}}%
\pgfpathcurveto{\pgfqpoint{1.818108in}{2.021016in}}{\pgfqpoint{1.814835in}{2.028916in}}{\pgfqpoint{1.809011in}{2.034740in}}%
\pgfpathcurveto{\pgfqpoint{1.803187in}{2.040564in}}{\pgfqpoint{1.795287in}{2.043836in}}{\pgfqpoint{1.787051in}{2.043836in}}%
\pgfpathcurveto{\pgfqpoint{1.778815in}{2.043836in}}{\pgfqpoint{1.770915in}{2.040564in}}{\pgfqpoint{1.765091in}{2.034740in}}%
\pgfpathcurveto{\pgfqpoint{1.759267in}{2.028916in}}{\pgfqpoint{1.755995in}{2.021016in}}{\pgfqpoint{1.755995in}{2.012780in}}%
\pgfpathcurveto{\pgfqpoint{1.755995in}{2.004544in}}{\pgfqpoint{1.759267in}{1.996644in}}{\pgfqpoint{1.765091in}{1.990820in}}%
\pgfpathcurveto{\pgfqpoint{1.770915in}{1.984996in}}{\pgfqpoint{1.778815in}{1.981723in}}{\pgfqpoint{1.787051in}{1.981723in}}%
\pgfpathclose%
\pgfusepath{stroke,fill}%
\end{pgfscope}%
\begin{pgfscope}%
\pgfpathrectangle{\pgfqpoint{0.100000in}{0.212622in}}{\pgfqpoint{3.696000in}{3.696000in}}%
\pgfusepath{clip}%
\pgfsetbuttcap%
\pgfsetroundjoin%
\definecolor{currentfill}{rgb}{0.121569,0.466667,0.705882}%
\pgfsetfillcolor{currentfill}%
\pgfsetfillopacity{0.347106}%
\pgfsetlinewidth{1.003750pt}%
\definecolor{currentstroke}{rgb}{0.121569,0.466667,0.705882}%
\pgfsetstrokecolor{currentstroke}%
\pgfsetstrokeopacity{0.347106}%
\pgfsetdash{}{0pt}%
\pgfpathmoveto{\pgfqpoint{1.783760in}{1.979427in}}%
\pgfpathcurveto{\pgfqpoint{1.791997in}{1.979427in}}{\pgfqpoint{1.799897in}{1.982699in}}{\pgfqpoint{1.805721in}{1.988523in}}%
\pgfpathcurveto{\pgfqpoint{1.811545in}{1.994347in}}{\pgfqpoint{1.814817in}{2.002247in}}{\pgfqpoint{1.814817in}{2.010483in}}%
\pgfpathcurveto{\pgfqpoint{1.814817in}{2.018719in}}{\pgfqpoint{1.811545in}{2.026619in}}{\pgfqpoint{1.805721in}{2.032443in}}%
\pgfpathcurveto{\pgfqpoint{1.799897in}{2.038267in}}{\pgfqpoint{1.791997in}{2.041540in}}{\pgfqpoint{1.783760in}{2.041540in}}%
\pgfpathcurveto{\pgfqpoint{1.775524in}{2.041540in}}{\pgfqpoint{1.767624in}{2.038267in}}{\pgfqpoint{1.761800in}{2.032443in}}%
\pgfpathcurveto{\pgfqpoint{1.755976in}{2.026619in}}{\pgfqpoint{1.752704in}{2.018719in}}{\pgfqpoint{1.752704in}{2.010483in}}%
\pgfpathcurveto{\pgfqpoint{1.752704in}{2.002247in}}{\pgfqpoint{1.755976in}{1.994347in}}{\pgfqpoint{1.761800in}{1.988523in}}%
\pgfpathcurveto{\pgfqpoint{1.767624in}{1.982699in}}{\pgfqpoint{1.775524in}{1.979427in}}{\pgfqpoint{1.783760in}{1.979427in}}%
\pgfpathclose%
\pgfusepath{stroke,fill}%
\end{pgfscope}%
\begin{pgfscope}%
\pgfpathrectangle{\pgfqpoint{0.100000in}{0.212622in}}{\pgfqpoint{3.696000in}{3.696000in}}%
\pgfusepath{clip}%
\pgfsetbuttcap%
\pgfsetroundjoin%
\definecolor{currentfill}{rgb}{0.121569,0.466667,0.705882}%
\pgfsetfillcolor{currentfill}%
\pgfsetfillopacity{0.347591}%
\pgfsetlinewidth{1.003750pt}%
\definecolor{currentstroke}{rgb}{0.121569,0.466667,0.705882}%
\pgfsetstrokecolor{currentstroke}%
\pgfsetstrokeopacity{0.347591}%
\pgfsetdash{}{0pt}%
\pgfpathmoveto{\pgfqpoint{1.782110in}{1.977659in}}%
\pgfpathcurveto{\pgfqpoint{1.790346in}{1.977659in}}{\pgfqpoint{1.798246in}{1.980932in}}{\pgfqpoint{1.804070in}{1.986756in}}%
\pgfpathcurveto{\pgfqpoint{1.809894in}{1.992580in}}{\pgfqpoint{1.813166in}{2.000480in}}{\pgfqpoint{1.813166in}{2.008716in}}%
\pgfpathcurveto{\pgfqpoint{1.813166in}{2.016952in}}{\pgfqpoint{1.809894in}{2.024852in}}{\pgfqpoint{1.804070in}{2.030676in}}%
\pgfpathcurveto{\pgfqpoint{1.798246in}{2.036500in}}{\pgfqpoint{1.790346in}{2.039772in}}{\pgfqpoint{1.782110in}{2.039772in}}%
\pgfpathcurveto{\pgfqpoint{1.773873in}{2.039772in}}{\pgfqpoint{1.765973in}{2.036500in}}{\pgfqpoint{1.760149in}{2.030676in}}%
\pgfpathcurveto{\pgfqpoint{1.754326in}{2.024852in}}{\pgfqpoint{1.751053in}{2.016952in}}{\pgfqpoint{1.751053in}{2.008716in}}%
\pgfpathcurveto{\pgfqpoint{1.751053in}{2.000480in}}{\pgfqpoint{1.754326in}{1.992580in}}{\pgfqpoint{1.760149in}{1.986756in}}%
\pgfpathcurveto{\pgfqpoint{1.765973in}{1.980932in}}{\pgfqpoint{1.773873in}{1.977659in}}{\pgfqpoint{1.782110in}{1.977659in}}%
\pgfpathclose%
\pgfusepath{stroke,fill}%
\end{pgfscope}%
\begin{pgfscope}%
\pgfpathrectangle{\pgfqpoint{0.100000in}{0.212622in}}{\pgfqpoint{3.696000in}{3.696000in}}%
\pgfusepath{clip}%
\pgfsetbuttcap%
\pgfsetroundjoin%
\definecolor{currentfill}{rgb}{0.121569,0.466667,0.705882}%
\pgfsetfillcolor{currentfill}%
\pgfsetfillopacity{0.348296}%
\pgfsetlinewidth{1.003750pt}%
\definecolor{currentstroke}{rgb}{0.121569,0.466667,0.705882}%
\pgfsetstrokecolor{currentstroke}%
\pgfsetstrokeopacity{0.348296}%
\pgfsetdash{}{0pt}%
\pgfpathmoveto{\pgfqpoint{1.778335in}{1.974520in}}%
\pgfpathcurveto{\pgfqpoint{1.786571in}{1.974520in}}{\pgfqpoint{1.794471in}{1.977793in}}{\pgfqpoint{1.800295in}{1.983617in}}%
\pgfpathcurveto{\pgfqpoint{1.806119in}{1.989441in}}{\pgfqpoint{1.809391in}{1.997341in}}{\pgfqpoint{1.809391in}{2.005577in}}%
\pgfpathcurveto{\pgfqpoint{1.809391in}{2.013813in}}{\pgfqpoint{1.806119in}{2.021713in}}{\pgfqpoint{1.800295in}{2.027537in}}%
\pgfpathcurveto{\pgfqpoint{1.794471in}{2.033361in}}{\pgfqpoint{1.786571in}{2.036633in}}{\pgfqpoint{1.778335in}{2.036633in}}%
\pgfpathcurveto{\pgfqpoint{1.770098in}{2.036633in}}{\pgfqpoint{1.762198in}{2.033361in}}{\pgfqpoint{1.756374in}{2.027537in}}%
\pgfpathcurveto{\pgfqpoint{1.750550in}{2.021713in}}{\pgfqpoint{1.747278in}{2.013813in}}{\pgfqpoint{1.747278in}{2.005577in}}%
\pgfpathcurveto{\pgfqpoint{1.747278in}{1.997341in}}{\pgfqpoint{1.750550in}{1.989441in}}{\pgfqpoint{1.756374in}{1.983617in}}%
\pgfpathcurveto{\pgfqpoint{1.762198in}{1.977793in}}{\pgfqpoint{1.770098in}{1.974520in}}{\pgfqpoint{1.778335in}{1.974520in}}%
\pgfpathclose%
\pgfusepath{stroke,fill}%
\end{pgfscope}%
\begin{pgfscope}%
\pgfpathrectangle{\pgfqpoint{0.100000in}{0.212622in}}{\pgfqpoint{3.696000in}{3.696000in}}%
\pgfusepath{clip}%
\pgfsetbuttcap%
\pgfsetroundjoin%
\definecolor{currentfill}{rgb}{0.121569,0.466667,0.705882}%
\pgfsetfillcolor{currentfill}%
\pgfsetfillopacity{0.350254}%
\pgfsetlinewidth{1.003750pt}%
\definecolor{currentstroke}{rgb}{0.121569,0.466667,0.705882}%
\pgfsetstrokecolor{currentstroke}%
\pgfsetstrokeopacity{0.350254}%
\pgfsetdash{}{0pt}%
\pgfpathmoveto{\pgfqpoint{1.978125in}{2.053295in}}%
\pgfpathcurveto{\pgfqpoint{1.986361in}{2.053295in}}{\pgfqpoint{1.994261in}{2.056568in}}{\pgfqpoint{2.000085in}{2.062392in}}%
\pgfpathcurveto{\pgfqpoint{2.005909in}{2.068216in}}{\pgfqpoint{2.009181in}{2.076116in}}{\pgfqpoint{2.009181in}{2.084352in}}%
\pgfpathcurveto{\pgfqpoint{2.009181in}{2.092588in}}{\pgfqpoint{2.005909in}{2.100488in}}{\pgfqpoint{2.000085in}{2.106312in}}%
\pgfpathcurveto{\pgfqpoint{1.994261in}{2.112136in}}{\pgfqpoint{1.986361in}{2.115408in}}{\pgfqpoint{1.978125in}{2.115408in}}%
\pgfpathcurveto{\pgfqpoint{1.969889in}{2.115408in}}{\pgfqpoint{1.961988in}{2.112136in}}{\pgfqpoint{1.956165in}{2.106312in}}%
\pgfpathcurveto{\pgfqpoint{1.950341in}{2.100488in}}{\pgfqpoint{1.947068in}{2.092588in}}{\pgfqpoint{1.947068in}{2.084352in}}%
\pgfpathcurveto{\pgfqpoint{1.947068in}{2.076116in}}{\pgfqpoint{1.950341in}{2.068216in}}{\pgfqpoint{1.956165in}{2.062392in}}%
\pgfpathcurveto{\pgfqpoint{1.961988in}{2.056568in}}{\pgfqpoint{1.969889in}{2.053295in}}{\pgfqpoint{1.978125in}{2.053295in}}%
\pgfpathclose%
\pgfusepath{stroke,fill}%
\end{pgfscope}%
\begin{pgfscope}%
\pgfpathrectangle{\pgfqpoint{0.100000in}{0.212622in}}{\pgfqpoint{3.696000in}{3.696000in}}%
\pgfusepath{clip}%
\pgfsetbuttcap%
\pgfsetroundjoin%
\definecolor{currentfill}{rgb}{0.121569,0.466667,0.705882}%
\pgfsetfillcolor{currentfill}%
\pgfsetfillopacity{0.350310}%
\pgfsetlinewidth{1.003750pt}%
\definecolor{currentstroke}{rgb}{0.121569,0.466667,0.705882}%
\pgfsetstrokecolor{currentstroke}%
\pgfsetstrokeopacity{0.350310}%
\pgfsetdash{}{0pt}%
\pgfpathmoveto{\pgfqpoint{1.772542in}{1.971938in}}%
\pgfpathcurveto{\pgfqpoint{1.780779in}{1.971938in}}{\pgfqpoint{1.788679in}{1.975210in}}{\pgfqpoint{1.794503in}{1.981034in}}%
\pgfpathcurveto{\pgfqpoint{1.800327in}{1.986858in}}{\pgfqpoint{1.803599in}{1.994758in}}{\pgfqpoint{1.803599in}{2.002994in}}%
\pgfpathcurveto{\pgfqpoint{1.803599in}{2.011231in}}{\pgfqpoint{1.800327in}{2.019131in}}{\pgfqpoint{1.794503in}{2.024955in}}%
\pgfpathcurveto{\pgfqpoint{1.788679in}{2.030779in}}{\pgfqpoint{1.780779in}{2.034051in}}{\pgfqpoint{1.772542in}{2.034051in}}%
\pgfpathcurveto{\pgfqpoint{1.764306in}{2.034051in}}{\pgfqpoint{1.756406in}{2.030779in}}{\pgfqpoint{1.750582in}{2.024955in}}%
\pgfpathcurveto{\pgfqpoint{1.744758in}{2.019131in}}{\pgfqpoint{1.741486in}{2.011231in}}{\pgfqpoint{1.741486in}{2.002994in}}%
\pgfpathcurveto{\pgfqpoint{1.741486in}{1.994758in}}{\pgfqpoint{1.744758in}{1.986858in}}{\pgfqpoint{1.750582in}{1.981034in}}%
\pgfpathcurveto{\pgfqpoint{1.756406in}{1.975210in}}{\pgfqpoint{1.764306in}{1.971938in}}{\pgfqpoint{1.772542in}{1.971938in}}%
\pgfpathclose%
\pgfusepath{stroke,fill}%
\end{pgfscope}%
\begin{pgfscope}%
\pgfpathrectangle{\pgfqpoint{0.100000in}{0.212622in}}{\pgfqpoint{3.696000in}{3.696000in}}%
\pgfusepath{clip}%
\pgfsetbuttcap%
\pgfsetroundjoin%
\definecolor{currentfill}{rgb}{0.121569,0.466667,0.705882}%
\pgfsetfillcolor{currentfill}%
\pgfsetfillopacity{0.351625}%
\pgfsetlinewidth{1.003750pt}%
\definecolor{currentstroke}{rgb}{0.121569,0.466667,0.705882}%
\pgfsetstrokecolor{currentstroke}%
\pgfsetstrokeopacity{0.351625}%
\pgfsetdash{}{0pt}%
\pgfpathmoveto{\pgfqpoint{1.768624in}{1.967136in}}%
\pgfpathcurveto{\pgfqpoint{1.776861in}{1.967136in}}{\pgfqpoint{1.784761in}{1.970408in}}{\pgfqpoint{1.790585in}{1.976232in}}%
\pgfpathcurveto{\pgfqpoint{1.796409in}{1.982056in}}{\pgfqpoint{1.799681in}{1.989956in}}{\pgfqpoint{1.799681in}{1.998192in}}%
\pgfpathcurveto{\pgfqpoint{1.799681in}{2.006428in}}{\pgfqpoint{1.796409in}{2.014329in}}{\pgfqpoint{1.790585in}{2.020152in}}%
\pgfpathcurveto{\pgfqpoint{1.784761in}{2.025976in}}{\pgfqpoint{1.776861in}{2.029249in}}{\pgfqpoint{1.768624in}{2.029249in}}%
\pgfpathcurveto{\pgfqpoint{1.760388in}{2.029249in}}{\pgfqpoint{1.752488in}{2.025976in}}{\pgfqpoint{1.746664in}{2.020152in}}%
\pgfpathcurveto{\pgfqpoint{1.740840in}{2.014329in}}{\pgfqpoint{1.737568in}{2.006428in}}{\pgfqpoint{1.737568in}{1.998192in}}%
\pgfpathcurveto{\pgfqpoint{1.737568in}{1.989956in}}{\pgfqpoint{1.740840in}{1.982056in}}{\pgfqpoint{1.746664in}{1.976232in}}%
\pgfpathcurveto{\pgfqpoint{1.752488in}{1.970408in}}{\pgfqpoint{1.760388in}{1.967136in}}{\pgfqpoint{1.768624in}{1.967136in}}%
\pgfpathclose%
\pgfusepath{stroke,fill}%
\end{pgfscope}%
\begin{pgfscope}%
\pgfpathrectangle{\pgfqpoint{0.100000in}{0.212622in}}{\pgfqpoint{3.696000in}{3.696000in}}%
\pgfusepath{clip}%
\pgfsetbuttcap%
\pgfsetroundjoin%
\definecolor{currentfill}{rgb}{0.121569,0.466667,0.705882}%
\pgfsetfillcolor{currentfill}%
\pgfsetfillopacity{0.352635}%
\pgfsetlinewidth{1.003750pt}%
\definecolor{currentstroke}{rgb}{0.121569,0.466667,0.705882}%
\pgfsetstrokecolor{currentstroke}%
\pgfsetstrokeopacity{0.352635}%
\pgfsetdash{}{0pt}%
\pgfpathmoveto{\pgfqpoint{1.764462in}{1.963481in}}%
\pgfpathcurveto{\pgfqpoint{1.772698in}{1.963481in}}{\pgfqpoint{1.780598in}{1.966753in}}{\pgfqpoint{1.786422in}{1.972577in}}%
\pgfpathcurveto{\pgfqpoint{1.792246in}{1.978401in}}{\pgfqpoint{1.795518in}{1.986301in}}{\pgfqpoint{1.795518in}{1.994537in}}%
\pgfpathcurveto{\pgfqpoint{1.795518in}{2.002774in}}{\pgfqpoint{1.792246in}{2.010674in}}{\pgfqpoint{1.786422in}{2.016498in}}%
\pgfpathcurveto{\pgfqpoint{1.780598in}{2.022321in}}{\pgfqpoint{1.772698in}{2.025594in}}{\pgfqpoint{1.764462in}{2.025594in}}%
\pgfpathcurveto{\pgfqpoint{1.756225in}{2.025594in}}{\pgfqpoint{1.748325in}{2.022321in}}{\pgfqpoint{1.742501in}{2.016498in}}%
\pgfpathcurveto{\pgfqpoint{1.736677in}{2.010674in}}{\pgfqpoint{1.733405in}{2.002774in}}{\pgfqpoint{1.733405in}{1.994537in}}%
\pgfpathcurveto{\pgfqpoint{1.733405in}{1.986301in}}{\pgfqpoint{1.736677in}{1.978401in}}{\pgfqpoint{1.742501in}{1.972577in}}%
\pgfpathcurveto{\pgfqpoint{1.748325in}{1.966753in}}{\pgfqpoint{1.756225in}{1.963481in}}{\pgfqpoint{1.764462in}{1.963481in}}%
\pgfpathclose%
\pgfusepath{stroke,fill}%
\end{pgfscope}%
\begin{pgfscope}%
\pgfpathrectangle{\pgfqpoint{0.100000in}{0.212622in}}{\pgfqpoint{3.696000in}{3.696000in}}%
\pgfusepath{clip}%
\pgfsetbuttcap%
\pgfsetroundjoin%
\definecolor{currentfill}{rgb}{0.121569,0.466667,0.705882}%
\pgfsetfillcolor{currentfill}%
\pgfsetfillopacity{0.354059}%
\pgfsetlinewidth{1.003750pt}%
\definecolor{currentstroke}{rgb}{0.121569,0.466667,0.705882}%
\pgfsetstrokecolor{currentstroke}%
\pgfsetstrokeopacity{0.354059}%
\pgfsetdash{}{0pt}%
\pgfpathmoveto{\pgfqpoint{1.983485in}{2.044384in}}%
\pgfpathcurveto{\pgfqpoint{1.991721in}{2.044384in}}{\pgfqpoint{1.999621in}{2.047657in}}{\pgfqpoint{2.005445in}{2.053480in}}%
\pgfpathcurveto{\pgfqpoint{2.011269in}{2.059304in}}{\pgfqpoint{2.014541in}{2.067204in}}{\pgfqpoint{2.014541in}{2.075441in}}%
\pgfpathcurveto{\pgfqpoint{2.014541in}{2.083677in}}{\pgfqpoint{2.011269in}{2.091577in}}{\pgfqpoint{2.005445in}{2.097401in}}%
\pgfpathcurveto{\pgfqpoint{1.999621in}{2.103225in}}{\pgfqpoint{1.991721in}{2.106497in}}{\pgfqpoint{1.983485in}{2.106497in}}%
\pgfpathcurveto{\pgfqpoint{1.975249in}{2.106497in}}{\pgfqpoint{1.967348in}{2.103225in}}{\pgfqpoint{1.961525in}{2.097401in}}%
\pgfpathcurveto{\pgfqpoint{1.955701in}{2.091577in}}{\pgfqpoint{1.952428in}{2.083677in}}{\pgfqpoint{1.952428in}{2.075441in}}%
\pgfpathcurveto{\pgfqpoint{1.952428in}{2.067204in}}{\pgfqpoint{1.955701in}{2.059304in}}{\pgfqpoint{1.961525in}{2.053480in}}%
\pgfpathcurveto{\pgfqpoint{1.967348in}{2.047657in}}{\pgfqpoint{1.975249in}{2.044384in}}{\pgfqpoint{1.983485in}{2.044384in}}%
\pgfpathclose%
\pgfusepath{stroke,fill}%
\end{pgfscope}%
\begin{pgfscope}%
\pgfpathrectangle{\pgfqpoint{0.100000in}{0.212622in}}{\pgfqpoint{3.696000in}{3.696000in}}%
\pgfusepath{clip}%
\pgfsetbuttcap%
\pgfsetroundjoin%
\definecolor{currentfill}{rgb}{0.121569,0.466667,0.705882}%
\pgfsetfillcolor{currentfill}%
\pgfsetfillopacity{0.354995}%
\pgfsetlinewidth{1.003750pt}%
\definecolor{currentstroke}{rgb}{0.121569,0.466667,0.705882}%
\pgfsetstrokecolor{currentstroke}%
\pgfsetstrokeopacity{0.354995}%
\pgfsetdash{}{0pt}%
\pgfpathmoveto{\pgfqpoint{1.757273in}{1.959780in}}%
\pgfpathcurveto{\pgfqpoint{1.765509in}{1.959780in}}{\pgfqpoint{1.773409in}{1.963053in}}{\pgfqpoint{1.779233in}{1.968877in}}%
\pgfpathcurveto{\pgfqpoint{1.785057in}{1.974701in}}{\pgfqpoint{1.788330in}{1.982601in}}{\pgfqpoint{1.788330in}{1.990837in}}%
\pgfpathcurveto{\pgfqpoint{1.788330in}{1.999073in}}{\pgfqpoint{1.785057in}{2.006973in}}{\pgfqpoint{1.779233in}{2.012797in}}%
\pgfpathcurveto{\pgfqpoint{1.773409in}{2.018621in}}{\pgfqpoint{1.765509in}{2.021893in}}{\pgfqpoint{1.757273in}{2.021893in}}%
\pgfpathcurveto{\pgfqpoint{1.749037in}{2.021893in}}{\pgfqpoint{1.741137in}{2.018621in}}{\pgfqpoint{1.735313in}{2.012797in}}%
\pgfpathcurveto{\pgfqpoint{1.729489in}{2.006973in}}{\pgfqpoint{1.726217in}{1.999073in}}{\pgfqpoint{1.726217in}{1.990837in}}%
\pgfpathcurveto{\pgfqpoint{1.726217in}{1.982601in}}{\pgfqpoint{1.729489in}{1.974701in}}{\pgfqpoint{1.735313in}{1.968877in}}%
\pgfpathcurveto{\pgfqpoint{1.741137in}{1.963053in}}{\pgfqpoint{1.749037in}{1.959780in}}{\pgfqpoint{1.757273in}{1.959780in}}%
\pgfpathclose%
\pgfusepath{stroke,fill}%
\end{pgfscope}%
\begin{pgfscope}%
\pgfpathrectangle{\pgfqpoint{0.100000in}{0.212622in}}{\pgfqpoint{3.696000in}{3.696000in}}%
\pgfusepath{clip}%
\pgfsetbuttcap%
\pgfsetroundjoin%
\definecolor{currentfill}{rgb}{0.121569,0.466667,0.705882}%
\pgfsetfillcolor{currentfill}%
\pgfsetfillopacity{0.357076}%
\pgfsetlinewidth{1.003750pt}%
\definecolor{currentstroke}{rgb}{0.121569,0.466667,0.705882}%
\pgfsetstrokecolor{currentstroke}%
\pgfsetstrokeopacity{0.357076}%
\pgfsetdash{}{0pt}%
\pgfpathmoveto{\pgfqpoint{1.750968in}{1.954226in}}%
\pgfpathcurveto{\pgfqpoint{1.759205in}{1.954226in}}{\pgfqpoint{1.767105in}{1.957499in}}{\pgfqpoint{1.772929in}{1.963323in}}%
\pgfpathcurveto{\pgfqpoint{1.778752in}{1.969147in}}{\pgfqpoint{1.782025in}{1.977047in}}{\pgfqpoint{1.782025in}{1.985283in}}%
\pgfpathcurveto{\pgfqpoint{1.782025in}{1.993519in}}{\pgfqpoint{1.778752in}{2.001419in}}{\pgfqpoint{1.772929in}{2.007243in}}%
\pgfpathcurveto{\pgfqpoint{1.767105in}{2.013067in}}{\pgfqpoint{1.759205in}{2.016339in}}{\pgfqpoint{1.750968in}{2.016339in}}%
\pgfpathcurveto{\pgfqpoint{1.742732in}{2.016339in}}{\pgfqpoint{1.734832in}{2.013067in}}{\pgfqpoint{1.729008in}{2.007243in}}%
\pgfpathcurveto{\pgfqpoint{1.723184in}{2.001419in}}{\pgfqpoint{1.719912in}{1.993519in}}{\pgfqpoint{1.719912in}{1.985283in}}%
\pgfpathcurveto{\pgfqpoint{1.719912in}{1.977047in}}{\pgfqpoint{1.723184in}{1.969147in}}{\pgfqpoint{1.729008in}{1.963323in}}%
\pgfpathcurveto{\pgfqpoint{1.734832in}{1.957499in}}{\pgfqpoint{1.742732in}{1.954226in}}{\pgfqpoint{1.750968in}{1.954226in}}%
\pgfpathclose%
\pgfusepath{stroke,fill}%
\end{pgfscope}%
\begin{pgfscope}%
\pgfpathrectangle{\pgfqpoint{0.100000in}{0.212622in}}{\pgfqpoint{3.696000in}{3.696000in}}%
\pgfusepath{clip}%
\pgfsetbuttcap%
\pgfsetroundjoin%
\definecolor{currentfill}{rgb}{0.121569,0.466667,0.705882}%
\pgfsetfillcolor{currentfill}%
\pgfsetfillopacity{0.358619}%
\pgfsetlinewidth{1.003750pt}%
\definecolor{currentstroke}{rgb}{0.121569,0.466667,0.705882}%
\pgfsetstrokecolor{currentstroke}%
\pgfsetstrokeopacity{0.358619}%
\pgfsetdash{}{0pt}%
\pgfpathmoveto{\pgfqpoint{1.746234in}{1.952370in}}%
\pgfpathcurveto{\pgfqpoint{1.754471in}{1.952370in}}{\pgfqpoint{1.762371in}{1.955642in}}{\pgfqpoint{1.768195in}{1.961466in}}%
\pgfpathcurveto{\pgfqpoint{1.774019in}{1.967290in}}{\pgfqpoint{1.777291in}{1.975190in}}{\pgfqpoint{1.777291in}{1.983427in}}%
\pgfpathcurveto{\pgfqpoint{1.777291in}{1.991663in}}{\pgfqpoint{1.774019in}{1.999563in}}{\pgfqpoint{1.768195in}{2.005387in}}%
\pgfpathcurveto{\pgfqpoint{1.762371in}{2.011211in}}{\pgfqpoint{1.754471in}{2.014483in}}{\pgfqpoint{1.746234in}{2.014483in}}%
\pgfpathcurveto{\pgfqpoint{1.737998in}{2.014483in}}{\pgfqpoint{1.730098in}{2.011211in}}{\pgfqpoint{1.724274in}{2.005387in}}%
\pgfpathcurveto{\pgfqpoint{1.718450in}{1.999563in}}{\pgfqpoint{1.715178in}{1.991663in}}{\pgfqpoint{1.715178in}{1.983427in}}%
\pgfpathcurveto{\pgfqpoint{1.715178in}{1.975190in}}{\pgfqpoint{1.718450in}{1.967290in}}{\pgfqpoint{1.724274in}{1.961466in}}%
\pgfpathcurveto{\pgfqpoint{1.730098in}{1.955642in}}{\pgfqpoint{1.737998in}{1.952370in}}{\pgfqpoint{1.746234in}{1.952370in}}%
\pgfpathclose%
\pgfusepath{stroke,fill}%
\end{pgfscope}%
\begin{pgfscope}%
\pgfpathrectangle{\pgfqpoint{0.100000in}{0.212622in}}{\pgfqpoint{3.696000in}{3.696000in}}%
\pgfusepath{clip}%
\pgfsetbuttcap%
\pgfsetroundjoin%
\definecolor{currentfill}{rgb}{0.121569,0.466667,0.705882}%
\pgfsetfillcolor{currentfill}%
\pgfsetfillopacity{0.358775}%
\pgfsetlinewidth{1.003750pt}%
\definecolor{currentstroke}{rgb}{0.121569,0.466667,0.705882}%
\pgfsetstrokecolor{currentstroke}%
\pgfsetstrokeopacity{0.358775}%
\pgfsetdash{}{0pt}%
\pgfpathmoveto{\pgfqpoint{1.986542in}{2.037070in}}%
\pgfpathcurveto{\pgfqpoint{1.994778in}{2.037070in}}{\pgfqpoint{2.002678in}{2.040342in}}{\pgfqpoint{2.008502in}{2.046166in}}%
\pgfpathcurveto{\pgfqpoint{2.014326in}{2.051990in}}{\pgfqpoint{2.017598in}{2.059890in}}{\pgfqpoint{2.017598in}{2.068126in}}%
\pgfpathcurveto{\pgfqpoint{2.017598in}{2.076363in}}{\pgfqpoint{2.014326in}{2.084263in}}{\pgfqpoint{2.008502in}{2.090086in}}%
\pgfpathcurveto{\pgfqpoint{2.002678in}{2.095910in}}{\pgfqpoint{1.994778in}{2.099183in}}{\pgfqpoint{1.986542in}{2.099183in}}%
\pgfpathcurveto{\pgfqpoint{1.978306in}{2.099183in}}{\pgfqpoint{1.970406in}{2.095910in}}{\pgfqpoint{1.964582in}{2.090086in}}%
\pgfpathcurveto{\pgfqpoint{1.958758in}{2.084263in}}{\pgfqpoint{1.955485in}{2.076363in}}{\pgfqpoint{1.955485in}{2.068126in}}%
\pgfpathcurveto{\pgfqpoint{1.955485in}{2.059890in}}{\pgfqpoint{1.958758in}{2.051990in}}{\pgfqpoint{1.964582in}{2.046166in}}%
\pgfpathcurveto{\pgfqpoint{1.970406in}{2.040342in}}{\pgfqpoint{1.978306in}{2.037070in}}{\pgfqpoint{1.986542in}{2.037070in}}%
\pgfpathclose%
\pgfusepath{stroke,fill}%
\end{pgfscope}%
\begin{pgfscope}%
\pgfpathrectangle{\pgfqpoint{0.100000in}{0.212622in}}{\pgfqpoint{3.696000in}{3.696000in}}%
\pgfusepath{clip}%
\pgfsetbuttcap%
\pgfsetroundjoin%
\definecolor{currentfill}{rgb}{0.121569,0.466667,0.705882}%
\pgfsetfillcolor{currentfill}%
\pgfsetfillopacity{0.359913}%
\pgfsetlinewidth{1.003750pt}%
\definecolor{currentstroke}{rgb}{0.121569,0.466667,0.705882}%
\pgfsetstrokecolor{currentstroke}%
\pgfsetstrokeopacity{0.359913}%
\pgfsetdash{}{0pt}%
\pgfpathmoveto{\pgfqpoint{1.741943in}{1.950155in}}%
\pgfpathcurveto{\pgfqpoint{1.750180in}{1.950155in}}{\pgfqpoint{1.758080in}{1.953427in}}{\pgfqpoint{1.763904in}{1.959251in}}%
\pgfpathcurveto{\pgfqpoint{1.769728in}{1.965075in}}{\pgfqpoint{1.773000in}{1.972975in}}{\pgfqpoint{1.773000in}{1.981211in}}%
\pgfpathcurveto{\pgfqpoint{1.773000in}{1.989448in}}{\pgfqpoint{1.769728in}{1.997348in}}{\pgfqpoint{1.763904in}{2.003172in}}%
\pgfpathcurveto{\pgfqpoint{1.758080in}{2.008995in}}{\pgfqpoint{1.750180in}{2.012268in}}{\pgfqpoint{1.741943in}{2.012268in}}%
\pgfpathcurveto{\pgfqpoint{1.733707in}{2.012268in}}{\pgfqpoint{1.725807in}{2.008995in}}{\pgfqpoint{1.719983in}{2.003172in}}%
\pgfpathcurveto{\pgfqpoint{1.714159in}{1.997348in}}{\pgfqpoint{1.710887in}{1.989448in}}{\pgfqpoint{1.710887in}{1.981211in}}%
\pgfpathcurveto{\pgfqpoint{1.710887in}{1.972975in}}{\pgfqpoint{1.714159in}{1.965075in}}{\pgfqpoint{1.719983in}{1.959251in}}%
\pgfpathcurveto{\pgfqpoint{1.725807in}{1.953427in}}{\pgfqpoint{1.733707in}{1.950155in}}{\pgfqpoint{1.741943in}{1.950155in}}%
\pgfpathclose%
\pgfusepath{stroke,fill}%
\end{pgfscope}%
\begin{pgfscope}%
\pgfpathrectangle{\pgfqpoint{0.100000in}{0.212622in}}{\pgfqpoint{3.696000in}{3.696000in}}%
\pgfusepath{clip}%
\pgfsetbuttcap%
\pgfsetroundjoin%
\definecolor{currentfill}{rgb}{0.121569,0.466667,0.705882}%
\pgfsetfillcolor{currentfill}%
\pgfsetfillopacity{0.362016}%
\pgfsetlinewidth{1.003750pt}%
\definecolor{currentstroke}{rgb}{0.121569,0.466667,0.705882}%
\pgfsetstrokecolor{currentstroke}%
\pgfsetstrokeopacity{0.362016}%
\pgfsetdash{}{0pt}%
\pgfpathmoveto{\pgfqpoint{1.734151in}{1.944409in}}%
\pgfpathcurveto{\pgfqpoint{1.742387in}{1.944409in}}{\pgfqpoint{1.750287in}{1.947681in}}{\pgfqpoint{1.756111in}{1.953505in}}%
\pgfpathcurveto{\pgfqpoint{1.761935in}{1.959329in}}{\pgfqpoint{1.765207in}{1.967229in}}{\pgfqpoint{1.765207in}{1.975465in}}%
\pgfpathcurveto{\pgfqpoint{1.765207in}{1.983701in}}{\pgfqpoint{1.761935in}{1.991602in}}{\pgfqpoint{1.756111in}{1.997425in}}%
\pgfpathcurveto{\pgfqpoint{1.750287in}{2.003249in}}{\pgfqpoint{1.742387in}{2.006522in}}{\pgfqpoint{1.734151in}{2.006522in}}%
\pgfpathcurveto{\pgfqpoint{1.725914in}{2.006522in}}{\pgfqpoint{1.718014in}{2.003249in}}{\pgfqpoint{1.712190in}{1.997425in}}%
\pgfpathcurveto{\pgfqpoint{1.706366in}{1.991602in}}{\pgfqpoint{1.703094in}{1.983701in}}{\pgfqpoint{1.703094in}{1.975465in}}%
\pgfpathcurveto{\pgfqpoint{1.703094in}{1.967229in}}{\pgfqpoint{1.706366in}{1.959329in}}{\pgfqpoint{1.712190in}{1.953505in}}%
\pgfpathcurveto{\pgfqpoint{1.718014in}{1.947681in}}{\pgfqpoint{1.725914in}{1.944409in}}{\pgfqpoint{1.734151in}{1.944409in}}%
\pgfpathclose%
\pgfusepath{stroke,fill}%
\end{pgfscope}%
\begin{pgfscope}%
\pgfpathrectangle{\pgfqpoint{0.100000in}{0.212622in}}{\pgfqpoint{3.696000in}{3.696000in}}%
\pgfusepath{clip}%
\pgfsetbuttcap%
\pgfsetroundjoin%
\definecolor{currentfill}{rgb}{0.121569,0.466667,0.705882}%
\pgfsetfillcolor{currentfill}%
\pgfsetfillopacity{0.363856}%
\pgfsetlinewidth{1.003750pt}%
\definecolor{currentstroke}{rgb}{0.121569,0.466667,0.705882}%
\pgfsetstrokecolor{currentstroke}%
\pgfsetstrokeopacity{0.363856}%
\pgfsetdash{}{0pt}%
\pgfpathmoveto{\pgfqpoint{1.728678in}{1.941846in}}%
\pgfpathcurveto{\pgfqpoint{1.736915in}{1.941846in}}{\pgfqpoint{1.744815in}{1.945118in}}{\pgfqpoint{1.750639in}{1.950942in}}%
\pgfpathcurveto{\pgfqpoint{1.756463in}{1.956766in}}{\pgfqpoint{1.759735in}{1.964666in}}{\pgfqpoint{1.759735in}{1.972902in}}%
\pgfpathcurveto{\pgfqpoint{1.759735in}{1.981138in}}{\pgfqpoint{1.756463in}{1.989038in}}{\pgfqpoint{1.750639in}{1.994862in}}%
\pgfpathcurveto{\pgfqpoint{1.744815in}{2.000686in}}{\pgfqpoint{1.736915in}{2.003959in}}{\pgfqpoint{1.728678in}{2.003959in}}%
\pgfpathcurveto{\pgfqpoint{1.720442in}{2.003959in}}{\pgfqpoint{1.712542in}{2.000686in}}{\pgfqpoint{1.706718in}{1.994862in}}%
\pgfpathcurveto{\pgfqpoint{1.700894in}{1.989038in}}{\pgfqpoint{1.697622in}{1.981138in}}{\pgfqpoint{1.697622in}{1.972902in}}%
\pgfpathcurveto{\pgfqpoint{1.697622in}{1.964666in}}{\pgfqpoint{1.700894in}{1.956766in}}{\pgfqpoint{1.706718in}{1.950942in}}%
\pgfpathcurveto{\pgfqpoint{1.712542in}{1.945118in}}{\pgfqpoint{1.720442in}{1.941846in}}{\pgfqpoint{1.728678in}{1.941846in}}%
\pgfpathclose%
\pgfusepath{stroke,fill}%
\end{pgfscope}%
\begin{pgfscope}%
\pgfpathrectangle{\pgfqpoint{0.100000in}{0.212622in}}{\pgfqpoint{3.696000in}{3.696000in}}%
\pgfusepath{clip}%
\pgfsetbuttcap%
\pgfsetroundjoin%
\definecolor{currentfill}{rgb}{0.121569,0.466667,0.705882}%
\pgfsetfillcolor{currentfill}%
\pgfsetfillopacity{0.364384}%
\pgfsetlinewidth{1.003750pt}%
\definecolor{currentstroke}{rgb}{0.121569,0.466667,0.705882}%
\pgfsetstrokecolor{currentstroke}%
\pgfsetstrokeopacity{0.364384}%
\pgfsetdash{}{0pt}%
\pgfpathmoveto{\pgfqpoint{1.988793in}{2.031694in}}%
\pgfpathcurveto{\pgfqpoint{1.997029in}{2.031694in}}{\pgfqpoint{2.004929in}{2.034967in}}{\pgfqpoint{2.010753in}{2.040791in}}%
\pgfpathcurveto{\pgfqpoint{2.016577in}{2.046615in}}{\pgfqpoint{2.019850in}{2.054515in}}{\pgfqpoint{2.019850in}{2.062751in}}%
\pgfpathcurveto{\pgfqpoint{2.019850in}{2.070987in}}{\pgfqpoint{2.016577in}{2.078887in}}{\pgfqpoint{2.010753in}{2.084711in}}%
\pgfpathcurveto{\pgfqpoint{2.004929in}{2.090535in}}{\pgfqpoint{1.997029in}{2.093807in}}{\pgfqpoint{1.988793in}{2.093807in}}%
\pgfpathcurveto{\pgfqpoint{1.980557in}{2.093807in}}{\pgfqpoint{1.972657in}{2.090535in}}{\pgfqpoint{1.966833in}{2.084711in}}%
\pgfpathcurveto{\pgfqpoint{1.961009in}{2.078887in}}{\pgfqpoint{1.957737in}{2.070987in}}{\pgfqpoint{1.957737in}{2.062751in}}%
\pgfpathcurveto{\pgfqpoint{1.957737in}{2.054515in}}{\pgfqpoint{1.961009in}{2.046615in}}{\pgfqpoint{1.966833in}{2.040791in}}%
\pgfpathcurveto{\pgfqpoint{1.972657in}{2.034967in}}{\pgfqpoint{1.980557in}{2.031694in}}{\pgfqpoint{1.988793in}{2.031694in}}%
\pgfpathclose%
\pgfusepath{stroke,fill}%
\end{pgfscope}%
\begin{pgfscope}%
\pgfpathrectangle{\pgfqpoint{0.100000in}{0.212622in}}{\pgfqpoint{3.696000in}{3.696000in}}%
\pgfusepath{clip}%
\pgfsetbuttcap%
\pgfsetroundjoin%
\definecolor{currentfill}{rgb}{0.121569,0.466667,0.705882}%
\pgfsetfillcolor{currentfill}%
\pgfsetfillopacity{0.365440}%
\pgfsetlinewidth{1.003750pt}%
\definecolor{currentstroke}{rgb}{0.121569,0.466667,0.705882}%
\pgfsetstrokecolor{currentstroke}%
\pgfsetstrokeopacity{0.365440}%
\pgfsetdash{}{0pt}%
\pgfpathmoveto{\pgfqpoint{1.723833in}{1.939624in}}%
\pgfpathcurveto{\pgfqpoint{1.732069in}{1.939624in}}{\pgfqpoint{1.739969in}{1.942896in}}{\pgfqpoint{1.745793in}{1.948720in}}%
\pgfpathcurveto{\pgfqpoint{1.751617in}{1.954544in}}{\pgfqpoint{1.754889in}{1.962444in}}{\pgfqpoint{1.754889in}{1.970681in}}%
\pgfpathcurveto{\pgfqpoint{1.754889in}{1.978917in}}{\pgfqpoint{1.751617in}{1.986817in}}{\pgfqpoint{1.745793in}{1.992641in}}%
\pgfpathcurveto{\pgfqpoint{1.739969in}{1.998465in}}{\pgfqpoint{1.732069in}{2.001737in}}{\pgfqpoint{1.723833in}{2.001737in}}%
\pgfpathcurveto{\pgfqpoint{1.715596in}{2.001737in}}{\pgfqpoint{1.707696in}{1.998465in}}{\pgfqpoint{1.701872in}{1.992641in}}%
\pgfpathcurveto{\pgfqpoint{1.696049in}{1.986817in}}{\pgfqpoint{1.692776in}{1.978917in}}{\pgfqpoint{1.692776in}{1.970681in}}%
\pgfpathcurveto{\pgfqpoint{1.692776in}{1.962444in}}{\pgfqpoint{1.696049in}{1.954544in}}{\pgfqpoint{1.701872in}{1.948720in}}%
\pgfpathcurveto{\pgfqpoint{1.707696in}{1.942896in}}{\pgfqpoint{1.715596in}{1.939624in}}{\pgfqpoint{1.723833in}{1.939624in}}%
\pgfpathclose%
\pgfusepath{stroke,fill}%
\end{pgfscope}%
\begin{pgfscope}%
\pgfpathrectangle{\pgfqpoint{0.100000in}{0.212622in}}{\pgfqpoint{3.696000in}{3.696000in}}%
\pgfusepath{clip}%
\pgfsetbuttcap%
\pgfsetroundjoin%
\definecolor{currentfill}{rgb}{0.121569,0.466667,0.705882}%
\pgfsetfillcolor{currentfill}%
\pgfsetfillopacity{0.367702}%
\pgfsetlinewidth{1.003750pt}%
\definecolor{currentstroke}{rgb}{0.121569,0.466667,0.705882}%
\pgfsetstrokecolor{currentstroke}%
\pgfsetstrokeopacity{0.367702}%
\pgfsetdash{}{0pt}%
\pgfpathmoveto{\pgfqpoint{1.714850in}{1.931640in}}%
\pgfpathcurveto{\pgfqpoint{1.723086in}{1.931640in}}{\pgfqpoint{1.730986in}{1.934913in}}{\pgfqpoint{1.736810in}{1.940736in}}%
\pgfpathcurveto{\pgfqpoint{1.742634in}{1.946560in}}{\pgfqpoint{1.745906in}{1.954460in}}{\pgfqpoint{1.745906in}{1.962697in}}%
\pgfpathcurveto{\pgfqpoint{1.745906in}{1.970933in}}{\pgfqpoint{1.742634in}{1.978833in}}{\pgfqpoint{1.736810in}{1.984657in}}%
\pgfpathcurveto{\pgfqpoint{1.730986in}{1.990481in}}{\pgfqpoint{1.723086in}{1.993753in}}{\pgfqpoint{1.714850in}{1.993753in}}%
\pgfpathcurveto{\pgfqpoint{1.706613in}{1.993753in}}{\pgfqpoint{1.698713in}{1.990481in}}{\pgfqpoint{1.692889in}{1.984657in}}%
\pgfpathcurveto{\pgfqpoint{1.687065in}{1.978833in}}{\pgfqpoint{1.683793in}{1.970933in}}{\pgfqpoint{1.683793in}{1.962697in}}%
\pgfpathcurveto{\pgfqpoint{1.683793in}{1.954460in}}{\pgfqpoint{1.687065in}{1.946560in}}{\pgfqpoint{1.692889in}{1.940736in}}%
\pgfpathcurveto{\pgfqpoint{1.698713in}{1.934913in}}{\pgfqpoint{1.706613in}{1.931640in}}{\pgfqpoint{1.714850in}{1.931640in}}%
\pgfpathclose%
\pgfusepath{stroke,fill}%
\end{pgfscope}%
\begin{pgfscope}%
\pgfpathrectangle{\pgfqpoint{0.100000in}{0.212622in}}{\pgfqpoint{3.696000in}{3.696000in}}%
\pgfusepath{clip}%
\pgfsetbuttcap%
\pgfsetroundjoin%
\definecolor{currentfill}{rgb}{0.121569,0.466667,0.705882}%
\pgfsetfillcolor{currentfill}%
\pgfsetfillopacity{0.370279}%
\pgfsetlinewidth{1.003750pt}%
\definecolor{currentstroke}{rgb}{0.121569,0.466667,0.705882}%
\pgfsetstrokecolor{currentstroke}%
\pgfsetstrokeopacity{0.370279}%
\pgfsetdash{}{0pt}%
\pgfpathmoveto{\pgfqpoint{1.993279in}{2.027046in}}%
\pgfpathcurveto{\pgfqpoint{2.001515in}{2.027046in}}{\pgfqpoint{2.009416in}{2.030319in}}{\pgfqpoint{2.015239in}{2.036143in}}%
\pgfpathcurveto{\pgfqpoint{2.021063in}{2.041966in}}{\pgfqpoint{2.024336in}{2.049866in}}{\pgfqpoint{2.024336in}{2.058103in}}%
\pgfpathcurveto{\pgfqpoint{2.024336in}{2.066339in}}{\pgfqpoint{2.021063in}{2.074239in}}{\pgfqpoint{2.015239in}{2.080063in}}%
\pgfpathcurveto{\pgfqpoint{2.009416in}{2.085887in}}{\pgfqpoint{2.001515in}{2.089159in}}{\pgfqpoint{1.993279in}{2.089159in}}%
\pgfpathcurveto{\pgfqpoint{1.985043in}{2.089159in}}{\pgfqpoint{1.977143in}{2.085887in}}{\pgfqpoint{1.971319in}{2.080063in}}%
\pgfpathcurveto{\pgfqpoint{1.965495in}{2.074239in}}{\pgfqpoint{1.962223in}{2.066339in}}{\pgfqpoint{1.962223in}{2.058103in}}%
\pgfpathcurveto{\pgfqpoint{1.962223in}{2.049866in}}{\pgfqpoint{1.965495in}{2.041966in}}{\pgfqpoint{1.971319in}{2.036143in}}%
\pgfpathcurveto{\pgfqpoint{1.977143in}{2.030319in}}{\pgfqpoint{1.985043in}{2.027046in}}{\pgfqpoint{1.993279in}{2.027046in}}%
\pgfpathclose%
\pgfusepath{stroke,fill}%
\end{pgfscope}%
\begin{pgfscope}%
\pgfpathrectangle{\pgfqpoint{0.100000in}{0.212622in}}{\pgfqpoint{3.696000in}{3.696000in}}%
\pgfusepath{clip}%
\pgfsetbuttcap%
\pgfsetroundjoin%
\definecolor{currentfill}{rgb}{0.121569,0.466667,0.705882}%
\pgfsetfillcolor{currentfill}%
\pgfsetfillopacity{0.370839}%
\pgfsetlinewidth{1.003750pt}%
\definecolor{currentstroke}{rgb}{0.121569,0.466667,0.705882}%
\pgfsetstrokecolor{currentstroke}%
\pgfsetstrokeopacity{0.370839}%
\pgfsetdash{}{0pt}%
\pgfpathmoveto{\pgfqpoint{1.706593in}{1.930700in}}%
\pgfpathcurveto{\pgfqpoint{1.714830in}{1.930700in}}{\pgfqpoint{1.722730in}{1.933972in}}{\pgfqpoint{1.728554in}{1.939796in}}%
\pgfpathcurveto{\pgfqpoint{1.734378in}{1.945620in}}{\pgfqpoint{1.737650in}{1.953520in}}{\pgfqpoint{1.737650in}{1.961757in}}%
\pgfpathcurveto{\pgfqpoint{1.737650in}{1.969993in}}{\pgfqpoint{1.734378in}{1.977893in}}{\pgfqpoint{1.728554in}{1.983717in}}%
\pgfpathcurveto{\pgfqpoint{1.722730in}{1.989541in}}{\pgfqpoint{1.714830in}{1.992813in}}{\pgfqpoint{1.706593in}{1.992813in}}%
\pgfpathcurveto{\pgfqpoint{1.698357in}{1.992813in}}{\pgfqpoint{1.690457in}{1.989541in}}{\pgfqpoint{1.684633in}{1.983717in}}%
\pgfpathcurveto{\pgfqpoint{1.678809in}{1.977893in}}{\pgfqpoint{1.675537in}{1.969993in}}{\pgfqpoint{1.675537in}{1.961757in}}%
\pgfpathcurveto{\pgfqpoint{1.675537in}{1.953520in}}{\pgfqpoint{1.678809in}{1.945620in}}{\pgfqpoint{1.684633in}{1.939796in}}%
\pgfpathcurveto{\pgfqpoint{1.690457in}{1.933972in}}{\pgfqpoint{1.698357in}{1.930700in}}{\pgfqpoint{1.706593in}{1.930700in}}%
\pgfpathclose%
\pgfusepath{stroke,fill}%
\end{pgfscope}%
\begin{pgfscope}%
\pgfpathrectangle{\pgfqpoint{0.100000in}{0.212622in}}{\pgfqpoint{3.696000in}{3.696000in}}%
\pgfusepath{clip}%
\pgfsetbuttcap%
\pgfsetroundjoin%
\definecolor{currentfill}{rgb}{0.121569,0.466667,0.705882}%
\pgfsetfillcolor{currentfill}%
\pgfsetfillopacity{0.372786}%
\pgfsetlinewidth{1.003750pt}%
\definecolor{currentstroke}{rgb}{0.121569,0.466667,0.705882}%
\pgfsetstrokecolor{currentstroke}%
\pgfsetstrokeopacity{0.372786}%
\pgfsetdash{}{0pt}%
\pgfpathmoveto{\pgfqpoint{1.699975in}{1.924270in}}%
\pgfpathcurveto{\pgfqpoint{1.708211in}{1.924270in}}{\pgfqpoint{1.716111in}{1.927542in}}{\pgfqpoint{1.721935in}{1.933366in}}%
\pgfpathcurveto{\pgfqpoint{1.727759in}{1.939190in}}{\pgfqpoint{1.731031in}{1.947090in}}{\pgfqpoint{1.731031in}{1.955326in}}%
\pgfpathcurveto{\pgfqpoint{1.731031in}{1.963563in}}{\pgfqpoint{1.727759in}{1.971463in}}{\pgfqpoint{1.721935in}{1.977287in}}%
\pgfpathcurveto{\pgfqpoint{1.716111in}{1.983111in}}{\pgfqpoint{1.708211in}{1.986383in}}{\pgfqpoint{1.699975in}{1.986383in}}%
\pgfpathcurveto{\pgfqpoint{1.691739in}{1.986383in}}{\pgfqpoint{1.683839in}{1.983111in}}{\pgfqpoint{1.678015in}{1.977287in}}%
\pgfpathcurveto{\pgfqpoint{1.672191in}{1.971463in}}{\pgfqpoint{1.668918in}{1.963563in}}{\pgfqpoint{1.668918in}{1.955326in}}%
\pgfpathcurveto{\pgfqpoint{1.668918in}{1.947090in}}{\pgfqpoint{1.672191in}{1.939190in}}{\pgfqpoint{1.678015in}{1.933366in}}%
\pgfpathcurveto{\pgfqpoint{1.683839in}{1.927542in}}{\pgfqpoint{1.691739in}{1.924270in}}{\pgfqpoint{1.699975in}{1.924270in}}%
\pgfpathclose%
\pgfusepath{stroke,fill}%
\end{pgfscope}%
\begin{pgfscope}%
\pgfpathrectangle{\pgfqpoint{0.100000in}{0.212622in}}{\pgfqpoint{3.696000in}{3.696000in}}%
\pgfusepath{clip}%
\pgfsetbuttcap%
\pgfsetroundjoin%
\definecolor{currentfill}{rgb}{0.121569,0.466667,0.705882}%
\pgfsetfillcolor{currentfill}%
\pgfsetfillopacity{0.374793}%
\pgfsetlinewidth{1.003750pt}%
\definecolor{currentstroke}{rgb}{0.121569,0.466667,0.705882}%
\pgfsetstrokecolor{currentstroke}%
\pgfsetstrokeopacity{0.374793}%
\pgfsetdash{}{0pt}%
\pgfpathmoveto{\pgfqpoint{1.693105in}{1.919975in}}%
\pgfpathcurveto{\pgfqpoint{1.701342in}{1.919975in}}{\pgfqpoint{1.709242in}{1.923247in}}{\pgfqpoint{1.715066in}{1.929071in}}%
\pgfpathcurveto{\pgfqpoint{1.720890in}{1.934895in}}{\pgfqpoint{1.724162in}{1.942795in}}{\pgfqpoint{1.724162in}{1.951031in}}%
\pgfpathcurveto{\pgfqpoint{1.724162in}{1.959267in}}{\pgfqpoint{1.720890in}{1.967167in}}{\pgfqpoint{1.715066in}{1.972991in}}%
\pgfpathcurveto{\pgfqpoint{1.709242in}{1.978815in}}{\pgfqpoint{1.701342in}{1.982087in}}{\pgfqpoint{1.693105in}{1.982087in}}%
\pgfpathcurveto{\pgfqpoint{1.684869in}{1.982087in}}{\pgfqpoint{1.676969in}{1.978815in}}{\pgfqpoint{1.671145in}{1.972991in}}%
\pgfpathcurveto{\pgfqpoint{1.665321in}{1.967167in}}{\pgfqpoint{1.662049in}{1.959267in}}{\pgfqpoint{1.662049in}{1.951031in}}%
\pgfpathcurveto{\pgfqpoint{1.662049in}{1.942795in}}{\pgfqpoint{1.665321in}{1.934895in}}{\pgfqpoint{1.671145in}{1.929071in}}%
\pgfpathcurveto{\pgfqpoint{1.676969in}{1.923247in}}{\pgfqpoint{1.684869in}{1.919975in}}{\pgfqpoint{1.693105in}{1.919975in}}%
\pgfpathclose%
\pgfusepath{stroke,fill}%
\end{pgfscope}%
\begin{pgfscope}%
\pgfpathrectangle{\pgfqpoint{0.100000in}{0.212622in}}{\pgfqpoint{3.696000in}{3.696000in}}%
\pgfusepath{clip}%
\pgfsetbuttcap%
\pgfsetroundjoin%
\definecolor{currentfill}{rgb}{0.121569,0.466667,0.705882}%
\pgfsetfillcolor{currentfill}%
\pgfsetfillopacity{0.376719}%
\pgfsetlinewidth{1.003750pt}%
\definecolor{currentstroke}{rgb}{0.121569,0.466667,0.705882}%
\pgfsetstrokecolor{currentstroke}%
\pgfsetstrokeopacity{0.376719}%
\pgfsetdash{}{0pt}%
\pgfpathmoveto{\pgfqpoint{1.687395in}{1.916497in}}%
\pgfpathcurveto{\pgfqpoint{1.695632in}{1.916497in}}{\pgfqpoint{1.703532in}{1.919769in}}{\pgfqpoint{1.709356in}{1.925593in}}%
\pgfpathcurveto{\pgfqpoint{1.715180in}{1.931417in}}{\pgfqpoint{1.718452in}{1.939317in}}{\pgfqpoint{1.718452in}{1.947553in}}%
\pgfpathcurveto{\pgfqpoint{1.718452in}{1.955790in}}{\pgfqpoint{1.715180in}{1.963690in}}{\pgfqpoint{1.709356in}{1.969514in}}%
\pgfpathcurveto{\pgfqpoint{1.703532in}{1.975338in}}{\pgfqpoint{1.695632in}{1.978610in}}{\pgfqpoint{1.687395in}{1.978610in}}%
\pgfpathcurveto{\pgfqpoint{1.679159in}{1.978610in}}{\pgfqpoint{1.671259in}{1.975338in}}{\pgfqpoint{1.665435in}{1.969514in}}%
\pgfpathcurveto{\pgfqpoint{1.659611in}{1.963690in}}{\pgfqpoint{1.656339in}{1.955790in}}{\pgfqpoint{1.656339in}{1.947553in}}%
\pgfpathcurveto{\pgfqpoint{1.656339in}{1.939317in}}{\pgfqpoint{1.659611in}{1.931417in}}{\pgfqpoint{1.665435in}{1.925593in}}%
\pgfpathcurveto{\pgfqpoint{1.671259in}{1.919769in}}{\pgfqpoint{1.679159in}{1.916497in}}{\pgfqpoint{1.687395in}{1.916497in}}%
\pgfpathclose%
\pgfusepath{stroke,fill}%
\end{pgfscope}%
\begin{pgfscope}%
\pgfpathrectangle{\pgfqpoint{0.100000in}{0.212622in}}{\pgfqpoint{3.696000in}{3.696000in}}%
\pgfusepath{clip}%
\pgfsetbuttcap%
\pgfsetroundjoin%
\definecolor{currentfill}{rgb}{0.121569,0.466667,0.705882}%
\pgfsetfillcolor{currentfill}%
\pgfsetfillopacity{0.376981}%
\pgfsetlinewidth{1.003750pt}%
\definecolor{currentstroke}{rgb}{0.121569,0.466667,0.705882}%
\pgfsetstrokecolor{currentstroke}%
\pgfsetstrokeopacity{0.376981}%
\pgfsetdash{}{0pt}%
\pgfpathmoveto{\pgfqpoint{1.995215in}{2.025193in}}%
\pgfpathcurveto{\pgfqpoint{2.003452in}{2.025193in}}{\pgfqpoint{2.011352in}{2.028465in}}{\pgfqpoint{2.017176in}{2.034289in}}%
\pgfpathcurveto{\pgfqpoint{2.022999in}{2.040113in}}{\pgfqpoint{2.026272in}{2.048013in}}{\pgfqpoint{2.026272in}{2.056249in}}%
\pgfpathcurveto{\pgfqpoint{2.026272in}{2.064486in}}{\pgfqpoint{2.022999in}{2.072386in}}{\pgfqpoint{2.017176in}{2.078210in}}%
\pgfpathcurveto{\pgfqpoint{2.011352in}{2.084034in}}{\pgfqpoint{2.003452in}{2.087306in}}{\pgfqpoint{1.995215in}{2.087306in}}%
\pgfpathcurveto{\pgfqpoint{1.986979in}{2.087306in}}{\pgfqpoint{1.979079in}{2.084034in}}{\pgfqpoint{1.973255in}{2.078210in}}%
\pgfpathcurveto{\pgfqpoint{1.967431in}{2.072386in}}{\pgfqpoint{1.964159in}{2.064486in}}{\pgfqpoint{1.964159in}{2.056249in}}%
\pgfpathcurveto{\pgfqpoint{1.964159in}{2.048013in}}{\pgfqpoint{1.967431in}{2.040113in}}{\pgfqpoint{1.973255in}{2.034289in}}%
\pgfpathcurveto{\pgfqpoint{1.979079in}{2.028465in}}{\pgfqpoint{1.986979in}{2.025193in}}{\pgfqpoint{1.995215in}{2.025193in}}%
\pgfpathclose%
\pgfusepath{stroke,fill}%
\end{pgfscope}%
\begin{pgfscope}%
\pgfpathrectangle{\pgfqpoint{0.100000in}{0.212622in}}{\pgfqpoint{3.696000in}{3.696000in}}%
\pgfusepath{clip}%
\pgfsetbuttcap%
\pgfsetroundjoin%
\definecolor{currentfill}{rgb}{0.121569,0.466667,0.705882}%
\pgfsetfillcolor{currentfill}%
\pgfsetfillopacity{0.377983}%
\pgfsetlinewidth{1.003750pt}%
\definecolor{currentstroke}{rgb}{0.121569,0.466667,0.705882}%
\pgfsetstrokecolor{currentstroke}%
\pgfsetstrokeopacity{0.377983}%
\pgfsetdash{}{0pt}%
\pgfpathmoveto{\pgfqpoint{1.682798in}{1.910164in}}%
\pgfpathcurveto{\pgfqpoint{1.691034in}{1.910164in}}{\pgfqpoint{1.698934in}{1.913436in}}{\pgfqpoint{1.704758in}{1.919260in}}%
\pgfpathcurveto{\pgfqpoint{1.710582in}{1.925084in}}{\pgfqpoint{1.713854in}{1.932984in}}{\pgfqpoint{1.713854in}{1.941220in}}%
\pgfpathcurveto{\pgfqpoint{1.713854in}{1.949456in}}{\pgfqpoint{1.710582in}{1.957356in}}{\pgfqpoint{1.704758in}{1.963180in}}%
\pgfpathcurveto{\pgfqpoint{1.698934in}{1.969004in}}{\pgfqpoint{1.691034in}{1.972277in}}{\pgfqpoint{1.682798in}{1.972277in}}%
\pgfpathcurveto{\pgfqpoint{1.674562in}{1.972277in}}{\pgfqpoint{1.666662in}{1.969004in}}{\pgfqpoint{1.660838in}{1.963180in}}%
\pgfpathcurveto{\pgfqpoint{1.655014in}{1.957356in}}{\pgfqpoint{1.651741in}{1.949456in}}{\pgfqpoint{1.651741in}{1.941220in}}%
\pgfpathcurveto{\pgfqpoint{1.651741in}{1.932984in}}{\pgfqpoint{1.655014in}{1.925084in}}{\pgfqpoint{1.660838in}{1.919260in}}%
\pgfpathcurveto{\pgfqpoint{1.666662in}{1.913436in}}{\pgfqpoint{1.674562in}{1.910164in}}{\pgfqpoint{1.682798in}{1.910164in}}%
\pgfpathclose%
\pgfusepath{stroke,fill}%
\end{pgfscope}%
\begin{pgfscope}%
\pgfpathrectangle{\pgfqpoint{0.100000in}{0.212622in}}{\pgfqpoint{3.696000in}{3.696000in}}%
\pgfusepath{clip}%
\pgfsetbuttcap%
\pgfsetroundjoin%
\definecolor{currentfill}{rgb}{0.121569,0.466667,0.705882}%
\pgfsetfillcolor{currentfill}%
\pgfsetfillopacity{0.378875}%
\pgfsetlinewidth{1.003750pt}%
\definecolor{currentstroke}{rgb}{0.121569,0.466667,0.705882}%
\pgfsetstrokecolor{currentstroke}%
\pgfsetstrokeopacity{0.378875}%
\pgfsetdash{}{0pt}%
\pgfpathmoveto{\pgfqpoint{1.679857in}{1.907925in}}%
\pgfpathcurveto{\pgfqpoint{1.688093in}{1.907925in}}{\pgfqpoint{1.695993in}{1.911197in}}{\pgfqpoint{1.701817in}{1.917021in}}%
\pgfpathcurveto{\pgfqpoint{1.707641in}{1.922845in}}{\pgfqpoint{1.710913in}{1.930745in}}{\pgfqpoint{1.710913in}{1.938981in}}%
\pgfpathcurveto{\pgfqpoint{1.710913in}{1.947217in}}{\pgfqpoint{1.707641in}{1.955117in}}{\pgfqpoint{1.701817in}{1.960941in}}%
\pgfpathcurveto{\pgfqpoint{1.695993in}{1.966765in}}{\pgfqpoint{1.688093in}{1.970038in}}{\pgfqpoint{1.679857in}{1.970038in}}%
\pgfpathcurveto{\pgfqpoint{1.671620in}{1.970038in}}{\pgfqpoint{1.663720in}{1.966765in}}{\pgfqpoint{1.657896in}{1.960941in}}%
\pgfpathcurveto{\pgfqpoint{1.652073in}{1.955117in}}{\pgfqpoint{1.648800in}{1.947217in}}{\pgfqpoint{1.648800in}{1.938981in}}%
\pgfpathcurveto{\pgfqpoint{1.648800in}{1.930745in}}{\pgfqpoint{1.652073in}{1.922845in}}{\pgfqpoint{1.657896in}{1.917021in}}%
\pgfpathcurveto{\pgfqpoint{1.663720in}{1.911197in}}{\pgfqpoint{1.671620in}{1.907925in}}{\pgfqpoint{1.679857in}{1.907925in}}%
\pgfpathclose%
\pgfusepath{stroke,fill}%
\end{pgfscope}%
\begin{pgfscope}%
\pgfpathrectangle{\pgfqpoint{0.100000in}{0.212622in}}{\pgfqpoint{3.696000in}{3.696000in}}%
\pgfusepath{clip}%
\pgfsetbuttcap%
\pgfsetroundjoin%
\definecolor{currentfill}{rgb}{0.121569,0.466667,0.705882}%
\pgfsetfillcolor{currentfill}%
\pgfsetfillopacity{0.379619}%
\pgfsetlinewidth{1.003750pt}%
\definecolor{currentstroke}{rgb}{0.121569,0.466667,0.705882}%
\pgfsetstrokecolor{currentstroke}%
\pgfsetstrokeopacity{0.379619}%
\pgfsetdash{}{0pt}%
\pgfpathmoveto{\pgfqpoint{1.677337in}{1.906266in}}%
\pgfpathcurveto{\pgfqpoint{1.685573in}{1.906266in}}{\pgfqpoint{1.693473in}{1.909538in}}{\pgfqpoint{1.699297in}{1.915362in}}%
\pgfpathcurveto{\pgfqpoint{1.705121in}{1.921186in}}{\pgfqpoint{1.708393in}{1.929086in}}{\pgfqpoint{1.708393in}{1.937322in}}%
\pgfpathcurveto{\pgfqpoint{1.708393in}{1.945558in}}{\pgfqpoint{1.705121in}{1.953458in}}{\pgfqpoint{1.699297in}{1.959282in}}%
\pgfpathcurveto{\pgfqpoint{1.693473in}{1.965106in}}{\pgfqpoint{1.685573in}{1.968379in}}{\pgfqpoint{1.677337in}{1.968379in}}%
\pgfpathcurveto{\pgfqpoint{1.669101in}{1.968379in}}{\pgfqpoint{1.661201in}{1.965106in}}{\pgfqpoint{1.655377in}{1.959282in}}%
\pgfpathcurveto{\pgfqpoint{1.649553in}{1.953458in}}{\pgfqpoint{1.646280in}{1.945558in}}{\pgfqpoint{1.646280in}{1.937322in}}%
\pgfpathcurveto{\pgfqpoint{1.646280in}{1.929086in}}{\pgfqpoint{1.649553in}{1.921186in}}{\pgfqpoint{1.655377in}{1.915362in}}%
\pgfpathcurveto{\pgfqpoint{1.661201in}{1.909538in}}{\pgfqpoint{1.669101in}{1.906266in}}{\pgfqpoint{1.677337in}{1.906266in}}%
\pgfpathclose%
\pgfusepath{stroke,fill}%
\end{pgfscope}%
\begin{pgfscope}%
\pgfpathrectangle{\pgfqpoint{0.100000in}{0.212622in}}{\pgfqpoint{3.696000in}{3.696000in}}%
\pgfusepath{clip}%
\pgfsetbuttcap%
\pgfsetroundjoin%
\definecolor{currentfill}{rgb}{0.121569,0.466667,0.705882}%
\pgfsetfillcolor{currentfill}%
\pgfsetfillopacity{0.380790}%
\pgfsetlinewidth{1.003750pt}%
\definecolor{currentstroke}{rgb}{0.121569,0.466667,0.705882}%
\pgfsetstrokecolor{currentstroke}%
\pgfsetstrokeopacity{0.380790}%
\pgfsetdash{}{0pt}%
\pgfpathmoveto{\pgfqpoint{1.673425in}{1.901120in}}%
\pgfpathcurveto{\pgfqpoint{1.681661in}{1.901120in}}{\pgfqpoint{1.689561in}{1.904392in}}{\pgfqpoint{1.695385in}{1.910216in}}%
\pgfpathcurveto{\pgfqpoint{1.701209in}{1.916040in}}{\pgfqpoint{1.704482in}{1.923940in}}{\pgfqpoint{1.704482in}{1.932176in}}%
\pgfpathcurveto{\pgfqpoint{1.704482in}{1.940412in}}{\pgfqpoint{1.701209in}{1.948313in}}{\pgfqpoint{1.695385in}{1.954136in}}%
\pgfpathcurveto{\pgfqpoint{1.689561in}{1.959960in}}{\pgfqpoint{1.681661in}{1.963233in}}{\pgfqpoint{1.673425in}{1.963233in}}%
\pgfpathcurveto{\pgfqpoint{1.665189in}{1.963233in}}{\pgfqpoint{1.657289in}{1.959960in}}{\pgfqpoint{1.651465in}{1.954136in}}%
\pgfpathcurveto{\pgfqpoint{1.645641in}{1.948313in}}{\pgfqpoint{1.642369in}{1.940412in}}{\pgfqpoint{1.642369in}{1.932176in}}%
\pgfpathcurveto{\pgfqpoint{1.642369in}{1.923940in}}{\pgfqpoint{1.645641in}{1.916040in}}{\pgfqpoint{1.651465in}{1.910216in}}%
\pgfpathcurveto{\pgfqpoint{1.657289in}{1.904392in}}{\pgfqpoint{1.665189in}{1.901120in}}{\pgfqpoint{1.673425in}{1.901120in}}%
\pgfpathclose%
\pgfusepath{stroke,fill}%
\end{pgfscope}%
\begin{pgfscope}%
\pgfpathrectangle{\pgfqpoint{0.100000in}{0.212622in}}{\pgfqpoint{3.696000in}{3.696000in}}%
\pgfusepath{clip}%
\pgfsetbuttcap%
\pgfsetroundjoin%
\definecolor{currentfill}{rgb}{0.121569,0.466667,0.705882}%
\pgfsetfillcolor{currentfill}%
\pgfsetfillopacity{0.381319}%
\pgfsetlinewidth{1.003750pt}%
\definecolor{currentstroke}{rgb}{0.121569,0.466667,0.705882}%
\pgfsetstrokecolor{currentstroke}%
\pgfsetstrokeopacity{0.381319}%
\pgfsetdash{}{0pt}%
\pgfpathmoveto{\pgfqpoint{1.671356in}{1.899844in}}%
\pgfpathcurveto{\pgfqpoint{1.679592in}{1.899844in}}{\pgfqpoint{1.687492in}{1.903116in}}{\pgfqpoint{1.693316in}{1.908940in}}%
\pgfpathcurveto{\pgfqpoint{1.699140in}{1.914764in}}{\pgfqpoint{1.702413in}{1.922664in}}{\pgfqpoint{1.702413in}{1.930900in}}%
\pgfpathcurveto{\pgfqpoint{1.702413in}{1.939137in}}{\pgfqpoint{1.699140in}{1.947037in}}{\pgfqpoint{1.693316in}{1.952861in}}%
\pgfpathcurveto{\pgfqpoint{1.687492in}{1.958684in}}{\pgfqpoint{1.679592in}{1.961957in}}{\pgfqpoint{1.671356in}{1.961957in}}%
\pgfpathcurveto{\pgfqpoint{1.663120in}{1.961957in}}{\pgfqpoint{1.655220in}{1.958684in}}{\pgfqpoint{1.649396in}{1.952861in}}%
\pgfpathcurveto{\pgfqpoint{1.643572in}{1.947037in}}{\pgfqpoint{1.640300in}{1.939137in}}{\pgfqpoint{1.640300in}{1.930900in}}%
\pgfpathcurveto{\pgfqpoint{1.640300in}{1.922664in}}{\pgfqpoint{1.643572in}{1.914764in}}{\pgfqpoint{1.649396in}{1.908940in}}%
\pgfpathcurveto{\pgfqpoint{1.655220in}{1.903116in}}{\pgfqpoint{1.663120in}{1.899844in}}{\pgfqpoint{1.671356in}{1.899844in}}%
\pgfpathclose%
\pgfusepath{stroke,fill}%
\end{pgfscope}%
\begin{pgfscope}%
\pgfpathrectangle{\pgfqpoint{0.100000in}{0.212622in}}{\pgfqpoint{3.696000in}{3.696000in}}%
\pgfusepath{clip}%
\pgfsetbuttcap%
\pgfsetroundjoin%
\definecolor{currentfill}{rgb}{0.121569,0.466667,0.705882}%
\pgfsetfillcolor{currentfill}%
\pgfsetfillopacity{0.382259}%
\pgfsetlinewidth{1.003750pt}%
\definecolor{currentstroke}{rgb}{0.121569,0.466667,0.705882}%
\pgfsetstrokecolor{currentstroke}%
\pgfsetstrokeopacity{0.382259}%
\pgfsetdash{}{0pt}%
\pgfpathmoveto{\pgfqpoint{1.667875in}{1.896915in}}%
\pgfpathcurveto{\pgfqpoint{1.676111in}{1.896915in}}{\pgfqpoint{1.684011in}{1.900188in}}{\pgfqpoint{1.689835in}{1.906012in}}%
\pgfpathcurveto{\pgfqpoint{1.695659in}{1.911835in}}{\pgfqpoint{1.698932in}{1.919735in}}{\pgfqpoint{1.698932in}{1.927972in}}%
\pgfpathcurveto{\pgfqpoint{1.698932in}{1.936208in}}{\pgfqpoint{1.695659in}{1.944108in}}{\pgfqpoint{1.689835in}{1.949932in}}%
\pgfpathcurveto{\pgfqpoint{1.684011in}{1.955756in}}{\pgfqpoint{1.676111in}{1.959028in}}{\pgfqpoint{1.667875in}{1.959028in}}%
\pgfpathcurveto{\pgfqpoint{1.659639in}{1.959028in}}{\pgfqpoint{1.651739in}{1.955756in}}{\pgfqpoint{1.645915in}{1.949932in}}%
\pgfpathcurveto{\pgfqpoint{1.640091in}{1.944108in}}{\pgfqpoint{1.636819in}{1.936208in}}{\pgfqpoint{1.636819in}{1.927972in}}%
\pgfpathcurveto{\pgfqpoint{1.636819in}{1.919735in}}{\pgfqpoint{1.640091in}{1.911835in}}{\pgfqpoint{1.645915in}{1.906012in}}%
\pgfpathcurveto{\pgfqpoint{1.651739in}{1.900188in}}{\pgfqpoint{1.659639in}{1.896915in}}{\pgfqpoint{1.667875in}{1.896915in}}%
\pgfpathclose%
\pgfusepath{stroke,fill}%
\end{pgfscope}%
\begin{pgfscope}%
\pgfpathrectangle{\pgfqpoint{0.100000in}{0.212622in}}{\pgfqpoint{3.696000in}{3.696000in}}%
\pgfusepath{clip}%
\pgfsetbuttcap%
\pgfsetroundjoin%
\definecolor{currentfill}{rgb}{0.121569,0.466667,0.705882}%
\pgfsetfillcolor{currentfill}%
\pgfsetfillopacity{0.383429}%
\pgfsetlinewidth{1.003750pt}%
\definecolor{currentstroke}{rgb}{0.121569,0.466667,0.705882}%
\pgfsetstrokecolor{currentstroke}%
\pgfsetstrokeopacity{0.383429}%
\pgfsetdash{}{0pt}%
\pgfpathmoveto{\pgfqpoint{1.995974in}{2.014816in}}%
\pgfpathcurveto{\pgfqpoint{2.004211in}{2.014816in}}{\pgfqpoint{2.012111in}{2.018088in}}{\pgfqpoint{2.017935in}{2.023912in}}%
\pgfpathcurveto{\pgfqpoint{2.023759in}{2.029736in}}{\pgfqpoint{2.027031in}{2.037636in}}{\pgfqpoint{2.027031in}{2.045872in}}%
\pgfpathcurveto{\pgfqpoint{2.027031in}{2.054108in}}{\pgfqpoint{2.023759in}{2.062008in}}{\pgfqpoint{2.017935in}{2.067832in}}%
\pgfpathcurveto{\pgfqpoint{2.012111in}{2.073656in}}{\pgfqpoint{2.004211in}{2.076929in}}{\pgfqpoint{1.995974in}{2.076929in}}%
\pgfpathcurveto{\pgfqpoint{1.987738in}{2.076929in}}{\pgfqpoint{1.979838in}{2.073656in}}{\pgfqpoint{1.974014in}{2.067832in}}%
\pgfpathcurveto{\pgfqpoint{1.968190in}{2.062008in}}{\pgfqpoint{1.964918in}{2.054108in}}{\pgfqpoint{1.964918in}{2.045872in}}%
\pgfpathcurveto{\pgfqpoint{1.964918in}{2.037636in}}{\pgfqpoint{1.968190in}{2.029736in}}{\pgfqpoint{1.974014in}{2.023912in}}%
\pgfpathcurveto{\pgfqpoint{1.979838in}{2.018088in}}{\pgfqpoint{1.987738in}{2.014816in}}{\pgfqpoint{1.995974in}{2.014816in}}%
\pgfpathclose%
\pgfusepath{stroke,fill}%
\end{pgfscope}%
\begin{pgfscope}%
\pgfpathrectangle{\pgfqpoint{0.100000in}{0.212622in}}{\pgfqpoint{3.696000in}{3.696000in}}%
\pgfusepath{clip}%
\pgfsetbuttcap%
\pgfsetroundjoin%
\definecolor{currentfill}{rgb}{0.121569,0.466667,0.705882}%
\pgfsetfillcolor{currentfill}%
\pgfsetfillopacity{0.383666}%
\pgfsetlinewidth{1.003750pt}%
\definecolor{currentstroke}{rgb}{0.121569,0.466667,0.705882}%
\pgfsetstrokecolor{currentstroke}%
\pgfsetstrokeopacity{0.383666}%
\pgfsetdash{}{0pt}%
\pgfpathmoveto{\pgfqpoint{1.662488in}{1.888282in}}%
\pgfpathcurveto{\pgfqpoint{1.670724in}{1.888282in}}{\pgfqpoint{1.678624in}{1.891554in}}{\pgfqpoint{1.684448in}{1.897378in}}%
\pgfpathcurveto{\pgfqpoint{1.690272in}{1.903202in}}{\pgfqpoint{1.693544in}{1.911102in}}{\pgfqpoint{1.693544in}{1.919338in}}%
\pgfpathcurveto{\pgfqpoint{1.693544in}{1.927574in}}{\pgfqpoint{1.690272in}{1.935475in}}{\pgfqpoint{1.684448in}{1.941298in}}%
\pgfpathcurveto{\pgfqpoint{1.678624in}{1.947122in}}{\pgfqpoint{1.670724in}{1.950395in}}{\pgfqpoint{1.662488in}{1.950395in}}%
\pgfpathcurveto{\pgfqpoint{1.654252in}{1.950395in}}{\pgfqpoint{1.646352in}{1.947122in}}{\pgfqpoint{1.640528in}{1.941298in}}%
\pgfpathcurveto{\pgfqpoint{1.634704in}{1.935475in}}{\pgfqpoint{1.631431in}{1.927574in}}{\pgfqpoint{1.631431in}{1.919338in}}%
\pgfpathcurveto{\pgfqpoint{1.631431in}{1.911102in}}{\pgfqpoint{1.634704in}{1.903202in}}{\pgfqpoint{1.640528in}{1.897378in}}%
\pgfpathcurveto{\pgfqpoint{1.646352in}{1.891554in}}{\pgfqpoint{1.654252in}{1.888282in}}{\pgfqpoint{1.662488in}{1.888282in}}%
\pgfpathclose%
\pgfusepath{stroke,fill}%
\end{pgfscope}%
\begin{pgfscope}%
\pgfpathrectangle{\pgfqpoint{0.100000in}{0.212622in}}{\pgfqpoint{3.696000in}{3.696000in}}%
\pgfusepath{clip}%
\pgfsetbuttcap%
\pgfsetroundjoin%
\definecolor{currentfill}{rgb}{0.121569,0.466667,0.705882}%
\pgfsetfillcolor{currentfill}%
\pgfsetfillopacity{0.385219}%
\pgfsetlinewidth{1.003750pt}%
\definecolor{currentstroke}{rgb}{0.121569,0.466667,0.705882}%
\pgfsetstrokecolor{currentstroke}%
\pgfsetstrokeopacity{0.385219}%
\pgfsetdash{}{0pt}%
\pgfpathmoveto{\pgfqpoint{1.658288in}{1.887369in}}%
\pgfpathcurveto{\pgfqpoint{1.666524in}{1.887369in}}{\pgfqpoint{1.674424in}{1.890641in}}{\pgfqpoint{1.680248in}{1.896465in}}%
\pgfpathcurveto{\pgfqpoint{1.686072in}{1.902289in}}{\pgfqpoint{1.689344in}{1.910189in}}{\pgfqpoint{1.689344in}{1.918426in}}%
\pgfpathcurveto{\pgfqpoint{1.689344in}{1.926662in}}{\pgfqpoint{1.686072in}{1.934562in}}{\pgfqpoint{1.680248in}{1.940386in}}%
\pgfpathcurveto{\pgfqpoint{1.674424in}{1.946210in}}{\pgfqpoint{1.666524in}{1.949482in}}{\pgfqpoint{1.658288in}{1.949482in}}%
\pgfpathcurveto{\pgfqpoint{1.650052in}{1.949482in}}{\pgfqpoint{1.642151in}{1.946210in}}{\pgfqpoint{1.636328in}{1.940386in}}%
\pgfpathcurveto{\pgfqpoint{1.630504in}{1.934562in}}{\pgfqpoint{1.627231in}{1.926662in}}{\pgfqpoint{1.627231in}{1.918426in}}%
\pgfpathcurveto{\pgfqpoint{1.627231in}{1.910189in}}{\pgfqpoint{1.630504in}{1.902289in}}{\pgfqpoint{1.636328in}{1.896465in}}%
\pgfpathcurveto{\pgfqpoint{1.642151in}{1.890641in}}{\pgfqpoint{1.650052in}{1.887369in}}{\pgfqpoint{1.658288in}{1.887369in}}%
\pgfpathclose%
\pgfusepath{stroke,fill}%
\end{pgfscope}%
\begin{pgfscope}%
\pgfpathrectangle{\pgfqpoint{0.100000in}{0.212622in}}{\pgfqpoint{3.696000in}{3.696000in}}%
\pgfusepath{clip}%
\pgfsetbuttcap%
\pgfsetroundjoin%
\definecolor{currentfill}{rgb}{0.121569,0.466667,0.705882}%
\pgfsetfillcolor{currentfill}%
\pgfsetfillopacity{0.386312}%
\pgfsetlinewidth{1.003750pt}%
\definecolor{currentstroke}{rgb}{0.121569,0.466667,0.705882}%
\pgfsetstrokecolor{currentstroke}%
\pgfsetstrokeopacity{0.386312}%
\pgfsetdash{}{0pt}%
\pgfpathmoveto{\pgfqpoint{1.654959in}{1.886295in}}%
\pgfpathcurveto{\pgfqpoint{1.663196in}{1.886295in}}{\pgfqpoint{1.671096in}{1.889568in}}{\pgfqpoint{1.676920in}{1.895392in}}%
\pgfpathcurveto{\pgfqpoint{1.682743in}{1.901216in}}{\pgfqpoint{1.686016in}{1.909116in}}{\pgfqpoint{1.686016in}{1.917352in}}%
\pgfpathcurveto{\pgfqpoint{1.686016in}{1.925588in}}{\pgfqpoint{1.682743in}{1.933488in}}{\pgfqpoint{1.676920in}{1.939312in}}%
\pgfpathcurveto{\pgfqpoint{1.671096in}{1.945136in}}{\pgfqpoint{1.663196in}{1.948408in}}{\pgfqpoint{1.654959in}{1.948408in}}%
\pgfpathcurveto{\pgfqpoint{1.646723in}{1.948408in}}{\pgfqpoint{1.638823in}{1.945136in}}{\pgfqpoint{1.632999in}{1.939312in}}%
\pgfpathcurveto{\pgfqpoint{1.627175in}{1.933488in}}{\pgfqpoint{1.623903in}{1.925588in}}{\pgfqpoint{1.623903in}{1.917352in}}%
\pgfpathcurveto{\pgfqpoint{1.623903in}{1.909116in}}{\pgfqpoint{1.627175in}{1.901216in}}{\pgfqpoint{1.632999in}{1.895392in}}%
\pgfpathcurveto{\pgfqpoint{1.638823in}{1.889568in}}{\pgfqpoint{1.646723in}{1.886295in}}{\pgfqpoint{1.654959in}{1.886295in}}%
\pgfpathclose%
\pgfusepath{stroke,fill}%
\end{pgfscope}%
\begin{pgfscope}%
\pgfpathrectangle{\pgfqpoint{0.100000in}{0.212622in}}{\pgfqpoint{3.696000in}{3.696000in}}%
\pgfusepath{clip}%
\pgfsetbuttcap%
\pgfsetroundjoin%
\definecolor{currentfill}{rgb}{0.121569,0.466667,0.705882}%
\pgfsetfillcolor{currentfill}%
\pgfsetfillopacity{0.387966}%
\pgfsetlinewidth{1.003750pt}%
\definecolor{currentstroke}{rgb}{0.121569,0.466667,0.705882}%
\pgfsetstrokecolor{currentstroke}%
\pgfsetstrokeopacity{0.387966}%
\pgfsetdash{}{0pt}%
\pgfpathmoveto{\pgfqpoint{1.649248in}{1.881574in}}%
\pgfpathcurveto{\pgfqpoint{1.657485in}{1.881574in}}{\pgfqpoint{1.665385in}{1.884847in}}{\pgfqpoint{1.671209in}{1.890671in}}%
\pgfpathcurveto{\pgfqpoint{1.677033in}{1.896495in}}{\pgfqpoint{1.680305in}{1.904395in}}{\pgfqpoint{1.680305in}{1.912631in}}%
\pgfpathcurveto{\pgfqpoint{1.680305in}{1.920867in}}{\pgfqpoint{1.677033in}{1.928767in}}{\pgfqpoint{1.671209in}{1.934591in}}%
\pgfpathcurveto{\pgfqpoint{1.665385in}{1.940415in}}{\pgfqpoint{1.657485in}{1.943687in}}{\pgfqpoint{1.649248in}{1.943687in}}%
\pgfpathcurveto{\pgfqpoint{1.641012in}{1.943687in}}{\pgfqpoint{1.633112in}{1.940415in}}{\pgfqpoint{1.627288in}{1.934591in}}%
\pgfpathcurveto{\pgfqpoint{1.621464in}{1.928767in}}{\pgfqpoint{1.618192in}{1.920867in}}{\pgfqpoint{1.618192in}{1.912631in}}%
\pgfpathcurveto{\pgfqpoint{1.618192in}{1.904395in}}{\pgfqpoint{1.621464in}{1.896495in}}{\pgfqpoint{1.627288in}{1.890671in}}%
\pgfpathcurveto{\pgfqpoint{1.633112in}{1.884847in}}{\pgfqpoint{1.641012in}{1.881574in}}{\pgfqpoint{1.649248in}{1.881574in}}%
\pgfpathclose%
\pgfusepath{stroke,fill}%
\end{pgfscope}%
\begin{pgfscope}%
\pgfpathrectangle{\pgfqpoint{0.100000in}{0.212622in}}{\pgfqpoint{3.696000in}{3.696000in}}%
\pgfusepath{clip}%
\pgfsetbuttcap%
\pgfsetroundjoin%
\definecolor{currentfill}{rgb}{0.121569,0.466667,0.705882}%
\pgfsetfillcolor{currentfill}%
\pgfsetfillopacity{0.389677}%
\pgfsetlinewidth{1.003750pt}%
\definecolor{currentstroke}{rgb}{0.121569,0.466667,0.705882}%
\pgfsetstrokecolor{currentstroke}%
\pgfsetstrokeopacity{0.389677}%
\pgfsetdash{}{0pt}%
\pgfpathmoveto{\pgfqpoint{1.644460in}{1.881315in}}%
\pgfpathcurveto{\pgfqpoint{1.652696in}{1.881315in}}{\pgfqpoint{1.660596in}{1.884588in}}{\pgfqpoint{1.666420in}{1.890412in}}%
\pgfpathcurveto{\pgfqpoint{1.672244in}{1.896235in}}{\pgfqpoint{1.675516in}{1.904136in}}{\pgfqpoint{1.675516in}{1.912372in}}%
\pgfpathcurveto{\pgfqpoint{1.675516in}{1.920608in}}{\pgfqpoint{1.672244in}{1.928508in}}{\pgfqpoint{1.666420in}{1.934332in}}%
\pgfpathcurveto{\pgfqpoint{1.660596in}{1.940156in}}{\pgfqpoint{1.652696in}{1.943428in}}{\pgfqpoint{1.644460in}{1.943428in}}%
\pgfpathcurveto{\pgfqpoint{1.636224in}{1.943428in}}{\pgfqpoint{1.628324in}{1.940156in}}{\pgfqpoint{1.622500in}{1.934332in}}%
\pgfpathcurveto{\pgfqpoint{1.616676in}{1.928508in}}{\pgfqpoint{1.613403in}{1.920608in}}{\pgfqpoint{1.613403in}{1.912372in}}%
\pgfpathcurveto{\pgfqpoint{1.613403in}{1.904136in}}{\pgfqpoint{1.616676in}{1.896235in}}{\pgfqpoint{1.622500in}{1.890412in}}%
\pgfpathcurveto{\pgfqpoint{1.628324in}{1.884588in}}{\pgfqpoint{1.636224in}{1.881315in}}{\pgfqpoint{1.644460in}{1.881315in}}%
\pgfpathclose%
\pgfusepath{stroke,fill}%
\end{pgfscope}%
\begin{pgfscope}%
\pgfpathrectangle{\pgfqpoint{0.100000in}{0.212622in}}{\pgfqpoint{3.696000in}{3.696000in}}%
\pgfusepath{clip}%
\pgfsetbuttcap%
\pgfsetroundjoin%
\definecolor{currentfill}{rgb}{0.121569,0.466667,0.705882}%
\pgfsetfillcolor{currentfill}%
\pgfsetfillopacity{0.390359}%
\pgfsetlinewidth{1.003750pt}%
\definecolor{currentstroke}{rgb}{0.121569,0.466667,0.705882}%
\pgfsetstrokecolor{currentstroke}%
\pgfsetstrokeopacity{0.390359}%
\pgfsetdash{}{0pt}%
\pgfpathmoveto{\pgfqpoint{1.997699in}{2.005528in}}%
\pgfpathcurveto{\pgfqpoint{2.005936in}{2.005528in}}{\pgfqpoint{2.013836in}{2.008800in}}{\pgfqpoint{2.019660in}{2.014624in}}%
\pgfpathcurveto{\pgfqpoint{2.025483in}{2.020448in}}{\pgfqpoint{2.028756in}{2.028348in}}{\pgfqpoint{2.028756in}{2.036584in}}%
\pgfpathcurveto{\pgfqpoint{2.028756in}{2.044821in}}{\pgfqpoint{2.025483in}{2.052721in}}{\pgfqpoint{2.019660in}{2.058545in}}%
\pgfpathcurveto{\pgfqpoint{2.013836in}{2.064369in}}{\pgfqpoint{2.005936in}{2.067641in}}{\pgfqpoint{1.997699in}{2.067641in}}%
\pgfpathcurveto{\pgfqpoint{1.989463in}{2.067641in}}{\pgfqpoint{1.981563in}{2.064369in}}{\pgfqpoint{1.975739in}{2.058545in}}%
\pgfpathcurveto{\pgfqpoint{1.969915in}{2.052721in}}{\pgfqpoint{1.966643in}{2.044821in}}{\pgfqpoint{1.966643in}{2.036584in}}%
\pgfpathcurveto{\pgfqpoint{1.966643in}{2.028348in}}{\pgfqpoint{1.969915in}{2.020448in}}{\pgfqpoint{1.975739in}{2.014624in}}%
\pgfpathcurveto{\pgfqpoint{1.981563in}{2.008800in}}{\pgfqpoint{1.989463in}{2.005528in}}{\pgfqpoint{1.997699in}{2.005528in}}%
\pgfpathclose%
\pgfusepath{stroke,fill}%
\end{pgfscope}%
\begin{pgfscope}%
\pgfpathrectangle{\pgfqpoint{0.100000in}{0.212622in}}{\pgfqpoint{3.696000in}{3.696000in}}%
\pgfusepath{clip}%
\pgfsetbuttcap%
\pgfsetroundjoin%
\definecolor{currentfill}{rgb}{0.121569,0.466667,0.705882}%
\pgfsetfillcolor{currentfill}%
\pgfsetfillopacity{0.390817}%
\pgfsetlinewidth{1.003750pt}%
\definecolor{currentstroke}{rgb}{0.121569,0.466667,0.705882}%
\pgfsetstrokecolor{currentstroke}%
\pgfsetstrokeopacity{0.390817}%
\pgfsetdash{}{0pt}%
\pgfpathmoveto{\pgfqpoint{1.640760in}{1.878705in}}%
\pgfpathcurveto{\pgfqpoint{1.648996in}{1.878705in}}{\pgfqpoint{1.656896in}{1.881977in}}{\pgfqpoint{1.662720in}{1.887801in}}%
\pgfpathcurveto{\pgfqpoint{1.668544in}{1.893625in}}{\pgfqpoint{1.671816in}{1.901525in}}{\pgfqpoint{1.671816in}{1.909762in}}%
\pgfpathcurveto{\pgfqpoint{1.671816in}{1.917998in}}{\pgfqpoint{1.668544in}{1.925898in}}{\pgfqpoint{1.662720in}{1.931722in}}%
\pgfpathcurveto{\pgfqpoint{1.656896in}{1.937546in}}{\pgfqpoint{1.648996in}{1.940818in}}{\pgfqpoint{1.640760in}{1.940818in}}%
\pgfpathcurveto{\pgfqpoint{1.632524in}{1.940818in}}{\pgfqpoint{1.624624in}{1.937546in}}{\pgfqpoint{1.618800in}{1.931722in}}%
\pgfpathcurveto{\pgfqpoint{1.612976in}{1.925898in}}{\pgfqpoint{1.609703in}{1.917998in}}{\pgfqpoint{1.609703in}{1.909762in}}%
\pgfpathcurveto{\pgfqpoint{1.609703in}{1.901525in}}{\pgfqpoint{1.612976in}{1.893625in}}{\pgfqpoint{1.618800in}{1.887801in}}%
\pgfpathcurveto{\pgfqpoint{1.624624in}{1.881977in}}{\pgfqpoint{1.632524in}{1.878705in}}{\pgfqpoint{1.640760in}{1.878705in}}%
\pgfpathclose%
\pgfusepath{stroke,fill}%
\end{pgfscope}%
\begin{pgfscope}%
\pgfpathrectangle{\pgfqpoint{0.100000in}{0.212622in}}{\pgfqpoint{3.696000in}{3.696000in}}%
\pgfusepath{clip}%
\pgfsetbuttcap%
\pgfsetroundjoin%
\definecolor{currentfill}{rgb}{0.121569,0.466667,0.705882}%
\pgfsetfillcolor{currentfill}%
\pgfsetfillopacity{0.392759}%
\pgfsetlinewidth{1.003750pt}%
\definecolor{currentstroke}{rgb}{0.121569,0.466667,0.705882}%
\pgfsetstrokecolor{currentstroke}%
\pgfsetstrokeopacity{0.392759}%
\pgfsetdash{}{0pt}%
\pgfpathmoveto{\pgfqpoint{1.634612in}{1.872289in}}%
\pgfpathcurveto{\pgfqpoint{1.642849in}{1.872289in}}{\pgfqpoint{1.650749in}{1.875561in}}{\pgfqpoint{1.656573in}{1.881385in}}%
\pgfpathcurveto{\pgfqpoint{1.662396in}{1.887209in}}{\pgfqpoint{1.665669in}{1.895109in}}{\pgfqpoint{1.665669in}{1.903345in}}%
\pgfpathcurveto{\pgfqpoint{1.665669in}{1.911582in}}{\pgfqpoint{1.662396in}{1.919482in}}{\pgfqpoint{1.656573in}{1.925306in}}%
\pgfpathcurveto{\pgfqpoint{1.650749in}{1.931130in}}{\pgfqpoint{1.642849in}{1.934402in}}{\pgfqpoint{1.634612in}{1.934402in}}%
\pgfpathcurveto{\pgfqpoint{1.626376in}{1.934402in}}{\pgfqpoint{1.618476in}{1.931130in}}{\pgfqpoint{1.612652in}{1.925306in}}%
\pgfpathcurveto{\pgfqpoint{1.606828in}{1.919482in}}{\pgfqpoint{1.603556in}{1.911582in}}{\pgfqpoint{1.603556in}{1.903345in}}%
\pgfpathcurveto{\pgfqpoint{1.603556in}{1.895109in}}{\pgfqpoint{1.606828in}{1.887209in}}{\pgfqpoint{1.612652in}{1.881385in}}%
\pgfpathcurveto{\pgfqpoint{1.618476in}{1.875561in}}{\pgfqpoint{1.626376in}{1.872289in}}{\pgfqpoint{1.634612in}{1.872289in}}%
\pgfpathclose%
\pgfusepath{stroke,fill}%
\end{pgfscope}%
\begin{pgfscope}%
\pgfpathrectangle{\pgfqpoint{0.100000in}{0.212622in}}{\pgfqpoint{3.696000in}{3.696000in}}%
\pgfusepath{clip}%
\pgfsetbuttcap%
\pgfsetroundjoin%
\definecolor{currentfill}{rgb}{0.121569,0.466667,0.705882}%
\pgfsetfillcolor{currentfill}%
\pgfsetfillopacity{0.393843}%
\pgfsetlinewidth{1.003750pt}%
\definecolor{currentstroke}{rgb}{0.121569,0.466667,0.705882}%
\pgfsetstrokecolor{currentstroke}%
\pgfsetstrokeopacity{0.393843}%
\pgfsetdash{}{0pt}%
\pgfpathmoveto{\pgfqpoint{1.631046in}{1.870463in}}%
\pgfpathcurveto{\pgfqpoint{1.639283in}{1.870463in}}{\pgfqpoint{1.647183in}{1.873736in}}{\pgfqpoint{1.653007in}{1.879560in}}%
\pgfpathcurveto{\pgfqpoint{1.658830in}{1.885384in}}{\pgfqpoint{1.662103in}{1.893284in}}{\pgfqpoint{1.662103in}{1.901520in}}%
\pgfpathcurveto{\pgfqpoint{1.662103in}{1.909756in}}{\pgfqpoint{1.658830in}{1.917656in}}{\pgfqpoint{1.653007in}{1.923480in}}%
\pgfpathcurveto{\pgfqpoint{1.647183in}{1.929304in}}{\pgfqpoint{1.639283in}{1.932576in}}{\pgfqpoint{1.631046in}{1.932576in}}%
\pgfpathcurveto{\pgfqpoint{1.622810in}{1.932576in}}{\pgfqpoint{1.614910in}{1.929304in}}{\pgfqpoint{1.609086in}{1.923480in}}%
\pgfpathcurveto{\pgfqpoint{1.603262in}{1.917656in}}{\pgfqpoint{1.599990in}{1.909756in}}{\pgfqpoint{1.599990in}{1.901520in}}%
\pgfpathcurveto{\pgfqpoint{1.599990in}{1.893284in}}{\pgfqpoint{1.603262in}{1.885384in}}{\pgfqpoint{1.609086in}{1.879560in}}%
\pgfpathcurveto{\pgfqpoint{1.614910in}{1.873736in}}{\pgfqpoint{1.622810in}{1.870463in}}{\pgfqpoint{1.631046in}{1.870463in}}%
\pgfpathclose%
\pgfusepath{stroke,fill}%
\end{pgfscope}%
\begin{pgfscope}%
\pgfpathrectangle{\pgfqpoint{0.100000in}{0.212622in}}{\pgfqpoint{3.696000in}{3.696000in}}%
\pgfusepath{clip}%
\pgfsetbuttcap%
\pgfsetroundjoin%
\definecolor{currentfill}{rgb}{0.121569,0.466667,0.705882}%
\pgfsetfillcolor{currentfill}%
\pgfsetfillopacity{0.396042}%
\pgfsetlinewidth{1.003750pt}%
\definecolor{currentstroke}{rgb}{0.121569,0.466667,0.705882}%
\pgfsetstrokecolor{currentstroke}%
\pgfsetstrokeopacity{0.396042}%
\pgfsetdash{}{0pt}%
\pgfpathmoveto{\pgfqpoint{1.624244in}{1.869177in}}%
\pgfpathcurveto{\pgfqpoint{1.632481in}{1.869177in}}{\pgfqpoint{1.640381in}{1.872449in}}{\pgfqpoint{1.646205in}{1.878273in}}%
\pgfpathcurveto{\pgfqpoint{1.652029in}{1.884097in}}{\pgfqpoint{1.655301in}{1.891997in}}{\pgfqpoint{1.655301in}{1.900233in}}%
\pgfpathcurveto{\pgfqpoint{1.655301in}{1.908469in}}{\pgfqpoint{1.652029in}{1.916369in}}{\pgfqpoint{1.646205in}{1.922193in}}%
\pgfpathcurveto{\pgfqpoint{1.640381in}{1.928017in}}{\pgfqpoint{1.632481in}{1.931290in}}{\pgfqpoint{1.624244in}{1.931290in}}%
\pgfpathcurveto{\pgfqpoint{1.616008in}{1.931290in}}{\pgfqpoint{1.608108in}{1.928017in}}{\pgfqpoint{1.602284in}{1.922193in}}%
\pgfpathcurveto{\pgfqpoint{1.596460in}{1.916369in}}{\pgfqpoint{1.593188in}{1.908469in}}{\pgfqpoint{1.593188in}{1.900233in}}%
\pgfpathcurveto{\pgfqpoint{1.593188in}{1.891997in}}{\pgfqpoint{1.596460in}{1.884097in}}{\pgfqpoint{1.602284in}{1.878273in}}%
\pgfpathcurveto{\pgfqpoint{1.608108in}{1.872449in}}{\pgfqpoint{1.616008in}{1.869177in}}{\pgfqpoint{1.624244in}{1.869177in}}%
\pgfpathclose%
\pgfusepath{stroke,fill}%
\end{pgfscope}%
\begin{pgfscope}%
\pgfpathrectangle{\pgfqpoint{0.100000in}{0.212622in}}{\pgfqpoint{3.696000in}{3.696000in}}%
\pgfusepath{clip}%
\pgfsetbuttcap%
\pgfsetroundjoin%
\definecolor{currentfill}{rgb}{0.121569,0.466667,0.705882}%
\pgfsetfillcolor{currentfill}%
\pgfsetfillopacity{0.397892}%
\pgfsetlinewidth{1.003750pt}%
\definecolor{currentstroke}{rgb}{0.121569,0.466667,0.705882}%
\pgfsetstrokecolor{currentstroke}%
\pgfsetstrokeopacity{0.397892}%
\pgfsetdash{}{0pt}%
\pgfpathmoveto{\pgfqpoint{1.999269in}{1.995350in}}%
\pgfpathcurveto{\pgfqpoint{2.007506in}{1.995350in}}{\pgfqpoint{2.015406in}{1.998622in}}{\pgfqpoint{2.021230in}{2.004446in}}%
\pgfpathcurveto{\pgfqpoint{2.027054in}{2.010270in}}{\pgfqpoint{2.030326in}{2.018170in}}{\pgfqpoint{2.030326in}{2.026407in}}%
\pgfpathcurveto{\pgfqpoint{2.030326in}{2.034643in}}{\pgfqpoint{2.027054in}{2.042543in}}{\pgfqpoint{2.021230in}{2.048367in}}%
\pgfpathcurveto{\pgfqpoint{2.015406in}{2.054191in}}{\pgfqpoint{2.007506in}{2.057463in}}{\pgfqpoint{1.999269in}{2.057463in}}%
\pgfpathcurveto{\pgfqpoint{1.991033in}{2.057463in}}{\pgfqpoint{1.983133in}{2.054191in}}{\pgfqpoint{1.977309in}{2.048367in}}%
\pgfpathcurveto{\pgfqpoint{1.971485in}{2.042543in}}{\pgfqpoint{1.968213in}{2.034643in}}{\pgfqpoint{1.968213in}{2.026407in}}%
\pgfpathcurveto{\pgfqpoint{1.968213in}{2.018170in}}{\pgfqpoint{1.971485in}{2.010270in}}{\pgfqpoint{1.977309in}{2.004446in}}%
\pgfpathcurveto{\pgfqpoint{1.983133in}{1.998622in}}{\pgfqpoint{1.991033in}{1.995350in}}{\pgfqpoint{1.999269in}{1.995350in}}%
\pgfpathclose%
\pgfusepath{stroke,fill}%
\end{pgfscope}%
\begin{pgfscope}%
\pgfpathrectangle{\pgfqpoint{0.100000in}{0.212622in}}{\pgfqpoint{3.696000in}{3.696000in}}%
\pgfusepath{clip}%
\pgfsetbuttcap%
\pgfsetroundjoin%
\definecolor{currentfill}{rgb}{0.121569,0.466667,0.705882}%
\pgfsetfillcolor{currentfill}%
\pgfsetfillopacity{0.399467}%
\pgfsetlinewidth{1.003750pt}%
\definecolor{currentstroke}{rgb}{0.121569,0.466667,0.705882}%
\pgfsetstrokecolor{currentstroke}%
\pgfsetstrokeopacity{0.399467}%
\pgfsetdash{}{0pt}%
\pgfpathmoveto{\pgfqpoint{1.613500in}{1.860457in}}%
\pgfpathcurveto{\pgfqpoint{1.621736in}{1.860457in}}{\pgfqpoint{1.629636in}{1.863729in}}{\pgfqpoint{1.635460in}{1.869553in}}%
\pgfpathcurveto{\pgfqpoint{1.641284in}{1.875377in}}{\pgfqpoint{1.644556in}{1.883277in}}{\pgfqpoint{1.644556in}{1.891513in}}%
\pgfpathcurveto{\pgfqpoint{1.644556in}{1.899749in}}{\pgfqpoint{1.641284in}{1.907649in}}{\pgfqpoint{1.635460in}{1.913473in}}%
\pgfpathcurveto{\pgfqpoint{1.629636in}{1.919297in}}{\pgfqpoint{1.621736in}{1.922570in}}{\pgfqpoint{1.613500in}{1.922570in}}%
\pgfpathcurveto{\pgfqpoint{1.605263in}{1.922570in}}{\pgfqpoint{1.597363in}{1.919297in}}{\pgfqpoint{1.591539in}{1.913473in}}%
\pgfpathcurveto{\pgfqpoint{1.585715in}{1.907649in}}{\pgfqpoint{1.582443in}{1.899749in}}{\pgfqpoint{1.582443in}{1.891513in}}%
\pgfpathcurveto{\pgfqpoint{1.582443in}{1.883277in}}{\pgfqpoint{1.585715in}{1.875377in}}{\pgfqpoint{1.591539in}{1.869553in}}%
\pgfpathcurveto{\pgfqpoint{1.597363in}{1.863729in}}{\pgfqpoint{1.605263in}{1.860457in}}{\pgfqpoint{1.613500in}{1.860457in}}%
\pgfpathclose%
\pgfusepath{stroke,fill}%
\end{pgfscope}%
\begin{pgfscope}%
\pgfpathrectangle{\pgfqpoint{0.100000in}{0.212622in}}{\pgfqpoint{3.696000in}{3.696000in}}%
\pgfusepath{clip}%
\pgfsetbuttcap%
\pgfsetroundjoin%
\definecolor{currentfill}{rgb}{0.121569,0.466667,0.705882}%
\pgfsetfillcolor{currentfill}%
\pgfsetfillopacity{0.402140}%
\pgfsetlinewidth{1.003750pt}%
\definecolor{currentstroke}{rgb}{0.121569,0.466667,0.705882}%
\pgfsetstrokecolor{currentstroke}%
\pgfsetstrokeopacity{0.402140}%
\pgfsetdash{}{0pt}%
\pgfpathmoveto{\pgfqpoint{1.603648in}{1.856326in}}%
\pgfpathcurveto{\pgfqpoint{1.611884in}{1.856326in}}{\pgfqpoint{1.619784in}{1.859599in}}{\pgfqpoint{1.625608in}{1.865422in}}%
\pgfpathcurveto{\pgfqpoint{1.631432in}{1.871246in}}{\pgfqpoint{1.634705in}{1.879146in}}{\pgfqpoint{1.634705in}{1.887383in}}%
\pgfpathcurveto{\pgfqpoint{1.634705in}{1.895619in}}{\pgfqpoint{1.631432in}{1.903519in}}{\pgfqpoint{1.625608in}{1.909343in}}%
\pgfpathcurveto{\pgfqpoint{1.619784in}{1.915167in}}{\pgfqpoint{1.611884in}{1.918439in}}{\pgfqpoint{1.603648in}{1.918439in}}%
\pgfpathcurveto{\pgfqpoint{1.595412in}{1.918439in}}{\pgfqpoint{1.587512in}{1.915167in}}{\pgfqpoint{1.581688in}{1.909343in}}%
\pgfpathcurveto{\pgfqpoint{1.575864in}{1.903519in}}{\pgfqpoint{1.572592in}{1.895619in}}{\pgfqpoint{1.572592in}{1.887383in}}%
\pgfpathcurveto{\pgfqpoint{1.572592in}{1.879146in}}{\pgfqpoint{1.575864in}{1.871246in}}{\pgfqpoint{1.581688in}{1.865422in}}%
\pgfpathcurveto{\pgfqpoint{1.587512in}{1.859599in}}{\pgfqpoint{1.595412in}{1.856326in}}{\pgfqpoint{1.603648in}{1.856326in}}%
\pgfpathclose%
\pgfusepath{stroke,fill}%
\end{pgfscope}%
\begin{pgfscope}%
\pgfpathrectangle{\pgfqpoint{0.100000in}{0.212622in}}{\pgfqpoint{3.696000in}{3.696000in}}%
\pgfusepath{clip}%
\pgfsetbuttcap%
\pgfsetroundjoin%
\definecolor{currentfill}{rgb}{0.121569,0.466667,0.705882}%
\pgfsetfillcolor{currentfill}%
\pgfsetfillopacity{0.402312}%
\pgfsetlinewidth{1.003750pt}%
\definecolor{currentstroke}{rgb}{0.121569,0.466667,0.705882}%
\pgfsetstrokecolor{currentstroke}%
\pgfsetstrokeopacity{0.402312}%
\pgfsetdash{}{0pt}%
\pgfpathmoveto{\pgfqpoint{2.001390in}{1.992012in}}%
\pgfpathcurveto{\pgfqpoint{2.009626in}{1.992012in}}{\pgfqpoint{2.017526in}{1.995284in}}{\pgfqpoint{2.023350in}{2.001108in}}%
\pgfpathcurveto{\pgfqpoint{2.029174in}{2.006932in}}{\pgfqpoint{2.032446in}{2.014832in}}{\pgfqpoint{2.032446in}{2.023068in}}%
\pgfpathcurveto{\pgfqpoint{2.032446in}{2.031305in}}{\pgfqpoint{2.029174in}{2.039205in}}{\pgfqpoint{2.023350in}{2.045029in}}%
\pgfpathcurveto{\pgfqpoint{2.017526in}{2.050853in}}{\pgfqpoint{2.009626in}{2.054125in}}{\pgfqpoint{2.001390in}{2.054125in}}%
\pgfpathcurveto{\pgfqpoint{1.993154in}{2.054125in}}{\pgfqpoint{1.985254in}{2.050853in}}{\pgfqpoint{1.979430in}{2.045029in}}%
\pgfpathcurveto{\pgfqpoint{1.973606in}{2.039205in}}{\pgfqpoint{1.970333in}{2.031305in}}{\pgfqpoint{1.970333in}{2.023068in}}%
\pgfpathcurveto{\pgfqpoint{1.970333in}{2.014832in}}{\pgfqpoint{1.973606in}{2.006932in}}{\pgfqpoint{1.979430in}{2.001108in}}%
\pgfpathcurveto{\pgfqpoint{1.985254in}{1.995284in}}{\pgfqpoint{1.993154in}{1.992012in}}{\pgfqpoint{2.001390in}{1.992012in}}%
\pgfpathclose%
\pgfusepath{stroke,fill}%
\end{pgfscope}%
\begin{pgfscope}%
\pgfpathrectangle{\pgfqpoint{0.100000in}{0.212622in}}{\pgfqpoint{3.696000in}{3.696000in}}%
\pgfusepath{clip}%
\pgfsetbuttcap%
\pgfsetroundjoin%
\definecolor{currentfill}{rgb}{0.121569,0.466667,0.705882}%
\pgfsetfillcolor{currentfill}%
\pgfsetfillopacity{0.404126}%
\pgfsetlinewidth{1.003750pt}%
\definecolor{currentstroke}{rgb}{0.121569,0.466667,0.705882}%
\pgfsetstrokecolor{currentstroke}%
\pgfsetstrokeopacity{0.404126}%
\pgfsetdash{}{0pt}%
\pgfpathmoveto{\pgfqpoint{1.597756in}{1.854687in}}%
\pgfpathcurveto{\pgfqpoint{1.605992in}{1.854687in}}{\pgfqpoint{1.613892in}{1.857959in}}{\pgfqpoint{1.619716in}{1.863783in}}%
\pgfpathcurveto{\pgfqpoint{1.625540in}{1.869607in}}{\pgfqpoint{1.628812in}{1.877507in}}{\pgfqpoint{1.628812in}{1.885743in}}%
\pgfpathcurveto{\pgfqpoint{1.628812in}{1.893979in}}{\pgfqpoint{1.625540in}{1.901879in}}{\pgfqpoint{1.619716in}{1.907703in}}%
\pgfpathcurveto{\pgfqpoint{1.613892in}{1.913527in}}{\pgfqpoint{1.605992in}{1.916800in}}{\pgfqpoint{1.597756in}{1.916800in}}%
\pgfpathcurveto{\pgfqpoint{1.589519in}{1.916800in}}{\pgfqpoint{1.581619in}{1.913527in}}{\pgfqpoint{1.575795in}{1.907703in}}%
\pgfpathcurveto{\pgfqpoint{1.569971in}{1.901879in}}{\pgfqpoint{1.566699in}{1.893979in}}{\pgfqpoint{1.566699in}{1.885743in}}%
\pgfpathcurveto{\pgfqpoint{1.566699in}{1.877507in}}{\pgfqpoint{1.569971in}{1.869607in}}{\pgfqpoint{1.575795in}{1.863783in}}%
\pgfpathcurveto{\pgfqpoint{1.581619in}{1.857959in}}{\pgfqpoint{1.589519in}{1.854687in}}{\pgfqpoint{1.597756in}{1.854687in}}%
\pgfpathclose%
\pgfusepath{stroke,fill}%
\end{pgfscope}%
\begin{pgfscope}%
\pgfpathrectangle{\pgfqpoint{0.100000in}{0.212622in}}{\pgfqpoint{3.696000in}{3.696000in}}%
\pgfusepath{clip}%
\pgfsetbuttcap%
\pgfsetroundjoin%
\definecolor{currentfill}{rgb}{0.121569,0.466667,0.705882}%
\pgfsetfillcolor{currentfill}%
\pgfsetfillopacity{0.407200}%
\pgfsetlinewidth{1.003750pt}%
\definecolor{currentstroke}{rgb}{0.121569,0.466667,0.705882}%
\pgfsetstrokecolor{currentstroke}%
\pgfsetstrokeopacity{0.407200}%
\pgfsetdash{}{0pt}%
\pgfpathmoveto{\pgfqpoint{1.587941in}{1.846749in}}%
\pgfpathcurveto{\pgfqpoint{1.596178in}{1.846749in}}{\pgfqpoint{1.604078in}{1.850021in}}{\pgfqpoint{1.609902in}{1.855845in}}%
\pgfpathcurveto{\pgfqpoint{1.615726in}{1.861669in}}{\pgfqpoint{1.618998in}{1.869569in}}{\pgfqpoint{1.618998in}{1.877805in}}%
\pgfpathcurveto{\pgfqpoint{1.618998in}{1.886041in}}{\pgfqpoint{1.615726in}{1.893942in}}{\pgfqpoint{1.609902in}{1.899765in}}%
\pgfpathcurveto{\pgfqpoint{1.604078in}{1.905589in}}{\pgfqpoint{1.596178in}{1.908862in}}{\pgfqpoint{1.587941in}{1.908862in}}%
\pgfpathcurveto{\pgfqpoint{1.579705in}{1.908862in}}{\pgfqpoint{1.571805in}{1.905589in}}{\pgfqpoint{1.565981in}{1.899765in}}%
\pgfpathcurveto{\pgfqpoint{1.560157in}{1.893942in}}{\pgfqpoint{1.556885in}{1.886041in}}{\pgfqpoint{1.556885in}{1.877805in}}%
\pgfpathcurveto{\pgfqpoint{1.556885in}{1.869569in}}{\pgfqpoint{1.560157in}{1.861669in}}{\pgfqpoint{1.565981in}{1.855845in}}%
\pgfpathcurveto{\pgfqpoint{1.571805in}{1.850021in}}{\pgfqpoint{1.579705in}{1.846749in}}{\pgfqpoint{1.587941in}{1.846749in}}%
\pgfpathclose%
\pgfusepath{stroke,fill}%
\end{pgfscope}%
\begin{pgfscope}%
\pgfpathrectangle{\pgfqpoint{0.100000in}{0.212622in}}{\pgfqpoint{3.696000in}{3.696000in}}%
\pgfusepath{clip}%
\pgfsetbuttcap%
\pgfsetroundjoin%
\definecolor{currentfill}{rgb}{0.121569,0.466667,0.705882}%
\pgfsetfillcolor{currentfill}%
\pgfsetfillopacity{0.407250}%
\pgfsetlinewidth{1.003750pt}%
\definecolor{currentstroke}{rgb}{0.121569,0.466667,0.705882}%
\pgfsetstrokecolor{currentstroke}%
\pgfsetstrokeopacity{0.407250}%
\pgfsetdash{}{0pt}%
\pgfpathmoveto{\pgfqpoint{2.002784in}{1.988953in}}%
\pgfpathcurveto{\pgfqpoint{2.011020in}{1.988953in}}{\pgfqpoint{2.018920in}{1.992225in}}{\pgfqpoint{2.024744in}{1.998049in}}%
\pgfpathcurveto{\pgfqpoint{2.030568in}{2.003873in}}{\pgfqpoint{2.033841in}{2.011773in}}{\pgfqpoint{2.033841in}{2.020009in}}%
\pgfpathcurveto{\pgfqpoint{2.033841in}{2.028246in}}{\pgfqpoint{2.030568in}{2.036146in}}{\pgfqpoint{2.024744in}{2.041970in}}%
\pgfpathcurveto{\pgfqpoint{2.018920in}{2.047793in}}{\pgfqpoint{2.011020in}{2.051066in}}{\pgfqpoint{2.002784in}{2.051066in}}%
\pgfpathcurveto{\pgfqpoint{1.994548in}{2.051066in}}{\pgfqpoint{1.986648in}{2.047793in}}{\pgfqpoint{1.980824in}{2.041970in}}%
\pgfpathcurveto{\pgfqpoint{1.975000in}{2.036146in}}{\pgfqpoint{1.971728in}{2.028246in}}{\pgfqpoint{1.971728in}{2.020009in}}%
\pgfpathcurveto{\pgfqpoint{1.971728in}{2.011773in}}{\pgfqpoint{1.975000in}{2.003873in}}{\pgfqpoint{1.980824in}{1.998049in}}%
\pgfpathcurveto{\pgfqpoint{1.986648in}{1.992225in}}{\pgfqpoint{1.994548in}{1.988953in}}{\pgfqpoint{2.002784in}{1.988953in}}%
\pgfpathclose%
\pgfusepath{stroke,fill}%
\end{pgfscope}%
\begin{pgfscope}%
\pgfpathrectangle{\pgfqpoint{0.100000in}{0.212622in}}{\pgfqpoint{3.696000in}{3.696000in}}%
\pgfusepath{clip}%
\pgfsetbuttcap%
\pgfsetroundjoin%
\definecolor{currentfill}{rgb}{0.121569,0.466667,0.705882}%
\pgfsetfillcolor{currentfill}%
\pgfsetfillopacity{0.409106}%
\pgfsetlinewidth{1.003750pt}%
\definecolor{currentstroke}{rgb}{0.121569,0.466667,0.705882}%
\pgfsetstrokecolor{currentstroke}%
\pgfsetstrokeopacity{0.409106}%
\pgfsetdash{}{0pt}%
\pgfpathmoveto{\pgfqpoint{1.580092in}{1.840426in}}%
\pgfpathcurveto{\pgfqpoint{1.588329in}{1.840426in}}{\pgfqpoint{1.596229in}{1.843699in}}{\pgfqpoint{1.602053in}{1.849523in}}%
\pgfpathcurveto{\pgfqpoint{1.607877in}{1.855346in}}{\pgfqpoint{1.611149in}{1.863247in}}{\pgfqpoint{1.611149in}{1.871483in}}%
\pgfpathcurveto{\pgfqpoint{1.611149in}{1.879719in}}{\pgfqpoint{1.607877in}{1.887619in}}{\pgfqpoint{1.602053in}{1.893443in}}%
\pgfpathcurveto{\pgfqpoint{1.596229in}{1.899267in}}{\pgfqpoint{1.588329in}{1.902539in}}{\pgfqpoint{1.580092in}{1.902539in}}%
\pgfpathcurveto{\pgfqpoint{1.571856in}{1.902539in}}{\pgfqpoint{1.563956in}{1.899267in}}{\pgfqpoint{1.558132in}{1.893443in}}%
\pgfpathcurveto{\pgfqpoint{1.552308in}{1.887619in}}{\pgfqpoint{1.549036in}{1.879719in}}{\pgfqpoint{1.549036in}{1.871483in}}%
\pgfpathcurveto{\pgfqpoint{1.549036in}{1.863247in}}{\pgfqpoint{1.552308in}{1.855346in}}{\pgfqpoint{1.558132in}{1.849523in}}%
\pgfpathcurveto{\pgfqpoint{1.563956in}{1.843699in}}{\pgfqpoint{1.571856in}{1.840426in}}{\pgfqpoint{1.580092in}{1.840426in}}%
\pgfpathclose%
\pgfusepath{stroke,fill}%
\end{pgfscope}%
\begin{pgfscope}%
\pgfpathrectangle{\pgfqpoint{0.100000in}{0.212622in}}{\pgfqpoint{3.696000in}{3.696000in}}%
\pgfusepath{clip}%
\pgfsetbuttcap%
\pgfsetroundjoin%
\definecolor{currentfill}{rgb}{0.121569,0.466667,0.705882}%
\pgfsetfillcolor{currentfill}%
\pgfsetfillopacity{0.410721}%
\pgfsetlinewidth{1.003750pt}%
\definecolor{currentstroke}{rgb}{0.121569,0.466667,0.705882}%
\pgfsetstrokecolor{currentstroke}%
\pgfsetstrokeopacity{0.410721}%
\pgfsetdash{}{0pt}%
\pgfpathmoveto{\pgfqpoint{1.575002in}{1.839343in}}%
\pgfpathcurveto{\pgfqpoint{1.583239in}{1.839343in}}{\pgfqpoint{1.591139in}{1.842616in}}{\pgfqpoint{1.596963in}{1.848440in}}%
\pgfpathcurveto{\pgfqpoint{1.602787in}{1.854264in}}{\pgfqpoint{1.606059in}{1.862164in}}{\pgfqpoint{1.606059in}{1.870400in}}%
\pgfpathcurveto{\pgfqpoint{1.606059in}{1.878636in}}{\pgfqpoint{1.602787in}{1.886536in}}{\pgfqpoint{1.596963in}{1.892360in}}%
\pgfpathcurveto{\pgfqpoint{1.591139in}{1.898184in}}{\pgfqpoint{1.583239in}{1.901456in}}{\pgfqpoint{1.575002in}{1.901456in}}%
\pgfpathcurveto{\pgfqpoint{1.566766in}{1.901456in}}{\pgfqpoint{1.558866in}{1.898184in}}{\pgfqpoint{1.553042in}{1.892360in}}%
\pgfpathcurveto{\pgfqpoint{1.547218in}{1.886536in}}{\pgfqpoint{1.543946in}{1.878636in}}{\pgfqpoint{1.543946in}{1.870400in}}%
\pgfpathcurveto{\pgfqpoint{1.543946in}{1.862164in}}{\pgfqpoint{1.547218in}{1.854264in}}{\pgfqpoint{1.553042in}{1.848440in}}%
\pgfpathcurveto{\pgfqpoint{1.558866in}{1.842616in}}{\pgfqpoint{1.566766in}{1.839343in}}{\pgfqpoint{1.575002in}{1.839343in}}%
\pgfpathclose%
\pgfusepath{stroke,fill}%
\end{pgfscope}%
\begin{pgfscope}%
\pgfpathrectangle{\pgfqpoint{0.100000in}{0.212622in}}{\pgfqpoint{3.696000in}{3.696000in}}%
\pgfusepath{clip}%
\pgfsetbuttcap%
\pgfsetroundjoin%
\definecolor{currentfill}{rgb}{0.121569,0.466667,0.705882}%
\pgfsetfillcolor{currentfill}%
\pgfsetfillopacity{0.412470}%
\pgfsetlinewidth{1.003750pt}%
\definecolor{currentstroke}{rgb}{0.121569,0.466667,0.705882}%
\pgfsetstrokecolor{currentstroke}%
\pgfsetstrokeopacity{0.412470}%
\pgfsetdash{}{0pt}%
\pgfpathmoveto{\pgfqpoint{2.004685in}{1.982974in}}%
\pgfpathcurveto{\pgfqpoint{2.012922in}{1.982974in}}{\pgfqpoint{2.020822in}{1.986247in}}{\pgfqpoint{2.026646in}{1.992071in}}%
\pgfpathcurveto{\pgfqpoint{2.032470in}{1.997894in}}{\pgfqpoint{2.035742in}{2.005795in}}{\pgfqpoint{2.035742in}{2.014031in}}%
\pgfpathcurveto{\pgfqpoint{2.035742in}{2.022267in}}{\pgfqpoint{2.032470in}{2.030167in}}{\pgfqpoint{2.026646in}{2.035991in}}%
\pgfpathcurveto{\pgfqpoint{2.020822in}{2.041815in}}{\pgfqpoint{2.012922in}{2.045087in}}{\pgfqpoint{2.004685in}{2.045087in}}%
\pgfpathcurveto{\pgfqpoint{1.996449in}{2.045087in}}{\pgfqpoint{1.988549in}{2.041815in}}{\pgfqpoint{1.982725in}{2.035991in}}%
\pgfpathcurveto{\pgfqpoint{1.976901in}{2.030167in}}{\pgfqpoint{1.973629in}{2.022267in}}{\pgfqpoint{1.973629in}{2.014031in}}%
\pgfpathcurveto{\pgfqpoint{1.973629in}{2.005795in}}{\pgfqpoint{1.976901in}{1.997894in}}{\pgfqpoint{1.982725in}{1.992071in}}%
\pgfpathcurveto{\pgfqpoint{1.988549in}{1.986247in}}{\pgfqpoint{1.996449in}{1.982974in}}{\pgfqpoint{2.004685in}{1.982974in}}%
\pgfpathclose%
\pgfusepath{stroke,fill}%
\end{pgfscope}%
\begin{pgfscope}%
\pgfpathrectangle{\pgfqpoint{0.100000in}{0.212622in}}{\pgfqpoint{3.696000in}{3.696000in}}%
\pgfusepath{clip}%
\pgfsetbuttcap%
\pgfsetroundjoin%
\definecolor{currentfill}{rgb}{0.121569,0.466667,0.705882}%
\pgfsetfillcolor{currentfill}%
\pgfsetfillopacity{0.413010}%
\pgfsetlinewidth{1.003750pt}%
\definecolor{currentstroke}{rgb}{0.121569,0.466667,0.705882}%
\pgfsetstrokecolor{currentstroke}%
\pgfsetstrokeopacity{0.413010}%
\pgfsetdash{}{0pt}%
\pgfpathmoveto{\pgfqpoint{1.566692in}{1.831550in}}%
\pgfpathcurveto{\pgfqpoint{1.574928in}{1.831550in}}{\pgfqpoint{1.582828in}{1.834822in}}{\pgfqpoint{1.588652in}{1.840646in}}%
\pgfpathcurveto{\pgfqpoint{1.594476in}{1.846470in}}{\pgfqpoint{1.597748in}{1.854370in}}{\pgfqpoint{1.597748in}{1.862606in}}%
\pgfpathcurveto{\pgfqpoint{1.597748in}{1.870842in}}{\pgfqpoint{1.594476in}{1.878742in}}{\pgfqpoint{1.588652in}{1.884566in}}%
\pgfpathcurveto{\pgfqpoint{1.582828in}{1.890390in}}{\pgfqpoint{1.574928in}{1.893663in}}{\pgfqpoint{1.566692in}{1.893663in}}%
\pgfpathcurveto{\pgfqpoint{1.558456in}{1.893663in}}{\pgfqpoint{1.550556in}{1.890390in}}{\pgfqpoint{1.544732in}{1.884566in}}%
\pgfpathcurveto{\pgfqpoint{1.538908in}{1.878742in}}{\pgfqpoint{1.535635in}{1.870842in}}{\pgfqpoint{1.535635in}{1.862606in}}%
\pgfpathcurveto{\pgfqpoint{1.535635in}{1.854370in}}{\pgfqpoint{1.538908in}{1.846470in}}{\pgfqpoint{1.544732in}{1.840646in}}%
\pgfpathcurveto{\pgfqpoint{1.550556in}{1.834822in}}{\pgfqpoint{1.558456in}{1.831550in}}{\pgfqpoint{1.566692in}{1.831550in}}%
\pgfpathclose%
\pgfusepath{stroke,fill}%
\end{pgfscope}%
\begin{pgfscope}%
\pgfpathrectangle{\pgfqpoint{0.100000in}{0.212622in}}{\pgfqpoint{3.696000in}{3.696000in}}%
\pgfusepath{clip}%
\pgfsetbuttcap%
\pgfsetroundjoin%
\definecolor{currentfill}{rgb}{0.121569,0.466667,0.705882}%
\pgfsetfillcolor{currentfill}%
\pgfsetfillopacity{0.414363}%
\pgfsetlinewidth{1.003750pt}%
\definecolor{currentstroke}{rgb}{0.121569,0.466667,0.705882}%
\pgfsetstrokecolor{currentstroke}%
\pgfsetstrokeopacity{0.414363}%
\pgfsetdash{}{0pt}%
\pgfpathmoveto{\pgfqpoint{1.560410in}{1.826632in}}%
\pgfpathcurveto{\pgfqpoint{1.568646in}{1.826632in}}{\pgfqpoint{1.576546in}{1.829904in}}{\pgfqpoint{1.582370in}{1.835728in}}%
\pgfpathcurveto{\pgfqpoint{1.588194in}{1.841552in}}{\pgfqpoint{1.591466in}{1.849452in}}{\pgfqpoint{1.591466in}{1.857688in}}%
\pgfpathcurveto{\pgfqpoint{1.591466in}{1.865925in}}{\pgfqpoint{1.588194in}{1.873825in}}{\pgfqpoint{1.582370in}{1.879648in}}%
\pgfpathcurveto{\pgfqpoint{1.576546in}{1.885472in}}{\pgfqpoint{1.568646in}{1.888745in}}{\pgfqpoint{1.560410in}{1.888745in}}%
\pgfpathcurveto{\pgfqpoint{1.552174in}{1.888745in}}{\pgfqpoint{1.544273in}{1.885472in}}{\pgfqpoint{1.538450in}{1.879648in}}%
\pgfpathcurveto{\pgfqpoint{1.532626in}{1.873825in}}{\pgfqpoint{1.529353in}{1.865925in}}{\pgfqpoint{1.529353in}{1.857688in}}%
\pgfpathcurveto{\pgfqpoint{1.529353in}{1.849452in}}{\pgfqpoint{1.532626in}{1.841552in}}{\pgfqpoint{1.538450in}{1.835728in}}%
\pgfpathcurveto{\pgfqpoint{1.544273in}{1.829904in}}{\pgfqpoint{1.552174in}{1.826632in}}{\pgfqpoint{1.560410in}{1.826632in}}%
\pgfpathclose%
\pgfusepath{stroke,fill}%
\end{pgfscope}%
\begin{pgfscope}%
\pgfpathrectangle{\pgfqpoint{0.100000in}{0.212622in}}{\pgfqpoint{3.696000in}{3.696000in}}%
\pgfusepath{clip}%
\pgfsetbuttcap%
\pgfsetroundjoin%
\definecolor{currentfill}{rgb}{0.121569,0.466667,0.705882}%
\pgfsetfillcolor{currentfill}%
\pgfsetfillopacity{0.415368}%
\pgfsetlinewidth{1.003750pt}%
\definecolor{currentstroke}{rgb}{0.121569,0.466667,0.705882}%
\pgfsetstrokecolor{currentstroke}%
\pgfsetstrokeopacity{0.415368}%
\pgfsetdash{}{0pt}%
\pgfpathmoveto{\pgfqpoint{1.557484in}{1.825463in}}%
\pgfpathcurveto{\pgfqpoint{1.565720in}{1.825463in}}{\pgfqpoint{1.573620in}{1.828735in}}{\pgfqpoint{1.579444in}{1.834559in}}%
\pgfpathcurveto{\pgfqpoint{1.585268in}{1.840383in}}{\pgfqpoint{1.588540in}{1.848283in}}{\pgfqpoint{1.588540in}{1.856520in}}%
\pgfpathcurveto{\pgfqpoint{1.588540in}{1.864756in}}{\pgfqpoint{1.585268in}{1.872656in}}{\pgfqpoint{1.579444in}{1.878480in}}%
\pgfpathcurveto{\pgfqpoint{1.573620in}{1.884304in}}{\pgfqpoint{1.565720in}{1.887576in}}{\pgfqpoint{1.557484in}{1.887576in}}%
\pgfpathcurveto{\pgfqpoint{1.549248in}{1.887576in}}{\pgfqpoint{1.541348in}{1.884304in}}{\pgfqpoint{1.535524in}{1.878480in}}%
\pgfpathcurveto{\pgfqpoint{1.529700in}{1.872656in}}{\pgfqpoint{1.526427in}{1.864756in}}{\pgfqpoint{1.526427in}{1.856520in}}%
\pgfpathcurveto{\pgfqpoint{1.526427in}{1.848283in}}{\pgfqpoint{1.529700in}{1.840383in}}{\pgfqpoint{1.535524in}{1.834559in}}%
\pgfpathcurveto{\pgfqpoint{1.541348in}{1.828735in}}{\pgfqpoint{1.549248in}{1.825463in}}{\pgfqpoint{1.557484in}{1.825463in}}%
\pgfpathclose%
\pgfusepath{stroke,fill}%
\end{pgfscope}%
\begin{pgfscope}%
\pgfpathrectangle{\pgfqpoint{0.100000in}{0.212622in}}{\pgfqpoint{3.696000in}{3.696000in}}%
\pgfusepath{clip}%
\pgfsetbuttcap%
\pgfsetroundjoin%
\definecolor{currentfill}{rgb}{0.121569,0.466667,0.705882}%
\pgfsetfillcolor{currentfill}%
\pgfsetfillopacity{0.416953}%
\pgfsetlinewidth{1.003750pt}%
\definecolor{currentstroke}{rgb}{0.121569,0.466667,0.705882}%
\pgfsetstrokecolor{currentstroke}%
\pgfsetstrokeopacity{0.416953}%
\pgfsetdash{}{0pt}%
\pgfpathmoveto{\pgfqpoint{1.552557in}{1.821162in}}%
\pgfpathcurveto{\pgfqpoint{1.560794in}{1.821162in}}{\pgfqpoint{1.568694in}{1.824434in}}{\pgfqpoint{1.574518in}{1.830258in}}%
\pgfpathcurveto{\pgfqpoint{1.580341in}{1.836082in}}{\pgfqpoint{1.583614in}{1.843982in}}{\pgfqpoint{1.583614in}{1.852219in}}%
\pgfpathcurveto{\pgfqpoint{1.583614in}{1.860455in}}{\pgfqpoint{1.580341in}{1.868355in}}{\pgfqpoint{1.574518in}{1.874179in}}%
\pgfpathcurveto{\pgfqpoint{1.568694in}{1.880003in}}{\pgfqpoint{1.560794in}{1.883275in}}{\pgfqpoint{1.552557in}{1.883275in}}%
\pgfpathcurveto{\pgfqpoint{1.544321in}{1.883275in}}{\pgfqpoint{1.536421in}{1.880003in}}{\pgfqpoint{1.530597in}{1.874179in}}%
\pgfpathcurveto{\pgfqpoint{1.524773in}{1.868355in}}{\pgfqpoint{1.521501in}{1.860455in}}{\pgfqpoint{1.521501in}{1.852219in}}%
\pgfpathcurveto{\pgfqpoint{1.521501in}{1.843982in}}{\pgfqpoint{1.524773in}{1.836082in}}{\pgfqpoint{1.530597in}{1.830258in}}%
\pgfpathcurveto{\pgfqpoint{1.536421in}{1.824434in}}{\pgfqpoint{1.544321in}{1.821162in}}{\pgfqpoint{1.552557in}{1.821162in}}%
\pgfpathclose%
\pgfusepath{stroke,fill}%
\end{pgfscope}%
\begin{pgfscope}%
\pgfpathrectangle{\pgfqpoint{0.100000in}{0.212622in}}{\pgfqpoint{3.696000in}{3.696000in}}%
\pgfusepath{clip}%
\pgfsetbuttcap%
\pgfsetroundjoin%
\definecolor{currentfill}{rgb}{0.121569,0.466667,0.705882}%
\pgfsetfillcolor{currentfill}%
\pgfsetfillopacity{0.417901}%
\pgfsetlinewidth{1.003750pt}%
\definecolor{currentstroke}{rgb}{0.121569,0.466667,0.705882}%
\pgfsetstrokecolor{currentstroke}%
\pgfsetstrokeopacity{0.417901}%
\pgfsetdash{}{0pt}%
\pgfpathmoveto{\pgfqpoint{1.549660in}{1.820752in}}%
\pgfpathcurveto{\pgfqpoint{1.557896in}{1.820752in}}{\pgfqpoint{1.565796in}{1.824025in}}{\pgfqpoint{1.571620in}{1.829849in}}%
\pgfpathcurveto{\pgfqpoint{1.577444in}{1.835672in}}{\pgfqpoint{1.580716in}{1.843573in}}{\pgfqpoint{1.580716in}{1.851809in}}%
\pgfpathcurveto{\pgfqpoint{1.580716in}{1.860045in}}{\pgfqpoint{1.577444in}{1.867945in}}{\pgfqpoint{1.571620in}{1.873769in}}%
\pgfpathcurveto{\pgfqpoint{1.565796in}{1.879593in}}{\pgfqpoint{1.557896in}{1.882865in}}{\pgfqpoint{1.549660in}{1.882865in}}%
\pgfpathcurveto{\pgfqpoint{1.541424in}{1.882865in}}{\pgfqpoint{1.533523in}{1.879593in}}{\pgfqpoint{1.527700in}{1.873769in}}%
\pgfpathcurveto{\pgfqpoint{1.521876in}{1.867945in}}{\pgfqpoint{1.518603in}{1.860045in}}{\pgfqpoint{1.518603in}{1.851809in}}%
\pgfpathcurveto{\pgfqpoint{1.518603in}{1.843573in}}{\pgfqpoint{1.521876in}{1.835672in}}{\pgfqpoint{1.527700in}{1.829849in}}%
\pgfpathcurveto{\pgfqpoint{1.533523in}{1.824025in}}{\pgfqpoint{1.541424in}{1.820752in}}{\pgfqpoint{1.549660in}{1.820752in}}%
\pgfpathclose%
\pgfusepath{stroke,fill}%
\end{pgfscope}%
\begin{pgfscope}%
\pgfpathrectangle{\pgfqpoint{0.100000in}{0.212622in}}{\pgfqpoint{3.696000in}{3.696000in}}%
\pgfusepath{clip}%
\pgfsetbuttcap%
\pgfsetroundjoin%
\definecolor{currentfill}{rgb}{0.121569,0.466667,0.705882}%
\pgfsetfillcolor{currentfill}%
\pgfsetfillopacity{0.418199}%
\pgfsetlinewidth{1.003750pt}%
\definecolor{currentstroke}{rgb}{0.121569,0.466667,0.705882}%
\pgfsetstrokecolor{currentstroke}%
\pgfsetstrokeopacity{0.418199}%
\pgfsetdash{}{0pt}%
\pgfpathmoveto{\pgfqpoint{2.007266in}{1.979018in}}%
\pgfpathcurveto{\pgfqpoint{2.015502in}{1.979018in}}{\pgfqpoint{2.023402in}{1.982291in}}{\pgfqpoint{2.029226in}{1.988115in}}%
\pgfpathcurveto{\pgfqpoint{2.035050in}{1.993938in}}{\pgfqpoint{2.038323in}{2.001839in}}{\pgfqpoint{2.038323in}{2.010075in}}%
\pgfpathcurveto{\pgfqpoint{2.038323in}{2.018311in}}{\pgfqpoint{2.035050in}{2.026211in}}{\pgfqpoint{2.029226in}{2.032035in}}%
\pgfpathcurveto{\pgfqpoint{2.023402in}{2.037859in}}{\pgfqpoint{2.015502in}{2.041131in}}{\pgfqpoint{2.007266in}{2.041131in}}%
\pgfpathcurveto{\pgfqpoint{1.999030in}{2.041131in}}{\pgfqpoint{1.991130in}{2.037859in}}{\pgfqpoint{1.985306in}{2.032035in}}%
\pgfpathcurveto{\pgfqpoint{1.979482in}{2.026211in}}{\pgfqpoint{1.976210in}{2.018311in}}{\pgfqpoint{1.976210in}{2.010075in}}%
\pgfpathcurveto{\pgfqpoint{1.976210in}{2.001839in}}{\pgfqpoint{1.979482in}{1.993938in}}{\pgfqpoint{1.985306in}{1.988115in}}%
\pgfpathcurveto{\pgfqpoint{1.991130in}{1.982291in}}{\pgfqpoint{1.999030in}{1.979018in}}{\pgfqpoint{2.007266in}{1.979018in}}%
\pgfpathclose%
\pgfusepath{stroke,fill}%
\end{pgfscope}%
\begin{pgfscope}%
\pgfpathrectangle{\pgfqpoint{0.100000in}{0.212622in}}{\pgfqpoint{3.696000in}{3.696000in}}%
\pgfusepath{clip}%
\pgfsetbuttcap%
\pgfsetroundjoin%
\definecolor{currentfill}{rgb}{0.121569,0.466667,0.705882}%
\pgfsetfillcolor{currentfill}%
\pgfsetfillopacity{0.418680}%
\pgfsetlinewidth{1.003750pt}%
\definecolor{currentstroke}{rgb}{0.121569,0.466667,0.705882}%
\pgfsetstrokecolor{currentstroke}%
\pgfsetstrokeopacity{0.418680}%
\pgfsetdash{}{0pt}%
\pgfpathmoveto{\pgfqpoint{1.547281in}{1.819828in}}%
\pgfpathcurveto{\pgfqpoint{1.555517in}{1.819828in}}{\pgfqpoint{1.563418in}{1.823100in}}{\pgfqpoint{1.569241in}{1.828924in}}%
\pgfpathcurveto{\pgfqpoint{1.575065in}{1.834748in}}{\pgfqpoint{1.578338in}{1.842648in}}{\pgfqpoint{1.578338in}{1.850885in}}%
\pgfpathcurveto{\pgfqpoint{1.578338in}{1.859121in}}{\pgfqpoint{1.575065in}{1.867021in}}{\pgfqpoint{1.569241in}{1.872845in}}%
\pgfpathcurveto{\pgfqpoint{1.563418in}{1.878669in}}{\pgfqpoint{1.555517in}{1.881941in}}{\pgfqpoint{1.547281in}{1.881941in}}%
\pgfpathcurveto{\pgfqpoint{1.539045in}{1.881941in}}{\pgfqpoint{1.531145in}{1.878669in}}{\pgfqpoint{1.525321in}{1.872845in}}%
\pgfpathcurveto{\pgfqpoint{1.519497in}{1.867021in}}{\pgfqpoint{1.516225in}{1.859121in}}{\pgfqpoint{1.516225in}{1.850885in}}%
\pgfpathcurveto{\pgfqpoint{1.516225in}{1.842648in}}{\pgfqpoint{1.519497in}{1.834748in}}{\pgfqpoint{1.525321in}{1.828924in}}%
\pgfpathcurveto{\pgfqpoint{1.531145in}{1.823100in}}{\pgfqpoint{1.539045in}{1.819828in}}{\pgfqpoint{1.547281in}{1.819828in}}%
\pgfpathclose%
\pgfusepath{stroke,fill}%
\end{pgfscope}%
\begin{pgfscope}%
\pgfpathrectangle{\pgfqpoint{0.100000in}{0.212622in}}{\pgfqpoint{3.696000in}{3.696000in}}%
\pgfusepath{clip}%
\pgfsetbuttcap%
\pgfsetroundjoin%
\definecolor{currentfill}{rgb}{0.121569,0.466667,0.705882}%
\pgfsetfillcolor{currentfill}%
\pgfsetfillopacity{0.419924}%
\pgfsetlinewidth{1.003750pt}%
\definecolor{currentstroke}{rgb}{0.121569,0.466667,0.705882}%
\pgfsetstrokecolor{currentstroke}%
\pgfsetstrokeopacity{0.419924}%
\pgfsetdash{}{0pt}%
\pgfpathmoveto{\pgfqpoint{1.543235in}{1.816593in}}%
\pgfpathcurveto{\pgfqpoint{1.551471in}{1.816593in}}{\pgfqpoint{1.559371in}{1.819865in}}{\pgfqpoint{1.565195in}{1.825689in}}%
\pgfpathcurveto{\pgfqpoint{1.571019in}{1.831513in}}{\pgfqpoint{1.574291in}{1.839413in}}{\pgfqpoint{1.574291in}{1.847650in}}%
\pgfpathcurveto{\pgfqpoint{1.574291in}{1.855886in}}{\pgfqpoint{1.571019in}{1.863786in}}{\pgfqpoint{1.565195in}{1.869610in}}%
\pgfpathcurveto{\pgfqpoint{1.559371in}{1.875434in}}{\pgfqpoint{1.551471in}{1.878706in}}{\pgfqpoint{1.543235in}{1.878706in}}%
\pgfpathcurveto{\pgfqpoint{1.534998in}{1.878706in}}{\pgfqpoint{1.527098in}{1.875434in}}{\pgfqpoint{1.521274in}{1.869610in}}%
\pgfpathcurveto{\pgfqpoint{1.515450in}{1.863786in}}{\pgfqpoint{1.512178in}{1.855886in}}{\pgfqpoint{1.512178in}{1.847650in}}%
\pgfpathcurveto{\pgfqpoint{1.512178in}{1.839413in}}{\pgfqpoint{1.515450in}{1.831513in}}{\pgfqpoint{1.521274in}{1.825689in}}%
\pgfpathcurveto{\pgfqpoint{1.527098in}{1.819865in}}{\pgfqpoint{1.534998in}{1.816593in}}{\pgfqpoint{1.543235in}{1.816593in}}%
\pgfpathclose%
\pgfusepath{stroke,fill}%
\end{pgfscope}%
\begin{pgfscope}%
\pgfpathrectangle{\pgfqpoint{0.100000in}{0.212622in}}{\pgfqpoint{3.696000in}{3.696000in}}%
\pgfusepath{clip}%
\pgfsetbuttcap%
\pgfsetroundjoin%
\definecolor{currentfill}{rgb}{0.121569,0.466667,0.705882}%
\pgfsetfillcolor{currentfill}%
\pgfsetfillopacity{0.420975}%
\pgfsetlinewidth{1.003750pt}%
\definecolor{currentstroke}{rgb}{0.121569,0.466667,0.705882}%
\pgfsetstrokecolor{currentstroke}%
\pgfsetstrokeopacity{0.420975}%
\pgfsetdash{}{0pt}%
\pgfpathmoveto{\pgfqpoint{1.540504in}{1.816243in}}%
\pgfpathcurveto{\pgfqpoint{1.548740in}{1.816243in}}{\pgfqpoint{1.556640in}{1.819515in}}{\pgfqpoint{1.562464in}{1.825339in}}%
\pgfpathcurveto{\pgfqpoint{1.568288in}{1.831163in}}{\pgfqpoint{1.571560in}{1.839063in}}{\pgfqpoint{1.571560in}{1.847299in}}%
\pgfpathcurveto{\pgfqpoint{1.571560in}{1.855535in}}{\pgfqpoint{1.568288in}{1.863436in}}{\pgfqpoint{1.562464in}{1.869259in}}%
\pgfpathcurveto{\pgfqpoint{1.556640in}{1.875083in}}{\pgfqpoint{1.548740in}{1.878356in}}{\pgfqpoint{1.540504in}{1.878356in}}%
\pgfpathcurveto{\pgfqpoint{1.532268in}{1.878356in}}{\pgfqpoint{1.524367in}{1.875083in}}{\pgfqpoint{1.518544in}{1.869259in}}%
\pgfpathcurveto{\pgfqpoint{1.512720in}{1.863436in}}{\pgfqpoint{1.509447in}{1.855535in}}{\pgfqpoint{1.509447in}{1.847299in}}%
\pgfpathcurveto{\pgfqpoint{1.509447in}{1.839063in}}{\pgfqpoint{1.512720in}{1.831163in}}{\pgfqpoint{1.518544in}{1.825339in}}%
\pgfpathcurveto{\pgfqpoint{1.524367in}{1.819515in}}{\pgfqpoint{1.532268in}{1.816243in}}{\pgfqpoint{1.540504in}{1.816243in}}%
\pgfpathclose%
\pgfusepath{stroke,fill}%
\end{pgfscope}%
\begin{pgfscope}%
\pgfpathrectangle{\pgfqpoint{0.100000in}{0.212622in}}{\pgfqpoint{3.696000in}{3.696000in}}%
\pgfusepath{clip}%
\pgfsetbuttcap%
\pgfsetroundjoin%
\definecolor{currentfill}{rgb}{0.121569,0.466667,0.705882}%
\pgfsetfillcolor{currentfill}%
\pgfsetfillopacity{0.421352}%
\pgfsetlinewidth{1.003750pt}%
\definecolor{currentstroke}{rgb}{0.121569,0.466667,0.705882}%
\pgfsetstrokecolor{currentstroke}%
\pgfsetstrokeopacity{0.421352}%
\pgfsetdash{}{0pt}%
\pgfpathmoveto{\pgfqpoint{2.009175in}{1.977041in}}%
\pgfpathcurveto{\pgfqpoint{2.017411in}{1.977041in}}{\pgfqpoint{2.025311in}{1.980313in}}{\pgfqpoint{2.031135in}{1.986137in}}%
\pgfpathcurveto{\pgfqpoint{2.036959in}{1.991961in}}{\pgfqpoint{2.040231in}{1.999861in}}{\pgfqpoint{2.040231in}{2.008098in}}%
\pgfpathcurveto{\pgfqpoint{2.040231in}{2.016334in}}{\pgfqpoint{2.036959in}{2.024234in}}{\pgfqpoint{2.031135in}{2.030058in}}%
\pgfpathcurveto{\pgfqpoint{2.025311in}{2.035882in}}{\pgfqpoint{2.017411in}{2.039154in}}{\pgfqpoint{2.009175in}{2.039154in}}%
\pgfpathcurveto{\pgfqpoint{2.000939in}{2.039154in}}{\pgfqpoint{1.993039in}{2.035882in}}{\pgfqpoint{1.987215in}{2.030058in}}%
\pgfpathcurveto{\pgfqpoint{1.981391in}{2.024234in}}{\pgfqpoint{1.978118in}{2.016334in}}{\pgfqpoint{1.978118in}{2.008098in}}%
\pgfpathcurveto{\pgfqpoint{1.978118in}{1.999861in}}{\pgfqpoint{1.981391in}{1.991961in}}{\pgfqpoint{1.987215in}{1.986137in}}%
\pgfpathcurveto{\pgfqpoint{1.993039in}{1.980313in}}{\pgfqpoint{2.000939in}{1.977041in}}{\pgfqpoint{2.009175in}{1.977041in}}%
\pgfpathclose%
\pgfusepath{stroke,fill}%
\end{pgfscope}%
\begin{pgfscope}%
\pgfpathrectangle{\pgfqpoint{0.100000in}{0.212622in}}{\pgfqpoint{3.696000in}{3.696000in}}%
\pgfusepath{clip}%
\pgfsetbuttcap%
\pgfsetroundjoin%
\definecolor{currentfill}{rgb}{0.121569,0.466667,0.705882}%
\pgfsetfillcolor{currentfill}%
\pgfsetfillopacity{0.421649}%
\pgfsetlinewidth{1.003750pt}%
\definecolor{currentstroke}{rgb}{0.121569,0.466667,0.705882}%
\pgfsetstrokecolor{currentstroke}%
\pgfsetstrokeopacity{0.421649}%
\pgfsetdash{}{0pt}%
\pgfpathmoveto{\pgfqpoint{1.538363in}{1.814916in}}%
\pgfpathcurveto{\pgfqpoint{1.546599in}{1.814916in}}{\pgfqpoint{1.554499in}{1.818188in}}{\pgfqpoint{1.560323in}{1.824012in}}%
\pgfpathcurveto{\pgfqpoint{1.566147in}{1.829836in}}{\pgfqpoint{1.569419in}{1.837736in}}{\pgfqpoint{1.569419in}{1.845972in}}%
\pgfpathcurveto{\pgfqpoint{1.569419in}{1.854209in}}{\pgfqpoint{1.566147in}{1.862109in}}{\pgfqpoint{1.560323in}{1.867933in}}%
\pgfpathcurveto{\pgfqpoint{1.554499in}{1.873757in}}{\pgfqpoint{1.546599in}{1.877029in}}{\pgfqpoint{1.538363in}{1.877029in}}%
\pgfpathcurveto{\pgfqpoint{1.530126in}{1.877029in}}{\pgfqpoint{1.522226in}{1.873757in}}{\pgfqpoint{1.516402in}{1.867933in}}%
\pgfpathcurveto{\pgfqpoint{1.510578in}{1.862109in}}{\pgfqpoint{1.507306in}{1.854209in}}{\pgfqpoint{1.507306in}{1.845972in}}%
\pgfpathcurveto{\pgfqpoint{1.507306in}{1.837736in}}{\pgfqpoint{1.510578in}{1.829836in}}{\pgfqpoint{1.516402in}{1.824012in}}%
\pgfpathcurveto{\pgfqpoint{1.522226in}{1.818188in}}{\pgfqpoint{1.530126in}{1.814916in}}{\pgfqpoint{1.538363in}{1.814916in}}%
\pgfpathclose%
\pgfusepath{stroke,fill}%
\end{pgfscope}%
\begin{pgfscope}%
\pgfpathrectangle{\pgfqpoint{0.100000in}{0.212622in}}{\pgfqpoint{3.696000in}{3.696000in}}%
\pgfusepath{clip}%
\pgfsetbuttcap%
\pgfsetroundjoin%
\definecolor{currentfill}{rgb}{0.121569,0.466667,0.705882}%
\pgfsetfillcolor{currentfill}%
\pgfsetfillopacity{0.422743}%
\pgfsetlinewidth{1.003750pt}%
\definecolor{currentstroke}{rgb}{0.121569,0.466667,0.705882}%
\pgfsetstrokecolor{currentstroke}%
\pgfsetstrokeopacity{0.422743}%
\pgfsetdash{}{0pt}%
\pgfpathmoveto{\pgfqpoint{1.534883in}{1.811072in}}%
\pgfpathcurveto{\pgfqpoint{1.543119in}{1.811072in}}{\pgfqpoint{1.551019in}{1.814345in}}{\pgfqpoint{1.556843in}{1.820169in}}%
\pgfpathcurveto{\pgfqpoint{1.562667in}{1.825993in}}{\pgfqpoint{1.565939in}{1.833893in}}{\pgfqpoint{1.565939in}{1.842129in}}%
\pgfpathcurveto{\pgfqpoint{1.565939in}{1.850365in}}{\pgfqpoint{1.562667in}{1.858265in}}{\pgfqpoint{1.556843in}{1.864089in}}%
\pgfpathcurveto{\pgfqpoint{1.551019in}{1.869913in}}{\pgfqpoint{1.543119in}{1.873185in}}{\pgfqpoint{1.534883in}{1.873185in}}%
\pgfpathcurveto{\pgfqpoint{1.526647in}{1.873185in}}{\pgfqpoint{1.518747in}{1.869913in}}{\pgfqpoint{1.512923in}{1.864089in}}%
\pgfpathcurveto{\pgfqpoint{1.507099in}{1.858265in}}{\pgfqpoint{1.503826in}{1.850365in}}{\pgfqpoint{1.503826in}{1.842129in}}%
\pgfpathcurveto{\pgfqpoint{1.503826in}{1.833893in}}{\pgfqpoint{1.507099in}{1.825993in}}{\pgfqpoint{1.512923in}{1.820169in}}%
\pgfpathcurveto{\pgfqpoint{1.518747in}{1.814345in}}{\pgfqpoint{1.526647in}{1.811072in}}{\pgfqpoint{1.534883in}{1.811072in}}%
\pgfpathclose%
\pgfusepath{stroke,fill}%
\end{pgfscope}%
\begin{pgfscope}%
\pgfpathrectangle{\pgfqpoint{0.100000in}{0.212622in}}{\pgfqpoint{3.696000in}{3.696000in}}%
\pgfusepath{clip}%
\pgfsetbuttcap%
\pgfsetroundjoin%
\definecolor{currentfill}{rgb}{0.121569,0.466667,0.705882}%
\pgfsetfillcolor{currentfill}%
\pgfsetfillopacity{0.422958}%
\pgfsetlinewidth{1.003750pt}%
\definecolor{currentstroke}{rgb}{0.121569,0.466667,0.705882}%
\pgfsetstrokecolor{currentstroke}%
\pgfsetstrokeopacity{0.422958}%
\pgfsetdash{}{0pt}%
\pgfpathmoveto{\pgfqpoint{2.009929in}{1.974938in}}%
\pgfpathcurveto{\pgfqpoint{2.018165in}{1.974938in}}{\pgfqpoint{2.026065in}{1.978210in}}{\pgfqpoint{2.031889in}{1.984034in}}%
\pgfpathcurveto{\pgfqpoint{2.037713in}{1.989858in}}{\pgfqpoint{2.040985in}{1.997758in}}{\pgfqpoint{2.040985in}{2.005994in}}%
\pgfpathcurveto{\pgfqpoint{2.040985in}{2.014231in}}{\pgfqpoint{2.037713in}{2.022131in}}{\pgfqpoint{2.031889in}{2.027955in}}%
\pgfpathcurveto{\pgfqpoint{2.026065in}{2.033779in}}{\pgfqpoint{2.018165in}{2.037051in}}{\pgfqpoint{2.009929in}{2.037051in}}%
\pgfpathcurveto{\pgfqpoint{2.001692in}{2.037051in}}{\pgfqpoint{1.993792in}{2.033779in}}{\pgfqpoint{1.987968in}{2.027955in}}%
\pgfpathcurveto{\pgfqpoint{1.982144in}{2.022131in}}{\pgfqpoint{1.978872in}{2.014231in}}{\pgfqpoint{1.978872in}{2.005994in}}%
\pgfpathcurveto{\pgfqpoint{1.978872in}{1.997758in}}{\pgfqpoint{1.982144in}{1.989858in}}{\pgfqpoint{1.987968in}{1.984034in}}%
\pgfpathcurveto{\pgfqpoint{1.993792in}{1.978210in}}{\pgfqpoint{2.001692in}{1.974938in}}{\pgfqpoint{2.009929in}{1.974938in}}%
\pgfpathclose%
\pgfusepath{stroke,fill}%
\end{pgfscope}%
\begin{pgfscope}%
\pgfpathrectangle{\pgfqpoint{0.100000in}{0.212622in}}{\pgfqpoint{3.696000in}{3.696000in}}%
\pgfusepath{clip}%
\pgfsetbuttcap%
\pgfsetroundjoin%
\definecolor{currentfill}{rgb}{0.121569,0.466667,0.705882}%
\pgfsetfillcolor{currentfill}%
\pgfsetfillopacity{0.423504}%
\pgfsetlinewidth{1.003750pt}%
\definecolor{currentstroke}{rgb}{0.121569,0.466667,0.705882}%
\pgfsetstrokecolor{currentstroke}%
\pgfsetstrokeopacity{0.423504}%
\pgfsetdash{}{0pt}%
\pgfpathmoveto{\pgfqpoint{1.532893in}{1.810915in}}%
\pgfpathcurveto{\pgfqpoint{1.541129in}{1.810915in}}{\pgfqpoint{1.549029in}{1.814187in}}{\pgfqpoint{1.554853in}{1.820011in}}%
\pgfpathcurveto{\pgfqpoint{1.560677in}{1.825835in}}{\pgfqpoint{1.563949in}{1.833735in}}{\pgfqpoint{1.563949in}{1.841972in}}%
\pgfpathcurveto{\pgfqpoint{1.563949in}{1.850208in}}{\pgfqpoint{1.560677in}{1.858108in}}{\pgfqpoint{1.554853in}{1.863932in}}%
\pgfpathcurveto{\pgfqpoint{1.549029in}{1.869756in}}{\pgfqpoint{1.541129in}{1.873028in}}{\pgfqpoint{1.532893in}{1.873028in}}%
\pgfpathcurveto{\pgfqpoint{1.524657in}{1.873028in}}{\pgfqpoint{1.516757in}{1.869756in}}{\pgfqpoint{1.510933in}{1.863932in}}%
\pgfpathcurveto{\pgfqpoint{1.505109in}{1.858108in}}{\pgfqpoint{1.501836in}{1.850208in}}{\pgfqpoint{1.501836in}{1.841972in}}%
\pgfpathcurveto{\pgfqpoint{1.501836in}{1.833735in}}{\pgfqpoint{1.505109in}{1.825835in}}{\pgfqpoint{1.510933in}{1.820011in}}%
\pgfpathcurveto{\pgfqpoint{1.516757in}{1.814187in}}{\pgfqpoint{1.524657in}{1.810915in}}{\pgfqpoint{1.532893in}{1.810915in}}%
\pgfpathclose%
\pgfusepath{stroke,fill}%
\end{pgfscope}%
\begin{pgfscope}%
\pgfpathrectangle{\pgfqpoint{0.100000in}{0.212622in}}{\pgfqpoint{3.696000in}{3.696000in}}%
\pgfusepath{clip}%
\pgfsetbuttcap%
\pgfsetroundjoin%
\definecolor{currentfill}{rgb}{0.121569,0.466667,0.705882}%
\pgfsetfillcolor{currentfill}%
\pgfsetfillopacity{0.424035}%
\pgfsetlinewidth{1.003750pt}%
\definecolor{currentstroke}{rgb}{0.121569,0.466667,0.705882}%
\pgfsetstrokecolor{currentstroke}%
\pgfsetstrokeopacity{0.424035}%
\pgfsetdash{}{0pt}%
\pgfpathmoveto{\pgfqpoint{1.531454in}{1.810403in}}%
\pgfpathcurveto{\pgfqpoint{1.539690in}{1.810403in}}{\pgfqpoint{1.547590in}{1.813675in}}{\pgfqpoint{1.553414in}{1.819499in}}%
\pgfpathcurveto{\pgfqpoint{1.559238in}{1.825323in}}{\pgfqpoint{1.562510in}{1.833223in}}{\pgfqpoint{1.562510in}{1.841459in}}%
\pgfpathcurveto{\pgfqpoint{1.562510in}{1.849696in}}{\pgfqpoint{1.559238in}{1.857596in}}{\pgfqpoint{1.553414in}{1.863420in}}%
\pgfpathcurveto{\pgfqpoint{1.547590in}{1.869243in}}{\pgfqpoint{1.539690in}{1.872516in}}{\pgfqpoint{1.531454in}{1.872516in}}%
\pgfpathcurveto{\pgfqpoint{1.523217in}{1.872516in}}{\pgfqpoint{1.515317in}{1.869243in}}{\pgfqpoint{1.509493in}{1.863420in}}%
\pgfpathcurveto{\pgfqpoint{1.503669in}{1.857596in}}{\pgfqpoint{1.500397in}{1.849696in}}{\pgfqpoint{1.500397in}{1.841459in}}%
\pgfpathcurveto{\pgfqpoint{1.500397in}{1.833223in}}{\pgfqpoint{1.503669in}{1.825323in}}{\pgfqpoint{1.509493in}{1.819499in}}%
\pgfpathcurveto{\pgfqpoint{1.515317in}{1.813675in}}{\pgfqpoint{1.523217in}{1.810403in}}{\pgfqpoint{1.531454in}{1.810403in}}%
\pgfpathclose%
\pgfusepath{stroke,fill}%
\end{pgfscope}%
\begin{pgfscope}%
\pgfpathrectangle{\pgfqpoint{0.100000in}{0.212622in}}{\pgfqpoint{3.696000in}{3.696000in}}%
\pgfusepath{clip}%
\pgfsetbuttcap%
\pgfsetroundjoin%
\definecolor{currentfill}{rgb}{0.121569,0.466667,0.705882}%
\pgfsetfillcolor{currentfill}%
\pgfsetfillopacity{0.424770}%
\pgfsetlinewidth{1.003750pt}%
\definecolor{currentstroke}{rgb}{0.121569,0.466667,0.705882}%
\pgfsetstrokecolor{currentstroke}%
\pgfsetstrokeopacity{0.424770}%
\pgfsetdash{}{0pt}%
\pgfpathmoveto{\pgfqpoint{1.528703in}{1.808131in}}%
\pgfpathcurveto{\pgfqpoint{1.536939in}{1.808131in}}{\pgfqpoint{1.544839in}{1.811403in}}{\pgfqpoint{1.550663in}{1.817227in}}%
\pgfpathcurveto{\pgfqpoint{1.556487in}{1.823051in}}{\pgfqpoint{1.559759in}{1.830951in}}{\pgfqpoint{1.559759in}{1.839187in}}%
\pgfpathcurveto{\pgfqpoint{1.559759in}{1.847423in}}{\pgfqpoint{1.556487in}{1.855323in}}{\pgfqpoint{1.550663in}{1.861147in}}%
\pgfpathcurveto{\pgfqpoint{1.544839in}{1.866971in}}{\pgfqpoint{1.536939in}{1.870244in}}{\pgfqpoint{1.528703in}{1.870244in}}%
\pgfpathcurveto{\pgfqpoint{1.520466in}{1.870244in}}{\pgfqpoint{1.512566in}{1.866971in}}{\pgfqpoint{1.506742in}{1.861147in}}%
\pgfpathcurveto{\pgfqpoint{1.500918in}{1.855323in}}{\pgfqpoint{1.497646in}{1.847423in}}{\pgfqpoint{1.497646in}{1.839187in}}%
\pgfpathcurveto{\pgfqpoint{1.497646in}{1.830951in}}{\pgfqpoint{1.500918in}{1.823051in}}{\pgfqpoint{1.506742in}{1.817227in}}%
\pgfpathcurveto{\pgfqpoint{1.512566in}{1.811403in}}{\pgfqpoint{1.520466in}{1.808131in}}{\pgfqpoint{1.528703in}{1.808131in}}%
\pgfpathclose%
\pgfusepath{stroke,fill}%
\end{pgfscope}%
\begin{pgfscope}%
\pgfpathrectangle{\pgfqpoint{0.100000in}{0.212622in}}{\pgfqpoint{3.696000in}{3.696000in}}%
\pgfusepath{clip}%
\pgfsetbuttcap%
\pgfsetroundjoin%
\definecolor{currentfill}{rgb}{0.121569,0.466667,0.705882}%
\pgfsetfillcolor{currentfill}%
\pgfsetfillopacity{0.424903}%
\pgfsetlinewidth{1.003750pt}%
\definecolor{currentstroke}{rgb}{0.121569,0.466667,0.705882}%
\pgfsetstrokecolor{currentstroke}%
\pgfsetstrokeopacity{0.424903}%
\pgfsetdash{}{0pt}%
\pgfpathmoveto{\pgfqpoint{2.011301in}{1.972812in}}%
\pgfpathcurveto{\pgfqpoint{2.019538in}{1.972812in}}{\pgfqpoint{2.027438in}{1.976085in}}{\pgfqpoint{2.033262in}{1.981909in}}%
\pgfpathcurveto{\pgfqpoint{2.039086in}{1.987732in}}{\pgfqpoint{2.042358in}{1.995633in}}{\pgfqpoint{2.042358in}{2.003869in}}%
\pgfpathcurveto{\pgfqpoint{2.042358in}{2.012105in}}{\pgfqpoint{2.039086in}{2.020005in}}{\pgfqpoint{2.033262in}{2.025829in}}%
\pgfpathcurveto{\pgfqpoint{2.027438in}{2.031653in}}{\pgfqpoint{2.019538in}{2.034925in}}{\pgfqpoint{2.011301in}{2.034925in}}%
\pgfpathcurveto{\pgfqpoint{2.003065in}{2.034925in}}{\pgfqpoint{1.995165in}{2.031653in}}{\pgfqpoint{1.989341in}{2.025829in}}%
\pgfpathcurveto{\pgfqpoint{1.983517in}{2.020005in}}{\pgfqpoint{1.980245in}{2.012105in}}{\pgfqpoint{1.980245in}{2.003869in}}%
\pgfpathcurveto{\pgfqpoint{1.980245in}{1.995633in}}{\pgfqpoint{1.983517in}{1.987732in}}{\pgfqpoint{1.989341in}{1.981909in}}%
\pgfpathcurveto{\pgfqpoint{1.995165in}{1.976085in}}{\pgfqpoint{2.003065in}{1.972812in}}{\pgfqpoint{2.011301in}{1.972812in}}%
\pgfpathclose%
\pgfusepath{stroke,fill}%
\end{pgfscope}%
\begin{pgfscope}%
\pgfpathrectangle{\pgfqpoint{0.100000in}{0.212622in}}{\pgfqpoint{3.696000in}{3.696000in}}%
\pgfusepath{clip}%
\pgfsetbuttcap%
\pgfsetroundjoin%
\definecolor{currentfill}{rgb}{0.121569,0.466667,0.705882}%
\pgfsetfillcolor{currentfill}%
\pgfsetfillopacity{0.425139}%
\pgfsetlinewidth{1.003750pt}%
\definecolor{currentstroke}{rgb}{0.121569,0.466667,0.705882}%
\pgfsetstrokecolor{currentstroke}%
\pgfsetstrokeopacity{0.425139}%
\pgfsetdash{}{0pt}%
\pgfpathmoveto{\pgfqpoint{1.527693in}{1.807889in}}%
\pgfpathcurveto{\pgfqpoint{1.535929in}{1.807889in}}{\pgfqpoint{1.543829in}{1.811162in}}{\pgfqpoint{1.549653in}{1.816986in}}%
\pgfpathcurveto{\pgfqpoint{1.555477in}{1.822809in}}{\pgfqpoint{1.558749in}{1.830710in}}{\pgfqpoint{1.558749in}{1.838946in}}%
\pgfpathcurveto{\pgfqpoint{1.558749in}{1.847182in}}{\pgfqpoint{1.555477in}{1.855082in}}{\pgfqpoint{1.549653in}{1.860906in}}%
\pgfpathcurveto{\pgfqpoint{1.543829in}{1.866730in}}{\pgfqpoint{1.535929in}{1.870002in}}{\pgfqpoint{1.527693in}{1.870002in}}%
\pgfpathcurveto{\pgfqpoint{1.519456in}{1.870002in}}{\pgfqpoint{1.511556in}{1.866730in}}{\pgfqpoint{1.505732in}{1.860906in}}%
\pgfpathcurveto{\pgfqpoint{1.499909in}{1.855082in}}{\pgfqpoint{1.496636in}{1.847182in}}{\pgfqpoint{1.496636in}{1.838946in}}%
\pgfpathcurveto{\pgfqpoint{1.496636in}{1.830710in}}{\pgfqpoint{1.499909in}{1.822809in}}{\pgfqpoint{1.505732in}{1.816986in}}%
\pgfpathcurveto{\pgfqpoint{1.511556in}{1.811162in}}{\pgfqpoint{1.519456in}{1.807889in}}{\pgfqpoint{1.527693in}{1.807889in}}%
\pgfpathclose%
\pgfusepath{stroke,fill}%
\end{pgfscope}%
\begin{pgfscope}%
\pgfpathrectangle{\pgfqpoint{0.100000in}{0.212622in}}{\pgfqpoint{3.696000in}{3.696000in}}%
\pgfusepath{clip}%
\pgfsetbuttcap%
\pgfsetroundjoin%
\definecolor{currentfill}{rgb}{0.121569,0.466667,0.705882}%
\pgfsetfillcolor{currentfill}%
\pgfsetfillopacity{0.425769}%
\pgfsetlinewidth{1.003750pt}%
\definecolor{currentstroke}{rgb}{0.121569,0.466667,0.705882}%
\pgfsetstrokecolor{currentstroke}%
\pgfsetstrokeopacity{0.425769}%
\pgfsetdash{}{0pt}%
\pgfpathmoveto{\pgfqpoint{1.525844in}{1.807187in}}%
\pgfpathcurveto{\pgfqpoint{1.534081in}{1.807187in}}{\pgfqpoint{1.541981in}{1.810459in}}{\pgfqpoint{1.547805in}{1.816283in}}%
\pgfpathcurveto{\pgfqpoint{1.553629in}{1.822107in}}{\pgfqpoint{1.556901in}{1.830007in}}{\pgfqpoint{1.556901in}{1.838243in}}%
\pgfpathcurveto{\pgfqpoint{1.556901in}{1.846480in}}{\pgfqpoint{1.553629in}{1.854380in}}{\pgfqpoint{1.547805in}{1.860204in}}%
\pgfpathcurveto{\pgfqpoint{1.541981in}{1.866027in}}{\pgfqpoint{1.534081in}{1.869300in}}{\pgfqpoint{1.525844in}{1.869300in}}%
\pgfpathcurveto{\pgfqpoint{1.517608in}{1.869300in}}{\pgfqpoint{1.509708in}{1.866027in}}{\pgfqpoint{1.503884in}{1.860204in}}%
\pgfpathcurveto{\pgfqpoint{1.498060in}{1.854380in}}{\pgfqpoint{1.494788in}{1.846480in}}{\pgfqpoint{1.494788in}{1.838243in}}%
\pgfpathcurveto{\pgfqpoint{1.494788in}{1.830007in}}{\pgfqpoint{1.498060in}{1.822107in}}{\pgfqpoint{1.503884in}{1.816283in}}%
\pgfpathcurveto{\pgfqpoint{1.509708in}{1.810459in}}{\pgfqpoint{1.517608in}{1.807187in}}{\pgfqpoint{1.525844in}{1.807187in}}%
\pgfpathclose%
\pgfusepath{stroke,fill}%
\end{pgfscope}%
\begin{pgfscope}%
\pgfpathrectangle{\pgfqpoint{0.100000in}{0.212622in}}{\pgfqpoint{3.696000in}{3.696000in}}%
\pgfusepath{clip}%
\pgfsetbuttcap%
\pgfsetroundjoin%
\definecolor{currentfill}{rgb}{0.121569,0.466667,0.705882}%
\pgfsetfillcolor{currentfill}%
\pgfsetfillopacity{0.426680}%
\pgfsetlinewidth{1.003750pt}%
\definecolor{currentstroke}{rgb}{0.121569,0.466667,0.705882}%
\pgfsetstrokecolor{currentstroke}%
\pgfsetstrokeopacity{0.426680}%
\pgfsetdash{}{0pt}%
\pgfpathmoveto{\pgfqpoint{1.522570in}{1.804237in}}%
\pgfpathcurveto{\pgfqpoint{1.530806in}{1.804237in}}{\pgfqpoint{1.538706in}{1.807509in}}{\pgfqpoint{1.544530in}{1.813333in}}%
\pgfpathcurveto{\pgfqpoint{1.550354in}{1.819157in}}{\pgfqpoint{1.553627in}{1.827057in}}{\pgfqpoint{1.553627in}{1.835293in}}%
\pgfpathcurveto{\pgfqpoint{1.553627in}{1.843530in}}{\pgfqpoint{1.550354in}{1.851430in}}{\pgfqpoint{1.544530in}{1.857254in}}%
\pgfpathcurveto{\pgfqpoint{1.538706in}{1.863078in}}{\pgfqpoint{1.530806in}{1.866350in}}{\pgfqpoint{1.522570in}{1.866350in}}%
\pgfpathcurveto{\pgfqpoint{1.514334in}{1.866350in}}{\pgfqpoint{1.506434in}{1.863078in}}{\pgfqpoint{1.500610in}{1.857254in}}%
\pgfpathcurveto{\pgfqpoint{1.494786in}{1.851430in}}{\pgfqpoint{1.491514in}{1.843530in}}{\pgfqpoint{1.491514in}{1.835293in}}%
\pgfpathcurveto{\pgfqpoint{1.491514in}{1.827057in}}{\pgfqpoint{1.494786in}{1.819157in}}{\pgfqpoint{1.500610in}{1.813333in}}%
\pgfpathcurveto{\pgfqpoint{1.506434in}{1.807509in}}{\pgfqpoint{1.514334in}{1.804237in}}{\pgfqpoint{1.522570in}{1.804237in}}%
\pgfpathclose%
\pgfusepath{stroke,fill}%
\end{pgfscope}%
\begin{pgfscope}%
\pgfpathrectangle{\pgfqpoint{0.100000in}{0.212622in}}{\pgfqpoint{3.696000in}{3.696000in}}%
\pgfusepath{clip}%
\pgfsetbuttcap%
\pgfsetroundjoin%
\definecolor{currentfill}{rgb}{0.121569,0.466667,0.705882}%
\pgfsetfillcolor{currentfill}%
\pgfsetfillopacity{0.426956}%
\pgfsetlinewidth{1.003750pt}%
\definecolor{currentstroke}{rgb}{0.121569,0.466667,0.705882}%
\pgfsetstrokecolor{currentstroke}%
\pgfsetstrokeopacity{0.426956}%
\pgfsetdash{}{0pt}%
\pgfpathmoveto{\pgfqpoint{1.521318in}{1.803149in}}%
\pgfpathcurveto{\pgfqpoint{1.529554in}{1.803149in}}{\pgfqpoint{1.537454in}{1.806421in}}{\pgfqpoint{1.543278in}{1.812245in}}%
\pgfpathcurveto{\pgfqpoint{1.549102in}{1.818069in}}{\pgfqpoint{1.552374in}{1.825969in}}{\pgfqpoint{1.552374in}{1.834206in}}%
\pgfpathcurveto{\pgfqpoint{1.552374in}{1.842442in}}{\pgfqpoint{1.549102in}{1.850342in}}{\pgfqpoint{1.543278in}{1.856166in}}%
\pgfpathcurveto{\pgfqpoint{1.537454in}{1.861990in}}{\pgfqpoint{1.529554in}{1.865262in}}{\pgfqpoint{1.521318in}{1.865262in}}%
\pgfpathcurveto{\pgfqpoint{1.513081in}{1.865262in}}{\pgfqpoint{1.505181in}{1.861990in}}{\pgfqpoint{1.499357in}{1.856166in}}%
\pgfpathcurveto{\pgfqpoint{1.493533in}{1.850342in}}{\pgfqpoint{1.490261in}{1.842442in}}{\pgfqpoint{1.490261in}{1.834206in}}%
\pgfpathcurveto{\pgfqpoint{1.490261in}{1.825969in}}{\pgfqpoint{1.493533in}{1.818069in}}{\pgfqpoint{1.499357in}{1.812245in}}%
\pgfpathcurveto{\pgfqpoint{1.505181in}{1.806421in}}{\pgfqpoint{1.513081in}{1.803149in}}{\pgfqpoint{1.521318in}{1.803149in}}%
\pgfpathclose%
\pgfusepath{stroke,fill}%
\end{pgfscope}%
\begin{pgfscope}%
\pgfpathrectangle{\pgfqpoint{0.100000in}{0.212622in}}{\pgfqpoint{3.696000in}{3.696000in}}%
\pgfusepath{clip}%
\pgfsetbuttcap%
\pgfsetroundjoin%
\definecolor{currentfill}{rgb}{0.121569,0.466667,0.705882}%
\pgfsetfillcolor{currentfill}%
\pgfsetfillopacity{0.427649}%
\pgfsetlinewidth{1.003750pt}%
\definecolor{currentstroke}{rgb}{0.121569,0.466667,0.705882}%
\pgfsetstrokecolor{currentstroke}%
\pgfsetstrokeopacity{0.427649}%
\pgfsetdash{}{0pt}%
\pgfpathmoveto{\pgfqpoint{1.519095in}{1.802336in}}%
\pgfpathcurveto{\pgfqpoint{1.527331in}{1.802336in}}{\pgfqpoint{1.535231in}{1.805608in}}{\pgfqpoint{1.541055in}{1.811432in}}%
\pgfpathcurveto{\pgfqpoint{1.546879in}{1.817256in}}{\pgfqpoint{1.550151in}{1.825156in}}{\pgfqpoint{1.550151in}{1.833393in}}%
\pgfpathcurveto{\pgfqpoint{1.550151in}{1.841629in}}{\pgfqpoint{1.546879in}{1.849529in}}{\pgfqpoint{1.541055in}{1.855353in}}%
\pgfpathcurveto{\pgfqpoint{1.535231in}{1.861177in}}{\pgfqpoint{1.527331in}{1.864449in}}{\pgfqpoint{1.519095in}{1.864449in}}%
\pgfpathcurveto{\pgfqpoint{1.510858in}{1.864449in}}{\pgfqpoint{1.502958in}{1.861177in}}{\pgfqpoint{1.497134in}{1.855353in}}%
\pgfpathcurveto{\pgfqpoint{1.491310in}{1.849529in}}{\pgfqpoint{1.488038in}{1.841629in}}{\pgfqpoint{1.488038in}{1.833393in}}%
\pgfpathcurveto{\pgfqpoint{1.488038in}{1.825156in}}{\pgfqpoint{1.491310in}{1.817256in}}{\pgfqpoint{1.497134in}{1.811432in}}%
\pgfpathcurveto{\pgfqpoint{1.502958in}{1.805608in}}{\pgfqpoint{1.510858in}{1.802336in}}{\pgfqpoint{1.519095in}{1.802336in}}%
\pgfpathclose%
\pgfusepath{stroke,fill}%
\end{pgfscope}%
\begin{pgfscope}%
\pgfpathrectangle{\pgfqpoint{0.100000in}{0.212622in}}{\pgfqpoint{3.696000in}{3.696000in}}%
\pgfusepath{clip}%
\pgfsetbuttcap%
\pgfsetroundjoin%
\definecolor{currentfill}{rgb}{0.121569,0.466667,0.705882}%
\pgfsetfillcolor{currentfill}%
\pgfsetfillopacity{0.427781}%
\pgfsetlinewidth{1.003750pt}%
\definecolor{currentstroke}{rgb}{0.121569,0.466667,0.705882}%
\pgfsetstrokecolor{currentstroke}%
\pgfsetstrokeopacity{0.427781}%
\pgfsetdash{}{0pt}%
\pgfpathmoveto{\pgfqpoint{2.013103in}{1.971139in}}%
\pgfpathcurveto{\pgfqpoint{2.021339in}{1.971139in}}{\pgfqpoint{2.029239in}{1.974411in}}{\pgfqpoint{2.035063in}{1.980235in}}%
\pgfpathcurveto{\pgfqpoint{2.040887in}{1.986059in}}{\pgfqpoint{2.044159in}{1.993959in}}{\pgfqpoint{2.044159in}{2.002196in}}%
\pgfpathcurveto{\pgfqpoint{2.044159in}{2.010432in}}{\pgfqpoint{2.040887in}{2.018332in}}{\pgfqpoint{2.035063in}{2.024156in}}%
\pgfpathcurveto{\pgfqpoint{2.029239in}{2.029980in}}{\pgfqpoint{2.021339in}{2.033252in}}{\pgfqpoint{2.013103in}{2.033252in}}%
\pgfpathcurveto{\pgfqpoint{2.004866in}{2.033252in}}{\pgfqpoint{1.996966in}{2.029980in}}{\pgfqpoint{1.991142in}{2.024156in}}%
\pgfpathcurveto{\pgfqpoint{1.985318in}{2.018332in}}{\pgfqpoint{1.982046in}{2.010432in}}{\pgfqpoint{1.982046in}{2.002196in}}%
\pgfpathcurveto{\pgfqpoint{1.982046in}{1.993959in}}{\pgfqpoint{1.985318in}{1.986059in}}{\pgfqpoint{1.991142in}{1.980235in}}%
\pgfpathcurveto{\pgfqpoint{1.996966in}{1.974411in}}{\pgfqpoint{2.004866in}{1.971139in}}{\pgfqpoint{2.013103in}{1.971139in}}%
\pgfpathclose%
\pgfusepath{stroke,fill}%
\end{pgfscope}%
\begin{pgfscope}%
\pgfpathrectangle{\pgfqpoint{0.100000in}{0.212622in}}{\pgfqpoint{3.696000in}{3.696000in}}%
\pgfusepath{clip}%
\pgfsetbuttcap%
\pgfsetroundjoin%
\definecolor{currentfill}{rgb}{0.121569,0.466667,0.705882}%
\pgfsetfillcolor{currentfill}%
\pgfsetfillopacity{0.428654}%
\pgfsetlinewidth{1.003750pt}%
\definecolor{currentstroke}{rgb}{0.121569,0.466667,0.705882}%
\pgfsetstrokecolor{currentstroke}%
\pgfsetstrokeopacity{0.428654}%
\pgfsetdash{}{0pt}%
\pgfpathmoveto{\pgfqpoint{1.515775in}{1.798153in}}%
\pgfpathcurveto{\pgfqpoint{1.524011in}{1.798153in}}{\pgfqpoint{1.531911in}{1.801425in}}{\pgfqpoint{1.537735in}{1.807249in}}%
\pgfpathcurveto{\pgfqpoint{1.543559in}{1.813073in}}{\pgfqpoint{1.546831in}{1.820973in}}{\pgfqpoint{1.546831in}{1.829210in}}%
\pgfpathcurveto{\pgfqpoint{1.546831in}{1.837446in}}{\pgfqpoint{1.543559in}{1.845346in}}{\pgfqpoint{1.537735in}{1.851170in}}%
\pgfpathcurveto{\pgfqpoint{1.531911in}{1.856994in}}{\pgfqpoint{1.524011in}{1.860266in}}{\pgfqpoint{1.515775in}{1.860266in}}%
\pgfpathcurveto{\pgfqpoint{1.507539in}{1.860266in}}{\pgfqpoint{1.499639in}{1.856994in}}{\pgfqpoint{1.493815in}{1.851170in}}%
\pgfpathcurveto{\pgfqpoint{1.487991in}{1.845346in}}{\pgfqpoint{1.484718in}{1.837446in}}{\pgfqpoint{1.484718in}{1.829210in}}%
\pgfpathcurveto{\pgfqpoint{1.484718in}{1.820973in}}{\pgfqpoint{1.487991in}{1.813073in}}{\pgfqpoint{1.493815in}{1.807249in}}%
\pgfpathcurveto{\pgfqpoint{1.499639in}{1.801425in}}{\pgfqpoint{1.507539in}{1.798153in}}{\pgfqpoint{1.515775in}{1.798153in}}%
\pgfpathclose%
\pgfusepath{stroke,fill}%
\end{pgfscope}%
\begin{pgfscope}%
\pgfpathrectangle{\pgfqpoint{0.100000in}{0.212622in}}{\pgfqpoint{3.696000in}{3.696000in}}%
\pgfusepath{clip}%
\pgfsetbuttcap%
\pgfsetroundjoin%
\definecolor{currentfill}{rgb}{0.121569,0.466667,0.705882}%
\pgfsetfillcolor{currentfill}%
\pgfsetfillopacity{0.429309}%
\pgfsetlinewidth{1.003750pt}%
\definecolor{currentstroke}{rgb}{0.121569,0.466667,0.705882}%
\pgfsetstrokecolor{currentstroke}%
\pgfsetstrokeopacity{0.429309}%
\pgfsetdash{}{0pt}%
\pgfpathmoveto{\pgfqpoint{1.513937in}{1.797758in}}%
\pgfpathcurveto{\pgfqpoint{1.522173in}{1.797758in}}{\pgfqpoint{1.530073in}{1.801030in}}{\pgfqpoint{1.535897in}{1.806854in}}%
\pgfpathcurveto{\pgfqpoint{1.541721in}{1.812678in}}{\pgfqpoint{1.544993in}{1.820578in}}{\pgfqpoint{1.544993in}{1.828815in}}%
\pgfpathcurveto{\pgfqpoint{1.544993in}{1.837051in}}{\pgfqpoint{1.541721in}{1.844951in}}{\pgfqpoint{1.535897in}{1.850775in}}%
\pgfpathcurveto{\pgfqpoint{1.530073in}{1.856599in}}{\pgfqpoint{1.522173in}{1.859871in}}{\pgfqpoint{1.513937in}{1.859871in}}%
\pgfpathcurveto{\pgfqpoint{1.505700in}{1.859871in}}{\pgfqpoint{1.497800in}{1.856599in}}{\pgfqpoint{1.491976in}{1.850775in}}%
\pgfpathcurveto{\pgfqpoint{1.486153in}{1.844951in}}{\pgfqpoint{1.482880in}{1.837051in}}{\pgfqpoint{1.482880in}{1.828815in}}%
\pgfpathcurveto{\pgfqpoint{1.482880in}{1.820578in}}{\pgfqpoint{1.486153in}{1.812678in}}{\pgfqpoint{1.491976in}{1.806854in}}%
\pgfpathcurveto{\pgfqpoint{1.497800in}{1.801030in}}{\pgfqpoint{1.505700in}{1.797758in}}{\pgfqpoint{1.513937in}{1.797758in}}%
\pgfpathclose%
\pgfusepath{stroke,fill}%
\end{pgfscope}%
\begin{pgfscope}%
\pgfpathrectangle{\pgfqpoint{0.100000in}{0.212622in}}{\pgfqpoint{3.696000in}{3.696000in}}%
\pgfusepath{clip}%
\pgfsetbuttcap%
\pgfsetroundjoin%
\definecolor{currentfill}{rgb}{0.121569,0.466667,0.705882}%
\pgfsetfillcolor{currentfill}%
\pgfsetfillopacity{0.430437}%
\pgfsetlinewidth{1.003750pt}%
\definecolor{currentstroke}{rgb}{0.121569,0.466667,0.705882}%
\pgfsetstrokecolor{currentstroke}%
\pgfsetstrokeopacity{0.430437}%
\pgfsetdash{}{0pt}%
\pgfpathmoveto{\pgfqpoint{1.510565in}{1.796666in}}%
\pgfpathcurveto{\pgfqpoint{1.518802in}{1.796666in}}{\pgfqpoint{1.526702in}{1.799938in}}{\pgfqpoint{1.532525in}{1.805762in}}%
\pgfpathcurveto{\pgfqpoint{1.538349in}{1.811586in}}{\pgfqpoint{1.541622in}{1.819486in}}{\pgfqpoint{1.541622in}{1.827722in}}%
\pgfpathcurveto{\pgfqpoint{1.541622in}{1.835959in}}{\pgfqpoint{1.538349in}{1.843859in}}{\pgfqpoint{1.532525in}{1.849683in}}%
\pgfpathcurveto{\pgfqpoint{1.526702in}{1.855507in}}{\pgfqpoint{1.518802in}{1.858779in}}{\pgfqpoint{1.510565in}{1.858779in}}%
\pgfpathcurveto{\pgfqpoint{1.502329in}{1.858779in}}{\pgfqpoint{1.494429in}{1.855507in}}{\pgfqpoint{1.488605in}{1.849683in}}%
\pgfpathcurveto{\pgfqpoint{1.482781in}{1.843859in}}{\pgfqpoint{1.479509in}{1.835959in}}{\pgfqpoint{1.479509in}{1.827722in}}%
\pgfpathcurveto{\pgfqpoint{1.479509in}{1.819486in}}{\pgfqpoint{1.482781in}{1.811586in}}{\pgfqpoint{1.488605in}{1.805762in}}%
\pgfpathcurveto{\pgfqpoint{1.494429in}{1.799938in}}{\pgfqpoint{1.502329in}{1.796666in}}{\pgfqpoint{1.510565in}{1.796666in}}%
\pgfpathclose%
\pgfusepath{stroke,fill}%
\end{pgfscope}%
\begin{pgfscope}%
\pgfpathrectangle{\pgfqpoint{0.100000in}{0.212622in}}{\pgfqpoint{3.696000in}{3.696000in}}%
\pgfusepath{clip}%
\pgfsetbuttcap%
\pgfsetroundjoin%
\definecolor{currentfill}{rgb}{0.121569,0.466667,0.705882}%
\pgfsetfillcolor{currentfill}%
\pgfsetfillopacity{0.430639}%
\pgfsetlinewidth{1.003750pt}%
\definecolor{currentstroke}{rgb}{0.121569,0.466667,0.705882}%
\pgfsetstrokecolor{currentstroke}%
\pgfsetstrokeopacity{0.430639}%
\pgfsetdash{}{0pt}%
\pgfpathmoveto{\pgfqpoint{2.014618in}{1.967202in}}%
\pgfpathcurveto{\pgfqpoint{2.022854in}{1.967202in}}{\pgfqpoint{2.030754in}{1.970474in}}{\pgfqpoint{2.036578in}{1.976298in}}%
\pgfpathcurveto{\pgfqpoint{2.042402in}{1.982122in}}{\pgfqpoint{2.045674in}{1.990022in}}{\pgfqpoint{2.045674in}{1.998258in}}%
\pgfpathcurveto{\pgfqpoint{2.045674in}{2.006494in}}{\pgfqpoint{2.042402in}{2.014394in}}{\pgfqpoint{2.036578in}{2.020218in}}%
\pgfpathcurveto{\pgfqpoint{2.030754in}{2.026042in}}{\pgfqpoint{2.022854in}{2.029315in}}{\pgfqpoint{2.014618in}{2.029315in}}%
\pgfpathcurveto{\pgfqpoint{2.006381in}{2.029315in}}{\pgfqpoint{1.998481in}{2.026042in}}{\pgfqpoint{1.992657in}{2.020218in}}%
\pgfpathcurveto{\pgfqpoint{1.986834in}{2.014394in}}{\pgfqpoint{1.983561in}{2.006494in}}{\pgfqpoint{1.983561in}{1.998258in}}%
\pgfpathcurveto{\pgfqpoint{1.983561in}{1.990022in}}{\pgfqpoint{1.986834in}{1.982122in}}{\pgfqpoint{1.992657in}{1.976298in}}%
\pgfpathcurveto{\pgfqpoint{1.998481in}{1.970474in}}{\pgfqpoint{2.006381in}{1.967202in}}{\pgfqpoint{2.014618in}{1.967202in}}%
\pgfpathclose%
\pgfusepath{stroke,fill}%
\end{pgfscope}%
\begin{pgfscope}%
\pgfpathrectangle{\pgfqpoint{0.100000in}{0.212622in}}{\pgfqpoint{3.696000in}{3.696000in}}%
\pgfusepath{clip}%
\pgfsetbuttcap%
\pgfsetroundjoin%
\definecolor{currentfill}{rgb}{0.121569,0.466667,0.705882}%
\pgfsetfillcolor{currentfill}%
\pgfsetfillopacity{0.432050}%
\pgfsetlinewidth{1.003750pt}%
\definecolor{currentstroke}{rgb}{0.121569,0.466667,0.705882}%
\pgfsetstrokecolor{currentstroke}%
\pgfsetstrokeopacity{0.432050}%
\pgfsetdash{}{0pt}%
\pgfpathmoveto{\pgfqpoint{1.505343in}{1.790524in}}%
\pgfpathcurveto{\pgfqpoint{1.513579in}{1.790524in}}{\pgfqpoint{1.521479in}{1.793797in}}{\pgfqpoint{1.527303in}{1.799621in}}%
\pgfpathcurveto{\pgfqpoint{1.533127in}{1.805445in}}{\pgfqpoint{1.536400in}{1.813345in}}{\pgfqpoint{1.536400in}{1.821581in}}%
\pgfpathcurveto{\pgfqpoint{1.536400in}{1.829817in}}{\pgfqpoint{1.533127in}{1.837717in}}{\pgfqpoint{1.527303in}{1.843541in}}%
\pgfpathcurveto{\pgfqpoint{1.521479in}{1.849365in}}{\pgfqpoint{1.513579in}{1.852637in}}{\pgfqpoint{1.505343in}{1.852637in}}%
\pgfpathcurveto{\pgfqpoint{1.497107in}{1.852637in}}{\pgfqpoint{1.489207in}{1.849365in}}{\pgfqpoint{1.483383in}{1.843541in}}%
\pgfpathcurveto{\pgfqpoint{1.477559in}{1.837717in}}{\pgfqpoint{1.474287in}{1.829817in}}{\pgfqpoint{1.474287in}{1.821581in}}%
\pgfpathcurveto{\pgfqpoint{1.474287in}{1.813345in}}{\pgfqpoint{1.477559in}{1.805445in}}{\pgfqpoint{1.483383in}{1.799621in}}%
\pgfpathcurveto{\pgfqpoint{1.489207in}{1.793797in}}{\pgfqpoint{1.497107in}{1.790524in}}{\pgfqpoint{1.505343in}{1.790524in}}%
\pgfpathclose%
\pgfusepath{stroke,fill}%
\end{pgfscope}%
\begin{pgfscope}%
\pgfpathrectangle{\pgfqpoint{0.100000in}{0.212622in}}{\pgfqpoint{3.696000in}{3.696000in}}%
\pgfusepath{clip}%
\pgfsetbuttcap%
\pgfsetroundjoin%
\definecolor{currentfill}{rgb}{0.121569,0.466667,0.705882}%
\pgfsetfillcolor{currentfill}%
\pgfsetfillopacity{0.433490}%
\pgfsetlinewidth{1.003750pt}%
\definecolor{currentstroke}{rgb}{0.121569,0.466667,0.705882}%
\pgfsetstrokecolor{currentstroke}%
\pgfsetstrokeopacity{0.433490}%
\pgfsetdash{}{0pt}%
\pgfpathmoveto{\pgfqpoint{1.500538in}{1.788501in}}%
\pgfpathcurveto{\pgfqpoint{1.508774in}{1.788501in}}{\pgfqpoint{1.516674in}{1.791773in}}{\pgfqpoint{1.522498in}{1.797597in}}%
\pgfpathcurveto{\pgfqpoint{1.528322in}{1.803421in}}{\pgfqpoint{1.531594in}{1.811321in}}{\pgfqpoint{1.531594in}{1.819557in}}%
\pgfpathcurveto{\pgfqpoint{1.531594in}{1.827793in}}{\pgfqpoint{1.528322in}{1.835693in}}{\pgfqpoint{1.522498in}{1.841517in}}%
\pgfpathcurveto{\pgfqpoint{1.516674in}{1.847341in}}{\pgfqpoint{1.508774in}{1.850614in}}{\pgfqpoint{1.500538in}{1.850614in}}%
\pgfpathcurveto{\pgfqpoint{1.492301in}{1.850614in}}{\pgfqpoint{1.484401in}{1.847341in}}{\pgfqpoint{1.478577in}{1.841517in}}%
\pgfpathcurveto{\pgfqpoint{1.472754in}{1.835693in}}{\pgfqpoint{1.469481in}{1.827793in}}{\pgfqpoint{1.469481in}{1.819557in}}%
\pgfpathcurveto{\pgfqpoint{1.469481in}{1.811321in}}{\pgfqpoint{1.472754in}{1.803421in}}{\pgfqpoint{1.478577in}{1.797597in}}%
\pgfpathcurveto{\pgfqpoint{1.484401in}{1.791773in}}{\pgfqpoint{1.492301in}{1.788501in}}{\pgfqpoint{1.500538in}{1.788501in}}%
\pgfpathclose%
\pgfusepath{stroke,fill}%
\end{pgfscope}%
\begin{pgfscope}%
\pgfpathrectangle{\pgfqpoint{0.100000in}{0.212622in}}{\pgfqpoint{3.696000in}{3.696000in}}%
\pgfusepath{clip}%
\pgfsetbuttcap%
\pgfsetroundjoin%
\definecolor{currentfill}{rgb}{0.121569,0.466667,0.705882}%
\pgfsetfillcolor{currentfill}%
\pgfsetfillopacity{0.433787}%
\pgfsetlinewidth{1.003750pt}%
\definecolor{currentstroke}{rgb}{0.121569,0.466667,0.705882}%
\pgfsetstrokecolor{currentstroke}%
\pgfsetstrokeopacity{0.433787}%
\pgfsetdash{}{0pt}%
\pgfpathmoveto{\pgfqpoint{2.016637in}{1.962729in}}%
\pgfpathcurveto{\pgfqpoint{2.024873in}{1.962729in}}{\pgfqpoint{2.032773in}{1.966001in}}{\pgfqpoint{2.038597in}{1.971825in}}%
\pgfpathcurveto{\pgfqpoint{2.044421in}{1.977649in}}{\pgfqpoint{2.047693in}{1.985549in}}{\pgfqpoint{2.047693in}{1.993786in}}%
\pgfpathcurveto{\pgfqpoint{2.047693in}{2.002022in}}{\pgfqpoint{2.044421in}{2.009922in}}{\pgfqpoint{2.038597in}{2.015746in}}%
\pgfpathcurveto{\pgfqpoint{2.032773in}{2.021570in}}{\pgfqpoint{2.024873in}{2.024842in}}{\pgfqpoint{2.016637in}{2.024842in}}%
\pgfpathcurveto{\pgfqpoint{2.008400in}{2.024842in}}{\pgfqpoint{2.000500in}{2.021570in}}{\pgfqpoint{1.994676in}{2.015746in}}%
\pgfpathcurveto{\pgfqpoint{1.988852in}{2.009922in}}{\pgfqpoint{1.985580in}{2.002022in}}{\pgfqpoint{1.985580in}{1.993786in}}%
\pgfpathcurveto{\pgfqpoint{1.985580in}{1.985549in}}{\pgfqpoint{1.988852in}{1.977649in}}{\pgfqpoint{1.994676in}{1.971825in}}%
\pgfpathcurveto{\pgfqpoint{2.000500in}{1.966001in}}{\pgfqpoint{2.008400in}{1.962729in}}{\pgfqpoint{2.016637in}{1.962729in}}%
\pgfpathclose%
\pgfusepath{stroke,fill}%
\end{pgfscope}%
\begin{pgfscope}%
\pgfpathrectangle{\pgfqpoint{0.100000in}{0.212622in}}{\pgfqpoint{3.696000in}{3.696000in}}%
\pgfusepath{clip}%
\pgfsetbuttcap%
\pgfsetroundjoin%
\definecolor{currentfill}{rgb}{0.121569,0.466667,0.705882}%
\pgfsetfillcolor{currentfill}%
\pgfsetfillopacity{0.434719}%
\pgfsetlinewidth{1.003750pt}%
\definecolor{currentstroke}{rgb}{0.121569,0.466667,0.705882}%
\pgfsetstrokecolor{currentstroke}%
\pgfsetstrokeopacity{0.434719}%
\pgfsetdash{}{0pt}%
\pgfpathmoveto{\pgfqpoint{1.496965in}{1.786724in}}%
\pgfpathcurveto{\pgfqpoint{1.505201in}{1.786724in}}{\pgfqpoint{1.513101in}{1.789997in}}{\pgfqpoint{1.518925in}{1.795821in}}%
\pgfpathcurveto{\pgfqpoint{1.524749in}{1.801645in}}{\pgfqpoint{1.528022in}{1.809545in}}{\pgfqpoint{1.528022in}{1.817781in}}%
\pgfpathcurveto{\pgfqpoint{1.528022in}{1.826017in}}{\pgfqpoint{1.524749in}{1.833917in}}{\pgfqpoint{1.518925in}{1.839741in}}%
\pgfpathcurveto{\pgfqpoint{1.513101in}{1.845565in}}{\pgfqpoint{1.505201in}{1.848837in}}{\pgfqpoint{1.496965in}{1.848837in}}%
\pgfpathcurveto{\pgfqpoint{1.488729in}{1.848837in}}{\pgfqpoint{1.480829in}{1.845565in}}{\pgfqpoint{1.475005in}{1.839741in}}%
\pgfpathcurveto{\pgfqpoint{1.469181in}{1.833917in}}{\pgfqpoint{1.465909in}{1.826017in}}{\pgfqpoint{1.465909in}{1.817781in}}%
\pgfpathcurveto{\pgfqpoint{1.465909in}{1.809545in}}{\pgfqpoint{1.469181in}{1.801645in}}{\pgfqpoint{1.475005in}{1.795821in}}%
\pgfpathcurveto{\pgfqpoint{1.480829in}{1.789997in}}{\pgfqpoint{1.488729in}{1.786724in}}{\pgfqpoint{1.496965in}{1.786724in}}%
\pgfpathclose%
\pgfusepath{stroke,fill}%
\end{pgfscope}%
\begin{pgfscope}%
\pgfpathrectangle{\pgfqpoint{0.100000in}{0.212622in}}{\pgfqpoint{3.696000in}{3.696000in}}%
\pgfusepath{clip}%
\pgfsetbuttcap%
\pgfsetroundjoin%
\definecolor{currentfill}{rgb}{0.121569,0.466667,0.705882}%
\pgfsetfillcolor{currentfill}%
\pgfsetfillopacity{0.435565}%
\pgfsetlinewidth{1.003750pt}%
\definecolor{currentstroke}{rgb}{0.121569,0.466667,0.705882}%
\pgfsetstrokecolor{currentstroke}%
\pgfsetstrokeopacity{0.435565}%
\pgfsetdash{}{0pt}%
\pgfpathmoveto{\pgfqpoint{2.017818in}{1.960612in}}%
\pgfpathcurveto{\pgfqpoint{2.026054in}{1.960612in}}{\pgfqpoint{2.033954in}{1.963885in}}{\pgfqpoint{2.039778in}{1.969709in}}%
\pgfpathcurveto{\pgfqpoint{2.045602in}{1.975532in}}{\pgfqpoint{2.048875in}{1.983433in}}{\pgfqpoint{2.048875in}{1.991669in}}%
\pgfpathcurveto{\pgfqpoint{2.048875in}{1.999905in}}{\pgfqpoint{2.045602in}{2.007805in}}{\pgfqpoint{2.039778in}{2.013629in}}%
\pgfpathcurveto{\pgfqpoint{2.033954in}{2.019453in}}{\pgfqpoint{2.026054in}{2.022725in}}{\pgfqpoint{2.017818in}{2.022725in}}%
\pgfpathcurveto{\pgfqpoint{2.009582in}{2.022725in}}{\pgfqpoint{2.001682in}{2.019453in}}{\pgfqpoint{1.995858in}{2.013629in}}%
\pgfpathcurveto{\pgfqpoint{1.990034in}{2.007805in}}{\pgfqpoint{1.986762in}{1.999905in}}{\pgfqpoint{1.986762in}{1.991669in}}%
\pgfpathcurveto{\pgfqpoint{1.986762in}{1.983433in}}{\pgfqpoint{1.990034in}{1.975532in}}{\pgfqpoint{1.995858in}{1.969709in}}%
\pgfpathcurveto{\pgfqpoint{2.001682in}{1.963885in}}{\pgfqpoint{2.009582in}{1.960612in}}{\pgfqpoint{2.017818in}{1.960612in}}%
\pgfpathclose%
\pgfusepath{stroke,fill}%
\end{pgfscope}%
\begin{pgfscope}%
\pgfpathrectangle{\pgfqpoint{0.100000in}{0.212622in}}{\pgfqpoint{3.696000in}{3.696000in}}%
\pgfusepath{clip}%
\pgfsetbuttcap%
\pgfsetroundjoin%
\definecolor{currentfill}{rgb}{0.121569,0.466667,0.705882}%
\pgfsetfillcolor{currentfill}%
\pgfsetfillopacity{0.435641}%
\pgfsetlinewidth{1.003750pt}%
\definecolor{currentstroke}{rgb}{0.121569,0.466667,0.705882}%
\pgfsetstrokecolor{currentstroke}%
\pgfsetstrokeopacity{0.435641}%
\pgfsetdash{}{0pt}%
\pgfpathmoveto{\pgfqpoint{1.493830in}{1.783780in}}%
\pgfpathcurveto{\pgfqpoint{1.502066in}{1.783780in}}{\pgfqpoint{1.509966in}{1.787052in}}{\pgfqpoint{1.515790in}{1.792876in}}%
\pgfpathcurveto{\pgfqpoint{1.521614in}{1.798700in}}{\pgfqpoint{1.524887in}{1.806600in}}{\pgfqpoint{1.524887in}{1.814836in}}%
\pgfpathcurveto{\pgfqpoint{1.524887in}{1.823073in}}{\pgfqpoint{1.521614in}{1.830973in}}{\pgfqpoint{1.515790in}{1.836797in}}%
\pgfpathcurveto{\pgfqpoint{1.509966in}{1.842621in}}{\pgfqpoint{1.502066in}{1.845893in}}{\pgfqpoint{1.493830in}{1.845893in}}%
\pgfpathcurveto{\pgfqpoint{1.485594in}{1.845893in}}{\pgfqpoint{1.477694in}{1.842621in}}{\pgfqpoint{1.471870in}{1.836797in}}%
\pgfpathcurveto{\pgfqpoint{1.466046in}{1.830973in}}{\pgfqpoint{1.462774in}{1.823073in}}{\pgfqpoint{1.462774in}{1.814836in}}%
\pgfpathcurveto{\pgfqpoint{1.462774in}{1.806600in}}{\pgfqpoint{1.466046in}{1.798700in}}{\pgfqpoint{1.471870in}{1.792876in}}%
\pgfpathcurveto{\pgfqpoint{1.477694in}{1.787052in}}{\pgfqpoint{1.485594in}{1.783780in}}{\pgfqpoint{1.493830in}{1.783780in}}%
\pgfpathclose%
\pgfusepath{stroke,fill}%
\end{pgfscope}%
\begin{pgfscope}%
\pgfpathrectangle{\pgfqpoint{0.100000in}{0.212622in}}{\pgfqpoint{3.696000in}{3.696000in}}%
\pgfusepath{clip}%
\pgfsetbuttcap%
\pgfsetroundjoin%
\definecolor{currentfill}{rgb}{0.121569,0.466667,0.705882}%
\pgfsetfillcolor{currentfill}%
\pgfsetfillopacity{0.436447}%
\pgfsetlinewidth{1.003750pt}%
\definecolor{currentstroke}{rgb}{0.121569,0.466667,0.705882}%
\pgfsetstrokecolor{currentstroke}%
\pgfsetstrokeopacity{0.436447}%
\pgfsetdash{}{0pt}%
\pgfpathmoveto{\pgfqpoint{1.491698in}{1.783145in}}%
\pgfpathcurveto{\pgfqpoint{1.499934in}{1.783145in}}{\pgfqpoint{1.507834in}{1.786417in}}{\pgfqpoint{1.513658in}{1.792241in}}%
\pgfpathcurveto{\pgfqpoint{1.519482in}{1.798065in}}{\pgfqpoint{1.522754in}{1.805965in}}{\pgfqpoint{1.522754in}{1.814201in}}%
\pgfpathcurveto{\pgfqpoint{1.522754in}{1.822438in}}{\pgfqpoint{1.519482in}{1.830338in}}{\pgfqpoint{1.513658in}{1.836162in}}%
\pgfpathcurveto{\pgfqpoint{1.507834in}{1.841986in}}{\pgfqpoint{1.499934in}{1.845258in}}{\pgfqpoint{1.491698in}{1.845258in}}%
\pgfpathcurveto{\pgfqpoint{1.483461in}{1.845258in}}{\pgfqpoint{1.475561in}{1.841986in}}{\pgfqpoint{1.469737in}{1.836162in}}%
\pgfpathcurveto{\pgfqpoint{1.463914in}{1.830338in}}{\pgfqpoint{1.460641in}{1.822438in}}{\pgfqpoint{1.460641in}{1.814201in}}%
\pgfpathcurveto{\pgfqpoint{1.460641in}{1.805965in}}{\pgfqpoint{1.463914in}{1.798065in}}{\pgfqpoint{1.469737in}{1.792241in}}%
\pgfpathcurveto{\pgfqpoint{1.475561in}{1.786417in}}{\pgfqpoint{1.483461in}{1.783145in}}{\pgfqpoint{1.491698in}{1.783145in}}%
\pgfpathclose%
\pgfusepath{stroke,fill}%
\end{pgfscope}%
\begin{pgfscope}%
\pgfpathrectangle{\pgfqpoint{0.100000in}{0.212622in}}{\pgfqpoint{3.696000in}{3.696000in}}%
\pgfusepath{clip}%
\pgfsetbuttcap%
\pgfsetroundjoin%
\definecolor{currentfill}{rgb}{0.121569,0.466667,0.705882}%
\pgfsetfillcolor{currentfill}%
\pgfsetfillopacity{0.436934}%
\pgfsetlinewidth{1.003750pt}%
\definecolor{currentstroke}{rgb}{0.121569,0.466667,0.705882}%
\pgfsetstrokecolor{currentstroke}%
\pgfsetstrokeopacity{0.436934}%
\pgfsetdash{}{0pt}%
\pgfpathmoveto{\pgfqpoint{1.490174in}{1.782270in}}%
\pgfpathcurveto{\pgfqpoint{1.498411in}{1.782270in}}{\pgfqpoint{1.506311in}{1.785542in}}{\pgfqpoint{1.512134in}{1.791366in}}%
\pgfpathcurveto{\pgfqpoint{1.517958in}{1.797190in}}{\pgfqpoint{1.521231in}{1.805090in}}{\pgfqpoint{1.521231in}{1.813326in}}%
\pgfpathcurveto{\pgfqpoint{1.521231in}{1.821562in}}{\pgfqpoint{1.517958in}{1.829462in}}{\pgfqpoint{1.512134in}{1.835286in}}%
\pgfpathcurveto{\pgfqpoint{1.506311in}{1.841110in}}{\pgfqpoint{1.498411in}{1.844383in}}{\pgfqpoint{1.490174in}{1.844383in}}%
\pgfpathcurveto{\pgfqpoint{1.481938in}{1.844383in}}{\pgfqpoint{1.474038in}{1.841110in}}{\pgfqpoint{1.468214in}{1.835286in}}%
\pgfpathcurveto{\pgfqpoint{1.462390in}{1.829462in}}{\pgfqpoint{1.459118in}{1.821562in}}{\pgfqpoint{1.459118in}{1.813326in}}%
\pgfpathcurveto{\pgfqpoint{1.459118in}{1.805090in}}{\pgfqpoint{1.462390in}{1.797190in}}{\pgfqpoint{1.468214in}{1.791366in}}%
\pgfpathcurveto{\pgfqpoint{1.474038in}{1.785542in}}{\pgfqpoint{1.481938in}{1.782270in}}{\pgfqpoint{1.490174in}{1.782270in}}%
\pgfpathclose%
\pgfusepath{stroke,fill}%
\end{pgfscope}%
\begin{pgfscope}%
\pgfpathrectangle{\pgfqpoint{0.100000in}{0.212622in}}{\pgfqpoint{3.696000in}{3.696000in}}%
\pgfusepath{clip}%
\pgfsetbuttcap%
\pgfsetroundjoin%
\definecolor{currentfill}{rgb}{0.121569,0.466667,0.705882}%
\pgfsetfillcolor{currentfill}%
\pgfsetfillopacity{0.437718}%
\pgfsetlinewidth{1.003750pt}%
\definecolor{currentstroke}{rgb}{0.121569,0.466667,0.705882}%
\pgfsetstrokecolor{currentstroke}%
\pgfsetstrokeopacity{0.437718}%
\pgfsetdash{}{0pt}%
\pgfpathmoveto{\pgfqpoint{1.487592in}{1.779756in}}%
\pgfpathcurveto{\pgfqpoint{1.495829in}{1.779756in}}{\pgfqpoint{1.503729in}{1.783028in}}{\pgfqpoint{1.509553in}{1.788852in}}%
\pgfpathcurveto{\pgfqpoint{1.515377in}{1.794676in}}{\pgfqpoint{1.518649in}{1.802576in}}{\pgfqpoint{1.518649in}{1.810812in}}%
\pgfpathcurveto{\pgfqpoint{1.518649in}{1.819048in}}{\pgfqpoint{1.515377in}{1.826949in}}{\pgfqpoint{1.509553in}{1.832772in}}%
\pgfpathcurveto{\pgfqpoint{1.503729in}{1.838596in}}{\pgfqpoint{1.495829in}{1.841869in}}{\pgfqpoint{1.487592in}{1.841869in}}%
\pgfpathcurveto{\pgfqpoint{1.479356in}{1.841869in}}{\pgfqpoint{1.471456in}{1.838596in}}{\pgfqpoint{1.465632in}{1.832772in}}%
\pgfpathcurveto{\pgfqpoint{1.459808in}{1.826949in}}{\pgfqpoint{1.456536in}{1.819048in}}{\pgfqpoint{1.456536in}{1.810812in}}%
\pgfpathcurveto{\pgfqpoint{1.456536in}{1.802576in}}{\pgfqpoint{1.459808in}{1.794676in}}{\pgfqpoint{1.465632in}{1.788852in}}%
\pgfpathcurveto{\pgfqpoint{1.471456in}{1.783028in}}{\pgfqpoint{1.479356in}{1.779756in}}{\pgfqpoint{1.487592in}{1.779756in}}%
\pgfpathclose%
\pgfusepath{stroke,fill}%
\end{pgfscope}%
\begin{pgfscope}%
\pgfpathrectangle{\pgfqpoint{0.100000in}{0.212622in}}{\pgfqpoint{3.696000in}{3.696000in}}%
\pgfusepath{clip}%
\pgfsetbuttcap%
\pgfsetroundjoin%
\definecolor{currentfill}{rgb}{0.121569,0.466667,0.705882}%
\pgfsetfillcolor{currentfill}%
\pgfsetfillopacity{0.438061}%
\pgfsetlinewidth{1.003750pt}%
\definecolor{currentstroke}{rgb}{0.121569,0.466667,0.705882}%
\pgfsetstrokecolor{currentstroke}%
\pgfsetstrokeopacity{0.438061}%
\pgfsetdash{}{0pt}%
\pgfpathmoveto{\pgfqpoint{2.018621in}{1.960592in}}%
\pgfpathcurveto{\pgfqpoint{2.026857in}{1.960592in}}{\pgfqpoint{2.034757in}{1.963864in}}{\pgfqpoint{2.040581in}{1.969688in}}%
\pgfpathcurveto{\pgfqpoint{2.046405in}{1.975512in}}{\pgfqpoint{2.049678in}{1.983412in}}{\pgfqpoint{2.049678in}{1.991648in}}%
\pgfpathcurveto{\pgfqpoint{2.049678in}{1.999885in}}{\pgfqpoint{2.046405in}{2.007785in}}{\pgfqpoint{2.040581in}{2.013609in}}%
\pgfpathcurveto{\pgfqpoint{2.034757in}{2.019433in}}{\pgfqpoint{2.026857in}{2.022705in}}{\pgfqpoint{2.018621in}{2.022705in}}%
\pgfpathcurveto{\pgfqpoint{2.010385in}{2.022705in}}{\pgfqpoint{2.002485in}{2.019433in}}{\pgfqpoint{1.996661in}{2.013609in}}%
\pgfpathcurveto{\pgfqpoint{1.990837in}{2.007785in}}{\pgfqpoint{1.987565in}{1.999885in}}{\pgfqpoint{1.987565in}{1.991648in}}%
\pgfpathcurveto{\pgfqpoint{1.987565in}{1.983412in}}{\pgfqpoint{1.990837in}{1.975512in}}{\pgfqpoint{1.996661in}{1.969688in}}%
\pgfpathcurveto{\pgfqpoint{2.002485in}{1.963864in}}{\pgfqpoint{2.010385in}{1.960592in}}{\pgfqpoint{2.018621in}{1.960592in}}%
\pgfpathclose%
\pgfusepath{stroke,fill}%
\end{pgfscope}%
\begin{pgfscope}%
\pgfpathrectangle{\pgfqpoint{0.100000in}{0.212622in}}{\pgfqpoint{3.696000in}{3.696000in}}%
\pgfusepath{clip}%
\pgfsetbuttcap%
\pgfsetroundjoin%
\definecolor{currentfill}{rgb}{0.121569,0.466667,0.705882}%
\pgfsetfillcolor{currentfill}%
\pgfsetfillopacity{0.439787}%
\pgfsetlinewidth{1.003750pt}%
\definecolor{currentstroke}{rgb}{0.121569,0.466667,0.705882}%
\pgfsetstrokecolor{currentstroke}%
\pgfsetstrokeopacity{0.439787}%
\pgfsetdash{}{0pt}%
\pgfpathmoveto{\pgfqpoint{1.482328in}{1.780217in}}%
\pgfpathcurveto{\pgfqpoint{1.490565in}{1.780217in}}{\pgfqpoint{1.498465in}{1.783489in}}{\pgfqpoint{1.504289in}{1.789313in}}%
\pgfpathcurveto{\pgfqpoint{1.510113in}{1.795137in}}{\pgfqpoint{1.513385in}{1.803037in}}{\pgfqpoint{1.513385in}{1.811273in}}%
\pgfpathcurveto{\pgfqpoint{1.513385in}{1.819509in}}{\pgfqpoint{1.510113in}{1.827409in}}{\pgfqpoint{1.504289in}{1.833233in}}%
\pgfpathcurveto{\pgfqpoint{1.498465in}{1.839057in}}{\pgfqpoint{1.490565in}{1.842330in}}{\pgfqpoint{1.482328in}{1.842330in}}%
\pgfpathcurveto{\pgfqpoint{1.474092in}{1.842330in}}{\pgfqpoint{1.466192in}{1.839057in}}{\pgfqpoint{1.460368in}{1.833233in}}%
\pgfpathcurveto{\pgfqpoint{1.454544in}{1.827409in}}{\pgfqpoint{1.451272in}{1.819509in}}{\pgfqpoint{1.451272in}{1.811273in}}%
\pgfpathcurveto{\pgfqpoint{1.451272in}{1.803037in}}{\pgfqpoint{1.454544in}{1.795137in}}{\pgfqpoint{1.460368in}{1.789313in}}%
\pgfpathcurveto{\pgfqpoint{1.466192in}{1.783489in}}{\pgfqpoint{1.474092in}{1.780217in}}{\pgfqpoint{1.482328in}{1.780217in}}%
\pgfpathclose%
\pgfusepath{stroke,fill}%
\end{pgfscope}%
\begin{pgfscope}%
\pgfpathrectangle{\pgfqpoint{0.100000in}{0.212622in}}{\pgfqpoint{3.696000in}{3.696000in}}%
\pgfusepath{clip}%
\pgfsetbuttcap%
\pgfsetroundjoin%
\definecolor{currentfill}{rgb}{0.121569,0.466667,0.705882}%
\pgfsetfillcolor{currentfill}%
\pgfsetfillopacity{0.440774}%
\pgfsetlinewidth{1.003750pt}%
\definecolor{currentstroke}{rgb}{0.121569,0.466667,0.705882}%
\pgfsetstrokecolor{currentstroke}%
\pgfsetstrokeopacity{0.440774}%
\pgfsetdash{}{0pt}%
\pgfpathmoveto{\pgfqpoint{2.019409in}{1.958447in}}%
\pgfpathcurveto{\pgfqpoint{2.027645in}{1.958447in}}{\pgfqpoint{2.035546in}{1.961719in}}{\pgfqpoint{2.041369in}{1.967543in}}%
\pgfpathcurveto{\pgfqpoint{2.047193in}{1.973367in}}{\pgfqpoint{2.050466in}{1.981267in}}{\pgfqpoint{2.050466in}{1.989503in}}%
\pgfpathcurveto{\pgfqpoint{2.050466in}{1.997739in}}{\pgfqpoint{2.047193in}{2.005639in}}{\pgfqpoint{2.041369in}{2.011463in}}%
\pgfpathcurveto{\pgfqpoint{2.035546in}{2.017287in}}{\pgfqpoint{2.027645in}{2.020560in}}{\pgfqpoint{2.019409in}{2.020560in}}%
\pgfpathcurveto{\pgfqpoint{2.011173in}{2.020560in}}{\pgfqpoint{2.003273in}{2.017287in}}{\pgfqpoint{1.997449in}{2.011463in}}%
\pgfpathcurveto{\pgfqpoint{1.991625in}{2.005639in}}{\pgfqpoint{1.988353in}{1.997739in}}{\pgfqpoint{1.988353in}{1.989503in}}%
\pgfpathcurveto{\pgfqpoint{1.988353in}{1.981267in}}{\pgfqpoint{1.991625in}{1.973367in}}{\pgfqpoint{1.997449in}{1.967543in}}%
\pgfpathcurveto{\pgfqpoint{2.003273in}{1.961719in}}{\pgfqpoint{2.011173in}{1.958447in}}{\pgfqpoint{2.019409in}{1.958447in}}%
\pgfpathclose%
\pgfusepath{stroke,fill}%
\end{pgfscope}%
\begin{pgfscope}%
\pgfpathrectangle{\pgfqpoint{0.100000in}{0.212622in}}{\pgfqpoint{3.696000in}{3.696000in}}%
\pgfusepath{clip}%
\pgfsetbuttcap%
\pgfsetroundjoin%
\definecolor{currentfill}{rgb}{0.121569,0.466667,0.705882}%
\pgfsetfillcolor{currentfill}%
\pgfsetfillopacity{0.440930}%
\pgfsetlinewidth{1.003750pt}%
\definecolor{currentstroke}{rgb}{0.121569,0.466667,0.705882}%
\pgfsetstrokecolor{currentstroke}%
\pgfsetstrokeopacity{0.440930}%
\pgfsetdash{}{0pt}%
\pgfpathmoveto{\pgfqpoint{1.478152in}{1.776462in}}%
\pgfpathcurveto{\pgfqpoint{1.486388in}{1.776462in}}{\pgfqpoint{1.494288in}{1.779734in}}{\pgfqpoint{1.500112in}{1.785558in}}%
\pgfpathcurveto{\pgfqpoint{1.505936in}{1.791382in}}{\pgfqpoint{1.509209in}{1.799282in}}{\pgfqpoint{1.509209in}{1.807518in}}%
\pgfpathcurveto{\pgfqpoint{1.509209in}{1.815755in}}{\pgfqpoint{1.505936in}{1.823655in}}{\pgfqpoint{1.500112in}{1.829479in}}%
\pgfpathcurveto{\pgfqpoint{1.494288in}{1.835303in}}{\pgfqpoint{1.486388in}{1.838575in}}{\pgfqpoint{1.478152in}{1.838575in}}%
\pgfpathcurveto{\pgfqpoint{1.469916in}{1.838575in}}{\pgfqpoint{1.462016in}{1.835303in}}{\pgfqpoint{1.456192in}{1.829479in}}%
\pgfpathcurveto{\pgfqpoint{1.450368in}{1.823655in}}{\pgfqpoint{1.447096in}{1.815755in}}{\pgfqpoint{1.447096in}{1.807518in}}%
\pgfpathcurveto{\pgfqpoint{1.447096in}{1.799282in}}{\pgfqpoint{1.450368in}{1.791382in}}{\pgfqpoint{1.456192in}{1.785558in}}%
\pgfpathcurveto{\pgfqpoint{1.462016in}{1.779734in}}{\pgfqpoint{1.469916in}{1.776462in}}{\pgfqpoint{1.478152in}{1.776462in}}%
\pgfpathclose%
\pgfusepath{stroke,fill}%
\end{pgfscope}%
\begin{pgfscope}%
\pgfpathrectangle{\pgfqpoint{0.100000in}{0.212622in}}{\pgfqpoint{3.696000in}{3.696000in}}%
\pgfusepath{clip}%
\pgfsetbuttcap%
\pgfsetroundjoin%
\definecolor{currentfill}{rgb}{0.121569,0.466667,0.705882}%
\pgfsetfillcolor{currentfill}%
\pgfsetfillopacity{0.442231}%
\pgfsetlinewidth{1.003750pt}%
\definecolor{currentstroke}{rgb}{0.121569,0.466667,0.705882}%
\pgfsetstrokecolor{currentstroke}%
\pgfsetstrokeopacity{0.442231}%
\pgfsetdash{}{0pt}%
\pgfpathmoveto{\pgfqpoint{2.019682in}{1.956975in}}%
\pgfpathcurveto{\pgfqpoint{2.027918in}{1.956975in}}{\pgfqpoint{2.035818in}{1.960247in}}{\pgfqpoint{2.041642in}{1.966071in}}%
\pgfpathcurveto{\pgfqpoint{2.047466in}{1.971895in}}{\pgfqpoint{2.050738in}{1.979795in}}{\pgfqpoint{2.050738in}{1.988031in}}%
\pgfpathcurveto{\pgfqpoint{2.050738in}{1.996268in}}{\pgfqpoint{2.047466in}{2.004168in}}{\pgfqpoint{2.041642in}{2.009992in}}%
\pgfpathcurveto{\pgfqpoint{2.035818in}{2.015816in}}{\pgfqpoint{2.027918in}{2.019088in}}{\pgfqpoint{2.019682in}{2.019088in}}%
\pgfpathcurveto{\pgfqpoint{2.011446in}{2.019088in}}{\pgfqpoint{2.003546in}{2.015816in}}{\pgfqpoint{1.997722in}{2.009992in}}%
\pgfpathcurveto{\pgfqpoint{1.991898in}{2.004168in}}{\pgfqpoint{1.988625in}{1.996268in}}{\pgfqpoint{1.988625in}{1.988031in}}%
\pgfpathcurveto{\pgfqpoint{1.988625in}{1.979795in}}{\pgfqpoint{1.991898in}{1.971895in}}{\pgfqpoint{1.997722in}{1.966071in}}%
\pgfpathcurveto{\pgfqpoint{2.003546in}{1.960247in}}{\pgfqpoint{2.011446in}{1.956975in}}{\pgfqpoint{2.019682in}{1.956975in}}%
\pgfpathclose%
\pgfusepath{stroke,fill}%
\end{pgfscope}%
\begin{pgfscope}%
\pgfpathrectangle{\pgfqpoint{0.100000in}{0.212622in}}{\pgfqpoint{3.696000in}{3.696000in}}%
\pgfusepath{clip}%
\pgfsetbuttcap%
\pgfsetroundjoin%
\definecolor{currentfill}{rgb}{0.121569,0.466667,0.705882}%
\pgfsetfillcolor{currentfill}%
\pgfsetfillopacity{0.442602}%
\pgfsetlinewidth{1.003750pt}%
\definecolor{currentstroke}{rgb}{0.121569,0.466667,0.705882}%
\pgfsetstrokecolor{currentstroke}%
\pgfsetstrokeopacity{0.442602}%
\pgfsetdash{}{0pt}%
\pgfpathmoveto{\pgfqpoint{1.474387in}{1.778324in}}%
\pgfpathcurveto{\pgfqpoint{1.482624in}{1.778324in}}{\pgfqpoint{1.490524in}{1.781597in}}{\pgfqpoint{1.496348in}{1.787421in}}%
\pgfpathcurveto{\pgfqpoint{1.502172in}{1.793245in}}{\pgfqpoint{1.505444in}{1.801145in}}{\pgfqpoint{1.505444in}{1.809381in}}%
\pgfpathcurveto{\pgfqpoint{1.505444in}{1.817617in}}{\pgfqpoint{1.502172in}{1.825517in}}{\pgfqpoint{1.496348in}{1.831341in}}%
\pgfpathcurveto{\pgfqpoint{1.490524in}{1.837165in}}{\pgfqpoint{1.482624in}{1.840437in}}{\pgfqpoint{1.474387in}{1.840437in}}%
\pgfpathcurveto{\pgfqpoint{1.466151in}{1.840437in}}{\pgfqpoint{1.458251in}{1.837165in}}{\pgfqpoint{1.452427in}{1.831341in}}%
\pgfpathcurveto{\pgfqpoint{1.446603in}{1.825517in}}{\pgfqpoint{1.443331in}{1.817617in}}{\pgfqpoint{1.443331in}{1.809381in}}%
\pgfpathcurveto{\pgfqpoint{1.443331in}{1.801145in}}{\pgfqpoint{1.446603in}{1.793245in}}{\pgfqpoint{1.452427in}{1.787421in}}%
\pgfpathcurveto{\pgfqpoint{1.458251in}{1.781597in}}{\pgfqpoint{1.466151in}{1.778324in}}{\pgfqpoint{1.474387in}{1.778324in}}%
\pgfpathclose%
\pgfusepath{stroke,fill}%
\end{pgfscope}%
\begin{pgfscope}%
\pgfpathrectangle{\pgfqpoint{0.100000in}{0.212622in}}{\pgfqpoint{3.696000in}{3.696000in}}%
\pgfusepath{clip}%
\pgfsetbuttcap%
\pgfsetroundjoin%
\definecolor{currentfill}{rgb}{0.121569,0.466667,0.705882}%
\pgfsetfillcolor{currentfill}%
\pgfsetfillopacity{0.444332}%
\pgfsetlinewidth{1.003750pt}%
\definecolor{currentstroke}{rgb}{0.121569,0.466667,0.705882}%
\pgfsetstrokecolor{currentstroke}%
\pgfsetstrokeopacity{0.444332}%
\pgfsetdash{}{0pt}%
\pgfpathmoveto{\pgfqpoint{2.020614in}{1.955425in}}%
\pgfpathcurveto{\pgfqpoint{2.028850in}{1.955425in}}{\pgfqpoint{2.036750in}{1.958698in}}{\pgfqpoint{2.042574in}{1.964521in}}%
\pgfpathcurveto{\pgfqpoint{2.048398in}{1.970345in}}{\pgfqpoint{2.051670in}{1.978245in}}{\pgfqpoint{2.051670in}{1.986482in}}%
\pgfpathcurveto{\pgfqpoint{2.051670in}{1.994718in}}{\pgfqpoint{2.048398in}{2.002618in}}{\pgfqpoint{2.042574in}{2.008442in}}%
\pgfpathcurveto{\pgfqpoint{2.036750in}{2.014266in}}{\pgfqpoint{2.028850in}{2.017538in}}{\pgfqpoint{2.020614in}{2.017538in}}%
\pgfpathcurveto{\pgfqpoint{2.012378in}{2.017538in}}{\pgfqpoint{2.004478in}{2.014266in}}{\pgfqpoint{1.998654in}{2.008442in}}%
\pgfpathcurveto{\pgfqpoint{1.992830in}{2.002618in}}{\pgfqpoint{1.989557in}{1.994718in}}{\pgfqpoint{1.989557in}{1.986482in}}%
\pgfpathcurveto{\pgfqpoint{1.989557in}{1.978245in}}{\pgfqpoint{1.992830in}{1.970345in}}{\pgfqpoint{1.998654in}{1.964521in}}%
\pgfpathcurveto{\pgfqpoint{2.004478in}{1.958698in}}{\pgfqpoint{2.012378in}{1.955425in}}{\pgfqpoint{2.020614in}{1.955425in}}%
\pgfpathclose%
\pgfusepath{stroke,fill}%
\end{pgfscope}%
\begin{pgfscope}%
\pgfpathrectangle{\pgfqpoint{0.100000in}{0.212622in}}{\pgfqpoint{3.696000in}{3.696000in}}%
\pgfusepath{clip}%
\pgfsetbuttcap%
\pgfsetroundjoin%
\definecolor{currentfill}{rgb}{0.121569,0.466667,0.705882}%
\pgfsetfillcolor{currentfill}%
\pgfsetfillopacity{0.445147}%
\pgfsetlinewidth{1.003750pt}%
\definecolor{currentstroke}{rgb}{0.121569,0.466667,0.705882}%
\pgfsetstrokecolor{currentstroke}%
\pgfsetstrokeopacity{0.445147}%
\pgfsetdash{}{0pt}%
\pgfpathmoveto{\pgfqpoint{1.467767in}{1.778098in}}%
\pgfpathcurveto{\pgfqpoint{1.476003in}{1.778098in}}{\pgfqpoint{1.483903in}{1.781371in}}{\pgfqpoint{1.489727in}{1.787195in}}%
\pgfpathcurveto{\pgfqpoint{1.495551in}{1.793018in}}{\pgfqpoint{1.498823in}{1.800918in}}{\pgfqpoint{1.498823in}{1.809155in}}%
\pgfpathcurveto{\pgfqpoint{1.498823in}{1.817391in}}{\pgfqpoint{1.495551in}{1.825291in}}{\pgfqpoint{1.489727in}{1.831115in}}%
\pgfpathcurveto{\pgfqpoint{1.483903in}{1.836939in}}{\pgfqpoint{1.476003in}{1.840211in}}{\pgfqpoint{1.467767in}{1.840211in}}%
\pgfpathcurveto{\pgfqpoint{1.459530in}{1.840211in}}{\pgfqpoint{1.451630in}{1.836939in}}{\pgfqpoint{1.445806in}{1.831115in}}%
\pgfpathcurveto{\pgfqpoint{1.439982in}{1.825291in}}{\pgfqpoint{1.436710in}{1.817391in}}{\pgfqpoint{1.436710in}{1.809155in}}%
\pgfpathcurveto{\pgfqpoint{1.436710in}{1.800918in}}{\pgfqpoint{1.439982in}{1.793018in}}{\pgfqpoint{1.445806in}{1.787195in}}%
\pgfpathcurveto{\pgfqpoint{1.451630in}{1.781371in}}{\pgfqpoint{1.459530in}{1.778098in}}{\pgfqpoint{1.467767in}{1.778098in}}%
\pgfpathclose%
\pgfusepath{stroke,fill}%
\end{pgfscope}%
\begin{pgfscope}%
\pgfpathrectangle{\pgfqpoint{0.100000in}{0.212622in}}{\pgfqpoint{3.696000in}{3.696000in}}%
\pgfusepath{clip}%
\pgfsetbuttcap%
\pgfsetroundjoin%
\definecolor{currentfill}{rgb}{0.121569,0.466667,0.705882}%
\pgfsetfillcolor{currentfill}%
\pgfsetfillopacity{0.445500}%
\pgfsetlinewidth{1.003750pt}%
\definecolor{currentstroke}{rgb}{0.121569,0.466667,0.705882}%
\pgfsetstrokecolor{currentstroke}%
\pgfsetstrokeopacity{0.445500}%
\pgfsetdash{}{0pt}%
\pgfpathmoveto{\pgfqpoint{2.021164in}{1.954664in}}%
\pgfpathcurveto{\pgfqpoint{2.029401in}{1.954664in}}{\pgfqpoint{2.037301in}{1.957936in}}{\pgfqpoint{2.043125in}{1.963760in}}%
\pgfpathcurveto{\pgfqpoint{2.048949in}{1.969584in}}{\pgfqpoint{2.052221in}{1.977484in}}{\pgfqpoint{2.052221in}{1.985720in}}%
\pgfpathcurveto{\pgfqpoint{2.052221in}{1.993957in}}{\pgfqpoint{2.048949in}{2.001857in}}{\pgfqpoint{2.043125in}{2.007681in}}%
\pgfpathcurveto{\pgfqpoint{2.037301in}{2.013505in}}{\pgfqpoint{2.029401in}{2.016777in}}{\pgfqpoint{2.021164in}{2.016777in}}%
\pgfpathcurveto{\pgfqpoint{2.012928in}{2.016777in}}{\pgfqpoint{2.005028in}{2.013505in}}{\pgfqpoint{1.999204in}{2.007681in}}%
\pgfpathcurveto{\pgfqpoint{1.993380in}{2.001857in}}{\pgfqpoint{1.990108in}{1.993957in}}{\pgfqpoint{1.990108in}{1.985720in}}%
\pgfpathcurveto{\pgfqpoint{1.990108in}{1.977484in}}{\pgfqpoint{1.993380in}{1.969584in}}{\pgfqpoint{1.999204in}{1.963760in}}%
\pgfpathcurveto{\pgfqpoint{2.005028in}{1.957936in}}{\pgfqpoint{2.012928in}{1.954664in}}{\pgfqpoint{2.021164in}{1.954664in}}%
\pgfpathclose%
\pgfusepath{stroke,fill}%
\end{pgfscope}%
\begin{pgfscope}%
\pgfpathrectangle{\pgfqpoint{0.100000in}{0.212622in}}{\pgfqpoint{3.696000in}{3.696000in}}%
\pgfusepath{clip}%
\pgfsetbuttcap%
\pgfsetroundjoin%
\definecolor{currentfill}{rgb}{0.121569,0.466667,0.705882}%
\pgfsetfillcolor{currentfill}%
\pgfsetfillopacity{0.446124}%
\pgfsetlinewidth{1.003750pt}%
\definecolor{currentstroke}{rgb}{0.121569,0.466667,0.705882}%
\pgfsetstrokecolor{currentstroke}%
\pgfsetstrokeopacity{0.446124}%
\pgfsetdash{}{0pt}%
\pgfpathmoveto{\pgfqpoint{2.021466in}{1.954122in}}%
\pgfpathcurveto{\pgfqpoint{2.029702in}{1.954122in}}{\pgfqpoint{2.037602in}{1.957395in}}{\pgfqpoint{2.043426in}{1.963219in}}%
\pgfpathcurveto{\pgfqpoint{2.049250in}{1.969042in}}{\pgfqpoint{2.052522in}{1.976943in}}{\pgfqpoint{2.052522in}{1.985179in}}%
\pgfpathcurveto{\pgfqpoint{2.052522in}{1.993415in}}{\pgfqpoint{2.049250in}{2.001315in}}{\pgfqpoint{2.043426in}{2.007139in}}%
\pgfpathcurveto{\pgfqpoint{2.037602in}{2.012963in}}{\pgfqpoint{2.029702in}{2.016235in}}{\pgfqpoint{2.021466in}{2.016235in}}%
\pgfpathcurveto{\pgfqpoint{2.013229in}{2.016235in}}{\pgfqpoint{2.005329in}{2.012963in}}{\pgfqpoint{1.999505in}{2.007139in}}%
\pgfpathcurveto{\pgfqpoint{1.993681in}{2.001315in}}{\pgfqpoint{1.990409in}{1.993415in}}{\pgfqpoint{1.990409in}{1.985179in}}%
\pgfpathcurveto{\pgfqpoint{1.990409in}{1.976943in}}{\pgfqpoint{1.993681in}{1.969042in}}{\pgfqpoint{1.999505in}{1.963219in}}%
\pgfpathcurveto{\pgfqpoint{2.005329in}{1.957395in}}{\pgfqpoint{2.013229in}{1.954122in}}{\pgfqpoint{2.021466in}{1.954122in}}%
\pgfpathclose%
\pgfusepath{stroke,fill}%
\end{pgfscope}%
\begin{pgfscope}%
\pgfpathrectangle{\pgfqpoint{0.100000in}{0.212622in}}{\pgfqpoint{3.696000in}{3.696000in}}%
\pgfusepath{clip}%
\pgfsetbuttcap%
\pgfsetroundjoin%
\definecolor{currentfill}{rgb}{0.121569,0.466667,0.705882}%
\pgfsetfillcolor{currentfill}%
\pgfsetfillopacity{0.446466}%
\pgfsetlinewidth{1.003750pt}%
\definecolor{currentstroke}{rgb}{0.121569,0.466667,0.705882}%
\pgfsetstrokecolor{currentstroke}%
\pgfsetstrokeopacity{0.446466}%
\pgfsetdash{}{0pt}%
\pgfpathmoveto{\pgfqpoint{2.021652in}{1.953822in}}%
\pgfpathcurveto{\pgfqpoint{2.029888in}{1.953822in}}{\pgfqpoint{2.037788in}{1.957094in}}{\pgfqpoint{2.043612in}{1.962918in}}%
\pgfpathcurveto{\pgfqpoint{2.049436in}{1.968742in}}{\pgfqpoint{2.052708in}{1.976642in}}{\pgfqpoint{2.052708in}{1.984878in}}%
\pgfpathcurveto{\pgfqpoint{2.052708in}{1.993115in}}{\pgfqpoint{2.049436in}{2.001015in}}{\pgfqpoint{2.043612in}{2.006839in}}%
\pgfpathcurveto{\pgfqpoint{2.037788in}{2.012663in}}{\pgfqpoint{2.029888in}{2.015935in}}{\pgfqpoint{2.021652in}{2.015935in}}%
\pgfpathcurveto{\pgfqpoint{2.013415in}{2.015935in}}{\pgfqpoint{2.005515in}{2.012663in}}{\pgfqpoint{1.999691in}{2.006839in}}%
\pgfpathcurveto{\pgfqpoint{1.993867in}{2.001015in}}{\pgfqpoint{1.990595in}{1.993115in}}{\pgfqpoint{1.990595in}{1.984878in}}%
\pgfpathcurveto{\pgfqpoint{1.990595in}{1.976642in}}{\pgfqpoint{1.993867in}{1.968742in}}{\pgfqpoint{1.999691in}{1.962918in}}%
\pgfpathcurveto{\pgfqpoint{2.005515in}{1.957094in}}{\pgfqpoint{2.013415in}{1.953822in}}{\pgfqpoint{2.021652in}{1.953822in}}%
\pgfpathclose%
\pgfusepath{stroke,fill}%
\end{pgfscope}%
\begin{pgfscope}%
\pgfpathrectangle{\pgfqpoint{0.100000in}{0.212622in}}{\pgfqpoint{3.696000in}{3.696000in}}%
\pgfusepath{clip}%
\pgfsetbuttcap%
\pgfsetroundjoin%
\definecolor{currentfill}{rgb}{0.121569,0.466667,0.705882}%
\pgfsetfillcolor{currentfill}%
\pgfsetfillopacity{0.446546}%
\pgfsetlinewidth{1.003750pt}%
\definecolor{currentstroke}{rgb}{0.121569,0.466667,0.705882}%
\pgfsetstrokecolor{currentstroke}%
\pgfsetstrokeopacity{0.446546}%
\pgfsetdash{}{0pt}%
\pgfpathmoveto{\pgfqpoint{1.463012in}{1.774130in}}%
\pgfpathcurveto{\pgfqpoint{1.471248in}{1.774130in}}{\pgfqpoint{1.479148in}{1.777402in}}{\pgfqpoint{1.484972in}{1.783226in}}%
\pgfpathcurveto{\pgfqpoint{1.490796in}{1.789050in}}{\pgfqpoint{1.494068in}{1.796950in}}{\pgfqpoint{1.494068in}{1.805186in}}%
\pgfpathcurveto{\pgfqpoint{1.494068in}{1.813422in}}{\pgfqpoint{1.490796in}{1.821323in}}{\pgfqpoint{1.484972in}{1.827146in}}%
\pgfpathcurveto{\pgfqpoint{1.479148in}{1.832970in}}{\pgfqpoint{1.471248in}{1.836243in}}{\pgfqpoint{1.463012in}{1.836243in}}%
\pgfpathcurveto{\pgfqpoint{1.454775in}{1.836243in}}{\pgfqpoint{1.446875in}{1.832970in}}{\pgfqpoint{1.441051in}{1.827146in}}%
\pgfpathcurveto{\pgfqpoint{1.435228in}{1.821323in}}{\pgfqpoint{1.431955in}{1.813422in}}{\pgfqpoint{1.431955in}{1.805186in}}%
\pgfpathcurveto{\pgfqpoint{1.431955in}{1.796950in}}{\pgfqpoint{1.435228in}{1.789050in}}{\pgfqpoint{1.441051in}{1.783226in}}%
\pgfpathcurveto{\pgfqpoint{1.446875in}{1.777402in}}{\pgfqpoint{1.454775in}{1.774130in}}{\pgfqpoint{1.463012in}{1.774130in}}%
\pgfpathclose%
\pgfusepath{stroke,fill}%
\end{pgfscope}%
\begin{pgfscope}%
\pgfpathrectangle{\pgfqpoint{0.100000in}{0.212622in}}{\pgfqpoint{3.696000in}{3.696000in}}%
\pgfusepath{clip}%
\pgfsetbuttcap%
\pgfsetroundjoin%
\definecolor{currentfill}{rgb}{0.121569,0.466667,0.705882}%
\pgfsetfillcolor{currentfill}%
\pgfsetfillopacity{0.446690}%
\pgfsetlinewidth{1.003750pt}%
\definecolor{currentstroke}{rgb}{0.121569,0.466667,0.705882}%
\pgfsetstrokecolor{currentstroke}%
\pgfsetstrokeopacity{0.446690}%
\pgfsetdash{}{0pt}%
\pgfpathmoveto{\pgfqpoint{2.021692in}{1.953882in}}%
\pgfpathcurveto{\pgfqpoint{2.029929in}{1.953882in}}{\pgfqpoint{2.037829in}{1.957155in}}{\pgfqpoint{2.043652in}{1.962979in}}%
\pgfpathcurveto{\pgfqpoint{2.049476in}{1.968802in}}{\pgfqpoint{2.052749in}{1.976703in}}{\pgfqpoint{2.052749in}{1.984939in}}%
\pgfpathcurveto{\pgfqpoint{2.052749in}{1.993175in}}{\pgfqpoint{2.049476in}{2.001075in}}{\pgfqpoint{2.043652in}{2.006899in}}%
\pgfpathcurveto{\pgfqpoint{2.037829in}{2.012723in}}{\pgfqpoint{2.029929in}{2.015995in}}{\pgfqpoint{2.021692in}{2.015995in}}%
\pgfpathcurveto{\pgfqpoint{2.013456in}{2.015995in}}{\pgfqpoint{2.005556in}{2.012723in}}{\pgfqpoint{1.999732in}{2.006899in}}%
\pgfpathcurveto{\pgfqpoint{1.993908in}{2.001075in}}{\pgfqpoint{1.990636in}{1.993175in}}{\pgfqpoint{1.990636in}{1.984939in}}%
\pgfpathcurveto{\pgfqpoint{1.990636in}{1.976703in}}{\pgfqpoint{1.993908in}{1.968802in}}{\pgfqpoint{1.999732in}{1.962979in}}%
\pgfpathcurveto{\pgfqpoint{2.005556in}{1.957155in}}{\pgfqpoint{2.013456in}{1.953882in}}{\pgfqpoint{2.021692in}{1.953882in}}%
\pgfpathclose%
\pgfusepath{stroke,fill}%
\end{pgfscope}%
\begin{pgfscope}%
\pgfpathrectangle{\pgfqpoint{0.100000in}{0.212622in}}{\pgfqpoint{3.696000in}{3.696000in}}%
\pgfusepath{clip}%
\pgfsetbuttcap%
\pgfsetroundjoin%
\definecolor{currentfill}{rgb}{0.121569,0.466667,0.705882}%
\pgfsetfillcolor{currentfill}%
\pgfsetfillopacity{0.447295}%
\pgfsetlinewidth{1.003750pt}%
\definecolor{currentstroke}{rgb}{0.121569,0.466667,0.705882}%
\pgfsetstrokecolor{currentstroke}%
\pgfsetstrokeopacity{0.447295}%
\pgfsetdash{}{0pt}%
\pgfpathmoveto{\pgfqpoint{2.022112in}{1.953348in}}%
\pgfpathcurveto{\pgfqpoint{2.030348in}{1.953348in}}{\pgfqpoint{2.038248in}{1.956620in}}{\pgfqpoint{2.044072in}{1.962444in}}%
\pgfpathcurveto{\pgfqpoint{2.049896in}{1.968268in}}{\pgfqpoint{2.053168in}{1.976168in}}{\pgfqpoint{2.053168in}{1.984405in}}%
\pgfpathcurveto{\pgfqpoint{2.053168in}{1.992641in}}{\pgfqpoint{2.049896in}{2.000541in}}{\pgfqpoint{2.044072in}{2.006365in}}%
\pgfpathcurveto{\pgfqpoint{2.038248in}{2.012189in}}{\pgfqpoint{2.030348in}{2.015461in}}{\pgfqpoint{2.022112in}{2.015461in}}%
\pgfpathcurveto{\pgfqpoint{2.013875in}{2.015461in}}{\pgfqpoint{2.005975in}{2.012189in}}{\pgfqpoint{2.000151in}{2.006365in}}%
\pgfpathcurveto{\pgfqpoint{1.994328in}{2.000541in}}{\pgfqpoint{1.991055in}{1.992641in}}{\pgfqpoint{1.991055in}{1.984405in}}%
\pgfpathcurveto{\pgfqpoint{1.991055in}{1.976168in}}{\pgfqpoint{1.994328in}{1.968268in}}{\pgfqpoint{2.000151in}{1.962444in}}%
\pgfpathcurveto{\pgfqpoint{2.005975in}{1.956620in}}{\pgfqpoint{2.013875in}{1.953348in}}{\pgfqpoint{2.022112in}{1.953348in}}%
\pgfpathclose%
\pgfusepath{stroke,fill}%
\end{pgfscope}%
\begin{pgfscope}%
\pgfpathrectangle{\pgfqpoint{0.100000in}{0.212622in}}{\pgfqpoint{3.696000in}{3.696000in}}%
\pgfusepath{clip}%
\pgfsetbuttcap%
\pgfsetroundjoin%
\definecolor{currentfill}{rgb}{0.121569,0.466667,0.705882}%
\pgfsetfillcolor{currentfill}%
\pgfsetfillopacity{0.447594}%
\pgfsetlinewidth{1.003750pt}%
\definecolor{currentstroke}{rgb}{0.121569,0.466667,0.705882}%
\pgfsetstrokecolor{currentstroke}%
\pgfsetstrokeopacity{0.447594}%
\pgfsetdash{}{0pt}%
\pgfpathmoveto{\pgfqpoint{2.022319in}{1.952813in}}%
\pgfpathcurveto{\pgfqpoint{2.030556in}{1.952813in}}{\pgfqpoint{2.038456in}{1.956085in}}{\pgfqpoint{2.044280in}{1.961909in}}%
\pgfpathcurveto{\pgfqpoint{2.050104in}{1.967733in}}{\pgfqpoint{2.053376in}{1.975633in}}{\pgfqpoint{2.053376in}{1.983869in}}%
\pgfpathcurveto{\pgfqpoint{2.053376in}{1.992106in}}{\pgfqpoint{2.050104in}{2.000006in}}{\pgfqpoint{2.044280in}{2.005830in}}%
\pgfpathcurveto{\pgfqpoint{2.038456in}{2.011654in}}{\pgfqpoint{2.030556in}{2.014926in}}{\pgfqpoint{2.022319in}{2.014926in}}%
\pgfpathcurveto{\pgfqpoint{2.014083in}{2.014926in}}{\pgfqpoint{2.006183in}{2.011654in}}{\pgfqpoint{2.000359in}{2.005830in}}%
\pgfpathcurveto{\pgfqpoint{1.994535in}{2.000006in}}{\pgfqpoint{1.991263in}{1.992106in}}{\pgfqpoint{1.991263in}{1.983869in}}%
\pgfpathcurveto{\pgfqpoint{1.991263in}{1.975633in}}{\pgfqpoint{1.994535in}{1.967733in}}{\pgfqpoint{2.000359in}{1.961909in}}%
\pgfpathcurveto{\pgfqpoint{2.006183in}{1.956085in}}{\pgfqpoint{2.014083in}{1.952813in}}{\pgfqpoint{2.022319in}{1.952813in}}%
\pgfpathclose%
\pgfusepath{stroke,fill}%
\end{pgfscope}%
\begin{pgfscope}%
\pgfpathrectangle{\pgfqpoint{0.100000in}{0.212622in}}{\pgfqpoint{3.696000in}{3.696000in}}%
\pgfusepath{clip}%
\pgfsetbuttcap%
\pgfsetroundjoin%
\definecolor{currentfill}{rgb}{0.121569,0.466667,0.705882}%
\pgfsetfillcolor{currentfill}%
\pgfsetfillopacity{0.448604}%
\pgfsetlinewidth{1.003750pt}%
\definecolor{currentstroke}{rgb}{0.121569,0.466667,0.705882}%
\pgfsetstrokecolor{currentstroke}%
\pgfsetstrokeopacity{0.448604}%
\pgfsetdash{}{0pt}%
\pgfpathmoveto{\pgfqpoint{2.022901in}{1.952320in}}%
\pgfpathcurveto{\pgfqpoint{2.031138in}{1.952320in}}{\pgfqpoint{2.039038in}{1.955593in}}{\pgfqpoint{2.044862in}{1.961416in}}%
\pgfpathcurveto{\pgfqpoint{2.050686in}{1.967240in}}{\pgfqpoint{2.053958in}{1.975140in}}{\pgfqpoint{2.053958in}{1.983377in}}%
\pgfpathcurveto{\pgfqpoint{2.053958in}{1.991613in}}{\pgfqpoint{2.050686in}{1.999513in}}{\pgfqpoint{2.044862in}{2.005337in}}%
\pgfpathcurveto{\pgfqpoint{2.039038in}{2.011161in}}{\pgfqpoint{2.031138in}{2.014433in}}{\pgfqpoint{2.022901in}{2.014433in}}%
\pgfpathcurveto{\pgfqpoint{2.014665in}{2.014433in}}{\pgfqpoint{2.006765in}{2.011161in}}{\pgfqpoint{2.000941in}{2.005337in}}%
\pgfpathcurveto{\pgfqpoint{1.995117in}{1.999513in}}{\pgfqpoint{1.991845in}{1.991613in}}{\pgfqpoint{1.991845in}{1.983377in}}%
\pgfpathcurveto{\pgfqpoint{1.991845in}{1.975140in}}{\pgfqpoint{1.995117in}{1.967240in}}{\pgfqpoint{2.000941in}{1.961416in}}%
\pgfpathcurveto{\pgfqpoint{2.006765in}{1.955593in}}{\pgfqpoint{2.014665in}{1.952320in}}{\pgfqpoint{2.022901in}{1.952320in}}%
\pgfpathclose%
\pgfusepath{stroke,fill}%
\end{pgfscope}%
\begin{pgfscope}%
\pgfpathrectangle{\pgfqpoint{0.100000in}{0.212622in}}{\pgfqpoint{3.696000in}{3.696000in}}%
\pgfusepath{clip}%
\pgfsetbuttcap%
\pgfsetroundjoin%
\definecolor{currentfill}{rgb}{0.121569,0.466667,0.705882}%
\pgfsetfillcolor{currentfill}%
\pgfsetfillopacity{0.449842}%
\pgfsetlinewidth{1.003750pt}%
\definecolor{currentstroke}{rgb}{0.121569,0.466667,0.705882}%
\pgfsetstrokecolor{currentstroke}%
\pgfsetstrokeopacity{0.449842}%
\pgfsetdash{}{0pt}%
\pgfpathmoveto{\pgfqpoint{2.023575in}{1.951909in}}%
\pgfpathcurveto{\pgfqpoint{2.031812in}{1.951909in}}{\pgfqpoint{2.039712in}{1.955181in}}{\pgfqpoint{2.045536in}{1.961005in}}%
\pgfpathcurveto{\pgfqpoint{2.051360in}{1.966829in}}{\pgfqpoint{2.054632in}{1.974729in}}{\pgfqpoint{2.054632in}{1.982965in}}%
\pgfpathcurveto{\pgfqpoint{2.054632in}{1.991202in}}{\pgfqpoint{2.051360in}{1.999102in}}{\pgfqpoint{2.045536in}{2.004926in}}%
\pgfpathcurveto{\pgfqpoint{2.039712in}{2.010749in}}{\pgfqpoint{2.031812in}{2.014022in}}{\pgfqpoint{2.023575in}{2.014022in}}%
\pgfpathcurveto{\pgfqpoint{2.015339in}{2.014022in}}{\pgfqpoint{2.007439in}{2.010749in}}{\pgfqpoint{2.001615in}{2.004926in}}%
\pgfpathcurveto{\pgfqpoint{1.995791in}{1.999102in}}{\pgfqpoint{1.992519in}{1.991202in}}{\pgfqpoint{1.992519in}{1.982965in}}%
\pgfpathcurveto{\pgfqpoint{1.992519in}{1.974729in}}{\pgfqpoint{1.995791in}{1.966829in}}{\pgfqpoint{2.001615in}{1.961005in}}%
\pgfpathcurveto{\pgfqpoint{2.007439in}{1.955181in}}{\pgfqpoint{2.015339in}{1.951909in}}{\pgfqpoint{2.023575in}{1.951909in}}%
\pgfpathclose%
\pgfusepath{stroke,fill}%
\end{pgfscope}%
\begin{pgfscope}%
\pgfpathrectangle{\pgfqpoint{0.100000in}{0.212622in}}{\pgfqpoint{3.696000in}{3.696000in}}%
\pgfusepath{clip}%
\pgfsetbuttcap%
\pgfsetroundjoin%
\definecolor{currentfill}{rgb}{0.121569,0.466667,0.705882}%
\pgfsetfillcolor{currentfill}%
\pgfsetfillopacity{0.449938}%
\pgfsetlinewidth{1.003750pt}%
\definecolor{currentstroke}{rgb}{0.121569,0.466667,0.705882}%
\pgfsetstrokecolor{currentstroke}%
\pgfsetstrokeopacity{0.449938}%
\pgfsetdash{}{0pt}%
\pgfpathmoveto{\pgfqpoint{1.460747in}{1.782388in}}%
\pgfpathcurveto{\pgfqpoint{1.468984in}{1.782388in}}{\pgfqpoint{1.476884in}{1.785660in}}{\pgfqpoint{1.482708in}{1.791484in}}%
\pgfpathcurveto{\pgfqpoint{1.488532in}{1.797308in}}{\pgfqpoint{1.491804in}{1.805208in}}{\pgfqpoint{1.491804in}{1.813444in}}%
\pgfpathcurveto{\pgfqpoint{1.491804in}{1.821680in}}{\pgfqpoint{1.488532in}{1.829580in}}{\pgfqpoint{1.482708in}{1.835404in}}%
\pgfpathcurveto{\pgfqpoint{1.476884in}{1.841228in}}{\pgfqpoint{1.468984in}{1.844501in}}{\pgfqpoint{1.460747in}{1.844501in}}%
\pgfpathcurveto{\pgfqpoint{1.452511in}{1.844501in}}{\pgfqpoint{1.444611in}{1.841228in}}{\pgfqpoint{1.438787in}{1.835404in}}%
\pgfpathcurveto{\pgfqpoint{1.432963in}{1.829580in}}{\pgfqpoint{1.429691in}{1.821680in}}{\pgfqpoint{1.429691in}{1.813444in}}%
\pgfpathcurveto{\pgfqpoint{1.429691in}{1.805208in}}{\pgfqpoint{1.432963in}{1.797308in}}{\pgfqpoint{1.438787in}{1.791484in}}%
\pgfpathcurveto{\pgfqpoint{1.444611in}{1.785660in}}{\pgfqpoint{1.452511in}{1.782388in}}{\pgfqpoint{1.460747in}{1.782388in}}%
\pgfpathclose%
\pgfusepath{stroke,fill}%
\end{pgfscope}%
\begin{pgfscope}%
\pgfpathrectangle{\pgfqpoint{0.100000in}{0.212622in}}{\pgfqpoint{3.696000in}{3.696000in}}%
\pgfusepath{clip}%
\pgfsetbuttcap%
\pgfsetroundjoin%
\definecolor{currentfill}{rgb}{0.121569,0.466667,0.705882}%
\pgfsetfillcolor{currentfill}%
\pgfsetfillopacity{0.451292}%
\pgfsetlinewidth{1.003750pt}%
\definecolor{currentstroke}{rgb}{0.121569,0.466667,0.705882}%
\pgfsetstrokecolor{currentstroke}%
\pgfsetstrokeopacity{0.451292}%
\pgfsetdash{}{0pt}%
\pgfpathmoveto{\pgfqpoint{2.024574in}{1.950798in}}%
\pgfpathcurveto{\pgfqpoint{2.032810in}{1.950798in}}{\pgfqpoint{2.040710in}{1.954070in}}{\pgfqpoint{2.046534in}{1.959894in}}%
\pgfpathcurveto{\pgfqpoint{2.052358in}{1.965718in}}{\pgfqpoint{2.055630in}{1.973618in}}{\pgfqpoint{2.055630in}{1.981855in}}%
\pgfpathcurveto{\pgfqpoint{2.055630in}{1.990091in}}{\pgfqpoint{2.052358in}{1.997991in}}{\pgfqpoint{2.046534in}{2.003815in}}%
\pgfpathcurveto{\pgfqpoint{2.040710in}{2.009639in}}{\pgfqpoint{2.032810in}{2.012911in}}{\pgfqpoint{2.024574in}{2.012911in}}%
\pgfpathcurveto{\pgfqpoint{2.016338in}{2.012911in}}{\pgfqpoint{2.008438in}{2.009639in}}{\pgfqpoint{2.002614in}{2.003815in}}%
\pgfpathcurveto{\pgfqpoint{1.996790in}{1.997991in}}{\pgfqpoint{1.993517in}{1.990091in}}{\pgfqpoint{1.993517in}{1.981855in}}%
\pgfpathcurveto{\pgfqpoint{1.993517in}{1.973618in}}{\pgfqpoint{1.996790in}{1.965718in}}{\pgfqpoint{2.002614in}{1.959894in}}%
\pgfpathcurveto{\pgfqpoint{2.008438in}{1.954070in}}{\pgfqpoint{2.016338in}{1.950798in}}{\pgfqpoint{2.024574in}{1.950798in}}%
\pgfpathclose%
\pgfusepath{stroke,fill}%
\end{pgfscope}%
\begin{pgfscope}%
\pgfpathrectangle{\pgfqpoint{0.100000in}{0.212622in}}{\pgfqpoint{3.696000in}{3.696000in}}%
\pgfusepath{clip}%
\pgfsetbuttcap%
\pgfsetroundjoin%
\definecolor{currentfill}{rgb}{0.121569,0.466667,0.705882}%
\pgfsetfillcolor{currentfill}%
\pgfsetfillopacity{0.452478}%
\pgfsetlinewidth{1.003750pt}%
\definecolor{currentstroke}{rgb}{0.121569,0.466667,0.705882}%
\pgfsetstrokecolor{currentstroke}%
\pgfsetstrokeopacity{0.452478}%
\pgfsetdash{}{0pt}%
\pgfpathmoveto{\pgfqpoint{1.452431in}{1.777179in}}%
\pgfpathcurveto{\pgfqpoint{1.460667in}{1.777179in}}{\pgfqpoint{1.468567in}{1.780451in}}{\pgfqpoint{1.474391in}{1.786275in}}%
\pgfpathcurveto{\pgfqpoint{1.480215in}{1.792099in}}{\pgfqpoint{1.483487in}{1.799999in}}{\pgfqpoint{1.483487in}{1.808236in}}%
\pgfpathcurveto{\pgfqpoint{1.483487in}{1.816472in}}{\pgfqpoint{1.480215in}{1.824372in}}{\pgfqpoint{1.474391in}{1.830196in}}%
\pgfpathcurveto{\pgfqpoint{1.468567in}{1.836020in}}{\pgfqpoint{1.460667in}{1.839292in}}{\pgfqpoint{1.452431in}{1.839292in}}%
\pgfpathcurveto{\pgfqpoint{1.444194in}{1.839292in}}{\pgfqpoint{1.436294in}{1.836020in}}{\pgfqpoint{1.430471in}{1.830196in}}%
\pgfpathcurveto{\pgfqpoint{1.424647in}{1.824372in}}{\pgfqpoint{1.421374in}{1.816472in}}{\pgfqpoint{1.421374in}{1.808236in}}%
\pgfpathcurveto{\pgfqpoint{1.421374in}{1.799999in}}{\pgfqpoint{1.424647in}{1.792099in}}{\pgfqpoint{1.430471in}{1.786275in}}%
\pgfpathcurveto{\pgfqpoint{1.436294in}{1.780451in}}{\pgfqpoint{1.444194in}{1.777179in}}{\pgfqpoint{1.452431in}{1.777179in}}%
\pgfpathclose%
\pgfusepath{stroke,fill}%
\end{pgfscope}%
\begin{pgfscope}%
\pgfpathrectangle{\pgfqpoint{0.100000in}{0.212622in}}{\pgfqpoint{3.696000in}{3.696000in}}%
\pgfusepath{clip}%
\pgfsetbuttcap%
\pgfsetroundjoin%
\definecolor{currentfill}{rgb}{0.121569,0.466667,0.705882}%
\pgfsetfillcolor{currentfill}%
\pgfsetfillopacity{0.453035}%
\pgfsetlinewidth{1.003750pt}%
\definecolor{currentstroke}{rgb}{0.121569,0.466667,0.705882}%
\pgfsetstrokecolor{currentstroke}%
\pgfsetstrokeopacity{0.453035}%
\pgfsetdash{}{0pt}%
\pgfpathmoveto{\pgfqpoint{2.025300in}{1.949006in}}%
\pgfpathcurveto{\pgfqpoint{2.033536in}{1.949006in}}{\pgfqpoint{2.041436in}{1.952278in}}{\pgfqpoint{2.047260in}{1.958102in}}%
\pgfpathcurveto{\pgfqpoint{2.053084in}{1.963926in}}{\pgfqpoint{2.056357in}{1.971826in}}{\pgfqpoint{2.056357in}{1.980062in}}%
\pgfpathcurveto{\pgfqpoint{2.056357in}{1.988298in}}{\pgfqpoint{2.053084in}{1.996199in}}{\pgfqpoint{2.047260in}{2.002022in}}%
\pgfpathcurveto{\pgfqpoint{2.041436in}{2.007846in}}{\pgfqpoint{2.033536in}{2.011119in}}{\pgfqpoint{2.025300in}{2.011119in}}%
\pgfpathcurveto{\pgfqpoint{2.017064in}{2.011119in}}{\pgfqpoint{2.009164in}{2.007846in}}{\pgfqpoint{2.003340in}{2.002022in}}%
\pgfpathcurveto{\pgfqpoint{1.997516in}{1.996199in}}{\pgfqpoint{1.994244in}{1.988298in}}{\pgfqpoint{1.994244in}{1.980062in}}%
\pgfpathcurveto{\pgfqpoint{1.994244in}{1.971826in}}{\pgfqpoint{1.997516in}{1.963926in}}{\pgfqpoint{2.003340in}{1.958102in}}%
\pgfpathcurveto{\pgfqpoint{2.009164in}{1.952278in}}{\pgfqpoint{2.017064in}{1.949006in}}{\pgfqpoint{2.025300in}{1.949006in}}%
\pgfpathclose%
\pgfusepath{stroke,fill}%
\end{pgfscope}%
\begin{pgfscope}%
\pgfpathrectangle{\pgfqpoint{0.100000in}{0.212622in}}{\pgfqpoint{3.696000in}{3.696000in}}%
\pgfusepath{clip}%
\pgfsetbuttcap%
\pgfsetroundjoin%
\definecolor{currentfill}{rgb}{0.121569,0.466667,0.705882}%
\pgfsetfillcolor{currentfill}%
\pgfsetfillopacity{0.454461}%
\pgfsetlinewidth{1.003750pt}%
\definecolor{currentstroke}{rgb}{0.121569,0.466667,0.705882}%
\pgfsetstrokecolor{currentstroke}%
\pgfsetstrokeopacity{0.454461}%
\pgfsetdash{}{0pt}%
\pgfpathmoveto{\pgfqpoint{1.445416in}{1.769570in}}%
\pgfpathcurveto{\pgfqpoint{1.453652in}{1.769570in}}{\pgfqpoint{1.461552in}{1.772843in}}{\pgfqpoint{1.467376in}{1.778667in}}%
\pgfpathcurveto{\pgfqpoint{1.473200in}{1.784491in}}{\pgfqpoint{1.476473in}{1.792391in}}{\pgfqpoint{1.476473in}{1.800627in}}%
\pgfpathcurveto{\pgfqpoint{1.476473in}{1.808863in}}{\pgfqpoint{1.473200in}{1.816763in}}{\pgfqpoint{1.467376in}{1.822587in}}%
\pgfpathcurveto{\pgfqpoint{1.461552in}{1.828411in}}{\pgfqpoint{1.453652in}{1.831683in}}{\pgfqpoint{1.445416in}{1.831683in}}%
\pgfpathcurveto{\pgfqpoint{1.437180in}{1.831683in}}{\pgfqpoint{1.429280in}{1.828411in}}{\pgfqpoint{1.423456in}{1.822587in}}%
\pgfpathcurveto{\pgfqpoint{1.417632in}{1.816763in}}{\pgfqpoint{1.414360in}{1.808863in}}{\pgfqpoint{1.414360in}{1.800627in}}%
\pgfpathcurveto{\pgfqpoint{1.414360in}{1.792391in}}{\pgfqpoint{1.417632in}{1.784491in}}{\pgfqpoint{1.423456in}{1.778667in}}%
\pgfpathcurveto{\pgfqpoint{1.429280in}{1.772843in}}{\pgfqpoint{1.437180in}{1.769570in}}{\pgfqpoint{1.445416in}{1.769570in}}%
\pgfpathclose%
\pgfusepath{stroke,fill}%
\end{pgfscope}%
\begin{pgfscope}%
\pgfpathrectangle{\pgfqpoint{0.100000in}{0.212622in}}{\pgfqpoint{3.696000in}{3.696000in}}%
\pgfusepath{clip}%
\pgfsetbuttcap%
\pgfsetroundjoin%
\definecolor{currentfill}{rgb}{0.121569,0.466667,0.705882}%
\pgfsetfillcolor{currentfill}%
\pgfsetfillopacity{0.455173}%
\pgfsetlinewidth{1.003750pt}%
\definecolor{currentstroke}{rgb}{0.121569,0.466667,0.705882}%
\pgfsetstrokecolor{currentstroke}%
\pgfsetstrokeopacity{0.455173}%
\pgfsetdash{}{0pt}%
\pgfpathmoveto{\pgfqpoint{2.026315in}{1.947648in}}%
\pgfpathcurveto{\pgfqpoint{2.034551in}{1.947648in}}{\pgfqpoint{2.042451in}{1.950921in}}{\pgfqpoint{2.048275in}{1.956745in}}%
\pgfpathcurveto{\pgfqpoint{2.054099in}{1.962569in}}{\pgfqpoint{2.057371in}{1.970469in}}{\pgfqpoint{2.057371in}{1.978705in}}%
\pgfpathcurveto{\pgfqpoint{2.057371in}{1.986941in}}{\pgfqpoint{2.054099in}{1.994841in}}{\pgfqpoint{2.048275in}{2.000665in}}%
\pgfpathcurveto{\pgfqpoint{2.042451in}{2.006489in}}{\pgfqpoint{2.034551in}{2.009761in}}{\pgfqpoint{2.026315in}{2.009761in}}%
\pgfpathcurveto{\pgfqpoint{2.018079in}{2.009761in}}{\pgfqpoint{2.010179in}{2.006489in}}{\pgfqpoint{2.004355in}{2.000665in}}%
\pgfpathcurveto{\pgfqpoint{1.998531in}{1.994841in}}{\pgfqpoint{1.995258in}{1.986941in}}{\pgfqpoint{1.995258in}{1.978705in}}%
\pgfpathcurveto{\pgfqpoint{1.995258in}{1.970469in}}{\pgfqpoint{1.998531in}{1.962569in}}{\pgfqpoint{2.004355in}{1.956745in}}%
\pgfpathcurveto{\pgfqpoint{2.010179in}{1.950921in}}{\pgfqpoint{2.018079in}{1.947648in}}{\pgfqpoint{2.026315in}{1.947648in}}%
\pgfpathclose%
\pgfusepath{stroke,fill}%
\end{pgfscope}%
\begin{pgfscope}%
\pgfpathrectangle{\pgfqpoint{0.100000in}{0.212622in}}{\pgfqpoint{3.696000in}{3.696000in}}%
\pgfusepath{clip}%
\pgfsetbuttcap%
\pgfsetroundjoin%
\definecolor{currentfill}{rgb}{0.121569,0.466667,0.705882}%
\pgfsetfillcolor{currentfill}%
\pgfsetfillopacity{0.456354}%
\pgfsetlinewidth{1.003750pt}%
\definecolor{currentstroke}{rgb}{0.121569,0.466667,0.705882}%
\pgfsetstrokecolor{currentstroke}%
\pgfsetstrokeopacity{0.456354}%
\pgfsetdash{}{0pt}%
\pgfpathmoveto{\pgfqpoint{1.439319in}{1.765957in}}%
\pgfpathcurveto{\pgfqpoint{1.447555in}{1.765957in}}{\pgfqpoint{1.455455in}{1.769229in}}{\pgfqpoint{1.461279in}{1.775053in}}%
\pgfpathcurveto{\pgfqpoint{1.467103in}{1.780877in}}{\pgfqpoint{1.470375in}{1.788777in}}{\pgfqpoint{1.470375in}{1.797014in}}%
\pgfpathcurveto{\pgfqpoint{1.470375in}{1.805250in}}{\pgfqpoint{1.467103in}{1.813150in}}{\pgfqpoint{1.461279in}{1.818974in}}%
\pgfpathcurveto{\pgfqpoint{1.455455in}{1.824798in}}{\pgfqpoint{1.447555in}{1.828070in}}{\pgfqpoint{1.439319in}{1.828070in}}%
\pgfpathcurveto{\pgfqpoint{1.431082in}{1.828070in}}{\pgfqpoint{1.423182in}{1.824798in}}{\pgfqpoint{1.417358in}{1.818974in}}%
\pgfpathcurveto{\pgfqpoint{1.411534in}{1.813150in}}{\pgfqpoint{1.408262in}{1.805250in}}{\pgfqpoint{1.408262in}{1.797014in}}%
\pgfpathcurveto{\pgfqpoint{1.408262in}{1.788777in}}{\pgfqpoint{1.411534in}{1.780877in}}{\pgfqpoint{1.417358in}{1.775053in}}%
\pgfpathcurveto{\pgfqpoint{1.423182in}{1.769229in}}{\pgfqpoint{1.431082in}{1.765957in}}{\pgfqpoint{1.439319in}{1.765957in}}%
\pgfpathclose%
\pgfusepath{stroke,fill}%
\end{pgfscope}%
\begin{pgfscope}%
\pgfpathrectangle{\pgfqpoint{0.100000in}{0.212622in}}{\pgfqpoint{3.696000in}{3.696000in}}%
\pgfusepath{clip}%
\pgfsetbuttcap%
\pgfsetroundjoin%
\definecolor{currentfill}{rgb}{0.121569,0.466667,0.705882}%
\pgfsetfillcolor{currentfill}%
\pgfsetfillopacity{0.457677}%
\pgfsetlinewidth{1.003750pt}%
\definecolor{currentstroke}{rgb}{0.121569,0.466667,0.705882}%
\pgfsetstrokecolor{currentstroke}%
\pgfsetstrokeopacity{0.457677}%
\pgfsetdash{}{0pt}%
\pgfpathmoveto{\pgfqpoint{2.027710in}{1.945118in}}%
\pgfpathcurveto{\pgfqpoint{2.035946in}{1.945118in}}{\pgfqpoint{2.043846in}{1.948391in}}{\pgfqpoint{2.049670in}{1.954214in}}%
\pgfpathcurveto{\pgfqpoint{2.055494in}{1.960038in}}{\pgfqpoint{2.058767in}{1.967938in}}{\pgfqpoint{2.058767in}{1.976175in}}%
\pgfpathcurveto{\pgfqpoint{2.058767in}{1.984411in}}{\pgfqpoint{2.055494in}{1.992311in}}{\pgfqpoint{2.049670in}{1.998135in}}%
\pgfpathcurveto{\pgfqpoint{2.043846in}{2.003959in}}{\pgfqpoint{2.035946in}{2.007231in}}{\pgfqpoint{2.027710in}{2.007231in}}%
\pgfpathcurveto{\pgfqpoint{2.019474in}{2.007231in}}{\pgfqpoint{2.011574in}{2.003959in}}{\pgfqpoint{2.005750in}{1.998135in}}%
\pgfpathcurveto{\pgfqpoint{1.999926in}{1.992311in}}{\pgfqpoint{1.996654in}{1.984411in}}{\pgfqpoint{1.996654in}{1.976175in}}%
\pgfpathcurveto{\pgfqpoint{1.996654in}{1.967938in}}{\pgfqpoint{1.999926in}{1.960038in}}{\pgfqpoint{2.005750in}{1.954214in}}%
\pgfpathcurveto{\pgfqpoint{2.011574in}{1.948391in}}{\pgfqpoint{2.019474in}{1.945118in}}{\pgfqpoint{2.027710in}{1.945118in}}%
\pgfpathclose%
\pgfusepath{stroke,fill}%
\end{pgfscope}%
\begin{pgfscope}%
\pgfpathrectangle{\pgfqpoint{0.100000in}{0.212622in}}{\pgfqpoint{3.696000in}{3.696000in}}%
\pgfusepath{clip}%
\pgfsetbuttcap%
\pgfsetroundjoin%
\definecolor{currentfill}{rgb}{0.121569,0.466667,0.705882}%
\pgfsetfillcolor{currentfill}%
\pgfsetfillopacity{0.458088}%
\pgfsetlinewidth{1.003750pt}%
\definecolor{currentstroke}{rgb}{0.121569,0.466667,0.705882}%
\pgfsetstrokecolor{currentstroke}%
\pgfsetstrokeopacity{0.458088}%
\pgfsetdash{}{0pt}%
\pgfpathmoveto{\pgfqpoint{1.433408in}{1.762443in}}%
\pgfpathcurveto{\pgfqpoint{1.441644in}{1.762443in}}{\pgfqpoint{1.449544in}{1.765715in}}{\pgfqpoint{1.455368in}{1.771539in}}%
\pgfpathcurveto{\pgfqpoint{1.461192in}{1.777363in}}{\pgfqpoint{1.464465in}{1.785263in}}{\pgfqpoint{1.464465in}{1.793500in}}%
\pgfpathcurveto{\pgfqpoint{1.464465in}{1.801736in}}{\pgfqpoint{1.461192in}{1.809636in}}{\pgfqpoint{1.455368in}{1.815460in}}%
\pgfpathcurveto{\pgfqpoint{1.449544in}{1.821284in}}{\pgfqpoint{1.441644in}{1.824556in}}{\pgfqpoint{1.433408in}{1.824556in}}%
\pgfpathcurveto{\pgfqpoint{1.425172in}{1.824556in}}{\pgfqpoint{1.417272in}{1.821284in}}{\pgfqpoint{1.411448in}{1.815460in}}%
\pgfpathcurveto{\pgfqpoint{1.405624in}{1.809636in}}{\pgfqpoint{1.402352in}{1.801736in}}{\pgfqpoint{1.402352in}{1.793500in}}%
\pgfpathcurveto{\pgfqpoint{1.402352in}{1.785263in}}{\pgfqpoint{1.405624in}{1.777363in}}{\pgfqpoint{1.411448in}{1.771539in}}%
\pgfpathcurveto{\pgfqpoint{1.417272in}{1.765715in}}{\pgfqpoint{1.425172in}{1.762443in}}{\pgfqpoint{1.433408in}{1.762443in}}%
\pgfpathclose%
\pgfusepath{stroke,fill}%
\end{pgfscope}%
\begin{pgfscope}%
\pgfpathrectangle{\pgfqpoint{0.100000in}{0.212622in}}{\pgfqpoint{3.696000in}{3.696000in}}%
\pgfusepath{clip}%
\pgfsetbuttcap%
\pgfsetroundjoin%
\definecolor{currentfill}{rgb}{0.121569,0.466667,0.705882}%
\pgfsetfillcolor{currentfill}%
\pgfsetfillopacity{0.459312}%
\pgfsetlinewidth{1.003750pt}%
\definecolor{currentstroke}{rgb}{0.121569,0.466667,0.705882}%
\pgfsetstrokecolor{currentstroke}%
\pgfsetstrokeopacity{0.459312}%
\pgfsetdash{}{0pt}%
\pgfpathmoveto{\pgfqpoint{2.028141in}{1.945321in}}%
\pgfpathcurveto{\pgfqpoint{2.036377in}{1.945321in}}{\pgfqpoint{2.044277in}{1.948593in}}{\pgfqpoint{2.050101in}{1.954417in}}%
\pgfpathcurveto{\pgfqpoint{2.055925in}{1.960241in}}{\pgfqpoint{2.059197in}{1.968141in}}{\pgfqpoint{2.059197in}{1.976377in}}%
\pgfpathcurveto{\pgfqpoint{2.059197in}{1.984613in}}{\pgfqpoint{2.055925in}{1.992513in}}{\pgfqpoint{2.050101in}{1.998337in}}%
\pgfpathcurveto{\pgfqpoint{2.044277in}{2.004161in}}{\pgfqpoint{2.036377in}{2.007434in}}{\pgfqpoint{2.028141in}{2.007434in}}%
\pgfpathcurveto{\pgfqpoint{2.019904in}{2.007434in}}{\pgfqpoint{2.012004in}{2.004161in}}{\pgfqpoint{2.006180in}{1.998337in}}%
\pgfpathcurveto{\pgfqpoint{2.000356in}{1.992513in}}{\pgfqpoint{1.997084in}{1.984613in}}{\pgfqpoint{1.997084in}{1.976377in}}%
\pgfpathcurveto{\pgfqpoint{1.997084in}{1.968141in}}{\pgfqpoint{2.000356in}{1.960241in}}{\pgfqpoint{2.006180in}{1.954417in}}%
\pgfpathcurveto{\pgfqpoint{2.012004in}{1.948593in}}{\pgfqpoint{2.019904in}{1.945321in}}{\pgfqpoint{2.028141in}{1.945321in}}%
\pgfpathclose%
\pgfusepath{stroke,fill}%
\end{pgfscope}%
\begin{pgfscope}%
\pgfpathrectangle{\pgfqpoint{0.100000in}{0.212622in}}{\pgfqpoint{3.696000in}{3.696000in}}%
\pgfusepath{clip}%
\pgfsetbuttcap%
\pgfsetroundjoin%
\definecolor{currentfill}{rgb}{0.121569,0.466667,0.705882}%
\pgfsetfillcolor{currentfill}%
\pgfsetfillopacity{0.459418}%
\pgfsetlinewidth{1.003750pt}%
\definecolor{currentstroke}{rgb}{0.121569,0.466667,0.705882}%
\pgfsetstrokecolor{currentstroke}%
\pgfsetstrokeopacity{0.459418}%
\pgfsetdash{}{0pt}%
\pgfpathmoveto{\pgfqpoint{1.428518in}{1.756709in}}%
\pgfpathcurveto{\pgfqpoint{1.436754in}{1.756709in}}{\pgfqpoint{1.444654in}{1.759982in}}{\pgfqpoint{1.450478in}{1.765806in}}%
\pgfpathcurveto{\pgfqpoint{1.456302in}{1.771629in}}{\pgfqpoint{1.459574in}{1.779530in}}{\pgfqpoint{1.459574in}{1.787766in}}%
\pgfpathcurveto{\pgfqpoint{1.459574in}{1.796002in}}{\pgfqpoint{1.456302in}{1.803902in}}{\pgfqpoint{1.450478in}{1.809726in}}%
\pgfpathcurveto{\pgfqpoint{1.444654in}{1.815550in}}{\pgfqpoint{1.436754in}{1.818822in}}{\pgfqpoint{1.428518in}{1.818822in}}%
\pgfpathcurveto{\pgfqpoint{1.420281in}{1.818822in}}{\pgfqpoint{1.412381in}{1.815550in}}{\pgfqpoint{1.406557in}{1.809726in}}%
\pgfpathcurveto{\pgfqpoint{1.400734in}{1.803902in}}{\pgfqpoint{1.397461in}{1.796002in}}{\pgfqpoint{1.397461in}{1.787766in}}%
\pgfpathcurveto{\pgfqpoint{1.397461in}{1.779530in}}{\pgfqpoint{1.400734in}{1.771629in}}{\pgfqpoint{1.406557in}{1.765806in}}%
\pgfpathcurveto{\pgfqpoint{1.412381in}{1.759982in}}{\pgfqpoint{1.420281in}{1.756709in}}{\pgfqpoint{1.428518in}{1.756709in}}%
\pgfpathclose%
\pgfusepath{stroke,fill}%
\end{pgfscope}%
\begin{pgfscope}%
\pgfpathrectangle{\pgfqpoint{0.100000in}{0.212622in}}{\pgfqpoint{3.696000in}{3.696000in}}%
\pgfusepath{clip}%
\pgfsetbuttcap%
\pgfsetroundjoin%
\definecolor{currentfill}{rgb}{0.121569,0.466667,0.705882}%
\pgfsetfillcolor{currentfill}%
\pgfsetfillopacity{0.460763}%
\pgfsetlinewidth{1.003750pt}%
\definecolor{currentstroke}{rgb}{0.121569,0.466667,0.705882}%
\pgfsetstrokecolor{currentstroke}%
\pgfsetstrokeopacity{0.460763}%
\pgfsetdash{}{0pt}%
\pgfpathmoveto{\pgfqpoint{1.424592in}{1.754374in}}%
\pgfpathcurveto{\pgfqpoint{1.432828in}{1.754374in}}{\pgfqpoint{1.440728in}{1.757647in}}{\pgfqpoint{1.446552in}{1.763471in}}%
\pgfpathcurveto{\pgfqpoint{1.452376in}{1.769295in}}{\pgfqpoint{1.455648in}{1.777195in}}{\pgfqpoint{1.455648in}{1.785431in}}%
\pgfpathcurveto{\pgfqpoint{1.455648in}{1.793667in}}{\pgfqpoint{1.452376in}{1.801567in}}{\pgfqpoint{1.446552in}{1.807391in}}%
\pgfpathcurveto{\pgfqpoint{1.440728in}{1.813215in}}{\pgfqpoint{1.432828in}{1.816487in}}{\pgfqpoint{1.424592in}{1.816487in}}%
\pgfpathcurveto{\pgfqpoint{1.416355in}{1.816487in}}{\pgfqpoint{1.408455in}{1.813215in}}{\pgfqpoint{1.402631in}{1.807391in}}%
\pgfpathcurveto{\pgfqpoint{1.396807in}{1.801567in}}{\pgfqpoint{1.393535in}{1.793667in}}{\pgfqpoint{1.393535in}{1.785431in}}%
\pgfpathcurveto{\pgfqpoint{1.393535in}{1.777195in}}{\pgfqpoint{1.396807in}{1.769295in}}{\pgfqpoint{1.402631in}{1.763471in}}%
\pgfpathcurveto{\pgfqpoint{1.408455in}{1.757647in}}{\pgfqpoint{1.416355in}{1.754374in}}{\pgfqpoint{1.424592in}{1.754374in}}%
\pgfpathclose%
\pgfusepath{stroke,fill}%
\end{pgfscope}%
\begin{pgfscope}%
\pgfpathrectangle{\pgfqpoint{0.100000in}{0.212622in}}{\pgfqpoint{3.696000in}{3.696000in}}%
\pgfusepath{clip}%
\pgfsetbuttcap%
\pgfsetroundjoin%
\definecolor{currentfill}{rgb}{0.121569,0.466667,0.705882}%
\pgfsetfillcolor{currentfill}%
\pgfsetfillopacity{0.461211}%
\pgfsetlinewidth{1.003750pt}%
\definecolor{currentstroke}{rgb}{0.121569,0.466667,0.705882}%
\pgfsetstrokecolor{currentstroke}%
\pgfsetstrokeopacity{0.461211}%
\pgfsetdash{}{0pt}%
\pgfpathmoveto{\pgfqpoint{2.029395in}{1.943746in}}%
\pgfpathcurveto{\pgfqpoint{2.037632in}{1.943746in}}{\pgfqpoint{2.045532in}{1.947018in}}{\pgfqpoint{2.051356in}{1.952842in}}%
\pgfpathcurveto{\pgfqpoint{2.057179in}{1.958666in}}{\pgfqpoint{2.060452in}{1.966566in}}{\pgfqpoint{2.060452in}{1.974802in}}%
\pgfpathcurveto{\pgfqpoint{2.060452in}{1.983038in}}{\pgfqpoint{2.057179in}{1.990939in}}{\pgfqpoint{2.051356in}{1.996762in}}%
\pgfpathcurveto{\pgfqpoint{2.045532in}{2.002586in}}{\pgfqpoint{2.037632in}{2.005859in}}{\pgfqpoint{2.029395in}{2.005859in}}%
\pgfpathcurveto{\pgfqpoint{2.021159in}{2.005859in}}{\pgfqpoint{2.013259in}{2.002586in}}{\pgfqpoint{2.007435in}{1.996762in}}%
\pgfpathcurveto{\pgfqpoint{2.001611in}{1.990939in}}{\pgfqpoint{1.998339in}{1.983038in}}{\pgfqpoint{1.998339in}{1.974802in}}%
\pgfpathcurveto{\pgfqpoint{1.998339in}{1.966566in}}{\pgfqpoint{2.001611in}{1.958666in}}{\pgfqpoint{2.007435in}{1.952842in}}%
\pgfpathcurveto{\pgfqpoint{2.013259in}{1.947018in}}{\pgfqpoint{2.021159in}{1.943746in}}{\pgfqpoint{2.029395in}{1.943746in}}%
\pgfpathclose%
\pgfusepath{stroke,fill}%
\end{pgfscope}%
\begin{pgfscope}%
\pgfpathrectangle{\pgfqpoint{0.100000in}{0.212622in}}{\pgfqpoint{3.696000in}{3.696000in}}%
\pgfusepath{clip}%
\pgfsetbuttcap%
\pgfsetroundjoin%
\definecolor{currentfill}{rgb}{0.121569,0.466667,0.705882}%
\pgfsetfillcolor{currentfill}%
\pgfsetfillopacity{0.462343}%
\pgfsetlinewidth{1.003750pt}%
\definecolor{currentstroke}{rgb}{0.121569,0.466667,0.705882}%
\pgfsetstrokecolor{currentstroke}%
\pgfsetstrokeopacity{0.462343}%
\pgfsetdash{}{0pt}%
\pgfpathmoveto{\pgfqpoint{2.030043in}{1.943437in}}%
\pgfpathcurveto{\pgfqpoint{2.038279in}{1.943437in}}{\pgfqpoint{2.046179in}{1.946709in}}{\pgfqpoint{2.052003in}{1.952533in}}%
\pgfpathcurveto{\pgfqpoint{2.057827in}{1.958357in}}{\pgfqpoint{2.061099in}{1.966257in}}{\pgfqpoint{2.061099in}{1.974493in}}%
\pgfpathcurveto{\pgfqpoint{2.061099in}{1.982730in}}{\pgfqpoint{2.057827in}{1.990630in}}{\pgfqpoint{2.052003in}{1.996454in}}%
\pgfpathcurveto{\pgfqpoint{2.046179in}{2.002278in}}{\pgfqpoint{2.038279in}{2.005550in}}{\pgfqpoint{2.030043in}{2.005550in}}%
\pgfpathcurveto{\pgfqpoint{2.021807in}{2.005550in}}{\pgfqpoint{2.013907in}{2.002278in}}{\pgfqpoint{2.008083in}{1.996454in}}%
\pgfpathcurveto{\pgfqpoint{2.002259in}{1.990630in}}{\pgfqpoint{1.998986in}{1.982730in}}{\pgfqpoint{1.998986in}{1.974493in}}%
\pgfpathcurveto{\pgfqpoint{1.998986in}{1.966257in}}{\pgfqpoint{2.002259in}{1.958357in}}{\pgfqpoint{2.008083in}{1.952533in}}%
\pgfpathcurveto{\pgfqpoint{2.013907in}{1.946709in}}{\pgfqpoint{2.021807in}{1.943437in}}{\pgfqpoint{2.030043in}{1.943437in}}%
\pgfpathclose%
\pgfusepath{stroke,fill}%
\end{pgfscope}%
\begin{pgfscope}%
\pgfpathrectangle{\pgfqpoint{0.100000in}{0.212622in}}{\pgfqpoint{3.696000in}{3.696000in}}%
\pgfusepath{clip}%
\pgfsetbuttcap%
\pgfsetroundjoin%
\definecolor{currentfill}{rgb}{0.121569,0.466667,0.705882}%
\pgfsetfillcolor{currentfill}%
\pgfsetfillopacity{0.463151}%
\pgfsetlinewidth{1.003750pt}%
\definecolor{currentstroke}{rgb}{0.121569,0.466667,0.705882}%
\pgfsetstrokecolor{currentstroke}%
\pgfsetstrokeopacity{0.463151}%
\pgfsetdash{}{0pt}%
\pgfpathmoveto{\pgfqpoint{1.416891in}{1.750571in}}%
\pgfpathcurveto{\pgfqpoint{1.425127in}{1.750571in}}{\pgfqpoint{1.433027in}{1.753844in}}{\pgfqpoint{1.438851in}{1.759668in}}%
\pgfpathcurveto{\pgfqpoint{1.444675in}{1.765492in}}{\pgfqpoint{1.447948in}{1.773392in}}{\pgfqpoint{1.447948in}{1.781628in}}%
\pgfpathcurveto{\pgfqpoint{1.447948in}{1.789864in}}{\pgfqpoint{1.444675in}{1.797764in}}{\pgfqpoint{1.438851in}{1.803588in}}%
\pgfpathcurveto{\pgfqpoint{1.433027in}{1.809412in}}{\pgfqpoint{1.425127in}{1.812684in}}{\pgfqpoint{1.416891in}{1.812684in}}%
\pgfpathcurveto{\pgfqpoint{1.408655in}{1.812684in}}{\pgfqpoint{1.400755in}{1.809412in}}{\pgfqpoint{1.394931in}{1.803588in}}%
\pgfpathcurveto{\pgfqpoint{1.389107in}{1.797764in}}{\pgfqpoint{1.385835in}{1.789864in}}{\pgfqpoint{1.385835in}{1.781628in}}%
\pgfpathcurveto{\pgfqpoint{1.385835in}{1.773392in}}{\pgfqpoint{1.389107in}{1.765492in}}{\pgfqpoint{1.394931in}{1.759668in}}%
\pgfpathcurveto{\pgfqpoint{1.400755in}{1.753844in}}{\pgfqpoint{1.408655in}{1.750571in}}{\pgfqpoint{1.416891in}{1.750571in}}%
\pgfpathclose%
\pgfusepath{stroke,fill}%
\end{pgfscope}%
\begin{pgfscope}%
\pgfpathrectangle{\pgfqpoint{0.100000in}{0.212622in}}{\pgfqpoint{3.696000in}{3.696000in}}%
\pgfusepath{clip}%
\pgfsetbuttcap%
\pgfsetroundjoin%
\definecolor{currentfill}{rgb}{0.121569,0.466667,0.705882}%
\pgfsetfillcolor{currentfill}%
\pgfsetfillopacity{0.463510}%
\pgfsetlinewidth{1.003750pt}%
\definecolor{currentstroke}{rgb}{0.121569,0.466667,0.705882}%
\pgfsetstrokecolor{currentstroke}%
\pgfsetstrokeopacity{0.463510}%
\pgfsetdash{}{0pt}%
\pgfpathmoveto{\pgfqpoint{2.030731in}{1.942327in}}%
\pgfpathcurveto{\pgfqpoint{2.038967in}{1.942327in}}{\pgfqpoint{2.046867in}{1.945600in}}{\pgfqpoint{2.052691in}{1.951424in}}%
\pgfpathcurveto{\pgfqpoint{2.058515in}{1.957247in}}{\pgfqpoint{2.061788in}{1.965147in}}{\pgfqpoint{2.061788in}{1.973384in}}%
\pgfpathcurveto{\pgfqpoint{2.061788in}{1.981620in}}{\pgfqpoint{2.058515in}{1.989520in}}{\pgfqpoint{2.052691in}{1.995344in}}%
\pgfpathcurveto{\pgfqpoint{2.046867in}{2.001168in}}{\pgfqpoint{2.038967in}{2.004440in}}{\pgfqpoint{2.030731in}{2.004440in}}%
\pgfpathcurveto{\pgfqpoint{2.022495in}{2.004440in}}{\pgfqpoint{2.014595in}{2.001168in}}{\pgfqpoint{2.008771in}{1.995344in}}%
\pgfpathcurveto{\pgfqpoint{2.002947in}{1.989520in}}{\pgfqpoint{1.999675in}{1.981620in}}{\pgfqpoint{1.999675in}{1.973384in}}%
\pgfpathcurveto{\pgfqpoint{1.999675in}{1.965147in}}{\pgfqpoint{2.002947in}{1.957247in}}{\pgfqpoint{2.008771in}{1.951424in}}%
\pgfpathcurveto{\pgfqpoint{2.014595in}{1.945600in}}{\pgfqpoint{2.022495in}{1.942327in}}{\pgfqpoint{2.030731in}{1.942327in}}%
\pgfpathclose%
\pgfusepath{stroke,fill}%
\end{pgfscope}%
\begin{pgfscope}%
\pgfpathrectangle{\pgfqpoint{0.100000in}{0.212622in}}{\pgfqpoint{3.696000in}{3.696000in}}%
\pgfusepath{clip}%
\pgfsetbuttcap%
\pgfsetroundjoin%
\definecolor{currentfill}{rgb}{0.121569,0.466667,0.705882}%
\pgfsetfillcolor{currentfill}%
\pgfsetfillopacity{0.464179}%
\pgfsetlinewidth{1.003750pt}%
\definecolor{currentstroke}{rgb}{0.121569,0.466667,0.705882}%
\pgfsetstrokecolor{currentstroke}%
\pgfsetstrokeopacity{0.464179}%
\pgfsetdash{}{0pt}%
\pgfpathmoveto{\pgfqpoint{2.031077in}{1.941886in}}%
\pgfpathcurveto{\pgfqpoint{2.039313in}{1.941886in}}{\pgfqpoint{2.047213in}{1.945159in}}{\pgfqpoint{2.053037in}{1.950983in}}%
\pgfpathcurveto{\pgfqpoint{2.058861in}{1.956806in}}{\pgfqpoint{2.062134in}{1.964707in}}{\pgfqpoint{2.062134in}{1.972943in}}%
\pgfpathcurveto{\pgfqpoint{2.062134in}{1.981179in}}{\pgfqpoint{2.058861in}{1.989079in}}{\pgfqpoint{2.053037in}{1.994903in}}%
\pgfpathcurveto{\pgfqpoint{2.047213in}{2.000727in}}{\pgfqpoint{2.039313in}{2.003999in}}{\pgfqpoint{2.031077in}{2.003999in}}%
\pgfpathcurveto{\pgfqpoint{2.022841in}{2.003999in}}{\pgfqpoint{2.014941in}{2.000727in}}{\pgfqpoint{2.009117in}{1.994903in}}%
\pgfpathcurveto{\pgfqpoint{2.003293in}{1.989079in}}{\pgfqpoint{2.000021in}{1.981179in}}{\pgfqpoint{2.000021in}{1.972943in}}%
\pgfpathcurveto{\pgfqpoint{2.000021in}{1.964707in}}{\pgfqpoint{2.003293in}{1.956806in}}{\pgfqpoint{2.009117in}{1.950983in}}%
\pgfpathcurveto{\pgfqpoint{2.014941in}{1.945159in}}{\pgfqpoint{2.022841in}{1.941886in}}{\pgfqpoint{2.031077in}{1.941886in}}%
\pgfpathclose%
\pgfusepath{stroke,fill}%
\end{pgfscope}%
\begin{pgfscope}%
\pgfpathrectangle{\pgfqpoint{0.100000in}{0.212622in}}{\pgfqpoint{3.696000in}{3.696000in}}%
\pgfusepath{clip}%
\pgfsetbuttcap%
\pgfsetroundjoin%
\definecolor{currentfill}{rgb}{0.121569,0.466667,0.705882}%
\pgfsetfillcolor{currentfill}%
\pgfsetfillopacity{0.464833}%
\pgfsetlinewidth{1.003750pt}%
\definecolor{currentstroke}{rgb}{0.121569,0.466667,0.705882}%
\pgfsetstrokecolor{currentstroke}%
\pgfsetstrokeopacity{0.464833}%
\pgfsetdash{}{0pt}%
\pgfpathmoveto{\pgfqpoint{1.410927in}{1.743449in}}%
\pgfpathcurveto{\pgfqpoint{1.419163in}{1.743449in}}{\pgfqpoint{1.427063in}{1.746722in}}{\pgfqpoint{1.432887in}{1.752546in}}%
\pgfpathcurveto{\pgfqpoint{1.438711in}{1.758369in}}{\pgfqpoint{1.441983in}{1.766269in}}{\pgfqpoint{1.441983in}{1.774506in}}%
\pgfpathcurveto{\pgfqpoint{1.441983in}{1.782742in}}{\pgfqpoint{1.438711in}{1.790642in}}{\pgfqpoint{1.432887in}{1.796466in}}%
\pgfpathcurveto{\pgfqpoint{1.427063in}{1.802290in}}{\pgfqpoint{1.419163in}{1.805562in}}{\pgfqpoint{1.410927in}{1.805562in}}%
\pgfpathcurveto{\pgfqpoint{1.402691in}{1.805562in}}{\pgfqpoint{1.394791in}{1.802290in}}{\pgfqpoint{1.388967in}{1.796466in}}%
\pgfpathcurveto{\pgfqpoint{1.383143in}{1.790642in}}{\pgfqpoint{1.379870in}{1.782742in}}{\pgfqpoint{1.379870in}{1.774506in}}%
\pgfpathcurveto{\pgfqpoint{1.379870in}{1.766269in}}{\pgfqpoint{1.383143in}{1.758369in}}{\pgfqpoint{1.388967in}{1.752546in}}%
\pgfpathcurveto{\pgfqpoint{1.394791in}{1.746722in}}{\pgfqpoint{1.402691in}{1.743449in}}{\pgfqpoint{1.410927in}{1.743449in}}%
\pgfpathclose%
\pgfusepath{stroke,fill}%
\end{pgfscope}%
\begin{pgfscope}%
\pgfpathrectangle{\pgfqpoint{0.100000in}{0.212622in}}{\pgfqpoint{3.696000in}{3.696000in}}%
\pgfusepath{clip}%
\pgfsetbuttcap%
\pgfsetroundjoin%
\definecolor{currentfill}{rgb}{0.121569,0.466667,0.705882}%
\pgfsetfillcolor{currentfill}%
\pgfsetfillopacity{0.465124}%
\pgfsetlinewidth{1.003750pt}%
\definecolor{currentstroke}{rgb}{0.121569,0.466667,0.705882}%
\pgfsetstrokecolor{currentstroke}%
\pgfsetstrokeopacity{0.465124}%
\pgfsetdash{}{0pt}%
\pgfpathmoveto{\pgfqpoint{2.031486in}{1.941831in}}%
\pgfpathcurveto{\pgfqpoint{2.039722in}{1.941831in}}{\pgfqpoint{2.047622in}{1.945104in}}{\pgfqpoint{2.053446in}{1.950928in}}%
\pgfpathcurveto{\pgfqpoint{2.059270in}{1.956751in}}{\pgfqpoint{2.062542in}{1.964652in}}{\pgfqpoint{2.062542in}{1.972888in}}%
\pgfpathcurveto{\pgfqpoint{2.062542in}{1.981124in}}{\pgfqpoint{2.059270in}{1.989024in}}{\pgfqpoint{2.053446in}{1.994848in}}%
\pgfpathcurveto{\pgfqpoint{2.047622in}{2.000672in}}{\pgfqpoint{2.039722in}{2.003944in}}{\pgfqpoint{2.031486in}{2.003944in}}%
\pgfpathcurveto{\pgfqpoint{2.023250in}{2.003944in}}{\pgfqpoint{2.015349in}{2.000672in}}{\pgfqpoint{2.009526in}{1.994848in}}%
\pgfpathcurveto{\pgfqpoint{2.003702in}{1.989024in}}{\pgfqpoint{2.000429in}{1.981124in}}{\pgfqpoint{2.000429in}{1.972888in}}%
\pgfpathcurveto{\pgfqpoint{2.000429in}{1.964652in}}{\pgfqpoint{2.003702in}{1.956751in}}{\pgfqpoint{2.009526in}{1.950928in}}%
\pgfpathcurveto{\pgfqpoint{2.015349in}{1.945104in}}{\pgfqpoint{2.023250in}{1.941831in}}{\pgfqpoint{2.031486in}{1.941831in}}%
\pgfpathclose%
\pgfusepath{stroke,fill}%
\end{pgfscope}%
\begin{pgfscope}%
\pgfpathrectangle{\pgfqpoint{0.100000in}{0.212622in}}{\pgfqpoint{3.696000in}{3.696000in}}%
\pgfusepath{clip}%
\pgfsetbuttcap%
\pgfsetroundjoin%
\definecolor{currentfill}{rgb}{0.121569,0.466667,0.705882}%
\pgfsetfillcolor{currentfill}%
\pgfsetfillopacity{0.466326}%
\pgfsetlinewidth{1.003750pt}%
\definecolor{currentstroke}{rgb}{0.121569,0.466667,0.705882}%
\pgfsetstrokecolor{currentstroke}%
\pgfsetstrokeopacity{0.466326}%
\pgfsetdash{}{0pt}%
\pgfpathmoveto{\pgfqpoint{1.406302in}{1.740086in}}%
\pgfpathcurveto{\pgfqpoint{1.414538in}{1.740086in}}{\pgfqpoint{1.422439in}{1.743358in}}{\pgfqpoint{1.428262in}{1.749182in}}%
\pgfpathcurveto{\pgfqpoint{1.434086in}{1.755006in}}{\pgfqpoint{1.437359in}{1.762906in}}{\pgfqpoint{1.437359in}{1.771142in}}%
\pgfpathcurveto{\pgfqpoint{1.437359in}{1.779378in}}{\pgfqpoint{1.434086in}{1.787278in}}{\pgfqpoint{1.428262in}{1.793102in}}%
\pgfpathcurveto{\pgfqpoint{1.422439in}{1.798926in}}{\pgfqpoint{1.414538in}{1.802199in}}{\pgfqpoint{1.406302in}{1.802199in}}%
\pgfpathcurveto{\pgfqpoint{1.398066in}{1.802199in}}{\pgfqpoint{1.390166in}{1.798926in}}{\pgfqpoint{1.384342in}{1.793102in}}%
\pgfpathcurveto{\pgfqpoint{1.378518in}{1.787278in}}{\pgfqpoint{1.375246in}{1.779378in}}{\pgfqpoint{1.375246in}{1.771142in}}%
\pgfpathcurveto{\pgfqpoint{1.375246in}{1.762906in}}{\pgfqpoint{1.378518in}{1.755006in}}{\pgfqpoint{1.384342in}{1.749182in}}%
\pgfpathcurveto{\pgfqpoint{1.390166in}{1.743358in}}{\pgfqpoint{1.398066in}{1.740086in}}{\pgfqpoint{1.406302in}{1.740086in}}%
\pgfpathclose%
\pgfusepath{stroke,fill}%
\end{pgfscope}%
\begin{pgfscope}%
\pgfpathrectangle{\pgfqpoint{0.100000in}{0.212622in}}{\pgfqpoint{3.696000in}{3.696000in}}%
\pgfusepath{clip}%
\pgfsetbuttcap%
\pgfsetroundjoin%
\definecolor{currentfill}{rgb}{0.121569,0.466667,0.705882}%
\pgfsetfillcolor{currentfill}%
\pgfsetfillopacity{0.466628}%
\pgfsetlinewidth{1.003750pt}%
\definecolor{currentstroke}{rgb}{0.121569,0.466667,0.705882}%
\pgfsetstrokecolor{currentstroke}%
\pgfsetstrokeopacity{0.466628}%
\pgfsetdash{}{0pt}%
\pgfpathmoveto{\pgfqpoint{2.032112in}{1.940604in}}%
\pgfpathcurveto{\pgfqpoint{2.040348in}{1.940604in}}{\pgfqpoint{2.048248in}{1.943876in}}{\pgfqpoint{2.054072in}{1.949700in}}%
\pgfpathcurveto{\pgfqpoint{2.059896in}{1.955524in}}{\pgfqpoint{2.063168in}{1.963424in}}{\pgfqpoint{2.063168in}{1.971660in}}%
\pgfpathcurveto{\pgfqpoint{2.063168in}{1.979897in}}{\pgfqpoint{2.059896in}{1.987797in}}{\pgfqpoint{2.054072in}{1.993621in}}%
\pgfpathcurveto{\pgfqpoint{2.048248in}{1.999445in}}{\pgfqpoint{2.040348in}{2.002717in}}{\pgfqpoint{2.032112in}{2.002717in}}%
\pgfpathcurveto{\pgfqpoint{2.023876in}{2.002717in}}{\pgfqpoint{2.015976in}{1.999445in}}{\pgfqpoint{2.010152in}{1.993621in}}%
\pgfpathcurveto{\pgfqpoint{2.004328in}{1.987797in}}{\pgfqpoint{2.001055in}{1.979897in}}{\pgfqpoint{2.001055in}{1.971660in}}%
\pgfpathcurveto{\pgfqpoint{2.001055in}{1.963424in}}{\pgfqpoint{2.004328in}{1.955524in}}{\pgfqpoint{2.010152in}{1.949700in}}%
\pgfpathcurveto{\pgfqpoint{2.015976in}{1.943876in}}{\pgfqpoint{2.023876in}{1.940604in}}{\pgfqpoint{2.032112in}{1.940604in}}%
\pgfpathclose%
\pgfusepath{stroke,fill}%
\end{pgfscope}%
\begin{pgfscope}%
\pgfpathrectangle{\pgfqpoint{0.100000in}{0.212622in}}{\pgfqpoint{3.696000in}{3.696000in}}%
\pgfusepath{clip}%
\pgfsetbuttcap%
\pgfsetroundjoin%
\definecolor{currentfill}{rgb}{0.121569,0.466667,0.705882}%
\pgfsetfillcolor{currentfill}%
\pgfsetfillopacity{0.467402}%
\pgfsetlinewidth{1.003750pt}%
\definecolor{currentstroke}{rgb}{0.121569,0.466667,0.705882}%
\pgfsetstrokecolor{currentstroke}%
\pgfsetstrokeopacity{0.467402}%
\pgfsetdash{}{0pt}%
\pgfpathmoveto{\pgfqpoint{2.032537in}{1.939596in}}%
\pgfpathcurveto{\pgfqpoint{2.040773in}{1.939596in}}{\pgfqpoint{2.048674in}{1.942869in}}{\pgfqpoint{2.054497in}{1.948693in}}%
\pgfpathcurveto{\pgfqpoint{2.060321in}{1.954516in}}{\pgfqpoint{2.063594in}{1.962417in}}{\pgfqpoint{2.063594in}{1.970653in}}%
\pgfpathcurveto{\pgfqpoint{2.063594in}{1.978889in}}{\pgfqpoint{2.060321in}{1.986789in}}{\pgfqpoint{2.054497in}{1.992613in}}%
\pgfpathcurveto{\pgfqpoint{2.048674in}{1.998437in}}{\pgfqpoint{2.040773in}{2.001709in}}{\pgfqpoint{2.032537in}{2.001709in}}%
\pgfpathcurveto{\pgfqpoint{2.024301in}{2.001709in}}{\pgfqpoint{2.016401in}{1.998437in}}{\pgfqpoint{2.010577in}{1.992613in}}%
\pgfpathcurveto{\pgfqpoint{2.004753in}{1.986789in}}{\pgfqpoint{2.001481in}{1.978889in}}{\pgfqpoint{2.001481in}{1.970653in}}%
\pgfpathcurveto{\pgfqpoint{2.001481in}{1.962417in}}{\pgfqpoint{2.004753in}{1.954516in}}{\pgfqpoint{2.010577in}{1.948693in}}%
\pgfpathcurveto{\pgfqpoint{2.016401in}{1.942869in}}{\pgfqpoint{2.024301in}{1.939596in}}{\pgfqpoint{2.032537in}{1.939596in}}%
\pgfpathclose%
\pgfusepath{stroke,fill}%
\end{pgfscope}%
\begin{pgfscope}%
\pgfpathrectangle{\pgfqpoint{0.100000in}{0.212622in}}{\pgfqpoint{3.696000in}{3.696000in}}%
\pgfusepath{clip}%
\pgfsetbuttcap%
\pgfsetroundjoin%
\definecolor{currentfill}{rgb}{0.121569,0.466667,0.705882}%
\pgfsetfillcolor{currentfill}%
\pgfsetfillopacity{0.467690}%
\pgfsetlinewidth{1.003750pt}%
\definecolor{currentstroke}{rgb}{0.121569,0.466667,0.705882}%
\pgfsetstrokecolor{currentstroke}%
\pgfsetstrokeopacity{0.467690}%
\pgfsetdash{}{0pt}%
\pgfpathmoveto{\pgfqpoint{1.401649in}{1.737289in}}%
\pgfpathcurveto{\pgfqpoint{1.409886in}{1.737289in}}{\pgfqpoint{1.417786in}{1.740562in}}{\pgfqpoint{1.423610in}{1.746385in}}%
\pgfpathcurveto{\pgfqpoint{1.429434in}{1.752209in}}{\pgfqpoint{1.432706in}{1.760109in}}{\pgfqpoint{1.432706in}{1.768346in}}%
\pgfpathcurveto{\pgfqpoint{1.432706in}{1.776582in}}{\pgfqpoint{1.429434in}{1.784482in}}{\pgfqpoint{1.423610in}{1.790306in}}%
\pgfpathcurveto{\pgfqpoint{1.417786in}{1.796130in}}{\pgfqpoint{1.409886in}{1.799402in}}{\pgfqpoint{1.401649in}{1.799402in}}%
\pgfpathcurveto{\pgfqpoint{1.393413in}{1.799402in}}{\pgfqpoint{1.385513in}{1.796130in}}{\pgfqpoint{1.379689in}{1.790306in}}%
\pgfpathcurveto{\pgfqpoint{1.373865in}{1.784482in}}{\pgfqpoint{1.370593in}{1.776582in}}{\pgfqpoint{1.370593in}{1.768346in}}%
\pgfpathcurveto{\pgfqpoint{1.370593in}{1.760109in}}{\pgfqpoint{1.373865in}{1.752209in}}{\pgfqpoint{1.379689in}{1.746385in}}%
\pgfpathcurveto{\pgfqpoint{1.385513in}{1.740562in}}{\pgfqpoint{1.393413in}{1.737289in}}{\pgfqpoint{1.401649in}{1.737289in}}%
\pgfpathclose%
\pgfusepath{stroke,fill}%
\end{pgfscope}%
\begin{pgfscope}%
\pgfpathrectangle{\pgfqpoint{0.100000in}{0.212622in}}{\pgfqpoint{3.696000in}{3.696000in}}%
\pgfusepath{clip}%
\pgfsetbuttcap%
\pgfsetroundjoin%
\definecolor{currentfill}{rgb}{0.121569,0.466667,0.705882}%
\pgfsetfillcolor{currentfill}%
\pgfsetfillopacity{0.468899}%
\pgfsetlinewidth{1.003750pt}%
\definecolor{currentstroke}{rgb}{0.121569,0.466667,0.705882}%
\pgfsetstrokecolor{currentstroke}%
\pgfsetstrokeopacity{0.468899}%
\pgfsetdash{}{0pt}%
\pgfpathmoveto{\pgfqpoint{2.033336in}{1.939288in}}%
\pgfpathcurveto{\pgfqpoint{2.041572in}{1.939288in}}{\pgfqpoint{2.049472in}{1.942560in}}{\pgfqpoint{2.055296in}{1.948384in}}%
\pgfpathcurveto{\pgfqpoint{2.061120in}{1.954208in}}{\pgfqpoint{2.064393in}{1.962108in}}{\pgfqpoint{2.064393in}{1.970344in}}%
\pgfpathcurveto{\pgfqpoint{2.064393in}{1.978581in}}{\pgfqpoint{2.061120in}{1.986481in}}{\pgfqpoint{2.055296in}{1.992304in}}%
\pgfpathcurveto{\pgfqpoint{2.049472in}{1.998128in}}{\pgfqpoint{2.041572in}{2.001401in}}{\pgfqpoint{2.033336in}{2.001401in}}%
\pgfpathcurveto{\pgfqpoint{2.025100in}{2.001401in}}{\pgfqpoint{2.017200in}{1.998128in}}{\pgfqpoint{2.011376in}{1.992304in}}%
\pgfpathcurveto{\pgfqpoint{2.005552in}{1.986481in}}{\pgfqpoint{2.002280in}{1.978581in}}{\pgfqpoint{2.002280in}{1.970344in}}%
\pgfpathcurveto{\pgfqpoint{2.002280in}{1.962108in}}{\pgfqpoint{2.005552in}{1.954208in}}{\pgfqpoint{2.011376in}{1.948384in}}%
\pgfpathcurveto{\pgfqpoint{2.017200in}{1.942560in}}{\pgfqpoint{2.025100in}{1.939288in}}{\pgfqpoint{2.033336in}{1.939288in}}%
\pgfpathclose%
\pgfusepath{stroke,fill}%
\end{pgfscope}%
\begin{pgfscope}%
\pgfpathrectangle{\pgfqpoint{0.100000in}{0.212622in}}{\pgfqpoint{3.696000in}{3.696000in}}%
\pgfusepath{clip}%
\pgfsetbuttcap%
\pgfsetroundjoin%
\definecolor{currentfill}{rgb}{0.121569,0.466667,0.705882}%
\pgfsetfillcolor{currentfill}%
\pgfsetfillopacity{0.469985}%
\pgfsetlinewidth{1.003750pt}%
\definecolor{currentstroke}{rgb}{0.121569,0.466667,0.705882}%
\pgfsetstrokecolor{currentstroke}%
\pgfsetstrokeopacity{0.469985}%
\pgfsetdash{}{0pt}%
\pgfpathmoveto{\pgfqpoint{1.394375in}{1.729325in}}%
\pgfpathcurveto{\pgfqpoint{1.402612in}{1.729325in}}{\pgfqpoint{1.410512in}{1.732597in}}{\pgfqpoint{1.416336in}{1.738421in}}%
\pgfpathcurveto{\pgfqpoint{1.422160in}{1.744245in}}{\pgfqpoint{1.425432in}{1.752145in}}{\pgfqpoint{1.425432in}{1.760381in}}%
\pgfpathcurveto{\pgfqpoint{1.425432in}{1.768617in}}{\pgfqpoint{1.422160in}{1.776517in}}{\pgfqpoint{1.416336in}{1.782341in}}%
\pgfpathcurveto{\pgfqpoint{1.410512in}{1.788165in}}{\pgfqpoint{1.402612in}{1.791438in}}{\pgfqpoint{1.394375in}{1.791438in}}%
\pgfpathcurveto{\pgfqpoint{1.386139in}{1.791438in}}{\pgfqpoint{1.378239in}{1.788165in}}{\pgfqpoint{1.372415in}{1.782341in}}%
\pgfpathcurveto{\pgfqpoint{1.366591in}{1.776517in}}{\pgfqpoint{1.363319in}{1.768617in}}{\pgfqpoint{1.363319in}{1.760381in}}%
\pgfpathcurveto{\pgfqpoint{1.363319in}{1.752145in}}{\pgfqpoint{1.366591in}{1.744245in}}{\pgfqpoint{1.372415in}{1.738421in}}%
\pgfpathcurveto{\pgfqpoint{1.378239in}{1.732597in}}{\pgfqpoint{1.386139in}{1.729325in}}{\pgfqpoint{1.394375in}{1.729325in}}%
\pgfpathclose%
\pgfusepath{stroke,fill}%
\end{pgfscope}%
\begin{pgfscope}%
\pgfpathrectangle{\pgfqpoint{0.100000in}{0.212622in}}{\pgfqpoint{3.696000in}{3.696000in}}%
\pgfusepath{clip}%
\pgfsetbuttcap%
\pgfsetroundjoin%
\definecolor{currentfill}{rgb}{0.121569,0.466667,0.705882}%
\pgfsetfillcolor{currentfill}%
\pgfsetfillopacity{0.470744}%
\pgfsetlinewidth{1.003750pt}%
\definecolor{currentstroke}{rgb}{0.121569,0.466667,0.705882}%
\pgfsetstrokecolor{currentstroke}%
\pgfsetstrokeopacity{0.470744}%
\pgfsetdash{}{0pt}%
\pgfpathmoveto{\pgfqpoint{2.034106in}{1.937739in}}%
\pgfpathcurveto{\pgfqpoint{2.042342in}{1.937739in}}{\pgfqpoint{2.050242in}{1.941012in}}{\pgfqpoint{2.056066in}{1.946835in}}%
\pgfpathcurveto{\pgfqpoint{2.061890in}{1.952659in}}{\pgfqpoint{2.065162in}{1.960559in}}{\pgfqpoint{2.065162in}{1.968796in}}%
\pgfpathcurveto{\pgfqpoint{2.065162in}{1.977032in}}{\pgfqpoint{2.061890in}{1.984932in}}{\pgfqpoint{2.056066in}{1.990756in}}%
\pgfpathcurveto{\pgfqpoint{2.050242in}{1.996580in}}{\pgfqpoint{2.042342in}{1.999852in}}{\pgfqpoint{2.034106in}{1.999852in}}%
\pgfpathcurveto{\pgfqpoint{2.025870in}{1.999852in}}{\pgfqpoint{2.017970in}{1.996580in}}{\pgfqpoint{2.012146in}{1.990756in}}%
\pgfpathcurveto{\pgfqpoint{2.006322in}{1.984932in}}{\pgfqpoint{2.003049in}{1.977032in}}{\pgfqpoint{2.003049in}{1.968796in}}%
\pgfpathcurveto{\pgfqpoint{2.003049in}{1.960559in}}{\pgfqpoint{2.006322in}{1.952659in}}{\pgfqpoint{2.012146in}{1.946835in}}%
\pgfpathcurveto{\pgfqpoint{2.017970in}{1.941012in}}{\pgfqpoint{2.025870in}{1.937739in}}{\pgfqpoint{2.034106in}{1.937739in}}%
\pgfpathclose%
\pgfusepath{stroke,fill}%
\end{pgfscope}%
\begin{pgfscope}%
\pgfpathrectangle{\pgfqpoint{0.100000in}{0.212622in}}{\pgfqpoint{3.696000in}{3.696000in}}%
\pgfusepath{clip}%
\pgfsetbuttcap%
\pgfsetroundjoin%
\definecolor{currentfill}{rgb}{0.121569,0.466667,0.705882}%
\pgfsetfillcolor{currentfill}%
\pgfsetfillopacity{0.471691}%
\pgfsetlinewidth{1.003750pt}%
\definecolor{currentstroke}{rgb}{0.121569,0.466667,0.705882}%
\pgfsetstrokecolor{currentstroke}%
\pgfsetstrokeopacity{0.471691}%
\pgfsetdash{}{0pt}%
\pgfpathmoveto{\pgfqpoint{1.388344in}{1.726349in}}%
\pgfpathcurveto{\pgfqpoint{1.396580in}{1.726349in}}{\pgfqpoint{1.404480in}{1.729621in}}{\pgfqpoint{1.410304in}{1.735445in}}%
\pgfpathcurveto{\pgfqpoint{1.416128in}{1.741269in}}{\pgfqpoint{1.419401in}{1.749169in}}{\pgfqpoint{1.419401in}{1.757405in}}%
\pgfpathcurveto{\pgfqpoint{1.419401in}{1.765641in}}{\pgfqpoint{1.416128in}{1.773541in}}{\pgfqpoint{1.410304in}{1.779365in}}%
\pgfpathcurveto{\pgfqpoint{1.404480in}{1.785189in}}{\pgfqpoint{1.396580in}{1.788462in}}{\pgfqpoint{1.388344in}{1.788462in}}%
\pgfpathcurveto{\pgfqpoint{1.380108in}{1.788462in}}{\pgfqpoint{1.372208in}{1.785189in}}{\pgfqpoint{1.366384in}{1.779365in}}%
\pgfpathcurveto{\pgfqpoint{1.360560in}{1.773541in}}{\pgfqpoint{1.357288in}{1.765641in}}{\pgfqpoint{1.357288in}{1.757405in}}%
\pgfpathcurveto{\pgfqpoint{1.357288in}{1.749169in}}{\pgfqpoint{1.360560in}{1.741269in}}{\pgfqpoint{1.366384in}{1.735445in}}%
\pgfpathcurveto{\pgfqpoint{1.372208in}{1.729621in}}{\pgfqpoint{1.380108in}{1.726349in}}{\pgfqpoint{1.388344in}{1.726349in}}%
\pgfpathclose%
\pgfusepath{stroke,fill}%
\end{pgfscope}%
\begin{pgfscope}%
\pgfpathrectangle{\pgfqpoint{0.100000in}{0.212622in}}{\pgfqpoint{3.696000in}{3.696000in}}%
\pgfusepath{clip}%
\pgfsetbuttcap%
\pgfsetroundjoin%
\definecolor{currentfill}{rgb}{0.121569,0.466667,0.705882}%
\pgfsetfillcolor{currentfill}%
\pgfsetfillopacity{0.472913}%
\pgfsetlinewidth{1.003750pt}%
\definecolor{currentstroke}{rgb}{0.121569,0.466667,0.705882}%
\pgfsetstrokecolor{currentstroke}%
\pgfsetstrokeopacity{0.472913}%
\pgfsetdash{}{0pt}%
\pgfpathmoveto{\pgfqpoint{1.384185in}{1.723918in}}%
\pgfpathcurveto{\pgfqpoint{1.392421in}{1.723918in}}{\pgfqpoint{1.400321in}{1.727190in}}{\pgfqpoint{1.406145in}{1.733014in}}%
\pgfpathcurveto{\pgfqpoint{1.411969in}{1.738838in}}{\pgfqpoint{1.415241in}{1.746738in}}{\pgfqpoint{1.415241in}{1.754974in}}%
\pgfpathcurveto{\pgfqpoint{1.415241in}{1.763210in}}{\pgfqpoint{1.411969in}{1.771110in}}{\pgfqpoint{1.406145in}{1.776934in}}%
\pgfpathcurveto{\pgfqpoint{1.400321in}{1.782758in}}{\pgfqpoint{1.392421in}{1.786031in}}{\pgfqpoint{1.384185in}{1.786031in}}%
\pgfpathcurveto{\pgfqpoint{1.375948in}{1.786031in}}{\pgfqpoint{1.368048in}{1.782758in}}{\pgfqpoint{1.362224in}{1.776934in}}%
\pgfpathcurveto{\pgfqpoint{1.356400in}{1.771110in}}{\pgfqpoint{1.353128in}{1.763210in}}{\pgfqpoint{1.353128in}{1.754974in}}%
\pgfpathcurveto{\pgfqpoint{1.353128in}{1.746738in}}{\pgfqpoint{1.356400in}{1.738838in}}{\pgfqpoint{1.362224in}{1.733014in}}%
\pgfpathcurveto{\pgfqpoint{1.368048in}{1.727190in}}{\pgfqpoint{1.375948in}{1.723918in}}{\pgfqpoint{1.384185in}{1.723918in}}%
\pgfpathclose%
\pgfusepath{stroke,fill}%
\end{pgfscope}%
\begin{pgfscope}%
\pgfpathrectangle{\pgfqpoint{0.100000in}{0.212622in}}{\pgfqpoint{3.696000in}{3.696000in}}%
\pgfusepath{clip}%
\pgfsetbuttcap%
\pgfsetroundjoin%
\definecolor{currentfill}{rgb}{0.121569,0.466667,0.705882}%
\pgfsetfillcolor{currentfill}%
\pgfsetfillopacity{0.473104}%
\pgfsetlinewidth{1.003750pt}%
\definecolor{currentstroke}{rgb}{0.121569,0.466667,0.705882}%
\pgfsetstrokecolor{currentstroke}%
\pgfsetstrokeopacity{0.473104}%
\pgfsetdash{}{0pt}%
\pgfpathmoveto{\pgfqpoint{2.035128in}{1.936838in}}%
\pgfpathcurveto{\pgfqpoint{2.043365in}{1.936838in}}{\pgfqpoint{2.051265in}{1.940110in}}{\pgfqpoint{2.057089in}{1.945934in}}%
\pgfpathcurveto{\pgfqpoint{2.062913in}{1.951758in}}{\pgfqpoint{2.066185in}{1.959658in}}{\pgfqpoint{2.066185in}{1.967895in}}%
\pgfpathcurveto{\pgfqpoint{2.066185in}{1.976131in}}{\pgfqpoint{2.062913in}{1.984031in}}{\pgfqpoint{2.057089in}{1.989855in}}%
\pgfpathcurveto{\pgfqpoint{2.051265in}{1.995679in}}{\pgfqpoint{2.043365in}{1.998951in}}{\pgfqpoint{2.035128in}{1.998951in}}%
\pgfpathcurveto{\pgfqpoint{2.026892in}{1.998951in}}{\pgfqpoint{2.018992in}{1.995679in}}{\pgfqpoint{2.013168in}{1.989855in}}%
\pgfpathcurveto{\pgfqpoint{2.007344in}{1.984031in}}{\pgfqpoint{2.004072in}{1.976131in}}{\pgfqpoint{2.004072in}{1.967895in}}%
\pgfpathcurveto{\pgfqpoint{2.004072in}{1.959658in}}{\pgfqpoint{2.007344in}{1.951758in}}{\pgfqpoint{2.013168in}{1.945934in}}%
\pgfpathcurveto{\pgfqpoint{2.018992in}{1.940110in}}{\pgfqpoint{2.026892in}{1.936838in}}{\pgfqpoint{2.035128in}{1.936838in}}%
\pgfpathclose%
\pgfusepath{stroke,fill}%
\end{pgfscope}%
\begin{pgfscope}%
\pgfpathrectangle{\pgfqpoint{0.100000in}{0.212622in}}{\pgfqpoint{3.696000in}{3.696000in}}%
\pgfusepath{clip}%
\pgfsetbuttcap%
\pgfsetroundjoin%
\definecolor{currentfill}{rgb}{0.121569,0.466667,0.705882}%
\pgfsetfillcolor{currentfill}%
\pgfsetfillopacity{0.475035}%
\pgfsetlinewidth{1.003750pt}%
\definecolor{currentstroke}{rgb}{0.121569,0.466667,0.705882}%
\pgfsetstrokecolor{currentstroke}%
\pgfsetstrokeopacity{0.475035}%
\pgfsetdash{}{0pt}%
\pgfpathmoveto{\pgfqpoint{1.377304in}{1.717839in}}%
\pgfpathcurveto{\pgfqpoint{1.385540in}{1.717839in}}{\pgfqpoint{1.393440in}{1.721112in}}{\pgfqpoint{1.399264in}{1.726936in}}%
\pgfpathcurveto{\pgfqpoint{1.405088in}{1.732760in}}{\pgfqpoint{1.408361in}{1.740660in}}{\pgfqpoint{1.408361in}{1.748896in}}%
\pgfpathcurveto{\pgfqpoint{1.408361in}{1.757132in}}{\pgfqpoint{1.405088in}{1.765032in}}{\pgfqpoint{1.399264in}{1.770856in}}%
\pgfpathcurveto{\pgfqpoint{1.393440in}{1.776680in}}{\pgfqpoint{1.385540in}{1.779952in}}{\pgfqpoint{1.377304in}{1.779952in}}%
\pgfpathcurveto{\pgfqpoint{1.369068in}{1.779952in}}{\pgfqpoint{1.361168in}{1.776680in}}{\pgfqpoint{1.355344in}{1.770856in}}%
\pgfpathcurveto{\pgfqpoint{1.349520in}{1.765032in}}{\pgfqpoint{1.346248in}{1.757132in}}{\pgfqpoint{1.346248in}{1.748896in}}%
\pgfpathcurveto{\pgfqpoint{1.346248in}{1.740660in}}{\pgfqpoint{1.349520in}{1.732760in}}{\pgfqpoint{1.355344in}{1.726936in}}%
\pgfpathcurveto{\pgfqpoint{1.361168in}{1.721112in}}{\pgfqpoint{1.369068in}{1.717839in}}{\pgfqpoint{1.377304in}{1.717839in}}%
\pgfpathclose%
\pgfusepath{stroke,fill}%
\end{pgfscope}%
\begin{pgfscope}%
\pgfpathrectangle{\pgfqpoint{0.100000in}{0.212622in}}{\pgfqpoint{3.696000in}{3.696000in}}%
\pgfusepath{clip}%
\pgfsetbuttcap%
\pgfsetroundjoin%
\definecolor{currentfill}{rgb}{0.121569,0.466667,0.705882}%
\pgfsetfillcolor{currentfill}%
\pgfsetfillopacity{0.475589}%
\pgfsetlinewidth{1.003750pt}%
\definecolor{currentstroke}{rgb}{0.121569,0.466667,0.705882}%
\pgfsetstrokecolor{currentstroke}%
\pgfsetstrokeopacity{0.475589}%
\pgfsetdash{}{0pt}%
\pgfpathmoveto{\pgfqpoint{2.036130in}{1.934662in}}%
\pgfpathcurveto{\pgfqpoint{2.044366in}{1.934662in}}{\pgfqpoint{2.052266in}{1.937934in}}{\pgfqpoint{2.058090in}{1.943758in}}%
\pgfpathcurveto{\pgfqpoint{2.063914in}{1.949582in}}{\pgfqpoint{2.067186in}{1.957482in}}{\pgfqpoint{2.067186in}{1.965718in}}%
\pgfpathcurveto{\pgfqpoint{2.067186in}{1.973954in}}{\pgfqpoint{2.063914in}{1.981854in}}{\pgfqpoint{2.058090in}{1.987678in}}%
\pgfpathcurveto{\pgfqpoint{2.052266in}{1.993502in}}{\pgfqpoint{2.044366in}{1.996775in}}{\pgfqpoint{2.036130in}{1.996775in}}%
\pgfpathcurveto{\pgfqpoint{2.027894in}{1.996775in}}{\pgfqpoint{2.019994in}{1.993502in}}{\pgfqpoint{2.014170in}{1.987678in}}%
\pgfpathcurveto{\pgfqpoint{2.008346in}{1.981854in}}{\pgfqpoint{2.005073in}{1.973954in}}{\pgfqpoint{2.005073in}{1.965718in}}%
\pgfpathcurveto{\pgfqpoint{2.005073in}{1.957482in}}{\pgfqpoint{2.008346in}{1.949582in}}{\pgfqpoint{2.014170in}{1.943758in}}%
\pgfpathcurveto{\pgfqpoint{2.019994in}{1.937934in}}{\pgfqpoint{2.027894in}{1.934662in}}{\pgfqpoint{2.036130in}{1.934662in}}%
\pgfpathclose%
\pgfusepath{stroke,fill}%
\end{pgfscope}%
\begin{pgfscope}%
\pgfpathrectangle{\pgfqpoint{0.100000in}{0.212622in}}{\pgfqpoint{3.696000in}{3.696000in}}%
\pgfusepath{clip}%
\pgfsetbuttcap%
\pgfsetroundjoin%
\definecolor{currentfill}{rgb}{0.121569,0.466667,0.705882}%
\pgfsetfillcolor{currentfill}%
\pgfsetfillopacity{0.476462}%
\pgfsetlinewidth{1.003750pt}%
\definecolor{currentstroke}{rgb}{0.121569,0.466667,0.705882}%
\pgfsetstrokecolor{currentstroke}%
\pgfsetstrokeopacity{0.476462}%
\pgfsetdash{}{0pt}%
\pgfpathmoveto{\pgfqpoint{1.372239in}{1.713554in}}%
\pgfpathcurveto{\pgfqpoint{1.380476in}{1.713554in}}{\pgfqpoint{1.388376in}{1.716826in}}{\pgfqpoint{1.394200in}{1.722650in}}%
\pgfpathcurveto{\pgfqpoint{1.400024in}{1.728474in}}{\pgfqpoint{1.403296in}{1.736374in}}{\pgfqpoint{1.403296in}{1.744610in}}%
\pgfpathcurveto{\pgfqpoint{1.403296in}{1.752847in}}{\pgfqpoint{1.400024in}{1.760747in}}{\pgfqpoint{1.394200in}{1.766571in}}%
\pgfpathcurveto{\pgfqpoint{1.388376in}{1.772394in}}{\pgfqpoint{1.380476in}{1.775667in}}{\pgfqpoint{1.372239in}{1.775667in}}%
\pgfpathcurveto{\pgfqpoint{1.364003in}{1.775667in}}{\pgfqpoint{1.356103in}{1.772394in}}{\pgfqpoint{1.350279in}{1.766571in}}%
\pgfpathcurveto{\pgfqpoint{1.344455in}{1.760747in}}{\pgfqpoint{1.341183in}{1.752847in}}{\pgfqpoint{1.341183in}{1.744610in}}%
\pgfpathcurveto{\pgfqpoint{1.341183in}{1.736374in}}{\pgfqpoint{1.344455in}{1.728474in}}{\pgfqpoint{1.350279in}{1.722650in}}%
\pgfpathcurveto{\pgfqpoint{1.356103in}{1.716826in}}{\pgfqpoint{1.364003in}{1.713554in}}{\pgfqpoint{1.372239in}{1.713554in}}%
\pgfpathclose%
\pgfusepath{stroke,fill}%
\end{pgfscope}%
\begin{pgfscope}%
\pgfpathrectangle{\pgfqpoint{0.100000in}{0.212622in}}{\pgfqpoint{3.696000in}{3.696000in}}%
\pgfusepath{clip}%
\pgfsetbuttcap%
\pgfsetroundjoin%
\definecolor{currentfill}{rgb}{0.121569,0.466667,0.705882}%
\pgfsetfillcolor{currentfill}%
\pgfsetfillopacity{0.477662}%
\pgfsetlinewidth{1.003750pt}%
\definecolor{currentstroke}{rgb}{0.121569,0.466667,0.705882}%
\pgfsetstrokecolor{currentstroke}%
\pgfsetstrokeopacity{0.477662}%
\pgfsetdash{}{0pt}%
\pgfpathmoveto{\pgfqpoint{1.368422in}{1.711226in}}%
\pgfpathcurveto{\pgfqpoint{1.376659in}{1.711226in}}{\pgfqpoint{1.384559in}{1.714498in}}{\pgfqpoint{1.390383in}{1.720322in}}%
\pgfpathcurveto{\pgfqpoint{1.396207in}{1.726146in}}{\pgfqpoint{1.399479in}{1.734046in}}{\pgfqpoint{1.399479in}{1.742282in}}%
\pgfpathcurveto{\pgfqpoint{1.399479in}{1.750518in}}{\pgfqpoint{1.396207in}{1.758418in}}{\pgfqpoint{1.390383in}{1.764242in}}%
\pgfpathcurveto{\pgfqpoint{1.384559in}{1.770066in}}{\pgfqpoint{1.376659in}{1.773339in}}{\pgfqpoint{1.368422in}{1.773339in}}%
\pgfpathcurveto{\pgfqpoint{1.360186in}{1.773339in}}{\pgfqpoint{1.352286in}{1.770066in}}{\pgfqpoint{1.346462in}{1.764242in}}%
\pgfpathcurveto{\pgfqpoint{1.340638in}{1.758418in}}{\pgfqpoint{1.337366in}{1.750518in}}{\pgfqpoint{1.337366in}{1.742282in}}%
\pgfpathcurveto{\pgfqpoint{1.337366in}{1.734046in}}{\pgfqpoint{1.340638in}{1.726146in}}{\pgfqpoint{1.346462in}{1.720322in}}%
\pgfpathcurveto{\pgfqpoint{1.352286in}{1.714498in}}{\pgfqpoint{1.360186in}{1.711226in}}{\pgfqpoint{1.368422in}{1.711226in}}%
\pgfpathclose%
\pgfusepath{stroke,fill}%
\end{pgfscope}%
\begin{pgfscope}%
\pgfpathrectangle{\pgfqpoint{0.100000in}{0.212622in}}{\pgfqpoint{3.696000in}{3.696000in}}%
\pgfusepath{clip}%
\pgfsetbuttcap%
\pgfsetroundjoin%
\definecolor{currentfill}{rgb}{0.121569,0.466667,0.705882}%
\pgfsetfillcolor{currentfill}%
\pgfsetfillopacity{0.478446}%
\pgfsetlinewidth{1.003750pt}%
\definecolor{currentstroke}{rgb}{0.121569,0.466667,0.705882}%
\pgfsetstrokecolor{currentstroke}%
\pgfsetstrokeopacity{0.478446}%
\pgfsetdash{}{0pt}%
\pgfpathmoveto{\pgfqpoint{2.037059in}{1.932213in}}%
\pgfpathcurveto{\pgfqpoint{2.045296in}{1.932213in}}{\pgfqpoint{2.053196in}{1.935485in}}{\pgfqpoint{2.059020in}{1.941309in}}%
\pgfpathcurveto{\pgfqpoint{2.064843in}{1.947133in}}{\pgfqpoint{2.068116in}{1.955033in}}{\pgfqpoint{2.068116in}{1.963269in}}%
\pgfpathcurveto{\pgfqpoint{2.068116in}{1.971506in}}{\pgfqpoint{2.064843in}{1.979406in}}{\pgfqpoint{2.059020in}{1.985230in}}%
\pgfpathcurveto{\pgfqpoint{2.053196in}{1.991053in}}{\pgfqpoint{2.045296in}{1.994326in}}{\pgfqpoint{2.037059in}{1.994326in}}%
\pgfpathcurveto{\pgfqpoint{2.028823in}{1.994326in}}{\pgfqpoint{2.020923in}{1.991053in}}{\pgfqpoint{2.015099in}{1.985230in}}%
\pgfpathcurveto{\pgfqpoint{2.009275in}{1.979406in}}{\pgfqpoint{2.006003in}{1.971506in}}{\pgfqpoint{2.006003in}{1.963269in}}%
\pgfpathcurveto{\pgfqpoint{2.006003in}{1.955033in}}{\pgfqpoint{2.009275in}{1.947133in}}{\pgfqpoint{2.015099in}{1.941309in}}%
\pgfpathcurveto{\pgfqpoint{2.020923in}{1.935485in}}{\pgfqpoint{2.028823in}{1.932213in}}{\pgfqpoint{2.037059in}{1.932213in}}%
\pgfpathclose%
\pgfusepath{stroke,fill}%
\end{pgfscope}%
\begin{pgfscope}%
\pgfpathrectangle{\pgfqpoint{0.100000in}{0.212622in}}{\pgfqpoint{3.696000in}{3.696000in}}%
\pgfusepath{clip}%
\pgfsetbuttcap%
\pgfsetroundjoin%
\definecolor{currentfill}{rgb}{0.121569,0.466667,0.705882}%
\pgfsetfillcolor{currentfill}%
\pgfsetfillopacity{0.478505}%
\pgfsetlinewidth{1.003750pt}%
\definecolor{currentstroke}{rgb}{0.121569,0.466667,0.705882}%
\pgfsetstrokecolor{currentstroke}%
\pgfsetstrokeopacity{0.478505}%
\pgfsetdash{}{0pt}%
\pgfpathmoveto{\pgfqpoint{1.365606in}{1.708483in}}%
\pgfpathcurveto{\pgfqpoint{1.373842in}{1.708483in}}{\pgfqpoint{1.381742in}{1.711755in}}{\pgfqpoint{1.387566in}{1.717579in}}%
\pgfpathcurveto{\pgfqpoint{1.393390in}{1.723403in}}{\pgfqpoint{1.396662in}{1.731303in}}{\pgfqpoint{1.396662in}{1.739539in}}%
\pgfpathcurveto{\pgfqpoint{1.396662in}{1.747775in}}{\pgfqpoint{1.393390in}{1.755675in}}{\pgfqpoint{1.387566in}{1.761499in}}%
\pgfpathcurveto{\pgfqpoint{1.381742in}{1.767323in}}{\pgfqpoint{1.373842in}{1.770596in}}{\pgfqpoint{1.365606in}{1.770596in}}%
\pgfpathcurveto{\pgfqpoint{1.357369in}{1.770596in}}{\pgfqpoint{1.349469in}{1.767323in}}{\pgfqpoint{1.343645in}{1.761499in}}%
\pgfpathcurveto{\pgfqpoint{1.337821in}{1.755675in}}{\pgfqpoint{1.334549in}{1.747775in}}{\pgfqpoint{1.334549in}{1.739539in}}%
\pgfpathcurveto{\pgfqpoint{1.334549in}{1.731303in}}{\pgfqpoint{1.337821in}{1.723403in}}{\pgfqpoint{1.343645in}{1.717579in}}%
\pgfpathcurveto{\pgfqpoint{1.349469in}{1.711755in}}{\pgfqpoint{1.357369in}{1.708483in}}{\pgfqpoint{1.365606in}{1.708483in}}%
\pgfpathclose%
\pgfusepath{stroke,fill}%
\end{pgfscope}%
\begin{pgfscope}%
\pgfpathrectangle{\pgfqpoint{0.100000in}{0.212622in}}{\pgfqpoint{3.696000in}{3.696000in}}%
\pgfusepath{clip}%
\pgfsetbuttcap%
\pgfsetroundjoin%
\definecolor{currentfill}{rgb}{0.121569,0.466667,0.705882}%
\pgfsetfillcolor{currentfill}%
\pgfsetfillopacity{0.479315}%
\pgfsetlinewidth{1.003750pt}%
\definecolor{currentstroke}{rgb}{0.121569,0.466667,0.705882}%
\pgfsetstrokecolor{currentstroke}%
\pgfsetstrokeopacity{0.479315}%
\pgfsetdash{}{0pt}%
\pgfpathmoveto{\pgfqpoint{1.362901in}{1.707154in}}%
\pgfpathcurveto{\pgfqpoint{1.371137in}{1.707154in}}{\pgfqpoint{1.379037in}{1.710426in}}{\pgfqpoint{1.384861in}{1.716250in}}%
\pgfpathcurveto{\pgfqpoint{1.390685in}{1.722074in}}{\pgfqpoint{1.393957in}{1.729974in}}{\pgfqpoint{1.393957in}{1.738211in}}%
\pgfpathcurveto{\pgfqpoint{1.393957in}{1.746447in}}{\pgfqpoint{1.390685in}{1.754347in}}{\pgfqpoint{1.384861in}{1.760171in}}%
\pgfpathcurveto{\pgfqpoint{1.379037in}{1.765995in}}{\pgfqpoint{1.371137in}{1.769267in}}{\pgfqpoint{1.362901in}{1.769267in}}%
\pgfpathcurveto{\pgfqpoint{1.354664in}{1.769267in}}{\pgfqpoint{1.346764in}{1.765995in}}{\pgfqpoint{1.340940in}{1.760171in}}%
\pgfpathcurveto{\pgfqpoint{1.335116in}{1.754347in}}{\pgfqpoint{1.331844in}{1.746447in}}{\pgfqpoint{1.331844in}{1.738211in}}%
\pgfpathcurveto{\pgfqpoint{1.331844in}{1.729974in}}{\pgfqpoint{1.335116in}{1.722074in}}{\pgfqpoint{1.340940in}{1.716250in}}%
\pgfpathcurveto{\pgfqpoint{1.346764in}{1.710426in}}{\pgfqpoint{1.354664in}{1.707154in}}{\pgfqpoint{1.362901in}{1.707154in}}%
\pgfpathclose%
\pgfusepath{stroke,fill}%
\end{pgfscope}%
\begin{pgfscope}%
\pgfpathrectangle{\pgfqpoint{0.100000in}{0.212622in}}{\pgfqpoint{3.696000in}{3.696000in}}%
\pgfusepath{clip}%
\pgfsetbuttcap%
\pgfsetroundjoin%
\definecolor{currentfill}{rgb}{0.121569,0.466667,0.705882}%
\pgfsetfillcolor{currentfill}%
\pgfsetfillopacity{0.479813}%
\pgfsetlinewidth{1.003750pt}%
\definecolor{currentstroke}{rgb}{0.121569,0.466667,0.705882}%
\pgfsetstrokecolor{currentstroke}%
\pgfsetstrokeopacity{0.479813}%
\pgfsetdash{}{0pt}%
\pgfpathmoveto{\pgfqpoint{2.038490in}{1.929963in}}%
\pgfpathcurveto{\pgfqpoint{2.046727in}{1.929963in}}{\pgfqpoint{2.054627in}{1.933236in}}{\pgfqpoint{2.060451in}{1.939059in}}%
\pgfpathcurveto{\pgfqpoint{2.066275in}{1.944883in}}{\pgfqpoint{2.069547in}{1.952783in}}{\pgfqpoint{2.069547in}{1.961020in}}%
\pgfpathcurveto{\pgfqpoint{2.069547in}{1.969256in}}{\pgfqpoint{2.066275in}{1.977156in}}{\pgfqpoint{2.060451in}{1.982980in}}%
\pgfpathcurveto{\pgfqpoint{2.054627in}{1.988804in}}{\pgfqpoint{2.046727in}{1.992076in}}{\pgfqpoint{2.038490in}{1.992076in}}%
\pgfpathcurveto{\pgfqpoint{2.030254in}{1.992076in}}{\pgfqpoint{2.022354in}{1.988804in}}{\pgfqpoint{2.016530in}{1.982980in}}%
\pgfpathcurveto{\pgfqpoint{2.010706in}{1.977156in}}{\pgfqpoint{2.007434in}{1.969256in}}{\pgfqpoint{2.007434in}{1.961020in}}%
\pgfpathcurveto{\pgfqpoint{2.007434in}{1.952783in}}{\pgfqpoint{2.010706in}{1.944883in}}{\pgfqpoint{2.016530in}{1.939059in}}%
\pgfpathcurveto{\pgfqpoint{2.022354in}{1.933236in}}{\pgfqpoint{2.030254in}{1.929963in}}{\pgfqpoint{2.038490in}{1.929963in}}%
\pgfpathclose%
\pgfusepath{stroke,fill}%
\end{pgfscope}%
\begin{pgfscope}%
\pgfpathrectangle{\pgfqpoint{0.100000in}{0.212622in}}{\pgfqpoint{3.696000in}{3.696000in}}%
\pgfusepath{clip}%
\pgfsetbuttcap%
\pgfsetroundjoin%
\definecolor{currentfill}{rgb}{0.121569,0.466667,0.705882}%
\pgfsetfillcolor{currentfill}%
\pgfsetfillopacity{0.480787}%
\pgfsetlinewidth{1.003750pt}%
\definecolor{currentstroke}{rgb}{0.121569,0.466667,0.705882}%
\pgfsetstrokecolor{currentstroke}%
\pgfsetstrokeopacity{0.480787}%
\pgfsetdash{}{0pt}%
\pgfpathmoveto{\pgfqpoint{1.358011in}{1.704680in}}%
\pgfpathcurveto{\pgfqpoint{1.366247in}{1.704680in}}{\pgfqpoint{1.374147in}{1.707952in}}{\pgfqpoint{1.379971in}{1.713776in}}%
\pgfpathcurveto{\pgfqpoint{1.385795in}{1.719600in}}{\pgfqpoint{1.389068in}{1.727500in}}{\pgfqpoint{1.389068in}{1.735736in}}%
\pgfpathcurveto{\pgfqpoint{1.389068in}{1.743973in}}{\pgfqpoint{1.385795in}{1.751873in}}{\pgfqpoint{1.379971in}{1.757697in}}%
\pgfpathcurveto{\pgfqpoint{1.374147in}{1.763521in}}{\pgfqpoint{1.366247in}{1.766793in}}{\pgfqpoint{1.358011in}{1.766793in}}%
\pgfpathcurveto{\pgfqpoint{1.349775in}{1.766793in}}{\pgfqpoint{1.341875in}{1.763521in}}{\pgfqpoint{1.336051in}{1.757697in}}%
\pgfpathcurveto{\pgfqpoint{1.330227in}{1.751873in}}{\pgfqpoint{1.326955in}{1.743973in}}{\pgfqpoint{1.326955in}{1.735736in}}%
\pgfpathcurveto{\pgfqpoint{1.326955in}{1.727500in}}{\pgfqpoint{1.330227in}{1.719600in}}{\pgfqpoint{1.336051in}{1.713776in}}%
\pgfpathcurveto{\pgfqpoint{1.341875in}{1.707952in}}{\pgfqpoint{1.349775in}{1.704680in}}{\pgfqpoint{1.358011in}{1.704680in}}%
\pgfpathclose%
\pgfusepath{stroke,fill}%
\end{pgfscope}%
\begin{pgfscope}%
\pgfpathrectangle{\pgfqpoint{0.100000in}{0.212622in}}{\pgfqpoint{3.696000in}{3.696000in}}%
\pgfusepath{clip}%
\pgfsetbuttcap%
\pgfsetroundjoin%
\definecolor{currentfill}{rgb}{0.121569,0.466667,0.705882}%
\pgfsetfillcolor{currentfill}%
\pgfsetfillopacity{0.481835}%
\pgfsetlinewidth{1.003750pt}%
\definecolor{currentstroke}{rgb}{0.121569,0.466667,0.705882}%
\pgfsetstrokecolor{currentstroke}%
\pgfsetstrokeopacity{0.481835}%
\pgfsetdash{}{0pt}%
\pgfpathmoveto{\pgfqpoint{2.039437in}{1.928920in}}%
\pgfpathcurveto{\pgfqpoint{2.047674in}{1.928920in}}{\pgfqpoint{2.055574in}{1.932193in}}{\pgfqpoint{2.061398in}{1.938017in}}%
\pgfpathcurveto{\pgfqpoint{2.067222in}{1.943840in}}{\pgfqpoint{2.070494in}{1.951741in}}{\pgfqpoint{2.070494in}{1.959977in}}%
\pgfpathcurveto{\pgfqpoint{2.070494in}{1.968213in}}{\pgfqpoint{2.067222in}{1.976113in}}{\pgfqpoint{2.061398in}{1.981937in}}%
\pgfpathcurveto{\pgfqpoint{2.055574in}{1.987761in}}{\pgfqpoint{2.047674in}{1.991033in}}{\pgfqpoint{2.039437in}{1.991033in}}%
\pgfpathcurveto{\pgfqpoint{2.031201in}{1.991033in}}{\pgfqpoint{2.023301in}{1.987761in}}{\pgfqpoint{2.017477in}{1.981937in}}%
\pgfpathcurveto{\pgfqpoint{2.011653in}{1.976113in}}{\pgfqpoint{2.008381in}{1.968213in}}{\pgfqpoint{2.008381in}{1.959977in}}%
\pgfpathcurveto{\pgfqpoint{2.008381in}{1.951741in}}{\pgfqpoint{2.011653in}{1.943840in}}{\pgfqpoint{2.017477in}{1.938017in}}%
\pgfpathcurveto{\pgfqpoint{2.023301in}{1.932193in}}{\pgfqpoint{2.031201in}{1.928920in}}{\pgfqpoint{2.039437in}{1.928920in}}%
\pgfpathclose%
\pgfusepath{stroke,fill}%
\end{pgfscope}%
\begin{pgfscope}%
\pgfpathrectangle{\pgfqpoint{0.100000in}{0.212622in}}{\pgfqpoint{3.696000in}{3.696000in}}%
\pgfusepath{clip}%
\pgfsetbuttcap%
\pgfsetroundjoin%
\definecolor{currentfill}{rgb}{0.121569,0.466667,0.705882}%
\pgfsetfillcolor{currentfill}%
\pgfsetfillopacity{0.483303}%
\pgfsetlinewidth{1.003750pt}%
\definecolor{currentstroke}{rgb}{0.121569,0.466667,0.705882}%
\pgfsetstrokecolor{currentstroke}%
\pgfsetstrokeopacity{0.483303}%
\pgfsetdash{}{0pt}%
\pgfpathmoveto{\pgfqpoint{1.350150in}{1.697623in}}%
\pgfpathcurveto{\pgfqpoint{1.358387in}{1.697623in}}{\pgfqpoint{1.366287in}{1.700895in}}{\pgfqpoint{1.372111in}{1.706719in}}%
\pgfpathcurveto{\pgfqpoint{1.377935in}{1.712543in}}{\pgfqpoint{1.381207in}{1.720443in}}{\pgfqpoint{1.381207in}{1.728679in}}%
\pgfpathcurveto{\pgfqpoint{1.381207in}{1.736916in}}{\pgfqpoint{1.377935in}{1.744816in}}{\pgfqpoint{1.372111in}{1.750640in}}%
\pgfpathcurveto{\pgfqpoint{1.366287in}{1.756464in}}{\pgfqpoint{1.358387in}{1.759736in}}{\pgfqpoint{1.350150in}{1.759736in}}%
\pgfpathcurveto{\pgfqpoint{1.341914in}{1.759736in}}{\pgfqpoint{1.334014in}{1.756464in}}{\pgfqpoint{1.328190in}{1.750640in}}%
\pgfpathcurveto{\pgfqpoint{1.322366in}{1.744816in}}{\pgfqpoint{1.319094in}{1.736916in}}{\pgfqpoint{1.319094in}{1.728679in}}%
\pgfpathcurveto{\pgfqpoint{1.319094in}{1.720443in}}{\pgfqpoint{1.322366in}{1.712543in}}{\pgfqpoint{1.328190in}{1.706719in}}%
\pgfpathcurveto{\pgfqpoint{1.334014in}{1.700895in}}{\pgfqpoint{1.341914in}{1.697623in}}{\pgfqpoint{1.350150in}{1.697623in}}%
\pgfpathclose%
\pgfusepath{stroke,fill}%
\end{pgfscope}%
\begin{pgfscope}%
\pgfpathrectangle{\pgfqpoint{0.100000in}{0.212622in}}{\pgfqpoint{3.696000in}{3.696000in}}%
\pgfusepath{clip}%
\pgfsetbuttcap%
\pgfsetroundjoin%
\definecolor{currentfill}{rgb}{0.121569,0.466667,0.705882}%
\pgfsetfillcolor{currentfill}%
\pgfsetfillopacity{0.484439}%
\pgfsetlinewidth{1.003750pt}%
\definecolor{currentstroke}{rgb}{0.121569,0.466667,0.705882}%
\pgfsetstrokecolor{currentstroke}%
\pgfsetstrokeopacity{0.484439}%
\pgfsetdash{}{0pt}%
\pgfpathmoveto{\pgfqpoint{2.040730in}{1.926177in}}%
\pgfpathcurveto{\pgfqpoint{2.048966in}{1.926177in}}{\pgfqpoint{2.056866in}{1.929449in}}{\pgfqpoint{2.062690in}{1.935273in}}%
\pgfpathcurveto{\pgfqpoint{2.068514in}{1.941097in}}{\pgfqpoint{2.071787in}{1.948997in}}{\pgfqpoint{2.071787in}{1.957233in}}%
\pgfpathcurveto{\pgfqpoint{2.071787in}{1.965469in}}{\pgfqpoint{2.068514in}{1.973369in}}{\pgfqpoint{2.062690in}{1.979193in}}%
\pgfpathcurveto{\pgfqpoint{2.056866in}{1.985017in}}{\pgfqpoint{2.048966in}{1.988290in}}{\pgfqpoint{2.040730in}{1.988290in}}%
\pgfpathcurveto{\pgfqpoint{2.032494in}{1.988290in}}{\pgfqpoint{2.024594in}{1.985017in}}{\pgfqpoint{2.018770in}{1.979193in}}%
\pgfpathcurveto{\pgfqpoint{2.012946in}{1.973369in}}{\pgfqpoint{2.009674in}{1.965469in}}{\pgfqpoint{2.009674in}{1.957233in}}%
\pgfpathcurveto{\pgfqpoint{2.009674in}{1.948997in}}{\pgfqpoint{2.012946in}{1.941097in}}{\pgfqpoint{2.018770in}{1.935273in}}%
\pgfpathcurveto{\pgfqpoint{2.024594in}{1.929449in}}{\pgfqpoint{2.032494in}{1.926177in}}{\pgfqpoint{2.040730in}{1.926177in}}%
\pgfpathclose%
\pgfusepath{stroke,fill}%
\end{pgfscope}%
\begin{pgfscope}%
\pgfpathrectangle{\pgfqpoint{0.100000in}{0.212622in}}{\pgfqpoint{3.696000in}{3.696000in}}%
\pgfusepath{clip}%
\pgfsetbuttcap%
\pgfsetroundjoin%
\definecolor{currentfill}{rgb}{0.121569,0.466667,0.705882}%
\pgfsetfillcolor{currentfill}%
\pgfsetfillopacity{0.484905}%
\pgfsetlinewidth{1.003750pt}%
\definecolor{currentstroke}{rgb}{0.121569,0.466667,0.705882}%
\pgfsetstrokecolor{currentstroke}%
\pgfsetstrokeopacity{0.484905}%
\pgfsetdash{}{0pt}%
\pgfpathmoveto{\pgfqpoint{1.344845in}{1.693317in}}%
\pgfpathcurveto{\pgfqpoint{1.353081in}{1.693317in}}{\pgfqpoint{1.360981in}{1.696590in}}{\pgfqpoint{1.366805in}{1.702414in}}%
\pgfpathcurveto{\pgfqpoint{1.372629in}{1.708237in}}{\pgfqpoint{1.375901in}{1.716138in}}{\pgfqpoint{1.375901in}{1.724374in}}%
\pgfpathcurveto{\pgfqpoint{1.375901in}{1.732610in}}{\pgfqpoint{1.372629in}{1.740510in}}{\pgfqpoint{1.366805in}{1.746334in}}%
\pgfpathcurveto{\pgfqpoint{1.360981in}{1.752158in}}{\pgfqpoint{1.353081in}{1.755430in}}{\pgfqpoint{1.344845in}{1.755430in}}%
\pgfpathcurveto{\pgfqpoint{1.336608in}{1.755430in}}{\pgfqpoint{1.328708in}{1.752158in}}{\pgfqpoint{1.322884in}{1.746334in}}%
\pgfpathcurveto{\pgfqpoint{1.317061in}{1.740510in}}{\pgfqpoint{1.313788in}{1.732610in}}{\pgfqpoint{1.313788in}{1.724374in}}%
\pgfpathcurveto{\pgfqpoint{1.313788in}{1.716138in}}{\pgfqpoint{1.317061in}{1.708237in}}{\pgfqpoint{1.322884in}{1.702414in}}%
\pgfpathcurveto{\pgfqpoint{1.328708in}{1.696590in}}{\pgfqpoint{1.336608in}{1.693317in}}{\pgfqpoint{1.344845in}{1.693317in}}%
\pgfpathclose%
\pgfusepath{stroke,fill}%
\end{pgfscope}%
\begin{pgfscope}%
\pgfpathrectangle{\pgfqpoint{0.100000in}{0.212622in}}{\pgfqpoint{3.696000in}{3.696000in}}%
\pgfusepath{clip}%
\pgfsetbuttcap%
\pgfsetroundjoin%
\definecolor{currentfill}{rgb}{0.121569,0.466667,0.705882}%
\pgfsetfillcolor{currentfill}%
\pgfsetfillopacity{0.486244}%
\pgfsetlinewidth{1.003750pt}%
\definecolor{currentstroke}{rgb}{0.121569,0.466667,0.705882}%
\pgfsetstrokecolor{currentstroke}%
\pgfsetstrokeopacity{0.486244}%
\pgfsetdash{}{0pt}%
\pgfpathmoveto{\pgfqpoint{1.340577in}{1.691077in}}%
\pgfpathcurveto{\pgfqpoint{1.348813in}{1.691077in}}{\pgfqpoint{1.356713in}{1.694350in}}{\pgfqpoint{1.362537in}{1.700174in}}%
\pgfpathcurveto{\pgfqpoint{1.368361in}{1.705997in}}{\pgfqpoint{1.371633in}{1.713898in}}{\pgfqpoint{1.371633in}{1.722134in}}%
\pgfpathcurveto{\pgfqpoint{1.371633in}{1.730370in}}{\pgfqpoint{1.368361in}{1.738270in}}{\pgfqpoint{1.362537in}{1.744094in}}%
\pgfpathcurveto{\pgfqpoint{1.356713in}{1.749918in}}{\pgfqpoint{1.348813in}{1.753190in}}{\pgfqpoint{1.340577in}{1.753190in}}%
\pgfpathcurveto{\pgfqpoint{1.332341in}{1.753190in}}{\pgfqpoint{1.324441in}{1.749918in}}{\pgfqpoint{1.318617in}{1.744094in}}%
\pgfpathcurveto{\pgfqpoint{1.312793in}{1.738270in}}{\pgfqpoint{1.309520in}{1.730370in}}{\pgfqpoint{1.309520in}{1.722134in}}%
\pgfpathcurveto{\pgfqpoint{1.309520in}{1.713898in}}{\pgfqpoint{1.312793in}{1.705997in}}{\pgfqpoint{1.318617in}{1.700174in}}%
\pgfpathcurveto{\pgfqpoint{1.324441in}{1.694350in}}{\pgfqpoint{1.332341in}{1.691077in}}{\pgfqpoint{1.340577in}{1.691077in}}%
\pgfpathclose%
\pgfusepath{stroke,fill}%
\end{pgfscope}%
\begin{pgfscope}%
\pgfpathrectangle{\pgfqpoint{0.100000in}{0.212622in}}{\pgfqpoint{3.696000in}{3.696000in}}%
\pgfusepath{clip}%
\pgfsetbuttcap%
\pgfsetroundjoin%
\definecolor{currentfill}{rgb}{0.121569,0.466667,0.705882}%
\pgfsetfillcolor{currentfill}%
\pgfsetfillopacity{0.487490}%
\pgfsetlinewidth{1.003750pt}%
\definecolor{currentstroke}{rgb}{0.121569,0.466667,0.705882}%
\pgfsetstrokecolor{currentstroke}%
\pgfsetstrokeopacity{0.487490}%
\pgfsetdash{}{0pt}%
\pgfpathmoveto{\pgfqpoint{2.042335in}{1.924496in}}%
\pgfpathcurveto{\pgfqpoint{2.050571in}{1.924496in}}{\pgfqpoint{2.058471in}{1.927768in}}{\pgfqpoint{2.064295in}{1.933592in}}%
\pgfpathcurveto{\pgfqpoint{2.070119in}{1.939416in}}{\pgfqpoint{2.073391in}{1.947316in}}{\pgfqpoint{2.073391in}{1.955552in}}%
\pgfpathcurveto{\pgfqpoint{2.073391in}{1.963789in}}{\pgfqpoint{2.070119in}{1.971689in}}{\pgfqpoint{2.064295in}{1.977513in}}%
\pgfpathcurveto{\pgfqpoint{2.058471in}{1.983337in}}{\pgfqpoint{2.050571in}{1.986609in}}{\pgfqpoint{2.042335in}{1.986609in}}%
\pgfpathcurveto{\pgfqpoint{2.034099in}{1.986609in}}{\pgfqpoint{2.026199in}{1.983337in}}{\pgfqpoint{2.020375in}{1.977513in}}%
\pgfpathcurveto{\pgfqpoint{2.014551in}{1.971689in}}{\pgfqpoint{2.011278in}{1.963789in}}{\pgfqpoint{2.011278in}{1.955552in}}%
\pgfpathcurveto{\pgfqpoint{2.011278in}{1.947316in}}{\pgfqpoint{2.014551in}{1.939416in}}{\pgfqpoint{2.020375in}{1.933592in}}%
\pgfpathcurveto{\pgfqpoint{2.026199in}{1.927768in}}{\pgfqpoint{2.034099in}{1.924496in}}{\pgfqpoint{2.042335in}{1.924496in}}%
\pgfpathclose%
\pgfusepath{stroke,fill}%
\end{pgfscope}%
\begin{pgfscope}%
\pgfpathrectangle{\pgfqpoint{0.100000in}{0.212622in}}{\pgfqpoint{3.696000in}{3.696000in}}%
\pgfusepath{clip}%
\pgfsetbuttcap%
\pgfsetroundjoin%
\definecolor{currentfill}{rgb}{0.121569,0.466667,0.705882}%
\pgfsetfillcolor{currentfill}%
\pgfsetfillopacity{0.488378}%
\pgfsetlinewidth{1.003750pt}%
\definecolor{currentstroke}{rgb}{0.121569,0.466667,0.705882}%
\pgfsetstrokecolor{currentstroke}%
\pgfsetstrokeopacity{0.488378}%
\pgfsetdash{}{0pt}%
\pgfpathmoveto{\pgfqpoint{1.333566in}{1.683994in}}%
\pgfpathcurveto{\pgfqpoint{1.341802in}{1.683994in}}{\pgfqpoint{1.349702in}{1.687266in}}{\pgfqpoint{1.355526in}{1.693090in}}%
\pgfpathcurveto{\pgfqpoint{1.361350in}{1.698914in}}{\pgfqpoint{1.364622in}{1.706814in}}{\pgfqpoint{1.364622in}{1.715050in}}%
\pgfpathcurveto{\pgfqpoint{1.364622in}{1.723286in}}{\pgfqpoint{1.361350in}{1.731186in}}{\pgfqpoint{1.355526in}{1.737010in}}%
\pgfpathcurveto{\pgfqpoint{1.349702in}{1.742834in}}{\pgfqpoint{1.341802in}{1.746107in}}{\pgfqpoint{1.333566in}{1.746107in}}%
\pgfpathcurveto{\pgfqpoint{1.325329in}{1.746107in}}{\pgfqpoint{1.317429in}{1.742834in}}{\pgfqpoint{1.311605in}{1.737010in}}%
\pgfpathcurveto{\pgfqpoint{1.305781in}{1.731186in}}{\pgfqpoint{1.302509in}{1.723286in}}{\pgfqpoint{1.302509in}{1.715050in}}%
\pgfpathcurveto{\pgfqpoint{1.302509in}{1.706814in}}{\pgfqpoint{1.305781in}{1.698914in}}{\pgfqpoint{1.311605in}{1.693090in}}%
\pgfpathcurveto{\pgfqpoint{1.317429in}{1.687266in}}{\pgfqpoint{1.325329in}{1.683994in}}{\pgfqpoint{1.333566in}{1.683994in}}%
\pgfpathclose%
\pgfusepath{stroke,fill}%
\end{pgfscope}%
\begin{pgfscope}%
\pgfpathrectangle{\pgfqpoint{0.100000in}{0.212622in}}{\pgfqpoint{3.696000in}{3.696000in}}%
\pgfusepath{clip}%
\pgfsetbuttcap%
\pgfsetroundjoin%
\definecolor{currentfill}{rgb}{0.121569,0.466667,0.705882}%
\pgfsetfillcolor{currentfill}%
\pgfsetfillopacity{0.489896}%
\pgfsetlinewidth{1.003750pt}%
\definecolor{currentstroke}{rgb}{0.121569,0.466667,0.705882}%
\pgfsetstrokecolor{currentstroke}%
\pgfsetstrokeopacity{0.489896}%
\pgfsetdash{}{0pt}%
\pgfpathmoveto{\pgfqpoint{1.328178in}{1.680187in}}%
\pgfpathcurveto{\pgfqpoint{1.336414in}{1.680187in}}{\pgfqpoint{1.344314in}{1.683460in}}{\pgfqpoint{1.350138in}{1.689284in}}%
\pgfpathcurveto{\pgfqpoint{1.355962in}{1.695107in}}{\pgfqpoint{1.359234in}{1.703007in}}{\pgfqpoint{1.359234in}{1.711244in}}%
\pgfpathcurveto{\pgfqpoint{1.359234in}{1.719480in}}{\pgfqpoint{1.355962in}{1.727380in}}{\pgfqpoint{1.350138in}{1.733204in}}%
\pgfpathcurveto{\pgfqpoint{1.344314in}{1.739028in}}{\pgfqpoint{1.336414in}{1.742300in}}{\pgfqpoint{1.328178in}{1.742300in}}%
\pgfpathcurveto{\pgfqpoint{1.319942in}{1.742300in}}{\pgfqpoint{1.312042in}{1.739028in}}{\pgfqpoint{1.306218in}{1.733204in}}%
\pgfpathcurveto{\pgfqpoint{1.300394in}{1.727380in}}{\pgfqpoint{1.297121in}{1.719480in}}{\pgfqpoint{1.297121in}{1.711244in}}%
\pgfpathcurveto{\pgfqpoint{1.297121in}{1.703007in}}{\pgfqpoint{1.300394in}{1.695107in}}{\pgfqpoint{1.306218in}{1.689284in}}%
\pgfpathcurveto{\pgfqpoint{1.312042in}{1.683460in}}{\pgfqpoint{1.319942in}{1.680187in}}{\pgfqpoint{1.328178in}{1.680187in}}%
\pgfpathclose%
\pgfusepath{stroke,fill}%
\end{pgfscope}%
\begin{pgfscope}%
\pgfpathrectangle{\pgfqpoint{0.100000in}{0.212622in}}{\pgfqpoint{3.696000in}{3.696000in}}%
\pgfusepath{clip}%
\pgfsetbuttcap%
\pgfsetroundjoin%
\definecolor{currentfill}{rgb}{0.121569,0.466667,0.705882}%
\pgfsetfillcolor{currentfill}%
\pgfsetfillopacity{0.490882}%
\pgfsetlinewidth{1.003750pt}%
\definecolor{currentstroke}{rgb}{0.121569,0.466667,0.705882}%
\pgfsetstrokecolor{currentstroke}%
\pgfsetstrokeopacity{0.490882}%
\pgfsetdash{}{0pt}%
\pgfpathmoveto{\pgfqpoint{1.324939in}{1.678984in}}%
\pgfpathcurveto{\pgfqpoint{1.333175in}{1.678984in}}{\pgfqpoint{1.341075in}{1.682256in}}{\pgfqpoint{1.346899in}{1.688080in}}%
\pgfpathcurveto{\pgfqpoint{1.352723in}{1.693904in}}{\pgfqpoint{1.355996in}{1.701804in}}{\pgfqpoint{1.355996in}{1.710040in}}%
\pgfpathcurveto{\pgfqpoint{1.355996in}{1.718277in}}{\pgfqpoint{1.352723in}{1.726177in}}{\pgfqpoint{1.346899in}{1.732001in}}%
\pgfpathcurveto{\pgfqpoint{1.341075in}{1.737824in}}{\pgfqpoint{1.333175in}{1.741097in}}{\pgfqpoint{1.324939in}{1.741097in}}%
\pgfpathcurveto{\pgfqpoint{1.316703in}{1.741097in}}{\pgfqpoint{1.308803in}{1.737824in}}{\pgfqpoint{1.302979in}{1.732001in}}%
\pgfpathcurveto{\pgfqpoint{1.297155in}{1.726177in}}{\pgfqpoint{1.293883in}{1.718277in}}{\pgfqpoint{1.293883in}{1.710040in}}%
\pgfpathcurveto{\pgfqpoint{1.293883in}{1.701804in}}{\pgfqpoint{1.297155in}{1.693904in}}{\pgfqpoint{1.302979in}{1.688080in}}%
\pgfpathcurveto{\pgfqpoint{1.308803in}{1.682256in}}{\pgfqpoint{1.316703in}{1.678984in}}{\pgfqpoint{1.324939in}{1.678984in}}%
\pgfpathclose%
\pgfusepath{stroke,fill}%
\end{pgfscope}%
\begin{pgfscope}%
\pgfpathrectangle{\pgfqpoint{0.100000in}{0.212622in}}{\pgfqpoint{3.696000in}{3.696000in}}%
\pgfusepath{clip}%
\pgfsetbuttcap%
\pgfsetroundjoin%
\definecolor{currentfill}{rgb}{0.121569,0.466667,0.705882}%
\pgfsetfillcolor{currentfill}%
\pgfsetfillopacity{0.491380}%
\pgfsetlinewidth{1.003750pt}%
\definecolor{currentstroke}{rgb}{0.121569,0.466667,0.705882}%
\pgfsetstrokecolor{currentstroke}%
\pgfsetstrokeopacity{0.491380}%
\pgfsetdash{}{0pt}%
\pgfpathmoveto{\pgfqpoint{2.043923in}{1.923926in}}%
\pgfpathcurveto{\pgfqpoint{2.052160in}{1.923926in}}{\pgfqpoint{2.060060in}{1.927198in}}{\pgfqpoint{2.065884in}{1.933022in}}%
\pgfpathcurveto{\pgfqpoint{2.071708in}{1.938846in}}{\pgfqpoint{2.074980in}{1.946746in}}{\pgfqpoint{2.074980in}{1.954982in}}%
\pgfpathcurveto{\pgfqpoint{2.074980in}{1.963219in}}{\pgfqpoint{2.071708in}{1.971119in}}{\pgfqpoint{2.065884in}{1.976943in}}%
\pgfpathcurveto{\pgfqpoint{2.060060in}{1.982766in}}{\pgfqpoint{2.052160in}{1.986039in}}{\pgfqpoint{2.043923in}{1.986039in}}%
\pgfpathcurveto{\pgfqpoint{2.035687in}{1.986039in}}{\pgfqpoint{2.027787in}{1.982766in}}{\pgfqpoint{2.021963in}{1.976943in}}%
\pgfpathcurveto{\pgfqpoint{2.016139in}{1.971119in}}{\pgfqpoint{2.012867in}{1.963219in}}{\pgfqpoint{2.012867in}{1.954982in}}%
\pgfpathcurveto{\pgfqpoint{2.012867in}{1.946746in}}{\pgfqpoint{2.016139in}{1.938846in}}{\pgfqpoint{2.021963in}{1.933022in}}%
\pgfpathcurveto{\pgfqpoint{2.027787in}{1.927198in}}{\pgfqpoint{2.035687in}{1.923926in}}{\pgfqpoint{2.043923in}{1.923926in}}%
\pgfpathclose%
\pgfusepath{stroke,fill}%
\end{pgfscope}%
\begin{pgfscope}%
\pgfpathrectangle{\pgfqpoint{0.100000in}{0.212622in}}{\pgfqpoint{3.696000in}{3.696000in}}%
\pgfusepath{clip}%
\pgfsetbuttcap%
\pgfsetroundjoin%
\definecolor{currentfill}{rgb}{0.121569,0.466667,0.705882}%
\pgfsetfillcolor{currentfill}%
\pgfsetfillopacity{0.492454}%
\pgfsetlinewidth{1.003750pt}%
\definecolor{currentstroke}{rgb}{0.121569,0.466667,0.705882}%
\pgfsetstrokecolor{currentstroke}%
\pgfsetstrokeopacity{0.492454}%
\pgfsetdash{}{0pt}%
\pgfpathmoveto{\pgfqpoint{1.319609in}{1.674489in}}%
\pgfpathcurveto{\pgfqpoint{1.327845in}{1.674489in}}{\pgfqpoint{1.335745in}{1.677762in}}{\pgfqpoint{1.341569in}{1.683586in}}%
\pgfpathcurveto{\pgfqpoint{1.347393in}{1.689410in}}{\pgfqpoint{1.350666in}{1.697310in}}{\pgfqpoint{1.350666in}{1.705546in}}%
\pgfpathcurveto{\pgfqpoint{1.350666in}{1.713782in}}{\pgfqpoint{1.347393in}{1.721682in}}{\pgfqpoint{1.341569in}{1.727506in}}%
\pgfpathcurveto{\pgfqpoint{1.335745in}{1.733330in}}{\pgfqpoint{1.327845in}{1.736602in}}{\pgfqpoint{1.319609in}{1.736602in}}%
\pgfpathcurveto{\pgfqpoint{1.311373in}{1.736602in}}{\pgfqpoint{1.303473in}{1.733330in}}{\pgfqpoint{1.297649in}{1.727506in}}%
\pgfpathcurveto{\pgfqpoint{1.291825in}{1.721682in}}{\pgfqpoint{1.288553in}{1.713782in}}{\pgfqpoint{1.288553in}{1.705546in}}%
\pgfpathcurveto{\pgfqpoint{1.288553in}{1.697310in}}{\pgfqpoint{1.291825in}{1.689410in}}{\pgfqpoint{1.297649in}{1.683586in}}%
\pgfpathcurveto{\pgfqpoint{1.303473in}{1.677762in}}{\pgfqpoint{1.311373in}{1.674489in}}{\pgfqpoint{1.319609in}{1.674489in}}%
\pgfpathclose%
\pgfusepath{stroke,fill}%
\end{pgfscope}%
\begin{pgfscope}%
\pgfpathrectangle{\pgfqpoint{0.100000in}{0.212622in}}{\pgfqpoint{3.696000in}{3.696000in}}%
\pgfusepath{clip}%
\pgfsetbuttcap%
\pgfsetroundjoin%
\definecolor{currentfill}{rgb}{0.121569,0.466667,0.705882}%
\pgfsetfillcolor{currentfill}%
\pgfsetfillopacity{0.493784}%
\pgfsetlinewidth{1.003750pt}%
\definecolor{currentstroke}{rgb}{0.121569,0.466667,0.705882}%
\pgfsetstrokecolor{currentstroke}%
\pgfsetstrokeopacity{0.493784}%
\pgfsetdash{}{0pt}%
\pgfpathmoveto{\pgfqpoint{1.315993in}{1.673457in}}%
\pgfpathcurveto{\pgfqpoint{1.324229in}{1.673457in}}{\pgfqpoint{1.332129in}{1.676730in}}{\pgfqpoint{1.337953in}{1.682553in}}%
\pgfpathcurveto{\pgfqpoint{1.343777in}{1.688377in}}{\pgfqpoint{1.347049in}{1.696277in}}{\pgfqpoint{1.347049in}{1.704514in}}%
\pgfpathcurveto{\pgfqpoint{1.347049in}{1.712750in}}{\pgfqpoint{1.343777in}{1.720650in}}{\pgfqpoint{1.337953in}{1.726474in}}%
\pgfpathcurveto{\pgfqpoint{1.332129in}{1.732298in}}{\pgfqpoint{1.324229in}{1.735570in}}{\pgfqpoint{1.315993in}{1.735570in}}%
\pgfpathcurveto{\pgfqpoint{1.307756in}{1.735570in}}{\pgfqpoint{1.299856in}{1.732298in}}{\pgfqpoint{1.294032in}{1.726474in}}%
\pgfpathcurveto{\pgfqpoint{1.288208in}{1.720650in}}{\pgfqpoint{1.284936in}{1.712750in}}{\pgfqpoint{1.284936in}{1.704514in}}%
\pgfpathcurveto{\pgfqpoint{1.284936in}{1.696277in}}{\pgfqpoint{1.288208in}{1.688377in}}{\pgfqpoint{1.294032in}{1.682553in}}%
\pgfpathcurveto{\pgfqpoint{1.299856in}{1.676730in}}{\pgfqpoint{1.307756in}{1.673457in}}{\pgfqpoint{1.315993in}{1.673457in}}%
\pgfpathclose%
\pgfusepath{stroke,fill}%
\end{pgfscope}%
\begin{pgfscope}%
\pgfpathrectangle{\pgfqpoint{0.100000in}{0.212622in}}{\pgfqpoint{3.696000in}{3.696000in}}%
\pgfusepath{clip}%
\pgfsetbuttcap%
\pgfsetroundjoin%
\definecolor{currentfill}{rgb}{0.121569,0.466667,0.705882}%
\pgfsetfillcolor{currentfill}%
\pgfsetfillopacity{0.495118}%
\pgfsetlinewidth{1.003750pt}%
\definecolor{currentstroke}{rgb}{0.121569,0.466667,0.705882}%
\pgfsetstrokecolor{currentstroke}%
\pgfsetstrokeopacity{0.495118}%
\pgfsetdash{}{0pt}%
\pgfpathmoveto{\pgfqpoint{2.046215in}{1.919414in}}%
\pgfpathcurveto{\pgfqpoint{2.054451in}{1.919414in}}{\pgfqpoint{2.062351in}{1.922686in}}{\pgfqpoint{2.068175in}{1.928510in}}%
\pgfpathcurveto{\pgfqpoint{2.073999in}{1.934334in}}{\pgfqpoint{2.077271in}{1.942234in}}{\pgfqpoint{2.077271in}{1.950471in}}%
\pgfpathcurveto{\pgfqpoint{2.077271in}{1.958707in}}{\pgfqpoint{2.073999in}{1.966607in}}{\pgfqpoint{2.068175in}{1.972431in}}%
\pgfpathcurveto{\pgfqpoint{2.062351in}{1.978255in}}{\pgfqpoint{2.054451in}{1.981527in}}{\pgfqpoint{2.046215in}{1.981527in}}%
\pgfpathcurveto{\pgfqpoint{2.037979in}{1.981527in}}{\pgfqpoint{2.030079in}{1.978255in}}{\pgfqpoint{2.024255in}{1.972431in}}%
\pgfpathcurveto{\pgfqpoint{2.018431in}{1.966607in}}{\pgfqpoint{2.015158in}{1.958707in}}{\pgfqpoint{2.015158in}{1.950471in}}%
\pgfpathcurveto{\pgfqpoint{2.015158in}{1.942234in}}{\pgfqpoint{2.018431in}{1.934334in}}{\pgfqpoint{2.024255in}{1.928510in}}%
\pgfpathcurveto{\pgfqpoint{2.030079in}{1.922686in}}{\pgfqpoint{2.037979in}{1.919414in}}{\pgfqpoint{2.046215in}{1.919414in}}%
\pgfpathclose%
\pgfusepath{stroke,fill}%
\end{pgfscope}%
\begin{pgfscope}%
\pgfpathrectangle{\pgfqpoint{0.100000in}{0.212622in}}{\pgfqpoint{3.696000in}{3.696000in}}%
\pgfusepath{clip}%
\pgfsetbuttcap%
\pgfsetroundjoin%
\definecolor{currentfill}{rgb}{0.121569,0.466667,0.705882}%
\pgfsetfillcolor{currentfill}%
\pgfsetfillopacity{0.495825}%
\pgfsetlinewidth{1.003750pt}%
\definecolor{currentstroke}{rgb}{0.121569,0.466667,0.705882}%
\pgfsetstrokecolor{currentstroke}%
\pgfsetstrokeopacity{0.495825}%
\pgfsetdash{}{0pt}%
\pgfpathmoveto{\pgfqpoint{1.309355in}{1.669214in}}%
\pgfpathcurveto{\pgfqpoint{1.317592in}{1.669214in}}{\pgfqpoint{1.325492in}{1.672486in}}{\pgfqpoint{1.331316in}{1.678310in}}%
\pgfpathcurveto{\pgfqpoint{1.337140in}{1.684134in}}{\pgfqpoint{1.340412in}{1.692034in}}{\pgfqpoint{1.340412in}{1.700270in}}%
\pgfpathcurveto{\pgfqpoint{1.340412in}{1.708506in}}{\pgfqpoint{1.337140in}{1.716406in}}{\pgfqpoint{1.331316in}{1.722230in}}%
\pgfpathcurveto{\pgfqpoint{1.325492in}{1.728054in}}{\pgfqpoint{1.317592in}{1.731327in}}{\pgfqpoint{1.309355in}{1.731327in}}%
\pgfpathcurveto{\pgfqpoint{1.301119in}{1.731327in}}{\pgfqpoint{1.293219in}{1.728054in}}{\pgfqpoint{1.287395in}{1.722230in}}%
\pgfpathcurveto{\pgfqpoint{1.281571in}{1.716406in}}{\pgfqpoint{1.278299in}{1.708506in}}{\pgfqpoint{1.278299in}{1.700270in}}%
\pgfpathcurveto{\pgfqpoint{1.278299in}{1.692034in}}{\pgfqpoint{1.281571in}{1.684134in}}{\pgfqpoint{1.287395in}{1.678310in}}%
\pgfpathcurveto{\pgfqpoint{1.293219in}{1.672486in}}{\pgfqpoint{1.301119in}{1.669214in}}{\pgfqpoint{1.309355in}{1.669214in}}%
\pgfpathclose%
\pgfusepath{stroke,fill}%
\end{pgfscope}%
\begin{pgfscope}%
\pgfpathrectangle{\pgfqpoint{0.100000in}{0.212622in}}{\pgfqpoint{3.696000in}{3.696000in}}%
\pgfusepath{clip}%
\pgfsetbuttcap%
\pgfsetroundjoin%
\definecolor{currentfill}{rgb}{0.121569,0.466667,0.705882}%
\pgfsetfillcolor{currentfill}%
\pgfsetfillopacity{0.499123}%
\pgfsetlinewidth{1.003750pt}%
\definecolor{currentstroke}{rgb}{0.121569,0.466667,0.705882}%
\pgfsetstrokecolor{currentstroke}%
\pgfsetstrokeopacity{0.499123}%
\pgfsetdash{}{0pt}%
\pgfpathmoveto{\pgfqpoint{1.297271in}{1.658818in}}%
\pgfpathcurveto{\pgfqpoint{1.305508in}{1.658818in}}{\pgfqpoint{1.313408in}{1.662091in}}{\pgfqpoint{1.319232in}{1.667915in}}%
\pgfpathcurveto{\pgfqpoint{1.325056in}{1.673738in}}{\pgfqpoint{1.328328in}{1.681639in}}{\pgfqpoint{1.328328in}{1.689875in}}%
\pgfpathcurveto{\pgfqpoint{1.328328in}{1.698111in}}{\pgfqpoint{1.325056in}{1.706011in}}{\pgfqpoint{1.319232in}{1.711835in}}%
\pgfpathcurveto{\pgfqpoint{1.313408in}{1.717659in}}{\pgfqpoint{1.305508in}{1.720931in}}{\pgfqpoint{1.297271in}{1.720931in}}%
\pgfpathcurveto{\pgfqpoint{1.289035in}{1.720931in}}{\pgfqpoint{1.281135in}{1.717659in}}{\pgfqpoint{1.275311in}{1.711835in}}%
\pgfpathcurveto{\pgfqpoint{1.269487in}{1.706011in}}{\pgfqpoint{1.266215in}{1.698111in}}{\pgfqpoint{1.266215in}{1.689875in}}%
\pgfpathcurveto{\pgfqpoint{1.266215in}{1.681639in}}{\pgfqpoint{1.269487in}{1.673738in}}{\pgfqpoint{1.275311in}{1.667915in}}%
\pgfpathcurveto{\pgfqpoint{1.281135in}{1.662091in}}{\pgfqpoint{1.289035in}{1.658818in}}{\pgfqpoint{1.297271in}{1.658818in}}%
\pgfpathclose%
\pgfusepath{stroke,fill}%
\end{pgfscope}%
\begin{pgfscope}%
\pgfpathrectangle{\pgfqpoint{0.100000in}{0.212622in}}{\pgfqpoint{3.696000in}{3.696000in}}%
\pgfusepath{clip}%
\pgfsetbuttcap%
\pgfsetroundjoin%
\definecolor{currentfill}{rgb}{0.121569,0.466667,0.705882}%
\pgfsetfillcolor{currentfill}%
\pgfsetfillopacity{0.499444}%
\pgfsetlinewidth{1.003750pt}%
\definecolor{currentstroke}{rgb}{0.121569,0.466667,0.705882}%
\pgfsetstrokecolor{currentstroke}%
\pgfsetstrokeopacity{0.499444}%
\pgfsetdash{}{0pt}%
\pgfpathmoveto{\pgfqpoint{2.047627in}{1.914768in}}%
\pgfpathcurveto{\pgfqpoint{2.055863in}{1.914768in}}{\pgfqpoint{2.063763in}{1.918040in}}{\pgfqpoint{2.069587in}{1.923864in}}%
\pgfpathcurveto{\pgfqpoint{2.075411in}{1.929688in}}{\pgfqpoint{2.078683in}{1.937588in}}{\pgfqpoint{2.078683in}{1.945824in}}%
\pgfpathcurveto{\pgfqpoint{2.078683in}{1.954061in}}{\pgfqpoint{2.075411in}{1.961961in}}{\pgfqpoint{2.069587in}{1.967785in}}%
\pgfpathcurveto{\pgfqpoint{2.063763in}{1.973609in}}{\pgfqpoint{2.055863in}{1.976881in}}{\pgfqpoint{2.047627in}{1.976881in}}%
\pgfpathcurveto{\pgfqpoint{2.039391in}{1.976881in}}{\pgfqpoint{2.031491in}{1.973609in}}{\pgfqpoint{2.025667in}{1.967785in}}%
\pgfpathcurveto{\pgfqpoint{2.019843in}{1.961961in}}{\pgfqpoint{2.016570in}{1.954061in}}{\pgfqpoint{2.016570in}{1.945824in}}%
\pgfpathcurveto{\pgfqpoint{2.016570in}{1.937588in}}{\pgfqpoint{2.019843in}{1.929688in}}{\pgfqpoint{2.025667in}{1.923864in}}%
\pgfpathcurveto{\pgfqpoint{2.031491in}{1.918040in}}{\pgfqpoint{2.039391in}{1.914768in}}{\pgfqpoint{2.047627in}{1.914768in}}%
\pgfpathclose%
\pgfusepath{stroke,fill}%
\end{pgfscope}%
\begin{pgfscope}%
\pgfpathrectangle{\pgfqpoint{0.100000in}{0.212622in}}{\pgfqpoint{3.696000in}{3.696000in}}%
\pgfusepath{clip}%
\pgfsetbuttcap%
\pgfsetroundjoin%
\definecolor{currentfill}{rgb}{0.121569,0.466667,0.705882}%
\pgfsetfillcolor{currentfill}%
\pgfsetfillopacity{0.502667}%
\pgfsetlinewidth{1.003750pt}%
\definecolor{currentstroke}{rgb}{0.121569,0.466667,0.705882}%
\pgfsetstrokecolor{currentstroke}%
\pgfsetstrokeopacity{0.502667}%
\pgfsetdash{}{0pt}%
\pgfpathmoveto{\pgfqpoint{1.286970in}{1.655826in}}%
\pgfpathcurveto{\pgfqpoint{1.295206in}{1.655826in}}{\pgfqpoint{1.303106in}{1.659099in}}{\pgfqpoint{1.308930in}{1.664922in}}%
\pgfpathcurveto{\pgfqpoint{1.314754in}{1.670746in}}{\pgfqpoint{1.318027in}{1.678646in}}{\pgfqpoint{1.318027in}{1.686883in}}%
\pgfpathcurveto{\pgfqpoint{1.318027in}{1.695119in}}{\pgfqpoint{1.314754in}{1.703019in}}{\pgfqpoint{1.308930in}{1.708843in}}%
\pgfpathcurveto{\pgfqpoint{1.303106in}{1.714667in}}{\pgfqpoint{1.295206in}{1.717939in}}{\pgfqpoint{1.286970in}{1.717939in}}%
\pgfpathcurveto{\pgfqpoint{1.278734in}{1.717939in}}{\pgfqpoint{1.270834in}{1.714667in}}{\pgfqpoint{1.265010in}{1.708843in}}%
\pgfpathcurveto{\pgfqpoint{1.259186in}{1.703019in}}{\pgfqpoint{1.255914in}{1.695119in}}{\pgfqpoint{1.255914in}{1.686883in}}%
\pgfpathcurveto{\pgfqpoint{1.255914in}{1.678646in}}{\pgfqpoint{1.259186in}{1.670746in}}{\pgfqpoint{1.265010in}{1.664922in}}%
\pgfpathcurveto{\pgfqpoint{1.270834in}{1.659099in}}{\pgfqpoint{1.278734in}{1.655826in}}{\pgfqpoint{1.286970in}{1.655826in}}%
\pgfpathclose%
\pgfusepath{stroke,fill}%
\end{pgfscope}%
\begin{pgfscope}%
\pgfpathrectangle{\pgfqpoint{0.100000in}{0.212622in}}{\pgfqpoint{3.696000in}{3.696000in}}%
\pgfusepath{clip}%
\pgfsetbuttcap%
\pgfsetroundjoin%
\definecolor{currentfill}{rgb}{0.121569,0.466667,0.705882}%
\pgfsetfillcolor{currentfill}%
\pgfsetfillopacity{0.504396}%
\pgfsetlinewidth{1.003750pt}%
\definecolor{currentstroke}{rgb}{0.121569,0.466667,0.705882}%
\pgfsetstrokecolor{currentstroke}%
\pgfsetstrokeopacity{0.504396}%
\pgfsetdash{}{0pt}%
\pgfpathmoveto{\pgfqpoint{2.051172in}{1.908683in}}%
\pgfpathcurveto{\pgfqpoint{2.059408in}{1.908683in}}{\pgfqpoint{2.067308in}{1.911956in}}{\pgfqpoint{2.073132in}{1.917780in}}%
\pgfpathcurveto{\pgfqpoint{2.078956in}{1.923604in}}{\pgfqpoint{2.082228in}{1.931504in}}{\pgfqpoint{2.082228in}{1.939740in}}%
\pgfpathcurveto{\pgfqpoint{2.082228in}{1.947976in}}{\pgfqpoint{2.078956in}{1.955876in}}{\pgfqpoint{2.073132in}{1.961700in}}%
\pgfpathcurveto{\pgfqpoint{2.067308in}{1.967524in}}{\pgfqpoint{2.059408in}{1.970796in}}{\pgfqpoint{2.051172in}{1.970796in}}%
\pgfpathcurveto{\pgfqpoint{2.042936in}{1.970796in}}{\pgfqpoint{2.035036in}{1.967524in}}{\pgfqpoint{2.029212in}{1.961700in}}%
\pgfpathcurveto{\pgfqpoint{2.023388in}{1.955876in}}{\pgfqpoint{2.020115in}{1.947976in}}{\pgfqpoint{2.020115in}{1.939740in}}%
\pgfpathcurveto{\pgfqpoint{2.020115in}{1.931504in}}{\pgfqpoint{2.023388in}{1.923604in}}{\pgfqpoint{2.029212in}{1.917780in}}%
\pgfpathcurveto{\pgfqpoint{2.035036in}{1.911956in}}{\pgfqpoint{2.042936in}{1.908683in}}{\pgfqpoint{2.051172in}{1.908683in}}%
\pgfpathclose%
\pgfusepath{stroke,fill}%
\end{pgfscope}%
\begin{pgfscope}%
\pgfpathrectangle{\pgfqpoint{0.100000in}{0.212622in}}{\pgfqpoint{3.696000in}{3.696000in}}%
\pgfusepath{clip}%
\pgfsetbuttcap%
\pgfsetroundjoin%
\definecolor{currentfill}{rgb}{0.121569,0.466667,0.705882}%
\pgfsetfillcolor{currentfill}%
\pgfsetfillopacity{0.505389}%
\pgfsetlinewidth{1.003750pt}%
\definecolor{currentstroke}{rgb}{0.121569,0.466667,0.705882}%
\pgfsetstrokecolor{currentstroke}%
\pgfsetstrokeopacity{0.505389}%
\pgfsetdash{}{0pt}%
\pgfpathmoveto{\pgfqpoint{1.278905in}{1.651623in}}%
\pgfpathcurveto{\pgfqpoint{1.287142in}{1.651623in}}{\pgfqpoint{1.295042in}{1.654895in}}{\pgfqpoint{1.300866in}{1.660719in}}%
\pgfpathcurveto{\pgfqpoint{1.306689in}{1.666543in}}{\pgfqpoint{1.309962in}{1.674443in}}{\pgfqpoint{1.309962in}{1.682679in}}%
\pgfpathcurveto{\pgfqpoint{1.309962in}{1.690915in}}{\pgfqpoint{1.306689in}{1.698815in}}{\pgfqpoint{1.300866in}{1.704639in}}%
\pgfpathcurveto{\pgfqpoint{1.295042in}{1.710463in}}{\pgfqpoint{1.287142in}{1.713736in}}{\pgfqpoint{1.278905in}{1.713736in}}%
\pgfpathcurveto{\pgfqpoint{1.270669in}{1.713736in}}{\pgfqpoint{1.262769in}{1.710463in}}{\pgfqpoint{1.256945in}{1.704639in}}%
\pgfpathcurveto{\pgfqpoint{1.251121in}{1.698815in}}{\pgfqpoint{1.247849in}{1.690915in}}{\pgfqpoint{1.247849in}{1.682679in}}%
\pgfpathcurveto{\pgfqpoint{1.247849in}{1.674443in}}{\pgfqpoint{1.251121in}{1.666543in}}{\pgfqpoint{1.256945in}{1.660719in}}%
\pgfpathcurveto{\pgfqpoint{1.262769in}{1.654895in}}{\pgfqpoint{1.270669in}{1.651623in}}{\pgfqpoint{1.278905in}{1.651623in}}%
\pgfpathclose%
\pgfusepath{stroke,fill}%
\end{pgfscope}%
\begin{pgfscope}%
\pgfpathrectangle{\pgfqpoint{0.100000in}{0.212622in}}{\pgfqpoint{3.696000in}{3.696000in}}%
\pgfusepath{clip}%
\pgfsetbuttcap%
\pgfsetroundjoin%
\definecolor{currentfill}{rgb}{0.121569,0.466667,0.705882}%
\pgfsetfillcolor{currentfill}%
\pgfsetfillopacity{0.507155}%
\pgfsetlinewidth{1.003750pt}%
\definecolor{currentstroke}{rgb}{0.121569,0.466667,0.705882}%
\pgfsetstrokecolor{currentstroke}%
\pgfsetstrokeopacity{0.507155}%
\pgfsetdash{}{0pt}%
\pgfpathmoveto{\pgfqpoint{2.052542in}{1.905261in}}%
\pgfpathcurveto{\pgfqpoint{2.060779in}{1.905261in}}{\pgfqpoint{2.068679in}{1.908533in}}{\pgfqpoint{2.074503in}{1.914357in}}%
\pgfpathcurveto{\pgfqpoint{2.080326in}{1.920181in}}{\pgfqpoint{2.083599in}{1.928081in}}{\pgfqpoint{2.083599in}{1.936317in}}%
\pgfpathcurveto{\pgfqpoint{2.083599in}{1.944553in}}{\pgfqpoint{2.080326in}{1.952453in}}{\pgfqpoint{2.074503in}{1.958277in}}%
\pgfpathcurveto{\pgfqpoint{2.068679in}{1.964101in}}{\pgfqpoint{2.060779in}{1.967374in}}{\pgfqpoint{2.052542in}{1.967374in}}%
\pgfpathcurveto{\pgfqpoint{2.044306in}{1.967374in}}{\pgfqpoint{2.036406in}{1.964101in}}{\pgfqpoint{2.030582in}{1.958277in}}%
\pgfpathcurveto{\pgfqpoint{2.024758in}{1.952453in}}{\pgfqpoint{2.021486in}{1.944553in}}{\pgfqpoint{2.021486in}{1.936317in}}%
\pgfpathcurveto{\pgfqpoint{2.021486in}{1.928081in}}{\pgfqpoint{2.024758in}{1.920181in}}{\pgfqpoint{2.030582in}{1.914357in}}%
\pgfpathcurveto{\pgfqpoint{2.036406in}{1.908533in}}{\pgfqpoint{2.044306in}{1.905261in}}{\pgfqpoint{2.052542in}{1.905261in}}%
\pgfpathclose%
\pgfusepath{stroke,fill}%
\end{pgfscope}%
\begin{pgfscope}%
\pgfpathrectangle{\pgfqpoint{0.100000in}{0.212622in}}{\pgfqpoint{3.696000in}{3.696000in}}%
\pgfusepath{clip}%
\pgfsetbuttcap%
\pgfsetroundjoin%
\definecolor{currentfill}{rgb}{0.121569,0.466667,0.705882}%
\pgfsetfillcolor{currentfill}%
\pgfsetfillopacity{0.509669}%
\pgfsetlinewidth{1.003750pt}%
\definecolor{currentstroke}{rgb}{0.121569,0.466667,0.705882}%
\pgfsetstrokecolor{currentstroke}%
\pgfsetstrokeopacity{0.509669}%
\pgfsetdash{}{0pt}%
\pgfpathmoveto{\pgfqpoint{1.264592in}{1.639115in}}%
\pgfpathcurveto{\pgfqpoint{1.272828in}{1.639115in}}{\pgfqpoint{1.280728in}{1.642388in}}{\pgfqpoint{1.286552in}{1.648212in}}%
\pgfpathcurveto{\pgfqpoint{1.292376in}{1.654036in}}{\pgfqpoint{1.295648in}{1.661936in}}{\pgfqpoint{1.295648in}{1.670172in}}%
\pgfpathcurveto{\pgfqpoint{1.295648in}{1.678408in}}{\pgfqpoint{1.292376in}{1.686308in}}{\pgfqpoint{1.286552in}{1.692132in}}%
\pgfpathcurveto{\pgfqpoint{1.280728in}{1.697956in}}{\pgfqpoint{1.272828in}{1.701228in}}{\pgfqpoint{1.264592in}{1.701228in}}%
\pgfpathcurveto{\pgfqpoint{1.256355in}{1.701228in}}{\pgfqpoint{1.248455in}{1.697956in}}{\pgfqpoint{1.242631in}{1.692132in}}%
\pgfpathcurveto{\pgfqpoint{1.236808in}{1.686308in}}{\pgfqpoint{1.233535in}{1.678408in}}{\pgfqpoint{1.233535in}{1.670172in}}%
\pgfpathcurveto{\pgfqpoint{1.233535in}{1.661936in}}{\pgfqpoint{1.236808in}{1.654036in}}{\pgfqpoint{1.242631in}{1.648212in}}%
\pgfpathcurveto{\pgfqpoint{1.248455in}{1.642388in}}{\pgfqpoint{1.256355in}{1.639115in}}{\pgfqpoint{1.264592in}{1.639115in}}%
\pgfpathclose%
\pgfusepath{stroke,fill}%
\end{pgfscope}%
\begin{pgfscope}%
\pgfpathrectangle{\pgfqpoint{0.100000in}{0.212622in}}{\pgfqpoint{3.696000in}{3.696000in}}%
\pgfusepath{clip}%
\pgfsetbuttcap%
\pgfsetroundjoin%
\definecolor{currentfill}{rgb}{0.121569,0.466667,0.705882}%
\pgfsetfillcolor{currentfill}%
\pgfsetfillopacity{0.510784}%
\pgfsetlinewidth{1.003750pt}%
\definecolor{currentstroke}{rgb}{0.121569,0.466667,0.705882}%
\pgfsetstrokecolor{currentstroke}%
\pgfsetstrokeopacity{0.510784}%
\pgfsetdash{}{0pt}%
\pgfpathmoveto{\pgfqpoint{2.053565in}{1.902928in}}%
\pgfpathcurveto{\pgfqpoint{2.061801in}{1.902928in}}{\pgfqpoint{2.069701in}{1.906200in}}{\pgfqpoint{2.075525in}{1.912024in}}%
\pgfpathcurveto{\pgfqpoint{2.081349in}{1.917848in}}{\pgfqpoint{2.084622in}{1.925748in}}{\pgfqpoint{2.084622in}{1.933985in}}%
\pgfpathcurveto{\pgfqpoint{2.084622in}{1.942221in}}{\pgfqpoint{2.081349in}{1.950121in}}{\pgfqpoint{2.075525in}{1.955945in}}%
\pgfpathcurveto{\pgfqpoint{2.069701in}{1.961769in}}{\pgfqpoint{2.061801in}{1.965041in}}{\pgfqpoint{2.053565in}{1.965041in}}%
\pgfpathcurveto{\pgfqpoint{2.045329in}{1.965041in}}{\pgfqpoint{2.037429in}{1.961769in}}{\pgfqpoint{2.031605in}{1.955945in}}%
\pgfpathcurveto{\pgfqpoint{2.025781in}{1.950121in}}{\pgfqpoint{2.022509in}{1.942221in}}{\pgfqpoint{2.022509in}{1.933985in}}%
\pgfpathcurveto{\pgfqpoint{2.022509in}{1.925748in}}{\pgfqpoint{2.025781in}{1.917848in}}{\pgfqpoint{2.031605in}{1.912024in}}%
\pgfpathcurveto{\pgfqpoint{2.037429in}{1.906200in}}{\pgfqpoint{2.045329in}{1.902928in}}{\pgfqpoint{2.053565in}{1.902928in}}%
\pgfpathclose%
\pgfusepath{stroke,fill}%
\end{pgfscope}%
\begin{pgfscope}%
\pgfpathrectangle{\pgfqpoint{0.100000in}{0.212622in}}{\pgfqpoint{3.696000in}{3.696000in}}%
\pgfusepath{clip}%
\pgfsetbuttcap%
\pgfsetroundjoin%
\definecolor{currentfill}{rgb}{0.121569,0.466667,0.705882}%
\pgfsetfillcolor{currentfill}%
\pgfsetfillopacity{0.513724}%
\pgfsetlinewidth{1.003750pt}%
\definecolor{currentstroke}{rgb}{0.121569,0.466667,0.705882}%
\pgfsetstrokecolor{currentstroke}%
\pgfsetstrokeopacity{0.513724}%
\pgfsetdash{}{0pt}%
\pgfpathmoveto{\pgfqpoint{1.253220in}{1.633783in}}%
\pgfpathcurveto{\pgfqpoint{1.261456in}{1.633783in}}{\pgfqpoint{1.269356in}{1.637056in}}{\pgfqpoint{1.275180in}{1.642879in}}%
\pgfpathcurveto{\pgfqpoint{1.281004in}{1.648703in}}{\pgfqpoint{1.284276in}{1.656603in}}{\pgfqpoint{1.284276in}{1.664840in}}%
\pgfpathcurveto{\pgfqpoint{1.284276in}{1.673076in}}{\pgfqpoint{1.281004in}{1.680976in}}{\pgfqpoint{1.275180in}{1.686800in}}%
\pgfpathcurveto{\pgfqpoint{1.269356in}{1.692624in}}{\pgfqpoint{1.261456in}{1.695896in}}{\pgfqpoint{1.253220in}{1.695896in}}%
\pgfpathcurveto{\pgfqpoint{1.244984in}{1.695896in}}{\pgfqpoint{1.237084in}{1.692624in}}{\pgfqpoint{1.231260in}{1.686800in}}%
\pgfpathcurveto{\pgfqpoint{1.225436in}{1.680976in}}{\pgfqpoint{1.222163in}{1.673076in}}{\pgfqpoint{1.222163in}{1.664840in}}%
\pgfpathcurveto{\pgfqpoint{1.222163in}{1.656603in}}{\pgfqpoint{1.225436in}{1.648703in}}{\pgfqpoint{1.231260in}{1.642879in}}%
\pgfpathcurveto{\pgfqpoint{1.237084in}{1.637056in}}{\pgfqpoint{1.244984in}{1.633783in}}{\pgfqpoint{1.253220in}{1.633783in}}%
\pgfpathclose%
\pgfusepath{stroke,fill}%
\end{pgfscope}%
\begin{pgfscope}%
\pgfpathrectangle{\pgfqpoint{0.100000in}{0.212622in}}{\pgfqpoint{3.696000in}{3.696000in}}%
\pgfusepath{clip}%
\pgfsetbuttcap%
\pgfsetroundjoin%
\definecolor{currentfill}{rgb}{0.121569,0.466667,0.705882}%
\pgfsetfillcolor{currentfill}%
\pgfsetfillopacity{0.514728}%
\pgfsetlinewidth{1.003750pt}%
\definecolor{currentstroke}{rgb}{0.121569,0.466667,0.705882}%
\pgfsetstrokecolor{currentstroke}%
\pgfsetstrokeopacity{0.514728}%
\pgfsetdash{}{0pt}%
\pgfpathmoveto{\pgfqpoint{2.056001in}{1.901011in}}%
\pgfpathcurveto{\pgfqpoint{2.064238in}{1.901011in}}{\pgfqpoint{2.072138in}{1.904283in}}{\pgfqpoint{2.077962in}{1.910107in}}%
\pgfpathcurveto{\pgfqpoint{2.083786in}{1.915931in}}{\pgfqpoint{2.087058in}{1.923831in}}{\pgfqpoint{2.087058in}{1.932067in}}%
\pgfpathcurveto{\pgfqpoint{2.087058in}{1.940303in}}{\pgfqpoint{2.083786in}{1.948203in}}{\pgfqpoint{2.077962in}{1.954027in}}%
\pgfpathcurveto{\pgfqpoint{2.072138in}{1.959851in}}{\pgfqpoint{2.064238in}{1.963124in}}{\pgfqpoint{2.056001in}{1.963124in}}%
\pgfpathcurveto{\pgfqpoint{2.047765in}{1.963124in}}{\pgfqpoint{2.039865in}{1.959851in}}{\pgfqpoint{2.034041in}{1.954027in}}%
\pgfpathcurveto{\pgfqpoint{2.028217in}{1.948203in}}{\pgfqpoint{2.024945in}{1.940303in}}{\pgfqpoint{2.024945in}{1.932067in}}%
\pgfpathcurveto{\pgfqpoint{2.024945in}{1.923831in}}{\pgfqpoint{2.028217in}{1.915931in}}{\pgfqpoint{2.034041in}{1.910107in}}%
\pgfpathcurveto{\pgfqpoint{2.039865in}{1.904283in}}{\pgfqpoint{2.047765in}{1.901011in}}{\pgfqpoint{2.056001in}{1.901011in}}%
\pgfpathclose%
\pgfusepath{stroke,fill}%
\end{pgfscope}%
\begin{pgfscope}%
\pgfpathrectangle{\pgfqpoint{0.100000in}{0.212622in}}{\pgfqpoint{3.696000in}{3.696000in}}%
\pgfusepath{clip}%
\pgfsetbuttcap%
\pgfsetroundjoin%
\definecolor{currentfill}{rgb}{0.121569,0.466667,0.705882}%
\pgfsetfillcolor{currentfill}%
\pgfsetfillopacity{0.516874}%
\pgfsetlinewidth{1.003750pt}%
\definecolor{currentstroke}{rgb}{0.121569,0.466667,0.705882}%
\pgfsetstrokecolor{currentstroke}%
\pgfsetstrokeopacity{0.516874}%
\pgfsetdash{}{0pt}%
\pgfpathmoveto{\pgfqpoint{1.242987in}{1.627837in}}%
\pgfpathcurveto{\pgfqpoint{1.251223in}{1.627837in}}{\pgfqpoint{1.259123in}{1.631109in}}{\pgfqpoint{1.264947in}{1.636933in}}%
\pgfpathcurveto{\pgfqpoint{1.270771in}{1.642757in}}{\pgfqpoint{1.274043in}{1.650657in}}{\pgfqpoint{1.274043in}{1.658894in}}%
\pgfpathcurveto{\pgfqpoint{1.274043in}{1.667130in}}{\pgfqpoint{1.270771in}{1.675030in}}{\pgfqpoint{1.264947in}{1.680854in}}%
\pgfpathcurveto{\pgfqpoint{1.259123in}{1.686678in}}{\pgfqpoint{1.251223in}{1.689950in}}{\pgfqpoint{1.242987in}{1.689950in}}%
\pgfpathcurveto{\pgfqpoint{1.234750in}{1.689950in}}{\pgfqpoint{1.226850in}{1.686678in}}{\pgfqpoint{1.221026in}{1.680854in}}%
\pgfpathcurveto{\pgfqpoint{1.215203in}{1.675030in}}{\pgfqpoint{1.211930in}{1.667130in}}{\pgfqpoint{1.211930in}{1.658894in}}%
\pgfpathcurveto{\pgfqpoint{1.211930in}{1.650657in}}{\pgfqpoint{1.215203in}{1.642757in}}{\pgfqpoint{1.221026in}{1.636933in}}%
\pgfpathcurveto{\pgfqpoint{1.226850in}{1.631109in}}{\pgfqpoint{1.234750in}{1.627837in}}{\pgfqpoint{1.242987in}{1.627837in}}%
\pgfpathclose%
\pgfusepath{stroke,fill}%
\end{pgfscope}%
\begin{pgfscope}%
\pgfpathrectangle{\pgfqpoint{0.100000in}{0.212622in}}{\pgfqpoint{3.696000in}{3.696000in}}%
\pgfusepath{clip}%
\pgfsetbuttcap%
\pgfsetroundjoin%
\definecolor{currentfill}{rgb}{0.121569,0.466667,0.705882}%
\pgfsetfillcolor{currentfill}%
\pgfsetfillopacity{0.519118}%
\pgfsetlinewidth{1.003750pt}%
\definecolor{currentstroke}{rgb}{0.121569,0.466667,0.705882}%
\pgfsetstrokecolor{currentstroke}%
\pgfsetstrokeopacity{0.519118}%
\pgfsetdash{}{0pt}%
\pgfpathmoveto{\pgfqpoint{2.058893in}{1.898197in}}%
\pgfpathcurveto{\pgfqpoint{2.067129in}{1.898197in}}{\pgfqpoint{2.075029in}{1.901469in}}{\pgfqpoint{2.080853in}{1.907293in}}%
\pgfpathcurveto{\pgfqpoint{2.086677in}{1.913117in}}{\pgfqpoint{2.089949in}{1.921017in}}{\pgfqpoint{2.089949in}{1.929253in}}%
\pgfpathcurveto{\pgfqpoint{2.089949in}{1.937489in}}{\pgfqpoint{2.086677in}{1.945389in}}{\pgfqpoint{2.080853in}{1.951213in}}%
\pgfpathcurveto{\pgfqpoint{2.075029in}{1.957037in}}{\pgfqpoint{2.067129in}{1.960310in}}{\pgfqpoint{2.058893in}{1.960310in}}%
\pgfpathcurveto{\pgfqpoint{2.050657in}{1.960310in}}{\pgfqpoint{2.042757in}{1.957037in}}{\pgfqpoint{2.036933in}{1.951213in}}%
\pgfpathcurveto{\pgfqpoint{2.031109in}{1.945389in}}{\pgfqpoint{2.027836in}{1.937489in}}{\pgfqpoint{2.027836in}{1.929253in}}%
\pgfpathcurveto{\pgfqpoint{2.027836in}{1.921017in}}{\pgfqpoint{2.031109in}{1.913117in}}{\pgfqpoint{2.036933in}{1.907293in}}%
\pgfpathcurveto{\pgfqpoint{2.042757in}{1.901469in}}{\pgfqpoint{2.050657in}{1.898197in}}{\pgfqpoint{2.058893in}{1.898197in}}%
\pgfpathclose%
\pgfusepath{stroke,fill}%
\end{pgfscope}%
\begin{pgfscope}%
\pgfpathrectangle{\pgfqpoint{0.100000in}{0.212622in}}{\pgfqpoint{3.696000in}{3.696000in}}%
\pgfusepath{clip}%
\pgfsetbuttcap%
\pgfsetroundjoin%
\definecolor{currentfill}{rgb}{0.121569,0.466667,0.705882}%
\pgfsetfillcolor{currentfill}%
\pgfsetfillopacity{0.519348}%
\pgfsetlinewidth{1.003750pt}%
\definecolor{currentstroke}{rgb}{0.121569,0.466667,0.705882}%
\pgfsetstrokecolor{currentstroke}%
\pgfsetstrokeopacity{0.519348}%
\pgfsetdash{}{0pt}%
\pgfpathmoveto{\pgfqpoint{1.233724in}{1.617475in}}%
\pgfpathcurveto{\pgfqpoint{1.241960in}{1.617475in}}{\pgfqpoint{1.249860in}{1.620747in}}{\pgfqpoint{1.255684in}{1.626571in}}%
\pgfpathcurveto{\pgfqpoint{1.261508in}{1.632395in}}{\pgfqpoint{1.264780in}{1.640295in}}{\pgfqpoint{1.264780in}{1.648531in}}%
\pgfpathcurveto{\pgfqpoint{1.264780in}{1.656767in}}{\pgfqpoint{1.261508in}{1.664667in}}{\pgfqpoint{1.255684in}{1.670491in}}%
\pgfpathcurveto{\pgfqpoint{1.249860in}{1.676315in}}{\pgfqpoint{1.241960in}{1.679588in}}{\pgfqpoint{1.233724in}{1.679588in}}%
\pgfpathcurveto{\pgfqpoint{1.225487in}{1.679588in}}{\pgfqpoint{1.217587in}{1.676315in}}{\pgfqpoint{1.211763in}{1.670491in}}%
\pgfpathcurveto{\pgfqpoint{1.205940in}{1.664667in}}{\pgfqpoint{1.202667in}{1.656767in}}{\pgfqpoint{1.202667in}{1.648531in}}%
\pgfpathcurveto{\pgfqpoint{1.202667in}{1.640295in}}{\pgfqpoint{1.205940in}{1.632395in}}{\pgfqpoint{1.211763in}{1.626571in}}%
\pgfpathcurveto{\pgfqpoint{1.217587in}{1.620747in}}{\pgfqpoint{1.225487in}{1.617475in}}{\pgfqpoint{1.233724in}{1.617475in}}%
\pgfpathclose%
\pgfusepath{stroke,fill}%
\end{pgfscope}%
\begin{pgfscope}%
\pgfpathrectangle{\pgfqpoint{0.100000in}{0.212622in}}{\pgfqpoint{3.696000in}{3.696000in}}%
\pgfusepath{clip}%
\pgfsetbuttcap%
\pgfsetroundjoin%
\definecolor{currentfill}{rgb}{0.121569,0.466667,0.705882}%
\pgfsetfillcolor{currentfill}%
\pgfsetfillopacity{0.521277}%
\pgfsetlinewidth{1.003750pt}%
\definecolor{currentstroke}{rgb}{0.121569,0.466667,0.705882}%
\pgfsetstrokecolor{currentstroke}%
\pgfsetstrokeopacity{0.521277}%
\pgfsetdash{}{0pt}%
\pgfpathmoveto{\pgfqpoint{1.226606in}{1.612491in}}%
\pgfpathcurveto{\pgfqpoint{1.234842in}{1.612491in}}{\pgfqpoint{1.242742in}{1.615763in}}{\pgfqpoint{1.248566in}{1.621587in}}%
\pgfpathcurveto{\pgfqpoint{1.254390in}{1.627411in}}{\pgfqpoint{1.257662in}{1.635311in}}{\pgfqpoint{1.257662in}{1.643547in}}%
\pgfpathcurveto{\pgfqpoint{1.257662in}{1.651784in}}{\pgfqpoint{1.254390in}{1.659684in}}{\pgfqpoint{1.248566in}{1.665508in}}%
\pgfpathcurveto{\pgfqpoint{1.242742in}{1.671332in}}{\pgfqpoint{1.234842in}{1.674604in}}{\pgfqpoint{1.226606in}{1.674604in}}%
\pgfpathcurveto{\pgfqpoint{1.218370in}{1.674604in}}{\pgfqpoint{1.210470in}{1.671332in}}{\pgfqpoint{1.204646in}{1.665508in}}%
\pgfpathcurveto{\pgfqpoint{1.198822in}{1.659684in}}{\pgfqpoint{1.195549in}{1.651784in}}{\pgfqpoint{1.195549in}{1.643547in}}%
\pgfpathcurveto{\pgfqpoint{1.195549in}{1.635311in}}{\pgfqpoint{1.198822in}{1.627411in}}{\pgfqpoint{1.204646in}{1.621587in}}%
\pgfpathcurveto{\pgfqpoint{1.210470in}{1.615763in}}{\pgfqpoint{1.218370in}{1.612491in}}{\pgfqpoint{1.226606in}{1.612491in}}%
\pgfpathclose%
\pgfusepath{stroke,fill}%
\end{pgfscope}%
\begin{pgfscope}%
\pgfpathrectangle{\pgfqpoint{0.100000in}{0.212622in}}{\pgfqpoint{3.696000in}{3.696000in}}%
\pgfusepath{clip}%
\pgfsetbuttcap%
\pgfsetroundjoin%
\definecolor{currentfill}{rgb}{0.121569,0.466667,0.705882}%
\pgfsetfillcolor{currentfill}%
\pgfsetfillopacity{0.522616}%
\pgfsetlinewidth{1.003750pt}%
\definecolor{currentstroke}{rgb}{0.121569,0.466667,0.705882}%
\pgfsetstrokecolor{currentstroke}%
\pgfsetstrokeopacity{0.522616}%
\pgfsetdash{}{0pt}%
\pgfpathmoveto{\pgfqpoint{1.222195in}{1.609228in}}%
\pgfpathcurveto{\pgfqpoint{1.230432in}{1.609228in}}{\pgfqpoint{1.238332in}{1.612501in}}{\pgfqpoint{1.244156in}{1.618325in}}%
\pgfpathcurveto{\pgfqpoint{1.249980in}{1.624148in}}{\pgfqpoint{1.253252in}{1.632048in}}{\pgfqpoint{1.253252in}{1.640285in}}%
\pgfpathcurveto{\pgfqpoint{1.253252in}{1.648521in}}{\pgfqpoint{1.249980in}{1.656421in}}{\pgfqpoint{1.244156in}{1.662245in}}%
\pgfpathcurveto{\pgfqpoint{1.238332in}{1.668069in}}{\pgfqpoint{1.230432in}{1.671341in}}{\pgfqpoint{1.222195in}{1.671341in}}%
\pgfpathcurveto{\pgfqpoint{1.213959in}{1.671341in}}{\pgfqpoint{1.206059in}{1.668069in}}{\pgfqpoint{1.200235in}{1.662245in}}%
\pgfpathcurveto{\pgfqpoint{1.194411in}{1.656421in}}{\pgfqpoint{1.191139in}{1.648521in}}{\pgfqpoint{1.191139in}{1.640285in}}%
\pgfpathcurveto{\pgfqpoint{1.191139in}{1.632048in}}{\pgfqpoint{1.194411in}{1.624148in}}{\pgfqpoint{1.200235in}{1.618325in}}%
\pgfpathcurveto{\pgfqpoint{1.206059in}{1.612501in}}{\pgfqpoint{1.213959in}{1.609228in}}{\pgfqpoint{1.222195in}{1.609228in}}%
\pgfpathclose%
\pgfusepath{stroke,fill}%
\end{pgfscope}%
\begin{pgfscope}%
\pgfpathrectangle{\pgfqpoint{0.100000in}{0.212622in}}{\pgfqpoint{3.696000in}{3.696000in}}%
\pgfusepath{clip}%
\pgfsetbuttcap%
\pgfsetroundjoin%
\definecolor{currentfill}{rgb}{0.121569,0.466667,0.705882}%
\pgfsetfillcolor{currentfill}%
\pgfsetfillopacity{0.523593}%
\pgfsetlinewidth{1.003750pt}%
\definecolor{currentstroke}{rgb}{0.121569,0.466667,0.705882}%
\pgfsetstrokecolor{currentstroke}%
\pgfsetstrokeopacity{0.523593}%
\pgfsetdash{}{0pt}%
\pgfpathmoveto{\pgfqpoint{1.218666in}{1.604420in}}%
\pgfpathcurveto{\pgfqpoint{1.226903in}{1.604420in}}{\pgfqpoint{1.234803in}{1.607693in}}{\pgfqpoint{1.240627in}{1.613517in}}%
\pgfpathcurveto{\pgfqpoint{1.246450in}{1.619341in}}{\pgfqpoint{1.249723in}{1.627241in}}{\pgfqpoint{1.249723in}{1.635477in}}%
\pgfpathcurveto{\pgfqpoint{1.249723in}{1.643713in}}{\pgfqpoint{1.246450in}{1.651613in}}{\pgfqpoint{1.240627in}{1.657437in}}%
\pgfpathcurveto{\pgfqpoint{1.234803in}{1.663261in}}{\pgfqpoint{1.226903in}{1.666533in}}{\pgfqpoint{1.218666in}{1.666533in}}%
\pgfpathcurveto{\pgfqpoint{1.210430in}{1.666533in}}{\pgfqpoint{1.202530in}{1.663261in}}{\pgfqpoint{1.196706in}{1.657437in}}%
\pgfpathcurveto{\pgfqpoint{1.190882in}{1.651613in}}{\pgfqpoint{1.187610in}{1.643713in}}{\pgfqpoint{1.187610in}{1.635477in}}%
\pgfpathcurveto{\pgfqpoint{1.187610in}{1.627241in}}{\pgfqpoint{1.190882in}{1.619341in}}{\pgfqpoint{1.196706in}{1.613517in}}%
\pgfpathcurveto{\pgfqpoint{1.202530in}{1.607693in}}{\pgfqpoint{1.210430in}{1.604420in}}{\pgfqpoint{1.218666in}{1.604420in}}%
\pgfpathclose%
\pgfusepath{stroke,fill}%
\end{pgfscope}%
\begin{pgfscope}%
\pgfpathrectangle{\pgfqpoint{0.100000in}{0.212622in}}{\pgfqpoint{3.696000in}{3.696000in}}%
\pgfusepath{clip}%
\pgfsetbuttcap%
\pgfsetroundjoin%
\definecolor{currentfill}{rgb}{0.121569,0.466667,0.705882}%
\pgfsetfillcolor{currentfill}%
\pgfsetfillopacity{0.524102}%
\pgfsetlinewidth{1.003750pt}%
\definecolor{currentstroke}{rgb}{0.121569,0.466667,0.705882}%
\pgfsetstrokecolor{currentstroke}%
\pgfsetstrokeopacity{0.524102}%
\pgfsetdash{}{0pt}%
\pgfpathmoveto{\pgfqpoint{2.061068in}{1.896231in}}%
\pgfpathcurveto{\pgfqpoint{2.069304in}{1.896231in}}{\pgfqpoint{2.077205in}{1.899503in}}{\pgfqpoint{2.083028in}{1.905327in}}%
\pgfpathcurveto{\pgfqpoint{2.088852in}{1.911151in}}{\pgfqpoint{2.092125in}{1.919051in}}{\pgfqpoint{2.092125in}{1.927288in}}%
\pgfpathcurveto{\pgfqpoint{2.092125in}{1.935524in}}{\pgfqpoint{2.088852in}{1.943424in}}{\pgfqpoint{2.083028in}{1.949248in}}%
\pgfpathcurveto{\pgfqpoint{2.077205in}{1.955072in}}{\pgfqpoint{2.069304in}{1.958344in}}{\pgfqpoint{2.061068in}{1.958344in}}%
\pgfpathcurveto{\pgfqpoint{2.052832in}{1.958344in}}{\pgfqpoint{2.044932in}{1.955072in}}{\pgfqpoint{2.039108in}{1.949248in}}%
\pgfpathcurveto{\pgfqpoint{2.033284in}{1.943424in}}{\pgfqpoint{2.030012in}{1.935524in}}{\pgfqpoint{2.030012in}{1.927288in}}%
\pgfpathcurveto{\pgfqpoint{2.030012in}{1.919051in}}{\pgfqpoint{2.033284in}{1.911151in}}{\pgfqpoint{2.039108in}{1.905327in}}%
\pgfpathcurveto{\pgfqpoint{2.044932in}{1.899503in}}{\pgfqpoint{2.052832in}{1.896231in}}{\pgfqpoint{2.061068in}{1.896231in}}%
\pgfpathclose%
\pgfusepath{stroke,fill}%
\end{pgfscope}%
\begin{pgfscope}%
\pgfpathrectangle{\pgfqpoint{0.100000in}{0.212622in}}{\pgfqpoint{3.696000in}{3.696000in}}%
\pgfusepath{clip}%
\pgfsetbuttcap%
\pgfsetroundjoin%
\definecolor{currentfill}{rgb}{0.121569,0.466667,0.705882}%
\pgfsetfillcolor{currentfill}%
\pgfsetfillopacity{0.524202}%
\pgfsetlinewidth{1.003750pt}%
\definecolor{currentstroke}{rgb}{0.121569,0.466667,0.705882}%
\pgfsetstrokecolor{currentstroke}%
\pgfsetstrokeopacity{0.524202}%
\pgfsetdash{}{0pt}%
\pgfpathmoveto{\pgfqpoint{1.215919in}{1.602663in}}%
\pgfpathcurveto{\pgfqpoint{1.224155in}{1.602663in}}{\pgfqpoint{1.232055in}{1.605935in}}{\pgfqpoint{1.237879in}{1.611759in}}%
\pgfpathcurveto{\pgfqpoint{1.243703in}{1.617583in}}{\pgfqpoint{1.246975in}{1.625483in}}{\pgfqpoint{1.246975in}{1.633719in}}%
\pgfpathcurveto{\pgfqpoint{1.246975in}{1.641956in}}{\pgfqpoint{1.243703in}{1.649856in}}{\pgfqpoint{1.237879in}{1.655680in}}%
\pgfpathcurveto{\pgfqpoint{1.232055in}{1.661503in}}{\pgfqpoint{1.224155in}{1.664776in}}{\pgfqpoint{1.215919in}{1.664776in}}%
\pgfpathcurveto{\pgfqpoint{1.207683in}{1.664776in}}{\pgfqpoint{1.199783in}{1.661503in}}{\pgfqpoint{1.193959in}{1.655680in}}%
\pgfpathcurveto{\pgfqpoint{1.188135in}{1.649856in}}{\pgfqpoint{1.184862in}{1.641956in}}{\pgfqpoint{1.184862in}{1.633719in}}%
\pgfpathcurveto{\pgfqpoint{1.184862in}{1.625483in}}{\pgfqpoint{1.188135in}{1.617583in}}{\pgfqpoint{1.193959in}{1.611759in}}%
\pgfpathcurveto{\pgfqpoint{1.199783in}{1.605935in}}{\pgfqpoint{1.207683in}{1.602663in}}{\pgfqpoint{1.215919in}{1.602663in}}%
\pgfpathclose%
\pgfusepath{stroke,fill}%
\end{pgfscope}%
\begin{pgfscope}%
\pgfpathrectangle{\pgfqpoint{0.100000in}{0.212622in}}{\pgfqpoint{3.696000in}{3.696000in}}%
\pgfusepath{clip}%
\pgfsetbuttcap%
\pgfsetroundjoin%
\definecolor{currentfill}{rgb}{0.121569,0.466667,0.705882}%
\pgfsetfillcolor{currentfill}%
\pgfsetfillopacity{0.525530}%
\pgfsetlinewidth{1.003750pt}%
\definecolor{currentstroke}{rgb}{0.121569,0.466667,0.705882}%
\pgfsetstrokecolor{currentstroke}%
\pgfsetstrokeopacity{0.525530}%
\pgfsetdash{}{0pt}%
\pgfpathmoveto{\pgfqpoint{1.211591in}{1.599695in}}%
\pgfpathcurveto{\pgfqpoint{1.219828in}{1.599695in}}{\pgfqpoint{1.227728in}{1.602967in}}{\pgfqpoint{1.233552in}{1.608791in}}%
\pgfpathcurveto{\pgfqpoint{1.239376in}{1.614615in}}{\pgfqpoint{1.242648in}{1.622515in}}{\pgfqpoint{1.242648in}{1.630752in}}%
\pgfpathcurveto{\pgfqpoint{1.242648in}{1.638988in}}{\pgfqpoint{1.239376in}{1.646888in}}{\pgfqpoint{1.233552in}{1.652712in}}%
\pgfpathcurveto{\pgfqpoint{1.227728in}{1.658536in}}{\pgfqpoint{1.219828in}{1.661808in}}{\pgfqpoint{1.211591in}{1.661808in}}%
\pgfpathcurveto{\pgfqpoint{1.203355in}{1.661808in}}{\pgfqpoint{1.195455in}{1.658536in}}{\pgfqpoint{1.189631in}{1.652712in}}%
\pgfpathcurveto{\pgfqpoint{1.183807in}{1.646888in}}{\pgfqpoint{1.180535in}{1.638988in}}{\pgfqpoint{1.180535in}{1.630752in}}%
\pgfpathcurveto{\pgfqpoint{1.180535in}{1.622515in}}{\pgfqpoint{1.183807in}{1.614615in}}{\pgfqpoint{1.189631in}{1.608791in}}%
\pgfpathcurveto{\pgfqpoint{1.195455in}{1.602967in}}{\pgfqpoint{1.203355in}{1.599695in}}{\pgfqpoint{1.211591in}{1.599695in}}%
\pgfpathclose%
\pgfusepath{stroke,fill}%
\end{pgfscope}%
\begin{pgfscope}%
\pgfpathrectangle{\pgfqpoint{0.100000in}{0.212622in}}{\pgfqpoint{3.696000in}{3.696000in}}%
\pgfusepath{clip}%
\pgfsetbuttcap%
\pgfsetroundjoin%
\definecolor{currentfill}{rgb}{0.121569,0.466667,0.705882}%
\pgfsetfillcolor{currentfill}%
\pgfsetfillopacity{0.526183}%
\pgfsetlinewidth{1.003750pt}%
\definecolor{currentstroke}{rgb}{0.121569,0.466667,0.705882}%
\pgfsetstrokecolor{currentstroke}%
\pgfsetstrokeopacity{0.526183}%
\pgfsetdash{}{0pt}%
\pgfpathmoveto{\pgfqpoint{1.208962in}{1.596050in}}%
\pgfpathcurveto{\pgfqpoint{1.217199in}{1.596050in}}{\pgfqpoint{1.225099in}{1.599322in}}{\pgfqpoint{1.230923in}{1.605146in}}%
\pgfpathcurveto{\pgfqpoint{1.236747in}{1.610970in}}{\pgfqpoint{1.240019in}{1.618870in}}{\pgfqpoint{1.240019in}{1.627106in}}%
\pgfpathcurveto{\pgfqpoint{1.240019in}{1.635343in}}{\pgfqpoint{1.236747in}{1.643243in}}{\pgfqpoint{1.230923in}{1.649067in}}%
\pgfpathcurveto{\pgfqpoint{1.225099in}{1.654890in}}{\pgfqpoint{1.217199in}{1.658163in}}{\pgfqpoint{1.208962in}{1.658163in}}%
\pgfpathcurveto{\pgfqpoint{1.200726in}{1.658163in}}{\pgfqpoint{1.192826in}{1.654890in}}{\pgfqpoint{1.187002in}{1.649067in}}%
\pgfpathcurveto{\pgfqpoint{1.181178in}{1.643243in}}{\pgfqpoint{1.177906in}{1.635343in}}{\pgfqpoint{1.177906in}{1.627106in}}%
\pgfpathcurveto{\pgfqpoint{1.177906in}{1.618870in}}{\pgfqpoint{1.181178in}{1.610970in}}{\pgfqpoint{1.187002in}{1.605146in}}%
\pgfpathcurveto{\pgfqpoint{1.192826in}{1.599322in}}{\pgfqpoint{1.200726in}{1.596050in}}{\pgfqpoint{1.208962in}{1.596050in}}%
\pgfpathclose%
\pgfusepath{stroke,fill}%
\end{pgfscope}%
\begin{pgfscope}%
\pgfpathrectangle{\pgfqpoint{0.100000in}{0.212622in}}{\pgfqpoint{3.696000in}{3.696000in}}%
\pgfusepath{clip}%
\pgfsetbuttcap%
\pgfsetroundjoin%
\definecolor{currentfill}{rgb}{0.121569,0.466667,0.705882}%
\pgfsetfillcolor{currentfill}%
\pgfsetfillopacity{0.526495}%
\pgfsetlinewidth{1.003750pt}%
\definecolor{currentstroke}{rgb}{0.121569,0.466667,0.705882}%
\pgfsetstrokecolor{currentstroke}%
\pgfsetstrokeopacity{0.526495}%
\pgfsetdash{}{0pt}%
\pgfpathmoveto{\pgfqpoint{1.206980in}{1.593933in}}%
\pgfpathcurveto{\pgfqpoint{1.215216in}{1.593933in}}{\pgfqpoint{1.223116in}{1.597206in}}{\pgfqpoint{1.228940in}{1.603030in}}%
\pgfpathcurveto{\pgfqpoint{1.234764in}{1.608854in}}{\pgfqpoint{1.238036in}{1.616754in}}{\pgfqpoint{1.238036in}{1.624990in}}%
\pgfpathcurveto{\pgfqpoint{1.238036in}{1.633226in}}{\pgfqpoint{1.234764in}{1.641126in}}{\pgfqpoint{1.228940in}{1.646950in}}%
\pgfpathcurveto{\pgfqpoint{1.223116in}{1.652774in}}{\pgfqpoint{1.215216in}{1.656046in}}{\pgfqpoint{1.206980in}{1.656046in}}%
\pgfpathcurveto{\pgfqpoint{1.198743in}{1.656046in}}{\pgfqpoint{1.190843in}{1.652774in}}{\pgfqpoint{1.185019in}{1.646950in}}%
\pgfpathcurveto{\pgfqpoint{1.179195in}{1.641126in}}{\pgfqpoint{1.175923in}{1.633226in}}{\pgfqpoint{1.175923in}{1.624990in}}%
\pgfpathcurveto{\pgfqpoint{1.175923in}{1.616754in}}{\pgfqpoint{1.179195in}{1.608854in}}{\pgfqpoint{1.185019in}{1.603030in}}%
\pgfpathcurveto{\pgfqpoint{1.190843in}{1.597206in}}{\pgfqpoint{1.198743in}{1.593933in}}{\pgfqpoint{1.206980in}{1.593933in}}%
\pgfpathclose%
\pgfusepath{stroke,fill}%
\end{pgfscope}%
\begin{pgfscope}%
\pgfpathrectangle{\pgfqpoint{0.100000in}{0.212622in}}{\pgfqpoint{3.696000in}{3.696000in}}%
\pgfusepath{clip}%
\pgfsetbuttcap%
\pgfsetroundjoin%
\definecolor{currentfill}{rgb}{0.121569,0.466667,0.705882}%
\pgfsetfillcolor{currentfill}%
\pgfsetfillopacity{0.527584}%
\pgfsetlinewidth{1.003750pt}%
\definecolor{currentstroke}{rgb}{0.121569,0.466667,0.705882}%
\pgfsetstrokecolor{currentstroke}%
\pgfsetstrokeopacity{0.527584}%
\pgfsetdash{}{0pt}%
\pgfpathmoveto{\pgfqpoint{1.203791in}{1.592651in}}%
\pgfpathcurveto{\pgfqpoint{1.212027in}{1.592651in}}{\pgfqpoint{1.219927in}{1.595923in}}{\pgfqpoint{1.225751in}{1.601747in}}%
\pgfpathcurveto{\pgfqpoint{1.231575in}{1.607571in}}{\pgfqpoint{1.234848in}{1.615471in}}{\pgfqpoint{1.234848in}{1.623708in}}%
\pgfpathcurveto{\pgfqpoint{1.234848in}{1.631944in}}{\pgfqpoint{1.231575in}{1.639844in}}{\pgfqpoint{1.225751in}{1.645668in}}%
\pgfpathcurveto{\pgfqpoint{1.219927in}{1.651492in}}{\pgfqpoint{1.212027in}{1.654764in}}{\pgfqpoint{1.203791in}{1.654764in}}%
\pgfpathcurveto{\pgfqpoint{1.195555in}{1.654764in}}{\pgfqpoint{1.187655in}{1.651492in}}{\pgfqpoint{1.181831in}{1.645668in}}%
\pgfpathcurveto{\pgfqpoint{1.176007in}{1.639844in}}{\pgfqpoint{1.172735in}{1.631944in}}{\pgfqpoint{1.172735in}{1.623708in}}%
\pgfpathcurveto{\pgfqpoint{1.172735in}{1.615471in}}{\pgfqpoint{1.176007in}{1.607571in}}{\pgfqpoint{1.181831in}{1.601747in}}%
\pgfpathcurveto{\pgfqpoint{1.187655in}{1.595923in}}{\pgfqpoint{1.195555in}{1.592651in}}{\pgfqpoint{1.203791in}{1.592651in}}%
\pgfpathclose%
\pgfusepath{stroke,fill}%
\end{pgfscope}%
\begin{pgfscope}%
\pgfpathrectangle{\pgfqpoint{0.100000in}{0.212622in}}{\pgfqpoint{3.696000in}{3.696000in}}%
\pgfusepath{clip}%
\pgfsetbuttcap%
\pgfsetroundjoin%
\definecolor{currentfill}{rgb}{0.121569,0.466667,0.705882}%
\pgfsetfillcolor{currentfill}%
\pgfsetfillopacity{0.527962}%
\pgfsetlinewidth{1.003750pt}%
\definecolor{currentstroke}{rgb}{0.121569,0.466667,0.705882}%
\pgfsetstrokecolor{currentstroke}%
\pgfsetstrokeopacity{0.527962}%
\pgfsetdash{}{0pt}%
\pgfpathmoveto{\pgfqpoint{1.202367in}{1.591037in}}%
\pgfpathcurveto{\pgfqpoint{1.210604in}{1.591037in}}{\pgfqpoint{1.218504in}{1.594310in}}{\pgfqpoint{1.224328in}{1.600134in}}%
\pgfpathcurveto{\pgfqpoint{1.230151in}{1.605958in}}{\pgfqpoint{1.233424in}{1.613858in}}{\pgfqpoint{1.233424in}{1.622094in}}%
\pgfpathcurveto{\pgfqpoint{1.233424in}{1.630330in}}{\pgfqpoint{1.230151in}{1.638230in}}{\pgfqpoint{1.224328in}{1.644054in}}%
\pgfpathcurveto{\pgfqpoint{1.218504in}{1.649878in}}{\pgfqpoint{1.210604in}{1.653150in}}{\pgfqpoint{1.202367in}{1.653150in}}%
\pgfpathcurveto{\pgfqpoint{1.194131in}{1.653150in}}{\pgfqpoint{1.186231in}{1.649878in}}{\pgfqpoint{1.180407in}{1.644054in}}%
\pgfpathcurveto{\pgfqpoint{1.174583in}{1.638230in}}{\pgfqpoint{1.171311in}{1.630330in}}{\pgfqpoint{1.171311in}{1.622094in}}%
\pgfpathcurveto{\pgfqpoint{1.171311in}{1.613858in}}{\pgfqpoint{1.174583in}{1.605958in}}{\pgfqpoint{1.180407in}{1.600134in}}%
\pgfpathcurveto{\pgfqpoint{1.186231in}{1.594310in}}{\pgfqpoint{1.194131in}{1.591037in}}{\pgfqpoint{1.202367in}{1.591037in}}%
\pgfpathclose%
\pgfusepath{stroke,fill}%
\end{pgfscope}%
\begin{pgfscope}%
\pgfpathrectangle{\pgfqpoint{0.100000in}{0.212622in}}{\pgfqpoint{3.696000in}{3.696000in}}%
\pgfusepath{clip}%
\pgfsetbuttcap%
\pgfsetroundjoin%
\definecolor{currentfill}{rgb}{0.121569,0.466667,0.705882}%
\pgfsetfillcolor{currentfill}%
\pgfsetfillopacity{0.528810}%
\pgfsetlinewidth{1.003750pt}%
\definecolor{currentstroke}{rgb}{0.121569,0.466667,0.705882}%
\pgfsetstrokecolor{currentstroke}%
\pgfsetstrokeopacity{0.528810}%
\pgfsetdash{}{0pt}%
\pgfpathmoveto{\pgfqpoint{1.199782in}{1.589115in}}%
\pgfpathcurveto{\pgfqpoint{1.208018in}{1.589115in}}{\pgfqpoint{1.215918in}{1.592388in}}{\pgfqpoint{1.221742in}{1.598212in}}%
\pgfpathcurveto{\pgfqpoint{1.227566in}{1.604036in}}{\pgfqpoint{1.230839in}{1.611936in}}{\pgfqpoint{1.230839in}{1.620172in}}%
\pgfpathcurveto{\pgfqpoint{1.230839in}{1.628408in}}{\pgfqpoint{1.227566in}{1.636308in}}{\pgfqpoint{1.221742in}{1.642132in}}%
\pgfpathcurveto{\pgfqpoint{1.215918in}{1.647956in}}{\pgfqpoint{1.208018in}{1.651228in}}{\pgfqpoint{1.199782in}{1.651228in}}%
\pgfpathcurveto{\pgfqpoint{1.191546in}{1.651228in}}{\pgfqpoint{1.183646in}{1.647956in}}{\pgfqpoint{1.177822in}{1.642132in}}%
\pgfpathcurveto{\pgfqpoint{1.171998in}{1.636308in}}{\pgfqpoint{1.168726in}{1.628408in}}{\pgfqpoint{1.168726in}{1.620172in}}%
\pgfpathcurveto{\pgfqpoint{1.168726in}{1.611936in}}{\pgfqpoint{1.171998in}{1.604036in}}{\pgfqpoint{1.177822in}{1.598212in}}%
\pgfpathcurveto{\pgfqpoint{1.183646in}{1.592388in}}{\pgfqpoint{1.191546in}{1.589115in}}{\pgfqpoint{1.199782in}{1.589115in}}%
\pgfpathclose%
\pgfusepath{stroke,fill}%
\end{pgfscope}%
\begin{pgfscope}%
\pgfpathrectangle{\pgfqpoint{0.100000in}{0.212622in}}{\pgfqpoint{3.696000in}{3.696000in}}%
\pgfusepath{clip}%
\pgfsetbuttcap%
\pgfsetroundjoin%
\definecolor{currentfill}{rgb}{0.121569,0.466667,0.705882}%
\pgfsetfillcolor{currentfill}%
\pgfsetfillopacity{0.529364}%
\pgfsetlinewidth{1.003750pt}%
\definecolor{currentstroke}{rgb}{0.121569,0.466667,0.705882}%
\pgfsetstrokecolor{currentstroke}%
\pgfsetstrokeopacity{0.529364}%
\pgfsetdash{}{0pt}%
\pgfpathmoveto{\pgfqpoint{2.064066in}{1.893655in}}%
\pgfpathcurveto{\pgfqpoint{2.072302in}{1.893655in}}{\pgfqpoint{2.080202in}{1.896928in}}{\pgfqpoint{2.086026in}{1.902752in}}%
\pgfpathcurveto{\pgfqpoint{2.091850in}{1.908575in}}{\pgfqpoint{2.095122in}{1.916475in}}{\pgfqpoint{2.095122in}{1.924712in}}%
\pgfpathcurveto{\pgfqpoint{2.095122in}{1.932948in}}{\pgfqpoint{2.091850in}{1.940848in}}{\pgfqpoint{2.086026in}{1.946672in}}%
\pgfpathcurveto{\pgfqpoint{2.080202in}{1.952496in}}{\pgfqpoint{2.072302in}{1.955768in}}{\pgfqpoint{2.064066in}{1.955768in}}%
\pgfpathcurveto{\pgfqpoint{2.055830in}{1.955768in}}{\pgfqpoint{2.047929in}{1.952496in}}{\pgfqpoint{2.042106in}{1.946672in}}%
\pgfpathcurveto{\pgfqpoint{2.036282in}{1.940848in}}{\pgfqpoint{2.033009in}{1.932948in}}{\pgfqpoint{2.033009in}{1.924712in}}%
\pgfpathcurveto{\pgfqpoint{2.033009in}{1.916475in}}{\pgfqpoint{2.036282in}{1.908575in}}{\pgfqpoint{2.042106in}{1.902752in}}%
\pgfpathcurveto{\pgfqpoint{2.047929in}{1.896928in}}{\pgfqpoint{2.055830in}{1.893655in}}{\pgfqpoint{2.064066in}{1.893655in}}%
\pgfpathclose%
\pgfusepath{stroke,fill}%
\end{pgfscope}%
\begin{pgfscope}%
\pgfpathrectangle{\pgfqpoint{0.100000in}{0.212622in}}{\pgfqpoint{3.696000in}{3.696000in}}%
\pgfusepath{clip}%
\pgfsetbuttcap%
\pgfsetroundjoin%
\definecolor{currentfill}{rgb}{0.121569,0.466667,0.705882}%
\pgfsetfillcolor{currentfill}%
\pgfsetfillopacity{0.530583}%
\pgfsetlinewidth{1.003750pt}%
\definecolor{currentstroke}{rgb}{0.121569,0.466667,0.705882}%
\pgfsetstrokecolor{currentstroke}%
\pgfsetstrokeopacity{0.530583}%
\pgfsetdash{}{0pt}%
\pgfpathmoveto{\pgfqpoint{1.194901in}{1.587337in}}%
\pgfpathcurveto{\pgfqpoint{1.203137in}{1.587337in}}{\pgfqpoint{1.211037in}{1.590609in}}{\pgfqpoint{1.216861in}{1.596433in}}%
\pgfpathcurveto{\pgfqpoint{1.222685in}{1.602257in}}{\pgfqpoint{1.225957in}{1.610157in}}{\pgfqpoint{1.225957in}{1.618393in}}%
\pgfpathcurveto{\pgfqpoint{1.225957in}{1.626630in}}{\pgfqpoint{1.222685in}{1.634530in}}{\pgfqpoint{1.216861in}{1.640354in}}%
\pgfpathcurveto{\pgfqpoint{1.211037in}{1.646178in}}{\pgfqpoint{1.203137in}{1.649450in}}{\pgfqpoint{1.194901in}{1.649450in}}%
\pgfpathcurveto{\pgfqpoint{1.186664in}{1.649450in}}{\pgfqpoint{1.178764in}{1.646178in}}{\pgfqpoint{1.172940in}{1.640354in}}%
\pgfpathcurveto{\pgfqpoint{1.167116in}{1.634530in}}{\pgfqpoint{1.163844in}{1.626630in}}{\pgfqpoint{1.163844in}{1.618393in}}%
\pgfpathcurveto{\pgfqpoint{1.163844in}{1.610157in}}{\pgfqpoint{1.167116in}{1.602257in}}{\pgfqpoint{1.172940in}{1.596433in}}%
\pgfpathcurveto{\pgfqpoint{1.178764in}{1.590609in}}{\pgfqpoint{1.186664in}{1.587337in}}{\pgfqpoint{1.194901in}{1.587337in}}%
\pgfpathclose%
\pgfusepath{stroke,fill}%
\end{pgfscope}%
\begin{pgfscope}%
\pgfpathrectangle{\pgfqpoint{0.100000in}{0.212622in}}{\pgfqpoint{3.696000in}{3.696000in}}%
\pgfusepath{clip}%
\pgfsetbuttcap%
\pgfsetroundjoin%
\definecolor{currentfill}{rgb}{0.121569,0.466667,0.705882}%
\pgfsetfillcolor{currentfill}%
\pgfsetfillopacity{0.531353}%
\pgfsetlinewidth{1.003750pt}%
\definecolor{currentstroke}{rgb}{0.121569,0.466667,0.705882}%
\pgfsetstrokecolor{currentstroke}%
\pgfsetstrokeopacity{0.531353}%
\pgfsetdash{}{0pt}%
\pgfpathmoveto{\pgfqpoint{1.192051in}{1.583534in}}%
\pgfpathcurveto{\pgfqpoint{1.200288in}{1.583534in}}{\pgfqpoint{1.208188in}{1.586806in}}{\pgfqpoint{1.214012in}{1.592630in}}%
\pgfpathcurveto{\pgfqpoint{1.219836in}{1.598454in}}{\pgfqpoint{1.223108in}{1.606354in}}{\pgfqpoint{1.223108in}{1.614590in}}%
\pgfpathcurveto{\pgfqpoint{1.223108in}{1.622827in}}{\pgfqpoint{1.219836in}{1.630727in}}{\pgfqpoint{1.214012in}{1.636551in}}%
\pgfpathcurveto{\pgfqpoint{1.208188in}{1.642375in}}{\pgfqpoint{1.200288in}{1.645647in}}{\pgfqpoint{1.192051in}{1.645647in}}%
\pgfpathcurveto{\pgfqpoint{1.183815in}{1.645647in}}{\pgfqpoint{1.175915in}{1.642375in}}{\pgfqpoint{1.170091in}{1.636551in}}%
\pgfpathcurveto{\pgfqpoint{1.164267in}{1.630727in}}{\pgfqpoint{1.160995in}{1.622827in}}{\pgfqpoint{1.160995in}{1.614590in}}%
\pgfpathcurveto{\pgfqpoint{1.160995in}{1.606354in}}{\pgfqpoint{1.164267in}{1.598454in}}{\pgfqpoint{1.170091in}{1.592630in}}%
\pgfpathcurveto{\pgfqpoint{1.175915in}{1.586806in}}{\pgfqpoint{1.183815in}{1.583534in}}{\pgfqpoint{1.192051in}{1.583534in}}%
\pgfpathclose%
\pgfusepath{stroke,fill}%
\end{pgfscope}%
\begin{pgfscope}%
\pgfpathrectangle{\pgfqpoint{0.100000in}{0.212622in}}{\pgfqpoint{3.696000in}{3.696000in}}%
\pgfusepath{clip}%
\pgfsetbuttcap%
\pgfsetroundjoin%
\definecolor{currentfill}{rgb}{0.121569,0.466667,0.705882}%
\pgfsetfillcolor{currentfill}%
\pgfsetfillopacity{0.532289}%
\pgfsetlinewidth{1.003750pt}%
\definecolor{currentstroke}{rgb}{0.121569,0.466667,0.705882}%
\pgfsetstrokecolor{currentstroke}%
\pgfsetstrokeopacity{0.532289}%
\pgfsetdash{}{0pt}%
\pgfpathmoveto{\pgfqpoint{1.185226in}{1.576466in}}%
\pgfpathcurveto{\pgfqpoint{1.193463in}{1.576466in}}{\pgfqpoint{1.201363in}{1.579739in}}{\pgfqpoint{1.207187in}{1.585562in}}%
\pgfpathcurveto{\pgfqpoint{1.213011in}{1.591386in}}{\pgfqpoint{1.216283in}{1.599286in}}{\pgfqpoint{1.216283in}{1.607523in}}%
\pgfpathcurveto{\pgfqpoint{1.216283in}{1.615759in}}{\pgfqpoint{1.213011in}{1.623659in}}{\pgfqpoint{1.207187in}{1.629483in}}%
\pgfpathcurveto{\pgfqpoint{1.201363in}{1.635307in}}{\pgfqpoint{1.193463in}{1.638579in}}{\pgfqpoint{1.185226in}{1.638579in}}%
\pgfpathcurveto{\pgfqpoint{1.176990in}{1.638579in}}{\pgfqpoint{1.169090in}{1.635307in}}{\pgfqpoint{1.163266in}{1.629483in}}%
\pgfpathcurveto{\pgfqpoint{1.157442in}{1.623659in}}{\pgfqpoint{1.154170in}{1.615759in}}{\pgfqpoint{1.154170in}{1.607523in}}%
\pgfpathcurveto{\pgfqpoint{1.154170in}{1.599286in}}{\pgfqpoint{1.157442in}{1.591386in}}{\pgfqpoint{1.163266in}{1.585562in}}%
\pgfpathcurveto{\pgfqpoint{1.169090in}{1.579739in}}{\pgfqpoint{1.176990in}{1.576466in}}{\pgfqpoint{1.185226in}{1.576466in}}%
\pgfpathclose%
\pgfusepath{stroke,fill}%
\end{pgfscope}%
\begin{pgfscope}%
\pgfpathrectangle{\pgfqpoint{0.100000in}{0.212622in}}{\pgfqpoint{3.696000in}{3.696000in}}%
\pgfusepath{clip}%
\pgfsetbuttcap%
\pgfsetroundjoin%
\definecolor{currentfill}{rgb}{0.121569,0.466667,0.705882}%
\pgfsetfillcolor{currentfill}%
\pgfsetfillopacity{0.534692}%
\pgfsetlinewidth{1.003750pt}%
\definecolor{currentstroke}{rgb}{0.121569,0.466667,0.705882}%
\pgfsetstrokecolor{currentstroke}%
\pgfsetstrokeopacity{0.534692}%
\pgfsetdash{}{0pt}%
\pgfpathmoveto{\pgfqpoint{2.067230in}{1.887914in}}%
\pgfpathcurveto{\pgfqpoint{2.075466in}{1.887914in}}{\pgfqpoint{2.083366in}{1.891186in}}{\pgfqpoint{2.089190in}{1.897010in}}%
\pgfpathcurveto{\pgfqpoint{2.095014in}{1.902834in}}{\pgfqpoint{2.098287in}{1.910734in}}{\pgfqpoint{2.098287in}{1.918970in}}%
\pgfpathcurveto{\pgfqpoint{2.098287in}{1.927206in}}{\pgfqpoint{2.095014in}{1.935106in}}{\pgfqpoint{2.089190in}{1.940930in}}%
\pgfpathcurveto{\pgfqpoint{2.083366in}{1.946754in}}{\pgfqpoint{2.075466in}{1.950027in}}{\pgfqpoint{2.067230in}{1.950027in}}%
\pgfpathcurveto{\pgfqpoint{2.058994in}{1.950027in}}{\pgfqpoint{2.051094in}{1.946754in}}{\pgfqpoint{2.045270in}{1.940930in}}%
\pgfpathcurveto{\pgfqpoint{2.039446in}{1.935106in}}{\pgfqpoint{2.036174in}{1.927206in}}{\pgfqpoint{2.036174in}{1.918970in}}%
\pgfpathcurveto{\pgfqpoint{2.036174in}{1.910734in}}{\pgfqpoint{2.039446in}{1.902834in}}{\pgfqpoint{2.045270in}{1.897010in}}%
\pgfpathcurveto{\pgfqpoint{2.051094in}{1.891186in}}{\pgfqpoint{2.058994in}{1.887914in}}{\pgfqpoint{2.067230in}{1.887914in}}%
\pgfpathclose%
\pgfusepath{stroke,fill}%
\end{pgfscope}%
\begin{pgfscope}%
\pgfpathrectangle{\pgfqpoint{0.100000in}{0.212622in}}{\pgfqpoint{3.696000in}{3.696000in}}%
\pgfusepath{clip}%
\pgfsetbuttcap%
\pgfsetroundjoin%
\definecolor{currentfill}{rgb}{0.121569,0.466667,0.705882}%
\pgfsetfillcolor{currentfill}%
\pgfsetfillopacity{0.535611}%
\pgfsetlinewidth{1.003750pt}%
\definecolor{currentstroke}{rgb}{0.121569,0.466667,0.705882}%
\pgfsetstrokecolor{currentstroke}%
\pgfsetstrokeopacity{0.535611}%
\pgfsetdash{}{0pt}%
\pgfpathmoveto{\pgfqpoint{1.174496in}{1.570612in}}%
\pgfpathcurveto{\pgfqpoint{1.182732in}{1.570612in}}{\pgfqpoint{1.190632in}{1.573884in}}{\pgfqpoint{1.196456in}{1.579708in}}%
\pgfpathcurveto{\pgfqpoint{1.202280in}{1.585532in}}{\pgfqpoint{1.205553in}{1.593432in}}{\pgfqpoint{1.205553in}{1.601668in}}%
\pgfpathcurveto{\pgfqpoint{1.205553in}{1.609904in}}{\pgfqpoint{1.202280in}{1.617805in}}{\pgfqpoint{1.196456in}{1.623628in}}%
\pgfpathcurveto{\pgfqpoint{1.190632in}{1.629452in}}{\pgfqpoint{1.182732in}{1.632725in}}{\pgfqpoint{1.174496in}{1.632725in}}%
\pgfpathcurveto{\pgfqpoint{1.166260in}{1.632725in}}{\pgfqpoint{1.158360in}{1.629452in}}{\pgfqpoint{1.152536in}{1.623628in}}%
\pgfpathcurveto{\pgfqpoint{1.146712in}{1.617805in}}{\pgfqpoint{1.143440in}{1.609904in}}{\pgfqpoint{1.143440in}{1.601668in}}%
\pgfpathcurveto{\pgfqpoint{1.143440in}{1.593432in}}{\pgfqpoint{1.146712in}{1.585532in}}{\pgfqpoint{1.152536in}{1.579708in}}%
\pgfpathcurveto{\pgfqpoint{1.158360in}{1.573884in}}{\pgfqpoint{1.166260in}{1.570612in}}{\pgfqpoint{1.174496in}{1.570612in}}%
\pgfpathclose%
\pgfusepath{stroke,fill}%
\end{pgfscope}%
\begin{pgfscope}%
\pgfpathrectangle{\pgfqpoint{0.100000in}{0.212622in}}{\pgfqpoint{3.696000in}{3.696000in}}%
\pgfusepath{clip}%
\pgfsetbuttcap%
\pgfsetroundjoin%
\definecolor{currentfill}{rgb}{0.121569,0.466667,0.705882}%
\pgfsetfillcolor{currentfill}%
\pgfsetfillopacity{0.537556}%
\pgfsetlinewidth{1.003750pt}%
\definecolor{currentstroke}{rgb}{0.121569,0.466667,0.705882}%
\pgfsetstrokecolor{currentstroke}%
\pgfsetstrokeopacity{0.537556}%
\pgfsetdash{}{0pt}%
\pgfpathmoveto{\pgfqpoint{2.068423in}{1.884046in}}%
\pgfpathcurveto{\pgfqpoint{2.076660in}{1.884046in}}{\pgfqpoint{2.084560in}{1.887318in}}{\pgfqpoint{2.090384in}{1.893142in}}%
\pgfpathcurveto{\pgfqpoint{2.096208in}{1.898966in}}{\pgfqpoint{2.099480in}{1.906866in}}{\pgfqpoint{2.099480in}{1.915103in}}%
\pgfpathcurveto{\pgfqpoint{2.099480in}{1.923339in}}{\pgfqpoint{2.096208in}{1.931239in}}{\pgfqpoint{2.090384in}{1.937063in}}%
\pgfpathcurveto{\pgfqpoint{2.084560in}{1.942887in}}{\pgfqpoint{2.076660in}{1.946159in}}{\pgfqpoint{2.068423in}{1.946159in}}%
\pgfpathcurveto{\pgfqpoint{2.060187in}{1.946159in}}{\pgfqpoint{2.052287in}{1.942887in}}{\pgfqpoint{2.046463in}{1.937063in}}%
\pgfpathcurveto{\pgfqpoint{2.040639in}{1.931239in}}{\pgfqpoint{2.037367in}{1.923339in}}{\pgfqpoint{2.037367in}{1.915103in}}%
\pgfpathcurveto{\pgfqpoint{2.037367in}{1.906866in}}{\pgfqpoint{2.040639in}{1.898966in}}{\pgfqpoint{2.046463in}{1.893142in}}%
\pgfpathcurveto{\pgfqpoint{2.052287in}{1.887318in}}{\pgfqpoint{2.060187in}{1.884046in}}{\pgfqpoint{2.068423in}{1.884046in}}%
\pgfpathclose%
\pgfusepath{stroke,fill}%
\end{pgfscope}%
\begin{pgfscope}%
\pgfpathrectangle{\pgfqpoint{0.100000in}{0.212622in}}{\pgfqpoint{3.696000in}{3.696000in}}%
\pgfusepath{clip}%
\pgfsetbuttcap%
\pgfsetroundjoin%
\definecolor{currentfill}{rgb}{0.121569,0.466667,0.705882}%
\pgfsetfillcolor{currentfill}%
\pgfsetfillopacity{0.538545}%
\pgfsetlinewidth{1.003750pt}%
\definecolor{currentstroke}{rgb}{0.121569,0.466667,0.705882}%
\pgfsetstrokecolor{currentstroke}%
\pgfsetstrokeopacity{0.538545}%
\pgfsetdash{}{0pt}%
\pgfpathmoveto{\pgfqpoint{1.165088in}{1.561685in}}%
\pgfpathcurveto{\pgfqpoint{1.173325in}{1.561685in}}{\pgfqpoint{1.181225in}{1.564957in}}{\pgfqpoint{1.187049in}{1.570781in}}%
\pgfpathcurveto{\pgfqpoint{1.192873in}{1.576605in}}{\pgfqpoint{1.196145in}{1.584505in}}{\pgfqpoint{1.196145in}{1.592742in}}%
\pgfpathcurveto{\pgfqpoint{1.196145in}{1.600978in}}{\pgfqpoint{1.192873in}{1.608878in}}{\pgfqpoint{1.187049in}{1.614702in}}%
\pgfpathcurveto{\pgfqpoint{1.181225in}{1.620526in}}{\pgfqpoint{1.173325in}{1.623798in}}{\pgfqpoint{1.165088in}{1.623798in}}%
\pgfpathcurveto{\pgfqpoint{1.156852in}{1.623798in}}{\pgfqpoint{1.148952in}{1.620526in}}{\pgfqpoint{1.143128in}{1.614702in}}%
\pgfpathcurveto{\pgfqpoint{1.137304in}{1.608878in}}{\pgfqpoint{1.134032in}{1.600978in}}{\pgfqpoint{1.134032in}{1.592742in}}%
\pgfpathcurveto{\pgfqpoint{1.134032in}{1.584505in}}{\pgfqpoint{1.137304in}{1.576605in}}{\pgfqpoint{1.143128in}{1.570781in}}%
\pgfpathcurveto{\pgfqpoint{1.148952in}{1.564957in}}{\pgfqpoint{1.156852in}{1.561685in}}{\pgfqpoint{1.165088in}{1.561685in}}%
\pgfpathclose%
\pgfusepath{stroke,fill}%
\end{pgfscope}%
\begin{pgfscope}%
\pgfpathrectangle{\pgfqpoint{0.100000in}{0.212622in}}{\pgfqpoint{3.696000in}{3.696000in}}%
\pgfusepath{clip}%
\pgfsetbuttcap%
\pgfsetroundjoin%
\definecolor{currentfill}{rgb}{0.121569,0.466667,0.705882}%
\pgfsetfillcolor{currentfill}%
\pgfsetfillopacity{0.540770}%
\pgfsetlinewidth{1.003750pt}%
\definecolor{currentstroke}{rgb}{0.121569,0.466667,0.705882}%
\pgfsetstrokecolor{currentstroke}%
\pgfsetstrokeopacity{0.540770}%
\pgfsetdash{}{0pt}%
\pgfpathmoveto{\pgfqpoint{1.157900in}{1.555288in}}%
\pgfpathcurveto{\pgfqpoint{1.166136in}{1.555288in}}{\pgfqpoint{1.174036in}{1.558560in}}{\pgfqpoint{1.179860in}{1.564384in}}%
\pgfpathcurveto{\pgfqpoint{1.185684in}{1.570208in}}{\pgfqpoint{1.188956in}{1.578108in}}{\pgfqpoint{1.188956in}{1.586344in}}%
\pgfpathcurveto{\pgfqpoint{1.188956in}{1.594581in}}{\pgfqpoint{1.185684in}{1.602481in}}{\pgfqpoint{1.179860in}{1.608305in}}%
\pgfpathcurveto{\pgfqpoint{1.174036in}{1.614128in}}{\pgfqpoint{1.166136in}{1.617401in}}{\pgfqpoint{1.157900in}{1.617401in}}%
\pgfpathcurveto{\pgfqpoint{1.149664in}{1.617401in}}{\pgfqpoint{1.141764in}{1.614128in}}{\pgfqpoint{1.135940in}{1.608305in}}%
\pgfpathcurveto{\pgfqpoint{1.130116in}{1.602481in}}{\pgfqpoint{1.126843in}{1.594581in}}{\pgfqpoint{1.126843in}{1.586344in}}%
\pgfpathcurveto{\pgfqpoint{1.126843in}{1.578108in}}{\pgfqpoint{1.130116in}{1.570208in}}{\pgfqpoint{1.135940in}{1.564384in}}%
\pgfpathcurveto{\pgfqpoint{1.141764in}{1.558560in}}{\pgfqpoint{1.149664in}{1.555288in}}{\pgfqpoint{1.157900in}{1.555288in}}%
\pgfpathclose%
\pgfusepath{stroke,fill}%
\end{pgfscope}%
\begin{pgfscope}%
\pgfpathrectangle{\pgfqpoint{0.100000in}{0.212622in}}{\pgfqpoint{3.696000in}{3.696000in}}%
\pgfusepath{clip}%
\pgfsetbuttcap%
\pgfsetroundjoin%
\definecolor{currentfill}{rgb}{0.121569,0.466667,0.705882}%
\pgfsetfillcolor{currentfill}%
\pgfsetfillopacity{0.540919}%
\pgfsetlinewidth{1.003750pt}%
\definecolor{currentstroke}{rgb}{0.121569,0.466667,0.705882}%
\pgfsetstrokecolor{currentstroke}%
\pgfsetstrokeopacity{0.540919}%
\pgfsetdash{}{0pt}%
\pgfpathmoveto{\pgfqpoint{2.070474in}{1.881940in}}%
\pgfpathcurveto{\pgfqpoint{2.078711in}{1.881940in}}{\pgfqpoint{2.086611in}{1.885212in}}{\pgfqpoint{2.092435in}{1.891036in}}%
\pgfpathcurveto{\pgfqpoint{2.098258in}{1.896860in}}{\pgfqpoint{2.101531in}{1.904760in}}{\pgfqpoint{2.101531in}{1.912996in}}%
\pgfpathcurveto{\pgfqpoint{2.101531in}{1.921232in}}{\pgfqpoint{2.098258in}{1.929132in}}{\pgfqpoint{2.092435in}{1.934956in}}%
\pgfpathcurveto{\pgfqpoint{2.086611in}{1.940780in}}{\pgfqpoint{2.078711in}{1.944053in}}{\pgfqpoint{2.070474in}{1.944053in}}%
\pgfpathcurveto{\pgfqpoint{2.062238in}{1.944053in}}{\pgfqpoint{2.054338in}{1.940780in}}{\pgfqpoint{2.048514in}{1.934956in}}%
\pgfpathcurveto{\pgfqpoint{2.042690in}{1.929132in}}{\pgfqpoint{2.039418in}{1.921232in}}{\pgfqpoint{2.039418in}{1.912996in}}%
\pgfpathcurveto{\pgfqpoint{2.039418in}{1.904760in}}{\pgfqpoint{2.042690in}{1.896860in}}{\pgfqpoint{2.048514in}{1.891036in}}%
\pgfpathcurveto{\pgfqpoint{2.054338in}{1.885212in}}{\pgfqpoint{2.062238in}{1.881940in}}{\pgfqpoint{2.070474in}{1.881940in}}%
\pgfpathclose%
\pgfusepath{stroke,fill}%
\end{pgfscope}%
\begin{pgfscope}%
\pgfpathrectangle{\pgfqpoint{0.100000in}{0.212622in}}{\pgfqpoint{3.696000in}{3.696000in}}%
\pgfusepath{clip}%
\pgfsetbuttcap%
\pgfsetroundjoin%
\definecolor{currentfill}{rgb}{0.121569,0.466667,0.705882}%
\pgfsetfillcolor{currentfill}%
\pgfsetfillopacity{0.542139}%
\pgfsetlinewidth{1.003750pt}%
\definecolor{currentstroke}{rgb}{0.121569,0.466667,0.705882}%
\pgfsetstrokecolor{currentstroke}%
\pgfsetstrokeopacity{0.542139}%
\pgfsetdash{}{0pt}%
\pgfpathmoveto{\pgfqpoint{1.153082in}{1.551475in}}%
\pgfpathcurveto{\pgfqpoint{1.161319in}{1.551475in}}{\pgfqpoint{1.169219in}{1.554748in}}{\pgfqpoint{1.175043in}{1.560572in}}%
\pgfpathcurveto{\pgfqpoint{1.180867in}{1.566396in}}{\pgfqpoint{1.184139in}{1.574296in}}{\pgfqpoint{1.184139in}{1.582532in}}%
\pgfpathcurveto{\pgfqpoint{1.184139in}{1.590768in}}{\pgfqpoint{1.180867in}{1.598668in}}{\pgfqpoint{1.175043in}{1.604492in}}%
\pgfpathcurveto{\pgfqpoint{1.169219in}{1.610316in}}{\pgfqpoint{1.161319in}{1.613588in}}{\pgfqpoint{1.153082in}{1.613588in}}%
\pgfpathcurveto{\pgfqpoint{1.144846in}{1.613588in}}{\pgfqpoint{1.136946in}{1.610316in}}{\pgfqpoint{1.131122in}{1.604492in}}%
\pgfpathcurveto{\pgfqpoint{1.125298in}{1.598668in}}{\pgfqpoint{1.122026in}{1.590768in}}{\pgfqpoint{1.122026in}{1.582532in}}%
\pgfpathcurveto{\pgfqpoint{1.122026in}{1.574296in}}{\pgfqpoint{1.125298in}{1.566396in}}{\pgfqpoint{1.131122in}{1.560572in}}%
\pgfpathcurveto{\pgfqpoint{1.136946in}{1.554748in}}{\pgfqpoint{1.144846in}{1.551475in}}{\pgfqpoint{1.153082in}{1.551475in}}%
\pgfpathclose%
\pgfusepath{stroke,fill}%
\end{pgfscope}%
\begin{pgfscope}%
\pgfpathrectangle{\pgfqpoint{0.100000in}{0.212622in}}{\pgfqpoint{3.696000in}{3.696000in}}%
\pgfusepath{clip}%
\pgfsetbuttcap%
\pgfsetroundjoin%
\definecolor{currentfill}{rgb}{0.121569,0.466667,0.705882}%
\pgfsetfillcolor{currentfill}%
\pgfsetfillopacity{0.544333}%
\pgfsetlinewidth{1.003750pt}%
\definecolor{currentstroke}{rgb}{0.121569,0.466667,0.705882}%
\pgfsetstrokecolor{currentstroke}%
\pgfsetstrokeopacity{0.544333}%
\pgfsetdash{}{0pt}%
\pgfpathmoveto{\pgfqpoint{1.144518in}{1.542382in}}%
\pgfpathcurveto{\pgfqpoint{1.152754in}{1.542382in}}{\pgfqpoint{1.160654in}{1.545654in}}{\pgfqpoint{1.166478in}{1.551478in}}%
\pgfpathcurveto{\pgfqpoint{1.172302in}{1.557302in}}{\pgfqpoint{1.175574in}{1.565202in}}{\pgfqpoint{1.175574in}{1.573438in}}%
\pgfpathcurveto{\pgfqpoint{1.175574in}{1.581675in}}{\pgfqpoint{1.172302in}{1.589575in}}{\pgfqpoint{1.166478in}{1.595399in}}%
\pgfpathcurveto{\pgfqpoint{1.160654in}{1.601222in}}{\pgfqpoint{1.152754in}{1.604495in}}{\pgfqpoint{1.144518in}{1.604495in}}%
\pgfpathcurveto{\pgfqpoint{1.136281in}{1.604495in}}{\pgfqpoint{1.128381in}{1.601222in}}{\pgfqpoint{1.122557in}{1.595399in}}%
\pgfpathcurveto{\pgfqpoint{1.116733in}{1.589575in}}{\pgfqpoint{1.113461in}{1.581675in}}{\pgfqpoint{1.113461in}{1.573438in}}%
\pgfpathcurveto{\pgfqpoint{1.113461in}{1.565202in}}{\pgfqpoint{1.116733in}{1.557302in}}{\pgfqpoint{1.122557in}{1.551478in}}%
\pgfpathcurveto{\pgfqpoint{1.128381in}{1.545654in}}{\pgfqpoint{1.136281in}{1.542382in}}{\pgfqpoint{1.144518in}{1.542382in}}%
\pgfpathclose%
\pgfusepath{stroke,fill}%
\end{pgfscope}%
\begin{pgfscope}%
\pgfpathrectangle{\pgfqpoint{0.100000in}{0.212622in}}{\pgfqpoint{3.696000in}{3.696000in}}%
\pgfusepath{clip}%
\pgfsetbuttcap%
\pgfsetroundjoin%
\definecolor{currentfill}{rgb}{0.121569,0.466667,0.705882}%
\pgfsetfillcolor{currentfill}%
\pgfsetfillopacity{0.544664}%
\pgfsetlinewidth{1.003750pt}%
\definecolor{currentstroke}{rgb}{0.121569,0.466667,0.705882}%
\pgfsetstrokecolor{currentstroke}%
\pgfsetstrokeopacity{0.544664}%
\pgfsetdash{}{0pt}%
\pgfpathmoveto{\pgfqpoint{2.073072in}{1.879811in}}%
\pgfpathcurveto{\pgfqpoint{2.081309in}{1.879811in}}{\pgfqpoint{2.089209in}{1.883083in}}{\pgfqpoint{2.095033in}{1.888907in}}%
\pgfpathcurveto{\pgfqpoint{2.100857in}{1.894731in}}{\pgfqpoint{2.104129in}{1.902631in}}{\pgfqpoint{2.104129in}{1.910867in}}%
\pgfpathcurveto{\pgfqpoint{2.104129in}{1.919104in}}{\pgfqpoint{2.100857in}{1.927004in}}{\pgfqpoint{2.095033in}{1.932828in}}%
\pgfpathcurveto{\pgfqpoint{2.089209in}{1.938652in}}{\pgfqpoint{2.081309in}{1.941924in}}{\pgfqpoint{2.073072in}{1.941924in}}%
\pgfpathcurveto{\pgfqpoint{2.064836in}{1.941924in}}{\pgfqpoint{2.056936in}{1.938652in}}{\pgfqpoint{2.051112in}{1.932828in}}%
\pgfpathcurveto{\pgfqpoint{2.045288in}{1.927004in}}{\pgfqpoint{2.042016in}{1.919104in}}{\pgfqpoint{2.042016in}{1.910867in}}%
\pgfpathcurveto{\pgfqpoint{2.042016in}{1.902631in}}{\pgfqpoint{2.045288in}{1.894731in}}{\pgfqpoint{2.051112in}{1.888907in}}%
\pgfpathcurveto{\pgfqpoint{2.056936in}{1.883083in}}{\pgfqpoint{2.064836in}{1.879811in}}{\pgfqpoint{2.073072in}{1.879811in}}%
\pgfpathclose%
\pgfusepath{stroke,fill}%
\end{pgfscope}%
\begin{pgfscope}%
\pgfpathrectangle{\pgfqpoint{0.100000in}{0.212622in}}{\pgfqpoint{3.696000in}{3.696000in}}%
\pgfusepath{clip}%
\pgfsetbuttcap%
\pgfsetroundjoin%
\definecolor{currentfill}{rgb}{0.121569,0.466667,0.705882}%
\pgfsetfillcolor{currentfill}%
\pgfsetfillopacity{0.547126}%
\pgfsetlinewidth{1.003750pt}%
\definecolor{currentstroke}{rgb}{0.121569,0.466667,0.705882}%
\pgfsetstrokecolor{currentstroke}%
\pgfsetstrokeopacity{0.547126}%
\pgfsetdash{}{0pt}%
\pgfpathmoveto{\pgfqpoint{1.137302in}{1.539555in}}%
\pgfpathcurveto{\pgfqpoint{1.145538in}{1.539555in}}{\pgfqpoint{1.153438in}{1.542827in}}{\pgfqpoint{1.159262in}{1.548651in}}%
\pgfpathcurveto{\pgfqpoint{1.165086in}{1.554475in}}{\pgfqpoint{1.168359in}{1.562375in}}{\pgfqpoint{1.168359in}{1.570611in}}%
\pgfpathcurveto{\pgfqpoint{1.168359in}{1.578847in}}{\pgfqpoint{1.165086in}{1.586747in}}{\pgfqpoint{1.159262in}{1.592571in}}%
\pgfpathcurveto{\pgfqpoint{1.153438in}{1.598395in}}{\pgfqpoint{1.145538in}{1.601668in}}{\pgfqpoint{1.137302in}{1.601668in}}%
\pgfpathcurveto{\pgfqpoint{1.129066in}{1.601668in}}{\pgfqpoint{1.121166in}{1.598395in}}{\pgfqpoint{1.115342in}{1.592571in}}%
\pgfpathcurveto{\pgfqpoint{1.109518in}{1.586747in}}{\pgfqpoint{1.106246in}{1.578847in}}{\pgfqpoint{1.106246in}{1.570611in}}%
\pgfpathcurveto{\pgfqpoint{1.106246in}{1.562375in}}{\pgfqpoint{1.109518in}{1.554475in}}{\pgfqpoint{1.115342in}{1.548651in}}%
\pgfpathcurveto{\pgfqpoint{1.121166in}{1.542827in}}{\pgfqpoint{1.129066in}{1.539555in}}{\pgfqpoint{1.137302in}{1.539555in}}%
\pgfpathclose%
\pgfusepath{stroke,fill}%
\end{pgfscope}%
\begin{pgfscope}%
\pgfpathrectangle{\pgfqpoint{0.100000in}{0.212622in}}{\pgfqpoint{3.696000in}{3.696000in}}%
\pgfusepath{clip}%
\pgfsetbuttcap%
\pgfsetroundjoin%
\definecolor{currentfill}{rgb}{0.121569,0.466667,0.705882}%
\pgfsetfillcolor{currentfill}%
\pgfsetfillopacity{0.548536}%
\pgfsetlinewidth{1.003750pt}%
\definecolor{currentstroke}{rgb}{0.121569,0.466667,0.705882}%
\pgfsetstrokecolor{currentstroke}%
\pgfsetstrokeopacity{0.548536}%
\pgfsetdash{}{0pt}%
\pgfpathmoveto{\pgfqpoint{1.131735in}{1.532937in}}%
\pgfpathcurveto{\pgfqpoint{1.139971in}{1.532937in}}{\pgfqpoint{1.147871in}{1.536209in}}{\pgfqpoint{1.153695in}{1.542033in}}%
\pgfpathcurveto{\pgfqpoint{1.159519in}{1.547857in}}{\pgfqpoint{1.162792in}{1.555757in}}{\pgfqpoint{1.162792in}{1.563993in}}%
\pgfpathcurveto{\pgfqpoint{1.162792in}{1.572230in}}{\pgfqpoint{1.159519in}{1.580130in}}{\pgfqpoint{1.153695in}{1.585954in}}%
\pgfpathcurveto{\pgfqpoint{1.147871in}{1.591778in}}{\pgfqpoint{1.139971in}{1.595050in}}{\pgfqpoint{1.131735in}{1.595050in}}%
\pgfpathcurveto{\pgfqpoint{1.123499in}{1.595050in}}{\pgfqpoint{1.115599in}{1.591778in}}{\pgfqpoint{1.109775in}{1.585954in}}%
\pgfpathcurveto{\pgfqpoint{1.103951in}{1.580130in}}{\pgfqpoint{1.100679in}{1.572230in}}{\pgfqpoint{1.100679in}{1.563993in}}%
\pgfpathcurveto{\pgfqpoint{1.100679in}{1.555757in}}{\pgfqpoint{1.103951in}{1.547857in}}{\pgfqpoint{1.109775in}{1.542033in}}%
\pgfpathcurveto{\pgfqpoint{1.115599in}{1.536209in}}{\pgfqpoint{1.123499in}{1.532937in}}{\pgfqpoint{1.131735in}{1.532937in}}%
\pgfpathclose%
\pgfusepath{stroke,fill}%
\end{pgfscope}%
\begin{pgfscope}%
\pgfpathrectangle{\pgfqpoint{0.100000in}{0.212622in}}{\pgfqpoint{3.696000in}{3.696000in}}%
\pgfusepath{clip}%
\pgfsetbuttcap%
\pgfsetroundjoin%
\definecolor{currentfill}{rgb}{0.121569,0.466667,0.705882}%
\pgfsetfillcolor{currentfill}%
\pgfsetfillopacity{0.548664}%
\pgfsetlinewidth{1.003750pt}%
\definecolor{currentstroke}{rgb}{0.121569,0.466667,0.705882}%
\pgfsetstrokecolor{currentstroke}%
\pgfsetstrokeopacity{0.548664}%
\pgfsetdash{}{0pt}%
\pgfpathmoveto{\pgfqpoint{2.075562in}{1.874915in}}%
\pgfpathcurveto{\pgfqpoint{2.083798in}{1.874915in}}{\pgfqpoint{2.091698in}{1.878187in}}{\pgfqpoint{2.097522in}{1.884011in}}%
\pgfpathcurveto{\pgfqpoint{2.103346in}{1.889835in}}{\pgfqpoint{2.106619in}{1.897735in}}{\pgfqpoint{2.106619in}{1.905972in}}%
\pgfpathcurveto{\pgfqpoint{2.106619in}{1.914208in}}{\pgfqpoint{2.103346in}{1.922108in}}{\pgfqpoint{2.097522in}{1.927932in}}%
\pgfpathcurveto{\pgfqpoint{2.091698in}{1.933756in}}{\pgfqpoint{2.083798in}{1.937028in}}{\pgfqpoint{2.075562in}{1.937028in}}%
\pgfpathcurveto{\pgfqpoint{2.067326in}{1.937028in}}{\pgfqpoint{2.059426in}{1.933756in}}{\pgfqpoint{2.053602in}{1.927932in}}%
\pgfpathcurveto{\pgfqpoint{2.047778in}{1.922108in}}{\pgfqpoint{2.044506in}{1.914208in}}{\pgfqpoint{2.044506in}{1.905972in}}%
\pgfpathcurveto{\pgfqpoint{2.044506in}{1.897735in}}{\pgfqpoint{2.047778in}{1.889835in}}{\pgfqpoint{2.053602in}{1.884011in}}%
\pgfpathcurveto{\pgfqpoint{2.059426in}{1.878187in}}{\pgfqpoint{2.067326in}{1.874915in}}{\pgfqpoint{2.075562in}{1.874915in}}%
\pgfpathclose%
\pgfusepath{stroke,fill}%
\end{pgfscope}%
\begin{pgfscope}%
\pgfpathrectangle{\pgfqpoint{0.100000in}{0.212622in}}{\pgfqpoint{3.696000in}{3.696000in}}%
\pgfusepath{clip}%
\pgfsetbuttcap%
\pgfsetroundjoin%
\definecolor{currentfill}{rgb}{0.121569,0.466667,0.705882}%
\pgfsetfillcolor{currentfill}%
\pgfsetfillopacity{0.550135}%
\pgfsetlinewidth{1.003750pt}%
\definecolor{currentstroke}{rgb}{0.121569,0.466667,0.705882}%
\pgfsetstrokecolor{currentstroke}%
\pgfsetstrokeopacity{0.550135}%
\pgfsetdash{}{0pt}%
\pgfpathmoveto{\pgfqpoint{1.126484in}{1.528507in}}%
\pgfpathcurveto{\pgfqpoint{1.134721in}{1.528507in}}{\pgfqpoint{1.142621in}{1.531779in}}{\pgfqpoint{1.148445in}{1.537603in}}%
\pgfpathcurveto{\pgfqpoint{1.154268in}{1.543427in}}{\pgfqpoint{1.157541in}{1.551327in}}{\pgfqpoint{1.157541in}{1.559564in}}%
\pgfpathcurveto{\pgfqpoint{1.157541in}{1.567800in}}{\pgfqpoint{1.154268in}{1.575700in}}{\pgfqpoint{1.148445in}{1.581524in}}%
\pgfpathcurveto{\pgfqpoint{1.142621in}{1.587348in}}{\pgfqpoint{1.134721in}{1.590620in}}{\pgfqpoint{1.126484in}{1.590620in}}%
\pgfpathcurveto{\pgfqpoint{1.118248in}{1.590620in}}{\pgfqpoint{1.110348in}{1.587348in}}{\pgfqpoint{1.104524in}{1.581524in}}%
\pgfpathcurveto{\pgfqpoint{1.098700in}{1.575700in}}{\pgfqpoint{1.095428in}{1.567800in}}{\pgfqpoint{1.095428in}{1.559564in}}%
\pgfpathcurveto{\pgfqpoint{1.095428in}{1.551327in}}{\pgfqpoint{1.098700in}{1.543427in}}{\pgfqpoint{1.104524in}{1.537603in}}%
\pgfpathcurveto{\pgfqpoint{1.110348in}{1.531779in}}{\pgfqpoint{1.118248in}{1.528507in}}{\pgfqpoint{1.126484in}{1.528507in}}%
\pgfpathclose%
\pgfusepath{stroke,fill}%
\end{pgfscope}%
\begin{pgfscope}%
\pgfpathrectangle{\pgfqpoint{0.100000in}{0.212622in}}{\pgfqpoint{3.696000in}{3.696000in}}%
\pgfusepath{clip}%
\pgfsetbuttcap%
\pgfsetroundjoin%
\definecolor{currentfill}{rgb}{0.121569,0.466667,0.705882}%
\pgfsetfillcolor{currentfill}%
\pgfsetfillopacity{0.552649}%
\pgfsetlinewidth{1.003750pt}%
\definecolor{currentstroke}{rgb}{0.121569,0.466667,0.705882}%
\pgfsetstrokecolor{currentstroke}%
\pgfsetstrokeopacity{0.552649}%
\pgfsetdash{}{0pt}%
\pgfpathmoveto{\pgfqpoint{2.077942in}{1.867875in}}%
\pgfpathcurveto{\pgfqpoint{2.086178in}{1.867875in}}{\pgfqpoint{2.094079in}{1.871147in}}{\pgfqpoint{2.099902in}{1.876971in}}%
\pgfpathcurveto{\pgfqpoint{2.105726in}{1.882795in}}{\pgfqpoint{2.108999in}{1.890695in}}{\pgfqpoint{2.108999in}{1.898931in}}%
\pgfpathcurveto{\pgfqpoint{2.108999in}{1.907168in}}{\pgfqpoint{2.105726in}{1.915068in}}{\pgfqpoint{2.099902in}{1.920892in}}%
\pgfpathcurveto{\pgfqpoint{2.094079in}{1.926715in}}{\pgfqpoint{2.086178in}{1.929988in}}{\pgfqpoint{2.077942in}{1.929988in}}%
\pgfpathcurveto{\pgfqpoint{2.069706in}{1.929988in}}{\pgfqpoint{2.061806in}{1.926715in}}{\pgfqpoint{2.055982in}{1.920892in}}%
\pgfpathcurveto{\pgfqpoint{2.050158in}{1.915068in}}{\pgfqpoint{2.046886in}{1.907168in}}{\pgfqpoint{2.046886in}{1.898931in}}%
\pgfpathcurveto{\pgfqpoint{2.046886in}{1.890695in}}{\pgfqpoint{2.050158in}{1.882795in}}{\pgfqpoint{2.055982in}{1.876971in}}%
\pgfpathcurveto{\pgfqpoint{2.061806in}{1.871147in}}{\pgfqpoint{2.069706in}{1.867875in}}{\pgfqpoint{2.077942in}{1.867875in}}%
\pgfpathclose%
\pgfusepath{stroke,fill}%
\end{pgfscope}%
\begin{pgfscope}%
\pgfpathrectangle{\pgfqpoint{0.100000in}{0.212622in}}{\pgfqpoint{3.696000in}{3.696000in}}%
\pgfusepath{clip}%
\pgfsetbuttcap%
\pgfsetroundjoin%
\definecolor{currentfill}{rgb}{0.121569,0.466667,0.705882}%
\pgfsetfillcolor{currentfill}%
\pgfsetfillopacity{0.553064}%
\pgfsetlinewidth{1.003750pt}%
\definecolor{currentstroke}{rgb}{0.121569,0.466667,0.705882}%
\pgfsetstrokecolor{currentstroke}%
\pgfsetstrokeopacity{0.553064}%
\pgfsetdash{}{0pt}%
\pgfpathmoveto{\pgfqpoint{1.115897in}{1.522141in}}%
\pgfpathcurveto{\pgfqpoint{1.124133in}{1.522141in}}{\pgfqpoint{1.132033in}{1.525414in}}{\pgfqpoint{1.137857in}{1.531238in}}%
\pgfpathcurveto{\pgfqpoint{1.143681in}{1.537061in}}{\pgfqpoint{1.146953in}{1.544962in}}{\pgfqpoint{1.146953in}{1.553198in}}%
\pgfpathcurveto{\pgfqpoint{1.146953in}{1.561434in}}{\pgfqpoint{1.143681in}{1.569334in}}{\pgfqpoint{1.137857in}{1.575158in}}%
\pgfpathcurveto{\pgfqpoint{1.132033in}{1.580982in}}{\pgfqpoint{1.124133in}{1.584254in}}{\pgfqpoint{1.115897in}{1.584254in}}%
\pgfpathcurveto{\pgfqpoint{1.107660in}{1.584254in}}{\pgfqpoint{1.099760in}{1.580982in}}{\pgfqpoint{1.093936in}{1.575158in}}%
\pgfpathcurveto{\pgfqpoint{1.088112in}{1.569334in}}{\pgfqpoint{1.084840in}{1.561434in}}{\pgfqpoint{1.084840in}{1.553198in}}%
\pgfpathcurveto{\pgfqpoint{1.084840in}{1.544962in}}{\pgfqpoint{1.088112in}{1.537061in}}{\pgfqpoint{1.093936in}{1.531238in}}%
\pgfpathcurveto{\pgfqpoint{1.099760in}{1.525414in}}{\pgfqpoint{1.107660in}{1.522141in}}{\pgfqpoint{1.115897in}{1.522141in}}%
\pgfpathclose%
\pgfusepath{stroke,fill}%
\end{pgfscope}%
\begin{pgfscope}%
\pgfpathrectangle{\pgfqpoint{0.100000in}{0.212622in}}{\pgfqpoint{3.696000in}{3.696000in}}%
\pgfusepath{clip}%
\pgfsetbuttcap%
\pgfsetroundjoin%
\definecolor{currentfill}{rgb}{0.121569,0.466667,0.705882}%
\pgfsetfillcolor{currentfill}%
\pgfsetfillopacity{0.555149}%
\pgfsetlinewidth{1.003750pt}%
\definecolor{currentstroke}{rgb}{0.121569,0.466667,0.705882}%
\pgfsetstrokecolor{currentstroke}%
\pgfsetstrokeopacity{0.555149}%
\pgfsetdash{}{0pt}%
\pgfpathmoveto{\pgfqpoint{1.108754in}{1.510773in}}%
\pgfpathcurveto{\pgfqpoint{1.116990in}{1.510773in}}{\pgfqpoint{1.124890in}{1.514045in}}{\pgfqpoint{1.130714in}{1.519869in}}%
\pgfpathcurveto{\pgfqpoint{1.136538in}{1.525693in}}{\pgfqpoint{1.139810in}{1.533593in}}{\pgfqpoint{1.139810in}{1.541829in}}%
\pgfpathcurveto{\pgfqpoint{1.139810in}{1.550065in}}{\pgfqpoint{1.136538in}{1.557965in}}{\pgfqpoint{1.130714in}{1.563789in}}%
\pgfpathcurveto{\pgfqpoint{1.124890in}{1.569613in}}{\pgfqpoint{1.116990in}{1.572886in}}{\pgfqpoint{1.108754in}{1.572886in}}%
\pgfpathcurveto{\pgfqpoint{1.100517in}{1.572886in}}{\pgfqpoint{1.092617in}{1.569613in}}{\pgfqpoint{1.086793in}{1.563789in}}%
\pgfpathcurveto{\pgfqpoint{1.080969in}{1.557965in}}{\pgfqpoint{1.077697in}{1.550065in}}{\pgfqpoint{1.077697in}{1.541829in}}%
\pgfpathcurveto{\pgfqpoint{1.077697in}{1.533593in}}{\pgfqpoint{1.080969in}{1.525693in}}{\pgfqpoint{1.086793in}{1.519869in}}%
\pgfpathcurveto{\pgfqpoint{1.092617in}{1.514045in}}{\pgfqpoint{1.100517in}{1.510773in}}{\pgfqpoint{1.108754in}{1.510773in}}%
\pgfpathclose%
\pgfusepath{stroke,fill}%
\end{pgfscope}%
\begin{pgfscope}%
\pgfpathrectangle{\pgfqpoint{0.100000in}{0.212622in}}{\pgfqpoint{3.696000in}{3.696000in}}%
\pgfusepath{clip}%
\pgfsetbuttcap%
\pgfsetroundjoin%
\definecolor{currentfill}{rgb}{0.121569,0.466667,0.705882}%
\pgfsetfillcolor{currentfill}%
\pgfsetfillopacity{0.556139}%
\pgfsetlinewidth{1.003750pt}%
\definecolor{currentstroke}{rgb}{0.121569,0.466667,0.705882}%
\pgfsetstrokecolor{currentstroke}%
\pgfsetstrokeopacity{0.556139}%
\pgfsetdash{}{0pt}%
\pgfpathmoveto{\pgfqpoint{1.100439in}{1.501123in}}%
\pgfpathcurveto{\pgfqpoint{1.108675in}{1.501123in}}{\pgfqpoint{1.116575in}{1.504395in}}{\pgfqpoint{1.122399in}{1.510219in}}%
\pgfpathcurveto{\pgfqpoint{1.128223in}{1.516043in}}{\pgfqpoint{1.131495in}{1.523943in}}{\pgfqpoint{1.131495in}{1.532180in}}%
\pgfpathcurveto{\pgfqpoint{1.131495in}{1.540416in}}{\pgfqpoint{1.128223in}{1.548316in}}{\pgfqpoint{1.122399in}{1.554140in}}%
\pgfpathcurveto{\pgfqpoint{1.116575in}{1.559964in}}{\pgfqpoint{1.108675in}{1.563236in}}{\pgfqpoint{1.100439in}{1.563236in}}%
\pgfpathcurveto{\pgfqpoint{1.092203in}{1.563236in}}{\pgfqpoint{1.084303in}{1.559964in}}{\pgfqpoint{1.078479in}{1.554140in}}%
\pgfpathcurveto{\pgfqpoint{1.072655in}{1.548316in}}{\pgfqpoint{1.069382in}{1.540416in}}{\pgfqpoint{1.069382in}{1.532180in}}%
\pgfpathcurveto{\pgfqpoint{1.069382in}{1.523943in}}{\pgfqpoint{1.072655in}{1.516043in}}{\pgfqpoint{1.078479in}{1.510219in}}%
\pgfpathcurveto{\pgfqpoint{1.084303in}{1.504395in}}{\pgfqpoint{1.092203in}{1.501123in}}{\pgfqpoint{1.100439in}{1.501123in}}%
\pgfpathclose%
\pgfusepath{stroke,fill}%
\end{pgfscope}%
\begin{pgfscope}%
\pgfpathrectangle{\pgfqpoint{0.100000in}{0.212622in}}{\pgfqpoint{3.696000in}{3.696000in}}%
\pgfusepath{clip}%
\pgfsetbuttcap%
\pgfsetroundjoin%
\definecolor{currentfill}{rgb}{0.121569,0.466667,0.705882}%
\pgfsetfillcolor{currentfill}%
\pgfsetfillopacity{0.557455}%
\pgfsetlinewidth{1.003750pt}%
\definecolor{currentstroke}{rgb}{0.121569,0.466667,0.705882}%
\pgfsetstrokecolor{currentstroke}%
\pgfsetstrokeopacity{0.557455}%
\pgfsetdash{}{0pt}%
\pgfpathmoveto{\pgfqpoint{2.081079in}{1.864326in}}%
\pgfpathcurveto{\pgfqpoint{2.089315in}{1.864326in}}{\pgfqpoint{2.097215in}{1.867598in}}{\pgfqpoint{2.103039in}{1.873422in}}%
\pgfpathcurveto{\pgfqpoint{2.108863in}{1.879246in}}{\pgfqpoint{2.112135in}{1.887146in}}{\pgfqpoint{2.112135in}{1.895382in}}%
\pgfpathcurveto{\pgfqpoint{2.112135in}{1.903619in}}{\pgfqpoint{2.108863in}{1.911519in}}{\pgfqpoint{2.103039in}{1.917343in}}%
\pgfpathcurveto{\pgfqpoint{2.097215in}{1.923167in}}{\pgfqpoint{2.089315in}{1.926439in}}{\pgfqpoint{2.081079in}{1.926439in}}%
\pgfpathcurveto{\pgfqpoint{2.072842in}{1.926439in}}{\pgfqpoint{2.064942in}{1.923167in}}{\pgfqpoint{2.059118in}{1.917343in}}%
\pgfpathcurveto{\pgfqpoint{2.053294in}{1.911519in}}{\pgfqpoint{2.050022in}{1.903619in}}{\pgfqpoint{2.050022in}{1.895382in}}%
\pgfpathcurveto{\pgfqpoint{2.050022in}{1.887146in}}{\pgfqpoint{2.053294in}{1.879246in}}{\pgfqpoint{2.059118in}{1.873422in}}%
\pgfpathcurveto{\pgfqpoint{2.064942in}{1.867598in}}{\pgfqpoint{2.072842in}{1.864326in}}{\pgfqpoint{2.081079in}{1.864326in}}%
\pgfpathclose%
\pgfusepath{stroke,fill}%
\end{pgfscope}%
\begin{pgfscope}%
\pgfpathrectangle{\pgfqpoint{0.100000in}{0.212622in}}{\pgfqpoint{3.696000in}{3.696000in}}%
\pgfusepath{clip}%
\pgfsetbuttcap%
\pgfsetroundjoin%
\definecolor{currentfill}{rgb}{0.121569,0.466667,0.705882}%
\pgfsetfillcolor{currentfill}%
\pgfsetfillopacity{0.558108}%
\pgfsetlinewidth{1.003750pt}%
\definecolor{currentstroke}{rgb}{0.121569,0.466667,0.705882}%
\pgfsetstrokecolor{currentstroke}%
\pgfsetstrokeopacity{0.558108}%
\pgfsetdash{}{0pt}%
\pgfpathmoveto{\pgfqpoint{1.093551in}{1.496456in}}%
\pgfpathcurveto{\pgfqpoint{1.101787in}{1.496456in}}{\pgfqpoint{1.109687in}{1.499728in}}{\pgfqpoint{1.115511in}{1.505552in}}%
\pgfpathcurveto{\pgfqpoint{1.121335in}{1.511376in}}{\pgfqpoint{1.124607in}{1.519276in}}{\pgfqpoint{1.124607in}{1.527512in}}%
\pgfpathcurveto{\pgfqpoint{1.124607in}{1.535749in}}{\pgfqpoint{1.121335in}{1.543649in}}{\pgfqpoint{1.115511in}{1.549473in}}%
\pgfpathcurveto{\pgfqpoint{1.109687in}{1.555296in}}{\pgfqpoint{1.101787in}{1.558569in}}{\pgfqpoint{1.093551in}{1.558569in}}%
\pgfpathcurveto{\pgfqpoint{1.085315in}{1.558569in}}{\pgfqpoint{1.077414in}{1.555296in}}{\pgfqpoint{1.071591in}{1.549473in}}%
\pgfpathcurveto{\pgfqpoint{1.065767in}{1.543649in}}{\pgfqpoint{1.062494in}{1.535749in}}{\pgfqpoint{1.062494in}{1.527512in}}%
\pgfpathcurveto{\pgfqpoint{1.062494in}{1.519276in}}{\pgfqpoint{1.065767in}{1.511376in}}{\pgfqpoint{1.071591in}{1.505552in}}%
\pgfpathcurveto{\pgfqpoint{1.077414in}{1.499728in}}{\pgfqpoint{1.085315in}{1.496456in}}{\pgfqpoint{1.093551in}{1.496456in}}%
\pgfpathclose%
\pgfusepath{stroke,fill}%
\end{pgfscope}%
\begin{pgfscope}%
\pgfpathrectangle{\pgfqpoint{0.100000in}{0.212622in}}{\pgfqpoint{3.696000in}{3.696000in}}%
\pgfusepath{clip}%
\pgfsetbuttcap%
\pgfsetroundjoin%
\definecolor{currentfill}{rgb}{0.121569,0.466667,0.705882}%
\pgfsetfillcolor{currentfill}%
\pgfsetfillopacity{0.559588}%
\pgfsetlinewidth{1.003750pt}%
\definecolor{currentstroke}{rgb}{0.121569,0.466667,0.705882}%
\pgfsetstrokecolor{currentstroke}%
\pgfsetstrokeopacity{0.559588}%
\pgfsetdash{}{0pt}%
\pgfpathmoveto{\pgfqpoint{1.089065in}{1.488428in}}%
\pgfpathcurveto{\pgfqpoint{1.097301in}{1.488428in}}{\pgfqpoint{1.105202in}{1.491701in}}{\pgfqpoint{1.111025in}{1.497525in}}%
\pgfpathcurveto{\pgfqpoint{1.116849in}{1.503349in}}{\pgfqpoint{1.120122in}{1.511249in}}{\pgfqpoint{1.120122in}{1.519485in}}%
\pgfpathcurveto{\pgfqpoint{1.120122in}{1.527721in}}{\pgfqpoint{1.116849in}{1.535621in}}{\pgfqpoint{1.111025in}{1.541445in}}%
\pgfpathcurveto{\pgfqpoint{1.105202in}{1.547269in}}{\pgfqpoint{1.097301in}{1.550541in}}{\pgfqpoint{1.089065in}{1.550541in}}%
\pgfpathcurveto{\pgfqpoint{1.080829in}{1.550541in}}{\pgfqpoint{1.072929in}{1.547269in}}{\pgfqpoint{1.067105in}{1.541445in}}%
\pgfpathcurveto{\pgfqpoint{1.061281in}{1.535621in}}{\pgfqpoint{1.058009in}{1.527721in}}{\pgfqpoint{1.058009in}{1.519485in}}%
\pgfpathcurveto{\pgfqpoint{1.058009in}{1.511249in}}{\pgfqpoint{1.061281in}{1.503349in}}{\pgfqpoint{1.067105in}{1.497525in}}%
\pgfpathcurveto{\pgfqpoint{1.072929in}{1.491701in}}{\pgfqpoint{1.080829in}{1.488428in}}{\pgfqpoint{1.089065in}{1.488428in}}%
\pgfpathclose%
\pgfusepath{stroke,fill}%
\end{pgfscope}%
\begin{pgfscope}%
\pgfpathrectangle{\pgfqpoint{0.100000in}{0.212622in}}{\pgfqpoint{3.696000in}{3.696000in}}%
\pgfusepath{clip}%
\pgfsetbuttcap%
\pgfsetroundjoin%
\definecolor{currentfill}{rgb}{0.121569,0.466667,0.705882}%
\pgfsetfillcolor{currentfill}%
\pgfsetfillopacity{0.560766}%
\pgfsetlinewidth{1.003750pt}%
\definecolor{currentstroke}{rgb}{0.121569,0.466667,0.705882}%
\pgfsetstrokecolor{currentstroke}%
\pgfsetstrokeopacity{0.560766}%
\pgfsetdash{}{0pt}%
\pgfpathmoveto{\pgfqpoint{1.085137in}{1.484923in}}%
\pgfpathcurveto{\pgfqpoint{1.093373in}{1.484923in}}{\pgfqpoint{1.101273in}{1.488195in}}{\pgfqpoint{1.107097in}{1.494019in}}%
\pgfpathcurveto{\pgfqpoint{1.112921in}{1.499843in}}{\pgfqpoint{1.116193in}{1.507743in}}{\pgfqpoint{1.116193in}{1.515979in}}%
\pgfpathcurveto{\pgfqpoint{1.116193in}{1.524216in}}{\pgfqpoint{1.112921in}{1.532116in}}{\pgfqpoint{1.107097in}{1.537940in}}%
\pgfpathcurveto{\pgfqpoint{1.101273in}{1.543763in}}{\pgfqpoint{1.093373in}{1.547036in}}{\pgfqpoint{1.085137in}{1.547036in}}%
\pgfpathcurveto{\pgfqpoint{1.076900in}{1.547036in}}{\pgfqpoint{1.069000in}{1.543763in}}{\pgfqpoint{1.063176in}{1.537940in}}%
\pgfpathcurveto{\pgfqpoint{1.057352in}{1.532116in}}{\pgfqpoint{1.054080in}{1.524216in}}{\pgfqpoint{1.054080in}{1.515979in}}%
\pgfpathcurveto{\pgfqpoint{1.054080in}{1.507743in}}{\pgfqpoint{1.057352in}{1.499843in}}{\pgfqpoint{1.063176in}{1.494019in}}%
\pgfpathcurveto{\pgfqpoint{1.069000in}{1.488195in}}{\pgfqpoint{1.076900in}{1.484923in}}{\pgfqpoint{1.085137in}{1.484923in}}%
\pgfpathclose%
\pgfusepath{stroke,fill}%
\end{pgfscope}%
\begin{pgfscope}%
\pgfpathrectangle{\pgfqpoint{0.100000in}{0.212622in}}{\pgfqpoint{3.696000in}{3.696000in}}%
\pgfusepath{clip}%
\pgfsetbuttcap%
\pgfsetroundjoin%
\definecolor{currentfill}{rgb}{0.121569,0.466667,0.705882}%
\pgfsetfillcolor{currentfill}%
\pgfsetfillopacity{0.562649}%
\pgfsetlinewidth{1.003750pt}%
\definecolor{currentstroke}{rgb}{0.121569,0.466667,0.705882}%
\pgfsetstrokecolor{currentstroke}%
\pgfsetstrokeopacity{0.562649}%
\pgfsetdash{}{0pt}%
\pgfpathmoveto{\pgfqpoint{2.083961in}{1.860722in}}%
\pgfpathcurveto{\pgfqpoint{2.092197in}{1.860722in}}{\pgfqpoint{2.100097in}{1.863995in}}{\pgfqpoint{2.105921in}{1.869819in}}%
\pgfpathcurveto{\pgfqpoint{2.111745in}{1.875643in}}{\pgfqpoint{2.115017in}{1.883543in}}{\pgfqpoint{2.115017in}{1.891779in}}%
\pgfpathcurveto{\pgfqpoint{2.115017in}{1.900015in}}{\pgfqpoint{2.111745in}{1.907915in}}{\pgfqpoint{2.105921in}{1.913739in}}%
\pgfpathcurveto{\pgfqpoint{2.100097in}{1.919563in}}{\pgfqpoint{2.092197in}{1.922835in}}{\pgfqpoint{2.083961in}{1.922835in}}%
\pgfpathcurveto{\pgfqpoint{2.075725in}{1.922835in}}{\pgfqpoint{2.067825in}{1.919563in}}{\pgfqpoint{2.062001in}{1.913739in}}%
\pgfpathcurveto{\pgfqpoint{2.056177in}{1.907915in}}{\pgfqpoint{2.052904in}{1.900015in}}{\pgfqpoint{2.052904in}{1.891779in}}%
\pgfpathcurveto{\pgfqpoint{2.052904in}{1.883543in}}{\pgfqpoint{2.056177in}{1.875643in}}{\pgfqpoint{2.062001in}{1.869819in}}%
\pgfpathcurveto{\pgfqpoint{2.067825in}{1.863995in}}{\pgfqpoint{2.075725in}{1.860722in}}{\pgfqpoint{2.083961in}{1.860722in}}%
\pgfpathclose%
\pgfusepath{stroke,fill}%
\end{pgfscope}%
\begin{pgfscope}%
\pgfpathrectangle{\pgfqpoint{0.100000in}{0.212622in}}{\pgfqpoint{3.696000in}{3.696000in}}%
\pgfusepath{clip}%
\pgfsetbuttcap%
\pgfsetroundjoin%
\definecolor{currentfill}{rgb}{0.121569,0.466667,0.705882}%
\pgfsetfillcolor{currentfill}%
\pgfsetfillopacity{0.563109}%
\pgfsetlinewidth{1.003750pt}%
\definecolor{currentstroke}{rgb}{0.121569,0.466667,0.705882}%
\pgfsetstrokecolor{currentstroke}%
\pgfsetstrokeopacity{0.563109}%
\pgfsetdash{}{0pt}%
\pgfpathmoveto{\pgfqpoint{1.078363in}{1.479314in}}%
\pgfpathcurveto{\pgfqpoint{1.086599in}{1.479314in}}{\pgfqpoint{1.094499in}{1.482586in}}{\pgfqpoint{1.100323in}{1.488410in}}%
\pgfpathcurveto{\pgfqpoint{1.106147in}{1.494234in}}{\pgfqpoint{1.109419in}{1.502134in}}{\pgfqpoint{1.109419in}{1.510371in}}%
\pgfpathcurveto{\pgfqpoint{1.109419in}{1.518607in}}{\pgfqpoint{1.106147in}{1.526507in}}{\pgfqpoint{1.100323in}{1.532331in}}%
\pgfpathcurveto{\pgfqpoint{1.094499in}{1.538155in}}{\pgfqpoint{1.086599in}{1.541427in}}{\pgfqpoint{1.078363in}{1.541427in}}%
\pgfpathcurveto{\pgfqpoint{1.070126in}{1.541427in}}{\pgfqpoint{1.062226in}{1.538155in}}{\pgfqpoint{1.056402in}{1.532331in}}%
\pgfpathcurveto{\pgfqpoint{1.050578in}{1.526507in}}{\pgfqpoint{1.047306in}{1.518607in}}{\pgfqpoint{1.047306in}{1.510371in}}%
\pgfpathcurveto{\pgfqpoint{1.047306in}{1.502134in}}{\pgfqpoint{1.050578in}{1.494234in}}{\pgfqpoint{1.056402in}{1.488410in}}%
\pgfpathcurveto{\pgfqpoint{1.062226in}{1.482586in}}{\pgfqpoint{1.070126in}{1.479314in}}{\pgfqpoint{1.078363in}{1.479314in}}%
\pgfpathclose%
\pgfusepath{stroke,fill}%
\end{pgfscope}%
\begin{pgfscope}%
\pgfpathrectangle{\pgfqpoint{0.100000in}{0.212622in}}{\pgfqpoint{3.696000in}{3.696000in}}%
\pgfusepath{clip}%
\pgfsetbuttcap%
\pgfsetroundjoin%
\definecolor{currentfill}{rgb}{0.121569,0.466667,0.705882}%
\pgfsetfillcolor{currentfill}%
\pgfsetfillopacity{0.566754}%
\pgfsetlinewidth{1.003750pt}%
\definecolor{currentstroke}{rgb}{0.121569,0.466667,0.705882}%
\pgfsetstrokecolor{currentstroke}%
\pgfsetstrokeopacity{0.566754}%
\pgfsetdash{}{0pt}%
\pgfpathmoveto{\pgfqpoint{1.067612in}{1.463442in}}%
\pgfpathcurveto{\pgfqpoint{1.075849in}{1.463442in}}{\pgfqpoint{1.083749in}{1.466714in}}{\pgfqpoint{1.089573in}{1.472538in}}%
\pgfpathcurveto{\pgfqpoint{1.095397in}{1.478362in}}{\pgfqpoint{1.098669in}{1.486262in}}{\pgfqpoint{1.098669in}{1.494498in}}%
\pgfpathcurveto{\pgfqpoint{1.098669in}{1.502735in}}{\pgfqpoint{1.095397in}{1.510635in}}{\pgfqpoint{1.089573in}{1.516459in}}%
\pgfpathcurveto{\pgfqpoint{1.083749in}{1.522283in}}{\pgfqpoint{1.075849in}{1.525555in}}{\pgfqpoint{1.067612in}{1.525555in}}%
\pgfpathcurveto{\pgfqpoint{1.059376in}{1.525555in}}{\pgfqpoint{1.051476in}{1.522283in}}{\pgfqpoint{1.045652in}{1.516459in}}%
\pgfpathcurveto{\pgfqpoint{1.039828in}{1.510635in}}{\pgfqpoint{1.036556in}{1.502735in}}{\pgfqpoint{1.036556in}{1.494498in}}%
\pgfpathcurveto{\pgfqpoint{1.036556in}{1.486262in}}{\pgfqpoint{1.039828in}{1.478362in}}{\pgfqpoint{1.045652in}{1.472538in}}%
\pgfpathcurveto{\pgfqpoint{1.051476in}{1.466714in}}{\pgfqpoint{1.059376in}{1.463442in}}{\pgfqpoint{1.067612in}{1.463442in}}%
\pgfpathclose%
\pgfusepath{stroke,fill}%
\end{pgfscope}%
\begin{pgfscope}%
\pgfpathrectangle{\pgfqpoint{0.100000in}{0.212622in}}{\pgfqpoint{3.696000in}{3.696000in}}%
\pgfusepath{clip}%
\pgfsetbuttcap%
\pgfsetroundjoin%
\definecolor{currentfill}{rgb}{0.121569,0.466667,0.705882}%
\pgfsetfillcolor{currentfill}%
\pgfsetfillopacity{0.568627}%
\pgfsetlinewidth{1.003750pt}%
\definecolor{currentstroke}{rgb}{0.121569,0.466667,0.705882}%
\pgfsetstrokecolor{currentstroke}%
\pgfsetstrokeopacity{0.568627}%
\pgfsetdash{}{0pt}%
\pgfpathmoveto{\pgfqpoint{2.087834in}{1.856452in}}%
\pgfpathcurveto{\pgfqpoint{2.096070in}{1.856452in}}{\pgfqpoint{2.103970in}{1.859724in}}{\pgfqpoint{2.109794in}{1.865548in}}%
\pgfpathcurveto{\pgfqpoint{2.115618in}{1.871372in}}{\pgfqpoint{2.118891in}{1.879272in}}{\pgfqpoint{2.118891in}{1.887508in}}%
\pgfpathcurveto{\pgfqpoint{2.118891in}{1.895744in}}{\pgfqpoint{2.115618in}{1.903645in}}{\pgfqpoint{2.109794in}{1.909468in}}%
\pgfpathcurveto{\pgfqpoint{2.103970in}{1.915292in}}{\pgfqpoint{2.096070in}{1.918565in}}{\pgfqpoint{2.087834in}{1.918565in}}%
\pgfpathcurveto{\pgfqpoint{2.079598in}{1.918565in}}{\pgfqpoint{2.071698in}{1.915292in}}{\pgfqpoint{2.065874in}{1.909468in}}%
\pgfpathcurveto{\pgfqpoint{2.060050in}{1.903645in}}{\pgfqpoint{2.056778in}{1.895744in}}{\pgfqpoint{2.056778in}{1.887508in}}%
\pgfpathcurveto{\pgfqpoint{2.056778in}{1.879272in}}{\pgfqpoint{2.060050in}{1.871372in}}{\pgfqpoint{2.065874in}{1.865548in}}%
\pgfpathcurveto{\pgfqpoint{2.071698in}{1.859724in}}{\pgfqpoint{2.079598in}{1.856452in}}{\pgfqpoint{2.087834in}{1.856452in}}%
\pgfpathclose%
\pgfusepath{stroke,fill}%
\end{pgfscope}%
\begin{pgfscope}%
\pgfpathrectangle{\pgfqpoint{0.100000in}{0.212622in}}{\pgfqpoint{3.696000in}{3.696000in}}%
\pgfusepath{clip}%
\pgfsetbuttcap%
\pgfsetroundjoin%
\definecolor{currentfill}{rgb}{0.121569,0.466667,0.705882}%
\pgfsetfillcolor{currentfill}%
\pgfsetfillopacity{0.569857}%
\pgfsetlinewidth{1.003750pt}%
\definecolor{currentstroke}{rgb}{0.121569,0.466667,0.705882}%
\pgfsetstrokecolor{currentstroke}%
\pgfsetstrokeopacity{0.569857}%
\pgfsetdash{}{0pt}%
\pgfpathmoveto{\pgfqpoint{1.056102in}{1.454956in}}%
\pgfpathcurveto{\pgfqpoint{1.064338in}{1.454956in}}{\pgfqpoint{1.072239in}{1.458228in}}{\pgfqpoint{1.078062in}{1.464052in}}%
\pgfpathcurveto{\pgfqpoint{1.083886in}{1.469876in}}{\pgfqpoint{1.087159in}{1.477776in}}{\pgfqpoint{1.087159in}{1.486012in}}%
\pgfpathcurveto{\pgfqpoint{1.087159in}{1.494249in}}{\pgfqpoint{1.083886in}{1.502149in}}{\pgfqpoint{1.078062in}{1.507973in}}%
\pgfpathcurveto{\pgfqpoint{1.072239in}{1.513797in}}{\pgfqpoint{1.064338in}{1.517069in}}{\pgfqpoint{1.056102in}{1.517069in}}%
\pgfpathcurveto{\pgfqpoint{1.047866in}{1.517069in}}{\pgfqpoint{1.039966in}{1.513797in}}{\pgfqpoint{1.034142in}{1.507973in}}%
\pgfpathcurveto{\pgfqpoint{1.028318in}{1.502149in}}{\pgfqpoint{1.025046in}{1.494249in}}{\pgfqpoint{1.025046in}{1.486012in}}%
\pgfpathcurveto{\pgfqpoint{1.025046in}{1.477776in}}{\pgfqpoint{1.028318in}{1.469876in}}{\pgfqpoint{1.034142in}{1.464052in}}%
\pgfpathcurveto{\pgfqpoint{1.039966in}{1.458228in}}{\pgfqpoint{1.047866in}{1.454956in}}{\pgfqpoint{1.056102in}{1.454956in}}%
\pgfpathclose%
\pgfusepath{stroke,fill}%
\end{pgfscope}%
\begin{pgfscope}%
\pgfpathrectangle{\pgfqpoint{0.100000in}{0.212622in}}{\pgfqpoint{3.696000in}{3.696000in}}%
\pgfusepath{clip}%
\pgfsetbuttcap%
\pgfsetroundjoin%
\definecolor{currentfill}{rgb}{0.121569,0.466667,0.705882}%
\pgfsetfillcolor{currentfill}%
\pgfsetfillopacity{0.571872}%
\pgfsetlinewidth{1.003750pt}%
\definecolor{currentstroke}{rgb}{0.121569,0.466667,0.705882}%
\pgfsetstrokecolor{currentstroke}%
\pgfsetstrokeopacity{0.571872}%
\pgfsetdash{}{0pt}%
\pgfpathmoveto{\pgfqpoint{2.089962in}{1.853774in}}%
\pgfpathcurveto{\pgfqpoint{2.098198in}{1.853774in}}{\pgfqpoint{2.106098in}{1.857046in}}{\pgfqpoint{2.111922in}{1.862870in}}%
\pgfpathcurveto{\pgfqpoint{2.117746in}{1.868694in}}{\pgfqpoint{2.121018in}{1.876594in}}{\pgfqpoint{2.121018in}{1.884830in}}%
\pgfpathcurveto{\pgfqpoint{2.121018in}{1.893067in}}{\pgfqpoint{2.117746in}{1.900967in}}{\pgfqpoint{2.111922in}{1.906791in}}%
\pgfpathcurveto{\pgfqpoint{2.106098in}{1.912615in}}{\pgfqpoint{2.098198in}{1.915887in}}{\pgfqpoint{2.089962in}{1.915887in}}%
\pgfpathcurveto{\pgfqpoint{2.081726in}{1.915887in}}{\pgfqpoint{2.073826in}{1.912615in}}{\pgfqpoint{2.068002in}{1.906791in}}%
\pgfpathcurveto{\pgfqpoint{2.062178in}{1.900967in}}{\pgfqpoint{2.058905in}{1.893067in}}{\pgfqpoint{2.058905in}{1.884830in}}%
\pgfpathcurveto{\pgfqpoint{2.058905in}{1.876594in}}{\pgfqpoint{2.062178in}{1.868694in}}{\pgfqpoint{2.068002in}{1.862870in}}%
\pgfpathcurveto{\pgfqpoint{2.073826in}{1.857046in}}{\pgfqpoint{2.081726in}{1.853774in}}{\pgfqpoint{2.089962in}{1.853774in}}%
\pgfpathclose%
\pgfusepath{stroke,fill}%
\end{pgfscope}%
\begin{pgfscope}%
\pgfpathrectangle{\pgfqpoint{0.100000in}{0.212622in}}{\pgfqpoint{3.696000in}{3.696000in}}%
\pgfusepath{clip}%
\pgfsetbuttcap%
\pgfsetroundjoin%
\definecolor{currentfill}{rgb}{0.121569,0.466667,0.705882}%
\pgfsetfillcolor{currentfill}%
\pgfsetfillopacity{0.572744}%
\pgfsetlinewidth{1.003750pt}%
\definecolor{currentstroke}{rgb}{0.121569,0.466667,0.705882}%
\pgfsetstrokecolor{currentstroke}%
\pgfsetstrokeopacity{0.572744}%
\pgfsetdash{}{0pt}%
\pgfpathmoveto{\pgfqpoint{1.045913in}{1.446992in}}%
\pgfpathcurveto{\pgfqpoint{1.054149in}{1.446992in}}{\pgfqpoint{1.062049in}{1.450264in}}{\pgfqpoint{1.067873in}{1.456088in}}%
\pgfpathcurveto{\pgfqpoint{1.073697in}{1.461912in}}{\pgfqpoint{1.076969in}{1.469812in}}{\pgfqpoint{1.076969in}{1.478049in}}%
\pgfpathcurveto{\pgfqpoint{1.076969in}{1.486285in}}{\pgfqpoint{1.073697in}{1.494185in}}{\pgfqpoint{1.067873in}{1.500009in}}%
\pgfpathcurveto{\pgfqpoint{1.062049in}{1.505833in}}{\pgfqpoint{1.054149in}{1.509105in}}{\pgfqpoint{1.045913in}{1.509105in}}%
\pgfpathcurveto{\pgfqpoint{1.037676in}{1.509105in}}{\pgfqpoint{1.029776in}{1.505833in}}{\pgfqpoint{1.023952in}{1.500009in}}%
\pgfpathcurveto{\pgfqpoint{1.018128in}{1.494185in}}{\pgfqpoint{1.014856in}{1.486285in}}{\pgfqpoint{1.014856in}{1.478049in}}%
\pgfpathcurveto{\pgfqpoint{1.014856in}{1.469812in}}{\pgfqpoint{1.018128in}{1.461912in}}{\pgfqpoint{1.023952in}{1.456088in}}%
\pgfpathcurveto{\pgfqpoint{1.029776in}{1.450264in}}{\pgfqpoint{1.037676in}{1.446992in}}{\pgfqpoint{1.045913in}{1.446992in}}%
\pgfpathclose%
\pgfusepath{stroke,fill}%
\end{pgfscope}%
\begin{pgfscope}%
\pgfpathrectangle{\pgfqpoint{0.100000in}{0.212622in}}{\pgfqpoint{3.696000in}{3.696000in}}%
\pgfusepath{clip}%
\pgfsetbuttcap%
\pgfsetroundjoin%
\definecolor{currentfill}{rgb}{0.121569,0.466667,0.705882}%
\pgfsetfillcolor{currentfill}%
\pgfsetfillopacity{0.575160}%
\pgfsetlinewidth{1.003750pt}%
\definecolor{currentstroke}{rgb}{0.121569,0.466667,0.705882}%
\pgfsetstrokecolor{currentstroke}%
\pgfsetstrokeopacity{0.575160}%
\pgfsetdash{}{0pt}%
\pgfpathmoveto{\pgfqpoint{1.038305in}{1.434304in}}%
\pgfpathcurveto{\pgfqpoint{1.046541in}{1.434304in}}{\pgfqpoint{1.054441in}{1.437576in}}{\pgfqpoint{1.060265in}{1.443400in}}%
\pgfpathcurveto{\pgfqpoint{1.066089in}{1.449224in}}{\pgfqpoint{1.069361in}{1.457124in}}{\pgfqpoint{1.069361in}{1.465360in}}%
\pgfpathcurveto{\pgfqpoint{1.069361in}{1.473596in}}{\pgfqpoint{1.066089in}{1.481496in}}{\pgfqpoint{1.060265in}{1.487320in}}%
\pgfpathcurveto{\pgfqpoint{1.054441in}{1.493144in}}{\pgfqpoint{1.046541in}{1.496417in}}{\pgfqpoint{1.038305in}{1.496417in}}%
\pgfpathcurveto{\pgfqpoint{1.030068in}{1.496417in}}{\pgfqpoint{1.022168in}{1.493144in}}{\pgfqpoint{1.016344in}{1.487320in}}%
\pgfpathcurveto{\pgfqpoint{1.010520in}{1.481496in}}{\pgfqpoint{1.007248in}{1.473596in}}{\pgfqpoint{1.007248in}{1.465360in}}%
\pgfpathcurveto{\pgfqpoint{1.007248in}{1.457124in}}{\pgfqpoint{1.010520in}{1.449224in}}{\pgfqpoint{1.016344in}{1.443400in}}%
\pgfpathcurveto{\pgfqpoint{1.022168in}{1.437576in}}{\pgfqpoint{1.030068in}{1.434304in}}{\pgfqpoint{1.038305in}{1.434304in}}%
\pgfpathclose%
\pgfusepath{stroke,fill}%
\end{pgfscope}%
\begin{pgfscope}%
\pgfpathrectangle{\pgfqpoint{0.100000in}{0.212622in}}{\pgfqpoint{3.696000in}{3.696000in}}%
\pgfusepath{clip}%
\pgfsetbuttcap%
\pgfsetroundjoin%
\definecolor{currentfill}{rgb}{0.121569,0.466667,0.705882}%
\pgfsetfillcolor{currentfill}%
\pgfsetfillopacity{0.575777}%
\pgfsetlinewidth{1.003750pt}%
\definecolor{currentstroke}{rgb}{0.121569,0.466667,0.705882}%
\pgfsetstrokecolor{currentstroke}%
\pgfsetstrokeopacity{0.575777}%
\pgfsetdash{}{0pt}%
\pgfpathmoveto{\pgfqpoint{2.092038in}{1.850586in}}%
\pgfpathcurveto{\pgfqpoint{2.100275in}{1.850586in}}{\pgfqpoint{2.108175in}{1.853859in}}{\pgfqpoint{2.113999in}{1.859683in}}%
\pgfpathcurveto{\pgfqpoint{2.119823in}{1.865507in}}{\pgfqpoint{2.123095in}{1.873407in}}{\pgfqpoint{2.123095in}{1.881643in}}%
\pgfpathcurveto{\pgfqpoint{2.123095in}{1.889879in}}{\pgfqpoint{2.119823in}{1.897779in}}{\pgfqpoint{2.113999in}{1.903603in}}%
\pgfpathcurveto{\pgfqpoint{2.108175in}{1.909427in}}{\pgfqpoint{2.100275in}{1.912699in}}{\pgfqpoint{2.092038in}{1.912699in}}%
\pgfpathcurveto{\pgfqpoint{2.083802in}{1.912699in}}{\pgfqpoint{2.075902in}{1.909427in}}{\pgfqpoint{2.070078in}{1.903603in}}%
\pgfpathcurveto{\pgfqpoint{2.064254in}{1.897779in}}{\pgfqpoint{2.060982in}{1.889879in}}{\pgfqpoint{2.060982in}{1.881643in}}%
\pgfpathcurveto{\pgfqpoint{2.060982in}{1.873407in}}{\pgfqpoint{2.064254in}{1.865507in}}{\pgfqpoint{2.070078in}{1.859683in}}%
\pgfpathcurveto{\pgfqpoint{2.075902in}{1.853859in}}{\pgfqpoint{2.083802in}{1.850586in}}{\pgfqpoint{2.092038in}{1.850586in}}%
\pgfpathclose%
\pgfusepath{stroke,fill}%
\end{pgfscope}%
\begin{pgfscope}%
\pgfpathrectangle{\pgfqpoint{0.100000in}{0.212622in}}{\pgfqpoint{3.696000in}{3.696000in}}%
\pgfusepath{clip}%
\pgfsetbuttcap%
\pgfsetroundjoin%
\definecolor{currentfill}{rgb}{0.121569,0.466667,0.705882}%
\pgfsetfillcolor{currentfill}%
\pgfsetfillopacity{0.577546}%
\pgfsetlinewidth{1.003750pt}%
\definecolor{currentstroke}{rgb}{0.121569,0.466667,0.705882}%
\pgfsetstrokecolor{currentstroke}%
\pgfsetstrokeopacity{0.577546}%
\pgfsetdash{}{0pt}%
\pgfpathmoveto{\pgfqpoint{1.030911in}{1.429349in}}%
\pgfpathcurveto{\pgfqpoint{1.039147in}{1.429349in}}{\pgfqpoint{1.047047in}{1.432622in}}{\pgfqpoint{1.052871in}{1.438445in}}%
\pgfpathcurveto{\pgfqpoint{1.058695in}{1.444269in}}{\pgfqpoint{1.061967in}{1.452169in}}{\pgfqpoint{1.061967in}{1.460406in}}%
\pgfpathcurveto{\pgfqpoint{1.061967in}{1.468642in}}{\pgfqpoint{1.058695in}{1.476542in}}{\pgfqpoint{1.052871in}{1.482366in}}%
\pgfpathcurveto{\pgfqpoint{1.047047in}{1.488190in}}{\pgfqpoint{1.039147in}{1.491462in}}{\pgfqpoint{1.030911in}{1.491462in}}%
\pgfpathcurveto{\pgfqpoint{1.022674in}{1.491462in}}{\pgfqpoint{1.014774in}{1.488190in}}{\pgfqpoint{1.008950in}{1.482366in}}%
\pgfpathcurveto{\pgfqpoint{1.003127in}{1.476542in}}{\pgfqpoint{0.999854in}{1.468642in}}{\pgfqpoint{0.999854in}{1.460406in}}%
\pgfpathcurveto{\pgfqpoint{0.999854in}{1.452169in}}{\pgfqpoint{1.003127in}{1.444269in}}{\pgfqpoint{1.008950in}{1.438445in}}%
\pgfpathcurveto{\pgfqpoint{1.014774in}{1.432622in}}{\pgfqpoint{1.022674in}{1.429349in}}{\pgfqpoint{1.030911in}{1.429349in}}%
\pgfpathclose%
\pgfusepath{stroke,fill}%
\end{pgfscope}%
\begin{pgfscope}%
\pgfpathrectangle{\pgfqpoint{0.100000in}{0.212622in}}{\pgfqpoint{3.696000in}{3.696000in}}%
\pgfusepath{clip}%
\pgfsetbuttcap%
\pgfsetroundjoin%
\definecolor{currentfill}{rgb}{0.121569,0.466667,0.705882}%
\pgfsetfillcolor{currentfill}%
\pgfsetfillopacity{0.577945}%
\pgfsetlinewidth{1.003750pt}%
\definecolor{currentstroke}{rgb}{0.121569,0.466667,0.705882}%
\pgfsetstrokecolor{currentstroke}%
\pgfsetstrokeopacity{0.577945}%
\pgfsetdash{}{0pt}%
\pgfpathmoveto{\pgfqpoint{2.093486in}{1.849104in}}%
\pgfpathcurveto{\pgfqpoint{2.101722in}{1.849104in}}{\pgfqpoint{2.109623in}{1.852376in}}{\pgfqpoint{2.115446in}{1.858200in}}%
\pgfpathcurveto{\pgfqpoint{2.121270in}{1.864024in}}{\pgfqpoint{2.124543in}{1.871924in}}{\pgfqpoint{2.124543in}{1.880160in}}%
\pgfpathcurveto{\pgfqpoint{2.124543in}{1.888397in}}{\pgfqpoint{2.121270in}{1.896297in}}{\pgfqpoint{2.115446in}{1.902121in}}%
\pgfpathcurveto{\pgfqpoint{2.109623in}{1.907945in}}{\pgfqpoint{2.101722in}{1.911217in}}{\pgfqpoint{2.093486in}{1.911217in}}%
\pgfpathcurveto{\pgfqpoint{2.085250in}{1.911217in}}{\pgfqpoint{2.077350in}{1.907945in}}{\pgfqpoint{2.071526in}{1.902121in}}%
\pgfpathcurveto{\pgfqpoint{2.065702in}{1.896297in}}{\pgfqpoint{2.062430in}{1.888397in}}{\pgfqpoint{2.062430in}{1.880160in}}%
\pgfpathcurveto{\pgfqpoint{2.062430in}{1.871924in}}{\pgfqpoint{2.065702in}{1.864024in}}{\pgfqpoint{2.071526in}{1.858200in}}%
\pgfpathcurveto{\pgfqpoint{2.077350in}{1.852376in}}{\pgfqpoint{2.085250in}{1.849104in}}{\pgfqpoint{2.093486in}{1.849104in}}%
\pgfpathclose%
\pgfusepath{stroke,fill}%
\end{pgfscope}%
\begin{pgfscope}%
\pgfpathrectangle{\pgfqpoint{0.100000in}{0.212622in}}{\pgfqpoint{3.696000in}{3.696000in}}%
\pgfusepath{clip}%
\pgfsetbuttcap%
\pgfsetroundjoin%
\definecolor{currentfill}{rgb}{0.121569,0.466667,0.705882}%
\pgfsetfillcolor{currentfill}%
\pgfsetfillopacity{0.579096}%
\pgfsetlinewidth{1.003750pt}%
\definecolor{currentstroke}{rgb}{0.121569,0.466667,0.705882}%
\pgfsetstrokecolor{currentstroke}%
\pgfsetstrokeopacity{0.579096}%
\pgfsetdash{}{0pt}%
\pgfpathmoveto{\pgfqpoint{2.094417in}{1.848093in}}%
\pgfpathcurveto{\pgfqpoint{2.102653in}{1.848093in}}{\pgfqpoint{2.110553in}{1.851365in}}{\pgfqpoint{2.116377in}{1.857189in}}%
\pgfpathcurveto{\pgfqpoint{2.122201in}{1.863013in}}{\pgfqpoint{2.125474in}{1.870913in}}{\pgfqpoint{2.125474in}{1.879149in}}%
\pgfpathcurveto{\pgfqpoint{2.125474in}{1.887385in}}{\pgfqpoint{2.122201in}{1.895286in}}{\pgfqpoint{2.116377in}{1.901109in}}%
\pgfpathcurveto{\pgfqpoint{2.110553in}{1.906933in}}{\pgfqpoint{2.102653in}{1.910206in}}{\pgfqpoint{2.094417in}{1.910206in}}%
\pgfpathcurveto{\pgfqpoint{2.086181in}{1.910206in}}{\pgfqpoint{2.078281in}{1.906933in}}{\pgfqpoint{2.072457in}{1.901109in}}%
\pgfpathcurveto{\pgfqpoint{2.066633in}{1.895286in}}{\pgfqpoint{2.063361in}{1.887385in}}{\pgfqpoint{2.063361in}{1.879149in}}%
\pgfpathcurveto{\pgfqpoint{2.063361in}{1.870913in}}{\pgfqpoint{2.066633in}{1.863013in}}{\pgfqpoint{2.072457in}{1.857189in}}%
\pgfpathcurveto{\pgfqpoint{2.078281in}{1.851365in}}{\pgfqpoint{2.086181in}{1.848093in}}{\pgfqpoint{2.094417in}{1.848093in}}%
\pgfpathclose%
\pgfusepath{stroke,fill}%
\end{pgfscope}%
\begin{pgfscope}%
\pgfpathrectangle{\pgfqpoint{0.100000in}{0.212622in}}{\pgfqpoint{3.696000in}{3.696000in}}%
\pgfusepath{clip}%
\pgfsetbuttcap%
\pgfsetroundjoin%
\definecolor{currentfill}{rgb}{0.121569,0.466667,0.705882}%
\pgfsetfillcolor{currentfill}%
\pgfsetfillopacity{0.579652}%
\pgfsetlinewidth{1.003750pt}%
\definecolor{currentstroke}{rgb}{0.121569,0.466667,0.705882}%
\pgfsetstrokecolor{currentstroke}%
\pgfsetstrokeopacity{0.579652}%
\pgfsetdash{}{0pt}%
\pgfpathmoveto{\pgfqpoint{1.023691in}{1.425027in}}%
\pgfpathcurveto{\pgfqpoint{1.031927in}{1.425027in}}{\pgfqpoint{1.039827in}{1.428299in}}{\pgfqpoint{1.045651in}{1.434123in}}%
\pgfpathcurveto{\pgfqpoint{1.051475in}{1.439947in}}{\pgfqpoint{1.054748in}{1.447847in}}{\pgfqpoint{1.054748in}{1.456083in}}%
\pgfpathcurveto{\pgfqpoint{1.054748in}{1.464319in}}{\pgfqpoint{1.051475in}{1.472219in}}{\pgfqpoint{1.045651in}{1.478043in}}%
\pgfpathcurveto{\pgfqpoint{1.039827in}{1.483867in}}{\pgfqpoint{1.031927in}{1.487140in}}{\pgfqpoint{1.023691in}{1.487140in}}%
\pgfpathcurveto{\pgfqpoint{1.015455in}{1.487140in}}{\pgfqpoint{1.007555in}{1.483867in}}{\pgfqpoint{1.001731in}{1.478043in}}%
\pgfpathcurveto{\pgfqpoint{0.995907in}{1.472219in}}{\pgfqpoint{0.992635in}{1.464319in}}{\pgfqpoint{0.992635in}{1.456083in}}%
\pgfpathcurveto{\pgfqpoint{0.992635in}{1.447847in}}{\pgfqpoint{0.995907in}{1.439947in}}{\pgfqpoint{1.001731in}{1.434123in}}%
\pgfpathcurveto{\pgfqpoint{1.007555in}{1.428299in}}{\pgfqpoint{1.015455in}{1.425027in}}{\pgfqpoint{1.023691in}{1.425027in}}%
\pgfpathclose%
\pgfusepath{stroke,fill}%
\end{pgfscope}%
\begin{pgfscope}%
\pgfpathrectangle{\pgfqpoint{0.100000in}{0.212622in}}{\pgfqpoint{3.696000in}{3.696000in}}%
\pgfusepath{clip}%
\pgfsetbuttcap%
\pgfsetroundjoin%
\definecolor{currentfill}{rgb}{0.121569,0.466667,0.705882}%
\pgfsetfillcolor{currentfill}%
\pgfsetfillopacity{0.580425}%
\pgfsetlinewidth{1.003750pt}%
\definecolor{currentstroke}{rgb}{0.121569,0.466667,0.705882}%
\pgfsetstrokecolor{currentstroke}%
\pgfsetstrokeopacity{0.580425}%
\pgfsetdash{}{0pt}%
\pgfpathmoveto{\pgfqpoint{2.095112in}{1.846555in}}%
\pgfpathcurveto{\pgfqpoint{2.103349in}{1.846555in}}{\pgfqpoint{2.111249in}{1.849828in}}{\pgfqpoint{2.117073in}{1.855652in}}%
\pgfpathcurveto{\pgfqpoint{2.122897in}{1.861476in}}{\pgfqpoint{2.126169in}{1.869376in}}{\pgfqpoint{2.126169in}{1.877612in}}%
\pgfpathcurveto{\pgfqpoint{2.126169in}{1.885848in}}{\pgfqpoint{2.122897in}{1.893748in}}{\pgfqpoint{2.117073in}{1.899572in}}%
\pgfpathcurveto{\pgfqpoint{2.111249in}{1.905396in}}{\pgfqpoint{2.103349in}{1.908668in}}{\pgfqpoint{2.095112in}{1.908668in}}%
\pgfpathcurveto{\pgfqpoint{2.086876in}{1.908668in}}{\pgfqpoint{2.078976in}{1.905396in}}{\pgfqpoint{2.073152in}{1.899572in}}%
\pgfpathcurveto{\pgfqpoint{2.067328in}{1.893748in}}{\pgfqpoint{2.064056in}{1.885848in}}{\pgfqpoint{2.064056in}{1.877612in}}%
\pgfpathcurveto{\pgfqpoint{2.064056in}{1.869376in}}{\pgfqpoint{2.067328in}{1.861476in}}{\pgfqpoint{2.073152in}{1.855652in}}%
\pgfpathcurveto{\pgfqpoint{2.078976in}{1.849828in}}{\pgfqpoint{2.086876in}{1.846555in}}{\pgfqpoint{2.095112in}{1.846555in}}%
\pgfpathclose%
\pgfusepath{stroke,fill}%
\end{pgfscope}%
\begin{pgfscope}%
\pgfpathrectangle{\pgfqpoint{0.100000in}{0.212622in}}{\pgfqpoint{3.696000in}{3.696000in}}%
\pgfusepath{clip}%
\pgfsetbuttcap%
\pgfsetroundjoin%
\definecolor{currentfill}{rgb}{0.121569,0.466667,0.705882}%
\pgfsetfillcolor{currentfill}%
\pgfsetfillopacity{0.582137}%
\pgfsetlinewidth{1.003750pt}%
\definecolor{currentstroke}{rgb}{0.121569,0.466667,0.705882}%
\pgfsetstrokecolor{currentstroke}%
\pgfsetstrokeopacity{0.582137}%
\pgfsetdash{}{0pt}%
\pgfpathmoveto{\pgfqpoint{2.096322in}{1.844008in}}%
\pgfpathcurveto{\pgfqpoint{2.104558in}{1.844008in}}{\pgfqpoint{2.112458in}{1.847281in}}{\pgfqpoint{2.118282in}{1.853105in}}%
\pgfpathcurveto{\pgfqpoint{2.124106in}{1.858928in}}{\pgfqpoint{2.127378in}{1.866828in}}{\pgfqpoint{2.127378in}{1.875065in}}%
\pgfpathcurveto{\pgfqpoint{2.127378in}{1.883301in}}{\pgfqpoint{2.124106in}{1.891201in}}{\pgfqpoint{2.118282in}{1.897025in}}%
\pgfpathcurveto{\pgfqpoint{2.112458in}{1.902849in}}{\pgfqpoint{2.104558in}{1.906121in}}{\pgfqpoint{2.096322in}{1.906121in}}%
\pgfpathcurveto{\pgfqpoint{2.088086in}{1.906121in}}{\pgfqpoint{2.080186in}{1.902849in}}{\pgfqpoint{2.074362in}{1.897025in}}%
\pgfpathcurveto{\pgfqpoint{2.068538in}{1.891201in}}{\pgfqpoint{2.065265in}{1.883301in}}{\pgfqpoint{2.065265in}{1.875065in}}%
\pgfpathcurveto{\pgfqpoint{2.065265in}{1.866828in}}{\pgfqpoint{2.068538in}{1.858928in}}{\pgfqpoint{2.074362in}{1.853105in}}%
\pgfpathcurveto{\pgfqpoint{2.080186in}{1.847281in}}{\pgfqpoint{2.088086in}{1.844008in}}{\pgfqpoint{2.096322in}{1.844008in}}%
\pgfpathclose%
\pgfusepath{stroke,fill}%
\end{pgfscope}%
\begin{pgfscope}%
\pgfpathrectangle{\pgfqpoint{0.100000in}{0.212622in}}{\pgfqpoint{3.696000in}{3.696000in}}%
\pgfusepath{clip}%
\pgfsetbuttcap%
\pgfsetroundjoin%
\definecolor{currentfill}{rgb}{0.121569,0.466667,0.705882}%
\pgfsetfillcolor{currentfill}%
\pgfsetfillopacity{0.583125}%
\pgfsetlinewidth{1.003750pt}%
\definecolor{currentstroke}{rgb}{0.121569,0.466667,0.705882}%
\pgfsetstrokecolor{currentstroke}%
\pgfsetstrokeopacity{0.583125}%
\pgfsetdash{}{0pt}%
\pgfpathmoveto{\pgfqpoint{1.012738in}{1.411917in}}%
\pgfpathcurveto{\pgfqpoint{1.020974in}{1.411917in}}{\pgfqpoint{1.028874in}{1.415189in}}{\pgfqpoint{1.034698in}{1.421013in}}%
\pgfpathcurveto{\pgfqpoint{1.040522in}{1.426837in}}{\pgfqpoint{1.043795in}{1.434737in}}{\pgfqpoint{1.043795in}{1.442973in}}%
\pgfpathcurveto{\pgfqpoint{1.043795in}{1.451209in}}{\pgfqpoint{1.040522in}{1.459110in}}{\pgfqpoint{1.034698in}{1.464933in}}%
\pgfpathcurveto{\pgfqpoint{1.028874in}{1.470757in}}{\pgfqpoint{1.020974in}{1.474030in}}{\pgfqpoint{1.012738in}{1.474030in}}%
\pgfpathcurveto{\pgfqpoint{1.004502in}{1.474030in}}{\pgfqpoint{0.996602in}{1.470757in}}{\pgfqpoint{0.990778in}{1.464933in}}%
\pgfpathcurveto{\pgfqpoint{0.984954in}{1.459110in}}{\pgfqpoint{0.981682in}{1.451209in}}{\pgfqpoint{0.981682in}{1.442973in}}%
\pgfpathcurveto{\pgfqpoint{0.981682in}{1.434737in}}{\pgfqpoint{0.984954in}{1.426837in}}{\pgfqpoint{0.990778in}{1.421013in}}%
\pgfpathcurveto{\pgfqpoint{0.996602in}{1.415189in}}{\pgfqpoint{1.004502in}{1.411917in}}{\pgfqpoint{1.012738in}{1.411917in}}%
\pgfpathclose%
\pgfusepath{stroke,fill}%
\end{pgfscope}%
\begin{pgfscope}%
\pgfpathrectangle{\pgfqpoint{0.100000in}{0.212622in}}{\pgfqpoint{3.696000in}{3.696000in}}%
\pgfusepath{clip}%
\pgfsetbuttcap%
\pgfsetroundjoin%
\definecolor{currentfill}{rgb}{0.121569,0.466667,0.705882}%
\pgfsetfillcolor{currentfill}%
\pgfsetfillopacity{0.584635}%
\pgfsetlinewidth{1.003750pt}%
\definecolor{currentstroke}{rgb}{0.121569,0.466667,0.705882}%
\pgfsetstrokecolor{currentstroke}%
\pgfsetstrokeopacity{0.584635}%
\pgfsetdash{}{0pt}%
\pgfpathmoveto{\pgfqpoint{2.098161in}{1.841124in}}%
\pgfpathcurveto{\pgfqpoint{2.106397in}{1.841124in}}{\pgfqpoint{2.114297in}{1.844396in}}{\pgfqpoint{2.120121in}{1.850220in}}%
\pgfpathcurveto{\pgfqpoint{2.125945in}{1.856044in}}{\pgfqpoint{2.129217in}{1.863944in}}{\pgfqpoint{2.129217in}{1.872180in}}%
\pgfpathcurveto{\pgfqpoint{2.129217in}{1.880417in}}{\pgfqpoint{2.125945in}{1.888317in}}{\pgfqpoint{2.120121in}{1.894140in}}%
\pgfpathcurveto{\pgfqpoint{2.114297in}{1.899964in}}{\pgfqpoint{2.106397in}{1.903237in}}{\pgfqpoint{2.098161in}{1.903237in}}%
\pgfpathcurveto{\pgfqpoint{2.089924in}{1.903237in}}{\pgfqpoint{2.082024in}{1.899964in}}{\pgfqpoint{2.076200in}{1.894140in}}%
\pgfpathcurveto{\pgfqpoint{2.070376in}{1.888317in}}{\pgfqpoint{2.067104in}{1.880417in}}{\pgfqpoint{2.067104in}{1.872180in}}%
\pgfpathcurveto{\pgfqpoint{2.067104in}{1.863944in}}{\pgfqpoint{2.070376in}{1.856044in}}{\pgfqpoint{2.076200in}{1.850220in}}%
\pgfpathcurveto{\pgfqpoint{2.082024in}{1.844396in}}{\pgfqpoint{2.089924in}{1.841124in}}{\pgfqpoint{2.098161in}{1.841124in}}%
\pgfpathclose%
\pgfusepath{stroke,fill}%
\end{pgfscope}%
\begin{pgfscope}%
\pgfpathrectangle{\pgfqpoint{0.100000in}{0.212622in}}{\pgfqpoint{3.696000in}{3.696000in}}%
\pgfusepath{clip}%
\pgfsetbuttcap%
\pgfsetroundjoin%
\definecolor{currentfill}{rgb}{0.121569,0.466667,0.705882}%
\pgfsetfillcolor{currentfill}%
\pgfsetfillopacity{0.587092}%
\pgfsetlinewidth{1.003750pt}%
\definecolor{currentstroke}{rgb}{0.121569,0.466667,0.705882}%
\pgfsetstrokecolor{currentstroke}%
\pgfsetstrokeopacity{0.587092}%
\pgfsetdash{}{0pt}%
\pgfpathmoveto{\pgfqpoint{1.000284in}{1.405634in}}%
\pgfpathcurveto{\pgfqpoint{1.008520in}{1.405634in}}{\pgfqpoint{1.016420in}{1.408906in}}{\pgfqpoint{1.022244in}{1.414730in}}%
\pgfpathcurveto{\pgfqpoint{1.028068in}{1.420554in}}{\pgfqpoint{1.031340in}{1.428454in}}{\pgfqpoint{1.031340in}{1.436690in}}%
\pgfpathcurveto{\pgfqpoint{1.031340in}{1.444926in}}{\pgfqpoint{1.028068in}{1.452827in}}{\pgfqpoint{1.022244in}{1.458650in}}%
\pgfpathcurveto{\pgfqpoint{1.016420in}{1.464474in}}{\pgfqpoint{1.008520in}{1.467747in}}{\pgfqpoint{1.000284in}{1.467747in}}%
\pgfpathcurveto{\pgfqpoint{0.992048in}{1.467747in}}{\pgfqpoint{0.984147in}{1.464474in}}{\pgfqpoint{0.978324in}{1.458650in}}%
\pgfpathcurveto{\pgfqpoint{0.972500in}{1.452827in}}{\pgfqpoint{0.969227in}{1.444926in}}{\pgfqpoint{0.969227in}{1.436690in}}%
\pgfpathcurveto{\pgfqpoint{0.969227in}{1.428454in}}{\pgfqpoint{0.972500in}{1.420554in}}{\pgfqpoint{0.978324in}{1.414730in}}%
\pgfpathcurveto{\pgfqpoint{0.984147in}{1.408906in}}{\pgfqpoint{0.992048in}{1.405634in}}{\pgfqpoint{1.000284in}{1.405634in}}%
\pgfpathclose%
\pgfusepath{stroke,fill}%
\end{pgfscope}%
\begin{pgfscope}%
\pgfpathrectangle{\pgfqpoint{0.100000in}{0.212622in}}{\pgfqpoint{3.696000in}{3.696000in}}%
\pgfusepath{clip}%
\pgfsetbuttcap%
\pgfsetroundjoin%
\definecolor{currentfill}{rgb}{0.121569,0.466667,0.705882}%
\pgfsetfillcolor{currentfill}%
\pgfsetfillopacity{0.587635}%
\pgfsetlinewidth{1.003750pt}%
\definecolor{currentstroke}{rgb}{0.121569,0.466667,0.705882}%
\pgfsetstrokecolor{currentstroke}%
\pgfsetstrokeopacity{0.587635}%
\pgfsetdash{}{0pt}%
\pgfpathmoveto{\pgfqpoint{2.100161in}{1.838306in}}%
\pgfpathcurveto{\pgfqpoint{2.108397in}{1.838306in}}{\pgfqpoint{2.116297in}{1.841578in}}{\pgfqpoint{2.122121in}{1.847402in}}%
\pgfpathcurveto{\pgfqpoint{2.127945in}{1.853226in}}{\pgfqpoint{2.131218in}{1.861126in}}{\pgfqpoint{2.131218in}{1.869362in}}%
\pgfpathcurveto{\pgfqpoint{2.131218in}{1.877599in}}{\pgfqpoint{2.127945in}{1.885499in}}{\pgfqpoint{2.122121in}{1.891323in}}%
\pgfpathcurveto{\pgfqpoint{2.116297in}{1.897147in}}{\pgfqpoint{2.108397in}{1.900419in}}{\pgfqpoint{2.100161in}{1.900419in}}%
\pgfpathcurveto{\pgfqpoint{2.091925in}{1.900419in}}{\pgfqpoint{2.084025in}{1.897147in}}{\pgfqpoint{2.078201in}{1.891323in}}%
\pgfpathcurveto{\pgfqpoint{2.072377in}{1.885499in}}{\pgfqpoint{2.069105in}{1.877599in}}{\pgfqpoint{2.069105in}{1.869362in}}%
\pgfpathcurveto{\pgfqpoint{2.069105in}{1.861126in}}{\pgfqpoint{2.072377in}{1.853226in}}{\pgfqpoint{2.078201in}{1.847402in}}%
\pgfpathcurveto{\pgfqpoint{2.084025in}{1.841578in}}{\pgfqpoint{2.091925in}{1.838306in}}{\pgfqpoint{2.100161in}{1.838306in}}%
\pgfpathclose%
\pgfusepath{stroke,fill}%
\end{pgfscope}%
\begin{pgfscope}%
\pgfpathrectangle{\pgfqpoint{0.100000in}{0.212622in}}{\pgfqpoint{3.696000in}{3.696000in}}%
\pgfusepath{clip}%
\pgfsetbuttcap%
\pgfsetroundjoin%
\definecolor{currentfill}{rgb}{0.121569,0.466667,0.705882}%
\pgfsetfillcolor{currentfill}%
\pgfsetfillopacity{0.589657}%
\pgfsetlinewidth{1.003750pt}%
\definecolor{currentstroke}{rgb}{0.121569,0.466667,0.705882}%
\pgfsetstrokecolor{currentstroke}%
\pgfsetstrokeopacity{0.589657}%
\pgfsetdash{}{0pt}%
\pgfpathmoveto{\pgfqpoint{0.992393in}{1.393462in}}%
\pgfpathcurveto{\pgfqpoint{1.000629in}{1.393462in}}{\pgfqpoint{1.008529in}{1.396734in}}{\pgfqpoint{1.014353in}{1.402558in}}%
\pgfpathcurveto{\pgfqpoint{1.020177in}{1.408382in}}{\pgfqpoint{1.023450in}{1.416282in}}{\pgfqpoint{1.023450in}{1.424519in}}%
\pgfpathcurveto{\pgfqpoint{1.023450in}{1.432755in}}{\pgfqpoint{1.020177in}{1.440655in}}{\pgfqpoint{1.014353in}{1.446479in}}%
\pgfpathcurveto{\pgfqpoint{1.008529in}{1.452303in}}{\pgfqpoint{1.000629in}{1.455575in}}{\pgfqpoint{0.992393in}{1.455575in}}%
\pgfpathcurveto{\pgfqpoint{0.984157in}{1.455575in}}{\pgfqpoint{0.976257in}{1.452303in}}{\pgfqpoint{0.970433in}{1.446479in}}%
\pgfpathcurveto{\pgfqpoint{0.964609in}{1.440655in}}{\pgfqpoint{0.961337in}{1.432755in}}{\pgfqpoint{0.961337in}{1.424519in}}%
\pgfpathcurveto{\pgfqpoint{0.961337in}{1.416282in}}{\pgfqpoint{0.964609in}{1.408382in}}{\pgfqpoint{0.970433in}{1.402558in}}%
\pgfpathcurveto{\pgfqpoint{0.976257in}{1.396734in}}{\pgfqpoint{0.984157in}{1.393462in}}{\pgfqpoint{0.992393in}{1.393462in}}%
\pgfpathclose%
\pgfusepath{stroke,fill}%
\end{pgfscope}%
\begin{pgfscope}%
\pgfpathrectangle{\pgfqpoint{0.100000in}{0.212622in}}{\pgfqpoint{3.696000in}{3.696000in}}%
\pgfusepath{clip}%
\pgfsetbuttcap%
\pgfsetroundjoin%
\definecolor{currentfill}{rgb}{0.121569,0.466667,0.705882}%
\pgfsetfillcolor{currentfill}%
\pgfsetfillopacity{0.591369}%
\pgfsetlinewidth{1.003750pt}%
\definecolor{currentstroke}{rgb}{0.121569,0.466667,0.705882}%
\pgfsetstrokecolor{currentstroke}%
\pgfsetstrokeopacity{0.591369}%
\pgfsetdash{}{0pt}%
\pgfpathmoveto{\pgfqpoint{2.101837in}{1.836170in}}%
\pgfpathcurveto{\pgfqpoint{2.110074in}{1.836170in}}{\pgfqpoint{2.117974in}{1.839442in}}{\pgfqpoint{2.123798in}{1.845266in}}%
\pgfpathcurveto{\pgfqpoint{2.129622in}{1.851090in}}{\pgfqpoint{2.132894in}{1.858990in}}{\pgfqpoint{2.132894in}{1.867226in}}%
\pgfpathcurveto{\pgfqpoint{2.132894in}{1.875463in}}{\pgfqpoint{2.129622in}{1.883363in}}{\pgfqpoint{2.123798in}{1.889187in}}%
\pgfpathcurveto{\pgfqpoint{2.117974in}{1.895011in}}{\pgfqpoint{2.110074in}{1.898283in}}{\pgfqpoint{2.101837in}{1.898283in}}%
\pgfpathcurveto{\pgfqpoint{2.093601in}{1.898283in}}{\pgfqpoint{2.085701in}{1.895011in}}{\pgfqpoint{2.079877in}{1.889187in}}%
\pgfpathcurveto{\pgfqpoint{2.074053in}{1.883363in}}{\pgfqpoint{2.070781in}{1.875463in}}{\pgfqpoint{2.070781in}{1.867226in}}%
\pgfpathcurveto{\pgfqpoint{2.070781in}{1.858990in}}{\pgfqpoint{2.074053in}{1.851090in}}{\pgfqpoint{2.079877in}{1.845266in}}%
\pgfpathcurveto{\pgfqpoint{2.085701in}{1.839442in}}{\pgfqpoint{2.093601in}{1.836170in}}{\pgfqpoint{2.101837in}{1.836170in}}%
\pgfpathclose%
\pgfusepath{stroke,fill}%
\end{pgfscope}%
\begin{pgfscope}%
\pgfpathrectangle{\pgfqpoint{0.100000in}{0.212622in}}{\pgfqpoint{3.696000in}{3.696000in}}%
\pgfusepath{clip}%
\pgfsetbuttcap%
\pgfsetroundjoin%
\definecolor{currentfill}{rgb}{0.121569,0.466667,0.705882}%
\pgfsetfillcolor{currentfill}%
\pgfsetfillopacity{0.592576}%
\pgfsetlinewidth{1.003750pt}%
\definecolor{currentstroke}{rgb}{0.121569,0.466667,0.705882}%
\pgfsetstrokecolor{currentstroke}%
\pgfsetstrokeopacity{0.592576}%
\pgfsetdash{}{0pt}%
\pgfpathmoveto{\pgfqpoint{0.983883in}{1.385780in}}%
\pgfpathcurveto{\pgfqpoint{0.992120in}{1.385780in}}{\pgfqpoint{1.000020in}{1.389052in}}{\pgfqpoint{1.005844in}{1.394876in}}%
\pgfpathcurveto{\pgfqpoint{1.011667in}{1.400700in}}{\pgfqpoint{1.014940in}{1.408600in}}{\pgfqpoint{1.014940in}{1.416836in}}%
\pgfpathcurveto{\pgfqpoint{1.014940in}{1.425073in}}{\pgfqpoint{1.011667in}{1.432973in}}{\pgfqpoint{1.005844in}{1.438797in}}%
\pgfpathcurveto{\pgfqpoint{1.000020in}{1.444621in}}{\pgfqpoint{0.992120in}{1.447893in}}{\pgfqpoint{0.983883in}{1.447893in}}%
\pgfpathcurveto{\pgfqpoint{0.975647in}{1.447893in}}{\pgfqpoint{0.967747in}{1.444621in}}{\pgfqpoint{0.961923in}{1.438797in}}%
\pgfpathcurveto{\pgfqpoint{0.956099in}{1.432973in}}{\pgfqpoint{0.952827in}{1.425073in}}{\pgfqpoint{0.952827in}{1.416836in}}%
\pgfpathcurveto{\pgfqpoint{0.952827in}{1.408600in}}{\pgfqpoint{0.956099in}{1.400700in}}{\pgfqpoint{0.961923in}{1.394876in}}%
\pgfpathcurveto{\pgfqpoint{0.967747in}{1.389052in}}{\pgfqpoint{0.975647in}{1.385780in}}{\pgfqpoint{0.983883in}{1.385780in}}%
\pgfpathclose%
\pgfusepath{stroke,fill}%
\end{pgfscope}%
\begin{pgfscope}%
\pgfpathrectangle{\pgfqpoint{0.100000in}{0.212622in}}{\pgfqpoint{3.696000in}{3.696000in}}%
\pgfusepath{clip}%
\pgfsetbuttcap%
\pgfsetroundjoin%
\definecolor{currentfill}{rgb}{0.121569,0.466667,0.705882}%
\pgfsetfillcolor{currentfill}%
\pgfsetfillopacity{0.594995}%
\pgfsetlinewidth{1.003750pt}%
\definecolor{currentstroke}{rgb}{0.121569,0.466667,0.705882}%
\pgfsetstrokecolor{currentstroke}%
\pgfsetstrokeopacity{0.594995}%
\pgfsetdash{}{0pt}%
\pgfpathmoveto{\pgfqpoint{2.104544in}{1.832165in}}%
\pgfpathcurveto{\pgfqpoint{2.112780in}{1.832165in}}{\pgfqpoint{2.120680in}{1.835438in}}{\pgfqpoint{2.126504in}{1.841261in}}%
\pgfpathcurveto{\pgfqpoint{2.132328in}{1.847085in}}{\pgfqpoint{2.135601in}{1.854985in}}{\pgfqpoint{2.135601in}{1.863222in}}%
\pgfpathcurveto{\pgfqpoint{2.135601in}{1.871458in}}{\pgfqpoint{2.132328in}{1.879358in}}{\pgfqpoint{2.126504in}{1.885182in}}%
\pgfpathcurveto{\pgfqpoint{2.120680in}{1.891006in}}{\pgfqpoint{2.112780in}{1.894278in}}{\pgfqpoint{2.104544in}{1.894278in}}%
\pgfpathcurveto{\pgfqpoint{2.096308in}{1.894278in}}{\pgfqpoint{2.088408in}{1.891006in}}{\pgfqpoint{2.082584in}{1.885182in}}%
\pgfpathcurveto{\pgfqpoint{2.076760in}{1.879358in}}{\pgfqpoint{2.073488in}{1.871458in}}{\pgfqpoint{2.073488in}{1.863222in}}%
\pgfpathcurveto{\pgfqpoint{2.073488in}{1.854985in}}{\pgfqpoint{2.076760in}{1.847085in}}{\pgfqpoint{2.082584in}{1.841261in}}%
\pgfpathcurveto{\pgfqpoint{2.088408in}{1.835438in}}{\pgfqpoint{2.096308in}{1.832165in}}{\pgfqpoint{2.104544in}{1.832165in}}%
\pgfpathclose%
\pgfusepath{stroke,fill}%
\end{pgfscope}%
\begin{pgfscope}%
\pgfpathrectangle{\pgfqpoint{0.100000in}{0.212622in}}{\pgfqpoint{3.696000in}{3.696000in}}%
\pgfusepath{clip}%
\pgfsetbuttcap%
\pgfsetroundjoin%
\definecolor{currentfill}{rgb}{0.121569,0.466667,0.705882}%
\pgfsetfillcolor{currentfill}%
\pgfsetfillopacity{0.595215}%
\pgfsetlinewidth{1.003750pt}%
\definecolor{currentstroke}{rgb}{0.121569,0.466667,0.705882}%
\pgfsetstrokecolor{currentstroke}%
\pgfsetstrokeopacity{0.595215}%
\pgfsetdash{}{0pt}%
\pgfpathmoveto{\pgfqpoint{0.976199in}{1.377739in}}%
\pgfpathcurveto{\pgfqpoint{0.984435in}{1.377739in}}{\pgfqpoint{0.992335in}{1.381012in}}{\pgfqpoint{0.998159in}{1.386836in}}%
\pgfpathcurveto{\pgfqpoint{1.003983in}{1.392659in}}{\pgfqpoint{1.007255in}{1.400560in}}{\pgfqpoint{1.007255in}{1.408796in}}%
\pgfpathcurveto{\pgfqpoint{1.007255in}{1.417032in}}{\pgfqpoint{1.003983in}{1.424932in}}{\pgfqpoint{0.998159in}{1.430756in}}%
\pgfpathcurveto{\pgfqpoint{0.992335in}{1.436580in}}{\pgfqpoint{0.984435in}{1.439852in}}{\pgfqpoint{0.976199in}{1.439852in}}%
\pgfpathcurveto{\pgfqpoint{0.967962in}{1.439852in}}{\pgfqpoint{0.960062in}{1.436580in}}{\pgfqpoint{0.954238in}{1.430756in}}%
\pgfpathcurveto{\pgfqpoint{0.948414in}{1.424932in}}{\pgfqpoint{0.945142in}{1.417032in}}{\pgfqpoint{0.945142in}{1.408796in}}%
\pgfpathcurveto{\pgfqpoint{0.945142in}{1.400560in}}{\pgfqpoint{0.948414in}{1.392659in}}{\pgfqpoint{0.954238in}{1.386836in}}%
\pgfpathcurveto{\pgfqpoint{0.960062in}{1.381012in}}{\pgfqpoint{0.967962in}{1.377739in}}{\pgfqpoint{0.976199in}{1.377739in}}%
\pgfpathclose%
\pgfusepath{stroke,fill}%
\end{pgfscope}%
\begin{pgfscope}%
\pgfpathrectangle{\pgfqpoint{0.100000in}{0.212622in}}{\pgfqpoint{3.696000in}{3.696000in}}%
\pgfusepath{clip}%
\pgfsetbuttcap%
\pgfsetroundjoin%
\definecolor{currentfill}{rgb}{0.121569,0.466667,0.705882}%
\pgfsetfillcolor{currentfill}%
\pgfsetfillopacity{0.597244}%
\pgfsetlinewidth{1.003750pt}%
\definecolor{currentstroke}{rgb}{0.121569,0.466667,0.705882}%
\pgfsetstrokecolor{currentstroke}%
\pgfsetstrokeopacity{0.597244}%
\pgfsetdash{}{0pt}%
\pgfpathmoveto{\pgfqpoint{0.970947in}{1.367272in}}%
\pgfpathcurveto{\pgfqpoint{0.979184in}{1.367272in}}{\pgfqpoint{0.987084in}{1.370545in}}{\pgfqpoint{0.992908in}{1.376369in}}%
\pgfpathcurveto{\pgfqpoint{0.998732in}{1.382193in}}{\pgfqpoint{1.002004in}{1.390093in}}{\pgfqpoint{1.002004in}{1.398329in}}%
\pgfpathcurveto{\pgfqpoint{1.002004in}{1.406565in}}{\pgfqpoint{0.998732in}{1.414465in}}{\pgfqpoint{0.992908in}{1.420289in}}%
\pgfpathcurveto{\pgfqpoint{0.987084in}{1.426113in}}{\pgfqpoint{0.979184in}{1.429385in}}{\pgfqpoint{0.970947in}{1.429385in}}%
\pgfpathcurveto{\pgfqpoint{0.962711in}{1.429385in}}{\pgfqpoint{0.954811in}{1.426113in}}{\pgfqpoint{0.948987in}{1.420289in}}%
\pgfpathcurveto{\pgfqpoint{0.943163in}{1.414465in}}{\pgfqpoint{0.939891in}{1.406565in}}{\pgfqpoint{0.939891in}{1.398329in}}%
\pgfpathcurveto{\pgfqpoint{0.939891in}{1.390093in}}{\pgfqpoint{0.943163in}{1.382193in}}{\pgfqpoint{0.948987in}{1.376369in}}%
\pgfpathcurveto{\pgfqpoint{0.954811in}{1.370545in}}{\pgfqpoint{0.962711in}{1.367272in}}{\pgfqpoint{0.970947in}{1.367272in}}%
\pgfpathclose%
\pgfusepath{stroke,fill}%
\end{pgfscope}%
\begin{pgfscope}%
\pgfpathrectangle{\pgfqpoint{0.100000in}{0.212622in}}{\pgfqpoint{3.696000in}{3.696000in}}%
\pgfusepath{clip}%
\pgfsetbuttcap%
\pgfsetroundjoin%
\definecolor{currentfill}{rgb}{0.121569,0.466667,0.705882}%
\pgfsetfillcolor{currentfill}%
\pgfsetfillopacity{0.598974}%
\pgfsetlinewidth{1.003750pt}%
\definecolor{currentstroke}{rgb}{0.121569,0.466667,0.705882}%
\pgfsetstrokecolor{currentstroke}%
\pgfsetstrokeopacity{0.598974}%
\pgfsetdash{}{0pt}%
\pgfpathmoveto{\pgfqpoint{0.965523in}{1.363111in}}%
\pgfpathcurveto{\pgfqpoint{0.973759in}{1.363111in}}{\pgfqpoint{0.981659in}{1.366384in}}{\pgfqpoint{0.987483in}{1.372208in}}%
\pgfpathcurveto{\pgfqpoint{0.993307in}{1.378032in}}{\pgfqpoint{0.996580in}{1.385932in}}{\pgfqpoint{0.996580in}{1.394168in}}%
\pgfpathcurveto{\pgfqpoint{0.996580in}{1.402404in}}{\pgfqpoint{0.993307in}{1.410304in}}{\pgfqpoint{0.987483in}{1.416128in}}%
\pgfpathcurveto{\pgfqpoint{0.981659in}{1.421952in}}{\pgfqpoint{0.973759in}{1.425224in}}{\pgfqpoint{0.965523in}{1.425224in}}%
\pgfpathcurveto{\pgfqpoint{0.957287in}{1.425224in}}{\pgfqpoint{0.949387in}{1.421952in}}{\pgfqpoint{0.943563in}{1.416128in}}%
\pgfpathcurveto{\pgfqpoint{0.937739in}{1.410304in}}{\pgfqpoint{0.934467in}{1.402404in}}{\pgfqpoint{0.934467in}{1.394168in}}%
\pgfpathcurveto{\pgfqpoint{0.934467in}{1.385932in}}{\pgfqpoint{0.937739in}{1.378032in}}{\pgfqpoint{0.943563in}{1.372208in}}%
\pgfpathcurveto{\pgfqpoint{0.949387in}{1.366384in}}{\pgfqpoint{0.957287in}{1.363111in}}{\pgfqpoint{0.965523in}{1.363111in}}%
\pgfpathclose%
\pgfusepath{stroke,fill}%
\end{pgfscope}%
\begin{pgfscope}%
\pgfpathrectangle{\pgfqpoint{0.100000in}{0.212622in}}{\pgfqpoint{3.696000in}{3.696000in}}%
\pgfusepath{clip}%
\pgfsetbuttcap%
\pgfsetroundjoin%
\definecolor{currentfill}{rgb}{0.121569,0.466667,0.705882}%
\pgfsetfillcolor{currentfill}%
\pgfsetfillopacity{0.599006}%
\pgfsetlinewidth{1.003750pt}%
\definecolor{currentstroke}{rgb}{0.121569,0.466667,0.705882}%
\pgfsetstrokecolor{currentstroke}%
\pgfsetstrokeopacity{0.599006}%
\pgfsetdash{}{0pt}%
\pgfpathmoveto{\pgfqpoint{2.107045in}{1.827132in}}%
\pgfpathcurveto{\pgfqpoint{2.115281in}{1.827132in}}{\pgfqpoint{2.123181in}{1.830404in}}{\pgfqpoint{2.129005in}{1.836228in}}%
\pgfpathcurveto{\pgfqpoint{2.134829in}{1.842052in}}{\pgfqpoint{2.138102in}{1.849952in}}{\pgfqpoint{2.138102in}{1.858188in}}%
\pgfpathcurveto{\pgfqpoint{2.138102in}{1.866425in}}{\pgfqpoint{2.134829in}{1.874325in}}{\pgfqpoint{2.129005in}{1.880149in}}%
\pgfpathcurveto{\pgfqpoint{2.123181in}{1.885973in}}{\pgfqpoint{2.115281in}{1.889245in}}{\pgfqpoint{2.107045in}{1.889245in}}%
\pgfpathcurveto{\pgfqpoint{2.098809in}{1.889245in}}{\pgfqpoint{2.090909in}{1.885973in}}{\pgfqpoint{2.085085in}{1.880149in}}%
\pgfpathcurveto{\pgfqpoint{2.079261in}{1.874325in}}{\pgfqpoint{2.075989in}{1.866425in}}{\pgfqpoint{2.075989in}{1.858188in}}%
\pgfpathcurveto{\pgfqpoint{2.075989in}{1.849952in}}{\pgfqpoint{2.079261in}{1.842052in}}{\pgfqpoint{2.085085in}{1.836228in}}%
\pgfpathcurveto{\pgfqpoint{2.090909in}{1.830404in}}{\pgfqpoint{2.098809in}{1.827132in}}{\pgfqpoint{2.107045in}{1.827132in}}%
\pgfpathclose%
\pgfusepath{stroke,fill}%
\end{pgfscope}%
\begin{pgfscope}%
\pgfpathrectangle{\pgfqpoint{0.100000in}{0.212622in}}{\pgfqpoint{3.696000in}{3.696000in}}%
\pgfusepath{clip}%
\pgfsetbuttcap%
\pgfsetroundjoin%
\definecolor{currentfill}{rgb}{0.121569,0.466667,0.705882}%
\pgfsetfillcolor{currentfill}%
\pgfsetfillopacity{0.600004}%
\pgfsetlinewidth{1.003750pt}%
\definecolor{currentstroke}{rgb}{0.121569,0.466667,0.705882}%
\pgfsetstrokecolor{currentstroke}%
\pgfsetstrokeopacity{0.600004}%
\pgfsetdash{}{0pt}%
\pgfpathmoveto{\pgfqpoint{0.963138in}{1.357581in}}%
\pgfpathcurveto{\pgfqpoint{0.971375in}{1.357581in}}{\pgfqpoint{0.979275in}{1.360853in}}{\pgfqpoint{0.985099in}{1.366677in}}%
\pgfpathcurveto{\pgfqpoint{0.990923in}{1.372501in}}{\pgfqpoint{0.994195in}{1.380401in}}{\pgfqpoint{0.994195in}{1.388637in}}%
\pgfpathcurveto{\pgfqpoint{0.994195in}{1.396874in}}{\pgfqpoint{0.990923in}{1.404774in}}{\pgfqpoint{0.985099in}{1.410598in}}%
\pgfpathcurveto{\pgfqpoint{0.979275in}{1.416422in}}{\pgfqpoint{0.971375in}{1.419694in}}{\pgfqpoint{0.963138in}{1.419694in}}%
\pgfpathcurveto{\pgfqpoint{0.954902in}{1.419694in}}{\pgfqpoint{0.947002in}{1.416422in}}{\pgfqpoint{0.941178in}{1.410598in}}%
\pgfpathcurveto{\pgfqpoint{0.935354in}{1.404774in}}{\pgfqpoint{0.932082in}{1.396874in}}{\pgfqpoint{0.932082in}{1.388637in}}%
\pgfpathcurveto{\pgfqpoint{0.932082in}{1.380401in}}{\pgfqpoint{0.935354in}{1.372501in}}{\pgfqpoint{0.941178in}{1.366677in}}%
\pgfpathcurveto{\pgfqpoint{0.947002in}{1.360853in}}{\pgfqpoint{0.954902in}{1.357581in}}{\pgfqpoint{0.963138in}{1.357581in}}%
\pgfpathclose%
\pgfusepath{stroke,fill}%
\end{pgfscope}%
\begin{pgfscope}%
\pgfpathrectangle{\pgfqpoint{0.100000in}{0.212622in}}{\pgfqpoint{3.696000in}{3.696000in}}%
\pgfusepath{clip}%
\pgfsetbuttcap%
\pgfsetroundjoin%
\definecolor{currentfill}{rgb}{0.121569,0.466667,0.705882}%
\pgfsetfillcolor{currentfill}%
\pgfsetfillopacity{0.600994}%
\pgfsetlinewidth{1.003750pt}%
\definecolor{currentstroke}{rgb}{0.121569,0.466667,0.705882}%
\pgfsetstrokecolor{currentstroke}%
\pgfsetstrokeopacity{0.600994}%
\pgfsetdash{}{0pt}%
\pgfpathmoveto{\pgfqpoint{0.960177in}{1.356085in}}%
\pgfpathcurveto{\pgfqpoint{0.968414in}{1.356085in}}{\pgfqpoint{0.976314in}{1.359357in}}{\pgfqpoint{0.982138in}{1.365181in}}%
\pgfpathcurveto{\pgfqpoint{0.987962in}{1.371005in}}{\pgfqpoint{0.991234in}{1.378905in}}{\pgfqpoint{0.991234in}{1.387141in}}%
\pgfpathcurveto{\pgfqpoint{0.991234in}{1.395377in}}{\pgfqpoint{0.987962in}{1.403277in}}{\pgfqpoint{0.982138in}{1.409101in}}%
\pgfpathcurveto{\pgfqpoint{0.976314in}{1.414925in}}{\pgfqpoint{0.968414in}{1.418198in}}{\pgfqpoint{0.960177in}{1.418198in}}%
\pgfpathcurveto{\pgfqpoint{0.951941in}{1.418198in}}{\pgfqpoint{0.944041in}{1.414925in}}{\pgfqpoint{0.938217in}{1.409101in}}%
\pgfpathcurveto{\pgfqpoint{0.932393in}{1.403277in}}{\pgfqpoint{0.929121in}{1.395377in}}{\pgfqpoint{0.929121in}{1.387141in}}%
\pgfpathcurveto{\pgfqpoint{0.929121in}{1.378905in}}{\pgfqpoint{0.932393in}{1.371005in}}{\pgfqpoint{0.938217in}{1.365181in}}%
\pgfpathcurveto{\pgfqpoint{0.944041in}{1.359357in}}{\pgfqpoint{0.951941in}{1.356085in}}{\pgfqpoint{0.960177in}{1.356085in}}%
\pgfpathclose%
\pgfusepath{stroke,fill}%
\end{pgfscope}%
\begin{pgfscope}%
\pgfpathrectangle{\pgfqpoint{0.100000in}{0.212622in}}{\pgfqpoint{3.696000in}{3.696000in}}%
\pgfusepath{clip}%
\pgfsetbuttcap%
\pgfsetroundjoin%
\definecolor{currentfill}{rgb}{0.121569,0.466667,0.705882}%
\pgfsetfillcolor{currentfill}%
\pgfsetfillopacity{0.602536}%
\pgfsetlinewidth{1.003750pt}%
\definecolor{currentstroke}{rgb}{0.121569,0.466667,0.705882}%
\pgfsetstrokecolor{currentstroke}%
\pgfsetstrokeopacity{0.602536}%
\pgfsetdash{}{0pt}%
\pgfpathmoveto{\pgfqpoint{0.955442in}{1.350900in}}%
\pgfpathcurveto{\pgfqpoint{0.963679in}{1.350900in}}{\pgfqpoint{0.971579in}{1.354173in}}{\pgfqpoint{0.977403in}{1.359997in}}%
\pgfpathcurveto{\pgfqpoint{0.983227in}{1.365821in}}{\pgfqpoint{0.986499in}{1.373721in}}{\pgfqpoint{0.986499in}{1.381957in}}%
\pgfpathcurveto{\pgfqpoint{0.986499in}{1.390193in}}{\pgfqpoint{0.983227in}{1.398093in}}{\pgfqpoint{0.977403in}{1.403917in}}%
\pgfpathcurveto{\pgfqpoint{0.971579in}{1.409741in}}{\pgfqpoint{0.963679in}{1.413013in}}{\pgfqpoint{0.955442in}{1.413013in}}%
\pgfpathcurveto{\pgfqpoint{0.947206in}{1.413013in}}{\pgfqpoint{0.939306in}{1.409741in}}{\pgfqpoint{0.933482in}{1.403917in}}%
\pgfpathcurveto{\pgfqpoint{0.927658in}{1.398093in}}{\pgfqpoint{0.924386in}{1.390193in}}{\pgfqpoint{0.924386in}{1.381957in}}%
\pgfpathcurveto{\pgfqpoint{0.924386in}{1.373721in}}{\pgfqpoint{0.927658in}{1.365821in}}{\pgfqpoint{0.933482in}{1.359997in}}%
\pgfpathcurveto{\pgfqpoint{0.939306in}{1.354173in}}{\pgfqpoint{0.947206in}{1.350900in}}{\pgfqpoint{0.955442in}{1.350900in}}%
\pgfpathclose%
\pgfusepath{stroke,fill}%
\end{pgfscope}%
\begin{pgfscope}%
\pgfpathrectangle{\pgfqpoint{0.100000in}{0.212622in}}{\pgfqpoint{3.696000in}{3.696000in}}%
\pgfusepath{clip}%
\pgfsetbuttcap%
\pgfsetroundjoin%
\definecolor{currentfill}{rgb}{0.121569,0.466667,0.705882}%
\pgfsetfillcolor{currentfill}%
\pgfsetfillopacity{0.603353}%
\pgfsetlinewidth{1.003750pt}%
\definecolor{currentstroke}{rgb}{0.121569,0.466667,0.705882}%
\pgfsetstrokecolor{currentstroke}%
\pgfsetstrokeopacity{0.603353}%
\pgfsetdash{}{0pt}%
\pgfpathmoveto{\pgfqpoint{2.109802in}{1.820543in}}%
\pgfpathcurveto{\pgfqpoint{2.118038in}{1.820543in}}{\pgfqpoint{2.125939in}{1.823816in}}{\pgfqpoint{2.131762in}{1.829640in}}%
\pgfpathcurveto{\pgfqpoint{2.137586in}{1.835463in}}{\pgfqpoint{2.140859in}{1.843364in}}{\pgfqpoint{2.140859in}{1.851600in}}%
\pgfpathcurveto{\pgfqpoint{2.140859in}{1.859836in}}{\pgfqpoint{2.137586in}{1.867736in}}{\pgfqpoint{2.131762in}{1.873560in}}%
\pgfpathcurveto{\pgfqpoint{2.125939in}{1.879384in}}{\pgfqpoint{2.118038in}{1.882656in}}{\pgfqpoint{2.109802in}{1.882656in}}%
\pgfpathcurveto{\pgfqpoint{2.101566in}{1.882656in}}{\pgfqpoint{2.093666in}{1.879384in}}{\pgfqpoint{2.087842in}{1.873560in}}%
\pgfpathcurveto{\pgfqpoint{2.082018in}{1.867736in}}{\pgfqpoint{2.078746in}{1.859836in}}{\pgfqpoint{2.078746in}{1.851600in}}%
\pgfpathcurveto{\pgfqpoint{2.078746in}{1.843364in}}{\pgfqpoint{2.082018in}{1.835463in}}{\pgfqpoint{2.087842in}{1.829640in}}%
\pgfpathcurveto{\pgfqpoint{2.093666in}{1.823816in}}{\pgfqpoint{2.101566in}{1.820543in}}{\pgfqpoint{2.109802in}{1.820543in}}%
\pgfpathclose%
\pgfusepath{stroke,fill}%
\end{pgfscope}%
\begin{pgfscope}%
\pgfpathrectangle{\pgfqpoint{0.100000in}{0.212622in}}{\pgfqpoint{3.696000in}{3.696000in}}%
\pgfusepath{clip}%
\pgfsetbuttcap%
\pgfsetroundjoin%
\definecolor{currentfill}{rgb}{0.121569,0.466667,0.705882}%
\pgfsetfillcolor{currentfill}%
\pgfsetfillopacity{0.605582}%
\pgfsetlinewidth{1.003750pt}%
\definecolor{currentstroke}{rgb}{0.121569,0.466667,0.705882}%
\pgfsetstrokecolor{currentstroke}%
\pgfsetstrokeopacity{0.605582}%
\pgfsetdash{}{0pt}%
\pgfpathmoveto{\pgfqpoint{0.946675in}{1.343147in}}%
\pgfpathcurveto{\pgfqpoint{0.954911in}{1.343147in}}{\pgfqpoint{0.962811in}{1.346420in}}{\pgfqpoint{0.968635in}{1.352244in}}%
\pgfpathcurveto{\pgfqpoint{0.974459in}{1.358068in}}{\pgfqpoint{0.977731in}{1.365968in}}{\pgfqpoint{0.977731in}{1.374204in}}%
\pgfpathcurveto{\pgfqpoint{0.977731in}{1.382440in}}{\pgfqpoint{0.974459in}{1.390340in}}{\pgfqpoint{0.968635in}{1.396164in}}%
\pgfpathcurveto{\pgfqpoint{0.962811in}{1.401988in}}{\pgfqpoint{0.954911in}{1.405260in}}{\pgfqpoint{0.946675in}{1.405260in}}%
\pgfpathcurveto{\pgfqpoint{0.938438in}{1.405260in}}{\pgfqpoint{0.930538in}{1.401988in}}{\pgfqpoint{0.924714in}{1.396164in}}%
\pgfpathcurveto{\pgfqpoint{0.918890in}{1.390340in}}{\pgfqpoint{0.915618in}{1.382440in}}{\pgfqpoint{0.915618in}{1.374204in}}%
\pgfpathcurveto{\pgfqpoint{0.915618in}{1.365968in}}{\pgfqpoint{0.918890in}{1.358068in}}{\pgfqpoint{0.924714in}{1.352244in}}%
\pgfpathcurveto{\pgfqpoint{0.930538in}{1.346420in}}{\pgfqpoint{0.938438in}{1.343147in}}{\pgfqpoint{0.946675in}{1.343147in}}%
\pgfpathclose%
\pgfusepath{stroke,fill}%
\end{pgfscope}%
\begin{pgfscope}%
\pgfpathrectangle{\pgfqpoint{0.100000in}{0.212622in}}{\pgfqpoint{3.696000in}{3.696000in}}%
\pgfusepath{clip}%
\pgfsetbuttcap%
\pgfsetroundjoin%
\definecolor{currentfill}{rgb}{0.121569,0.466667,0.705882}%
\pgfsetfillcolor{currentfill}%
\pgfsetfillopacity{0.605905}%
\pgfsetlinewidth{1.003750pt}%
\definecolor{currentstroke}{rgb}{0.121569,0.466667,0.705882}%
\pgfsetstrokecolor{currentstroke}%
\pgfsetstrokeopacity{0.605905}%
\pgfsetdash{}{0pt}%
\pgfpathmoveto{\pgfqpoint{2.111854in}{1.818241in}}%
\pgfpathcurveto{\pgfqpoint{2.120090in}{1.818241in}}{\pgfqpoint{2.127990in}{1.821513in}}{\pgfqpoint{2.133814in}{1.827337in}}%
\pgfpathcurveto{\pgfqpoint{2.139638in}{1.833161in}}{\pgfqpoint{2.142910in}{1.841061in}}{\pgfqpoint{2.142910in}{1.849297in}}%
\pgfpathcurveto{\pgfqpoint{2.142910in}{1.857533in}}{\pgfqpoint{2.139638in}{1.865434in}}{\pgfqpoint{2.133814in}{1.871257in}}%
\pgfpathcurveto{\pgfqpoint{2.127990in}{1.877081in}}{\pgfqpoint{2.120090in}{1.880354in}}{\pgfqpoint{2.111854in}{1.880354in}}%
\pgfpathcurveto{\pgfqpoint{2.103618in}{1.880354in}}{\pgfqpoint{2.095718in}{1.877081in}}{\pgfqpoint{2.089894in}{1.871257in}}%
\pgfpathcurveto{\pgfqpoint{2.084070in}{1.865434in}}{\pgfqpoint{2.080797in}{1.857533in}}{\pgfqpoint{2.080797in}{1.849297in}}%
\pgfpathcurveto{\pgfqpoint{2.080797in}{1.841061in}}{\pgfqpoint{2.084070in}{1.833161in}}{\pgfqpoint{2.089894in}{1.827337in}}%
\pgfpathcurveto{\pgfqpoint{2.095718in}{1.821513in}}{\pgfqpoint{2.103618in}{1.818241in}}{\pgfqpoint{2.111854in}{1.818241in}}%
\pgfpathclose%
\pgfusepath{stroke,fill}%
\end{pgfscope}%
\begin{pgfscope}%
\pgfpathrectangle{\pgfqpoint{0.100000in}{0.212622in}}{\pgfqpoint{3.696000in}{3.696000in}}%
\pgfusepath{clip}%
\pgfsetbuttcap%
\pgfsetroundjoin%
\definecolor{currentfill}{rgb}{0.121569,0.466667,0.705882}%
\pgfsetfillcolor{currentfill}%
\pgfsetfillopacity{0.608043}%
\pgfsetlinewidth{1.003750pt}%
\definecolor{currentstroke}{rgb}{0.121569,0.466667,0.705882}%
\pgfsetstrokecolor{currentstroke}%
\pgfsetstrokeopacity{0.608043}%
\pgfsetdash{}{0pt}%
\pgfpathmoveto{\pgfqpoint{0.937851in}{1.336552in}}%
\pgfpathcurveto{\pgfqpoint{0.946087in}{1.336552in}}{\pgfqpoint{0.953987in}{1.339824in}}{\pgfqpoint{0.959811in}{1.345648in}}%
\pgfpathcurveto{\pgfqpoint{0.965635in}{1.351472in}}{\pgfqpoint{0.968907in}{1.359372in}}{\pgfqpoint{0.968907in}{1.367608in}}%
\pgfpathcurveto{\pgfqpoint{0.968907in}{1.375844in}}{\pgfqpoint{0.965635in}{1.383744in}}{\pgfqpoint{0.959811in}{1.389568in}}%
\pgfpathcurveto{\pgfqpoint{0.953987in}{1.395392in}}{\pgfqpoint{0.946087in}{1.398665in}}{\pgfqpoint{0.937851in}{1.398665in}}%
\pgfpathcurveto{\pgfqpoint{0.929615in}{1.398665in}}{\pgfqpoint{0.921715in}{1.395392in}}{\pgfqpoint{0.915891in}{1.389568in}}%
\pgfpathcurveto{\pgfqpoint{0.910067in}{1.383744in}}{\pgfqpoint{0.906794in}{1.375844in}}{\pgfqpoint{0.906794in}{1.367608in}}%
\pgfpathcurveto{\pgfqpoint{0.906794in}{1.359372in}}{\pgfqpoint{0.910067in}{1.351472in}}{\pgfqpoint{0.915891in}{1.345648in}}%
\pgfpathcurveto{\pgfqpoint{0.921715in}{1.339824in}}{\pgfqpoint{0.929615in}{1.336552in}}{\pgfqpoint{0.937851in}{1.336552in}}%
\pgfpathclose%
\pgfusepath{stroke,fill}%
\end{pgfscope}%
\begin{pgfscope}%
\pgfpathrectangle{\pgfqpoint{0.100000in}{0.212622in}}{\pgfqpoint{3.696000in}{3.696000in}}%
\pgfusepath{clip}%
\pgfsetbuttcap%
\pgfsetroundjoin%
\definecolor{currentfill}{rgb}{0.121569,0.466667,0.705882}%
\pgfsetfillcolor{currentfill}%
\pgfsetfillopacity{0.608594}%
\pgfsetlinewidth{1.003750pt}%
\definecolor{currentstroke}{rgb}{0.121569,0.466667,0.705882}%
\pgfsetstrokecolor{currentstroke}%
\pgfsetstrokeopacity{0.608594}%
\pgfsetdash{}{0pt}%
\pgfpathmoveto{\pgfqpoint{2.113386in}{1.814236in}}%
\pgfpathcurveto{\pgfqpoint{2.121623in}{1.814236in}}{\pgfqpoint{2.129523in}{1.817508in}}{\pgfqpoint{2.135347in}{1.823332in}}%
\pgfpathcurveto{\pgfqpoint{2.141170in}{1.829156in}}{\pgfqpoint{2.144443in}{1.837056in}}{\pgfqpoint{2.144443in}{1.845292in}}%
\pgfpathcurveto{\pgfqpoint{2.144443in}{1.853529in}}{\pgfqpoint{2.141170in}{1.861429in}}{\pgfqpoint{2.135347in}{1.867253in}}%
\pgfpathcurveto{\pgfqpoint{2.129523in}{1.873077in}}{\pgfqpoint{2.121623in}{1.876349in}}{\pgfqpoint{2.113386in}{1.876349in}}%
\pgfpathcurveto{\pgfqpoint{2.105150in}{1.876349in}}{\pgfqpoint{2.097250in}{1.873077in}}{\pgfqpoint{2.091426in}{1.867253in}}%
\pgfpathcurveto{\pgfqpoint{2.085602in}{1.861429in}}{\pgfqpoint{2.082330in}{1.853529in}}{\pgfqpoint{2.082330in}{1.845292in}}%
\pgfpathcurveto{\pgfqpoint{2.082330in}{1.837056in}}{\pgfqpoint{2.085602in}{1.829156in}}{\pgfqpoint{2.091426in}{1.823332in}}%
\pgfpathcurveto{\pgfqpoint{2.097250in}{1.817508in}}{\pgfqpoint{2.105150in}{1.814236in}}{\pgfqpoint{2.113386in}{1.814236in}}%
\pgfpathclose%
\pgfusepath{stroke,fill}%
\end{pgfscope}%
\begin{pgfscope}%
\pgfpathrectangle{\pgfqpoint{0.100000in}{0.212622in}}{\pgfqpoint{3.696000in}{3.696000in}}%
\pgfusepath{clip}%
\pgfsetbuttcap%
\pgfsetroundjoin%
\definecolor{currentfill}{rgb}{0.121569,0.466667,0.705882}%
\pgfsetfillcolor{currentfill}%
\pgfsetfillopacity{0.610226}%
\pgfsetlinewidth{1.003750pt}%
\definecolor{currentstroke}{rgb}{0.121569,0.466667,0.705882}%
\pgfsetstrokecolor{currentstroke}%
\pgfsetstrokeopacity{0.610226}%
\pgfsetdash{}{0pt}%
\pgfpathmoveto{\pgfqpoint{0.933224in}{1.325868in}}%
\pgfpathcurveto{\pgfqpoint{0.941460in}{1.325868in}}{\pgfqpoint{0.949360in}{1.329140in}}{\pgfqpoint{0.955184in}{1.334964in}}%
\pgfpathcurveto{\pgfqpoint{0.961008in}{1.340788in}}{\pgfqpoint{0.964280in}{1.348688in}}{\pgfqpoint{0.964280in}{1.356924in}}%
\pgfpathcurveto{\pgfqpoint{0.964280in}{1.365161in}}{\pgfqpoint{0.961008in}{1.373061in}}{\pgfqpoint{0.955184in}{1.378885in}}%
\pgfpathcurveto{\pgfqpoint{0.949360in}{1.384709in}}{\pgfqpoint{0.941460in}{1.387981in}}{\pgfqpoint{0.933224in}{1.387981in}}%
\pgfpathcurveto{\pgfqpoint{0.924987in}{1.387981in}}{\pgfqpoint{0.917087in}{1.384709in}}{\pgfqpoint{0.911263in}{1.378885in}}%
\pgfpathcurveto{\pgfqpoint{0.905439in}{1.373061in}}{\pgfqpoint{0.902167in}{1.365161in}}{\pgfqpoint{0.902167in}{1.356924in}}%
\pgfpathcurveto{\pgfqpoint{0.902167in}{1.348688in}}{\pgfqpoint{0.905439in}{1.340788in}}{\pgfqpoint{0.911263in}{1.334964in}}%
\pgfpathcurveto{\pgfqpoint{0.917087in}{1.329140in}}{\pgfqpoint{0.924987in}{1.325868in}}{\pgfqpoint{0.933224in}{1.325868in}}%
\pgfpathclose%
\pgfusepath{stroke,fill}%
\end{pgfscope}%
\begin{pgfscope}%
\pgfpathrectangle{\pgfqpoint{0.100000in}{0.212622in}}{\pgfqpoint{3.696000in}{3.696000in}}%
\pgfusepath{clip}%
\pgfsetbuttcap%
\pgfsetroundjoin%
\definecolor{currentfill}{rgb}{0.121569,0.466667,0.705882}%
\pgfsetfillcolor{currentfill}%
\pgfsetfillopacity{0.611643}%
\pgfsetlinewidth{1.003750pt}%
\definecolor{currentstroke}{rgb}{0.121569,0.466667,0.705882}%
\pgfsetstrokecolor{currentstroke}%
\pgfsetstrokeopacity{0.611643}%
\pgfsetdash{}{0pt}%
\pgfpathmoveto{\pgfqpoint{2.115446in}{1.810615in}}%
\pgfpathcurveto{\pgfqpoint{2.123682in}{1.810615in}}{\pgfqpoint{2.131583in}{1.813887in}}{\pgfqpoint{2.137406in}{1.819711in}}%
\pgfpathcurveto{\pgfqpoint{2.143230in}{1.825535in}}{\pgfqpoint{2.146503in}{1.833435in}}{\pgfqpoint{2.146503in}{1.841671in}}%
\pgfpathcurveto{\pgfqpoint{2.146503in}{1.849908in}}{\pgfqpoint{2.143230in}{1.857808in}}{\pgfqpoint{2.137406in}{1.863632in}}%
\pgfpathcurveto{\pgfqpoint{2.131583in}{1.869456in}}{\pgfqpoint{2.123682in}{1.872728in}}{\pgfqpoint{2.115446in}{1.872728in}}%
\pgfpathcurveto{\pgfqpoint{2.107210in}{1.872728in}}{\pgfqpoint{2.099310in}{1.869456in}}{\pgfqpoint{2.093486in}{1.863632in}}%
\pgfpathcurveto{\pgfqpoint{2.087662in}{1.857808in}}{\pgfqpoint{2.084390in}{1.849908in}}{\pgfqpoint{2.084390in}{1.841671in}}%
\pgfpathcurveto{\pgfqpoint{2.084390in}{1.833435in}}{\pgfqpoint{2.087662in}{1.825535in}}{\pgfqpoint{2.093486in}{1.819711in}}%
\pgfpathcurveto{\pgfqpoint{2.099310in}{1.813887in}}{\pgfqpoint{2.107210in}{1.810615in}}{\pgfqpoint{2.115446in}{1.810615in}}%
\pgfpathclose%
\pgfusepath{stroke,fill}%
\end{pgfscope}%
\begin{pgfscope}%
\pgfpathrectangle{\pgfqpoint{0.100000in}{0.212622in}}{\pgfqpoint{3.696000in}{3.696000in}}%
\pgfusepath{clip}%
\pgfsetbuttcap%
\pgfsetroundjoin%
\definecolor{currentfill}{rgb}{0.121569,0.466667,0.705882}%
\pgfsetfillcolor{currentfill}%
\pgfsetfillopacity{0.611904}%
\pgfsetlinewidth{1.003750pt}%
\definecolor{currentstroke}{rgb}{0.121569,0.466667,0.705882}%
\pgfsetstrokecolor{currentstroke}%
\pgfsetstrokeopacity{0.611904}%
\pgfsetdash{}{0pt}%
\pgfpathmoveto{\pgfqpoint{0.927694in}{1.322232in}}%
\pgfpathcurveto{\pgfqpoint{0.935930in}{1.322232in}}{\pgfqpoint{0.943830in}{1.325504in}}{\pgfqpoint{0.949654in}{1.331328in}}%
\pgfpathcurveto{\pgfqpoint{0.955478in}{1.337152in}}{\pgfqpoint{0.958750in}{1.345052in}}{\pgfqpoint{0.958750in}{1.353288in}}%
\pgfpathcurveto{\pgfqpoint{0.958750in}{1.361525in}}{\pgfqpoint{0.955478in}{1.369425in}}{\pgfqpoint{0.949654in}{1.375249in}}%
\pgfpathcurveto{\pgfqpoint{0.943830in}{1.381072in}}{\pgfqpoint{0.935930in}{1.384345in}}{\pgfqpoint{0.927694in}{1.384345in}}%
\pgfpathcurveto{\pgfqpoint{0.919458in}{1.384345in}}{\pgfqpoint{0.911558in}{1.381072in}}{\pgfqpoint{0.905734in}{1.375249in}}%
\pgfpathcurveto{\pgfqpoint{0.899910in}{1.369425in}}{\pgfqpoint{0.896638in}{1.361525in}}{\pgfqpoint{0.896638in}{1.353288in}}%
\pgfpathcurveto{\pgfqpoint{0.896638in}{1.345052in}}{\pgfqpoint{0.899910in}{1.337152in}}{\pgfqpoint{0.905734in}{1.331328in}}%
\pgfpathcurveto{\pgfqpoint{0.911558in}{1.325504in}}{\pgfqpoint{0.919458in}{1.322232in}}{\pgfqpoint{0.927694in}{1.322232in}}%
\pgfpathclose%
\pgfusepath{stroke,fill}%
\end{pgfscope}%
\begin{pgfscope}%
\pgfpathrectangle{\pgfqpoint{0.100000in}{0.212622in}}{\pgfqpoint{3.696000in}{3.696000in}}%
\pgfusepath{clip}%
\pgfsetbuttcap%
\pgfsetroundjoin%
\definecolor{currentfill}{rgb}{0.121569,0.466667,0.705882}%
\pgfsetfillcolor{currentfill}%
\pgfsetfillopacity{0.613186}%
\pgfsetlinewidth{1.003750pt}%
\definecolor{currentstroke}{rgb}{0.121569,0.466667,0.705882}%
\pgfsetstrokecolor{currentstroke}%
\pgfsetstrokeopacity{0.613186}%
\pgfsetdash{}{0pt}%
\pgfpathmoveto{\pgfqpoint{0.924409in}{1.315336in}}%
\pgfpathcurveto{\pgfqpoint{0.932645in}{1.315336in}}{\pgfqpoint{0.940545in}{1.318608in}}{\pgfqpoint{0.946369in}{1.324432in}}%
\pgfpathcurveto{\pgfqpoint{0.952193in}{1.330256in}}{\pgfqpoint{0.955466in}{1.338156in}}{\pgfqpoint{0.955466in}{1.346392in}}%
\pgfpathcurveto{\pgfqpoint{0.955466in}{1.354628in}}{\pgfqpoint{0.952193in}{1.362528in}}{\pgfqpoint{0.946369in}{1.368352in}}%
\pgfpathcurveto{\pgfqpoint{0.940545in}{1.374176in}}{\pgfqpoint{0.932645in}{1.377449in}}{\pgfqpoint{0.924409in}{1.377449in}}%
\pgfpathcurveto{\pgfqpoint{0.916173in}{1.377449in}}{\pgfqpoint{0.908273in}{1.374176in}}{\pgfqpoint{0.902449in}{1.368352in}}%
\pgfpathcurveto{\pgfqpoint{0.896625in}{1.362528in}}{\pgfqpoint{0.893353in}{1.354628in}}{\pgfqpoint{0.893353in}{1.346392in}}%
\pgfpathcurveto{\pgfqpoint{0.893353in}{1.338156in}}{\pgfqpoint{0.896625in}{1.330256in}}{\pgfqpoint{0.902449in}{1.324432in}}%
\pgfpathcurveto{\pgfqpoint{0.908273in}{1.318608in}}{\pgfqpoint{0.916173in}{1.315336in}}{\pgfqpoint{0.924409in}{1.315336in}}%
\pgfpathclose%
\pgfusepath{stroke,fill}%
\end{pgfscope}%
\begin{pgfscope}%
\pgfpathrectangle{\pgfqpoint{0.100000in}{0.212622in}}{\pgfqpoint{3.696000in}{3.696000in}}%
\pgfusepath{clip}%
\pgfsetbuttcap%
\pgfsetroundjoin%
\definecolor{currentfill}{rgb}{0.121569,0.466667,0.705882}%
\pgfsetfillcolor{currentfill}%
\pgfsetfillopacity{0.614082}%
\pgfsetlinewidth{1.003750pt}%
\definecolor{currentstroke}{rgb}{0.121569,0.466667,0.705882}%
\pgfsetstrokecolor{currentstroke}%
\pgfsetstrokeopacity{0.614082}%
\pgfsetdash{}{0pt}%
\pgfpathmoveto{\pgfqpoint{0.921854in}{1.314010in}}%
\pgfpathcurveto{\pgfqpoint{0.930090in}{1.314010in}}{\pgfqpoint{0.937990in}{1.317282in}}{\pgfqpoint{0.943814in}{1.323106in}}%
\pgfpathcurveto{\pgfqpoint{0.949638in}{1.328930in}}{\pgfqpoint{0.952910in}{1.336830in}}{\pgfqpoint{0.952910in}{1.345066in}}%
\pgfpathcurveto{\pgfqpoint{0.952910in}{1.353302in}}{\pgfqpoint{0.949638in}{1.361203in}}{\pgfqpoint{0.943814in}{1.367026in}}%
\pgfpathcurveto{\pgfqpoint{0.937990in}{1.372850in}}{\pgfqpoint{0.930090in}{1.376123in}}{\pgfqpoint{0.921854in}{1.376123in}}%
\pgfpathcurveto{\pgfqpoint{0.913618in}{1.376123in}}{\pgfqpoint{0.905717in}{1.372850in}}{\pgfqpoint{0.899894in}{1.367026in}}%
\pgfpathcurveto{\pgfqpoint{0.894070in}{1.361203in}}{\pgfqpoint{0.890797in}{1.353302in}}{\pgfqpoint{0.890797in}{1.345066in}}%
\pgfpathcurveto{\pgfqpoint{0.890797in}{1.336830in}}{\pgfqpoint{0.894070in}{1.328930in}}{\pgfqpoint{0.899894in}{1.323106in}}%
\pgfpathcurveto{\pgfqpoint{0.905717in}{1.317282in}}{\pgfqpoint{0.913618in}{1.314010in}}{\pgfqpoint{0.921854in}{1.314010in}}%
\pgfpathclose%
\pgfusepath{stroke,fill}%
\end{pgfscope}%
\begin{pgfscope}%
\pgfpathrectangle{\pgfqpoint{0.100000in}{0.212622in}}{\pgfqpoint{3.696000in}{3.696000in}}%
\pgfusepath{clip}%
\pgfsetbuttcap%
\pgfsetroundjoin%
\definecolor{currentfill}{rgb}{0.121569,0.466667,0.705882}%
\pgfsetfillcolor{currentfill}%
\pgfsetfillopacity{0.615022}%
\pgfsetlinewidth{1.003750pt}%
\definecolor{currentstroke}{rgb}{0.121569,0.466667,0.705882}%
\pgfsetstrokecolor{currentstroke}%
\pgfsetstrokeopacity{0.615022}%
\pgfsetdash{}{0pt}%
\pgfpathmoveto{\pgfqpoint{2.117745in}{1.806112in}}%
\pgfpathcurveto{\pgfqpoint{2.125981in}{1.806112in}}{\pgfqpoint{2.133881in}{1.809385in}}{\pgfqpoint{2.139705in}{1.815209in}}%
\pgfpathcurveto{\pgfqpoint{2.145529in}{1.821033in}}{\pgfqpoint{2.148801in}{1.828933in}}{\pgfqpoint{2.148801in}{1.837169in}}%
\pgfpathcurveto{\pgfqpoint{2.148801in}{1.845405in}}{\pgfqpoint{2.145529in}{1.853305in}}{\pgfqpoint{2.139705in}{1.859129in}}%
\pgfpathcurveto{\pgfqpoint{2.133881in}{1.864953in}}{\pgfqpoint{2.125981in}{1.868225in}}{\pgfqpoint{2.117745in}{1.868225in}}%
\pgfpathcurveto{\pgfqpoint{2.109508in}{1.868225in}}{\pgfqpoint{2.101608in}{1.864953in}}{\pgfqpoint{2.095784in}{1.859129in}}%
\pgfpathcurveto{\pgfqpoint{2.089960in}{1.853305in}}{\pgfqpoint{2.086688in}{1.845405in}}{\pgfqpoint{2.086688in}{1.837169in}}%
\pgfpathcurveto{\pgfqpoint{2.086688in}{1.828933in}}{\pgfqpoint{2.089960in}{1.821033in}}{\pgfqpoint{2.095784in}{1.815209in}}%
\pgfpathcurveto{\pgfqpoint{2.101608in}{1.809385in}}{\pgfqpoint{2.109508in}{1.806112in}}{\pgfqpoint{2.117745in}{1.806112in}}%
\pgfpathclose%
\pgfusepath{stroke,fill}%
\end{pgfscope}%
\begin{pgfscope}%
\pgfpathrectangle{\pgfqpoint{0.100000in}{0.212622in}}{\pgfqpoint{3.696000in}{3.696000in}}%
\pgfusepath{clip}%
\pgfsetbuttcap%
\pgfsetroundjoin%
\definecolor{currentfill}{rgb}{0.121569,0.466667,0.705882}%
\pgfsetfillcolor{currentfill}%
\pgfsetfillopacity{0.615616}%
\pgfsetlinewidth{1.003750pt}%
\definecolor{currentstroke}{rgb}{0.121569,0.466667,0.705882}%
\pgfsetstrokecolor{currentstroke}%
\pgfsetstrokeopacity{0.615616}%
\pgfsetdash{}{0pt}%
\pgfpathmoveto{\pgfqpoint{0.917750in}{1.310330in}}%
\pgfpathcurveto{\pgfqpoint{0.925987in}{1.310330in}}{\pgfqpoint{0.933887in}{1.313602in}}{\pgfqpoint{0.939711in}{1.319426in}}%
\pgfpathcurveto{\pgfqpoint{0.945534in}{1.325250in}}{\pgfqpoint{0.948807in}{1.333150in}}{\pgfqpoint{0.948807in}{1.341386in}}%
\pgfpathcurveto{\pgfqpoint{0.948807in}{1.349622in}}{\pgfqpoint{0.945534in}{1.357522in}}{\pgfqpoint{0.939711in}{1.363346in}}%
\pgfpathcurveto{\pgfqpoint{0.933887in}{1.369170in}}{\pgfqpoint{0.925987in}{1.372443in}}{\pgfqpoint{0.917750in}{1.372443in}}%
\pgfpathcurveto{\pgfqpoint{0.909514in}{1.372443in}}{\pgfqpoint{0.901614in}{1.369170in}}{\pgfqpoint{0.895790in}{1.363346in}}%
\pgfpathcurveto{\pgfqpoint{0.889966in}{1.357522in}}{\pgfqpoint{0.886694in}{1.349622in}}{\pgfqpoint{0.886694in}{1.341386in}}%
\pgfpathcurveto{\pgfqpoint{0.886694in}{1.333150in}}{\pgfqpoint{0.889966in}{1.325250in}}{\pgfqpoint{0.895790in}{1.319426in}}%
\pgfpathcurveto{\pgfqpoint{0.901614in}{1.313602in}}{\pgfqpoint{0.909514in}{1.310330in}}{\pgfqpoint{0.917750in}{1.310330in}}%
\pgfpathclose%
\pgfusepath{stroke,fill}%
\end{pgfscope}%
\begin{pgfscope}%
\pgfpathrectangle{\pgfqpoint{0.100000in}{0.212622in}}{\pgfqpoint{3.696000in}{3.696000in}}%
\pgfusepath{clip}%
\pgfsetbuttcap%
\pgfsetroundjoin%
\definecolor{currentfill}{rgb}{0.121569,0.466667,0.705882}%
\pgfsetfillcolor{currentfill}%
\pgfsetfillopacity{0.618191}%
\pgfsetlinewidth{1.003750pt}%
\definecolor{currentstroke}{rgb}{0.121569,0.466667,0.705882}%
\pgfsetstrokecolor{currentstroke}%
\pgfsetstrokeopacity{0.618191}%
\pgfsetdash{}{0pt}%
\pgfpathmoveto{\pgfqpoint{0.909591in}{1.303149in}}%
\pgfpathcurveto{\pgfqpoint{0.917827in}{1.303149in}}{\pgfqpoint{0.925727in}{1.306421in}}{\pgfqpoint{0.931551in}{1.312245in}}%
\pgfpathcurveto{\pgfqpoint{0.937375in}{1.318069in}}{\pgfqpoint{0.940647in}{1.325969in}}{\pgfqpoint{0.940647in}{1.334205in}}%
\pgfpathcurveto{\pgfqpoint{0.940647in}{1.342441in}}{\pgfqpoint{0.937375in}{1.350341in}}{\pgfqpoint{0.931551in}{1.356165in}}%
\pgfpathcurveto{\pgfqpoint{0.925727in}{1.361989in}}{\pgfqpoint{0.917827in}{1.365262in}}{\pgfqpoint{0.909591in}{1.365262in}}%
\pgfpathcurveto{\pgfqpoint{0.901355in}{1.365262in}}{\pgfqpoint{0.893454in}{1.361989in}}{\pgfqpoint{0.887631in}{1.356165in}}%
\pgfpathcurveto{\pgfqpoint{0.881807in}{1.350341in}}{\pgfqpoint{0.878534in}{1.342441in}}{\pgfqpoint{0.878534in}{1.334205in}}%
\pgfpathcurveto{\pgfqpoint{0.878534in}{1.325969in}}{\pgfqpoint{0.881807in}{1.318069in}}{\pgfqpoint{0.887631in}{1.312245in}}%
\pgfpathcurveto{\pgfqpoint{0.893454in}{1.306421in}}{\pgfqpoint{0.901355in}{1.303149in}}{\pgfqpoint{0.909591in}{1.303149in}}%
\pgfpathclose%
\pgfusepath{stroke,fill}%
\end{pgfscope}%
\begin{pgfscope}%
\pgfpathrectangle{\pgfqpoint{0.100000in}{0.212622in}}{\pgfqpoint{3.696000in}{3.696000in}}%
\pgfusepath{clip}%
\pgfsetbuttcap%
\pgfsetroundjoin%
\definecolor{currentfill}{rgb}{0.121569,0.466667,0.705882}%
\pgfsetfillcolor{currentfill}%
\pgfsetfillopacity{0.618962}%
\pgfsetlinewidth{1.003750pt}%
\definecolor{currentstroke}{rgb}{0.121569,0.466667,0.705882}%
\pgfsetstrokecolor{currentstroke}%
\pgfsetstrokeopacity{0.618962}%
\pgfsetdash{}{0pt}%
\pgfpathmoveto{\pgfqpoint{2.120504in}{1.802850in}}%
\pgfpathcurveto{\pgfqpoint{2.128741in}{1.802850in}}{\pgfqpoint{2.136641in}{1.806123in}}{\pgfqpoint{2.142465in}{1.811946in}}%
\pgfpathcurveto{\pgfqpoint{2.148289in}{1.817770in}}{\pgfqpoint{2.151561in}{1.825670in}}{\pgfqpoint{2.151561in}{1.833907in}}%
\pgfpathcurveto{\pgfqpoint{2.151561in}{1.842143in}}{\pgfqpoint{2.148289in}{1.850043in}}{\pgfqpoint{2.142465in}{1.855867in}}%
\pgfpathcurveto{\pgfqpoint{2.136641in}{1.861691in}}{\pgfqpoint{2.128741in}{1.864963in}}{\pgfqpoint{2.120504in}{1.864963in}}%
\pgfpathcurveto{\pgfqpoint{2.112268in}{1.864963in}}{\pgfqpoint{2.104368in}{1.861691in}}{\pgfqpoint{2.098544in}{1.855867in}}%
\pgfpathcurveto{\pgfqpoint{2.092720in}{1.850043in}}{\pgfqpoint{2.089448in}{1.842143in}}{\pgfqpoint{2.089448in}{1.833907in}}%
\pgfpathcurveto{\pgfqpoint{2.089448in}{1.825670in}}{\pgfqpoint{2.092720in}{1.817770in}}{\pgfqpoint{2.098544in}{1.811946in}}%
\pgfpathcurveto{\pgfqpoint{2.104368in}{1.806123in}}{\pgfqpoint{2.112268in}{1.802850in}}{\pgfqpoint{2.120504in}{1.802850in}}%
\pgfpathclose%
\pgfusepath{stroke,fill}%
\end{pgfscope}%
\begin{pgfscope}%
\pgfpathrectangle{\pgfqpoint{0.100000in}{0.212622in}}{\pgfqpoint{3.696000in}{3.696000in}}%
\pgfusepath{clip}%
\pgfsetbuttcap%
\pgfsetroundjoin%
\definecolor{currentfill}{rgb}{0.121569,0.466667,0.705882}%
\pgfsetfillcolor{currentfill}%
\pgfsetfillopacity{0.620174}%
\pgfsetlinewidth{1.003750pt}%
\definecolor{currentstroke}{rgb}{0.121569,0.466667,0.705882}%
\pgfsetstrokecolor{currentstroke}%
\pgfsetstrokeopacity{0.620174}%
\pgfsetdash{}{0pt}%
\pgfpathmoveto{\pgfqpoint{0.902359in}{1.297422in}}%
\pgfpathcurveto{\pgfqpoint{0.910595in}{1.297422in}}{\pgfqpoint{0.918495in}{1.300695in}}{\pgfqpoint{0.924319in}{1.306519in}}%
\pgfpathcurveto{\pgfqpoint{0.930143in}{1.312343in}}{\pgfqpoint{0.933415in}{1.320243in}}{\pgfqpoint{0.933415in}{1.328479in}}%
\pgfpathcurveto{\pgfqpoint{0.933415in}{1.336715in}}{\pgfqpoint{0.930143in}{1.344615in}}{\pgfqpoint{0.924319in}{1.350439in}}%
\pgfpathcurveto{\pgfqpoint{0.918495in}{1.356263in}}{\pgfqpoint{0.910595in}{1.359535in}}{\pgfqpoint{0.902359in}{1.359535in}}%
\pgfpathcurveto{\pgfqpoint{0.894123in}{1.359535in}}{\pgfqpoint{0.886223in}{1.356263in}}{\pgfqpoint{0.880399in}{1.350439in}}%
\pgfpathcurveto{\pgfqpoint{0.874575in}{1.344615in}}{\pgfqpoint{0.871302in}{1.336715in}}{\pgfqpoint{0.871302in}{1.328479in}}%
\pgfpathcurveto{\pgfqpoint{0.871302in}{1.320243in}}{\pgfqpoint{0.874575in}{1.312343in}}{\pgfqpoint{0.880399in}{1.306519in}}%
\pgfpathcurveto{\pgfqpoint{0.886223in}{1.300695in}}{\pgfqpoint{0.894123in}{1.297422in}}{\pgfqpoint{0.902359in}{1.297422in}}%
\pgfpathclose%
\pgfusepath{stroke,fill}%
\end{pgfscope}%
\begin{pgfscope}%
\pgfpathrectangle{\pgfqpoint{0.100000in}{0.212622in}}{\pgfqpoint{3.696000in}{3.696000in}}%
\pgfusepath{clip}%
\pgfsetbuttcap%
\pgfsetroundjoin%
\definecolor{currentfill}{rgb}{0.121569,0.466667,0.705882}%
\pgfsetfillcolor{currentfill}%
\pgfsetfillopacity{0.621748}%
\pgfsetlinewidth{1.003750pt}%
\definecolor{currentstroke}{rgb}{0.121569,0.466667,0.705882}%
\pgfsetstrokecolor{currentstroke}%
\pgfsetstrokeopacity{0.621748}%
\pgfsetdash{}{0pt}%
\pgfpathmoveto{\pgfqpoint{0.897808in}{1.292766in}}%
\pgfpathcurveto{\pgfqpoint{0.906044in}{1.292766in}}{\pgfqpoint{0.913944in}{1.296038in}}{\pgfqpoint{0.919768in}{1.301862in}}%
\pgfpathcurveto{\pgfqpoint{0.925592in}{1.307686in}}{\pgfqpoint{0.928864in}{1.315586in}}{\pgfqpoint{0.928864in}{1.323822in}}%
\pgfpathcurveto{\pgfqpoint{0.928864in}{1.332059in}}{\pgfqpoint{0.925592in}{1.339959in}}{\pgfqpoint{0.919768in}{1.345783in}}%
\pgfpathcurveto{\pgfqpoint{0.913944in}{1.351607in}}{\pgfqpoint{0.906044in}{1.354879in}}{\pgfqpoint{0.897808in}{1.354879in}}%
\pgfpathcurveto{\pgfqpoint{0.889571in}{1.354879in}}{\pgfqpoint{0.881671in}{1.351607in}}{\pgfqpoint{0.875847in}{1.345783in}}%
\pgfpathcurveto{\pgfqpoint{0.870023in}{1.339959in}}{\pgfqpoint{0.866751in}{1.332059in}}{\pgfqpoint{0.866751in}{1.323822in}}%
\pgfpathcurveto{\pgfqpoint{0.866751in}{1.315586in}}{\pgfqpoint{0.870023in}{1.307686in}}{\pgfqpoint{0.875847in}{1.301862in}}%
\pgfpathcurveto{\pgfqpoint{0.881671in}{1.296038in}}{\pgfqpoint{0.889571in}{1.292766in}}{\pgfqpoint{0.897808in}{1.292766in}}%
\pgfpathclose%
\pgfusepath{stroke,fill}%
\end{pgfscope}%
\begin{pgfscope}%
\pgfpathrectangle{\pgfqpoint{0.100000in}{0.212622in}}{\pgfqpoint{3.696000in}{3.696000in}}%
\pgfusepath{clip}%
\pgfsetbuttcap%
\pgfsetroundjoin%
\definecolor{currentfill}{rgb}{0.121569,0.466667,0.705882}%
\pgfsetfillcolor{currentfill}%
\pgfsetfillopacity{0.622094}%
\pgfsetlinewidth{1.003750pt}%
\definecolor{currentstroke}{rgb}{0.121569,0.466667,0.705882}%
\pgfsetstrokecolor{currentstroke}%
\pgfsetstrokeopacity{0.622094}%
\pgfsetdash{}{0pt}%
\pgfpathmoveto{\pgfqpoint{0.913757in}{1.254265in}}%
\pgfpathcurveto{\pgfqpoint{0.921993in}{1.254265in}}{\pgfqpoint{0.929893in}{1.257538in}}{\pgfqpoint{0.935717in}{1.263362in}}%
\pgfpathcurveto{\pgfqpoint{0.941541in}{1.269186in}}{\pgfqpoint{0.944813in}{1.277086in}}{\pgfqpoint{0.944813in}{1.285322in}}%
\pgfpathcurveto{\pgfqpoint{0.944813in}{1.293558in}}{\pgfqpoint{0.941541in}{1.301458in}}{\pgfqpoint{0.935717in}{1.307282in}}%
\pgfpathcurveto{\pgfqpoint{0.929893in}{1.313106in}}{\pgfqpoint{0.921993in}{1.316378in}}{\pgfqpoint{0.913757in}{1.316378in}}%
\pgfpathcurveto{\pgfqpoint{0.905520in}{1.316378in}}{\pgfqpoint{0.897620in}{1.313106in}}{\pgfqpoint{0.891796in}{1.307282in}}%
\pgfpathcurveto{\pgfqpoint{0.885972in}{1.301458in}}{\pgfqpoint{0.882700in}{1.293558in}}{\pgfqpoint{0.882700in}{1.285322in}}%
\pgfpathcurveto{\pgfqpoint{0.882700in}{1.277086in}}{\pgfqpoint{0.885972in}{1.269186in}}{\pgfqpoint{0.891796in}{1.263362in}}%
\pgfpathcurveto{\pgfqpoint{0.897620in}{1.257538in}}{\pgfqpoint{0.905520in}{1.254265in}}{\pgfqpoint{0.913757in}{1.254265in}}%
\pgfpathclose%
\pgfusepath{stroke,fill}%
\end{pgfscope}%
\begin{pgfscope}%
\pgfpathrectangle{\pgfqpoint{0.100000in}{0.212622in}}{\pgfqpoint{3.696000in}{3.696000in}}%
\pgfusepath{clip}%
\pgfsetbuttcap%
\pgfsetroundjoin%
\definecolor{currentfill}{rgb}{0.121569,0.466667,0.705882}%
\pgfsetfillcolor{currentfill}%
\pgfsetfillopacity{0.622215}%
\pgfsetlinewidth{1.003750pt}%
\definecolor{currentstroke}{rgb}{0.121569,0.466667,0.705882}%
\pgfsetstrokecolor{currentstroke}%
\pgfsetstrokeopacity{0.622215}%
\pgfsetdash{}{0pt}%
\pgfpathmoveto{\pgfqpoint{0.913535in}{1.254352in}}%
\pgfpathcurveto{\pgfqpoint{0.921772in}{1.254352in}}{\pgfqpoint{0.929672in}{1.257624in}}{\pgfqpoint{0.935496in}{1.263448in}}%
\pgfpathcurveto{\pgfqpoint{0.941320in}{1.269272in}}{\pgfqpoint{0.944592in}{1.277172in}}{\pgfqpoint{0.944592in}{1.285408in}}%
\pgfpathcurveto{\pgfqpoint{0.944592in}{1.293644in}}{\pgfqpoint{0.941320in}{1.301544in}}{\pgfqpoint{0.935496in}{1.307368in}}%
\pgfpathcurveto{\pgfqpoint{0.929672in}{1.313192in}}{\pgfqpoint{0.921772in}{1.316465in}}{\pgfqpoint{0.913535in}{1.316465in}}%
\pgfpathcurveto{\pgfqpoint{0.905299in}{1.316465in}}{\pgfqpoint{0.897399in}{1.313192in}}{\pgfqpoint{0.891575in}{1.307368in}}%
\pgfpathcurveto{\pgfqpoint{0.885751in}{1.301544in}}{\pgfqpoint{0.882479in}{1.293644in}}{\pgfqpoint{0.882479in}{1.285408in}}%
\pgfpathcurveto{\pgfqpoint{0.882479in}{1.277172in}}{\pgfqpoint{0.885751in}{1.269272in}}{\pgfqpoint{0.891575in}{1.263448in}}%
\pgfpathcurveto{\pgfqpoint{0.897399in}{1.257624in}}{\pgfqpoint{0.905299in}{1.254352in}}{\pgfqpoint{0.913535in}{1.254352in}}%
\pgfpathclose%
\pgfusepath{stroke,fill}%
\end{pgfscope}%
\begin{pgfscope}%
\pgfpathrectangle{\pgfqpoint{0.100000in}{0.212622in}}{\pgfqpoint{3.696000in}{3.696000in}}%
\pgfusepath{clip}%
\pgfsetbuttcap%
\pgfsetroundjoin%
\definecolor{currentfill}{rgb}{0.121569,0.466667,0.705882}%
\pgfsetfillcolor{currentfill}%
\pgfsetfillopacity{0.622555}%
\pgfsetlinewidth{1.003750pt}%
\definecolor{currentstroke}{rgb}{0.121569,0.466667,0.705882}%
\pgfsetstrokecolor{currentstroke}%
\pgfsetstrokeopacity{0.622555}%
\pgfsetdash{}{0pt}%
\pgfpathmoveto{\pgfqpoint{0.912907in}{1.254601in}}%
\pgfpathcurveto{\pgfqpoint{0.921144in}{1.254601in}}{\pgfqpoint{0.929044in}{1.257874in}}{\pgfqpoint{0.934868in}{1.263698in}}%
\pgfpathcurveto{\pgfqpoint{0.940692in}{1.269522in}}{\pgfqpoint{0.943964in}{1.277422in}}{\pgfqpoint{0.943964in}{1.285658in}}%
\pgfpathcurveto{\pgfqpoint{0.943964in}{1.293894in}}{\pgfqpoint{0.940692in}{1.301794in}}{\pgfqpoint{0.934868in}{1.307618in}}%
\pgfpathcurveto{\pgfqpoint{0.929044in}{1.313442in}}{\pgfqpoint{0.921144in}{1.316714in}}{\pgfqpoint{0.912907in}{1.316714in}}%
\pgfpathcurveto{\pgfqpoint{0.904671in}{1.316714in}}{\pgfqpoint{0.896771in}{1.313442in}}{\pgfqpoint{0.890947in}{1.307618in}}%
\pgfpathcurveto{\pgfqpoint{0.885123in}{1.301794in}}{\pgfqpoint{0.881851in}{1.293894in}}{\pgfqpoint{0.881851in}{1.285658in}}%
\pgfpathcurveto{\pgfqpoint{0.881851in}{1.277422in}}{\pgfqpoint{0.885123in}{1.269522in}}{\pgfqpoint{0.890947in}{1.263698in}}%
\pgfpathcurveto{\pgfqpoint{0.896771in}{1.257874in}}{\pgfqpoint{0.904671in}{1.254601in}}{\pgfqpoint{0.912907in}{1.254601in}}%
\pgfpathclose%
\pgfusepath{stroke,fill}%
\end{pgfscope}%
\begin{pgfscope}%
\pgfpathrectangle{\pgfqpoint{0.100000in}{0.212622in}}{\pgfqpoint{3.696000in}{3.696000in}}%
\pgfusepath{clip}%
\pgfsetbuttcap%
\pgfsetroundjoin%
\definecolor{currentfill}{rgb}{0.121569,0.466667,0.705882}%
\pgfsetfillcolor{currentfill}%
\pgfsetfillopacity{0.622781}%
\pgfsetlinewidth{1.003750pt}%
\definecolor{currentstroke}{rgb}{0.121569,0.466667,0.705882}%
\pgfsetstrokecolor{currentstroke}%
\pgfsetstrokeopacity{0.622781}%
\pgfsetdash{}{0pt}%
\pgfpathmoveto{\pgfqpoint{0.894630in}{1.291939in}}%
\pgfpathcurveto{\pgfqpoint{0.902866in}{1.291939in}}{\pgfqpoint{0.910766in}{1.295211in}}{\pgfqpoint{0.916590in}{1.301035in}}%
\pgfpathcurveto{\pgfqpoint{0.922414in}{1.306859in}}{\pgfqpoint{0.925686in}{1.314759in}}{\pgfqpoint{0.925686in}{1.322996in}}%
\pgfpathcurveto{\pgfqpoint{0.925686in}{1.331232in}}{\pgfqpoint{0.922414in}{1.339132in}}{\pgfqpoint{0.916590in}{1.344956in}}%
\pgfpathcurveto{\pgfqpoint{0.910766in}{1.350780in}}{\pgfqpoint{0.902866in}{1.354052in}}{\pgfqpoint{0.894630in}{1.354052in}}%
\pgfpathcurveto{\pgfqpoint{0.886394in}{1.354052in}}{\pgfqpoint{0.878494in}{1.350780in}}{\pgfqpoint{0.872670in}{1.344956in}}%
\pgfpathcurveto{\pgfqpoint{0.866846in}{1.339132in}}{\pgfqpoint{0.863573in}{1.331232in}}{\pgfqpoint{0.863573in}{1.322996in}}%
\pgfpathcurveto{\pgfqpoint{0.863573in}{1.314759in}}{\pgfqpoint{0.866846in}{1.306859in}}{\pgfqpoint{0.872670in}{1.301035in}}%
\pgfpathcurveto{\pgfqpoint{0.878494in}{1.295211in}}{\pgfqpoint{0.886394in}{1.291939in}}{\pgfqpoint{0.894630in}{1.291939in}}%
\pgfpathclose%
\pgfusepath{stroke,fill}%
\end{pgfscope}%
\begin{pgfscope}%
\pgfpathrectangle{\pgfqpoint{0.100000in}{0.212622in}}{\pgfqpoint{3.696000in}{3.696000in}}%
\pgfusepath{clip}%
\pgfsetbuttcap%
\pgfsetroundjoin%
\definecolor{currentfill}{rgb}{0.121569,0.466667,0.705882}%
\pgfsetfillcolor{currentfill}%
\pgfsetfillopacity{0.623107}%
\pgfsetlinewidth{1.003750pt}%
\definecolor{currentstroke}{rgb}{0.121569,0.466667,0.705882}%
\pgfsetstrokecolor{currentstroke}%
\pgfsetstrokeopacity{0.623107}%
\pgfsetdash{}{0pt}%
\pgfpathmoveto{\pgfqpoint{2.122341in}{1.799056in}}%
\pgfpathcurveto{\pgfqpoint{2.130577in}{1.799056in}}{\pgfqpoint{2.138477in}{1.802329in}}{\pgfqpoint{2.144301in}{1.808153in}}%
\pgfpathcurveto{\pgfqpoint{2.150125in}{1.813977in}}{\pgfqpoint{2.153397in}{1.821877in}}{\pgfqpoint{2.153397in}{1.830113in}}%
\pgfpathcurveto{\pgfqpoint{2.153397in}{1.838349in}}{\pgfqpoint{2.150125in}{1.846249in}}{\pgfqpoint{2.144301in}{1.852073in}}%
\pgfpathcurveto{\pgfqpoint{2.138477in}{1.857897in}}{\pgfqpoint{2.130577in}{1.861169in}}{\pgfqpoint{2.122341in}{1.861169in}}%
\pgfpathcurveto{\pgfqpoint{2.114104in}{1.861169in}}{\pgfqpoint{2.106204in}{1.857897in}}{\pgfqpoint{2.100380in}{1.852073in}}%
\pgfpathcurveto{\pgfqpoint{2.094557in}{1.846249in}}{\pgfqpoint{2.091284in}{1.838349in}}{\pgfqpoint{2.091284in}{1.830113in}}%
\pgfpathcurveto{\pgfqpoint{2.091284in}{1.821877in}}{\pgfqpoint{2.094557in}{1.813977in}}{\pgfqpoint{2.100380in}{1.808153in}}%
\pgfpathcurveto{\pgfqpoint{2.106204in}{1.802329in}}{\pgfqpoint{2.114104in}{1.799056in}}{\pgfqpoint{2.122341in}{1.799056in}}%
\pgfpathclose%
\pgfusepath{stroke,fill}%
\end{pgfscope}%
\begin{pgfscope}%
\pgfpathrectangle{\pgfqpoint{0.100000in}{0.212622in}}{\pgfqpoint{3.696000in}{3.696000in}}%
\pgfusepath{clip}%
\pgfsetbuttcap%
\pgfsetroundjoin%
\definecolor{currentfill}{rgb}{0.121569,0.466667,0.705882}%
\pgfsetfillcolor{currentfill}%
\pgfsetfillopacity{0.623181}%
\pgfsetlinewidth{1.003750pt}%
\definecolor{currentstroke}{rgb}{0.121569,0.466667,0.705882}%
\pgfsetstrokecolor{currentstroke}%
\pgfsetstrokeopacity{0.623181}%
\pgfsetdash{}{0pt}%
\pgfpathmoveto{\pgfqpoint{0.893467in}{1.290280in}}%
\pgfpathcurveto{\pgfqpoint{0.901704in}{1.290280in}}{\pgfqpoint{0.909604in}{1.293553in}}{\pgfqpoint{0.915428in}{1.299376in}}%
\pgfpathcurveto{\pgfqpoint{0.921252in}{1.305200in}}{\pgfqpoint{0.924524in}{1.313100in}}{\pgfqpoint{0.924524in}{1.321337in}}%
\pgfpathcurveto{\pgfqpoint{0.924524in}{1.329573in}}{\pgfqpoint{0.921252in}{1.337473in}}{\pgfqpoint{0.915428in}{1.343297in}}%
\pgfpathcurveto{\pgfqpoint{0.909604in}{1.349121in}}{\pgfqpoint{0.901704in}{1.352393in}}{\pgfqpoint{0.893467in}{1.352393in}}%
\pgfpathcurveto{\pgfqpoint{0.885231in}{1.352393in}}{\pgfqpoint{0.877331in}{1.349121in}}{\pgfqpoint{0.871507in}{1.343297in}}%
\pgfpathcurveto{\pgfqpoint{0.865683in}{1.337473in}}{\pgfqpoint{0.862411in}{1.329573in}}{\pgfqpoint{0.862411in}{1.321337in}}%
\pgfpathcurveto{\pgfqpoint{0.862411in}{1.313100in}}{\pgfqpoint{0.865683in}{1.305200in}}{\pgfqpoint{0.871507in}{1.299376in}}%
\pgfpathcurveto{\pgfqpoint{0.877331in}{1.293553in}}{\pgfqpoint{0.885231in}{1.290280in}}{\pgfqpoint{0.893467in}{1.290280in}}%
\pgfpathclose%
\pgfusepath{stroke,fill}%
\end{pgfscope}%
\begin{pgfscope}%
\pgfpathrectangle{\pgfqpoint{0.100000in}{0.212622in}}{\pgfqpoint{3.696000in}{3.696000in}}%
\pgfusepath{clip}%
\pgfsetbuttcap%
\pgfsetroundjoin%
\definecolor{currentfill}{rgb}{0.121569,0.466667,0.705882}%
\pgfsetfillcolor{currentfill}%
\pgfsetfillopacity{0.623181}%
\pgfsetlinewidth{1.003750pt}%
\definecolor{currentstroke}{rgb}{0.121569,0.466667,0.705882}%
\pgfsetstrokecolor{currentstroke}%
\pgfsetstrokeopacity{0.623181}%
\pgfsetdash{}{0pt}%
\pgfpathmoveto{\pgfqpoint{0.893467in}{1.290280in}}%
\pgfpathcurveto{\pgfqpoint{0.901704in}{1.290280in}}{\pgfqpoint{0.909604in}{1.293552in}}{\pgfqpoint{0.915428in}{1.299376in}}%
\pgfpathcurveto{\pgfqpoint{0.921252in}{1.305200in}}{\pgfqpoint{0.924524in}{1.313100in}}{\pgfqpoint{0.924524in}{1.321337in}}%
\pgfpathcurveto{\pgfqpoint{0.924524in}{1.329573in}}{\pgfqpoint{0.921252in}{1.337473in}}{\pgfqpoint{0.915428in}{1.343297in}}%
\pgfpathcurveto{\pgfqpoint{0.909604in}{1.349121in}}{\pgfqpoint{0.901704in}{1.352393in}}{\pgfqpoint{0.893467in}{1.352393in}}%
\pgfpathcurveto{\pgfqpoint{0.885231in}{1.352393in}}{\pgfqpoint{0.877331in}{1.349121in}}{\pgfqpoint{0.871507in}{1.343297in}}%
\pgfpathcurveto{\pgfqpoint{0.865683in}{1.337473in}}{\pgfqpoint{0.862411in}{1.329573in}}{\pgfqpoint{0.862411in}{1.321337in}}%
\pgfpathcurveto{\pgfqpoint{0.862411in}{1.313100in}}{\pgfqpoint{0.865683in}{1.305200in}}{\pgfqpoint{0.871507in}{1.299376in}}%
\pgfpathcurveto{\pgfqpoint{0.877331in}{1.293552in}}{\pgfqpoint{0.885231in}{1.290280in}}{\pgfqpoint{0.893467in}{1.290280in}}%
\pgfpathclose%
\pgfusepath{stroke,fill}%
\end{pgfscope}%
\begin{pgfscope}%
\pgfpathrectangle{\pgfqpoint{0.100000in}{0.212622in}}{\pgfqpoint{3.696000in}{3.696000in}}%
\pgfusepath{clip}%
\pgfsetbuttcap%
\pgfsetroundjoin%
\definecolor{currentfill}{rgb}{0.121569,0.466667,0.705882}%
\pgfsetfillcolor{currentfill}%
\pgfsetfillopacity{0.623181}%
\pgfsetlinewidth{1.003750pt}%
\definecolor{currentstroke}{rgb}{0.121569,0.466667,0.705882}%
\pgfsetstrokecolor{currentstroke}%
\pgfsetstrokeopacity{0.623181}%
\pgfsetdash{}{0pt}%
\pgfpathmoveto{\pgfqpoint{0.893467in}{1.290280in}}%
\pgfpathcurveto{\pgfqpoint{0.901704in}{1.290280in}}{\pgfqpoint{0.909604in}{1.293552in}}{\pgfqpoint{0.915428in}{1.299376in}}%
\pgfpathcurveto{\pgfqpoint{0.921252in}{1.305200in}}{\pgfqpoint{0.924524in}{1.313100in}}{\pgfqpoint{0.924524in}{1.321337in}}%
\pgfpathcurveto{\pgfqpoint{0.924524in}{1.329573in}}{\pgfqpoint{0.921252in}{1.337473in}}{\pgfqpoint{0.915428in}{1.343297in}}%
\pgfpathcurveto{\pgfqpoint{0.909604in}{1.349121in}}{\pgfqpoint{0.901704in}{1.352393in}}{\pgfqpoint{0.893467in}{1.352393in}}%
\pgfpathcurveto{\pgfqpoint{0.885231in}{1.352393in}}{\pgfqpoint{0.877331in}{1.349121in}}{\pgfqpoint{0.871507in}{1.343297in}}%
\pgfpathcurveto{\pgfqpoint{0.865683in}{1.337473in}}{\pgfqpoint{0.862411in}{1.329573in}}{\pgfqpoint{0.862411in}{1.321337in}}%
\pgfpathcurveto{\pgfqpoint{0.862411in}{1.313100in}}{\pgfqpoint{0.865683in}{1.305200in}}{\pgfqpoint{0.871507in}{1.299376in}}%
\pgfpathcurveto{\pgfqpoint{0.877331in}{1.293552in}}{\pgfqpoint{0.885231in}{1.290280in}}{\pgfqpoint{0.893467in}{1.290280in}}%
\pgfpathclose%
\pgfusepath{stroke,fill}%
\end{pgfscope}%
\begin{pgfscope}%
\pgfpathrectangle{\pgfqpoint{0.100000in}{0.212622in}}{\pgfqpoint{3.696000in}{3.696000in}}%
\pgfusepath{clip}%
\pgfsetbuttcap%
\pgfsetroundjoin%
\definecolor{currentfill}{rgb}{0.121569,0.466667,0.705882}%
\pgfsetfillcolor{currentfill}%
\pgfsetfillopacity{0.623181}%
\pgfsetlinewidth{1.003750pt}%
\definecolor{currentstroke}{rgb}{0.121569,0.466667,0.705882}%
\pgfsetstrokecolor{currentstroke}%
\pgfsetstrokeopacity{0.623181}%
\pgfsetdash{}{0pt}%
\pgfpathmoveto{\pgfqpoint{0.893467in}{1.290280in}}%
\pgfpathcurveto{\pgfqpoint{0.901703in}{1.290280in}}{\pgfqpoint{0.909603in}{1.293552in}}{\pgfqpoint{0.915427in}{1.299376in}}%
\pgfpathcurveto{\pgfqpoint{0.921251in}{1.305200in}}{\pgfqpoint{0.924524in}{1.313100in}}{\pgfqpoint{0.924524in}{1.321336in}}%
\pgfpathcurveto{\pgfqpoint{0.924524in}{1.329573in}}{\pgfqpoint{0.921251in}{1.337473in}}{\pgfqpoint{0.915427in}{1.343297in}}%
\pgfpathcurveto{\pgfqpoint{0.909603in}{1.349121in}}{\pgfqpoint{0.901703in}{1.352393in}}{\pgfqpoint{0.893467in}{1.352393in}}%
\pgfpathcurveto{\pgfqpoint{0.885231in}{1.352393in}}{\pgfqpoint{0.877331in}{1.349121in}}{\pgfqpoint{0.871507in}{1.343297in}}%
\pgfpathcurveto{\pgfqpoint{0.865683in}{1.337473in}}{\pgfqpoint{0.862411in}{1.329573in}}{\pgfqpoint{0.862411in}{1.321336in}}%
\pgfpathcurveto{\pgfqpoint{0.862411in}{1.313100in}}{\pgfqpoint{0.865683in}{1.305200in}}{\pgfqpoint{0.871507in}{1.299376in}}%
\pgfpathcurveto{\pgfqpoint{0.877331in}{1.293552in}}{\pgfqpoint{0.885231in}{1.290280in}}{\pgfqpoint{0.893467in}{1.290280in}}%
\pgfpathclose%
\pgfusepath{stroke,fill}%
\end{pgfscope}%
\begin{pgfscope}%
\pgfpathrectangle{\pgfqpoint{0.100000in}{0.212622in}}{\pgfqpoint{3.696000in}{3.696000in}}%
\pgfusepath{clip}%
\pgfsetbuttcap%
\pgfsetroundjoin%
\definecolor{currentfill}{rgb}{0.121569,0.466667,0.705882}%
\pgfsetfillcolor{currentfill}%
\pgfsetfillopacity{0.623181}%
\pgfsetlinewidth{1.003750pt}%
\definecolor{currentstroke}{rgb}{0.121569,0.466667,0.705882}%
\pgfsetstrokecolor{currentstroke}%
\pgfsetstrokeopacity{0.623181}%
\pgfsetdash{}{0pt}%
\pgfpathmoveto{\pgfqpoint{0.893467in}{1.290280in}}%
\pgfpathcurveto{\pgfqpoint{0.901703in}{1.290280in}}{\pgfqpoint{0.909603in}{1.293552in}}{\pgfqpoint{0.915427in}{1.299376in}}%
\pgfpathcurveto{\pgfqpoint{0.921251in}{1.305200in}}{\pgfqpoint{0.924523in}{1.313100in}}{\pgfqpoint{0.924523in}{1.321336in}}%
\pgfpathcurveto{\pgfqpoint{0.924523in}{1.329573in}}{\pgfqpoint{0.921251in}{1.337473in}}{\pgfqpoint{0.915427in}{1.343297in}}%
\pgfpathcurveto{\pgfqpoint{0.909603in}{1.349120in}}{\pgfqpoint{0.901703in}{1.352393in}}{\pgfqpoint{0.893467in}{1.352393in}}%
\pgfpathcurveto{\pgfqpoint{0.885231in}{1.352393in}}{\pgfqpoint{0.877331in}{1.349120in}}{\pgfqpoint{0.871507in}{1.343297in}}%
\pgfpathcurveto{\pgfqpoint{0.865683in}{1.337473in}}{\pgfqpoint{0.862410in}{1.329573in}}{\pgfqpoint{0.862410in}{1.321336in}}%
\pgfpathcurveto{\pgfqpoint{0.862410in}{1.313100in}}{\pgfqpoint{0.865683in}{1.305200in}}{\pgfqpoint{0.871507in}{1.299376in}}%
\pgfpathcurveto{\pgfqpoint{0.877331in}{1.293552in}}{\pgfqpoint{0.885231in}{1.290280in}}{\pgfqpoint{0.893467in}{1.290280in}}%
\pgfpathclose%
\pgfusepath{stroke,fill}%
\end{pgfscope}%
\begin{pgfscope}%
\pgfpathrectangle{\pgfqpoint{0.100000in}{0.212622in}}{\pgfqpoint{3.696000in}{3.696000in}}%
\pgfusepath{clip}%
\pgfsetbuttcap%
\pgfsetroundjoin%
\definecolor{currentfill}{rgb}{0.121569,0.466667,0.705882}%
\pgfsetfillcolor{currentfill}%
\pgfsetfillopacity{0.623181}%
\pgfsetlinewidth{1.003750pt}%
\definecolor{currentstroke}{rgb}{0.121569,0.466667,0.705882}%
\pgfsetstrokecolor{currentstroke}%
\pgfsetstrokeopacity{0.623181}%
\pgfsetdash{}{0pt}%
\pgfpathmoveto{\pgfqpoint{0.893466in}{1.290280in}}%
\pgfpathcurveto{\pgfqpoint{0.901703in}{1.290280in}}{\pgfqpoint{0.909603in}{1.293552in}}{\pgfqpoint{0.915427in}{1.299376in}}%
\pgfpathcurveto{\pgfqpoint{0.921250in}{1.305200in}}{\pgfqpoint{0.924523in}{1.313100in}}{\pgfqpoint{0.924523in}{1.321336in}}%
\pgfpathcurveto{\pgfqpoint{0.924523in}{1.329572in}}{\pgfqpoint{0.921250in}{1.337472in}}{\pgfqpoint{0.915427in}{1.343296in}}%
\pgfpathcurveto{\pgfqpoint{0.909603in}{1.349120in}}{\pgfqpoint{0.901703in}{1.352393in}}{\pgfqpoint{0.893466in}{1.352393in}}%
\pgfpathcurveto{\pgfqpoint{0.885230in}{1.352393in}}{\pgfqpoint{0.877330in}{1.349120in}}{\pgfqpoint{0.871506in}{1.343296in}}%
\pgfpathcurveto{\pgfqpoint{0.865682in}{1.337472in}}{\pgfqpoint{0.862410in}{1.329572in}}{\pgfqpoint{0.862410in}{1.321336in}}%
\pgfpathcurveto{\pgfqpoint{0.862410in}{1.313100in}}{\pgfqpoint{0.865682in}{1.305200in}}{\pgfqpoint{0.871506in}{1.299376in}}%
\pgfpathcurveto{\pgfqpoint{0.877330in}{1.293552in}}{\pgfqpoint{0.885230in}{1.290280in}}{\pgfqpoint{0.893466in}{1.290280in}}%
\pgfpathclose%
\pgfusepath{stroke,fill}%
\end{pgfscope}%
\begin{pgfscope}%
\pgfpathrectangle{\pgfqpoint{0.100000in}{0.212622in}}{\pgfqpoint{3.696000in}{3.696000in}}%
\pgfusepath{clip}%
\pgfsetbuttcap%
\pgfsetroundjoin%
\definecolor{currentfill}{rgb}{0.121569,0.466667,0.705882}%
\pgfsetfillcolor{currentfill}%
\pgfsetfillopacity{0.623181}%
\pgfsetlinewidth{1.003750pt}%
\definecolor{currentstroke}{rgb}{0.121569,0.466667,0.705882}%
\pgfsetstrokecolor{currentstroke}%
\pgfsetstrokeopacity{0.623181}%
\pgfsetdash{}{0pt}%
\pgfpathmoveto{\pgfqpoint{0.893465in}{1.290279in}}%
\pgfpathcurveto{\pgfqpoint{0.901702in}{1.290279in}}{\pgfqpoint{0.909602in}{1.293551in}}{\pgfqpoint{0.915426in}{1.299375in}}%
\pgfpathcurveto{\pgfqpoint{0.921249in}{1.305199in}}{\pgfqpoint{0.924522in}{1.313099in}}{\pgfqpoint{0.924522in}{1.321335in}}%
\pgfpathcurveto{\pgfqpoint{0.924522in}{1.329572in}}{\pgfqpoint{0.921249in}{1.337472in}}{\pgfqpoint{0.915426in}{1.343296in}}%
\pgfpathcurveto{\pgfqpoint{0.909602in}{1.349120in}}{\pgfqpoint{0.901702in}{1.352392in}}{\pgfqpoint{0.893465in}{1.352392in}}%
\pgfpathcurveto{\pgfqpoint{0.885229in}{1.352392in}}{\pgfqpoint{0.877329in}{1.349120in}}{\pgfqpoint{0.871505in}{1.343296in}}%
\pgfpathcurveto{\pgfqpoint{0.865681in}{1.337472in}}{\pgfqpoint{0.862409in}{1.329572in}}{\pgfqpoint{0.862409in}{1.321335in}}%
\pgfpathcurveto{\pgfqpoint{0.862409in}{1.313099in}}{\pgfqpoint{0.865681in}{1.305199in}}{\pgfqpoint{0.871505in}{1.299375in}}%
\pgfpathcurveto{\pgfqpoint{0.877329in}{1.293551in}}{\pgfqpoint{0.885229in}{1.290279in}}{\pgfqpoint{0.893465in}{1.290279in}}%
\pgfpathclose%
\pgfusepath{stroke,fill}%
\end{pgfscope}%
\begin{pgfscope}%
\pgfpathrectangle{\pgfqpoint{0.100000in}{0.212622in}}{\pgfqpoint{3.696000in}{3.696000in}}%
\pgfusepath{clip}%
\pgfsetbuttcap%
\pgfsetroundjoin%
\definecolor{currentfill}{rgb}{0.121569,0.466667,0.705882}%
\pgfsetfillcolor{currentfill}%
\pgfsetfillopacity{0.623182}%
\pgfsetlinewidth{1.003750pt}%
\definecolor{currentstroke}{rgb}{0.121569,0.466667,0.705882}%
\pgfsetstrokecolor{currentstroke}%
\pgfsetstrokeopacity{0.623182}%
\pgfsetdash{}{0pt}%
\pgfpathmoveto{\pgfqpoint{0.893463in}{1.290277in}}%
\pgfpathcurveto{\pgfqpoint{0.901700in}{1.290277in}}{\pgfqpoint{0.909600in}{1.293550in}}{\pgfqpoint{0.915424in}{1.299374in}}%
\pgfpathcurveto{\pgfqpoint{0.921247in}{1.305198in}}{\pgfqpoint{0.924520in}{1.313098in}}{\pgfqpoint{0.924520in}{1.321334in}}%
\pgfpathcurveto{\pgfqpoint{0.924520in}{1.329570in}}{\pgfqpoint{0.921247in}{1.337470in}}{\pgfqpoint{0.915424in}{1.343294in}}%
\pgfpathcurveto{\pgfqpoint{0.909600in}{1.349118in}}{\pgfqpoint{0.901700in}{1.352390in}}{\pgfqpoint{0.893463in}{1.352390in}}%
\pgfpathcurveto{\pgfqpoint{0.885227in}{1.352390in}}{\pgfqpoint{0.877327in}{1.349118in}}{\pgfqpoint{0.871503in}{1.343294in}}%
\pgfpathcurveto{\pgfqpoint{0.865679in}{1.337470in}}{\pgfqpoint{0.862407in}{1.329570in}}{\pgfqpoint{0.862407in}{1.321334in}}%
\pgfpathcurveto{\pgfqpoint{0.862407in}{1.313098in}}{\pgfqpoint{0.865679in}{1.305198in}}{\pgfqpoint{0.871503in}{1.299374in}}%
\pgfpathcurveto{\pgfqpoint{0.877327in}{1.293550in}}{\pgfqpoint{0.885227in}{1.290277in}}{\pgfqpoint{0.893463in}{1.290277in}}%
\pgfpathclose%
\pgfusepath{stroke,fill}%
\end{pgfscope}%
\begin{pgfscope}%
\pgfpathrectangle{\pgfqpoint{0.100000in}{0.212622in}}{\pgfqpoint{3.696000in}{3.696000in}}%
\pgfusepath{clip}%
\pgfsetbuttcap%
\pgfsetroundjoin%
\definecolor{currentfill}{rgb}{0.121569,0.466667,0.705882}%
\pgfsetfillcolor{currentfill}%
\pgfsetfillopacity{0.623183}%
\pgfsetlinewidth{1.003750pt}%
\definecolor{currentstroke}{rgb}{0.121569,0.466667,0.705882}%
\pgfsetstrokecolor{currentstroke}%
\pgfsetstrokeopacity{0.623183}%
\pgfsetdash{}{0pt}%
\pgfpathmoveto{\pgfqpoint{0.893460in}{1.290275in}}%
\pgfpathcurveto{\pgfqpoint{0.901696in}{1.290275in}}{\pgfqpoint{0.909596in}{1.293547in}}{\pgfqpoint{0.915420in}{1.299371in}}%
\pgfpathcurveto{\pgfqpoint{0.921244in}{1.305195in}}{\pgfqpoint{0.924516in}{1.313095in}}{\pgfqpoint{0.924516in}{1.321331in}}%
\pgfpathcurveto{\pgfqpoint{0.924516in}{1.329567in}}{\pgfqpoint{0.921244in}{1.337468in}}{\pgfqpoint{0.915420in}{1.343291in}}%
\pgfpathcurveto{\pgfqpoint{0.909596in}{1.349115in}}{\pgfqpoint{0.901696in}{1.352388in}}{\pgfqpoint{0.893460in}{1.352388in}}%
\pgfpathcurveto{\pgfqpoint{0.885224in}{1.352388in}}{\pgfqpoint{0.877324in}{1.349115in}}{\pgfqpoint{0.871500in}{1.343291in}}%
\pgfpathcurveto{\pgfqpoint{0.865676in}{1.337468in}}{\pgfqpoint{0.862403in}{1.329567in}}{\pgfqpoint{0.862403in}{1.321331in}}%
\pgfpathcurveto{\pgfqpoint{0.862403in}{1.313095in}}{\pgfqpoint{0.865676in}{1.305195in}}{\pgfqpoint{0.871500in}{1.299371in}}%
\pgfpathcurveto{\pgfqpoint{0.877324in}{1.293547in}}{\pgfqpoint{0.885224in}{1.290275in}}{\pgfqpoint{0.893460in}{1.290275in}}%
\pgfpathclose%
\pgfusepath{stroke,fill}%
\end{pgfscope}%
\begin{pgfscope}%
\pgfpathrectangle{\pgfqpoint{0.100000in}{0.212622in}}{\pgfqpoint{3.696000in}{3.696000in}}%
\pgfusepath{clip}%
\pgfsetbuttcap%
\pgfsetroundjoin%
\definecolor{currentfill}{rgb}{0.121569,0.466667,0.705882}%
\pgfsetfillcolor{currentfill}%
\pgfsetfillopacity{0.623185}%
\pgfsetlinewidth{1.003750pt}%
\definecolor{currentstroke}{rgb}{0.121569,0.466667,0.705882}%
\pgfsetstrokecolor{currentstroke}%
\pgfsetstrokeopacity{0.623185}%
\pgfsetdash{}{0pt}%
\pgfpathmoveto{\pgfqpoint{0.893453in}{1.290271in}}%
\pgfpathcurveto{\pgfqpoint{0.901690in}{1.290271in}}{\pgfqpoint{0.909590in}{1.293544in}}{\pgfqpoint{0.915413in}{1.299368in}}%
\pgfpathcurveto{\pgfqpoint{0.921237in}{1.305192in}}{\pgfqpoint{0.924510in}{1.313092in}}{\pgfqpoint{0.924510in}{1.321328in}}%
\pgfpathcurveto{\pgfqpoint{0.924510in}{1.329564in}}{\pgfqpoint{0.921237in}{1.337464in}}{\pgfqpoint{0.915413in}{1.343288in}}%
\pgfpathcurveto{\pgfqpoint{0.909590in}{1.349112in}}{\pgfqpoint{0.901690in}{1.352384in}}{\pgfqpoint{0.893453in}{1.352384in}}%
\pgfpathcurveto{\pgfqpoint{0.885217in}{1.352384in}}{\pgfqpoint{0.877317in}{1.349112in}}{\pgfqpoint{0.871493in}{1.343288in}}%
\pgfpathcurveto{\pgfqpoint{0.865669in}{1.337464in}}{\pgfqpoint{0.862397in}{1.329564in}}{\pgfqpoint{0.862397in}{1.321328in}}%
\pgfpathcurveto{\pgfqpoint{0.862397in}{1.313092in}}{\pgfqpoint{0.865669in}{1.305192in}}{\pgfqpoint{0.871493in}{1.299368in}}%
\pgfpathcurveto{\pgfqpoint{0.877317in}{1.293544in}}{\pgfqpoint{0.885217in}{1.290271in}}{\pgfqpoint{0.893453in}{1.290271in}}%
\pgfpathclose%
\pgfusepath{stroke,fill}%
\end{pgfscope}%
\begin{pgfscope}%
\pgfpathrectangle{\pgfqpoint{0.100000in}{0.212622in}}{\pgfqpoint{3.696000in}{3.696000in}}%
\pgfusepath{clip}%
\pgfsetbuttcap%
\pgfsetroundjoin%
\definecolor{currentfill}{rgb}{0.121569,0.466667,0.705882}%
\pgfsetfillcolor{currentfill}%
\pgfsetfillopacity{0.623189}%
\pgfsetlinewidth{1.003750pt}%
\definecolor{currentstroke}{rgb}{0.121569,0.466667,0.705882}%
\pgfsetstrokecolor{currentstroke}%
\pgfsetstrokeopacity{0.623189}%
\pgfsetdash{}{0pt}%
\pgfpathmoveto{\pgfqpoint{0.893442in}{1.290263in}}%
\pgfpathcurveto{\pgfqpoint{0.901678in}{1.290263in}}{\pgfqpoint{0.909578in}{1.293535in}}{\pgfqpoint{0.915402in}{1.299359in}}%
\pgfpathcurveto{\pgfqpoint{0.921226in}{1.305183in}}{\pgfqpoint{0.924498in}{1.313083in}}{\pgfqpoint{0.924498in}{1.321319in}}%
\pgfpathcurveto{\pgfqpoint{0.924498in}{1.329556in}}{\pgfqpoint{0.921226in}{1.337456in}}{\pgfqpoint{0.915402in}{1.343280in}}%
\pgfpathcurveto{\pgfqpoint{0.909578in}{1.349104in}}{\pgfqpoint{0.901678in}{1.352376in}}{\pgfqpoint{0.893442in}{1.352376in}}%
\pgfpathcurveto{\pgfqpoint{0.885205in}{1.352376in}}{\pgfqpoint{0.877305in}{1.349104in}}{\pgfqpoint{0.871481in}{1.343280in}}%
\pgfpathcurveto{\pgfqpoint{0.865657in}{1.337456in}}{\pgfqpoint{0.862385in}{1.329556in}}{\pgfqpoint{0.862385in}{1.321319in}}%
\pgfpathcurveto{\pgfqpoint{0.862385in}{1.313083in}}{\pgfqpoint{0.865657in}{1.305183in}}{\pgfqpoint{0.871481in}{1.299359in}}%
\pgfpathcurveto{\pgfqpoint{0.877305in}{1.293535in}}{\pgfqpoint{0.885205in}{1.290263in}}{\pgfqpoint{0.893442in}{1.290263in}}%
\pgfpathclose%
\pgfusepath{stroke,fill}%
\end{pgfscope}%
\begin{pgfscope}%
\pgfpathrectangle{\pgfqpoint{0.100000in}{0.212622in}}{\pgfqpoint{3.696000in}{3.696000in}}%
\pgfusepath{clip}%
\pgfsetbuttcap%
\pgfsetroundjoin%
\definecolor{currentfill}{rgb}{0.121569,0.466667,0.705882}%
\pgfsetfillcolor{currentfill}%
\pgfsetfillopacity{0.623196}%
\pgfsetlinewidth{1.003750pt}%
\definecolor{currentstroke}{rgb}{0.121569,0.466667,0.705882}%
\pgfsetstrokecolor{currentstroke}%
\pgfsetstrokeopacity{0.623196}%
\pgfsetdash{}{0pt}%
\pgfpathmoveto{\pgfqpoint{0.893421in}{1.290252in}}%
\pgfpathcurveto{\pgfqpoint{0.901657in}{1.290252in}}{\pgfqpoint{0.909557in}{1.293524in}}{\pgfqpoint{0.915381in}{1.299348in}}%
\pgfpathcurveto{\pgfqpoint{0.921205in}{1.305172in}}{\pgfqpoint{0.924477in}{1.313072in}}{\pgfqpoint{0.924477in}{1.321308in}}%
\pgfpathcurveto{\pgfqpoint{0.924477in}{1.329544in}}{\pgfqpoint{0.921205in}{1.337444in}}{\pgfqpoint{0.915381in}{1.343268in}}%
\pgfpathcurveto{\pgfqpoint{0.909557in}{1.349092in}}{\pgfqpoint{0.901657in}{1.352365in}}{\pgfqpoint{0.893421in}{1.352365in}}%
\pgfpathcurveto{\pgfqpoint{0.885184in}{1.352365in}}{\pgfqpoint{0.877284in}{1.349092in}}{\pgfqpoint{0.871460in}{1.343268in}}%
\pgfpathcurveto{\pgfqpoint{0.865637in}{1.337444in}}{\pgfqpoint{0.862364in}{1.329544in}}{\pgfqpoint{0.862364in}{1.321308in}}%
\pgfpathcurveto{\pgfqpoint{0.862364in}{1.313072in}}{\pgfqpoint{0.865637in}{1.305172in}}{\pgfqpoint{0.871460in}{1.299348in}}%
\pgfpathcurveto{\pgfqpoint{0.877284in}{1.293524in}}{\pgfqpoint{0.885184in}{1.290252in}}{\pgfqpoint{0.893421in}{1.290252in}}%
\pgfpathclose%
\pgfusepath{stroke,fill}%
\end{pgfscope}%
\begin{pgfscope}%
\pgfpathrectangle{\pgfqpoint{0.100000in}{0.212622in}}{\pgfqpoint{3.696000in}{3.696000in}}%
\pgfusepath{clip}%
\pgfsetbuttcap%
\pgfsetroundjoin%
\definecolor{currentfill}{rgb}{0.121569,0.466667,0.705882}%
\pgfsetfillcolor{currentfill}%
\pgfsetfillopacity{0.623208}%
\pgfsetlinewidth{1.003750pt}%
\definecolor{currentstroke}{rgb}{0.121569,0.466667,0.705882}%
\pgfsetstrokecolor{currentstroke}%
\pgfsetstrokeopacity{0.623208}%
\pgfsetdash{}{0pt}%
\pgfpathmoveto{\pgfqpoint{0.893379in}{1.290234in}}%
\pgfpathcurveto{\pgfqpoint{0.901616in}{1.290234in}}{\pgfqpoint{0.909516in}{1.293506in}}{\pgfqpoint{0.915340in}{1.299330in}}%
\pgfpathcurveto{\pgfqpoint{0.921163in}{1.305154in}}{\pgfqpoint{0.924436in}{1.313054in}}{\pgfqpoint{0.924436in}{1.321290in}}%
\pgfpathcurveto{\pgfqpoint{0.924436in}{1.329526in}}{\pgfqpoint{0.921163in}{1.337426in}}{\pgfqpoint{0.915340in}{1.343250in}}%
\pgfpathcurveto{\pgfqpoint{0.909516in}{1.349074in}}{\pgfqpoint{0.901616in}{1.352347in}}{\pgfqpoint{0.893379in}{1.352347in}}%
\pgfpathcurveto{\pgfqpoint{0.885143in}{1.352347in}}{\pgfqpoint{0.877243in}{1.349074in}}{\pgfqpoint{0.871419in}{1.343250in}}%
\pgfpathcurveto{\pgfqpoint{0.865595in}{1.337426in}}{\pgfqpoint{0.862323in}{1.329526in}}{\pgfqpoint{0.862323in}{1.321290in}}%
\pgfpathcurveto{\pgfqpoint{0.862323in}{1.313054in}}{\pgfqpoint{0.865595in}{1.305154in}}{\pgfqpoint{0.871419in}{1.299330in}}%
\pgfpathcurveto{\pgfqpoint{0.877243in}{1.293506in}}{\pgfqpoint{0.885143in}{1.290234in}}{\pgfqpoint{0.893379in}{1.290234in}}%
\pgfpathclose%
\pgfusepath{stroke,fill}%
\end{pgfscope}%
\begin{pgfscope}%
\pgfpathrectangle{\pgfqpoint{0.100000in}{0.212622in}}{\pgfqpoint{3.696000in}{3.696000in}}%
\pgfusepath{clip}%
\pgfsetbuttcap%
\pgfsetroundjoin%
\definecolor{currentfill}{rgb}{0.121569,0.466667,0.705882}%
\pgfsetfillcolor{currentfill}%
\pgfsetfillopacity{0.623227}%
\pgfsetlinewidth{1.003750pt}%
\definecolor{currentstroke}{rgb}{0.121569,0.466667,0.705882}%
\pgfsetstrokecolor{currentstroke}%
\pgfsetstrokeopacity{0.623227}%
\pgfsetdash{}{0pt}%
\pgfpathmoveto{\pgfqpoint{0.893307in}{1.290172in}}%
\pgfpathcurveto{\pgfqpoint{0.901543in}{1.290172in}}{\pgfqpoint{0.909443in}{1.293445in}}{\pgfqpoint{0.915267in}{1.299269in}}%
\pgfpathcurveto{\pgfqpoint{0.921091in}{1.305092in}}{\pgfqpoint{0.924363in}{1.312993in}}{\pgfqpoint{0.924363in}{1.321229in}}%
\pgfpathcurveto{\pgfqpoint{0.924363in}{1.329465in}}{\pgfqpoint{0.921091in}{1.337365in}}{\pgfqpoint{0.915267in}{1.343189in}}%
\pgfpathcurveto{\pgfqpoint{0.909443in}{1.349013in}}{\pgfqpoint{0.901543in}{1.352285in}}{\pgfqpoint{0.893307in}{1.352285in}}%
\pgfpathcurveto{\pgfqpoint{0.885070in}{1.352285in}}{\pgfqpoint{0.877170in}{1.349013in}}{\pgfqpoint{0.871346in}{1.343189in}}%
\pgfpathcurveto{\pgfqpoint{0.865522in}{1.337365in}}{\pgfqpoint{0.862250in}{1.329465in}}{\pgfqpoint{0.862250in}{1.321229in}}%
\pgfpathcurveto{\pgfqpoint{0.862250in}{1.312993in}}{\pgfqpoint{0.865522in}{1.305092in}}{\pgfqpoint{0.871346in}{1.299269in}}%
\pgfpathcurveto{\pgfqpoint{0.877170in}{1.293445in}}{\pgfqpoint{0.885070in}{1.290172in}}{\pgfqpoint{0.893307in}{1.290172in}}%
\pgfpathclose%
\pgfusepath{stroke,fill}%
\end{pgfscope}%
\begin{pgfscope}%
\pgfpathrectangle{\pgfqpoint{0.100000in}{0.212622in}}{\pgfqpoint{3.696000in}{3.696000in}}%
\pgfusepath{clip}%
\pgfsetbuttcap%
\pgfsetroundjoin%
\definecolor{currentfill}{rgb}{0.121569,0.466667,0.705882}%
\pgfsetfillcolor{currentfill}%
\pgfsetfillopacity{0.623261}%
\pgfsetlinewidth{1.003750pt}%
\definecolor{currentstroke}{rgb}{0.121569,0.466667,0.705882}%
\pgfsetstrokecolor{currentstroke}%
\pgfsetstrokeopacity{0.623261}%
\pgfsetdash{}{0pt}%
\pgfpathmoveto{\pgfqpoint{0.911548in}{1.255141in}}%
\pgfpathcurveto{\pgfqpoint{0.919785in}{1.255141in}}{\pgfqpoint{0.927685in}{1.258413in}}{\pgfqpoint{0.933509in}{1.264237in}}%
\pgfpathcurveto{\pgfqpoint{0.939333in}{1.270061in}}{\pgfqpoint{0.942605in}{1.277961in}}{\pgfqpoint{0.942605in}{1.286197in}}%
\pgfpathcurveto{\pgfqpoint{0.942605in}{1.294434in}}{\pgfqpoint{0.939333in}{1.302334in}}{\pgfqpoint{0.933509in}{1.308158in}}%
\pgfpathcurveto{\pgfqpoint{0.927685in}{1.313982in}}{\pgfqpoint{0.919785in}{1.317254in}}{\pgfqpoint{0.911548in}{1.317254in}}%
\pgfpathcurveto{\pgfqpoint{0.903312in}{1.317254in}}{\pgfqpoint{0.895412in}{1.313982in}}{\pgfqpoint{0.889588in}{1.308158in}}%
\pgfpathcurveto{\pgfqpoint{0.883764in}{1.302334in}}{\pgfqpoint{0.880492in}{1.294434in}}{\pgfqpoint{0.880492in}{1.286197in}}%
\pgfpathcurveto{\pgfqpoint{0.880492in}{1.277961in}}{\pgfqpoint{0.883764in}{1.270061in}}{\pgfqpoint{0.889588in}{1.264237in}}%
\pgfpathcurveto{\pgfqpoint{0.895412in}{1.258413in}}{\pgfqpoint{0.903312in}{1.255141in}}{\pgfqpoint{0.911548in}{1.255141in}}%
\pgfpathclose%
\pgfusepath{stroke,fill}%
\end{pgfscope}%
\begin{pgfscope}%
\pgfpathrectangle{\pgfqpoint{0.100000in}{0.212622in}}{\pgfqpoint{3.696000in}{3.696000in}}%
\pgfusepath{clip}%
\pgfsetbuttcap%
\pgfsetroundjoin%
\definecolor{currentfill}{rgb}{0.121569,0.466667,0.705882}%
\pgfsetfillcolor{currentfill}%
\pgfsetfillopacity{0.623269}%
\pgfsetlinewidth{1.003750pt}%
\definecolor{currentstroke}{rgb}{0.121569,0.466667,0.705882}%
\pgfsetstrokecolor{currentstroke}%
\pgfsetstrokeopacity{0.623269}%
\pgfsetdash{}{0pt}%
\pgfpathmoveto{\pgfqpoint{0.893181in}{1.290095in}}%
\pgfpathcurveto{\pgfqpoint{0.901417in}{1.290095in}}{\pgfqpoint{0.909318in}{1.293367in}}{\pgfqpoint{0.915141in}{1.299191in}}%
\pgfpathcurveto{\pgfqpoint{0.920965in}{1.305015in}}{\pgfqpoint{0.924238in}{1.312915in}}{\pgfqpoint{0.924238in}{1.321152in}}%
\pgfpathcurveto{\pgfqpoint{0.924238in}{1.329388in}}{\pgfqpoint{0.920965in}{1.337288in}}{\pgfqpoint{0.915141in}{1.343112in}}%
\pgfpathcurveto{\pgfqpoint{0.909318in}{1.348936in}}{\pgfqpoint{0.901417in}{1.352208in}}{\pgfqpoint{0.893181in}{1.352208in}}%
\pgfpathcurveto{\pgfqpoint{0.884945in}{1.352208in}}{\pgfqpoint{0.877045in}{1.348936in}}{\pgfqpoint{0.871221in}{1.343112in}}%
\pgfpathcurveto{\pgfqpoint{0.865397in}{1.337288in}}{\pgfqpoint{0.862125in}{1.329388in}}{\pgfqpoint{0.862125in}{1.321152in}}%
\pgfpathcurveto{\pgfqpoint{0.862125in}{1.312915in}}{\pgfqpoint{0.865397in}{1.305015in}}{\pgfqpoint{0.871221in}{1.299191in}}%
\pgfpathcurveto{\pgfqpoint{0.877045in}{1.293367in}}{\pgfqpoint{0.884945in}{1.290095in}}{\pgfqpoint{0.893181in}{1.290095in}}%
\pgfpathclose%
\pgfusepath{stroke,fill}%
\end{pgfscope}%
\begin{pgfscope}%
\pgfpathrectangle{\pgfqpoint{0.100000in}{0.212622in}}{\pgfqpoint{3.696000in}{3.696000in}}%
\pgfusepath{clip}%
\pgfsetbuttcap%
\pgfsetroundjoin%
\definecolor{currentfill}{rgb}{0.121569,0.466667,0.705882}%
\pgfsetfillcolor{currentfill}%
\pgfsetfillopacity{0.623340}%
\pgfsetlinewidth{1.003750pt}%
\definecolor{currentstroke}{rgb}{0.121569,0.466667,0.705882}%
\pgfsetstrokecolor{currentstroke}%
\pgfsetstrokeopacity{0.623340}%
\pgfsetdash{}{0pt}%
\pgfpathmoveto{\pgfqpoint{0.892947in}{1.289933in}}%
\pgfpathcurveto{\pgfqpoint{0.901183in}{1.289933in}}{\pgfqpoint{0.909083in}{1.293205in}}{\pgfqpoint{0.914907in}{1.299029in}}%
\pgfpathcurveto{\pgfqpoint{0.920731in}{1.304853in}}{\pgfqpoint{0.924003in}{1.312753in}}{\pgfqpoint{0.924003in}{1.320989in}}%
\pgfpathcurveto{\pgfqpoint{0.924003in}{1.329226in}}{\pgfqpoint{0.920731in}{1.337126in}}{\pgfqpoint{0.914907in}{1.342950in}}%
\pgfpathcurveto{\pgfqpoint{0.909083in}{1.348773in}}{\pgfqpoint{0.901183in}{1.352046in}}{\pgfqpoint{0.892947in}{1.352046in}}%
\pgfpathcurveto{\pgfqpoint{0.884710in}{1.352046in}}{\pgfqpoint{0.876810in}{1.348773in}}{\pgfqpoint{0.870986in}{1.342950in}}%
\pgfpathcurveto{\pgfqpoint{0.865162in}{1.337126in}}{\pgfqpoint{0.861890in}{1.329226in}}{\pgfqpoint{0.861890in}{1.320989in}}%
\pgfpathcurveto{\pgfqpoint{0.861890in}{1.312753in}}{\pgfqpoint{0.865162in}{1.304853in}}{\pgfqpoint{0.870986in}{1.299029in}}%
\pgfpathcurveto{\pgfqpoint{0.876810in}{1.293205in}}{\pgfqpoint{0.884710in}{1.289933in}}{\pgfqpoint{0.892947in}{1.289933in}}%
\pgfpathclose%
\pgfusepath{stroke,fill}%
\end{pgfscope}%
\begin{pgfscope}%
\pgfpathrectangle{\pgfqpoint{0.100000in}{0.212622in}}{\pgfqpoint{3.696000in}{3.696000in}}%
\pgfusepath{clip}%
\pgfsetbuttcap%
\pgfsetroundjoin%
\definecolor{currentfill}{rgb}{0.121569,0.466667,0.705882}%
\pgfsetfillcolor{currentfill}%
\pgfsetfillopacity{0.623479}%
\pgfsetlinewidth{1.003750pt}%
\definecolor{currentstroke}{rgb}{0.121569,0.466667,0.705882}%
\pgfsetstrokecolor{currentstroke}%
\pgfsetstrokeopacity{0.623479}%
\pgfsetdash{}{0pt}%
\pgfpathmoveto{\pgfqpoint{0.892543in}{1.289670in}}%
\pgfpathcurveto{\pgfqpoint{0.900780in}{1.289670in}}{\pgfqpoint{0.908680in}{1.292942in}}{\pgfqpoint{0.914504in}{1.298766in}}%
\pgfpathcurveto{\pgfqpoint{0.920327in}{1.304590in}}{\pgfqpoint{0.923600in}{1.312490in}}{\pgfqpoint{0.923600in}{1.320726in}}%
\pgfpathcurveto{\pgfqpoint{0.923600in}{1.328963in}}{\pgfqpoint{0.920327in}{1.336863in}}{\pgfqpoint{0.914504in}{1.342687in}}%
\pgfpathcurveto{\pgfqpoint{0.908680in}{1.348510in}}{\pgfqpoint{0.900780in}{1.351783in}}{\pgfqpoint{0.892543in}{1.351783in}}%
\pgfpathcurveto{\pgfqpoint{0.884307in}{1.351783in}}{\pgfqpoint{0.876407in}{1.348510in}}{\pgfqpoint{0.870583in}{1.342687in}}%
\pgfpathcurveto{\pgfqpoint{0.864759in}{1.336863in}}{\pgfqpoint{0.861487in}{1.328963in}}{\pgfqpoint{0.861487in}{1.320726in}}%
\pgfpathcurveto{\pgfqpoint{0.861487in}{1.312490in}}{\pgfqpoint{0.864759in}{1.304590in}}{\pgfqpoint{0.870583in}{1.298766in}}%
\pgfpathcurveto{\pgfqpoint{0.876407in}{1.292942in}}{\pgfqpoint{0.884307in}{1.289670in}}{\pgfqpoint{0.892543in}{1.289670in}}%
\pgfpathclose%
\pgfusepath{stroke,fill}%
\end{pgfscope}%
\begin{pgfscope}%
\pgfpathrectangle{\pgfqpoint{0.100000in}{0.212622in}}{\pgfqpoint{3.696000in}{3.696000in}}%
\pgfusepath{clip}%
\pgfsetbuttcap%
\pgfsetroundjoin%
\definecolor{currentfill}{rgb}{0.121569,0.466667,0.705882}%
\pgfsetfillcolor{currentfill}%
\pgfsetfillopacity{0.623710}%
\pgfsetlinewidth{1.003750pt}%
\definecolor{currentstroke}{rgb}{0.121569,0.466667,0.705882}%
\pgfsetstrokecolor{currentstroke}%
\pgfsetstrokeopacity{0.623710}%
\pgfsetdash{}{0pt}%
\pgfpathmoveto{\pgfqpoint{0.891797in}{1.289073in}}%
\pgfpathcurveto{\pgfqpoint{0.900033in}{1.289073in}}{\pgfqpoint{0.907933in}{1.292346in}}{\pgfqpoint{0.913757in}{1.298169in}}%
\pgfpathcurveto{\pgfqpoint{0.919581in}{1.303993in}}{\pgfqpoint{0.922853in}{1.311893in}}{\pgfqpoint{0.922853in}{1.320130in}}%
\pgfpathcurveto{\pgfqpoint{0.922853in}{1.328366in}}{\pgfqpoint{0.919581in}{1.336266in}}{\pgfqpoint{0.913757in}{1.342090in}}%
\pgfpathcurveto{\pgfqpoint{0.907933in}{1.347914in}}{\pgfqpoint{0.900033in}{1.351186in}}{\pgfqpoint{0.891797in}{1.351186in}}%
\pgfpathcurveto{\pgfqpoint{0.883560in}{1.351186in}}{\pgfqpoint{0.875660in}{1.347914in}}{\pgfqpoint{0.869836in}{1.342090in}}%
\pgfpathcurveto{\pgfqpoint{0.864012in}{1.336266in}}{\pgfqpoint{0.860740in}{1.328366in}}{\pgfqpoint{0.860740in}{1.320130in}}%
\pgfpathcurveto{\pgfqpoint{0.860740in}{1.311893in}}{\pgfqpoint{0.864012in}{1.303993in}}{\pgfqpoint{0.869836in}{1.298169in}}%
\pgfpathcurveto{\pgfqpoint{0.875660in}{1.292346in}}{\pgfqpoint{0.883560in}{1.289073in}}{\pgfqpoint{0.891797in}{1.289073in}}%
\pgfpathclose%
\pgfusepath{stroke,fill}%
\end{pgfscope}%
\begin{pgfscope}%
\pgfpathrectangle{\pgfqpoint{0.100000in}{0.212622in}}{\pgfqpoint{3.696000in}{3.696000in}}%
\pgfusepath{clip}%
\pgfsetbuttcap%
\pgfsetroundjoin%
\definecolor{currentfill}{rgb}{0.121569,0.466667,0.705882}%
\pgfsetfillcolor{currentfill}%
\pgfsetfillopacity{0.624155}%
\pgfsetlinewidth{1.003750pt}%
\definecolor{currentstroke}{rgb}{0.121569,0.466667,0.705882}%
\pgfsetstrokecolor{currentstroke}%
\pgfsetstrokeopacity{0.624155}%
\pgfsetdash{}{0pt}%
\pgfpathmoveto{\pgfqpoint{0.890353in}{1.288267in}}%
\pgfpathcurveto{\pgfqpoint{0.898589in}{1.288267in}}{\pgfqpoint{0.906489in}{1.291539in}}{\pgfqpoint{0.912313in}{1.297363in}}%
\pgfpathcurveto{\pgfqpoint{0.918137in}{1.303187in}}{\pgfqpoint{0.921409in}{1.311087in}}{\pgfqpoint{0.921409in}{1.319324in}}%
\pgfpathcurveto{\pgfqpoint{0.921409in}{1.327560in}}{\pgfqpoint{0.918137in}{1.335460in}}{\pgfqpoint{0.912313in}{1.341284in}}%
\pgfpathcurveto{\pgfqpoint{0.906489in}{1.347108in}}{\pgfqpoint{0.898589in}{1.350380in}}{\pgfqpoint{0.890353in}{1.350380in}}%
\pgfpathcurveto{\pgfqpoint{0.882117in}{1.350380in}}{\pgfqpoint{0.874217in}{1.347108in}}{\pgfqpoint{0.868393in}{1.341284in}}%
\pgfpathcurveto{\pgfqpoint{0.862569in}{1.335460in}}{\pgfqpoint{0.859296in}{1.327560in}}{\pgfqpoint{0.859296in}{1.319324in}}%
\pgfpathcurveto{\pgfqpoint{0.859296in}{1.311087in}}{\pgfqpoint{0.862569in}{1.303187in}}{\pgfqpoint{0.868393in}{1.297363in}}%
\pgfpathcurveto{\pgfqpoint{0.874217in}{1.291539in}}{\pgfqpoint{0.882117in}{1.288267in}}{\pgfqpoint{0.890353in}{1.288267in}}%
\pgfpathclose%
\pgfusepath{stroke,fill}%
\end{pgfscope}%
\begin{pgfscope}%
\pgfpathrectangle{\pgfqpoint{0.100000in}{0.212622in}}{\pgfqpoint{3.696000in}{3.696000in}}%
\pgfusepath{clip}%
\pgfsetbuttcap%
\pgfsetroundjoin%
\definecolor{currentfill}{rgb}{0.121569,0.466667,0.705882}%
\pgfsetfillcolor{currentfill}%
\pgfsetfillopacity{0.624436}%
\pgfsetlinewidth{1.003750pt}%
\definecolor{currentstroke}{rgb}{0.121569,0.466667,0.705882}%
\pgfsetstrokecolor{currentstroke}%
\pgfsetstrokeopacity{0.624436}%
\pgfsetdash{}{0pt}%
\pgfpathmoveto{\pgfqpoint{0.909677in}{1.255886in}}%
\pgfpathcurveto{\pgfqpoint{0.917913in}{1.255886in}}{\pgfqpoint{0.925813in}{1.259158in}}{\pgfqpoint{0.931637in}{1.264982in}}%
\pgfpathcurveto{\pgfqpoint{0.937461in}{1.270806in}}{\pgfqpoint{0.940733in}{1.278706in}}{\pgfqpoint{0.940733in}{1.286942in}}%
\pgfpathcurveto{\pgfqpoint{0.940733in}{1.295178in}}{\pgfqpoint{0.937461in}{1.303078in}}{\pgfqpoint{0.931637in}{1.308902in}}%
\pgfpathcurveto{\pgfqpoint{0.925813in}{1.314726in}}{\pgfqpoint{0.917913in}{1.317999in}}{\pgfqpoint{0.909677in}{1.317999in}}%
\pgfpathcurveto{\pgfqpoint{0.901441in}{1.317999in}}{\pgfqpoint{0.893541in}{1.314726in}}{\pgfqpoint{0.887717in}{1.308902in}}%
\pgfpathcurveto{\pgfqpoint{0.881893in}{1.303078in}}{\pgfqpoint{0.878620in}{1.295178in}}{\pgfqpoint{0.878620in}{1.286942in}}%
\pgfpathcurveto{\pgfqpoint{0.878620in}{1.278706in}}{\pgfqpoint{0.881893in}{1.270806in}}{\pgfqpoint{0.887717in}{1.264982in}}%
\pgfpathcurveto{\pgfqpoint{0.893541in}{1.259158in}}{\pgfqpoint{0.901441in}{1.255886in}}{\pgfqpoint{0.909677in}{1.255886in}}%
\pgfpathclose%
\pgfusepath{stroke,fill}%
\end{pgfscope}%
\begin{pgfscope}%
\pgfpathrectangle{\pgfqpoint{0.100000in}{0.212622in}}{\pgfqpoint{3.696000in}{3.696000in}}%
\pgfusepath{clip}%
\pgfsetbuttcap%
\pgfsetroundjoin%
\definecolor{currentfill}{rgb}{0.121569,0.466667,0.705882}%
\pgfsetfillcolor{currentfill}%
\pgfsetfillopacity{0.625023}%
\pgfsetlinewidth{1.003750pt}%
\definecolor{currentstroke}{rgb}{0.121569,0.466667,0.705882}%
\pgfsetstrokecolor{currentstroke}%
\pgfsetstrokeopacity{0.625023}%
\pgfsetdash{}{0pt}%
\pgfpathmoveto{\pgfqpoint{0.887698in}{1.287210in}}%
\pgfpathcurveto{\pgfqpoint{0.895934in}{1.287210in}}{\pgfqpoint{0.903834in}{1.290482in}}{\pgfqpoint{0.909658in}{1.296306in}}%
\pgfpathcurveto{\pgfqpoint{0.915482in}{1.302130in}}{\pgfqpoint{0.918755in}{1.310030in}}{\pgfqpoint{0.918755in}{1.318266in}}%
\pgfpathcurveto{\pgfqpoint{0.918755in}{1.326503in}}{\pgfqpoint{0.915482in}{1.334403in}}{\pgfqpoint{0.909658in}{1.340227in}}%
\pgfpathcurveto{\pgfqpoint{0.903834in}{1.346051in}}{\pgfqpoint{0.895934in}{1.349323in}}{\pgfqpoint{0.887698in}{1.349323in}}%
\pgfpathcurveto{\pgfqpoint{0.879462in}{1.349323in}}{\pgfqpoint{0.871562in}{1.346051in}}{\pgfqpoint{0.865738in}{1.340227in}}%
\pgfpathcurveto{\pgfqpoint{0.859914in}{1.334403in}}{\pgfqpoint{0.856642in}{1.326503in}}{\pgfqpoint{0.856642in}{1.318266in}}%
\pgfpathcurveto{\pgfqpoint{0.856642in}{1.310030in}}{\pgfqpoint{0.859914in}{1.302130in}}{\pgfqpoint{0.865738in}{1.296306in}}%
\pgfpathcurveto{\pgfqpoint{0.871562in}{1.290482in}}{\pgfqpoint{0.879462in}{1.287210in}}{\pgfqpoint{0.887698in}{1.287210in}}%
\pgfpathclose%
\pgfusepath{stroke,fill}%
\end{pgfscope}%
\begin{pgfscope}%
\pgfpathrectangle{\pgfqpoint{0.100000in}{0.212622in}}{\pgfqpoint{3.696000in}{3.696000in}}%
\pgfusepath{clip}%
\pgfsetbuttcap%
\pgfsetroundjoin%
\definecolor{currentfill}{rgb}{0.121569,0.466667,0.705882}%
\pgfsetfillcolor{currentfill}%
\pgfsetfillopacity{0.626006}%
\pgfsetlinewidth{1.003750pt}%
\definecolor{currentstroke}{rgb}{0.121569,0.466667,0.705882}%
\pgfsetstrokecolor{currentstroke}%
\pgfsetstrokeopacity{0.626006}%
\pgfsetdash{}{0pt}%
\pgfpathmoveto{\pgfqpoint{0.906892in}{1.257017in}}%
\pgfpathcurveto{\pgfqpoint{0.915128in}{1.257017in}}{\pgfqpoint{0.923028in}{1.260290in}}{\pgfqpoint{0.928852in}{1.266114in}}%
\pgfpathcurveto{\pgfqpoint{0.934676in}{1.271937in}}{\pgfqpoint{0.937948in}{1.279837in}}{\pgfqpoint{0.937948in}{1.288074in}}%
\pgfpathcurveto{\pgfqpoint{0.937948in}{1.296310in}}{\pgfqpoint{0.934676in}{1.304210in}}{\pgfqpoint{0.928852in}{1.310034in}}%
\pgfpathcurveto{\pgfqpoint{0.923028in}{1.315858in}}{\pgfqpoint{0.915128in}{1.319130in}}{\pgfqpoint{0.906892in}{1.319130in}}%
\pgfpathcurveto{\pgfqpoint{0.898656in}{1.319130in}}{\pgfqpoint{0.890756in}{1.315858in}}{\pgfqpoint{0.884932in}{1.310034in}}%
\pgfpathcurveto{\pgfqpoint{0.879108in}{1.304210in}}{\pgfqpoint{0.875835in}{1.296310in}}{\pgfqpoint{0.875835in}{1.288074in}}%
\pgfpathcurveto{\pgfqpoint{0.875835in}{1.279837in}}{\pgfqpoint{0.879108in}{1.271937in}}{\pgfqpoint{0.884932in}{1.266114in}}%
\pgfpathcurveto{\pgfqpoint{0.890756in}{1.260290in}}{\pgfqpoint{0.898656in}{1.257017in}}{\pgfqpoint{0.906892in}{1.257017in}}%
\pgfpathclose%
\pgfusepath{stroke,fill}%
\end{pgfscope}%
\begin{pgfscope}%
\pgfpathrectangle{\pgfqpoint{0.100000in}{0.212622in}}{\pgfqpoint{3.696000in}{3.696000in}}%
\pgfusepath{clip}%
\pgfsetbuttcap%
\pgfsetroundjoin%
\definecolor{currentfill}{rgb}{0.121569,0.466667,0.705882}%
\pgfsetfillcolor{currentfill}%
\pgfsetfillopacity{0.626273}%
\pgfsetlinewidth{1.003750pt}%
\definecolor{currentstroke}{rgb}{0.121569,0.466667,0.705882}%
\pgfsetstrokecolor{currentstroke}%
\pgfsetstrokeopacity{0.626273}%
\pgfsetdash{}{0pt}%
\pgfpathmoveto{\pgfqpoint{0.883044in}{1.283018in}}%
\pgfpathcurveto{\pgfqpoint{0.891280in}{1.283018in}}{\pgfqpoint{0.899180in}{1.286290in}}{\pgfqpoint{0.905004in}{1.292114in}}%
\pgfpathcurveto{\pgfqpoint{0.910828in}{1.297938in}}{\pgfqpoint{0.914100in}{1.305838in}}{\pgfqpoint{0.914100in}{1.314074in}}%
\pgfpathcurveto{\pgfqpoint{0.914100in}{1.322311in}}{\pgfqpoint{0.910828in}{1.330211in}}{\pgfqpoint{0.905004in}{1.336035in}}%
\pgfpathcurveto{\pgfqpoint{0.899180in}{1.341859in}}{\pgfqpoint{0.891280in}{1.345131in}}{\pgfqpoint{0.883044in}{1.345131in}}%
\pgfpathcurveto{\pgfqpoint{0.874807in}{1.345131in}}{\pgfqpoint{0.866907in}{1.341859in}}{\pgfqpoint{0.861083in}{1.336035in}}%
\pgfpathcurveto{\pgfqpoint{0.855260in}{1.330211in}}{\pgfqpoint{0.851987in}{1.322311in}}{\pgfqpoint{0.851987in}{1.314074in}}%
\pgfpathcurveto{\pgfqpoint{0.851987in}{1.305838in}}{\pgfqpoint{0.855260in}{1.297938in}}{\pgfqpoint{0.861083in}{1.292114in}}%
\pgfpathcurveto{\pgfqpoint{0.866907in}{1.286290in}}{\pgfqpoint{0.874807in}{1.283018in}}{\pgfqpoint{0.883044in}{1.283018in}}%
\pgfpathclose%
\pgfusepath{stroke,fill}%
\end{pgfscope}%
\begin{pgfscope}%
\pgfpathrectangle{\pgfqpoint{0.100000in}{0.212622in}}{\pgfqpoint{3.696000in}{3.696000in}}%
\pgfusepath{clip}%
\pgfsetbuttcap%
\pgfsetroundjoin%
\definecolor{currentfill}{rgb}{0.121569,0.466667,0.705882}%
\pgfsetfillcolor{currentfill}%
\pgfsetfillopacity{0.627595}%
\pgfsetlinewidth{1.003750pt}%
\definecolor{currentstroke}{rgb}{0.121569,0.466667,0.705882}%
\pgfsetstrokecolor{currentstroke}%
\pgfsetstrokeopacity{0.627595}%
\pgfsetdash{}{0pt}%
\pgfpathmoveto{\pgfqpoint{0.878826in}{1.281282in}}%
\pgfpathcurveto{\pgfqpoint{0.887063in}{1.281282in}}{\pgfqpoint{0.894963in}{1.284554in}}{\pgfqpoint{0.900787in}{1.290378in}}%
\pgfpathcurveto{\pgfqpoint{0.906610in}{1.296202in}}{\pgfqpoint{0.909883in}{1.304102in}}{\pgfqpoint{0.909883in}{1.312338in}}%
\pgfpathcurveto{\pgfqpoint{0.909883in}{1.320574in}}{\pgfqpoint{0.906610in}{1.328475in}}{\pgfqpoint{0.900787in}{1.334298in}}%
\pgfpathcurveto{\pgfqpoint{0.894963in}{1.340122in}}{\pgfqpoint{0.887063in}{1.343395in}}{\pgfqpoint{0.878826in}{1.343395in}}%
\pgfpathcurveto{\pgfqpoint{0.870590in}{1.343395in}}{\pgfqpoint{0.862690in}{1.340122in}}{\pgfqpoint{0.856866in}{1.334298in}}%
\pgfpathcurveto{\pgfqpoint{0.851042in}{1.328475in}}{\pgfqpoint{0.847770in}{1.320574in}}{\pgfqpoint{0.847770in}{1.312338in}}%
\pgfpathcurveto{\pgfqpoint{0.847770in}{1.304102in}}{\pgfqpoint{0.851042in}{1.296202in}}{\pgfqpoint{0.856866in}{1.290378in}}%
\pgfpathcurveto{\pgfqpoint{0.862690in}{1.284554in}}{\pgfqpoint{0.870590in}{1.281282in}}{\pgfqpoint{0.878826in}{1.281282in}}%
\pgfpathclose%
\pgfusepath{stroke,fill}%
\end{pgfscope}%
\begin{pgfscope}%
\pgfpathrectangle{\pgfqpoint{0.100000in}{0.212622in}}{\pgfqpoint{3.696000in}{3.696000in}}%
\pgfusepath{clip}%
\pgfsetbuttcap%
\pgfsetroundjoin%
\definecolor{currentfill}{rgb}{0.121569,0.466667,0.705882}%
\pgfsetfillcolor{currentfill}%
\pgfsetfillopacity{0.627638}%
\pgfsetlinewidth{1.003750pt}%
\definecolor{currentstroke}{rgb}{0.121569,0.466667,0.705882}%
\pgfsetstrokecolor{currentstroke}%
\pgfsetstrokeopacity{0.627638}%
\pgfsetdash{}{0pt}%
\pgfpathmoveto{\pgfqpoint{2.125770in}{1.796527in}}%
\pgfpathcurveto{\pgfqpoint{2.134006in}{1.796527in}}{\pgfqpoint{2.141906in}{1.799799in}}{\pgfqpoint{2.147730in}{1.805623in}}%
\pgfpathcurveto{\pgfqpoint{2.153554in}{1.811447in}}{\pgfqpoint{2.156826in}{1.819347in}}{\pgfqpoint{2.156826in}{1.827584in}}%
\pgfpathcurveto{\pgfqpoint{2.156826in}{1.835820in}}{\pgfqpoint{2.153554in}{1.843720in}}{\pgfqpoint{2.147730in}{1.849544in}}%
\pgfpathcurveto{\pgfqpoint{2.141906in}{1.855368in}}{\pgfqpoint{2.134006in}{1.858640in}}{\pgfqpoint{2.125770in}{1.858640in}}%
\pgfpathcurveto{\pgfqpoint{2.117534in}{1.858640in}}{\pgfqpoint{2.109634in}{1.855368in}}{\pgfqpoint{2.103810in}{1.849544in}}%
\pgfpathcurveto{\pgfqpoint{2.097986in}{1.843720in}}{\pgfqpoint{2.094713in}{1.835820in}}{\pgfqpoint{2.094713in}{1.827584in}}%
\pgfpathcurveto{\pgfqpoint{2.094713in}{1.819347in}}{\pgfqpoint{2.097986in}{1.811447in}}{\pgfqpoint{2.103810in}{1.805623in}}%
\pgfpathcurveto{\pgfqpoint{2.109634in}{1.799799in}}{\pgfqpoint{2.117534in}{1.796527in}}{\pgfqpoint{2.125770in}{1.796527in}}%
\pgfpathclose%
\pgfusepath{stroke,fill}%
\end{pgfscope}%
\begin{pgfscope}%
\pgfpathrectangle{\pgfqpoint{0.100000in}{0.212622in}}{\pgfqpoint{3.696000in}{3.696000in}}%
\pgfusepath{clip}%
\pgfsetbuttcap%
\pgfsetroundjoin%
\definecolor{currentfill}{rgb}{0.121569,0.466667,0.705882}%
\pgfsetfillcolor{currentfill}%
\pgfsetfillopacity{0.627920}%
\pgfsetlinewidth{1.003750pt}%
\definecolor{currentstroke}{rgb}{0.121569,0.466667,0.705882}%
\pgfsetstrokecolor{currentstroke}%
\pgfsetstrokeopacity{0.627920}%
\pgfsetdash{}{0pt}%
\pgfpathmoveto{\pgfqpoint{0.903610in}{1.258392in}}%
\pgfpathcurveto{\pgfqpoint{0.911846in}{1.258392in}}{\pgfqpoint{0.919746in}{1.261664in}}{\pgfqpoint{0.925570in}{1.267488in}}%
\pgfpathcurveto{\pgfqpoint{0.931394in}{1.273312in}}{\pgfqpoint{0.934666in}{1.281212in}}{\pgfqpoint{0.934666in}{1.289448in}}%
\pgfpathcurveto{\pgfqpoint{0.934666in}{1.297685in}}{\pgfqpoint{0.931394in}{1.305585in}}{\pgfqpoint{0.925570in}{1.311409in}}%
\pgfpathcurveto{\pgfqpoint{0.919746in}{1.317233in}}{\pgfqpoint{0.911846in}{1.320505in}}{\pgfqpoint{0.903610in}{1.320505in}}%
\pgfpathcurveto{\pgfqpoint{0.895374in}{1.320505in}}{\pgfqpoint{0.887474in}{1.317233in}}{\pgfqpoint{0.881650in}{1.311409in}}%
\pgfpathcurveto{\pgfqpoint{0.875826in}{1.305585in}}{\pgfqpoint{0.872553in}{1.297685in}}{\pgfqpoint{0.872553in}{1.289448in}}%
\pgfpathcurveto{\pgfqpoint{0.872553in}{1.281212in}}{\pgfqpoint{0.875826in}{1.273312in}}{\pgfqpoint{0.881650in}{1.267488in}}%
\pgfpathcurveto{\pgfqpoint{0.887474in}{1.261664in}}{\pgfqpoint{0.895374in}{1.258392in}}{\pgfqpoint{0.903610in}{1.258392in}}%
\pgfpathclose%
\pgfusepath{stroke,fill}%
\end{pgfscope}%
\begin{pgfscope}%
\pgfpathrectangle{\pgfqpoint{0.100000in}{0.212622in}}{\pgfqpoint{3.696000in}{3.696000in}}%
\pgfusepath{clip}%
\pgfsetbuttcap%
\pgfsetroundjoin%
\definecolor{currentfill}{rgb}{0.121569,0.466667,0.705882}%
\pgfsetfillcolor{currentfill}%
\pgfsetfillopacity{0.628739}%
\pgfsetlinewidth{1.003750pt}%
\definecolor{currentstroke}{rgb}{0.121569,0.466667,0.705882}%
\pgfsetstrokecolor{currentstroke}%
\pgfsetstrokeopacity{0.628739}%
\pgfsetdash{}{0pt}%
\pgfpathmoveto{\pgfqpoint{0.875418in}{1.279945in}}%
\pgfpathcurveto{\pgfqpoint{0.883654in}{1.279945in}}{\pgfqpoint{0.891554in}{1.283217in}}{\pgfqpoint{0.897378in}{1.289041in}}%
\pgfpathcurveto{\pgfqpoint{0.903202in}{1.294865in}}{\pgfqpoint{0.906474in}{1.302765in}}{\pgfqpoint{0.906474in}{1.311002in}}%
\pgfpathcurveto{\pgfqpoint{0.906474in}{1.319238in}}{\pgfqpoint{0.903202in}{1.327138in}}{\pgfqpoint{0.897378in}{1.332962in}}%
\pgfpathcurveto{\pgfqpoint{0.891554in}{1.338786in}}{\pgfqpoint{0.883654in}{1.342058in}}{\pgfqpoint{0.875418in}{1.342058in}}%
\pgfpathcurveto{\pgfqpoint{0.867182in}{1.342058in}}{\pgfqpoint{0.859282in}{1.338786in}}{\pgfqpoint{0.853458in}{1.332962in}}%
\pgfpathcurveto{\pgfqpoint{0.847634in}{1.327138in}}{\pgfqpoint{0.844361in}{1.319238in}}{\pgfqpoint{0.844361in}{1.311002in}}%
\pgfpathcurveto{\pgfqpoint{0.844361in}{1.302765in}}{\pgfqpoint{0.847634in}{1.294865in}}{\pgfqpoint{0.853458in}{1.289041in}}%
\pgfpathcurveto{\pgfqpoint{0.859282in}{1.283217in}}{\pgfqpoint{0.867182in}{1.279945in}}{\pgfqpoint{0.875418in}{1.279945in}}%
\pgfpathclose%
\pgfusepath{stroke,fill}%
\end{pgfscope}%
\begin{pgfscope}%
\pgfpathrectangle{\pgfqpoint{0.100000in}{0.212622in}}{\pgfqpoint{3.696000in}{3.696000in}}%
\pgfusepath{clip}%
\pgfsetbuttcap%
\pgfsetroundjoin%
\definecolor{currentfill}{rgb}{0.121569,0.466667,0.705882}%
\pgfsetfillcolor{currentfill}%
\pgfsetfillopacity{0.629732}%
\pgfsetlinewidth{1.003750pt}%
\definecolor{currentstroke}{rgb}{0.121569,0.466667,0.705882}%
\pgfsetstrokecolor{currentstroke}%
\pgfsetstrokeopacity{0.629732}%
\pgfsetdash{}{0pt}%
\pgfpathmoveto{\pgfqpoint{0.872755in}{1.278842in}}%
\pgfpathcurveto{\pgfqpoint{0.880992in}{1.278842in}}{\pgfqpoint{0.888892in}{1.282115in}}{\pgfqpoint{0.894716in}{1.287938in}}%
\pgfpathcurveto{\pgfqpoint{0.900540in}{1.293762in}}{\pgfqpoint{0.903812in}{1.301662in}}{\pgfqpoint{0.903812in}{1.309899in}}%
\pgfpathcurveto{\pgfqpoint{0.903812in}{1.318135in}}{\pgfqpoint{0.900540in}{1.326035in}}{\pgfqpoint{0.894716in}{1.331859in}}%
\pgfpathcurveto{\pgfqpoint{0.888892in}{1.337683in}}{\pgfqpoint{0.880992in}{1.340955in}}{\pgfqpoint{0.872755in}{1.340955in}}%
\pgfpathcurveto{\pgfqpoint{0.864519in}{1.340955in}}{\pgfqpoint{0.856619in}{1.337683in}}{\pgfqpoint{0.850795in}{1.331859in}}%
\pgfpathcurveto{\pgfqpoint{0.844971in}{1.326035in}}{\pgfqpoint{0.841699in}{1.318135in}}{\pgfqpoint{0.841699in}{1.309899in}}%
\pgfpathcurveto{\pgfqpoint{0.841699in}{1.301662in}}{\pgfqpoint{0.844971in}{1.293762in}}{\pgfqpoint{0.850795in}{1.287938in}}%
\pgfpathcurveto{\pgfqpoint{0.856619in}{1.282115in}}{\pgfqpoint{0.864519in}{1.278842in}}{\pgfqpoint{0.872755in}{1.278842in}}%
\pgfpathclose%
\pgfusepath{stroke,fill}%
\end{pgfscope}%
\begin{pgfscope}%
\pgfpathrectangle{\pgfqpoint{0.100000in}{0.212622in}}{\pgfqpoint{3.696000in}{3.696000in}}%
\pgfusepath{clip}%
\pgfsetbuttcap%
\pgfsetroundjoin%
\definecolor{currentfill}{rgb}{0.121569,0.466667,0.705882}%
\pgfsetfillcolor{currentfill}%
\pgfsetfillopacity{0.630008}%
\pgfsetlinewidth{1.003750pt}%
\definecolor{currentstroke}{rgb}{0.121569,0.466667,0.705882}%
\pgfsetstrokecolor{currentstroke}%
\pgfsetstrokeopacity{0.630008}%
\pgfsetdash{}{0pt}%
\pgfpathmoveto{\pgfqpoint{0.899897in}{1.259847in}}%
\pgfpathcurveto{\pgfqpoint{0.908134in}{1.259847in}}{\pgfqpoint{0.916034in}{1.263120in}}{\pgfqpoint{0.921858in}{1.268944in}}%
\pgfpathcurveto{\pgfqpoint{0.927682in}{1.274767in}}{\pgfqpoint{0.930954in}{1.282667in}}{\pgfqpoint{0.930954in}{1.290904in}}%
\pgfpathcurveto{\pgfqpoint{0.930954in}{1.299140in}}{\pgfqpoint{0.927682in}{1.307040in}}{\pgfqpoint{0.921858in}{1.312864in}}%
\pgfpathcurveto{\pgfqpoint{0.916034in}{1.318688in}}{\pgfqpoint{0.908134in}{1.321960in}}{\pgfqpoint{0.899897in}{1.321960in}}%
\pgfpathcurveto{\pgfqpoint{0.891661in}{1.321960in}}{\pgfqpoint{0.883761in}{1.318688in}}{\pgfqpoint{0.877937in}{1.312864in}}%
\pgfpathcurveto{\pgfqpoint{0.872113in}{1.307040in}}{\pgfqpoint{0.868841in}{1.299140in}}{\pgfqpoint{0.868841in}{1.290904in}}%
\pgfpathcurveto{\pgfqpoint{0.868841in}{1.282667in}}{\pgfqpoint{0.872113in}{1.274767in}}{\pgfqpoint{0.877937in}{1.268944in}}%
\pgfpathcurveto{\pgfqpoint{0.883761in}{1.263120in}}{\pgfqpoint{0.891661in}{1.259847in}}{\pgfqpoint{0.899897in}{1.259847in}}%
\pgfpathclose%
\pgfusepath{stroke,fill}%
\end{pgfscope}%
\begin{pgfscope}%
\pgfpathrectangle{\pgfqpoint{0.100000in}{0.212622in}}{\pgfqpoint{3.696000in}{3.696000in}}%
\pgfusepath{clip}%
\pgfsetbuttcap%
\pgfsetroundjoin%
\definecolor{currentfill}{rgb}{0.121569,0.466667,0.705882}%
\pgfsetfillcolor{currentfill}%
\pgfsetfillopacity{0.630173}%
\pgfsetlinewidth{1.003750pt}%
\definecolor{currentstroke}{rgb}{0.121569,0.466667,0.705882}%
\pgfsetstrokecolor{currentstroke}%
\pgfsetstrokeopacity{0.630173}%
\pgfsetdash{}{0pt}%
\pgfpathmoveto{\pgfqpoint{2.127084in}{1.795085in}}%
\pgfpathcurveto{\pgfqpoint{2.135321in}{1.795085in}}{\pgfqpoint{2.143221in}{1.798357in}}{\pgfqpoint{2.149045in}{1.804181in}}%
\pgfpathcurveto{\pgfqpoint{2.154869in}{1.810005in}}{\pgfqpoint{2.158141in}{1.817905in}}{\pgfqpoint{2.158141in}{1.826141in}}%
\pgfpathcurveto{\pgfqpoint{2.158141in}{1.834377in}}{\pgfqpoint{2.154869in}{1.842277in}}{\pgfqpoint{2.149045in}{1.848101in}}%
\pgfpathcurveto{\pgfqpoint{2.143221in}{1.853925in}}{\pgfqpoint{2.135321in}{1.857198in}}{\pgfqpoint{2.127084in}{1.857198in}}%
\pgfpathcurveto{\pgfqpoint{2.118848in}{1.857198in}}{\pgfqpoint{2.110948in}{1.853925in}}{\pgfqpoint{2.105124in}{1.848101in}}%
\pgfpathcurveto{\pgfqpoint{2.099300in}{1.842277in}}{\pgfqpoint{2.096028in}{1.834377in}}{\pgfqpoint{2.096028in}{1.826141in}}%
\pgfpathcurveto{\pgfqpoint{2.096028in}{1.817905in}}{\pgfqpoint{2.099300in}{1.810005in}}{\pgfqpoint{2.105124in}{1.804181in}}%
\pgfpathcurveto{\pgfqpoint{2.110948in}{1.798357in}}{\pgfqpoint{2.118848in}{1.795085in}}{\pgfqpoint{2.127084in}{1.795085in}}%
\pgfpathclose%
\pgfusepath{stroke,fill}%
\end{pgfscope}%
\begin{pgfscope}%
\pgfpathrectangle{\pgfqpoint{0.100000in}{0.212622in}}{\pgfqpoint{3.696000in}{3.696000in}}%
\pgfusepath{clip}%
\pgfsetbuttcap%
\pgfsetroundjoin%
\definecolor{currentfill}{rgb}{0.121569,0.466667,0.705882}%
\pgfsetfillcolor{currentfill}%
\pgfsetfillopacity{0.630313}%
\pgfsetlinewidth{1.003750pt}%
\definecolor{currentstroke}{rgb}{0.121569,0.466667,0.705882}%
\pgfsetstrokecolor{currentstroke}%
\pgfsetstrokeopacity{0.630313}%
\pgfsetdash{}{0pt}%
\pgfpathmoveto{\pgfqpoint{0.871633in}{1.277328in}}%
\pgfpathcurveto{\pgfqpoint{0.879869in}{1.277328in}}{\pgfqpoint{0.887769in}{1.280601in}}{\pgfqpoint{0.893593in}{1.286424in}}%
\pgfpathcurveto{\pgfqpoint{0.899417in}{1.292248in}}{\pgfqpoint{0.902690in}{1.300148in}}{\pgfqpoint{0.902690in}{1.308385in}}%
\pgfpathcurveto{\pgfqpoint{0.902690in}{1.316621in}}{\pgfqpoint{0.899417in}{1.324521in}}{\pgfqpoint{0.893593in}{1.330345in}}%
\pgfpathcurveto{\pgfqpoint{0.887769in}{1.336169in}}{\pgfqpoint{0.879869in}{1.339441in}}{\pgfqpoint{0.871633in}{1.339441in}}%
\pgfpathcurveto{\pgfqpoint{0.863397in}{1.339441in}}{\pgfqpoint{0.855497in}{1.336169in}}{\pgfqpoint{0.849673in}{1.330345in}}%
\pgfpathcurveto{\pgfqpoint{0.843849in}{1.324521in}}{\pgfqpoint{0.840577in}{1.316621in}}{\pgfqpoint{0.840577in}{1.308385in}}%
\pgfpathcurveto{\pgfqpoint{0.840577in}{1.300148in}}{\pgfqpoint{0.843849in}{1.292248in}}{\pgfqpoint{0.849673in}{1.286424in}}%
\pgfpathcurveto{\pgfqpoint{0.855497in}{1.280601in}}{\pgfqpoint{0.863397in}{1.277328in}}{\pgfqpoint{0.871633in}{1.277328in}}%
\pgfpathclose%
\pgfusepath{stroke,fill}%
\end{pgfscope}%
\begin{pgfscope}%
\pgfpathrectangle{\pgfqpoint{0.100000in}{0.212622in}}{\pgfqpoint{3.696000in}{3.696000in}}%
\pgfusepath{clip}%
\pgfsetbuttcap%
\pgfsetroundjoin%
\definecolor{currentfill}{rgb}{0.121569,0.466667,0.705882}%
\pgfsetfillcolor{currentfill}%
\pgfsetfillopacity{0.631119}%
\pgfsetlinewidth{1.003750pt}%
\definecolor{currentstroke}{rgb}{0.121569,0.466667,0.705882}%
\pgfsetstrokecolor{currentstroke}%
\pgfsetstrokeopacity{0.631119}%
\pgfsetdash{}{0pt}%
\pgfpathmoveto{\pgfqpoint{0.897784in}{1.260490in}}%
\pgfpathcurveto{\pgfqpoint{0.906020in}{1.260490in}}{\pgfqpoint{0.913920in}{1.263762in}}{\pgfqpoint{0.919744in}{1.269586in}}%
\pgfpathcurveto{\pgfqpoint{0.925568in}{1.275410in}}{\pgfqpoint{0.928840in}{1.283310in}}{\pgfqpoint{0.928840in}{1.291547in}}%
\pgfpathcurveto{\pgfqpoint{0.928840in}{1.299783in}}{\pgfqpoint{0.925568in}{1.307683in}}{\pgfqpoint{0.919744in}{1.313507in}}%
\pgfpathcurveto{\pgfqpoint{0.913920in}{1.319331in}}{\pgfqpoint{0.906020in}{1.322603in}}{\pgfqpoint{0.897784in}{1.322603in}}%
\pgfpathcurveto{\pgfqpoint{0.889548in}{1.322603in}}{\pgfqpoint{0.881647in}{1.319331in}}{\pgfqpoint{0.875824in}{1.313507in}}%
\pgfpathcurveto{\pgfqpoint{0.870000in}{1.307683in}}{\pgfqpoint{0.866727in}{1.299783in}}{\pgfqpoint{0.866727in}{1.291547in}}%
\pgfpathcurveto{\pgfqpoint{0.866727in}{1.283310in}}{\pgfqpoint{0.870000in}{1.275410in}}{\pgfqpoint{0.875824in}{1.269586in}}%
\pgfpathcurveto{\pgfqpoint{0.881647in}{1.263762in}}{\pgfqpoint{0.889548in}{1.260490in}}{\pgfqpoint{0.897784in}{1.260490in}}%
\pgfpathclose%
\pgfusepath{stroke,fill}%
\end{pgfscope}%
\begin{pgfscope}%
\pgfpathrectangle{\pgfqpoint{0.100000in}{0.212622in}}{\pgfqpoint{3.696000in}{3.696000in}}%
\pgfusepath{clip}%
\pgfsetbuttcap%
\pgfsetroundjoin%
\definecolor{currentfill}{rgb}{0.121569,0.466667,0.705882}%
\pgfsetfillcolor{currentfill}%
\pgfsetfillopacity{0.631357}%
\pgfsetlinewidth{1.003750pt}%
\definecolor{currentstroke}{rgb}{0.121569,0.466667,0.705882}%
\pgfsetstrokecolor{currentstroke}%
\pgfsetstrokeopacity{0.631357}%
\pgfsetdash{}{0pt}%
\pgfpathmoveto{\pgfqpoint{0.871442in}{1.274041in}}%
\pgfpathcurveto{\pgfqpoint{0.879678in}{1.274041in}}{\pgfqpoint{0.887578in}{1.277314in}}{\pgfqpoint{0.893402in}{1.283137in}}%
\pgfpathcurveto{\pgfqpoint{0.899226in}{1.288961in}}{\pgfqpoint{0.902499in}{1.296861in}}{\pgfqpoint{0.902499in}{1.305098in}}%
\pgfpathcurveto{\pgfqpoint{0.902499in}{1.313334in}}{\pgfqpoint{0.899226in}{1.321234in}}{\pgfqpoint{0.893402in}{1.327058in}}%
\pgfpathcurveto{\pgfqpoint{0.887578in}{1.332882in}}{\pgfqpoint{0.879678in}{1.336154in}}{\pgfqpoint{0.871442in}{1.336154in}}%
\pgfpathcurveto{\pgfqpoint{0.863206in}{1.336154in}}{\pgfqpoint{0.855306in}{1.332882in}}{\pgfqpoint{0.849482in}{1.327058in}}%
\pgfpathcurveto{\pgfqpoint{0.843658in}{1.321234in}}{\pgfqpoint{0.840386in}{1.313334in}}{\pgfqpoint{0.840386in}{1.305098in}}%
\pgfpathcurveto{\pgfqpoint{0.840386in}{1.296861in}}{\pgfqpoint{0.843658in}{1.288961in}}{\pgfqpoint{0.849482in}{1.283137in}}%
\pgfpathcurveto{\pgfqpoint{0.855306in}{1.277314in}}{\pgfqpoint{0.863206in}{1.274041in}}{\pgfqpoint{0.871442in}{1.274041in}}%
\pgfpathclose%
\pgfusepath{stroke,fill}%
\end{pgfscope}%
\begin{pgfscope}%
\pgfpathrectangle{\pgfqpoint{0.100000in}{0.212622in}}{\pgfqpoint{3.696000in}{3.696000in}}%
\pgfusepath{clip}%
\pgfsetbuttcap%
\pgfsetroundjoin%
\definecolor{currentfill}{rgb}{0.121569,0.466667,0.705882}%
\pgfsetfillcolor{currentfill}%
\pgfsetfillopacity{0.631427}%
\pgfsetlinewidth{1.003750pt}%
\definecolor{currentstroke}{rgb}{0.121569,0.466667,0.705882}%
\pgfsetstrokecolor{currentstroke}%
\pgfsetstrokeopacity{0.631427}%
\pgfsetdash{}{0pt}%
\pgfpathmoveto{\pgfqpoint{2.128117in}{1.793540in}}%
\pgfpathcurveto{\pgfqpoint{2.136353in}{1.793540in}}{\pgfqpoint{2.144254in}{1.796813in}}{\pgfqpoint{2.150077in}{1.802637in}}%
\pgfpathcurveto{\pgfqpoint{2.155901in}{1.808461in}}{\pgfqpoint{2.159174in}{1.816361in}}{\pgfqpoint{2.159174in}{1.824597in}}%
\pgfpathcurveto{\pgfqpoint{2.159174in}{1.832833in}}{\pgfqpoint{2.155901in}{1.840733in}}{\pgfqpoint{2.150077in}{1.846557in}}%
\pgfpathcurveto{\pgfqpoint{2.144254in}{1.852381in}}{\pgfqpoint{2.136353in}{1.855653in}}{\pgfqpoint{2.128117in}{1.855653in}}%
\pgfpathcurveto{\pgfqpoint{2.119881in}{1.855653in}}{\pgfqpoint{2.111981in}{1.852381in}}{\pgfqpoint{2.106157in}{1.846557in}}%
\pgfpathcurveto{\pgfqpoint{2.100333in}{1.840733in}}{\pgfqpoint{2.097061in}{1.832833in}}{\pgfqpoint{2.097061in}{1.824597in}}%
\pgfpathcurveto{\pgfqpoint{2.097061in}{1.816361in}}{\pgfqpoint{2.100333in}{1.808461in}}{\pgfqpoint{2.106157in}{1.802637in}}%
\pgfpathcurveto{\pgfqpoint{2.111981in}{1.796813in}}{\pgfqpoint{2.119881in}{1.793540in}}{\pgfqpoint{2.128117in}{1.793540in}}%
\pgfpathclose%
\pgfusepath{stroke,fill}%
\end{pgfscope}%
\begin{pgfscope}%
\pgfpathrectangle{\pgfqpoint{0.100000in}{0.212622in}}{\pgfqpoint{3.696000in}{3.696000in}}%
\pgfusepath{clip}%
\pgfsetbuttcap%
\pgfsetroundjoin%
\definecolor{currentfill}{rgb}{0.121569,0.466667,0.705882}%
\pgfsetfillcolor{currentfill}%
\pgfsetfillopacity{0.631754}%
\pgfsetlinewidth{1.003750pt}%
\definecolor{currentstroke}{rgb}{0.121569,0.466667,0.705882}%
\pgfsetstrokecolor{currentstroke}%
\pgfsetstrokeopacity{0.631754}%
\pgfsetdash{}{0pt}%
\pgfpathmoveto{\pgfqpoint{0.896626in}{1.260984in}}%
\pgfpathcurveto{\pgfqpoint{0.904862in}{1.260984in}}{\pgfqpoint{0.912762in}{1.264256in}}{\pgfqpoint{0.918586in}{1.270080in}}%
\pgfpathcurveto{\pgfqpoint{0.924410in}{1.275904in}}{\pgfqpoint{0.927682in}{1.283804in}}{\pgfqpoint{0.927682in}{1.292041in}}%
\pgfpathcurveto{\pgfqpoint{0.927682in}{1.300277in}}{\pgfqpoint{0.924410in}{1.308177in}}{\pgfqpoint{0.918586in}{1.314001in}}%
\pgfpathcurveto{\pgfqpoint{0.912762in}{1.319825in}}{\pgfqpoint{0.904862in}{1.323097in}}{\pgfqpoint{0.896626in}{1.323097in}}%
\pgfpathcurveto{\pgfqpoint{0.888389in}{1.323097in}}{\pgfqpoint{0.880489in}{1.319825in}}{\pgfqpoint{0.874665in}{1.314001in}}%
\pgfpathcurveto{\pgfqpoint{0.868841in}{1.308177in}}{\pgfqpoint{0.865569in}{1.300277in}}{\pgfqpoint{0.865569in}{1.292041in}}%
\pgfpathcurveto{\pgfqpoint{0.865569in}{1.283804in}}{\pgfqpoint{0.868841in}{1.275904in}}{\pgfqpoint{0.874665in}{1.270080in}}%
\pgfpathcurveto{\pgfqpoint{0.880489in}{1.264256in}}{\pgfqpoint{0.888389in}{1.260984in}}{\pgfqpoint{0.896626in}{1.260984in}}%
\pgfpathclose%
\pgfusepath{stroke,fill}%
\end{pgfscope}%
\begin{pgfscope}%
\pgfpathrectangle{\pgfqpoint{0.100000in}{0.212622in}}{\pgfqpoint{3.696000in}{3.696000in}}%
\pgfusepath{clip}%
\pgfsetbuttcap%
\pgfsetroundjoin%
\definecolor{currentfill}{rgb}{0.121569,0.466667,0.705882}%
\pgfsetfillcolor{currentfill}%
\pgfsetfillopacity{0.632103}%
\pgfsetlinewidth{1.003750pt}%
\definecolor{currentstroke}{rgb}{0.121569,0.466667,0.705882}%
\pgfsetstrokecolor{currentstroke}%
\pgfsetstrokeopacity{0.632103}%
\pgfsetdash{}{0pt}%
\pgfpathmoveto{\pgfqpoint{0.895996in}{1.261248in}}%
\pgfpathcurveto{\pgfqpoint{0.904233in}{1.261248in}}{\pgfqpoint{0.912133in}{1.264521in}}{\pgfqpoint{0.917957in}{1.270344in}}%
\pgfpathcurveto{\pgfqpoint{0.923781in}{1.276168in}}{\pgfqpoint{0.927053in}{1.284068in}}{\pgfqpoint{0.927053in}{1.292305in}}%
\pgfpathcurveto{\pgfqpoint{0.927053in}{1.300541in}}{\pgfqpoint{0.923781in}{1.308441in}}{\pgfqpoint{0.917957in}{1.314265in}}%
\pgfpathcurveto{\pgfqpoint{0.912133in}{1.320089in}}{\pgfqpoint{0.904233in}{1.323361in}}{\pgfqpoint{0.895996in}{1.323361in}}%
\pgfpathcurveto{\pgfqpoint{0.887760in}{1.323361in}}{\pgfqpoint{0.879860in}{1.320089in}}{\pgfqpoint{0.874036in}{1.314265in}}%
\pgfpathcurveto{\pgfqpoint{0.868212in}{1.308441in}}{\pgfqpoint{0.864940in}{1.300541in}}{\pgfqpoint{0.864940in}{1.292305in}}%
\pgfpathcurveto{\pgfqpoint{0.864940in}{1.284068in}}{\pgfqpoint{0.868212in}{1.276168in}}{\pgfqpoint{0.874036in}{1.270344in}}%
\pgfpathcurveto{\pgfqpoint{0.879860in}{1.264521in}}{\pgfqpoint{0.887760in}{1.261248in}}{\pgfqpoint{0.895996in}{1.261248in}}%
\pgfpathclose%
\pgfusepath{stroke,fill}%
\end{pgfscope}%
\begin{pgfscope}%
\pgfpathrectangle{\pgfqpoint{0.100000in}{0.212622in}}{\pgfqpoint{3.696000in}{3.696000in}}%
\pgfusepath{clip}%
\pgfsetbuttcap%
\pgfsetroundjoin%
\definecolor{currentfill}{rgb}{0.121569,0.466667,0.705882}%
\pgfsetfillcolor{currentfill}%
\pgfsetfillopacity{0.632298}%
\pgfsetlinewidth{1.003750pt}%
\definecolor{currentstroke}{rgb}{0.121569,0.466667,0.705882}%
\pgfsetstrokecolor{currentstroke}%
\pgfsetstrokeopacity{0.632298}%
\pgfsetdash{}{0pt}%
\pgfpathmoveto{\pgfqpoint{0.895661in}{1.261401in}}%
\pgfpathcurveto{\pgfqpoint{0.903897in}{1.261401in}}{\pgfqpoint{0.911797in}{1.264673in}}{\pgfqpoint{0.917621in}{1.270497in}}%
\pgfpathcurveto{\pgfqpoint{0.923445in}{1.276321in}}{\pgfqpoint{0.926717in}{1.284221in}}{\pgfqpoint{0.926717in}{1.292457in}}%
\pgfpathcurveto{\pgfqpoint{0.926717in}{1.300693in}}{\pgfqpoint{0.923445in}{1.308593in}}{\pgfqpoint{0.917621in}{1.314417in}}%
\pgfpathcurveto{\pgfqpoint{0.911797in}{1.320241in}}{\pgfqpoint{0.903897in}{1.323514in}}{\pgfqpoint{0.895661in}{1.323514in}}%
\pgfpathcurveto{\pgfqpoint{0.887425in}{1.323514in}}{\pgfqpoint{0.879525in}{1.320241in}}{\pgfqpoint{0.873701in}{1.314417in}}%
\pgfpathcurveto{\pgfqpoint{0.867877in}{1.308593in}}{\pgfqpoint{0.864604in}{1.300693in}}{\pgfqpoint{0.864604in}{1.292457in}}%
\pgfpathcurveto{\pgfqpoint{0.864604in}{1.284221in}}{\pgfqpoint{0.867877in}{1.276321in}}{\pgfqpoint{0.873701in}{1.270497in}}%
\pgfpathcurveto{\pgfqpoint{0.879525in}{1.264673in}}{\pgfqpoint{0.887425in}{1.261401in}}{\pgfqpoint{0.895661in}{1.261401in}}%
\pgfpathclose%
\pgfusepath{stroke,fill}%
\end{pgfscope}%
\begin{pgfscope}%
\pgfpathrectangle{\pgfqpoint{0.100000in}{0.212622in}}{\pgfqpoint{3.696000in}{3.696000in}}%
\pgfusepath{clip}%
\pgfsetbuttcap%
\pgfsetroundjoin%
\definecolor{currentfill}{rgb}{0.121569,0.466667,0.705882}%
\pgfsetfillcolor{currentfill}%
\pgfsetfillopacity{0.632400}%
\pgfsetlinewidth{1.003750pt}%
\definecolor{currentstroke}{rgb}{0.121569,0.466667,0.705882}%
\pgfsetstrokecolor{currentstroke}%
\pgfsetstrokeopacity{0.632400}%
\pgfsetdash{}{0pt}%
\pgfpathmoveto{\pgfqpoint{0.895467in}{1.261461in}}%
\pgfpathcurveto{\pgfqpoint{0.903704in}{1.261461in}}{\pgfqpoint{0.911604in}{1.264733in}}{\pgfqpoint{0.917428in}{1.270557in}}%
\pgfpathcurveto{\pgfqpoint{0.923252in}{1.276381in}}{\pgfqpoint{0.926524in}{1.284281in}}{\pgfqpoint{0.926524in}{1.292517in}}%
\pgfpathcurveto{\pgfqpoint{0.926524in}{1.300754in}}{\pgfqpoint{0.923252in}{1.308654in}}{\pgfqpoint{0.917428in}{1.314478in}}%
\pgfpathcurveto{\pgfqpoint{0.911604in}{1.320302in}}{\pgfqpoint{0.903704in}{1.323574in}}{\pgfqpoint{0.895467in}{1.323574in}}%
\pgfpathcurveto{\pgfqpoint{0.887231in}{1.323574in}}{\pgfqpoint{0.879331in}{1.320302in}}{\pgfqpoint{0.873507in}{1.314478in}}%
\pgfpathcurveto{\pgfqpoint{0.867683in}{1.308654in}}{\pgfqpoint{0.864411in}{1.300754in}}{\pgfqpoint{0.864411in}{1.292517in}}%
\pgfpathcurveto{\pgfqpoint{0.864411in}{1.284281in}}{\pgfqpoint{0.867683in}{1.276381in}}{\pgfqpoint{0.873507in}{1.270557in}}%
\pgfpathcurveto{\pgfqpoint{0.879331in}{1.264733in}}{\pgfqpoint{0.887231in}{1.261461in}}{\pgfqpoint{0.895467in}{1.261461in}}%
\pgfpathclose%
\pgfusepath{stroke,fill}%
\end{pgfscope}%
\begin{pgfscope}%
\pgfpathrectangle{\pgfqpoint{0.100000in}{0.212622in}}{\pgfqpoint{3.696000in}{3.696000in}}%
\pgfusepath{clip}%
\pgfsetbuttcap%
\pgfsetroundjoin%
\definecolor{currentfill}{rgb}{0.121569,0.466667,0.705882}%
\pgfsetfillcolor{currentfill}%
\pgfsetfillopacity{0.632459}%
\pgfsetlinewidth{1.003750pt}%
\definecolor{currentstroke}{rgb}{0.121569,0.466667,0.705882}%
\pgfsetstrokecolor{currentstroke}%
\pgfsetstrokeopacity{0.632459}%
\pgfsetdash{}{0pt}%
\pgfpathmoveto{\pgfqpoint{0.895365in}{1.261506in}}%
\pgfpathcurveto{\pgfqpoint{0.903601in}{1.261506in}}{\pgfqpoint{0.911501in}{1.264779in}}{\pgfqpoint{0.917325in}{1.270603in}}%
\pgfpathcurveto{\pgfqpoint{0.923149in}{1.276427in}}{\pgfqpoint{0.926421in}{1.284327in}}{\pgfqpoint{0.926421in}{1.292563in}}%
\pgfpathcurveto{\pgfqpoint{0.926421in}{1.300799in}}{\pgfqpoint{0.923149in}{1.308699in}}{\pgfqpoint{0.917325in}{1.314523in}}%
\pgfpathcurveto{\pgfqpoint{0.911501in}{1.320347in}}{\pgfqpoint{0.903601in}{1.323619in}}{\pgfqpoint{0.895365in}{1.323619in}}%
\pgfpathcurveto{\pgfqpoint{0.887129in}{1.323619in}}{\pgfqpoint{0.879229in}{1.320347in}}{\pgfqpoint{0.873405in}{1.314523in}}%
\pgfpathcurveto{\pgfqpoint{0.867581in}{1.308699in}}{\pgfqpoint{0.864308in}{1.300799in}}{\pgfqpoint{0.864308in}{1.292563in}}%
\pgfpathcurveto{\pgfqpoint{0.864308in}{1.284327in}}{\pgfqpoint{0.867581in}{1.276427in}}{\pgfqpoint{0.873405in}{1.270603in}}%
\pgfpathcurveto{\pgfqpoint{0.879229in}{1.264779in}}{\pgfqpoint{0.887129in}{1.261506in}}{\pgfqpoint{0.895365in}{1.261506in}}%
\pgfpathclose%
\pgfusepath{stroke,fill}%
\end{pgfscope}%
\begin{pgfscope}%
\pgfpathrectangle{\pgfqpoint{0.100000in}{0.212622in}}{\pgfqpoint{3.696000in}{3.696000in}}%
\pgfusepath{clip}%
\pgfsetbuttcap%
\pgfsetroundjoin%
\definecolor{currentfill}{rgb}{0.121569,0.466667,0.705882}%
\pgfsetfillcolor{currentfill}%
\pgfsetfillopacity{0.632491}%
\pgfsetlinewidth{1.003750pt}%
\definecolor{currentstroke}{rgb}{0.121569,0.466667,0.705882}%
\pgfsetstrokecolor{currentstroke}%
\pgfsetstrokeopacity{0.632491}%
\pgfsetdash{}{0pt}%
\pgfpathmoveto{\pgfqpoint{0.895310in}{1.261527in}}%
\pgfpathcurveto{\pgfqpoint{0.903546in}{1.261527in}}{\pgfqpoint{0.911446in}{1.264799in}}{\pgfqpoint{0.917270in}{1.270623in}}%
\pgfpathcurveto{\pgfqpoint{0.923094in}{1.276447in}}{\pgfqpoint{0.926366in}{1.284347in}}{\pgfqpoint{0.926366in}{1.292583in}}%
\pgfpathcurveto{\pgfqpoint{0.926366in}{1.300820in}}{\pgfqpoint{0.923094in}{1.308720in}}{\pgfqpoint{0.917270in}{1.314544in}}%
\pgfpathcurveto{\pgfqpoint{0.911446in}{1.320367in}}{\pgfqpoint{0.903546in}{1.323640in}}{\pgfqpoint{0.895310in}{1.323640in}}%
\pgfpathcurveto{\pgfqpoint{0.887073in}{1.323640in}}{\pgfqpoint{0.879173in}{1.320367in}}{\pgfqpoint{0.873349in}{1.314544in}}%
\pgfpathcurveto{\pgfqpoint{0.867525in}{1.308720in}}{\pgfqpoint{0.864253in}{1.300820in}}{\pgfqpoint{0.864253in}{1.292583in}}%
\pgfpathcurveto{\pgfqpoint{0.864253in}{1.284347in}}{\pgfqpoint{0.867525in}{1.276447in}}{\pgfqpoint{0.873349in}{1.270623in}}%
\pgfpathcurveto{\pgfqpoint{0.879173in}{1.264799in}}{\pgfqpoint{0.887073in}{1.261527in}}{\pgfqpoint{0.895310in}{1.261527in}}%
\pgfpathclose%
\pgfusepath{stroke,fill}%
\end{pgfscope}%
\begin{pgfscope}%
\pgfpathrectangle{\pgfqpoint{0.100000in}{0.212622in}}{\pgfqpoint{3.696000in}{3.696000in}}%
\pgfusepath{clip}%
\pgfsetbuttcap%
\pgfsetroundjoin%
\definecolor{currentfill}{rgb}{0.121569,0.466667,0.705882}%
\pgfsetfillcolor{currentfill}%
\pgfsetfillopacity{0.632509}%
\pgfsetlinewidth{1.003750pt}%
\definecolor{currentstroke}{rgb}{0.121569,0.466667,0.705882}%
\pgfsetstrokecolor{currentstroke}%
\pgfsetstrokeopacity{0.632509}%
\pgfsetdash{}{0pt}%
\pgfpathmoveto{\pgfqpoint{0.895278in}{1.261541in}}%
\pgfpathcurveto{\pgfqpoint{0.903514in}{1.261541in}}{\pgfqpoint{0.911414in}{1.264813in}}{\pgfqpoint{0.917238in}{1.270637in}}%
\pgfpathcurveto{\pgfqpoint{0.923062in}{1.276461in}}{\pgfqpoint{0.926334in}{1.284361in}}{\pgfqpoint{0.926334in}{1.292597in}}%
\pgfpathcurveto{\pgfqpoint{0.926334in}{1.300834in}}{\pgfqpoint{0.923062in}{1.308734in}}{\pgfqpoint{0.917238in}{1.314558in}}%
\pgfpathcurveto{\pgfqpoint{0.911414in}{1.320382in}}{\pgfqpoint{0.903514in}{1.323654in}}{\pgfqpoint{0.895278in}{1.323654in}}%
\pgfpathcurveto{\pgfqpoint{0.887041in}{1.323654in}}{\pgfqpoint{0.879141in}{1.320382in}}{\pgfqpoint{0.873317in}{1.314558in}}%
\pgfpathcurveto{\pgfqpoint{0.867493in}{1.308734in}}{\pgfqpoint{0.864221in}{1.300834in}}{\pgfqpoint{0.864221in}{1.292597in}}%
\pgfpathcurveto{\pgfqpoint{0.864221in}{1.284361in}}{\pgfqpoint{0.867493in}{1.276461in}}{\pgfqpoint{0.873317in}{1.270637in}}%
\pgfpathcurveto{\pgfqpoint{0.879141in}{1.264813in}}{\pgfqpoint{0.887041in}{1.261541in}}{\pgfqpoint{0.895278in}{1.261541in}}%
\pgfpathclose%
\pgfusepath{stroke,fill}%
\end{pgfscope}%
\begin{pgfscope}%
\pgfpathrectangle{\pgfqpoint{0.100000in}{0.212622in}}{\pgfqpoint{3.696000in}{3.696000in}}%
\pgfusepath{clip}%
\pgfsetbuttcap%
\pgfsetroundjoin%
\definecolor{currentfill}{rgb}{0.121569,0.466667,0.705882}%
\pgfsetfillcolor{currentfill}%
\pgfsetfillopacity{0.632519}%
\pgfsetlinewidth{1.003750pt}%
\definecolor{currentstroke}{rgb}{0.121569,0.466667,0.705882}%
\pgfsetstrokecolor{currentstroke}%
\pgfsetstrokeopacity{0.632519}%
\pgfsetdash{}{0pt}%
\pgfpathmoveto{\pgfqpoint{0.895261in}{1.261549in}}%
\pgfpathcurveto{\pgfqpoint{0.903497in}{1.261549in}}{\pgfqpoint{0.911397in}{1.264821in}}{\pgfqpoint{0.917221in}{1.270645in}}%
\pgfpathcurveto{\pgfqpoint{0.923045in}{1.276469in}}{\pgfqpoint{0.926317in}{1.284369in}}{\pgfqpoint{0.926317in}{1.292605in}}%
\pgfpathcurveto{\pgfqpoint{0.926317in}{1.300842in}}{\pgfqpoint{0.923045in}{1.308742in}}{\pgfqpoint{0.917221in}{1.314566in}}%
\pgfpathcurveto{\pgfqpoint{0.911397in}{1.320390in}}{\pgfqpoint{0.903497in}{1.323662in}}{\pgfqpoint{0.895261in}{1.323662in}}%
\pgfpathcurveto{\pgfqpoint{0.887025in}{1.323662in}}{\pgfqpoint{0.879125in}{1.320390in}}{\pgfqpoint{0.873301in}{1.314566in}}%
\pgfpathcurveto{\pgfqpoint{0.867477in}{1.308742in}}{\pgfqpoint{0.864204in}{1.300842in}}{\pgfqpoint{0.864204in}{1.292605in}}%
\pgfpathcurveto{\pgfqpoint{0.864204in}{1.284369in}}{\pgfqpoint{0.867477in}{1.276469in}}{\pgfqpoint{0.873301in}{1.270645in}}%
\pgfpathcurveto{\pgfqpoint{0.879125in}{1.264821in}}{\pgfqpoint{0.887025in}{1.261549in}}{\pgfqpoint{0.895261in}{1.261549in}}%
\pgfpathclose%
\pgfusepath{stroke,fill}%
\end{pgfscope}%
\begin{pgfscope}%
\pgfpathrectangle{\pgfqpoint{0.100000in}{0.212622in}}{\pgfqpoint{3.696000in}{3.696000in}}%
\pgfusepath{clip}%
\pgfsetbuttcap%
\pgfsetroundjoin%
\definecolor{currentfill}{rgb}{0.121569,0.466667,0.705882}%
\pgfsetfillcolor{currentfill}%
\pgfsetfillopacity{0.632524}%
\pgfsetlinewidth{1.003750pt}%
\definecolor{currentstroke}{rgb}{0.121569,0.466667,0.705882}%
\pgfsetstrokecolor{currentstroke}%
\pgfsetstrokeopacity{0.632524}%
\pgfsetdash{}{0pt}%
\pgfpathmoveto{\pgfqpoint{0.895251in}{1.261554in}}%
\pgfpathcurveto{\pgfqpoint{0.903487in}{1.261554in}}{\pgfqpoint{0.911387in}{1.264827in}}{\pgfqpoint{0.917211in}{1.270651in}}%
\pgfpathcurveto{\pgfqpoint{0.923035in}{1.276475in}}{\pgfqpoint{0.926307in}{1.284375in}}{\pgfqpoint{0.926307in}{1.292611in}}%
\pgfpathcurveto{\pgfqpoint{0.926307in}{1.300847in}}{\pgfqpoint{0.923035in}{1.308747in}}{\pgfqpoint{0.917211in}{1.314571in}}%
\pgfpathcurveto{\pgfqpoint{0.911387in}{1.320395in}}{\pgfqpoint{0.903487in}{1.323667in}}{\pgfqpoint{0.895251in}{1.323667in}}%
\pgfpathcurveto{\pgfqpoint{0.887015in}{1.323667in}}{\pgfqpoint{0.879115in}{1.320395in}}{\pgfqpoint{0.873291in}{1.314571in}}%
\pgfpathcurveto{\pgfqpoint{0.867467in}{1.308747in}}{\pgfqpoint{0.864194in}{1.300847in}}{\pgfqpoint{0.864194in}{1.292611in}}%
\pgfpathcurveto{\pgfqpoint{0.864194in}{1.284375in}}{\pgfqpoint{0.867467in}{1.276475in}}{\pgfqpoint{0.873291in}{1.270651in}}%
\pgfpathcurveto{\pgfqpoint{0.879115in}{1.264827in}}{\pgfqpoint{0.887015in}{1.261554in}}{\pgfqpoint{0.895251in}{1.261554in}}%
\pgfpathclose%
\pgfusepath{stroke,fill}%
\end{pgfscope}%
\begin{pgfscope}%
\pgfpathrectangle{\pgfqpoint{0.100000in}{0.212622in}}{\pgfqpoint{3.696000in}{3.696000in}}%
\pgfusepath{clip}%
\pgfsetbuttcap%
\pgfsetroundjoin%
\definecolor{currentfill}{rgb}{0.121569,0.466667,0.705882}%
\pgfsetfillcolor{currentfill}%
\pgfsetfillopacity{0.632527}%
\pgfsetlinewidth{1.003750pt}%
\definecolor{currentstroke}{rgb}{0.121569,0.466667,0.705882}%
\pgfsetstrokecolor{currentstroke}%
\pgfsetstrokeopacity{0.632527}%
\pgfsetdash{}{0pt}%
\pgfpathmoveto{\pgfqpoint{0.895246in}{1.261557in}}%
\pgfpathcurveto{\pgfqpoint{0.903482in}{1.261557in}}{\pgfqpoint{0.911382in}{1.264829in}}{\pgfqpoint{0.917206in}{1.270653in}}%
\pgfpathcurveto{\pgfqpoint{0.923030in}{1.276477in}}{\pgfqpoint{0.926302in}{1.284377in}}{\pgfqpoint{0.926302in}{1.292613in}}%
\pgfpathcurveto{\pgfqpoint{0.926302in}{1.300849in}}{\pgfqpoint{0.923030in}{1.308749in}}{\pgfqpoint{0.917206in}{1.314573in}}%
\pgfpathcurveto{\pgfqpoint{0.911382in}{1.320397in}}{\pgfqpoint{0.903482in}{1.323670in}}{\pgfqpoint{0.895246in}{1.323670in}}%
\pgfpathcurveto{\pgfqpoint{0.887009in}{1.323670in}}{\pgfqpoint{0.879109in}{1.320397in}}{\pgfqpoint{0.873285in}{1.314573in}}%
\pgfpathcurveto{\pgfqpoint{0.867461in}{1.308749in}}{\pgfqpoint{0.864189in}{1.300849in}}{\pgfqpoint{0.864189in}{1.292613in}}%
\pgfpathcurveto{\pgfqpoint{0.864189in}{1.284377in}}{\pgfqpoint{0.867461in}{1.276477in}}{\pgfqpoint{0.873285in}{1.270653in}}%
\pgfpathcurveto{\pgfqpoint{0.879109in}{1.264829in}}{\pgfqpoint{0.887009in}{1.261557in}}{\pgfqpoint{0.895246in}{1.261557in}}%
\pgfpathclose%
\pgfusepath{stroke,fill}%
\end{pgfscope}%
\begin{pgfscope}%
\pgfpathrectangle{\pgfqpoint{0.100000in}{0.212622in}}{\pgfqpoint{3.696000in}{3.696000in}}%
\pgfusepath{clip}%
\pgfsetbuttcap%
\pgfsetroundjoin%
\definecolor{currentfill}{rgb}{0.121569,0.466667,0.705882}%
\pgfsetfillcolor{currentfill}%
\pgfsetfillopacity{0.632529}%
\pgfsetlinewidth{1.003750pt}%
\definecolor{currentstroke}{rgb}{0.121569,0.466667,0.705882}%
\pgfsetstrokecolor{currentstroke}%
\pgfsetstrokeopacity{0.632529}%
\pgfsetdash{}{0pt}%
\pgfpathmoveto{\pgfqpoint{0.895243in}{1.261558in}}%
\pgfpathcurveto{\pgfqpoint{0.903479in}{1.261558in}}{\pgfqpoint{0.911379in}{1.264830in}}{\pgfqpoint{0.917203in}{1.270654in}}%
\pgfpathcurveto{\pgfqpoint{0.923027in}{1.276478in}}{\pgfqpoint{0.926299in}{1.284378in}}{\pgfqpoint{0.926299in}{1.292615in}}%
\pgfpathcurveto{\pgfqpoint{0.926299in}{1.300851in}}{\pgfqpoint{0.923027in}{1.308751in}}{\pgfqpoint{0.917203in}{1.314575in}}%
\pgfpathcurveto{\pgfqpoint{0.911379in}{1.320399in}}{\pgfqpoint{0.903479in}{1.323671in}}{\pgfqpoint{0.895243in}{1.323671in}}%
\pgfpathcurveto{\pgfqpoint{0.887006in}{1.323671in}}{\pgfqpoint{0.879106in}{1.320399in}}{\pgfqpoint{0.873282in}{1.314575in}}%
\pgfpathcurveto{\pgfqpoint{0.867458in}{1.308751in}}{\pgfqpoint{0.864186in}{1.300851in}}{\pgfqpoint{0.864186in}{1.292615in}}%
\pgfpathcurveto{\pgfqpoint{0.864186in}{1.284378in}}{\pgfqpoint{0.867458in}{1.276478in}}{\pgfqpoint{0.873282in}{1.270654in}}%
\pgfpathcurveto{\pgfqpoint{0.879106in}{1.264830in}}{\pgfqpoint{0.887006in}{1.261558in}}{\pgfqpoint{0.895243in}{1.261558in}}%
\pgfpathclose%
\pgfusepath{stroke,fill}%
\end{pgfscope}%
\begin{pgfscope}%
\pgfpathrectangle{\pgfqpoint{0.100000in}{0.212622in}}{\pgfqpoint{3.696000in}{3.696000in}}%
\pgfusepath{clip}%
\pgfsetbuttcap%
\pgfsetroundjoin%
\definecolor{currentfill}{rgb}{0.121569,0.466667,0.705882}%
\pgfsetfillcolor{currentfill}%
\pgfsetfillopacity{0.632529}%
\pgfsetlinewidth{1.003750pt}%
\definecolor{currentstroke}{rgb}{0.121569,0.466667,0.705882}%
\pgfsetstrokecolor{currentstroke}%
\pgfsetstrokeopacity{0.632529}%
\pgfsetdash{}{0pt}%
\pgfpathmoveto{\pgfqpoint{0.895241in}{1.261559in}}%
\pgfpathcurveto{\pgfqpoint{0.903477in}{1.261559in}}{\pgfqpoint{0.911377in}{1.264831in}}{\pgfqpoint{0.917201in}{1.270655in}}%
\pgfpathcurveto{\pgfqpoint{0.923025in}{1.276479in}}{\pgfqpoint{0.926298in}{1.284379in}}{\pgfqpoint{0.926298in}{1.292615in}}%
\pgfpathcurveto{\pgfqpoint{0.926298in}{1.300851in}}{\pgfqpoint{0.923025in}{1.308751in}}{\pgfqpoint{0.917201in}{1.314575in}}%
\pgfpathcurveto{\pgfqpoint{0.911377in}{1.320399in}}{\pgfqpoint{0.903477in}{1.323672in}}{\pgfqpoint{0.895241in}{1.323672in}}%
\pgfpathcurveto{\pgfqpoint{0.887005in}{1.323672in}}{\pgfqpoint{0.879105in}{1.320399in}}{\pgfqpoint{0.873281in}{1.314575in}}%
\pgfpathcurveto{\pgfqpoint{0.867457in}{1.308751in}}{\pgfqpoint{0.864185in}{1.300851in}}{\pgfqpoint{0.864185in}{1.292615in}}%
\pgfpathcurveto{\pgfqpoint{0.864185in}{1.284379in}}{\pgfqpoint{0.867457in}{1.276479in}}{\pgfqpoint{0.873281in}{1.270655in}}%
\pgfpathcurveto{\pgfqpoint{0.879105in}{1.264831in}}{\pgfqpoint{0.887005in}{1.261559in}}{\pgfqpoint{0.895241in}{1.261559in}}%
\pgfpathclose%
\pgfusepath{stroke,fill}%
\end{pgfscope}%
\begin{pgfscope}%
\pgfpathrectangle{\pgfqpoint{0.100000in}{0.212622in}}{\pgfqpoint{3.696000in}{3.696000in}}%
\pgfusepath{clip}%
\pgfsetbuttcap%
\pgfsetroundjoin%
\definecolor{currentfill}{rgb}{0.121569,0.466667,0.705882}%
\pgfsetfillcolor{currentfill}%
\pgfsetfillopacity{0.632530}%
\pgfsetlinewidth{1.003750pt}%
\definecolor{currentstroke}{rgb}{0.121569,0.466667,0.705882}%
\pgfsetstrokecolor{currentstroke}%
\pgfsetstrokeopacity{0.632530}%
\pgfsetdash{}{0pt}%
\pgfpathmoveto{\pgfqpoint{0.895240in}{1.261559in}}%
\pgfpathcurveto{\pgfqpoint{0.903476in}{1.261559in}}{\pgfqpoint{0.911376in}{1.264831in}}{\pgfqpoint{0.917200in}{1.270655in}}%
\pgfpathcurveto{\pgfqpoint{0.923024in}{1.276479in}}{\pgfqpoint{0.926297in}{1.284379in}}{\pgfqpoint{0.926297in}{1.292615in}}%
\pgfpathcurveto{\pgfqpoint{0.926297in}{1.300852in}}{\pgfqpoint{0.923024in}{1.308752in}}{\pgfqpoint{0.917200in}{1.314576in}}%
\pgfpathcurveto{\pgfqpoint{0.911376in}{1.320400in}}{\pgfqpoint{0.903476in}{1.323672in}}{\pgfqpoint{0.895240in}{1.323672in}}%
\pgfpathcurveto{\pgfqpoint{0.887004in}{1.323672in}}{\pgfqpoint{0.879104in}{1.320400in}}{\pgfqpoint{0.873280in}{1.314576in}}%
\pgfpathcurveto{\pgfqpoint{0.867456in}{1.308752in}}{\pgfqpoint{0.864184in}{1.300852in}}{\pgfqpoint{0.864184in}{1.292615in}}%
\pgfpathcurveto{\pgfqpoint{0.864184in}{1.284379in}}{\pgfqpoint{0.867456in}{1.276479in}}{\pgfqpoint{0.873280in}{1.270655in}}%
\pgfpathcurveto{\pgfqpoint{0.879104in}{1.264831in}}{\pgfqpoint{0.887004in}{1.261559in}}{\pgfqpoint{0.895240in}{1.261559in}}%
\pgfpathclose%
\pgfusepath{stroke,fill}%
\end{pgfscope}%
\begin{pgfscope}%
\pgfpathrectangle{\pgfqpoint{0.100000in}{0.212622in}}{\pgfqpoint{3.696000in}{3.696000in}}%
\pgfusepath{clip}%
\pgfsetbuttcap%
\pgfsetroundjoin%
\definecolor{currentfill}{rgb}{0.121569,0.466667,0.705882}%
\pgfsetfillcolor{currentfill}%
\pgfsetfillopacity{0.632530}%
\pgfsetlinewidth{1.003750pt}%
\definecolor{currentstroke}{rgb}{0.121569,0.466667,0.705882}%
\pgfsetstrokecolor{currentstroke}%
\pgfsetstrokeopacity{0.632530}%
\pgfsetdash{}{0pt}%
\pgfpathmoveto{\pgfqpoint{0.895240in}{1.261559in}}%
\pgfpathcurveto{\pgfqpoint{0.903476in}{1.261559in}}{\pgfqpoint{0.911376in}{1.264831in}}{\pgfqpoint{0.917200in}{1.270655in}}%
\pgfpathcurveto{\pgfqpoint{0.923024in}{1.276479in}}{\pgfqpoint{0.926296in}{1.284379in}}{\pgfqpoint{0.926296in}{1.292616in}}%
\pgfpathcurveto{\pgfqpoint{0.926296in}{1.300852in}}{\pgfqpoint{0.923024in}{1.308752in}}{\pgfqpoint{0.917200in}{1.314576in}}%
\pgfpathcurveto{\pgfqpoint{0.911376in}{1.320400in}}{\pgfqpoint{0.903476in}{1.323672in}}{\pgfqpoint{0.895240in}{1.323672in}}%
\pgfpathcurveto{\pgfqpoint{0.887003in}{1.323672in}}{\pgfqpoint{0.879103in}{1.320400in}}{\pgfqpoint{0.873279in}{1.314576in}}%
\pgfpathcurveto{\pgfqpoint{0.867455in}{1.308752in}}{\pgfqpoint{0.864183in}{1.300852in}}{\pgfqpoint{0.864183in}{1.292616in}}%
\pgfpathcurveto{\pgfqpoint{0.864183in}{1.284379in}}{\pgfqpoint{0.867455in}{1.276479in}}{\pgfqpoint{0.873279in}{1.270655in}}%
\pgfpathcurveto{\pgfqpoint{0.879103in}{1.264831in}}{\pgfqpoint{0.887003in}{1.261559in}}{\pgfqpoint{0.895240in}{1.261559in}}%
\pgfpathclose%
\pgfusepath{stroke,fill}%
\end{pgfscope}%
\begin{pgfscope}%
\pgfpathrectangle{\pgfqpoint{0.100000in}{0.212622in}}{\pgfqpoint{3.696000in}{3.696000in}}%
\pgfusepath{clip}%
\pgfsetbuttcap%
\pgfsetroundjoin%
\definecolor{currentfill}{rgb}{0.121569,0.466667,0.705882}%
\pgfsetfillcolor{currentfill}%
\pgfsetfillopacity{0.632530}%
\pgfsetlinewidth{1.003750pt}%
\definecolor{currentstroke}{rgb}{0.121569,0.466667,0.705882}%
\pgfsetstrokecolor{currentstroke}%
\pgfsetstrokeopacity{0.632530}%
\pgfsetdash{}{0pt}%
\pgfpathmoveto{\pgfqpoint{0.895239in}{1.261559in}}%
\pgfpathcurveto{\pgfqpoint{0.903476in}{1.261559in}}{\pgfqpoint{0.911376in}{1.264831in}}{\pgfqpoint{0.917200in}{1.270655in}}%
\pgfpathcurveto{\pgfqpoint{0.923024in}{1.276479in}}{\pgfqpoint{0.926296in}{1.284379in}}{\pgfqpoint{0.926296in}{1.292616in}}%
\pgfpathcurveto{\pgfqpoint{0.926296in}{1.300852in}}{\pgfqpoint{0.923024in}{1.308752in}}{\pgfqpoint{0.917200in}{1.314576in}}%
\pgfpathcurveto{\pgfqpoint{0.911376in}{1.320400in}}{\pgfqpoint{0.903476in}{1.323672in}}{\pgfqpoint{0.895239in}{1.323672in}}%
\pgfpathcurveto{\pgfqpoint{0.887003in}{1.323672in}}{\pgfqpoint{0.879103in}{1.320400in}}{\pgfqpoint{0.873279in}{1.314576in}}%
\pgfpathcurveto{\pgfqpoint{0.867455in}{1.308752in}}{\pgfqpoint{0.864183in}{1.300852in}}{\pgfqpoint{0.864183in}{1.292616in}}%
\pgfpathcurveto{\pgfqpoint{0.864183in}{1.284379in}}{\pgfqpoint{0.867455in}{1.276479in}}{\pgfqpoint{0.873279in}{1.270655in}}%
\pgfpathcurveto{\pgfqpoint{0.879103in}{1.264831in}}{\pgfqpoint{0.887003in}{1.261559in}}{\pgfqpoint{0.895239in}{1.261559in}}%
\pgfpathclose%
\pgfusepath{stroke,fill}%
\end{pgfscope}%
\begin{pgfscope}%
\pgfpathrectangle{\pgfqpoint{0.100000in}{0.212622in}}{\pgfqpoint{3.696000in}{3.696000in}}%
\pgfusepath{clip}%
\pgfsetbuttcap%
\pgfsetroundjoin%
\definecolor{currentfill}{rgb}{0.121569,0.466667,0.705882}%
\pgfsetfillcolor{currentfill}%
\pgfsetfillopacity{0.632530}%
\pgfsetlinewidth{1.003750pt}%
\definecolor{currentstroke}{rgb}{0.121569,0.466667,0.705882}%
\pgfsetstrokecolor{currentstroke}%
\pgfsetstrokeopacity{0.632530}%
\pgfsetdash{}{0pt}%
\pgfpathmoveto{\pgfqpoint{0.895239in}{1.261559in}}%
\pgfpathcurveto{\pgfqpoint{0.903476in}{1.261559in}}{\pgfqpoint{0.911376in}{1.264831in}}{\pgfqpoint{0.917200in}{1.270655in}}%
\pgfpathcurveto{\pgfqpoint{0.923023in}{1.276479in}}{\pgfqpoint{0.926296in}{1.284379in}}{\pgfqpoint{0.926296in}{1.292616in}}%
\pgfpathcurveto{\pgfqpoint{0.926296in}{1.300852in}}{\pgfqpoint{0.923023in}{1.308752in}}{\pgfqpoint{0.917200in}{1.314576in}}%
\pgfpathcurveto{\pgfqpoint{0.911376in}{1.320400in}}{\pgfqpoint{0.903476in}{1.323672in}}{\pgfqpoint{0.895239in}{1.323672in}}%
\pgfpathcurveto{\pgfqpoint{0.887003in}{1.323672in}}{\pgfqpoint{0.879103in}{1.320400in}}{\pgfqpoint{0.873279in}{1.314576in}}%
\pgfpathcurveto{\pgfqpoint{0.867455in}{1.308752in}}{\pgfqpoint{0.864183in}{1.300852in}}{\pgfqpoint{0.864183in}{1.292616in}}%
\pgfpathcurveto{\pgfqpoint{0.864183in}{1.284379in}}{\pgfqpoint{0.867455in}{1.276479in}}{\pgfqpoint{0.873279in}{1.270655in}}%
\pgfpathcurveto{\pgfqpoint{0.879103in}{1.264831in}}{\pgfqpoint{0.887003in}{1.261559in}}{\pgfqpoint{0.895239in}{1.261559in}}%
\pgfpathclose%
\pgfusepath{stroke,fill}%
\end{pgfscope}%
\begin{pgfscope}%
\pgfpathrectangle{\pgfqpoint{0.100000in}{0.212622in}}{\pgfqpoint{3.696000in}{3.696000in}}%
\pgfusepath{clip}%
\pgfsetbuttcap%
\pgfsetroundjoin%
\definecolor{currentfill}{rgb}{0.121569,0.466667,0.705882}%
\pgfsetfillcolor{currentfill}%
\pgfsetfillopacity{0.632530}%
\pgfsetlinewidth{1.003750pt}%
\definecolor{currentstroke}{rgb}{0.121569,0.466667,0.705882}%
\pgfsetstrokecolor{currentstroke}%
\pgfsetstrokeopacity{0.632530}%
\pgfsetdash{}{0pt}%
\pgfpathmoveto{\pgfqpoint{0.895239in}{1.261559in}}%
\pgfpathcurveto{\pgfqpoint{0.903475in}{1.261559in}}{\pgfqpoint{0.911376in}{1.264831in}}{\pgfqpoint{0.917199in}{1.270655in}}%
\pgfpathcurveto{\pgfqpoint{0.923023in}{1.276479in}}{\pgfqpoint{0.926296in}{1.284379in}}{\pgfqpoint{0.926296in}{1.292616in}}%
\pgfpathcurveto{\pgfqpoint{0.926296in}{1.300852in}}{\pgfqpoint{0.923023in}{1.308752in}}{\pgfqpoint{0.917199in}{1.314576in}}%
\pgfpathcurveto{\pgfqpoint{0.911376in}{1.320400in}}{\pgfqpoint{0.903475in}{1.323672in}}{\pgfqpoint{0.895239in}{1.323672in}}%
\pgfpathcurveto{\pgfqpoint{0.887003in}{1.323672in}}{\pgfqpoint{0.879103in}{1.320400in}}{\pgfqpoint{0.873279in}{1.314576in}}%
\pgfpathcurveto{\pgfqpoint{0.867455in}{1.308752in}}{\pgfqpoint{0.864183in}{1.300852in}}{\pgfqpoint{0.864183in}{1.292616in}}%
\pgfpathcurveto{\pgfqpoint{0.864183in}{1.284379in}}{\pgfqpoint{0.867455in}{1.276479in}}{\pgfqpoint{0.873279in}{1.270655in}}%
\pgfpathcurveto{\pgfqpoint{0.879103in}{1.264831in}}{\pgfqpoint{0.887003in}{1.261559in}}{\pgfqpoint{0.895239in}{1.261559in}}%
\pgfpathclose%
\pgfusepath{stroke,fill}%
\end{pgfscope}%
\begin{pgfscope}%
\pgfpathrectangle{\pgfqpoint{0.100000in}{0.212622in}}{\pgfqpoint{3.696000in}{3.696000in}}%
\pgfusepath{clip}%
\pgfsetbuttcap%
\pgfsetroundjoin%
\definecolor{currentfill}{rgb}{0.121569,0.466667,0.705882}%
\pgfsetfillcolor{currentfill}%
\pgfsetfillopacity{0.632530}%
\pgfsetlinewidth{1.003750pt}%
\definecolor{currentstroke}{rgb}{0.121569,0.466667,0.705882}%
\pgfsetstrokecolor{currentstroke}%
\pgfsetstrokeopacity{0.632530}%
\pgfsetdash{}{0pt}%
\pgfpathmoveto{\pgfqpoint{0.895239in}{1.261559in}}%
\pgfpathcurveto{\pgfqpoint{0.903475in}{1.261559in}}{\pgfqpoint{0.911375in}{1.264831in}}{\pgfqpoint{0.917199in}{1.270655in}}%
\pgfpathcurveto{\pgfqpoint{0.923023in}{1.276479in}}{\pgfqpoint{0.926296in}{1.284379in}}{\pgfqpoint{0.926296in}{1.292616in}}%
\pgfpathcurveto{\pgfqpoint{0.926296in}{1.300852in}}{\pgfqpoint{0.923023in}{1.308752in}}{\pgfqpoint{0.917199in}{1.314576in}}%
\pgfpathcurveto{\pgfqpoint{0.911375in}{1.320400in}}{\pgfqpoint{0.903475in}{1.323672in}}{\pgfqpoint{0.895239in}{1.323672in}}%
\pgfpathcurveto{\pgfqpoint{0.887003in}{1.323672in}}{\pgfqpoint{0.879103in}{1.320400in}}{\pgfqpoint{0.873279in}{1.314576in}}%
\pgfpathcurveto{\pgfqpoint{0.867455in}{1.308752in}}{\pgfqpoint{0.864183in}{1.300852in}}{\pgfqpoint{0.864183in}{1.292616in}}%
\pgfpathcurveto{\pgfqpoint{0.864183in}{1.284379in}}{\pgfqpoint{0.867455in}{1.276479in}}{\pgfqpoint{0.873279in}{1.270655in}}%
\pgfpathcurveto{\pgfqpoint{0.879103in}{1.264831in}}{\pgfqpoint{0.887003in}{1.261559in}}{\pgfqpoint{0.895239in}{1.261559in}}%
\pgfpathclose%
\pgfusepath{stroke,fill}%
\end{pgfscope}%
\begin{pgfscope}%
\pgfpathrectangle{\pgfqpoint{0.100000in}{0.212622in}}{\pgfqpoint{3.696000in}{3.696000in}}%
\pgfusepath{clip}%
\pgfsetbuttcap%
\pgfsetroundjoin%
\definecolor{currentfill}{rgb}{0.121569,0.466667,0.705882}%
\pgfsetfillcolor{currentfill}%
\pgfsetfillopacity{0.632530}%
\pgfsetlinewidth{1.003750pt}%
\definecolor{currentstroke}{rgb}{0.121569,0.466667,0.705882}%
\pgfsetstrokecolor{currentstroke}%
\pgfsetstrokeopacity{0.632530}%
\pgfsetdash{}{0pt}%
\pgfpathmoveto{\pgfqpoint{0.895239in}{1.261559in}}%
\pgfpathcurveto{\pgfqpoint{0.903475in}{1.261559in}}{\pgfqpoint{0.911375in}{1.264831in}}{\pgfqpoint{0.917199in}{1.270655in}}%
\pgfpathcurveto{\pgfqpoint{0.923023in}{1.276479in}}{\pgfqpoint{0.926296in}{1.284379in}}{\pgfqpoint{0.926296in}{1.292616in}}%
\pgfpathcurveto{\pgfqpoint{0.926296in}{1.300852in}}{\pgfqpoint{0.923023in}{1.308752in}}{\pgfqpoint{0.917199in}{1.314576in}}%
\pgfpathcurveto{\pgfqpoint{0.911375in}{1.320400in}}{\pgfqpoint{0.903475in}{1.323672in}}{\pgfqpoint{0.895239in}{1.323672in}}%
\pgfpathcurveto{\pgfqpoint{0.887003in}{1.323672in}}{\pgfqpoint{0.879103in}{1.320400in}}{\pgfqpoint{0.873279in}{1.314576in}}%
\pgfpathcurveto{\pgfqpoint{0.867455in}{1.308752in}}{\pgfqpoint{0.864183in}{1.300852in}}{\pgfqpoint{0.864183in}{1.292616in}}%
\pgfpathcurveto{\pgfqpoint{0.864183in}{1.284379in}}{\pgfqpoint{0.867455in}{1.276479in}}{\pgfqpoint{0.873279in}{1.270655in}}%
\pgfpathcurveto{\pgfqpoint{0.879103in}{1.264831in}}{\pgfqpoint{0.887003in}{1.261559in}}{\pgfqpoint{0.895239in}{1.261559in}}%
\pgfpathclose%
\pgfusepath{stroke,fill}%
\end{pgfscope}%
\begin{pgfscope}%
\pgfpathrectangle{\pgfqpoint{0.100000in}{0.212622in}}{\pgfqpoint{3.696000in}{3.696000in}}%
\pgfusepath{clip}%
\pgfsetbuttcap%
\pgfsetroundjoin%
\definecolor{currentfill}{rgb}{0.121569,0.466667,0.705882}%
\pgfsetfillcolor{currentfill}%
\pgfsetfillopacity{0.632530}%
\pgfsetlinewidth{1.003750pt}%
\definecolor{currentstroke}{rgb}{0.121569,0.466667,0.705882}%
\pgfsetstrokecolor{currentstroke}%
\pgfsetstrokeopacity{0.632530}%
\pgfsetdash{}{0pt}%
\pgfpathmoveto{\pgfqpoint{0.895239in}{1.261559in}}%
\pgfpathcurveto{\pgfqpoint{0.903475in}{1.261559in}}{\pgfqpoint{0.911375in}{1.264831in}}{\pgfqpoint{0.917199in}{1.270655in}}%
\pgfpathcurveto{\pgfqpoint{0.923023in}{1.276479in}}{\pgfqpoint{0.926296in}{1.284379in}}{\pgfqpoint{0.926296in}{1.292616in}}%
\pgfpathcurveto{\pgfqpoint{0.926296in}{1.300852in}}{\pgfqpoint{0.923023in}{1.308752in}}{\pgfqpoint{0.917199in}{1.314576in}}%
\pgfpathcurveto{\pgfqpoint{0.911375in}{1.320400in}}{\pgfqpoint{0.903475in}{1.323672in}}{\pgfqpoint{0.895239in}{1.323672in}}%
\pgfpathcurveto{\pgfqpoint{0.887003in}{1.323672in}}{\pgfqpoint{0.879103in}{1.320400in}}{\pgfqpoint{0.873279in}{1.314576in}}%
\pgfpathcurveto{\pgfqpoint{0.867455in}{1.308752in}}{\pgfqpoint{0.864183in}{1.300852in}}{\pgfqpoint{0.864183in}{1.292616in}}%
\pgfpathcurveto{\pgfqpoint{0.864183in}{1.284379in}}{\pgfqpoint{0.867455in}{1.276479in}}{\pgfqpoint{0.873279in}{1.270655in}}%
\pgfpathcurveto{\pgfqpoint{0.879103in}{1.264831in}}{\pgfqpoint{0.887003in}{1.261559in}}{\pgfqpoint{0.895239in}{1.261559in}}%
\pgfpathclose%
\pgfusepath{stroke,fill}%
\end{pgfscope}%
\begin{pgfscope}%
\pgfpathrectangle{\pgfqpoint{0.100000in}{0.212622in}}{\pgfqpoint{3.696000in}{3.696000in}}%
\pgfusepath{clip}%
\pgfsetbuttcap%
\pgfsetroundjoin%
\definecolor{currentfill}{rgb}{0.121569,0.466667,0.705882}%
\pgfsetfillcolor{currentfill}%
\pgfsetfillopacity{0.632530}%
\pgfsetlinewidth{1.003750pt}%
\definecolor{currentstroke}{rgb}{0.121569,0.466667,0.705882}%
\pgfsetstrokecolor{currentstroke}%
\pgfsetstrokeopacity{0.632530}%
\pgfsetdash{}{0pt}%
\pgfpathmoveto{\pgfqpoint{0.895239in}{1.261559in}}%
\pgfpathcurveto{\pgfqpoint{0.903475in}{1.261559in}}{\pgfqpoint{0.911375in}{1.264831in}}{\pgfqpoint{0.917199in}{1.270655in}}%
\pgfpathcurveto{\pgfqpoint{0.923023in}{1.276479in}}{\pgfqpoint{0.926296in}{1.284379in}}{\pgfqpoint{0.926296in}{1.292616in}}%
\pgfpathcurveto{\pgfqpoint{0.926296in}{1.300852in}}{\pgfqpoint{0.923023in}{1.308752in}}{\pgfqpoint{0.917199in}{1.314576in}}%
\pgfpathcurveto{\pgfqpoint{0.911375in}{1.320400in}}{\pgfqpoint{0.903475in}{1.323672in}}{\pgfqpoint{0.895239in}{1.323672in}}%
\pgfpathcurveto{\pgfqpoint{0.887003in}{1.323672in}}{\pgfqpoint{0.879103in}{1.320400in}}{\pgfqpoint{0.873279in}{1.314576in}}%
\pgfpathcurveto{\pgfqpoint{0.867455in}{1.308752in}}{\pgfqpoint{0.864183in}{1.300852in}}{\pgfqpoint{0.864183in}{1.292616in}}%
\pgfpathcurveto{\pgfqpoint{0.864183in}{1.284379in}}{\pgfqpoint{0.867455in}{1.276479in}}{\pgfqpoint{0.873279in}{1.270655in}}%
\pgfpathcurveto{\pgfqpoint{0.879103in}{1.264831in}}{\pgfqpoint{0.887003in}{1.261559in}}{\pgfqpoint{0.895239in}{1.261559in}}%
\pgfpathclose%
\pgfusepath{stroke,fill}%
\end{pgfscope}%
\begin{pgfscope}%
\pgfpathrectangle{\pgfqpoint{0.100000in}{0.212622in}}{\pgfqpoint{3.696000in}{3.696000in}}%
\pgfusepath{clip}%
\pgfsetbuttcap%
\pgfsetroundjoin%
\definecolor{currentfill}{rgb}{0.121569,0.466667,0.705882}%
\pgfsetfillcolor{currentfill}%
\pgfsetfillopacity{0.632530}%
\pgfsetlinewidth{1.003750pt}%
\definecolor{currentstroke}{rgb}{0.121569,0.466667,0.705882}%
\pgfsetstrokecolor{currentstroke}%
\pgfsetstrokeopacity{0.632530}%
\pgfsetdash{}{0pt}%
\pgfpathmoveto{\pgfqpoint{0.895239in}{1.261559in}}%
\pgfpathcurveto{\pgfqpoint{0.903475in}{1.261559in}}{\pgfqpoint{0.911375in}{1.264831in}}{\pgfqpoint{0.917199in}{1.270655in}}%
\pgfpathcurveto{\pgfqpoint{0.923023in}{1.276479in}}{\pgfqpoint{0.926296in}{1.284379in}}{\pgfqpoint{0.926296in}{1.292616in}}%
\pgfpathcurveto{\pgfqpoint{0.926296in}{1.300852in}}{\pgfqpoint{0.923023in}{1.308752in}}{\pgfqpoint{0.917199in}{1.314576in}}%
\pgfpathcurveto{\pgfqpoint{0.911375in}{1.320400in}}{\pgfqpoint{0.903475in}{1.323672in}}{\pgfqpoint{0.895239in}{1.323672in}}%
\pgfpathcurveto{\pgfqpoint{0.887003in}{1.323672in}}{\pgfqpoint{0.879103in}{1.320400in}}{\pgfqpoint{0.873279in}{1.314576in}}%
\pgfpathcurveto{\pgfqpoint{0.867455in}{1.308752in}}{\pgfqpoint{0.864183in}{1.300852in}}{\pgfqpoint{0.864183in}{1.292616in}}%
\pgfpathcurveto{\pgfqpoint{0.864183in}{1.284379in}}{\pgfqpoint{0.867455in}{1.276479in}}{\pgfqpoint{0.873279in}{1.270655in}}%
\pgfpathcurveto{\pgfqpoint{0.879103in}{1.264831in}}{\pgfqpoint{0.887003in}{1.261559in}}{\pgfqpoint{0.895239in}{1.261559in}}%
\pgfpathclose%
\pgfusepath{stroke,fill}%
\end{pgfscope}%
\begin{pgfscope}%
\pgfpathrectangle{\pgfqpoint{0.100000in}{0.212622in}}{\pgfqpoint{3.696000in}{3.696000in}}%
\pgfusepath{clip}%
\pgfsetbuttcap%
\pgfsetroundjoin%
\definecolor{currentfill}{rgb}{0.121569,0.466667,0.705882}%
\pgfsetfillcolor{currentfill}%
\pgfsetfillopacity{0.632530}%
\pgfsetlinewidth{1.003750pt}%
\definecolor{currentstroke}{rgb}{0.121569,0.466667,0.705882}%
\pgfsetstrokecolor{currentstroke}%
\pgfsetstrokeopacity{0.632530}%
\pgfsetdash{}{0pt}%
\pgfpathmoveto{\pgfqpoint{0.895239in}{1.261559in}}%
\pgfpathcurveto{\pgfqpoint{0.903475in}{1.261559in}}{\pgfqpoint{0.911375in}{1.264831in}}{\pgfqpoint{0.917199in}{1.270655in}}%
\pgfpathcurveto{\pgfqpoint{0.923023in}{1.276479in}}{\pgfqpoint{0.926296in}{1.284379in}}{\pgfqpoint{0.926296in}{1.292616in}}%
\pgfpathcurveto{\pgfqpoint{0.926296in}{1.300852in}}{\pgfqpoint{0.923023in}{1.308752in}}{\pgfqpoint{0.917199in}{1.314576in}}%
\pgfpathcurveto{\pgfqpoint{0.911375in}{1.320400in}}{\pgfqpoint{0.903475in}{1.323672in}}{\pgfqpoint{0.895239in}{1.323672in}}%
\pgfpathcurveto{\pgfqpoint{0.887003in}{1.323672in}}{\pgfqpoint{0.879103in}{1.320400in}}{\pgfqpoint{0.873279in}{1.314576in}}%
\pgfpathcurveto{\pgfqpoint{0.867455in}{1.308752in}}{\pgfqpoint{0.864183in}{1.300852in}}{\pgfqpoint{0.864183in}{1.292616in}}%
\pgfpathcurveto{\pgfqpoint{0.864183in}{1.284379in}}{\pgfqpoint{0.867455in}{1.276479in}}{\pgfqpoint{0.873279in}{1.270655in}}%
\pgfpathcurveto{\pgfqpoint{0.879103in}{1.264831in}}{\pgfqpoint{0.887003in}{1.261559in}}{\pgfqpoint{0.895239in}{1.261559in}}%
\pgfpathclose%
\pgfusepath{stroke,fill}%
\end{pgfscope}%
\begin{pgfscope}%
\pgfpathrectangle{\pgfqpoint{0.100000in}{0.212622in}}{\pgfqpoint{3.696000in}{3.696000in}}%
\pgfusepath{clip}%
\pgfsetbuttcap%
\pgfsetroundjoin%
\definecolor{currentfill}{rgb}{0.121569,0.466667,0.705882}%
\pgfsetfillcolor{currentfill}%
\pgfsetfillopacity{0.632530}%
\pgfsetlinewidth{1.003750pt}%
\definecolor{currentstroke}{rgb}{0.121569,0.466667,0.705882}%
\pgfsetstrokecolor{currentstroke}%
\pgfsetstrokeopacity{0.632530}%
\pgfsetdash{}{0pt}%
\pgfpathmoveto{\pgfqpoint{0.895239in}{1.261559in}}%
\pgfpathcurveto{\pgfqpoint{0.903475in}{1.261559in}}{\pgfqpoint{0.911375in}{1.264831in}}{\pgfqpoint{0.917199in}{1.270655in}}%
\pgfpathcurveto{\pgfqpoint{0.923023in}{1.276479in}}{\pgfqpoint{0.926296in}{1.284379in}}{\pgfqpoint{0.926296in}{1.292616in}}%
\pgfpathcurveto{\pgfqpoint{0.926296in}{1.300852in}}{\pgfqpoint{0.923023in}{1.308752in}}{\pgfqpoint{0.917199in}{1.314576in}}%
\pgfpathcurveto{\pgfqpoint{0.911375in}{1.320400in}}{\pgfqpoint{0.903475in}{1.323672in}}{\pgfqpoint{0.895239in}{1.323672in}}%
\pgfpathcurveto{\pgfqpoint{0.887003in}{1.323672in}}{\pgfqpoint{0.879103in}{1.320400in}}{\pgfqpoint{0.873279in}{1.314576in}}%
\pgfpathcurveto{\pgfqpoint{0.867455in}{1.308752in}}{\pgfqpoint{0.864183in}{1.300852in}}{\pgfqpoint{0.864183in}{1.292616in}}%
\pgfpathcurveto{\pgfqpoint{0.864183in}{1.284379in}}{\pgfqpoint{0.867455in}{1.276479in}}{\pgfqpoint{0.873279in}{1.270655in}}%
\pgfpathcurveto{\pgfqpoint{0.879103in}{1.264831in}}{\pgfqpoint{0.887003in}{1.261559in}}{\pgfqpoint{0.895239in}{1.261559in}}%
\pgfpathclose%
\pgfusepath{stroke,fill}%
\end{pgfscope}%
\begin{pgfscope}%
\pgfpathrectangle{\pgfqpoint{0.100000in}{0.212622in}}{\pgfqpoint{3.696000in}{3.696000in}}%
\pgfusepath{clip}%
\pgfsetbuttcap%
\pgfsetroundjoin%
\definecolor{currentfill}{rgb}{0.121569,0.466667,0.705882}%
\pgfsetfillcolor{currentfill}%
\pgfsetfillopacity{0.632530}%
\pgfsetlinewidth{1.003750pt}%
\definecolor{currentstroke}{rgb}{0.121569,0.466667,0.705882}%
\pgfsetstrokecolor{currentstroke}%
\pgfsetstrokeopacity{0.632530}%
\pgfsetdash{}{0pt}%
\pgfpathmoveto{\pgfqpoint{0.895239in}{1.261559in}}%
\pgfpathcurveto{\pgfqpoint{0.903475in}{1.261559in}}{\pgfqpoint{0.911375in}{1.264831in}}{\pgfqpoint{0.917199in}{1.270655in}}%
\pgfpathcurveto{\pgfqpoint{0.923023in}{1.276479in}}{\pgfqpoint{0.926296in}{1.284379in}}{\pgfqpoint{0.926296in}{1.292616in}}%
\pgfpathcurveto{\pgfqpoint{0.926296in}{1.300852in}}{\pgfqpoint{0.923023in}{1.308752in}}{\pgfqpoint{0.917199in}{1.314576in}}%
\pgfpathcurveto{\pgfqpoint{0.911375in}{1.320400in}}{\pgfqpoint{0.903475in}{1.323672in}}{\pgfqpoint{0.895239in}{1.323672in}}%
\pgfpathcurveto{\pgfqpoint{0.887003in}{1.323672in}}{\pgfqpoint{0.879103in}{1.320400in}}{\pgfqpoint{0.873279in}{1.314576in}}%
\pgfpathcurveto{\pgfqpoint{0.867455in}{1.308752in}}{\pgfqpoint{0.864183in}{1.300852in}}{\pgfqpoint{0.864183in}{1.292616in}}%
\pgfpathcurveto{\pgfqpoint{0.864183in}{1.284379in}}{\pgfqpoint{0.867455in}{1.276479in}}{\pgfqpoint{0.873279in}{1.270655in}}%
\pgfpathcurveto{\pgfqpoint{0.879103in}{1.264831in}}{\pgfqpoint{0.887003in}{1.261559in}}{\pgfqpoint{0.895239in}{1.261559in}}%
\pgfpathclose%
\pgfusepath{stroke,fill}%
\end{pgfscope}%
\begin{pgfscope}%
\pgfpathrectangle{\pgfqpoint{0.100000in}{0.212622in}}{\pgfqpoint{3.696000in}{3.696000in}}%
\pgfusepath{clip}%
\pgfsetbuttcap%
\pgfsetroundjoin%
\definecolor{currentfill}{rgb}{0.121569,0.466667,0.705882}%
\pgfsetfillcolor{currentfill}%
\pgfsetfillopacity{0.632530}%
\pgfsetlinewidth{1.003750pt}%
\definecolor{currentstroke}{rgb}{0.121569,0.466667,0.705882}%
\pgfsetstrokecolor{currentstroke}%
\pgfsetstrokeopacity{0.632530}%
\pgfsetdash{}{0pt}%
\pgfpathmoveto{\pgfqpoint{0.895239in}{1.261559in}}%
\pgfpathcurveto{\pgfqpoint{0.903475in}{1.261559in}}{\pgfqpoint{0.911375in}{1.264831in}}{\pgfqpoint{0.917199in}{1.270655in}}%
\pgfpathcurveto{\pgfqpoint{0.923023in}{1.276479in}}{\pgfqpoint{0.926296in}{1.284379in}}{\pgfqpoint{0.926296in}{1.292616in}}%
\pgfpathcurveto{\pgfqpoint{0.926296in}{1.300852in}}{\pgfqpoint{0.923023in}{1.308752in}}{\pgfqpoint{0.917199in}{1.314576in}}%
\pgfpathcurveto{\pgfqpoint{0.911375in}{1.320400in}}{\pgfqpoint{0.903475in}{1.323672in}}{\pgfqpoint{0.895239in}{1.323672in}}%
\pgfpathcurveto{\pgfqpoint{0.887003in}{1.323672in}}{\pgfqpoint{0.879103in}{1.320400in}}{\pgfqpoint{0.873279in}{1.314576in}}%
\pgfpathcurveto{\pgfqpoint{0.867455in}{1.308752in}}{\pgfqpoint{0.864183in}{1.300852in}}{\pgfqpoint{0.864183in}{1.292616in}}%
\pgfpathcurveto{\pgfqpoint{0.864183in}{1.284379in}}{\pgfqpoint{0.867455in}{1.276479in}}{\pgfqpoint{0.873279in}{1.270655in}}%
\pgfpathcurveto{\pgfqpoint{0.879103in}{1.264831in}}{\pgfqpoint{0.887003in}{1.261559in}}{\pgfqpoint{0.895239in}{1.261559in}}%
\pgfpathclose%
\pgfusepath{stroke,fill}%
\end{pgfscope}%
\begin{pgfscope}%
\pgfpathrectangle{\pgfqpoint{0.100000in}{0.212622in}}{\pgfqpoint{3.696000in}{3.696000in}}%
\pgfusepath{clip}%
\pgfsetbuttcap%
\pgfsetroundjoin%
\definecolor{currentfill}{rgb}{0.121569,0.466667,0.705882}%
\pgfsetfillcolor{currentfill}%
\pgfsetfillopacity{0.632530}%
\pgfsetlinewidth{1.003750pt}%
\definecolor{currentstroke}{rgb}{0.121569,0.466667,0.705882}%
\pgfsetstrokecolor{currentstroke}%
\pgfsetstrokeopacity{0.632530}%
\pgfsetdash{}{0pt}%
\pgfpathmoveto{\pgfqpoint{0.895239in}{1.261559in}}%
\pgfpathcurveto{\pgfqpoint{0.903475in}{1.261559in}}{\pgfqpoint{0.911375in}{1.264831in}}{\pgfqpoint{0.917199in}{1.270655in}}%
\pgfpathcurveto{\pgfqpoint{0.923023in}{1.276479in}}{\pgfqpoint{0.926296in}{1.284379in}}{\pgfqpoint{0.926296in}{1.292616in}}%
\pgfpathcurveto{\pgfqpoint{0.926296in}{1.300852in}}{\pgfqpoint{0.923023in}{1.308752in}}{\pgfqpoint{0.917199in}{1.314576in}}%
\pgfpathcurveto{\pgfqpoint{0.911375in}{1.320400in}}{\pgfqpoint{0.903475in}{1.323672in}}{\pgfqpoint{0.895239in}{1.323672in}}%
\pgfpathcurveto{\pgfqpoint{0.887003in}{1.323672in}}{\pgfqpoint{0.879103in}{1.320400in}}{\pgfqpoint{0.873279in}{1.314576in}}%
\pgfpathcurveto{\pgfqpoint{0.867455in}{1.308752in}}{\pgfqpoint{0.864183in}{1.300852in}}{\pgfqpoint{0.864183in}{1.292616in}}%
\pgfpathcurveto{\pgfqpoint{0.864183in}{1.284379in}}{\pgfqpoint{0.867455in}{1.276479in}}{\pgfqpoint{0.873279in}{1.270655in}}%
\pgfpathcurveto{\pgfqpoint{0.879103in}{1.264831in}}{\pgfqpoint{0.887003in}{1.261559in}}{\pgfqpoint{0.895239in}{1.261559in}}%
\pgfpathclose%
\pgfusepath{stroke,fill}%
\end{pgfscope}%
\begin{pgfscope}%
\pgfpathrectangle{\pgfqpoint{0.100000in}{0.212622in}}{\pgfqpoint{3.696000in}{3.696000in}}%
\pgfusepath{clip}%
\pgfsetbuttcap%
\pgfsetroundjoin%
\definecolor{currentfill}{rgb}{0.121569,0.466667,0.705882}%
\pgfsetfillcolor{currentfill}%
\pgfsetfillopacity{0.632530}%
\pgfsetlinewidth{1.003750pt}%
\definecolor{currentstroke}{rgb}{0.121569,0.466667,0.705882}%
\pgfsetstrokecolor{currentstroke}%
\pgfsetstrokeopacity{0.632530}%
\pgfsetdash{}{0pt}%
\pgfpathmoveto{\pgfqpoint{0.895239in}{1.261559in}}%
\pgfpathcurveto{\pgfqpoint{0.903475in}{1.261559in}}{\pgfqpoint{0.911375in}{1.264831in}}{\pgfqpoint{0.917199in}{1.270655in}}%
\pgfpathcurveto{\pgfqpoint{0.923023in}{1.276479in}}{\pgfqpoint{0.926296in}{1.284379in}}{\pgfqpoint{0.926296in}{1.292616in}}%
\pgfpathcurveto{\pgfqpoint{0.926296in}{1.300852in}}{\pgfqpoint{0.923023in}{1.308752in}}{\pgfqpoint{0.917199in}{1.314576in}}%
\pgfpathcurveto{\pgfqpoint{0.911375in}{1.320400in}}{\pgfqpoint{0.903475in}{1.323672in}}{\pgfqpoint{0.895239in}{1.323672in}}%
\pgfpathcurveto{\pgfqpoint{0.887003in}{1.323672in}}{\pgfqpoint{0.879103in}{1.320400in}}{\pgfqpoint{0.873279in}{1.314576in}}%
\pgfpathcurveto{\pgfqpoint{0.867455in}{1.308752in}}{\pgfqpoint{0.864183in}{1.300852in}}{\pgfqpoint{0.864183in}{1.292616in}}%
\pgfpathcurveto{\pgfqpoint{0.864183in}{1.284379in}}{\pgfqpoint{0.867455in}{1.276479in}}{\pgfqpoint{0.873279in}{1.270655in}}%
\pgfpathcurveto{\pgfqpoint{0.879103in}{1.264831in}}{\pgfqpoint{0.887003in}{1.261559in}}{\pgfqpoint{0.895239in}{1.261559in}}%
\pgfpathclose%
\pgfusepath{stroke,fill}%
\end{pgfscope}%
\begin{pgfscope}%
\pgfpathrectangle{\pgfqpoint{0.100000in}{0.212622in}}{\pgfqpoint{3.696000in}{3.696000in}}%
\pgfusepath{clip}%
\pgfsetbuttcap%
\pgfsetroundjoin%
\definecolor{currentfill}{rgb}{0.121569,0.466667,0.705882}%
\pgfsetfillcolor{currentfill}%
\pgfsetfillopacity{0.632530}%
\pgfsetlinewidth{1.003750pt}%
\definecolor{currentstroke}{rgb}{0.121569,0.466667,0.705882}%
\pgfsetstrokecolor{currentstroke}%
\pgfsetstrokeopacity{0.632530}%
\pgfsetdash{}{0pt}%
\pgfpathmoveto{\pgfqpoint{0.895239in}{1.261559in}}%
\pgfpathcurveto{\pgfqpoint{0.903475in}{1.261559in}}{\pgfqpoint{0.911375in}{1.264831in}}{\pgfqpoint{0.917199in}{1.270655in}}%
\pgfpathcurveto{\pgfqpoint{0.923023in}{1.276479in}}{\pgfqpoint{0.926296in}{1.284379in}}{\pgfqpoint{0.926296in}{1.292616in}}%
\pgfpathcurveto{\pgfqpoint{0.926296in}{1.300852in}}{\pgfqpoint{0.923023in}{1.308752in}}{\pgfqpoint{0.917199in}{1.314576in}}%
\pgfpathcurveto{\pgfqpoint{0.911375in}{1.320400in}}{\pgfqpoint{0.903475in}{1.323672in}}{\pgfqpoint{0.895239in}{1.323672in}}%
\pgfpathcurveto{\pgfqpoint{0.887003in}{1.323672in}}{\pgfqpoint{0.879103in}{1.320400in}}{\pgfqpoint{0.873279in}{1.314576in}}%
\pgfpathcurveto{\pgfqpoint{0.867455in}{1.308752in}}{\pgfqpoint{0.864183in}{1.300852in}}{\pgfqpoint{0.864183in}{1.292616in}}%
\pgfpathcurveto{\pgfqpoint{0.864183in}{1.284379in}}{\pgfqpoint{0.867455in}{1.276479in}}{\pgfqpoint{0.873279in}{1.270655in}}%
\pgfpathcurveto{\pgfqpoint{0.879103in}{1.264831in}}{\pgfqpoint{0.887003in}{1.261559in}}{\pgfqpoint{0.895239in}{1.261559in}}%
\pgfpathclose%
\pgfusepath{stroke,fill}%
\end{pgfscope}%
\begin{pgfscope}%
\pgfpathrectangle{\pgfqpoint{0.100000in}{0.212622in}}{\pgfqpoint{3.696000in}{3.696000in}}%
\pgfusepath{clip}%
\pgfsetbuttcap%
\pgfsetroundjoin%
\definecolor{currentfill}{rgb}{0.121569,0.466667,0.705882}%
\pgfsetfillcolor{currentfill}%
\pgfsetfillopacity{0.632530}%
\pgfsetlinewidth{1.003750pt}%
\definecolor{currentstroke}{rgb}{0.121569,0.466667,0.705882}%
\pgfsetstrokecolor{currentstroke}%
\pgfsetstrokeopacity{0.632530}%
\pgfsetdash{}{0pt}%
\pgfpathmoveto{\pgfqpoint{0.895239in}{1.261559in}}%
\pgfpathcurveto{\pgfqpoint{0.903475in}{1.261559in}}{\pgfqpoint{0.911375in}{1.264831in}}{\pgfqpoint{0.917199in}{1.270655in}}%
\pgfpathcurveto{\pgfqpoint{0.923023in}{1.276479in}}{\pgfqpoint{0.926296in}{1.284379in}}{\pgfqpoint{0.926296in}{1.292616in}}%
\pgfpathcurveto{\pgfqpoint{0.926296in}{1.300852in}}{\pgfqpoint{0.923023in}{1.308752in}}{\pgfqpoint{0.917199in}{1.314576in}}%
\pgfpathcurveto{\pgfqpoint{0.911375in}{1.320400in}}{\pgfqpoint{0.903475in}{1.323672in}}{\pgfqpoint{0.895239in}{1.323672in}}%
\pgfpathcurveto{\pgfqpoint{0.887003in}{1.323672in}}{\pgfqpoint{0.879103in}{1.320400in}}{\pgfqpoint{0.873279in}{1.314576in}}%
\pgfpathcurveto{\pgfqpoint{0.867455in}{1.308752in}}{\pgfqpoint{0.864183in}{1.300852in}}{\pgfqpoint{0.864183in}{1.292616in}}%
\pgfpathcurveto{\pgfqpoint{0.864183in}{1.284379in}}{\pgfqpoint{0.867455in}{1.276479in}}{\pgfqpoint{0.873279in}{1.270655in}}%
\pgfpathcurveto{\pgfqpoint{0.879103in}{1.264831in}}{\pgfqpoint{0.887003in}{1.261559in}}{\pgfqpoint{0.895239in}{1.261559in}}%
\pgfpathclose%
\pgfusepath{stroke,fill}%
\end{pgfscope}%
\begin{pgfscope}%
\pgfpathrectangle{\pgfqpoint{0.100000in}{0.212622in}}{\pgfqpoint{3.696000in}{3.696000in}}%
\pgfusepath{clip}%
\pgfsetbuttcap%
\pgfsetroundjoin%
\definecolor{currentfill}{rgb}{0.121569,0.466667,0.705882}%
\pgfsetfillcolor{currentfill}%
\pgfsetfillopacity{0.632530}%
\pgfsetlinewidth{1.003750pt}%
\definecolor{currentstroke}{rgb}{0.121569,0.466667,0.705882}%
\pgfsetstrokecolor{currentstroke}%
\pgfsetstrokeopacity{0.632530}%
\pgfsetdash{}{0pt}%
\pgfpathmoveto{\pgfqpoint{0.895239in}{1.261559in}}%
\pgfpathcurveto{\pgfqpoint{0.903475in}{1.261559in}}{\pgfqpoint{0.911375in}{1.264831in}}{\pgfqpoint{0.917199in}{1.270655in}}%
\pgfpathcurveto{\pgfqpoint{0.923023in}{1.276479in}}{\pgfqpoint{0.926296in}{1.284379in}}{\pgfqpoint{0.926296in}{1.292616in}}%
\pgfpathcurveto{\pgfqpoint{0.926296in}{1.300852in}}{\pgfqpoint{0.923023in}{1.308752in}}{\pgfqpoint{0.917199in}{1.314576in}}%
\pgfpathcurveto{\pgfqpoint{0.911375in}{1.320400in}}{\pgfqpoint{0.903475in}{1.323672in}}{\pgfqpoint{0.895239in}{1.323672in}}%
\pgfpathcurveto{\pgfqpoint{0.887003in}{1.323672in}}{\pgfqpoint{0.879103in}{1.320400in}}{\pgfqpoint{0.873279in}{1.314576in}}%
\pgfpathcurveto{\pgfqpoint{0.867455in}{1.308752in}}{\pgfqpoint{0.864183in}{1.300852in}}{\pgfqpoint{0.864183in}{1.292616in}}%
\pgfpathcurveto{\pgfqpoint{0.864183in}{1.284379in}}{\pgfqpoint{0.867455in}{1.276479in}}{\pgfqpoint{0.873279in}{1.270655in}}%
\pgfpathcurveto{\pgfqpoint{0.879103in}{1.264831in}}{\pgfqpoint{0.887003in}{1.261559in}}{\pgfqpoint{0.895239in}{1.261559in}}%
\pgfpathclose%
\pgfusepath{stroke,fill}%
\end{pgfscope}%
\begin{pgfscope}%
\pgfpathrectangle{\pgfqpoint{0.100000in}{0.212622in}}{\pgfqpoint{3.696000in}{3.696000in}}%
\pgfusepath{clip}%
\pgfsetbuttcap%
\pgfsetroundjoin%
\definecolor{currentfill}{rgb}{0.121569,0.466667,0.705882}%
\pgfsetfillcolor{currentfill}%
\pgfsetfillopacity{0.632530}%
\pgfsetlinewidth{1.003750pt}%
\definecolor{currentstroke}{rgb}{0.121569,0.466667,0.705882}%
\pgfsetstrokecolor{currentstroke}%
\pgfsetstrokeopacity{0.632530}%
\pgfsetdash{}{0pt}%
\pgfpathmoveto{\pgfqpoint{0.895239in}{1.261559in}}%
\pgfpathcurveto{\pgfqpoint{0.903475in}{1.261559in}}{\pgfqpoint{0.911375in}{1.264831in}}{\pgfqpoint{0.917199in}{1.270655in}}%
\pgfpathcurveto{\pgfqpoint{0.923023in}{1.276479in}}{\pgfqpoint{0.926296in}{1.284379in}}{\pgfqpoint{0.926296in}{1.292616in}}%
\pgfpathcurveto{\pgfqpoint{0.926296in}{1.300852in}}{\pgfqpoint{0.923023in}{1.308752in}}{\pgfqpoint{0.917199in}{1.314576in}}%
\pgfpathcurveto{\pgfqpoint{0.911375in}{1.320400in}}{\pgfqpoint{0.903475in}{1.323672in}}{\pgfqpoint{0.895239in}{1.323672in}}%
\pgfpathcurveto{\pgfqpoint{0.887003in}{1.323672in}}{\pgfqpoint{0.879103in}{1.320400in}}{\pgfqpoint{0.873279in}{1.314576in}}%
\pgfpathcurveto{\pgfqpoint{0.867455in}{1.308752in}}{\pgfqpoint{0.864183in}{1.300852in}}{\pgfqpoint{0.864183in}{1.292616in}}%
\pgfpathcurveto{\pgfqpoint{0.864183in}{1.284379in}}{\pgfqpoint{0.867455in}{1.276479in}}{\pgfqpoint{0.873279in}{1.270655in}}%
\pgfpathcurveto{\pgfqpoint{0.879103in}{1.264831in}}{\pgfqpoint{0.887003in}{1.261559in}}{\pgfqpoint{0.895239in}{1.261559in}}%
\pgfpathclose%
\pgfusepath{stroke,fill}%
\end{pgfscope}%
\begin{pgfscope}%
\pgfpathrectangle{\pgfqpoint{0.100000in}{0.212622in}}{\pgfqpoint{3.696000in}{3.696000in}}%
\pgfusepath{clip}%
\pgfsetbuttcap%
\pgfsetroundjoin%
\definecolor{currentfill}{rgb}{0.121569,0.466667,0.705882}%
\pgfsetfillcolor{currentfill}%
\pgfsetfillopacity{0.632530}%
\pgfsetlinewidth{1.003750pt}%
\definecolor{currentstroke}{rgb}{0.121569,0.466667,0.705882}%
\pgfsetstrokecolor{currentstroke}%
\pgfsetstrokeopacity{0.632530}%
\pgfsetdash{}{0pt}%
\pgfpathmoveto{\pgfqpoint{0.895239in}{1.261559in}}%
\pgfpathcurveto{\pgfqpoint{0.903475in}{1.261559in}}{\pgfqpoint{0.911375in}{1.264831in}}{\pgfqpoint{0.917199in}{1.270655in}}%
\pgfpathcurveto{\pgfqpoint{0.923023in}{1.276479in}}{\pgfqpoint{0.926296in}{1.284379in}}{\pgfqpoint{0.926296in}{1.292616in}}%
\pgfpathcurveto{\pgfqpoint{0.926296in}{1.300852in}}{\pgfqpoint{0.923023in}{1.308752in}}{\pgfqpoint{0.917199in}{1.314576in}}%
\pgfpathcurveto{\pgfqpoint{0.911375in}{1.320400in}}{\pgfqpoint{0.903475in}{1.323672in}}{\pgfqpoint{0.895239in}{1.323672in}}%
\pgfpathcurveto{\pgfqpoint{0.887003in}{1.323672in}}{\pgfqpoint{0.879103in}{1.320400in}}{\pgfqpoint{0.873279in}{1.314576in}}%
\pgfpathcurveto{\pgfqpoint{0.867455in}{1.308752in}}{\pgfqpoint{0.864183in}{1.300852in}}{\pgfqpoint{0.864183in}{1.292616in}}%
\pgfpathcurveto{\pgfqpoint{0.864183in}{1.284379in}}{\pgfqpoint{0.867455in}{1.276479in}}{\pgfqpoint{0.873279in}{1.270655in}}%
\pgfpathcurveto{\pgfqpoint{0.879103in}{1.264831in}}{\pgfqpoint{0.887003in}{1.261559in}}{\pgfqpoint{0.895239in}{1.261559in}}%
\pgfpathclose%
\pgfusepath{stroke,fill}%
\end{pgfscope}%
\begin{pgfscope}%
\pgfpathrectangle{\pgfqpoint{0.100000in}{0.212622in}}{\pgfqpoint{3.696000in}{3.696000in}}%
\pgfusepath{clip}%
\pgfsetbuttcap%
\pgfsetroundjoin%
\definecolor{currentfill}{rgb}{0.121569,0.466667,0.705882}%
\pgfsetfillcolor{currentfill}%
\pgfsetfillopacity{0.632530}%
\pgfsetlinewidth{1.003750pt}%
\definecolor{currentstroke}{rgb}{0.121569,0.466667,0.705882}%
\pgfsetstrokecolor{currentstroke}%
\pgfsetstrokeopacity{0.632530}%
\pgfsetdash{}{0pt}%
\pgfpathmoveto{\pgfqpoint{0.895239in}{1.261559in}}%
\pgfpathcurveto{\pgfqpoint{0.903475in}{1.261559in}}{\pgfqpoint{0.911375in}{1.264831in}}{\pgfqpoint{0.917199in}{1.270655in}}%
\pgfpathcurveto{\pgfqpoint{0.923023in}{1.276479in}}{\pgfqpoint{0.926296in}{1.284379in}}{\pgfqpoint{0.926296in}{1.292616in}}%
\pgfpathcurveto{\pgfqpoint{0.926296in}{1.300852in}}{\pgfqpoint{0.923023in}{1.308752in}}{\pgfqpoint{0.917199in}{1.314576in}}%
\pgfpathcurveto{\pgfqpoint{0.911375in}{1.320400in}}{\pgfqpoint{0.903475in}{1.323672in}}{\pgfqpoint{0.895239in}{1.323672in}}%
\pgfpathcurveto{\pgfqpoint{0.887003in}{1.323672in}}{\pgfqpoint{0.879103in}{1.320400in}}{\pgfqpoint{0.873279in}{1.314576in}}%
\pgfpathcurveto{\pgfqpoint{0.867455in}{1.308752in}}{\pgfqpoint{0.864183in}{1.300852in}}{\pgfqpoint{0.864183in}{1.292616in}}%
\pgfpathcurveto{\pgfqpoint{0.864183in}{1.284379in}}{\pgfqpoint{0.867455in}{1.276479in}}{\pgfqpoint{0.873279in}{1.270655in}}%
\pgfpathcurveto{\pgfqpoint{0.879103in}{1.264831in}}{\pgfqpoint{0.887003in}{1.261559in}}{\pgfqpoint{0.895239in}{1.261559in}}%
\pgfpathclose%
\pgfusepath{stroke,fill}%
\end{pgfscope}%
\begin{pgfscope}%
\pgfpathrectangle{\pgfqpoint{0.100000in}{0.212622in}}{\pgfqpoint{3.696000in}{3.696000in}}%
\pgfusepath{clip}%
\pgfsetbuttcap%
\pgfsetroundjoin%
\definecolor{currentfill}{rgb}{0.121569,0.466667,0.705882}%
\pgfsetfillcolor{currentfill}%
\pgfsetfillopacity{0.632530}%
\pgfsetlinewidth{1.003750pt}%
\definecolor{currentstroke}{rgb}{0.121569,0.466667,0.705882}%
\pgfsetstrokecolor{currentstroke}%
\pgfsetstrokeopacity{0.632530}%
\pgfsetdash{}{0pt}%
\pgfpathmoveto{\pgfqpoint{0.895239in}{1.261559in}}%
\pgfpathcurveto{\pgfqpoint{0.903475in}{1.261559in}}{\pgfqpoint{0.911375in}{1.264831in}}{\pgfqpoint{0.917199in}{1.270655in}}%
\pgfpathcurveto{\pgfqpoint{0.923023in}{1.276479in}}{\pgfqpoint{0.926296in}{1.284379in}}{\pgfqpoint{0.926296in}{1.292616in}}%
\pgfpathcurveto{\pgfqpoint{0.926296in}{1.300852in}}{\pgfqpoint{0.923023in}{1.308752in}}{\pgfqpoint{0.917199in}{1.314576in}}%
\pgfpathcurveto{\pgfqpoint{0.911375in}{1.320400in}}{\pgfqpoint{0.903475in}{1.323672in}}{\pgfqpoint{0.895239in}{1.323672in}}%
\pgfpathcurveto{\pgfqpoint{0.887003in}{1.323672in}}{\pgfqpoint{0.879103in}{1.320400in}}{\pgfqpoint{0.873279in}{1.314576in}}%
\pgfpathcurveto{\pgfqpoint{0.867455in}{1.308752in}}{\pgfqpoint{0.864183in}{1.300852in}}{\pgfqpoint{0.864183in}{1.292616in}}%
\pgfpathcurveto{\pgfqpoint{0.864183in}{1.284379in}}{\pgfqpoint{0.867455in}{1.276479in}}{\pgfqpoint{0.873279in}{1.270655in}}%
\pgfpathcurveto{\pgfqpoint{0.879103in}{1.264831in}}{\pgfqpoint{0.887003in}{1.261559in}}{\pgfqpoint{0.895239in}{1.261559in}}%
\pgfpathclose%
\pgfusepath{stroke,fill}%
\end{pgfscope}%
\begin{pgfscope}%
\pgfpathrectangle{\pgfqpoint{0.100000in}{0.212622in}}{\pgfqpoint{3.696000in}{3.696000in}}%
\pgfusepath{clip}%
\pgfsetbuttcap%
\pgfsetroundjoin%
\definecolor{currentfill}{rgb}{0.121569,0.466667,0.705882}%
\pgfsetfillcolor{currentfill}%
\pgfsetfillopacity{0.632530}%
\pgfsetlinewidth{1.003750pt}%
\definecolor{currentstroke}{rgb}{0.121569,0.466667,0.705882}%
\pgfsetstrokecolor{currentstroke}%
\pgfsetstrokeopacity{0.632530}%
\pgfsetdash{}{0pt}%
\pgfpathmoveto{\pgfqpoint{0.895239in}{1.261559in}}%
\pgfpathcurveto{\pgfqpoint{0.903475in}{1.261559in}}{\pgfqpoint{0.911375in}{1.264831in}}{\pgfqpoint{0.917199in}{1.270655in}}%
\pgfpathcurveto{\pgfqpoint{0.923023in}{1.276479in}}{\pgfqpoint{0.926296in}{1.284379in}}{\pgfqpoint{0.926296in}{1.292616in}}%
\pgfpathcurveto{\pgfqpoint{0.926296in}{1.300852in}}{\pgfqpoint{0.923023in}{1.308752in}}{\pgfqpoint{0.917199in}{1.314576in}}%
\pgfpathcurveto{\pgfqpoint{0.911375in}{1.320400in}}{\pgfqpoint{0.903475in}{1.323672in}}{\pgfqpoint{0.895239in}{1.323672in}}%
\pgfpathcurveto{\pgfqpoint{0.887003in}{1.323672in}}{\pgfqpoint{0.879103in}{1.320400in}}{\pgfqpoint{0.873279in}{1.314576in}}%
\pgfpathcurveto{\pgfqpoint{0.867455in}{1.308752in}}{\pgfqpoint{0.864183in}{1.300852in}}{\pgfqpoint{0.864183in}{1.292616in}}%
\pgfpathcurveto{\pgfqpoint{0.864183in}{1.284379in}}{\pgfqpoint{0.867455in}{1.276479in}}{\pgfqpoint{0.873279in}{1.270655in}}%
\pgfpathcurveto{\pgfqpoint{0.879103in}{1.264831in}}{\pgfqpoint{0.887003in}{1.261559in}}{\pgfqpoint{0.895239in}{1.261559in}}%
\pgfpathclose%
\pgfusepath{stroke,fill}%
\end{pgfscope}%
\begin{pgfscope}%
\pgfpathrectangle{\pgfqpoint{0.100000in}{0.212622in}}{\pgfqpoint{3.696000in}{3.696000in}}%
\pgfusepath{clip}%
\pgfsetbuttcap%
\pgfsetroundjoin%
\definecolor{currentfill}{rgb}{0.121569,0.466667,0.705882}%
\pgfsetfillcolor{currentfill}%
\pgfsetfillopacity{0.632530}%
\pgfsetlinewidth{1.003750pt}%
\definecolor{currentstroke}{rgb}{0.121569,0.466667,0.705882}%
\pgfsetstrokecolor{currentstroke}%
\pgfsetstrokeopacity{0.632530}%
\pgfsetdash{}{0pt}%
\pgfpathmoveto{\pgfqpoint{0.895239in}{1.261559in}}%
\pgfpathcurveto{\pgfqpoint{0.903475in}{1.261559in}}{\pgfqpoint{0.911375in}{1.264831in}}{\pgfqpoint{0.917199in}{1.270655in}}%
\pgfpathcurveto{\pgfqpoint{0.923023in}{1.276479in}}{\pgfqpoint{0.926296in}{1.284379in}}{\pgfqpoint{0.926296in}{1.292616in}}%
\pgfpathcurveto{\pgfqpoint{0.926296in}{1.300852in}}{\pgfqpoint{0.923023in}{1.308752in}}{\pgfqpoint{0.917199in}{1.314576in}}%
\pgfpathcurveto{\pgfqpoint{0.911375in}{1.320400in}}{\pgfqpoint{0.903475in}{1.323672in}}{\pgfqpoint{0.895239in}{1.323672in}}%
\pgfpathcurveto{\pgfqpoint{0.887003in}{1.323672in}}{\pgfqpoint{0.879103in}{1.320400in}}{\pgfqpoint{0.873279in}{1.314576in}}%
\pgfpathcurveto{\pgfqpoint{0.867455in}{1.308752in}}{\pgfqpoint{0.864183in}{1.300852in}}{\pgfqpoint{0.864183in}{1.292616in}}%
\pgfpathcurveto{\pgfqpoint{0.864183in}{1.284379in}}{\pgfqpoint{0.867455in}{1.276479in}}{\pgfqpoint{0.873279in}{1.270655in}}%
\pgfpathcurveto{\pgfqpoint{0.879103in}{1.264831in}}{\pgfqpoint{0.887003in}{1.261559in}}{\pgfqpoint{0.895239in}{1.261559in}}%
\pgfpathclose%
\pgfusepath{stroke,fill}%
\end{pgfscope}%
\begin{pgfscope}%
\pgfpathrectangle{\pgfqpoint{0.100000in}{0.212622in}}{\pgfqpoint{3.696000in}{3.696000in}}%
\pgfusepath{clip}%
\pgfsetbuttcap%
\pgfsetroundjoin%
\definecolor{currentfill}{rgb}{0.121569,0.466667,0.705882}%
\pgfsetfillcolor{currentfill}%
\pgfsetfillopacity{0.632530}%
\pgfsetlinewidth{1.003750pt}%
\definecolor{currentstroke}{rgb}{0.121569,0.466667,0.705882}%
\pgfsetstrokecolor{currentstroke}%
\pgfsetstrokeopacity{0.632530}%
\pgfsetdash{}{0pt}%
\pgfpathmoveto{\pgfqpoint{0.895239in}{1.261559in}}%
\pgfpathcurveto{\pgfqpoint{0.903475in}{1.261559in}}{\pgfqpoint{0.911375in}{1.264831in}}{\pgfqpoint{0.917199in}{1.270655in}}%
\pgfpathcurveto{\pgfqpoint{0.923023in}{1.276479in}}{\pgfqpoint{0.926296in}{1.284379in}}{\pgfqpoint{0.926296in}{1.292616in}}%
\pgfpathcurveto{\pgfqpoint{0.926296in}{1.300852in}}{\pgfqpoint{0.923023in}{1.308752in}}{\pgfqpoint{0.917199in}{1.314576in}}%
\pgfpathcurveto{\pgfqpoint{0.911375in}{1.320400in}}{\pgfqpoint{0.903475in}{1.323672in}}{\pgfqpoint{0.895239in}{1.323672in}}%
\pgfpathcurveto{\pgfqpoint{0.887003in}{1.323672in}}{\pgfqpoint{0.879103in}{1.320400in}}{\pgfqpoint{0.873279in}{1.314576in}}%
\pgfpathcurveto{\pgfqpoint{0.867455in}{1.308752in}}{\pgfqpoint{0.864183in}{1.300852in}}{\pgfqpoint{0.864183in}{1.292616in}}%
\pgfpathcurveto{\pgfqpoint{0.864183in}{1.284379in}}{\pgfqpoint{0.867455in}{1.276479in}}{\pgfqpoint{0.873279in}{1.270655in}}%
\pgfpathcurveto{\pgfqpoint{0.879103in}{1.264831in}}{\pgfqpoint{0.887003in}{1.261559in}}{\pgfqpoint{0.895239in}{1.261559in}}%
\pgfpathclose%
\pgfusepath{stroke,fill}%
\end{pgfscope}%
\begin{pgfscope}%
\pgfpathrectangle{\pgfqpoint{0.100000in}{0.212622in}}{\pgfqpoint{3.696000in}{3.696000in}}%
\pgfusepath{clip}%
\pgfsetbuttcap%
\pgfsetroundjoin%
\definecolor{currentfill}{rgb}{0.121569,0.466667,0.705882}%
\pgfsetfillcolor{currentfill}%
\pgfsetfillopacity{0.632530}%
\pgfsetlinewidth{1.003750pt}%
\definecolor{currentstroke}{rgb}{0.121569,0.466667,0.705882}%
\pgfsetstrokecolor{currentstroke}%
\pgfsetstrokeopacity{0.632530}%
\pgfsetdash{}{0pt}%
\pgfpathmoveto{\pgfqpoint{0.895239in}{1.261559in}}%
\pgfpathcurveto{\pgfqpoint{0.903475in}{1.261559in}}{\pgfqpoint{0.911375in}{1.264831in}}{\pgfqpoint{0.917199in}{1.270655in}}%
\pgfpathcurveto{\pgfqpoint{0.923023in}{1.276479in}}{\pgfqpoint{0.926296in}{1.284379in}}{\pgfqpoint{0.926296in}{1.292616in}}%
\pgfpathcurveto{\pgfqpoint{0.926296in}{1.300852in}}{\pgfqpoint{0.923023in}{1.308752in}}{\pgfqpoint{0.917199in}{1.314576in}}%
\pgfpathcurveto{\pgfqpoint{0.911375in}{1.320400in}}{\pgfqpoint{0.903475in}{1.323672in}}{\pgfqpoint{0.895239in}{1.323672in}}%
\pgfpathcurveto{\pgfqpoint{0.887003in}{1.323672in}}{\pgfqpoint{0.879103in}{1.320400in}}{\pgfqpoint{0.873279in}{1.314576in}}%
\pgfpathcurveto{\pgfqpoint{0.867455in}{1.308752in}}{\pgfqpoint{0.864183in}{1.300852in}}{\pgfqpoint{0.864183in}{1.292616in}}%
\pgfpathcurveto{\pgfqpoint{0.864183in}{1.284379in}}{\pgfqpoint{0.867455in}{1.276479in}}{\pgfqpoint{0.873279in}{1.270655in}}%
\pgfpathcurveto{\pgfqpoint{0.879103in}{1.264831in}}{\pgfqpoint{0.887003in}{1.261559in}}{\pgfqpoint{0.895239in}{1.261559in}}%
\pgfpathclose%
\pgfusepath{stroke,fill}%
\end{pgfscope}%
\begin{pgfscope}%
\pgfpathrectangle{\pgfqpoint{0.100000in}{0.212622in}}{\pgfqpoint{3.696000in}{3.696000in}}%
\pgfusepath{clip}%
\pgfsetbuttcap%
\pgfsetroundjoin%
\definecolor{currentfill}{rgb}{0.121569,0.466667,0.705882}%
\pgfsetfillcolor{currentfill}%
\pgfsetfillopacity{0.632530}%
\pgfsetlinewidth{1.003750pt}%
\definecolor{currentstroke}{rgb}{0.121569,0.466667,0.705882}%
\pgfsetstrokecolor{currentstroke}%
\pgfsetstrokeopacity{0.632530}%
\pgfsetdash{}{0pt}%
\pgfpathmoveto{\pgfqpoint{0.895239in}{1.261559in}}%
\pgfpathcurveto{\pgfqpoint{0.903475in}{1.261559in}}{\pgfqpoint{0.911375in}{1.264831in}}{\pgfqpoint{0.917199in}{1.270655in}}%
\pgfpathcurveto{\pgfqpoint{0.923023in}{1.276479in}}{\pgfqpoint{0.926296in}{1.284379in}}{\pgfqpoint{0.926296in}{1.292616in}}%
\pgfpathcurveto{\pgfqpoint{0.926296in}{1.300852in}}{\pgfqpoint{0.923023in}{1.308752in}}{\pgfqpoint{0.917199in}{1.314576in}}%
\pgfpathcurveto{\pgfqpoint{0.911375in}{1.320400in}}{\pgfqpoint{0.903475in}{1.323672in}}{\pgfqpoint{0.895239in}{1.323672in}}%
\pgfpathcurveto{\pgfqpoint{0.887003in}{1.323672in}}{\pgfqpoint{0.879103in}{1.320400in}}{\pgfqpoint{0.873279in}{1.314576in}}%
\pgfpathcurveto{\pgfqpoint{0.867455in}{1.308752in}}{\pgfqpoint{0.864183in}{1.300852in}}{\pgfqpoint{0.864183in}{1.292616in}}%
\pgfpathcurveto{\pgfqpoint{0.864183in}{1.284379in}}{\pgfqpoint{0.867455in}{1.276479in}}{\pgfqpoint{0.873279in}{1.270655in}}%
\pgfpathcurveto{\pgfqpoint{0.879103in}{1.264831in}}{\pgfqpoint{0.887003in}{1.261559in}}{\pgfqpoint{0.895239in}{1.261559in}}%
\pgfpathclose%
\pgfusepath{stroke,fill}%
\end{pgfscope}%
\begin{pgfscope}%
\pgfpathrectangle{\pgfqpoint{0.100000in}{0.212622in}}{\pgfqpoint{3.696000in}{3.696000in}}%
\pgfusepath{clip}%
\pgfsetbuttcap%
\pgfsetroundjoin%
\definecolor{currentfill}{rgb}{0.121569,0.466667,0.705882}%
\pgfsetfillcolor{currentfill}%
\pgfsetfillopacity{0.632530}%
\pgfsetlinewidth{1.003750pt}%
\definecolor{currentstroke}{rgb}{0.121569,0.466667,0.705882}%
\pgfsetstrokecolor{currentstroke}%
\pgfsetstrokeopacity{0.632530}%
\pgfsetdash{}{0pt}%
\pgfpathmoveto{\pgfqpoint{0.895239in}{1.261559in}}%
\pgfpathcurveto{\pgfqpoint{0.903475in}{1.261559in}}{\pgfqpoint{0.911375in}{1.264831in}}{\pgfqpoint{0.917199in}{1.270655in}}%
\pgfpathcurveto{\pgfqpoint{0.923023in}{1.276479in}}{\pgfqpoint{0.926296in}{1.284379in}}{\pgfqpoint{0.926296in}{1.292616in}}%
\pgfpathcurveto{\pgfqpoint{0.926296in}{1.300852in}}{\pgfqpoint{0.923023in}{1.308752in}}{\pgfqpoint{0.917199in}{1.314576in}}%
\pgfpathcurveto{\pgfqpoint{0.911375in}{1.320400in}}{\pgfqpoint{0.903475in}{1.323672in}}{\pgfqpoint{0.895239in}{1.323672in}}%
\pgfpathcurveto{\pgfqpoint{0.887003in}{1.323672in}}{\pgfqpoint{0.879103in}{1.320400in}}{\pgfqpoint{0.873279in}{1.314576in}}%
\pgfpathcurveto{\pgfqpoint{0.867455in}{1.308752in}}{\pgfqpoint{0.864183in}{1.300852in}}{\pgfqpoint{0.864183in}{1.292616in}}%
\pgfpathcurveto{\pgfqpoint{0.864183in}{1.284379in}}{\pgfqpoint{0.867455in}{1.276479in}}{\pgfqpoint{0.873279in}{1.270655in}}%
\pgfpathcurveto{\pgfqpoint{0.879103in}{1.264831in}}{\pgfqpoint{0.887003in}{1.261559in}}{\pgfqpoint{0.895239in}{1.261559in}}%
\pgfpathclose%
\pgfusepath{stroke,fill}%
\end{pgfscope}%
\begin{pgfscope}%
\pgfpathrectangle{\pgfqpoint{0.100000in}{0.212622in}}{\pgfqpoint{3.696000in}{3.696000in}}%
\pgfusepath{clip}%
\pgfsetbuttcap%
\pgfsetroundjoin%
\definecolor{currentfill}{rgb}{0.121569,0.466667,0.705882}%
\pgfsetfillcolor{currentfill}%
\pgfsetfillopacity{0.632530}%
\pgfsetlinewidth{1.003750pt}%
\definecolor{currentstroke}{rgb}{0.121569,0.466667,0.705882}%
\pgfsetstrokecolor{currentstroke}%
\pgfsetstrokeopacity{0.632530}%
\pgfsetdash{}{0pt}%
\pgfpathmoveto{\pgfqpoint{0.895239in}{1.261559in}}%
\pgfpathcurveto{\pgfqpoint{0.903475in}{1.261559in}}{\pgfqpoint{0.911375in}{1.264831in}}{\pgfqpoint{0.917199in}{1.270655in}}%
\pgfpathcurveto{\pgfqpoint{0.923023in}{1.276479in}}{\pgfqpoint{0.926296in}{1.284379in}}{\pgfqpoint{0.926296in}{1.292616in}}%
\pgfpathcurveto{\pgfqpoint{0.926296in}{1.300852in}}{\pgfqpoint{0.923023in}{1.308752in}}{\pgfqpoint{0.917199in}{1.314576in}}%
\pgfpathcurveto{\pgfqpoint{0.911375in}{1.320400in}}{\pgfqpoint{0.903475in}{1.323672in}}{\pgfqpoint{0.895239in}{1.323672in}}%
\pgfpathcurveto{\pgfqpoint{0.887003in}{1.323672in}}{\pgfqpoint{0.879103in}{1.320400in}}{\pgfqpoint{0.873279in}{1.314576in}}%
\pgfpathcurveto{\pgfqpoint{0.867455in}{1.308752in}}{\pgfqpoint{0.864183in}{1.300852in}}{\pgfqpoint{0.864183in}{1.292616in}}%
\pgfpathcurveto{\pgfqpoint{0.864183in}{1.284379in}}{\pgfqpoint{0.867455in}{1.276479in}}{\pgfqpoint{0.873279in}{1.270655in}}%
\pgfpathcurveto{\pgfqpoint{0.879103in}{1.264831in}}{\pgfqpoint{0.887003in}{1.261559in}}{\pgfqpoint{0.895239in}{1.261559in}}%
\pgfpathclose%
\pgfusepath{stroke,fill}%
\end{pgfscope}%
\begin{pgfscope}%
\pgfpathrectangle{\pgfqpoint{0.100000in}{0.212622in}}{\pgfqpoint{3.696000in}{3.696000in}}%
\pgfusepath{clip}%
\pgfsetbuttcap%
\pgfsetroundjoin%
\definecolor{currentfill}{rgb}{0.121569,0.466667,0.705882}%
\pgfsetfillcolor{currentfill}%
\pgfsetfillopacity{0.632530}%
\pgfsetlinewidth{1.003750pt}%
\definecolor{currentstroke}{rgb}{0.121569,0.466667,0.705882}%
\pgfsetstrokecolor{currentstroke}%
\pgfsetstrokeopacity{0.632530}%
\pgfsetdash{}{0pt}%
\pgfpathmoveto{\pgfqpoint{0.895239in}{1.261559in}}%
\pgfpathcurveto{\pgfqpoint{0.903475in}{1.261559in}}{\pgfqpoint{0.911375in}{1.264831in}}{\pgfqpoint{0.917199in}{1.270655in}}%
\pgfpathcurveto{\pgfqpoint{0.923023in}{1.276479in}}{\pgfqpoint{0.926296in}{1.284379in}}{\pgfqpoint{0.926296in}{1.292616in}}%
\pgfpathcurveto{\pgfqpoint{0.926296in}{1.300852in}}{\pgfqpoint{0.923023in}{1.308752in}}{\pgfqpoint{0.917199in}{1.314576in}}%
\pgfpathcurveto{\pgfqpoint{0.911375in}{1.320400in}}{\pgfqpoint{0.903475in}{1.323672in}}{\pgfqpoint{0.895239in}{1.323672in}}%
\pgfpathcurveto{\pgfqpoint{0.887003in}{1.323672in}}{\pgfqpoint{0.879103in}{1.320400in}}{\pgfqpoint{0.873279in}{1.314576in}}%
\pgfpathcurveto{\pgfqpoint{0.867455in}{1.308752in}}{\pgfqpoint{0.864183in}{1.300852in}}{\pgfqpoint{0.864183in}{1.292616in}}%
\pgfpathcurveto{\pgfqpoint{0.864183in}{1.284379in}}{\pgfqpoint{0.867455in}{1.276479in}}{\pgfqpoint{0.873279in}{1.270655in}}%
\pgfpathcurveto{\pgfqpoint{0.879103in}{1.264831in}}{\pgfqpoint{0.887003in}{1.261559in}}{\pgfqpoint{0.895239in}{1.261559in}}%
\pgfpathclose%
\pgfusepath{stroke,fill}%
\end{pgfscope}%
\begin{pgfscope}%
\pgfpathrectangle{\pgfqpoint{0.100000in}{0.212622in}}{\pgfqpoint{3.696000in}{3.696000in}}%
\pgfusepath{clip}%
\pgfsetbuttcap%
\pgfsetroundjoin%
\definecolor{currentfill}{rgb}{0.121569,0.466667,0.705882}%
\pgfsetfillcolor{currentfill}%
\pgfsetfillopacity{0.632530}%
\pgfsetlinewidth{1.003750pt}%
\definecolor{currentstroke}{rgb}{0.121569,0.466667,0.705882}%
\pgfsetstrokecolor{currentstroke}%
\pgfsetstrokeopacity{0.632530}%
\pgfsetdash{}{0pt}%
\pgfpathmoveto{\pgfqpoint{0.895239in}{1.261559in}}%
\pgfpathcurveto{\pgfqpoint{0.903475in}{1.261559in}}{\pgfqpoint{0.911375in}{1.264831in}}{\pgfqpoint{0.917199in}{1.270655in}}%
\pgfpathcurveto{\pgfqpoint{0.923023in}{1.276479in}}{\pgfqpoint{0.926296in}{1.284379in}}{\pgfqpoint{0.926296in}{1.292616in}}%
\pgfpathcurveto{\pgfqpoint{0.926296in}{1.300852in}}{\pgfqpoint{0.923023in}{1.308752in}}{\pgfqpoint{0.917199in}{1.314576in}}%
\pgfpathcurveto{\pgfqpoint{0.911375in}{1.320400in}}{\pgfqpoint{0.903475in}{1.323672in}}{\pgfqpoint{0.895239in}{1.323672in}}%
\pgfpathcurveto{\pgfqpoint{0.887003in}{1.323672in}}{\pgfqpoint{0.879103in}{1.320400in}}{\pgfqpoint{0.873279in}{1.314576in}}%
\pgfpathcurveto{\pgfqpoint{0.867455in}{1.308752in}}{\pgfqpoint{0.864183in}{1.300852in}}{\pgfqpoint{0.864183in}{1.292616in}}%
\pgfpathcurveto{\pgfqpoint{0.864183in}{1.284379in}}{\pgfqpoint{0.867455in}{1.276479in}}{\pgfqpoint{0.873279in}{1.270655in}}%
\pgfpathcurveto{\pgfqpoint{0.879103in}{1.264831in}}{\pgfqpoint{0.887003in}{1.261559in}}{\pgfqpoint{0.895239in}{1.261559in}}%
\pgfpathclose%
\pgfusepath{stroke,fill}%
\end{pgfscope}%
\begin{pgfscope}%
\pgfpathrectangle{\pgfqpoint{0.100000in}{0.212622in}}{\pgfqpoint{3.696000in}{3.696000in}}%
\pgfusepath{clip}%
\pgfsetbuttcap%
\pgfsetroundjoin%
\definecolor{currentfill}{rgb}{0.121569,0.466667,0.705882}%
\pgfsetfillcolor{currentfill}%
\pgfsetfillopacity{0.632530}%
\pgfsetlinewidth{1.003750pt}%
\definecolor{currentstroke}{rgb}{0.121569,0.466667,0.705882}%
\pgfsetstrokecolor{currentstroke}%
\pgfsetstrokeopacity{0.632530}%
\pgfsetdash{}{0pt}%
\pgfpathmoveto{\pgfqpoint{0.895239in}{1.261559in}}%
\pgfpathcurveto{\pgfqpoint{0.903475in}{1.261559in}}{\pgfqpoint{0.911375in}{1.264831in}}{\pgfqpoint{0.917199in}{1.270655in}}%
\pgfpathcurveto{\pgfqpoint{0.923023in}{1.276479in}}{\pgfqpoint{0.926296in}{1.284379in}}{\pgfqpoint{0.926296in}{1.292616in}}%
\pgfpathcurveto{\pgfqpoint{0.926296in}{1.300852in}}{\pgfqpoint{0.923023in}{1.308752in}}{\pgfqpoint{0.917199in}{1.314576in}}%
\pgfpathcurveto{\pgfqpoint{0.911375in}{1.320400in}}{\pgfqpoint{0.903475in}{1.323672in}}{\pgfqpoint{0.895239in}{1.323672in}}%
\pgfpathcurveto{\pgfqpoint{0.887003in}{1.323672in}}{\pgfqpoint{0.879103in}{1.320400in}}{\pgfqpoint{0.873279in}{1.314576in}}%
\pgfpathcurveto{\pgfqpoint{0.867455in}{1.308752in}}{\pgfqpoint{0.864183in}{1.300852in}}{\pgfqpoint{0.864183in}{1.292616in}}%
\pgfpathcurveto{\pgfqpoint{0.864183in}{1.284379in}}{\pgfqpoint{0.867455in}{1.276479in}}{\pgfqpoint{0.873279in}{1.270655in}}%
\pgfpathcurveto{\pgfqpoint{0.879103in}{1.264831in}}{\pgfqpoint{0.887003in}{1.261559in}}{\pgfqpoint{0.895239in}{1.261559in}}%
\pgfpathclose%
\pgfusepath{stroke,fill}%
\end{pgfscope}%
\begin{pgfscope}%
\pgfpathrectangle{\pgfqpoint{0.100000in}{0.212622in}}{\pgfqpoint{3.696000in}{3.696000in}}%
\pgfusepath{clip}%
\pgfsetbuttcap%
\pgfsetroundjoin%
\definecolor{currentfill}{rgb}{0.121569,0.466667,0.705882}%
\pgfsetfillcolor{currentfill}%
\pgfsetfillopacity{0.632530}%
\pgfsetlinewidth{1.003750pt}%
\definecolor{currentstroke}{rgb}{0.121569,0.466667,0.705882}%
\pgfsetstrokecolor{currentstroke}%
\pgfsetstrokeopacity{0.632530}%
\pgfsetdash{}{0pt}%
\pgfpathmoveto{\pgfqpoint{0.895239in}{1.261559in}}%
\pgfpathcurveto{\pgfqpoint{0.903475in}{1.261559in}}{\pgfqpoint{0.911375in}{1.264831in}}{\pgfqpoint{0.917199in}{1.270655in}}%
\pgfpathcurveto{\pgfqpoint{0.923023in}{1.276479in}}{\pgfqpoint{0.926296in}{1.284379in}}{\pgfqpoint{0.926296in}{1.292616in}}%
\pgfpathcurveto{\pgfqpoint{0.926296in}{1.300852in}}{\pgfqpoint{0.923023in}{1.308752in}}{\pgfqpoint{0.917199in}{1.314576in}}%
\pgfpathcurveto{\pgfqpoint{0.911375in}{1.320400in}}{\pgfqpoint{0.903475in}{1.323672in}}{\pgfqpoint{0.895239in}{1.323672in}}%
\pgfpathcurveto{\pgfqpoint{0.887003in}{1.323672in}}{\pgfqpoint{0.879103in}{1.320400in}}{\pgfqpoint{0.873279in}{1.314576in}}%
\pgfpathcurveto{\pgfqpoint{0.867455in}{1.308752in}}{\pgfqpoint{0.864183in}{1.300852in}}{\pgfqpoint{0.864183in}{1.292616in}}%
\pgfpathcurveto{\pgfqpoint{0.864183in}{1.284379in}}{\pgfqpoint{0.867455in}{1.276479in}}{\pgfqpoint{0.873279in}{1.270655in}}%
\pgfpathcurveto{\pgfqpoint{0.879103in}{1.264831in}}{\pgfqpoint{0.887003in}{1.261559in}}{\pgfqpoint{0.895239in}{1.261559in}}%
\pgfpathclose%
\pgfusepath{stroke,fill}%
\end{pgfscope}%
\begin{pgfscope}%
\pgfpathrectangle{\pgfqpoint{0.100000in}{0.212622in}}{\pgfqpoint{3.696000in}{3.696000in}}%
\pgfusepath{clip}%
\pgfsetbuttcap%
\pgfsetroundjoin%
\definecolor{currentfill}{rgb}{0.121569,0.466667,0.705882}%
\pgfsetfillcolor{currentfill}%
\pgfsetfillopacity{0.632530}%
\pgfsetlinewidth{1.003750pt}%
\definecolor{currentstroke}{rgb}{0.121569,0.466667,0.705882}%
\pgfsetstrokecolor{currentstroke}%
\pgfsetstrokeopacity{0.632530}%
\pgfsetdash{}{0pt}%
\pgfpathmoveto{\pgfqpoint{0.895239in}{1.261559in}}%
\pgfpathcurveto{\pgfqpoint{0.903475in}{1.261559in}}{\pgfqpoint{0.911375in}{1.264831in}}{\pgfqpoint{0.917199in}{1.270655in}}%
\pgfpathcurveto{\pgfqpoint{0.923023in}{1.276479in}}{\pgfqpoint{0.926296in}{1.284379in}}{\pgfqpoint{0.926296in}{1.292616in}}%
\pgfpathcurveto{\pgfqpoint{0.926296in}{1.300852in}}{\pgfqpoint{0.923023in}{1.308752in}}{\pgfqpoint{0.917199in}{1.314576in}}%
\pgfpathcurveto{\pgfqpoint{0.911375in}{1.320400in}}{\pgfqpoint{0.903475in}{1.323672in}}{\pgfqpoint{0.895239in}{1.323672in}}%
\pgfpathcurveto{\pgfqpoint{0.887003in}{1.323672in}}{\pgfqpoint{0.879103in}{1.320400in}}{\pgfqpoint{0.873279in}{1.314576in}}%
\pgfpathcurveto{\pgfqpoint{0.867455in}{1.308752in}}{\pgfqpoint{0.864183in}{1.300852in}}{\pgfqpoint{0.864183in}{1.292616in}}%
\pgfpathcurveto{\pgfqpoint{0.864183in}{1.284379in}}{\pgfqpoint{0.867455in}{1.276479in}}{\pgfqpoint{0.873279in}{1.270655in}}%
\pgfpathcurveto{\pgfqpoint{0.879103in}{1.264831in}}{\pgfqpoint{0.887003in}{1.261559in}}{\pgfqpoint{0.895239in}{1.261559in}}%
\pgfpathclose%
\pgfusepath{stroke,fill}%
\end{pgfscope}%
\begin{pgfscope}%
\pgfpathrectangle{\pgfqpoint{0.100000in}{0.212622in}}{\pgfqpoint{3.696000in}{3.696000in}}%
\pgfusepath{clip}%
\pgfsetbuttcap%
\pgfsetroundjoin%
\definecolor{currentfill}{rgb}{0.121569,0.466667,0.705882}%
\pgfsetfillcolor{currentfill}%
\pgfsetfillopacity{0.632530}%
\pgfsetlinewidth{1.003750pt}%
\definecolor{currentstroke}{rgb}{0.121569,0.466667,0.705882}%
\pgfsetstrokecolor{currentstroke}%
\pgfsetstrokeopacity{0.632530}%
\pgfsetdash{}{0pt}%
\pgfpathmoveto{\pgfqpoint{0.895239in}{1.261559in}}%
\pgfpathcurveto{\pgfqpoint{0.903475in}{1.261559in}}{\pgfqpoint{0.911375in}{1.264831in}}{\pgfqpoint{0.917199in}{1.270655in}}%
\pgfpathcurveto{\pgfqpoint{0.923023in}{1.276479in}}{\pgfqpoint{0.926296in}{1.284379in}}{\pgfqpoint{0.926296in}{1.292616in}}%
\pgfpathcurveto{\pgfqpoint{0.926296in}{1.300852in}}{\pgfqpoint{0.923023in}{1.308752in}}{\pgfqpoint{0.917199in}{1.314576in}}%
\pgfpathcurveto{\pgfqpoint{0.911375in}{1.320400in}}{\pgfqpoint{0.903475in}{1.323672in}}{\pgfqpoint{0.895239in}{1.323672in}}%
\pgfpathcurveto{\pgfqpoint{0.887003in}{1.323672in}}{\pgfqpoint{0.879103in}{1.320400in}}{\pgfqpoint{0.873279in}{1.314576in}}%
\pgfpathcurveto{\pgfqpoint{0.867455in}{1.308752in}}{\pgfqpoint{0.864183in}{1.300852in}}{\pgfqpoint{0.864183in}{1.292616in}}%
\pgfpathcurveto{\pgfqpoint{0.864183in}{1.284379in}}{\pgfqpoint{0.867455in}{1.276479in}}{\pgfqpoint{0.873279in}{1.270655in}}%
\pgfpathcurveto{\pgfqpoint{0.879103in}{1.264831in}}{\pgfqpoint{0.887003in}{1.261559in}}{\pgfqpoint{0.895239in}{1.261559in}}%
\pgfpathclose%
\pgfusepath{stroke,fill}%
\end{pgfscope}%
\begin{pgfscope}%
\pgfpathrectangle{\pgfqpoint{0.100000in}{0.212622in}}{\pgfqpoint{3.696000in}{3.696000in}}%
\pgfusepath{clip}%
\pgfsetbuttcap%
\pgfsetroundjoin%
\definecolor{currentfill}{rgb}{0.121569,0.466667,0.705882}%
\pgfsetfillcolor{currentfill}%
\pgfsetfillopacity{0.632530}%
\pgfsetlinewidth{1.003750pt}%
\definecolor{currentstroke}{rgb}{0.121569,0.466667,0.705882}%
\pgfsetstrokecolor{currentstroke}%
\pgfsetstrokeopacity{0.632530}%
\pgfsetdash{}{0pt}%
\pgfpathmoveto{\pgfqpoint{0.895239in}{1.261559in}}%
\pgfpathcurveto{\pgfqpoint{0.903475in}{1.261559in}}{\pgfqpoint{0.911375in}{1.264831in}}{\pgfqpoint{0.917199in}{1.270655in}}%
\pgfpathcurveto{\pgfqpoint{0.923023in}{1.276479in}}{\pgfqpoint{0.926296in}{1.284379in}}{\pgfqpoint{0.926296in}{1.292616in}}%
\pgfpathcurveto{\pgfqpoint{0.926296in}{1.300852in}}{\pgfqpoint{0.923023in}{1.308752in}}{\pgfqpoint{0.917199in}{1.314576in}}%
\pgfpathcurveto{\pgfqpoint{0.911375in}{1.320400in}}{\pgfqpoint{0.903475in}{1.323672in}}{\pgfqpoint{0.895239in}{1.323672in}}%
\pgfpathcurveto{\pgfqpoint{0.887003in}{1.323672in}}{\pgfqpoint{0.879103in}{1.320400in}}{\pgfqpoint{0.873279in}{1.314576in}}%
\pgfpathcurveto{\pgfqpoint{0.867455in}{1.308752in}}{\pgfqpoint{0.864183in}{1.300852in}}{\pgfqpoint{0.864183in}{1.292616in}}%
\pgfpathcurveto{\pgfqpoint{0.864183in}{1.284379in}}{\pgfqpoint{0.867455in}{1.276479in}}{\pgfqpoint{0.873279in}{1.270655in}}%
\pgfpathcurveto{\pgfqpoint{0.879103in}{1.264831in}}{\pgfqpoint{0.887003in}{1.261559in}}{\pgfqpoint{0.895239in}{1.261559in}}%
\pgfpathclose%
\pgfusepath{stroke,fill}%
\end{pgfscope}%
\begin{pgfscope}%
\pgfpathrectangle{\pgfqpoint{0.100000in}{0.212622in}}{\pgfqpoint{3.696000in}{3.696000in}}%
\pgfusepath{clip}%
\pgfsetbuttcap%
\pgfsetroundjoin%
\definecolor{currentfill}{rgb}{0.121569,0.466667,0.705882}%
\pgfsetfillcolor{currentfill}%
\pgfsetfillopacity{0.632530}%
\pgfsetlinewidth{1.003750pt}%
\definecolor{currentstroke}{rgb}{0.121569,0.466667,0.705882}%
\pgfsetstrokecolor{currentstroke}%
\pgfsetstrokeopacity{0.632530}%
\pgfsetdash{}{0pt}%
\pgfpathmoveto{\pgfqpoint{0.895239in}{1.261559in}}%
\pgfpathcurveto{\pgfqpoint{0.903475in}{1.261559in}}{\pgfqpoint{0.911375in}{1.264831in}}{\pgfqpoint{0.917199in}{1.270655in}}%
\pgfpathcurveto{\pgfqpoint{0.923023in}{1.276479in}}{\pgfqpoint{0.926296in}{1.284379in}}{\pgfqpoint{0.926296in}{1.292616in}}%
\pgfpathcurveto{\pgfqpoint{0.926296in}{1.300852in}}{\pgfqpoint{0.923023in}{1.308752in}}{\pgfqpoint{0.917199in}{1.314576in}}%
\pgfpathcurveto{\pgfqpoint{0.911375in}{1.320400in}}{\pgfqpoint{0.903475in}{1.323672in}}{\pgfqpoint{0.895239in}{1.323672in}}%
\pgfpathcurveto{\pgfqpoint{0.887003in}{1.323672in}}{\pgfqpoint{0.879103in}{1.320400in}}{\pgfqpoint{0.873279in}{1.314576in}}%
\pgfpathcurveto{\pgfqpoint{0.867455in}{1.308752in}}{\pgfqpoint{0.864183in}{1.300852in}}{\pgfqpoint{0.864183in}{1.292616in}}%
\pgfpathcurveto{\pgfqpoint{0.864183in}{1.284379in}}{\pgfqpoint{0.867455in}{1.276479in}}{\pgfqpoint{0.873279in}{1.270655in}}%
\pgfpathcurveto{\pgfqpoint{0.879103in}{1.264831in}}{\pgfqpoint{0.887003in}{1.261559in}}{\pgfqpoint{0.895239in}{1.261559in}}%
\pgfpathclose%
\pgfusepath{stroke,fill}%
\end{pgfscope}%
\begin{pgfscope}%
\pgfpathrectangle{\pgfqpoint{0.100000in}{0.212622in}}{\pgfqpoint{3.696000in}{3.696000in}}%
\pgfusepath{clip}%
\pgfsetbuttcap%
\pgfsetroundjoin%
\definecolor{currentfill}{rgb}{0.121569,0.466667,0.705882}%
\pgfsetfillcolor{currentfill}%
\pgfsetfillopacity{0.632530}%
\pgfsetlinewidth{1.003750pt}%
\definecolor{currentstroke}{rgb}{0.121569,0.466667,0.705882}%
\pgfsetstrokecolor{currentstroke}%
\pgfsetstrokeopacity{0.632530}%
\pgfsetdash{}{0pt}%
\pgfpathmoveto{\pgfqpoint{0.895239in}{1.261559in}}%
\pgfpathcurveto{\pgfqpoint{0.903475in}{1.261559in}}{\pgfqpoint{0.911375in}{1.264831in}}{\pgfqpoint{0.917199in}{1.270655in}}%
\pgfpathcurveto{\pgfqpoint{0.923023in}{1.276479in}}{\pgfqpoint{0.926296in}{1.284379in}}{\pgfqpoint{0.926296in}{1.292616in}}%
\pgfpathcurveto{\pgfqpoint{0.926296in}{1.300852in}}{\pgfqpoint{0.923023in}{1.308752in}}{\pgfqpoint{0.917199in}{1.314576in}}%
\pgfpathcurveto{\pgfqpoint{0.911375in}{1.320400in}}{\pgfqpoint{0.903475in}{1.323672in}}{\pgfqpoint{0.895239in}{1.323672in}}%
\pgfpathcurveto{\pgfqpoint{0.887003in}{1.323672in}}{\pgfqpoint{0.879103in}{1.320400in}}{\pgfqpoint{0.873279in}{1.314576in}}%
\pgfpathcurveto{\pgfqpoint{0.867455in}{1.308752in}}{\pgfqpoint{0.864183in}{1.300852in}}{\pgfqpoint{0.864183in}{1.292616in}}%
\pgfpathcurveto{\pgfqpoint{0.864183in}{1.284379in}}{\pgfqpoint{0.867455in}{1.276479in}}{\pgfqpoint{0.873279in}{1.270655in}}%
\pgfpathcurveto{\pgfqpoint{0.879103in}{1.264831in}}{\pgfqpoint{0.887003in}{1.261559in}}{\pgfqpoint{0.895239in}{1.261559in}}%
\pgfpathclose%
\pgfusepath{stroke,fill}%
\end{pgfscope}%
\begin{pgfscope}%
\pgfpathrectangle{\pgfqpoint{0.100000in}{0.212622in}}{\pgfqpoint{3.696000in}{3.696000in}}%
\pgfusepath{clip}%
\pgfsetbuttcap%
\pgfsetroundjoin%
\definecolor{currentfill}{rgb}{0.121569,0.466667,0.705882}%
\pgfsetfillcolor{currentfill}%
\pgfsetfillopacity{0.632530}%
\pgfsetlinewidth{1.003750pt}%
\definecolor{currentstroke}{rgb}{0.121569,0.466667,0.705882}%
\pgfsetstrokecolor{currentstroke}%
\pgfsetstrokeopacity{0.632530}%
\pgfsetdash{}{0pt}%
\pgfpathmoveto{\pgfqpoint{0.895239in}{1.261559in}}%
\pgfpathcurveto{\pgfqpoint{0.903475in}{1.261559in}}{\pgfqpoint{0.911375in}{1.264831in}}{\pgfqpoint{0.917199in}{1.270655in}}%
\pgfpathcurveto{\pgfqpoint{0.923023in}{1.276479in}}{\pgfqpoint{0.926296in}{1.284379in}}{\pgfqpoint{0.926296in}{1.292616in}}%
\pgfpathcurveto{\pgfqpoint{0.926296in}{1.300852in}}{\pgfqpoint{0.923023in}{1.308752in}}{\pgfqpoint{0.917199in}{1.314576in}}%
\pgfpathcurveto{\pgfqpoint{0.911375in}{1.320400in}}{\pgfqpoint{0.903475in}{1.323672in}}{\pgfqpoint{0.895239in}{1.323672in}}%
\pgfpathcurveto{\pgfqpoint{0.887003in}{1.323672in}}{\pgfqpoint{0.879103in}{1.320400in}}{\pgfqpoint{0.873279in}{1.314576in}}%
\pgfpathcurveto{\pgfqpoint{0.867455in}{1.308752in}}{\pgfqpoint{0.864183in}{1.300852in}}{\pgfqpoint{0.864183in}{1.292616in}}%
\pgfpathcurveto{\pgfqpoint{0.864183in}{1.284379in}}{\pgfqpoint{0.867455in}{1.276479in}}{\pgfqpoint{0.873279in}{1.270655in}}%
\pgfpathcurveto{\pgfqpoint{0.879103in}{1.264831in}}{\pgfqpoint{0.887003in}{1.261559in}}{\pgfqpoint{0.895239in}{1.261559in}}%
\pgfpathclose%
\pgfusepath{stroke,fill}%
\end{pgfscope}%
\begin{pgfscope}%
\pgfpathrectangle{\pgfqpoint{0.100000in}{0.212622in}}{\pgfqpoint{3.696000in}{3.696000in}}%
\pgfusepath{clip}%
\pgfsetbuttcap%
\pgfsetroundjoin%
\definecolor{currentfill}{rgb}{0.121569,0.466667,0.705882}%
\pgfsetfillcolor{currentfill}%
\pgfsetfillopacity{0.632530}%
\pgfsetlinewidth{1.003750pt}%
\definecolor{currentstroke}{rgb}{0.121569,0.466667,0.705882}%
\pgfsetstrokecolor{currentstroke}%
\pgfsetstrokeopacity{0.632530}%
\pgfsetdash{}{0pt}%
\pgfpathmoveto{\pgfqpoint{0.895239in}{1.261559in}}%
\pgfpathcurveto{\pgfqpoint{0.903475in}{1.261559in}}{\pgfqpoint{0.911375in}{1.264831in}}{\pgfqpoint{0.917199in}{1.270655in}}%
\pgfpathcurveto{\pgfqpoint{0.923023in}{1.276479in}}{\pgfqpoint{0.926296in}{1.284379in}}{\pgfqpoint{0.926296in}{1.292616in}}%
\pgfpathcurveto{\pgfqpoint{0.926296in}{1.300852in}}{\pgfqpoint{0.923023in}{1.308752in}}{\pgfqpoint{0.917199in}{1.314576in}}%
\pgfpathcurveto{\pgfqpoint{0.911375in}{1.320400in}}{\pgfqpoint{0.903475in}{1.323672in}}{\pgfqpoint{0.895239in}{1.323672in}}%
\pgfpathcurveto{\pgfqpoint{0.887003in}{1.323672in}}{\pgfqpoint{0.879103in}{1.320400in}}{\pgfqpoint{0.873279in}{1.314576in}}%
\pgfpathcurveto{\pgfqpoint{0.867455in}{1.308752in}}{\pgfqpoint{0.864183in}{1.300852in}}{\pgfqpoint{0.864183in}{1.292616in}}%
\pgfpathcurveto{\pgfqpoint{0.864183in}{1.284379in}}{\pgfqpoint{0.867455in}{1.276479in}}{\pgfqpoint{0.873279in}{1.270655in}}%
\pgfpathcurveto{\pgfqpoint{0.879103in}{1.264831in}}{\pgfqpoint{0.887003in}{1.261559in}}{\pgfqpoint{0.895239in}{1.261559in}}%
\pgfpathclose%
\pgfusepath{stroke,fill}%
\end{pgfscope}%
\begin{pgfscope}%
\pgfpathrectangle{\pgfqpoint{0.100000in}{0.212622in}}{\pgfqpoint{3.696000in}{3.696000in}}%
\pgfusepath{clip}%
\pgfsetbuttcap%
\pgfsetroundjoin%
\definecolor{currentfill}{rgb}{0.121569,0.466667,0.705882}%
\pgfsetfillcolor{currentfill}%
\pgfsetfillopacity{0.632530}%
\pgfsetlinewidth{1.003750pt}%
\definecolor{currentstroke}{rgb}{0.121569,0.466667,0.705882}%
\pgfsetstrokecolor{currentstroke}%
\pgfsetstrokeopacity{0.632530}%
\pgfsetdash{}{0pt}%
\pgfpathmoveto{\pgfqpoint{0.895239in}{1.261559in}}%
\pgfpathcurveto{\pgfqpoint{0.903475in}{1.261559in}}{\pgfqpoint{0.911375in}{1.264831in}}{\pgfqpoint{0.917199in}{1.270655in}}%
\pgfpathcurveto{\pgfqpoint{0.923023in}{1.276479in}}{\pgfqpoint{0.926296in}{1.284379in}}{\pgfqpoint{0.926296in}{1.292616in}}%
\pgfpathcurveto{\pgfqpoint{0.926296in}{1.300852in}}{\pgfqpoint{0.923023in}{1.308752in}}{\pgfqpoint{0.917199in}{1.314576in}}%
\pgfpathcurveto{\pgfqpoint{0.911375in}{1.320400in}}{\pgfqpoint{0.903475in}{1.323672in}}{\pgfqpoint{0.895239in}{1.323672in}}%
\pgfpathcurveto{\pgfqpoint{0.887003in}{1.323672in}}{\pgfqpoint{0.879103in}{1.320400in}}{\pgfqpoint{0.873279in}{1.314576in}}%
\pgfpathcurveto{\pgfqpoint{0.867455in}{1.308752in}}{\pgfqpoint{0.864183in}{1.300852in}}{\pgfqpoint{0.864183in}{1.292616in}}%
\pgfpathcurveto{\pgfqpoint{0.864183in}{1.284379in}}{\pgfqpoint{0.867455in}{1.276479in}}{\pgfqpoint{0.873279in}{1.270655in}}%
\pgfpathcurveto{\pgfqpoint{0.879103in}{1.264831in}}{\pgfqpoint{0.887003in}{1.261559in}}{\pgfqpoint{0.895239in}{1.261559in}}%
\pgfpathclose%
\pgfusepath{stroke,fill}%
\end{pgfscope}%
\begin{pgfscope}%
\pgfpathrectangle{\pgfqpoint{0.100000in}{0.212622in}}{\pgfqpoint{3.696000in}{3.696000in}}%
\pgfusepath{clip}%
\pgfsetbuttcap%
\pgfsetroundjoin%
\definecolor{currentfill}{rgb}{0.121569,0.466667,0.705882}%
\pgfsetfillcolor{currentfill}%
\pgfsetfillopacity{0.632530}%
\pgfsetlinewidth{1.003750pt}%
\definecolor{currentstroke}{rgb}{0.121569,0.466667,0.705882}%
\pgfsetstrokecolor{currentstroke}%
\pgfsetstrokeopacity{0.632530}%
\pgfsetdash{}{0pt}%
\pgfpathmoveto{\pgfqpoint{0.895239in}{1.261559in}}%
\pgfpathcurveto{\pgfqpoint{0.903475in}{1.261559in}}{\pgfqpoint{0.911375in}{1.264831in}}{\pgfqpoint{0.917199in}{1.270655in}}%
\pgfpathcurveto{\pgfqpoint{0.923023in}{1.276479in}}{\pgfqpoint{0.926296in}{1.284379in}}{\pgfqpoint{0.926296in}{1.292616in}}%
\pgfpathcurveto{\pgfqpoint{0.926296in}{1.300852in}}{\pgfqpoint{0.923023in}{1.308752in}}{\pgfqpoint{0.917199in}{1.314576in}}%
\pgfpathcurveto{\pgfqpoint{0.911375in}{1.320400in}}{\pgfqpoint{0.903475in}{1.323672in}}{\pgfqpoint{0.895239in}{1.323672in}}%
\pgfpathcurveto{\pgfqpoint{0.887003in}{1.323672in}}{\pgfqpoint{0.879103in}{1.320400in}}{\pgfqpoint{0.873279in}{1.314576in}}%
\pgfpathcurveto{\pgfqpoint{0.867455in}{1.308752in}}{\pgfqpoint{0.864183in}{1.300852in}}{\pgfqpoint{0.864183in}{1.292616in}}%
\pgfpathcurveto{\pgfqpoint{0.864183in}{1.284379in}}{\pgfqpoint{0.867455in}{1.276479in}}{\pgfqpoint{0.873279in}{1.270655in}}%
\pgfpathcurveto{\pgfqpoint{0.879103in}{1.264831in}}{\pgfqpoint{0.887003in}{1.261559in}}{\pgfqpoint{0.895239in}{1.261559in}}%
\pgfpathclose%
\pgfusepath{stroke,fill}%
\end{pgfscope}%
\begin{pgfscope}%
\pgfpathrectangle{\pgfqpoint{0.100000in}{0.212622in}}{\pgfqpoint{3.696000in}{3.696000in}}%
\pgfusepath{clip}%
\pgfsetbuttcap%
\pgfsetroundjoin%
\definecolor{currentfill}{rgb}{0.121569,0.466667,0.705882}%
\pgfsetfillcolor{currentfill}%
\pgfsetfillopacity{0.632530}%
\pgfsetlinewidth{1.003750pt}%
\definecolor{currentstroke}{rgb}{0.121569,0.466667,0.705882}%
\pgfsetstrokecolor{currentstroke}%
\pgfsetstrokeopacity{0.632530}%
\pgfsetdash{}{0pt}%
\pgfpathmoveto{\pgfqpoint{0.895239in}{1.261559in}}%
\pgfpathcurveto{\pgfqpoint{0.903475in}{1.261559in}}{\pgfqpoint{0.911375in}{1.264831in}}{\pgfqpoint{0.917199in}{1.270655in}}%
\pgfpathcurveto{\pgfqpoint{0.923023in}{1.276479in}}{\pgfqpoint{0.926296in}{1.284379in}}{\pgfqpoint{0.926296in}{1.292616in}}%
\pgfpathcurveto{\pgfqpoint{0.926296in}{1.300852in}}{\pgfqpoint{0.923023in}{1.308752in}}{\pgfqpoint{0.917199in}{1.314576in}}%
\pgfpathcurveto{\pgfqpoint{0.911375in}{1.320400in}}{\pgfqpoint{0.903475in}{1.323672in}}{\pgfqpoint{0.895239in}{1.323672in}}%
\pgfpathcurveto{\pgfqpoint{0.887003in}{1.323672in}}{\pgfqpoint{0.879103in}{1.320400in}}{\pgfqpoint{0.873279in}{1.314576in}}%
\pgfpathcurveto{\pgfqpoint{0.867455in}{1.308752in}}{\pgfqpoint{0.864183in}{1.300852in}}{\pgfqpoint{0.864183in}{1.292616in}}%
\pgfpathcurveto{\pgfqpoint{0.864183in}{1.284379in}}{\pgfqpoint{0.867455in}{1.276479in}}{\pgfqpoint{0.873279in}{1.270655in}}%
\pgfpathcurveto{\pgfqpoint{0.879103in}{1.264831in}}{\pgfqpoint{0.887003in}{1.261559in}}{\pgfqpoint{0.895239in}{1.261559in}}%
\pgfpathclose%
\pgfusepath{stroke,fill}%
\end{pgfscope}%
\begin{pgfscope}%
\pgfpathrectangle{\pgfqpoint{0.100000in}{0.212622in}}{\pgfqpoint{3.696000in}{3.696000in}}%
\pgfusepath{clip}%
\pgfsetbuttcap%
\pgfsetroundjoin%
\definecolor{currentfill}{rgb}{0.121569,0.466667,0.705882}%
\pgfsetfillcolor{currentfill}%
\pgfsetfillopacity{0.632530}%
\pgfsetlinewidth{1.003750pt}%
\definecolor{currentstroke}{rgb}{0.121569,0.466667,0.705882}%
\pgfsetstrokecolor{currentstroke}%
\pgfsetstrokeopacity{0.632530}%
\pgfsetdash{}{0pt}%
\pgfpathmoveto{\pgfqpoint{0.895239in}{1.261559in}}%
\pgfpathcurveto{\pgfqpoint{0.903475in}{1.261559in}}{\pgfqpoint{0.911375in}{1.264831in}}{\pgfqpoint{0.917199in}{1.270655in}}%
\pgfpathcurveto{\pgfqpoint{0.923023in}{1.276479in}}{\pgfqpoint{0.926296in}{1.284379in}}{\pgfqpoint{0.926296in}{1.292616in}}%
\pgfpathcurveto{\pgfqpoint{0.926296in}{1.300852in}}{\pgfqpoint{0.923023in}{1.308752in}}{\pgfqpoint{0.917199in}{1.314576in}}%
\pgfpathcurveto{\pgfqpoint{0.911375in}{1.320400in}}{\pgfqpoint{0.903475in}{1.323672in}}{\pgfqpoint{0.895239in}{1.323672in}}%
\pgfpathcurveto{\pgfqpoint{0.887003in}{1.323672in}}{\pgfqpoint{0.879103in}{1.320400in}}{\pgfqpoint{0.873279in}{1.314576in}}%
\pgfpathcurveto{\pgfqpoint{0.867455in}{1.308752in}}{\pgfqpoint{0.864183in}{1.300852in}}{\pgfqpoint{0.864183in}{1.292616in}}%
\pgfpathcurveto{\pgfqpoint{0.864183in}{1.284379in}}{\pgfqpoint{0.867455in}{1.276479in}}{\pgfqpoint{0.873279in}{1.270655in}}%
\pgfpathcurveto{\pgfqpoint{0.879103in}{1.264831in}}{\pgfqpoint{0.887003in}{1.261559in}}{\pgfqpoint{0.895239in}{1.261559in}}%
\pgfpathclose%
\pgfusepath{stroke,fill}%
\end{pgfscope}%
\begin{pgfscope}%
\pgfpathrectangle{\pgfqpoint{0.100000in}{0.212622in}}{\pgfqpoint{3.696000in}{3.696000in}}%
\pgfusepath{clip}%
\pgfsetbuttcap%
\pgfsetroundjoin%
\definecolor{currentfill}{rgb}{0.121569,0.466667,0.705882}%
\pgfsetfillcolor{currentfill}%
\pgfsetfillopacity{0.632530}%
\pgfsetlinewidth{1.003750pt}%
\definecolor{currentstroke}{rgb}{0.121569,0.466667,0.705882}%
\pgfsetstrokecolor{currentstroke}%
\pgfsetstrokeopacity{0.632530}%
\pgfsetdash{}{0pt}%
\pgfpathmoveto{\pgfqpoint{0.895239in}{1.261559in}}%
\pgfpathcurveto{\pgfqpoint{0.903475in}{1.261559in}}{\pgfqpoint{0.911375in}{1.264831in}}{\pgfqpoint{0.917199in}{1.270655in}}%
\pgfpathcurveto{\pgfqpoint{0.923023in}{1.276479in}}{\pgfqpoint{0.926296in}{1.284379in}}{\pgfqpoint{0.926296in}{1.292616in}}%
\pgfpathcurveto{\pgfqpoint{0.926296in}{1.300852in}}{\pgfqpoint{0.923023in}{1.308752in}}{\pgfqpoint{0.917199in}{1.314576in}}%
\pgfpathcurveto{\pgfqpoint{0.911375in}{1.320400in}}{\pgfqpoint{0.903475in}{1.323672in}}{\pgfqpoint{0.895239in}{1.323672in}}%
\pgfpathcurveto{\pgfqpoint{0.887003in}{1.323672in}}{\pgfqpoint{0.879103in}{1.320400in}}{\pgfqpoint{0.873279in}{1.314576in}}%
\pgfpathcurveto{\pgfqpoint{0.867455in}{1.308752in}}{\pgfqpoint{0.864183in}{1.300852in}}{\pgfqpoint{0.864183in}{1.292616in}}%
\pgfpathcurveto{\pgfqpoint{0.864183in}{1.284379in}}{\pgfqpoint{0.867455in}{1.276479in}}{\pgfqpoint{0.873279in}{1.270655in}}%
\pgfpathcurveto{\pgfqpoint{0.879103in}{1.264831in}}{\pgfqpoint{0.887003in}{1.261559in}}{\pgfqpoint{0.895239in}{1.261559in}}%
\pgfpathclose%
\pgfusepath{stroke,fill}%
\end{pgfscope}%
\begin{pgfscope}%
\pgfpathrectangle{\pgfqpoint{0.100000in}{0.212622in}}{\pgfqpoint{3.696000in}{3.696000in}}%
\pgfusepath{clip}%
\pgfsetbuttcap%
\pgfsetroundjoin%
\definecolor{currentfill}{rgb}{0.121569,0.466667,0.705882}%
\pgfsetfillcolor{currentfill}%
\pgfsetfillopacity{0.632530}%
\pgfsetlinewidth{1.003750pt}%
\definecolor{currentstroke}{rgb}{0.121569,0.466667,0.705882}%
\pgfsetstrokecolor{currentstroke}%
\pgfsetstrokeopacity{0.632530}%
\pgfsetdash{}{0pt}%
\pgfpathmoveto{\pgfqpoint{0.895239in}{1.261559in}}%
\pgfpathcurveto{\pgfqpoint{0.903475in}{1.261559in}}{\pgfqpoint{0.911375in}{1.264831in}}{\pgfqpoint{0.917199in}{1.270655in}}%
\pgfpathcurveto{\pgfqpoint{0.923023in}{1.276479in}}{\pgfqpoint{0.926296in}{1.284379in}}{\pgfqpoint{0.926296in}{1.292616in}}%
\pgfpathcurveto{\pgfqpoint{0.926296in}{1.300852in}}{\pgfqpoint{0.923023in}{1.308752in}}{\pgfqpoint{0.917199in}{1.314576in}}%
\pgfpathcurveto{\pgfqpoint{0.911375in}{1.320400in}}{\pgfqpoint{0.903475in}{1.323672in}}{\pgfqpoint{0.895239in}{1.323672in}}%
\pgfpathcurveto{\pgfqpoint{0.887003in}{1.323672in}}{\pgfqpoint{0.879103in}{1.320400in}}{\pgfqpoint{0.873279in}{1.314576in}}%
\pgfpathcurveto{\pgfqpoint{0.867455in}{1.308752in}}{\pgfqpoint{0.864183in}{1.300852in}}{\pgfqpoint{0.864183in}{1.292616in}}%
\pgfpathcurveto{\pgfqpoint{0.864183in}{1.284379in}}{\pgfqpoint{0.867455in}{1.276479in}}{\pgfqpoint{0.873279in}{1.270655in}}%
\pgfpathcurveto{\pgfqpoint{0.879103in}{1.264831in}}{\pgfqpoint{0.887003in}{1.261559in}}{\pgfqpoint{0.895239in}{1.261559in}}%
\pgfpathclose%
\pgfusepath{stroke,fill}%
\end{pgfscope}%
\begin{pgfscope}%
\pgfpathrectangle{\pgfqpoint{0.100000in}{0.212622in}}{\pgfqpoint{3.696000in}{3.696000in}}%
\pgfusepath{clip}%
\pgfsetbuttcap%
\pgfsetroundjoin%
\definecolor{currentfill}{rgb}{0.121569,0.466667,0.705882}%
\pgfsetfillcolor{currentfill}%
\pgfsetfillopacity{0.632530}%
\pgfsetlinewidth{1.003750pt}%
\definecolor{currentstroke}{rgb}{0.121569,0.466667,0.705882}%
\pgfsetstrokecolor{currentstroke}%
\pgfsetstrokeopacity{0.632530}%
\pgfsetdash{}{0pt}%
\pgfpathmoveto{\pgfqpoint{0.895239in}{1.261559in}}%
\pgfpathcurveto{\pgfqpoint{0.903475in}{1.261559in}}{\pgfqpoint{0.911375in}{1.264831in}}{\pgfqpoint{0.917199in}{1.270655in}}%
\pgfpathcurveto{\pgfqpoint{0.923023in}{1.276479in}}{\pgfqpoint{0.926296in}{1.284379in}}{\pgfqpoint{0.926296in}{1.292616in}}%
\pgfpathcurveto{\pgfqpoint{0.926296in}{1.300852in}}{\pgfqpoint{0.923023in}{1.308752in}}{\pgfqpoint{0.917199in}{1.314576in}}%
\pgfpathcurveto{\pgfqpoint{0.911375in}{1.320400in}}{\pgfqpoint{0.903475in}{1.323672in}}{\pgfqpoint{0.895239in}{1.323672in}}%
\pgfpathcurveto{\pgfqpoint{0.887003in}{1.323672in}}{\pgfqpoint{0.879103in}{1.320400in}}{\pgfqpoint{0.873279in}{1.314576in}}%
\pgfpathcurveto{\pgfqpoint{0.867455in}{1.308752in}}{\pgfqpoint{0.864183in}{1.300852in}}{\pgfqpoint{0.864183in}{1.292616in}}%
\pgfpathcurveto{\pgfqpoint{0.864183in}{1.284379in}}{\pgfqpoint{0.867455in}{1.276479in}}{\pgfqpoint{0.873279in}{1.270655in}}%
\pgfpathcurveto{\pgfqpoint{0.879103in}{1.264831in}}{\pgfqpoint{0.887003in}{1.261559in}}{\pgfqpoint{0.895239in}{1.261559in}}%
\pgfpathclose%
\pgfusepath{stroke,fill}%
\end{pgfscope}%
\begin{pgfscope}%
\pgfpathrectangle{\pgfqpoint{0.100000in}{0.212622in}}{\pgfqpoint{3.696000in}{3.696000in}}%
\pgfusepath{clip}%
\pgfsetbuttcap%
\pgfsetroundjoin%
\definecolor{currentfill}{rgb}{0.121569,0.466667,0.705882}%
\pgfsetfillcolor{currentfill}%
\pgfsetfillopacity{0.632530}%
\pgfsetlinewidth{1.003750pt}%
\definecolor{currentstroke}{rgb}{0.121569,0.466667,0.705882}%
\pgfsetstrokecolor{currentstroke}%
\pgfsetstrokeopacity{0.632530}%
\pgfsetdash{}{0pt}%
\pgfpathmoveto{\pgfqpoint{0.895239in}{1.261559in}}%
\pgfpathcurveto{\pgfqpoint{0.903475in}{1.261559in}}{\pgfqpoint{0.911375in}{1.264831in}}{\pgfqpoint{0.917199in}{1.270655in}}%
\pgfpathcurveto{\pgfqpoint{0.923023in}{1.276479in}}{\pgfqpoint{0.926296in}{1.284379in}}{\pgfqpoint{0.926296in}{1.292616in}}%
\pgfpathcurveto{\pgfqpoint{0.926296in}{1.300852in}}{\pgfqpoint{0.923023in}{1.308752in}}{\pgfqpoint{0.917199in}{1.314576in}}%
\pgfpathcurveto{\pgfqpoint{0.911375in}{1.320400in}}{\pgfqpoint{0.903475in}{1.323672in}}{\pgfqpoint{0.895239in}{1.323672in}}%
\pgfpathcurveto{\pgfqpoint{0.887003in}{1.323672in}}{\pgfqpoint{0.879103in}{1.320400in}}{\pgfqpoint{0.873279in}{1.314576in}}%
\pgfpathcurveto{\pgfqpoint{0.867455in}{1.308752in}}{\pgfqpoint{0.864183in}{1.300852in}}{\pgfqpoint{0.864183in}{1.292616in}}%
\pgfpathcurveto{\pgfqpoint{0.864183in}{1.284379in}}{\pgfqpoint{0.867455in}{1.276479in}}{\pgfqpoint{0.873279in}{1.270655in}}%
\pgfpathcurveto{\pgfqpoint{0.879103in}{1.264831in}}{\pgfqpoint{0.887003in}{1.261559in}}{\pgfqpoint{0.895239in}{1.261559in}}%
\pgfpathclose%
\pgfusepath{stroke,fill}%
\end{pgfscope}%
\begin{pgfscope}%
\pgfpathrectangle{\pgfqpoint{0.100000in}{0.212622in}}{\pgfqpoint{3.696000in}{3.696000in}}%
\pgfusepath{clip}%
\pgfsetbuttcap%
\pgfsetroundjoin%
\definecolor{currentfill}{rgb}{0.121569,0.466667,0.705882}%
\pgfsetfillcolor{currentfill}%
\pgfsetfillopacity{0.632530}%
\pgfsetlinewidth{1.003750pt}%
\definecolor{currentstroke}{rgb}{0.121569,0.466667,0.705882}%
\pgfsetstrokecolor{currentstroke}%
\pgfsetstrokeopacity{0.632530}%
\pgfsetdash{}{0pt}%
\pgfpathmoveto{\pgfqpoint{0.895239in}{1.261559in}}%
\pgfpathcurveto{\pgfqpoint{0.903475in}{1.261559in}}{\pgfqpoint{0.911375in}{1.264831in}}{\pgfqpoint{0.917199in}{1.270655in}}%
\pgfpathcurveto{\pgfqpoint{0.923023in}{1.276479in}}{\pgfqpoint{0.926296in}{1.284379in}}{\pgfqpoint{0.926296in}{1.292616in}}%
\pgfpathcurveto{\pgfqpoint{0.926296in}{1.300852in}}{\pgfqpoint{0.923023in}{1.308752in}}{\pgfqpoint{0.917199in}{1.314576in}}%
\pgfpathcurveto{\pgfqpoint{0.911375in}{1.320400in}}{\pgfqpoint{0.903475in}{1.323672in}}{\pgfqpoint{0.895239in}{1.323672in}}%
\pgfpathcurveto{\pgfqpoint{0.887003in}{1.323672in}}{\pgfqpoint{0.879103in}{1.320400in}}{\pgfqpoint{0.873279in}{1.314576in}}%
\pgfpathcurveto{\pgfqpoint{0.867455in}{1.308752in}}{\pgfqpoint{0.864183in}{1.300852in}}{\pgfqpoint{0.864183in}{1.292616in}}%
\pgfpathcurveto{\pgfqpoint{0.864183in}{1.284379in}}{\pgfqpoint{0.867455in}{1.276479in}}{\pgfqpoint{0.873279in}{1.270655in}}%
\pgfpathcurveto{\pgfqpoint{0.879103in}{1.264831in}}{\pgfqpoint{0.887003in}{1.261559in}}{\pgfqpoint{0.895239in}{1.261559in}}%
\pgfpathclose%
\pgfusepath{stroke,fill}%
\end{pgfscope}%
\begin{pgfscope}%
\pgfpathrectangle{\pgfqpoint{0.100000in}{0.212622in}}{\pgfqpoint{3.696000in}{3.696000in}}%
\pgfusepath{clip}%
\pgfsetbuttcap%
\pgfsetroundjoin%
\definecolor{currentfill}{rgb}{0.121569,0.466667,0.705882}%
\pgfsetfillcolor{currentfill}%
\pgfsetfillopacity{0.632530}%
\pgfsetlinewidth{1.003750pt}%
\definecolor{currentstroke}{rgb}{0.121569,0.466667,0.705882}%
\pgfsetstrokecolor{currentstroke}%
\pgfsetstrokeopacity{0.632530}%
\pgfsetdash{}{0pt}%
\pgfpathmoveto{\pgfqpoint{0.895239in}{1.261559in}}%
\pgfpathcurveto{\pgfqpoint{0.903475in}{1.261559in}}{\pgfqpoint{0.911375in}{1.264831in}}{\pgfqpoint{0.917199in}{1.270655in}}%
\pgfpathcurveto{\pgfqpoint{0.923023in}{1.276479in}}{\pgfqpoint{0.926296in}{1.284379in}}{\pgfqpoint{0.926296in}{1.292616in}}%
\pgfpathcurveto{\pgfqpoint{0.926296in}{1.300852in}}{\pgfqpoint{0.923023in}{1.308752in}}{\pgfqpoint{0.917199in}{1.314576in}}%
\pgfpathcurveto{\pgfqpoint{0.911375in}{1.320400in}}{\pgfqpoint{0.903475in}{1.323672in}}{\pgfqpoint{0.895239in}{1.323672in}}%
\pgfpathcurveto{\pgfqpoint{0.887003in}{1.323672in}}{\pgfqpoint{0.879103in}{1.320400in}}{\pgfqpoint{0.873279in}{1.314576in}}%
\pgfpathcurveto{\pgfqpoint{0.867455in}{1.308752in}}{\pgfqpoint{0.864183in}{1.300852in}}{\pgfqpoint{0.864183in}{1.292616in}}%
\pgfpathcurveto{\pgfqpoint{0.864183in}{1.284379in}}{\pgfqpoint{0.867455in}{1.276479in}}{\pgfqpoint{0.873279in}{1.270655in}}%
\pgfpathcurveto{\pgfqpoint{0.879103in}{1.264831in}}{\pgfqpoint{0.887003in}{1.261559in}}{\pgfqpoint{0.895239in}{1.261559in}}%
\pgfpathclose%
\pgfusepath{stroke,fill}%
\end{pgfscope}%
\begin{pgfscope}%
\pgfpathrectangle{\pgfqpoint{0.100000in}{0.212622in}}{\pgfqpoint{3.696000in}{3.696000in}}%
\pgfusepath{clip}%
\pgfsetbuttcap%
\pgfsetroundjoin%
\definecolor{currentfill}{rgb}{0.121569,0.466667,0.705882}%
\pgfsetfillcolor{currentfill}%
\pgfsetfillopacity{0.632530}%
\pgfsetlinewidth{1.003750pt}%
\definecolor{currentstroke}{rgb}{0.121569,0.466667,0.705882}%
\pgfsetstrokecolor{currentstroke}%
\pgfsetstrokeopacity{0.632530}%
\pgfsetdash{}{0pt}%
\pgfpathmoveto{\pgfqpoint{0.895239in}{1.261559in}}%
\pgfpathcurveto{\pgfqpoint{0.903475in}{1.261559in}}{\pgfqpoint{0.911375in}{1.264831in}}{\pgfqpoint{0.917199in}{1.270655in}}%
\pgfpathcurveto{\pgfqpoint{0.923023in}{1.276479in}}{\pgfqpoint{0.926296in}{1.284379in}}{\pgfqpoint{0.926296in}{1.292616in}}%
\pgfpathcurveto{\pgfqpoint{0.926296in}{1.300852in}}{\pgfqpoint{0.923023in}{1.308752in}}{\pgfqpoint{0.917199in}{1.314576in}}%
\pgfpathcurveto{\pgfqpoint{0.911375in}{1.320400in}}{\pgfqpoint{0.903475in}{1.323672in}}{\pgfqpoint{0.895239in}{1.323672in}}%
\pgfpathcurveto{\pgfqpoint{0.887003in}{1.323672in}}{\pgfqpoint{0.879103in}{1.320400in}}{\pgfqpoint{0.873279in}{1.314576in}}%
\pgfpathcurveto{\pgfqpoint{0.867455in}{1.308752in}}{\pgfqpoint{0.864183in}{1.300852in}}{\pgfqpoint{0.864183in}{1.292616in}}%
\pgfpathcurveto{\pgfqpoint{0.864183in}{1.284379in}}{\pgfqpoint{0.867455in}{1.276479in}}{\pgfqpoint{0.873279in}{1.270655in}}%
\pgfpathcurveto{\pgfqpoint{0.879103in}{1.264831in}}{\pgfqpoint{0.887003in}{1.261559in}}{\pgfqpoint{0.895239in}{1.261559in}}%
\pgfpathclose%
\pgfusepath{stroke,fill}%
\end{pgfscope}%
\begin{pgfscope}%
\pgfpathrectangle{\pgfqpoint{0.100000in}{0.212622in}}{\pgfqpoint{3.696000in}{3.696000in}}%
\pgfusepath{clip}%
\pgfsetbuttcap%
\pgfsetroundjoin%
\definecolor{currentfill}{rgb}{0.121569,0.466667,0.705882}%
\pgfsetfillcolor{currentfill}%
\pgfsetfillopacity{0.632530}%
\pgfsetlinewidth{1.003750pt}%
\definecolor{currentstroke}{rgb}{0.121569,0.466667,0.705882}%
\pgfsetstrokecolor{currentstroke}%
\pgfsetstrokeopacity{0.632530}%
\pgfsetdash{}{0pt}%
\pgfpathmoveto{\pgfqpoint{0.895239in}{1.261559in}}%
\pgfpathcurveto{\pgfqpoint{0.903475in}{1.261559in}}{\pgfqpoint{0.911375in}{1.264831in}}{\pgfqpoint{0.917199in}{1.270655in}}%
\pgfpathcurveto{\pgfqpoint{0.923023in}{1.276479in}}{\pgfqpoint{0.926296in}{1.284379in}}{\pgfqpoint{0.926296in}{1.292616in}}%
\pgfpathcurveto{\pgfqpoint{0.926296in}{1.300852in}}{\pgfqpoint{0.923023in}{1.308752in}}{\pgfqpoint{0.917199in}{1.314576in}}%
\pgfpathcurveto{\pgfqpoint{0.911375in}{1.320400in}}{\pgfqpoint{0.903475in}{1.323672in}}{\pgfqpoint{0.895239in}{1.323672in}}%
\pgfpathcurveto{\pgfqpoint{0.887003in}{1.323672in}}{\pgfqpoint{0.879103in}{1.320400in}}{\pgfqpoint{0.873279in}{1.314576in}}%
\pgfpathcurveto{\pgfqpoint{0.867455in}{1.308752in}}{\pgfqpoint{0.864183in}{1.300852in}}{\pgfqpoint{0.864183in}{1.292616in}}%
\pgfpathcurveto{\pgfqpoint{0.864183in}{1.284379in}}{\pgfqpoint{0.867455in}{1.276479in}}{\pgfqpoint{0.873279in}{1.270655in}}%
\pgfpathcurveto{\pgfqpoint{0.879103in}{1.264831in}}{\pgfqpoint{0.887003in}{1.261559in}}{\pgfqpoint{0.895239in}{1.261559in}}%
\pgfpathclose%
\pgfusepath{stroke,fill}%
\end{pgfscope}%
\begin{pgfscope}%
\pgfpathrectangle{\pgfqpoint{0.100000in}{0.212622in}}{\pgfqpoint{3.696000in}{3.696000in}}%
\pgfusepath{clip}%
\pgfsetbuttcap%
\pgfsetroundjoin%
\definecolor{currentfill}{rgb}{0.121569,0.466667,0.705882}%
\pgfsetfillcolor{currentfill}%
\pgfsetfillopacity{0.632530}%
\pgfsetlinewidth{1.003750pt}%
\definecolor{currentstroke}{rgb}{0.121569,0.466667,0.705882}%
\pgfsetstrokecolor{currentstroke}%
\pgfsetstrokeopacity{0.632530}%
\pgfsetdash{}{0pt}%
\pgfpathmoveto{\pgfqpoint{0.895239in}{1.261559in}}%
\pgfpathcurveto{\pgfqpoint{0.903475in}{1.261559in}}{\pgfqpoint{0.911375in}{1.264831in}}{\pgfqpoint{0.917199in}{1.270655in}}%
\pgfpathcurveto{\pgfqpoint{0.923023in}{1.276479in}}{\pgfqpoint{0.926296in}{1.284379in}}{\pgfqpoint{0.926296in}{1.292616in}}%
\pgfpathcurveto{\pgfqpoint{0.926296in}{1.300852in}}{\pgfqpoint{0.923023in}{1.308752in}}{\pgfqpoint{0.917199in}{1.314576in}}%
\pgfpathcurveto{\pgfqpoint{0.911375in}{1.320400in}}{\pgfqpoint{0.903475in}{1.323672in}}{\pgfqpoint{0.895239in}{1.323672in}}%
\pgfpathcurveto{\pgfqpoint{0.887003in}{1.323672in}}{\pgfqpoint{0.879103in}{1.320400in}}{\pgfqpoint{0.873279in}{1.314576in}}%
\pgfpathcurveto{\pgfqpoint{0.867455in}{1.308752in}}{\pgfqpoint{0.864183in}{1.300852in}}{\pgfqpoint{0.864183in}{1.292616in}}%
\pgfpathcurveto{\pgfqpoint{0.864183in}{1.284379in}}{\pgfqpoint{0.867455in}{1.276479in}}{\pgfqpoint{0.873279in}{1.270655in}}%
\pgfpathcurveto{\pgfqpoint{0.879103in}{1.264831in}}{\pgfqpoint{0.887003in}{1.261559in}}{\pgfqpoint{0.895239in}{1.261559in}}%
\pgfpathclose%
\pgfusepath{stroke,fill}%
\end{pgfscope}%
\begin{pgfscope}%
\pgfpathrectangle{\pgfqpoint{0.100000in}{0.212622in}}{\pgfqpoint{3.696000in}{3.696000in}}%
\pgfusepath{clip}%
\pgfsetbuttcap%
\pgfsetroundjoin%
\definecolor{currentfill}{rgb}{0.121569,0.466667,0.705882}%
\pgfsetfillcolor{currentfill}%
\pgfsetfillopacity{0.632530}%
\pgfsetlinewidth{1.003750pt}%
\definecolor{currentstroke}{rgb}{0.121569,0.466667,0.705882}%
\pgfsetstrokecolor{currentstroke}%
\pgfsetstrokeopacity{0.632530}%
\pgfsetdash{}{0pt}%
\pgfpathmoveto{\pgfqpoint{0.895239in}{1.261559in}}%
\pgfpathcurveto{\pgfqpoint{0.903475in}{1.261559in}}{\pgfqpoint{0.911375in}{1.264831in}}{\pgfqpoint{0.917199in}{1.270655in}}%
\pgfpathcurveto{\pgfqpoint{0.923023in}{1.276479in}}{\pgfqpoint{0.926296in}{1.284379in}}{\pgfqpoint{0.926296in}{1.292616in}}%
\pgfpathcurveto{\pgfqpoint{0.926296in}{1.300852in}}{\pgfqpoint{0.923023in}{1.308752in}}{\pgfqpoint{0.917199in}{1.314576in}}%
\pgfpathcurveto{\pgfqpoint{0.911375in}{1.320400in}}{\pgfqpoint{0.903475in}{1.323672in}}{\pgfqpoint{0.895239in}{1.323672in}}%
\pgfpathcurveto{\pgfqpoint{0.887003in}{1.323672in}}{\pgfqpoint{0.879103in}{1.320400in}}{\pgfqpoint{0.873279in}{1.314576in}}%
\pgfpathcurveto{\pgfqpoint{0.867455in}{1.308752in}}{\pgfqpoint{0.864183in}{1.300852in}}{\pgfqpoint{0.864183in}{1.292616in}}%
\pgfpathcurveto{\pgfqpoint{0.864183in}{1.284379in}}{\pgfqpoint{0.867455in}{1.276479in}}{\pgfqpoint{0.873279in}{1.270655in}}%
\pgfpathcurveto{\pgfqpoint{0.879103in}{1.264831in}}{\pgfqpoint{0.887003in}{1.261559in}}{\pgfqpoint{0.895239in}{1.261559in}}%
\pgfpathclose%
\pgfusepath{stroke,fill}%
\end{pgfscope}%
\begin{pgfscope}%
\pgfpathrectangle{\pgfqpoint{0.100000in}{0.212622in}}{\pgfqpoint{3.696000in}{3.696000in}}%
\pgfusepath{clip}%
\pgfsetbuttcap%
\pgfsetroundjoin%
\definecolor{currentfill}{rgb}{0.121569,0.466667,0.705882}%
\pgfsetfillcolor{currentfill}%
\pgfsetfillopacity{0.632530}%
\pgfsetlinewidth{1.003750pt}%
\definecolor{currentstroke}{rgb}{0.121569,0.466667,0.705882}%
\pgfsetstrokecolor{currentstroke}%
\pgfsetstrokeopacity{0.632530}%
\pgfsetdash{}{0pt}%
\pgfpathmoveto{\pgfqpoint{0.895239in}{1.261559in}}%
\pgfpathcurveto{\pgfqpoint{0.903475in}{1.261559in}}{\pgfqpoint{0.911375in}{1.264831in}}{\pgfqpoint{0.917199in}{1.270655in}}%
\pgfpathcurveto{\pgfqpoint{0.923023in}{1.276479in}}{\pgfqpoint{0.926296in}{1.284379in}}{\pgfqpoint{0.926296in}{1.292616in}}%
\pgfpathcurveto{\pgfqpoint{0.926296in}{1.300852in}}{\pgfqpoint{0.923023in}{1.308752in}}{\pgfqpoint{0.917199in}{1.314576in}}%
\pgfpathcurveto{\pgfqpoint{0.911375in}{1.320400in}}{\pgfqpoint{0.903475in}{1.323672in}}{\pgfqpoint{0.895239in}{1.323672in}}%
\pgfpathcurveto{\pgfqpoint{0.887003in}{1.323672in}}{\pgfqpoint{0.879103in}{1.320400in}}{\pgfqpoint{0.873279in}{1.314576in}}%
\pgfpathcurveto{\pgfqpoint{0.867455in}{1.308752in}}{\pgfqpoint{0.864183in}{1.300852in}}{\pgfqpoint{0.864183in}{1.292616in}}%
\pgfpathcurveto{\pgfqpoint{0.864183in}{1.284379in}}{\pgfqpoint{0.867455in}{1.276479in}}{\pgfqpoint{0.873279in}{1.270655in}}%
\pgfpathcurveto{\pgfqpoint{0.879103in}{1.264831in}}{\pgfqpoint{0.887003in}{1.261559in}}{\pgfqpoint{0.895239in}{1.261559in}}%
\pgfpathclose%
\pgfusepath{stroke,fill}%
\end{pgfscope}%
\begin{pgfscope}%
\pgfpathrectangle{\pgfqpoint{0.100000in}{0.212622in}}{\pgfqpoint{3.696000in}{3.696000in}}%
\pgfusepath{clip}%
\pgfsetbuttcap%
\pgfsetroundjoin%
\definecolor{currentfill}{rgb}{0.121569,0.466667,0.705882}%
\pgfsetfillcolor{currentfill}%
\pgfsetfillopacity{0.632530}%
\pgfsetlinewidth{1.003750pt}%
\definecolor{currentstroke}{rgb}{0.121569,0.466667,0.705882}%
\pgfsetstrokecolor{currentstroke}%
\pgfsetstrokeopacity{0.632530}%
\pgfsetdash{}{0pt}%
\pgfpathmoveto{\pgfqpoint{0.895239in}{1.261559in}}%
\pgfpathcurveto{\pgfqpoint{0.903475in}{1.261559in}}{\pgfqpoint{0.911375in}{1.264831in}}{\pgfqpoint{0.917199in}{1.270655in}}%
\pgfpathcurveto{\pgfqpoint{0.923023in}{1.276479in}}{\pgfqpoint{0.926296in}{1.284379in}}{\pgfqpoint{0.926296in}{1.292616in}}%
\pgfpathcurveto{\pgfqpoint{0.926296in}{1.300852in}}{\pgfqpoint{0.923023in}{1.308752in}}{\pgfqpoint{0.917199in}{1.314576in}}%
\pgfpathcurveto{\pgfqpoint{0.911375in}{1.320400in}}{\pgfqpoint{0.903475in}{1.323672in}}{\pgfqpoint{0.895239in}{1.323672in}}%
\pgfpathcurveto{\pgfqpoint{0.887003in}{1.323672in}}{\pgfqpoint{0.879103in}{1.320400in}}{\pgfqpoint{0.873279in}{1.314576in}}%
\pgfpathcurveto{\pgfqpoint{0.867455in}{1.308752in}}{\pgfqpoint{0.864183in}{1.300852in}}{\pgfqpoint{0.864183in}{1.292616in}}%
\pgfpathcurveto{\pgfqpoint{0.864183in}{1.284379in}}{\pgfqpoint{0.867455in}{1.276479in}}{\pgfqpoint{0.873279in}{1.270655in}}%
\pgfpathcurveto{\pgfqpoint{0.879103in}{1.264831in}}{\pgfqpoint{0.887003in}{1.261559in}}{\pgfqpoint{0.895239in}{1.261559in}}%
\pgfpathclose%
\pgfusepath{stroke,fill}%
\end{pgfscope}%
\begin{pgfscope}%
\pgfpathrectangle{\pgfqpoint{0.100000in}{0.212622in}}{\pgfqpoint{3.696000in}{3.696000in}}%
\pgfusepath{clip}%
\pgfsetbuttcap%
\pgfsetroundjoin%
\definecolor{currentfill}{rgb}{0.121569,0.466667,0.705882}%
\pgfsetfillcolor{currentfill}%
\pgfsetfillopacity{0.632530}%
\pgfsetlinewidth{1.003750pt}%
\definecolor{currentstroke}{rgb}{0.121569,0.466667,0.705882}%
\pgfsetstrokecolor{currentstroke}%
\pgfsetstrokeopacity{0.632530}%
\pgfsetdash{}{0pt}%
\pgfpathmoveto{\pgfqpoint{0.895239in}{1.261559in}}%
\pgfpathcurveto{\pgfqpoint{0.903475in}{1.261559in}}{\pgfqpoint{0.911375in}{1.264831in}}{\pgfqpoint{0.917199in}{1.270655in}}%
\pgfpathcurveto{\pgfqpoint{0.923023in}{1.276479in}}{\pgfqpoint{0.926296in}{1.284379in}}{\pgfqpoint{0.926296in}{1.292616in}}%
\pgfpathcurveto{\pgfqpoint{0.926296in}{1.300852in}}{\pgfqpoint{0.923023in}{1.308752in}}{\pgfqpoint{0.917199in}{1.314576in}}%
\pgfpathcurveto{\pgfqpoint{0.911375in}{1.320400in}}{\pgfqpoint{0.903475in}{1.323672in}}{\pgfqpoint{0.895239in}{1.323672in}}%
\pgfpathcurveto{\pgfqpoint{0.887003in}{1.323672in}}{\pgfqpoint{0.879103in}{1.320400in}}{\pgfqpoint{0.873279in}{1.314576in}}%
\pgfpathcurveto{\pgfqpoint{0.867455in}{1.308752in}}{\pgfqpoint{0.864183in}{1.300852in}}{\pgfqpoint{0.864183in}{1.292616in}}%
\pgfpathcurveto{\pgfqpoint{0.864183in}{1.284379in}}{\pgfqpoint{0.867455in}{1.276479in}}{\pgfqpoint{0.873279in}{1.270655in}}%
\pgfpathcurveto{\pgfqpoint{0.879103in}{1.264831in}}{\pgfqpoint{0.887003in}{1.261559in}}{\pgfqpoint{0.895239in}{1.261559in}}%
\pgfpathclose%
\pgfusepath{stroke,fill}%
\end{pgfscope}%
\begin{pgfscope}%
\pgfpathrectangle{\pgfqpoint{0.100000in}{0.212622in}}{\pgfqpoint{3.696000in}{3.696000in}}%
\pgfusepath{clip}%
\pgfsetbuttcap%
\pgfsetroundjoin%
\definecolor{currentfill}{rgb}{0.121569,0.466667,0.705882}%
\pgfsetfillcolor{currentfill}%
\pgfsetfillopacity{0.632530}%
\pgfsetlinewidth{1.003750pt}%
\definecolor{currentstroke}{rgb}{0.121569,0.466667,0.705882}%
\pgfsetstrokecolor{currentstroke}%
\pgfsetstrokeopacity{0.632530}%
\pgfsetdash{}{0pt}%
\pgfpathmoveto{\pgfqpoint{0.895239in}{1.261559in}}%
\pgfpathcurveto{\pgfqpoint{0.903475in}{1.261559in}}{\pgfqpoint{0.911375in}{1.264831in}}{\pgfqpoint{0.917199in}{1.270655in}}%
\pgfpathcurveto{\pgfqpoint{0.923023in}{1.276479in}}{\pgfqpoint{0.926296in}{1.284379in}}{\pgfqpoint{0.926296in}{1.292616in}}%
\pgfpathcurveto{\pgfqpoint{0.926296in}{1.300852in}}{\pgfqpoint{0.923023in}{1.308752in}}{\pgfqpoint{0.917199in}{1.314576in}}%
\pgfpathcurveto{\pgfqpoint{0.911375in}{1.320400in}}{\pgfqpoint{0.903475in}{1.323672in}}{\pgfqpoint{0.895239in}{1.323672in}}%
\pgfpathcurveto{\pgfqpoint{0.887003in}{1.323672in}}{\pgfqpoint{0.879103in}{1.320400in}}{\pgfqpoint{0.873279in}{1.314576in}}%
\pgfpathcurveto{\pgfqpoint{0.867455in}{1.308752in}}{\pgfqpoint{0.864183in}{1.300852in}}{\pgfqpoint{0.864183in}{1.292616in}}%
\pgfpathcurveto{\pgfqpoint{0.864183in}{1.284379in}}{\pgfqpoint{0.867455in}{1.276479in}}{\pgfqpoint{0.873279in}{1.270655in}}%
\pgfpathcurveto{\pgfqpoint{0.879103in}{1.264831in}}{\pgfqpoint{0.887003in}{1.261559in}}{\pgfqpoint{0.895239in}{1.261559in}}%
\pgfpathclose%
\pgfusepath{stroke,fill}%
\end{pgfscope}%
\begin{pgfscope}%
\pgfpathrectangle{\pgfqpoint{0.100000in}{0.212622in}}{\pgfqpoint{3.696000in}{3.696000in}}%
\pgfusepath{clip}%
\pgfsetbuttcap%
\pgfsetroundjoin%
\definecolor{currentfill}{rgb}{0.121569,0.466667,0.705882}%
\pgfsetfillcolor{currentfill}%
\pgfsetfillopacity{0.632530}%
\pgfsetlinewidth{1.003750pt}%
\definecolor{currentstroke}{rgb}{0.121569,0.466667,0.705882}%
\pgfsetstrokecolor{currentstroke}%
\pgfsetstrokeopacity{0.632530}%
\pgfsetdash{}{0pt}%
\pgfpathmoveto{\pgfqpoint{0.895239in}{1.261559in}}%
\pgfpathcurveto{\pgfqpoint{0.903475in}{1.261559in}}{\pgfqpoint{0.911375in}{1.264831in}}{\pgfqpoint{0.917199in}{1.270655in}}%
\pgfpathcurveto{\pgfqpoint{0.923023in}{1.276479in}}{\pgfqpoint{0.926296in}{1.284379in}}{\pgfqpoint{0.926296in}{1.292616in}}%
\pgfpathcurveto{\pgfqpoint{0.926296in}{1.300852in}}{\pgfqpoint{0.923023in}{1.308752in}}{\pgfqpoint{0.917199in}{1.314576in}}%
\pgfpathcurveto{\pgfqpoint{0.911375in}{1.320400in}}{\pgfqpoint{0.903475in}{1.323672in}}{\pgfqpoint{0.895239in}{1.323672in}}%
\pgfpathcurveto{\pgfqpoint{0.887003in}{1.323672in}}{\pgfqpoint{0.879103in}{1.320400in}}{\pgfqpoint{0.873279in}{1.314576in}}%
\pgfpathcurveto{\pgfqpoint{0.867455in}{1.308752in}}{\pgfqpoint{0.864183in}{1.300852in}}{\pgfqpoint{0.864183in}{1.292616in}}%
\pgfpathcurveto{\pgfqpoint{0.864183in}{1.284379in}}{\pgfqpoint{0.867455in}{1.276479in}}{\pgfqpoint{0.873279in}{1.270655in}}%
\pgfpathcurveto{\pgfqpoint{0.879103in}{1.264831in}}{\pgfqpoint{0.887003in}{1.261559in}}{\pgfqpoint{0.895239in}{1.261559in}}%
\pgfpathclose%
\pgfusepath{stroke,fill}%
\end{pgfscope}%
\begin{pgfscope}%
\pgfpathrectangle{\pgfqpoint{0.100000in}{0.212622in}}{\pgfqpoint{3.696000in}{3.696000in}}%
\pgfusepath{clip}%
\pgfsetbuttcap%
\pgfsetroundjoin%
\definecolor{currentfill}{rgb}{0.121569,0.466667,0.705882}%
\pgfsetfillcolor{currentfill}%
\pgfsetfillopacity{0.632530}%
\pgfsetlinewidth{1.003750pt}%
\definecolor{currentstroke}{rgb}{0.121569,0.466667,0.705882}%
\pgfsetstrokecolor{currentstroke}%
\pgfsetstrokeopacity{0.632530}%
\pgfsetdash{}{0pt}%
\pgfpathmoveto{\pgfqpoint{0.895239in}{1.261559in}}%
\pgfpathcurveto{\pgfqpoint{0.903475in}{1.261559in}}{\pgfqpoint{0.911375in}{1.264831in}}{\pgfqpoint{0.917199in}{1.270655in}}%
\pgfpathcurveto{\pgfqpoint{0.923023in}{1.276479in}}{\pgfqpoint{0.926296in}{1.284379in}}{\pgfqpoint{0.926296in}{1.292616in}}%
\pgfpathcurveto{\pgfqpoint{0.926296in}{1.300852in}}{\pgfqpoint{0.923023in}{1.308752in}}{\pgfqpoint{0.917199in}{1.314576in}}%
\pgfpathcurveto{\pgfqpoint{0.911375in}{1.320400in}}{\pgfqpoint{0.903475in}{1.323672in}}{\pgfqpoint{0.895239in}{1.323672in}}%
\pgfpathcurveto{\pgfqpoint{0.887003in}{1.323672in}}{\pgfqpoint{0.879103in}{1.320400in}}{\pgfqpoint{0.873279in}{1.314576in}}%
\pgfpathcurveto{\pgfqpoint{0.867455in}{1.308752in}}{\pgfqpoint{0.864183in}{1.300852in}}{\pgfqpoint{0.864183in}{1.292616in}}%
\pgfpathcurveto{\pgfqpoint{0.864183in}{1.284379in}}{\pgfqpoint{0.867455in}{1.276479in}}{\pgfqpoint{0.873279in}{1.270655in}}%
\pgfpathcurveto{\pgfqpoint{0.879103in}{1.264831in}}{\pgfqpoint{0.887003in}{1.261559in}}{\pgfqpoint{0.895239in}{1.261559in}}%
\pgfpathclose%
\pgfusepath{stroke,fill}%
\end{pgfscope}%
\begin{pgfscope}%
\pgfpathrectangle{\pgfqpoint{0.100000in}{0.212622in}}{\pgfqpoint{3.696000in}{3.696000in}}%
\pgfusepath{clip}%
\pgfsetbuttcap%
\pgfsetroundjoin%
\definecolor{currentfill}{rgb}{0.121569,0.466667,0.705882}%
\pgfsetfillcolor{currentfill}%
\pgfsetfillopacity{0.632530}%
\pgfsetlinewidth{1.003750pt}%
\definecolor{currentstroke}{rgb}{0.121569,0.466667,0.705882}%
\pgfsetstrokecolor{currentstroke}%
\pgfsetstrokeopacity{0.632530}%
\pgfsetdash{}{0pt}%
\pgfpathmoveto{\pgfqpoint{0.895239in}{1.261559in}}%
\pgfpathcurveto{\pgfqpoint{0.903475in}{1.261559in}}{\pgfqpoint{0.911375in}{1.264831in}}{\pgfqpoint{0.917199in}{1.270655in}}%
\pgfpathcurveto{\pgfqpoint{0.923023in}{1.276479in}}{\pgfqpoint{0.926296in}{1.284379in}}{\pgfqpoint{0.926296in}{1.292616in}}%
\pgfpathcurveto{\pgfqpoint{0.926296in}{1.300852in}}{\pgfqpoint{0.923023in}{1.308752in}}{\pgfqpoint{0.917199in}{1.314576in}}%
\pgfpathcurveto{\pgfqpoint{0.911375in}{1.320400in}}{\pgfqpoint{0.903475in}{1.323672in}}{\pgfqpoint{0.895239in}{1.323672in}}%
\pgfpathcurveto{\pgfqpoint{0.887003in}{1.323672in}}{\pgfqpoint{0.879103in}{1.320400in}}{\pgfqpoint{0.873279in}{1.314576in}}%
\pgfpathcurveto{\pgfqpoint{0.867455in}{1.308752in}}{\pgfqpoint{0.864183in}{1.300852in}}{\pgfqpoint{0.864183in}{1.292616in}}%
\pgfpathcurveto{\pgfqpoint{0.864183in}{1.284379in}}{\pgfqpoint{0.867455in}{1.276479in}}{\pgfqpoint{0.873279in}{1.270655in}}%
\pgfpathcurveto{\pgfqpoint{0.879103in}{1.264831in}}{\pgfqpoint{0.887003in}{1.261559in}}{\pgfqpoint{0.895239in}{1.261559in}}%
\pgfpathclose%
\pgfusepath{stroke,fill}%
\end{pgfscope}%
\begin{pgfscope}%
\pgfpathrectangle{\pgfqpoint{0.100000in}{0.212622in}}{\pgfqpoint{3.696000in}{3.696000in}}%
\pgfusepath{clip}%
\pgfsetbuttcap%
\pgfsetroundjoin%
\definecolor{currentfill}{rgb}{0.121569,0.466667,0.705882}%
\pgfsetfillcolor{currentfill}%
\pgfsetfillopacity{0.632530}%
\pgfsetlinewidth{1.003750pt}%
\definecolor{currentstroke}{rgb}{0.121569,0.466667,0.705882}%
\pgfsetstrokecolor{currentstroke}%
\pgfsetstrokeopacity{0.632530}%
\pgfsetdash{}{0pt}%
\pgfpathmoveto{\pgfqpoint{0.895239in}{1.261559in}}%
\pgfpathcurveto{\pgfqpoint{0.903475in}{1.261559in}}{\pgfqpoint{0.911375in}{1.264831in}}{\pgfqpoint{0.917199in}{1.270655in}}%
\pgfpathcurveto{\pgfqpoint{0.923023in}{1.276479in}}{\pgfqpoint{0.926296in}{1.284379in}}{\pgfqpoint{0.926296in}{1.292616in}}%
\pgfpathcurveto{\pgfqpoint{0.926296in}{1.300852in}}{\pgfqpoint{0.923023in}{1.308752in}}{\pgfqpoint{0.917199in}{1.314576in}}%
\pgfpathcurveto{\pgfqpoint{0.911375in}{1.320400in}}{\pgfqpoint{0.903475in}{1.323672in}}{\pgfqpoint{0.895239in}{1.323672in}}%
\pgfpathcurveto{\pgfqpoint{0.887003in}{1.323672in}}{\pgfqpoint{0.879103in}{1.320400in}}{\pgfqpoint{0.873279in}{1.314576in}}%
\pgfpathcurveto{\pgfqpoint{0.867455in}{1.308752in}}{\pgfqpoint{0.864183in}{1.300852in}}{\pgfqpoint{0.864183in}{1.292616in}}%
\pgfpathcurveto{\pgfqpoint{0.864183in}{1.284379in}}{\pgfqpoint{0.867455in}{1.276479in}}{\pgfqpoint{0.873279in}{1.270655in}}%
\pgfpathcurveto{\pgfqpoint{0.879103in}{1.264831in}}{\pgfqpoint{0.887003in}{1.261559in}}{\pgfqpoint{0.895239in}{1.261559in}}%
\pgfpathclose%
\pgfusepath{stroke,fill}%
\end{pgfscope}%
\begin{pgfscope}%
\pgfpathrectangle{\pgfqpoint{0.100000in}{0.212622in}}{\pgfqpoint{3.696000in}{3.696000in}}%
\pgfusepath{clip}%
\pgfsetbuttcap%
\pgfsetroundjoin%
\definecolor{currentfill}{rgb}{0.121569,0.466667,0.705882}%
\pgfsetfillcolor{currentfill}%
\pgfsetfillopacity{0.632530}%
\pgfsetlinewidth{1.003750pt}%
\definecolor{currentstroke}{rgb}{0.121569,0.466667,0.705882}%
\pgfsetstrokecolor{currentstroke}%
\pgfsetstrokeopacity{0.632530}%
\pgfsetdash{}{0pt}%
\pgfpathmoveto{\pgfqpoint{0.895239in}{1.261559in}}%
\pgfpathcurveto{\pgfqpoint{0.903475in}{1.261559in}}{\pgfqpoint{0.911375in}{1.264831in}}{\pgfqpoint{0.917199in}{1.270655in}}%
\pgfpathcurveto{\pgfqpoint{0.923023in}{1.276479in}}{\pgfqpoint{0.926296in}{1.284379in}}{\pgfqpoint{0.926296in}{1.292616in}}%
\pgfpathcurveto{\pgfqpoint{0.926296in}{1.300852in}}{\pgfqpoint{0.923023in}{1.308752in}}{\pgfqpoint{0.917199in}{1.314576in}}%
\pgfpathcurveto{\pgfqpoint{0.911375in}{1.320400in}}{\pgfqpoint{0.903475in}{1.323672in}}{\pgfqpoint{0.895239in}{1.323672in}}%
\pgfpathcurveto{\pgfqpoint{0.887003in}{1.323672in}}{\pgfqpoint{0.879103in}{1.320400in}}{\pgfqpoint{0.873279in}{1.314576in}}%
\pgfpathcurveto{\pgfqpoint{0.867455in}{1.308752in}}{\pgfqpoint{0.864183in}{1.300852in}}{\pgfqpoint{0.864183in}{1.292616in}}%
\pgfpathcurveto{\pgfqpoint{0.864183in}{1.284379in}}{\pgfqpoint{0.867455in}{1.276479in}}{\pgfqpoint{0.873279in}{1.270655in}}%
\pgfpathcurveto{\pgfqpoint{0.879103in}{1.264831in}}{\pgfqpoint{0.887003in}{1.261559in}}{\pgfqpoint{0.895239in}{1.261559in}}%
\pgfpathclose%
\pgfusepath{stroke,fill}%
\end{pgfscope}%
\begin{pgfscope}%
\pgfpathrectangle{\pgfqpoint{0.100000in}{0.212622in}}{\pgfqpoint{3.696000in}{3.696000in}}%
\pgfusepath{clip}%
\pgfsetbuttcap%
\pgfsetroundjoin%
\definecolor{currentfill}{rgb}{0.121569,0.466667,0.705882}%
\pgfsetfillcolor{currentfill}%
\pgfsetfillopacity{0.632530}%
\pgfsetlinewidth{1.003750pt}%
\definecolor{currentstroke}{rgb}{0.121569,0.466667,0.705882}%
\pgfsetstrokecolor{currentstroke}%
\pgfsetstrokeopacity{0.632530}%
\pgfsetdash{}{0pt}%
\pgfpathmoveto{\pgfqpoint{0.895239in}{1.261559in}}%
\pgfpathcurveto{\pgfqpoint{0.903475in}{1.261559in}}{\pgfqpoint{0.911375in}{1.264831in}}{\pgfqpoint{0.917199in}{1.270655in}}%
\pgfpathcurveto{\pgfqpoint{0.923023in}{1.276479in}}{\pgfqpoint{0.926296in}{1.284379in}}{\pgfqpoint{0.926296in}{1.292616in}}%
\pgfpathcurveto{\pgfqpoint{0.926296in}{1.300852in}}{\pgfqpoint{0.923023in}{1.308752in}}{\pgfqpoint{0.917199in}{1.314576in}}%
\pgfpathcurveto{\pgfqpoint{0.911375in}{1.320400in}}{\pgfqpoint{0.903475in}{1.323672in}}{\pgfqpoint{0.895239in}{1.323672in}}%
\pgfpathcurveto{\pgfqpoint{0.887003in}{1.323672in}}{\pgfqpoint{0.879103in}{1.320400in}}{\pgfqpoint{0.873279in}{1.314576in}}%
\pgfpathcurveto{\pgfqpoint{0.867455in}{1.308752in}}{\pgfqpoint{0.864183in}{1.300852in}}{\pgfqpoint{0.864183in}{1.292616in}}%
\pgfpathcurveto{\pgfqpoint{0.864183in}{1.284379in}}{\pgfqpoint{0.867455in}{1.276479in}}{\pgfqpoint{0.873279in}{1.270655in}}%
\pgfpathcurveto{\pgfqpoint{0.879103in}{1.264831in}}{\pgfqpoint{0.887003in}{1.261559in}}{\pgfqpoint{0.895239in}{1.261559in}}%
\pgfpathclose%
\pgfusepath{stroke,fill}%
\end{pgfscope}%
\begin{pgfscope}%
\pgfpathrectangle{\pgfqpoint{0.100000in}{0.212622in}}{\pgfqpoint{3.696000in}{3.696000in}}%
\pgfusepath{clip}%
\pgfsetbuttcap%
\pgfsetroundjoin%
\definecolor{currentfill}{rgb}{0.121569,0.466667,0.705882}%
\pgfsetfillcolor{currentfill}%
\pgfsetfillopacity{0.632530}%
\pgfsetlinewidth{1.003750pt}%
\definecolor{currentstroke}{rgb}{0.121569,0.466667,0.705882}%
\pgfsetstrokecolor{currentstroke}%
\pgfsetstrokeopacity{0.632530}%
\pgfsetdash{}{0pt}%
\pgfpathmoveto{\pgfqpoint{0.895239in}{1.261559in}}%
\pgfpathcurveto{\pgfqpoint{0.903475in}{1.261559in}}{\pgfqpoint{0.911375in}{1.264831in}}{\pgfqpoint{0.917199in}{1.270655in}}%
\pgfpathcurveto{\pgfqpoint{0.923023in}{1.276479in}}{\pgfqpoint{0.926296in}{1.284379in}}{\pgfqpoint{0.926296in}{1.292616in}}%
\pgfpathcurveto{\pgfqpoint{0.926296in}{1.300852in}}{\pgfqpoint{0.923023in}{1.308752in}}{\pgfqpoint{0.917199in}{1.314576in}}%
\pgfpathcurveto{\pgfqpoint{0.911375in}{1.320400in}}{\pgfqpoint{0.903475in}{1.323672in}}{\pgfqpoint{0.895239in}{1.323672in}}%
\pgfpathcurveto{\pgfqpoint{0.887003in}{1.323672in}}{\pgfqpoint{0.879103in}{1.320400in}}{\pgfqpoint{0.873279in}{1.314576in}}%
\pgfpathcurveto{\pgfqpoint{0.867455in}{1.308752in}}{\pgfqpoint{0.864183in}{1.300852in}}{\pgfqpoint{0.864183in}{1.292616in}}%
\pgfpathcurveto{\pgfqpoint{0.864183in}{1.284379in}}{\pgfqpoint{0.867455in}{1.276479in}}{\pgfqpoint{0.873279in}{1.270655in}}%
\pgfpathcurveto{\pgfqpoint{0.879103in}{1.264831in}}{\pgfqpoint{0.887003in}{1.261559in}}{\pgfqpoint{0.895239in}{1.261559in}}%
\pgfpathclose%
\pgfusepath{stroke,fill}%
\end{pgfscope}%
\begin{pgfscope}%
\pgfpathrectangle{\pgfqpoint{0.100000in}{0.212622in}}{\pgfqpoint{3.696000in}{3.696000in}}%
\pgfusepath{clip}%
\pgfsetbuttcap%
\pgfsetroundjoin%
\definecolor{currentfill}{rgb}{0.121569,0.466667,0.705882}%
\pgfsetfillcolor{currentfill}%
\pgfsetfillopacity{0.632530}%
\pgfsetlinewidth{1.003750pt}%
\definecolor{currentstroke}{rgb}{0.121569,0.466667,0.705882}%
\pgfsetstrokecolor{currentstroke}%
\pgfsetstrokeopacity{0.632530}%
\pgfsetdash{}{0pt}%
\pgfpathmoveto{\pgfqpoint{0.895239in}{1.261559in}}%
\pgfpathcurveto{\pgfqpoint{0.903475in}{1.261559in}}{\pgfqpoint{0.911375in}{1.264831in}}{\pgfqpoint{0.917199in}{1.270655in}}%
\pgfpathcurveto{\pgfqpoint{0.923023in}{1.276479in}}{\pgfqpoint{0.926296in}{1.284379in}}{\pgfqpoint{0.926296in}{1.292616in}}%
\pgfpathcurveto{\pgfqpoint{0.926296in}{1.300852in}}{\pgfqpoint{0.923023in}{1.308752in}}{\pgfqpoint{0.917199in}{1.314576in}}%
\pgfpathcurveto{\pgfqpoint{0.911375in}{1.320400in}}{\pgfqpoint{0.903475in}{1.323672in}}{\pgfqpoint{0.895239in}{1.323672in}}%
\pgfpathcurveto{\pgfqpoint{0.887003in}{1.323672in}}{\pgfqpoint{0.879103in}{1.320400in}}{\pgfqpoint{0.873279in}{1.314576in}}%
\pgfpathcurveto{\pgfqpoint{0.867455in}{1.308752in}}{\pgfqpoint{0.864183in}{1.300852in}}{\pgfqpoint{0.864183in}{1.292616in}}%
\pgfpathcurveto{\pgfqpoint{0.864183in}{1.284379in}}{\pgfqpoint{0.867455in}{1.276479in}}{\pgfqpoint{0.873279in}{1.270655in}}%
\pgfpathcurveto{\pgfqpoint{0.879103in}{1.264831in}}{\pgfqpoint{0.887003in}{1.261559in}}{\pgfqpoint{0.895239in}{1.261559in}}%
\pgfpathclose%
\pgfusepath{stroke,fill}%
\end{pgfscope}%
\begin{pgfscope}%
\pgfpathrectangle{\pgfqpoint{0.100000in}{0.212622in}}{\pgfqpoint{3.696000in}{3.696000in}}%
\pgfusepath{clip}%
\pgfsetbuttcap%
\pgfsetroundjoin%
\definecolor{currentfill}{rgb}{0.121569,0.466667,0.705882}%
\pgfsetfillcolor{currentfill}%
\pgfsetfillopacity{0.632530}%
\pgfsetlinewidth{1.003750pt}%
\definecolor{currentstroke}{rgb}{0.121569,0.466667,0.705882}%
\pgfsetstrokecolor{currentstroke}%
\pgfsetstrokeopacity{0.632530}%
\pgfsetdash{}{0pt}%
\pgfpathmoveto{\pgfqpoint{0.895239in}{1.261559in}}%
\pgfpathcurveto{\pgfqpoint{0.903475in}{1.261559in}}{\pgfqpoint{0.911375in}{1.264831in}}{\pgfqpoint{0.917199in}{1.270655in}}%
\pgfpathcurveto{\pgfqpoint{0.923023in}{1.276479in}}{\pgfqpoint{0.926296in}{1.284379in}}{\pgfqpoint{0.926296in}{1.292616in}}%
\pgfpathcurveto{\pgfqpoint{0.926296in}{1.300852in}}{\pgfqpoint{0.923023in}{1.308752in}}{\pgfqpoint{0.917199in}{1.314576in}}%
\pgfpathcurveto{\pgfqpoint{0.911375in}{1.320400in}}{\pgfqpoint{0.903475in}{1.323672in}}{\pgfqpoint{0.895239in}{1.323672in}}%
\pgfpathcurveto{\pgfqpoint{0.887003in}{1.323672in}}{\pgfqpoint{0.879103in}{1.320400in}}{\pgfqpoint{0.873279in}{1.314576in}}%
\pgfpathcurveto{\pgfqpoint{0.867455in}{1.308752in}}{\pgfqpoint{0.864183in}{1.300852in}}{\pgfqpoint{0.864183in}{1.292616in}}%
\pgfpathcurveto{\pgfqpoint{0.864183in}{1.284379in}}{\pgfqpoint{0.867455in}{1.276479in}}{\pgfqpoint{0.873279in}{1.270655in}}%
\pgfpathcurveto{\pgfqpoint{0.879103in}{1.264831in}}{\pgfqpoint{0.887003in}{1.261559in}}{\pgfqpoint{0.895239in}{1.261559in}}%
\pgfpathclose%
\pgfusepath{stroke,fill}%
\end{pgfscope}%
\begin{pgfscope}%
\pgfpathrectangle{\pgfqpoint{0.100000in}{0.212622in}}{\pgfqpoint{3.696000in}{3.696000in}}%
\pgfusepath{clip}%
\pgfsetbuttcap%
\pgfsetroundjoin%
\definecolor{currentfill}{rgb}{0.121569,0.466667,0.705882}%
\pgfsetfillcolor{currentfill}%
\pgfsetfillopacity{0.632530}%
\pgfsetlinewidth{1.003750pt}%
\definecolor{currentstroke}{rgb}{0.121569,0.466667,0.705882}%
\pgfsetstrokecolor{currentstroke}%
\pgfsetstrokeopacity{0.632530}%
\pgfsetdash{}{0pt}%
\pgfpathmoveto{\pgfqpoint{0.895239in}{1.261559in}}%
\pgfpathcurveto{\pgfqpoint{0.903475in}{1.261559in}}{\pgfqpoint{0.911375in}{1.264831in}}{\pgfqpoint{0.917199in}{1.270655in}}%
\pgfpathcurveto{\pgfqpoint{0.923023in}{1.276479in}}{\pgfqpoint{0.926296in}{1.284379in}}{\pgfqpoint{0.926296in}{1.292616in}}%
\pgfpathcurveto{\pgfqpoint{0.926296in}{1.300852in}}{\pgfqpoint{0.923023in}{1.308752in}}{\pgfqpoint{0.917199in}{1.314576in}}%
\pgfpathcurveto{\pgfqpoint{0.911375in}{1.320400in}}{\pgfqpoint{0.903475in}{1.323672in}}{\pgfqpoint{0.895239in}{1.323672in}}%
\pgfpathcurveto{\pgfqpoint{0.887003in}{1.323672in}}{\pgfqpoint{0.879103in}{1.320400in}}{\pgfqpoint{0.873279in}{1.314576in}}%
\pgfpathcurveto{\pgfqpoint{0.867455in}{1.308752in}}{\pgfqpoint{0.864183in}{1.300852in}}{\pgfqpoint{0.864183in}{1.292616in}}%
\pgfpathcurveto{\pgfqpoint{0.864183in}{1.284379in}}{\pgfqpoint{0.867455in}{1.276479in}}{\pgfqpoint{0.873279in}{1.270655in}}%
\pgfpathcurveto{\pgfqpoint{0.879103in}{1.264831in}}{\pgfqpoint{0.887003in}{1.261559in}}{\pgfqpoint{0.895239in}{1.261559in}}%
\pgfpathclose%
\pgfusepath{stroke,fill}%
\end{pgfscope}%
\begin{pgfscope}%
\pgfpathrectangle{\pgfqpoint{0.100000in}{0.212622in}}{\pgfqpoint{3.696000in}{3.696000in}}%
\pgfusepath{clip}%
\pgfsetbuttcap%
\pgfsetroundjoin%
\definecolor{currentfill}{rgb}{0.121569,0.466667,0.705882}%
\pgfsetfillcolor{currentfill}%
\pgfsetfillopacity{0.632530}%
\pgfsetlinewidth{1.003750pt}%
\definecolor{currentstroke}{rgb}{0.121569,0.466667,0.705882}%
\pgfsetstrokecolor{currentstroke}%
\pgfsetstrokeopacity{0.632530}%
\pgfsetdash{}{0pt}%
\pgfpathmoveto{\pgfqpoint{0.895239in}{1.261559in}}%
\pgfpathcurveto{\pgfqpoint{0.903475in}{1.261559in}}{\pgfqpoint{0.911375in}{1.264831in}}{\pgfqpoint{0.917199in}{1.270655in}}%
\pgfpathcurveto{\pgfqpoint{0.923023in}{1.276479in}}{\pgfqpoint{0.926296in}{1.284379in}}{\pgfqpoint{0.926296in}{1.292616in}}%
\pgfpathcurveto{\pgfqpoint{0.926296in}{1.300852in}}{\pgfqpoint{0.923023in}{1.308752in}}{\pgfqpoint{0.917199in}{1.314576in}}%
\pgfpathcurveto{\pgfqpoint{0.911375in}{1.320400in}}{\pgfqpoint{0.903475in}{1.323672in}}{\pgfqpoint{0.895239in}{1.323672in}}%
\pgfpathcurveto{\pgfqpoint{0.887003in}{1.323672in}}{\pgfqpoint{0.879103in}{1.320400in}}{\pgfqpoint{0.873279in}{1.314576in}}%
\pgfpathcurveto{\pgfqpoint{0.867455in}{1.308752in}}{\pgfqpoint{0.864183in}{1.300852in}}{\pgfqpoint{0.864183in}{1.292616in}}%
\pgfpathcurveto{\pgfqpoint{0.864183in}{1.284379in}}{\pgfqpoint{0.867455in}{1.276479in}}{\pgfqpoint{0.873279in}{1.270655in}}%
\pgfpathcurveto{\pgfqpoint{0.879103in}{1.264831in}}{\pgfqpoint{0.887003in}{1.261559in}}{\pgfqpoint{0.895239in}{1.261559in}}%
\pgfpathclose%
\pgfusepath{stroke,fill}%
\end{pgfscope}%
\begin{pgfscope}%
\pgfpathrectangle{\pgfqpoint{0.100000in}{0.212622in}}{\pgfqpoint{3.696000in}{3.696000in}}%
\pgfusepath{clip}%
\pgfsetbuttcap%
\pgfsetroundjoin%
\definecolor{currentfill}{rgb}{0.121569,0.466667,0.705882}%
\pgfsetfillcolor{currentfill}%
\pgfsetfillopacity{0.632530}%
\pgfsetlinewidth{1.003750pt}%
\definecolor{currentstroke}{rgb}{0.121569,0.466667,0.705882}%
\pgfsetstrokecolor{currentstroke}%
\pgfsetstrokeopacity{0.632530}%
\pgfsetdash{}{0pt}%
\pgfpathmoveto{\pgfqpoint{0.895239in}{1.261559in}}%
\pgfpathcurveto{\pgfqpoint{0.903475in}{1.261559in}}{\pgfqpoint{0.911375in}{1.264831in}}{\pgfqpoint{0.917199in}{1.270655in}}%
\pgfpathcurveto{\pgfqpoint{0.923023in}{1.276479in}}{\pgfqpoint{0.926296in}{1.284379in}}{\pgfqpoint{0.926296in}{1.292616in}}%
\pgfpathcurveto{\pgfqpoint{0.926296in}{1.300852in}}{\pgfqpoint{0.923023in}{1.308752in}}{\pgfqpoint{0.917199in}{1.314576in}}%
\pgfpathcurveto{\pgfqpoint{0.911375in}{1.320400in}}{\pgfqpoint{0.903475in}{1.323672in}}{\pgfqpoint{0.895239in}{1.323672in}}%
\pgfpathcurveto{\pgfqpoint{0.887003in}{1.323672in}}{\pgfqpoint{0.879103in}{1.320400in}}{\pgfqpoint{0.873279in}{1.314576in}}%
\pgfpathcurveto{\pgfqpoint{0.867455in}{1.308752in}}{\pgfqpoint{0.864183in}{1.300852in}}{\pgfqpoint{0.864183in}{1.292616in}}%
\pgfpathcurveto{\pgfqpoint{0.864183in}{1.284379in}}{\pgfqpoint{0.867455in}{1.276479in}}{\pgfqpoint{0.873279in}{1.270655in}}%
\pgfpathcurveto{\pgfqpoint{0.879103in}{1.264831in}}{\pgfqpoint{0.887003in}{1.261559in}}{\pgfqpoint{0.895239in}{1.261559in}}%
\pgfpathclose%
\pgfusepath{stroke,fill}%
\end{pgfscope}%
\begin{pgfscope}%
\pgfpathrectangle{\pgfqpoint{0.100000in}{0.212622in}}{\pgfqpoint{3.696000in}{3.696000in}}%
\pgfusepath{clip}%
\pgfsetbuttcap%
\pgfsetroundjoin%
\definecolor{currentfill}{rgb}{0.121569,0.466667,0.705882}%
\pgfsetfillcolor{currentfill}%
\pgfsetfillopacity{0.632530}%
\pgfsetlinewidth{1.003750pt}%
\definecolor{currentstroke}{rgb}{0.121569,0.466667,0.705882}%
\pgfsetstrokecolor{currentstroke}%
\pgfsetstrokeopacity{0.632530}%
\pgfsetdash{}{0pt}%
\pgfpathmoveto{\pgfqpoint{0.895239in}{1.261559in}}%
\pgfpathcurveto{\pgfqpoint{0.903475in}{1.261559in}}{\pgfqpoint{0.911375in}{1.264831in}}{\pgfqpoint{0.917199in}{1.270655in}}%
\pgfpathcurveto{\pgfqpoint{0.923023in}{1.276479in}}{\pgfqpoint{0.926296in}{1.284379in}}{\pgfqpoint{0.926296in}{1.292616in}}%
\pgfpathcurveto{\pgfqpoint{0.926296in}{1.300852in}}{\pgfqpoint{0.923023in}{1.308752in}}{\pgfqpoint{0.917199in}{1.314576in}}%
\pgfpathcurveto{\pgfqpoint{0.911375in}{1.320400in}}{\pgfqpoint{0.903475in}{1.323672in}}{\pgfqpoint{0.895239in}{1.323672in}}%
\pgfpathcurveto{\pgfqpoint{0.887003in}{1.323672in}}{\pgfqpoint{0.879103in}{1.320400in}}{\pgfqpoint{0.873279in}{1.314576in}}%
\pgfpathcurveto{\pgfqpoint{0.867455in}{1.308752in}}{\pgfqpoint{0.864183in}{1.300852in}}{\pgfqpoint{0.864183in}{1.292616in}}%
\pgfpathcurveto{\pgfqpoint{0.864183in}{1.284379in}}{\pgfqpoint{0.867455in}{1.276479in}}{\pgfqpoint{0.873279in}{1.270655in}}%
\pgfpathcurveto{\pgfqpoint{0.879103in}{1.264831in}}{\pgfqpoint{0.887003in}{1.261559in}}{\pgfqpoint{0.895239in}{1.261559in}}%
\pgfpathclose%
\pgfusepath{stroke,fill}%
\end{pgfscope}%
\begin{pgfscope}%
\pgfpathrectangle{\pgfqpoint{0.100000in}{0.212622in}}{\pgfqpoint{3.696000in}{3.696000in}}%
\pgfusepath{clip}%
\pgfsetbuttcap%
\pgfsetroundjoin%
\definecolor{currentfill}{rgb}{0.121569,0.466667,0.705882}%
\pgfsetfillcolor{currentfill}%
\pgfsetfillopacity{0.632530}%
\pgfsetlinewidth{1.003750pt}%
\definecolor{currentstroke}{rgb}{0.121569,0.466667,0.705882}%
\pgfsetstrokecolor{currentstroke}%
\pgfsetstrokeopacity{0.632530}%
\pgfsetdash{}{0pt}%
\pgfpathmoveto{\pgfqpoint{0.895239in}{1.261559in}}%
\pgfpathcurveto{\pgfqpoint{0.903475in}{1.261559in}}{\pgfqpoint{0.911375in}{1.264831in}}{\pgfqpoint{0.917199in}{1.270655in}}%
\pgfpathcurveto{\pgfqpoint{0.923023in}{1.276479in}}{\pgfqpoint{0.926296in}{1.284379in}}{\pgfqpoint{0.926296in}{1.292616in}}%
\pgfpathcurveto{\pgfqpoint{0.926296in}{1.300852in}}{\pgfqpoint{0.923023in}{1.308752in}}{\pgfqpoint{0.917199in}{1.314576in}}%
\pgfpathcurveto{\pgfqpoint{0.911375in}{1.320400in}}{\pgfqpoint{0.903475in}{1.323672in}}{\pgfqpoint{0.895239in}{1.323672in}}%
\pgfpathcurveto{\pgfqpoint{0.887003in}{1.323672in}}{\pgfqpoint{0.879103in}{1.320400in}}{\pgfqpoint{0.873279in}{1.314576in}}%
\pgfpathcurveto{\pgfqpoint{0.867455in}{1.308752in}}{\pgfqpoint{0.864183in}{1.300852in}}{\pgfqpoint{0.864183in}{1.292616in}}%
\pgfpathcurveto{\pgfqpoint{0.864183in}{1.284379in}}{\pgfqpoint{0.867455in}{1.276479in}}{\pgfqpoint{0.873279in}{1.270655in}}%
\pgfpathcurveto{\pgfqpoint{0.879103in}{1.264831in}}{\pgfqpoint{0.887003in}{1.261559in}}{\pgfqpoint{0.895239in}{1.261559in}}%
\pgfpathclose%
\pgfusepath{stroke,fill}%
\end{pgfscope}%
\begin{pgfscope}%
\pgfpathrectangle{\pgfqpoint{0.100000in}{0.212622in}}{\pgfqpoint{3.696000in}{3.696000in}}%
\pgfusepath{clip}%
\pgfsetbuttcap%
\pgfsetroundjoin%
\definecolor{currentfill}{rgb}{0.121569,0.466667,0.705882}%
\pgfsetfillcolor{currentfill}%
\pgfsetfillopacity{0.632530}%
\pgfsetlinewidth{1.003750pt}%
\definecolor{currentstroke}{rgb}{0.121569,0.466667,0.705882}%
\pgfsetstrokecolor{currentstroke}%
\pgfsetstrokeopacity{0.632530}%
\pgfsetdash{}{0pt}%
\pgfpathmoveto{\pgfqpoint{0.895239in}{1.261559in}}%
\pgfpathcurveto{\pgfqpoint{0.903475in}{1.261559in}}{\pgfqpoint{0.911375in}{1.264831in}}{\pgfqpoint{0.917199in}{1.270655in}}%
\pgfpathcurveto{\pgfqpoint{0.923023in}{1.276479in}}{\pgfqpoint{0.926296in}{1.284379in}}{\pgfqpoint{0.926296in}{1.292616in}}%
\pgfpathcurveto{\pgfqpoint{0.926296in}{1.300852in}}{\pgfqpoint{0.923023in}{1.308752in}}{\pgfqpoint{0.917199in}{1.314576in}}%
\pgfpathcurveto{\pgfqpoint{0.911375in}{1.320400in}}{\pgfqpoint{0.903475in}{1.323672in}}{\pgfqpoint{0.895239in}{1.323672in}}%
\pgfpathcurveto{\pgfqpoint{0.887003in}{1.323672in}}{\pgfqpoint{0.879103in}{1.320400in}}{\pgfqpoint{0.873279in}{1.314576in}}%
\pgfpathcurveto{\pgfqpoint{0.867455in}{1.308752in}}{\pgfqpoint{0.864183in}{1.300852in}}{\pgfqpoint{0.864183in}{1.292616in}}%
\pgfpathcurveto{\pgfqpoint{0.864183in}{1.284379in}}{\pgfqpoint{0.867455in}{1.276479in}}{\pgfqpoint{0.873279in}{1.270655in}}%
\pgfpathcurveto{\pgfqpoint{0.879103in}{1.264831in}}{\pgfqpoint{0.887003in}{1.261559in}}{\pgfqpoint{0.895239in}{1.261559in}}%
\pgfpathclose%
\pgfusepath{stroke,fill}%
\end{pgfscope}%
\begin{pgfscope}%
\pgfpathrectangle{\pgfqpoint{0.100000in}{0.212622in}}{\pgfqpoint{3.696000in}{3.696000in}}%
\pgfusepath{clip}%
\pgfsetbuttcap%
\pgfsetroundjoin%
\definecolor{currentfill}{rgb}{0.121569,0.466667,0.705882}%
\pgfsetfillcolor{currentfill}%
\pgfsetfillopacity{0.632530}%
\pgfsetlinewidth{1.003750pt}%
\definecolor{currentstroke}{rgb}{0.121569,0.466667,0.705882}%
\pgfsetstrokecolor{currentstroke}%
\pgfsetstrokeopacity{0.632530}%
\pgfsetdash{}{0pt}%
\pgfpathmoveto{\pgfqpoint{0.895239in}{1.261559in}}%
\pgfpathcurveto{\pgfqpoint{0.903475in}{1.261559in}}{\pgfqpoint{0.911375in}{1.264831in}}{\pgfqpoint{0.917199in}{1.270655in}}%
\pgfpathcurveto{\pgfqpoint{0.923023in}{1.276479in}}{\pgfqpoint{0.926296in}{1.284379in}}{\pgfqpoint{0.926296in}{1.292616in}}%
\pgfpathcurveto{\pgfqpoint{0.926296in}{1.300852in}}{\pgfqpoint{0.923023in}{1.308752in}}{\pgfqpoint{0.917199in}{1.314576in}}%
\pgfpathcurveto{\pgfqpoint{0.911375in}{1.320400in}}{\pgfqpoint{0.903475in}{1.323672in}}{\pgfqpoint{0.895239in}{1.323672in}}%
\pgfpathcurveto{\pgfqpoint{0.887003in}{1.323672in}}{\pgfqpoint{0.879103in}{1.320400in}}{\pgfqpoint{0.873279in}{1.314576in}}%
\pgfpathcurveto{\pgfqpoint{0.867455in}{1.308752in}}{\pgfqpoint{0.864183in}{1.300852in}}{\pgfqpoint{0.864183in}{1.292616in}}%
\pgfpathcurveto{\pgfqpoint{0.864183in}{1.284379in}}{\pgfqpoint{0.867455in}{1.276479in}}{\pgfqpoint{0.873279in}{1.270655in}}%
\pgfpathcurveto{\pgfqpoint{0.879103in}{1.264831in}}{\pgfqpoint{0.887003in}{1.261559in}}{\pgfqpoint{0.895239in}{1.261559in}}%
\pgfpathclose%
\pgfusepath{stroke,fill}%
\end{pgfscope}%
\begin{pgfscope}%
\pgfpathrectangle{\pgfqpoint{0.100000in}{0.212622in}}{\pgfqpoint{3.696000in}{3.696000in}}%
\pgfusepath{clip}%
\pgfsetbuttcap%
\pgfsetroundjoin%
\definecolor{currentfill}{rgb}{0.121569,0.466667,0.705882}%
\pgfsetfillcolor{currentfill}%
\pgfsetfillopacity{0.632530}%
\pgfsetlinewidth{1.003750pt}%
\definecolor{currentstroke}{rgb}{0.121569,0.466667,0.705882}%
\pgfsetstrokecolor{currentstroke}%
\pgfsetstrokeopacity{0.632530}%
\pgfsetdash{}{0pt}%
\pgfpathmoveto{\pgfqpoint{0.895239in}{1.261559in}}%
\pgfpathcurveto{\pgfqpoint{0.903475in}{1.261559in}}{\pgfqpoint{0.911375in}{1.264831in}}{\pgfqpoint{0.917199in}{1.270655in}}%
\pgfpathcurveto{\pgfqpoint{0.923023in}{1.276479in}}{\pgfqpoint{0.926296in}{1.284379in}}{\pgfqpoint{0.926296in}{1.292616in}}%
\pgfpathcurveto{\pgfqpoint{0.926296in}{1.300852in}}{\pgfqpoint{0.923023in}{1.308752in}}{\pgfqpoint{0.917199in}{1.314576in}}%
\pgfpathcurveto{\pgfqpoint{0.911375in}{1.320400in}}{\pgfqpoint{0.903475in}{1.323672in}}{\pgfqpoint{0.895239in}{1.323672in}}%
\pgfpathcurveto{\pgfqpoint{0.887003in}{1.323672in}}{\pgfqpoint{0.879103in}{1.320400in}}{\pgfqpoint{0.873279in}{1.314576in}}%
\pgfpathcurveto{\pgfqpoint{0.867455in}{1.308752in}}{\pgfqpoint{0.864183in}{1.300852in}}{\pgfqpoint{0.864183in}{1.292616in}}%
\pgfpathcurveto{\pgfqpoint{0.864183in}{1.284379in}}{\pgfqpoint{0.867455in}{1.276479in}}{\pgfqpoint{0.873279in}{1.270655in}}%
\pgfpathcurveto{\pgfqpoint{0.879103in}{1.264831in}}{\pgfqpoint{0.887003in}{1.261559in}}{\pgfqpoint{0.895239in}{1.261559in}}%
\pgfpathclose%
\pgfusepath{stroke,fill}%
\end{pgfscope}%
\begin{pgfscope}%
\pgfpathrectangle{\pgfqpoint{0.100000in}{0.212622in}}{\pgfqpoint{3.696000in}{3.696000in}}%
\pgfusepath{clip}%
\pgfsetbuttcap%
\pgfsetroundjoin%
\definecolor{currentfill}{rgb}{0.121569,0.466667,0.705882}%
\pgfsetfillcolor{currentfill}%
\pgfsetfillopacity{0.632530}%
\pgfsetlinewidth{1.003750pt}%
\definecolor{currentstroke}{rgb}{0.121569,0.466667,0.705882}%
\pgfsetstrokecolor{currentstroke}%
\pgfsetstrokeopacity{0.632530}%
\pgfsetdash{}{0pt}%
\pgfpathmoveto{\pgfqpoint{0.895239in}{1.261559in}}%
\pgfpathcurveto{\pgfqpoint{0.903475in}{1.261559in}}{\pgfqpoint{0.911375in}{1.264831in}}{\pgfqpoint{0.917199in}{1.270655in}}%
\pgfpathcurveto{\pgfqpoint{0.923023in}{1.276479in}}{\pgfqpoint{0.926296in}{1.284379in}}{\pgfqpoint{0.926296in}{1.292616in}}%
\pgfpathcurveto{\pgfqpoint{0.926296in}{1.300852in}}{\pgfqpoint{0.923023in}{1.308752in}}{\pgfqpoint{0.917199in}{1.314576in}}%
\pgfpathcurveto{\pgfqpoint{0.911375in}{1.320400in}}{\pgfqpoint{0.903475in}{1.323672in}}{\pgfqpoint{0.895239in}{1.323672in}}%
\pgfpathcurveto{\pgfqpoint{0.887003in}{1.323672in}}{\pgfqpoint{0.879103in}{1.320400in}}{\pgfqpoint{0.873279in}{1.314576in}}%
\pgfpathcurveto{\pgfqpoint{0.867455in}{1.308752in}}{\pgfqpoint{0.864183in}{1.300852in}}{\pgfqpoint{0.864183in}{1.292616in}}%
\pgfpathcurveto{\pgfqpoint{0.864183in}{1.284379in}}{\pgfqpoint{0.867455in}{1.276479in}}{\pgfqpoint{0.873279in}{1.270655in}}%
\pgfpathcurveto{\pgfqpoint{0.879103in}{1.264831in}}{\pgfqpoint{0.887003in}{1.261559in}}{\pgfqpoint{0.895239in}{1.261559in}}%
\pgfpathclose%
\pgfusepath{stroke,fill}%
\end{pgfscope}%
\begin{pgfscope}%
\pgfpathrectangle{\pgfqpoint{0.100000in}{0.212622in}}{\pgfqpoint{3.696000in}{3.696000in}}%
\pgfusepath{clip}%
\pgfsetbuttcap%
\pgfsetroundjoin%
\definecolor{currentfill}{rgb}{0.121569,0.466667,0.705882}%
\pgfsetfillcolor{currentfill}%
\pgfsetfillopacity{0.632689}%
\pgfsetlinewidth{1.003750pt}%
\definecolor{currentstroke}{rgb}{0.121569,0.466667,0.705882}%
\pgfsetstrokecolor{currentstroke}%
\pgfsetstrokeopacity{0.632689}%
\pgfsetdash{}{0pt}%
\pgfpathmoveto{\pgfqpoint{0.894771in}{1.261477in}}%
\pgfpathcurveto{\pgfqpoint{0.903007in}{1.261477in}}{\pgfqpoint{0.910908in}{1.264749in}}{\pgfqpoint{0.916731in}{1.270573in}}%
\pgfpathcurveto{\pgfqpoint{0.922555in}{1.276397in}}{\pgfqpoint{0.925828in}{1.284297in}}{\pgfqpoint{0.925828in}{1.292534in}}%
\pgfpathcurveto{\pgfqpoint{0.925828in}{1.300770in}}{\pgfqpoint{0.922555in}{1.308670in}}{\pgfqpoint{0.916731in}{1.314494in}}%
\pgfpathcurveto{\pgfqpoint{0.910908in}{1.320318in}}{\pgfqpoint{0.903007in}{1.323590in}}{\pgfqpoint{0.894771in}{1.323590in}}%
\pgfpathcurveto{\pgfqpoint{0.886535in}{1.323590in}}{\pgfqpoint{0.878635in}{1.320318in}}{\pgfqpoint{0.872811in}{1.314494in}}%
\pgfpathcurveto{\pgfqpoint{0.866987in}{1.308670in}}{\pgfqpoint{0.863715in}{1.300770in}}{\pgfqpoint{0.863715in}{1.292534in}}%
\pgfpathcurveto{\pgfqpoint{0.863715in}{1.284297in}}{\pgfqpoint{0.866987in}{1.276397in}}{\pgfqpoint{0.872811in}{1.270573in}}%
\pgfpathcurveto{\pgfqpoint{0.878635in}{1.264749in}}{\pgfqpoint{0.886535in}{1.261477in}}{\pgfqpoint{0.894771in}{1.261477in}}%
\pgfpathclose%
\pgfusepath{stroke,fill}%
\end{pgfscope}%
\begin{pgfscope}%
\pgfpathrectangle{\pgfqpoint{0.100000in}{0.212622in}}{\pgfqpoint{3.696000in}{3.696000in}}%
\pgfusepath{clip}%
\pgfsetbuttcap%
\pgfsetroundjoin%
\definecolor{currentfill}{rgb}{0.121569,0.466667,0.705882}%
\pgfsetfillcolor{currentfill}%
\pgfsetfillopacity{0.632713}%
\pgfsetlinewidth{1.003750pt}%
\definecolor{currentstroke}{rgb}{0.121569,0.466667,0.705882}%
\pgfsetstrokecolor{currentstroke}%
\pgfsetstrokeopacity{0.632713}%
\pgfsetdash{}{0pt}%
\pgfpathmoveto{\pgfqpoint{0.875248in}{1.267411in}}%
\pgfpathcurveto{\pgfqpoint{0.883484in}{1.267411in}}{\pgfqpoint{0.891384in}{1.270683in}}{\pgfqpoint{0.897208in}{1.276507in}}%
\pgfpathcurveto{\pgfqpoint{0.903032in}{1.282331in}}{\pgfqpoint{0.906304in}{1.290231in}}{\pgfqpoint{0.906304in}{1.298468in}}%
\pgfpathcurveto{\pgfqpoint{0.906304in}{1.306704in}}{\pgfqpoint{0.903032in}{1.314604in}}{\pgfqpoint{0.897208in}{1.320428in}}%
\pgfpathcurveto{\pgfqpoint{0.891384in}{1.326252in}}{\pgfqpoint{0.883484in}{1.329524in}}{\pgfqpoint{0.875248in}{1.329524in}}%
\pgfpathcurveto{\pgfqpoint{0.867011in}{1.329524in}}{\pgfqpoint{0.859111in}{1.326252in}}{\pgfqpoint{0.853287in}{1.320428in}}%
\pgfpathcurveto{\pgfqpoint{0.847464in}{1.314604in}}{\pgfqpoint{0.844191in}{1.306704in}}{\pgfqpoint{0.844191in}{1.298468in}}%
\pgfpathcurveto{\pgfqpoint{0.844191in}{1.290231in}}{\pgfqpoint{0.847464in}{1.282331in}}{\pgfqpoint{0.853287in}{1.276507in}}%
\pgfpathcurveto{\pgfqpoint{0.859111in}{1.270683in}}{\pgfqpoint{0.867011in}{1.267411in}}{\pgfqpoint{0.875248in}{1.267411in}}%
\pgfpathclose%
\pgfusepath{stroke,fill}%
\end{pgfscope}%
\begin{pgfscope}%
\pgfpathrectangle{\pgfqpoint{0.100000in}{0.212622in}}{\pgfqpoint{3.696000in}{3.696000in}}%
\pgfusepath{clip}%
\pgfsetbuttcap%
\pgfsetroundjoin%
\definecolor{currentfill}{rgb}{0.121569,0.466667,0.705882}%
\pgfsetfillcolor{currentfill}%
\pgfsetfillopacity{0.632831}%
\pgfsetlinewidth{1.003750pt}%
\definecolor{currentstroke}{rgb}{0.121569,0.466667,0.705882}%
\pgfsetstrokecolor{currentstroke}%
\pgfsetstrokeopacity{0.632831}%
\pgfsetdash{}{0pt}%
\pgfpathmoveto{\pgfqpoint{2.129281in}{1.791383in}}%
\pgfpathcurveto{\pgfqpoint{2.137517in}{1.791383in}}{\pgfqpoint{2.145417in}{1.794655in}}{\pgfqpoint{2.151241in}{1.800479in}}%
\pgfpathcurveto{\pgfqpoint{2.157065in}{1.806303in}}{\pgfqpoint{2.160337in}{1.814203in}}{\pgfqpoint{2.160337in}{1.822439in}}%
\pgfpathcurveto{\pgfqpoint{2.160337in}{1.830676in}}{\pgfqpoint{2.157065in}{1.838576in}}{\pgfqpoint{2.151241in}{1.844400in}}%
\pgfpathcurveto{\pgfqpoint{2.145417in}{1.850224in}}{\pgfqpoint{2.137517in}{1.853496in}}{\pgfqpoint{2.129281in}{1.853496in}}%
\pgfpathcurveto{\pgfqpoint{2.121044in}{1.853496in}}{\pgfqpoint{2.113144in}{1.850224in}}{\pgfqpoint{2.107320in}{1.844400in}}%
\pgfpathcurveto{\pgfqpoint{2.101496in}{1.838576in}}{\pgfqpoint{2.098224in}{1.830676in}}{\pgfqpoint{2.098224in}{1.822439in}}%
\pgfpathcurveto{\pgfqpoint{2.098224in}{1.814203in}}{\pgfqpoint{2.101496in}{1.806303in}}{\pgfqpoint{2.107320in}{1.800479in}}%
\pgfpathcurveto{\pgfqpoint{2.113144in}{1.794655in}}{\pgfqpoint{2.121044in}{1.791383in}}{\pgfqpoint{2.129281in}{1.791383in}}%
\pgfpathclose%
\pgfusepath{stroke,fill}%
\end{pgfscope}%
\begin{pgfscope}%
\pgfpathrectangle{\pgfqpoint{0.100000in}{0.212622in}}{\pgfqpoint{3.696000in}{3.696000in}}%
\pgfusepath{clip}%
\pgfsetbuttcap%
\pgfsetroundjoin%
\definecolor{currentfill}{rgb}{0.121569,0.466667,0.705882}%
\pgfsetfillcolor{currentfill}%
\pgfsetfillopacity{0.632954}%
\pgfsetlinewidth{1.003750pt}%
\definecolor{currentstroke}{rgb}{0.121569,0.466667,0.705882}%
\pgfsetstrokecolor{currentstroke}%
\pgfsetstrokeopacity{0.632954}%
\pgfsetdash{}{0pt}%
\pgfpathmoveto{\pgfqpoint{0.893528in}{1.261178in}}%
\pgfpathcurveto{\pgfqpoint{0.901764in}{1.261178in}}{\pgfqpoint{0.909664in}{1.264450in}}{\pgfqpoint{0.915488in}{1.270274in}}%
\pgfpathcurveto{\pgfqpoint{0.921312in}{1.276098in}}{\pgfqpoint{0.924585in}{1.283998in}}{\pgfqpoint{0.924585in}{1.292234in}}%
\pgfpathcurveto{\pgfqpoint{0.924585in}{1.300471in}}{\pgfqpoint{0.921312in}{1.308371in}}{\pgfqpoint{0.915488in}{1.314195in}}%
\pgfpathcurveto{\pgfqpoint{0.909664in}{1.320018in}}{\pgfqpoint{0.901764in}{1.323291in}}{\pgfqpoint{0.893528in}{1.323291in}}%
\pgfpathcurveto{\pgfqpoint{0.885292in}{1.323291in}}{\pgfqpoint{0.877392in}{1.320018in}}{\pgfqpoint{0.871568in}{1.314195in}}%
\pgfpathcurveto{\pgfqpoint{0.865744in}{1.308371in}}{\pgfqpoint{0.862472in}{1.300471in}}{\pgfqpoint{0.862472in}{1.292234in}}%
\pgfpathcurveto{\pgfqpoint{0.862472in}{1.283998in}}{\pgfqpoint{0.865744in}{1.276098in}}{\pgfqpoint{0.871568in}{1.270274in}}%
\pgfpathcurveto{\pgfqpoint{0.877392in}{1.264450in}}{\pgfqpoint{0.885292in}{1.261178in}}{\pgfqpoint{0.893528in}{1.261178in}}%
\pgfpathclose%
\pgfusepath{stroke,fill}%
\end{pgfscope}%
\begin{pgfscope}%
\pgfpathrectangle{\pgfqpoint{0.100000in}{0.212622in}}{\pgfqpoint{3.696000in}{3.696000in}}%
\pgfusepath{clip}%
\pgfsetbuttcap%
\pgfsetroundjoin%
\definecolor{currentfill}{rgb}{0.121569,0.466667,0.705882}%
\pgfsetfillcolor{currentfill}%
\pgfsetfillopacity{0.633323}%
\pgfsetlinewidth{1.003750pt}%
\definecolor{currentstroke}{rgb}{0.121569,0.466667,0.705882}%
\pgfsetstrokecolor{currentstroke}%
\pgfsetstrokeopacity{0.633323}%
\pgfsetdash{}{0pt}%
\pgfpathmoveto{\pgfqpoint{0.891066in}{1.261071in}}%
\pgfpathcurveto{\pgfqpoint{0.899302in}{1.261071in}}{\pgfqpoint{0.907202in}{1.264344in}}{\pgfqpoint{0.913026in}{1.270168in}}%
\pgfpathcurveto{\pgfqpoint{0.918850in}{1.275992in}}{\pgfqpoint{0.922122in}{1.283892in}}{\pgfqpoint{0.922122in}{1.292128in}}%
\pgfpathcurveto{\pgfqpoint{0.922122in}{1.300364in}}{\pgfqpoint{0.918850in}{1.308264in}}{\pgfqpoint{0.913026in}{1.314088in}}%
\pgfpathcurveto{\pgfqpoint{0.907202in}{1.319912in}}{\pgfqpoint{0.899302in}{1.323184in}}{\pgfqpoint{0.891066in}{1.323184in}}%
\pgfpathcurveto{\pgfqpoint{0.882829in}{1.323184in}}{\pgfqpoint{0.874929in}{1.319912in}}{\pgfqpoint{0.869105in}{1.314088in}}%
\pgfpathcurveto{\pgfqpoint{0.863282in}{1.308264in}}{\pgfqpoint{0.860009in}{1.300364in}}{\pgfqpoint{0.860009in}{1.292128in}}%
\pgfpathcurveto{\pgfqpoint{0.860009in}{1.283892in}}{\pgfqpoint{0.863282in}{1.275992in}}{\pgfqpoint{0.869105in}{1.270168in}}%
\pgfpathcurveto{\pgfqpoint{0.874929in}{1.264344in}}{\pgfqpoint{0.882829in}{1.261071in}}{\pgfqpoint{0.891066in}{1.261071in}}%
\pgfpathclose%
\pgfusepath{stroke,fill}%
\end{pgfscope}%
\begin{pgfscope}%
\pgfpathrectangle{\pgfqpoint{0.100000in}{0.212622in}}{\pgfqpoint{3.696000in}{3.696000in}}%
\pgfusepath{clip}%
\pgfsetbuttcap%
\pgfsetroundjoin%
\definecolor{currentfill}{rgb}{0.121569,0.466667,0.705882}%
\pgfsetfillcolor{currentfill}%
\pgfsetfillopacity{0.633523}%
\pgfsetlinewidth{1.003750pt}%
\definecolor{currentstroke}{rgb}{0.121569,0.466667,0.705882}%
\pgfsetstrokecolor{currentstroke}%
\pgfsetstrokeopacity{0.633523}%
\pgfsetdash{}{0pt}%
\pgfpathmoveto{\pgfqpoint{0.881123in}{1.263164in}}%
\pgfpathcurveto{\pgfqpoint{0.889359in}{1.263164in}}{\pgfqpoint{0.897259in}{1.266436in}}{\pgfqpoint{0.903083in}{1.272260in}}%
\pgfpathcurveto{\pgfqpoint{0.908907in}{1.278084in}}{\pgfqpoint{0.912179in}{1.285984in}}{\pgfqpoint{0.912179in}{1.294221in}}%
\pgfpathcurveto{\pgfqpoint{0.912179in}{1.302457in}}{\pgfqpoint{0.908907in}{1.310357in}}{\pgfqpoint{0.903083in}{1.316181in}}%
\pgfpathcurveto{\pgfqpoint{0.897259in}{1.322005in}}{\pgfqpoint{0.889359in}{1.325277in}}{\pgfqpoint{0.881123in}{1.325277in}}%
\pgfpathcurveto{\pgfqpoint{0.872887in}{1.325277in}}{\pgfqpoint{0.864987in}{1.322005in}}{\pgfqpoint{0.859163in}{1.316181in}}%
\pgfpathcurveto{\pgfqpoint{0.853339in}{1.310357in}}{\pgfqpoint{0.850066in}{1.302457in}}{\pgfqpoint{0.850066in}{1.294221in}}%
\pgfpathcurveto{\pgfqpoint{0.850066in}{1.285984in}}{\pgfqpoint{0.853339in}{1.278084in}}{\pgfqpoint{0.859163in}{1.272260in}}%
\pgfpathcurveto{\pgfqpoint{0.864987in}{1.266436in}}{\pgfqpoint{0.872887in}{1.263164in}}{\pgfqpoint{0.881123in}{1.263164in}}%
\pgfpathclose%
\pgfusepath{stroke,fill}%
\end{pgfscope}%
\begin{pgfscope}%
\pgfpathrectangle{\pgfqpoint{0.100000in}{0.212622in}}{\pgfqpoint{3.696000in}{3.696000in}}%
\pgfusepath{clip}%
\pgfsetbuttcap%
\pgfsetroundjoin%
\definecolor{currentfill}{rgb}{0.121569,0.466667,0.705882}%
\pgfsetfillcolor{currentfill}%
\pgfsetfillopacity{0.633700}%
\pgfsetlinewidth{1.003750pt}%
\definecolor{currentstroke}{rgb}{0.121569,0.466667,0.705882}%
\pgfsetstrokecolor{currentstroke}%
\pgfsetstrokeopacity{0.633700}%
\pgfsetdash{}{0pt}%
\pgfpathmoveto{\pgfqpoint{0.886624in}{1.261963in}}%
\pgfpathcurveto{\pgfqpoint{0.894860in}{1.261963in}}{\pgfqpoint{0.902760in}{1.265235in}}{\pgfqpoint{0.908584in}{1.271059in}}%
\pgfpathcurveto{\pgfqpoint{0.914408in}{1.276883in}}{\pgfqpoint{0.917680in}{1.284783in}}{\pgfqpoint{0.917680in}{1.293019in}}%
\pgfpathcurveto{\pgfqpoint{0.917680in}{1.301256in}}{\pgfqpoint{0.914408in}{1.309156in}}{\pgfqpoint{0.908584in}{1.314980in}}%
\pgfpathcurveto{\pgfqpoint{0.902760in}{1.320804in}}{\pgfqpoint{0.894860in}{1.324076in}}{\pgfqpoint{0.886624in}{1.324076in}}%
\pgfpathcurveto{\pgfqpoint{0.878387in}{1.324076in}}{\pgfqpoint{0.870487in}{1.320804in}}{\pgfqpoint{0.864663in}{1.314980in}}%
\pgfpathcurveto{\pgfqpoint{0.858839in}{1.309156in}}{\pgfqpoint{0.855567in}{1.301256in}}{\pgfqpoint{0.855567in}{1.293019in}}%
\pgfpathcurveto{\pgfqpoint{0.855567in}{1.284783in}}{\pgfqpoint{0.858839in}{1.276883in}}{\pgfqpoint{0.864663in}{1.271059in}}%
\pgfpathcurveto{\pgfqpoint{0.870487in}{1.265235in}}{\pgfqpoint{0.878387in}{1.261963in}}{\pgfqpoint{0.886624in}{1.261963in}}%
\pgfpathclose%
\pgfusepath{stroke,fill}%
\end{pgfscope}%
\begin{pgfscope}%
\pgfpathrectangle{\pgfqpoint{0.100000in}{0.212622in}}{\pgfqpoint{3.696000in}{3.696000in}}%
\pgfusepath{clip}%
\pgfsetbuttcap%
\pgfsetroundjoin%
\definecolor{currentfill}{rgb}{0.121569,0.466667,0.705882}%
\pgfsetfillcolor{currentfill}%
\pgfsetfillopacity{0.634664}%
\pgfsetlinewidth{1.003750pt}%
\definecolor{currentstroke}{rgb}{0.121569,0.466667,0.705882}%
\pgfsetstrokecolor{currentstroke}%
\pgfsetstrokeopacity{0.634664}%
\pgfsetdash{}{0pt}%
\pgfpathmoveto{\pgfqpoint{2.130528in}{1.789586in}}%
\pgfpathcurveto{\pgfqpoint{2.138765in}{1.789586in}}{\pgfqpoint{2.146665in}{1.792858in}}{\pgfqpoint{2.152489in}{1.798682in}}%
\pgfpathcurveto{\pgfqpoint{2.158313in}{1.804506in}}{\pgfqpoint{2.161585in}{1.812406in}}{\pgfqpoint{2.161585in}{1.820642in}}%
\pgfpathcurveto{\pgfqpoint{2.161585in}{1.828879in}}{\pgfqpoint{2.158313in}{1.836779in}}{\pgfqpoint{2.152489in}{1.842603in}}%
\pgfpathcurveto{\pgfqpoint{2.146665in}{1.848426in}}{\pgfqpoint{2.138765in}{1.851699in}}{\pgfqpoint{2.130528in}{1.851699in}}%
\pgfpathcurveto{\pgfqpoint{2.122292in}{1.851699in}}{\pgfqpoint{2.114392in}{1.848426in}}{\pgfqpoint{2.108568in}{1.842603in}}%
\pgfpathcurveto{\pgfqpoint{2.102744in}{1.836779in}}{\pgfqpoint{2.099472in}{1.828879in}}{\pgfqpoint{2.099472in}{1.820642in}}%
\pgfpathcurveto{\pgfqpoint{2.099472in}{1.812406in}}{\pgfqpoint{2.102744in}{1.804506in}}{\pgfqpoint{2.108568in}{1.798682in}}%
\pgfpathcurveto{\pgfqpoint{2.114392in}{1.792858in}}{\pgfqpoint{2.122292in}{1.789586in}}{\pgfqpoint{2.130528in}{1.789586in}}%
\pgfpathclose%
\pgfusepath{stroke,fill}%
\end{pgfscope}%
\begin{pgfscope}%
\pgfpathrectangle{\pgfqpoint{0.100000in}{0.212622in}}{\pgfqpoint{3.696000in}{3.696000in}}%
\pgfusepath{clip}%
\pgfsetbuttcap%
\pgfsetroundjoin%
\definecolor{currentfill}{rgb}{0.121569,0.466667,0.705882}%
\pgfsetfillcolor{currentfill}%
\pgfsetfillopacity{0.635730}%
\pgfsetlinewidth{1.003750pt}%
\definecolor{currentstroke}{rgb}{0.121569,0.466667,0.705882}%
\pgfsetstrokecolor{currentstroke}%
\pgfsetstrokeopacity{0.635730}%
\pgfsetdash{}{0pt}%
\pgfpathmoveto{\pgfqpoint{2.131117in}{1.788919in}}%
\pgfpathcurveto{\pgfqpoint{2.139353in}{1.788919in}}{\pgfqpoint{2.147253in}{1.792192in}}{\pgfqpoint{2.153077in}{1.798015in}}%
\pgfpathcurveto{\pgfqpoint{2.158901in}{1.803839in}}{\pgfqpoint{2.162173in}{1.811739in}}{\pgfqpoint{2.162173in}{1.819976in}}%
\pgfpathcurveto{\pgfqpoint{2.162173in}{1.828212in}}{\pgfqpoint{2.158901in}{1.836112in}}{\pgfqpoint{2.153077in}{1.841936in}}%
\pgfpathcurveto{\pgfqpoint{2.147253in}{1.847760in}}{\pgfqpoint{2.139353in}{1.851032in}}{\pgfqpoint{2.131117in}{1.851032in}}%
\pgfpathcurveto{\pgfqpoint{2.122881in}{1.851032in}}{\pgfqpoint{2.114981in}{1.847760in}}{\pgfqpoint{2.109157in}{1.841936in}}%
\pgfpathcurveto{\pgfqpoint{2.103333in}{1.836112in}}{\pgfqpoint{2.100060in}{1.828212in}}{\pgfqpoint{2.100060in}{1.819976in}}%
\pgfpathcurveto{\pgfqpoint{2.100060in}{1.811739in}}{\pgfqpoint{2.103333in}{1.803839in}}{\pgfqpoint{2.109157in}{1.798015in}}%
\pgfpathcurveto{\pgfqpoint{2.114981in}{1.792192in}}{\pgfqpoint{2.122881in}{1.788919in}}{\pgfqpoint{2.131117in}{1.788919in}}%
\pgfpathclose%
\pgfusepath{stroke,fill}%
\end{pgfscope}%
\begin{pgfscope}%
\pgfpathrectangle{\pgfqpoint{0.100000in}{0.212622in}}{\pgfqpoint{3.696000in}{3.696000in}}%
\pgfusepath{clip}%
\pgfsetbuttcap%
\pgfsetroundjoin%
\definecolor{currentfill}{rgb}{0.121569,0.466667,0.705882}%
\pgfsetfillcolor{currentfill}%
\pgfsetfillopacity{0.637155}%
\pgfsetlinewidth{1.003750pt}%
\definecolor{currentstroke}{rgb}{0.121569,0.466667,0.705882}%
\pgfsetstrokecolor{currentstroke}%
\pgfsetstrokeopacity{0.637155}%
\pgfsetdash{}{0pt}%
\pgfpathmoveto{\pgfqpoint{2.132581in}{1.787095in}}%
\pgfpathcurveto{\pgfqpoint{2.140817in}{1.787095in}}{\pgfqpoint{2.148718in}{1.790367in}}{\pgfqpoint{2.154541in}{1.796191in}}%
\pgfpathcurveto{\pgfqpoint{2.160365in}{1.802015in}}{\pgfqpoint{2.163638in}{1.809915in}}{\pgfqpoint{2.163638in}{1.818151in}}%
\pgfpathcurveto{\pgfqpoint{2.163638in}{1.826388in}}{\pgfqpoint{2.160365in}{1.834288in}}{\pgfqpoint{2.154541in}{1.840111in}}%
\pgfpathcurveto{\pgfqpoint{2.148718in}{1.845935in}}{\pgfqpoint{2.140817in}{1.849208in}}{\pgfqpoint{2.132581in}{1.849208in}}%
\pgfpathcurveto{\pgfqpoint{2.124345in}{1.849208in}}{\pgfqpoint{2.116445in}{1.845935in}}{\pgfqpoint{2.110621in}{1.840111in}}%
\pgfpathcurveto{\pgfqpoint{2.104797in}{1.834288in}}{\pgfqpoint{2.101525in}{1.826388in}}{\pgfqpoint{2.101525in}{1.818151in}}%
\pgfpathcurveto{\pgfqpoint{2.101525in}{1.809915in}}{\pgfqpoint{2.104797in}{1.802015in}}{\pgfqpoint{2.110621in}{1.796191in}}%
\pgfpathcurveto{\pgfqpoint{2.116445in}{1.790367in}}{\pgfqpoint{2.124345in}{1.787095in}}{\pgfqpoint{2.132581in}{1.787095in}}%
\pgfpathclose%
\pgfusepath{stroke,fill}%
\end{pgfscope}%
\begin{pgfscope}%
\pgfpathrectangle{\pgfqpoint{0.100000in}{0.212622in}}{\pgfqpoint{3.696000in}{3.696000in}}%
\pgfusepath{clip}%
\pgfsetbuttcap%
\pgfsetroundjoin%
\definecolor{currentfill}{rgb}{0.121569,0.466667,0.705882}%
\pgfsetfillcolor{currentfill}%
\pgfsetfillopacity{0.638912}%
\pgfsetlinewidth{1.003750pt}%
\definecolor{currentstroke}{rgb}{0.121569,0.466667,0.705882}%
\pgfsetstrokecolor{currentstroke}%
\pgfsetstrokeopacity{0.638912}%
\pgfsetdash{}{0pt}%
\pgfpathmoveto{\pgfqpoint{2.133693in}{1.784593in}}%
\pgfpathcurveto{\pgfqpoint{2.141929in}{1.784593in}}{\pgfqpoint{2.149829in}{1.787865in}}{\pgfqpoint{2.155653in}{1.793689in}}%
\pgfpathcurveto{\pgfqpoint{2.161477in}{1.799513in}}{\pgfqpoint{2.164749in}{1.807413in}}{\pgfqpoint{2.164749in}{1.815649in}}%
\pgfpathcurveto{\pgfqpoint{2.164749in}{1.823886in}}{\pgfqpoint{2.161477in}{1.831786in}}{\pgfqpoint{2.155653in}{1.837610in}}%
\pgfpathcurveto{\pgfqpoint{2.149829in}{1.843434in}}{\pgfqpoint{2.141929in}{1.846706in}}{\pgfqpoint{2.133693in}{1.846706in}}%
\pgfpathcurveto{\pgfqpoint{2.125457in}{1.846706in}}{\pgfqpoint{2.117557in}{1.843434in}}{\pgfqpoint{2.111733in}{1.837610in}}%
\pgfpathcurveto{\pgfqpoint{2.105909in}{1.831786in}}{\pgfqpoint{2.102636in}{1.823886in}}{\pgfqpoint{2.102636in}{1.815649in}}%
\pgfpathcurveto{\pgfqpoint{2.102636in}{1.807413in}}{\pgfqpoint{2.105909in}{1.799513in}}{\pgfqpoint{2.111733in}{1.793689in}}%
\pgfpathcurveto{\pgfqpoint{2.117557in}{1.787865in}}{\pgfqpoint{2.125457in}{1.784593in}}{\pgfqpoint{2.133693in}{1.784593in}}%
\pgfpathclose%
\pgfusepath{stroke,fill}%
\end{pgfscope}%
\begin{pgfscope}%
\pgfpathrectangle{\pgfqpoint{0.100000in}{0.212622in}}{\pgfqpoint{3.696000in}{3.696000in}}%
\pgfusepath{clip}%
\pgfsetbuttcap%
\pgfsetroundjoin%
\definecolor{currentfill}{rgb}{0.121569,0.466667,0.705882}%
\pgfsetfillcolor{currentfill}%
\pgfsetfillopacity{0.641414}%
\pgfsetlinewidth{1.003750pt}%
\definecolor{currentstroke}{rgb}{0.121569,0.466667,0.705882}%
\pgfsetstrokecolor{currentstroke}%
\pgfsetstrokeopacity{0.641414}%
\pgfsetdash{}{0pt}%
\pgfpathmoveto{\pgfqpoint{2.135208in}{1.782311in}}%
\pgfpathcurveto{\pgfqpoint{2.143445in}{1.782311in}}{\pgfqpoint{2.151345in}{1.785584in}}{\pgfqpoint{2.157169in}{1.791407in}}%
\pgfpathcurveto{\pgfqpoint{2.162993in}{1.797231in}}{\pgfqpoint{2.166265in}{1.805131in}}{\pgfqpoint{2.166265in}{1.813368in}}%
\pgfpathcurveto{\pgfqpoint{2.166265in}{1.821604in}}{\pgfqpoint{2.162993in}{1.829504in}}{\pgfqpoint{2.157169in}{1.835328in}}%
\pgfpathcurveto{\pgfqpoint{2.151345in}{1.841152in}}{\pgfqpoint{2.143445in}{1.844424in}}{\pgfqpoint{2.135208in}{1.844424in}}%
\pgfpathcurveto{\pgfqpoint{2.126972in}{1.844424in}}{\pgfqpoint{2.119072in}{1.841152in}}{\pgfqpoint{2.113248in}{1.835328in}}%
\pgfpathcurveto{\pgfqpoint{2.107424in}{1.829504in}}{\pgfqpoint{2.104152in}{1.821604in}}{\pgfqpoint{2.104152in}{1.813368in}}%
\pgfpathcurveto{\pgfqpoint{2.104152in}{1.805131in}}{\pgfqpoint{2.107424in}{1.797231in}}{\pgfqpoint{2.113248in}{1.791407in}}%
\pgfpathcurveto{\pgfqpoint{2.119072in}{1.785584in}}{\pgfqpoint{2.126972in}{1.782311in}}{\pgfqpoint{2.135208in}{1.782311in}}%
\pgfpathclose%
\pgfusepath{stroke,fill}%
\end{pgfscope}%
\begin{pgfscope}%
\pgfpathrectangle{\pgfqpoint{0.100000in}{0.212622in}}{\pgfqpoint{3.696000in}{3.696000in}}%
\pgfusepath{clip}%
\pgfsetbuttcap%
\pgfsetroundjoin%
\definecolor{currentfill}{rgb}{0.121569,0.466667,0.705882}%
\pgfsetfillcolor{currentfill}%
\pgfsetfillopacity{0.644354}%
\pgfsetlinewidth{1.003750pt}%
\definecolor{currentstroke}{rgb}{0.121569,0.466667,0.705882}%
\pgfsetstrokecolor{currentstroke}%
\pgfsetstrokeopacity{0.644354}%
\pgfsetdash{}{0pt}%
\pgfpathmoveto{\pgfqpoint{2.137212in}{1.781498in}}%
\pgfpathcurveto{\pgfqpoint{2.145448in}{1.781498in}}{\pgfqpoint{2.153348in}{1.784770in}}{\pgfqpoint{2.159172in}{1.790594in}}%
\pgfpathcurveto{\pgfqpoint{2.164996in}{1.796418in}}{\pgfqpoint{2.168269in}{1.804318in}}{\pgfqpoint{2.168269in}{1.812554in}}%
\pgfpathcurveto{\pgfqpoint{2.168269in}{1.820790in}}{\pgfqpoint{2.164996in}{1.828690in}}{\pgfqpoint{2.159172in}{1.834514in}}%
\pgfpathcurveto{\pgfqpoint{2.153348in}{1.840338in}}{\pgfqpoint{2.145448in}{1.843611in}}{\pgfqpoint{2.137212in}{1.843611in}}%
\pgfpathcurveto{\pgfqpoint{2.128976in}{1.843611in}}{\pgfqpoint{2.121076in}{1.840338in}}{\pgfqpoint{2.115252in}{1.834514in}}%
\pgfpathcurveto{\pgfqpoint{2.109428in}{1.828690in}}{\pgfqpoint{2.106156in}{1.820790in}}{\pgfqpoint{2.106156in}{1.812554in}}%
\pgfpathcurveto{\pgfqpoint{2.106156in}{1.804318in}}{\pgfqpoint{2.109428in}{1.796418in}}{\pgfqpoint{2.115252in}{1.790594in}}%
\pgfpathcurveto{\pgfqpoint{2.121076in}{1.784770in}}{\pgfqpoint{2.128976in}{1.781498in}}{\pgfqpoint{2.137212in}{1.781498in}}%
\pgfpathclose%
\pgfusepath{stroke,fill}%
\end{pgfscope}%
\begin{pgfscope}%
\pgfpathrectangle{\pgfqpoint{0.100000in}{0.212622in}}{\pgfqpoint{3.696000in}{3.696000in}}%
\pgfusepath{clip}%
\pgfsetbuttcap%
\pgfsetroundjoin%
\definecolor{currentfill}{rgb}{0.121569,0.466667,0.705882}%
\pgfsetfillcolor{currentfill}%
\pgfsetfillopacity{0.647499}%
\pgfsetlinewidth{1.003750pt}%
\definecolor{currentstroke}{rgb}{0.121569,0.466667,0.705882}%
\pgfsetstrokecolor{currentstroke}%
\pgfsetstrokeopacity{0.647499}%
\pgfsetdash{}{0pt}%
\pgfpathmoveto{\pgfqpoint{2.139628in}{1.777754in}}%
\pgfpathcurveto{\pgfqpoint{2.147864in}{1.777754in}}{\pgfqpoint{2.155764in}{1.781027in}}{\pgfqpoint{2.161588in}{1.786851in}}%
\pgfpathcurveto{\pgfqpoint{2.167412in}{1.792675in}}{\pgfqpoint{2.170685in}{1.800575in}}{\pgfqpoint{2.170685in}{1.808811in}}%
\pgfpathcurveto{\pgfqpoint{2.170685in}{1.817047in}}{\pgfqpoint{2.167412in}{1.824947in}}{\pgfqpoint{2.161588in}{1.830771in}}%
\pgfpathcurveto{\pgfqpoint{2.155764in}{1.836595in}}{\pgfqpoint{2.147864in}{1.839867in}}{\pgfqpoint{2.139628in}{1.839867in}}%
\pgfpathcurveto{\pgfqpoint{2.131392in}{1.839867in}}{\pgfqpoint{2.123492in}{1.836595in}}{\pgfqpoint{2.117668in}{1.830771in}}%
\pgfpathcurveto{\pgfqpoint{2.111844in}{1.824947in}}{\pgfqpoint{2.108572in}{1.817047in}}{\pgfqpoint{2.108572in}{1.808811in}}%
\pgfpathcurveto{\pgfqpoint{2.108572in}{1.800575in}}{\pgfqpoint{2.111844in}{1.792675in}}{\pgfqpoint{2.117668in}{1.786851in}}%
\pgfpathcurveto{\pgfqpoint{2.123492in}{1.781027in}}{\pgfqpoint{2.131392in}{1.777754in}}{\pgfqpoint{2.139628in}{1.777754in}}%
\pgfpathclose%
\pgfusepath{stroke,fill}%
\end{pgfscope}%
\begin{pgfscope}%
\pgfpathrectangle{\pgfqpoint{0.100000in}{0.212622in}}{\pgfqpoint{3.696000in}{3.696000in}}%
\pgfusepath{clip}%
\pgfsetbuttcap%
\pgfsetroundjoin%
\definecolor{currentfill}{rgb}{0.121569,0.466667,0.705882}%
\pgfsetfillcolor{currentfill}%
\pgfsetfillopacity{0.650604}%
\pgfsetlinewidth{1.003750pt}%
\definecolor{currentstroke}{rgb}{0.121569,0.466667,0.705882}%
\pgfsetstrokecolor{currentstroke}%
\pgfsetstrokeopacity{0.650604}%
\pgfsetdash{}{0pt}%
\pgfpathmoveto{\pgfqpoint{2.141981in}{1.772097in}}%
\pgfpathcurveto{\pgfqpoint{2.150218in}{1.772097in}}{\pgfqpoint{2.158118in}{1.775369in}}{\pgfqpoint{2.163942in}{1.781193in}}%
\pgfpathcurveto{\pgfqpoint{2.169766in}{1.787017in}}{\pgfqpoint{2.173038in}{1.794917in}}{\pgfqpoint{2.173038in}{1.803153in}}%
\pgfpathcurveto{\pgfqpoint{2.173038in}{1.811390in}}{\pgfqpoint{2.169766in}{1.819290in}}{\pgfqpoint{2.163942in}{1.825114in}}%
\pgfpathcurveto{\pgfqpoint{2.158118in}{1.830937in}}{\pgfqpoint{2.150218in}{1.834210in}}{\pgfqpoint{2.141981in}{1.834210in}}%
\pgfpathcurveto{\pgfqpoint{2.133745in}{1.834210in}}{\pgfqpoint{2.125845in}{1.830937in}}{\pgfqpoint{2.120021in}{1.825114in}}%
\pgfpathcurveto{\pgfqpoint{2.114197in}{1.819290in}}{\pgfqpoint{2.110925in}{1.811390in}}{\pgfqpoint{2.110925in}{1.803153in}}%
\pgfpathcurveto{\pgfqpoint{2.110925in}{1.794917in}}{\pgfqpoint{2.114197in}{1.787017in}}{\pgfqpoint{2.120021in}{1.781193in}}%
\pgfpathcurveto{\pgfqpoint{2.125845in}{1.775369in}}{\pgfqpoint{2.133745in}{1.772097in}}{\pgfqpoint{2.141981in}{1.772097in}}%
\pgfpathclose%
\pgfusepath{stroke,fill}%
\end{pgfscope}%
\begin{pgfscope}%
\pgfpathrectangle{\pgfqpoint{0.100000in}{0.212622in}}{\pgfqpoint{3.696000in}{3.696000in}}%
\pgfusepath{clip}%
\pgfsetbuttcap%
\pgfsetroundjoin%
\definecolor{currentfill}{rgb}{0.121569,0.466667,0.705882}%
\pgfsetfillcolor{currentfill}%
\pgfsetfillopacity{0.655020}%
\pgfsetlinewidth{1.003750pt}%
\definecolor{currentstroke}{rgb}{0.121569,0.466667,0.705882}%
\pgfsetstrokecolor{currentstroke}%
\pgfsetstrokeopacity{0.655020}%
\pgfsetdash{}{0pt}%
\pgfpathmoveto{\pgfqpoint{2.144415in}{1.768939in}}%
\pgfpathcurveto{\pgfqpoint{2.152651in}{1.768939in}}{\pgfqpoint{2.160551in}{1.772212in}}{\pgfqpoint{2.166375in}{1.778036in}}%
\pgfpathcurveto{\pgfqpoint{2.172199in}{1.783860in}}{\pgfqpoint{2.175471in}{1.791760in}}{\pgfqpoint{2.175471in}{1.799996in}}%
\pgfpathcurveto{\pgfqpoint{2.175471in}{1.808232in}}{\pgfqpoint{2.172199in}{1.816132in}}{\pgfqpoint{2.166375in}{1.821956in}}%
\pgfpathcurveto{\pgfqpoint{2.160551in}{1.827780in}}{\pgfqpoint{2.152651in}{1.831052in}}{\pgfqpoint{2.144415in}{1.831052in}}%
\pgfpathcurveto{\pgfqpoint{2.136178in}{1.831052in}}{\pgfqpoint{2.128278in}{1.827780in}}{\pgfqpoint{2.122454in}{1.821956in}}%
\pgfpathcurveto{\pgfqpoint{2.116631in}{1.816132in}}{\pgfqpoint{2.113358in}{1.808232in}}{\pgfqpoint{2.113358in}{1.799996in}}%
\pgfpathcurveto{\pgfqpoint{2.113358in}{1.791760in}}{\pgfqpoint{2.116631in}{1.783860in}}{\pgfqpoint{2.122454in}{1.778036in}}%
\pgfpathcurveto{\pgfqpoint{2.128278in}{1.772212in}}{\pgfqpoint{2.136178in}{1.768939in}}{\pgfqpoint{2.144415in}{1.768939in}}%
\pgfpathclose%
\pgfusepath{stroke,fill}%
\end{pgfscope}%
\begin{pgfscope}%
\pgfpathrectangle{\pgfqpoint{0.100000in}{0.212622in}}{\pgfqpoint{3.696000in}{3.696000in}}%
\pgfusepath{clip}%
\pgfsetbuttcap%
\pgfsetroundjoin%
\definecolor{currentfill}{rgb}{0.121569,0.466667,0.705882}%
\pgfsetfillcolor{currentfill}%
\pgfsetfillopacity{0.660013}%
\pgfsetlinewidth{1.003750pt}%
\definecolor{currentstroke}{rgb}{0.121569,0.466667,0.705882}%
\pgfsetstrokecolor{currentstroke}%
\pgfsetstrokeopacity{0.660013}%
\pgfsetdash{}{0pt}%
\pgfpathmoveto{\pgfqpoint{2.146556in}{1.766948in}}%
\pgfpathcurveto{\pgfqpoint{2.154793in}{1.766948in}}{\pgfqpoint{2.162693in}{1.770221in}}{\pgfqpoint{2.168517in}{1.776045in}}%
\pgfpathcurveto{\pgfqpoint{2.174341in}{1.781869in}}{\pgfqpoint{2.177613in}{1.789769in}}{\pgfqpoint{2.177613in}{1.798005in}}%
\pgfpathcurveto{\pgfqpoint{2.177613in}{1.806241in}}{\pgfqpoint{2.174341in}{1.814141in}}{\pgfqpoint{2.168517in}{1.819965in}}%
\pgfpathcurveto{\pgfqpoint{2.162693in}{1.825789in}}{\pgfqpoint{2.154793in}{1.829061in}}{\pgfqpoint{2.146556in}{1.829061in}}%
\pgfpathcurveto{\pgfqpoint{2.138320in}{1.829061in}}{\pgfqpoint{2.130420in}{1.825789in}}{\pgfqpoint{2.124596in}{1.819965in}}%
\pgfpathcurveto{\pgfqpoint{2.118772in}{1.814141in}}{\pgfqpoint{2.115500in}{1.806241in}}{\pgfqpoint{2.115500in}{1.798005in}}%
\pgfpathcurveto{\pgfqpoint{2.115500in}{1.789769in}}{\pgfqpoint{2.118772in}{1.781869in}}{\pgfqpoint{2.124596in}{1.776045in}}%
\pgfpathcurveto{\pgfqpoint{2.130420in}{1.770221in}}{\pgfqpoint{2.138320in}{1.766948in}}{\pgfqpoint{2.146556in}{1.766948in}}%
\pgfpathclose%
\pgfusepath{stroke,fill}%
\end{pgfscope}%
\begin{pgfscope}%
\pgfpathrectangle{\pgfqpoint{0.100000in}{0.212622in}}{\pgfqpoint{3.696000in}{3.696000in}}%
\pgfusepath{clip}%
\pgfsetbuttcap%
\pgfsetroundjoin%
\definecolor{currentfill}{rgb}{0.121569,0.466667,0.705882}%
\pgfsetfillcolor{currentfill}%
\pgfsetfillopacity{0.665210}%
\pgfsetlinewidth{1.003750pt}%
\definecolor{currentstroke}{rgb}{0.121569,0.466667,0.705882}%
\pgfsetstrokecolor{currentstroke}%
\pgfsetstrokeopacity{0.665210}%
\pgfsetdash{}{0pt}%
\pgfpathmoveto{\pgfqpoint{2.150170in}{1.762055in}}%
\pgfpathcurveto{\pgfqpoint{2.158406in}{1.762055in}}{\pgfqpoint{2.166306in}{1.765327in}}{\pgfqpoint{2.172130in}{1.771151in}}%
\pgfpathcurveto{\pgfqpoint{2.177954in}{1.776975in}}{\pgfqpoint{2.181226in}{1.784875in}}{\pgfqpoint{2.181226in}{1.793112in}}%
\pgfpathcurveto{\pgfqpoint{2.181226in}{1.801348in}}{\pgfqpoint{2.177954in}{1.809248in}}{\pgfqpoint{2.172130in}{1.815072in}}%
\pgfpathcurveto{\pgfqpoint{2.166306in}{1.820896in}}{\pgfqpoint{2.158406in}{1.824168in}}{\pgfqpoint{2.150170in}{1.824168in}}%
\pgfpathcurveto{\pgfqpoint{2.141933in}{1.824168in}}{\pgfqpoint{2.134033in}{1.820896in}}{\pgfqpoint{2.128209in}{1.815072in}}%
\pgfpathcurveto{\pgfqpoint{2.122386in}{1.809248in}}{\pgfqpoint{2.119113in}{1.801348in}}{\pgfqpoint{2.119113in}{1.793112in}}%
\pgfpathcurveto{\pgfqpoint{2.119113in}{1.784875in}}{\pgfqpoint{2.122386in}{1.776975in}}{\pgfqpoint{2.128209in}{1.771151in}}%
\pgfpathcurveto{\pgfqpoint{2.134033in}{1.765327in}}{\pgfqpoint{2.141933in}{1.762055in}}{\pgfqpoint{2.150170in}{1.762055in}}%
\pgfpathclose%
\pgfusepath{stroke,fill}%
\end{pgfscope}%
\begin{pgfscope}%
\pgfpathrectangle{\pgfqpoint{0.100000in}{0.212622in}}{\pgfqpoint{3.696000in}{3.696000in}}%
\pgfusepath{clip}%
\pgfsetbuttcap%
\pgfsetroundjoin%
\definecolor{currentfill}{rgb}{0.121569,0.466667,0.705882}%
\pgfsetfillcolor{currentfill}%
\pgfsetfillopacity{0.670508}%
\pgfsetlinewidth{1.003750pt}%
\definecolor{currentstroke}{rgb}{0.121569,0.466667,0.705882}%
\pgfsetstrokecolor{currentstroke}%
\pgfsetstrokeopacity{0.670508}%
\pgfsetdash{}{0pt}%
\pgfpathmoveto{\pgfqpoint{2.153918in}{1.754811in}}%
\pgfpathcurveto{\pgfqpoint{2.162154in}{1.754811in}}{\pgfqpoint{2.170054in}{1.758084in}}{\pgfqpoint{2.175878in}{1.763907in}}%
\pgfpathcurveto{\pgfqpoint{2.181702in}{1.769731in}}{\pgfqpoint{2.184974in}{1.777631in}}{\pgfqpoint{2.184974in}{1.785868in}}%
\pgfpathcurveto{\pgfqpoint{2.184974in}{1.794104in}}{\pgfqpoint{2.181702in}{1.802004in}}{\pgfqpoint{2.175878in}{1.807828in}}%
\pgfpathcurveto{\pgfqpoint{2.170054in}{1.813652in}}{\pgfqpoint{2.162154in}{1.816924in}}{\pgfqpoint{2.153918in}{1.816924in}}%
\pgfpathcurveto{\pgfqpoint{2.145681in}{1.816924in}}{\pgfqpoint{2.137781in}{1.813652in}}{\pgfqpoint{2.131957in}{1.807828in}}%
\pgfpathcurveto{\pgfqpoint{2.126134in}{1.802004in}}{\pgfqpoint{2.122861in}{1.794104in}}{\pgfqpoint{2.122861in}{1.785868in}}%
\pgfpathcurveto{\pgfqpoint{2.122861in}{1.777631in}}{\pgfqpoint{2.126134in}{1.769731in}}{\pgfqpoint{2.131957in}{1.763907in}}%
\pgfpathcurveto{\pgfqpoint{2.137781in}{1.758084in}}{\pgfqpoint{2.145681in}{1.754811in}}{\pgfqpoint{2.153918in}{1.754811in}}%
\pgfpathclose%
\pgfusepath{stroke,fill}%
\end{pgfscope}%
\begin{pgfscope}%
\pgfpathrectangle{\pgfqpoint{0.100000in}{0.212622in}}{\pgfqpoint{3.696000in}{3.696000in}}%
\pgfusepath{clip}%
\pgfsetbuttcap%
\pgfsetroundjoin%
\definecolor{currentfill}{rgb}{0.121569,0.466667,0.705882}%
\pgfsetfillcolor{currentfill}%
\pgfsetfillopacity{0.676803}%
\pgfsetlinewidth{1.003750pt}%
\definecolor{currentstroke}{rgb}{0.121569,0.466667,0.705882}%
\pgfsetstrokecolor{currentstroke}%
\pgfsetstrokeopacity{0.676803}%
\pgfsetdash{}{0pt}%
\pgfpathmoveto{\pgfqpoint{2.158878in}{1.748746in}}%
\pgfpathcurveto{\pgfqpoint{2.167114in}{1.748746in}}{\pgfqpoint{2.175015in}{1.752018in}}{\pgfqpoint{2.180838in}{1.757842in}}%
\pgfpathcurveto{\pgfqpoint{2.186662in}{1.763666in}}{\pgfqpoint{2.189935in}{1.771566in}}{\pgfqpoint{2.189935in}{1.779802in}}%
\pgfpathcurveto{\pgfqpoint{2.189935in}{1.788039in}}{\pgfqpoint{2.186662in}{1.795939in}}{\pgfqpoint{2.180838in}{1.801763in}}%
\pgfpathcurveto{\pgfqpoint{2.175015in}{1.807587in}}{\pgfqpoint{2.167114in}{1.810859in}}{\pgfqpoint{2.158878in}{1.810859in}}%
\pgfpathcurveto{\pgfqpoint{2.150642in}{1.810859in}}{\pgfqpoint{2.142742in}{1.807587in}}{\pgfqpoint{2.136918in}{1.801763in}}%
\pgfpathcurveto{\pgfqpoint{2.131094in}{1.795939in}}{\pgfqpoint{2.127822in}{1.788039in}}{\pgfqpoint{2.127822in}{1.779802in}}%
\pgfpathcurveto{\pgfqpoint{2.127822in}{1.771566in}}{\pgfqpoint{2.131094in}{1.763666in}}{\pgfqpoint{2.136918in}{1.757842in}}%
\pgfpathcurveto{\pgfqpoint{2.142742in}{1.752018in}}{\pgfqpoint{2.150642in}{1.748746in}}{\pgfqpoint{2.158878in}{1.748746in}}%
\pgfpathclose%
\pgfusepath{stroke,fill}%
\end{pgfscope}%
\begin{pgfscope}%
\pgfpathrectangle{\pgfqpoint{0.100000in}{0.212622in}}{\pgfqpoint{3.696000in}{3.696000in}}%
\pgfusepath{clip}%
\pgfsetbuttcap%
\pgfsetroundjoin%
\definecolor{currentfill}{rgb}{0.121569,0.466667,0.705882}%
\pgfsetfillcolor{currentfill}%
\pgfsetfillopacity{0.683863}%
\pgfsetlinewidth{1.003750pt}%
\definecolor{currentstroke}{rgb}{0.121569,0.466667,0.705882}%
\pgfsetstrokecolor{currentstroke}%
\pgfsetstrokeopacity{0.683863}%
\pgfsetdash{}{0pt}%
\pgfpathmoveto{\pgfqpoint{2.163397in}{1.744262in}}%
\pgfpathcurveto{\pgfqpoint{2.171634in}{1.744262in}}{\pgfqpoint{2.179534in}{1.747535in}}{\pgfqpoint{2.185358in}{1.753359in}}%
\pgfpathcurveto{\pgfqpoint{2.191182in}{1.759183in}}{\pgfqpoint{2.194454in}{1.767083in}}{\pgfqpoint{2.194454in}{1.775319in}}%
\pgfpathcurveto{\pgfqpoint{2.194454in}{1.783555in}}{\pgfqpoint{2.191182in}{1.791455in}}{\pgfqpoint{2.185358in}{1.797279in}}%
\pgfpathcurveto{\pgfqpoint{2.179534in}{1.803103in}}{\pgfqpoint{2.171634in}{1.806375in}}{\pgfqpoint{2.163397in}{1.806375in}}%
\pgfpathcurveto{\pgfqpoint{2.155161in}{1.806375in}}{\pgfqpoint{2.147261in}{1.803103in}}{\pgfqpoint{2.141437in}{1.797279in}}%
\pgfpathcurveto{\pgfqpoint{2.135613in}{1.791455in}}{\pgfqpoint{2.132341in}{1.783555in}}{\pgfqpoint{2.132341in}{1.775319in}}%
\pgfpathcurveto{\pgfqpoint{2.132341in}{1.767083in}}{\pgfqpoint{2.135613in}{1.759183in}}{\pgfqpoint{2.141437in}{1.753359in}}%
\pgfpathcurveto{\pgfqpoint{2.147261in}{1.747535in}}{\pgfqpoint{2.155161in}{1.744262in}}{\pgfqpoint{2.163397in}{1.744262in}}%
\pgfpathclose%
\pgfusepath{stroke,fill}%
\end{pgfscope}%
\begin{pgfscope}%
\pgfpathrectangle{\pgfqpoint{0.100000in}{0.212622in}}{\pgfqpoint{3.696000in}{3.696000in}}%
\pgfusepath{clip}%
\pgfsetbuttcap%
\pgfsetroundjoin%
\definecolor{currentfill}{rgb}{0.121569,0.466667,0.705882}%
\pgfsetfillcolor{currentfill}%
\pgfsetfillopacity{0.690389}%
\pgfsetlinewidth{1.003750pt}%
\definecolor{currentstroke}{rgb}{0.121569,0.466667,0.705882}%
\pgfsetstrokecolor{currentstroke}%
\pgfsetstrokeopacity{0.690389}%
\pgfsetdash{}{0pt}%
\pgfpathmoveto{\pgfqpoint{2.167401in}{1.732142in}}%
\pgfpathcurveto{\pgfqpoint{2.175638in}{1.732142in}}{\pgfqpoint{2.183538in}{1.735414in}}{\pgfqpoint{2.189362in}{1.741238in}}%
\pgfpathcurveto{\pgfqpoint{2.195185in}{1.747062in}}{\pgfqpoint{2.198458in}{1.754962in}}{\pgfqpoint{2.198458in}{1.763198in}}%
\pgfpathcurveto{\pgfqpoint{2.198458in}{1.771434in}}{\pgfqpoint{2.195185in}{1.779335in}}{\pgfqpoint{2.189362in}{1.785158in}}%
\pgfpathcurveto{\pgfqpoint{2.183538in}{1.790982in}}{\pgfqpoint{2.175638in}{1.794255in}}{\pgfqpoint{2.167401in}{1.794255in}}%
\pgfpathcurveto{\pgfqpoint{2.159165in}{1.794255in}}{\pgfqpoint{2.151265in}{1.790982in}}{\pgfqpoint{2.145441in}{1.785158in}}%
\pgfpathcurveto{\pgfqpoint{2.139617in}{1.779335in}}{\pgfqpoint{2.136345in}{1.771434in}}{\pgfqpoint{2.136345in}{1.763198in}}%
\pgfpathcurveto{\pgfqpoint{2.136345in}{1.754962in}}{\pgfqpoint{2.139617in}{1.747062in}}{\pgfqpoint{2.145441in}{1.741238in}}%
\pgfpathcurveto{\pgfqpoint{2.151265in}{1.735414in}}{\pgfqpoint{2.159165in}{1.732142in}}{\pgfqpoint{2.167401in}{1.732142in}}%
\pgfpathclose%
\pgfusepath{stroke,fill}%
\end{pgfscope}%
\begin{pgfscope}%
\pgfpathrectangle{\pgfqpoint{0.100000in}{0.212622in}}{\pgfqpoint{3.696000in}{3.696000in}}%
\pgfusepath{clip}%
\pgfsetbuttcap%
\pgfsetroundjoin%
\definecolor{currentfill}{rgb}{0.121569,0.466667,0.705882}%
\pgfsetfillcolor{currentfill}%
\pgfsetfillopacity{0.696981}%
\pgfsetlinewidth{1.003750pt}%
\definecolor{currentstroke}{rgb}{0.121569,0.466667,0.705882}%
\pgfsetstrokecolor{currentstroke}%
\pgfsetstrokeopacity{0.696981}%
\pgfsetdash{}{0pt}%
\pgfpathmoveto{\pgfqpoint{2.171736in}{1.719305in}}%
\pgfpathcurveto{\pgfqpoint{2.179972in}{1.719305in}}{\pgfqpoint{2.187872in}{1.722578in}}{\pgfqpoint{2.193696in}{1.728402in}}%
\pgfpathcurveto{\pgfqpoint{2.199520in}{1.734225in}}{\pgfqpoint{2.202792in}{1.742126in}}{\pgfqpoint{2.202792in}{1.750362in}}%
\pgfpathcurveto{\pgfqpoint{2.202792in}{1.758598in}}{\pgfqpoint{2.199520in}{1.766498in}}{\pgfqpoint{2.193696in}{1.772322in}}%
\pgfpathcurveto{\pgfqpoint{2.187872in}{1.778146in}}{\pgfqpoint{2.179972in}{1.781418in}}{\pgfqpoint{2.171736in}{1.781418in}}%
\pgfpathcurveto{\pgfqpoint{2.163500in}{1.781418in}}{\pgfqpoint{2.155600in}{1.778146in}}{\pgfqpoint{2.149776in}{1.772322in}}%
\pgfpathcurveto{\pgfqpoint{2.143952in}{1.766498in}}{\pgfqpoint{2.140679in}{1.758598in}}{\pgfqpoint{2.140679in}{1.750362in}}%
\pgfpathcurveto{\pgfqpoint{2.140679in}{1.742126in}}{\pgfqpoint{2.143952in}{1.734225in}}{\pgfqpoint{2.149776in}{1.728402in}}%
\pgfpathcurveto{\pgfqpoint{2.155600in}{1.722578in}}{\pgfqpoint{2.163500in}{1.719305in}}{\pgfqpoint{2.171736in}{1.719305in}}%
\pgfpathclose%
\pgfusepath{stroke,fill}%
\end{pgfscope}%
\begin{pgfscope}%
\pgfpathrectangle{\pgfqpoint{0.100000in}{0.212622in}}{\pgfqpoint{3.696000in}{3.696000in}}%
\pgfusepath{clip}%
\pgfsetbuttcap%
\pgfsetroundjoin%
\definecolor{currentfill}{rgb}{0.121569,0.466667,0.705882}%
\pgfsetfillcolor{currentfill}%
\pgfsetfillopacity{0.701312}%
\pgfsetlinewidth{1.003750pt}%
\definecolor{currentstroke}{rgb}{0.121569,0.466667,0.705882}%
\pgfsetstrokecolor{currentstroke}%
\pgfsetstrokeopacity{0.701312}%
\pgfsetdash{}{0pt}%
\pgfpathmoveto{\pgfqpoint{2.175177in}{1.717263in}}%
\pgfpathcurveto{\pgfqpoint{2.183413in}{1.717263in}}{\pgfqpoint{2.191313in}{1.720535in}}{\pgfqpoint{2.197137in}{1.726359in}}%
\pgfpathcurveto{\pgfqpoint{2.202961in}{1.732183in}}{\pgfqpoint{2.206234in}{1.740083in}}{\pgfqpoint{2.206234in}{1.748319in}}%
\pgfpathcurveto{\pgfqpoint{2.206234in}{1.756556in}}{\pgfqpoint{2.202961in}{1.764456in}}{\pgfqpoint{2.197137in}{1.770280in}}%
\pgfpathcurveto{\pgfqpoint{2.191313in}{1.776104in}}{\pgfqpoint{2.183413in}{1.779376in}}{\pgfqpoint{2.175177in}{1.779376in}}%
\pgfpathcurveto{\pgfqpoint{2.166941in}{1.779376in}}{\pgfqpoint{2.159041in}{1.776104in}}{\pgfqpoint{2.153217in}{1.770280in}}%
\pgfpathcurveto{\pgfqpoint{2.147393in}{1.764456in}}{\pgfqpoint{2.144121in}{1.756556in}}{\pgfqpoint{2.144121in}{1.748319in}}%
\pgfpathcurveto{\pgfqpoint{2.144121in}{1.740083in}}{\pgfqpoint{2.147393in}{1.732183in}}{\pgfqpoint{2.153217in}{1.726359in}}%
\pgfpathcurveto{\pgfqpoint{2.159041in}{1.720535in}}{\pgfqpoint{2.166941in}{1.717263in}}{\pgfqpoint{2.175177in}{1.717263in}}%
\pgfpathclose%
\pgfusepath{stroke,fill}%
\end{pgfscope}%
\begin{pgfscope}%
\pgfpathrectangle{\pgfqpoint{0.100000in}{0.212622in}}{\pgfqpoint{3.696000in}{3.696000in}}%
\pgfusepath{clip}%
\pgfsetbuttcap%
\pgfsetroundjoin%
\definecolor{currentfill}{rgb}{0.121569,0.466667,0.705882}%
\pgfsetfillcolor{currentfill}%
\pgfsetfillopacity{0.705604}%
\pgfsetlinewidth{1.003750pt}%
\definecolor{currentstroke}{rgb}{0.121569,0.466667,0.705882}%
\pgfsetstrokecolor{currentstroke}%
\pgfsetstrokeopacity{0.705604}%
\pgfsetdash{}{0pt}%
\pgfpathmoveto{\pgfqpoint{2.178600in}{1.713794in}}%
\pgfpathcurveto{\pgfqpoint{2.186836in}{1.713794in}}{\pgfqpoint{2.194736in}{1.717067in}}{\pgfqpoint{2.200560in}{1.722890in}}%
\pgfpathcurveto{\pgfqpoint{2.206384in}{1.728714in}}{\pgfqpoint{2.209656in}{1.736614in}}{\pgfqpoint{2.209656in}{1.744851in}}%
\pgfpathcurveto{\pgfqpoint{2.209656in}{1.753087in}}{\pgfqpoint{2.206384in}{1.760987in}}{\pgfqpoint{2.200560in}{1.766811in}}%
\pgfpathcurveto{\pgfqpoint{2.194736in}{1.772635in}}{\pgfqpoint{2.186836in}{1.775907in}}{\pgfqpoint{2.178600in}{1.775907in}}%
\pgfpathcurveto{\pgfqpoint{2.170364in}{1.775907in}}{\pgfqpoint{2.162464in}{1.772635in}}{\pgfqpoint{2.156640in}{1.766811in}}%
\pgfpathcurveto{\pgfqpoint{2.150816in}{1.760987in}}{\pgfqpoint{2.147543in}{1.753087in}}{\pgfqpoint{2.147543in}{1.744851in}}%
\pgfpathcurveto{\pgfqpoint{2.147543in}{1.736614in}}{\pgfqpoint{2.150816in}{1.728714in}}{\pgfqpoint{2.156640in}{1.722890in}}%
\pgfpathcurveto{\pgfqpoint{2.162464in}{1.717067in}}{\pgfqpoint{2.170364in}{1.713794in}}{\pgfqpoint{2.178600in}{1.713794in}}%
\pgfpathclose%
\pgfusepath{stroke,fill}%
\end{pgfscope}%
\begin{pgfscope}%
\pgfpathrectangle{\pgfqpoint{0.100000in}{0.212622in}}{\pgfqpoint{3.696000in}{3.696000in}}%
\pgfusepath{clip}%
\pgfsetbuttcap%
\pgfsetroundjoin%
\definecolor{currentfill}{rgb}{0.121569,0.466667,0.705882}%
\pgfsetfillcolor{currentfill}%
\pgfsetfillopacity{0.707539}%
\pgfsetlinewidth{1.003750pt}%
\definecolor{currentstroke}{rgb}{0.121569,0.466667,0.705882}%
\pgfsetstrokecolor{currentstroke}%
\pgfsetstrokeopacity{0.707539}%
\pgfsetdash{}{0pt}%
\pgfpathmoveto{\pgfqpoint{2.180508in}{1.709173in}}%
\pgfpathcurveto{\pgfqpoint{2.188744in}{1.709173in}}{\pgfqpoint{2.196644in}{1.712446in}}{\pgfqpoint{2.202468in}{1.718270in}}%
\pgfpathcurveto{\pgfqpoint{2.208292in}{1.724094in}}{\pgfqpoint{2.211564in}{1.731994in}}{\pgfqpoint{2.211564in}{1.740230in}}%
\pgfpathcurveto{\pgfqpoint{2.211564in}{1.748466in}}{\pgfqpoint{2.208292in}{1.756366in}}{\pgfqpoint{2.202468in}{1.762190in}}%
\pgfpathcurveto{\pgfqpoint{2.196644in}{1.768014in}}{\pgfqpoint{2.188744in}{1.771286in}}{\pgfqpoint{2.180508in}{1.771286in}}%
\pgfpathcurveto{\pgfqpoint{2.172272in}{1.771286in}}{\pgfqpoint{2.164372in}{1.768014in}}{\pgfqpoint{2.158548in}{1.762190in}}%
\pgfpathcurveto{\pgfqpoint{2.152724in}{1.756366in}}{\pgfqpoint{2.149451in}{1.748466in}}{\pgfqpoint{2.149451in}{1.740230in}}%
\pgfpathcurveto{\pgfqpoint{2.149451in}{1.731994in}}{\pgfqpoint{2.152724in}{1.724094in}}{\pgfqpoint{2.158548in}{1.718270in}}%
\pgfpathcurveto{\pgfqpoint{2.164372in}{1.712446in}}{\pgfqpoint{2.172272in}{1.709173in}}{\pgfqpoint{2.180508in}{1.709173in}}%
\pgfpathclose%
\pgfusepath{stroke,fill}%
\end{pgfscope}%
\begin{pgfscope}%
\pgfpathrectangle{\pgfqpoint{0.100000in}{0.212622in}}{\pgfqpoint{3.696000in}{3.696000in}}%
\pgfusepath{clip}%
\pgfsetbuttcap%
\pgfsetroundjoin%
\definecolor{currentfill}{rgb}{0.121569,0.466667,0.705882}%
\pgfsetfillcolor{currentfill}%
\pgfsetfillopacity{0.708668}%
\pgfsetlinewidth{1.003750pt}%
\definecolor{currentstroke}{rgb}{0.121569,0.466667,0.705882}%
\pgfsetstrokecolor{currentstroke}%
\pgfsetstrokeopacity{0.708668}%
\pgfsetdash{}{0pt}%
\pgfpathmoveto{\pgfqpoint{2.181577in}{1.707050in}}%
\pgfpathcurveto{\pgfqpoint{2.189813in}{1.707050in}}{\pgfqpoint{2.197713in}{1.710322in}}{\pgfqpoint{2.203537in}{1.716146in}}%
\pgfpathcurveto{\pgfqpoint{2.209361in}{1.721970in}}{\pgfqpoint{2.212634in}{1.729870in}}{\pgfqpoint{2.212634in}{1.738106in}}%
\pgfpathcurveto{\pgfqpoint{2.212634in}{1.746343in}}{\pgfqpoint{2.209361in}{1.754243in}}{\pgfqpoint{2.203537in}{1.760067in}}%
\pgfpathcurveto{\pgfqpoint{2.197713in}{1.765891in}}{\pgfqpoint{2.189813in}{1.769163in}}{\pgfqpoint{2.181577in}{1.769163in}}%
\pgfpathcurveto{\pgfqpoint{2.173341in}{1.769163in}}{\pgfqpoint{2.165441in}{1.765891in}}{\pgfqpoint{2.159617in}{1.760067in}}%
\pgfpathcurveto{\pgfqpoint{2.153793in}{1.754243in}}{\pgfqpoint{2.150521in}{1.746343in}}{\pgfqpoint{2.150521in}{1.738106in}}%
\pgfpathcurveto{\pgfqpoint{2.150521in}{1.729870in}}{\pgfqpoint{2.153793in}{1.721970in}}{\pgfqpoint{2.159617in}{1.716146in}}%
\pgfpathcurveto{\pgfqpoint{2.165441in}{1.710322in}}{\pgfqpoint{2.173341in}{1.707050in}}{\pgfqpoint{2.181577in}{1.707050in}}%
\pgfpathclose%
\pgfusepath{stroke,fill}%
\end{pgfscope}%
\begin{pgfscope}%
\pgfpathrectangle{\pgfqpoint{0.100000in}{0.212622in}}{\pgfqpoint{3.696000in}{3.696000in}}%
\pgfusepath{clip}%
\pgfsetbuttcap%
\pgfsetroundjoin%
\definecolor{currentfill}{rgb}{0.121569,0.466667,0.705882}%
\pgfsetfillcolor{currentfill}%
\pgfsetfillopacity{0.709420}%
\pgfsetlinewidth{1.003750pt}%
\definecolor{currentstroke}{rgb}{0.121569,0.466667,0.705882}%
\pgfsetstrokecolor{currentstroke}%
\pgfsetstrokeopacity{0.709420}%
\pgfsetdash{}{0pt}%
\pgfpathmoveto{\pgfqpoint{2.182106in}{1.706675in}}%
\pgfpathcurveto{\pgfqpoint{2.190342in}{1.706675in}}{\pgfqpoint{2.198242in}{1.709948in}}{\pgfqpoint{2.204066in}{1.715772in}}%
\pgfpathcurveto{\pgfqpoint{2.209890in}{1.721596in}}{\pgfqpoint{2.213163in}{1.729496in}}{\pgfqpoint{2.213163in}{1.737732in}}%
\pgfpathcurveto{\pgfqpoint{2.213163in}{1.745968in}}{\pgfqpoint{2.209890in}{1.753868in}}{\pgfqpoint{2.204066in}{1.759692in}}%
\pgfpathcurveto{\pgfqpoint{2.198242in}{1.765516in}}{\pgfqpoint{2.190342in}{1.768788in}}{\pgfqpoint{2.182106in}{1.768788in}}%
\pgfpathcurveto{\pgfqpoint{2.173870in}{1.768788in}}{\pgfqpoint{2.165970in}{1.765516in}}{\pgfqpoint{2.160146in}{1.759692in}}%
\pgfpathcurveto{\pgfqpoint{2.154322in}{1.753868in}}{\pgfqpoint{2.151050in}{1.745968in}}{\pgfqpoint{2.151050in}{1.737732in}}%
\pgfpathcurveto{\pgfqpoint{2.151050in}{1.729496in}}{\pgfqpoint{2.154322in}{1.721596in}}{\pgfqpoint{2.160146in}{1.715772in}}%
\pgfpathcurveto{\pgfqpoint{2.165970in}{1.709948in}}{\pgfqpoint{2.173870in}{1.706675in}}{\pgfqpoint{2.182106in}{1.706675in}}%
\pgfpathclose%
\pgfusepath{stroke,fill}%
\end{pgfscope}%
\begin{pgfscope}%
\pgfpathrectangle{\pgfqpoint{0.100000in}{0.212622in}}{\pgfqpoint{3.696000in}{3.696000in}}%
\pgfusepath{clip}%
\pgfsetbuttcap%
\pgfsetroundjoin%
\definecolor{currentfill}{rgb}{0.121569,0.466667,0.705882}%
\pgfsetfillcolor{currentfill}%
\pgfsetfillopacity{0.710310}%
\pgfsetlinewidth{1.003750pt}%
\definecolor{currentstroke}{rgb}{0.121569,0.466667,0.705882}%
\pgfsetstrokecolor{currentstroke}%
\pgfsetstrokeopacity{0.710310}%
\pgfsetdash{}{0pt}%
\pgfpathmoveto{\pgfqpoint{2.182912in}{1.705126in}}%
\pgfpathcurveto{\pgfqpoint{2.191148in}{1.705126in}}{\pgfqpoint{2.199048in}{1.708398in}}{\pgfqpoint{2.204872in}{1.714222in}}%
\pgfpathcurveto{\pgfqpoint{2.210696in}{1.720046in}}{\pgfqpoint{2.213968in}{1.727946in}}{\pgfqpoint{2.213968in}{1.736183in}}%
\pgfpathcurveto{\pgfqpoint{2.213968in}{1.744419in}}{\pgfqpoint{2.210696in}{1.752319in}}{\pgfqpoint{2.204872in}{1.758143in}}%
\pgfpathcurveto{\pgfqpoint{2.199048in}{1.763967in}}{\pgfqpoint{2.191148in}{1.767239in}}{\pgfqpoint{2.182912in}{1.767239in}}%
\pgfpathcurveto{\pgfqpoint{2.174676in}{1.767239in}}{\pgfqpoint{2.166776in}{1.763967in}}{\pgfqpoint{2.160952in}{1.758143in}}%
\pgfpathcurveto{\pgfqpoint{2.155128in}{1.752319in}}{\pgfqpoint{2.151855in}{1.744419in}}{\pgfqpoint{2.151855in}{1.736183in}}%
\pgfpathcurveto{\pgfqpoint{2.151855in}{1.727946in}}{\pgfqpoint{2.155128in}{1.720046in}}{\pgfqpoint{2.160952in}{1.714222in}}%
\pgfpathcurveto{\pgfqpoint{2.166776in}{1.708398in}}{\pgfqpoint{2.174676in}{1.705126in}}{\pgfqpoint{2.182912in}{1.705126in}}%
\pgfpathclose%
\pgfusepath{stroke,fill}%
\end{pgfscope}%
\begin{pgfscope}%
\pgfpathrectangle{\pgfqpoint{0.100000in}{0.212622in}}{\pgfqpoint{3.696000in}{3.696000in}}%
\pgfusepath{clip}%
\pgfsetbuttcap%
\pgfsetroundjoin%
\definecolor{currentfill}{rgb}{0.121569,0.466667,0.705882}%
\pgfsetfillcolor{currentfill}%
\pgfsetfillopacity{0.710777}%
\pgfsetlinewidth{1.003750pt}%
\definecolor{currentstroke}{rgb}{0.121569,0.466667,0.705882}%
\pgfsetstrokecolor{currentstroke}%
\pgfsetstrokeopacity{0.710777}%
\pgfsetdash{}{0pt}%
\pgfpathmoveto{\pgfqpoint{2.183241in}{1.704065in}}%
\pgfpathcurveto{\pgfqpoint{2.191477in}{1.704065in}}{\pgfqpoint{2.199377in}{1.707337in}}{\pgfqpoint{2.205201in}{1.713161in}}%
\pgfpathcurveto{\pgfqpoint{2.211025in}{1.718985in}}{\pgfqpoint{2.214297in}{1.726885in}}{\pgfqpoint{2.214297in}{1.735121in}}%
\pgfpathcurveto{\pgfqpoint{2.214297in}{1.743357in}}{\pgfqpoint{2.211025in}{1.751258in}}{\pgfqpoint{2.205201in}{1.757081in}}%
\pgfpathcurveto{\pgfqpoint{2.199377in}{1.762905in}}{\pgfqpoint{2.191477in}{1.766178in}}{\pgfqpoint{2.183241in}{1.766178in}}%
\pgfpathcurveto{\pgfqpoint{2.175004in}{1.766178in}}{\pgfqpoint{2.167104in}{1.762905in}}{\pgfqpoint{2.161280in}{1.757081in}}%
\pgfpathcurveto{\pgfqpoint{2.155457in}{1.751258in}}{\pgfqpoint{2.152184in}{1.743357in}}{\pgfqpoint{2.152184in}{1.735121in}}%
\pgfpathcurveto{\pgfqpoint{2.152184in}{1.726885in}}{\pgfqpoint{2.155457in}{1.718985in}}{\pgfqpoint{2.161280in}{1.713161in}}%
\pgfpathcurveto{\pgfqpoint{2.167104in}{1.707337in}}{\pgfqpoint{2.175004in}{1.704065in}}{\pgfqpoint{2.183241in}{1.704065in}}%
\pgfpathclose%
\pgfusepath{stroke,fill}%
\end{pgfscope}%
\begin{pgfscope}%
\pgfpathrectangle{\pgfqpoint{0.100000in}{0.212622in}}{\pgfqpoint{3.696000in}{3.696000in}}%
\pgfusepath{clip}%
\pgfsetbuttcap%
\pgfsetroundjoin%
\definecolor{currentfill}{rgb}{0.121569,0.466667,0.705882}%
\pgfsetfillcolor{currentfill}%
\pgfsetfillopacity{0.711462}%
\pgfsetlinewidth{1.003750pt}%
\definecolor{currentstroke}{rgb}{0.121569,0.466667,0.705882}%
\pgfsetstrokecolor{currentstroke}%
\pgfsetstrokeopacity{0.711462}%
\pgfsetdash{}{0pt}%
\pgfpathmoveto{\pgfqpoint{2.183830in}{1.702908in}}%
\pgfpathcurveto{\pgfqpoint{2.192066in}{1.702908in}}{\pgfqpoint{2.199966in}{1.706180in}}{\pgfqpoint{2.205790in}{1.712004in}}%
\pgfpathcurveto{\pgfqpoint{2.211614in}{1.717828in}}{\pgfqpoint{2.214886in}{1.725728in}}{\pgfqpoint{2.214886in}{1.733964in}}%
\pgfpathcurveto{\pgfqpoint{2.214886in}{1.742201in}}{\pgfqpoint{2.211614in}{1.750101in}}{\pgfqpoint{2.205790in}{1.755925in}}%
\pgfpathcurveto{\pgfqpoint{2.199966in}{1.761749in}}{\pgfqpoint{2.192066in}{1.765021in}}{\pgfqpoint{2.183830in}{1.765021in}}%
\pgfpathcurveto{\pgfqpoint{2.175594in}{1.765021in}}{\pgfqpoint{2.167694in}{1.761749in}}{\pgfqpoint{2.161870in}{1.755925in}}%
\pgfpathcurveto{\pgfqpoint{2.156046in}{1.750101in}}{\pgfqpoint{2.152773in}{1.742201in}}{\pgfqpoint{2.152773in}{1.733964in}}%
\pgfpathcurveto{\pgfqpoint{2.152773in}{1.725728in}}{\pgfqpoint{2.156046in}{1.717828in}}{\pgfqpoint{2.161870in}{1.712004in}}%
\pgfpathcurveto{\pgfqpoint{2.167694in}{1.706180in}}{\pgfqpoint{2.175594in}{1.702908in}}{\pgfqpoint{2.183830in}{1.702908in}}%
\pgfpathclose%
\pgfusepath{stroke,fill}%
\end{pgfscope}%
\begin{pgfscope}%
\pgfpathrectangle{\pgfqpoint{0.100000in}{0.212622in}}{\pgfqpoint{3.696000in}{3.696000in}}%
\pgfusepath{clip}%
\pgfsetbuttcap%
\pgfsetroundjoin%
\definecolor{currentfill}{rgb}{0.121569,0.466667,0.705882}%
\pgfsetfillcolor{currentfill}%
\pgfsetfillopacity{0.711926}%
\pgfsetlinewidth{1.003750pt}%
\definecolor{currentstroke}{rgb}{0.121569,0.466667,0.705882}%
\pgfsetstrokecolor{currentstroke}%
\pgfsetstrokeopacity{0.711926}%
\pgfsetdash{}{0pt}%
\pgfpathmoveto{\pgfqpoint{2.184033in}{1.702768in}}%
\pgfpathcurveto{\pgfqpoint{2.192270in}{1.702768in}}{\pgfqpoint{2.200170in}{1.706040in}}{\pgfqpoint{2.205994in}{1.711864in}}%
\pgfpathcurveto{\pgfqpoint{2.211818in}{1.717688in}}{\pgfqpoint{2.215090in}{1.725588in}}{\pgfqpoint{2.215090in}{1.733824in}}%
\pgfpathcurveto{\pgfqpoint{2.215090in}{1.742060in}}{\pgfqpoint{2.211818in}{1.749960in}}{\pgfqpoint{2.205994in}{1.755784in}}%
\pgfpathcurveto{\pgfqpoint{2.200170in}{1.761608in}}{\pgfqpoint{2.192270in}{1.764881in}}{\pgfqpoint{2.184033in}{1.764881in}}%
\pgfpathcurveto{\pgfqpoint{2.175797in}{1.764881in}}{\pgfqpoint{2.167897in}{1.761608in}}{\pgfqpoint{2.162073in}{1.755784in}}%
\pgfpathcurveto{\pgfqpoint{2.156249in}{1.749960in}}{\pgfqpoint{2.152977in}{1.742060in}}{\pgfqpoint{2.152977in}{1.733824in}}%
\pgfpathcurveto{\pgfqpoint{2.152977in}{1.725588in}}{\pgfqpoint{2.156249in}{1.717688in}}{\pgfqpoint{2.162073in}{1.711864in}}%
\pgfpathcurveto{\pgfqpoint{2.167897in}{1.706040in}}{\pgfqpoint{2.175797in}{1.702768in}}{\pgfqpoint{2.184033in}{1.702768in}}%
\pgfpathclose%
\pgfusepath{stroke,fill}%
\end{pgfscope}%
\begin{pgfscope}%
\pgfpathrectangle{\pgfqpoint{0.100000in}{0.212622in}}{\pgfqpoint{3.696000in}{3.696000in}}%
\pgfusepath{clip}%
\pgfsetbuttcap%
\pgfsetroundjoin%
\definecolor{currentfill}{rgb}{0.121569,0.466667,0.705882}%
\pgfsetfillcolor{currentfill}%
\pgfsetfillopacity{0.712640}%
\pgfsetlinewidth{1.003750pt}%
\definecolor{currentstroke}{rgb}{0.121569,0.466667,0.705882}%
\pgfsetstrokecolor{currentstroke}%
\pgfsetstrokeopacity{0.712640}%
\pgfsetdash{}{0pt}%
\pgfpathmoveto{\pgfqpoint{2.184686in}{1.701169in}}%
\pgfpathcurveto{\pgfqpoint{2.192922in}{1.701169in}}{\pgfqpoint{2.200822in}{1.704441in}}{\pgfqpoint{2.206646in}{1.710265in}}%
\pgfpathcurveto{\pgfqpoint{2.212470in}{1.716089in}}{\pgfqpoint{2.215743in}{1.723989in}}{\pgfqpoint{2.215743in}{1.732225in}}%
\pgfpathcurveto{\pgfqpoint{2.215743in}{1.740461in}}{\pgfqpoint{2.212470in}{1.748361in}}{\pgfqpoint{2.206646in}{1.754185in}}%
\pgfpathcurveto{\pgfqpoint{2.200822in}{1.760009in}}{\pgfqpoint{2.192922in}{1.763282in}}{\pgfqpoint{2.184686in}{1.763282in}}%
\pgfpathcurveto{\pgfqpoint{2.176450in}{1.763282in}}{\pgfqpoint{2.168550in}{1.760009in}}{\pgfqpoint{2.162726in}{1.754185in}}%
\pgfpathcurveto{\pgfqpoint{2.156902in}{1.748361in}}{\pgfqpoint{2.153630in}{1.740461in}}{\pgfqpoint{2.153630in}{1.732225in}}%
\pgfpathcurveto{\pgfqpoint{2.153630in}{1.723989in}}{\pgfqpoint{2.156902in}{1.716089in}}{\pgfqpoint{2.162726in}{1.710265in}}%
\pgfpathcurveto{\pgfqpoint{2.168550in}{1.704441in}}{\pgfqpoint{2.176450in}{1.701169in}}{\pgfqpoint{2.184686in}{1.701169in}}%
\pgfpathclose%
\pgfusepath{stroke,fill}%
\end{pgfscope}%
\begin{pgfscope}%
\pgfpathrectangle{\pgfqpoint{0.100000in}{0.212622in}}{\pgfqpoint{3.696000in}{3.696000in}}%
\pgfusepath{clip}%
\pgfsetbuttcap%
\pgfsetroundjoin%
\definecolor{currentfill}{rgb}{0.121569,0.466667,0.705882}%
\pgfsetfillcolor{currentfill}%
\pgfsetfillopacity{0.713010}%
\pgfsetlinewidth{1.003750pt}%
\definecolor{currentstroke}{rgb}{0.121569,0.466667,0.705882}%
\pgfsetstrokecolor{currentstroke}%
\pgfsetstrokeopacity{0.713010}%
\pgfsetdash{}{0pt}%
\pgfpathmoveto{\pgfqpoint{2.185121in}{1.700199in}}%
\pgfpathcurveto{\pgfqpoint{2.193357in}{1.700199in}}{\pgfqpoint{2.201257in}{1.703471in}}{\pgfqpoint{2.207081in}{1.709295in}}%
\pgfpathcurveto{\pgfqpoint{2.212905in}{1.715119in}}{\pgfqpoint{2.216177in}{1.723019in}}{\pgfqpoint{2.216177in}{1.731255in}}%
\pgfpathcurveto{\pgfqpoint{2.216177in}{1.739492in}}{\pgfqpoint{2.212905in}{1.747392in}}{\pgfqpoint{2.207081in}{1.753216in}}%
\pgfpathcurveto{\pgfqpoint{2.201257in}{1.759040in}}{\pgfqpoint{2.193357in}{1.762312in}}{\pgfqpoint{2.185121in}{1.762312in}}%
\pgfpathcurveto{\pgfqpoint{2.176884in}{1.762312in}}{\pgfqpoint{2.168984in}{1.759040in}}{\pgfqpoint{2.163161in}{1.753216in}}%
\pgfpathcurveto{\pgfqpoint{2.157337in}{1.747392in}}{\pgfqpoint{2.154064in}{1.739492in}}{\pgfqpoint{2.154064in}{1.731255in}}%
\pgfpathcurveto{\pgfqpoint{2.154064in}{1.723019in}}{\pgfqpoint{2.157337in}{1.715119in}}{\pgfqpoint{2.163161in}{1.709295in}}%
\pgfpathcurveto{\pgfqpoint{2.168984in}{1.703471in}}{\pgfqpoint{2.176884in}{1.700199in}}{\pgfqpoint{2.185121in}{1.700199in}}%
\pgfpathclose%
\pgfusepath{stroke,fill}%
\end{pgfscope}%
\begin{pgfscope}%
\pgfpathrectangle{\pgfqpoint{0.100000in}{0.212622in}}{\pgfqpoint{3.696000in}{3.696000in}}%
\pgfusepath{clip}%
\pgfsetbuttcap%
\pgfsetroundjoin%
\definecolor{currentfill}{rgb}{0.121569,0.466667,0.705882}%
\pgfsetfillcolor{currentfill}%
\pgfsetfillopacity{0.714022}%
\pgfsetlinewidth{1.003750pt}%
\definecolor{currentstroke}{rgb}{0.121569,0.466667,0.705882}%
\pgfsetstrokecolor{currentstroke}%
\pgfsetstrokeopacity{0.714022}%
\pgfsetdash{}{0pt}%
\pgfpathmoveto{\pgfqpoint{2.185731in}{1.699251in}}%
\pgfpathcurveto{\pgfqpoint{2.193968in}{1.699251in}}{\pgfqpoint{2.201868in}{1.702523in}}{\pgfqpoint{2.207692in}{1.708347in}}%
\pgfpathcurveto{\pgfqpoint{2.213516in}{1.714171in}}{\pgfqpoint{2.216788in}{1.722071in}}{\pgfqpoint{2.216788in}{1.730307in}}%
\pgfpathcurveto{\pgfqpoint{2.216788in}{1.738543in}}{\pgfqpoint{2.213516in}{1.746444in}}{\pgfqpoint{2.207692in}{1.752267in}}%
\pgfpathcurveto{\pgfqpoint{2.201868in}{1.758091in}}{\pgfqpoint{2.193968in}{1.761364in}}{\pgfqpoint{2.185731in}{1.761364in}}%
\pgfpathcurveto{\pgfqpoint{2.177495in}{1.761364in}}{\pgfqpoint{2.169595in}{1.758091in}}{\pgfqpoint{2.163771in}{1.752267in}}%
\pgfpathcurveto{\pgfqpoint{2.157947in}{1.746444in}}{\pgfqpoint{2.154675in}{1.738543in}}{\pgfqpoint{2.154675in}{1.730307in}}%
\pgfpathcurveto{\pgfqpoint{2.154675in}{1.722071in}}{\pgfqpoint{2.157947in}{1.714171in}}{\pgfqpoint{2.163771in}{1.708347in}}%
\pgfpathcurveto{\pgfqpoint{2.169595in}{1.702523in}}{\pgfqpoint{2.177495in}{1.699251in}}{\pgfqpoint{2.185731in}{1.699251in}}%
\pgfpathclose%
\pgfusepath{stroke,fill}%
\end{pgfscope}%
\begin{pgfscope}%
\pgfpathrectangle{\pgfqpoint{0.100000in}{0.212622in}}{\pgfqpoint{3.696000in}{3.696000in}}%
\pgfusepath{clip}%
\pgfsetbuttcap%
\pgfsetroundjoin%
\definecolor{currentfill}{rgb}{0.121569,0.466667,0.705882}%
\pgfsetfillcolor{currentfill}%
\pgfsetfillopacity{0.715340}%
\pgfsetlinewidth{1.003750pt}%
\definecolor{currentstroke}{rgb}{0.121569,0.466667,0.705882}%
\pgfsetstrokecolor{currentstroke}%
\pgfsetstrokeopacity{0.715340}%
\pgfsetdash{}{0pt}%
\pgfpathmoveto{\pgfqpoint{2.186478in}{1.698988in}}%
\pgfpathcurveto{\pgfqpoint{2.194714in}{1.698988in}}{\pgfqpoint{2.202614in}{1.702261in}}{\pgfqpoint{2.208438in}{1.708085in}}%
\pgfpathcurveto{\pgfqpoint{2.214262in}{1.713909in}}{\pgfqpoint{2.217535in}{1.721809in}}{\pgfqpoint{2.217535in}{1.730045in}}%
\pgfpathcurveto{\pgfqpoint{2.217535in}{1.738281in}}{\pgfqpoint{2.214262in}{1.746181in}}{\pgfqpoint{2.208438in}{1.752005in}}%
\pgfpathcurveto{\pgfqpoint{2.202614in}{1.757829in}}{\pgfqpoint{2.194714in}{1.761101in}}{\pgfqpoint{2.186478in}{1.761101in}}%
\pgfpathcurveto{\pgfqpoint{2.178242in}{1.761101in}}{\pgfqpoint{2.170342in}{1.757829in}}{\pgfqpoint{2.164518in}{1.752005in}}%
\pgfpathcurveto{\pgfqpoint{2.158694in}{1.746181in}}{\pgfqpoint{2.155422in}{1.738281in}}{\pgfqpoint{2.155422in}{1.730045in}}%
\pgfpathcurveto{\pgfqpoint{2.155422in}{1.721809in}}{\pgfqpoint{2.158694in}{1.713909in}}{\pgfqpoint{2.164518in}{1.708085in}}%
\pgfpathcurveto{\pgfqpoint{2.170342in}{1.702261in}}{\pgfqpoint{2.178242in}{1.698988in}}{\pgfqpoint{2.186478in}{1.698988in}}%
\pgfpathclose%
\pgfusepath{stroke,fill}%
\end{pgfscope}%
\begin{pgfscope}%
\pgfpathrectangle{\pgfqpoint{0.100000in}{0.212622in}}{\pgfqpoint{3.696000in}{3.696000in}}%
\pgfusepath{clip}%
\pgfsetbuttcap%
\pgfsetroundjoin%
\definecolor{currentfill}{rgb}{0.121569,0.466667,0.705882}%
\pgfsetfillcolor{currentfill}%
\pgfsetfillopacity{0.717189}%
\pgfsetlinewidth{1.003750pt}%
\definecolor{currentstroke}{rgb}{0.121569,0.466667,0.705882}%
\pgfsetstrokecolor{currentstroke}%
\pgfsetstrokeopacity{0.717189}%
\pgfsetdash{}{0pt}%
\pgfpathmoveto{\pgfqpoint{2.188410in}{1.696014in}}%
\pgfpathcurveto{\pgfqpoint{2.196647in}{1.696014in}}{\pgfqpoint{2.204547in}{1.699287in}}{\pgfqpoint{2.210371in}{1.705111in}}%
\pgfpathcurveto{\pgfqpoint{2.216195in}{1.710935in}}{\pgfqpoint{2.219467in}{1.718835in}}{\pgfqpoint{2.219467in}{1.727071in}}%
\pgfpathcurveto{\pgfqpoint{2.219467in}{1.735307in}}{\pgfqpoint{2.216195in}{1.743207in}}{\pgfqpoint{2.210371in}{1.749031in}}%
\pgfpathcurveto{\pgfqpoint{2.204547in}{1.754855in}}{\pgfqpoint{2.196647in}{1.758127in}}{\pgfqpoint{2.188410in}{1.758127in}}%
\pgfpathcurveto{\pgfqpoint{2.180174in}{1.758127in}}{\pgfqpoint{2.172274in}{1.754855in}}{\pgfqpoint{2.166450in}{1.749031in}}%
\pgfpathcurveto{\pgfqpoint{2.160626in}{1.743207in}}{\pgfqpoint{2.157354in}{1.735307in}}{\pgfqpoint{2.157354in}{1.727071in}}%
\pgfpathcurveto{\pgfqpoint{2.157354in}{1.718835in}}{\pgfqpoint{2.160626in}{1.710935in}}{\pgfqpoint{2.166450in}{1.705111in}}%
\pgfpathcurveto{\pgfqpoint{2.172274in}{1.699287in}}{\pgfqpoint{2.180174in}{1.696014in}}{\pgfqpoint{2.188410in}{1.696014in}}%
\pgfpathclose%
\pgfusepath{stroke,fill}%
\end{pgfscope}%
\begin{pgfscope}%
\pgfpathrectangle{\pgfqpoint{0.100000in}{0.212622in}}{\pgfqpoint{3.696000in}{3.696000in}}%
\pgfusepath{clip}%
\pgfsetbuttcap%
\pgfsetroundjoin%
\definecolor{currentfill}{rgb}{0.121569,0.466667,0.705882}%
\pgfsetfillcolor{currentfill}%
\pgfsetfillopacity{0.718210}%
\pgfsetlinewidth{1.003750pt}%
\definecolor{currentstroke}{rgb}{0.121569,0.466667,0.705882}%
\pgfsetstrokecolor{currentstroke}%
\pgfsetstrokeopacity{0.718210}%
\pgfsetdash{}{0pt}%
\pgfpathmoveto{\pgfqpoint{2.189103in}{1.694176in}}%
\pgfpathcurveto{\pgfqpoint{2.197340in}{1.694176in}}{\pgfqpoint{2.205240in}{1.697449in}}{\pgfqpoint{2.211063in}{1.703273in}}%
\pgfpathcurveto{\pgfqpoint{2.216887in}{1.709096in}}{\pgfqpoint{2.220160in}{1.716997in}}{\pgfqpoint{2.220160in}{1.725233in}}%
\pgfpathcurveto{\pgfqpoint{2.220160in}{1.733469in}}{\pgfqpoint{2.216887in}{1.741369in}}{\pgfqpoint{2.211063in}{1.747193in}}%
\pgfpathcurveto{\pgfqpoint{2.205240in}{1.753017in}}{\pgfqpoint{2.197340in}{1.756289in}}{\pgfqpoint{2.189103in}{1.756289in}}%
\pgfpathcurveto{\pgfqpoint{2.180867in}{1.756289in}}{\pgfqpoint{2.172967in}{1.753017in}}{\pgfqpoint{2.167143in}{1.747193in}}%
\pgfpathcurveto{\pgfqpoint{2.161319in}{1.741369in}}{\pgfqpoint{2.158047in}{1.733469in}}{\pgfqpoint{2.158047in}{1.725233in}}%
\pgfpathcurveto{\pgfqpoint{2.158047in}{1.716997in}}{\pgfqpoint{2.161319in}{1.709096in}}{\pgfqpoint{2.167143in}{1.703273in}}%
\pgfpathcurveto{\pgfqpoint{2.172967in}{1.697449in}}{\pgfqpoint{2.180867in}{1.694176in}}{\pgfqpoint{2.189103in}{1.694176in}}%
\pgfpathclose%
\pgfusepath{stroke,fill}%
\end{pgfscope}%
\begin{pgfscope}%
\pgfpathrectangle{\pgfqpoint{0.100000in}{0.212622in}}{\pgfqpoint{3.696000in}{3.696000in}}%
\pgfusepath{clip}%
\pgfsetbuttcap%
\pgfsetroundjoin%
\definecolor{currentfill}{rgb}{0.121569,0.466667,0.705882}%
\pgfsetfillcolor{currentfill}%
\pgfsetfillopacity{0.720426}%
\pgfsetlinewidth{1.003750pt}%
\definecolor{currentstroke}{rgb}{0.121569,0.466667,0.705882}%
\pgfsetstrokecolor{currentstroke}%
\pgfsetstrokeopacity{0.720426}%
\pgfsetdash{}{0pt}%
\pgfpathmoveto{\pgfqpoint{2.190291in}{1.693728in}}%
\pgfpathcurveto{\pgfqpoint{2.198528in}{1.693728in}}{\pgfqpoint{2.206428in}{1.697000in}}{\pgfqpoint{2.212252in}{1.702824in}}%
\pgfpathcurveto{\pgfqpoint{2.218076in}{1.708648in}}{\pgfqpoint{2.221348in}{1.716548in}}{\pgfqpoint{2.221348in}{1.724785in}}%
\pgfpathcurveto{\pgfqpoint{2.221348in}{1.733021in}}{\pgfqpoint{2.218076in}{1.740921in}}{\pgfqpoint{2.212252in}{1.746745in}}%
\pgfpathcurveto{\pgfqpoint{2.206428in}{1.752569in}}{\pgfqpoint{2.198528in}{1.755841in}}{\pgfqpoint{2.190291in}{1.755841in}}%
\pgfpathcurveto{\pgfqpoint{2.182055in}{1.755841in}}{\pgfqpoint{2.174155in}{1.752569in}}{\pgfqpoint{2.168331in}{1.746745in}}%
\pgfpathcurveto{\pgfqpoint{2.162507in}{1.740921in}}{\pgfqpoint{2.159235in}{1.733021in}}{\pgfqpoint{2.159235in}{1.724785in}}%
\pgfpathcurveto{\pgfqpoint{2.159235in}{1.716548in}}{\pgfqpoint{2.162507in}{1.708648in}}{\pgfqpoint{2.168331in}{1.702824in}}%
\pgfpathcurveto{\pgfqpoint{2.174155in}{1.697000in}}{\pgfqpoint{2.182055in}{1.693728in}}{\pgfqpoint{2.190291in}{1.693728in}}%
\pgfpathclose%
\pgfusepath{stroke,fill}%
\end{pgfscope}%
\begin{pgfscope}%
\pgfpathrectangle{\pgfqpoint{0.100000in}{0.212622in}}{\pgfqpoint{3.696000in}{3.696000in}}%
\pgfusepath{clip}%
\pgfsetbuttcap%
\pgfsetroundjoin%
\definecolor{currentfill}{rgb}{0.121569,0.466667,0.705882}%
\pgfsetfillcolor{currentfill}%
\pgfsetfillopacity{0.722916}%
\pgfsetlinewidth{1.003750pt}%
\definecolor{currentstroke}{rgb}{0.121569,0.466667,0.705882}%
\pgfsetstrokecolor{currentstroke}%
\pgfsetstrokeopacity{0.722916}%
\pgfsetdash{}{0pt}%
\pgfpathmoveto{\pgfqpoint{2.192096in}{1.692286in}}%
\pgfpathcurveto{\pgfqpoint{2.200332in}{1.692286in}}{\pgfqpoint{2.208232in}{1.695558in}}{\pgfqpoint{2.214056in}{1.701382in}}%
\pgfpathcurveto{\pgfqpoint{2.219880in}{1.707206in}}{\pgfqpoint{2.223152in}{1.715106in}}{\pgfqpoint{2.223152in}{1.723343in}}%
\pgfpathcurveto{\pgfqpoint{2.223152in}{1.731579in}}{\pgfqpoint{2.219880in}{1.739479in}}{\pgfqpoint{2.214056in}{1.745303in}}%
\pgfpathcurveto{\pgfqpoint{2.208232in}{1.751127in}}{\pgfqpoint{2.200332in}{1.754399in}}{\pgfqpoint{2.192096in}{1.754399in}}%
\pgfpathcurveto{\pgfqpoint{2.183860in}{1.754399in}}{\pgfqpoint{2.175959in}{1.751127in}}{\pgfqpoint{2.170136in}{1.745303in}}%
\pgfpathcurveto{\pgfqpoint{2.164312in}{1.739479in}}{\pgfqpoint{2.161039in}{1.731579in}}{\pgfqpoint{2.161039in}{1.723343in}}%
\pgfpathcurveto{\pgfqpoint{2.161039in}{1.715106in}}{\pgfqpoint{2.164312in}{1.707206in}}{\pgfqpoint{2.170136in}{1.701382in}}%
\pgfpathcurveto{\pgfqpoint{2.175959in}{1.695558in}}{\pgfqpoint{2.183860in}{1.692286in}}{\pgfqpoint{2.192096in}{1.692286in}}%
\pgfpathclose%
\pgfusepath{stroke,fill}%
\end{pgfscope}%
\begin{pgfscope}%
\pgfpathrectangle{\pgfqpoint{0.100000in}{0.212622in}}{\pgfqpoint{3.696000in}{3.696000in}}%
\pgfusepath{clip}%
\pgfsetbuttcap%
\pgfsetroundjoin%
\definecolor{currentfill}{rgb}{0.121569,0.466667,0.705882}%
\pgfsetfillcolor{currentfill}%
\pgfsetfillopacity{0.725357}%
\pgfsetlinewidth{1.003750pt}%
\definecolor{currentstroke}{rgb}{0.121569,0.466667,0.705882}%
\pgfsetstrokecolor{currentstroke}%
\pgfsetstrokeopacity{0.725357}%
\pgfsetdash{}{0pt}%
\pgfpathmoveto{\pgfqpoint{2.194646in}{1.686944in}}%
\pgfpathcurveto{\pgfqpoint{2.202882in}{1.686944in}}{\pgfqpoint{2.210782in}{1.690217in}}{\pgfqpoint{2.216606in}{1.696040in}}%
\pgfpathcurveto{\pgfqpoint{2.222430in}{1.701864in}}{\pgfqpoint{2.225702in}{1.709764in}}{\pgfqpoint{2.225702in}{1.718001in}}%
\pgfpathcurveto{\pgfqpoint{2.225702in}{1.726237in}}{\pgfqpoint{2.222430in}{1.734137in}}{\pgfqpoint{2.216606in}{1.739961in}}%
\pgfpathcurveto{\pgfqpoint{2.210782in}{1.745785in}}{\pgfqpoint{2.202882in}{1.749057in}}{\pgfqpoint{2.194646in}{1.749057in}}%
\pgfpathcurveto{\pgfqpoint{2.186410in}{1.749057in}}{\pgfqpoint{2.178510in}{1.745785in}}{\pgfqpoint{2.172686in}{1.739961in}}%
\pgfpathcurveto{\pgfqpoint{2.166862in}{1.734137in}}{\pgfqpoint{2.163589in}{1.726237in}}{\pgfqpoint{2.163589in}{1.718001in}}%
\pgfpathcurveto{\pgfqpoint{2.163589in}{1.709764in}}{\pgfqpoint{2.166862in}{1.701864in}}{\pgfqpoint{2.172686in}{1.696040in}}%
\pgfpathcurveto{\pgfqpoint{2.178510in}{1.690217in}}{\pgfqpoint{2.186410in}{1.686944in}}{\pgfqpoint{2.194646in}{1.686944in}}%
\pgfpathclose%
\pgfusepath{stroke,fill}%
\end{pgfscope}%
\begin{pgfscope}%
\pgfpathrectangle{\pgfqpoint{0.100000in}{0.212622in}}{\pgfqpoint{3.696000in}{3.696000in}}%
\pgfusepath{clip}%
\pgfsetbuttcap%
\pgfsetroundjoin%
\definecolor{currentfill}{rgb}{0.121569,0.466667,0.705882}%
\pgfsetfillcolor{currentfill}%
\pgfsetfillopacity{0.727993}%
\pgfsetlinewidth{1.003750pt}%
\definecolor{currentstroke}{rgb}{0.121569,0.466667,0.705882}%
\pgfsetstrokecolor{currentstroke}%
\pgfsetstrokeopacity{0.727993}%
\pgfsetdash{}{0pt}%
\pgfpathmoveto{\pgfqpoint{2.196628in}{1.681333in}}%
\pgfpathcurveto{\pgfqpoint{2.204864in}{1.681333in}}{\pgfqpoint{2.212764in}{1.684606in}}{\pgfqpoint{2.218588in}{1.690429in}}%
\pgfpathcurveto{\pgfqpoint{2.224412in}{1.696253in}}{\pgfqpoint{2.227684in}{1.704153in}}{\pgfqpoint{2.227684in}{1.712390in}}%
\pgfpathcurveto{\pgfqpoint{2.227684in}{1.720626in}}{\pgfqpoint{2.224412in}{1.728526in}}{\pgfqpoint{2.218588in}{1.734350in}}%
\pgfpathcurveto{\pgfqpoint{2.212764in}{1.740174in}}{\pgfqpoint{2.204864in}{1.743446in}}{\pgfqpoint{2.196628in}{1.743446in}}%
\pgfpathcurveto{\pgfqpoint{2.188392in}{1.743446in}}{\pgfqpoint{2.180492in}{1.740174in}}{\pgfqpoint{2.174668in}{1.734350in}}%
\pgfpathcurveto{\pgfqpoint{2.168844in}{1.728526in}}{\pgfqpoint{2.165571in}{1.720626in}}{\pgfqpoint{2.165571in}{1.712390in}}%
\pgfpathcurveto{\pgfqpoint{2.165571in}{1.704153in}}{\pgfqpoint{2.168844in}{1.696253in}}{\pgfqpoint{2.174668in}{1.690429in}}%
\pgfpathcurveto{\pgfqpoint{2.180492in}{1.684606in}}{\pgfqpoint{2.188392in}{1.681333in}}{\pgfqpoint{2.196628in}{1.681333in}}%
\pgfpathclose%
\pgfusepath{stroke,fill}%
\end{pgfscope}%
\begin{pgfscope}%
\pgfpathrectangle{\pgfqpoint{0.100000in}{0.212622in}}{\pgfqpoint{3.696000in}{3.696000in}}%
\pgfusepath{clip}%
\pgfsetbuttcap%
\pgfsetroundjoin%
\definecolor{currentfill}{rgb}{0.121569,0.466667,0.705882}%
\pgfsetfillcolor{currentfill}%
\pgfsetfillopacity{0.729912}%
\pgfsetlinewidth{1.003750pt}%
\definecolor{currentstroke}{rgb}{0.121569,0.466667,0.705882}%
\pgfsetstrokecolor{currentstroke}%
\pgfsetstrokeopacity{0.729912}%
\pgfsetdash{}{0pt}%
\pgfpathmoveto{\pgfqpoint{2.197704in}{1.681196in}}%
\pgfpathcurveto{\pgfqpoint{2.205940in}{1.681196in}}{\pgfqpoint{2.213840in}{1.684468in}}{\pgfqpoint{2.219664in}{1.690292in}}%
\pgfpathcurveto{\pgfqpoint{2.225488in}{1.696116in}}{\pgfqpoint{2.228760in}{1.704016in}}{\pgfqpoint{2.228760in}{1.712252in}}%
\pgfpathcurveto{\pgfqpoint{2.228760in}{1.720489in}}{\pgfqpoint{2.225488in}{1.728389in}}{\pgfqpoint{2.219664in}{1.734212in}}%
\pgfpathcurveto{\pgfqpoint{2.213840in}{1.740036in}}{\pgfqpoint{2.205940in}{1.743309in}}{\pgfqpoint{2.197704in}{1.743309in}}%
\pgfpathcurveto{\pgfqpoint{2.189468in}{1.743309in}}{\pgfqpoint{2.181567in}{1.740036in}}{\pgfqpoint{2.175744in}{1.734212in}}%
\pgfpathcurveto{\pgfqpoint{2.169920in}{1.728389in}}{\pgfqpoint{2.166647in}{1.720489in}}{\pgfqpoint{2.166647in}{1.712252in}}%
\pgfpathcurveto{\pgfqpoint{2.166647in}{1.704016in}}{\pgfqpoint{2.169920in}{1.696116in}}{\pgfqpoint{2.175744in}{1.690292in}}%
\pgfpathcurveto{\pgfqpoint{2.181567in}{1.684468in}}{\pgfqpoint{2.189468in}{1.681196in}}{\pgfqpoint{2.197704in}{1.681196in}}%
\pgfpathclose%
\pgfusepath{stroke,fill}%
\end{pgfscope}%
\begin{pgfscope}%
\pgfpathrectangle{\pgfqpoint{0.100000in}{0.212622in}}{\pgfqpoint{3.696000in}{3.696000in}}%
\pgfusepath{clip}%
\pgfsetbuttcap%
\pgfsetroundjoin%
\definecolor{currentfill}{rgb}{0.121569,0.466667,0.705882}%
\pgfsetfillcolor{currentfill}%
\pgfsetfillopacity{0.731887}%
\pgfsetlinewidth{1.003750pt}%
\definecolor{currentstroke}{rgb}{0.121569,0.466667,0.705882}%
\pgfsetstrokecolor{currentstroke}%
\pgfsetstrokeopacity{0.731887}%
\pgfsetdash{}{0pt}%
\pgfpathmoveto{\pgfqpoint{2.199035in}{1.679114in}}%
\pgfpathcurveto{\pgfqpoint{2.207271in}{1.679114in}}{\pgfqpoint{2.215171in}{1.682387in}}{\pgfqpoint{2.220995in}{1.688211in}}%
\pgfpathcurveto{\pgfqpoint{2.226819in}{1.694035in}}{\pgfqpoint{2.230091in}{1.701935in}}{\pgfqpoint{2.230091in}{1.710171in}}%
\pgfpathcurveto{\pgfqpoint{2.230091in}{1.718407in}}{\pgfqpoint{2.226819in}{1.726307in}}{\pgfqpoint{2.220995in}{1.732131in}}%
\pgfpathcurveto{\pgfqpoint{2.215171in}{1.737955in}}{\pgfqpoint{2.207271in}{1.741227in}}{\pgfqpoint{2.199035in}{1.741227in}}%
\pgfpathcurveto{\pgfqpoint{2.190799in}{1.741227in}}{\pgfqpoint{2.182898in}{1.737955in}}{\pgfqpoint{2.177075in}{1.732131in}}%
\pgfpathcurveto{\pgfqpoint{2.171251in}{1.726307in}}{\pgfqpoint{2.167978in}{1.718407in}}{\pgfqpoint{2.167978in}{1.710171in}}%
\pgfpathcurveto{\pgfqpoint{2.167978in}{1.701935in}}{\pgfqpoint{2.171251in}{1.694035in}}{\pgfqpoint{2.177075in}{1.688211in}}%
\pgfpathcurveto{\pgfqpoint{2.182898in}{1.682387in}}{\pgfqpoint{2.190799in}{1.679114in}}{\pgfqpoint{2.199035in}{1.679114in}}%
\pgfpathclose%
\pgfusepath{stroke,fill}%
\end{pgfscope}%
\begin{pgfscope}%
\pgfpathrectangle{\pgfqpoint{0.100000in}{0.212622in}}{\pgfqpoint{3.696000in}{3.696000in}}%
\pgfusepath{clip}%
\pgfsetbuttcap%
\pgfsetroundjoin%
\definecolor{currentfill}{rgb}{0.121569,0.466667,0.705882}%
\pgfsetfillcolor{currentfill}%
\pgfsetfillopacity{0.732821}%
\pgfsetlinewidth{1.003750pt}%
\definecolor{currentstroke}{rgb}{0.121569,0.466667,0.705882}%
\pgfsetstrokecolor{currentstroke}%
\pgfsetstrokeopacity{0.732821}%
\pgfsetdash{}{0pt}%
\pgfpathmoveto{\pgfqpoint{2.199756in}{1.676995in}}%
\pgfpathcurveto{\pgfqpoint{2.207992in}{1.676995in}}{\pgfqpoint{2.215892in}{1.680268in}}{\pgfqpoint{2.221716in}{1.686092in}}%
\pgfpathcurveto{\pgfqpoint{2.227540in}{1.691916in}}{\pgfqpoint{2.230812in}{1.699816in}}{\pgfqpoint{2.230812in}{1.708052in}}%
\pgfpathcurveto{\pgfqpoint{2.230812in}{1.716288in}}{\pgfqpoint{2.227540in}{1.724188in}}{\pgfqpoint{2.221716in}{1.730012in}}%
\pgfpathcurveto{\pgfqpoint{2.215892in}{1.735836in}}{\pgfqpoint{2.207992in}{1.739108in}}{\pgfqpoint{2.199756in}{1.739108in}}%
\pgfpathcurveto{\pgfqpoint{2.191520in}{1.739108in}}{\pgfqpoint{2.183620in}{1.735836in}}{\pgfqpoint{2.177796in}{1.730012in}}%
\pgfpathcurveto{\pgfqpoint{2.171972in}{1.724188in}}{\pgfqpoint{2.168699in}{1.716288in}}{\pgfqpoint{2.168699in}{1.708052in}}%
\pgfpathcurveto{\pgfqpoint{2.168699in}{1.699816in}}{\pgfqpoint{2.171972in}{1.691916in}}{\pgfqpoint{2.177796in}{1.686092in}}%
\pgfpathcurveto{\pgfqpoint{2.183620in}{1.680268in}}{\pgfqpoint{2.191520in}{1.676995in}}{\pgfqpoint{2.199756in}{1.676995in}}%
\pgfpathclose%
\pgfusepath{stroke,fill}%
\end{pgfscope}%
\begin{pgfscope}%
\pgfpathrectangle{\pgfqpoint{0.100000in}{0.212622in}}{\pgfqpoint{3.696000in}{3.696000in}}%
\pgfusepath{clip}%
\pgfsetbuttcap%
\pgfsetroundjoin%
\definecolor{currentfill}{rgb}{0.121569,0.466667,0.705882}%
\pgfsetfillcolor{currentfill}%
\pgfsetfillopacity{0.734182}%
\pgfsetlinewidth{1.003750pt}%
\definecolor{currentstroke}{rgb}{0.121569,0.466667,0.705882}%
\pgfsetstrokecolor{currentstroke}%
\pgfsetstrokeopacity{0.734182}%
\pgfsetdash{}{0pt}%
\pgfpathmoveto{\pgfqpoint{2.200619in}{1.675226in}}%
\pgfpathcurveto{\pgfqpoint{2.208856in}{1.675226in}}{\pgfqpoint{2.216756in}{1.678498in}}{\pgfqpoint{2.222580in}{1.684322in}}%
\pgfpathcurveto{\pgfqpoint{2.228404in}{1.690146in}}{\pgfqpoint{2.231676in}{1.698046in}}{\pgfqpoint{2.231676in}{1.706282in}}%
\pgfpathcurveto{\pgfqpoint{2.231676in}{1.714519in}}{\pgfqpoint{2.228404in}{1.722419in}}{\pgfqpoint{2.222580in}{1.728243in}}%
\pgfpathcurveto{\pgfqpoint{2.216756in}{1.734067in}}{\pgfqpoint{2.208856in}{1.737339in}}{\pgfqpoint{2.200619in}{1.737339in}}%
\pgfpathcurveto{\pgfqpoint{2.192383in}{1.737339in}}{\pgfqpoint{2.184483in}{1.734067in}}{\pgfqpoint{2.178659in}{1.728243in}}%
\pgfpathcurveto{\pgfqpoint{2.172835in}{1.722419in}}{\pgfqpoint{2.169563in}{1.714519in}}{\pgfqpoint{2.169563in}{1.706282in}}%
\pgfpathcurveto{\pgfqpoint{2.169563in}{1.698046in}}{\pgfqpoint{2.172835in}{1.690146in}}{\pgfqpoint{2.178659in}{1.684322in}}%
\pgfpathcurveto{\pgfqpoint{2.184483in}{1.678498in}}{\pgfqpoint{2.192383in}{1.675226in}}{\pgfqpoint{2.200619in}{1.675226in}}%
\pgfpathclose%
\pgfusepath{stroke,fill}%
\end{pgfscope}%
\begin{pgfscope}%
\pgfpathrectangle{\pgfqpoint{0.100000in}{0.212622in}}{\pgfqpoint{3.696000in}{3.696000in}}%
\pgfusepath{clip}%
\pgfsetbuttcap%
\pgfsetroundjoin%
\definecolor{currentfill}{rgb}{0.121569,0.466667,0.705882}%
\pgfsetfillcolor{currentfill}%
\pgfsetfillopacity{0.735077}%
\pgfsetlinewidth{1.003750pt}%
\definecolor{currentstroke}{rgb}{0.121569,0.466667,0.705882}%
\pgfsetstrokecolor{currentstroke}%
\pgfsetstrokeopacity{0.735077}%
\pgfsetdash{}{0pt}%
\pgfpathmoveto{\pgfqpoint{2.201292in}{1.675291in}}%
\pgfpathcurveto{\pgfqpoint{2.209528in}{1.675291in}}{\pgfqpoint{2.217428in}{1.678563in}}{\pgfqpoint{2.223252in}{1.684387in}}%
\pgfpathcurveto{\pgfqpoint{2.229076in}{1.690211in}}{\pgfqpoint{2.232348in}{1.698111in}}{\pgfqpoint{2.232348in}{1.706347in}}%
\pgfpathcurveto{\pgfqpoint{2.232348in}{1.714583in}}{\pgfqpoint{2.229076in}{1.722483in}}{\pgfqpoint{2.223252in}{1.728307in}}%
\pgfpathcurveto{\pgfqpoint{2.217428in}{1.734131in}}{\pgfqpoint{2.209528in}{1.737404in}}{\pgfqpoint{2.201292in}{1.737404in}}%
\pgfpathcurveto{\pgfqpoint{2.193056in}{1.737404in}}{\pgfqpoint{2.185156in}{1.734131in}}{\pgfqpoint{2.179332in}{1.728307in}}%
\pgfpathcurveto{\pgfqpoint{2.173508in}{1.722483in}}{\pgfqpoint{2.170235in}{1.714583in}}{\pgfqpoint{2.170235in}{1.706347in}}%
\pgfpathcurveto{\pgfqpoint{2.170235in}{1.698111in}}{\pgfqpoint{2.173508in}{1.690211in}}{\pgfqpoint{2.179332in}{1.684387in}}%
\pgfpathcurveto{\pgfqpoint{2.185156in}{1.678563in}}{\pgfqpoint{2.193056in}{1.675291in}}{\pgfqpoint{2.201292in}{1.675291in}}%
\pgfpathclose%
\pgfusepath{stroke,fill}%
\end{pgfscope}%
\begin{pgfscope}%
\pgfpathrectangle{\pgfqpoint{0.100000in}{0.212622in}}{\pgfqpoint{3.696000in}{3.696000in}}%
\pgfusepath{clip}%
\pgfsetbuttcap%
\pgfsetroundjoin%
\definecolor{currentfill}{rgb}{0.121569,0.466667,0.705882}%
\pgfsetfillcolor{currentfill}%
\pgfsetfillopacity{0.736201}%
\pgfsetlinewidth{1.003750pt}%
\definecolor{currentstroke}{rgb}{0.121569,0.466667,0.705882}%
\pgfsetstrokecolor{currentstroke}%
\pgfsetstrokeopacity{0.736201}%
\pgfsetdash{}{0pt}%
\pgfpathmoveto{\pgfqpoint{2.202153in}{1.673253in}}%
\pgfpathcurveto{\pgfqpoint{2.210389in}{1.673253in}}{\pgfqpoint{2.218289in}{1.676525in}}{\pgfqpoint{2.224113in}{1.682349in}}%
\pgfpathcurveto{\pgfqpoint{2.229937in}{1.688173in}}{\pgfqpoint{2.233209in}{1.696073in}}{\pgfqpoint{2.233209in}{1.704309in}}%
\pgfpathcurveto{\pgfqpoint{2.233209in}{1.712545in}}{\pgfqpoint{2.229937in}{1.720445in}}{\pgfqpoint{2.224113in}{1.726269in}}%
\pgfpathcurveto{\pgfqpoint{2.218289in}{1.732093in}}{\pgfqpoint{2.210389in}{1.735366in}}{\pgfqpoint{2.202153in}{1.735366in}}%
\pgfpathcurveto{\pgfqpoint{2.193917in}{1.735366in}}{\pgfqpoint{2.186016in}{1.732093in}}{\pgfqpoint{2.180193in}{1.726269in}}%
\pgfpathcurveto{\pgfqpoint{2.174369in}{1.720445in}}{\pgfqpoint{2.171096in}{1.712545in}}{\pgfqpoint{2.171096in}{1.704309in}}%
\pgfpathcurveto{\pgfqpoint{2.171096in}{1.696073in}}{\pgfqpoint{2.174369in}{1.688173in}}{\pgfqpoint{2.180193in}{1.682349in}}%
\pgfpathcurveto{\pgfqpoint{2.186016in}{1.676525in}}{\pgfqpoint{2.193917in}{1.673253in}}{\pgfqpoint{2.202153in}{1.673253in}}%
\pgfpathclose%
\pgfusepath{stroke,fill}%
\end{pgfscope}%
\begin{pgfscope}%
\pgfpathrectangle{\pgfqpoint{0.100000in}{0.212622in}}{\pgfqpoint{3.696000in}{3.696000in}}%
\pgfusepath{clip}%
\pgfsetbuttcap%
\pgfsetroundjoin%
\definecolor{currentfill}{rgb}{0.121569,0.466667,0.705882}%
\pgfsetfillcolor{currentfill}%
\pgfsetfillopacity{0.737461}%
\pgfsetlinewidth{1.003750pt}%
\definecolor{currentstroke}{rgb}{0.121569,0.466667,0.705882}%
\pgfsetstrokecolor{currentstroke}%
\pgfsetstrokeopacity{0.737461}%
\pgfsetdash{}{0pt}%
\pgfpathmoveto{\pgfqpoint{2.203339in}{1.670682in}}%
\pgfpathcurveto{\pgfqpoint{2.211576in}{1.670682in}}{\pgfqpoint{2.219476in}{1.673955in}}{\pgfqpoint{2.225300in}{1.679779in}}%
\pgfpathcurveto{\pgfqpoint{2.231124in}{1.685603in}}{\pgfqpoint{2.234396in}{1.693503in}}{\pgfqpoint{2.234396in}{1.701739in}}%
\pgfpathcurveto{\pgfqpoint{2.234396in}{1.709975in}}{\pgfqpoint{2.231124in}{1.717875in}}{\pgfqpoint{2.225300in}{1.723699in}}%
\pgfpathcurveto{\pgfqpoint{2.219476in}{1.729523in}}{\pgfqpoint{2.211576in}{1.732795in}}{\pgfqpoint{2.203339in}{1.732795in}}%
\pgfpathcurveto{\pgfqpoint{2.195103in}{1.732795in}}{\pgfqpoint{2.187203in}{1.729523in}}{\pgfqpoint{2.181379in}{1.723699in}}%
\pgfpathcurveto{\pgfqpoint{2.175555in}{1.717875in}}{\pgfqpoint{2.172283in}{1.709975in}}{\pgfqpoint{2.172283in}{1.701739in}}%
\pgfpathcurveto{\pgfqpoint{2.172283in}{1.693503in}}{\pgfqpoint{2.175555in}{1.685603in}}{\pgfqpoint{2.181379in}{1.679779in}}%
\pgfpathcurveto{\pgfqpoint{2.187203in}{1.673955in}}{\pgfqpoint{2.195103in}{1.670682in}}{\pgfqpoint{2.203339in}{1.670682in}}%
\pgfpathclose%
\pgfusepath{stroke,fill}%
\end{pgfscope}%
\begin{pgfscope}%
\pgfpathrectangle{\pgfqpoint{0.100000in}{0.212622in}}{\pgfqpoint{3.696000in}{3.696000in}}%
\pgfusepath{clip}%
\pgfsetbuttcap%
\pgfsetroundjoin%
\definecolor{currentfill}{rgb}{0.121569,0.466667,0.705882}%
\pgfsetfillcolor{currentfill}%
\pgfsetfillopacity{0.738940}%
\pgfsetlinewidth{1.003750pt}%
\definecolor{currentstroke}{rgb}{0.121569,0.466667,0.705882}%
\pgfsetstrokecolor{currentstroke}%
\pgfsetstrokeopacity{0.738940}%
\pgfsetdash{}{0pt}%
\pgfpathmoveto{\pgfqpoint{2.204901in}{1.666168in}}%
\pgfpathcurveto{\pgfqpoint{2.213138in}{1.666168in}}{\pgfqpoint{2.221038in}{1.669440in}}{\pgfqpoint{2.226862in}{1.675264in}}%
\pgfpathcurveto{\pgfqpoint{2.232686in}{1.681088in}}{\pgfqpoint{2.235958in}{1.688988in}}{\pgfqpoint{2.235958in}{1.697224in}}%
\pgfpathcurveto{\pgfqpoint{2.235958in}{1.705460in}}{\pgfqpoint{2.232686in}{1.713360in}}{\pgfqpoint{2.226862in}{1.719184in}}%
\pgfpathcurveto{\pgfqpoint{2.221038in}{1.725008in}}{\pgfqpoint{2.213138in}{1.728281in}}{\pgfqpoint{2.204901in}{1.728281in}}%
\pgfpathcurveto{\pgfqpoint{2.196665in}{1.728281in}}{\pgfqpoint{2.188765in}{1.725008in}}{\pgfqpoint{2.182941in}{1.719184in}}%
\pgfpathcurveto{\pgfqpoint{2.177117in}{1.713360in}}{\pgfqpoint{2.173845in}{1.705460in}}{\pgfqpoint{2.173845in}{1.697224in}}%
\pgfpathcurveto{\pgfqpoint{2.173845in}{1.688988in}}{\pgfqpoint{2.177117in}{1.681088in}}{\pgfqpoint{2.182941in}{1.675264in}}%
\pgfpathcurveto{\pgfqpoint{2.188765in}{1.669440in}}{\pgfqpoint{2.196665in}{1.666168in}}{\pgfqpoint{2.204901in}{1.666168in}}%
\pgfpathclose%
\pgfusepath{stroke,fill}%
\end{pgfscope}%
\begin{pgfscope}%
\pgfpathrectangle{\pgfqpoint{0.100000in}{0.212622in}}{\pgfqpoint{3.696000in}{3.696000in}}%
\pgfusepath{clip}%
\pgfsetbuttcap%
\pgfsetroundjoin%
\definecolor{currentfill}{rgb}{0.121569,0.466667,0.705882}%
\pgfsetfillcolor{currentfill}%
\pgfsetfillopacity{0.740128}%
\pgfsetlinewidth{1.003750pt}%
\definecolor{currentstroke}{rgb}{0.121569,0.466667,0.705882}%
\pgfsetstrokecolor{currentstroke}%
\pgfsetstrokeopacity{0.740128}%
\pgfsetdash{}{0pt}%
\pgfpathmoveto{\pgfqpoint{2.205769in}{1.666048in}}%
\pgfpathcurveto{\pgfqpoint{2.214005in}{1.666048in}}{\pgfqpoint{2.221905in}{1.669320in}}{\pgfqpoint{2.227729in}{1.675144in}}%
\pgfpathcurveto{\pgfqpoint{2.233553in}{1.680968in}}{\pgfqpoint{2.236825in}{1.688868in}}{\pgfqpoint{2.236825in}{1.697104in}}%
\pgfpathcurveto{\pgfqpoint{2.236825in}{1.705341in}}{\pgfqpoint{2.233553in}{1.713241in}}{\pgfqpoint{2.227729in}{1.719065in}}%
\pgfpathcurveto{\pgfqpoint{2.221905in}{1.724888in}}{\pgfqpoint{2.214005in}{1.728161in}}{\pgfqpoint{2.205769in}{1.728161in}}%
\pgfpathcurveto{\pgfqpoint{2.197533in}{1.728161in}}{\pgfqpoint{2.189632in}{1.724888in}}{\pgfqpoint{2.183809in}{1.719065in}}%
\pgfpathcurveto{\pgfqpoint{2.177985in}{1.713241in}}{\pgfqpoint{2.174712in}{1.705341in}}{\pgfqpoint{2.174712in}{1.697104in}}%
\pgfpathcurveto{\pgfqpoint{2.174712in}{1.688868in}}{\pgfqpoint{2.177985in}{1.680968in}}{\pgfqpoint{2.183809in}{1.675144in}}%
\pgfpathcurveto{\pgfqpoint{2.189632in}{1.669320in}}{\pgfqpoint{2.197533in}{1.666048in}}{\pgfqpoint{2.205769in}{1.666048in}}%
\pgfpathclose%
\pgfusepath{stroke,fill}%
\end{pgfscope}%
\begin{pgfscope}%
\pgfpathrectangle{\pgfqpoint{0.100000in}{0.212622in}}{\pgfqpoint{3.696000in}{3.696000in}}%
\pgfusepath{clip}%
\pgfsetbuttcap%
\pgfsetroundjoin%
\definecolor{currentfill}{rgb}{0.121569,0.466667,0.705882}%
\pgfsetfillcolor{currentfill}%
\pgfsetfillopacity{0.741453}%
\pgfsetlinewidth{1.003750pt}%
\definecolor{currentstroke}{rgb}{0.121569,0.466667,0.705882}%
\pgfsetstrokecolor{currentstroke}%
\pgfsetstrokeopacity{0.741453}%
\pgfsetdash{}{0pt}%
\pgfpathmoveto{\pgfqpoint{2.207221in}{1.665100in}}%
\pgfpathcurveto{\pgfqpoint{2.215458in}{1.665100in}}{\pgfqpoint{2.223358in}{1.668372in}}{\pgfqpoint{2.229182in}{1.674196in}}%
\pgfpathcurveto{\pgfqpoint{2.235006in}{1.680020in}}{\pgfqpoint{2.238278in}{1.687920in}}{\pgfqpoint{2.238278in}{1.696156in}}%
\pgfpathcurveto{\pgfqpoint{2.238278in}{1.704392in}}{\pgfqpoint{2.235006in}{1.712292in}}{\pgfqpoint{2.229182in}{1.718116in}}%
\pgfpathcurveto{\pgfqpoint{2.223358in}{1.723940in}}{\pgfqpoint{2.215458in}{1.727213in}}{\pgfqpoint{2.207221in}{1.727213in}}%
\pgfpathcurveto{\pgfqpoint{2.198985in}{1.727213in}}{\pgfqpoint{2.191085in}{1.723940in}}{\pgfqpoint{2.185261in}{1.718116in}}%
\pgfpathcurveto{\pgfqpoint{2.179437in}{1.712292in}}{\pgfqpoint{2.176165in}{1.704392in}}{\pgfqpoint{2.176165in}{1.696156in}}%
\pgfpathcurveto{\pgfqpoint{2.176165in}{1.687920in}}{\pgfqpoint{2.179437in}{1.680020in}}{\pgfqpoint{2.185261in}{1.674196in}}%
\pgfpathcurveto{\pgfqpoint{2.191085in}{1.668372in}}{\pgfqpoint{2.198985in}{1.665100in}}{\pgfqpoint{2.207221in}{1.665100in}}%
\pgfpathclose%
\pgfusepath{stroke,fill}%
\end{pgfscope}%
\begin{pgfscope}%
\pgfpathrectangle{\pgfqpoint{0.100000in}{0.212622in}}{\pgfqpoint{3.696000in}{3.696000in}}%
\pgfusepath{clip}%
\pgfsetbuttcap%
\pgfsetroundjoin%
\definecolor{currentfill}{rgb}{0.121569,0.466667,0.705882}%
\pgfsetfillcolor{currentfill}%
\pgfsetfillopacity{0.742181}%
\pgfsetlinewidth{1.003750pt}%
\definecolor{currentstroke}{rgb}{0.121569,0.466667,0.705882}%
\pgfsetstrokecolor{currentstroke}%
\pgfsetstrokeopacity{0.742181}%
\pgfsetdash{}{0pt}%
\pgfpathmoveto{\pgfqpoint{2.207720in}{1.664370in}}%
\pgfpathcurveto{\pgfqpoint{2.215957in}{1.664370in}}{\pgfqpoint{2.223857in}{1.667642in}}{\pgfqpoint{2.229681in}{1.673466in}}%
\pgfpathcurveto{\pgfqpoint{2.235504in}{1.679290in}}{\pgfqpoint{2.238777in}{1.687190in}}{\pgfqpoint{2.238777in}{1.695427in}}%
\pgfpathcurveto{\pgfqpoint{2.238777in}{1.703663in}}{\pgfqpoint{2.235504in}{1.711563in}}{\pgfqpoint{2.229681in}{1.717387in}}%
\pgfpathcurveto{\pgfqpoint{2.223857in}{1.723211in}}{\pgfqpoint{2.215957in}{1.726483in}}{\pgfqpoint{2.207720in}{1.726483in}}%
\pgfpathcurveto{\pgfqpoint{2.199484in}{1.726483in}}{\pgfqpoint{2.191584in}{1.723211in}}{\pgfqpoint{2.185760in}{1.717387in}}%
\pgfpathcurveto{\pgfqpoint{2.179936in}{1.711563in}}{\pgfqpoint{2.176664in}{1.703663in}}{\pgfqpoint{2.176664in}{1.695427in}}%
\pgfpathcurveto{\pgfqpoint{2.176664in}{1.687190in}}{\pgfqpoint{2.179936in}{1.679290in}}{\pgfqpoint{2.185760in}{1.673466in}}%
\pgfpathcurveto{\pgfqpoint{2.191584in}{1.667642in}}{\pgfqpoint{2.199484in}{1.664370in}}{\pgfqpoint{2.207720in}{1.664370in}}%
\pgfpathclose%
\pgfusepath{stroke,fill}%
\end{pgfscope}%
\begin{pgfscope}%
\pgfpathrectangle{\pgfqpoint{0.100000in}{0.212622in}}{\pgfqpoint{3.696000in}{3.696000in}}%
\pgfusepath{clip}%
\pgfsetbuttcap%
\pgfsetroundjoin%
\definecolor{currentfill}{rgb}{0.121569,0.466667,0.705882}%
\pgfsetfillcolor{currentfill}%
\pgfsetfillopacity{0.742922}%
\pgfsetlinewidth{1.003750pt}%
\definecolor{currentstroke}{rgb}{0.121569,0.466667,0.705882}%
\pgfsetstrokecolor{currentstroke}%
\pgfsetstrokeopacity{0.742922}%
\pgfsetdash{}{0pt}%
\pgfpathmoveto{\pgfqpoint{2.208473in}{1.662751in}}%
\pgfpathcurveto{\pgfqpoint{2.216709in}{1.662751in}}{\pgfqpoint{2.224610in}{1.666023in}}{\pgfqpoint{2.230433in}{1.671847in}}%
\pgfpathcurveto{\pgfqpoint{2.236257in}{1.677671in}}{\pgfqpoint{2.239530in}{1.685571in}}{\pgfqpoint{2.239530in}{1.693807in}}%
\pgfpathcurveto{\pgfqpoint{2.239530in}{1.702044in}}{\pgfqpoint{2.236257in}{1.709944in}}{\pgfqpoint{2.230433in}{1.715768in}}%
\pgfpathcurveto{\pgfqpoint{2.224610in}{1.721592in}}{\pgfqpoint{2.216709in}{1.724864in}}{\pgfqpoint{2.208473in}{1.724864in}}%
\pgfpathcurveto{\pgfqpoint{2.200237in}{1.724864in}}{\pgfqpoint{2.192337in}{1.721592in}}{\pgfqpoint{2.186513in}{1.715768in}}%
\pgfpathcurveto{\pgfqpoint{2.180689in}{1.709944in}}{\pgfqpoint{2.177417in}{1.702044in}}{\pgfqpoint{2.177417in}{1.693807in}}%
\pgfpathcurveto{\pgfqpoint{2.177417in}{1.685571in}}{\pgfqpoint{2.180689in}{1.677671in}}{\pgfqpoint{2.186513in}{1.671847in}}%
\pgfpathcurveto{\pgfqpoint{2.192337in}{1.666023in}}{\pgfqpoint{2.200237in}{1.662751in}}{\pgfqpoint{2.208473in}{1.662751in}}%
\pgfpathclose%
\pgfusepath{stroke,fill}%
\end{pgfscope}%
\begin{pgfscope}%
\pgfpathrectangle{\pgfqpoint{0.100000in}{0.212622in}}{\pgfqpoint{3.696000in}{3.696000in}}%
\pgfusepath{clip}%
\pgfsetbuttcap%
\pgfsetroundjoin%
\definecolor{currentfill}{rgb}{0.121569,0.466667,0.705882}%
\pgfsetfillcolor{currentfill}%
\pgfsetfillopacity{0.743457}%
\pgfsetlinewidth{1.003750pt}%
\definecolor{currentstroke}{rgb}{0.121569,0.466667,0.705882}%
\pgfsetstrokecolor{currentstroke}%
\pgfsetstrokeopacity{0.743457}%
\pgfsetdash{}{0pt}%
\pgfpathmoveto{\pgfqpoint{2.208849in}{1.662646in}}%
\pgfpathcurveto{\pgfqpoint{2.217085in}{1.662646in}}{\pgfqpoint{2.224986in}{1.665918in}}{\pgfqpoint{2.230809in}{1.671742in}}%
\pgfpathcurveto{\pgfqpoint{2.236633in}{1.677566in}}{\pgfqpoint{2.239906in}{1.685466in}}{\pgfqpoint{2.239906in}{1.693702in}}%
\pgfpathcurveto{\pgfqpoint{2.239906in}{1.701938in}}{\pgfqpoint{2.236633in}{1.709838in}}{\pgfqpoint{2.230809in}{1.715662in}}%
\pgfpathcurveto{\pgfqpoint{2.224986in}{1.721486in}}{\pgfqpoint{2.217085in}{1.724759in}}{\pgfqpoint{2.208849in}{1.724759in}}%
\pgfpathcurveto{\pgfqpoint{2.200613in}{1.724759in}}{\pgfqpoint{2.192713in}{1.721486in}}{\pgfqpoint{2.186889in}{1.715662in}}%
\pgfpathcurveto{\pgfqpoint{2.181065in}{1.709838in}}{\pgfqpoint{2.177793in}{1.701938in}}{\pgfqpoint{2.177793in}{1.693702in}}%
\pgfpathcurveto{\pgfqpoint{2.177793in}{1.685466in}}{\pgfqpoint{2.181065in}{1.677566in}}{\pgfqpoint{2.186889in}{1.671742in}}%
\pgfpathcurveto{\pgfqpoint{2.192713in}{1.665918in}}{\pgfqpoint{2.200613in}{1.662646in}}{\pgfqpoint{2.208849in}{1.662646in}}%
\pgfpathclose%
\pgfusepath{stroke,fill}%
\end{pgfscope}%
\begin{pgfscope}%
\pgfpathrectangle{\pgfqpoint{0.100000in}{0.212622in}}{\pgfqpoint{3.696000in}{3.696000in}}%
\pgfusepath{clip}%
\pgfsetbuttcap%
\pgfsetroundjoin%
\definecolor{currentfill}{rgb}{0.121569,0.466667,0.705882}%
\pgfsetfillcolor{currentfill}%
\pgfsetfillopacity{0.744097}%
\pgfsetlinewidth{1.003750pt}%
\definecolor{currentstroke}{rgb}{0.121569,0.466667,0.705882}%
\pgfsetstrokecolor{currentstroke}%
\pgfsetstrokeopacity{0.744097}%
\pgfsetdash{}{0pt}%
\pgfpathmoveto{\pgfqpoint{2.209550in}{1.661358in}}%
\pgfpathcurveto{\pgfqpoint{2.217787in}{1.661358in}}{\pgfqpoint{2.225687in}{1.664630in}}{\pgfqpoint{2.231511in}{1.670454in}}%
\pgfpathcurveto{\pgfqpoint{2.237334in}{1.676278in}}{\pgfqpoint{2.240607in}{1.684178in}}{\pgfqpoint{2.240607in}{1.692414in}}%
\pgfpathcurveto{\pgfqpoint{2.240607in}{1.700651in}}{\pgfqpoint{2.237334in}{1.708551in}}{\pgfqpoint{2.231511in}{1.714375in}}%
\pgfpathcurveto{\pgfqpoint{2.225687in}{1.720198in}}{\pgfqpoint{2.217787in}{1.723471in}}{\pgfqpoint{2.209550in}{1.723471in}}%
\pgfpathcurveto{\pgfqpoint{2.201314in}{1.723471in}}{\pgfqpoint{2.193414in}{1.720198in}}{\pgfqpoint{2.187590in}{1.714375in}}%
\pgfpathcurveto{\pgfqpoint{2.181766in}{1.708551in}}{\pgfqpoint{2.178494in}{1.700651in}}{\pgfqpoint{2.178494in}{1.692414in}}%
\pgfpathcurveto{\pgfqpoint{2.178494in}{1.684178in}}{\pgfqpoint{2.181766in}{1.676278in}}{\pgfqpoint{2.187590in}{1.670454in}}%
\pgfpathcurveto{\pgfqpoint{2.193414in}{1.664630in}}{\pgfqpoint{2.201314in}{1.661358in}}{\pgfqpoint{2.209550in}{1.661358in}}%
\pgfpathclose%
\pgfusepath{stroke,fill}%
\end{pgfscope}%
\begin{pgfscope}%
\pgfpathrectangle{\pgfqpoint{0.100000in}{0.212622in}}{\pgfqpoint{3.696000in}{3.696000in}}%
\pgfusepath{clip}%
\pgfsetbuttcap%
\pgfsetroundjoin%
\definecolor{currentfill}{rgb}{0.121569,0.466667,0.705882}%
\pgfsetfillcolor{currentfill}%
\pgfsetfillopacity{0.744487}%
\pgfsetlinewidth{1.003750pt}%
\definecolor{currentstroke}{rgb}{0.121569,0.466667,0.705882}%
\pgfsetstrokecolor{currentstroke}%
\pgfsetstrokeopacity{0.744487}%
\pgfsetdash{}{0pt}%
\pgfpathmoveto{\pgfqpoint{2.209902in}{1.660863in}}%
\pgfpathcurveto{\pgfqpoint{2.218138in}{1.660863in}}{\pgfqpoint{2.226038in}{1.664136in}}{\pgfqpoint{2.231862in}{1.669960in}}%
\pgfpathcurveto{\pgfqpoint{2.237686in}{1.675783in}}{\pgfqpoint{2.240958in}{1.683684in}}{\pgfqpoint{2.240958in}{1.691920in}}%
\pgfpathcurveto{\pgfqpoint{2.240958in}{1.700156in}}{\pgfqpoint{2.237686in}{1.708056in}}{\pgfqpoint{2.231862in}{1.713880in}}%
\pgfpathcurveto{\pgfqpoint{2.226038in}{1.719704in}}{\pgfqpoint{2.218138in}{1.722976in}}{\pgfqpoint{2.209902in}{1.722976in}}%
\pgfpathcurveto{\pgfqpoint{2.201666in}{1.722976in}}{\pgfqpoint{2.193766in}{1.719704in}}{\pgfqpoint{2.187942in}{1.713880in}}%
\pgfpathcurveto{\pgfqpoint{2.182118in}{1.708056in}}{\pgfqpoint{2.178845in}{1.700156in}}{\pgfqpoint{2.178845in}{1.691920in}}%
\pgfpathcurveto{\pgfqpoint{2.178845in}{1.683684in}}{\pgfqpoint{2.182118in}{1.675783in}}{\pgfqpoint{2.187942in}{1.669960in}}%
\pgfpathcurveto{\pgfqpoint{2.193766in}{1.664136in}}{\pgfqpoint{2.201666in}{1.660863in}}{\pgfqpoint{2.209902in}{1.660863in}}%
\pgfpathclose%
\pgfusepath{stroke,fill}%
\end{pgfscope}%
\begin{pgfscope}%
\pgfpathrectangle{\pgfqpoint{0.100000in}{0.212622in}}{\pgfqpoint{3.696000in}{3.696000in}}%
\pgfusepath{clip}%
\pgfsetbuttcap%
\pgfsetroundjoin%
\definecolor{currentfill}{rgb}{0.121569,0.466667,0.705882}%
\pgfsetfillcolor{currentfill}%
\pgfsetfillopacity{0.745262}%
\pgfsetlinewidth{1.003750pt}%
\definecolor{currentstroke}{rgb}{0.121569,0.466667,0.705882}%
\pgfsetstrokecolor{currentstroke}%
\pgfsetstrokeopacity{0.745262}%
\pgfsetdash{}{0pt}%
\pgfpathmoveto{\pgfqpoint{2.210377in}{1.660616in}}%
\pgfpathcurveto{\pgfqpoint{2.218614in}{1.660616in}}{\pgfqpoint{2.226514in}{1.663888in}}{\pgfqpoint{2.232338in}{1.669712in}}%
\pgfpathcurveto{\pgfqpoint{2.238162in}{1.675536in}}{\pgfqpoint{2.241434in}{1.683436in}}{\pgfqpoint{2.241434in}{1.691672in}}%
\pgfpathcurveto{\pgfqpoint{2.241434in}{1.699908in}}{\pgfqpoint{2.238162in}{1.707808in}}{\pgfqpoint{2.232338in}{1.713632in}}%
\pgfpathcurveto{\pgfqpoint{2.226514in}{1.719456in}}{\pgfqpoint{2.218614in}{1.722729in}}{\pgfqpoint{2.210377in}{1.722729in}}%
\pgfpathcurveto{\pgfqpoint{2.202141in}{1.722729in}}{\pgfqpoint{2.194241in}{1.719456in}}{\pgfqpoint{2.188417in}{1.713632in}}%
\pgfpathcurveto{\pgfqpoint{2.182593in}{1.707808in}}{\pgfqpoint{2.179321in}{1.699908in}}{\pgfqpoint{2.179321in}{1.691672in}}%
\pgfpathcurveto{\pgfqpoint{2.179321in}{1.683436in}}{\pgfqpoint{2.182593in}{1.675536in}}{\pgfqpoint{2.188417in}{1.669712in}}%
\pgfpathcurveto{\pgfqpoint{2.194241in}{1.663888in}}{\pgfqpoint{2.202141in}{1.660616in}}{\pgfqpoint{2.210377in}{1.660616in}}%
\pgfpathclose%
\pgfusepath{stroke,fill}%
\end{pgfscope}%
\begin{pgfscope}%
\pgfpathrectangle{\pgfqpoint{0.100000in}{0.212622in}}{\pgfqpoint{3.696000in}{3.696000in}}%
\pgfusepath{clip}%
\pgfsetbuttcap%
\pgfsetroundjoin%
\definecolor{currentfill}{rgb}{0.121569,0.466667,0.705882}%
\pgfsetfillcolor{currentfill}%
\pgfsetfillopacity{0.746193}%
\pgfsetlinewidth{1.003750pt}%
\definecolor{currentstroke}{rgb}{0.121569,0.466667,0.705882}%
\pgfsetstrokecolor{currentstroke}%
\pgfsetstrokeopacity{0.746193}%
\pgfsetdash{}{0pt}%
\pgfpathmoveto{\pgfqpoint{2.211206in}{1.660109in}}%
\pgfpathcurveto{\pgfqpoint{2.219442in}{1.660109in}}{\pgfqpoint{2.227342in}{1.663382in}}{\pgfqpoint{2.233166in}{1.669206in}}%
\pgfpathcurveto{\pgfqpoint{2.238990in}{1.675030in}}{\pgfqpoint{2.242262in}{1.682930in}}{\pgfqpoint{2.242262in}{1.691166in}}%
\pgfpathcurveto{\pgfqpoint{2.242262in}{1.699402in}}{\pgfqpoint{2.238990in}{1.707302in}}{\pgfqpoint{2.233166in}{1.713126in}}%
\pgfpathcurveto{\pgfqpoint{2.227342in}{1.718950in}}{\pgfqpoint{2.219442in}{1.722222in}}{\pgfqpoint{2.211206in}{1.722222in}}%
\pgfpathcurveto{\pgfqpoint{2.202970in}{1.722222in}}{\pgfqpoint{2.195070in}{1.718950in}}{\pgfqpoint{2.189246in}{1.713126in}}%
\pgfpathcurveto{\pgfqpoint{2.183422in}{1.707302in}}{\pgfqpoint{2.180149in}{1.699402in}}{\pgfqpoint{2.180149in}{1.691166in}}%
\pgfpathcurveto{\pgfqpoint{2.180149in}{1.682930in}}{\pgfqpoint{2.183422in}{1.675030in}}{\pgfqpoint{2.189246in}{1.669206in}}%
\pgfpathcurveto{\pgfqpoint{2.195070in}{1.663382in}}{\pgfqpoint{2.202970in}{1.660109in}}{\pgfqpoint{2.211206in}{1.660109in}}%
\pgfpathclose%
\pgfusepath{stroke,fill}%
\end{pgfscope}%
\begin{pgfscope}%
\pgfpathrectangle{\pgfqpoint{0.100000in}{0.212622in}}{\pgfqpoint{3.696000in}{3.696000in}}%
\pgfusepath{clip}%
\pgfsetbuttcap%
\pgfsetroundjoin%
\definecolor{currentfill}{rgb}{0.121569,0.466667,0.705882}%
\pgfsetfillcolor{currentfill}%
\pgfsetfillopacity{0.747744}%
\pgfsetlinewidth{1.003750pt}%
\definecolor{currentstroke}{rgb}{0.121569,0.466667,0.705882}%
\pgfsetstrokecolor{currentstroke}%
\pgfsetstrokeopacity{0.747744}%
\pgfsetdash{}{0pt}%
\pgfpathmoveto{\pgfqpoint{2.212465in}{1.658055in}}%
\pgfpathcurveto{\pgfqpoint{2.220701in}{1.658055in}}{\pgfqpoint{2.228601in}{1.661327in}}{\pgfqpoint{2.234425in}{1.667151in}}%
\pgfpathcurveto{\pgfqpoint{2.240249in}{1.672975in}}{\pgfqpoint{2.243521in}{1.680875in}}{\pgfqpoint{2.243521in}{1.689111in}}%
\pgfpathcurveto{\pgfqpoint{2.243521in}{1.697348in}}{\pgfqpoint{2.240249in}{1.705248in}}{\pgfqpoint{2.234425in}{1.711072in}}%
\pgfpathcurveto{\pgfqpoint{2.228601in}{1.716896in}}{\pgfqpoint{2.220701in}{1.720168in}}{\pgfqpoint{2.212465in}{1.720168in}}%
\pgfpathcurveto{\pgfqpoint{2.204229in}{1.720168in}}{\pgfqpoint{2.196328in}{1.716896in}}{\pgfqpoint{2.190505in}{1.711072in}}%
\pgfpathcurveto{\pgfqpoint{2.184681in}{1.705248in}}{\pgfqpoint{2.181408in}{1.697348in}}{\pgfqpoint{2.181408in}{1.689111in}}%
\pgfpathcurveto{\pgfqpoint{2.181408in}{1.680875in}}{\pgfqpoint{2.184681in}{1.672975in}}{\pgfqpoint{2.190505in}{1.667151in}}%
\pgfpathcurveto{\pgfqpoint{2.196328in}{1.661327in}}{\pgfqpoint{2.204229in}{1.658055in}}{\pgfqpoint{2.212465in}{1.658055in}}%
\pgfpathclose%
\pgfusepath{stroke,fill}%
\end{pgfscope}%
\begin{pgfscope}%
\pgfpathrectangle{\pgfqpoint{0.100000in}{0.212622in}}{\pgfqpoint{3.696000in}{3.696000in}}%
\pgfusepath{clip}%
\pgfsetbuttcap%
\pgfsetroundjoin%
\definecolor{currentfill}{rgb}{0.121569,0.466667,0.705882}%
\pgfsetfillcolor{currentfill}%
\pgfsetfillopacity{0.749267}%
\pgfsetlinewidth{1.003750pt}%
\definecolor{currentstroke}{rgb}{0.121569,0.466667,0.705882}%
\pgfsetstrokecolor{currentstroke}%
\pgfsetstrokeopacity{0.749267}%
\pgfsetdash{}{0pt}%
\pgfpathmoveto{\pgfqpoint{2.213988in}{1.653775in}}%
\pgfpathcurveto{\pgfqpoint{2.222225in}{1.653775in}}{\pgfqpoint{2.230125in}{1.657047in}}{\pgfqpoint{2.235949in}{1.662871in}}%
\pgfpathcurveto{\pgfqpoint{2.241773in}{1.668695in}}{\pgfqpoint{2.245045in}{1.676595in}}{\pgfqpoint{2.245045in}{1.684831in}}%
\pgfpathcurveto{\pgfqpoint{2.245045in}{1.693068in}}{\pgfqpoint{2.241773in}{1.700968in}}{\pgfqpoint{2.235949in}{1.706792in}}%
\pgfpathcurveto{\pgfqpoint{2.230125in}{1.712616in}}{\pgfqpoint{2.222225in}{1.715888in}}{\pgfqpoint{2.213988in}{1.715888in}}%
\pgfpathcurveto{\pgfqpoint{2.205752in}{1.715888in}}{\pgfqpoint{2.197852in}{1.712616in}}{\pgfqpoint{2.192028in}{1.706792in}}%
\pgfpathcurveto{\pgfqpoint{2.186204in}{1.700968in}}{\pgfqpoint{2.182932in}{1.693068in}}{\pgfqpoint{2.182932in}{1.684831in}}%
\pgfpathcurveto{\pgfqpoint{2.182932in}{1.676595in}}{\pgfqpoint{2.186204in}{1.668695in}}{\pgfqpoint{2.192028in}{1.662871in}}%
\pgfpathcurveto{\pgfqpoint{2.197852in}{1.657047in}}{\pgfqpoint{2.205752in}{1.653775in}}{\pgfqpoint{2.213988in}{1.653775in}}%
\pgfpathclose%
\pgfusepath{stroke,fill}%
\end{pgfscope}%
\begin{pgfscope}%
\pgfpathrectangle{\pgfqpoint{0.100000in}{0.212622in}}{\pgfqpoint{3.696000in}{3.696000in}}%
\pgfusepath{clip}%
\pgfsetbuttcap%
\pgfsetroundjoin%
\definecolor{currentfill}{rgb}{0.121569,0.466667,0.705882}%
\pgfsetfillcolor{currentfill}%
\pgfsetfillopacity{0.751831}%
\pgfsetlinewidth{1.003750pt}%
\definecolor{currentstroke}{rgb}{0.121569,0.466667,0.705882}%
\pgfsetstrokecolor{currentstroke}%
\pgfsetstrokeopacity{0.751831}%
\pgfsetdash{}{0pt}%
\pgfpathmoveto{\pgfqpoint{2.216340in}{1.652296in}}%
\pgfpathcurveto{\pgfqpoint{2.224577in}{1.652296in}}{\pgfqpoint{2.232477in}{1.655568in}}{\pgfqpoint{2.238301in}{1.661392in}}%
\pgfpathcurveto{\pgfqpoint{2.244124in}{1.667216in}}{\pgfqpoint{2.247397in}{1.675116in}}{\pgfqpoint{2.247397in}{1.683353in}}%
\pgfpathcurveto{\pgfqpoint{2.247397in}{1.691589in}}{\pgfqpoint{2.244124in}{1.699489in}}{\pgfqpoint{2.238301in}{1.705313in}}%
\pgfpathcurveto{\pgfqpoint{2.232477in}{1.711137in}}{\pgfqpoint{2.224577in}{1.714409in}}{\pgfqpoint{2.216340in}{1.714409in}}%
\pgfpathcurveto{\pgfqpoint{2.208104in}{1.714409in}}{\pgfqpoint{2.200204in}{1.711137in}}{\pgfqpoint{2.194380in}{1.705313in}}%
\pgfpathcurveto{\pgfqpoint{2.188556in}{1.699489in}}{\pgfqpoint{2.185284in}{1.691589in}}{\pgfqpoint{2.185284in}{1.683353in}}%
\pgfpathcurveto{\pgfqpoint{2.185284in}{1.675116in}}{\pgfqpoint{2.188556in}{1.667216in}}{\pgfqpoint{2.194380in}{1.661392in}}%
\pgfpathcurveto{\pgfqpoint{2.200204in}{1.655568in}}{\pgfqpoint{2.208104in}{1.652296in}}{\pgfqpoint{2.216340in}{1.652296in}}%
\pgfpathclose%
\pgfusepath{stroke,fill}%
\end{pgfscope}%
\begin{pgfscope}%
\pgfpathrectangle{\pgfqpoint{0.100000in}{0.212622in}}{\pgfqpoint{3.696000in}{3.696000in}}%
\pgfusepath{clip}%
\pgfsetbuttcap%
\pgfsetroundjoin%
\definecolor{currentfill}{rgb}{0.121569,0.466667,0.705882}%
\pgfsetfillcolor{currentfill}%
\pgfsetfillopacity{0.754978}%
\pgfsetlinewidth{1.003750pt}%
\definecolor{currentstroke}{rgb}{0.121569,0.466667,0.705882}%
\pgfsetstrokecolor{currentstroke}%
\pgfsetstrokeopacity{0.754978}%
\pgfsetdash{}{0pt}%
\pgfpathmoveto{\pgfqpoint{2.218597in}{1.650903in}}%
\pgfpathcurveto{\pgfqpoint{2.226833in}{1.650903in}}{\pgfqpoint{2.234733in}{1.654175in}}{\pgfqpoint{2.240557in}{1.659999in}}%
\pgfpathcurveto{\pgfqpoint{2.246381in}{1.665823in}}{\pgfqpoint{2.249653in}{1.673723in}}{\pgfqpoint{2.249653in}{1.681959in}}%
\pgfpathcurveto{\pgfqpoint{2.249653in}{1.690195in}}{\pgfqpoint{2.246381in}{1.698095in}}{\pgfqpoint{2.240557in}{1.703919in}}%
\pgfpathcurveto{\pgfqpoint{2.234733in}{1.709743in}}{\pgfqpoint{2.226833in}{1.713016in}}{\pgfqpoint{2.218597in}{1.713016in}}%
\pgfpathcurveto{\pgfqpoint{2.210360in}{1.713016in}}{\pgfqpoint{2.202460in}{1.709743in}}{\pgfqpoint{2.196636in}{1.703919in}}%
\pgfpathcurveto{\pgfqpoint{2.190812in}{1.698095in}}{\pgfqpoint{2.187540in}{1.690195in}}{\pgfqpoint{2.187540in}{1.681959in}}%
\pgfpathcurveto{\pgfqpoint{2.187540in}{1.673723in}}{\pgfqpoint{2.190812in}{1.665823in}}{\pgfqpoint{2.196636in}{1.659999in}}%
\pgfpathcurveto{\pgfqpoint{2.202460in}{1.654175in}}{\pgfqpoint{2.210360in}{1.650903in}}{\pgfqpoint{2.218597in}{1.650903in}}%
\pgfpathclose%
\pgfusepath{stroke,fill}%
\end{pgfscope}%
\begin{pgfscope}%
\pgfpathrectangle{\pgfqpoint{0.100000in}{0.212622in}}{\pgfqpoint{3.696000in}{3.696000in}}%
\pgfusepath{clip}%
\pgfsetbuttcap%
\pgfsetroundjoin%
\definecolor{currentfill}{rgb}{0.121569,0.466667,0.705882}%
\pgfsetfillcolor{currentfill}%
\pgfsetfillopacity{0.758381}%
\pgfsetlinewidth{1.003750pt}%
\definecolor{currentstroke}{rgb}{0.121569,0.466667,0.705882}%
\pgfsetstrokecolor{currentstroke}%
\pgfsetstrokeopacity{0.758381}%
\pgfsetdash{}{0pt}%
\pgfpathmoveto{\pgfqpoint{2.220438in}{1.647701in}}%
\pgfpathcurveto{\pgfqpoint{2.228675in}{1.647701in}}{\pgfqpoint{2.236575in}{1.650973in}}{\pgfqpoint{2.242399in}{1.656797in}}%
\pgfpathcurveto{\pgfqpoint{2.248222in}{1.662621in}}{\pgfqpoint{2.251495in}{1.670521in}}{\pgfqpoint{2.251495in}{1.678758in}}%
\pgfpathcurveto{\pgfqpoint{2.251495in}{1.686994in}}{\pgfqpoint{2.248222in}{1.694894in}}{\pgfqpoint{2.242399in}{1.700718in}}%
\pgfpathcurveto{\pgfqpoint{2.236575in}{1.706542in}}{\pgfqpoint{2.228675in}{1.709814in}}{\pgfqpoint{2.220438in}{1.709814in}}%
\pgfpathcurveto{\pgfqpoint{2.212202in}{1.709814in}}{\pgfqpoint{2.204302in}{1.706542in}}{\pgfqpoint{2.198478in}{1.700718in}}%
\pgfpathcurveto{\pgfqpoint{2.192654in}{1.694894in}}{\pgfqpoint{2.189382in}{1.686994in}}{\pgfqpoint{2.189382in}{1.678758in}}%
\pgfpathcurveto{\pgfqpoint{2.189382in}{1.670521in}}{\pgfqpoint{2.192654in}{1.662621in}}{\pgfqpoint{2.198478in}{1.656797in}}%
\pgfpathcurveto{\pgfqpoint{2.204302in}{1.650973in}}{\pgfqpoint{2.212202in}{1.647701in}}{\pgfqpoint{2.220438in}{1.647701in}}%
\pgfpathclose%
\pgfusepath{stroke,fill}%
\end{pgfscope}%
\begin{pgfscope}%
\pgfpathrectangle{\pgfqpoint{0.100000in}{0.212622in}}{\pgfqpoint{3.696000in}{3.696000in}}%
\pgfusepath{clip}%
\pgfsetbuttcap%
\pgfsetroundjoin%
\definecolor{currentfill}{rgb}{0.121569,0.466667,0.705882}%
\pgfsetfillcolor{currentfill}%
\pgfsetfillopacity{0.761578}%
\pgfsetlinewidth{1.003750pt}%
\definecolor{currentstroke}{rgb}{0.121569,0.466667,0.705882}%
\pgfsetstrokecolor{currentstroke}%
\pgfsetstrokeopacity{0.761578}%
\pgfsetdash{}{0pt}%
\pgfpathmoveto{\pgfqpoint{2.223342in}{1.642457in}}%
\pgfpathcurveto{\pgfqpoint{2.231578in}{1.642457in}}{\pgfqpoint{2.239478in}{1.645730in}}{\pgfqpoint{2.245302in}{1.651554in}}%
\pgfpathcurveto{\pgfqpoint{2.251126in}{1.657377in}}{\pgfqpoint{2.254398in}{1.665278in}}{\pgfqpoint{2.254398in}{1.673514in}}%
\pgfpathcurveto{\pgfqpoint{2.254398in}{1.681750in}}{\pgfqpoint{2.251126in}{1.689650in}}{\pgfqpoint{2.245302in}{1.695474in}}%
\pgfpathcurveto{\pgfqpoint{2.239478in}{1.701298in}}{\pgfqpoint{2.231578in}{1.704570in}}{\pgfqpoint{2.223342in}{1.704570in}}%
\pgfpathcurveto{\pgfqpoint{2.215105in}{1.704570in}}{\pgfqpoint{2.207205in}{1.701298in}}{\pgfqpoint{2.201381in}{1.695474in}}%
\pgfpathcurveto{\pgfqpoint{2.195557in}{1.689650in}}{\pgfqpoint{2.192285in}{1.681750in}}{\pgfqpoint{2.192285in}{1.673514in}}%
\pgfpathcurveto{\pgfqpoint{2.192285in}{1.665278in}}{\pgfqpoint{2.195557in}{1.657377in}}{\pgfqpoint{2.201381in}{1.651554in}}%
\pgfpathcurveto{\pgfqpoint{2.207205in}{1.645730in}}{\pgfqpoint{2.215105in}{1.642457in}}{\pgfqpoint{2.223342in}{1.642457in}}%
\pgfpathclose%
\pgfusepath{stroke,fill}%
\end{pgfscope}%
\begin{pgfscope}%
\pgfpathrectangle{\pgfqpoint{0.100000in}{0.212622in}}{\pgfqpoint{3.696000in}{3.696000in}}%
\pgfusepath{clip}%
\pgfsetbuttcap%
\pgfsetroundjoin%
\definecolor{currentfill}{rgb}{0.121569,0.466667,0.705882}%
\pgfsetfillcolor{currentfill}%
\pgfsetfillopacity{0.765729}%
\pgfsetlinewidth{1.003750pt}%
\definecolor{currentstroke}{rgb}{0.121569,0.466667,0.705882}%
\pgfsetstrokecolor{currentstroke}%
\pgfsetstrokeopacity{0.765729}%
\pgfsetdash{}{0pt}%
\pgfpathmoveto{\pgfqpoint{2.226135in}{1.641099in}}%
\pgfpathcurveto{\pgfqpoint{2.234371in}{1.641099in}}{\pgfqpoint{2.242271in}{1.644371in}}{\pgfqpoint{2.248095in}{1.650195in}}%
\pgfpathcurveto{\pgfqpoint{2.253919in}{1.656019in}}{\pgfqpoint{2.257191in}{1.663919in}}{\pgfqpoint{2.257191in}{1.672155in}}%
\pgfpathcurveto{\pgfqpoint{2.257191in}{1.680392in}}{\pgfqpoint{2.253919in}{1.688292in}}{\pgfqpoint{2.248095in}{1.694116in}}%
\pgfpathcurveto{\pgfqpoint{2.242271in}{1.699940in}}{\pgfqpoint{2.234371in}{1.703212in}}{\pgfqpoint{2.226135in}{1.703212in}}%
\pgfpathcurveto{\pgfqpoint{2.217898in}{1.703212in}}{\pgfqpoint{2.209998in}{1.699940in}}{\pgfqpoint{2.204174in}{1.694116in}}%
\pgfpathcurveto{\pgfqpoint{2.198351in}{1.688292in}}{\pgfqpoint{2.195078in}{1.680392in}}{\pgfqpoint{2.195078in}{1.672155in}}%
\pgfpathcurveto{\pgfqpoint{2.195078in}{1.663919in}}{\pgfqpoint{2.198351in}{1.656019in}}{\pgfqpoint{2.204174in}{1.650195in}}%
\pgfpathcurveto{\pgfqpoint{2.209998in}{1.644371in}}{\pgfqpoint{2.217898in}{1.641099in}}{\pgfqpoint{2.226135in}{1.641099in}}%
\pgfpathclose%
\pgfusepath{stroke,fill}%
\end{pgfscope}%
\begin{pgfscope}%
\pgfpathrectangle{\pgfqpoint{0.100000in}{0.212622in}}{\pgfqpoint{3.696000in}{3.696000in}}%
\pgfusepath{clip}%
\pgfsetbuttcap%
\pgfsetroundjoin%
\definecolor{currentfill}{rgb}{0.121569,0.466667,0.705882}%
\pgfsetfillcolor{currentfill}%
\pgfsetfillopacity{0.767780}%
\pgfsetlinewidth{1.003750pt}%
\definecolor{currentstroke}{rgb}{0.121569,0.466667,0.705882}%
\pgfsetstrokecolor{currentstroke}%
\pgfsetstrokeopacity{0.767780}%
\pgfsetdash{}{0pt}%
\pgfpathmoveto{\pgfqpoint{2.227981in}{1.639068in}}%
\pgfpathcurveto{\pgfqpoint{2.236217in}{1.639068in}}{\pgfqpoint{2.244117in}{1.642340in}}{\pgfqpoint{2.249941in}{1.648164in}}%
\pgfpathcurveto{\pgfqpoint{2.255765in}{1.653988in}}{\pgfqpoint{2.259038in}{1.661888in}}{\pgfqpoint{2.259038in}{1.670124in}}%
\pgfpathcurveto{\pgfqpoint{2.259038in}{1.678361in}}{\pgfqpoint{2.255765in}{1.686261in}}{\pgfqpoint{2.249941in}{1.692085in}}%
\pgfpathcurveto{\pgfqpoint{2.244117in}{1.697908in}}{\pgfqpoint{2.236217in}{1.701181in}}{\pgfqpoint{2.227981in}{1.701181in}}%
\pgfpathcurveto{\pgfqpoint{2.219745in}{1.701181in}}{\pgfqpoint{2.211845in}{1.697908in}}{\pgfqpoint{2.206021in}{1.692085in}}%
\pgfpathcurveto{\pgfqpoint{2.200197in}{1.686261in}}{\pgfqpoint{2.196925in}{1.678361in}}{\pgfqpoint{2.196925in}{1.670124in}}%
\pgfpathcurveto{\pgfqpoint{2.196925in}{1.661888in}}{\pgfqpoint{2.200197in}{1.653988in}}{\pgfqpoint{2.206021in}{1.648164in}}%
\pgfpathcurveto{\pgfqpoint{2.211845in}{1.642340in}}{\pgfqpoint{2.219745in}{1.639068in}}{\pgfqpoint{2.227981in}{1.639068in}}%
\pgfpathclose%
\pgfusepath{stroke,fill}%
\end{pgfscope}%
\begin{pgfscope}%
\pgfpathrectangle{\pgfqpoint{0.100000in}{0.212622in}}{\pgfqpoint{3.696000in}{3.696000in}}%
\pgfusepath{clip}%
\pgfsetbuttcap%
\pgfsetroundjoin%
\definecolor{currentfill}{rgb}{0.121569,0.466667,0.705882}%
\pgfsetfillcolor{currentfill}%
\pgfsetfillopacity{0.769790}%
\pgfsetlinewidth{1.003750pt}%
\definecolor{currentstroke}{rgb}{0.121569,0.466667,0.705882}%
\pgfsetstrokecolor{currentstroke}%
\pgfsetstrokeopacity{0.769790}%
\pgfsetdash{}{0pt}%
\pgfpathmoveto{\pgfqpoint{2.230187in}{1.634916in}}%
\pgfpathcurveto{\pgfqpoint{2.238423in}{1.634916in}}{\pgfqpoint{2.246323in}{1.638189in}}{\pgfqpoint{2.252147in}{1.644013in}}%
\pgfpathcurveto{\pgfqpoint{2.257971in}{1.649837in}}{\pgfqpoint{2.261243in}{1.657737in}}{\pgfqpoint{2.261243in}{1.665973in}}%
\pgfpathcurveto{\pgfqpoint{2.261243in}{1.674209in}}{\pgfqpoint{2.257971in}{1.682109in}}{\pgfqpoint{2.252147in}{1.687933in}}%
\pgfpathcurveto{\pgfqpoint{2.246323in}{1.693757in}}{\pgfqpoint{2.238423in}{1.697029in}}{\pgfqpoint{2.230187in}{1.697029in}}%
\pgfpathcurveto{\pgfqpoint{2.221951in}{1.697029in}}{\pgfqpoint{2.214051in}{1.693757in}}{\pgfqpoint{2.208227in}{1.687933in}}%
\pgfpathcurveto{\pgfqpoint{2.202403in}{1.682109in}}{\pgfqpoint{2.199130in}{1.674209in}}{\pgfqpoint{2.199130in}{1.665973in}}%
\pgfpathcurveto{\pgfqpoint{2.199130in}{1.657737in}}{\pgfqpoint{2.202403in}{1.649837in}}{\pgfqpoint{2.208227in}{1.644013in}}%
\pgfpathcurveto{\pgfqpoint{2.214051in}{1.638189in}}{\pgfqpoint{2.221951in}{1.634916in}}{\pgfqpoint{2.230187in}{1.634916in}}%
\pgfpathclose%
\pgfusepath{stroke,fill}%
\end{pgfscope}%
\begin{pgfscope}%
\pgfpathrectangle{\pgfqpoint{0.100000in}{0.212622in}}{\pgfqpoint{3.696000in}{3.696000in}}%
\pgfusepath{clip}%
\pgfsetbuttcap%
\pgfsetroundjoin%
\definecolor{currentfill}{rgb}{0.121569,0.466667,0.705882}%
\pgfsetfillcolor{currentfill}%
\pgfsetfillopacity{0.770998}%
\pgfsetlinewidth{1.003750pt}%
\definecolor{currentstroke}{rgb}{0.121569,0.466667,0.705882}%
\pgfsetstrokecolor{currentstroke}%
\pgfsetstrokeopacity{0.770998}%
\pgfsetdash{}{0pt}%
\pgfpathmoveto{\pgfqpoint{2.231231in}{1.633152in}}%
\pgfpathcurveto{\pgfqpoint{2.239467in}{1.633152in}}{\pgfqpoint{2.247367in}{1.636425in}}{\pgfqpoint{2.253191in}{1.642249in}}%
\pgfpathcurveto{\pgfqpoint{2.259015in}{1.648073in}}{\pgfqpoint{2.262287in}{1.655973in}}{\pgfqpoint{2.262287in}{1.664209in}}%
\pgfpathcurveto{\pgfqpoint{2.262287in}{1.672445in}}{\pgfqpoint{2.259015in}{1.680345in}}{\pgfqpoint{2.253191in}{1.686169in}}%
\pgfpathcurveto{\pgfqpoint{2.247367in}{1.691993in}}{\pgfqpoint{2.239467in}{1.695265in}}{\pgfqpoint{2.231231in}{1.695265in}}%
\pgfpathcurveto{\pgfqpoint{2.222995in}{1.695265in}}{\pgfqpoint{2.215095in}{1.691993in}}{\pgfqpoint{2.209271in}{1.686169in}}%
\pgfpathcurveto{\pgfqpoint{2.203447in}{1.680345in}}{\pgfqpoint{2.200174in}{1.672445in}}{\pgfqpoint{2.200174in}{1.664209in}}%
\pgfpathcurveto{\pgfqpoint{2.200174in}{1.655973in}}{\pgfqpoint{2.203447in}{1.648073in}}{\pgfqpoint{2.209271in}{1.642249in}}%
\pgfpathcurveto{\pgfqpoint{2.215095in}{1.636425in}}{\pgfqpoint{2.222995in}{1.633152in}}{\pgfqpoint{2.231231in}{1.633152in}}%
\pgfpathclose%
\pgfusepath{stroke,fill}%
\end{pgfscope}%
\begin{pgfscope}%
\pgfpathrectangle{\pgfqpoint{0.100000in}{0.212622in}}{\pgfqpoint{3.696000in}{3.696000in}}%
\pgfusepath{clip}%
\pgfsetbuttcap%
\pgfsetroundjoin%
\definecolor{currentfill}{rgb}{0.121569,0.466667,0.705882}%
\pgfsetfillcolor{currentfill}%
\pgfsetfillopacity{0.771765}%
\pgfsetlinewidth{1.003750pt}%
\definecolor{currentstroke}{rgb}{0.121569,0.466667,0.705882}%
\pgfsetstrokecolor{currentstroke}%
\pgfsetstrokeopacity{0.771765}%
\pgfsetdash{}{0pt}%
\pgfpathmoveto{\pgfqpoint{2.231759in}{1.632795in}}%
\pgfpathcurveto{\pgfqpoint{2.239995in}{1.632795in}}{\pgfqpoint{2.247896in}{1.636068in}}{\pgfqpoint{2.253719in}{1.641892in}}%
\pgfpathcurveto{\pgfqpoint{2.259543in}{1.647716in}}{\pgfqpoint{2.262816in}{1.655616in}}{\pgfqpoint{2.262816in}{1.663852in}}%
\pgfpathcurveto{\pgfqpoint{2.262816in}{1.672088in}}{\pgfqpoint{2.259543in}{1.679988in}}{\pgfqpoint{2.253719in}{1.685812in}}%
\pgfpathcurveto{\pgfqpoint{2.247896in}{1.691636in}}{\pgfqpoint{2.239995in}{1.694908in}}{\pgfqpoint{2.231759in}{1.694908in}}%
\pgfpathcurveto{\pgfqpoint{2.223523in}{1.694908in}}{\pgfqpoint{2.215623in}{1.691636in}}{\pgfqpoint{2.209799in}{1.685812in}}%
\pgfpathcurveto{\pgfqpoint{2.203975in}{1.679988in}}{\pgfqpoint{2.200703in}{1.672088in}}{\pgfqpoint{2.200703in}{1.663852in}}%
\pgfpathcurveto{\pgfqpoint{2.200703in}{1.655616in}}{\pgfqpoint{2.203975in}{1.647716in}}{\pgfqpoint{2.209799in}{1.641892in}}%
\pgfpathcurveto{\pgfqpoint{2.215623in}{1.636068in}}{\pgfqpoint{2.223523in}{1.632795in}}{\pgfqpoint{2.231759in}{1.632795in}}%
\pgfpathclose%
\pgfusepath{stroke,fill}%
\end{pgfscope}%
\begin{pgfscope}%
\pgfpathrectangle{\pgfqpoint{0.100000in}{0.212622in}}{\pgfqpoint{3.696000in}{3.696000in}}%
\pgfusepath{clip}%
\pgfsetbuttcap%
\pgfsetroundjoin%
\definecolor{currentfill}{rgb}{0.121569,0.466667,0.705882}%
\pgfsetfillcolor{currentfill}%
\pgfsetfillopacity{0.772758}%
\pgfsetlinewidth{1.003750pt}%
\definecolor{currentstroke}{rgb}{0.121569,0.466667,0.705882}%
\pgfsetstrokecolor{currentstroke}%
\pgfsetstrokeopacity{0.772758}%
\pgfsetdash{}{0pt}%
\pgfpathmoveto{\pgfqpoint{2.232792in}{1.630890in}}%
\pgfpathcurveto{\pgfqpoint{2.241028in}{1.630890in}}{\pgfqpoint{2.248929in}{1.634163in}}{\pgfqpoint{2.254752in}{1.639987in}}%
\pgfpathcurveto{\pgfqpoint{2.260576in}{1.645810in}}{\pgfqpoint{2.263849in}{1.653711in}}{\pgfqpoint{2.263849in}{1.661947in}}%
\pgfpathcurveto{\pgfqpoint{2.263849in}{1.670183in}}{\pgfqpoint{2.260576in}{1.678083in}}{\pgfqpoint{2.254752in}{1.683907in}}%
\pgfpathcurveto{\pgfqpoint{2.248929in}{1.689731in}}{\pgfqpoint{2.241028in}{1.693003in}}{\pgfqpoint{2.232792in}{1.693003in}}%
\pgfpathcurveto{\pgfqpoint{2.224556in}{1.693003in}}{\pgfqpoint{2.216656in}{1.689731in}}{\pgfqpoint{2.210832in}{1.683907in}}%
\pgfpathcurveto{\pgfqpoint{2.205008in}{1.678083in}}{\pgfqpoint{2.201736in}{1.670183in}}{\pgfqpoint{2.201736in}{1.661947in}}%
\pgfpathcurveto{\pgfqpoint{2.201736in}{1.653711in}}{\pgfqpoint{2.205008in}{1.645810in}}{\pgfqpoint{2.210832in}{1.639987in}}%
\pgfpathcurveto{\pgfqpoint{2.216656in}{1.634163in}}{\pgfqpoint{2.224556in}{1.630890in}}{\pgfqpoint{2.232792in}{1.630890in}}%
\pgfpathclose%
\pgfusepath{stroke,fill}%
\end{pgfscope}%
\begin{pgfscope}%
\pgfpathrectangle{\pgfqpoint{0.100000in}{0.212622in}}{\pgfqpoint{3.696000in}{3.696000in}}%
\pgfusepath{clip}%
\pgfsetbuttcap%
\pgfsetroundjoin%
\definecolor{currentfill}{rgb}{0.121569,0.466667,0.705882}%
\pgfsetfillcolor{currentfill}%
\pgfsetfillopacity{0.773271}%
\pgfsetlinewidth{1.003750pt}%
\definecolor{currentstroke}{rgb}{0.121569,0.466667,0.705882}%
\pgfsetstrokecolor{currentstroke}%
\pgfsetstrokeopacity{0.773271}%
\pgfsetdash{}{0pt}%
\pgfpathmoveto{\pgfqpoint{2.233364in}{1.629631in}}%
\pgfpathcurveto{\pgfqpoint{2.241600in}{1.629631in}}{\pgfqpoint{2.249500in}{1.632903in}}{\pgfqpoint{2.255324in}{1.638727in}}%
\pgfpathcurveto{\pgfqpoint{2.261148in}{1.644551in}}{\pgfqpoint{2.264421in}{1.652451in}}{\pgfqpoint{2.264421in}{1.660687in}}%
\pgfpathcurveto{\pgfqpoint{2.264421in}{1.668923in}}{\pgfqpoint{2.261148in}{1.676823in}}{\pgfqpoint{2.255324in}{1.682647in}}%
\pgfpathcurveto{\pgfqpoint{2.249500in}{1.688471in}}{\pgfqpoint{2.241600in}{1.691744in}}{\pgfqpoint{2.233364in}{1.691744in}}%
\pgfpathcurveto{\pgfqpoint{2.225128in}{1.691744in}}{\pgfqpoint{2.217228in}{1.688471in}}{\pgfqpoint{2.211404in}{1.682647in}}%
\pgfpathcurveto{\pgfqpoint{2.205580in}{1.676823in}}{\pgfqpoint{2.202308in}{1.668923in}}{\pgfqpoint{2.202308in}{1.660687in}}%
\pgfpathcurveto{\pgfqpoint{2.202308in}{1.652451in}}{\pgfqpoint{2.205580in}{1.644551in}}{\pgfqpoint{2.211404in}{1.638727in}}%
\pgfpathcurveto{\pgfqpoint{2.217228in}{1.632903in}}{\pgfqpoint{2.225128in}{1.629631in}}{\pgfqpoint{2.233364in}{1.629631in}}%
\pgfpathclose%
\pgfusepath{stroke,fill}%
\end{pgfscope}%
\begin{pgfscope}%
\pgfpathrectangle{\pgfqpoint{0.100000in}{0.212622in}}{\pgfqpoint{3.696000in}{3.696000in}}%
\pgfusepath{clip}%
\pgfsetbuttcap%
\pgfsetroundjoin%
\definecolor{currentfill}{rgb}{0.121569,0.466667,0.705882}%
\pgfsetfillcolor{currentfill}%
\pgfsetfillopacity{0.774032}%
\pgfsetlinewidth{1.003750pt}%
\definecolor{currentstroke}{rgb}{0.121569,0.466667,0.705882}%
\pgfsetstrokecolor{currentstroke}%
\pgfsetstrokeopacity{0.774032}%
\pgfsetdash{}{0pt}%
\pgfpathmoveto{\pgfqpoint{2.233981in}{1.628460in}}%
\pgfpathcurveto{\pgfqpoint{2.242217in}{1.628460in}}{\pgfqpoint{2.250118in}{1.631732in}}{\pgfqpoint{2.255941in}{1.637556in}}%
\pgfpathcurveto{\pgfqpoint{2.261765in}{1.643380in}}{\pgfqpoint{2.265038in}{1.651280in}}{\pgfqpoint{2.265038in}{1.659516in}}%
\pgfpathcurveto{\pgfqpoint{2.265038in}{1.667752in}}{\pgfqpoint{2.261765in}{1.675653in}}{\pgfqpoint{2.255941in}{1.681476in}}%
\pgfpathcurveto{\pgfqpoint{2.250118in}{1.687300in}}{\pgfqpoint{2.242217in}{1.690573in}}{\pgfqpoint{2.233981in}{1.690573in}}%
\pgfpathcurveto{\pgfqpoint{2.225745in}{1.690573in}}{\pgfqpoint{2.217845in}{1.687300in}}{\pgfqpoint{2.212021in}{1.681476in}}%
\pgfpathcurveto{\pgfqpoint{2.206197in}{1.675653in}}{\pgfqpoint{2.202925in}{1.667752in}}{\pgfqpoint{2.202925in}{1.659516in}}%
\pgfpathcurveto{\pgfqpoint{2.202925in}{1.651280in}}{\pgfqpoint{2.206197in}{1.643380in}}{\pgfqpoint{2.212021in}{1.637556in}}%
\pgfpathcurveto{\pgfqpoint{2.217845in}{1.631732in}}{\pgfqpoint{2.225745in}{1.628460in}}{\pgfqpoint{2.233981in}{1.628460in}}%
\pgfpathclose%
\pgfusepath{stroke,fill}%
\end{pgfscope}%
\begin{pgfscope}%
\pgfpathrectangle{\pgfqpoint{0.100000in}{0.212622in}}{\pgfqpoint{3.696000in}{3.696000in}}%
\pgfusepath{clip}%
\pgfsetbuttcap%
\pgfsetroundjoin%
\definecolor{currentfill}{rgb}{0.121569,0.466667,0.705882}%
\pgfsetfillcolor{currentfill}%
\pgfsetfillopacity{0.774500}%
\pgfsetlinewidth{1.003750pt}%
\definecolor{currentstroke}{rgb}{0.121569,0.466667,0.705882}%
\pgfsetstrokecolor{currentstroke}%
\pgfsetstrokeopacity{0.774500}%
\pgfsetdash{}{0pt}%
\pgfpathmoveto{\pgfqpoint{2.234380in}{1.628156in}}%
\pgfpathcurveto{\pgfqpoint{2.242616in}{1.628156in}}{\pgfqpoint{2.250516in}{1.631428in}}{\pgfqpoint{2.256340in}{1.637252in}}%
\pgfpathcurveto{\pgfqpoint{2.262164in}{1.643076in}}{\pgfqpoint{2.265436in}{1.650976in}}{\pgfqpoint{2.265436in}{1.659212in}}%
\pgfpathcurveto{\pgfqpoint{2.265436in}{1.667449in}}{\pgfqpoint{2.262164in}{1.675349in}}{\pgfqpoint{2.256340in}{1.681173in}}%
\pgfpathcurveto{\pgfqpoint{2.250516in}{1.686997in}}{\pgfqpoint{2.242616in}{1.690269in}}{\pgfqpoint{2.234380in}{1.690269in}}%
\pgfpathcurveto{\pgfqpoint{2.226143in}{1.690269in}}{\pgfqpoint{2.218243in}{1.686997in}}{\pgfqpoint{2.212420in}{1.681173in}}%
\pgfpathcurveto{\pgfqpoint{2.206596in}{1.675349in}}{\pgfqpoint{2.203323in}{1.667449in}}{\pgfqpoint{2.203323in}{1.659212in}}%
\pgfpathcurveto{\pgfqpoint{2.203323in}{1.650976in}}{\pgfqpoint{2.206596in}{1.643076in}}{\pgfqpoint{2.212420in}{1.637252in}}%
\pgfpathcurveto{\pgfqpoint{2.218243in}{1.631428in}}{\pgfqpoint{2.226143in}{1.628156in}}{\pgfqpoint{2.234380in}{1.628156in}}%
\pgfpathclose%
\pgfusepath{stroke,fill}%
\end{pgfscope}%
\begin{pgfscope}%
\pgfpathrectangle{\pgfqpoint{0.100000in}{0.212622in}}{\pgfqpoint{3.696000in}{3.696000in}}%
\pgfusepath{clip}%
\pgfsetbuttcap%
\pgfsetroundjoin%
\definecolor{currentfill}{rgb}{0.121569,0.466667,0.705882}%
\pgfsetfillcolor{currentfill}%
\pgfsetfillopacity{0.775546}%
\pgfsetlinewidth{1.003750pt}%
\definecolor{currentstroke}{rgb}{0.121569,0.466667,0.705882}%
\pgfsetstrokecolor{currentstroke}%
\pgfsetstrokeopacity{0.775546}%
\pgfsetdash{}{0pt}%
\pgfpathmoveto{\pgfqpoint{2.235527in}{1.626418in}}%
\pgfpathcurveto{\pgfqpoint{2.243763in}{1.626418in}}{\pgfqpoint{2.251663in}{1.629690in}}{\pgfqpoint{2.257487in}{1.635514in}}%
\pgfpathcurveto{\pgfqpoint{2.263311in}{1.641338in}}{\pgfqpoint{2.266584in}{1.649238in}}{\pgfqpoint{2.266584in}{1.657474in}}%
\pgfpathcurveto{\pgfqpoint{2.266584in}{1.665711in}}{\pgfqpoint{2.263311in}{1.673611in}}{\pgfqpoint{2.257487in}{1.679435in}}%
\pgfpathcurveto{\pgfqpoint{2.251663in}{1.685258in}}{\pgfqpoint{2.243763in}{1.688531in}}{\pgfqpoint{2.235527in}{1.688531in}}%
\pgfpathcurveto{\pgfqpoint{2.227291in}{1.688531in}}{\pgfqpoint{2.219391in}{1.685258in}}{\pgfqpoint{2.213567in}{1.679435in}}%
\pgfpathcurveto{\pgfqpoint{2.207743in}{1.673611in}}{\pgfqpoint{2.204471in}{1.665711in}}{\pgfqpoint{2.204471in}{1.657474in}}%
\pgfpathcurveto{\pgfqpoint{2.204471in}{1.649238in}}{\pgfqpoint{2.207743in}{1.641338in}}{\pgfqpoint{2.213567in}{1.635514in}}%
\pgfpathcurveto{\pgfqpoint{2.219391in}{1.629690in}}{\pgfqpoint{2.227291in}{1.626418in}}{\pgfqpoint{2.235527in}{1.626418in}}%
\pgfpathclose%
\pgfusepath{stroke,fill}%
\end{pgfscope}%
\begin{pgfscope}%
\pgfpathrectangle{\pgfqpoint{0.100000in}{0.212622in}}{\pgfqpoint{3.696000in}{3.696000in}}%
\pgfusepath{clip}%
\pgfsetbuttcap%
\pgfsetroundjoin%
\definecolor{currentfill}{rgb}{0.121569,0.466667,0.705882}%
\pgfsetfillcolor{currentfill}%
\pgfsetfillopacity{0.776601}%
\pgfsetlinewidth{1.003750pt}%
\definecolor{currentstroke}{rgb}{0.121569,0.466667,0.705882}%
\pgfsetstrokecolor{currentstroke}%
\pgfsetstrokeopacity{0.776601}%
\pgfsetdash{}{0pt}%
\pgfpathmoveto{\pgfqpoint{2.236637in}{1.623337in}}%
\pgfpathcurveto{\pgfqpoint{2.244873in}{1.623337in}}{\pgfqpoint{2.252773in}{1.626610in}}{\pgfqpoint{2.258597in}{1.632433in}}%
\pgfpathcurveto{\pgfqpoint{2.264421in}{1.638257in}}{\pgfqpoint{2.267693in}{1.646157in}}{\pgfqpoint{2.267693in}{1.654394in}}%
\pgfpathcurveto{\pgfqpoint{2.267693in}{1.662630in}}{\pgfqpoint{2.264421in}{1.670530in}}{\pgfqpoint{2.258597in}{1.676354in}}%
\pgfpathcurveto{\pgfqpoint{2.252773in}{1.682178in}}{\pgfqpoint{2.244873in}{1.685450in}}{\pgfqpoint{2.236637in}{1.685450in}}%
\pgfpathcurveto{\pgfqpoint{2.228401in}{1.685450in}}{\pgfqpoint{2.220501in}{1.682178in}}{\pgfqpoint{2.214677in}{1.676354in}}%
\pgfpathcurveto{\pgfqpoint{2.208853in}{1.670530in}}{\pgfqpoint{2.205580in}{1.662630in}}{\pgfqpoint{2.205580in}{1.654394in}}%
\pgfpathcurveto{\pgfqpoint{2.205580in}{1.646157in}}{\pgfqpoint{2.208853in}{1.638257in}}{\pgfqpoint{2.214677in}{1.632433in}}%
\pgfpathcurveto{\pgfqpoint{2.220501in}{1.626610in}}{\pgfqpoint{2.228401in}{1.623337in}}{\pgfqpoint{2.236637in}{1.623337in}}%
\pgfpathclose%
\pgfusepath{stroke,fill}%
\end{pgfscope}%
\begin{pgfscope}%
\pgfpathrectangle{\pgfqpoint{0.100000in}{0.212622in}}{\pgfqpoint{3.696000in}{3.696000in}}%
\pgfusepath{clip}%
\pgfsetbuttcap%
\pgfsetroundjoin%
\definecolor{currentfill}{rgb}{0.121569,0.466667,0.705882}%
\pgfsetfillcolor{currentfill}%
\pgfsetfillopacity{0.778445}%
\pgfsetlinewidth{1.003750pt}%
\definecolor{currentstroke}{rgb}{0.121569,0.466667,0.705882}%
\pgfsetstrokecolor{currentstroke}%
\pgfsetstrokeopacity{0.778445}%
\pgfsetdash{}{0pt}%
\pgfpathmoveto{\pgfqpoint{2.237889in}{1.620584in}}%
\pgfpathcurveto{\pgfqpoint{2.246125in}{1.620584in}}{\pgfqpoint{2.254025in}{1.623856in}}{\pgfqpoint{2.259849in}{1.629680in}}%
\pgfpathcurveto{\pgfqpoint{2.265673in}{1.635504in}}{\pgfqpoint{2.268945in}{1.643404in}}{\pgfqpoint{2.268945in}{1.651640in}}%
\pgfpathcurveto{\pgfqpoint{2.268945in}{1.659877in}}{\pgfqpoint{2.265673in}{1.667777in}}{\pgfqpoint{2.259849in}{1.673601in}}%
\pgfpathcurveto{\pgfqpoint{2.254025in}{1.679425in}}{\pgfqpoint{2.246125in}{1.682697in}}{\pgfqpoint{2.237889in}{1.682697in}}%
\pgfpathcurveto{\pgfqpoint{2.229652in}{1.682697in}}{\pgfqpoint{2.221752in}{1.679425in}}{\pgfqpoint{2.215928in}{1.673601in}}%
\pgfpathcurveto{\pgfqpoint{2.210104in}{1.667777in}}{\pgfqpoint{2.206832in}{1.659877in}}{\pgfqpoint{2.206832in}{1.651640in}}%
\pgfpathcurveto{\pgfqpoint{2.206832in}{1.643404in}}{\pgfqpoint{2.210104in}{1.635504in}}{\pgfqpoint{2.215928in}{1.629680in}}%
\pgfpathcurveto{\pgfqpoint{2.221752in}{1.623856in}}{\pgfqpoint{2.229652in}{1.620584in}}{\pgfqpoint{2.237889in}{1.620584in}}%
\pgfpathclose%
\pgfusepath{stroke,fill}%
\end{pgfscope}%
\begin{pgfscope}%
\pgfpathrectangle{\pgfqpoint{0.100000in}{0.212622in}}{\pgfqpoint{3.696000in}{3.696000in}}%
\pgfusepath{clip}%
\pgfsetbuttcap%
\pgfsetroundjoin%
\definecolor{currentfill}{rgb}{0.121569,0.466667,0.705882}%
\pgfsetfillcolor{currentfill}%
\pgfsetfillopacity{0.779483}%
\pgfsetlinewidth{1.003750pt}%
\definecolor{currentstroke}{rgb}{0.121569,0.466667,0.705882}%
\pgfsetstrokecolor{currentstroke}%
\pgfsetstrokeopacity{0.779483}%
\pgfsetdash{}{0pt}%
\pgfpathmoveto{\pgfqpoint{2.238922in}{1.619424in}}%
\pgfpathcurveto{\pgfqpoint{2.247159in}{1.619424in}}{\pgfqpoint{2.255059in}{1.622696in}}{\pgfqpoint{2.260883in}{1.628520in}}%
\pgfpathcurveto{\pgfqpoint{2.266707in}{1.634344in}}{\pgfqpoint{2.269979in}{1.642244in}}{\pgfqpoint{2.269979in}{1.650481in}}%
\pgfpathcurveto{\pgfqpoint{2.269979in}{1.658717in}}{\pgfqpoint{2.266707in}{1.666617in}}{\pgfqpoint{2.260883in}{1.672441in}}%
\pgfpathcurveto{\pgfqpoint{2.255059in}{1.678265in}}{\pgfqpoint{2.247159in}{1.681537in}}{\pgfqpoint{2.238922in}{1.681537in}}%
\pgfpathcurveto{\pgfqpoint{2.230686in}{1.681537in}}{\pgfqpoint{2.222786in}{1.678265in}}{\pgfqpoint{2.216962in}{1.672441in}}%
\pgfpathcurveto{\pgfqpoint{2.211138in}{1.666617in}}{\pgfqpoint{2.207866in}{1.658717in}}{\pgfqpoint{2.207866in}{1.650481in}}%
\pgfpathcurveto{\pgfqpoint{2.207866in}{1.642244in}}{\pgfqpoint{2.211138in}{1.634344in}}{\pgfqpoint{2.216962in}{1.628520in}}%
\pgfpathcurveto{\pgfqpoint{2.222786in}{1.622696in}}{\pgfqpoint{2.230686in}{1.619424in}}{\pgfqpoint{2.238922in}{1.619424in}}%
\pgfpathclose%
\pgfusepath{stroke,fill}%
\end{pgfscope}%
\begin{pgfscope}%
\pgfpathrectangle{\pgfqpoint{0.100000in}{0.212622in}}{\pgfqpoint{3.696000in}{3.696000in}}%
\pgfusepath{clip}%
\pgfsetbuttcap%
\pgfsetroundjoin%
\definecolor{currentfill}{rgb}{0.121569,0.466667,0.705882}%
\pgfsetfillcolor{currentfill}%
\pgfsetfillopacity{0.781080}%
\pgfsetlinewidth{1.003750pt}%
\definecolor{currentstroke}{rgb}{0.121569,0.466667,0.705882}%
\pgfsetstrokecolor{currentstroke}%
\pgfsetstrokeopacity{0.781080}%
\pgfsetdash{}{0pt}%
\pgfpathmoveto{\pgfqpoint{2.240324in}{1.616622in}}%
\pgfpathcurveto{\pgfqpoint{2.248561in}{1.616622in}}{\pgfqpoint{2.256461in}{1.619894in}}{\pgfqpoint{2.262284in}{1.625718in}}%
\pgfpathcurveto{\pgfqpoint{2.268108in}{1.631542in}}{\pgfqpoint{2.271381in}{1.639442in}}{\pgfqpoint{2.271381in}{1.647678in}}%
\pgfpathcurveto{\pgfqpoint{2.271381in}{1.655914in}}{\pgfqpoint{2.268108in}{1.663814in}}{\pgfqpoint{2.262284in}{1.669638in}}%
\pgfpathcurveto{\pgfqpoint{2.256461in}{1.675462in}}{\pgfqpoint{2.248561in}{1.678735in}}{\pgfqpoint{2.240324in}{1.678735in}}%
\pgfpathcurveto{\pgfqpoint{2.232088in}{1.678735in}}{\pgfqpoint{2.224188in}{1.675462in}}{\pgfqpoint{2.218364in}{1.669638in}}%
\pgfpathcurveto{\pgfqpoint{2.212540in}{1.663814in}}{\pgfqpoint{2.209268in}{1.655914in}}{\pgfqpoint{2.209268in}{1.647678in}}%
\pgfpathcurveto{\pgfqpoint{2.209268in}{1.639442in}}{\pgfqpoint{2.212540in}{1.631542in}}{\pgfqpoint{2.218364in}{1.625718in}}%
\pgfpathcurveto{\pgfqpoint{2.224188in}{1.619894in}}{\pgfqpoint{2.232088in}{1.616622in}}{\pgfqpoint{2.240324in}{1.616622in}}%
\pgfpathclose%
\pgfusepath{stroke,fill}%
\end{pgfscope}%
\begin{pgfscope}%
\pgfpathrectangle{\pgfqpoint{0.100000in}{0.212622in}}{\pgfqpoint{3.696000in}{3.696000in}}%
\pgfusepath{clip}%
\pgfsetbuttcap%
\pgfsetroundjoin%
\definecolor{currentfill}{rgb}{0.121569,0.466667,0.705882}%
\pgfsetfillcolor{currentfill}%
\pgfsetfillopacity{0.781799}%
\pgfsetlinewidth{1.003750pt}%
\definecolor{currentstroke}{rgb}{0.121569,0.466667,0.705882}%
\pgfsetstrokecolor{currentstroke}%
\pgfsetstrokeopacity{0.781799}%
\pgfsetdash{}{0pt}%
\pgfpathmoveto{\pgfqpoint{2.240997in}{1.614021in}}%
\pgfpathcurveto{\pgfqpoint{2.249234in}{1.614021in}}{\pgfqpoint{2.257134in}{1.617294in}}{\pgfqpoint{2.262958in}{1.623118in}}%
\pgfpathcurveto{\pgfqpoint{2.268781in}{1.628941in}}{\pgfqpoint{2.272054in}{1.636841in}}{\pgfqpoint{2.272054in}{1.645078in}}%
\pgfpathcurveto{\pgfqpoint{2.272054in}{1.653314in}}{\pgfqpoint{2.268781in}{1.661214in}}{\pgfqpoint{2.262958in}{1.667038in}}%
\pgfpathcurveto{\pgfqpoint{2.257134in}{1.672862in}}{\pgfqpoint{2.249234in}{1.676134in}}{\pgfqpoint{2.240997in}{1.676134in}}%
\pgfpathcurveto{\pgfqpoint{2.232761in}{1.676134in}}{\pgfqpoint{2.224861in}{1.672862in}}{\pgfqpoint{2.219037in}{1.667038in}}%
\pgfpathcurveto{\pgfqpoint{2.213213in}{1.661214in}}{\pgfqpoint{2.209941in}{1.653314in}}{\pgfqpoint{2.209941in}{1.645078in}}%
\pgfpathcurveto{\pgfqpoint{2.209941in}{1.636841in}}{\pgfqpoint{2.213213in}{1.628941in}}{\pgfqpoint{2.219037in}{1.623118in}}%
\pgfpathcurveto{\pgfqpoint{2.224861in}{1.617294in}}{\pgfqpoint{2.232761in}{1.614021in}}{\pgfqpoint{2.240997in}{1.614021in}}%
\pgfpathclose%
\pgfusepath{stroke,fill}%
\end{pgfscope}%
\begin{pgfscope}%
\pgfpathrectangle{\pgfqpoint{0.100000in}{0.212622in}}{\pgfqpoint{3.696000in}{3.696000in}}%
\pgfusepath{clip}%
\pgfsetbuttcap%
\pgfsetroundjoin%
\definecolor{currentfill}{rgb}{0.121569,0.466667,0.705882}%
\pgfsetfillcolor{currentfill}%
\pgfsetfillopacity{0.782940}%
\pgfsetlinewidth{1.003750pt}%
\definecolor{currentstroke}{rgb}{0.121569,0.466667,0.705882}%
\pgfsetstrokecolor{currentstroke}%
\pgfsetstrokeopacity{0.782940}%
\pgfsetdash{}{0pt}%
\pgfpathmoveto{\pgfqpoint{2.241979in}{1.612504in}}%
\pgfpathcurveto{\pgfqpoint{2.250215in}{1.612504in}}{\pgfqpoint{2.258115in}{1.615776in}}{\pgfqpoint{2.263939in}{1.621600in}}%
\pgfpathcurveto{\pgfqpoint{2.269763in}{1.627424in}}{\pgfqpoint{2.273035in}{1.635324in}}{\pgfqpoint{2.273035in}{1.643561in}}%
\pgfpathcurveto{\pgfqpoint{2.273035in}{1.651797in}}{\pgfqpoint{2.269763in}{1.659697in}}{\pgfqpoint{2.263939in}{1.665521in}}%
\pgfpathcurveto{\pgfqpoint{2.258115in}{1.671345in}}{\pgfqpoint{2.250215in}{1.674617in}}{\pgfqpoint{2.241979in}{1.674617in}}%
\pgfpathcurveto{\pgfqpoint{2.233742in}{1.674617in}}{\pgfqpoint{2.225842in}{1.671345in}}{\pgfqpoint{2.220018in}{1.665521in}}%
\pgfpathcurveto{\pgfqpoint{2.214194in}{1.659697in}}{\pgfqpoint{2.210922in}{1.651797in}}{\pgfqpoint{2.210922in}{1.643561in}}%
\pgfpathcurveto{\pgfqpoint{2.210922in}{1.635324in}}{\pgfqpoint{2.214194in}{1.627424in}}{\pgfqpoint{2.220018in}{1.621600in}}%
\pgfpathcurveto{\pgfqpoint{2.225842in}{1.615776in}}{\pgfqpoint{2.233742in}{1.612504in}}{\pgfqpoint{2.241979in}{1.612504in}}%
\pgfpathclose%
\pgfusepath{stroke,fill}%
\end{pgfscope}%
\begin{pgfscope}%
\pgfpathrectangle{\pgfqpoint{0.100000in}{0.212622in}}{\pgfqpoint{3.696000in}{3.696000in}}%
\pgfusepath{clip}%
\pgfsetbuttcap%
\pgfsetroundjoin%
\definecolor{currentfill}{rgb}{0.121569,0.466667,0.705882}%
\pgfsetfillcolor{currentfill}%
\pgfsetfillopacity{0.783645}%
\pgfsetlinewidth{1.003750pt}%
\definecolor{currentstroke}{rgb}{0.121569,0.466667,0.705882}%
\pgfsetstrokecolor{currentstroke}%
\pgfsetstrokeopacity{0.783645}%
\pgfsetdash{}{0pt}%
\pgfpathmoveto{\pgfqpoint{2.242420in}{1.612094in}}%
\pgfpathcurveto{\pgfqpoint{2.250656in}{1.612094in}}{\pgfqpoint{2.258556in}{1.615366in}}{\pgfqpoint{2.264380in}{1.621190in}}%
\pgfpathcurveto{\pgfqpoint{2.270204in}{1.627014in}}{\pgfqpoint{2.273477in}{1.634914in}}{\pgfqpoint{2.273477in}{1.643150in}}%
\pgfpathcurveto{\pgfqpoint{2.273477in}{1.651387in}}{\pgfqpoint{2.270204in}{1.659287in}}{\pgfqpoint{2.264380in}{1.665111in}}%
\pgfpathcurveto{\pgfqpoint{2.258556in}{1.670935in}}{\pgfqpoint{2.250656in}{1.674207in}}{\pgfqpoint{2.242420in}{1.674207in}}%
\pgfpathcurveto{\pgfqpoint{2.234184in}{1.674207in}}{\pgfqpoint{2.226284in}{1.670935in}}{\pgfqpoint{2.220460in}{1.665111in}}%
\pgfpathcurveto{\pgfqpoint{2.214636in}{1.659287in}}{\pgfqpoint{2.211364in}{1.651387in}}{\pgfqpoint{2.211364in}{1.643150in}}%
\pgfpathcurveto{\pgfqpoint{2.211364in}{1.634914in}}{\pgfqpoint{2.214636in}{1.627014in}}{\pgfqpoint{2.220460in}{1.621190in}}%
\pgfpathcurveto{\pgfqpoint{2.226284in}{1.615366in}}{\pgfqpoint{2.234184in}{1.612094in}}{\pgfqpoint{2.242420in}{1.612094in}}%
\pgfpathclose%
\pgfusepath{stroke,fill}%
\end{pgfscope}%
\begin{pgfscope}%
\pgfpathrectangle{\pgfqpoint{0.100000in}{0.212622in}}{\pgfqpoint{3.696000in}{3.696000in}}%
\pgfusepath{clip}%
\pgfsetbuttcap%
\pgfsetroundjoin%
\definecolor{currentfill}{rgb}{0.121569,0.466667,0.705882}%
\pgfsetfillcolor{currentfill}%
\pgfsetfillopacity{0.784821}%
\pgfsetlinewidth{1.003750pt}%
\definecolor{currentstroke}{rgb}{0.121569,0.466667,0.705882}%
\pgfsetstrokecolor{currentstroke}%
\pgfsetstrokeopacity{0.784821}%
\pgfsetdash{}{0pt}%
\pgfpathmoveto{\pgfqpoint{2.243389in}{1.610785in}}%
\pgfpathcurveto{\pgfqpoint{2.251625in}{1.610785in}}{\pgfqpoint{2.259525in}{1.614058in}}{\pgfqpoint{2.265349in}{1.619882in}}%
\pgfpathcurveto{\pgfqpoint{2.271173in}{1.625706in}}{\pgfqpoint{2.274445in}{1.633606in}}{\pgfqpoint{2.274445in}{1.641842in}}%
\pgfpathcurveto{\pgfqpoint{2.274445in}{1.650078in}}{\pgfqpoint{2.271173in}{1.657978in}}{\pgfqpoint{2.265349in}{1.663802in}}%
\pgfpathcurveto{\pgfqpoint{2.259525in}{1.669626in}}{\pgfqpoint{2.251625in}{1.672898in}}{\pgfqpoint{2.243389in}{1.672898in}}%
\pgfpathcurveto{\pgfqpoint{2.235152in}{1.672898in}}{\pgfqpoint{2.227252in}{1.669626in}}{\pgfqpoint{2.221429in}{1.663802in}}%
\pgfpathcurveto{\pgfqpoint{2.215605in}{1.657978in}}{\pgfqpoint{2.212332in}{1.650078in}}{\pgfqpoint{2.212332in}{1.641842in}}%
\pgfpathcurveto{\pgfqpoint{2.212332in}{1.633606in}}{\pgfqpoint{2.215605in}{1.625706in}}{\pgfqpoint{2.221429in}{1.619882in}}%
\pgfpathcurveto{\pgfqpoint{2.227252in}{1.614058in}}{\pgfqpoint{2.235152in}{1.610785in}}{\pgfqpoint{2.243389in}{1.610785in}}%
\pgfpathclose%
\pgfusepath{stroke,fill}%
\end{pgfscope}%
\begin{pgfscope}%
\pgfpathrectangle{\pgfqpoint{0.100000in}{0.212622in}}{\pgfqpoint{3.696000in}{3.696000in}}%
\pgfusepath{clip}%
\pgfsetbuttcap%
\pgfsetroundjoin%
\definecolor{currentfill}{rgb}{0.121569,0.466667,0.705882}%
\pgfsetfillcolor{currentfill}%
\pgfsetfillopacity{0.785360}%
\pgfsetlinewidth{1.003750pt}%
\definecolor{currentstroke}{rgb}{0.121569,0.466667,0.705882}%
\pgfsetstrokecolor{currentstroke}%
\pgfsetstrokeopacity{0.785360}%
\pgfsetdash{}{0pt}%
\pgfpathmoveto{\pgfqpoint{2.243929in}{1.609393in}}%
\pgfpathcurveto{\pgfqpoint{2.252165in}{1.609393in}}{\pgfqpoint{2.260065in}{1.612665in}}{\pgfqpoint{2.265889in}{1.618489in}}%
\pgfpathcurveto{\pgfqpoint{2.271713in}{1.624313in}}{\pgfqpoint{2.274986in}{1.632213in}}{\pgfqpoint{2.274986in}{1.640449in}}%
\pgfpathcurveto{\pgfqpoint{2.274986in}{1.648685in}}{\pgfqpoint{2.271713in}{1.656585in}}{\pgfqpoint{2.265889in}{1.662409in}}%
\pgfpathcurveto{\pgfqpoint{2.260065in}{1.668233in}}{\pgfqpoint{2.252165in}{1.671506in}}{\pgfqpoint{2.243929in}{1.671506in}}%
\pgfpathcurveto{\pgfqpoint{2.235693in}{1.671506in}}{\pgfqpoint{2.227793in}{1.668233in}}{\pgfqpoint{2.221969in}{1.662409in}}%
\pgfpathcurveto{\pgfqpoint{2.216145in}{1.656585in}}{\pgfqpoint{2.212873in}{1.648685in}}{\pgfqpoint{2.212873in}{1.640449in}}%
\pgfpathcurveto{\pgfqpoint{2.212873in}{1.632213in}}{\pgfqpoint{2.216145in}{1.624313in}}{\pgfqpoint{2.221969in}{1.618489in}}%
\pgfpathcurveto{\pgfqpoint{2.227793in}{1.612665in}}{\pgfqpoint{2.235693in}{1.609393in}}{\pgfqpoint{2.243929in}{1.609393in}}%
\pgfpathclose%
\pgfusepath{stroke,fill}%
\end{pgfscope}%
\begin{pgfscope}%
\pgfpathrectangle{\pgfqpoint{0.100000in}{0.212622in}}{\pgfqpoint{3.696000in}{3.696000in}}%
\pgfusepath{clip}%
\pgfsetbuttcap%
\pgfsetroundjoin%
\definecolor{currentfill}{rgb}{0.121569,0.466667,0.705882}%
\pgfsetfillcolor{currentfill}%
\pgfsetfillopacity{0.786130}%
\pgfsetlinewidth{1.003750pt}%
\definecolor{currentstroke}{rgb}{0.121569,0.466667,0.705882}%
\pgfsetstrokecolor{currentstroke}%
\pgfsetstrokeopacity{0.786130}%
\pgfsetdash{}{0pt}%
\pgfpathmoveto{\pgfqpoint{2.244728in}{1.608237in}}%
\pgfpathcurveto{\pgfqpoint{2.252964in}{1.608237in}}{\pgfqpoint{2.260864in}{1.611509in}}{\pgfqpoint{2.266688in}{1.617333in}}%
\pgfpathcurveto{\pgfqpoint{2.272512in}{1.623157in}}{\pgfqpoint{2.275785in}{1.631057in}}{\pgfqpoint{2.275785in}{1.639293in}}%
\pgfpathcurveto{\pgfqpoint{2.275785in}{1.647530in}}{\pgfqpoint{2.272512in}{1.655430in}}{\pgfqpoint{2.266688in}{1.661254in}}%
\pgfpathcurveto{\pgfqpoint{2.260864in}{1.667078in}}{\pgfqpoint{2.252964in}{1.670350in}}{\pgfqpoint{2.244728in}{1.670350in}}%
\pgfpathcurveto{\pgfqpoint{2.236492in}{1.670350in}}{\pgfqpoint{2.228592in}{1.667078in}}{\pgfqpoint{2.222768in}{1.661254in}}%
\pgfpathcurveto{\pgfqpoint{2.216944in}{1.655430in}}{\pgfqpoint{2.213672in}{1.647530in}}{\pgfqpoint{2.213672in}{1.639293in}}%
\pgfpathcurveto{\pgfqpoint{2.213672in}{1.631057in}}{\pgfqpoint{2.216944in}{1.623157in}}{\pgfqpoint{2.222768in}{1.617333in}}%
\pgfpathcurveto{\pgfqpoint{2.228592in}{1.611509in}}{\pgfqpoint{2.236492in}{1.608237in}}{\pgfqpoint{2.244728in}{1.608237in}}%
\pgfpathclose%
\pgfusepath{stroke,fill}%
\end{pgfscope}%
\begin{pgfscope}%
\pgfpathrectangle{\pgfqpoint{0.100000in}{0.212622in}}{\pgfqpoint{3.696000in}{3.696000in}}%
\pgfusepath{clip}%
\pgfsetbuttcap%
\pgfsetroundjoin%
\definecolor{currentfill}{rgb}{0.121569,0.466667,0.705882}%
\pgfsetfillcolor{currentfill}%
\pgfsetfillopacity{0.786642}%
\pgfsetlinewidth{1.003750pt}%
\definecolor{currentstroke}{rgb}{0.121569,0.466667,0.705882}%
\pgfsetstrokecolor{currentstroke}%
\pgfsetstrokeopacity{0.786642}%
\pgfsetdash{}{0pt}%
\pgfpathmoveto{\pgfqpoint{2.245082in}{1.608095in}}%
\pgfpathcurveto{\pgfqpoint{2.253318in}{1.608095in}}{\pgfqpoint{2.261218in}{1.611368in}}{\pgfqpoint{2.267042in}{1.617191in}}%
\pgfpathcurveto{\pgfqpoint{2.272866in}{1.623015in}}{\pgfqpoint{2.276138in}{1.630915in}}{\pgfqpoint{2.276138in}{1.639152in}}%
\pgfpathcurveto{\pgfqpoint{2.276138in}{1.647388in}}{\pgfqpoint{2.272866in}{1.655288in}}{\pgfqpoint{2.267042in}{1.661112in}}%
\pgfpathcurveto{\pgfqpoint{2.261218in}{1.666936in}}{\pgfqpoint{2.253318in}{1.670208in}}{\pgfqpoint{2.245082in}{1.670208in}}%
\pgfpathcurveto{\pgfqpoint{2.236846in}{1.670208in}}{\pgfqpoint{2.228946in}{1.666936in}}{\pgfqpoint{2.223122in}{1.661112in}}%
\pgfpathcurveto{\pgfqpoint{2.217298in}{1.655288in}}{\pgfqpoint{2.214025in}{1.647388in}}{\pgfqpoint{2.214025in}{1.639152in}}%
\pgfpathcurveto{\pgfqpoint{2.214025in}{1.630915in}}{\pgfqpoint{2.217298in}{1.623015in}}{\pgfqpoint{2.223122in}{1.617191in}}%
\pgfpathcurveto{\pgfqpoint{2.228946in}{1.611368in}}{\pgfqpoint{2.236846in}{1.608095in}}{\pgfqpoint{2.245082in}{1.608095in}}%
\pgfpathclose%
\pgfusepath{stroke,fill}%
\end{pgfscope}%
\begin{pgfscope}%
\pgfpathrectangle{\pgfqpoint{0.100000in}{0.212622in}}{\pgfqpoint{3.696000in}{3.696000in}}%
\pgfusepath{clip}%
\pgfsetbuttcap%
\pgfsetroundjoin%
\definecolor{currentfill}{rgb}{0.121569,0.466667,0.705882}%
\pgfsetfillcolor{currentfill}%
\pgfsetfillopacity{0.787404}%
\pgfsetlinewidth{1.003750pt}%
\definecolor{currentstroke}{rgb}{0.121569,0.466667,0.705882}%
\pgfsetstrokecolor{currentstroke}%
\pgfsetstrokeopacity{0.787404}%
\pgfsetdash{}{0pt}%
\pgfpathmoveto{\pgfqpoint{2.245697in}{1.607249in}}%
\pgfpathcurveto{\pgfqpoint{2.253934in}{1.607249in}}{\pgfqpoint{2.261834in}{1.610522in}}{\pgfqpoint{2.267657in}{1.616345in}}%
\pgfpathcurveto{\pgfqpoint{2.273481in}{1.622169in}}{\pgfqpoint{2.276754in}{1.630069in}}{\pgfqpoint{2.276754in}{1.638306in}}%
\pgfpathcurveto{\pgfqpoint{2.276754in}{1.646542in}}{\pgfqpoint{2.273481in}{1.654442in}}{\pgfqpoint{2.267657in}{1.660266in}}%
\pgfpathcurveto{\pgfqpoint{2.261834in}{1.666090in}}{\pgfqpoint{2.253934in}{1.669362in}}{\pgfqpoint{2.245697in}{1.669362in}}%
\pgfpathcurveto{\pgfqpoint{2.237461in}{1.669362in}}{\pgfqpoint{2.229561in}{1.666090in}}{\pgfqpoint{2.223737in}{1.660266in}}%
\pgfpathcurveto{\pgfqpoint{2.217913in}{1.654442in}}{\pgfqpoint{2.214641in}{1.646542in}}{\pgfqpoint{2.214641in}{1.638306in}}%
\pgfpathcurveto{\pgfqpoint{2.214641in}{1.630069in}}{\pgfqpoint{2.217913in}{1.622169in}}{\pgfqpoint{2.223737in}{1.616345in}}%
\pgfpathcurveto{\pgfqpoint{2.229561in}{1.610522in}}{\pgfqpoint{2.237461in}{1.607249in}}{\pgfqpoint{2.245697in}{1.607249in}}%
\pgfpathclose%
\pgfusepath{stroke,fill}%
\end{pgfscope}%
\begin{pgfscope}%
\pgfpathrectangle{\pgfqpoint{0.100000in}{0.212622in}}{\pgfqpoint{3.696000in}{3.696000in}}%
\pgfusepath{clip}%
\pgfsetbuttcap%
\pgfsetroundjoin%
\definecolor{currentfill}{rgb}{0.121569,0.466667,0.705882}%
\pgfsetfillcolor{currentfill}%
\pgfsetfillopacity{0.788284}%
\pgfsetlinewidth{1.003750pt}%
\definecolor{currentstroke}{rgb}{0.121569,0.466667,0.705882}%
\pgfsetstrokecolor{currentstroke}%
\pgfsetstrokeopacity{0.788284}%
\pgfsetdash{}{0pt}%
\pgfpathmoveto{\pgfqpoint{2.246474in}{1.606040in}}%
\pgfpathcurveto{\pgfqpoint{2.254710in}{1.606040in}}{\pgfqpoint{2.262610in}{1.609313in}}{\pgfqpoint{2.268434in}{1.615136in}}%
\pgfpathcurveto{\pgfqpoint{2.274258in}{1.620960in}}{\pgfqpoint{2.277530in}{1.628860in}}{\pgfqpoint{2.277530in}{1.637097in}}%
\pgfpathcurveto{\pgfqpoint{2.277530in}{1.645333in}}{\pgfqpoint{2.274258in}{1.653233in}}{\pgfqpoint{2.268434in}{1.659057in}}%
\pgfpathcurveto{\pgfqpoint{2.262610in}{1.664881in}}{\pgfqpoint{2.254710in}{1.668153in}}{\pgfqpoint{2.246474in}{1.668153in}}%
\pgfpathcurveto{\pgfqpoint{2.238238in}{1.668153in}}{\pgfqpoint{2.230337in}{1.664881in}}{\pgfqpoint{2.224514in}{1.659057in}}%
\pgfpathcurveto{\pgfqpoint{2.218690in}{1.653233in}}{\pgfqpoint{2.215417in}{1.645333in}}{\pgfqpoint{2.215417in}{1.637097in}}%
\pgfpathcurveto{\pgfqpoint{2.215417in}{1.628860in}}{\pgfqpoint{2.218690in}{1.620960in}}{\pgfqpoint{2.224514in}{1.615136in}}%
\pgfpathcurveto{\pgfqpoint{2.230337in}{1.609313in}}{\pgfqpoint{2.238238in}{1.606040in}}{\pgfqpoint{2.246474in}{1.606040in}}%
\pgfpathclose%
\pgfusepath{stroke,fill}%
\end{pgfscope}%
\begin{pgfscope}%
\pgfpathrectangle{\pgfqpoint{0.100000in}{0.212622in}}{\pgfqpoint{3.696000in}{3.696000in}}%
\pgfusepath{clip}%
\pgfsetbuttcap%
\pgfsetroundjoin%
\definecolor{currentfill}{rgb}{0.121569,0.466667,0.705882}%
\pgfsetfillcolor{currentfill}%
\pgfsetfillopacity{0.788647}%
\pgfsetlinewidth{1.003750pt}%
\definecolor{currentstroke}{rgb}{0.121569,0.466667,0.705882}%
\pgfsetstrokecolor{currentstroke}%
\pgfsetstrokeopacity{0.788647}%
\pgfsetdash{}{0pt}%
\pgfpathmoveto{\pgfqpoint{2.247000in}{1.604686in}}%
\pgfpathcurveto{\pgfqpoint{2.255236in}{1.604686in}}{\pgfqpoint{2.263136in}{1.607958in}}{\pgfqpoint{2.268960in}{1.613782in}}%
\pgfpathcurveto{\pgfqpoint{2.274784in}{1.619606in}}{\pgfqpoint{2.278056in}{1.627506in}}{\pgfqpoint{2.278056in}{1.635742in}}%
\pgfpathcurveto{\pgfqpoint{2.278056in}{1.643978in}}{\pgfqpoint{2.274784in}{1.651878in}}{\pgfqpoint{2.268960in}{1.657702in}}%
\pgfpathcurveto{\pgfqpoint{2.263136in}{1.663526in}}{\pgfqpoint{2.255236in}{1.666799in}}{\pgfqpoint{2.247000in}{1.666799in}}%
\pgfpathcurveto{\pgfqpoint{2.238763in}{1.666799in}}{\pgfqpoint{2.230863in}{1.663526in}}{\pgfqpoint{2.225040in}{1.657702in}}%
\pgfpathcurveto{\pgfqpoint{2.219216in}{1.651878in}}{\pgfqpoint{2.215943in}{1.643978in}}{\pgfqpoint{2.215943in}{1.635742in}}%
\pgfpathcurveto{\pgfqpoint{2.215943in}{1.627506in}}{\pgfqpoint{2.219216in}{1.619606in}}{\pgfqpoint{2.225040in}{1.613782in}}%
\pgfpathcurveto{\pgfqpoint{2.230863in}{1.607958in}}{\pgfqpoint{2.238763in}{1.604686in}}{\pgfqpoint{2.247000in}{1.604686in}}%
\pgfpathclose%
\pgfusepath{stroke,fill}%
\end{pgfscope}%
\begin{pgfscope}%
\pgfpathrectangle{\pgfqpoint{0.100000in}{0.212622in}}{\pgfqpoint{3.696000in}{3.696000in}}%
\pgfusepath{clip}%
\pgfsetbuttcap%
\pgfsetroundjoin%
\definecolor{currentfill}{rgb}{0.121569,0.466667,0.705882}%
\pgfsetfillcolor{currentfill}%
\pgfsetfillopacity{0.789365}%
\pgfsetlinewidth{1.003750pt}%
\definecolor{currentstroke}{rgb}{0.121569,0.466667,0.705882}%
\pgfsetstrokecolor{currentstroke}%
\pgfsetstrokeopacity{0.789365}%
\pgfsetdash{}{0pt}%
\pgfpathmoveto{\pgfqpoint{2.247671in}{1.603922in}}%
\pgfpathcurveto{\pgfqpoint{2.255907in}{1.603922in}}{\pgfqpoint{2.263807in}{1.607194in}}{\pgfqpoint{2.269631in}{1.613018in}}%
\pgfpathcurveto{\pgfqpoint{2.275455in}{1.618842in}}{\pgfqpoint{2.278727in}{1.626742in}}{\pgfqpoint{2.278727in}{1.634978in}}%
\pgfpathcurveto{\pgfqpoint{2.278727in}{1.643215in}}{\pgfqpoint{2.275455in}{1.651115in}}{\pgfqpoint{2.269631in}{1.656939in}}%
\pgfpathcurveto{\pgfqpoint{2.263807in}{1.662763in}}{\pgfqpoint{2.255907in}{1.666035in}}{\pgfqpoint{2.247671in}{1.666035in}}%
\pgfpathcurveto{\pgfqpoint{2.239435in}{1.666035in}}{\pgfqpoint{2.231534in}{1.662763in}}{\pgfqpoint{2.225711in}{1.656939in}}%
\pgfpathcurveto{\pgfqpoint{2.219887in}{1.651115in}}{\pgfqpoint{2.216614in}{1.643215in}}{\pgfqpoint{2.216614in}{1.634978in}}%
\pgfpathcurveto{\pgfqpoint{2.216614in}{1.626742in}}{\pgfqpoint{2.219887in}{1.618842in}}{\pgfqpoint{2.225711in}{1.613018in}}%
\pgfpathcurveto{\pgfqpoint{2.231534in}{1.607194in}}{\pgfqpoint{2.239435in}{1.603922in}}{\pgfqpoint{2.247671in}{1.603922in}}%
\pgfpathclose%
\pgfusepath{stroke,fill}%
\end{pgfscope}%
\begin{pgfscope}%
\pgfpathrectangle{\pgfqpoint{0.100000in}{0.212622in}}{\pgfqpoint{3.696000in}{3.696000in}}%
\pgfusepath{clip}%
\pgfsetbuttcap%
\pgfsetroundjoin%
\definecolor{currentfill}{rgb}{0.121569,0.466667,0.705882}%
\pgfsetfillcolor{currentfill}%
\pgfsetfillopacity{0.790709}%
\pgfsetlinewidth{1.003750pt}%
\definecolor{currentstroke}{rgb}{0.121569,0.466667,0.705882}%
\pgfsetstrokecolor{currentstroke}%
\pgfsetstrokeopacity{0.790709}%
\pgfsetdash{}{0pt}%
\pgfpathmoveto{\pgfqpoint{2.248318in}{1.603624in}}%
\pgfpathcurveto{\pgfqpoint{2.256554in}{1.603624in}}{\pgfqpoint{2.264454in}{1.606896in}}{\pgfqpoint{2.270278in}{1.612720in}}%
\pgfpathcurveto{\pgfqpoint{2.276102in}{1.618544in}}{\pgfqpoint{2.279374in}{1.626444in}}{\pgfqpoint{2.279374in}{1.634680in}}%
\pgfpathcurveto{\pgfqpoint{2.279374in}{1.642916in}}{\pgfqpoint{2.276102in}{1.650816in}}{\pgfqpoint{2.270278in}{1.656640in}}%
\pgfpathcurveto{\pgfqpoint{2.264454in}{1.662464in}}{\pgfqpoint{2.256554in}{1.665737in}}{\pgfqpoint{2.248318in}{1.665737in}}%
\pgfpathcurveto{\pgfqpoint{2.240081in}{1.665737in}}{\pgfqpoint{2.232181in}{1.662464in}}{\pgfqpoint{2.226357in}{1.656640in}}%
\pgfpathcurveto{\pgfqpoint{2.220533in}{1.650816in}}{\pgfqpoint{2.217261in}{1.642916in}}{\pgfqpoint{2.217261in}{1.634680in}}%
\pgfpathcurveto{\pgfqpoint{2.217261in}{1.626444in}}{\pgfqpoint{2.220533in}{1.618544in}}{\pgfqpoint{2.226357in}{1.612720in}}%
\pgfpathcurveto{\pgfqpoint{2.232181in}{1.606896in}}{\pgfqpoint{2.240081in}{1.603624in}}{\pgfqpoint{2.248318in}{1.603624in}}%
\pgfpathclose%
\pgfusepath{stroke,fill}%
\end{pgfscope}%
\begin{pgfscope}%
\pgfpathrectangle{\pgfqpoint{0.100000in}{0.212622in}}{\pgfqpoint{3.696000in}{3.696000in}}%
\pgfusepath{clip}%
\pgfsetbuttcap%
\pgfsetroundjoin%
\definecolor{currentfill}{rgb}{0.121569,0.466667,0.705882}%
\pgfsetfillcolor{currentfill}%
\pgfsetfillopacity{0.792626}%
\pgfsetlinewidth{1.003750pt}%
\definecolor{currentstroke}{rgb}{0.121569,0.466667,0.705882}%
\pgfsetstrokecolor{currentstroke}%
\pgfsetstrokeopacity{0.792626}%
\pgfsetdash{}{0pt}%
\pgfpathmoveto{\pgfqpoint{2.249716in}{1.601333in}}%
\pgfpathcurveto{\pgfqpoint{2.257952in}{1.601333in}}{\pgfqpoint{2.265852in}{1.604605in}}{\pgfqpoint{2.271676in}{1.610429in}}%
\pgfpathcurveto{\pgfqpoint{2.277500in}{1.616253in}}{\pgfqpoint{2.280772in}{1.624153in}}{\pgfqpoint{2.280772in}{1.632389in}}%
\pgfpathcurveto{\pgfqpoint{2.280772in}{1.640626in}}{\pgfqpoint{2.277500in}{1.648526in}}{\pgfqpoint{2.271676in}{1.654350in}}%
\pgfpathcurveto{\pgfqpoint{2.265852in}{1.660174in}}{\pgfqpoint{2.257952in}{1.663446in}}{\pgfqpoint{2.249716in}{1.663446in}}%
\pgfpathcurveto{\pgfqpoint{2.241479in}{1.663446in}}{\pgfqpoint{2.233579in}{1.660174in}}{\pgfqpoint{2.227755in}{1.654350in}}%
\pgfpathcurveto{\pgfqpoint{2.221932in}{1.648526in}}{\pgfqpoint{2.218659in}{1.640626in}}{\pgfqpoint{2.218659in}{1.632389in}}%
\pgfpathcurveto{\pgfqpoint{2.218659in}{1.624153in}}{\pgfqpoint{2.221932in}{1.616253in}}{\pgfqpoint{2.227755in}{1.610429in}}%
\pgfpathcurveto{\pgfqpoint{2.233579in}{1.604605in}}{\pgfqpoint{2.241479in}{1.601333in}}{\pgfqpoint{2.249716in}{1.601333in}}%
\pgfpathclose%
\pgfusepath{stroke,fill}%
\end{pgfscope}%
\begin{pgfscope}%
\pgfpathrectangle{\pgfqpoint{0.100000in}{0.212622in}}{\pgfqpoint{3.696000in}{3.696000in}}%
\pgfusepath{clip}%
\pgfsetbuttcap%
\pgfsetroundjoin%
\definecolor{currentfill}{rgb}{0.121569,0.466667,0.705882}%
\pgfsetfillcolor{currentfill}%
\pgfsetfillopacity{0.793575}%
\pgfsetlinewidth{1.003750pt}%
\definecolor{currentstroke}{rgb}{0.121569,0.466667,0.705882}%
\pgfsetstrokecolor{currentstroke}%
\pgfsetstrokeopacity{0.793575}%
\pgfsetdash{}{0pt}%
\pgfpathmoveto{\pgfqpoint{2.250360in}{1.599347in}}%
\pgfpathcurveto{\pgfqpoint{2.258596in}{1.599347in}}{\pgfqpoint{2.266496in}{1.602619in}}{\pgfqpoint{2.272320in}{1.608443in}}%
\pgfpathcurveto{\pgfqpoint{2.278144in}{1.614267in}}{\pgfqpoint{2.281416in}{1.622167in}}{\pgfqpoint{2.281416in}{1.630403in}}%
\pgfpathcurveto{\pgfqpoint{2.281416in}{1.638639in}}{\pgfqpoint{2.278144in}{1.646540in}}{\pgfqpoint{2.272320in}{1.652363in}}%
\pgfpathcurveto{\pgfqpoint{2.266496in}{1.658187in}}{\pgfqpoint{2.258596in}{1.661460in}}{\pgfqpoint{2.250360in}{1.661460in}}%
\pgfpathcurveto{\pgfqpoint{2.242124in}{1.661460in}}{\pgfqpoint{2.234224in}{1.658187in}}{\pgfqpoint{2.228400in}{1.652363in}}%
\pgfpathcurveto{\pgfqpoint{2.222576in}{1.646540in}}{\pgfqpoint{2.219303in}{1.638639in}}{\pgfqpoint{2.219303in}{1.630403in}}%
\pgfpathcurveto{\pgfqpoint{2.219303in}{1.622167in}}{\pgfqpoint{2.222576in}{1.614267in}}{\pgfqpoint{2.228400in}{1.608443in}}%
\pgfpathcurveto{\pgfqpoint{2.234224in}{1.602619in}}{\pgfqpoint{2.242124in}{1.599347in}}{\pgfqpoint{2.250360in}{1.599347in}}%
\pgfpathclose%
\pgfusepath{stroke,fill}%
\end{pgfscope}%
\begin{pgfscope}%
\pgfpathrectangle{\pgfqpoint{0.100000in}{0.212622in}}{\pgfqpoint{3.696000in}{3.696000in}}%
\pgfusepath{clip}%
\pgfsetbuttcap%
\pgfsetroundjoin%
\definecolor{currentfill}{rgb}{0.121569,0.466667,0.705882}%
\pgfsetfillcolor{currentfill}%
\pgfsetfillopacity{0.795046}%
\pgfsetlinewidth{1.003750pt}%
\definecolor{currentstroke}{rgb}{0.121569,0.466667,0.705882}%
\pgfsetstrokecolor{currentstroke}%
\pgfsetstrokeopacity{0.795046}%
\pgfsetdash{}{0pt}%
\pgfpathmoveto{\pgfqpoint{2.251534in}{1.598331in}}%
\pgfpathcurveto{\pgfqpoint{2.259771in}{1.598331in}}{\pgfqpoint{2.267671in}{1.601603in}}{\pgfqpoint{2.273495in}{1.607427in}}%
\pgfpathcurveto{\pgfqpoint{2.279318in}{1.613251in}}{\pgfqpoint{2.282591in}{1.621151in}}{\pgfqpoint{2.282591in}{1.629388in}}%
\pgfpathcurveto{\pgfqpoint{2.282591in}{1.637624in}}{\pgfqpoint{2.279318in}{1.645524in}}{\pgfqpoint{2.273495in}{1.651348in}}%
\pgfpathcurveto{\pgfqpoint{2.267671in}{1.657172in}}{\pgfqpoint{2.259771in}{1.660444in}}{\pgfqpoint{2.251534in}{1.660444in}}%
\pgfpathcurveto{\pgfqpoint{2.243298in}{1.660444in}}{\pgfqpoint{2.235398in}{1.657172in}}{\pgfqpoint{2.229574in}{1.651348in}}%
\pgfpathcurveto{\pgfqpoint{2.223750in}{1.645524in}}{\pgfqpoint{2.220478in}{1.637624in}}{\pgfqpoint{2.220478in}{1.629388in}}%
\pgfpathcurveto{\pgfqpoint{2.220478in}{1.621151in}}{\pgfqpoint{2.223750in}{1.613251in}}{\pgfqpoint{2.229574in}{1.607427in}}%
\pgfpathcurveto{\pgfqpoint{2.235398in}{1.601603in}}{\pgfqpoint{2.243298in}{1.598331in}}{\pgfqpoint{2.251534in}{1.598331in}}%
\pgfpathclose%
\pgfusepath{stroke,fill}%
\end{pgfscope}%
\begin{pgfscope}%
\pgfpathrectangle{\pgfqpoint{0.100000in}{0.212622in}}{\pgfqpoint{3.696000in}{3.696000in}}%
\pgfusepath{clip}%
\pgfsetbuttcap%
\pgfsetroundjoin%
\definecolor{currentfill}{rgb}{0.121569,0.466667,0.705882}%
\pgfsetfillcolor{currentfill}%
\pgfsetfillopacity{0.796872}%
\pgfsetlinewidth{1.003750pt}%
\definecolor{currentstroke}{rgb}{0.121569,0.466667,0.705882}%
\pgfsetstrokecolor{currentstroke}%
\pgfsetstrokeopacity{0.796872}%
\pgfsetdash{}{0pt}%
\pgfpathmoveto{\pgfqpoint{2.252661in}{1.597634in}}%
\pgfpathcurveto{\pgfqpoint{2.260897in}{1.597634in}}{\pgfqpoint{2.268797in}{1.600907in}}{\pgfqpoint{2.274621in}{1.606730in}}%
\pgfpathcurveto{\pgfqpoint{2.280445in}{1.612554in}}{\pgfqpoint{2.283717in}{1.620454in}}{\pgfqpoint{2.283717in}{1.628691in}}%
\pgfpathcurveto{\pgfqpoint{2.283717in}{1.636927in}}{\pgfqpoint{2.280445in}{1.644827in}}{\pgfqpoint{2.274621in}{1.650651in}}%
\pgfpathcurveto{\pgfqpoint{2.268797in}{1.656475in}}{\pgfqpoint{2.260897in}{1.659747in}}{\pgfqpoint{2.252661in}{1.659747in}}%
\pgfpathcurveto{\pgfqpoint{2.244425in}{1.659747in}}{\pgfqpoint{2.236525in}{1.656475in}}{\pgfqpoint{2.230701in}{1.650651in}}%
\pgfpathcurveto{\pgfqpoint{2.224877in}{1.644827in}}{\pgfqpoint{2.221604in}{1.636927in}}{\pgfqpoint{2.221604in}{1.628691in}}%
\pgfpathcurveto{\pgfqpoint{2.221604in}{1.620454in}}{\pgfqpoint{2.224877in}{1.612554in}}{\pgfqpoint{2.230701in}{1.606730in}}%
\pgfpathcurveto{\pgfqpoint{2.236525in}{1.600907in}}{\pgfqpoint{2.244425in}{1.597634in}}{\pgfqpoint{2.252661in}{1.597634in}}%
\pgfpathclose%
\pgfusepath{stroke,fill}%
\end{pgfscope}%
\begin{pgfscope}%
\pgfpathrectangle{\pgfqpoint{0.100000in}{0.212622in}}{\pgfqpoint{3.696000in}{3.696000in}}%
\pgfusepath{clip}%
\pgfsetbuttcap%
\pgfsetroundjoin%
\definecolor{currentfill}{rgb}{0.121569,0.466667,0.705882}%
\pgfsetfillcolor{currentfill}%
\pgfsetfillopacity{0.798901}%
\pgfsetlinewidth{1.003750pt}%
\definecolor{currentstroke}{rgb}{0.121569,0.466667,0.705882}%
\pgfsetstrokecolor{currentstroke}%
\pgfsetstrokeopacity{0.798901}%
\pgfsetdash{}{0pt}%
\pgfpathmoveto{\pgfqpoint{2.254766in}{1.593755in}}%
\pgfpathcurveto{\pgfqpoint{2.263002in}{1.593755in}}{\pgfqpoint{2.270902in}{1.597028in}}{\pgfqpoint{2.276726in}{1.602852in}}%
\pgfpathcurveto{\pgfqpoint{2.282550in}{1.608676in}}{\pgfqpoint{2.285822in}{1.616576in}}{\pgfqpoint{2.285822in}{1.624812in}}%
\pgfpathcurveto{\pgfqpoint{2.285822in}{1.633048in}}{\pgfqpoint{2.282550in}{1.640948in}}{\pgfqpoint{2.276726in}{1.646772in}}%
\pgfpathcurveto{\pgfqpoint{2.270902in}{1.652596in}}{\pgfqpoint{2.263002in}{1.655868in}}{\pgfqpoint{2.254766in}{1.655868in}}%
\pgfpathcurveto{\pgfqpoint{2.246529in}{1.655868in}}{\pgfqpoint{2.238629in}{1.652596in}}{\pgfqpoint{2.232805in}{1.646772in}}%
\pgfpathcurveto{\pgfqpoint{2.226982in}{1.640948in}}{\pgfqpoint{2.223709in}{1.633048in}}{\pgfqpoint{2.223709in}{1.624812in}}%
\pgfpathcurveto{\pgfqpoint{2.223709in}{1.616576in}}{\pgfqpoint{2.226982in}{1.608676in}}{\pgfqpoint{2.232805in}{1.602852in}}%
\pgfpathcurveto{\pgfqpoint{2.238629in}{1.597028in}}{\pgfqpoint{2.246529in}{1.593755in}}{\pgfqpoint{2.254766in}{1.593755in}}%
\pgfpathclose%
\pgfusepath{stroke,fill}%
\end{pgfscope}%
\begin{pgfscope}%
\pgfpathrectangle{\pgfqpoint{0.100000in}{0.212622in}}{\pgfqpoint{3.696000in}{3.696000in}}%
\pgfusepath{clip}%
\pgfsetbuttcap%
\pgfsetroundjoin%
\definecolor{currentfill}{rgb}{0.121569,0.466667,0.705882}%
\pgfsetfillcolor{currentfill}%
\pgfsetfillopacity{0.800779}%
\pgfsetlinewidth{1.003750pt}%
\definecolor{currentstroke}{rgb}{0.121569,0.466667,0.705882}%
\pgfsetstrokecolor{currentstroke}%
\pgfsetstrokeopacity{0.800779}%
\pgfsetdash{}{0pt}%
\pgfpathmoveto{\pgfqpoint{2.257206in}{1.586747in}}%
\pgfpathcurveto{\pgfqpoint{2.265442in}{1.586747in}}{\pgfqpoint{2.273342in}{1.590020in}}{\pgfqpoint{2.279166in}{1.595844in}}%
\pgfpathcurveto{\pgfqpoint{2.284990in}{1.601667in}}{\pgfqpoint{2.288262in}{1.609568in}}{\pgfqpoint{2.288262in}{1.617804in}}%
\pgfpathcurveto{\pgfqpoint{2.288262in}{1.626040in}}{\pgfqpoint{2.284990in}{1.633940in}}{\pgfqpoint{2.279166in}{1.639764in}}%
\pgfpathcurveto{\pgfqpoint{2.273342in}{1.645588in}}{\pgfqpoint{2.265442in}{1.648860in}}{\pgfqpoint{2.257206in}{1.648860in}}%
\pgfpathcurveto{\pgfqpoint{2.248969in}{1.648860in}}{\pgfqpoint{2.241069in}{1.645588in}}{\pgfqpoint{2.235245in}{1.639764in}}%
\pgfpathcurveto{\pgfqpoint{2.229422in}{1.633940in}}{\pgfqpoint{2.226149in}{1.626040in}}{\pgfqpoint{2.226149in}{1.617804in}}%
\pgfpathcurveto{\pgfqpoint{2.226149in}{1.609568in}}{\pgfqpoint{2.229422in}{1.601667in}}{\pgfqpoint{2.235245in}{1.595844in}}%
\pgfpathcurveto{\pgfqpoint{2.241069in}{1.590020in}}{\pgfqpoint{2.248969in}{1.586747in}}{\pgfqpoint{2.257206in}{1.586747in}}%
\pgfpathclose%
\pgfusepath{stroke,fill}%
\end{pgfscope}%
\begin{pgfscope}%
\pgfpathrectangle{\pgfqpoint{0.100000in}{0.212622in}}{\pgfqpoint{3.696000in}{3.696000in}}%
\pgfusepath{clip}%
\pgfsetbuttcap%
\pgfsetroundjoin%
\definecolor{currentfill}{rgb}{0.121569,0.466667,0.705882}%
\pgfsetfillcolor{currentfill}%
\pgfsetfillopacity{0.803137}%
\pgfsetlinewidth{1.003750pt}%
\definecolor{currentstroke}{rgb}{0.121569,0.466667,0.705882}%
\pgfsetstrokecolor{currentstroke}%
\pgfsetstrokeopacity{0.803137}%
\pgfsetdash{}{0pt}%
\pgfpathmoveto{\pgfqpoint{2.259077in}{1.580863in}}%
\pgfpathcurveto{\pgfqpoint{2.267314in}{1.580863in}}{\pgfqpoint{2.275214in}{1.584136in}}{\pgfqpoint{2.281038in}{1.589960in}}%
\pgfpathcurveto{\pgfqpoint{2.286861in}{1.595784in}}{\pgfqpoint{2.290134in}{1.603684in}}{\pgfqpoint{2.290134in}{1.611920in}}%
\pgfpathcurveto{\pgfqpoint{2.290134in}{1.620156in}}{\pgfqpoint{2.286861in}{1.628056in}}{\pgfqpoint{2.281038in}{1.633880in}}%
\pgfpathcurveto{\pgfqpoint{2.275214in}{1.639704in}}{\pgfqpoint{2.267314in}{1.642976in}}{\pgfqpoint{2.259077in}{1.642976in}}%
\pgfpathcurveto{\pgfqpoint{2.250841in}{1.642976in}}{\pgfqpoint{2.242941in}{1.639704in}}{\pgfqpoint{2.237117in}{1.633880in}}%
\pgfpathcurveto{\pgfqpoint{2.231293in}{1.628056in}}{\pgfqpoint{2.228021in}{1.620156in}}{\pgfqpoint{2.228021in}{1.611920in}}%
\pgfpathcurveto{\pgfqpoint{2.228021in}{1.603684in}}{\pgfqpoint{2.231293in}{1.595784in}}{\pgfqpoint{2.237117in}{1.589960in}}%
\pgfpathcurveto{\pgfqpoint{2.242941in}{1.584136in}}{\pgfqpoint{2.250841in}{1.580863in}}{\pgfqpoint{2.259077in}{1.580863in}}%
\pgfpathclose%
\pgfusepath{stroke,fill}%
\end{pgfscope}%
\begin{pgfscope}%
\pgfpathrectangle{\pgfqpoint{0.100000in}{0.212622in}}{\pgfqpoint{3.696000in}{3.696000in}}%
\pgfusepath{clip}%
\pgfsetbuttcap%
\pgfsetroundjoin%
\definecolor{currentfill}{rgb}{0.121569,0.466667,0.705882}%
\pgfsetfillcolor{currentfill}%
\pgfsetfillopacity{0.804846}%
\pgfsetlinewidth{1.003750pt}%
\definecolor{currentstroke}{rgb}{0.121569,0.466667,0.705882}%
\pgfsetstrokecolor{currentstroke}%
\pgfsetstrokeopacity{0.804846}%
\pgfsetdash{}{0pt}%
\pgfpathmoveto{\pgfqpoint{2.260306in}{1.580295in}}%
\pgfpathcurveto{\pgfqpoint{2.268543in}{1.580295in}}{\pgfqpoint{2.276443in}{1.583568in}}{\pgfqpoint{2.282267in}{1.589392in}}%
\pgfpathcurveto{\pgfqpoint{2.288091in}{1.595216in}}{\pgfqpoint{2.291363in}{1.603116in}}{\pgfqpoint{2.291363in}{1.611352in}}%
\pgfpathcurveto{\pgfqpoint{2.291363in}{1.619588in}}{\pgfqpoint{2.288091in}{1.627488in}}{\pgfqpoint{2.282267in}{1.633312in}}%
\pgfpathcurveto{\pgfqpoint{2.276443in}{1.639136in}}{\pgfqpoint{2.268543in}{1.642408in}}{\pgfqpoint{2.260306in}{1.642408in}}%
\pgfpathcurveto{\pgfqpoint{2.252070in}{1.642408in}}{\pgfqpoint{2.244170in}{1.639136in}}{\pgfqpoint{2.238346in}{1.633312in}}%
\pgfpathcurveto{\pgfqpoint{2.232522in}{1.627488in}}{\pgfqpoint{2.229250in}{1.619588in}}{\pgfqpoint{2.229250in}{1.611352in}}%
\pgfpathcurveto{\pgfqpoint{2.229250in}{1.603116in}}{\pgfqpoint{2.232522in}{1.595216in}}{\pgfqpoint{2.238346in}{1.589392in}}%
\pgfpathcurveto{\pgfqpoint{2.244170in}{1.583568in}}{\pgfqpoint{2.252070in}{1.580295in}}{\pgfqpoint{2.260306in}{1.580295in}}%
\pgfpathclose%
\pgfusepath{stroke,fill}%
\end{pgfscope}%
\begin{pgfscope}%
\pgfpathrectangle{\pgfqpoint{0.100000in}{0.212622in}}{\pgfqpoint{3.696000in}{3.696000in}}%
\pgfusepath{clip}%
\pgfsetbuttcap%
\pgfsetroundjoin%
\definecolor{currentfill}{rgb}{0.121569,0.466667,0.705882}%
\pgfsetfillcolor{currentfill}%
\pgfsetfillopacity{0.806804}%
\pgfsetlinewidth{1.003750pt}%
\definecolor{currentstroke}{rgb}{0.121569,0.466667,0.705882}%
\pgfsetstrokecolor{currentstroke}%
\pgfsetstrokeopacity{0.806804}%
\pgfsetdash{}{0pt}%
\pgfpathmoveto{\pgfqpoint{2.261932in}{1.576971in}}%
\pgfpathcurveto{\pgfqpoint{2.270169in}{1.576971in}}{\pgfqpoint{2.278069in}{1.580244in}}{\pgfqpoint{2.283893in}{1.586067in}}%
\pgfpathcurveto{\pgfqpoint{2.289717in}{1.591891in}}{\pgfqpoint{2.292989in}{1.599791in}}{\pgfqpoint{2.292989in}{1.608028in}}%
\pgfpathcurveto{\pgfqpoint{2.292989in}{1.616264in}}{\pgfqpoint{2.289717in}{1.624164in}}{\pgfqpoint{2.283893in}{1.629988in}}%
\pgfpathcurveto{\pgfqpoint{2.278069in}{1.635812in}}{\pgfqpoint{2.270169in}{1.639084in}}{\pgfqpoint{2.261932in}{1.639084in}}%
\pgfpathcurveto{\pgfqpoint{2.253696in}{1.639084in}}{\pgfqpoint{2.245796in}{1.635812in}}{\pgfqpoint{2.239972in}{1.629988in}}%
\pgfpathcurveto{\pgfqpoint{2.234148in}{1.624164in}}{\pgfqpoint{2.230876in}{1.616264in}}{\pgfqpoint{2.230876in}{1.608028in}}%
\pgfpathcurveto{\pgfqpoint{2.230876in}{1.599791in}}{\pgfqpoint{2.234148in}{1.591891in}}{\pgfqpoint{2.239972in}{1.586067in}}%
\pgfpathcurveto{\pgfqpoint{2.245796in}{1.580244in}}{\pgfqpoint{2.253696in}{1.576971in}}{\pgfqpoint{2.261932in}{1.576971in}}%
\pgfpathclose%
\pgfusepath{stroke,fill}%
\end{pgfscope}%
\begin{pgfscope}%
\pgfpathrectangle{\pgfqpoint{0.100000in}{0.212622in}}{\pgfqpoint{3.696000in}{3.696000in}}%
\pgfusepath{clip}%
\pgfsetbuttcap%
\pgfsetroundjoin%
\definecolor{currentfill}{rgb}{0.121569,0.466667,0.705882}%
\pgfsetfillcolor{currentfill}%
\pgfsetfillopacity{0.808773}%
\pgfsetlinewidth{1.003750pt}%
\definecolor{currentstroke}{rgb}{0.121569,0.466667,0.705882}%
\pgfsetstrokecolor{currentstroke}%
\pgfsetstrokeopacity{0.808773}%
\pgfsetdash{}{0pt}%
\pgfpathmoveto{\pgfqpoint{2.264102in}{1.572549in}}%
\pgfpathcurveto{\pgfqpoint{2.272339in}{1.572549in}}{\pgfqpoint{2.280239in}{1.575821in}}{\pgfqpoint{2.286063in}{1.581645in}}%
\pgfpathcurveto{\pgfqpoint{2.291887in}{1.587469in}}{\pgfqpoint{2.295159in}{1.595369in}}{\pgfqpoint{2.295159in}{1.603605in}}%
\pgfpathcurveto{\pgfqpoint{2.295159in}{1.611841in}}{\pgfqpoint{2.291887in}{1.619741in}}{\pgfqpoint{2.286063in}{1.625565in}}%
\pgfpathcurveto{\pgfqpoint{2.280239in}{1.631389in}}{\pgfqpoint{2.272339in}{1.634662in}}{\pgfqpoint{2.264102in}{1.634662in}}%
\pgfpathcurveto{\pgfqpoint{2.255866in}{1.634662in}}{\pgfqpoint{2.247966in}{1.631389in}}{\pgfqpoint{2.242142in}{1.625565in}}%
\pgfpathcurveto{\pgfqpoint{2.236318in}{1.619741in}}{\pgfqpoint{2.233046in}{1.611841in}}{\pgfqpoint{2.233046in}{1.603605in}}%
\pgfpathcurveto{\pgfqpoint{2.233046in}{1.595369in}}{\pgfqpoint{2.236318in}{1.587469in}}{\pgfqpoint{2.242142in}{1.581645in}}%
\pgfpathcurveto{\pgfqpoint{2.247966in}{1.575821in}}{\pgfqpoint{2.255866in}{1.572549in}}{\pgfqpoint{2.264102in}{1.572549in}}%
\pgfpathclose%
\pgfusepath{stroke,fill}%
\end{pgfscope}%
\begin{pgfscope}%
\pgfpathrectangle{\pgfqpoint{0.100000in}{0.212622in}}{\pgfqpoint{3.696000in}{3.696000in}}%
\pgfusepath{clip}%
\pgfsetbuttcap%
\pgfsetroundjoin%
\definecolor{currentfill}{rgb}{0.121569,0.466667,0.705882}%
\pgfsetfillcolor{currentfill}%
\pgfsetfillopacity{0.809825}%
\pgfsetlinewidth{1.003750pt}%
\definecolor{currentstroke}{rgb}{0.121569,0.466667,0.705882}%
\pgfsetstrokecolor{currentstroke}%
\pgfsetstrokeopacity{0.809825}%
\pgfsetdash{}{0pt}%
\pgfpathmoveto{\pgfqpoint{2.264977in}{1.569727in}}%
\pgfpathcurveto{\pgfqpoint{2.273214in}{1.569727in}}{\pgfqpoint{2.281114in}{1.572999in}}{\pgfqpoint{2.286938in}{1.578823in}}%
\pgfpathcurveto{\pgfqpoint{2.292762in}{1.584647in}}{\pgfqpoint{2.296034in}{1.592547in}}{\pgfqpoint{2.296034in}{1.600783in}}%
\pgfpathcurveto{\pgfqpoint{2.296034in}{1.609020in}}{\pgfqpoint{2.292762in}{1.616920in}}{\pgfqpoint{2.286938in}{1.622744in}}%
\pgfpathcurveto{\pgfqpoint{2.281114in}{1.628568in}}{\pgfqpoint{2.273214in}{1.631840in}}{\pgfqpoint{2.264977in}{1.631840in}}%
\pgfpathcurveto{\pgfqpoint{2.256741in}{1.631840in}}{\pgfqpoint{2.248841in}{1.628568in}}{\pgfqpoint{2.243017in}{1.622744in}}%
\pgfpathcurveto{\pgfqpoint{2.237193in}{1.616920in}}{\pgfqpoint{2.233921in}{1.609020in}}{\pgfqpoint{2.233921in}{1.600783in}}%
\pgfpathcurveto{\pgfqpoint{2.233921in}{1.592547in}}{\pgfqpoint{2.237193in}{1.584647in}}{\pgfqpoint{2.243017in}{1.578823in}}%
\pgfpathcurveto{\pgfqpoint{2.248841in}{1.572999in}}{\pgfqpoint{2.256741in}{1.569727in}}{\pgfqpoint{2.264977in}{1.569727in}}%
\pgfpathclose%
\pgfusepath{stroke,fill}%
\end{pgfscope}%
\begin{pgfscope}%
\pgfpathrectangle{\pgfqpoint{0.100000in}{0.212622in}}{\pgfqpoint{3.696000in}{3.696000in}}%
\pgfusepath{clip}%
\pgfsetbuttcap%
\pgfsetroundjoin%
\definecolor{currentfill}{rgb}{0.121569,0.466667,0.705882}%
\pgfsetfillcolor{currentfill}%
\pgfsetfillopacity{0.810573}%
\pgfsetlinewidth{1.003750pt}%
\definecolor{currentstroke}{rgb}{0.121569,0.466667,0.705882}%
\pgfsetstrokecolor{currentstroke}%
\pgfsetstrokeopacity{0.810573}%
\pgfsetdash{}{0pt}%
\pgfpathmoveto{\pgfqpoint{2.265482in}{1.569235in}}%
\pgfpathcurveto{\pgfqpoint{2.273718in}{1.569235in}}{\pgfqpoint{2.281618in}{1.572507in}}{\pgfqpoint{2.287442in}{1.578331in}}%
\pgfpathcurveto{\pgfqpoint{2.293266in}{1.584155in}}{\pgfqpoint{2.296538in}{1.592055in}}{\pgfqpoint{2.296538in}{1.600291in}}%
\pgfpathcurveto{\pgfqpoint{2.296538in}{1.608527in}}{\pgfqpoint{2.293266in}{1.616427in}}{\pgfqpoint{2.287442in}{1.622251in}}%
\pgfpathcurveto{\pgfqpoint{2.281618in}{1.628075in}}{\pgfqpoint{2.273718in}{1.631348in}}{\pgfqpoint{2.265482in}{1.631348in}}%
\pgfpathcurveto{\pgfqpoint{2.257245in}{1.631348in}}{\pgfqpoint{2.249345in}{1.628075in}}{\pgfqpoint{2.243521in}{1.622251in}}%
\pgfpathcurveto{\pgfqpoint{2.237697in}{1.616427in}}{\pgfqpoint{2.234425in}{1.608527in}}{\pgfqpoint{2.234425in}{1.600291in}}%
\pgfpathcurveto{\pgfqpoint{2.234425in}{1.592055in}}{\pgfqpoint{2.237697in}{1.584155in}}{\pgfqpoint{2.243521in}{1.578331in}}%
\pgfpathcurveto{\pgfqpoint{2.249345in}{1.572507in}}{\pgfqpoint{2.257245in}{1.569235in}}{\pgfqpoint{2.265482in}{1.569235in}}%
\pgfpathclose%
\pgfusepath{stroke,fill}%
\end{pgfscope}%
\begin{pgfscope}%
\pgfpathrectangle{\pgfqpoint{0.100000in}{0.212622in}}{\pgfqpoint{3.696000in}{3.696000in}}%
\pgfusepath{clip}%
\pgfsetbuttcap%
\pgfsetroundjoin%
\definecolor{currentfill}{rgb}{0.121569,0.466667,0.705882}%
\pgfsetfillcolor{currentfill}%
\pgfsetfillopacity{0.811766}%
\pgfsetlinewidth{1.003750pt}%
\definecolor{currentstroke}{rgb}{0.121569,0.466667,0.705882}%
\pgfsetstrokecolor{currentstroke}%
\pgfsetstrokeopacity{0.811766}%
\pgfsetdash{}{0pt}%
\pgfpathmoveto{\pgfqpoint{2.266200in}{1.568776in}}%
\pgfpathcurveto{\pgfqpoint{2.274436in}{1.568776in}}{\pgfqpoint{2.282336in}{1.572048in}}{\pgfqpoint{2.288160in}{1.577872in}}%
\pgfpathcurveto{\pgfqpoint{2.293984in}{1.583696in}}{\pgfqpoint{2.297256in}{1.591596in}}{\pgfqpoint{2.297256in}{1.599833in}}%
\pgfpathcurveto{\pgfqpoint{2.297256in}{1.608069in}}{\pgfqpoint{2.293984in}{1.615969in}}{\pgfqpoint{2.288160in}{1.621793in}}%
\pgfpathcurveto{\pgfqpoint{2.282336in}{1.627617in}}{\pgfqpoint{2.274436in}{1.630889in}}{\pgfqpoint{2.266200in}{1.630889in}}%
\pgfpathcurveto{\pgfqpoint{2.257964in}{1.630889in}}{\pgfqpoint{2.250064in}{1.627617in}}{\pgfqpoint{2.244240in}{1.621793in}}%
\pgfpathcurveto{\pgfqpoint{2.238416in}{1.615969in}}{\pgfqpoint{2.235143in}{1.608069in}}{\pgfqpoint{2.235143in}{1.599833in}}%
\pgfpathcurveto{\pgfqpoint{2.235143in}{1.591596in}}{\pgfqpoint{2.238416in}{1.583696in}}{\pgfqpoint{2.244240in}{1.577872in}}%
\pgfpathcurveto{\pgfqpoint{2.250064in}{1.572048in}}{\pgfqpoint{2.257964in}{1.568776in}}{\pgfqpoint{2.266200in}{1.568776in}}%
\pgfpathclose%
\pgfusepath{stroke,fill}%
\end{pgfscope}%
\begin{pgfscope}%
\pgfpathrectangle{\pgfqpoint{0.100000in}{0.212622in}}{\pgfqpoint{3.696000in}{3.696000in}}%
\pgfusepath{clip}%
\pgfsetbuttcap%
\pgfsetroundjoin%
\definecolor{currentfill}{rgb}{0.121569,0.466667,0.705882}%
\pgfsetfillcolor{currentfill}%
\pgfsetfillopacity{0.812960}%
\pgfsetlinewidth{1.003750pt}%
\definecolor{currentstroke}{rgb}{0.121569,0.466667,0.705882}%
\pgfsetstrokecolor{currentstroke}%
\pgfsetstrokeopacity{0.812960}%
\pgfsetdash{}{0pt}%
\pgfpathmoveto{\pgfqpoint{2.267190in}{1.567176in}}%
\pgfpathcurveto{\pgfqpoint{2.275427in}{1.567176in}}{\pgfqpoint{2.283327in}{1.570448in}}{\pgfqpoint{2.289151in}{1.576272in}}%
\pgfpathcurveto{\pgfqpoint{2.294975in}{1.582096in}}{\pgfqpoint{2.298247in}{1.589996in}}{\pgfqpoint{2.298247in}{1.598232in}}%
\pgfpathcurveto{\pgfqpoint{2.298247in}{1.606469in}}{\pgfqpoint{2.294975in}{1.614369in}}{\pgfqpoint{2.289151in}{1.620193in}}%
\pgfpathcurveto{\pgfqpoint{2.283327in}{1.626017in}}{\pgfqpoint{2.275427in}{1.629289in}}{\pgfqpoint{2.267190in}{1.629289in}}%
\pgfpathcurveto{\pgfqpoint{2.258954in}{1.629289in}}{\pgfqpoint{2.251054in}{1.626017in}}{\pgfqpoint{2.245230in}{1.620193in}}%
\pgfpathcurveto{\pgfqpoint{2.239406in}{1.614369in}}{\pgfqpoint{2.236134in}{1.606469in}}{\pgfqpoint{2.236134in}{1.598232in}}%
\pgfpathcurveto{\pgfqpoint{2.236134in}{1.589996in}}{\pgfqpoint{2.239406in}{1.582096in}}{\pgfqpoint{2.245230in}{1.576272in}}%
\pgfpathcurveto{\pgfqpoint{2.251054in}{1.570448in}}{\pgfqpoint{2.258954in}{1.567176in}}{\pgfqpoint{2.267190in}{1.567176in}}%
\pgfpathclose%
\pgfusepath{stroke,fill}%
\end{pgfscope}%
\begin{pgfscope}%
\pgfpathrectangle{\pgfqpoint{0.100000in}{0.212622in}}{\pgfqpoint{3.696000in}{3.696000in}}%
\pgfusepath{clip}%
\pgfsetbuttcap%
\pgfsetroundjoin%
\definecolor{currentfill}{rgb}{0.121569,0.466667,0.705882}%
\pgfsetfillcolor{currentfill}%
\pgfsetfillopacity{0.814156}%
\pgfsetlinewidth{1.003750pt}%
\definecolor{currentstroke}{rgb}{0.121569,0.466667,0.705882}%
\pgfsetstrokecolor{currentstroke}%
\pgfsetstrokeopacity{0.814156}%
\pgfsetdash{}{0pt}%
\pgfpathmoveto{\pgfqpoint{2.268223in}{1.563959in}}%
\pgfpathcurveto{\pgfqpoint{2.276459in}{1.563959in}}{\pgfqpoint{2.284360in}{1.567232in}}{\pgfqpoint{2.290183in}{1.573056in}}%
\pgfpathcurveto{\pgfqpoint{2.296007in}{1.578880in}}{\pgfqpoint{2.299280in}{1.586780in}}{\pgfqpoint{2.299280in}{1.595016in}}%
\pgfpathcurveto{\pgfqpoint{2.299280in}{1.603252in}}{\pgfqpoint{2.296007in}{1.611152in}}{\pgfqpoint{2.290183in}{1.616976in}}%
\pgfpathcurveto{\pgfqpoint{2.284360in}{1.622800in}}{\pgfqpoint{2.276459in}{1.626072in}}{\pgfqpoint{2.268223in}{1.626072in}}%
\pgfpathcurveto{\pgfqpoint{2.259987in}{1.626072in}}{\pgfqpoint{2.252087in}{1.622800in}}{\pgfqpoint{2.246263in}{1.616976in}}%
\pgfpathcurveto{\pgfqpoint{2.240439in}{1.611152in}}{\pgfqpoint{2.237167in}{1.603252in}}{\pgfqpoint{2.237167in}{1.595016in}}%
\pgfpathcurveto{\pgfqpoint{2.237167in}{1.586780in}}{\pgfqpoint{2.240439in}{1.578880in}}{\pgfqpoint{2.246263in}{1.573056in}}%
\pgfpathcurveto{\pgfqpoint{2.252087in}{1.567232in}}{\pgfqpoint{2.259987in}{1.563959in}}{\pgfqpoint{2.268223in}{1.563959in}}%
\pgfpathclose%
\pgfusepath{stroke,fill}%
\end{pgfscope}%
\begin{pgfscope}%
\pgfpathrectangle{\pgfqpoint{0.100000in}{0.212622in}}{\pgfqpoint{3.696000in}{3.696000in}}%
\pgfusepath{clip}%
\pgfsetbuttcap%
\pgfsetroundjoin%
\definecolor{currentfill}{rgb}{0.121569,0.466667,0.705882}%
\pgfsetfillcolor{currentfill}%
\pgfsetfillopacity{0.814916}%
\pgfsetlinewidth{1.003750pt}%
\definecolor{currentstroke}{rgb}{0.121569,0.466667,0.705882}%
\pgfsetstrokecolor{currentstroke}%
\pgfsetstrokeopacity{0.814916}%
\pgfsetdash{}{0pt}%
\pgfpathmoveto{\pgfqpoint{2.268962in}{1.562924in}}%
\pgfpathcurveto{\pgfqpoint{2.277198in}{1.562924in}}{\pgfqpoint{2.285098in}{1.566196in}}{\pgfqpoint{2.290922in}{1.572020in}}%
\pgfpathcurveto{\pgfqpoint{2.296746in}{1.577844in}}{\pgfqpoint{2.300018in}{1.585744in}}{\pgfqpoint{2.300018in}{1.593980in}}%
\pgfpathcurveto{\pgfqpoint{2.300018in}{1.602216in}}{\pgfqpoint{2.296746in}{1.610117in}}{\pgfqpoint{2.290922in}{1.615940in}}%
\pgfpathcurveto{\pgfqpoint{2.285098in}{1.621764in}}{\pgfqpoint{2.277198in}{1.625037in}}{\pgfqpoint{2.268962in}{1.625037in}}%
\pgfpathcurveto{\pgfqpoint{2.260725in}{1.625037in}}{\pgfqpoint{2.252825in}{1.621764in}}{\pgfqpoint{2.247002in}{1.615940in}}%
\pgfpathcurveto{\pgfqpoint{2.241178in}{1.610117in}}{\pgfqpoint{2.237905in}{1.602216in}}{\pgfqpoint{2.237905in}{1.593980in}}%
\pgfpathcurveto{\pgfqpoint{2.237905in}{1.585744in}}{\pgfqpoint{2.241178in}{1.577844in}}{\pgfqpoint{2.247002in}{1.572020in}}%
\pgfpathcurveto{\pgfqpoint{2.252825in}{1.566196in}}{\pgfqpoint{2.260725in}{1.562924in}}{\pgfqpoint{2.268962in}{1.562924in}}%
\pgfpathclose%
\pgfusepath{stroke,fill}%
\end{pgfscope}%
\begin{pgfscope}%
\pgfpathrectangle{\pgfqpoint{0.100000in}{0.212622in}}{\pgfqpoint{3.696000in}{3.696000in}}%
\pgfusepath{clip}%
\pgfsetbuttcap%
\pgfsetroundjoin%
\definecolor{currentfill}{rgb}{0.121569,0.466667,0.705882}%
\pgfsetfillcolor{currentfill}%
\pgfsetfillopacity{0.816117}%
\pgfsetlinewidth{1.003750pt}%
\definecolor{currentstroke}{rgb}{0.121569,0.466667,0.705882}%
\pgfsetstrokecolor{currentstroke}%
\pgfsetstrokeopacity{0.816117}%
\pgfsetdash{}{0pt}%
\pgfpathmoveto{\pgfqpoint{2.269727in}{1.562323in}}%
\pgfpathcurveto{\pgfqpoint{2.277963in}{1.562323in}}{\pgfqpoint{2.285863in}{1.565596in}}{\pgfqpoint{2.291687in}{1.571419in}}%
\pgfpathcurveto{\pgfqpoint{2.297511in}{1.577243in}}{\pgfqpoint{2.300783in}{1.585143in}}{\pgfqpoint{2.300783in}{1.593380in}}%
\pgfpathcurveto{\pgfqpoint{2.300783in}{1.601616in}}{\pgfqpoint{2.297511in}{1.609516in}}{\pgfqpoint{2.291687in}{1.615340in}}%
\pgfpathcurveto{\pgfqpoint{2.285863in}{1.621164in}}{\pgfqpoint{2.277963in}{1.624436in}}{\pgfqpoint{2.269727in}{1.624436in}}%
\pgfpathcurveto{\pgfqpoint{2.261490in}{1.624436in}}{\pgfqpoint{2.253590in}{1.621164in}}{\pgfqpoint{2.247766in}{1.615340in}}%
\pgfpathcurveto{\pgfqpoint{2.241942in}{1.609516in}}{\pgfqpoint{2.238670in}{1.601616in}}{\pgfqpoint{2.238670in}{1.593380in}}%
\pgfpathcurveto{\pgfqpoint{2.238670in}{1.585143in}}{\pgfqpoint{2.241942in}{1.577243in}}{\pgfqpoint{2.247766in}{1.571419in}}%
\pgfpathcurveto{\pgfqpoint{2.253590in}{1.565596in}}{\pgfqpoint{2.261490in}{1.562323in}}{\pgfqpoint{2.269727in}{1.562323in}}%
\pgfpathclose%
\pgfusepath{stroke,fill}%
\end{pgfscope}%
\begin{pgfscope}%
\pgfpathrectangle{\pgfqpoint{0.100000in}{0.212622in}}{\pgfqpoint{3.696000in}{3.696000in}}%
\pgfusepath{clip}%
\pgfsetbuttcap%
\pgfsetroundjoin%
\definecolor{currentfill}{rgb}{0.121569,0.466667,0.705882}%
\pgfsetfillcolor{currentfill}%
\pgfsetfillopacity{0.817879}%
\pgfsetlinewidth{1.003750pt}%
\definecolor{currentstroke}{rgb}{0.121569,0.466667,0.705882}%
\pgfsetstrokecolor{currentstroke}%
\pgfsetstrokeopacity{0.817879}%
\pgfsetdash{}{0pt}%
\pgfpathmoveto{\pgfqpoint{2.271062in}{1.559851in}}%
\pgfpathcurveto{\pgfqpoint{2.279298in}{1.559851in}}{\pgfqpoint{2.287198in}{1.563123in}}{\pgfqpoint{2.293022in}{1.568947in}}%
\pgfpathcurveto{\pgfqpoint{2.298846in}{1.574771in}}{\pgfqpoint{2.302119in}{1.582671in}}{\pgfqpoint{2.302119in}{1.590907in}}%
\pgfpathcurveto{\pgfqpoint{2.302119in}{1.599144in}}{\pgfqpoint{2.298846in}{1.607044in}}{\pgfqpoint{2.293022in}{1.612868in}}%
\pgfpathcurveto{\pgfqpoint{2.287198in}{1.618692in}}{\pgfqpoint{2.279298in}{1.621964in}}{\pgfqpoint{2.271062in}{1.621964in}}%
\pgfpathcurveto{\pgfqpoint{2.262826in}{1.621964in}}{\pgfqpoint{2.254926in}{1.618692in}}{\pgfqpoint{2.249102in}{1.612868in}}%
\pgfpathcurveto{\pgfqpoint{2.243278in}{1.607044in}}{\pgfqpoint{2.240006in}{1.599144in}}{\pgfqpoint{2.240006in}{1.590907in}}%
\pgfpathcurveto{\pgfqpoint{2.240006in}{1.582671in}}{\pgfqpoint{2.243278in}{1.574771in}}{\pgfqpoint{2.249102in}{1.568947in}}%
\pgfpathcurveto{\pgfqpoint{2.254926in}{1.563123in}}{\pgfqpoint{2.262826in}{1.559851in}}{\pgfqpoint{2.271062in}{1.559851in}}%
\pgfpathclose%
\pgfusepath{stroke,fill}%
\end{pgfscope}%
\begin{pgfscope}%
\pgfpathrectangle{\pgfqpoint{0.100000in}{0.212622in}}{\pgfqpoint{3.696000in}{3.696000in}}%
\pgfusepath{clip}%
\pgfsetbuttcap%
\pgfsetroundjoin%
\definecolor{currentfill}{rgb}{0.121569,0.466667,0.705882}%
\pgfsetfillcolor{currentfill}%
\pgfsetfillopacity{0.818640}%
\pgfsetlinewidth{1.003750pt}%
\definecolor{currentstroke}{rgb}{0.121569,0.466667,0.705882}%
\pgfsetstrokecolor{currentstroke}%
\pgfsetstrokeopacity{0.818640}%
\pgfsetdash{}{0pt}%
\pgfpathmoveto{\pgfqpoint{2.271633in}{1.557120in}}%
\pgfpathcurveto{\pgfqpoint{2.279869in}{1.557120in}}{\pgfqpoint{2.287769in}{1.560392in}}{\pgfqpoint{2.293593in}{1.566216in}}%
\pgfpathcurveto{\pgfqpoint{2.299417in}{1.572040in}}{\pgfqpoint{2.302689in}{1.579940in}}{\pgfqpoint{2.302689in}{1.588176in}}%
\pgfpathcurveto{\pgfqpoint{2.302689in}{1.596413in}}{\pgfqpoint{2.299417in}{1.604313in}}{\pgfqpoint{2.293593in}{1.610137in}}%
\pgfpathcurveto{\pgfqpoint{2.287769in}{1.615961in}}{\pgfqpoint{2.279869in}{1.619233in}}{\pgfqpoint{2.271633in}{1.619233in}}%
\pgfpathcurveto{\pgfqpoint{2.263397in}{1.619233in}}{\pgfqpoint{2.255497in}{1.615961in}}{\pgfqpoint{2.249673in}{1.610137in}}%
\pgfpathcurveto{\pgfqpoint{2.243849in}{1.604313in}}{\pgfqpoint{2.240576in}{1.596413in}}{\pgfqpoint{2.240576in}{1.588176in}}%
\pgfpathcurveto{\pgfqpoint{2.240576in}{1.579940in}}{\pgfqpoint{2.243849in}{1.572040in}}{\pgfqpoint{2.249673in}{1.566216in}}%
\pgfpathcurveto{\pgfqpoint{2.255497in}{1.560392in}}{\pgfqpoint{2.263397in}{1.557120in}}{\pgfqpoint{2.271633in}{1.557120in}}%
\pgfpathclose%
\pgfusepath{stroke,fill}%
\end{pgfscope}%
\begin{pgfscope}%
\pgfpathrectangle{\pgfqpoint{0.100000in}{0.212622in}}{\pgfqpoint{3.696000in}{3.696000in}}%
\pgfusepath{clip}%
\pgfsetbuttcap%
\pgfsetroundjoin%
\definecolor{currentfill}{rgb}{0.121569,0.466667,0.705882}%
\pgfsetfillcolor{currentfill}%
\pgfsetfillopacity{0.819811}%
\pgfsetlinewidth{1.003750pt}%
\definecolor{currentstroke}{rgb}{0.121569,0.466667,0.705882}%
\pgfsetstrokecolor{currentstroke}%
\pgfsetstrokeopacity{0.819811}%
\pgfsetdash{}{0pt}%
\pgfpathmoveto{\pgfqpoint{2.272667in}{1.555243in}}%
\pgfpathcurveto{\pgfqpoint{2.280903in}{1.555243in}}{\pgfqpoint{2.288803in}{1.558515in}}{\pgfqpoint{2.294627in}{1.564339in}}%
\pgfpathcurveto{\pgfqpoint{2.300451in}{1.570163in}}{\pgfqpoint{2.303723in}{1.578063in}}{\pgfqpoint{2.303723in}{1.586300in}}%
\pgfpathcurveto{\pgfqpoint{2.303723in}{1.594536in}}{\pgfqpoint{2.300451in}{1.602436in}}{\pgfqpoint{2.294627in}{1.608260in}}%
\pgfpathcurveto{\pgfqpoint{2.288803in}{1.614084in}}{\pgfqpoint{2.280903in}{1.617356in}}{\pgfqpoint{2.272667in}{1.617356in}}%
\pgfpathcurveto{\pgfqpoint{2.264430in}{1.617356in}}{\pgfqpoint{2.256530in}{1.614084in}}{\pgfqpoint{2.250706in}{1.608260in}}%
\pgfpathcurveto{\pgfqpoint{2.244883in}{1.602436in}}{\pgfqpoint{2.241610in}{1.594536in}}{\pgfqpoint{2.241610in}{1.586300in}}%
\pgfpathcurveto{\pgfqpoint{2.241610in}{1.578063in}}{\pgfqpoint{2.244883in}{1.570163in}}{\pgfqpoint{2.250706in}{1.564339in}}%
\pgfpathcurveto{\pgfqpoint{2.256530in}{1.558515in}}{\pgfqpoint{2.264430in}{1.555243in}}{\pgfqpoint{2.272667in}{1.555243in}}%
\pgfpathclose%
\pgfusepath{stroke,fill}%
\end{pgfscope}%
\begin{pgfscope}%
\pgfpathrectangle{\pgfqpoint{0.100000in}{0.212622in}}{\pgfqpoint{3.696000in}{3.696000in}}%
\pgfusepath{clip}%
\pgfsetbuttcap%
\pgfsetroundjoin%
\definecolor{currentfill}{rgb}{0.121569,0.466667,0.705882}%
\pgfsetfillcolor{currentfill}%
\pgfsetfillopacity{0.820299}%
\pgfsetlinewidth{1.003750pt}%
\definecolor{currentstroke}{rgb}{0.121569,0.466667,0.705882}%
\pgfsetstrokecolor{currentstroke}%
\pgfsetstrokeopacity{0.820299}%
\pgfsetdash{}{0pt}%
\pgfpathmoveto{\pgfqpoint{0.634814in}{2.616437in}}%
\pgfpathcurveto{\pgfqpoint{0.643050in}{2.616437in}}{\pgfqpoint{0.650950in}{2.619709in}}{\pgfqpoint{0.656774in}{2.625533in}}%
\pgfpathcurveto{\pgfqpoint{0.662598in}{2.631357in}}{\pgfqpoint{0.665870in}{2.639257in}}{\pgfqpoint{0.665870in}{2.647493in}}%
\pgfpathcurveto{\pgfqpoint{0.665870in}{2.655730in}}{\pgfqpoint{0.662598in}{2.663630in}}{\pgfqpoint{0.656774in}{2.669454in}}%
\pgfpathcurveto{\pgfqpoint{0.650950in}{2.675278in}}{\pgfqpoint{0.643050in}{2.678550in}}{\pgfqpoint{0.634814in}{2.678550in}}%
\pgfpathcurveto{\pgfqpoint{0.626577in}{2.678550in}}{\pgfqpoint{0.618677in}{2.675278in}}{\pgfqpoint{0.612853in}{2.669454in}}%
\pgfpathcurveto{\pgfqpoint{0.607030in}{2.663630in}}{\pgfqpoint{0.603757in}{2.655730in}}{\pgfqpoint{0.603757in}{2.647493in}}%
\pgfpathcurveto{\pgfqpoint{0.603757in}{2.639257in}}{\pgfqpoint{0.607030in}{2.631357in}}{\pgfqpoint{0.612853in}{2.625533in}}%
\pgfpathcurveto{\pgfqpoint{0.618677in}{2.619709in}}{\pgfqpoint{0.626577in}{2.616437in}}{\pgfqpoint{0.634814in}{2.616437in}}%
\pgfpathclose%
\pgfusepath{stroke,fill}%
\end{pgfscope}%
\begin{pgfscope}%
\pgfpathrectangle{\pgfqpoint{0.100000in}{0.212622in}}{\pgfqpoint{3.696000in}{3.696000in}}%
\pgfusepath{clip}%
\pgfsetbuttcap%
\pgfsetroundjoin%
\definecolor{currentfill}{rgb}{0.121569,0.466667,0.705882}%
\pgfsetfillcolor{currentfill}%
\pgfsetfillopacity{0.820370}%
\pgfsetlinewidth{1.003750pt}%
\definecolor{currentstroke}{rgb}{0.121569,0.466667,0.705882}%
\pgfsetstrokecolor{currentstroke}%
\pgfsetstrokeopacity{0.820370}%
\pgfsetdash{}{0pt}%
\pgfpathmoveto{\pgfqpoint{0.624828in}{2.629289in}}%
\pgfpathcurveto{\pgfqpoint{0.633064in}{2.629289in}}{\pgfqpoint{0.640964in}{2.632561in}}{\pgfqpoint{0.646788in}{2.638385in}}%
\pgfpathcurveto{\pgfqpoint{0.652612in}{2.644209in}}{\pgfqpoint{0.655884in}{2.652109in}}{\pgfqpoint{0.655884in}{2.660345in}}%
\pgfpathcurveto{\pgfqpoint{0.655884in}{2.668581in}}{\pgfqpoint{0.652612in}{2.676481in}}{\pgfqpoint{0.646788in}{2.682305in}}%
\pgfpathcurveto{\pgfqpoint{0.640964in}{2.688129in}}{\pgfqpoint{0.633064in}{2.691402in}}{\pgfqpoint{0.624828in}{2.691402in}}%
\pgfpathcurveto{\pgfqpoint{0.616591in}{2.691402in}}{\pgfqpoint{0.608691in}{2.688129in}}{\pgfqpoint{0.602867in}{2.682305in}}%
\pgfpathcurveto{\pgfqpoint{0.597043in}{2.676481in}}{\pgfqpoint{0.593771in}{2.668581in}}{\pgfqpoint{0.593771in}{2.660345in}}%
\pgfpathcurveto{\pgfqpoint{0.593771in}{2.652109in}}{\pgfqpoint{0.597043in}{2.644209in}}{\pgfqpoint{0.602867in}{2.638385in}}%
\pgfpathcurveto{\pgfqpoint{0.608691in}{2.632561in}}{\pgfqpoint{0.616591in}{2.629289in}}{\pgfqpoint{0.624828in}{2.629289in}}%
\pgfpathclose%
\pgfusepath{stroke,fill}%
\end{pgfscope}%
\begin{pgfscope}%
\pgfpathrectangle{\pgfqpoint{0.100000in}{0.212622in}}{\pgfqpoint{3.696000in}{3.696000in}}%
\pgfusepath{clip}%
\pgfsetbuttcap%
\pgfsetroundjoin%
\definecolor{currentfill}{rgb}{0.121569,0.466667,0.705882}%
\pgfsetfillcolor{currentfill}%
\pgfsetfillopacity{0.820581}%
\pgfsetlinewidth{1.003750pt}%
\definecolor{currentstroke}{rgb}{0.121569,0.466667,0.705882}%
\pgfsetstrokecolor{currentstroke}%
\pgfsetstrokeopacity{0.820581}%
\pgfsetdash{}{0pt}%
\pgfpathmoveto{\pgfqpoint{0.618227in}{2.631525in}}%
\pgfpathcurveto{\pgfqpoint{0.626464in}{2.631525in}}{\pgfqpoint{0.634364in}{2.634797in}}{\pgfqpoint{0.640188in}{2.640621in}}%
\pgfpathcurveto{\pgfqpoint{0.646011in}{2.646445in}}{\pgfqpoint{0.649284in}{2.654345in}}{\pgfqpoint{0.649284in}{2.662581in}}%
\pgfpathcurveto{\pgfqpoint{0.649284in}{2.670818in}}{\pgfqpoint{0.646011in}{2.678718in}}{\pgfqpoint{0.640188in}{2.684542in}}%
\pgfpathcurveto{\pgfqpoint{0.634364in}{2.690366in}}{\pgfqpoint{0.626464in}{2.693638in}}{\pgfqpoint{0.618227in}{2.693638in}}%
\pgfpathcurveto{\pgfqpoint{0.609991in}{2.693638in}}{\pgfqpoint{0.602091in}{2.690366in}}{\pgfqpoint{0.596267in}{2.684542in}}%
\pgfpathcurveto{\pgfqpoint{0.590443in}{2.678718in}}{\pgfqpoint{0.587171in}{2.670818in}}{\pgfqpoint{0.587171in}{2.662581in}}%
\pgfpathcurveto{\pgfqpoint{0.587171in}{2.654345in}}{\pgfqpoint{0.590443in}{2.646445in}}{\pgfqpoint{0.596267in}{2.640621in}}%
\pgfpathcurveto{\pgfqpoint{0.602091in}{2.634797in}}{\pgfqpoint{0.609991in}{2.631525in}}{\pgfqpoint{0.618227in}{2.631525in}}%
\pgfpathclose%
\pgfusepath{stroke,fill}%
\end{pgfscope}%
\begin{pgfscope}%
\pgfpathrectangle{\pgfqpoint{0.100000in}{0.212622in}}{\pgfqpoint{3.696000in}{3.696000in}}%
\pgfusepath{clip}%
\pgfsetbuttcap%
\pgfsetroundjoin%
\definecolor{currentfill}{rgb}{0.121569,0.466667,0.705882}%
\pgfsetfillcolor{currentfill}%
\pgfsetfillopacity{0.820593}%
\pgfsetlinewidth{1.003750pt}%
\definecolor{currentstroke}{rgb}{0.121569,0.466667,0.705882}%
\pgfsetstrokecolor{currentstroke}%
\pgfsetstrokeopacity{0.820593}%
\pgfsetdash{}{0pt}%
\pgfpathmoveto{\pgfqpoint{2.273179in}{1.555024in}}%
\pgfpathcurveto{\pgfqpoint{2.281416in}{1.555024in}}{\pgfqpoint{2.289316in}{1.558297in}}{\pgfqpoint{2.295140in}{1.564121in}}%
\pgfpathcurveto{\pgfqpoint{2.300963in}{1.569944in}}{\pgfqpoint{2.304236in}{1.577844in}}{\pgfqpoint{2.304236in}{1.586081in}}%
\pgfpathcurveto{\pgfqpoint{2.304236in}{1.594317in}}{\pgfqpoint{2.300963in}{1.602217in}}{\pgfqpoint{2.295140in}{1.608041in}}%
\pgfpathcurveto{\pgfqpoint{2.289316in}{1.613865in}}{\pgfqpoint{2.281416in}{1.617137in}}{\pgfqpoint{2.273179in}{1.617137in}}%
\pgfpathcurveto{\pgfqpoint{2.264943in}{1.617137in}}{\pgfqpoint{2.257043in}{1.613865in}}{\pgfqpoint{2.251219in}{1.608041in}}%
\pgfpathcurveto{\pgfqpoint{2.245395in}{1.602217in}}{\pgfqpoint{2.242123in}{1.594317in}}{\pgfqpoint{2.242123in}{1.586081in}}%
\pgfpathcurveto{\pgfqpoint{2.242123in}{1.577844in}}{\pgfqpoint{2.245395in}{1.569944in}}{\pgfqpoint{2.251219in}{1.564121in}}%
\pgfpathcurveto{\pgfqpoint{2.257043in}{1.558297in}}{\pgfqpoint{2.264943in}{1.555024in}}{\pgfqpoint{2.273179in}{1.555024in}}%
\pgfpathclose%
\pgfusepath{stroke,fill}%
\end{pgfscope}%
\begin{pgfscope}%
\pgfpathrectangle{\pgfqpoint{0.100000in}{0.212622in}}{\pgfqpoint{3.696000in}{3.696000in}}%
\pgfusepath{clip}%
\pgfsetbuttcap%
\pgfsetroundjoin%
\definecolor{currentfill}{rgb}{0.121569,0.466667,0.705882}%
\pgfsetfillcolor{currentfill}%
\pgfsetfillopacity{0.821371}%
\pgfsetlinewidth{1.003750pt}%
\definecolor{currentstroke}{rgb}{0.121569,0.466667,0.705882}%
\pgfsetstrokecolor{currentstroke}%
\pgfsetstrokeopacity{0.821371}%
\pgfsetdash{}{0pt}%
\pgfpathmoveto{\pgfqpoint{0.614876in}{2.639866in}}%
\pgfpathcurveto{\pgfqpoint{0.623112in}{2.639866in}}{\pgfqpoint{0.631012in}{2.643138in}}{\pgfqpoint{0.636836in}{2.648962in}}%
\pgfpathcurveto{\pgfqpoint{0.642660in}{2.654786in}}{\pgfqpoint{0.645933in}{2.662686in}}{\pgfqpoint{0.645933in}{2.670922in}}%
\pgfpathcurveto{\pgfqpoint{0.645933in}{2.679159in}}{\pgfqpoint{0.642660in}{2.687059in}}{\pgfqpoint{0.636836in}{2.692883in}}%
\pgfpathcurveto{\pgfqpoint{0.631012in}{2.698706in}}{\pgfqpoint{0.623112in}{2.701979in}}{\pgfqpoint{0.614876in}{2.701979in}}%
\pgfpathcurveto{\pgfqpoint{0.606640in}{2.701979in}}{\pgfqpoint{0.598740in}{2.698706in}}{\pgfqpoint{0.592916in}{2.692883in}}%
\pgfpathcurveto{\pgfqpoint{0.587092in}{2.687059in}}{\pgfqpoint{0.583820in}{2.679159in}}{\pgfqpoint{0.583820in}{2.670922in}}%
\pgfpathcurveto{\pgfqpoint{0.583820in}{2.662686in}}{\pgfqpoint{0.587092in}{2.654786in}}{\pgfqpoint{0.592916in}{2.648962in}}%
\pgfpathcurveto{\pgfqpoint{0.598740in}{2.643138in}}{\pgfqpoint{0.606640in}{2.639866in}}{\pgfqpoint{0.614876in}{2.639866in}}%
\pgfpathclose%
\pgfusepath{stroke,fill}%
\end{pgfscope}%
\begin{pgfscope}%
\pgfpathrectangle{\pgfqpoint{0.100000in}{0.212622in}}{\pgfqpoint{3.696000in}{3.696000in}}%
\pgfusepath{clip}%
\pgfsetbuttcap%
\pgfsetroundjoin%
\definecolor{currentfill}{rgb}{0.121569,0.466667,0.705882}%
\pgfsetfillcolor{currentfill}%
\pgfsetfillopacity{0.821599}%
\pgfsetlinewidth{1.003750pt}%
\definecolor{currentstroke}{rgb}{0.121569,0.466667,0.705882}%
\pgfsetstrokecolor{currentstroke}%
\pgfsetstrokeopacity{0.821599}%
\pgfsetdash{}{0pt}%
\pgfpathmoveto{\pgfqpoint{2.274090in}{1.553595in}}%
\pgfpathcurveto{\pgfqpoint{2.282326in}{1.553595in}}{\pgfqpoint{2.290226in}{1.556867in}}{\pgfqpoint{2.296050in}{1.562691in}}%
\pgfpathcurveto{\pgfqpoint{2.301874in}{1.568515in}}{\pgfqpoint{2.305147in}{1.576415in}}{\pgfqpoint{2.305147in}{1.584652in}}%
\pgfpathcurveto{\pgfqpoint{2.305147in}{1.592888in}}{\pgfqpoint{2.301874in}{1.600788in}}{\pgfqpoint{2.296050in}{1.606612in}}%
\pgfpathcurveto{\pgfqpoint{2.290226in}{1.612436in}}{\pgfqpoint{2.282326in}{1.615708in}}{\pgfqpoint{2.274090in}{1.615708in}}%
\pgfpathcurveto{\pgfqpoint{2.265854in}{1.615708in}}{\pgfqpoint{2.257954in}{1.612436in}}{\pgfqpoint{2.252130in}{1.606612in}}%
\pgfpathcurveto{\pgfqpoint{2.246306in}{1.600788in}}{\pgfqpoint{2.243034in}{1.592888in}}{\pgfqpoint{2.243034in}{1.584652in}}%
\pgfpathcurveto{\pgfqpoint{2.243034in}{1.576415in}}{\pgfqpoint{2.246306in}{1.568515in}}{\pgfqpoint{2.252130in}{1.562691in}}%
\pgfpathcurveto{\pgfqpoint{2.257954in}{1.556867in}}{\pgfqpoint{2.265854in}{1.553595in}}{\pgfqpoint{2.274090in}{1.553595in}}%
\pgfpathclose%
\pgfusepath{stroke,fill}%
\end{pgfscope}%
\begin{pgfscope}%
\pgfpathrectangle{\pgfqpoint{0.100000in}{0.212622in}}{\pgfqpoint{3.696000in}{3.696000in}}%
\pgfusepath{clip}%
\pgfsetbuttcap%
\pgfsetroundjoin%
\definecolor{currentfill}{rgb}{0.121569,0.466667,0.705882}%
\pgfsetfillcolor{currentfill}%
\pgfsetfillopacity{0.821631}%
\pgfsetlinewidth{1.003750pt}%
\definecolor{currentstroke}{rgb}{0.121569,0.466667,0.705882}%
\pgfsetstrokecolor{currentstroke}%
\pgfsetstrokeopacity{0.821631}%
\pgfsetdash{}{0pt}%
\pgfpathmoveto{\pgfqpoint{0.612810in}{2.641103in}}%
\pgfpathcurveto{\pgfqpoint{0.621046in}{2.641103in}}{\pgfqpoint{0.628946in}{2.644376in}}{\pgfqpoint{0.634770in}{2.650200in}}%
\pgfpathcurveto{\pgfqpoint{0.640594in}{2.656024in}}{\pgfqpoint{0.643866in}{2.663924in}}{\pgfqpoint{0.643866in}{2.672160in}}%
\pgfpathcurveto{\pgfqpoint{0.643866in}{2.680396in}}{\pgfqpoint{0.640594in}{2.688296in}}{\pgfqpoint{0.634770in}{2.694120in}}%
\pgfpathcurveto{\pgfqpoint{0.628946in}{2.699944in}}{\pgfqpoint{0.621046in}{2.703216in}}{\pgfqpoint{0.612810in}{2.703216in}}%
\pgfpathcurveto{\pgfqpoint{0.604573in}{2.703216in}}{\pgfqpoint{0.596673in}{2.699944in}}{\pgfqpoint{0.590850in}{2.694120in}}%
\pgfpathcurveto{\pgfqpoint{0.585026in}{2.688296in}}{\pgfqpoint{0.581753in}{2.680396in}}{\pgfqpoint{0.581753in}{2.672160in}}%
\pgfpathcurveto{\pgfqpoint{0.581753in}{2.663924in}}{\pgfqpoint{0.585026in}{2.656024in}}{\pgfqpoint{0.590850in}{2.650200in}}%
\pgfpathcurveto{\pgfqpoint{0.596673in}{2.644376in}}{\pgfqpoint{0.604573in}{2.641103in}}{\pgfqpoint{0.612810in}{2.641103in}}%
\pgfpathclose%
\pgfusepath{stroke,fill}%
\end{pgfscope}%
\begin{pgfscope}%
\pgfpathrectangle{\pgfqpoint{0.100000in}{0.212622in}}{\pgfqpoint{3.696000in}{3.696000in}}%
\pgfusepath{clip}%
\pgfsetbuttcap%
\pgfsetroundjoin%
\definecolor{currentfill}{rgb}{0.121569,0.466667,0.705882}%
\pgfsetfillcolor{currentfill}%
\pgfsetfillopacity{0.821749}%
\pgfsetlinewidth{1.003750pt}%
\definecolor{currentstroke}{rgb}{0.121569,0.466667,0.705882}%
\pgfsetstrokecolor{currentstroke}%
\pgfsetstrokeopacity{0.821749}%
\pgfsetdash{}{0pt}%
\pgfpathmoveto{\pgfqpoint{0.611680in}{2.641613in}}%
\pgfpathcurveto{\pgfqpoint{0.619916in}{2.641613in}}{\pgfqpoint{0.627816in}{2.644885in}}{\pgfqpoint{0.633640in}{2.650709in}}%
\pgfpathcurveto{\pgfqpoint{0.639464in}{2.656533in}}{\pgfqpoint{0.642736in}{2.664433in}}{\pgfqpoint{0.642736in}{2.672669in}}%
\pgfpathcurveto{\pgfqpoint{0.642736in}{2.680905in}}{\pgfqpoint{0.639464in}{2.688805in}}{\pgfqpoint{0.633640in}{2.694629in}}%
\pgfpathcurveto{\pgfqpoint{0.627816in}{2.700453in}}{\pgfqpoint{0.619916in}{2.703726in}}{\pgfqpoint{0.611680in}{2.703726in}}%
\pgfpathcurveto{\pgfqpoint{0.603443in}{2.703726in}}{\pgfqpoint{0.595543in}{2.700453in}}{\pgfqpoint{0.589719in}{2.694629in}}%
\pgfpathcurveto{\pgfqpoint{0.583895in}{2.688805in}}{\pgfqpoint{0.580623in}{2.680905in}}{\pgfqpoint{0.580623in}{2.672669in}}%
\pgfpathcurveto{\pgfqpoint{0.580623in}{2.664433in}}{\pgfqpoint{0.583895in}{2.656533in}}{\pgfqpoint{0.589719in}{2.650709in}}%
\pgfpathcurveto{\pgfqpoint{0.595543in}{2.644885in}}{\pgfqpoint{0.603443in}{2.641613in}}{\pgfqpoint{0.611680in}{2.641613in}}%
\pgfpathclose%
\pgfusepath{stroke,fill}%
\end{pgfscope}%
\begin{pgfscope}%
\pgfpathrectangle{\pgfqpoint{0.100000in}{0.212622in}}{\pgfqpoint{3.696000in}{3.696000in}}%
\pgfusepath{clip}%
\pgfsetbuttcap%
\pgfsetroundjoin%
\definecolor{currentfill}{rgb}{0.121569,0.466667,0.705882}%
\pgfsetfillcolor{currentfill}%
\pgfsetfillopacity{0.821813}%
\pgfsetlinewidth{1.003750pt}%
\definecolor{currentstroke}{rgb}{0.121569,0.466667,0.705882}%
\pgfsetstrokecolor{currentstroke}%
\pgfsetstrokeopacity{0.821813}%
\pgfsetdash{}{0pt}%
\pgfpathmoveto{\pgfqpoint{0.611067in}{2.641710in}}%
\pgfpathcurveto{\pgfqpoint{0.619303in}{2.641710in}}{\pgfqpoint{0.627203in}{2.644983in}}{\pgfqpoint{0.633027in}{2.650807in}}%
\pgfpathcurveto{\pgfqpoint{0.638851in}{2.656630in}}{\pgfqpoint{0.642123in}{2.664531in}}{\pgfqpoint{0.642123in}{2.672767in}}%
\pgfpathcurveto{\pgfqpoint{0.642123in}{2.681003in}}{\pgfqpoint{0.638851in}{2.688903in}}{\pgfqpoint{0.633027in}{2.694727in}}%
\pgfpathcurveto{\pgfqpoint{0.627203in}{2.700551in}}{\pgfqpoint{0.619303in}{2.703823in}}{\pgfqpoint{0.611067in}{2.703823in}}%
\pgfpathcurveto{\pgfqpoint{0.602831in}{2.703823in}}{\pgfqpoint{0.594931in}{2.700551in}}{\pgfqpoint{0.589107in}{2.694727in}}%
\pgfpathcurveto{\pgfqpoint{0.583283in}{2.688903in}}{\pgfqpoint{0.580010in}{2.681003in}}{\pgfqpoint{0.580010in}{2.672767in}}%
\pgfpathcurveto{\pgfqpoint{0.580010in}{2.664531in}}{\pgfqpoint{0.583283in}{2.656630in}}{\pgfqpoint{0.589107in}{2.650807in}}%
\pgfpathcurveto{\pgfqpoint{0.594931in}{2.644983in}}{\pgfqpoint{0.602831in}{2.641710in}}{\pgfqpoint{0.611067in}{2.641710in}}%
\pgfpathclose%
\pgfusepath{stroke,fill}%
\end{pgfscope}%
\begin{pgfscope}%
\pgfpathrectangle{\pgfqpoint{0.100000in}{0.212622in}}{\pgfqpoint{3.696000in}{3.696000in}}%
\pgfusepath{clip}%
\pgfsetbuttcap%
\pgfsetroundjoin%
\definecolor{currentfill}{rgb}{0.121569,0.466667,0.705882}%
\pgfsetfillcolor{currentfill}%
\pgfsetfillopacity{0.822134}%
\pgfsetlinewidth{1.003750pt}%
\definecolor{currentstroke}{rgb}{0.121569,0.466667,0.705882}%
\pgfsetstrokecolor{currentstroke}%
\pgfsetstrokeopacity{0.822134}%
\pgfsetdash{}{0pt}%
\pgfpathmoveto{\pgfqpoint{0.608365in}{2.640646in}}%
\pgfpathcurveto{\pgfqpoint{0.616601in}{2.640646in}}{\pgfqpoint{0.624501in}{2.643919in}}{\pgfqpoint{0.630325in}{2.649743in}}%
\pgfpathcurveto{\pgfqpoint{0.636149in}{2.655567in}}{\pgfqpoint{0.639421in}{2.663467in}}{\pgfqpoint{0.639421in}{2.671703in}}%
\pgfpathcurveto{\pgfqpoint{0.639421in}{2.679939in}}{\pgfqpoint{0.636149in}{2.687839in}}{\pgfqpoint{0.630325in}{2.693663in}}%
\pgfpathcurveto{\pgfqpoint{0.624501in}{2.699487in}}{\pgfqpoint{0.616601in}{2.702759in}}{\pgfqpoint{0.608365in}{2.702759in}}%
\pgfpathcurveto{\pgfqpoint{0.600129in}{2.702759in}}{\pgfqpoint{0.592229in}{2.699487in}}{\pgfqpoint{0.586405in}{2.693663in}}%
\pgfpathcurveto{\pgfqpoint{0.580581in}{2.687839in}}{\pgfqpoint{0.577308in}{2.679939in}}{\pgfqpoint{0.577308in}{2.671703in}}%
\pgfpathcurveto{\pgfqpoint{0.577308in}{2.663467in}}{\pgfqpoint{0.580581in}{2.655567in}}{\pgfqpoint{0.586405in}{2.649743in}}%
\pgfpathcurveto{\pgfqpoint{0.592229in}{2.643919in}}{\pgfqpoint{0.600129in}{2.640646in}}{\pgfqpoint{0.608365in}{2.640646in}}%
\pgfpathclose%
\pgfusepath{stroke,fill}%
\end{pgfscope}%
\begin{pgfscope}%
\pgfpathrectangle{\pgfqpoint{0.100000in}{0.212622in}}{\pgfqpoint{3.696000in}{3.696000in}}%
\pgfusepath{clip}%
\pgfsetbuttcap%
\pgfsetroundjoin%
\definecolor{currentfill}{rgb}{0.121569,0.466667,0.705882}%
\pgfsetfillcolor{currentfill}%
\pgfsetfillopacity{0.822135}%
\pgfsetlinewidth{1.003750pt}%
\definecolor{currentstroke}{rgb}{0.121569,0.466667,0.705882}%
\pgfsetstrokecolor{currentstroke}%
\pgfsetstrokeopacity{0.822135}%
\pgfsetdash{}{0pt}%
\pgfpathmoveto{\pgfqpoint{0.642282in}{2.612643in}}%
\pgfpathcurveto{\pgfqpoint{0.650518in}{2.612643in}}{\pgfqpoint{0.658418in}{2.615915in}}{\pgfqpoint{0.664242in}{2.621739in}}%
\pgfpathcurveto{\pgfqpoint{0.670066in}{2.627563in}}{\pgfqpoint{0.673338in}{2.635463in}}{\pgfqpoint{0.673338in}{2.643699in}}%
\pgfpathcurveto{\pgfqpoint{0.673338in}{2.651936in}}{\pgfqpoint{0.670066in}{2.659836in}}{\pgfqpoint{0.664242in}{2.665660in}}%
\pgfpathcurveto{\pgfqpoint{0.658418in}{2.671483in}}{\pgfqpoint{0.650518in}{2.674756in}}{\pgfqpoint{0.642282in}{2.674756in}}%
\pgfpathcurveto{\pgfqpoint{0.634045in}{2.674756in}}{\pgfqpoint{0.626145in}{2.671483in}}{\pgfqpoint{0.620321in}{2.665660in}}%
\pgfpathcurveto{\pgfqpoint{0.614498in}{2.659836in}}{\pgfqpoint{0.611225in}{2.651936in}}{\pgfqpoint{0.611225in}{2.643699in}}%
\pgfpathcurveto{\pgfqpoint{0.611225in}{2.635463in}}{\pgfqpoint{0.614498in}{2.627563in}}{\pgfqpoint{0.620321in}{2.621739in}}%
\pgfpathcurveto{\pgfqpoint{0.626145in}{2.615915in}}{\pgfqpoint{0.634045in}{2.612643in}}{\pgfqpoint{0.642282in}{2.612643in}}%
\pgfpathclose%
\pgfusepath{stroke,fill}%
\end{pgfscope}%
\begin{pgfscope}%
\pgfpathrectangle{\pgfqpoint{0.100000in}{0.212622in}}{\pgfqpoint{3.696000in}{3.696000in}}%
\pgfusepath{clip}%
\pgfsetbuttcap%
\pgfsetroundjoin%
\definecolor{currentfill}{rgb}{0.121569,0.466667,0.705882}%
\pgfsetfillcolor{currentfill}%
\pgfsetfillopacity{0.822220}%
\pgfsetlinewidth{1.003750pt}%
\definecolor{currentstroke}{rgb}{0.121569,0.466667,0.705882}%
\pgfsetstrokecolor{currentstroke}%
\pgfsetstrokeopacity{0.822220}%
\pgfsetdash{}{0pt}%
\pgfpathmoveto{\pgfqpoint{2.274345in}{1.553092in}}%
\pgfpathcurveto{\pgfqpoint{2.282581in}{1.553092in}}{\pgfqpoint{2.290481in}{1.556364in}}{\pgfqpoint{2.296305in}{1.562188in}}%
\pgfpathcurveto{\pgfqpoint{2.302129in}{1.568012in}}{\pgfqpoint{2.305401in}{1.575912in}}{\pgfqpoint{2.305401in}{1.584148in}}%
\pgfpathcurveto{\pgfqpoint{2.305401in}{1.592385in}}{\pgfqpoint{2.302129in}{1.600285in}}{\pgfqpoint{2.296305in}{1.606109in}}%
\pgfpathcurveto{\pgfqpoint{2.290481in}{1.611932in}}{\pgfqpoint{2.282581in}{1.615205in}}{\pgfqpoint{2.274345in}{1.615205in}}%
\pgfpathcurveto{\pgfqpoint{2.266109in}{1.615205in}}{\pgfqpoint{2.258209in}{1.611932in}}{\pgfqpoint{2.252385in}{1.606109in}}%
\pgfpathcurveto{\pgfqpoint{2.246561in}{1.600285in}}{\pgfqpoint{2.243288in}{1.592385in}}{\pgfqpoint{2.243288in}{1.584148in}}%
\pgfpathcurveto{\pgfqpoint{2.243288in}{1.575912in}}{\pgfqpoint{2.246561in}{1.568012in}}{\pgfqpoint{2.252385in}{1.562188in}}%
\pgfpathcurveto{\pgfqpoint{2.258209in}{1.556364in}}{\pgfqpoint{2.266109in}{1.553092in}}{\pgfqpoint{2.274345in}{1.553092in}}%
\pgfpathclose%
\pgfusepath{stroke,fill}%
\end{pgfscope}%
\begin{pgfscope}%
\pgfpathrectangle{\pgfqpoint{0.100000in}{0.212622in}}{\pgfqpoint{3.696000in}{3.696000in}}%
\pgfusepath{clip}%
\pgfsetbuttcap%
\pgfsetroundjoin%
\definecolor{currentfill}{rgb}{0.121569,0.466667,0.705882}%
\pgfsetfillcolor{currentfill}%
\pgfsetfillopacity{0.822728}%
\pgfsetlinewidth{1.003750pt}%
\definecolor{currentstroke}{rgb}{0.121569,0.466667,0.705882}%
\pgfsetstrokecolor{currentstroke}%
\pgfsetstrokeopacity{0.822728}%
\pgfsetdash{}{0pt}%
\pgfpathmoveto{\pgfqpoint{0.604784in}{2.637979in}}%
\pgfpathcurveto{\pgfqpoint{0.613021in}{2.637979in}}{\pgfqpoint{0.620921in}{2.641251in}}{\pgfqpoint{0.626745in}{2.647075in}}%
\pgfpathcurveto{\pgfqpoint{0.632569in}{2.652899in}}{\pgfqpoint{0.635841in}{2.660799in}}{\pgfqpoint{0.635841in}{2.669036in}}%
\pgfpathcurveto{\pgfqpoint{0.635841in}{2.677272in}}{\pgfqpoint{0.632569in}{2.685172in}}{\pgfqpoint{0.626745in}{2.690996in}}%
\pgfpathcurveto{\pgfqpoint{0.620921in}{2.696820in}}{\pgfqpoint{0.613021in}{2.700092in}}{\pgfqpoint{0.604784in}{2.700092in}}%
\pgfpathcurveto{\pgfqpoint{0.596548in}{2.700092in}}{\pgfqpoint{0.588648in}{2.696820in}}{\pgfqpoint{0.582824in}{2.690996in}}%
\pgfpathcurveto{\pgfqpoint{0.577000in}{2.685172in}}{\pgfqpoint{0.573728in}{2.677272in}}{\pgfqpoint{0.573728in}{2.669036in}}%
\pgfpathcurveto{\pgfqpoint{0.573728in}{2.660799in}}{\pgfqpoint{0.577000in}{2.652899in}}{\pgfqpoint{0.582824in}{2.647075in}}%
\pgfpathcurveto{\pgfqpoint{0.588648in}{2.641251in}}{\pgfqpoint{0.596548in}{2.637979in}}{\pgfqpoint{0.604784in}{2.637979in}}%
\pgfpathclose%
\pgfusepath{stroke,fill}%
\end{pgfscope}%
\begin{pgfscope}%
\pgfpathrectangle{\pgfqpoint{0.100000in}{0.212622in}}{\pgfqpoint{3.696000in}{3.696000in}}%
\pgfusepath{clip}%
\pgfsetbuttcap%
\pgfsetroundjoin%
\definecolor{currentfill}{rgb}{0.121569,0.466667,0.705882}%
\pgfsetfillcolor{currentfill}%
\pgfsetfillopacity{0.822989}%
\pgfsetlinewidth{1.003750pt}%
\definecolor{currentstroke}{rgb}{0.121569,0.466667,0.705882}%
\pgfsetstrokecolor{currentstroke}%
\pgfsetstrokeopacity{0.822989}%
\pgfsetdash{}{0pt}%
\pgfpathmoveto{\pgfqpoint{2.275129in}{1.551880in}}%
\pgfpathcurveto{\pgfqpoint{2.283365in}{1.551880in}}{\pgfqpoint{2.291265in}{1.555152in}}{\pgfqpoint{2.297089in}{1.560976in}}%
\pgfpathcurveto{\pgfqpoint{2.302913in}{1.566800in}}{\pgfqpoint{2.306186in}{1.574700in}}{\pgfqpoint{2.306186in}{1.582936in}}%
\pgfpathcurveto{\pgfqpoint{2.306186in}{1.591172in}}{\pgfqpoint{2.302913in}{1.599073in}}{\pgfqpoint{2.297089in}{1.604896in}}%
\pgfpathcurveto{\pgfqpoint{2.291265in}{1.610720in}}{\pgfqpoint{2.283365in}{1.613993in}}{\pgfqpoint{2.275129in}{1.613993in}}%
\pgfpathcurveto{\pgfqpoint{2.266893in}{1.613993in}}{\pgfqpoint{2.258993in}{1.610720in}}{\pgfqpoint{2.253169in}{1.604896in}}%
\pgfpathcurveto{\pgfqpoint{2.247345in}{1.599073in}}{\pgfqpoint{2.244073in}{1.591172in}}{\pgfqpoint{2.244073in}{1.582936in}}%
\pgfpathcurveto{\pgfqpoint{2.244073in}{1.574700in}}{\pgfqpoint{2.247345in}{1.566800in}}{\pgfqpoint{2.253169in}{1.560976in}}%
\pgfpathcurveto{\pgfqpoint{2.258993in}{1.555152in}}{\pgfqpoint{2.266893in}{1.551880in}}{\pgfqpoint{2.275129in}{1.551880in}}%
\pgfpathclose%
\pgfusepath{stroke,fill}%
\end{pgfscope}%
\begin{pgfscope}%
\pgfpathrectangle{\pgfqpoint{0.100000in}{0.212622in}}{\pgfqpoint{3.696000in}{3.696000in}}%
\pgfusepath{clip}%
\pgfsetbuttcap%
\pgfsetroundjoin%
\definecolor{currentfill}{rgb}{0.121569,0.466667,0.705882}%
\pgfsetfillcolor{currentfill}%
\pgfsetfillopacity{0.823244}%
\pgfsetlinewidth{1.003750pt}%
\definecolor{currentstroke}{rgb}{0.121569,0.466667,0.705882}%
\pgfsetstrokecolor{currentstroke}%
\pgfsetstrokeopacity{0.823244}%
\pgfsetdash{}{0pt}%
\pgfpathmoveto{\pgfqpoint{0.649330in}{2.605832in}}%
\pgfpathcurveto{\pgfqpoint{0.657566in}{2.605832in}}{\pgfqpoint{0.665466in}{2.609104in}}{\pgfqpoint{0.671290in}{2.614928in}}%
\pgfpathcurveto{\pgfqpoint{0.677114in}{2.620752in}}{\pgfqpoint{0.680386in}{2.628652in}}{\pgfqpoint{0.680386in}{2.636888in}}%
\pgfpathcurveto{\pgfqpoint{0.680386in}{2.645124in}}{\pgfqpoint{0.677114in}{2.653025in}}{\pgfqpoint{0.671290in}{2.658848in}}%
\pgfpathcurveto{\pgfqpoint{0.665466in}{2.664672in}}{\pgfqpoint{0.657566in}{2.667945in}}{\pgfqpoint{0.649330in}{2.667945in}}%
\pgfpathcurveto{\pgfqpoint{0.641093in}{2.667945in}}{\pgfqpoint{0.633193in}{2.664672in}}{\pgfqpoint{0.627369in}{2.658848in}}%
\pgfpathcurveto{\pgfqpoint{0.621545in}{2.653025in}}{\pgfqpoint{0.618273in}{2.645124in}}{\pgfqpoint{0.618273in}{2.636888in}}%
\pgfpathcurveto{\pgfqpoint{0.618273in}{2.628652in}}{\pgfqpoint{0.621545in}{2.620752in}}{\pgfqpoint{0.627369in}{2.614928in}}%
\pgfpathcurveto{\pgfqpoint{0.633193in}{2.609104in}}{\pgfqpoint{0.641093in}{2.605832in}}{\pgfqpoint{0.649330in}{2.605832in}}%
\pgfpathclose%
\pgfusepath{stroke,fill}%
\end{pgfscope}%
\begin{pgfscope}%
\pgfpathrectangle{\pgfqpoint{0.100000in}{0.212622in}}{\pgfqpoint{3.696000in}{3.696000in}}%
\pgfusepath{clip}%
\pgfsetbuttcap%
\pgfsetroundjoin%
\definecolor{currentfill}{rgb}{0.121569,0.466667,0.705882}%
\pgfsetfillcolor{currentfill}%
\pgfsetfillopacity{0.823514}%
\pgfsetlinewidth{1.003750pt}%
\definecolor{currentstroke}{rgb}{0.121569,0.466667,0.705882}%
\pgfsetstrokecolor{currentstroke}%
\pgfsetstrokeopacity{0.823514}%
\pgfsetdash{}{0pt}%
\pgfpathmoveto{\pgfqpoint{2.275549in}{1.551835in}}%
\pgfpathcurveto{\pgfqpoint{2.283786in}{1.551835in}}{\pgfqpoint{2.291686in}{1.555107in}}{\pgfqpoint{2.297510in}{1.560931in}}%
\pgfpathcurveto{\pgfqpoint{2.303334in}{1.566755in}}{\pgfqpoint{2.306606in}{1.574655in}}{\pgfqpoint{2.306606in}{1.582891in}}%
\pgfpathcurveto{\pgfqpoint{2.306606in}{1.591128in}}{\pgfqpoint{2.303334in}{1.599028in}}{\pgfqpoint{2.297510in}{1.604852in}}%
\pgfpathcurveto{\pgfqpoint{2.291686in}{1.610676in}}{\pgfqpoint{2.283786in}{1.613948in}}{\pgfqpoint{2.275549in}{1.613948in}}%
\pgfpathcurveto{\pgfqpoint{2.267313in}{1.613948in}}{\pgfqpoint{2.259413in}{1.610676in}}{\pgfqpoint{2.253589in}{1.604852in}}%
\pgfpathcurveto{\pgfqpoint{2.247765in}{1.599028in}}{\pgfqpoint{2.244493in}{1.591128in}}{\pgfqpoint{2.244493in}{1.582891in}}%
\pgfpathcurveto{\pgfqpoint{2.244493in}{1.574655in}}{\pgfqpoint{2.247765in}{1.566755in}}{\pgfqpoint{2.253589in}{1.560931in}}%
\pgfpathcurveto{\pgfqpoint{2.259413in}{1.555107in}}{\pgfqpoint{2.267313in}{1.551835in}}{\pgfqpoint{2.275549in}{1.551835in}}%
\pgfpathclose%
\pgfusepath{stroke,fill}%
\end{pgfscope}%
\begin{pgfscope}%
\pgfpathrectangle{\pgfqpoint{0.100000in}{0.212622in}}{\pgfqpoint{3.696000in}{3.696000in}}%
\pgfusepath{clip}%
\pgfsetbuttcap%
\pgfsetroundjoin%
\definecolor{currentfill}{rgb}{0.121569,0.466667,0.705882}%
\pgfsetfillcolor{currentfill}%
\pgfsetfillopacity{0.824346}%
\pgfsetlinewidth{1.003750pt}%
\definecolor{currentstroke}{rgb}{0.121569,0.466667,0.705882}%
\pgfsetstrokecolor{currentstroke}%
\pgfsetstrokeopacity{0.824346}%
\pgfsetdash{}{0pt}%
\pgfpathmoveto{\pgfqpoint{2.276037in}{1.551452in}}%
\pgfpathcurveto{\pgfqpoint{2.284273in}{1.551452in}}{\pgfqpoint{2.292173in}{1.554724in}}{\pgfqpoint{2.297997in}{1.560548in}}%
\pgfpathcurveto{\pgfqpoint{2.303821in}{1.566372in}}{\pgfqpoint{2.307094in}{1.574272in}}{\pgfqpoint{2.307094in}{1.582508in}}%
\pgfpathcurveto{\pgfqpoint{2.307094in}{1.590744in}}{\pgfqpoint{2.303821in}{1.598645in}}{\pgfqpoint{2.297997in}{1.604468in}}%
\pgfpathcurveto{\pgfqpoint{2.292173in}{1.610292in}}{\pgfqpoint{2.284273in}{1.613565in}}{\pgfqpoint{2.276037in}{1.613565in}}%
\pgfpathcurveto{\pgfqpoint{2.267801in}{1.613565in}}{\pgfqpoint{2.259901in}{1.610292in}}{\pgfqpoint{2.254077in}{1.604468in}}%
\pgfpathcurveto{\pgfqpoint{2.248253in}{1.598645in}}{\pgfqpoint{2.244981in}{1.590744in}}{\pgfqpoint{2.244981in}{1.582508in}}%
\pgfpathcurveto{\pgfqpoint{2.244981in}{1.574272in}}{\pgfqpoint{2.248253in}{1.566372in}}{\pgfqpoint{2.254077in}{1.560548in}}%
\pgfpathcurveto{\pgfqpoint{2.259901in}{1.554724in}}{\pgfqpoint{2.267801in}{1.551452in}}{\pgfqpoint{2.276037in}{1.551452in}}%
\pgfpathclose%
\pgfusepath{stroke,fill}%
\end{pgfscope}%
\begin{pgfscope}%
\pgfpathrectangle{\pgfqpoint{0.100000in}{0.212622in}}{\pgfqpoint{3.696000in}{3.696000in}}%
\pgfusepath{clip}%
\pgfsetbuttcap%
\pgfsetroundjoin%
\definecolor{currentfill}{rgb}{0.121569,0.466667,0.705882}%
\pgfsetfillcolor{currentfill}%
\pgfsetfillopacity{0.824521}%
\pgfsetlinewidth{1.003750pt}%
\definecolor{currentstroke}{rgb}{0.121569,0.466667,0.705882}%
\pgfsetstrokecolor{currentstroke}%
\pgfsetstrokeopacity{0.824521}%
\pgfsetdash{}{0pt}%
\pgfpathmoveto{\pgfqpoint{0.655737in}{2.602109in}}%
\pgfpathcurveto{\pgfqpoint{0.663973in}{2.602109in}}{\pgfqpoint{0.671874in}{2.605382in}}{\pgfqpoint{0.677697in}{2.611206in}}%
\pgfpathcurveto{\pgfqpoint{0.683521in}{2.617030in}}{\pgfqpoint{0.686794in}{2.624930in}}{\pgfqpoint{0.686794in}{2.633166in}}%
\pgfpathcurveto{\pgfqpoint{0.686794in}{2.641402in}}{\pgfqpoint{0.683521in}{2.649302in}}{\pgfqpoint{0.677697in}{2.655126in}}%
\pgfpathcurveto{\pgfqpoint{0.671874in}{2.660950in}}{\pgfqpoint{0.663973in}{2.664222in}}{\pgfqpoint{0.655737in}{2.664222in}}%
\pgfpathcurveto{\pgfqpoint{0.647501in}{2.664222in}}{\pgfqpoint{0.639601in}{2.660950in}}{\pgfqpoint{0.633777in}{2.655126in}}%
\pgfpathcurveto{\pgfqpoint{0.627953in}{2.649302in}}{\pgfqpoint{0.624681in}{2.641402in}}{\pgfqpoint{0.624681in}{2.633166in}}%
\pgfpathcurveto{\pgfqpoint{0.624681in}{2.624930in}}{\pgfqpoint{0.627953in}{2.617030in}}{\pgfqpoint{0.633777in}{2.611206in}}%
\pgfpathcurveto{\pgfqpoint{0.639601in}{2.605382in}}{\pgfqpoint{0.647501in}{2.602109in}}{\pgfqpoint{0.655737in}{2.602109in}}%
\pgfpathclose%
\pgfusepath{stroke,fill}%
\end{pgfscope}%
\begin{pgfscope}%
\pgfpathrectangle{\pgfqpoint{0.100000in}{0.212622in}}{\pgfqpoint{3.696000in}{3.696000in}}%
\pgfusepath{clip}%
\pgfsetbuttcap%
\pgfsetroundjoin%
\definecolor{currentfill}{rgb}{0.121569,0.466667,0.705882}%
\pgfsetfillcolor{currentfill}%
\pgfsetfillopacity{0.825157}%
\pgfsetlinewidth{1.003750pt}%
\definecolor{currentstroke}{rgb}{0.121569,0.466667,0.705882}%
\pgfsetstrokecolor{currentstroke}%
\pgfsetstrokeopacity{0.825157}%
\pgfsetdash{}{0pt}%
\pgfpathmoveto{\pgfqpoint{2.276707in}{1.549811in}}%
\pgfpathcurveto{\pgfqpoint{2.284944in}{1.549811in}}{\pgfqpoint{2.292844in}{1.553083in}}{\pgfqpoint{2.298668in}{1.558907in}}%
\pgfpathcurveto{\pgfqpoint{2.304492in}{1.564731in}}{\pgfqpoint{2.307764in}{1.572631in}}{\pgfqpoint{2.307764in}{1.580867in}}%
\pgfpathcurveto{\pgfqpoint{2.307764in}{1.589103in}}{\pgfqpoint{2.304492in}{1.597003in}}{\pgfqpoint{2.298668in}{1.602827in}}%
\pgfpathcurveto{\pgfqpoint{2.292844in}{1.608651in}}{\pgfqpoint{2.284944in}{1.611924in}}{\pgfqpoint{2.276707in}{1.611924in}}%
\pgfpathcurveto{\pgfqpoint{2.268471in}{1.611924in}}{\pgfqpoint{2.260571in}{1.608651in}}{\pgfqpoint{2.254747in}{1.602827in}}%
\pgfpathcurveto{\pgfqpoint{2.248923in}{1.597003in}}{\pgfqpoint{2.245651in}{1.589103in}}{\pgfqpoint{2.245651in}{1.580867in}}%
\pgfpathcurveto{\pgfqpoint{2.245651in}{1.572631in}}{\pgfqpoint{2.248923in}{1.564731in}}{\pgfqpoint{2.254747in}{1.558907in}}%
\pgfpathcurveto{\pgfqpoint{2.260571in}{1.553083in}}{\pgfqpoint{2.268471in}{1.549811in}}{\pgfqpoint{2.276707in}{1.549811in}}%
\pgfpathclose%
\pgfusepath{stroke,fill}%
\end{pgfscope}%
\begin{pgfscope}%
\pgfpathrectangle{\pgfqpoint{0.100000in}{0.212622in}}{\pgfqpoint{3.696000in}{3.696000in}}%
\pgfusepath{clip}%
\pgfsetbuttcap%
\pgfsetroundjoin%
\definecolor{currentfill}{rgb}{0.121569,0.466667,0.705882}%
\pgfsetfillcolor{currentfill}%
\pgfsetfillopacity{0.825850}%
\pgfsetlinewidth{1.003750pt}%
\definecolor{currentstroke}{rgb}{0.121569,0.466667,0.705882}%
\pgfsetstrokecolor{currentstroke}%
\pgfsetstrokeopacity{0.825850}%
\pgfsetdash{}{0pt}%
\pgfpathmoveto{\pgfqpoint{0.668211in}{2.590425in}}%
\pgfpathcurveto{\pgfqpoint{0.676447in}{2.590425in}}{\pgfqpoint{0.684347in}{2.593697in}}{\pgfqpoint{0.690171in}{2.599521in}}%
\pgfpathcurveto{\pgfqpoint{0.695995in}{2.605345in}}{\pgfqpoint{0.699268in}{2.613245in}}{\pgfqpoint{0.699268in}{2.621481in}}%
\pgfpathcurveto{\pgfqpoint{0.699268in}{2.629718in}}{\pgfqpoint{0.695995in}{2.637618in}}{\pgfqpoint{0.690171in}{2.643442in}}%
\pgfpathcurveto{\pgfqpoint{0.684347in}{2.649266in}}{\pgfqpoint{0.676447in}{2.652538in}}{\pgfqpoint{0.668211in}{2.652538in}}%
\pgfpathcurveto{\pgfqpoint{0.659975in}{2.652538in}}{\pgfqpoint{0.652075in}{2.649266in}}{\pgfqpoint{0.646251in}{2.643442in}}%
\pgfpathcurveto{\pgfqpoint{0.640427in}{2.637618in}}{\pgfqpoint{0.637155in}{2.629718in}}{\pgfqpoint{0.637155in}{2.621481in}}%
\pgfpathcurveto{\pgfqpoint{0.637155in}{2.613245in}}{\pgfqpoint{0.640427in}{2.605345in}}{\pgfqpoint{0.646251in}{2.599521in}}%
\pgfpathcurveto{\pgfqpoint{0.652075in}{2.593697in}}{\pgfqpoint{0.659975in}{2.590425in}}{\pgfqpoint{0.668211in}{2.590425in}}%
\pgfpathclose%
\pgfusepath{stroke,fill}%
\end{pgfscope}%
\begin{pgfscope}%
\pgfpathrectangle{\pgfqpoint{0.100000in}{0.212622in}}{\pgfqpoint{3.696000in}{3.696000in}}%
\pgfusepath{clip}%
\pgfsetbuttcap%
\pgfsetroundjoin%
\definecolor{currentfill}{rgb}{0.121569,0.466667,0.705882}%
\pgfsetfillcolor{currentfill}%
\pgfsetfillopacity{0.826165}%
\pgfsetlinewidth{1.003750pt}%
\definecolor{currentstroke}{rgb}{0.121569,0.466667,0.705882}%
\pgfsetstrokecolor{currentstroke}%
\pgfsetstrokeopacity{0.826165}%
\pgfsetdash{}{0pt}%
\pgfpathmoveto{\pgfqpoint{2.278159in}{1.545868in}}%
\pgfpathcurveto{\pgfqpoint{2.286395in}{1.545868in}}{\pgfqpoint{2.294295in}{1.549140in}}{\pgfqpoint{2.300119in}{1.554964in}}%
\pgfpathcurveto{\pgfqpoint{2.305943in}{1.560788in}}{\pgfqpoint{2.309215in}{1.568688in}}{\pgfqpoint{2.309215in}{1.576925in}}%
\pgfpathcurveto{\pgfqpoint{2.309215in}{1.585161in}}{\pgfqpoint{2.305943in}{1.593061in}}{\pgfqpoint{2.300119in}{1.598885in}}%
\pgfpathcurveto{\pgfqpoint{2.294295in}{1.604709in}}{\pgfqpoint{2.286395in}{1.607981in}}{\pgfqpoint{2.278159in}{1.607981in}}%
\pgfpathcurveto{\pgfqpoint{2.269922in}{1.607981in}}{\pgfqpoint{2.262022in}{1.604709in}}{\pgfqpoint{2.256198in}{1.598885in}}%
\pgfpathcurveto{\pgfqpoint{2.250374in}{1.593061in}}{\pgfqpoint{2.247102in}{1.585161in}}{\pgfqpoint{2.247102in}{1.576925in}}%
\pgfpathcurveto{\pgfqpoint{2.247102in}{1.568688in}}{\pgfqpoint{2.250374in}{1.560788in}}{\pgfqpoint{2.256198in}{1.554964in}}%
\pgfpathcurveto{\pgfqpoint{2.262022in}{1.549140in}}{\pgfqpoint{2.269922in}{1.545868in}}{\pgfqpoint{2.278159in}{1.545868in}}%
\pgfpathclose%
\pgfusepath{stroke,fill}%
\end{pgfscope}%
\begin{pgfscope}%
\pgfpathrectangle{\pgfqpoint{0.100000in}{0.212622in}}{\pgfqpoint{3.696000in}{3.696000in}}%
\pgfusepath{clip}%
\pgfsetbuttcap%
\pgfsetroundjoin%
\definecolor{currentfill}{rgb}{0.121569,0.466667,0.705882}%
\pgfsetfillcolor{currentfill}%
\pgfsetfillopacity{0.828077}%
\pgfsetlinewidth{1.003750pt}%
\definecolor{currentstroke}{rgb}{0.121569,0.466667,0.705882}%
\pgfsetstrokecolor{currentstroke}%
\pgfsetstrokeopacity{0.828077}%
\pgfsetdash{}{0pt}%
\pgfpathmoveto{\pgfqpoint{0.679892in}{2.584867in}}%
\pgfpathcurveto{\pgfqpoint{0.688129in}{2.584867in}}{\pgfqpoint{0.696029in}{2.588139in}}{\pgfqpoint{0.701853in}{2.593963in}}%
\pgfpathcurveto{\pgfqpoint{0.707676in}{2.599787in}}{\pgfqpoint{0.710949in}{2.607687in}}{\pgfqpoint{0.710949in}{2.615924in}}%
\pgfpathcurveto{\pgfqpoint{0.710949in}{2.624160in}}{\pgfqpoint{0.707676in}{2.632060in}}{\pgfqpoint{0.701853in}{2.637884in}}%
\pgfpathcurveto{\pgfqpoint{0.696029in}{2.643708in}}{\pgfqpoint{0.688129in}{2.646980in}}{\pgfqpoint{0.679892in}{2.646980in}}%
\pgfpathcurveto{\pgfqpoint{0.671656in}{2.646980in}}{\pgfqpoint{0.663756in}{2.643708in}}{\pgfqpoint{0.657932in}{2.637884in}}%
\pgfpathcurveto{\pgfqpoint{0.652108in}{2.632060in}}{\pgfqpoint{0.648836in}{2.624160in}}{\pgfqpoint{0.648836in}{2.615924in}}%
\pgfpathcurveto{\pgfqpoint{0.648836in}{2.607687in}}{\pgfqpoint{0.652108in}{2.599787in}}{\pgfqpoint{0.657932in}{2.593963in}}%
\pgfpathcurveto{\pgfqpoint{0.663756in}{2.588139in}}{\pgfqpoint{0.671656in}{2.584867in}}{\pgfqpoint{0.679892in}{2.584867in}}%
\pgfpathclose%
\pgfusepath{stroke,fill}%
\end{pgfscope}%
\begin{pgfscope}%
\pgfpathrectangle{\pgfqpoint{0.100000in}{0.212622in}}{\pgfqpoint{3.696000in}{3.696000in}}%
\pgfusepath{clip}%
\pgfsetbuttcap%
\pgfsetroundjoin%
\definecolor{currentfill}{rgb}{0.121569,0.466667,0.705882}%
\pgfsetfillcolor{currentfill}%
\pgfsetfillopacity{0.828215}%
\pgfsetlinewidth{1.003750pt}%
\definecolor{currentstroke}{rgb}{0.121569,0.466667,0.705882}%
\pgfsetstrokecolor{currentstroke}%
\pgfsetstrokeopacity{0.828215}%
\pgfsetdash{}{0pt}%
\pgfpathmoveto{\pgfqpoint{2.279837in}{1.543018in}}%
\pgfpathcurveto{\pgfqpoint{2.288073in}{1.543018in}}{\pgfqpoint{2.295973in}{1.546290in}}{\pgfqpoint{2.301797in}{1.552114in}}%
\pgfpathcurveto{\pgfqpoint{2.307621in}{1.557938in}}{\pgfqpoint{2.310894in}{1.565838in}}{\pgfqpoint{2.310894in}{1.574074in}}%
\pgfpathcurveto{\pgfqpoint{2.310894in}{1.582311in}}{\pgfqpoint{2.307621in}{1.590211in}}{\pgfqpoint{2.301797in}{1.596035in}}%
\pgfpathcurveto{\pgfqpoint{2.295973in}{1.601858in}}{\pgfqpoint{2.288073in}{1.605131in}}{\pgfqpoint{2.279837in}{1.605131in}}%
\pgfpathcurveto{\pgfqpoint{2.271601in}{1.605131in}}{\pgfqpoint{2.263701in}{1.601858in}}{\pgfqpoint{2.257877in}{1.596035in}}%
\pgfpathcurveto{\pgfqpoint{2.252053in}{1.590211in}}{\pgfqpoint{2.248781in}{1.582311in}}{\pgfqpoint{2.248781in}{1.574074in}}%
\pgfpathcurveto{\pgfqpoint{2.248781in}{1.565838in}}{\pgfqpoint{2.252053in}{1.557938in}}{\pgfqpoint{2.257877in}{1.552114in}}%
\pgfpathcurveto{\pgfqpoint{2.263701in}{1.546290in}}{\pgfqpoint{2.271601in}{1.543018in}}{\pgfqpoint{2.279837in}{1.543018in}}%
\pgfpathclose%
\pgfusepath{stroke,fill}%
\end{pgfscope}%
\begin{pgfscope}%
\pgfpathrectangle{\pgfqpoint{0.100000in}{0.212622in}}{\pgfqpoint{3.696000in}{3.696000in}}%
\pgfusepath{clip}%
\pgfsetbuttcap%
\pgfsetroundjoin%
\definecolor{currentfill}{rgb}{0.121569,0.466667,0.705882}%
\pgfsetfillcolor{currentfill}%
\pgfsetfillopacity{0.829644}%
\pgfsetlinewidth{1.003750pt}%
\definecolor{currentstroke}{rgb}{0.121569,0.466667,0.705882}%
\pgfsetstrokecolor{currentstroke}%
\pgfsetstrokeopacity{0.829644}%
\pgfsetdash{}{0pt}%
\pgfpathmoveto{\pgfqpoint{0.702792in}{2.561761in}}%
\pgfpathcurveto{\pgfqpoint{0.711029in}{2.561761in}}{\pgfqpoint{0.718929in}{2.565033in}}{\pgfqpoint{0.724753in}{2.570857in}}%
\pgfpathcurveto{\pgfqpoint{0.730577in}{2.576681in}}{\pgfqpoint{0.733849in}{2.584581in}}{\pgfqpoint{0.733849in}{2.592817in}}%
\pgfpathcurveto{\pgfqpoint{0.733849in}{2.601054in}}{\pgfqpoint{0.730577in}{2.608954in}}{\pgfqpoint{0.724753in}{2.614778in}}%
\pgfpathcurveto{\pgfqpoint{0.718929in}{2.620602in}}{\pgfqpoint{0.711029in}{2.623874in}}{\pgfqpoint{0.702792in}{2.623874in}}%
\pgfpathcurveto{\pgfqpoint{0.694556in}{2.623874in}}{\pgfqpoint{0.686656in}{2.620602in}}{\pgfqpoint{0.680832in}{2.614778in}}%
\pgfpathcurveto{\pgfqpoint{0.675008in}{2.608954in}}{\pgfqpoint{0.671736in}{2.601054in}}{\pgfqpoint{0.671736in}{2.592817in}}%
\pgfpathcurveto{\pgfqpoint{0.671736in}{2.584581in}}{\pgfqpoint{0.675008in}{2.576681in}}{\pgfqpoint{0.680832in}{2.570857in}}%
\pgfpathcurveto{\pgfqpoint{0.686656in}{2.565033in}}{\pgfqpoint{0.694556in}{2.561761in}}{\pgfqpoint{0.702792in}{2.561761in}}%
\pgfpathclose%
\pgfusepath{stroke,fill}%
\end{pgfscope}%
\begin{pgfscope}%
\pgfpathrectangle{\pgfqpoint{0.100000in}{0.212622in}}{\pgfqpoint{3.696000in}{3.696000in}}%
\pgfusepath{clip}%
\pgfsetbuttcap%
\pgfsetroundjoin%
\definecolor{currentfill}{rgb}{0.121569,0.466667,0.705882}%
\pgfsetfillcolor{currentfill}%
\pgfsetfillopacity{0.831031}%
\pgfsetlinewidth{1.003750pt}%
\definecolor{currentstroke}{rgb}{0.121569,0.466667,0.705882}%
\pgfsetstrokecolor{currentstroke}%
\pgfsetstrokeopacity{0.831031}%
\pgfsetdash{}{0pt}%
\pgfpathmoveto{\pgfqpoint{2.280848in}{1.540940in}}%
\pgfpathcurveto{\pgfqpoint{2.289085in}{1.540940in}}{\pgfqpoint{2.296985in}{1.544212in}}{\pgfqpoint{2.302809in}{1.550036in}}%
\pgfpathcurveto{\pgfqpoint{2.308632in}{1.555860in}}{\pgfqpoint{2.311905in}{1.563760in}}{\pgfqpoint{2.311905in}{1.571996in}}%
\pgfpathcurveto{\pgfqpoint{2.311905in}{1.580232in}}{\pgfqpoint{2.308632in}{1.588132in}}{\pgfqpoint{2.302809in}{1.593956in}}%
\pgfpathcurveto{\pgfqpoint{2.296985in}{1.599780in}}{\pgfqpoint{2.289085in}{1.603053in}}{\pgfqpoint{2.280848in}{1.603053in}}%
\pgfpathcurveto{\pgfqpoint{2.272612in}{1.603053in}}{\pgfqpoint{2.264712in}{1.599780in}}{\pgfqpoint{2.258888in}{1.593956in}}%
\pgfpathcurveto{\pgfqpoint{2.253064in}{1.588132in}}{\pgfqpoint{2.249792in}{1.580232in}}{\pgfqpoint{2.249792in}{1.571996in}}%
\pgfpathcurveto{\pgfqpoint{2.249792in}{1.563760in}}{\pgfqpoint{2.253064in}{1.555860in}}{\pgfqpoint{2.258888in}{1.550036in}}%
\pgfpathcurveto{\pgfqpoint{2.264712in}{1.544212in}}{\pgfqpoint{2.272612in}{1.540940in}}{\pgfqpoint{2.280848in}{1.540940in}}%
\pgfpathclose%
\pgfusepath{stroke,fill}%
\end{pgfscope}%
\begin{pgfscope}%
\pgfpathrectangle{\pgfqpoint{0.100000in}{0.212622in}}{\pgfqpoint{3.696000in}{3.696000in}}%
\pgfusepath{clip}%
\pgfsetbuttcap%
\pgfsetroundjoin%
\definecolor{currentfill}{rgb}{0.121569,0.466667,0.705882}%
\pgfsetfillcolor{currentfill}%
\pgfsetfillopacity{0.832474}%
\pgfsetlinewidth{1.003750pt}%
\definecolor{currentstroke}{rgb}{0.121569,0.466667,0.705882}%
\pgfsetstrokecolor{currentstroke}%
\pgfsetstrokeopacity{0.832474}%
\pgfsetdash{}{0pt}%
\pgfpathmoveto{\pgfqpoint{2.282156in}{1.539519in}}%
\pgfpathcurveto{\pgfqpoint{2.290392in}{1.539519in}}{\pgfqpoint{2.298292in}{1.542791in}}{\pgfqpoint{2.304116in}{1.548615in}}%
\pgfpathcurveto{\pgfqpoint{2.309940in}{1.554439in}}{\pgfqpoint{2.313212in}{1.562339in}}{\pgfqpoint{2.313212in}{1.570575in}}%
\pgfpathcurveto{\pgfqpoint{2.313212in}{1.578812in}}{\pgfqpoint{2.309940in}{1.586712in}}{\pgfqpoint{2.304116in}{1.592536in}}%
\pgfpathcurveto{\pgfqpoint{2.298292in}{1.598360in}}{\pgfqpoint{2.290392in}{1.601632in}}{\pgfqpoint{2.282156in}{1.601632in}}%
\pgfpathcurveto{\pgfqpoint{2.273920in}{1.601632in}}{\pgfqpoint{2.266019in}{1.598360in}}{\pgfqpoint{2.260196in}{1.592536in}}%
\pgfpathcurveto{\pgfqpoint{2.254372in}{1.586712in}}{\pgfqpoint{2.251099in}{1.578812in}}{\pgfqpoint{2.251099in}{1.570575in}}%
\pgfpathcurveto{\pgfqpoint{2.251099in}{1.562339in}}{\pgfqpoint{2.254372in}{1.554439in}}{\pgfqpoint{2.260196in}{1.548615in}}%
\pgfpathcurveto{\pgfqpoint{2.266019in}{1.542791in}}{\pgfqpoint{2.273920in}{1.539519in}}{\pgfqpoint{2.282156in}{1.539519in}}%
\pgfpathclose%
\pgfusepath{stroke,fill}%
\end{pgfscope}%
\begin{pgfscope}%
\pgfpathrectangle{\pgfqpoint{0.100000in}{0.212622in}}{\pgfqpoint{3.696000in}{3.696000in}}%
\pgfusepath{clip}%
\pgfsetbuttcap%
\pgfsetroundjoin%
\definecolor{currentfill}{rgb}{0.121569,0.466667,0.705882}%
\pgfsetfillcolor{currentfill}%
\pgfsetfillopacity{0.832991}%
\pgfsetlinewidth{1.003750pt}%
\definecolor{currentstroke}{rgb}{0.121569,0.466667,0.705882}%
\pgfsetstrokecolor{currentstroke}%
\pgfsetstrokeopacity{0.832991}%
\pgfsetdash{}{0pt}%
\pgfpathmoveto{\pgfqpoint{0.743751in}{2.521171in}}%
\pgfpathcurveto{\pgfqpoint{0.751988in}{2.521171in}}{\pgfqpoint{0.759888in}{2.524443in}}{\pgfqpoint{0.765712in}{2.530267in}}%
\pgfpathcurveto{\pgfqpoint{0.771536in}{2.536091in}}{\pgfqpoint{0.774808in}{2.543991in}}{\pgfqpoint{0.774808in}{2.552228in}}%
\pgfpathcurveto{\pgfqpoint{0.774808in}{2.560464in}}{\pgfqpoint{0.771536in}{2.568364in}}{\pgfqpoint{0.765712in}{2.574188in}}%
\pgfpathcurveto{\pgfqpoint{0.759888in}{2.580012in}}{\pgfqpoint{0.751988in}{2.583284in}}{\pgfqpoint{0.743751in}{2.583284in}}%
\pgfpathcurveto{\pgfqpoint{0.735515in}{2.583284in}}{\pgfqpoint{0.727615in}{2.580012in}}{\pgfqpoint{0.721791in}{2.574188in}}%
\pgfpathcurveto{\pgfqpoint{0.715967in}{2.568364in}}{\pgfqpoint{0.712695in}{2.560464in}}{\pgfqpoint{0.712695in}{2.552228in}}%
\pgfpathcurveto{\pgfqpoint{0.712695in}{2.543991in}}{\pgfqpoint{0.715967in}{2.536091in}}{\pgfqpoint{0.721791in}{2.530267in}}%
\pgfpathcurveto{\pgfqpoint{0.727615in}{2.524443in}}{\pgfqpoint{0.735515in}{2.521171in}}{\pgfqpoint{0.743751in}{2.521171in}}%
\pgfpathclose%
\pgfusepath{stroke,fill}%
\end{pgfscope}%
\begin{pgfscope}%
\pgfpathrectangle{\pgfqpoint{0.100000in}{0.212622in}}{\pgfqpoint{3.696000in}{3.696000in}}%
\pgfusepath{clip}%
\pgfsetbuttcap%
\pgfsetroundjoin%
\definecolor{currentfill}{rgb}{0.121569,0.466667,0.705882}%
\pgfsetfillcolor{currentfill}%
\pgfsetfillopacity{0.834216}%
\pgfsetlinewidth{1.003750pt}%
\definecolor{currentstroke}{rgb}{0.121569,0.466667,0.705882}%
\pgfsetstrokecolor{currentstroke}%
\pgfsetstrokeopacity{0.834216}%
\pgfsetdash{}{0pt}%
\pgfpathmoveto{\pgfqpoint{2.283360in}{1.537150in}}%
\pgfpathcurveto{\pgfqpoint{2.291596in}{1.537150in}}{\pgfqpoint{2.299496in}{1.540422in}}{\pgfqpoint{2.305320in}{1.546246in}}%
\pgfpathcurveto{\pgfqpoint{2.311144in}{1.552070in}}{\pgfqpoint{2.314416in}{1.559970in}}{\pgfqpoint{2.314416in}{1.568206in}}%
\pgfpathcurveto{\pgfqpoint{2.314416in}{1.576443in}}{\pgfqpoint{2.311144in}{1.584343in}}{\pgfqpoint{2.305320in}{1.590167in}}%
\pgfpathcurveto{\pgfqpoint{2.299496in}{1.595991in}}{\pgfqpoint{2.291596in}{1.599263in}}{\pgfqpoint{2.283360in}{1.599263in}}%
\pgfpathcurveto{\pgfqpoint{2.275124in}{1.599263in}}{\pgfqpoint{2.267223in}{1.595991in}}{\pgfqpoint{2.261400in}{1.590167in}}%
\pgfpathcurveto{\pgfqpoint{2.255576in}{1.584343in}}{\pgfqpoint{2.252303in}{1.576443in}}{\pgfqpoint{2.252303in}{1.568206in}}%
\pgfpathcurveto{\pgfqpoint{2.252303in}{1.559970in}}{\pgfqpoint{2.255576in}{1.552070in}}{\pgfqpoint{2.261400in}{1.546246in}}%
\pgfpathcurveto{\pgfqpoint{2.267223in}{1.540422in}}{\pgfqpoint{2.275124in}{1.537150in}}{\pgfqpoint{2.283360in}{1.537150in}}%
\pgfpathclose%
\pgfusepath{stroke,fill}%
\end{pgfscope}%
\begin{pgfscope}%
\pgfpathrectangle{\pgfqpoint{0.100000in}{0.212622in}}{\pgfqpoint{3.696000in}{3.696000in}}%
\pgfusepath{clip}%
\pgfsetbuttcap%
\pgfsetroundjoin%
\definecolor{currentfill}{rgb}{0.121569,0.466667,0.705882}%
\pgfsetfillcolor{currentfill}%
\pgfsetfillopacity{0.835084}%
\pgfsetlinewidth{1.003750pt}%
\definecolor{currentstroke}{rgb}{0.121569,0.466667,0.705882}%
\pgfsetstrokecolor{currentstroke}%
\pgfsetstrokeopacity{0.835084}%
\pgfsetdash{}{0pt}%
\pgfpathmoveto{\pgfqpoint{2.283954in}{1.535256in}}%
\pgfpathcurveto{\pgfqpoint{2.292190in}{1.535256in}}{\pgfqpoint{2.300090in}{1.538529in}}{\pgfqpoint{2.305914in}{1.544353in}}%
\pgfpathcurveto{\pgfqpoint{2.311738in}{1.550177in}}{\pgfqpoint{2.315010in}{1.558077in}}{\pgfqpoint{2.315010in}{1.566313in}}%
\pgfpathcurveto{\pgfqpoint{2.315010in}{1.574549in}}{\pgfqpoint{2.311738in}{1.582449in}}{\pgfqpoint{2.305914in}{1.588273in}}%
\pgfpathcurveto{\pgfqpoint{2.300090in}{1.594097in}}{\pgfqpoint{2.292190in}{1.597369in}}{\pgfqpoint{2.283954in}{1.597369in}}%
\pgfpathcurveto{\pgfqpoint{2.275717in}{1.597369in}}{\pgfqpoint{2.267817in}{1.594097in}}{\pgfqpoint{2.261993in}{1.588273in}}%
\pgfpathcurveto{\pgfqpoint{2.256169in}{1.582449in}}{\pgfqpoint{2.252897in}{1.574549in}}{\pgfqpoint{2.252897in}{1.566313in}}%
\pgfpathcurveto{\pgfqpoint{2.252897in}{1.558077in}}{\pgfqpoint{2.256169in}{1.550177in}}{\pgfqpoint{2.261993in}{1.544353in}}%
\pgfpathcurveto{\pgfqpoint{2.267817in}{1.538529in}}{\pgfqpoint{2.275717in}{1.535256in}}{\pgfqpoint{2.283954in}{1.535256in}}%
\pgfpathclose%
\pgfusepath{stroke,fill}%
\end{pgfscope}%
\begin{pgfscope}%
\pgfpathrectangle{\pgfqpoint{0.100000in}{0.212622in}}{\pgfqpoint{3.696000in}{3.696000in}}%
\pgfusepath{clip}%
\pgfsetbuttcap%
\pgfsetroundjoin%
\definecolor{currentfill}{rgb}{0.121569,0.466667,0.705882}%
\pgfsetfillcolor{currentfill}%
\pgfsetfillopacity{0.836412}%
\pgfsetlinewidth{1.003750pt}%
\definecolor{currentstroke}{rgb}{0.121569,0.466667,0.705882}%
\pgfsetstrokecolor{currentstroke}%
\pgfsetstrokeopacity{0.836412}%
\pgfsetdash{}{0pt}%
\pgfpathmoveto{\pgfqpoint{2.284390in}{1.534508in}}%
\pgfpathcurveto{\pgfqpoint{2.292626in}{1.534508in}}{\pgfqpoint{2.300526in}{1.537781in}}{\pgfqpoint{2.306350in}{1.543605in}}%
\pgfpathcurveto{\pgfqpoint{2.312174in}{1.549428in}}{\pgfqpoint{2.315446in}{1.557329in}}{\pgfqpoint{2.315446in}{1.565565in}}%
\pgfpathcurveto{\pgfqpoint{2.315446in}{1.573801in}}{\pgfqpoint{2.312174in}{1.581701in}}{\pgfqpoint{2.306350in}{1.587525in}}%
\pgfpathcurveto{\pgfqpoint{2.300526in}{1.593349in}}{\pgfqpoint{2.292626in}{1.596621in}}{\pgfqpoint{2.284390in}{1.596621in}}%
\pgfpathcurveto{\pgfqpoint{2.276153in}{1.596621in}}{\pgfqpoint{2.268253in}{1.593349in}}{\pgfqpoint{2.262429in}{1.587525in}}%
\pgfpathcurveto{\pgfqpoint{2.256606in}{1.581701in}}{\pgfqpoint{2.253333in}{1.573801in}}{\pgfqpoint{2.253333in}{1.565565in}}%
\pgfpathcurveto{\pgfqpoint{2.253333in}{1.557329in}}{\pgfqpoint{2.256606in}{1.549428in}}{\pgfqpoint{2.262429in}{1.543605in}}%
\pgfpathcurveto{\pgfqpoint{2.268253in}{1.537781in}}{\pgfqpoint{2.276153in}{1.534508in}}{\pgfqpoint{2.284390in}{1.534508in}}%
\pgfpathclose%
\pgfusepath{stroke,fill}%
\end{pgfscope}%
\begin{pgfscope}%
\pgfpathrectangle{\pgfqpoint{0.100000in}{0.212622in}}{\pgfqpoint{3.696000in}{3.696000in}}%
\pgfusepath{clip}%
\pgfsetbuttcap%
\pgfsetroundjoin%
\definecolor{currentfill}{rgb}{0.121569,0.466667,0.705882}%
\pgfsetfillcolor{currentfill}%
\pgfsetfillopacity{0.836735}%
\pgfsetlinewidth{1.003750pt}%
\definecolor{currentstroke}{rgb}{0.121569,0.466667,0.705882}%
\pgfsetstrokecolor{currentstroke}%
\pgfsetstrokeopacity{0.836735}%
\pgfsetdash{}{0pt}%
\pgfpathmoveto{\pgfqpoint{0.782231in}{2.478509in}}%
\pgfpathcurveto{\pgfqpoint{0.790467in}{2.478509in}}{\pgfqpoint{0.798367in}{2.481781in}}{\pgfqpoint{0.804191in}{2.487605in}}%
\pgfpathcurveto{\pgfqpoint{0.810015in}{2.493429in}}{\pgfqpoint{0.813287in}{2.501329in}}{\pgfqpoint{0.813287in}{2.509565in}}%
\pgfpathcurveto{\pgfqpoint{0.813287in}{2.517801in}}{\pgfqpoint{0.810015in}{2.525701in}}{\pgfqpoint{0.804191in}{2.531525in}}%
\pgfpathcurveto{\pgfqpoint{0.798367in}{2.537349in}}{\pgfqpoint{0.790467in}{2.540622in}}{\pgfqpoint{0.782231in}{2.540622in}}%
\pgfpathcurveto{\pgfqpoint{0.773995in}{2.540622in}}{\pgfqpoint{0.766095in}{2.537349in}}{\pgfqpoint{0.760271in}{2.531525in}}%
\pgfpathcurveto{\pgfqpoint{0.754447in}{2.525701in}}{\pgfqpoint{0.751174in}{2.517801in}}{\pgfqpoint{0.751174in}{2.509565in}}%
\pgfpathcurveto{\pgfqpoint{0.751174in}{2.501329in}}{\pgfqpoint{0.754447in}{2.493429in}}{\pgfqpoint{0.760271in}{2.487605in}}%
\pgfpathcurveto{\pgfqpoint{0.766095in}{2.481781in}}{\pgfqpoint{0.773995in}{2.478509in}}{\pgfqpoint{0.782231in}{2.478509in}}%
\pgfpathclose%
\pgfusepath{stroke,fill}%
\end{pgfscope}%
\begin{pgfscope}%
\pgfpathrectangle{\pgfqpoint{0.100000in}{0.212622in}}{\pgfqpoint{3.696000in}{3.696000in}}%
\pgfusepath{clip}%
\pgfsetbuttcap%
\pgfsetroundjoin%
\definecolor{currentfill}{rgb}{0.121569,0.466667,0.705882}%
\pgfsetfillcolor{currentfill}%
\pgfsetfillopacity{0.837110}%
\pgfsetlinewidth{1.003750pt}%
\definecolor{currentstroke}{rgb}{0.121569,0.466667,0.705882}%
\pgfsetstrokecolor{currentstroke}%
\pgfsetstrokeopacity{0.837110}%
\pgfsetdash{}{0pt}%
\pgfpathmoveto{\pgfqpoint{2.284820in}{1.533973in}}%
\pgfpathcurveto{\pgfqpoint{2.293056in}{1.533973in}}{\pgfqpoint{2.300956in}{1.537245in}}{\pgfqpoint{2.306780in}{1.543069in}}%
\pgfpathcurveto{\pgfqpoint{2.312604in}{1.548893in}}{\pgfqpoint{2.315877in}{1.556793in}}{\pgfqpoint{2.315877in}{1.565029in}}%
\pgfpathcurveto{\pgfqpoint{2.315877in}{1.573266in}}{\pgfqpoint{2.312604in}{1.581166in}}{\pgfqpoint{2.306780in}{1.586990in}}%
\pgfpathcurveto{\pgfqpoint{2.300956in}{1.592814in}}{\pgfqpoint{2.293056in}{1.596086in}}{\pgfqpoint{2.284820in}{1.596086in}}%
\pgfpathcurveto{\pgfqpoint{2.276584in}{1.596086in}}{\pgfqpoint{2.268684in}{1.592814in}}{\pgfqpoint{2.262860in}{1.586990in}}%
\pgfpathcurveto{\pgfqpoint{2.257036in}{1.581166in}}{\pgfqpoint{2.253764in}{1.573266in}}{\pgfqpoint{2.253764in}{1.565029in}}%
\pgfpathcurveto{\pgfqpoint{2.253764in}{1.556793in}}{\pgfqpoint{2.257036in}{1.548893in}}{\pgfqpoint{2.262860in}{1.543069in}}%
\pgfpathcurveto{\pgfqpoint{2.268684in}{1.537245in}}{\pgfqpoint{2.276584in}{1.533973in}}{\pgfqpoint{2.284820in}{1.533973in}}%
\pgfpathclose%
\pgfusepath{stroke,fill}%
\end{pgfscope}%
\begin{pgfscope}%
\pgfpathrectangle{\pgfqpoint{0.100000in}{0.212622in}}{\pgfqpoint{3.696000in}{3.696000in}}%
\pgfusepath{clip}%
\pgfsetbuttcap%
\pgfsetroundjoin%
\definecolor{currentfill}{rgb}{0.121569,0.466667,0.705882}%
\pgfsetfillcolor{currentfill}%
\pgfsetfillopacity{0.837952}%
\pgfsetlinewidth{1.003750pt}%
\definecolor{currentstroke}{rgb}{0.121569,0.466667,0.705882}%
\pgfsetstrokecolor{currentstroke}%
\pgfsetstrokeopacity{0.837952}%
\pgfsetdash{}{0pt}%
\pgfpathmoveto{\pgfqpoint{2.285540in}{1.532352in}}%
\pgfpathcurveto{\pgfqpoint{2.293776in}{1.532352in}}{\pgfqpoint{2.301676in}{1.535624in}}{\pgfqpoint{2.307500in}{1.541448in}}%
\pgfpathcurveto{\pgfqpoint{2.313324in}{1.547272in}}{\pgfqpoint{2.316596in}{1.555172in}}{\pgfqpoint{2.316596in}{1.563409in}}%
\pgfpathcurveto{\pgfqpoint{2.316596in}{1.571645in}}{\pgfqpoint{2.313324in}{1.579545in}}{\pgfqpoint{2.307500in}{1.585369in}}%
\pgfpathcurveto{\pgfqpoint{2.301676in}{1.591193in}}{\pgfqpoint{2.293776in}{1.594465in}}{\pgfqpoint{2.285540in}{1.594465in}}%
\pgfpathcurveto{\pgfqpoint{2.277303in}{1.594465in}}{\pgfqpoint{2.269403in}{1.591193in}}{\pgfqpoint{2.263579in}{1.585369in}}%
\pgfpathcurveto{\pgfqpoint{2.257756in}{1.579545in}}{\pgfqpoint{2.254483in}{1.571645in}}{\pgfqpoint{2.254483in}{1.563409in}}%
\pgfpathcurveto{\pgfqpoint{2.254483in}{1.555172in}}{\pgfqpoint{2.257756in}{1.547272in}}{\pgfqpoint{2.263579in}{1.541448in}}%
\pgfpathcurveto{\pgfqpoint{2.269403in}{1.535624in}}{\pgfqpoint{2.277303in}{1.532352in}}{\pgfqpoint{2.285540in}{1.532352in}}%
\pgfpathclose%
\pgfusepath{stroke,fill}%
\end{pgfscope}%
\begin{pgfscope}%
\pgfpathrectangle{\pgfqpoint{0.100000in}{0.212622in}}{\pgfqpoint{3.696000in}{3.696000in}}%
\pgfusepath{clip}%
\pgfsetbuttcap%
\pgfsetroundjoin%
\definecolor{currentfill}{rgb}{0.121569,0.466667,0.705882}%
\pgfsetfillcolor{currentfill}%
\pgfsetfillopacity{0.838946}%
\pgfsetlinewidth{1.003750pt}%
\definecolor{currentstroke}{rgb}{0.121569,0.466667,0.705882}%
\pgfsetstrokecolor{currentstroke}%
\pgfsetstrokeopacity{0.838946}%
\pgfsetdash{}{0pt}%
\pgfpathmoveto{\pgfqpoint{2.286430in}{1.528971in}}%
\pgfpathcurveto{\pgfqpoint{2.294667in}{1.528971in}}{\pgfqpoint{2.302567in}{1.532243in}}{\pgfqpoint{2.308391in}{1.538067in}}%
\pgfpathcurveto{\pgfqpoint{2.314214in}{1.543891in}}{\pgfqpoint{2.317487in}{1.551791in}}{\pgfqpoint{2.317487in}{1.560028in}}%
\pgfpathcurveto{\pgfqpoint{2.317487in}{1.568264in}}{\pgfqpoint{2.314214in}{1.576164in}}{\pgfqpoint{2.308391in}{1.581988in}}%
\pgfpathcurveto{\pgfqpoint{2.302567in}{1.587812in}}{\pgfqpoint{2.294667in}{1.591084in}}{\pgfqpoint{2.286430in}{1.591084in}}%
\pgfpathcurveto{\pgfqpoint{2.278194in}{1.591084in}}{\pgfqpoint{2.270294in}{1.587812in}}{\pgfqpoint{2.264470in}{1.581988in}}%
\pgfpathcurveto{\pgfqpoint{2.258646in}{1.576164in}}{\pgfqpoint{2.255374in}{1.568264in}}{\pgfqpoint{2.255374in}{1.560028in}}%
\pgfpathcurveto{\pgfqpoint{2.255374in}{1.551791in}}{\pgfqpoint{2.258646in}{1.543891in}}{\pgfqpoint{2.264470in}{1.538067in}}%
\pgfpathcurveto{\pgfqpoint{2.270294in}{1.532243in}}{\pgfqpoint{2.278194in}{1.528971in}}{\pgfqpoint{2.286430in}{1.528971in}}%
\pgfpathclose%
\pgfusepath{stroke,fill}%
\end{pgfscope}%
\begin{pgfscope}%
\pgfpathrectangle{\pgfqpoint{0.100000in}{0.212622in}}{\pgfqpoint{3.696000in}{3.696000in}}%
\pgfusepath{clip}%
\pgfsetbuttcap%
\pgfsetroundjoin%
\definecolor{currentfill}{rgb}{0.121569,0.466667,0.705882}%
\pgfsetfillcolor{currentfill}%
\pgfsetfillopacity{0.841148}%
\pgfsetlinewidth{1.003750pt}%
\definecolor{currentstroke}{rgb}{0.121569,0.466667,0.705882}%
\pgfsetstrokecolor{currentstroke}%
\pgfsetstrokeopacity{0.841148}%
\pgfsetdash{}{0pt}%
\pgfpathmoveto{\pgfqpoint{0.823209in}{2.450807in}}%
\pgfpathcurveto{\pgfqpoint{0.831445in}{2.450807in}}{\pgfqpoint{0.839345in}{2.454080in}}{\pgfqpoint{0.845169in}{2.459904in}}%
\pgfpathcurveto{\pgfqpoint{0.850993in}{2.465727in}}{\pgfqpoint{0.854265in}{2.473628in}}{\pgfqpoint{0.854265in}{2.481864in}}%
\pgfpathcurveto{\pgfqpoint{0.854265in}{2.490100in}}{\pgfqpoint{0.850993in}{2.498000in}}{\pgfqpoint{0.845169in}{2.503824in}}%
\pgfpathcurveto{\pgfqpoint{0.839345in}{2.509648in}}{\pgfqpoint{0.831445in}{2.512920in}}{\pgfqpoint{0.823209in}{2.512920in}}%
\pgfpathcurveto{\pgfqpoint{0.814973in}{2.512920in}}{\pgfqpoint{0.807073in}{2.509648in}}{\pgfqpoint{0.801249in}{2.503824in}}%
\pgfpathcurveto{\pgfqpoint{0.795425in}{2.498000in}}{\pgfqpoint{0.792152in}{2.490100in}}{\pgfqpoint{0.792152in}{2.481864in}}%
\pgfpathcurveto{\pgfqpoint{0.792152in}{2.473628in}}{\pgfqpoint{0.795425in}{2.465727in}}{\pgfqpoint{0.801249in}{2.459904in}}%
\pgfpathcurveto{\pgfqpoint{0.807073in}{2.454080in}}{\pgfqpoint{0.814973in}{2.450807in}}{\pgfqpoint{0.823209in}{2.450807in}}%
\pgfpathclose%
\pgfusepath{stroke,fill}%
\end{pgfscope}%
\begin{pgfscope}%
\pgfpathrectangle{\pgfqpoint{0.100000in}{0.212622in}}{\pgfqpoint{3.696000in}{3.696000in}}%
\pgfusepath{clip}%
\pgfsetbuttcap%
\pgfsetroundjoin%
\definecolor{currentfill}{rgb}{0.121569,0.466667,0.705882}%
\pgfsetfillcolor{currentfill}%
\pgfsetfillopacity{0.841200}%
\pgfsetlinewidth{1.003750pt}%
\definecolor{currentstroke}{rgb}{0.121569,0.466667,0.705882}%
\pgfsetstrokecolor{currentstroke}%
\pgfsetstrokeopacity{0.841200}%
\pgfsetdash{}{0pt}%
\pgfpathmoveto{\pgfqpoint{2.288098in}{1.527617in}}%
\pgfpathcurveto{\pgfqpoint{2.296334in}{1.527617in}}{\pgfqpoint{2.304234in}{1.530889in}}{\pgfqpoint{2.310058in}{1.536713in}}%
\pgfpathcurveto{\pgfqpoint{2.315882in}{1.542537in}}{\pgfqpoint{2.319154in}{1.550437in}}{\pgfqpoint{2.319154in}{1.558673in}}%
\pgfpathcurveto{\pgfqpoint{2.319154in}{1.566910in}}{\pgfqpoint{2.315882in}{1.574810in}}{\pgfqpoint{2.310058in}{1.580634in}}%
\pgfpathcurveto{\pgfqpoint{2.304234in}{1.586458in}}{\pgfqpoint{2.296334in}{1.589730in}}{\pgfqpoint{2.288098in}{1.589730in}}%
\pgfpathcurveto{\pgfqpoint{2.279861in}{1.589730in}}{\pgfqpoint{2.271961in}{1.586458in}}{\pgfqpoint{2.266137in}{1.580634in}}%
\pgfpathcurveto{\pgfqpoint{2.260313in}{1.574810in}}{\pgfqpoint{2.257041in}{1.566910in}}{\pgfqpoint{2.257041in}{1.558673in}}%
\pgfpathcurveto{\pgfqpoint{2.257041in}{1.550437in}}{\pgfqpoint{2.260313in}{1.542537in}}{\pgfqpoint{2.266137in}{1.536713in}}%
\pgfpathcurveto{\pgfqpoint{2.271961in}{1.530889in}}{\pgfqpoint{2.279861in}{1.527617in}}{\pgfqpoint{2.288098in}{1.527617in}}%
\pgfpathclose%
\pgfusepath{stroke,fill}%
\end{pgfscope}%
\begin{pgfscope}%
\pgfpathrectangle{\pgfqpoint{0.100000in}{0.212622in}}{\pgfqpoint{3.696000in}{3.696000in}}%
\pgfusepath{clip}%
\pgfsetbuttcap%
\pgfsetroundjoin%
\definecolor{currentfill}{rgb}{0.121569,0.466667,0.705882}%
\pgfsetfillcolor{currentfill}%
\pgfsetfillopacity{0.843928}%
\pgfsetlinewidth{1.003750pt}%
\definecolor{currentstroke}{rgb}{0.121569,0.466667,0.705882}%
\pgfsetstrokecolor{currentstroke}%
\pgfsetstrokeopacity{0.843928}%
\pgfsetdash{}{0pt}%
\pgfpathmoveto{\pgfqpoint{2.289429in}{1.527027in}}%
\pgfpathcurveto{\pgfqpoint{2.297665in}{1.527027in}}{\pgfqpoint{2.305565in}{1.530300in}}{\pgfqpoint{2.311389in}{1.536124in}}%
\pgfpathcurveto{\pgfqpoint{2.317213in}{1.541947in}}{\pgfqpoint{2.320485in}{1.549848in}}{\pgfqpoint{2.320485in}{1.558084in}}%
\pgfpathcurveto{\pgfqpoint{2.320485in}{1.566320in}}{\pgfqpoint{2.317213in}{1.574220in}}{\pgfqpoint{2.311389in}{1.580044in}}%
\pgfpathcurveto{\pgfqpoint{2.305565in}{1.585868in}}{\pgfqpoint{2.297665in}{1.589140in}}{\pgfqpoint{2.289429in}{1.589140in}}%
\pgfpathcurveto{\pgfqpoint{2.281192in}{1.589140in}}{\pgfqpoint{2.273292in}{1.585868in}}{\pgfqpoint{2.267468in}{1.580044in}}%
\pgfpathcurveto{\pgfqpoint{2.261644in}{1.574220in}}{\pgfqpoint{2.258372in}{1.566320in}}{\pgfqpoint{2.258372in}{1.558084in}}%
\pgfpathcurveto{\pgfqpoint{2.258372in}{1.549848in}}{\pgfqpoint{2.261644in}{1.541947in}}{\pgfqpoint{2.267468in}{1.536124in}}%
\pgfpathcurveto{\pgfqpoint{2.273292in}{1.530300in}}{\pgfqpoint{2.281192in}{1.527027in}}{\pgfqpoint{2.289429in}{1.527027in}}%
\pgfpathclose%
\pgfusepath{stroke,fill}%
\end{pgfscope}%
\begin{pgfscope}%
\pgfpathrectangle{\pgfqpoint{0.100000in}{0.212622in}}{\pgfqpoint{3.696000in}{3.696000in}}%
\pgfusepath{clip}%
\pgfsetbuttcap%
\pgfsetroundjoin%
\definecolor{currentfill}{rgb}{0.121569,0.466667,0.705882}%
\pgfsetfillcolor{currentfill}%
\pgfsetfillopacity{0.845372}%
\pgfsetlinewidth{1.003750pt}%
\definecolor{currentstroke}{rgb}{0.121569,0.466667,0.705882}%
\pgfsetstrokecolor{currentstroke}%
\pgfsetstrokeopacity{0.845372}%
\pgfsetdash{}{0pt}%
\pgfpathmoveto{\pgfqpoint{0.862418in}{2.426058in}}%
\pgfpathcurveto{\pgfqpoint{0.870654in}{2.426058in}}{\pgfqpoint{0.878554in}{2.429330in}}{\pgfqpoint{0.884378in}{2.435154in}}%
\pgfpathcurveto{\pgfqpoint{0.890202in}{2.440978in}}{\pgfqpoint{0.893474in}{2.448878in}}{\pgfqpoint{0.893474in}{2.457114in}}%
\pgfpathcurveto{\pgfqpoint{0.893474in}{2.465350in}}{\pgfqpoint{0.890202in}{2.473250in}}{\pgfqpoint{0.884378in}{2.479074in}}%
\pgfpathcurveto{\pgfqpoint{0.878554in}{2.484898in}}{\pgfqpoint{0.870654in}{2.488171in}}{\pgfqpoint{0.862418in}{2.488171in}}%
\pgfpathcurveto{\pgfqpoint{0.854181in}{2.488171in}}{\pgfqpoint{0.846281in}{2.484898in}}{\pgfqpoint{0.840457in}{2.479074in}}%
\pgfpathcurveto{\pgfqpoint{0.834633in}{2.473250in}}{\pgfqpoint{0.831361in}{2.465350in}}{\pgfqpoint{0.831361in}{2.457114in}}%
\pgfpathcurveto{\pgfqpoint{0.831361in}{2.448878in}}{\pgfqpoint{0.834633in}{2.440978in}}{\pgfqpoint{0.840457in}{2.435154in}}%
\pgfpathcurveto{\pgfqpoint{0.846281in}{2.429330in}}{\pgfqpoint{0.854181in}{2.426058in}}{\pgfqpoint{0.862418in}{2.426058in}}%
\pgfpathclose%
\pgfusepath{stroke,fill}%
\end{pgfscope}%
\begin{pgfscope}%
\pgfpathrectangle{\pgfqpoint{0.100000in}{0.212622in}}{\pgfqpoint{3.696000in}{3.696000in}}%
\pgfusepath{clip}%
\pgfsetbuttcap%
\pgfsetroundjoin%
\definecolor{currentfill}{rgb}{0.121569,0.466667,0.705882}%
\pgfsetfillcolor{currentfill}%
\pgfsetfillopacity{0.846416}%
\pgfsetlinewidth{1.003750pt}%
\definecolor{currentstroke}{rgb}{0.121569,0.466667,0.705882}%
\pgfsetstrokecolor{currentstroke}%
\pgfsetstrokeopacity{0.846416}%
\pgfsetdash{}{0pt}%
\pgfpathmoveto{\pgfqpoint{2.291351in}{1.524045in}}%
\pgfpathcurveto{\pgfqpoint{2.299587in}{1.524045in}}{\pgfqpoint{2.307487in}{1.527318in}}{\pgfqpoint{2.313311in}{1.533142in}}%
\pgfpathcurveto{\pgfqpoint{2.319135in}{1.538966in}}{\pgfqpoint{2.322407in}{1.546866in}}{\pgfqpoint{2.322407in}{1.555102in}}%
\pgfpathcurveto{\pgfqpoint{2.322407in}{1.563338in}}{\pgfqpoint{2.319135in}{1.571238in}}{\pgfqpoint{2.313311in}{1.577062in}}%
\pgfpathcurveto{\pgfqpoint{2.307487in}{1.582886in}}{\pgfqpoint{2.299587in}{1.586158in}}{\pgfqpoint{2.291351in}{1.586158in}}%
\pgfpathcurveto{\pgfqpoint{2.283114in}{1.586158in}}{\pgfqpoint{2.275214in}{1.582886in}}{\pgfqpoint{2.269390in}{1.577062in}}%
\pgfpathcurveto{\pgfqpoint{2.263567in}{1.571238in}}{\pgfqpoint{2.260294in}{1.563338in}}{\pgfqpoint{2.260294in}{1.555102in}}%
\pgfpathcurveto{\pgfqpoint{2.260294in}{1.546866in}}{\pgfqpoint{2.263567in}{1.538966in}}{\pgfqpoint{2.269390in}{1.533142in}}%
\pgfpathcurveto{\pgfqpoint{2.275214in}{1.527318in}}{\pgfqpoint{2.283114in}{1.524045in}}{\pgfqpoint{2.291351in}{1.524045in}}%
\pgfpathclose%
\pgfusepath{stroke,fill}%
\end{pgfscope}%
\begin{pgfscope}%
\pgfpathrectangle{\pgfqpoint{0.100000in}{0.212622in}}{\pgfqpoint{3.696000in}{3.696000in}}%
\pgfusepath{clip}%
\pgfsetbuttcap%
\pgfsetroundjoin%
\definecolor{currentfill}{rgb}{0.121569,0.466667,0.705882}%
\pgfsetfillcolor{currentfill}%
\pgfsetfillopacity{0.849134}%
\pgfsetlinewidth{1.003750pt}%
\definecolor{currentstroke}{rgb}{0.121569,0.466667,0.705882}%
\pgfsetstrokecolor{currentstroke}%
\pgfsetstrokeopacity{0.849134}%
\pgfsetdash{}{0pt}%
\pgfpathmoveto{\pgfqpoint{2.293487in}{1.520671in}}%
\pgfpathcurveto{\pgfqpoint{2.301723in}{1.520671in}}{\pgfqpoint{2.309623in}{1.523944in}}{\pgfqpoint{2.315447in}{1.529768in}}%
\pgfpathcurveto{\pgfqpoint{2.321271in}{1.535591in}}{\pgfqpoint{2.324543in}{1.543492in}}{\pgfqpoint{2.324543in}{1.551728in}}%
\pgfpathcurveto{\pgfqpoint{2.324543in}{1.559964in}}{\pgfqpoint{2.321271in}{1.567864in}}{\pgfqpoint{2.315447in}{1.573688in}}%
\pgfpathcurveto{\pgfqpoint{2.309623in}{1.579512in}}{\pgfqpoint{2.301723in}{1.582784in}}{\pgfqpoint{2.293487in}{1.582784in}}%
\pgfpathcurveto{\pgfqpoint{2.285250in}{1.582784in}}{\pgfqpoint{2.277350in}{1.579512in}}{\pgfqpoint{2.271526in}{1.573688in}}%
\pgfpathcurveto{\pgfqpoint{2.265702in}{1.567864in}}{\pgfqpoint{2.262430in}{1.559964in}}{\pgfqpoint{2.262430in}{1.551728in}}%
\pgfpathcurveto{\pgfqpoint{2.262430in}{1.543492in}}{\pgfqpoint{2.265702in}{1.535591in}}{\pgfqpoint{2.271526in}{1.529768in}}%
\pgfpathcurveto{\pgfqpoint{2.277350in}{1.523944in}}{\pgfqpoint{2.285250in}{1.520671in}}{\pgfqpoint{2.293487in}{1.520671in}}%
\pgfpathclose%
\pgfusepath{stroke,fill}%
\end{pgfscope}%
\begin{pgfscope}%
\pgfpathrectangle{\pgfqpoint{0.100000in}{0.212622in}}{\pgfqpoint{3.696000in}{3.696000in}}%
\pgfusepath{clip}%
\pgfsetbuttcap%
\pgfsetroundjoin%
\definecolor{currentfill}{rgb}{0.121569,0.466667,0.705882}%
\pgfsetfillcolor{currentfill}%
\pgfsetfillopacity{0.849237}%
\pgfsetlinewidth{1.003750pt}%
\definecolor{currentstroke}{rgb}{0.121569,0.466667,0.705882}%
\pgfsetstrokecolor{currentstroke}%
\pgfsetstrokeopacity{0.849237}%
\pgfsetdash{}{0pt}%
\pgfpathmoveto{\pgfqpoint{0.895559in}{2.394320in}}%
\pgfpathcurveto{\pgfqpoint{0.903795in}{2.394320in}}{\pgfqpoint{0.911695in}{2.397592in}}{\pgfqpoint{0.917519in}{2.403416in}}%
\pgfpathcurveto{\pgfqpoint{0.923343in}{2.409240in}}{\pgfqpoint{0.926616in}{2.417140in}}{\pgfqpoint{0.926616in}{2.425376in}}%
\pgfpathcurveto{\pgfqpoint{0.926616in}{2.433613in}}{\pgfqpoint{0.923343in}{2.441513in}}{\pgfqpoint{0.917519in}{2.447337in}}%
\pgfpathcurveto{\pgfqpoint{0.911695in}{2.453160in}}{\pgfqpoint{0.903795in}{2.456433in}}{\pgfqpoint{0.895559in}{2.456433in}}%
\pgfpathcurveto{\pgfqpoint{0.887323in}{2.456433in}}{\pgfqpoint{0.879423in}{2.453160in}}{\pgfqpoint{0.873599in}{2.447337in}}%
\pgfpathcurveto{\pgfqpoint{0.867775in}{2.441513in}}{\pgfqpoint{0.864503in}{2.433613in}}{\pgfqpoint{0.864503in}{2.425376in}}%
\pgfpathcurveto{\pgfqpoint{0.864503in}{2.417140in}}{\pgfqpoint{0.867775in}{2.409240in}}{\pgfqpoint{0.873599in}{2.403416in}}%
\pgfpathcurveto{\pgfqpoint{0.879423in}{2.397592in}}{\pgfqpoint{0.887323in}{2.394320in}}{\pgfqpoint{0.895559in}{2.394320in}}%
\pgfpathclose%
\pgfusepath{stroke,fill}%
\end{pgfscope}%
\begin{pgfscope}%
\pgfpathrectangle{\pgfqpoint{0.100000in}{0.212622in}}{\pgfqpoint{3.696000in}{3.696000in}}%
\pgfusepath{clip}%
\pgfsetbuttcap%
\pgfsetroundjoin%
\definecolor{currentfill}{rgb}{0.121569,0.466667,0.705882}%
\pgfsetfillcolor{currentfill}%
\pgfsetfillopacity{0.851924}%
\pgfsetlinewidth{1.003750pt}%
\definecolor{currentstroke}{rgb}{0.121569,0.466667,0.705882}%
\pgfsetstrokecolor{currentstroke}%
\pgfsetstrokeopacity{0.851924}%
\pgfsetdash{}{0pt}%
\pgfpathmoveto{\pgfqpoint{2.295865in}{1.516210in}}%
\pgfpathcurveto{\pgfqpoint{2.304101in}{1.516210in}}{\pgfqpoint{2.312001in}{1.519482in}}{\pgfqpoint{2.317825in}{1.525306in}}%
\pgfpathcurveto{\pgfqpoint{2.323649in}{1.531130in}}{\pgfqpoint{2.326921in}{1.539030in}}{\pgfqpoint{2.326921in}{1.547266in}}%
\pgfpathcurveto{\pgfqpoint{2.326921in}{1.555502in}}{\pgfqpoint{2.323649in}{1.563402in}}{\pgfqpoint{2.317825in}{1.569226in}}%
\pgfpathcurveto{\pgfqpoint{2.312001in}{1.575050in}}{\pgfqpoint{2.304101in}{1.578323in}}{\pgfqpoint{2.295865in}{1.578323in}}%
\pgfpathcurveto{\pgfqpoint{2.287628in}{1.578323in}}{\pgfqpoint{2.279728in}{1.575050in}}{\pgfqpoint{2.273904in}{1.569226in}}%
\pgfpathcurveto{\pgfqpoint{2.268081in}{1.563402in}}{\pgfqpoint{2.264808in}{1.555502in}}{\pgfqpoint{2.264808in}{1.547266in}}%
\pgfpathcurveto{\pgfqpoint{2.264808in}{1.539030in}}{\pgfqpoint{2.268081in}{1.531130in}}{\pgfqpoint{2.273904in}{1.525306in}}%
\pgfpathcurveto{\pgfqpoint{2.279728in}{1.519482in}}{\pgfqpoint{2.287628in}{1.516210in}}{\pgfqpoint{2.295865in}{1.516210in}}%
\pgfpathclose%
\pgfusepath{stroke,fill}%
\end{pgfscope}%
\begin{pgfscope}%
\pgfpathrectangle{\pgfqpoint{0.100000in}{0.212622in}}{\pgfqpoint{3.696000in}{3.696000in}}%
\pgfusepath{clip}%
\pgfsetbuttcap%
\pgfsetroundjoin%
\definecolor{currentfill}{rgb}{0.121569,0.466667,0.705882}%
\pgfsetfillcolor{currentfill}%
\pgfsetfillopacity{0.852852}%
\pgfsetlinewidth{1.003750pt}%
\definecolor{currentstroke}{rgb}{0.121569,0.466667,0.705882}%
\pgfsetstrokecolor{currentstroke}%
\pgfsetstrokeopacity{0.852852}%
\pgfsetdash{}{0pt}%
\pgfpathmoveto{\pgfqpoint{0.931179in}{2.371063in}}%
\pgfpathcurveto{\pgfqpoint{0.939416in}{2.371063in}}{\pgfqpoint{0.947316in}{2.374336in}}{\pgfqpoint{0.953140in}{2.380159in}}%
\pgfpathcurveto{\pgfqpoint{0.958964in}{2.385983in}}{\pgfqpoint{0.962236in}{2.393883in}}{\pgfqpoint{0.962236in}{2.402120in}}%
\pgfpathcurveto{\pgfqpoint{0.962236in}{2.410356in}}{\pgfqpoint{0.958964in}{2.418256in}}{\pgfqpoint{0.953140in}{2.424080in}}%
\pgfpathcurveto{\pgfqpoint{0.947316in}{2.429904in}}{\pgfqpoint{0.939416in}{2.433176in}}{\pgfqpoint{0.931179in}{2.433176in}}%
\pgfpathcurveto{\pgfqpoint{0.922943in}{2.433176in}}{\pgfqpoint{0.915043in}{2.429904in}}{\pgfqpoint{0.909219in}{2.424080in}}%
\pgfpathcurveto{\pgfqpoint{0.903395in}{2.418256in}}{\pgfqpoint{0.900123in}{2.410356in}}{\pgfqpoint{0.900123in}{2.402120in}}%
\pgfpathcurveto{\pgfqpoint{0.900123in}{2.393883in}}{\pgfqpoint{0.903395in}{2.385983in}}{\pgfqpoint{0.909219in}{2.380159in}}%
\pgfpathcurveto{\pgfqpoint{0.915043in}{2.374336in}}{\pgfqpoint{0.922943in}{2.371063in}}{\pgfqpoint{0.931179in}{2.371063in}}%
\pgfpathclose%
\pgfusepath{stroke,fill}%
\end{pgfscope}%
\begin{pgfscope}%
\pgfpathrectangle{\pgfqpoint{0.100000in}{0.212622in}}{\pgfqpoint{3.696000in}{3.696000in}}%
\pgfusepath{clip}%
\pgfsetbuttcap%
\pgfsetroundjoin%
\definecolor{currentfill}{rgb}{0.121569,0.466667,0.705882}%
\pgfsetfillcolor{currentfill}%
\pgfsetfillopacity{0.855218}%
\pgfsetlinewidth{1.003750pt}%
\definecolor{currentstroke}{rgb}{0.121569,0.466667,0.705882}%
\pgfsetstrokecolor{currentstroke}%
\pgfsetstrokeopacity{0.855218}%
\pgfsetdash{}{0pt}%
\pgfpathmoveto{\pgfqpoint{0.963437in}{2.343324in}}%
\pgfpathcurveto{\pgfqpoint{0.971673in}{2.343324in}}{\pgfqpoint{0.979573in}{2.346596in}}{\pgfqpoint{0.985397in}{2.352420in}}%
\pgfpathcurveto{\pgfqpoint{0.991221in}{2.358244in}}{\pgfqpoint{0.994493in}{2.366144in}}{\pgfqpoint{0.994493in}{2.374380in}}%
\pgfpathcurveto{\pgfqpoint{0.994493in}{2.382616in}}{\pgfqpoint{0.991221in}{2.390516in}}{\pgfqpoint{0.985397in}{2.396340in}}%
\pgfpathcurveto{\pgfqpoint{0.979573in}{2.402164in}}{\pgfqpoint{0.971673in}{2.405437in}}{\pgfqpoint{0.963437in}{2.405437in}}%
\pgfpathcurveto{\pgfqpoint{0.955200in}{2.405437in}}{\pgfqpoint{0.947300in}{2.402164in}}{\pgfqpoint{0.941476in}{2.396340in}}%
\pgfpathcurveto{\pgfqpoint{0.935652in}{2.390516in}}{\pgfqpoint{0.932380in}{2.382616in}}{\pgfqpoint{0.932380in}{2.374380in}}%
\pgfpathcurveto{\pgfqpoint{0.932380in}{2.366144in}}{\pgfqpoint{0.935652in}{2.358244in}}{\pgfqpoint{0.941476in}{2.352420in}}%
\pgfpathcurveto{\pgfqpoint{0.947300in}{2.346596in}}{\pgfqpoint{0.955200in}{2.343324in}}{\pgfqpoint{0.963437in}{2.343324in}}%
\pgfpathclose%
\pgfusepath{stroke,fill}%
\end{pgfscope}%
\begin{pgfscope}%
\pgfpathrectangle{\pgfqpoint{0.100000in}{0.212622in}}{\pgfqpoint{3.696000in}{3.696000in}}%
\pgfusepath{clip}%
\pgfsetbuttcap%
\pgfsetroundjoin%
\definecolor{currentfill}{rgb}{0.121569,0.466667,0.705882}%
\pgfsetfillcolor{currentfill}%
\pgfsetfillopacity{0.855897}%
\pgfsetlinewidth{1.003750pt}%
\definecolor{currentstroke}{rgb}{0.121569,0.466667,0.705882}%
\pgfsetstrokecolor{currentstroke}%
\pgfsetstrokeopacity{0.855897}%
\pgfsetdash{}{0pt}%
\pgfpathmoveto{\pgfqpoint{2.299196in}{1.516555in}}%
\pgfpathcurveto{\pgfqpoint{2.307432in}{1.516555in}}{\pgfqpoint{2.315332in}{1.519827in}}{\pgfqpoint{2.321156in}{1.525651in}}%
\pgfpathcurveto{\pgfqpoint{2.326980in}{1.531475in}}{\pgfqpoint{2.330252in}{1.539375in}}{\pgfqpoint{2.330252in}{1.547611in}}%
\pgfpathcurveto{\pgfqpoint{2.330252in}{1.555848in}}{\pgfqpoint{2.326980in}{1.563748in}}{\pgfqpoint{2.321156in}{1.569572in}}%
\pgfpathcurveto{\pgfqpoint{2.315332in}{1.575396in}}{\pgfqpoint{2.307432in}{1.578668in}}{\pgfqpoint{2.299196in}{1.578668in}}%
\pgfpathcurveto{\pgfqpoint{2.290960in}{1.578668in}}{\pgfqpoint{2.283059in}{1.575396in}}{\pgfqpoint{2.277236in}{1.569572in}}%
\pgfpathcurveto{\pgfqpoint{2.271412in}{1.563748in}}{\pgfqpoint{2.268139in}{1.555848in}}{\pgfqpoint{2.268139in}{1.547611in}}%
\pgfpathcurveto{\pgfqpoint{2.268139in}{1.539375in}}{\pgfqpoint{2.271412in}{1.531475in}}{\pgfqpoint{2.277236in}{1.525651in}}%
\pgfpathcurveto{\pgfqpoint{2.283059in}{1.519827in}}{\pgfqpoint{2.290960in}{1.516555in}}{\pgfqpoint{2.299196in}{1.516555in}}%
\pgfpathclose%
\pgfusepath{stroke,fill}%
\end{pgfscope}%
\begin{pgfscope}%
\pgfpathrectangle{\pgfqpoint{0.100000in}{0.212622in}}{\pgfqpoint{3.696000in}{3.696000in}}%
\pgfusepath{clip}%
\pgfsetbuttcap%
\pgfsetroundjoin%
\definecolor{currentfill}{rgb}{0.121569,0.466667,0.705882}%
\pgfsetfillcolor{currentfill}%
\pgfsetfillopacity{0.859815}%
\pgfsetlinewidth{1.003750pt}%
\definecolor{currentstroke}{rgb}{0.121569,0.466667,0.705882}%
\pgfsetstrokecolor{currentstroke}%
\pgfsetstrokeopacity{0.859815}%
\pgfsetdash{}{0pt}%
\pgfpathmoveto{\pgfqpoint{2.301830in}{1.513732in}}%
\pgfpathcurveto{\pgfqpoint{2.310066in}{1.513732in}}{\pgfqpoint{2.317966in}{1.517005in}}{\pgfqpoint{2.323790in}{1.522829in}}%
\pgfpathcurveto{\pgfqpoint{2.329614in}{1.528653in}}{\pgfqpoint{2.332887in}{1.536553in}}{\pgfqpoint{2.332887in}{1.544789in}}%
\pgfpathcurveto{\pgfqpoint{2.332887in}{1.553025in}}{\pgfqpoint{2.329614in}{1.560925in}}{\pgfqpoint{2.323790in}{1.566749in}}%
\pgfpathcurveto{\pgfqpoint{2.317966in}{1.572573in}}{\pgfqpoint{2.310066in}{1.575845in}}{\pgfqpoint{2.301830in}{1.575845in}}%
\pgfpathcurveto{\pgfqpoint{2.293594in}{1.575845in}}{\pgfqpoint{2.285694in}{1.572573in}}{\pgfqpoint{2.279870in}{1.566749in}}%
\pgfpathcurveto{\pgfqpoint{2.274046in}{1.560925in}}{\pgfqpoint{2.270774in}{1.553025in}}{\pgfqpoint{2.270774in}{1.544789in}}%
\pgfpathcurveto{\pgfqpoint{2.270774in}{1.536553in}}{\pgfqpoint{2.274046in}{1.528653in}}{\pgfqpoint{2.279870in}{1.522829in}}%
\pgfpathcurveto{\pgfqpoint{2.285694in}{1.517005in}}{\pgfqpoint{2.293594in}{1.513732in}}{\pgfqpoint{2.301830in}{1.513732in}}%
\pgfpathclose%
\pgfusepath{stroke,fill}%
\end{pgfscope}%
\begin{pgfscope}%
\pgfpathrectangle{\pgfqpoint{0.100000in}{0.212622in}}{\pgfqpoint{3.696000in}{3.696000in}}%
\pgfusepath{clip}%
\pgfsetbuttcap%
\pgfsetroundjoin%
\definecolor{currentfill}{rgb}{0.121569,0.466667,0.705882}%
\pgfsetfillcolor{currentfill}%
\pgfsetfillopacity{0.860819}%
\pgfsetlinewidth{1.003750pt}%
\definecolor{currentstroke}{rgb}{0.121569,0.466667,0.705882}%
\pgfsetstrokecolor{currentstroke}%
\pgfsetstrokeopacity{0.860819}%
\pgfsetdash{}{0pt}%
\pgfpathmoveto{\pgfqpoint{1.020175in}{2.295716in}}%
\pgfpathcurveto{\pgfqpoint{1.028411in}{2.295716in}}{\pgfqpoint{1.036311in}{2.298988in}}{\pgfqpoint{1.042135in}{2.304812in}}%
\pgfpathcurveto{\pgfqpoint{1.047959in}{2.310636in}}{\pgfqpoint{1.051231in}{2.318536in}}{\pgfqpoint{1.051231in}{2.326772in}}%
\pgfpathcurveto{\pgfqpoint{1.051231in}{2.335008in}}{\pgfqpoint{1.047959in}{2.342908in}}{\pgfqpoint{1.042135in}{2.348732in}}%
\pgfpathcurveto{\pgfqpoint{1.036311in}{2.354556in}}{\pgfqpoint{1.028411in}{2.357829in}}{\pgfqpoint{1.020175in}{2.357829in}}%
\pgfpathcurveto{\pgfqpoint{1.011939in}{2.357829in}}{\pgfqpoint{1.004039in}{2.354556in}}{\pgfqpoint{0.998215in}{2.348732in}}%
\pgfpathcurveto{\pgfqpoint{0.992391in}{2.342908in}}{\pgfqpoint{0.989118in}{2.335008in}}{\pgfqpoint{0.989118in}{2.326772in}}%
\pgfpathcurveto{\pgfqpoint{0.989118in}{2.318536in}}{\pgfqpoint{0.992391in}{2.310636in}}{\pgfqpoint{0.998215in}{2.304812in}}%
\pgfpathcurveto{\pgfqpoint{1.004039in}{2.298988in}}{\pgfqpoint{1.011939in}{2.295716in}}{\pgfqpoint{1.020175in}{2.295716in}}%
\pgfpathclose%
\pgfusepath{stroke,fill}%
\end{pgfscope}%
\begin{pgfscope}%
\pgfpathrectangle{\pgfqpoint{0.100000in}{0.212622in}}{\pgfqpoint{3.696000in}{3.696000in}}%
\pgfusepath{clip}%
\pgfsetbuttcap%
\pgfsetroundjoin%
\definecolor{currentfill}{rgb}{0.121569,0.466667,0.705882}%
\pgfsetfillcolor{currentfill}%
\pgfsetfillopacity{0.863791}%
\pgfsetlinewidth{1.003750pt}%
\definecolor{currentstroke}{rgb}{0.121569,0.466667,0.705882}%
\pgfsetstrokecolor{currentstroke}%
\pgfsetstrokeopacity{0.863791}%
\pgfsetdash{}{0pt}%
\pgfpathmoveto{\pgfqpoint{2.305441in}{1.509743in}}%
\pgfpathcurveto{\pgfqpoint{2.313677in}{1.509743in}}{\pgfqpoint{2.321577in}{1.513015in}}{\pgfqpoint{2.327401in}{1.518839in}}%
\pgfpathcurveto{\pgfqpoint{2.333225in}{1.524663in}}{\pgfqpoint{2.336497in}{1.532563in}}{\pgfqpoint{2.336497in}{1.540799in}}%
\pgfpathcurveto{\pgfqpoint{2.336497in}{1.549036in}}{\pgfqpoint{2.333225in}{1.556936in}}{\pgfqpoint{2.327401in}{1.562759in}}%
\pgfpathcurveto{\pgfqpoint{2.321577in}{1.568583in}}{\pgfqpoint{2.313677in}{1.571856in}}{\pgfqpoint{2.305441in}{1.571856in}}%
\pgfpathcurveto{\pgfqpoint{2.297204in}{1.571856in}}{\pgfqpoint{2.289304in}{1.568583in}}{\pgfqpoint{2.283480in}{1.562759in}}%
\pgfpathcurveto{\pgfqpoint{2.277657in}{1.556936in}}{\pgfqpoint{2.274384in}{1.549036in}}{\pgfqpoint{2.274384in}{1.540799in}}%
\pgfpathcurveto{\pgfqpoint{2.274384in}{1.532563in}}{\pgfqpoint{2.277657in}{1.524663in}}{\pgfqpoint{2.283480in}{1.518839in}}%
\pgfpathcurveto{\pgfqpoint{2.289304in}{1.513015in}}{\pgfqpoint{2.297204in}{1.509743in}}{\pgfqpoint{2.305441in}{1.509743in}}%
\pgfpathclose%
\pgfusepath{stroke,fill}%
\end{pgfscope}%
\begin{pgfscope}%
\pgfpathrectangle{\pgfqpoint{0.100000in}{0.212622in}}{\pgfqpoint{3.696000in}{3.696000in}}%
\pgfusepath{clip}%
\pgfsetbuttcap%
\pgfsetroundjoin%
\definecolor{currentfill}{rgb}{0.121569,0.466667,0.705882}%
\pgfsetfillcolor{currentfill}%
\pgfsetfillopacity{0.865360}%
\pgfsetlinewidth{1.003750pt}%
\definecolor{currentstroke}{rgb}{0.121569,0.466667,0.705882}%
\pgfsetstrokecolor{currentstroke}%
\pgfsetstrokeopacity{0.865360}%
\pgfsetdash{}{0pt}%
\pgfpathmoveto{\pgfqpoint{2.307120in}{1.503551in}}%
\pgfpathcurveto{\pgfqpoint{2.315356in}{1.503551in}}{\pgfqpoint{2.323256in}{1.506824in}}{\pgfqpoint{2.329080in}{1.512648in}}%
\pgfpathcurveto{\pgfqpoint{2.334904in}{1.518472in}}{\pgfqpoint{2.338176in}{1.526372in}}{\pgfqpoint{2.338176in}{1.534608in}}%
\pgfpathcurveto{\pgfqpoint{2.338176in}{1.542844in}}{\pgfqpoint{2.334904in}{1.550744in}}{\pgfqpoint{2.329080in}{1.556568in}}%
\pgfpathcurveto{\pgfqpoint{2.323256in}{1.562392in}}{\pgfqpoint{2.315356in}{1.565664in}}{\pgfqpoint{2.307120in}{1.565664in}}%
\pgfpathcurveto{\pgfqpoint{2.298883in}{1.565664in}}{\pgfqpoint{2.290983in}{1.562392in}}{\pgfqpoint{2.285159in}{1.556568in}}%
\pgfpathcurveto{\pgfqpoint{2.279336in}{1.550744in}}{\pgfqpoint{2.276063in}{1.542844in}}{\pgfqpoint{2.276063in}{1.534608in}}%
\pgfpathcurveto{\pgfqpoint{2.276063in}{1.526372in}}{\pgfqpoint{2.279336in}{1.518472in}}{\pgfqpoint{2.285159in}{1.512648in}}%
\pgfpathcurveto{\pgfqpoint{2.290983in}{1.506824in}}{\pgfqpoint{2.298883in}{1.503551in}}{\pgfqpoint{2.307120in}{1.503551in}}%
\pgfpathclose%
\pgfusepath{stroke,fill}%
\end{pgfscope}%
\begin{pgfscope}%
\pgfpathrectangle{\pgfqpoint{0.100000in}{0.212622in}}{\pgfqpoint{3.696000in}{3.696000in}}%
\pgfusepath{clip}%
\pgfsetbuttcap%
\pgfsetroundjoin%
\definecolor{currentfill}{rgb}{0.121569,0.466667,0.705882}%
\pgfsetfillcolor{currentfill}%
\pgfsetfillopacity{0.865986}%
\pgfsetlinewidth{1.003750pt}%
\definecolor{currentstroke}{rgb}{0.121569,0.466667,0.705882}%
\pgfsetstrokecolor{currentstroke}%
\pgfsetstrokeopacity{0.865986}%
\pgfsetdash{}{0pt}%
\pgfpathmoveto{\pgfqpoint{1.073716in}{2.250940in}}%
\pgfpathcurveto{\pgfqpoint{1.081952in}{2.250940in}}{\pgfqpoint{1.089852in}{2.254213in}}{\pgfqpoint{1.095676in}{2.260036in}}%
\pgfpathcurveto{\pgfqpoint{1.101500in}{2.265860in}}{\pgfqpoint{1.104772in}{2.273760in}}{\pgfqpoint{1.104772in}{2.281997in}}%
\pgfpathcurveto{\pgfqpoint{1.104772in}{2.290233in}}{\pgfqpoint{1.101500in}{2.298133in}}{\pgfqpoint{1.095676in}{2.303957in}}%
\pgfpathcurveto{\pgfqpoint{1.089852in}{2.309781in}}{\pgfqpoint{1.081952in}{2.313053in}}{\pgfqpoint{1.073716in}{2.313053in}}%
\pgfpathcurveto{\pgfqpoint{1.065480in}{2.313053in}}{\pgfqpoint{1.057580in}{2.309781in}}{\pgfqpoint{1.051756in}{2.303957in}}%
\pgfpathcurveto{\pgfqpoint{1.045932in}{2.298133in}}{\pgfqpoint{1.042659in}{2.290233in}}{\pgfqpoint{1.042659in}{2.281997in}}%
\pgfpathcurveto{\pgfqpoint{1.042659in}{2.273760in}}{\pgfqpoint{1.045932in}{2.265860in}}{\pgfqpoint{1.051756in}{2.260036in}}%
\pgfpathcurveto{\pgfqpoint{1.057580in}{2.254213in}}{\pgfqpoint{1.065480in}{2.250940in}}{\pgfqpoint{1.073716in}{2.250940in}}%
\pgfpathclose%
\pgfusepath{stroke,fill}%
\end{pgfscope}%
\begin{pgfscope}%
\pgfpathrectangle{\pgfqpoint{0.100000in}{0.212622in}}{\pgfqpoint{3.696000in}{3.696000in}}%
\pgfusepath{clip}%
\pgfsetbuttcap%
\pgfsetroundjoin%
\definecolor{currentfill}{rgb}{0.121569,0.466667,0.705882}%
\pgfsetfillcolor{currentfill}%
\pgfsetfillopacity{0.867771}%
\pgfsetlinewidth{1.003750pt}%
\definecolor{currentstroke}{rgb}{0.121569,0.466667,0.705882}%
\pgfsetstrokecolor{currentstroke}%
\pgfsetstrokeopacity{0.867771}%
\pgfsetdash{}{0pt}%
\pgfpathmoveto{\pgfqpoint{2.309210in}{1.500601in}}%
\pgfpathcurveto{\pgfqpoint{2.317446in}{1.500601in}}{\pgfqpoint{2.325346in}{1.503874in}}{\pgfqpoint{2.331170in}{1.509698in}}%
\pgfpathcurveto{\pgfqpoint{2.336994in}{1.515522in}}{\pgfqpoint{2.340266in}{1.523422in}}{\pgfqpoint{2.340266in}{1.531658in}}%
\pgfpathcurveto{\pgfqpoint{2.340266in}{1.539894in}}{\pgfqpoint{2.336994in}{1.547794in}}{\pgfqpoint{2.331170in}{1.553618in}}%
\pgfpathcurveto{\pgfqpoint{2.325346in}{1.559442in}}{\pgfqpoint{2.317446in}{1.562714in}}{\pgfqpoint{2.309210in}{1.562714in}}%
\pgfpathcurveto{\pgfqpoint{2.300973in}{1.562714in}}{\pgfqpoint{2.293073in}{1.559442in}}{\pgfqpoint{2.287249in}{1.553618in}}%
\pgfpathcurveto{\pgfqpoint{2.281425in}{1.547794in}}{\pgfqpoint{2.278153in}{1.539894in}}{\pgfqpoint{2.278153in}{1.531658in}}%
\pgfpathcurveto{\pgfqpoint{2.278153in}{1.523422in}}{\pgfqpoint{2.281425in}{1.515522in}}{\pgfqpoint{2.287249in}{1.509698in}}%
\pgfpathcurveto{\pgfqpoint{2.293073in}{1.503874in}}{\pgfqpoint{2.300973in}{1.500601in}}{\pgfqpoint{2.309210in}{1.500601in}}%
\pgfpathclose%
\pgfusepath{stroke,fill}%
\end{pgfscope}%
\begin{pgfscope}%
\pgfpathrectangle{\pgfqpoint{0.100000in}{0.212622in}}{\pgfqpoint{3.696000in}{3.696000in}}%
\pgfusepath{clip}%
\pgfsetbuttcap%
\pgfsetroundjoin%
\definecolor{currentfill}{rgb}{0.121569,0.466667,0.705882}%
\pgfsetfillcolor{currentfill}%
\pgfsetfillopacity{0.870631}%
\pgfsetlinewidth{1.003750pt}%
\definecolor{currentstroke}{rgb}{0.121569,0.466667,0.705882}%
\pgfsetstrokecolor{currentstroke}%
\pgfsetstrokeopacity{0.870631}%
\pgfsetdash{}{0pt}%
\pgfpathmoveto{\pgfqpoint{2.311490in}{1.499325in}}%
\pgfpathcurveto{\pgfqpoint{2.319727in}{1.499325in}}{\pgfqpoint{2.327627in}{1.502597in}}{\pgfqpoint{2.333450in}{1.508421in}}%
\pgfpathcurveto{\pgfqpoint{2.339274in}{1.514245in}}{\pgfqpoint{2.342547in}{1.522145in}}{\pgfqpoint{2.342547in}{1.530382in}}%
\pgfpathcurveto{\pgfqpoint{2.342547in}{1.538618in}}{\pgfqpoint{2.339274in}{1.546518in}}{\pgfqpoint{2.333450in}{1.552342in}}%
\pgfpathcurveto{\pgfqpoint{2.327627in}{1.558166in}}{\pgfqpoint{2.319727in}{1.561438in}}{\pgfqpoint{2.311490in}{1.561438in}}%
\pgfpathcurveto{\pgfqpoint{2.303254in}{1.561438in}}{\pgfqpoint{2.295354in}{1.558166in}}{\pgfqpoint{2.289530in}{1.552342in}}%
\pgfpathcurveto{\pgfqpoint{2.283706in}{1.546518in}}{\pgfqpoint{2.280434in}{1.538618in}}{\pgfqpoint{2.280434in}{1.530382in}}%
\pgfpathcurveto{\pgfqpoint{2.280434in}{1.522145in}}{\pgfqpoint{2.283706in}{1.514245in}}{\pgfqpoint{2.289530in}{1.508421in}}%
\pgfpathcurveto{\pgfqpoint{2.295354in}{1.502597in}}{\pgfqpoint{2.303254in}{1.499325in}}{\pgfqpoint{2.311490in}{1.499325in}}%
\pgfpathclose%
\pgfusepath{stroke,fill}%
\end{pgfscope}%
\begin{pgfscope}%
\pgfpathrectangle{\pgfqpoint{0.100000in}{0.212622in}}{\pgfqpoint{3.696000in}{3.696000in}}%
\pgfusepath{clip}%
\pgfsetbuttcap%
\pgfsetroundjoin%
\definecolor{currentfill}{rgb}{0.121569,0.466667,0.705882}%
\pgfsetfillcolor{currentfill}%
\pgfsetfillopacity{0.872047}%
\pgfsetlinewidth{1.003750pt}%
\definecolor{currentstroke}{rgb}{0.121569,0.466667,0.705882}%
\pgfsetstrokecolor{currentstroke}%
\pgfsetstrokeopacity{0.872047}%
\pgfsetdash{}{0pt}%
\pgfpathmoveto{\pgfqpoint{2.312621in}{1.497582in}}%
\pgfpathcurveto{\pgfqpoint{2.320857in}{1.497582in}}{\pgfqpoint{2.328757in}{1.500855in}}{\pgfqpoint{2.334581in}{1.506679in}}%
\pgfpathcurveto{\pgfqpoint{2.340405in}{1.512503in}}{\pgfqpoint{2.343677in}{1.520403in}}{\pgfqpoint{2.343677in}{1.528639in}}%
\pgfpathcurveto{\pgfqpoint{2.343677in}{1.536875in}}{\pgfqpoint{2.340405in}{1.544775in}}{\pgfqpoint{2.334581in}{1.550599in}}%
\pgfpathcurveto{\pgfqpoint{2.328757in}{1.556423in}}{\pgfqpoint{2.320857in}{1.559695in}}{\pgfqpoint{2.312621in}{1.559695in}}%
\pgfpathcurveto{\pgfqpoint{2.304384in}{1.559695in}}{\pgfqpoint{2.296484in}{1.556423in}}{\pgfqpoint{2.290661in}{1.550599in}}%
\pgfpathcurveto{\pgfqpoint{2.284837in}{1.544775in}}{\pgfqpoint{2.281564in}{1.536875in}}{\pgfqpoint{2.281564in}{1.528639in}}%
\pgfpathcurveto{\pgfqpoint{2.281564in}{1.520403in}}{\pgfqpoint{2.284837in}{1.512503in}}{\pgfqpoint{2.290661in}{1.506679in}}%
\pgfpathcurveto{\pgfqpoint{2.296484in}{1.500855in}}{\pgfqpoint{2.304384in}{1.497582in}}{\pgfqpoint{2.312621in}{1.497582in}}%
\pgfpathclose%
\pgfusepath{stroke,fill}%
\end{pgfscope}%
\begin{pgfscope}%
\pgfpathrectangle{\pgfqpoint{0.100000in}{0.212622in}}{\pgfqpoint{3.696000in}{3.696000in}}%
\pgfusepath{clip}%
\pgfsetbuttcap%
\pgfsetroundjoin%
\definecolor{currentfill}{rgb}{0.121569,0.466667,0.705882}%
\pgfsetfillcolor{currentfill}%
\pgfsetfillopacity{0.872395}%
\pgfsetlinewidth{1.003750pt}%
\definecolor{currentstroke}{rgb}{0.121569,0.466667,0.705882}%
\pgfsetstrokecolor{currentstroke}%
\pgfsetstrokeopacity{0.872395}%
\pgfsetdash{}{0pt}%
\pgfpathmoveto{\pgfqpoint{1.125519in}{2.211248in}}%
\pgfpathcurveto{\pgfqpoint{1.133755in}{2.211248in}}{\pgfqpoint{1.141655in}{2.214521in}}{\pgfqpoint{1.147479in}{2.220345in}}%
\pgfpathcurveto{\pgfqpoint{1.153303in}{2.226168in}}{\pgfqpoint{1.156575in}{2.234068in}}{\pgfqpoint{1.156575in}{2.242305in}}%
\pgfpathcurveto{\pgfqpoint{1.156575in}{2.250541in}}{\pgfqpoint{1.153303in}{2.258441in}}{\pgfqpoint{1.147479in}{2.264265in}}%
\pgfpathcurveto{\pgfqpoint{1.141655in}{2.270089in}}{\pgfqpoint{1.133755in}{2.273361in}}{\pgfqpoint{1.125519in}{2.273361in}}%
\pgfpathcurveto{\pgfqpoint{1.117282in}{2.273361in}}{\pgfqpoint{1.109382in}{2.270089in}}{\pgfqpoint{1.103558in}{2.264265in}}%
\pgfpathcurveto{\pgfqpoint{1.097735in}{2.258441in}}{\pgfqpoint{1.094462in}{2.250541in}}{\pgfqpoint{1.094462in}{2.242305in}}%
\pgfpathcurveto{\pgfqpoint{1.094462in}{2.234068in}}{\pgfqpoint{1.097735in}{2.226168in}}{\pgfqpoint{1.103558in}{2.220345in}}%
\pgfpathcurveto{\pgfqpoint{1.109382in}{2.214521in}}{\pgfqpoint{1.117282in}{2.211248in}}{\pgfqpoint{1.125519in}{2.211248in}}%
\pgfpathclose%
\pgfusepath{stroke,fill}%
\end{pgfscope}%
\begin{pgfscope}%
\pgfpathrectangle{\pgfqpoint{0.100000in}{0.212622in}}{\pgfqpoint{3.696000in}{3.696000in}}%
\pgfusepath{clip}%
\pgfsetbuttcap%
\pgfsetroundjoin%
\definecolor{currentfill}{rgb}{0.121569,0.466667,0.705882}%
\pgfsetfillcolor{currentfill}%
\pgfsetfillopacity{0.872857}%
\pgfsetlinewidth{1.003750pt}%
\definecolor{currentstroke}{rgb}{0.121569,0.466667,0.705882}%
\pgfsetstrokecolor{currentstroke}%
\pgfsetstrokeopacity{0.872857}%
\pgfsetdash{}{0pt}%
\pgfpathmoveto{\pgfqpoint{2.313249in}{1.496809in}}%
\pgfpathcurveto{\pgfqpoint{2.321485in}{1.496809in}}{\pgfqpoint{2.329385in}{1.500082in}}{\pgfqpoint{2.335209in}{1.505906in}}%
\pgfpathcurveto{\pgfqpoint{2.341033in}{1.511730in}}{\pgfqpoint{2.344305in}{1.519630in}}{\pgfqpoint{2.344305in}{1.527866in}}%
\pgfpathcurveto{\pgfqpoint{2.344305in}{1.536102in}}{\pgfqpoint{2.341033in}{1.544002in}}{\pgfqpoint{2.335209in}{1.549826in}}%
\pgfpathcurveto{\pgfqpoint{2.329385in}{1.555650in}}{\pgfqpoint{2.321485in}{1.558922in}}{\pgfqpoint{2.313249in}{1.558922in}}%
\pgfpathcurveto{\pgfqpoint{2.305012in}{1.558922in}}{\pgfqpoint{2.297112in}{1.555650in}}{\pgfqpoint{2.291288in}{1.549826in}}%
\pgfpathcurveto{\pgfqpoint{2.285465in}{1.544002in}}{\pgfqpoint{2.282192in}{1.536102in}}{\pgfqpoint{2.282192in}{1.527866in}}%
\pgfpathcurveto{\pgfqpoint{2.282192in}{1.519630in}}{\pgfqpoint{2.285465in}{1.511730in}}{\pgfqpoint{2.291288in}{1.505906in}}%
\pgfpathcurveto{\pgfqpoint{2.297112in}{1.500082in}}{\pgfqpoint{2.305012in}{1.496809in}}{\pgfqpoint{2.313249in}{1.496809in}}%
\pgfpathclose%
\pgfusepath{stroke,fill}%
\end{pgfscope}%
\begin{pgfscope}%
\pgfpathrectangle{\pgfqpoint{0.100000in}{0.212622in}}{\pgfqpoint{3.696000in}{3.696000in}}%
\pgfusepath{clip}%
\pgfsetbuttcap%
\pgfsetroundjoin%
\definecolor{currentfill}{rgb}{0.121569,0.466667,0.705882}%
\pgfsetfillcolor{currentfill}%
\pgfsetfillopacity{0.873198}%
\pgfsetlinewidth{1.003750pt}%
\definecolor{currentstroke}{rgb}{0.121569,0.466667,0.705882}%
\pgfsetstrokecolor{currentstroke}%
\pgfsetstrokeopacity{0.873198}%
\pgfsetdash{}{0pt}%
\pgfpathmoveto{\pgfqpoint{2.313617in}{1.495766in}}%
\pgfpathcurveto{\pgfqpoint{2.321853in}{1.495766in}}{\pgfqpoint{2.329753in}{1.499038in}}{\pgfqpoint{2.335577in}{1.504862in}}%
\pgfpathcurveto{\pgfqpoint{2.341401in}{1.510686in}}{\pgfqpoint{2.344673in}{1.518586in}}{\pgfqpoint{2.344673in}{1.526822in}}%
\pgfpathcurveto{\pgfqpoint{2.344673in}{1.535059in}}{\pgfqpoint{2.341401in}{1.542959in}}{\pgfqpoint{2.335577in}{1.548783in}}%
\pgfpathcurveto{\pgfqpoint{2.329753in}{1.554607in}}{\pgfqpoint{2.321853in}{1.557879in}}{\pgfqpoint{2.313617in}{1.557879in}}%
\pgfpathcurveto{\pgfqpoint{2.305380in}{1.557879in}}{\pgfqpoint{2.297480in}{1.554607in}}{\pgfqpoint{2.291657in}{1.548783in}}%
\pgfpathcurveto{\pgfqpoint{2.285833in}{1.542959in}}{\pgfqpoint{2.282560in}{1.535059in}}{\pgfqpoint{2.282560in}{1.526822in}}%
\pgfpathcurveto{\pgfqpoint{2.282560in}{1.518586in}}{\pgfqpoint{2.285833in}{1.510686in}}{\pgfqpoint{2.291657in}{1.504862in}}%
\pgfpathcurveto{\pgfqpoint{2.297480in}{1.499038in}}{\pgfqpoint{2.305380in}{1.495766in}}{\pgfqpoint{2.313617in}{1.495766in}}%
\pgfpathclose%
\pgfusepath{stroke,fill}%
\end{pgfscope}%
\begin{pgfscope}%
\pgfpathrectangle{\pgfqpoint{0.100000in}{0.212622in}}{\pgfqpoint{3.696000in}{3.696000in}}%
\pgfusepath{clip}%
\pgfsetbuttcap%
\pgfsetroundjoin%
\definecolor{currentfill}{rgb}{0.121569,0.466667,0.705882}%
\pgfsetfillcolor{currentfill}%
\pgfsetfillopacity{0.874176}%
\pgfsetlinewidth{1.003750pt}%
\definecolor{currentstroke}{rgb}{0.121569,0.466667,0.705882}%
\pgfsetstrokecolor{currentstroke}%
\pgfsetstrokeopacity{0.874176}%
\pgfsetdash{}{0pt}%
\pgfpathmoveto{\pgfqpoint{2.314193in}{1.494892in}}%
\pgfpathcurveto{\pgfqpoint{2.322429in}{1.494892in}}{\pgfqpoint{2.330329in}{1.498164in}}{\pgfqpoint{2.336153in}{1.503988in}}%
\pgfpathcurveto{\pgfqpoint{2.341977in}{1.509812in}}{\pgfqpoint{2.345249in}{1.517712in}}{\pgfqpoint{2.345249in}{1.525948in}}%
\pgfpathcurveto{\pgfqpoint{2.345249in}{1.534185in}}{\pgfqpoint{2.341977in}{1.542085in}}{\pgfqpoint{2.336153in}{1.547909in}}%
\pgfpathcurveto{\pgfqpoint{2.330329in}{1.553732in}}{\pgfqpoint{2.322429in}{1.557005in}}{\pgfqpoint{2.314193in}{1.557005in}}%
\pgfpathcurveto{\pgfqpoint{2.305956in}{1.557005in}}{\pgfqpoint{2.298056in}{1.553732in}}{\pgfqpoint{2.292232in}{1.547909in}}%
\pgfpathcurveto{\pgfqpoint{2.286408in}{1.542085in}}{\pgfqpoint{2.283136in}{1.534185in}}{\pgfqpoint{2.283136in}{1.525948in}}%
\pgfpathcurveto{\pgfqpoint{2.283136in}{1.517712in}}{\pgfqpoint{2.286408in}{1.509812in}}{\pgfqpoint{2.292232in}{1.503988in}}%
\pgfpathcurveto{\pgfqpoint{2.298056in}{1.498164in}}{\pgfqpoint{2.305956in}{1.494892in}}{\pgfqpoint{2.314193in}{1.494892in}}%
\pgfpathclose%
\pgfusepath{stroke,fill}%
\end{pgfscope}%
\begin{pgfscope}%
\pgfpathrectangle{\pgfqpoint{0.100000in}{0.212622in}}{\pgfqpoint{3.696000in}{3.696000in}}%
\pgfusepath{clip}%
\pgfsetbuttcap%
\pgfsetroundjoin%
\definecolor{currentfill}{rgb}{0.121569,0.466667,0.705882}%
\pgfsetfillcolor{currentfill}%
\pgfsetfillopacity{0.875798}%
\pgfsetlinewidth{1.003750pt}%
\definecolor{currentstroke}{rgb}{0.121569,0.466667,0.705882}%
\pgfsetstrokecolor{currentstroke}%
\pgfsetstrokeopacity{0.875798}%
\pgfsetdash{}{0pt}%
\pgfpathmoveto{\pgfqpoint{2.315229in}{1.494158in}}%
\pgfpathcurveto{\pgfqpoint{2.323466in}{1.494158in}}{\pgfqpoint{2.331366in}{1.497430in}}{\pgfqpoint{2.337190in}{1.503254in}}%
\pgfpathcurveto{\pgfqpoint{2.343013in}{1.509078in}}{\pgfqpoint{2.346286in}{1.516978in}}{\pgfqpoint{2.346286in}{1.525214in}}%
\pgfpathcurveto{\pgfqpoint{2.346286in}{1.533450in}}{\pgfqpoint{2.343013in}{1.541350in}}{\pgfqpoint{2.337190in}{1.547174in}}%
\pgfpathcurveto{\pgfqpoint{2.331366in}{1.552998in}}{\pgfqpoint{2.323466in}{1.556271in}}{\pgfqpoint{2.315229in}{1.556271in}}%
\pgfpathcurveto{\pgfqpoint{2.306993in}{1.556271in}}{\pgfqpoint{2.299093in}{1.552998in}}{\pgfqpoint{2.293269in}{1.547174in}}%
\pgfpathcurveto{\pgfqpoint{2.287445in}{1.541350in}}{\pgfqpoint{2.284173in}{1.533450in}}{\pgfqpoint{2.284173in}{1.525214in}}%
\pgfpathcurveto{\pgfqpoint{2.284173in}{1.516978in}}{\pgfqpoint{2.287445in}{1.509078in}}{\pgfqpoint{2.293269in}{1.503254in}}%
\pgfpathcurveto{\pgfqpoint{2.299093in}{1.497430in}}{\pgfqpoint{2.306993in}{1.494158in}}{\pgfqpoint{2.315229in}{1.494158in}}%
\pgfpathclose%
\pgfusepath{stroke,fill}%
\end{pgfscope}%
\begin{pgfscope}%
\pgfpathrectangle{\pgfqpoint{0.100000in}{0.212622in}}{\pgfqpoint{3.696000in}{3.696000in}}%
\pgfusepath{clip}%
\pgfsetbuttcap%
\pgfsetroundjoin%
\definecolor{currentfill}{rgb}{0.121569,0.466667,0.705882}%
\pgfsetfillcolor{currentfill}%
\pgfsetfillopacity{0.876976}%
\pgfsetlinewidth{1.003750pt}%
\definecolor{currentstroke}{rgb}{0.121569,0.466667,0.705882}%
\pgfsetstrokecolor{currentstroke}%
\pgfsetstrokeopacity{0.876976}%
\pgfsetdash{}{0pt}%
\pgfpathmoveto{\pgfqpoint{1.177501in}{2.169494in}}%
\pgfpathcurveto{\pgfqpoint{1.185738in}{2.169494in}}{\pgfqpoint{1.193638in}{2.172766in}}{\pgfqpoint{1.199462in}{2.178590in}}%
\pgfpathcurveto{\pgfqpoint{1.205286in}{2.184414in}}{\pgfqpoint{1.208558in}{2.192314in}}{\pgfqpoint{1.208558in}{2.200550in}}%
\pgfpathcurveto{\pgfqpoint{1.208558in}{2.208787in}}{\pgfqpoint{1.205286in}{2.216687in}}{\pgfqpoint{1.199462in}{2.222511in}}%
\pgfpathcurveto{\pgfqpoint{1.193638in}{2.228335in}}{\pgfqpoint{1.185738in}{2.231607in}}{\pgfqpoint{1.177501in}{2.231607in}}%
\pgfpathcurveto{\pgfqpoint{1.169265in}{2.231607in}}{\pgfqpoint{1.161365in}{2.228335in}}{\pgfqpoint{1.155541in}{2.222511in}}%
\pgfpathcurveto{\pgfqpoint{1.149717in}{2.216687in}}{\pgfqpoint{1.146445in}{2.208787in}}{\pgfqpoint{1.146445in}{2.200550in}}%
\pgfpathcurveto{\pgfqpoint{1.146445in}{2.192314in}}{\pgfqpoint{1.149717in}{2.184414in}}{\pgfqpoint{1.155541in}{2.178590in}}%
\pgfpathcurveto{\pgfqpoint{1.161365in}{2.172766in}}{\pgfqpoint{1.169265in}{2.169494in}}{\pgfqpoint{1.177501in}{2.169494in}}%
\pgfpathclose%
\pgfusepath{stroke,fill}%
\end{pgfscope}%
\begin{pgfscope}%
\pgfpathrectangle{\pgfqpoint{0.100000in}{0.212622in}}{\pgfqpoint{3.696000in}{3.696000in}}%
\pgfusepath{clip}%
\pgfsetbuttcap%
\pgfsetroundjoin%
\definecolor{currentfill}{rgb}{0.121569,0.466667,0.705882}%
\pgfsetfillcolor{currentfill}%
\pgfsetfillopacity{0.877854}%
\pgfsetlinewidth{1.003750pt}%
\definecolor{currentstroke}{rgb}{0.121569,0.466667,0.705882}%
\pgfsetstrokecolor{currentstroke}%
\pgfsetstrokeopacity{0.877854}%
\pgfsetdash{}{0pt}%
\pgfpathmoveto{\pgfqpoint{2.317063in}{1.490749in}}%
\pgfpathcurveto{\pgfqpoint{2.325299in}{1.490749in}}{\pgfqpoint{2.333199in}{1.494022in}}{\pgfqpoint{2.339023in}{1.499845in}}%
\pgfpathcurveto{\pgfqpoint{2.344847in}{1.505669in}}{\pgfqpoint{2.348119in}{1.513569in}}{\pgfqpoint{2.348119in}{1.521806in}}%
\pgfpathcurveto{\pgfqpoint{2.348119in}{1.530042in}}{\pgfqpoint{2.344847in}{1.537942in}}{\pgfqpoint{2.339023in}{1.543766in}}%
\pgfpathcurveto{\pgfqpoint{2.333199in}{1.549590in}}{\pgfqpoint{2.325299in}{1.552862in}}{\pgfqpoint{2.317063in}{1.552862in}}%
\pgfpathcurveto{\pgfqpoint{2.308826in}{1.552862in}}{\pgfqpoint{2.300926in}{1.549590in}}{\pgfqpoint{2.295102in}{1.543766in}}%
\pgfpathcurveto{\pgfqpoint{2.289278in}{1.537942in}}{\pgfqpoint{2.286006in}{1.530042in}}{\pgfqpoint{2.286006in}{1.521806in}}%
\pgfpathcurveto{\pgfqpoint{2.286006in}{1.513569in}}{\pgfqpoint{2.289278in}{1.505669in}}{\pgfqpoint{2.295102in}{1.499845in}}%
\pgfpathcurveto{\pgfqpoint{2.300926in}{1.494022in}}{\pgfqpoint{2.308826in}{1.490749in}}{\pgfqpoint{2.317063in}{1.490749in}}%
\pgfpathclose%
\pgfusepath{stroke,fill}%
\end{pgfscope}%
\begin{pgfscope}%
\pgfpathrectangle{\pgfqpoint{0.100000in}{0.212622in}}{\pgfqpoint{3.696000in}{3.696000in}}%
\pgfusepath{clip}%
\pgfsetbuttcap%
\pgfsetroundjoin%
\definecolor{currentfill}{rgb}{0.121569,0.466667,0.705882}%
\pgfsetfillcolor{currentfill}%
\pgfsetfillopacity{0.878986}%
\pgfsetlinewidth{1.003750pt}%
\definecolor{currentstroke}{rgb}{0.121569,0.466667,0.705882}%
\pgfsetstrokecolor{currentstroke}%
\pgfsetstrokeopacity{0.878986}%
\pgfsetdash{}{0pt}%
\pgfpathmoveto{\pgfqpoint{2.317771in}{1.488707in}}%
\pgfpathcurveto{\pgfqpoint{2.326007in}{1.488707in}}{\pgfqpoint{2.333907in}{1.491979in}}{\pgfqpoint{2.339731in}{1.497803in}}%
\pgfpathcurveto{\pgfqpoint{2.345555in}{1.503627in}}{\pgfqpoint{2.348827in}{1.511527in}}{\pgfqpoint{2.348827in}{1.519764in}}%
\pgfpathcurveto{\pgfqpoint{2.348827in}{1.528000in}}{\pgfqpoint{2.345555in}{1.535900in}}{\pgfqpoint{2.339731in}{1.541724in}}%
\pgfpathcurveto{\pgfqpoint{2.333907in}{1.547548in}}{\pgfqpoint{2.326007in}{1.550820in}}{\pgfqpoint{2.317771in}{1.550820in}}%
\pgfpathcurveto{\pgfqpoint{2.309534in}{1.550820in}}{\pgfqpoint{2.301634in}{1.547548in}}{\pgfqpoint{2.295810in}{1.541724in}}%
\pgfpathcurveto{\pgfqpoint{2.289986in}{1.535900in}}{\pgfqpoint{2.286714in}{1.528000in}}{\pgfqpoint{2.286714in}{1.519764in}}%
\pgfpathcurveto{\pgfqpoint{2.286714in}{1.511527in}}{\pgfqpoint{2.289986in}{1.503627in}}{\pgfqpoint{2.295810in}{1.497803in}}%
\pgfpathcurveto{\pgfqpoint{2.301634in}{1.491979in}}{\pgfqpoint{2.309534in}{1.488707in}}{\pgfqpoint{2.317771in}{1.488707in}}%
\pgfpathclose%
\pgfusepath{stroke,fill}%
\end{pgfscope}%
\begin{pgfscope}%
\pgfpathrectangle{\pgfqpoint{0.100000in}{0.212622in}}{\pgfqpoint{3.696000in}{3.696000in}}%
\pgfusepath{clip}%
\pgfsetbuttcap%
\pgfsetroundjoin%
\definecolor{currentfill}{rgb}{0.121569,0.466667,0.705882}%
\pgfsetfillcolor{currentfill}%
\pgfsetfillopacity{0.879650}%
\pgfsetlinewidth{1.003750pt}%
\definecolor{currentstroke}{rgb}{0.121569,0.466667,0.705882}%
\pgfsetstrokecolor{currentstroke}%
\pgfsetstrokeopacity{0.879650}%
\pgfsetdash{}{0pt}%
\pgfpathmoveto{\pgfqpoint{2.318177in}{1.487846in}}%
\pgfpathcurveto{\pgfqpoint{2.326414in}{1.487846in}}{\pgfqpoint{2.334314in}{1.491119in}}{\pgfqpoint{2.340138in}{1.496943in}}%
\pgfpathcurveto{\pgfqpoint{2.345962in}{1.502767in}}{\pgfqpoint{2.349234in}{1.510667in}}{\pgfqpoint{2.349234in}{1.518903in}}%
\pgfpathcurveto{\pgfqpoint{2.349234in}{1.527139in}}{\pgfqpoint{2.345962in}{1.535039in}}{\pgfqpoint{2.340138in}{1.540863in}}%
\pgfpathcurveto{\pgfqpoint{2.334314in}{1.546687in}}{\pgfqpoint{2.326414in}{1.549959in}}{\pgfqpoint{2.318177in}{1.549959in}}%
\pgfpathcurveto{\pgfqpoint{2.309941in}{1.549959in}}{\pgfqpoint{2.302041in}{1.546687in}}{\pgfqpoint{2.296217in}{1.540863in}}%
\pgfpathcurveto{\pgfqpoint{2.290393in}{1.535039in}}{\pgfqpoint{2.287121in}{1.527139in}}{\pgfqpoint{2.287121in}{1.518903in}}%
\pgfpathcurveto{\pgfqpoint{2.287121in}{1.510667in}}{\pgfqpoint{2.290393in}{1.502767in}}{\pgfqpoint{2.296217in}{1.496943in}}%
\pgfpathcurveto{\pgfqpoint{2.302041in}{1.491119in}}{\pgfqpoint{2.309941in}{1.487846in}}{\pgfqpoint{2.318177in}{1.487846in}}%
\pgfpathclose%
\pgfusepath{stroke,fill}%
\end{pgfscope}%
\begin{pgfscope}%
\pgfpathrectangle{\pgfqpoint{0.100000in}{0.212622in}}{\pgfqpoint{3.696000in}{3.696000in}}%
\pgfusepath{clip}%
\pgfsetbuttcap%
\pgfsetroundjoin%
\definecolor{currentfill}{rgb}{0.121569,0.466667,0.705882}%
\pgfsetfillcolor{currentfill}%
\pgfsetfillopacity{0.880019}%
\pgfsetlinewidth{1.003750pt}%
\definecolor{currentstroke}{rgb}{0.121569,0.466667,0.705882}%
\pgfsetstrokecolor{currentstroke}%
\pgfsetstrokeopacity{0.880019}%
\pgfsetdash{}{0pt}%
\pgfpathmoveto{\pgfqpoint{2.318392in}{1.487394in}}%
\pgfpathcurveto{\pgfqpoint{2.326628in}{1.487394in}}{\pgfqpoint{2.334528in}{1.490666in}}{\pgfqpoint{2.340352in}{1.496490in}}%
\pgfpathcurveto{\pgfqpoint{2.346176in}{1.502314in}}{\pgfqpoint{2.349448in}{1.510214in}}{\pgfqpoint{2.349448in}{1.518450in}}%
\pgfpathcurveto{\pgfqpoint{2.349448in}{1.526687in}}{\pgfqpoint{2.346176in}{1.534587in}}{\pgfqpoint{2.340352in}{1.540411in}}%
\pgfpathcurveto{\pgfqpoint{2.334528in}{1.546235in}}{\pgfqpoint{2.326628in}{1.549507in}}{\pgfqpoint{2.318392in}{1.549507in}}%
\pgfpathcurveto{\pgfqpoint{2.310155in}{1.549507in}}{\pgfqpoint{2.302255in}{1.546235in}}{\pgfqpoint{2.296431in}{1.540411in}}%
\pgfpathcurveto{\pgfqpoint{2.290607in}{1.534587in}}{\pgfqpoint{2.287335in}{1.526687in}}{\pgfqpoint{2.287335in}{1.518450in}}%
\pgfpathcurveto{\pgfqpoint{2.287335in}{1.510214in}}{\pgfqpoint{2.290607in}{1.502314in}}{\pgfqpoint{2.296431in}{1.496490in}}%
\pgfpathcurveto{\pgfqpoint{2.302255in}{1.490666in}}{\pgfqpoint{2.310155in}{1.487394in}}{\pgfqpoint{2.318392in}{1.487394in}}%
\pgfpathclose%
\pgfusepath{stroke,fill}%
\end{pgfscope}%
\begin{pgfscope}%
\pgfpathrectangle{\pgfqpoint{0.100000in}{0.212622in}}{\pgfqpoint{3.696000in}{3.696000in}}%
\pgfusepath{clip}%
\pgfsetbuttcap%
\pgfsetroundjoin%
\definecolor{currentfill}{rgb}{0.121569,0.466667,0.705882}%
\pgfsetfillcolor{currentfill}%
\pgfsetfillopacity{0.880698}%
\pgfsetlinewidth{1.003750pt}%
\definecolor{currentstroke}{rgb}{0.121569,0.466667,0.705882}%
\pgfsetstrokecolor{currentstroke}%
\pgfsetstrokeopacity{0.880698}%
\pgfsetdash{}{0pt}%
\pgfpathmoveto{\pgfqpoint{2.319006in}{1.486506in}}%
\pgfpathcurveto{\pgfqpoint{2.327242in}{1.486506in}}{\pgfqpoint{2.335142in}{1.489778in}}{\pgfqpoint{2.340966in}{1.495602in}}%
\pgfpathcurveto{\pgfqpoint{2.346790in}{1.501426in}}{\pgfqpoint{2.350062in}{1.509326in}}{\pgfqpoint{2.350062in}{1.517562in}}%
\pgfpathcurveto{\pgfqpoint{2.350062in}{1.525799in}}{\pgfqpoint{2.346790in}{1.533699in}}{\pgfqpoint{2.340966in}{1.539523in}}%
\pgfpathcurveto{\pgfqpoint{2.335142in}{1.545347in}}{\pgfqpoint{2.327242in}{1.548619in}}{\pgfqpoint{2.319006in}{1.548619in}}%
\pgfpathcurveto{\pgfqpoint{2.310769in}{1.548619in}}{\pgfqpoint{2.302869in}{1.545347in}}{\pgfqpoint{2.297045in}{1.539523in}}%
\pgfpathcurveto{\pgfqpoint{2.291221in}{1.533699in}}{\pgfqpoint{2.287949in}{1.525799in}}{\pgfqpoint{2.287949in}{1.517562in}}%
\pgfpathcurveto{\pgfqpoint{2.287949in}{1.509326in}}{\pgfqpoint{2.291221in}{1.501426in}}{\pgfqpoint{2.297045in}{1.495602in}}%
\pgfpathcurveto{\pgfqpoint{2.302869in}{1.489778in}}{\pgfqpoint{2.310769in}{1.486506in}}{\pgfqpoint{2.319006in}{1.486506in}}%
\pgfpathclose%
\pgfusepath{stroke,fill}%
\end{pgfscope}%
\begin{pgfscope}%
\pgfpathrectangle{\pgfqpoint{0.100000in}{0.212622in}}{\pgfqpoint{3.696000in}{3.696000in}}%
\pgfusepath{clip}%
\pgfsetbuttcap%
\pgfsetroundjoin%
\definecolor{currentfill}{rgb}{0.121569,0.466667,0.705882}%
\pgfsetfillcolor{currentfill}%
\pgfsetfillopacity{0.881095}%
\pgfsetlinewidth{1.003750pt}%
\definecolor{currentstroke}{rgb}{0.121569,0.466667,0.705882}%
\pgfsetstrokecolor{currentstroke}%
\pgfsetstrokeopacity{0.881095}%
\pgfsetdash{}{0pt}%
\pgfpathmoveto{\pgfqpoint{2.319359in}{1.486173in}}%
\pgfpathcurveto{\pgfqpoint{2.327595in}{1.486173in}}{\pgfqpoint{2.335495in}{1.489445in}}{\pgfqpoint{2.341319in}{1.495269in}}%
\pgfpathcurveto{\pgfqpoint{2.347143in}{1.501093in}}{\pgfqpoint{2.350415in}{1.508993in}}{\pgfqpoint{2.350415in}{1.517229in}}%
\pgfpathcurveto{\pgfqpoint{2.350415in}{1.525465in}}{\pgfqpoint{2.347143in}{1.533365in}}{\pgfqpoint{2.341319in}{1.539189in}}%
\pgfpathcurveto{\pgfqpoint{2.335495in}{1.545013in}}{\pgfqpoint{2.327595in}{1.548286in}}{\pgfqpoint{2.319359in}{1.548286in}}%
\pgfpathcurveto{\pgfqpoint{2.311123in}{1.548286in}}{\pgfqpoint{2.303223in}{1.545013in}}{\pgfqpoint{2.297399in}{1.539189in}}%
\pgfpathcurveto{\pgfqpoint{2.291575in}{1.533365in}}{\pgfqpoint{2.288302in}{1.525465in}}{\pgfqpoint{2.288302in}{1.517229in}}%
\pgfpathcurveto{\pgfqpoint{2.288302in}{1.508993in}}{\pgfqpoint{2.291575in}{1.501093in}}{\pgfqpoint{2.297399in}{1.495269in}}%
\pgfpathcurveto{\pgfqpoint{2.303223in}{1.489445in}}{\pgfqpoint{2.311123in}{1.486173in}}{\pgfqpoint{2.319359in}{1.486173in}}%
\pgfpathclose%
\pgfusepath{stroke,fill}%
\end{pgfscope}%
\begin{pgfscope}%
\pgfpathrectangle{\pgfqpoint{0.100000in}{0.212622in}}{\pgfqpoint{3.696000in}{3.696000in}}%
\pgfusepath{clip}%
\pgfsetbuttcap%
\pgfsetroundjoin%
\definecolor{currentfill}{rgb}{0.121569,0.466667,0.705882}%
\pgfsetfillcolor{currentfill}%
\pgfsetfillopacity{0.881631}%
\pgfsetlinewidth{1.003750pt}%
\definecolor{currentstroke}{rgb}{0.121569,0.466667,0.705882}%
\pgfsetstrokecolor{currentstroke}%
\pgfsetstrokeopacity{0.881631}%
\pgfsetdash{}{0pt}%
\pgfpathmoveto{\pgfqpoint{2.319749in}{1.484840in}}%
\pgfpathcurveto{\pgfqpoint{2.327985in}{1.484840in}}{\pgfqpoint{2.335885in}{1.488112in}}{\pgfqpoint{2.341709in}{1.493936in}}%
\pgfpathcurveto{\pgfqpoint{2.347533in}{1.499760in}}{\pgfqpoint{2.350805in}{1.507660in}}{\pgfqpoint{2.350805in}{1.515896in}}%
\pgfpathcurveto{\pgfqpoint{2.350805in}{1.524133in}}{\pgfqpoint{2.347533in}{1.532033in}}{\pgfqpoint{2.341709in}{1.537856in}}%
\pgfpathcurveto{\pgfqpoint{2.335885in}{1.543680in}}{\pgfqpoint{2.327985in}{1.546953in}}{\pgfqpoint{2.319749in}{1.546953in}}%
\pgfpathcurveto{\pgfqpoint{2.311512in}{1.546953in}}{\pgfqpoint{2.303612in}{1.543680in}}{\pgfqpoint{2.297788in}{1.537856in}}%
\pgfpathcurveto{\pgfqpoint{2.291964in}{1.532033in}}{\pgfqpoint{2.288692in}{1.524133in}}{\pgfqpoint{2.288692in}{1.515896in}}%
\pgfpathcurveto{\pgfqpoint{2.288692in}{1.507660in}}{\pgfqpoint{2.291964in}{1.499760in}}{\pgfqpoint{2.297788in}{1.493936in}}%
\pgfpathcurveto{\pgfqpoint{2.303612in}{1.488112in}}{\pgfqpoint{2.311512in}{1.484840in}}{\pgfqpoint{2.319749in}{1.484840in}}%
\pgfpathclose%
\pgfusepath{stroke,fill}%
\end{pgfscope}%
\begin{pgfscope}%
\pgfpathrectangle{\pgfqpoint{0.100000in}{0.212622in}}{\pgfqpoint{3.696000in}{3.696000in}}%
\pgfusepath{clip}%
\pgfsetbuttcap%
\pgfsetroundjoin%
\definecolor{currentfill}{rgb}{0.121569,0.466667,0.705882}%
\pgfsetfillcolor{currentfill}%
\pgfsetfillopacity{0.881945}%
\pgfsetlinewidth{1.003750pt}%
\definecolor{currentstroke}{rgb}{0.121569,0.466667,0.705882}%
\pgfsetstrokecolor{currentstroke}%
\pgfsetstrokeopacity{0.881945}%
\pgfsetdash{}{0pt}%
\pgfpathmoveto{\pgfqpoint{1.230234in}{2.139198in}}%
\pgfpathcurveto{\pgfqpoint{1.238470in}{2.139198in}}{\pgfqpoint{1.246370in}{2.142470in}}{\pgfqpoint{1.252194in}{2.148294in}}%
\pgfpathcurveto{\pgfqpoint{1.258018in}{2.154118in}}{\pgfqpoint{1.261291in}{2.162018in}}{\pgfqpoint{1.261291in}{2.170254in}}%
\pgfpathcurveto{\pgfqpoint{1.261291in}{2.178491in}}{\pgfqpoint{1.258018in}{2.186391in}}{\pgfqpoint{1.252194in}{2.192215in}}%
\pgfpathcurveto{\pgfqpoint{1.246370in}{2.198039in}}{\pgfqpoint{1.238470in}{2.201311in}}{\pgfqpoint{1.230234in}{2.201311in}}%
\pgfpathcurveto{\pgfqpoint{1.221998in}{2.201311in}}{\pgfqpoint{1.214098in}{2.198039in}}{\pgfqpoint{1.208274in}{2.192215in}}%
\pgfpathcurveto{\pgfqpoint{1.202450in}{2.186391in}}{\pgfqpoint{1.199178in}{2.178491in}}{\pgfqpoint{1.199178in}{2.170254in}}%
\pgfpathcurveto{\pgfqpoint{1.199178in}{2.162018in}}{\pgfqpoint{1.202450in}{2.154118in}}{\pgfqpoint{1.208274in}{2.148294in}}%
\pgfpathcurveto{\pgfqpoint{1.214098in}{2.142470in}}{\pgfqpoint{1.221998in}{2.139198in}}{\pgfqpoint{1.230234in}{2.139198in}}%
\pgfpathclose%
\pgfusepath{stroke,fill}%
\end{pgfscope}%
\begin{pgfscope}%
\pgfpathrectangle{\pgfqpoint{0.100000in}{0.212622in}}{\pgfqpoint{3.696000in}{3.696000in}}%
\pgfusepath{clip}%
\pgfsetbuttcap%
\pgfsetroundjoin%
\definecolor{currentfill}{rgb}{0.121569,0.466667,0.705882}%
\pgfsetfillcolor{currentfill}%
\pgfsetfillopacity{0.881997}%
\pgfsetlinewidth{1.003750pt}%
\definecolor{currentstroke}{rgb}{0.121569,0.466667,0.705882}%
\pgfsetstrokecolor{currentstroke}%
\pgfsetstrokeopacity{0.881997}%
\pgfsetdash{}{0pt}%
\pgfpathmoveto{\pgfqpoint{2.319953in}{1.484538in}}%
\pgfpathcurveto{\pgfqpoint{2.328190in}{1.484538in}}{\pgfqpoint{2.336090in}{1.487811in}}{\pgfqpoint{2.341914in}{1.493635in}}%
\pgfpathcurveto{\pgfqpoint{2.347737in}{1.499458in}}{\pgfqpoint{2.351010in}{1.507358in}}{\pgfqpoint{2.351010in}{1.515595in}}%
\pgfpathcurveto{\pgfqpoint{2.351010in}{1.523831in}}{\pgfqpoint{2.347737in}{1.531731in}}{\pgfqpoint{2.341914in}{1.537555in}}%
\pgfpathcurveto{\pgfqpoint{2.336090in}{1.543379in}}{\pgfqpoint{2.328190in}{1.546651in}}{\pgfqpoint{2.319953in}{1.546651in}}%
\pgfpathcurveto{\pgfqpoint{2.311717in}{1.546651in}}{\pgfqpoint{2.303817in}{1.543379in}}{\pgfqpoint{2.297993in}{1.537555in}}%
\pgfpathcurveto{\pgfqpoint{2.292169in}{1.531731in}}{\pgfqpoint{2.288897in}{1.523831in}}{\pgfqpoint{2.288897in}{1.515595in}}%
\pgfpathcurveto{\pgfqpoint{2.288897in}{1.507358in}}{\pgfqpoint{2.292169in}{1.499458in}}{\pgfqpoint{2.297993in}{1.493635in}}%
\pgfpathcurveto{\pgfqpoint{2.303817in}{1.487811in}}{\pgfqpoint{2.311717in}{1.484538in}}{\pgfqpoint{2.319953in}{1.484538in}}%
\pgfpathclose%
\pgfusepath{stroke,fill}%
\end{pgfscope}%
\begin{pgfscope}%
\pgfpathrectangle{\pgfqpoint{0.100000in}{0.212622in}}{\pgfqpoint{3.696000in}{3.696000in}}%
\pgfusepath{clip}%
\pgfsetbuttcap%
\pgfsetroundjoin%
\definecolor{currentfill}{rgb}{0.121569,0.466667,0.705882}%
\pgfsetfillcolor{currentfill}%
\pgfsetfillopacity{0.882205}%
\pgfsetlinewidth{1.003750pt}%
\definecolor{currentstroke}{rgb}{0.121569,0.466667,0.705882}%
\pgfsetstrokecolor{currentstroke}%
\pgfsetstrokeopacity{0.882205}%
\pgfsetdash{}{0pt}%
\pgfpathmoveto{\pgfqpoint{2.320093in}{1.484421in}}%
\pgfpathcurveto{\pgfqpoint{2.328329in}{1.484421in}}{\pgfqpoint{2.336229in}{1.487693in}}{\pgfqpoint{2.342053in}{1.493517in}}%
\pgfpathcurveto{\pgfqpoint{2.347877in}{1.499341in}}{\pgfqpoint{2.351149in}{1.507241in}}{\pgfqpoint{2.351149in}{1.515478in}}%
\pgfpathcurveto{\pgfqpoint{2.351149in}{1.523714in}}{\pgfqpoint{2.347877in}{1.531614in}}{\pgfqpoint{2.342053in}{1.537438in}}%
\pgfpathcurveto{\pgfqpoint{2.336229in}{1.543262in}}{\pgfqpoint{2.328329in}{1.546534in}}{\pgfqpoint{2.320093in}{1.546534in}}%
\pgfpathcurveto{\pgfqpoint{2.311857in}{1.546534in}}{\pgfqpoint{2.303957in}{1.543262in}}{\pgfqpoint{2.298133in}{1.537438in}}%
\pgfpathcurveto{\pgfqpoint{2.292309in}{1.531614in}}{\pgfqpoint{2.289036in}{1.523714in}}{\pgfqpoint{2.289036in}{1.515478in}}%
\pgfpathcurveto{\pgfqpoint{2.289036in}{1.507241in}}{\pgfqpoint{2.292309in}{1.499341in}}{\pgfqpoint{2.298133in}{1.493517in}}%
\pgfpathcurveto{\pgfqpoint{2.303957in}{1.487693in}}{\pgfqpoint{2.311857in}{1.484421in}}{\pgfqpoint{2.320093in}{1.484421in}}%
\pgfpathclose%
\pgfusepath{stroke,fill}%
\end{pgfscope}%
\begin{pgfscope}%
\pgfpathrectangle{\pgfqpoint{0.100000in}{0.212622in}}{\pgfqpoint{3.696000in}{3.696000in}}%
\pgfusepath{clip}%
\pgfsetbuttcap%
\pgfsetroundjoin%
\definecolor{currentfill}{rgb}{0.121569,0.466667,0.705882}%
\pgfsetfillcolor{currentfill}%
\pgfsetfillopacity{0.882832}%
\pgfsetlinewidth{1.003750pt}%
\definecolor{currentstroke}{rgb}{0.121569,0.466667,0.705882}%
\pgfsetstrokecolor{currentstroke}%
\pgfsetstrokeopacity{0.882832}%
\pgfsetdash{}{0pt}%
\pgfpathmoveto{\pgfqpoint{2.320615in}{1.483043in}}%
\pgfpathcurveto{\pgfqpoint{2.328851in}{1.483043in}}{\pgfqpoint{2.336751in}{1.486316in}}{\pgfqpoint{2.342575in}{1.492140in}}%
\pgfpathcurveto{\pgfqpoint{2.348399in}{1.497964in}}{\pgfqpoint{2.351671in}{1.505864in}}{\pgfqpoint{2.351671in}{1.514100in}}%
\pgfpathcurveto{\pgfqpoint{2.351671in}{1.522336in}}{\pgfqpoint{2.348399in}{1.530236in}}{\pgfqpoint{2.342575in}{1.536060in}}%
\pgfpathcurveto{\pgfqpoint{2.336751in}{1.541884in}}{\pgfqpoint{2.328851in}{1.545156in}}{\pgfqpoint{2.320615in}{1.545156in}}%
\pgfpathcurveto{\pgfqpoint{2.312378in}{1.545156in}}{\pgfqpoint{2.304478in}{1.541884in}}{\pgfqpoint{2.298654in}{1.536060in}}%
\pgfpathcurveto{\pgfqpoint{2.292830in}{1.530236in}}{\pgfqpoint{2.289558in}{1.522336in}}{\pgfqpoint{2.289558in}{1.514100in}}%
\pgfpathcurveto{\pgfqpoint{2.289558in}{1.505864in}}{\pgfqpoint{2.292830in}{1.497964in}}{\pgfqpoint{2.298654in}{1.492140in}}%
\pgfpathcurveto{\pgfqpoint{2.304478in}{1.486316in}}{\pgfqpoint{2.312378in}{1.483043in}}{\pgfqpoint{2.320615in}{1.483043in}}%
\pgfpathclose%
\pgfusepath{stroke,fill}%
\end{pgfscope}%
\begin{pgfscope}%
\pgfpathrectangle{\pgfqpoint{0.100000in}{0.212622in}}{\pgfqpoint{3.696000in}{3.696000in}}%
\pgfusepath{clip}%
\pgfsetbuttcap%
\pgfsetroundjoin%
\definecolor{currentfill}{rgb}{0.121569,0.466667,0.705882}%
\pgfsetfillcolor{currentfill}%
\pgfsetfillopacity{0.883565}%
\pgfsetlinewidth{1.003750pt}%
\definecolor{currentstroke}{rgb}{0.121569,0.466667,0.705882}%
\pgfsetstrokecolor{currentstroke}%
\pgfsetstrokeopacity{0.883565}%
\pgfsetdash{}{0pt}%
\pgfpathmoveto{\pgfqpoint{2.321332in}{1.480606in}}%
\pgfpathcurveto{\pgfqpoint{2.329569in}{1.480606in}}{\pgfqpoint{2.337469in}{1.483878in}}{\pgfqpoint{2.343293in}{1.489702in}}%
\pgfpathcurveto{\pgfqpoint{2.349117in}{1.495526in}}{\pgfqpoint{2.352389in}{1.503426in}}{\pgfqpoint{2.352389in}{1.511662in}}%
\pgfpathcurveto{\pgfqpoint{2.352389in}{1.519899in}}{\pgfqpoint{2.349117in}{1.527799in}}{\pgfqpoint{2.343293in}{1.533623in}}%
\pgfpathcurveto{\pgfqpoint{2.337469in}{1.539446in}}{\pgfqpoint{2.329569in}{1.542719in}}{\pgfqpoint{2.321332in}{1.542719in}}%
\pgfpathcurveto{\pgfqpoint{2.313096in}{1.542719in}}{\pgfqpoint{2.305196in}{1.539446in}}{\pgfqpoint{2.299372in}{1.533623in}}%
\pgfpathcurveto{\pgfqpoint{2.293548in}{1.527799in}}{\pgfqpoint{2.290276in}{1.519899in}}{\pgfqpoint{2.290276in}{1.511662in}}%
\pgfpathcurveto{\pgfqpoint{2.290276in}{1.503426in}}{\pgfqpoint{2.293548in}{1.495526in}}{\pgfqpoint{2.299372in}{1.489702in}}%
\pgfpathcurveto{\pgfqpoint{2.305196in}{1.483878in}}{\pgfqpoint{2.313096in}{1.480606in}}{\pgfqpoint{2.321332in}{1.480606in}}%
\pgfpathclose%
\pgfusepath{stroke,fill}%
\end{pgfscope}%
\begin{pgfscope}%
\pgfpathrectangle{\pgfqpoint{0.100000in}{0.212622in}}{\pgfqpoint{3.696000in}{3.696000in}}%
\pgfusepath{clip}%
\pgfsetbuttcap%
\pgfsetroundjoin%
\definecolor{currentfill}{rgb}{0.121569,0.466667,0.705882}%
\pgfsetfillcolor{currentfill}%
\pgfsetfillopacity{0.885011}%
\pgfsetlinewidth{1.003750pt}%
\definecolor{currentstroke}{rgb}{0.121569,0.466667,0.705882}%
\pgfsetstrokecolor{currentstroke}%
\pgfsetstrokeopacity{0.885011}%
\pgfsetdash{}{0pt}%
\pgfpathmoveto{\pgfqpoint{2.322289in}{1.478816in}}%
\pgfpathcurveto{\pgfqpoint{2.330526in}{1.478816in}}{\pgfqpoint{2.338426in}{1.482089in}}{\pgfqpoint{2.344250in}{1.487913in}}%
\pgfpathcurveto{\pgfqpoint{2.350073in}{1.493737in}}{\pgfqpoint{2.353346in}{1.501637in}}{\pgfqpoint{2.353346in}{1.509873in}}%
\pgfpathcurveto{\pgfqpoint{2.353346in}{1.518109in}}{\pgfqpoint{2.350073in}{1.526009in}}{\pgfqpoint{2.344250in}{1.531833in}}%
\pgfpathcurveto{\pgfqpoint{2.338426in}{1.537657in}}{\pgfqpoint{2.330526in}{1.540929in}}{\pgfqpoint{2.322289in}{1.540929in}}%
\pgfpathcurveto{\pgfqpoint{2.314053in}{1.540929in}}{\pgfqpoint{2.306153in}{1.537657in}}{\pgfqpoint{2.300329in}{1.531833in}}%
\pgfpathcurveto{\pgfqpoint{2.294505in}{1.526009in}}{\pgfqpoint{2.291233in}{1.518109in}}{\pgfqpoint{2.291233in}{1.509873in}}%
\pgfpathcurveto{\pgfqpoint{2.291233in}{1.501637in}}{\pgfqpoint{2.294505in}{1.493737in}}{\pgfqpoint{2.300329in}{1.487913in}}%
\pgfpathcurveto{\pgfqpoint{2.306153in}{1.482089in}}{\pgfqpoint{2.314053in}{1.478816in}}{\pgfqpoint{2.322289in}{1.478816in}}%
\pgfpathclose%
\pgfusepath{stroke,fill}%
\end{pgfscope}%
\begin{pgfscope}%
\pgfpathrectangle{\pgfqpoint{0.100000in}{0.212622in}}{\pgfqpoint{3.696000in}{3.696000in}}%
\pgfusepath{clip}%
\pgfsetbuttcap%
\pgfsetroundjoin%
\definecolor{currentfill}{rgb}{0.121569,0.466667,0.705882}%
\pgfsetfillcolor{currentfill}%
\pgfsetfillopacity{0.886792}%
\pgfsetlinewidth{1.003750pt}%
\definecolor{currentstroke}{rgb}{0.121569,0.466667,0.705882}%
\pgfsetstrokecolor{currentstroke}%
\pgfsetstrokeopacity{0.886792}%
\pgfsetdash{}{0pt}%
\pgfpathmoveto{\pgfqpoint{2.323272in}{1.477241in}}%
\pgfpathcurveto{\pgfqpoint{2.331508in}{1.477241in}}{\pgfqpoint{2.339408in}{1.480514in}}{\pgfqpoint{2.345232in}{1.486338in}}%
\pgfpathcurveto{\pgfqpoint{2.351056in}{1.492162in}}{\pgfqpoint{2.354329in}{1.500062in}}{\pgfqpoint{2.354329in}{1.508298in}}%
\pgfpathcurveto{\pgfqpoint{2.354329in}{1.516534in}}{\pgfqpoint{2.351056in}{1.524434in}}{\pgfqpoint{2.345232in}{1.530258in}}%
\pgfpathcurveto{\pgfqpoint{2.339408in}{1.536082in}}{\pgfqpoint{2.331508in}{1.539354in}}{\pgfqpoint{2.323272in}{1.539354in}}%
\pgfpathcurveto{\pgfqpoint{2.315036in}{1.539354in}}{\pgfqpoint{2.307136in}{1.536082in}}{\pgfqpoint{2.301312in}{1.530258in}}%
\pgfpathcurveto{\pgfqpoint{2.295488in}{1.524434in}}{\pgfqpoint{2.292216in}{1.516534in}}{\pgfqpoint{2.292216in}{1.508298in}}%
\pgfpathcurveto{\pgfqpoint{2.292216in}{1.500062in}}{\pgfqpoint{2.295488in}{1.492162in}}{\pgfqpoint{2.301312in}{1.486338in}}%
\pgfpathcurveto{\pgfqpoint{2.307136in}{1.480514in}}{\pgfqpoint{2.315036in}{1.477241in}}{\pgfqpoint{2.323272in}{1.477241in}}%
\pgfpathclose%
\pgfusepath{stroke,fill}%
\end{pgfscope}%
\begin{pgfscope}%
\pgfpathrectangle{\pgfqpoint{0.100000in}{0.212622in}}{\pgfqpoint{3.696000in}{3.696000in}}%
\pgfusepath{clip}%
\pgfsetbuttcap%
\pgfsetroundjoin%
\definecolor{currentfill}{rgb}{0.121569,0.466667,0.705882}%
\pgfsetfillcolor{currentfill}%
\pgfsetfillopacity{0.887800}%
\pgfsetlinewidth{1.003750pt}%
\definecolor{currentstroke}{rgb}{0.121569,0.466667,0.705882}%
\pgfsetstrokecolor{currentstroke}%
\pgfsetstrokeopacity{0.887800}%
\pgfsetdash{}{0pt}%
\pgfpathmoveto{\pgfqpoint{2.323894in}{1.476583in}}%
\pgfpathcurveto{\pgfqpoint{2.332130in}{1.476583in}}{\pgfqpoint{2.340030in}{1.479855in}}{\pgfqpoint{2.345854in}{1.485679in}}%
\pgfpathcurveto{\pgfqpoint{2.351678in}{1.491503in}}{\pgfqpoint{2.354950in}{1.499403in}}{\pgfqpoint{2.354950in}{1.507639in}}%
\pgfpathcurveto{\pgfqpoint{2.354950in}{1.515875in}}{\pgfqpoint{2.351678in}{1.523775in}}{\pgfqpoint{2.345854in}{1.529599in}}%
\pgfpathcurveto{\pgfqpoint{2.340030in}{1.535423in}}{\pgfqpoint{2.332130in}{1.538696in}}{\pgfqpoint{2.323894in}{1.538696in}}%
\pgfpathcurveto{\pgfqpoint{2.315657in}{1.538696in}}{\pgfqpoint{2.307757in}{1.535423in}}{\pgfqpoint{2.301933in}{1.529599in}}%
\pgfpathcurveto{\pgfqpoint{2.296109in}{1.523775in}}{\pgfqpoint{2.292837in}{1.515875in}}{\pgfqpoint{2.292837in}{1.507639in}}%
\pgfpathcurveto{\pgfqpoint{2.292837in}{1.499403in}}{\pgfqpoint{2.296109in}{1.491503in}}{\pgfqpoint{2.301933in}{1.485679in}}%
\pgfpathcurveto{\pgfqpoint{2.307757in}{1.479855in}}{\pgfqpoint{2.315657in}{1.476583in}}{\pgfqpoint{2.323894in}{1.476583in}}%
\pgfpathclose%
\pgfusepath{stroke,fill}%
\end{pgfscope}%
\begin{pgfscope}%
\pgfpathrectangle{\pgfqpoint{0.100000in}{0.212622in}}{\pgfqpoint{3.696000in}{3.696000in}}%
\pgfusepath{clip}%
\pgfsetbuttcap%
\pgfsetroundjoin%
\definecolor{currentfill}{rgb}{0.121569,0.466667,0.705882}%
\pgfsetfillcolor{currentfill}%
\pgfsetfillopacity{0.887833}%
\pgfsetlinewidth{1.003750pt}%
\definecolor{currentstroke}{rgb}{0.121569,0.466667,0.705882}%
\pgfsetstrokecolor{currentstroke}%
\pgfsetstrokeopacity{0.887833}%
\pgfsetdash{}{0pt}%
\pgfpathmoveto{\pgfqpoint{1.277585in}{2.106482in}}%
\pgfpathcurveto{\pgfqpoint{1.285821in}{2.106482in}}{\pgfqpoint{1.293721in}{2.109754in}}{\pgfqpoint{1.299545in}{2.115578in}}%
\pgfpathcurveto{\pgfqpoint{1.305369in}{2.121402in}}{\pgfqpoint{1.308641in}{2.129302in}}{\pgfqpoint{1.308641in}{2.137538in}}%
\pgfpathcurveto{\pgfqpoint{1.308641in}{2.145774in}}{\pgfqpoint{1.305369in}{2.153675in}}{\pgfqpoint{1.299545in}{2.159498in}}%
\pgfpathcurveto{\pgfqpoint{1.293721in}{2.165322in}}{\pgfqpoint{1.285821in}{2.168595in}}{\pgfqpoint{1.277585in}{2.168595in}}%
\pgfpathcurveto{\pgfqpoint{1.269349in}{2.168595in}}{\pgfqpoint{1.261448in}{2.165322in}}{\pgfqpoint{1.255625in}{2.159498in}}%
\pgfpathcurveto{\pgfqpoint{1.249801in}{2.153675in}}{\pgfqpoint{1.246528in}{2.145774in}}{\pgfqpoint{1.246528in}{2.137538in}}%
\pgfpathcurveto{\pgfqpoint{1.246528in}{2.129302in}}{\pgfqpoint{1.249801in}{2.121402in}}{\pgfqpoint{1.255625in}{2.115578in}}%
\pgfpathcurveto{\pgfqpoint{1.261448in}{2.109754in}}{\pgfqpoint{1.269349in}{2.106482in}}{\pgfqpoint{1.277585in}{2.106482in}}%
\pgfpathclose%
\pgfusepath{stroke,fill}%
\end{pgfscope}%
\begin{pgfscope}%
\pgfpathrectangle{\pgfqpoint{0.100000in}{0.212622in}}{\pgfqpoint{3.696000in}{3.696000in}}%
\pgfusepath{clip}%
\pgfsetbuttcap%
\pgfsetroundjoin%
\definecolor{currentfill}{rgb}{0.121569,0.466667,0.705882}%
\pgfsetfillcolor{currentfill}%
\pgfsetfillopacity{0.889006}%
\pgfsetlinewidth{1.003750pt}%
\definecolor{currentstroke}{rgb}{0.121569,0.466667,0.705882}%
\pgfsetstrokecolor{currentstroke}%
\pgfsetstrokeopacity{0.889006}%
\pgfsetdash{}{0pt}%
\pgfpathmoveto{\pgfqpoint{2.324612in}{1.475443in}}%
\pgfpathcurveto{\pgfqpoint{2.332848in}{1.475443in}}{\pgfqpoint{2.340748in}{1.478715in}}{\pgfqpoint{2.346572in}{1.484539in}}%
\pgfpathcurveto{\pgfqpoint{2.352396in}{1.490363in}}{\pgfqpoint{2.355668in}{1.498263in}}{\pgfqpoint{2.355668in}{1.506499in}}%
\pgfpathcurveto{\pgfqpoint{2.355668in}{1.514735in}}{\pgfqpoint{2.352396in}{1.522635in}}{\pgfqpoint{2.346572in}{1.528459in}}%
\pgfpathcurveto{\pgfqpoint{2.340748in}{1.534283in}}{\pgfqpoint{2.332848in}{1.537556in}}{\pgfqpoint{2.324612in}{1.537556in}}%
\pgfpathcurveto{\pgfqpoint{2.316376in}{1.537556in}}{\pgfqpoint{2.308476in}{1.534283in}}{\pgfqpoint{2.302652in}{1.528459in}}%
\pgfpathcurveto{\pgfqpoint{2.296828in}{1.522635in}}{\pgfqpoint{2.293555in}{1.514735in}}{\pgfqpoint{2.293555in}{1.506499in}}%
\pgfpathcurveto{\pgfqpoint{2.293555in}{1.498263in}}{\pgfqpoint{2.296828in}{1.490363in}}{\pgfqpoint{2.302652in}{1.484539in}}%
\pgfpathcurveto{\pgfqpoint{2.308476in}{1.478715in}}{\pgfqpoint{2.316376in}{1.475443in}}{\pgfqpoint{2.324612in}{1.475443in}}%
\pgfpathclose%
\pgfusepath{stroke,fill}%
\end{pgfscope}%
\begin{pgfscope}%
\pgfpathrectangle{\pgfqpoint{0.100000in}{0.212622in}}{\pgfqpoint{3.696000in}{3.696000in}}%
\pgfusepath{clip}%
\pgfsetbuttcap%
\pgfsetroundjoin%
\definecolor{currentfill}{rgb}{0.121569,0.466667,0.705882}%
\pgfsetfillcolor{currentfill}%
\pgfsetfillopacity{0.889580}%
\pgfsetlinewidth{1.003750pt}%
\definecolor{currentstroke}{rgb}{0.121569,0.466667,0.705882}%
\pgfsetstrokecolor{currentstroke}%
\pgfsetstrokeopacity{0.889580}%
\pgfsetdash{}{0pt}%
\pgfpathmoveto{\pgfqpoint{2.325207in}{1.474385in}}%
\pgfpathcurveto{\pgfqpoint{2.333444in}{1.474385in}}{\pgfqpoint{2.341344in}{1.477658in}}{\pgfqpoint{2.347168in}{1.483482in}}%
\pgfpathcurveto{\pgfqpoint{2.352992in}{1.489306in}}{\pgfqpoint{2.356264in}{1.497206in}}{\pgfqpoint{2.356264in}{1.505442in}}%
\pgfpathcurveto{\pgfqpoint{2.356264in}{1.513678in}}{\pgfqpoint{2.352992in}{1.521578in}}{\pgfqpoint{2.347168in}{1.527402in}}%
\pgfpathcurveto{\pgfqpoint{2.341344in}{1.533226in}}{\pgfqpoint{2.333444in}{1.536498in}}{\pgfqpoint{2.325207in}{1.536498in}}%
\pgfpathcurveto{\pgfqpoint{2.316971in}{1.536498in}}{\pgfqpoint{2.309071in}{1.533226in}}{\pgfqpoint{2.303247in}{1.527402in}}%
\pgfpathcurveto{\pgfqpoint{2.297423in}{1.521578in}}{\pgfqpoint{2.294151in}{1.513678in}}{\pgfqpoint{2.294151in}{1.505442in}}%
\pgfpathcurveto{\pgfqpoint{2.294151in}{1.497206in}}{\pgfqpoint{2.297423in}{1.489306in}}{\pgfqpoint{2.303247in}{1.483482in}}%
\pgfpathcurveto{\pgfqpoint{2.309071in}{1.477658in}}{\pgfqpoint{2.316971in}{1.474385in}}{\pgfqpoint{2.325207in}{1.474385in}}%
\pgfpathclose%
\pgfusepath{stroke,fill}%
\end{pgfscope}%
\begin{pgfscope}%
\pgfpathrectangle{\pgfqpoint{0.100000in}{0.212622in}}{\pgfqpoint{3.696000in}{3.696000in}}%
\pgfusepath{clip}%
\pgfsetbuttcap%
\pgfsetroundjoin%
\definecolor{currentfill}{rgb}{0.121569,0.466667,0.705882}%
\pgfsetfillcolor{currentfill}%
\pgfsetfillopacity{0.890649}%
\pgfsetlinewidth{1.003750pt}%
\definecolor{currentstroke}{rgb}{0.121569,0.466667,0.705882}%
\pgfsetstrokecolor{currentstroke}%
\pgfsetstrokeopacity{0.890649}%
\pgfsetdash{}{0pt}%
\pgfpathmoveto{\pgfqpoint{2.325826in}{1.474141in}}%
\pgfpathcurveto{\pgfqpoint{2.334062in}{1.474141in}}{\pgfqpoint{2.341962in}{1.477413in}}{\pgfqpoint{2.347786in}{1.483237in}}%
\pgfpathcurveto{\pgfqpoint{2.353610in}{1.489061in}}{\pgfqpoint{2.356882in}{1.496961in}}{\pgfqpoint{2.356882in}{1.505197in}}%
\pgfpathcurveto{\pgfqpoint{2.356882in}{1.513434in}}{\pgfqpoint{2.353610in}{1.521334in}}{\pgfqpoint{2.347786in}{1.527158in}}%
\pgfpathcurveto{\pgfqpoint{2.341962in}{1.532982in}}{\pgfqpoint{2.334062in}{1.536254in}}{\pgfqpoint{2.325826in}{1.536254in}}%
\pgfpathcurveto{\pgfqpoint{2.317589in}{1.536254in}}{\pgfqpoint{2.309689in}{1.532982in}}{\pgfqpoint{2.303865in}{1.527158in}}%
\pgfpathcurveto{\pgfqpoint{2.298041in}{1.521334in}}{\pgfqpoint{2.294769in}{1.513434in}}{\pgfqpoint{2.294769in}{1.505197in}}%
\pgfpathcurveto{\pgfqpoint{2.294769in}{1.496961in}}{\pgfqpoint{2.298041in}{1.489061in}}{\pgfqpoint{2.303865in}{1.483237in}}%
\pgfpathcurveto{\pgfqpoint{2.309689in}{1.477413in}}{\pgfqpoint{2.317589in}{1.474141in}}{\pgfqpoint{2.325826in}{1.474141in}}%
\pgfpathclose%
\pgfusepath{stroke,fill}%
\end{pgfscope}%
\begin{pgfscope}%
\pgfpathrectangle{\pgfqpoint{0.100000in}{0.212622in}}{\pgfqpoint{3.696000in}{3.696000in}}%
\pgfusepath{clip}%
\pgfsetbuttcap%
\pgfsetroundjoin%
\definecolor{currentfill}{rgb}{0.121569,0.466667,0.705882}%
\pgfsetfillcolor{currentfill}%
\pgfsetfillopacity{0.891246}%
\pgfsetlinewidth{1.003750pt}%
\definecolor{currentstroke}{rgb}{0.121569,0.466667,0.705882}%
\pgfsetstrokecolor{currentstroke}%
\pgfsetstrokeopacity{0.891246}%
\pgfsetdash{}{0pt}%
\pgfpathmoveto{\pgfqpoint{2.326154in}{1.474060in}}%
\pgfpathcurveto{\pgfqpoint{2.334391in}{1.474060in}}{\pgfqpoint{2.342291in}{1.477333in}}{\pgfqpoint{2.348115in}{1.483156in}}%
\pgfpathcurveto{\pgfqpoint{2.353939in}{1.488980in}}{\pgfqpoint{2.357211in}{1.496880in}}{\pgfqpoint{2.357211in}{1.505117in}}%
\pgfpathcurveto{\pgfqpoint{2.357211in}{1.513353in}}{\pgfqpoint{2.353939in}{1.521253in}}{\pgfqpoint{2.348115in}{1.527077in}}%
\pgfpathcurveto{\pgfqpoint{2.342291in}{1.532901in}}{\pgfqpoint{2.334391in}{1.536173in}}{\pgfqpoint{2.326154in}{1.536173in}}%
\pgfpathcurveto{\pgfqpoint{2.317918in}{1.536173in}}{\pgfqpoint{2.310018in}{1.532901in}}{\pgfqpoint{2.304194in}{1.527077in}}%
\pgfpathcurveto{\pgfqpoint{2.298370in}{1.521253in}}{\pgfqpoint{2.295098in}{1.513353in}}{\pgfqpoint{2.295098in}{1.505117in}}%
\pgfpathcurveto{\pgfqpoint{2.295098in}{1.496880in}}{\pgfqpoint{2.298370in}{1.488980in}}{\pgfqpoint{2.304194in}{1.483156in}}%
\pgfpathcurveto{\pgfqpoint{2.310018in}{1.477333in}}{\pgfqpoint{2.317918in}{1.474060in}}{\pgfqpoint{2.326154in}{1.474060in}}%
\pgfpathclose%
\pgfusepath{stroke,fill}%
\end{pgfscope}%
\begin{pgfscope}%
\pgfpathrectangle{\pgfqpoint{0.100000in}{0.212622in}}{\pgfqpoint{3.696000in}{3.696000in}}%
\pgfusepath{clip}%
\pgfsetbuttcap%
\pgfsetroundjoin%
\definecolor{currentfill}{rgb}{0.121569,0.466667,0.705882}%
\pgfsetfillcolor{currentfill}%
\pgfsetfillopacity{0.891297}%
\pgfsetlinewidth{1.003750pt}%
\definecolor{currentstroke}{rgb}{0.121569,0.466667,0.705882}%
\pgfsetstrokecolor{currentstroke}%
\pgfsetstrokeopacity{0.891297}%
\pgfsetdash{}{0pt}%
\pgfpathmoveto{\pgfqpoint{1.324012in}{2.065687in}}%
\pgfpathcurveto{\pgfqpoint{1.332248in}{2.065687in}}{\pgfqpoint{1.340148in}{2.068959in}}{\pgfqpoint{1.345972in}{2.074783in}}%
\pgfpathcurveto{\pgfqpoint{1.351796in}{2.080607in}}{\pgfqpoint{1.355068in}{2.088507in}}{\pgfqpoint{1.355068in}{2.096743in}}%
\pgfpathcurveto{\pgfqpoint{1.355068in}{2.104980in}}{\pgfqpoint{1.351796in}{2.112880in}}{\pgfqpoint{1.345972in}{2.118704in}}%
\pgfpathcurveto{\pgfqpoint{1.340148in}{2.124528in}}{\pgfqpoint{1.332248in}{2.127800in}}{\pgfqpoint{1.324012in}{2.127800in}}%
\pgfpathcurveto{\pgfqpoint{1.315775in}{2.127800in}}{\pgfqpoint{1.307875in}{2.124528in}}{\pgfqpoint{1.302051in}{2.118704in}}%
\pgfpathcurveto{\pgfqpoint{1.296227in}{2.112880in}}{\pgfqpoint{1.292955in}{2.104980in}}{\pgfqpoint{1.292955in}{2.096743in}}%
\pgfpathcurveto{\pgfqpoint{1.292955in}{2.088507in}}{\pgfqpoint{1.296227in}{2.080607in}}{\pgfqpoint{1.302051in}{2.074783in}}%
\pgfpathcurveto{\pgfqpoint{1.307875in}{2.068959in}}{\pgfqpoint{1.315775in}{2.065687in}}{\pgfqpoint{1.324012in}{2.065687in}}%
\pgfpathclose%
\pgfusepath{stroke,fill}%
\end{pgfscope}%
\begin{pgfscope}%
\pgfpathrectangle{\pgfqpoint{0.100000in}{0.212622in}}{\pgfqpoint{3.696000in}{3.696000in}}%
\pgfusepath{clip}%
\pgfsetbuttcap%
\pgfsetroundjoin%
\definecolor{currentfill}{rgb}{0.121569,0.466667,0.705882}%
\pgfsetfillcolor{currentfill}%
\pgfsetfillopacity{0.891859}%
\pgfsetlinewidth{1.003750pt}%
\definecolor{currentstroke}{rgb}{0.121569,0.466667,0.705882}%
\pgfsetstrokecolor{currentstroke}%
\pgfsetstrokeopacity{0.891859}%
\pgfsetdash{}{0pt}%
\pgfpathmoveto{\pgfqpoint{2.326766in}{1.472930in}}%
\pgfpathcurveto{\pgfqpoint{2.335002in}{1.472930in}}{\pgfqpoint{2.342902in}{1.476202in}}{\pgfqpoint{2.348726in}{1.482026in}}%
\pgfpathcurveto{\pgfqpoint{2.354550in}{1.487850in}}{\pgfqpoint{2.357822in}{1.495750in}}{\pgfqpoint{2.357822in}{1.503986in}}%
\pgfpathcurveto{\pgfqpoint{2.357822in}{1.512222in}}{\pgfqpoint{2.354550in}{1.520122in}}{\pgfqpoint{2.348726in}{1.525946in}}%
\pgfpathcurveto{\pgfqpoint{2.342902in}{1.531770in}}{\pgfqpoint{2.335002in}{1.535043in}}{\pgfqpoint{2.326766in}{1.535043in}}%
\pgfpathcurveto{\pgfqpoint{2.318529in}{1.535043in}}{\pgfqpoint{2.310629in}{1.531770in}}{\pgfqpoint{2.304805in}{1.525946in}}%
\pgfpathcurveto{\pgfqpoint{2.298981in}{1.520122in}}{\pgfqpoint{2.295709in}{1.512222in}}{\pgfqpoint{2.295709in}{1.503986in}}%
\pgfpathcurveto{\pgfqpoint{2.295709in}{1.495750in}}{\pgfqpoint{2.298981in}{1.487850in}}{\pgfqpoint{2.304805in}{1.482026in}}%
\pgfpathcurveto{\pgfqpoint{2.310629in}{1.476202in}}{\pgfqpoint{2.318529in}{1.472930in}}{\pgfqpoint{2.326766in}{1.472930in}}%
\pgfpathclose%
\pgfusepath{stroke,fill}%
\end{pgfscope}%
\begin{pgfscope}%
\pgfpathrectangle{\pgfqpoint{0.100000in}{0.212622in}}{\pgfqpoint{3.696000in}{3.696000in}}%
\pgfusepath{clip}%
\pgfsetbuttcap%
\pgfsetroundjoin%
\definecolor{currentfill}{rgb}{0.121569,0.466667,0.705882}%
\pgfsetfillcolor{currentfill}%
\pgfsetfillopacity{0.892682}%
\pgfsetlinewidth{1.003750pt}%
\definecolor{currentstroke}{rgb}{0.121569,0.466667,0.705882}%
\pgfsetstrokecolor{currentstroke}%
\pgfsetstrokeopacity{0.892682}%
\pgfsetdash{}{0pt}%
\pgfpathmoveto{\pgfqpoint{2.327892in}{1.469829in}}%
\pgfpathcurveto{\pgfqpoint{2.336128in}{1.469829in}}{\pgfqpoint{2.344028in}{1.473101in}}{\pgfqpoint{2.349852in}{1.478925in}}%
\pgfpathcurveto{\pgfqpoint{2.355676in}{1.484749in}}{\pgfqpoint{2.358948in}{1.492649in}}{\pgfqpoint{2.358948in}{1.500885in}}%
\pgfpathcurveto{\pgfqpoint{2.358948in}{1.509121in}}{\pgfqpoint{2.355676in}{1.517021in}}{\pgfqpoint{2.349852in}{1.522845in}}%
\pgfpathcurveto{\pgfqpoint{2.344028in}{1.528669in}}{\pgfqpoint{2.336128in}{1.531942in}}{\pgfqpoint{2.327892in}{1.531942in}}%
\pgfpathcurveto{\pgfqpoint{2.319655in}{1.531942in}}{\pgfqpoint{2.311755in}{1.528669in}}{\pgfqpoint{2.305931in}{1.522845in}}%
\pgfpathcurveto{\pgfqpoint{2.300107in}{1.517021in}}{\pgfqpoint{2.296835in}{1.509121in}}{\pgfqpoint{2.296835in}{1.500885in}}%
\pgfpathcurveto{\pgfqpoint{2.296835in}{1.492649in}}{\pgfqpoint{2.300107in}{1.484749in}}{\pgfqpoint{2.305931in}{1.478925in}}%
\pgfpathcurveto{\pgfqpoint{2.311755in}{1.473101in}}{\pgfqpoint{2.319655in}{1.469829in}}{\pgfqpoint{2.327892in}{1.469829in}}%
\pgfpathclose%
\pgfusepath{stroke,fill}%
\end{pgfscope}%
\begin{pgfscope}%
\pgfpathrectangle{\pgfqpoint{0.100000in}{0.212622in}}{\pgfqpoint{3.696000in}{3.696000in}}%
\pgfusepath{clip}%
\pgfsetbuttcap%
\pgfsetroundjoin%
\definecolor{currentfill}{rgb}{0.121569,0.466667,0.705882}%
\pgfsetfillcolor{currentfill}%
\pgfsetfillopacity{0.894739}%
\pgfsetlinewidth{1.003750pt}%
\definecolor{currentstroke}{rgb}{0.121569,0.466667,0.705882}%
\pgfsetstrokecolor{currentstroke}%
\pgfsetstrokeopacity{0.894739}%
\pgfsetdash{}{0pt}%
\pgfpathmoveto{\pgfqpoint{2.329384in}{1.468248in}}%
\pgfpathcurveto{\pgfqpoint{2.337620in}{1.468248in}}{\pgfqpoint{2.345520in}{1.471520in}}{\pgfqpoint{2.351344in}{1.477344in}}%
\pgfpathcurveto{\pgfqpoint{2.357168in}{1.483168in}}{\pgfqpoint{2.360441in}{1.491068in}}{\pgfqpoint{2.360441in}{1.499304in}}%
\pgfpathcurveto{\pgfqpoint{2.360441in}{1.507540in}}{\pgfqpoint{2.357168in}{1.515440in}}{\pgfqpoint{2.351344in}{1.521264in}}%
\pgfpathcurveto{\pgfqpoint{2.345520in}{1.527088in}}{\pgfqpoint{2.337620in}{1.530361in}}{\pgfqpoint{2.329384in}{1.530361in}}%
\pgfpathcurveto{\pgfqpoint{2.321148in}{1.530361in}}{\pgfqpoint{2.313248in}{1.527088in}}{\pgfqpoint{2.307424in}{1.521264in}}%
\pgfpathcurveto{\pgfqpoint{2.301600in}{1.515440in}}{\pgfqpoint{2.298328in}{1.507540in}}{\pgfqpoint{2.298328in}{1.499304in}}%
\pgfpathcurveto{\pgfqpoint{2.298328in}{1.491068in}}{\pgfqpoint{2.301600in}{1.483168in}}{\pgfqpoint{2.307424in}{1.477344in}}%
\pgfpathcurveto{\pgfqpoint{2.313248in}{1.471520in}}{\pgfqpoint{2.321148in}{1.468248in}}{\pgfqpoint{2.329384in}{1.468248in}}%
\pgfpathclose%
\pgfusepath{stroke,fill}%
\end{pgfscope}%
\begin{pgfscope}%
\pgfpathrectangle{\pgfqpoint{0.100000in}{0.212622in}}{\pgfqpoint{3.696000in}{3.696000in}}%
\pgfusepath{clip}%
\pgfsetbuttcap%
\pgfsetroundjoin%
\definecolor{currentfill}{rgb}{0.121569,0.466667,0.705882}%
\pgfsetfillcolor{currentfill}%
\pgfsetfillopacity{0.896241}%
\pgfsetlinewidth{1.003750pt}%
\definecolor{currentstroke}{rgb}{0.121569,0.466667,0.705882}%
\pgfsetstrokecolor{currentstroke}%
\pgfsetstrokeopacity{0.896241}%
\pgfsetdash{}{0pt}%
\pgfpathmoveto{\pgfqpoint{1.366645in}{2.030412in}}%
\pgfpathcurveto{\pgfqpoint{1.374881in}{2.030412in}}{\pgfqpoint{1.382781in}{2.033684in}}{\pgfqpoint{1.388605in}{2.039508in}}%
\pgfpathcurveto{\pgfqpoint{1.394429in}{2.045332in}}{\pgfqpoint{1.397701in}{2.053232in}}{\pgfqpoint{1.397701in}{2.061469in}}%
\pgfpathcurveto{\pgfqpoint{1.397701in}{2.069705in}}{\pgfqpoint{1.394429in}{2.077605in}}{\pgfqpoint{1.388605in}{2.083429in}}%
\pgfpathcurveto{\pgfqpoint{1.382781in}{2.089253in}}{\pgfqpoint{1.374881in}{2.092525in}}{\pgfqpoint{1.366645in}{2.092525in}}%
\pgfpathcurveto{\pgfqpoint{1.358408in}{2.092525in}}{\pgfqpoint{1.350508in}{2.089253in}}{\pgfqpoint{1.344684in}{2.083429in}}%
\pgfpathcurveto{\pgfqpoint{1.338861in}{2.077605in}}{\pgfqpoint{1.335588in}{2.069705in}}{\pgfqpoint{1.335588in}{2.061469in}}%
\pgfpathcurveto{\pgfqpoint{1.335588in}{2.053232in}}{\pgfqpoint{1.338861in}{2.045332in}}{\pgfqpoint{1.344684in}{2.039508in}}%
\pgfpathcurveto{\pgfqpoint{1.350508in}{2.033684in}}{\pgfqpoint{1.358408in}{2.030412in}}{\pgfqpoint{1.366645in}{2.030412in}}%
\pgfpathclose%
\pgfusepath{stroke,fill}%
\end{pgfscope}%
\begin{pgfscope}%
\pgfpathrectangle{\pgfqpoint{0.100000in}{0.212622in}}{\pgfqpoint{3.696000in}{3.696000in}}%
\pgfusepath{clip}%
\pgfsetbuttcap%
\pgfsetroundjoin%
\definecolor{currentfill}{rgb}{0.121569,0.466667,0.705882}%
\pgfsetfillcolor{currentfill}%
\pgfsetfillopacity{0.896931}%
\pgfsetlinewidth{1.003750pt}%
\definecolor{currentstroke}{rgb}{0.121569,0.466667,0.705882}%
\pgfsetstrokecolor{currentstroke}%
\pgfsetstrokeopacity{0.896931}%
\pgfsetdash{}{0pt}%
\pgfpathmoveto{\pgfqpoint{2.331348in}{1.466630in}}%
\pgfpathcurveto{\pgfqpoint{2.339584in}{1.466630in}}{\pgfqpoint{2.347484in}{1.469902in}}{\pgfqpoint{2.353308in}{1.475726in}}%
\pgfpathcurveto{\pgfqpoint{2.359132in}{1.481550in}}{\pgfqpoint{2.362404in}{1.489450in}}{\pgfqpoint{2.362404in}{1.497686in}}%
\pgfpathcurveto{\pgfqpoint{2.362404in}{1.505922in}}{\pgfqpoint{2.359132in}{1.513823in}}{\pgfqpoint{2.353308in}{1.519646in}}%
\pgfpathcurveto{\pgfqpoint{2.347484in}{1.525470in}}{\pgfqpoint{2.339584in}{1.528743in}}{\pgfqpoint{2.331348in}{1.528743in}}%
\pgfpathcurveto{\pgfqpoint{2.323112in}{1.528743in}}{\pgfqpoint{2.315211in}{1.525470in}}{\pgfqpoint{2.309388in}{1.519646in}}%
\pgfpathcurveto{\pgfqpoint{2.303564in}{1.513823in}}{\pgfqpoint{2.300291in}{1.505922in}}{\pgfqpoint{2.300291in}{1.497686in}}%
\pgfpathcurveto{\pgfqpoint{2.300291in}{1.489450in}}{\pgfqpoint{2.303564in}{1.481550in}}{\pgfqpoint{2.309388in}{1.475726in}}%
\pgfpathcurveto{\pgfqpoint{2.315211in}{1.469902in}}{\pgfqpoint{2.323112in}{1.466630in}}{\pgfqpoint{2.331348in}{1.466630in}}%
\pgfpathclose%
\pgfusepath{stroke,fill}%
\end{pgfscope}%
\begin{pgfscope}%
\pgfpathrectangle{\pgfqpoint{0.100000in}{0.212622in}}{\pgfqpoint{3.696000in}{3.696000in}}%
\pgfusepath{clip}%
\pgfsetbuttcap%
\pgfsetroundjoin%
\definecolor{currentfill}{rgb}{0.121569,0.466667,0.705882}%
\pgfsetfillcolor{currentfill}%
\pgfsetfillopacity{0.899219}%
\pgfsetlinewidth{1.003750pt}%
\definecolor{currentstroke}{rgb}{0.121569,0.466667,0.705882}%
\pgfsetstrokecolor{currentstroke}%
\pgfsetstrokeopacity{0.899219}%
\pgfsetdash{}{0pt}%
\pgfpathmoveto{\pgfqpoint{2.333613in}{1.463275in}}%
\pgfpathcurveto{\pgfqpoint{2.341850in}{1.463275in}}{\pgfqpoint{2.349750in}{1.466547in}}{\pgfqpoint{2.355574in}{1.472371in}}%
\pgfpathcurveto{\pgfqpoint{2.361397in}{1.478195in}}{\pgfqpoint{2.364670in}{1.486095in}}{\pgfqpoint{2.364670in}{1.494331in}}%
\pgfpathcurveto{\pgfqpoint{2.364670in}{1.502568in}}{\pgfqpoint{2.361397in}{1.510468in}}{\pgfqpoint{2.355574in}{1.516292in}}%
\pgfpathcurveto{\pgfqpoint{2.349750in}{1.522116in}}{\pgfqpoint{2.341850in}{1.525388in}}{\pgfqpoint{2.333613in}{1.525388in}}%
\pgfpathcurveto{\pgfqpoint{2.325377in}{1.525388in}}{\pgfqpoint{2.317477in}{1.522116in}}{\pgfqpoint{2.311653in}{1.516292in}}%
\pgfpathcurveto{\pgfqpoint{2.305829in}{1.510468in}}{\pgfqpoint{2.302557in}{1.502568in}}{\pgfqpoint{2.302557in}{1.494331in}}%
\pgfpathcurveto{\pgfqpoint{2.302557in}{1.486095in}}{\pgfqpoint{2.305829in}{1.478195in}}{\pgfqpoint{2.311653in}{1.472371in}}%
\pgfpathcurveto{\pgfqpoint{2.317477in}{1.466547in}}{\pgfqpoint{2.325377in}{1.463275in}}{\pgfqpoint{2.333613in}{1.463275in}}%
\pgfpathclose%
\pgfusepath{stroke,fill}%
\end{pgfscope}%
\begin{pgfscope}%
\pgfpathrectangle{\pgfqpoint{0.100000in}{0.212622in}}{\pgfqpoint{3.696000in}{3.696000in}}%
\pgfusepath{clip}%
\pgfsetbuttcap%
\pgfsetroundjoin%
\definecolor{currentfill}{rgb}{0.121569,0.466667,0.705882}%
\pgfsetfillcolor{currentfill}%
\pgfsetfillopacity{0.900856}%
\pgfsetlinewidth{1.003750pt}%
\definecolor{currentstroke}{rgb}{0.121569,0.466667,0.705882}%
\pgfsetstrokecolor{currentstroke}%
\pgfsetstrokeopacity{0.900856}%
\pgfsetdash{}{0pt}%
\pgfpathmoveto{\pgfqpoint{1.408997in}{2.007079in}}%
\pgfpathcurveto{\pgfqpoint{1.417234in}{2.007079in}}{\pgfqpoint{1.425134in}{2.010351in}}{\pgfqpoint{1.430958in}{2.016175in}}%
\pgfpathcurveto{\pgfqpoint{1.436781in}{2.021999in}}{\pgfqpoint{1.440054in}{2.029899in}}{\pgfqpoint{1.440054in}{2.038135in}}%
\pgfpathcurveto{\pgfqpoint{1.440054in}{2.046372in}}{\pgfqpoint{1.436781in}{2.054272in}}{\pgfqpoint{1.430958in}{2.060096in}}%
\pgfpathcurveto{\pgfqpoint{1.425134in}{2.065920in}}{\pgfqpoint{1.417234in}{2.069192in}}{\pgfqpoint{1.408997in}{2.069192in}}%
\pgfpathcurveto{\pgfqpoint{1.400761in}{2.069192in}}{\pgfqpoint{1.392861in}{2.065920in}}{\pgfqpoint{1.387037in}{2.060096in}}%
\pgfpathcurveto{\pgfqpoint{1.381213in}{2.054272in}}{\pgfqpoint{1.377941in}{2.046372in}}{\pgfqpoint{1.377941in}{2.038135in}}%
\pgfpathcurveto{\pgfqpoint{1.377941in}{2.029899in}}{\pgfqpoint{1.381213in}{2.021999in}}{\pgfqpoint{1.387037in}{2.016175in}}%
\pgfpathcurveto{\pgfqpoint{1.392861in}{2.010351in}}{\pgfqpoint{1.400761in}{2.007079in}}{\pgfqpoint{1.408997in}{2.007079in}}%
\pgfpathclose%
\pgfusepath{stroke,fill}%
\end{pgfscope}%
\begin{pgfscope}%
\pgfpathrectangle{\pgfqpoint{0.100000in}{0.212622in}}{\pgfqpoint{3.696000in}{3.696000in}}%
\pgfusepath{clip}%
\pgfsetbuttcap%
\pgfsetroundjoin%
\definecolor{currentfill}{rgb}{0.121569,0.466667,0.705882}%
\pgfsetfillcolor{currentfill}%
\pgfsetfillopacity{0.901602}%
\pgfsetlinewidth{1.003750pt}%
\definecolor{currentstroke}{rgb}{0.121569,0.466667,0.705882}%
\pgfsetstrokecolor{currentstroke}%
\pgfsetstrokeopacity{0.901602}%
\pgfsetdash{}{0pt}%
\pgfpathmoveto{\pgfqpoint{2.335638in}{1.458560in}}%
\pgfpathcurveto{\pgfqpoint{2.343874in}{1.458560in}}{\pgfqpoint{2.351774in}{1.461833in}}{\pgfqpoint{2.357598in}{1.467656in}}%
\pgfpathcurveto{\pgfqpoint{2.363422in}{1.473480in}}{\pgfqpoint{2.366694in}{1.481380in}}{\pgfqpoint{2.366694in}{1.489617in}}%
\pgfpathcurveto{\pgfqpoint{2.366694in}{1.497853in}}{\pgfqpoint{2.363422in}{1.505753in}}{\pgfqpoint{2.357598in}{1.511577in}}%
\pgfpathcurveto{\pgfqpoint{2.351774in}{1.517401in}}{\pgfqpoint{2.343874in}{1.520673in}}{\pgfqpoint{2.335638in}{1.520673in}}%
\pgfpathcurveto{\pgfqpoint{2.327401in}{1.520673in}}{\pgfqpoint{2.319501in}{1.517401in}}{\pgfqpoint{2.313677in}{1.511577in}}%
\pgfpathcurveto{\pgfqpoint{2.307853in}{1.505753in}}{\pgfqpoint{2.304581in}{1.497853in}}{\pgfqpoint{2.304581in}{1.489617in}}%
\pgfpathcurveto{\pgfqpoint{2.304581in}{1.481380in}}{\pgfqpoint{2.307853in}{1.473480in}}{\pgfqpoint{2.313677in}{1.467656in}}%
\pgfpathcurveto{\pgfqpoint{2.319501in}{1.461833in}}{\pgfqpoint{2.327401in}{1.458560in}}{\pgfqpoint{2.335638in}{1.458560in}}%
\pgfpathclose%
\pgfusepath{stroke,fill}%
\end{pgfscope}%
\begin{pgfscope}%
\pgfpathrectangle{\pgfqpoint{0.100000in}{0.212622in}}{\pgfqpoint{3.696000in}{3.696000in}}%
\pgfusepath{clip}%
\pgfsetbuttcap%
\pgfsetroundjoin%
\definecolor{currentfill}{rgb}{0.121569,0.466667,0.705882}%
\pgfsetfillcolor{currentfill}%
\pgfsetfillopacity{0.902995}%
\pgfsetlinewidth{1.003750pt}%
\definecolor{currentstroke}{rgb}{0.121569,0.466667,0.705882}%
\pgfsetstrokecolor{currentstroke}%
\pgfsetstrokeopacity{0.902995}%
\pgfsetdash{}{0pt}%
\pgfpathmoveto{\pgfqpoint{2.336697in}{1.456427in}}%
\pgfpathcurveto{\pgfqpoint{2.344933in}{1.456427in}}{\pgfqpoint{2.352833in}{1.459699in}}{\pgfqpoint{2.358657in}{1.465523in}}%
\pgfpathcurveto{\pgfqpoint{2.364481in}{1.471347in}}{\pgfqpoint{2.367753in}{1.479247in}}{\pgfqpoint{2.367753in}{1.487483in}}%
\pgfpathcurveto{\pgfqpoint{2.367753in}{1.495719in}}{\pgfqpoint{2.364481in}{1.503619in}}{\pgfqpoint{2.358657in}{1.509443in}}%
\pgfpathcurveto{\pgfqpoint{2.352833in}{1.515267in}}{\pgfqpoint{2.344933in}{1.518540in}}{\pgfqpoint{2.336697in}{1.518540in}}%
\pgfpathcurveto{\pgfqpoint{2.328461in}{1.518540in}}{\pgfqpoint{2.320560in}{1.515267in}}{\pgfqpoint{2.314737in}{1.509443in}}%
\pgfpathcurveto{\pgfqpoint{2.308913in}{1.503619in}}{\pgfqpoint{2.305640in}{1.495719in}}{\pgfqpoint{2.305640in}{1.487483in}}%
\pgfpathcurveto{\pgfqpoint{2.305640in}{1.479247in}}{\pgfqpoint{2.308913in}{1.471347in}}{\pgfqpoint{2.314737in}{1.465523in}}%
\pgfpathcurveto{\pgfqpoint{2.320560in}{1.459699in}}{\pgfqpoint{2.328461in}{1.456427in}}{\pgfqpoint{2.336697in}{1.456427in}}%
\pgfpathclose%
\pgfusepath{stroke,fill}%
\end{pgfscope}%
\begin{pgfscope}%
\pgfpathrectangle{\pgfqpoint{0.100000in}{0.212622in}}{\pgfqpoint{3.696000in}{3.696000in}}%
\pgfusepath{clip}%
\pgfsetbuttcap%
\pgfsetroundjoin%
\definecolor{currentfill}{rgb}{0.121569,0.466667,0.705882}%
\pgfsetfillcolor{currentfill}%
\pgfsetfillopacity{0.903568}%
\pgfsetlinewidth{1.003750pt}%
\definecolor{currentstroke}{rgb}{0.121569,0.466667,0.705882}%
\pgfsetstrokecolor{currentstroke}%
\pgfsetstrokeopacity{0.903568}%
\pgfsetdash{}{0pt}%
\pgfpathmoveto{\pgfqpoint{1.450582in}{1.971127in}}%
\pgfpathcurveto{\pgfqpoint{1.458819in}{1.971127in}}{\pgfqpoint{1.466719in}{1.974399in}}{\pgfqpoint{1.472543in}{1.980223in}}%
\pgfpathcurveto{\pgfqpoint{1.478367in}{1.986047in}}{\pgfqpoint{1.481639in}{1.993947in}}{\pgfqpoint{1.481639in}{2.002183in}}%
\pgfpathcurveto{\pgfqpoint{1.481639in}{2.010420in}}{\pgfqpoint{1.478367in}{2.018320in}}{\pgfqpoint{1.472543in}{2.024144in}}%
\pgfpathcurveto{\pgfqpoint{1.466719in}{2.029968in}}{\pgfqpoint{1.458819in}{2.033240in}}{\pgfqpoint{1.450582in}{2.033240in}}%
\pgfpathcurveto{\pgfqpoint{1.442346in}{2.033240in}}{\pgfqpoint{1.434446in}{2.029968in}}{\pgfqpoint{1.428622in}{2.024144in}}%
\pgfpathcurveto{\pgfqpoint{1.422798in}{2.018320in}}{\pgfqpoint{1.419526in}{2.010420in}}{\pgfqpoint{1.419526in}{2.002183in}}%
\pgfpathcurveto{\pgfqpoint{1.419526in}{1.993947in}}{\pgfqpoint{1.422798in}{1.986047in}}{\pgfqpoint{1.428622in}{1.980223in}}%
\pgfpathcurveto{\pgfqpoint{1.434446in}{1.974399in}}{\pgfqpoint{1.442346in}{1.971127in}}{\pgfqpoint{1.450582in}{1.971127in}}%
\pgfpathclose%
\pgfusepath{stroke,fill}%
\end{pgfscope}%
\begin{pgfscope}%
\pgfpathrectangle{\pgfqpoint{0.100000in}{0.212622in}}{\pgfqpoint{3.696000in}{3.696000in}}%
\pgfusepath{clip}%
\pgfsetbuttcap%
\pgfsetroundjoin%
\definecolor{currentfill}{rgb}{0.121569,0.466667,0.705882}%
\pgfsetfillcolor{currentfill}%
\pgfsetfillopacity{0.904803}%
\pgfsetlinewidth{1.003750pt}%
\definecolor{currentstroke}{rgb}{0.121569,0.466667,0.705882}%
\pgfsetstrokecolor{currentstroke}%
\pgfsetstrokeopacity{0.904803}%
\pgfsetdash{}{0pt}%
\pgfpathmoveto{\pgfqpoint{2.337999in}{1.455048in}}%
\pgfpathcurveto{\pgfqpoint{2.346235in}{1.455048in}}{\pgfqpoint{2.354135in}{1.458321in}}{\pgfqpoint{2.359959in}{1.464144in}}%
\pgfpathcurveto{\pgfqpoint{2.365783in}{1.469968in}}{\pgfqpoint{2.369056in}{1.477868in}}{\pgfqpoint{2.369056in}{1.486105in}}%
\pgfpathcurveto{\pgfqpoint{2.369056in}{1.494341in}}{\pgfqpoint{2.365783in}{1.502241in}}{\pgfqpoint{2.359959in}{1.508065in}}%
\pgfpathcurveto{\pgfqpoint{2.354135in}{1.513889in}}{\pgfqpoint{2.346235in}{1.517161in}}{\pgfqpoint{2.337999in}{1.517161in}}%
\pgfpathcurveto{\pgfqpoint{2.329763in}{1.517161in}}{\pgfqpoint{2.321863in}{1.513889in}}{\pgfqpoint{2.316039in}{1.508065in}}%
\pgfpathcurveto{\pgfqpoint{2.310215in}{1.502241in}}{\pgfqpoint{2.306943in}{1.494341in}}{\pgfqpoint{2.306943in}{1.486105in}}%
\pgfpathcurveto{\pgfqpoint{2.306943in}{1.477868in}}{\pgfqpoint{2.310215in}{1.469968in}}{\pgfqpoint{2.316039in}{1.464144in}}%
\pgfpathcurveto{\pgfqpoint{2.321863in}{1.458321in}}{\pgfqpoint{2.329763in}{1.455048in}}{\pgfqpoint{2.337999in}{1.455048in}}%
\pgfpathclose%
\pgfusepath{stroke,fill}%
\end{pgfscope}%
\begin{pgfscope}%
\pgfpathrectangle{\pgfqpoint{0.100000in}{0.212622in}}{\pgfqpoint{3.696000in}{3.696000in}}%
\pgfusepath{clip}%
\pgfsetbuttcap%
\pgfsetroundjoin%
\definecolor{currentfill}{rgb}{0.121569,0.466667,0.705882}%
\pgfsetfillcolor{currentfill}%
\pgfsetfillopacity{0.906786}%
\pgfsetlinewidth{1.003750pt}%
\definecolor{currentstroke}{rgb}{0.121569,0.466667,0.705882}%
\pgfsetstrokecolor{currentstroke}%
\pgfsetstrokeopacity{0.906786}%
\pgfsetdash{}{0pt}%
\pgfpathmoveto{\pgfqpoint{2.339673in}{1.453008in}}%
\pgfpathcurveto{\pgfqpoint{2.347909in}{1.453008in}}{\pgfqpoint{2.355809in}{1.456280in}}{\pgfqpoint{2.361633in}{1.462104in}}%
\pgfpathcurveto{\pgfqpoint{2.367457in}{1.467928in}}{\pgfqpoint{2.370729in}{1.475828in}}{\pgfqpoint{2.370729in}{1.484064in}}%
\pgfpathcurveto{\pgfqpoint{2.370729in}{1.492301in}}{\pgfqpoint{2.367457in}{1.500201in}}{\pgfqpoint{2.361633in}{1.506025in}}%
\pgfpathcurveto{\pgfqpoint{2.355809in}{1.511849in}}{\pgfqpoint{2.347909in}{1.515121in}}{\pgfqpoint{2.339673in}{1.515121in}}%
\pgfpathcurveto{\pgfqpoint{2.331436in}{1.515121in}}{\pgfqpoint{2.323536in}{1.511849in}}{\pgfqpoint{2.317712in}{1.506025in}}%
\pgfpathcurveto{\pgfqpoint{2.311888in}{1.500201in}}{\pgfqpoint{2.308616in}{1.492301in}}{\pgfqpoint{2.308616in}{1.484064in}}%
\pgfpathcurveto{\pgfqpoint{2.308616in}{1.475828in}}{\pgfqpoint{2.311888in}{1.467928in}}{\pgfqpoint{2.317712in}{1.462104in}}%
\pgfpathcurveto{\pgfqpoint{2.323536in}{1.456280in}}{\pgfqpoint{2.331436in}{1.453008in}}{\pgfqpoint{2.339673in}{1.453008in}}%
\pgfpathclose%
\pgfusepath{stroke,fill}%
\end{pgfscope}%
\begin{pgfscope}%
\pgfpathrectangle{\pgfqpoint{0.100000in}{0.212622in}}{\pgfqpoint{3.696000in}{3.696000in}}%
\pgfusepath{clip}%
\pgfsetbuttcap%
\pgfsetroundjoin%
\definecolor{currentfill}{rgb}{0.121569,0.466667,0.705882}%
\pgfsetfillcolor{currentfill}%
\pgfsetfillopacity{0.907232}%
\pgfsetlinewidth{1.003750pt}%
\definecolor{currentstroke}{rgb}{0.121569,0.466667,0.705882}%
\pgfsetstrokecolor{currentstroke}%
\pgfsetstrokeopacity{0.907232}%
\pgfsetdash{}{0pt}%
\pgfpathmoveto{\pgfqpoint{1.488898in}{1.947819in}}%
\pgfpathcurveto{\pgfqpoint{1.497135in}{1.947819in}}{\pgfqpoint{1.505035in}{1.951091in}}{\pgfqpoint{1.510859in}{1.956915in}}%
\pgfpathcurveto{\pgfqpoint{1.516682in}{1.962739in}}{\pgfqpoint{1.519955in}{1.970639in}}{\pgfqpoint{1.519955in}{1.978876in}}%
\pgfpathcurveto{\pgfqpoint{1.519955in}{1.987112in}}{\pgfqpoint{1.516682in}{1.995012in}}{\pgfqpoint{1.510859in}{2.000836in}}%
\pgfpathcurveto{\pgfqpoint{1.505035in}{2.006660in}}{\pgfqpoint{1.497135in}{2.009932in}}{\pgfqpoint{1.488898in}{2.009932in}}%
\pgfpathcurveto{\pgfqpoint{1.480662in}{2.009932in}}{\pgfqpoint{1.472762in}{2.006660in}}{\pgfqpoint{1.466938in}{2.000836in}}%
\pgfpathcurveto{\pgfqpoint{1.461114in}{1.995012in}}{\pgfqpoint{1.457842in}{1.987112in}}{\pgfqpoint{1.457842in}{1.978876in}}%
\pgfpathcurveto{\pgfqpoint{1.457842in}{1.970639in}}{\pgfqpoint{1.461114in}{1.962739in}}{\pgfqpoint{1.466938in}{1.956915in}}%
\pgfpathcurveto{\pgfqpoint{1.472762in}{1.951091in}}{\pgfqpoint{1.480662in}{1.947819in}}{\pgfqpoint{1.488898in}{1.947819in}}%
\pgfpathclose%
\pgfusepath{stroke,fill}%
\end{pgfscope}%
\begin{pgfscope}%
\pgfpathrectangle{\pgfqpoint{0.100000in}{0.212622in}}{\pgfqpoint{3.696000in}{3.696000in}}%
\pgfusepath{clip}%
\pgfsetbuttcap%
\pgfsetroundjoin%
\definecolor{currentfill}{rgb}{0.121569,0.466667,0.705882}%
\pgfsetfillcolor{currentfill}%
\pgfsetfillopacity{0.908793}%
\pgfsetlinewidth{1.003750pt}%
\definecolor{currentstroke}{rgb}{0.121569,0.466667,0.705882}%
\pgfsetstrokecolor{currentstroke}%
\pgfsetstrokeopacity{0.908793}%
\pgfsetdash{}{0pt}%
\pgfpathmoveto{\pgfqpoint{2.341883in}{1.449484in}}%
\pgfpathcurveto{\pgfqpoint{2.350119in}{1.449484in}}{\pgfqpoint{2.358019in}{1.452757in}}{\pgfqpoint{2.363843in}{1.458581in}}%
\pgfpathcurveto{\pgfqpoint{2.369667in}{1.464405in}}{\pgfqpoint{2.372939in}{1.472305in}}{\pgfqpoint{2.372939in}{1.480541in}}%
\pgfpathcurveto{\pgfqpoint{2.372939in}{1.488777in}}{\pgfqpoint{2.369667in}{1.496677in}}{\pgfqpoint{2.363843in}{1.502501in}}%
\pgfpathcurveto{\pgfqpoint{2.358019in}{1.508325in}}{\pgfqpoint{2.350119in}{1.511597in}}{\pgfqpoint{2.341883in}{1.511597in}}%
\pgfpathcurveto{\pgfqpoint{2.333646in}{1.511597in}}{\pgfqpoint{2.325746in}{1.508325in}}{\pgfqpoint{2.319922in}{1.502501in}}%
\pgfpathcurveto{\pgfqpoint{2.314098in}{1.496677in}}{\pgfqpoint{2.310826in}{1.488777in}}{\pgfqpoint{2.310826in}{1.480541in}}%
\pgfpathcurveto{\pgfqpoint{2.310826in}{1.472305in}}{\pgfqpoint{2.314098in}{1.464405in}}{\pgfqpoint{2.319922in}{1.458581in}}%
\pgfpathcurveto{\pgfqpoint{2.325746in}{1.452757in}}{\pgfqpoint{2.333646in}{1.449484in}}{\pgfqpoint{2.341883in}{1.449484in}}%
\pgfpathclose%
\pgfusepath{stroke,fill}%
\end{pgfscope}%
\begin{pgfscope}%
\pgfpathrectangle{\pgfqpoint{0.100000in}{0.212622in}}{\pgfqpoint{3.696000in}{3.696000in}}%
\pgfusepath{clip}%
\pgfsetbuttcap%
\pgfsetroundjoin%
\definecolor{currentfill}{rgb}{0.121569,0.466667,0.705882}%
\pgfsetfillcolor{currentfill}%
\pgfsetfillopacity{0.910767}%
\pgfsetlinewidth{1.003750pt}%
\definecolor{currentstroke}{rgb}{0.121569,0.466667,0.705882}%
\pgfsetstrokecolor{currentstroke}%
\pgfsetstrokeopacity{0.910767}%
\pgfsetdash{}{0pt}%
\pgfpathmoveto{\pgfqpoint{1.525720in}{1.922795in}}%
\pgfpathcurveto{\pgfqpoint{1.533956in}{1.922795in}}{\pgfqpoint{1.541856in}{1.926067in}}{\pgfqpoint{1.547680in}{1.931891in}}%
\pgfpathcurveto{\pgfqpoint{1.553504in}{1.937715in}}{\pgfqpoint{1.556777in}{1.945615in}}{\pgfqpoint{1.556777in}{1.953851in}}%
\pgfpathcurveto{\pgfqpoint{1.556777in}{1.962087in}}{\pgfqpoint{1.553504in}{1.969987in}}{\pgfqpoint{1.547680in}{1.975811in}}%
\pgfpathcurveto{\pgfqpoint{1.541856in}{1.981635in}}{\pgfqpoint{1.533956in}{1.984908in}}{\pgfqpoint{1.525720in}{1.984908in}}%
\pgfpathcurveto{\pgfqpoint{1.517484in}{1.984908in}}{\pgfqpoint{1.509584in}{1.981635in}}{\pgfqpoint{1.503760in}{1.975811in}}%
\pgfpathcurveto{\pgfqpoint{1.497936in}{1.969987in}}{\pgfqpoint{1.494664in}{1.962087in}}{\pgfqpoint{1.494664in}{1.953851in}}%
\pgfpathcurveto{\pgfqpoint{1.494664in}{1.945615in}}{\pgfqpoint{1.497936in}{1.937715in}}{\pgfqpoint{1.503760in}{1.931891in}}%
\pgfpathcurveto{\pgfqpoint{1.509584in}{1.926067in}}{\pgfqpoint{1.517484in}{1.922795in}}{\pgfqpoint{1.525720in}{1.922795in}}%
\pgfpathclose%
\pgfusepath{stroke,fill}%
\end{pgfscope}%
\begin{pgfscope}%
\pgfpathrectangle{\pgfqpoint{0.100000in}{0.212622in}}{\pgfqpoint{3.696000in}{3.696000in}}%
\pgfusepath{clip}%
\pgfsetbuttcap%
\pgfsetroundjoin%
\definecolor{currentfill}{rgb}{0.121569,0.466667,0.705882}%
\pgfsetfillcolor{currentfill}%
\pgfsetfillopacity{0.911001}%
\pgfsetlinewidth{1.003750pt}%
\definecolor{currentstroke}{rgb}{0.121569,0.466667,0.705882}%
\pgfsetstrokecolor{currentstroke}%
\pgfsetstrokeopacity{0.911001}%
\pgfsetdash{}{0pt}%
\pgfpathmoveto{\pgfqpoint{2.344316in}{1.442377in}}%
\pgfpathcurveto{\pgfqpoint{2.352553in}{1.442377in}}{\pgfqpoint{2.360453in}{1.445649in}}{\pgfqpoint{2.366277in}{1.451473in}}%
\pgfpathcurveto{\pgfqpoint{2.372101in}{1.457297in}}{\pgfqpoint{2.375373in}{1.465197in}}{\pgfqpoint{2.375373in}{1.473433in}}%
\pgfpathcurveto{\pgfqpoint{2.375373in}{1.481670in}}{\pgfqpoint{2.372101in}{1.489570in}}{\pgfqpoint{2.366277in}{1.495394in}}%
\pgfpathcurveto{\pgfqpoint{2.360453in}{1.501218in}}{\pgfqpoint{2.352553in}{1.504490in}}{\pgfqpoint{2.344316in}{1.504490in}}%
\pgfpathcurveto{\pgfqpoint{2.336080in}{1.504490in}}{\pgfqpoint{2.328180in}{1.501218in}}{\pgfqpoint{2.322356in}{1.495394in}}%
\pgfpathcurveto{\pgfqpoint{2.316532in}{1.489570in}}{\pgfqpoint{2.313260in}{1.481670in}}{\pgfqpoint{2.313260in}{1.473433in}}%
\pgfpathcurveto{\pgfqpoint{2.313260in}{1.465197in}}{\pgfqpoint{2.316532in}{1.457297in}}{\pgfqpoint{2.322356in}{1.451473in}}%
\pgfpathcurveto{\pgfqpoint{2.328180in}{1.445649in}}{\pgfqpoint{2.336080in}{1.442377in}}{\pgfqpoint{2.344316in}{1.442377in}}%
\pgfpathclose%
\pgfusepath{stroke,fill}%
\end{pgfscope}%
\begin{pgfscope}%
\pgfpathrectangle{\pgfqpoint{0.100000in}{0.212622in}}{\pgfqpoint{3.696000in}{3.696000in}}%
\pgfusepath{clip}%
\pgfsetbuttcap%
\pgfsetroundjoin%
\definecolor{currentfill}{rgb}{0.121569,0.466667,0.705882}%
\pgfsetfillcolor{currentfill}%
\pgfsetfillopacity{0.914031}%
\pgfsetlinewidth{1.003750pt}%
\definecolor{currentstroke}{rgb}{0.121569,0.466667,0.705882}%
\pgfsetstrokecolor{currentstroke}%
\pgfsetstrokeopacity{0.914031}%
\pgfsetdash{}{0pt}%
\pgfpathmoveto{\pgfqpoint{1.560248in}{1.898332in}}%
\pgfpathcurveto{\pgfqpoint{1.568484in}{1.898332in}}{\pgfqpoint{1.576384in}{1.901604in}}{\pgfqpoint{1.582208in}{1.907428in}}%
\pgfpathcurveto{\pgfqpoint{1.588032in}{1.913252in}}{\pgfqpoint{1.591304in}{1.921152in}}{\pgfqpoint{1.591304in}{1.929388in}}%
\pgfpathcurveto{\pgfqpoint{1.591304in}{1.937624in}}{\pgfqpoint{1.588032in}{1.945524in}}{\pgfqpoint{1.582208in}{1.951348in}}%
\pgfpathcurveto{\pgfqpoint{1.576384in}{1.957172in}}{\pgfqpoint{1.568484in}{1.960445in}}{\pgfqpoint{1.560248in}{1.960445in}}%
\pgfpathcurveto{\pgfqpoint{1.552012in}{1.960445in}}{\pgfqpoint{1.544111in}{1.957172in}}{\pgfqpoint{1.538288in}{1.951348in}}%
\pgfpathcurveto{\pgfqpoint{1.532464in}{1.945524in}}{\pgfqpoint{1.529191in}{1.937624in}}{\pgfqpoint{1.529191in}{1.929388in}}%
\pgfpathcurveto{\pgfqpoint{1.529191in}{1.921152in}}{\pgfqpoint{1.532464in}{1.913252in}}{\pgfqpoint{1.538288in}{1.907428in}}%
\pgfpathcurveto{\pgfqpoint{1.544111in}{1.901604in}}{\pgfqpoint{1.552012in}{1.898332in}}{\pgfqpoint{1.560248in}{1.898332in}}%
\pgfpathclose%
\pgfusepath{stroke,fill}%
\end{pgfscope}%
\begin{pgfscope}%
\pgfpathrectangle{\pgfqpoint{0.100000in}{0.212622in}}{\pgfqpoint{3.696000in}{3.696000in}}%
\pgfusepath{clip}%
\pgfsetbuttcap%
\pgfsetroundjoin%
\definecolor{currentfill}{rgb}{0.121569,0.466667,0.705882}%
\pgfsetfillcolor{currentfill}%
\pgfsetfillopacity{0.914575}%
\pgfsetlinewidth{1.003750pt}%
\definecolor{currentstroke}{rgb}{0.121569,0.466667,0.705882}%
\pgfsetstrokecolor{currentstroke}%
\pgfsetstrokeopacity{0.914575}%
\pgfsetdash{}{0pt}%
\pgfpathmoveto{\pgfqpoint{2.346754in}{1.438424in}}%
\pgfpathcurveto{\pgfqpoint{2.354990in}{1.438424in}}{\pgfqpoint{2.362890in}{1.441696in}}{\pgfqpoint{2.368714in}{1.447520in}}%
\pgfpathcurveto{\pgfqpoint{2.374538in}{1.453344in}}{\pgfqpoint{2.377810in}{1.461244in}}{\pgfqpoint{2.377810in}{1.469481in}}%
\pgfpathcurveto{\pgfqpoint{2.377810in}{1.477717in}}{\pgfqpoint{2.374538in}{1.485617in}}{\pgfqpoint{2.368714in}{1.491441in}}%
\pgfpathcurveto{\pgfqpoint{2.362890in}{1.497265in}}{\pgfqpoint{2.354990in}{1.500537in}}{\pgfqpoint{2.346754in}{1.500537in}}%
\pgfpathcurveto{\pgfqpoint{2.338517in}{1.500537in}}{\pgfqpoint{2.330617in}{1.497265in}}{\pgfqpoint{2.324793in}{1.491441in}}%
\pgfpathcurveto{\pgfqpoint{2.318969in}{1.485617in}}{\pgfqpoint{2.315697in}{1.477717in}}{\pgfqpoint{2.315697in}{1.469481in}}%
\pgfpathcurveto{\pgfqpoint{2.315697in}{1.461244in}}{\pgfqpoint{2.318969in}{1.453344in}}{\pgfqpoint{2.324793in}{1.447520in}}%
\pgfpathcurveto{\pgfqpoint{2.330617in}{1.441696in}}{\pgfqpoint{2.338517in}{1.438424in}}{\pgfqpoint{2.346754in}{1.438424in}}%
\pgfpathclose%
\pgfusepath{stroke,fill}%
\end{pgfscope}%
\begin{pgfscope}%
\pgfpathrectangle{\pgfqpoint{0.100000in}{0.212622in}}{\pgfqpoint{3.696000in}{3.696000in}}%
\pgfusepath{clip}%
\pgfsetbuttcap%
\pgfsetroundjoin%
\definecolor{currentfill}{rgb}{0.121569,0.466667,0.705882}%
\pgfsetfillcolor{currentfill}%
\pgfsetfillopacity{0.916324}%
\pgfsetlinewidth{1.003750pt}%
\definecolor{currentstroke}{rgb}{0.121569,0.466667,0.705882}%
\pgfsetstrokecolor{currentstroke}%
\pgfsetstrokeopacity{0.916324}%
\pgfsetdash{}{0pt}%
\pgfpathmoveto{\pgfqpoint{1.592585in}{1.871522in}}%
\pgfpathcurveto{\pgfqpoint{1.600821in}{1.871522in}}{\pgfqpoint{1.608721in}{1.874794in}}{\pgfqpoint{1.614545in}{1.880618in}}%
\pgfpathcurveto{\pgfqpoint{1.620369in}{1.886442in}}{\pgfqpoint{1.623641in}{1.894342in}}{\pgfqpoint{1.623641in}{1.902579in}}%
\pgfpathcurveto{\pgfqpoint{1.623641in}{1.910815in}}{\pgfqpoint{1.620369in}{1.918715in}}{\pgfqpoint{1.614545in}{1.924539in}}%
\pgfpathcurveto{\pgfqpoint{1.608721in}{1.930363in}}{\pgfqpoint{1.600821in}{1.933635in}}{\pgfqpoint{1.592585in}{1.933635in}}%
\pgfpathcurveto{\pgfqpoint{1.584348in}{1.933635in}}{\pgfqpoint{1.576448in}{1.930363in}}{\pgfqpoint{1.570624in}{1.924539in}}%
\pgfpathcurveto{\pgfqpoint{1.564800in}{1.918715in}}{\pgfqpoint{1.561528in}{1.910815in}}{\pgfqpoint{1.561528in}{1.902579in}}%
\pgfpathcurveto{\pgfqpoint{1.561528in}{1.894342in}}{\pgfqpoint{1.564800in}{1.886442in}}{\pgfqpoint{1.570624in}{1.880618in}}%
\pgfpathcurveto{\pgfqpoint{1.576448in}{1.874794in}}{\pgfqpoint{1.584348in}{1.871522in}}{\pgfqpoint{1.592585in}{1.871522in}}%
\pgfpathclose%
\pgfusepath{stroke,fill}%
\end{pgfscope}%
\begin{pgfscope}%
\pgfpathrectangle{\pgfqpoint{0.100000in}{0.212622in}}{\pgfqpoint{3.696000in}{3.696000in}}%
\pgfusepath{clip}%
\pgfsetbuttcap%
\pgfsetroundjoin%
\definecolor{currentfill}{rgb}{0.121569,0.466667,0.705882}%
\pgfsetfillcolor{currentfill}%
\pgfsetfillopacity{0.918666}%
\pgfsetlinewidth{1.003750pt}%
\definecolor{currentstroke}{rgb}{0.121569,0.466667,0.705882}%
\pgfsetstrokecolor{currentstroke}%
\pgfsetstrokeopacity{0.918666}%
\pgfsetdash{}{0pt}%
\pgfpathmoveto{\pgfqpoint{2.350004in}{1.434740in}}%
\pgfpathcurveto{\pgfqpoint{2.358240in}{1.434740in}}{\pgfqpoint{2.366140in}{1.438012in}}{\pgfqpoint{2.371964in}{1.443836in}}%
\pgfpathcurveto{\pgfqpoint{2.377788in}{1.449660in}}{\pgfqpoint{2.381060in}{1.457560in}}{\pgfqpoint{2.381060in}{1.465797in}}%
\pgfpathcurveto{\pgfqpoint{2.381060in}{1.474033in}}{\pgfqpoint{2.377788in}{1.481933in}}{\pgfqpoint{2.371964in}{1.487757in}}%
\pgfpathcurveto{\pgfqpoint{2.366140in}{1.493581in}}{\pgfqpoint{2.358240in}{1.496853in}}{\pgfqpoint{2.350004in}{1.496853in}}%
\pgfpathcurveto{\pgfqpoint{2.341767in}{1.496853in}}{\pgfqpoint{2.333867in}{1.493581in}}{\pgfqpoint{2.328043in}{1.487757in}}%
\pgfpathcurveto{\pgfqpoint{2.322220in}{1.481933in}}{\pgfqpoint{2.318947in}{1.474033in}}{\pgfqpoint{2.318947in}{1.465797in}}%
\pgfpathcurveto{\pgfqpoint{2.318947in}{1.457560in}}{\pgfqpoint{2.322220in}{1.449660in}}{\pgfqpoint{2.328043in}{1.443836in}}%
\pgfpathcurveto{\pgfqpoint{2.333867in}{1.438012in}}{\pgfqpoint{2.341767in}{1.434740in}}{\pgfqpoint{2.350004in}{1.434740in}}%
\pgfpathclose%
\pgfusepath{stroke,fill}%
\end{pgfscope}%
\begin{pgfscope}%
\pgfpathrectangle{\pgfqpoint{0.100000in}{0.212622in}}{\pgfqpoint{3.696000in}{3.696000in}}%
\pgfusepath{clip}%
\pgfsetbuttcap%
\pgfsetroundjoin%
\definecolor{currentfill}{rgb}{0.121569,0.466667,0.705882}%
\pgfsetfillcolor{currentfill}%
\pgfsetfillopacity{0.920516}%
\pgfsetlinewidth{1.003750pt}%
\definecolor{currentstroke}{rgb}{0.121569,0.466667,0.705882}%
\pgfsetstrokecolor{currentstroke}%
\pgfsetstrokeopacity{0.920516}%
\pgfsetdash{}{0pt}%
\pgfpathmoveto{\pgfqpoint{1.621565in}{1.854677in}}%
\pgfpathcurveto{\pgfqpoint{1.629802in}{1.854677in}}{\pgfqpoint{1.637702in}{1.857950in}}{\pgfqpoint{1.643526in}{1.863773in}}%
\pgfpathcurveto{\pgfqpoint{1.649350in}{1.869597in}}{\pgfqpoint{1.652622in}{1.877497in}}{\pgfqpoint{1.652622in}{1.885734in}}%
\pgfpathcurveto{\pgfqpoint{1.652622in}{1.893970in}}{\pgfqpoint{1.649350in}{1.901870in}}{\pgfqpoint{1.643526in}{1.907694in}}%
\pgfpathcurveto{\pgfqpoint{1.637702in}{1.913518in}}{\pgfqpoint{1.629802in}{1.916790in}}{\pgfqpoint{1.621565in}{1.916790in}}%
\pgfpathcurveto{\pgfqpoint{1.613329in}{1.916790in}}{\pgfqpoint{1.605429in}{1.913518in}}{\pgfqpoint{1.599605in}{1.907694in}}%
\pgfpathcurveto{\pgfqpoint{1.593781in}{1.901870in}}{\pgfqpoint{1.590509in}{1.893970in}}{\pgfqpoint{1.590509in}{1.885734in}}%
\pgfpathcurveto{\pgfqpoint{1.590509in}{1.877497in}}{\pgfqpoint{1.593781in}{1.869597in}}{\pgfqpoint{1.599605in}{1.863773in}}%
\pgfpathcurveto{\pgfqpoint{1.605429in}{1.857950in}}{\pgfqpoint{1.613329in}{1.854677in}}{\pgfqpoint{1.621565in}{1.854677in}}%
\pgfpathclose%
\pgfusepath{stroke,fill}%
\end{pgfscope}%
\begin{pgfscope}%
\pgfpathrectangle{\pgfqpoint{0.100000in}{0.212622in}}{\pgfqpoint{3.696000in}{3.696000in}}%
\pgfusepath{clip}%
\pgfsetbuttcap%
\pgfsetroundjoin%
\definecolor{currentfill}{rgb}{0.121569,0.466667,0.705882}%
\pgfsetfillcolor{currentfill}%
\pgfsetfillopacity{0.922784}%
\pgfsetlinewidth{1.003750pt}%
\definecolor{currentstroke}{rgb}{0.121569,0.466667,0.705882}%
\pgfsetstrokecolor{currentstroke}%
\pgfsetstrokeopacity{0.922784}%
\pgfsetdash{}{0pt}%
\pgfpathmoveto{\pgfqpoint{1.646815in}{1.829981in}}%
\pgfpathcurveto{\pgfqpoint{1.655051in}{1.829981in}}{\pgfqpoint{1.662951in}{1.833253in}}{\pgfqpoint{1.668775in}{1.839077in}}%
\pgfpathcurveto{\pgfqpoint{1.674599in}{1.844901in}}{\pgfqpoint{1.677871in}{1.852801in}}{\pgfqpoint{1.677871in}{1.861037in}}%
\pgfpathcurveto{\pgfqpoint{1.677871in}{1.869273in}}{\pgfqpoint{1.674599in}{1.877173in}}{\pgfqpoint{1.668775in}{1.882997in}}%
\pgfpathcurveto{\pgfqpoint{1.662951in}{1.888821in}}{\pgfqpoint{1.655051in}{1.892094in}}{\pgfqpoint{1.646815in}{1.892094in}}%
\pgfpathcurveto{\pgfqpoint{1.638579in}{1.892094in}}{\pgfqpoint{1.630679in}{1.888821in}}{\pgfqpoint{1.624855in}{1.882997in}}%
\pgfpathcurveto{\pgfqpoint{1.619031in}{1.877173in}}{\pgfqpoint{1.615758in}{1.869273in}}{\pgfqpoint{1.615758in}{1.861037in}}%
\pgfpathcurveto{\pgfqpoint{1.615758in}{1.852801in}}{\pgfqpoint{1.619031in}{1.844901in}}{\pgfqpoint{1.624855in}{1.839077in}}%
\pgfpathcurveto{\pgfqpoint{1.630679in}{1.833253in}}{\pgfqpoint{1.638579in}{1.829981in}}{\pgfqpoint{1.646815in}{1.829981in}}%
\pgfpathclose%
\pgfusepath{stroke,fill}%
\end{pgfscope}%
\begin{pgfscope}%
\pgfpathrectangle{\pgfqpoint{0.100000in}{0.212622in}}{\pgfqpoint{3.696000in}{3.696000in}}%
\pgfusepath{clip}%
\pgfsetbuttcap%
\pgfsetroundjoin%
\definecolor{currentfill}{rgb}{0.121569,0.466667,0.705882}%
\pgfsetfillcolor{currentfill}%
\pgfsetfillopacity{0.923299}%
\pgfsetlinewidth{1.003750pt}%
\definecolor{currentstroke}{rgb}{0.121569,0.466667,0.705882}%
\pgfsetstrokecolor{currentstroke}%
\pgfsetstrokeopacity{0.923299}%
\pgfsetdash{}{0pt}%
\pgfpathmoveto{\pgfqpoint{2.353605in}{1.432088in}}%
\pgfpathcurveto{\pgfqpoint{2.361841in}{1.432088in}}{\pgfqpoint{2.369741in}{1.435360in}}{\pgfqpoint{2.375565in}{1.441184in}}%
\pgfpathcurveto{\pgfqpoint{2.381389in}{1.447008in}}{\pgfqpoint{2.384661in}{1.454908in}}{\pgfqpoint{2.384661in}{1.463145in}}%
\pgfpathcurveto{\pgfqpoint{2.384661in}{1.471381in}}{\pgfqpoint{2.381389in}{1.479281in}}{\pgfqpoint{2.375565in}{1.485105in}}%
\pgfpathcurveto{\pgfqpoint{2.369741in}{1.490929in}}{\pgfqpoint{2.361841in}{1.494201in}}{\pgfqpoint{2.353605in}{1.494201in}}%
\pgfpathcurveto{\pgfqpoint{2.345369in}{1.494201in}}{\pgfqpoint{2.337469in}{1.490929in}}{\pgfqpoint{2.331645in}{1.485105in}}%
\pgfpathcurveto{\pgfqpoint{2.325821in}{1.479281in}}{\pgfqpoint{2.322548in}{1.471381in}}{\pgfqpoint{2.322548in}{1.463145in}}%
\pgfpathcurveto{\pgfqpoint{2.322548in}{1.454908in}}{\pgfqpoint{2.325821in}{1.447008in}}{\pgfqpoint{2.331645in}{1.441184in}}%
\pgfpathcurveto{\pgfqpoint{2.337469in}{1.435360in}}{\pgfqpoint{2.345369in}{1.432088in}}{\pgfqpoint{2.353605in}{1.432088in}}%
\pgfpathclose%
\pgfusepath{stroke,fill}%
\end{pgfscope}%
\begin{pgfscope}%
\pgfpathrectangle{\pgfqpoint{0.100000in}{0.212622in}}{\pgfqpoint{3.696000in}{3.696000in}}%
\pgfusepath{clip}%
\pgfsetbuttcap%
\pgfsetroundjoin%
\definecolor{currentfill}{rgb}{0.121569,0.466667,0.705882}%
\pgfsetfillcolor{currentfill}%
\pgfsetfillopacity{0.925450}%
\pgfsetlinewidth{1.003750pt}%
\definecolor{currentstroke}{rgb}{0.121569,0.466667,0.705882}%
\pgfsetstrokecolor{currentstroke}%
\pgfsetstrokeopacity{0.925450}%
\pgfsetdash{}{0pt}%
\pgfpathmoveto{\pgfqpoint{1.671104in}{1.809160in}}%
\pgfpathcurveto{\pgfqpoint{1.679340in}{1.809160in}}{\pgfqpoint{1.687240in}{1.812432in}}{\pgfqpoint{1.693064in}{1.818256in}}%
\pgfpathcurveto{\pgfqpoint{1.698888in}{1.824080in}}{\pgfqpoint{1.702161in}{1.831980in}}{\pgfqpoint{1.702161in}{1.840216in}}%
\pgfpathcurveto{\pgfqpoint{1.702161in}{1.848453in}}{\pgfqpoint{1.698888in}{1.856353in}}{\pgfqpoint{1.693064in}{1.862177in}}%
\pgfpathcurveto{\pgfqpoint{1.687240in}{1.868001in}}{\pgfqpoint{1.679340in}{1.871273in}}{\pgfqpoint{1.671104in}{1.871273in}}%
\pgfpathcurveto{\pgfqpoint{1.662868in}{1.871273in}}{\pgfqpoint{1.654968in}{1.868001in}}{\pgfqpoint{1.649144in}{1.862177in}}%
\pgfpathcurveto{\pgfqpoint{1.643320in}{1.856353in}}{\pgfqpoint{1.640048in}{1.848453in}}{\pgfqpoint{1.640048in}{1.840216in}}%
\pgfpathcurveto{\pgfqpoint{1.640048in}{1.831980in}}{\pgfqpoint{1.643320in}{1.824080in}}{\pgfqpoint{1.649144in}{1.818256in}}%
\pgfpathcurveto{\pgfqpoint{1.654968in}{1.812432in}}{\pgfqpoint{1.662868in}{1.809160in}}{\pgfqpoint{1.671104in}{1.809160in}}%
\pgfpathclose%
\pgfusepath{stroke,fill}%
\end{pgfscope}%
\begin{pgfscope}%
\pgfpathrectangle{\pgfqpoint{0.100000in}{0.212622in}}{\pgfqpoint{3.696000in}{3.696000in}}%
\pgfusepath{clip}%
\pgfsetbuttcap%
\pgfsetroundjoin%
\definecolor{currentfill}{rgb}{0.121569,0.466667,0.705882}%
\pgfsetfillcolor{currentfill}%
\pgfsetfillopacity{0.927660}%
\pgfsetlinewidth{1.003750pt}%
\definecolor{currentstroke}{rgb}{0.121569,0.466667,0.705882}%
\pgfsetstrokecolor{currentstroke}%
\pgfsetstrokeopacity{0.927660}%
\pgfsetdash{}{0pt}%
\pgfpathmoveto{\pgfqpoint{2.357267in}{1.426372in}}%
\pgfpathcurveto{\pgfqpoint{2.365503in}{1.426372in}}{\pgfqpoint{2.373403in}{1.429645in}}{\pgfqpoint{2.379227in}{1.435469in}}%
\pgfpathcurveto{\pgfqpoint{2.385051in}{1.441292in}}{\pgfqpoint{2.388323in}{1.449193in}}{\pgfqpoint{2.388323in}{1.457429in}}%
\pgfpathcurveto{\pgfqpoint{2.388323in}{1.465665in}}{\pgfqpoint{2.385051in}{1.473565in}}{\pgfqpoint{2.379227in}{1.479389in}}%
\pgfpathcurveto{\pgfqpoint{2.373403in}{1.485213in}}{\pgfqpoint{2.365503in}{1.488485in}}{\pgfqpoint{2.357267in}{1.488485in}}%
\pgfpathcurveto{\pgfqpoint{2.349030in}{1.488485in}}{\pgfqpoint{2.341130in}{1.485213in}}{\pgfqpoint{2.335306in}{1.479389in}}%
\pgfpathcurveto{\pgfqpoint{2.329483in}{1.473565in}}{\pgfqpoint{2.326210in}{1.465665in}}{\pgfqpoint{2.326210in}{1.457429in}}%
\pgfpathcurveto{\pgfqpoint{2.326210in}{1.449193in}}{\pgfqpoint{2.329483in}{1.441292in}}{\pgfqpoint{2.335306in}{1.435469in}}%
\pgfpathcurveto{\pgfqpoint{2.341130in}{1.429645in}}{\pgfqpoint{2.349030in}{1.426372in}}{\pgfqpoint{2.357267in}{1.426372in}}%
\pgfpathclose%
\pgfusepath{stroke,fill}%
\end{pgfscope}%
\begin{pgfscope}%
\pgfpathrectangle{\pgfqpoint{0.100000in}{0.212622in}}{\pgfqpoint{3.696000in}{3.696000in}}%
\pgfusepath{clip}%
\pgfsetbuttcap%
\pgfsetroundjoin%
\definecolor{currentfill}{rgb}{0.121569,0.466667,0.705882}%
\pgfsetfillcolor{currentfill}%
\pgfsetfillopacity{0.928082}%
\pgfsetlinewidth{1.003750pt}%
\definecolor{currentstroke}{rgb}{0.121569,0.466667,0.705882}%
\pgfsetstrokecolor{currentstroke}%
\pgfsetstrokeopacity{0.928082}%
\pgfsetdash{}{0pt}%
\pgfpathmoveto{\pgfqpoint{1.692748in}{1.794553in}}%
\pgfpathcurveto{\pgfqpoint{1.700984in}{1.794553in}}{\pgfqpoint{1.708884in}{1.797825in}}{\pgfqpoint{1.714708in}{1.803649in}}%
\pgfpathcurveto{\pgfqpoint{1.720532in}{1.809473in}}{\pgfqpoint{1.723805in}{1.817373in}}{\pgfqpoint{1.723805in}{1.825609in}}%
\pgfpathcurveto{\pgfqpoint{1.723805in}{1.833845in}}{\pgfqpoint{1.720532in}{1.841745in}}{\pgfqpoint{1.714708in}{1.847569in}}%
\pgfpathcurveto{\pgfqpoint{1.708884in}{1.853393in}}{\pgfqpoint{1.700984in}{1.856666in}}{\pgfqpoint{1.692748in}{1.856666in}}%
\pgfpathcurveto{\pgfqpoint{1.684512in}{1.856666in}}{\pgfqpoint{1.676612in}{1.853393in}}{\pgfqpoint{1.670788in}{1.847569in}}%
\pgfpathcurveto{\pgfqpoint{1.664964in}{1.841745in}}{\pgfqpoint{1.661692in}{1.833845in}}{\pgfqpoint{1.661692in}{1.825609in}}%
\pgfpathcurveto{\pgfqpoint{1.661692in}{1.817373in}}{\pgfqpoint{1.664964in}{1.809473in}}{\pgfqpoint{1.670788in}{1.803649in}}%
\pgfpathcurveto{\pgfqpoint{1.676612in}{1.797825in}}{\pgfqpoint{1.684512in}{1.794553in}}{\pgfqpoint{1.692748in}{1.794553in}}%
\pgfpathclose%
\pgfusepath{stroke,fill}%
\end{pgfscope}%
\begin{pgfscope}%
\pgfpathrectangle{\pgfqpoint{0.100000in}{0.212622in}}{\pgfqpoint{3.696000in}{3.696000in}}%
\pgfusepath{clip}%
\pgfsetbuttcap%
\pgfsetroundjoin%
\definecolor{currentfill}{rgb}{0.121569,0.466667,0.705882}%
\pgfsetfillcolor{currentfill}%
\pgfsetfillopacity{0.931596}%
\pgfsetlinewidth{1.003750pt}%
\definecolor{currentstroke}{rgb}{0.121569,0.466667,0.705882}%
\pgfsetstrokecolor{currentstroke}%
\pgfsetstrokeopacity{0.931596}%
\pgfsetdash{}{0pt}%
\pgfpathmoveto{\pgfqpoint{2.360058in}{1.414951in}}%
\pgfpathcurveto{\pgfqpoint{2.368294in}{1.414951in}}{\pgfqpoint{2.376194in}{1.418223in}}{\pgfqpoint{2.382018in}{1.424047in}}%
\pgfpathcurveto{\pgfqpoint{2.387842in}{1.429871in}}{\pgfqpoint{2.391114in}{1.437771in}}{\pgfqpoint{2.391114in}{1.446007in}}%
\pgfpathcurveto{\pgfqpoint{2.391114in}{1.454243in}}{\pgfqpoint{2.387842in}{1.462143in}}{\pgfqpoint{2.382018in}{1.467967in}}%
\pgfpathcurveto{\pgfqpoint{2.376194in}{1.473791in}}{\pgfqpoint{2.368294in}{1.477064in}}{\pgfqpoint{2.360058in}{1.477064in}}%
\pgfpathcurveto{\pgfqpoint{2.351822in}{1.477064in}}{\pgfqpoint{2.343922in}{1.473791in}}{\pgfqpoint{2.338098in}{1.467967in}}%
\pgfpathcurveto{\pgfqpoint{2.332274in}{1.462143in}}{\pgfqpoint{2.329001in}{1.454243in}}{\pgfqpoint{2.329001in}{1.446007in}}%
\pgfpathcurveto{\pgfqpoint{2.329001in}{1.437771in}}{\pgfqpoint{2.332274in}{1.429871in}}{\pgfqpoint{2.338098in}{1.424047in}}%
\pgfpathcurveto{\pgfqpoint{2.343922in}{1.418223in}}{\pgfqpoint{2.351822in}{1.414951in}}{\pgfqpoint{2.360058in}{1.414951in}}%
\pgfpathclose%
\pgfusepath{stroke,fill}%
\end{pgfscope}%
\begin{pgfscope}%
\pgfpathrectangle{\pgfqpoint{0.100000in}{0.212622in}}{\pgfqpoint{3.696000in}{3.696000in}}%
\pgfusepath{clip}%
\pgfsetbuttcap%
\pgfsetroundjoin%
\definecolor{currentfill}{rgb}{0.121569,0.466667,0.705882}%
\pgfsetfillcolor{currentfill}%
\pgfsetfillopacity{0.932124}%
\pgfsetlinewidth{1.003750pt}%
\definecolor{currentstroke}{rgb}{0.121569,0.466667,0.705882}%
\pgfsetstrokecolor{currentstroke}%
\pgfsetstrokeopacity{0.932124}%
\pgfsetdash{}{0pt}%
\pgfpathmoveto{\pgfqpoint{1.734200in}{1.769675in}}%
\pgfpathcurveto{\pgfqpoint{1.742437in}{1.769675in}}{\pgfqpoint{1.750337in}{1.772947in}}{\pgfqpoint{1.756161in}{1.778771in}}%
\pgfpathcurveto{\pgfqpoint{1.761985in}{1.784595in}}{\pgfqpoint{1.765257in}{1.792495in}}{\pgfqpoint{1.765257in}{1.800731in}}%
\pgfpathcurveto{\pgfqpoint{1.765257in}{1.808967in}}{\pgfqpoint{1.761985in}{1.816868in}}{\pgfqpoint{1.756161in}{1.822691in}}%
\pgfpathcurveto{\pgfqpoint{1.750337in}{1.828515in}}{\pgfqpoint{1.742437in}{1.831788in}}{\pgfqpoint{1.734200in}{1.831788in}}%
\pgfpathcurveto{\pgfqpoint{1.725964in}{1.831788in}}{\pgfqpoint{1.718064in}{1.828515in}}{\pgfqpoint{1.712240in}{1.822691in}}%
\pgfpathcurveto{\pgfqpoint{1.706416in}{1.816868in}}{\pgfqpoint{1.703144in}{1.808967in}}{\pgfqpoint{1.703144in}{1.800731in}}%
\pgfpathcurveto{\pgfqpoint{1.703144in}{1.792495in}}{\pgfqpoint{1.706416in}{1.784595in}}{\pgfqpoint{1.712240in}{1.778771in}}%
\pgfpathcurveto{\pgfqpoint{1.718064in}{1.772947in}}{\pgfqpoint{1.725964in}{1.769675in}}{\pgfqpoint{1.734200in}{1.769675in}}%
\pgfpathclose%
\pgfusepath{stroke,fill}%
\end{pgfscope}%
\begin{pgfscope}%
\pgfpathrectangle{\pgfqpoint{0.100000in}{0.212622in}}{\pgfqpoint{3.696000in}{3.696000in}}%
\pgfusepath{clip}%
\pgfsetbuttcap%
\pgfsetroundjoin%
\definecolor{currentfill}{rgb}{0.121569,0.466667,0.705882}%
\pgfsetfillcolor{currentfill}%
\pgfsetfillopacity{0.936411}%
\pgfsetlinewidth{1.003750pt}%
\definecolor{currentstroke}{rgb}{0.121569,0.466667,0.705882}%
\pgfsetstrokecolor{currentstroke}%
\pgfsetstrokeopacity{0.936411}%
\pgfsetdash{}{0pt}%
\pgfpathmoveto{\pgfqpoint{2.364347in}{1.407210in}}%
\pgfpathcurveto{\pgfqpoint{2.372583in}{1.407210in}}{\pgfqpoint{2.380483in}{1.410482in}}{\pgfqpoint{2.386307in}{1.416306in}}%
\pgfpathcurveto{\pgfqpoint{2.392131in}{1.422130in}}{\pgfqpoint{2.395403in}{1.430030in}}{\pgfqpoint{2.395403in}{1.438266in}}%
\pgfpathcurveto{\pgfqpoint{2.395403in}{1.446502in}}{\pgfqpoint{2.392131in}{1.454402in}}{\pgfqpoint{2.386307in}{1.460226in}}%
\pgfpathcurveto{\pgfqpoint{2.380483in}{1.466050in}}{\pgfqpoint{2.372583in}{1.469323in}}{\pgfqpoint{2.364347in}{1.469323in}}%
\pgfpathcurveto{\pgfqpoint{2.356110in}{1.469323in}}{\pgfqpoint{2.348210in}{1.466050in}}{\pgfqpoint{2.342386in}{1.460226in}}%
\pgfpathcurveto{\pgfqpoint{2.336563in}{1.454402in}}{\pgfqpoint{2.333290in}{1.446502in}}{\pgfqpoint{2.333290in}{1.438266in}}%
\pgfpathcurveto{\pgfqpoint{2.333290in}{1.430030in}}{\pgfqpoint{2.336563in}{1.422130in}}{\pgfqpoint{2.342386in}{1.416306in}}%
\pgfpathcurveto{\pgfqpoint{2.348210in}{1.410482in}}{\pgfqpoint{2.356110in}{1.407210in}}{\pgfqpoint{2.364347in}{1.407210in}}%
\pgfpathclose%
\pgfusepath{stroke,fill}%
\end{pgfscope}%
\begin{pgfscope}%
\pgfpathrectangle{\pgfqpoint{0.100000in}{0.212622in}}{\pgfqpoint{3.696000in}{3.696000in}}%
\pgfusepath{clip}%
\pgfsetbuttcap%
\pgfsetroundjoin%
\definecolor{currentfill}{rgb}{0.121569,0.466667,0.705882}%
\pgfsetfillcolor{currentfill}%
\pgfsetfillopacity{0.936437}%
\pgfsetlinewidth{1.003750pt}%
\definecolor{currentstroke}{rgb}{0.121569,0.466667,0.705882}%
\pgfsetstrokecolor{currentstroke}%
\pgfsetstrokeopacity{0.936437}%
\pgfsetdash{}{0pt}%
\pgfpathmoveto{\pgfqpoint{1.771609in}{1.747384in}}%
\pgfpathcurveto{\pgfqpoint{1.779845in}{1.747384in}}{\pgfqpoint{1.787745in}{1.750656in}}{\pgfqpoint{1.793569in}{1.756480in}}%
\pgfpathcurveto{\pgfqpoint{1.799393in}{1.762304in}}{\pgfqpoint{1.802665in}{1.770204in}}{\pgfqpoint{1.802665in}{1.778441in}}%
\pgfpathcurveto{\pgfqpoint{1.802665in}{1.786677in}}{\pgfqpoint{1.799393in}{1.794577in}}{\pgfqpoint{1.793569in}{1.800401in}}%
\pgfpathcurveto{\pgfqpoint{1.787745in}{1.806225in}}{\pgfqpoint{1.779845in}{1.809497in}}{\pgfqpoint{1.771609in}{1.809497in}}%
\pgfpathcurveto{\pgfqpoint{1.763372in}{1.809497in}}{\pgfqpoint{1.755472in}{1.806225in}}{\pgfqpoint{1.749648in}{1.800401in}}%
\pgfpathcurveto{\pgfqpoint{1.743825in}{1.794577in}}{\pgfqpoint{1.740552in}{1.786677in}}{\pgfqpoint{1.740552in}{1.778441in}}%
\pgfpathcurveto{\pgfqpoint{1.740552in}{1.770204in}}{\pgfqpoint{1.743825in}{1.762304in}}{\pgfqpoint{1.749648in}{1.756480in}}%
\pgfpathcurveto{\pgfqpoint{1.755472in}{1.750656in}}{\pgfqpoint{1.763372in}{1.747384in}}{\pgfqpoint{1.771609in}{1.747384in}}%
\pgfpathclose%
\pgfusepath{stroke,fill}%
\end{pgfscope}%
\begin{pgfscope}%
\pgfpathrectangle{\pgfqpoint{0.100000in}{0.212622in}}{\pgfqpoint{3.696000in}{3.696000in}}%
\pgfusepath{clip}%
\pgfsetbuttcap%
\pgfsetroundjoin%
\definecolor{currentfill}{rgb}{0.121569,0.466667,0.705882}%
\pgfsetfillcolor{currentfill}%
\pgfsetfillopacity{0.938917}%
\pgfsetlinewidth{1.003750pt}%
\definecolor{currentstroke}{rgb}{0.121569,0.466667,0.705882}%
\pgfsetstrokecolor{currentstroke}%
\pgfsetstrokeopacity{0.938917}%
\pgfsetdash{}{0pt}%
\pgfpathmoveto{\pgfqpoint{1.808083in}{1.715568in}}%
\pgfpathcurveto{\pgfqpoint{1.816319in}{1.715568in}}{\pgfqpoint{1.824219in}{1.718841in}}{\pgfqpoint{1.830043in}{1.724665in}}%
\pgfpathcurveto{\pgfqpoint{1.835867in}{1.730489in}}{\pgfqpoint{1.839139in}{1.738389in}}{\pgfqpoint{1.839139in}{1.746625in}}%
\pgfpathcurveto{\pgfqpoint{1.839139in}{1.754861in}}{\pgfqpoint{1.835867in}{1.762761in}}{\pgfqpoint{1.830043in}{1.768585in}}%
\pgfpathcurveto{\pgfqpoint{1.824219in}{1.774409in}}{\pgfqpoint{1.816319in}{1.777681in}}{\pgfqpoint{1.808083in}{1.777681in}}%
\pgfpathcurveto{\pgfqpoint{1.799847in}{1.777681in}}{\pgfqpoint{1.791947in}{1.774409in}}{\pgfqpoint{1.786123in}{1.768585in}}%
\pgfpathcurveto{\pgfqpoint{1.780299in}{1.762761in}}{\pgfqpoint{1.777026in}{1.754861in}}{\pgfqpoint{1.777026in}{1.746625in}}%
\pgfpathcurveto{\pgfqpoint{1.777026in}{1.738389in}}{\pgfqpoint{1.780299in}{1.730489in}}{\pgfqpoint{1.786123in}{1.724665in}}%
\pgfpathcurveto{\pgfqpoint{1.791947in}{1.718841in}}{\pgfqpoint{1.799847in}{1.715568in}}{\pgfqpoint{1.808083in}{1.715568in}}%
\pgfpathclose%
\pgfusepath{stroke,fill}%
\end{pgfscope}%
\begin{pgfscope}%
\pgfpathrectangle{\pgfqpoint{0.100000in}{0.212622in}}{\pgfqpoint{3.696000in}{3.696000in}}%
\pgfusepath{clip}%
\pgfsetbuttcap%
\pgfsetroundjoin%
\definecolor{currentfill}{rgb}{0.121569,0.466667,0.705882}%
\pgfsetfillcolor{currentfill}%
\pgfsetfillopacity{0.941852}%
\pgfsetlinewidth{1.003750pt}%
\definecolor{currentstroke}{rgb}{0.121569,0.466667,0.705882}%
\pgfsetstrokecolor{currentstroke}%
\pgfsetstrokeopacity{0.941852}%
\pgfsetdash{}{0pt}%
\pgfpathmoveto{\pgfqpoint{2.368543in}{1.401636in}}%
\pgfpathcurveto{\pgfqpoint{2.376779in}{1.401636in}}{\pgfqpoint{2.384680in}{1.404908in}}{\pgfqpoint{2.390503in}{1.410732in}}%
\pgfpathcurveto{\pgfqpoint{2.396327in}{1.416556in}}{\pgfqpoint{2.399600in}{1.424456in}}{\pgfqpoint{2.399600in}{1.432692in}}%
\pgfpathcurveto{\pgfqpoint{2.399600in}{1.440929in}}{\pgfqpoint{2.396327in}{1.448829in}}{\pgfqpoint{2.390503in}{1.454653in}}%
\pgfpathcurveto{\pgfqpoint{2.384680in}{1.460477in}}{\pgfqpoint{2.376779in}{1.463749in}}{\pgfqpoint{2.368543in}{1.463749in}}%
\pgfpathcurveto{\pgfqpoint{2.360307in}{1.463749in}}{\pgfqpoint{2.352407in}{1.460477in}}{\pgfqpoint{2.346583in}{1.454653in}}%
\pgfpathcurveto{\pgfqpoint{2.340759in}{1.448829in}}{\pgfqpoint{2.337487in}{1.440929in}}{\pgfqpoint{2.337487in}{1.432692in}}%
\pgfpathcurveto{\pgfqpoint{2.337487in}{1.424456in}}{\pgfqpoint{2.340759in}{1.416556in}}{\pgfqpoint{2.346583in}{1.410732in}}%
\pgfpathcurveto{\pgfqpoint{2.352407in}{1.404908in}}{\pgfqpoint{2.360307in}{1.401636in}}{\pgfqpoint{2.368543in}{1.401636in}}%
\pgfpathclose%
\pgfusepath{stroke,fill}%
\end{pgfscope}%
\begin{pgfscope}%
\pgfpathrectangle{\pgfqpoint{0.100000in}{0.212622in}}{\pgfqpoint{3.696000in}{3.696000in}}%
\pgfusepath{clip}%
\pgfsetbuttcap%
\pgfsetroundjoin%
\definecolor{currentfill}{rgb}{0.121569,0.466667,0.705882}%
\pgfsetfillcolor{currentfill}%
\pgfsetfillopacity{0.942187}%
\pgfsetlinewidth{1.003750pt}%
\definecolor{currentstroke}{rgb}{0.121569,0.466667,0.705882}%
\pgfsetstrokecolor{currentstroke}%
\pgfsetstrokeopacity{0.942187}%
\pgfsetdash{}{0pt}%
\pgfpathmoveto{\pgfqpoint{1.841948in}{1.691897in}}%
\pgfpathcurveto{\pgfqpoint{1.850184in}{1.691897in}}{\pgfqpoint{1.858084in}{1.695170in}}{\pgfqpoint{1.863908in}{1.700994in}}%
\pgfpathcurveto{\pgfqpoint{1.869732in}{1.706818in}}{\pgfqpoint{1.873004in}{1.714718in}}{\pgfqpoint{1.873004in}{1.722954in}}%
\pgfpathcurveto{\pgfqpoint{1.873004in}{1.731190in}}{\pgfqpoint{1.869732in}{1.739090in}}{\pgfqpoint{1.863908in}{1.744914in}}%
\pgfpathcurveto{\pgfqpoint{1.858084in}{1.750738in}}{\pgfqpoint{1.850184in}{1.754010in}}{\pgfqpoint{1.841948in}{1.754010in}}%
\pgfpathcurveto{\pgfqpoint{1.833712in}{1.754010in}}{\pgfqpoint{1.825812in}{1.750738in}}{\pgfqpoint{1.819988in}{1.744914in}}%
\pgfpathcurveto{\pgfqpoint{1.814164in}{1.739090in}}{\pgfqpoint{1.810891in}{1.731190in}}{\pgfqpoint{1.810891in}{1.722954in}}%
\pgfpathcurveto{\pgfqpoint{1.810891in}{1.714718in}}{\pgfqpoint{1.814164in}{1.706818in}}{\pgfqpoint{1.819988in}{1.700994in}}%
\pgfpathcurveto{\pgfqpoint{1.825812in}{1.695170in}}{\pgfqpoint{1.833712in}{1.691897in}}{\pgfqpoint{1.841948in}{1.691897in}}%
\pgfpathclose%
\pgfusepath{stroke,fill}%
\end{pgfscope}%
\begin{pgfscope}%
\pgfpathrectangle{\pgfqpoint{0.100000in}{0.212622in}}{\pgfqpoint{3.696000in}{3.696000in}}%
\pgfusepath{clip}%
\pgfsetbuttcap%
\pgfsetroundjoin%
\definecolor{currentfill}{rgb}{0.121569,0.466667,0.705882}%
\pgfsetfillcolor{currentfill}%
\pgfsetfillopacity{0.945897}%
\pgfsetlinewidth{1.003750pt}%
\definecolor{currentstroke}{rgb}{0.121569,0.466667,0.705882}%
\pgfsetstrokecolor{currentstroke}%
\pgfsetstrokeopacity{0.945897}%
\pgfsetdash{}{0pt}%
\pgfpathmoveto{\pgfqpoint{1.873272in}{1.672055in}}%
\pgfpathcurveto{\pgfqpoint{1.881508in}{1.672055in}}{\pgfqpoint{1.889408in}{1.675327in}}{\pgfqpoint{1.895232in}{1.681151in}}%
\pgfpathcurveto{\pgfqpoint{1.901056in}{1.686975in}}{\pgfqpoint{1.904328in}{1.694875in}}{\pgfqpoint{1.904328in}{1.703112in}}%
\pgfpathcurveto{\pgfqpoint{1.904328in}{1.711348in}}{\pgfqpoint{1.901056in}{1.719248in}}{\pgfqpoint{1.895232in}{1.725072in}}%
\pgfpathcurveto{\pgfqpoint{1.889408in}{1.730896in}}{\pgfqpoint{1.881508in}{1.734168in}}{\pgfqpoint{1.873272in}{1.734168in}}%
\pgfpathcurveto{\pgfqpoint{1.865036in}{1.734168in}}{\pgfqpoint{1.857136in}{1.730896in}}{\pgfqpoint{1.851312in}{1.725072in}}%
\pgfpathcurveto{\pgfqpoint{1.845488in}{1.719248in}}{\pgfqpoint{1.842215in}{1.711348in}}{\pgfqpoint{1.842215in}{1.703112in}}%
\pgfpathcurveto{\pgfqpoint{1.842215in}{1.694875in}}{\pgfqpoint{1.845488in}{1.686975in}}{\pgfqpoint{1.851312in}{1.681151in}}%
\pgfpathcurveto{\pgfqpoint{1.857136in}{1.675327in}}{\pgfqpoint{1.865036in}{1.672055in}}{\pgfqpoint{1.873272in}{1.672055in}}%
\pgfpathclose%
\pgfusepath{stroke,fill}%
\end{pgfscope}%
\begin{pgfscope}%
\pgfpathrectangle{\pgfqpoint{0.100000in}{0.212622in}}{\pgfqpoint{3.696000in}{3.696000in}}%
\pgfusepath{clip}%
\pgfsetbuttcap%
\pgfsetroundjoin%
\definecolor{currentfill}{rgb}{0.121569,0.466667,0.705882}%
\pgfsetfillcolor{currentfill}%
\pgfsetfillopacity{0.948310}%
\pgfsetlinewidth{1.003750pt}%
\definecolor{currentstroke}{rgb}{0.121569,0.466667,0.705882}%
\pgfsetstrokecolor{currentstroke}%
\pgfsetstrokeopacity{0.948310}%
\pgfsetdash{}{0pt}%
\pgfpathmoveto{\pgfqpoint{2.372834in}{1.399154in}}%
\pgfpathcurveto{\pgfqpoint{2.381070in}{1.399154in}}{\pgfqpoint{2.388970in}{1.402426in}}{\pgfqpoint{2.394794in}{1.408250in}}%
\pgfpathcurveto{\pgfqpoint{2.400618in}{1.414074in}}{\pgfqpoint{2.403891in}{1.421974in}}{\pgfqpoint{2.403891in}{1.430210in}}%
\pgfpathcurveto{\pgfqpoint{2.403891in}{1.438447in}}{\pgfqpoint{2.400618in}{1.446347in}}{\pgfqpoint{2.394794in}{1.452170in}}%
\pgfpathcurveto{\pgfqpoint{2.388970in}{1.457994in}}{\pgfqpoint{2.381070in}{1.461267in}}{\pgfqpoint{2.372834in}{1.461267in}}%
\pgfpathcurveto{\pgfqpoint{2.364598in}{1.461267in}}{\pgfqpoint{2.356698in}{1.457994in}}{\pgfqpoint{2.350874in}{1.452170in}}%
\pgfpathcurveto{\pgfqpoint{2.345050in}{1.446347in}}{\pgfqpoint{2.341778in}{1.438447in}}{\pgfqpoint{2.341778in}{1.430210in}}%
\pgfpathcurveto{\pgfqpoint{2.341778in}{1.421974in}}{\pgfqpoint{2.345050in}{1.414074in}}{\pgfqpoint{2.350874in}{1.408250in}}%
\pgfpathcurveto{\pgfqpoint{2.356698in}{1.402426in}}{\pgfqpoint{2.364598in}{1.399154in}}{\pgfqpoint{2.372834in}{1.399154in}}%
\pgfpathclose%
\pgfusepath{stroke,fill}%
\end{pgfscope}%
\begin{pgfscope}%
\pgfpathrectangle{\pgfqpoint{0.100000in}{0.212622in}}{\pgfqpoint{3.696000in}{3.696000in}}%
\pgfusepath{clip}%
\pgfsetbuttcap%
\pgfsetroundjoin%
\definecolor{currentfill}{rgb}{0.121569,0.466667,0.705882}%
\pgfsetfillcolor{currentfill}%
\pgfsetfillopacity{0.948733}%
\pgfsetlinewidth{1.003750pt}%
\definecolor{currentstroke}{rgb}{0.121569,0.466667,0.705882}%
\pgfsetstrokecolor{currentstroke}%
\pgfsetstrokeopacity{0.948733}%
\pgfsetdash{}{0pt}%
\pgfpathmoveto{\pgfqpoint{1.903168in}{1.653243in}}%
\pgfpathcurveto{\pgfqpoint{1.911404in}{1.653243in}}{\pgfqpoint{1.919304in}{1.656516in}}{\pgfqpoint{1.925128in}{1.662340in}}%
\pgfpathcurveto{\pgfqpoint{1.930952in}{1.668164in}}{\pgfqpoint{1.934224in}{1.676064in}}{\pgfqpoint{1.934224in}{1.684300in}}%
\pgfpathcurveto{\pgfqpoint{1.934224in}{1.692536in}}{\pgfqpoint{1.930952in}{1.700436in}}{\pgfqpoint{1.925128in}{1.706260in}}%
\pgfpathcurveto{\pgfqpoint{1.919304in}{1.712084in}}{\pgfqpoint{1.911404in}{1.715356in}}{\pgfqpoint{1.903168in}{1.715356in}}%
\pgfpathcurveto{\pgfqpoint{1.894931in}{1.715356in}}{\pgfqpoint{1.887031in}{1.712084in}}{\pgfqpoint{1.881207in}{1.706260in}}%
\pgfpathcurveto{\pgfqpoint{1.875383in}{1.700436in}}{\pgfqpoint{1.872111in}{1.692536in}}{\pgfqpoint{1.872111in}{1.684300in}}%
\pgfpathcurveto{\pgfqpoint{1.872111in}{1.676064in}}{\pgfqpoint{1.875383in}{1.668164in}}{\pgfqpoint{1.881207in}{1.662340in}}%
\pgfpathcurveto{\pgfqpoint{1.887031in}{1.656516in}}{\pgfqpoint{1.894931in}{1.653243in}}{\pgfqpoint{1.903168in}{1.653243in}}%
\pgfpathclose%
\pgfusepath{stroke,fill}%
\end{pgfscope}%
\begin{pgfscope}%
\pgfpathrectangle{\pgfqpoint{0.100000in}{0.212622in}}{\pgfqpoint{3.696000in}{3.696000in}}%
\pgfusepath{clip}%
\pgfsetbuttcap%
\pgfsetroundjoin%
\definecolor{currentfill}{rgb}{0.121569,0.466667,0.705882}%
\pgfsetfillcolor{currentfill}%
\pgfsetfillopacity{0.950400}%
\pgfsetlinewidth{1.003750pt}%
\definecolor{currentstroke}{rgb}{0.121569,0.466667,0.705882}%
\pgfsetstrokecolor{currentstroke}%
\pgfsetstrokeopacity{0.950400}%
\pgfsetdash{}{0pt}%
\pgfpathmoveto{\pgfqpoint{1.930513in}{1.630215in}}%
\pgfpathcurveto{\pgfqpoint{1.938750in}{1.630215in}}{\pgfqpoint{1.946650in}{1.633487in}}{\pgfqpoint{1.952474in}{1.639311in}}%
\pgfpathcurveto{\pgfqpoint{1.958298in}{1.645135in}}{\pgfqpoint{1.961570in}{1.653035in}}{\pgfqpoint{1.961570in}{1.661271in}}%
\pgfpathcurveto{\pgfqpoint{1.961570in}{1.669508in}}{\pgfqpoint{1.958298in}{1.677408in}}{\pgfqpoint{1.952474in}{1.683232in}}%
\pgfpathcurveto{\pgfqpoint{1.946650in}{1.689056in}}{\pgfqpoint{1.938750in}{1.692328in}}{\pgfqpoint{1.930513in}{1.692328in}}%
\pgfpathcurveto{\pgfqpoint{1.922277in}{1.692328in}}{\pgfqpoint{1.914377in}{1.689056in}}{\pgfqpoint{1.908553in}{1.683232in}}%
\pgfpathcurveto{\pgfqpoint{1.902729in}{1.677408in}}{\pgfqpoint{1.899457in}{1.669508in}}{\pgfqpoint{1.899457in}{1.661271in}}%
\pgfpathcurveto{\pgfqpoint{1.899457in}{1.653035in}}{\pgfqpoint{1.902729in}{1.645135in}}{\pgfqpoint{1.908553in}{1.639311in}}%
\pgfpathcurveto{\pgfqpoint{1.914377in}{1.633487in}}{\pgfqpoint{1.922277in}{1.630215in}}{\pgfqpoint{1.930513in}{1.630215in}}%
\pgfpathclose%
\pgfusepath{stroke,fill}%
\end{pgfscope}%
\begin{pgfscope}%
\pgfpathrectangle{\pgfqpoint{0.100000in}{0.212622in}}{\pgfqpoint{3.696000in}{3.696000in}}%
\pgfusepath{clip}%
\pgfsetbuttcap%
\pgfsetroundjoin%
\definecolor{currentfill}{rgb}{0.121569,0.466667,0.705882}%
\pgfsetfillcolor{currentfill}%
\pgfsetfillopacity{0.952748}%
\pgfsetlinewidth{1.003750pt}%
\definecolor{currentstroke}{rgb}{0.121569,0.466667,0.705882}%
\pgfsetstrokecolor{currentstroke}%
\pgfsetstrokeopacity{0.952748}%
\pgfsetdash{}{0pt}%
\pgfpathmoveto{\pgfqpoint{1.955890in}{1.618172in}}%
\pgfpathcurveto{\pgfqpoint{1.964127in}{1.618172in}}{\pgfqpoint{1.972027in}{1.621445in}}{\pgfqpoint{1.977851in}{1.627268in}}%
\pgfpathcurveto{\pgfqpoint{1.983675in}{1.633092in}}{\pgfqpoint{1.986947in}{1.640992in}}{\pgfqpoint{1.986947in}{1.649229in}}%
\pgfpathcurveto{\pgfqpoint{1.986947in}{1.657465in}}{\pgfqpoint{1.983675in}{1.665365in}}{\pgfqpoint{1.977851in}{1.671189in}}%
\pgfpathcurveto{\pgfqpoint{1.972027in}{1.677013in}}{\pgfqpoint{1.964127in}{1.680285in}}{\pgfqpoint{1.955890in}{1.680285in}}%
\pgfpathcurveto{\pgfqpoint{1.947654in}{1.680285in}}{\pgfqpoint{1.939754in}{1.677013in}}{\pgfqpoint{1.933930in}{1.671189in}}%
\pgfpathcurveto{\pgfqpoint{1.928106in}{1.665365in}}{\pgfqpoint{1.924834in}{1.657465in}}{\pgfqpoint{1.924834in}{1.649229in}}%
\pgfpathcurveto{\pgfqpoint{1.924834in}{1.640992in}}{\pgfqpoint{1.928106in}{1.633092in}}{\pgfqpoint{1.933930in}{1.627268in}}%
\pgfpathcurveto{\pgfqpoint{1.939754in}{1.621445in}}{\pgfqpoint{1.947654in}{1.618172in}}{\pgfqpoint{1.955890in}{1.618172in}}%
\pgfpathclose%
\pgfusepath{stroke,fill}%
\end{pgfscope}%
\begin{pgfscope}%
\pgfpathrectangle{\pgfqpoint{0.100000in}{0.212622in}}{\pgfqpoint{3.696000in}{3.696000in}}%
\pgfusepath{clip}%
\pgfsetbuttcap%
\pgfsetroundjoin%
\definecolor{currentfill}{rgb}{0.121569,0.466667,0.705882}%
\pgfsetfillcolor{currentfill}%
\pgfsetfillopacity{0.953981}%
\pgfsetlinewidth{1.003750pt}%
\definecolor{currentstroke}{rgb}{0.121569,0.466667,0.705882}%
\pgfsetstrokecolor{currentstroke}%
\pgfsetstrokeopacity{0.953981}%
\pgfsetdash{}{0pt}%
\pgfpathmoveto{\pgfqpoint{2.377103in}{1.389337in}}%
\pgfpathcurveto{\pgfqpoint{2.385340in}{1.389337in}}{\pgfqpoint{2.393240in}{1.392609in}}{\pgfqpoint{2.399064in}{1.398433in}}%
\pgfpathcurveto{\pgfqpoint{2.404888in}{1.404257in}}{\pgfqpoint{2.408160in}{1.412157in}}{\pgfqpoint{2.408160in}{1.420394in}}%
\pgfpathcurveto{\pgfqpoint{2.408160in}{1.428630in}}{\pgfqpoint{2.404888in}{1.436530in}}{\pgfqpoint{2.399064in}{1.442354in}}%
\pgfpathcurveto{\pgfqpoint{2.393240in}{1.448178in}}{\pgfqpoint{2.385340in}{1.451450in}}{\pgfqpoint{2.377103in}{1.451450in}}%
\pgfpathcurveto{\pgfqpoint{2.368867in}{1.451450in}}{\pgfqpoint{2.360967in}{1.448178in}}{\pgfqpoint{2.355143in}{1.442354in}}%
\pgfpathcurveto{\pgfqpoint{2.349319in}{1.436530in}}{\pgfqpoint{2.346047in}{1.428630in}}{\pgfqpoint{2.346047in}{1.420394in}}%
\pgfpathcurveto{\pgfqpoint{2.346047in}{1.412157in}}{\pgfqpoint{2.349319in}{1.404257in}}{\pgfqpoint{2.355143in}{1.398433in}}%
\pgfpathcurveto{\pgfqpoint{2.360967in}{1.392609in}}{\pgfqpoint{2.368867in}{1.389337in}}{\pgfqpoint{2.377103in}{1.389337in}}%
\pgfpathclose%
\pgfusepath{stroke,fill}%
\end{pgfscope}%
\begin{pgfscope}%
\pgfpathrectangle{\pgfqpoint{0.100000in}{0.212622in}}{\pgfqpoint{3.696000in}{3.696000in}}%
\pgfusepath{clip}%
\pgfsetbuttcap%
\pgfsetroundjoin%
\definecolor{currentfill}{rgb}{0.121569,0.466667,0.705882}%
\pgfsetfillcolor{currentfill}%
\pgfsetfillopacity{0.954517}%
\pgfsetlinewidth{1.003750pt}%
\definecolor{currentstroke}{rgb}{0.121569,0.466667,0.705882}%
\pgfsetstrokecolor{currentstroke}%
\pgfsetstrokeopacity{0.954517}%
\pgfsetdash{}{0pt}%
\pgfpathmoveto{\pgfqpoint{1.976013in}{1.606557in}}%
\pgfpathcurveto{\pgfqpoint{1.984249in}{1.606557in}}{\pgfqpoint{1.992149in}{1.609830in}}{\pgfqpoint{1.997973in}{1.615654in}}%
\pgfpathcurveto{\pgfqpoint{2.003797in}{1.621478in}}{\pgfqpoint{2.007069in}{1.629378in}}{\pgfqpoint{2.007069in}{1.637614in}}%
\pgfpathcurveto{\pgfqpoint{2.007069in}{1.645850in}}{\pgfqpoint{2.003797in}{1.653750in}}{\pgfqpoint{1.997973in}{1.659574in}}%
\pgfpathcurveto{\pgfqpoint{1.992149in}{1.665398in}}{\pgfqpoint{1.984249in}{1.668670in}}{\pgfqpoint{1.976013in}{1.668670in}}%
\pgfpathcurveto{\pgfqpoint{1.967777in}{1.668670in}}{\pgfqpoint{1.959877in}{1.665398in}}{\pgfqpoint{1.954053in}{1.659574in}}%
\pgfpathcurveto{\pgfqpoint{1.948229in}{1.653750in}}{\pgfqpoint{1.944956in}{1.645850in}}{\pgfqpoint{1.944956in}{1.637614in}}%
\pgfpathcurveto{\pgfqpoint{1.944956in}{1.629378in}}{\pgfqpoint{1.948229in}{1.621478in}}{\pgfqpoint{1.954053in}{1.615654in}}%
\pgfpathcurveto{\pgfqpoint{1.959877in}{1.609830in}}{\pgfqpoint{1.967777in}{1.606557in}}{\pgfqpoint{1.976013in}{1.606557in}}%
\pgfpathclose%
\pgfusepath{stroke,fill}%
\end{pgfscope}%
\begin{pgfscope}%
\pgfpathrectangle{\pgfqpoint{0.100000in}{0.212622in}}{\pgfqpoint{3.696000in}{3.696000in}}%
\pgfusepath{clip}%
\pgfsetbuttcap%
\pgfsetroundjoin%
\definecolor{currentfill}{rgb}{0.121569,0.466667,0.705882}%
\pgfsetfillcolor{currentfill}%
\pgfsetfillopacity{0.955680}%
\pgfsetlinewidth{1.003750pt}%
\definecolor{currentstroke}{rgb}{0.121569,0.466667,0.705882}%
\pgfsetstrokecolor{currentstroke}%
\pgfsetstrokeopacity{0.955680}%
\pgfsetdash{}{0pt}%
\pgfpathmoveto{\pgfqpoint{1.994301in}{1.593684in}}%
\pgfpathcurveto{\pgfqpoint{2.002538in}{1.593684in}}{\pgfqpoint{2.010438in}{1.596956in}}{\pgfqpoint{2.016262in}{1.602780in}}%
\pgfpathcurveto{\pgfqpoint{2.022085in}{1.608604in}}{\pgfqpoint{2.025358in}{1.616504in}}{\pgfqpoint{2.025358in}{1.624740in}}%
\pgfpathcurveto{\pgfqpoint{2.025358in}{1.632976in}}{\pgfqpoint{2.022085in}{1.640876in}}{\pgfqpoint{2.016262in}{1.646700in}}%
\pgfpathcurveto{\pgfqpoint{2.010438in}{1.652524in}}{\pgfqpoint{2.002538in}{1.655797in}}{\pgfqpoint{1.994301in}{1.655797in}}%
\pgfpathcurveto{\pgfqpoint{1.986065in}{1.655797in}}{\pgfqpoint{1.978165in}{1.652524in}}{\pgfqpoint{1.972341in}{1.646700in}}%
\pgfpathcurveto{\pgfqpoint{1.966517in}{1.640876in}}{\pgfqpoint{1.963245in}{1.632976in}}{\pgfqpoint{1.963245in}{1.624740in}}%
\pgfpathcurveto{\pgfqpoint{1.963245in}{1.616504in}}{\pgfqpoint{1.966517in}{1.608604in}}{\pgfqpoint{1.972341in}{1.602780in}}%
\pgfpathcurveto{\pgfqpoint{1.978165in}{1.596956in}}{\pgfqpoint{1.986065in}{1.593684in}}{\pgfqpoint{1.994301in}{1.593684in}}%
\pgfpathclose%
\pgfusepath{stroke,fill}%
\end{pgfscope}%
\begin{pgfscope}%
\pgfpathrectangle{\pgfqpoint{0.100000in}{0.212622in}}{\pgfqpoint{3.696000in}{3.696000in}}%
\pgfusepath{clip}%
\pgfsetbuttcap%
\pgfsetroundjoin%
\definecolor{currentfill}{rgb}{0.121569,0.466667,0.705882}%
\pgfsetfillcolor{currentfill}%
\pgfsetfillopacity{0.957052}%
\pgfsetlinewidth{1.003750pt}%
\definecolor{currentstroke}{rgb}{0.121569,0.466667,0.705882}%
\pgfsetstrokecolor{currentstroke}%
\pgfsetstrokeopacity{0.957052}%
\pgfsetdash{}{0pt}%
\pgfpathmoveto{\pgfqpoint{2.010186in}{1.582270in}}%
\pgfpathcurveto{\pgfqpoint{2.018422in}{1.582270in}}{\pgfqpoint{2.026322in}{1.585542in}}{\pgfqpoint{2.032146in}{1.591366in}}%
\pgfpathcurveto{\pgfqpoint{2.037970in}{1.597190in}}{\pgfqpoint{2.041242in}{1.605090in}}{\pgfqpoint{2.041242in}{1.613327in}}%
\pgfpathcurveto{\pgfqpoint{2.041242in}{1.621563in}}{\pgfqpoint{2.037970in}{1.629463in}}{\pgfqpoint{2.032146in}{1.635287in}}%
\pgfpathcurveto{\pgfqpoint{2.026322in}{1.641111in}}{\pgfqpoint{2.018422in}{1.644383in}}{\pgfqpoint{2.010186in}{1.644383in}}%
\pgfpathcurveto{\pgfqpoint{2.001949in}{1.644383in}}{\pgfqpoint{1.994049in}{1.641111in}}{\pgfqpoint{1.988225in}{1.635287in}}%
\pgfpathcurveto{\pgfqpoint{1.982402in}{1.629463in}}{\pgfqpoint{1.979129in}{1.621563in}}{\pgfqpoint{1.979129in}{1.613327in}}%
\pgfpathcurveto{\pgfqpoint{1.979129in}{1.605090in}}{\pgfqpoint{1.982402in}{1.597190in}}{\pgfqpoint{1.988225in}{1.591366in}}%
\pgfpathcurveto{\pgfqpoint{1.994049in}{1.585542in}}{\pgfqpoint{2.001949in}{1.582270in}}{\pgfqpoint{2.010186in}{1.582270in}}%
\pgfpathclose%
\pgfusepath{stroke,fill}%
\end{pgfscope}%
\begin{pgfscope}%
\pgfpathrectangle{\pgfqpoint{0.100000in}{0.212622in}}{\pgfqpoint{3.696000in}{3.696000in}}%
\pgfusepath{clip}%
\pgfsetbuttcap%
\pgfsetroundjoin%
\definecolor{currentfill}{rgb}{0.121569,0.466667,0.705882}%
\pgfsetfillcolor{currentfill}%
\pgfsetfillopacity{0.958204}%
\pgfsetlinewidth{1.003750pt}%
\definecolor{currentstroke}{rgb}{0.121569,0.466667,0.705882}%
\pgfsetstrokecolor{currentstroke}%
\pgfsetstrokeopacity{0.958204}%
\pgfsetdash{}{0pt}%
\pgfpathmoveto{\pgfqpoint{2.023278in}{1.570451in}}%
\pgfpathcurveto{\pgfqpoint{2.031515in}{1.570451in}}{\pgfqpoint{2.039415in}{1.573723in}}{\pgfqpoint{2.045239in}{1.579547in}}%
\pgfpathcurveto{\pgfqpoint{2.051062in}{1.585371in}}{\pgfqpoint{2.054335in}{1.593271in}}{\pgfqpoint{2.054335in}{1.601507in}}%
\pgfpathcurveto{\pgfqpoint{2.054335in}{1.609744in}}{\pgfqpoint{2.051062in}{1.617644in}}{\pgfqpoint{2.045239in}{1.623468in}}%
\pgfpathcurveto{\pgfqpoint{2.039415in}{1.629292in}}{\pgfqpoint{2.031515in}{1.632564in}}{\pgfqpoint{2.023278in}{1.632564in}}%
\pgfpathcurveto{\pgfqpoint{2.015042in}{1.632564in}}{\pgfqpoint{2.007142in}{1.629292in}}{\pgfqpoint{2.001318in}{1.623468in}}%
\pgfpathcurveto{\pgfqpoint{1.995494in}{1.617644in}}{\pgfqpoint{1.992222in}{1.609744in}}{\pgfqpoint{1.992222in}{1.601507in}}%
\pgfpathcurveto{\pgfqpoint{1.992222in}{1.593271in}}{\pgfqpoint{1.995494in}{1.585371in}}{\pgfqpoint{2.001318in}{1.579547in}}%
\pgfpathcurveto{\pgfqpoint{2.007142in}{1.573723in}}{\pgfqpoint{2.015042in}{1.570451in}}{\pgfqpoint{2.023278in}{1.570451in}}%
\pgfpathclose%
\pgfusepath{stroke,fill}%
\end{pgfscope}%
\begin{pgfscope}%
\pgfpathrectangle{\pgfqpoint{0.100000in}{0.212622in}}{\pgfqpoint{3.696000in}{3.696000in}}%
\pgfusepath{clip}%
\pgfsetbuttcap%
\pgfsetroundjoin%
\definecolor{currentfill}{rgb}{0.121569,0.466667,0.705882}%
\pgfsetfillcolor{currentfill}%
\pgfsetfillopacity{0.959318}%
\pgfsetlinewidth{1.003750pt}%
\definecolor{currentstroke}{rgb}{0.121569,0.466667,0.705882}%
\pgfsetstrokecolor{currentstroke}%
\pgfsetstrokeopacity{0.959318}%
\pgfsetdash{}{0pt}%
\pgfpathmoveto{\pgfqpoint{2.036184in}{1.561120in}}%
\pgfpathcurveto{\pgfqpoint{2.044420in}{1.561120in}}{\pgfqpoint{2.052320in}{1.564392in}}{\pgfqpoint{2.058144in}{1.570216in}}%
\pgfpathcurveto{\pgfqpoint{2.063968in}{1.576040in}}{\pgfqpoint{2.067240in}{1.583940in}}{\pgfqpoint{2.067240in}{1.592176in}}%
\pgfpathcurveto{\pgfqpoint{2.067240in}{1.600413in}}{\pgfqpoint{2.063968in}{1.608313in}}{\pgfqpoint{2.058144in}{1.614137in}}%
\pgfpathcurveto{\pgfqpoint{2.052320in}{1.619961in}}{\pgfqpoint{2.044420in}{1.623233in}}{\pgfqpoint{2.036184in}{1.623233in}}%
\pgfpathcurveto{\pgfqpoint{2.027948in}{1.623233in}}{\pgfqpoint{2.020048in}{1.619961in}}{\pgfqpoint{2.014224in}{1.614137in}}%
\pgfpathcurveto{\pgfqpoint{2.008400in}{1.608313in}}{\pgfqpoint{2.005127in}{1.600413in}}{\pgfqpoint{2.005127in}{1.592176in}}%
\pgfpathcurveto{\pgfqpoint{2.005127in}{1.583940in}}{\pgfqpoint{2.008400in}{1.576040in}}{\pgfqpoint{2.014224in}{1.570216in}}%
\pgfpathcurveto{\pgfqpoint{2.020048in}{1.564392in}}{\pgfqpoint{2.027948in}{1.561120in}}{\pgfqpoint{2.036184in}{1.561120in}}%
\pgfpathclose%
\pgfusepath{stroke,fill}%
\end{pgfscope}%
\begin{pgfscope}%
\pgfpathrectangle{\pgfqpoint{0.100000in}{0.212622in}}{\pgfqpoint{3.696000in}{3.696000in}}%
\pgfusepath{clip}%
\pgfsetbuttcap%
\pgfsetroundjoin%
\definecolor{currentfill}{rgb}{0.121569,0.466667,0.705882}%
\pgfsetfillcolor{currentfill}%
\pgfsetfillopacity{0.959796}%
\pgfsetlinewidth{1.003750pt}%
\definecolor{currentstroke}{rgb}{0.121569,0.466667,0.705882}%
\pgfsetstrokecolor{currentstroke}%
\pgfsetstrokeopacity{0.959796}%
\pgfsetdash{}{0pt}%
\pgfpathmoveto{\pgfqpoint{2.381758in}{1.378398in}}%
\pgfpathcurveto{\pgfqpoint{2.389994in}{1.378398in}}{\pgfqpoint{2.397894in}{1.381671in}}{\pgfqpoint{2.403718in}{1.387494in}}%
\pgfpathcurveto{\pgfqpoint{2.409542in}{1.393318in}}{\pgfqpoint{2.412814in}{1.401218in}}{\pgfqpoint{2.412814in}{1.409455in}}%
\pgfpathcurveto{\pgfqpoint{2.412814in}{1.417691in}}{\pgfqpoint{2.409542in}{1.425591in}}{\pgfqpoint{2.403718in}{1.431415in}}%
\pgfpathcurveto{\pgfqpoint{2.397894in}{1.437239in}}{\pgfqpoint{2.389994in}{1.440511in}}{\pgfqpoint{2.381758in}{1.440511in}}%
\pgfpathcurveto{\pgfqpoint{2.373521in}{1.440511in}}{\pgfqpoint{2.365621in}{1.437239in}}{\pgfqpoint{2.359797in}{1.431415in}}%
\pgfpathcurveto{\pgfqpoint{2.353973in}{1.425591in}}{\pgfqpoint{2.350701in}{1.417691in}}{\pgfqpoint{2.350701in}{1.409455in}}%
\pgfpathcurveto{\pgfqpoint{2.350701in}{1.401218in}}{\pgfqpoint{2.353973in}{1.393318in}}{\pgfqpoint{2.359797in}{1.387494in}}%
\pgfpathcurveto{\pgfqpoint{2.365621in}{1.381671in}}{\pgfqpoint{2.373521in}{1.378398in}}{\pgfqpoint{2.381758in}{1.378398in}}%
\pgfpathclose%
\pgfusepath{stroke,fill}%
\end{pgfscope}%
\begin{pgfscope}%
\pgfpathrectangle{\pgfqpoint{0.100000in}{0.212622in}}{\pgfqpoint{3.696000in}{3.696000in}}%
\pgfusepath{clip}%
\pgfsetbuttcap%
\pgfsetroundjoin%
\definecolor{currentfill}{rgb}{0.121569,0.466667,0.705882}%
\pgfsetfillcolor{currentfill}%
\pgfsetfillopacity{0.960520}%
\pgfsetlinewidth{1.003750pt}%
\definecolor{currentstroke}{rgb}{0.121569,0.466667,0.705882}%
\pgfsetstrokecolor{currentstroke}%
\pgfsetstrokeopacity{0.960520}%
\pgfsetdash{}{0pt}%
\pgfpathmoveto{\pgfqpoint{2.046074in}{1.555341in}}%
\pgfpathcurveto{\pgfqpoint{2.054310in}{1.555341in}}{\pgfqpoint{2.062210in}{1.558613in}}{\pgfqpoint{2.068034in}{1.564437in}}%
\pgfpathcurveto{\pgfqpoint{2.073858in}{1.570261in}}{\pgfqpoint{2.077130in}{1.578161in}}{\pgfqpoint{2.077130in}{1.586398in}}%
\pgfpathcurveto{\pgfqpoint{2.077130in}{1.594634in}}{\pgfqpoint{2.073858in}{1.602534in}}{\pgfqpoint{2.068034in}{1.608358in}}%
\pgfpathcurveto{\pgfqpoint{2.062210in}{1.614182in}}{\pgfqpoint{2.054310in}{1.617454in}}{\pgfqpoint{2.046074in}{1.617454in}}%
\pgfpathcurveto{\pgfqpoint{2.037837in}{1.617454in}}{\pgfqpoint{2.029937in}{1.614182in}}{\pgfqpoint{2.024113in}{1.608358in}}%
\pgfpathcurveto{\pgfqpoint{2.018289in}{1.602534in}}{\pgfqpoint{2.015017in}{1.594634in}}{\pgfqpoint{2.015017in}{1.586398in}}%
\pgfpathcurveto{\pgfqpoint{2.015017in}{1.578161in}}{\pgfqpoint{2.018289in}{1.570261in}}{\pgfqpoint{2.024113in}{1.564437in}}%
\pgfpathcurveto{\pgfqpoint{2.029937in}{1.558613in}}{\pgfqpoint{2.037837in}{1.555341in}}{\pgfqpoint{2.046074in}{1.555341in}}%
\pgfpathclose%
\pgfusepath{stroke,fill}%
\end{pgfscope}%
\begin{pgfscope}%
\pgfpathrectangle{\pgfqpoint{0.100000in}{0.212622in}}{\pgfqpoint{3.696000in}{3.696000in}}%
\pgfusepath{clip}%
\pgfsetbuttcap%
\pgfsetroundjoin%
\definecolor{currentfill}{rgb}{0.121569,0.466667,0.705882}%
\pgfsetfillcolor{currentfill}%
\pgfsetfillopacity{0.961180}%
\pgfsetlinewidth{1.003750pt}%
\definecolor{currentstroke}{rgb}{0.121569,0.466667,0.705882}%
\pgfsetstrokecolor{currentstroke}%
\pgfsetstrokeopacity{0.961180}%
\pgfsetdash{}{0pt}%
\pgfpathmoveto{\pgfqpoint{2.054763in}{1.548152in}}%
\pgfpathcurveto{\pgfqpoint{2.062999in}{1.548152in}}{\pgfqpoint{2.070899in}{1.551425in}}{\pgfqpoint{2.076723in}{1.557248in}}%
\pgfpathcurveto{\pgfqpoint{2.082547in}{1.563072in}}{\pgfqpoint{2.085819in}{1.570972in}}{\pgfqpoint{2.085819in}{1.579209in}}%
\pgfpathcurveto{\pgfqpoint{2.085819in}{1.587445in}}{\pgfqpoint{2.082547in}{1.595345in}}{\pgfqpoint{2.076723in}{1.601169in}}%
\pgfpathcurveto{\pgfqpoint{2.070899in}{1.606993in}}{\pgfqpoint{2.062999in}{1.610265in}}{\pgfqpoint{2.054763in}{1.610265in}}%
\pgfpathcurveto{\pgfqpoint{2.046526in}{1.610265in}}{\pgfqpoint{2.038626in}{1.606993in}}{\pgfqpoint{2.032803in}{1.601169in}}%
\pgfpathcurveto{\pgfqpoint{2.026979in}{1.595345in}}{\pgfqpoint{2.023706in}{1.587445in}}{\pgfqpoint{2.023706in}{1.579209in}}%
\pgfpathcurveto{\pgfqpoint{2.023706in}{1.570972in}}{\pgfqpoint{2.026979in}{1.563072in}}{\pgfqpoint{2.032803in}{1.557248in}}%
\pgfpathcurveto{\pgfqpoint{2.038626in}{1.551425in}}{\pgfqpoint{2.046526in}{1.548152in}}{\pgfqpoint{2.054763in}{1.548152in}}%
\pgfpathclose%
\pgfusepath{stroke,fill}%
\end{pgfscope}%
\begin{pgfscope}%
\pgfpathrectangle{\pgfqpoint{0.100000in}{0.212622in}}{\pgfqpoint{3.696000in}{3.696000in}}%
\pgfusepath{clip}%
\pgfsetbuttcap%
\pgfsetroundjoin%
\definecolor{currentfill}{rgb}{0.121569,0.466667,0.705882}%
\pgfsetfillcolor{currentfill}%
\pgfsetfillopacity{0.961814}%
\pgfsetlinewidth{1.003750pt}%
\definecolor{currentstroke}{rgb}{0.121569,0.466667,0.705882}%
\pgfsetstrokecolor{currentstroke}%
\pgfsetstrokeopacity{0.961814}%
\pgfsetdash{}{0pt}%
\pgfpathmoveto{\pgfqpoint{2.061253in}{1.543721in}}%
\pgfpathcurveto{\pgfqpoint{2.069489in}{1.543721in}}{\pgfqpoint{2.077389in}{1.546994in}}{\pgfqpoint{2.083213in}{1.552817in}}%
\pgfpathcurveto{\pgfqpoint{2.089037in}{1.558641in}}{\pgfqpoint{2.092309in}{1.566541in}}{\pgfqpoint{2.092309in}{1.574778in}}%
\pgfpathcurveto{\pgfqpoint{2.092309in}{1.583014in}}{\pgfqpoint{2.089037in}{1.590914in}}{\pgfqpoint{2.083213in}{1.596738in}}%
\pgfpathcurveto{\pgfqpoint{2.077389in}{1.602562in}}{\pgfqpoint{2.069489in}{1.605834in}}{\pgfqpoint{2.061253in}{1.605834in}}%
\pgfpathcurveto{\pgfqpoint{2.053016in}{1.605834in}}{\pgfqpoint{2.045116in}{1.602562in}}{\pgfqpoint{2.039292in}{1.596738in}}%
\pgfpathcurveto{\pgfqpoint{2.033468in}{1.590914in}}{\pgfqpoint{2.030196in}{1.583014in}}{\pgfqpoint{2.030196in}{1.574778in}}%
\pgfpathcurveto{\pgfqpoint{2.030196in}{1.566541in}}{\pgfqpoint{2.033468in}{1.558641in}}{\pgfqpoint{2.039292in}{1.552817in}}%
\pgfpathcurveto{\pgfqpoint{2.045116in}{1.546994in}}{\pgfqpoint{2.053016in}{1.543721in}}{\pgfqpoint{2.061253in}{1.543721in}}%
\pgfpathclose%
\pgfusepath{stroke,fill}%
\end{pgfscope}%
\begin{pgfscope}%
\pgfpathrectangle{\pgfqpoint{0.100000in}{0.212622in}}{\pgfqpoint{3.696000in}{3.696000in}}%
\pgfusepath{clip}%
\pgfsetbuttcap%
\pgfsetroundjoin%
\definecolor{currentfill}{rgb}{0.121569,0.466667,0.705882}%
\pgfsetfillcolor{currentfill}%
\pgfsetfillopacity{0.962737}%
\pgfsetlinewidth{1.003750pt}%
\definecolor{currentstroke}{rgb}{0.121569,0.466667,0.705882}%
\pgfsetstrokecolor{currentstroke}%
\pgfsetstrokeopacity{0.962737}%
\pgfsetdash{}{0pt}%
\pgfpathmoveto{\pgfqpoint{2.384587in}{1.370954in}}%
\pgfpathcurveto{\pgfqpoint{2.392824in}{1.370954in}}{\pgfqpoint{2.400724in}{1.374226in}}{\pgfqpoint{2.406548in}{1.380050in}}%
\pgfpathcurveto{\pgfqpoint{2.412371in}{1.385874in}}{\pgfqpoint{2.415644in}{1.393774in}}{\pgfqpoint{2.415644in}{1.402011in}}%
\pgfpathcurveto{\pgfqpoint{2.415644in}{1.410247in}}{\pgfqpoint{2.412371in}{1.418147in}}{\pgfqpoint{2.406548in}{1.423971in}}%
\pgfpathcurveto{\pgfqpoint{2.400724in}{1.429795in}}{\pgfqpoint{2.392824in}{1.433067in}}{\pgfqpoint{2.384587in}{1.433067in}}%
\pgfpathcurveto{\pgfqpoint{2.376351in}{1.433067in}}{\pgfqpoint{2.368451in}{1.429795in}}{\pgfqpoint{2.362627in}{1.423971in}}%
\pgfpathcurveto{\pgfqpoint{2.356803in}{1.418147in}}{\pgfqpoint{2.353531in}{1.410247in}}{\pgfqpoint{2.353531in}{1.402011in}}%
\pgfpathcurveto{\pgfqpoint{2.353531in}{1.393774in}}{\pgfqpoint{2.356803in}{1.385874in}}{\pgfqpoint{2.362627in}{1.380050in}}%
\pgfpathcurveto{\pgfqpoint{2.368451in}{1.374226in}}{\pgfqpoint{2.376351in}{1.370954in}}{\pgfqpoint{2.384587in}{1.370954in}}%
\pgfpathclose%
\pgfusepath{stroke,fill}%
\end{pgfscope}%
\begin{pgfscope}%
\pgfpathrectangle{\pgfqpoint{0.100000in}{0.212622in}}{\pgfqpoint{3.696000in}{3.696000in}}%
\pgfusepath{clip}%
\pgfsetbuttcap%
\pgfsetroundjoin%
\definecolor{currentfill}{rgb}{0.121569,0.466667,0.705882}%
\pgfsetfillcolor{currentfill}%
\pgfsetfillopacity{0.963377}%
\pgfsetlinewidth{1.003750pt}%
\definecolor{currentstroke}{rgb}{0.121569,0.466667,0.705882}%
\pgfsetstrokecolor{currentstroke}%
\pgfsetstrokeopacity{0.963377}%
\pgfsetdash{}{0pt}%
\pgfpathmoveto{\pgfqpoint{2.073719in}{1.540424in}}%
\pgfpathcurveto{\pgfqpoint{2.081955in}{1.540424in}}{\pgfqpoint{2.089855in}{1.543697in}}{\pgfqpoint{2.095679in}{1.549521in}}%
\pgfpathcurveto{\pgfqpoint{2.101503in}{1.555345in}}{\pgfqpoint{2.104776in}{1.563245in}}{\pgfqpoint{2.104776in}{1.571481in}}%
\pgfpathcurveto{\pgfqpoint{2.104776in}{1.579717in}}{\pgfqpoint{2.101503in}{1.587617in}}{\pgfqpoint{2.095679in}{1.593441in}}%
\pgfpathcurveto{\pgfqpoint{2.089855in}{1.599265in}}{\pgfqpoint{2.081955in}{1.602537in}}{\pgfqpoint{2.073719in}{1.602537in}}%
\pgfpathcurveto{\pgfqpoint{2.065483in}{1.602537in}}{\pgfqpoint{2.057583in}{1.599265in}}{\pgfqpoint{2.051759in}{1.593441in}}%
\pgfpathcurveto{\pgfqpoint{2.045935in}{1.587617in}}{\pgfqpoint{2.042663in}{1.579717in}}{\pgfqpoint{2.042663in}{1.571481in}}%
\pgfpathcurveto{\pgfqpoint{2.042663in}{1.563245in}}{\pgfqpoint{2.045935in}{1.555345in}}{\pgfqpoint{2.051759in}{1.549521in}}%
\pgfpathcurveto{\pgfqpoint{2.057583in}{1.543697in}}{\pgfqpoint{2.065483in}{1.540424in}}{\pgfqpoint{2.073719in}{1.540424in}}%
\pgfpathclose%
\pgfusepath{stroke,fill}%
\end{pgfscope}%
\begin{pgfscope}%
\pgfpathrectangle{\pgfqpoint{0.100000in}{0.212622in}}{\pgfqpoint{3.696000in}{3.696000in}}%
\pgfusepath{clip}%
\pgfsetbuttcap%
\pgfsetroundjoin%
\definecolor{currentfill}{rgb}{0.121569,0.466667,0.705882}%
\pgfsetfillcolor{currentfill}%
\pgfsetfillopacity{0.964038}%
\pgfsetlinewidth{1.003750pt}%
\definecolor{currentstroke}{rgb}{0.121569,0.466667,0.705882}%
\pgfsetstrokecolor{currentstroke}%
\pgfsetstrokeopacity{0.964038}%
\pgfsetdash{}{0pt}%
\pgfpathmoveto{\pgfqpoint{2.081739in}{1.534080in}}%
\pgfpathcurveto{\pgfqpoint{2.089975in}{1.534080in}}{\pgfqpoint{2.097875in}{1.537353in}}{\pgfqpoint{2.103699in}{1.543176in}}%
\pgfpathcurveto{\pgfqpoint{2.109523in}{1.549000in}}{\pgfqpoint{2.112795in}{1.556900in}}{\pgfqpoint{2.112795in}{1.565137in}}%
\pgfpathcurveto{\pgfqpoint{2.112795in}{1.573373in}}{\pgfqpoint{2.109523in}{1.581273in}}{\pgfqpoint{2.103699in}{1.587097in}}%
\pgfpathcurveto{\pgfqpoint{2.097875in}{1.592921in}}{\pgfqpoint{2.089975in}{1.596193in}}{\pgfqpoint{2.081739in}{1.596193in}}%
\pgfpathcurveto{\pgfqpoint{2.073502in}{1.596193in}}{\pgfqpoint{2.065602in}{1.592921in}}{\pgfqpoint{2.059778in}{1.587097in}}%
\pgfpathcurveto{\pgfqpoint{2.053954in}{1.581273in}}{\pgfqpoint{2.050682in}{1.573373in}}{\pgfqpoint{2.050682in}{1.565137in}}%
\pgfpathcurveto{\pgfqpoint{2.050682in}{1.556900in}}{\pgfqpoint{2.053954in}{1.549000in}}{\pgfqpoint{2.059778in}{1.543176in}}%
\pgfpathcurveto{\pgfqpoint{2.065602in}{1.537353in}}{\pgfqpoint{2.073502in}{1.534080in}}{\pgfqpoint{2.081739in}{1.534080in}}%
\pgfpathclose%
\pgfusepath{stroke,fill}%
\end{pgfscope}%
\begin{pgfscope}%
\pgfpathrectangle{\pgfqpoint{0.100000in}{0.212622in}}{\pgfqpoint{3.696000in}{3.696000in}}%
\pgfusepath{clip}%
\pgfsetbuttcap%
\pgfsetroundjoin%
\definecolor{currentfill}{rgb}{0.121569,0.466667,0.705882}%
\pgfsetfillcolor{currentfill}%
\pgfsetfillopacity{0.965910}%
\pgfsetlinewidth{1.003750pt}%
\definecolor{currentstroke}{rgb}{0.121569,0.466667,0.705882}%
\pgfsetstrokecolor{currentstroke}%
\pgfsetstrokeopacity{0.965910}%
\pgfsetdash{}{0pt}%
\pgfpathmoveto{\pgfqpoint{2.096012in}{1.525611in}}%
\pgfpathcurveto{\pgfqpoint{2.104249in}{1.525611in}}{\pgfqpoint{2.112149in}{1.528884in}}{\pgfqpoint{2.117973in}{1.534708in}}%
\pgfpathcurveto{\pgfqpoint{2.123797in}{1.540532in}}{\pgfqpoint{2.127069in}{1.548432in}}{\pgfqpoint{2.127069in}{1.556668in}}%
\pgfpathcurveto{\pgfqpoint{2.127069in}{1.564904in}}{\pgfqpoint{2.123797in}{1.572804in}}{\pgfqpoint{2.117973in}{1.578628in}}%
\pgfpathcurveto{\pgfqpoint{2.112149in}{1.584452in}}{\pgfqpoint{2.104249in}{1.587724in}}{\pgfqpoint{2.096012in}{1.587724in}}%
\pgfpathcurveto{\pgfqpoint{2.087776in}{1.587724in}}{\pgfqpoint{2.079876in}{1.584452in}}{\pgfqpoint{2.074052in}{1.578628in}}%
\pgfpathcurveto{\pgfqpoint{2.068228in}{1.572804in}}{\pgfqpoint{2.064956in}{1.564904in}}{\pgfqpoint{2.064956in}{1.556668in}}%
\pgfpathcurveto{\pgfqpoint{2.064956in}{1.548432in}}{\pgfqpoint{2.068228in}{1.540532in}}{\pgfqpoint{2.074052in}{1.534708in}}%
\pgfpathcurveto{\pgfqpoint{2.079876in}{1.528884in}}{\pgfqpoint{2.087776in}{1.525611in}}{\pgfqpoint{2.096012in}{1.525611in}}%
\pgfpathclose%
\pgfusepath{stroke,fill}%
\end{pgfscope}%
\begin{pgfscope}%
\pgfpathrectangle{\pgfqpoint{0.100000in}{0.212622in}}{\pgfqpoint{3.696000in}{3.696000in}}%
\pgfusepath{clip}%
\pgfsetbuttcap%
\pgfsetroundjoin%
\definecolor{currentfill}{rgb}{0.121569,0.466667,0.705882}%
\pgfsetfillcolor{currentfill}%
\pgfsetfillopacity{0.966086}%
\pgfsetlinewidth{1.003750pt}%
\definecolor{currentstroke}{rgb}{0.121569,0.466667,0.705882}%
\pgfsetstrokecolor{currentstroke}%
\pgfsetstrokeopacity{0.966086}%
\pgfsetdash{}{0pt}%
\pgfpathmoveto{\pgfqpoint{2.387147in}{1.364234in}}%
\pgfpathcurveto{\pgfqpoint{2.395383in}{1.364234in}}{\pgfqpoint{2.403283in}{1.367506in}}{\pgfqpoint{2.409107in}{1.373330in}}%
\pgfpathcurveto{\pgfqpoint{2.414931in}{1.379154in}}{\pgfqpoint{2.418203in}{1.387054in}}{\pgfqpoint{2.418203in}{1.395291in}}%
\pgfpathcurveto{\pgfqpoint{2.418203in}{1.403527in}}{\pgfqpoint{2.414931in}{1.411427in}}{\pgfqpoint{2.409107in}{1.417251in}}%
\pgfpathcurveto{\pgfqpoint{2.403283in}{1.423075in}}{\pgfqpoint{2.395383in}{1.426347in}}{\pgfqpoint{2.387147in}{1.426347in}}%
\pgfpathcurveto{\pgfqpoint{2.378911in}{1.426347in}}{\pgfqpoint{2.371011in}{1.423075in}}{\pgfqpoint{2.365187in}{1.417251in}}%
\pgfpathcurveto{\pgfqpoint{2.359363in}{1.411427in}}{\pgfqpoint{2.356090in}{1.403527in}}{\pgfqpoint{2.356090in}{1.395291in}}%
\pgfpathcurveto{\pgfqpoint{2.356090in}{1.387054in}}{\pgfqpoint{2.359363in}{1.379154in}}{\pgfqpoint{2.365187in}{1.373330in}}%
\pgfpathcurveto{\pgfqpoint{2.371011in}{1.367506in}}{\pgfqpoint{2.378911in}{1.364234in}}{\pgfqpoint{2.387147in}{1.364234in}}%
\pgfpathclose%
\pgfusepath{stroke,fill}%
\end{pgfscope}%
\begin{pgfscope}%
\pgfpathrectangle{\pgfqpoint{0.100000in}{0.212622in}}{\pgfqpoint{3.696000in}{3.696000in}}%
\pgfusepath{clip}%
\pgfsetbuttcap%
\pgfsetroundjoin%
\definecolor{currentfill}{rgb}{0.121569,0.466667,0.705882}%
\pgfsetfillcolor{currentfill}%
\pgfsetfillopacity{0.966666}%
\pgfsetlinewidth{1.003750pt}%
\definecolor{currentstroke}{rgb}{0.121569,0.466667,0.705882}%
\pgfsetstrokecolor{currentstroke}%
\pgfsetstrokeopacity{0.966666}%
\pgfsetdash{}{0pt}%
\pgfpathmoveto{\pgfqpoint{2.107363in}{1.513517in}}%
\pgfpathcurveto{\pgfqpoint{2.115600in}{1.513517in}}{\pgfqpoint{2.123500in}{1.516790in}}{\pgfqpoint{2.129323in}{1.522614in}}%
\pgfpathcurveto{\pgfqpoint{2.135147in}{1.528438in}}{\pgfqpoint{2.138420in}{1.536338in}}{\pgfqpoint{2.138420in}{1.544574in}}%
\pgfpathcurveto{\pgfqpoint{2.138420in}{1.552810in}}{\pgfqpoint{2.135147in}{1.560710in}}{\pgfqpoint{2.129323in}{1.566534in}}%
\pgfpathcurveto{\pgfqpoint{2.123500in}{1.572358in}}{\pgfqpoint{2.115600in}{1.575630in}}{\pgfqpoint{2.107363in}{1.575630in}}%
\pgfpathcurveto{\pgfqpoint{2.099127in}{1.575630in}}{\pgfqpoint{2.091227in}{1.572358in}}{\pgfqpoint{2.085403in}{1.566534in}}%
\pgfpathcurveto{\pgfqpoint{2.079579in}{1.560710in}}{\pgfqpoint{2.076307in}{1.552810in}}{\pgfqpoint{2.076307in}{1.544574in}}%
\pgfpathcurveto{\pgfqpoint{2.076307in}{1.536338in}}{\pgfqpoint{2.079579in}{1.528438in}}{\pgfqpoint{2.085403in}{1.522614in}}%
\pgfpathcurveto{\pgfqpoint{2.091227in}{1.516790in}}{\pgfqpoint{2.099127in}{1.513517in}}{\pgfqpoint{2.107363in}{1.513517in}}%
\pgfpathclose%
\pgfusepath{stroke,fill}%
\end{pgfscope}%
\begin{pgfscope}%
\pgfpathrectangle{\pgfqpoint{0.100000in}{0.212622in}}{\pgfqpoint{3.696000in}{3.696000in}}%
\pgfusepath{clip}%
\pgfsetbuttcap%
\pgfsetroundjoin%
\definecolor{currentfill}{rgb}{0.121569,0.466667,0.705882}%
\pgfsetfillcolor{currentfill}%
\pgfsetfillopacity{0.969160}%
\pgfsetlinewidth{1.003750pt}%
\definecolor{currentstroke}{rgb}{0.121569,0.466667,0.705882}%
\pgfsetstrokecolor{currentstroke}%
\pgfsetstrokeopacity{0.969160}%
\pgfsetdash{}{0pt}%
\pgfpathmoveto{\pgfqpoint{2.128525in}{1.499702in}}%
\pgfpathcurveto{\pgfqpoint{2.136761in}{1.499702in}}{\pgfqpoint{2.144661in}{1.502974in}}{\pgfqpoint{2.150485in}{1.508798in}}%
\pgfpathcurveto{\pgfqpoint{2.156309in}{1.514622in}}{\pgfqpoint{2.159582in}{1.522522in}}{\pgfqpoint{2.159582in}{1.530758in}}%
\pgfpathcurveto{\pgfqpoint{2.159582in}{1.538994in}}{\pgfqpoint{2.156309in}{1.546894in}}{\pgfqpoint{2.150485in}{1.552718in}}%
\pgfpathcurveto{\pgfqpoint{2.144661in}{1.558542in}}{\pgfqpoint{2.136761in}{1.561815in}}{\pgfqpoint{2.128525in}{1.561815in}}%
\pgfpathcurveto{\pgfqpoint{2.120289in}{1.561815in}}{\pgfqpoint{2.112389in}{1.558542in}}{\pgfqpoint{2.106565in}{1.552718in}}%
\pgfpathcurveto{\pgfqpoint{2.100741in}{1.546894in}}{\pgfqpoint{2.097469in}{1.538994in}}{\pgfqpoint{2.097469in}{1.530758in}}%
\pgfpathcurveto{\pgfqpoint{2.097469in}{1.522522in}}{\pgfqpoint{2.100741in}{1.514622in}}{\pgfqpoint{2.106565in}{1.508798in}}%
\pgfpathcurveto{\pgfqpoint{2.112389in}{1.502974in}}{\pgfqpoint{2.120289in}{1.499702in}}{\pgfqpoint{2.128525in}{1.499702in}}%
\pgfpathclose%
\pgfusepath{stroke,fill}%
\end{pgfscope}%
\begin{pgfscope}%
\pgfpathrectangle{\pgfqpoint{0.100000in}{0.212622in}}{\pgfqpoint{3.696000in}{3.696000in}}%
\pgfusepath{clip}%
\pgfsetbuttcap%
\pgfsetroundjoin%
\definecolor{currentfill}{rgb}{0.121569,0.466667,0.705882}%
\pgfsetfillcolor{currentfill}%
\pgfsetfillopacity{0.969969}%
\pgfsetlinewidth{1.003750pt}%
\definecolor{currentstroke}{rgb}{0.121569,0.466667,0.705882}%
\pgfsetstrokecolor{currentstroke}%
\pgfsetstrokeopacity{0.969969}%
\pgfsetdash{}{0pt}%
\pgfpathmoveto{\pgfqpoint{2.390982in}{1.357469in}}%
\pgfpathcurveto{\pgfqpoint{2.399219in}{1.357469in}}{\pgfqpoint{2.407119in}{1.360741in}}{\pgfqpoint{2.412943in}{1.366565in}}%
\pgfpathcurveto{\pgfqpoint{2.418766in}{1.372389in}}{\pgfqpoint{2.422039in}{1.380289in}}{\pgfqpoint{2.422039in}{1.388525in}}%
\pgfpathcurveto{\pgfqpoint{2.422039in}{1.396762in}}{\pgfqpoint{2.418766in}{1.404662in}}{\pgfqpoint{2.412943in}{1.410486in}}%
\pgfpathcurveto{\pgfqpoint{2.407119in}{1.416310in}}{\pgfqpoint{2.399219in}{1.419582in}}{\pgfqpoint{2.390982in}{1.419582in}}%
\pgfpathcurveto{\pgfqpoint{2.382746in}{1.419582in}}{\pgfqpoint{2.374846in}{1.416310in}}{\pgfqpoint{2.369022in}{1.410486in}}%
\pgfpathcurveto{\pgfqpoint{2.363198in}{1.404662in}}{\pgfqpoint{2.359926in}{1.396762in}}{\pgfqpoint{2.359926in}{1.388525in}}%
\pgfpathcurveto{\pgfqpoint{2.359926in}{1.380289in}}{\pgfqpoint{2.363198in}{1.372389in}}{\pgfqpoint{2.369022in}{1.366565in}}%
\pgfpathcurveto{\pgfqpoint{2.374846in}{1.360741in}}{\pgfqpoint{2.382746in}{1.357469in}}{\pgfqpoint{2.390982in}{1.357469in}}%
\pgfpathclose%
\pgfusepath{stroke,fill}%
\end{pgfscope}%
\begin{pgfscope}%
\pgfpathrectangle{\pgfqpoint{0.100000in}{0.212622in}}{\pgfqpoint{3.696000in}{3.696000in}}%
\pgfusepath{clip}%
\pgfsetbuttcap%
\pgfsetroundjoin%
\definecolor{currentfill}{rgb}{0.121569,0.466667,0.705882}%
\pgfsetfillcolor{currentfill}%
\pgfsetfillopacity{0.971428}%
\pgfsetlinewidth{1.003750pt}%
\definecolor{currentstroke}{rgb}{0.121569,0.466667,0.705882}%
\pgfsetstrokecolor{currentstroke}%
\pgfsetstrokeopacity{0.971428}%
\pgfsetdash{}{0pt}%
\pgfpathmoveto{\pgfqpoint{2.145462in}{1.487243in}}%
\pgfpathcurveto{\pgfqpoint{2.153698in}{1.487243in}}{\pgfqpoint{2.161598in}{1.490515in}}{\pgfqpoint{2.167422in}{1.496339in}}%
\pgfpathcurveto{\pgfqpoint{2.173246in}{1.502163in}}{\pgfqpoint{2.176518in}{1.510063in}}{\pgfqpoint{2.176518in}{1.518299in}}%
\pgfpathcurveto{\pgfqpoint{2.176518in}{1.526535in}}{\pgfqpoint{2.173246in}{1.534435in}}{\pgfqpoint{2.167422in}{1.540259in}}%
\pgfpathcurveto{\pgfqpoint{2.161598in}{1.546083in}}{\pgfqpoint{2.153698in}{1.549356in}}{\pgfqpoint{2.145462in}{1.549356in}}%
\pgfpathcurveto{\pgfqpoint{2.137226in}{1.549356in}}{\pgfqpoint{2.129326in}{1.546083in}}{\pgfqpoint{2.123502in}{1.540259in}}%
\pgfpathcurveto{\pgfqpoint{2.117678in}{1.534435in}}{\pgfqpoint{2.114405in}{1.526535in}}{\pgfqpoint{2.114405in}{1.518299in}}%
\pgfpathcurveto{\pgfqpoint{2.114405in}{1.510063in}}{\pgfqpoint{2.117678in}{1.502163in}}{\pgfqpoint{2.123502in}{1.496339in}}%
\pgfpathcurveto{\pgfqpoint{2.129326in}{1.490515in}}{\pgfqpoint{2.137226in}{1.487243in}}{\pgfqpoint{2.145462in}{1.487243in}}%
\pgfpathclose%
\pgfusepath{stroke,fill}%
\end{pgfscope}%
\begin{pgfscope}%
\pgfpathrectangle{\pgfqpoint{0.100000in}{0.212622in}}{\pgfqpoint{3.696000in}{3.696000in}}%
\pgfusepath{clip}%
\pgfsetbuttcap%
\pgfsetroundjoin%
\definecolor{currentfill}{rgb}{0.121569,0.466667,0.705882}%
\pgfsetfillcolor{currentfill}%
\pgfsetfillopacity{0.973360}%
\pgfsetlinewidth{1.003750pt}%
\definecolor{currentstroke}{rgb}{0.121569,0.466667,0.705882}%
\pgfsetstrokecolor{currentstroke}%
\pgfsetstrokeopacity{0.973360}%
\pgfsetdash{}{0pt}%
\pgfpathmoveto{\pgfqpoint{2.162849in}{1.475937in}}%
\pgfpathcurveto{\pgfqpoint{2.171085in}{1.475937in}}{\pgfqpoint{2.178985in}{1.479209in}}{\pgfqpoint{2.184809in}{1.485033in}}%
\pgfpathcurveto{\pgfqpoint{2.190633in}{1.490857in}}{\pgfqpoint{2.193905in}{1.498757in}}{\pgfqpoint{2.193905in}{1.506993in}}%
\pgfpathcurveto{\pgfqpoint{2.193905in}{1.515230in}}{\pgfqpoint{2.190633in}{1.523130in}}{\pgfqpoint{2.184809in}{1.528954in}}%
\pgfpathcurveto{\pgfqpoint{2.178985in}{1.534778in}}{\pgfqpoint{2.171085in}{1.538050in}}{\pgfqpoint{2.162849in}{1.538050in}}%
\pgfpathcurveto{\pgfqpoint{2.154613in}{1.538050in}}{\pgfqpoint{2.146712in}{1.534778in}}{\pgfqpoint{2.140889in}{1.528954in}}%
\pgfpathcurveto{\pgfqpoint{2.135065in}{1.523130in}}{\pgfqpoint{2.131792in}{1.515230in}}{\pgfqpoint{2.131792in}{1.506993in}}%
\pgfpathcurveto{\pgfqpoint{2.131792in}{1.498757in}}{\pgfqpoint{2.135065in}{1.490857in}}{\pgfqpoint{2.140889in}{1.485033in}}%
\pgfpathcurveto{\pgfqpoint{2.146712in}{1.479209in}}{\pgfqpoint{2.154613in}{1.475937in}}{\pgfqpoint{2.162849in}{1.475937in}}%
\pgfpathclose%
\pgfusepath{stroke,fill}%
\end{pgfscope}%
\begin{pgfscope}%
\pgfpathrectangle{\pgfqpoint{0.100000in}{0.212622in}}{\pgfqpoint{3.696000in}{3.696000in}}%
\pgfusepath{clip}%
\pgfsetbuttcap%
\pgfsetroundjoin%
\definecolor{currentfill}{rgb}{0.121569,0.466667,0.705882}%
\pgfsetfillcolor{currentfill}%
\pgfsetfillopacity{0.974571}%
\pgfsetlinewidth{1.003750pt}%
\definecolor{currentstroke}{rgb}{0.121569,0.466667,0.705882}%
\pgfsetstrokecolor{currentstroke}%
\pgfsetstrokeopacity{0.974571}%
\pgfsetdash{}{0pt}%
\pgfpathmoveto{\pgfqpoint{2.393770in}{1.352182in}}%
\pgfpathcurveto{\pgfqpoint{2.402006in}{1.352182in}}{\pgfqpoint{2.409906in}{1.355454in}}{\pgfqpoint{2.415730in}{1.361278in}}%
\pgfpathcurveto{\pgfqpoint{2.421554in}{1.367102in}}{\pgfqpoint{2.424826in}{1.375002in}}{\pgfqpoint{2.424826in}{1.383239in}}%
\pgfpathcurveto{\pgfqpoint{2.424826in}{1.391475in}}{\pgfqpoint{2.421554in}{1.399375in}}{\pgfqpoint{2.415730in}{1.405199in}}%
\pgfpathcurveto{\pgfqpoint{2.409906in}{1.411023in}}{\pgfqpoint{2.402006in}{1.414295in}}{\pgfqpoint{2.393770in}{1.414295in}}%
\pgfpathcurveto{\pgfqpoint{2.385534in}{1.414295in}}{\pgfqpoint{2.377634in}{1.411023in}}{\pgfqpoint{2.371810in}{1.405199in}}%
\pgfpathcurveto{\pgfqpoint{2.365986in}{1.399375in}}{\pgfqpoint{2.362713in}{1.391475in}}{\pgfqpoint{2.362713in}{1.383239in}}%
\pgfpathcurveto{\pgfqpoint{2.362713in}{1.375002in}}{\pgfqpoint{2.365986in}{1.367102in}}{\pgfqpoint{2.371810in}{1.361278in}}%
\pgfpathcurveto{\pgfqpoint{2.377634in}{1.355454in}}{\pgfqpoint{2.385534in}{1.352182in}}{\pgfqpoint{2.393770in}{1.352182in}}%
\pgfpathclose%
\pgfusepath{stroke,fill}%
\end{pgfscope}%
\begin{pgfscope}%
\pgfpathrectangle{\pgfqpoint{0.100000in}{0.212622in}}{\pgfqpoint{3.696000in}{3.696000in}}%
\pgfusepath{clip}%
\pgfsetbuttcap%
\pgfsetroundjoin%
\definecolor{currentfill}{rgb}{0.121569,0.466667,0.705882}%
\pgfsetfillcolor{currentfill}%
\pgfsetfillopacity{0.975599}%
\pgfsetlinewidth{1.003750pt}%
\definecolor{currentstroke}{rgb}{0.121569,0.466667,0.705882}%
\pgfsetstrokecolor{currentstroke}%
\pgfsetstrokeopacity{0.975599}%
\pgfsetdash{}{0pt}%
\pgfpathmoveto{\pgfqpoint{2.178016in}{1.468598in}}%
\pgfpathcurveto{\pgfqpoint{2.186252in}{1.468598in}}{\pgfqpoint{2.194152in}{1.471871in}}{\pgfqpoint{2.199976in}{1.477695in}}%
\pgfpathcurveto{\pgfqpoint{2.205800in}{1.483519in}}{\pgfqpoint{2.209072in}{1.491419in}}{\pgfqpoint{2.209072in}{1.499655in}}%
\pgfpathcurveto{\pgfqpoint{2.209072in}{1.507891in}}{\pgfqpoint{2.205800in}{1.515791in}}{\pgfqpoint{2.199976in}{1.521615in}}%
\pgfpathcurveto{\pgfqpoint{2.194152in}{1.527439in}}{\pgfqpoint{2.186252in}{1.530711in}}{\pgfqpoint{2.178016in}{1.530711in}}%
\pgfpathcurveto{\pgfqpoint{2.169779in}{1.530711in}}{\pgfqpoint{2.161879in}{1.527439in}}{\pgfqpoint{2.156055in}{1.521615in}}%
\pgfpathcurveto{\pgfqpoint{2.150231in}{1.515791in}}{\pgfqpoint{2.146959in}{1.507891in}}{\pgfqpoint{2.146959in}{1.499655in}}%
\pgfpathcurveto{\pgfqpoint{2.146959in}{1.491419in}}{\pgfqpoint{2.150231in}{1.483519in}}{\pgfqpoint{2.156055in}{1.477695in}}%
\pgfpathcurveto{\pgfqpoint{2.161879in}{1.471871in}}{\pgfqpoint{2.169779in}{1.468598in}}{\pgfqpoint{2.178016in}{1.468598in}}%
\pgfpathclose%
\pgfusepath{stroke,fill}%
\end{pgfscope}%
\begin{pgfscope}%
\pgfpathrectangle{\pgfqpoint{0.100000in}{0.212622in}}{\pgfqpoint{3.696000in}{3.696000in}}%
\pgfusepath{clip}%
\pgfsetbuttcap%
\pgfsetroundjoin%
\definecolor{currentfill}{rgb}{0.121569,0.466667,0.705882}%
\pgfsetfillcolor{currentfill}%
\pgfsetfillopacity{0.977403}%
\pgfsetlinewidth{1.003750pt}%
\definecolor{currentstroke}{rgb}{0.121569,0.466667,0.705882}%
\pgfsetstrokecolor{currentstroke}%
\pgfsetstrokeopacity{0.977403}%
\pgfsetdash{}{0pt}%
\pgfpathmoveto{\pgfqpoint{2.192424in}{1.458667in}}%
\pgfpathcurveto{\pgfqpoint{2.200660in}{1.458667in}}{\pgfqpoint{2.208560in}{1.461939in}}{\pgfqpoint{2.214384in}{1.467763in}}%
\pgfpathcurveto{\pgfqpoint{2.220208in}{1.473587in}}{\pgfqpoint{2.223480in}{1.481487in}}{\pgfqpoint{2.223480in}{1.489724in}}%
\pgfpathcurveto{\pgfqpoint{2.223480in}{1.497960in}}{\pgfqpoint{2.220208in}{1.505860in}}{\pgfqpoint{2.214384in}{1.511684in}}%
\pgfpathcurveto{\pgfqpoint{2.208560in}{1.517508in}}{\pgfqpoint{2.200660in}{1.520780in}}{\pgfqpoint{2.192424in}{1.520780in}}%
\pgfpathcurveto{\pgfqpoint{2.184187in}{1.520780in}}{\pgfqpoint{2.176287in}{1.517508in}}{\pgfqpoint{2.170463in}{1.511684in}}%
\pgfpathcurveto{\pgfqpoint{2.164639in}{1.505860in}}{\pgfqpoint{2.161367in}{1.497960in}}{\pgfqpoint{2.161367in}{1.489724in}}%
\pgfpathcurveto{\pgfqpoint{2.161367in}{1.481487in}}{\pgfqpoint{2.164639in}{1.473587in}}{\pgfqpoint{2.170463in}{1.467763in}}%
\pgfpathcurveto{\pgfqpoint{2.176287in}{1.461939in}}{\pgfqpoint{2.184187in}{1.458667in}}{\pgfqpoint{2.192424in}{1.458667in}}%
\pgfpathclose%
\pgfusepath{stroke,fill}%
\end{pgfscope}%
\begin{pgfscope}%
\pgfpathrectangle{\pgfqpoint{0.100000in}{0.212622in}}{\pgfqpoint{3.696000in}{3.696000in}}%
\pgfusepath{clip}%
\pgfsetbuttcap%
\pgfsetroundjoin%
\definecolor{currentfill}{rgb}{0.121569,0.466667,0.705882}%
\pgfsetfillcolor{currentfill}%
\pgfsetfillopacity{0.978744}%
\pgfsetlinewidth{1.003750pt}%
\definecolor{currentstroke}{rgb}{0.121569,0.466667,0.705882}%
\pgfsetstrokecolor{currentstroke}%
\pgfsetstrokeopacity{0.978744}%
\pgfsetdash{}{0pt}%
\pgfpathmoveto{\pgfqpoint{2.204667in}{1.451614in}}%
\pgfpathcurveto{\pgfqpoint{2.212903in}{1.451614in}}{\pgfqpoint{2.220804in}{1.454887in}}{\pgfqpoint{2.226627in}{1.460711in}}%
\pgfpathcurveto{\pgfqpoint{2.232451in}{1.466535in}}{\pgfqpoint{2.235724in}{1.474435in}}{\pgfqpoint{2.235724in}{1.482671in}}%
\pgfpathcurveto{\pgfqpoint{2.235724in}{1.490907in}}{\pgfqpoint{2.232451in}{1.498807in}}{\pgfqpoint{2.226627in}{1.504631in}}%
\pgfpathcurveto{\pgfqpoint{2.220804in}{1.510455in}}{\pgfqpoint{2.212903in}{1.513727in}}{\pgfqpoint{2.204667in}{1.513727in}}%
\pgfpathcurveto{\pgfqpoint{2.196431in}{1.513727in}}{\pgfqpoint{2.188531in}{1.510455in}}{\pgfqpoint{2.182707in}{1.504631in}}%
\pgfpathcurveto{\pgfqpoint{2.176883in}{1.498807in}}{\pgfqpoint{2.173611in}{1.490907in}}{\pgfqpoint{2.173611in}{1.482671in}}%
\pgfpathcurveto{\pgfqpoint{2.173611in}{1.474435in}}{\pgfqpoint{2.176883in}{1.466535in}}{\pgfqpoint{2.182707in}{1.460711in}}%
\pgfpathcurveto{\pgfqpoint{2.188531in}{1.454887in}}{\pgfqpoint{2.196431in}{1.451614in}}{\pgfqpoint{2.204667in}{1.451614in}}%
\pgfpathclose%
\pgfusepath{stroke,fill}%
\end{pgfscope}%
\begin{pgfscope}%
\pgfpathrectangle{\pgfqpoint{0.100000in}{0.212622in}}{\pgfqpoint{3.696000in}{3.696000in}}%
\pgfusepath{clip}%
\pgfsetbuttcap%
\pgfsetroundjoin%
\definecolor{currentfill}{rgb}{0.121569,0.466667,0.705882}%
\pgfsetfillcolor{currentfill}%
\pgfsetfillopacity{0.979203}%
\pgfsetlinewidth{1.003750pt}%
\definecolor{currentstroke}{rgb}{0.121569,0.466667,0.705882}%
\pgfsetstrokecolor{currentstroke}%
\pgfsetstrokeopacity{0.979203}%
\pgfsetdash{}{0pt}%
\pgfpathmoveto{\pgfqpoint{2.396824in}{1.345580in}}%
\pgfpathcurveto{\pgfqpoint{2.405060in}{1.345580in}}{\pgfqpoint{2.412960in}{1.348852in}}{\pgfqpoint{2.418784in}{1.354676in}}%
\pgfpathcurveto{\pgfqpoint{2.424608in}{1.360500in}}{\pgfqpoint{2.427880in}{1.368400in}}{\pgfqpoint{2.427880in}{1.376636in}}%
\pgfpathcurveto{\pgfqpoint{2.427880in}{1.384872in}}{\pgfqpoint{2.424608in}{1.392772in}}{\pgfqpoint{2.418784in}{1.398596in}}%
\pgfpathcurveto{\pgfqpoint{2.412960in}{1.404420in}}{\pgfqpoint{2.405060in}{1.407693in}}{\pgfqpoint{2.396824in}{1.407693in}}%
\pgfpathcurveto{\pgfqpoint{2.388587in}{1.407693in}}{\pgfqpoint{2.380687in}{1.404420in}}{\pgfqpoint{2.374863in}{1.398596in}}%
\pgfpathcurveto{\pgfqpoint{2.369039in}{1.392772in}}{\pgfqpoint{2.365767in}{1.384872in}}{\pgfqpoint{2.365767in}{1.376636in}}%
\pgfpathcurveto{\pgfqpoint{2.365767in}{1.368400in}}{\pgfqpoint{2.369039in}{1.360500in}}{\pgfqpoint{2.374863in}{1.354676in}}%
\pgfpathcurveto{\pgfqpoint{2.380687in}{1.348852in}}{\pgfqpoint{2.388587in}{1.345580in}}{\pgfqpoint{2.396824in}{1.345580in}}%
\pgfpathclose%
\pgfusepath{stroke,fill}%
\end{pgfscope}%
\begin{pgfscope}%
\pgfpathrectangle{\pgfqpoint{0.100000in}{0.212622in}}{\pgfqpoint{3.696000in}{3.696000in}}%
\pgfusepath{clip}%
\pgfsetbuttcap%
\pgfsetroundjoin%
\definecolor{currentfill}{rgb}{0.121569,0.466667,0.705882}%
\pgfsetfillcolor{currentfill}%
\pgfsetfillopacity{0.979867}%
\pgfsetlinewidth{1.003750pt}%
\definecolor{currentstroke}{rgb}{0.121569,0.466667,0.705882}%
\pgfsetstrokecolor{currentstroke}%
\pgfsetstrokeopacity{0.979867}%
\pgfsetdash{}{0pt}%
\pgfpathmoveto{\pgfqpoint{2.214057in}{1.442797in}}%
\pgfpathcurveto{\pgfqpoint{2.222293in}{1.442797in}}{\pgfqpoint{2.230193in}{1.446069in}}{\pgfqpoint{2.236017in}{1.451893in}}%
\pgfpathcurveto{\pgfqpoint{2.241841in}{1.457717in}}{\pgfqpoint{2.245113in}{1.465617in}}{\pgfqpoint{2.245113in}{1.473854in}}%
\pgfpathcurveto{\pgfqpoint{2.245113in}{1.482090in}}{\pgfqpoint{2.241841in}{1.489990in}}{\pgfqpoint{2.236017in}{1.495814in}}%
\pgfpathcurveto{\pgfqpoint{2.230193in}{1.501638in}}{\pgfqpoint{2.222293in}{1.504910in}}{\pgfqpoint{2.214057in}{1.504910in}}%
\pgfpathcurveto{\pgfqpoint{2.205820in}{1.504910in}}{\pgfqpoint{2.197920in}{1.501638in}}{\pgfqpoint{2.192096in}{1.495814in}}%
\pgfpathcurveto{\pgfqpoint{2.186272in}{1.489990in}}{\pgfqpoint{2.183000in}{1.482090in}}{\pgfqpoint{2.183000in}{1.473854in}}%
\pgfpathcurveto{\pgfqpoint{2.183000in}{1.465617in}}{\pgfqpoint{2.186272in}{1.457717in}}{\pgfqpoint{2.192096in}{1.451893in}}%
\pgfpathcurveto{\pgfqpoint{2.197920in}{1.446069in}}{\pgfqpoint{2.205820in}{1.442797in}}{\pgfqpoint{2.214057in}{1.442797in}}%
\pgfpathclose%
\pgfusepath{stroke,fill}%
\end{pgfscope}%
\begin{pgfscope}%
\pgfpathrectangle{\pgfqpoint{0.100000in}{0.212622in}}{\pgfqpoint{3.696000in}{3.696000in}}%
\pgfusepath{clip}%
\pgfsetbuttcap%
\pgfsetroundjoin%
\definecolor{currentfill}{rgb}{0.121569,0.466667,0.705882}%
\pgfsetfillcolor{currentfill}%
\pgfsetfillopacity{0.981867}%
\pgfsetlinewidth{1.003750pt}%
\definecolor{currentstroke}{rgb}{0.121569,0.466667,0.705882}%
\pgfsetstrokecolor{currentstroke}%
\pgfsetstrokeopacity{0.981867}%
\pgfsetdash{}{0pt}%
\pgfpathmoveto{\pgfqpoint{2.231541in}{1.427555in}}%
\pgfpathcurveto{\pgfqpoint{2.239778in}{1.427555in}}{\pgfqpoint{2.247678in}{1.430827in}}{\pgfqpoint{2.253502in}{1.436651in}}%
\pgfpathcurveto{\pgfqpoint{2.259326in}{1.442475in}}{\pgfqpoint{2.262598in}{1.450375in}}{\pgfqpoint{2.262598in}{1.458612in}}%
\pgfpathcurveto{\pgfqpoint{2.262598in}{1.466848in}}{\pgfqpoint{2.259326in}{1.474748in}}{\pgfqpoint{2.253502in}{1.480572in}}%
\pgfpathcurveto{\pgfqpoint{2.247678in}{1.486396in}}{\pgfqpoint{2.239778in}{1.489668in}}{\pgfqpoint{2.231541in}{1.489668in}}%
\pgfpathcurveto{\pgfqpoint{2.223305in}{1.489668in}}{\pgfqpoint{2.215405in}{1.486396in}}{\pgfqpoint{2.209581in}{1.480572in}}%
\pgfpathcurveto{\pgfqpoint{2.203757in}{1.474748in}}{\pgfqpoint{2.200485in}{1.466848in}}{\pgfqpoint{2.200485in}{1.458612in}}%
\pgfpathcurveto{\pgfqpoint{2.200485in}{1.450375in}}{\pgfqpoint{2.203757in}{1.442475in}}{\pgfqpoint{2.209581in}{1.436651in}}%
\pgfpathcurveto{\pgfqpoint{2.215405in}{1.430827in}}{\pgfqpoint{2.223305in}{1.427555in}}{\pgfqpoint{2.231541in}{1.427555in}}%
\pgfpathclose%
\pgfusepath{stroke,fill}%
\end{pgfscope}%
\begin{pgfscope}%
\pgfpathrectangle{\pgfqpoint{0.100000in}{0.212622in}}{\pgfqpoint{3.696000in}{3.696000in}}%
\pgfusepath{clip}%
\pgfsetbuttcap%
\pgfsetroundjoin%
\definecolor{currentfill}{rgb}{0.121569,0.466667,0.705882}%
\pgfsetfillcolor{currentfill}%
\pgfsetfillopacity{0.983282}%
\pgfsetlinewidth{1.003750pt}%
\definecolor{currentstroke}{rgb}{0.121569,0.466667,0.705882}%
\pgfsetstrokecolor{currentstroke}%
\pgfsetstrokeopacity{0.983282}%
\pgfsetdash{}{0pt}%
\pgfpathmoveto{\pgfqpoint{2.249412in}{1.415770in}}%
\pgfpathcurveto{\pgfqpoint{2.257649in}{1.415770in}}{\pgfqpoint{2.265549in}{1.419043in}}{\pgfqpoint{2.271373in}{1.424866in}}%
\pgfpathcurveto{\pgfqpoint{2.277197in}{1.430690in}}{\pgfqpoint{2.280469in}{1.438590in}}{\pgfqpoint{2.280469in}{1.446827in}}%
\pgfpathcurveto{\pgfqpoint{2.280469in}{1.455063in}}{\pgfqpoint{2.277197in}{1.462963in}}{\pgfqpoint{2.271373in}{1.468787in}}%
\pgfpathcurveto{\pgfqpoint{2.265549in}{1.474611in}}{\pgfqpoint{2.257649in}{1.477883in}}{\pgfqpoint{2.249412in}{1.477883in}}%
\pgfpathcurveto{\pgfqpoint{2.241176in}{1.477883in}}{\pgfqpoint{2.233276in}{1.474611in}}{\pgfqpoint{2.227452in}{1.468787in}}%
\pgfpathcurveto{\pgfqpoint{2.221628in}{1.462963in}}{\pgfqpoint{2.218356in}{1.455063in}}{\pgfqpoint{2.218356in}{1.446827in}}%
\pgfpathcurveto{\pgfqpoint{2.218356in}{1.438590in}}{\pgfqpoint{2.221628in}{1.430690in}}{\pgfqpoint{2.227452in}{1.424866in}}%
\pgfpathcurveto{\pgfqpoint{2.233276in}{1.419043in}}{\pgfqpoint{2.241176in}{1.415770in}}{\pgfqpoint{2.249412in}{1.415770in}}%
\pgfpathclose%
\pgfusepath{stroke,fill}%
\end{pgfscope}%
\begin{pgfscope}%
\pgfpathrectangle{\pgfqpoint{0.100000in}{0.212622in}}{\pgfqpoint{3.696000in}{3.696000in}}%
\pgfusepath{clip}%
\pgfsetbuttcap%
\pgfsetroundjoin%
\definecolor{currentfill}{rgb}{0.121569,0.466667,0.705882}%
\pgfsetfillcolor{currentfill}%
\pgfsetfillopacity{0.984400}%
\pgfsetlinewidth{1.003750pt}%
\definecolor{currentstroke}{rgb}{0.121569,0.466667,0.705882}%
\pgfsetstrokecolor{currentstroke}%
\pgfsetstrokeopacity{0.984400}%
\pgfsetdash{}{0pt}%
\pgfpathmoveto{\pgfqpoint{2.400853in}{1.340714in}}%
\pgfpathcurveto{\pgfqpoint{2.409090in}{1.340714in}}{\pgfqpoint{2.416990in}{1.343986in}}{\pgfqpoint{2.422814in}{1.349810in}}%
\pgfpathcurveto{\pgfqpoint{2.428638in}{1.355634in}}{\pgfqpoint{2.431910in}{1.363534in}}{\pgfqpoint{2.431910in}{1.371771in}}%
\pgfpathcurveto{\pgfqpoint{2.431910in}{1.380007in}}{\pgfqpoint{2.428638in}{1.387907in}}{\pgfqpoint{2.422814in}{1.393731in}}%
\pgfpathcurveto{\pgfqpoint{2.416990in}{1.399555in}}{\pgfqpoint{2.409090in}{1.402827in}}{\pgfqpoint{2.400853in}{1.402827in}}%
\pgfpathcurveto{\pgfqpoint{2.392617in}{1.402827in}}{\pgfqpoint{2.384717in}{1.399555in}}{\pgfqpoint{2.378893in}{1.393731in}}%
\pgfpathcurveto{\pgfqpoint{2.373069in}{1.387907in}}{\pgfqpoint{2.369797in}{1.380007in}}{\pgfqpoint{2.369797in}{1.371771in}}%
\pgfpathcurveto{\pgfqpoint{2.369797in}{1.363534in}}{\pgfqpoint{2.373069in}{1.355634in}}{\pgfqpoint{2.378893in}{1.349810in}}%
\pgfpathcurveto{\pgfqpoint{2.384717in}{1.343986in}}{\pgfqpoint{2.392617in}{1.340714in}}{\pgfqpoint{2.400853in}{1.340714in}}%
\pgfpathclose%
\pgfusepath{stroke,fill}%
\end{pgfscope}%
\begin{pgfscope}%
\pgfpathrectangle{\pgfqpoint{0.100000in}{0.212622in}}{\pgfqpoint{3.696000in}{3.696000in}}%
\pgfusepath{clip}%
\pgfsetbuttcap%
\pgfsetroundjoin%
\definecolor{currentfill}{rgb}{0.121569,0.466667,0.705882}%
\pgfsetfillcolor{currentfill}%
\pgfsetfillopacity{0.985856}%
\pgfsetlinewidth{1.003750pt}%
\definecolor{currentstroke}{rgb}{0.121569,0.466667,0.705882}%
\pgfsetstrokecolor{currentstroke}%
\pgfsetstrokeopacity{0.985856}%
\pgfsetdash{}{0pt}%
\pgfpathmoveto{\pgfqpoint{2.265430in}{1.409540in}}%
\pgfpathcurveto{\pgfqpoint{2.273666in}{1.409540in}}{\pgfqpoint{2.281566in}{1.412812in}}{\pgfqpoint{2.287390in}{1.418636in}}%
\pgfpathcurveto{\pgfqpoint{2.293214in}{1.424460in}}{\pgfqpoint{2.296486in}{1.432360in}}{\pgfqpoint{2.296486in}{1.440597in}}%
\pgfpathcurveto{\pgfqpoint{2.296486in}{1.448833in}}{\pgfqpoint{2.293214in}{1.456733in}}{\pgfqpoint{2.287390in}{1.462557in}}%
\pgfpathcurveto{\pgfqpoint{2.281566in}{1.468381in}}{\pgfqpoint{2.273666in}{1.471653in}}{\pgfqpoint{2.265430in}{1.471653in}}%
\pgfpathcurveto{\pgfqpoint{2.257193in}{1.471653in}}{\pgfqpoint{2.249293in}{1.468381in}}{\pgfqpoint{2.243469in}{1.462557in}}%
\pgfpathcurveto{\pgfqpoint{2.237645in}{1.456733in}}{\pgfqpoint{2.234373in}{1.448833in}}{\pgfqpoint{2.234373in}{1.440597in}}%
\pgfpathcurveto{\pgfqpoint{2.234373in}{1.432360in}}{\pgfqpoint{2.237645in}{1.424460in}}{\pgfqpoint{2.243469in}{1.418636in}}%
\pgfpathcurveto{\pgfqpoint{2.249293in}{1.412812in}}{\pgfqpoint{2.257193in}{1.409540in}}{\pgfqpoint{2.265430in}{1.409540in}}%
\pgfpathclose%
\pgfusepath{stroke,fill}%
\end{pgfscope}%
\begin{pgfscope}%
\pgfpathrectangle{\pgfqpoint{0.100000in}{0.212622in}}{\pgfqpoint{3.696000in}{3.696000in}}%
\pgfusepath{clip}%
\pgfsetbuttcap%
\pgfsetroundjoin%
\definecolor{currentfill}{rgb}{0.121569,0.466667,0.705882}%
\pgfsetfillcolor{currentfill}%
\pgfsetfillopacity{0.987609}%
\pgfsetlinewidth{1.003750pt}%
\definecolor{currentstroke}{rgb}{0.121569,0.466667,0.705882}%
\pgfsetstrokecolor{currentstroke}%
\pgfsetstrokeopacity{0.987609}%
\pgfsetdash{}{0pt}%
\pgfpathmoveto{\pgfqpoint{2.278934in}{1.401680in}}%
\pgfpathcurveto{\pgfqpoint{2.287170in}{1.401680in}}{\pgfqpoint{2.295070in}{1.404952in}}{\pgfqpoint{2.300894in}{1.410776in}}%
\pgfpathcurveto{\pgfqpoint{2.306718in}{1.416600in}}{\pgfqpoint{2.309991in}{1.424500in}}{\pgfqpoint{2.309991in}{1.432736in}}%
\pgfpathcurveto{\pgfqpoint{2.309991in}{1.440973in}}{\pgfqpoint{2.306718in}{1.448873in}}{\pgfqpoint{2.300894in}{1.454697in}}%
\pgfpathcurveto{\pgfqpoint{2.295070in}{1.460521in}}{\pgfqpoint{2.287170in}{1.463793in}}{\pgfqpoint{2.278934in}{1.463793in}}%
\pgfpathcurveto{\pgfqpoint{2.270698in}{1.463793in}}{\pgfqpoint{2.262798in}{1.460521in}}{\pgfqpoint{2.256974in}{1.454697in}}%
\pgfpathcurveto{\pgfqpoint{2.251150in}{1.448873in}}{\pgfqpoint{2.247878in}{1.440973in}}{\pgfqpoint{2.247878in}{1.432736in}}%
\pgfpathcurveto{\pgfqpoint{2.247878in}{1.424500in}}{\pgfqpoint{2.251150in}{1.416600in}}{\pgfqpoint{2.256974in}{1.410776in}}%
\pgfpathcurveto{\pgfqpoint{2.262798in}{1.404952in}}{\pgfqpoint{2.270698in}{1.401680in}}{\pgfqpoint{2.278934in}{1.401680in}}%
\pgfpathclose%
\pgfusepath{stroke,fill}%
\end{pgfscope}%
\begin{pgfscope}%
\pgfpathrectangle{\pgfqpoint{0.100000in}{0.212622in}}{\pgfqpoint{3.696000in}{3.696000in}}%
\pgfusepath{clip}%
\pgfsetbuttcap%
\pgfsetroundjoin%
\definecolor{currentfill}{rgb}{0.121569,0.466667,0.705882}%
\pgfsetfillcolor{currentfill}%
\pgfsetfillopacity{0.988636}%
\pgfsetlinewidth{1.003750pt}%
\definecolor{currentstroke}{rgb}{0.121569,0.466667,0.705882}%
\pgfsetstrokecolor{currentstroke}%
\pgfsetstrokeopacity{0.988636}%
\pgfsetdash{}{0pt}%
\pgfpathmoveto{\pgfqpoint{2.290850in}{1.390279in}}%
\pgfpathcurveto{\pgfqpoint{2.299086in}{1.390279in}}{\pgfqpoint{2.306986in}{1.393551in}}{\pgfqpoint{2.312810in}{1.399375in}}%
\pgfpathcurveto{\pgfqpoint{2.318634in}{1.405199in}}{\pgfqpoint{2.321906in}{1.413099in}}{\pgfqpoint{2.321906in}{1.421335in}}%
\pgfpathcurveto{\pgfqpoint{2.321906in}{1.429572in}}{\pgfqpoint{2.318634in}{1.437472in}}{\pgfqpoint{2.312810in}{1.443296in}}%
\pgfpathcurveto{\pgfqpoint{2.306986in}{1.449120in}}{\pgfqpoint{2.299086in}{1.452392in}}{\pgfqpoint{2.290850in}{1.452392in}}%
\pgfpathcurveto{\pgfqpoint{2.282613in}{1.452392in}}{\pgfqpoint{2.274713in}{1.449120in}}{\pgfqpoint{2.268889in}{1.443296in}}%
\pgfpathcurveto{\pgfqpoint{2.263065in}{1.437472in}}{\pgfqpoint{2.259793in}{1.429572in}}{\pgfqpoint{2.259793in}{1.421335in}}%
\pgfpathcurveto{\pgfqpoint{2.259793in}{1.413099in}}{\pgfqpoint{2.263065in}{1.405199in}}{\pgfqpoint{2.268889in}{1.399375in}}%
\pgfpathcurveto{\pgfqpoint{2.274713in}{1.393551in}}{\pgfqpoint{2.282613in}{1.390279in}}{\pgfqpoint{2.290850in}{1.390279in}}%
\pgfpathclose%
\pgfusepath{stroke,fill}%
\end{pgfscope}%
\begin{pgfscope}%
\pgfpathrectangle{\pgfqpoint{0.100000in}{0.212622in}}{\pgfqpoint{3.696000in}{3.696000in}}%
\pgfusepath{clip}%
\pgfsetbuttcap%
\pgfsetroundjoin%
\definecolor{currentfill}{rgb}{0.121569,0.466667,0.705882}%
\pgfsetfillcolor{currentfill}%
\pgfsetfillopacity{0.989904}%
\pgfsetlinewidth{1.003750pt}%
\definecolor{currentstroke}{rgb}{0.121569,0.466667,0.705882}%
\pgfsetstrokecolor{currentstroke}%
\pgfsetstrokeopacity{0.989904}%
\pgfsetdash{}{0pt}%
\pgfpathmoveto{\pgfqpoint{2.404450in}{1.335339in}}%
\pgfpathcurveto{\pgfqpoint{2.412686in}{1.335339in}}{\pgfqpoint{2.420586in}{1.338611in}}{\pgfqpoint{2.426410in}{1.344435in}}%
\pgfpathcurveto{\pgfqpoint{2.432234in}{1.350259in}}{\pgfqpoint{2.435507in}{1.358159in}}{\pgfqpoint{2.435507in}{1.366395in}}%
\pgfpathcurveto{\pgfqpoint{2.435507in}{1.374631in}}{\pgfqpoint{2.432234in}{1.382531in}}{\pgfqpoint{2.426410in}{1.388355in}}%
\pgfpathcurveto{\pgfqpoint{2.420586in}{1.394179in}}{\pgfqpoint{2.412686in}{1.397452in}}{\pgfqpoint{2.404450in}{1.397452in}}%
\pgfpathcurveto{\pgfqpoint{2.396214in}{1.397452in}}{\pgfqpoint{2.388314in}{1.394179in}}{\pgfqpoint{2.382490in}{1.388355in}}%
\pgfpathcurveto{\pgfqpoint{2.376666in}{1.382531in}}{\pgfqpoint{2.373394in}{1.374631in}}{\pgfqpoint{2.373394in}{1.366395in}}%
\pgfpathcurveto{\pgfqpoint{2.373394in}{1.358159in}}{\pgfqpoint{2.376666in}{1.350259in}}{\pgfqpoint{2.382490in}{1.344435in}}%
\pgfpathcurveto{\pgfqpoint{2.388314in}{1.338611in}}{\pgfqpoint{2.396214in}{1.335339in}}{\pgfqpoint{2.404450in}{1.335339in}}%
\pgfpathclose%
\pgfusepath{stroke,fill}%
\end{pgfscope}%
\begin{pgfscope}%
\pgfpathrectangle{\pgfqpoint{0.100000in}{0.212622in}}{\pgfqpoint{3.696000in}{3.696000in}}%
\pgfusepath{clip}%
\pgfsetbuttcap%
\pgfsetroundjoin%
\definecolor{currentfill}{rgb}{0.121569,0.466667,0.705882}%
\pgfsetfillcolor{currentfill}%
\pgfsetfillopacity{0.989984}%
\pgfsetlinewidth{1.003750pt}%
\definecolor{currentstroke}{rgb}{0.121569,0.466667,0.705882}%
\pgfsetstrokecolor{currentstroke}%
\pgfsetstrokeopacity{0.989984}%
\pgfsetdash{}{0pt}%
\pgfpathmoveto{\pgfqpoint{2.302294in}{1.383857in}}%
\pgfpathcurveto{\pgfqpoint{2.310530in}{1.383857in}}{\pgfqpoint{2.318430in}{1.387130in}}{\pgfqpoint{2.324254in}{1.392954in}}%
\pgfpathcurveto{\pgfqpoint{2.330078in}{1.398778in}}{\pgfqpoint{2.333350in}{1.406678in}}{\pgfqpoint{2.333350in}{1.414914in}}%
\pgfpathcurveto{\pgfqpoint{2.333350in}{1.423150in}}{\pgfqpoint{2.330078in}{1.431050in}}{\pgfqpoint{2.324254in}{1.436874in}}%
\pgfpathcurveto{\pgfqpoint{2.318430in}{1.442698in}}{\pgfqpoint{2.310530in}{1.445970in}}{\pgfqpoint{2.302294in}{1.445970in}}%
\pgfpathcurveto{\pgfqpoint{2.294057in}{1.445970in}}{\pgfqpoint{2.286157in}{1.442698in}}{\pgfqpoint{2.280333in}{1.436874in}}%
\pgfpathcurveto{\pgfqpoint{2.274510in}{1.431050in}}{\pgfqpoint{2.271237in}{1.423150in}}{\pgfqpoint{2.271237in}{1.414914in}}%
\pgfpathcurveto{\pgfqpoint{2.271237in}{1.406678in}}{\pgfqpoint{2.274510in}{1.398778in}}{\pgfqpoint{2.280333in}{1.392954in}}%
\pgfpathcurveto{\pgfqpoint{2.286157in}{1.387130in}}{\pgfqpoint{2.294057in}{1.383857in}}{\pgfqpoint{2.302294in}{1.383857in}}%
\pgfpathclose%
\pgfusepath{stroke,fill}%
\end{pgfscope}%
\begin{pgfscope}%
\pgfpathrectangle{\pgfqpoint{0.100000in}{0.212622in}}{\pgfqpoint{3.696000in}{3.696000in}}%
\pgfusepath{clip}%
\pgfsetbuttcap%
\pgfsetroundjoin%
\definecolor{currentfill}{rgb}{0.121569,0.466667,0.705882}%
\pgfsetfillcolor{currentfill}%
\pgfsetfillopacity{0.991218}%
\pgfsetlinewidth{1.003750pt}%
\definecolor{currentstroke}{rgb}{0.121569,0.466667,0.705882}%
\pgfsetstrokecolor{currentstroke}%
\pgfsetstrokeopacity{0.991218}%
\pgfsetdash{}{0pt}%
\pgfpathmoveto{\pgfqpoint{2.312040in}{1.378408in}}%
\pgfpathcurveto{\pgfqpoint{2.320276in}{1.378408in}}{\pgfqpoint{2.328176in}{1.381681in}}{\pgfqpoint{2.334000in}{1.387505in}}%
\pgfpathcurveto{\pgfqpoint{2.339824in}{1.393329in}}{\pgfqpoint{2.343096in}{1.401229in}}{\pgfqpoint{2.343096in}{1.409465in}}%
\pgfpathcurveto{\pgfqpoint{2.343096in}{1.417701in}}{\pgfqpoint{2.339824in}{1.425601in}}{\pgfqpoint{2.334000in}{1.431425in}}%
\pgfpathcurveto{\pgfqpoint{2.328176in}{1.437249in}}{\pgfqpoint{2.320276in}{1.440521in}}{\pgfqpoint{2.312040in}{1.440521in}}%
\pgfpathcurveto{\pgfqpoint{2.303803in}{1.440521in}}{\pgfqpoint{2.295903in}{1.437249in}}{\pgfqpoint{2.290079in}{1.431425in}}%
\pgfpathcurveto{\pgfqpoint{2.284255in}{1.425601in}}{\pgfqpoint{2.280983in}{1.417701in}}{\pgfqpoint{2.280983in}{1.409465in}}%
\pgfpathcurveto{\pgfqpoint{2.280983in}{1.401229in}}{\pgfqpoint{2.284255in}{1.393329in}}{\pgfqpoint{2.290079in}{1.387505in}}%
\pgfpathcurveto{\pgfqpoint{2.295903in}{1.381681in}}{\pgfqpoint{2.303803in}{1.378408in}}{\pgfqpoint{2.312040in}{1.378408in}}%
\pgfpathclose%
\pgfusepath{stroke,fill}%
\end{pgfscope}%
\begin{pgfscope}%
\pgfpathrectangle{\pgfqpoint{0.100000in}{0.212622in}}{\pgfqpoint{3.696000in}{3.696000in}}%
\pgfusepath{clip}%
\pgfsetbuttcap%
\pgfsetroundjoin%
\definecolor{currentfill}{rgb}{0.121569,0.466667,0.705882}%
\pgfsetfillcolor{currentfill}%
\pgfsetfillopacity{0.991976}%
\pgfsetlinewidth{1.003750pt}%
\definecolor{currentstroke}{rgb}{0.121569,0.466667,0.705882}%
\pgfsetstrokecolor{currentstroke}%
\pgfsetstrokeopacity{0.991976}%
\pgfsetdash{}{0pt}%
\pgfpathmoveto{\pgfqpoint{2.318684in}{1.374155in}}%
\pgfpathcurveto{\pgfqpoint{2.326920in}{1.374155in}}{\pgfqpoint{2.334820in}{1.377428in}}{\pgfqpoint{2.340644in}{1.383252in}}%
\pgfpathcurveto{\pgfqpoint{2.346468in}{1.389076in}}{\pgfqpoint{2.349740in}{1.396976in}}{\pgfqpoint{2.349740in}{1.405212in}}%
\pgfpathcurveto{\pgfqpoint{2.349740in}{1.413448in}}{\pgfqpoint{2.346468in}{1.421348in}}{\pgfqpoint{2.340644in}{1.427172in}}%
\pgfpathcurveto{\pgfqpoint{2.334820in}{1.432996in}}{\pgfqpoint{2.326920in}{1.436268in}}{\pgfqpoint{2.318684in}{1.436268in}}%
\pgfpathcurveto{\pgfqpoint{2.310448in}{1.436268in}}{\pgfqpoint{2.302548in}{1.432996in}}{\pgfqpoint{2.296724in}{1.427172in}}%
\pgfpathcurveto{\pgfqpoint{2.290900in}{1.421348in}}{\pgfqpoint{2.287627in}{1.413448in}}{\pgfqpoint{2.287627in}{1.405212in}}%
\pgfpathcurveto{\pgfqpoint{2.287627in}{1.396976in}}{\pgfqpoint{2.290900in}{1.389076in}}{\pgfqpoint{2.296724in}{1.383252in}}%
\pgfpathcurveto{\pgfqpoint{2.302548in}{1.377428in}}{\pgfqpoint{2.310448in}{1.374155in}}{\pgfqpoint{2.318684in}{1.374155in}}%
\pgfpathclose%
\pgfusepath{stroke,fill}%
\end{pgfscope}%
\begin{pgfscope}%
\pgfpathrectangle{\pgfqpoint{0.100000in}{0.212622in}}{\pgfqpoint{3.696000in}{3.696000in}}%
\pgfusepath{clip}%
\pgfsetbuttcap%
\pgfsetroundjoin%
\definecolor{currentfill}{rgb}{0.121569,0.466667,0.705882}%
\pgfsetfillcolor{currentfill}%
\pgfsetfillopacity{0.992578}%
\pgfsetlinewidth{1.003750pt}%
\definecolor{currentstroke}{rgb}{0.121569,0.466667,0.705882}%
\pgfsetstrokecolor{currentstroke}%
\pgfsetstrokeopacity{0.992578}%
\pgfsetdash{}{0pt}%
\pgfpathmoveto{\pgfqpoint{2.324688in}{1.370191in}}%
\pgfpathcurveto{\pgfqpoint{2.332924in}{1.370191in}}{\pgfqpoint{2.340824in}{1.373463in}}{\pgfqpoint{2.346648in}{1.379287in}}%
\pgfpathcurveto{\pgfqpoint{2.352472in}{1.385111in}}{\pgfqpoint{2.355744in}{1.393011in}}{\pgfqpoint{2.355744in}{1.401247in}}%
\pgfpathcurveto{\pgfqpoint{2.355744in}{1.409484in}}{\pgfqpoint{2.352472in}{1.417384in}}{\pgfqpoint{2.346648in}{1.423208in}}%
\pgfpathcurveto{\pgfqpoint{2.340824in}{1.429031in}}{\pgfqpoint{2.332924in}{1.432304in}}{\pgfqpoint{2.324688in}{1.432304in}}%
\pgfpathcurveto{\pgfqpoint{2.316451in}{1.432304in}}{\pgfqpoint{2.308551in}{1.429031in}}{\pgfqpoint{2.302727in}{1.423208in}}%
\pgfpathcurveto{\pgfqpoint{2.296903in}{1.417384in}}{\pgfqpoint{2.293631in}{1.409484in}}{\pgfqpoint{2.293631in}{1.401247in}}%
\pgfpathcurveto{\pgfqpoint{2.293631in}{1.393011in}}{\pgfqpoint{2.296903in}{1.385111in}}{\pgfqpoint{2.302727in}{1.379287in}}%
\pgfpathcurveto{\pgfqpoint{2.308551in}{1.373463in}}{\pgfqpoint{2.316451in}{1.370191in}}{\pgfqpoint{2.324688in}{1.370191in}}%
\pgfpathclose%
\pgfusepath{stroke,fill}%
\end{pgfscope}%
\begin{pgfscope}%
\pgfpathrectangle{\pgfqpoint{0.100000in}{0.212622in}}{\pgfqpoint{3.696000in}{3.696000in}}%
\pgfusepath{clip}%
\pgfsetbuttcap%
\pgfsetroundjoin%
\definecolor{currentfill}{rgb}{0.121569,0.466667,0.705882}%
\pgfsetfillcolor{currentfill}%
\pgfsetfillopacity{0.992649}%
\pgfsetlinewidth{1.003750pt}%
\definecolor{currentstroke}{rgb}{0.121569,0.466667,0.705882}%
\pgfsetstrokecolor{currentstroke}%
\pgfsetstrokeopacity{0.992649}%
\pgfsetdash{}{0pt}%
\pgfpathmoveto{\pgfqpoint{2.406384in}{1.330645in}}%
\pgfpathcurveto{\pgfqpoint{2.414620in}{1.330645in}}{\pgfqpoint{2.422520in}{1.333918in}}{\pgfqpoint{2.428344in}{1.339742in}}%
\pgfpathcurveto{\pgfqpoint{2.434168in}{1.345566in}}{\pgfqpoint{2.437440in}{1.353466in}}{\pgfqpoint{2.437440in}{1.361702in}}%
\pgfpathcurveto{\pgfqpoint{2.437440in}{1.369938in}}{\pgfqpoint{2.434168in}{1.377838in}}{\pgfqpoint{2.428344in}{1.383662in}}%
\pgfpathcurveto{\pgfqpoint{2.422520in}{1.389486in}}{\pgfqpoint{2.414620in}{1.392758in}}{\pgfqpoint{2.406384in}{1.392758in}}%
\pgfpathcurveto{\pgfqpoint{2.398148in}{1.392758in}}{\pgfqpoint{2.390248in}{1.389486in}}{\pgfqpoint{2.384424in}{1.383662in}}%
\pgfpathcurveto{\pgfqpoint{2.378600in}{1.377838in}}{\pgfqpoint{2.375327in}{1.369938in}}{\pgfqpoint{2.375327in}{1.361702in}}%
\pgfpathcurveto{\pgfqpoint{2.375327in}{1.353466in}}{\pgfqpoint{2.378600in}{1.345566in}}{\pgfqpoint{2.384424in}{1.339742in}}%
\pgfpathcurveto{\pgfqpoint{2.390248in}{1.333918in}}{\pgfqpoint{2.398148in}{1.330645in}}{\pgfqpoint{2.406384in}{1.330645in}}%
\pgfpathclose%
\pgfusepath{stroke,fill}%
\end{pgfscope}%
\begin{pgfscope}%
\pgfpathrectangle{\pgfqpoint{0.100000in}{0.212622in}}{\pgfqpoint{3.696000in}{3.696000in}}%
\pgfusepath{clip}%
\pgfsetbuttcap%
\pgfsetroundjoin%
\definecolor{currentfill}{rgb}{0.121569,0.466667,0.705882}%
\pgfsetfillcolor{currentfill}%
\pgfsetfillopacity{0.993011}%
\pgfsetlinewidth{1.003750pt}%
\definecolor{currentstroke}{rgb}{0.121569,0.466667,0.705882}%
\pgfsetstrokecolor{currentstroke}%
\pgfsetstrokeopacity{0.993011}%
\pgfsetdash{}{0pt}%
\pgfpathmoveto{\pgfqpoint{2.329200in}{1.367358in}}%
\pgfpathcurveto{\pgfqpoint{2.337437in}{1.367358in}}{\pgfqpoint{2.345337in}{1.370630in}}{\pgfqpoint{2.351161in}{1.376454in}}%
\pgfpathcurveto{\pgfqpoint{2.356985in}{1.382278in}}{\pgfqpoint{2.360257in}{1.390178in}}{\pgfqpoint{2.360257in}{1.398414in}}%
\pgfpathcurveto{\pgfqpoint{2.360257in}{1.406651in}}{\pgfqpoint{2.356985in}{1.414551in}}{\pgfqpoint{2.351161in}{1.420375in}}%
\pgfpathcurveto{\pgfqpoint{2.345337in}{1.426199in}}{\pgfqpoint{2.337437in}{1.429471in}}{\pgfqpoint{2.329200in}{1.429471in}}%
\pgfpathcurveto{\pgfqpoint{2.320964in}{1.429471in}}{\pgfqpoint{2.313064in}{1.426199in}}{\pgfqpoint{2.307240in}{1.420375in}}%
\pgfpathcurveto{\pgfqpoint{2.301416in}{1.414551in}}{\pgfqpoint{2.298144in}{1.406651in}}{\pgfqpoint{2.298144in}{1.398414in}}%
\pgfpathcurveto{\pgfqpoint{2.298144in}{1.390178in}}{\pgfqpoint{2.301416in}{1.382278in}}{\pgfqpoint{2.307240in}{1.376454in}}%
\pgfpathcurveto{\pgfqpoint{2.313064in}{1.370630in}}{\pgfqpoint{2.320964in}{1.367358in}}{\pgfqpoint{2.329200in}{1.367358in}}%
\pgfpathclose%
\pgfusepath{stroke,fill}%
\end{pgfscope}%
\begin{pgfscope}%
\pgfpathrectangle{\pgfqpoint{0.100000in}{0.212622in}}{\pgfqpoint{3.696000in}{3.696000in}}%
\pgfusepath{clip}%
\pgfsetbuttcap%
\pgfsetroundjoin%
\definecolor{currentfill}{rgb}{0.121569,0.466667,0.705882}%
\pgfsetfillcolor{currentfill}%
\pgfsetfillopacity{0.994120}%
\pgfsetlinewidth{1.003750pt}%
\definecolor{currentstroke}{rgb}{0.121569,0.466667,0.705882}%
\pgfsetstrokecolor{currentstroke}%
\pgfsetstrokeopacity{0.994120}%
\pgfsetdash{}{0pt}%
\pgfpathmoveto{\pgfqpoint{2.337604in}{1.364747in}}%
\pgfpathcurveto{\pgfqpoint{2.345840in}{1.364747in}}{\pgfqpoint{2.353741in}{1.368020in}}{\pgfqpoint{2.359564in}{1.373844in}}%
\pgfpathcurveto{\pgfqpoint{2.365388in}{1.379668in}}{\pgfqpoint{2.368661in}{1.387568in}}{\pgfqpoint{2.368661in}{1.395804in}}%
\pgfpathcurveto{\pgfqpoint{2.368661in}{1.404040in}}{\pgfqpoint{2.365388in}{1.411940in}}{\pgfqpoint{2.359564in}{1.417764in}}%
\pgfpathcurveto{\pgfqpoint{2.353741in}{1.423588in}}{\pgfqpoint{2.345840in}{1.426860in}}{\pgfqpoint{2.337604in}{1.426860in}}%
\pgfpathcurveto{\pgfqpoint{2.329368in}{1.426860in}}{\pgfqpoint{2.321468in}{1.423588in}}{\pgfqpoint{2.315644in}{1.417764in}}%
\pgfpathcurveto{\pgfqpoint{2.309820in}{1.411940in}}{\pgfqpoint{2.306548in}{1.404040in}}{\pgfqpoint{2.306548in}{1.395804in}}%
\pgfpathcurveto{\pgfqpoint{2.306548in}{1.387568in}}{\pgfqpoint{2.309820in}{1.379668in}}{\pgfqpoint{2.315644in}{1.373844in}}%
\pgfpathcurveto{\pgfqpoint{2.321468in}{1.368020in}}{\pgfqpoint{2.329368in}{1.364747in}}{\pgfqpoint{2.337604in}{1.364747in}}%
\pgfpathclose%
\pgfusepath{stroke,fill}%
\end{pgfscope}%
\begin{pgfscope}%
\pgfpathrectangle{\pgfqpoint{0.100000in}{0.212622in}}{\pgfqpoint{3.696000in}{3.696000in}}%
\pgfusepath{clip}%
\pgfsetbuttcap%
\pgfsetroundjoin%
\definecolor{currentfill}{rgb}{0.121569,0.466667,0.705882}%
\pgfsetfillcolor{currentfill}%
\pgfsetfillopacity{0.994538}%
\pgfsetlinewidth{1.003750pt}%
\definecolor{currentstroke}{rgb}{0.121569,0.466667,0.705882}%
\pgfsetstrokecolor{currentstroke}%
\pgfsetstrokeopacity{0.994538}%
\pgfsetdash{}{0pt}%
\pgfpathmoveto{\pgfqpoint{2.342137in}{1.360800in}}%
\pgfpathcurveto{\pgfqpoint{2.350374in}{1.360800in}}{\pgfqpoint{2.358274in}{1.364072in}}{\pgfqpoint{2.364097in}{1.369896in}}%
\pgfpathcurveto{\pgfqpoint{2.369921in}{1.375720in}}{\pgfqpoint{2.373194in}{1.383620in}}{\pgfqpoint{2.373194in}{1.391857in}}%
\pgfpathcurveto{\pgfqpoint{2.373194in}{1.400093in}}{\pgfqpoint{2.369921in}{1.407993in}}{\pgfqpoint{2.364097in}{1.413817in}}%
\pgfpathcurveto{\pgfqpoint{2.358274in}{1.419641in}}{\pgfqpoint{2.350374in}{1.422913in}}{\pgfqpoint{2.342137in}{1.422913in}}%
\pgfpathcurveto{\pgfqpoint{2.333901in}{1.422913in}}{\pgfqpoint{2.326001in}{1.419641in}}{\pgfqpoint{2.320177in}{1.413817in}}%
\pgfpathcurveto{\pgfqpoint{2.314353in}{1.407993in}}{\pgfqpoint{2.311081in}{1.400093in}}{\pgfqpoint{2.311081in}{1.391857in}}%
\pgfpathcurveto{\pgfqpoint{2.311081in}{1.383620in}}{\pgfqpoint{2.314353in}{1.375720in}}{\pgfqpoint{2.320177in}{1.369896in}}%
\pgfpathcurveto{\pgfqpoint{2.326001in}{1.364072in}}{\pgfqpoint{2.333901in}{1.360800in}}{\pgfqpoint{2.342137in}{1.360800in}}%
\pgfpathclose%
\pgfusepath{stroke,fill}%
\end{pgfscope}%
\begin{pgfscope}%
\pgfpathrectangle{\pgfqpoint{0.100000in}{0.212622in}}{\pgfqpoint{3.696000in}{3.696000in}}%
\pgfusepath{clip}%
\pgfsetbuttcap%
\pgfsetroundjoin%
\definecolor{currentfill}{rgb}{0.121569,0.466667,0.705882}%
\pgfsetfillcolor{currentfill}%
\pgfsetfillopacity{0.994852}%
\pgfsetlinewidth{1.003750pt}%
\definecolor{currentstroke}{rgb}{0.121569,0.466667,0.705882}%
\pgfsetstrokecolor{currentstroke}%
\pgfsetstrokeopacity{0.994852}%
\pgfsetdash{}{0pt}%
\pgfpathmoveto{\pgfqpoint{2.346332in}{1.357830in}}%
\pgfpathcurveto{\pgfqpoint{2.354569in}{1.357830in}}{\pgfqpoint{2.362469in}{1.361103in}}{\pgfqpoint{2.368293in}{1.366927in}}%
\pgfpathcurveto{\pgfqpoint{2.374117in}{1.372751in}}{\pgfqpoint{2.377389in}{1.380651in}}{\pgfqpoint{2.377389in}{1.388887in}}%
\pgfpathcurveto{\pgfqpoint{2.377389in}{1.397123in}}{\pgfqpoint{2.374117in}{1.405023in}}{\pgfqpoint{2.368293in}{1.410847in}}%
\pgfpathcurveto{\pgfqpoint{2.362469in}{1.416671in}}{\pgfqpoint{2.354569in}{1.419943in}}{\pgfqpoint{2.346332in}{1.419943in}}%
\pgfpathcurveto{\pgfqpoint{2.338096in}{1.419943in}}{\pgfqpoint{2.330196in}{1.416671in}}{\pgfqpoint{2.324372in}{1.410847in}}%
\pgfpathcurveto{\pgfqpoint{2.318548in}{1.405023in}}{\pgfqpoint{2.315276in}{1.397123in}}{\pgfqpoint{2.315276in}{1.388887in}}%
\pgfpathcurveto{\pgfqpoint{2.315276in}{1.380651in}}{\pgfqpoint{2.318548in}{1.372751in}}{\pgfqpoint{2.324372in}{1.366927in}}%
\pgfpathcurveto{\pgfqpoint{2.330196in}{1.361103in}}{\pgfqpoint{2.338096in}{1.357830in}}{\pgfqpoint{2.346332in}{1.357830in}}%
\pgfpathclose%
\pgfusepath{stroke,fill}%
\end{pgfscope}%
\begin{pgfscope}%
\pgfpathrectangle{\pgfqpoint{0.100000in}{0.212622in}}{\pgfqpoint{3.696000in}{3.696000in}}%
\pgfusepath{clip}%
\pgfsetbuttcap%
\pgfsetroundjoin%
\definecolor{currentfill}{rgb}{0.121569,0.466667,0.705882}%
\pgfsetfillcolor{currentfill}%
\pgfsetfillopacity{0.995052}%
\pgfsetlinewidth{1.003750pt}%
\definecolor{currentstroke}{rgb}{0.121569,0.466667,0.705882}%
\pgfsetstrokecolor{currentstroke}%
\pgfsetstrokeopacity{0.995052}%
\pgfsetdash{}{0pt}%
\pgfpathmoveto{\pgfqpoint{2.348124in}{1.356969in}}%
\pgfpathcurveto{\pgfqpoint{2.356360in}{1.356969in}}{\pgfqpoint{2.364260in}{1.360241in}}{\pgfqpoint{2.370084in}{1.366065in}}%
\pgfpathcurveto{\pgfqpoint{2.375908in}{1.371889in}}{\pgfqpoint{2.379181in}{1.379789in}}{\pgfqpoint{2.379181in}{1.388025in}}%
\pgfpathcurveto{\pgfqpoint{2.379181in}{1.396262in}}{\pgfqpoint{2.375908in}{1.404162in}}{\pgfqpoint{2.370084in}{1.409986in}}%
\pgfpathcurveto{\pgfqpoint{2.364260in}{1.415809in}}{\pgfqpoint{2.356360in}{1.419082in}}{\pgfqpoint{2.348124in}{1.419082in}}%
\pgfpathcurveto{\pgfqpoint{2.339888in}{1.419082in}}{\pgfqpoint{2.331988in}{1.415809in}}{\pgfqpoint{2.326164in}{1.409986in}}%
\pgfpathcurveto{\pgfqpoint{2.320340in}{1.404162in}}{\pgfqpoint{2.317068in}{1.396262in}}{\pgfqpoint{2.317068in}{1.388025in}}%
\pgfpathcurveto{\pgfqpoint{2.317068in}{1.379789in}}{\pgfqpoint{2.320340in}{1.371889in}}{\pgfqpoint{2.326164in}{1.366065in}}%
\pgfpathcurveto{\pgfqpoint{2.331988in}{1.360241in}}{\pgfqpoint{2.339888in}{1.356969in}}{\pgfqpoint{2.348124in}{1.356969in}}%
\pgfpathclose%
\pgfusepath{stroke,fill}%
\end{pgfscope}%
\begin{pgfscope}%
\pgfpathrectangle{\pgfqpoint{0.100000in}{0.212622in}}{\pgfqpoint{3.696000in}{3.696000in}}%
\pgfusepath{clip}%
\pgfsetbuttcap%
\pgfsetroundjoin%
\definecolor{currentfill}{rgb}{0.121569,0.466667,0.705882}%
\pgfsetfillcolor{currentfill}%
\pgfsetfillopacity{0.995403}%
\pgfsetlinewidth{1.003750pt}%
\definecolor{currentstroke}{rgb}{0.121569,0.466667,0.705882}%
\pgfsetstrokecolor{currentstroke}%
\pgfsetstrokeopacity{0.995403}%
\pgfsetdash{}{0pt}%
\pgfpathmoveto{\pgfqpoint{2.351352in}{1.355222in}}%
\pgfpathcurveto{\pgfqpoint{2.359588in}{1.355222in}}{\pgfqpoint{2.367488in}{1.358494in}}{\pgfqpoint{2.373312in}{1.364318in}}%
\pgfpathcurveto{\pgfqpoint{2.379136in}{1.370142in}}{\pgfqpoint{2.382408in}{1.378042in}}{\pgfqpoint{2.382408in}{1.386279in}}%
\pgfpathcurveto{\pgfqpoint{2.382408in}{1.394515in}}{\pgfqpoint{2.379136in}{1.402415in}}{\pgfqpoint{2.373312in}{1.408239in}}%
\pgfpathcurveto{\pgfqpoint{2.367488in}{1.414063in}}{\pgfqpoint{2.359588in}{1.417335in}}{\pgfqpoint{2.351352in}{1.417335in}}%
\pgfpathcurveto{\pgfqpoint{2.343116in}{1.417335in}}{\pgfqpoint{2.335216in}{1.414063in}}{\pgfqpoint{2.329392in}{1.408239in}}%
\pgfpathcurveto{\pgfqpoint{2.323568in}{1.402415in}}{\pgfqpoint{2.320295in}{1.394515in}}{\pgfqpoint{2.320295in}{1.386279in}}%
\pgfpathcurveto{\pgfqpoint{2.320295in}{1.378042in}}{\pgfqpoint{2.323568in}{1.370142in}}{\pgfqpoint{2.329392in}{1.364318in}}%
\pgfpathcurveto{\pgfqpoint{2.335216in}{1.358494in}}{\pgfqpoint{2.343116in}{1.355222in}}{\pgfqpoint{2.351352in}{1.355222in}}%
\pgfpathclose%
\pgfusepath{stroke,fill}%
\end{pgfscope}%
\begin{pgfscope}%
\pgfpathrectangle{\pgfqpoint{0.100000in}{0.212622in}}{\pgfqpoint{3.696000in}{3.696000in}}%
\pgfusepath{clip}%
\pgfsetbuttcap%
\pgfsetroundjoin%
\definecolor{currentfill}{rgb}{0.121569,0.466667,0.705882}%
\pgfsetfillcolor{currentfill}%
\pgfsetfillopacity{0.995515}%
\pgfsetlinewidth{1.003750pt}%
\definecolor{currentstroke}{rgb}{0.121569,0.466667,0.705882}%
\pgfsetstrokecolor{currentstroke}%
\pgfsetstrokeopacity{0.995515}%
\pgfsetdash{}{0pt}%
\pgfpathmoveto{\pgfqpoint{2.352482in}{1.354499in}}%
\pgfpathcurveto{\pgfqpoint{2.360718in}{1.354499in}}{\pgfqpoint{2.368618in}{1.357771in}}{\pgfqpoint{2.374442in}{1.363595in}}%
\pgfpathcurveto{\pgfqpoint{2.380266in}{1.369419in}}{\pgfqpoint{2.383538in}{1.377319in}}{\pgfqpoint{2.383538in}{1.385555in}}%
\pgfpathcurveto{\pgfqpoint{2.383538in}{1.393792in}}{\pgfqpoint{2.380266in}{1.401692in}}{\pgfqpoint{2.374442in}{1.407516in}}%
\pgfpathcurveto{\pgfqpoint{2.368618in}{1.413340in}}{\pgfqpoint{2.360718in}{1.416612in}}{\pgfqpoint{2.352482in}{1.416612in}}%
\pgfpathcurveto{\pgfqpoint{2.344246in}{1.416612in}}{\pgfqpoint{2.336346in}{1.413340in}}{\pgfqpoint{2.330522in}{1.407516in}}%
\pgfpathcurveto{\pgfqpoint{2.324698in}{1.401692in}}{\pgfqpoint{2.321425in}{1.393792in}}{\pgfqpoint{2.321425in}{1.385555in}}%
\pgfpathcurveto{\pgfqpoint{2.321425in}{1.377319in}}{\pgfqpoint{2.324698in}{1.369419in}}{\pgfqpoint{2.330522in}{1.363595in}}%
\pgfpathcurveto{\pgfqpoint{2.336346in}{1.357771in}}{\pgfqpoint{2.344246in}{1.354499in}}{\pgfqpoint{2.352482in}{1.354499in}}%
\pgfpathclose%
\pgfusepath{stroke,fill}%
\end{pgfscope}%
\begin{pgfscope}%
\pgfpathrectangle{\pgfqpoint{0.100000in}{0.212622in}}{\pgfqpoint{3.696000in}{3.696000in}}%
\pgfusepath{clip}%
\pgfsetbuttcap%
\pgfsetroundjoin%
\definecolor{currentfill}{rgb}{0.121569,0.466667,0.705882}%
\pgfsetfillcolor{currentfill}%
\pgfsetfillopacity{0.995546}%
\pgfsetlinewidth{1.003750pt}%
\definecolor{currentstroke}{rgb}{0.121569,0.466667,0.705882}%
\pgfsetstrokecolor{currentstroke}%
\pgfsetstrokeopacity{0.995546}%
\pgfsetdash{}{0pt}%
\pgfpathmoveto{\pgfqpoint{2.352711in}{1.354367in}}%
\pgfpathcurveto{\pgfqpoint{2.360947in}{1.354367in}}{\pgfqpoint{2.368847in}{1.357639in}}{\pgfqpoint{2.374671in}{1.363463in}}%
\pgfpathcurveto{\pgfqpoint{2.380495in}{1.369287in}}{\pgfqpoint{2.383767in}{1.377187in}}{\pgfqpoint{2.383767in}{1.385424in}}%
\pgfpathcurveto{\pgfqpoint{2.383767in}{1.393660in}}{\pgfqpoint{2.380495in}{1.401560in}}{\pgfqpoint{2.374671in}{1.407384in}}%
\pgfpathcurveto{\pgfqpoint{2.368847in}{1.413208in}}{\pgfqpoint{2.360947in}{1.416480in}}{\pgfqpoint{2.352711in}{1.416480in}}%
\pgfpathcurveto{\pgfqpoint{2.344474in}{1.416480in}}{\pgfqpoint{2.336574in}{1.413208in}}{\pgfqpoint{2.330750in}{1.407384in}}%
\pgfpathcurveto{\pgfqpoint{2.324927in}{1.401560in}}{\pgfqpoint{2.321654in}{1.393660in}}{\pgfqpoint{2.321654in}{1.385424in}}%
\pgfpathcurveto{\pgfqpoint{2.321654in}{1.377187in}}{\pgfqpoint{2.324927in}{1.369287in}}{\pgfqpoint{2.330750in}{1.363463in}}%
\pgfpathcurveto{\pgfqpoint{2.336574in}{1.357639in}}{\pgfqpoint{2.344474in}{1.354367in}}{\pgfqpoint{2.352711in}{1.354367in}}%
\pgfpathclose%
\pgfusepath{stroke,fill}%
\end{pgfscope}%
\begin{pgfscope}%
\pgfpathrectangle{\pgfqpoint{0.100000in}{0.212622in}}{\pgfqpoint{3.696000in}{3.696000in}}%
\pgfusepath{clip}%
\pgfsetbuttcap%
\pgfsetroundjoin%
\definecolor{currentfill}{rgb}{0.121569,0.466667,0.705882}%
\pgfsetfillcolor{currentfill}%
\pgfsetfillopacity{0.995601}%
\pgfsetlinewidth{1.003750pt}%
\definecolor{currentstroke}{rgb}{0.121569,0.466667,0.705882}%
\pgfsetstrokecolor{currentstroke}%
\pgfsetstrokeopacity{0.995601}%
\pgfsetdash{}{0pt}%
\pgfpathmoveto{\pgfqpoint{2.353118in}{1.354097in}}%
\pgfpathcurveto{\pgfqpoint{2.361354in}{1.354097in}}{\pgfqpoint{2.369254in}{1.357370in}}{\pgfqpoint{2.375078in}{1.363194in}}%
\pgfpathcurveto{\pgfqpoint{2.380902in}{1.369018in}}{\pgfqpoint{2.384174in}{1.376918in}}{\pgfqpoint{2.384174in}{1.385154in}}%
\pgfpathcurveto{\pgfqpoint{2.384174in}{1.393390in}}{\pgfqpoint{2.380902in}{1.401290in}}{\pgfqpoint{2.375078in}{1.407114in}}%
\pgfpathcurveto{\pgfqpoint{2.369254in}{1.412938in}}{\pgfqpoint{2.361354in}{1.416210in}}{\pgfqpoint{2.353118in}{1.416210in}}%
\pgfpathcurveto{\pgfqpoint{2.344882in}{1.416210in}}{\pgfqpoint{2.336982in}{1.412938in}}{\pgfqpoint{2.331158in}{1.407114in}}%
\pgfpathcurveto{\pgfqpoint{2.325334in}{1.401290in}}{\pgfqpoint{2.322061in}{1.393390in}}{\pgfqpoint{2.322061in}{1.385154in}}%
\pgfpathcurveto{\pgfqpoint{2.322061in}{1.376918in}}{\pgfqpoint{2.325334in}{1.369018in}}{\pgfqpoint{2.331158in}{1.363194in}}%
\pgfpathcurveto{\pgfqpoint{2.336982in}{1.357370in}}{\pgfqpoint{2.344882in}{1.354097in}}{\pgfqpoint{2.353118in}{1.354097in}}%
\pgfpathclose%
\pgfusepath{stroke,fill}%
\end{pgfscope}%
\begin{pgfscope}%
\pgfpathrectangle{\pgfqpoint{0.100000in}{0.212622in}}{\pgfqpoint{3.696000in}{3.696000in}}%
\pgfusepath{clip}%
\pgfsetbuttcap%
\pgfsetroundjoin%
\definecolor{currentfill}{rgb}{0.121569,0.466667,0.705882}%
\pgfsetfillcolor{currentfill}%
\pgfsetfillopacity{0.995692}%
\pgfsetlinewidth{1.003750pt}%
\definecolor{currentstroke}{rgb}{0.121569,0.466667,0.705882}%
\pgfsetstrokecolor{currentstroke}%
\pgfsetstrokeopacity{0.995692}%
\pgfsetdash{}{0pt}%
\pgfpathmoveto{\pgfqpoint{2.353848in}{1.353525in}}%
\pgfpathcurveto{\pgfqpoint{2.362084in}{1.353525in}}{\pgfqpoint{2.369984in}{1.356797in}}{\pgfqpoint{2.375808in}{1.362621in}}%
\pgfpathcurveto{\pgfqpoint{2.381632in}{1.368445in}}{\pgfqpoint{2.384904in}{1.376345in}}{\pgfqpoint{2.384904in}{1.384581in}}%
\pgfpathcurveto{\pgfqpoint{2.384904in}{1.392818in}}{\pgfqpoint{2.381632in}{1.400718in}}{\pgfqpoint{2.375808in}{1.406542in}}%
\pgfpathcurveto{\pgfqpoint{2.369984in}{1.412366in}}{\pgfqpoint{2.362084in}{1.415638in}}{\pgfqpoint{2.353848in}{1.415638in}}%
\pgfpathcurveto{\pgfqpoint{2.345612in}{1.415638in}}{\pgfqpoint{2.337712in}{1.412366in}}{\pgfqpoint{2.331888in}{1.406542in}}%
\pgfpathcurveto{\pgfqpoint{2.326064in}{1.400718in}}{\pgfqpoint{2.322791in}{1.392818in}}{\pgfqpoint{2.322791in}{1.384581in}}%
\pgfpathcurveto{\pgfqpoint{2.322791in}{1.376345in}}{\pgfqpoint{2.326064in}{1.368445in}}{\pgfqpoint{2.331888in}{1.362621in}}%
\pgfpathcurveto{\pgfqpoint{2.337712in}{1.356797in}}{\pgfqpoint{2.345612in}{1.353525in}}{\pgfqpoint{2.353848in}{1.353525in}}%
\pgfpathclose%
\pgfusepath{stroke,fill}%
\end{pgfscope}%
\begin{pgfscope}%
\pgfpathrectangle{\pgfqpoint{0.100000in}{0.212622in}}{\pgfqpoint{3.696000in}{3.696000in}}%
\pgfusepath{clip}%
\pgfsetbuttcap%
\pgfsetroundjoin%
\definecolor{currentfill}{rgb}{0.121569,0.466667,0.705882}%
\pgfsetfillcolor{currentfill}%
\pgfsetfillopacity{0.995699}%
\pgfsetlinewidth{1.003750pt}%
\definecolor{currentstroke}{rgb}{0.121569,0.466667,0.705882}%
\pgfsetstrokecolor{currentstroke}%
\pgfsetstrokeopacity{0.995699}%
\pgfsetdash{}{0pt}%
\pgfpathmoveto{\pgfqpoint{2.407960in}{1.326216in}}%
\pgfpathcurveto{\pgfqpoint{2.416196in}{1.326216in}}{\pgfqpoint{2.424096in}{1.329489in}}{\pgfqpoint{2.429920in}{1.335313in}}%
\pgfpathcurveto{\pgfqpoint{2.435744in}{1.341137in}}{\pgfqpoint{2.439017in}{1.349037in}}{\pgfqpoint{2.439017in}{1.357273in}}%
\pgfpathcurveto{\pgfqpoint{2.439017in}{1.365509in}}{\pgfqpoint{2.435744in}{1.373409in}}{\pgfqpoint{2.429920in}{1.379233in}}%
\pgfpathcurveto{\pgfqpoint{2.424096in}{1.385057in}}{\pgfqpoint{2.416196in}{1.388329in}}{\pgfqpoint{2.407960in}{1.388329in}}%
\pgfpathcurveto{\pgfqpoint{2.399724in}{1.388329in}}{\pgfqpoint{2.391824in}{1.385057in}}{\pgfqpoint{2.386000in}{1.379233in}}%
\pgfpathcurveto{\pgfqpoint{2.380176in}{1.373409in}}{\pgfqpoint{2.376904in}{1.365509in}}{\pgfqpoint{2.376904in}{1.357273in}}%
\pgfpathcurveto{\pgfqpoint{2.376904in}{1.349037in}}{\pgfqpoint{2.380176in}{1.341137in}}{\pgfqpoint{2.386000in}{1.335313in}}%
\pgfpathcurveto{\pgfqpoint{2.391824in}{1.329489in}}{\pgfqpoint{2.399724in}{1.326216in}}{\pgfqpoint{2.407960in}{1.326216in}}%
\pgfpathclose%
\pgfusepath{stroke,fill}%
\end{pgfscope}%
\begin{pgfscope}%
\pgfpathrectangle{\pgfqpoint{0.100000in}{0.212622in}}{\pgfqpoint{3.696000in}{3.696000in}}%
\pgfusepath{clip}%
\pgfsetbuttcap%
\pgfsetroundjoin%
\definecolor{currentfill}{rgb}{0.121569,0.466667,0.705882}%
\pgfsetfillcolor{currentfill}%
\pgfsetfillopacity{0.995847}%
\pgfsetlinewidth{1.003750pt}%
\definecolor{currentstroke}{rgb}{0.121569,0.466667,0.705882}%
\pgfsetstrokecolor{currentstroke}%
\pgfsetstrokeopacity{0.995847}%
\pgfsetdash{}{0pt}%
\pgfpathmoveto{\pgfqpoint{2.355185in}{1.352455in}}%
\pgfpathcurveto{\pgfqpoint{2.363421in}{1.352455in}}{\pgfqpoint{2.371321in}{1.355728in}}{\pgfqpoint{2.377145in}{1.361551in}}%
\pgfpathcurveto{\pgfqpoint{2.382969in}{1.367375in}}{\pgfqpoint{2.386241in}{1.375275in}}{\pgfqpoint{2.386241in}{1.383512in}}%
\pgfpathcurveto{\pgfqpoint{2.386241in}{1.391748in}}{\pgfqpoint{2.382969in}{1.399648in}}{\pgfqpoint{2.377145in}{1.405472in}}%
\pgfpathcurveto{\pgfqpoint{2.371321in}{1.411296in}}{\pgfqpoint{2.363421in}{1.414568in}}{\pgfqpoint{2.355185in}{1.414568in}}%
\pgfpathcurveto{\pgfqpoint{2.346948in}{1.414568in}}{\pgfqpoint{2.339048in}{1.411296in}}{\pgfqpoint{2.333224in}{1.405472in}}%
\pgfpathcurveto{\pgfqpoint{2.327401in}{1.399648in}}{\pgfqpoint{2.324128in}{1.391748in}}{\pgfqpoint{2.324128in}{1.383512in}}%
\pgfpathcurveto{\pgfqpoint{2.324128in}{1.375275in}}{\pgfqpoint{2.327401in}{1.367375in}}{\pgfqpoint{2.333224in}{1.361551in}}%
\pgfpathcurveto{\pgfqpoint{2.339048in}{1.355728in}}{\pgfqpoint{2.346948in}{1.352455in}}{\pgfqpoint{2.355185in}{1.352455in}}%
\pgfpathclose%
\pgfusepath{stroke,fill}%
\end{pgfscope}%
\begin{pgfscope}%
\pgfpathrectangle{\pgfqpoint{0.100000in}{0.212622in}}{\pgfqpoint{3.696000in}{3.696000in}}%
\pgfusepath{clip}%
\pgfsetbuttcap%
\pgfsetroundjoin%
\definecolor{currentfill}{rgb}{0.121569,0.466667,0.705882}%
\pgfsetfillcolor{currentfill}%
\pgfsetfillopacity{0.996126}%
\pgfsetlinewidth{1.003750pt}%
\definecolor{currentstroke}{rgb}{0.121569,0.466667,0.705882}%
\pgfsetstrokecolor{currentstroke}%
\pgfsetstrokeopacity{0.996126}%
\pgfsetdash{}{0pt}%
\pgfpathmoveto{\pgfqpoint{2.357629in}{1.350523in}}%
\pgfpathcurveto{\pgfqpoint{2.365866in}{1.350523in}}{\pgfqpoint{2.373766in}{1.353795in}}{\pgfqpoint{2.379590in}{1.359619in}}%
\pgfpathcurveto{\pgfqpoint{2.385413in}{1.365443in}}{\pgfqpoint{2.388686in}{1.373343in}}{\pgfqpoint{2.388686in}{1.381580in}}%
\pgfpathcurveto{\pgfqpoint{2.388686in}{1.389816in}}{\pgfqpoint{2.385413in}{1.397716in}}{\pgfqpoint{2.379590in}{1.403540in}}%
\pgfpathcurveto{\pgfqpoint{2.373766in}{1.409364in}}{\pgfqpoint{2.365866in}{1.412636in}}{\pgfqpoint{2.357629in}{1.412636in}}%
\pgfpathcurveto{\pgfqpoint{2.349393in}{1.412636in}}{\pgfqpoint{2.341493in}{1.409364in}}{\pgfqpoint{2.335669in}{1.403540in}}%
\pgfpathcurveto{\pgfqpoint{2.329845in}{1.397716in}}{\pgfqpoint{2.326573in}{1.389816in}}{\pgfqpoint{2.326573in}{1.381580in}}%
\pgfpathcurveto{\pgfqpoint{2.326573in}{1.373343in}}{\pgfqpoint{2.329845in}{1.365443in}}{\pgfqpoint{2.335669in}{1.359619in}}%
\pgfpathcurveto{\pgfqpoint{2.341493in}{1.353795in}}{\pgfqpoint{2.349393in}{1.350523in}}{\pgfqpoint{2.357629in}{1.350523in}}%
\pgfpathclose%
\pgfusepath{stroke,fill}%
\end{pgfscope}%
\begin{pgfscope}%
\pgfpathrectangle{\pgfqpoint{0.100000in}{0.212622in}}{\pgfqpoint{3.696000in}{3.696000in}}%
\pgfusepath{clip}%
\pgfsetbuttcap%
\pgfsetroundjoin%
\definecolor{currentfill}{rgb}{0.121569,0.466667,0.705882}%
\pgfsetfillcolor{currentfill}%
\pgfsetfillopacity{0.996609}%
\pgfsetlinewidth{1.003750pt}%
\definecolor{currentstroke}{rgb}{0.121569,0.466667,0.705882}%
\pgfsetstrokecolor{currentstroke}%
\pgfsetstrokeopacity{0.996609}%
\pgfsetdash{}{0pt}%
\pgfpathmoveto{\pgfqpoint{2.362080in}{1.346892in}}%
\pgfpathcurveto{\pgfqpoint{2.370316in}{1.346892in}}{\pgfqpoint{2.378216in}{1.350165in}}{\pgfqpoint{2.384040in}{1.355989in}}%
\pgfpathcurveto{\pgfqpoint{2.389864in}{1.361812in}}{\pgfqpoint{2.393136in}{1.369713in}}{\pgfqpoint{2.393136in}{1.377949in}}%
\pgfpathcurveto{\pgfqpoint{2.393136in}{1.386185in}}{\pgfqpoint{2.389864in}{1.394085in}}{\pgfqpoint{2.384040in}{1.399909in}}%
\pgfpathcurveto{\pgfqpoint{2.378216in}{1.405733in}}{\pgfqpoint{2.370316in}{1.409005in}}{\pgfqpoint{2.362080in}{1.409005in}}%
\pgfpathcurveto{\pgfqpoint{2.353843in}{1.409005in}}{\pgfqpoint{2.345943in}{1.405733in}}{\pgfqpoint{2.340119in}{1.399909in}}%
\pgfpathcurveto{\pgfqpoint{2.334295in}{1.394085in}}{\pgfqpoint{2.331023in}{1.386185in}}{\pgfqpoint{2.331023in}{1.377949in}}%
\pgfpathcurveto{\pgfqpoint{2.331023in}{1.369713in}}{\pgfqpoint{2.334295in}{1.361812in}}{\pgfqpoint{2.340119in}{1.355989in}}%
\pgfpathcurveto{\pgfqpoint{2.345943in}{1.350165in}}{\pgfqpoint{2.353843in}{1.346892in}}{\pgfqpoint{2.362080in}{1.346892in}}%
\pgfpathclose%
\pgfusepath{stroke,fill}%
\end{pgfscope}%
\begin{pgfscope}%
\pgfpathrectangle{\pgfqpoint{0.100000in}{0.212622in}}{\pgfqpoint{3.696000in}{3.696000in}}%
\pgfusepath{clip}%
\pgfsetbuttcap%
\pgfsetroundjoin%
\definecolor{currentfill}{rgb}{0.121569,0.466667,0.705882}%
\pgfsetfillcolor{currentfill}%
\pgfsetfillopacity{0.997225}%
\pgfsetlinewidth{1.003750pt}%
\definecolor{currentstroke}{rgb}{0.121569,0.466667,0.705882}%
\pgfsetstrokecolor{currentstroke}%
\pgfsetstrokeopacity{0.997225}%
\pgfsetdash{}{0pt}%
\pgfpathmoveto{\pgfqpoint{2.407914in}{1.322630in}}%
\pgfpathcurveto{\pgfqpoint{2.416150in}{1.322630in}}{\pgfqpoint{2.424050in}{1.325902in}}{\pgfqpoint{2.429874in}{1.331726in}}%
\pgfpathcurveto{\pgfqpoint{2.435698in}{1.337550in}}{\pgfqpoint{2.438970in}{1.345450in}}{\pgfqpoint{2.438970in}{1.353686in}}%
\pgfpathcurveto{\pgfqpoint{2.438970in}{1.361923in}}{\pgfqpoint{2.435698in}{1.369823in}}{\pgfqpoint{2.429874in}{1.375647in}}%
\pgfpathcurveto{\pgfqpoint{2.424050in}{1.381471in}}{\pgfqpoint{2.416150in}{1.384743in}}{\pgfqpoint{2.407914in}{1.384743in}}%
\pgfpathcurveto{\pgfqpoint{2.399677in}{1.384743in}}{\pgfqpoint{2.391777in}{1.381471in}}{\pgfqpoint{2.385953in}{1.375647in}}%
\pgfpathcurveto{\pgfqpoint{2.380129in}{1.369823in}}{\pgfqpoint{2.376857in}{1.361923in}}{\pgfqpoint{2.376857in}{1.353686in}}%
\pgfpathcurveto{\pgfqpoint{2.376857in}{1.345450in}}{\pgfqpoint{2.380129in}{1.337550in}}{\pgfqpoint{2.385953in}{1.331726in}}%
\pgfpathcurveto{\pgfqpoint{2.391777in}{1.325902in}}{\pgfqpoint{2.399677in}{1.322630in}}{\pgfqpoint{2.407914in}{1.322630in}}%
\pgfpathclose%
\pgfusepath{stroke,fill}%
\end{pgfscope}%
\begin{pgfscope}%
\pgfpathrectangle{\pgfqpoint{0.100000in}{0.212622in}}{\pgfqpoint{3.696000in}{3.696000in}}%
\pgfusepath{clip}%
\pgfsetbuttcap%
\pgfsetroundjoin%
\definecolor{currentfill}{rgb}{0.121569,0.466667,0.705882}%
\pgfsetfillcolor{currentfill}%
\pgfsetfillopacity{0.997489}%
\pgfsetlinewidth{1.003750pt}%
\definecolor{currentstroke}{rgb}{0.121569,0.466667,0.705882}%
\pgfsetstrokecolor{currentstroke}%
\pgfsetstrokeopacity{0.997489}%
\pgfsetdash{}{0pt}%
\pgfpathmoveto{\pgfqpoint{2.370049in}{1.339954in}}%
\pgfpathcurveto{\pgfqpoint{2.378286in}{1.339954in}}{\pgfqpoint{2.386186in}{1.343226in}}{\pgfqpoint{2.392010in}{1.349050in}}%
\pgfpathcurveto{\pgfqpoint{2.397833in}{1.354874in}}{\pgfqpoint{2.401106in}{1.362774in}}{\pgfqpoint{2.401106in}{1.371010in}}%
\pgfpathcurveto{\pgfqpoint{2.401106in}{1.379246in}}{\pgfqpoint{2.397833in}{1.387146in}}{\pgfqpoint{2.392010in}{1.392970in}}%
\pgfpathcurveto{\pgfqpoint{2.386186in}{1.398794in}}{\pgfqpoint{2.378286in}{1.402067in}}{\pgfqpoint{2.370049in}{1.402067in}}%
\pgfpathcurveto{\pgfqpoint{2.361813in}{1.402067in}}{\pgfqpoint{2.353913in}{1.398794in}}{\pgfqpoint{2.348089in}{1.392970in}}%
\pgfpathcurveto{\pgfqpoint{2.342265in}{1.387146in}}{\pgfqpoint{2.338993in}{1.379246in}}{\pgfqpoint{2.338993in}{1.371010in}}%
\pgfpathcurveto{\pgfqpoint{2.338993in}{1.362774in}}{\pgfqpoint{2.342265in}{1.354874in}}{\pgfqpoint{2.348089in}{1.349050in}}%
\pgfpathcurveto{\pgfqpoint{2.353913in}{1.343226in}}{\pgfqpoint{2.361813in}{1.339954in}}{\pgfqpoint{2.370049in}{1.339954in}}%
\pgfpathclose%
\pgfusepath{stroke,fill}%
\end{pgfscope}%
\begin{pgfscope}%
\pgfpathrectangle{\pgfqpoint{0.100000in}{0.212622in}}{\pgfqpoint{3.696000in}{3.696000in}}%
\pgfusepath{clip}%
\pgfsetbuttcap%
\pgfsetroundjoin%
\definecolor{currentfill}{rgb}{0.121569,0.466667,0.705882}%
\pgfsetfillcolor{currentfill}%
\pgfsetfillopacity{0.998000}%
\pgfsetlinewidth{1.003750pt}%
\definecolor{currentstroke}{rgb}{0.121569,0.466667,0.705882}%
\pgfsetstrokecolor{currentstroke}%
\pgfsetstrokeopacity{0.998000}%
\pgfsetdash{}{0pt}%
\pgfpathmoveto{\pgfqpoint{2.377666in}{1.331930in}}%
\pgfpathcurveto{\pgfqpoint{2.385902in}{1.331930in}}{\pgfqpoint{2.393802in}{1.335202in}}{\pgfqpoint{2.399626in}{1.341026in}}%
\pgfpathcurveto{\pgfqpoint{2.405450in}{1.346850in}}{\pgfqpoint{2.408723in}{1.354750in}}{\pgfqpoint{2.408723in}{1.362986in}}%
\pgfpathcurveto{\pgfqpoint{2.408723in}{1.371223in}}{\pgfqpoint{2.405450in}{1.379123in}}{\pgfqpoint{2.399626in}{1.384947in}}%
\pgfpathcurveto{\pgfqpoint{2.393802in}{1.390770in}}{\pgfqpoint{2.385902in}{1.394043in}}{\pgfqpoint{2.377666in}{1.394043in}}%
\pgfpathcurveto{\pgfqpoint{2.369430in}{1.394043in}}{\pgfqpoint{2.361530in}{1.390770in}}{\pgfqpoint{2.355706in}{1.384947in}}%
\pgfpathcurveto{\pgfqpoint{2.349882in}{1.379123in}}{\pgfqpoint{2.346610in}{1.371223in}}{\pgfqpoint{2.346610in}{1.362986in}}%
\pgfpathcurveto{\pgfqpoint{2.346610in}{1.354750in}}{\pgfqpoint{2.349882in}{1.346850in}}{\pgfqpoint{2.355706in}{1.341026in}}%
\pgfpathcurveto{\pgfqpoint{2.361530in}{1.335202in}}{\pgfqpoint{2.369430in}{1.331930in}}{\pgfqpoint{2.377666in}{1.331930in}}%
\pgfpathclose%
\pgfusepath{stroke,fill}%
\end{pgfscope}%
\begin{pgfscope}%
\pgfpathrectangle{\pgfqpoint{0.100000in}{0.212622in}}{\pgfqpoint{3.696000in}{3.696000in}}%
\pgfusepath{clip}%
\pgfsetbuttcap%
\pgfsetroundjoin%
\definecolor{currentfill}{rgb}{0.121569,0.466667,0.705882}%
\pgfsetfillcolor{currentfill}%
\pgfsetfillopacity{0.998198}%
\pgfsetlinewidth{1.003750pt}%
\definecolor{currentstroke}{rgb}{0.121569,0.466667,0.705882}%
\pgfsetstrokecolor{currentstroke}%
\pgfsetstrokeopacity{0.998198}%
\pgfsetdash{}{0pt}%
\pgfpathmoveto{\pgfqpoint{2.407420in}{1.321469in}}%
\pgfpathcurveto{\pgfqpoint{2.415656in}{1.321469in}}{\pgfqpoint{2.423556in}{1.324741in}}{\pgfqpoint{2.429380in}{1.330565in}}%
\pgfpathcurveto{\pgfqpoint{2.435204in}{1.336389in}}{\pgfqpoint{2.438476in}{1.344289in}}{\pgfqpoint{2.438476in}{1.352526in}}%
\pgfpathcurveto{\pgfqpoint{2.438476in}{1.360762in}}{\pgfqpoint{2.435204in}{1.368662in}}{\pgfqpoint{2.429380in}{1.374486in}}%
\pgfpathcurveto{\pgfqpoint{2.423556in}{1.380310in}}{\pgfqpoint{2.415656in}{1.383582in}}{\pgfqpoint{2.407420in}{1.383582in}}%
\pgfpathcurveto{\pgfqpoint{2.399183in}{1.383582in}}{\pgfqpoint{2.391283in}{1.380310in}}{\pgfqpoint{2.385459in}{1.374486in}}%
\pgfpathcurveto{\pgfqpoint{2.379636in}{1.368662in}}{\pgfqpoint{2.376363in}{1.360762in}}{\pgfqpoint{2.376363in}{1.352526in}}%
\pgfpathcurveto{\pgfqpoint{2.376363in}{1.344289in}}{\pgfqpoint{2.379636in}{1.336389in}}{\pgfqpoint{2.385459in}{1.330565in}}%
\pgfpathcurveto{\pgfqpoint{2.391283in}{1.324741in}}{\pgfqpoint{2.399183in}{1.321469in}}{\pgfqpoint{2.407420in}{1.321469in}}%
\pgfpathclose%
\pgfusepath{stroke,fill}%
\end{pgfscope}%
\begin{pgfscope}%
\pgfpathrectangle{\pgfqpoint{0.100000in}{0.212622in}}{\pgfqpoint{3.696000in}{3.696000in}}%
\pgfusepath{clip}%
\pgfsetbuttcap%
\pgfsetroundjoin%
\definecolor{currentfill}{rgb}{0.121569,0.466667,0.705882}%
\pgfsetfillcolor{currentfill}%
\pgfsetfillopacity{0.998691}%
\pgfsetlinewidth{1.003750pt}%
\definecolor{currentstroke}{rgb}{0.121569,0.466667,0.705882}%
\pgfsetstrokecolor{currentstroke}%
\pgfsetstrokeopacity{0.998691}%
\pgfsetdash{}{0pt}%
\pgfpathmoveto{\pgfqpoint{2.406668in}{1.320812in}}%
\pgfpathcurveto{\pgfqpoint{2.414904in}{1.320812in}}{\pgfqpoint{2.422804in}{1.324084in}}{\pgfqpoint{2.428628in}{1.329908in}}%
\pgfpathcurveto{\pgfqpoint{2.434452in}{1.335732in}}{\pgfqpoint{2.437724in}{1.343632in}}{\pgfqpoint{2.437724in}{1.351868in}}%
\pgfpathcurveto{\pgfqpoint{2.437724in}{1.360104in}}{\pgfqpoint{2.434452in}{1.368004in}}{\pgfqpoint{2.428628in}{1.373828in}}%
\pgfpathcurveto{\pgfqpoint{2.422804in}{1.379652in}}{\pgfqpoint{2.414904in}{1.382925in}}{\pgfqpoint{2.406668in}{1.382925in}}%
\pgfpathcurveto{\pgfqpoint{2.398432in}{1.382925in}}{\pgfqpoint{2.390531in}{1.379652in}}{\pgfqpoint{2.384708in}{1.373828in}}%
\pgfpathcurveto{\pgfqpoint{2.378884in}{1.368004in}}{\pgfqpoint{2.375611in}{1.360104in}}{\pgfqpoint{2.375611in}{1.351868in}}%
\pgfpathcurveto{\pgfqpoint{2.375611in}{1.343632in}}{\pgfqpoint{2.378884in}{1.335732in}}{\pgfqpoint{2.384708in}{1.329908in}}%
\pgfpathcurveto{\pgfqpoint{2.390531in}{1.324084in}}{\pgfqpoint{2.398432in}{1.320812in}}{\pgfqpoint{2.406668in}{1.320812in}}%
\pgfpathclose%
\pgfusepath{stroke,fill}%
\end{pgfscope}%
\begin{pgfscope}%
\pgfpathrectangle{\pgfqpoint{0.100000in}{0.212622in}}{\pgfqpoint{3.696000in}{3.696000in}}%
\pgfusepath{clip}%
\pgfsetbuttcap%
\pgfsetroundjoin%
\definecolor{currentfill}{rgb}{0.121569,0.466667,0.705882}%
\pgfsetfillcolor{currentfill}%
\pgfsetfillopacity{0.999214}%
\pgfsetlinewidth{1.003750pt}%
\definecolor{currentstroke}{rgb}{0.121569,0.466667,0.705882}%
\pgfsetstrokecolor{currentstroke}%
\pgfsetstrokeopacity{0.999214}%
\pgfsetdash{}{0pt}%
\pgfpathmoveto{\pgfqpoint{2.384933in}{1.329938in}}%
\pgfpathcurveto{\pgfqpoint{2.393170in}{1.329938in}}{\pgfqpoint{2.401070in}{1.333210in}}{\pgfqpoint{2.406894in}{1.339034in}}%
\pgfpathcurveto{\pgfqpoint{2.412718in}{1.344858in}}{\pgfqpoint{2.415990in}{1.352758in}}{\pgfqpoint{2.415990in}{1.360994in}}%
\pgfpathcurveto{\pgfqpoint{2.415990in}{1.369231in}}{\pgfqpoint{2.412718in}{1.377131in}}{\pgfqpoint{2.406894in}{1.382955in}}%
\pgfpathcurveto{\pgfqpoint{2.401070in}{1.388779in}}{\pgfqpoint{2.393170in}{1.392051in}}{\pgfqpoint{2.384933in}{1.392051in}}%
\pgfpathcurveto{\pgfqpoint{2.376697in}{1.392051in}}{\pgfqpoint{2.368797in}{1.388779in}}{\pgfqpoint{2.362973in}{1.382955in}}%
\pgfpathcurveto{\pgfqpoint{2.357149in}{1.377131in}}{\pgfqpoint{2.353877in}{1.369231in}}{\pgfqpoint{2.353877in}{1.360994in}}%
\pgfpathcurveto{\pgfqpoint{2.353877in}{1.352758in}}{\pgfqpoint{2.357149in}{1.344858in}}{\pgfqpoint{2.362973in}{1.339034in}}%
\pgfpathcurveto{\pgfqpoint{2.368797in}{1.333210in}}{\pgfqpoint{2.376697in}{1.329938in}}{\pgfqpoint{2.384933in}{1.329938in}}%
\pgfpathclose%
\pgfusepath{stroke,fill}%
\end{pgfscope}%
\begin{pgfscope}%
\pgfpathrectangle{\pgfqpoint{0.100000in}{0.212622in}}{\pgfqpoint{3.696000in}{3.696000in}}%
\pgfusepath{clip}%
\pgfsetbuttcap%
\pgfsetroundjoin%
\definecolor{currentfill}{rgb}{0.121569,0.466667,0.705882}%
\pgfsetfillcolor{currentfill}%
\pgfsetfillopacity{0.999388}%
\pgfsetlinewidth{1.003750pt}%
\definecolor{currentstroke}{rgb}{0.121569,0.466667,0.705882}%
\pgfsetstrokecolor{currentstroke}%
\pgfsetstrokeopacity{0.999388}%
\pgfsetdash{}{0pt}%
\pgfpathmoveto{\pgfqpoint{2.404987in}{1.320609in}}%
\pgfpathcurveto{\pgfqpoint{2.413223in}{1.320609in}}{\pgfqpoint{2.421123in}{1.323881in}}{\pgfqpoint{2.426947in}{1.329705in}}%
\pgfpathcurveto{\pgfqpoint{2.432771in}{1.335529in}}{\pgfqpoint{2.436043in}{1.343429in}}{\pgfqpoint{2.436043in}{1.351665in}}%
\pgfpathcurveto{\pgfqpoint{2.436043in}{1.359901in}}{\pgfqpoint{2.432771in}{1.367802in}}{\pgfqpoint{2.426947in}{1.373625in}}%
\pgfpathcurveto{\pgfqpoint{2.421123in}{1.379449in}}{\pgfqpoint{2.413223in}{1.382722in}}{\pgfqpoint{2.404987in}{1.382722in}}%
\pgfpathcurveto{\pgfqpoint{2.396751in}{1.382722in}}{\pgfqpoint{2.388851in}{1.379449in}}{\pgfqpoint{2.383027in}{1.373625in}}%
\pgfpathcurveto{\pgfqpoint{2.377203in}{1.367802in}}{\pgfqpoint{2.373930in}{1.359901in}}{\pgfqpoint{2.373930in}{1.351665in}}%
\pgfpathcurveto{\pgfqpoint{2.373930in}{1.343429in}}{\pgfqpoint{2.377203in}{1.335529in}}{\pgfqpoint{2.383027in}{1.329705in}}%
\pgfpathcurveto{\pgfqpoint{2.388851in}{1.323881in}}{\pgfqpoint{2.396751in}{1.320609in}}{\pgfqpoint{2.404987in}{1.320609in}}%
\pgfpathclose%
\pgfusepath{stroke,fill}%
\end{pgfscope}%
\begin{pgfscope}%
\pgfpathrectangle{\pgfqpoint{0.100000in}{0.212622in}}{\pgfqpoint{3.696000in}{3.696000in}}%
\pgfusepath{clip}%
\pgfsetbuttcap%
\pgfsetroundjoin%
\definecolor{currentfill}{rgb}{0.121569,0.466667,0.705882}%
\pgfsetfillcolor{currentfill}%
\pgfsetfillopacity{0.999735}%
\pgfsetlinewidth{1.003750pt}%
\definecolor{currentstroke}{rgb}{0.121569,0.466667,0.705882}%
\pgfsetstrokecolor{currentstroke}%
\pgfsetstrokeopacity{0.999735}%
\pgfsetdash{}{0pt}%
\pgfpathmoveto{\pgfqpoint{2.391363in}{1.325471in}}%
\pgfpathcurveto{\pgfqpoint{2.399599in}{1.325471in}}{\pgfqpoint{2.407499in}{1.328743in}}{\pgfqpoint{2.413323in}{1.334567in}}%
\pgfpathcurveto{\pgfqpoint{2.419147in}{1.340391in}}{\pgfqpoint{2.422419in}{1.348291in}}{\pgfqpoint{2.422419in}{1.356527in}}%
\pgfpathcurveto{\pgfqpoint{2.422419in}{1.364764in}}{\pgfqpoint{2.419147in}{1.372664in}}{\pgfqpoint{2.413323in}{1.378488in}}%
\pgfpathcurveto{\pgfqpoint{2.407499in}{1.384312in}}{\pgfqpoint{2.399599in}{1.387584in}}{\pgfqpoint{2.391363in}{1.387584in}}%
\pgfpathcurveto{\pgfqpoint{2.383126in}{1.387584in}}{\pgfqpoint{2.375226in}{1.384312in}}{\pgfqpoint{2.369402in}{1.378488in}}%
\pgfpathcurveto{\pgfqpoint{2.363579in}{1.372664in}}{\pgfqpoint{2.360306in}{1.364764in}}{\pgfqpoint{2.360306in}{1.356527in}}%
\pgfpathcurveto{\pgfqpoint{2.360306in}{1.348291in}}{\pgfqpoint{2.363579in}{1.340391in}}{\pgfqpoint{2.369402in}{1.334567in}}%
\pgfpathcurveto{\pgfqpoint{2.375226in}{1.328743in}}{\pgfqpoint{2.383126in}{1.325471in}}{\pgfqpoint{2.391363in}{1.325471in}}%
\pgfpathclose%
\pgfusepath{stroke,fill}%
\end{pgfscope}%
\begin{pgfscope}%
\pgfpathrectangle{\pgfqpoint{0.100000in}{0.212622in}}{\pgfqpoint{3.696000in}{3.696000in}}%
\pgfusepath{clip}%
\pgfsetbuttcap%
\pgfsetroundjoin%
\definecolor{currentfill}{rgb}{0.121569,0.466667,0.705882}%
\pgfsetfillcolor{currentfill}%
\pgfsetfillopacity{0.999861}%
\pgfsetlinewidth{1.003750pt}%
\definecolor{currentstroke}{rgb}{0.121569,0.466667,0.705882}%
\pgfsetstrokecolor{currentstroke}%
\pgfsetstrokeopacity{0.999861}%
\pgfsetdash{}{0pt}%
\pgfpathmoveto{\pgfqpoint{2.401906in}{1.320183in}}%
\pgfpathcurveto{\pgfqpoint{2.410142in}{1.320183in}}{\pgfqpoint{2.418043in}{1.323455in}}{\pgfqpoint{2.423866in}{1.329279in}}%
\pgfpathcurveto{\pgfqpoint{2.429690in}{1.335103in}}{\pgfqpoint{2.432963in}{1.343003in}}{\pgfqpoint{2.432963in}{1.351240in}}%
\pgfpathcurveto{\pgfqpoint{2.432963in}{1.359476in}}{\pgfqpoint{2.429690in}{1.367376in}}{\pgfqpoint{2.423866in}{1.373200in}}%
\pgfpathcurveto{\pgfqpoint{2.418043in}{1.379024in}}{\pgfqpoint{2.410142in}{1.382296in}}{\pgfqpoint{2.401906in}{1.382296in}}%
\pgfpathcurveto{\pgfqpoint{2.393670in}{1.382296in}}{\pgfqpoint{2.385770in}{1.379024in}}{\pgfqpoint{2.379946in}{1.373200in}}%
\pgfpathcurveto{\pgfqpoint{2.374122in}{1.367376in}}{\pgfqpoint{2.370850in}{1.359476in}}{\pgfqpoint{2.370850in}{1.351240in}}%
\pgfpathcurveto{\pgfqpoint{2.370850in}{1.343003in}}{\pgfqpoint{2.374122in}{1.335103in}}{\pgfqpoint{2.379946in}{1.329279in}}%
\pgfpathcurveto{\pgfqpoint{2.385770in}{1.323455in}}{\pgfqpoint{2.393670in}{1.320183in}}{\pgfqpoint{2.401906in}{1.320183in}}%
\pgfpathclose%
\pgfusepath{stroke,fill}%
\end{pgfscope}%
\begin{pgfscope}%
\pgfpathrectangle{\pgfqpoint{0.100000in}{0.212622in}}{\pgfqpoint{3.696000in}{3.696000in}}%
\pgfusepath{clip}%
\pgfsetbuttcap%
\pgfsetroundjoin%
\definecolor{currentfill}{rgb}{0.121569,0.466667,0.705882}%
\pgfsetfillcolor{currentfill}%
\pgfsetfillopacity{0.999920}%
\pgfsetlinewidth{1.003750pt}%
\definecolor{currentstroke}{rgb}{0.121569,0.466667,0.705882}%
\pgfsetstrokecolor{currentstroke}%
\pgfsetstrokeopacity{0.999920}%
\pgfsetdash{}{0pt}%
\pgfpathmoveto{\pgfqpoint{2.397839in}{1.321109in}}%
\pgfpathcurveto{\pgfqpoint{2.406076in}{1.321109in}}{\pgfqpoint{2.413976in}{1.324382in}}{\pgfqpoint{2.419800in}{1.330205in}}%
\pgfpathcurveto{\pgfqpoint{2.425624in}{1.336029in}}{\pgfqpoint{2.428896in}{1.343929in}}{\pgfqpoint{2.428896in}{1.352166in}}%
\pgfpathcurveto{\pgfqpoint{2.428896in}{1.360402in}}{\pgfqpoint{2.425624in}{1.368302in}}{\pgfqpoint{2.419800in}{1.374126in}}%
\pgfpathcurveto{\pgfqpoint{2.413976in}{1.379950in}}{\pgfqpoint{2.406076in}{1.383222in}}{\pgfqpoint{2.397839in}{1.383222in}}%
\pgfpathcurveto{\pgfqpoint{2.389603in}{1.383222in}}{\pgfqpoint{2.381703in}{1.379950in}}{\pgfqpoint{2.375879in}{1.374126in}}%
\pgfpathcurveto{\pgfqpoint{2.370055in}{1.368302in}}{\pgfqpoint{2.366783in}{1.360402in}}{\pgfqpoint{2.366783in}{1.352166in}}%
\pgfpathcurveto{\pgfqpoint{2.366783in}{1.343929in}}{\pgfqpoint{2.370055in}{1.336029in}}{\pgfqpoint{2.375879in}{1.330205in}}%
\pgfpathcurveto{\pgfqpoint{2.381703in}{1.324382in}}{\pgfqpoint{2.389603in}{1.321109in}}{\pgfqpoint{2.397839in}{1.321109in}}%
\pgfpathclose%
\pgfusepath{stroke,fill}%
\end{pgfscope}%
\begin{pgfscope}%
\pgfpathrectangle{\pgfqpoint{0.100000in}{0.212622in}}{\pgfqpoint{3.696000in}{3.696000in}}%
\pgfusepath{clip}%
\pgfsetbuttcap%
\pgfsetroundjoin%
\definecolor{currentfill}{rgb}{0.121569,0.466667,0.705882}%
\pgfsetfillcolor{currentfill}%
\pgfsetlinewidth{1.003750pt}%
\definecolor{currentstroke}{rgb}{0.121569,0.466667,0.705882}%
\pgfsetstrokecolor{currentstroke}%
\pgfsetdash{}{0pt}%
\pgfpathmoveto{\pgfqpoint{2.395697in}{1.322720in}}%
\pgfpathcurveto{\pgfqpoint{2.403933in}{1.322720in}}{\pgfqpoint{2.411833in}{1.325992in}}{\pgfqpoint{2.417657in}{1.331816in}}%
\pgfpathcurveto{\pgfqpoint{2.423481in}{1.337640in}}{\pgfqpoint{2.426753in}{1.345540in}}{\pgfqpoint{2.426753in}{1.353776in}}%
\pgfpathcurveto{\pgfqpoint{2.426753in}{1.362013in}}{\pgfqpoint{2.423481in}{1.369913in}}{\pgfqpoint{2.417657in}{1.375737in}}%
\pgfpathcurveto{\pgfqpoint{2.411833in}{1.381561in}}{\pgfqpoint{2.403933in}{1.384833in}}{\pgfqpoint{2.395697in}{1.384833in}}%
\pgfpathcurveto{\pgfqpoint{2.387460in}{1.384833in}}{\pgfqpoint{2.379560in}{1.381561in}}{\pgfqpoint{2.373736in}{1.375737in}}%
\pgfpathcurveto{\pgfqpoint{2.367912in}{1.369913in}}{\pgfqpoint{2.364640in}{1.362013in}}{\pgfqpoint{2.364640in}{1.353776in}}%
\pgfpathcurveto{\pgfqpoint{2.364640in}{1.345540in}}{\pgfqpoint{2.367912in}{1.337640in}}{\pgfqpoint{2.373736in}{1.331816in}}%
\pgfpathcurveto{\pgfqpoint{2.379560in}{1.325992in}}{\pgfqpoint{2.387460in}{1.322720in}}{\pgfqpoint{2.395697in}{1.322720in}}%
\pgfpathclose%
\pgfusepath{stroke,fill}%
\end{pgfscope}%
\begin{pgfscope}%
\pgfsetbuttcap%
\pgfsetmiterjoin%
\definecolor{currentfill}{rgb}{1.000000,1.000000,1.000000}%
\pgfsetfillcolor{currentfill}%
\pgfsetfillopacity{0.800000}%
\pgfsetlinewidth{1.003750pt}%
\definecolor{currentstroke}{rgb}{0.800000,0.800000,0.800000}%
\pgfsetstrokecolor{currentstroke}%
\pgfsetstrokeopacity{0.800000}%
\pgfsetdash{}{0pt}%
\pgfpathmoveto{\pgfqpoint{2.104889in}{3.216678in}}%
\pgfpathlineto{\pgfqpoint{3.698778in}{3.216678in}}%
\pgfpathquadraticcurveto{\pgfqpoint{3.726556in}{3.216678in}}{\pgfqpoint{3.726556in}{3.244456in}}%
\pgfpathlineto{\pgfqpoint{3.726556in}{3.811400in}}%
\pgfpathquadraticcurveto{\pgfqpoint{3.726556in}{3.839178in}}{\pgfqpoint{3.698778in}{3.839178in}}%
\pgfpathlineto{\pgfqpoint{2.104889in}{3.839178in}}%
\pgfpathquadraticcurveto{\pgfqpoint{2.077111in}{3.839178in}}{\pgfqpoint{2.077111in}{3.811400in}}%
\pgfpathlineto{\pgfqpoint{2.077111in}{3.244456in}}%
\pgfpathquadraticcurveto{\pgfqpoint{2.077111in}{3.216678in}}{\pgfqpoint{2.104889in}{3.216678in}}%
\pgfpathclose%
\pgfusepath{stroke,fill}%
\end{pgfscope}%
\begin{pgfscope}%
\pgfsetrectcap%
\pgfsetroundjoin%
\pgfsetlinewidth{1.505625pt}%
\definecolor{currentstroke}{rgb}{0.121569,0.466667,0.705882}%
\pgfsetstrokecolor{currentstroke}%
\pgfsetdash{}{0pt}%
\pgfpathmoveto{\pgfqpoint{2.132667in}{3.735011in}}%
\pgfpathlineto{\pgfqpoint{2.410444in}{3.735011in}}%
\pgfusepath{stroke}%
\end{pgfscope}%
\begin{pgfscope}%
\definecolor{textcolor}{rgb}{0.000000,0.000000,0.000000}%
\pgfsetstrokecolor{textcolor}%
\pgfsetfillcolor{textcolor}%
\pgftext[x=2.521555in,y=3.686400in,left,base]{\color{textcolor}\rmfamily\fontsize{10.000000}{12.000000}\selectfont Ground truth}%
\end{pgfscope}%
\begin{pgfscope}%
\pgfsetbuttcap%
\pgfsetroundjoin%
\definecolor{currentfill}{rgb}{0.121569,0.466667,0.705882}%
\pgfsetfillcolor{currentfill}%
\pgfsetlinewidth{1.003750pt}%
\definecolor{currentstroke}{rgb}{0.121569,0.466667,0.705882}%
\pgfsetstrokecolor{currentstroke}%
\pgfsetdash{}{0pt}%
\pgfsys@defobject{currentmarker}{\pgfqpoint{-0.031056in}{-0.031056in}}{\pgfqpoint{0.031056in}{0.031056in}}{%
\pgfpathmoveto{\pgfqpoint{0.000000in}{-0.031056in}}%
\pgfpathcurveto{\pgfqpoint{0.008236in}{-0.031056in}}{\pgfqpoint{0.016136in}{-0.027784in}}{\pgfqpoint{0.021960in}{-0.021960in}}%
\pgfpathcurveto{\pgfqpoint{0.027784in}{-0.016136in}}{\pgfqpoint{0.031056in}{-0.008236in}}{\pgfqpoint{0.031056in}{0.000000in}}%
\pgfpathcurveto{\pgfqpoint{0.031056in}{0.008236in}}{\pgfqpoint{0.027784in}{0.016136in}}{\pgfqpoint{0.021960in}{0.021960in}}%
\pgfpathcurveto{\pgfqpoint{0.016136in}{0.027784in}}{\pgfqpoint{0.008236in}{0.031056in}}{\pgfqpoint{0.000000in}{0.031056in}}%
\pgfpathcurveto{\pgfqpoint{-0.008236in}{0.031056in}}{\pgfqpoint{-0.016136in}{0.027784in}}{\pgfqpoint{-0.021960in}{0.021960in}}%
\pgfpathcurveto{\pgfqpoint{-0.027784in}{0.016136in}}{\pgfqpoint{-0.031056in}{0.008236in}}{\pgfqpoint{-0.031056in}{0.000000in}}%
\pgfpathcurveto{\pgfqpoint{-0.031056in}{-0.008236in}}{\pgfqpoint{-0.027784in}{-0.016136in}}{\pgfqpoint{-0.021960in}{-0.021960in}}%
\pgfpathcurveto{\pgfqpoint{-0.016136in}{-0.027784in}}{\pgfqpoint{-0.008236in}{-0.031056in}}{\pgfqpoint{0.000000in}{-0.031056in}}%
\pgfpathclose%
\pgfusepath{stroke,fill}%
}%
\begin{pgfscope}%
\pgfsys@transformshift{2.271555in}{3.529248in}%
\pgfsys@useobject{currentmarker}{}%
\end{pgfscope}%
\end{pgfscope}%
\begin{pgfscope}%
\definecolor{textcolor}{rgb}{0.000000,0.000000,0.000000}%
\pgfsetstrokecolor{textcolor}%
\pgfsetfillcolor{textcolor}%
\pgftext[x=2.521555in,y=3.492789in,left,base]{\color{textcolor}\rmfamily\fontsize{10.000000}{12.000000}\selectfont Estimated position}%
\end{pgfscope}%
\begin{pgfscope}%
\pgfsetbuttcap%
\pgfsetroundjoin%
\definecolor{currentfill}{rgb}{1.000000,0.498039,0.054902}%
\pgfsetfillcolor{currentfill}%
\pgfsetlinewidth{1.003750pt}%
\definecolor{currentstroke}{rgb}{1.000000,0.498039,0.054902}%
\pgfsetstrokecolor{currentstroke}%
\pgfsetdash{}{0pt}%
\pgfsys@defobject{currentmarker}{\pgfqpoint{-0.031056in}{-0.031056in}}{\pgfqpoint{0.031056in}{0.031056in}}{%
\pgfpathmoveto{\pgfqpoint{0.000000in}{-0.031056in}}%
\pgfpathcurveto{\pgfqpoint{0.008236in}{-0.031056in}}{\pgfqpoint{0.016136in}{-0.027784in}}{\pgfqpoint{0.021960in}{-0.021960in}}%
\pgfpathcurveto{\pgfqpoint{0.027784in}{-0.016136in}}{\pgfqpoint{0.031056in}{-0.008236in}}{\pgfqpoint{0.031056in}{0.000000in}}%
\pgfpathcurveto{\pgfqpoint{0.031056in}{0.008236in}}{\pgfqpoint{0.027784in}{0.016136in}}{\pgfqpoint{0.021960in}{0.021960in}}%
\pgfpathcurveto{\pgfqpoint{0.016136in}{0.027784in}}{\pgfqpoint{0.008236in}{0.031056in}}{\pgfqpoint{0.000000in}{0.031056in}}%
\pgfpathcurveto{\pgfqpoint{-0.008236in}{0.031056in}}{\pgfqpoint{-0.016136in}{0.027784in}}{\pgfqpoint{-0.021960in}{0.021960in}}%
\pgfpathcurveto{\pgfqpoint{-0.027784in}{0.016136in}}{\pgfqpoint{-0.031056in}{0.008236in}}{\pgfqpoint{-0.031056in}{0.000000in}}%
\pgfpathcurveto{\pgfqpoint{-0.031056in}{-0.008236in}}{\pgfqpoint{-0.027784in}{-0.016136in}}{\pgfqpoint{-0.021960in}{-0.021960in}}%
\pgfpathcurveto{\pgfqpoint{-0.016136in}{-0.027784in}}{\pgfqpoint{-0.008236in}{-0.031056in}}{\pgfqpoint{0.000000in}{-0.031056in}}%
\pgfpathclose%
\pgfusepath{stroke,fill}%
}%
\begin{pgfscope}%
\pgfsys@transformshift{2.271555in}{3.335637in}%
\pgfsys@useobject{currentmarker}{}%
\end{pgfscope}%
\end{pgfscope}%
\begin{pgfscope}%
\definecolor{textcolor}{rgb}{0.000000,0.000000,0.000000}%
\pgfsetstrokecolor{textcolor}%
\pgfsetfillcolor{textcolor}%
\pgftext[x=2.521555in,y=3.299178in,left,base]{\color{textcolor}\rmfamily\fontsize{10.000000}{12.000000}\selectfont Estimated turn}%
\end{pgfscope}%
\end{pgfpicture}%
\makeatother%
\endgroup%
}
%         \caption{MPU-9250 Breakout}
%         \label{fig:triangle16_3D}
%     \end{subfigure}
%     \caption{Position estimation by the best performing algorithms in the 4-meter line experiment.}
%     \label{fig:triangle16}
% \end{figure}

% % \subsubsection{28 meter}

% % For the 16-meter line experiment, the Mahony algorithm which had the lowest displacement error with an average of 0.48 meters (16\% of error margin), and ROLEQ with an average of 0.24 meters of turn error (7\% of error margin).

% % \begin{figure}[!h]
% %     \centering
% %     \begin{table}[H]
    \begin{center}
        \resizebox{1\linewidth}{!}{

            \begin{tabular}[t]{lcccc}
                \hline
                Algorithm     & Displacement Error[$m$] & Displacement Error[\%] & Turn Error[$m$] & Turn Error[\%] \\
                \hline
                AngularRate   & 22.91                   & 27.28                  & 48.97           & 58.29          \\
                AQUA          & 6.77                    & 8.06                   & 15.53           & 18.49          \\
                Complementary & 5.69                    & 6.78                   & 10.22           & 12.16          \\
                Davenport     & 3.93                    & 4.68                   & 5.06            & 6.02           \\
                EKF           & 3.52                    & 4.19                   & 5.04            & 5.99           \\
                FAMC          & 21.07                   & 25.08                  & 46.74           & 55.64          \\
                FLAE          & 3.88                    & 4.62                   & 4.98            & 5.93           \\
                Fourati       & 35.63                   & 42.41                  & 53.27           & 63.42          \\
                Madgwick      & 15.58                   & 18.55                  & 17.78           & 21.16          \\
                Mahony        & 2.93                    & 3.49                   & 4.69            & 5.59           \\
                OLEQ          & 4.69                    & 5.58                   & 5.05            & 6.01           \\
                QUEST         & 15.18                   & 18.08                  & 30.32           & 36.09          \\
                ROLEQ         & 4.38                    & 5.22                   & 5.40            & 6.43           \\
                SAAM          & 2.93                    & 3.48                   & 4.54            & 5.40           \\
                Tilt          & 2.93                    & 3.48                   & 4.54            & 5.40           \\

                \hline
                Average       & 10.14                   & 12.07                  & 17.47           & 20.80
            \end{tabular}
        }
        \caption{28 meter square position estimation error (displacement and turn) of the sensor fusion algorithms. }
        \label{tab:28_triangle}
    \end{center}
\end{table}
% % \end{figure}

% % \begin{figure}[!h]
% %     \centering
% %     \begin{subfigure}{0.49\textwidth}
% %         \centering
% %         \resizebox{1\linewidth}{!}{%% Creator: Matplotlib, PGF backend
%%
%% To include the figure in your LaTeX document, write
%%   \input{<filename>.pgf}
%%
%% Make sure the required packages are loaded in your preamble
%%   \usepackage{pgf}
%%
%% and, on pdftex
%%   \usepackage[utf8]{inputenc}\DeclareUnicodeCharacter{2212}{-}
%%
%% or, on luatex and xetex
%%   \usepackage{unicode-math}
%%
%% Figures using additional raster images can only be included by \input if
%% they are in the same directory as the main LaTeX file. For loading figures
%% from other directories you can use the `import` package
%%   \usepackage{import}
%%
%% and then include the figures with
%%   \import{<path to file>}{<filename>.pgf}
%%
%% Matplotlib used the following preamble
%%   \usepackage{fontspec}
%%
\begingroup%
\makeatletter%
\begin{pgfpicture}%
\pgfpathrectangle{\pgfpointorigin}{\pgfqpoint{5.590556in}{4.322338in}}%
\pgfusepath{use as bounding box, clip}%
\begin{pgfscope}%
\pgfsetbuttcap%
\pgfsetmiterjoin%
\definecolor{currentfill}{rgb}{1.000000,1.000000,1.000000}%
\pgfsetfillcolor{currentfill}%
\pgfsetlinewidth{0.000000pt}%
\definecolor{currentstroke}{rgb}{1.000000,1.000000,1.000000}%
\pgfsetstrokecolor{currentstroke}%
\pgfsetdash{}{0pt}%
\pgfpathmoveto{\pgfqpoint{0.000000in}{0.000000in}}%
\pgfpathlineto{\pgfqpoint{5.590556in}{0.000000in}}%
\pgfpathlineto{\pgfqpoint{5.590556in}{4.322338in}}%
\pgfpathlineto{\pgfqpoint{0.000000in}{4.322338in}}%
\pgfpathclose%
\pgfusepath{fill}%
\end{pgfscope}%
\begin{pgfscope}%
\pgfsetbuttcap%
\pgfsetmiterjoin%
\definecolor{currentfill}{rgb}{1.000000,1.000000,1.000000}%
\pgfsetfillcolor{currentfill}%
\pgfsetlinewidth{0.000000pt}%
\definecolor{currentstroke}{rgb}{0.000000,0.000000,0.000000}%
\pgfsetstrokecolor{currentstroke}%
\pgfsetstrokeopacity{0.000000}%
\pgfsetdash{}{0pt}%
\pgfpathmoveto{\pgfqpoint{0.530556in}{0.515000in}}%
\pgfpathlineto{\pgfqpoint{5.490556in}{0.515000in}}%
\pgfpathlineto{\pgfqpoint{5.490556in}{4.211000in}}%
\pgfpathlineto{\pgfqpoint{0.530556in}{4.211000in}}%
\pgfpathclose%
\pgfusepath{fill}%
\end{pgfscope}%
\begin{pgfscope}%
\pgfpathrectangle{\pgfqpoint{0.530556in}{0.515000in}}{\pgfqpoint{4.960000in}{3.696000in}}%
\pgfusepath{clip}%
\pgfsetbuttcap%
\pgfsetroundjoin%
\definecolor{currentfill}{rgb}{0.121569,0.466667,0.705882}%
\pgfsetfillcolor{currentfill}%
\pgfsetlinewidth{1.003750pt}%
\definecolor{currentstroke}{rgb}{0.121569,0.466667,0.705882}%
\pgfsetstrokecolor{currentstroke}%
\pgfsetdash{}{0pt}%
\pgfsys@defobject{currentmarker}{\pgfqpoint{-0.041667in}{-0.041667in}}{\pgfqpoint{0.041667in}{0.041667in}}{%
\pgfpathmoveto{\pgfqpoint{0.000000in}{-0.041667in}}%
\pgfpathcurveto{\pgfqpoint{0.011050in}{-0.041667in}}{\pgfqpoint{0.021649in}{-0.037276in}}{\pgfqpoint{0.029463in}{-0.029463in}}%
\pgfpathcurveto{\pgfqpoint{0.037276in}{-0.021649in}}{\pgfqpoint{0.041667in}{-0.011050in}}{\pgfqpoint{0.041667in}{0.000000in}}%
\pgfpathcurveto{\pgfqpoint{0.041667in}{0.011050in}}{\pgfqpoint{0.037276in}{0.021649in}}{\pgfqpoint{0.029463in}{0.029463in}}%
\pgfpathcurveto{\pgfqpoint{0.021649in}{0.037276in}}{\pgfqpoint{0.011050in}{0.041667in}}{\pgfqpoint{0.000000in}{0.041667in}}%
\pgfpathcurveto{\pgfqpoint{-0.011050in}{0.041667in}}{\pgfqpoint{-0.021649in}{0.037276in}}{\pgfqpoint{-0.029463in}{0.029463in}}%
\pgfpathcurveto{\pgfqpoint{-0.037276in}{0.021649in}}{\pgfqpoint{-0.041667in}{0.011050in}}{\pgfqpoint{-0.041667in}{0.000000in}}%
\pgfpathcurveto{\pgfqpoint{-0.041667in}{-0.011050in}}{\pgfqpoint{-0.037276in}{-0.021649in}}{\pgfqpoint{-0.029463in}{-0.029463in}}%
\pgfpathcurveto{\pgfqpoint{-0.021649in}{-0.037276in}}{\pgfqpoint{-0.011050in}{-0.041667in}}{\pgfqpoint{0.000000in}{-0.041667in}}%
\pgfpathclose%
\pgfusepath{stroke,fill}%
}%
\begin{pgfscope}%
\pgfsys@transformshift{0.894919in}{0.998511in}%
\pgfsys@useobject{currentmarker}{}%
\end{pgfscope}%
\begin{pgfscope}%
\pgfsys@transformshift{0.894089in}{0.995309in}%
\pgfsys@useobject{currentmarker}{}%
\end{pgfscope}%
\begin{pgfscope}%
\pgfsys@transformshift{0.893668in}{0.991294in}%
\pgfsys@useobject{currentmarker}{}%
\end{pgfscope}%
\begin{pgfscope}%
\pgfsys@transformshift{0.893129in}{0.985671in}%
\pgfsys@useobject{currentmarker}{}%
\end{pgfscope}%
\begin{pgfscope}%
\pgfsys@transformshift{0.894545in}{0.979533in}%
\pgfsys@useobject{currentmarker}{}%
\end{pgfscope}%
\begin{pgfscope}%
\pgfsys@transformshift{0.896936in}{0.972465in}%
\pgfsys@useobject{currentmarker}{}%
\end{pgfscope}%
\begin{pgfscope}%
\pgfsys@transformshift{0.898726in}{0.964249in}%
\pgfsys@useobject{currentmarker}{}%
\end{pgfscope}%
\begin{pgfscope}%
\pgfsys@transformshift{0.901669in}{0.955447in}%
\pgfsys@useobject{currentmarker}{}%
\end{pgfscope}%
\begin{pgfscope}%
\pgfsys@transformshift{0.903050in}{0.950532in}%
\pgfsys@useobject{currentmarker}{}%
\end{pgfscope}%
\begin{pgfscope}%
\pgfsys@transformshift{0.904986in}{0.944951in}%
\pgfsys@useobject{currentmarker}{}%
\end{pgfscope}%
\begin{pgfscope}%
\pgfsys@transformshift{0.906822in}{0.938634in}%
\pgfsys@useobject{currentmarker}{}%
\end{pgfscope}%
\begin{pgfscope}%
\pgfsys@transformshift{0.907789in}{0.935148in}%
\pgfsys@useobject{currentmarker}{}%
\end{pgfscope}%
\begin{pgfscope}%
\pgfsys@transformshift{0.908291in}{0.933222in}%
\pgfsys@useobject{currentmarker}{}%
\end{pgfscope}%
\begin{pgfscope}%
\pgfsys@transformshift{0.908971in}{0.930700in}%
\pgfsys@useobject{currentmarker}{}%
\end{pgfscope}%
\begin{pgfscope}%
\pgfsys@transformshift{0.909409in}{0.929331in}%
\pgfsys@useobject{currentmarker}{}%
\end{pgfscope}%
\begin{pgfscope}%
\pgfsys@transformshift{0.910123in}{0.926903in}%
\pgfsys@useobject{currentmarker}{}%
\end{pgfscope}%
\begin{pgfscope}%
\pgfsys@transformshift{0.910515in}{0.925567in}%
\pgfsys@useobject{currentmarker}{}%
\end{pgfscope}%
\begin{pgfscope}%
\pgfsys@transformshift{0.910754in}{0.924840in}%
\pgfsys@useobject{currentmarker}{}%
\end{pgfscope}%
\begin{pgfscope}%
\pgfsys@transformshift{0.910846in}{0.924429in}%
\pgfsys@useobject{currentmarker}{}%
\end{pgfscope}%
\begin{pgfscope}%
\pgfsys@transformshift{0.910915in}{0.924208in}%
\pgfsys@useobject{currentmarker}{}%
\end{pgfscope}%
\begin{pgfscope}%
\pgfsys@transformshift{0.910947in}{0.924085in}%
\pgfsys@useobject{currentmarker}{}%
\end{pgfscope}%
\begin{pgfscope}%
\pgfsys@transformshift{0.910967in}{0.924017in}%
\pgfsys@useobject{currentmarker}{}%
\end{pgfscope}%
\begin{pgfscope}%
\pgfsys@transformshift{0.910972in}{0.923979in}%
\pgfsys@useobject{currentmarker}{}%
\end{pgfscope}%
\begin{pgfscope}%
\pgfsys@transformshift{0.910978in}{0.923959in}%
\pgfsys@useobject{currentmarker}{}%
\end{pgfscope}%
\begin{pgfscope}%
\pgfsys@transformshift{0.911185in}{0.922967in}%
\pgfsys@useobject{currentmarker}{}%
\end{pgfscope}%
\begin{pgfscope}%
\pgfsys@transformshift{0.911318in}{0.922425in}%
\pgfsys@useobject{currentmarker}{}%
\end{pgfscope}%
\begin{pgfscope}%
\pgfsys@transformshift{0.911516in}{0.921203in}%
\pgfsys@useobject{currentmarker}{}%
\end{pgfscope}%
\begin{pgfscope}%
\pgfsys@transformshift{0.912006in}{0.919269in}%
\pgfsys@useobject{currentmarker}{}%
\end{pgfscope}%
\begin{pgfscope}%
\pgfsys@transformshift{0.912290in}{0.918209in}%
\pgfsys@useobject{currentmarker}{}%
\end{pgfscope}%
\begin{pgfscope}%
\pgfsys@transformshift{0.912763in}{0.916500in}%
\pgfsys@useobject{currentmarker}{}%
\end{pgfscope}%
\begin{pgfscope}%
\pgfsys@transformshift{0.913016in}{0.915558in}%
\pgfsys@useobject{currentmarker}{}%
\end{pgfscope}%
\begin{pgfscope}%
\pgfsys@transformshift{0.913145in}{0.915037in}%
\pgfsys@useobject{currentmarker}{}%
\end{pgfscope}%
\begin{pgfscope}%
\pgfsys@transformshift{0.913215in}{0.914750in}%
\pgfsys@useobject{currentmarker}{}%
\end{pgfscope}%
\begin{pgfscope}%
\pgfsys@transformshift{0.913469in}{0.913797in}%
\pgfsys@useobject{currentmarker}{}%
\end{pgfscope}%
\begin{pgfscope}%
\pgfsys@transformshift{0.913805in}{0.912004in}%
\pgfsys@useobject{currentmarker}{}%
\end{pgfscope}%
\begin{pgfscope}%
\pgfsys@transformshift{0.914540in}{0.909764in}%
\pgfsys@useobject{currentmarker}{}%
\end{pgfscope}%
\begin{pgfscope}%
\pgfsys@transformshift{0.914735in}{0.908482in}%
\pgfsys@useobject{currentmarker}{}%
\end{pgfscope}%
\begin{pgfscope}%
\pgfsys@transformshift{0.915337in}{0.906316in}%
\pgfsys@useobject{currentmarker}{}%
\end{pgfscope}%
\begin{pgfscope}%
\pgfsys@transformshift{0.915629in}{0.905114in}%
\pgfsys@useobject{currentmarker}{}%
\end{pgfscope}%
\begin{pgfscope}%
\pgfsys@transformshift{0.915817in}{0.904461in}%
\pgfsys@useobject{currentmarker}{}%
\end{pgfscope}%
\begin{pgfscope}%
\pgfsys@transformshift{0.916374in}{0.902775in}%
\pgfsys@useobject{currentmarker}{}%
\end{pgfscope}%
\begin{pgfscope}%
\pgfsys@transformshift{0.916611in}{0.901828in}%
\pgfsys@useobject{currentmarker}{}%
\end{pgfscope}%
\begin{pgfscope}%
\pgfsys@transformshift{0.917028in}{0.900230in}%
\pgfsys@useobject{currentmarker}{}%
\end{pgfscope}%
\begin{pgfscope}%
\pgfsys@transformshift{0.917255in}{0.899350in}%
\pgfsys@useobject{currentmarker}{}%
\end{pgfscope}%
\begin{pgfscope}%
\pgfsys@transformshift{0.917390in}{0.898869in}%
\pgfsys@useobject{currentmarker}{}%
\end{pgfscope}%
\begin{pgfscope}%
\pgfsys@transformshift{0.917773in}{0.897861in}%
\pgfsys@useobject{currentmarker}{}%
\end{pgfscope}%
\begin{pgfscope}%
\pgfsys@transformshift{0.917876in}{0.897277in}%
\pgfsys@useobject{currentmarker}{}%
\end{pgfscope}%
\begin{pgfscope}%
\pgfsys@transformshift{0.918305in}{0.896148in}%
\pgfsys@useobject{currentmarker}{}%
\end{pgfscope}%
\begin{pgfscope}%
\pgfsys@transformshift{0.918491in}{0.895509in}%
\pgfsys@useobject{currentmarker}{}%
\end{pgfscope}%
\begin{pgfscope}%
\pgfsys@transformshift{0.918740in}{0.894336in}%
\pgfsys@useobject{currentmarker}{}%
\end{pgfscope}%
\begin{pgfscope}%
\pgfsys@transformshift{0.919221in}{0.892646in}%
\pgfsys@useobject{currentmarker}{}%
\end{pgfscope}%
\begin{pgfscope}%
\pgfsys@transformshift{0.919471in}{0.891713in}%
\pgfsys@useobject{currentmarker}{}%
\end{pgfscope}%
\begin{pgfscope}%
\pgfsys@transformshift{0.919882in}{0.890264in}%
\pgfsys@useobject{currentmarker}{}%
\end{pgfscope}%
\begin{pgfscope}%
\pgfsys@transformshift{0.920550in}{0.888296in}%
\pgfsys@useobject{currentmarker}{}%
\end{pgfscope}%
\begin{pgfscope}%
\pgfsys@transformshift{0.920911in}{0.887211in}%
\pgfsys@useobject{currentmarker}{}%
\end{pgfscope}%
\begin{pgfscope}%
\pgfsys@transformshift{0.921083in}{0.886607in}%
\pgfsys@useobject{currentmarker}{}%
\end{pgfscope}%
\begin{pgfscope}%
\pgfsys@transformshift{0.921186in}{0.886276in}%
\pgfsys@useobject{currentmarker}{}%
\end{pgfscope}%
\begin{pgfscope}%
\pgfsys@transformshift{0.921501in}{0.885391in}%
\pgfsys@useobject{currentmarker}{}%
\end{pgfscope}%
\begin{pgfscope}%
\pgfsys@transformshift{0.921648in}{0.884895in}%
\pgfsys@useobject{currentmarker}{}%
\end{pgfscope}%
\begin{pgfscope}%
\pgfsys@transformshift{0.921905in}{0.883597in}%
\pgfsys@useobject{currentmarker}{}%
\end{pgfscope}%
\begin{pgfscope}%
\pgfsys@transformshift{0.922158in}{0.882915in}%
\pgfsys@useobject{currentmarker}{}%
\end{pgfscope}%
\begin{pgfscope}%
\pgfsys@transformshift{0.922629in}{0.881592in}%
\pgfsys@useobject{currentmarker}{}%
\end{pgfscope}%
\begin{pgfscope}%
\pgfsys@transformshift{0.923222in}{0.879515in}%
\pgfsys@useobject{currentmarker}{}%
\end{pgfscope}%
\begin{pgfscope}%
\pgfsys@transformshift{0.924107in}{0.876496in}%
\pgfsys@useobject{currentmarker}{}%
\end{pgfscope}%
\begin{pgfscope}%
\pgfsys@transformshift{0.925429in}{0.872523in}%
\pgfsys@useobject{currentmarker}{}%
\end{pgfscope}%
\begin{pgfscope}%
\pgfsys@transformshift{0.926044in}{0.870303in}%
\pgfsys@useobject{currentmarker}{}%
\end{pgfscope}%
\begin{pgfscope}%
\pgfsys@transformshift{0.926461in}{0.869107in}%
\pgfsys@useobject{currentmarker}{}%
\end{pgfscope}%
\begin{pgfscope}%
\pgfsys@transformshift{0.926713in}{0.868458in}%
\pgfsys@useobject{currentmarker}{}%
\end{pgfscope}%
\begin{pgfscope}%
\pgfsys@transformshift{0.926850in}{0.868100in}%
\pgfsys@useobject{currentmarker}{}%
\end{pgfscope}%
\begin{pgfscope}%
\pgfsys@transformshift{0.926922in}{0.867902in}%
\pgfsys@useobject{currentmarker}{}%
\end{pgfscope}%
\begin{pgfscope}%
\pgfsys@transformshift{0.926944in}{0.867788in}%
\pgfsys@useobject{currentmarker}{}%
\end{pgfscope}%
\begin{pgfscope}%
\pgfsys@transformshift{0.926969in}{0.867729in}%
\pgfsys@useobject{currentmarker}{}%
\end{pgfscope}%
\begin{pgfscope}%
\pgfsys@transformshift{0.926984in}{0.867698in}%
\pgfsys@useobject{currentmarker}{}%
\end{pgfscope}%
\begin{pgfscope}%
\pgfsys@transformshift{0.926992in}{0.867680in}%
\pgfsys@useobject{currentmarker}{}%
\end{pgfscope}%
\begin{pgfscope}%
\pgfsys@transformshift{0.926996in}{0.867670in}%
\pgfsys@useobject{currentmarker}{}%
\end{pgfscope}%
\begin{pgfscope}%
\pgfsys@transformshift{0.926998in}{0.867665in}%
\pgfsys@useobject{currentmarker}{}%
\end{pgfscope}%
\begin{pgfscope}%
\pgfsys@transformshift{0.926999in}{0.867662in}%
\pgfsys@useobject{currentmarker}{}%
\end{pgfscope}%
\begin{pgfscope}%
\pgfsys@transformshift{0.927092in}{0.867053in}%
\pgfsys@useobject{currentmarker}{}%
\end{pgfscope}%
\begin{pgfscope}%
\pgfsys@transformshift{0.927150in}{0.866719in}%
\pgfsys@useobject{currentmarker}{}%
\end{pgfscope}%
\begin{pgfscope}%
\pgfsys@transformshift{0.927408in}{0.865827in}%
\pgfsys@useobject{currentmarker}{}%
\end{pgfscope}%
\begin{pgfscope}%
\pgfsys@transformshift{0.927473in}{0.865320in}%
\pgfsys@useobject{currentmarker}{}%
\end{pgfscope}%
\begin{pgfscope}%
\pgfsys@transformshift{0.927522in}{0.865044in}%
\pgfsys@useobject{currentmarker}{}%
\end{pgfscope}%
\begin{pgfscope}%
\pgfsys@transformshift{0.927537in}{0.864890in}%
\pgfsys@useobject{currentmarker}{}%
\end{pgfscope}%
\begin{pgfscope}%
\pgfsys@transformshift{0.927551in}{0.864806in}%
\pgfsys@useobject{currentmarker}{}%
\end{pgfscope}%
\begin{pgfscope}%
\pgfsys@transformshift{0.927558in}{0.864760in}%
\pgfsys@useobject{currentmarker}{}%
\end{pgfscope}%
\begin{pgfscope}%
\pgfsys@transformshift{0.927564in}{0.864735in}%
\pgfsys@useobject{currentmarker}{}%
\end{pgfscope}%
\begin{pgfscope}%
\pgfsys@transformshift{0.927566in}{0.864721in}%
\pgfsys@useobject{currentmarker}{}%
\end{pgfscope}%
\begin{pgfscope}%
\pgfsys@transformshift{0.927568in}{0.864713in}%
\pgfsys@useobject{currentmarker}{}%
\end{pgfscope}%
\begin{pgfscope}%
\pgfsys@transformshift{0.927666in}{0.864164in}%
\pgfsys@useobject{currentmarker}{}%
\end{pgfscope}%
\begin{pgfscope}%
\pgfsys@transformshift{0.927715in}{0.863862in}%
\pgfsys@useobject{currentmarker}{}%
\end{pgfscope}%
\begin{pgfscope}%
\pgfsys@transformshift{0.927760in}{0.863699in}%
\pgfsys@useobject{currentmarker}{}%
\end{pgfscope}%
\begin{pgfscope}%
\pgfsys@transformshift{0.927770in}{0.863607in}%
\pgfsys@useobject{currentmarker}{}%
\end{pgfscope}%
\begin{pgfscope}%
\pgfsys@transformshift{0.927780in}{0.863557in}%
\pgfsys@useobject{currentmarker}{}%
\end{pgfscope}%
\begin{pgfscope}%
\pgfsys@transformshift{0.927784in}{0.863529in}%
\pgfsys@useobject{currentmarker}{}%
\end{pgfscope}%
\begin{pgfscope}%
\pgfsys@transformshift{0.927788in}{0.863514in}%
\pgfsys@useobject{currentmarker}{}%
\end{pgfscope}%
\begin{pgfscope}%
\pgfsys@transformshift{0.927914in}{0.862701in}%
\pgfsys@useobject{currentmarker}{}%
\end{pgfscope}%
\begin{pgfscope}%
\pgfsys@transformshift{0.928001in}{0.862257in}%
\pgfsys@useobject{currentmarker}{}%
\end{pgfscope}%
\begin{pgfscope}%
\pgfsys@transformshift{0.928154in}{0.861173in}%
\pgfsys@useobject{currentmarker}{}%
\end{pgfscope}%
\begin{pgfscope}%
\pgfsys@transformshift{0.928298in}{0.860589in}%
\pgfsys@useobject{currentmarker}{}%
\end{pgfscope}%
\begin{pgfscope}%
\pgfsys@transformshift{0.928371in}{0.860266in}%
\pgfsys@useobject{currentmarker}{}%
\end{pgfscope}%
\begin{pgfscope}%
\pgfsys@transformshift{0.928405in}{0.860087in}%
\pgfsys@useobject{currentmarker}{}%
\end{pgfscope}%
\begin{pgfscope}%
\pgfsys@transformshift{0.928438in}{0.859992in}%
\pgfsys@useobject{currentmarker}{}%
\end{pgfscope}%
\begin{pgfscope}%
\pgfsys@transformshift{0.928448in}{0.859938in}%
\pgfsys@useobject{currentmarker}{}%
\end{pgfscope}%
\begin{pgfscope}%
\pgfsys@transformshift{0.928456in}{0.859909in}%
\pgfsys@useobject{currentmarker}{}%
\end{pgfscope}%
\begin{pgfscope}%
\pgfsys@transformshift{0.928460in}{0.859892in}%
\pgfsys@useobject{currentmarker}{}%
\end{pgfscope}%
\begin{pgfscope}%
\pgfsys@transformshift{0.928461in}{0.859883in}%
\pgfsys@useobject{currentmarker}{}%
\end{pgfscope}%
\begin{pgfscope}%
\pgfsys@transformshift{0.928463in}{0.859879in}%
\pgfsys@useobject{currentmarker}{}%
\end{pgfscope}%
\begin{pgfscope}%
\pgfsys@transformshift{0.928464in}{0.859876in}%
\pgfsys@useobject{currentmarker}{}%
\end{pgfscope}%
\begin{pgfscope}%
\pgfsys@transformshift{0.928464in}{0.859874in}%
\pgfsys@useobject{currentmarker}{}%
\end{pgfscope}%
\begin{pgfscope}%
\pgfsys@transformshift{0.928464in}{0.859874in}%
\pgfsys@useobject{currentmarker}{}%
\end{pgfscope}%
\begin{pgfscope}%
\pgfsys@transformshift{0.928464in}{0.859873in}%
\pgfsys@useobject{currentmarker}{}%
\end{pgfscope}%
\begin{pgfscope}%
\pgfsys@transformshift{0.928464in}{0.859873in}%
\pgfsys@useobject{currentmarker}{}%
\end{pgfscope}%
\begin{pgfscope}%
\pgfsys@transformshift{0.928464in}{0.859873in}%
\pgfsys@useobject{currentmarker}{}%
\end{pgfscope}%
\begin{pgfscope}%
\pgfsys@transformshift{0.928464in}{0.859873in}%
\pgfsys@useobject{currentmarker}{}%
\end{pgfscope}%
\begin{pgfscope}%
\pgfsys@transformshift{0.928464in}{0.859873in}%
\pgfsys@useobject{currentmarker}{}%
\end{pgfscope}%
\begin{pgfscope}%
\pgfsys@transformshift{0.928464in}{0.859873in}%
\pgfsys@useobject{currentmarker}{}%
\end{pgfscope}%
\begin{pgfscope}%
\pgfsys@transformshift{0.928464in}{0.859873in}%
\pgfsys@useobject{currentmarker}{}%
\end{pgfscope}%
\begin{pgfscope}%
\pgfsys@transformshift{0.928464in}{0.859873in}%
\pgfsys@useobject{currentmarker}{}%
\end{pgfscope}%
\begin{pgfscope}%
\pgfsys@transformshift{0.928464in}{0.859873in}%
\pgfsys@useobject{currentmarker}{}%
\end{pgfscope}%
\begin{pgfscope}%
\pgfsys@transformshift{0.928464in}{0.859873in}%
\pgfsys@useobject{currentmarker}{}%
\end{pgfscope}%
\begin{pgfscope}%
\pgfsys@transformshift{0.928464in}{0.859873in}%
\pgfsys@useobject{currentmarker}{}%
\end{pgfscope}%
\begin{pgfscope}%
\pgfsys@transformshift{0.928464in}{0.859873in}%
\pgfsys@useobject{currentmarker}{}%
\end{pgfscope}%
\begin{pgfscope}%
\pgfsys@transformshift{0.928464in}{0.859873in}%
\pgfsys@useobject{currentmarker}{}%
\end{pgfscope}%
\begin{pgfscope}%
\pgfsys@transformshift{0.928464in}{0.859873in}%
\pgfsys@useobject{currentmarker}{}%
\end{pgfscope}%
\begin{pgfscope}%
\pgfsys@transformshift{0.928464in}{0.859873in}%
\pgfsys@useobject{currentmarker}{}%
\end{pgfscope}%
\begin{pgfscope}%
\pgfsys@transformshift{0.928464in}{0.859873in}%
\pgfsys@useobject{currentmarker}{}%
\end{pgfscope}%
\begin{pgfscope}%
\pgfsys@transformshift{0.928464in}{0.859873in}%
\pgfsys@useobject{currentmarker}{}%
\end{pgfscope}%
\begin{pgfscope}%
\pgfsys@transformshift{0.928464in}{0.859873in}%
\pgfsys@useobject{currentmarker}{}%
\end{pgfscope}%
\begin{pgfscope}%
\pgfsys@transformshift{0.928464in}{0.859873in}%
\pgfsys@useobject{currentmarker}{}%
\end{pgfscope}%
\begin{pgfscope}%
\pgfsys@transformshift{0.928464in}{0.859873in}%
\pgfsys@useobject{currentmarker}{}%
\end{pgfscope}%
\begin{pgfscope}%
\pgfsys@transformshift{0.928464in}{0.859873in}%
\pgfsys@useobject{currentmarker}{}%
\end{pgfscope}%
\begin{pgfscope}%
\pgfsys@transformshift{0.928464in}{0.859873in}%
\pgfsys@useobject{currentmarker}{}%
\end{pgfscope}%
\begin{pgfscope}%
\pgfsys@transformshift{0.928464in}{0.859873in}%
\pgfsys@useobject{currentmarker}{}%
\end{pgfscope}%
\begin{pgfscope}%
\pgfsys@transformshift{0.928464in}{0.859873in}%
\pgfsys@useobject{currentmarker}{}%
\end{pgfscope}%
\begin{pgfscope}%
\pgfsys@transformshift{0.928464in}{0.859873in}%
\pgfsys@useobject{currentmarker}{}%
\end{pgfscope}%
\begin{pgfscope}%
\pgfsys@transformshift{0.928464in}{0.859873in}%
\pgfsys@useobject{currentmarker}{}%
\end{pgfscope}%
\begin{pgfscope}%
\pgfsys@transformshift{0.928464in}{0.859873in}%
\pgfsys@useobject{currentmarker}{}%
\end{pgfscope}%
\begin{pgfscope}%
\pgfsys@transformshift{0.928464in}{0.859873in}%
\pgfsys@useobject{currentmarker}{}%
\end{pgfscope}%
\begin{pgfscope}%
\pgfsys@transformshift{0.928464in}{0.859873in}%
\pgfsys@useobject{currentmarker}{}%
\end{pgfscope}%
\begin{pgfscope}%
\pgfsys@transformshift{0.928464in}{0.859873in}%
\pgfsys@useobject{currentmarker}{}%
\end{pgfscope}%
\begin{pgfscope}%
\pgfsys@transformshift{0.928464in}{0.859873in}%
\pgfsys@useobject{currentmarker}{}%
\end{pgfscope}%
\begin{pgfscope}%
\pgfsys@transformshift{0.928464in}{0.859873in}%
\pgfsys@useobject{currentmarker}{}%
\end{pgfscope}%
\begin{pgfscope}%
\pgfsys@transformshift{0.928084in}{0.859466in}%
\pgfsys@useobject{currentmarker}{}%
\end{pgfscope}%
\begin{pgfscope}%
\pgfsys@transformshift{0.927832in}{0.859293in}%
\pgfsys@useobject{currentmarker}{}%
\end{pgfscope}%
\begin{pgfscope}%
\pgfsys@transformshift{0.925253in}{0.859066in}%
\pgfsys@useobject{currentmarker}{}%
\end{pgfscope}%
\begin{pgfscope}%
\pgfsys@transformshift{0.923999in}{0.859741in}%
\pgfsys@useobject{currentmarker}{}%
\end{pgfscope}%
\begin{pgfscope}%
\pgfsys@transformshift{0.923693in}{0.860462in}%
\pgfsys@useobject{currentmarker}{}%
\end{pgfscope}%
\begin{pgfscope}%
\pgfsys@transformshift{0.923650in}{0.860891in}%
\pgfsys@useobject{currentmarker}{}%
\end{pgfscope}%
\begin{pgfscope}%
\pgfsys@transformshift{0.924275in}{0.862077in}%
\pgfsys@useobject{currentmarker}{}%
\end{pgfscope}%
\begin{pgfscope}%
\pgfsys@transformshift{0.924556in}{0.862759in}%
\pgfsys@useobject{currentmarker}{}%
\end{pgfscope}%
\begin{pgfscope}%
\pgfsys@transformshift{0.925928in}{0.865421in}%
\pgfsys@useobject{currentmarker}{}%
\end{pgfscope}%
\begin{pgfscope}%
\pgfsys@transformshift{0.928209in}{0.869966in}%
\pgfsys@useobject{currentmarker}{}%
\end{pgfscope}%
\begin{pgfscope}%
\pgfsys@transformshift{0.931688in}{0.876564in}%
\pgfsys@useobject{currentmarker}{}%
\end{pgfscope}%
\begin{pgfscope}%
\pgfsys@transformshift{0.935767in}{0.885155in}%
\pgfsys@useobject{currentmarker}{}%
\end{pgfscope}%
\begin{pgfscope}%
\pgfsys@transformshift{0.940638in}{0.896134in}%
\pgfsys@useobject{currentmarker}{}%
\end{pgfscope}%
\begin{pgfscope}%
\pgfsys@transformshift{0.946645in}{0.908995in}%
\pgfsys@useobject{currentmarker}{}%
\end{pgfscope}%
\begin{pgfscope}%
\pgfsys@transformshift{0.952705in}{0.923596in}%
\pgfsys@useobject{currentmarker}{}%
\end{pgfscope}%
\begin{pgfscope}%
\pgfsys@transformshift{0.958639in}{0.939398in}%
\pgfsys@useobject{currentmarker}{}%
\end{pgfscope}%
\begin{pgfscope}%
\pgfsys@transformshift{0.966546in}{0.954943in}%
\pgfsys@useobject{currentmarker}{}%
\end{pgfscope}%
\begin{pgfscope}%
\pgfsys@transformshift{0.970416in}{0.963720in}%
\pgfsys@useobject{currentmarker}{}%
\end{pgfscope}%
\begin{pgfscope}%
\pgfsys@transformshift{0.972542in}{0.968548in}%
\pgfsys@useobject{currentmarker}{}%
\end{pgfscope}%
\begin{pgfscope}%
\pgfsys@transformshift{0.973624in}{0.971240in}%
\pgfsys@useobject{currentmarker}{}%
\end{pgfscope}%
\begin{pgfscope}%
\pgfsys@transformshift{0.974234in}{0.972715in}%
\pgfsys@useobject{currentmarker}{}%
\end{pgfscope}%
\begin{pgfscope}%
\pgfsys@transformshift{0.975025in}{0.974753in}%
\pgfsys@useobject{currentmarker}{}%
\end{pgfscope}%
\begin{pgfscope}%
\pgfsys@transformshift{0.975556in}{0.975832in}%
\pgfsys@useobject{currentmarker}{}%
\end{pgfscope}%
\begin{pgfscope}%
\pgfsys@transformshift{0.976104in}{0.977751in}%
\pgfsys@useobject{currentmarker}{}%
\end{pgfscope}%
\begin{pgfscope}%
\pgfsys@transformshift{0.976497in}{0.978776in}%
\pgfsys@useobject{currentmarker}{}%
\end{pgfscope}%
\begin{pgfscope}%
\pgfsys@transformshift{0.976664in}{0.979356in}%
\pgfsys@useobject{currentmarker}{}%
\end{pgfscope}%
\begin{pgfscope}%
\pgfsys@transformshift{0.976786in}{0.979665in}%
\pgfsys@useobject{currentmarker}{}%
\end{pgfscope}%
\begin{pgfscope}%
\pgfsys@transformshift{0.976834in}{0.979841in}%
\pgfsys@useobject{currentmarker}{}%
\end{pgfscope}%
\begin{pgfscope}%
\pgfsys@transformshift{0.977113in}{0.980515in}%
\pgfsys@useobject{currentmarker}{}%
\end{pgfscope}%
\begin{pgfscope}%
\pgfsys@transformshift{0.977209in}{0.980905in}%
\pgfsys@useobject{currentmarker}{}%
\end{pgfscope}%
\begin{pgfscope}%
\pgfsys@transformshift{0.977274in}{0.981115in}%
\pgfsys@useobject{currentmarker}{}%
\end{pgfscope}%
\begin{pgfscope}%
\pgfsys@transformshift{0.977286in}{0.981236in}%
\pgfsys@useobject{currentmarker}{}%
\end{pgfscope}%
\begin{pgfscope}%
\pgfsys@transformshift{0.977457in}{0.981901in}%
\pgfsys@useobject{currentmarker}{}%
\end{pgfscope}%
\begin{pgfscope}%
\pgfsys@transformshift{0.977492in}{0.982277in}%
\pgfsys@useobject{currentmarker}{}%
\end{pgfscope}%
\begin{pgfscope}%
\pgfsys@transformshift{0.977525in}{0.982483in}%
\pgfsys@useobject{currentmarker}{}%
\end{pgfscope}%
\begin{pgfscope}%
\pgfsys@transformshift{0.977540in}{0.982596in}%
\pgfsys@useobject{currentmarker}{}%
\end{pgfscope}%
\begin{pgfscope}%
\pgfsys@transformshift{0.977542in}{0.982659in}%
\pgfsys@useobject{currentmarker}{}%
\end{pgfscope}%
\begin{pgfscope}%
\pgfsys@transformshift{0.977544in}{0.982693in}%
\pgfsys@useobject{currentmarker}{}%
\end{pgfscope}%
\begin{pgfscope}%
\pgfsys@transformshift{0.977544in}{0.982712in}%
\pgfsys@useobject{currentmarker}{}%
\end{pgfscope}%
\begin{pgfscope}%
\pgfsys@transformshift{0.977545in}{0.982723in}%
\pgfsys@useobject{currentmarker}{}%
\end{pgfscope}%
\begin{pgfscope}%
\pgfsys@transformshift{0.977546in}{0.982728in}%
\pgfsys@useobject{currentmarker}{}%
\end{pgfscope}%
\begin{pgfscope}%
\pgfsys@transformshift{0.977546in}{0.982731in}%
\pgfsys@useobject{currentmarker}{}%
\end{pgfscope}%
\begin{pgfscope}%
\pgfsys@transformshift{0.977546in}{0.982733in}%
\pgfsys@useobject{currentmarker}{}%
\end{pgfscope}%
\begin{pgfscope}%
\pgfsys@transformshift{0.977546in}{0.982734in}%
\pgfsys@useobject{currentmarker}{}%
\end{pgfscope}%
\begin{pgfscope}%
\pgfsys@transformshift{0.977546in}{0.982735in}%
\pgfsys@useobject{currentmarker}{}%
\end{pgfscope}%
\begin{pgfscope}%
\pgfsys@transformshift{0.977546in}{0.982735in}%
\pgfsys@useobject{currentmarker}{}%
\end{pgfscope}%
\begin{pgfscope}%
\pgfsys@transformshift{0.977546in}{0.982735in}%
\pgfsys@useobject{currentmarker}{}%
\end{pgfscope}%
\begin{pgfscope}%
\pgfsys@transformshift{0.977546in}{0.982735in}%
\pgfsys@useobject{currentmarker}{}%
\end{pgfscope}%
\begin{pgfscope}%
\pgfsys@transformshift{0.977546in}{0.982735in}%
\pgfsys@useobject{currentmarker}{}%
\end{pgfscope}%
\begin{pgfscope}%
\pgfsys@transformshift{0.977546in}{0.982735in}%
\pgfsys@useobject{currentmarker}{}%
\end{pgfscope}%
\begin{pgfscope}%
\pgfsys@transformshift{0.977546in}{0.982735in}%
\pgfsys@useobject{currentmarker}{}%
\end{pgfscope}%
\begin{pgfscope}%
\pgfsys@transformshift{0.977546in}{0.982735in}%
\pgfsys@useobject{currentmarker}{}%
\end{pgfscope}%
\begin{pgfscope}%
\pgfsys@transformshift{0.977546in}{0.982735in}%
\pgfsys@useobject{currentmarker}{}%
\end{pgfscope}%
\begin{pgfscope}%
\pgfsys@transformshift{0.977546in}{0.982735in}%
\pgfsys@useobject{currentmarker}{}%
\end{pgfscope}%
\begin{pgfscope}%
\pgfsys@transformshift{0.977546in}{0.982735in}%
\pgfsys@useobject{currentmarker}{}%
\end{pgfscope}%
\begin{pgfscope}%
\pgfsys@transformshift{0.977546in}{0.982735in}%
\pgfsys@useobject{currentmarker}{}%
\end{pgfscope}%
\begin{pgfscope}%
\pgfsys@transformshift{0.977546in}{0.982735in}%
\pgfsys@useobject{currentmarker}{}%
\end{pgfscope}%
\begin{pgfscope}%
\pgfsys@transformshift{0.977546in}{0.982735in}%
\pgfsys@useobject{currentmarker}{}%
\end{pgfscope}%
\begin{pgfscope}%
\pgfsys@transformshift{0.977546in}{0.982735in}%
\pgfsys@useobject{currentmarker}{}%
\end{pgfscope}%
\begin{pgfscope}%
\pgfsys@transformshift{0.977546in}{0.982735in}%
\pgfsys@useobject{currentmarker}{}%
\end{pgfscope}%
\begin{pgfscope}%
\pgfsys@transformshift{0.977546in}{0.982735in}%
\pgfsys@useobject{currentmarker}{}%
\end{pgfscope}%
\begin{pgfscope}%
\pgfsys@transformshift{0.977546in}{0.982735in}%
\pgfsys@useobject{currentmarker}{}%
\end{pgfscope}%
\begin{pgfscope}%
\pgfsys@transformshift{0.977546in}{0.982735in}%
\pgfsys@useobject{currentmarker}{}%
\end{pgfscope}%
\begin{pgfscope}%
\pgfsys@transformshift{0.977546in}{0.982735in}%
\pgfsys@useobject{currentmarker}{}%
\end{pgfscope}%
\begin{pgfscope}%
\pgfsys@transformshift{0.977546in}{0.982735in}%
\pgfsys@useobject{currentmarker}{}%
\end{pgfscope}%
\begin{pgfscope}%
\pgfsys@transformshift{0.977546in}{0.982735in}%
\pgfsys@useobject{currentmarker}{}%
\end{pgfscope}%
\begin{pgfscope}%
\pgfsys@transformshift{0.977546in}{0.982735in}%
\pgfsys@useobject{currentmarker}{}%
\end{pgfscope}%
\begin{pgfscope}%
\pgfsys@transformshift{0.977546in}{0.982735in}%
\pgfsys@useobject{currentmarker}{}%
\end{pgfscope}%
\begin{pgfscope}%
\pgfsys@transformshift{0.977546in}{0.982735in}%
\pgfsys@useobject{currentmarker}{}%
\end{pgfscope}%
\begin{pgfscope}%
\pgfsys@transformshift{0.977546in}{0.982735in}%
\pgfsys@useobject{currentmarker}{}%
\end{pgfscope}%
\begin{pgfscope}%
\pgfsys@transformshift{0.977546in}{0.982735in}%
\pgfsys@useobject{currentmarker}{}%
\end{pgfscope}%
\begin{pgfscope}%
\pgfsys@transformshift{0.977546in}{0.982735in}%
\pgfsys@useobject{currentmarker}{}%
\end{pgfscope}%
\begin{pgfscope}%
\pgfsys@transformshift{0.977546in}{0.982735in}%
\pgfsys@useobject{currentmarker}{}%
\end{pgfscope}%
\begin{pgfscope}%
\pgfsys@transformshift{0.977546in}{0.982735in}%
\pgfsys@useobject{currentmarker}{}%
\end{pgfscope}%
\begin{pgfscope}%
\pgfsys@transformshift{0.977546in}{0.982735in}%
\pgfsys@useobject{currentmarker}{}%
\end{pgfscope}%
\begin{pgfscope}%
\pgfsys@transformshift{0.977546in}{0.982735in}%
\pgfsys@useobject{currentmarker}{}%
\end{pgfscope}%
\begin{pgfscope}%
\pgfsys@transformshift{0.977546in}{0.982735in}%
\pgfsys@useobject{currentmarker}{}%
\end{pgfscope}%
\begin{pgfscope}%
\pgfsys@transformshift{0.977546in}{0.982735in}%
\pgfsys@useobject{currentmarker}{}%
\end{pgfscope}%
\begin{pgfscope}%
\pgfsys@transformshift{0.977546in}{0.982735in}%
\pgfsys@useobject{currentmarker}{}%
\end{pgfscope}%
\begin{pgfscope}%
\pgfsys@transformshift{0.977546in}{0.982735in}%
\pgfsys@useobject{currentmarker}{}%
\end{pgfscope}%
\begin{pgfscope}%
\pgfsys@transformshift{0.977546in}{0.982735in}%
\pgfsys@useobject{currentmarker}{}%
\end{pgfscope}%
\begin{pgfscope}%
\pgfsys@transformshift{0.977546in}{0.982735in}%
\pgfsys@useobject{currentmarker}{}%
\end{pgfscope}%
\begin{pgfscope}%
\pgfsys@transformshift{0.977546in}{0.982735in}%
\pgfsys@useobject{currentmarker}{}%
\end{pgfscope}%
\begin{pgfscope}%
\pgfsys@transformshift{0.977546in}{0.982735in}%
\pgfsys@useobject{currentmarker}{}%
\end{pgfscope}%
\begin{pgfscope}%
\pgfsys@transformshift{0.977546in}{0.982735in}%
\pgfsys@useobject{currentmarker}{}%
\end{pgfscope}%
\begin{pgfscope}%
\pgfsys@transformshift{0.977546in}{0.982735in}%
\pgfsys@useobject{currentmarker}{}%
\end{pgfscope}%
\begin{pgfscope}%
\pgfsys@transformshift{0.977546in}{0.982735in}%
\pgfsys@useobject{currentmarker}{}%
\end{pgfscope}%
\begin{pgfscope}%
\pgfsys@transformshift{0.977546in}{0.982735in}%
\pgfsys@useobject{currentmarker}{}%
\end{pgfscope}%
\begin{pgfscope}%
\pgfsys@transformshift{0.977546in}{0.982735in}%
\pgfsys@useobject{currentmarker}{}%
\end{pgfscope}%
\begin{pgfscope}%
\pgfsys@transformshift{0.977546in}{0.982735in}%
\pgfsys@useobject{currentmarker}{}%
\end{pgfscope}%
\begin{pgfscope}%
\pgfsys@transformshift{0.977546in}{0.982735in}%
\pgfsys@useobject{currentmarker}{}%
\end{pgfscope}%
\begin{pgfscope}%
\pgfsys@transformshift{0.977546in}{0.982735in}%
\pgfsys@useobject{currentmarker}{}%
\end{pgfscope}%
\begin{pgfscope}%
\pgfsys@transformshift{0.977546in}{0.982735in}%
\pgfsys@useobject{currentmarker}{}%
\end{pgfscope}%
\begin{pgfscope}%
\pgfsys@transformshift{0.977546in}{0.982735in}%
\pgfsys@useobject{currentmarker}{}%
\end{pgfscope}%
\begin{pgfscope}%
\pgfsys@transformshift{0.977546in}{0.982735in}%
\pgfsys@useobject{currentmarker}{}%
\end{pgfscope}%
\begin{pgfscope}%
\pgfsys@transformshift{0.977546in}{0.982735in}%
\pgfsys@useobject{currentmarker}{}%
\end{pgfscope}%
\begin{pgfscope}%
\pgfsys@transformshift{0.977546in}{0.982735in}%
\pgfsys@useobject{currentmarker}{}%
\end{pgfscope}%
\begin{pgfscope}%
\pgfsys@transformshift{0.977546in}{0.982735in}%
\pgfsys@useobject{currentmarker}{}%
\end{pgfscope}%
\begin{pgfscope}%
\pgfsys@transformshift{0.977546in}{0.982735in}%
\pgfsys@useobject{currentmarker}{}%
\end{pgfscope}%
\begin{pgfscope}%
\pgfsys@transformshift{0.977546in}{0.982735in}%
\pgfsys@useobject{currentmarker}{}%
\end{pgfscope}%
\begin{pgfscope}%
\pgfsys@transformshift{0.977546in}{0.982735in}%
\pgfsys@useobject{currentmarker}{}%
\end{pgfscope}%
\begin{pgfscope}%
\pgfsys@transformshift{0.977546in}{0.982735in}%
\pgfsys@useobject{currentmarker}{}%
\end{pgfscope}%
\begin{pgfscope}%
\pgfsys@transformshift{0.977546in}{0.982735in}%
\pgfsys@useobject{currentmarker}{}%
\end{pgfscope}%
\begin{pgfscope}%
\pgfsys@transformshift{0.977546in}{0.982735in}%
\pgfsys@useobject{currentmarker}{}%
\end{pgfscope}%
\begin{pgfscope}%
\pgfsys@transformshift{0.977546in}{0.982735in}%
\pgfsys@useobject{currentmarker}{}%
\end{pgfscope}%
\begin{pgfscope}%
\pgfsys@transformshift{0.977546in}{0.982735in}%
\pgfsys@useobject{currentmarker}{}%
\end{pgfscope}%
\begin{pgfscope}%
\pgfsys@transformshift{0.977546in}{0.982735in}%
\pgfsys@useobject{currentmarker}{}%
\end{pgfscope}%
\begin{pgfscope}%
\pgfsys@transformshift{0.977546in}{0.982735in}%
\pgfsys@useobject{currentmarker}{}%
\end{pgfscope}%
\begin{pgfscope}%
\pgfsys@transformshift{0.977546in}{0.982735in}%
\pgfsys@useobject{currentmarker}{}%
\end{pgfscope}%
\begin{pgfscope}%
\pgfsys@transformshift{0.977546in}{0.982735in}%
\pgfsys@useobject{currentmarker}{}%
\end{pgfscope}%
\begin{pgfscope}%
\pgfsys@transformshift{0.977546in}{0.982735in}%
\pgfsys@useobject{currentmarker}{}%
\end{pgfscope}%
\begin{pgfscope}%
\pgfsys@transformshift{0.977546in}{0.982735in}%
\pgfsys@useobject{currentmarker}{}%
\end{pgfscope}%
\begin{pgfscope}%
\pgfsys@transformshift{0.977546in}{0.982735in}%
\pgfsys@useobject{currentmarker}{}%
\end{pgfscope}%
\begin{pgfscope}%
\pgfsys@transformshift{0.977546in}{0.982735in}%
\pgfsys@useobject{currentmarker}{}%
\end{pgfscope}%
\begin{pgfscope}%
\pgfsys@transformshift{0.977546in}{0.982735in}%
\pgfsys@useobject{currentmarker}{}%
\end{pgfscope}%
\begin{pgfscope}%
\pgfsys@transformshift{0.977546in}{0.982735in}%
\pgfsys@useobject{currentmarker}{}%
\end{pgfscope}%
\begin{pgfscope}%
\pgfsys@transformshift{0.977546in}{0.982735in}%
\pgfsys@useobject{currentmarker}{}%
\end{pgfscope}%
\begin{pgfscope}%
\pgfsys@transformshift{0.977546in}{0.982735in}%
\pgfsys@useobject{currentmarker}{}%
\end{pgfscope}%
\begin{pgfscope}%
\pgfsys@transformshift{0.977546in}{0.982735in}%
\pgfsys@useobject{currentmarker}{}%
\end{pgfscope}%
\begin{pgfscope}%
\pgfsys@transformshift{0.977546in}{0.982735in}%
\pgfsys@useobject{currentmarker}{}%
\end{pgfscope}%
\begin{pgfscope}%
\pgfsys@transformshift{0.977546in}{0.982735in}%
\pgfsys@useobject{currentmarker}{}%
\end{pgfscope}%
\begin{pgfscope}%
\pgfsys@transformshift{0.977546in}{0.982735in}%
\pgfsys@useobject{currentmarker}{}%
\end{pgfscope}%
\begin{pgfscope}%
\pgfsys@transformshift{0.977546in}{0.982735in}%
\pgfsys@useobject{currentmarker}{}%
\end{pgfscope}%
\begin{pgfscope}%
\pgfsys@transformshift{0.977546in}{0.982735in}%
\pgfsys@useobject{currentmarker}{}%
\end{pgfscope}%
\begin{pgfscope}%
\pgfsys@transformshift{0.977546in}{0.982735in}%
\pgfsys@useobject{currentmarker}{}%
\end{pgfscope}%
\begin{pgfscope}%
\pgfsys@transformshift{0.977546in}{0.982735in}%
\pgfsys@useobject{currentmarker}{}%
\end{pgfscope}%
\begin{pgfscope}%
\pgfsys@transformshift{0.977546in}{0.982735in}%
\pgfsys@useobject{currentmarker}{}%
\end{pgfscope}%
\begin{pgfscope}%
\pgfsys@transformshift{0.977546in}{0.982735in}%
\pgfsys@useobject{currentmarker}{}%
\end{pgfscope}%
\begin{pgfscope}%
\pgfsys@transformshift{0.977546in}{0.982735in}%
\pgfsys@useobject{currentmarker}{}%
\end{pgfscope}%
\begin{pgfscope}%
\pgfsys@transformshift{0.977546in}{0.982735in}%
\pgfsys@useobject{currentmarker}{}%
\end{pgfscope}%
\begin{pgfscope}%
\pgfsys@transformshift{0.977546in}{0.982735in}%
\pgfsys@useobject{currentmarker}{}%
\end{pgfscope}%
\begin{pgfscope}%
\pgfsys@transformshift{0.977546in}{0.982735in}%
\pgfsys@useobject{currentmarker}{}%
\end{pgfscope}%
\begin{pgfscope}%
\pgfsys@transformshift{0.977546in}{0.982735in}%
\pgfsys@useobject{currentmarker}{}%
\end{pgfscope}%
\begin{pgfscope}%
\pgfsys@transformshift{0.977546in}{0.982735in}%
\pgfsys@useobject{currentmarker}{}%
\end{pgfscope}%
\begin{pgfscope}%
\pgfsys@transformshift{0.977546in}{0.982735in}%
\pgfsys@useobject{currentmarker}{}%
\end{pgfscope}%
\begin{pgfscope}%
\pgfsys@transformshift{0.977546in}{0.982735in}%
\pgfsys@useobject{currentmarker}{}%
\end{pgfscope}%
\begin{pgfscope}%
\pgfsys@transformshift{0.977546in}{0.982735in}%
\pgfsys@useobject{currentmarker}{}%
\end{pgfscope}%
\begin{pgfscope}%
\pgfsys@transformshift{0.977546in}{0.982735in}%
\pgfsys@useobject{currentmarker}{}%
\end{pgfscope}%
\begin{pgfscope}%
\pgfsys@transformshift{0.977810in}{0.983395in}%
\pgfsys@useobject{currentmarker}{}%
\end{pgfscope}%
\begin{pgfscope}%
\pgfsys@transformshift{0.977832in}{0.983785in}%
\pgfsys@useobject{currentmarker}{}%
\end{pgfscope}%
\begin{pgfscope}%
\pgfsys@transformshift{0.977903in}{0.983988in}%
\pgfsys@useobject{currentmarker}{}%
\end{pgfscope}%
\begin{pgfscope}%
\pgfsys@transformshift{0.977688in}{0.987238in}%
\pgfsys@useobject{currentmarker}{}%
\end{pgfscope}%
\begin{pgfscope}%
\pgfsys@transformshift{0.979575in}{0.995639in}%
\pgfsys@useobject{currentmarker}{}%
\end{pgfscope}%
\begin{pgfscope}%
\pgfsys@transformshift{0.980136in}{1.008683in}%
\pgfsys@useobject{currentmarker}{}%
\end{pgfscope}%
\begin{pgfscope}%
\pgfsys@transformshift{0.988267in}{1.022261in}%
\pgfsys@useobject{currentmarker}{}%
\end{pgfscope}%
\begin{pgfscope}%
\pgfsys@transformshift{0.989420in}{1.040526in}%
\pgfsys@useobject{currentmarker}{}%
\end{pgfscope}%
\begin{pgfscope}%
\pgfsys@transformshift{0.993756in}{1.049611in}%
\pgfsys@useobject{currentmarker}{}%
\end{pgfscope}%
\begin{pgfscope}%
\pgfsys@transformshift{0.995283in}{1.061161in}%
\pgfsys@useobject{currentmarker}{}%
\end{pgfscope}%
\begin{pgfscope}%
\pgfsys@transformshift{0.995744in}{1.067553in}%
\pgfsys@useobject{currentmarker}{}%
\end{pgfscope}%
\begin{pgfscope}%
\pgfsys@transformshift{0.997184in}{1.070770in}%
\pgfsys@useobject{currentmarker}{}%
\end{pgfscope}%
\begin{pgfscope}%
\pgfsys@transformshift{0.997238in}{1.077236in}%
\pgfsys@useobject{currentmarker}{}%
\end{pgfscope}%
\begin{pgfscope}%
\pgfsys@transformshift{0.998753in}{1.080454in}%
\pgfsys@useobject{currentmarker}{}%
\end{pgfscope}%
\begin{pgfscope}%
\pgfsys@transformshift{0.999043in}{1.085721in}%
\pgfsys@useobject{currentmarker}{}%
\end{pgfscope}%
\begin{pgfscope}%
\pgfsys@transformshift{0.999942in}{1.091546in}%
\pgfsys@useobject{currentmarker}{}%
\end{pgfscope}%
\begin{pgfscope}%
\pgfsys@transformshift{0.999777in}{1.094783in}%
\pgfsys@useobject{currentmarker}{}%
\end{pgfscope}%
\begin{pgfscope}%
\pgfsys@transformshift{0.999200in}{1.098702in}%
\pgfsys@useobject{currentmarker}{}%
\end{pgfscope}%
\begin{pgfscope}%
\pgfsys@transformshift{0.999812in}{1.100793in}%
\pgfsys@useobject{currentmarker}{}%
\end{pgfscope}%
\begin{pgfscope}%
\pgfsys@transformshift{0.999576in}{1.101968in}%
\pgfsys@useobject{currentmarker}{}%
\end{pgfscope}%
\begin{pgfscope}%
\pgfsys@transformshift{1.000414in}{1.104728in}%
\pgfsys@useobject{currentmarker}{}%
\end{pgfscope}%
\begin{pgfscope}%
\pgfsys@transformshift{0.999568in}{1.108573in}%
\pgfsys@useobject{currentmarker}{}%
\end{pgfscope}%
\begin{pgfscope}%
\pgfsys@transformshift{1.000382in}{1.113815in}%
\pgfsys@useobject{currentmarker}{}%
\end{pgfscope}%
\begin{pgfscope}%
\pgfsys@transformshift{0.999814in}{1.116676in}%
\pgfsys@useobject{currentmarker}{}%
\end{pgfscope}%
\begin{pgfscope}%
\pgfsys@transformshift{0.999612in}{1.118268in}%
\pgfsys@useobject{currentmarker}{}%
\end{pgfscope}%
\begin{pgfscope}%
\pgfsys@transformshift{0.999826in}{1.119124in}%
\pgfsys@useobject{currentmarker}{}%
\end{pgfscope}%
\begin{pgfscope}%
\pgfsys@transformshift{0.999148in}{1.121786in}%
\pgfsys@useobject{currentmarker}{}%
\end{pgfscope}%
\begin{pgfscope}%
\pgfsys@transformshift{1.000188in}{1.126649in}%
\pgfsys@useobject{currentmarker}{}%
\end{pgfscope}%
\begin{pgfscope}%
\pgfsys@transformshift{0.998601in}{1.132998in}%
\pgfsys@useobject{currentmarker}{}%
\end{pgfscope}%
\begin{pgfscope}%
\pgfsys@transformshift{1.000887in}{1.140165in}%
\pgfsys@useobject{currentmarker}{}%
\end{pgfscope}%
\begin{pgfscope}%
\pgfsys@transformshift{1.006048in}{1.147623in}%
\pgfsys@useobject{currentmarker}{}%
\end{pgfscope}%
\begin{pgfscope}%
\pgfsys@transformshift{1.006595in}{1.152581in}%
\pgfsys@useobject{currentmarker}{}%
\end{pgfscope}%
\begin{pgfscope}%
\pgfsys@transformshift{1.008689in}{1.159039in}%
\pgfsys@useobject{currentmarker}{}%
\end{pgfscope}%
\begin{pgfscope}%
\pgfsys@transformshift{1.009069in}{1.166733in}%
\pgfsys@useobject{currentmarker}{}%
\end{pgfscope}%
\begin{pgfscope}%
\pgfsys@transformshift{1.011683in}{1.176080in}%
\pgfsys@useobject{currentmarker}{}%
\end{pgfscope}%
\begin{pgfscope}%
\pgfsys@transformshift{1.011627in}{1.186941in}%
\pgfsys@useobject{currentmarker}{}%
\end{pgfscope}%
\begin{pgfscope}%
\pgfsys@transformshift{1.011946in}{1.192906in}%
\pgfsys@useobject{currentmarker}{}%
\end{pgfscope}%
\begin{pgfscope}%
\pgfsys@transformshift{1.013612in}{1.195738in}%
\pgfsys@useobject{currentmarker}{}%
\end{pgfscope}%
\begin{pgfscope}%
\pgfsys@transformshift{1.013260in}{1.199644in}%
\pgfsys@useobject{currentmarker}{}%
\end{pgfscope}%
\begin{pgfscope}%
\pgfsys@transformshift{1.014440in}{1.204219in}%
\pgfsys@useobject{currentmarker}{}%
\end{pgfscope}%
\begin{pgfscope}%
\pgfsys@transformshift{1.013937in}{1.206768in}%
\pgfsys@useobject{currentmarker}{}%
\end{pgfscope}%
\begin{pgfscope}%
\pgfsys@transformshift{1.015006in}{1.211261in}%
\pgfsys@useobject{currentmarker}{}%
\end{pgfscope}%
\begin{pgfscope}%
\pgfsys@transformshift{1.014315in}{1.216556in}%
\pgfsys@useobject{currentmarker}{}%
\end{pgfscope}%
\begin{pgfscope}%
\pgfsys@transformshift{1.015884in}{1.222323in}%
\pgfsys@useobject{currentmarker}{}%
\end{pgfscope}%
\begin{pgfscope}%
\pgfsys@transformshift{1.014602in}{1.229931in}%
\pgfsys@useobject{currentmarker}{}%
\end{pgfscope}%
\begin{pgfscope}%
\pgfsys@transformshift{1.015527in}{1.234072in}%
\pgfsys@useobject{currentmarker}{}%
\end{pgfscope}%
\begin{pgfscope}%
\pgfsys@transformshift{1.013677in}{1.240260in}%
\pgfsys@useobject{currentmarker}{}%
\end{pgfscope}%
\begin{pgfscope}%
\pgfsys@transformshift{1.014008in}{1.243797in}%
\pgfsys@useobject{currentmarker}{}%
\end{pgfscope}%
\begin{pgfscope}%
\pgfsys@transformshift{1.013428in}{1.248820in}%
\pgfsys@useobject{currentmarker}{}%
\end{pgfscope}%
\begin{pgfscope}%
\pgfsys@transformshift{1.012767in}{1.251521in}%
\pgfsys@useobject{currentmarker}{}%
\end{pgfscope}%
\begin{pgfscope}%
\pgfsys@transformshift{1.013443in}{1.252894in}%
\pgfsys@useobject{currentmarker}{}%
\end{pgfscope}%
\begin{pgfscope}%
\pgfsys@transformshift{1.013366in}{1.255271in}%
\pgfsys@useobject{currentmarker}{}%
\end{pgfscope}%
\begin{pgfscope}%
\pgfsys@transformshift{1.013877in}{1.256474in}%
\pgfsys@useobject{currentmarker}{}%
\end{pgfscope}%
\begin{pgfscope}%
\pgfsys@transformshift{1.013218in}{1.259632in}%
\pgfsys@useobject{currentmarker}{}%
\end{pgfscope}%
\begin{pgfscope}%
\pgfsys@transformshift{1.015797in}{1.266252in}%
\pgfsys@useobject{currentmarker}{}%
\end{pgfscope}%
\begin{pgfscope}%
\pgfsys@transformshift{1.014778in}{1.275717in}%
\pgfsys@useobject{currentmarker}{}%
\end{pgfscope}%
\begin{pgfscope}%
\pgfsys@transformshift{1.020148in}{1.288021in}%
\pgfsys@useobject{currentmarker}{}%
\end{pgfscope}%
\begin{pgfscope}%
\pgfsys@transformshift{1.017522in}{1.303697in}%
\pgfsys@useobject{currentmarker}{}%
\end{pgfscope}%
\begin{pgfscope}%
\pgfsys@transformshift{1.024332in}{1.321617in}%
\pgfsys@useobject{currentmarker}{}%
\end{pgfscope}%
\begin{pgfscope}%
\pgfsys@transformshift{1.022019in}{1.342930in}%
\pgfsys@useobject{currentmarker}{}%
\end{pgfscope}%
\begin{pgfscope}%
\pgfsys@transformshift{1.030798in}{1.364603in}%
\pgfsys@useobject{currentmarker}{}%
\end{pgfscope}%
\begin{pgfscope}%
\pgfsys@transformshift{1.027290in}{1.389113in}%
\pgfsys@useobject{currentmarker}{}%
\end{pgfscope}%
\begin{pgfscope}%
\pgfsys@transformshift{1.034711in}{1.413386in}%
\pgfsys@useobject{currentmarker}{}%
\end{pgfscope}%
\begin{pgfscope}%
\pgfsys@transformshift{1.034477in}{1.427344in}%
\pgfsys@useobject{currentmarker}{}%
\end{pgfscope}%
\begin{pgfscope}%
\pgfsys@transformshift{1.035866in}{1.434895in}%
\pgfsys@useobject{currentmarker}{}%
\end{pgfscope}%
\begin{pgfscope}%
\pgfsys@transformshift{1.036733in}{1.443231in}%
\pgfsys@useobject{currentmarker}{}%
\end{pgfscope}%
\begin{pgfscope}%
\pgfsys@transformshift{1.036855in}{1.447839in}%
\pgfsys@useobject{currentmarker}{}%
\end{pgfscope}%
\begin{pgfscope}%
\pgfsys@transformshift{1.037633in}{1.450252in}%
\pgfsys@useobject{currentmarker}{}%
\end{pgfscope}%
\begin{pgfscope}%
\pgfsys@transformshift{1.037399in}{1.451626in}%
\pgfsys@useobject{currentmarker}{}%
\end{pgfscope}%
\begin{pgfscope}%
\pgfsys@transformshift{1.037718in}{1.452324in}%
\pgfsys@useobject{currentmarker}{}%
\end{pgfscope}%
\begin{pgfscope}%
\pgfsys@transformshift{1.037131in}{1.454071in}%
\pgfsys@useobject{currentmarker}{}%
\end{pgfscope}%
\begin{pgfscope}%
\pgfsys@transformshift{1.035173in}{1.463904in}%
\pgfsys@useobject{currentmarker}{}%
\end{pgfscope}%
\begin{pgfscope}%
\pgfsys@transformshift{1.034434in}{1.475162in}%
\pgfsys@useobject{currentmarker}{}%
\end{pgfscope}%
\begin{pgfscope}%
\pgfsys@transformshift{1.040047in}{1.486411in}%
\pgfsys@useobject{currentmarker}{}%
\end{pgfscope}%
\begin{pgfscope}%
\pgfsys@transformshift{1.037703in}{1.500016in}%
\pgfsys@useobject{currentmarker}{}%
\end{pgfscope}%
\begin{pgfscope}%
\pgfsys@transformshift{1.044657in}{1.515690in}%
\pgfsys@useobject{currentmarker}{}%
\end{pgfscope}%
\begin{pgfscope}%
\pgfsys@transformshift{1.043080in}{1.535431in}%
\pgfsys@useobject{currentmarker}{}%
\end{pgfscope}%
\begin{pgfscope}%
\pgfsys@transformshift{1.046588in}{1.545743in}%
\pgfsys@useobject{currentmarker}{}%
\end{pgfscope}%
\begin{pgfscope}%
\pgfsys@transformshift{1.046386in}{1.551730in}%
\pgfsys@useobject{currentmarker}{}%
\end{pgfscope}%
\begin{pgfscope}%
\pgfsys@transformshift{1.047246in}{1.554911in}%
\pgfsys@useobject{currentmarker}{}%
\end{pgfscope}%
\begin{pgfscope}%
\pgfsys@transformshift{1.047294in}{1.556723in}%
\pgfsys@useobject{currentmarker}{}%
\end{pgfscope}%
\begin{pgfscope}%
\pgfsys@transformshift{1.047432in}{1.557710in}%
\pgfsys@useobject{currentmarker}{}%
\end{pgfscope}%
\begin{pgfscope}%
\pgfsys@transformshift{1.047324in}{1.559333in}%
\pgfsys@useobject{currentmarker}{}%
\end{pgfscope}%
\begin{pgfscope}%
\pgfsys@transformshift{1.046812in}{1.561836in}%
\pgfsys@useobject{currentmarker}{}%
\end{pgfscope}%
\begin{pgfscope}%
\pgfsys@transformshift{1.043515in}{1.571441in}%
\pgfsys@useobject{currentmarker}{}%
\end{pgfscope}%
\begin{pgfscope}%
\pgfsys@transformshift{1.044946in}{1.582296in}%
\pgfsys@useobject{currentmarker}{}%
\end{pgfscope}%
\begin{pgfscope}%
\pgfsys@transformshift{1.048037in}{1.593367in}%
\pgfsys@useobject{currentmarker}{}%
\end{pgfscope}%
\begin{pgfscope}%
\pgfsys@transformshift{1.047093in}{1.599619in}%
\pgfsys@useobject{currentmarker}{}%
\end{pgfscope}%
\begin{pgfscope}%
\pgfsys@transformshift{1.050669in}{1.609074in}%
\pgfsys@useobject{currentmarker}{}%
\end{pgfscope}%
\begin{pgfscope}%
\pgfsys@transformshift{1.050373in}{1.614626in}%
\pgfsys@useobject{currentmarker}{}%
\end{pgfscope}%
\begin{pgfscope}%
\pgfsys@transformshift{1.054047in}{1.624590in}%
\pgfsys@useobject{currentmarker}{}%
\end{pgfscope}%
\begin{pgfscope}%
\pgfsys@transformshift{1.052368in}{1.636540in}%
\pgfsys@useobject{currentmarker}{}%
\end{pgfscope}%
\begin{pgfscope}%
\pgfsys@transformshift{1.056117in}{1.649922in}%
\pgfsys@useobject{currentmarker}{}%
\end{pgfscope}%
\begin{pgfscope}%
\pgfsys@transformshift{1.053290in}{1.664605in}%
\pgfsys@useobject{currentmarker}{}%
\end{pgfscope}%
\begin{pgfscope}%
\pgfsys@transformshift{1.054763in}{1.672697in}%
\pgfsys@useobject{currentmarker}{}%
\end{pgfscope}%
\begin{pgfscope}%
\pgfsys@transformshift{1.054697in}{1.681869in}%
\pgfsys@useobject{currentmarker}{}%
\end{pgfscope}%
\begin{pgfscope}%
\pgfsys@transformshift{1.055255in}{1.686883in}%
\pgfsys@useobject{currentmarker}{}%
\end{pgfscope}%
\begin{pgfscope}%
\pgfsys@transformshift{1.056052in}{1.692837in}%
\pgfsys@useobject{currentmarker}{}%
\end{pgfscope}%
\begin{pgfscope}%
\pgfsys@transformshift{1.054851in}{1.695915in}%
\pgfsys@useobject{currentmarker}{}%
\end{pgfscope}%
\begin{pgfscope}%
\pgfsys@transformshift{1.055503in}{1.697611in}%
\pgfsys@useobject{currentmarker}{}%
\end{pgfscope}%
\begin{pgfscope}%
\pgfsys@transformshift{1.055340in}{1.700062in}%
\pgfsys@useobject{currentmarker}{}%
\end{pgfscope}%
\begin{pgfscope}%
\pgfsys@transformshift{1.055744in}{1.701352in}%
\pgfsys@useobject{currentmarker}{}%
\end{pgfscope}%
\begin{pgfscope}%
\pgfsys@transformshift{1.055391in}{1.703911in}%
\pgfsys@useobject{currentmarker}{}%
\end{pgfscope}%
\begin{pgfscope}%
\pgfsys@transformshift{1.058651in}{1.711139in}%
\pgfsys@useobject{currentmarker}{}%
\end{pgfscope}%
\begin{pgfscope}%
\pgfsys@transformshift{1.057989in}{1.715450in}%
\pgfsys@useobject{currentmarker}{}%
\end{pgfscope}%
\begin{pgfscope}%
\pgfsys@transformshift{1.061142in}{1.723283in}%
\pgfsys@useobject{currentmarker}{}%
\end{pgfscope}%
\begin{pgfscope}%
\pgfsys@transformshift{1.060731in}{1.727909in}%
\pgfsys@useobject{currentmarker}{}%
\end{pgfscope}%
\begin{pgfscope}%
\pgfsys@transformshift{1.063940in}{1.735333in}%
\pgfsys@useobject{currentmarker}{}%
\end{pgfscope}%
\begin{pgfscope}%
\pgfsys@transformshift{1.062526in}{1.743965in}%
\pgfsys@useobject{currentmarker}{}%
\end{pgfscope}%
\begin{pgfscope}%
\pgfsys@transformshift{1.067365in}{1.755589in}%
\pgfsys@useobject{currentmarker}{}%
\end{pgfscope}%
\begin{pgfscope}%
\pgfsys@transformshift{1.066611in}{1.769362in}%
\pgfsys@useobject{currentmarker}{}%
\end{pgfscope}%
\begin{pgfscope}%
\pgfsys@transformshift{1.072600in}{1.787394in}%
\pgfsys@useobject{currentmarker}{}%
\end{pgfscope}%
\begin{pgfscope}%
\pgfsys@transformshift{1.069731in}{1.807058in}%
\pgfsys@useobject{currentmarker}{}%
\end{pgfscope}%
\begin{pgfscope}%
\pgfsys@transformshift{1.073299in}{1.817389in}%
\pgfsys@useobject{currentmarker}{}%
\end{pgfscope}%
\begin{pgfscope}%
\pgfsys@transformshift{1.072835in}{1.823382in}%
\pgfsys@useobject{currentmarker}{}%
\end{pgfscope}%
\begin{pgfscope}%
\pgfsys@transformshift{1.073709in}{1.826571in}%
\pgfsys@useobject{currentmarker}{}%
\end{pgfscope}%
\begin{pgfscope}%
\pgfsys@transformshift{1.073394in}{1.830975in}%
\pgfsys@useobject{currentmarker}{}%
\end{pgfscope}%
\begin{pgfscope}%
\pgfsys@transformshift{1.074000in}{1.833326in}%
\pgfsys@useobject{currentmarker}{}%
\end{pgfscope}%
\begin{pgfscope}%
\pgfsys@transformshift{1.073927in}{1.834660in}%
\pgfsys@useobject{currentmarker}{}%
\end{pgfscope}%
\begin{pgfscope}%
\pgfsys@transformshift{1.074186in}{1.835347in}%
\pgfsys@useobject{currentmarker}{}%
\end{pgfscope}%
\begin{pgfscope}%
\pgfsys@transformshift{1.074193in}{1.835751in}%
\pgfsys@useobject{currentmarker}{}%
\end{pgfscope}%
\begin{pgfscope}%
\pgfsys@transformshift{1.074236in}{1.835969in}%
\pgfsys@useobject{currentmarker}{}%
\end{pgfscope}%
\begin{pgfscope}%
\pgfsys@transformshift{1.074237in}{1.836091in}%
\pgfsys@useobject{currentmarker}{}%
\end{pgfscope}%
\begin{pgfscope}%
\pgfsys@transformshift{1.074236in}{1.836158in}%
\pgfsys@useobject{currentmarker}{}%
\end{pgfscope}%
\begin{pgfscope}%
\pgfsys@transformshift{1.074495in}{1.837363in}%
\pgfsys@useobject{currentmarker}{}%
\end{pgfscope}%
\begin{pgfscope}%
\pgfsys@transformshift{1.074155in}{1.839727in}%
\pgfsys@useobject{currentmarker}{}%
\end{pgfscope}%
\begin{pgfscope}%
\pgfsys@transformshift{1.076012in}{1.844323in}%
\pgfsys@useobject{currentmarker}{}%
\end{pgfscope}%
\begin{pgfscope}%
\pgfsys@transformshift{1.075149in}{1.849772in}%
\pgfsys@useobject{currentmarker}{}%
\end{pgfscope}%
\begin{pgfscope}%
\pgfsys@transformshift{1.076672in}{1.863583in}%
\pgfsys@useobject{currentmarker}{}%
\end{pgfscope}%
\begin{pgfscope}%
\pgfsys@transformshift{1.074460in}{1.878158in}%
\pgfsys@useobject{currentmarker}{}%
\end{pgfscope}%
\begin{pgfscope}%
\pgfsys@transformshift{1.077545in}{1.894271in}%
\pgfsys@useobject{currentmarker}{}%
\end{pgfscope}%
\begin{pgfscope}%
\pgfsys@transformshift{1.074281in}{1.912167in}%
\pgfsys@useobject{currentmarker}{}%
\end{pgfscope}%
\begin{pgfscope}%
\pgfsys@transformshift{1.082905in}{1.931128in}%
\pgfsys@useobject{currentmarker}{}%
\end{pgfscope}%
\begin{pgfscope}%
\pgfsys@transformshift{1.080057in}{1.953872in}%
\pgfsys@useobject{currentmarker}{}%
\end{pgfscope}%
\begin{pgfscope}%
\pgfsys@transformshift{1.085849in}{1.977835in}%
\pgfsys@useobject{currentmarker}{}%
\end{pgfscope}%
\begin{pgfscope}%
\pgfsys@transformshift{1.084060in}{1.991276in}%
\pgfsys@useobject{currentmarker}{}%
\end{pgfscope}%
\begin{pgfscope}%
\pgfsys@transformshift{1.085634in}{1.998566in}%
\pgfsys@useobject{currentmarker}{}%
\end{pgfscope}%
\begin{pgfscope}%
\pgfsys@transformshift{1.085234in}{2.006947in}%
\pgfsys@useobject{currentmarker}{}%
\end{pgfscope}%
\begin{pgfscope}%
\pgfsys@transformshift{1.086078in}{2.011484in}%
\pgfsys@useobject{currentmarker}{}%
\end{pgfscope}%
\begin{pgfscope}%
\pgfsys@transformshift{1.085711in}{2.016893in}%
\pgfsys@useobject{currentmarker}{}%
\end{pgfscope}%
\begin{pgfscope}%
\pgfsys@transformshift{1.086130in}{2.019846in}%
\pgfsys@useobject{currentmarker}{}%
\end{pgfscope}%
\begin{pgfscope}%
\pgfsys@transformshift{1.086658in}{2.023439in}%
\pgfsys@useobject{currentmarker}{}%
\end{pgfscope}%
\begin{pgfscope}%
\pgfsys@transformshift{1.086394in}{2.025419in}%
\pgfsys@useobject{currentmarker}{}%
\end{pgfscope}%
\begin{pgfscope}%
\pgfsys@transformshift{1.086841in}{2.026422in}%
\pgfsys@useobject{currentmarker}{}%
\end{pgfscope}%
\begin{pgfscope}%
\pgfsys@transformshift{1.086443in}{2.029198in}%
\pgfsys@useobject{currentmarker}{}%
\end{pgfscope}%
\begin{pgfscope}%
\pgfsys@transformshift{1.089327in}{2.037542in}%
\pgfsys@useobject{currentmarker}{}%
\end{pgfscope}%
\begin{pgfscope}%
\pgfsys@transformshift{1.087242in}{2.048624in}%
\pgfsys@useobject{currentmarker}{}%
\end{pgfscope}%
\begin{pgfscope}%
\pgfsys@transformshift{1.090896in}{2.060747in}%
\pgfsys@useobject{currentmarker}{}%
\end{pgfscope}%
\begin{pgfscope}%
\pgfsys@transformshift{1.089711in}{2.075001in}%
\pgfsys@useobject{currentmarker}{}%
\end{pgfscope}%
\begin{pgfscope}%
\pgfsys@transformshift{1.092052in}{2.082512in}%
\pgfsys@useobject{currentmarker}{}%
\end{pgfscope}%
\begin{pgfscope}%
\pgfsys@transformshift{1.090645in}{2.091003in}%
\pgfsys@useobject{currentmarker}{}%
\end{pgfscope}%
\begin{pgfscope}%
\pgfsys@transformshift{1.091866in}{2.095576in}%
\pgfsys@useobject{currentmarker}{}%
\end{pgfscope}%
\begin{pgfscope}%
\pgfsys@transformshift{1.091761in}{2.100940in}%
\pgfsys@useobject{currentmarker}{}%
\end{pgfscope}%
\begin{pgfscope}%
\pgfsys@transformshift{1.092062in}{2.103875in}%
\pgfsys@useobject{currentmarker}{}%
\end{pgfscope}%
\begin{pgfscope}%
\pgfsys@transformshift{1.091895in}{2.107493in}%
\pgfsys@useobject{currentmarker}{}%
\end{pgfscope}%
\begin{pgfscope}%
\pgfsys@transformshift{1.091232in}{2.111686in}%
\pgfsys@useobject{currentmarker}{}%
\end{pgfscope}%
\begin{pgfscope}%
\pgfsys@transformshift{1.091814in}{2.123766in}%
\pgfsys@useobject{currentmarker}{}%
\end{pgfscope}%
\begin{pgfscope}%
\pgfsys@transformshift{1.088914in}{2.136936in}%
\pgfsys@useobject{currentmarker}{}%
\end{pgfscope}%
\begin{pgfscope}%
\pgfsys@transformshift{1.091802in}{2.150940in}%
\pgfsys@useobject{currentmarker}{}%
\end{pgfscope}%
\begin{pgfscope}%
\pgfsys@transformshift{1.090532in}{2.166152in}%
\pgfsys@useobject{currentmarker}{}%
\end{pgfscope}%
\begin{pgfscope}%
\pgfsys@transformshift{1.094064in}{2.182298in}%
\pgfsys@useobject{currentmarker}{}%
\end{pgfscope}%
\begin{pgfscope}%
\pgfsys@transformshift{1.091967in}{2.199567in}%
\pgfsys@useobject{currentmarker}{}%
\end{pgfscope}%
\begin{pgfscope}%
\pgfsys@transformshift{1.093328in}{2.209037in}%
\pgfsys@useobject{currentmarker}{}%
\end{pgfscope}%
\begin{pgfscope}%
\pgfsys@transformshift{1.093176in}{2.214297in}%
\pgfsys@useobject{currentmarker}{}%
\end{pgfscope}%
\begin{pgfscope}%
\pgfsys@transformshift{1.093411in}{2.217181in}%
\pgfsys@useobject{currentmarker}{}%
\end{pgfscope}%
\begin{pgfscope}%
\pgfsys@transformshift{1.094421in}{2.220573in}%
\pgfsys@useobject{currentmarker}{}%
\end{pgfscope}%
\begin{pgfscope}%
\pgfsys@transformshift{1.094531in}{2.222516in}%
\pgfsys@useobject{currentmarker}{}%
\end{pgfscope}%
\begin{pgfscope}%
\pgfsys@transformshift{1.094553in}{2.223587in}%
\pgfsys@useobject{currentmarker}{}%
\end{pgfscope}%
\begin{pgfscope}%
\pgfsys@transformshift{1.094523in}{2.224175in}%
\pgfsys@useobject{currentmarker}{}%
\end{pgfscope}%
\begin{pgfscope}%
\pgfsys@transformshift{1.094752in}{2.232344in}%
\pgfsys@useobject{currentmarker}{}%
\end{pgfscope}%
\begin{pgfscope}%
\pgfsys@transformshift{1.093496in}{2.242251in}%
\pgfsys@useobject{currentmarker}{}%
\end{pgfscope}%
\begin{pgfscope}%
\pgfsys@transformshift{1.094457in}{2.247658in}%
\pgfsys@useobject{currentmarker}{}%
\end{pgfscope}%
\begin{pgfscope}%
\pgfsys@transformshift{1.094146in}{2.250663in}%
\pgfsys@useobject{currentmarker}{}%
\end{pgfscope}%
\begin{pgfscope}%
\pgfsys@transformshift{1.096451in}{2.256997in}%
\pgfsys@useobject{currentmarker}{}%
\end{pgfscope}%
\begin{pgfscope}%
\pgfsys@transformshift{1.095409in}{2.264483in}%
\pgfsys@useobject{currentmarker}{}%
\end{pgfscope}%
\begin{pgfscope}%
\pgfsys@transformshift{1.096452in}{2.268508in}%
\pgfsys@useobject{currentmarker}{}%
\end{pgfscope}%
\begin{pgfscope}%
\pgfsys@transformshift{1.096000in}{2.270749in}%
\pgfsys@useobject{currentmarker}{}%
\end{pgfscope}%
\begin{pgfscope}%
\pgfsys@transformshift{1.096925in}{2.273997in}%
\pgfsys@useobject{currentmarker}{}%
\end{pgfscope}%
\begin{pgfscope}%
\pgfsys@transformshift{1.097363in}{2.278451in}%
\pgfsys@useobject{currentmarker}{}%
\end{pgfscope}%
\begin{pgfscope}%
\pgfsys@transformshift{1.097103in}{2.280898in}%
\pgfsys@useobject{currentmarker}{}%
\end{pgfscope}%
\begin{pgfscope}%
\pgfsys@transformshift{1.097472in}{2.282201in}%
\pgfsys@useobject{currentmarker}{}%
\end{pgfscope}%
\begin{pgfscope}%
\pgfsys@transformshift{1.097394in}{2.282942in}%
\pgfsys@useobject{currentmarker}{}%
\end{pgfscope}%
\begin{pgfscope}%
\pgfsys@transformshift{1.097475in}{2.283343in}%
\pgfsys@useobject{currentmarker}{}%
\end{pgfscope}%
\begin{pgfscope}%
\pgfsys@transformshift{1.097447in}{2.283567in}%
\pgfsys@useobject{currentmarker}{}%
\end{pgfscope}%
\begin{pgfscope}%
\pgfsys@transformshift{1.098795in}{2.287247in}%
\pgfsys@useobject{currentmarker}{}%
\end{pgfscope}%
\begin{pgfscope}%
\pgfsys@transformshift{1.097951in}{2.293140in}%
\pgfsys@useobject{currentmarker}{}%
\end{pgfscope}%
\begin{pgfscope}%
\pgfsys@transformshift{1.100006in}{2.301033in}%
\pgfsys@useobject{currentmarker}{}%
\end{pgfscope}%
\begin{pgfscope}%
\pgfsys@transformshift{1.099754in}{2.310705in}%
\pgfsys@useobject{currentmarker}{}%
\end{pgfscope}%
\begin{pgfscope}%
\pgfsys@transformshift{1.101132in}{2.315845in}%
\pgfsys@useobject{currentmarker}{}%
\end{pgfscope}%
\begin{pgfscope}%
\pgfsys@transformshift{1.101005in}{2.318769in}%
\pgfsys@useobject{currentmarker}{}%
\end{pgfscope}%
\begin{pgfscope}%
\pgfsys@transformshift{1.100654in}{2.320340in}%
\pgfsys@useobject{currentmarker}{}%
\end{pgfscope}%
\begin{pgfscope}%
\pgfsys@transformshift{1.101264in}{2.322455in}%
\pgfsys@useobject{currentmarker}{}%
\end{pgfscope}%
\begin{pgfscope}%
\pgfsys@transformshift{1.101169in}{2.326485in}%
\pgfsys@useobject{currentmarker}{}%
\end{pgfscope}%
\begin{pgfscope}%
\pgfsys@transformshift{1.102859in}{2.331535in}%
\pgfsys@useobject{currentmarker}{}%
\end{pgfscope}%
\begin{pgfscope}%
\pgfsys@transformshift{1.101676in}{2.337356in}%
\pgfsys@useobject{currentmarker}{}%
\end{pgfscope}%
\begin{pgfscope}%
\pgfsys@transformshift{1.105371in}{2.347848in}%
\pgfsys@useobject{currentmarker}{}%
\end{pgfscope}%
\begin{pgfscope}%
\pgfsys@transformshift{1.105010in}{2.353956in}%
\pgfsys@useobject{currentmarker}{}%
\end{pgfscope}%
\begin{pgfscope}%
\pgfsys@transformshift{1.106673in}{2.361041in}%
\pgfsys@useobject{currentmarker}{}%
\end{pgfscope}%
\begin{pgfscope}%
\pgfsys@transformshift{1.106817in}{2.369207in}%
\pgfsys@useobject{currentmarker}{}%
\end{pgfscope}%
\begin{pgfscope}%
\pgfsys@transformshift{1.106693in}{2.373697in}%
\pgfsys@useobject{currentmarker}{}%
\end{pgfscope}%
\begin{pgfscope}%
\pgfsys@transformshift{1.106118in}{2.383321in}%
\pgfsys@useobject{currentmarker}{}%
\end{pgfscope}%
\begin{pgfscope}%
\pgfsys@transformshift{1.105220in}{2.395752in}%
\pgfsys@useobject{currentmarker}{}%
\end{pgfscope}%
\begin{pgfscope}%
\pgfsys@transformshift{1.108923in}{2.409443in}%
\pgfsys@useobject{currentmarker}{}%
\end{pgfscope}%
\begin{pgfscope}%
\pgfsys@transformshift{1.107906in}{2.425285in}%
\pgfsys@useobject{currentmarker}{}%
\end{pgfscope}%
\begin{pgfscope}%
\pgfsys@transformshift{1.109388in}{2.433889in}%
\pgfsys@useobject{currentmarker}{}%
\end{pgfscope}%
\begin{pgfscope}%
\pgfsys@transformshift{1.110681in}{2.443519in}%
\pgfsys@useobject{currentmarker}{}%
\end{pgfscope}%
\begin{pgfscope}%
\pgfsys@transformshift{1.111359in}{2.448820in}%
\pgfsys@useobject{currentmarker}{}%
\end{pgfscope}%
\begin{pgfscope}%
\pgfsys@transformshift{1.112868in}{2.451343in}%
\pgfsys@useobject{currentmarker}{}%
\end{pgfscope}%
\begin{pgfscope}%
\pgfsys@transformshift{1.112465in}{2.457386in}%
\pgfsys@useobject{currentmarker}{}%
\end{pgfscope}%
\begin{pgfscope}%
\pgfsys@transformshift{1.113423in}{2.460577in}%
\pgfsys@useobject{currentmarker}{}%
\end{pgfscope}%
\begin{pgfscope}%
\pgfsys@transformshift{1.113042in}{2.464775in}%
\pgfsys@useobject{currentmarker}{}%
\end{pgfscope}%
\begin{pgfscope}%
\pgfsys@transformshift{1.116194in}{2.472909in}%
\pgfsys@useobject{currentmarker}{}%
\end{pgfscope}%
\begin{pgfscope}%
\pgfsys@transformshift{1.115883in}{2.477697in}%
\pgfsys@useobject{currentmarker}{}%
\end{pgfscope}%
\begin{pgfscope}%
\pgfsys@transformshift{1.117204in}{2.482859in}%
\pgfsys@useobject{currentmarker}{}%
\end{pgfscope}%
\begin{pgfscope}%
\pgfsys@transformshift{1.117825in}{2.489054in}%
\pgfsys@useobject{currentmarker}{}%
\end{pgfscope}%
\begin{pgfscope}%
\pgfsys@transformshift{1.117537in}{2.492466in}%
\pgfsys@useobject{currentmarker}{}%
\end{pgfscope}%
\begin{pgfscope}%
\pgfsys@transformshift{1.119263in}{2.496922in}%
\pgfsys@useobject{currentmarker}{}%
\end{pgfscope}%
\begin{pgfscope}%
\pgfsys@transformshift{1.119200in}{2.505454in}%
\pgfsys@useobject{currentmarker}{}%
\end{pgfscope}%
\begin{pgfscope}%
\pgfsys@transformshift{1.123712in}{2.515085in}%
\pgfsys@useobject{currentmarker}{}%
\end{pgfscope}%
\begin{pgfscope}%
\pgfsys@transformshift{1.123639in}{2.527475in}%
\pgfsys@useobject{currentmarker}{}%
\end{pgfscope}%
\begin{pgfscope}%
\pgfsys@transformshift{1.124954in}{2.534161in}%
\pgfsys@useobject{currentmarker}{}%
\end{pgfscope}%
\begin{pgfscope}%
\pgfsys@transformshift{1.126767in}{2.541394in}%
\pgfsys@useobject{currentmarker}{}%
\end{pgfscope}%
\begin{pgfscope}%
\pgfsys@transformshift{1.126382in}{2.549514in}%
\pgfsys@useobject{currentmarker}{}%
\end{pgfscope}%
\begin{pgfscope}%
\pgfsys@transformshift{1.130030in}{2.560919in}%
\pgfsys@useobject{currentmarker}{}%
\end{pgfscope}%
\begin{pgfscope}%
\pgfsys@transformshift{1.128576in}{2.575718in}%
\pgfsys@useobject{currentmarker}{}%
\end{pgfscope}%
\begin{pgfscope}%
\pgfsys@transformshift{1.135062in}{2.592178in}%
\pgfsys@useobject{currentmarker}{}%
\end{pgfscope}%
\begin{pgfscope}%
\pgfsys@transformshift{1.136823in}{2.611055in}%
\pgfsys@useobject{currentmarker}{}%
\end{pgfscope}%
\begin{pgfscope}%
\pgfsys@transformshift{1.138665in}{2.630658in}%
\pgfsys@useobject{currentmarker}{}%
\end{pgfscope}%
\begin{pgfscope}%
\pgfsys@transformshift{1.141983in}{2.640966in}%
\pgfsys@useobject{currentmarker}{}%
\end{pgfscope}%
\begin{pgfscope}%
\pgfsys@transformshift{1.142553in}{2.646894in}%
\pgfsys@useobject{currentmarker}{}%
\end{pgfscope}%
\begin{pgfscope}%
\pgfsys@transformshift{1.142771in}{2.650163in}%
\pgfsys@useobject{currentmarker}{}%
\end{pgfscope}%
\begin{pgfscope}%
\pgfsys@transformshift{1.144327in}{2.655740in}%
\pgfsys@useobject{currentmarker}{}%
\end{pgfscope}%
\begin{pgfscope}%
\pgfsys@transformshift{1.145035in}{2.664531in}%
\pgfsys@useobject{currentmarker}{}%
\end{pgfscope}%
\begin{pgfscope}%
\pgfsys@transformshift{1.146001in}{2.669285in}%
\pgfsys@useobject{currentmarker}{}%
\end{pgfscope}%
\begin{pgfscope}%
\pgfsys@transformshift{1.147416in}{2.674816in}%
\pgfsys@useobject{currentmarker}{}%
\end{pgfscope}%
\begin{pgfscope}%
\pgfsys@transformshift{1.147033in}{2.684129in}%
\pgfsys@useobject{currentmarker}{}%
\end{pgfscope}%
\begin{pgfscope}%
\pgfsys@transformshift{1.149622in}{2.694002in}%
\pgfsys@useobject{currentmarker}{}%
\end{pgfscope}%
\begin{pgfscope}%
\pgfsys@transformshift{1.150418in}{2.705181in}%
\pgfsys@useobject{currentmarker}{}%
\end{pgfscope}%
\begin{pgfscope}%
\pgfsys@transformshift{1.150020in}{2.711333in}%
\pgfsys@useobject{currentmarker}{}%
\end{pgfscope}%
\begin{pgfscope}%
\pgfsys@transformshift{1.152226in}{2.718696in}%
\pgfsys@useobject{currentmarker}{}%
\end{pgfscope}%
\begin{pgfscope}%
\pgfsys@transformshift{1.152041in}{2.729169in}%
\pgfsys@useobject{currentmarker}{}%
\end{pgfscope}%
\begin{pgfscope}%
\pgfsys@transformshift{1.154743in}{2.740484in}%
\pgfsys@useobject{currentmarker}{}%
\end{pgfscope}%
\begin{pgfscope}%
\pgfsys@transformshift{1.156171in}{2.752642in}%
\pgfsys@useobject{currentmarker}{}%
\end{pgfscope}%
\begin{pgfscope}%
\pgfsys@transformshift{1.154642in}{2.765448in}%
\pgfsys@useobject{currentmarker}{}%
\end{pgfscope}%
\begin{pgfscope}%
\pgfsys@transformshift{1.160681in}{2.780399in}%
\pgfsys@useobject{currentmarker}{}%
\end{pgfscope}%
\begin{pgfscope}%
\pgfsys@transformshift{1.159755in}{2.798370in}%
\pgfsys@useobject{currentmarker}{}%
\end{pgfscope}%
\begin{pgfscope}%
\pgfsys@transformshift{1.163364in}{2.817555in}%
\pgfsys@useobject{currentmarker}{}%
\end{pgfscope}%
\begin{pgfscope}%
\pgfsys@transformshift{1.167227in}{2.837481in}%
\pgfsys@useobject{currentmarker}{}%
\end{pgfscope}%
\begin{pgfscope}%
\pgfsys@transformshift{1.164704in}{2.848355in}%
\pgfsys@useobject{currentmarker}{}%
\end{pgfscope}%
\begin{pgfscope}%
\pgfsys@transformshift{1.167920in}{2.863446in}%
\pgfsys@useobject{currentmarker}{}%
\end{pgfscope}%
\begin{pgfscope}%
\pgfsys@transformshift{1.166854in}{2.880810in}%
\pgfsys@useobject{currentmarker}{}%
\end{pgfscope}%
\begin{pgfscope}%
\pgfsys@transformshift{1.170172in}{2.899123in}%
\pgfsys@useobject{currentmarker}{}%
\end{pgfscope}%
\begin{pgfscope}%
\pgfsys@transformshift{1.173940in}{2.918201in}%
\pgfsys@useobject{currentmarker}{}%
\end{pgfscope}%
\begin{pgfscope}%
\pgfsys@transformshift{1.172187in}{2.928752in}%
\pgfsys@useobject{currentmarker}{}%
\end{pgfscope}%
\begin{pgfscope}%
\pgfsys@transformshift{1.177456in}{2.943797in}%
\pgfsys@useobject{currentmarker}{}%
\end{pgfscope}%
\begin{pgfscope}%
\pgfsys@transformshift{1.175114in}{2.960514in}%
\pgfsys@useobject{currentmarker}{}%
\end{pgfscope}%
\begin{pgfscope}%
\pgfsys@transformshift{1.176734in}{2.969655in}%
\pgfsys@useobject{currentmarker}{}%
\end{pgfscope}%
\begin{pgfscope}%
\pgfsys@transformshift{1.177795in}{2.974650in}%
\pgfsys@useobject{currentmarker}{}%
\end{pgfscope}%
\begin{pgfscope}%
\pgfsys@transformshift{1.176490in}{2.981103in}%
\pgfsys@useobject{currentmarker}{}%
\end{pgfscope}%
\begin{pgfscope}%
\pgfsys@transformshift{1.180644in}{2.992221in}%
\pgfsys@useobject{currentmarker}{}%
\end{pgfscope}%
\begin{pgfscope}%
\pgfsys@transformshift{1.179021in}{3.005548in}%
\pgfsys@useobject{currentmarker}{}%
\end{pgfscope}%
\begin{pgfscope}%
\pgfsys@transformshift{1.180864in}{3.012699in}%
\pgfsys@useobject{currentmarker}{}%
\end{pgfscope}%
\begin{pgfscope}%
\pgfsys@transformshift{1.182912in}{3.020370in}%
\pgfsys@useobject{currentmarker}{}%
\end{pgfscope}%
\begin{pgfscope}%
\pgfsys@transformshift{1.182243in}{3.024686in}%
\pgfsys@useobject{currentmarker}{}%
\end{pgfscope}%
\begin{pgfscope}%
\pgfsys@transformshift{1.184620in}{3.033833in}%
\pgfsys@useobject{currentmarker}{}%
\end{pgfscope}%
\begin{pgfscope}%
\pgfsys@transformshift{1.184308in}{3.043876in}%
\pgfsys@useobject{currentmarker}{}%
\end{pgfscope}%
\begin{pgfscope}%
\pgfsys@transformshift{1.184416in}{3.049401in}%
\pgfsys@useobject{currentmarker}{}%
\end{pgfscope}%
\begin{pgfscope}%
\pgfsys@transformshift{1.187144in}{3.055450in}%
\pgfsys@useobject{currentmarker}{}%
\end{pgfscope}%
\begin{pgfscope}%
\pgfsys@transformshift{1.186675in}{3.065680in}%
\pgfsys@useobject{currentmarker}{}%
\end{pgfscope}%
\begin{pgfscope}%
\pgfsys@transformshift{1.187921in}{3.071174in}%
\pgfsys@useobject{currentmarker}{}%
\end{pgfscope}%
\begin{pgfscope}%
\pgfsys@transformshift{1.188635in}{3.074189in}%
\pgfsys@useobject{currentmarker}{}%
\end{pgfscope}%
\begin{pgfscope}%
\pgfsys@transformshift{1.187891in}{3.078608in}%
\pgfsys@useobject{currentmarker}{}%
\end{pgfscope}%
\begin{pgfscope}%
\pgfsys@transformshift{1.190040in}{3.085951in}%
\pgfsys@useobject{currentmarker}{}%
\end{pgfscope}%
\begin{pgfscope}%
\pgfsys@transformshift{1.189631in}{3.094561in}%
\pgfsys@useobject{currentmarker}{}%
\end{pgfscope}%
\begin{pgfscope}%
\pgfsys@transformshift{1.189766in}{3.099300in}%
\pgfsys@useobject{currentmarker}{}%
\end{pgfscope}%
\begin{pgfscope}%
\pgfsys@transformshift{1.190656in}{3.101751in}%
\pgfsys@useobject{currentmarker}{}%
\end{pgfscope}%
\begin{pgfscope}%
\pgfsys@transformshift{1.190025in}{3.107435in}%
\pgfsys@useobject{currentmarker}{}%
\end{pgfscope}%
\begin{pgfscope}%
\pgfsys@transformshift{1.190813in}{3.110480in}%
\pgfsys@useobject{currentmarker}{}%
\end{pgfscope}%
\begin{pgfscope}%
\pgfsys@transformshift{1.190806in}{3.112210in}%
\pgfsys@useobject{currentmarker}{}%
\end{pgfscope}%
\begin{pgfscope}%
\pgfsys@transformshift{1.190511in}{3.114573in}%
\pgfsys@useobject{currentmarker}{}%
\end{pgfscope}%
\begin{pgfscope}%
\pgfsys@transformshift{1.193823in}{3.121119in}%
\pgfsys@useobject{currentmarker}{}%
\end{pgfscope}%
\begin{pgfscope}%
\pgfsys@transformshift{1.192840in}{3.130468in}%
\pgfsys@useobject{currentmarker}{}%
\end{pgfscope}%
\begin{pgfscope}%
\pgfsys@transformshift{1.196041in}{3.139931in}%
\pgfsys@useobject{currentmarker}{}%
\end{pgfscope}%
\begin{pgfscope}%
\pgfsys@transformshift{1.196708in}{3.145384in}%
\pgfsys@useobject{currentmarker}{}%
\end{pgfscope}%
\begin{pgfscope}%
\pgfsys@transformshift{1.195634in}{3.151901in}%
\pgfsys@useobject{currentmarker}{}%
\end{pgfscope}%
\begin{pgfscope}%
\pgfsys@transformshift{1.199247in}{3.161728in}%
\pgfsys@useobject{currentmarker}{}%
\end{pgfscope}%
\begin{pgfscope}%
\pgfsys@transformshift{1.198855in}{3.173766in}%
\pgfsys@useobject{currentmarker}{}%
\end{pgfscope}%
\begin{pgfscope}%
\pgfsys@transformshift{1.199562in}{3.180353in}%
\pgfsys@useobject{currentmarker}{}%
\end{pgfscope}%
\begin{pgfscope}%
\pgfsys@transformshift{1.200619in}{3.183840in}%
\pgfsys@useobject{currentmarker}{}%
\end{pgfscope}%
\begin{pgfscope}%
\pgfsys@transformshift{1.199876in}{3.190187in}%
\pgfsys@useobject{currentmarker}{}%
\end{pgfscope}%
\begin{pgfscope}%
\pgfsys@transformshift{1.201614in}{3.198296in}%
\pgfsys@useobject{currentmarker}{}%
\end{pgfscope}%
\begin{pgfscope}%
\pgfsys@transformshift{1.202113in}{3.202830in}%
\pgfsys@useobject{currentmarker}{}%
\end{pgfscope}%
\begin{pgfscope}%
\pgfsys@transformshift{1.201686in}{3.205302in}%
\pgfsys@useobject{currentmarker}{}%
\end{pgfscope}%
\begin{pgfscope}%
\pgfsys@transformshift{1.203739in}{3.211538in}%
\pgfsys@useobject{currentmarker}{}%
\end{pgfscope}%
\begin{pgfscope}%
\pgfsys@transformshift{1.203213in}{3.220162in}%
\pgfsys@useobject{currentmarker}{}%
\end{pgfscope}%
\begin{pgfscope}%
\pgfsys@transformshift{1.204389in}{3.224766in}%
\pgfsys@useobject{currentmarker}{}%
\end{pgfscope}%
\begin{pgfscope}%
\pgfsys@transformshift{1.205130in}{3.227272in}%
\pgfsys@useobject{currentmarker}{}%
\end{pgfscope}%
\begin{pgfscope}%
\pgfsys@transformshift{1.204302in}{3.231600in}%
\pgfsys@useobject{currentmarker}{}%
\end{pgfscope}%
\begin{pgfscope}%
\pgfsys@transformshift{1.206340in}{3.239123in}%
\pgfsys@useobject{currentmarker}{}%
\end{pgfscope}%
\begin{pgfscope}%
\pgfsys@transformshift{1.206371in}{3.248413in}%
\pgfsys@useobject{currentmarker}{}%
\end{pgfscope}%
\begin{pgfscope}%
\pgfsys@transformshift{1.206095in}{3.253515in}%
\pgfsys@useobject{currentmarker}{}%
\end{pgfscope}%
\begin{pgfscope}%
\pgfsys@transformshift{1.208544in}{3.259837in}%
\pgfsys@useobject{currentmarker}{}%
\end{pgfscope}%
\begin{pgfscope}%
\pgfsys@transformshift{1.207689in}{3.269118in}%
\pgfsys@useobject{currentmarker}{}%
\end{pgfscope}%
\begin{pgfscope}%
\pgfsys@transformshift{1.208779in}{3.274127in}%
\pgfsys@useobject{currentmarker}{}%
\end{pgfscope}%
\begin{pgfscope}%
\pgfsys@transformshift{1.209320in}{3.276894in}%
\pgfsys@useobject{currentmarker}{}%
\end{pgfscope}%
\begin{pgfscope}%
\pgfsys@transformshift{1.208946in}{3.278398in}%
\pgfsys@useobject{currentmarker}{}%
\end{pgfscope}%
\begin{pgfscope}%
\pgfsys@transformshift{1.210702in}{3.285444in}%
\pgfsys@useobject{currentmarker}{}%
\end{pgfscope}%
\begin{pgfscope}%
\pgfsys@transformshift{1.210264in}{3.289414in}%
\pgfsys@useobject{currentmarker}{}%
\end{pgfscope}%
\begin{pgfscope}%
\pgfsys@transformshift{1.210752in}{3.291555in}%
\pgfsys@useobject{currentmarker}{}%
\end{pgfscope}%
\begin{pgfscope}%
\pgfsys@transformshift{1.211282in}{3.294285in}%
\pgfsys@useobject{currentmarker}{}%
\end{pgfscope}%
\begin{pgfscope}%
\pgfsys@transformshift{1.209781in}{3.300475in}%
\pgfsys@useobject{currentmarker}{}%
\end{pgfscope}%
\begin{pgfscope}%
\pgfsys@transformshift{1.211561in}{3.308425in}%
\pgfsys@useobject{currentmarker}{}%
\end{pgfscope}%
\begin{pgfscope}%
\pgfsys@transformshift{1.212009in}{3.312883in}%
\pgfsys@useobject{currentmarker}{}%
\end{pgfscope}%
\begin{pgfscope}%
\pgfsys@transformshift{1.211464in}{3.318065in}%
\pgfsys@useobject{currentmarker}{}%
\end{pgfscope}%
\begin{pgfscope}%
\pgfsys@transformshift{1.214940in}{3.327973in}%
\pgfsys@useobject{currentmarker}{}%
\end{pgfscope}%
\begin{pgfscope}%
\pgfsys@transformshift{1.214363in}{3.339899in}%
\pgfsys@useobject{currentmarker}{}%
\end{pgfscope}%
\begin{pgfscope}%
\pgfsys@transformshift{1.215027in}{3.346432in}%
\pgfsys@useobject{currentmarker}{}%
\end{pgfscope}%
\begin{pgfscope}%
\pgfsys@transformshift{1.216273in}{3.349822in}%
\pgfsys@useobject{currentmarker}{}%
\end{pgfscope}%
\begin{pgfscope}%
\pgfsys@transformshift{1.215289in}{3.356391in}%
\pgfsys@useobject{currentmarker}{}%
\end{pgfscope}%
\begin{pgfscope}%
\pgfsys@transformshift{1.217318in}{3.363371in}%
\pgfsys@useobject{currentmarker}{}%
\end{pgfscope}%
\begin{pgfscope}%
\pgfsys@transformshift{1.217637in}{3.371518in}%
\pgfsys@useobject{currentmarker}{}%
\end{pgfscope}%
\begin{pgfscope}%
\pgfsys@transformshift{1.215989in}{3.380048in}%
\pgfsys@useobject{currentmarker}{}%
\end{pgfscope}%
\begin{pgfscope}%
\pgfsys@transformshift{1.220231in}{3.392257in}%
\pgfsys@useobject{currentmarker}{}%
\end{pgfscope}%
\begin{pgfscope}%
\pgfsys@transformshift{1.219723in}{3.406632in}%
\pgfsys@useobject{currentmarker}{}%
\end{pgfscope}%
\begin{pgfscope}%
\pgfsys@transformshift{1.221755in}{3.414278in}%
\pgfsys@useobject{currentmarker}{}%
\end{pgfscope}%
\begin{pgfscope}%
\pgfsys@transformshift{1.223226in}{3.418373in}%
\pgfsys@useobject{currentmarker}{}%
\end{pgfscope}%
\begin{pgfscope}%
\pgfsys@transformshift{1.222128in}{3.426129in}%
\pgfsys@useobject{currentmarker}{}%
\end{pgfscope}%
\begin{pgfscope}%
\pgfsys@transformshift{1.225190in}{3.435176in}%
\pgfsys@useobject{currentmarker}{}%
\end{pgfscope}%
\begin{pgfscope}%
\pgfsys@transformshift{1.225316in}{3.446769in}%
\pgfsys@useobject{currentmarker}{}%
\end{pgfscope}%
\begin{pgfscope}%
\pgfsys@transformshift{1.225317in}{3.453145in}%
\pgfsys@useobject{currentmarker}{}%
\end{pgfscope}%
\begin{pgfscope}%
\pgfsys@transformshift{1.227528in}{3.460000in}%
\pgfsys@useobject{currentmarker}{}%
\end{pgfscope}%
\begin{pgfscope}%
\pgfsys@transformshift{1.225999in}{3.469597in}%
\pgfsys@useobject{currentmarker}{}%
\end{pgfscope}%
\begin{pgfscope}%
\pgfsys@transformshift{1.230218in}{3.479666in}%
\pgfsys@useobject{currentmarker}{}%
\end{pgfscope}%
\begin{pgfscope}%
\pgfsys@transformshift{1.230286in}{3.485670in}%
\pgfsys@useobject{currentmarker}{}%
\end{pgfscope}%
\begin{pgfscope}%
\pgfsys@transformshift{1.229589in}{3.492173in}%
\pgfsys@useobject{currentmarker}{}%
\end{pgfscope}%
\begin{pgfscope}%
\pgfsys@transformshift{1.232990in}{3.502394in}%
\pgfsys@useobject{currentmarker}{}%
\end{pgfscope}%
\begin{pgfscope}%
\pgfsys@transformshift{1.232546in}{3.514343in}%
\pgfsys@useobject{currentmarker}{}%
\end{pgfscope}%
\begin{pgfscope}%
\pgfsys@transformshift{1.233624in}{3.520830in}%
\pgfsys@useobject{currentmarker}{}%
\end{pgfscope}%
\begin{pgfscope}%
\pgfsys@transformshift{1.234499in}{3.524339in}%
\pgfsys@useobject{currentmarker}{}%
\end{pgfscope}%
\begin{pgfscope}%
\pgfsys@transformshift{1.233850in}{3.531976in}%
\pgfsys@useobject{currentmarker}{}%
\end{pgfscope}%
\begin{pgfscope}%
\pgfsys@transformshift{1.235021in}{3.536026in}%
\pgfsys@useobject{currentmarker}{}%
\end{pgfscope}%
\begin{pgfscope}%
\pgfsys@transformshift{1.235205in}{3.538337in}%
\pgfsys@useobject{currentmarker}{}%
\end{pgfscope}%
\begin{pgfscope}%
\pgfsys@transformshift{1.235046in}{3.539602in}%
\pgfsys@useobject{currentmarker}{}%
\end{pgfscope}%
\begin{pgfscope}%
\pgfsys@transformshift{1.237284in}{3.545068in}%
\pgfsys@useobject{currentmarker}{}%
\end{pgfscope}%
\begin{pgfscope}%
\pgfsys@transformshift{1.236450in}{3.551825in}%
\pgfsys@useobject{currentmarker}{}%
\end{pgfscope}%
\begin{pgfscope}%
\pgfsys@transformshift{1.236977in}{3.555532in}%
\pgfsys@useobject{currentmarker}{}%
\end{pgfscope}%
\begin{pgfscope}%
\pgfsys@transformshift{1.238881in}{3.559590in}%
\pgfsys@useobject{currentmarker}{}%
\end{pgfscope}%
\begin{pgfscope}%
\pgfsys@transformshift{1.238285in}{3.566808in}%
\pgfsys@useobject{currentmarker}{}%
\end{pgfscope}%
\begin{pgfscope}%
\pgfsys@transformshift{1.240528in}{3.575528in}%
\pgfsys@useobject{currentmarker}{}%
\end{pgfscope}%
\begin{pgfscope}%
\pgfsys@transformshift{1.240966in}{3.585308in}%
\pgfsys@useobject{currentmarker}{}%
\end{pgfscope}%
\begin{pgfscope}%
\pgfsys@transformshift{1.240306in}{3.590651in}%
\pgfsys@useobject{currentmarker}{}%
\end{pgfscope}%
\begin{pgfscope}%
\pgfsys@transformshift{1.243711in}{3.598570in}%
\pgfsys@useobject{currentmarker}{}%
\end{pgfscope}%
\begin{pgfscope}%
\pgfsys@transformshift{1.241970in}{3.609695in}%
\pgfsys@useobject{currentmarker}{}%
\end{pgfscope}%
\begin{pgfscope}%
\pgfsys@transformshift{1.244953in}{3.621928in}%
\pgfsys@useobject{currentmarker}{}%
\end{pgfscope}%
\begin{pgfscope}%
\pgfsys@transformshift{1.246239in}{3.628733in}%
\pgfsys@useobject{currentmarker}{}%
\end{pgfscope}%
\begin{pgfscope}%
\pgfsys@transformshift{1.245197in}{3.636794in}%
\pgfsys@useobject{currentmarker}{}%
\end{pgfscope}%
\begin{pgfscope}%
\pgfsys@transformshift{1.248942in}{3.649399in}%
\pgfsys@useobject{currentmarker}{}%
\end{pgfscope}%
\begin{pgfscope}%
\pgfsys@transformshift{1.248364in}{3.656609in}%
\pgfsys@useobject{currentmarker}{}%
\end{pgfscope}%
\begin{pgfscope}%
\pgfsys@transformshift{1.249332in}{3.664371in}%
\pgfsys@useobject{currentmarker}{}%
\end{pgfscope}%
\begin{pgfscope}%
\pgfsys@transformshift{1.251530in}{3.672602in}%
\pgfsys@useobject{currentmarker}{}%
\end{pgfscope}%
\begin{pgfscope}%
\pgfsys@transformshift{1.250799in}{3.683523in}%
\pgfsys@useobject{currentmarker}{}%
\end{pgfscope}%
\begin{pgfscope}%
\pgfsys@transformshift{1.254817in}{3.696877in}%
\pgfsys@useobject{currentmarker}{}%
\end{pgfscope}%
\begin{pgfscope}%
\pgfsys@transformshift{1.254496in}{3.712034in}%
\pgfsys@useobject{currentmarker}{}%
\end{pgfscope}%
\begin{pgfscope}%
\pgfsys@transformshift{1.255557in}{3.720305in}%
\pgfsys@useobject{currentmarker}{}%
\end{pgfscope}%
\begin{pgfscope}%
\pgfsys@transformshift{1.256845in}{3.724707in}%
\pgfsys@useobject{currentmarker}{}%
\end{pgfscope}%
\begin{pgfscope}%
\pgfsys@transformshift{1.255940in}{3.731671in}%
\pgfsys@useobject{currentmarker}{}%
\end{pgfscope}%
\begin{pgfscope}%
\pgfsys@transformshift{1.258158in}{3.739840in}%
\pgfsys@useobject{currentmarker}{}%
\end{pgfscope}%
\begin{pgfscope}%
\pgfsys@transformshift{1.258439in}{3.744488in}%
\pgfsys@useobject{currentmarker}{}%
\end{pgfscope}%
\begin{pgfscope}%
\pgfsys@transformshift{1.258202in}{3.747038in}%
\pgfsys@useobject{currentmarker}{}%
\end{pgfscope}%
\begin{pgfscope}%
\pgfsys@transformshift{1.260015in}{3.752536in}%
\pgfsys@useobject{currentmarker}{}%
\end{pgfscope}%
\begin{pgfscope}%
\pgfsys@transformshift{1.259756in}{3.760647in}%
\pgfsys@useobject{currentmarker}{}%
\end{pgfscope}%
\begin{pgfscope}%
\pgfsys@transformshift{1.260397in}{3.765064in}%
\pgfsys@useobject{currentmarker}{}%
\end{pgfscope}%
\begin{pgfscope}%
\pgfsys@transformshift{1.262412in}{3.769957in}%
\pgfsys@useobject{currentmarker}{}%
\end{pgfscope}%
\begin{pgfscope}%
\pgfsys@transformshift{1.261071in}{3.778068in}%
\pgfsys@useobject{currentmarker}{}%
\end{pgfscope}%
\begin{pgfscope}%
\pgfsys@transformshift{1.264642in}{3.788489in}%
\pgfsys@useobject{currentmarker}{}%
\end{pgfscope}%
\begin{pgfscope}%
\pgfsys@transformshift{1.263343in}{3.801192in}%
\pgfsys@useobject{currentmarker}{}%
\end{pgfscope}%
\begin{pgfscope}%
\pgfsys@transformshift{1.263538in}{3.808212in}%
\pgfsys@useobject{currentmarker}{}%
\end{pgfscope}%
\begin{pgfscope}%
\pgfsys@transformshift{1.266407in}{3.817061in}%
\pgfsys@useobject{currentmarker}{}%
\end{pgfscope}%
\begin{pgfscope}%
\pgfsys@transformshift{1.264307in}{3.828491in}%
\pgfsys@useobject{currentmarker}{}%
\end{pgfscope}%
\begin{pgfscope}%
\pgfsys@transformshift{1.268126in}{3.840913in}%
\pgfsys@useobject{currentmarker}{}%
\end{pgfscope}%
\begin{pgfscope}%
\pgfsys@transformshift{1.269178in}{3.854588in}%
\pgfsys@useobject{currentmarker}{}%
\end{pgfscope}%
\begin{pgfscope}%
\pgfsys@transformshift{1.268035in}{3.862045in}%
\pgfsys@useobject{currentmarker}{}%
\end{pgfscope}%
\begin{pgfscope}%
\pgfsys@transformshift{1.271083in}{3.873644in}%
\pgfsys@useobject{currentmarker}{}%
\end{pgfscope}%
\begin{pgfscope}%
\pgfsys@transformshift{1.270896in}{3.886331in}%
\pgfsys@useobject{currentmarker}{}%
\end{pgfscope}%
\begin{pgfscope}%
\pgfsys@transformshift{1.271555in}{3.893279in}%
\pgfsys@useobject{currentmarker}{}%
\end{pgfscope}%
\begin{pgfscope}%
\pgfsys@transformshift{1.274098in}{3.900380in}%
\pgfsys@useobject{currentmarker}{}%
\end{pgfscope}%
\begin{pgfscope}%
\pgfsys@transformshift{1.273369in}{3.911273in}%
\pgfsys@useobject{currentmarker}{}%
\end{pgfscope}%
\begin{pgfscope}%
\pgfsys@transformshift{1.276105in}{3.922699in}%
\pgfsys@useobject{currentmarker}{}%
\end{pgfscope}%
\begin{pgfscope}%
\pgfsys@transformshift{1.276159in}{3.929161in}%
\pgfsys@useobject{currentmarker}{}%
\end{pgfscope}%
\begin{pgfscope}%
\pgfsys@transformshift{1.275448in}{3.932644in}%
\pgfsys@useobject{currentmarker}{}%
\end{pgfscope}%
\begin{pgfscope}%
\pgfsys@transformshift{1.277425in}{3.939200in}%
\pgfsys@useobject{currentmarker}{}%
\end{pgfscope}%
\begin{pgfscope}%
\pgfsys@transformshift{1.277358in}{3.948211in}%
\pgfsys@useobject{currentmarker}{}%
\end{pgfscope}%
\begin{pgfscope}%
\pgfsys@transformshift{1.278082in}{3.953115in}%
\pgfsys@useobject{currentmarker}{}%
\end{pgfscope}%
\begin{pgfscope}%
\pgfsys@transformshift{1.279765in}{3.958440in}%
\pgfsys@useobject{currentmarker}{}%
\end{pgfscope}%
\begin{pgfscope}%
\pgfsys@transformshift{1.278624in}{3.966100in}%
\pgfsys@useobject{currentmarker}{}%
\end{pgfscope}%
\begin{pgfscope}%
\pgfsys@transformshift{1.282050in}{3.975869in}%
\pgfsys@useobject{currentmarker}{}%
\end{pgfscope}%
\begin{pgfscope}%
\pgfsys@transformshift{1.282171in}{3.986917in}%
\pgfsys@useobject{currentmarker}{}%
\end{pgfscope}%
\begin{pgfscope}%
\pgfsys@transformshift{1.282384in}{3.992990in}%
\pgfsys@useobject{currentmarker}{}%
\end{pgfscope}%
\begin{pgfscope}%
\pgfsys@transformshift{1.283567in}{3.996116in}%
\pgfsys@useobject{currentmarker}{}%
\end{pgfscope}%
\begin{pgfscope}%
\pgfsys@transformshift{1.283014in}{4.002368in}%
\pgfsys@useobject{currentmarker}{}%
\end{pgfscope}%
\begin{pgfscope}%
\pgfsys@transformshift{1.283826in}{4.005722in}%
\pgfsys@useobject{currentmarker}{}%
\end{pgfscope}%
\begin{pgfscope}%
\pgfsys@transformshift{1.284142in}{4.007594in}%
\pgfsys@useobject{currentmarker}{}%
\end{pgfscope}%
\begin{pgfscope}%
\pgfsys@transformshift{1.283686in}{4.010579in}%
\pgfsys@useobject{currentmarker}{}%
\end{pgfscope}%
\begin{pgfscope}%
\pgfsys@transformshift{1.285644in}{4.017683in}%
\pgfsys@useobject{currentmarker}{}%
\end{pgfscope}%
\begin{pgfscope}%
\pgfsys@transformshift{1.285163in}{4.025714in}%
\pgfsys@useobject{currentmarker}{}%
\end{pgfscope}%
\begin{pgfscope}%
\pgfsys@transformshift{1.285245in}{4.030139in}%
\pgfsys@useobject{currentmarker}{}%
\end{pgfscope}%
\begin{pgfscope}%
\pgfsys@transformshift{1.285610in}{4.032545in}%
\pgfsys@useobject{currentmarker}{}%
\end{pgfscope}%
\begin{pgfscope}%
\pgfsys@transformshift{1.285772in}{4.033874in}%
\pgfsys@useobject{currentmarker}{}%
\end{pgfscope}%
\begin{pgfscope}%
\pgfsys@transformshift{1.286039in}{4.035830in}%
\pgfsys@useobject{currentmarker}{}%
\end{pgfscope}%
\begin{pgfscope}%
\pgfsys@transformshift{1.286151in}{4.036910in}%
\pgfsys@useobject{currentmarker}{}%
\end{pgfscope}%
\begin{pgfscope}%
\pgfsys@transformshift{1.286232in}{4.037501in}%
\pgfsys@useobject{currentmarker}{}%
\end{pgfscope}%
\begin{pgfscope}%
\pgfsys@transformshift{1.286282in}{4.037826in}%
\pgfsys@useobject{currentmarker}{}%
\end{pgfscope}%
\begin{pgfscope}%
\pgfsys@transformshift{1.286306in}{4.038005in}%
\pgfsys@useobject{currentmarker}{}%
\end{pgfscope}%
\begin{pgfscope}%
\pgfsys@transformshift{1.286319in}{4.038103in}%
\pgfsys@useobject{currentmarker}{}%
\end{pgfscope}%
\begin{pgfscope}%
\pgfsys@transformshift{1.286333in}{4.038156in}%
\pgfsys@useobject{currentmarker}{}%
\end{pgfscope}%
\begin{pgfscope}%
\pgfsys@transformshift{1.286340in}{4.038185in}%
\pgfsys@useobject{currentmarker}{}%
\end{pgfscope}%
\begin{pgfscope}%
\pgfsys@transformshift{1.286343in}{4.038201in}%
\pgfsys@useobject{currentmarker}{}%
\end{pgfscope}%
\begin{pgfscope}%
\pgfsys@transformshift{1.286346in}{4.038210in}%
\pgfsys@useobject{currentmarker}{}%
\end{pgfscope}%
\begin{pgfscope}%
\pgfsys@transformshift{1.286347in}{4.038215in}%
\pgfsys@useobject{currentmarker}{}%
\end{pgfscope}%
\begin{pgfscope}%
\pgfsys@transformshift{1.286348in}{4.038218in}%
\pgfsys@useobject{currentmarker}{}%
\end{pgfscope}%
\begin{pgfscope}%
\pgfsys@transformshift{1.286349in}{4.038219in}%
\pgfsys@useobject{currentmarker}{}%
\end{pgfscope}%
\begin{pgfscope}%
\pgfsys@transformshift{1.286764in}{4.038873in}%
\pgfsys@useobject{currentmarker}{}%
\end{pgfscope}%
\begin{pgfscope}%
\pgfsys@transformshift{1.288084in}{4.040349in}%
\pgfsys@useobject{currentmarker}{}%
\end{pgfscope}%
\begin{pgfscope}%
\pgfsys@transformshift{1.290247in}{4.041949in}%
\pgfsys@useobject{currentmarker}{}%
\end{pgfscope}%
\begin{pgfscope}%
\pgfsys@transformshift{1.291541in}{4.042664in}%
\pgfsys@useobject{currentmarker}{}%
\end{pgfscope}%
\begin{pgfscope}%
\pgfsys@transformshift{1.292305in}{4.042943in}%
\pgfsys@useobject{currentmarker}{}%
\end{pgfscope}%
\begin{pgfscope}%
\pgfsys@transformshift{1.292749in}{4.043000in}%
\pgfsys@useobject{currentmarker}{}%
\end{pgfscope}%
\begin{pgfscope}%
\pgfsys@transformshift{1.294459in}{4.042815in}%
\pgfsys@useobject{currentmarker}{}%
\end{pgfscope}%
\begin{pgfscope}%
\pgfsys@transformshift{1.296662in}{4.042278in}%
\pgfsys@useobject{currentmarker}{}%
\end{pgfscope}%
\begin{pgfscope}%
\pgfsys@transformshift{1.297816in}{4.041804in}%
\pgfsys@useobject{currentmarker}{}%
\end{pgfscope}%
\begin{pgfscope}%
\pgfsys@transformshift{1.298480in}{4.041634in}%
\pgfsys@useobject{currentmarker}{}%
\end{pgfscope}%
\begin{pgfscope}%
\pgfsys@transformshift{1.300398in}{4.040676in}%
\pgfsys@useobject{currentmarker}{}%
\end{pgfscope}%
\begin{pgfscope}%
\pgfsys@transformshift{1.301452in}{4.040147in}%
\pgfsys@useobject{currentmarker}{}%
\end{pgfscope}%
\begin{pgfscope}%
\pgfsys@transformshift{1.303535in}{4.039053in}%
\pgfsys@useobject{currentmarker}{}%
\end{pgfscope}%
\begin{pgfscope}%
\pgfsys@transformshift{1.306566in}{4.037238in}%
\pgfsys@useobject{currentmarker}{}%
\end{pgfscope}%
\begin{pgfscope}%
\pgfsys@transformshift{1.308245in}{4.036261in}%
\pgfsys@useobject{currentmarker}{}%
\end{pgfscope}%
\begin{pgfscope}%
\pgfsys@transformshift{1.309088in}{4.035604in}%
\pgfsys@useobject{currentmarker}{}%
\end{pgfscope}%
\begin{pgfscope}%
\pgfsys@transformshift{1.309544in}{4.035233in}%
\pgfsys@useobject{currentmarker}{}%
\end{pgfscope}%
\begin{pgfscope}%
\pgfsys@transformshift{1.311257in}{4.033955in}%
\pgfsys@useobject{currentmarker}{}%
\end{pgfscope}%
\begin{pgfscope}%
\pgfsys@transformshift{1.312139in}{4.033178in}%
\pgfsys@useobject{currentmarker}{}%
\end{pgfscope}%
\begin{pgfscope}%
\pgfsys@transformshift{1.314434in}{4.030878in}%
\pgfsys@useobject{currentmarker}{}%
\end{pgfscope}%
\begin{pgfscope}%
\pgfsys@transformshift{1.315698in}{4.029615in}%
\pgfsys@useobject{currentmarker}{}%
\end{pgfscope}%
\begin{pgfscope}%
\pgfsys@transformshift{1.318320in}{4.027259in}%
\pgfsys@useobject{currentmarker}{}%
\end{pgfscope}%
\begin{pgfscope}%
\pgfsys@transformshift{1.322278in}{4.024488in}%
\pgfsys@useobject{currentmarker}{}%
\end{pgfscope}%
\begin{pgfscope}%
\pgfsys@transformshift{1.327465in}{4.017349in}%
\pgfsys@useobject{currentmarker}{}%
\end{pgfscope}%
\begin{pgfscope}%
\pgfsys@transformshift{1.334867in}{4.010134in}%
\pgfsys@useobject{currentmarker}{}%
\end{pgfscope}%
\begin{pgfscope}%
\pgfsys@transformshift{1.339162in}{4.006410in}%
\pgfsys@useobject{currentmarker}{}%
\end{pgfscope}%
\begin{pgfscope}%
\pgfsys@transformshift{1.343442in}{3.999856in}%
\pgfsys@useobject{currentmarker}{}%
\end{pgfscope}%
\begin{pgfscope}%
\pgfsys@transformshift{1.350977in}{3.992690in}%
\pgfsys@useobject{currentmarker}{}%
\end{pgfscope}%
\begin{pgfscope}%
\pgfsys@transformshift{1.354209in}{3.987971in}%
\pgfsys@useobject{currentmarker}{}%
\end{pgfscope}%
\begin{pgfscope}%
\pgfsys@transformshift{1.358638in}{3.983007in}%
\pgfsys@useobject{currentmarker}{}%
\end{pgfscope}%
\begin{pgfscope}%
\pgfsys@transformshift{1.365174in}{3.978828in}%
\pgfsys@useobject{currentmarker}{}%
\end{pgfscope}%
\begin{pgfscope}%
\pgfsys@transformshift{1.371386in}{3.970162in}%
\pgfsys@useobject{currentmarker}{}%
\end{pgfscope}%
\begin{pgfscope}%
\pgfsys@transformshift{1.380435in}{3.962780in}%
\pgfsys@useobject{currentmarker}{}%
\end{pgfscope}%
\begin{pgfscope}%
\pgfsys@transformshift{1.387941in}{3.952396in}%
\pgfsys@useobject{currentmarker}{}%
\end{pgfscope}%
\begin{pgfscope}%
\pgfsys@transformshift{1.397749in}{3.942234in}%
\pgfsys@useobject{currentmarker}{}%
\end{pgfscope}%
\begin{pgfscope}%
\pgfsys@transformshift{1.410066in}{3.932908in}%
\pgfsys@useobject{currentmarker}{}%
\end{pgfscope}%
\begin{pgfscope}%
\pgfsys@transformshift{1.421635in}{3.918335in}%
\pgfsys@useobject{currentmarker}{}%
\end{pgfscope}%
\begin{pgfscope}%
\pgfsys@transformshift{1.429500in}{3.911787in}%
\pgfsys@useobject{currentmarker}{}%
\end{pgfscope}%
\begin{pgfscope}%
\pgfsys@transformshift{1.433370in}{3.907700in}%
\pgfsys@useobject{currentmarker}{}%
\end{pgfscope}%
\begin{pgfscope}%
\pgfsys@transformshift{1.437884in}{3.902203in}%
\pgfsys@useobject{currentmarker}{}%
\end{pgfscope}%
\begin{pgfscope}%
\pgfsys@transformshift{1.446439in}{3.896358in}%
\pgfsys@useobject{currentmarker}{}%
\end{pgfscope}%
\begin{pgfscope}%
\pgfsys@transformshift{1.449974in}{3.891888in}%
\pgfsys@useobject{currentmarker}{}%
\end{pgfscope}%
\begin{pgfscope}%
\pgfsys@transformshift{1.452384in}{3.889885in}%
\pgfsys@useobject{currentmarker}{}%
\end{pgfscope}%
\begin{pgfscope}%
\pgfsys@transformshift{1.455542in}{3.887529in}%
\pgfsys@useobject{currentmarker}{}%
\end{pgfscope}%
\begin{pgfscope}%
\pgfsys@transformshift{1.459802in}{3.881949in}%
\pgfsys@useobject{currentmarker}{}%
\end{pgfscope}%
\begin{pgfscope}%
\pgfsys@transformshift{1.465946in}{3.876582in}%
\pgfsys@useobject{currentmarker}{}%
\end{pgfscope}%
\begin{pgfscope}%
\pgfsys@transformshift{1.472894in}{3.870635in}%
\pgfsys@useobject{currentmarker}{}%
\end{pgfscope}%
\begin{pgfscope}%
\pgfsys@transformshift{1.478216in}{3.862336in}%
\pgfsys@useobject{currentmarker}{}%
\end{pgfscope}%
\begin{pgfscope}%
\pgfsys@transformshift{1.488275in}{3.853697in}%
\pgfsys@useobject{currentmarker}{}%
\end{pgfscope}%
\begin{pgfscope}%
\pgfsys@transformshift{1.493035in}{3.848172in}%
\pgfsys@useobject{currentmarker}{}%
\end{pgfscope}%
\begin{pgfscope}%
\pgfsys@transformshift{1.495642in}{3.845123in}%
\pgfsys@useobject{currentmarker}{}%
\end{pgfscope}%
\begin{pgfscope}%
\pgfsys@transformshift{1.497445in}{3.843853in}%
\pgfsys@useobject{currentmarker}{}%
\end{pgfscope}%
\begin{pgfscope}%
\pgfsys@transformshift{1.500933in}{3.840098in}%
\pgfsys@useobject{currentmarker}{}%
\end{pgfscope}%
\begin{pgfscope}%
\pgfsys@transformshift{1.503098in}{3.838292in}%
\pgfsys@useobject{currentmarker}{}%
\end{pgfscope}%
\begin{pgfscope}%
\pgfsys@transformshift{1.506518in}{3.836033in}%
\pgfsys@useobject{currentmarker}{}%
\end{pgfscope}%
\begin{pgfscope}%
\pgfsys@transformshift{1.509270in}{3.832278in}%
\pgfsys@useobject{currentmarker}{}%
\end{pgfscope}%
\begin{pgfscope}%
\pgfsys@transformshift{1.516193in}{3.826681in}%
\pgfsys@useobject{currentmarker}{}%
\end{pgfscope}%
\begin{pgfscope}%
\pgfsys@transformshift{1.523383in}{3.820145in}%
\pgfsys@useobject{currentmarker}{}%
\end{pgfscope}%
\begin{pgfscope}%
\pgfsys@transformshift{1.529950in}{3.811531in}%
\pgfsys@useobject{currentmarker}{}%
\end{pgfscope}%
\begin{pgfscope}%
\pgfsys@transformshift{1.542344in}{3.802932in}%
\pgfsys@useobject{currentmarker}{}%
\end{pgfscope}%
\begin{pgfscope}%
\pgfsys@transformshift{1.547446in}{3.796388in}%
\pgfsys@useobject{currentmarker}{}%
\end{pgfscope}%
\begin{pgfscope}%
\pgfsys@transformshift{1.553797in}{3.790190in}%
\pgfsys@useobject{currentmarker}{}%
\end{pgfscope}%
\begin{pgfscope}%
\pgfsys@transformshift{1.561796in}{3.784879in}%
\pgfsys@useobject{currentmarker}{}%
\end{pgfscope}%
\begin{pgfscope}%
\pgfsys@transformshift{1.569021in}{3.774706in}%
\pgfsys@useobject{currentmarker}{}%
\end{pgfscope}%
\begin{pgfscope}%
\pgfsys@transformshift{1.579912in}{3.767388in}%
\pgfsys@useobject{currentmarker}{}%
\end{pgfscope}%
\begin{pgfscope}%
\pgfsys@transformshift{1.585106in}{3.762378in}%
\pgfsys@useobject{currentmarker}{}%
\end{pgfscope}%
\begin{pgfscope}%
\pgfsys@transformshift{1.589971in}{3.755815in}%
\pgfsys@useobject{currentmarker}{}%
\end{pgfscope}%
\begin{pgfscope}%
\pgfsys@transformshift{1.598152in}{3.750073in}%
\pgfsys@useobject{currentmarker}{}%
\end{pgfscope}%
\begin{pgfscope}%
\pgfsys@transformshift{1.606586in}{3.741893in}%
\pgfsys@useobject{currentmarker}{}%
\end{pgfscope}%
\begin{pgfscope}%
\pgfsys@transformshift{1.615187in}{3.732202in}%
\pgfsys@useobject{currentmarker}{}%
\end{pgfscope}%
\begin{pgfscope}%
\pgfsys@transformshift{1.628157in}{3.721973in}%
\pgfsys@useobject{currentmarker}{}%
\end{pgfscope}%
\begin{pgfscope}%
\pgfsys@transformshift{1.639958in}{3.709462in}%
\pgfsys@useobject{currentmarker}{}%
\end{pgfscope}%
\begin{pgfscope}%
\pgfsys@transformshift{1.646496in}{3.702625in}%
\pgfsys@useobject{currentmarker}{}%
\end{pgfscope}%
\begin{pgfscope}%
\pgfsys@transformshift{1.650831in}{3.699748in}%
\pgfsys@useobject{currentmarker}{}%
\end{pgfscope}%
\begin{pgfscope}%
\pgfsys@transformshift{1.655877in}{3.693850in}%
\pgfsys@useobject{currentmarker}{}%
\end{pgfscope}%
\begin{pgfscope}%
\pgfsys@transformshift{1.662912in}{3.688092in}%
\pgfsys@useobject{currentmarker}{}%
\end{pgfscope}%
\begin{pgfscope}%
\pgfsys@transformshift{1.671070in}{3.681553in}%
\pgfsys@useobject{currentmarker}{}%
\end{pgfscope}%
\begin{pgfscope}%
\pgfsys@transformshift{1.678375in}{3.672823in}%
\pgfsys@useobject{currentmarker}{}%
\end{pgfscope}%
\begin{pgfscope}%
\pgfsys@transformshift{1.690351in}{3.664504in}%
\pgfsys@useobject{currentmarker}{}%
\end{pgfscope}%
\begin{pgfscope}%
\pgfsys@transformshift{1.696207in}{3.659025in}%
\pgfsys@useobject{currentmarker}{}%
\end{pgfscope}%
\begin{pgfscope}%
\pgfsys@transformshift{1.701721in}{3.652012in}%
\pgfsys@useobject{currentmarker}{}%
\end{pgfscope}%
\begin{pgfscope}%
\pgfsys@transformshift{1.711059in}{3.645048in}%
\pgfsys@useobject{currentmarker}{}%
\end{pgfscope}%
\begin{pgfscope}%
\pgfsys@transformshift{1.720301in}{3.636060in}%
\pgfsys@useobject{currentmarker}{}%
\end{pgfscope}%
\begin{pgfscope}%
\pgfsys@transformshift{1.729991in}{3.626106in}%
\pgfsys@useobject{currentmarker}{}%
\end{pgfscope}%
\begin{pgfscope}%
\pgfsys@transformshift{1.736297in}{3.621791in}%
\pgfsys@useobject{currentmarker}{}%
\end{pgfscope}%
\begin{pgfscope}%
\pgfsys@transformshift{1.742885in}{3.614335in}%
\pgfsys@useobject{currentmarker}{}%
\end{pgfscope}%
\begin{pgfscope}%
\pgfsys@transformshift{1.750445in}{3.606684in}%
\pgfsys@useobject{currentmarker}{}%
\end{pgfscope}%
\begin{pgfscope}%
\pgfsys@transformshift{1.760286in}{3.599626in}%
\pgfsys@useobject{currentmarker}{}%
\end{pgfscope}%
\begin{pgfscope}%
\pgfsys@transformshift{1.769180in}{3.589130in}%
\pgfsys@useobject{currentmarker}{}%
\end{pgfscope}%
\begin{pgfscope}%
\pgfsys@transformshift{1.781251in}{3.580312in}%
\pgfsys@useobject{currentmarker}{}%
\end{pgfscope}%
\begin{pgfscope}%
\pgfsys@transformshift{1.795097in}{3.571824in}%
\pgfsys@useobject{currentmarker}{}%
\end{pgfscope}%
\begin{pgfscope}%
\pgfsys@transformshift{1.807628in}{3.555882in}%
\pgfsys@useobject{currentmarker}{}%
\end{pgfscope}%
\begin{pgfscope}%
\pgfsys@transformshift{1.816337in}{3.548916in}%
\pgfsys@useobject{currentmarker}{}%
\end{pgfscope}%
\begin{pgfscope}%
\pgfsys@transformshift{1.821126in}{3.545083in}%
\pgfsys@useobject{currentmarker}{}%
\end{pgfscope}%
\begin{pgfscope}%
\pgfsys@transformshift{1.827620in}{3.538228in}%
\pgfsys@useobject{currentmarker}{}%
\end{pgfscope}%
\begin{pgfscope}%
\pgfsys@transformshift{1.835229in}{3.530838in}%
\pgfsys@useobject{currentmarker}{}%
\end{pgfscope}%
\begin{pgfscope}%
\pgfsys@transformshift{1.839727in}{3.527123in}%
\pgfsys@useobject{currentmarker}{}%
\end{pgfscope}%
\begin{pgfscope}%
\pgfsys@transformshift{1.845696in}{3.519970in}%
\pgfsys@useobject{currentmarker}{}%
\end{pgfscope}%
\begin{pgfscope}%
\pgfsys@transformshift{1.852749in}{3.511327in}%
\pgfsys@useobject{currentmarker}{}%
\end{pgfscope}%
\begin{pgfscope}%
\pgfsys@transformshift{1.863044in}{3.504297in}%
\pgfsys@useobject{currentmarker}{}%
\end{pgfscope}%
\begin{pgfscope}%
\pgfsys@transformshift{1.873199in}{3.492231in}%
\pgfsys@useobject{currentmarker}{}%
\end{pgfscope}%
\begin{pgfscope}%
\pgfsys@transformshift{1.885760in}{3.481035in}%
\pgfsys@useobject{currentmarker}{}%
\end{pgfscope}%
\begin{pgfscope}%
\pgfsys@transformshift{1.899850in}{3.469494in}%
\pgfsys@useobject{currentmarker}{}%
\end{pgfscope}%
\begin{pgfscope}%
\pgfsys@transformshift{1.911201in}{3.454388in}%
\pgfsys@useobject{currentmarker}{}%
\end{pgfscope}%
\begin{pgfscope}%
\pgfsys@transformshift{1.927022in}{3.441595in}%
\pgfsys@useobject{currentmarker}{}%
\end{pgfscope}%
\begin{pgfscope}%
\pgfsys@transformshift{1.943915in}{3.427478in}%
\pgfsys@useobject{currentmarker}{}%
\end{pgfscope}%
\begin{pgfscope}%
\pgfsys@transformshift{1.960471in}{3.408250in}%
\pgfsys@useobject{currentmarker}{}%
\end{pgfscope}%
\begin{pgfscope}%
\pgfsys@transformshift{1.980870in}{3.391653in}%
\pgfsys@useobject{currentmarker}{}%
\end{pgfscope}%
\begin{pgfscope}%
\pgfsys@transformshift{2.003710in}{3.376978in}%
\pgfsys@useobject{currentmarker}{}%
\end{pgfscope}%
\begin{pgfscope}%
\pgfsys@transformshift{2.020193in}{3.354656in}%
\pgfsys@useobject{currentmarker}{}%
\end{pgfscope}%
\begin{pgfscope}%
\pgfsys@transformshift{2.043315in}{3.336937in}%
\pgfsys@useobject{currentmarker}{}%
\end{pgfscope}%
\begin{pgfscope}%
\pgfsys@transformshift{2.068031in}{3.318159in}%
\pgfsys@useobject{currentmarker}{}%
\end{pgfscope}%
\begin{pgfscope}%
\pgfsys@transformshift{2.088830in}{3.291528in}%
\pgfsys@useobject{currentmarker}{}%
\end{pgfscope}%
\begin{pgfscope}%
\pgfsys@transformshift{2.117809in}{3.272829in}%
\pgfsys@useobject{currentmarker}{}%
\end{pgfscope}%
\begin{pgfscope}%
\pgfsys@transformshift{2.145273in}{3.250900in}%
\pgfsys@useobject{currentmarker}{}%
\end{pgfscope}%
\begin{pgfscope}%
\pgfsys@transformshift{2.170174in}{3.223222in}%
\pgfsys@useobject{currentmarker}{}%
\end{pgfscope}%
\begin{pgfscope}%
\pgfsys@transformshift{2.185705in}{3.209877in}%
\pgfsys@useobject{currentmarker}{}%
\end{pgfscope}%
\begin{pgfscope}%
\pgfsys@transformshift{2.204393in}{3.197076in}%
\pgfsys@useobject{currentmarker}{}%
\end{pgfscope}%
\begin{pgfscope}%
\pgfsys@transformshift{2.221932in}{3.177460in}%
\pgfsys@useobject{currentmarker}{}%
\end{pgfscope}%
\begin{pgfscope}%
\pgfsys@transformshift{2.243587in}{3.159928in}%
\pgfsys@useobject{currentmarker}{}%
\end{pgfscope}%
\begin{pgfscope}%
\pgfsys@transformshift{2.266310in}{3.142147in}%
\pgfsys@useobject{currentmarker}{}%
\end{pgfscope}%
\begin{pgfscope}%
\pgfsys@transformshift{2.284606in}{3.117770in}%
\pgfsys@useobject{currentmarker}{}%
\end{pgfscope}%
\begin{pgfscope}%
\pgfsys@transformshift{2.297855in}{3.107500in}%
\pgfsys@useobject{currentmarker}{}%
\end{pgfscope}%
\begin{pgfscope}%
\pgfsys@transformshift{2.311450in}{3.094813in}%
\pgfsys@useobject{currentmarker}{}%
\end{pgfscope}%
\begin{pgfscope}%
\pgfsys@transformshift{2.323156in}{3.076827in}%
\pgfsys@useobject{currentmarker}{}%
\end{pgfscope}%
\begin{pgfscope}%
\pgfsys@transformshift{2.339334in}{3.061527in}%
\pgfsys@useobject{currentmarker}{}%
\end{pgfscope}%
\begin{pgfscope}%
\pgfsys@transformshift{2.356890in}{3.045979in}%
\pgfsys@useobject{currentmarker}{}%
\end{pgfscope}%
\begin{pgfscope}%
\pgfsys@transformshift{2.364370in}{3.035472in}%
\pgfsys@useobject{currentmarker}{}%
\end{pgfscope}%
\begin{pgfscope}%
\pgfsys@transformshift{2.376577in}{3.024671in}%
\pgfsys@useobject{currentmarker}{}%
\end{pgfscope}%
\begin{pgfscope}%
\pgfsys@transformshift{2.382972in}{3.018389in}%
\pgfsys@useobject{currentmarker}{}%
\end{pgfscope}%
\begin{pgfscope}%
\pgfsys@transformshift{2.385621in}{3.014230in}%
\pgfsys@useobject{currentmarker}{}%
\end{pgfscope}%
\begin{pgfscope}%
\pgfsys@transformshift{2.391316in}{3.009311in}%
\pgfsys@useobject{currentmarker}{}%
\end{pgfscope}%
\begin{pgfscope}%
\pgfsys@transformshift{2.396880in}{3.003087in}%
\pgfsys@useobject{currentmarker}{}%
\end{pgfscope}%
\begin{pgfscope}%
\pgfsys@transformshift{2.403449in}{2.996093in}%
\pgfsys@useobject{currentmarker}{}%
\end{pgfscope}%
\begin{pgfscope}%
\pgfsys@transformshift{2.411724in}{2.989591in}%
\pgfsys@useobject{currentmarker}{}%
\end{pgfscope}%
\begin{pgfscope}%
\pgfsys@transformshift{2.419568in}{2.980457in}%
\pgfsys@useobject{currentmarker}{}%
\end{pgfscope}%
\begin{pgfscope}%
\pgfsys@transformshift{2.428258in}{2.970231in}%
\pgfsys@useobject{currentmarker}{}%
\end{pgfscope}%
\begin{pgfscope}%
\pgfsys@transformshift{2.440477in}{2.959813in}%
\pgfsys@useobject{currentmarker}{}%
\end{pgfscope}%
\begin{pgfscope}%
\pgfsys@transformshift{2.452688in}{2.947224in}%
\pgfsys@useobject{currentmarker}{}%
\end{pgfscope}%
\begin{pgfscope}%
\pgfsys@transformshift{2.464322in}{2.932191in}%
\pgfsys@useobject{currentmarker}{}%
\end{pgfscope}%
\begin{pgfscope}%
\pgfsys@transformshift{2.480885in}{2.916791in}%
\pgfsys@useobject{currentmarker}{}%
\end{pgfscope}%
\begin{pgfscope}%
\pgfsys@transformshift{2.498518in}{2.901280in}%
\pgfsys@useobject{currentmarker}{}%
\end{pgfscope}%
\begin{pgfscope}%
\pgfsys@transformshift{2.506594in}{2.891200in}%
\pgfsys@useobject{currentmarker}{}%
\end{pgfscope}%
\begin{pgfscope}%
\pgfsys@transformshift{2.517770in}{2.880815in}%
\pgfsys@useobject{currentmarker}{}%
\end{pgfscope}%
\begin{pgfscope}%
\pgfsys@transformshift{2.524107in}{2.875316in}%
\pgfsys@useobject{currentmarker}{}%
\end{pgfscope}%
\begin{pgfscope}%
\pgfsys@transformshift{2.529291in}{2.867650in}%
\pgfsys@useobject{currentmarker}{}%
\end{pgfscope}%
\begin{pgfscope}%
\pgfsys@transformshift{2.537556in}{2.860219in}%
\pgfsys@useobject{currentmarker}{}%
\end{pgfscope}%
\begin{pgfscope}%
\pgfsys@transformshift{2.547363in}{2.852659in}%
\pgfsys@useobject{currentmarker}{}%
\end{pgfscope}%
\begin{pgfscope}%
\pgfsys@transformshift{2.556649in}{2.841617in}%
\pgfsys@useobject{currentmarker}{}%
\end{pgfscope}%
\begin{pgfscope}%
\pgfsys@transformshift{2.568747in}{2.831362in}%
\pgfsys@useobject{currentmarker}{}%
\end{pgfscope}%
\begin{pgfscope}%
\pgfsys@transformshift{2.582588in}{2.821399in}%
\pgfsys@useobject{currentmarker}{}%
\end{pgfscope}%
\begin{pgfscope}%
\pgfsys@transformshift{2.588340in}{2.813990in}%
\pgfsys@useobject{currentmarker}{}%
\end{pgfscope}%
\begin{pgfscope}%
\pgfsys@transformshift{2.597465in}{2.806402in}%
\pgfsys@useobject{currentmarker}{}%
\end{pgfscope}%
\begin{pgfscope}%
\pgfsys@transformshift{2.608163in}{2.799360in}%
\pgfsys@useobject{currentmarker}{}%
\end{pgfscope}%
\begin{pgfscope}%
\pgfsys@transformshift{2.618525in}{2.787591in}%
\pgfsys@useobject{currentmarker}{}%
\end{pgfscope}%
\begin{pgfscope}%
\pgfsys@transformshift{2.632153in}{2.776577in}%
\pgfsys@useobject{currentmarker}{}%
\end{pgfscope}%
\begin{pgfscope}%
\pgfsys@transformshift{2.648291in}{2.765457in}%
\pgfsys@useobject{currentmarker}{}%
\end{pgfscope}%
\begin{pgfscope}%
\pgfsys@transformshift{2.663053in}{2.746958in}%
\pgfsys@useobject{currentmarker}{}%
\end{pgfscope}%
\begin{pgfscope}%
\pgfsys@transformshift{2.681534in}{2.730068in}%
\pgfsys@useobject{currentmarker}{}%
\end{pgfscope}%
\begin{pgfscope}%
\pgfsys@transformshift{2.702364in}{2.713447in}%
\pgfsys@useobject{currentmarker}{}%
\end{pgfscope}%
\begin{pgfscope}%
\pgfsys@transformshift{2.724602in}{2.697004in}%
\pgfsys@useobject{currentmarker}{}%
\end{pgfscope}%
\begin{pgfscope}%
\pgfsys@transformshift{2.743684in}{2.674584in}%
\pgfsys@useobject{currentmarker}{}%
\end{pgfscope}%
\begin{pgfscope}%
\pgfsys@transformshift{2.770089in}{2.657613in}%
\pgfsys@useobject{currentmarker}{}%
\end{pgfscope}%
\begin{pgfscope}%
\pgfsys@transformshift{2.797137in}{2.638392in}%
\pgfsys@useobject{currentmarker}{}%
\end{pgfscope}%
\begin{pgfscope}%
\pgfsys@transformshift{2.821104in}{2.613900in}%
\pgfsys@useobject{currentmarker}{}%
\end{pgfscope}%
\begin{pgfscope}%
\pgfsys@transformshift{2.845004in}{2.586967in}%
\pgfsys@useobject{currentmarker}{}%
\end{pgfscope}%
\begin{pgfscope}%
\pgfsys@transformshift{2.861136in}{2.575480in}%
\pgfsys@useobject{currentmarker}{}%
\end{pgfscope}%
\begin{pgfscope}%
\pgfsys@transformshift{2.877097in}{2.559742in}%
\pgfsys@useobject{currentmarker}{}%
\end{pgfscope}%
\begin{pgfscope}%
\pgfsys@transformshift{2.894101in}{2.542677in}%
\pgfsys@useobject{currentmarker}{}%
\end{pgfscope}%
\begin{pgfscope}%
\pgfsys@transformshift{2.915339in}{2.526136in}%
\pgfsys@useobject{currentmarker}{}%
\end{pgfscope}%
\begin{pgfscope}%
\pgfsys@transformshift{2.937331in}{2.508495in}%
\pgfsys@useobject{currentmarker}{}%
\end{pgfscope}%
\begin{pgfscope}%
\pgfsys@transformshift{2.947766in}{2.497026in}%
\pgfsys@useobject{currentmarker}{}%
\end{pgfscope}%
\begin{pgfscope}%
\pgfsys@transformshift{2.961332in}{2.485293in}%
\pgfsys@useobject{currentmarker}{}%
\end{pgfscope}%
\begin{pgfscope}%
\pgfsys@transformshift{2.975927in}{2.472312in}%
\pgfsys@useobject{currentmarker}{}%
\end{pgfscope}%
\begin{pgfscope}%
\pgfsys@transformshift{2.983045in}{2.464265in}%
\pgfsys@useobject{currentmarker}{}%
\end{pgfscope}%
\begin{pgfscope}%
\pgfsys@transformshift{2.994139in}{2.455779in}%
\pgfsys@useobject{currentmarker}{}%
\end{pgfscope}%
\begin{pgfscope}%
\pgfsys@transformshift{3.006357in}{2.446193in}%
\pgfsys@useobject{currentmarker}{}%
\end{pgfscope}%
\begin{pgfscope}%
\pgfsys@transformshift{3.019022in}{2.432090in}%
\pgfsys@useobject{currentmarker}{}%
\end{pgfscope}%
\begin{pgfscope}%
\pgfsys@transformshift{3.027093in}{2.425492in}%
\pgfsys@useobject{currentmarker}{}%
\end{pgfscope}%
\begin{pgfscope}%
\pgfsys@transformshift{3.037418in}{2.418661in}%
\pgfsys@useobject{currentmarker}{}%
\end{pgfscope}%
\begin{pgfscope}%
\pgfsys@transformshift{3.049739in}{2.405935in}%
\pgfsys@useobject{currentmarker}{}%
\end{pgfscope}%
\begin{pgfscope}%
\pgfsys@transformshift{3.057564in}{2.400132in}%
\pgfsys@useobject{currentmarker}{}%
\end{pgfscope}%
\begin{pgfscope}%
\pgfsys@transformshift{3.062011in}{2.397143in}%
\pgfsys@useobject{currentmarker}{}%
\end{pgfscope}%
\begin{pgfscope}%
\pgfsys@transformshift{3.067773in}{2.390453in}%
\pgfsys@useobject{currentmarker}{}%
\end{pgfscope}%
\begin{pgfscope}%
\pgfsys@transformshift{3.075379in}{2.384714in}%
\pgfsys@useobject{currentmarker}{}%
\end{pgfscope}%
\begin{pgfscope}%
\pgfsys@transformshift{3.079173in}{2.381099in}%
\pgfsys@useobject{currentmarker}{}%
\end{pgfscope}%
\begin{pgfscope}%
\pgfsys@transformshift{3.084304in}{2.375374in}%
\pgfsys@useobject{currentmarker}{}%
\end{pgfscope}%
\begin{pgfscope}%
\pgfsys@transformshift{3.087685in}{2.372836in}%
\pgfsys@useobject{currentmarker}{}%
\end{pgfscope}%
\begin{pgfscope}%
\pgfsys@transformshift{3.092558in}{2.369364in}%
\pgfsys@useobject{currentmarker}{}%
\end{pgfscope}%
\begin{pgfscope}%
\pgfsys@transformshift{3.099069in}{2.361450in}%
\pgfsys@useobject{currentmarker}{}%
\end{pgfscope}%
\begin{pgfscope}%
\pgfsys@transformshift{3.108292in}{2.354111in}%
\pgfsys@useobject{currentmarker}{}%
\end{pgfscope}%
\begin{pgfscope}%
\pgfsys@transformshift{3.118050in}{2.346477in}%
\pgfsys@useobject{currentmarker}{}%
\end{pgfscope}%
\begin{pgfscope}%
\pgfsys@transformshift{3.130384in}{2.336066in}%
\pgfsys@useobject{currentmarker}{}%
\end{pgfscope}%
\begin{pgfscope}%
\pgfsys@transformshift{3.136824in}{2.329955in}%
\pgfsys@useobject{currentmarker}{}%
\end{pgfscope}%
\begin{pgfscope}%
\pgfsys@transformshift{3.145304in}{2.324271in}%
\pgfsys@useobject{currentmarker}{}%
\end{pgfscope}%
\begin{pgfscope}%
\pgfsys@transformshift{3.154639in}{2.316439in}%
\pgfsys@useobject{currentmarker}{}%
\end{pgfscope}%
\begin{pgfscope}%
\pgfsys@transformshift{3.163959in}{2.307262in}%
\pgfsys@useobject{currentmarker}{}%
\end{pgfscope}%
\begin{pgfscope}%
\pgfsys@transformshift{3.176084in}{2.297670in}%
\pgfsys@useobject{currentmarker}{}%
\end{pgfscope}%
\begin{pgfscope}%
\pgfsys@transformshift{3.182506in}{2.292097in}%
\pgfsys@useobject{currentmarker}{}%
\end{pgfscope}%
\begin{pgfscope}%
\pgfsys@transformshift{3.189650in}{2.284363in}%
\pgfsys@useobject{currentmarker}{}%
\end{pgfscope}%
\begin{pgfscope}%
\pgfsys@transformshift{3.201540in}{2.275711in}%
\pgfsys@useobject{currentmarker}{}%
\end{pgfscope}%
\begin{pgfscope}%
\pgfsys@transformshift{3.214368in}{2.265936in}%
\pgfsys@useobject{currentmarker}{}%
\end{pgfscope}%
\begin{pgfscope}%
\pgfsys@transformshift{3.225353in}{2.253412in}%
\pgfsys@useobject{currentmarker}{}%
\end{pgfscope}%
\begin{pgfscope}%
\pgfsys@transformshift{3.240436in}{2.242446in}%
\pgfsys@useobject{currentmarker}{}%
\end{pgfscope}%
\begin{pgfscope}%
\pgfsys@transformshift{3.255992in}{2.230279in}%
\pgfsys@useobject{currentmarker}{}%
\end{pgfscope}%
\begin{pgfscope}%
\pgfsys@transformshift{3.269960in}{2.215087in}%
\pgfsys@useobject{currentmarker}{}%
\end{pgfscope}%
\begin{pgfscope}%
\pgfsys@transformshift{3.289210in}{2.199722in}%
\pgfsys@useobject{currentmarker}{}%
\end{pgfscope}%
\begin{pgfscope}%
\pgfsys@transformshift{3.311107in}{2.186182in}%
\pgfsys@useobject{currentmarker}{}%
\end{pgfscope}%
\begin{pgfscope}%
\pgfsys@transformshift{3.329773in}{2.164918in}%
\pgfsys@useobject{currentmarker}{}%
\end{pgfscope}%
\begin{pgfscope}%
\pgfsys@transformshift{3.353696in}{2.148633in}%
\pgfsys@useobject{currentmarker}{}%
\end{pgfscope}%
\begin{pgfscope}%
\pgfsys@transformshift{3.379616in}{2.131578in}%
\pgfsys@useobject{currentmarker}{}%
\end{pgfscope}%
\begin{pgfscope}%
\pgfsys@transformshift{3.403269in}{2.105893in}%
\pgfsys@useobject{currentmarker}{}%
\end{pgfscope}%
\begin{pgfscope}%
\pgfsys@transformshift{3.432542in}{2.083359in}%
\pgfsys@useobject{currentmarker}{}%
\end{pgfscope}%
\begin{pgfscope}%
\pgfsys@transformshift{3.464401in}{2.062243in}%
\pgfsys@useobject{currentmarker}{}%
\end{pgfscope}%
\begin{pgfscope}%
\pgfsys@transformshift{3.493819in}{2.031616in}%
\pgfsys@useobject{currentmarker}{}%
\end{pgfscope}%
\begin{pgfscope}%
\pgfsys@transformshift{3.511082in}{2.015882in}%
\pgfsys@useobject{currentmarker}{}%
\end{pgfscope}%
\begin{pgfscope}%
\pgfsys@transformshift{3.531221in}{1.999819in}%
\pgfsys@useobject{currentmarker}{}%
\end{pgfscope}%
\begin{pgfscope}%
\pgfsys@transformshift{3.551372in}{1.981473in}%
\pgfsys@useobject{currentmarker}{}%
\end{pgfscope}%
\begin{pgfscope}%
\pgfsys@transformshift{3.571770in}{1.961868in}%
\pgfsys@useobject{currentmarker}{}%
\end{pgfscope}%
\begin{pgfscope}%
\pgfsys@transformshift{3.583987in}{1.952231in}%
\pgfsys@useobject{currentmarker}{}%
\end{pgfscope}%
\begin{pgfscope}%
\pgfsys@transformshift{3.596899in}{1.941402in}%
\pgfsys@useobject{currentmarker}{}%
\end{pgfscope}%
\begin{pgfscope}%
\pgfsys@transformshift{3.609512in}{1.929320in}%
\pgfsys@useobject{currentmarker}{}%
\end{pgfscope}%
\begin{pgfscope}%
\pgfsys@transformshift{3.625877in}{1.917611in}%
\pgfsys@useobject{currentmarker}{}%
\end{pgfscope}%
\begin{pgfscope}%
\pgfsys@transformshift{3.643239in}{1.905233in}%
\pgfsys@useobject{currentmarker}{}%
\end{pgfscope}%
\begin{pgfscope}%
\pgfsys@transformshift{3.659202in}{1.889518in}%
\pgfsys@useobject{currentmarker}{}%
\end{pgfscope}%
\begin{pgfscope}%
\pgfsys@transformshift{3.678843in}{1.875793in}%
\pgfsys@useobject{currentmarker}{}%
\end{pgfscope}%
\begin{pgfscope}%
\pgfsys@transformshift{3.700354in}{1.860929in}%
\pgfsys@useobject{currentmarker}{}%
\end{pgfscope}%
\begin{pgfscope}%
\pgfsys@transformshift{3.721427in}{1.840452in}%
\pgfsys@useobject{currentmarker}{}%
\end{pgfscope}%
\begin{pgfscope}%
\pgfsys@transformshift{3.746173in}{1.823343in}%
\pgfsys@useobject{currentmarker}{}%
\end{pgfscope}%
\begin{pgfscope}%
\pgfsys@transformshift{3.771437in}{1.805420in}%
\pgfsys@useobject{currentmarker}{}%
\end{pgfscope}%
\begin{pgfscope}%
\pgfsys@transformshift{3.797545in}{1.783652in}%
\pgfsys@useobject{currentmarker}{}%
\end{pgfscope}%
\begin{pgfscope}%
\pgfsys@transformshift{3.824851in}{1.761264in}%
\pgfsys@useobject{currentmarker}{}%
\end{pgfscope}%
\begin{pgfscope}%
\pgfsys@transformshift{3.855001in}{1.739309in}%
\pgfsys@useobject{currentmarker}{}%
\end{pgfscope}%
\begin{pgfscope}%
\pgfsys@transformshift{3.884099in}{1.714776in}%
\pgfsys@useobject{currentmarker}{}%
\end{pgfscope}%
\begin{pgfscope}%
\pgfsys@transformshift{3.898627in}{1.699706in}%
\pgfsys@useobject{currentmarker}{}%
\end{pgfscope}%
\begin{pgfscope}%
\pgfsys@transformshift{3.917200in}{1.687117in}%
\pgfsys@useobject{currentmarker}{}%
\end{pgfscope}%
\begin{pgfscope}%
\pgfsys@transformshift{3.936346in}{1.672599in}%
\pgfsys@useobject{currentmarker}{}%
\end{pgfscope}%
\begin{pgfscope}%
\pgfsys@transformshift{3.955200in}{1.655505in}%
\pgfsys@useobject{currentmarker}{}%
\end{pgfscope}%
\begin{pgfscope}%
\pgfsys@transformshift{3.977985in}{1.638012in}%
\pgfsys@useobject{currentmarker}{}%
\end{pgfscope}%
\begin{pgfscope}%
\pgfsys@transformshift{4.003747in}{1.622001in}%
\pgfsys@useobject{currentmarker}{}%
\end{pgfscope}%
\begin{pgfscope}%
\pgfsys@transformshift{4.027582in}{1.599948in}%
\pgfsys@useobject{currentmarker}{}%
\end{pgfscope}%
\begin{pgfscope}%
\pgfsys@transformshift{4.053667in}{1.579663in}%
\pgfsys@useobject{currentmarker}{}%
\end{pgfscope}%
\begin{pgfscope}%
\pgfsys@transformshift{4.083059in}{1.561732in}%
\pgfsys@useobject{currentmarker}{}%
\end{pgfscope}%
\begin{pgfscope}%
\pgfsys@transformshift{4.111180in}{1.536782in}%
\pgfsys@useobject{currentmarker}{}%
\end{pgfscope}%
\begin{pgfscope}%
\pgfsys@transformshift{4.140834in}{1.512737in}%
\pgfsys@useobject{currentmarker}{}%
\end{pgfscope}%
\begin{pgfscope}%
\pgfsys@transformshift{4.174754in}{1.492184in}%
\pgfsys@useobject{currentmarker}{}%
\end{pgfscope}%
\begin{pgfscope}%
\pgfsys@transformshift{4.207673in}{1.464090in}%
\pgfsys@useobject{currentmarker}{}%
\end{pgfscope}%
\begin{pgfscope}%
\pgfsys@transformshift{4.241664in}{1.435169in}%
\pgfsys@useobject{currentmarker}{}%
\end{pgfscope}%
\begin{pgfscope}%
\pgfsys@transformshift{4.279045in}{1.407788in}%
\pgfsys@useobject{currentmarker}{}%
\end{pgfscope}%
\begin{pgfscope}%
\pgfsys@transformshift{4.316819in}{1.378870in}%
\pgfsys@useobject{currentmarker}{}%
\end{pgfscope}%
\begin{pgfscope}%
\pgfsys@transformshift{4.350665in}{1.344311in}%
\pgfsys@useobject{currentmarker}{}%
\end{pgfscope}%
\begin{pgfscope}%
\pgfsys@transformshift{4.392133in}{1.316902in}%
\pgfsys@useobject{currentmarker}{}%
\end{pgfscope}%
\begin{pgfscope}%
\pgfsys@transformshift{4.434387in}{1.288086in}%
\pgfsys@useobject{currentmarker}{}%
\end{pgfscope}%
\begin{pgfscope}%
\pgfsys@transformshift{4.471207in}{1.250587in}%
\pgfsys@useobject{currentmarker}{}%
\end{pgfscope}%
\begin{pgfscope}%
\pgfsys@transformshift{4.516229in}{1.218863in}%
\pgfsys@useobject{currentmarker}{}%
\end{pgfscope}%
\begin{pgfscope}%
\pgfsys@transformshift{4.541173in}{1.201674in}%
\pgfsys@useobject{currentmarker}{}%
\end{pgfscope}%
\begin{pgfscope}%
\pgfsys@transformshift{4.564916in}{1.180361in}%
\pgfsys@useobject{currentmarker}{}%
\end{pgfscope}%
\begin{pgfscope}%
\pgfsys@transformshift{4.591760in}{1.161524in}%
\pgfsys@useobject{currentmarker}{}%
\end{pgfscope}%
\begin{pgfscope}%
\pgfsys@transformshift{4.605899in}{1.150326in}%
\pgfsys@useobject{currentmarker}{}%
\end{pgfscope}%
\begin{pgfscope}%
\pgfsys@transformshift{4.621075in}{1.138870in}%
\pgfsys@useobject{currentmarker}{}%
\end{pgfscope}%
\begin{pgfscope}%
\pgfsys@transformshift{4.635437in}{1.124793in}%
\pgfsys@useobject{currentmarker}{}%
\end{pgfscope}%
\begin{pgfscope}%
\pgfsys@transformshift{4.654040in}{1.111727in}%
\pgfsys@useobject{currentmarker}{}%
\end{pgfscope}%
\begin{pgfscope}%
\pgfsys@transformshift{4.674164in}{1.097803in}%
\pgfsys@useobject{currentmarker}{}%
\end{pgfscope}%
\begin{pgfscope}%
\pgfsys@transformshift{4.694295in}{1.081637in}%
\pgfsys@useobject{currentmarker}{}%
\end{pgfscope}%
\begin{pgfscope}%
\pgfsys@transformshift{4.716651in}{1.064863in}%
\pgfsys@useobject{currentmarker}{}%
\end{pgfscope}%
\begin{pgfscope}%
\pgfsys@transformshift{4.739433in}{1.045982in}%
\pgfsys@useobject{currentmarker}{}%
\end{pgfscope}%
\begin{pgfscope}%
\pgfsys@transformshift{4.764650in}{1.027784in}%
\pgfsys@useobject{currentmarker}{}%
\end{pgfscope}%
\begin{pgfscope}%
\pgfsys@transformshift{4.788802in}{1.005589in}%
\pgfsys@useobject{currentmarker}{}%
\end{pgfscope}%
\begin{pgfscope}%
\pgfsys@transformshift{4.815227in}{0.983607in}%
\pgfsys@useobject{currentmarker}{}%
\end{pgfscope}%
\begin{pgfscope}%
\pgfsys@transformshift{4.842361in}{0.960560in}%
\pgfsys@useobject{currentmarker}{}%
\end{pgfscope}%
\begin{pgfscope}%
\pgfsys@transformshift{4.869783in}{0.935753in}%
\pgfsys@useobject{currentmarker}{}%
\end{pgfscope}%
\begin{pgfscope}%
\pgfsys@transformshift{4.898756in}{0.910094in}%
\pgfsys@useobject{currentmarker}{}%
\end{pgfscope}%
\begin{pgfscope}%
\pgfsys@transformshift{4.930421in}{0.884569in}%
\pgfsys@useobject{currentmarker}{}%
\end{pgfscope}%
\begin{pgfscope}%
\pgfsys@transformshift{4.962807in}{0.857318in}%
\pgfsys@useobject{currentmarker}{}%
\end{pgfscope}%
\begin{pgfscope}%
\pgfsys@transformshift{4.998058in}{0.831078in}%
\pgfsys@useobject{currentmarker}{}%
\end{pgfscope}%
\begin{pgfscope}%
\pgfsys@transformshift{5.031373in}{0.800023in}%
\pgfsys@useobject{currentmarker}{}%
\end{pgfscope}%
\begin{pgfscope}%
\pgfsys@transformshift{5.065794in}{0.768154in}%
\pgfsys@useobject{currentmarker}{}%
\end{pgfscope}%
\begin{pgfscope}%
\pgfsys@transformshift{5.099219in}{0.732977in}%
\pgfsys@useobject{currentmarker}{}%
\end{pgfscope}%
\begin{pgfscope}%
\pgfsys@transformshift{5.117316in}{0.713362in}%
\pgfsys@useobject{currentmarker}{}%
\end{pgfscope}%
\begin{pgfscope}%
\pgfsys@transformshift{5.125492in}{0.701171in}%
\pgfsys@useobject{currentmarker}{}%
\end{pgfscope}%
\begin{pgfscope}%
\pgfsys@transformshift{5.127737in}{0.693416in}%
\pgfsys@useobject{currentmarker}{}%
\end{pgfscope}%
\begin{pgfscope}%
\pgfsys@transformshift{5.128106in}{0.688991in}%
\pgfsys@useobject{currentmarker}{}%
\end{pgfscope}%
\begin{pgfscope}%
\pgfsys@transformshift{5.128129in}{0.686549in}%
\pgfsys@useobject{currentmarker}{}%
\end{pgfscope}%
\begin{pgfscope}%
\pgfsys@transformshift{5.127751in}{0.685260in}%
\pgfsys@useobject{currentmarker}{}%
\end{pgfscope}%
\begin{pgfscope}%
\pgfsys@transformshift{5.127559in}{0.684547in}%
\pgfsys@useobject{currentmarker}{}%
\end{pgfscope}%
\begin{pgfscope}%
\pgfsys@transformshift{5.127360in}{0.684193in}%
\pgfsys@useobject{currentmarker}{}%
\end{pgfscope}%
\begin{pgfscope}%
\pgfsys@transformshift{5.126529in}{0.683414in}%
\pgfsys@useobject{currentmarker}{}%
\end{pgfscope}%
\begin{pgfscope}%
\pgfsys@transformshift{5.125965in}{0.683141in}%
\pgfsys@useobject{currentmarker}{}%
\end{pgfscope}%
\begin{pgfscope}%
\pgfsys@transformshift{5.125636in}{0.683041in}%
\pgfsys@useobject{currentmarker}{}%
\end{pgfscope}%
\begin{pgfscope}%
\pgfsys@transformshift{5.124375in}{0.683042in}%
\pgfsys@useobject{currentmarker}{}%
\end{pgfscope}%
\begin{pgfscope}%
\pgfsys@transformshift{5.122066in}{0.683000in}%
\pgfsys@useobject{currentmarker}{}%
\end{pgfscope}%
\begin{pgfscope}%
\pgfsys@transformshift{5.118793in}{0.683401in}%
\pgfsys@useobject{currentmarker}{}%
\end{pgfscope}%
\begin{pgfscope}%
\pgfsys@transformshift{5.116984in}{0.683528in}%
\pgfsys@useobject{currentmarker}{}%
\end{pgfscope}%
\begin{pgfscope}%
\pgfsys@transformshift{5.115992in}{0.683635in}%
\pgfsys@useobject{currentmarker}{}%
\end{pgfscope}%
\begin{pgfscope}%
\pgfsys@transformshift{5.115453in}{0.683735in}%
\pgfsys@useobject{currentmarker}{}%
\end{pgfscope}%
\begin{pgfscope}%
\pgfsys@transformshift{5.115154in}{0.683778in}%
\pgfsys@useobject{currentmarker}{}%
\end{pgfscope}%
\begin{pgfscope}%
\pgfsys@transformshift{5.114992in}{0.683817in}%
\pgfsys@useobject{currentmarker}{}%
\end{pgfscope}%
\begin{pgfscope}%
\pgfsys@transformshift{5.114902in}{0.683830in}%
\pgfsys@useobject{currentmarker}{}%
\end{pgfscope}%
\begin{pgfscope}%
\pgfsys@transformshift{5.114854in}{0.683844in}%
\pgfsys@useobject{currentmarker}{}%
\end{pgfscope}%
\begin{pgfscope}%
\pgfsys@transformshift{5.114827in}{0.683851in}%
\pgfsys@useobject{currentmarker}{}%
\end{pgfscope}%
\begin{pgfscope}%
\pgfsys@transformshift{5.114812in}{0.683853in}%
\pgfsys@useobject{currentmarker}{}%
\end{pgfscope}%
\begin{pgfscope}%
\pgfsys@transformshift{5.114272in}{0.684041in}%
\pgfsys@useobject{currentmarker}{}%
\end{pgfscope}%
\begin{pgfscope}%
\pgfsys@transformshift{5.113160in}{0.684324in}%
\pgfsys@useobject{currentmarker}{}%
\end{pgfscope}%
\begin{pgfscope}%
\pgfsys@transformshift{5.111269in}{0.684866in}%
\pgfsys@useobject{currentmarker}{}%
\end{pgfscope}%
\begin{pgfscope}%
\pgfsys@transformshift{5.108714in}{0.685561in}%
\pgfsys@useobject{currentmarker}{}%
\end{pgfscope}%
\begin{pgfscope}%
\pgfsys@transformshift{5.107336in}{0.686035in}%
\pgfsys@useobject{currentmarker}{}%
\end{pgfscope}%
\begin{pgfscope}%
\pgfsys@transformshift{5.105412in}{0.686540in}%
\pgfsys@useobject{currentmarker}{}%
\end{pgfscope}%
\begin{pgfscope}%
\pgfsys@transformshift{5.102950in}{0.687167in}%
\pgfsys@useobject{currentmarker}{}%
\end{pgfscope}%
\begin{pgfscope}%
\pgfsys@transformshift{5.100043in}{0.688304in}%
\pgfsys@useobject{currentmarker}{}%
\end{pgfscope}%
\begin{pgfscope}%
\pgfsys@transformshift{5.096438in}{0.689088in}%
\pgfsys@useobject{currentmarker}{}%
\end{pgfscope}%
\begin{pgfscope}%
\pgfsys@transformshift{5.094489in}{0.689652in}%
\pgfsys@useobject{currentmarker}{}%
\end{pgfscope}%
\begin{pgfscope}%
\pgfsys@transformshift{5.091971in}{0.690228in}%
\pgfsys@useobject{currentmarker}{}%
\end{pgfscope}%
\begin{pgfscope}%
\pgfsys@transformshift{5.090625in}{0.690682in}%
\pgfsys@useobject{currentmarker}{}%
\end{pgfscope}%
\begin{pgfscope}%
\pgfsys@transformshift{5.088706in}{0.691086in}%
\pgfsys@useobject{currentmarker}{}%
\end{pgfscope}%
\begin{pgfscope}%
\pgfsys@transformshift{5.086226in}{0.691723in}%
\pgfsys@useobject{currentmarker}{}%
\end{pgfscope}%
\begin{pgfscope}%
\pgfsys@transformshift{5.083203in}{0.692482in}%
\pgfsys@useobject{currentmarker}{}%
\end{pgfscope}%
\begin{pgfscope}%
\pgfsys@transformshift{5.079493in}{0.693297in}%
\pgfsys@useobject{currentmarker}{}%
\end{pgfscope}%
\begin{pgfscope}%
\pgfsys@transformshift{5.075078in}{0.693870in}%
\pgfsys@useobject{currentmarker}{}%
\end{pgfscope}%
\begin{pgfscope}%
\pgfsys@transformshift{5.069962in}{0.694594in}%
\pgfsys@useobject{currentmarker}{}%
\end{pgfscope}%
\begin{pgfscope}%
\pgfsys@transformshift{5.064217in}{0.695633in}%
\pgfsys@useobject{currentmarker}{}%
\end{pgfscope}%
\begin{pgfscope}%
\pgfsys@transformshift{5.057660in}{0.696320in}%
\pgfsys@useobject{currentmarker}{}%
\end{pgfscope}%
\begin{pgfscope}%
\pgfsys@transformshift{5.050484in}{0.697506in}%
\pgfsys@useobject{currentmarker}{}%
\end{pgfscope}%
\begin{pgfscope}%
\pgfsys@transformshift{5.042642in}{0.698470in}%
\pgfsys@useobject{currentmarker}{}%
\end{pgfscope}%
\begin{pgfscope}%
\pgfsys@transformshift{5.034216in}{0.699904in}%
\pgfsys@useobject{currentmarker}{}%
\end{pgfscope}%
\begin{pgfscope}%
\pgfsys@transformshift{5.025092in}{0.701131in}%
\pgfsys@useobject{currentmarker}{}%
\end{pgfscope}%
\begin{pgfscope}%
\pgfsys@transformshift{5.015177in}{0.702070in}%
\pgfsys@useobject{currentmarker}{}%
\end{pgfscope}%
\begin{pgfscope}%
\pgfsys@transformshift{5.004519in}{0.703027in}%
\pgfsys@useobject{currentmarker}{}%
\end{pgfscope}%
\begin{pgfscope}%
\pgfsys@transformshift{4.993114in}{0.703660in}%
\pgfsys@useobject{currentmarker}{}%
\end{pgfscope}%
\begin{pgfscope}%
\pgfsys@transformshift{4.980886in}{0.703636in}%
\pgfsys@useobject{currentmarker}{}%
\end{pgfscope}%
\begin{pgfscope}%
\pgfsys@transformshift{4.968030in}{0.705140in}%
\pgfsys@useobject{currentmarker}{}%
\end{pgfscope}%
\begin{pgfscope}%
\pgfsys@transformshift{4.954388in}{0.706214in}%
\pgfsys@useobject{currentmarker}{}%
\end{pgfscope}%
\begin{pgfscope}%
\pgfsys@transformshift{4.940062in}{0.707377in}%
\pgfsys@useobject{currentmarker}{}%
\end{pgfscope}%
\begin{pgfscope}%
\pgfsys@transformshift{4.925159in}{0.709046in}%
\pgfsys@useobject{currentmarker}{}%
\end{pgfscope}%
\begin{pgfscope}%
\pgfsys@transformshift{4.909652in}{0.710729in}%
\pgfsys@useobject{currentmarker}{}%
\end{pgfscope}%
\begin{pgfscope}%
\pgfsys@transformshift{4.893752in}{0.714607in}%
\pgfsys@useobject{currentmarker}{}%
\end{pgfscope}%
\begin{pgfscope}%
\pgfsys@transformshift{4.876568in}{0.716285in}%
\pgfsys@useobject{currentmarker}{}%
\end{pgfscope}%
\begin{pgfscope}%
\pgfsys@transformshift{4.867354in}{0.718586in}%
\pgfsys@useobject{currentmarker}{}%
\end{pgfscope}%
\begin{pgfscope}%
\pgfsys@transformshift{4.856972in}{0.720200in}%
\pgfsys@useobject{currentmarker}{}%
\end{pgfscope}%
\begin{pgfscope}%
\pgfsys@transformshift{4.846268in}{0.723268in}%
\pgfsys@useobject{currentmarker}{}%
\end{pgfscope}%
\begin{pgfscope}%
\pgfsys@transformshift{4.834443in}{0.725257in}%
\pgfsys@useobject{currentmarker}{}%
\end{pgfscope}%
\begin{pgfscope}%
\pgfsys@transformshift{4.822245in}{0.729174in}%
\pgfsys@useobject{currentmarker}{}%
\end{pgfscope}%
\begin{pgfscope}%
\pgfsys@transformshift{4.808676in}{0.730697in}%
\pgfsys@useobject{currentmarker}{}%
\end{pgfscope}%
\begin{pgfscope}%
\pgfsys@transformshift{4.794956in}{0.735431in}%
\pgfsys@useobject{currentmarker}{}%
\end{pgfscope}%
\begin{pgfscope}%
\pgfsys@transformshift{4.779692in}{0.737446in}%
\pgfsys@useobject{currentmarker}{}%
\end{pgfscope}%
\begin{pgfscope}%
\pgfsys@transformshift{4.764188in}{0.742322in}%
\pgfsys@useobject{currentmarker}{}%
\end{pgfscope}%
\begin{pgfscope}%
\pgfsys@transformshift{4.747039in}{0.743420in}%
\pgfsys@useobject{currentmarker}{}%
\end{pgfscope}%
\begin{pgfscope}%
\pgfsys@transformshift{4.729611in}{0.747994in}%
\pgfsys@useobject{currentmarker}{}%
\end{pgfscope}%
\begin{pgfscope}%
\pgfsys@transformshift{4.710816in}{0.749407in}%
\pgfsys@useobject{currentmarker}{}%
\end{pgfscope}%
\begin{pgfscope}%
\pgfsys@transformshift{4.692018in}{0.755391in}%
\pgfsys@useobject{currentmarker}{}%
\end{pgfscope}%
\begin{pgfscope}%
\pgfsys@transformshift{4.671585in}{0.757607in}%
\pgfsys@useobject{currentmarker}{}%
\end{pgfscope}%
\begin{pgfscope}%
\pgfsys@transformshift{4.651245in}{0.764160in}%
\pgfsys@useobject{currentmarker}{}%
\end{pgfscope}%
\begin{pgfscope}%
\pgfsys@transformshift{4.629215in}{0.766943in}%
\pgfsys@useobject{currentmarker}{}%
\end{pgfscope}%
\begin{pgfscope}%
\pgfsys@transformshift{4.607269in}{0.774309in}%
\pgfsys@useobject{currentmarker}{}%
\end{pgfscope}%
\begin{pgfscope}%
\pgfsys@transformshift{4.583458in}{0.777869in}%
\pgfsys@useobject{currentmarker}{}%
\end{pgfscope}%
\begin{pgfscope}%
\pgfsys@transformshift{4.559240in}{0.783674in}%
\pgfsys@useobject{currentmarker}{}%
\end{pgfscope}%
\begin{pgfscope}%
\pgfsys@transformshift{4.533630in}{0.786733in}%
\pgfsys@useobject{currentmarker}{}%
\end{pgfscope}%
\begin{pgfscope}%
\pgfsys@transformshift{4.507777in}{0.793559in}%
\pgfsys@useobject{currentmarker}{}%
\end{pgfscope}%
\begin{pgfscope}%
\pgfsys@transformshift{4.480456in}{0.797933in}%
\pgfsys@useobject{currentmarker}{}%
\end{pgfscope}%
\begin{pgfscope}%
\pgfsys@transformshift{4.453114in}{0.806297in}%
\pgfsys@useobject{currentmarker}{}%
\end{pgfscope}%
\begin{pgfscope}%
\pgfsys@transformshift{4.423924in}{0.810739in}%
\pgfsys@useobject{currentmarker}{}%
\end{pgfscope}%
\begin{pgfscope}%
\pgfsys@transformshift{4.393899in}{0.816387in}%
\pgfsys@useobject{currentmarker}{}%
\end{pgfscope}%
\begin{pgfscope}%
\pgfsys@transformshift{4.362553in}{0.819712in}%
\pgfsys@useobject{currentmarker}{}%
\end{pgfscope}%
\begin{pgfscope}%
\pgfsys@transformshift{4.330347in}{0.823232in}%
\pgfsys@useobject{currentmarker}{}%
\end{pgfscope}%
\begin{pgfscope}%
\pgfsys@transformshift{4.297049in}{0.823014in}%
\pgfsys@useobject{currentmarker}{}%
\end{pgfscope}%
\begin{pgfscope}%
\pgfsys@transformshift{4.278750in}{0.822264in}%
\pgfsys@useobject{currentmarker}{}%
\end{pgfscope}%
\begin{pgfscope}%
\pgfsys@transformshift{4.257708in}{0.820440in}%
\pgfsys@useobject{currentmarker}{}%
\end{pgfscope}%
\begin{pgfscope}%
\pgfsys@transformshift{4.235310in}{0.819585in}%
\pgfsys@useobject{currentmarker}{}%
\end{pgfscope}%
\begin{pgfscope}%
\pgfsys@transformshift{4.212613in}{0.815788in}%
\pgfsys@useobject{currentmarker}{}%
\end{pgfscope}%
\begin{pgfscope}%
\pgfsys@transformshift{4.187320in}{0.814612in}%
\pgfsys@useobject{currentmarker}{}%
\end{pgfscope}%
\begin{pgfscope}%
\pgfsys@transformshift{4.159375in}{0.812114in}%
\pgfsys@useobject{currentmarker}{}%
\end{pgfscope}%
\begin{pgfscope}%
\pgfsys@transformshift{4.144067in}{0.810177in}%
\pgfsys@useobject{currentmarker}{}%
\end{pgfscope}%
\begin{pgfscope}%
\pgfsys@transformshift{4.126400in}{0.810242in}%
\pgfsys@useobject{currentmarker}{}%
\end{pgfscope}%
\begin{pgfscope}%
\pgfsys@transformshift{4.106613in}{0.808372in}%
\pgfsys@useobject{currentmarker}{}%
\end{pgfscope}%
\begin{pgfscope}%
\pgfsys@transformshift{4.084628in}{0.808392in}%
\pgfsys@useobject{currentmarker}{}%
\end{pgfscope}%
\begin{pgfscope}%
\pgfsys@transformshift{4.072579in}{0.807376in}%
\pgfsys@useobject{currentmarker}{}%
\end{pgfscope}%
\begin{pgfscope}%
\pgfsys@transformshift{4.057711in}{0.805719in}%
\pgfsys@useobject{currentmarker}{}%
\end{pgfscope}%
\begin{pgfscope}%
\pgfsys@transformshift{4.041129in}{0.805381in}%
\pgfsys@useobject{currentmarker}{}%
\end{pgfscope}%
\begin{pgfscope}%
\pgfsys@transformshift{4.032020in}{0.804902in}%
\pgfsys@useobject{currentmarker}{}%
\end{pgfscope}%
\begin{pgfscope}%
\pgfsys@transformshift{4.020121in}{0.804732in}%
\pgfsys@useobject{currentmarker}{}%
\end{pgfscope}%
\begin{pgfscope}%
\pgfsys@transformshift{4.006802in}{0.803580in}%
\pgfsys@useobject{currentmarker}{}%
\end{pgfscope}%
\begin{pgfscope}%
\pgfsys@transformshift{3.999455in}{0.803851in}%
\pgfsys@useobject{currentmarker}{}%
\end{pgfscope}%
\begin{pgfscope}%
\pgfsys@transformshift{3.988867in}{0.803003in}%
\pgfsys@useobject{currentmarker}{}%
\end{pgfscope}%
\begin{pgfscope}%
\pgfsys@transformshift{3.977269in}{0.803213in}%
\pgfsys@useobject{currentmarker}{}%
\end{pgfscope}%
\begin{pgfscope}%
\pgfsys@transformshift{3.964657in}{0.802658in}%
\pgfsys@useobject{currentmarker}{}%
\end{pgfscope}%
\begin{pgfscope}%
\pgfsys@transformshift{3.948758in}{0.803345in}%
\pgfsys@useobject{currentmarker}{}%
\end{pgfscope}%
\begin{pgfscope}%
\pgfsys@transformshift{3.940028in}{0.802727in}%
\pgfsys@useobject{currentmarker}{}%
\end{pgfscope}%
\begin{pgfscope}%
\pgfsys@transformshift{3.930226in}{0.802814in}%
\pgfsys@useobject{currentmarker}{}%
\end{pgfscope}%
\begin{pgfscope}%
\pgfsys@transformshift{3.918091in}{0.803011in}%
\pgfsys@useobject{currentmarker}{}%
\end{pgfscope}%
\begin{pgfscope}%
\pgfsys@transformshift{3.905337in}{0.802741in}%
\pgfsys@useobject{currentmarker}{}%
\end{pgfscope}%
\begin{pgfscope}%
\pgfsys@transformshift{3.891644in}{0.803701in}%
\pgfsys@useobject{currentmarker}{}%
\end{pgfscope}%
\begin{pgfscope}%
\pgfsys@transformshift{3.874671in}{0.803065in}%
\pgfsys@useobject{currentmarker}{}%
\end{pgfscope}%
\begin{pgfscope}%
\pgfsys@transformshift{3.865330in}{0.802967in}%
\pgfsys@useobject{currentmarker}{}%
\end{pgfscope}%
\begin{pgfscope}%
\pgfsys@transformshift{3.853562in}{0.803332in}%
\pgfsys@useobject{currentmarker}{}%
\end{pgfscope}%
\begin{pgfscope}%
\pgfsys@transformshift{3.838577in}{0.803966in}%
\pgfsys@useobject{currentmarker}{}%
\end{pgfscope}%
\begin{pgfscope}%
\pgfsys@transformshift{3.830339in}{0.804410in}%
\pgfsys@useobject{currentmarker}{}%
\end{pgfscope}%
\begin{pgfscope}%
\pgfsys@transformshift{3.818763in}{0.804412in}%
\pgfsys@useobject{currentmarker}{}%
\end{pgfscope}%
\begin{pgfscope}%
\pgfsys@transformshift{3.804083in}{0.805031in}%
\pgfsys@useobject{currentmarker}{}%
\end{pgfscope}%
\begin{pgfscope}%
\pgfsys@transformshift{3.788133in}{0.804349in}%
\pgfsys@useobject{currentmarker}{}%
\end{pgfscope}%
\begin{pgfscope}%
\pgfsys@transformshift{3.767305in}{0.804556in}%
\pgfsys@useobject{currentmarker}{}%
\end{pgfscope}%
\begin{pgfscope}%
\pgfsys@transformshift{3.755849in}{0.804635in}%
\pgfsys@useobject{currentmarker}{}%
\end{pgfscope}%
\begin{pgfscope}%
\pgfsys@transformshift{3.741686in}{0.804589in}%
\pgfsys@useobject{currentmarker}{}%
\end{pgfscope}%
\begin{pgfscope}%
\pgfsys@transformshift{3.725331in}{0.806124in}%
\pgfsys@useobject{currentmarker}{}%
\end{pgfscope}%
\begin{pgfscope}%
\pgfsys@transformshift{3.707700in}{0.806570in}%
\pgfsys@useobject{currentmarker}{}%
\end{pgfscope}%
\begin{pgfscope}%
\pgfsys@transformshift{3.684976in}{0.807174in}%
\pgfsys@useobject{currentmarker}{}%
\end{pgfscope}%
\begin{pgfscope}%
\pgfsys@transformshift{3.672501in}{0.808008in}%
\pgfsys@useobject{currentmarker}{}%
\end{pgfscope}%
\begin{pgfscope}%
\pgfsys@transformshift{3.657842in}{0.810114in}%
\pgfsys@useobject{currentmarker}{}%
\end{pgfscope}%
\begin{pgfscope}%
\pgfsys@transformshift{3.638816in}{0.810745in}%
\pgfsys@useobject{currentmarker}{}%
\end{pgfscope}%
\begin{pgfscope}%
\pgfsys@transformshift{3.628350in}{0.811010in}%
\pgfsys@useobject{currentmarker}{}%
\end{pgfscope}%
\begin{pgfscope}%
\pgfsys@transformshift{3.615366in}{0.810712in}%
\pgfsys@useobject{currentmarker}{}%
\end{pgfscope}%
\begin{pgfscope}%
\pgfsys@transformshift{3.598013in}{0.814143in}%
\pgfsys@useobject{currentmarker}{}%
\end{pgfscope}%
\begin{pgfscope}%
\pgfsys@transformshift{3.588292in}{0.814526in}%
\pgfsys@useobject{currentmarker}{}%
\end{pgfscope}%
\begin{pgfscope}%
\pgfsys@transformshift{3.574972in}{0.816503in}%
\pgfsys@useobject{currentmarker}{}%
\end{pgfscope}%
\begin{pgfscope}%
\pgfsys@transformshift{3.559388in}{0.815421in}%
\pgfsys@useobject{currentmarker}{}%
\end{pgfscope}%
\begin{pgfscope}%
\pgfsys@transformshift{3.543197in}{0.815330in}%
\pgfsys@useobject{currentmarker}{}%
\end{pgfscope}%
\begin{pgfscope}%
\pgfsys@transformshift{3.522587in}{0.816051in}%
\pgfsys@useobject{currentmarker}{}%
\end{pgfscope}%
\begin{pgfscope}%
\pgfsys@transformshift{3.501391in}{0.817916in}%
\pgfsys@useobject{currentmarker}{}%
\end{pgfscope}%
\begin{pgfscope}%
\pgfsys@transformshift{3.478171in}{0.818922in}%
\pgfsys@useobject{currentmarker}{}%
\end{pgfscope}%
\begin{pgfscope}%
\pgfsys@transformshift{3.451246in}{0.819418in}%
\pgfsys@useobject{currentmarker}{}%
\end{pgfscope}%
\begin{pgfscope}%
\pgfsys@transformshift{3.423595in}{0.820908in}%
\pgfsys@useobject{currentmarker}{}%
\end{pgfscope}%
\begin{pgfscope}%
\pgfsys@transformshift{3.391266in}{0.822069in}%
\pgfsys@useobject{currentmarker}{}%
\end{pgfscope}%
\begin{pgfscope}%
\pgfsys@transformshift{3.358316in}{0.824453in}%
\pgfsys@useobject{currentmarker}{}%
\end{pgfscope}%
\begin{pgfscope}%
\pgfsys@transformshift{3.323923in}{0.827579in}%
\pgfsys@useobject{currentmarker}{}%
\end{pgfscope}%
\begin{pgfscope}%
\pgfsys@transformshift{3.285271in}{0.831366in}%
\pgfsys@useobject{currentmarker}{}%
\end{pgfscope}%
\begin{pgfscope}%
\pgfsys@transformshift{3.245782in}{0.830794in}%
\pgfsys@useobject{currentmarker}{}%
\end{pgfscope}%
\begin{pgfscope}%
\pgfsys@transformshift{3.202713in}{0.834456in}%
\pgfsys@useobject{currentmarker}{}%
\end{pgfscope}%
\begin{pgfscope}%
\pgfsys@transformshift{3.179153in}{0.837638in}%
\pgfsys@useobject{currentmarker}{}%
\end{pgfscope}%
\begin{pgfscope}%
\pgfsys@transformshift{3.153257in}{0.842140in}%
\pgfsys@useobject{currentmarker}{}%
\end{pgfscope}%
\begin{pgfscope}%
\pgfsys@transformshift{3.123822in}{0.843598in}%
\pgfsys@useobject{currentmarker}{}%
\end{pgfscope}%
\begin{pgfscope}%
\pgfsys@transformshift{3.094247in}{0.849155in}%
\pgfsys@useobject{currentmarker}{}%
\end{pgfscope}%
\begin{pgfscope}%
\pgfsys@transformshift{3.060809in}{0.848600in}%
\pgfsys@useobject{currentmarker}{}%
\end{pgfscope}%
\begin{pgfscope}%
\pgfsys@transformshift{3.026268in}{0.849324in}%
\pgfsys@useobject{currentmarker}{}%
\end{pgfscope}%
\begin{pgfscope}%
\pgfsys@transformshift{3.007272in}{0.849767in}%
\pgfsys@useobject{currentmarker}{}%
\end{pgfscope}%
\begin{pgfscope}%
\pgfsys@transformshift{2.984849in}{0.851704in}%
\pgfsys@useobject{currentmarker}{}%
\end{pgfscope}%
\begin{pgfscope}%
\pgfsys@transformshift{2.972475in}{0.852041in}%
\pgfsys@useobject{currentmarker}{}%
\end{pgfscope}%
\begin{pgfscope}%
\pgfsys@transformshift{2.958415in}{0.852593in}%
\pgfsys@useobject{currentmarker}{}%
\end{pgfscope}%
\begin{pgfscope}%
\pgfsys@transformshift{2.939319in}{0.852903in}%
\pgfsys@useobject{currentmarker}{}%
\end{pgfscope}%
\begin{pgfscope}%
\pgfsys@transformshift{2.919019in}{0.854184in}%
\pgfsys@useobject{currentmarker}{}%
\end{pgfscope}%
\begin{pgfscope}%
\pgfsys@transformshift{2.893817in}{0.856196in}%
\pgfsys@useobject{currentmarker}{}%
\end{pgfscope}%
\begin{pgfscope}%
\pgfsys@transformshift{2.879919in}{0.856662in}%
\pgfsys@useobject{currentmarker}{}%
\end{pgfscope}%
\begin{pgfscope}%
\pgfsys@transformshift{2.863164in}{0.858560in}%
\pgfsys@useobject{currentmarker}{}%
\end{pgfscope}%
\begin{pgfscope}%
\pgfsys@transformshift{2.844565in}{0.858640in}%
\pgfsys@useobject{currentmarker}{}%
\end{pgfscope}%
\begin{pgfscope}%
\pgfsys@transformshift{2.824505in}{0.859495in}%
\pgfsys@useobject{currentmarker}{}%
\end{pgfscope}%
\begin{pgfscope}%
\pgfsys@transformshift{2.799646in}{0.863500in}%
\pgfsys@useobject{currentmarker}{}%
\end{pgfscope}%
\begin{pgfscope}%
\pgfsys@transformshift{2.785798in}{0.863648in}%
\pgfsys@useobject{currentmarker}{}%
\end{pgfscope}%
\begin{pgfscope}%
\pgfsys@transformshift{2.767607in}{0.866074in}%
\pgfsys@useobject{currentmarker}{}%
\end{pgfscope}%
\begin{pgfscope}%
\pgfsys@transformshift{2.748061in}{0.867256in}%
\pgfsys@useobject{currentmarker}{}%
\end{pgfscope}%
\begin{pgfscope}%
\pgfsys@transformshift{2.726897in}{0.872154in}%
\pgfsys@useobject{currentmarker}{}%
\end{pgfscope}%
\begin{pgfscope}%
\pgfsys@transformshift{2.703032in}{0.872263in}%
\pgfsys@useobject{currentmarker}{}%
\end{pgfscope}%
\begin{pgfscope}%
\pgfsys@transformshift{2.678149in}{0.873907in}%
\pgfsys@useobject{currentmarker}{}%
\end{pgfscope}%
\begin{pgfscope}%
\pgfsys@transformshift{2.649628in}{0.875890in}%
\pgfsys@useobject{currentmarker}{}%
\end{pgfscope}%
\begin{pgfscope}%
\pgfsys@transformshift{2.633910in}{0.875463in}%
\pgfsys@useobject{currentmarker}{}%
\end{pgfscope}%
\begin{pgfscope}%
\pgfsys@transformshift{2.614940in}{0.875091in}%
\pgfsys@useobject{currentmarker}{}%
\end{pgfscope}%
\begin{pgfscope}%
\pgfsys@transformshift{2.592227in}{0.873505in}%
\pgfsys@useobject{currentmarker}{}%
\end{pgfscope}%
\begin{pgfscope}%
\pgfsys@transformshift{2.579745in}{0.872491in}%
\pgfsys@useobject{currentmarker}{}%
\end{pgfscope}%
\begin{pgfscope}%
\pgfsys@transformshift{2.563363in}{0.870558in}%
\pgfsys@useobject{currentmarker}{}%
\end{pgfscope}%
\begin{pgfscope}%
\pgfsys@transformshift{2.546160in}{0.870891in}%
\pgfsys@useobject{currentmarker}{}%
\end{pgfscope}%
\begin{pgfscope}%
\pgfsys@transformshift{2.527658in}{0.868863in}%
\pgfsys@useobject{currentmarker}{}%
\end{pgfscope}%
\begin{pgfscope}%
\pgfsys@transformshift{2.505366in}{0.869561in}%
\pgfsys@useobject{currentmarker}{}%
\end{pgfscope}%
\begin{pgfscope}%
\pgfsys@transformshift{2.481556in}{0.868996in}%
\pgfsys@useobject{currentmarker}{}%
\end{pgfscope}%
\begin{pgfscope}%
\pgfsys@transformshift{2.456566in}{0.869325in}%
\pgfsys@useobject{currentmarker}{}%
\end{pgfscope}%
\begin{pgfscope}%
\pgfsys@transformshift{2.429846in}{0.868041in}%
\pgfsys@useobject{currentmarker}{}%
\end{pgfscope}%
\begin{pgfscope}%
\pgfsys@transformshift{2.400854in}{0.869426in}%
\pgfsys@useobject{currentmarker}{}%
\end{pgfscope}%
\begin{pgfscope}%
\pgfsys@transformshift{2.367097in}{0.870051in}%
\pgfsys@useobject{currentmarker}{}%
\end{pgfscope}%
\begin{pgfscope}%
\pgfsys@transformshift{2.348529in}{0.869769in}%
\pgfsys@useobject{currentmarker}{}%
\end{pgfscope}%
\begin{pgfscope}%
\pgfsys@transformshift{2.325743in}{0.870048in}%
\pgfsys@useobject{currentmarker}{}%
\end{pgfscope}%
\begin{pgfscope}%
\pgfsys@transformshift{2.300811in}{0.868350in}%
\pgfsys@useobject{currentmarker}{}%
\end{pgfscope}%
\begin{pgfscope}%
\pgfsys@transformshift{2.287355in}{0.865552in}%
\pgfsys@useobject{currentmarker}{}%
\end{pgfscope}%
\begin{pgfscope}%
\pgfsys@transformshift{2.271641in}{0.865609in}%
\pgfsys@useobject{currentmarker}{}%
\end{pgfscope}%
\begin{pgfscope}%
\pgfsys@transformshift{2.254982in}{0.865874in}%
\pgfsys@useobject{currentmarker}{}%
\end{pgfscope}%
\begin{pgfscope}%
\pgfsys@transformshift{2.233236in}{0.866622in}%
\pgfsys@useobject{currentmarker}{}%
\end{pgfscope}%
\begin{pgfscope}%
\pgfsys@transformshift{2.208680in}{0.867027in}%
\pgfsys@useobject{currentmarker}{}%
\end{pgfscope}%
\begin{pgfscope}%
\pgfsys@transformshift{2.182709in}{0.869791in}%
\pgfsys@useobject{currentmarker}{}%
\end{pgfscope}%
\begin{pgfscope}%
\pgfsys@transformshift{2.150507in}{0.871892in}%
\pgfsys@useobject{currentmarker}{}%
\end{pgfscope}%
\begin{pgfscope}%
\pgfsys@transformshift{2.132922in}{0.869491in}%
\pgfsys@useobject{currentmarker}{}%
\end{pgfscope}%
\begin{pgfscope}%
\pgfsys@transformshift{2.112485in}{0.869555in}%
\pgfsys@useobject{currentmarker}{}%
\end{pgfscope}%
\begin{pgfscope}%
\pgfsys@transformshift{2.089578in}{0.869136in}%
\pgfsys@useobject{currentmarker}{}%
\end{pgfscope}%
\begin{pgfscope}%
\pgfsys@transformshift{2.065298in}{0.866252in}%
\pgfsys@useobject{currentmarker}{}%
\end{pgfscope}%
\begin{pgfscope}%
\pgfsys@transformshift{2.036633in}{0.863463in}%
\pgfsys@useobject{currentmarker}{}%
\end{pgfscope}%
\begin{pgfscope}%
\pgfsys@transformshift{2.007198in}{0.862661in}%
\pgfsys@useobject{currentmarker}{}%
\end{pgfscope}%
\begin{pgfscope}%
\pgfsys@transformshift{1.975668in}{0.858039in}%
\pgfsys@useobject{currentmarker}{}%
\end{pgfscope}%
\begin{pgfscope}%
\pgfsys@transformshift{1.941656in}{0.858180in}%
\pgfsys@useobject{currentmarker}{}%
\end{pgfscope}%
\begin{pgfscope}%
\pgfsys@transformshift{1.923005in}{0.856743in}%
\pgfsys@useobject{currentmarker}{}%
\end{pgfscope}%
\begin{pgfscope}%
\pgfsys@transformshift{1.902458in}{0.856914in}%
\pgfsys@useobject{currentmarker}{}%
\end{pgfscope}%
\begin{pgfscope}%
\pgfsys@transformshift{1.891330in}{0.854941in}%
\pgfsys@useobject{currentmarker}{}%
\end{pgfscope}%
\begin{pgfscope}%
\pgfsys@transformshift{1.876643in}{0.853763in}%
\pgfsys@useobject{currentmarker}{}%
\end{pgfscope}%
\begin{pgfscope}%
\pgfsys@transformshift{1.860025in}{0.853732in}%
\pgfsys@useobject{currentmarker}{}%
\end{pgfscope}%
\begin{pgfscope}%
\pgfsys@transformshift{1.842599in}{0.852118in}%
\pgfsys@useobject{currentmarker}{}%
\end{pgfscope}%
\begin{pgfscope}%
\pgfsys@transformshift{1.821967in}{0.850190in}%
\pgfsys@useobject{currentmarker}{}%
\end{pgfscope}%
\begin{pgfscope}%
\pgfsys@transformshift{1.810571in}{0.850187in}%
\pgfsys@useobject{currentmarker}{}%
\end{pgfscope}%
\begin{pgfscope}%
\pgfsys@transformshift{1.797361in}{0.849373in}%
\pgfsys@useobject{currentmarker}{}%
\end{pgfscope}%
\begin{pgfscope}%
\pgfsys@transformshift{1.790154in}{0.848352in}%
\pgfsys@useobject{currentmarker}{}%
\end{pgfscope}%
\begin{pgfscope}%
\pgfsys@transformshift{1.780772in}{0.847283in}%
\pgfsys@useobject{currentmarker}{}%
\end{pgfscope}%
\begin{pgfscope}%
\pgfsys@transformshift{1.768957in}{0.846790in}%
\pgfsys@useobject{currentmarker}{}%
\end{pgfscope}%
\begin{pgfscope}%
\pgfsys@transformshift{1.755913in}{0.844779in}%
\pgfsys@useobject{currentmarker}{}%
\end{pgfscope}%
\begin{pgfscope}%
\pgfsys@transformshift{1.739554in}{0.844640in}%
\pgfsys@useobject{currentmarker}{}%
\end{pgfscope}%
\begin{pgfscope}%
\pgfsys@transformshift{1.730576in}{0.845244in}%
\pgfsys@useobject{currentmarker}{}%
\end{pgfscope}%
\begin{pgfscope}%
\pgfsys@transformshift{1.719230in}{0.844484in}%
\pgfsys@useobject{currentmarker}{}%
\end{pgfscope}%
\begin{pgfscope}%
\pgfsys@transformshift{1.706118in}{0.844326in}%
\pgfsys@useobject{currentmarker}{}%
\end{pgfscope}%
\begin{pgfscope}%
\pgfsys@transformshift{1.698906in}{0.844365in}%
\pgfsys@useobject{currentmarker}{}%
\end{pgfscope}%
\begin{pgfscope}%
\pgfsys@transformshift{1.687265in}{0.843113in}%
\pgfsys@useobject{currentmarker}{}%
\end{pgfscope}%
\begin{pgfscope}%
\pgfsys@transformshift{1.680875in}{0.842323in}%
\pgfsys@useobject{currentmarker}{}%
\end{pgfscope}%
\begin{pgfscope}%
\pgfsys@transformshift{1.670821in}{0.841158in}%
\pgfsys@useobject{currentmarker}{}%
\end{pgfscope}%
\begin{pgfscope}%
\pgfsys@transformshift{1.659719in}{0.840858in}%
\pgfsys@useobject{currentmarker}{}%
\end{pgfscope}%
\begin{pgfscope}%
\pgfsys@transformshift{1.645623in}{0.838586in}%
\pgfsys@useobject{currentmarker}{}%
\end{pgfscope}%
\begin{pgfscope}%
\pgfsys@transformshift{1.629646in}{0.837148in}%
\pgfsys@useobject{currentmarker}{}%
\end{pgfscope}%
\begin{pgfscope}%
\pgfsys@transformshift{1.612731in}{0.836904in}%
\pgfsys@useobject{currentmarker}{}%
\end{pgfscope}%
\begin{pgfscope}%
\pgfsys@transformshift{1.591971in}{0.835576in}%
\pgfsys@useobject{currentmarker}{}%
\end{pgfscope}%
\begin{pgfscope}%
\pgfsys@transformshift{1.580590in}{0.834394in}%
\pgfsys@useobject{currentmarker}{}%
\end{pgfscope}%
\begin{pgfscope}%
\pgfsys@transformshift{1.565799in}{0.832785in}%
\pgfsys@useobject{currentmarker}{}%
\end{pgfscope}%
\begin{pgfscope}%
\pgfsys@transformshift{1.549202in}{0.832841in}%
\pgfsys@useobject{currentmarker}{}%
\end{pgfscope}%
\begin{pgfscope}%
\pgfsys@transformshift{1.532115in}{0.831612in}%
\pgfsys@useobject{currentmarker}{}%
\end{pgfscope}%
\begin{pgfscope}%
\pgfsys@transformshift{1.511625in}{0.829805in}%
\pgfsys@useobject{currentmarker}{}%
\end{pgfscope}%
\begin{pgfscope}%
\pgfsys@transformshift{1.490190in}{0.829044in}%
\pgfsys@useobject{currentmarker}{}%
\end{pgfscope}%
\begin{pgfscope}%
\pgfsys@transformshift{1.467638in}{0.827379in}%
\pgfsys@useobject{currentmarker}{}%
\end{pgfscope}%
\begin{pgfscope}%
\pgfsys@transformshift{1.444599in}{0.824886in}%
\pgfsys@useobject{currentmarker}{}%
\end{pgfscope}%
\begin{pgfscope}%
\pgfsys@transformshift{1.418999in}{0.822839in}%
\pgfsys@useobject{currentmarker}{}%
\end{pgfscope}%
\begin{pgfscope}%
\pgfsys@transformshift{1.390713in}{0.820952in}%
\pgfsys@useobject{currentmarker}{}%
\end{pgfscope}%
\begin{pgfscope}%
\pgfsys@transformshift{1.361604in}{0.821631in}%
\pgfsys@useobject{currentmarker}{}%
\end{pgfscope}%
\begin{pgfscope}%
\pgfsys@transformshift{1.329296in}{0.821588in}%
\pgfsys@useobject{currentmarker}{}%
\end{pgfscope}%
\begin{pgfscope}%
\pgfsys@transformshift{1.295871in}{0.825086in}%
\pgfsys@useobject{currentmarker}{}%
\end{pgfscope}%
\begin{pgfscope}%
\pgfsys@transformshift{1.261427in}{0.823486in}%
\pgfsys@useobject{currentmarker}{}%
\end{pgfscope}%
\begin{pgfscope}%
\pgfsys@transformshift{1.225335in}{0.826669in}%
\pgfsys@useobject{currentmarker}{}%
\end{pgfscope}%
\begin{pgfscope}%
\pgfsys@transformshift{1.184387in}{0.822178in}%
\pgfsys@useobject{currentmarker}{}%
\end{pgfscope}%
\begin{pgfscope}%
\pgfsys@transformshift{1.142529in}{0.820733in}%
\pgfsys@useobject{currentmarker}{}%
\end{pgfscope}%
\begin{pgfscope}%
\pgfsys@transformshift{1.096791in}{0.819255in}%
\pgfsys@useobject{currentmarker}{}%
\end{pgfscope}%
\begin{pgfscope}%
\pgfsys@transformshift{1.050350in}{0.825616in}%
\pgfsys@useobject{currentmarker}{}%
\end{pgfscope}%
\begin{pgfscope}%
\pgfsys@transformshift{1.001550in}{0.822045in}%
\pgfsys@useobject{currentmarker}{}%
\end{pgfscope}%
\begin{pgfscope}%
\pgfsys@transformshift{0.949062in}{0.825465in}%
\pgfsys@useobject{currentmarker}{}%
\end{pgfscope}%
\begin{pgfscope}%
\pgfsys@transformshift{0.895563in}{0.825611in}%
\pgfsys@useobject{currentmarker}{}%
\end{pgfscope}%
\begin{pgfscope}%
\pgfsys@transformshift{0.919827in}{0.825666in}%
\pgfsys@useobject{currentmarker}{}%
\end{pgfscope}%
\begin{pgfscope}%
\pgfsys@transformshift{0.950146in}{0.831301in}%
\pgfsys@useobject{currentmarker}{}%
\end{pgfscope}%
\begin{pgfscope}%
\pgfsys@transformshift{0.986769in}{0.836518in}%
\pgfsys@useobject{currentmarker}{}%
\end{pgfscope}%
\begin{pgfscope}%
\pgfsys@transformshift{1.027064in}{0.846899in}%
\pgfsys@useobject{currentmarker}{}%
\end{pgfscope}%
\begin{pgfscope}%
\pgfsys@transformshift{1.066739in}{0.862276in}%
\pgfsys@useobject{currentmarker}{}%
\end{pgfscope}%
\begin{pgfscope}%
\pgfsys@transformshift{1.098858in}{0.891031in}%
\pgfsys@useobject{currentmarker}{}%
\end{pgfscope}%
\begin{pgfscope}%
\pgfsys@transformshift{1.122334in}{0.894355in}%
\pgfsys@useobject{currentmarker}{}%
\end{pgfscope}%
\begin{pgfscope}%
\pgfsys@transformshift{1.131668in}{0.903463in}%
\pgfsys@useobject{currentmarker}{}%
\end{pgfscope}%
\begin{pgfscope}%
\pgfsys@transformshift{1.138616in}{0.905241in}%
\pgfsys@useobject{currentmarker}{}%
\end{pgfscope}%
\begin{pgfscope}%
\pgfsys@transformshift{1.146520in}{0.907265in}%
\pgfsys@useobject{currentmarker}{}%
\end{pgfscope}%
\end{pgfscope}%
\begin{pgfscope}%
\pgfsetbuttcap%
\pgfsetroundjoin%
\definecolor{currentfill}{rgb}{0.000000,0.000000,0.000000}%
\pgfsetfillcolor{currentfill}%
\pgfsetlinewidth{0.803000pt}%
\definecolor{currentstroke}{rgb}{0.000000,0.000000,0.000000}%
\pgfsetstrokecolor{currentstroke}%
\pgfsetdash{}{0pt}%
\pgfsys@defobject{currentmarker}{\pgfqpoint{0.000000in}{-0.048611in}}{\pgfqpoint{0.000000in}{0.000000in}}{%
\pgfpathmoveto{\pgfqpoint{0.000000in}{0.000000in}}%
\pgfpathlineto{\pgfqpoint{0.000000in}{-0.048611in}}%
\pgfusepath{stroke,fill}%
}%
\begin{pgfscope}%
\pgfsys@transformshift{0.892982in}{0.515000in}%
\pgfsys@useobject{currentmarker}{}%
\end{pgfscope}%
\end{pgfscope}%
\begin{pgfscope}%
\definecolor{textcolor}{rgb}{0.000000,0.000000,0.000000}%
\pgfsetstrokecolor{textcolor}%
\pgfsetfillcolor{textcolor}%
\pgftext[x=0.892982in,y=0.417777in,,top]{\color{textcolor}\rmfamily\fontsize{10.000000}{12.000000}\selectfont \(\displaystyle {0}\)}%
\end{pgfscope}%
\begin{pgfscope}%
\pgfsetbuttcap%
\pgfsetroundjoin%
\definecolor{currentfill}{rgb}{0.000000,0.000000,0.000000}%
\pgfsetfillcolor{currentfill}%
\pgfsetlinewidth{0.803000pt}%
\definecolor{currentstroke}{rgb}{0.000000,0.000000,0.000000}%
\pgfsetstrokecolor{currentstroke}%
\pgfsetdash{}{0pt}%
\pgfsys@defobject{currentmarker}{\pgfqpoint{0.000000in}{-0.048611in}}{\pgfqpoint{0.000000in}{0.000000in}}{%
\pgfpathmoveto{\pgfqpoint{0.000000in}{0.000000in}}%
\pgfpathlineto{\pgfqpoint{0.000000in}{-0.048611in}}%
\pgfusepath{stroke,fill}%
}%
\begin{pgfscope}%
\pgfsys@transformshift{1.951692in}{0.515000in}%
\pgfsys@useobject{currentmarker}{}%
\end{pgfscope}%
\end{pgfscope}%
\begin{pgfscope}%
\definecolor{textcolor}{rgb}{0.000000,0.000000,0.000000}%
\pgfsetstrokecolor{textcolor}%
\pgfsetfillcolor{textcolor}%
\pgftext[x=1.951692in,y=0.417777in,,top]{\color{textcolor}\rmfamily\fontsize{10.000000}{12.000000}\selectfont \(\displaystyle {10}\)}%
\end{pgfscope}%
\begin{pgfscope}%
\pgfsetbuttcap%
\pgfsetroundjoin%
\definecolor{currentfill}{rgb}{0.000000,0.000000,0.000000}%
\pgfsetfillcolor{currentfill}%
\pgfsetlinewidth{0.803000pt}%
\definecolor{currentstroke}{rgb}{0.000000,0.000000,0.000000}%
\pgfsetstrokecolor{currentstroke}%
\pgfsetdash{}{0pt}%
\pgfsys@defobject{currentmarker}{\pgfqpoint{0.000000in}{-0.048611in}}{\pgfqpoint{0.000000in}{0.000000in}}{%
\pgfpathmoveto{\pgfqpoint{0.000000in}{0.000000in}}%
\pgfpathlineto{\pgfqpoint{0.000000in}{-0.048611in}}%
\pgfusepath{stroke,fill}%
}%
\begin{pgfscope}%
\pgfsys@transformshift{3.010401in}{0.515000in}%
\pgfsys@useobject{currentmarker}{}%
\end{pgfscope}%
\end{pgfscope}%
\begin{pgfscope}%
\definecolor{textcolor}{rgb}{0.000000,0.000000,0.000000}%
\pgfsetstrokecolor{textcolor}%
\pgfsetfillcolor{textcolor}%
\pgftext[x=3.010401in,y=0.417777in,,top]{\color{textcolor}\rmfamily\fontsize{10.000000}{12.000000}\selectfont \(\displaystyle {20}\)}%
\end{pgfscope}%
\begin{pgfscope}%
\pgfsetbuttcap%
\pgfsetroundjoin%
\definecolor{currentfill}{rgb}{0.000000,0.000000,0.000000}%
\pgfsetfillcolor{currentfill}%
\pgfsetlinewidth{0.803000pt}%
\definecolor{currentstroke}{rgb}{0.000000,0.000000,0.000000}%
\pgfsetstrokecolor{currentstroke}%
\pgfsetdash{}{0pt}%
\pgfsys@defobject{currentmarker}{\pgfqpoint{0.000000in}{-0.048611in}}{\pgfqpoint{0.000000in}{0.000000in}}{%
\pgfpathmoveto{\pgfqpoint{0.000000in}{0.000000in}}%
\pgfpathlineto{\pgfqpoint{0.000000in}{-0.048611in}}%
\pgfusepath{stroke,fill}%
}%
\begin{pgfscope}%
\pgfsys@transformshift{4.069110in}{0.515000in}%
\pgfsys@useobject{currentmarker}{}%
\end{pgfscope}%
\end{pgfscope}%
\begin{pgfscope}%
\definecolor{textcolor}{rgb}{0.000000,0.000000,0.000000}%
\pgfsetstrokecolor{textcolor}%
\pgfsetfillcolor{textcolor}%
\pgftext[x=4.069110in,y=0.417777in,,top]{\color{textcolor}\rmfamily\fontsize{10.000000}{12.000000}\selectfont \(\displaystyle {30}\)}%
\end{pgfscope}%
\begin{pgfscope}%
\pgfsetbuttcap%
\pgfsetroundjoin%
\definecolor{currentfill}{rgb}{0.000000,0.000000,0.000000}%
\pgfsetfillcolor{currentfill}%
\pgfsetlinewidth{0.803000pt}%
\definecolor{currentstroke}{rgb}{0.000000,0.000000,0.000000}%
\pgfsetstrokecolor{currentstroke}%
\pgfsetdash{}{0pt}%
\pgfsys@defobject{currentmarker}{\pgfqpoint{0.000000in}{-0.048611in}}{\pgfqpoint{0.000000in}{0.000000in}}{%
\pgfpathmoveto{\pgfqpoint{0.000000in}{0.000000in}}%
\pgfpathlineto{\pgfqpoint{0.000000in}{-0.048611in}}%
\pgfusepath{stroke,fill}%
}%
\begin{pgfscope}%
\pgfsys@transformshift{5.127820in}{0.515000in}%
\pgfsys@useobject{currentmarker}{}%
\end{pgfscope}%
\end{pgfscope}%
\begin{pgfscope}%
\definecolor{textcolor}{rgb}{0.000000,0.000000,0.000000}%
\pgfsetstrokecolor{textcolor}%
\pgfsetfillcolor{textcolor}%
\pgftext[x=5.127820in,y=0.417777in,,top]{\color{textcolor}\rmfamily\fontsize{10.000000}{12.000000}\selectfont \(\displaystyle {40}\)}%
\end{pgfscope}%
\begin{pgfscope}%
\definecolor{textcolor}{rgb}{0.000000,0.000000,0.000000}%
\pgfsetstrokecolor{textcolor}%
\pgfsetfillcolor{textcolor}%
\pgftext[x=3.010556in,y=0.238889in,,top]{\color{textcolor}\rmfamily\fontsize{10.000000}{12.000000}\selectfont Position X [\(\displaystyle m\)]}%
\end{pgfscope}%
\begin{pgfscope}%
\pgfsetbuttcap%
\pgfsetroundjoin%
\definecolor{currentfill}{rgb}{0.000000,0.000000,0.000000}%
\pgfsetfillcolor{currentfill}%
\pgfsetlinewidth{0.803000pt}%
\definecolor{currentstroke}{rgb}{0.000000,0.000000,0.000000}%
\pgfsetstrokecolor{currentstroke}%
\pgfsetdash{}{0pt}%
\pgfsys@defobject{currentmarker}{\pgfqpoint{-0.048611in}{0.000000in}}{\pgfqpoint{-0.000000in}{0.000000in}}{%
\pgfpathmoveto{\pgfqpoint{-0.000000in}{0.000000in}}%
\pgfpathlineto{\pgfqpoint{-0.048611in}{0.000000in}}%
\pgfusepath{stroke,fill}%
}%
\begin{pgfscope}%
\pgfsys@transformshift{0.530556in}{0.998015in}%
\pgfsys@useobject{currentmarker}{}%
\end{pgfscope}%
\end{pgfscope}%
\begin{pgfscope}%
\definecolor{textcolor}{rgb}{0.000000,0.000000,0.000000}%
\pgfsetstrokecolor{textcolor}%
\pgfsetfillcolor{textcolor}%
\pgftext[x=0.363889in, y=0.949821in, left, base]{\color{textcolor}\rmfamily\fontsize{10.000000}{12.000000}\selectfont \(\displaystyle {0}\)}%
\end{pgfscope}%
\begin{pgfscope}%
\pgfsetbuttcap%
\pgfsetroundjoin%
\definecolor{currentfill}{rgb}{0.000000,0.000000,0.000000}%
\pgfsetfillcolor{currentfill}%
\pgfsetlinewidth{0.803000pt}%
\definecolor{currentstroke}{rgb}{0.000000,0.000000,0.000000}%
\pgfsetstrokecolor{currentstroke}%
\pgfsetdash{}{0pt}%
\pgfsys@defobject{currentmarker}{\pgfqpoint{-0.048611in}{0.000000in}}{\pgfqpoint{-0.000000in}{0.000000in}}{%
\pgfpathmoveto{\pgfqpoint{-0.000000in}{0.000000in}}%
\pgfpathlineto{\pgfqpoint{-0.048611in}{0.000000in}}%
\pgfusepath{stroke,fill}%
}%
\begin{pgfscope}%
\pgfsys@transformshift{0.530556in}{1.527370in}%
\pgfsys@useobject{currentmarker}{}%
\end{pgfscope}%
\end{pgfscope}%
\begin{pgfscope}%
\definecolor{textcolor}{rgb}{0.000000,0.000000,0.000000}%
\pgfsetstrokecolor{textcolor}%
\pgfsetfillcolor{textcolor}%
\pgftext[x=0.363889in, y=1.479175in, left, base]{\color{textcolor}\rmfamily\fontsize{10.000000}{12.000000}\selectfont \(\displaystyle {5}\)}%
\end{pgfscope}%
\begin{pgfscope}%
\pgfsetbuttcap%
\pgfsetroundjoin%
\definecolor{currentfill}{rgb}{0.000000,0.000000,0.000000}%
\pgfsetfillcolor{currentfill}%
\pgfsetlinewidth{0.803000pt}%
\definecolor{currentstroke}{rgb}{0.000000,0.000000,0.000000}%
\pgfsetstrokecolor{currentstroke}%
\pgfsetdash{}{0pt}%
\pgfsys@defobject{currentmarker}{\pgfqpoint{-0.048611in}{0.000000in}}{\pgfqpoint{-0.000000in}{0.000000in}}{%
\pgfpathmoveto{\pgfqpoint{-0.000000in}{0.000000in}}%
\pgfpathlineto{\pgfqpoint{-0.048611in}{0.000000in}}%
\pgfusepath{stroke,fill}%
}%
\begin{pgfscope}%
\pgfsys@transformshift{0.530556in}{2.056724in}%
\pgfsys@useobject{currentmarker}{}%
\end{pgfscope}%
\end{pgfscope}%
\begin{pgfscope}%
\definecolor{textcolor}{rgb}{0.000000,0.000000,0.000000}%
\pgfsetstrokecolor{textcolor}%
\pgfsetfillcolor{textcolor}%
\pgftext[x=0.294444in, y=2.008530in, left, base]{\color{textcolor}\rmfamily\fontsize{10.000000}{12.000000}\selectfont \(\displaystyle {10}\)}%
\end{pgfscope}%
\begin{pgfscope}%
\pgfsetbuttcap%
\pgfsetroundjoin%
\definecolor{currentfill}{rgb}{0.000000,0.000000,0.000000}%
\pgfsetfillcolor{currentfill}%
\pgfsetlinewidth{0.803000pt}%
\definecolor{currentstroke}{rgb}{0.000000,0.000000,0.000000}%
\pgfsetstrokecolor{currentstroke}%
\pgfsetdash{}{0pt}%
\pgfsys@defobject{currentmarker}{\pgfqpoint{-0.048611in}{0.000000in}}{\pgfqpoint{-0.000000in}{0.000000in}}{%
\pgfpathmoveto{\pgfqpoint{-0.000000in}{0.000000in}}%
\pgfpathlineto{\pgfqpoint{-0.048611in}{0.000000in}}%
\pgfusepath{stroke,fill}%
}%
\begin{pgfscope}%
\pgfsys@transformshift{0.530556in}{2.586079in}%
\pgfsys@useobject{currentmarker}{}%
\end{pgfscope}%
\end{pgfscope}%
\begin{pgfscope}%
\definecolor{textcolor}{rgb}{0.000000,0.000000,0.000000}%
\pgfsetstrokecolor{textcolor}%
\pgfsetfillcolor{textcolor}%
\pgftext[x=0.294444in, y=2.537885in, left, base]{\color{textcolor}\rmfamily\fontsize{10.000000}{12.000000}\selectfont \(\displaystyle {15}\)}%
\end{pgfscope}%
\begin{pgfscope}%
\pgfsetbuttcap%
\pgfsetroundjoin%
\definecolor{currentfill}{rgb}{0.000000,0.000000,0.000000}%
\pgfsetfillcolor{currentfill}%
\pgfsetlinewidth{0.803000pt}%
\definecolor{currentstroke}{rgb}{0.000000,0.000000,0.000000}%
\pgfsetstrokecolor{currentstroke}%
\pgfsetdash{}{0pt}%
\pgfsys@defobject{currentmarker}{\pgfqpoint{-0.048611in}{0.000000in}}{\pgfqpoint{-0.000000in}{0.000000in}}{%
\pgfpathmoveto{\pgfqpoint{-0.000000in}{0.000000in}}%
\pgfpathlineto{\pgfqpoint{-0.048611in}{0.000000in}}%
\pgfusepath{stroke,fill}%
}%
\begin{pgfscope}%
\pgfsys@transformshift{0.530556in}{3.115434in}%
\pgfsys@useobject{currentmarker}{}%
\end{pgfscope}%
\end{pgfscope}%
\begin{pgfscope}%
\definecolor{textcolor}{rgb}{0.000000,0.000000,0.000000}%
\pgfsetstrokecolor{textcolor}%
\pgfsetfillcolor{textcolor}%
\pgftext[x=0.294444in, y=3.067239in, left, base]{\color{textcolor}\rmfamily\fontsize{10.000000}{12.000000}\selectfont \(\displaystyle {20}\)}%
\end{pgfscope}%
\begin{pgfscope}%
\pgfsetbuttcap%
\pgfsetroundjoin%
\definecolor{currentfill}{rgb}{0.000000,0.000000,0.000000}%
\pgfsetfillcolor{currentfill}%
\pgfsetlinewidth{0.803000pt}%
\definecolor{currentstroke}{rgb}{0.000000,0.000000,0.000000}%
\pgfsetstrokecolor{currentstroke}%
\pgfsetdash{}{0pt}%
\pgfsys@defobject{currentmarker}{\pgfqpoint{-0.048611in}{0.000000in}}{\pgfqpoint{-0.000000in}{0.000000in}}{%
\pgfpathmoveto{\pgfqpoint{-0.000000in}{0.000000in}}%
\pgfpathlineto{\pgfqpoint{-0.048611in}{0.000000in}}%
\pgfusepath{stroke,fill}%
}%
\begin{pgfscope}%
\pgfsys@transformshift{0.530556in}{3.644788in}%
\pgfsys@useobject{currentmarker}{}%
\end{pgfscope}%
\end{pgfscope}%
\begin{pgfscope}%
\definecolor{textcolor}{rgb}{0.000000,0.000000,0.000000}%
\pgfsetstrokecolor{textcolor}%
\pgfsetfillcolor{textcolor}%
\pgftext[x=0.294444in, y=3.596594in, left, base]{\color{textcolor}\rmfamily\fontsize{10.000000}{12.000000}\selectfont \(\displaystyle {25}\)}%
\end{pgfscope}%
\begin{pgfscope}%
\pgfsetbuttcap%
\pgfsetroundjoin%
\definecolor{currentfill}{rgb}{0.000000,0.000000,0.000000}%
\pgfsetfillcolor{currentfill}%
\pgfsetlinewidth{0.803000pt}%
\definecolor{currentstroke}{rgb}{0.000000,0.000000,0.000000}%
\pgfsetstrokecolor{currentstroke}%
\pgfsetdash{}{0pt}%
\pgfsys@defobject{currentmarker}{\pgfqpoint{-0.048611in}{0.000000in}}{\pgfqpoint{-0.000000in}{0.000000in}}{%
\pgfpathmoveto{\pgfqpoint{-0.000000in}{0.000000in}}%
\pgfpathlineto{\pgfqpoint{-0.048611in}{0.000000in}}%
\pgfusepath{stroke,fill}%
}%
\begin{pgfscope}%
\pgfsys@transformshift{0.530556in}{4.174143in}%
\pgfsys@useobject{currentmarker}{}%
\end{pgfscope}%
\end{pgfscope}%
\begin{pgfscope}%
\definecolor{textcolor}{rgb}{0.000000,0.000000,0.000000}%
\pgfsetstrokecolor{textcolor}%
\pgfsetfillcolor{textcolor}%
\pgftext[x=0.294444in, y=4.125949in, left, base]{\color{textcolor}\rmfamily\fontsize{10.000000}{12.000000}\selectfont \(\displaystyle {30}\)}%
\end{pgfscope}%
\begin{pgfscope}%
\definecolor{textcolor}{rgb}{0.000000,0.000000,0.000000}%
\pgfsetstrokecolor{textcolor}%
\pgfsetfillcolor{textcolor}%
\pgftext[x=0.238889in,y=2.363000in,,bottom,rotate=90.000000]{\color{textcolor}\rmfamily\fontsize{10.000000}{12.000000}\selectfont Position Y [\(\displaystyle m\)]}%
\end{pgfscope}%
\begin{pgfscope}%
\pgfpathrectangle{\pgfqpoint{0.530556in}{0.515000in}}{\pgfqpoint{4.960000in}{3.696000in}}%
\pgfusepath{clip}%
\pgfsetrectcap%
\pgfsetroundjoin%
\pgfsetlinewidth{1.505625pt}%
\definecolor{currentstroke}{rgb}{0.121569,0.466667,0.705882}%
\pgfsetstrokecolor{currentstroke}%
\pgfsetdash{}{0pt}%
\pgfpathmoveto{\pgfqpoint{0.892982in}{0.998015in}}%
\pgfpathlineto{\pgfqpoint{0.892982in}{3.962401in}}%
\pgfpathlineto{\pgfqpoint{3.857368in}{0.998015in}}%
\pgfpathlineto{\pgfqpoint{0.892982in}{0.998015in}}%
\pgfusepath{stroke}%
\end{pgfscope}%
\begin{pgfscope}%
\pgfsetrectcap%
\pgfsetmiterjoin%
\pgfsetlinewidth{0.803000pt}%
\definecolor{currentstroke}{rgb}{0.000000,0.000000,0.000000}%
\pgfsetstrokecolor{currentstroke}%
\pgfsetdash{}{0pt}%
\pgfpathmoveto{\pgfqpoint{0.530556in}{0.515000in}}%
\pgfpathlineto{\pgfqpoint{0.530556in}{4.211000in}}%
\pgfusepath{stroke}%
\end{pgfscope}%
\begin{pgfscope}%
\pgfsetrectcap%
\pgfsetmiterjoin%
\pgfsetlinewidth{0.803000pt}%
\definecolor{currentstroke}{rgb}{0.000000,0.000000,0.000000}%
\pgfsetstrokecolor{currentstroke}%
\pgfsetdash{}{0pt}%
\pgfpathmoveto{\pgfqpoint{5.490556in}{0.515000in}}%
\pgfpathlineto{\pgfqpoint{5.490556in}{4.211000in}}%
\pgfusepath{stroke}%
\end{pgfscope}%
\begin{pgfscope}%
\pgfsetrectcap%
\pgfsetmiterjoin%
\pgfsetlinewidth{0.803000pt}%
\definecolor{currentstroke}{rgb}{0.000000,0.000000,0.000000}%
\pgfsetstrokecolor{currentstroke}%
\pgfsetdash{}{0pt}%
\pgfpathmoveto{\pgfqpoint{0.530556in}{0.515000in}}%
\pgfpathlineto{\pgfqpoint{5.490556in}{0.515000in}}%
\pgfusepath{stroke}%
\end{pgfscope}%
\begin{pgfscope}%
\pgfsetrectcap%
\pgfsetmiterjoin%
\pgfsetlinewidth{0.803000pt}%
\definecolor{currentstroke}{rgb}{0.000000,0.000000,0.000000}%
\pgfsetstrokecolor{currentstroke}%
\pgfsetdash{}{0pt}%
\pgfpathmoveto{\pgfqpoint{0.530556in}{4.211000in}}%
\pgfpathlineto{\pgfqpoint{5.490556in}{4.211000in}}%
\pgfusepath{stroke}%
\end{pgfscope}%
\begin{pgfscope}%
\pgfsetbuttcap%
\pgfsetmiterjoin%
\definecolor{currentfill}{rgb}{1.000000,1.000000,1.000000}%
\pgfsetfillcolor{currentfill}%
\pgfsetfillopacity{0.800000}%
\pgfsetlinewidth{1.003750pt}%
\definecolor{currentstroke}{rgb}{0.800000,0.800000,0.800000}%
\pgfsetstrokecolor{currentstroke}%
\pgfsetstrokeopacity{0.800000}%
\pgfsetdash{}{0pt}%
\pgfpathmoveto{\pgfqpoint{3.799444in}{3.712667in}}%
\pgfpathlineto{\pgfqpoint{5.393333in}{3.712667in}}%
\pgfpathquadraticcurveto{\pgfqpoint{5.421111in}{3.712667in}}{\pgfqpoint{5.421111in}{3.740444in}}%
\pgfpathlineto{\pgfqpoint{5.421111in}{4.113777in}}%
\pgfpathquadraticcurveto{\pgfqpoint{5.421111in}{4.141555in}}{\pgfqpoint{5.393333in}{4.141555in}}%
\pgfpathlineto{\pgfqpoint{3.799444in}{4.141555in}}%
\pgfpathquadraticcurveto{\pgfqpoint{3.771667in}{4.141555in}}{\pgfqpoint{3.771667in}{4.113777in}}%
\pgfpathlineto{\pgfqpoint{3.771667in}{3.740444in}}%
\pgfpathquadraticcurveto{\pgfqpoint{3.771667in}{3.712667in}}{\pgfqpoint{3.799444in}{3.712667in}}%
\pgfpathclose%
\pgfusepath{stroke,fill}%
\end{pgfscope}%
\begin{pgfscope}%
\pgfsetrectcap%
\pgfsetroundjoin%
\pgfsetlinewidth{1.505625pt}%
\definecolor{currentstroke}{rgb}{0.121569,0.466667,0.705882}%
\pgfsetstrokecolor{currentstroke}%
\pgfsetdash{}{0pt}%
\pgfpathmoveto{\pgfqpoint{3.827222in}{4.037388in}}%
\pgfpathlineto{\pgfqpoint{4.105000in}{4.037388in}}%
\pgfusepath{stroke}%
\end{pgfscope}%
\begin{pgfscope}%
\definecolor{textcolor}{rgb}{0.000000,0.000000,0.000000}%
\pgfsetstrokecolor{textcolor}%
\pgfsetfillcolor{textcolor}%
\pgftext[x=4.216111in,y=3.988777in,left,base]{\color{textcolor}\rmfamily\fontsize{10.000000}{12.000000}\selectfont Ground truth}%
\end{pgfscope}%
\begin{pgfscope}%
\pgfsetbuttcap%
\pgfsetroundjoin%
\definecolor{currentfill}{rgb}{0.121569,0.466667,0.705882}%
\pgfsetfillcolor{currentfill}%
\pgfsetlinewidth{1.003750pt}%
\definecolor{currentstroke}{rgb}{0.121569,0.466667,0.705882}%
\pgfsetstrokecolor{currentstroke}%
\pgfsetdash{}{0pt}%
\pgfsys@defobject{currentmarker}{\pgfqpoint{-0.041667in}{-0.041667in}}{\pgfqpoint{0.041667in}{0.041667in}}{%
\pgfpathmoveto{\pgfqpoint{0.000000in}{-0.041667in}}%
\pgfpathcurveto{\pgfqpoint{0.011050in}{-0.041667in}}{\pgfqpoint{0.021649in}{-0.037276in}}{\pgfqpoint{0.029463in}{-0.029463in}}%
\pgfpathcurveto{\pgfqpoint{0.037276in}{-0.021649in}}{\pgfqpoint{0.041667in}{-0.011050in}}{\pgfqpoint{0.041667in}{0.000000in}}%
\pgfpathcurveto{\pgfqpoint{0.041667in}{0.011050in}}{\pgfqpoint{0.037276in}{0.021649in}}{\pgfqpoint{0.029463in}{0.029463in}}%
\pgfpathcurveto{\pgfqpoint{0.021649in}{0.037276in}}{\pgfqpoint{0.011050in}{0.041667in}}{\pgfqpoint{0.000000in}{0.041667in}}%
\pgfpathcurveto{\pgfqpoint{-0.011050in}{0.041667in}}{\pgfqpoint{-0.021649in}{0.037276in}}{\pgfqpoint{-0.029463in}{0.029463in}}%
\pgfpathcurveto{\pgfqpoint{-0.037276in}{0.021649in}}{\pgfqpoint{-0.041667in}{0.011050in}}{\pgfqpoint{-0.041667in}{0.000000in}}%
\pgfpathcurveto{\pgfqpoint{-0.041667in}{-0.011050in}}{\pgfqpoint{-0.037276in}{-0.021649in}}{\pgfqpoint{-0.029463in}{-0.029463in}}%
\pgfpathcurveto{\pgfqpoint{-0.021649in}{-0.037276in}}{\pgfqpoint{-0.011050in}{-0.041667in}}{\pgfqpoint{0.000000in}{-0.041667in}}%
\pgfpathclose%
\pgfusepath{stroke,fill}%
}%
\begin{pgfscope}%
\pgfsys@transformshift{3.966111in}{3.831625in}%
\pgfsys@useobject{currentmarker}{}%
\end{pgfscope}%
\end{pgfscope}%
\begin{pgfscope}%
\definecolor{textcolor}{rgb}{0.000000,0.000000,0.000000}%
\pgfsetstrokecolor{textcolor}%
\pgfsetfillcolor{textcolor}%
\pgftext[x=4.216111in,y=3.795166in,left,base]{\color{textcolor}\rmfamily\fontsize{10.000000}{12.000000}\selectfont Estimated position}%
\end{pgfscope}%
\end{pgfpicture}%
\makeatother%
\endgroup%
}
% %         \caption{INS Hardware}
% %         \label{fig:triangle28_2D}
% %     \end{subfigure}
% %     \begin{subfigure}{0.49\textwidth}
% %         \centering
% %         \resizebox{1\linewidth}{!}{%% Creator: Matplotlib, PGF backend
%%
%% To include the figure in your LaTeX document, write
%%   \input{<filename>.pgf}
%%
%% Make sure the required packages are loaded in your preamble
%%   \usepackage{pgf}
%%
%% and, on pdftex
%%   \usepackage[utf8]{inputenc}\DeclareUnicodeCharacter{2212}{-}
%%
%% or, on luatex and xetex
%%   \usepackage{unicode-math}
%%
%% Figures using additional raster images can only be included by \input if
%% they are in the same directory as the main LaTeX file. For loading figures
%% from other directories you can use the `import` package
%%   \usepackage{import}
%%
%% and then include the figures with
%%   \import{<path to file>}{<filename>.pgf}
%%
%% Matplotlib used the following preamble
%%   \usepackage{fontspec}
%%
\begingroup%
\makeatletter%
\begin{pgfpicture}%
\pgfpathrectangle{\pgfpointorigin}{\pgfqpoint{4.342355in}{4.207622in}}%
\pgfusepath{use as bounding box, clip}%
\begin{pgfscope}%
\pgfsetbuttcap%
\pgfsetmiterjoin%
\definecolor{currentfill}{rgb}{1.000000,1.000000,1.000000}%
\pgfsetfillcolor{currentfill}%
\pgfsetlinewidth{0.000000pt}%
\definecolor{currentstroke}{rgb}{1.000000,1.000000,1.000000}%
\pgfsetstrokecolor{currentstroke}%
\pgfsetdash{}{0pt}%
\pgfpathmoveto{\pgfqpoint{0.000000in}{-0.000000in}}%
\pgfpathlineto{\pgfqpoint{4.342355in}{-0.000000in}}%
\pgfpathlineto{\pgfqpoint{4.342355in}{4.207622in}}%
\pgfpathlineto{\pgfqpoint{0.000000in}{4.207622in}}%
\pgfpathclose%
\pgfusepath{fill}%
\end{pgfscope}%
\begin{pgfscope}%
\pgfsetbuttcap%
\pgfsetmiterjoin%
\definecolor{currentfill}{rgb}{1.000000,1.000000,1.000000}%
\pgfsetfillcolor{currentfill}%
\pgfsetlinewidth{0.000000pt}%
\definecolor{currentstroke}{rgb}{0.000000,0.000000,0.000000}%
\pgfsetstrokecolor{currentstroke}%
\pgfsetstrokeopacity{0.000000}%
\pgfsetdash{}{0pt}%
\pgfpathmoveto{\pgfqpoint{0.100000in}{0.212622in}}%
\pgfpathlineto{\pgfqpoint{3.796000in}{0.212622in}}%
\pgfpathlineto{\pgfqpoint{3.796000in}{3.908622in}}%
\pgfpathlineto{\pgfqpoint{0.100000in}{3.908622in}}%
\pgfpathclose%
\pgfusepath{fill}%
\end{pgfscope}%
\begin{pgfscope}%
\pgfsetbuttcap%
\pgfsetmiterjoin%
\definecolor{currentfill}{rgb}{0.950000,0.950000,0.950000}%
\pgfsetfillcolor{currentfill}%
\pgfsetfillopacity{0.500000}%
\pgfsetlinewidth{1.003750pt}%
\definecolor{currentstroke}{rgb}{0.950000,0.950000,0.950000}%
\pgfsetstrokecolor{currentstroke}%
\pgfsetstrokeopacity{0.500000}%
\pgfsetdash{}{0pt}%
\pgfpathmoveto{\pgfqpoint{0.379073in}{1.123938in}}%
\pgfpathlineto{\pgfqpoint{1.599613in}{2.147018in}}%
\pgfpathlineto{\pgfqpoint{1.582647in}{3.622484in}}%
\pgfpathlineto{\pgfqpoint{0.303698in}{2.689165in}}%
\pgfusepath{stroke,fill}%
\end{pgfscope}%
\begin{pgfscope}%
\pgfsetbuttcap%
\pgfsetmiterjoin%
\definecolor{currentfill}{rgb}{0.900000,0.900000,0.900000}%
\pgfsetfillcolor{currentfill}%
\pgfsetfillopacity{0.500000}%
\pgfsetlinewidth{1.003750pt}%
\definecolor{currentstroke}{rgb}{0.900000,0.900000,0.900000}%
\pgfsetstrokecolor{currentstroke}%
\pgfsetstrokeopacity{0.500000}%
\pgfsetdash{}{0pt}%
\pgfpathmoveto{\pgfqpoint{1.599613in}{2.147018in}}%
\pgfpathlineto{\pgfqpoint{3.558144in}{1.577751in}}%
\pgfpathlineto{\pgfqpoint{3.628038in}{3.104037in}}%
\pgfpathlineto{\pgfqpoint{1.582647in}{3.622484in}}%
\pgfusepath{stroke,fill}%
\end{pgfscope}%
\begin{pgfscope}%
\pgfsetbuttcap%
\pgfsetmiterjoin%
\definecolor{currentfill}{rgb}{0.925000,0.925000,0.925000}%
\pgfsetfillcolor{currentfill}%
\pgfsetfillopacity{0.500000}%
\pgfsetlinewidth{1.003750pt}%
\definecolor{currentstroke}{rgb}{0.925000,0.925000,0.925000}%
\pgfsetstrokecolor{currentstroke}%
\pgfsetstrokeopacity{0.500000}%
\pgfsetdash{}{0pt}%
\pgfpathmoveto{\pgfqpoint{0.379073in}{1.123938in}}%
\pgfpathlineto{\pgfqpoint{2.455212in}{0.445871in}}%
\pgfpathlineto{\pgfqpoint{3.558144in}{1.577751in}}%
\pgfpathlineto{\pgfqpoint{1.599613in}{2.147018in}}%
\pgfusepath{stroke,fill}%
\end{pgfscope}%
\begin{pgfscope}%
\pgfsetrectcap%
\pgfsetroundjoin%
\pgfsetlinewidth{0.803000pt}%
\definecolor{currentstroke}{rgb}{0.000000,0.000000,0.000000}%
\pgfsetstrokecolor{currentstroke}%
\pgfsetdash{}{0pt}%
\pgfpathmoveto{\pgfqpoint{0.379073in}{1.123938in}}%
\pgfpathlineto{\pgfqpoint{2.455212in}{0.445871in}}%
\pgfusepath{stroke}%
\end{pgfscope}%
\begin{pgfscope}%
\definecolor{textcolor}{rgb}{0.000000,0.000000,0.000000}%
\pgfsetstrokecolor{textcolor}%
\pgfsetfillcolor{textcolor}%
\pgftext[x=0.730374in, y=0.408886in, left, base,rotate=341.912962]{\color{textcolor}\rmfamily\fontsize{10.000000}{12.000000}\selectfont Position X [\(\displaystyle m\)]}%
\end{pgfscope}%
\begin{pgfscope}%
\pgfsetbuttcap%
\pgfsetroundjoin%
\pgfsetlinewidth{0.803000pt}%
\definecolor{currentstroke}{rgb}{0.690196,0.690196,0.690196}%
\pgfsetstrokecolor{currentstroke}%
\pgfsetdash{}{0pt}%
\pgfpathmoveto{\pgfqpoint{0.638825in}{1.039103in}}%
\pgfpathlineto{\pgfqpoint{1.845599in}{2.075520in}}%
\pgfpathlineto{\pgfqpoint{1.839067in}{3.557488in}}%
\pgfusepath{stroke}%
\end{pgfscope}%
\begin{pgfscope}%
\pgfsetbuttcap%
\pgfsetroundjoin%
\pgfsetlinewidth{0.803000pt}%
\definecolor{currentstroke}{rgb}{0.690196,0.690196,0.690196}%
\pgfsetstrokecolor{currentstroke}%
\pgfsetdash{}{0pt}%
\pgfpathmoveto{\pgfqpoint{1.052229in}{0.904085in}}%
\pgfpathlineto{\pgfqpoint{2.236533in}{1.961891in}}%
\pgfpathlineto{\pgfqpoint{2.246866in}{3.454124in}}%
\pgfusepath{stroke}%
\end{pgfscope}%
\begin{pgfscope}%
\pgfsetbuttcap%
\pgfsetroundjoin%
\pgfsetlinewidth{0.803000pt}%
\definecolor{currentstroke}{rgb}{0.690196,0.690196,0.690196}%
\pgfsetstrokecolor{currentstroke}%
\pgfsetdash{}{0pt}%
\pgfpathmoveto{\pgfqpoint{1.474585in}{0.766144in}}%
\pgfpathlineto{\pgfqpoint{2.635222in}{1.846008in}}%
\pgfpathlineto{\pgfqpoint{2.663108in}{3.348618in}}%
\pgfusepath{stroke}%
\end{pgfscope}%
\begin{pgfscope}%
\pgfsetbuttcap%
\pgfsetroundjoin%
\pgfsetlinewidth{0.803000pt}%
\definecolor{currentstroke}{rgb}{0.690196,0.690196,0.690196}%
\pgfsetstrokecolor{currentstroke}%
\pgfsetdash{}{0pt}%
\pgfpathmoveto{\pgfqpoint{1.906186in}{0.625183in}}%
\pgfpathlineto{\pgfqpoint{3.041899in}{1.727803in}}%
\pgfpathlineto{\pgfqpoint{3.088058in}{3.240906in}}%
\pgfusepath{stroke}%
\end{pgfscope}%
\begin{pgfscope}%
\pgfsetbuttcap%
\pgfsetroundjoin%
\pgfsetlinewidth{0.803000pt}%
\definecolor{currentstroke}{rgb}{0.690196,0.690196,0.690196}%
\pgfsetstrokecolor{currentstroke}%
\pgfsetdash{}{0pt}%
\pgfpathmoveto{\pgfqpoint{2.347339in}{0.481102in}}%
\pgfpathlineto{\pgfqpoint{3.456807in}{1.607205in}}%
\pgfpathlineto{\pgfqpoint{3.521994in}{3.130916in}}%
\pgfusepath{stroke}%
\end{pgfscope}%
\begin{pgfscope}%
\pgfsetrectcap%
\pgfsetroundjoin%
\pgfsetlinewidth{0.803000pt}%
\definecolor{currentstroke}{rgb}{0.000000,0.000000,0.000000}%
\pgfsetstrokecolor{currentstroke}%
\pgfsetdash{}{0pt}%
\pgfpathmoveto{\pgfqpoint{0.649336in}{1.048130in}}%
\pgfpathlineto{\pgfqpoint{0.617757in}{1.021009in}}%
\pgfusepath{stroke}%
\end{pgfscope}%
\begin{pgfscope}%
\definecolor{textcolor}{rgb}{0.000000,0.000000,0.000000}%
\pgfsetstrokecolor{textcolor}%
\pgfsetfillcolor{textcolor}%
\pgftext[x=0.534389in,y=0.819972in,,top]{\color{textcolor}\rmfamily\fontsize{10.000000}{12.000000}\selectfont \(\displaystyle {0}\)}%
\end{pgfscope}%
\begin{pgfscope}%
\pgfsetrectcap%
\pgfsetroundjoin%
\pgfsetlinewidth{0.803000pt}%
\definecolor{currentstroke}{rgb}{0.000000,0.000000,0.000000}%
\pgfsetstrokecolor{currentstroke}%
\pgfsetdash{}{0pt}%
\pgfpathmoveto{\pgfqpoint{1.062554in}{0.913307in}}%
\pgfpathlineto{\pgfqpoint{1.031535in}{0.885601in}}%
\pgfusepath{stroke}%
\end{pgfscope}%
\begin{pgfscope}%
\definecolor{textcolor}{rgb}{0.000000,0.000000,0.000000}%
\pgfsetstrokecolor{textcolor}%
\pgfsetfillcolor{textcolor}%
\pgftext[x=0.948229in,y=0.682089in,,top]{\color{textcolor}\rmfamily\fontsize{10.000000}{12.000000}\selectfont \(\displaystyle {10}\)}%
\end{pgfscope}%
\begin{pgfscope}%
\pgfsetrectcap%
\pgfsetroundjoin%
\pgfsetlinewidth{0.803000pt}%
\definecolor{currentstroke}{rgb}{0.000000,0.000000,0.000000}%
\pgfsetstrokecolor{currentstroke}%
\pgfsetdash{}{0pt}%
\pgfpathmoveto{\pgfqpoint{1.484712in}{0.775566in}}%
\pgfpathlineto{\pgfqpoint{1.454285in}{0.747257in}}%
\pgfusepath{stroke}%
\end{pgfscope}%
\begin{pgfscope}%
\definecolor{textcolor}{rgb}{0.000000,0.000000,0.000000}%
\pgfsetstrokecolor{textcolor}%
\pgfsetfillcolor{textcolor}%
\pgftext[x=1.371064in,y=0.541210in,,top]{\color{textcolor}\rmfamily\fontsize{10.000000}{12.000000}\selectfont \(\displaystyle {20}\)}%
\end{pgfscope}%
\begin{pgfscope}%
\pgfsetrectcap%
\pgfsetroundjoin%
\pgfsetlinewidth{0.803000pt}%
\definecolor{currentstroke}{rgb}{0.000000,0.000000,0.000000}%
\pgfsetstrokecolor{currentstroke}%
\pgfsetdash{}{0pt}%
\pgfpathmoveto{\pgfqpoint{1.916105in}{0.634813in}}%
\pgfpathlineto{\pgfqpoint{1.886304in}{0.605880in}}%
\pgfusepath{stroke}%
\end{pgfscope}%
\begin{pgfscope}%
\definecolor{textcolor}{rgb}{0.000000,0.000000,0.000000}%
\pgfsetstrokecolor{textcolor}%
\pgfsetfillcolor{textcolor}%
\pgftext[x=1.803191in,y=0.397234in,,top]{\color{textcolor}\rmfamily\fontsize{10.000000}{12.000000}\selectfont \(\displaystyle {30}\)}%
\end{pgfscope}%
\begin{pgfscope}%
\pgfsetrectcap%
\pgfsetroundjoin%
\pgfsetlinewidth{0.803000pt}%
\definecolor{currentstroke}{rgb}{0.000000,0.000000,0.000000}%
\pgfsetstrokecolor{currentstroke}%
\pgfsetdash{}{0pt}%
\pgfpathmoveto{\pgfqpoint{2.357038in}{0.490946in}}%
\pgfpathlineto{\pgfqpoint{2.327898in}{0.461369in}}%
\pgfusepath{stroke}%
\end{pgfscope}%
\begin{pgfscope}%
\definecolor{textcolor}{rgb}{0.000000,0.000000,0.000000}%
\pgfsetstrokecolor{textcolor}%
\pgfsetfillcolor{textcolor}%
\pgftext[x=2.244919in,y=0.250060in,,top]{\color{textcolor}\rmfamily\fontsize{10.000000}{12.000000}\selectfont \(\displaystyle {40}\)}%
\end{pgfscope}%
\begin{pgfscope}%
\pgfsetrectcap%
\pgfsetroundjoin%
\pgfsetlinewidth{0.803000pt}%
\definecolor{currentstroke}{rgb}{0.000000,0.000000,0.000000}%
\pgfsetstrokecolor{currentstroke}%
\pgfsetdash{}{0pt}%
\pgfpathmoveto{\pgfqpoint{3.558144in}{1.577751in}}%
\pgfpathlineto{\pgfqpoint{2.455212in}{0.445871in}}%
\pgfusepath{stroke}%
\end{pgfscope}%
\begin{pgfscope}%
\definecolor{textcolor}{rgb}{0.000000,0.000000,0.000000}%
\pgfsetstrokecolor{textcolor}%
\pgfsetfillcolor{textcolor}%
\pgftext[x=3.120747in, y=0.305657in, left, base,rotate=45.742112]{\color{textcolor}\rmfamily\fontsize{10.000000}{12.000000}\selectfont Position Y [\(\displaystyle m\)]}%
\end{pgfscope}%
\begin{pgfscope}%
\pgfsetbuttcap%
\pgfsetroundjoin%
\pgfsetlinewidth{0.803000pt}%
\definecolor{currentstroke}{rgb}{0.690196,0.690196,0.690196}%
\pgfsetstrokecolor{currentstroke}%
\pgfsetdash{}{0pt}%
\pgfpathmoveto{\pgfqpoint{0.375869in}{2.741832in}}%
\pgfpathlineto{\pgfqpoint{0.447702in}{1.181464in}}%
\pgfpathlineto{\pgfqpoint{2.517487in}{0.509780in}}%
\pgfusepath{stroke}%
\end{pgfscope}%
\begin{pgfscope}%
\pgfsetbuttcap%
\pgfsetroundjoin%
\pgfsetlinewidth{0.803000pt}%
\definecolor{currentstroke}{rgb}{0.690196,0.690196,0.690196}%
\pgfsetstrokecolor{currentstroke}%
\pgfsetdash{}{0pt}%
\pgfpathmoveto{\pgfqpoint{0.557754in}{2.874564in}}%
\pgfpathlineto{\pgfqpoint{0.620790in}{1.326549in}}%
\pgfpathlineto{\pgfqpoint{2.674412in}{0.670824in}}%
\pgfusepath{stroke}%
\end{pgfscope}%
\begin{pgfscope}%
\pgfsetbuttcap%
\pgfsetroundjoin%
\pgfsetlinewidth{0.803000pt}%
\definecolor{currentstroke}{rgb}{0.690196,0.690196,0.690196}%
\pgfsetstrokecolor{currentstroke}%
\pgfsetdash{}{0pt}%
\pgfpathmoveto{\pgfqpoint{0.735264in}{3.004102in}}%
\pgfpathlineto{\pgfqpoint{0.789895in}{1.468296in}}%
\pgfpathlineto{\pgfqpoint{2.827536in}{0.827967in}}%
\pgfusepath{stroke}%
\end{pgfscope}%
\begin{pgfscope}%
\pgfsetbuttcap%
\pgfsetroundjoin%
\pgfsetlinewidth{0.803000pt}%
\definecolor{currentstroke}{rgb}{0.690196,0.690196,0.690196}%
\pgfsetstrokecolor{currentstroke}%
\pgfsetdash{}{0pt}%
\pgfpathmoveto{\pgfqpoint{0.908555in}{3.130562in}}%
\pgfpathlineto{\pgfqpoint{0.955152in}{1.606818in}}%
\pgfpathlineto{\pgfqpoint{2.976995in}{0.981349in}}%
\pgfusepath{stroke}%
\end{pgfscope}%
\begin{pgfscope}%
\pgfsetbuttcap%
\pgfsetroundjoin%
\pgfsetlinewidth{0.803000pt}%
\definecolor{currentstroke}{rgb}{0.690196,0.690196,0.690196}%
\pgfsetstrokecolor{currentstroke}%
\pgfsetdash{}{0pt}%
\pgfpathmoveto{\pgfqpoint{1.077775in}{3.254052in}}%
\pgfpathlineto{\pgfqpoint{1.116692in}{1.742224in}}%
\pgfpathlineto{\pgfqpoint{3.122919in}{1.131103in}}%
\pgfusepath{stroke}%
\end{pgfscope}%
\begin{pgfscope}%
\pgfsetbuttcap%
\pgfsetroundjoin%
\pgfsetlinewidth{0.803000pt}%
\definecolor{currentstroke}{rgb}{0.690196,0.690196,0.690196}%
\pgfsetstrokecolor{currentstroke}%
\pgfsetdash{}{0pt}%
\pgfpathmoveto{\pgfqpoint{1.243067in}{3.374674in}}%
\pgfpathlineto{\pgfqpoint{1.274638in}{1.874617in}}%
\pgfpathlineto{\pgfqpoint{3.265433in}{1.277356in}}%
\pgfusepath{stroke}%
\end{pgfscope}%
\begin{pgfscope}%
\pgfsetbuttcap%
\pgfsetroundjoin%
\pgfsetlinewidth{0.803000pt}%
\definecolor{currentstroke}{rgb}{0.690196,0.690196,0.690196}%
\pgfsetstrokecolor{currentstroke}%
\pgfsetdash{}{0pt}%
\pgfpathmoveto{\pgfqpoint{1.404565in}{3.492528in}}%
\pgfpathlineto{\pgfqpoint{1.429109in}{2.004098in}}%
\pgfpathlineto{\pgfqpoint{3.404653in}{1.420231in}}%
\pgfusepath{stroke}%
\end{pgfscope}%
\begin{pgfscope}%
\pgfsetrectcap%
\pgfsetroundjoin%
\pgfsetlinewidth{0.803000pt}%
\definecolor{currentstroke}{rgb}{0.000000,0.000000,0.000000}%
\pgfsetstrokecolor{currentstroke}%
\pgfsetdash{}{0pt}%
\pgfpathmoveto{\pgfqpoint{2.500044in}{0.515441in}}%
\pgfpathlineto{\pgfqpoint{2.552418in}{0.498444in}}%
\pgfusepath{stroke}%
\end{pgfscope}%
\begin{pgfscope}%
\definecolor{textcolor}{rgb}{0.000000,0.000000,0.000000}%
\pgfsetstrokecolor{textcolor}%
\pgfsetfillcolor{textcolor}%
\pgftext[x=2.696573in,y=0.323132in,,top]{\color{textcolor}\rmfamily\fontsize{10.000000}{12.000000}\selectfont \(\displaystyle {-5}\)}%
\end{pgfscope}%
\begin{pgfscope}%
\pgfsetrectcap%
\pgfsetroundjoin%
\pgfsetlinewidth{0.803000pt}%
\definecolor{currentstroke}{rgb}{0.000000,0.000000,0.000000}%
\pgfsetstrokecolor{currentstroke}%
\pgfsetdash{}{0pt}%
\pgfpathmoveto{\pgfqpoint{2.657116in}{0.676347in}}%
\pgfpathlineto{\pgfqpoint{2.709049in}{0.659765in}}%
\pgfusepath{stroke}%
\end{pgfscope}%
\begin{pgfscope}%
\definecolor{textcolor}{rgb}{0.000000,0.000000,0.000000}%
\pgfsetstrokecolor{textcolor}%
\pgfsetfillcolor{textcolor}%
\pgftext[x=2.851394in,y=0.486561in,,top]{\color{textcolor}\rmfamily\fontsize{10.000000}{12.000000}\selectfont \(\displaystyle {0}\)}%
\end{pgfscope}%
\begin{pgfscope}%
\pgfsetrectcap%
\pgfsetroundjoin%
\pgfsetlinewidth{0.803000pt}%
\definecolor{currentstroke}{rgb}{0.000000,0.000000,0.000000}%
\pgfsetstrokecolor{currentstroke}%
\pgfsetdash{}{0pt}%
\pgfpathmoveto{\pgfqpoint{2.810385in}{0.833357in}}%
\pgfpathlineto{\pgfqpoint{2.861882in}{0.817174in}}%
\pgfusepath{stroke}%
\end{pgfscope}%
\begin{pgfscope}%
\definecolor{textcolor}{rgb}{0.000000,0.000000,0.000000}%
\pgfsetstrokecolor{textcolor}%
\pgfsetfillcolor{textcolor}%
\pgftext[x=3.002463in,y=0.646029in,,top]{\color{textcolor}\rmfamily\fontsize{10.000000}{12.000000}\selectfont \(\displaystyle {5}\)}%
\end{pgfscope}%
\begin{pgfscope}%
\pgfsetrectcap%
\pgfsetroundjoin%
\pgfsetlinewidth{0.803000pt}%
\definecolor{currentstroke}{rgb}{0.000000,0.000000,0.000000}%
\pgfsetstrokecolor{currentstroke}%
\pgfsetdash{}{0pt}%
\pgfpathmoveto{\pgfqpoint{2.959987in}{0.986610in}}%
\pgfpathlineto{\pgfqpoint{3.011054in}{0.970812in}}%
\pgfusepath{stroke}%
\end{pgfscope}%
\begin{pgfscope}%
\definecolor{textcolor}{rgb}{0.000000,0.000000,0.000000}%
\pgfsetstrokecolor{textcolor}%
\pgfsetfillcolor{textcolor}%
\pgftext[x=3.149913in,y=0.801678in,,top]{\color{textcolor}\rmfamily\fontsize{10.000000}{12.000000}\selectfont \(\displaystyle {10}\)}%
\end{pgfscope}%
\begin{pgfscope}%
\pgfsetrectcap%
\pgfsetroundjoin%
\pgfsetlinewidth{0.803000pt}%
\definecolor{currentstroke}{rgb}{0.000000,0.000000,0.000000}%
\pgfsetstrokecolor{currentstroke}%
\pgfsetdash{}{0pt}%
\pgfpathmoveto{\pgfqpoint{3.106052in}{1.136241in}}%
\pgfpathlineto{\pgfqpoint{3.156695in}{1.120814in}}%
\pgfusepath{stroke}%
\end{pgfscope}%
\begin{pgfscope}%
\definecolor{textcolor}{rgb}{0.000000,0.000000,0.000000}%
\pgfsetstrokecolor{textcolor}%
\pgfsetfillcolor{textcolor}%
\pgftext[x=3.293875in,y=0.953643in,,top]{\color{textcolor}\rmfamily\fontsize{10.000000}{12.000000}\selectfont \(\displaystyle {15}\)}%
\end{pgfscope}%
\begin{pgfscope}%
\pgfsetrectcap%
\pgfsetroundjoin%
\pgfsetlinewidth{0.803000pt}%
\definecolor{currentstroke}{rgb}{0.000000,0.000000,0.000000}%
\pgfsetstrokecolor{currentstroke}%
\pgfsetdash{}{0pt}%
\pgfpathmoveto{\pgfqpoint{3.248705in}{1.282375in}}%
\pgfpathlineto{\pgfqpoint{3.298929in}{1.267307in}}%
\pgfusepath{stroke}%
\end{pgfscope}%
\begin{pgfscope}%
\definecolor{textcolor}{rgb}{0.000000,0.000000,0.000000}%
\pgfsetstrokecolor{textcolor}%
\pgfsetfillcolor{textcolor}%
\pgftext[x=3.434470in,y=1.102055in,,top]{\color{textcolor}\rmfamily\fontsize{10.000000}{12.000000}\selectfont \(\displaystyle {20}\)}%
\end{pgfscope}%
\begin{pgfscope}%
\pgfsetrectcap%
\pgfsetroundjoin%
\pgfsetlinewidth{0.803000pt}%
\definecolor{currentstroke}{rgb}{0.000000,0.000000,0.000000}%
\pgfsetstrokecolor{currentstroke}%
\pgfsetdash{}{0pt}%
\pgfpathmoveto{\pgfqpoint{3.388063in}{1.425134in}}%
\pgfpathlineto{\pgfqpoint{3.437874in}{1.410413in}}%
\pgfusepath{stroke}%
\end{pgfscope}%
\begin{pgfscope}%
\definecolor{textcolor}{rgb}{0.000000,0.000000,0.000000}%
\pgfsetstrokecolor{textcolor}%
\pgfsetfillcolor{textcolor}%
\pgftext[x=3.571814in,y=1.247035in,,top]{\color{textcolor}\rmfamily\fontsize{10.000000}{12.000000}\selectfont \(\displaystyle {25}\)}%
\end{pgfscope}%
\begin{pgfscope}%
\pgfsetrectcap%
\pgfsetroundjoin%
\pgfsetlinewidth{0.803000pt}%
\definecolor{currentstroke}{rgb}{0.000000,0.000000,0.000000}%
\pgfsetstrokecolor{currentstroke}%
\pgfsetdash{}{0pt}%
\pgfpathmoveto{\pgfqpoint{3.558144in}{1.577751in}}%
\pgfpathlineto{\pgfqpoint{3.628038in}{3.104037in}}%
\pgfusepath{stroke}%
\end{pgfscope}%
\begin{pgfscope}%
\definecolor{textcolor}{rgb}{0.000000,0.000000,0.000000}%
\pgfsetstrokecolor{textcolor}%
\pgfsetfillcolor{textcolor}%
\pgftext[x=4.167903in, y=1.963517in, left, base,rotate=87.378092]{\color{textcolor}\rmfamily\fontsize{10.000000}{12.000000}\selectfont Position Z [\(\displaystyle m\)]}%
\end{pgfscope}%
\begin{pgfscope}%
\pgfsetbuttcap%
\pgfsetroundjoin%
\pgfsetlinewidth{0.803000pt}%
\definecolor{currentstroke}{rgb}{0.690196,0.690196,0.690196}%
\pgfsetstrokecolor{currentstroke}%
\pgfsetdash{}{0pt}%
\pgfpathmoveto{\pgfqpoint{3.563216in}{1.688511in}}%
\pgfpathlineto{\pgfqpoint{1.598380in}{2.254301in}}%
\pgfpathlineto{\pgfqpoint{0.373612in}{1.237346in}}%
\pgfusepath{stroke}%
\end{pgfscope}%
\begin{pgfscope}%
\pgfsetbuttcap%
\pgfsetroundjoin%
\pgfsetlinewidth{0.803000pt}%
\definecolor{currentstroke}{rgb}{0.690196,0.690196,0.690196}%
\pgfsetstrokecolor{currentstroke}%
\pgfsetdash{}{0pt}%
\pgfpathmoveto{\pgfqpoint{3.574656in}{1.938323in}}%
\pgfpathlineto{\pgfqpoint{1.595599in}{2.496147in}}%
\pgfpathlineto{\pgfqpoint{0.361290in}{1.493233in}}%
\pgfusepath{stroke}%
\end{pgfscope}%
\begin{pgfscope}%
\pgfsetbuttcap%
\pgfsetroundjoin%
\pgfsetlinewidth{0.803000pt}%
\definecolor{currentstroke}{rgb}{0.690196,0.690196,0.690196}%
\pgfsetstrokecolor{currentstroke}%
\pgfsetdash{}{0pt}%
\pgfpathmoveto{\pgfqpoint{3.586264in}{2.191812in}}%
\pgfpathlineto{\pgfqpoint{1.592779in}{2.741383in}}%
\pgfpathlineto{\pgfqpoint{0.348779in}{1.753032in}}%
\pgfusepath{stroke}%
\end{pgfscope}%
\begin{pgfscope}%
\pgfsetbuttcap%
\pgfsetroundjoin%
\pgfsetlinewidth{0.803000pt}%
\definecolor{currentstroke}{rgb}{0.690196,0.690196,0.690196}%
\pgfsetstrokecolor{currentstroke}%
\pgfsetdash{}{0pt}%
\pgfpathmoveto{\pgfqpoint{3.598044in}{2.449062in}}%
\pgfpathlineto{\pgfqpoint{1.589919in}{2.990081in}}%
\pgfpathlineto{\pgfqpoint{0.336075in}{2.016833in}}%
\pgfusepath{stroke}%
\end{pgfscope}%
\begin{pgfscope}%
\pgfsetbuttcap%
\pgfsetroundjoin%
\pgfsetlinewidth{0.803000pt}%
\definecolor{currentstroke}{rgb}{0.690196,0.690196,0.690196}%
\pgfsetstrokecolor{currentstroke}%
\pgfsetdash{}{0pt}%
\pgfpathmoveto{\pgfqpoint{3.610001in}{2.710157in}}%
\pgfpathlineto{\pgfqpoint{1.587018in}{3.242315in}}%
\pgfpathlineto{\pgfqpoint{0.323174in}{2.284730in}}%
\pgfusepath{stroke}%
\end{pgfscope}%
\begin{pgfscope}%
\pgfsetbuttcap%
\pgfsetroundjoin%
\pgfsetlinewidth{0.803000pt}%
\definecolor{currentstroke}{rgb}{0.690196,0.690196,0.690196}%
\pgfsetstrokecolor{currentstroke}%
\pgfsetdash{}{0pt}%
\pgfpathmoveto{\pgfqpoint{3.622137in}{2.975183in}}%
\pgfpathlineto{\pgfqpoint{1.584076in}{3.498161in}}%
\pgfpathlineto{\pgfqpoint{0.310071in}{2.556819in}}%
\pgfusepath{stroke}%
\end{pgfscope}%
\begin{pgfscope}%
\pgfsetrectcap%
\pgfsetroundjoin%
\pgfsetlinewidth{0.803000pt}%
\definecolor{currentstroke}{rgb}{0.000000,0.000000,0.000000}%
\pgfsetstrokecolor{currentstroke}%
\pgfsetdash{}{0pt}%
\pgfpathmoveto{\pgfqpoint{3.546724in}{1.693260in}}%
\pgfpathlineto{\pgfqpoint{3.596242in}{1.679001in}}%
\pgfusepath{stroke}%
\end{pgfscope}%
\begin{pgfscope}%
\definecolor{textcolor}{rgb}{0.000000,0.000000,0.000000}%
\pgfsetstrokecolor{textcolor}%
\pgfsetfillcolor{textcolor}%
\pgftext[x=3.817476in,y=1.724476in,,top]{\color{textcolor}\rmfamily\fontsize{10.000000}{12.000000}\selectfont \(\displaystyle {-2}\)}%
\end{pgfscope}%
\begin{pgfscope}%
\pgfsetrectcap%
\pgfsetroundjoin%
\pgfsetlinewidth{0.803000pt}%
\definecolor{currentstroke}{rgb}{0.000000,0.000000,0.000000}%
\pgfsetstrokecolor{currentstroke}%
\pgfsetdash{}{0pt}%
\pgfpathmoveto{\pgfqpoint{3.558038in}{1.943007in}}%
\pgfpathlineto{\pgfqpoint{3.607932in}{1.928943in}}%
\pgfusepath{stroke}%
\end{pgfscope}%
\begin{pgfscope}%
\definecolor{textcolor}{rgb}{0.000000,0.000000,0.000000}%
\pgfsetstrokecolor{textcolor}%
\pgfsetfillcolor{textcolor}%
\pgftext[x=3.830734in,y=1.973792in,,top]{\color{textcolor}\rmfamily\fontsize{10.000000}{12.000000}\selectfont \(\displaystyle {0}\)}%
\end{pgfscope}%
\begin{pgfscope}%
\pgfsetrectcap%
\pgfsetroundjoin%
\pgfsetlinewidth{0.803000pt}%
\definecolor{currentstroke}{rgb}{0.000000,0.000000,0.000000}%
\pgfsetstrokecolor{currentstroke}%
\pgfsetdash{}{0pt}%
\pgfpathmoveto{\pgfqpoint{3.569519in}{2.196429in}}%
\pgfpathlineto{\pgfqpoint{3.619795in}{2.182569in}}%
\pgfusepath{stroke}%
\end{pgfscope}%
\begin{pgfscope}%
\definecolor{textcolor}{rgb}{0.000000,0.000000,0.000000}%
\pgfsetstrokecolor{textcolor}%
\pgfsetfillcolor{textcolor}%
\pgftext[x=3.844186in,y=2.226769in,,top]{\color{textcolor}\rmfamily\fontsize{10.000000}{12.000000}\selectfont \(\displaystyle {2}\)}%
\end{pgfscope}%
\begin{pgfscope}%
\pgfsetrectcap%
\pgfsetroundjoin%
\pgfsetlinewidth{0.803000pt}%
\definecolor{currentstroke}{rgb}{0.000000,0.000000,0.000000}%
\pgfsetstrokecolor{currentstroke}%
\pgfsetdash{}{0pt}%
\pgfpathmoveto{\pgfqpoint{3.581170in}{2.453609in}}%
\pgfpathlineto{\pgfqpoint{3.631833in}{2.439959in}}%
\pgfusepath{stroke}%
\end{pgfscope}%
\begin{pgfscope}%
\definecolor{textcolor}{rgb}{0.000000,0.000000,0.000000}%
\pgfsetstrokecolor{textcolor}%
\pgfsetfillcolor{textcolor}%
\pgftext[x=3.857837in,y=2.483486in,,top]{\color{textcolor}\rmfamily\fontsize{10.000000}{12.000000}\selectfont \(\displaystyle {4}\)}%
\end{pgfscope}%
\begin{pgfscope}%
\pgfsetrectcap%
\pgfsetroundjoin%
\pgfsetlinewidth{0.803000pt}%
\definecolor{currentstroke}{rgb}{0.000000,0.000000,0.000000}%
\pgfsetstrokecolor{currentstroke}%
\pgfsetdash{}{0pt}%
\pgfpathmoveto{\pgfqpoint{3.592996in}{2.714630in}}%
\pgfpathlineto{\pgfqpoint{3.644052in}{2.701199in}}%
\pgfusepath{stroke}%
\end{pgfscope}%
\begin{pgfscope}%
\definecolor{textcolor}{rgb}{0.000000,0.000000,0.000000}%
\pgfsetstrokecolor{textcolor}%
\pgfsetfillcolor{textcolor}%
\pgftext[x=3.871691in,y=2.744029in,,top]{\color{textcolor}\rmfamily\fontsize{10.000000}{12.000000}\selectfont \(\displaystyle {6}\)}%
\end{pgfscope}%
\begin{pgfscope}%
\pgfsetrectcap%
\pgfsetroundjoin%
\pgfsetlinewidth{0.803000pt}%
\definecolor{currentstroke}{rgb}{0.000000,0.000000,0.000000}%
\pgfsetstrokecolor{currentstroke}%
\pgfsetdash{}{0pt}%
\pgfpathmoveto{\pgfqpoint{3.604999in}{2.979580in}}%
\pgfpathlineto{\pgfqpoint{3.656455in}{2.966376in}}%
\pgfusepath{stroke}%
\end{pgfscope}%
\begin{pgfscope}%
\definecolor{textcolor}{rgb}{0.000000,0.000000,0.000000}%
\pgfsetstrokecolor{textcolor}%
\pgfsetfillcolor{textcolor}%
\pgftext[x=3.885754in,y=3.008481in,,top]{\color{textcolor}\rmfamily\fontsize{10.000000}{12.000000}\selectfont \(\displaystyle {8}\)}%
\end{pgfscope}%
\begin{pgfscope}%
\pgfpathrectangle{\pgfqpoint{0.100000in}{0.212622in}}{\pgfqpoint{3.696000in}{3.696000in}}%
\pgfusepath{clip}%
\pgfsetrectcap%
\pgfsetroundjoin%
\pgfsetlinewidth{1.505625pt}%
\definecolor{currentstroke}{rgb}{0.121569,0.466667,0.705882}%
\pgfsetstrokecolor{currentstroke}%
\pgfsetdash{}{0pt}%
\pgfpathmoveto{\pgfqpoint{0.865737in}{1.611460in}}%
\pgfpathlineto{\pgfqpoint{1.764737in}{2.360069in}}%
\pgfpathlineto{\pgfqpoint{2.046127in}{1.245600in}}%
\pgfpathlineto{\pgfqpoint{0.865737in}{1.611460in}}%
\pgfusepath{stroke}%
\end{pgfscope}%
\begin{pgfscope}%
\pgfpathrectangle{\pgfqpoint{0.100000in}{0.212622in}}{\pgfqpoint{3.696000in}{3.696000in}}%
\pgfusepath{clip}%
\pgfsetrectcap%
\pgfsetroundjoin%
\pgfsetlinewidth{1.505625pt}%
\definecolor{currentstroke}{rgb}{1.000000,0.000000,0.000000}%
\pgfsetstrokecolor{currentstroke}%
\pgfsetdash{}{0pt}%
\pgfpathmoveto{\pgfqpoint{0.864999in}{1.612012in}}%
\pgfpathlineto{\pgfqpoint{0.865737in}{1.611460in}}%
\pgfusepath{stroke}%
\end{pgfscope}%
\begin{pgfscope}%
\pgfpathrectangle{\pgfqpoint{0.100000in}{0.212622in}}{\pgfqpoint{3.696000in}{3.696000in}}%
\pgfusepath{clip}%
\pgfsetrectcap%
\pgfsetroundjoin%
\pgfsetlinewidth{1.505625pt}%
\definecolor{currentstroke}{rgb}{1.000000,0.000000,0.000000}%
\pgfsetstrokecolor{currentstroke}%
\pgfsetdash{}{0pt}%
\pgfpathmoveto{\pgfqpoint{0.898778in}{1.558189in}}%
\pgfpathlineto{\pgfqpoint{0.865737in}{1.611460in}}%
\pgfusepath{stroke}%
\end{pgfscope}%
\begin{pgfscope}%
\pgfpathrectangle{\pgfqpoint{0.100000in}{0.212622in}}{\pgfqpoint{3.696000in}{3.696000in}}%
\pgfusepath{clip}%
\pgfsetrectcap%
\pgfsetroundjoin%
\pgfsetlinewidth{1.505625pt}%
\definecolor{currentstroke}{rgb}{1.000000,0.000000,0.000000}%
\pgfsetstrokecolor{currentstroke}%
\pgfsetdash{}{0pt}%
\pgfpathmoveto{\pgfqpoint{0.859925in}{1.523662in}}%
\pgfpathlineto{\pgfqpoint{0.865737in}{1.611460in}}%
\pgfusepath{stroke}%
\end{pgfscope}%
\begin{pgfscope}%
\pgfpathrectangle{\pgfqpoint{0.100000in}{0.212622in}}{\pgfqpoint{3.696000in}{3.696000in}}%
\pgfusepath{clip}%
\pgfsetrectcap%
\pgfsetroundjoin%
\pgfsetlinewidth{1.505625pt}%
\definecolor{currentstroke}{rgb}{1.000000,0.000000,0.000000}%
\pgfsetstrokecolor{currentstroke}%
\pgfsetdash{}{0pt}%
\pgfpathmoveto{\pgfqpoint{0.902127in}{1.609687in}}%
\pgfpathlineto{\pgfqpoint{0.865737in}{1.611460in}}%
\pgfusepath{stroke}%
\end{pgfscope}%
\begin{pgfscope}%
\pgfpathrectangle{\pgfqpoint{0.100000in}{0.212622in}}{\pgfqpoint{3.696000in}{3.696000in}}%
\pgfusepath{clip}%
\pgfsetrectcap%
\pgfsetroundjoin%
\pgfsetlinewidth{1.505625pt}%
\definecolor{currentstroke}{rgb}{1.000000,0.000000,0.000000}%
\pgfsetstrokecolor{currentstroke}%
\pgfsetdash{}{0pt}%
\pgfpathmoveto{\pgfqpoint{1.878396in}{3.370216in}}%
\pgfpathlineto{\pgfqpoint{1.764737in}{2.360069in}}%
\pgfusepath{stroke}%
\end{pgfscope}%
\begin{pgfscope}%
\pgfpathrectangle{\pgfqpoint{0.100000in}{0.212622in}}{\pgfqpoint{3.696000in}{3.696000in}}%
\pgfusepath{clip}%
\pgfsetrectcap%
\pgfsetroundjoin%
\pgfsetlinewidth{1.505625pt}%
\definecolor{currentstroke}{rgb}{1.000000,0.000000,0.000000}%
\pgfsetstrokecolor{currentstroke}%
\pgfsetdash{}{0pt}%
\pgfpathmoveto{\pgfqpoint{1.931998in}{3.220125in}}%
\pgfpathlineto{\pgfqpoint{1.764737in}{2.360069in}}%
\pgfusepath{stroke}%
\end{pgfscope}%
\begin{pgfscope}%
\pgfpathrectangle{\pgfqpoint{0.100000in}{0.212622in}}{\pgfqpoint{3.696000in}{3.696000in}}%
\pgfusepath{clip}%
\pgfsetrectcap%
\pgfsetroundjoin%
\pgfsetlinewidth{1.505625pt}%
\definecolor{currentstroke}{rgb}{1.000000,0.000000,0.000000}%
\pgfsetstrokecolor{currentstroke}%
\pgfsetdash{}{0pt}%
\pgfpathmoveto{\pgfqpoint{2.106736in}{2.101721in}}%
\pgfpathlineto{\pgfqpoint{2.046127in}{1.245600in}}%
\pgfusepath{stroke}%
\end{pgfscope}%
\begin{pgfscope}%
\pgfpathrectangle{\pgfqpoint{0.100000in}{0.212622in}}{\pgfqpoint{3.696000in}{3.696000in}}%
\pgfusepath{clip}%
\pgfsetrectcap%
\pgfsetroundjoin%
\pgfsetlinewidth{1.505625pt}%
\definecolor{currentstroke}{rgb}{1.000000,0.000000,0.000000}%
\pgfsetstrokecolor{currentstroke}%
\pgfsetdash{}{0pt}%
\pgfpathmoveto{\pgfqpoint{2.443691in}{0.787390in}}%
\pgfpathlineto{\pgfqpoint{2.046127in}{1.245600in}}%
\pgfusepath{stroke}%
\end{pgfscope}%
\begin{pgfscope}%
\pgfpathrectangle{\pgfqpoint{0.100000in}{0.212622in}}{\pgfqpoint{3.696000in}{3.696000in}}%
\pgfusepath{clip}%
\pgfsetrectcap%
\pgfsetroundjoin%
\pgfsetlinewidth{1.505625pt}%
\definecolor{currentstroke}{rgb}{1.000000,0.000000,0.000000}%
\pgfsetstrokecolor{currentstroke}%
\pgfsetdash{}{0pt}%
\pgfpathmoveto{\pgfqpoint{2.084811in}{0.993331in}}%
\pgfpathlineto{\pgfqpoint{2.046127in}{1.245600in}}%
\pgfusepath{stroke}%
\end{pgfscope}%
\begin{pgfscope}%
\pgfpathrectangle{\pgfqpoint{0.100000in}{0.212622in}}{\pgfqpoint{3.696000in}{3.696000in}}%
\pgfusepath{clip}%
\pgfsetrectcap%
\pgfsetroundjoin%
\pgfsetlinewidth{1.505625pt}%
\definecolor{currentstroke}{rgb}{1.000000,0.000000,0.000000}%
\pgfsetstrokecolor{currentstroke}%
\pgfsetdash{}{0pt}%
\pgfpathmoveto{\pgfqpoint{1.400390in}{1.215425in}}%
\pgfpathlineto{\pgfqpoint{2.046127in}{1.245600in}}%
\pgfusepath{stroke}%
\end{pgfscope}%
\begin{pgfscope}%
\pgfpathrectangle{\pgfqpoint{0.100000in}{0.212622in}}{\pgfqpoint{3.696000in}{3.696000in}}%
\pgfusepath{clip}%
\pgfsetrectcap%
\pgfsetroundjoin%
\pgfsetlinewidth{1.505625pt}%
\definecolor{currentstroke}{rgb}{1.000000,0.000000,0.000000}%
\pgfsetstrokecolor{currentstroke}%
\pgfsetdash{}{0pt}%
\pgfpathmoveto{\pgfqpoint{0.585743in}{1.251650in}}%
\pgfpathlineto{\pgfqpoint{0.865737in}{1.611460in}}%
\pgfusepath{stroke}%
\end{pgfscope}%
\begin{pgfscope}%
\pgfpathrectangle{\pgfqpoint{0.100000in}{0.212622in}}{\pgfqpoint{3.696000in}{3.696000in}}%
\pgfusepath{clip}%
\pgfsetbuttcap%
\pgfsetroundjoin%
\definecolor{currentfill}{rgb}{1.000000,0.498039,0.054902}%
\pgfsetfillcolor{currentfill}%
\pgfsetfillopacity{0.300000}%
\pgfsetlinewidth{1.003750pt}%
\definecolor{currentstroke}{rgb}{1.000000,0.498039,0.054902}%
\pgfsetstrokecolor{currentstroke}%
\pgfsetstrokeopacity{0.300000}%
\pgfsetdash{}{0pt}%
\pgfpathmoveto{\pgfqpoint{1.878396in}{3.339159in}}%
\pgfpathcurveto{\pgfqpoint{1.886632in}{3.339159in}}{\pgfqpoint{1.894532in}{3.342432in}}{\pgfqpoint{1.900356in}{3.348255in}}%
\pgfpathcurveto{\pgfqpoint{1.906180in}{3.354079in}}{\pgfqpoint{1.909453in}{3.361979in}}{\pgfqpoint{1.909453in}{3.370216in}}%
\pgfpathcurveto{\pgfqpoint{1.909453in}{3.378452in}}{\pgfqpoint{1.906180in}{3.386352in}}{\pgfqpoint{1.900356in}{3.392176in}}%
\pgfpathcurveto{\pgfqpoint{1.894532in}{3.398000in}}{\pgfqpoint{1.886632in}{3.401272in}}{\pgfqpoint{1.878396in}{3.401272in}}%
\pgfpathcurveto{\pgfqpoint{1.870160in}{3.401272in}}{\pgfqpoint{1.862260in}{3.398000in}}{\pgfqpoint{1.856436in}{3.392176in}}%
\pgfpathcurveto{\pgfqpoint{1.850612in}{3.386352in}}{\pgfqpoint{1.847340in}{3.378452in}}{\pgfqpoint{1.847340in}{3.370216in}}%
\pgfpathcurveto{\pgfqpoint{1.847340in}{3.361979in}}{\pgfqpoint{1.850612in}{3.354079in}}{\pgfqpoint{1.856436in}{3.348255in}}%
\pgfpathcurveto{\pgfqpoint{1.862260in}{3.342432in}}{\pgfqpoint{1.870160in}{3.339159in}}{\pgfqpoint{1.878396in}{3.339159in}}%
\pgfpathclose%
\pgfusepath{stroke,fill}%
\end{pgfscope}%
\begin{pgfscope}%
\pgfpathrectangle{\pgfqpoint{0.100000in}{0.212622in}}{\pgfqpoint{3.696000in}{3.696000in}}%
\pgfusepath{clip}%
\pgfsetbuttcap%
\pgfsetroundjoin%
\definecolor{currentfill}{rgb}{1.000000,0.498039,0.054902}%
\pgfsetfillcolor{currentfill}%
\pgfsetfillopacity{0.329481}%
\pgfsetlinewidth{1.003750pt}%
\definecolor{currentstroke}{rgb}{1.000000,0.498039,0.054902}%
\pgfsetstrokecolor{currentstroke}%
\pgfsetstrokeopacity{0.329481}%
\pgfsetdash{}{0pt}%
\pgfpathmoveto{\pgfqpoint{1.931998in}{3.189069in}}%
\pgfpathcurveto{\pgfqpoint{1.940235in}{3.189069in}}{\pgfqpoint{1.948135in}{3.192341in}}{\pgfqpoint{1.953959in}{3.198165in}}%
\pgfpathcurveto{\pgfqpoint{1.959783in}{3.203989in}}{\pgfqpoint{1.963055in}{3.211889in}}{\pgfqpoint{1.963055in}{3.220125in}}%
\pgfpathcurveto{\pgfqpoint{1.963055in}{3.228361in}}{\pgfqpoint{1.959783in}{3.236261in}}{\pgfqpoint{1.953959in}{3.242085in}}%
\pgfpathcurveto{\pgfqpoint{1.948135in}{3.247909in}}{\pgfqpoint{1.940235in}{3.251182in}}{\pgfqpoint{1.931998in}{3.251182in}}%
\pgfpathcurveto{\pgfqpoint{1.923762in}{3.251182in}}{\pgfqpoint{1.915862in}{3.247909in}}{\pgfqpoint{1.910038in}{3.242085in}}%
\pgfpathcurveto{\pgfqpoint{1.904214in}{3.236261in}}{\pgfqpoint{1.900942in}{3.228361in}}{\pgfqpoint{1.900942in}{3.220125in}}%
\pgfpathcurveto{\pgfqpoint{1.900942in}{3.211889in}}{\pgfqpoint{1.904214in}{3.203989in}}{\pgfqpoint{1.910038in}{3.198165in}}%
\pgfpathcurveto{\pgfqpoint{1.915862in}{3.192341in}}{\pgfqpoint{1.923762in}{3.189069in}}{\pgfqpoint{1.931998in}{3.189069in}}%
\pgfpathclose%
\pgfusepath{stroke,fill}%
\end{pgfscope}%
\begin{pgfscope}%
\pgfpathrectangle{\pgfqpoint{0.100000in}{0.212622in}}{\pgfqpoint{3.696000in}{3.696000in}}%
\pgfusepath{clip}%
\pgfsetbuttcap%
\pgfsetroundjoin%
\definecolor{currentfill}{rgb}{1.000000,0.498039,0.054902}%
\pgfsetfillcolor{currentfill}%
\pgfsetfillopacity{0.578496}%
\pgfsetlinewidth{1.003750pt}%
\definecolor{currentstroke}{rgb}{1.000000,0.498039,0.054902}%
\pgfsetstrokecolor{currentstroke}%
\pgfsetstrokeopacity{0.578496}%
\pgfsetdash{}{0pt}%
\pgfpathmoveto{\pgfqpoint{0.898778in}{1.527132in}}%
\pgfpathcurveto{\pgfqpoint{0.907015in}{1.527132in}}{\pgfqpoint{0.914915in}{1.530405in}}{\pgfqpoint{0.920739in}{1.536229in}}%
\pgfpathcurveto{\pgfqpoint{0.926563in}{1.542053in}}{\pgfqpoint{0.929835in}{1.549953in}}{\pgfqpoint{0.929835in}{1.558189in}}%
\pgfpathcurveto{\pgfqpoint{0.929835in}{1.566425in}}{\pgfqpoint{0.926563in}{1.574325in}}{\pgfqpoint{0.920739in}{1.580149in}}%
\pgfpathcurveto{\pgfqpoint{0.914915in}{1.585973in}}{\pgfqpoint{0.907015in}{1.589245in}}{\pgfqpoint{0.898778in}{1.589245in}}%
\pgfpathcurveto{\pgfqpoint{0.890542in}{1.589245in}}{\pgfqpoint{0.882642in}{1.585973in}}{\pgfqpoint{0.876818in}{1.580149in}}%
\pgfpathcurveto{\pgfqpoint{0.870994in}{1.574325in}}{\pgfqpoint{0.867722in}{1.566425in}}{\pgfqpoint{0.867722in}{1.558189in}}%
\pgfpathcurveto{\pgfqpoint{0.867722in}{1.549953in}}{\pgfqpoint{0.870994in}{1.542053in}}{\pgfqpoint{0.876818in}{1.536229in}}%
\pgfpathcurveto{\pgfqpoint{0.882642in}{1.530405in}}{\pgfqpoint{0.890542in}{1.527132in}}{\pgfqpoint{0.898778in}{1.527132in}}%
\pgfpathclose%
\pgfusepath{stroke,fill}%
\end{pgfscope}%
\begin{pgfscope}%
\pgfpathrectangle{\pgfqpoint{0.100000in}{0.212622in}}{\pgfqpoint{3.696000in}{3.696000in}}%
\pgfusepath{clip}%
\pgfsetbuttcap%
\pgfsetroundjoin%
\definecolor{currentfill}{rgb}{1.000000,0.498039,0.054902}%
\pgfsetfillcolor{currentfill}%
\pgfsetfillopacity{0.587159}%
\pgfsetlinewidth{1.003750pt}%
\definecolor{currentstroke}{rgb}{1.000000,0.498039,0.054902}%
\pgfsetstrokecolor{currentstroke}%
\pgfsetstrokeopacity{0.587159}%
\pgfsetdash{}{0pt}%
\pgfpathmoveto{\pgfqpoint{0.585743in}{1.220593in}}%
\pgfpathcurveto{\pgfqpoint{0.593979in}{1.220593in}}{\pgfqpoint{0.601879in}{1.223866in}}{\pgfqpoint{0.607703in}{1.229690in}}%
\pgfpathcurveto{\pgfqpoint{0.613527in}{1.235514in}}{\pgfqpoint{0.616799in}{1.243414in}}{\pgfqpoint{0.616799in}{1.251650in}}%
\pgfpathcurveto{\pgfqpoint{0.616799in}{1.259886in}}{\pgfqpoint{0.613527in}{1.267786in}}{\pgfqpoint{0.607703in}{1.273610in}}%
\pgfpathcurveto{\pgfqpoint{0.601879in}{1.279434in}}{\pgfqpoint{0.593979in}{1.282706in}}{\pgfqpoint{0.585743in}{1.282706in}}%
\pgfpathcurveto{\pgfqpoint{0.577507in}{1.282706in}}{\pgfqpoint{0.569606in}{1.279434in}}{\pgfqpoint{0.563783in}{1.273610in}}%
\pgfpathcurveto{\pgfqpoint{0.557959in}{1.267786in}}{\pgfqpoint{0.554686in}{1.259886in}}{\pgfqpoint{0.554686in}{1.251650in}}%
\pgfpathcurveto{\pgfqpoint{0.554686in}{1.243414in}}{\pgfqpoint{0.557959in}{1.235514in}}{\pgfqpoint{0.563783in}{1.229690in}}%
\pgfpathcurveto{\pgfqpoint{0.569606in}{1.223866in}}{\pgfqpoint{0.577507in}{1.220593in}}{\pgfqpoint{0.585743in}{1.220593in}}%
\pgfpathclose%
\pgfusepath{stroke,fill}%
\end{pgfscope}%
\begin{pgfscope}%
\pgfpathrectangle{\pgfqpoint{0.100000in}{0.212622in}}{\pgfqpoint{3.696000in}{3.696000in}}%
\pgfusepath{clip}%
\pgfsetbuttcap%
\pgfsetroundjoin%
\definecolor{currentfill}{rgb}{1.000000,0.498039,0.054902}%
\pgfsetfillcolor{currentfill}%
\pgfsetfillopacity{0.593933}%
\pgfsetlinewidth{1.003750pt}%
\definecolor{currentstroke}{rgb}{1.000000,0.498039,0.054902}%
\pgfsetstrokecolor{currentstroke}%
\pgfsetstrokeopacity{0.593933}%
\pgfsetdash{}{0pt}%
\pgfpathmoveto{\pgfqpoint{0.864999in}{1.580955in}}%
\pgfpathcurveto{\pgfqpoint{0.873235in}{1.580955in}}{\pgfqpoint{0.881135in}{1.584228in}}{\pgfqpoint{0.886959in}{1.590052in}}%
\pgfpathcurveto{\pgfqpoint{0.892783in}{1.595875in}}{\pgfqpoint{0.896055in}{1.603776in}}{\pgfqpoint{0.896055in}{1.612012in}}%
\pgfpathcurveto{\pgfqpoint{0.896055in}{1.620248in}}{\pgfqpoint{0.892783in}{1.628148in}}{\pgfqpoint{0.886959in}{1.633972in}}%
\pgfpathcurveto{\pgfqpoint{0.881135in}{1.639796in}}{\pgfqpoint{0.873235in}{1.643068in}}{\pgfqpoint{0.864999in}{1.643068in}}%
\pgfpathcurveto{\pgfqpoint{0.856763in}{1.643068in}}{\pgfqpoint{0.848863in}{1.639796in}}{\pgfqpoint{0.843039in}{1.633972in}}%
\pgfpathcurveto{\pgfqpoint{0.837215in}{1.628148in}}{\pgfqpoint{0.833942in}{1.620248in}}{\pgfqpoint{0.833942in}{1.612012in}}%
\pgfpathcurveto{\pgfqpoint{0.833942in}{1.603776in}}{\pgfqpoint{0.837215in}{1.595875in}}{\pgfqpoint{0.843039in}{1.590052in}}%
\pgfpathcurveto{\pgfqpoint{0.848863in}{1.584228in}}{\pgfqpoint{0.856763in}{1.580955in}}{\pgfqpoint{0.864999in}{1.580955in}}%
\pgfpathclose%
\pgfusepath{stroke,fill}%
\end{pgfscope}%
\begin{pgfscope}%
\pgfpathrectangle{\pgfqpoint{0.100000in}{0.212622in}}{\pgfqpoint{3.696000in}{3.696000in}}%
\pgfusepath{clip}%
\pgfsetbuttcap%
\pgfsetroundjoin%
\definecolor{currentfill}{rgb}{1.000000,0.498039,0.054902}%
\pgfsetfillcolor{currentfill}%
\pgfsetfillopacity{0.606580}%
\pgfsetlinewidth{1.003750pt}%
\definecolor{currentstroke}{rgb}{1.000000,0.498039,0.054902}%
\pgfsetstrokecolor{currentstroke}%
\pgfsetstrokeopacity{0.606580}%
\pgfsetdash{}{0pt}%
\pgfpathmoveto{\pgfqpoint{2.106736in}{2.070664in}}%
\pgfpathcurveto{\pgfqpoint{2.114973in}{2.070664in}}{\pgfqpoint{2.122873in}{2.073937in}}{\pgfqpoint{2.128697in}{2.079760in}}%
\pgfpathcurveto{\pgfqpoint{2.134521in}{2.085584in}}{\pgfqpoint{2.137793in}{2.093484in}}{\pgfqpoint{2.137793in}{2.101721in}}%
\pgfpathcurveto{\pgfqpoint{2.137793in}{2.109957in}}{\pgfqpoint{2.134521in}{2.117857in}}{\pgfqpoint{2.128697in}{2.123681in}}%
\pgfpathcurveto{\pgfqpoint{2.122873in}{2.129505in}}{\pgfqpoint{2.114973in}{2.132777in}}{\pgfqpoint{2.106736in}{2.132777in}}%
\pgfpathcurveto{\pgfqpoint{2.098500in}{2.132777in}}{\pgfqpoint{2.090600in}{2.129505in}}{\pgfqpoint{2.084776in}{2.123681in}}%
\pgfpathcurveto{\pgfqpoint{2.078952in}{2.117857in}}{\pgfqpoint{2.075680in}{2.109957in}}{\pgfqpoint{2.075680in}{2.101721in}}%
\pgfpathcurveto{\pgfqpoint{2.075680in}{2.093484in}}{\pgfqpoint{2.078952in}{2.085584in}}{\pgfqpoint{2.084776in}{2.079760in}}%
\pgfpathcurveto{\pgfqpoint{2.090600in}{2.073937in}}{\pgfqpoint{2.098500in}{2.070664in}}{\pgfqpoint{2.106736in}{2.070664in}}%
\pgfpathclose%
\pgfusepath{stroke,fill}%
\end{pgfscope}%
\begin{pgfscope}%
\pgfpathrectangle{\pgfqpoint{0.100000in}{0.212622in}}{\pgfqpoint{3.696000in}{3.696000in}}%
\pgfusepath{clip}%
\pgfsetbuttcap%
\pgfsetroundjoin%
\definecolor{currentfill}{rgb}{1.000000,0.498039,0.054902}%
\pgfsetfillcolor{currentfill}%
\pgfsetfillopacity{0.618828}%
\pgfsetlinewidth{1.003750pt}%
\definecolor{currentstroke}{rgb}{1.000000,0.498039,0.054902}%
\pgfsetstrokecolor{currentstroke}%
\pgfsetstrokeopacity{0.618828}%
\pgfsetdash{}{0pt}%
\pgfpathmoveto{\pgfqpoint{0.902127in}{1.578631in}}%
\pgfpathcurveto{\pgfqpoint{0.910363in}{1.578631in}}{\pgfqpoint{0.918263in}{1.581903in}}{\pgfqpoint{0.924087in}{1.587727in}}%
\pgfpathcurveto{\pgfqpoint{0.929911in}{1.593551in}}{\pgfqpoint{0.933183in}{1.601451in}}{\pgfqpoint{0.933183in}{1.609687in}}%
\pgfpathcurveto{\pgfqpoint{0.933183in}{1.617924in}}{\pgfqpoint{0.929911in}{1.625824in}}{\pgfqpoint{0.924087in}{1.631648in}}%
\pgfpathcurveto{\pgfqpoint{0.918263in}{1.637472in}}{\pgfqpoint{0.910363in}{1.640744in}}{\pgfqpoint{0.902127in}{1.640744in}}%
\pgfpathcurveto{\pgfqpoint{0.893890in}{1.640744in}}{\pgfqpoint{0.885990in}{1.637472in}}{\pgfqpoint{0.880167in}{1.631648in}}%
\pgfpathcurveto{\pgfqpoint{0.874343in}{1.625824in}}{\pgfqpoint{0.871070in}{1.617924in}}{\pgfqpoint{0.871070in}{1.609687in}}%
\pgfpathcurveto{\pgfqpoint{0.871070in}{1.601451in}}{\pgfqpoint{0.874343in}{1.593551in}}{\pgfqpoint{0.880167in}{1.587727in}}%
\pgfpathcurveto{\pgfqpoint{0.885990in}{1.581903in}}{\pgfqpoint{0.893890in}{1.578631in}}{\pgfqpoint{0.902127in}{1.578631in}}%
\pgfpathclose%
\pgfusepath{stroke,fill}%
\end{pgfscope}%
\begin{pgfscope}%
\pgfpathrectangle{\pgfqpoint{0.100000in}{0.212622in}}{\pgfqpoint{3.696000in}{3.696000in}}%
\pgfusepath{clip}%
\pgfsetbuttcap%
\pgfsetroundjoin%
\definecolor{currentfill}{rgb}{1.000000,0.498039,0.054902}%
\pgfsetfillcolor{currentfill}%
\pgfsetfillopacity{0.627170}%
\pgfsetlinewidth{1.003750pt}%
\definecolor{currentstroke}{rgb}{1.000000,0.498039,0.054902}%
\pgfsetstrokecolor{currentstroke}%
\pgfsetstrokeopacity{0.627170}%
\pgfsetdash{}{0pt}%
\pgfpathmoveto{\pgfqpoint{0.859925in}{1.492605in}}%
\pgfpathcurveto{\pgfqpoint{0.868161in}{1.492605in}}{\pgfqpoint{0.876061in}{1.495877in}}{\pgfqpoint{0.881885in}{1.501701in}}%
\pgfpathcurveto{\pgfqpoint{0.887709in}{1.507525in}}{\pgfqpoint{0.890981in}{1.515425in}}{\pgfqpoint{0.890981in}{1.523662in}}%
\pgfpathcurveto{\pgfqpoint{0.890981in}{1.531898in}}{\pgfqpoint{0.887709in}{1.539798in}}{\pgfqpoint{0.881885in}{1.545622in}}%
\pgfpathcurveto{\pgfqpoint{0.876061in}{1.551446in}}{\pgfqpoint{0.868161in}{1.554718in}}{\pgfqpoint{0.859925in}{1.554718in}}%
\pgfpathcurveto{\pgfqpoint{0.851688in}{1.554718in}}{\pgfqpoint{0.843788in}{1.551446in}}{\pgfqpoint{0.837965in}{1.545622in}}%
\pgfpathcurveto{\pgfqpoint{0.832141in}{1.539798in}}{\pgfqpoint{0.828868in}{1.531898in}}{\pgfqpoint{0.828868in}{1.523662in}}%
\pgfpathcurveto{\pgfqpoint{0.828868in}{1.515425in}}{\pgfqpoint{0.832141in}{1.507525in}}{\pgfqpoint{0.837965in}{1.501701in}}%
\pgfpathcurveto{\pgfqpoint{0.843788in}{1.495877in}}{\pgfqpoint{0.851688in}{1.492605in}}{\pgfqpoint{0.859925in}{1.492605in}}%
\pgfpathclose%
\pgfusepath{stroke,fill}%
\end{pgfscope}%
\begin{pgfscope}%
\pgfpathrectangle{\pgfqpoint{0.100000in}{0.212622in}}{\pgfqpoint{3.696000in}{3.696000in}}%
\pgfusepath{clip}%
\pgfsetbuttcap%
\pgfsetroundjoin%
\definecolor{currentfill}{rgb}{1.000000,0.498039,0.054902}%
\pgfsetfillcolor{currentfill}%
\pgfsetfillopacity{0.778128}%
\pgfsetlinewidth{1.003750pt}%
\definecolor{currentstroke}{rgb}{1.000000,0.498039,0.054902}%
\pgfsetstrokecolor{currentstroke}%
\pgfsetstrokeopacity{0.778128}%
\pgfsetdash{}{0pt}%
\pgfpathmoveto{\pgfqpoint{1.400390in}{1.184368in}}%
\pgfpathcurveto{\pgfqpoint{1.408626in}{1.184368in}}{\pgfqpoint{1.416527in}{1.187641in}}{\pgfqpoint{1.422350in}{1.193464in}}%
\pgfpathcurveto{\pgfqpoint{1.428174in}{1.199288in}}{\pgfqpoint{1.431447in}{1.207188in}}{\pgfqpoint{1.431447in}{1.215425in}}%
\pgfpathcurveto{\pgfqpoint{1.431447in}{1.223661in}}{\pgfqpoint{1.428174in}{1.231561in}}{\pgfqpoint{1.422350in}{1.237385in}}%
\pgfpathcurveto{\pgfqpoint{1.416527in}{1.243209in}}{\pgfqpoint{1.408626in}{1.246481in}}{\pgfqpoint{1.400390in}{1.246481in}}%
\pgfpathcurveto{\pgfqpoint{1.392154in}{1.246481in}}{\pgfqpoint{1.384254in}{1.243209in}}{\pgfqpoint{1.378430in}{1.237385in}}%
\pgfpathcurveto{\pgfqpoint{1.372606in}{1.231561in}}{\pgfqpoint{1.369334in}{1.223661in}}{\pgfqpoint{1.369334in}{1.215425in}}%
\pgfpathcurveto{\pgfqpoint{1.369334in}{1.207188in}}{\pgfqpoint{1.372606in}{1.199288in}}{\pgfqpoint{1.378430in}{1.193464in}}%
\pgfpathcurveto{\pgfqpoint{1.384254in}{1.187641in}}{\pgfqpoint{1.392154in}{1.184368in}}{\pgfqpoint{1.400390in}{1.184368in}}%
\pgfpathclose%
\pgfusepath{stroke,fill}%
\end{pgfscope}%
\begin{pgfscope}%
\pgfpathrectangle{\pgfqpoint{0.100000in}{0.212622in}}{\pgfqpoint{3.696000in}{3.696000in}}%
\pgfusepath{clip}%
\pgfsetbuttcap%
\pgfsetroundjoin%
\definecolor{currentfill}{rgb}{1.000000,0.498039,0.054902}%
\pgfsetfillcolor{currentfill}%
\pgfsetfillopacity{0.925805}%
\pgfsetlinewidth{1.003750pt}%
\definecolor{currentstroke}{rgb}{1.000000,0.498039,0.054902}%
\pgfsetstrokecolor{currentstroke}%
\pgfsetstrokeopacity{0.925805}%
\pgfsetdash{}{0pt}%
\pgfpathmoveto{\pgfqpoint{2.084811in}{0.962275in}}%
\pgfpathcurveto{\pgfqpoint{2.093047in}{0.962275in}}{\pgfqpoint{2.100947in}{0.965547in}}{\pgfqpoint{2.106771in}{0.971371in}}%
\pgfpathcurveto{\pgfqpoint{2.112595in}{0.977195in}}{\pgfqpoint{2.115867in}{0.985095in}}{\pgfqpoint{2.115867in}{0.993331in}}%
\pgfpathcurveto{\pgfqpoint{2.115867in}{1.001567in}}{\pgfqpoint{2.112595in}{1.009468in}}{\pgfqpoint{2.106771in}{1.015291in}}%
\pgfpathcurveto{\pgfqpoint{2.100947in}{1.021115in}}{\pgfqpoint{2.093047in}{1.024388in}}{\pgfqpoint{2.084811in}{1.024388in}}%
\pgfpathcurveto{\pgfqpoint{2.076575in}{1.024388in}}{\pgfqpoint{2.068674in}{1.021115in}}{\pgfqpoint{2.062851in}{1.015291in}}%
\pgfpathcurveto{\pgfqpoint{2.057027in}{1.009468in}}{\pgfqpoint{2.053754in}{1.001567in}}{\pgfqpoint{2.053754in}{0.993331in}}%
\pgfpathcurveto{\pgfqpoint{2.053754in}{0.985095in}}{\pgfqpoint{2.057027in}{0.977195in}}{\pgfqpoint{2.062851in}{0.971371in}}%
\pgfpathcurveto{\pgfqpoint{2.068674in}{0.965547in}}{\pgfqpoint{2.076575in}{0.962275in}}{\pgfqpoint{2.084811in}{0.962275in}}%
\pgfpathclose%
\pgfusepath{stroke,fill}%
\end{pgfscope}%
\begin{pgfscope}%
\pgfpathrectangle{\pgfqpoint{0.100000in}{0.212622in}}{\pgfqpoint{3.696000in}{3.696000in}}%
\pgfusepath{clip}%
\pgfsetbuttcap%
\pgfsetroundjoin%
\definecolor{currentfill}{rgb}{1.000000,0.498039,0.054902}%
\pgfsetfillcolor{currentfill}%
\pgfsetlinewidth{1.003750pt}%
\definecolor{currentstroke}{rgb}{1.000000,0.498039,0.054902}%
\pgfsetstrokecolor{currentstroke}%
\pgfsetdash{}{0pt}%
\pgfpathmoveto{\pgfqpoint{2.443691in}{0.756333in}}%
\pgfpathcurveto{\pgfqpoint{2.451927in}{0.756333in}}{\pgfqpoint{2.459827in}{0.759605in}}{\pgfqpoint{2.465651in}{0.765429in}}%
\pgfpathcurveto{\pgfqpoint{2.471475in}{0.771253in}}{\pgfqpoint{2.474747in}{0.779153in}}{\pgfqpoint{2.474747in}{0.787390in}}%
\pgfpathcurveto{\pgfqpoint{2.474747in}{0.795626in}}{\pgfqpoint{2.471475in}{0.803526in}}{\pgfqpoint{2.465651in}{0.809350in}}%
\pgfpathcurveto{\pgfqpoint{2.459827in}{0.815174in}}{\pgfqpoint{2.451927in}{0.818446in}}{\pgfqpoint{2.443691in}{0.818446in}}%
\pgfpathcurveto{\pgfqpoint{2.435454in}{0.818446in}}{\pgfqpoint{2.427554in}{0.815174in}}{\pgfqpoint{2.421730in}{0.809350in}}%
\pgfpathcurveto{\pgfqpoint{2.415906in}{0.803526in}}{\pgfqpoint{2.412634in}{0.795626in}}{\pgfqpoint{2.412634in}{0.787390in}}%
\pgfpathcurveto{\pgfqpoint{2.412634in}{0.779153in}}{\pgfqpoint{2.415906in}{0.771253in}}{\pgfqpoint{2.421730in}{0.765429in}}%
\pgfpathcurveto{\pgfqpoint{2.427554in}{0.759605in}}{\pgfqpoint{2.435454in}{0.756333in}}{\pgfqpoint{2.443691in}{0.756333in}}%
\pgfpathclose%
\pgfusepath{stroke,fill}%
\end{pgfscope}%
\begin{pgfscope}%
\definecolor{textcolor}{rgb}{0.000000,0.000000,0.000000}%
\pgfsetstrokecolor{textcolor}%
\pgfsetfillcolor{textcolor}%
\pgftext[x=1.948000in,y=3.991956in,,base]{\color{textcolor}\rmfamily\fontsize{12.000000}{14.400000}\selectfont EKF}%
\end{pgfscope}%
\begin{pgfscope}%
\pgfpathrectangle{\pgfqpoint{0.100000in}{0.212622in}}{\pgfqpoint{3.696000in}{3.696000in}}%
\pgfusepath{clip}%
\pgfsetbuttcap%
\pgfsetroundjoin%
\definecolor{currentfill}{rgb}{0.121569,0.466667,0.705882}%
\pgfsetfillcolor{currentfill}%
\pgfsetfillopacity{0.300000}%
\pgfsetlinewidth{1.003750pt}%
\definecolor{currentstroke}{rgb}{0.121569,0.466667,0.705882}%
\pgfsetstrokecolor{currentstroke}%
\pgfsetstrokeopacity{0.300000}%
\pgfsetdash{}{0pt}%
\pgfpathmoveto{\pgfqpoint{1.874951in}{3.342068in}}%
\pgfpathcurveto{\pgfqpoint{1.883187in}{3.342068in}}{\pgfqpoint{1.891088in}{3.345341in}}{\pgfqpoint{1.896911in}{3.351164in}}%
\pgfpathcurveto{\pgfqpoint{1.902735in}{3.356988in}}{\pgfqpoint{1.906008in}{3.364888in}}{\pgfqpoint{1.906008in}{3.373125in}}%
\pgfpathcurveto{\pgfqpoint{1.906008in}{3.381361in}}{\pgfqpoint{1.902735in}{3.389261in}}{\pgfqpoint{1.896911in}{3.395085in}}%
\pgfpathcurveto{\pgfqpoint{1.891088in}{3.400909in}}{\pgfqpoint{1.883187in}{3.404181in}}{\pgfqpoint{1.874951in}{3.404181in}}%
\pgfpathcurveto{\pgfqpoint{1.866715in}{3.404181in}}{\pgfqpoint{1.858815in}{3.400909in}}{\pgfqpoint{1.852991in}{3.395085in}}%
\pgfpathcurveto{\pgfqpoint{1.847167in}{3.389261in}}{\pgfqpoint{1.843895in}{3.381361in}}{\pgfqpoint{1.843895in}{3.373125in}}%
\pgfpathcurveto{\pgfqpoint{1.843895in}{3.364888in}}{\pgfqpoint{1.847167in}{3.356988in}}{\pgfqpoint{1.852991in}{3.351164in}}%
\pgfpathcurveto{\pgfqpoint{1.858815in}{3.345341in}}{\pgfqpoint{1.866715in}{3.342068in}}{\pgfqpoint{1.874951in}{3.342068in}}%
\pgfpathclose%
\pgfusepath{stroke,fill}%
\end{pgfscope}%
\begin{pgfscope}%
\pgfpathrectangle{\pgfqpoint{0.100000in}{0.212622in}}{\pgfqpoint{3.696000in}{3.696000in}}%
\pgfusepath{clip}%
\pgfsetbuttcap%
\pgfsetroundjoin%
\definecolor{currentfill}{rgb}{0.121569,0.466667,0.705882}%
\pgfsetfillcolor{currentfill}%
\pgfsetfillopacity{0.300016}%
\pgfsetlinewidth{1.003750pt}%
\definecolor{currentstroke}{rgb}{0.121569,0.466667,0.705882}%
\pgfsetstrokecolor{currentstroke}%
\pgfsetstrokeopacity{0.300016}%
\pgfsetdash{}{0pt}%
\pgfpathmoveto{\pgfqpoint{1.876319in}{3.341152in}}%
\pgfpathcurveto{\pgfqpoint{1.884555in}{3.341152in}}{\pgfqpoint{1.892455in}{3.344424in}}{\pgfqpoint{1.898279in}{3.350248in}}%
\pgfpathcurveto{\pgfqpoint{1.904103in}{3.356072in}}{\pgfqpoint{1.907376in}{3.363972in}}{\pgfqpoint{1.907376in}{3.372208in}}%
\pgfpathcurveto{\pgfqpoint{1.907376in}{3.380445in}}{\pgfqpoint{1.904103in}{3.388345in}}{\pgfqpoint{1.898279in}{3.394169in}}%
\pgfpathcurveto{\pgfqpoint{1.892455in}{3.399992in}}{\pgfqpoint{1.884555in}{3.403265in}}{\pgfqpoint{1.876319in}{3.403265in}}%
\pgfpathcurveto{\pgfqpoint{1.868083in}{3.403265in}}{\pgfqpoint{1.860183in}{3.399992in}}{\pgfqpoint{1.854359in}{3.394169in}}%
\pgfpathcurveto{\pgfqpoint{1.848535in}{3.388345in}}{\pgfqpoint{1.845263in}{3.380445in}}{\pgfqpoint{1.845263in}{3.372208in}}%
\pgfpathcurveto{\pgfqpoint{1.845263in}{3.363972in}}{\pgfqpoint{1.848535in}{3.356072in}}{\pgfqpoint{1.854359in}{3.350248in}}%
\pgfpathcurveto{\pgfqpoint{1.860183in}{3.344424in}}{\pgfqpoint{1.868083in}{3.341152in}}{\pgfqpoint{1.876319in}{3.341152in}}%
\pgfpathclose%
\pgfusepath{stroke,fill}%
\end{pgfscope}%
\begin{pgfscope}%
\pgfpathrectangle{\pgfqpoint{0.100000in}{0.212622in}}{\pgfqpoint{3.696000in}{3.696000in}}%
\pgfusepath{clip}%
\pgfsetbuttcap%
\pgfsetroundjoin%
\definecolor{currentfill}{rgb}{0.121569,0.466667,0.705882}%
\pgfsetfillcolor{currentfill}%
\pgfsetfillopacity{0.300020}%
\pgfsetlinewidth{1.003750pt}%
\definecolor{currentstroke}{rgb}{0.121569,0.466667,0.705882}%
\pgfsetstrokecolor{currentstroke}%
\pgfsetstrokeopacity{0.300020}%
\pgfsetdash{}{0pt}%
\pgfpathmoveto{\pgfqpoint{1.874440in}{3.342430in}}%
\pgfpathcurveto{\pgfqpoint{1.882677in}{3.342430in}}{\pgfqpoint{1.890577in}{3.345702in}}{\pgfqpoint{1.896401in}{3.351526in}}%
\pgfpathcurveto{\pgfqpoint{1.902225in}{3.357350in}}{\pgfqpoint{1.905497in}{3.365250in}}{\pgfqpoint{1.905497in}{3.373486in}}%
\pgfpathcurveto{\pgfqpoint{1.905497in}{3.381722in}}{\pgfqpoint{1.902225in}{3.389623in}}{\pgfqpoint{1.896401in}{3.395446in}}%
\pgfpathcurveto{\pgfqpoint{1.890577in}{3.401270in}}{\pgfqpoint{1.882677in}{3.404543in}}{\pgfqpoint{1.874440in}{3.404543in}}%
\pgfpathcurveto{\pgfqpoint{1.866204in}{3.404543in}}{\pgfqpoint{1.858304in}{3.401270in}}{\pgfqpoint{1.852480in}{3.395446in}}%
\pgfpathcurveto{\pgfqpoint{1.846656in}{3.389623in}}{\pgfqpoint{1.843384in}{3.381722in}}{\pgfqpoint{1.843384in}{3.373486in}}%
\pgfpathcurveto{\pgfqpoint{1.843384in}{3.365250in}}{\pgfqpoint{1.846656in}{3.357350in}}{\pgfqpoint{1.852480in}{3.351526in}}%
\pgfpathcurveto{\pgfqpoint{1.858304in}{3.345702in}}{\pgfqpoint{1.866204in}{3.342430in}}{\pgfqpoint{1.874440in}{3.342430in}}%
\pgfpathclose%
\pgfusepath{stroke,fill}%
\end{pgfscope}%
\begin{pgfscope}%
\pgfpathrectangle{\pgfqpoint{0.100000in}{0.212622in}}{\pgfqpoint{3.696000in}{3.696000in}}%
\pgfusepath{clip}%
\pgfsetbuttcap%
\pgfsetroundjoin%
\definecolor{currentfill}{rgb}{0.121569,0.466667,0.705882}%
\pgfsetfillcolor{currentfill}%
\pgfsetfillopacity{0.300021}%
\pgfsetlinewidth{1.003750pt}%
\definecolor{currentstroke}{rgb}{0.121569,0.466667,0.705882}%
\pgfsetstrokecolor{currentstroke}%
\pgfsetstrokeopacity{0.300021}%
\pgfsetdash{}{0pt}%
\pgfpathmoveto{\pgfqpoint{1.877042in}{3.340509in}}%
\pgfpathcurveto{\pgfqpoint{1.885278in}{3.340509in}}{\pgfqpoint{1.893178in}{3.343781in}}{\pgfqpoint{1.899002in}{3.349605in}}%
\pgfpathcurveto{\pgfqpoint{1.904826in}{3.355429in}}{\pgfqpoint{1.908098in}{3.363329in}}{\pgfqpoint{1.908098in}{3.371565in}}%
\pgfpathcurveto{\pgfqpoint{1.908098in}{3.379802in}}{\pgfqpoint{1.904826in}{3.387702in}}{\pgfqpoint{1.899002in}{3.393526in}}%
\pgfpathcurveto{\pgfqpoint{1.893178in}{3.399350in}}{\pgfqpoint{1.885278in}{3.402622in}}{\pgfqpoint{1.877042in}{3.402622in}}%
\pgfpathcurveto{\pgfqpoint{1.868805in}{3.402622in}}{\pgfqpoint{1.860905in}{3.399350in}}{\pgfqpoint{1.855081in}{3.393526in}}%
\pgfpathcurveto{\pgfqpoint{1.849257in}{3.387702in}}{\pgfqpoint{1.845985in}{3.379802in}}{\pgfqpoint{1.845985in}{3.371565in}}%
\pgfpathcurveto{\pgfqpoint{1.845985in}{3.363329in}}{\pgfqpoint{1.849257in}{3.355429in}}{\pgfqpoint{1.855081in}{3.349605in}}%
\pgfpathcurveto{\pgfqpoint{1.860905in}{3.343781in}}{\pgfqpoint{1.868805in}{3.340509in}}{\pgfqpoint{1.877042in}{3.340509in}}%
\pgfpathclose%
\pgfusepath{stroke,fill}%
\end{pgfscope}%
\begin{pgfscope}%
\pgfpathrectangle{\pgfqpoint{0.100000in}{0.212622in}}{\pgfqpoint{3.696000in}{3.696000in}}%
\pgfusepath{clip}%
\pgfsetbuttcap%
\pgfsetroundjoin%
\definecolor{currentfill}{rgb}{0.121569,0.466667,0.705882}%
\pgfsetfillcolor{currentfill}%
\pgfsetfillopacity{0.300076}%
\pgfsetlinewidth{1.003750pt}%
\definecolor{currentstroke}{rgb}{0.121569,0.466667,0.705882}%
\pgfsetstrokecolor{currentstroke}%
\pgfsetstrokeopacity{0.300076}%
\pgfsetdash{}{0pt}%
\pgfpathmoveto{\pgfqpoint{1.878396in}{3.339159in}}%
\pgfpathcurveto{\pgfqpoint{1.886632in}{3.339159in}}{\pgfqpoint{1.894532in}{3.342432in}}{\pgfqpoint{1.900356in}{3.348255in}}%
\pgfpathcurveto{\pgfqpoint{1.906180in}{3.354079in}}{\pgfqpoint{1.909453in}{3.361979in}}{\pgfqpoint{1.909453in}{3.370216in}}%
\pgfpathcurveto{\pgfqpoint{1.909453in}{3.378452in}}{\pgfqpoint{1.906180in}{3.386352in}}{\pgfqpoint{1.900356in}{3.392176in}}%
\pgfpathcurveto{\pgfqpoint{1.894532in}{3.398000in}}{\pgfqpoint{1.886632in}{3.401272in}}{\pgfqpoint{1.878396in}{3.401272in}}%
\pgfpathcurveto{\pgfqpoint{1.870160in}{3.401272in}}{\pgfqpoint{1.862260in}{3.398000in}}{\pgfqpoint{1.856436in}{3.392176in}}%
\pgfpathcurveto{\pgfqpoint{1.850612in}{3.386352in}}{\pgfqpoint{1.847340in}{3.378452in}}{\pgfqpoint{1.847340in}{3.370216in}}%
\pgfpathcurveto{\pgfqpoint{1.847340in}{3.361979in}}{\pgfqpoint{1.850612in}{3.354079in}}{\pgfqpoint{1.856436in}{3.348255in}}%
\pgfpathcurveto{\pgfqpoint{1.862260in}{3.342432in}}{\pgfqpoint{1.870160in}{3.339159in}}{\pgfqpoint{1.878396in}{3.339159in}}%
\pgfpathclose%
\pgfusepath{stroke,fill}%
\end{pgfscope}%
\begin{pgfscope}%
\pgfpathrectangle{\pgfqpoint{0.100000in}{0.212622in}}{\pgfqpoint{3.696000in}{3.696000in}}%
\pgfusepath{clip}%
\pgfsetbuttcap%
\pgfsetroundjoin%
\definecolor{currentfill}{rgb}{0.121569,0.466667,0.705882}%
\pgfsetfillcolor{currentfill}%
\pgfsetfillopacity{0.300089}%
\pgfsetlinewidth{1.003750pt}%
\definecolor{currentstroke}{rgb}{0.121569,0.466667,0.705882}%
\pgfsetstrokecolor{currentstroke}%
\pgfsetstrokeopacity{0.300089}%
\pgfsetdash{}{0pt}%
\pgfpathmoveto{\pgfqpoint{1.873484in}{3.343063in}}%
\pgfpathcurveto{\pgfqpoint{1.881720in}{3.343063in}}{\pgfqpoint{1.889620in}{3.346335in}}{\pgfqpoint{1.895444in}{3.352159in}}%
\pgfpathcurveto{\pgfqpoint{1.901268in}{3.357983in}}{\pgfqpoint{1.904540in}{3.365883in}}{\pgfqpoint{1.904540in}{3.374119in}}%
\pgfpathcurveto{\pgfqpoint{1.904540in}{3.382356in}}{\pgfqpoint{1.901268in}{3.390256in}}{\pgfqpoint{1.895444in}{3.396080in}}%
\pgfpathcurveto{\pgfqpoint{1.889620in}{3.401904in}}{\pgfqpoint{1.881720in}{3.405176in}}{\pgfqpoint{1.873484in}{3.405176in}}%
\pgfpathcurveto{\pgfqpoint{1.865247in}{3.405176in}}{\pgfqpoint{1.857347in}{3.401904in}}{\pgfqpoint{1.851523in}{3.396080in}}%
\pgfpathcurveto{\pgfqpoint{1.845700in}{3.390256in}}{\pgfqpoint{1.842427in}{3.382356in}}{\pgfqpoint{1.842427in}{3.374119in}}%
\pgfpathcurveto{\pgfqpoint{1.842427in}{3.365883in}}{\pgfqpoint{1.845700in}{3.357983in}}{\pgfqpoint{1.851523in}{3.352159in}}%
\pgfpathcurveto{\pgfqpoint{1.857347in}{3.346335in}}{\pgfqpoint{1.865247in}{3.343063in}}{\pgfqpoint{1.873484in}{3.343063in}}%
\pgfpathclose%
\pgfusepath{stroke,fill}%
\end{pgfscope}%
\begin{pgfscope}%
\pgfpathrectangle{\pgfqpoint{0.100000in}{0.212622in}}{\pgfqpoint{3.696000in}{3.696000in}}%
\pgfusepath{clip}%
\pgfsetbuttcap%
\pgfsetroundjoin%
\definecolor{currentfill}{rgb}{0.121569,0.466667,0.705882}%
\pgfsetfillcolor{currentfill}%
\pgfsetfillopacity{0.300116}%
\pgfsetlinewidth{1.003750pt}%
\definecolor{currentstroke}{rgb}{0.121569,0.466667,0.705882}%
\pgfsetstrokecolor{currentstroke}%
\pgfsetstrokeopacity{0.300116}%
\pgfsetdash{}{0pt}%
\pgfpathmoveto{\pgfqpoint{1.873213in}{3.343217in}}%
\pgfpathcurveto{\pgfqpoint{1.881449in}{3.343217in}}{\pgfqpoint{1.889349in}{3.346489in}}{\pgfqpoint{1.895173in}{3.352313in}}%
\pgfpathcurveto{\pgfqpoint{1.900997in}{3.358137in}}{\pgfqpoint{1.904269in}{3.366037in}}{\pgfqpoint{1.904269in}{3.374273in}}%
\pgfpathcurveto{\pgfqpoint{1.904269in}{3.382510in}}{\pgfqpoint{1.900997in}{3.390410in}}{\pgfqpoint{1.895173in}{3.396234in}}%
\pgfpathcurveto{\pgfqpoint{1.889349in}{3.402058in}}{\pgfqpoint{1.881449in}{3.405330in}}{\pgfqpoint{1.873213in}{3.405330in}}%
\pgfpathcurveto{\pgfqpoint{1.864976in}{3.405330in}}{\pgfqpoint{1.857076in}{3.402058in}}{\pgfqpoint{1.851252in}{3.396234in}}%
\pgfpathcurveto{\pgfqpoint{1.845428in}{3.390410in}}{\pgfqpoint{1.842156in}{3.382510in}}{\pgfqpoint{1.842156in}{3.374273in}}%
\pgfpathcurveto{\pgfqpoint{1.842156in}{3.366037in}}{\pgfqpoint{1.845428in}{3.358137in}}{\pgfqpoint{1.851252in}{3.352313in}}%
\pgfpathcurveto{\pgfqpoint{1.857076in}{3.346489in}}{\pgfqpoint{1.864976in}{3.343217in}}{\pgfqpoint{1.873213in}{3.343217in}}%
\pgfpathclose%
\pgfusepath{stroke,fill}%
\end{pgfscope}%
\begin{pgfscope}%
\pgfpathrectangle{\pgfqpoint{0.100000in}{0.212622in}}{\pgfqpoint{3.696000in}{3.696000in}}%
\pgfusepath{clip}%
\pgfsetbuttcap%
\pgfsetroundjoin%
\definecolor{currentfill}{rgb}{0.121569,0.466667,0.705882}%
\pgfsetfillcolor{currentfill}%
\pgfsetfillopacity{0.300172}%
\pgfsetlinewidth{1.003750pt}%
\definecolor{currentstroke}{rgb}{0.121569,0.466667,0.705882}%
\pgfsetstrokecolor{currentstroke}%
\pgfsetstrokeopacity{0.300172}%
\pgfsetdash{}{0pt}%
\pgfpathmoveto{\pgfqpoint{1.872708in}{3.343418in}}%
\pgfpathcurveto{\pgfqpoint{1.880944in}{3.343418in}}{\pgfqpoint{1.888845in}{3.346690in}}{\pgfqpoint{1.894668in}{3.352514in}}%
\pgfpathcurveto{\pgfqpoint{1.900492in}{3.358338in}}{\pgfqpoint{1.903765in}{3.366238in}}{\pgfqpoint{1.903765in}{3.374474in}}%
\pgfpathcurveto{\pgfqpoint{1.903765in}{3.382710in}}{\pgfqpoint{1.900492in}{3.390611in}}{\pgfqpoint{1.894668in}{3.396434in}}%
\pgfpathcurveto{\pgfqpoint{1.888845in}{3.402258in}}{\pgfqpoint{1.880944in}{3.405531in}}{\pgfqpoint{1.872708in}{3.405531in}}%
\pgfpathcurveto{\pgfqpoint{1.864472in}{3.405531in}}{\pgfqpoint{1.856572in}{3.402258in}}{\pgfqpoint{1.850748in}{3.396434in}}%
\pgfpathcurveto{\pgfqpoint{1.844924in}{3.390611in}}{\pgfqpoint{1.841652in}{3.382710in}}{\pgfqpoint{1.841652in}{3.374474in}}%
\pgfpathcurveto{\pgfqpoint{1.841652in}{3.366238in}}{\pgfqpoint{1.844924in}{3.358338in}}{\pgfqpoint{1.850748in}{3.352514in}}%
\pgfpathcurveto{\pgfqpoint{1.856572in}{3.346690in}}{\pgfqpoint{1.864472in}{3.343418in}}{\pgfqpoint{1.872708in}{3.343418in}}%
\pgfpathclose%
\pgfusepath{stroke,fill}%
\end{pgfscope}%
\begin{pgfscope}%
\pgfpathrectangle{\pgfqpoint{0.100000in}{0.212622in}}{\pgfqpoint{3.696000in}{3.696000in}}%
\pgfusepath{clip}%
\pgfsetbuttcap%
\pgfsetroundjoin%
\definecolor{currentfill}{rgb}{0.121569,0.466667,0.705882}%
\pgfsetfillcolor{currentfill}%
\pgfsetfillopacity{0.300293}%
\pgfsetlinewidth{1.003750pt}%
\definecolor{currentstroke}{rgb}{0.121569,0.466667,0.705882}%
\pgfsetstrokecolor{currentstroke}%
\pgfsetstrokeopacity{0.300293}%
\pgfsetdash{}{0pt}%
\pgfpathmoveto{\pgfqpoint{1.880161in}{3.337594in}}%
\pgfpathcurveto{\pgfqpoint{1.888398in}{3.337594in}}{\pgfqpoint{1.896298in}{3.340866in}}{\pgfqpoint{1.902122in}{3.346690in}}%
\pgfpathcurveto{\pgfqpoint{1.907945in}{3.352514in}}{\pgfqpoint{1.911218in}{3.360414in}}{\pgfqpoint{1.911218in}{3.368650in}}%
\pgfpathcurveto{\pgfqpoint{1.911218in}{3.376886in}}{\pgfqpoint{1.907945in}{3.384786in}}{\pgfqpoint{1.902122in}{3.390610in}}%
\pgfpathcurveto{\pgfqpoint{1.896298in}{3.396434in}}{\pgfqpoint{1.888398in}{3.399707in}}{\pgfqpoint{1.880161in}{3.399707in}}%
\pgfpathcurveto{\pgfqpoint{1.871925in}{3.399707in}}{\pgfqpoint{1.864025in}{3.396434in}}{\pgfqpoint{1.858201in}{3.390610in}}%
\pgfpathcurveto{\pgfqpoint{1.852377in}{3.384786in}}{\pgfqpoint{1.849105in}{3.376886in}}{\pgfqpoint{1.849105in}{3.368650in}}%
\pgfpathcurveto{\pgfqpoint{1.849105in}{3.360414in}}{\pgfqpoint{1.852377in}{3.352514in}}{\pgfqpoint{1.858201in}{3.346690in}}%
\pgfpathcurveto{\pgfqpoint{1.864025in}{3.340866in}}{\pgfqpoint{1.871925in}{3.337594in}}{\pgfqpoint{1.880161in}{3.337594in}}%
\pgfpathclose%
\pgfusepath{stroke,fill}%
\end{pgfscope}%
\begin{pgfscope}%
\pgfpathrectangle{\pgfqpoint{0.100000in}{0.212622in}}{\pgfqpoint{3.696000in}{3.696000in}}%
\pgfusepath{clip}%
\pgfsetbuttcap%
\pgfsetroundjoin%
\definecolor{currentfill}{rgb}{0.121569,0.466667,0.705882}%
\pgfsetfillcolor{currentfill}%
\pgfsetfillopacity{0.300296}%
\pgfsetlinewidth{1.003750pt}%
\definecolor{currentstroke}{rgb}{0.121569,0.466667,0.705882}%
\pgfsetstrokecolor{currentstroke}%
\pgfsetstrokeopacity{0.300296}%
\pgfsetdash{}{0pt}%
\pgfpathmoveto{\pgfqpoint{1.871778in}{3.343683in}}%
\pgfpathcurveto{\pgfqpoint{1.880014in}{3.343683in}}{\pgfqpoint{1.887914in}{3.346956in}}{\pgfqpoint{1.893738in}{3.352780in}}%
\pgfpathcurveto{\pgfqpoint{1.899562in}{3.358604in}}{\pgfqpoint{1.902834in}{3.366504in}}{\pgfqpoint{1.902834in}{3.374740in}}%
\pgfpathcurveto{\pgfqpoint{1.902834in}{3.382976in}}{\pgfqpoint{1.899562in}{3.390876in}}{\pgfqpoint{1.893738in}{3.396700in}}%
\pgfpathcurveto{\pgfqpoint{1.887914in}{3.402524in}}{\pgfqpoint{1.880014in}{3.405796in}}{\pgfqpoint{1.871778in}{3.405796in}}%
\pgfpathcurveto{\pgfqpoint{1.863542in}{3.405796in}}{\pgfqpoint{1.855641in}{3.402524in}}{\pgfqpoint{1.849818in}{3.396700in}}%
\pgfpathcurveto{\pgfqpoint{1.843994in}{3.390876in}}{\pgfqpoint{1.840721in}{3.382976in}}{\pgfqpoint{1.840721in}{3.374740in}}%
\pgfpathcurveto{\pgfqpoint{1.840721in}{3.366504in}}{\pgfqpoint{1.843994in}{3.358604in}}{\pgfqpoint{1.849818in}{3.352780in}}%
\pgfpathcurveto{\pgfqpoint{1.855641in}{3.346956in}}{\pgfqpoint{1.863542in}{3.343683in}}{\pgfqpoint{1.871778in}{3.343683in}}%
\pgfpathclose%
\pgfusepath{stroke,fill}%
\end{pgfscope}%
\begin{pgfscope}%
\pgfpathrectangle{\pgfqpoint{0.100000in}{0.212622in}}{\pgfqpoint{3.696000in}{3.696000in}}%
\pgfusepath{clip}%
\pgfsetbuttcap%
\pgfsetroundjoin%
\definecolor{currentfill}{rgb}{0.121569,0.466667,0.705882}%
\pgfsetfillcolor{currentfill}%
\pgfsetfillopacity{0.300526}%
\pgfsetlinewidth{1.003750pt}%
\definecolor{currentstroke}{rgb}{0.121569,0.466667,0.705882}%
\pgfsetstrokecolor{currentstroke}%
\pgfsetstrokeopacity{0.300526}%
\pgfsetdash{}{0pt}%
\pgfpathmoveto{\pgfqpoint{1.870077in}{3.343918in}}%
\pgfpathcurveto{\pgfqpoint{1.878313in}{3.343918in}}{\pgfqpoint{1.886213in}{3.347190in}}{\pgfqpoint{1.892037in}{3.353014in}}%
\pgfpathcurveto{\pgfqpoint{1.897861in}{3.358838in}}{\pgfqpoint{1.901133in}{3.366738in}}{\pgfqpoint{1.901133in}{3.374974in}}%
\pgfpathcurveto{\pgfqpoint{1.901133in}{3.383210in}}{\pgfqpoint{1.897861in}{3.391110in}}{\pgfqpoint{1.892037in}{3.396934in}}%
\pgfpathcurveto{\pgfqpoint{1.886213in}{3.402758in}}{\pgfqpoint{1.878313in}{3.406031in}}{\pgfqpoint{1.870077in}{3.406031in}}%
\pgfpathcurveto{\pgfqpoint{1.861840in}{3.406031in}}{\pgfqpoint{1.853940in}{3.402758in}}{\pgfqpoint{1.848116in}{3.396934in}}%
\pgfpathcurveto{\pgfqpoint{1.842293in}{3.391110in}}{\pgfqpoint{1.839020in}{3.383210in}}{\pgfqpoint{1.839020in}{3.374974in}}%
\pgfpathcurveto{\pgfqpoint{1.839020in}{3.366738in}}{\pgfqpoint{1.842293in}{3.358838in}}{\pgfqpoint{1.848116in}{3.353014in}}%
\pgfpathcurveto{\pgfqpoint{1.853940in}{3.347190in}}{\pgfqpoint{1.861840in}{3.343918in}}{\pgfqpoint{1.870077in}{3.343918in}}%
\pgfpathclose%
\pgfusepath{stroke,fill}%
\end{pgfscope}%
\begin{pgfscope}%
\pgfpathrectangle{\pgfqpoint{0.100000in}{0.212622in}}{\pgfqpoint{3.696000in}{3.696000in}}%
\pgfusepath{clip}%
\pgfsetbuttcap%
\pgfsetroundjoin%
\definecolor{currentfill}{rgb}{0.121569,0.466667,0.705882}%
\pgfsetfillcolor{currentfill}%
\pgfsetfillopacity{0.300614}%
\pgfsetlinewidth{1.003750pt}%
\definecolor{currentstroke}{rgb}{0.121569,0.466667,0.705882}%
\pgfsetstrokecolor{currentstroke}%
\pgfsetstrokeopacity{0.300614}%
\pgfsetdash{}{0pt}%
\pgfpathmoveto{\pgfqpoint{1.883332in}{3.334262in}}%
\pgfpathcurveto{\pgfqpoint{1.891568in}{3.334262in}}{\pgfqpoint{1.899468in}{3.337534in}}{\pgfqpoint{1.905292in}{3.343358in}}%
\pgfpathcurveto{\pgfqpoint{1.911116in}{3.349182in}}{\pgfqpoint{1.914389in}{3.357082in}}{\pgfqpoint{1.914389in}{3.365318in}}%
\pgfpathcurveto{\pgfqpoint{1.914389in}{3.373555in}}{\pgfqpoint{1.911116in}{3.381455in}}{\pgfqpoint{1.905292in}{3.387279in}}%
\pgfpathcurveto{\pgfqpoint{1.899468in}{3.393103in}}{\pgfqpoint{1.891568in}{3.396375in}}{\pgfqpoint{1.883332in}{3.396375in}}%
\pgfpathcurveto{\pgfqpoint{1.875096in}{3.396375in}}{\pgfqpoint{1.867196in}{3.393103in}}{\pgfqpoint{1.861372in}{3.387279in}}%
\pgfpathcurveto{\pgfqpoint{1.855548in}{3.381455in}}{\pgfqpoint{1.852276in}{3.373555in}}{\pgfqpoint{1.852276in}{3.365318in}}%
\pgfpathcurveto{\pgfqpoint{1.852276in}{3.357082in}}{\pgfqpoint{1.855548in}{3.349182in}}{\pgfqpoint{1.861372in}{3.343358in}}%
\pgfpathcurveto{\pgfqpoint{1.867196in}{3.337534in}}{\pgfqpoint{1.875096in}{3.334262in}}{\pgfqpoint{1.883332in}{3.334262in}}%
\pgfpathclose%
\pgfusepath{stroke,fill}%
\end{pgfscope}%
\begin{pgfscope}%
\pgfpathrectangle{\pgfqpoint{0.100000in}{0.212622in}}{\pgfqpoint{3.696000in}{3.696000in}}%
\pgfusepath{clip}%
\pgfsetbuttcap%
\pgfsetroundjoin%
\definecolor{currentfill}{rgb}{0.121569,0.466667,0.705882}%
\pgfsetfillcolor{currentfill}%
\pgfsetfillopacity{0.300720}%
\pgfsetlinewidth{1.003750pt}%
\definecolor{currentstroke}{rgb}{0.121569,0.466667,0.705882}%
\pgfsetstrokecolor{currentstroke}%
\pgfsetstrokeopacity{0.300720}%
\pgfsetdash{}{0pt}%
\pgfpathmoveto{\pgfqpoint{1.868943in}{3.344025in}}%
\pgfpathcurveto{\pgfqpoint{1.877179in}{3.344025in}}{\pgfqpoint{1.885079in}{3.347297in}}{\pgfqpoint{1.890903in}{3.353121in}}%
\pgfpathcurveto{\pgfqpoint{1.896727in}{3.358945in}}{\pgfqpoint{1.900000in}{3.366845in}}{\pgfqpoint{1.900000in}{3.375081in}}%
\pgfpathcurveto{\pgfqpoint{1.900000in}{3.383318in}}{\pgfqpoint{1.896727in}{3.391218in}}{\pgfqpoint{1.890903in}{3.397042in}}%
\pgfpathcurveto{\pgfqpoint{1.885079in}{3.402866in}}{\pgfqpoint{1.877179in}{3.406138in}}{\pgfqpoint{1.868943in}{3.406138in}}%
\pgfpathcurveto{\pgfqpoint{1.860707in}{3.406138in}}{\pgfqpoint{1.852807in}{3.402866in}}{\pgfqpoint{1.846983in}{3.397042in}}%
\pgfpathcurveto{\pgfqpoint{1.841159in}{3.391218in}}{\pgfqpoint{1.837887in}{3.383318in}}{\pgfqpoint{1.837887in}{3.375081in}}%
\pgfpathcurveto{\pgfqpoint{1.837887in}{3.366845in}}{\pgfqpoint{1.841159in}{3.358945in}}{\pgfqpoint{1.846983in}{3.353121in}}%
\pgfpathcurveto{\pgfqpoint{1.852807in}{3.347297in}}{\pgfqpoint{1.860707in}{3.344025in}}{\pgfqpoint{1.868943in}{3.344025in}}%
\pgfpathclose%
\pgfusepath{stroke,fill}%
\end{pgfscope}%
\begin{pgfscope}%
\pgfpathrectangle{\pgfqpoint{0.100000in}{0.212622in}}{\pgfqpoint{3.696000in}{3.696000in}}%
\pgfusepath{clip}%
\pgfsetbuttcap%
\pgfsetroundjoin%
\definecolor{currentfill}{rgb}{0.121569,0.466667,0.705882}%
\pgfsetfillcolor{currentfill}%
\pgfsetfillopacity{0.300816}%
\pgfsetlinewidth{1.003750pt}%
\definecolor{currentstroke}{rgb}{0.121569,0.466667,0.705882}%
\pgfsetstrokecolor{currentstroke}%
\pgfsetstrokeopacity{0.300816}%
\pgfsetdash{}{0pt}%
\pgfpathmoveto{\pgfqpoint{1.868378in}{3.343992in}}%
\pgfpathcurveto{\pgfqpoint{1.876614in}{3.343992in}}{\pgfqpoint{1.884514in}{3.347264in}}{\pgfqpoint{1.890338in}{3.353088in}}%
\pgfpathcurveto{\pgfqpoint{1.896162in}{3.358912in}}{\pgfqpoint{1.899435in}{3.366812in}}{\pgfqpoint{1.899435in}{3.375048in}}%
\pgfpathcurveto{\pgfqpoint{1.899435in}{3.383285in}}{\pgfqpoint{1.896162in}{3.391185in}}{\pgfqpoint{1.890338in}{3.397009in}}%
\pgfpathcurveto{\pgfqpoint{1.884514in}{3.402832in}}{\pgfqpoint{1.876614in}{3.406105in}}{\pgfqpoint{1.868378in}{3.406105in}}%
\pgfpathcurveto{\pgfqpoint{1.860142in}{3.406105in}}{\pgfqpoint{1.852242in}{3.402832in}}{\pgfqpoint{1.846418in}{3.397009in}}%
\pgfpathcurveto{\pgfqpoint{1.840594in}{3.391185in}}{\pgfqpoint{1.837322in}{3.383285in}}{\pgfqpoint{1.837322in}{3.375048in}}%
\pgfpathcurveto{\pgfqpoint{1.837322in}{3.366812in}}{\pgfqpoint{1.840594in}{3.358912in}}{\pgfqpoint{1.846418in}{3.353088in}}%
\pgfpathcurveto{\pgfqpoint{1.852242in}{3.347264in}}{\pgfqpoint{1.860142in}{3.343992in}}{\pgfqpoint{1.868378in}{3.343992in}}%
\pgfpathclose%
\pgfusepath{stroke,fill}%
\end{pgfscope}%
\begin{pgfscope}%
\pgfpathrectangle{\pgfqpoint{0.100000in}{0.212622in}}{\pgfqpoint{3.696000in}{3.696000in}}%
\pgfusepath{clip}%
\pgfsetbuttcap%
\pgfsetroundjoin%
\definecolor{currentfill}{rgb}{0.121569,0.466667,0.705882}%
\pgfsetfillcolor{currentfill}%
\pgfsetfillopacity{0.300994}%
\pgfsetlinewidth{1.003750pt}%
\definecolor{currentstroke}{rgb}{0.121569,0.466667,0.705882}%
\pgfsetstrokecolor{currentstroke}%
\pgfsetstrokeopacity{0.300994}%
\pgfsetdash{}{0pt}%
\pgfpathmoveto{\pgfqpoint{1.867364in}{3.343756in}}%
\pgfpathcurveto{\pgfqpoint{1.875601in}{3.343756in}}{\pgfqpoint{1.883501in}{3.347028in}}{\pgfqpoint{1.889325in}{3.352852in}}%
\pgfpathcurveto{\pgfqpoint{1.895148in}{3.358676in}}{\pgfqpoint{1.898421in}{3.366576in}}{\pgfqpoint{1.898421in}{3.374812in}}%
\pgfpathcurveto{\pgfqpoint{1.898421in}{3.383048in}}{\pgfqpoint{1.895148in}{3.390948in}}{\pgfqpoint{1.889325in}{3.396772in}}%
\pgfpathcurveto{\pgfqpoint{1.883501in}{3.402596in}}{\pgfqpoint{1.875601in}{3.405869in}}{\pgfqpoint{1.867364in}{3.405869in}}%
\pgfpathcurveto{\pgfqpoint{1.859128in}{3.405869in}}{\pgfqpoint{1.851228in}{3.402596in}}{\pgfqpoint{1.845404in}{3.396772in}}%
\pgfpathcurveto{\pgfqpoint{1.839580in}{3.390948in}}{\pgfqpoint{1.836308in}{3.383048in}}{\pgfqpoint{1.836308in}{3.374812in}}%
\pgfpathcurveto{\pgfqpoint{1.836308in}{3.366576in}}{\pgfqpoint{1.839580in}{3.358676in}}{\pgfqpoint{1.845404in}{3.352852in}}%
\pgfpathcurveto{\pgfqpoint{1.851228in}{3.347028in}}{\pgfqpoint{1.859128in}{3.343756in}}{\pgfqpoint{1.867364in}{3.343756in}}%
\pgfpathclose%
\pgfusepath{stroke,fill}%
\end{pgfscope}%
\begin{pgfscope}%
\pgfpathrectangle{\pgfqpoint{0.100000in}{0.212622in}}{\pgfqpoint{3.696000in}{3.696000in}}%
\pgfusepath{clip}%
\pgfsetbuttcap%
\pgfsetroundjoin%
\definecolor{currentfill}{rgb}{0.121569,0.466667,0.705882}%
\pgfsetfillcolor{currentfill}%
\pgfsetfillopacity{0.301042}%
\pgfsetlinewidth{1.003750pt}%
\definecolor{currentstroke}{rgb}{0.121569,0.466667,0.705882}%
\pgfsetstrokecolor{currentstroke}%
\pgfsetstrokeopacity{0.301042}%
\pgfsetdash{}{0pt}%
\pgfpathmoveto{\pgfqpoint{1.867044in}{3.343604in}}%
\pgfpathcurveto{\pgfqpoint{1.875280in}{3.343604in}}{\pgfqpoint{1.883180in}{3.346876in}}{\pgfqpoint{1.889004in}{3.352700in}}%
\pgfpathcurveto{\pgfqpoint{1.894828in}{3.358524in}}{\pgfqpoint{1.898101in}{3.366424in}}{\pgfqpoint{1.898101in}{3.374661in}}%
\pgfpathcurveto{\pgfqpoint{1.898101in}{3.382897in}}{\pgfqpoint{1.894828in}{3.390797in}}{\pgfqpoint{1.889004in}{3.396621in}}%
\pgfpathcurveto{\pgfqpoint{1.883180in}{3.402445in}}{\pgfqpoint{1.875280in}{3.405717in}}{\pgfqpoint{1.867044in}{3.405717in}}%
\pgfpathcurveto{\pgfqpoint{1.858808in}{3.405717in}}{\pgfqpoint{1.850908in}{3.402445in}}{\pgfqpoint{1.845084in}{3.396621in}}%
\pgfpathcurveto{\pgfqpoint{1.839260in}{3.390797in}}{\pgfqpoint{1.835988in}{3.382897in}}{\pgfqpoint{1.835988in}{3.374661in}}%
\pgfpathcurveto{\pgfqpoint{1.835988in}{3.366424in}}{\pgfqpoint{1.839260in}{3.358524in}}{\pgfqpoint{1.845084in}{3.352700in}}%
\pgfpathcurveto{\pgfqpoint{1.850908in}{3.346876in}}{\pgfqpoint{1.858808in}{3.343604in}}{\pgfqpoint{1.867044in}{3.343604in}}%
\pgfpathclose%
\pgfusepath{stroke,fill}%
\end{pgfscope}%
\begin{pgfscope}%
\pgfpathrectangle{\pgfqpoint{0.100000in}{0.212622in}}{\pgfqpoint{3.696000in}{3.696000in}}%
\pgfusepath{clip}%
\pgfsetbuttcap%
\pgfsetroundjoin%
\definecolor{currentfill}{rgb}{0.121569,0.466667,0.705882}%
\pgfsetfillcolor{currentfill}%
\pgfsetfillopacity{0.301125}%
\pgfsetlinewidth{1.003750pt}%
\definecolor{currentstroke}{rgb}{0.121569,0.466667,0.705882}%
\pgfsetstrokecolor{currentstroke}%
\pgfsetstrokeopacity{0.301125}%
\pgfsetdash{}{0pt}%
\pgfpathmoveto{\pgfqpoint{1.886761in}{3.330115in}}%
\pgfpathcurveto{\pgfqpoint{1.894998in}{3.330115in}}{\pgfqpoint{1.902898in}{3.333387in}}{\pgfqpoint{1.908722in}{3.339211in}}%
\pgfpathcurveto{\pgfqpoint{1.914546in}{3.345035in}}{\pgfqpoint{1.917818in}{3.352935in}}{\pgfqpoint{1.917818in}{3.361171in}}%
\pgfpathcurveto{\pgfqpoint{1.917818in}{3.369408in}}{\pgfqpoint{1.914546in}{3.377308in}}{\pgfqpoint{1.908722in}{3.383132in}}%
\pgfpathcurveto{\pgfqpoint{1.902898in}{3.388956in}}{\pgfqpoint{1.894998in}{3.392228in}}{\pgfqpoint{1.886761in}{3.392228in}}%
\pgfpathcurveto{\pgfqpoint{1.878525in}{3.392228in}}{\pgfqpoint{1.870625in}{3.388956in}}{\pgfqpoint{1.864801in}{3.383132in}}%
\pgfpathcurveto{\pgfqpoint{1.858977in}{3.377308in}}{\pgfqpoint{1.855705in}{3.369408in}}{\pgfqpoint{1.855705in}{3.361171in}}%
\pgfpathcurveto{\pgfqpoint{1.855705in}{3.352935in}}{\pgfqpoint{1.858977in}{3.345035in}}{\pgfqpoint{1.864801in}{3.339211in}}%
\pgfpathcurveto{\pgfqpoint{1.870625in}{3.333387in}}{\pgfqpoint{1.878525in}{3.330115in}}{\pgfqpoint{1.886761in}{3.330115in}}%
\pgfpathclose%
\pgfusepath{stroke,fill}%
\end{pgfscope}%
\begin{pgfscope}%
\pgfpathrectangle{\pgfqpoint{0.100000in}{0.212622in}}{\pgfqpoint{3.696000in}{3.696000in}}%
\pgfusepath{clip}%
\pgfsetbuttcap%
\pgfsetroundjoin%
\definecolor{currentfill}{rgb}{0.121569,0.466667,0.705882}%
\pgfsetfillcolor{currentfill}%
\pgfsetfillopacity{0.301165}%
\pgfsetlinewidth{1.003750pt}%
\definecolor{currentstroke}{rgb}{0.121569,0.466667,0.705882}%
\pgfsetstrokecolor{currentstroke}%
\pgfsetstrokeopacity{0.301165}%
\pgfsetdash{}{0pt}%
\pgfpathmoveto{\pgfqpoint{1.866474in}{3.343393in}}%
\pgfpathcurveto{\pgfqpoint{1.874710in}{3.343393in}}{\pgfqpoint{1.882610in}{3.346665in}}{\pgfqpoint{1.888434in}{3.352489in}}%
\pgfpathcurveto{\pgfqpoint{1.894258in}{3.358313in}}{\pgfqpoint{1.897530in}{3.366213in}}{\pgfqpoint{1.897530in}{3.374449in}}%
\pgfpathcurveto{\pgfqpoint{1.897530in}{3.382686in}}{\pgfqpoint{1.894258in}{3.390586in}}{\pgfqpoint{1.888434in}{3.396410in}}%
\pgfpathcurveto{\pgfqpoint{1.882610in}{3.402233in}}{\pgfqpoint{1.874710in}{3.405506in}}{\pgfqpoint{1.866474in}{3.405506in}}%
\pgfpathcurveto{\pgfqpoint{1.858238in}{3.405506in}}{\pgfqpoint{1.850337in}{3.402233in}}{\pgfqpoint{1.844514in}{3.396410in}}%
\pgfpathcurveto{\pgfqpoint{1.838690in}{3.390586in}}{\pgfqpoint{1.835417in}{3.382686in}}{\pgfqpoint{1.835417in}{3.374449in}}%
\pgfpathcurveto{\pgfqpoint{1.835417in}{3.366213in}}{\pgfqpoint{1.838690in}{3.358313in}}{\pgfqpoint{1.844514in}{3.352489in}}%
\pgfpathcurveto{\pgfqpoint{1.850337in}{3.346665in}}{\pgfqpoint{1.858238in}{3.343393in}}{\pgfqpoint{1.866474in}{3.343393in}}%
\pgfpathclose%
\pgfusepath{stroke,fill}%
\end{pgfscope}%
\begin{pgfscope}%
\pgfpathrectangle{\pgfqpoint{0.100000in}{0.212622in}}{\pgfqpoint{3.696000in}{3.696000in}}%
\pgfusepath{clip}%
\pgfsetbuttcap%
\pgfsetroundjoin%
\definecolor{currentfill}{rgb}{0.121569,0.466667,0.705882}%
\pgfsetfillcolor{currentfill}%
\pgfsetfillopacity{0.301383}%
\pgfsetlinewidth{1.003750pt}%
\definecolor{currentstroke}{rgb}{0.121569,0.466667,0.705882}%
\pgfsetstrokecolor{currentstroke}%
\pgfsetstrokeopacity{0.301383}%
\pgfsetdash{}{0pt}%
\pgfpathmoveto{\pgfqpoint{1.865467in}{3.342823in}}%
\pgfpathcurveto{\pgfqpoint{1.873703in}{3.342823in}}{\pgfqpoint{1.881603in}{3.346096in}}{\pgfqpoint{1.887427in}{3.351920in}}%
\pgfpathcurveto{\pgfqpoint{1.893251in}{3.357744in}}{\pgfqpoint{1.896523in}{3.365644in}}{\pgfqpoint{1.896523in}{3.373880in}}%
\pgfpathcurveto{\pgfqpoint{1.896523in}{3.382116in}}{\pgfqpoint{1.893251in}{3.390016in}}{\pgfqpoint{1.887427in}{3.395840in}}%
\pgfpathcurveto{\pgfqpoint{1.881603in}{3.401664in}}{\pgfqpoint{1.873703in}{3.404936in}}{\pgfqpoint{1.865467in}{3.404936in}}%
\pgfpathcurveto{\pgfqpoint{1.857231in}{3.404936in}}{\pgfqpoint{1.849330in}{3.401664in}}{\pgfqpoint{1.843507in}{3.395840in}}%
\pgfpathcurveto{\pgfqpoint{1.837683in}{3.390016in}}{\pgfqpoint{1.834410in}{3.382116in}}{\pgfqpoint{1.834410in}{3.373880in}}%
\pgfpathcurveto{\pgfqpoint{1.834410in}{3.365644in}}{\pgfqpoint{1.837683in}{3.357744in}}{\pgfqpoint{1.843507in}{3.351920in}}%
\pgfpathcurveto{\pgfqpoint{1.849330in}{3.346096in}}{\pgfqpoint{1.857231in}{3.342823in}}{\pgfqpoint{1.865467in}{3.342823in}}%
\pgfpathclose%
\pgfusepath{stroke,fill}%
\end{pgfscope}%
\begin{pgfscope}%
\pgfpathrectangle{\pgfqpoint{0.100000in}{0.212622in}}{\pgfqpoint{3.696000in}{3.696000in}}%
\pgfusepath{clip}%
\pgfsetbuttcap%
\pgfsetroundjoin%
\definecolor{currentfill}{rgb}{0.121569,0.466667,0.705882}%
\pgfsetfillcolor{currentfill}%
\pgfsetfillopacity{0.301535}%
\pgfsetlinewidth{1.003750pt}%
\definecolor{currentstroke}{rgb}{0.121569,0.466667,0.705882}%
\pgfsetstrokecolor{currentstroke}%
\pgfsetstrokeopacity{0.301535}%
\pgfsetdash{}{0pt}%
\pgfpathmoveto{\pgfqpoint{1.888509in}{3.328017in}}%
\pgfpathcurveto{\pgfqpoint{1.896745in}{3.328017in}}{\pgfqpoint{1.904645in}{3.331289in}}{\pgfqpoint{1.910469in}{3.337113in}}%
\pgfpathcurveto{\pgfqpoint{1.916293in}{3.342937in}}{\pgfqpoint{1.919566in}{3.350837in}}{\pgfqpoint{1.919566in}{3.359073in}}%
\pgfpathcurveto{\pgfqpoint{1.919566in}{3.367310in}}{\pgfqpoint{1.916293in}{3.375210in}}{\pgfqpoint{1.910469in}{3.381034in}}%
\pgfpathcurveto{\pgfqpoint{1.904645in}{3.386857in}}{\pgfqpoint{1.896745in}{3.390130in}}{\pgfqpoint{1.888509in}{3.390130in}}%
\pgfpathcurveto{\pgfqpoint{1.880273in}{3.390130in}}{\pgfqpoint{1.872373in}{3.386857in}}{\pgfqpoint{1.866549in}{3.381034in}}%
\pgfpathcurveto{\pgfqpoint{1.860725in}{3.375210in}}{\pgfqpoint{1.857453in}{3.367310in}}{\pgfqpoint{1.857453in}{3.359073in}}%
\pgfpathcurveto{\pgfqpoint{1.857453in}{3.350837in}}{\pgfqpoint{1.860725in}{3.342937in}}{\pgfqpoint{1.866549in}{3.337113in}}%
\pgfpathcurveto{\pgfqpoint{1.872373in}{3.331289in}}{\pgfqpoint{1.880273in}{3.328017in}}{\pgfqpoint{1.888509in}{3.328017in}}%
\pgfpathclose%
\pgfusepath{stroke,fill}%
\end{pgfscope}%
\begin{pgfscope}%
\pgfpathrectangle{\pgfqpoint{0.100000in}{0.212622in}}{\pgfqpoint{3.696000in}{3.696000in}}%
\pgfusepath{clip}%
\pgfsetbuttcap%
\pgfsetroundjoin%
\definecolor{currentfill}{rgb}{0.121569,0.466667,0.705882}%
\pgfsetfillcolor{currentfill}%
\pgfsetfillopacity{0.301546}%
\pgfsetlinewidth{1.003750pt}%
\definecolor{currentstroke}{rgb}{0.121569,0.466667,0.705882}%
\pgfsetstrokecolor{currentstroke}%
\pgfsetstrokeopacity{0.301546}%
\pgfsetdash{}{0pt}%
\pgfpathmoveto{\pgfqpoint{1.864736in}{3.342249in}}%
\pgfpathcurveto{\pgfqpoint{1.872972in}{3.342249in}}{\pgfqpoint{1.880872in}{3.345521in}}{\pgfqpoint{1.886696in}{3.351345in}}%
\pgfpathcurveto{\pgfqpoint{1.892520in}{3.357169in}}{\pgfqpoint{1.895792in}{3.365069in}}{\pgfqpoint{1.895792in}{3.373305in}}%
\pgfpathcurveto{\pgfqpoint{1.895792in}{3.381541in}}{\pgfqpoint{1.892520in}{3.389442in}}{\pgfqpoint{1.886696in}{3.395265in}}%
\pgfpathcurveto{\pgfqpoint{1.880872in}{3.401089in}}{\pgfqpoint{1.872972in}{3.404362in}}{\pgfqpoint{1.864736in}{3.404362in}}%
\pgfpathcurveto{\pgfqpoint{1.856499in}{3.404362in}}{\pgfqpoint{1.848599in}{3.401089in}}{\pgfqpoint{1.842775in}{3.395265in}}%
\pgfpathcurveto{\pgfqpoint{1.836951in}{3.389442in}}{\pgfqpoint{1.833679in}{3.381541in}}{\pgfqpoint{1.833679in}{3.373305in}}%
\pgfpathcurveto{\pgfqpoint{1.833679in}{3.365069in}}{\pgfqpoint{1.836951in}{3.357169in}}{\pgfqpoint{1.842775in}{3.351345in}}%
\pgfpathcurveto{\pgfqpoint{1.848599in}{3.345521in}}{\pgfqpoint{1.856499in}{3.342249in}}{\pgfqpoint{1.864736in}{3.342249in}}%
\pgfpathclose%
\pgfusepath{stroke,fill}%
\end{pgfscope}%
\begin{pgfscope}%
\pgfpathrectangle{\pgfqpoint{0.100000in}{0.212622in}}{\pgfqpoint{3.696000in}{3.696000in}}%
\pgfusepath{clip}%
\pgfsetbuttcap%
\pgfsetroundjoin%
\definecolor{currentfill}{rgb}{0.121569,0.466667,0.705882}%
\pgfsetfillcolor{currentfill}%
\pgfsetfillopacity{0.301587}%
\pgfsetlinewidth{1.003750pt}%
\definecolor{currentstroke}{rgb}{0.121569,0.466667,0.705882}%
\pgfsetstrokecolor{currentstroke}%
\pgfsetstrokeopacity{0.301587}%
\pgfsetdash{}{0pt}%
\pgfpathmoveto{\pgfqpoint{1.864551in}{3.342075in}}%
\pgfpathcurveto{\pgfqpoint{1.872788in}{3.342075in}}{\pgfqpoint{1.880688in}{3.345347in}}{\pgfqpoint{1.886511in}{3.351171in}}%
\pgfpathcurveto{\pgfqpoint{1.892335in}{3.356995in}}{\pgfqpoint{1.895608in}{3.364895in}}{\pgfqpoint{1.895608in}{3.373131in}}%
\pgfpathcurveto{\pgfqpoint{1.895608in}{3.381368in}}{\pgfqpoint{1.892335in}{3.389268in}}{\pgfqpoint{1.886511in}{3.395092in}}%
\pgfpathcurveto{\pgfqpoint{1.880688in}{3.400915in}}{\pgfqpoint{1.872788in}{3.404188in}}{\pgfqpoint{1.864551in}{3.404188in}}%
\pgfpathcurveto{\pgfqpoint{1.856315in}{3.404188in}}{\pgfqpoint{1.848415in}{3.400915in}}{\pgfqpoint{1.842591in}{3.395092in}}%
\pgfpathcurveto{\pgfqpoint{1.836767in}{3.389268in}}{\pgfqpoint{1.833495in}{3.381368in}}{\pgfqpoint{1.833495in}{3.373131in}}%
\pgfpathcurveto{\pgfqpoint{1.833495in}{3.364895in}}{\pgfqpoint{1.836767in}{3.356995in}}{\pgfqpoint{1.842591in}{3.351171in}}%
\pgfpathcurveto{\pgfqpoint{1.848415in}{3.345347in}}{\pgfqpoint{1.856315in}{3.342075in}}{\pgfqpoint{1.864551in}{3.342075in}}%
\pgfpathclose%
\pgfusepath{stroke,fill}%
\end{pgfscope}%
\begin{pgfscope}%
\pgfpathrectangle{\pgfqpoint{0.100000in}{0.212622in}}{\pgfqpoint{3.696000in}{3.696000in}}%
\pgfusepath{clip}%
\pgfsetbuttcap%
\pgfsetroundjoin%
\definecolor{currentfill}{rgb}{0.121569,0.466667,0.705882}%
\pgfsetfillcolor{currentfill}%
\pgfsetfillopacity{0.301664}%
\pgfsetlinewidth{1.003750pt}%
\definecolor{currentstroke}{rgb}{0.121569,0.466667,0.705882}%
\pgfsetstrokecolor{currentstroke}%
\pgfsetstrokeopacity{0.301664}%
\pgfsetdash{}{0pt}%
\pgfpathmoveto{\pgfqpoint{1.864228in}{3.341731in}}%
\pgfpathcurveto{\pgfqpoint{1.872465in}{3.341731in}}{\pgfqpoint{1.880365in}{3.345003in}}{\pgfqpoint{1.886189in}{3.350827in}}%
\pgfpathcurveto{\pgfqpoint{1.892012in}{3.356651in}}{\pgfqpoint{1.895285in}{3.364551in}}{\pgfqpoint{1.895285in}{3.372787in}}%
\pgfpathcurveto{\pgfqpoint{1.895285in}{3.381024in}}{\pgfqpoint{1.892012in}{3.388924in}}{\pgfqpoint{1.886189in}{3.394748in}}%
\pgfpathcurveto{\pgfqpoint{1.880365in}{3.400572in}}{\pgfqpoint{1.872465in}{3.403844in}}{\pgfqpoint{1.864228in}{3.403844in}}%
\pgfpathcurveto{\pgfqpoint{1.855992in}{3.403844in}}{\pgfqpoint{1.848092in}{3.400572in}}{\pgfqpoint{1.842268in}{3.394748in}}%
\pgfpathcurveto{\pgfqpoint{1.836444in}{3.388924in}}{\pgfqpoint{1.833172in}{3.381024in}}{\pgfqpoint{1.833172in}{3.372787in}}%
\pgfpathcurveto{\pgfqpoint{1.833172in}{3.364551in}}{\pgfqpoint{1.836444in}{3.356651in}}{\pgfqpoint{1.842268in}{3.350827in}}%
\pgfpathcurveto{\pgfqpoint{1.848092in}{3.345003in}}{\pgfqpoint{1.855992in}{3.341731in}}{\pgfqpoint{1.864228in}{3.341731in}}%
\pgfpathclose%
\pgfusepath{stroke,fill}%
\end{pgfscope}%
\begin{pgfscope}%
\pgfpathrectangle{\pgfqpoint{0.100000in}{0.212622in}}{\pgfqpoint{3.696000in}{3.696000in}}%
\pgfusepath{clip}%
\pgfsetbuttcap%
\pgfsetroundjoin%
\definecolor{currentfill}{rgb}{0.121569,0.466667,0.705882}%
\pgfsetfillcolor{currentfill}%
\pgfsetfillopacity{0.301800}%
\pgfsetlinewidth{1.003750pt}%
\definecolor{currentstroke}{rgb}{0.121569,0.466667,0.705882}%
\pgfsetstrokecolor{currentstroke}%
\pgfsetstrokeopacity{0.301800}%
\pgfsetdash{}{0pt}%
\pgfpathmoveto{\pgfqpoint{1.863661in}{3.341041in}}%
\pgfpathcurveto{\pgfqpoint{1.871897in}{3.341041in}}{\pgfqpoint{1.879797in}{3.344314in}}{\pgfqpoint{1.885621in}{3.350138in}}%
\pgfpathcurveto{\pgfqpoint{1.891445in}{3.355962in}}{\pgfqpoint{1.894717in}{3.363862in}}{\pgfqpoint{1.894717in}{3.372098in}}%
\pgfpathcurveto{\pgfqpoint{1.894717in}{3.380334in}}{\pgfqpoint{1.891445in}{3.388234in}}{\pgfqpoint{1.885621in}{3.394058in}}%
\pgfpathcurveto{\pgfqpoint{1.879797in}{3.399882in}}{\pgfqpoint{1.871897in}{3.403154in}}{\pgfqpoint{1.863661in}{3.403154in}}%
\pgfpathcurveto{\pgfqpoint{1.855424in}{3.403154in}}{\pgfqpoint{1.847524in}{3.399882in}}{\pgfqpoint{1.841700in}{3.394058in}}%
\pgfpathcurveto{\pgfqpoint{1.835876in}{3.388234in}}{\pgfqpoint{1.832604in}{3.380334in}}{\pgfqpoint{1.832604in}{3.372098in}}%
\pgfpathcurveto{\pgfqpoint{1.832604in}{3.363862in}}{\pgfqpoint{1.835876in}{3.355962in}}{\pgfqpoint{1.841700in}{3.350138in}}%
\pgfpathcurveto{\pgfqpoint{1.847524in}{3.344314in}}{\pgfqpoint{1.855424in}{3.341041in}}{\pgfqpoint{1.863661in}{3.341041in}}%
\pgfpathclose%
\pgfusepath{stroke,fill}%
\end{pgfscope}%
\begin{pgfscope}%
\pgfpathrectangle{\pgfqpoint{0.100000in}{0.212622in}}{\pgfqpoint{3.696000in}{3.696000in}}%
\pgfusepath{clip}%
\pgfsetbuttcap%
\pgfsetroundjoin%
\definecolor{currentfill}{rgb}{0.121569,0.466667,0.705882}%
\pgfsetfillcolor{currentfill}%
\pgfsetfillopacity{0.301812}%
\pgfsetlinewidth{1.003750pt}%
\definecolor{currentstroke}{rgb}{0.121569,0.466667,0.705882}%
\pgfsetstrokecolor{currentstroke}%
\pgfsetstrokeopacity{0.301812}%
\pgfsetdash{}{0pt}%
\pgfpathmoveto{\pgfqpoint{1.890910in}{3.323936in}}%
\pgfpathcurveto{\pgfqpoint{1.899146in}{3.323936in}}{\pgfqpoint{1.907046in}{3.327208in}}{\pgfqpoint{1.912870in}{3.333032in}}%
\pgfpathcurveto{\pgfqpoint{1.918694in}{3.338856in}}{\pgfqpoint{1.921967in}{3.346756in}}{\pgfqpoint{1.921967in}{3.354992in}}%
\pgfpathcurveto{\pgfqpoint{1.921967in}{3.363228in}}{\pgfqpoint{1.918694in}{3.371129in}}{\pgfqpoint{1.912870in}{3.376952in}}%
\pgfpathcurveto{\pgfqpoint{1.907046in}{3.382776in}}{\pgfqpoint{1.899146in}{3.386049in}}{\pgfqpoint{1.890910in}{3.386049in}}%
\pgfpathcurveto{\pgfqpoint{1.882674in}{3.386049in}}{\pgfqpoint{1.874774in}{3.382776in}}{\pgfqpoint{1.868950in}{3.376952in}}%
\pgfpathcurveto{\pgfqpoint{1.863126in}{3.371129in}}{\pgfqpoint{1.859854in}{3.363228in}}{\pgfqpoint{1.859854in}{3.354992in}}%
\pgfpathcurveto{\pgfqpoint{1.859854in}{3.346756in}}{\pgfqpoint{1.863126in}{3.338856in}}{\pgfqpoint{1.868950in}{3.333032in}}%
\pgfpathcurveto{\pgfqpoint{1.874774in}{3.327208in}}{\pgfqpoint{1.882674in}{3.323936in}}{\pgfqpoint{1.890910in}{3.323936in}}%
\pgfpathclose%
\pgfusepath{stroke,fill}%
\end{pgfscope}%
\begin{pgfscope}%
\pgfpathrectangle{\pgfqpoint{0.100000in}{0.212622in}}{\pgfqpoint{3.696000in}{3.696000in}}%
\pgfusepath{clip}%
\pgfsetbuttcap%
\pgfsetroundjoin%
\definecolor{currentfill}{rgb}{0.121569,0.466667,0.705882}%
\pgfsetfillcolor{currentfill}%
\pgfsetfillopacity{0.302045}%
\pgfsetlinewidth{1.003750pt}%
\definecolor{currentstroke}{rgb}{0.121569,0.466667,0.705882}%
\pgfsetstrokecolor{currentstroke}%
\pgfsetstrokeopacity{0.302045}%
\pgfsetdash{}{0pt}%
\pgfpathmoveto{\pgfqpoint{1.862678in}{3.339663in}}%
\pgfpathcurveto{\pgfqpoint{1.870915in}{3.339663in}}{\pgfqpoint{1.878815in}{3.342935in}}{\pgfqpoint{1.884639in}{3.348759in}}%
\pgfpathcurveto{\pgfqpoint{1.890462in}{3.354583in}}{\pgfqpoint{1.893735in}{3.362483in}}{\pgfqpoint{1.893735in}{3.370719in}}%
\pgfpathcurveto{\pgfqpoint{1.893735in}{3.378956in}}{\pgfqpoint{1.890462in}{3.386856in}}{\pgfqpoint{1.884639in}{3.392680in}}%
\pgfpathcurveto{\pgfqpoint{1.878815in}{3.398504in}}{\pgfqpoint{1.870915in}{3.401776in}}{\pgfqpoint{1.862678in}{3.401776in}}%
\pgfpathcurveto{\pgfqpoint{1.854442in}{3.401776in}}{\pgfqpoint{1.846542in}{3.398504in}}{\pgfqpoint{1.840718in}{3.392680in}}%
\pgfpathcurveto{\pgfqpoint{1.834894in}{3.386856in}}{\pgfqpoint{1.831622in}{3.378956in}}{\pgfqpoint{1.831622in}{3.370719in}}%
\pgfpathcurveto{\pgfqpoint{1.831622in}{3.362483in}}{\pgfqpoint{1.834894in}{3.354583in}}{\pgfqpoint{1.840718in}{3.348759in}}%
\pgfpathcurveto{\pgfqpoint{1.846542in}{3.342935in}}{\pgfqpoint{1.854442in}{3.339663in}}{\pgfqpoint{1.862678in}{3.339663in}}%
\pgfpathclose%
\pgfusepath{stroke,fill}%
\end{pgfscope}%
\begin{pgfscope}%
\pgfpathrectangle{\pgfqpoint{0.100000in}{0.212622in}}{\pgfqpoint{3.696000in}{3.696000in}}%
\pgfusepath{clip}%
\pgfsetbuttcap%
\pgfsetroundjoin%
\definecolor{currentfill}{rgb}{0.121569,0.466667,0.705882}%
\pgfsetfillcolor{currentfill}%
\pgfsetfillopacity{0.302221}%
\pgfsetlinewidth{1.003750pt}%
\definecolor{currentstroke}{rgb}{0.121569,0.466667,0.705882}%
\pgfsetstrokecolor{currentstroke}%
\pgfsetstrokeopacity{0.302221}%
\pgfsetdash{}{0pt}%
\pgfpathmoveto{\pgfqpoint{1.861977in}{3.338588in}}%
\pgfpathcurveto{\pgfqpoint{1.870214in}{3.338588in}}{\pgfqpoint{1.878114in}{3.341860in}}{\pgfqpoint{1.883938in}{3.347684in}}%
\pgfpathcurveto{\pgfqpoint{1.889761in}{3.353508in}}{\pgfqpoint{1.893034in}{3.361408in}}{\pgfqpoint{1.893034in}{3.369644in}}%
\pgfpathcurveto{\pgfqpoint{1.893034in}{3.377880in}}{\pgfqpoint{1.889761in}{3.385780in}}{\pgfqpoint{1.883938in}{3.391604in}}%
\pgfpathcurveto{\pgfqpoint{1.878114in}{3.397428in}}{\pgfqpoint{1.870214in}{3.400701in}}{\pgfqpoint{1.861977in}{3.400701in}}%
\pgfpathcurveto{\pgfqpoint{1.853741in}{3.400701in}}{\pgfqpoint{1.845841in}{3.397428in}}{\pgfqpoint{1.840017in}{3.391604in}}%
\pgfpathcurveto{\pgfqpoint{1.834193in}{3.385780in}}{\pgfqpoint{1.830921in}{3.377880in}}{\pgfqpoint{1.830921in}{3.369644in}}%
\pgfpathcurveto{\pgfqpoint{1.830921in}{3.361408in}}{\pgfqpoint{1.834193in}{3.353508in}}{\pgfqpoint{1.840017in}{3.347684in}}%
\pgfpathcurveto{\pgfqpoint{1.845841in}{3.341860in}}{\pgfqpoint{1.853741in}{3.338588in}}{\pgfqpoint{1.861977in}{3.338588in}}%
\pgfpathclose%
\pgfusepath{stroke,fill}%
\end{pgfscope}%
\begin{pgfscope}%
\pgfpathrectangle{\pgfqpoint{0.100000in}{0.212622in}}{\pgfqpoint{3.696000in}{3.696000in}}%
\pgfusepath{clip}%
\pgfsetbuttcap%
\pgfsetroundjoin%
\definecolor{currentfill}{rgb}{0.121569,0.466667,0.705882}%
\pgfsetfillcolor{currentfill}%
\pgfsetfillopacity{0.302289}%
\pgfsetlinewidth{1.003750pt}%
\definecolor{currentstroke}{rgb}{0.121569,0.466667,0.705882}%
\pgfsetstrokecolor{currentstroke}%
\pgfsetstrokeopacity{0.302289}%
\pgfsetdash{}{0pt}%
\pgfpathmoveto{\pgfqpoint{1.861712in}{3.338151in}}%
\pgfpathcurveto{\pgfqpoint{1.869948in}{3.338151in}}{\pgfqpoint{1.877848in}{3.341423in}}{\pgfqpoint{1.883672in}{3.347247in}}%
\pgfpathcurveto{\pgfqpoint{1.889496in}{3.353071in}}{\pgfqpoint{1.892768in}{3.360971in}}{\pgfqpoint{1.892768in}{3.369207in}}%
\pgfpathcurveto{\pgfqpoint{1.892768in}{3.377443in}}{\pgfqpoint{1.889496in}{3.385343in}}{\pgfqpoint{1.883672in}{3.391167in}}%
\pgfpathcurveto{\pgfqpoint{1.877848in}{3.396991in}}{\pgfqpoint{1.869948in}{3.400264in}}{\pgfqpoint{1.861712in}{3.400264in}}%
\pgfpathcurveto{\pgfqpoint{1.853475in}{3.400264in}}{\pgfqpoint{1.845575in}{3.396991in}}{\pgfqpoint{1.839751in}{3.391167in}}%
\pgfpathcurveto{\pgfqpoint{1.833927in}{3.385343in}}{\pgfqpoint{1.830655in}{3.377443in}}{\pgfqpoint{1.830655in}{3.369207in}}%
\pgfpathcurveto{\pgfqpoint{1.830655in}{3.360971in}}{\pgfqpoint{1.833927in}{3.353071in}}{\pgfqpoint{1.839751in}{3.347247in}}%
\pgfpathcurveto{\pgfqpoint{1.845575in}{3.341423in}}{\pgfqpoint{1.853475in}{3.338151in}}{\pgfqpoint{1.861712in}{3.338151in}}%
\pgfpathclose%
\pgfusepath{stroke,fill}%
\end{pgfscope}%
\begin{pgfscope}%
\pgfpathrectangle{\pgfqpoint{0.100000in}{0.212622in}}{\pgfqpoint{3.696000in}{3.696000in}}%
\pgfusepath{clip}%
\pgfsetbuttcap%
\pgfsetroundjoin%
\definecolor{currentfill}{rgb}{0.121569,0.466667,0.705882}%
\pgfsetfillcolor{currentfill}%
\pgfsetfillopacity{0.302289}%
\pgfsetlinewidth{1.003750pt}%
\definecolor{currentstroke}{rgb}{0.121569,0.466667,0.705882}%
\pgfsetstrokecolor{currentstroke}%
\pgfsetstrokeopacity{0.302289}%
\pgfsetdash{}{0pt}%
\pgfpathmoveto{\pgfqpoint{1.861711in}{3.338150in}}%
\pgfpathcurveto{\pgfqpoint{1.869947in}{3.338150in}}{\pgfqpoint{1.877847in}{3.341422in}}{\pgfqpoint{1.883671in}{3.347246in}}%
\pgfpathcurveto{\pgfqpoint{1.889495in}{3.353070in}}{\pgfqpoint{1.892768in}{3.360970in}}{\pgfqpoint{1.892768in}{3.369206in}}%
\pgfpathcurveto{\pgfqpoint{1.892768in}{3.377443in}}{\pgfqpoint{1.889495in}{3.385343in}}{\pgfqpoint{1.883671in}{3.391167in}}%
\pgfpathcurveto{\pgfqpoint{1.877847in}{3.396990in}}{\pgfqpoint{1.869947in}{3.400263in}}{\pgfqpoint{1.861711in}{3.400263in}}%
\pgfpathcurveto{\pgfqpoint{1.853475in}{3.400263in}}{\pgfqpoint{1.845575in}{3.396990in}}{\pgfqpoint{1.839751in}{3.391167in}}%
\pgfpathcurveto{\pgfqpoint{1.833927in}{3.385343in}}{\pgfqpoint{1.830655in}{3.377443in}}{\pgfqpoint{1.830655in}{3.369206in}}%
\pgfpathcurveto{\pgfqpoint{1.830655in}{3.360970in}}{\pgfqpoint{1.833927in}{3.353070in}}{\pgfqpoint{1.839751in}{3.347246in}}%
\pgfpathcurveto{\pgfqpoint{1.845575in}{3.341422in}}{\pgfqpoint{1.853475in}{3.338150in}}{\pgfqpoint{1.861711in}{3.338150in}}%
\pgfpathclose%
\pgfusepath{stroke,fill}%
\end{pgfscope}%
\begin{pgfscope}%
\pgfpathrectangle{\pgfqpoint{0.100000in}{0.212622in}}{\pgfqpoint{3.696000in}{3.696000in}}%
\pgfusepath{clip}%
\pgfsetbuttcap%
\pgfsetroundjoin%
\definecolor{currentfill}{rgb}{0.121569,0.466667,0.705882}%
\pgfsetfillcolor{currentfill}%
\pgfsetfillopacity{0.302290}%
\pgfsetlinewidth{1.003750pt}%
\definecolor{currentstroke}{rgb}{0.121569,0.466667,0.705882}%
\pgfsetstrokecolor{currentstroke}%
\pgfsetstrokeopacity{0.302290}%
\pgfsetdash{}{0pt}%
\pgfpathmoveto{\pgfqpoint{1.861710in}{3.338148in}}%
\pgfpathcurveto{\pgfqpoint{1.869946in}{3.338148in}}{\pgfqpoint{1.877847in}{3.341421in}}{\pgfqpoint{1.883670in}{3.347245in}}%
\pgfpathcurveto{\pgfqpoint{1.889494in}{3.353068in}}{\pgfqpoint{1.892767in}{3.360968in}}{\pgfqpoint{1.892767in}{3.369205in}}%
\pgfpathcurveto{\pgfqpoint{1.892767in}{3.377441in}}{\pgfqpoint{1.889494in}{3.385341in}}{\pgfqpoint{1.883670in}{3.391165in}}%
\pgfpathcurveto{\pgfqpoint{1.877847in}{3.396989in}}{\pgfqpoint{1.869946in}{3.400261in}}{\pgfqpoint{1.861710in}{3.400261in}}%
\pgfpathcurveto{\pgfqpoint{1.853474in}{3.400261in}}{\pgfqpoint{1.845574in}{3.396989in}}{\pgfqpoint{1.839750in}{3.391165in}}%
\pgfpathcurveto{\pgfqpoint{1.833926in}{3.385341in}}{\pgfqpoint{1.830654in}{3.377441in}}{\pgfqpoint{1.830654in}{3.369205in}}%
\pgfpathcurveto{\pgfqpoint{1.830654in}{3.360968in}}{\pgfqpoint{1.833926in}{3.353068in}}{\pgfqpoint{1.839750in}{3.347245in}}%
\pgfpathcurveto{\pgfqpoint{1.845574in}{3.341421in}}{\pgfqpoint{1.853474in}{3.338148in}}{\pgfqpoint{1.861710in}{3.338148in}}%
\pgfpathclose%
\pgfusepath{stroke,fill}%
\end{pgfscope}%
\begin{pgfscope}%
\pgfpathrectangle{\pgfqpoint{0.100000in}{0.212622in}}{\pgfqpoint{3.696000in}{3.696000in}}%
\pgfusepath{clip}%
\pgfsetbuttcap%
\pgfsetroundjoin%
\definecolor{currentfill}{rgb}{0.121569,0.466667,0.705882}%
\pgfsetfillcolor{currentfill}%
\pgfsetfillopacity{0.302290}%
\pgfsetlinewidth{1.003750pt}%
\definecolor{currentstroke}{rgb}{0.121569,0.466667,0.705882}%
\pgfsetstrokecolor{currentstroke}%
\pgfsetstrokeopacity{0.302290}%
\pgfsetdash{}{0pt}%
\pgfpathmoveto{\pgfqpoint{1.861709in}{3.338145in}}%
\pgfpathcurveto{\pgfqpoint{1.869945in}{3.338145in}}{\pgfqpoint{1.877845in}{3.341418in}}{\pgfqpoint{1.883669in}{3.347242in}}%
\pgfpathcurveto{\pgfqpoint{1.889493in}{3.353066in}}{\pgfqpoint{1.892765in}{3.360966in}}{\pgfqpoint{1.892765in}{3.369202in}}%
\pgfpathcurveto{\pgfqpoint{1.892765in}{3.377438in}}{\pgfqpoint{1.889493in}{3.385338in}}{\pgfqpoint{1.883669in}{3.391162in}}%
\pgfpathcurveto{\pgfqpoint{1.877845in}{3.396986in}}{\pgfqpoint{1.869945in}{3.400258in}}{\pgfqpoint{1.861709in}{3.400258in}}%
\pgfpathcurveto{\pgfqpoint{1.853472in}{3.400258in}}{\pgfqpoint{1.845572in}{3.396986in}}{\pgfqpoint{1.839748in}{3.391162in}}%
\pgfpathcurveto{\pgfqpoint{1.833924in}{3.385338in}}{\pgfqpoint{1.830652in}{3.377438in}}{\pgfqpoint{1.830652in}{3.369202in}}%
\pgfpathcurveto{\pgfqpoint{1.830652in}{3.360966in}}{\pgfqpoint{1.833924in}{3.353066in}}{\pgfqpoint{1.839748in}{3.347242in}}%
\pgfpathcurveto{\pgfqpoint{1.845572in}{3.341418in}}{\pgfqpoint{1.853472in}{3.338145in}}{\pgfqpoint{1.861709in}{3.338145in}}%
\pgfpathclose%
\pgfusepath{stroke,fill}%
\end{pgfscope}%
\begin{pgfscope}%
\pgfpathrectangle{\pgfqpoint{0.100000in}{0.212622in}}{\pgfqpoint{3.696000in}{3.696000in}}%
\pgfusepath{clip}%
\pgfsetbuttcap%
\pgfsetroundjoin%
\definecolor{currentfill}{rgb}{0.121569,0.466667,0.705882}%
\pgfsetfillcolor{currentfill}%
\pgfsetfillopacity{0.302291}%
\pgfsetlinewidth{1.003750pt}%
\definecolor{currentstroke}{rgb}{0.121569,0.466667,0.705882}%
\pgfsetstrokecolor{currentstroke}%
\pgfsetstrokeopacity{0.302291}%
\pgfsetdash{}{0pt}%
\pgfpathmoveto{\pgfqpoint{1.861706in}{3.338140in}}%
\pgfpathcurveto{\pgfqpoint{1.869942in}{3.338140in}}{\pgfqpoint{1.877842in}{3.341413in}}{\pgfqpoint{1.883666in}{3.347237in}}%
\pgfpathcurveto{\pgfqpoint{1.889490in}{3.353061in}}{\pgfqpoint{1.892762in}{3.360961in}}{\pgfqpoint{1.892762in}{3.369197in}}%
\pgfpathcurveto{\pgfqpoint{1.892762in}{3.377433in}}{\pgfqpoint{1.889490in}{3.385333in}}{\pgfqpoint{1.883666in}{3.391157in}}%
\pgfpathcurveto{\pgfqpoint{1.877842in}{3.396981in}}{\pgfqpoint{1.869942in}{3.400253in}}{\pgfqpoint{1.861706in}{3.400253in}}%
\pgfpathcurveto{\pgfqpoint{1.853469in}{3.400253in}}{\pgfqpoint{1.845569in}{3.396981in}}{\pgfqpoint{1.839745in}{3.391157in}}%
\pgfpathcurveto{\pgfqpoint{1.833921in}{3.385333in}}{\pgfqpoint{1.830649in}{3.377433in}}{\pgfqpoint{1.830649in}{3.369197in}}%
\pgfpathcurveto{\pgfqpoint{1.830649in}{3.360961in}}{\pgfqpoint{1.833921in}{3.353061in}}{\pgfqpoint{1.839745in}{3.347237in}}%
\pgfpathcurveto{\pgfqpoint{1.845569in}{3.341413in}}{\pgfqpoint{1.853469in}{3.338140in}}{\pgfqpoint{1.861706in}{3.338140in}}%
\pgfpathclose%
\pgfusepath{stroke,fill}%
\end{pgfscope}%
\begin{pgfscope}%
\pgfpathrectangle{\pgfqpoint{0.100000in}{0.212622in}}{\pgfqpoint{3.696000in}{3.696000in}}%
\pgfusepath{clip}%
\pgfsetbuttcap%
\pgfsetroundjoin%
\definecolor{currentfill}{rgb}{0.121569,0.466667,0.705882}%
\pgfsetfillcolor{currentfill}%
\pgfsetfillopacity{0.302293}%
\pgfsetlinewidth{1.003750pt}%
\definecolor{currentstroke}{rgb}{0.121569,0.466667,0.705882}%
\pgfsetstrokecolor{currentstroke}%
\pgfsetstrokeopacity{0.302293}%
\pgfsetdash{}{0pt}%
\pgfpathmoveto{\pgfqpoint{1.861700in}{3.338131in}}%
\pgfpathcurveto{\pgfqpoint{1.869937in}{3.338131in}}{\pgfqpoint{1.877837in}{3.341404in}}{\pgfqpoint{1.883661in}{3.347228in}}%
\pgfpathcurveto{\pgfqpoint{1.889484in}{3.353051in}}{\pgfqpoint{1.892757in}{3.360951in}}{\pgfqpoint{1.892757in}{3.369188in}}%
\pgfpathcurveto{\pgfqpoint{1.892757in}{3.377424in}}{\pgfqpoint{1.889484in}{3.385324in}}{\pgfqpoint{1.883661in}{3.391148in}}%
\pgfpathcurveto{\pgfqpoint{1.877837in}{3.396972in}}{\pgfqpoint{1.869937in}{3.400244in}}{\pgfqpoint{1.861700in}{3.400244in}}%
\pgfpathcurveto{\pgfqpoint{1.853464in}{3.400244in}}{\pgfqpoint{1.845564in}{3.396972in}}{\pgfqpoint{1.839740in}{3.391148in}}%
\pgfpathcurveto{\pgfqpoint{1.833916in}{3.385324in}}{\pgfqpoint{1.830644in}{3.377424in}}{\pgfqpoint{1.830644in}{3.369188in}}%
\pgfpathcurveto{\pgfqpoint{1.830644in}{3.360951in}}{\pgfqpoint{1.833916in}{3.353051in}}{\pgfqpoint{1.839740in}{3.347228in}}%
\pgfpathcurveto{\pgfqpoint{1.845564in}{3.341404in}}{\pgfqpoint{1.853464in}{3.338131in}}{\pgfqpoint{1.861700in}{3.338131in}}%
\pgfpathclose%
\pgfusepath{stroke,fill}%
\end{pgfscope}%
\begin{pgfscope}%
\pgfpathrectangle{\pgfqpoint{0.100000in}{0.212622in}}{\pgfqpoint{3.696000in}{3.696000in}}%
\pgfusepath{clip}%
\pgfsetbuttcap%
\pgfsetroundjoin%
\definecolor{currentfill}{rgb}{0.121569,0.466667,0.705882}%
\pgfsetfillcolor{currentfill}%
\pgfsetfillopacity{0.302296}%
\pgfsetlinewidth{1.003750pt}%
\definecolor{currentstroke}{rgb}{0.121569,0.466667,0.705882}%
\pgfsetstrokecolor{currentstroke}%
\pgfsetstrokeopacity{0.302296}%
\pgfsetdash{}{0pt}%
\pgfpathmoveto{\pgfqpoint{1.861691in}{3.338115in}}%
\pgfpathcurveto{\pgfqpoint{1.869927in}{3.338115in}}{\pgfqpoint{1.877827in}{3.341387in}}{\pgfqpoint{1.883651in}{3.347211in}}%
\pgfpathcurveto{\pgfqpoint{1.889475in}{3.353035in}}{\pgfqpoint{1.892747in}{3.360935in}}{\pgfqpoint{1.892747in}{3.369171in}}%
\pgfpathcurveto{\pgfqpoint{1.892747in}{3.377408in}}{\pgfqpoint{1.889475in}{3.385308in}}{\pgfqpoint{1.883651in}{3.391132in}}%
\pgfpathcurveto{\pgfqpoint{1.877827in}{3.396956in}}{\pgfqpoint{1.869927in}{3.400228in}}{\pgfqpoint{1.861691in}{3.400228in}}%
\pgfpathcurveto{\pgfqpoint{1.853454in}{3.400228in}}{\pgfqpoint{1.845554in}{3.396956in}}{\pgfqpoint{1.839730in}{3.391132in}}%
\pgfpathcurveto{\pgfqpoint{1.833906in}{3.385308in}}{\pgfqpoint{1.830634in}{3.377408in}}{\pgfqpoint{1.830634in}{3.369171in}}%
\pgfpathcurveto{\pgfqpoint{1.830634in}{3.360935in}}{\pgfqpoint{1.833906in}{3.353035in}}{\pgfqpoint{1.839730in}{3.347211in}}%
\pgfpathcurveto{\pgfqpoint{1.845554in}{3.341387in}}{\pgfqpoint{1.853454in}{3.338115in}}{\pgfqpoint{1.861691in}{3.338115in}}%
\pgfpathclose%
\pgfusepath{stroke,fill}%
\end{pgfscope}%
\begin{pgfscope}%
\pgfpathrectangle{\pgfqpoint{0.100000in}{0.212622in}}{\pgfqpoint{3.696000in}{3.696000in}}%
\pgfusepath{clip}%
\pgfsetbuttcap%
\pgfsetroundjoin%
\definecolor{currentfill}{rgb}{0.121569,0.466667,0.705882}%
\pgfsetfillcolor{currentfill}%
\pgfsetfillopacity{0.302301}%
\pgfsetlinewidth{1.003750pt}%
\definecolor{currentstroke}{rgb}{0.121569,0.466667,0.705882}%
\pgfsetstrokecolor{currentstroke}%
\pgfsetstrokeopacity{0.302301}%
\pgfsetdash{}{0pt}%
\pgfpathmoveto{\pgfqpoint{1.861674in}{3.338085in}}%
\pgfpathcurveto{\pgfqpoint{1.869910in}{3.338085in}}{\pgfqpoint{1.877810in}{3.341357in}}{\pgfqpoint{1.883634in}{3.347181in}}%
\pgfpathcurveto{\pgfqpoint{1.889458in}{3.353005in}}{\pgfqpoint{1.892730in}{3.360905in}}{\pgfqpoint{1.892730in}{3.369141in}}%
\pgfpathcurveto{\pgfqpoint{1.892730in}{3.377378in}}{\pgfqpoint{1.889458in}{3.385278in}}{\pgfqpoint{1.883634in}{3.391102in}}%
\pgfpathcurveto{\pgfqpoint{1.877810in}{3.396926in}}{\pgfqpoint{1.869910in}{3.400198in}}{\pgfqpoint{1.861674in}{3.400198in}}%
\pgfpathcurveto{\pgfqpoint{1.853437in}{3.400198in}}{\pgfqpoint{1.845537in}{3.396926in}}{\pgfqpoint{1.839713in}{3.391102in}}%
\pgfpathcurveto{\pgfqpoint{1.833889in}{3.385278in}}{\pgfqpoint{1.830617in}{3.377378in}}{\pgfqpoint{1.830617in}{3.369141in}}%
\pgfpathcurveto{\pgfqpoint{1.830617in}{3.360905in}}{\pgfqpoint{1.833889in}{3.353005in}}{\pgfqpoint{1.839713in}{3.347181in}}%
\pgfpathcurveto{\pgfqpoint{1.845537in}{3.341357in}}{\pgfqpoint{1.853437in}{3.338085in}}{\pgfqpoint{1.861674in}{3.338085in}}%
\pgfpathclose%
\pgfusepath{stroke,fill}%
\end{pgfscope}%
\begin{pgfscope}%
\pgfpathrectangle{\pgfqpoint{0.100000in}{0.212622in}}{\pgfqpoint{3.696000in}{3.696000in}}%
\pgfusepath{clip}%
\pgfsetbuttcap%
\pgfsetroundjoin%
\definecolor{currentfill}{rgb}{0.121569,0.466667,0.705882}%
\pgfsetfillcolor{currentfill}%
\pgfsetfillopacity{0.302312}%
\pgfsetlinewidth{1.003750pt}%
\definecolor{currentstroke}{rgb}{0.121569,0.466667,0.705882}%
\pgfsetstrokecolor{currentstroke}%
\pgfsetstrokeopacity{0.302312}%
\pgfsetdash{}{0pt}%
\pgfpathmoveto{\pgfqpoint{1.861643in}{3.338031in}}%
\pgfpathcurveto{\pgfqpoint{1.869879in}{3.338031in}}{\pgfqpoint{1.877779in}{3.341303in}}{\pgfqpoint{1.883603in}{3.347127in}}%
\pgfpathcurveto{\pgfqpoint{1.889427in}{3.352951in}}{\pgfqpoint{1.892699in}{3.360851in}}{\pgfqpoint{1.892699in}{3.369087in}}%
\pgfpathcurveto{\pgfqpoint{1.892699in}{3.377324in}}{\pgfqpoint{1.889427in}{3.385224in}}{\pgfqpoint{1.883603in}{3.391048in}}%
\pgfpathcurveto{\pgfqpoint{1.877779in}{3.396872in}}{\pgfqpoint{1.869879in}{3.400144in}}{\pgfqpoint{1.861643in}{3.400144in}}%
\pgfpathcurveto{\pgfqpoint{1.853406in}{3.400144in}}{\pgfqpoint{1.845506in}{3.396872in}}{\pgfqpoint{1.839682in}{3.391048in}}%
\pgfpathcurveto{\pgfqpoint{1.833858in}{3.385224in}}{\pgfqpoint{1.830586in}{3.377324in}}{\pgfqpoint{1.830586in}{3.369087in}}%
\pgfpathcurveto{\pgfqpoint{1.830586in}{3.360851in}}{\pgfqpoint{1.833858in}{3.352951in}}{\pgfqpoint{1.839682in}{3.347127in}}%
\pgfpathcurveto{\pgfqpoint{1.845506in}{3.341303in}}{\pgfqpoint{1.853406in}{3.338031in}}{\pgfqpoint{1.861643in}{3.338031in}}%
\pgfpathclose%
\pgfusepath{stroke,fill}%
\end{pgfscope}%
\begin{pgfscope}%
\pgfpathrectangle{\pgfqpoint{0.100000in}{0.212622in}}{\pgfqpoint{3.696000in}{3.696000in}}%
\pgfusepath{clip}%
\pgfsetbuttcap%
\pgfsetroundjoin%
\definecolor{currentfill}{rgb}{0.121569,0.466667,0.705882}%
\pgfsetfillcolor{currentfill}%
\pgfsetfillopacity{0.302331}%
\pgfsetlinewidth{1.003750pt}%
\definecolor{currentstroke}{rgb}{0.121569,0.466667,0.705882}%
\pgfsetstrokecolor{currentstroke}%
\pgfsetstrokeopacity{0.302331}%
\pgfsetdash{}{0pt}%
\pgfpathmoveto{\pgfqpoint{1.861586in}{3.337933in}}%
\pgfpathcurveto{\pgfqpoint{1.869822in}{3.337933in}}{\pgfqpoint{1.877722in}{3.341205in}}{\pgfqpoint{1.883546in}{3.347029in}}%
\pgfpathcurveto{\pgfqpoint{1.889370in}{3.352853in}}{\pgfqpoint{1.892643in}{3.360753in}}{\pgfqpoint{1.892643in}{3.368989in}}%
\pgfpathcurveto{\pgfqpoint{1.892643in}{3.377226in}}{\pgfqpoint{1.889370in}{3.385126in}}{\pgfqpoint{1.883546in}{3.390950in}}%
\pgfpathcurveto{\pgfqpoint{1.877722in}{3.396774in}}{\pgfqpoint{1.869822in}{3.400046in}}{\pgfqpoint{1.861586in}{3.400046in}}%
\pgfpathcurveto{\pgfqpoint{1.853350in}{3.400046in}}{\pgfqpoint{1.845450in}{3.396774in}}{\pgfqpoint{1.839626in}{3.390950in}}%
\pgfpathcurveto{\pgfqpoint{1.833802in}{3.385126in}}{\pgfqpoint{1.830530in}{3.377226in}}{\pgfqpoint{1.830530in}{3.368989in}}%
\pgfpathcurveto{\pgfqpoint{1.830530in}{3.360753in}}{\pgfqpoint{1.833802in}{3.352853in}}{\pgfqpoint{1.839626in}{3.347029in}}%
\pgfpathcurveto{\pgfqpoint{1.845450in}{3.341205in}}{\pgfqpoint{1.853350in}{3.337933in}}{\pgfqpoint{1.861586in}{3.337933in}}%
\pgfpathclose%
\pgfusepath{stroke,fill}%
\end{pgfscope}%
\begin{pgfscope}%
\pgfpathrectangle{\pgfqpoint{0.100000in}{0.212622in}}{\pgfqpoint{3.696000in}{3.696000in}}%
\pgfusepath{clip}%
\pgfsetbuttcap%
\pgfsetroundjoin%
\definecolor{currentfill}{rgb}{0.121569,0.466667,0.705882}%
\pgfsetfillcolor{currentfill}%
\pgfsetfillopacity{0.302366}%
\pgfsetlinewidth{1.003750pt}%
\definecolor{currentstroke}{rgb}{0.121569,0.466667,0.705882}%
\pgfsetstrokecolor{currentstroke}%
\pgfsetstrokeopacity{0.302366}%
\pgfsetdash{}{0pt}%
\pgfpathmoveto{\pgfqpoint{1.861484in}{3.337755in}}%
\pgfpathcurveto{\pgfqpoint{1.869720in}{3.337755in}}{\pgfqpoint{1.877621in}{3.341028in}}{\pgfqpoint{1.883444in}{3.346852in}}%
\pgfpathcurveto{\pgfqpoint{1.889268in}{3.352676in}}{\pgfqpoint{1.892541in}{3.360576in}}{\pgfqpoint{1.892541in}{3.368812in}}%
\pgfpathcurveto{\pgfqpoint{1.892541in}{3.377048in}}{\pgfqpoint{1.889268in}{3.384948in}}{\pgfqpoint{1.883444in}{3.390772in}}%
\pgfpathcurveto{\pgfqpoint{1.877621in}{3.396596in}}{\pgfqpoint{1.869720in}{3.399868in}}{\pgfqpoint{1.861484in}{3.399868in}}%
\pgfpathcurveto{\pgfqpoint{1.853248in}{3.399868in}}{\pgfqpoint{1.845348in}{3.396596in}}{\pgfqpoint{1.839524in}{3.390772in}}%
\pgfpathcurveto{\pgfqpoint{1.833700in}{3.384948in}}{\pgfqpoint{1.830428in}{3.377048in}}{\pgfqpoint{1.830428in}{3.368812in}}%
\pgfpathcurveto{\pgfqpoint{1.830428in}{3.360576in}}{\pgfqpoint{1.833700in}{3.352676in}}{\pgfqpoint{1.839524in}{3.346852in}}%
\pgfpathcurveto{\pgfqpoint{1.845348in}{3.341028in}}{\pgfqpoint{1.853248in}{3.337755in}}{\pgfqpoint{1.861484in}{3.337755in}}%
\pgfpathclose%
\pgfusepath{stroke,fill}%
\end{pgfscope}%
\begin{pgfscope}%
\pgfpathrectangle{\pgfqpoint{0.100000in}{0.212622in}}{\pgfqpoint{3.696000in}{3.696000in}}%
\pgfusepath{clip}%
\pgfsetbuttcap%
\pgfsetroundjoin%
\definecolor{currentfill}{rgb}{0.121569,0.466667,0.705882}%
\pgfsetfillcolor{currentfill}%
\pgfsetfillopacity{0.302429}%
\pgfsetlinewidth{1.003750pt}%
\definecolor{currentstroke}{rgb}{0.121569,0.466667,0.705882}%
\pgfsetstrokecolor{currentstroke}%
\pgfsetstrokeopacity{0.302429}%
\pgfsetdash{}{0pt}%
\pgfpathmoveto{\pgfqpoint{1.861301in}{3.337430in}}%
\pgfpathcurveto{\pgfqpoint{1.869537in}{3.337430in}}{\pgfqpoint{1.877437in}{3.340702in}}{\pgfqpoint{1.883261in}{3.346526in}}%
\pgfpathcurveto{\pgfqpoint{1.889085in}{3.352350in}}{\pgfqpoint{1.892357in}{3.360250in}}{\pgfqpoint{1.892357in}{3.368487in}}%
\pgfpathcurveto{\pgfqpoint{1.892357in}{3.376723in}}{\pgfqpoint{1.889085in}{3.384623in}}{\pgfqpoint{1.883261in}{3.390447in}}%
\pgfpathcurveto{\pgfqpoint{1.877437in}{3.396271in}}{\pgfqpoint{1.869537in}{3.399543in}}{\pgfqpoint{1.861301in}{3.399543in}}%
\pgfpathcurveto{\pgfqpoint{1.853064in}{3.399543in}}{\pgfqpoint{1.845164in}{3.396271in}}{\pgfqpoint{1.839340in}{3.390447in}}%
\pgfpathcurveto{\pgfqpoint{1.833516in}{3.384623in}}{\pgfqpoint{1.830244in}{3.376723in}}{\pgfqpoint{1.830244in}{3.368487in}}%
\pgfpathcurveto{\pgfqpoint{1.830244in}{3.360250in}}{\pgfqpoint{1.833516in}{3.352350in}}{\pgfqpoint{1.839340in}{3.346526in}}%
\pgfpathcurveto{\pgfqpoint{1.845164in}{3.340702in}}{\pgfqpoint{1.853064in}{3.337430in}}{\pgfqpoint{1.861301in}{3.337430in}}%
\pgfpathclose%
\pgfusepath{stroke,fill}%
\end{pgfscope}%
\begin{pgfscope}%
\pgfpathrectangle{\pgfqpoint{0.100000in}{0.212622in}}{\pgfqpoint{3.696000in}{3.696000in}}%
\pgfusepath{clip}%
\pgfsetbuttcap%
\pgfsetroundjoin%
\definecolor{currentfill}{rgb}{0.121569,0.466667,0.705882}%
\pgfsetfillcolor{currentfill}%
\pgfsetfillopacity{0.302531}%
\pgfsetlinewidth{1.003750pt}%
\definecolor{currentstroke}{rgb}{0.121569,0.466667,0.705882}%
\pgfsetstrokecolor{currentstroke}%
\pgfsetstrokeopacity{0.302531}%
\pgfsetdash{}{0pt}%
\pgfpathmoveto{\pgfqpoint{1.893693in}{3.319069in}}%
\pgfpathcurveto{\pgfqpoint{1.901929in}{3.319069in}}{\pgfqpoint{1.909829in}{3.322341in}}{\pgfqpoint{1.915653in}{3.328165in}}%
\pgfpathcurveto{\pgfqpoint{1.921477in}{3.333989in}}{\pgfqpoint{1.924750in}{3.341889in}}{\pgfqpoint{1.924750in}{3.350125in}}%
\pgfpathcurveto{\pgfqpoint{1.924750in}{3.358361in}}{\pgfqpoint{1.921477in}{3.366261in}}{\pgfqpoint{1.915653in}{3.372085in}}%
\pgfpathcurveto{\pgfqpoint{1.909829in}{3.377909in}}{\pgfqpoint{1.901929in}{3.381182in}}{\pgfqpoint{1.893693in}{3.381182in}}%
\pgfpathcurveto{\pgfqpoint{1.885457in}{3.381182in}}{\pgfqpoint{1.877557in}{3.377909in}}{\pgfqpoint{1.871733in}{3.372085in}}%
\pgfpathcurveto{\pgfqpoint{1.865909in}{3.366261in}}{\pgfqpoint{1.862637in}{3.358361in}}{\pgfqpoint{1.862637in}{3.350125in}}%
\pgfpathcurveto{\pgfqpoint{1.862637in}{3.341889in}}{\pgfqpoint{1.865909in}{3.333989in}}{\pgfqpoint{1.871733in}{3.328165in}}%
\pgfpathcurveto{\pgfqpoint{1.877557in}{3.322341in}}{\pgfqpoint{1.885457in}{3.319069in}}{\pgfqpoint{1.893693in}{3.319069in}}%
\pgfpathclose%
\pgfusepath{stroke,fill}%
\end{pgfscope}%
\begin{pgfscope}%
\pgfpathrectangle{\pgfqpoint{0.100000in}{0.212622in}}{\pgfqpoint{3.696000in}{3.696000in}}%
\pgfusepath{clip}%
\pgfsetbuttcap%
\pgfsetroundjoin%
\definecolor{currentfill}{rgb}{0.121569,0.466667,0.705882}%
\pgfsetfillcolor{currentfill}%
\pgfsetfillopacity{0.302545}%
\pgfsetlinewidth{1.003750pt}%
\definecolor{currentstroke}{rgb}{0.121569,0.466667,0.705882}%
\pgfsetstrokecolor{currentstroke}%
\pgfsetstrokeopacity{0.302545}%
\pgfsetdash{}{0pt}%
\pgfpathmoveto{\pgfqpoint{1.860964in}{3.336843in}}%
\pgfpathcurveto{\pgfqpoint{1.869200in}{3.336843in}}{\pgfqpoint{1.877100in}{3.340115in}}{\pgfqpoint{1.882924in}{3.345939in}}%
\pgfpathcurveto{\pgfqpoint{1.888748in}{3.351763in}}{\pgfqpoint{1.892020in}{3.359663in}}{\pgfqpoint{1.892020in}{3.367899in}}%
\pgfpathcurveto{\pgfqpoint{1.892020in}{3.376136in}}{\pgfqpoint{1.888748in}{3.384036in}}{\pgfqpoint{1.882924in}{3.389860in}}%
\pgfpathcurveto{\pgfqpoint{1.877100in}{3.395683in}}{\pgfqpoint{1.869200in}{3.398956in}}{\pgfqpoint{1.860964in}{3.398956in}}%
\pgfpathcurveto{\pgfqpoint{1.852728in}{3.398956in}}{\pgfqpoint{1.844827in}{3.395683in}}{\pgfqpoint{1.839004in}{3.389860in}}%
\pgfpathcurveto{\pgfqpoint{1.833180in}{3.384036in}}{\pgfqpoint{1.829907in}{3.376136in}}{\pgfqpoint{1.829907in}{3.367899in}}%
\pgfpathcurveto{\pgfqpoint{1.829907in}{3.359663in}}{\pgfqpoint{1.833180in}{3.351763in}}{\pgfqpoint{1.839004in}{3.345939in}}%
\pgfpathcurveto{\pgfqpoint{1.844827in}{3.340115in}}{\pgfqpoint{1.852728in}{3.336843in}}{\pgfqpoint{1.860964in}{3.336843in}}%
\pgfpathclose%
\pgfusepath{stroke,fill}%
\end{pgfscope}%
\begin{pgfscope}%
\pgfpathrectangle{\pgfqpoint{0.100000in}{0.212622in}}{\pgfqpoint{3.696000in}{3.696000in}}%
\pgfusepath{clip}%
\pgfsetbuttcap%
\pgfsetroundjoin%
\definecolor{currentfill}{rgb}{0.121569,0.466667,0.705882}%
\pgfsetfillcolor{currentfill}%
\pgfsetfillopacity{0.302757}%
\pgfsetlinewidth{1.003750pt}%
\definecolor{currentstroke}{rgb}{0.121569,0.466667,0.705882}%
\pgfsetstrokecolor{currentstroke}%
\pgfsetstrokeopacity{0.302757}%
\pgfsetdash{}{0pt}%
\pgfpathmoveto{\pgfqpoint{1.860341in}{3.335793in}}%
\pgfpathcurveto{\pgfqpoint{1.868577in}{3.335793in}}{\pgfqpoint{1.876477in}{3.339065in}}{\pgfqpoint{1.882301in}{3.344889in}}%
\pgfpathcurveto{\pgfqpoint{1.888125in}{3.350713in}}{\pgfqpoint{1.891397in}{3.358613in}}{\pgfqpoint{1.891397in}{3.366850in}}%
\pgfpathcurveto{\pgfqpoint{1.891397in}{3.375086in}}{\pgfqpoint{1.888125in}{3.382986in}}{\pgfqpoint{1.882301in}{3.388810in}}%
\pgfpathcurveto{\pgfqpoint{1.876477in}{3.394634in}}{\pgfqpoint{1.868577in}{3.397906in}}{\pgfqpoint{1.860341in}{3.397906in}}%
\pgfpathcurveto{\pgfqpoint{1.852104in}{3.397906in}}{\pgfqpoint{1.844204in}{3.394634in}}{\pgfqpoint{1.838380in}{3.388810in}}%
\pgfpathcurveto{\pgfqpoint{1.832556in}{3.382986in}}{\pgfqpoint{1.829284in}{3.375086in}}{\pgfqpoint{1.829284in}{3.366850in}}%
\pgfpathcurveto{\pgfqpoint{1.829284in}{3.358613in}}{\pgfqpoint{1.832556in}{3.350713in}}{\pgfqpoint{1.838380in}{3.344889in}}%
\pgfpathcurveto{\pgfqpoint{1.844204in}{3.339065in}}{\pgfqpoint{1.852104in}{3.335793in}}{\pgfqpoint{1.860341in}{3.335793in}}%
\pgfpathclose%
\pgfusepath{stroke,fill}%
\end{pgfscope}%
\begin{pgfscope}%
\pgfpathrectangle{\pgfqpoint{0.100000in}{0.212622in}}{\pgfqpoint{3.696000in}{3.696000in}}%
\pgfusepath{clip}%
\pgfsetbuttcap%
\pgfsetroundjoin%
\definecolor{currentfill}{rgb}{0.121569,0.466667,0.705882}%
\pgfsetfillcolor{currentfill}%
\pgfsetfillopacity{0.302905}%
\pgfsetlinewidth{1.003750pt}%
\definecolor{currentstroke}{rgb}{0.121569,0.466667,0.705882}%
\pgfsetstrokecolor{currentstroke}%
\pgfsetstrokeopacity{0.302905}%
\pgfsetdash{}{0pt}%
\pgfpathmoveto{\pgfqpoint{1.859926in}{3.335093in}}%
\pgfpathcurveto{\pgfqpoint{1.868162in}{3.335093in}}{\pgfqpoint{1.876062in}{3.338365in}}{\pgfqpoint{1.881886in}{3.344189in}}%
\pgfpathcurveto{\pgfqpoint{1.887710in}{3.350013in}}{\pgfqpoint{1.890982in}{3.357913in}}{\pgfqpoint{1.890982in}{3.366149in}}%
\pgfpathcurveto{\pgfqpoint{1.890982in}{3.374386in}}{\pgfqpoint{1.887710in}{3.382286in}}{\pgfqpoint{1.881886in}{3.388110in}}%
\pgfpathcurveto{\pgfqpoint{1.876062in}{3.393934in}}{\pgfqpoint{1.868162in}{3.397206in}}{\pgfqpoint{1.859926in}{3.397206in}}%
\pgfpathcurveto{\pgfqpoint{1.851689in}{3.397206in}}{\pgfqpoint{1.843789in}{3.393934in}}{\pgfqpoint{1.837965in}{3.388110in}}%
\pgfpathcurveto{\pgfqpoint{1.832141in}{3.382286in}}{\pgfqpoint{1.828869in}{3.374386in}}{\pgfqpoint{1.828869in}{3.366149in}}%
\pgfpathcurveto{\pgfqpoint{1.828869in}{3.357913in}}{\pgfqpoint{1.832141in}{3.350013in}}{\pgfqpoint{1.837965in}{3.344189in}}%
\pgfpathcurveto{\pgfqpoint{1.843789in}{3.338365in}}{\pgfqpoint{1.851689in}{3.335093in}}{\pgfqpoint{1.859926in}{3.335093in}}%
\pgfpathclose%
\pgfusepath{stroke,fill}%
\end{pgfscope}%
\begin{pgfscope}%
\pgfpathrectangle{\pgfqpoint{0.100000in}{0.212622in}}{\pgfqpoint{3.696000in}{3.696000in}}%
\pgfusepath{clip}%
\pgfsetbuttcap%
\pgfsetroundjoin%
\definecolor{currentfill}{rgb}{0.121569,0.466667,0.705882}%
\pgfsetfillcolor{currentfill}%
\pgfsetfillopacity{0.302933}%
\pgfsetlinewidth{1.003750pt}%
\definecolor{currentstroke}{rgb}{0.121569,0.466667,0.705882}%
\pgfsetstrokecolor{currentstroke}%
\pgfsetstrokeopacity{0.302933}%
\pgfsetdash{}{0pt}%
\pgfpathmoveto{\pgfqpoint{1.895182in}{3.316337in}}%
\pgfpathcurveto{\pgfqpoint{1.903418in}{3.316337in}}{\pgfqpoint{1.911318in}{3.319609in}}{\pgfqpoint{1.917142in}{3.325433in}}%
\pgfpathcurveto{\pgfqpoint{1.922966in}{3.331257in}}{\pgfqpoint{1.926238in}{3.339157in}}{\pgfqpoint{1.926238in}{3.347394in}}%
\pgfpathcurveto{\pgfqpoint{1.926238in}{3.355630in}}{\pgfqpoint{1.922966in}{3.363530in}}{\pgfqpoint{1.917142in}{3.369354in}}%
\pgfpathcurveto{\pgfqpoint{1.911318in}{3.375178in}}{\pgfqpoint{1.903418in}{3.378450in}}{\pgfqpoint{1.895182in}{3.378450in}}%
\pgfpathcurveto{\pgfqpoint{1.886946in}{3.378450in}}{\pgfqpoint{1.879046in}{3.375178in}}{\pgfqpoint{1.873222in}{3.369354in}}%
\pgfpathcurveto{\pgfqpoint{1.867398in}{3.363530in}}{\pgfqpoint{1.864125in}{3.355630in}}{\pgfqpoint{1.864125in}{3.347394in}}%
\pgfpathcurveto{\pgfqpoint{1.864125in}{3.339157in}}{\pgfqpoint{1.867398in}{3.331257in}}{\pgfqpoint{1.873222in}{3.325433in}}%
\pgfpathcurveto{\pgfqpoint{1.879046in}{3.319609in}}{\pgfqpoint{1.886946in}{3.316337in}}{\pgfqpoint{1.895182in}{3.316337in}}%
\pgfpathclose%
\pgfusepath{stroke,fill}%
\end{pgfscope}%
\begin{pgfscope}%
\pgfpathrectangle{\pgfqpoint{0.100000in}{0.212622in}}{\pgfqpoint{3.696000in}{3.696000in}}%
\pgfusepath{clip}%
\pgfsetbuttcap%
\pgfsetroundjoin%
\definecolor{currentfill}{rgb}{0.121569,0.466667,0.705882}%
\pgfsetfillcolor{currentfill}%
\pgfsetfillopacity{0.303175}%
\pgfsetlinewidth{1.003750pt}%
\definecolor{currentstroke}{rgb}{0.121569,0.466667,0.705882}%
\pgfsetstrokecolor{currentstroke}%
\pgfsetstrokeopacity{0.303175}%
\pgfsetdash{}{0pt}%
\pgfpathmoveto{\pgfqpoint{1.859175in}{3.333811in}}%
\pgfpathcurveto{\pgfqpoint{1.867411in}{3.333811in}}{\pgfqpoint{1.875311in}{3.337083in}}{\pgfqpoint{1.881135in}{3.342907in}}%
\pgfpathcurveto{\pgfqpoint{1.886959in}{3.348731in}}{\pgfqpoint{1.890231in}{3.356631in}}{\pgfqpoint{1.890231in}{3.364867in}}%
\pgfpathcurveto{\pgfqpoint{1.890231in}{3.373103in}}{\pgfqpoint{1.886959in}{3.381003in}}{\pgfqpoint{1.881135in}{3.386827in}}%
\pgfpathcurveto{\pgfqpoint{1.875311in}{3.392651in}}{\pgfqpoint{1.867411in}{3.395924in}}{\pgfqpoint{1.859175in}{3.395924in}}%
\pgfpathcurveto{\pgfqpoint{1.850938in}{3.395924in}}{\pgfqpoint{1.843038in}{3.392651in}}{\pgfqpoint{1.837214in}{3.386827in}}%
\pgfpathcurveto{\pgfqpoint{1.831390in}{3.381003in}}{\pgfqpoint{1.828118in}{3.373103in}}{\pgfqpoint{1.828118in}{3.364867in}}%
\pgfpathcurveto{\pgfqpoint{1.828118in}{3.356631in}}{\pgfqpoint{1.831390in}{3.348731in}}{\pgfqpoint{1.837214in}{3.342907in}}%
\pgfpathcurveto{\pgfqpoint{1.843038in}{3.337083in}}{\pgfqpoint{1.850938in}{3.333811in}}{\pgfqpoint{1.859175in}{3.333811in}}%
\pgfpathclose%
\pgfusepath{stroke,fill}%
\end{pgfscope}%
\begin{pgfscope}%
\pgfpathrectangle{\pgfqpoint{0.100000in}{0.212622in}}{\pgfqpoint{3.696000in}{3.696000in}}%
\pgfusepath{clip}%
\pgfsetbuttcap%
\pgfsetroundjoin%
\definecolor{currentfill}{rgb}{0.121569,0.466667,0.705882}%
\pgfsetfillcolor{currentfill}%
\pgfsetfillopacity{0.303388}%
\pgfsetlinewidth{1.003750pt}%
\definecolor{currentstroke}{rgb}{0.121569,0.466667,0.705882}%
\pgfsetstrokecolor{currentstroke}%
\pgfsetstrokeopacity{0.303388}%
\pgfsetdash{}{0pt}%
\pgfpathmoveto{\pgfqpoint{1.896786in}{3.312867in}}%
\pgfpathcurveto{\pgfqpoint{1.905023in}{3.312867in}}{\pgfqpoint{1.912923in}{3.316139in}}{\pgfqpoint{1.918747in}{3.321963in}}%
\pgfpathcurveto{\pgfqpoint{1.924571in}{3.327787in}}{\pgfqpoint{1.927843in}{3.335687in}}{\pgfqpoint{1.927843in}{3.343923in}}%
\pgfpathcurveto{\pgfqpoint{1.927843in}{3.352160in}}{\pgfqpoint{1.924571in}{3.360060in}}{\pgfqpoint{1.918747in}{3.365884in}}%
\pgfpathcurveto{\pgfqpoint{1.912923in}{3.371708in}}{\pgfqpoint{1.905023in}{3.374980in}}{\pgfqpoint{1.896786in}{3.374980in}}%
\pgfpathcurveto{\pgfqpoint{1.888550in}{3.374980in}}{\pgfqpoint{1.880650in}{3.371708in}}{\pgfqpoint{1.874826in}{3.365884in}}%
\pgfpathcurveto{\pgfqpoint{1.869002in}{3.360060in}}{\pgfqpoint{1.865730in}{3.352160in}}{\pgfqpoint{1.865730in}{3.343923in}}%
\pgfpathcurveto{\pgfqpoint{1.865730in}{3.335687in}}{\pgfqpoint{1.869002in}{3.327787in}}{\pgfqpoint{1.874826in}{3.321963in}}%
\pgfpathcurveto{\pgfqpoint{1.880650in}{3.316139in}}{\pgfqpoint{1.888550in}{3.312867in}}{\pgfqpoint{1.896786in}{3.312867in}}%
\pgfpathclose%
\pgfusepath{stroke,fill}%
\end{pgfscope}%
\begin{pgfscope}%
\pgfpathrectangle{\pgfqpoint{0.100000in}{0.212622in}}{\pgfqpoint{3.696000in}{3.696000in}}%
\pgfusepath{clip}%
\pgfsetbuttcap%
\pgfsetroundjoin%
\definecolor{currentfill}{rgb}{0.121569,0.466667,0.705882}%
\pgfsetfillcolor{currentfill}%
\pgfsetfillopacity{0.303637}%
\pgfsetlinewidth{1.003750pt}%
\definecolor{currentstroke}{rgb}{0.121569,0.466667,0.705882}%
\pgfsetstrokecolor{currentstroke}%
\pgfsetstrokeopacity{0.303637}%
\pgfsetdash{}{0pt}%
\pgfpathmoveto{\pgfqpoint{1.857719in}{3.331495in}}%
\pgfpathcurveto{\pgfqpoint{1.865955in}{3.331495in}}{\pgfqpoint{1.873855in}{3.334767in}}{\pgfqpoint{1.879679in}{3.340591in}}%
\pgfpathcurveto{\pgfqpoint{1.885503in}{3.346415in}}{\pgfqpoint{1.888775in}{3.354315in}}{\pgfqpoint{1.888775in}{3.362552in}}%
\pgfpathcurveto{\pgfqpoint{1.888775in}{3.370788in}}{\pgfqpoint{1.885503in}{3.378688in}}{\pgfqpoint{1.879679in}{3.384512in}}%
\pgfpathcurveto{\pgfqpoint{1.873855in}{3.390336in}}{\pgfqpoint{1.865955in}{3.393608in}}{\pgfqpoint{1.857719in}{3.393608in}}%
\pgfpathcurveto{\pgfqpoint{1.849483in}{3.393608in}}{\pgfqpoint{1.841583in}{3.390336in}}{\pgfqpoint{1.835759in}{3.384512in}}%
\pgfpathcurveto{\pgfqpoint{1.829935in}{3.378688in}}{\pgfqpoint{1.826662in}{3.370788in}}{\pgfqpoint{1.826662in}{3.362552in}}%
\pgfpathcurveto{\pgfqpoint{1.826662in}{3.354315in}}{\pgfqpoint{1.829935in}{3.346415in}}{\pgfqpoint{1.835759in}{3.340591in}}%
\pgfpathcurveto{\pgfqpoint{1.841583in}{3.334767in}}{\pgfqpoint{1.849483in}{3.331495in}}{\pgfqpoint{1.857719in}{3.331495in}}%
\pgfpathclose%
\pgfusepath{stroke,fill}%
\end{pgfscope}%
\begin{pgfscope}%
\pgfpathrectangle{\pgfqpoint{0.100000in}{0.212622in}}{\pgfqpoint{3.696000in}{3.696000in}}%
\pgfusepath{clip}%
\pgfsetbuttcap%
\pgfsetroundjoin%
\definecolor{currentfill}{rgb}{0.121569,0.466667,0.705882}%
\pgfsetfillcolor{currentfill}%
\pgfsetfillopacity{0.304160}%
\pgfsetlinewidth{1.003750pt}%
\definecolor{currentstroke}{rgb}{0.121569,0.466667,0.705882}%
\pgfsetstrokecolor{currentstroke}%
\pgfsetstrokeopacity{0.304160}%
\pgfsetdash{}{0pt}%
\pgfpathmoveto{\pgfqpoint{1.898439in}{3.309433in}}%
\pgfpathcurveto{\pgfqpoint{1.906676in}{3.309433in}}{\pgfqpoint{1.914576in}{3.312705in}}{\pgfqpoint{1.920400in}{3.318529in}}%
\pgfpathcurveto{\pgfqpoint{1.926223in}{3.324353in}}{\pgfqpoint{1.929496in}{3.332253in}}{\pgfqpoint{1.929496in}{3.340490in}}%
\pgfpathcurveto{\pgfqpoint{1.929496in}{3.348726in}}{\pgfqpoint{1.926223in}{3.356626in}}{\pgfqpoint{1.920400in}{3.362450in}}%
\pgfpathcurveto{\pgfqpoint{1.914576in}{3.368274in}}{\pgfqpoint{1.906676in}{3.371546in}}{\pgfqpoint{1.898439in}{3.371546in}}%
\pgfpathcurveto{\pgfqpoint{1.890203in}{3.371546in}}{\pgfqpoint{1.882303in}{3.368274in}}{\pgfqpoint{1.876479in}{3.362450in}}%
\pgfpathcurveto{\pgfqpoint{1.870655in}{3.356626in}}{\pgfqpoint{1.867383in}{3.348726in}}{\pgfqpoint{1.867383in}{3.340490in}}%
\pgfpathcurveto{\pgfqpoint{1.867383in}{3.332253in}}{\pgfqpoint{1.870655in}{3.324353in}}{\pgfqpoint{1.876479in}{3.318529in}}%
\pgfpathcurveto{\pgfqpoint{1.882303in}{3.312705in}}{\pgfqpoint{1.890203in}{3.309433in}}{\pgfqpoint{1.898439in}{3.309433in}}%
\pgfpathclose%
\pgfusepath{stroke,fill}%
\end{pgfscope}%
\begin{pgfscope}%
\pgfpathrectangle{\pgfqpoint{0.100000in}{0.212622in}}{\pgfqpoint{3.696000in}{3.696000in}}%
\pgfusepath{clip}%
\pgfsetbuttcap%
\pgfsetroundjoin%
\definecolor{currentfill}{rgb}{0.121569,0.466667,0.705882}%
\pgfsetfillcolor{currentfill}%
\pgfsetfillopacity{0.304477}%
\pgfsetlinewidth{1.003750pt}%
\definecolor{currentstroke}{rgb}{0.121569,0.466667,0.705882}%
\pgfsetstrokecolor{currentstroke}%
\pgfsetstrokeopacity{0.304477}%
\pgfsetdash{}{0pt}%
\pgfpathmoveto{\pgfqpoint{1.855165in}{3.327145in}}%
\pgfpathcurveto{\pgfqpoint{1.863401in}{3.327145in}}{\pgfqpoint{1.871302in}{3.330417in}}{\pgfqpoint{1.877125in}{3.336241in}}%
\pgfpathcurveto{\pgfqpoint{1.882949in}{3.342065in}}{\pgfqpoint{1.886222in}{3.349965in}}{\pgfqpoint{1.886222in}{3.358201in}}%
\pgfpathcurveto{\pgfqpoint{1.886222in}{3.366437in}}{\pgfqpoint{1.882949in}{3.374337in}}{\pgfqpoint{1.877125in}{3.380161in}}%
\pgfpathcurveto{\pgfqpoint{1.871302in}{3.385985in}}{\pgfqpoint{1.863401in}{3.389258in}}{\pgfqpoint{1.855165in}{3.389258in}}%
\pgfpathcurveto{\pgfqpoint{1.846929in}{3.389258in}}{\pgfqpoint{1.839029in}{3.385985in}}{\pgfqpoint{1.833205in}{3.380161in}}%
\pgfpathcurveto{\pgfqpoint{1.827381in}{3.374337in}}{\pgfqpoint{1.824109in}{3.366437in}}{\pgfqpoint{1.824109in}{3.358201in}}%
\pgfpathcurveto{\pgfqpoint{1.824109in}{3.349965in}}{\pgfqpoint{1.827381in}{3.342065in}}{\pgfqpoint{1.833205in}{3.336241in}}%
\pgfpathcurveto{\pgfqpoint{1.839029in}{3.330417in}}{\pgfqpoint{1.846929in}{3.327145in}}{\pgfqpoint{1.855165in}{3.327145in}}%
\pgfpathclose%
\pgfusepath{stroke,fill}%
\end{pgfscope}%
\begin{pgfscope}%
\pgfpathrectangle{\pgfqpoint{0.100000in}{0.212622in}}{\pgfqpoint{3.696000in}{3.696000in}}%
\pgfusepath{clip}%
\pgfsetbuttcap%
\pgfsetroundjoin%
\definecolor{currentfill}{rgb}{0.121569,0.466667,0.705882}%
\pgfsetfillcolor{currentfill}%
\pgfsetfillopacity{0.304871}%
\pgfsetlinewidth{1.003750pt}%
\definecolor{currentstroke}{rgb}{0.121569,0.466667,0.705882}%
\pgfsetstrokecolor{currentstroke}%
\pgfsetstrokeopacity{0.304871}%
\pgfsetdash{}{0pt}%
\pgfpathmoveto{\pgfqpoint{1.900914in}{3.303623in}}%
\pgfpathcurveto{\pgfqpoint{1.909151in}{3.303623in}}{\pgfqpoint{1.917051in}{3.306896in}}{\pgfqpoint{1.922875in}{3.312720in}}%
\pgfpathcurveto{\pgfqpoint{1.928699in}{3.318544in}}{\pgfqpoint{1.931971in}{3.326444in}}{\pgfqpoint{1.931971in}{3.334680in}}%
\pgfpathcurveto{\pgfqpoint{1.931971in}{3.342916in}}{\pgfqpoint{1.928699in}{3.350816in}}{\pgfqpoint{1.922875in}{3.356640in}}%
\pgfpathcurveto{\pgfqpoint{1.917051in}{3.362464in}}{\pgfqpoint{1.909151in}{3.365736in}}{\pgfqpoint{1.900914in}{3.365736in}}%
\pgfpathcurveto{\pgfqpoint{1.892678in}{3.365736in}}{\pgfqpoint{1.884778in}{3.362464in}}{\pgfqpoint{1.878954in}{3.356640in}}%
\pgfpathcurveto{\pgfqpoint{1.873130in}{3.350816in}}{\pgfqpoint{1.869858in}{3.342916in}}{\pgfqpoint{1.869858in}{3.334680in}}%
\pgfpathcurveto{\pgfqpoint{1.869858in}{3.326444in}}{\pgfqpoint{1.873130in}{3.318544in}}{\pgfqpoint{1.878954in}{3.312720in}}%
\pgfpathcurveto{\pgfqpoint{1.884778in}{3.306896in}}{\pgfqpoint{1.892678in}{3.303623in}}{\pgfqpoint{1.900914in}{3.303623in}}%
\pgfpathclose%
\pgfusepath{stroke,fill}%
\end{pgfscope}%
\begin{pgfscope}%
\pgfpathrectangle{\pgfqpoint{0.100000in}{0.212622in}}{\pgfqpoint{3.696000in}{3.696000in}}%
\pgfusepath{clip}%
\pgfsetbuttcap%
\pgfsetroundjoin%
\definecolor{currentfill}{rgb}{0.121569,0.466667,0.705882}%
\pgfsetfillcolor{currentfill}%
\pgfsetfillopacity{0.305277}%
\pgfsetlinewidth{1.003750pt}%
\definecolor{currentstroke}{rgb}{0.121569,0.466667,0.705882}%
\pgfsetstrokecolor{currentstroke}%
\pgfsetstrokeopacity{0.305277}%
\pgfsetdash{}{0pt}%
\pgfpathmoveto{\pgfqpoint{1.853075in}{3.322973in}}%
\pgfpathcurveto{\pgfqpoint{1.861311in}{3.322973in}}{\pgfqpoint{1.869211in}{3.326245in}}{\pgfqpoint{1.875035in}{3.332069in}}%
\pgfpathcurveto{\pgfqpoint{1.880859in}{3.337893in}}{\pgfqpoint{1.884132in}{3.345793in}}{\pgfqpoint{1.884132in}{3.354030in}}%
\pgfpathcurveto{\pgfqpoint{1.884132in}{3.362266in}}{\pgfqpoint{1.880859in}{3.370166in}}{\pgfqpoint{1.875035in}{3.375990in}}%
\pgfpathcurveto{\pgfqpoint{1.869211in}{3.381814in}}{\pgfqpoint{1.861311in}{3.385086in}}{\pgfqpoint{1.853075in}{3.385086in}}%
\pgfpathcurveto{\pgfqpoint{1.844839in}{3.385086in}}{\pgfqpoint{1.836939in}{3.381814in}}{\pgfqpoint{1.831115in}{3.375990in}}%
\pgfpathcurveto{\pgfqpoint{1.825291in}{3.370166in}}{\pgfqpoint{1.822019in}{3.362266in}}{\pgfqpoint{1.822019in}{3.354030in}}%
\pgfpathcurveto{\pgfqpoint{1.822019in}{3.345793in}}{\pgfqpoint{1.825291in}{3.337893in}}{\pgfqpoint{1.831115in}{3.332069in}}%
\pgfpathcurveto{\pgfqpoint{1.836939in}{3.326245in}}{\pgfqpoint{1.844839in}{3.322973in}}{\pgfqpoint{1.853075in}{3.322973in}}%
\pgfpathclose%
\pgfusepath{stroke,fill}%
\end{pgfscope}%
\begin{pgfscope}%
\pgfpathrectangle{\pgfqpoint{0.100000in}{0.212622in}}{\pgfqpoint{3.696000in}{3.696000in}}%
\pgfusepath{clip}%
\pgfsetbuttcap%
\pgfsetroundjoin%
\definecolor{currentfill}{rgb}{0.121569,0.466667,0.705882}%
\pgfsetfillcolor{currentfill}%
\pgfsetfillopacity{0.305581}%
\pgfsetlinewidth{1.003750pt}%
\definecolor{currentstroke}{rgb}{0.121569,0.466667,0.705882}%
\pgfsetstrokecolor{currentstroke}%
\pgfsetstrokeopacity{0.305581}%
\pgfsetdash{}{0pt}%
\pgfpathmoveto{\pgfqpoint{1.852067in}{3.321370in}}%
\pgfpathcurveto{\pgfqpoint{1.860304in}{3.321370in}}{\pgfqpoint{1.868204in}{3.324643in}}{\pgfqpoint{1.874028in}{3.330467in}}%
\pgfpathcurveto{\pgfqpoint{1.879852in}{3.336291in}}{\pgfqpoint{1.883124in}{3.344191in}}{\pgfqpoint{1.883124in}{3.352427in}}%
\pgfpathcurveto{\pgfqpoint{1.883124in}{3.360663in}}{\pgfqpoint{1.879852in}{3.368563in}}{\pgfqpoint{1.874028in}{3.374387in}}%
\pgfpathcurveto{\pgfqpoint{1.868204in}{3.380211in}}{\pgfqpoint{1.860304in}{3.383483in}}{\pgfqpoint{1.852067in}{3.383483in}}%
\pgfpathcurveto{\pgfqpoint{1.843831in}{3.383483in}}{\pgfqpoint{1.835931in}{3.380211in}}{\pgfqpoint{1.830107in}{3.374387in}}%
\pgfpathcurveto{\pgfqpoint{1.824283in}{3.368563in}}{\pgfqpoint{1.821011in}{3.360663in}}{\pgfqpoint{1.821011in}{3.352427in}}%
\pgfpathcurveto{\pgfqpoint{1.821011in}{3.344191in}}{\pgfqpoint{1.824283in}{3.336291in}}{\pgfqpoint{1.830107in}{3.330467in}}%
\pgfpathcurveto{\pgfqpoint{1.835931in}{3.324643in}}{\pgfqpoint{1.843831in}{3.321370in}}{\pgfqpoint{1.852067in}{3.321370in}}%
\pgfpathclose%
\pgfusepath{stroke,fill}%
\end{pgfscope}%
\begin{pgfscope}%
\pgfpathrectangle{\pgfqpoint{0.100000in}{0.212622in}}{\pgfqpoint{3.696000in}{3.696000in}}%
\pgfusepath{clip}%
\pgfsetbuttcap%
\pgfsetroundjoin%
\definecolor{currentfill}{rgb}{0.121569,0.466667,0.705882}%
\pgfsetfillcolor{currentfill}%
\pgfsetfillopacity{0.305789}%
\pgfsetlinewidth{1.003750pt}%
\definecolor{currentstroke}{rgb}{0.121569,0.466667,0.705882}%
\pgfsetstrokecolor{currentstroke}%
\pgfsetstrokeopacity{0.305789}%
\pgfsetdash{}{0pt}%
\pgfpathmoveto{\pgfqpoint{1.851518in}{3.320312in}}%
\pgfpathcurveto{\pgfqpoint{1.859754in}{3.320312in}}{\pgfqpoint{1.867654in}{3.323585in}}{\pgfqpoint{1.873478in}{3.329409in}}%
\pgfpathcurveto{\pgfqpoint{1.879302in}{3.335233in}}{\pgfqpoint{1.882574in}{3.343133in}}{\pgfqpoint{1.882574in}{3.351369in}}%
\pgfpathcurveto{\pgfqpoint{1.882574in}{3.359605in}}{\pgfqpoint{1.879302in}{3.367505in}}{\pgfqpoint{1.873478in}{3.373329in}}%
\pgfpathcurveto{\pgfqpoint{1.867654in}{3.379153in}}{\pgfqpoint{1.859754in}{3.382425in}}{\pgfqpoint{1.851518in}{3.382425in}}%
\pgfpathcurveto{\pgfqpoint{1.843282in}{3.382425in}}{\pgfqpoint{1.835382in}{3.379153in}}{\pgfqpoint{1.829558in}{3.373329in}}%
\pgfpathcurveto{\pgfqpoint{1.823734in}{3.367505in}}{\pgfqpoint{1.820461in}{3.359605in}}{\pgfqpoint{1.820461in}{3.351369in}}%
\pgfpathcurveto{\pgfqpoint{1.820461in}{3.343133in}}{\pgfqpoint{1.823734in}{3.335233in}}{\pgfqpoint{1.829558in}{3.329409in}}%
\pgfpathcurveto{\pgfqpoint{1.835382in}{3.323585in}}{\pgfqpoint{1.843282in}{3.320312in}}{\pgfqpoint{1.851518in}{3.320312in}}%
\pgfpathclose%
\pgfusepath{stroke,fill}%
\end{pgfscope}%
\begin{pgfscope}%
\pgfpathrectangle{\pgfqpoint{0.100000in}{0.212622in}}{\pgfqpoint{3.696000in}{3.696000in}}%
\pgfusepath{clip}%
\pgfsetbuttcap%
\pgfsetroundjoin%
\definecolor{currentfill}{rgb}{0.121569,0.466667,0.705882}%
\pgfsetfillcolor{currentfill}%
\pgfsetfillopacity{0.305936}%
\pgfsetlinewidth{1.003750pt}%
\definecolor{currentstroke}{rgb}{0.121569,0.466667,0.705882}%
\pgfsetstrokecolor{currentstroke}%
\pgfsetstrokeopacity{0.305936}%
\pgfsetdash{}{0pt}%
\pgfpathmoveto{\pgfqpoint{1.903146in}{3.297662in}}%
\pgfpathcurveto{\pgfqpoint{1.911382in}{3.297662in}}{\pgfqpoint{1.919282in}{3.300934in}}{\pgfqpoint{1.925106in}{3.306758in}}%
\pgfpathcurveto{\pgfqpoint{1.930930in}{3.312582in}}{\pgfqpoint{1.934202in}{3.320482in}}{\pgfqpoint{1.934202in}{3.328718in}}%
\pgfpathcurveto{\pgfqpoint{1.934202in}{3.336954in}}{\pgfqpoint{1.930930in}{3.344854in}}{\pgfqpoint{1.925106in}{3.350678in}}%
\pgfpathcurveto{\pgfqpoint{1.919282in}{3.356502in}}{\pgfqpoint{1.911382in}{3.359775in}}{\pgfqpoint{1.903146in}{3.359775in}}%
\pgfpathcurveto{\pgfqpoint{1.894909in}{3.359775in}}{\pgfqpoint{1.887009in}{3.356502in}}{\pgfqpoint{1.881185in}{3.350678in}}%
\pgfpathcurveto{\pgfqpoint{1.875361in}{3.344854in}}{\pgfqpoint{1.872089in}{3.336954in}}{\pgfqpoint{1.872089in}{3.328718in}}%
\pgfpathcurveto{\pgfqpoint{1.872089in}{3.320482in}}{\pgfqpoint{1.875361in}{3.312582in}}{\pgfqpoint{1.881185in}{3.306758in}}%
\pgfpathcurveto{\pgfqpoint{1.887009in}{3.300934in}}{\pgfqpoint{1.894909in}{3.297662in}}{\pgfqpoint{1.903146in}{3.297662in}}%
\pgfpathclose%
\pgfusepath{stroke,fill}%
\end{pgfscope}%
\begin{pgfscope}%
\pgfpathrectangle{\pgfqpoint{0.100000in}{0.212622in}}{\pgfqpoint{3.696000in}{3.696000in}}%
\pgfusepath{clip}%
\pgfsetbuttcap%
\pgfsetroundjoin%
\definecolor{currentfill}{rgb}{0.121569,0.466667,0.705882}%
\pgfsetfillcolor{currentfill}%
\pgfsetfillopacity{0.306159}%
\pgfsetlinewidth{1.003750pt}%
\definecolor{currentstroke}{rgb}{0.121569,0.466667,0.705882}%
\pgfsetstrokecolor{currentstroke}%
\pgfsetstrokeopacity{0.306159}%
\pgfsetdash{}{0pt}%
\pgfpathmoveto{\pgfqpoint{1.850460in}{3.318434in}}%
\pgfpathcurveto{\pgfqpoint{1.858697in}{3.318434in}}{\pgfqpoint{1.866597in}{3.321706in}}{\pgfqpoint{1.872421in}{3.327530in}}%
\pgfpathcurveto{\pgfqpoint{1.878245in}{3.333354in}}{\pgfqpoint{1.881517in}{3.341254in}}{\pgfqpoint{1.881517in}{3.349491in}}%
\pgfpathcurveto{\pgfqpoint{1.881517in}{3.357727in}}{\pgfqpoint{1.878245in}{3.365627in}}{\pgfqpoint{1.872421in}{3.371451in}}%
\pgfpathcurveto{\pgfqpoint{1.866597in}{3.377275in}}{\pgfqpoint{1.858697in}{3.380547in}}{\pgfqpoint{1.850460in}{3.380547in}}%
\pgfpathcurveto{\pgfqpoint{1.842224in}{3.380547in}}{\pgfqpoint{1.834324in}{3.377275in}}{\pgfqpoint{1.828500in}{3.371451in}}%
\pgfpathcurveto{\pgfqpoint{1.822676in}{3.365627in}}{\pgfqpoint{1.819404in}{3.357727in}}{\pgfqpoint{1.819404in}{3.349491in}}%
\pgfpathcurveto{\pgfqpoint{1.819404in}{3.341254in}}{\pgfqpoint{1.822676in}{3.333354in}}{\pgfqpoint{1.828500in}{3.327530in}}%
\pgfpathcurveto{\pgfqpoint{1.834324in}{3.321706in}}{\pgfqpoint{1.842224in}{3.318434in}}{\pgfqpoint{1.850460in}{3.318434in}}%
\pgfpathclose%
\pgfusepath{stroke,fill}%
\end{pgfscope}%
\begin{pgfscope}%
\pgfpathrectangle{\pgfqpoint{0.100000in}{0.212622in}}{\pgfqpoint{3.696000in}{3.696000in}}%
\pgfusepath{clip}%
\pgfsetbuttcap%
\pgfsetroundjoin%
\definecolor{currentfill}{rgb}{0.121569,0.466667,0.705882}%
\pgfsetfillcolor{currentfill}%
\pgfsetfillopacity{0.306758}%
\pgfsetlinewidth{1.003750pt}%
\definecolor{currentstroke}{rgb}{0.121569,0.466667,0.705882}%
\pgfsetstrokecolor{currentstroke}%
\pgfsetstrokeopacity{0.306758}%
\pgfsetdash{}{0pt}%
\pgfpathmoveto{\pgfqpoint{1.848354in}{3.314986in}}%
\pgfpathcurveto{\pgfqpoint{1.856590in}{3.314986in}}{\pgfqpoint{1.864490in}{3.318258in}}{\pgfqpoint{1.870314in}{3.324082in}}%
\pgfpathcurveto{\pgfqpoint{1.876138in}{3.329906in}}{\pgfqpoint{1.879410in}{3.337806in}}{\pgfqpoint{1.879410in}{3.346042in}}%
\pgfpathcurveto{\pgfqpoint{1.879410in}{3.354279in}}{\pgfqpoint{1.876138in}{3.362179in}}{\pgfqpoint{1.870314in}{3.368002in}}%
\pgfpathcurveto{\pgfqpoint{1.864490in}{3.373826in}}{\pgfqpoint{1.856590in}{3.377099in}}{\pgfqpoint{1.848354in}{3.377099in}}%
\pgfpathcurveto{\pgfqpoint{1.840117in}{3.377099in}}{\pgfqpoint{1.832217in}{3.373826in}}{\pgfqpoint{1.826393in}{3.368002in}}%
\pgfpathcurveto{\pgfqpoint{1.820569in}{3.362179in}}{\pgfqpoint{1.817297in}{3.354279in}}{\pgfqpoint{1.817297in}{3.346042in}}%
\pgfpathcurveto{\pgfqpoint{1.817297in}{3.337806in}}{\pgfqpoint{1.820569in}{3.329906in}}{\pgfqpoint{1.826393in}{3.324082in}}%
\pgfpathcurveto{\pgfqpoint{1.832217in}{3.318258in}}{\pgfqpoint{1.840117in}{3.314986in}}{\pgfqpoint{1.848354in}{3.314986in}}%
\pgfpathclose%
\pgfusepath{stroke,fill}%
\end{pgfscope}%
\begin{pgfscope}%
\pgfpathrectangle{\pgfqpoint{0.100000in}{0.212622in}}{\pgfqpoint{3.696000in}{3.696000in}}%
\pgfusepath{clip}%
\pgfsetbuttcap%
\pgfsetroundjoin%
\definecolor{currentfill}{rgb}{0.121569,0.466667,0.705882}%
\pgfsetfillcolor{currentfill}%
\pgfsetfillopacity{0.307103}%
\pgfsetlinewidth{1.003750pt}%
\definecolor{currentstroke}{rgb}{0.121569,0.466667,0.705882}%
\pgfsetstrokecolor{currentstroke}%
\pgfsetstrokeopacity{0.307103}%
\pgfsetdash{}{0pt}%
\pgfpathmoveto{\pgfqpoint{1.905637in}{3.291170in}}%
\pgfpathcurveto{\pgfqpoint{1.913873in}{3.291170in}}{\pgfqpoint{1.921773in}{3.294442in}}{\pgfqpoint{1.927597in}{3.300266in}}%
\pgfpathcurveto{\pgfqpoint{1.933421in}{3.306090in}}{\pgfqpoint{1.936693in}{3.313990in}}{\pgfqpoint{1.936693in}{3.322227in}}%
\pgfpathcurveto{\pgfqpoint{1.936693in}{3.330463in}}{\pgfqpoint{1.933421in}{3.338363in}}{\pgfqpoint{1.927597in}{3.344187in}}%
\pgfpathcurveto{\pgfqpoint{1.921773in}{3.350011in}}{\pgfqpoint{1.913873in}{3.353283in}}{\pgfqpoint{1.905637in}{3.353283in}}%
\pgfpathcurveto{\pgfqpoint{1.897400in}{3.353283in}}{\pgfqpoint{1.889500in}{3.350011in}}{\pgfqpoint{1.883676in}{3.344187in}}%
\pgfpathcurveto{\pgfqpoint{1.877852in}{3.338363in}}{\pgfqpoint{1.874580in}{3.330463in}}{\pgfqpoint{1.874580in}{3.322227in}}%
\pgfpathcurveto{\pgfqpoint{1.874580in}{3.313990in}}{\pgfqpoint{1.877852in}{3.306090in}}{\pgfqpoint{1.883676in}{3.300266in}}%
\pgfpathcurveto{\pgfqpoint{1.889500in}{3.294442in}}{\pgfqpoint{1.897400in}{3.291170in}}{\pgfqpoint{1.905637in}{3.291170in}}%
\pgfpathclose%
\pgfusepath{stroke,fill}%
\end{pgfscope}%
\begin{pgfscope}%
\pgfpathrectangle{\pgfqpoint{0.100000in}{0.212622in}}{\pgfqpoint{3.696000in}{3.696000in}}%
\pgfusepath{clip}%
\pgfsetbuttcap%
\pgfsetroundjoin%
\definecolor{currentfill}{rgb}{0.121569,0.466667,0.705882}%
\pgfsetfillcolor{currentfill}%
\pgfsetfillopacity{0.307104}%
\pgfsetlinewidth{1.003750pt}%
\definecolor{currentstroke}{rgb}{0.121569,0.466667,0.705882}%
\pgfsetstrokecolor{currentstroke}%
\pgfsetstrokeopacity{0.307104}%
\pgfsetdash{}{0pt}%
\pgfpathmoveto{\pgfqpoint{1.847430in}{3.312997in}}%
\pgfpathcurveto{\pgfqpoint{1.855666in}{3.312997in}}{\pgfqpoint{1.863566in}{3.316269in}}{\pgfqpoint{1.869390in}{3.322093in}}%
\pgfpathcurveto{\pgfqpoint{1.875214in}{3.327917in}}{\pgfqpoint{1.878486in}{3.335817in}}{\pgfqpoint{1.878486in}{3.344053in}}%
\pgfpathcurveto{\pgfqpoint{1.878486in}{3.352290in}}{\pgfqpoint{1.875214in}{3.360190in}}{\pgfqpoint{1.869390in}{3.366014in}}%
\pgfpathcurveto{\pgfqpoint{1.863566in}{3.371838in}}{\pgfqpoint{1.855666in}{3.375110in}}{\pgfqpoint{1.847430in}{3.375110in}}%
\pgfpathcurveto{\pgfqpoint{1.839193in}{3.375110in}}{\pgfqpoint{1.831293in}{3.371838in}}{\pgfqpoint{1.825469in}{3.366014in}}%
\pgfpathcurveto{\pgfqpoint{1.819646in}{3.360190in}}{\pgfqpoint{1.816373in}{3.352290in}}{\pgfqpoint{1.816373in}{3.344053in}}%
\pgfpathcurveto{\pgfqpoint{1.816373in}{3.335817in}}{\pgfqpoint{1.819646in}{3.327917in}}{\pgfqpoint{1.825469in}{3.322093in}}%
\pgfpathcurveto{\pgfqpoint{1.831293in}{3.316269in}}{\pgfqpoint{1.839193in}{3.312997in}}{\pgfqpoint{1.847430in}{3.312997in}}%
\pgfpathclose%
\pgfusepath{stroke,fill}%
\end{pgfscope}%
\begin{pgfscope}%
\pgfpathrectangle{\pgfqpoint{0.100000in}{0.212622in}}{\pgfqpoint{3.696000in}{3.696000in}}%
\pgfusepath{clip}%
\pgfsetbuttcap%
\pgfsetroundjoin%
\definecolor{currentfill}{rgb}{0.121569,0.466667,0.705882}%
\pgfsetfillcolor{currentfill}%
\pgfsetfillopacity{0.307729}%
\pgfsetlinewidth{1.003750pt}%
\definecolor{currentstroke}{rgb}{0.121569,0.466667,0.705882}%
\pgfsetstrokecolor{currentstroke}%
\pgfsetstrokeopacity{0.307729}%
\pgfsetdash{}{0pt}%
\pgfpathmoveto{\pgfqpoint{1.845473in}{3.309705in}}%
\pgfpathcurveto{\pgfqpoint{1.853710in}{3.309705in}}{\pgfqpoint{1.861610in}{3.312978in}}{\pgfqpoint{1.867434in}{3.318802in}}%
\pgfpathcurveto{\pgfqpoint{1.873258in}{3.324626in}}{\pgfqpoint{1.876530in}{3.332526in}}{\pgfqpoint{1.876530in}{3.340762in}}%
\pgfpathcurveto{\pgfqpoint{1.876530in}{3.348998in}}{\pgfqpoint{1.873258in}{3.356898in}}{\pgfqpoint{1.867434in}{3.362722in}}%
\pgfpathcurveto{\pgfqpoint{1.861610in}{3.368546in}}{\pgfqpoint{1.853710in}{3.371818in}}{\pgfqpoint{1.845473in}{3.371818in}}%
\pgfpathcurveto{\pgfqpoint{1.837237in}{3.371818in}}{\pgfqpoint{1.829337in}{3.368546in}}{\pgfqpoint{1.823513in}{3.362722in}}%
\pgfpathcurveto{\pgfqpoint{1.817689in}{3.356898in}}{\pgfqpoint{1.814417in}{3.348998in}}{\pgfqpoint{1.814417in}{3.340762in}}%
\pgfpathcurveto{\pgfqpoint{1.814417in}{3.332526in}}{\pgfqpoint{1.817689in}{3.324626in}}{\pgfqpoint{1.823513in}{3.318802in}}%
\pgfpathcurveto{\pgfqpoint{1.829337in}{3.312978in}}{\pgfqpoint{1.837237in}{3.309705in}}{\pgfqpoint{1.845473in}{3.309705in}}%
\pgfpathclose%
\pgfusepath{stroke,fill}%
\end{pgfscope}%
\begin{pgfscope}%
\pgfpathrectangle{\pgfqpoint{0.100000in}{0.212622in}}{\pgfqpoint{3.696000in}{3.696000in}}%
\pgfusepath{clip}%
\pgfsetbuttcap%
\pgfsetroundjoin%
\definecolor{currentfill}{rgb}{0.121569,0.466667,0.705882}%
\pgfsetfillcolor{currentfill}%
\pgfsetfillopacity{0.308322}%
\pgfsetlinewidth{1.003750pt}%
\definecolor{currentstroke}{rgb}{0.121569,0.466667,0.705882}%
\pgfsetstrokecolor{currentstroke}%
\pgfsetstrokeopacity{0.308322}%
\pgfsetdash{}{0pt}%
\pgfpathmoveto{\pgfqpoint{1.908316in}{3.283640in}}%
\pgfpathcurveto{\pgfqpoint{1.916552in}{3.283640in}}{\pgfqpoint{1.924452in}{3.286913in}}{\pgfqpoint{1.930276in}{3.292736in}}%
\pgfpathcurveto{\pgfqpoint{1.936100in}{3.298560in}}{\pgfqpoint{1.939372in}{3.306460in}}{\pgfqpoint{1.939372in}{3.314697in}}%
\pgfpathcurveto{\pgfqpoint{1.939372in}{3.322933in}}{\pgfqpoint{1.936100in}{3.330833in}}{\pgfqpoint{1.930276in}{3.336657in}}%
\pgfpathcurveto{\pgfqpoint{1.924452in}{3.342481in}}{\pgfqpoint{1.916552in}{3.345753in}}{\pgfqpoint{1.908316in}{3.345753in}}%
\pgfpathcurveto{\pgfqpoint{1.900079in}{3.345753in}}{\pgfqpoint{1.892179in}{3.342481in}}{\pgfqpoint{1.886355in}{3.336657in}}%
\pgfpathcurveto{\pgfqpoint{1.880531in}{3.330833in}}{\pgfqpoint{1.877259in}{3.322933in}}{\pgfqpoint{1.877259in}{3.314697in}}%
\pgfpathcurveto{\pgfqpoint{1.877259in}{3.306460in}}{\pgfqpoint{1.880531in}{3.298560in}}{\pgfqpoint{1.886355in}{3.292736in}}%
\pgfpathcurveto{\pgfqpoint{1.892179in}{3.286913in}}{\pgfqpoint{1.900079in}{3.283640in}}{\pgfqpoint{1.908316in}{3.283640in}}%
\pgfpathclose%
\pgfusepath{stroke,fill}%
\end{pgfscope}%
\begin{pgfscope}%
\pgfpathrectangle{\pgfqpoint{0.100000in}{0.212622in}}{\pgfqpoint{3.696000in}{3.696000in}}%
\pgfusepath{clip}%
\pgfsetbuttcap%
\pgfsetroundjoin%
\definecolor{currentfill}{rgb}{0.121569,0.466667,0.705882}%
\pgfsetfillcolor{currentfill}%
\pgfsetfillopacity{0.308868}%
\pgfsetlinewidth{1.003750pt}%
\definecolor{currentstroke}{rgb}{0.121569,0.466667,0.705882}%
\pgfsetstrokecolor{currentstroke}%
\pgfsetstrokeopacity{0.308868}%
\pgfsetdash{}{0pt}%
\pgfpathmoveto{\pgfqpoint{1.841967in}{3.303660in}}%
\pgfpathcurveto{\pgfqpoint{1.850203in}{3.303660in}}{\pgfqpoint{1.858103in}{3.306933in}}{\pgfqpoint{1.863927in}{3.312757in}}%
\pgfpathcurveto{\pgfqpoint{1.869751in}{3.318581in}}{\pgfqpoint{1.873023in}{3.326481in}}{\pgfqpoint{1.873023in}{3.334717in}}%
\pgfpathcurveto{\pgfqpoint{1.873023in}{3.342953in}}{\pgfqpoint{1.869751in}{3.350853in}}{\pgfqpoint{1.863927in}{3.356677in}}%
\pgfpathcurveto{\pgfqpoint{1.858103in}{3.362501in}}{\pgfqpoint{1.850203in}{3.365773in}}{\pgfqpoint{1.841967in}{3.365773in}}%
\pgfpathcurveto{\pgfqpoint{1.833730in}{3.365773in}}{\pgfqpoint{1.825830in}{3.362501in}}{\pgfqpoint{1.820006in}{3.356677in}}%
\pgfpathcurveto{\pgfqpoint{1.814183in}{3.350853in}}{\pgfqpoint{1.810910in}{3.342953in}}{\pgfqpoint{1.810910in}{3.334717in}}%
\pgfpathcurveto{\pgfqpoint{1.810910in}{3.326481in}}{\pgfqpoint{1.814183in}{3.318581in}}{\pgfqpoint{1.820006in}{3.312757in}}%
\pgfpathcurveto{\pgfqpoint{1.825830in}{3.306933in}}{\pgfqpoint{1.833730in}{3.303660in}}{\pgfqpoint{1.841967in}{3.303660in}}%
\pgfpathclose%
\pgfusepath{stroke,fill}%
\end{pgfscope}%
\begin{pgfscope}%
\pgfpathrectangle{\pgfqpoint{0.100000in}{0.212622in}}{\pgfqpoint{3.696000in}{3.696000in}}%
\pgfusepath{clip}%
\pgfsetbuttcap%
\pgfsetroundjoin%
\definecolor{currentfill}{rgb}{0.121569,0.466667,0.705882}%
\pgfsetfillcolor{currentfill}%
\pgfsetfillopacity{0.309994}%
\pgfsetlinewidth{1.003750pt}%
\definecolor{currentstroke}{rgb}{0.121569,0.466667,0.705882}%
\pgfsetstrokecolor{currentstroke}%
\pgfsetstrokeopacity{0.309994}%
\pgfsetdash{}{0pt}%
\pgfpathmoveto{\pgfqpoint{1.839009in}{3.297816in}}%
\pgfpathcurveto{\pgfqpoint{1.847245in}{3.297816in}}{\pgfqpoint{1.855145in}{3.301089in}}{\pgfqpoint{1.860969in}{3.306913in}}%
\pgfpathcurveto{\pgfqpoint{1.866793in}{3.312736in}}{\pgfqpoint{1.870065in}{3.320636in}}{\pgfqpoint{1.870065in}{3.328873in}}%
\pgfpathcurveto{\pgfqpoint{1.870065in}{3.337109in}}{\pgfqpoint{1.866793in}{3.345009in}}{\pgfqpoint{1.860969in}{3.350833in}}%
\pgfpathcurveto{\pgfqpoint{1.855145in}{3.356657in}}{\pgfqpoint{1.847245in}{3.359929in}}{\pgfqpoint{1.839009in}{3.359929in}}%
\pgfpathcurveto{\pgfqpoint{1.830773in}{3.359929in}}{\pgfqpoint{1.822873in}{3.356657in}}{\pgfqpoint{1.817049in}{3.350833in}}%
\pgfpathcurveto{\pgfqpoint{1.811225in}{3.345009in}}{\pgfqpoint{1.807952in}{3.337109in}}{\pgfqpoint{1.807952in}{3.328873in}}%
\pgfpathcurveto{\pgfqpoint{1.807952in}{3.320636in}}{\pgfqpoint{1.811225in}{3.312736in}}{\pgfqpoint{1.817049in}{3.306913in}}%
\pgfpathcurveto{\pgfqpoint{1.822873in}{3.301089in}}{\pgfqpoint{1.830773in}{3.297816in}}{\pgfqpoint{1.839009in}{3.297816in}}%
\pgfpathclose%
\pgfusepath{stroke,fill}%
\end{pgfscope}%
\begin{pgfscope}%
\pgfpathrectangle{\pgfqpoint{0.100000in}{0.212622in}}{\pgfqpoint{3.696000in}{3.696000in}}%
\pgfusepath{clip}%
\pgfsetbuttcap%
\pgfsetroundjoin%
\definecolor{currentfill}{rgb}{0.121569,0.466667,0.705882}%
\pgfsetfillcolor{currentfill}%
\pgfsetfillopacity{0.310010}%
\pgfsetlinewidth{1.003750pt}%
\definecolor{currentstroke}{rgb}{0.121569,0.466667,0.705882}%
\pgfsetstrokecolor{currentstroke}%
\pgfsetstrokeopacity{0.310010}%
\pgfsetdash{}{0pt}%
\pgfpathmoveto{\pgfqpoint{1.910772in}{3.276190in}}%
\pgfpathcurveto{\pgfqpoint{1.919008in}{3.276190in}}{\pgfqpoint{1.926908in}{3.279463in}}{\pgfqpoint{1.932732in}{3.285287in}}%
\pgfpathcurveto{\pgfqpoint{1.938556in}{3.291111in}}{\pgfqpoint{1.941828in}{3.299011in}}{\pgfqpoint{1.941828in}{3.307247in}}%
\pgfpathcurveto{\pgfqpoint{1.941828in}{3.315483in}}{\pgfqpoint{1.938556in}{3.323383in}}{\pgfqpoint{1.932732in}{3.329207in}}%
\pgfpathcurveto{\pgfqpoint{1.926908in}{3.335031in}}{\pgfqpoint{1.919008in}{3.338303in}}{\pgfqpoint{1.910772in}{3.338303in}}%
\pgfpathcurveto{\pgfqpoint{1.902535in}{3.338303in}}{\pgfqpoint{1.894635in}{3.335031in}}{\pgfqpoint{1.888812in}{3.329207in}}%
\pgfpathcurveto{\pgfqpoint{1.882988in}{3.323383in}}{\pgfqpoint{1.879715in}{3.315483in}}{\pgfqpoint{1.879715in}{3.307247in}}%
\pgfpathcurveto{\pgfqpoint{1.879715in}{3.299011in}}{\pgfqpoint{1.882988in}{3.291111in}}{\pgfqpoint{1.888812in}{3.285287in}}%
\pgfpathcurveto{\pgfqpoint{1.894635in}{3.279463in}}{\pgfqpoint{1.902535in}{3.276190in}}{\pgfqpoint{1.910772in}{3.276190in}}%
\pgfpathclose%
\pgfusepath{stroke,fill}%
\end{pgfscope}%
\begin{pgfscope}%
\pgfpathrectangle{\pgfqpoint{0.100000in}{0.212622in}}{\pgfqpoint{3.696000in}{3.696000in}}%
\pgfusepath{clip}%
\pgfsetbuttcap%
\pgfsetroundjoin%
\definecolor{currentfill}{rgb}{0.121569,0.466667,0.705882}%
\pgfsetfillcolor{currentfill}%
\pgfsetfillopacity{0.310755}%
\pgfsetlinewidth{1.003750pt}%
\definecolor{currentstroke}{rgb}{0.121569,0.466667,0.705882}%
\pgfsetstrokecolor{currentstroke}%
\pgfsetstrokeopacity{0.310755}%
\pgfsetdash{}{0pt}%
\pgfpathmoveto{\pgfqpoint{1.836400in}{3.293651in}}%
\pgfpathcurveto{\pgfqpoint{1.844636in}{3.293651in}}{\pgfqpoint{1.852536in}{3.296924in}}{\pgfqpoint{1.858360in}{3.302747in}}%
\pgfpathcurveto{\pgfqpoint{1.864184in}{3.308571in}}{\pgfqpoint{1.867456in}{3.316471in}}{\pgfqpoint{1.867456in}{3.324708in}}%
\pgfpathcurveto{\pgfqpoint{1.867456in}{3.332944in}}{\pgfqpoint{1.864184in}{3.340844in}}{\pgfqpoint{1.858360in}{3.346668in}}%
\pgfpathcurveto{\pgfqpoint{1.852536in}{3.352492in}}{\pgfqpoint{1.844636in}{3.355764in}}{\pgfqpoint{1.836400in}{3.355764in}}%
\pgfpathcurveto{\pgfqpoint{1.828164in}{3.355764in}}{\pgfqpoint{1.820263in}{3.352492in}}{\pgfqpoint{1.814440in}{3.346668in}}%
\pgfpathcurveto{\pgfqpoint{1.808616in}{3.340844in}}{\pgfqpoint{1.805343in}{3.332944in}}{\pgfqpoint{1.805343in}{3.324708in}}%
\pgfpathcurveto{\pgfqpoint{1.805343in}{3.316471in}}{\pgfqpoint{1.808616in}{3.308571in}}{\pgfqpoint{1.814440in}{3.302747in}}%
\pgfpathcurveto{\pgfqpoint{1.820263in}{3.296924in}}{\pgfqpoint{1.828164in}{3.293651in}}{\pgfqpoint{1.836400in}{3.293651in}}%
\pgfpathclose%
\pgfusepath{stroke,fill}%
\end{pgfscope}%
\begin{pgfscope}%
\pgfpathrectangle{\pgfqpoint{0.100000in}{0.212622in}}{\pgfqpoint{3.696000in}{3.696000in}}%
\pgfusepath{clip}%
\pgfsetbuttcap%
\pgfsetroundjoin%
\definecolor{currentfill}{rgb}{0.121569,0.466667,0.705882}%
\pgfsetfillcolor{currentfill}%
\pgfsetfillopacity{0.311357}%
\pgfsetlinewidth{1.003750pt}%
\definecolor{currentstroke}{rgb}{0.121569,0.466667,0.705882}%
\pgfsetstrokecolor{currentstroke}%
\pgfsetstrokeopacity{0.311357}%
\pgfsetdash{}{0pt}%
\pgfpathmoveto{\pgfqpoint{1.834814in}{3.290463in}}%
\pgfpathcurveto{\pgfqpoint{1.843050in}{3.290463in}}{\pgfqpoint{1.850950in}{3.293735in}}{\pgfqpoint{1.856774in}{3.299559in}}%
\pgfpathcurveto{\pgfqpoint{1.862598in}{3.305383in}}{\pgfqpoint{1.865870in}{3.313283in}}{\pgfqpoint{1.865870in}{3.321519in}}%
\pgfpathcurveto{\pgfqpoint{1.865870in}{3.329756in}}{\pgfqpoint{1.862598in}{3.337656in}}{\pgfqpoint{1.856774in}{3.343480in}}%
\pgfpathcurveto{\pgfqpoint{1.850950in}{3.349303in}}{\pgfqpoint{1.843050in}{3.352576in}}{\pgfqpoint{1.834814in}{3.352576in}}%
\pgfpathcurveto{\pgfqpoint{1.826578in}{3.352576in}}{\pgfqpoint{1.818678in}{3.349303in}}{\pgfqpoint{1.812854in}{3.343480in}}%
\pgfpathcurveto{\pgfqpoint{1.807030in}{3.337656in}}{\pgfqpoint{1.803757in}{3.329756in}}{\pgfqpoint{1.803757in}{3.321519in}}%
\pgfpathcurveto{\pgfqpoint{1.803757in}{3.313283in}}{\pgfqpoint{1.807030in}{3.305383in}}{\pgfqpoint{1.812854in}{3.299559in}}%
\pgfpathcurveto{\pgfqpoint{1.818678in}{3.293735in}}{\pgfqpoint{1.826578in}{3.290463in}}{\pgfqpoint{1.834814in}{3.290463in}}%
\pgfpathclose%
\pgfusepath{stroke,fill}%
\end{pgfscope}%
\begin{pgfscope}%
\pgfpathrectangle{\pgfqpoint{0.100000in}{0.212622in}}{\pgfqpoint{3.696000in}{3.696000in}}%
\pgfusepath{clip}%
\pgfsetbuttcap%
\pgfsetroundjoin%
\definecolor{currentfill}{rgb}{0.121569,0.466667,0.705882}%
\pgfsetfillcolor{currentfill}%
\pgfsetfillopacity{0.311751}%
\pgfsetlinewidth{1.003750pt}%
\definecolor{currentstroke}{rgb}{0.121569,0.466667,0.705882}%
\pgfsetstrokecolor{currentstroke}%
\pgfsetstrokeopacity{0.311751}%
\pgfsetdash{}{0pt}%
\pgfpathmoveto{\pgfqpoint{1.914122in}{3.266545in}}%
\pgfpathcurveto{\pgfqpoint{1.922359in}{3.266545in}}{\pgfqpoint{1.930259in}{3.269818in}}{\pgfqpoint{1.936083in}{3.275641in}}%
\pgfpathcurveto{\pgfqpoint{1.941906in}{3.281465in}}{\pgfqpoint{1.945179in}{3.289365in}}{\pgfqpoint{1.945179in}{3.297602in}}%
\pgfpathcurveto{\pgfqpoint{1.945179in}{3.305838in}}{\pgfqpoint{1.941906in}{3.313738in}}{\pgfqpoint{1.936083in}{3.319562in}}%
\pgfpathcurveto{\pgfqpoint{1.930259in}{3.325386in}}{\pgfqpoint{1.922359in}{3.328658in}}{\pgfqpoint{1.914122in}{3.328658in}}%
\pgfpathcurveto{\pgfqpoint{1.905886in}{3.328658in}}{\pgfqpoint{1.897986in}{3.325386in}}{\pgfqpoint{1.892162in}{3.319562in}}%
\pgfpathcurveto{\pgfqpoint{1.886338in}{3.313738in}}{\pgfqpoint{1.883066in}{3.305838in}}{\pgfqpoint{1.883066in}{3.297602in}}%
\pgfpathcurveto{\pgfqpoint{1.883066in}{3.289365in}}{\pgfqpoint{1.886338in}{3.281465in}}{\pgfqpoint{1.892162in}{3.275641in}}%
\pgfpathcurveto{\pgfqpoint{1.897986in}{3.269818in}}{\pgfqpoint{1.905886in}{3.266545in}}{\pgfqpoint{1.914122in}{3.266545in}}%
\pgfpathclose%
\pgfusepath{stroke,fill}%
\end{pgfscope}%
\begin{pgfscope}%
\pgfpathrectangle{\pgfqpoint{0.100000in}{0.212622in}}{\pgfqpoint{3.696000in}{3.696000in}}%
\pgfusepath{clip}%
\pgfsetbuttcap%
\pgfsetroundjoin%
\definecolor{currentfill}{rgb}{0.121569,0.466667,0.705882}%
\pgfsetfillcolor{currentfill}%
\pgfsetfillopacity{0.311887}%
\pgfsetlinewidth{1.003750pt}%
\definecolor{currentstroke}{rgb}{0.121569,0.466667,0.705882}%
\pgfsetstrokecolor{currentstroke}%
\pgfsetstrokeopacity{0.311887}%
\pgfsetdash{}{0pt}%
\pgfpathmoveto{\pgfqpoint{1.833262in}{3.287794in}}%
\pgfpathcurveto{\pgfqpoint{1.841499in}{3.287794in}}{\pgfqpoint{1.849399in}{3.291066in}}{\pgfqpoint{1.855223in}{3.296890in}}%
\pgfpathcurveto{\pgfqpoint{1.861047in}{3.302714in}}{\pgfqpoint{1.864319in}{3.310614in}}{\pgfqpoint{1.864319in}{3.318850in}}%
\pgfpathcurveto{\pgfqpoint{1.864319in}{3.327086in}}{\pgfqpoint{1.861047in}{3.334986in}}{\pgfqpoint{1.855223in}{3.340810in}}%
\pgfpathcurveto{\pgfqpoint{1.849399in}{3.346634in}}{\pgfqpoint{1.841499in}{3.349907in}}{\pgfqpoint{1.833262in}{3.349907in}}%
\pgfpathcurveto{\pgfqpoint{1.825026in}{3.349907in}}{\pgfqpoint{1.817126in}{3.346634in}}{\pgfqpoint{1.811302in}{3.340810in}}%
\pgfpathcurveto{\pgfqpoint{1.805478in}{3.334986in}}{\pgfqpoint{1.802206in}{3.327086in}}{\pgfqpoint{1.802206in}{3.318850in}}%
\pgfpathcurveto{\pgfqpoint{1.802206in}{3.310614in}}{\pgfqpoint{1.805478in}{3.302714in}}{\pgfqpoint{1.811302in}{3.296890in}}%
\pgfpathcurveto{\pgfqpoint{1.817126in}{3.291066in}}{\pgfqpoint{1.825026in}{3.287794in}}{\pgfqpoint{1.833262in}{3.287794in}}%
\pgfpathclose%
\pgfusepath{stroke,fill}%
\end{pgfscope}%
\begin{pgfscope}%
\pgfpathrectangle{\pgfqpoint{0.100000in}{0.212622in}}{\pgfqpoint{3.696000in}{3.696000in}}%
\pgfusepath{clip}%
\pgfsetbuttcap%
\pgfsetroundjoin%
\definecolor{currentfill}{rgb}{0.121569,0.466667,0.705882}%
\pgfsetfillcolor{currentfill}%
\pgfsetfillopacity{0.312802}%
\pgfsetlinewidth{1.003750pt}%
\definecolor{currentstroke}{rgb}{0.121569,0.466667,0.705882}%
\pgfsetstrokecolor{currentstroke}%
\pgfsetstrokeopacity{0.312802}%
\pgfsetdash{}{0pt}%
\pgfpathmoveto{\pgfqpoint{1.830326in}{3.282902in}}%
\pgfpathcurveto{\pgfqpoint{1.838562in}{3.282902in}}{\pgfqpoint{1.846462in}{3.286175in}}{\pgfqpoint{1.852286in}{3.291999in}}%
\pgfpathcurveto{\pgfqpoint{1.858110in}{3.297822in}}{\pgfqpoint{1.861382in}{3.305723in}}{\pgfqpoint{1.861382in}{3.313959in}}%
\pgfpathcurveto{\pgfqpoint{1.861382in}{3.322195in}}{\pgfqpoint{1.858110in}{3.330095in}}{\pgfqpoint{1.852286in}{3.335919in}}%
\pgfpathcurveto{\pgfqpoint{1.846462in}{3.341743in}}{\pgfqpoint{1.838562in}{3.345015in}}{\pgfqpoint{1.830326in}{3.345015in}}%
\pgfpathcurveto{\pgfqpoint{1.822090in}{3.345015in}}{\pgfqpoint{1.814190in}{3.341743in}}{\pgfqpoint{1.808366in}{3.335919in}}%
\pgfpathcurveto{\pgfqpoint{1.802542in}{3.330095in}}{\pgfqpoint{1.799269in}{3.322195in}}{\pgfqpoint{1.799269in}{3.313959in}}%
\pgfpathcurveto{\pgfqpoint{1.799269in}{3.305723in}}{\pgfqpoint{1.802542in}{3.297822in}}{\pgfqpoint{1.808366in}{3.291999in}}%
\pgfpathcurveto{\pgfqpoint{1.814190in}{3.286175in}}{\pgfqpoint{1.822090in}{3.282902in}}{\pgfqpoint{1.830326in}{3.282902in}}%
\pgfpathclose%
\pgfusepath{stroke,fill}%
\end{pgfscope}%
\begin{pgfscope}%
\pgfpathrectangle{\pgfqpoint{0.100000in}{0.212622in}}{\pgfqpoint{3.696000in}{3.696000in}}%
\pgfusepath{clip}%
\pgfsetbuttcap%
\pgfsetroundjoin%
\definecolor{currentfill}{rgb}{0.121569,0.466667,0.705882}%
\pgfsetfillcolor{currentfill}%
\pgfsetfillopacity{0.312860}%
\pgfsetlinewidth{1.003750pt}%
\definecolor{currentstroke}{rgb}{0.121569,0.466667,0.705882}%
\pgfsetstrokecolor{currentstroke}%
\pgfsetstrokeopacity{0.312860}%
\pgfsetdash{}{0pt}%
\pgfpathmoveto{\pgfqpoint{1.915643in}{3.261431in}}%
\pgfpathcurveto{\pgfqpoint{1.923879in}{3.261431in}}{\pgfqpoint{1.931779in}{3.264704in}}{\pgfqpoint{1.937603in}{3.270528in}}%
\pgfpathcurveto{\pgfqpoint{1.943427in}{3.276352in}}{\pgfqpoint{1.946700in}{3.284252in}}{\pgfqpoint{1.946700in}{3.292488in}}%
\pgfpathcurveto{\pgfqpoint{1.946700in}{3.300724in}}{\pgfqpoint{1.943427in}{3.308624in}}{\pgfqpoint{1.937603in}{3.314448in}}%
\pgfpathcurveto{\pgfqpoint{1.931779in}{3.320272in}}{\pgfqpoint{1.923879in}{3.323544in}}{\pgfqpoint{1.915643in}{3.323544in}}%
\pgfpathcurveto{\pgfqpoint{1.907407in}{3.323544in}}{\pgfqpoint{1.899507in}{3.320272in}}{\pgfqpoint{1.893683in}{3.314448in}}%
\pgfpathcurveto{\pgfqpoint{1.887859in}{3.308624in}}{\pgfqpoint{1.884587in}{3.300724in}}{\pgfqpoint{1.884587in}{3.292488in}}%
\pgfpathcurveto{\pgfqpoint{1.884587in}{3.284252in}}{\pgfqpoint{1.887859in}{3.276352in}}{\pgfqpoint{1.893683in}{3.270528in}}%
\pgfpathcurveto{\pgfqpoint{1.899507in}{3.264704in}}{\pgfqpoint{1.907407in}{3.261431in}}{\pgfqpoint{1.915643in}{3.261431in}}%
\pgfpathclose%
\pgfusepath{stroke,fill}%
\end{pgfscope}%
\begin{pgfscope}%
\pgfpathrectangle{\pgfqpoint{0.100000in}{0.212622in}}{\pgfqpoint{3.696000in}{3.696000in}}%
\pgfusepath{clip}%
\pgfsetbuttcap%
\pgfsetroundjoin%
\definecolor{currentfill}{rgb}{0.121569,0.466667,0.705882}%
\pgfsetfillcolor{currentfill}%
\pgfsetfillopacity{0.313499}%
\pgfsetlinewidth{1.003750pt}%
\definecolor{currentstroke}{rgb}{0.121569,0.466667,0.705882}%
\pgfsetstrokecolor{currentstroke}%
\pgfsetstrokeopacity{0.313499}%
\pgfsetdash{}{0pt}%
\pgfpathmoveto{\pgfqpoint{1.916463in}{3.258715in}}%
\pgfpathcurveto{\pgfqpoint{1.924699in}{3.258715in}}{\pgfqpoint{1.932599in}{3.261987in}}{\pgfqpoint{1.938423in}{3.267811in}}%
\pgfpathcurveto{\pgfqpoint{1.944247in}{3.273635in}}{\pgfqpoint{1.947519in}{3.281535in}}{\pgfqpoint{1.947519in}{3.289771in}}%
\pgfpathcurveto{\pgfqpoint{1.947519in}{3.298007in}}{\pgfqpoint{1.944247in}{3.305908in}}{\pgfqpoint{1.938423in}{3.311731in}}%
\pgfpathcurveto{\pgfqpoint{1.932599in}{3.317555in}}{\pgfqpoint{1.924699in}{3.320828in}}{\pgfqpoint{1.916463in}{3.320828in}}%
\pgfpathcurveto{\pgfqpoint{1.908226in}{3.320828in}}{\pgfqpoint{1.900326in}{3.317555in}}{\pgfqpoint{1.894502in}{3.311731in}}%
\pgfpathcurveto{\pgfqpoint{1.888678in}{3.305908in}}{\pgfqpoint{1.885406in}{3.298007in}}{\pgfqpoint{1.885406in}{3.289771in}}%
\pgfpathcurveto{\pgfqpoint{1.885406in}{3.281535in}}{\pgfqpoint{1.888678in}{3.273635in}}{\pgfqpoint{1.894502in}{3.267811in}}%
\pgfpathcurveto{\pgfqpoint{1.900326in}{3.261987in}}{\pgfqpoint{1.908226in}{3.258715in}}{\pgfqpoint{1.916463in}{3.258715in}}%
\pgfpathclose%
\pgfusepath{stroke,fill}%
\end{pgfscope}%
\begin{pgfscope}%
\pgfpathrectangle{\pgfqpoint{0.100000in}{0.212622in}}{\pgfqpoint{3.696000in}{3.696000in}}%
\pgfusepath{clip}%
\pgfsetbuttcap%
\pgfsetroundjoin%
\definecolor{currentfill}{rgb}{0.121569,0.466667,0.705882}%
\pgfsetfillcolor{currentfill}%
\pgfsetfillopacity{0.313536}%
\pgfsetlinewidth{1.003750pt}%
\definecolor{currentstroke}{rgb}{0.121569,0.466667,0.705882}%
\pgfsetstrokecolor{currentstroke}%
\pgfsetstrokeopacity{0.313536}%
\pgfsetdash{}{0pt}%
\pgfpathmoveto{\pgfqpoint{1.828502in}{3.278851in}}%
\pgfpathcurveto{\pgfqpoint{1.836738in}{3.278851in}}{\pgfqpoint{1.844638in}{3.282123in}}{\pgfqpoint{1.850462in}{3.287947in}}%
\pgfpathcurveto{\pgfqpoint{1.856286in}{3.293771in}}{\pgfqpoint{1.859559in}{3.301671in}}{\pgfqpoint{1.859559in}{3.309907in}}%
\pgfpathcurveto{\pgfqpoint{1.859559in}{3.318143in}}{\pgfqpoint{1.856286in}{3.326044in}}{\pgfqpoint{1.850462in}{3.331867in}}%
\pgfpathcurveto{\pgfqpoint{1.844638in}{3.337691in}}{\pgfqpoint{1.836738in}{3.340964in}}{\pgfqpoint{1.828502in}{3.340964in}}%
\pgfpathcurveto{\pgfqpoint{1.820266in}{3.340964in}}{\pgfqpoint{1.812366in}{3.337691in}}{\pgfqpoint{1.806542in}{3.331867in}}%
\pgfpathcurveto{\pgfqpoint{1.800718in}{3.326044in}}{\pgfqpoint{1.797446in}{3.318143in}}{\pgfqpoint{1.797446in}{3.309907in}}%
\pgfpathcurveto{\pgfqpoint{1.797446in}{3.301671in}}{\pgfqpoint{1.800718in}{3.293771in}}{\pgfqpoint{1.806542in}{3.287947in}}%
\pgfpathcurveto{\pgfqpoint{1.812366in}{3.282123in}}{\pgfqpoint{1.820266in}{3.278851in}}{\pgfqpoint{1.828502in}{3.278851in}}%
\pgfpathclose%
\pgfusepath{stroke,fill}%
\end{pgfscope}%
\begin{pgfscope}%
\pgfpathrectangle{\pgfqpoint{0.100000in}{0.212622in}}{\pgfqpoint{3.696000in}{3.696000in}}%
\pgfusepath{clip}%
\pgfsetbuttcap%
\pgfsetroundjoin%
\definecolor{currentfill}{rgb}{0.121569,0.466667,0.705882}%
\pgfsetfillcolor{currentfill}%
\pgfsetfillopacity{0.313917}%
\pgfsetlinewidth{1.003750pt}%
\definecolor{currentstroke}{rgb}{0.121569,0.466667,0.705882}%
\pgfsetstrokecolor{currentstroke}%
\pgfsetstrokeopacity{0.313917}%
\pgfsetdash{}{0pt}%
\pgfpathmoveto{\pgfqpoint{1.827330in}{3.277025in}}%
\pgfpathcurveto{\pgfqpoint{1.835566in}{3.277025in}}{\pgfqpoint{1.843466in}{3.280297in}}{\pgfqpoint{1.849290in}{3.286121in}}%
\pgfpathcurveto{\pgfqpoint{1.855114in}{3.291945in}}{\pgfqpoint{1.858386in}{3.299845in}}{\pgfqpoint{1.858386in}{3.308081in}}%
\pgfpathcurveto{\pgfqpoint{1.858386in}{3.316318in}}{\pgfqpoint{1.855114in}{3.324218in}}{\pgfqpoint{1.849290in}{3.330042in}}%
\pgfpathcurveto{\pgfqpoint{1.843466in}{3.335866in}}{\pgfqpoint{1.835566in}{3.339138in}}{\pgfqpoint{1.827330in}{3.339138in}}%
\pgfpathcurveto{\pgfqpoint{1.819094in}{3.339138in}}{\pgfqpoint{1.811193in}{3.335866in}}{\pgfqpoint{1.805370in}{3.330042in}}%
\pgfpathcurveto{\pgfqpoint{1.799546in}{3.324218in}}{\pgfqpoint{1.796273in}{3.316318in}}{\pgfqpoint{1.796273in}{3.308081in}}%
\pgfpathcurveto{\pgfqpoint{1.796273in}{3.299845in}}{\pgfqpoint{1.799546in}{3.291945in}}{\pgfqpoint{1.805370in}{3.286121in}}%
\pgfpathcurveto{\pgfqpoint{1.811193in}{3.280297in}}{\pgfqpoint{1.819094in}{3.277025in}}{\pgfqpoint{1.827330in}{3.277025in}}%
\pgfpathclose%
\pgfusepath{stroke,fill}%
\end{pgfscope}%
\begin{pgfscope}%
\pgfpathrectangle{\pgfqpoint{0.100000in}{0.212622in}}{\pgfqpoint{3.696000in}{3.696000in}}%
\pgfusepath{clip}%
\pgfsetbuttcap%
\pgfsetroundjoin%
\definecolor{currentfill}{rgb}{0.121569,0.466667,0.705882}%
\pgfsetfillcolor{currentfill}%
\pgfsetfillopacity{0.314160}%
\pgfsetlinewidth{1.003750pt}%
\definecolor{currentstroke}{rgb}{0.121569,0.466667,0.705882}%
\pgfsetstrokecolor{currentstroke}%
\pgfsetstrokeopacity{0.314160}%
\pgfsetdash{}{0pt}%
\pgfpathmoveto{\pgfqpoint{1.917606in}{3.254818in}}%
\pgfpathcurveto{\pgfqpoint{1.925842in}{3.254818in}}{\pgfqpoint{1.933742in}{3.258091in}}{\pgfqpoint{1.939566in}{3.263915in}}%
\pgfpathcurveto{\pgfqpoint{1.945390in}{3.269739in}}{\pgfqpoint{1.948663in}{3.277639in}}{\pgfqpoint{1.948663in}{3.285875in}}%
\pgfpathcurveto{\pgfqpoint{1.948663in}{3.294111in}}{\pgfqpoint{1.945390in}{3.302011in}}{\pgfqpoint{1.939566in}{3.307835in}}%
\pgfpathcurveto{\pgfqpoint{1.933742in}{3.313659in}}{\pgfqpoint{1.925842in}{3.316931in}}{\pgfqpoint{1.917606in}{3.316931in}}%
\pgfpathcurveto{\pgfqpoint{1.909370in}{3.316931in}}{\pgfqpoint{1.901470in}{3.313659in}}{\pgfqpoint{1.895646in}{3.307835in}}%
\pgfpathcurveto{\pgfqpoint{1.889822in}{3.302011in}}{\pgfqpoint{1.886550in}{3.294111in}}{\pgfqpoint{1.886550in}{3.285875in}}%
\pgfpathcurveto{\pgfqpoint{1.886550in}{3.277639in}}{\pgfqpoint{1.889822in}{3.269739in}}{\pgfqpoint{1.895646in}{3.263915in}}%
\pgfpathcurveto{\pgfqpoint{1.901470in}{3.258091in}}{\pgfqpoint{1.909370in}{3.254818in}}{\pgfqpoint{1.917606in}{3.254818in}}%
\pgfpathclose%
\pgfusepath{stroke,fill}%
\end{pgfscope}%
\begin{pgfscope}%
\pgfpathrectangle{\pgfqpoint{0.100000in}{0.212622in}}{\pgfqpoint{3.696000in}{3.696000in}}%
\pgfusepath{clip}%
\pgfsetbuttcap%
\pgfsetroundjoin%
\definecolor{currentfill}{rgb}{0.121569,0.466667,0.705882}%
\pgfsetfillcolor{currentfill}%
\pgfsetfillopacity{0.314635}%
\pgfsetlinewidth{1.003750pt}%
\definecolor{currentstroke}{rgb}{0.121569,0.466667,0.705882}%
\pgfsetstrokecolor{currentstroke}%
\pgfsetstrokeopacity{0.314635}%
\pgfsetdash{}{0pt}%
\pgfpathmoveto{\pgfqpoint{1.825374in}{3.273554in}}%
\pgfpathcurveto{\pgfqpoint{1.833610in}{3.273554in}}{\pgfqpoint{1.841510in}{3.276827in}}{\pgfqpoint{1.847334in}{3.282651in}}%
\pgfpathcurveto{\pgfqpoint{1.853158in}{3.288474in}}{\pgfqpoint{1.856430in}{3.296375in}}{\pgfqpoint{1.856430in}{3.304611in}}%
\pgfpathcurveto{\pgfqpoint{1.856430in}{3.312847in}}{\pgfqpoint{1.853158in}{3.320747in}}{\pgfqpoint{1.847334in}{3.326571in}}%
\pgfpathcurveto{\pgfqpoint{1.841510in}{3.332395in}}{\pgfqpoint{1.833610in}{3.335667in}}{\pgfqpoint{1.825374in}{3.335667in}}%
\pgfpathcurveto{\pgfqpoint{1.817138in}{3.335667in}}{\pgfqpoint{1.809238in}{3.332395in}}{\pgfqpoint{1.803414in}{3.326571in}}%
\pgfpathcurveto{\pgfqpoint{1.797590in}{3.320747in}}{\pgfqpoint{1.794317in}{3.312847in}}{\pgfqpoint{1.794317in}{3.304611in}}%
\pgfpathcurveto{\pgfqpoint{1.794317in}{3.296375in}}{\pgfqpoint{1.797590in}{3.288474in}}{\pgfqpoint{1.803414in}{3.282651in}}%
\pgfpathcurveto{\pgfqpoint{1.809238in}{3.276827in}}{\pgfqpoint{1.817138in}{3.273554in}}{\pgfqpoint{1.825374in}{3.273554in}}%
\pgfpathclose%
\pgfusepath{stroke,fill}%
\end{pgfscope}%
\begin{pgfscope}%
\pgfpathrectangle{\pgfqpoint{0.100000in}{0.212622in}}{\pgfqpoint{3.696000in}{3.696000in}}%
\pgfusepath{clip}%
\pgfsetbuttcap%
\pgfsetroundjoin%
\definecolor{currentfill}{rgb}{0.121569,0.466667,0.705882}%
\pgfsetfillcolor{currentfill}%
\pgfsetfillopacity{0.315395}%
\pgfsetlinewidth{1.003750pt}%
\definecolor{currentstroke}{rgb}{0.121569,0.466667,0.705882}%
\pgfsetstrokecolor{currentstroke}%
\pgfsetstrokeopacity{0.315395}%
\pgfsetdash{}{0pt}%
\pgfpathmoveto{\pgfqpoint{1.918860in}{3.249757in}}%
\pgfpathcurveto{\pgfqpoint{1.927096in}{3.249757in}}{\pgfqpoint{1.934996in}{3.253029in}}{\pgfqpoint{1.940820in}{3.258853in}}%
\pgfpathcurveto{\pgfqpoint{1.946644in}{3.264677in}}{\pgfqpoint{1.949916in}{3.272577in}}{\pgfqpoint{1.949916in}{3.280813in}}%
\pgfpathcurveto{\pgfqpoint{1.949916in}{3.289050in}}{\pgfqpoint{1.946644in}{3.296950in}}{\pgfqpoint{1.940820in}{3.302774in}}%
\pgfpathcurveto{\pgfqpoint{1.934996in}{3.308598in}}{\pgfqpoint{1.927096in}{3.311870in}}{\pgfqpoint{1.918860in}{3.311870in}}%
\pgfpathcurveto{\pgfqpoint{1.910624in}{3.311870in}}{\pgfqpoint{1.902724in}{3.308598in}}{\pgfqpoint{1.896900in}{3.302774in}}%
\pgfpathcurveto{\pgfqpoint{1.891076in}{3.296950in}}{\pgfqpoint{1.887803in}{3.289050in}}{\pgfqpoint{1.887803in}{3.280813in}}%
\pgfpathcurveto{\pgfqpoint{1.887803in}{3.272577in}}{\pgfqpoint{1.891076in}{3.264677in}}{\pgfqpoint{1.896900in}{3.258853in}}%
\pgfpathcurveto{\pgfqpoint{1.902724in}{3.253029in}}{\pgfqpoint{1.910624in}{3.249757in}}{\pgfqpoint{1.918860in}{3.249757in}}%
\pgfpathclose%
\pgfusepath{stroke,fill}%
\end{pgfscope}%
\begin{pgfscope}%
\pgfpathrectangle{\pgfqpoint{0.100000in}{0.212622in}}{\pgfqpoint{3.696000in}{3.696000in}}%
\pgfusepath{clip}%
\pgfsetbuttcap%
\pgfsetroundjoin%
\definecolor{currentfill}{rgb}{0.121569,0.466667,0.705882}%
\pgfsetfillcolor{currentfill}%
\pgfsetfillopacity{0.315962}%
\pgfsetlinewidth{1.003750pt}%
\definecolor{currentstroke}{rgb}{0.121569,0.466667,0.705882}%
\pgfsetstrokecolor{currentstroke}%
\pgfsetstrokeopacity{0.315962}%
\pgfsetdash{}{0pt}%
\pgfpathmoveto{\pgfqpoint{1.821817in}{3.267333in}}%
\pgfpathcurveto{\pgfqpoint{1.830054in}{3.267333in}}{\pgfqpoint{1.837954in}{3.270606in}}{\pgfqpoint{1.843778in}{3.276430in}}%
\pgfpathcurveto{\pgfqpoint{1.849602in}{3.282254in}}{\pgfqpoint{1.852874in}{3.290154in}}{\pgfqpoint{1.852874in}{3.298390in}}%
\pgfpathcurveto{\pgfqpoint{1.852874in}{3.306626in}}{\pgfqpoint{1.849602in}{3.314526in}}{\pgfqpoint{1.843778in}{3.320350in}}%
\pgfpathcurveto{\pgfqpoint{1.837954in}{3.326174in}}{\pgfqpoint{1.830054in}{3.329446in}}{\pgfqpoint{1.821817in}{3.329446in}}%
\pgfpathcurveto{\pgfqpoint{1.813581in}{3.329446in}}{\pgfqpoint{1.805681in}{3.326174in}}{\pgfqpoint{1.799857in}{3.320350in}}%
\pgfpathcurveto{\pgfqpoint{1.794033in}{3.314526in}}{\pgfqpoint{1.790761in}{3.306626in}}{\pgfqpoint{1.790761in}{3.298390in}}%
\pgfpathcurveto{\pgfqpoint{1.790761in}{3.290154in}}{\pgfqpoint{1.794033in}{3.282254in}}{\pgfqpoint{1.799857in}{3.276430in}}%
\pgfpathcurveto{\pgfqpoint{1.805681in}{3.270606in}}{\pgfqpoint{1.813581in}{3.267333in}}{\pgfqpoint{1.821817in}{3.267333in}}%
\pgfpathclose%
\pgfusepath{stroke,fill}%
\end{pgfscope}%
\begin{pgfscope}%
\pgfpathrectangle{\pgfqpoint{0.100000in}{0.212622in}}{\pgfqpoint{3.696000in}{3.696000in}}%
\pgfusepath{clip}%
\pgfsetbuttcap%
\pgfsetroundjoin%
\definecolor{currentfill}{rgb}{0.121569,0.466667,0.705882}%
\pgfsetfillcolor{currentfill}%
\pgfsetfillopacity{0.316027}%
\pgfsetlinewidth{1.003750pt}%
\definecolor{currentstroke}{rgb}{0.121569,0.466667,0.705882}%
\pgfsetstrokecolor{currentstroke}%
\pgfsetstrokeopacity{0.316027}%
\pgfsetdash{}{0pt}%
\pgfpathmoveto{\pgfqpoint{1.919703in}{3.246958in}}%
\pgfpathcurveto{\pgfqpoint{1.927939in}{3.246958in}}{\pgfqpoint{1.935839in}{3.250230in}}{\pgfqpoint{1.941663in}{3.256054in}}%
\pgfpathcurveto{\pgfqpoint{1.947487in}{3.261878in}}{\pgfqpoint{1.950759in}{3.269778in}}{\pgfqpoint{1.950759in}{3.278015in}}%
\pgfpathcurveto{\pgfqpoint{1.950759in}{3.286251in}}{\pgfqpoint{1.947487in}{3.294151in}}{\pgfqpoint{1.941663in}{3.299975in}}%
\pgfpathcurveto{\pgfqpoint{1.935839in}{3.305799in}}{\pgfqpoint{1.927939in}{3.309071in}}{\pgfqpoint{1.919703in}{3.309071in}}%
\pgfpathcurveto{\pgfqpoint{1.911467in}{3.309071in}}{\pgfqpoint{1.903567in}{3.305799in}}{\pgfqpoint{1.897743in}{3.299975in}}%
\pgfpathcurveto{\pgfqpoint{1.891919in}{3.294151in}}{\pgfqpoint{1.888646in}{3.286251in}}{\pgfqpoint{1.888646in}{3.278015in}}%
\pgfpathcurveto{\pgfqpoint{1.888646in}{3.269778in}}{\pgfqpoint{1.891919in}{3.261878in}}{\pgfqpoint{1.897743in}{3.256054in}}%
\pgfpathcurveto{\pgfqpoint{1.903567in}{3.250230in}}{\pgfqpoint{1.911467in}{3.246958in}}{\pgfqpoint{1.919703in}{3.246958in}}%
\pgfpathclose%
\pgfusepath{stroke,fill}%
\end{pgfscope}%
\begin{pgfscope}%
\pgfpathrectangle{\pgfqpoint{0.100000in}{0.212622in}}{\pgfqpoint{3.696000in}{3.696000in}}%
\pgfusepath{clip}%
\pgfsetbuttcap%
\pgfsetroundjoin%
\definecolor{currentfill}{rgb}{0.121569,0.466667,0.705882}%
\pgfsetfillcolor{currentfill}%
\pgfsetfillopacity{0.316373}%
\pgfsetlinewidth{1.003750pt}%
\definecolor{currentstroke}{rgb}{0.121569,0.466667,0.705882}%
\pgfsetstrokecolor{currentstroke}%
\pgfsetstrokeopacity{0.316373}%
\pgfsetdash{}{0pt}%
\pgfpathmoveto{\pgfqpoint{1.920147in}{3.245385in}}%
\pgfpathcurveto{\pgfqpoint{1.928384in}{3.245385in}}{\pgfqpoint{1.936284in}{3.248658in}}{\pgfqpoint{1.942108in}{3.254482in}}%
\pgfpathcurveto{\pgfqpoint{1.947932in}{3.260306in}}{\pgfqpoint{1.951204in}{3.268206in}}{\pgfqpoint{1.951204in}{3.276442in}}%
\pgfpathcurveto{\pgfqpoint{1.951204in}{3.284678in}}{\pgfqpoint{1.947932in}{3.292578in}}{\pgfqpoint{1.942108in}{3.298402in}}%
\pgfpathcurveto{\pgfqpoint{1.936284in}{3.304226in}}{\pgfqpoint{1.928384in}{3.307498in}}{\pgfqpoint{1.920147in}{3.307498in}}%
\pgfpathcurveto{\pgfqpoint{1.911911in}{3.307498in}}{\pgfqpoint{1.904011in}{3.304226in}}{\pgfqpoint{1.898187in}{3.298402in}}%
\pgfpathcurveto{\pgfqpoint{1.892363in}{3.292578in}}{\pgfqpoint{1.889091in}{3.284678in}}{\pgfqpoint{1.889091in}{3.276442in}}%
\pgfpathcurveto{\pgfqpoint{1.889091in}{3.268206in}}{\pgfqpoint{1.892363in}{3.260306in}}{\pgfqpoint{1.898187in}{3.254482in}}%
\pgfpathcurveto{\pgfqpoint{1.904011in}{3.248658in}}{\pgfqpoint{1.911911in}{3.245385in}}{\pgfqpoint{1.920147in}{3.245385in}}%
\pgfpathclose%
\pgfusepath{stroke,fill}%
\end{pgfscope}%
\begin{pgfscope}%
\pgfpathrectangle{\pgfqpoint{0.100000in}{0.212622in}}{\pgfqpoint{3.696000in}{3.696000in}}%
\pgfusepath{clip}%
\pgfsetbuttcap%
\pgfsetroundjoin%
\definecolor{currentfill}{rgb}{0.121569,0.466667,0.705882}%
\pgfsetfillcolor{currentfill}%
\pgfsetfillopacity{0.316882}%
\pgfsetlinewidth{1.003750pt}%
\definecolor{currentstroke}{rgb}{0.121569,0.466667,0.705882}%
\pgfsetstrokecolor{currentstroke}%
\pgfsetstrokeopacity{0.316882}%
\pgfsetdash{}{0pt}%
\pgfpathmoveto{\pgfqpoint{1.920588in}{3.243583in}}%
\pgfpathcurveto{\pgfqpoint{1.928824in}{3.243583in}}{\pgfqpoint{1.936724in}{3.246855in}}{\pgfqpoint{1.942548in}{3.252679in}}%
\pgfpathcurveto{\pgfqpoint{1.948372in}{3.258503in}}{\pgfqpoint{1.951644in}{3.266403in}}{\pgfqpoint{1.951644in}{3.274639in}}%
\pgfpathcurveto{\pgfqpoint{1.951644in}{3.282876in}}{\pgfqpoint{1.948372in}{3.290776in}}{\pgfqpoint{1.942548in}{3.296600in}}%
\pgfpathcurveto{\pgfqpoint{1.936724in}{3.302424in}}{\pgfqpoint{1.928824in}{3.305696in}}{\pgfqpoint{1.920588in}{3.305696in}}%
\pgfpathcurveto{\pgfqpoint{1.912352in}{3.305696in}}{\pgfqpoint{1.904452in}{3.302424in}}{\pgfqpoint{1.898628in}{3.296600in}}%
\pgfpathcurveto{\pgfqpoint{1.892804in}{3.290776in}}{\pgfqpoint{1.889531in}{3.282876in}}{\pgfqpoint{1.889531in}{3.274639in}}%
\pgfpathcurveto{\pgfqpoint{1.889531in}{3.266403in}}{\pgfqpoint{1.892804in}{3.258503in}}{\pgfqpoint{1.898628in}{3.252679in}}%
\pgfpathcurveto{\pgfqpoint{1.904452in}{3.246855in}}{\pgfqpoint{1.912352in}{3.243583in}}{\pgfqpoint{1.920588in}{3.243583in}}%
\pgfpathclose%
\pgfusepath{stroke,fill}%
\end{pgfscope}%
\begin{pgfscope}%
\pgfpathrectangle{\pgfqpoint{0.100000in}{0.212622in}}{\pgfqpoint{3.696000in}{3.696000in}}%
\pgfusepath{clip}%
\pgfsetbuttcap%
\pgfsetroundjoin%
\definecolor{currentfill}{rgb}{0.121569,0.466667,0.705882}%
\pgfsetfillcolor{currentfill}%
\pgfsetfillopacity{0.317055}%
\pgfsetlinewidth{1.003750pt}%
\definecolor{currentstroke}{rgb}{0.121569,0.466667,0.705882}%
\pgfsetstrokecolor{currentstroke}%
\pgfsetstrokeopacity{0.317055}%
\pgfsetdash{}{0pt}%
\pgfpathmoveto{\pgfqpoint{1.818145in}{3.261519in}}%
\pgfpathcurveto{\pgfqpoint{1.826381in}{3.261519in}}{\pgfqpoint{1.834282in}{3.264791in}}{\pgfqpoint{1.840105in}{3.270615in}}%
\pgfpathcurveto{\pgfqpoint{1.845929in}{3.276439in}}{\pgfqpoint{1.849202in}{3.284339in}}{\pgfqpoint{1.849202in}{3.292575in}}%
\pgfpathcurveto{\pgfqpoint{1.849202in}{3.300811in}}{\pgfqpoint{1.845929in}{3.308711in}}{\pgfqpoint{1.840105in}{3.314535in}}%
\pgfpathcurveto{\pgfqpoint{1.834282in}{3.320359in}}{\pgfqpoint{1.826381in}{3.323632in}}{\pgfqpoint{1.818145in}{3.323632in}}%
\pgfpathcurveto{\pgfqpoint{1.809909in}{3.323632in}}{\pgfqpoint{1.802009in}{3.320359in}}{\pgfqpoint{1.796185in}{3.314535in}}%
\pgfpathcurveto{\pgfqpoint{1.790361in}{3.308711in}}{\pgfqpoint{1.787089in}{3.300811in}}{\pgfqpoint{1.787089in}{3.292575in}}%
\pgfpathcurveto{\pgfqpoint{1.787089in}{3.284339in}}{\pgfqpoint{1.790361in}{3.276439in}}{\pgfqpoint{1.796185in}{3.270615in}}%
\pgfpathcurveto{\pgfqpoint{1.802009in}{3.264791in}}{\pgfqpoint{1.809909in}{3.261519in}}{\pgfqpoint{1.818145in}{3.261519in}}%
\pgfpathclose%
\pgfusepath{stroke,fill}%
\end{pgfscope}%
\begin{pgfscope}%
\pgfpathrectangle{\pgfqpoint{0.100000in}{0.212622in}}{\pgfqpoint{3.696000in}{3.696000in}}%
\pgfusepath{clip}%
\pgfsetbuttcap%
\pgfsetroundjoin%
\definecolor{currentfill}{rgb}{0.121569,0.466667,0.705882}%
\pgfsetfillcolor{currentfill}%
\pgfsetfillopacity{0.317647}%
\pgfsetlinewidth{1.003750pt}%
\definecolor{currentstroke}{rgb}{0.121569,0.466667,0.705882}%
\pgfsetstrokecolor{currentstroke}%
\pgfsetstrokeopacity{0.317647}%
\pgfsetdash{}{0pt}%
\pgfpathmoveto{\pgfqpoint{1.921617in}{3.240062in}}%
\pgfpathcurveto{\pgfqpoint{1.929854in}{3.240062in}}{\pgfqpoint{1.937754in}{3.243335in}}{\pgfqpoint{1.943578in}{3.249159in}}%
\pgfpathcurveto{\pgfqpoint{1.949402in}{3.254983in}}{\pgfqpoint{1.952674in}{3.262883in}}{\pgfqpoint{1.952674in}{3.271119in}}%
\pgfpathcurveto{\pgfqpoint{1.952674in}{3.279355in}}{\pgfqpoint{1.949402in}{3.287255in}}{\pgfqpoint{1.943578in}{3.293079in}}%
\pgfpathcurveto{\pgfqpoint{1.937754in}{3.298903in}}{\pgfqpoint{1.929854in}{3.302175in}}{\pgfqpoint{1.921617in}{3.302175in}}%
\pgfpathcurveto{\pgfqpoint{1.913381in}{3.302175in}}{\pgfqpoint{1.905481in}{3.298903in}}{\pgfqpoint{1.899657in}{3.293079in}}%
\pgfpathcurveto{\pgfqpoint{1.893833in}{3.287255in}}{\pgfqpoint{1.890561in}{3.279355in}}{\pgfqpoint{1.890561in}{3.271119in}}%
\pgfpathcurveto{\pgfqpoint{1.890561in}{3.262883in}}{\pgfqpoint{1.893833in}{3.254983in}}{\pgfqpoint{1.899657in}{3.249159in}}%
\pgfpathcurveto{\pgfqpoint{1.905481in}{3.243335in}}{\pgfqpoint{1.913381in}{3.240062in}}{\pgfqpoint{1.921617in}{3.240062in}}%
\pgfpathclose%
\pgfusepath{stroke,fill}%
\end{pgfscope}%
\begin{pgfscope}%
\pgfpathrectangle{\pgfqpoint{0.100000in}{0.212622in}}{\pgfqpoint{3.696000in}{3.696000in}}%
\pgfusepath{clip}%
\pgfsetbuttcap%
\pgfsetroundjoin%
\definecolor{currentfill}{rgb}{0.121569,0.466667,0.705882}%
\pgfsetfillcolor{currentfill}%
\pgfsetfillopacity{0.317878}%
\pgfsetlinewidth{1.003750pt}%
\definecolor{currentstroke}{rgb}{0.121569,0.466667,0.705882}%
\pgfsetstrokecolor{currentstroke}%
\pgfsetstrokeopacity{0.317878}%
\pgfsetdash{}{0pt}%
\pgfpathmoveto{\pgfqpoint{1.816026in}{3.257211in}}%
\pgfpathcurveto{\pgfqpoint{1.824263in}{3.257211in}}{\pgfqpoint{1.832163in}{3.260483in}}{\pgfqpoint{1.837987in}{3.266307in}}%
\pgfpathcurveto{\pgfqpoint{1.843811in}{3.272131in}}{\pgfqpoint{1.847083in}{3.280031in}}{\pgfqpoint{1.847083in}{3.288267in}}%
\pgfpathcurveto{\pgfqpoint{1.847083in}{3.296504in}}{\pgfqpoint{1.843811in}{3.304404in}}{\pgfqpoint{1.837987in}{3.310228in}}%
\pgfpathcurveto{\pgfqpoint{1.832163in}{3.316052in}}{\pgfqpoint{1.824263in}{3.319324in}}{\pgfqpoint{1.816026in}{3.319324in}}%
\pgfpathcurveto{\pgfqpoint{1.807790in}{3.319324in}}{\pgfqpoint{1.799890in}{3.316052in}}{\pgfqpoint{1.794066in}{3.310228in}}%
\pgfpathcurveto{\pgfqpoint{1.788242in}{3.304404in}}{\pgfqpoint{1.784970in}{3.296504in}}{\pgfqpoint{1.784970in}{3.288267in}}%
\pgfpathcurveto{\pgfqpoint{1.784970in}{3.280031in}}{\pgfqpoint{1.788242in}{3.272131in}}{\pgfqpoint{1.794066in}{3.266307in}}%
\pgfpathcurveto{\pgfqpoint{1.799890in}{3.260483in}}{\pgfqpoint{1.807790in}{3.257211in}}{\pgfqpoint{1.816026in}{3.257211in}}%
\pgfpathclose%
\pgfusepath{stroke,fill}%
\end{pgfscope}%
\begin{pgfscope}%
\pgfpathrectangle{\pgfqpoint{0.100000in}{0.212622in}}{\pgfqpoint{3.696000in}{3.696000in}}%
\pgfusepath{clip}%
\pgfsetbuttcap%
\pgfsetroundjoin%
\definecolor{currentfill}{rgb}{0.121569,0.466667,0.705882}%
\pgfsetfillcolor{currentfill}%
\pgfsetfillopacity{0.318628}%
\pgfsetlinewidth{1.003750pt}%
\definecolor{currentstroke}{rgb}{0.121569,0.466667,0.705882}%
\pgfsetstrokecolor{currentstroke}%
\pgfsetstrokeopacity{0.318628}%
\pgfsetdash{}{0pt}%
\pgfpathmoveto{\pgfqpoint{1.813778in}{3.253540in}}%
\pgfpathcurveto{\pgfqpoint{1.822014in}{3.253540in}}{\pgfqpoint{1.829914in}{3.256812in}}{\pgfqpoint{1.835738in}{3.262636in}}%
\pgfpathcurveto{\pgfqpoint{1.841562in}{3.268460in}}{\pgfqpoint{1.844835in}{3.276360in}}{\pgfqpoint{1.844835in}{3.284597in}}%
\pgfpathcurveto{\pgfqpoint{1.844835in}{3.292833in}}{\pgfqpoint{1.841562in}{3.300733in}}{\pgfqpoint{1.835738in}{3.306557in}}%
\pgfpathcurveto{\pgfqpoint{1.829914in}{3.312381in}}{\pgfqpoint{1.822014in}{3.315653in}}{\pgfqpoint{1.813778in}{3.315653in}}%
\pgfpathcurveto{\pgfqpoint{1.805542in}{3.315653in}}{\pgfqpoint{1.797642in}{3.312381in}}{\pgfqpoint{1.791818in}{3.306557in}}%
\pgfpathcurveto{\pgfqpoint{1.785994in}{3.300733in}}{\pgfqpoint{1.782722in}{3.292833in}}{\pgfqpoint{1.782722in}{3.284597in}}%
\pgfpathcurveto{\pgfqpoint{1.782722in}{3.276360in}}{\pgfqpoint{1.785994in}{3.268460in}}{\pgfqpoint{1.791818in}{3.262636in}}%
\pgfpathcurveto{\pgfqpoint{1.797642in}{3.256812in}}{\pgfqpoint{1.805542in}{3.253540in}}{\pgfqpoint{1.813778in}{3.253540in}}%
\pgfpathclose%
\pgfusepath{stroke,fill}%
\end{pgfscope}%
\begin{pgfscope}%
\pgfpathrectangle{\pgfqpoint{0.100000in}{0.212622in}}{\pgfqpoint{3.696000in}{3.696000in}}%
\pgfusepath{clip}%
\pgfsetbuttcap%
\pgfsetroundjoin%
\definecolor{currentfill}{rgb}{0.121569,0.466667,0.705882}%
\pgfsetfillcolor{currentfill}%
\pgfsetfillopacity{0.318631}%
\pgfsetlinewidth{1.003750pt}%
\definecolor{currentstroke}{rgb}{0.121569,0.466667,0.705882}%
\pgfsetstrokecolor{currentstroke}%
\pgfsetstrokeopacity{0.318631}%
\pgfsetdash{}{0pt}%
\pgfpathmoveto{\pgfqpoint{1.922553in}{3.236064in}}%
\pgfpathcurveto{\pgfqpoint{1.930789in}{3.236064in}}{\pgfqpoint{1.938689in}{3.239337in}}{\pgfqpoint{1.944513in}{3.245161in}}%
\pgfpathcurveto{\pgfqpoint{1.950337in}{3.250985in}}{\pgfqpoint{1.953609in}{3.258885in}}{\pgfqpoint{1.953609in}{3.267121in}}%
\pgfpathcurveto{\pgfqpoint{1.953609in}{3.275357in}}{\pgfqpoint{1.950337in}{3.283257in}}{\pgfqpoint{1.944513in}{3.289081in}}%
\pgfpathcurveto{\pgfqpoint{1.938689in}{3.294905in}}{\pgfqpoint{1.930789in}{3.298177in}}{\pgfqpoint{1.922553in}{3.298177in}}%
\pgfpathcurveto{\pgfqpoint{1.914317in}{3.298177in}}{\pgfqpoint{1.906416in}{3.294905in}}{\pgfqpoint{1.900593in}{3.289081in}}%
\pgfpathcurveto{\pgfqpoint{1.894769in}{3.283257in}}{\pgfqpoint{1.891496in}{3.275357in}}{\pgfqpoint{1.891496in}{3.267121in}}%
\pgfpathcurveto{\pgfqpoint{1.891496in}{3.258885in}}{\pgfqpoint{1.894769in}{3.250985in}}{\pgfqpoint{1.900593in}{3.245161in}}%
\pgfpathcurveto{\pgfqpoint{1.906416in}{3.239337in}}{\pgfqpoint{1.914317in}{3.236064in}}{\pgfqpoint{1.922553in}{3.236064in}}%
\pgfpathclose%
\pgfusepath{stroke,fill}%
\end{pgfscope}%
\begin{pgfscope}%
\pgfpathrectangle{\pgfqpoint{0.100000in}{0.212622in}}{\pgfqpoint{3.696000in}{3.696000in}}%
\pgfusepath{clip}%
\pgfsetbuttcap%
\pgfsetroundjoin%
\definecolor{currentfill}{rgb}{0.121569,0.466667,0.705882}%
\pgfsetfillcolor{currentfill}%
\pgfsetfillopacity{0.319801}%
\pgfsetlinewidth{1.003750pt}%
\definecolor{currentstroke}{rgb}{0.121569,0.466667,0.705882}%
\pgfsetstrokecolor{currentstroke}%
\pgfsetstrokeopacity{0.319801}%
\pgfsetdash{}{0pt}%
\pgfpathmoveto{\pgfqpoint{1.923467in}{3.231709in}}%
\pgfpathcurveto{\pgfqpoint{1.931703in}{3.231709in}}{\pgfqpoint{1.939603in}{3.234981in}}{\pgfqpoint{1.945427in}{3.240805in}}%
\pgfpathcurveto{\pgfqpoint{1.951251in}{3.246629in}}{\pgfqpoint{1.954523in}{3.254529in}}{\pgfqpoint{1.954523in}{3.262765in}}%
\pgfpathcurveto{\pgfqpoint{1.954523in}{3.271002in}}{\pgfqpoint{1.951251in}{3.278902in}}{\pgfqpoint{1.945427in}{3.284726in}}%
\pgfpathcurveto{\pgfqpoint{1.939603in}{3.290550in}}{\pgfqpoint{1.931703in}{3.293822in}}{\pgfqpoint{1.923467in}{3.293822in}}%
\pgfpathcurveto{\pgfqpoint{1.915230in}{3.293822in}}{\pgfqpoint{1.907330in}{3.290550in}}{\pgfqpoint{1.901506in}{3.284726in}}%
\pgfpathcurveto{\pgfqpoint{1.895682in}{3.278902in}}{\pgfqpoint{1.892410in}{3.271002in}}{\pgfqpoint{1.892410in}{3.262765in}}%
\pgfpathcurveto{\pgfqpoint{1.892410in}{3.254529in}}{\pgfqpoint{1.895682in}{3.246629in}}{\pgfqpoint{1.901506in}{3.240805in}}%
\pgfpathcurveto{\pgfqpoint{1.907330in}{3.234981in}}{\pgfqpoint{1.915230in}{3.231709in}}{\pgfqpoint{1.923467in}{3.231709in}}%
\pgfpathclose%
\pgfusepath{stroke,fill}%
\end{pgfscope}%
\begin{pgfscope}%
\pgfpathrectangle{\pgfqpoint{0.100000in}{0.212622in}}{\pgfqpoint{3.696000in}{3.696000in}}%
\pgfusepath{clip}%
\pgfsetbuttcap%
\pgfsetroundjoin%
\definecolor{currentfill}{rgb}{0.121569,0.466667,0.705882}%
\pgfsetfillcolor{currentfill}%
\pgfsetfillopacity{0.320001}%
\pgfsetlinewidth{1.003750pt}%
\definecolor{currentstroke}{rgb}{0.121569,0.466667,0.705882}%
\pgfsetstrokecolor{currentstroke}%
\pgfsetstrokeopacity{0.320001}%
\pgfsetdash{}{0pt}%
\pgfpathmoveto{\pgfqpoint{1.809662in}{3.246942in}}%
\pgfpathcurveto{\pgfqpoint{1.817898in}{3.246942in}}{\pgfqpoint{1.825798in}{3.250215in}}{\pgfqpoint{1.831622in}{3.256039in}}%
\pgfpathcurveto{\pgfqpoint{1.837446in}{3.261863in}}{\pgfqpoint{1.840718in}{3.269763in}}{\pgfqpoint{1.840718in}{3.277999in}}%
\pgfpathcurveto{\pgfqpoint{1.840718in}{3.286235in}}{\pgfqpoint{1.837446in}{3.294135in}}{\pgfqpoint{1.831622in}{3.299959in}}%
\pgfpathcurveto{\pgfqpoint{1.825798in}{3.305783in}}{\pgfqpoint{1.817898in}{3.309055in}}{\pgfqpoint{1.809662in}{3.309055in}}%
\pgfpathcurveto{\pgfqpoint{1.801426in}{3.309055in}}{\pgfqpoint{1.793526in}{3.305783in}}{\pgfqpoint{1.787702in}{3.299959in}}%
\pgfpathcurveto{\pgfqpoint{1.781878in}{3.294135in}}{\pgfqpoint{1.778605in}{3.286235in}}{\pgfqpoint{1.778605in}{3.277999in}}%
\pgfpathcurveto{\pgfqpoint{1.778605in}{3.269763in}}{\pgfqpoint{1.781878in}{3.261863in}}{\pgfqpoint{1.787702in}{3.256039in}}%
\pgfpathcurveto{\pgfqpoint{1.793526in}{3.250215in}}{\pgfqpoint{1.801426in}{3.246942in}}{\pgfqpoint{1.809662in}{3.246942in}}%
\pgfpathclose%
\pgfusepath{stroke,fill}%
\end{pgfscope}%
\begin{pgfscope}%
\pgfpathrectangle{\pgfqpoint{0.100000in}{0.212622in}}{\pgfqpoint{3.696000in}{3.696000in}}%
\pgfusepath{clip}%
\pgfsetbuttcap%
\pgfsetroundjoin%
\definecolor{currentfill}{rgb}{0.121569,0.466667,0.705882}%
\pgfsetfillcolor{currentfill}%
\pgfsetfillopacity{0.320864}%
\pgfsetlinewidth{1.003750pt}%
\definecolor{currentstroke}{rgb}{0.121569,0.466667,0.705882}%
\pgfsetstrokecolor{currentstroke}%
\pgfsetstrokeopacity{0.320864}%
\pgfsetdash{}{0pt}%
\pgfpathmoveto{\pgfqpoint{1.924793in}{3.226573in}}%
\pgfpathcurveto{\pgfqpoint{1.933029in}{3.226573in}}{\pgfqpoint{1.940929in}{3.229845in}}{\pgfqpoint{1.946753in}{3.235669in}}%
\pgfpathcurveto{\pgfqpoint{1.952577in}{3.241493in}}{\pgfqpoint{1.955849in}{3.249393in}}{\pgfqpoint{1.955849in}{3.257629in}}%
\pgfpathcurveto{\pgfqpoint{1.955849in}{3.265866in}}{\pgfqpoint{1.952577in}{3.273766in}}{\pgfqpoint{1.946753in}{3.279590in}}%
\pgfpathcurveto{\pgfqpoint{1.940929in}{3.285414in}}{\pgfqpoint{1.933029in}{3.288686in}}{\pgfqpoint{1.924793in}{3.288686in}}%
\pgfpathcurveto{\pgfqpoint{1.916557in}{3.288686in}}{\pgfqpoint{1.908657in}{3.285414in}}{\pgfqpoint{1.902833in}{3.279590in}}%
\pgfpathcurveto{\pgfqpoint{1.897009in}{3.273766in}}{\pgfqpoint{1.893736in}{3.265866in}}{\pgfqpoint{1.893736in}{3.257629in}}%
\pgfpathcurveto{\pgfqpoint{1.893736in}{3.249393in}}{\pgfqpoint{1.897009in}{3.241493in}}{\pgfqpoint{1.902833in}{3.235669in}}%
\pgfpathcurveto{\pgfqpoint{1.908657in}{3.229845in}}{\pgfqpoint{1.916557in}{3.226573in}}{\pgfqpoint{1.924793in}{3.226573in}}%
\pgfpathclose%
\pgfusepath{stroke,fill}%
\end{pgfscope}%
\begin{pgfscope}%
\pgfpathrectangle{\pgfqpoint{0.100000in}{0.212622in}}{\pgfqpoint{3.696000in}{3.696000in}}%
\pgfusepath{clip}%
\pgfsetbuttcap%
\pgfsetroundjoin%
\definecolor{currentfill}{rgb}{0.121569,0.466667,0.705882}%
\pgfsetfillcolor{currentfill}%
\pgfsetfillopacity{0.321329}%
\pgfsetlinewidth{1.003750pt}%
\definecolor{currentstroke}{rgb}{0.121569,0.466667,0.705882}%
\pgfsetstrokecolor{currentstroke}%
\pgfsetstrokeopacity{0.321329}%
\pgfsetdash{}{0pt}%
\pgfpathmoveto{\pgfqpoint{1.806312in}{3.240144in}}%
\pgfpathcurveto{\pgfqpoint{1.814548in}{3.240144in}}{\pgfqpoint{1.822448in}{3.243416in}}{\pgfqpoint{1.828272in}{3.249240in}}%
\pgfpathcurveto{\pgfqpoint{1.834096in}{3.255064in}}{\pgfqpoint{1.837369in}{3.262964in}}{\pgfqpoint{1.837369in}{3.271200in}}%
\pgfpathcurveto{\pgfqpoint{1.837369in}{3.279436in}}{\pgfqpoint{1.834096in}{3.287336in}}{\pgfqpoint{1.828272in}{3.293160in}}%
\pgfpathcurveto{\pgfqpoint{1.822448in}{3.298984in}}{\pgfqpoint{1.814548in}{3.302257in}}{\pgfqpoint{1.806312in}{3.302257in}}%
\pgfpathcurveto{\pgfqpoint{1.798076in}{3.302257in}}{\pgfqpoint{1.790176in}{3.298984in}}{\pgfqpoint{1.784352in}{3.293160in}}%
\pgfpathcurveto{\pgfqpoint{1.778528in}{3.287336in}}{\pgfqpoint{1.775256in}{3.279436in}}{\pgfqpoint{1.775256in}{3.271200in}}%
\pgfpathcurveto{\pgfqpoint{1.775256in}{3.262964in}}{\pgfqpoint{1.778528in}{3.255064in}}{\pgfqpoint{1.784352in}{3.249240in}}%
\pgfpathcurveto{\pgfqpoint{1.790176in}{3.243416in}}{\pgfqpoint{1.798076in}{3.240144in}}{\pgfqpoint{1.806312in}{3.240144in}}%
\pgfpathclose%
\pgfusepath{stroke,fill}%
\end{pgfscope}%
\begin{pgfscope}%
\pgfpathrectangle{\pgfqpoint{0.100000in}{0.212622in}}{\pgfqpoint{3.696000in}{3.696000in}}%
\pgfusepath{clip}%
\pgfsetbuttcap%
\pgfsetroundjoin%
\definecolor{currentfill}{rgb}{0.121569,0.466667,0.705882}%
\pgfsetfillcolor{currentfill}%
\pgfsetfillopacity{0.322126}%
\pgfsetlinewidth{1.003750pt}%
\definecolor{currentstroke}{rgb}{0.121569,0.466667,0.705882}%
\pgfsetstrokecolor{currentstroke}%
\pgfsetstrokeopacity{0.322126}%
\pgfsetdash{}{0pt}%
\pgfpathmoveto{\pgfqpoint{1.803783in}{3.236266in}}%
\pgfpathcurveto{\pgfqpoint{1.812019in}{3.236266in}}{\pgfqpoint{1.819919in}{3.239539in}}{\pgfqpoint{1.825743in}{3.245362in}}%
\pgfpathcurveto{\pgfqpoint{1.831567in}{3.251186in}}{\pgfqpoint{1.834839in}{3.259086in}}{\pgfqpoint{1.834839in}{3.267323in}}%
\pgfpathcurveto{\pgfqpoint{1.834839in}{3.275559in}}{\pgfqpoint{1.831567in}{3.283459in}}{\pgfqpoint{1.825743in}{3.289283in}}%
\pgfpathcurveto{\pgfqpoint{1.819919in}{3.295107in}}{\pgfqpoint{1.812019in}{3.298379in}}{\pgfqpoint{1.803783in}{3.298379in}}%
\pgfpathcurveto{\pgfqpoint{1.795546in}{3.298379in}}{\pgfqpoint{1.787646in}{3.295107in}}{\pgfqpoint{1.781822in}{3.289283in}}%
\pgfpathcurveto{\pgfqpoint{1.775998in}{3.283459in}}{\pgfqpoint{1.772726in}{3.275559in}}{\pgfqpoint{1.772726in}{3.267323in}}%
\pgfpathcurveto{\pgfqpoint{1.772726in}{3.259086in}}{\pgfqpoint{1.775998in}{3.251186in}}{\pgfqpoint{1.781822in}{3.245362in}}%
\pgfpathcurveto{\pgfqpoint{1.787646in}{3.239539in}}{\pgfqpoint{1.795546in}{3.236266in}}{\pgfqpoint{1.803783in}{3.236266in}}%
\pgfpathclose%
\pgfusepath{stroke,fill}%
\end{pgfscope}%
\begin{pgfscope}%
\pgfpathrectangle{\pgfqpoint{0.100000in}{0.212622in}}{\pgfqpoint{3.696000in}{3.696000in}}%
\pgfusepath{clip}%
\pgfsetbuttcap%
\pgfsetroundjoin%
\definecolor{currentfill}{rgb}{0.121569,0.466667,0.705882}%
\pgfsetfillcolor{currentfill}%
\pgfsetfillopacity{0.322573}%
\pgfsetlinewidth{1.003750pt}%
\definecolor{currentstroke}{rgb}{0.121569,0.466667,0.705882}%
\pgfsetstrokecolor{currentstroke}%
\pgfsetstrokeopacity{0.322573}%
\pgfsetdash{}{0pt}%
\pgfpathmoveto{\pgfqpoint{1.925917in}{3.220118in}}%
\pgfpathcurveto{\pgfqpoint{1.934154in}{3.220118in}}{\pgfqpoint{1.942054in}{3.223391in}}{\pgfqpoint{1.947878in}{3.229215in}}%
\pgfpathcurveto{\pgfqpoint{1.953702in}{3.235039in}}{\pgfqpoint{1.956974in}{3.242939in}}{\pgfqpoint{1.956974in}{3.251175in}}%
\pgfpathcurveto{\pgfqpoint{1.956974in}{3.259411in}}{\pgfqpoint{1.953702in}{3.267311in}}{\pgfqpoint{1.947878in}{3.273135in}}%
\pgfpathcurveto{\pgfqpoint{1.942054in}{3.278959in}}{\pgfqpoint{1.934154in}{3.282231in}}{\pgfqpoint{1.925917in}{3.282231in}}%
\pgfpathcurveto{\pgfqpoint{1.917681in}{3.282231in}}{\pgfqpoint{1.909781in}{3.278959in}}{\pgfqpoint{1.903957in}{3.273135in}}%
\pgfpathcurveto{\pgfqpoint{1.898133in}{3.267311in}}{\pgfqpoint{1.894861in}{3.259411in}}{\pgfqpoint{1.894861in}{3.251175in}}%
\pgfpathcurveto{\pgfqpoint{1.894861in}{3.242939in}}{\pgfqpoint{1.898133in}{3.235039in}}{\pgfqpoint{1.903957in}{3.229215in}}%
\pgfpathcurveto{\pgfqpoint{1.909781in}{3.223391in}}{\pgfqpoint{1.917681in}{3.220118in}}{\pgfqpoint{1.925917in}{3.220118in}}%
\pgfpathclose%
\pgfusepath{stroke,fill}%
\end{pgfscope}%
\begin{pgfscope}%
\pgfpathrectangle{\pgfqpoint{0.100000in}{0.212622in}}{\pgfqpoint{3.696000in}{3.696000in}}%
\pgfusepath{clip}%
\pgfsetbuttcap%
\pgfsetroundjoin%
\definecolor{currentfill}{rgb}{0.121569,0.466667,0.705882}%
\pgfsetfillcolor{currentfill}%
\pgfsetfillopacity{0.323443}%
\pgfsetlinewidth{1.003750pt}%
\definecolor{currentstroke}{rgb}{0.121569,0.466667,0.705882}%
\pgfsetstrokecolor{currentstroke}%
\pgfsetstrokeopacity{0.323443}%
\pgfsetdash{}{0pt}%
\pgfpathmoveto{\pgfqpoint{1.926687in}{3.216425in}}%
\pgfpathcurveto{\pgfqpoint{1.934924in}{3.216425in}}{\pgfqpoint{1.942824in}{3.219697in}}{\pgfqpoint{1.948648in}{3.225521in}}%
\pgfpathcurveto{\pgfqpoint{1.954472in}{3.231345in}}{\pgfqpoint{1.957744in}{3.239245in}}{\pgfqpoint{1.957744in}{3.247481in}}%
\pgfpathcurveto{\pgfqpoint{1.957744in}{3.255718in}}{\pgfqpoint{1.954472in}{3.263618in}}{\pgfqpoint{1.948648in}{3.269441in}}%
\pgfpathcurveto{\pgfqpoint{1.942824in}{3.275265in}}{\pgfqpoint{1.934924in}{3.278538in}}{\pgfqpoint{1.926687in}{3.278538in}}%
\pgfpathcurveto{\pgfqpoint{1.918451in}{3.278538in}}{\pgfqpoint{1.910551in}{3.275265in}}{\pgfqpoint{1.904727in}{3.269441in}}%
\pgfpathcurveto{\pgfqpoint{1.898903in}{3.263618in}}{\pgfqpoint{1.895631in}{3.255718in}}{\pgfqpoint{1.895631in}{3.247481in}}%
\pgfpathcurveto{\pgfqpoint{1.895631in}{3.239245in}}{\pgfqpoint{1.898903in}{3.231345in}}{\pgfqpoint{1.904727in}{3.225521in}}%
\pgfpathcurveto{\pgfqpoint{1.910551in}{3.219697in}}{\pgfqpoint{1.918451in}{3.216425in}}{\pgfqpoint{1.926687in}{3.216425in}}%
\pgfpathclose%
\pgfusepath{stroke,fill}%
\end{pgfscope}%
\begin{pgfscope}%
\pgfpathrectangle{\pgfqpoint{0.100000in}{0.212622in}}{\pgfqpoint{3.696000in}{3.696000in}}%
\pgfusepath{clip}%
\pgfsetbuttcap%
\pgfsetroundjoin%
\definecolor{currentfill}{rgb}{0.121569,0.466667,0.705882}%
\pgfsetfillcolor{currentfill}%
\pgfsetfillopacity{0.323700}%
\pgfsetlinewidth{1.003750pt}%
\definecolor{currentstroke}{rgb}{0.121569,0.466667,0.705882}%
\pgfsetstrokecolor{currentstroke}%
\pgfsetstrokeopacity{0.323700}%
\pgfsetdash{}{0pt}%
\pgfpathmoveto{\pgfqpoint{1.799707in}{3.228978in}}%
\pgfpathcurveto{\pgfqpoint{1.807944in}{3.228978in}}{\pgfqpoint{1.815844in}{3.232251in}}{\pgfqpoint{1.821667in}{3.238075in}}%
\pgfpathcurveto{\pgfqpoint{1.827491in}{3.243899in}}{\pgfqpoint{1.830764in}{3.251799in}}{\pgfqpoint{1.830764in}{3.260035in}}%
\pgfpathcurveto{\pgfqpoint{1.830764in}{3.268271in}}{\pgfqpoint{1.827491in}{3.276171in}}{\pgfqpoint{1.821667in}{3.281995in}}%
\pgfpathcurveto{\pgfqpoint{1.815844in}{3.287819in}}{\pgfqpoint{1.807944in}{3.291091in}}{\pgfqpoint{1.799707in}{3.291091in}}%
\pgfpathcurveto{\pgfqpoint{1.791471in}{3.291091in}}{\pgfqpoint{1.783571in}{3.287819in}}{\pgfqpoint{1.777747in}{3.281995in}}%
\pgfpathcurveto{\pgfqpoint{1.771923in}{3.276171in}}{\pgfqpoint{1.768651in}{3.268271in}}{\pgfqpoint{1.768651in}{3.260035in}}%
\pgfpathcurveto{\pgfqpoint{1.768651in}{3.251799in}}{\pgfqpoint{1.771923in}{3.243899in}}{\pgfqpoint{1.777747in}{3.238075in}}%
\pgfpathcurveto{\pgfqpoint{1.783571in}{3.232251in}}{\pgfqpoint{1.791471in}{3.228978in}}{\pgfqpoint{1.799707in}{3.228978in}}%
\pgfpathclose%
\pgfusepath{stroke,fill}%
\end{pgfscope}%
\begin{pgfscope}%
\pgfpathrectangle{\pgfqpoint{0.100000in}{0.212622in}}{\pgfqpoint{3.696000in}{3.696000in}}%
\pgfusepath{clip}%
\pgfsetbuttcap%
\pgfsetroundjoin%
\definecolor{currentfill}{rgb}{0.121569,0.466667,0.705882}%
\pgfsetfillcolor{currentfill}%
\pgfsetfillopacity{0.323871}%
\pgfsetlinewidth{1.003750pt}%
\definecolor{currentstroke}{rgb}{0.121569,0.466667,0.705882}%
\pgfsetstrokecolor{currentstroke}%
\pgfsetstrokeopacity{0.323871}%
\pgfsetdash{}{0pt}%
\pgfpathmoveto{\pgfqpoint{1.927158in}{3.214237in}}%
\pgfpathcurveto{\pgfqpoint{1.935394in}{3.214237in}}{\pgfqpoint{1.943294in}{3.217510in}}{\pgfqpoint{1.949118in}{3.223334in}}%
\pgfpathcurveto{\pgfqpoint{1.954942in}{3.229157in}}{\pgfqpoint{1.958214in}{3.237058in}}{\pgfqpoint{1.958214in}{3.245294in}}%
\pgfpathcurveto{\pgfqpoint{1.958214in}{3.253530in}}{\pgfqpoint{1.954942in}{3.261430in}}{\pgfqpoint{1.949118in}{3.267254in}}%
\pgfpathcurveto{\pgfqpoint{1.943294in}{3.273078in}}{\pgfqpoint{1.935394in}{3.276350in}}{\pgfqpoint{1.927158in}{3.276350in}}%
\pgfpathcurveto{\pgfqpoint{1.918921in}{3.276350in}}{\pgfqpoint{1.911021in}{3.273078in}}{\pgfqpoint{1.905197in}{3.267254in}}%
\pgfpathcurveto{\pgfqpoint{1.899373in}{3.261430in}}{\pgfqpoint{1.896101in}{3.253530in}}{\pgfqpoint{1.896101in}{3.245294in}}%
\pgfpathcurveto{\pgfqpoint{1.896101in}{3.237058in}}{\pgfqpoint{1.899373in}{3.229157in}}{\pgfqpoint{1.905197in}{3.223334in}}%
\pgfpathcurveto{\pgfqpoint{1.911021in}{3.217510in}}{\pgfqpoint{1.918921in}{3.214237in}}{\pgfqpoint{1.927158in}{3.214237in}}%
\pgfpathclose%
\pgfusepath{stroke,fill}%
\end{pgfscope}%
\begin{pgfscope}%
\pgfpathrectangle{\pgfqpoint{0.100000in}{0.212622in}}{\pgfqpoint{3.696000in}{3.696000in}}%
\pgfusepath{clip}%
\pgfsetbuttcap%
\pgfsetroundjoin%
\definecolor{currentfill}{rgb}{0.121569,0.466667,0.705882}%
\pgfsetfillcolor{currentfill}%
\pgfsetfillopacity{0.324162}%
\pgfsetlinewidth{1.003750pt}%
\definecolor{currentstroke}{rgb}{0.121569,0.466667,0.705882}%
\pgfsetstrokecolor{currentstroke}%
\pgfsetstrokeopacity{0.324162}%
\pgfsetdash{}{0pt}%
\pgfpathmoveto{\pgfqpoint{1.927313in}{3.213161in}}%
\pgfpathcurveto{\pgfqpoint{1.935550in}{3.213161in}}{\pgfqpoint{1.943450in}{3.216433in}}{\pgfqpoint{1.949274in}{3.222257in}}%
\pgfpathcurveto{\pgfqpoint{1.955098in}{3.228081in}}{\pgfqpoint{1.958370in}{3.235981in}}{\pgfqpoint{1.958370in}{3.244217in}}%
\pgfpathcurveto{\pgfqpoint{1.958370in}{3.252453in}}{\pgfqpoint{1.955098in}{3.260353in}}{\pgfqpoint{1.949274in}{3.266177in}}%
\pgfpathcurveto{\pgfqpoint{1.943450in}{3.272001in}}{\pgfqpoint{1.935550in}{3.275274in}}{\pgfqpoint{1.927313in}{3.275274in}}%
\pgfpathcurveto{\pgfqpoint{1.919077in}{3.275274in}}{\pgfqpoint{1.911177in}{3.272001in}}{\pgfqpoint{1.905353in}{3.266177in}}%
\pgfpathcurveto{\pgfqpoint{1.899529in}{3.260353in}}{\pgfqpoint{1.896257in}{3.252453in}}{\pgfqpoint{1.896257in}{3.244217in}}%
\pgfpathcurveto{\pgfqpoint{1.896257in}{3.235981in}}{\pgfqpoint{1.899529in}{3.228081in}}{\pgfqpoint{1.905353in}{3.222257in}}%
\pgfpathcurveto{\pgfqpoint{1.911177in}{3.216433in}}{\pgfqpoint{1.919077in}{3.213161in}}{\pgfqpoint{1.927313in}{3.213161in}}%
\pgfpathclose%
\pgfusepath{stroke,fill}%
\end{pgfscope}%
\begin{pgfscope}%
\pgfpathrectangle{\pgfqpoint{0.100000in}{0.212622in}}{\pgfqpoint{3.696000in}{3.696000in}}%
\pgfusepath{clip}%
\pgfsetbuttcap%
\pgfsetroundjoin%
\definecolor{currentfill}{rgb}{0.121569,0.466667,0.705882}%
\pgfsetfillcolor{currentfill}%
\pgfsetfillopacity{0.324723}%
\pgfsetlinewidth{1.003750pt}%
\definecolor{currentstroke}{rgb}{0.121569,0.466667,0.705882}%
\pgfsetstrokecolor{currentstroke}%
\pgfsetstrokeopacity{0.324723}%
\pgfsetdash{}{0pt}%
\pgfpathmoveto{\pgfqpoint{1.927882in}{3.210388in}}%
\pgfpathcurveto{\pgfqpoint{1.936118in}{3.210388in}}{\pgfqpoint{1.944018in}{3.213661in}}{\pgfqpoint{1.949842in}{3.219485in}}%
\pgfpathcurveto{\pgfqpoint{1.955666in}{3.225308in}}{\pgfqpoint{1.958939in}{3.233209in}}{\pgfqpoint{1.958939in}{3.241445in}}%
\pgfpathcurveto{\pgfqpoint{1.958939in}{3.249681in}}{\pgfqpoint{1.955666in}{3.257581in}}{\pgfqpoint{1.949842in}{3.263405in}}%
\pgfpathcurveto{\pgfqpoint{1.944018in}{3.269229in}}{\pgfqpoint{1.936118in}{3.272501in}}{\pgfqpoint{1.927882in}{3.272501in}}%
\pgfpathcurveto{\pgfqpoint{1.919646in}{3.272501in}}{\pgfqpoint{1.911746in}{3.269229in}}{\pgfqpoint{1.905922in}{3.263405in}}%
\pgfpathcurveto{\pgfqpoint{1.900098in}{3.257581in}}{\pgfqpoint{1.896826in}{3.249681in}}{\pgfqpoint{1.896826in}{3.241445in}}%
\pgfpathcurveto{\pgfqpoint{1.896826in}{3.233209in}}{\pgfqpoint{1.900098in}{3.225308in}}{\pgfqpoint{1.905922in}{3.219485in}}%
\pgfpathcurveto{\pgfqpoint{1.911746in}{3.213661in}}{\pgfqpoint{1.919646in}{3.210388in}}{\pgfqpoint{1.927882in}{3.210388in}}%
\pgfpathclose%
\pgfusepath{stroke,fill}%
\end{pgfscope}%
\begin{pgfscope}%
\pgfpathrectangle{\pgfqpoint{0.100000in}{0.212622in}}{\pgfqpoint{3.696000in}{3.696000in}}%
\pgfusepath{clip}%
\pgfsetbuttcap%
\pgfsetroundjoin%
\definecolor{currentfill}{rgb}{0.121569,0.466667,0.705882}%
\pgfsetfillcolor{currentfill}%
\pgfsetfillopacity{0.325066}%
\pgfsetlinewidth{1.003750pt}%
\definecolor{currentstroke}{rgb}{0.121569,0.466667,0.705882}%
\pgfsetstrokecolor{currentstroke}%
\pgfsetstrokeopacity{0.325066}%
\pgfsetdash{}{0pt}%
\pgfpathmoveto{\pgfqpoint{1.928142in}{3.208948in}}%
\pgfpathcurveto{\pgfqpoint{1.936378in}{3.208948in}}{\pgfqpoint{1.944278in}{3.212220in}}{\pgfqpoint{1.950102in}{3.218044in}}%
\pgfpathcurveto{\pgfqpoint{1.955926in}{3.223868in}}{\pgfqpoint{1.959198in}{3.231768in}}{\pgfqpoint{1.959198in}{3.240004in}}%
\pgfpathcurveto{\pgfqpoint{1.959198in}{3.248240in}}{\pgfqpoint{1.955926in}{3.256140in}}{\pgfqpoint{1.950102in}{3.261964in}}%
\pgfpathcurveto{\pgfqpoint{1.944278in}{3.267788in}}{\pgfqpoint{1.936378in}{3.271061in}}{\pgfqpoint{1.928142in}{3.271061in}}%
\pgfpathcurveto{\pgfqpoint{1.919906in}{3.271061in}}{\pgfqpoint{1.912005in}{3.267788in}}{\pgfqpoint{1.906182in}{3.261964in}}%
\pgfpathcurveto{\pgfqpoint{1.900358in}{3.256140in}}{\pgfqpoint{1.897085in}{3.248240in}}{\pgfqpoint{1.897085in}{3.240004in}}%
\pgfpathcurveto{\pgfqpoint{1.897085in}{3.231768in}}{\pgfqpoint{1.900358in}{3.223868in}}{\pgfqpoint{1.906182in}{3.218044in}}%
\pgfpathcurveto{\pgfqpoint{1.912005in}{3.212220in}}{\pgfqpoint{1.919906in}{3.208948in}}{\pgfqpoint{1.928142in}{3.208948in}}%
\pgfpathclose%
\pgfusepath{stroke,fill}%
\end{pgfscope}%
\begin{pgfscope}%
\pgfpathrectangle{\pgfqpoint{0.100000in}{0.212622in}}{\pgfqpoint{3.696000in}{3.696000in}}%
\pgfusepath{clip}%
\pgfsetbuttcap%
\pgfsetroundjoin%
\definecolor{currentfill}{rgb}{0.121569,0.466667,0.705882}%
\pgfsetfillcolor{currentfill}%
\pgfsetfillopacity{0.325189}%
\pgfsetlinewidth{1.003750pt}%
\definecolor{currentstroke}{rgb}{0.121569,0.466667,0.705882}%
\pgfsetstrokecolor{currentstroke}%
\pgfsetstrokeopacity{0.325189}%
\pgfsetdash{}{0pt}%
\pgfpathmoveto{\pgfqpoint{1.795877in}{3.222016in}}%
\pgfpathcurveto{\pgfqpoint{1.804113in}{3.222016in}}{\pgfqpoint{1.812013in}{3.225289in}}{\pgfqpoint{1.817837in}{3.231113in}}%
\pgfpathcurveto{\pgfqpoint{1.823661in}{3.236937in}}{\pgfqpoint{1.826934in}{3.244837in}}{\pgfqpoint{1.826934in}{3.253073in}}%
\pgfpathcurveto{\pgfqpoint{1.826934in}{3.261309in}}{\pgfqpoint{1.823661in}{3.269209in}}{\pgfqpoint{1.817837in}{3.275033in}}%
\pgfpathcurveto{\pgfqpoint{1.812013in}{3.280857in}}{\pgfqpoint{1.804113in}{3.284129in}}{\pgfqpoint{1.795877in}{3.284129in}}%
\pgfpathcurveto{\pgfqpoint{1.787641in}{3.284129in}}{\pgfqpoint{1.779741in}{3.280857in}}{\pgfqpoint{1.773917in}{3.275033in}}%
\pgfpathcurveto{\pgfqpoint{1.768093in}{3.269209in}}{\pgfqpoint{1.764821in}{3.261309in}}{\pgfqpoint{1.764821in}{3.253073in}}%
\pgfpathcurveto{\pgfqpoint{1.764821in}{3.244837in}}{\pgfqpoint{1.768093in}{3.236937in}}{\pgfqpoint{1.773917in}{3.231113in}}%
\pgfpathcurveto{\pgfqpoint{1.779741in}{3.225289in}}{\pgfqpoint{1.787641in}{3.222016in}}{\pgfqpoint{1.795877in}{3.222016in}}%
\pgfpathclose%
\pgfusepath{stroke,fill}%
\end{pgfscope}%
\begin{pgfscope}%
\pgfpathrectangle{\pgfqpoint{0.100000in}{0.212622in}}{\pgfqpoint{3.696000in}{3.696000in}}%
\pgfusepath{clip}%
\pgfsetbuttcap%
\pgfsetroundjoin%
\definecolor{currentfill}{rgb}{0.121569,0.466667,0.705882}%
\pgfsetfillcolor{currentfill}%
\pgfsetfillopacity{0.325596}%
\pgfsetlinewidth{1.003750pt}%
\definecolor{currentstroke}{rgb}{0.121569,0.466667,0.705882}%
\pgfsetstrokecolor{currentstroke}%
\pgfsetstrokeopacity{0.325596}%
\pgfsetdash{}{0pt}%
\pgfpathmoveto{\pgfqpoint{1.928476in}{3.206945in}}%
\pgfpathcurveto{\pgfqpoint{1.936713in}{3.206945in}}{\pgfqpoint{1.944613in}{3.210217in}}{\pgfqpoint{1.950437in}{3.216041in}}%
\pgfpathcurveto{\pgfqpoint{1.956260in}{3.221865in}}{\pgfqpoint{1.959533in}{3.229765in}}{\pgfqpoint{1.959533in}{3.238001in}}%
\pgfpathcurveto{\pgfqpoint{1.959533in}{3.246238in}}{\pgfqpoint{1.956260in}{3.254138in}}{\pgfqpoint{1.950437in}{3.259962in}}%
\pgfpathcurveto{\pgfqpoint{1.944613in}{3.265786in}}{\pgfqpoint{1.936713in}{3.269058in}}{\pgfqpoint{1.928476in}{3.269058in}}%
\pgfpathcurveto{\pgfqpoint{1.920240in}{3.269058in}}{\pgfqpoint{1.912340in}{3.265786in}}{\pgfqpoint{1.906516in}{3.259962in}}%
\pgfpathcurveto{\pgfqpoint{1.900692in}{3.254138in}}{\pgfqpoint{1.897420in}{3.246238in}}{\pgfqpoint{1.897420in}{3.238001in}}%
\pgfpathcurveto{\pgfqpoint{1.897420in}{3.229765in}}{\pgfqpoint{1.900692in}{3.221865in}}{\pgfqpoint{1.906516in}{3.216041in}}%
\pgfpathcurveto{\pgfqpoint{1.912340in}{3.210217in}}{\pgfqpoint{1.920240in}{3.206945in}}{\pgfqpoint{1.928476in}{3.206945in}}%
\pgfpathclose%
\pgfusepath{stroke,fill}%
\end{pgfscope}%
\begin{pgfscope}%
\pgfpathrectangle{\pgfqpoint{0.100000in}{0.212622in}}{\pgfqpoint{3.696000in}{3.696000in}}%
\pgfusepath{clip}%
\pgfsetbuttcap%
\pgfsetroundjoin%
\definecolor{currentfill}{rgb}{0.121569,0.466667,0.705882}%
\pgfsetfillcolor{currentfill}%
\pgfsetfillopacity{0.326077}%
\pgfsetlinewidth{1.003750pt}%
\definecolor{currentstroke}{rgb}{0.121569,0.466667,0.705882}%
\pgfsetstrokecolor{currentstroke}%
\pgfsetstrokeopacity{0.326077}%
\pgfsetdash{}{0pt}%
\pgfpathmoveto{\pgfqpoint{1.929098in}{3.204421in}}%
\pgfpathcurveto{\pgfqpoint{1.937334in}{3.204421in}}{\pgfqpoint{1.945234in}{3.207693in}}{\pgfqpoint{1.951058in}{3.213517in}}%
\pgfpathcurveto{\pgfqpoint{1.956882in}{3.219341in}}{\pgfqpoint{1.960154in}{3.227241in}}{\pgfqpoint{1.960154in}{3.235477in}}%
\pgfpathcurveto{\pgfqpoint{1.960154in}{3.243713in}}{\pgfqpoint{1.956882in}{3.251614in}}{\pgfqpoint{1.951058in}{3.257437in}}%
\pgfpathcurveto{\pgfqpoint{1.945234in}{3.263261in}}{\pgfqpoint{1.937334in}{3.266534in}}{\pgfqpoint{1.929098in}{3.266534in}}%
\pgfpathcurveto{\pgfqpoint{1.920861in}{3.266534in}}{\pgfqpoint{1.912961in}{3.263261in}}{\pgfqpoint{1.907137in}{3.257437in}}%
\pgfpathcurveto{\pgfqpoint{1.901313in}{3.251614in}}{\pgfqpoint{1.898041in}{3.243713in}}{\pgfqpoint{1.898041in}{3.235477in}}%
\pgfpathcurveto{\pgfqpoint{1.898041in}{3.227241in}}{\pgfqpoint{1.901313in}{3.219341in}}{\pgfqpoint{1.907137in}{3.213517in}}%
\pgfpathcurveto{\pgfqpoint{1.912961in}{3.207693in}}{\pgfqpoint{1.920861in}{3.204421in}}{\pgfqpoint{1.929098in}{3.204421in}}%
\pgfpathclose%
\pgfusepath{stroke,fill}%
\end{pgfscope}%
\begin{pgfscope}%
\pgfpathrectangle{\pgfqpoint{0.100000in}{0.212622in}}{\pgfqpoint{3.696000in}{3.696000in}}%
\pgfusepath{clip}%
\pgfsetbuttcap%
\pgfsetroundjoin%
\definecolor{currentfill}{rgb}{0.121569,0.466667,0.705882}%
\pgfsetfillcolor{currentfill}%
\pgfsetfillopacity{0.326363}%
\pgfsetlinewidth{1.003750pt}%
\definecolor{currentstroke}{rgb}{0.121569,0.466667,0.705882}%
\pgfsetstrokecolor{currentstroke}%
\pgfsetstrokeopacity{0.326363}%
\pgfsetdash{}{0pt}%
\pgfpathmoveto{\pgfqpoint{1.791965in}{3.215838in}}%
\pgfpathcurveto{\pgfqpoint{1.800202in}{3.215838in}}{\pgfqpoint{1.808102in}{3.219110in}}{\pgfqpoint{1.813926in}{3.224934in}}%
\pgfpathcurveto{\pgfqpoint{1.819749in}{3.230758in}}{\pgfqpoint{1.823022in}{3.238658in}}{\pgfqpoint{1.823022in}{3.246894in}}%
\pgfpathcurveto{\pgfqpoint{1.823022in}{3.255130in}}{\pgfqpoint{1.819749in}{3.263030in}}{\pgfqpoint{1.813926in}{3.268854in}}%
\pgfpathcurveto{\pgfqpoint{1.808102in}{3.274678in}}{\pgfqpoint{1.800202in}{3.277951in}}{\pgfqpoint{1.791965in}{3.277951in}}%
\pgfpathcurveto{\pgfqpoint{1.783729in}{3.277951in}}{\pgfqpoint{1.775829in}{3.274678in}}{\pgfqpoint{1.770005in}{3.268854in}}%
\pgfpathcurveto{\pgfqpoint{1.764181in}{3.263030in}}{\pgfqpoint{1.760909in}{3.255130in}}{\pgfqpoint{1.760909in}{3.246894in}}%
\pgfpathcurveto{\pgfqpoint{1.760909in}{3.238658in}}{\pgfqpoint{1.764181in}{3.230758in}}{\pgfqpoint{1.770005in}{3.224934in}}%
\pgfpathcurveto{\pgfqpoint{1.775829in}{3.219110in}}{\pgfqpoint{1.783729in}{3.215838in}}{\pgfqpoint{1.791965in}{3.215838in}}%
\pgfpathclose%
\pgfusepath{stroke,fill}%
\end{pgfscope}%
\begin{pgfscope}%
\pgfpathrectangle{\pgfqpoint{0.100000in}{0.212622in}}{\pgfqpoint{3.696000in}{3.696000in}}%
\pgfusepath{clip}%
\pgfsetbuttcap%
\pgfsetroundjoin%
\definecolor{currentfill}{rgb}{0.121569,0.466667,0.705882}%
\pgfsetfillcolor{currentfill}%
\pgfsetfillopacity{0.327183}%
\pgfsetlinewidth{1.003750pt}%
\definecolor{currentstroke}{rgb}{0.121569,0.466667,0.705882}%
\pgfsetstrokecolor{currentstroke}%
\pgfsetstrokeopacity{0.327183}%
\pgfsetdash{}{0pt}%
\pgfpathmoveto{\pgfqpoint{1.929839in}{3.199893in}}%
\pgfpathcurveto{\pgfqpoint{1.938075in}{3.199893in}}{\pgfqpoint{1.945975in}{3.203165in}}{\pgfqpoint{1.951799in}{3.208989in}}%
\pgfpathcurveto{\pgfqpoint{1.957623in}{3.214813in}}{\pgfqpoint{1.960895in}{3.222713in}}{\pgfqpoint{1.960895in}{3.230950in}}%
\pgfpathcurveto{\pgfqpoint{1.960895in}{3.239186in}}{\pgfqpoint{1.957623in}{3.247086in}}{\pgfqpoint{1.951799in}{3.252910in}}%
\pgfpathcurveto{\pgfqpoint{1.945975in}{3.258734in}}{\pgfqpoint{1.938075in}{3.262006in}}{\pgfqpoint{1.929839in}{3.262006in}}%
\pgfpathcurveto{\pgfqpoint{1.921603in}{3.262006in}}{\pgfqpoint{1.913703in}{3.258734in}}{\pgfqpoint{1.907879in}{3.252910in}}%
\pgfpathcurveto{\pgfqpoint{1.902055in}{3.247086in}}{\pgfqpoint{1.898782in}{3.239186in}}{\pgfqpoint{1.898782in}{3.230950in}}%
\pgfpathcurveto{\pgfqpoint{1.898782in}{3.222713in}}{\pgfqpoint{1.902055in}{3.214813in}}{\pgfqpoint{1.907879in}{3.208989in}}%
\pgfpathcurveto{\pgfqpoint{1.913703in}{3.203165in}}{\pgfqpoint{1.921603in}{3.199893in}}{\pgfqpoint{1.929839in}{3.199893in}}%
\pgfpathclose%
\pgfusepath{stroke,fill}%
\end{pgfscope}%
\begin{pgfscope}%
\pgfpathrectangle{\pgfqpoint{0.100000in}{0.212622in}}{\pgfqpoint{3.696000in}{3.696000in}}%
\pgfusepath{clip}%
\pgfsetbuttcap%
\pgfsetroundjoin%
\definecolor{currentfill}{rgb}{0.121569,0.466667,0.705882}%
\pgfsetfillcolor{currentfill}%
\pgfsetfillopacity{0.327368}%
\pgfsetlinewidth{1.003750pt}%
\definecolor{currentstroke}{rgb}{0.121569,0.466667,0.705882}%
\pgfsetstrokecolor{currentstroke}%
\pgfsetstrokeopacity{0.327368}%
\pgfsetdash{}{0pt}%
\pgfpathmoveto{\pgfqpoint{1.789406in}{3.210406in}}%
\pgfpathcurveto{\pgfqpoint{1.797642in}{3.210406in}}{\pgfqpoint{1.805542in}{3.213679in}}{\pgfqpoint{1.811366in}{3.219503in}}%
\pgfpathcurveto{\pgfqpoint{1.817190in}{3.225327in}}{\pgfqpoint{1.820462in}{3.233227in}}{\pgfqpoint{1.820462in}{3.241463in}}%
\pgfpathcurveto{\pgfqpoint{1.820462in}{3.249699in}}{\pgfqpoint{1.817190in}{3.257599in}}{\pgfqpoint{1.811366in}{3.263423in}}%
\pgfpathcurveto{\pgfqpoint{1.805542in}{3.269247in}}{\pgfqpoint{1.797642in}{3.272519in}}{\pgfqpoint{1.789406in}{3.272519in}}%
\pgfpathcurveto{\pgfqpoint{1.781170in}{3.272519in}}{\pgfqpoint{1.773270in}{3.269247in}}{\pgfqpoint{1.767446in}{3.263423in}}%
\pgfpathcurveto{\pgfqpoint{1.761622in}{3.257599in}}{\pgfqpoint{1.758349in}{3.249699in}}{\pgfqpoint{1.758349in}{3.241463in}}%
\pgfpathcurveto{\pgfqpoint{1.758349in}{3.233227in}}{\pgfqpoint{1.761622in}{3.225327in}}{\pgfqpoint{1.767446in}{3.219503in}}%
\pgfpathcurveto{\pgfqpoint{1.773270in}{3.213679in}}{\pgfqpoint{1.781170in}{3.210406in}}{\pgfqpoint{1.789406in}{3.210406in}}%
\pgfpathclose%
\pgfusepath{stroke,fill}%
\end{pgfscope}%
\begin{pgfscope}%
\pgfpathrectangle{\pgfqpoint{0.100000in}{0.212622in}}{\pgfqpoint{3.696000in}{3.696000in}}%
\pgfusepath{clip}%
\pgfsetbuttcap%
\pgfsetroundjoin%
\definecolor{currentfill}{rgb}{0.121569,0.466667,0.705882}%
\pgfsetfillcolor{currentfill}%
\pgfsetfillopacity{0.328135}%
\pgfsetlinewidth{1.003750pt}%
\definecolor{currentstroke}{rgb}{0.121569,0.466667,0.705882}%
\pgfsetstrokecolor{currentstroke}%
\pgfsetstrokeopacity{0.328135}%
\pgfsetdash{}{0pt}%
\pgfpathmoveto{\pgfqpoint{1.787108in}{3.206793in}}%
\pgfpathcurveto{\pgfqpoint{1.795344in}{3.206793in}}{\pgfqpoint{1.803244in}{3.210066in}}{\pgfqpoint{1.809068in}{3.215889in}}%
\pgfpathcurveto{\pgfqpoint{1.814892in}{3.221713in}}{\pgfqpoint{1.818164in}{3.229613in}}{\pgfqpoint{1.818164in}{3.237850in}}%
\pgfpathcurveto{\pgfqpoint{1.818164in}{3.246086in}}{\pgfqpoint{1.814892in}{3.253986in}}{\pgfqpoint{1.809068in}{3.259810in}}%
\pgfpathcurveto{\pgfqpoint{1.803244in}{3.265634in}}{\pgfqpoint{1.795344in}{3.268906in}}{\pgfqpoint{1.787108in}{3.268906in}}%
\pgfpathcurveto{\pgfqpoint{1.778871in}{3.268906in}}{\pgfqpoint{1.770971in}{3.265634in}}{\pgfqpoint{1.765147in}{3.259810in}}%
\pgfpathcurveto{\pgfqpoint{1.759324in}{3.253986in}}{\pgfqpoint{1.756051in}{3.246086in}}{\pgfqpoint{1.756051in}{3.237850in}}%
\pgfpathcurveto{\pgfqpoint{1.756051in}{3.229613in}}{\pgfqpoint{1.759324in}{3.221713in}}{\pgfqpoint{1.765147in}{3.215889in}}%
\pgfpathcurveto{\pgfqpoint{1.770971in}{3.210066in}}{\pgfqpoint{1.778871in}{3.206793in}}{\pgfqpoint{1.787108in}{3.206793in}}%
\pgfpathclose%
\pgfusepath{stroke,fill}%
\end{pgfscope}%
\begin{pgfscope}%
\pgfpathrectangle{\pgfqpoint{0.100000in}{0.212622in}}{\pgfqpoint{3.696000in}{3.696000in}}%
\pgfusepath{clip}%
\pgfsetbuttcap%
\pgfsetroundjoin%
\definecolor{currentfill}{rgb}{0.121569,0.466667,0.705882}%
\pgfsetfillcolor{currentfill}%
\pgfsetfillopacity{0.328412}%
\pgfsetlinewidth{1.003750pt}%
\definecolor{currentstroke}{rgb}{0.121569,0.466667,0.705882}%
\pgfsetstrokecolor{currentstroke}%
\pgfsetstrokeopacity{0.328412}%
\pgfsetdash{}{0pt}%
\pgfpathmoveto{\pgfqpoint{1.930690in}{3.195076in}}%
\pgfpathcurveto{\pgfqpoint{1.938926in}{3.195076in}}{\pgfqpoint{1.946826in}{3.198348in}}{\pgfqpoint{1.952650in}{3.204172in}}%
\pgfpathcurveto{\pgfqpoint{1.958474in}{3.209996in}}{\pgfqpoint{1.961746in}{3.217896in}}{\pgfqpoint{1.961746in}{3.226133in}}%
\pgfpathcurveto{\pgfqpoint{1.961746in}{3.234369in}}{\pgfqpoint{1.958474in}{3.242269in}}{\pgfqpoint{1.952650in}{3.248093in}}%
\pgfpathcurveto{\pgfqpoint{1.946826in}{3.253917in}}{\pgfqpoint{1.938926in}{3.257189in}}{\pgfqpoint{1.930690in}{3.257189in}}%
\pgfpathcurveto{\pgfqpoint{1.922453in}{3.257189in}}{\pgfqpoint{1.914553in}{3.253917in}}{\pgfqpoint{1.908729in}{3.248093in}}%
\pgfpathcurveto{\pgfqpoint{1.902905in}{3.242269in}}{\pgfqpoint{1.899633in}{3.234369in}}{\pgfqpoint{1.899633in}{3.226133in}}%
\pgfpathcurveto{\pgfqpoint{1.899633in}{3.217896in}}{\pgfqpoint{1.902905in}{3.209996in}}{\pgfqpoint{1.908729in}{3.204172in}}%
\pgfpathcurveto{\pgfqpoint{1.914553in}{3.198348in}}{\pgfqpoint{1.922453in}{3.195076in}}{\pgfqpoint{1.930690in}{3.195076in}}%
\pgfpathclose%
\pgfusepath{stroke,fill}%
\end{pgfscope}%
\begin{pgfscope}%
\pgfpathrectangle{\pgfqpoint{0.100000in}{0.212622in}}{\pgfqpoint{3.696000in}{3.696000in}}%
\pgfusepath{clip}%
\pgfsetbuttcap%
\pgfsetroundjoin%
\definecolor{currentfill}{rgb}{0.121569,0.466667,0.705882}%
\pgfsetfillcolor{currentfill}%
\pgfsetfillopacity{0.329505}%
\pgfsetlinewidth{1.003750pt}%
\definecolor{currentstroke}{rgb}{0.121569,0.466667,0.705882}%
\pgfsetstrokecolor{currentstroke}%
\pgfsetstrokeopacity{0.329505}%
\pgfsetdash{}{0pt}%
\pgfpathmoveto{\pgfqpoint{1.782896in}{3.200171in}}%
\pgfpathcurveto{\pgfqpoint{1.791132in}{3.200171in}}{\pgfqpoint{1.799032in}{3.203443in}}{\pgfqpoint{1.804856in}{3.209267in}}%
\pgfpathcurveto{\pgfqpoint{1.810680in}{3.215091in}}{\pgfqpoint{1.813952in}{3.222991in}}{\pgfqpoint{1.813952in}{3.231227in}}%
\pgfpathcurveto{\pgfqpoint{1.813952in}{3.239464in}}{\pgfqpoint{1.810680in}{3.247364in}}{\pgfqpoint{1.804856in}{3.253188in}}%
\pgfpathcurveto{\pgfqpoint{1.799032in}{3.259012in}}{\pgfqpoint{1.791132in}{3.262284in}}{\pgfqpoint{1.782896in}{3.262284in}}%
\pgfpathcurveto{\pgfqpoint{1.774660in}{3.262284in}}{\pgfqpoint{1.766760in}{3.259012in}}{\pgfqpoint{1.760936in}{3.253188in}}%
\pgfpathcurveto{\pgfqpoint{1.755112in}{3.247364in}}{\pgfqpoint{1.751839in}{3.239464in}}{\pgfqpoint{1.751839in}{3.231227in}}%
\pgfpathcurveto{\pgfqpoint{1.751839in}{3.222991in}}{\pgfqpoint{1.755112in}{3.215091in}}{\pgfqpoint{1.760936in}{3.209267in}}%
\pgfpathcurveto{\pgfqpoint{1.766760in}{3.203443in}}{\pgfqpoint{1.774660in}{3.200171in}}{\pgfqpoint{1.782896in}{3.200171in}}%
\pgfpathclose%
\pgfusepath{stroke,fill}%
\end{pgfscope}%
\begin{pgfscope}%
\pgfpathrectangle{\pgfqpoint{0.100000in}{0.212622in}}{\pgfqpoint{3.696000in}{3.696000in}}%
\pgfusepath{clip}%
\pgfsetbuttcap%
\pgfsetroundjoin%
\definecolor{currentfill}{rgb}{0.121569,0.466667,0.705882}%
\pgfsetfillcolor{currentfill}%
\pgfsetfillopacity{0.329536}%
\pgfsetlinewidth{1.003750pt}%
\definecolor{currentstroke}{rgb}{0.121569,0.466667,0.705882}%
\pgfsetstrokecolor{currentstroke}%
\pgfsetstrokeopacity{0.329536}%
\pgfsetdash{}{0pt}%
\pgfpathmoveto{\pgfqpoint{1.931998in}{3.189069in}}%
\pgfpathcurveto{\pgfqpoint{1.940235in}{3.189069in}}{\pgfqpoint{1.948135in}{3.192341in}}{\pgfqpoint{1.953959in}{3.198165in}}%
\pgfpathcurveto{\pgfqpoint{1.959783in}{3.203989in}}{\pgfqpoint{1.963055in}{3.211889in}}{\pgfqpoint{1.963055in}{3.220125in}}%
\pgfpathcurveto{\pgfqpoint{1.963055in}{3.228361in}}{\pgfqpoint{1.959783in}{3.236261in}}{\pgfqpoint{1.953959in}{3.242085in}}%
\pgfpathcurveto{\pgfqpoint{1.948135in}{3.247909in}}{\pgfqpoint{1.940235in}{3.251182in}}{\pgfqpoint{1.931998in}{3.251182in}}%
\pgfpathcurveto{\pgfqpoint{1.923762in}{3.251182in}}{\pgfqpoint{1.915862in}{3.247909in}}{\pgfqpoint{1.910038in}{3.242085in}}%
\pgfpathcurveto{\pgfqpoint{1.904214in}{3.236261in}}{\pgfqpoint{1.900942in}{3.228361in}}{\pgfqpoint{1.900942in}{3.220125in}}%
\pgfpathcurveto{\pgfqpoint{1.900942in}{3.211889in}}{\pgfqpoint{1.904214in}{3.203989in}}{\pgfqpoint{1.910038in}{3.198165in}}%
\pgfpathcurveto{\pgfqpoint{1.915862in}{3.192341in}}{\pgfqpoint{1.923762in}{3.189069in}}{\pgfqpoint{1.931998in}{3.189069in}}%
\pgfpathclose%
\pgfusepath{stroke,fill}%
\end{pgfscope}%
\begin{pgfscope}%
\pgfpathrectangle{\pgfqpoint{0.100000in}{0.212622in}}{\pgfqpoint{3.696000in}{3.696000in}}%
\pgfusepath{clip}%
\pgfsetbuttcap%
\pgfsetroundjoin%
\definecolor{currentfill}{rgb}{0.121569,0.466667,0.705882}%
\pgfsetfillcolor{currentfill}%
\pgfsetfillopacity{0.330697}%
\pgfsetlinewidth{1.003750pt}%
\definecolor{currentstroke}{rgb}{0.121569,0.466667,0.705882}%
\pgfsetstrokecolor{currentstroke}%
\pgfsetstrokeopacity{0.330697}%
\pgfsetdash{}{0pt}%
\pgfpathmoveto{\pgfqpoint{1.779697in}{3.193919in}}%
\pgfpathcurveto{\pgfqpoint{1.787934in}{3.193919in}}{\pgfqpoint{1.795834in}{3.197192in}}{\pgfqpoint{1.801658in}{3.203016in}}%
\pgfpathcurveto{\pgfqpoint{1.807482in}{3.208840in}}{\pgfqpoint{1.810754in}{3.216740in}}{\pgfqpoint{1.810754in}{3.224976in}}%
\pgfpathcurveto{\pgfqpoint{1.810754in}{3.233212in}}{\pgfqpoint{1.807482in}{3.241112in}}{\pgfqpoint{1.801658in}{3.246936in}}%
\pgfpathcurveto{\pgfqpoint{1.795834in}{3.252760in}}{\pgfqpoint{1.787934in}{3.256032in}}{\pgfqpoint{1.779697in}{3.256032in}}%
\pgfpathcurveto{\pgfqpoint{1.771461in}{3.256032in}}{\pgfqpoint{1.763561in}{3.252760in}}{\pgfqpoint{1.757737in}{3.246936in}}%
\pgfpathcurveto{\pgfqpoint{1.751913in}{3.241112in}}{\pgfqpoint{1.748641in}{3.233212in}}{\pgfqpoint{1.748641in}{3.224976in}}%
\pgfpathcurveto{\pgfqpoint{1.748641in}{3.216740in}}{\pgfqpoint{1.751913in}{3.208840in}}{\pgfqpoint{1.757737in}{3.203016in}}%
\pgfpathcurveto{\pgfqpoint{1.763561in}{3.197192in}}{\pgfqpoint{1.771461in}{3.193919in}}{\pgfqpoint{1.779697in}{3.193919in}}%
\pgfpathclose%
\pgfusepath{stroke,fill}%
\end{pgfscope}%
\begin{pgfscope}%
\pgfpathrectangle{\pgfqpoint{0.100000in}{0.212622in}}{\pgfqpoint{3.696000in}{3.696000in}}%
\pgfusepath{clip}%
\pgfsetbuttcap%
\pgfsetroundjoin%
\definecolor{currentfill}{rgb}{0.121569,0.466667,0.705882}%
\pgfsetfillcolor{currentfill}%
\pgfsetfillopacity{0.331441}%
\pgfsetlinewidth{1.003750pt}%
\definecolor{currentstroke}{rgb}{0.121569,0.466667,0.705882}%
\pgfsetstrokecolor{currentstroke}%
\pgfsetstrokeopacity{0.331441}%
\pgfsetdash{}{0pt}%
\pgfpathmoveto{\pgfqpoint{1.776851in}{3.189316in}}%
\pgfpathcurveto{\pgfqpoint{1.785087in}{3.189316in}}{\pgfqpoint{1.792987in}{3.192588in}}{\pgfqpoint{1.798811in}{3.198412in}}%
\pgfpathcurveto{\pgfqpoint{1.804635in}{3.204236in}}{\pgfqpoint{1.807907in}{3.212136in}}{\pgfqpoint{1.807907in}{3.220373in}}%
\pgfpathcurveto{\pgfqpoint{1.807907in}{3.228609in}}{\pgfqpoint{1.804635in}{3.236509in}}{\pgfqpoint{1.798811in}{3.242333in}}%
\pgfpathcurveto{\pgfqpoint{1.792987in}{3.248157in}}{\pgfqpoint{1.785087in}{3.251429in}}{\pgfqpoint{1.776851in}{3.251429in}}%
\pgfpathcurveto{\pgfqpoint{1.768615in}{3.251429in}}{\pgfqpoint{1.760715in}{3.248157in}}{\pgfqpoint{1.754891in}{3.242333in}}%
\pgfpathcurveto{\pgfqpoint{1.749067in}{3.236509in}}{\pgfqpoint{1.745794in}{3.228609in}}{\pgfqpoint{1.745794in}{3.220373in}}%
\pgfpathcurveto{\pgfqpoint{1.745794in}{3.212136in}}{\pgfqpoint{1.749067in}{3.204236in}}{\pgfqpoint{1.754891in}{3.198412in}}%
\pgfpathcurveto{\pgfqpoint{1.760715in}{3.192588in}}{\pgfqpoint{1.768615in}{3.189316in}}{\pgfqpoint{1.776851in}{3.189316in}}%
\pgfpathclose%
\pgfusepath{stroke,fill}%
\end{pgfscope}%
\begin{pgfscope}%
\pgfpathrectangle{\pgfqpoint{0.100000in}{0.212622in}}{\pgfqpoint{3.696000in}{3.696000in}}%
\pgfusepath{clip}%
\pgfsetbuttcap%
\pgfsetroundjoin%
\definecolor{currentfill}{rgb}{0.121569,0.466667,0.705882}%
\pgfsetfillcolor{currentfill}%
\pgfsetfillopacity{0.331497}%
\pgfsetlinewidth{1.003750pt}%
\definecolor{currentstroke}{rgb}{0.121569,0.466667,0.705882}%
\pgfsetstrokecolor{currentstroke}%
\pgfsetstrokeopacity{0.331497}%
\pgfsetdash{}{0pt}%
\pgfpathmoveto{\pgfqpoint{1.932931in}{3.181460in}}%
\pgfpathcurveto{\pgfqpoint{1.941167in}{3.181460in}}{\pgfqpoint{1.949067in}{3.184732in}}{\pgfqpoint{1.954891in}{3.190556in}}%
\pgfpathcurveto{\pgfqpoint{1.960715in}{3.196380in}}{\pgfqpoint{1.963987in}{3.204280in}}{\pgfqpoint{1.963987in}{3.212517in}}%
\pgfpathcurveto{\pgfqpoint{1.963987in}{3.220753in}}{\pgfqpoint{1.960715in}{3.228653in}}{\pgfqpoint{1.954891in}{3.234477in}}%
\pgfpathcurveto{\pgfqpoint{1.949067in}{3.240301in}}{\pgfqpoint{1.941167in}{3.243573in}}{\pgfqpoint{1.932931in}{3.243573in}}%
\pgfpathcurveto{\pgfqpoint{1.924695in}{3.243573in}}{\pgfqpoint{1.916795in}{3.240301in}}{\pgfqpoint{1.910971in}{3.234477in}}%
\pgfpathcurveto{\pgfqpoint{1.905147in}{3.228653in}}{\pgfqpoint{1.901874in}{3.220753in}}{\pgfqpoint{1.901874in}{3.212517in}}%
\pgfpathcurveto{\pgfqpoint{1.901874in}{3.204280in}}{\pgfqpoint{1.905147in}{3.196380in}}{\pgfqpoint{1.910971in}{3.190556in}}%
\pgfpathcurveto{\pgfqpoint{1.916795in}{3.184732in}}{\pgfqpoint{1.924695in}{3.181460in}}{\pgfqpoint{1.932931in}{3.181460in}}%
\pgfpathclose%
\pgfusepath{stroke,fill}%
\end{pgfscope}%
\begin{pgfscope}%
\pgfpathrectangle{\pgfqpoint{0.100000in}{0.212622in}}{\pgfqpoint{3.696000in}{3.696000in}}%
\pgfusepath{clip}%
\pgfsetbuttcap%
\pgfsetroundjoin%
\definecolor{currentfill}{rgb}{0.121569,0.466667,0.705882}%
\pgfsetfillcolor{currentfill}%
\pgfsetfillopacity{0.331984}%
\pgfsetlinewidth{1.003750pt}%
\definecolor{currentstroke}{rgb}{0.121569,0.466667,0.705882}%
\pgfsetstrokecolor{currentstroke}%
\pgfsetstrokeopacity{0.331984}%
\pgfsetdash{}{0pt}%
\pgfpathmoveto{\pgfqpoint{1.775364in}{3.186138in}}%
\pgfpathcurveto{\pgfqpoint{1.783600in}{3.186138in}}{\pgfqpoint{1.791500in}{3.189410in}}{\pgfqpoint{1.797324in}{3.195234in}}%
\pgfpathcurveto{\pgfqpoint{1.803148in}{3.201058in}}{\pgfqpoint{1.806421in}{3.208958in}}{\pgfqpoint{1.806421in}{3.217194in}}%
\pgfpathcurveto{\pgfqpoint{1.806421in}{3.225430in}}{\pgfqpoint{1.803148in}{3.233331in}}{\pgfqpoint{1.797324in}{3.239154in}}%
\pgfpathcurveto{\pgfqpoint{1.791500in}{3.244978in}}{\pgfqpoint{1.783600in}{3.248251in}}{\pgfqpoint{1.775364in}{3.248251in}}%
\pgfpathcurveto{\pgfqpoint{1.767128in}{3.248251in}}{\pgfqpoint{1.759228in}{3.244978in}}{\pgfqpoint{1.753404in}{3.239154in}}%
\pgfpathcurveto{\pgfqpoint{1.747580in}{3.233331in}}{\pgfqpoint{1.744308in}{3.225430in}}{\pgfqpoint{1.744308in}{3.217194in}}%
\pgfpathcurveto{\pgfqpoint{1.744308in}{3.208958in}}{\pgfqpoint{1.747580in}{3.201058in}}{\pgfqpoint{1.753404in}{3.195234in}}%
\pgfpathcurveto{\pgfqpoint{1.759228in}{3.189410in}}{\pgfqpoint{1.767128in}{3.186138in}}{\pgfqpoint{1.775364in}{3.186138in}}%
\pgfpathclose%
\pgfusepath{stroke,fill}%
\end{pgfscope}%
\begin{pgfscope}%
\pgfpathrectangle{\pgfqpoint{0.100000in}{0.212622in}}{\pgfqpoint{3.696000in}{3.696000in}}%
\pgfusepath{clip}%
\pgfsetbuttcap%
\pgfsetroundjoin%
\definecolor{currentfill}{rgb}{0.121569,0.466667,0.705882}%
\pgfsetfillcolor{currentfill}%
\pgfsetfillopacity{0.332452}%
\pgfsetlinewidth{1.003750pt}%
\definecolor{currentstroke}{rgb}{0.121569,0.466667,0.705882}%
\pgfsetstrokecolor{currentstroke}%
\pgfsetstrokeopacity{0.332452}%
\pgfsetdash{}{0pt}%
\pgfpathmoveto{\pgfqpoint{1.773959in}{3.183677in}}%
\pgfpathcurveto{\pgfqpoint{1.782196in}{3.183677in}}{\pgfqpoint{1.790096in}{3.186950in}}{\pgfqpoint{1.795920in}{3.192773in}}%
\pgfpathcurveto{\pgfqpoint{1.801744in}{3.198597in}}{\pgfqpoint{1.805016in}{3.206497in}}{\pgfqpoint{1.805016in}{3.214734in}}%
\pgfpathcurveto{\pgfqpoint{1.805016in}{3.222970in}}{\pgfqpoint{1.801744in}{3.230870in}}{\pgfqpoint{1.795920in}{3.236694in}}%
\pgfpathcurveto{\pgfqpoint{1.790096in}{3.242518in}}{\pgfqpoint{1.782196in}{3.245790in}}{\pgfqpoint{1.773959in}{3.245790in}}%
\pgfpathcurveto{\pgfqpoint{1.765723in}{3.245790in}}{\pgfqpoint{1.757823in}{3.242518in}}{\pgfqpoint{1.751999in}{3.236694in}}%
\pgfpathcurveto{\pgfqpoint{1.746175in}{3.230870in}}{\pgfqpoint{1.742903in}{3.222970in}}{\pgfqpoint{1.742903in}{3.214734in}}%
\pgfpathcurveto{\pgfqpoint{1.742903in}{3.206497in}}{\pgfqpoint{1.746175in}{3.198597in}}{\pgfqpoint{1.751999in}{3.192773in}}%
\pgfpathcurveto{\pgfqpoint{1.757823in}{3.186950in}}{\pgfqpoint{1.765723in}{3.183677in}}{\pgfqpoint{1.773959in}{3.183677in}}%
\pgfpathclose%
\pgfusepath{stroke,fill}%
\end{pgfscope}%
\begin{pgfscope}%
\pgfpathrectangle{\pgfqpoint{0.100000in}{0.212622in}}{\pgfqpoint{3.696000in}{3.696000in}}%
\pgfusepath{clip}%
\pgfsetbuttcap%
\pgfsetroundjoin%
\definecolor{currentfill}{rgb}{0.121569,0.466667,0.705882}%
\pgfsetfillcolor{currentfill}%
\pgfsetfillopacity{0.332517}%
\pgfsetlinewidth{1.003750pt}%
\definecolor{currentstroke}{rgb}{0.121569,0.466667,0.705882}%
\pgfsetstrokecolor{currentstroke}%
\pgfsetstrokeopacity{0.332517}%
\pgfsetdash{}{0pt}%
\pgfpathmoveto{\pgfqpoint{1.933796in}{3.177354in}}%
\pgfpathcurveto{\pgfqpoint{1.942033in}{3.177354in}}{\pgfqpoint{1.949933in}{3.180627in}}{\pgfqpoint{1.955757in}{3.186450in}}%
\pgfpathcurveto{\pgfqpoint{1.961580in}{3.192274in}}{\pgfqpoint{1.964853in}{3.200174in}}{\pgfqpoint{1.964853in}{3.208411in}}%
\pgfpathcurveto{\pgfqpoint{1.964853in}{3.216647in}}{\pgfqpoint{1.961580in}{3.224547in}}{\pgfqpoint{1.955757in}{3.230371in}}%
\pgfpathcurveto{\pgfqpoint{1.949933in}{3.236195in}}{\pgfqpoint{1.942033in}{3.239467in}}{\pgfqpoint{1.933796in}{3.239467in}}%
\pgfpathcurveto{\pgfqpoint{1.925560in}{3.239467in}}{\pgfqpoint{1.917660in}{3.236195in}}{\pgfqpoint{1.911836in}{3.230371in}}%
\pgfpathcurveto{\pgfqpoint{1.906012in}{3.224547in}}{\pgfqpoint{1.902740in}{3.216647in}}{\pgfqpoint{1.902740in}{3.208411in}}%
\pgfpathcurveto{\pgfqpoint{1.902740in}{3.200174in}}{\pgfqpoint{1.906012in}{3.192274in}}{\pgfqpoint{1.911836in}{3.186450in}}%
\pgfpathcurveto{\pgfqpoint{1.917660in}{3.180627in}}{\pgfqpoint{1.925560in}{3.177354in}}{\pgfqpoint{1.933796in}{3.177354in}}%
\pgfpathclose%
\pgfusepath{stroke,fill}%
\end{pgfscope}%
\begin{pgfscope}%
\pgfpathrectangle{\pgfqpoint{0.100000in}{0.212622in}}{\pgfqpoint{3.696000in}{3.696000in}}%
\pgfusepath{clip}%
\pgfsetbuttcap%
\pgfsetroundjoin%
\definecolor{currentfill}{rgb}{0.121569,0.466667,0.705882}%
\pgfsetfillcolor{currentfill}%
\pgfsetfillopacity{0.333245}%
\pgfsetlinewidth{1.003750pt}%
\definecolor{currentstroke}{rgb}{0.121569,0.466667,0.705882}%
\pgfsetstrokecolor{currentstroke}%
\pgfsetstrokeopacity{0.333245}%
\pgfsetdash{}{0pt}%
\pgfpathmoveto{\pgfqpoint{1.771276in}{3.179158in}}%
\pgfpathcurveto{\pgfqpoint{1.779513in}{3.179158in}}{\pgfqpoint{1.787413in}{3.182430in}}{\pgfqpoint{1.793237in}{3.188254in}}%
\pgfpathcurveto{\pgfqpoint{1.799061in}{3.194078in}}{\pgfqpoint{1.802333in}{3.201978in}}{\pgfqpoint{1.802333in}{3.210215in}}%
\pgfpathcurveto{\pgfqpoint{1.802333in}{3.218451in}}{\pgfqpoint{1.799061in}{3.226351in}}{\pgfqpoint{1.793237in}{3.232175in}}%
\pgfpathcurveto{\pgfqpoint{1.787413in}{3.237999in}}{\pgfqpoint{1.779513in}{3.241271in}}{\pgfqpoint{1.771276in}{3.241271in}}%
\pgfpathcurveto{\pgfqpoint{1.763040in}{3.241271in}}{\pgfqpoint{1.755140in}{3.237999in}}{\pgfqpoint{1.749316in}{3.232175in}}%
\pgfpathcurveto{\pgfqpoint{1.743492in}{3.226351in}}{\pgfqpoint{1.740220in}{3.218451in}}{\pgfqpoint{1.740220in}{3.210215in}}%
\pgfpathcurveto{\pgfqpoint{1.740220in}{3.201978in}}{\pgfqpoint{1.743492in}{3.194078in}}{\pgfqpoint{1.749316in}{3.188254in}}%
\pgfpathcurveto{\pgfqpoint{1.755140in}{3.182430in}}{\pgfqpoint{1.763040in}{3.179158in}}{\pgfqpoint{1.771276in}{3.179158in}}%
\pgfpathclose%
\pgfusepath{stroke,fill}%
\end{pgfscope}%
\begin{pgfscope}%
\pgfpathrectangle{\pgfqpoint{0.100000in}{0.212622in}}{\pgfqpoint{3.696000in}{3.696000in}}%
\pgfusepath{clip}%
\pgfsetbuttcap%
\pgfsetroundjoin%
\definecolor{currentfill}{rgb}{0.121569,0.466667,0.705882}%
\pgfsetfillcolor{currentfill}%
\pgfsetfillopacity{0.333602}%
\pgfsetlinewidth{1.003750pt}%
\definecolor{currentstroke}{rgb}{0.121569,0.466667,0.705882}%
\pgfsetstrokecolor{currentstroke}%
\pgfsetstrokeopacity{0.333602}%
\pgfsetdash{}{0pt}%
\pgfpathmoveto{\pgfqpoint{1.934699in}{3.172908in}}%
\pgfpathcurveto{\pgfqpoint{1.942936in}{3.172908in}}{\pgfqpoint{1.950836in}{3.176180in}}{\pgfqpoint{1.956660in}{3.182004in}}%
\pgfpathcurveto{\pgfqpoint{1.962484in}{3.187828in}}{\pgfqpoint{1.965756in}{3.195728in}}{\pgfqpoint{1.965756in}{3.203965in}}%
\pgfpathcurveto{\pgfqpoint{1.965756in}{3.212201in}}{\pgfqpoint{1.962484in}{3.220101in}}{\pgfqpoint{1.956660in}{3.225925in}}%
\pgfpathcurveto{\pgfqpoint{1.950836in}{3.231749in}}{\pgfqpoint{1.942936in}{3.235021in}}{\pgfqpoint{1.934699in}{3.235021in}}%
\pgfpathcurveto{\pgfqpoint{1.926463in}{3.235021in}}{\pgfqpoint{1.918563in}{3.231749in}}{\pgfqpoint{1.912739in}{3.225925in}}%
\pgfpathcurveto{\pgfqpoint{1.906915in}{3.220101in}}{\pgfqpoint{1.903643in}{3.212201in}}{\pgfqpoint{1.903643in}{3.203965in}}%
\pgfpathcurveto{\pgfqpoint{1.903643in}{3.195728in}}{\pgfqpoint{1.906915in}{3.187828in}}{\pgfqpoint{1.912739in}{3.182004in}}%
\pgfpathcurveto{\pgfqpoint{1.918563in}{3.176180in}}{\pgfqpoint{1.926463in}{3.172908in}}{\pgfqpoint{1.934699in}{3.172908in}}%
\pgfpathclose%
\pgfusepath{stroke,fill}%
\end{pgfscope}%
\begin{pgfscope}%
\pgfpathrectangle{\pgfqpoint{0.100000in}{0.212622in}}{\pgfqpoint{3.696000in}{3.696000in}}%
\pgfusepath{clip}%
\pgfsetbuttcap%
\pgfsetroundjoin%
\definecolor{currentfill}{rgb}{0.121569,0.466667,0.705882}%
\pgfsetfillcolor{currentfill}%
\pgfsetfillopacity{0.333855}%
\pgfsetlinewidth{1.003750pt}%
\definecolor{currentstroke}{rgb}{0.121569,0.466667,0.705882}%
\pgfsetstrokecolor{currentstroke}%
\pgfsetstrokeopacity{0.333855}%
\pgfsetdash{}{0pt}%
\pgfpathmoveto{\pgfqpoint{1.769681in}{3.175704in}}%
\pgfpathcurveto{\pgfqpoint{1.777917in}{3.175704in}}{\pgfqpoint{1.785817in}{3.178976in}}{\pgfqpoint{1.791641in}{3.184800in}}%
\pgfpathcurveto{\pgfqpoint{1.797465in}{3.190624in}}{\pgfqpoint{1.800738in}{3.198524in}}{\pgfqpoint{1.800738in}{3.206761in}}%
\pgfpathcurveto{\pgfqpoint{1.800738in}{3.214997in}}{\pgfqpoint{1.797465in}{3.222897in}}{\pgfqpoint{1.791641in}{3.228721in}}%
\pgfpathcurveto{\pgfqpoint{1.785817in}{3.234545in}}{\pgfqpoint{1.777917in}{3.237817in}}{\pgfqpoint{1.769681in}{3.237817in}}%
\pgfpathcurveto{\pgfqpoint{1.761445in}{3.237817in}}{\pgfqpoint{1.753545in}{3.234545in}}{\pgfqpoint{1.747721in}{3.228721in}}%
\pgfpathcurveto{\pgfqpoint{1.741897in}{3.222897in}}{\pgfqpoint{1.738625in}{3.214997in}}{\pgfqpoint{1.738625in}{3.206761in}}%
\pgfpathcurveto{\pgfqpoint{1.738625in}{3.198524in}}{\pgfqpoint{1.741897in}{3.190624in}}{\pgfqpoint{1.747721in}{3.184800in}}%
\pgfpathcurveto{\pgfqpoint{1.753545in}{3.178976in}}{\pgfqpoint{1.761445in}{3.175704in}}{\pgfqpoint{1.769681in}{3.175704in}}%
\pgfpathclose%
\pgfusepath{stroke,fill}%
\end{pgfscope}%
\begin{pgfscope}%
\pgfpathrectangle{\pgfqpoint{0.100000in}{0.212622in}}{\pgfqpoint{3.696000in}{3.696000in}}%
\pgfusepath{clip}%
\pgfsetbuttcap%
\pgfsetroundjoin%
\definecolor{currentfill}{rgb}{0.121569,0.466667,0.705882}%
\pgfsetfillcolor{currentfill}%
\pgfsetfillopacity{0.334112}%
\pgfsetlinewidth{1.003750pt}%
\definecolor{currentstroke}{rgb}{0.121569,0.466667,0.705882}%
\pgfsetstrokecolor{currentstroke}%
\pgfsetstrokeopacity{0.334112}%
\pgfsetdash{}{0pt}%
\pgfpathmoveto{\pgfqpoint{1.768831in}{3.174311in}}%
\pgfpathcurveto{\pgfqpoint{1.777067in}{3.174311in}}{\pgfqpoint{1.784968in}{3.177583in}}{\pgfqpoint{1.790791in}{3.183407in}}%
\pgfpathcurveto{\pgfqpoint{1.796615in}{3.189231in}}{\pgfqpoint{1.799888in}{3.197131in}}{\pgfqpoint{1.799888in}{3.205368in}}%
\pgfpathcurveto{\pgfqpoint{1.799888in}{3.213604in}}{\pgfqpoint{1.796615in}{3.221504in}}{\pgfqpoint{1.790791in}{3.227328in}}%
\pgfpathcurveto{\pgfqpoint{1.784968in}{3.233152in}}{\pgfqpoint{1.777067in}{3.236424in}}{\pgfqpoint{1.768831in}{3.236424in}}%
\pgfpathcurveto{\pgfqpoint{1.760595in}{3.236424in}}{\pgfqpoint{1.752695in}{3.233152in}}{\pgfqpoint{1.746871in}{3.227328in}}%
\pgfpathcurveto{\pgfqpoint{1.741047in}{3.221504in}}{\pgfqpoint{1.737775in}{3.213604in}}{\pgfqpoint{1.737775in}{3.205368in}}%
\pgfpathcurveto{\pgfqpoint{1.737775in}{3.197131in}}{\pgfqpoint{1.741047in}{3.189231in}}{\pgfqpoint{1.746871in}{3.183407in}}%
\pgfpathcurveto{\pgfqpoint{1.752695in}{3.177583in}}{\pgfqpoint{1.760595in}{3.174311in}}{\pgfqpoint{1.768831in}{3.174311in}}%
\pgfpathclose%
\pgfusepath{stroke,fill}%
\end{pgfscope}%
\begin{pgfscope}%
\pgfpathrectangle{\pgfqpoint{0.100000in}{0.212622in}}{\pgfqpoint{3.696000in}{3.696000in}}%
\pgfusepath{clip}%
\pgfsetbuttcap%
\pgfsetroundjoin%
\definecolor{currentfill}{rgb}{0.121569,0.466667,0.705882}%
\pgfsetfillcolor{currentfill}%
\pgfsetfillopacity{0.334605}%
\pgfsetlinewidth{1.003750pt}%
\definecolor{currentstroke}{rgb}{0.121569,0.466667,0.705882}%
\pgfsetstrokecolor{currentstroke}%
\pgfsetstrokeopacity{0.334605}%
\pgfsetdash{}{0pt}%
\pgfpathmoveto{\pgfqpoint{1.767400in}{3.171715in}}%
\pgfpathcurveto{\pgfqpoint{1.775636in}{3.171715in}}{\pgfqpoint{1.783536in}{3.174988in}}{\pgfqpoint{1.789360in}{3.180812in}}%
\pgfpathcurveto{\pgfqpoint{1.795184in}{3.186636in}}{\pgfqpoint{1.798456in}{3.194536in}}{\pgfqpoint{1.798456in}{3.202772in}}%
\pgfpathcurveto{\pgfqpoint{1.798456in}{3.211008in}}{\pgfqpoint{1.795184in}{3.218908in}}{\pgfqpoint{1.789360in}{3.224732in}}%
\pgfpathcurveto{\pgfqpoint{1.783536in}{3.230556in}}{\pgfqpoint{1.775636in}{3.233828in}}{\pgfqpoint{1.767400in}{3.233828in}}%
\pgfpathcurveto{\pgfqpoint{1.759164in}{3.233828in}}{\pgfqpoint{1.751263in}{3.230556in}}{\pgfqpoint{1.745440in}{3.224732in}}%
\pgfpathcurveto{\pgfqpoint{1.739616in}{3.218908in}}{\pgfqpoint{1.736343in}{3.211008in}}{\pgfqpoint{1.736343in}{3.202772in}}%
\pgfpathcurveto{\pgfqpoint{1.736343in}{3.194536in}}{\pgfqpoint{1.739616in}{3.186636in}}{\pgfqpoint{1.745440in}{3.180812in}}%
\pgfpathcurveto{\pgfqpoint{1.751263in}{3.174988in}}{\pgfqpoint{1.759164in}{3.171715in}}{\pgfqpoint{1.767400in}{3.171715in}}%
\pgfpathclose%
\pgfusepath{stroke,fill}%
\end{pgfscope}%
\begin{pgfscope}%
\pgfpathrectangle{\pgfqpoint{0.100000in}{0.212622in}}{\pgfqpoint{3.696000in}{3.696000in}}%
\pgfusepath{clip}%
\pgfsetbuttcap%
\pgfsetroundjoin%
\definecolor{currentfill}{rgb}{0.121569,0.466667,0.705882}%
\pgfsetfillcolor{currentfill}%
\pgfsetfillopacity{0.334940}%
\pgfsetlinewidth{1.003750pt}%
\definecolor{currentstroke}{rgb}{0.121569,0.466667,0.705882}%
\pgfsetstrokecolor{currentstroke}%
\pgfsetstrokeopacity{0.334940}%
\pgfsetdash{}{0pt}%
\pgfpathmoveto{\pgfqpoint{1.935294in}{3.168422in}}%
\pgfpathcurveto{\pgfqpoint{1.943531in}{3.168422in}}{\pgfqpoint{1.951431in}{3.171694in}}{\pgfqpoint{1.957255in}{3.177518in}}%
\pgfpathcurveto{\pgfqpoint{1.963079in}{3.183342in}}{\pgfqpoint{1.966351in}{3.191242in}}{\pgfqpoint{1.966351in}{3.199479in}}%
\pgfpathcurveto{\pgfqpoint{1.966351in}{3.207715in}}{\pgfqpoint{1.963079in}{3.215615in}}{\pgfqpoint{1.957255in}{3.221439in}}%
\pgfpathcurveto{\pgfqpoint{1.951431in}{3.227263in}}{\pgfqpoint{1.943531in}{3.230535in}}{\pgfqpoint{1.935294in}{3.230535in}}%
\pgfpathcurveto{\pgfqpoint{1.927058in}{3.230535in}}{\pgfqpoint{1.919158in}{3.227263in}}{\pgfqpoint{1.913334in}{3.221439in}}%
\pgfpathcurveto{\pgfqpoint{1.907510in}{3.215615in}}{\pgfqpoint{1.904238in}{3.207715in}}{\pgfqpoint{1.904238in}{3.199479in}}%
\pgfpathcurveto{\pgfqpoint{1.904238in}{3.191242in}}{\pgfqpoint{1.907510in}{3.183342in}}{\pgfqpoint{1.913334in}{3.177518in}}%
\pgfpathcurveto{\pgfqpoint{1.919158in}{3.171694in}}{\pgfqpoint{1.927058in}{3.168422in}}{\pgfqpoint{1.935294in}{3.168422in}}%
\pgfpathclose%
\pgfusepath{stroke,fill}%
\end{pgfscope}%
\begin{pgfscope}%
\pgfpathrectangle{\pgfqpoint{0.100000in}{0.212622in}}{\pgfqpoint{3.696000in}{3.696000in}}%
\pgfusepath{clip}%
\pgfsetbuttcap%
\pgfsetroundjoin%
\definecolor{currentfill}{rgb}{0.121569,0.466667,0.705882}%
\pgfsetfillcolor{currentfill}%
\pgfsetfillopacity{0.335514}%
\pgfsetlinewidth{1.003750pt}%
\definecolor{currentstroke}{rgb}{0.121569,0.466667,0.705882}%
\pgfsetstrokecolor{currentstroke}%
\pgfsetstrokeopacity{0.335514}%
\pgfsetdash{}{0pt}%
\pgfpathmoveto{\pgfqpoint{1.764870in}{3.166949in}}%
\pgfpathcurveto{\pgfqpoint{1.773106in}{3.166949in}}{\pgfqpoint{1.781006in}{3.170221in}}{\pgfqpoint{1.786830in}{3.176045in}}%
\pgfpathcurveto{\pgfqpoint{1.792654in}{3.181869in}}{\pgfqpoint{1.795926in}{3.189769in}}{\pgfqpoint{1.795926in}{3.198006in}}%
\pgfpathcurveto{\pgfqpoint{1.795926in}{3.206242in}}{\pgfqpoint{1.792654in}{3.214142in}}{\pgfqpoint{1.786830in}{3.219966in}}%
\pgfpathcurveto{\pgfqpoint{1.781006in}{3.225790in}}{\pgfqpoint{1.773106in}{3.229062in}}{\pgfqpoint{1.764870in}{3.229062in}}%
\pgfpathcurveto{\pgfqpoint{1.756633in}{3.229062in}}{\pgfqpoint{1.748733in}{3.225790in}}{\pgfqpoint{1.742909in}{3.219966in}}%
\pgfpathcurveto{\pgfqpoint{1.737085in}{3.214142in}}{\pgfqpoint{1.733813in}{3.206242in}}{\pgfqpoint{1.733813in}{3.198006in}}%
\pgfpathcurveto{\pgfqpoint{1.733813in}{3.189769in}}{\pgfqpoint{1.737085in}{3.181869in}}{\pgfqpoint{1.742909in}{3.176045in}}%
\pgfpathcurveto{\pgfqpoint{1.748733in}{3.170221in}}{\pgfqpoint{1.756633in}{3.166949in}}{\pgfqpoint{1.764870in}{3.166949in}}%
\pgfpathclose%
\pgfusepath{stroke,fill}%
\end{pgfscope}%
\begin{pgfscope}%
\pgfpathrectangle{\pgfqpoint{0.100000in}{0.212622in}}{\pgfqpoint{3.696000in}{3.696000in}}%
\pgfusepath{clip}%
\pgfsetbuttcap%
\pgfsetroundjoin%
\definecolor{currentfill}{rgb}{0.121569,0.466667,0.705882}%
\pgfsetfillcolor{currentfill}%
\pgfsetfillopacity{0.336197}%
\pgfsetlinewidth{1.003750pt}%
\definecolor{currentstroke}{rgb}{0.121569,0.466667,0.705882}%
\pgfsetstrokecolor{currentstroke}%
\pgfsetstrokeopacity{0.336197}%
\pgfsetdash{}{0pt}%
\pgfpathmoveto{\pgfqpoint{1.762508in}{3.163085in}}%
\pgfpathcurveto{\pgfqpoint{1.770744in}{3.163085in}}{\pgfqpoint{1.778644in}{3.166358in}}{\pgfqpoint{1.784468in}{3.172181in}}%
\pgfpathcurveto{\pgfqpoint{1.790292in}{3.178005in}}{\pgfqpoint{1.793564in}{3.185905in}}{\pgfqpoint{1.793564in}{3.194142in}}%
\pgfpathcurveto{\pgfqpoint{1.793564in}{3.202378in}}{\pgfqpoint{1.790292in}{3.210278in}}{\pgfqpoint{1.784468in}{3.216102in}}%
\pgfpathcurveto{\pgfqpoint{1.778644in}{3.221926in}}{\pgfqpoint{1.770744in}{3.225198in}}{\pgfqpoint{1.762508in}{3.225198in}}%
\pgfpathcurveto{\pgfqpoint{1.754271in}{3.225198in}}{\pgfqpoint{1.746371in}{3.221926in}}{\pgfqpoint{1.740547in}{3.216102in}}%
\pgfpathcurveto{\pgfqpoint{1.734723in}{3.210278in}}{\pgfqpoint{1.731451in}{3.202378in}}{\pgfqpoint{1.731451in}{3.194142in}}%
\pgfpathcurveto{\pgfqpoint{1.731451in}{3.185905in}}{\pgfqpoint{1.734723in}{3.178005in}}{\pgfqpoint{1.740547in}{3.172181in}}%
\pgfpathcurveto{\pgfqpoint{1.746371in}{3.166358in}}{\pgfqpoint{1.754271in}{3.163085in}}{\pgfqpoint{1.762508in}{3.163085in}}%
\pgfpathclose%
\pgfusepath{stroke,fill}%
\end{pgfscope}%
\begin{pgfscope}%
\pgfpathrectangle{\pgfqpoint{0.100000in}{0.212622in}}{\pgfqpoint{3.696000in}{3.696000in}}%
\pgfusepath{clip}%
\pgfsetbuttcap%
\pgfsetroundjoin%
\definecolor{currentfill}{rgb}{0.121569,0.466667,0.705882}%
\pgfsetfillcolor{currentfill}%
\pgfsetfillopacity{0.336481}%
\pgfsetlinewidth{1.003750pt}%
\definecolor{currentstroke}{rgb}{0.121569,0.466667,0.705882}%
\pgfsetstrokecolor{currentstroke}%
\pgfsetstrokeopacity{0.336481}%
\pgfsetdash{}{0pt}%
\pgfpathmoveto{\pgfqpoint{1.936714in}{3.162383in}}%
\pgfpathcurveto{\pgfqpoint{1.944950in}{3.162383in}}{\pgfqpoint{1.952850in}{3.165656in}}{\pgfqpoint{1.958674in}{3.171480in}}%
\pgfpathcurveto{\pgfqpoint{1.964498in}{3.177304in}}{\pgfqpoint{1.967770in}{3.185204in}}{\pgfqpoint{1.967770in}{3.193440in}}%
\pgfpathcurveto{\pgfqpoint{1.967770in}{3.201676in}}{\pgfqpoint{1.964498in}{3.209576in}}{\pgfqpoint{1.958674in}{3.215400in}}%
\pgfpathcurveto{\pgfqpoint{1.952850in}{3.221224in}}{\pgfqpoint{1.944950in}{3.224496in}}{\pgfqpoint{1.936714in}{3.224496in}}%
\pgfpathcurveto{\pgfqpoint{1.928477in}{3.224496in}}{\pgfqpoint{1.920577in}{3.221224in}}{\pgfqpoint{1.914753in}{3.215400in}}%
\pgfpathcurveto{\pgfqpoint{1.908929in}{3.209576in}}{\pgfqpoint{1.905657in}{3.201676in}}{\pgfqpoint{1.905657in}{3.193440in}}%
\pgfpathcurveto{\pgfqpoint{1.905657in}{3.185204in}}{\pgfqpoint{1.908929in}{3.177304in}}{\pgfqpoint{1.914753in}{3.171480in}}%
\pgfpathcurveto{\pgfqpoint{1.920577in}{3.165656in}}{\pgfqpoint{1.928477in}{3.162383in}}{\pgfqpoint{1.936714in}{3.162383in}}%
\pgfpathclose%
\pgfusepath{stroke,fill}%
\end{pgfscope}%
\begin{pgfscope}%
\pgfpathrectangle{\pgfqpoint{0.100000in}{0.212622in}}{\pgfqpoint{3.696000in}{3.696000in}}%
\pgfusepath{clip}%
\pgfsetbuttcap%
\pgfsetroundjoin%
\definecolor{currentfill}{rgb}{0.121569,0.466667,0.705882}%
\pgfsetfillcolor{currentfill}%
\pgfsetfillopacity{0.336679}%
\pgfsetlinewidth{1.003750pt}%
\definecolor{currentstroke}{rgb}{0.121569,0.466667,0.705882}%
\pgfsetstrokecolor{currentstroke}%
\pgfsetstrokeopacity{0.336679}%
\pgfsetdash{}{0pt}%
\pgfpathmoveto{\pgfqpoint{1.761184in}{3.160405in}}%
\pgfpathcurveto{\pgfqpoint{1.769420in}{3.160405in}}{\pgfqpoint{1.777320in}{3.163677in}}{\pgfqpoint{1.783144in}{3.169501in}}%
\pgfpathcurveto{\pgfqpoint{1.788968in}{3.175325in}}{\pgfqpoint{1.792240in}{3.183225in}}{\pgfqpoint{1.792240in}{3.191461in}}%
\pgfpathcurveto{\pgfqpoint{1.792240in}{3.199697in}}{\pgfqpoint{1.788968in}{3.207597in}}{\pgfqpoint{1.783144in}{3.213421in}}%
\pgfpathcurveto{\pgfqpoint{1.777320in}{3.219245in}}{\pgfqpoint{1.769420in}{3.222518in}}{\pgfqpoint{1.761184in}{3.222518in}}%
\pgfpathcurveto{\pgfqpoint{1.752947in}{3.222518in}}{\pgfqpoint{1.745047in}{3.219245in}}{\pgfqpoint{1.739223in}{3.213421in}}%
\pgfpathcurveto{\pgfqpoint{1.733399in}{3.207597in}}{\pgfqpoint{1.730127in}{3.199697in}}{\pgfqpoint{1.730127in}{3.191461in}}%
\pgfpathcurveto{\pgfqpoint{1.730127in}{3.183225in}}{\pgfqpoint{1.733399in}{3.175325in}}{\pgfqpoint{1.739223in}{3.169501in}}%
\pgfpathcurveto{\pgfqpoint{1.745047in}{3.163677in}}{\pgfqpoint{1.752947in}{3.160405in}}{\pgfqpoint{1.761184in}{3.160405in}}%
\pgfpathclose%
\pgfusepath{stroke,fill}%
\end{pgfscope}%
\begin{pgfscope}%
\pgfpathrectangle{\pgfqpoint{0.100000in}{0.212622in}}{\pgfqpoint{3.696000in}{3.696000in}}%
\pgfusepath{clip}%
\pgfsetbuttcap%
\pgfsetroundjoin%
\definecolor{currentfill}{rgb}{0.121569,0.466667,0.705882}%
\pgfsetfillcolor{currentfill}%
\pgfsetfillopacity{0.337570}%
\pgfsetlinewidth{1.003750pt}%
\definecolor{currentstroke}{rgb}{0.121569,0.466667,0.705882}%
\pgfsetstrokecolor{currentstroke}%
\pgfsetstrokeopacity{0.337570}%
\pgfsetdash{}{0pt}%
\pgfpathmoveto{\pgfqpoint{1.758476in}{3.155978in}}%
\pgfpathcurveto{\pgfqpoint{1.766713in}{3.155978in}}{\pgfqpoint{1.774613in}{3.159250in}}{\pgfqpoint{1.780437in}{3.165074in}}%
\pgfpathcurveto{\pgfqpoint{1.786260in}{3.170898in}}{\pgfqpoint{1.789533in}{3.178798in}}{\pgfqpoint{1.789533in}{3.187035in}}%
\pgfpathcurveto{\pgfqpoint{1.789533in}{3.195271in}}{\pgfqpoint{1.786260in}{3.203171in}}{\pgfqpoint{1.780437in}{3.208995in}}%
\pgfpathcurveto{\pgfqpoint{1.774613in}{3.214819in}}{\pgfqpoint{1.766713in}{3.218091in}}{\pgfqpoint{1.758476in}{3.218091in}}%
\pgfpathcurveto{\pgfqpoint{1.750240in}{3.218091in}}{\pgfqpoint{1.742340in}{3.214819in}}{\pgfqpoint{1.736516in}{3.208995in}}%
\pgfpathcurveto{\pgfqpoint{1.730692in}{3.203171in}}{\pgfqpoint{1.727420in}{3.195271in}}{\pgfqpoint{1.727420in}{3.187035in}}%
\pgfpathcurveto{\pgfqpoint{1.727420in}{3.178798in}}{\pgfqpoint{1.730692in}{3.170898in}}{\pgfqpoint{1.736516in}{3.165074in}}%
\pgfpathcurveto{\pgfqpoint{1.742340in}{3.159250in}}{\pgfqpoint{1.750240in}{3.155978in}}{\pgfqpoint{1.758476in}{3.155978in}}%
\pgfpathclose%
\pgfusepath{stroke,fill}%
\end{pgfscope}%
\begin{pgfscope}%
\pgfpathrectangle{\pgfqpoint{0.100000in}{0.212622in}}{\pgfqpoint{3.696000in}{3.696000in}}%
\pgfusepath{clip}%
\pgfsetbuttcap%
\pgfsetroundjoin%
\definecolor{currentfill}{rgb}{0.121569,0.466667,0.705882}%
\pgfsetfillcolor{currentfill}%
\pgfsetfillopacity{0.338174}%
\pgfsetlinewidth{1.003750pt}%
\definecolor{currentstroke}{rgb}{0.121569,0.466667,0.705882}%
\pgfsetstrokecolor{currentstroke}%
\pgfsetstrokeopacity{0.338174}%
\pgfsetdash{}{0pt}%
\pgfpathmoveto{\pgfqpoint{1.937808in}{3.155929in}}%
\pgfpathcurveto{\pgfqpoint{1.946044in}{3.155929in}}{\pgfqpoint{1.953944in}{3.159201in}}{\pgfqpoint{1.959768in}{3.165025in}}%
\pgfpathcurveto{\pgfqpoint{1.965592in}{3.170849in}}{\pgfqpoint{1.968864in}{3.178749in}}{\pgfqpoint{1.968864in}{3.186985in}}%
\pgfpathcurveto{\pgfqpoint{1.968864in}{3.195221in}}{\pgfqpoint{1.965592in}{3.203121in}}{\pgfqpoint{1.959768in}{3.208945in}}%
\pgfpathcurveto{\pgfqpoint{1.953944in}{3.214769in}}{\pgfqpoint{1.946044in}{3.218042in}}{\pgfqpoint{1.937808in}{3.218042in}}%
\pgfpathcurveto{\pgfqpoint{1.929571in}{3.218042in}}{\pgfqpoint{1.921671in}{3.214769in}}{\pgfqpoint{1.915847in}{3.208945in}}%
\pgfpathcurveto{\pgfqpoint{1.910023in}{3.203121in}}{\pgfqpoint{1.906751in}{3.195221in}}{\pgfqpoint{1.906751in}{3.186985in}}%
\pgfpathcurveto{\pgfqpoint{1.906751in}{3.178749in}}{\pgfqpoint{1.910023in}{3.170849in}}{\pgfqpoint{1.915847in}{3.165025in}}%
\pgfpathcurveto{\pgfqpoint{1.921671in}{3.159201in}}{\pgfqpoint{1.929571in}{3.155929in}}{\pgfqpoint{1.937808in}{3.155929in}}%
\pgfpathclose%
\pgfusepath{stroke,fill}%
\end{pgfscope}%
\begin{pgfscope}%
\pgfpathrectangle{\pgfqpoint{0.100000in}{0.212622in}}{\pgfqpoint{3.696000in}{3.696000in}}%
\pgfusepath{clip}%
\pgfsetbuttcap%
\pgfsetroundjoin%
\definecolor{currentfill}{rgb}{0.121569,0.466667,0.705882}%
\pgfsetfillcolor{currentfill}%
\pgfsetfillopacity{0.339164}%
\pgfsetlinewidth{1.003750pt}%
\definecolor{currentstroke}{rgb}{0.121569,0.466667,0.705882}%
\pgfsetstrokecolor{currentstroke}%
\pgfsetstrokeopacity{0.339164}%
\pgfsetdash{}{0pt}%
\pgfpathmoveto{\pgfqpoint{1.938345in}{3.152561in}}%
\pgfpathcurveto{\pgfqpoint{1.946582in}{3.152561in}}{\pgfqpoint{1.954482in}{3.155834in}}{\pgfqpoint{1.960306in}{3.161658in}}%
\pgfpathcurveto{\pgfqpoint{1.966130in}{3.167482in}}{\pgfqpoint{1.969402in}{3.175382in}}{\pgfqpoint{1.969402in}{3.183618in}}%
\pgfpathcurveto{\pgfqpoint{1.969402in}{3.191854in}}{\pgfqpoint{1.966130in}{3.199754in}}{\pgfqpoint{1.960306in}{3.205578in}}%
\pgfpathcurveto{\pgfqpoint{1.954482in}{3.211402in}}{\pgfqpoint{1.946582in}{3.214674in}}{\pgfqpoint{1.938345in}{3.214674in}}%
\pgfpathcurveto{\pgfqpoint{1.930109in}{3.214674in}}{\pgfqpoint{1.922209in}{3.211402in}}{\pgfqpoint{1.916385in}{3.205578in}}%
\pgfpathcurveto{\pgfqpoint{1.910561in}{3.199754in}}{\pgfqpoint{1.907289in}{3.191854in}}{\pgfqpoint{1.907289in}{3.183618in}}%
\pgfpathcurveto{\pgfqpoint{1.907289in}{3.175382in}}{\pgfqpoint{1.910561in}{3.167482in}}{\pgfqpoint{1.916385in}{3.161658in}}%
\pgfpathcurveto{\pgfqpoint{1.922209in}{3.155834in}}{\pgfqpoint{1.930109in}{3.152561in}}{\pgfqpoint{1.938345in}{3.152561in}}%
\pgfpathclose%
\pgfusepath{stroke,fill}%
\end{pgfscope}%
\begin{pgfscope}%
\pgfpathrectangle{\pgfqpoint{0.100000in}{0.212622in}}{\pgfqpoint{3.696000in}{3.696000in}}%
\pgfusepath{clip}%
\pgfsetbuttcap%
\pgfsetroundjoin%
\definecolor{currentfill}{rgb}{0.121569,0.466667,0.705882}%
\pgfsetfillcolor{currentfill}%
\pgfsetfillopacity{0.339173}%
\pgfsetlinewidth{1.003750pt}%
\definecolor{currentstroke}{rgb}{0.121569,0.466667,0.705882}%
\pgfsetstrokecolor{currentstroke}%
\pgfsetstrokeopacity{0.339173}%
\pgfsetdash{}{0pt}%
\pgfpathmoveto{\pgfqpoint{1.753584in}{3.147815in}}%
\pgfpathcurveto{\pgfqpoint{1.761820in}{3.147815in}}{\pgfqpoint{1.769720in}{3.151087in}}{\pgfqpoint{1.775544in}{3.156911in}}%
\pgfpathcurveto{\pgfqpoint{1.781368in}{3.162735in}}{\pgfqpoint{1.784640in}{3.170635in}}{\pgfqpoint{1.784640in}{3.178871in}}%
\pgfpathcurveto{\pgfqpoint{1.784640in}{3.187107in}}{\pgfqpoint{1.781368in}{3.195007in}}{\pgfqpoint{1.775544in}{3.200831in}}%
\pgfpathcurveto{\pgfqpoint{1.769720in}{3.206655in}}{\pgfqpoint{1.761820in}{3.209928in}}{\pgfqpoint{1.753584in}{3.209928in}}%
\pgfpathcurveto{\pgfqpoint{1.745348in}{3.209928in}}{\pgfqpoint{1.737448in}{3.206655in}}{\pgfqpoint{1.731624in}{3.200831in}}%
\pgfpathcurveto{\pgfqpoint{1.725800in}{3.195007in}}{\pgfqpoint{1.722527in}{3.187107in}}{\pgfqpoint{1.722527in}{3.178871in}}%
\pgfpathcurveto{\pgfqpoint{1.722527in}{3.170635in}}{\pgfqpoint{1.725800in}{3.162735in}}{\pgfqpoint{1.731624in}{3.156911in}}%
\pgfpathcurveto{\pgfqpoint{1.737448in}{3.151087in}}{\pgfqpoint{1.745348in}{3.147815in}}{\pgfqpoint{1.753584in}{3.147815in}}%
\pgfpathclose%
\pgfusepath{stroke,fill}%
\end{pgfscope}%
\begin{pgfscope}%
\pgfpathrectangle{\pgfqpoint{0.100000in}{0.212622in}}{\pgfqpoint{3.696000in}{3.696000in}}%
\pgfusepath{clip}%
\pgfsetbuttcap%
\pgfsetroundjoin%
\definecolor{currentfill}{rgb}{0.121569,0.466667,0.705882}%
\pgfsetfillcolor{currentfill}%
\pgfsetfillopacity{0.340164}%
\pgfsetlinewidth{1.003750pt}%
\definecolor{currentstroke}{rgb}{0.121569,0.466667,0.705882}%
\pgfsetstrokecolor{currentstroke}%
\pgfsetstrokeopacity{0.340164}%
\pgfsetdash{}{0pt}%
\pgfpathmoveto{\pgfqpoint{1.939217in}{3.148503in}}%
\pgfpathcurveto{\pgfqpoint{1.947453in}{3.148503in}}{\pgfqpoint{1.955353in}{3.151775in}}{\pgfqpoint{1.961177in}{3.157599in}}%
\pgfpathcurveto{\pgfqpoint{1.967001in}{3.163423in}}{\pgfqpoint{1.970273in}{3.171323in}}{\pgfqpoint{1.970273in}{3.179559in}}%
\pgfpathcurveto{\pgfqpoint{1.970273in}{3.187796in}}{\pgfqpoint{1.967001in}{3.195696in}}{\pgfqpoint{1.961177in}{3.201520in}}%
\pgfpathcurveto{\pgfqpoint{1.955353in}{3.207343in}}{\pgfqpoint{1.947453in}{3.210616in}}{\pgfqpoint{1.939217in}{3.210616in}}%
\pgfpathcurveto{\pgfqpoint{1.930980in}{3.210616in}}{\pgfqpoint{1.923080in}{3.207343in}}{\pgfqpoint{1.917256in}{3.201520in}}%
\pgfpathcurveto{\pgfqpoint{1.911432in}{3.195696in}}{\pgfqpoint{1.908160in}{3.187796in}}{\pgfqpoint{1.908160in}{3.179559in}}%
\pgfpathcurveto{\pgfqpoint{1.908160in}{3.171323in}}{\pgfqpoint{1.911432in}{3.163423in}}{\pgfqpoint{1.917256in}{3.157599in}}%
\pgfpathcurveto{\pgfqpoint{1.923080in}{3.151775in}}{\pgfqpoint{1.930980in}{3.148503in}}{\pgfqpoint{1.939217in}{3.148503in}}%
\pgfpathclose%
\pgfusepath{stroke,fill}%
\end{pgfscope}%
\begin{pgfscope}%
\pgfpathrectangle{\pgfqpoint{0.100000in}{0.212622in}}{\pgfqpoint{3.696000in}{3.696000in}}%
\pgfusepath{clip}%
\pgfsetbuttcap%
\pgfsetroundjoin%
\definecolor{currentfill}{rgb}{0.121569,0.466667,0.705882}%
\pgfsetfillcolor{currentfill}%
\pgfsetfillopacity{0.340687}%
\pgfsetlinewidth{1.003750pt}%
\definecolor{currentstroke}{rgb}{0.121569,0.466667,0.705882}%
\pgfsetstrokecolor{currentstroke}%
\pgfsetstrokeopacity{0.340687}%
\pgfsetdash{}{0pt}%
\pgfpathmoveto{\pgfqpoint{1.749440in}{3.139982in}}%
\pgfpathcurveto{\pgfqpoint{1.757676in}{3.139982in}}{\pgfqpoint{1.765576in}{3.143254in}}{\pgfqpoint{1.771400in}{3.149078in}}%
\pgfpathcurveto{\pgfqpoint{1.777224in}{3.154902in}}{\pgfqpoint{1.780497in}{3.162802in}}{\pgfqpoint{1.780497in}{3.171039in}}%
\pgfpathcurveto{\pgfqpoint{1.780497in}{3.179275in}}{\pgfqpoint{1.777224in}{3.187175in}}{\pgfqpoint{1.771400in}{3.192999in}}%
\pgfpathcurveto{\pgfqpoint{1.765576in}{3.198823in}}{\pgfqpoint{1.757676in}{3.202095in}}{\pgfqpoint{1.749440in}{3.202095in}}%
\pgfpathcurveto{\pgfqpoint{1.741204in}{3.202095in}}{\pgfqpoint{1.733304in}{3.198823in}}{\pgfqpoint{1.727480in}{3.192999in}}%
\pgfpathcurveto{\pgfqpoint{1.721656in}{3.187175in}}{\pgfqpoint{1.718384in}{3.179275in}}{\pgfqpoint{1.718384in}{3.171039in}}%
\pgfpathcurveto{\pgfqpoint{1.718384in}{3.162802in}}{\pgfqpoint{1.721656in}{3.154902in}}{\pgfqpoint{1.727480in}{3.149078in}}%
\pgfpathcurveto{\pgfqpoint{1.733304in}{3.143254in}}{\pgfqpoint{1.741204in}{3.139982in}}{\pgfqpoint{1.749440in}{3.139982in}}%
\pgfpathclose%
\pgfusepath{stroke,fill}%
\end{pgfscope}%
\begin{pgfscope}%
\pgfpathrectangle{\pgfqpoint{0.100000in}{0.212622in}}{\pgfqpoint{3.696000in}{3.696000in}}%
\pgfusepath{clip}%
\pgfsetbuttcap%
\pgfsetroundjoin%
\definecolor{currentfill}{rgb}{0.121569,0.466667,0.705882}%
\pgfsetfillcolor{currentfill}%
\pgfsetfillopacity{0.341495}%
\pgfsetlinewidth{1.003750pt}%
\definecolor{currentstroke}{rgb}{0.121569,0.466667,0.705882}%
\pgfsetstrokecolor{currentstroke}%
\pgfsetstrokeopacity{0.341495}%
\pgfsetdash{}{0pt}%
\pgfpathmoveto{\pgfqpoint{1.939902in}{3.143621in}}%
\pgfpathcurveto{\pgfqpoint{1.948139in}{3.143621in}}{\pgfqpoint{1.956039in}{3.146893in}}{\pgfqpoint{1.961863in}{3.152717in}}%
\pgfpathcurveto{\pgfqpoint{1.967686in}{3.158541in}}{\pgfqpoint{1.970959in}{3.166441in}}{\pgfqpoint{1.970959in}{3.174677in}}%
\pgfpathcurveto{\pgfqpoint{1.970959in}{3.182913in}}{\pgfqpoint{1.967686in}{3.190813in}}{\pgfqpoint{1.961863in}{3.196637in}}%
\pgfpathcurveto{\pgfqpoint{1.956039in}{3.202461in}}{\pgfqpoint{1.948139in}{3.205734in}}{\pgfqpoint{1.939902in}{3.205734in}}%
\pgfpathcurveto{\pgfqpoint{1.931666in}{3.205734in}}{\pgfqpoint{1.923766in}{3.202461in}}{\pgfqpoint{1.917942in}{3.196637in}}%
\pgfpathcurveto{\pgfqpoint{1.912118in}{3.190813in}}{\pgfqpoint{1.908846in}{3.182913in}}{\pgfqpoint{1.908846in}{3.174677in}}%
\pgfpathcurveto{\pgfqpoint{1.908846in}{3.166441in}}{\pgfqpoint{1.912118in}{3.158541in}}{\pgfqpoint{1.917942in}{3.152717in}}%
\pgfpathcurveto{\pgfqpoint{1.923766in}{3.146893in}}{\pgfqpoint{1.931666in}{3.143621in}}{\pgfqpoint{1.939902in}{3.143621in}}%
\pgfpathclose%
\pgfusepath{stroke,fill}%
\end{pgfscope}%
\begin{pgfscope}%
\pgfpathrectangle{\pgfqpoint{0.100000in}{0.212622in}}{\pgfqpoint{3.696000in}{3.696000in}}%
\pgfusepath{clip}%
\pgfsetbuttcap%
\pgfsetroundjoin%
\definecolor{currentfill}{rgb}{0.121569,0.466667,0.705882}%
\pgfsetfillcolor{currentfill}%
\pgfsetfillopacity{0.341765}%
\pgfsetlinewidth{1.003750pt}%
\definecolor{currentstroke}{rgb}{0.121569,0.466667,0.705882}%
\pgfsetstrokecolor{currentstroke}%
\pgfsetstrokeopacity{0.341765}%
\pgfsetdash{}{0pt}%
\pgfpathmoveto{\pgfqpoint{1.745707in}{3.134075in}}%
\pgfpathcurveto{\pgfqpoint{1.753944in}{3.134075in}}{\pgfqpoint{1.761844in}{3.137347in}}{\pgfqpoint{1.767668in}{3.143171in}}%
\pgfpathcurveto{\pgfqpoint{1.773492in}{3.148995in}}{\pgfqpoint{1.776764in}{3.156895in}}{\pgfqpoint{1.776764in}{3.165131in}}%
\pgfpathcurveto{\pgfqpoint{1.776764in}{3.173368in}}{\pgfqpoint{1.773492in}{3.181268in}}{\pgfqpoint{1.767668in}{3.187092in}}%
\pgfpathcurveto{\pgfqpoint{1.761844in}{3.192915in}}{\pgfqpoint{1.753944in}{3.196188in}}{\pgfqpoint{1.745707in}{3.196188in}}%
\pgfpathcurveto{\pgfqpoint{1.737471in}{3.196188in}}{\pgfqpoint{1.729571in}{3.192915in}}{\pgfqpoint{1.723747in}{3.187092in}}%
\pgfpathcurveto{\pgfqpoint{1.717923in}{3.181268in}}{\pgfqpoint{1.714651in}{3.173368in}}{\pgfqpoint{1.714651in}{3.165131in}}%
\pgfpathcurveto{\pgfqpoint{1.714651in}{3.156895in}}{\pgfqpoint{1.717923in}{3.148995in}}{\pgfqpoint{1.723747in}{3.143171in}}%
\pgfpathcurveto{\pgfqpoint{1.729571in}{3.137347in}}{\pgfqpoint{1.737471in}{3.134075in}}{\pgfqpoint{1.745707in}{3.134075in}}%
\pgfpathclose%
\pgfusepath{stroke,fill}%
\end{pgfscope}%
\begin{pgfscope}%
\pgfpathrectangle{\pgfqpoint{0.100000in}{0.212622in}}{\pgfqpoint{3.696000in}{3.696000in}}%
\pgfusepath{clip}%
\pgfsetbuttcap%
\pgfsetroundjoin%
\definecolor{currentfill}{rgb}{0.121569,0.466667,0.705882}%
\pgfsetfillcolor{currentfill}%
\pgfsetfillopacity{0.342700}%
\pgfsetlinewidth{1.003750pt}%
\definecolor{currentstroke}{rgb}{0.121569,0.466667,0.705882}%
\pgfsetstrokecolor{currentstroke}%
\pgfsetstrokeopacity{0.342700}%
\pgfsetdash{}{0pt}%
\pgfpathmoveto{\pgfqpoint{1.743265in}{3.129223in}}%
\pgfpathcurveto{\pgfqpoint{1.751501in}{3.129223in}}{\pgfqpoint{1.759401in}{3.132495in}}{\pgfqpoint{1.765225in}{3.138319in}}%
\pgfpathcurveto{\pgfqpoint{1.771049in}{3.144143in}}{\pgfqpoint{1.774322in}{3.152043in}}{\pgfqpoint{1.774322in}{3.160279in}}%
\pgfpathcurveto{\pgfqpoint{1.774322in}{3.168515in}}{\pgfqpoint{1.771049in}{3.176415in}}{\pgfqpoint{1.765225in}{3.182239in}}%
\pgfpathcurveto{\pgfqpoint{1.759401in}{3.188063in}}{\pgfqpoint{1.751501in}{3.191336in}}{\pgfqpoint{1.743265in}{3.191336in}}%
\pgfpathcurveto{\pgfqpoint{1.735029in}{3.191336in}}{\pgfqpoint{1.727129in}{3.188063in}}{\pgfqpoint{1.721305in}{3.182239in}}%
\pgfpathcurveto{\pgfqpoint{1.715481in}{3.176415in}}{\pgfqpoint{1.712209in}{3.168515in}}{\pgfqpoint{1.712209in}{3.160279in}}%
\pgfpathcurveto{\pgfqpoint{1.712209in}{3.152043in}}{\pgfqpoint{1.715481in}{3.144143in}}{\pgfqpoint{1.721305in}{3.138319in}}%
\pgfpathcurveto{\pgfqpoint{1.727129in}{3.132495in}}{\pgfqpoint{1.735029in}{3.129223in}}{\pgfqpoint{1.743265in}{3.129223in}}%
\pgfpathclose%
\pgfusepath{stroke,fill}%
\end{pgfscope}%
\begin{pgfscope}%
\pgfpathrectangle{\pgfqpoint{0.100000in}{0.212622in}}{\pgfqpoint{3.696000in}{3.696000in}}%
\pgfusepath{clip}%
\pgfsetbuttcap%
\pgfsetroundjoin%
\definecolor{currentfill}{rgb}{0.121569,0.466667,0.705882}%
\pgfsetfillcolor{currentfill}%
\pgfsetfillopacity{0.343050}%
\pgfsetlinewidth{1.003750pt}%
\definecolor{currentstroke}{rgb}{0.121569,0.466667,0.705882}%
\pgfsetstrokecolor{currentstroke}%
\pgfsetstrokeopacity{0.343050}%
\pgfsetdash{}{0pt}%
\pgfpathmoveto{\pgfqpoint{1.940891in}{3.137992in}}%
\pgfpathcurveto{\pgfqpoint{1.949128in}{3.137992in}}{\pgfqpoint{1.957028in}{3.141264in}}{\pgfqpoint{1.962852in}{3.147088in}}%
\pgfpathcurveto{\pgfqpoint{1.968676in}{3.152912in}}{\pgfqpoint{1.971948in}{3.160812in}}{\pgfqpoint{1.971948in}{3.169048in}}%
\pgfpathcurveto{\pgfqpoint{1.971948in}{3.177284in}}{\pgfqpoint{1.968676in}{3.185184in}}{\pgfqpoint{1.962852in}{3.191008in}}%
\pgfpathcurveto{\pgfqpoint{1.957028in}{3.196832in}}{\pgfqpoint{1.949128in}{3.200105in}}{\pgfqpoint{1.940891in}{3.200105in}}%
\pgfpathcurveto{\pgfqpoint{1.932655in}{3.200105in}}{\pgfqpoint{1.924755in}{3.196832in}}{\pgfqpoint{1.918931in}{3.191008in}}%
\pgfpathcurveto{\pgfqpoint{1.913107in}{3.185184in}}{\pgfqpoint{1.909835in}{3.177284in}}{\pgfqpoint{1.909835in}{3.169048in}}%
\pgfpathcurveto{\pgfqpoint{1.909835in}{3.160812in}}{\pgfqpoint{1.913107in}{3.152912in}}{\pgfqpoint{1.918931in}{3.147088in}}%
\pgfpathcurveto{\pgfqpoint{1.924755in}{3.141264in}}{\pgfqpoint{1.932655in}{3.137992in}}{\pgfqpoint{1.940891in}{3.137992in}}%
\pgfpathclose%
\pgfusepath{stroke,fill}%
\end{pgfscope}%
\begin{pgfscope}%
\pgfpathrectangle{\pgfqpoint{0.100000in}{0.212622in}}{\pgfqpoint{3.696000in}{3.696000in}}%
\pgfusepath{clip}%
\pgfsetbuttcap%
\pgfsetroundjoin%
\definecolor{currentfill}{rgb}{0.121569,0.466667,0.705882}%
\pgfsetfillcolor{currentfill}%
\pgfsetfillopacity{0.343533}%
\pgfsetlinewidth{1.003750pt}%
\definecolor{currentstroke}{rgb}{0.121569,0.466667,0.705882}%
\pgfsetstrokecolor{currentstroke}%
\pgfsetstrokeopacity{0.343533}%
\pgfsetdash{}{0pt}%
\pgfpathmoveto{\pgfqpoint{1.740820in}{3.124912in}}%
\pgfpathcurveto{\pgfqpoint{1.749057in}{3.124912in}}{\pgfqpoint{1.756957in}{3.128184in}}{\pgfqpoint{1.762781in}{3.134008in}}%
\pgfpathcurveto{\pgfqpoint{1.768604in}{3.139832in}}{\pgfqpoint{1.771877in}{3.147732in}}{\pgfqpoint{1.771877in}{3.155968in}}%
\pgfpathcurveto{\pgfqpoint{1.771877in}{3.164205in}}{\pgfqpoint{1.768604in}{3.172105in}}{\pgfqpoint{1.762781in}{3.177929in}}%
\pgfpathcurveto{\pgfqpoint{1.756957in}{3.183753in}}{\pgfqpoint{1.749057in}{3.187025in}}{\pgfqpoint{1.740820in}{3.187025in}}%
\pgfpathcurveto{\pgfqpoint{1.732584in}{3.187025in}}{\pgfqpoint{1.724684in}{3.183753in}}{\pgfqpoint{1.718860in}{3.177929in}}%
\pgfpathcurveto{\pgfqpoint{1.713036in}{3.172105in}}{\pgfqpoint{1.709764in}{3.164205in}}{\pgfqpoint{1.709764in}{3.155968in}}%
\pgfpathcurveto{\pgfqpoint{1.709764in}{3.147732in}}{\pgfqpoint{1.713036in}{3.139832in}}{\pgfqpoint{1.718860in}{3.134008in}}%
\pgfpathcurveto{\pgfqpoint{1.724684in}{3.128184in}}{\pgfqpoint{1.732584in}{3.124912in}}{\pgfqpoint{1.740820in}{3.124912in}}%
\pgfpathclose%
\pgfusepath{stroke,fill}%
\end{pgfscope}%
\begin{pgfscope}%
\pgfpathrectangle{\pgfqpoint{0.100000in}{0.212622in}}{\pgfqpoint{3.696000in}{3.696000in}}%
\pgfusepath{clip}%
\pgfsetbuttcap%
\pgfsetroundjoin%
\definecolor{currentfill}{rgb}{0.121569,0.466667,0.705882}%
\pgfsetfillcolor{currentfill}%
\pgfsetfillopacity{0.344257}%
\pgfsetlinewidth{1.003750pt}%
\definecolor{currentstroke}{rgb}{0.121569,0.466667,0.705882}%
\pgfsetstrokecolor{currentstroke}%
\pgfsetstrokeopacity{0.344257}%
\pgfsetdash{}{0pt}%
\pgfpathmoveto{\pgfqpoint{1.738423in}{3.120936in}}%
\pgfpathcurveto{\pgfqpoint{1.746659in}{3.120936in}}{\pgfqpoint{1.754559in}{3.124209in}}{\pgfqpoint{1.760383in}{3.130033in}}%
\pgfpathcurveto{\pgfqpoint{1.766207in}{3.135856in}}{\pgfqpoint{1.769479in}{3.143757in}}{\pgfqpoint{1.769479in}{3.151993in}}%
\pgfpathcurveto{\pgfqpoint{1.769479in}{3.160229in}}{\pgfqpoint{1.766207in}{3.168129in}}{\pgfqpoint{1.760383in}{3.173953in}}%
\pgfpathcurveto{\pgfqpoint{1.754559in}{3.179777in}}{\pgfqpoint{1.746659in}{3.183049in}}{\pgfqpoint{1.738423in}{3.183049in}}%
\pgfpathcurveto{\pgfqpoint{1.730186in}{3.183049in}}{\pgfqpoint{1.722286in}{3.179777in}}{\pgfqpoint{1.716462in}{3.173953in}}%
\pgfpathcurveto{\pgfqpoint{1.710639in}{3.168129in}}{\pgfqpoint{1.707366in}{3.160229in}}{\pgfqpoint{1.707366in}{3.151993in}}%
\pgfpathcurveto{\pgfqpoint{1.707366in}{3.143757in}}{\pgfqpoint{1.710639in}{3.135856in}}{\pgfqpoint{1.716462in}{3.130033in}}%
\pgfpathcurveto{\pgfqpoint{1.722286in}{3.124209in}}{\pgfqpoint{1.730186in}{3.120936in}}{\pgfqpoint{1.738423in}{3.120936in}}%
\pgfpathclose%
\pgfusepath{stroke,fill}%
\end{pgfscope}%
\begin{pgfscope}%
\pgfpathrectangle{\pgfqpoint{0.100000in}{0.212622in}}{\pgfqpoint{3.696000in}{3.696000in}}%
\pgfusepath{clip}%
\pgfsetbuttcap%
\pgfsetroundjoin%
\definecolor{currentfill}{rgb}{0.121569,0.466667,0.705882}%
\pgfsetfillcolor{currentfill}%
\pgfsetfillopacity{0.344462}%
\pgfsetlinewidth{1.003750pt}%
\definecolor{currentstroke}{rgb}{0.121569,0.466667,0.705882}%
\pgfsetstrokecolor{currentstroke}%
\pgfsetstrokeopacity{0.344462}%
\pgfsetdash{}{0pt}%
\pgfpathmoveto{\pgfqpoint{1.942286in}{3.130855in}}%
\pgfpathcurveto{\pgfqpoint{1.950523in}{3.130855in}}{\pgfqpoint{1.958423in}{3.134127in}}{\pgfqpoint{1.964247in}{3.139951in}}%
\pgfpathcurveto{\pgfqpoint{1.970071in}{3.145775in}}{\pgfqpoint{1.973343in}{3.153675in}}{\pgfqpoint{1.973343in}{3.161911in}}%
\pgfpathcurveto{\pgfqpoint{1.973343in}{3.170147in}}{\pgfqpoint{1.970071in}{3.178047in}}{\pgfqpoint{1.964247in}{3.183871in}}%
\pgfpathcurveto{\pgfqpoint{1.958423in}{3.189695in}}{\pgfqpoint{1.950523in}{3.192968in}}{\pgfqpoint{1.942286in}{3.192968in}}%
\pgfpathcurveto{\pgfqpoint{1.934050in}{3.192968in}}{\pgfqpoint{1.926150in}{3.189695in}}{\pgfqpoint{1.920326in}{3.183871in}}%
\pgfpathcurveto{\pgfqpoint{1.914502in}{3.178047in}}{\pgfqpoint{1.911230in}{3.170147in}}{\pgfqpoint{1.911230in}{3.161911in}}%
\pgfpathcurveto{\pgfqpoint{1.911230in}{3.153675in}}{\pgfqpoint{1.914502in}{3.145775in}}{\pgfqpoint{1.920326in}{3.139951in}}%
\pgfpathcurveto{\pgfqpoint{1.926150in}{3.134127in}}{\pgfqpoint{1.934050in}{3.130855in}}{\pgfqpoint{1.942286in}{3.130855in}}%
\pgfpathclose%
\pgfusepath{stroke,fill}%
\end{pgfscope}%
\begin{pgfscope}%
\pgfpathrectangle{\pgfqpoint{0.100000in}{0.212622in}}{\pgfqpoint{3.696000in}{3.696000in}}%
\pgfusepath{clip}%
\pgfsetbuttcap%
\pgfsetroundjoin%
\definecolor{currentfill}{rgb}{0.121569,0.466667,0.705882}%
\pgfsetfillcolor{currentfill}%
\pgfsetfillopacity{0.345636}%
\pgfsetlinewidth{1.003750pt}%
\definecolor{currentstroke}{rgb}{0.121569,0.466667,0.705882}%
\pgfsetstrokecolor{currentstroke}%
\pgfsetstrokeopacity{0.345636}%
\pgfsetdash{}{0pt}%
\pgfpathmoveto{\pgfqpoint{1.734747in}{3.113079in}}%
\pgfpathcurveto{\pgfqpoint{1.742983in}{3.113079in}}{\pgfqpoint{1.750883in}{3.116351in}}{\pgfqpoint{1.756707in}{3.122175in}}%
\pgfpathcurveto{\pgfqpoint{1.762531in}{3.127999in}}{\pgfqpoint{1.765803in}{3.135899in}}{\pgfqpoint{1.765803in}{3.144135in}}%
\pgfpathcurveto{\pgfqpoint{1.765803in}{3.152372in}}{\pgfqpoint{1.762531in}{3.160272in}}{\pgfqpoint{1.756707in}{3.166096in}}%
\pgfpathcurveto{\pgfqpoint{1.750883in}{3.171920in}}{\pgfqpoint{1.742983in}{3.175192in}}{\pgfqpoint{1.734747in}{3.175192in}}%
\pgfpathcurveto{\pgfqpoint{1.726511in}{3.175192in}}{\pgfqpoint{1.718611in}{3.171920in}}{\pgfqpoint{1.712787in}{3.166096in}}%
\pgfpathcurveto{\pgfqpoint{1.706963in}{3.160272in}}{\pgfqpoint{1.703690in}{3.152372in}}{\pgfqpoint{1.703690in}{3.144135in}}%
\pgfpathcurveto{\pgfqpoint{1.703690in}{3.135899in}}{\pgfqpoint{1.706963in}{3.127999in}}{\pgfqpoint{1.712787in}{3.122175in}}%
\pgfpathcurveto{\pgfqpoint{1.718611in}{3.116351in}}{\pgfqpoint{1.726511in}{3.113079in}}{\pgfqpoint{1.734747in}{3.113079in}}%
\pgfpathclose%
\pgfusepath{stroke,fill}%
\end{pgfscope}%
\begin{pgfscope}%
\pgfpathrectangle{\pgfqpoint{0.100000in}{0.212622in}}{\pgfqpoint{3.696000in}{3.696000in}}%
\pgfusepath{clip}%
\pgfsetbuttcap%
\pgfsetroundjoin%
\definecolor{currentfill}{rgb}{0.121569,0.466667,0.705882}%
\pgfsetfillcolor{currentfill}%
\pgfsetfillopacity{0.346474}%
\pgfsetlinewidth{1.003750pt}%
\definecolor{currentstroke}{rgb}{0.121569,0.466667,0.705882}%
\pgfsetstrokecolor{currentstroke}%
\pgfsetstrokeopacity{0.346474}%
\pgfsetdash{}{0pt}%
\pgfpathmoveto{\pgfqpoint{1.731981in}{3.108816in}}%
\pgfpathcurveto{\pgfqpoint{1.740217in}{3.108816in}}{\pgfqpoint{1.748118in}{3.112089in}}{\pgfqpoint{1.753941in}{3.117913in}}%
\pgfpathcurveto{\pgfqpoint{1.759765in}{3.123736in}}{\pgfqpoint{1.763038in}{3.131637in}}{\pgfqpoint{1.763038in}{3.139873in}}%
\pgfpathcurveto{\pgfqpoint{1.763038in}{3.148109in}}{\pgfqpoint{1.759765in}{3.156009in}}{\pgfqpoint{1.753941in}{3.161833in}}%
\pgfpathcurveto{\pgfqpoint{1.748118in}{3.167657in}}{\pgfqpoint{1.740217in}{3.170929in}}{\pgfqpoint{1.731981in}{3.170929in}}%
\pgfpathcurveto{\pgfqpoint{1.723745in}{3.170929in}}{\pgfqpoint{1.715845in}{3.167657in}}{\pgfqpoint{1.710021in}{3.161833in}}%
\pgfpathcurveto{\pgfqpoint{1.704197in}{3.156009in}}{\pgfqpoint{1.700925in}{3.148109in}}{\pgfqpoint{1.700925in}{3.139873in}}%
\pgfpathcurveto{\pgfqpoint{1.700925in}{3.131637in}}{\pgfqpoint{1.704197in}{3.123736in}}{\pgfqpoint{1.710021in}{3.117913in}}%
\pgfpathcurveto{\pgfqpoint{1.715845in}{3.112089in}}{\pgfqpoint{1.723745in}{3.108816in}}{\pgfqpoint{1.731981in}{3.108816in}}%
\pgfpathclose%
\pgfusepath{stroke,fill}%
\end{pgfscope}%
\begin{pgfscope}%
\pgfpathrectangle{\pgfqpoint{0.100000in}{0.212622in}}{\pgfqpoint{3.696000in}{3.696000in}}%
\pgfusepath{clip}%
\pgfsetbuttcap%
\pgfsetroundjoin%
\definecolor{currentfill}{rgb}{0.121569,0.466667,0.705882}%
\pgfsetfillcolor{currentfill}%
\pgfsetfillopacity{0.346677}%
\pgfsetlinewidth{1.003750pt}%
\definecolor{currentstroke}{rgb}{0.121569,0.466667,0.705882}%
\pgfsetstrokecolor{currentstroke}%
\pgfsetstrokeopacity{0.346677}%
\pgfsetdash{}{0pt}%
\pgfpathmoveto{\pgfqpoint{1.943341in}{3.122761in}}%
\pgfpathcurveto{\pgfqpoint{1.951577in}{3.122761in}}{\pgfqpoint{1.959477in}{3.126033in}}{\pgfqpoint{1.965301in}{3.131857in}}%
\pgfpathcurveto{\pgfqpoint{1.971125in}{3.137681in}}{\pgfqpoint{1.974398in}{3.145581in}}{\pgfqpoint{1.974398in}{3.153818in}}%
\pgfpathcurveto{\pgfqpoint{1.974398in}{3.162054in}}{\pgfqpoint{1.971125in}{3.169954in}}{\pgfqpoint{1.965301in}{3.175778in}}%
\pgfpathcurveto{\pgfqpoint{1.959477in}{3.181602in}}{\pgfqpoint{1.951577in}{3.184874in}}{\pgfqpoint{1.943341in}{3.184874in}}%
\pgfpathcurveto{\pgfqpoint{1.935105in}{3.184874in}}{\pgfqpoint{1.927205in}{3.181602in}}{\pgfqpoint{1.921381in}{3.175778in}}%
\pgfpathcurveto{\pgfqpoint{1.915557in}{3.169954in}}{\pgfqpoint{1.912285in}{3.162054in}}{\pgfqpoint{1.912285in}{3.153818in}}%
\pgfpathcurveto{\pgfqpoint{1.912285in}{3.145581in}}{\pgfqpoint{1.915557in}{3.137681in}}{\pgfqpoint{1.921381in}{3.131857in}}%
\pgfpathcurveto{\pgfqpoint{1.927205in}{3.126033in}}{\pgfqpoint{1.935105in}{3.122761in}}{\pgfqpoint{1.943341in}{3.122761in}}%
\pgfpathclose%
\pgfusepath{stroke,fill}%
\end{pgfscope}%
\begin{pgfscope}%
\pgfpathrectangle{\pgfqpoint{0.100000in}{0.212622in}}{\pgfqpoint{3.696000in}{3.696000in}}%
\pgfusepath{clip}%
\pgfsetbuttcap%
\pgfsetroundjoin%
\definecolor{currentfill}{rgb}{0.121569,0.466667,0.705882}%
\pgfsetfillcolor{currentfill}%
\pgfsetfillopacity{0.347247}%
\pgfsetlinewidth{1.003750pt}%
\definecolor{currentstroke}{rgb}{0.121569,0.466667,0.705882}%
\pgfsetstrokecolor{currentstroke}%
\pgfsetstrokeopacity{0.347247}%
\pgfsetdash{}{0pt}%
\pgfpathmoveto{\pgfqpoint{1.729904in}{3.105021in}}%
\pgfpathcurveto{\pgfqpoint{1.738141in}{3.105021in}}{\pgfqpoint{1.746041in}{3.108293in}}{\pgfqpoint{1.751865in}{3.114117in}}%
\pgfpathcurveto{\pgfqpoint{1.757688in}{3.119941in}}{\pgfqpoint{1.760961in}{3.127841in}}{\pgfqpoint{1.760961in}{3.136077in}}%
\pgfpathcurveto{\pgfqpoint{1.760961in}{3.144314in}}{\pgfqpoint{1.757688in}{3.152214in}}{\pgfqpoint{1.751865in}{3.158038in}}%
\pgfpathcurveto{\pgfqpoint{1.746041in}{3.163862in}}{\pgfqpoint{1.738141in}{3.167134in}}{\pgfqpoint{1.729904in}{3.167134in}}%
\pgfpathcurveto{\pgfqpoint{1.721668in}{3.167134in}}{\pgfqpoint{1.713768in}{3.163862in}}{\pgfqpoint{1.707944in}{3.158038in}}%
\pgfpathcurveto{\pgfqpoint{1.702120in}{3.152214in}}{\pgfqpoint{1.698848in}{3.144314in}}{\pgfqpoint{1.698848in}{3.136077in}}%
\pgfpathcurveto{\pgfqpoint{1.698848in}{3.127841in}}{\pgfqpoint{1.702120in}{3.119941in}}{\pgfqpoint{1.707944in}{3.114117in}}%
\pgfpathcurveto{\pgfqpoint{1.713768in}{3.108293in}}{\pgfqpoint{1.721668in}{3.105021in}}{\pgfqpoint{1.729904in}{3.105021in}}%
\pgfpathclose%
\pgfusepath{stroke,fill}%
\end{pgfscope}%
\begin{pgfscope}%
\pgfpathrectangle{\pgfqpoint{0.100000in}{0.212622in}}{\pgfqpoint{3.696000in}{3.696000in}}%
\pgfusepath{clip}%
\pgfsetbuttcap%
\pgfsetroundjoin%
\definecolor{currentfill}{rgb}{0.121569,0.466667,0.705882}%
\pgfsetfillcolor{currentfill}%
\pgfsetfillopacity{0.348678}%
\pgfsetlinewidth{1.003750pt}%
\definecolor{currentstroke}{rgb}{0.121569,0.466667,0.705882}%
\pgfsetstrokecolor{currentstroke}%
\pgfsetstrokeopacity{0.348678}%
\pgfsetdash{}{0pt}%
\pgfpathmoveto{\pgfqpoint{1.726116in}{3.098233in}}%
\pgfpathcurveto{\pgfqpoint{1.734352in}{3.098233in}}{\pgfqpoint{1.742252in}{3.101505in}}{\pgfqpoint{1.748076in}{3.107329in}}%
\pgfpathcurveto{\pgfqpoint{1.753900in}{3.113153in}}{\pgfqpoint{1.757172in}{3.121053in}}{\pgfqpoint{1.757172in}{3.129289in}}%
\pgfpathcurveto{\pgfqpoint{1.757172in}{3.137525in}}{\pgfqpoint{1.753900in}{3.145425in}}{\pgfqpoint{1.748076in}{3.151249in}}%
\pgfpathcurveto{\pgfqpoint{1.742252in}{3.157073in}}{\pgfqpoint{1.734352in}{3.160346in}}{\pgfqpoint{1.726116in}{3.160346in}}%
\pgfpathcurveto{\pgfqpoint{1.717879in}{3.160346in}}{\pgfqpoint{1.709979in}{3.157073in}}{\pgfqpoint{1.704155in}{3.151249in}}%
\pgfpathcurveto{\pgfqpoint{1.698332in}{3.145425in}}{\pgfqpoint{1.695059in}{3.137525in}}{\pgfqpoint{1.695059in}{3.129289in}}%
\pgfpathcurveto{\pgfqpoint{1.695059in}{3.121053in}}{\pgfqpoint{1.698332in}{3.113153in}}{\pgfqpoint{1.704155in}{3.107329in}}%
\pgfpathcurveto{\pgfqpoint{1.709979in}{3.101505in}}{\pgfqpoint{1.717879in}{3.098233in}}{\pgfqpoint{1.726116in}{3.098233in}}%
\pgfpathclose%
\pgfusepath{stroke,fill}%
\end{pgfscope}%
\begin{pgfscope}%
\pgfpathrectangle{\pgfqpoint{0.100000in}{0.212622in}}{\pgfqpoint{3.696000in}{3.696000in}}%
\pgfusepath{clip}%
\pgfsetbuttcap%
\pgfsetroundjoin%
\definecolor{currentfill}{rgb}{0.121569,0.466667,0.705882}%
\pgfsetfillcolor{currentfill}%
\pgfsetfillopacity{0.348909}%
\pgfsetlinewidth{1.003750pt}%
\definecolor{currentstroke}{rgb}{0.121569,0.466667,0.705882}%
\pgfsetstrokecolor{currentstroke}%
\pgfsetstrokeopacity{0.348909}%
\pgfsetdash{}{0pt}%
\pgfpathmoveto{\pgfqpoint{1.944758in}{3.114297in}}%
\pgfpathcurveto{\pgfqpoint{1.952995in}{3.114297in}}{\pgfqpoint{1.960895in}{3.117569in}}{\pgfqpoint{1.966719in}{3.123393in}}%
\pgfpathcurveto{\pgfqpoint{1.972542in}{3.129217in}}{\pgfqpoint{1.975815in}{3.137117in}}{\pgfqpoint{1.975815in}{3.145353in}}%
\pgfpathcurveto{\pgfqpoint{1.975815in}{3.153590in}}{\pgfqpoint{1.972542in}{3.161490in}}{\pgfqpoint{1.966719in}{3.167314in}}%
\pgfpathcurveto{\pgfqpoint{1.960895in}{3.173138in}}{\pgfqpoint{1.952995in}{3.176410in}}{\pgfqpoint{1.944758in}{3.176410in}}%
\pgfpathcurveto{\pgfqpoint{1.936522in}{3.176410in}}{\pgfqpoint{1.928622in}{3.173138in}}{\pgfqpoint{1.922798in}{3.167314in}}%
\pgfpathcurveto{\pgfqpoint{1.916974in}{3.161490in}}{\pgfqpoint{1.913702in}{3.153590in}}{\pgfqpoint{1.913702in}{3.145353in}}%
\pgfpathcurveto{\pgfqpoint{1.913702in}{3.137117in}}{\pgfqpoint{1.916974in}{3.129217in}}{\pgfqpoint{1.922798in}{3.123393in}}%
\pgfpathcurveto{\pgfqpoint{1.928622in}{3.117569in}}{\pgfqpoint{1.936522in}{3.114297in}}{\pgfqpoint{1.944758in}{3.114297in}}%
\pgfpathclose%
\pgfusepath{stroke,fill}%
\end{pgfscope}%
\begin{pgfscope}%
\pgfpathrectangle{\pgfqpoint{0.100000in}{0.212622in}}{\pgfqpoint{3.696000in}{3.696000in}}%
\pgfusepath{clip}%
\pgfsetbuttcap%
\pgfsetroundjoin%
\definecolor{currentfill}{rgb}{0.121569,0.466667,0.705882}%
\pgfsetfillcolor{currentfill}%
\pgfsetfillopacity{0.349761}%
\pgfsetlinewidth{1.003750pt}%
\definecolor{currentstroke}{rgb}{0.121569,0.466667,0.705882}%
\pgfsetstrokecolor{currentstroke}%
\pgfsetstrokeopacity{0.349761}%
\pgfsetdash{}{0pt}%
\pgfpathmoveto{\pgfqpoint{1.722273in}{3.092026in}}%
\pgfpathcurveto{\pgfqpoint{1.730510in}{3.092026in}}{\pgfqpoint{1.738410in}{3.095298in}}{\pgfqpoint{1.744234in}{3.101122in}}%
\pgfpathcurveto{\pgfqpoint{1.750057in}{3.106946in}}{\pgfqpoint{1.753330in}{3.114846in}}{\pgfqpoint{1.753330in}{3.123083in}}%
\pgfpathcurveto{\pgfqpoint{1.753330in}{3.131319in}}{\pgfqpoint{1.750057in}{3.139219in}}{\pgfqpoint{1.744234in}{3.145043in}}%
\pgfpathcurveto{\pgfqpoint{1.738410in}{3.150867in}}{\pgfqpoint{1.730510in}{3.154139in}}{\pgfqpoint{1.722273in}{3.154139in}}%
\pgfpathcurveto{\pgfqpoint{1.714037in}{3.154139in}}{\pgfqpoint{1.706137in}{3.150867in}}{\pgfqpoint{1.700313in}{3.145043in}}%
\pgfpathcurveto{\pgfqpoint{1.694489in}{3.139219in}}{\pgfqpoint{1.691217in}{3.131319in}}{\pgfqpoint{1.691217in}{3.123083in}}%
\pgfpathcurveto{\pgfqpoint{1.691217in}{3.114846in}}{\pgfqpoint{1.694489in}{3.106946in}}{\pgfqpoint{1.700313in}{3.101122in}}%
\pgfpathcurveto{\pgfqpoint{1.706137in}{3.095298in}}{\pgfqpoint{1.714037in}{3.092026in}}{\pgfqpoint{1.722273in}{3.092026in}}%
\pgfpathclose%
\pgfusepath{stroke,fill}%
\end{pgfscope}%
\begin{pgfscope}%
\pgfpathrectangle{\pgfqpoint{0.100000in}{0.212622in}}{\pgfqpoint{3.696000in}{3.696000in}}%
\pgfusepath{clip}%
\pgfsetbuttcap%
\pgfsetroundjoin%
\definecolor{currentfill}{rgb}{0.121569,0.466667,0.705882}%
\pgfsetfillcolor{currentfill}%
\pgfsetfillopacity{0.350050}%
\pgfsetlinewidth{1.003750pt}%
\definecolor{currentstroke}{rgb}{0.121569,0.466667,0.705882}%
\pgfsetstrokecolor{currentstroke}%
\pgfsetstrokeopacity{0.350050}%
\pgfsetdash{}{0pt}%
\pgfpathmoveto{\pgfqpoint{1.945670in}{3.109412in}}%
\pgfpathcurveto{\pgfqpoint{1.953906in}{3.109412in}}{\pgfqpoint{1.961806in}{3.112684in}}{\pgfqpoint{1.967630in}{3.118508in}}%
\pgfpathcurveto{\pgfqpoint{1.973454in}{3.124332in}}{\pgfqpoint{1.976727in}{3.132232in}}{\pgfqpoint{1.976727in}{3.140468in}}%
\pgfpathcurveto{\pgfqpoint{1.976727in}{3.148705in}}{\pgfqpoint{1.973454in}{3.156605in}}{\pgfqpoint{1.967630in}{3.162429in}}%
\pgfpathcurveto{\pgfqpoint{1.961806in}{3.168253in}}{\pgfqpoint{1.953906in}{3.171525in}}{\pgfqpoint{1.945670in}{3.171525in}}%
\pgfpathcurveto{\pgfqpoint{1.937434in}{3.171525in}}{\pgfqpoint{1.929534in}{3.168253in}}{\pgfqpoint{1.923710in}{3.162429in}}%
\pgfpathcurveto{\pgfqpoint{1.917886in}{3.156605in}}{\pgfqpoint{1.914614in}{3.148705in}}{\pgfqpoint{1.914614in}{3.140468in}}%
\pgfpathcurveto{\pgfqpoint{1.914614in}{3.132232in}}{\pgfqpoint{1.917886in}{3.124332in}}{\pgfqpoint{1.923710in}{3.118508in}}%
\pgfpathcurveto{\pgfqpoint{1.929534in}{3.112684in}}{\pgfqpoint{1.937434in}{3.109412in}}{\pgfqpoint{1.945670in}{3.109412in}}%
\pgfpathclose%
\pgfusepath{stroke,fill}%
\end{pgfscope}%
\begin{pgfscope}%
\pgfpathrectangle{\pgfqpoint{0.100000in}{0.212622in}}{\pgfqpoint{3.696000in}{3.696000in}}%
\pgfusepath{clip}%
\pgfsetbuttcap%
\pgfsetroundjoin%
\definecolor{currentfill}{rgb}{0.121569,0.466667,0.705882}%
\pgfsetfillcolor{currentfill}%
\pgfsetfillopacity{0.350703}%
\pgfsetlinewidth{1.003750pt}%
\definecolor{currentstroke}{rgb}{0.121569,0.466667,0.705882}%
\pgfsetstrokecolor{currentstroke}%
\pgfsetstrokeopacity{0.350703}%
\pgfsetdash{}{0pt}%
\pgfpathmoveto{\pgfqpoint{1.719911in}{3.086967in}}%
\pgfpathcurveto{\pgfqpoint{1.728147in}{3.086967in}}{\pgfqpoint{1.736047in}{3.090239in}}{\pgfqpoint{1.741871in}{3.096063in}}%
\pgfpathcurveto{\pgfqpoint{1.747695in}{3.101887in}}{\pgfqpoint{1.750967in}{3.109787in}}{\pgfqpoint{1.750967in}{3.118023in}}%
\pgfpathcurveto{\pgfqpoint{1.750967in}{3.126260in}}{\pgfqpoint{1.747695in}{3.134160in}}{\pgfqpoint{1.741871in}{3.139984in}}%
\pgfpathcurveto{\pgfqpoint{1.736047in}{3.145808in}}{\pgfqpoint{1.728147in}{3.149080in}}{\pgfqpoint{1.719911in}{3.149080in}}%
\pgfpathcurveto{\pgfqpoint{1.711674in}{3.149080in}}{\pgfqpoint{1.703774in}{3.145808in}}{\pgfqpoint{1.697950in}{3.139984in}}%
\pgfpathcurveto{\pgfqpoint{1.692126in}{3.134160in}}{\pgfqpoint{1.688854in}{3.126260in}}{\pgfqpoint{1.688854in}{3.118023in}}%
\pgfpathcurveto{\pgfqpoint{1.688854in}{3.109787in}}{\pgfqpoint{1.692126in}{3.101887in}}{\pgfqpoint{1.697950in}{3.096063in}}%
\pgfpathcurveto{\pgfqpoint{1.703774in}{3.090239in}}{\pgfqpoint{1.711674in}{3.086967in}}{\pgfqpoint{1.719911in}{3.086967in}}%
\pgfpathclose%
\pgfusepath{stroke,fill}%
\end{pgfscope}%
\begin{pgfscope}%
\pgfpathrectangle{\pgfqpoint{0.100000in}{0.212622in}}{\pgfqpoint{3.696000in}{3.696000in}}%
\pgfusepath{clip}%
\pgfsetbuttcap%
\pgfsetroundjoin%
\definecolor{currentfill}{rgb}{0.121569,0.466667,0.705882}%
\pgfsetfillcolor{currentfill}%
\pgfsetfillopacity{0.350783}%
\pgfsetlinewidth{1.003750pt}%
\definecolor{currentstroke}{rgb}{0.121569,0.466667,0.705882}%
\pgfsetstrokecolor{currentstroke}%
\pgfsetstrokeopacity{0.350783}%
\pgfsetdash{}{0pt}%
\pgfpathmoveto{\pgfqpoint{1.945948in}{3.106959in}}%
\pgfpathcurveto{\pgfqpoint{1.954185in}{3.106959in}}{\pgfqpoint{1.962085in}{3.110232in}}{\pgfqpoint{1.967909in}{3.116055in}}%
\pgfpathcurveto{\pgfqpoint{1.973733in}{3.121879in}}{\pgfqpoint{1.977005in}{3.129779in}}{\pgfqpoint{1.977005in}{3.138016in}}%
\pgfpathcurveto{\pgfqpoint{1.977005in}{3.146252in}}{\pgfqpoint{1.973733in}{3.154152in}}{\pgfqpoint{1.967909in}{3.159976in}}%
\pgfpathcurveto{\pgfqpoint{1.962085in}{3.165800in}}{\pgfqpoint{1.954185in}{3.169072in}}{\pgfqpoint{1.945948in}{3.169072in}}%
\pgfpathcurveto{\pgfqpoint{1.937712in}{3.169072in}}{\pgfqpoint{1.929812in}{3.165800in}}{\pgfqpoint{1.923988in}{3.159976in}}%
\pgfpathcurveto{\pgfqpoint{1.918164in}{3.154152in}}{\pgfqpoint{1.914892in}{3.146252in}}{\pgfqpoint{1.914892in}{3.138016in}}%
\pgfpathcurveto{\pgfqpoint{1.914892in}{3.129779in}}{\pgfqpoint{1.918164in}{3.121879in}}{\pgfqpoint{1.923988in}{3.116055in}}%
\pgfpathcurveto{\pgfqpoint{1.929812in}{3.110232in}}{\pgfqpoint{1.937712in}{3.106959in}}{\pgfqpoint{1.945948in}{3.106959in}}%
\pgfpathclose%
\pgfusepath{stroke,fill}%
\end{pgfscope}%
\begin{pgfscope}%
\pgfpathrectangle{\pgfqpoint{0.100000in}{0.212622in}}{\pgfqpoint{3.696000in}{3.696000in}}%
\pgfusepath{clip}%
\pgfsetbuttcap%
\pgfsetroundjoin%
\definecolor{currentfill}{rgb}{0.121569,0.466667,0.705882}%
\pgfsetfillcolor{currentfill}%
\pgfsetfillopacity{0.351272}%
\pgfsetlinewidth{1.003750pt}%
\definecolor{currentstroke}{rgb}{0.121569,0.466667,0.705882}%
\pgfsetstrokecolor{currentstroke}%
\pgfsetstrokeopacity{0.351272}%
\pgfsetdash{}{0pt}%
\pgfpathmoveto{\pgfqpoint{1.718095in}{3.084160in}}%
\pgfpathcurveto{\pgfqpoint{1.726331in}{3.084160in}}{\pgfqpoint{1.734231in}{3.087433in}}{\pgfqpoint{1.740055in}{3.093257in}}%
\pgfpathcurveto{\pgfqpoint{1.745879in}{3.099080in}}{\pgfqpoint{1.749152in}{3.106981in}}{\pgfqpoint{1.749152in}{3.115217in}}%
\pgfpathcurveto{\pgfqpoint{1.749152in}{3.123453in}}{\pgfqpoint{1.745879in}{3.131353in}}{\pgfqpoint{1.740055in}{3.137177in}}%
\pgfpathcurveto{\pgfqpoint{1.734231in}{3.143001in}}{\pgfqpoint{1.726331in}{3.146273in}}{\pgfqpoint{1.718095in}{3.146273in}}%
\pgfpathcurveto{\pgfqpoint{1.709859in}{3.146273in}}{\pgfqpoint{1.701959in}{3.143001in}}{\pgfqpoint{1.696135in}{3.137177in}}%
\pgfpathcurveto{\pgfqpoint{1.690311in}{3.131353in}}{\pgfqpoint{1.687039in}{3.123453in}}{\pgfqpoint{1.687039in}{3.115217in}}%
\pgfpathcurveto{\pgfqpoint{1.687039in}{3.106981in}}{\pgfqpoint{1.690311in}{3.099080in}}{\pgfqpoint{1.696135in}{3.093257in}}%
\pgfpathcurveto{\pgfqpoint{1.701959in}{3.087433in}}{\pgfqpoint{1.709859in}{3.084160in}}{\pgfqpoint{1.718095in}{3.084160in}}%
\pgfpathclose%
\pgfusepath{stroke,fill}%
\end{pgfscope}%
\begin{pgfscope}%
\pgfpathrectangle{\pgfqpoint{0.100000in}{0.212622in}}{\pgfqpoint{3.696000in}{3.696000in}}%
\pgfusepath{clip}%
\pgfsetbuttcap%
\pgfsetroundjoin%
\definecolor{currentfill}{rgb}{0.121569,0.466667,0.705882}%
\pgfsetfillcolor{currentfill}%
\pgfsetfillopacity{0.351766}%
\pgfsetlinewidth{1.003750pt}%
\definecolor{currentstroke}{rgb}{0.121569,0.466667,0.705882}%
\pgfsetstrokecolor{currentstroke}%
\pgfsetstrokeopacity{0.351766}%
\pgfsetdash{}{0pt}%
\pgfpathmoveto{\pgfqpoint{1.946729in}{3.103163in}}%
\pgfpathcurveto{\pgfqpoint{1.954965in}{3.103163in}}{\pgfqpoint{1.962865in}{3.106436in}}{\pgfqpoint{1.968689in}{3.112260in}}%
\pgfpathcurveto{\pgfqpoint{1.974513in}{3.118084in}}{\pgfqpoint{1.977785in}{3.125984in}}{\pgfqpoint{1.977785in}{3.134220in}}%
\pgfpathcurveto{\pgfqpoint{1.977785in}{3.142456in}}{\pgfqpoint{1.974513in}{3.150356in}}{\pgfqpoint{1.968689in}{3.156180in}}%
\pgfpathcurveto{\pgfqpoint{1.962865in}{3.162004in}}{\pgfqpoint{1.954965in}{3.165276in}}{\pgfqpoint{1.946729in}{3.165276in}}%
\pgfpathcurveto{\pgfqpoint{1.938492in}{3.165276in}}{\pgfqpoint{1.930592in}{3.162004in}}{\pgfqpoint{1.924768in}{3.156180in}}%
\pgfpathcurveto{\pgfqpoint{1.918945in}{3.150356in}}{\pgfqpoint{1.915672in}{3.142456in}}{\pgfqpoint{1.915672in}{3.134220in}}%
\pgfpathcurveto{\pgfqpoint{1.915672in}{3.125984in}}{\pgfqpoint{1.918945in}{3.118084in}}{\pgfqpoint{1.924768in}{3.112260in}}%
\pgfpathcurveto{\pgfqpoint{1.930592in}{3.106436in}}{\pgfqpoint{1.938492in}{3.103163in}}{\pgfqpoint{1.946729in}{3.103163in}}%
\pgfpathclose%
\pgfusepath{stroke,fill}%
\end{pgfscope}%
\begin{pgfscope}%
\pgfpathrectangle{\pgfqpoint{0.100000in}{0.212622in}}{\pgfqpoint{3.696000in}{3.696000in}}%
\pgfusepath{clip}%
\pgfsetbuttcap%
\pgfsetroundjoin%
\definecolor{currentfill}{rgb}{0.121569,0.466667,0.705882}%
\pgfsetfillcolor{currentfill}%
\pgfsetfillopacity{0.352339}%
\pgfsetlinewidth{1.003750pt}%
\definecolor{currentstroke}{rgb}{0.121569,0.466667,0.705882}%
\pgfsetstrokecolor{currentstroke}%
\pgfsetstrokeopacity{0.352339}%
\pgfsetdash{}{0pt}%
\pgfpathmoveto{\pgfqpoint{1.715028in}{3.078846in}}%
\pgfpathcurveto{\pgfqpoint{1.723264in}{3.078846in}}{\pgfqpoint{1.731164in}{3.082118in}}{\pgfqpoint{1.736988in}{3.087942in}}%
\pgfpathcurveto{\pgfqpoint{1.742812in}{3.093766in}}{\pgfqpoint{1.746085in}{3.101666in}}{\pgfqpoint{1.746085in}{3.109903in}}%
\pgfpathcurveto{\pgfqpoint{1.746085in}{3.118139in}}{\pgfqpoint{1.742812in}{3.126039in}}{\pgfqpoint{1.736988in}{3.131863in}}%
\pgfpathcurveto{\pgfqpoint{1.731164in}{3.137687in}}{\pgfqpoint{1.723264in}{3.140959in}}{\pgfqpoint{1.715028in}{3.140959in}}%
\pgfpathcurveto{\pgfqpoint{1.706792in}{3.140959in}}{\pgfqpoint{1.698892in}{3.137687in}}{\pgfqpoint{1.693068in}{3.131863in}}%
\pgfpathcurveto{\pgfqpoint{1.687244in}{3.126039in}}{\pgfqpoint{1.683972in}{3.118139in}}{\pgfqpoint{1.683972in}{3.109903in}}%
\pgfpathcurveto{\pgfqpoint{1.683972in}{3.101666in}}{\pgfqpoint{1.687244in}{3.093766in}}{\pgfqpoint{1.693068in}{3.087942in}}%
\pgfpathcurveto{\pgfqpoint{1.698892in}{3.082118in}}{\pgfqpoint{1.706792in}{3.078846in}}{\pgfqpoint{1.715028in}{3.078846in}}%
\pgfpathclose%
\pgfusepath{stroke,fill}%
\end{pgfscope}%
\begin{pgfscope}%
\pgfpathrectangle{\pgfqpoint{0.100000in}{0.212622in}}{\pgfqpoint{3.696000in}{3.696000in}}%
\pgfusepath{clip}%
\pgfsetbuttcap%
\pgfsetroundjoin%
\definecolor{currentfill}{rgb}{0.121569,0.466667,0.705882}%
\pgfsetfillcolor{currentfill}%
\pgfsetfillopacity{0.352905}%
\pgfsetlinewidth{1.003750pt}%
\definecolor{currentstroke}{rgb}{0.121569,0.466667,0.705882}%
\pgfsetstrokecolor{currentstroke}%
\pgfsetstrokeopacity{0.352905}%
\pgfsetdash{}{0pt}%
\pgfpathmoveto{\pgfqpoint{1.947534in}{3.098568in}}%
\pgfpathcurveto{\pgfqpoint{1.955771in}{3.098568in}}{\pgfqpoint{1.963671in}{3.101840in}}{\pgfqpoint{1.969495in}{3.107664in}}%
\pgfpathcurveto{\pgfqpoint{1.975319in}{3.113488in}}{\pgfqpoint{1.978591in}{3.121388in}}{\pgfqpoint{1.978591in}{3.129624in}}%
\pgfpathcurveto{\pgfqpoint{1.978591in}{3.137861in}}{\pgfqpoint{1.975319in}{3.145761in}}{\pgfqpoint{1.969495in}{3.151585in}}%
\pgfpathcurveto{\pgfqpoint{1.963671in}{3.157409in}}{\pgfqpoint{1.955771in}{3.160681in}}{\pgfqpoint{1.947534in}{3.160681in}}%
\pgfpathcurveto{\pgfqpoint{1.939298in}{3.160681in}}{\pgfqpoint{1.931398in}{3.157409in}}{\pgfqpoint{1.925574in}{3.151585in}}%
\pgfpathcurveto{\pgfqpoint{1.919750in}{3.145761in}}{\pgfqpoint{1.916478in}{3.137861in}}{\pgfqpoint{1.916478in}{3.129624in}}%
\pgfpathcurveto{\pgfqpoint{1.916478in}{3.121388in}}{\pgfqpoint{1.919750in}{3.113488in}}{\pgfqpoint{1.925574in}{3.107664in}}%
\pgfpathcurveto{\pgfqpoint{1.931398in}{3.101840in}}{\pgfqpoint{1.939298in}{3.098568in}}{\pgfqpoint{1.947534in}{3.098568in}}%
\pgfpathclose%
\pgfusepath{stroke,fill}%
\end{pgfscope}%
\begin{pgfscope}%
\pgfpathrectangle{\pgfqpoint{0.100000in}{0.212622in}}{\pgfqpoint{3.696000in}{3.696000in}}%
\pgfusepath{clip}%
\pgfsetbuttcap%
\pgfsetroundjoin%
\definecolor{currentfill}{rgb}{0.121569,0.466667,0.705882}%
\pgfsetfillcolor{currentfill}%
\pgfsetfillopacity{0.353328}%
\pgfsetlinewidth{1.003750pt}%
\definecolor{currentstroke}{rgb}{0.121569,0.466667,0.705882}%
\pgfsetstrokecolor{currentstroke}%
\pgfsetstrokeopacity{0.353328}%
\pgfsetdash{}{0pt}%
\pgfpathmoveto{\pgfqpoint{1.712312in}{3.073849in}}%
\pgfpathcurveto{\pgfqpoint{1.720548in}{3.073849in}}{\pgfqpoint{1.728448in}{3.077121in}}{\pgfqpoint{1.734272in}{3.082945in}}%
\pgfpathcurveto{\pgfqpoint{1.740096in}{3.088769in}}{\pgfqpoint{1.743369in}{3.096669in}}{\pgfqpoint{1.743369in}{3.104905in}}%
\pgfpathcurveto{\pgfqpoint{1.743369in}{3.113142in}}{\pgfqpoint{1.740096in}{3.121042in}}{\pgfqpoint{1.734272in}{3.126866in}}%
\pgfpathcurveto{\pgfqpoint{1.728448in}{3.132690in}}{\pgfqpoint{1.720548in}{3.135962in}}{\pgfqpoint{1.712312in}{3.135962in}}%
\pgfpathcurveto{\pgfqpoint{1.704076in}{3.135962in}}{\pgfqpoint{1.696176in}{3.132690in}}{\pgfqpoint{1.690352in}{3.126866in}}%
\pgfpathcurveto{\pgfqpoint{1.684528in}{3.121042in}}{\pgfqpoint{1.681256in}{3.113142in}}{\pgfqpoint{1.681256in}{3.104905in}}%
\pgfpathcurveto{\pgfqpoint{1.681256in}{3.096669in}}{\pgfqpoint{1.684528in}{3.088769in}}{\pgfqpoint{1.690352in}{3.082945in}}%
\pgfpathcurveto{\pgfqpoint{1.696176in}{3.077121in}}{\pgfqpoint{1.704076in}{3.073849in}}{\pgfqpoint{1.712312in}{3.073849in}}%
\pgfpathclose%
\pgfusepath{stroke,fill}%
\end{pgfscope}%
\begin{pgfscope}%
\pgfpathrectangle{\pgfqpoint{0.100000in}{0.212622in}}{\pgfqpoint{3.696000in}{3.696000in}}%
\pgfusepath{clip}%
\pgfsetbuttcap%
\pgfsetroundjoin%
\definecolor{currentfill}{rgb}{0.121569,0.466667,0.705882}%
\pgfsetfillcolor{currentfill}%
\pgfsetfillopacity{0.354029}%
\pgfsetlinewidth{1.003750pt}%
\definecolor{currentstroke}{rgb}{0.121569,0.466667,0.705882}%
\pgfsetstrokecolor{currentstroke}%
\pgfsetstrokeopacity{0.354029}%
\pgfsetdash{}{0pt}%
\pgfpathmoveto{\pgfqpoint{1.709788in}{3.069952in}}%
\pgfpathcurveto{\pgfqpoint{1.718024in}{3.069952in}}{\pgfqpoint{1.725925in}{3.073225in}}{\pgfqpoint{1.731748in}{3.079049in}}%
\pgfpathcurveto{\pgfqpoint{1.737572in}{3.084873in}}{\pgfqpoint{1.740845in}{3.092773in}}{\pgfqpoint{1.740845in}{3.101009in}}%
\pgfpathcurveto{\pgfqpoint{1.740845in}{3.109245in}}{\pgfqpoint{1.737572in}{3.117145in}}{\pgfqpoint{1.731748in}{3.122969in}}%
\pgfpathcurveto{\pgfqpoint{1.725925in}{3.128793in}}{\pgfqpoint{1.718024in}{3.132065in}}{\pgfqpoint{1.709788in}{3.132065in}}%
\pgfpathcurveto{\pgfqpoint{1.701552in}{3.132065in}}{\pgfqpoint{1.693652in}{3.128793in}}{\pgfqpoint{1.687828in}{3.122969in}}%
\pgfpathcurveto{\pgfqpoint{1.682004in}{3.117145in}}{\pgfqpoint{1.678732in}{3.109245in}}{\pgfqpoint{1.678732in}{3.101009in}}%
\pgfpathcurveto{\pgfqpoint{1.678732in}{3.092773in}}{\pgfqpoint{1.682004in}{3.084873in}}{\pgfqpoint{1.687828in}{3.079049in}}%
\pgfpathcurveto{\pgfqpoint{1.693652in}{3.073225in}}{\pgfqpoint{1.701552in}{3.069952in}}{\pgfqpoint{1.709788in}{3.069952in}}%
\pgfpathclose%
\pgfusepath{stroke,fill}%
\end{pgfscope}%
\begin{pgfscope}%
\pgfpathrectangle{\pgfqpoint{0.100000in}{0.212622in}}{\pgfqpoint{3.696000in}{3.696000in}}%
\pgfusepath{clip}%
\pgfsetbuttcap%
\pgfsetroundjoin%
\definecolor{currentfill}{rgb}{0.121569,0.466667,0.705882}%
\pgfsetfillcolor{currentfill}%
\pgfsetfillopacity{0.354338}%
\pgfsetlinewidth{1.003750pt}%
\definecolor{currentstroke}{rgb}{0.121569,0.466667,0.705882}%
\pgfsetstrokecolor{currentstroke}%
\pgfsetstrokeopacity{0.354338}%
\pgfsetdash{}{0pt}%
\pgfpathmoveto{\pgfqpoint{1.948215in}{3.093557in}}%
\pgfpathcurveto{\pgfqpoint{1.956451in}{3.093557in}}{\pgfqpoint{1.964351in}{3.096830in}}{\pgfqpoint{1.970175in}{3.102654in}}%
\pgfpathcurveto{\pgfqpoint{1.975999in}{3.108478in}}{\pgfqpoint{1.979271in}{3.116378in}}{\pgfqpoint{1.979271in}{3.124614in}}%
\pgfpathcurveto{\pgfqpoint{1.979271in}{3.132850in}}{\pgfqpoint{1.975999in}{3.140750in}}{\pgfqpoint{1.970175in}{3.146574in}}%
\pgfpathcurveto{\pgfqpoint{1.964351in}{3.152398in}}{\pgfqpoint{1.956451in}{3.155670in}}{\pgfqpoint{1.948215in}{3.155670in}}%
\pgfpathcurveto{\pgfqpoint{1.939978in}{3.155670in}}{\pgfqpoint{1.932078in}{3.152398in}}{\pgfqpoint{1.926254in}{3.146574in}}%
\pgfpathcurveto{\pgfqpoint{1.920430in}{3.140750in}}{\pgfqpoint{1.917158in}{3.132850in}}{\pgfqpoint{1.917158in}{3.124614in}}%
\pgfpathcurveto{\pgfqpoint{1.917158in}{3.116378in}}{\pgfqpoint{1.920430in}{3.108478in}}{\pgfqpoint{1.926254in}{3.102654in}}%
\pgfpathcurveto{\pgfqpoint{1.932078in}{3.096830in}}{\pgfqpoint{1.939978in}{3.093557in}}{\pgfqpoint{1.948215in}{3.093557in}}%
\pgfpathclose%
\pgfusepath{stroke,fill}%
\end{pgfscope}%
\begin{pgfscope}%
\pgfpathrectangle{\pgfqpoint{0.100000in}{0.212622in}}{\pgfqpoint{3.696000in}{3.696000in}}%
\pgfusepath{clip}%
\pgfsetbuttcap%
\pgfsetroundjoin%
\definecolor{currentfill}{rgb}{0.121569,0.466667,0.705882}%
\pgfsetfillcolor{currentfill}%
\pgfsetfillopacity{0.354526}%
\pgfsetlinewidth{1.003750pt}%
\definecolor{currentstroke}{rgb}{0.121569,0.466667,0.705882}%
\pgfsetstrokecolor{currentstroke}%
\pgfsetstrokeopacity{0.354526}%
\pgfsetdash{}{0pt}%
\pgfpathmoveto{\pgfqpoint{1.708518in}{3.067389in}}%
\pgfpathcurveto{\pgfqpoint{1.716755in}{3.067389in}}{\pgfqpoint{1.724655in}{3.070661in}}{\pgfqpoint{1.730479in}{3.076485in}}%
\pgfpathcurveto{\pgfqpoint{1.736302in}{3.082309in}}{\pgfqpoint{1.739575in}{3.090209in}}{\pgfqpoint{1.739575in}{3.098445in}}%
\pgfpathcurveto{\pgfqpoint{1.739575in}{3.106681in}}{\pgfqpoint{1.736302in}{3.114581in}}{\pgfqpoint{1.730479in}{3.120405in}}%
\pgfpathcurveto{\pgfqpoint{1.724655in}{3.126229in}}{\pgfqpoint{1.716755in}{3.129502in}}{\pgfqpoint{1.708518in}{3.129502in}}%
\pgfpathcurveto{\pgfqpoint{1.700282in}{3.129502in}}{\pgfqpoint{1.692382in}{3.126229in}}{\pgfqpoint{1.686558in}{3.120405in}}%
\pgfpathcurveto{\pgfqpoint{1.680734in}{3.114581in}}{\pgfqpoint{1.677462in}{3.106681in}}{\pgfqpoint{1.677462in}{3.098445in}}%
\pgfpathcurveto{\pgfqpoint{1.677462in}{3.090209in}}{\pgfqpoint{1.680734in}{3.082309in}}{\pgfqpoint{1.686558in}{3.076485in}}%
\pgfpathcurveto{\pgfqpoint{1.692382in}{3.070661in}}{\pgfqpoint{1.700282in}{3.067389in}}{\pgfqpoint{1.708518in}{3.067389in}}%
\pgfpathclose%
\pgfusepath{stroke,fill}%
\end{pgfscope}%
\begin{pgfscope}%
\pgfpathrectangle{\pgfqpoint{0.100000in}{0.212622in}}{\pgfqpoint{3.696000in}{3.696000in}}%
\pgfusepath{clip}%
\pgfsetbuttcap%
\pgfsetroundjoin%
\definecolor{currentfill}{rgb}{0.121569,0.466667,0.705882}%
\pgfsetfillcolor{currentfill}%
\pgfsetfillopacity{0.354939}%
\pgfsetlinewidth{1.003750pt}%
\definecolor{currentstroke}{rgb}{0.121569,0.466667,0.705882}%
\pgfsetstrokecolor{currentstroke}%
\pgfsetstrokeopacity{0.354939}%
\pgfsetdash{}{0pt}%
\pgfpathmoveto{\pgfqpoint{1.707321in}{3.065403in}}%
\pgfpathcurveto{\pgfqpoint{1.715557in}{3.065403in}}{\pgfqpoint{1.723457in}{3.068675in}}{\pgfqpoint{1.729281in}{3.074499in}}%
\pgfpathcurveto{\pgfqpoint{1.735105in}{3.080323in}}{\pgfqpoint{1.738377in}{3.088223in}}{\pgfqpoint{1.738377in}{3.096459in}}%
\pgfpathcurveto{\pgfqpoint{1.738377in}{3.104696in}}{\pgfqpoint{1.735105in}{3.112596in}}{\pgfqpoint{1.729281in}{3.118420in}}%
\pgfpathcurveto{\pgfqpoint{1.723457in}{3.124244in}}{\pgfqpoint{1.715557in}{3.127516in}}{\pgfqpoint{1.707321in}{3.127516in}}%
\pgfpathcurveto{\pgfqpoint{1.699084in}{3.127516in}}{\pgfqpoint{1.691184in}{3.124244in}}{\pgfqpoint{1.685360in}{3.118420in}}%
\pgfpathcurveto{\pgfqpoint{1.679536in}{3.112596in}}{\pgfqpoint{1.676264in}{3.104696in}}{\pgfqpoint{1.676264in}{3.096459in}}%
\pgfpathcurveto{\pgfqpoint{1.676264in}{3.088223in}}{\pgfqpoint{1.679536in}{3.080323in}}{\pgfqpoint{1.685360in}{3.074499in}}%
\pgfpathcurveto{\pgfqpoint{1.691184in}{3.068675in}}{\pgfqpoint{1.699084in}{3.065403in}}{\pgfqpoint{1.707321in}{3.065403in}}%
\pgfpathclose%
\pgfusepath{stroke,fill}%
\end{pgfscope}%
\begin{pgfscope}%
\pgfpathrectangle{\pgfqpoint{0.100000in}{0.212622in}}{\pgfqpoint{3.696000in}{3.696000in}}%
\pgfusepath{clip}%
\pgfsetbuttcap%
\pgfsetroundjoin%
\definecolor{currentfill}{rgb}{0.121569,0.466667,0.705882}%
\pgfsetfillcolor{currentfill}%
\pgfsetfillopacity{0.355637}%
\pgfsetlinewidth{1.003750pt}%
\definecolor{currentstroke}{rgb}{0.121569,0.466667,0.705882}%
\pgfsetstrokecolor{currentstroke}%
\pgfsetstrokeopacity{0.355637}%
\pgfsetdash{}{0pt}%
\pgfpathmoveto{\pgfqpoint{1.705037in}{3.061743in}}%
\pgfpathcurveto{\pgfqpoint{1.713274in}{3.061743in}}{\pgfqpoint{1.721174in}{3.065016in}}{\pgfqpoint{1.726998in}{3.070839in}}%
\pgfpathcurveto{\pgfqpoint{1.732822in}{3.076663in}}{\pgfqpoint{1.736094in}{3.084563in}}{\pgfqpoint{1.736094in}{3.092800in}}%
\pgfpathcurveto{\pgfqpoint{1.736094in}{3.101036in}}{\pgfqpoint{1.732822in}{3.108936in}}{\pgfqpoint{1.726998in}{3.114760in}}%
\pgfpathcurveto{\pgfqpoint{1.721174in}{3.120584in}}{\pgfqpoint{1.713274in}{3.123856in}}{\pgfqpoint{1.705037in}{3.123856in}}%
\pgfpathcurveto{\pgfqpoint{1.696801in}{3.123856in}}{\pgfqpoint{1.688901in}{3.120584in}}{\pgfqpoint{1.683077in}{3.114760in}}%
\pgfpathcurveto{\pgfqpoint{1.677253in}{3.108936in}}{\pgfqpoint{1.673981in}{3.101036in}}{\pgfqpoint{1.673981in}{3.092800in}}%
\pgfpathcurveto{\pgfqpoint{1.673981in}{3.084563in}}{\pgfqpoint{1.677253in}{3.076663in}}{\pgfqpoint{1.683077in}{3.070839in}}%
\pgfpathcurveto{\pgfqpoint{1.688901in}{3.065016in}}{\pgfqpoint{1.696801in}{3.061743in}}{\pgfqpoint{1.705037in}{3.061743in}}%
\pgfpathclose%
\pgfusepath{stroke,fill}%
\end{pgfscope}%
\begin{pgfscope}%
\pgfpathrectangle{\pgfqpoint{0.100000in}{0.212622in}}{\pgfqpoint{3.696000in}{3.696000in}}%
\pgfusepath{clip}%
\pgfsetbuttcap%
\pgfsetroundjoin%
\definecolor{currentfill}{rgb}{0.121569,0.466667,0.705882}%
\pgfsetfillcolor{currentfill}%
\pgfsetfillopacity{0.355714}%
\pgfsetlinewidth{1.003750pt}%
\definecolor{currentstroke}{rgb}{0.121569,0.466667,0.705882}%
\pgfsetstrokecolor{currentstroke}%
\pgfsetstrokeopacity{0.355714}%
\pgfsetdash{}{0pt}%
\pgfpathmoveto{\pgfqpoint{1.949352in}{3.087718in}}%
\pgfpathcurveto{\pgfqpoint{1.957588in}{3.087718in}}{\pgfqpoint{1.965488in}{3.090990in}}{\pgfqpoint{1.971312in}{3.096814in}}%
\pgfpathcurveto{\pgfqpoint{1.977136in}{3.102638in}}{\pgfqpoint{1.980409in}{3.110538in}}{\pgfqpoint{1.980409in}{3.118774in}}%
\pgfpathcurveto{\pgfqpoint{1.980409in}{3.127010in}}{\pgfqpoint{1.977136in}{3.134911in}}{\pgfqpoint{1.971312in}{3.140734in}}%
\pgfpathcurveto{\pgfqpoint{1.965488in}{3.146558in}}{\pgfqpoint{1.957588in}{3.149831in}}{\pgfqpoint{1.949352in}{3.149831in}}%
\pgfpathcurveto{\pgfqpoint{1.941116in}{3.149831in}}{\pgfqpoint{1.933216in}{3.146558in}}{\pgfqpoint{1.927392in}{3.140734in}}%
\pgfpathcurveto{\pgfqpoint{1.921568in}{3.134911in}}{\pgfqpoint{1.918296in}{3.127010in}}{\pgfqpoint{1.918296in}{3.118774in}}%
\pgfpathcurveto{\pgfqpoint{1.918296in}{3.110538in}}{\pgfqpoint{1.921568in}{3.102638in}}{\pgfqpoint{1.927392in}{3.096814in}}%
\pgfpathcurveto{\pgfqpoint{1.933216in}{3.090990in}}{\pgfqpoint{1.941116in}{3.087718in}}{\pgfqpoint{1.949352in}{3.087718in}}%
\pgfpathclose%
\pgfusepath{stroke,fill}%
\end{pgfscope}%
\begin{pgfscope}%
\pgfpathrectangle{\pgfqpoint{0.100000in}{0.212622in}}{\pgfqpoint{3.696000in}{3.696000in}}%
\pgfusepath{clip}%
\pgfsetbuttcap%
\pgfsetroundjoin%
\definecolor{currentfill}{rgb}{0.121569,0.466667,0.705882}%
\pgfsetfillcolor{currentfill}%
\pgfsetfillopacity{0.356291}%
\pgfsetlinewidth{1.003750pt}%
\definecolor{currentstroke}{rgb}{0.121569,0.466667,0.705882}%
\pgfsetstrokecolor{currentstroke}%
\pgfsetstrokeopacity{0.356291}%
\pgfsetdash{}{0pt}%
\pgfpathmoveto{\pgfqpoint{1.703391in}{3.058330in}}%
\pgfpathcurveto{\pgfqpoint{1.711627in}{3.058330in}}{\pgfqpoint{1.719528in}{3.061602in}}{\pgfqpoint{1.725351in}{3.067426in}}%
\pgfpathcurveto{\pgfqpoint{1.731175in}{3.073250in}}{\pgfqpoint{1.734448in}{3.081150in}}{\pgfqpoint{1.734448in}{3.089387in}}%
\pgfpathcurveto{\pgfqpoint{1.734448in}{3.097623in}}{\pgfqpoint{1.731175in}{3.105523in}}{\pgfqpoint{1.725351in}{3.111347in}}%
\pgfpathcurveto{\pgfqpoint{1.719528in}{3.117171in}}{\pgfqpoint{1.711627in}{3.120443in}}{\pgfqpoint{1.703391in}{3.120443in}}%
\pgfpathcurveto{\pgfqpoint{1.695155in}{3.120443in}}{\pgfqpoint{1.687255in}{3.117171in}}{\pgfqpoint{1.681431in}{3.111347in}}%
\pgfpathcurveto{\pgfqpoint{1.675607in}{3.105523in}}{\pgfqpoint{1.672335in}{3.097623in}}{\pgfqpoint{1.672335in}{3.089387in}}%
\pgfpathcurveto{\pgfqpoint{1.672335in}{3.081150in}}{\pgfqpoint{1.675607in}{3.073250in}}{\pgfqpoint{1.681431in}{3.067426in}}%
\pgfpathcurveto{\pgfqpoint{1.687255in}{3.061602in}}{\pgfqpoint{1.695155in}{3.058330in}}{\pgfqpoint{1.703391in}{3.058330in}}%
\pgfpathclose%
\pgfusepath{stroke,fill}%
\end{pgfscope}%
\begin{pgfscope}%
\pgfpathrectangle{\pgfqpoint{0.100000in}{0.212622in}}{\pgfqpoint{3.696000in}{3.696000in}}%
\pgfusepath{clip}%
\pgfsetbuttcap%
\pgfsetroundjoin%
\definecolor{currentfill}{rgb}{0.121569,0.466667,0.705882}%
\pgfsetfillcolor{currentfill}%
\pgfsetfillopacity{0.356428}%
\pgfsetlinewidth{1.003750pt}%
\definecolor{currentstroke}{rgb}{0.121569,0.466667,0.705882}%
\pgfsetstrokecolor{currentstroke}%
\pgfsetstrokeopacity{0.356428}%
\pgfsetdash{}{0pt}%
\pgfpathmoveto{\pgfqpoint{1.702959in}{3.057675in}}%
\pgfpathcurveto{\pgfqpoint{1.711195in}{3.057675in}}{\pgfqpoint{1.719095in}{3.060947in}}{\pgfqpoint{1.724919in}{3.066771in}}%
\pgfpathcurveto{\pgfqpoint{1.730743in}{3.072595in}}{\pgfqpoint{1.734016in}{3.080495in}}{\pgfqpoint{1.734016in}{3.088732in}}%
\pgfpathcurveto{\pgfqpoint{1.734016in}{3.096968in}}{\pgfqpoint{1.730743in}{3.104868in}}{\pgfqpoint{1.724919in}{3.110692in}}%
\pgfpathcurveto{\pgfqpoint{1.719095in}{3.116516in}}{\pgfqpoint{1.711195in}{3.119788in}}{\pgfqpoint{1.702959in}{3.119788in}}%
\pgfpathcurveto{\pgfqpoint{1.694723in}{3.119788in}}{\pgfqpoint{1.686823in}{3.116516in}}{\pgfqpoint{1.680999in}{3.110692in}}%
\pgfpathcurveto{\pgfqpoint{1.675175in}{3.104868in}}{\pgfqpoint{1.671903in}{3.096968in}}{\pgfqpoint{1.671903in}{3.088732in}}%
\pgfpathcurveto{\pgfqpoint{1.671903in}{3.080495in}}{\pgfqpoint{1.675175in}{3.072595in}}{\pgfqpoint{1.680999in}{3.066771in}}%
\pgfpathcurveto{\pgfqpoint{1.686823in}{3.060947in}}{\pgfqpoint{1.694723in}{3.057675in}}{\pgfqpoint{1.702959in}{3.057675in}}%
\pgfpathclose%
\pgfusepath{stroke,fill}%
\end{pgfscope}%
\begin{pgfscope}%
\pgfpathrectangle{\pgfqpoint{0.100000in}{0.212622in}}{\pgfqpoint{3.696000in}{3.696000in}}%
\pgfusepath{clip}%
\pgfsetbuttcap%
\pgfsetroundjoin%
\definecolor{currentfill}{rgb}{0.121569,0.466667,0.705882}%
\pgfsetfillcolor{currentfill}%
\pgfsetfillopacity{0.356697}%
\pgfsetlinewidth{1.003750pt}%
\definecolor{currentstroke}{rgb}{0.121569,0.466667,0.705882}%
\pgfsetstrokecolor{currentstroke}%
\pgfsetstrokeopacity{0.356697}%
\pgfsetdash{}{0pt}%
\pgfpathmoveto{\pgfqpoint{1.702249in}{3.056455in}}%
\pgfpathcurveto{\pgfqpoint{1.710485in}{3.056455in}}{\pgfqpoint{1.718385in}{3.059727in}}{\pgfqpoint{1.724209in}{3.065551in}}%
\pgfpathcurveto{\pgfqpoint{1.730033in}{3.071375in}}{\pgfqpoint{1.733306in}{3.079275in}}{\pgfqpoint{1.733306in}{3.087512in}}%
\pgfpathcurveto{\pgfqpoint{1.733306in}{3.095748in}}{\pgfqpoint{1.730033in}{3.103648in}}{\pgfqpoint{1.724209in}{3.109472in}}%
\pgfpathcurveto{\pgfqpoint{1.718385in}{3.115296in}}{\pgfqpoint{1.710485in}{3.118568in}}{\pgfqpoint{1.702249in}{3.118568in}}%
\pgfpathcurveto{\pgfqpoint{1.694013in}{3.118568in}}{\pgfqpoint{1.686113in}{3.115296in}}{\pgfqpoint{1.680289in}{3.109472in}}%
\pgfpathcurveto{\pgfqpoint{1.674465in}{3.103648in}}{\pgfqpoint{1.671193in}{3.095748in}}{\pgfqpoint{1.671193in}{3.087512in}}%
\pgfpathcurveto{\pgfqpoint{1.671193in}{3.079275in}}{\pgfqpoint{1.674465in}{3.071375in}}{\pgfqpoint{1.680289in}{3.065551in}}%
\pgfpathcurveto{\pgfqpoint{1.686113in}{3.059727in}}{\pgfqpoint{1.694013in}{3.056455in}}{\pgfqpoint{1.702249in}{3.056455in}}%
\pgfpathclose%
\pgfusepath{stroke,fill}%
\end{pgfscope}%
\begin{pgfscope}%
\pgfpathrectangle{\pgfqpoint{0.100000in}{0.212622in}}{\pgfqpoint{3.696000in}{3.696000in}}%
\pgfusepath{clip}%
\pgfsetbuttcap%
\pgfsetroundjoin%
\definecolor{currentfill}{rgb}{0.121569,0.466667,0.705882}%
\pgfsetfillcolor{currentfill}%
\pgfsetfillopacity{0.357173}%
\pgfsetlinewidth{1.003750pt}%
\definecolor{currentstroke}{rgb}{0.121569,0.466667,0.705882}%
\pgfsetstrokecolor{currentstroke}%
\pgfsetstrokeopacity{0.357173}%
\pgfsetdash{}{0pt}%
\pgfpathmoveto{\pgfqpoint{1.700950in}{3.054196in}}%
\pgfpathcurveto{\pgfqpoint{1.709186in}{3.054196in}}{\pgfqpoint{1.717086in}{3.057469in}}{\pgfqpoint{1.722910in}{3.063293in}}%
\pgfpathcurveto{\pgfqpoint{1.728734in}{3.069116in}}{\pgfqpoint{1.732006in}{3.077017in}}{\pgfqpoint{1.732006in}{3.085253in}}%
\pgfpathcurveto{\pgfqpoint{1.732006in}{3.093489in}}{\pgfqpoint{1.728734in}{3.101389in}}{\pgfqpoint{1.722910in}{3.107213in}}%
\pgfpathcurveto{\pgfqpoint{1.717086in}{3.113037in}}{\pgfqpoint{1.709186in}{3.116309in}}{\pgfqpoint{1.700950in}{3.116309in}}%
\pgfpathcurveto{\pgfqpoint{1.692713in}{3.116309in}}{\pgfqpoint{1.684813in}{3.113037in}}{\pgfqpoint{1.678989in}{3.107213in}}%
\pgfpathcurveto{\pgfqpoint{1.673165in}{3.101389in}}{\pgfqpoint{1.669893in}{3.093489in}}{\pgfqpoint{1.669893in}{3.085253in}}%
\pgfpathcurveto{\pgfqpoint{1.669893in}{3.077017in}}{\pgfqpoint{1.673165in}{3.069116in}}{\pgfqpoint{1.678989in}{3.063293in}}%
\pgfpathcurveto{\pgfqpoint{1.684813in}{3.057469in}}{\pgfqpoint{1.692713in}{3.054196in}}{\pgfqpoint{1.700950in}{3.054196in}}%
\pgfpathclose%
\pgfusepath{stroke,fill}%
\end{pgfscope}%
\begin{pgfscope}%
\pgfpathrectangle{\pgfqpoint{0.100000in}{0.212622in}}{\pgfqpoint{3.696000in}{3.696000in}}%
\pgfusepath{clip}%
\pgfsetbuttcap%
\pgfsetroundjoin%
\definecolor{currentfill}{rgb}{0.121569,0.466667,0.705882}%
\pgfsetfillcolor{currentfill}%
\pgfsetfillopacity{0.357622}%
\pgfsetlinewidth{1.003750pt}%
\definecolor{currentstroke}{rgb}{0.121569,0.466667,0.705882}%
\pgfsetstrokecolor{currentstroke}%
\pgfsetstrokeopacity{0.357622}%
\pgfsetdash{}{0pt}%
\pgfpathmoveto{\pgfqpoint{1.950360in}{3.080402in}}%
\pgfpathcurveto{\pgfqpoint{1.958596in}{3.080402in}}{\pgfqpoint{1.966497in}{3.083675in}}{\pgfqpoint{1.972320in}{3.089498in}}%
\pgfpathcurveto{\pgfqpoint{1.978144in}{3.095322in}}{\pgfqpoint{1.981417in}{3.103222in}}{\pgfqpoint{1.981417in}{3.111459in}}%
\pgfpathcurveto{\pgfqpoint{1.981417in}{3.119695in}}{\pgfqpoint{1.978144in}{3.127595in}}{\pgfqpoint{1.972320in}{3.133419in}}%
\pgfpathcurveto{\pgfqpoint{1.966497in}{3.139243in}}{\pgfqpoint{1.958596in}{3.142515in}}{\pgfqpoint{1.950360in}{3.142515in}}%
\pgfpathcurveto{\pgfqpoint{1.942124in}{3.142515in}}{\pgfqpoint{1.934224in}{3.139243in}}{\pgfqpoint{1.928400in}{3.133419in}}%
\pgfpathcurveto{\pgfqpoint{1.922576in}{3.127595in}}{\pgfqpoint{1.919304in}{3.119695in}}{\pgfqpoint{1.919304in}{3.111459in}}%
\pgfpathcurveto{\pgfqpoint{1.919304in}{3.103222in}}{\pgfqpoint{1.922576in}{3.095322in}}{\pgfqpoint{1.928400in}{3.089498in}}%
\pgfpathcurveto{\pgfqpoint{1.934224in}{3.083675in}}{\pgfqpoint{1.942124in}{3.080402in}}{\pgfqpoint{1.950360in}{3.080402in}}%
\pgfpathclose%
\pgfusepath{stroke,fill}%
\end{pgfscope}%
\begin{pgfscope}%
\pgfpathrectangle{\pgfqpoint{0.100000in}{0.212622in}}{\pgfqpoint{3.696000in}{3.696000in}}%
\pgfusepath{clip}%
\pgfsetbuttcap%
\pgfsetroundjoin%
\definecolor{currentfill}{rgb}{0.121569,0.466667,0.705882}%
\pgfsetfillcolor{currentfill}%
\pgfsetfillopacity{0.357925}%
\pgfsetlinewidth{1.003750pt}%
\definecolor{currentstroke}{rgb}{0.121569,0.466667,0.705882}%
\pgfsetstrokecolor{currentstroke}%
\pgfsetstrokeopacity{0.357925}%
\pgfsetdash{}{0pt}%
\pgfpathmoveto{\pgfqpoint{1.698374in}{3.049943in}}%
\pgfpathcurveto{\pgfqpoint{1.706610in}{3.049943in}}{\pgfqpoint{1.714511in}{3.053215in}}{\pgfqpoint{1.720334in}{3.059039in}}%
\pgfpathcurveto{\pgfqpoint{1.726158in}{3.064863in}}{\pgfqpoint{1.729431in}{3.072763in}}{\pgfqpoint{1.729431in}{3.081000in}}%
\pgfpathcurveto{\pgfqpoint{1.729431in}{3.089236in}}{\pgfqpoint{1.726158in}{3.097136in}}{\pgfqpoint{1.720334in}{3.102960in}}%
\pgfpathcurveto{\pgfqpoint{1.714511in}{3.108784in}}{\pgfqpoint{1.706610in}{3.112056in}}{\pgfqpoint{1.698374in}{3.112056in}}%
\pgfpathcurveto{\pgfqpoint{1.690138in}{3.112056in}}{\pgfqpoint{1.682238in}{3.108784in}}{\pgfqpoint{1.676414in}{3.102960in}}%
\pgfpathcurveto{\pgfqpoint{1.670590in}{3.097136in}}{\pgfqpoint{1.667318in}{3.089236in}}{\pgfqpoint{1.667318in}{3.081000in}}%
\pgfpathcurveto{\pgfqpoint{1.667318in}{3.072763in}}{\pgfqpoint{1.670590in}{3.064863in}}{\pgfqpoint{1.676414in}{3.059039in}}%
\pgfpathcurveto{\pgfqpoint{1.682238in}{3.053215in}}{\pgfqpoint{1.690138in}{3.049943in}}{\pgfqpoint{1.698374in}{3.049943in}}%
\pgfpathclose%
\pgfusepath{stroke,fill}%
\end{pgfscope}%
\begin{pgfscope}%
\pgfpathrectangle{\pgfqpoint{0.100000in}{0.212622in}}{\pgfqpoint{3.696000in}{3.696000in}}%
\pgfusepath{clip}%
\pgfsetbuttcap%
\pgfsetroundjoin%
\definecolor{currentfill}{rgb}{0.121569,0.466667,0.705882}%
\pgfsetfillcolor{currentfill}%
\pgfsetfillopacity{0.358306}%
\pgfsetlinewidth{1.003750pt}%
\definecolor{currentstroke}{rgb}{0.121569,0.466667,0.705882}%
\pgfsetstrokecolor{currentstroke}%
\pgfsetstrokeopacity{0.358306}%
\pgfsetdash{}{0pt}%
\pgfpathmoveto{\pgfqpoint{1.697347in}{3.047792in}}%
\pgfpathcurveto{\pgfqpoint{1.705584in}{3.047792in}}{\pgfqpoint{1.713484in}{3.051065in}}{\pgfqpoint{1.719308in}{3.056889in}}%
\pgfpathcurveto{\pgfqpoint{1.725132in}{3.062713in}}{\pgfqpoint{1.728404in}{3.070613in}}{\pgfqpoint{1.728404in}{3.078849in}}%
\pgfpathcurveto{\pgfqpoint{1.728404in}{3.087085in}}{\pgfqpoint{1.725132in}{3.094985in}}{\pgfqpoint{1.719308in}{3.100809in}}%
\pgfpathcurveto{\pgfqpoint{1.713484in}{3.106633in}}{\pgfqpoint{1.705584in}{3.109905in}}{\pgfqpoint{1.697347in}{3.109905in}}%
\pgfpathcurveto{\pgfqpoint{1.689111in}{3.109905in}}{\pgfqpoint{1.681211in}{3.106633in}}{\pgfqpoint{1.675387in}{3.100809in}}%
\pgfpathcurveto{\pgfqpoint{1.669563in}{3.094985in}}{\pgfqpoint{1.666291in}{3.087085in}}{\pgfqpoint{1.666291in}{3.078849in}}%
\pgfpathcurveto{\pgfqpoint{1.666291in}{3.070613in}}{\pgfqpoint{1.669563in}{3.062713in}}{\pgfqpoint{1.675387in}{3.056889in}}%
\pgfpathcurveto{\pgfqpoint{1.681211in}{3.051065in}}{\pgfqpoint{1.689111in}{3.047792in}}{\pgfqpoint{1.697347in}{3.047792in}}%
\pgfpathclose%
\pgfusepath{stroke,fill}%
\end{pgfscope}%
\begin{pgfscope}%
\pgfpathrectangle{\pgfqpoint{0.100000in}{0.212622in}}{\pgfqpoint{3.696000in}{3.696000in}}%
\pgfusepath{clip}%
\pgfsetbuttcap%
\pgfsetroundjoin%
\definecolor{currentfill}{rgb}{0.121569,0.466667,0.705882}%
\pgfsetfillcolor{currentfill}%
\pgfsetfillopacity{0.358688}%
\pgfsetlinewidth{1.003750pt}%
\definecolor{currentstroke}{rgb}{0.121569,0.466667,0.705882}%
\pgfsetstrokecolor{currentstroke}%
\pgfsetstrokeopacity{0.358688}%
\pgfsetdash{}{0pt}%
\pgfpathmoveto{\pgfqpoint{1.951007in}{3.076527in}}%
\pgfpathcurveto{\pgfqpoint{1.959243in}{3.076527in}}{\pgfqpoint{1.967143in}{3.079799in}}{\pgfqpoint{1.972967in}{3.085623in}}%
\pgfpathcurveto{\pgfqpoint{1.978791in}{3.091447in}}{\pgfqpoint{1.982063in}{3.099347in}}{\pgfqpoint{1.982063in}{3.107583in}}%
\pgfpathcurveto{\pgfqpoint{1.982063in}{3.115820in}}{\pgfqpoint{1.978791in}{3.123720in}}{\pgfqpoint{1.972967in}{3.129544in}}%
\pgfpathcurveto{\pgfqpoint{1.967143in}{3.135367in}}{\pgfqpoint{1.959243in}{3.138640in}}{\pgfqpoint{1.951007in}{3.138640in}}%
\pgfpathcurveto{\pgfqpoint{1.942771in}{3.138640in}}{\pgfqpoint{1.934871in}{3.135367in}}{\pgfqpoint{1.929047in}{3.129544in}}%
\pgfpathcurveto{\pgfqpoint{1.923223in}{3.123720in}}{\pgfqpoint{1.919950in}{3.115820in}}{\pgfqpoint{1.919950in}{3.107583in}}%
\pgfpathcurveto{\pgfqpoint{1.919950in}{3.099347in}}{\pgfqpoint{1.923223in}{3.091447in}}{\pgfqpoint{1.929047in}{3.085623in}}%
\pgfpathcurveto{\pgfqpoint{1.934871in}{3.079799in}}{\pgfqpoint{1.942771in}{3.076527in}}{\pgfqpoint{1.951007in}{3.076527in}}%
\pgfpathclose%
\pgfusepath{stroke,fill}%
\end{pgfscope}%
\begin{pgfscope}%
\pgfpathrectangle{\pgfqpoint{0.100000in}{0.212622in}}{\pgfqpoint{3.696000in}{3.696000in}}%
\pgfusepath{clip}%
\pgfsetbuttcap%
\pgfsetroundjoin%
\definecolor{currentfill}{rgb}{0.121569,0.466667,0.705882}%
\pgfsetfillcolor{currentfill}%
\pgfsetfillopacity{0.359011}%
\pgfsetlinewidth{1.003750pt}%
\definecolor{currentstroke}{rgb}{0.121569,0.466667,0.705882}%
\pgfsetstrokecolor{currentstroke}%
\pgfsetstrokeopacity{0.359011}%
\pgfsetdash{}{0pt}%
\pgfpathmoveto{\pgfqpoint{1.695212in}{3.044265in}}%
\pgfpathcurveto{\pgfqpoint{1.703449in}{3.044265in}}{\pgfqpoint{1.711349in}{3.047538in}}{\pgfqpoint{1.717173in}{3.053361in}}%
\pgfpathcurveto{\pgfqpoint{1.722996in}{3.059185in}}{\pgfqpoint{1.726269in}{3.067085in}}{\pgfqpoint{1.726269in}{3.075322in}}%
\pgfpathcurveto{\pgfqpoint{1.726269in}{3.083558in}}{\pgfqpoint{1.722996in}{3.091458in}}{\pgfqpoint{1.717173in}{3.097282in}}%
\pgfpathcurveto{\pgfqpoint{1.711349in}{3.103106in}}{\pgfqpoint{1.703449in}{3.106378in}}{\pgfqpoint{1.695212in}{3.106378in}}%
\pgfpathcurveto{\pgfqpoint{1.686976in}{3.106378in}}{\pgfqpoint{1.679076in}{3.103106in}}{\pgfqpoint{1.673252in}{3.097282in}}%
\pgfpathcurveto{\pgfqpoint{1.667428in}{3.091458in}}{\pgfqpoint{1.664156in}{3.083558in}}{\pgfqpoint{1.664156in}{3.075322in}}%
\pgfpathcurveto{\pgfqpoint{1.664156in}{3.067085in}}{\pgfqpoint{1.667428in}{3.059185in}}{\pgfqpoint{1.673252in}{3.053361in}}%
\pgfpathcurveto{\pgfqpoint{1.679076in}{3.047538in}}{\pgfqpoint{1.686976in}{3.044265in}}{\pgfqpoint{1.695212in}{3.044265in}}%
\pgfpathclose%
\pgfusepath{stroke,fill}%
\end{pgfscope}%
\begin{pgfscope}%
\pgfpathrectangle{\pgfqpoint{0.100000in}{0.212622in}}{\pgfqpoint{3.696000in}{3.696000in}}%
\pgfusepath{clip}%
\pgfsetbuttcap%
\pgfsetroundjoin%
\definecolor{currentfill}{rgb}{0.121569,0.466667,0.705882}%
\pgfsetfillcolor{currentfill}%
\pgfsetfillopacity{0.359725}%
\pgfsetlinewidth{1.003750pt}%
\definecolor{currentstroke}{rgb}{0.121569,0.466667,0.705882}%
\pgfsetstrokecolor{currentstroke}%
\pgfsetstrokeopacity{0.359725}%
\pgfsetdash{}{0pt}%
\pgfpathmoveto{\pgfqpoint{1.951952in}{3.071825in}}%
\pgfpathcurveto{\pgfqpoint{1.960188in}{3.071825in}}{\pgfqpoint{1.968088in}{3.075097in}}{\pgfqpoint{1.973912in}{3.080921in}}%
\pgfpathcurveto{\pgfqpoint{1.979736in}{3.086745in}}{\pgfqpoint{1.983008in}{3.094645in}}{\pgfqpoint{1.983008in}{3.102881in}}%
\pgfpathcurveto{\pgfqpoint{1.983008in}{3.111117in}}{\pgfqpoint{1.979736in}{3.119017in}}{\pgfqpoint{1.973912in}{3.124841in}}%
\pgfpathcurveto{\pgfqpoint{1.968088in}{3.130665in}}{\pgfqpoint{1.960188in}{3.133938in}}{\pgfqpoint{1.951952in}{3.133938in}}%
\pgfpathcurveto{\pgfqpoint{1.943715in}{3.133938in}}{\pgfqpoint{1.935815in}{3.130665in}}{\pgfqpoint{1.929991in}{3.124841in}}%
\pgfpathcurveto{\pgfqpoint{1.924167in}{3.119017in}}{\pgfqpoint{1.920895in}{3.111117in}}{\pgfqpoint{1.920895in}{3.102881in}}%
\pgfpathcurveto{\pgfqpoint{1.920895in}{3.094645in}}{\pgfqpoint{1.924167in}{3.086745in}}{\pgfqpoint{1.929991in}{3.080921in}}%
\pgfpathcurveto{\pgfqpoint{1.935815in}{3.075097in}}{\pgfqpoint{1.943715in}{3.071825in}}{\pgfqpoint{1.951952in}{3.071825in}}%
\pgfpathclose%
\pgfusepath{stroke,fill}%
\end{pgfscope}%
\begin{pgfscope}%
\pgfpathrectangle{\pgfqpoint{0.100000in}{0.212622in}}{\pgfqpoint{3.696000in}{3.696000in}}%
\pgfusepath{clip}%
\pgfsetbuttcap%
\pgfsetroundjoin%
\definecolor{currentfill}{rgb}{0.121569,0.466667,0.705882}%
\pgfsetfillcolor{currentfill}%
\pgfsetfillopacity{0.360221}%
\pgfsetlinewidth{1.003750pt}%
\definecolor{currentstroke}{rgb}{0.121569,0.466667,0.705882}%
\pgfsetstrokecolor{currentstroke}%
\pgfsetstrokeopacity{0.360221}%
\pgfsetdash{}{0pt}%
\pgfpathmoveto{\pgfqpoint{1.691232in}{3.037713in}}%
\pgfpathcurveto{\pgfqpoint{1.699469in}{3.037713in}}{\pgfqpoint{1.707369in}{3.040985in}}{\pgfqpoint{1.713193in}{3.046809in}}%
\pgfpathcurveto{\pgfqpoint{1.719017in}{3.052633in}}{\pgfqpoint{1.722289in}{3.060533in}}{\pgfqpoint{1.722289in}{3.068769in}}%
\pgfpathcurveto{\pgfqpoint{1.722289in}{3.077006in}}{\pgfqpoint{1.719017in}{3.084906in}}{\pgfqpoint{1.713193in}{3.090730in}}%
\pgfpathcurveto{\pgfqpoint{1.707369in}{3.096554in}}{\pgfqpoint{1.699469in}{3.099826in}}{\pgfqpoint{1.691232in}{3.099826in}}%
\pgfpathcurveto{\pgfqpoint{1.682996in}{3.099826in}}{\pgfqpoint{1.675096in}{3.096554in}}{\pgfqpoint{1.669272in}{3.090730in}}%
\pgfpathcurveto{\pgfqpoint{1.663448in}{3.084906in}}{\pgfqpoint{1.660176in}{3.077006in}}{\pgfqpoint{1.660176in}{3.068769in}}%
\pgfpathcurveto{\pgfqpoint{1.660176in}{3.060533in}}{\pgfqpoint{1.663448in}{3.052633in}}{\pgfqpoint{1.669272in}{3.046809in}}%
\pgfpathcurveto{\pgfqpoint{1.675096in}{3.040985in}}{\pgfqpoint{1.682996in}{3.037713in}}{\pgfqpoint{1.691232in}{3.037713in}}%
\pgfpathclose%
\pgfusepath{stroke,fill}%
\end{pgfscope}%
\begin{pgfscope}%
\pgfpathrectangle{\pgfqpoint{0.100000in}{0.212622in}}{\pgfqpoint{3.696000in}{3.696000in}}%
\pgfusepath{clip}%
\pgfsetbuttcap%
\pgfsetroundjoin%
\definecolor{currentfill}{rgb}{0.121569,0.466667,0.705882}%
\pgfsetfillcolor{currentfill}%
\pgfsetfillopacity{0.361275}%
\pgfsetlinewidth{1.003750pt}%
\definecolor{currentstroke}{rgb}{0.121569,0.466667,0.705882}%
\pgfsetstrokecolor{currentstroke}%
\pgfsetstrokeopacity{0.361275}%
\pgfsetdash{}{0pt}%
\pgfpathmoveto{\pgfqpoint{1.952655in}{3.065991in}}%
\pgfpathcurveto{\pgfqpoint{1.960891in}{3.065991in}}{\pgfqpoint{1.968791in}{3.069264in}}{\pgfqpoint{1.974615in}{3.075088in}}%
\pgfpathcurveto{\pgfqpoint{1.980439in}{3.080911in}}{\pgfqpoint{1.983712in}{3.088812in}}{\pgfqpoint{1.983712in}{3.097048in}}%
\pgfpathcurveto{\pgfqpoint{1.983712in}{3.105284in}}{\pgfqpoint{1.980439in}{3.113184in}}{\pgfqpoint{1.974615in}{3.119008in}}%
\pgfpathcurveto{\pgfqpoint{1.968791in}{3.124832in}}{\pgfqpoint{1.960891in}{3.128104in}}{\pgfqpoint{1.952655in}{3.128104in}}%
\pgfpathcurveto{\pgfqpoint{1.944419in}{3.128104in}}{\pgfqpoint{1.936519in}{3.124832in}}{\pgfqpoint{1.930695in}{3.119008in}}%
\pgfpathcurveto{\pgfqpoint{1.924871in}{3.113184in}}{\pgfqpoint{1.921599in}{3.105284in}}{\pgfqpoint{1.921599in}{3.097048in}}%
\pgfpathcurveto{\pgfqpoint{1.921599in}{3.088812in}}{\pgfqpoint{1.924871in}{3.080911in}}{\pgfqpoint{1.930695in}{3.075088in}}%
\pgfpathcurveto{\pgfqpoint{1.936519in}{3.069264in}}{\pgfqpoint{1.944419in}{3.065991in}}{\pgfqpoint{1.952655in}{3.065991in}}%
\pgfpathclose%
\pgfusepath{stroke,fill}%
\end{pgfscope}%
\begin{pgfscope}%
\pgfpathrectangle{\pgfqpoint{0.100000in}{0.212622in}}{\pgfqpoint{3.696000in}{3.696000in}}%
\pgfusepath{clip}%
\pgfsetbuttcap%
\pgfsetroundjoin%
\definecolor{currentfill}{rgb}{0.121569,0.466667,0.705882}%
\pgfsetfillcolor{currentfill}%
\pgfsetfillopacity{0.361356}%
\pgfsetlinewidth{1.003750pt}%
\definecolor{currentstroke}{rgb}{0.121569,0.466667,0.705882}%
\pgfsetstrokecolor{currentstroke}%
\pgfsetstrokeopacity{0.361356}%
\pgfsetdash{}{0pt}%
\pgfpathmoveto{\pgfqpoint{1.688235in}{3.031233in}}%
\pgfpathcurveto{\pgfqpoint{1.696472in}{3.031233in}}{\pgfqpoint{1.704372in}{3.034505in}}{\pgfqpoint{1.710196in}{3.040329in}}%
\pgfpathcurveto{\pgfqpoint{1.716020in}{3.046153in}}{\pgfqpoint{1.719292in}{3.054053in}}{\pgfqpoint{1.719292in}{3.062289in}}%
\pgfpathcurveto{\pgfqpoint{1.719292in}{3.070526in}}{\pgfqpoint{1.716020in}{3.078426in}}{\pgfqpoint{1.710196in}{3.084250in}}%
\pgfpathcurveto{\pgfqpoint{1.704372in}{3.090074in}}{\pgfqpoint{1.696472in}{3.093346in}}{\pgfqpoint{1.688235in}{3.093346in}}%
\pgfpathcurveto{\pgfqpoint{1.679999in}{3.093346in}}{\pgfqpoint{1.672099in}{3.090074in}}{\pgfqpoint{1.666275in}{3.084250in}}%
\pgfpathcurveto{\pgfqpoint{1.660451in}{3.078426in}}{\pgfqpoint{1.657179in}{3.070526in}}{\pgfqpoint{1.657179in}{3.062289in}}%
\pgfpathcurveto{\pgfqpoint{1.657179in}{3.054053in}}{\pgfqpoint{1.660451in}{3.046153in}}{\pgfqpoint{1.666275in}{3.040329in}}%
\pgfpathcurveto{\pgfqpoint{1.672099in}{3.034505in}}{\pgfqpoint{1.679999in}{3.031233in}}{\pgfqpoint{1.688235in}{3.031233in}}%
\pgfpathclose%
\pgfusepath{stroke,fill}%
\end{pgfscope}%
\begin{pgfscope}%
\pgfpathrectangle{\pgfqpoint{0.100000in}{0.212622in}}{\pgfqpoint{3.696000in}{3.696000in}}%
\pgfusepath{clip}%
\pgfsetbuttcap%
\pgfsetroundjoin%
\definecolor{currentfill}{rgb}{0.121569,0.466667,0.705882}%
\pgfsetfillcolor{currentfill}%
\pgfsetfillopacity{0.362019}%
\pgfsetlinewidth{1.003750pt}%
\definecolor{currentstroke}{rgb}{0.121569,0.466667,0.705882}%
\pgfsetstrokecolor{currentstroke}%
\pgfsetstrokeopacity{0.362019}%
\pgfsetdash{}{0pt}%
\pgfpathmoveto{\pgfqpoint{1.686046in}{3.027668in}}%
\pgfpathcurveto{\pgfqpoint{1.694282in}{3.027668in}}{\pgfqpoint{1.702182in}{3.030940in}}{\pgfqpoint{1.708006in}{3.036764in}}%
\pgfpathcurveto{\pgfqpoint{1.713830in}{3.042588in}}{\pgfqpoint{1.717102in}{3.050488in}}{\pgfqpoint{1.717102in}{3.058724in}}%
\pgfpathcurveto{\pgfqpoint{1.717102in}{3.066960in}}{\pgfqpoint{1.713830in}{3.074861in}}{\pgfqpoint{1.708006in}{3.080684in}}%
\pgfpathcurveto{\pgfqpoint{1.702182in}{3.086508in}}{\pgfqpoint{1.694282in}{3.089781in}}{\pgfqpoint{1.686046in}{3.089781in}}%
\pgfpathcurveto{\pgfqpoint{1.677809in}{3.089781in}}{\pgfqpoint{1.669909in}{3.086508in}}{\pgfqpoint{1.664086in}{3.080684in}}%
\pgfpathcurveto{\pgfqpoint{1.658262in}{3.074861in}}{\pgfqpoint{1.654989in}{3.066960in}}{\pgfqpoint{1.654989in}{3.058724in}}%
\pgfpathcurveto{\pgfqpoint{1.654989in}{3.050488in}}{\pgfqpoint{1.658262in}{3.042588in}}{\pgfqpoint{1.664086in}{3.036764in}}%
\pgfpathcurveto{\pgfqpoint{1.669909in}{3.030940in}}{\pgfqpoint{1.677809in}{3.027668in}}{\pgfqpoint{1.686046in}{3.027668in}}%
\pgfpathclose%
\pgfusepath{stroke,fill}%
\end{pgfscope}%
\begin{pgfscope}%
\pgfpathrectangle{\pgfqpoint{0.100000in}{0.212622in}}{\pgfqpoint{3.696000in}{3.696000in}}%
\pgfusepath{clip}%
\pgfsetbuttcap%
\pgfsetroundjoin%
\definecolor{currentfill}{rgb}{0.121569,0.466667,0.705882}%
\pgfsetfillcolor{currentfill}%
\pgfsetfillopacity{0.362652}%
\pgfsetlinewidth{1.003750pt}%
\definecolor{currentstroke}{rgb}{0.121569,0.466667,0.705882}%
\pgfsetstrokecolor{currentstroke}%
\pgfsetstrokeopacity{0.362652}%
\pgfsetdash{}{0pt}%
\pgfpathmoveto{\pgfqpoint{1.684165in}{3.024303in}}%
\pgfpathcurveto{\pgfqpoint{1.692401in}{3.024303in}}{\pgfqpoint{1.700301in}{3.027575in}}{\pgfqpoint{1.706125in}{3.033399in}}%
\pgfpathcurveto{\pgfqpoint{1.711949in}{3.039223in}}{\pgfqpoint{1.715221in}{3.047123in}}{\pgfqpoint{1.715221in}{3.055359in}}%
\pgfpathcurveto{\pgfqpoint{1.715221in}{3.063595in}}{\pgfqpoint{1.711949in}{3.071495in}}{\pgfqpoint{1.706125in}{3.077319in}}%
\pgfpathcurveto{\pgfqpoint{1.700301in}{3.083143in}}{\pgfqpoint{1.692401in}{3.086416in}}{\pgfqpoint{1.684165in}{3.086416in}}%
\pgfpathcurveto{\pgfqpoint{1.675928in}{3.086416in}}{\pgfqpoint{1.668028in}{3.083143in}}{\pgfqpoint{1.662204in}{3.077319in}}%
\pgfpathcurveto{\pgfqpoint{1.656380in}{3.071495in}}{\pgfqpoint{1.653108in}{3.063595in}}{\pgfqpoint{1.653108in}{3.055359in}}%
\pgfpathcurveto{\pgfqpoint{1.653108in}{3.047123in}}{\pgfqpoint{1.656380in}{3.039223in}}{\pgfqpoint{1.662204in}{3.033399in}}%
\pgfpathcurveto{\pgfqpoint{1.668028in}{3.027575in}}{\pgfqpoint{1.675928in}{3.024303in}}{\pgfqpoint{1.684165in}{3.024303in}}%
\pgfpathclose%
\pgfusepath{stroke,fill}%
\end{pgfscope}%
\begin{pgfscope}%
\pgfpathrectangle{\pgfqpoint{0.100000in}{0.212622in}}{\pgfqpoint{3.696000in}{3.696000in}}%
\pgfusepath{clip}%
\pgfsetbuttcap%
\pgfsetroundjoin%
\definecolor{currentfill}{rgb}{0.121569,0.466667,0.705882}%
\pgfsetfillcolor{currentfill}%
\pgfsetfillopacity{0.362974}%
\pgfsetlinewidth{1.003750pt}%
\definecolor{currentstroke}{rgb}{0.121569,0.466667,0.705882}%
\pgfsetstrokecolor{currentstroke}%
\pgfsetstrokeopacity{0.362974}%
\pgfsetdash{}{0pt}%
\pgfpathmoveto{\pgfqpoint{1.953655in}{3.059652in}}%
\pgfpathcurveto{\pgfqpoint{1.961891in}{3.059652in}}{\pgfqpoint{1.969791in}{3.062924in}}{\pgfqpoint{1.975615in}{3.068748in}}%
\pgfpathcurveto{\pgfqpoint{1.981439in}{3.074572in}}{\pgfqpoint{1.984711in}{3.082472in}}{\pgfqpoint{1.984711in}{3.090709in}}%
\pgfpathcurveto{\pgfqpoint{1.984711in}{3.098945in}}{\pgfqpoint{1.981439in}{3.106845in}}{\pgfqpoint{1.975615in}{3.112669in}}%
\pgfpathcurveto{\pgfqpoint{1.969791in}{3.118493in}}{\pgfqpoint{1.961891in}{3.121765in}}{\pgfqpoint{1.953655in}{3.121765in}}%
\pgfpathcurveto{\pgfqpoint{1.945419in}{3.121765in}}{\pgfqpoint{1.937519in}{3.118493in}}{\pgfqpoint{1.931695in}{3.112669in}}%
\pgfpathcurveto{\pgfqpoint{1.925871in}{3.106845in}}{\pgfqpoint{1.922599in}{3.098945in}}{\pgfqpoint{1.922599in}{3.090709in}}%
\pgfpathcurveto{\pgfqpoint{1.922599in}{3.082472in}}{\pgfqpoint{1.925871in}{3.074572in}}{\pgfqpoint{1.931695in}{3.068748in}}%
\pgfpathcurveto{\pgfqpoint{1.937519in}{3.062924in}}{\pgfqpoint{1.945419in}{3.059652in}}{\pgfqpoint{1.953655in}{3.059652in}}%
\pgfpathclose%
\pgfusepath{stroke,fill}%
\end{pgfscope}%
\begin{pgfscope}%
\pgfpathrectangle{\pgfqpoint{0.100000in}{0.212622in}}{\pgfqpoint{3.696000in}{3.696000in}}%
\pgfusepath{clip}%
\pgfsetbuttcap%
\pgfsetroundjoin%
\definecolor{currentfill}{rgb}{0.121569,0.466667,0.705882}%
\pgfsetfillcolor{currentfill}%
\pgfsetfillopacity{0.363826}%
\pgfsetlinewidth{1.003750pt}%
\definecolor{currentstroke}{rgb}{0.121569,0.466667,0.705882}%
\pgfsetstrokecolor{currentstroke}%
\pgfsetstrokeopacity{0.363826}%
\pgfsetdash{}{0pt}%
\pgfpathmoveto{\pgfqpoint{1.680962in}{3.018000in}}%
\pgfpathcurveto{\pgfqpoint{1.689198in}{3.018000in}}{\pgfqpoint{1.697098in}{3.021272in}}{\pgfqpoint{1.702922in}{3.027096in}}%
\pgfpathcurveto{\pgfqpoint{1.708746in}{3.032920in}}{\pgfqpoint{1.712018in}{3.040820in}}{\pgfqpoint{1.712018in}{3.049056in}}%
\pgfpathcurveto{\pgfqpoint{1.712018in}{3.057292in}}{\pgfqpoint{1.708746in}{3.065193in}}{\pgfqpoint{1.702922in}{3.071016in}}%
\pgfpathcurveto{\pgfqpoint{1.697098in}{3.076840in}}{\pgfqpoint{1.689198in}{3.080113in}}{\pgfqpoint{1.680962in}{3.080113in}}%
\pgfpathcurveto{\pgfqpoint{1.672725in}{3.080113in}}{\pgfqpoint{1.664825in}{3.076840in}}{\pgfqpoint{1.659001in}{3.071016in}}%
\pgfpathcurveto{\pgfqpoint{1.653177in}{3.065193in}}{\pgfqpoint{1.649905in}{3.057292in}}{\pgfqpoint{1.649905in}{3.049056in}}%
\pgfpathcurveto{\pgfqpoint{1.649905in}{3.040820in}}{\pgfqpoint{1.653177in}{3.032920in}}{\pgfqpoint{1.659001in}{3.027096in}}%
\pgfpathcurveto{\pgfqpoint{1.664825in}{3.021272in}}{\pgfqpoint{1.672725in}{3.018000in}}{\pgfqpoint{1.680962in}{3.018000in}}%
\pgfpathclose%
\pgfusepath{stroke,fill}%
\end{pgfscope}%
\begin{pgfscope}%
\pgfpathrectangle{\pgfqpoint{0.100000in}{0.212622in}}{\pgfqpoint{3.696000in}{3.696000in}}%
\pgfusepath{clip}%
\pgfsetbuttcap%
\pgfsetroundjoin%
\definecolor{currentfill}{rgb}{0.121569,0.466667,0.705882}%
\pgfsetfillcolor{currentfill}%
\pgfsetfillopacity{0.364605}%
\pgfsetlinewidth{1.003750pt}%
\definecolor{currentstroke}{rgb}{0.121569,0.466667,0.705882}%
\pgfsetstrokecolor{currentstroke}%
\pgfsetstrokeopacity{0.364605}%
\pgfsetdash{}{0pt}%
\pgfpathmoveto{\pgfqpoint{1.955060in}{3.052315in}}%
\pgfpathcurveto{\pgfqpoint{1.963296in}{3.052315in}}{\pgfqpoint{1.971196in}{3.055588in}}{\pgfqpoint{1.977020in}{3.061412in}}%
\pgfpathcurveto{\pgfqpoint{1.982844in}{3.067236in}}{\pgfqpoint{1.986116in}{3.075136in}}{\pgfqpoint{1.986116in}{3.083372in}}%
\pgfpathcurveto{\pgfqpoint{1.986116in}{3.091608in}}{\pgfqpoint{1.982844in}{3.099508in}}{\pgfqpoint{1.977020in}{3.105332in}}%
\pgfpathcurveto{\pgfqpoint{1.971196in}{3.111156in}}{\pgfqpoint{1.963296in}{3.114428in}}{\pgfqpoint{1.955060in}{3.114428in}}%
\pgfpathcurveto{\pgfqpoint{1.946823in}{3.114428in}}{\pgfqpoint{1.938923in}{3.111156in}}{\pgfqpoint{1.933099in}{3.105332in}}%
\pgfpathcurveto{\pgfqpoint{1.927275in}{3.099508in}}{\pgfqpoint{1.924003in}{3.091608in}}{\pgfqpoint{1.924003in}{3.083372in}}%
\pgfpathcurveto{\pgfqpoint{1.924003in}{3.075136in}}{\pgfqpoint{1.927275in}{3.067236in}}{\pgfqpoint{1.933099in}{3.061412in}}%
\pgfpathcurveto{\pgfqpoint{1.938923in}{3.055588in}}{\pgfqpoint{1.946823in}{3.052315in}}{\pgfqpoint{1.955060in}{3.052315in}}%
\pgfpathclose%
\pgfusepath{stroke,fill}%
\end{pgfscope}%
\begin{pgfscope}%
\pgfpathrectangle{\pgfqpoint{0.100000in}{0.212622in}}{\pgfqpoint{3.696000in}{3.696000in}}%
\pgfusepath{clip}%
\pgfsetbuttcap%
\pgfsetroundjoin%
\definecolor{currentfill}{rgb}{0.121569,0.466667,0.705882}%
\pgfsetfillcolor{currentfill}%
\pgfsetfillopacity{0.364778}%
\pgfsetlinewidth{1.003750pt}%
\definecolor{currentstroke}{rgb}{0.121569,0.466667,0.705882}%
\pgfsetstrokecolor{currentstroke}%
\pgfsetstrokeopacity{0.364778}%
\pgfsetdash{}{0pt}%
\pgfpathmoveto{\pgfqpoint{1.677685in}{3.012600in}}%
\pgfpathcurveto{\pgfqpoint{1.685921in}{3.012600in}}{\pgfqpoint{1.693821in}{3.015873in}}{\pgfqpoint{1.699645in}{3.021697in}}%
\pgfpathcurveto{\pgfqpoint{1.705469in}{3.027521in}}{\pgfqpoint{1.708741in}{3.035421in}}{\pgfqpoint{1.708741in}{3.043657in}}%
\pgfpathcurveto{\pgfqpoint{1.708741in}{3.051893in}}{\pgfqpoint{1.705469in}{3.059793in}}{\pgfqpoint{1.699645in}{3.065617in}}%
\pgfpathcurveto{\pgfqpoint{1.693821in}{3.071441in}}{\pgfqpoint{1.685921in}{3.074713in}}{\pgfqpoint{1.677685in}{3.074713in}}%
\pgfpathcurveto{\pgfqpoint{1.669449in}{3.074713in}}{\pgfqpoint{1.661549in}{3.071441in}}{\pgfqpoint{1.655725in}{3.065617in}}%
\pgfpathcurveto{\pgfqpoint{1.649901in}{3.059793in}}{\pgfqpoint{1.646628in}{3.051893in}}{\pgfqpoint{1.646628in}{3.043657in}}%
\pgfpathcurveto{\pgfqpoint{1.646628in}{3.035421in}}{\pgfqpoint{1.649901in}{3.027521in}}{\pgfqpoint{1.655725in}{3.021697in}}%
\pgfpathcurveto{\pgfqpoint{1.661549in}{3.015873in}}{\pgfqpoint{1.669449in}{3.012600in}}{\pgfqpoint{1.677685in}{3.012600in}}%
\pgfpathclose%
\pgfusepath{stroke,fill}%
\end{pgfscope}%
\begin{pgfscope}%
\pgfpathrectangle{\pgfqpoint{0.100000in}{0.212622in}}{\pgfqpoint{3.696000in}{3.696000in}}%
\pgfusepath{clip}%
\pgfsetbuttcap%
\pgfsetroundjoin%
\definecolor{currentfill}{rgb}{0.121569,0.466667,0.705882}%
\pgfsetfillcolor{currentfill}%
\pgfsetfillopacity{0.365540}%
\pgfsetlinewidth{1.003750pt}%
\definecolor{currentstroke}{rgb}{0.121569,0.466667,0.705882}%
\pgfsetstrokecolor{currentstroke}%
\pgfsetstrokeopacity{0.365540}%
\pgfsetdash{}{0pt}%
\pgfpathmoveto{\pgfqpoint{1.675661in}{3.008291in}}%
\pgfpathcurveto{\pgfqpoint{1.683897in}{3.008291in}}{\pgfqpoint{1.691797in}{3.011564in}}{\pgfqpoint{1.697621in}{3.017388in}}%
\pgfpathcurveto{\pgfqpoint{1.703445in}{3.023212in}}{\pgfqpoint{1.706717in}{3.031112in}}{\pgfqpoint{1.706717in}{3.039348in}}%
\pgfpathcurveto{\pgfqpoint{1.706717in}{3.047584in}}{\pgfqpoint{1.703445in}{3.055484in}}{\pgfqpoint{1.697621in}{3.061308in}}%
\pgfpathcurveto{\pgfqpoint{1.691797in}{3.067132in}}{\pgfqpoint{1.683897in}{3.070404in}}{\pgfqpoint{1.675661in}{3.070404in}}%
\pgfpathcurveto{\pgfqpoint{1.667425in}{3.070404in}}{\pgfqpoint{1.659525in}{3.067132in}}{\pgfqpoint{1.653701in}{3.061308in}}%
\pgfpathcurveto{\pgfqpoint{1.647877in}{3.055484in}}{\pgfqpoint{1.644604in}{3.047584in}}{\pgfqpoint{1.644604in}{3.039348in}}%
\pgfpathcurveto{\pgfqpoint{1.644604in}{3.031112in}}{\pgfqpoint{1.647877in}{3.023212in}}{\pgfqpoint{1.653701in}{3.017388in}}%
\pgfpathcurveto{\pgfqpoint{1.659525in}{3.011564in}}{\pgfqpoint{1.667425in}{3.008291in}}{\pgfqpoint{1.675661in}{3.008291in}}%
\pgfpathclose%
\pgfusepath{stroke,fill}%
\end{pgfscope}%
\begin{pgfscope}%
\pgfpathrectangle{\pgfqpoint{0.100000in}{0.212622in}}{\pgfqpoint{3.696000in}{3.696000in}}%
\pgfusepath{clip}%
\pgfsetbuttcap%
\pgfsetroundjoin%
\definecolor{currentfill}{rgb}{0.121569,0.466667,0.705882}%
\pgfsetfillcolor{currentfill}%
\pgfsetfillopacity{0.365673}%
\pgfsetlinewidth{1.003750pt}%
\definecolor{currentstroke}{rgb}{0.121569,0.466667,0.705882}%
\pgfsetstrokecolor{currentstroke}%
\pgfsetstrokeopacity{0.365673}%
\pgfsetdash{}{0pt}%
\pgfpathmoveto{\pgfqpoint{1.955497in}{3.048666in}}%
\pgfpathcurveto{\pgfqpoint{1.963733in}{3.048666in}}{\pgfqpoint{1.971633in}{3.051939in}}{\pgfqpoint{1.977457in}{3.057763in}}%
\pgfpathcurveto{\pgfqpoint{1.983281in}{3.063587in}}{\pgfqpoint{1.986554in}{3.071487in}}{\pgfqpoint{1.986554in}{3.079723in}}%
\pgfpathcurveto{\pgfqpoint{1.986554in}{3.087959in}}{\pgfqpoint{1.983281in}{3.095859in}}{\pgfqpoint{1.977457in}{3.101683in}}%
\pgfpathcurveto{\pgfqpoint{1.971633in}{3.107507in}}{\pgfqpoint{1.963733in}{3.110779in}}{\pgfqpoint{1.955497in}{3.110779in}}%
\pgfpathcurveto{\pgfqpoint{1.947261in}{3.110779in}}{\pgfqpoint{1.939361in}{3.107507in}}{\pgfqpoint{1.933537in}{3.101683in}}%
\pgfpathcurveto{\pgfqpoint{1.927713in}{3.095859in}}{\pgfqpoint{1.924441in}{3.087959in}}{\pgfqpoint{1.924441in}{3.079723in}}%
\pgfpathcurveto{\pgfqpoint{1.924441in}{3.071487in}}{\pgfqpoint{1.927713in}{3.063587in}}{\pgfqpoint{1.933537in}{3.057763in}}%
\pgfpathcurveto{\pgfqpoint{1.939361in}{3.051939in}}{\pgfqpoint{1.947261in}{3.048666in}}{\pgfqpoint{1.955497in}{3.048666in}}%
\pgfpathclose%
\pgfusepath{stroke,fill}%
\end{pgfscope}%
\begin{pgfscope}%
\pgfpathrectangle{\pgfqpoint{0.100000in}{0.212622in}}{\pgfqpoint{3.696000in}{3.696000in}}%
\pgfusepath{clip}%
\pgfsetbuttcap%
\pgfsetroundjoin%
\definecolor{currentfill}{rgb}{0.121569,0.466667,0.705882}%
\pgfsetfillcolor{currentfill}%
\pgfsetfillopacity{0.366225}%
\pgfsetlinewidth{1.003750pt}%
\definecolor{currentstroke}{rgb}{0.121569,0.466667,0.705882}%
\pgfsetstrokecolor{currentstroke}%
\pgfsetstrokeopacity{0.366225}%
\pgfsetdash{}{0pt}%
\pgfpathmoveto{\pgfqpoint{1.673559in}{3.004910in}}%
\pgfpathcurveto{\pgfqpoint{1.681795in}{3.004910in}}{\pgfqpoint{1.689695in}{3.008182in}}{\pgfqpoint{1.695519in}{3.014006in}}%
\pgfpathcurveto{\pgfqpoint{1.701343in}{3.019830in}}{\pgfqpoint{1.704615in}{3.027730in}}{\pgfqpoint{1.704615in}{3.035966in}}%
\pgfpathcurveto{\pgfqpoint{1.704615in}{3.044203in}}{\pgfqpoint{1.701343in}{3.052103in}}{\pgfqpoint{1.695519in}{3.057927in}}%
\pgfpathcurveto{\pgfqpoint{1.689695in}{3.063751in}}{\pgfqpoint{1.681795in}{3.067023in}}{\pgfqpoint{1.673559in}{3.067023in}}%
\pgfpathcurveto{\pgfqpoint{1.665323in}{3.067023in}}{\pgfqpoint{1.657423in}{3.063751in}}{\pgfqpoint{1.651599in}{3.057927in}}%
\pgfpathcurveto{\pgfqpoint{1.645775in}{3.052103in}}{\pgfqpoint{1.642502in}{3.044203in}}{\pgfqpoint{1.642502in}{3.035966in}}%
\pgfpathcurveto{\pgfqpoint{1.642502in}{3.027730in}}{\pgfqpoint{1.645775in}{3.019830in}}{\pgfqpoint{1.651599in}{3.014006in}}%
\pgfpathcurveto{\pgfqpoint{1.657423in}{3.008182in}}{\pgfqpoint{1.665323in}{3.004910in}}{\pgfqpoint{1.673559in}{3.004910in}}%
\pgfpathclose%
\pgfusepath{stroke,fill}%
\end{pgfscope}%
\begin{pgfscope}%
\pgfpathrectangle{\pgfqpoint{0.100000in}{0.212622in}}{\pgfqpoint{3.696000in}{3.696000in}}%
\pgfusepath{clip}%
\pgfsetbuttcap%
\pgfsetroundjoin%
\definecolor{currentfill}{rgb}{0.121569,0.466667,0.705882}%
\pgfsetfillcolor{currentfill}%
\pgfsetfillopacity{0.367005}%
\pgfsetlinewidth{1.003750pt}%
\definecolor{currentstroke}{rgb}{0.121569,0.466667,0.705882}%
\pgfsetstrokecolor{currentstroke}%
\pgfsetstrokeopacity{0.367005}%
\pgfsetdash{}{0pt}%
\pgfpathmoveto{\pgfqpoint{1.956435in}{3.043999in}}%
\pgfpathcurveto{\pgfqpoint{1.964672in}{3.043999in}}{\pgfqpoint{1.972572in}{3.047272in}}{\pgfqpoint{1.978396in}{3.053096in}}%
\pgfpathcurveto{\pgfqpoint{1.984220in}{3.058920in}}{\pgfqpoint{1.987492in}{3.066820in}}{\pgfqpoint{1.987492in}{3.075056in}}%
\pgfpathcurveto{\pgfqpoint{1.987492in}{3.083292in}}{\pgfqpoint{1.984220in}{3.091192in}}{\pgfqpoint{1.978396in}{3.097016in}}%
\pgfpathcurveto{\pgfqpoint{1.972572in}{3.102840in}}{\pgfqpoint{1.964672in}{3.106112in}}{\pgfqpoint{1.956435in}{3.106112in}}%
\pgfpathcurveto{\pgfqpoint{1.948199in}{3.106112in}}{\pgfqpoint{1.940299in}{3.102840in}}{\pgfqpoint{1.934475in}{3.097016in}}%
\pgfpathcurveto{\pgfqpoint{1.928651in}{3.091192in}}{\pgfqpoint{1.925379in}{3.083292in}}{\pgfqpoint{1.925379in}{3.075056in}}%
\pgfpathcurveto{\pgfqpoint{1.925379in}{3.066820in}}{\pgfqpoint{1.928651in}{3.058920in}}{\pgfqpoint{1.934475in}{3.053096in}}%
\pgfpathcurveto{\pgfqpoint{1.940299in}{3.047272in}}{\pgfqpoint{1.948199in}{3.043999in}}{\pgfqpoint{1.956435in}{3.043999in}}%
\pgfpathclose%
\pgfusepath{stroke,fill}%
\end{pgfscope}%
\begin{pgfscope}%
\pgfpathrectangle{\pgfqpoint{0.100000in}{0.212622in}}{\pgfqpoint{3.696000in}{3.696000in}}%
\pgfusepath{clip}%
\pgfsetbuttcap%
\pgfsetroundjoin%
\definecolor{currentfill}{rgb}{0.121569,0.466667,0.705882}%
\pgfsetfillcolor{currentfill}%
\pgfsetfillopacity{0.367468}%
\pgfsetlinewidth{1.003750pt}%
\definecolor{currentstroke}{rgb}{0.121569,0.466667,0.705882}%
\pgfsetstrokecolor{currentstroke}%
\pgfsetstrokeopacity{0.367468}%
\pgfsetdash{}{0pt}%
\pgfpathmoveto{\pgfqpoint{1.669857in}{2.998572in}}%
\pgfpathcurveto{\pgfqpoint{1.678093in}{2.998572in}}{\pgfqpoint{1.685993in}{3.001845in}}{\pgfqpoint{1.691817in}{3.007668in}}%
\pgfpathcurveto{\pgfqpoint{1.697641in}{3.013492in}}{\pgfqpoint{1.700913in}{3.021392in}}{\pgfqpoint{1.700913in}{3.029629in}}%
\pgfpathcurveto{\pgfqpoint{1.700913in}{3.037865in}}{\pgfqpoint{1.697641in}{3.045765in}}{\pgfqpoint{1.691817in}{3.051589in}}%
\pgfpathcurveto{\pgfqpoint{1.685993in}{3.057413in}}{\pgfqpoint{1.678093in}{3.060685in}}{\pgfqpoint{1.669857in}{3.060685in}}%
\pgfpathcurveto{\pgfqpoint{1.661620in}{3.060685in}}{\pgfqpoint{1.653720in}{3.057413in}}{\pgfqpoint{1.647897in}{3.051589in}}%
\pgfpathcurveto{\pgfqpoint{1.642073in}{3.045765in}}{\pgfqpoint{1.638800in}{3.037865in}}{\pgfqpoint{1.638800in}{3.029629in}}%
\pgfpathcurveto{\pgfqpoint{1.638800in}{3.021392in}}{\pgfqpoint{1.642073in}{3.013492in}}{\pgfqpoint{1.647897in}{3.007668in}}%
\pgfpathcurveto{\pgfqpoint{1.653720in}{3.001845in}}{\pgfqpoint{1.661620in}{2.998572in}}{\pgfqpoint{1.669857in}{2.998572in}}%
\pgfpathclose%
\pgfusepath{stroke,fill}%
\end{pgfscope}%
\begin{pgfscope}%
\pgfpathrectangle{\pgfqpoint{0.100000in}{0.212622in}}{\pgfqpoint{3.696000in}{3.696000in}}%
\pgfusepath{clip}%
\pgfsetbuttcap%
\pgfsetroundjoin%
\definecolor{currentfill}{rgb}{0.121569,0.466667,0.705882}%
\pgfsetfillcolor{currentfill}%
\pgfsetfillopacity{0.368398}%
\pgfsetlinewidth{1.003750pt}%
\definecolor{currentstroke}{rgb}{0.121569,0.466667,0.705882}%
\pgfsetstrokecolor{currentstroke}%
\pgfsetstrokeopacity{0.368398}%
\pgfsetdash{}{0pt}%
\pgfpathmoveto{\pgfqpoint{1.957549in}{3.038854in}}%
\pgfpathcurveto{\pgfqpoint{1.965785in}{3.038854in}}{\pgfqpoint{1.973685in}{3.042127in}}{\pgfqpoint{1.979509in}{3.047950in}}%
\pgfpathcurveto{\pgfqpoint{1.985333in}{3.053774in}}{\pgfqpoint{1.988605in}{3.061674in}}{\pgfqpoint{1.988605in}{3.069911in}}%
\pgfpathcurveto{\pgfqpoint{1.988605in}{3.078147in}}{\pgfqpoint{1.985333in}{3.086047in}}{\pgfqpoint{1.979509in}{3.091871in}}%
\pgfpathcurveto{\pgfqpoint{1.973685in}{3.097695in}}{\pgfqpoint{1.965785in}{3.100967in}}{\pgfqpoint{1.957549in}{3.100967in}}%
\pgfpathcurveto{\pgfqpoint{1.949312in}{3.100967in}}{\pgfqpoint{1.941412in}{3.097695in}}{\pgfqpoint{1.935588in}{3.091871in}}%
\pgfpathcurveto{\pgfqpoint{1.929764in}{3.086047in}}{\pgfqpoint{1.926492in}{3.078147in}}{\pgfqpoint{1.926492in}{3.069911in}}%
\pgfpathcurveto{\pgfqpoint{1.926492in}{3.061674in}}{\pgfqpoint{1.929764in}{3.053774in}}{\pgfqpoint{1.935588in}{3.047950in}}%
\pgfpathcurveto{\pgfqpoint{1.941412in}{3.042127in}}{\pgfqpoint{1.949312in}{3.038854in}}{\pgfqpoint{1.957549in}{3.038854in}}%
\pgfpathclose%
\pgfusepath{stroke,fill}%
\end{pgfscope}%
\begin{pgfscope}%
\pgfpathrectangle{\pgfqpoint{0.100000in}{0.212622in}}{\pgfqpoint{3.696000in}{3.696000in}}%
\pgfusepath{clip}%
\pgfsetbuttcap%
\pgfsetroundjoin%
\definecolor{currentfill}{rgb}{0.121569,0.466667,0.705882}%
\pgfsetfillcolor{currentfill}%
\pgfsetfillopacity{0.368532}%
\pgfsetlinewidth{1.003750pt}%
\definecolor{currentstroke}{rgb}{0.121569,0.466667,0.705882}%
\pgfsetstrokecolor{currentstroke}%
\pgfsetstrokeopacity{0.368532}%
\pgfsetdash{}{0pt}%
\pgfpathmoveto{\pgfqpoint{1.667024in}{2.993225in}}%
\pgfpathcurveto{\pgfqpoint{1.675261in}{2.993225in}}{\pgfqpoint{1.683161in}{2.996498in}}{\pgfqpoint{1.688985in}{3.002322in}}%
\pgfpathcurveto{\pgfqpoint{1.694808in}{3.008145in}}{\pgfqpoint{1.698081in}{3.016045in}}{\pgfqpoint{1.698081in}{3.024282in}}%
\pgfpathcurveto{\pgfqpoint{1.698081in}{3.032518in}}{\pgfqpoint{1.694808in}{3.040418in}}{\pgfqpoint{1.688985in}{3.046242in}}%
\pgfpathcurveto{\pgfqpoint{1.683161in}{3.052066in}}{\pgfqpoint{1.675261in}{3.055338in}}{\pgfqpoint{1.667024in}{3.055338in}}%
\pgfpathcurveto{\pgfqpoint{1.658788in}{3.055338in}}{\pgfqpoint{1.650888in}{3.052066in}}{\pgfqpoint{1.645064in}{3.046242in}}%
\pgfpathcurveto{\pgfqpoint{1.639240in}{3.040418in}}{\pgfqpoint{1.635968in}{3.032518in}}{\pgfqpoint{1.635968in}{3.024282in}}%
\pgfpathcurveto{\pgfqpoint{1.635968in}{3.016045in}}{\pgfqpoint{1.639240in}{3.008145in}}{\pgfqpoint{1.645064in}{3.002322in}}%
\pgfpathcurveto{\pgfqpoint{1.650888in}{2.996498in}}{\pgfqpoint{1.658788in}{2.993225in}}{\pgfqpoint{1.667024in}{2.993225in}}%
\pgfpathclose%
\pgfusepath{stroke,fill}%
\end{pgfscope}%
\begin{pgfscope}%
\pgfpathrectangle{\pgfqpoint{0.100000in}{0.212622in}}{\pgfqpoint{3.696000in}{3.696000in}}%
\pgfusepath{clip}%
\pgfsetbuttcap%
\pgfsetroundjoin%
\definecolor{currentfill}{rgb}{0.121569,0.466667,0.705882}%
\pgfsetfillcolor{currentfill}%
\pgfsetfillopacity{0.369308}%
\pgfsetlinewidth{1.003750pt}%
\definecolor{currentstroke}{rgb}{0.121569,0.466667,0.705882}%
\pgfsetstrokecolor{currentstroke}%
\pgfsetstrokeopacity{0.369308}%
\pgfsetdash{}{0pt}%
\pgfpathmoveto{\pgfqpoint{1.664343in}{2.988964in}}%
\pgfpathcurveto{\pgfqpoint{1.672579in}{2.988964in}}{\pgfqpoint{1.680479in}{2.992236in}}{\pgfqpoint{1.686303in}{2.998060in}}%
\pgfpathcurveto{\pgfqpoint{1.692127in}{3.003884in}}{\pgfqpoint{1.695399in}{3.011784in}}{\pgfqpoint{1.695399in}{3.020021in}}%
\pgfpathcurveto{\pgfqpoint{1.695399in}{3.028257in}}{\pgfqpoint{1.692127in}{3.036157in}}{\pgfqpoint{1.686303in}{3.041981in}}%
\pgfpathcurveto{\pgfqpoint{1.680479in}{3.047805in}}{\pgfqpoint{1.672579in}{3.051077in}}{\pgfqpoint{1.664343in}{3.051077in}}%
\pgfpathcurveto{\pgfqpoint{1.656106in}{3.051077in}}{\pgfqpoint{1.648206in}{3.047805in}}{\pgfqpoint{1.642382in}{3.041981in}}%
\pgfpathcurveto{\pgfqpoint{1.636558in}{3.036157in}}{\pgfqpoint{1.633286in}{3.028257in}}{\pgfqpoint{1.633286in}{3.020021in}}%
\pgfpathcurveto{\pgfqpoint{1.633286in}{3.011784in}}{\pgfqpoint{1.636558in}{3.003884in}}{\pgfqpoint{1.642382in}{2.998060in}}%
\pgfpathcurveto{\pgfqpoint{1.648206in}{2.992236in}}{\pgfqpoint{1.656106in}{2.988964in}}{\pgfqpoint{1.664343in}{2.988964in}}%
\pgfpathclose%
\pgfusepath{stroke,fill}%
\end{pgfscope}%
\begin{pgfscope}%
\pgfpathrectangle{\pgfqpoint{0.100000in}{0.212622in}}{\pgfqpoint{3.696000in}{3.696000in}}%
\pgfusepath{clip}%
\pgfsetbuttcap%
\pgfsetroundjoin%
\definecolor{currentfill}{rgb}{0.121569,0.466667,0.705882}%
\pgfsetfillcolor{currentfill}%
\pgfsetfillopacity{0.369779}%
\pgfsetlinewidth{1.003750pt}%
\definecolor{currentstroke}{rgb}{0.121569,0.466667,0.705882}%
\pgfsetstrokecolor{currentstroke}%
\pgfsetstrokeopacity{0.369779}%
\pgfsetdash{}{0pt}%
\pgfpathmoveto{\pgfqpoint{1.663081in}{2.986437in}}%
\pgfpathcurveto{\pgfqpoint{1.671318in}{2.986437in}}{\pgfqpoint{1.679218in}{2.989709in}}{\pgfqpoint{1.685041in}{2.995533in}}%
\pgfpathcurveto{\pgfqpoint{1.690865in}{3.001357in}}{\pgfqpoint{1.694138in}{3.009257in}}{\pgfqpoint{1.694138in}{3.017493in}}%
\pgfpathcurveto{\pgfqpoint{1.694138in}{3.025730in}}{\pgfqpoint{1.690865in}{3.033630in}}{\pgfqpoint{1.685041in}{3.039454in}}%
\pgfpathcurveto{\pgfqpoint{1.679218in}{3.045277in}}{\pgfqpoint{1.671318in}{3.048550in}}{\pgfqpoint{1.663081in}{3.048550in}}%
\pgfpathcurveto{\pgfqpoint{1.654845in}{3.048550in}}{\pgfqpoint{1.646945in}{3.045277in}}{\pgfqpoint{1.641121in}{3.039454in}}%
\pgfpathcurveto{\pgfqpoint{1.635297in}{3.033630in}}{\pgfqpoint{1.632025in}{3.025730in}}{\pgfqpoint{1.632025in}{3.017493in}}%
\pgfpathcurveto{\pgfqpoint{1.632025in}{3.009257in}}{\pgfqpoint{1.635297in}{3.001357in}}{\pgfqpoint{1.641121in}{2.995533in}}%
\pgfpathcurveto{\pgfqpoint{1.646945in}{2.989709in}}{\pgfqpoint{1.654845in}{2.986437in}}{\pgfqpoint{1.663081in}{2.986437in}}%
\pgfpathclose%
\pgfusepath{stroke,fill}%
\end{pgfscope}%
\begin{pgfscope}%
\pgfpathrectangle{\pgfqpoint{0.100000in}{0.212622in}}{\pgfqpoint{3.696000in}{3.696000in}}%
\pgfusepath{clip}%
\pgfsetbuttcap%
\pgfsetroundjoin%
\definecolor{currentfill}{rgb}{0.121569,0.466667,0.705882}%
\pgfsetfillcolor{currentfill}%
\pgfsetfillopacity{0.370173}%
\pgfsetlinewidth{1.003750pt}%
\definecolor{currentstroke}{rgb}{0.121569,0.466667,0.705882}%
\pgfsetstrokecolor{currentstroke}%
\pgfsetstrokeopacity{0.370173}%
\pgfsetdash{}{0pt}%
\pgfpathmoveto{\pgfqpoint{1.958284in}{3.033425in}}%
\pgfpathcurveto{\pgfqpoint{1.966520in}{3.033425in}}{\pgfqpoint{1.974420in}{3.036698in}}{\pgfqpoint{1.980244in}{3.042521in}}%
\pgfpathcurveto{\pgfqpoint{1.986068in}{3.048345in}}{\pgfqpoint{1.989340in}{3.056245in}}{\pgfqpoint{1.989340in}{3.064482in}}%
\pgfpathcurveto{\pgfqpoint{1.989340in}{3.072718in}}{\pgfqpoint{1.986068in}{3.080618in}}{\pgfqpoint{1.980244in}{3.086442in}}%
\pgfpathcurveto{\pgfqpoint{1.974420in}{3.092266in}}{\pgfqpoint{1.966520in}{3.095538in}}{\pgfqpoint{1.958284in}{3.095538in}}%
\pgfpathcurveto{\pgfqpoint{1.950048in}{3.095538in}}{\pgfqpoint{1.942148in}{3.092266in}}{\pgfqpoint{1.936324in}{3.086442in}}%
\pgfpathcurveto{\pgfqpoint{1.930500in}{3.080618in}}{\pgfqpoint{1.927227in}{3.072718in}}{\pgfqpoint{1.927227in}{3.064482in}}%
\pgfpathcurveto{\pgfqpoint{1.927227in}{3.056245in}}{\pgfqpoint{1.930500in}{3.048345in}}{\pgfqpoint{1.936324in}{3.042521in}}%
\pgfpathcurveto{\pgfqpoint{1.942148in}{3.036698in}}{\pgfqpoint{1.950048in}{3.033425in}}{\pgfqpoint{1.958284in}{3.033425in}}%
\pgfpathclose%
\pgfusepath{stroke,fill}%
\end{pgfscope}%
\begin{pgfscope}%
\pgfpathrectangle{\pgfqpoint{0.100000in}{0.212622in}}{\pgfqpoint{3.696000in}{3.696000in}}%
\pgfusepath{clip}%
\pgfsetbuttcap%
\pgfsetroundjoin%
\definecolor{currentfill}{rgb}{0.121569,0.466667,0.705882}%
\pgfsetfillcolor{currentfill}%
\pgfsetfillopacity{0.370636}%
\pgfsetlinewidth{1.003750pt}%
\definecolor{currentstroke}{rgb}{0.121569,0.466667,0.705882}%
\pgfsetstrokecolor{currentstroke}%
\pgfsetstrokeopacity{0.370636}%
\pgfsetdash{}{0pt}%
\pgfpathmoveto{\pgfqpoint{1.660572in}{2.982114in}}%
\pgfpathcurveto{\pgfqpoint{1.668809in}{2.982114in}}{\pgfqpoint{1.676709in}{2.985386in}}{\pgfqpoint{1.682533in}{2.991210in}}%
\pgfpathcurveto{\pgfqpoint{1.688357in}{2.997034in}}{\pgfqpoint{1.691629in}{3.004934in}}{\pgfqpoint{1.691629in}{3.013170in}}%
\pgfpathcurveto{\pgfqpoint{1.691629in}{3.021407in}}{\pgfqpoint{1.688357in}{3.029307in}}{\pgfqpoint{1.682533in}{3.035131in}}%
\pgfpathcurveto{\pgfqpoint{1.676709in}{3.040955in}}{\pgfqpoint{1.668809in}{3.044227in}}{\pgfqpoint{1.660572in}{3.044227in}}%
\pgfpathcurveto{\pgfqpoint{1.652336in}{3.044227in}}{\pgfqpoint{1.644436in}{3.040955in}}{\pgfqpoint{1.638612in}{3.035131in}}%
\pgfpathcurveto{\pgfqpoint{1.632788in}{3.029307in}}{\pgfqpoint{1.629516in}{3.021407in}}{\pgfqpoint{1.629516in}{3.013170in}}%
\pgfpathcurveto{\pgfqpoint{1.629516in}{3.004934in}}{\pgfqpoint{1.632788in}{2.997034in}}{\pgfqpoint{1.638612in}{2.991210in}}%
\pgfpathcurveto{\pgfqpoint{1.644436in}{2.985386in}}{\pgfqpoint{1.652336in}{2.982114in}}{\pgfqpoint{1.660572in}{2.982114in}}%
\pgfpathclose%
\pgfusepath{stroke,fill}%
\end{pgfscope}%
\begin{pgfscope}%
\pgfpathrectangle{\pgfqpoint{0.100000in}{0.212622in}}{\pgfqpoint{3.696000in}{3.696000in}}%
\pgfusepath{clip}%
\pgfsetbuttcap%
\pgfsetroundjoin%
\definecolor{currentfill}{rgb}{0.121569,0.466667,0.705882}%
\pgfsetfillcolor{currentfill}%
\pgfsetfillopacity{0.372110}%
\pgfsetlinewidth{1.003750pt}%
\definecolor{currentstroke}{rgb}{0.121569,0.466667,0.705882}%
\pgfsetstrokecolor{currentstroke}%
\pgfsetstrokeopacity{0.372110}%
\pgfsetdash{}{0pt}%
\pgfpathmoveto{\pgfqpoint{1.655817in}{2.974196in}}%
\pgfpathcurveto{\pgfqpoint{1.664053in}{2.974196in}}{\pgfqpoint{1.671953in}{2.977468in}}{\pgfqpoint{1.677777in}{2.983292in}}%
\pgfpathcurveto{\pgfqpoint{1.683601in}{2.989116in}}{\pgfqpoint{1.686874in}{2.997016in}}{\pgfqpoint{1.686874in}{3.005253in}}%
\pgfpathcurveto{\pgfqpoint{1.686874in}{3.013489in}}{\pgfqpoint{1.683601in}{3.021389in}}{\pgfqpoint{1.677777in}{3.027213in}}%
\pgfpathcurveto{\pgfqpoint{1.671953in}{3.033037in}}{\pgfqpoint{1.664053in}{3.036309in}}{\pgfqpoint{1.655817in}{3.036309in}}%
\pgfpathcurveto{\pgfqpoint{1.647581in}{3.036309in}}{\pgfqpoint{1.639681in}{3.033037in}}{\pgfqpoint{1.633857in}{3.027213in}}%
\pgfpathcurveto{\pgfqpoint{1.628033in}{3.021389in}}{\pgfqpoint{1.624761in}{3.013489in}}{\pgfqpoint{1.624761in}{3.005253in}}%
\pgfpathcurveto{\pgfqpoint{1.624761in}{2.997016in}}{\pgfqpoint{1.628033in}{2.989116in}}{\pgfqpoint{1.633857in}{2.983292in}}%
\pgfpathcurveto{\pgfqpoint{1.639681in}{2.977468in}}{\pgfqpoint{1.647581in}{2.974196in}}{\pgfqpoint{1.655817in}{2.974196in}}%
\pgfpathclose%
\pgfusepath{stroke,fill}%
\end{pgfscope}%
\begin{pgfscope}%
\pgfpathrectangle{\pgfqpoint{0.100000in}{0.212622in}}{\pgfqpoint{3.696000in}{3.696000in}}%
\pgfusepath{clip}%
\pgfsetbuttcap%
\pgfsetroundjoin%
\definecolor{currentfill}{rgb}{0.121569,0.466667,0.705882}%
\pgfsetfillcolor{currentfill}%
\pgfsetfillopacity{0.372111}%
\pgfsetlinewidth{1.003750pt}%
\definecolor{currentstroke}{rgb}{0.121569,0.466667,0.705882}%
\pgfsetstrokecolor{currentstroke}%
\pgfsetstrokeopacity{0.372111}%
\pgfsetdash{}{0pt}%
\pgfpathmoveto{\pgfqpoint{1.959716in}{3.027465in}}%
\pgfpathcurveto{\pgfqpoint{1.967952in}{3.027465in}}{\pgfqpoint{1.975852in}{3.030737in}}{\pgfqpoint{1.981676in}{3.036561in}}%
\pgfpathcurveto{\pgfqpoint{1.987500in}{3.042385in}}{\pgfqpoint{1.990772in}{3.050285in}}{\pgfqpoint{1.990772in}{3.058522in}}%
\pgfpathcurveto{\pgfqpoint{1.990772in}{3.066758in}}{\pgfqpoint{1.987500in}{3.074658in}}{\pgfqpoint{1.981676in}{3.080482in}}%
\pgfpathcurveto{\pgfqpoint{1.975852in}{3.086306in}}{\pgfqpoint{1.967952in}{3.089578in}}{\pgfqpoint{1.959716in}{3.089578in}}%
\pgfpathcurveto{\pgfqpoint{1.951480in}{3.089578in}}{\pgfqpoint{1.943579in}{3.086306in}}{\pgfqpoint{1.937756in}{3.080482in}}%
\pgfpathcurveto{\pgfqpoint{1.931932in}{3.074658in}}{\pgfqpoint{1.928659in}{3.066758in}}{\pgfqpoint{1.928659in}{3.058522in}}%
\pgfpathcurveto{\pgfqpoint{1.928659in}{3.050285in}}{\pgfqpoint{1.931932in}{3.042385in}}{\pgfqpoint{1.937756in}{3.036561in}}%
\pgfpathcurveto{\pgfqpoint{1.943579in}{3.030737in}}{\pgfqpoint{1.951480in}{3.027465in}}{\pgfqpoint{1.959716in}{3.027465in}}%
\pgfpathclose%
\pgfusepath{stroke,fill}%
\end{pgfscope}%
\begin{pgfscope}%
\pgfpathrectangle{\pgfqpoint{0.100000in}{0.212622in}}{\pgfqpoint{3.696000in}{3.696000in}}%
\pgfusepath{clip}%
\pgfsetbuttcap%
\pgfsetroundjoin%
\definecolor{currentfill}{rgb}{0.121569,0.466667,0.705882}%
\pgfsetfillcolor{currentfill}%
\pgfsetfillopacity{0.373533}%
\pgfsetlinewidth{1.003750pt}%
\definecolor{currentstroke}{rgb}{0.121569,0.466667,0.705882}%
\pgfsetstrokecolor{currentstroke}%
\pgfsetstrokeopacity{0.373533}%
\pgfsetdash{}{0pt}%
\pgfpathmoveto{\pgfqpoint{1.652316in}{2.966527in}}%
\pgfpathcurveto{\pgfqpoint{1.660552in}{2.966527in}}{\pgfqpoint{1.668452in}{2.969799in}}{\pgfqpoint{1.674276in}{2.975623in}}%
\pgfpathcurveto{\pgfqpoint{1.680100in}{2.981447in}}{\pgfqpoint{1.683372in}{2.989347in}}{\pgfqpoint{1.683372in}{2.997583in}}%
\pgfpathcurveto{\pgfqpoint{1.683372in}{3.005819in}}{\pgfqpoint{1.680100in}{3.013719in}}{\pgfqpoint{1.674276in}{3.019543in}}%
\pgfpathcurveto{\pgfqpoint{1.668452in}{3.025367in}}{\pgfqpoint{1.660552in}{3.028640in}}{\pgfqpoint{1.652316in}{3.028640in}}%
\pgfpathcurveto{\pgfqpoint{1.644079in}{3.028640in}}{\pgfqpoint{1.636179in}{3.025367in}}{\pgfqpoint{1.630355in}{3.019543in}}%
\pgfpathcurveto{\pgfqpoint{1.624531in}{3.013719in}}{\pgfqpoint{1.621259in}{3.005819in}}{\pgfqpoint{1.621259in}{2.997583in}}%
\pgfpathcurveto{\pgfqpoint{1.621259in}{2.989347in}}{\pgfqpoint{1.624531in}{2.981447in}}{\pgfqpoint{1.630355in}{2.975623in}}%
\pgfpathcurveto{\pgfqpoint{1.636179in}{2.969799in}}{\pgfqpoint{1.644079in}{2.966527in}}{\pgfqpoint{1.652316in}{2.966527in}}%
\pgfpathclose%
\pgfusepath{stroke,fill}%
\end{pgfscope}%
\begin{pgfscope}%
\pgfpathrectangle{\pgfqpoint{0.100000in}{0.212622in}}{\pgfqpoint{3.696000in}{3.696000in}}%
\pgfusepath{clip}%
\pgfsetbuttcap%
\pgfsetroundjoin%
\definecolor{currentfill}{rgb}{0.121569,0.466667,0.705882}%
\pgfsetfillcolor{currentfill}%
\pgfsetfillopacity{0.374109}%
\pgfsetlinewidth{1.003750pt}%
\definecolor{currentstroke}{rgb}{0.121569,0.466667,0.705882}%
\pgfsetstrokecolor{currentstroke}%
\pgfsetstrokeopacity{0.374109}%
\pgfsetdash{}{0pt}%
\pgfpathmoveto{\pgfqpoint{1.961141in}{3.020427in}}%
\pgfpathcurveto{\pgfqpoint{1.969378in}{3.020427in}}{\pgfqpoint{1.977278in}{3.023700in}}{\pgfqpoint{1.983102in}{3.029524in}}%
\pgfpathcurveto{\pgfqpoint{1.988925in}{3.035348in}}{\pgfqpoint{1.992198in}{3.043248in}}{\pgfqpoint{1.992198in}{3.051484in}}%
\pgfpathcurveto{\pgfqpoint{1.992198in}{3.059720in}}{\pgfqpoint{1.988925in}{3.067620in}}{\pgfqpoint{1.983102in}{3.073444in}}%
\pgfpathcurveto{\pgfqpoint{1.977278in}{3.079268in}}{\pgfqpoint{1.969378in}{3.082540in}}{\pgfqpoint{1.961141in}{3.082540in}}%
\pgfpathcurveto{\pgfqpoint{1.952905in}{3.082540in}}{\pgfqpoint{1.945005in}{3.079268in}}{\pgfqpoint{1.939181in}{3.073444in}}%
\pgfpathcurveto{\pgfqpoint{1.933357in}{3.067620in}}{\pgfqpoint{1.930085in}{3.059720in}}{\pgfqpoint{1.930085in}{3.051484in}}%
\pgfpathcurveto{\pgfqpoint{1.930085in}{3.043248in}}{\pgfqpoint{1.933357in}{3.035348in}}{\pgfqpoint{1.939181in}{3.029524in}}%
\pgfpathcurveto{\pgfqpoint{1.945005in}{3.023700in}}{\pgfqpoint{1.952905in}{3.020427in}}{\pgfqpoint{1.961141in}{3.020427in}}%
\pgfpathclose%
\pgfusepath{stroke,fill}%
\end{pgfscope}%
\begin{pgfscope}%
\pgfpathrectangle{\pgfqpoint{0.100000in}{0.212622in}}{\pgfqpoint{3.696000in}{3.696000in}}%
\pgfusepath{clip}%
\pgfsetbuttcap%
\pgfsetroundjoin%
\definecolor{currentfill}{rgb}{0.121569,0.466667,0.705882}%
\pgfsetfillcolor{currentfill}%
\pgfsetfillopacity{0.374451}%
\pgfsetlinewidth{1.003750pt}%
\definecolor{currentstroke}{rgb}{0.121569,0.466667,0.705882}%
\pgfsetstrokecolor{currentstroke}%
\pgfsetstrokeopacity{0.374451}%
\pgfsetdash{}{0pt}%
\pgfpathmoveto{\pgfqpoint{1.649441in}{2.961850in}}%
\pgfpathcurveto{\pgfqpoint{1.657678in}{2.961850in}}{\pgfqpoint{1.665578in}{2.965123in}}{\pgfqpoint{1.671402in}{2.970947in}}%
\pgfpathcurveto{\pgfqpoint{1.677226in}{2.976771in}}{\pgfqpoint{1.680498in}{2.984671in}}{\pgfqpoint{1.680498in}{2.992907in}}%
\pgfpathcurveto{\pgfqpoint{1.680498in}{3.001143in}}{\pgfqpoint{1.677226in}{3.009043in}}{\pgfqpoint{1.671402in}{3.014867in}}%
\pgfpathcurveto{\pgfqpoint{1.665578in}{3.020691in}}{\pgfqpoint{1.657678in}{3.023963in}}{\pgfqpoint{1.649441in}{3.023963in}}%
\pgfpathcurveto{\pgfqpoint{1.641205in}{3.023963in}}{\pgfqpoint{1.633305in}{3.020691in}}{\pgfqpoint{1.627481in}{3.014867in}}%
\pgfpathcurveto{\pgfqpoint{1.621657in}{3.009043in}}{\pgfqpoint{1.618385in}{3.001143in}}{\pgfqpoint{1.618385in}{2.992907in}}%
\pgfpathcurveto{\pgfqpoint{1.618385in}{2.984671in}}{\pgfqpoint{1.621657in}{2.976771in}}{\pgfqpoint{1.627481in}{2.970947in}}%
\pgfpathcurveto{\pgfqpoint{1.633305in}{2.965123in}}{\pgfqpoint{1.641205in}{2.961850in}}{\pgfqpoint{1.649441in}{2.961850in}}%
\pgfpathclose%
\pgfusepath{stroke,fill}%
\end{pgfscope}%
\begin{pgfscope}%
\pgfpathrectangle{\pgfqpoint{0.100000in}{0.212622in}}{\pgfqpoint{3.696000in}{3.696000in}}%
\pgfusepath{clip}%
\pgfsetbuttcap%
\pgfsetroundjoin%
\definecolor{currentfill}{rgb}{0.121569,0.466667,0.705882}%
\pgfsetfillcolor{currentfill}%
\pgfsetfillopacity{0.375357}%
\pgfsetlinewidth{1.003750pt}%
\definecolor{currentstroke}{rgb}{0.121569,0.466667,0.705882}%
\pgfsetstrokecolor{currentstroke}%
\pgfsetstrokeopacity{0.375357}%
\pgfsetdash{}{0pt}%
\pgfpathmoveto{\pgfqpoint{1.646992in}{2.957300in}}%
\pgfpathcurveto{\pgfqpoint{1.655228in}{2.957300in}}{\pgfqpoint{1.663128in}{2.960572in}}{\pgfqpoint{1.668952in}{2.966396in}}%
\pgfpathcurveto{\pgfqpoint{1.674776in}{2.972220in}}{\pgfqpoint{1.678048in}{2.980120in}}{\pgfqpoint{1.678048in}{2.988356in}}%
\pgfpathcurveto{\pgfqpoint{1.678048in}{2.996593in}}{\pgfqpoint{1.674776in}{3.004493in}}{\pgfqpoint{1.668952in}{3.010317in}}%
\pgfpathcurveto{\pgfqpoint{1.663128in}{3.016141in}}{\pgfqpoint{1.655228in}{3.019413in}}{\pgfqpoint{1.646992in}{3.019413in}}%
\pgfpathcurveto{\pgfqpoint{1.638756in}{3.019413in}}{\pgfqpoint{1.630856in}{3.016141in}}{\pgfqpoint{1.625032in}{3.010317in}}%
\pgfpathcurveto{\pgfqpoint{1.619208in}{3.004493in}}{\pgfqpoint{1.615935in}{2.996593in}}{\pgfqpoint{1.615935in}{2.988356in}}%
\pgfpathcurveto{\pgfqpoint{1.615935in}{2.980120in}}{\pgfqpoint{1.619208in}{2.972220in}}{\pgfqpoint{1.625032in}{2.966396in}}%
\pgfpathcurveto{\pgfqpoint{1.630856in}{2.960572in}}{\pgfqpoint{1.638756in}{2.957300in}}{\pgfqpoint{1.646992in}{2.957300in}}%
\pgfpathclose%
\pgfusepath{stroke,fill}%
\end{pgfscope}%
\begin{pgfscope}%
\pgfpathrectangle{\pgfqpoint{0.100000in}{0.212622in}}{\pgfqpoint{3.696000in}{3.696000in}}%
\pgfusepath{clip}%
\pgfsetbuttcap%
\pgfsetroundjoin%
\definecolor{currentfill}{rgb}{0.121569,0.466667,0.705882}%
\pgfsetfillcolor{currentfill}%
\pgfsetfillopacity{0.376177}%
\pgfsetlinewidth{1.003750pt}%
\definecolor{currentstroke}{rgb}{0.121569,0.466667,0.705882}%
\pgfsetstrokecolor{currentstroke}%
\pgfsetstrokeopacity{0.376177}%
\pgfsetdash{}{0pt}%
\pgfpathmoveto{\pgfqpoint{1.644800in}{2.953297in}}%
\pgfpathcurveto{\pgfqpoint{1.653036in}{2.953297in}}{\pgfqpoint{1.660936in}{2.956569in}}{\pgfqpoint{1.666760in}{2.962393in}}%
\pgfpathcurveto{\pgfqpoint{1.672584in}{2.968217in}}{\pgfqpoint{1.675857in}{2.976117in}}{\pgfqpoint{1.675857in}{2.984353in}}%
\pgfpathcurveto{\pgfqpoint{1.675857in}{2.992590in}}{\pgfqpoint{1.672584in}{3.000490in}}{\pgfqpoint{1.666760in}{3.006314in}}%
\pgfpathcurveto{\pgfqpoint{1.660936in}{3.012138in}}{\pgfqpoint{1.653036in}{3.015410in}}{\pgfqpoint{1.644800in}{3.015410in}}%
\pgfpathcurveto{\pgfqpoint{1.636564in}{3.015410in}}{\pgfqpoint{1.628664in}{3.012138in}}{\pgfqpoint{1.622840in}{3.006314in}}%
\pgfpathcurveto{\pgfqpoint{1.617016in}{3.000490in}}{\pgfqpoint{1.613744in}{2.992590in}}{\pgfqpoint{1.613744in}{2.984353in}}%
\pgfpathcurveto{\pgfqpoint{1.613744in}{2.976117in}}{\pgfqpoint{1.617016in}{2.968217in}}{\pgfqpoint{1.622840in}{2.962393in}}%
\pgfpathcurveto{\pgfqpoint{1.628664in}{2.956569in}}{\pgfqpoint{1.636564in}{2.953297in}}{\pgfqpoint{1.644800in}{2.953297in}}%
\pgfpathclose%
\pgfusepath{stroke,fill}%
\end{pgfscope}%
\begin{pgfscope}%
\pgfpathrectangle{\pgfqpoint{0.100000in}{0.212622in}}{\pgfqpoint{3.696000in}{3.696000in}}%
\pgfusepath{clip}%
\pgfsetbuttcap%
\pgfsetroundjoin%
\definecolor{currentfill}{rgb}{0.121569,0.466667,0.705882}%
\pgfsetfillcolor{currentfill}%
\pgfsetfillopacity{0.376551}%
\pgfsetlinewidth{1.003750pt}%
\definecolor{currentstroke}{rgb}{0.121569,0.466667,0.705882}%
\pgfsetstrokecolor{currentstroke}%
\pgfsetstrokeopacity{0.376551}%
\pgfsetdash{}{0pt}%
\pgfpathmoveto{\pgfqpoint{1.962212in}{3.013436in}}%
\pgfpathcurveto{\pgfqpoint{1.970448in}{3.013436in}}{\pgfqpoint{1.978348in}{3.016708in}}{\pgfqpoint{1.984172in}{3.022532in}}%
\pgfpathcurveto{\pgfqpoint{1.989996in}{3.028356in}}{\pgfqpoint{1.993268in}{3.036256in}}{\pgfqpoint{1.993268in}{3.044493in}}%
\pgfpathcurveto{\pgfqpoint{1.993268in}{3.052729in}}{\pgfqpoint{1.989996in}{3.060629in}}{\pgfqpoint{1.984172in}{3.066453in}}%
\pgfpathcurveto{\pgfqpoint{1.978348in}{3.072277in}}{\pgfqpoint{1.970448in}{3.075549in}}{\pgfqpoint{1.962212in}{3.075549in}}%
\pgfpathcurveto{\pgfqpoint{1.953976in}{3.075549in}}{\pgfqpoint{1.946075in}{3.072277in}}{\pgfqpoint{1.940252in}{3.066453in}}%
\pgfpathcurveto{\pgfqpoint{1.934428in}{3.060629in}}{\pgfqpoint{1.931155in}{3.052729in}}{\pgfqpoint{1.931155in}{3.044493in}}%
\pgfpathcurveto{\pgfqpoint{1.931155in}{3.036256in}}{\pgfqpoint{1.934428in}{3.028356in}}{\pgfqpoint{1.940252in}{3.022532in}}%
\pgfpathcurveto{\pgfqpoint{1.946075in}{3.016708in}}{\pgfqpoint{1.953976in}{3.013436in}}{\pgfqpoint{1.962212in}{3.013436in}}%
\pgfpathclose%
\pgfusepath{stroke,fill}%
\end{pgfscope}%
\begin{pgfscope}%
\pgfpathrectangle{\pgfqpoint{0.100000in}{0.212622in}}{\pgfqpoint{3.696000in}{3.696000in}}%
\pgfusepath{clip}%
\pgfsetbuttcap%
\pgfsetroundjoin%
\definecolor{currentfill}{rgb}{0.121569,0.466667,0.705882}%
\pgfsetfillcolor{currentfill}%
\pgfsetfillopacity{0.376840}%
\pgfsetlinewidth{1.003750pt}%
\definecolor{currentstroke}{rgb}{0.121569,0.466667,0.705882}%
\pgfsetstrokecolor{currentstroke}%
\pgfsetstrokeopacity{0.376840}%
\pgfsetdash{}{0pt}%
\pgfpathmoveto{\pgfqpoint{1.642520in}{2.949696in}}%
\pgfpathcurveto{\pgfqpoint{1.650757in}{2.949696in}}{\pgfqpoint{1.658657in}{2.952968in}}{\pgfqpoint{1.664481in}{2.958792in}}%
\pgfpathcurveto{\pgfqpoint{1.670305in}{2.964616in}}{\pgfqpoint{1.673577in}{2.972516in}}{\pgfqpoint{1.673577in}{2.980753in}}%
\pgfpathcurveto{\pgfqpoint{1.673577in}{2.988989in}}{\pgfqpoint{1.670305in}{2.996889in}}{\pgfqpoint{1.664481in}{3.002713in}}%
\pgfpathcurveto{\pgfqpoint{1.658657in}{3.008537in}}{\pgfqpoint{1.650757in}{3.011809in}}{\pgfqpoint{1.642520in}{3.011809in}}%
\pgfpathcurveto{\pgfqpoint{1.634284in}{3.011809in}}{\pgfqpoint{1.626384in}{3.008537in}}{\pgfqpoint{1.620560in}{3.002713in}}%
\pgfpathcurveto{\pgfqpoint{1.614736in}{2.996889in}}{\pgfqpoint{1.611464in}{2.988989in}}{\pgfqpoint{1.611464in}{2.980753in}}%
\pgfpathcurveto{\pgfqpoint{1.611464in}{2.972516in}}{\pgfqpoint{1.614736in}{2.964616in}}{\pgfqpoint{1.620560in}{2.958792in}}%
\pgfpathcurveto{\pgfqpoint{1.626384in}{2.952968in}}{\pgfqpoint{1.634284in}{2.949696in}}{\pgfqpoint{1.642520in}{2.949696in}}%
\pgfpathclose%
\pgfusepath{stroke,fill}%
\end{pgfscope}%
\begin{pgfscope}%
\pgfpathrectangle{\pgfqpoint{0.100000in}{0.212622in}}{\pgfqpoint{3.696000in}{3.696000in}}%
\pgfusepath{clip}%
\pgfsetbuttcap%
\pgfsetroundjoin%
\definecolor{currentfill}{rgb}{0.121569,0.466667,0.705882}%
\pgfsetfillcolor{currentfill}%
\pgfsetfillopacity{0.377235}%
\pgfsetlinewidth{1.003750pt}%
\definecolor{currentstroke}{rgb}{0.121569,0.466667,0.705882}%
\pgfsetstrokecolor{currentstroke}%
\pgfsetstrokeopacity{0.377235}%
\pgfsetdash{}{0pt}%
\pgfpathmoveto{\pgfqpoint{1.641500in}{2.947577in}}%
\pgfpathcurveto{\pgfqpoint{1.649736in}{2.947577in}}{\pgfqpoint{1.657636in}{2.950849in}}{\pgfqpoint{1.663460in}{2.956673in}}%
\pgfpathcurveto{\pgfqpoint{1.669284in}{2.962497in}}{\pgfqpoint{1.672556in}{2.970397in}}{\pgfqpoint{1.672556in}{2.978634in}}%
\pgfpathcurveto{\pgfqpoint{1.672556in}{2.986870in}}{\pgfqpoint{1.669284in}{2.994770in}}{\pgfqpoint{1.663460in}{3.000594in}}%
\pgfpathcurveto{\pgfqpoint{1.657636in}{3.006418in}}{\pgfqpoint{1.649736in}{3.009690in}}{\pgfqpoint{1.641500in}{3.009690in}}%
\pgfpathcurveto{\pgfqpoint{1.633263in}{3.009690in}}{\pgfqpoint{1.625363in}{3.006418in}}{\pgfqpoint{1.619539in}{3.000594in}}%
\pgfpathcurveto{\pgfqpoint{1.613716in}{2.994770in}}{\pgfqpoint{1.610443in}{2.986870in}}{\pgfqpoint{1.610443in}{2.978634in}}%
\pgfpathcurveto{\pgfqpoint{1.610443in}{2.970397in}}{\pgfqpoint{1.613716in}{2.962497in}}{\pgfqpoint{1.619539in}{2.956673in}}%
\pgfpathcurveto{\pgfqpoint{1.625363in}{2.950849in}}{\pgfqpoint{1.633263in}{2.947577in}}{\pgfqpoint{1.641500in}{2.947577in}}%
\pgfpathclose%
\pgfusepath{stroke,fill}%
\end{pgfscope}%
\begin{pgfscope}%
\pgfpathrectangle{\pgfqpoint{0.100000in}{0.212622in}}{\pgfqpoint{3.696000in}{3.696000in}}%
\pgfusepath{clip}%
\pgfsetbuttcap%
\pgfsetroundjoin%
\definecolor{currentfill}{rgb}{0.121569,0.466667,0.705882}%
\pgfsetfillcolor{currentfill}%
\pgfsetfillopacity{0.377947}%
\pgfsetlinewidth{1.003750pt}%
\definecolor{currentstroke}{rgb}{0.121569,0.466667,0.705882}%
\pgfsetstrokecolor{currentstroke}%
\pgfsetstrokeopacity{0.377947}%
\pgfsetdash{}{0pt}%
\pgfpathmoveto{\pgfqpoint{1.639383in}{2.944025in}}%
\pgfpathcurveto{\pgfqpoint{1.647619in}{2.944025in}}{\pgfqpoint{1.655519in}{2.947297in}}{\pgfqpoint{1.661343in}{2.953121in}}%
\pgfpathcurveto{\pgfqpoint{1.667167in}{2.958945in}}{\pgfqpoint{1.670439in}{2.966845in}}{\pgfqpoint{1.670439in}{2.975081in}}%
\pgfpathcurveto{\pgfqpoint{1.670439in}{2.983318in}}{\pgfqpoint{1.667167in}{2.991218in}}{\pgfqpoint{1.661343in}{2.997042in}}%
\pgfpathcurveto{\pgfqpoint{1.655519in}{3.002865in}}{\pgfqpoint{1.647619in}{3.006138in}}{\pgfqpoint{1.639383in}{3.006138in}}%
\pgfpathcurveto{\pgfqpoint{1.631146in}{3.006138in}}{\pgfqpoint{1.623246in}{3.002865in}}{\pgfqpoint{1.617422in}{2.997042in}}%
\pgfpathcurveto{\pgfqpoint{1.611598in}{2.991218in}}{\pgfqpoint{1.608326in}{2.983318in}}{\pgfqpoint{1.608326in}{2.975081in}}%
\pgfpathcurveto{\pgfqpoint{1.608326in}{2.966845in}}{\pgfqpoint{1.611598in}{2.958945in}}{\pgfqpoint{1.617422in}{2.953121in}}%
\pgfpathcurveto{\pgfqpoint{1.623246in}{2.947297in}}{\pgfqpoint{1.631146in}{2.944025in}}{\pgfqpoint{1.639383in}{2.944025in}}%
\pgfpathclose%
\pgfusepath{stroke,fill}%
\end{pgfscope}%
\begin{pgfscope}%
\pgfpathrectangle{\pgfqpoint{0.100000in}{0.212622in}}{\pgfqpoint{3.696000in}{3.696000in}}%
\pgfusepath{clip}%
\pgfsetbuttcap%
\pgfsetroundjoin%
\definecolor{currentfill}{rgb}{0.121569,0.466667,0.705882}%
\pgfsetfillcolor{currentfill}%
\pgfsetfillopacity{0.379201}%
\pgfsetlinewidth{1.003750pt}%
\definecolor{currentstroke}{rgb}{0.121569,0.466667,0.705882}%
\pgfsetstrokecolor{currentstroke}%
\pgfsetstrokeopacity{0.379201}%
\pgfsetdash{}{0pt}%
\pgfpathmoveto{\pgfqpoint{1.635444in}{2.937532in}}%
\pgfpathcurveto{\pgfqpoint{1.643680in}{2.937532in}}{\pgfqpoint{1.651580in}{2.940804in}}{\pgfqpoint{1.657404in}{2.946628in}}%
\pgfpathcurveto{\pgfqpoint{1.663228in}{2.952452in}}{\pgfqpoint{1.666501in}{2.960352in}}{\pgfqpoint{1.666501in}{2.968588in}}%
\pgfpathcurveto{\pgfqpoint{1.666501in}{2.976825in}}{\pgfqpoint{1.663228in}{2.984725in}}{\pgfqpoint{1.657404in}{2.990549in}}%
\pgfpathcurveto{\pgfqpoint{1.651580in}{2.996373in}}{\pgfqpoint{1.643680in}{2.999645in}}{\pgfqpoint{1.635444in}{2.999645in}}%
\pgfpathcurveto{\pgfqpoint{1.627208in}{2.999645in}}{\pgfqpoint{1.619308in}{2.996373in}}{\pgfqpoint{1.613484in}{2.990549in}}%
\pgfpathcurveto{\pgfqpoint{1.607660in}{2.984725in}}{\pgfqpoint{1.604388in}{2.976825in}}{\pgfqpoint{1.604388in}{2.968588in}}%
\pgfpathcurveto{\pgfqpoint{1.604388in}{2.960352in}}{\pgfqpoint{1.607660in}{2.952452in}}{\pgfqpoint{1.613484in}{2.946628in}}%
\pgfpathcurveto{\pgfqpoint{1.619308in}{2.940804in}}{\pgfqpoint{1.627208in}{2.937532in}}{\pgfqpoint{1.635444in}{2.937532in}}%
\pgfpathclose%
\pgfusepath{stroke,fill}%
\end{pgfscope}%
\begin{pgfscope}%
\pgfpathrectangle{\pgfqpoint{0.100000in}{0.212622in}}{\pgfqpoint{3.696000in}{3.696000in}}%
\pgfusepath{clip}%
\pgfsetbuttcap%
\pgfsetroundjoin%
\definecolor{currentfill}{rgb}{0.121569,0.466667,0.705882}%
\pgfsetfillcolor{currentfill}%
\pgfsetfillopacity{0.379252}%
\pgfsetlinewidth{1.003750pt}%
\definecolor{currentstroke}{rgb}{0.121569,0.466667,0.705882}%
\pgfsetstrokecolor{currentstroke}%
\pgfsetstrokeopacity{0.379252}%
\pgfsetdash{}{0pt}%
\pgfpathmoveto{\pgfqpoint{1.964517in}{3.004198in}}%
\pgfpathcurveto{\pgfqpoint{1.972753in}{3.004198in}}{\pgfqpoint{1.980653in}{3.007470in}}{\pgfqpoint{1.986477in}{3.013294in}}%
\pgfpathcurveto{\pgfqpoint{1.992301in}{3.019118in}}{\pgfqpoint{1.995573in}{3.027018in}}{\pgfqpoint{1.995573in}{3.035254in}}%
\pgfpathcurveto{\pgfqpoint{1.995573in}{3.043490in}}{\pgfqpoint{1.992301in}{3.051390in}}{\pgfqpoint{1.986477in}{3.057214in}}%
\pgfpathcurveto{\pgfqpoint{1.980653in}{3.063038in}}{\pgfqpoint{1.972753in}{3.066311in}}{\pgfqpoint{1.964517in}{3.066311in}}%
\pgfpathcurveto{\pgfqpoint{1.956281in}{3.066311in}}{\pgfqpoint{1.948381in}{3.063038in}}{\pgfqpoint{1.942557in}{3.057214in}}%
\pgfpathcurveto{\pgfqpoint{1.936733in}{3.051390in}}{\pgfqpoint{1.933460in}{3.043490in}}{\pgfqpoint{1.933460in}{3.035254in}}%
\pgfpathcurveto{\pgfqpoint{1.933460in}{3.027018in}}{\pgfqpoint{1.936733in}{3.019118in}}{\pgfqpoint{1.942557in}{3.013294in}}%
\pgfpathcurveto{\pgfqpoint{1.948381in}{3.007470in}}{\pgfqpoint{1.956281in}{3.004198in}}{\pgfqpoint{1.964517in}{3.004198in}}%
\pgfpathclose%
\pgfusepath{stroke,fill}%
\end{pgfscope}%
\begin{pgfscope}%
\pgfpathrectangle{\pgfqpoint{0.100000in}{0.212622in}}{\pgfqpoint{3.696000in}{3.696000in}}%
\pgfusepath{clip}%
\pgfsetbuttcap%
\pgfsetroundjoin%
\definecolor{currentfill}{rgb}{0.121569,0.466667,0.705882}%
\pgfsetfillcolor{currentfill}%
\pgfsetfillopacity{0.380377}%
\pgfsetlinewidth{1.003750pt}%
\definecolor{currentstroke}{rgb}{0.121569,0.466667,0.705882}%
\pgfsetstrokecolor{currentstroke}%
\pgfsetstrokeopacity{0.380377}%
\pgfsetdash{}{0pt}%
\pgfpathmoveto{\pgfqpoint{1.632464in}{2.931495in}}%
\pgfpathcurveto{\pgfqpoint{1.640700in}{2.931495in}}{\pgfqpoint{1.648600in}{2.934767in}}{\pgfqpoint{1.654424in}{2.940591in}}%
\pgfpathcurveto{\pgfqpoint{1.660248in}{2.946415in}}{\pgfqpoint{1.663520in}{2.954315in}}{\pgfqpoint{1.663520in}{2.962552in}}%
\pgfpathcurveto{\pgfqpoint{1.663520in}{2.970788in}}{\pgfqpoint{1.660248in}{2.978688in}}{\pgfqpoint{1.654424in}{2.984512in}}%
\pgfpathcurveto{\pgfqpoint{1.648600in}{2.990336in}}{\pgfqpoint{1.640700in}{2.993608in}}{\pgfqpoint{1.632464in}{2.993608in}}%
\pgfpathcurveto{\pgfqpoint{1.624227in}{2.993608in}}{\pgfqpoint{1.616327in}{2.990336in}}{\pgfqpoint{1.610503in}{2.984512in}}%
\pgfpathcurveto{\pgfqpoint{1.604679in}{2.978688in}}{\pgfqpoint{1.601407in}{2.970788in}}{\pgfqpoint{1.601407in}{2.962552in}}%
\pgfpathcurveto{\pgfqpoint{1.601407in}{2.954315in}}{\pgfqpoint{1.604679in}{2.946415in}}{\pgfqpoint{1.610503in}{2.940591in}}%
\pgfpathcurveto{\pgfqpoint{1.616327in}{2.934767in}}{\pgfqpoint{1.624227in}{2.931495in}}{\pgfqpoint{1.632464in}{2.931495in}}%
\pgfpathclose%
\pgfusepath{stroke,fill}%
\end{pgfscope}%
\begin{pgfscope}%
\pgfpathrectangle{\pgfqpoint{0.100000in}{0.212622in}}{\pgfqpoint{3.696000in}{3.696000in}}%
\pgfusepath{clip}%
\pgfsetbuttcap%
\pgfsetroundjoin%
\definecolor{currentfill}{rgb}{0.121569,0.466667,0.705882}%
\pgfsetfillcolor{currentfill}%
\pgfsetfillopacity{0.380695}%
\pgfsetlinewidth{1.003750pt}%
\definecolor{currentstroke}{rgb}{0.121569,0.466667,0.705882}%
\pgfsetstrokecolor{currentstroke}%
\pgfsetstrokeopacity{0.380695}%
\pgfsetdash{}{0pt}%
\pgfpathmoveto{\pgfqpoint{1.965630in}{2.998785in}}%
\pgfpathcurveto{\pgfqpoint{1.973866in}{2.998785in}}{\pgfqpoint{1.981766in}{3.002057in}}{\pgfqpoint{1.987590in}{3.007881in}}%
\pgfpathcurveto{\pgfqpoint{1.993414in}{3.013705in}}{\pgfqpoint{1.996686in}{3.021605in}}{\pgfqpoint{1.996686in}{3.029841in}}%
\pgfpathcurveto{\pgfqpoint{1.996686in}{3.038078in}}{\pgfqpoint{1.993414in}{3.045978in}}{\pgfqpoint{1.987590in}{3.051802in}}%
\pgfpathcurveto{\pgfqpoint{1.981766in}{3.057626in}}{\pgfqpoint{1.973866in}{3.060898in}}{\pgfqpoint{1.965630in}{3.060898in}}%
\pgfpathcurveto{\pgfqpoint{1.957394in}{3.060898in}}{\pgfqpoint{1.949494in}{3.057626in}}{\pgfqpoint{1.943670in}{3.051802in}}%
\pgfpathcurveto{\pgfqpoint{1.937846in}{3.045978in}}{\pgfqpoint{1.934573in}{3.038078in}}{\pgfqpoint{1.934573in}{3.029841in}}%
\pgfpathcurveto{\pgfqpoint{1.934573in}{3.021605in}}{\pgfqpoint{1.937846in}{3.013705in}}{\pgfqpoint{1.943670in}{3.007881in}}%
\pgfpathcurveto{\pgfqpoint{1.949494in}{3.002057in}}{\pgfqpoint{1.957394in}{2.998785in}}{\pgfqpoint{1.965630in}{2.998785in}}%
\pgfpathclose%
\pgfusepath{stroke,fill}%
\end{pgfscope}%
\begin{pgfscope}%
\pgfpathrectangle{\pgfqpoint{0.100000in}{0.212622in}}{\pgfqpoint{3.696000in}{3.696000in}}%
\pgfusepath{clip}%
\pgfsetbuttcap%
\pgfsetroundjoin%
\definecolor{currentfill}{rgb}{0.121569,0.466667,0.705882}%
\pgfsetfillcolor{currentfill}%
\pgfsetfillopacity{0.380929}%
\pgfsetlinewidth{1.003750pt}%
\definecolor{currentstroke}{rgb}{0.121569,0.466667,0.705882}%
\pgfsetstrokecolor{currentstroke}%
\pgfsetstrokeopacity{0.380929}%
\pgfsetdash{}{0pt}%
\pgfpathmoveto{\pgfqpoint{1.630707in}{2.928735in}}%
\pgfpathcurveto{\pgfqpoint{1.638943in}{2.928735in}}{\pgfqpoint{1.646843in}{2.932007in}}{\pgfqpoint{1.652667in}{2.937831in}}%
\pgfpathcurveto{\pgfqpoint{1.658491in}{2.943655in}}{\pgfqpoint{1.661763in}{2.951555in}}{\pgfqpoint{1.661763in}{2.959792in}}%
\pgfpathcurveto{\pgfqpoint{1.661763in}{2.968028in}}{\pgfqpoint{1.658491in}{2.975928in}}{\pgfqpoint{1.652667in}{2.981752in}}%
\pgfpathcurveto{\pgfqpoint{1.646843in}{2.987576in}}{\pgfqpoint{1.638943in}{2.990848in}}{\pgfqpoint{1.630707in}{2.990848in}}%
\pgfpathcurveto{\pgfqpoint{1.622470in}{2.990848in}}{\pgfqpoint{1.614570in}{2.987576in}}{\pgfqpoint{1.608746in}{2.981752in}}%
\pgfpathcurveto{\pgfqpoint{1.602922in}{2.975928in}}{\pgfqpoint{1.599650in}{2.968028in}}{\pgfqpoint{1.599650in}{2.959792in}}%
\pgfpathcurveto{\pgfqpoint{1.599650in}{2.951555in}}{\pgfqpoint{1.602922in}{2.943655in}}{\pgfqpoint{1.608746in}{2.937831in}}%
\pgfpathcurveto{\pgfqpoint{1.614570in}{2.932007in}}{\pgfqpoint{1.622470in}{2.928735in}}{\pgfqpoint{1.630707in}{2.928735in}}%
\pgfpathclose%
\pgfusepath{stroke,fill}%
\end{pgfscope}%
\begin{pgfscope}%
\pgfpathrectangle{\pgfqpoint{0.100000in}{0.212622in}}{\pgfqpoint{3.696000in}{3.696000in}}%
\pgfusepath{clip}%
\pgfsetbuttcap%
\pgfsetroundjoin%
\definecolor{currentfill}{rgb}{0.121569,0.466667,0.705882}%
\pgfsetfillcolor{currentfill}%
\pgfsetfillopacity{0.381438}%
\pgfsetlinewidth{1.003750pt}%
\definecolor{currentstroke}{rgb}{0.121569,0.466667,0.705882}%
\pgfsetstrokecolor{currentstroke}%
\pgfsetstrokeopacity{0.381438}%
\pgfsetdash{}{0pt}%
\pgfpathmoveto{\pgfqpoint{1.629373in}{2.926254in}}%
\pgfpathcurveto{\pgfqpoint{1.637610in}{2.926254in}}{\pgfqpoint{1.645510in}{2.929526in}}{\pgfqpoint{1.651334in}{2.935350in}}%
\pgfpathcurveto{\pgfqpoint{1.657157in}{2.941174in}}{\pgfqpoint{1.660430in}{2.949074in}}{\pgfqpoint{1.660430in}{2.957310in}}%
\pgfpathcurveto{\pgfqpoint{1.660430in}{2.965547in}}{\pgfqpoint{1.657157in}{2.973447in}}{\pgfqpoint{1.651334in}{2.979271in}}%
\pgfpathcurveto{\pgfqpoint{1.645510in}{2.985094in}}{\pgfqpoint{1.637610in}{2.988367in}}{\pgfqpoint{1.629373in}{2.988367in}}%
\pgfpathcurveto{\pgfqpoint{1.621137in}{2.988367in}}{\pgfqpoint{1.613237in}{2.985094in}}{\pgfqpoint{1.607413in}{2.979271in}}%
\pgfpathcurveto{\pgfqpoint{1.601589in}{2.973447in}}{\pgfqpoint{1.598317in}{2.965547in}}{\pgfqpoint{1.598317in}{2.957310in}}%
\pgfpathcurveto{\pgfqpoint{1.598317in}{2.949074in}}{\pgfqpoint{1.601589in}{2.941174in}}{\pgfqpoint{1.607413in}{2.935350in}}%
\pgfpathcurveto{\pgfqpoint{1.613237in}{2.929526in}}{\pgfqpoint{1.621137in}{2.926254in}}{\pgfqpoint{1.629373in}{2.926254in}}%
\pgfpathclose%
\pgfusepath{stroke,fill}%
\end{pgfscope}%
\begin{pgfscope}%
\pgfpathrectangle{\pgfqpoint{0.100000in}{0.212622in}}{\pgfqpoint{3.696000in}{3.696000in}}%
\pgfusepath{clip}%
\pgfsetbuttcap%
\pgfsetroundjoin%
\definecolor{currentfill}{rgb}{0.121569,0.466667,0.705882}%
\pgfsetfillcolor{currentfill}%
\pgfsetfillopacity{0.381574}%
\pgfsetlinewidth{1.003750pt}%
\definecolor{currentstroke}{rgb}{0.121569,0.466667,0.705882}%
\pgfsetstrokecolor{currentstroke}%
\pgfsetstrokeopacity{0.381574}%
\pgfsetdash{}{0pt}%
\pgfpathmoveto{\pgfqpoint{1.966025in}{2.995957in}}%
\pgfpathcurveto{\pgfqpoint{1.974261in}{2.995957in}}{\pgfqpoint{1.982161in}{2.999230in}}{\pgfqpoint{1.987985in}{3.005054in}}%
\pgfpathcurveto{\pgfqpoint{1.993809in}{3.010878in}}{\pgfqpoint{1.997081in}{3.018778in}}{\pgfqpoint{1.997081in}{3.027014in}}%
\pgfpathcurveto{\pgfqpoint{1.997081in}{3.035250in}}{\pgfqpoint{1.993809in}{3.043150in}}{\pgfqpoint{1.987985in}{3.048974in}}%
\pgfpathcurveto{\pgfqpoint{1.982161in}{3.054798in}}{\pgfqpoint{1.974261in}{3.058070in}}{\pgfqpoint{1.966025in}{3.058070in}}%
\pgfpathcurveto{\pgfqpoint{1.957788in}{3.058070in}}{\pgfqpoint{1.949888in}{3.054798in}}{\pgfqpoint{1.944064in}{3.048974in}}%
\pgfpathcurveto{\pgfqpoint{1.938241in}{3.043150in}}{\pgfqpoint{1.934968in}{3.035250in}}{\pgfqpoint{1.934968in}{3.027014in}}%
\pgfpathcurveto{\pgfqpoint{1.934968in}{3.018778in}}{\pgfqpoint{1.938241in}{3.010878in}}{\pgfqpoint{1.944064in}{3.005054in}}%
\pgfpathcurveto{\pgfqpoint{1.949888in}{2.999230in}}{\pgfqpoint{1.957788in}{2.995957in}}{\pgfqpoint{1.966025in}{2.995957in}}%
\pgfpathclose%
\pgfusepath{stroke,fill}%
\end{pgfscope}%
\begin{pgfscope}%
\pgfpathrectangle{\pgfqpoint{0.100000in}{0.212622in}}{\pgfqpoint{3.696000in}{3.696000in}}%
\pgfusepath{clip}%
\pgfsetbuttcap%
\pgfsetroundjoin%
\definecolor{currentfill}{rgb}{0.121569,0.466667,0.705882}%
\pgfsetfillcolor{currentfill}%
\pgfsetfillopacity{0.382379}%
\pgfsetlinewidth{1.003750pt}%
\definecolor{currentstroke}{rgb}{0.121569,0.466667,0.705882}%
\pgfsetstrokecolor{currentstroke}%
\pgfsetstrokeopacity{0.382379}%
\pgfsetdash{}{0pt}%
\pgfpathmoveto{\pgfqpoint{1.626987in}{2.921756in}}%
\pgfpathcurveto{\pgfqpoint{1.635224in}{2.921756in}}{\pgfqpoint{1.643124in}{2.925028in}}{\pgfqpoint{1.648948in}{2.930852in}}%
\pgfpathcurveto{\pgfqpoint{1.654771in}{2.936676in}}{\pgfqpoint{1.658044in}{2.944576in}}{\pgfqpoint{1.658044in}{2.952812in}}%
\pgfpathcurveto{\pgfqpoint{1.658044in}{2.961049in}}{\pgfqpoint{1.654771in}{2.968949in}}{\pgfqpoint{1.648948in}{2.974773in}}%
\pgfpathcurveto{\pgfqpoint{1.643124in}{2.980596in}}{\pgfqpoint{1.635224in}{2.983869in}}{\pgfqpoint{1.626987in}{2.983869in}}%
\pgfpathcurveto{\pgfqpoint{1.618751in}{2.983869in}}{\pgfqpoint{1.610851in}{2.980596in}}{\pgfqpoint{1.605027in}{2.974773in}}%
\pgfpathcurveto{\pgfqpoint{1.599203in}{2.968949in}}{\pgfqpoint{1.595931in}{2.961049in}}{\pgfqpoint{1.595931in}{2.952812in}}%
\pgfpathcurveto{\pgfqpoint{1.595931in}{2.944576in}}{\pgfqpoint{1.599203in}{2.936676in}}{\pgfqpoint{1.605027in}{2.930852in}}%
\pgfpathcurveto{\pgfqpoint{1.610851in}{2.925028in}}{\pgfqpoint{1.618751in}{2.921756in}}{\pgfqpoint{1.626987in}{2.921756in}}%
\pgfpathclose%
\pgfusepath{stroke,fill}%
\end{pgfscope}%
\begin{pgfscope}%
\pgfpathrectangle{\pgfqpoint{0.100000in}{0.212622in}}{\pgfqpoint{3.696000in}{3.696000in}}%
\pgfusepath{clip}%
\pgfsetbuttcap%
\pgfsetroundjoin%
\definecolor{currentfill}{rgb}{0.121569,0.466667,0.705882}%
\pgfsetfillcolor{currentfill}%
\pgfsetfillopacity{0.382827}%
\pgfsetlinewidth{1.003750pt}%
\definecolor{currentstroke}{rgb}{0.121569,0.466667,0.705882}%
\pgfsetstrokecolor{currentstroke}%
\pgfsetstrokeopacity{0.382827}%
\pgfsetdash{}{0pt}%
\pgfpathmoveto{\pgfqpoint{1.966944in}{2.991470in}}%
\pgfpathcurveto{\pgfqpoint{1.975180in}{2.991470in}}{\pgfqpoint{1.983080in}{2.994742in}}{\pgfqpoint{1.988904in}{3.000566in}}%
\pgfpathcurveto{\pgfqpoint{1.994728in}{3.006390in}}{\pgfqpoint{1.998001in}{3.014290in}}{\pgfqpoint{1.998001in}{3.022526in}}%
\pgfpathcurveto{\pgfqpoint{1.998001in}{3.030763in}}{\pgfqpoint{1.994728in}{3.038663in}}{\pgfqpoint{1.988904in}{3.044487in}}%
\pgfpathcurveto{\pgfqpoint{1.983080in}{3.050311in}}{\pgfqpoint{1.975180in}{3.053583in}}{\pgfqpoint{1.966944in}{3.053583in}}%
\pgfpathcurveto{\pgfqpoint{1.958708in}{3.053583in}}{\pgfqpoint{1.950808in}{3.050311in}}{\pgfqpoint{1.944984in}{3.044487in}}%
\pgfpathcurveto{\pgfqpoint{1.939160in}{3.038663in}}{\pgfqpoint{1.935888in}{3.030763in}}{\pgfqpoint{1.935888in}{3.022526in}}%
\pgfpathcurveto{\pgfqpoint{1.935888in}{3.014290in}}{\pgfqpoint{1.939160in}{3.006390in}}{\pgfqpoint{1.944984in}{3.000566in}}%
\pgfpathcurveto{\pgfqpoint{1.950808in}{2.994742in}}{\pgfqpoint{1.958708in}{2.991470in}}{\pgfqpoint{1.966944in}{2.991470in}}%
\pgfpathclose%
\pgfusepath{stroke,fill}%
\end{pgfscope}%
\begin{pgfscope}%
\pgfpathrectangle{\pgfqpoint{0.100000in}{0.212622in}}{\pgfqpoint{3.696000in}{3.696000in}}%
\pgfusepath{clip}%
\pgfsetbuttcap%
\pgfsetroundjoin%
\definecolor{currentfill}{rgb}{0.121569,0.466667,0.705882}%
\pgfsetfillcolor{currentfill}%
\pgfsetfillopacity{0.383030}%
\pgfsetlinewidth{1.003750pt}%
\definecolor{currentstroke}{rgb}{0.121569,0.466667,0.705882}%
\pgfsetstrokecolor{currentstroke}%
\pgfsetstrokeopacity{0.383030}%
\pgfsetdash{}{0pt}%
\pgfpathmoveto{\pgfqpoint{1.624798in}{2.918357in}}%
\pgfpathcurveto{\pgfqpoint{1.633034in}{2.918357in}}{\pgfqpoint{1.640934in}{2.921629in}}{\pgfqpoint{1.646758in}{2.927453in}}%
\pgfpathcurveto{\pgfqpoint{1.652582in}{2.933277in}}{\pgfqpoint{1.655854in}{2.941177in}}{\pgfqpoint{1.655854in}{2.949414in}}%
\pgfpathcurveto{\pgfqpoint{1.655854in}{2.957650in}}{\pgfqpoint{1.652582in}{2.965550in}}{\pgfqpoint{1.646758in}{2.971374in}}%
\pgfpathcurveto{\pgfqpoint{1.640934in}{2.977198in}}{\pgfqpoint{1.633034in}{2.980470in}}{\pgfqpoint{1.624798in}{2.980470in}}%
\pgfpathcurveto{\pgfqpoint{1.616561in}{2.980470in}}{\pgfqpoint{1.608661in}{2.977198in}}{\pgfqpoint{1.602837in}{2.971374in}}%
\pgfpathcurveto{\pgfqpoint{1.597013in}{2.965550in}}{\pgfqpoint{1.593741in}{2.957650in}}{\pgfqpoint{1.593741in}{2.949414in}}%
\pgfpathcurveto{\pgfqpoint{1.593741in}{2.941177in}}{\pgfqpoint{1.597013in}{2.933277in}}{\pgfqpoint{1.602837in}{2.927453in}}%
\pgfpathcurveto{\pgfqpoint{1.608661in}{2.921629in}}{\pgfqpoint{1.616561in}{2.918357in}}{\pgfqpoint{1.624798in}{2.918357in}}%
\pgfpathclose%
\pgfusepath{stroke,fill}%
\end{pgfscope}%
\begin{pgfscope}%
\pgfpathrectangle{\pgfqpoint{0.100000in}{0.212622in}}{\pgfqpoint{3.696000in}{3.696000in}}%
\pgfusepath{clip}%
\pgfsetbuttcap%
\pgfsetroundjoin%
\definecolor{currentfill}{rgb}{0.121569,0.466667,0.705882}%
\pgfsetfillcolor{currentfill}%
\pgfsetfillopacity{0.383343}%
\pgfsetlinewidth{1.003750pt}%
\definecolor{currentstroke}{rgb}{0.121569,0.466667,0.705882}%
\pgfsetstrokecolor{currentstroke}%
\pgfsetstrokeopacity{0.383343}%
\pgfsetdash{}{0pt}%
\pgfpathmoveto{\pgfqpoint{1.623986in}{2.916782in}}%
\pgfpathcurveto{\pgfqpoint{1.632222in}{2.916782in}}{\pgfqpoint{1.640122in}{2.920054in}}{\pgfqpoint{1.645946in}{2.925878in}}%
\pgfpathcurveto{\pgfqpoint{1.651770in}{2.931702in}}{\pgfqpoint{1.655042in}{2.939602in}}{\pgfqpoint{1.655042in}{2.947839in}}%
\pgfpathcurveto{\pgfqpoint{1.655042in}{2.956075in}}{\pgfqpoint{1.651770in}{2.963975in}}{\pgfqpoint{1.645946in}{2.969799in}}%
\pgfpathcurveto{\pgfqpoint{1.640122in}{2.975623in}}{\pgfqpoint{1.632222in}{2.978895in}}{\pgfqpoint{1.623986in}{2.978895in}}%
\pgfpathcurveto{\pgfqpoint{1.615749in}{2.978895in}}{\pgfqpoint{1.607849in}{2.975623in}}{\pgfqpoint{1.602025in}{2.969799in}}%
\pgfpathcurveto{\pgfqpoint{1.596201in}{2.963975in}}{\pgfqpoint{1.592929in}{2.956075in}}{\pgfqpoint{1.592929in}{2.947839in}}%
\pgfpathcurveto{\pgfqpoint{1.592929in}{2.939602in}}{\pgfqpoint{1.596201in}{2.931702in}}{\pgfqpoint{1.602025in}{2.925878in}}%
\pgfpathcurveto{\pgfqpoint{1.607849in}{2.920054in}}{\pgfqpoint{1.615749in}{2.916782in}}{\pgfqpoint{1.623986in}{2.916782in}}%
\pgfpathclose%
\pgfusepath{stroke,fill}%
\end{pgfscope}%
\begin{pgfscope}%
\pgfpathrectangle{\pgfqpoint{0.100000in}{0.212622in}}{\pgfqpoint{3.696000in}{3.696000in}}%
\pgfusepath{clip}%
\pgfsetbuttcap%
\pgfsetroundjoin%
\definecolor{currentfill}{rgb}{0.121569,0.466667,0.705882}%
\pgfsetfillcolor{currentfill}%
\pgfsetfillopacity{0.383587}%
\pgfsetlinewidth{1.003750pt}%
\definecolor{currentstroke}{rgb}{0.121569,0.466667,0.705882}%
\pgfsetstrokecolor{currentstroke}%
\pgfsetstrokeopacity{0.383587}%
\pgfsetdash{}{0pt}%
\pgfpathmoveto{\pgfqpoint{1.623297in}{2.915589in}}%
\pgfpathcurveto{\pgfqpoint{1.631533in}{2.915589in}}{\pgfqpoint{1.639433in}{2.918861in}}{\pgfqpoint{1.645257in}{2.924685in}}%
\pgfpathcurveto{\pgfqpoint{1.651081in}{2.930509in}}{\pgfqpoint{1.654354in}{2.938409in}}{\pgfqpoint{1.654354in}{2.946645in}}%
\pgfpathcurveto{\pgfqpoint{1.654354in}{2.954881in}}{\pgfqpoint{1.651081in}{2.962781in}}{\pgfqpoint{1.645257in}{2.968605in}}%
\pgfpathcurveto{\pgfqpoint{1.639433in}{2.974429in}}{\pgfqpoint{1.631533in}{2.977702in}}{\pgfqpoint{1.623297in}{2.977702in}}%
\pgfpathcurveto{\pgfqpoint{1.615061in}{2.977702in}}{\pgfqpoint{1.607161in}{2.974429in}}{\pgfqpoint{1.601337in}{2.968605in}}%
\pgfpathcurveto{\pgfqpoint{1.595513in}{2.962781in}}{\pgfqpoint{1.592241in}{2.954881in}}{\pgfqpoint{1.592241in}{2.946645in}}%
\pgfpathcurveto{\pgfqpoint{1.592241in}{2.938409in}}{\pgfqpoint{1.595513in}{2.930509in}}{\pgfqpoint{1.601337in}{2.924685in}}%
\pgfpathcurveto{\pgfqpoint{1.607161in}{2.918861in}}{\pgfqpoint{1.615061in}{2.915589in}}{\pgfqpoint{1.623297in}{2.915589in}}%
\pgfpathclose%
\pgfusepath{stroke,fill}%
\end{pgfscope}%
\begin{pgfscope}%
\pgfpathrectangle{\pgfqpoint{0.100000in}{0.212622in}}{\pgfqpoint{3.696000in}{3.696000in}}%
\pgfusepath{clip}%
\pgfsetbuttcap%
\pgfsetroundjoin%
\definecolor{currentfill}{rgb}{0.121569,0.466667,0.705882}%
\pgfsetfillcolor{currentfill}%
\pgfsetfillopacity{0.384002}%
\pgfsetlinewidth{1.003750pt}%
\definecolor{currentstroke}{rgb}{0.121569,0.466667,0.705882}%
\pgfsetstrokecolor{currentstroke}%
\pgfsetstrokeopacity{0.384002}%
\pgfsetdash{}{0pt}%
\pgfpathmoveto{\pgfqpoint{1.621968in}{2.913409in}}%
\pgfpathcurveto{\pgfqpoint{1.630204in}{2.913409in}}{\pgfqpoint{1.638104in}{2.916681in}}{\pgfqpoint{1.643928in}{2.922505in}}%
\pgfpathcurveto{\pgfqpoint{1.649752in}{2.928329in}}{\pgfqpoint{1.653024in}{2.936229in}}{\pgfqpoint{1.653024in}{2.944465in}}%
\pgfpathcurveto{\pgfqpoint{1.653024in}{2.952702in}}{\pgfqpoint{1.649752in}{2.960602in}}{\pgfqpoint{1.643928in}{2.966426in}}%
\pgfpathcurveto{\pgfqpoint{1.638104in}{2.972249in}}{\pgfqpoint{1.630204in}{2.975522in}}{\pgfqpoint{1.621968in}{2.975522in}}%
\pgfpathcurveto{\pgfqpoint{1.613731in}{2.975522in}}{\pgfqpoint{1.605831in}{2.972249in}}{\pgfqpoint{1.600007in}{2.966426in}}%
\pgfpathcurveto{\pgfqpoint{1.594183in}{2.960602in}}{\pgfqpoint{1.590911in}{2.952702in}}{\pgfqpoint{1.590911in}{2.944465in}}%
\pgfpathcurveto{\pgfqpoint{1.590911in}{2.936229in}}{\pgfqpoint{1.594183in}{2.928329in}}{\pgfqpoint{1.600007in}{2.922505in}}%
\pgfpathcurveto{\pgfqpoint{1.605831in}{2.916681in}}{\pgfqpoint{1.613731in}{2.913409in}}{\pgfqpoint{1.621968in}{2.913409in}}%
\pgfpathclose%
\pgfusepath{stroke,fill}%
\end{pgfscope}%
\begin{pgfscope}%
\pgfpathrectangle{\pgfqpoint{0.100000in}{0.212622in}}{\pgfqpoint{3.696000in}{3.696000in}}%
\pgfusepath{clip}%
\pgfsetbuttcap%
\pgfsetroundjoin%
\definecolor{currentfill}{rgb}{0.121569,0.466667,0.705882}%
\pgfsetfillcolor{currentfill}%
\pgfsetfillopacity{0.384197}%
\pgfsetlinewidth{1.003750pt}%
\definecolor{currentstroke}{rgb}{0.121569,0.466667,0.705882}%
\pgfsetstrokecolor{currentstroke}%
\pgfsetstrokeopacity{0.384197}%
\pgfsetdash{}{0pt}%
\pgfpathmoveto{\pgfqpoint{1.968024in}{2.986323in}}%
\pgfpathcurveto{\pgfqpoint{1.976260in}{2.986323in}}{\pgfqpoint{1.984160in}{2.989596in}}{\pgfqpoint{1.989984in}{2.995420in}}%
\pgfpathcurveto{\pgfqpoint{1.995808in}{3.001244in}}{\pgfqpoint{1.999080in}{3.009144in}}{\pgfqpoint{1.999080in}{3.017380in}}%
\pgfpathcurveto{\pgfqpoint{1.999080in}{3.025616in}}{\pgfqpoint{1.995808in}{3.033516in}}{\pgfqpoint{1.989984in}{3.039340in}}%
\pgfpathcurveto{\pgfqpoint{1.984160in}{3.045164in}}{\pgfqpoint{1.976260in}{3.048436in}}{\pgfqpoint{1.968024in}{3.048436in}}%
\pgfpathcurveto{\pgfqpoint{1.959788in}{3.048436in}}{\pgfqpoint{1.951888in}{3.045164in}}{\pgfqpoint{1.946064in}{3.039340in}}%
\pgfpathcurveto{\pgfqpoint{1.940240in}{3.033516in}}{\pgfqpoint{1.936967in}{3.025616in}}{\pgfqpoint{1.936967in}{3.017380in}}%
\pgfpathcurveto{\pgfqpoint{1.936967in}{3.009144in}}{\pgfqpoint{1.940240in}{3.001244in}}{\pgfqpoint{1.946064in}{2.995420in}}%
\pgfpathcurveto{\pgfqpoint{1.951888in}{2.989596in}}{\pgfqpoint{1.959788in}{2.986323in}}{\pgfqpoint{1.968024in}{2.986323in}}%
\pgfpathclose%
\pgfusepath{stroke,fill}%
\end{pgfscope}%
\begin{pgfscope}%
\pgfpathrectangle{\pgfqpoint{0.100000in}{0.212622in}}{\pgfqpoint{3.696000in}{3.696000in}}%
\pgfusepath{clip}%
\pgfsetbuttcap%
\pgfsetroundjoin%
\definecolor{currentfill}{rgb}{0.121569,0.466667,0.705882}%
\pgfsetfillcolor{currentfill}%
\pgfsetfillopacity{0.384785}%
\pgfsetlinewidth{1.003750pt}%
\definecolor{currentstroke}{rgb}{0.121569,0.466667,0.705882}%
\pgfsetstrokecolor{currentstroke}%
\pgfsetstrokeopacity{0.384785}%
\pgfsetdash{}{0pt}%
\pgfpathmoveto{\pgfqpoint{1.619959in}{2.909051in}}%
\pgfpathcurveto{\pgfqpoint{1.628195in}{2.909051in}}{\pgfqpoint{1.636095in}{2.912323in}}{\pgfqpoint{1.641919in}{2.918147in}}%
\pgfpathcurveto{\pgfqpoint{1.647743in}{2.923971in}}{\pgfqpoint{1.651015in}{2.931871in}}{\pgfqpoint{1.651015in}{2.940107in}}%
\pgfpathcurveto{\pgfqpoint{1.651015in}{2.948344in}}{\pgfqpoint{1.647743in}{2.956244in}}{\pgfqpoint{1.641919in}{2.962068in}}%
\pgfpathcurveto{\pgfqpoint{1.636095in}{2.967891in}}{\pgfqpoint{1.628195in}{2.971164in}}{\pgfqpoint{1.619959in}{2.971164in}}%
\pgfpathcurveto{\pgfqpoint{1.611723in}{2.971164in}}{\pgfqpoint{1.603823in}{2.967891in}}{\pgfqpoint{1.597999in}{2.962068in}}%
\pgfpathcurveto{\pgfqpoint{1.592175in}{2.956244in}}{\pgfqpoint{1.588902in}{2.948344in}}{\pgfqpoint{1.588902in}{2.940107in}}%
\pgfpathcurveto{\pgfqpoint{1.588902in}{2.931871in}}{\pgfqpoint{1.592175in}{2.923971in}}{\pgfqpoint{1.597999in}{2.918147in}}%
\pgfpathcurveto{\pgfqpoint{1.603823in}{2.912323in}}{\pgfqpoint{1.611723in}{2.909051in}}{\pgfqpoint{1.619959in}{2.909051in}}%
\pgfpathclose%
\pgfusepath{stroke,fill}%
\end{pgfscope}%
\begin{pgfscope}%
\pgfpathrectangle{\pgfqpoint{0.100000in}{0.212622in}}{\pgfqpoint{3.696000in}{3.696000in}}%
\pgfusepath{clip}%
\pgfsetbuttcap%
\pgfsetroundjoin%
\definecolor{currentfill}{rgb}{0.121569,0.466667,0.705882}%
\pgfsetfillcolor{currentfill}%
\pgfsetfillopacity{0.384940}%
\pgfsetlinewidth{1.003750pt}%
\definecolor{currentstroke}{rgb}{0.121569,0.466667,0.705882}%
\pgfsetstrokecolor{currentstroke}%
\pgfsetstrokeopacity{0.384940}%
\pgfsetdash{}{0pt}%
\pgfpathmoveto{\pgfqpoint{1.619428in}{2.908204in}}%
\pgfpathcurveto{\pgfqpoint{1.627664in}{2.908204in}}{\pgfqpoint{1.635565in}{2.911476in}}{\pgfqpoint{1.641388in}{2.917300in}}%
\pgfpathcurveto{\pgfqpoint{1.647212in}{2.923124in}}{\pgfqpoint{1.650485in}{2.931024in}}{\pgfqpoint{1.650485in}{2.939260in}}%
\pgfpathcurveto{\pgfqpoint{1.650485in}{2.947496in}}{\pgfqpoint{1.647212in}{2.955397in}}{\pgfqpoint{1.641388in}{2.961220in}}%
\pgfpathcurveto{\pgfqpoint{1.635565in}{2.967044in}}{\pgfqpoint{1.627664in}{2.970317in}}{\pgfqpoint{1.619428in}{2.970317in}}%
\pgfpathcurveto{\pgfqpoint{1.611192in}{2.970317in}}{\pgfqpoint{1.603292in}{2.967044in}}{\pgfqpoint{1.597468in}{2.961220in}}%
\pgfpathcurveto{\pgfqpoint{1.591644in}{2.955397in}}{\pgfqpoint{1.588372in}{2.947496in}}{\pgfqpoint{1.588372in}{2.939260in}}%
\pgfpathcurveto{\pgfqpoint{1.588372in}{2.931024in}}{\pgfqpoint{1.591644in}{2.923124in}}{\pgfqpoint{1.597468in}{2.917300in}}%
\pgfpathcurveto{\pgfqpoint{1.603292in}{2.911476in}}{\pgfqpoint{1.611192in}{2.908204in}}{\pgfqpoint{1.619428in}{2.908204in}}%
\pgfpathclose%
\pgfusepath{stroke,fill}%
\end{pgfscope}%
\begin{pgfscope}%
\pgfpathrectangle{\pgfqpoint{0.100000in}{0.212622in}}{\pgfqpoint{3.696000in}{3.696000in}}%
\pgfusepath{clip}%
\pgfsetbuttcap%
\pgfsetroundjoin%
\definecolor{currentfill}{rgb}{0.121569,0.466667,0.705882}%
\pgfsetfillcolor{currentfill}%
\pgfsetfillopacity{0.385035}%
\pgfsetlinewidth{1.003750pt}%
\definecolor{currentstroke}{rgb}{0.121569,0.466667,0.705882}%
\pgfsetstrokecolor{currentstroke}%
\pgfsetstrokeopacity{0.385035}%
\pgfsetdash{}{0pt}%
\pgfpathmoveto{\pgfqpoint{1.968340in}{2.983587in}}%
\pgfpathcurveto{\pgfqpoint{1.976576in}{2.983587in}}{\pgfqpoint{1.984476in}{2.986860in}}{\pgfqpoint{1.990300in}{2.992684in}}%
\pgfpathcurveto{\pgfqpoint{1.996124in}{2.998508in}}{\pgfqpoint{1.999396in}{3.006408in}}{\pgfqpoint{1.999396in}{3.014644in}}%
\pgfpathcurveto{\pgfqpoint{1.999396in}{3.022880in}}{\pgfqpoint{1.996124in}{3.030780in}}{\pgfqpoint{1.990300in}{3.036604in}}%
\pgfpathcurveto{\pgfqpoint{1.984476in}{3.042428in}}{\pgfqpoint{1.976576in}{3.045700in}}{\pgfqpoint{1.968340in}{3.045700in}}%
\pgfpathcurveto{\pgfqpoint{1.960103in}{3.045700in}}{\pgfqpoint{1.952203in}{3.042428in}}{\pgfqpoint{1.946379in}{3.036604in}}%
\pgfpathcurveto{\pgfqpoint{1.940555in}{3.030780in}}{\pgfqpoint{1.937283in}{3.022880in}}{\pgfqpoint{1.937283in}{3.014644in}}%
\pgfpathcurveto{\pgfqpoint{1.937283in}{3.006408in}}{\pgfqpoint{1.940555in}{2.998508in}}{\pgfqpoint{1.946379in}{2.992684in}}%
\pgfpathcurveto{\pgfqpoint{1.952203in}{2.986860in}}{\pgfqpoint{1.960103in}{2.983587in}}{\pgfqpoint{1.968340in}{2.983587in}}%
\pgfpathclose%
\pgfusepath{stroke,fill}%
\end{pgfscope}%
\begin{pgfscope}%
\pgfpathrectangle{\pgfqpoint{0.100000in}{0.212622in}}{\pgfqpoint{3.696000in}{3.696000in}}%
\pgfusepath{clip}%
\pgfsetbuttcap%
\pgfsetroundjoin%
\definecolor{currentfill}{rgb}{0.121569,0.466667,0.705882}%
\pgfsetfillcolor{currentfill}%
\pgfsetfillopacity{0.385250}%
\pgfsetlinewidth{1.003750pt}%
\definecolor{currentstroke}{rgb}{0.121569,0.466667,0.705882}%
\pgfsetstrokecolor{currentstroke}%
\pgfsetstrokeopacity{0.385250}%
\pgfsetdash{}{0pt}%
\pgfpathmoveto{\pgfqpoint{1.618591in}{2.906599in}}%
\pgfpathcurveto{\pgfqpoint{1.626828in}{2.906599in}}{\pgfqpoint{1.634728in}{2.909872in}}{\pgfqpoint{1.640552in}{2.915696in}}%
\pgfpathcurveto{\pgfqpoint{1.646376in}{2.921520in}}{\pgfqpoint{1.649648in}{2.929420in}}{\pgfqpoint{1.649648in}{2.937656in}}%
\pgfpathcurveto{\pgfqpoint{1.649648in}{2.945892in}}{\pgfqpoint{1.646376in}{2.953792in}}{\pgfqpoint{1.640552in}{2.959616in}}%
\pgfpathcurveto{\pgfqpoint{1.634728in}{2.965440in}}{\pgfqpoint{1.626828in}{2.968712in}}{\pgfqpoint{1.618591in}{2.968712in}}%
\pgfpathcurveto{\pgfqpoint{1.610355in}{2.968712in}}{\pgfqpoint{1.602455in}{2.965440in}}{\pgfqpoint{1.596631in}{2.959616in}}%
\pgfpathcurveto{\pgfqpoint{1.590807in}{2.953792in}}{\pgfqpoint{1.587535in}{2.945892in}}{\pgfqpoint{1.587535in}{2.937656in}}%
\pgfpathcurveto{\pgfqpoint{1.587535in}{2.929420in}}{\pgfqpoint{1.590807in}{2.921520in}}{\pgfqpoint{1.596631in}{2.915696in}}%
\pgfpathcurveto{\pgfqpoint{1.602455in}{2.909872in}}{\pgfqpoint{1.610355in}{2.906599in}}{\pgfqpoint{1.618591in}{2.906599in}}%
\pgfpathclose%
\pgfusepath{stroke,fill}%
\end{pgfscope}%
\begin{pgfscope}%
\pgfpathrectangle{\pgfqpoint{0.100000in}{0.212622in}}{\pgfqpoint{3.696000in}{3.696000in}}%
\pgfusepath{clip}%
\pgfsetbuttcap%
\pgfsetroundjoin%
\definecolor{currentfill}{rgb}{0.121569,0.466667,0.705882}%
\pgfsetfillcolor{currentfill}%
\pgfsetfillopacity{0.385806}%
\pgfsetlinewidth{1.003750pt}%
\definecolor{currentstroke}{rgb}{0.121569,0.466667,0.705882}%
\pgfsetstrokecolor{currentstroke}%
\pgfsetstrokeopacity{0.385806}%
\pgfsetdash{}{0pt}%
\pgfpathmoveto{\pgfqpoint{1.616982in}{2.903752in}}%
\pgfpathcurveto{\pgfqpoint{1.625219in}{2.903752in}}{\pgfqpoint{1.633119in}{2.907025in}}{\pgfqpoint{1.638943in}{2.912849in}}%
\pgfpathcurveto{\pgfqpoint{1.644767in}{2.918673in}}{\pgfqpoint{1.648039in}{2.926573in}}{\pgfqpoint{1.648039in}{2.934809in}}%
\pgfpathcurveto{\pgfqpoint{1.648039in}{2.943045in}}{\pgfqpoint{1.644767in}{2.950945in}}{\pgfqpoint{1.638943in}{2.956769in}}%
\pgfpathcurveto{\pgfqpoint{1.633119in}{2.962593in}}{\pgfqpoint{1.625219in}{2.965865in}}{\pgfqpoint{1.616982in}{2.965865in}}%
\pgfpathcurveto{\pgfqpoint{1.608746in}{2.965865in}}{\pgfqpoint{1.600846in}{2.962593in}}{\pgfqpoint{1.595022in}{2.956769in}}%
\pgfpathcurveto{\pgfqpoint{1.589198in}{2.950945in}}{\pgfqpoint{1.585926in}{2.943045in}}{\pgfqpoint{1.585926in}{2.934809in}}%
\pgfpathcurveto{\pgfqpoint{1.585926in}{2.926573in}}{\pgfqpoint{1.589198in}{2.918673in}}{\pgfqpoint{1.595022in}{2.912849in}}%
\pgfpathcurveto{\pgfqpoint{1.600846in}{2.907025in}}{\pgfqpoint{1.608746in}{2.903752in}}{\pgfqpoint{1.616982in}{2.903752in}}%
\pgfpathclose%
\pgfusepath{stroke,fill}%
\end{pgfscope}%
\begin{pgfscope}%
\pgfpathrectangle{\pgfqpoint{0.100000in}{0.212622in}}{\pgfqpoint{3.696000in}{3.696000in}}%
\pgfusepath{clip}%
\pgfsetbuttcap%
\pgfsetroundjoin%
\definecolor{currentfill}{rgb}{0.121569,0.466667,0.705882}%
\pgfsetfillcolor{currentfill}%
\pgfsetfillopacity{0.386250}%
\pgfsetlinewidth{1.003750pt}%
\definecolor{currentstroke}{rgb}{0.121569,0.466667,0.705882}%
\pgfsetstrokecolor{currentstroke}%
\pgfsetstrokeopacity{0.386250}%
\pgfsetdash{}{0pt}%
\pgfpathmoveto{\pgfqpoint{1.969201in}{2.979029in}}%
\pgfpathcurveto{\pgfqpoint{1.977437in}{2.979029in}}{\pgfqpoint{1.985337in}{2.982301in}}{\pgfqpoint{1.991161in}{2.988125in}}%
\pgfpathcurveto{\pgfqpoint{1.996985in}{2.993949in}}{\pgfqpoint{2.000258in}{3.001849in}}{\pgfqpoint{2.000258in}{3.010085in}}%
\pgfpathcurveto{\pgfqpoint{2.000258in}{3.018321in}}{\pgfqpoint{1.996985in}{3.026222in}}{\pgfqpoint{1.991161in}{3.032045in}}%
\pgfpathcurveto{\pgfqpoint{1.985337in}{3.037869in}}{\pgfqpoint{1.977437in}{3.041142in}}{\pgfqpoint{1.969201in}{3.041142in}}%
\pgfpathcurveto{\pgfqpoint{1.960965in}{3.041142in}}{\pgfqpoint{1.953065in}{3.037869in}}{\pgfqpoint{1.947241in}{3.032045in}}%
\pgfpathcurveto{\pgfqpoint{1.941417in}{3.026222in}}{\pgfqpoint{1.938145in}{3.018321in}}{\pgfqpoint{1.938145in}{3.010085in}}%
\pgfpathcurveto{\pgfqpoint{1.938145in}{3.001849in}}{\pgfqpoint{1.941417in}{2.993949in}}{\pgfqpoint{1.947241in}{2.988125in}}%
\pgfpathcurveto{\pgfqpoint{1.953065in}{2.982301in}}{\pgfqpoint{1.960965in}{2.979029in}}{\pgfqpoint{1.969201in}{2.979029in}}%
\pgfpathclose%
\pgfusepath{stroke,fill}%
\end{pgfscope}%
\begin{pgfscope}%
\pgfpathrectangle{\pgfqpoint{0.100000in}{0.212622in}}{\pgfqpoint{3.696000in}{3.696000in}}%
\pgfusepath{clip}%
\pgfsetbuttcap%
\pgfsetroundjoin%
\definecolor{currentfill}{rgb}{0.121569,0.466667,0.705882}%
\pgfsetfillcolor{currentfill}%
\pgfsetfillopacity{0.386712}%
\pgfsetlinewidth{1.003750pt}%
\definecolor{currentstroke}{rgb}{0.121569,0.466667,0.705882}%
\pgfsetstrokecolor{currentstroke}%
\pgfsetstrokeopacity{0.386712}%
\pgfsetdash{}{0pt}%
\pgfpathmoveto{\pgfqpoint{1.613800in}{2.898545in}}%
\pgfpathcurveto{\pgfqpoint{1.622036in}{2.898545in}}{\pgfqpoint{1.629936in}{2.901818in}}{\pgfqpoint{1.635760in}{2.907642in}}%
\pgfpathcurveto{\pgfqpoint{1.641584in}{2.913465in}}{\pgfqpoint{1.644856in}{2.921366in}}{\pgfqpoint{1.644856in}{2.929602in}}%
\pgfpathcurveto{\pgfqpoint{1.644856in}{2.937838in}}{\pgfqpoint{1.641584in}{2.945738in}}{\pgfqpoint{1.635760in}{2.951562in}}%
\pgfpathcurveto{\pgfqpoint{1.629936in}{2.957386in}}{\pgfqpoint{1.622036in}{2.960658in}}{\pgfqpoint{1.613800in}{2.960658in}}%
\pgfpathcurveto{\pgfqpoint{1.605563in}{2.960658in}}{\pgfqpoint{1.597663in}{2.957386in}}{\pgfqpoint{1.591839in}{2.951562in}}%
\pgfpathcurveto{\pgfqpoint{1.586015in}{2.945738in}}{\pgfqpoint{1.582743in}{2.937838in}}{\pgfqpoint{1.582743in}{2.929602in}}%
\pgfpathcurveto{\pgfqpoint{1.582743in}{2.921366in}}{\pgfqpoint{1.586015in}{2.913465in}}{\pgfqpoint{1.591839in}{2.907642in}}%
\pgfpathcurveto{\pgfqpoint{1.597663in}{2.901818in}}{\pgfqpoint{1.605563in}{2.898545in}}{\pgfqpoint{1.613800in}{2.898545in}}%
\pgfpathclose%
\pgfusepath{stroke,fill}%
\end{pgfscope}%
\begin{pgfscope}%
\pgfpathrectangle{\pgfqpoint{0.100000in}{0.212622in}}{\pgfqpoint{3.696000in}{3.696000in}}%
\pgfusepath{clip}%
\pgfsetbuttcap%
\pgfsetroundjoin%
\definecolor{currentfill}{rgb}{0.121569,0.466667,0.705882}%
\pgfsetfillcolor{currentfill}%
\pgfsetfillopacity{0.387427}%
\pgfsetlinewidth{1.003750pt}%
\definecolor{currentstroke}{rgb}{0.121569,0.466667,0.705882}%
\pgfsetstrokecolor{currentstroke}%
\pgfsetstrokeopacity{0.387427}%
\pgfsetdash{}{0pt}%
\pgfpathmoveto{\pgfqpoint{1.611900in}{2.894431in}}%
\pgfpathcurveto{\pgfqpoint{1.620136in}{2.894431in}}{\pgfqpoint{1.628036in}{2.897703in}}{\pgfqpoint{1.633860in}{2.903527in}}%
\pgfpathcurveto{\pgfqpoint{1.639684in}{2.909351in}}{\pgfqpoint{1.642956in}{2.917251in}}{\pgfqpoint{1.642956in}{2.925487in}}%
\pgfpathcurveto{\pgfqpoint{1.642956in}{2.933723in}}{\pgfqpoint{1.639684in}{2.941623in}}{\pgfqpoint{1.633860in}{2.947447in}}%
\pgfpathcurveto{\pgfqpoint{1.628036in}{2.953271in}}{\pgfqpoint{1.620136in}{2.956544in}}{\pgfqpoint{1.611900in}{2.956544in}}%
\pgfpathcurveto{\pgfqpoint{1.603664in}{2.956544in}}{\pgfqpoint{1.595764in}{2.953271in}}{\pgfqpoint{1.589940in}{2.947447in}}%
\pgfpathcurveto{\pgfqpoint{1.584116in}{2.941623in}}{\pgfqpoint{1.580843in}{2.933723in}}{\pgfqpoint{1.580843in}{2.925487in}}%
\pgfpathcurveto{\pgfqpoint{1.580843in}{2.917251in}}{\pgfqpoint{1.584116in}{2.909351in}}{\pgfqpoint{1.589940in}{2.903527in}}%
\pgfpathcurveto{\pgfqpoint{1.595764in}{2.897703in}}{\pgfqpoint{1.603664in}{2.894431in}}{\pgfqpoint{1.611900in}{2.894431in}}%
\pgfpathclose%
\pgfusepath{stroke,fill}%
\end{pgfscope}%
\begin{pgfscope}%
\pgfpathrectangle{\pgfqpoint{0.100000in}{0.212622in}}{\pgfqpoint{3.696000in}{3.696000in}}%
\pgfusepath{clip}%
\pgfsetbuttcap%
\pgfsetroundjoin%
\definecolor{currentfill}{rgb}{0.121569,0.466667,0.705882}%
\pgfsetfillcolor{currentfill}%
\pgfsetfillopacity{0.387673}%
\pgfsetlinewidth{1.003750pt}%
\definecolor{currentstroke}{rgb}{0.121569,0.466667,0.705882}%
\pgfsetstrokecolor{currentstroke}%
\pgfsetstrokeopacity{0.387673}%
\pgfsetdash{}{0pt}%
\pgfpathmoveto{\pgfqpoint{1.970195in}{2.973399in}}%
\pgfpathcurveto{\pgfqpoint{1.978431in}{2.973399in}}{\pgfqpoint{1.986331in}{2.976671in}}{\pgfqpoint{1.992155in}{2.982495in}}%
\pgfpathcurveto{\pgfqpoint{1.997979in}{2.988319in}}{\pgfqpoint{2.001251in}{2.996219in}}{\pgfqpoint{2.001251in}{3.004455in}}%
\pgfpathcurveto{\pgfqpoint{2.001251in}{3.012692in}}{\pgfqpoint{1.997979in}{3.020592in}}{\pgfqpoint{1.992155in}{3.026416in}}%
\pgfpathcurveto{\pgfqpoint{1.986331in}{3.032240in}}{\pgfqpoint{1.978431in}{3.035512in}}{\pgfqpoint{1.970195in}{3.035512in}}%
\pgfpathcurveto{\pgfqpoint{1.961958in}{3.035512in}}{\pgfqpoint{1.954058in}{3.032240in}}{\pgfqpoint{1.948234in}{3.026416in}}%
\pgfpathcurveto{\pgfqpoint{1.942410in}{3.020592in}}{\pgfqpoint{1.939138in}{3.012692in}}{\pgfqpoint{1.939138in}{3.004455in}}%
\pgfpathcurveto{\pgfqpoint{1.939138in}{2.996219in}}{\pgfqpoint{1.942410in}{2.988319in}}{\pgfqpoint{1.948234in}{2.982495in}}%
\pgfpathcurveto{\pgfqpoint{1.954058in}{2.976671in}}{\pgfqpoint{1.961958in}{2.973399in}}{\pgfqpoint{1.970195in}{2.973399in}}%
\pgfpathclose%
\pgfusepath{stroke,fill}%
\end{pgfscope}%
\begin{pgfscope}%
\pgfpathrectangle{\pgfqpoint{0.100000in}{0.212622in}}{\pgfqpoint{3.696000in}{3.696000in}}%
\pgfusepath{clip}%
\pgfsetbuttcap%
\pgfsetroundjoin%
\definecolor{currentfill}{rgb}{0.121569,0.466667,0.705882}%
\pgfsetfillcolor{currentfill}%
\pgfsetfillopacity{0.387962}%
\pgfsetlinewidth{1.003750pt}%
\definecolor{currentstroke}{rgb}{0.121569,0.466667,0.705882}%
\pgfsetstrokecolor{currentstroke}%
\pgfsetstrokeopacity{0.387962}%
\pgfsetdash{}{0pt}%
\pgfpathmoveto{\pgfqpoint{1.610181in}{2.891684in}}%
\pgfpathcurveto{\pgfqpoint{1.618417in}{2.891684in}}{\pgfqpoint{1.626317in}{2.894956in}}{\pgfqpoint{1.632141in}{2.900780in}}%
\pgfpathcurveto{\pgfqpoint{1.637965in}{2.906604in}}{\pgfqpoint{1.641237in}{2.914504in}}{\pgfqpoint{1.641237in}{2.922741in}}%
\pgfpathcurveto{\pgfqpoint{1.641237in}{2.930977in}}{\pgfqpoint{1.637965in}{2.938877in}}{\pgfqpoint{1.632141in}{2.944701in}}%
\pgfpathcurveto{\pgfqpoint{1.626317in}{2.950525in}}{\pgfqpoint{1.618417in}{2.953797in}}{\pgfqpoint{1.610181in}{2.953797in}}%
\pgfpathcurveto{\pgfqpoint{1.601944in}{2.953797in}}{\pgfqpoint{1.594044in}{2.950525in}}{\pgfqpoint{1.588220in}{2.944701in}}%
\pgfpathcurveto{\pgfqpoint{1.582396in}{2.938877in}}{\pgfqpoint{1.579124in}{2.930977in}}{\pgfqpoint{1.579124in}{2.922741in}}%
\pgfpathcurveto{\pgfqpoint{1.579124in}{2.914504in}}{\pgfqpoint{1.582396in}{2.906604in}}{\pgfqpoint{1.588220in}{2.900780in}}%
\pgfpathcurveto{\pgfqpoint{1.594044in}{2.894956in}}{\pgfqpoint{1.601944in}{2.891684in}}{\pgfqpoint{1.610181in}{2.891684in}}%
\pgfpathclose%
\pgfusepath{stroke,fill}%
\end{pgfscope}%
\begin{pgfscope}%
\pgfpathrectangle{\pgfqpoint{0.100000in}{0.212622in}}{\pgfqpoint{3.696000in}{3.696000in}}%
\pgfusepath{clip}%
\pgfsetbuttcap%
\pgfsetroundjoin%
\definecolor{currentfill}{rgb}{0.121569,0.466667,0.705882}%
\pgfsetfillcolor{currentfill}%
\pgfsetfillopacity{0.388952}%
\pgfsetlinewidth{1.003750pt}%
\definecolor{currentstroke}{rgb}{0.121569,0.466667,0.705882}%
\pgfsetstrokecolor{currentstroke}%
\pgfsetstrokeopacity{0.388952}%
\pgfsetdash{}{0pt}%
\pgfpathmoveto{\pgfqpoint{1.607179in}{2.886570in}}%
\pgfpathcurveto{\pgfqpoint{1.615415in}{2.886570in}}{\pgfqpoint{1.623315in}{2.889842in}}{\pgfqpoint{1.629139in}{2.895666in}}%
\pgfpathcurveto{\pgfqpoint{1.634963in}{2.901490in}}{\pgfqpoint{1.638235in}{2.909390in}}{\pgfqpoint{1.638235in}{2.917626in}}%
\pgfpathcurveto{\pgfqpoint{1.638235in}{2.925863in}}{\pgfqpoint{1.634963in}{2.933763in}}{\pgfqpoint{1.629139in}{2.939586in}}%
\pgfpathcurveto{\pgfqpoint{1.623315in}{2.945410in}}{\pgfqpoint{1.615415in}{2.948683in}}{\pgfqpoint{1.607179in}{2.948683in}}%
\pgfpathcurveto{\pgfqpoint{1.598943in}{2.948683in}}{\pgfqpoint{1.591043in}{2.945410in}}{\pgfqpoint{1.585219in}{2.939586in}}%
\pgfpathcurveto{\pgfqpoint{1.579395in}{2.933763in}}{\pgfqpoint{1.576122in}{2.925863in}}{\pgfqpoint{1.576122in}{2.917626in}}%
\pgfpathcurveto{\pgfqpoint{1.576122in}{2.909390in}}{\pgfqpoint{1.579395in}{2.901490in}}{\pgfqpoint{1.585219in}{2.895666in}}%
\pgfpathcurveto{\pgfqpoint{1.591043in}{2.889842in}}{\pgfqpoint{1.598943in}{2.886570in}}{\pgfqpoint{1.607179in}{2.886570in}}%
\pgfpathclose%
\pgfusepath{stroke,fill}%
\end{pgfscope}%
\begin{pgfscope}%
\pgfpathrectangle{\pgfqpoint{0.100000in}{0.212622in}}{\pgfqpoint{3.696000in}{3.696000in}}%
\pgfusepath{clip}%
\pgfsetbuttcap%
\pgfsetroundjoin%
\definecolor{currentfill}{rgb}{0.121569,0.466667,0.705882}%
\pgfsetfillcolor{currentfill}%
\pgfsetfillopacity{0.389362}%
\pgfsetlinewidth{1.003750pt}%
\definecolor{currentstroke}{rgb}{0.121569,0.466667,0.705882}%
\pgfsetstrokecolor{currentstroke}%
\pgfsetstrokeopacity{0.389362}%
\pgfsetdash{}{0pt}%
\pgfpathmoveto{\pgfqpoint{1.970828in}{2.967104in}}%
\pgfpathcurveto{\pgfqpoint{1.979065in}{2.967104in}}{\pgfqpoint{1.986965in}{2.970376in}}{\pgfqpoint{1.992789in}{2.976200in}}%
\pgfpathcurveto{\pgfqpoint{1.998613in}{2.982024in}}{\pgfqpoint{2.001885in}{2.989924in}}{\pgfqpoint{2.001885in}{2.998160in}}%
\pgfpathcurveto{\pgfqpoint{2.001885in}{3.006396in}}{\pgfqpoint{1.998613in}{3.014296in}}{\pgfqpoint{1.992789in}{3.020120in}}%
\pgfpathcurveto{\pgfqpoint{1.986965in}{3.025944in}}{\pgfqpoint{1.979065in}{3.029217in}}{\pgfqpoint{1.970828in}{3.029217in}}%
\pgfpathcurveto{\pgfqpoint{1.962592in}{3.029217in}}{\pgfqpoint{1.954692in}{3.025944in}}{\pgfqpoint{1.948868in}{3.020120in}}%
\pgfpathcurveto{\pgfqpoint{1.943044in}{3.014296in}}{\pgfqpoint{1.939772in}{3.006396in}}{\pgfqpoint{1.939772in}{2.998160in}}%
\pgfpathcurveto{\pgfqpoint{1.939772in}{2.989924in}}{\pgfqpoint{1.943044in}{2.982024in}}{\pgfqpoint{1.948868in}{2.976200in}}%
\pgfpathcurveto{\pgfqpoint{1.954692in}{2.970376in}}{\pgfqpoint{1.962592in}{2.967104in}}{\pgfqpoint{1.970828in}{2.967104in}}%
\pgfpathclose%
\pgfusepath{stroke,fill}%
\end{pgfscope}%
\begin{pgfscope}%
\pgfpathrectangle{\pgfqpoint{0.100000in}{0.212622in}}{\pgfqpoint{3.696000in}{3.696000in}}%
\pgfusepath{clip}%
\pgfsetbuttcap%
\pgfsetroundjoin%
\definecolor{currentfill}{rgb}{0.121569,0.466667,0.705882}%
\pgfsetfillcolor{currentfill}%
\pgfsetfillopacity{0.389814}%
\pgfsetlinewidth{1.003750pt}%
\definecolor{currentstroke}{rgb}{0.121569,0.466667,0.705882}%
\pgfsetstrokecolor{currentstroke}%
\pgfsetstrokeopacity{0.389814}%
\pgfsetdash{}{0pt}%
\pgfpathmoveto{\pgfqpoint{1.604890in}{2.882107in}}%
\pgfpathcurveto{\pgfqpoint{1.613126in}{2.882107in}}{\pgfqpoint{1.621026in}{2.885379in}}{\pgfqpoint{1.626850in}{2.891203in}}%
\pgfpathcurveto{\pgfqpoint{1.632674in}{2.897027in}}{\pgfqpoint{1.635947in}{2.904927in}}{\pgfqpoint{1.635947in}{2.913163in}}%
\pgfpathcurveto{\pgfqpoint{1.635947in}{2.921400in}}{\pgfqpoint{1.632674in}{2.929300in}}{\pgfqpoint{1.626850in}{2.935124in}}%
\pgfpathcurveto{\pgfqpoint{1.621026in}{2.940948in}}{\pgfqpoint{1.613126in}{2.944220in}}{\pgfqpoint{1.604890in}{2.944220in}}%
\pgfpathcurveto{\pgfqpoint{1.596654in}{2.944220in}}{\pgfqpoint{1.588754in}{2.940948in}}{\pgfqpoint{1.582930in}{2.935124in}}%
\pgfpathcurveto{\pgfqpoint{1.577106in}{2.929300in}}{\pgfqpoint{1.573834in}{2.921400in}}{\pgfqpoint{1.573834in}{2.913163in}}%
\pgfpathcurveto{\pgfqpoint{1.573834in}{2.904927in}}{\pgfqpoint{1.577106in}{2.897027in}}{\pgfqpoint{1.582930in}{2.891203in}}%
\pgfpathcurveto{\pgfqpoint{1.588754in}{2.885379in}}{\pgfqpoint{1.596654in}{2.882107in}}{\pgfqpoint{1.604890in}{2.882107in}}%
\pgfpathclose%
\pgfusepath{stroke,fill}%
\end{pgfscope}%
\begin{pgfscope}%
\pgfpathrectangle{\pgfqpoint{0.100000in}{0.212622in}}{\pgfqpoint{3.696000in}{3.696000in}}%
\pgfusepath{clip}%
\pgfsetbuttcap%
\pgfsetroundjoin%
\definecolor{currentfill}{rgb}{0.121569,0.466667,0.705882}%
\pgfsetfillcolor{currentfill}%
\pgfsetfillopacity{0.390247}%
\pgfsetlinewidth{1.003750pt}%
\definecolor{currentstroke}{rgb}{0.121569,0.466667,0.705882}%
\pgfsetstrokecolor{currentstroke}%
\pgfsetstrokeopacity{0.390247}%
\pgfsetdash{}{0pt}%
\pgfpathmoveto{\pgfqpoint{1.603340in}{2.879734in}}%
\pgfpathcurveto{\pgfqpoint{1.611576in}{2.879734in}}{\pgfqpoint{1.619476in}{2.883006in}}{\pgfqpoint{1.625300in}{2.888830in}}%
\pgfpathcurveto{\pgfqpoint{1.631124in}{2.894654in}}{\pgfqpoint{1.634397in}{2.902554in}}{\pgfqpoint{1.634397in}{2.910790in}}%
\pgfpathcurveto{\pgfqpoint{1.634397in}{2.919027in}}{\pgfqpoint{1.631124in}{2.926927in}}{\pgfqpoint{1.625300in}{2.932751in}}%
\pgfpathcurveto{\pgfqpoint{1.619476in}{2.938575in}}{\pgfqpoint{1.611576in}{2.941847in}}{\pgfqpoint{1.603340in}{2.941847in}}%
\pgfpathcurveto{\pgfqpoint{1.595104in}{2.941847in}}{\pgfqpoint{1.587204in}{2.938575in}}{\pgfqpoint{1.581380in}{2.932751in}}%
\pgfpathcurveto{\pgfqpoint{1.575556in}{2.926927in}}{\pgfqpoint{1.572284in}{2.919027in}}{\pgfqpoint{1.572284in}{2.910790in}}%
\pgfpathcurveto{\pgfqpoint{1.572284in}{2.902554in}}{\pgfqpoint{1.575556in}{2.894654in}}{\pgfqpoint{1.581380in}{2.888830in}}%
\pgfpathcurveto{\pgfqpoint{1.587204in}{2.883006in}}{\pgfqpoint{1.595104in}{2.879734in}}{\pgfqpoint{1.603340in}{2.879734in}}%
\pgfpathclose%
\pgfusepath{stroke,fill}%
\end{pgfscope}%
\begin{pgfscope}%
\pgfpathrectangle{\pgfqpoint{0.100000in}{0.212622in}}{\pgfqpoint{3.696000in}{3.696000in}}%
\pgfusepath{clip}%
\pgfsetbuttcap%
\pgfsetroundjoin%
\definecolor{currentfill}{rgb}{0.121569,0.466667,0.705882}%
\pgfsetfillcolor{currentfill}%
\pgfsetfillopacity{0.390541}%
\pgfsetlinewidth{1.003750pt}%
\definecolor{currentstroke}{rgb}{0.121569,0.466667,0.705882}%
\pgfsetstrokecolor{currentstroke}%
\pgfsetstrokeopacity{0.390541}%
\pgfsetdash{}{0pt}%
\pgfpathmoveto{\pgfqpoint{1.602556in}{2.878273in}}%
\pgfpathcurveto{\pgfqpoint{1.610792in}{2.878273in}}{\pgfqpoint{1.618692in}{2.881546in}}{\pgfqpoint{1.624516in}{2.887370in}}%
\pgfpathcurveto{\pgfqpoint{1.630340in}{2.893194in}}{\pgfqpoint{1.633612in}{2.901094in}}{\pgfqpoint{1.633612in}{2.909330in}}%
\pgfpathcurveto{\pgfqpoint{1.633612in}{2.917566in}}{\pgfqpoint{1.630340in}{2.925466in}}{\pgfqpoint{1.624516in}{2.931290in}}%
\pgfpathcurveto{\pgfqpoint{1.618692in}{2.937114in}}{\pgfqpoint{1.610792in}{2.940386in}}{\pgfqpoint{1.602556in}{2.940386in}}%
\pgfpathcurveto{\pgfqpoint{1.594320in}{2.940386in}}{\pgfqpoint{1.586420in}{2.937114in}}{\pgfqpoint{1.580596in}{2.931290in}}%
\pgfpathcurveto{\pgfqpoint{1.574772in}{2.925466in}}{\pgfqpoint{1.571499in}{2.917566in}}{\pgfqpoint{1.571499in}{2.909330in}}%
\pgfpathcurveto{\pgfqpoint{1.571499in}{2.901094in}}{\pgfqpoint{1.574772in}{2.893194in}}{\pgfqpoint{1.580596in}{2.887370in}}%
\pgfpathcurveto{\pgfqpoint{1.586420in}{2.881546in}}{\pgfqpoint{1.594320in}{2.878273in}}{\pgfqpoint{1.602556in}{2.878273in}}%
\pgfpathclose%
\pgfusepath{stroke,fill}%
\end{pgfscope}%
\begin{pgfscope}%
\pgfpathrectangle{\pgfqpoint{0.100000in}{0.212622in}}{\pgfqpoint{3.696000in}{3.696000in}}%
\pgfusepath{clip}%
\pgfsetbuttcap%
\pgfsetroundjoin%
\definecolor{currentfill}{rgb}{0.121569,0.466667,0.705882}%
\pgfsetfillcolor{currentfill}%
\pgfsetfillopacity{0.391058}%
\pgfsetlinewidth{1.003750pt}%
\definecolor{currentstroke}{rgb}{0.121569,0.466667,0.705882}%
\pgfsetstrokecolor{currentstroke}%
\pgfsetstrokeopacity{0.391058}%
\pgfsetdash{}{0pt}%
\pgfpathmoveto{\pgfqpoint{1.601060in}{2.875641in}}%
\pgfpathcurveto{\pgfqpoint{1.609297in}{2.875641in}}{\pgfqpoint{1.617197in}{2.878913in}}{\pgfqpoint{1.623021in}{2.884737in}}%
\pgfpathcurveto{\pgfqpoint{1.628845in}{2.890561in}}{\pgfqpoint{1.632117in}{2.898461in}}{\pgfqpoint{1.632117in}{2.906697in}}%
\pgfpathcurveto{\pgfqpoint{1.632117in}{2.914933in}}{\pgfqpoint{1.628845in}{2.922834in}}{\pgfqpoint{1.623021in}{2.928657in}}%
\pgfpathcurveto{\pgfqpoint{1.617197in}{2.934481in}}{\pgfqpoint{1.609297in}{2.937754in}}{\pgfqpoint{1.601060in}{2.937754in}}%
\pgfpathcurveto{\pgfqpoint{1.592824in}{2.937754in}}{\pgfqpoint{1.584924in}{2.934481in}}{\pgfqpoint{1.579100in}{2.928657in}}%
\pgfpathcurveto{\pgfqpoint{1.573276in}{2.922834in}}{\pgfqpoint{1.570004in}{2.914933in}}{\pgfqpoint{1.570004in}{2.906697in}}%
\pgfpathcurveto{\pgfqpoint{1.570004in}{2.898461in}}{\pgfqpoint{1.573276in}{2.890561in}}{\pgfqpoint{1.579100in}{2.884737in}}%
\pgfpathcurveto{\pgfqpoint{1.584924in}{2.878913in}}{\pgfqpoint{1.592824in}{2.875641in}}{\pgfqpoint{1.601060in}{2.875641in}}%
\pgfpathclose%
\pgfusepath{stroke,fill}%
\end{pgfscope}%
\begin{pgfscope}%
\pgfpathrectangle{\pgfqpoint{0.100000in}{0.212622in}}{\pgfqpoint{3.696000in}{3.696000in}}%
\pgfusepath{clip}%
\pgfsetbuttcap%
\pgfsetroundjoin%
\definecolor{currentfill}{rgb}{0.121569,0.466667,0.705882}%
\pgfsetfillcolor{currentfill}%
\pgfsetfillopacity{0.391269}%
\pgfsetlinewidth{1.003750pt}%
\definecolor{currentstroke}{rgb}{0.121569,0.466667,0.705882}%
\pgfsetstrokecolor{currentstroke}%
\pgfsetstrokeopacity{0.391269}%
\pgfsetdash{}{0pt}%
\pgfpathmoveto{\pgfqpoint{1.972391in}{2.958868in}}%
\pgfpathcurveto{\pgfqpoint{1.980627in}{2.958868in}}{\pgfqpoint{1.988527in}{2.962141in}}{\pgfqpoint{1.994351in}{2.967964in}}%
\pgfpathcurveto{\pgfqpoint{2.000175in}{2.973788in}}{\pgfqpoint{2.003447in}{2.981688in}}{\pgfqpoint{2.003447in}{2.989925in}}%
\pgfpathcurveto{\pgfqpoint{2.003447in}{2.998161in}}{\pgfqpoint{2.000175in}{3.006061in}}{\pgfqpoint{1.994351in}{3.011885in}}%
\pgfpathcurveto{\pgfqpoint{1.988527in}{3.017709in}}{\pgfqpoint{1.980627in}{3.020981in}}{\pgfqpoint{1.972391in}{3.020981in}}%
\pgfpathcurveto{\pgfqpoint{1.964155in}{3.020981in}}{\pgfqpoint{1.956255in}{3.017709in}}{\pgfqpoint{1.950431in}{3.011885in}}%
\pgfpathcurveto{\pgfqpoint{1.944607in}{3.006061in}}{\pgfqpoint{1.941334in}{2.998161in}}{\pgfqpoint{1.941334in}{2.989925in}}%
\pgfpathcurveto{\pgfqpoint{1.941334in}{2.981688in}}{\pgfqpoint{1.944607in}{2.973788in}}{\pgfqpoint{1.950431in}{2.967964in}}%
\pgfpathcurveto{\pgfqpoint{1.956255in}{2.962141in}}{\pgfqpoint{1.964155in}{2.958868in}}{\pgfqpoint{1.972391in}{2.958868in}}%
\pgfpathclose%
\pgfusepath{stroke,fill}%
\end{pgfscope}%
\begin{pgfscope}%
\pgfpathrectangle{\pgfqpoint{0.100000in}{0.212622in}}{\pgfqpoint{3.696000in}{3.696000in}}%
\pgfusepath{clip}%
\pgfsetbuttcap%
\pgfsetroundjoin%
\definecolor{currentfill}{rgb}{0.121569,0.466667,0.705882}%
\pgfsetfillcolor{currentfill}%
\pgfsetfillopacity{0.391902}%
\pgfsetlinewidth{1.003750pt}%
\definecolor{currentstroke}{rgb}{0.121569,0.466667,0.705882}%
\pgfsetstrokecolor{currentstroke}%
\pgfsetstrokeopacity{0.391902}%
\pgfsetdash{}{0pt}%
\pgfpathmoveto{\pgfqpoint{1.598152in}{2.870749in}}%
\pgfpathcurveto{\pgfqpoint{1.606388in}{2.870749in}}{\pgfqpoint{1.614288in}{2.874022in}}{\pgfqpoint{1.620112in}{2.879846in}}%
\pgfpathcurveto{\pgfqpoint{1.625936in}{2.885670in}}{\pgfqpoint{1.629208in}{2.893570in}}{\pgfqpoint{1.629208in}{2.901806in}}%
\pgfpathcurveto{\pgfqpoint{1.629208in}{2.910042in}}{\pgfqpoint{1.625936in}{2.917942in}}{\pgfqpoint{1.620112in}{2.923766in}}%
\pgfpathcurveto{\pgfqpoint{1.614288in}{2.929590in}}{\pgfqpoint{1.606388in}{2.932862in}}{\pgfqpoint{1.598152in}{2.932862in}}%
\pgfpathcurveto{\pgfqpoint{1.589915in}{2.932862in}}{\pgfqpoint{1.582015in}{2.929590in}}{\pgfqpoint{1.576191in}{2.923766in}}%
\pgfpathcurveto{\pgfqpoint{1.570368in}{2.917942in}}{\pgfqpoint{1.567095in}{2.910042in}}{\pgfqpoint{1.567095in}{2.901806in}}%
\pgfpathcurveto{\pgfqpoint{1.567095in}{2.893570in}}{\pgfqpoint{1.570368in}{2.885670in}}{\pgfqpoint{1.576191in}{2.879846in}}%
\pgfpathcurveto{\pgfqpoint{1.582015in}{2.874022in}}{\pgfqpoint{1.589915in}{2.870749in}}{\pgfqpoint{1.598152in}{2.870749in}}%
\pgfpathclose%
\pgfusepath{stroke,fill}%
\end{pgfscope}%
\begin{pgfscope}%
\pgfpathrectangle{\pgfqpoint{0.100000in}{0.212622in}}{\pgfqpoint{3.696000in}{3.696000in}}%
\pgfusepath{clip}%
\pgfsetbuttcap%
\pgfsetroundjoin%
\definecolor{currentfill}{rgb}{0.121569,0.466667,0.705882}%
\pgfsetfillcolor{currentfill}%
\pgfsetfillopacity{0.392608}%
\pgfsetlinewidth{1.003750pt}%
\definecolor{currentstroke}{rgb}{0.121569,0.466667,0.705882}%
\pgfsetstrokecolor{currentstroke}%
\pgfsetstrokeopacity{0.392608}%
\pgfsetdash{}{0pt}%
\pgfpathmoveto{\pgfqpoint{1.596337in}{2.866786in}}%
\pgfpathcurveto{\pgfqpoint{1.604573in}{2.866786in}}{\pgfqpoint{1.612473in}{2.870058in}}{\pgfqpoint{1.618297in}{2.875882in}}%
\pgfpathcurveto{\pgfqpoint{1.624121in}{2.881706in}}{\pgfqpoint{1.627393in}{2.889606in}}{\pgfqpoint{1.627393in}{2.897842in}}%
\pgfpathcurveto{\pgfqpoint{1.627393in}{2.906079in}}{\pgfqpoint{1.624121in}{2.913979in}}{\pgfqpoint{1.618297in}{2.919803in}}%
\pgfpathcurveto{\pgfqpoint{1.612473in}{2.925627in}}{\pgfqpoint{1.604573in}{2.928899in}}{\pgfqpoint{1.596337in}{2.928899in}}%
\pgfpathcurveto{\pgfqpoint{1.588100in}{2.928899in}}{\pgfqpoint{1.580200in}{2.925627in}}{\pgfqpoint{1.574376in}{2.919803in}}%
\pgfpathcurveto{\pgfqpoint{1.568552in}{2.913979in}}{\pgfqpoint{1.565280in}{2.906079in}}{\pgfqpoint{1.565280in}{2.897842in}}%
\pgfpathcurveto{\pgfqpoint{1.565280in}{2.889606in}}{\pgfqpoint{1.568552in}{2.881706in}}{\pgfqpoint{1.574376in}{2.875882in}}%
\pgfpathcurveto{\pgfqpoint{1.580200in}{2.870058in}}{\pgfqpoint{1.588100in}{2.866786in}}{\pgfqpoint{1.596337in}{2.866786in}}%
\pgfpathclose%
\pgfusepath{stroke,fill}%
\end{pgfscope}%
\begin{pgfscope}%
\pgfpathrectangle{\pgfqpoint{0.100000in}{0.212622in}}{\pgfqpoint{3.696000in}{3.696000in}}%
\pgfusepath{clip}%
\pgfsetbuttcap%
\pgfsetroundjoin%
\definecolor{currentfill}{rgb}{0.121569,0.466667,0.705882}%
\pgfsetfillcolor{currentfill}%
\pgfsetfillopacity{0.392865}%
\pgfsetlinewidth{1.003750pt}%
\definecolor{currentstroke}{rgb}{0.121569,0.466667,0.705882}%
\pgfsetstrokecolor{currentstroke}%
\pgfsetstrokeopacity{0.392865}%
\pgfsetdash{}{0pt}%
\pgfpathmoveto{\pgfqpoint{1.595490in}{2.865413in}}%
\pgfpathcurveto{\pgfqpoint{1.603727in}{2.865413in}}{\pgfqpoint{1.611627in}{2.868685in}}{\pgfqpoint{1.617451in}{2.874509in}}%
\pgfpathcurveto{\pgfqpoint{1.623275in}{2.880333in}}{\pgfqpoint{1.626547in}{2.888233in}}{\pgfqpoint{1.626547in}{2.896470in}}%
\pgfpathcurveto{\pgfqpoint{1.626547in}{2.904706in}}{\pgfqpoint{1.623275in}{2.912606in}}{\pgfqpoint{1.617451in}{2.918430in}}%
\pgfpathcurveto{\pgfqpoint{1.611627in}{2.924254in}}{\pgfqpoint{1.603727in}{2.927526in}}{\pgfqpoint{1.595490in}{2.927526in}}%
\pgfpathcurveto{\pgfqpoint{1.587254in}{2.927526in}}{\pgfqpoint{1.579354in}{2.924254in}}{\pgfqpoint{1.573530in}{2.918430in}}%
\pgfpathcurveto{\pgfqpoint{1.567706in}{2.912606in}}{\pgfqpoint{1.564434in}{2.904706in}}{\pgfqpoint{1.564434in}{2.896470in}}%
\pgfpathcurveto{\pgfqpoint{1.564434in}{2.888233in}}{\pgfqpoint{1.567706in}{2.880333in}}{\pgfqpoint{1.573530in}{2.874509in}}%
\pgfpathcurveto{\pgfqpoint{1.579354in}{2.868685in}}{\pgfqpoint{1.587254in}{2.865413in}}{\pgfqpoint{1.595490in}{2.865413in}}%
\pgfpathclose%
\pgfusepath{stroke,fill}%
\end{pgfscope}%
\begin{pgfscope}%
\pgfpathrectangle{\pgfqpoint{0.100000in}{0.212622in}}{\pgfqpoint{3.696000in}{3.696000in}}%
\pgfusepath{clip}%
\pgfsetbuttcap%
\pgfsetroundjoin%
\definecolor{currentfill}{rgb}{0.121569,0.466667,0.705882}%
\pgfsetfillcolor{currentfill}%
\pgfsetfillopacity{0.393360}%
\pgfsetlinewidth{1.003750pt}%
\definecolor{currentstroke}{rgb}{0.121569,0.466667,0.705882}%
\pgfsetstrokecolor{currentstroke}%
\pgfsetstrokeopacity{0.393360}%
\pgfsetdash{}{0pt}%
\pgfpathmoveto{\pgfqpoint{1.594081in}{2.862844in}}%
\pgfpathcurveto{\pgfqpoint{1.602317in}{2.862844in}}{\pgfqpoint{1.610217in}{2.866116in}}{\pgfqpoint{1.616041in}{2.871940in}}%
\pgfpathcurveto{\pgfqpoint{1.621865in}{2.877764in}}{\pgfqpoint{1.625137in}{2.885664in}}{\pgfqpoint{1.625137in}{2.893900in}}%
\pgfpathcurveto{\pgfqpoint{1.625137in}{2.902136in}}{\pgfqpoint{1.621865in}{2.910037in}}{\pgfqpoint{1.616041in}{2.915860in}}%
\pgfpathcurveto{\pgfqpoint{1.610217in}{2.921684in}}{\pgfqpoint{1.602317in}{2.924957in}}{\pgfqpoint{1.594081in}{2.924957in}}%
\pgfpathcurveto{\pgfqpoint{1.585844in}{2.924957in}}{\pgfqpoint{1.577944in}{2.921684in}}{\pgfqpoint{1.572120in}{2.915860in}}%
\pgfpathcurveto{\pgfqpoint{1.566296in}{2.910037in}}{\pgfqpoint{1.563024in}{2.902136in}}{\pgfqpoint{1.563024in}{2.893900in}}%
\pgfpathcurveto{\pgfqpoint{1.563024in}{2.885664in}}{\pgfqpoint{1.566296in}{2.877764in}}{\pgfqpoint{1.572120in}{2.871940in}}%
\pgfpathcurveto{\pgfqpoint{1.577944in}{2.866116in}}{\pgfqpoint{1.585844in}{2.862844in}}{\pgfqpoint{1.594081in}{2.862844in}}%
\pgfpathclose%
\pgfusepath{stroke,fill}%
\end{pgfscope}%
\begin{pgfscope}%
\pgfpathrectangle{\pgfqpoint{0.100000in}{0.212622in}}{\pgfqpoint{3.696000in}{3.696000in}}%
\pgfusepath{clip}%
\pgfsetbuttcap%
\pgfsetroundjoin%
\definecolor{currentfill}{rgb}{0.121569,0.466667,0.705882}%
\pgfsetfillcolor{currentfill}%
\pgfsetfillopacity{0.393474}%
\pgfsetlinewidth{1.003750pt}%
\definecolor{currentstroke}{rgb}{0.121569,0.466667,0.705882}%
\pgfsetstrokecolor{currentstroke}%
\pgfsetstrokeopacity{0.393474}%
\pgfsetdash{}{0pt}%
\pgfpathmoveto{\pgfqpoint{1.973733in}{2.950452in}}%
\pgfpathcurveto{\pgfqpoint{1.981969in}{2.950452in}}{\pgfqpoint{1.989869in}{2.953725in}}{\pgfqpoint{1.995693in}{2.959548in}}%
\pgfpathcurveto{\pgfqpoint{2.001517in}{2.965372in}}{\pgfqpoint{2.004789in}{2.973272in}}{\pgfqpoint{2.004789in}{2.981509in}}%
\pgfpathcurveto{\pgfqpoint{2.004789in}{2.989745in}}{\pgfqpoint{2.001517in}{2.997645in}}{\pgfqpoint{1.995693in}{3.003469in}}%
\pgfpathcurveto{\pgfqpoint{1.989869in}{3.009293in}}{\pgfqpoint{1.981969in}{3.012565in}}{\pgfqpoint{1.973733in}{3.012565in}}%
\pgfpathcurveto{\pgfqpoint{1.965497in}{3.012565in}}{\pgfqpoint{1.957597in}{3.009293in}}{\pgfqpoint{1.951773in}{3.003469in}}%
\pgfpathcurveto{\pgfqpoint{1.945949in}{2.997645in}}{\pgfqpoint{1.942676in}{2.989745in}}{\pgfqpoint{1.942676in}{2.981509in}}%
\pgfpathcurveto{\pgfqpoint{1.942676in}{2.973272in}}{\pgfqpoint{1.945949in}{2.965372in}}{\pgfqpoint{1.951773in}{2.959548in}}%
\pgfpathcurveto{\pgfqpoint{1.957597in}{2.953725in}}{\pgfqpoint{1.965497in}{2.950452in}}{\pgfqpoint{1.973733in}{2.950452in}}%
\pgfpathclose%
\pgfusepath{stroke,fill}%
\end{pgfscope}%
\begin{pgfscope}%
\pgfpathrectangle{\pgfqpoint{0.100000in}{0.212622in}}{\pgfqpoint{3.696000in}{3.696000in}}%
\pgfusepath{clip}%
\pgfsetbuttcap%
\pgfsetroundjoin%
\definecolor{currentfill}{rgb}{0.121569,0.466667,0.705882}%
\pgfsetfillcolor{currentfill}%
\pgfsetfillopacity{0.394272}%
\pgfsetlinewidth{1.003750pt}%
\definecolor{currentstroke}{rgb}{0.121569,0.466667,0.705882}%
\pgfsetstrokecolor{currentstroke}%
\pgfsetstrokeopacity{0.394272}%
\pgfsetdash{}{0pt}%
\pgfpathmoveto{\pgfqpoint{1.591556in}{2.858166in}}%
\pgfpathcurveto{\pgfqpoint{1.599792in}{2.858166in}}{\pgfqpoint{1.607692in}{2.861438in}}{\pgfqpoint{1.613516in}{2.867262in}}%
\pgfpathcurveto{\pgfqpoint{1.619340in}{2.873086in}}{\pgfqpoint{1.622612in}{2.880986in}}{\pgfqpoint{1.622612in}{2.889222in}}%
\pgfpathcurveto{\pgfqpoint{1.622612in}{2.897458in}}{\pgfqpoint{1.619340in}{2.905359in}}{\pgfqpoint{1.613516in}{2.911182in}}%
\pgfpathcurveto{\pgfqpoint{1.607692in}{2.917006in}}{\pgfqpoint{1.599792in}{2.920279in}}{\pgfqpoint{1.591556in}{2.920279in}}%
\pgfpathcurveto{\pgfqpoint{1.583319in}{2.920279in}}{\pgfqpoint{1.575419in}{2.917006in}}{\pgfqpoint{1.569595in}{2.911182in}}%
\pgfpathcurveto{\pgfqpoint{1.563772in}{2.905359in}}{\pgfqpoint{1.560499in}{2.897458in}}{\pgfqpoint{1.560499in}{2.889222in}}%
\pgfpathcurveto{\pgfqpoint{1.560499in}{2.880986in}}{\pgfqpoint{1.563772in}{2.873086in}}{\pgfqpoint{1.569595in}{2.867262in}}%
\pgfpathcurveto{\pgfqpoint{1.575419in}{2.861438in}}{\pgfqpoint{1.583319in}{2.858166in}}{\pgfqpoint{1.591556in}{2.858166in}}%
\pgfpathclose%
\pgfusepath{stroke,fill}%
\end{pgfscope}%
\begin{pgfscope}%
\pgfpathrectangle{\pgfqpoint{0.100000in}{0.212622in}}{\pgfqpoint{3.696000in}{3.696000in}}%
\pgfusepath{clip}%
\pgfsetbuttcap%
\pgfsetroundjoin%
\definecolor{currentfill}{rgb}{0.121569,0.466667,0.705882}%
\pgfsetfillcolor{currentfill}%
\pgfsetfillopacity{0.394883}%
\pgfsetlinewidth{1.003750pt}%
\definecolor{currentstroke}{rgb}{0.121569,0.466667,0.705882}%
\pgfsetstrokecolor{currentstroke}%
\pgfsetstrokeopacity{0.394883}%
\pgfsetdash{}{0pt}%
\pgfpathmoveto{\pgfqpoint{1.589336in}{2.854601in}}%
\pgfpathcurveto{\pgfqpoint{1.597572in}{2.854601in}}{\pgfqpoint{1.605472in}{2.857873in}}{\pgfqpoint{1.611296in}{2.863697in}}%
\pgfpathcurveto{\pgfqpoint{1.617120in}{2.869521in}}{\pgfqpoint{1.620392in}{2.877421in}}{\pgfqpoint{1.620392in}{2.885657in}}%
\pgfpathcurveto{\pgfqpoint{1.620392in}{2.893894in}}{\pgfqpoint{1.617120in}{2.901794in}}{\pgfqpoint{1.611296in}{2.907618in}}%
\pgfpathcurveto{\pgfqpoint{1.605472in}{2.913442in}}{\pgfqpoint{1.597572in}{2.916714in}}{\pgfqpoint{1.589336in}{2.916714in}}%
\pgfpathcurveto{\pgfqpoint{1.581100in}{2.916714in}}{\pgfqpoint{1.573200in}{2.913442in}}{\pgfqpoint{1.567376in}{2.907618in}}%
\pgfpathcurveto{\pgfqpoint{1.561552in}{2.901794in}}{\pgfqpoint{1.558279in}{2.893894in}}{\pgfqpoint{1.558279in}{2.885657in}}%
\pgfpathcurveto{\pgfqpoint{1.558279in}{2.877421in}}{\pgfqpoint{1.561552in}{2.869521in}}{\pgfqpoint{1.567376in}{2.863697in}}%
\pgfpathcurveto{\pgfqpoint{1.573200in}{2.857873in}}{\pgfqpoint{1.581100in}{2.854601in}}{\pgfqpoint{1.589336in}{2.854601in}}%
\pgfpathclose%
\pgfusepath{stroke,fill}%
\end{pgfscope}%
\begin{pgfscope}%
\pgfpathrectangle{\pgfqpoint{0.100000in}{0.212622in}}{\pgfqpoint{3.696000in}{3.696000in}}%
\pgfusepath{clip}%
\pgfsetbuttcap%
\pgfsetroundjoin%
\definecolor{currentfill}{rgb}{0.121569,0.466667,0.705882}%
\pgfsetfillcolor{currentfill}%
\pgfsetfillopacity{0.395271}%
\pgfsetlinewidth{1.003750pt}%
\definecolor{currentstroke}{rgb}{0.121569,0.466667,0.705882}%
\pgfsetstrokecolor{currentstroke}%
\pgfsetstrokeopacity{0.395271}%
\pgfsetdash{}{0pt}%
\pgfpathmoveto{\pgfqpoint{1.588263in}{2.852455in}}%
\pgfpathcurveto{\pgfqpoint{1.596499in}{2.852455in}}{\pgfqpoint{1.604399in}{2.855728in}}{\pgfqpoint{1.610223in}{2.861552in}}%
\pgfpathcurveto{\pgfqpoint{1.616047in}{2.867376in}}{\pgfqpoint{1.619319in}{2.875276in}}{\pgfqpoint{1.619319in}{2.883512in}}%
\pgfpathcurveto{\pgfqpoint{1.619319in}{2.891748in}}{\pgfqpoint{1.616047in}{2.899648in}}{\pgfqpoint{1.610223in}{2.905472in}}%
\pgfpathcurveto{\pgfqpoint{1.604399in}{2.911296in}}{\pgfqpoint{1.596499in}{2.914568in}}{\pgfqpoint{1.588263in}{2.914568in}}%
\pgfpathcurveto{\pgfqpoint{1.580026in}{2.914568in}}{\pgfqpoint{1.572126in}{2.911296in}}{\pgfqpoint{1.566302in}{2.905472in}}%
\pgfpathcurveto{\pgfqpoint{1.560478in}{2.899648in}}{\pgfqpoint{1.557206in}{2.891748in}}{\pgfqpoint{1.557206in}{2.883512in}}%
\pgfpathcurveto{\pgfqpoint{1.557206in}{2.875276in}}{\pgfqpoint{1.560478in}{2.867376in}}{\pgfqpoint{1.566302in}{2.861552in}}%
\pgfpathcurveto{\pgfqpoint{1.572126in}{2.855728in}}{\pgfqpoint{1.580026in}{2.852455in}}{\pgfqpoint{1.588263in}{2.852455in}}%
\pgfpathclose%
\pgfusepath{stroke,fill}%
\end{pgfscope}%
\begin{pgfscope}%
\pgfpathrectangle{\pgfqpoint{0.100000in}{0.212622in}}{\pgfqpoint{3.696000in}{3.696000in}}%
\pgfusepath{clip}%
\pgfsetbuttcap%
\pgfsetroundjoin%
\definecolor{currentfill}{rgb}{0.121569,0.466667,0.705882}%
\pgfsetfillcolor{currentfill}%
\pgfsetfillopacity{0.395959}%
\pgfsetlinewidth{1.003750pt}%
\definecolor{currentstroke}{rgb}{0.121569,0.466667,0.705882}%
\pgfsetstrokecolor{currentstroke}%
\pgfsetstrokeopacity{0.395959}%
\pgfsetdash{}{0pt}%
\pgfpathmoveto{\pgfqpoint{1.586118in}{2.848725in}}%
\pgfpathcurveto{\pgfqpoint{1.594355in}{2.848725in}}{\pgfqpoint{1.602255in}{2.851997in}}{\pgfqpoint{1.608079in}{2.857821in}}%
\pgfpathcurveto{\pgfqpoint{1.613902in}{2.863645in}}{\pgfqpoint{1.617175in}{2.871545in}}{\pgfqpoint{1.617175in}{2.879781in}}%
\pgfpathcurveto{\pgfqpoint{1.617175in}{2.888018in}}{\pgfqpoint{1.613902in}{2.895918in}}{\pgfqpoint{1.608079in}{2.901741in}}%
\pgfpathcurveto{\pgfqpoint{1.602255in}{2.907565in}}{\pgfqpoint{1.594355in}{2.910838in}}{\pgfqpoint{1.586118in}{2.910838in}}%
\pgfpathcurveto{\pgfqpoint{1.577882in}{2.910838in}}{\pgfqpoint{1.569982in}{2.907565in}}{\pgfqpoint{1.564158in}{2.901741in}}%
\pgfpathcurveto{\pgfqpoint{1.558334in}{2.895918in}}{\pgfqpoint{1.555062in}{2.888018in}}{\pgfqpoint{1.555062in}{2.879781in}}%
\pgfpathcurveto{\pgfqpoint{1.555062in}{2.871545in}}{\pgfqpoint{1.558334in}{2.863645in}}{\pgfqpoint{1.564158in}{2.857821in}}%
\pgfpathcurveto{\pgfqpoint{1.569982in}{2.851997in}}{\pgfqpoint{1.577882in}{2.848725in}}{\pgfqpoint{1.586118in}{2.848725in}}%
\pgfpathclose%
\pgfusepath{stroke,fill}%
\end{pgfscope}%
\begin{pgfscope}%
\pgfpathrectangle{\pgfqpoint{0.100000in}{0.212622in}}{\pgfqpoint{3.696000in}{3.696000in}}%
\pgfusepath{clip}%
\pgfsetbuttcap%
\pgfsetroundjoin%
\definecolor{currentfill}{rgb}{0.121569,0.466667,0.705882}%
\pgfsetfillcolor{currentfill}%
\pgfsetfillopacity{0.395988}%
\pgfsetlinewidth{1.003750pt}%
\definecolor{currentstroke}{rgb}{0.121569,0.466667,0.705882}%
\pgfsetstrokecolor{currentstroke}%
\pgfsetstrokeopacity{0.395988}%
\pgfsetdash{}{0pt}%
\pgfpathmoveto{\pgfqpoint{1.974747in}{2.941476in}}%
\pgfpathcurveto{\pgfqpoint{1.982983in}{2.941476in}}{\pgfqpoint{1.990883in}{2.944748in}}{\pgfqpoint{1.996707in}{2.950572in}}%
\pgfpathcurveto{\pgfqpoint{2.002531in}{2.956396in}}{\pgfqpoint{2.005803in}{2.964296in}}{\pgfqpoint{2.005803in}{2.972532in}}%
\pgfpathcurveto{\pgfqpoint{2.005803in}{2.980769in}}{\pgfqpoint{2.002531in}{2.988669in}}{\pgfqpoint{1.996707in}{2.994493in}}%
\pgfpathcurveto{\pgfqpoint{1.990883in}{3.000316in}}{\pgfqpoint{1.982983in}{3.003589in}}{\pgfqpoint{1.974747in}{3.003589in}}%
\pgfpathcurveto{\pgfqpoint{1.966511in}{3.003589in}}{\pgfqpoint{1.958611in}{3.000316in}}{\pgfqpoint{1.952787in}{2.994493in}}%
\pgfpathcurveto{\pgfqpoint{1.946963in}{2.988669in}}{\pgfqpoint{1.943690in}{2.980769in}}{\pgfqpoint{1.943690in}{2.972532in}}%
\pgfpathcurveto{\pgfqpoint{1.943690in}{2.964296in}}{\pgfqpoint{1.946963in}{2.956396in}}{\pgfqpoint{1.952787in}{2.950572in}}%
\pgfpathcurveto{\pgfqpoint{1.958611in}{2.944748in}}{\pgfqpoint{1.966511in}{2.941476in}}{\pgfqpoint{1.974747in}{2.941476in}}%
\pgfpathclose%
\pgfusepath{stroke,fill}%
\end{pgfscope}%
\begin{pgfscope}%
\pgfpathrectangle{\pgfqpoint{0.100000in}{0.212622in}}{\pgfqpoint{3.696000in}{3.696000in}}%
\pgfusepath{clip}%
\pgfsetbuttcap%
\pgfsetroundjoin%
\definecolor{currentfill}{rgb}{0.121569,0.466667,0.705882}%
\pgfsetfillcolor{currentfill}%
\pgfsetfillopacity{0.397118}%
\pgfsetlinewidth{1.003750pt}%
\definecolor{currentstroke}{rgb}{0.121569,0.466667,0.705882}%
\pgfsetstrokecolor{currentstroke}%
\pgfsetstrokeopacity{0.397118}%
\pgfsetdash{}{0pt}%
\pgfpathmoveto{\pgfqpoint{1.582053in}{2.841834in}}%
\pgfpathcurveto{\pgfqpoint{1.590289in}{2.841834in}}{\pgfqpoint{1.598189in}{2.845106in}}{\pgfqpoint{1.604013in}{2.850930in}}%
\pgfpathcurveto{\pgfqpoint{1.609837in}{2.856754in}}{\pgfqpoint{1.613109in}{2.864654in}}{\pgfqpoint{1.613109in}{2.872891in}}%
\pgfpathcurveto{\pgfqpoint{1.613109in}{2.881127in}}{\pgfqpoint{1.609837in}{2.889027in}}{\pgfqpoint{1.604013in}{2.894851in}}%
\pgfpathcurveto{\pgfqpoint{1.598189in}{2.900675in}}{\pgfqpoint{1.590289in}{2.903947in}}{\pgfqpoint{1.582053in}{2.903947in}}%
\pgfpathcurveto{\pgfqpoint{1.573816in}{2.903947in}}{\pgfqpoint{1.565916in}{2.900675in}}{\pgfqpoint{1.560092in}{2.894851in}}%
\pgfpathcurveto{\pgfqpoint{1.554268in}{2.889027in}}{\pgfqpoint{1.550996in}{2.881127in}}{\pgfqpoint{1.550996in}{2.872891in}}%
\pgfpathcurveto{\pgfqpoint{1.550996in}{2.864654in}}{\pgfqpoint{1.554268in}{2.856754in}}{\pgfqpoint{1.560092in}{2.850930in}}%
\pgfpathcurveto{\pgfqpoint{1.565916in}{2.845106in}}{\pgfqpoint{1.573816in}{2.841834in}}{\pgfqpoint{1.582053in}{2.841834in}}%
\pgfpathclose%
\pgfusepath{stroke,fill}%
\end{pgfscope}%
\begin{pgfscope}%
\pgfpathrectangle{\pgfqpoint{0.100000in}{0.212622in}}{\pgfqpoint{3.696000in}{3.696000in}}%
\pgfusepath{clip}%
\pgfsetbuttcap%
\pgfsetroundjoin%
\definecolor{currentfill}{rgb}{0.121569,0.466667,0.705882}%
\pgfsetfillcolor{currentfill}%
\pgfsetfillopacity{0.398199}%
\pgfsetlinewidth{1.003750pt}%
\definecolor{currentstroke}{rgb}{0.121569,0.466667,0.705882}%
\pgfsetstrokecolor{currentstroke}%
\pgfsetstrokeopacity{0.398199}%
\pgfsetdash{}{0pt}%
\pgfpathmoveto{\pgfqpoint{1.579108in}{2.835371in}}%
\pgfpathcurveto{\pgfqpoint{1.587345in}{2.835371in}}{\pgfqpoint{1.595245in}{2.838643in}}{\pgfqpoint{1.601069in}{2.844467in}}%
\pgfpathcurveto{\pgfqpoint{1.606893in}{2.850291in}}{\pgfqpoint{1.610165in}{2.858191in}}{\pgfqpoint{1.610165in}{2.866427in}}%
\pgfpathcurveto{\pgfqpoint{1.610165in}{2.874664in}}{\pgfqpoint{1.606893in}{2.882564in}}{\pgfqpoint{1.601069in}{2.888388in}}%
\pgfpathcurveto{\pgfqpoint{1.595245in}{2.894211in}}{\pgfqpoint{1.587345in}{2.897484in}}{\pgfqpoint{1.579108in}{2.897484in}}%
\pgfpathcurveto{\pgfqpoint{1.570872in}{2.897484in}}{\pgfqpoint{1.562972in}{2.894211in}}{\pgfqpoint{1.557148in}{2.888388in}}%
\pgfpathcurveto{\pgfqpoint{1.551324in}{2.882564in}}{\pgfqpoint{1.548052in}{2.874664in}}{\pgfqpoint{1.548052in}{2.866427in}}%
\pgfpathcurveto{\pgfqpoint{1.548052in}{2.858191in}}{\pgfqpoint{1.551324in}{2.850291in}}{\pgfqpoint{1.557148in}{2.844467in}}%
\pgfpathcurveto{\pgfqpoint{1.562972in}{2.838643in}}{\pgfqpoint{1.570872in}{2.835371in}}{\pgfqpoint{1.579108in}{2.835371in}}%
\pgfpathclose%
\pgfusepath{stroke,fill}%
\end{pgfscope}%
\begin{pgfscope}%
\pgfpathrectangle{\pgfqpoint{0.100000in}{0.212622in}}{\pgfqpoint{3.696000in}{3.696000in}}%
\pgfusepath{clip}%
\pgfsetbuttcap%
\pgfsetroundjoin%
\definecolor{currentfill}{rgb}{0.121569,0.466667,0.705882}%
\pgfsetfillcolor{currentfill}%
\pgfsetfillopacity{0.398280}%
\pgfsetlinewidth{1.003750pt}%
\definecolor{currentstroke}{rgb}{0.121569,0.466667,0.705882}%
\pgfsetstrokecolor{currentstroke}%
\pgfsetstrokeopacity{0.398280}%
\pgfsetdash{}{0pt}%
\pgfpathmoveto{\pgfqpoint{1.976630in}{2.931614in}}%
\pgfpathcurveto{\pgfqpoint{1.984866in}{2.931614in}}{\pgfqpoint{1.992766in}{2.934886in}}{\pgfqpoint{1.998590in}{2.940710in}}%
\pgfpathcurveto{\pgfqpoint{2.004414in}{2.946534in}}{\pgfqpoint{2.007686in}{2.954434in}}{\pgfqpoint{2.007686in}{2.962670in}}%
\pgfpathcurveto{\pgfqpoint{2.007686in}{2.970907in}}{\pgfqpoint{2.004414in}{2.978807in}}{\pgfqpoint{1.998590in}{2.984631in}}%
\pgfpathcurveto{\pgfqpoint{1.992766in}{2.990454in}}{\pgfqpoint{1.984866in}{2.993727in}}{\pgfqpoint{1.976630in}{2.993727in}}%
\pgfpathcurveto{\pgfqpoint{1.968393in}{2.993727in}}{\pgfqpoint{1.960493in}{2.990454in}}{\pgfqpoint{1.954669in}{2.984631in}}%
\pgfpathcurveto{\pgfqpoint{1.948846in}{2.978807in}}{\pgfqpoint{1.945573in}{2.970907in}}{\pgfqpoint{1.945573in}{2.962670in}}%
\pgfpathcurveto{\pgfqpoint{1.945573in}{2.954434in}}{\pgfqpoint{1.948846in}{2.946534in}}{\pgfqpoint{1.954669in}{2.940710in}}%
\pgfpathcurveto{\pgfqpoint{1.960493in}{2.934886in}}{\pgfqpoint{1.968393in}{2.931614in}}{\pgfqpoint{1.976630in}{2.931614in}}%
\pgfpathclose%
\pgfusepath{stroke,fill}%
\end{pgfscope}%
\begin{pgfscope}%
\pgfpathrectangle{\pgfqpoint{0.100000in}{0.212622in}}{\pgfqpoint{3.696000in}{3.696000in}}%
\pgfusepath{clip}%
\pgfsetbuttcap%
\pgfsetroundjoin%
\definecolor{currentfill}{rgb}{0.121569,0.466667,0.705882}%
\pgfsetfillcolor{currentfill}%
\pgfsetfillopacity{0.398875}%
\pgfsetlinewidth{1.003750pt}%
\definecolor{currentstroke}{rgb}{0.121569,0.466667,0.705882}%
\pgfsetstrokecolor{currentstroke}%
\pgfsetstrokeopacity{0.398875}%
\pgfsetdash{}{0pt}%
\pgfpathmoveto{\pgfqpoint{1.576858in}{2.831779in}}%
\pgfpathcurveto{\pgfqpoint{1.585094in}{2.831779in}}{\pgfqpoint{1.592994in}{2.835052in}}{\pgfqpoint{1.598818in}{2.840875in}}%
\pgfpathcurveto{\pgfqpoint{1.604642in}{2.846699in}}{\pgfqpoint{1.607914in}{2.854599in}}{\pgfqpoint{1.607914in}{2.862836in}}%
\pgfpathcurveto{\pgfqpoint{1.607914in}{2.871072in}}{\pgfqpoint{1.604642in}{2.878972in}}{\pgfqpoint{1.598818in}{2.884796in}}%
\pgfpathcurveto{\pgfqpoint{1.592994in}{2.890620in}}{\pgfqpoint{1.585094in}{2.893892in}}{\pgfqpoint{1.576858in}{2.893892in}}%
\pgfpathcurveto{\pgfqpoint{1.568621in}{2.893892in}}{\pgfqpoint{1.560721in}{2.890620in}}{\pgfqpoint{1.554897in}{2.884796in}}%
\pgfpathcurveto{\pgfqpoint{1.549073in}{2.878972in}}{\pgfqpoint{1.545801in}{2.871072in}}{\pgfqpoint{1.545801in}{2.862836in}}%
\pgfpathcurveto{\pgfqpoint{1.545801in}{2.854599in}}{\pgfqpoint{1.549073in}{2.846699in}}{\pgfqpoint{1.554897in}{2.840875in}}%
\pgfpathcurveto{\pgfqpoint{1.560721in}{2.835052in}}{\pgfqpoint{1.568621in}{2.831779in}}{\pgfqpoint{1.576858in}{2.831779in}}%
\pgfpathclose%
\pgfusepath{stroke,fill}%
\end{pgfscope}%
\begin{pgfscope}%
\pgfpathrectangle{\pgfqpoint{0.100000in}{0.212622in}}{\pgfqpoint{3.696000in}{3.696000in}}%
\pgfusepath{clip}%
\pgfsetbuttcap%
\pgfsetroundjoin%
\definecolor{currentfill}{rgb}{0.121569,0.466667,0.705882}%
\pgfsetfillcolor{currentfill}%
\pgfsetfillopacity{0.399465}%
\pgfsetlinewidth{1.003750pt}%
\definecolor{currentstroke}{rgb}{0.121569,0.466667,0.705882}%
\pgfsetstrokecolor{currentstroke}%
\pgfsetstrokeopacity{0.399465}%
\pgfsetdash{}{0pt}%
\pgfpathmoveto{\pgfqpoint{1.575165in}{2.828641in}}%
\pgfpathcurveto{\pgfqpoint{1.583402in}{2.828641in}}{\pgfqpoint{1.591302in}{2.831913in}}{\pgfqpoint{1.597126in}{2.837737in}}%
\pgfpathcurveto{\pgfqpoint{1.602949in}{2.843561in}}{\pgfqpoint{1.606222in}{2.851461in}}{\pgfqpoint{1.606222in}{2.859697in}}%
\pgfpathcurveto{\pgfqpoint{1.606222in}{2.867934in}}{\pgfqpoint{1.602949in}{2.875834in}}{\pgfqpoint{1.597126in}{2.881658in}}%
\pgfpathcurveto{\pgfqpoint{1.591302in}{2.887482in}}{\pgfqpoint{1.583402in}{2.890754in}}{\pgfqpoint{1.575165in}{2.890754in}}%
\pgfpathcurveto{\pgfqpoint{1.566929in}{2.890754in}}{\pgfqpoint{1.559029in}{2.887482in}}{\pgfqpoint{1.553205in}{2.881658in}}%
\pgfpathcurveto{\pgfqpoint{1.547381in}{2.875834in}}{\pgfqpoint{1.544109in}{2.867934in}}{\pgfqpoint{1.544109in}{2.859697in}}%
\pgfpathcurveto{\pgfqpoint{1.544109in}{2.851461in}}{\pgfqpoint{1.547381in}{2.843561in}}{\pgfqpoint{1.553205in}{2.837737in}}%
\pgfpathcurveto{\pgfqpoint{1.559029in}{2.831913in}}{\pgfqpoint{1.566929in}{2.828641in}}{\pgfqpoint{1.575165in}{2.828641in}}%
\pgfpathclose%
\pgfusepath{stroke,fill}%
\end{pgfscope}%
\begin{pgfscope}%
\pgfpathrectangle{\pgfqpoint{0.100000in}{0.212622in}}{\pgfqpoint{3.696000in}{3.696000in}}%
\pgfusepath{clip}%
\pgfsetbuttcap%
\pgfsetroundjoin%
\definecolor{currentfill}{rgb}{0.121569,0.466667,0.705882}%
\pgfsetfillcolor{currentfill}%
\pgfsetfillopacity{0.400570}%
\pgfsetlinewidth{1.003750pt}%
\definecolor{currentstroke}{rgb}{0.121569,0.466667,0.705882}%
\pgfsetstrokecolor{currentstroke}%
\pgfsetstrokeopacity{0.400570}%
\pgfsetdash{}{0pt}%
\pgfpathmoveto{\pgfqpoint{1.571986in}{2.823193in}}%
\pgfpathcurveto{\pgfqpoint{1.580222in}{2.823193in}}{\pgfqpoint{1.588122in}{2.826465in}}{\pgfqpoint{1.593946in}{2.832289in}}%
\pgfpathcurveto{\pgfqpoint{1.599770in}{2.838113in}}{\pgfqpoint{1.603042in}{2.846013in}}{\pgfqpoint{1.603042in}{2.854249in}}%
\pgfpathcurveto{\pgfqpoint{1.603042in}{2.862485in}}{\pgfqpoint{1.599770in}{2.870385in}}{\pgfqpoint{1.593946in}{2.876209in}}%
\pgfpathcurveto{\pgfqpoint{1.588122in}{2.882033in}}{\pgfqpoint{1.580222in}{2.885306in}}{\pgfqpoint{1.571986in}{2.885306in}}%
\pgfpathcurveto{\pgfqpoint{1.563749in}{2.885306in}}{\pgfqpoint{1.555849in}{2.882033in}}{\pgfqpoint{1.550025in}{2.876209in}}%
\pgfpathcurveto{\pgfqpoint{1.544201in}{2.870385in}}{\pgfqpoint{1.540929in}{2.862485in}}{\pgfqpoint{1.540929in}{2.854249in}}%
\pgfpathcurveto{\pgfqpoint{1.540929in}{2.846013in}}{\pgfqpoint{1.544201in}{2.838113in}}{\pgfqpoint{1.550025in}{2.832289in}}%
\pgfpathcurveto{\pgfqpoint{1.555849in}{2.826465in}}{\pgfqpoint{1.563749in}{2.823193in}}{\pgfqpoint{1.571986in}{2.823193in}}%
\pgfpathclose%
\pgfusepath{stroke,fill}%
\end{pgfscope}%
\begin{pgfscope}%
\pgfpathrectangle{\pgfqpoint{0.100000in}{0.212622in}}{\pgfqpoint{3.696000in}{3.696000in}}%
\pgfusepath{clip}%
\pgfsetbuttcap%
\pgfsetroundjoin%
\definecolor{currentfill}{rgb}{0.121569,0.466667,0.705882}%
\pgfsetfillcolor{currentfill}%
\pgfsetfillopacity{0.400833}%
\pgfsetlinewidth{1.003750pt}%
\definecolor{currentstroke}{rgb}{0.121569,0.466667,0.705882}%
\pgfsetstrokecolor{currentstroke}%
\pgfsetstrokeopacity{0.400833}%
\pgfsetdash{}{0pt}%
\pgfpathmoveto{\pgfqpoint{1.978139in}{2.920853in}}%
\pgfpathcurveto{\pgfqpoint{1.986375in}{2.920853in}}{\pgfqpoint{1.994275in}{2.924125in}}{\pgfqpoint{2.000099in}{2.929949in}}%
\pgfpathcurveto{\pgfqpoint{2.005923in}{2.935773in}}{\pgfqpoint{2.009195in}{2.943673in}}{\pgfqpoint{2.009195in}{2.951909in}}%
\pgfpathcurveto{\pgfqpoint{2.009195in}{2.960146in}}{\pgfqpoint{2.005923in}{2.968046in}}{\pgfqpoint{2.000099in}{2.973870in}}%
\pgfpathcurveto{\pgfqpoint{1.994275in}{2.979694in}}{\pgfqpoint{1.986375in}{2.982966in}}{\pgfqpoint{1.978139in}{2.982966in}}%
\pgfpathcurveto{\pgfqpoint{1.969903in}{2.982966in}}{\pgfqpoint{1.962003in}{2.979694in}}{\pgfqpoint{1.956179in}{2.973870in}}%
\pgfpathcurveto{\pgfqpoint{1.950355in}{2.968046in}}{\pgfqpoint{1.947082in}{2.960146in}}{\pgfqpoint{1.947082in}{2.951909in}}%
\pgfpathcurveto{\pgfqpoint{1.947082in}{2.943673in}}{\pgfqpoint{1.950355in}{2.935773in}}{\pgfqpoint{1.956179in}{2.929949in}}%
\pgfpathcurveto{\pgfqpoint{1.962003in}{2.924125in}}{\pgfqpoint{1.969903in}{2.920853in}}{\pgfqpoint{1.978139in}{2.920853in}}%
\pgfpathclose%
\pgfusepath{stroke,fill}%
\end{pgfscope}%
\begin{pgfscope}%
\pgfpathrectangle{\pgfqpoint{0.100000in}{0.212622in}}{\pgfqpoint{3.696000in}{3.696000in}}%
\pgfusepath{clip}%
\pgfsetbuttcap%
\pgfsetroundjoin%
\definecolor{currentfill}{rgb}{0.121569,0.466667,0.705882}%
\pgfsetfillcolor{currentfill}%
\pgfsetfillopacity{0.401464}%
\pgfsetlinewidth{1.003750pt}%
\definecolor{currentstroke}{rgb}{0.121569,0.466667,0.705882}%
\pgfsetstrokecolor{currentstroke}%
\pgfsetstrokeopacity{0.401464}%
\pgfsetdash{}{0pt}%
\pgfpathmoveto{\pgfqpoint{1.568686in}{2.817970in}}%
\pgfpathcurveto{\pgfqpoint{1.576922in}{2.817970in}}{\pgfqpoint{1.584822in}{2.821242in}}{\pgfqpoint{1.590646in}{2.827066in}}%
\pgfpathcurveto{\pgfqpoint{1.596470in}{2.832890in}}{\pgfqpoint{1.599742in}{2.840790in}}{\pgfqpoint{1.599742in}{2.849026in}}%
\pgfpathcurveto{\pgfqpoint{1.599742in}{2.857262in}}{\pgfqpoint{1.596470in}{2.865162in}}{\pgfqpoint{1.590646in}{2.870986in}}%
\pgfpathcurveto{\pgfqpoint{1.584822in}{2.876810in}}{\pgfqpoint{1.576922in}{2.880083in}}{\pgfqpoint{1.568686in}{2.880083in}}%
\pgfpathcurveto{\pgfqpoint{1.560449in}{2.880083in}}{\pgfqpoint{1.552549in}{2.876810in}}{\pgfqpoint{1.546725in}{2.870986in}}%
\pgfpathcurveto{\pgfqpoint{1.540902in}{2.865162in}}{\pgfqpoint{1.537629in}{2.857262in}}{\pgfqpoint{1.537629in}{2.849026in}}%
\pgfpathcurveto{\pgfqpoint{1.537629in}{2.840790in}}{\pgfqpoint{1.540902in}{2.832890in}}{\pgfqpoint{1.546725in}{2.827066in}}%
\pgfpathcurveto{\pgfqpoint{1.552549in}{2.821242in}}{\pgfqpoint{1.560449in}{2.817970in}}{\pgfqpoint{1.568686in}{2.817970in}}%
\pgfpathclose%
\pgfusepath{stroke,fill}%
\end{pgfscope}%
\begin{pgfscope}%
\pgfpathrectangle{\pgfqpoint{0.100000in}{0.212622in}}{\pgfqpoint{3.696000in}{3.696000in}}%
\pgfusepath{clip}%
\pgfsetbuttcap%
\pgfsetroundjoin%
\definecolor{currentfill}{rgb}{0.121569,0.466667,0.705882}%
\pgfsetfillcolor{currentfill}%
\pgfsetfillopacity{0.402243}%
\pgfsetlinewidth{1.003750pt}%
\definecolor{currentstroke}{rgb}{0.121569,0.466667,0.705882}%
\pgfsetstrokecolor{currentstroke}%
\pgfsetstrokeopacity{0.402243}%
\pgfsetdash{}{0pt}%
\pgfpathmoveto{\pgfqpoint{1.566640in}{2.813496in}}%
\pgfpathcurveto{\pgfqpoint{1.574876in}{2.813496in}}{\pgfqpoint{1.582776in}{2.816768in}}{\pgfqpoint{1.588600in}{2.822592in}}%
\pgfpathcurveto{\pgfqpoint{1.594424in}{2.828416in}}{\pgfqpoint{1.597697in}{2.836316in}}{\pgfqpoint{1.597697in}{2.844553in}}%
\pgfpathcurveto{\pgfqpoint{1.597697in}{2.852789in}}{\pgfqpoint{1.594424in}{2.860689in}}{\pgfqpoint{1.588600in}{2.866513in}}%
\pgfpathcurveto{\pgfqpoint{1.582776in}{2.872337in}}{\pgfqpoint{1.574876in}{2.875609in}}{\pgfqpoint{1.566640in}{2.875609in}}%
\pgfpathcurveto{\pgfqpoint{1.558404in}{2.875609in}}{\pgfqpoint{1.550504in}{2.872337in}}{\pgfqpoint{1.544680in}{2.866513in}}%
\pgfpathcurveto{\pgfqpoint{1.538856in}{2.860689in}}{\pgfqpoint{1.535584in}{2.852789in}}{\pgfqpoint{1.535584in}{2.844553in}}%
\pgfpathcurveto{\pgfqpoint{1.535584in}{2.836316in}}{\pgfqpoint{1.538856in}{2.828416in}}{\pgfqpoint{1.544680in}{2.822592in}}%
\pgfpathcurveto{\pgfqpoint{1.550504in}{2.816768in}}{\pgfqpoint{1.558404in}{2.813496in}}{\pgfqpoint{1.566640in}{2.813496in}}%
\pgfpathclose%
\pgfusepath{stroke,fill}%
\end{pgfscope}%
\begin{pgfscope}%
\pgfpathrectangle{\pgfqpoint{0.100000in}{0.212622in}}{\pgfqpoint{3.696000in}{3.696000in}}%
\pgfusepath{clip}%
\pgfsetbuttcap%
\pgfsetroundjoin%
\definecolor{currentfill}{rgb}{0.121569,0.466667,0.705882}%
\pgfsetfillcolor{currentfill}%
\pgfsetfillopacity{0.402492}%
\pgfsetlinewidth{1.003750pt}%
\definecolor{currentstroke}{rgb}{0.121569,0.466667,0.705882}%
\pgfsetstrokecolor{currentstroke}%
\pgfsetstrokeopacity{0.402492}%
\pgfsetdash{}{0pt}%
\pgfpathmoveto{\pgfqpoint{1.565837in}{2.812205in}}%
\pgfpathcurveto{\pgfqpoint{1.574073in}{2.812205in}}{\pgfqpoint{1.581973in}{2.815478in}}{\pgfqpoint{1.587797in}{2.821302in}}%
\pgfpathcurveto{\pgfqpoint{1.593621in}{2.827126in}}{\pgfqpoint{1.596893in}{2.835026in}}{\pgfqpoint{1.596893in}{2.843262in}}%
\pgfpathcurveto{\pgfqpoint{1.596893in}{2.851498in}}{\pgfqpoint{1.593621in}{2.859398in}}{\pgfqpoint{1.587797in}{2.865222in}}%
\pgfpathcurveto{\pgfqpoint{1.581973in}{2.871046in}}{\pgfqpoint{1.574073in}{2.874318in}}{\pgfqpoint{1.565837in}{2.874318in}}%
\pgfpathcurveto{\pgfqpoint{1.557601in}{2.874318in}}{\pgfqpoint{1.549701in}{2.871046in}}{\pgfqpoint{1.543877in}{2.865222in}}%
\pgfpathcurveto{\pgfqpoint{1.538053in}{2.859398in}}{\pgfqpoint{1.534780in}{2.851498in}}{\pgfqpoint{1.534780in}{2.843262in}}%
\pgfpathcurveto{\pgfqpoint{1.534780in}{2.835026in}}{\pgfqpoint{1.538053in}{2.827126in}}{\pgfqpoint{1.543877in}{2.821302in}}%
\pgfpathcurveto{\pgfqpoint{1.549701in}{2.815478in}}{\pgfqpoint{1.557601in}{2.812205in}}{\pgfqpoint{1.565837in}{2.812205in}}%
\pgfpathclose%
\pgfusepath{stroke,fill}%
\end{pgfscope}%
\begin{pgfscope}%
\pgfpathrectangle{\pgfqpoint{0.100000in}{0.212622in}}{\pgfqpoint{3.696000in}{3.696000in}}%
\pgfusepath{clip}%
\pgfsetbuttcap%
\pgfsetroundjoin%
\definecolor{currentfill}{rgb}{0.121569,0.466667,0.705882}%
\pgfsetfillcolor{currentfill}%
\pgfsetfillopacity{0.402678}%
\pgfsetlinewidth{1.003750pt}%
\definecolor{currentstroke}{rgb}{0.121569,0.466667,0.705882}%
\pgfsetstrokecolor{currentstroke}%
\pgfsetstrokeopacity{0.402678}%
\pgfsetdash{}{0pt}%
\pgfpathmoveto{\pgfqpoint{1.565290in}{2.811234in}}%
\pgfpathcurveto{\pgfqpoint{1.573526in}{2.811234in}}{\pgfqpoint{1.581426in}{2.814507in}}{\pgfqpoint{1.587250in}{2.820330in}}%
\pgfpathcurveto{\pgfqpoint{1.593074in}{2.826154in}}{\pgfqpoint{1.596347in}{2.834054in}}{\pgfqpoint{1.596347in}{2.842291in}}%
\pgfpathcurveto{\pgfqpoint{1.596347in}{2.850527in}}{\pgfqpoint{1.593074in}{2.858427in}}{\pgfqpoint{1.587250in}{2.864251in}}%
\pgfpathcurveto{\pgfqpoint{1.581426in}{2.870075in}}{\pgfqpoint{1.573526in}{2.873347in}}{\pgfqpoint{1.565290in}{2.873347in}}%
\pgfpathcurveto{\pgfqpoint{1.557054in}{2.873347in}}{\pgfqpoint{1.549154in}{2.870075in}}{\pgfqpoint{1.543330in}{2.864251in}}%
\pgfpathcurveto{\pgfqpoint{1.537506in}{2.858427in}}{\pgfqpoint{1.534234in}{2.850527in}}{\pgfqpoint{1.534234in}{2.842291in}}%
\pgfpathcurveto{\pgfqpoint{1.534234in}{2.834054in}}{\pgfqpoint{1.537506in}{2.826154in}}{\pgfqpoint{1.543330in}{2.820330in}}%
\pgfpathcurveto{\pgfqpoint{1.549154in}{2.814507in}}{\pgfqpoint{1.557054in}{2.811234in}}{\pgfqpoint{1.565290in}{2.811234in}}%
\pgfpathclose%
\pgfusepath{stroke,fill}%
\end{pgfscope}%
\begin{pgfscope}%
\pgfpathrectangle{\pgfqpoint{0.100000in}{0.212622in}}{\pgfqpoint{3.696000in}{3.696000in}}%
\pgfusepath{clip}%
\pgfsetbuttcap%
\pgfsetroundjoin%
\definecolor{currentfill}{rgb}{0.121569,0.466667,0.705882}%
\pgfsetfillcolor{currentfill}%
\pgfsetfillopacity{0.403007}%
\pgfsetlinewidth{1.003750pt}%
\definecolor{currentstroke}{rgb}{0.121569,0.466667,0.705882}%
\pgfsetstrokecolor{currentstroke}%
\pgfsetstrokeopacity{0.403007}%
\pgfsetdash{}{0pt}%
\pgfpathmoveto{\pgfqpoint{1.564310in}{2.809415in}}%
\pgfpathcurveto{\pgfqpoint{1.572547in}{2.809415in}}{\pgfqpoint{1.580447in}{2.812688in}}{\pgfqpoint{1.586271in}{2.818512in}}%
\pgfpathcurveto{\pgfqpoint{1.592095in}{2.824336in}}{\pgfqpoint{1.595367in}{2.832236in}}{\pgfqpoint{1.595367in}{2.840472in}}%
\pgfpathcurveto{\pgfqpoint{1.595367in}{2.848708in}}{\pgfqpoint{1.592095in}{2.856608in}}{\pgfqpoint{1.586271in}{2.862432in}}%
\pgfpathcurveto{\pgfqpoint{1.580447in}{2.868256in}}{\pgfqpoint{1.572547in}{2.871528in}}{\pgfqpoint{1.564310in}{2.871528in}}%
\pgfpathcurveto{\pgfqpoint{1.556074in}{2.871528in}}{\pgfqpoint{1.548174in}{2.868256in}}{\pgfqpoint{1.542350in}{2.862432in}}%
\pgfpathcurveto{\pgfqpoint{1.536526in}{2.856608in}}{\pgfqpoint{1.533254in}{2.848708in}}{\pgfqpoint{1.533254in}{2.840472in}}%
\pgfpathcurveto{\pgfqpoint{1.533254in}{2.832236in}}{\pgfqpoint{1.536526in}{2.824336in}}{\pgfqpoint{1.542350in}{2.818512in}}%
\pgfpathcurveto{\pgfqpoint{1.548174in}{2.812688in}}{\pgfqpoint{1.556074in}{2.809415in}}{\pgfqpoint{1.564310in}{2.809415in}}%
\pgfpathclose%
\pgfusepath{stroke,fill}%
\end{pgfscope}%
\begin{pgfscope}%
\pgfpathrectangle{\pgfqpoint{0.100000in}{0.212622in}}{\pgfqpoint{3.696000in}{3.696000in}}%
\pgfusepath{clip}%
\pgfsetbuttcap%
\pgfsetroundjoin%
\definecolor{currentfill}{rgb}{0.121569,0.466667,0.705882}%
\pgfsetfillcolor{currentfill}%
\pgfsetfillopacity{0.403531}%
\pgfsetlinewidth{1.003750pt}%
\definecolor{currentstroke}{rgb}{0.121569,0.466667,0.705882}%
\pgfsetstrokecolor{currentstroke}%
\pgfsetstrokeopacity{0.403531}%
\pgfsetdash{}{0pt}%
\pgfpathmoveto{\pgfqpoint{1.562313in}{2.806139in}}%
\pgfpathcurveto{\pgfqpoint{1.570550in}{2.806139in}}{\pgfqpoint{1.578450in}{2.809411in}}{\pgfqpoint{1.584274in}{2.815235in}}%
\pgfpathcurveto{\pgfqpoint{1.590098in}{2.821059in}}{\pgfqpoint{1.593370in}{2.828959in}}{\pgfqpoint{1.593370in}{2.837195in}}%
\pgfpathcurveto{\pgfqpoint{1.593370in}{2.845432in}}{\pgfqpoint{1.590098in}{2.853332in}}{\pgfqpoint{1.584274in}{2.859156in}}%
\pgfpathcurveto{\pgfqpoint{1.578450in}{2.864980in}}{\pgfqpoint{1.570550in}{2.868252in}}{\pgfqpoint{1.562313in}{2.868252in}}%
\pgfpathcurveto{\pgfqpoint{1.554077in}{2.868252in}}{\pgfqpoint{1.546177in}{2.864980in}}{\pgfqpoint{1.540353in}{2.859156in}}%
\pgfpathcurveto{\pgfqpoint{1.534529in}{2.853332in}}{\pgfqpoint{1.531257in}{2.845432in}}{\pgfqpoint{1.531257in}{2.837195in}}%
\pgfpathcurveto{\pgfqpoint{1.531257in}{2.828959in}}{\pgfqpoint{1.534529in}{2.821059in}}{\pgfqpoint{1.540353in}{2.815235in}}%
\pgfpathcurveto{\pgfqpoint{1.546177in}{2.809411in}}{\pgfqpoint{1.554077in}{2.806139in}}{\pgfqpoint{1.562313in}{2.806139in}}%
\pgfpathclose%
\pgfusepath{stroke,fill}%
\end{pgfscope}%
\begin{pgfscope}%
\pgfpathrectangle{\pgfqpoint{0.100000in}{0.212622in}}{\pgfqpoint{3.696000in}{3.696000in}}%
\pgfusepath{clip}%
\pgfsetbuttcap%
\pgfsetroundjoin%
\definecolor{currentfill}{rgb}{0.121569,0.466667,0.705882}%
\pgfsetfillcolor{currentfill}%
\pgfsetfillopacity{0.403795}%
\pgfsetlinewidth{1.003750pt}%
\definecolor{currentstroke}{rgb}{0.121569,0.466667,0.705882}%
\pgfsetstrokecolor{currentstroke}%
\pgfsetstrokeopacity{0.403795}%
\pgfsetdash{}{0pt}%
\pgfpathmoveto{\pgfqpoint{1.561578in}{2.804505in}}%
\pgfpathcurveto{\pgfqpoint{1.569814in}{2.804505in}}{\pgfqpoint{1.577715in}{2.807777in}}{\pgfqpoint{1.583538in}{2.813601in}}%
\pgfpathcurveto{\pgfqpoint{1.589362in}{2.819425in}}{\pgfqpoint{1.592635in}{2.827325in}}{\pgfqpoint{1.592635in}{2.835561in}}%
\pgfpathcurveto{\pgfqpoint{1.592635in}{2.843797in}}{\pgfqpoint{1.589362in}{2.851697in}}{\pgfqpoint{1.583538in}{2.857521in}}%
\pgfpathcurveto{\pgfqpoint{1.577715in}{2.863345in}}{\pgfqpoint{1.569814in}{2.866618in}}{\pgfqpoint{1.561578in}{2.866618in}}%
\pgfpathcurveto{\pgfqpoint{1.553342in}{2.866618in}}{\pgfqpoint{1.545442in}{2.863345in}}{\pgfqpoint{1.539618in}{2.857521in}}%
\pgfpathcurveto{\pgfqpoint{1.533794in}{2.851697in}}{\pgfqpoint{1.530522in}{2.843797in}}{\pgfqpoint{1.530522in}{2.835561in}}%
\pgfpathcurveto{\pgfqpoint{1.530522in}{2.827325in}}{\pgfqpoint{1.533794in}{2.819425in}}{\pgfqpoint{1.539618in}{2.813601in}}%
\pgfpathcurveto{\pgfqpoint{1.545442in}{2.807777in}}{\pgfqpoint{1.553342in}{2.804505in}}{\pgfqpoint{1.561578in}{2.804505in}}%
\pgfpathclose%
\pgfusepath{stroke,fill}%
\end{pgfscope}%
\begin{pgfscope}%
\pgfpathrectangle{\pgfqpoint{0.100000in}{0.212622in}}{\pgfqpoint{3.696000in}{3.696000in}}%
\pgfusepath{clip}%
\pgfsetbuttcap%
\pgfsetroundjoin%
\definecolor{currentfill}{rgb}{0.121569,0.466667,0.705882}%
\pgfsetfillcolor{currentfill}%
\pgfsetfillopacity{0.403799}%
\pgfsetlinewidth{1.003750pt}%
\definecolor{currentstroke}{rgb}{0.121569,0.466667,0.705882}%
\pgfsetstrokecolor{currentstroke}%
\pgfsetstrokeopacity{0.403799}%
\pgfsetdash{}{0pt}%
\pgfpathmoveto{\pgfqpoint{1.979299in}{2.909624in}}%
\pgfpathcurveto{\pgfqpoint{1.987536in}{2.909624in}}{\pgfqpoint{1.995436in}{2.912896in}}{\pgfqpoint{2.001260in}{2.918720in}}%
\pgfpathcurveto{\pgfqpoint{2.007083in}{2.924544in}}{\pgfqpoint{2.010356in}{2.932444in}}{\pgfqpoint{2.010356in}{2.940680in}}%
\pgfpathcurveto{\pgfqpoint{2.010356in}{2.948916in}}{\pgfqpoint{2.007083in}{2.956816in}}{\pgfqpoint{2.001260in}{2.962640in}}%
\pgfpathcurveto{\pgfqpoint{1.995436in}{2.968464in}}{\pgfqpoint{1.987536in}{2.971737in}}{\pgfqpoint{1.979299in}{2.971737in}}%
\pgfpathcurveto{\pgfqpoint{1.971063in}{2.971737in}}{\pgfqpoint{1.963163in}{2.968464in}}{\pgfqpoint{1.957339in}{2.962640in}}%
\pgfpathcurveto{\pgfqpoint{1.951515in}{2.956816in}}{\pgfqpoint{1.948243in}{2.948916in}}{\pgfqpoint{1.948243in}{2.940680in}}%
\pgfpathcurveto{\pgfqpoint{1.948243in}{2.932444in}}{\pgfqpoint{1.951515in}{2.924544in}}{\pgfqpoint{1.957339in}{2.918720in}}%
\pgfpathcurveto{\pgfqpoint{1.963163in}{2.912896in}}{\pgfqpoint{1.971063in}{2.909624in}}{\pgfqpoint{1.979299in}{2.909624in}}%
\pgfpathclose%
\pgfusepath{stroke,fill}%
\end{pgfscope}%
\begin{pgfscope}%
\pgfpathrectangle{\pgfqpoint{0.100000in}{0.212622in}}{\pgfqpoint{3.696000in}{3.696000in}}%
\pgfusepath{clip}%
\pgfsetbuttcap%
\pgfsetroundjoin%
\definecolor{currentfill}{rgb}{0.121569,0.466667,0.705882}%
\pgfsetfillcolor{currentfill}%
\pgfsetfillopacity{0.404312}%
\pgfsetlinewidth{1.003750pt}%
\definecolor{currentstroke}{rgb}{0.121569,0.466667,0.705882}%
\pgfsetstrokecolor{currentstroke}%
\pgfsetstrokeopacity{0.404312}%
\pgfsetdash{}{0pt}%
\pgfpathmoveto{\pgfqpoint{1.560023in}{2.801948in}}%
\pgfpathcurveto{\pgfqpoint{1.568260in}{2.801948in}}{\pgfqpoint{1.576160in}{2.805220in}}{\pgfqpoint{1.581984in}{2.811044in}}%
\pgfpathcurveto{\pgfqpoint{1.587807in}{2.816868in}}{\pgfqpoint{1.591080in}{2.824768in}}{\pgfqpoint{1.591080in}{2.833005in}}%
\pgfpathcurveto{\pgfqpoint{1.591080in}{2.841241in}}{\pgfqpoint{1.587807in}{2.849141in}}{\pgfqpoint{1.581984in}{2.854965in}}%
\pgfpathcurveto{\pgfqpoint{1.576160in}{2.860789in}}{\pgfqpoint{1.568260in}{2.864061in}}{\pgfqpoint{1.560023in}{2.864061in}}%
\pgfpathcurveto{\pgfqpoint{1.551787in}{2.864061in}}{\pgfqpoint{1.543887in}{2.860789in}}{\pgfqpoint{1.538063in}{2.854965in}}%
\pgfpathcurveto{\pgfqpoint{1.532239in}{2.849141in}}{\pgfqpoint{1.528967in}{2.841241in}}{\pgfqpoint{1.528967in}{2.833005in}}%
\pgfpathcurveto{\pgfqpoint{1.528967in}{2.824768in}}{\pgfqpoint{1.532239in}{2.816868in}}{\pgfqpoint{1.538063in}{2.811044in}}%
\pgfpathcurveto{\pgfqpoint{1.543887in}{2.805220in}}{\pgfqpoint{1.551787in}{2.801948in}}{\pgfqpoint{1.560023in}{2.801948in}}%
\pgfpathclose%
\pgfusepath{stroke,fill}%
\end{pgfscope}%
\begin{pgfscope}%
\pgfpathrectangle{\pgfqpoint{0.100000in}{0.212622in}}{\pgfqpoint{3.696000in}{3.696000in}}%
\pgfusepath{clip}%
\pgfsetbuttcap%
\pgfsetroundjoin%
\definecolor{currentfill}{rgb}{0.121569,0.466667,0.705882}%
\pgfsetfillcolor{currentfill}%
\pgfsetfillopacity{0.405240}%
\pgfsetlinewidth{1.003750pt}%
\definecolor{currentstroke}{rgb}{0.121569,0.466667,0.705882}%
\pgfsetstrokecolor{currentstroke}%
\pgfsetstrokeopacity{0.405240}%
\pgfsetdash{}{0pt}%
\pgfpathmoveto{\pgfqpoint{1.557205in}{2.797240in}}%
\pgfpathcurveto{\pgfqpoint{1.565442in}{2.797240in}}{\pgfqpoint{1.573342in}{2.800512in}}{\pgfqpoint{1.579166in}{2.806336in}}%
\pgfpathcurveto{\pgfqpoint{1.584990in}{2.812160in}}{\pgfqpoint{1.588262in}{2.820060in}}{\pgfqpoint{1.588262in}{2.828296in}}%
\pgfpathcurveto{\pgfqpoint{1.588262in}{2.836533in}}{\pgfqpoint{1.584990in}{2.844433in}}{\pgfqpoint{1.579166in}{2.850257in}}%
\pgfpathcurveto{\pgfqpoint{1.573342in}{2.856081in}}{\pgfqpoint{1.565442in}{2.859353in}}{\pgfqpoint{1.557205in}{2.859353in}}%
\pgfpathcurveto{\pgfqpoint{1.548969in}{2.859353in}}{\pgfqpoint{1.541069in}{2.856081in}}{\pgfqpoint{1.535245in}{2.850257in}}%
\pgfpathcurveto{\pgfqpoint{1.529421in}{2.844433in}}{\pgfqpoint{1.526149in}{2.836533in}}{\pgfqpoint{1.526149in}{2.828296in}}%
\pgfpathcurveto{\pgfqpoint{1.526149in}{2.820060in}}{\pgfqpoint{1.529421in}{2.812160in}}{\pgfqpoint{1.535245in}{2.806336in}}%
\pgfpathcurveto{\pgfqpoint{1.541069in}{2.800512in}}{\pgfqpoint{1.548969in}{2.797240in}}{\pgfqpoint{1.557205in}{2.797240in}}%
\pgfpathclose%
\pgfusepath{stroke,fill}%
\end{pgfscope}%
\begin{pgfscope}%
\pgfpathrectangle{\pgfqpoint{0.100000in}{0.212622in}}{\pgfqpoint{3.696000in}{3.696000in}}%
\pgfusepath{clip}%
\pgfsetbuttcap%
\pgfsetroundjoin%
\definecolor{currentfill}{rgb}{0.121569,0.466667,0.705882}%
\pgfsetfillcolor{currentfill}%
\pgfsetfillopacity{0.406100}%
\pgfsetlinewidth{1.003750pt}%
\definecolor{currentstroke}{rgb}{0.121569,0.466667,0.705882}%
\pgfsetstrokecolor{currentstroke}%
\pgfsetstrokeopacity{0.406100}%
\pgfsetdash{}{0pt}%
\pgfpathmoveto{\pgfqpoint{1.554894in}{2.792949in}}%
\pgfpathcurveto{\pgfqpoint{1.563131in}{2.792949in}}{\pgfqpoint{1.571031in}{2.796221in}}{\pgfqpoint{1.576855in}{2.802045in}}%
\pgfpathcurveto{\pgfqpoint{1.582679in}{2.807869in}}{\pgfqpoint{1.585951in}{2.815769in}}{\pgfqpoint{1.585951in}{2.824005in}}%
\pgfpathcurveto{\pgfqpoint{1.585951in}{2.832241in}}{\pgfqpoint{1.582679in}{2.840142in}}{\pgfqpoint{1.576855in}{2.845965in}}%
\pgfpathcurveto{\pgfqpoint{1.571031in}{2.851789in}}{\pgfqpoint{1.563131in}{2.855062in}}{\pgfqpoint{1.554894in}{2.855062in}}%
\pgfpathcurveto{\pgfqpoint{1.546658in}{2.855062in}}{\pgfqpoint{1.538758in}{2.851789in}}{\pgfqpoint{1.532934in}{2.845965in}}%
\pgfpathcurveto{\pgfqpoint{1.527110in}{2.840142in}}{\pgfqpoint{1.523838in}{2.832241in}}{\pgfqpoint{1.523838in}{2.824005in}}%
\pgfpathcurveto{\pgfqpoint{1.523838in}{2.815769in}}{\pgfqpoint{1.527110in}{2.807869in}}{\pgfqpoint{1.532934in}{2.802045in}}%
\pgfpathcurveto{\pgfqpoint{1.538758in}{2.796221in}}{\pgfqpoint{1.546658in}{2.792949in}}{\pgfqpoint{1.554894in}{2.792949in}}%
\pgfpathclose%
\pgfusepath{stroke,fill}%
\end{pgfscope}%
\begin{pgfscope}%
\pgfpathrectangle{\pgfqpoint{0.100000in}{0.212622in}}{\pgfqpoint{3.696000in}{3.696000in}}%
\pgfusepath{clip}%
\pgfsetbuttcap%
\pgfsetroundjoin%
\definecolor{currentfill}{rgb}{0.121569,0.466667,0.705882}%
\pgfsetfillcolor{currentfill}%
\pgfsetfillopacity{0.406552}%
\pgfsetlinewidth{1.003750pt}%
\definecolor{currentstroke}{rgb}{0.121569,0.466667,0.705882}%
\pgfsetstrokecolor{currentstroke}%
\pgfsetstrokeopacity{0.406552}%
\pgfsetdash{}{0pt}%
\pgfpathmoveto{\pgfqpoint{1.553327in}{2.790540in}}%
\pgfpathcurveto{\pgfqpoint{1.561563in}{2.790540in}}{\pgfqpoint{1.569463in}{2.793813in}}{\pgfqpoint{1.575287in}{2.799637in}}%
\pgfpathcurveto{\pgfqpoint{1.581111in}{2.805460in}}{\pgfqpoint{1.584384in}{2.813361in}}{\pgfqpoint{1.584384in}{2.821597in}}%
\pgfpathcurveto{\pgfqpoint{1.584384in}{2.829833in}}{\pgfqpoint{1.581111in}{2.837733in}}{\pgfqpoint{1.575287in}{2.843557in}}%
\pgfpathcurveto{\pgfqpoint{1.569463in}{2.849381in}}{\pgfqpoint{1.561563in}{2.852653in}}{\pgfqpoint{1.553327in}{2.852653in}}%
\pgfpathcurveto{\pgfqpoint{1.545091in}{2.852653in}}{\pgfqpoint{1.537191in}{2.849381in}}{\pgfqpoint{1.531367in}{2.843557in}}%
\pgfpathcurveto{\pgfqpoint{1.525543in}{2.837733in}}{\pgfqpoint{1.522271in}{2.829833in}}{\pgfqpoint{1.522271in}{2.821597in}}%
\pgfpathcurveto{\pgfqpoint{1.522271in}{2.813361in}}{\pgfqpoint{1.525543in}{2.805460in}}{\pgfqpoint{1.531367in}{2.799637in}}%
\pgfpathcurveto{\pgfqpoint{1.537191in}{2.793813in}}{\pgfqpoint{1.545091in}{2.790540in}}{\pgfqpoint{1.553327in}{2.790540in}}%
\pgfpathclose%
\pgfusepath{stroke,fill}%
\end{pgfscope}%
\begin{pgfscope}%
\pgfpathrectangle{\pgfqpoint{0.100000in}{0.212622in}}{\pgfqpoint{3.696000in}{3.696000in}}%
\pgfusepath{clip}%
\pgfsetbuttcap%
\pgfsetroundjoin%
\definecolor{currentfill}{rgb}{0.121569,0.466667,0.705882}%
\pgfsetfillcolor{currentfill}%
\pgfsetfillopacity{0.406883}%
\pgfsetlinewidth{1.003750pt}%
\definecolor{currentstroke}{rgb}{0.121569,0.466667,0.705882}%
\pgfsetstrokecolor{currentstroke}%
\pgfsetstrokeopacity{0.406883}%
\pgfsetdash{}{0pt}%
\pgfpathmoveto{\pgfqpoint{1.981628in}{2.896166in}}%
\pgfpathcurveto{\pgfqpoint{1.989864in}{2.896166in}}{\pgfqpoint{1.997764in}{2.899438in}}{\pgfqpoint{2.003588in}{2.905262in}}%
\pgfpathcurveto{\pgfqpoint{2.009412in}{2.911086in}}{\pgfqpoint{2.012684in}{2.918986in}}{\pgfqpoint{2.012684in}{2.927222in}}%
\pgfpathcurveto{\pgfqpoint{2.012684in}{2.935459in}}{\pgfqpoint{2.009412in}{2.943359in}}{\pgfqpoint{2.003588in}{2.949183in}}%
\pgfpathcurveto{\pgfqpoint{1.997764in}{2.955007in}}{\pgfqpoint{1.989864in}{2.958279in}}{\pgfqpoint{1.981628in}{2.958279in}}%
\pgfpathcurveto{\pgfqpoint{1.973391in}{2.958279in}}{\pgfqpoint{1.965491in}{2.955007in}}{\pgfqpoint{1.959667in}{2.949183in}}%
\pgfpathcurveto{\pgfqpoint{1.953843in}{2.943359in}}{\pgfqpoint{1.950571in}{2.935459in}}{\pgfqpoint{1.950571in}{2.927222in}}%
\pgfpathcurveto{\pgfqpoint{1.950571in}{2.918986in}}{\pgfqpoint{1.953843in}{2.911086in}}{\pgfqpoint{1.959667in}{2.905262in}}%
\pgfpathcurveto{\pgfqpoint{1.965491in}{2.899438in}}{\pgfqpoint{1.973391in}{2.896166in}}{\pgfqpoint{1.981628in}{2.896166in}}%
\pgfpathclose%
\pgfusepath{stroke,fill}%
\end{pgfscope}%
\begin{pgfscope}%
\pgfpathrectangle{\pgfqpoint{0.100000in}{0.212622in}}{\pgfqpoint{3.696000in}{3.696000in}}%
\pgfusepath{clip}%
\pgfsetbuttcap%
\pgfsetroundjoin%
\definecolor{currentfill}{rgb}{0.121569,0.466667,0.705882}%
\pgfsetfillcolor{currentfill}%
\pgfsetfillopacity{0.406889}%
\pgfsetlinewidth{1.003750pt}%
\definecolor{currentstroke}{rgb}{0.121569,0.466667,0.705882}%
\pgfsetstrokecolor{currentstroke}%
\pgfsetstrokeopacity{0.406889}%
\pgfsetdash{}{0pt}%
\pgfpathmoveto{\pgfqpoint{1.552395in}{2.788750in}}%
\pgfpathcurveto{\pgfqpoint{1.560631in}{2.788750in}}{\pgfqpoint{1.568531in}{2.792023in}}{\pgfqpoint{1.574355in}{2.797847in}}%
\pgfpathcurveto{\pgfqpoint{1.580179in}{2.803671in}}{\pgfqpoint{1.583451in}{2.811571in}}{\pgfqpoint{1.583451in}{2.819807in}}%
\pgfpathcurveto{\pgfqpoint{1.583451in}{2.828043in}}{\pgfqpoint{1.580179in}{2.835943in}}{\pgfqpoint{1.574355in}{2.841767in}}%
\pgfpathcurveto{\pgfqpoint{1.568531in}{2.847591in}}{\pgfqpoint{1.560631in}{2.850863in}}{\pgfqpoint{1.552395in}{2.850863in}}%
\pgfpathcurveto{\pgfqpoint{1.544159in}{2.850863in}}{\pgfqpoint{1.536259in}{2.847591in}}{\pgfqpoint{1.530435in}{2.841767in}}%
\pgfpathcurveto{\pgfqpoint{1.524611in}{2.835943in}}{\pgfqpoint{1.521338in}{2.828043in}}{\pgfqpoint{1.521338in}{2.819807in}}%
\pgfpathcurveto{\pgfqpoint{1.521338in}{2.811571in}}{\pgfqpoint{1.524611in}{2.803671in}}{\pgfqpoint{1.530435in}{2.797847in}}%
\pgfpathcurveto{\pgfqpoint{1.536259in}{2.792023in}}{\pgfqpoint{1.544159in}{2.788750in}}{\pgfqpoint{1.552395in}{2.788750in}}%
\pgfpathclose%
\pgfusepath{stroke,fill}%
\end{pgfscope}%
\begin{pgfscope}%
\pgfpathrectangle{\pgfqpoint{0.100000in}{0.212622in}}{\pgfqpoint{3.696000in}{3.696000in}}%
\pgfusepath{clip}%
\pgfsetbuttcap%
\pgfsetroundjoin%
\definecolor{currentfill}{rgb}{0.121569,0.466667,0.705882}%
\pgfsetfillcolor{currentfill}%
\pgfsetfillopacity{0.407472}%
\pgfsetlinewidth{1.003750pt}%
\definecolor{currentstroke}{rgb}{0.121569,0.466667,0.705882}%
\pgfsetstrokecolor{currentstroke}%
\pgfsetstrokeopacity{0.407472}%
\pgfsetdash{}{0pt}%
\pgfpathmoveto{\pgfqpoint{1.550584in}{2.785534in}}%
\pgfpathcurveto{\pgfqpoint{1.558820in}{2.785534in}}{\pgfqpoint{1.566720in}{2.788806in}}{\pgfqpoint{1.572544in}{2.794630in}}%
\pgfpathcurveto{\pgfqpoint{1.578368in}{2.800454in}}{\pgfqpoint{1.581640in}{2.808354in}}{\pgfqpoint{1.581640in}{2.816591in}}%
\pgfpathcurveto{\pgfqpoint{1.581640in}{2.824827in}}{\pgfqpoint{1.578368in}{2.832727in}}{\pgfqpoint{1.572544in}{2.838551in}}%
\pgfpathcurveto{\pgfqpoint{1.566720in}{2.844375in}}{\pgfqpoint{1.558820in}{2.847647in}}{\pgfqpoint{1.550584in}{2.847647in}}%
\pgfpathcurveto{\pgfqpoint{1.542348in}{2.847647in}}{\pgfqpoint{1.534448in}{2.844375in}}{\pgfqpoint{1.528624in}{2.838551in}}%
\pgfpathcurveto{\pgfqpoint{1.522800in}{2.832727in}}{\pgfqpoint{1.519527in}{2.824827in}}{\pgfqpoint{1.519527in}{2.816591in}}%
\pgfpathcurveto{\pgfqpoint{1.519527in}{2.808354in}}{\pgfqpoint{1.522800in}{2.800454in}}{\pgfqpoint{1.528624in}{2.794630in}}%
\pgfpathcurveto{\pgfqpoint{1.534448in}{2.788806in}}{\pgfqpoint{1.542348in}{2.785534in}}{\pgfqpoint{1.550584in}{2.785534in}}%
\pgfpathclose%
\pgfusepath{stroke,fill}%
\end{pgfscope}%
\begin{pgfscope}%
\pgfpathrectangle{\pgfqpoint{0.100000in}{0.212622in}}{\pgfqpoint{3.696000in}{3.696000in}}%
\pgfusepath{clip}%
\pgfsetbuttcap%
\pgfsetroundjoin%
\definecolor{currentfill}{rgb}{0.121569,0.466667,0.705882}%
\pgfsetfillcolor{currentfill}%
\pgfsetfillopacity{0.408390}%
\pgfsetlinewidth{1.003750pt}%
\definecolor{currentstroke}{rgb}{0.121569,0.466667,0.705882}%
\pgfsetstrokecolor{currentstroke}%
\pgfsetstrokeopacity{0.408390}%
\pgfsetdash{}{0pt}%
\pgfpathmoveto{\pgfqpoint{1.546995in}{2.779595in}}%
\pgfpathcurveto{\pgfqpoint{1.555231in}{2.779595in}}{\pgfqpoint{1.563131in}{2.782867in}}{\pgfqpoint{1.568955in}{2.788691in}}%
\pgfpathcurveto{\pgfqpoint{1.574779in}{2.794515in}}{\pgfqpoint{1.578051in}{2.802415in}}{\pgfqpoint{1.578051in}{2.810651in}}%
\pgfpathcurveto{\pgfqpoint{1.578051in}{2.818888in}}{\pgfqpoint{1.574779in}{2.826788in}}{\pgfqpoint{1.568955in}{2.832612in}}%
\pgfpathcurveto{\pgfqpoint{1.563131in}{2.838436in}}{\pgfqpoint{1.555231in}{2.841708in}}{\pgfqpoint{1.546995in}{2.841708in}}%
\pgfpathcurveto{\pgfqpoint{1.538758in}{2.841708in}}{\pgfqpoint{1.530858in}{2.838436in}}{\pgfqpoint{1.525034in}{2.832612in}}%
\pgfpathcurveto{\pgfqpoint{1.519210in}{2.826788in}}{\pgfqpoint{1.515938in}{2.818888in}}{\pgfqpoint{1.515938in}{2.810651in}}%
\pgfpathcurveto{\pgfqpoint{1.515938in}{2.802415in}}{\pgfqpoint{1.519210in}{2.794515in}}{\pgfqpoint{1.525034in}{2.788691in}}%
\pgfpathcurveto{\pgfqpoint{1.530858in}{2.782867in}}{\pgfqpoint{1.538758in}{2.779595in}}{\pgfqpoint{1.546995in}{2.779595in}}%
\pgfpathclose%
\pgfusepath{stroke,fill}%
\end{pgfscope}%
\begin{pgfscope}%
\pgfpathrectangle{\pgfqpoint{0.100000in}{0.212622in}}{\pgfqpoint{3.696000in}{3.696000in}}%
\pgfusepath{clip}%
\pgfsetbuttcap%
\pgfsetroundjoin%
\definecolor{currentfill}{rgb}{0.121569,0.466667,0.705882}%
\pgfsetfillcolor{currentfill}%
\pgfsetfillopacity{0.409042}%
\pgfsetlinewidth{1.003750pt}%
\definecolor{currentstroke}{rgb}{0.121569,0.466667,0.705882}%
\pgfsetstrokecolor{currentstroke}%
\pgfsetstrokeopacity{0.409042}%
\pgfsetdash{}{0pt}%
\pgfpathmoveto{\pgfqpoint{1.545139in}{2.775338in}}%
\pgfpathcurveto{\pgfqpoint{1.553375in}{2.775338in}}{\pgfqpoint{1.561275in}{2.778610in}}{\pgfqpoint{1.567099in}{2.784434in}}%
\pgfpathcurveto{\pgfqpoint{1.572923in}{2.790258in}}{\pgfqpoint{1.576196in}{2.798158in}}{\pgfqpoint{1.576196in}{2.806394in}}%
\pgfpathcurveto{\pgfqpoint{1.576196in}{2.814630in}}{\pgfqpoint{1.572923in}{2.822530in}}{\pgfqpoint{1.567099in}{2.828354in}}%
\pgfpathcurveto{\pgfqpoint{1.561275in}{2.834178in}}{\pgfqpoint{1.553375in}{2.837451in}}{\pgfqpoint{1.545139in}{2.837451in}}%
\pgfpathcurveto{\pgfqpoint{1.536903in}{2.837451in}}{\pgfqpoint{1.529003in}{2.834178in}}{\pgfqpoint{1.523179in}{2.828354in}}%
\pgfpathcurveto{\pgfqpoint{1.517355in}{2.822530in}}{\pgfqpoint{1.514083in}{2.814630in}}{\pgfqpoint{1.514083in}{2.806394in}}%
\pgfpathcurveto{\pgfqpoint{1.514083in}{2.798158in}}{\pgfqpoint{1.517355in}{2.790258in}}{\pgfqpoint{1.523179in}{2.784434in}}%
\pgfpathcurveto{\pgfqpoint{1.529003in}{2.778610in}}{\pgfqpoint{1.536903in}{2.775338in}}{\pgfqpoint{1.545139in}{2.775338in}}%
\pgfpathclose%
\pgfusepath{stroke,fill}%
\end{pgfscope}%
\begin{pgfscope}%
\pgfpathrectangle{\pgfqpoint{0.100000in}{0.212622in}}{\pgfqpoint{3.696000in}{3.696000in}}%
\pgfusepath{clip}%
\pgfsetbuttcap%
\pgfsetroundjoin%
\definecolor{currentfill}{rgb}{0.121569,0.466667,0.705882}%
\pgfsetfillcolor{currentfill}%
\pgfsetfillopacity{0.409625}%
\pgfsetlinewidth{1.003750pt}%
\definecolor{currentstroke}{rgb}{0.121569,0.466667,0.705882}%
\pgfsetstrokecolor{currentstroke}%
\pgfsetstrokeopacity{0.409625}%
\pgfsetdash{}{0pt}%
\pgfpathmoveto{\pgfqpoint{1.543324in}{2.772276in}}%
\pgfpathcurveto{\pgfqpoint{1.551560in}{2.772276in}}{\pgfqpoint{1.559460in}{2.775548in}}{\pgfqpoint{1.565284in}{2.781372in}}%
\pgfpathcurveto{\pgfqpoint{1.571108in}{2.787196in}}{\pgfqpoint{1.574381in}{2.795096in}}{\pgfqpoint{1.574381in}{2.803332in}}%
\pgfpathcurveto{\pgfqpoint{1.574381in}{2.811569in}}{\pgfqpoint{1.571108in}{2.819469in}}{\pgfqpoint{1.565284in}{2.825293in}}%
\pgfpathcurveto{\pgfqpoint{1.559460in}{2.831117in}}{\pgfqpoint{1.551560in}{2.834389in}}{\pgfqpoint{1.543324in}{2.834389in}}%
\pgfpathcurveto{\pgfqpoint{1.535088in}{2.834389in}}{\pgfqpoint{1.527188in}{2.831117in}}{\pgfqpoint{1.521364in}{2.825293in}}%
\pgfpathcurveto{\pgfqpoint{1.515540in}{2.819469in}}{\pgfqpoint{1.512268in}{2.811569in}}{\pgfqpoint{1.512268in}{2.803332in}}%
\pgfpathcurveto{\pgfqpoint{1.512268in}{2.795096in}}{\pgfqpoint{1.515540in}{2.787196in}}{\pgfqpoint{1.521364in}{2.781372in}}%
\pgfpathcurveto{\pgfqpoint{1.527188in}{2.775548in}}{\pgfqpoint{1.535088in}{2.772276in}}{\pgfqpoint{1.543324in}{2.772276in}}%
\pgfpathclose%
\pgfusepath{stroke,fill}%
\end{pgfscope}%
\begin{pgfscope}%
\pgfpathrectangle{\pgfqpoint{0.100000in}{0.212622in}}{\pgfqpoint{3.696000in}{3.696000in}}%
\pgfusepath{clip}%
\pgfsetbuttcap%
\pgfsetroundjoin%
\definecolor{currentfill}{rgb}{0.121569,0.466667,0.705882}%
\pgfsetfillcolor{currentfill}%
\pgfsetfillopacity{0.410176}%
\pgfsetlinewidth{1.003750pt}%
\definecolor{currentstroke}{rgb}{0.121569,0.466667,0.705882}%
\pgfsetstrokecolor{currentstroke}%
\pgfsetstrokeopacity{0.410176}%
\pgfsetdash{}{0pt}%
\pgfpathmoveto{\pgfqpoint{1.983693in}{2.882269in}}%
\pgfpathcurveto{\pgfqpoint{1.991930in}{2.882269in}}{\pgfqpoint{1.999830in}{2.885541in}}{\pgfqpoint{2.005654in}{2.891365in}}%
\pgfpathcurveto{\pgfqpoint{2.011478in}{2.897189in}}{\pgfqpoint{2.014750in}{2.905089in}}{\pgfqpoint{2.014750in}{2.913325in}}%
\pgfpathcurveto{\pgfqpoint{2.014750in}{2.921561in}}{\pgfqpoint{2.011478in}{2.929462in}}{\pgfqpoint{2.005654in}{2.935285in}}%
\pgfpathcurveto{\pgfqpoint{1.999830in}{2.941109in}}{\pgfqpoint{1.991930in}{2.944382in}}{\pgfqpoint{1.983693in}{2.944382in}}%
\pgfpathcurveto{\pgfqpoint{1.975457in}{2.944382in}}{\pgfqpoint{1.967557in}{2.941109in}}{\pgfqpoint{1.961733in}{2.935285in}}%
\pgfpathcurveto{\pgfqpoint{1.955909in}{2.929462in}}{\pgfqpoint{1.952637in}{2.921561in}}{\pgfqpoint{1.952637in}{2.913325in}}%
\pgfpathcurveto{\pgfqpoint{1.952637in}{2.905089in}}{\pgfqpoint{1.955909in}{2.897189in}}{\pgfqpoint{1.961733in}{2.891365in}}%
\pgfpathcurveto{\pgfqpoint{1.967557in}{2.885541in}}{\pgfqpoint{1.975457in}{2.882269in}}{\pgfqpoint{1.983693in}{2.882269in}}%
\pgfpathclose%
\pgfusepath{stroke,fill}%
\end{pgfscope}%
\begin{pgfscope}%
\pgfpathrectangle{\pgfqpoint{0.100000in}{0.212622in}}{\pgfqpoint{3.696000in}{3.696000in}}%
\pgfusepath{clip}%
\pgfsetbuttcap%
\pgfsetroundjoin%
\definecolor{currentfill}{rgb}{0.121569,0.466667,0.705882}%
\pgfsetfillcolor{currentfill}%
\pgfsetfillopacity{0.410675}%
\pgfsetlinewidth{1.003750pt}%
\definecolor{currentstroke}{rgb}{0.121569,0.466667,0.705882}%
\pgfsetstrokecolor{currentstroke}%
\pgfsetstrokeopacity{0.410675}%
\pgfsetdash{}{0pt}%
\pgfpathmoveto{\pgfqpoint{1.540037in}{2.766650in}}%
\pgfpathcurveto{\pgfqpoint{1.548273in}{2.766650in}}{\pgfqpoint{1.556173in}{2.769922in}}{\pgfqpoint{1.561997in}{2.775746in}}%
\pgfpathcurveto{\pgfqpoint{1.567821in}{2.781570in}}{\pgfqpoint{1.571093in}{2.789470in}}{\pgfqpoint{1.571093in}{2.797706in}}%
\pgfpathcurveto{\pgfqpoint{1.571093in}{2.805942in}}{\pgfqpoint{1.567821in}{2.813842in}}{\pgfqpoint{1.561997in}{2.819666in}}%
\pgfpathcurveto{\pgfqpoint{1.556173in}{2.825490in}}{\pgfqpoint{1.548273in}{2.828763in}}{\pgfqpoint{1.540037in}{2.828763in}}%
\pgfpathcurveto{\pgfqpoint{1.531800in}{2.828763in}}{\pgfqpoint{1.523900in}{2.825490in}}{\pgfqpoint{1.518077in}{2.819666in}}%
\pgfpathcurveto{\pgfqpoint{1.512253in}{2.813842in}}{\pgfqpoint{1.508980in}{2.805942in}}{\pgfqpoint{1.508980in}{2.797706in}}%
\pgfpathcurveto{\pgfqpoint{1.508980in}{2.789470in}}{\pgfqpoint{1.512253in}{2.781570in}}{\pgfqpoint{1.518077in}{2.775746in}}%
\pgfpathcurveto{\pgfqpoint{1.523900in}{2.769922in}}{\pgfqpoint{1.531800in}{2.766650in}}{\pgfqpoint{1.540037in}{2.766650in}}%
\pgfpathclose%
\pgfusepath{stroke,fill}%
\end{pgfscope}%
\begin{pgfscope}%
\pgfpathrectangle{\pgfqpoint{0.100000in}{0.212622in}}{\pgfqpoint{3.696000in}{3.696000in}}%
\pgfusepath{clip}%
\pgfsetbuttcap%
\pgfsetroundjoin%
\definecolor{currentfill}{rgb}{0.121569,0.466667,0.705882}%
\pgfsetfillcolor{currentfill}%
\pgfsetfillopacity{0.411709}%
\pgfsetlinewidth{1.003750pt}%
\definecolor{currentstroke}{rgb}{0.121569,0.466667,0.705882}%
\pgfsetstrokecolor{currentstroke}%
\pgfsetstrokeopacity{0.411709}%
\pgfsetdash{}{0pt}%
\pgfpathmoveto{\pgfqpoint{1.537295in}{2.761092in}}%
\pgfpathcurveto{\pgfqpoint{1.545531in}{2.761092in}}{\pgfqpoint{1.553431in}{2.764365in}}{\pgfqpoint{1.559255in}{2.770188in}}%
\pgfpathcurveto{\pgfqpoint{1.565079in}{2.776012in}}{\pgfqpoint{1.568351in}{2.783912in}}{\pgfqpoint{1.568351in}{2.792149in}}%
\pgfpathcurveto{\pgfqpoint{1.568351in}{2.800385in}}{\pgfqpoint{1.565079in}{2.808285in}}{\pgfqpoint{1.559255in}{2.814109in}}%
\pgfpathcurveto{\pgfqpoint{1.553431in}{2.819933in}}{\pgfqpoint{1.545531in}{2.823205in}}{\pgfqpoint{1.537295in}{2.823205in}}%
\pgfpathcurveto{\pgfqpoint{1.529059in}{2.823205in}}{\pgfqpoint{1.521159in}{2.819933in}}{\pgfqpoint{1.515335in}{2.814109in}}%
\pgfpathcurveto{\pgfqpoint{1.509511in}{2.808285in}}{\pgfqpoint{1.506238in}{2.800385in}}{\pgfqpoint{1.506238in}{2.792149in}}%
\pgfpathcurveto{\pgfqpoint{1.506238in}{2.783912in}}{\pgfqpoint{1.509511in}{2.776012in}}{\pgfqpoint{1.515335in}{2.770188in}}%
\pgfpathcurveto{\pgfqpoint{1.521159in}{2.764365in}}{\pgfqpoint{1.529059in}{2.761092in}}{\pgfqpoint{1.537295in}{2.761092in}}%
\pgfpathclose%
\pgfusepath{stroke,fill}%
\end{pgfscope}%
\begin{pgfscope}%
\pgfpathrectangle{\pgfqpoint{0.100000in}{0.212622in}}{\pgfqpoint{3.696000in}{3.696000in}}%
\pgfusepath{clip}%
\pgfsetbuttcap%
\pgfsetroundjoin%
\definecolor{currentfill}{rgb}{0.121569,0.466667,0.705882}%
\pgfsetfillcolor{currentfill}%
\pgfsetfillopacity{0.412156}%
\pgfsetlinewidth{1.003750pt}%
\definecolor{currentstroke}{rgb}{0.121569,0.466667,0.705882}%
\pgfsetstrokecolor{currentstroke}%
\pgfsetstrokeopacity{0.412156}%
\pgfsetdash{}{0pt}%
\pgfpathmoveto{\pgfqpoint{1.535772in}{2.758757in}}%
\pgfpathcurveto{\pgfqpoint{1.544008in}{2.758757in}}{\pgfqpoint{1.551908in}{2.762029in}}{\pgfqpoint{1.557732in}{2.767853in}}%
\pgfpathcurveto{\pgfqpoint{1.563556in}{2.773677in}}{\pgfqpoint{1.566828in}{2.781577in}}{\pgfqpoint{1.566828in}{2.789814in}}%
\pgfpathcurveto{\pgfqpoint{1.566828in}{2.798050in}}{\pgfqpoint{1.563556in}{2.805950in}}{\pgfqpoint{1.557732in}{2.811774in}}%
\pgfpathcurveto{\pgfqpoint{1.551908in}{2.817598in}}{\pgfqpoint{1.544008in}{2.820870in}}{\pgfqpoint{1.535772in}{2.820870in}}%
\pgfpathcurveto{\pgfqpoint{1.527535in}{2.820870in}}{\pgfqpoint{1.519635in}{2.817598in}}{\pgfqpoint{1.513811in}{2.811774in}}%
\pgfpathcurveto{\pgfqpoint{1.507987in}{2.805950in}}{\pgfqpoint{1.504715in}{2.798050in}}{\pgfqpoint{1.504715in}{2.789814in}}%
\pgfpathcurveto{\pgfqpoint{1.504715in}{2.781577in}}{\pgfqpoint{1.507987in}{2.773677in}}{\pgfqpoint{1.513811in}{2.767853in}}%
\pgfpathcurveto{\pgfqpoint{1.519635in}{2.762029in}}{\pgfqpoint{1.527535in}{2.758757in}}{\pgfqpoint{1.535772in}{2.758757in}}%
\pgfpathclose%
\pgfusepath{stroke,fill}%
\end{pgfscope}%
\begin{pgfscope}%
\pgfpathrectangle{\pgfqpoint{0.100000in}{0.212622in}}{\pgfqpoint{3.696000in}{3.696000in}}%
\pgfusepath{clip}%
\pgfsetbuttcap%
\pgfsetroundjoin%
\definecolor{currentfill}{rgb}{0.121569,0.466667,0.705882}%
\pgfsetfillcolor{currentfill}%
\pgfsetfillopacity{0.413023}%
\pgfsetlinewidth{1.003750pt}%
\definecolor{currentstroke}{rgb}{0.121569,0.466667,0.705882}%
\pgfsetstrokecolor{currentstroke}%
\pgfsetstrokeopacity{0.413023}%
\pgfsetdash{}{0pt}%
\pgfpathmoveto{\pgfqpoint{1.533369in}{2.754191in}}%
\pgfpathcurveto{\pgfqpoint{1.541605in}{2.754191in}}{\pgfqpoint{1.549505in}{2.757464in}}{\pgfqpoint{1.555329in}{2.763287in}}%
\pgfpathcurveto{\pgfqpoint{1.561153in}{2.769111in}}{\pgfqpoint{1.564425in}{2.777011in}}{\pgfqpoint{1.564425in}{2.785248in}}%
\pgfpathcurveto{\pgfqpoint{1.564425in}{2.793484in}}{\pgfqpoint{1.561153in}{2.801384in}}{\pgfqpoint{1.555329in}{2.807208in}}%
\pgfpathcurveto{\pgfqpoint{1.549505in}{2.813032in}}{\pgfqpoint{1.541605in}{2.816304in}}{\pgfqpoint{1.533369in}{2.816304in}}%
\pgfpathcurveto{\pgfqpoint{1.525133in}{2.816304in}}{\pgfqpoint{1.517233in}{2.813032in}}{\pgfqpoint{1.511409in}{2.807208in}}%
\pgfpathcurveto{\pgfqpoint{1.505585in}{2.801384in}}{\pgfqpoint{1.502312in}{2.793484in}}{\pgfqpoint{1.502312in}{2.785248in}}%
\pgfpathcurveto{\pgfqpoint{1.502312in}{2.777011in}}{\pgfqpoint{1.505585in}{2.769111in}}{\pgfqpoint{1.511409in}{2.763287in}}%
\pgfpathcurveto{\pgfqpoint{1.517233in}{2.757464in}}{\pgfqpoint{1.525133in}{2.754191in}}{\pgfqpoint{1.533369in}{2.754191in}}%
\pgfpathclose%
\pgfusepath{stroke,fill}%
\end{pgfscope}%
\begin{pgfscope}%
\pgfpathrectangle{\pgfqpoint{0.100000in}{0.212622in}}{\pgfqpoint{3.696000in}{3.696000in}}%
\pgfusepath{clip}%
\pgfsetbuttcap%
\pgfsetroundjoin%
\definecolor{currentfill}{rgb}{0.121569,0.466667,0.705882}%
\pgfsetfillcolor{currentfill}%
\pgfsetfillopacity{0.413816}%
\pgfsetlinewidth{1.003750pt}%
\definecolor{currentstroke}{rgb}{0.121569,0.466667,0.705882}%
\pgfsetstrokecolor{currentstroke}%
\pgfsetstrokeopacity{0.413816}%
\pgfsetdash{}{0pt}%
\pgfpathmoveto{\pgfqpoint{1.531009in}{2.750051in}}%
\pgfpathcurveto{\pgfqpoint{1.539245in}{2.750051in}}{\pgfqpoint{1.547145in}{2.753323in}}{\pgfqpoint{1.552969in}{2.759147in}}%
\pgfpathcurveto{\pgfqpoint{1.558793in}{2.764971in}}{\pgfqpoint{1.562066in}{2.772871in}}{\pgfqpoint{1.562066in}{2.781107in}}%
\pgfpathcurveto{\pgfqpoint{1.562066in}{2.789344in}}{\pgfqpoint{1.558793in}{2.797244in}}{\pgfqpoint{1.552969in}{2.803068in}}%
\pgfpathcurveto{\pgfqpoint{1.547145in}{2.808892in}}{\pgfqpoint{1.539245in}{2.812164in}}{\pgfqpoint{1.531009in}{2.812164in}}%
\pgfpathcurveto{\pgfqpoint{1.522773in}{2.812164in}}{\pgfqpoint{1.514873in}{2.808892in}}{\pgfqpoint{1.509049in}{2.803068in}}%
\pgfpathcurveto{\pgfqpoint{1.503225in}{2.797244in}}{\pgfqpoint{1.499953in}{2.789344in}}{\pgfqpoint{1.499953in}{2.781107in}}%
\pgfpathcurveto{\pgfqpoint{1.499953in}{2.772871in}}{\pgfqpoint{1.503225in}{2.764971in}}{\pgfqpoint{1.509049in}{2.759147in}}%
\pgfpathcurveto{\pgfqpoint{1.514873in}{2.753323in}}{\pgfqpoint{1.522773in}{2.750051in}}{\pgfqpoint{1.531009in}{2.750051in}}%
\pgfpathclose%
\pgfusepath{stroke,fill}%
\end{pgfscope}%
\begin{pgfscope}%
\pgfpathrectangle{\pgfqpoint{0.100000in}{0.212622in}}{\pgfqpoint{3.696000in}{3.696000in}}%
\pgfusepath{clip}%
\pgfsetbuttcap%
\pgfsetroundjoin%
\definecolor{currentfill}{rgb}{0.121569,0.466667,0.705882}%
\pgfsetfillcolor{currentfill}%
\pgfsetfillopacity{0.413906}%
\pgfsetlinewidth{1.003750pt}%
\definecolor{currentstroke}{rgb}{0.121569,0.466667,0.705882}%
\pgfsetstrokecolor{currentstroke}%
\pgfsetstrokeopacity{0.413906}%
\pgfsetdash{}{0pt}%
\pgfpathmoveto{\pgfqpoint{1.985144in}{2.868660in}}%
\pgfpathcurveto{\pgfqpoint{1.993380in}{2.868660in}}{\pgfqpoint{2.001280in}{2.871932in}}{\pgfqpoint{2.007104in}{2.877756in}}%
\pgfpathcurveto{\pgfqpoint{2.012928in}{2.883580in}}{\pgfqpoint{2.016201in}{2.891480in}}{\pgfqpoint{2.016201in}{2.899716in}}%
\pgfpathcurveto{\pgfqpoint{2.016201in}{2.907952in}}{\pgfqpoint{2.012928in}{2.915853in}}{\pgfqpoint{2.007104in}{2.921676in}}%
\pgfpathcurveto{\pgfqpoint{2.001280in}{2.927500in}}{\pgfqpoint{1.993380in}{2.930773in}}{\pgfqpoint{1.985144in}{2.930773in}}%
\pgfpathcurveto{\pgfqpoint{1.976908in}{2.930773in}}{\pgfqpoint{1.969008in}{2.927500in}}{\pgfqpoint{1.963184in}{2.921676in}}%
\pgfpathcurveto{\pgfqpoint{1.957360in}{2.915853in}}{\pgfqpoint{1.954088in}{2.907952in}}{\pgfqpoint{1.954088in}{2.899716in}}%
\pgfpathcurveto{\pgfqpoint{1.954088in}{2.891480in}}{\pgfqpoint{1.957360in}{2.883580in}}{\pgfqpoint{1.963184in}{2.877756in}}%
\pgfpathcurveto{\pgfqpoint{1.969008in}{2.871932in}}{\pgfqpoint{1.976908in}{2.868660in}}{\pgfqpoint{1.985144in}{2.868660in}}%
\pgfpathclose%
\pgfusepath{stroke,fill}%
\end{pgfscope}%
\begin{pgfscope}%
\pgfpathrectangle{\pgfqpoint{0.100000in}{0.212622in}}{\pgfqpoint{3.696000in}{3.696000in}}%
\pgfusepath{clip}%
\pgfsetbuttcap%
\pgfsetroundjoin%
\definecolor{currentfill}{rgb}{0.121569,0.466667,0.705882}%
\pgfsetfillcolor{currentfill}%
\pgfsetfillopacity{0.415074}%
\pgfsetlinewidth{1.003750pt}%
\definecolor{currentstroke}{rgb}{0.121569,0.466667,0.705882}%
\pgfsetstrokecolor{currentstroke}%
\pgfsetstrokeopacity{0.415074}%
\pgfsetdash{}{0pt}%
\pgfpathmoveto{\pgfqpoint{1.526438in}{2.742221in}}%
\pgfpathcurveto{\pgfqpoint{1.534675in}{2.742221in}}{\pgfqpoint{1.542575in}{2.745494in}}{\pgfqpoint{1.548398in}{2.751318in}}%
\pgfpathcurveto{\pgfqpoint{1.554222in}{2.757142in}}{\pgfqpoint{1.557495in}{2.765042in}}{\pgfqpoint{1.557495in}{2.773278in}}%
\pgfpathcurveto{\pgfqpoint{1.557495in}{2.781514in}}{\pgfqpoint{1.554222in}{2.789414in}}{\pgfqpoint{1.548398in}{2.795238in}}%
\pgfpathcurveto{\pgfqpoint{1.542575in}{2.801062in}}{\pgfqpoint{1.534675in}{2.804334in}}{\pgfqpoint{1.526438in}{2.804334in}}%
\pgfpathcurveto{\pgfqpoint{1.518202in}{2.804334in}}{\pgfqpoint{1.510302in}{2.801062in}}{\pgfqpoint{1.504478in}{2.795238in}}%
\pgfpathcurveto{\pgfqpoint{1.498654in}{2.789414in}}{\pgfqpoint{1.495382in}{2.781514in}}{\pgfqpoint{1.495382in}{2.773278in}}%
\pgfpathcurveto{\pgfqpoint{1.495382in}{2.765042in}}{\pgfqpoint{1.498654in}{2.757142in}}{\pgfqpoint{1.504478in}{2.751318in}}%
\pgfpathcurveto{\pgfqpoint{1.510302in}{2.745494in}}{\pgfqpoint{1.518202in}{2.742221in}}{\pgfqpoint{1.526438in}{2.742221in}}%
\pgfpathclose%
\pgfusepath{stroke,fill}%
\end{pgfscope}%
\begin{pgfscope}%
\pgfpathrectangle{\pgfqpoint{0.100000in}{0.212622in}}{\pgfqpoint{3.696000in}{3.696000in}}%
\pgfusepath{clip}%
\pgfsetbuttcap%
\pgfsetroundjoin%
\definecolor{currentfill}{rgb}{0.121569,0.466667,0.705882}%
\pgfsetfillcolor{currentfill}%
\pgfsetfillopacity{0.416284}%
\pgfsetlinewidth{1.003750pt}%
\definecolor{currentstroke}{rgb}{0.121569,0.466667,0.705882}%
\pgfsetstrokecolor{currentstroke}%
\pgfsetstrokeopacity{0.416284}%
\pgfsetdash{}{0pt}%
\pgfpathmoveto{\pgfqpoint{1.523091in}{2.734781in}}%
\pgfpathcurveto{\pgfqpoint{1.531328in}{2.734781in}}{\pgfqpoint{1.539228in}{2.738053in}}{\pgfqpoint{1.545052in}{2.743877in}}%
\pgfpathcurveto{\pgfqpoint{1.550875in}{2.749701in}}{\pgfqpoint{1.554148in}{2.757601in}}{\pgfqpoint{1.554148in}{2.765837in}}%
\pgfpathcurveto{\pgfqpoint{1.554148in}{2.774074in}}{\pgfqpoint{1.550875in}{2.781974in}}{\pgfqpoint{1.545052in}{2.787798in}}%
\pgfpathcurveto{\pgfqpoint{1.539228in}{2.793622in}}{\pgfqpoint{1.531328in}{2.796894in}}{\pgfqpoint{1.523091in}{2.796894in}}%
\pgfpathcurveto{\pgfqpoint{1.514855in}{2.796894in}}{\pgfqpoint{1.506955in}{2.793622in}}{\pgfqpoint{1.501131in}{2.787798in}}%
\pgfpathcurveto{\pgfqpoint{1.495307in}{2.781974in}}{\pgfqpoint{1.492035in}{2.774074in}}{\pgfqpoint{1.492035in}{2.765837in}}%
\pgfpathcurveto{\pgfqpoint{1.492035in}{2.757601in}}{\pgfqpoint{1.495307in}{2.749701in}}{\pgfqpoint{1.501131in}{2.743877in}}%
\pgfpathcurveto{\pgfqpoint{1.506955in}{2.738053in}}{\pgfqpoint{1.514855in}{2.734781in}}{\pgfqpoint{1.523091in}{2.734781in}}%
\pgfpathclose%
\pgfusepath{stroke,fill}%
\end{pgfscope}%
\begin{pgfscope}%
\pgfpathrectangle{\pgfqpoint{0.100000in}{0.212622in}}{\pgfqpoint{3.696000in}{3.696000in}}%
\pgfusepath{clip}%
\pgfsetbuttcap%
\pgfsetroundjoin%
\definecolor{currentfill}{rgb}{0.121569,0.466667,0.705882}%
\pgfsetfillcolor{currentfill}%
\pgfsetfillopacity{0.416911}%
\pgfsetlinewidth{1.003750pt}%
\definecolor{currentstroke}{rgb}{0.121569,0.466667,0.705882}%
\pgfsetstrokecolor{currentstroke}%
\pgfsetstrokeopacity{0.416911}%
\pgfsetdash{}{0pt}%
\pgfpathmoveto{\pgfqpoint{1.520821in}{2.731024in}}%
\pgfpathcurveto{\pgfqpoint{1.529058in}{2.731024in}}{\pgfqpoint{1.536958in}{2.734296in}}{\pgfqpoint{1.542782in}{2.740120in}}%
\pgfpathcurveto{\pgfqpoint{1.548606in}{2.745944in}}{\pgfqpoint{1.551878in}{2.753844in}}{\pgfqpoint{1.551878in}{2.762080in}}%
\pgfpathcurveto{\pgfqpoint{1.551878in}{2.770317in}}{\pgfqpoint{1.548606in}{2.778217in}}{\pgfqpoint{1.542782in}{2.784041in}}%
\pgfpathcurveto{\pgfqpoint{1.536958in}{2.789864in}}{\pgfqpoint{1.529058in}{2.793137in}}{\pgfqpoint{1.520821in}{2.793137in}}%
\pgfpathcurveto{\pgfqpoint{1.512585in}{2.793137in}}{\pgfqpoint{1.504685in}{2.789864in}}{\pgfqpoint{1.498861in}{2.784041in}}%
\pgfpathcurveto{\pgfqpoint{1.493037in}{2.778217in}}{\pgfqpoint{1.489765in}{2.770317in}}{\pgfqpoint{1.489765in}{2.762080in}}%
\pgfpathcurveto{\pgfqpoint{1.489765in}{2.753844in}}{\pgfqpoint{1.493037in}{2.745944in}}{\pgfqpoint{1.498861in}{2.740120in}}%
\pgfpathcurveto{\pgfqpoint{1.504685in}{2.734296in}}{\pgfqpoint{1.512585in}{2.731024in}}{\pgfqpoint{1.520821in}{2.731024in}}%
\pgfpathclose%
\pgfusepath{stroke,fill}%
\end{pgfscope}%
\begin{pgfscope}%
\pgfpathrectangle{\pgfqpoint{0.100000in}{0.212622in}}{\pgfqpoint{3.696000in}{3.696000in}}%
\pgfusepath{clip}%
\pgfsetbuttcap%
\pgfsetroundjoin%
\definecolor{currentfill}{rgb}{0.121569,0.466667,0.705882}%
\pgfsetfillcolor{currentfill}%
\pgfsetfillopacity{0.417160}%
\pgfsetlinewidth{1.003750pt}%
\definecolor{currentstroke}{rgb}{0.121569,0.466667,0.705882}%
\pgfsetstrokecolor{currentstroke}%
\pgfsetstrokeopacity{0.417160}%
\pgfsetdash{}{0pt}%
\pgfpathmoveto{\pgfqpoint{1.988076in}{2.853776in}}%
\pgfpathcurveto{\pgfqpoint{1.996312in}{2.853776in}}{\pgfqpoint{2.004212in}{2.857049in}}{\pgfqpoint{2.010036in}{2.862872in}}%
\pgfpathcurveto{\pgfqpoint{2.015860in}{2.868696in}}{\pgfqpoint{2.019132in}{2.876596in}}{\pgfqpoint{2.019132in}{2.884833in}}%
\pgfpathcurveto{\pgfqpoint{2.019132in}{2.893069in}}{\pgfqpoint{2.015860in}{2.900969in}}{\pgfqpoint{2.010036in}{2.906793in}}%
\pgfpathcurveto{\pgfqpoint{2.004212in}{2.912617in}}{\pgfqpoint{1.996312in}{2.915889in}}{\pgfqpoint{1.988076in}{2.915889in}}%
\pgfpathcurveto{\pgfqpoint{1.979840in}{2.915889in}}{\pgfqpoint{1.971940in}{2.912617in}}{\pgfqpoint{1.966116in}{2.906793in}}%
\pgfpathcurveto{\pgfqpoint{1.960292in}{2.900969in}}{\pgfqpoint{1.957019in}{2.893069in}}{\pgfqpoint{1.957019in}{2.884833in}}%
\pgfpathcurveto{\pgfqpoint{1.957019in}{2.876596in}}{\pgfqpoint{1.960292in}{2.868696in}}{\pgfqpoint{1.966116in}{2.862872in}}%
\pgfpathcurveto{\pgfqpoint{1.971940in}{2.857049in}}{\pgfqpoint{1.979840in}{2.853776in}}{\pgfqpoint{1.988076in}{2.853776in}}%
\pgfpathclose%
\pgfusepath{stroke,fill}%
\end{pgfscope}%
\begin{pgfscope}%
\pgfpathrectangle{\pgfqpoint{0.100000in}{0.212622in}}{\pgfqpoint{3.696000in}{3.696000in}}%
\pgfusepath{clip}%
\pgfsetbuttcap%
\pgfsetroundjoin%
\definecolor{currentfill}{rgb}{0.121569,0.466667,0.705882}%
\pgfsetfillcolor{currentfill}%
\pgfsetfillopacity{0.417438}%
\pgfsetlinewidth{1.003750pt}%
\definecolor{currentstroke}{rgb}{0.121569,0.466667,0.705882}%
\pgfsetstrokecolor{currentstroke}%
\pgfsetstrokeopacity{0.417438}%
\pgfsetdash{}{0pt}%
\pgfpathmoveto{\pgfqpoint{1.519317in}{2.727915in}}%
\pgfpathcurveto{\pgfqpoint{1.527554in}{2.727915in}}{\pgfqpoint{1.535454in}{2.731187in}}{\pgfqpoint{1.541278in}{2.737011in}}%
\pgfpathcurveto{\pgfqpoint{1.547102in}{2.742835in}}{\pgfqpoint{1.550374in}{2.750735in}}{\pgfqpoint{1.550374in}{2.758971in}}%
\pgfpathcurveto{\pgfqpoint{1.550374in}{2.767208in}}{\pgfqpoint{1.547102in}{2.775108in}}{\pgfqpoint{1.541278in}{2.780932in}}%
\pgfpathcurveto{\pgfqpoint{1.535454in}{2.786756in}}{\pgfqpoint{1.527554in}{2.790028in}}{\pgfqpoint{1.519317in}{2.790028in}}%
\pgfpathcurveto{\pgfqpoint{1.511081in}{2.790028in}}{\pgfqpoint{1.503181in}{2.786756in}}{\pgfqpoint{1.497357in}{2.780932in}}%
\pgfpathcurveto{\pgfqpoint{1.491533in}{2.775108in}}{\pgfqpoint{1.488261in}{2.767208in}}{\pgfqpoint{1.488261in}{2.758971in}}%
\pgfpathcurveto{\pgfqpoint{1.488261in}{2.750735in}}{\pgfqpoint{1.491533in}{2.742835in}}{\pgfqpoint{1.497357in}{2.737011in}}%
\pgfpathcurveto{\pgfqpoint{1.503181in}{2.731187in}}{\pgfqpoint{1.511081in}{2.727915in}}{\pgfqpoint{1.519317in}{2.727915in}}%
\pgfpathclose%
\pgfusepath{stroke,fill}%
\end{pgfscope}%
\begin{pgfscope}%
\pgfpathrectangle{\pgfqpoint{0.100000in}{0.212622in}}{\pgfqpoint{3.696000in}{3.696000in}}%
\pgfusepath{clip}%
\pgfsetbuttcap%
\pgfsetroundjoin%
\definecolor{currentfill}{rgb}{0.121569,0.466667,0.705882}%
\pgfsetfillcolor{currentfill}%
\pgfsetfillopacity{0.418405}%
\pgfsetlinewidth{1.003750pt}%
\definecolor{currentstroke}{rgb}{0.121569,0.466667,0.705882}%
\pgfsetstrokecolor{currentstroke}%
\pgfsetstrokeopacity{0.418405}%
\pgfsetdash{}{0pt}%
\pgfpathmoveto{\pgfqpoint{1.516343in}{2.722590in}}%
\pgfpathcurveto{\pgfqpoint{1.524580in}{2.722590in}}{\pgfqpoint{1.532480in}{2.725862in}}{\pgfqpoint{1.538304in}{2.731686in}}%
\pgfpathcurveto{\pgfqpoint{1.544128in}{2.737510in}}{\pgfqpoint{1.547400in}{2.745410in}}{\pgfqpoint{1.547400in}{2.753646in}}%
\pgfpathcurveto{\pgfqpoint{1.547400in}{2.761883in}}{\pgfqpoint{1.544128in}{2.769783in}}{\pgfqpoint{1.538304in}{2.775607in}}%
\pgfpathcurveto{\pgfqpoint{1.532480in}{2.781431in}}{\pgfqpoint{1.524580in}{2.784703in}}{\pgfqpoint{1.516343in}{2.784703in}}%
\pgfpathcurveto{\pgfqpoint{1.508107in}{2.784703in}}{\pgfqpoint{1.500207in}{2.781431in}}{\pgfqpoint{1.494383in}{2.775607in}}%
\pgfpathcurveto{\pgfqpoint{1.488559in}{2.769783in}}{\pgfqpoint{1.485287in}{2.761883in}}{\pgfqpoint{1.485287in}{2.753646in}}%
\pgfpathcurveto{\pgfqpoint{1.485287in}{2.745410in}}{\pgfqpoint{1.488559in}{2.737510in}}{\pgfqpoint{1.494383in}{2.731686in}}%
\pgfpathcurveto{\pgfqpoint{1.500207in}{2.725862in}}{\pgfqpoint{1.508107in}{2.722590in}}{\pgfqpoint{1.516343in}{2.722590in}}%
\pgfpathclose%
\pgfusepath{stroke,fill}%
\end{pgfscope}%
\begin{pgfscope}%
\pgfpathrectangle{\pgfqpoint{0.100000in}{0.212622in}}{\pgfqpoint{3.696000in}{3.696000in}}%
\pgfusepath{clip}%
\pgfsetbuttcap%
\pgfsetroundjoin%
\definecolor{currentfill}{rgb}{0.121569,0.466667,0.705882}%
\pgfsetfillcolor{currentfill}%
\pgfsetfillopacity{0.420013}%
\pgfsetlinewidth{1.003750pt}%
\definecolor{currentstroke}{rgb}{0.121569,0.466667,0.705882}%
\pgfsetstrokecolor{currentstroke}%
\pgfsetstrokeopacity{0.420013}%
\pgfsetdash{}{0pt}%
\pgfpathmoveto{\pgfqpoint{1.510662in}{2.712724in}}%
\pgfpathcurveto{\pgfqpoint{1.518898in}{2.712724in}}{\pgfqpoint{1.526798in}{2.715996in}}{\pgfqpoint{1.532622in}{2.721820in}}%
\pgfpathcurveto{\pgfqpoint{1.538446in}{2.727644in}}{\pgfqpoint{1.541718in}{2.735544in}}{\pgfqpoint{1.541718in}{2.743780in}}%
\pgfpathcurveto{\pgfqpoint{1.541718in}{2.752016in}}{\pgfqpoint{1.538446in}{2.759916in}}{\pgfqpoint{1.532622in}{2.765740in}}%
\pgfpathcurveto{\pgfqpoint{1.526798in}{2.771564in}}{\pgfqpoint{1.518898in}{2.774837in}}{\pgfqpoint{1.510662in}{2.774837in}}%
\pgfpathcurveto{\pgfqpoint{1.502426in}{2.774837in}}{\pgfqpoint{1.494525in}{2.771564in}}{\pgfqpoint{1.488702in}{2.765740in}}%
\pgfpathcurveto{\pgfqpoint{1.482878in}{2.759916in}}{\pgfqpoint{1.479605in}{2.752016in}}{\pgfqpoint{1.479605in}{2.743780in}}%
\pgfpathcurveto{\pgfqpoint{1.479605in}{2.735544in}}{\pgfqpoint{1.482878in}{2.727644in}}{\pgfqpoint{1.488702in}{2.721820in}}%
\pgfpathcurveto{\pgfqpoint{1.494525in}{2.715996in}}{\pgfqpoint{1.502426in}{2.712724in}}{\pgfqpoint{1.510662in}{2.712724in}}%
\pgfpathclose%
\pgfusepath{stroke,fill}%
\end{pgfscope}%
\begin{pgfscope}%
\pgfpathrectangle{\pgfqpoint{0.100000in}{0.212622in}}{\pgfqpoint{3.696000in}{3.696000in}}%
\pgfusepath{clip}%
\pgfsetbuttcap%
\pgfsetroundjoin%
\definecolor{currentfill}{rgb}{0.121569,0.466667,0.705882}%
\pgfsetfillcolor{currentfill}%
\pgfsetfillopacity{0.420493}%
\pgfsetlinewidth{1.003750pt}%
\definecolor{currentstroke}{rgb}{0.121569,0.466667,0.705882}%
\pgfsetstrokecolor{currentstroke}%
\pgfsetstrokeopacity{0.420493}%
\pgfsetdash{}{0pt}%
\pgfpathmoveto{\pgfqpoint{1.990323in}{2.837026in}}%
\pgfpathcurveto{\pgfqpoint{1.998559in}{2.837026in}}{\pgfqpoint{2.006459in}{2.840299in}}{\pgfqpoint{2.012283in}{2.846122in}}%
\pgfpathcurveto{\pgfqpoint{2.018107in}{2.851946in}}{\pgfqpoint{2.021380in}{2.859846in}}{\pgfqpoint{2.021380in}{2.868083in}}%
\pgfpathcurveto{\pgfqpoint{2.021380in}{2.876319in}}{\pgfqpoint{2.018107in}{2.884219in}}{\pgfqpoint{2.012283in}{2.890043in}}%
\pgfpathcurveto{\pgfqpoint{2.006459in}{2.895867in}}{\pgfqpoint{1.998559in}{2.899139in}}{\pgfqpoint{1.990323in}{2.899139in}}%
\pgfpathcurveto{\pgfqpoint{1.982087in}{2.899139in}}{\pgfqpoint{1.974187in}{2.895867in}}{\pgfqpoint{1.968363in}{2.890043in}}%
\pgfpathcurveto{\pgfqpoint{1.962539in}{2.884219in}}{\pgfqpoint{1.959267in}{2.876319in}}{\pgfqpoint{1.959267in}{2.868083in}}%
\pgfpathcurveto{\pgfqpoint{1.959267in}{2.859846in}}{\pgfqpoint{1.962539in}{2.851946in}}{\pgfqpoint{1.968363in}{2.846122in}}%
\pgfpathcurveto{\pgfqpoint{1.974187in}{2.840299in}}{\pgfqpoint{1.982087in}{2.837026in}}{\pgfqpoint{1.990323in}{2.837026in}}%
\pgfpathclose%
\pgfusepath{stroke,fill}%
\end{pgfscope}%
\begin{pgfscope}%
\pgfpathrectangle{\pgfqpoint{0.100000in}{0.212622in}}{\pgfqpoint{3.696000in}{3.696000in}}%
\pgfusepath{clip}%
\pgfsetbuttcap%
\pgfsetroundjoin%
\definecolor{currentfill}{rgb}{0.121569,0.466667,0.705882}%
\pgfsetfillcolor{currentfill}%
\pgfsetfillopacity{0.421648}%
\pgfsetlinewidth{1.003750pt}%
\definecolor{currentstroke}{rgb}{0.121569,0.466667,0.705882}%
\pgfsetstrokecolor{currentstroke}%
\pgfsetstrokeopacity{0.421648}%
\pgfsetdash{}{0pt}%
\pgfpathmoveto{\pgfqpoint{1.506082in}{2.702830in}}%
\pgfpathcurveto{\pgfqpoint{1.514318in}{2.702830in}}{\pgfqpoint{1.522218in}{2.706103in}}{\pgfqpoint{1.528042in}{2.711927in}}%
\pgfpathcurveto{\pgfqpoint{1.533866in}{2.717750in}}{\pgfqpoint{1.537138in}{2.725651in}}{\pgfqpoint{1.537138in}{2.733887in}}%
\pgfpathcurveto{\pgfqpoint{1.537138in}{2.742123in}}{\pgfqpoint{1.533866in}{2.750023in}}{\pgfqpoint{1.528042in}{2.755847in}}%
\pgfpathcurveto{\pgfqpoint{1.522218in}{2.761671in}}{\pgfqpoint{1.514318in}{2.764943in}}{\pgfqpoint{1.506082in}{2.764943in}}%
\pgfpathcurveto{\pgfqpoint{1.497845in}{2.764943in}}{\pgfqpoint{1.489945in}{2.761671in}}{\pgfqpoint{1.484122in}{2.755847in}}%
\pgfpathcurveto{\pgfqpoint{1.478298in}{2.750023in}}{\pgfqpoint{1.475025in}{2.742123in}}{\pgfqpoint{1.475025in}{2.733887in}}%
\pgfpathcurveto{\pgfqpoint{1.475025in}{2.725651in}}{\pgfqpoint{1.478298in}{2.717750in}}{\pgfqpoint{1.484122in}{2.711927in}}%
\pgfpathcurveto{\pgfqpoint{1.489945in}{2.706103in}}{\pgfqpoint{1.497845in}{2.702830in}}{\pgfqpoint{1.506082in}{2.702830in}}%
\pgfpathclose%
\pgfusepath{stroke,fill}%
\end{pgfscope}%
\begin{pgfscope}%
\pgfpathrectangle{\pgfqpoint{0.100000in}{0.212622in}}{\pgfqpoint{3.696000in}{3.696000in}}%
\pgfusepath{clip}%
\pgfsetbuttcap%
\pgfsetroundjoin%
\definecolor{currentfill}{rgb}{0.121569,0.466667,0.705882}%
\pgfsetfillcolor{currentfill}%
\pgfsetfillopacity{0.422628}%
\pgfsetlinewidth{1.003750pt}%
\definecolor{currentstroke}{rgb}{0.121569,0.466667,0.705882}%
\pgfsetstrokecolor{currentstroke}%
\pgfsetstrokeopacity{0.422628}%
\pgfsetdash{}{0pt}%
\pgfpathmoveto{\pgfqpoint{1.502349in}{2.696633in}}%
\pgfpathcurveto{\pgfqpoint{1.510585in}{2.696633in}}{\pgfqpoint{1.518485in}{2.699905in}}{\pgfqpoint{1.524309in}{2.705729in}}%
\pgfpathcurveto{\pgfqpoint{1.530133in}{2.711553in}}{\pgfqpoint{1.533405in}{2.719453in}}{\pgfqpoint{1.533405in}{2.727689in}}%
\pgfpathcurveto{\pgfqpoint{1.533405in}{2.735925in}}{\pgfqpoint{1.530133in}{2.743825in}}{\pgfqpoint{1.524309in}{2.749649in}}%
\pgfpathcurveto{\pgfqpoint{1.518485in}{2.755473in}}{\pgfqpoint{1.510585in}{2.758746in}}{\pgfqpoint{1.502349in}{2.758746in}}%
\pgfpathcurveto{\pgfqpoint{1.494112in}{2.758746in}}{\pgfqpoint{1.486212in}{2.755473in}}{\pgfqpoint{1.480388in}{2.749649in}}%
\pgfpathcurveto{\pgfqpoint{1.474564in}{2.743825in}}{\pgfqpoint{1.471292in}{2.735925in}}{\pgfqpoint{1.471292in}{2.727689in}}%
\pgfpathcurveto{\pgfqpoint{1.471292in}{2.719453in}}{\pgfqpoint{1.474564in}{2.711553in}}{\pgfqpoint{1.480388in}{2.705729in}}%
\pgfpathcurveto{\pgfqpoint{1.486212in}{2.699905in}}{\pgfqpoint{1.494112in}{2.696633in}}{\pgfqpoint{1.502349in}{2.696633in}}%
\pgfpathclose%
\pgfusepath{stroke,fill}%
\end{pgfscope}%
\begin{pgfscope}%
\pgfpathrectangle{\pgfqpoint{0.100000in}{0.212622in}}{\pgfqpoint{3.696000in}{3.696000in}}%
\pgfusepath{clip}%
\pgfsetbuttcap%
\pgfsetroundjoin%
\definecolor{currentfill}{rgb}{0.121569,0.466667,0.705882}%
\pgfsetfillcolor{currentfill}%
\pgfsetfillopacity{0.424411}%
\pgfsetlinewidth{1.003750pt}%
\definecolor{currentstroke}{rgb}{0.121569,0.466667,0.705882}%
\pgfsetstrokecolor{currentstroke}%
\pgfsetstrokeopacity{0.424411}%
\pgfsetdash{}{0pt}%
\pgfpathmoveto{\pgfqpoint{1.992210in}{2.820184in}}%
\pgfpathcurveto{\pgfqpoint{2.000446in}{2.820184in}}{\pgfqpoint{2.008347in}{2.823457in}}{\pgfqpoint{2.014170in}{2.829280in}}%
\pgfpathcurveto{\pgfqpoint{2.019994in}{2.835104in}}{\pgfqpoint{2.023267in}{2.843004in}}{\pgfqpoint{2.023267in}{2.851241in}}%
\pgfpathcurveto{\pgfqpoint{2.023267in}{2.859477in}}{\pgfqpoint{2.019994in}{2.867377in}}{\pgfqpoint{2.014170in}{2.873201in}}%
\pgfpathcurveto{\pgfqpoint{2.008347in}{2.879025in}}{\pgfqpoint{2.000446in}{2.882297in}}{\pgfqpoint{1.992210in}{2.882297in}}%
\pgfpathcurveto{\pgfqpoint{1.983974in}{2.882297in}}{\pgfqpoint{1.976074in}{2.879025in}}{\pgfqpoint{1.970250in}{2.873201in}}%
\pgfpathcurveto{\pgfqpoint{1.964426in}{2.867377in}}{\pgfqpoint{1.961154in}{2.859477in}}{\pgfqpoint{1.961154in}{2.851241in}}%
\pgfpathcurveto{\pgfqpoint{1.961154in}{2.843004in}}{\pgfqpoint{1.964426in}{2.835104in}}{\pgfqpoint{1.970250in}{2.829280in}}%
\pgfpathcurveto{\pgfqpoint{1.976074in}{2.823457in}}{\pgfqpoint{1.983974in}{2.820184in}}{\pgfqpoint{1.992210in}{2.820184in}}%
\pgfpathclose%
\pgfusepath{stroke,fill}%
\end{pgfscope}%
\begin{pgfscope}%
\pgfpathrectangle{\pgfqpoint{0.100000in}{0.212622in}}{\pgfqpoint{3.696000in}{3.696000in}}%
\pgfusepath{clip}%
\pgfsetbuttcap%
\pgfsetroundjoin%
\definecolor{currentfill}{rgb}{0.121569,0.466667,0.705882}%
\pgfsetfillcolor{currentfill}%
\pgfsetfillopacity{0.424678}%
\pgfsetlinewidth{1.003750pt}%
\definecolor{currentstroke}{rgb}{0.121569,0.466667,0.705882}%
\pgfsetstrokecolor{currentstroke}%
\pgfsetstrokeopacity{0.424678}%
\pgfsetdash{}{0pt}%
\pgfpathmoveto{\pgfqpoint{1.496463in}{2.685102in}}%
\pgfpathcurveto{\pgfqpoint{1.504700in}{2.685102in}}{\pgfqpoint{1.512600in}{2.688374in}}{\pgfqpoint{1.518424in}{2.694198in}}%
\pgfpathcurveto{\pgfqpoint{1.524248in}{2.700022in}}{\pgfqpoint{1.527520in}{2.707922in}}{\pgfqpoint{1.527520in}{2.716158in}}%
\pgfpathcurveto{\pgfqpoint{1.527520in}{2.724394in}}{\pgfqpoint{1.524248in}{2.732294in}}{\pgfqpoint{1.518424in}{2.738118in}}%
\pgfpathcurveto{\pgfqpoint{1.512600in}{2.743942in}}{\pgfqpoint{1.504700in}{2.747215in}}{\pgfqpoint{1.496463in}{2.747215in}}%
\pgfpathcurveto{\pgfqpoint{1.488227in}{2.747215in}}{\pgfqpoint{1.480327in}{2.743942in}}{\pgfqpoint{1.474503in}{2.738118in}}%
\pgfpathcurveto{\pgfqpoint{1.468679in}{2.732294in}}{\pgfqpoint{1.465407in}{2.724394in}}{\pgfqpoint{1.465407in}{2.716158in}}%
\pgfpathcurveto{\pgfqpoint{1.465407in}{2.707922in}}{\pgfqpoint{1.468679in}{2.700022in}}{\pgfqpoint{1.474503in}{2.694198in}}%
\pgfpathcurveto{\pgfqpoint{1.480327in}{2.688374in}}{\pgfqpoint{1.488227in}{2.685102in}}{\pgfqpoint{1.496463in}{2.685102in}}%
\pgfpathclose%
\pgfusepath{stroke,fill}%
\end{pgfscope}%
\begin{pgfscope}%
\pgfpathrectangle{\pgfqpoint{0.100000in}{0.212622in}}{\pgfqpoint{3.696000in}{3.696000in}}%
\pgfusepath{clip}%
\pgfsetbuttcap%
\pgfsetroundjoin%
\definecolor{currentfill}{rgb}{0.121569,0.466667,0.705882}%
\pgfsetfillcolor{currentfill}%
\pgfsetfillopacity{0.426614}%
\pgfsetlinewidth{1.003750pt}%
\definecolor{currentstroke}{rgb}{0.121569,0.466667,0.705882}%
\pgfsetstrokecolor{currentstroke}%
\pgfsetstrokeopacity{0.426614}%
\pgfsetdash{}{0pt}%
\pgfpathmoveto{\pgfqpoint{1.490574in}{2.674275in}}%
\pgfpathcurveto{\pgfqpoint{1.498810in}{2.674275in}}{\pgfqpoint{1.506710in}{2.677547in}}{\pgfqpoint{1.512534in}{2.683371in}}%
\pgfpathcurveto{\pgfqpoint{1.518358in}{2.689195in}}{\pgfqpoint{1.521630in}{2.697095in}}{\pgfqpoint{1.521630in}{2.705331in}}%
\pgfpathcurveto{\pgfqpoint{1.521630in}{2.713567in}}{\pgfqpoint{1.518358in}{2.721467in}}{\pgfqpoint{1.512534in}{2.727291in}}%
\pgfpathcurveto{\pgfqpoint{1.506710in}{2.733115in}}{\pgfqpoint{1.498810in}{2.736388in}}{\pgfqpoint{1.490574in}{2.736388in}}%
\pgfpathcurveto{\pgfqpoint{1.482338in}{2.736388in}}{\pgfqpoint{1.474438in}{2.733115in}}{\pgfqpoint{1.468614in}{2.727291in}}%
\pgfpathcurveto{\pgfqpoint{1.462790in}{2.721467in}}{\pgfqpoint{1.459517in}{2.713567in}}{\pgfqpoint{1.459517in}{2.705331in}}%
\pgfpathcurveto{\pgfqpoint{1.459517in}{2.697095in}}{\pgfqpoint{1.462790in}{2.689195in}}{\pgfqpoint{1.468614in}{2.683371in}}%
\pgfpathcurveto{\pgfqpoint{1.474438in}{2.677547in}}{\pgfqpoint{1.482338in}{2.674275in}}{\pgfqpoint{1.490574in}{2.674275in}}%
\pgfpathclose%
\pgfusepath{stroke,fill}%
\end{pgfscope}%
\begin{pgfscope}%
\pgfpathrectangle{\pgfqpoint{0.100000in}{0.212622in}}{\pgfqpoint{3.696000in}{3.696000in}}%
\pgfusepath{clip}%
\pgfsetbuttcap%
\pgfsetroundjoin%
\definecolor{currentfill}{rgb}{0.121569,0.466667,0.705882}%
\pgfsetfillcolor{currentfill}%
\pgfsetfillopacity{0.428122}%
\pgfsetlinewidth{1.003750pt}%
\definecolor{currentstroke}{rgb}{0.121569,0.466667,0.705882}%
\pgfsetstrokecolor{currentstroke}%
\pgfsetstrokeopacity{0.428122}%
\pgfsetdash{}{0pt}%
\pgfpathmoveto{\pgfqpoint{1.484598in}{2.663675in}}%
\pgfpathcurveto{\pgfqpoint{1.492835in}{2.663675in}}{\pgfqpoint{1.500735in}{2.666947in}}{\pgfqpoint{1.506559in}{2.672771in}}%
\pgfpathcurveto{\pgfqpoint{1.512382in}{2.678595in}}{\pgfqpoint{1.515655in}{2.686495in}}{\pgfqpoint{1.515655in}{2.694731in}}%
\pgfpathcurveto{\pgfqpoint{1.515655in}{2.702967in}}{\pgfqpoint{1.512382in}{2.710868in}}{\pgfqpoint{1.506559in}{2.716691in}}%
\pgfpathcurveto{\pgfqpoint{1.500735in}{2.722515in}}{\pgfqpoint{1.492835in}{2.725788in}}{\pgfqpoint{1.484598in}{2.725788in}}%
\pgfpathcurveto{\pgfqpoint{1.476362in}{2.725788in}}{\pgfqpoint{1.468462in}{2.722515in}}{\pgfqpoint{1.462638in}{2.716691in}}%
\pgfpathcurveto{\pgfqpoint{1.456814in}{2.710868in}}{\pgfqpoint{1.453542in}{2.702967in}}{\pgfqpoint{1.453542in}{2.694731in}}%
\pgfpathcurveto{\pgfqpoint{1.453542in}{2.686495in}}{\pgfqpoint{1.456814in}{2.678595in}}{\pgfqpoint{1.462638in}{2.672771in}}%
\pgfpathcurveto{\pgfqpoint{1.468462in}{2.666947in}}{\pgfqpoint{1.476362in}{2.663675in}}{\pgfqpoint{1.484598in}{2.663675in}}%
\pgfpathclose%
\pgfusepath{stroke,fill}%
\end{pgfscope}%
\begin{pgfscope}%
\pgfpathrectangle{\pgfqpoint{0.100000in}{0.212622in}}{\pgfqpoint{3.696000in}{3.696000in}}%
\pgfusepath{clip}%
\pgfsetbuttcap%
\pgfsetroundjoin%
\definecolor{currentfill}{rgb}{0.121569,0.466667,0.705882}%
\pgfsetfillcolor{currentfill}%
\pgfsetfillopacity{0.428300}%
\pgfsetlinewidth{1.003750pt}%
\definecolor{currentstroke}{rgb}{0.121569,0.466667,0.705882}%
\pgfsetstrokecolor{currentstroke}%
\pgfsetstrokeopacity{0.428300}%
\pgfsetdash{}{0pt}%
\pgfpathmoveto{\pgfqpoint{1.995548in}{2.801457in}}%
\pgfpathcurveto{\pgfqpoint{2.003784in}{2.801457in}}{\pgfqpoint{2.011684in}{2.804729in}}{\pgfqpoint{2.017508in}{2.810553in}}%
\pgfpathcurveto{\pgfqpoint{2.023332in}{2.816377in}}{\pgfqpoint{2.026605in}{2.824277in}}{\pgfqpoint{2.026605in}{2.832513in}}%
\pgfpathcurveto{\pgfqpoint{2.026605in}{2.840750in}}{\pgfqpoint{2.023332in}{2.848650in}}{\pgfqpoint{2.017508in}{2.854474in}}%
\pgfpathcurveto{\pgfqpoint{2.011684in}{2.860298in}}{\pgfqpoint{2.003784in}{2.863570in}}{\pgfqpoint{1.995548in}{2.863570in}}%
\pgfpathcurveto{\pgfqpoint{1.987312in}{2.863570in}}{\pgfqpoint{1.979412in}{2.860298in}}{\pgfqpoint{1.973588in}{2.854474in}}%
\pgfpathcurveto{\pgfqpoint{1.967764in}{2.848650in}}{\pgfqpoint{1.964492in}{2.840750in}}{\pgfqpoint{1.964492in}{2.832513in}}%
\pgfpathcurveto{\pgfqpoint{1.964492in}{2.824277in}}{\pgfqpoint{1.967764in}{2.816377in}}{\pgfqpoint{1.973588in}{2.810553in}}%
\pgfpathcurveto{\pgfqpoint{1.979412in}{2.804729in}}{\pgfqpoint{1.987312in}{2.801457in}}{\pgfqpoint{1.995548in}{2.801457in}}%
\pgfpathclose%
\pgfusepath{stroke,fill}%
\end{pgfscope}%
\begin{pgfscope}%
\pgfpathrectangle{\pgfqpoint{0.100000in}{0.212622in}}{\pgfqpoint{3.696000in}{3.696000in}}%
\pgfusepath{clip}%
\pgfsetbuttcap%
\pgfsetroundjoin%
\definecolor{currentfill}{rgb}{0.121569,0.466667,0.705882}%
\pgfsetfillcolor{currentfill}%
\pgfsetfillopacity{0.429585}%
\pgfsetlinewidth{1.003750pt}%
\definecolor{currentstroke}{rgb}{0.121569,0.466667,0.705882}%
\pgfsetstrokecolor{currentstroke}%
\pgfsetstrokeopacity{0.429585}%
\pgfsetdash{}{0pt}%
\pgfpathmoveto{\pgfqpoint{1.480252in}{2.653510in}}%
\pgfpathcurveto{\pgfqpoint{1.488489in}{2.653510in}}{\pgfqpoint{1.496389in}{2.656782in}}{\pgfqpoint{1.502213in}{2.662606in}}%
\pgfpathcurveto{\pgfqpoint{1.508037in}{2.668430in}}{\pgfqpoint{1.511309in}{2.676330in}}{\pgfqpoint{1.511309in}{2.684566in}}%
\pgfpathcurveto{\pgfqpoint{1.511309in}{2.692802in}}{\pgfqpoint{1.508037in}{2.700702in}}{\pgfqpoint{1.502213in}{2.706526in}}%
\pgfpathcurveto{\pgfqpoint{1.496389in}{2.712350in}}{\pgfqpoint{1.488489in}{2.715623in}}{\pgfqpoint{1.480252in}{2.715623in}}%
\pgfpathcurveto{\pgfqpoint{1.472016in}{2.715623in}}{\pgfqpoint{1.464116in}{2.712350in}}{\pgfqpoint{1.458292in}{2.706526in}}%
\pgfpathcurveto{\pgfqpoint{1.452468in}{2.700702in}}{\pgfqpoint{1.449196in}{2.692802in}}{\pgfqpoint{1.449196in}{2.684566in}}%
\pgfpathcurveto{\pgfqpoint{1.449196in}{2.676330in}}{\pgfqpoint{1.452468in}{2.668430in}}{\pgfqpoint{1.458292in}{2.662606in}}%
\pgfpathcurveto{\pgfqpoint{1.464116in}{2.656782in}}{\pgfqpoint{1.472016in}{2.653510in}}{\pgfqpoint{1.480252in}{2.653510in}}%
\pgfpathclose%
\pgfusepath{stroke,fill}%
\end{pgfscope}%
\begin{pgfscope}%
\pgfpathrectangle{\pgfqpoint{0.100000in}{0.212622in}}{\pgfqpoint{3.696000in}{3.696000in}}%
\pgfusepath{clip}%
\pgfsetbuttcap%
\pgfsetroundjoin%
\definecolor{currentfill}{rgb}{0.121569,0.466667,0.705882}%
\pgfsetfillcolor{currentfill}%
\pgfsetfillopacity{0.430602}%
\pgfsetlinewidth{1.003750pt}%
\definecolor{currentstroke}{rgb}{0.121569,0.466667,0.705882}%
\pgfsetstrokecolor{currentstroke}%
\pgfsetstrokeopacity{0.430602}%
\pgfsetdash{}{0pt}%
\pgfpathmoveto{\pgfqpoint{1.476316in}{2.647057in}}%
\pgfpathcurveto{\pgfqpoint{1.484552in}{2.647057in}}{\pgfqpoint{1.492452in}{2.650329in}}{\pgfqpoint{1.498276in}{2.656153in}}%
\pgfpathcurveto{\pgfqpoint{1.504100in}{2.661977in}}{\pgfqpoint{1.507372in}{2.669877in}}{\pgfqpoint{1.507372in}{2.678113in}}%
\pgfpathcurveto{\pgfqpoint{1.507372in}{2.686349in}}{\pgfqpoint{1.504100in}{2.694249in}}{\pgfqpoint{1.498276in}{2.700073in}}%
\pgfpathcurveto{\pgfqpoint{1.492452in}{2.705897in}}{\pgfqpoint{1.484552in}{2.709170in}}{\pgfqpoint{1.476316in}{2.709170in}}%
\pgfpathcurveto{\pgfqpoint{1.468080in}{2.709170in}}{\pgfqpoint{1.460179in}{2.705897in}}{\pgfqpoint{1.454356in}{2.700073in}}%
\pgfpathcurveto{\pgfqpoint{1.448532in}{2.694249in}}{\pgfqpoint{1.445259in}{2.686349in}}{\pgfqpoint{1.445259in}{2.678113in}}%
\pgfpathcurveto{\pgfqpoint{1.445259in}{2.669877in}}{\pgfqpoint{1.448532in}{2.661977in}}{\pgfqpoint{1.454356in}{2.656153in}}%
\pgfpathcurveto{\pgfqpoint{1.460179in}{2.650329in}}{\pgfqpoint{1.468080in}{2.647057in}}{\pgfqpoint{1.476316in}{2.647057in}}%
\pgfpathclose%
\pgfusepath{stroke,fill}%
\end{pgfscope}%
\begin{pgfscope}%
\pgfpathrectangle{\pgfqpoint{0.100000in}{0.212622in}}{\pgfqpoint{3.696000in}{3.696000in}}%
\pgfusepath{clip}%
\pgfsetbuttcap%
\pgfsetroundjoin%
\definecolor{currentfill}{rgb}{0.121569,0.466667,0.705882}%
\pgfsetfillcolor{currentfill}%
\pgfsetfillopacity{0.432565}%
\pgfsetlinewidth{1.003750pt}%
\definecolor{currentstroke}{rgb}{0.121569,0.466667,0.705882}%
\pgfsetstrokecolor{currentstroke}%
\pgfsetstrokeopacity{0.432565}%
\pgfsetdash{}{0pt}%
\pgfpathmoveto{\pgfqpoint{1.998039in}{2.782645in}}%
\pgfpathcurveto{\pgfqpoint{2.006275in}{2.782645in}}{\pgfqpoint{2.014176in}{2.785917in}}{\pgfqpoint{2.019999in}{2.791741in}}%
\pgfpathcurveto{\pgfqpoint{2.025823in}{2.797565in}}{\pgfqpoint{2.029096in}{2.805465in}}{\pgfqpoint{2.029096in}{2.813701in}}%
\pgfpathcurveto{\pgfqpoint{2.029096in}{2.821938in}}{\pgfqpoint{2.025823in}{2.829838in}}{\pgfqpoint{2.019999in}{2.835662in}}%
\pgfpathcurveto{\pgfqpoint{2.014176in}{2.841486in}}{\pgfqpoint{2.006275in}{2.844758in}}{\pgfqpoint{1.998039in}{2.844758in}}%
\pgfpathcurveto{\pgfqpoint{1.989803in}{2.844758in}}{\pgfqpoint{1.981903in}{2.841486in}}{\pgfqpoint{1.976079in}{2.835662in}}%
\pgfpathcurveto{\pgfqpoint{1.970255in}{2.829838in}}{\pgfqpoint{1.966983in}{2.821938in}}{\pgfqpoint{1.966983in}{2.813701in}}%
\pgfpathcurveto{\pgfqpoint{1.966983in}{2.805465in}}{\pgfqpoint{1.970255in}{2.797565in}}{\pgfqpoint{1.976079in}{2.791741in}}%
\pgfpathcurveto{\pgfqpoint{1.981903in}{2.785917in}}{\pgfqpoint{1.989803in}{2.782645in}}{\pgfqpoint{1.998039in}{2.782645in}}%
\pgfpathclose%
\pgfusepath{stroke,fill}%
\end{pgfscope}%
\begin{pgfscope}%
\pgfpathrectangle{\pgfqpoint{0.100000in}{0.212622in}}{\pgfqpoint{3.696000in}{3.696000in}}%
\pgfusepath{clip}%
\pgfsetbuttcap%
\pgfsetroundjoin%
\definecolor{currentfill}{rgb}{0.121569,0.466667,0.705882}%
\pgfsetfillcolor{currentfill}%
\pgfsetfillopacity{0.432598}%
\pgfsetlinewidth{1.003750pt}%
\definecolor{currentstroke}{rgb}{0.121569,0.466667,0.705882}%
\pgfsetstrokecolor{currentstroke}%
\pgfsetstrokeopacity{0.432598}%
\pgfsetdash{}{0pt}%
\pgfpathmoveto{\pgfqpoint{1.469984in}{2.634650in}}%
\pgfpathcurveto{\pgfqpoint{1.478220in}{2.634650in}}{\pgfqpoint{1.486120in}{2.637922in}}{\pgfqpoint{1.491944in}{2.643746in}}%
\pgfpathcurveto{\pgfqpoint{1.497768in}{2.649570in}}{\pgfqpoint{1.501040in}{2.657470in}}{\pgfqpoint{1.501040in}{2.665706in}}%
\pgfpathcurveto{\pgfqpoint{1.501040in}{2.673942in}}{\pgfqpoint{1.497768in}{2.681842in}}{\pgfqpoint{1.491944in}{2.687666in}}%
\pgfpathcurveto{\pgfqpoint{1.486120in}{2.693490in}}{\pgfqpoint{1.478220in}{2.696763in}}{\pgfqpoint{1.469984in}{2.696763in}}%
\pgfpathcurveto{\pgfqpoint{1.461748in}{2.696763in}}{\pgfqpoint{1.453847in}{2.693490in}}{\pgfqpoint{1.448024in}{2.687666in}}%
\pgfpathcurveto{\pgfqpoint{1.442200in}{2.681842in}}{\pgfqpoint{1.438927in}{2.673942in}}{\pgfqpoint{1.438927in}{2.665706in}}%
\pgfpathcurveto{\pgfqpoint{1.438927in}{2.657470in}}{\pgfqpoint{1.442200in}{2.649570in}}{\pgfqpoint{1.448024in}{2.643746in}}%
\pgfpathcurveto{\pgfqpoint{1.453847in}{2.637922in}}{\pgfqpoint{1.461748in}{2.634650in}}{\pgfqpoint{1.469984in}{2.634650in}}%
\pgfpathclose%
\pgfusepath{stroke,fill}%
\end{pgfscope}%
\begin{pgfscope}%
\pgfpathrectangle{\pgfqpoint{0.100000in}{0.212622in}}{\pgfqpoint{3.696000in}{3.696000in}}%
\pgfusepath{clip}%
\pgfsetbuttcap%
\pgfsetroundjoin%
\definecolor{currentfill}{rgb}{0.121569,0.466667,0.705882}%
\pgfsetfillcolor{currentfill}%
\pgfsetfillopacity{0.434456}%
\pgfsetlinewidth{1.003750pt}%
\definecolor{currentstroke}{rgb}{0.121569,0.466667,0.705882}%
\pgfsetstrokecolor{currentstroke}%
\pgfsetstrokeopacity{0.434456}%
\pgfsetdash{}{0pt}%
\pgfpathmoveto{\pgfqpoint{1.463695in}{2.622727in}}%
\pgfpathcurveto{\pgfqpoint{1.471931in}{2.622727in}}{\pgfqpoint{1.479831in}{2.625999in}}{\pgfqpoint{1.485655in}{2.631823in}}%
\pgfpathcurveto{\pgfqpoint{1.491479in}{2.637647in}}{\pgfqpoint{1.494752in}{2.645547in}}{\pgfqpoint{1.494752in}{2.653784in}}%
\pgfpathcurveto{\pgfqpoint{1.494752in}{2.662020in}}{\pgfqpoint{1.491479in}{2.669920in}}{\pgfqpoint{1.485655in}{2.675744in}}%
\pgfpathcurveto{\pgfqpoint{1.479831in}{2.681568in}}{\pgfqpoint{1.471931in}{2.684840in}}{\pgfqpoint{1.463695in}{2.684840in}}%
\pgfpathcurveto{\pgfqpoint{1.455459in}{2.684840in}}{\pgfqpoint{1.447559in}{2.681568in}}{\pgfqpoint{1.441735in}{2.675744in}}%
\pgfpathcurveto{\pgfqpoint{1.435911in}{2.669920in}}{\pgfqpoint{1.432639in}{2.662020in}}{\pgfqpoint{1.432639in}{2.653784in}}%
\pgfpathcurveto{\pgfqpoint{1.432639in}{2.645547in}}{\pgfqpoint{1.435911in}{2.637647in}}{\pgfqpoint{1.441735in}{2.631823in}}%
\pgfpathcurveto{\pgfqpoint{1.447559in}{2.625999in}}{\pgfqpoint{1.455459in}{2.622727in}}{\pgfqpoint{1.463695in}{2.622727in}}%
\pgfpathclose%
\pgfusepath{stroke,fill}%
\end{pgfscope}%
\begin{pgfscope}%
\pgfpathrectangle{\pgfqpoint{0.100000in}{0.212622in}}{\pgfqpoint{3.696000in}{3.696000in}}%
\pgfusepath{clip}%
\pgfsetbuttcap%
\pgfsetroundjoin%
\definecolor{currentfill}{rgb}{0.121569,0.466667,0.705882}%
\pgfsetfillcolor{currentfill}%
\pgfsetfillopacity{0.435929}%
\pgfsetlinewidth{1.003750pt}%
\definecolor{currentstroke}{rgb}{0.121569,0.466667,0.705882}%
\pgfsetstrokecolor{currentstroke}%
\pgfsetstrokeopacity{0.435929}%
\pgfsetdash{}{0pt}%
\pgfpathmoveto{\pgfqpoint{1.457365in}{2.611635in}}%
\pgfpathcurveto{\pgfqpoint{1.465602in}{2.611635in}}{\pgfqpoint{1.473502in}{2.614908in}}{\pgfqpoint{1.479326in}{2.620732in}}%
\pgfpathcurveto{\pgfqpoint{1.485150in}{2.626556in}}{\pgfqpoint{1.488422in}{2.634456in}}{\pgfqpoint{1.488422in}{2.642692in}}%
\pgfpathcurveto{\pgfqpoint{1.488422in}{2.650928in}}{\pgfqpoint{1.485150in}{2.658828in}}{\pgfqpoint{1.479326in}{2.664652in}}%
\pgfpathcurveto{\pgfqpoint{1.473502in}{2.670476in}}{\pgfqpoint{1.465602in}{2.673748in}}{\pgfqpoint{1.457365in}{2.673748in}}%
\pgfpathcurveto{\pgfqpoint{1.449129in}{2.673748in}}{\pgfqpoint{1.441229in}{2.670476in}}{\pgfqpoint{1.435405in}{2.664652in}}%
\pgfpathcurveto{\pgfqpoint{1.429581in}{2.658828in}}{\pgfqpoint{1.426309in}{2.650928in}}{\pgfqpoint{1.426309in}{2.642692in}}%
\pgfpathcurveto{\pgfqpoint{1.426309in}{2.634456in}}{\pgfqpoint{1.429581in}{2.626556in}}{\pgfqpoint{1.435405in}{2.620732in}}%
\pgfpathcurveto{\pgfqpoint{1.441229in}{2.614908in}}{\pgfqpoint{1.449129in}{2.611635in}}{\pgfqpoint{1.457365in}{2.611635in}}%
\pgfpathclose%
\pgfusepath{stroke,fill}%
\end{pgfscope}%
\begin{pgfscope}%
\pgfpathrectangle{\pgfqpoint{0.100000in}{0.212622in}}{\pgfqpoint{3.696000in}{3.696000in}}%
\pgfusepath{clip}%
\pgfsetbuttcap%
\pgfsetroundjoin%
\definecolor{currentfill}{rgb}{0.121569,0.466667,0.705882}%
\pgfsetfillcolor{currentfill}%
\pgfsetfillopacity{0.437302}%
\pgfsetlinewidth{1.003750pt}%
\definecolor{currentstroke}{rgb}{0.121569,0.466667,0.705882}%
\pgfsetstrokecolor{currentstroke}%
\pgfsetstrokeopacity{0.437302}%
\pgfsetdash{}{0pt}%
\pgfpathmoveto{\pgfqpoint{2.000243in}{2.764706in}}%
\pgfpathcurveto{\pgfqpoint{2.008479in}{2.764706in}}{\pgfqpoint{2.016379in}{2.767978in}}{\pgfqpoint{2.022203in}{2.773802in}}%
\pgfpathcurveto{\pgfqpoint{2.028027in}{2.779626in}}{\pgfqpoint{2.031299in}{2.787526in}}{\pgfqpoint{2.031299in}{2.795763in}}%
\pgfpathcurveto{\pgfqpoint{2.031299in}{2.803999in}}{\pgfqpoint{2.028027in}{2.811899in}}{\pgfqpoint{2.022203in}{2.817723in}}%
\pgfpathcurveto{\pgfqpoint{2.016379in}{2.823547in}}{\pgfqpoint{2.008479in}{2.826819in}}{\pgfqpoint{2.000243in}{2.826819in}}%
\pgfpathcurveto{\pgfqpoint{1.992006in}{2.826819in}}{\pgfqpoint{1.984106in}{2.823547in}}{\pgfqpoint{1.978282in}{2.817723in}}%
\pgfpathcurveto{\pgfqpoint{1.972459in}{2.811899in}}{\pgfqpoint{1.969186in}{2.803999in}}{\pgfqpoint{1.969186in}{2.795763in}}%
\pgfpathcurveto{\pgfqpoint{1.969186in}{2.787526in}}{\pgfqpoint{1.972459in}{2.779626in}}{\pgfqpoint{1.978282in}{2.773802in}}%
\pgfpathcurveto{\pgfqpoint{1.984106in}{2.767978in}}{\pgfqpoint{1.992006in}{2.764706in}}{\pgfqpoint{2.000243in}{2.764706in}}%
\pgfpathclose%
\pgfusepath{stroke,fill}%
\end{pgfscope}%
\begin{pgfscope}%
\pgfpathrectangle{\pgfqpoint{0.100000in}{0.212622in}}{\pgfqpoint{3.696000in}{3.696000in}}%
\pgfusepath{clip}%
\pgfsetbuttcap%
\pgfsetroundjoin%
\definecolor{currentfill}{rgb}{0.121569,0.466667,0.705882}%
\pgfsetfillcolor{currentfill}%
\pgfsetfillopacity{0.437549}%
\pgfsetlinewidth{1.003750pt}%
\definecolor{currentstroke}{rgb}{0.121569,0.466667,0.705882}%
\pgfsetstrokecolor{currentstroke}%
\pgfsetstrokeopacity{0.437549}%
\pgfsetdash{}{0pt}%
\pgfpathmoveto{\pgfqpoint{1.452638in}{2.601524in}}%
\pgfpathcurveto{\pgfqpoint{1.460874in}{2.601524in}}{\pgfqpoint{1.468774in}{2.604796in}}{\pgfqpoint{1.474598in}{2.610620in}}%
\pgfpathcurveto{\pgfqpoint{1.480422in}{2.616444in}}{\pgfqpoint{1.483694in}{2.624344in}}{\pgfqpoint{1.483694in}{2.632580in}}%
\pgfpathcurveto{\pgfqpoint{1.483694in}{2.640817in}}{\pgfqpoint{1.480422in}{2.648717in}}{\pgfqpoint{1.474598in}{2.654541in}}%
\pgfpathcurveto{\pgfqpoint{1.468774in}{2.660365in}}{\pgfqpoint{1.460874in}{2.663637in}}{\pgfqpoint{1.452638in}{2.663637in}}%
\pgfpathcurveto{\pgfqpoint{1.444402in}{2.663637in}}{\pgfqpoint{1.436502in}{2.660365in}}{\pgfqpoint{1.430678in}{2.654541in}}%
\pgfpathcurveto{\pgfqpoint{1.424854in}{2.648717in}}{\pgfqpoint{1.421581in}{2.640817in}}{\pgfqpoint{1.421581in}{2.632580in}}%
\pgfpathcurveto{\pgfqpoint{1.421581in}{2.624344in}}{\pgfqpoint{1.424854in}{2.616444in}}{\pgfqpoint{1.430678in}{2.610620in}}%
\pgfpathcurveto{\pgfqpoint{1.436502in}{2.604796in}}{\pgfqpoint{1.444402in}{2.601524in}}{\pgfqpoint{1.452638in}{2.601524in}}%
\pgfpathclose%
\pgfusepath{stroke,fill}%
\end{pgfscope}%
\begin{pgfscope}%
\pgfpathrectangle{\pgfqpoint{0.100000in}{0.212622in}}{\pgfqpoint{3.696000in}{3.696000in}}%
\pgfusepath{clip}%
\pgfsetbuttcap%
\pgfsetroundjoin%
\definecolor{currentfill}{rgb}{0.121569,0.466667,0.705882}%
\pgfsetfillcolor{currentfill}%
\pgfsetfillopacity{0.438724}%
\pgfsetlinewidth{1.003750pt}%
\definecolor{currentstroke}{rgb}{0.121569,0.466667,0.705882}%
\pgfsetstrokecolor{currentstroke}%
\pgfsetstrokeopacity{0.438724}%
\pgfsetdash{}{0pt}%
\pgfpathmoveto{\pgfqpoint{1.448111in}{2.594001in}}%
\pgfpathcurveto{\pgfqpoint{1.456347in}{2.594001in}}{\pgfqpoint{1.464247in}{2.597273in}}{\pgfqpoint{1.470071in}{2.603097in}}%
\pgfpathcurveto{\pgfqpoint{1.475895in}{2.608921in}}{\pgfqpoint{1.479167in}{2.616821in}}{\pgfqpoint{1.479167in}{2.625058in}}%
\pgfpathcurveto{\pgfqpoint{1.479167in}{2.633294in}}{\pgfqpoint{1.475895in}{2.641194in}}{\pgfqpoint{1.470071in}{2.647018in}}%
\pgfpathcurveto{\pgfqpoint{1.464247in}{2.652842in}}{\pgfqpoint{1.456347in}{2.656114in}}{\pgfqpoint{1.448111in}{2.656114in}}%
\pgfpathcurveto{\pgfqpoint{1.439874in}{2.656114in}}{\pgfqpoint{1.431974in}{2.652842in}}{\pgfqpoint{1.426150in}{2.647018in}}%
\pgfpathcurveto{\pgfqpoint{1.420326in}{2.641194in}}{\pgfqpoint{1.417054in}{2.633294in}}{\pgfqpoint{1.417054in}{2.625058in}}%
\pgfpathcurveto{\pgfqpoint{1.417054in}{2.616821in}}{\pgfqpoint{1.420326in}{2.608921in}}{\pgfqpoint{1.426150in}{2.603097in}}%
\pgfpathcurveto{\pgfqpoint{1.431974in}{2.597273in}}{\pgfqpoint{1.439874in}{2.594001in}}{\pgfqpoint{1.448111in}{2.594001in}}%
\pgfpathclose%
\pgfusepath{stroke,fill}%
\end{pgfscope}%
\begin{pgfscope}%
\pgfpathrectangle{\pgfqpoint{0.100000in}{0.212622in}}{\pgfqpoint{3.696000in}{3.696000in}}%
\pgfusepath{clip}%
\pgfsetbuttcap%
\pgfsetroundjoin%
\definecolor{currentfill}{rgb}{0.121569,0.466667,0.705882}%
\pgfsetfillcolor{currentfill}%
\pgfsetfillopacity{0.439891}%
\pgfsetlinewidth{1.003750pt}%
\definecolor{currentstroke}{rgb}{0.121569,0.466667,0.705882}%
\pgfsetstrokecolor{currentstroke}%
\pgfsetstrokeopacity{0.439891}%
\pgfsetdash{}{0pt}%
\pgfpathmoveto{\pgfqpoint{1.444169in}{2.586509in}}%
\pgfpathcurveto{\pgfqpoint{1.452405in}{2.586509in}}{\pgfqpoint{1.460305in}{2.589781in}}{\pgfqpoint{1.466129in}{2.595605in}}%
\pgfpathcurveto{\pgfqpoint{1.471953in}{2.601429in}}{\pgfqpoint{1.475225in}{2.609329in}}{\pgfqpoint{1.475225in}{2.617566in}}%
\pgfpathcurveto{\pgfqpoint{1.475225in}{2.625802in}}{\pgfqpoint{1.471953in}{2.633702in}}{\pgfqpoint{1.466129in}{2.639526in}}%
\pgfpathcurveto{\pgfqpoint{1.460305in}{2.645350in}}{\pgfqpoint{1.452405in}{2.648622in}}{\pgfqpoint{1.444169in}{2.648622in}}%
\pgfpathcurveto{\pgfqpoint{1.435933in}{2.648622in}}{\pgfqpoint{1.428033in}{2.645350in}}{\pgfqpoint{1.422209in}{2.639526in}}%
\pgfpathcurveto{\pgfqpoint{1.416385in}{2.633702in}}{\pgfqpoint{1.413112in}{2.625802in}}{\pgfqpoint{1.413112in}{2.617566in}}%
\pgfpathcurveto{\pgfqpoint{1.413112in}{2.609329in}}{\pgfqpoint{1.416385in}{2.601429in}}{\pgfqpoint{1.422209in}{2.595605in}}%
\pgfpathcurveto{\pgfqpoint{1.428033in}{2.589781in}}{\pgfqpoint{1.435933in}{2.586509in}}{\pgfqpoint{1.444169in}{2.586509in}}%
\pgfpathclose%
\pgfusepath{stroke,fill}%
\end{pgfscope}%
\begin{pgfscope}%
\pgfpathrectangle{\pgfqpoint{0.100000in}{0.212622in}}{\pgfqpoint{3.696000in}{3.696000in}}%
\pgfusepath{clip}%
\pgfsetbuttcap%
\pgfsetroundjoin%
\definecolor{currentfill}{rgb}{0.121569,0.466667,0.705882}%
\pgfsetfillcolor{currentfill}%
\pgfsetfillopacity{0.440976}%
\pgfsetlinewidth{1.003750pt}%
\definecolor{currentstroke}{rgb}{0.121569,0.466667,0.705882}%
\pgfsetstrokecolor{currentstroke}%
\pgfsetstrokeopacity{0.440976}%
\pgfsetdash{}{0pt}%
\pgfpathmoveto{\pgfqpoint{1.440415in}{2.579302in}}%
\pgfpathcurveto{\pgfqpoint{1.448651in}{2.579302in}}{\pgfqpoint{1.456551in}{2.582575in}}{\pgfqpoint{1.462375in}{2.588399in}}%
\pgfpathcurveto{\pgfqpoint{1.468199in}{2.594223in}}{\pgfqpoint{1.471471in}{2.602123in}}{\pgfqpoint{1.471471in}{2.610359in}}%
\pgfpathcurveto{\pgfqpoint{1.471471in}{2.618595in}}{\pgfqpoint{1.468199in}{2.626495in}}{\pgfqpoint{1.462375in}{2.632319in}}%
\pgfpathcurveto{\pgfqpoint{1.456551in}{2.638143in}}{\pgfqpoint{1.448651in}{2.641415in}}{\pgfqpoint{1.440415in}{2.641415in}}%
\pgfpathcurveto{\pgfqpoint{1.432179in}{2.641415in}}{\pgfqpoint{1.424279in}{2.638143in}}{\pgfqpoint{1.418455in}{2.632319in}}%
\pgfpathcurveto{\pgfqpoint{1.412631in}{2.626495in}}{\pgfqpoint{1.409358in}{2.618595in}}{\pgfqpoint{1.409358in}{2.610359in}}%
\pgfpathcurveto{\pgfqpoint{1.409358in}{2.602123in}}{\pgfqpoint{1.412631in}{2.594223in}}{\pgfqpoint{1.418455in}{2.588399in}}%
\pgfpathcurveto{\pgfqpoint{1.424279in}{2.582575in}}{\pgfqpoint{1.432179in}{2.579302in}}{\pgfqpoint{1.440415in}{2.579302in}}%
\pgfpathclose%
\pgfusepath{stroke,fill}%
\end{pgfscope}%
\begin{pgfscope}%
\pgfpathrectangle{\pgfqpoint{0.100000in}{0.212622in}}{\pgfqpoint{3.696000in}{3.696000in}}%
\pgfusepath{clip}%
\pgfsetbuttcap%
\pgfsetroundjoin%
\definecolor{currentfill}{rgb}{0.121569,0.466667,0.705882}%
\pgfsetfillcolor{currentfill}%
\pgfsetfillopacity{0.441822}%
\pgfsetlinewidth{1.003750pt}%
\definecolor{currentstroke}{rgb}{0.121569,0.466667,0.705882}%
\pgfsetstrokecolor{currentstroke}%
\pgfsetstrokeopacity{0.441822}%
\pgfsetdash{}{0pt}%
\pgfpathmoveto{\pgfqpoint{1.436769in}{2.572712in}}%
\pgfpathcurveto{\pgfqpoint{1.445006in}{2.572712in}}{\pgfqpoint{1.452906in}{2.575984in}}{\pgfqpoint{1.458730in}{2.581808in}}%
\pgfpathcurveto{\pgfqpoint{1.464554in}{2.587632in}}{\pgfqpoint{1.467826in}{2.595532in}}{\pgfqpoint{1.467826in}{2.603768in}}%
\pgfpathcurveto{\pgfqpoint{1.467826in}{2.612005in}}{\pgfqpoint{1.464554in}{2.619905in}}{\pgfqpoint{1.458730in}{2.625729in}}%
\pgfpathcurveto{\pgfqpoint{1.452906in}{2.631553in}}{\pgfqpoint{1.445006in}{2.634825in}}{\pgfqpoint{1.436769in}{2.634825in}}%
\pgfpathcurveto{\pgfqpoint{1.428533in}{2.634825in}}{\pgfqpoint{1.420633in}{2.631553in}}{\pgfqpoint{1.414809in}{2.625729in}}%
\pgfpathcurveto{\pgfqpoint{1.408985in}{2.619905in}}{\pgfqpoint{1.405713in}{2.612005in}}{\pgfqpoint{1.405713in}{2.603768in}}%
\pgfpathcurveto{\pgfqpoint{1.405713in}{2.595532in}}{\pgfqpoint{1.408985in}{2.587632in}}{\pgfqpoint{1.414809in}{2.581808in}}%
\pgfpathcurveto{\pgfqpoint{1.420633in}{2.575984in}}{\pgfqpoint{1.428533in}{2.572712in}}{\pgfqpoint{1.436769in}{2.572712in}}%
\pgfpathclose%
\pgfusepath{stroke,fill}%
\end{pgfscope}%
\begin{pgfscope}%
\pgfpathrectangle{\pgfqpoint{0.100000in}{0.212622in}}{\pgfqpoint{3.696000in}{3.696000in}}%
\pgfusepath{clip}%
\pgfsetbuttcap%
\pgfsetroundjoin%
\definecolor{currentfill}{rgb}{0.121569,0.466667,0.705882}%
\pgfsetfillcolor{currentfill}%
\pgfsetfillopacity{0.441871}%
\pgfsetlinewidth{1.003750pt}%
\definecolor{currentstroke}{rgb}{0.121569,0.466667,0.705882}%
\pgfsetstrokecolor{currentstroke}%
\pgfsetstrokeopacity{0.441871}%
\pgfsetdash{}{0pt}%
\pgfpathmoveto{\pgfqpoint{2.004057in}{2.745205in}}%
\pgfpathcurveto{\pgfqpoint{2.012293in}{2.745205in}}{\pgfqpoint{2.020193in}{2.748477in}}{\pgfqpoint{2.026017in}{2.754301in}}%
\pgfpathcurveto{\pgfqpoint{2.031841in}{2.760125in}}{\pgfqpoint{2.035114in}{2.768025in}}{\pgfqpoint{2.035114in}{2.776262in}}%
\pgfpathcurveto{\pgfqpoint{2.035114in}{2.784498in}}{\pgfqpoint{2.031841in}{2.792398in}}{\pgfqpoint{2.026017in}{2.798222in}}%
\pgfpathcurveto{\pgfqpoint{2.020193in}{2.804046in}}{\pgfqpoint{2.012293in}{2.807318in}}{\pgfqpoint{2.004057in}{2.807318in}}%
\pgfpathcurveto{\pgfqpoint{1.995821in}{2.807318in}}{\pgfqpoint{1.987921in}{2.804046in}}{\pgfqpoint{1.982097in}{2.798222in}}%
\pgfpathcurveto{\pgfqpoint{1.976273in}{2.792398in}}{\pgfqpoint{1.973001in}{2.784498in}}{\pgfqpoint{1.973001in}{2.776262in}}%
\pgfpathcurveto{\pgfqpoint{1.973001in}{2.768025in}}{\pgfqpoint{1.976273in}{2.760125in}}{\pgfqpoint{1.982097in}{2.754301in}}%
\pgfpathcurveto{\pgfqpoint{1.987921in}{2.748477in}}{\pgfqpoint{1.995821in}{2.745205in}}{\pgfqpoint{2.004057in}{2.745205in}}%
\pgfpathclose%
\pgfusepath{stroke,fill}%
\end{pgfscope}%
\begin{pgfscope}%
\pgfpathrectangle{\pgfqpoint{0.100000in}{0.212622in}}{\pgfqpoint{3.696000in}{3.696000in}}%
\pgfusepath{clip}%
\pgfsetbuttcap%
\pgfsetroundjoin%
\definecolor{currentfill}{rgb}{0.121569,0.466667,0.705882}%
\pgfsetfillcolor{currentfill}%
\pgfsetfillopacity{0.442480}%
\pgfsetlinewidth{1.003750pt}%
\definecolor{currentstroke}{rgb}{0.121569,0.466667,0.705882}%
\pgfsetstrokecolor{currentstroke}%
\pgfsetstrokeopacity{0.442480}%
\pgfsetdash{}{0pt}%
\pgfpathmoveto{\pgfqpoint{1.434504in}{2.567454in}}%
\pgfpathcurveto{\pgfqpoint{1.442740in}{2.567454in}}{\pgfqpoint{1.450640in}{2.570726in}}{\pgfqpoint{1.456464in}{2.576550in}}%
\pgfpathcurveto{\pgfqpoint{1.462288in}{2.582374in}}{\pgfqpoint{1.465560in}{2.590274in}}{\pgfqpoint{1.465560in}{2.598511in}}%
\pgfpathcurveto{\pgfqpoint{1.465560in}{2.606747in}}{\pgfqpoint{1.462288in}{2.614647in}}{\pgfqpoint{1.456464in}{2.620471in}}%
\pgfpathcurveto{\pgfqpoint{1.450640in}{2.626295in}}{\pgfqpoint{1.442740in}{2.629567in}}{\pgfqpoint{1.434504in}{2.629567in}}%
\pgfpathcurveto{\pgfqpoint{1.426268in}{2.629567in}}{\pgfqpoint{1.418368in}{2.626295in}}{\pgfqpoint{1.412544in}{2.620471in}}%
\pgfpathcurveto{\pgfqpoint{1.406720in}{2.614647in}}{\pgfqpoint{1.403447in}{2.606747in}}{\pgfqpoint{1.403447in}{2.598511in}}%
\pgfpathcurveto{\pgfqpoint{1.403447in}{2.590274in}}{\pgfqpoint{1.406720in}{2.582374in}}{\pgfqpoint{1.412544in}{2.576550in}}%
\pgfpathcurveto{\pgfqpoint{1.418368in}{2.570726in}}{\pgfqpoint{1.426268in}{2.567454in}}{\pgfqpoint{1.434504in}{2.567454in}}%
\pgfpathclose%
\pgfusepath{stroke,fill}%
\end{pgfscope}%
\begin{pgfscope}%
\pgfpathrectangle{\pgfqpoint{0.100000in}{0.212622in}}{\pgfqpoint{3.696000in}{3.696000in}}%
\pgfusepath{clip}%
\pgfsetbuttcap%
\pgfsetroundjoin%
\definecolor{currentfill}{rgb}{0.121569,0.466667,0.705882}%
\pgfsetfillcolor{currentfill}%
\pgfsetfillopacity{0.443032}%
\pgfsetlinewidth{1.003750pt}%
\definecolor{currentstroke}{rgb}{0.121569,0.466667,0.705882}%
\pgfsetstrokecolor{currentstroke}%
\pgfsetstrokeopacity{0.443032}%
\pgfsetdash{}{0pt}%
\pgfpathmoveto{\pgfqpoint{1.432392in}{2.563727in}}%
\pgfpathcurveto{\pgfqpoint{1.440628in}{2.563727in}}{\pgfqpoint{1.448528in}{2.566999in}}{\pgfqpoint{1.454352in}{2.572823in}}%
\pgfpathcurveto{\pgfqpoint{1.460176in}{2.578647in}}{\pgfqpoint{1.463448in}{2.586547in}}{\pgfqpoint{1.463448in}{2.594784in}}%
\pgfpathcurveto{\pgfqpoint{1.463448in}{2.603020in}}{\pgfqpoint{1.460176in}{2.610920in}}{\pgfqpoint{1.454352in}{2.616744in}}%
\pgfpathcurveto{\pgfqpoint{1.448528in}{2.622568in}}{\pgfqpoint{1.440628in}{2.625840in}}{\pgfqpoint{1.432392in}{2.625840in}}%
\pgfpathcurveto{\pgfqpoint{1.424156in}{2.625840in}}{\pgfqpoint{1.416256in}{2.622568in}}{\pgfqpoint{1.410432in}{2.616744in}}%
\pgfpathcurveto{\pgfqpoint{1.404608in}{2.610920in}}{\pgfqpoint{1.401335in}{2.603020in}}{\pgfqpoint{1.401335in}{2.594784in}}%
\pgfpathcurveto{\pgfqpoint{1.401335in}{2.586547in}}{\pgfqpoint{1.404608in}{2.578647in}}{\pgfqpoint{1.410432in}{2.572823in}}%
\pgfpathcurveto{\pgfqpoint{1.416256in}{2.566999in}}{\pgfqpoint{1.424156in}{2.563727in}}{\pgfqpoint{1.432392in}{2.563727in}}%
\pgfpathclose%
\pgfusepath{stroke,fill}%
\end{pgfscope}%
\begin{pgfscope}%
\pgfpathrectangle{\pgfqpoint{0.100000in}{0.212622in}}{\pgfqpoint{3.696000in}{3.696000in}}%
\pgfusepath{clip}%
\pgfsetbuttcap%
\pgfsetroundjoin%
\definecolor{currentfill}{rgb}{0.121569,0.466667,0.705882}%
\pgfsetfillcolor{currentfill}%
\pgfsetfillopacity{0.444004}%
\pgfsetlinewidth{1.003750pt}%
\definecolor{currentstroke}{rgb}{0.121569,0.466667,0.705882}%
\pgfsetstrokecolor{currentstroke}%
\pgfsetstrokeopacity{0.444004}%
\pgfsetdash{}{0pt}%
\pgfpathmoveto{\pgfqpoint{1.428822in}{2.556429in}}%
\pgfpathcurveto{\pgfqpoint{1.437059in}{2.556429in}}{\pgfqpoint{1.444959in}{2.559701in}}{\pgfqpoint{1.450783in}{2.565525in}}%
\pgfpathcurveto{\pgfqpoint{1.456606in}{2.571349in}}{\pgfqpoint{1.459879in}{2.579249in}}{\pgfqpoint{1.459879in}{2.587486in}}%
\pgfpathcurveto{\pgfqpoint{1.459879in}{2.595722in}}{\pgfqpoint{1.456606in}{2.603622in}}{\pgfqpoint{1.450783in}{2.609446in}}%
\pgfpathcurveto{\pgfqpoint{1.444959in}{2.615270in}}{\pgfqpoint{1.437059in}{2.618542in}}{\pgfqpoint{1.428822in}{2.618542in}}%
\pgfpathcurveto{\pgfqpoint{1.420586in}{2.618542in}}{\pgfqpoint{1.412686in}{2.615270in}}{\pgfqpoint{1.406862in}{2.609446in}}%
\pgfpathcurveto{\pgfqpoint{1.401038in}{2.603622in}}{\pgfqpoint{1.397766in}{2.595722in}}{\pgfqpoint{1.397766in}{2.587486in}}%
\pgfpathcurveto{\pgfqpoint{1.397766in}{2.579249in}}{\pgfqpoint{1.401038in}{2.571349in}}{\pgfqpoint{1.406862in}{2.565525in}}%
\pgfpathcurveto{\pgfqpoint{1.412686in}{2.559701in}}{\pgfqpoint{1.420586in}{2.556429in}}{\pgfqpoint{1.428822in}{2.556429in}}%
\pgfpathclose%
\pgfusepath{stroke,fill}%
\end{pgfscope}%
\begin{pgfscope}%
\pgfpathrectangle{\pgfqpoint{0.100000in}{0.212622in}}{\pgfqpoint{3.696000in}{3.696000in}}%
\pgfusepath{clip}%
\pgfsetbuttcap%
\pgfsetroundjoin%
\definecolor{currentfill}{rgb}{0.121569,0.466667,0.705882}%
\pgfsetfillcolor{currentfill}%
\pgfsetfillopacity{0.444450}%
\pgfsetlinewidth{1.003750pt}%
\definecolor{currentstroke}{rgb}{0.121569,0.466667,0.705882}%
\pgfsetstrokecolor{currentstroke}%
\pgfsetstrokeopacity{0.444450}%
\pgfsetdash{}{0pt}%
\pgfpathmoveto{\pgfqpoint{2.005979in}{2.734551in}}%
\pgfpathcurveto{\pgfqpoint{2.014215in}{2.734551in}}{\pgfqpoint{2.022115in}{2.737824in}}{\pgfqpoint{2.027939in}{2.743648in}}%
\pgfpathcurveto{\pgfqpoint{2.033763in}{2.749472in}}{\pgfqpoint{2.037035in}{2.757372in}}{\pgfqpoint{2.037035in}{2.765608in}}%
\pgfpathcurveto{\pgfqpoint{2.037035in}{2.773844in}}{\pgfqpoint{2.033763in}{2.781744in}}{\pgfqpoint{2.027939in}{2.787568in}}%
\pgfpathcurveto{\pgfqpoint{2.022115in}{2.793392in}}{\pgfqpoint{2.014215in}{2.796664in}}{\pgfqpoint{2.005979in}{2.796664in}}%
\pgfpathcurveto{\pgfqpoint{1.997742in}{2.796664in}}{\pgfqpoint{1.989842in}{2.793392in}}{\pgfqpoint{1.984018in}{2.787568in}}%
\pgfpathcurveto{\pgfqpoint{1.978195in}{2.781744in}}{\pgfqpoint{1.974922in}{2.773844in}}{\pgfqpoint{1.974922in}{2.765608in}}%
\pgfpathcurveto{\pgfqpoint{1.974922in}{2.757372in}}{\pgfqpoint{1.978195in}{2.749472in}}{\pgfqpoint{1.984018in}{2.743648in}}%
\pgfpathcurveto{\pgfqpoint{1.989842in}{2.737824in}}{\pgfqpoint{1.997742in}{2.734551in}}{\pgfqpoint{2.005979in}{2.734551in}}%
\pgfpathclose%
\pgfusepath{stroke,fill}%
\end{pgfscope}%
\begin{pgfscope}%
\pgfpathrectangle{\pgfqpoint{0.100000in}{0.212622in}}{\pgfqpoint{3.696000in}{3.696000in}}%
\pgfusepath{clip}%
\pgfsetbuttcap%
\pgfsetroundjoin%
\definecolor{currentfill}{rgb}{0.121569,0.466667,0.705882}%
\pgfsetfillcolor{currentfill}%
\pgfsetfillopacity{0.444954}%
\pgfsetlinewidth{1.003750pt}%
\definecolor{currentstroke}{rgb}{0.121569,0.466667,0.705882}%
\pgfsetstrokecolor{currentstroke}%
\pgfsetstrokeopacity{0.444954}%
\pgfsetdash{}{0pt}%
\pgfpathmoveto{\pgfqpoint{1.425601in}{2.549988in}}%
\pgfpathcurveto{\pgfqpoint{1.433837in}{2.549988in}}{\pgfqpoint{1.441737in}{2.553260in}}{\pgfqpoint{1.447561in}{2.559084in}}%
\pgfpathcurveto{\pgfqpoint{1.453385in}{2.564908in}}{\pgfqpoint{1.456657in}{2.572808in}}{\pgfqpoint{1.456657in}{2.581044in}}%
\pgfpathcurveto{\pgfqpoint{1.456657in}{2.589280in}}{\pgfqpoint{1.453385in}{2.597180in}}{\pgfqpoint{1.447561in}{2.603004in}}%
\pgfpathcurveto{\pgfqpoint{1.441737in}{2.608828in}}{\pgfqpoint{1.433837in}{2.612101in}}{\pgfqpoint{1.425601in}{2.612101in}}%
\pgfpathcurveto{\pgfqpoint{1.417364in}{2.612101in}}{\pgfqpoint{1.409464in}{2.608828in}}{\pgfqpoint{1.403640in}{2.603004in}}%
\pgfpathcurveto{\pgfqpoint{1.397816in}{2.597180in}}{\pgfqpoint{1.394544in}{2.589280in}}{\pgfqpoint{1.394544in}{2.581044in}}%
\pgfpathcurveto{\pgfqpoint{1.394544in}{2.572808in}}{\pgfqpoint{1.397816in}{2.564908in}}{\pgfqpoint{1.403640in}{2.559084in}}%
\pgfpathcurveto{\pgfqpoint{1.409464in}{2.553260in}}{\pgfqpoint{1.417364in}{2.549988in}}{\pgfqpoint{1.425601in}{2.549988in}}%
\pgfpathclose%
\pgfusepath{stroke,fill}%
\end{pgfscope}%
\begin{pgfscope}%
\pgfpathrectangle{\pgfqpoint{0.100000in}{0.212622in}}{\pgfqpoint{3.696000in}{3.696000in}}%
\pgfusepath{clip}%
\pgfsetbuttcap%
\pgfsetroundjoin%
\definecolor{currentfill}{rgb}{0.121569,0.466667,0.705882}%
\pgfsetfillcolor{currentfill}%
\pgfsetfillopacity{0.445693}%
\pgfsetlinewidth{1.003750pt}%
\definecolor{currentstroke}{rgb}{0.121569,0.466667,0.705882}%
\pgfsetstrokecolor{currentstroke}%
\pgfsetstrokeopacity{0.445693}%
\pgfsetdash{}{0pt}%
\pgfpathmoveto{\pgfqpoint{1.422283in}{2.544187in}}%
\pgfpathcurveto{\pgfqpoint{1.430519in}{2.544187in}}{\pgfqpoint{1.438419in}{2.547459in}}{\pgfqpoint{1.444243in}{2.553283in}}%
\pgfpathcurveto{\pgfqpoint{1.450067in}{2.559107in}}{\pgfqpoint{1.453339in}{2.567007in}}{\pgfqpoint{1.453339in}{2.575244in}}%
\pgfpathcurveto{\pgfqpoint{1.453339in}{2.583480in}}{\pgfqpoint{1.450067in}{2.591380in}}{\pgfqpoint{1.444243in}{2.597204in}}%
\pgfpathcurveto{\pgfqpoint{1.438419in}{2.603028in}}{\pgfqpoint{1.430519in}{2.606300in}}{\pgfqpoint{1.422283in}{2.606300in}}%
\pgfpathcurveto{\pgfqpoint{1.414046in}{2.606300in}}{\pgfqpoint{1.406146in}{2.603028in}}{\pgfqpoint{1.400323in}{2.597204in}}%
\pgfpathcurveto{\pgfqpoint{1.394499in}{2.591380in}}{\pgfqpoint{1.391226in}{2.583480in}}{\pgfqpoint{1.391226in}{2.575244in}}%
\pgfpathcurveto{\pgfqpoint{1.391226in}{2.567007in}}{\pgfqpoint{1.394499in}{2.559107in}}{\pgfqpoint{1.400323in}{2.553283in}}%
\pgfpathcurveto{\pgfqpoint{1.406146in}{2.547459in}}{\pgfqpoint{1.414046in}{2.544187in}}{\pgfqpoint{1.422283in}{2.544187in}}%
\pgfpathclose%
\pgfusepath{stroke,fill}%
\end{pgfscope}%
\begin{pgfscope}%
\pgfpathrectangle{\pgfqpoint{0.100000in}{0.212622in}}{\pgfqpoint{3.696000in}{3.696000in}}%
\pgfusepath{clip}%
\pgfsetbuttcap%
\pgfsetroundjoin%
\definecolor{currentfill}{rgb}{0.121569,0.466667,0.705882}%
\pgfsetfillcolor{currentfill}%
\pgfsetfillopacity{0.446154}%
\pgfsetlinewidth{1.003750pt}%
\definecolor{currentstroke}{rgb}{0.121569,0.466667,0.705882}%
\pgfsetstrokecolor{currentstroke}%
\pgfsetstrokeopacity{0.446154}%
\pgfsetdash{}{0pt}%
\pgfpathmoveto{\pgfqpoint{1.420570in}{2.540205in}}%
\pgfpathcurveto{\pgfqpoint{1.428806in}{2.540205in}}{\pgfqpoint{1.436706in}{2.543477in}}{\pgfqpoint{1.442530in}{2.549301in}}%
\pgfpathcurveto{\pgfqpoint{1.448354in}{2.555125in}}{\pgfqpoint{1.451626in}{2.563025in}}{\pgfqpoint{1.451626in}{2.571261in}}%
\pgfpathcurveto{\pgfqpoint{1.451626in}{2.579498in}}{\pgfqpoint{1.448354in}{2.587398in}}{\pgfqpoint{1.442530in}{2.593222in}}%
\pgfpathcurveto{\pgfqpoint{1.436706in}{2.599046in}}{\pgfqpoint{1.428806in}{2.602318in}}{\pgfqpoint{1.420570in}{2.602318in}}%
\pgfpathcurveto{\pgfqpoint{1.412334in}{2.602318in}}{\pgfqpoint{1.404434in}{2.599046in}}{\pgfqpoint{1.398610in}{2.593222in}}%
\pgfpathcurveto{\pgfqpoint{1.392786in}{2.587398in}}{\pgfqpoint{1.389513in}{2.579498in}}{\pgfqpoint{1.389513in}{2.571261in}}%
\pgfpathcurveto{\pgfqpoint{1.389513in}{2.563025in}}{\pgfqpoint{1.392786in}{2.555125in}}{\pgfqpoint{1.398610in}{2.549301in}}%
\pgfpathcurveto{\pgfqpoint{1.404434in}{2.543477in}}{\pgfqpoint{1.412334in}{2.540205in}}{\pgfqpoint{1.420570in}{2.540205in}}%
\pgfpathclose%
\pgfusepath{stroke,fill}%
\end{pgfscope}%
\begin{pgfscope}%
\pgfpathrectangle{\pgfqpoint{0.100000in}{0.212622in}}{\pgfqpoint{3.696000in}{3.696000in}}%
\pgfusepath{clip}%
\pgfsetbuttcap%
\pgfsetroundjoin%
\definecolor{currentfill}{rgb}{0.121569,0.466667,0.705882}%
\pgfsetfillcolor{currentfill}%
\pgfsetfillopacity{0.446576}%
\pgfsetlinewidth{1.003750pt}%
\definecolor{currentstroke}{rgb}{0.121569,0.466667,0.705882}%
\pgfsetstrokecolor{currentstroke}%
\pgfsetstrokeopacity{0.446576}%
\pgfsetdash{}{0pt}%
\pgfpathmoveto{\pgfqpoint{1.418940in}{2.537166in}}%
\pgfpathcurveto{\pgfqpoint{1.427176in}{2.537166in}}{\pgfqpoint{1.435076in}{2.540439in}}{\pgfqpoint{1.440900in}{2.546263in}}%
\pgfpathcurveto{\pgfqpoint{1.446724in}{2.552086in}}{\pgfqpoint{1.449996in}{2.559987in}}{\pgfqpoint{1.449996in}{2.568223in}}%
\pgfpathcurveto{\pgfqpoint{1.449996in}{2.576459in}}{\pgfqpoint{1.446724in}{2.584359in}}{\pgfqpoint{1.440900in}{2.590183in}}%
\pgfpathcurveto{\pgfqpoint{1.435076in}{2.596007in}}{\pgfqpoint{1.427176in}{2.599279in}}{\pgfqpoint{1.418940in}{2.599279in}}%
\pgfpathcurveto{\pgfqpoint{1.410703in}{2.599279in}}{\pgfqpoint{1.402803in}{2.596007in}}{\pgfqpoint{1.396979in}{2.590183in}}%
\pgfpathcurveto{\pgfqpoint{1.391156in}{2.584359in}}{\pgfqpoint{1.387883in}{2.576459in}}{\pgfqpoint{1.387883in}{2.568223in}}%
\pgfpathcurveto{\pgfqpoint{1.387883in}{2.559987in}}{\pgfqpoint{1.391156in}{2.552086in}}{\pgfqpoint{1.396979in}{2.546263in}}%
\pgfpathcurveto{\pgfqpoint{1.402803in}{2.540439in}}{\pgfqpoint{1.410703in}{2.537166in}}{\pgfqpoint{1.418940in}{2.537166in}}%
\pgfpathclose%
\pgfusepath{stroke,fill}%
\end{pgfscope}%
\begin{pgfscope}%
\pgfpathrectangle{\pgfqpoint{0.100000in}{0.212622in}}{\pgfqpoint{3.696000in}{3.696000in}}%
\pgfusepath{clip}%
\pgfsetbuttcap%
\pgfsetroundjoin%
\definecolor{currentfill}{rgb}{0.121569,0.466667,0.705882}%
\pgfsetfillcolor{currentfill}%
\pgfsetfillopacity{0.447282}%
\pgfsetlinewidth{1.003750pt}%
\definecolor{currentstroke}{rgb}{0.121569,0.466667,0.705882}%
\pgfsetstrokecolor{currentstroke}%
\pgfsetstrokeopacity{0.447282}%
\pgfsetdash{}{0pt}%
\pgfpathmoveto{\pgfqpoint{1.415874in}{2.531577in}}%
\pgfpathcurveto{\pgfqpoint{1.424110in}{2.531577in}}{\pgfqpoint{1.432010in}{2.534849in}}{\pgfqpoint{1.437834in}{2.540673in}}%
\pgfpathcurveto{\pgfqpoint{1.443658in}{2.546497in}}{\pgfqpoint{1.446930in}{2.554397in}}{\pgfqpoint{1.446930in}{2.562633in}}%
\pgfpathcurveto{\pgfqpoint{1.446930in}{2.570870in}}{\pgfqpoint{1.443658in}{2.578770in}}{\pgfqpoint{1.437834in}{2.584594in}}%
\pgfpathcurveto{\pgfqpoint{1.432010in}{2.590418in}}{\pgfqpoint{1.424110in}{2.593690in}}{\pgfqpoint{1.415874in}{2.593690in}}%
\pgfpathcurveto{\pgfqpoint{1.407637in}{2.593690in}}{\pgfqpoint{1.399737in}{2.590418in}}{\pgfqpoint{1.393913in}{2.584594in}}%
\pgfpathcurveto{\pgfqpoint{1.388089in}{2.578770in}}{\pgfqpoint{1.384817in}{2.570870in}}{\pgfqpoint{1.384817in}{2.562633in}}%
\pgfpathcurveto{\pgfqpoint{1.384817in}{2.554397in}}{\pgfqpoint{1.388089in}{2.546497in}}{\pgfqpoint{1.393913in}{2.540673in}}%
\pgfpathcurveto{\pgfqpoint{1.399737in}{2.534849in}}{\pgfqpoint{1.407637in}{2.531577in}}{\pgfqpoint{1.415874in}{2.531577in}}%
\pgfpathclose%
\pgfusepath{stroke,fill}%
\end{pgfscope}%
\begin{pgfscope}%
\pgfpathrectangle{\pgfqpoint{0.100000in}{0.212622in}}{\pgfqpoint{3.696000in}{3.696000in}}%
\pgfusepath{clip}%
\pgfsetbuttcap%
\pgfsetroundjoin%
\definecolor{currentfill}{rgb}{0.121569,0.466667,0.705882}%
\pgfsetfillcolor{currentfill}%
\pgfsetfillopacity{0.447608}%
\pgfsetlinewidth{1.003750pt}%
\definecolor{currentstroke}{rgb}{0.121569,0.466667,0.705882}%
\pgfsetstrokecolor{currentstroke}%
\pgfsetstrokeopacity{0.447608}%
\pgfsetdash{}{0pt}%
\pgfpathmoveto{\pgfqpoint{2.007390in}{2.723331in}}%
\pgfpathcurveto{\pgfqpoint{2.015626in}{2.723331in}}{\pgfqpoint{2.023526in}{2.726603in}}{\pgfqpoint{2.029350in}{2.732427in}}%
\pgfpathcurveto{\pgfqpoint{2.035174in}{2.738251in}}{\pgfqpoint{2.038447in}{2.746151in}}{\pgfqpoint{2.038447in}{2.754388in}}%
\pgfpathcurveto{\pgfqpoint{2.038447in}{2.762624in}}{\pgfqpoint{2.035174in}{2.770524in}}{\pgfqpoint{2.029350in}{2.776348in}}%
\pgfpathcurveto{\pgfqpoint{2.023526in}{2.782172in}}{\pgfqpoint{2.015626in}{2.785444in}}{\pgfqpoint{2.007390in}{2.785444in}}%
\pgfpathcurveto{\pgfqpoint{1.999154in}{2.785444in}}{\pgfqpoint{1.991254in}{2.782172in}}{\pgfqpoint{1.985430in}{2.776348in}}%
\pgfpathcurveto{\pgfqpoint{1.979606in}{2.770524in}}{\pgfqpoint{1.976334in}{2.762624in}}{\pgfqpoint{1.976334in}{2.754388in}}%
\pgfpathcurveto{\pgfqpoint{1.976334in}{2.746151in}}{\pgfqpoint{1.979606in}{2.738251in}}{\pgfqpoint{1.985430in}{2.732427in}}%
\pgfpathcurveto{\pgfqpoint{1.991254in}{2.726603in}}{\pgfqpoint{1.999154in}{2.723331in}}{\pgfqpoint{2.007390in}{2.723331in}}%
\pgfpathclose%
\pgfusepath{stroke,fill}%
\end{pgfscope}%
\begin{pgfscope}%
\pgfpathrectangle{\pgfqpoint{0.100000in}{0.212622in}}{\pgfqpoint{3.696000in}{3.696000in}}%
\pgfusepath{clip}%
\pgfsetbuttcap%
\pgfsetroundjoin%
\definecolor{currentfill}{rgb}{0.121569,0.466667,0.705882}%
\pgfsetfillcolor{currentfill}%
\pgfsetfillopacity{0.447766}%
\pgfsetlinewidth{1.003750pt}%
\definecolor{currentstroke}{rgb}{0.121569,0.466667,0.705882}%
\pgfsetstrokecolor{currentstroke}%
\pgfsetstrokeopacity{0.447766}%
\pgfsetdash{}{0pt}%
\pgfpathmoveto{\pgfqpoint{1.414197in}{2.527528in}}%
\pgfpathcurveto{\pgfqpoint{1.422433in}{2.527528in}}{\pgfqpoint{1.430333in}{2.530801in}}{\pgfqpoint{1.436157in}{2.536625in}}%
\pgfpathcurveto{\pgfqpoint{1.441981in}{2.542449in}}{\pgfqpoint{1.445253in}{2.550349in}}{\pgfqpoint{1.445253in}{2.558585in}}%
\pgfpathcurveto{\pgfqpoint{1.445253in}{2.566821in}}{\pgfqpoint{1.441981in}{2.574721in}}{\pgfqpoint{1.436157in}{2.580545in}}%
\pgfpathcurveto{\pgfqpoint{1.430333in}{2.586369in}}{\pgfqpoint{1.422433in}{2.589641in}}{\pgfqpoint{1.414197in}{2.589641in}}%
\pgfpathcurveto{\pgfqpoint{1.405960in}{2.589641in}}{\pgfqpoint{1.398060in}{2.586369in}}{\pgfqpoint{1.392236in}{2.580545in}}%
\pgfpathcurveto{\pgfqpoint{1.386412in}{2.574721in}}{\pgfqpoint{1.383140in}{2.566821in}}{\pgfqpoint{1.383140in}{2.558585in}}%
\pgfpathcurveto{\pgfqpoint{1.383140in}{2.550349in}}{\pgfqpoint{1.386412in}{2.542449in}}{\pgfqpoint{1.392236in}{2.536625in}}%
\pgfpathcurveto{\pgfqpoint{1.398060in}{2.530801in}}{\pgfqpoint{1.405960in}{2.527528in}}{\pgfqpoint{1.414197in}{2.527528in}}%
\pgfpathclose%
\pgfusepath{stroke,fill}%
\end{pgfscope}%
\begin{pgfscope}%
\pgfpathrectangle{\pgfqpoint{0.100000in}{0.212622in}}{\pgfqpoint{3.696000in}{3.696000in}}%
\pgfusepath{clip}%
\pgfsetbuttcap%
\pgfsetroundjoin%
\definecolor{currentfill}{rgb}{0.121569,0.466667,0.705882}%
\pgfsetfillcolor{currentfill}%
\pgfsetfillopacity{0.448047}%
\pgfsetlinewidth{1.003750pt}%
\definecolor{currentstroke}{rgb}{0.121569,0.466667,0.705882}%
\pgfsetstrokecolor{currentstroke}%
\pgfsetstrokeopacity{0.448047}%
\pgfsetdash{}{0pt}%
\pgfpathmoveto{\pgfqpoint{1.413058in}{2.525519in}}%
\pgfpathcurveto{\pgfqpoint{1.421295in}{2.525519in}}{\pgfqpoint{1.429195in}{2.528791in}}{\pgfqpoint{1.435019in}{2.534615in}}%
\pgfpathcurveto{\pgfqpoint{1.440843in}{2.540439in}}{\pgfqpoint{1.444115in}{2.548339in}}{\pgfqpoint{1.444115in}{2.556575in}}%
\pgfpathcurveto{\pgfqpoint{1.444115in}{2.564812in}}{\pgfqpoint{1.440843in}{2.572712in}}{\pgfqpoint{1.435019in}{2.578536in}}%
\pgfpathcurveto{\pgfqpoint{1.429195in}{2.584360in}}{\pgfqpoint{1.421295in}{2.587632in}}{\pgfqpoint{1.413058in}{2.587632in}}%
\pgfpathcurveto{\pgfqpoint{1.404822in}{2.587632in}}{\pgfqpoint{1.396922in}{2.584360in}}{\pgfqpoint{1.391098in}{2.578536in}}%
\pgfpathcurveto{\pgfqpoint{1.385274in}{2.572712in}}{\pgfqpoint{1.382002in}{2.564812in}}{\pgfqpoint{1.382002in}{2.556575in}}%
\pgfpathcurveto{\pgfqpoint{1.382002in}{2.548339in}}{\pgfqpoint{1.385274in}{2.540439in}}{\pgfqpoint{1.391098in}{2.534615in}}%
\pgfpathcurveto{\pgfqpoint{1.396922in}{2.528791in}}{\pgfqpoint{1.404822in}{2.525519in}}{\pgfqpoint{1.413058in}{2.525519in}}%
\pgfpathclose%
\pgfusepath{stroke,fill}%
\end{pgfscope}%
\begin{pgfscope}%
\pgfpathrectangle{\pgfqpoint{0.100000in}{0.212622in}}{\pgfqpoint{3.696000in}{3.696000in}}%
\pgfusepath{clip}%
\pgfsetbuttcap%
\pgfsetroundjoin%
\definecolor{currentfill}{rgb}{0.121569,0.466667,0.705882}%
\pgfsetfillcolor{currentfill}%
\pgfsetfillopacity{0.448544}%
\pgfsetlinewidth{1.003750pt}%
\definecolor{currentstroke}{rgb}{0.121569,0.466667,0.705882}%
\pgfsetstrokecolor{currentstroke}%
\pgfsetstrokeopacity{0.448544}%
\pgfsetdash{}{0pt}%
\pgfpathmoveto{\pgfqpoint{1.411011in}{2.521777in}}%
\pgfpathcurveto{\pgfqpoint{1.419247in}{2.521777in}}{\pgfqpoint{1.427147in}{2.525049in}}{\pgfqpoint{1.432971in}{2.530873in}}%
\pgfpathcurveto{\pgfqpoint{1.438795in}{2.536697in}}{\pgfqpoint{1.442067in}{2.544597in}}{\pgfqpoint{1.442067in}{2.552833in}}%
\pgfpathcurveto{\pgfqpoint{1.442067in}{2.561070in}}{\pgfqpoint{1.438795in}{2.568970in}}{\pgfqpoint{1.432971in}{2.574794in}}%
\pgfpathcurveto{\pgfqpoint{1.427147in}{2.580618in}}{\pgfqpoint{1.419247in}{2.583890in}}{\pgfqpoint{1.411011in}{2.583890in}}%
\pgfpathcurveto{\pgfqpoint{1.402775in}{2.583890in}}{\pgfqpoint{1.394874in}{2.580618in}}{\pgfqpoint{1.389051in}{2.574794in}}%
\pgfpathcurveto{\pgfqpoint{1.383227in}{2.568970in}}{\pgfqpoint{1.379954in}{2.561070in}}{\pgfqpoint{1.379954in}{2.552833in}}%
\pgfpathcurveto{\pgfqpoint{1.379954in}{2.544597in}}{\pgfqpoint{1.383227in}{2.536697in}}{\pgfqpoint{1.389051in}{2.530873in}}%
\pgfpathcurveto{\pgfqpoint{1.394874in}{2.525049in}}{\pgfqpoint{1.402775in}{2.521777in}}{\pgfqpoint{1.411011in}{2.521777in}}%
\pgfpathclose%
\pgfusepath{stroke,fill}%
\end{pgfscope}%
\begin{pgfscope}%
\pgfpathrectangle{\pgfqpoint{0.100000in}{0.212622in}}{\pgfqpoint{3.696000in}{3.696000in}}%
\pgfusepath{clip}%
\pgfsetbuttcap%
\pgfsetroundjoin%
\definecolor{currentfill}{rgb}{0.121569,0.466667,0.705882}%
\pgfsetfillcolor{currentfill}%
\pgfsetfillopacity{0.449467}%
\pgfsetlinewidth{1.003750pt}%
\definecolor{currentstroke}{rgb}{0.121569,0.466667,0.705882}%
\pgfsetstrokecolor{currentstroke}%
\pgfsetstrokeopacity{0.449467}%
\pgfsetdash{}{0pt}%
\pgfpathmoveto{\pgfqpoint{1.407957in}{2.514185in}}%
\pgfpathcurveto{\pgfqpoint{1.416193in}{2.514185in}}{\pgfqpoint{1.424093in}{2.517457in}}{\pgfqpoint{1.429917in}{2.523281in}}%
\pgfpathcurveto{\pgfqpoint{1.435741in}{2.529105in}}{\pgfqpoint{1.439013in}{2.537005in}}{\pgfqpoint{1.439013in}{2.545241in}}%
\pgfpathcurveto{\pgfqpoint{1.439013in}{2.553477in}}{\pgfqpoint{1.435741in}{2.561378in}}{\pgfqpoint{1.429917in}{2.567201in}}%
\pgfpathcurveto{\pgfqpoint{1.424093in}{2.573025in}}{\pgfqpoint{1.416193in}{2.576298in}}{\pgfqpoint{1.407957in}{2.576298in}}%
\pgfpathcurveto{\pgfqpoint{1.399721in}{2.576298in}}{\pgfqpoint{1.391820in}{2.573025in}}{\pgfqpoint{1.385997in}{2.567201in}}%
\pgfpathcurveto{\pgfqpoint{1.380173in}{2.561378in}}{\pgfqpoint{1.376900in}{2.553477in}}{\pgfqpoint{1.376900in}{2.545241in}}%
\pgfpathcurveto{\pgfqpoint{1.376900in}{2.537005in}}{\pgfqpoint{1.380173in}{2.529105in}}{\pgfqpoint{1.385997in}{2.523281in}}%
\pgfpathcurveto{\pgfqpoint{1.391820in}{2.517457in}}{\pgfqpoint{1.399721in}{2.514185in}}{\pgfqpoint{1.407957in}{2.514185in}}%
\pgfpathclose%
\pgfusepath{stroke,fill}%
\end{pgfscope}%
\begin{pgfscope}%
\pgfpathrectangle{\pgfqpoint{0.100000in}{0.212622in}}{\pgfqpoint{3.696000in}{3.696000in}}%
\pgfusepath{clip}%
\pgfsetbuttcap%
\pgfsetroundjoin%
\definecolor{currentfill}{rgb}{0.121569,0.466667,0.705882}%
\pgfsetfillcolor{currentfill}%
\pgfsetfillopacity{0.450952}%
\pgfsetlinewidth{1.003750pt}%
\definecolor{currentstroke}{rgb}{0.121569,0.466667,0.705882}%
\pgfsetstrokecolor{currentstroke}%
\pgfsetstrokeopacity{0.450952}%
\pgfsetdash{}{0pt}%
\pgfpathmoveto{\pgfqpoint{2.010208in}{2.710057in}}%
\pgfpathcurveto{\pgfqpoint{2.018444in}{2.710057in}}{\pgfqpoint{2.026344in}{2.713330in}}{\pgfqpoint{2.032168in}{2.719154in}}%
\pgfpathcurveto{\pgfqpoint{2.037992in}{2.724977in}}{\pgfqpoint{2.041265in}{2.732878in}}{\pgfqpoint{2.041265in}{2.741114in}}%
\pgfpathcurveto{\pgfqpoint{2.041265in}{2.749350in}}{\pgfqpoint{2.037992in}{2.757250in}}{\pgfqpoint{2.032168in}{2.763074in}}%
\pgfpathcurveto{\pgfqpoint{2.026344in}{2.768898in}}{\pgfqpoint{2.018444in}{2.772170in}}{\pgfqpoint{2.010208in}{2.772170in}}%
\pgfpathcurveto{\pgfqpoint{2.001972in}{2.772170in}}{\pgfqpoint{1.994072in}{2.768898in}}{\pgfqpoint{1.988248in}{2.763074in}}%
\pgfpathcurveto{\pgfqpoint{1.982424in}{2.757250in}}{\pgfqpoint{1.979152in}{2.749350in}}{\pgfqpoint{1.979152in}{2.741114in}}%
\pgfpathcurveto{\pgfqpoint{1.979152in}{2.732878in}}{\pgfqpoint{1.982424in}{2.724977in}}{\pgfqpoint{1.988248in}{2.719154in}}%
\pgfpathcurveto{\pgfqpoint{1.994072in}{2.713330in}}{\pgfqpoint{2.001972in}{2.710057in}}{\pgfqpoint{2.010208in}{2.710057in}}%
\pgfpathclose%
\pgfusepath{stroke,fill}%
\end{pgfscope}%
\begin{pgfscope}%
\pgfpathrectangle{\pgfqpoint{0.100000in}{0.212622in}}{\pgfqpoint{3.696000in}{3.696000in}}%
\pgfusepath{clip}%
\pgfsetbuttcap%
\pgfsetroundjoin%
\definecolor{currentfill}{rgb}{0.121569,0.466667,0.705882}%
\pgfsetfillcolor{currentfill}%
\pgfsetfillopacity{0.451372}%
\pgfsetlinewidth{1.003750pt}%
\definecolor{currentstroke}{rgb}{0.121569,0.466667,0.705882}%
\pgfsetstrokecolor{currentstroke}%
\pgfsetstrokeopacity{0.451372}%
\pgfsetdash{}{0pt}%
\pgfpathmoveto{\pgfqpoint{1.401538in}{2.502277in}}%
\pgfpathcurveto{\pgfqpoint{1.409775in}{2.502277in}}{\pgfqpoint{1.417675in}{2.505549in}}{\pgfqpoint{1.423499in}{2.511373in}}%
\pgfpathcurveto{\pgfqpoint{1.429323in}{2.517197in}}{\pgfqpoint{1.432595in}{2.525097in}}{\pgfqpoint{1.432595in}{2.533334in}}%
\pgfpathcurveto{\pgfqpoint{1.432595in}{2.541570in}}{\pgfqpoint{1.429323in}{2.549470in}}{\pgfqpoint{1.423499in}{2.555294in}}%
\pgfpathcurveto{\pgfqpoint{1.417675in}{2.561118in}}{\pgfqpoint{1.409775in}{2.564390in}}{\pgfqpoint{1.401538in}{2.564390in}}%
\pgfpathcurveto{\pgfqpoint{1.393302in}{2.564390in}}{\pgfqpoint{1.385402in}{2.561118in}}{\pgfqpoint{1.379578in}{2.555294in}}%
\pgfpathcurveto{\pgfqpoint{1.373754in}{2.549470in}}{\pgfqpoint{1.370482in}{2.541570in}}{\pgfqpoint{1.370482in}{2.533334in}}%
\pgfpathcurveto{\pgfqpoint{1.370482in}{2.525097in}}{\pgfqpoint{1.373754in}{2.517197in}}{\pgfqpoint{1.379578in}{2.511373in}}%
\pgfpathcurveto{\pgfqpoint{1.385402in}{2.505549in}}{\pgfqpoint{1.393302in}{2.502277in}}{\pgfqpoint{1.401538in}{2.502277in}}%
\pgfpathclose%
\pgfusepath{stroke,fill}%
\end{pgfscope}%
\begin{pgfscope}%
\pgfpathrectangle{\pgfqpoint{0.100000in}{0.212622in}}{\pgfqpoint{3.696000in}{3.696000in}}%
\pgfusepath{clip}%
\pgfsetbuttcap%
\pgfsetroundjoin%
\definecolor{currentfill}{rgb}{0.121569,0.466667,0.705882}%
\pgfsetfillcolor{currentfill}%
\pgfsetfillopacity{0.453132}%
\pgfsetlinewidth{1.003750pt}%
\definecolor{currentstroke}{rgb}{0.121569,0.466667,0.705882}%
\pgfsetstrokecolor{currentstroke}%
\pgfsetstrokeopacity{0.453132}%
\pgfsetdash{}{0pt}%
\pgfpathmoveto{\pgfqpoint{1.395062in}{2.490955in}}%
\pgfpathcurveto{\pgfqpoint{1.403298in}{2.490955in}}{\pgfqpoint{1.411198in}{2.494228in}}{\pgfqpoint{1.417022in}{2.500052in}}%
\pgfpathcurveto{\pgfqpoint{1.422846in}{2.505876in}}{\pgfqpoint{1.426118in}{2.513776in}}{\pgfqpoint{1.426118in}{2.522012in}}%
\pgfpathcurveto{\pgfqpoint{1.426118in}{2.530248in}}{\pgfqpoint{1.422846in}{2.538148in}}{\pgfqpoint{1.417022in}{2.543972in}}%
\pgfpathcurveto{\pgfqpoint{1.411198in}{2.549796in}}{\pgfqpoint{1.403298in}{2.553068in}}{\pgfqpoint{1.395062in}{2.553068in}}%
\pgfpathcurveto{\pgfqpoint{1.386826in}{2.553068in}}{\pgfqpoint{1.378926in}{2.549796in}}{\pgfqpoint{1.373102in}{2.543972in}}%
\pgfpathcurveto{\pgfqpoint{1.367278in}{2.538148in}}{\pgfqpoint{1.364005in}{2.530248in}}{\pgfqpoint{1.364005in}{2.522012in}}%
\pgfpathcurveto{\pgfqpoint{1.364005in}{2.513776in}}{\pgfqpoint{1.367278in}{2.505876in}}{\pgfqpoint{1.373102in}{2.500052in}}%
\pgfpathcurveto{\pgfqpoint{1.378926in}{2.494228in}}{\pgfqpoint{1.386826in}{2.490955in}}{\pgfqpoint{1.395062in}{2.490955in}}%
\pgfpathclose%
\pgfusepath{stroke,fill}%
\end{pgfscope}%
\begin{pgfscope}%
\pgfpathrectangle{\pgfqpoint{0.100000in}{0.212622in}}{\pgfqpoint{3.696000in}{3.696000in}}%
\pgfusepath{clip}%
\pgfsetbuttcap%
\pgfsetroundjoin%
\definecolor{currentfill}{rgb}{0.121569,0.466667,0.705882}%
\pgfsetfillcolor{currentfill}%
\pgfsetfillopacity{0.454883}%
\pgfsetlinewidth{1.003750pt}%
\definecolor{currentstroke}{rgb}{0.121569,0.466667,0.705882}%
\pgfsetstrokecolor{currentstroke}%
\pgfsetstrokeopacity{0.454883}%
\pgfsetdash{}{0pt}%
\pgfpathmoveto{\pgfqpoint{1.389662in}{2.479909in}}%
\pgfpathcurveto{\pgfqpoint{1.397898in}{2.479909in}}{\pgfqpoint{1.405798in}{2.483181in}}{\pgfqpoint{1.411622in}{2.489005in}}%
\pgfpathcurveto{\pgfqpoint{1.417446in}{2.494829in}}{\pgfqpoint{1.420718in}{2.502729in}}{\pgfqpoint{1.420718in}{2.510965in}}%
\pgfpathcurveto{\pgfqpoint{1.420718in}{2.519202in}}{\pgfqpoint{1.417446in}{2.527102in}}{\pgfqpoint{1.411622in}{2.532926in}}%
\pgfpathcurveto{\pgfqpoint{1.405798in}{2.538749in}}{\pgfqpoint{1.397898in}{2.542022in}}{\pgfqpoint{1.389662in}{2.542022in}}%
\pgfpathcurveto{\pgfqpoint{1.381425in}{2.542022in}}{\pgfqpoint{1.373525in}{2.538749in}}{\pgfqpoint{1.367701in}{2.532926in}}%
\pgfpathcurveto{\pgfqpoint{1.361877in}{2.527102in}}{\pgfqpoint{1.358605in}{2.519202in}}{\pgfqpoint{1.358605in}{2.510965in}}%
\pgfpathcurveto{\pgfqpoint{1.358605in}{2.502729in}}{\pgfqpoint{1.361877in}{2.494829in}}{\pgfqpoint{1.367701in}{2.489005in}}%
\pgfpathcurveto{\pgfqpoint{1.373525in}{2.483181in}}{\pgfqpoint{1.381425in}{2.479909in}}{\pgfqpoint{1.389662in}{2.479909in}}%
\pgfpathclose%
\pgfusepath{stroke,fill}%
\end{pgfscope}%
\begin{pgfscope}%
\pgfpathrectangle{\pgfqpoint{0.100000in}{0.212622in}}{\pgfqpoint{3.696000in}{3.696000in}}%
\pgfusepath{clip}%
\pgfsetbuttcap%
\pgfsetroundjoin%
\definecolor{currentfill}{rgb}{0.121569,0.466667,0.705882}%
\pgfsetfillcolor{currentfill}%
\pgfsetfillopacity{0.454981}%
\pgfsetlinewidth{1.003750pt}%
\definecolor{currentstroke}{rgb}{0.121569,0.466667,0.705882}%
\pgfsetstrokecolor{currentstroke}%
\pgfsetstrokeopacity{0.454981}%
\pgfsetdash{}{0pt}%
\pgfpathmoveto{\pgfqpoint{2.012613in}{2.697344in}}%
\pgfpathcurveto{\pgfqpoint{2.020849in}{2.697344in}}{\pgfqpoint{2.028749in}{2.700616in}}{\pgfqpoint{2.034573in}{2.706440in}}%
\pgfpathcurveto{\pgfqpoint{2.040397in}{2.712264in}}{\pgfqpoint{2.043669in}{2.720164in}}{\pgfqpoint{2.043669in}{2.728400in}}%
\pgfpathcurveto{\pgfqpoint{2.043669in}{2.736637in}}{\pgfqpoint{2.040397in}{2.744537in}}{\pgfqpoint{2.034573in}{2.750361in}}%
\pgfpathcurveto{\pgfqpoint{2.028749in}{2.756184in}}{\pgfqpoint{2.020849in}{2.759457in}}{\pgfqpoint{2.012613in}{2.759457in}}%
\pgfpathcurveto{\pgfqpoint{2.004376in}{2.759457in}}{\pgfqpoint{1.996476in}{2.756184in}}{\pgfqpoint{1.990652in}{2.750361in}}%
\pgfpathcurveto{\pgfqpoint{1.984828in}{2.744537in}}{\pgfqpoint{1.981556in}{2.736637in}}{\pgfqpoint{1.981556in}{2.728400in}}%
\pgfpathcurveto{\pgfqpoint{1.981556in}{2.720164in}}{\pgfqpoint{1.984828in}{2.712264in}}{\pgfqpoint{1.990652in}{2.706440in}}%
\pgfpathcurveto{\pgfqpoint{1.996476in}{2.700616in}}{\pgfqpoint{2.004376in}{2.697344in}}{\pgfqpoint{2.012613in}{2.697344in}}%
\pgfpathclose%
\pgfusepath{stroke,fill}%
\end{pgfscope}%
\begin{pgfscope}%
\pgfpathrectangle{\pgfqpoint{0.100000in}{0.212622in}}{\pgfqpoint{3.696000in}{3.696000in}}%
\pgfusepath{clip}%
\pgfsetbuttcap%
\pgfsetroundjoin%
\definecolor{currentfill}{rgb}{0.121569,0.466667,0.705882}%
\pgfsetfillcolor{currentfill}%
\pgfsetfillopacity{0.456134}%
\pgfsetlinewidth{1.003750pt}%
\definecolor{currentstroke}{rgb}{0.121569,0.466667,0.705882}%
\pgfsetstrokecolor{currentstroke}%
\pgfsetstrokeopacity{0.456134}%
\pgfsetdash{}{0pt}%
\pgfpathmoveto{\pgfqpoint{1.384397in}{2.470827in}}%
\pgfpathcurveto{\pgfqpoint{1.392634in}{2.470827in}}{\pgfqpoint{1.400534in}{2.474099in}}{\pgfqpoint{1.406358in}{2.479923in}}%
\pgfpathcurveto{\pgfqpoint{1.412182in}{2.485747in}}{\pgfqpoint{1.415454in}{2.493647in}}{\pgfqpoint{1.415454in}{2.501884in}}%
\pgfpathcurveto{\pgfqpoint{1.415454in}{2.510120in}}{\pgfqpoint{1.412182in}{2.518020in}}{\pgfqpoint{1.406358in}{2.523844in}}%
\pgfpathcurveto{\pgfqpoint{1.400534in}{2.529668in}}{\pgfqpoint{1.392634in}{2.532940in}}{\pgfqpoint{1.384397in}{2.532940in}}%
\pgfpathcurveto{\pgfqpoint{1.376161in}{2.532940in}}{\pgfqpoint{1.368261in}{2.529668in}}{\pgfqpoint{1.362437in}{2.523844in}}%
\pgfpathcurveto{\pgfqpoint{1.356613in}{2.518020in}}{\pgfqpoint{1.353341in}{2.510120in}}{\pgfqpoint{1.353341in}{2.501884in}}%
\pgfpathcurveto{\pgfqpoint{1.353341in}{2.493647in}}{\pgfqpoint{1.356613in}{2.485747in}}{\pgfqpoint{1.362437in}{2.479923in}}%
\pgfpathcurveto{\pgfqpoint{1.368261in}{2.474099in}}{\pgfqpoint{1.376161in}{2.470827in}}{\pgfqpoint{1.384397in}{2.470827in}}%
\pgfpathclose%
\pgfusepath{stroke,fill}%
\end{pgfscope}%
\begin{pgfscope}%
\pgfpathrectangle{\pgfqpoint{0.100000in}{0.212622in}}{\pgfqpoint{3.696000in}{3.696000in}}%
\pgfusepath{clip}%
\pgfsetbuttcap%
\pgfsetroundjoin%
\definecolor{currentfill}{rgb}{0.121569,0.466667,0.705882}%
\pgfsetfillcolor{currentfill}%
\pgfsetfillopacity{0.457237}%
\pgfsetlinewidth{1.003750pt}%
\definecolor{currentstroke}{rgb}{0.121569,0.466667,0.705882}%
\pgfsetstrokecolor{currentstroke}%
\pgfsetstrokeopacity{0.457237}%
\pgfsetdash{}{0pt}%
\pgfpathmoveto{\pgfqpoint{1.380868in}{2.462880in}}%
\pgfpathcurveto{\pgfqpoint{1.389105in}{2.462880in}}{\pgfqpoint{1.397005in}{2.466152in}}{\pgfqpoint{1.402829in}{2.471976in}}%
\pgfpathcurveto{\pgfqpoint{1.408653in}{2.477800in}}{\pgfqpoint{1.411925in}{2.485700in}}{\pgfqpoint{1.411925in}{2.493936in}}%
\pgfpathcurveto{\pgfqpoint{1.411925in}{2.502173in}}{\pgfqpoint{1.408653in}{2.510073in}}{\pgfqpoint{1.402829in}{2.515897in}}%
\pgfpathcurveto{\pgfqpoint{1.397005in}{2.521721in}}{\pgfqpoint{1.389105in}{2.524993in}}{\pgfqpoint{1.380868in}{2.524993in}}%
\pgfpathcurveto{\pgfqpoint{1.372632in}{2.524993in}}{\pgfqpoint{1.364732in}{2.521721in}}{\pgfqpoint{1.358908in}{2.515897in}}%
\pgfpathcurveto{\pgfqpoint{1.353084in}{2.510073in}}{\pgfqpoint{1.349812in}{2.502173in}}{\pgfqpoint{1.349812in}{2.493936in}}%
\pgfpathcurveto{\pgfqpoint{1.349812in}{2.485700in}}{\pgfqpoint{1.353084in}{2.477800in}}{\pgfqpoint{1.358908in}{2.471976in}}%
\pgfpathcurveto{\pgfqpoint{1.364732in}{2.466152in}}{\pgfqpoint{1.372632in}{2.462880in}}{\pgfqpoint{1.380868in}{2.462880in}}%
\pgfpathclose%
\pgfusepath{stroke,fill}%
\end{pgfscope}%
\begin{pgfscope}%
\pgfpathrectangle{\pgfqpoint{0.100000in}{0.212622in}}{\pgfqpoint{3.696000in}{3.696000in}}%
\pgfusepath{clip}%
\pgfsetbuttcap%
\pgfsetroundjoin%
\definecolor{currentfill}{rgb}{0.121569,0.466667,0.705882}%
\pgfsetfillcolor{currentfill}%
\pgfsetfillopacity{0.457993}%
\pgfsetlinewidth{1.003750pt}%
\definecolor{currentstroke}{rgb}{0.121569,0.466667,0.705882}%
\pgfsetstrokecolor{currentstroke}%
\pgfsetstrokeopacity{0.457993}%
\pgfsetdash{}{0pt}%
\pgfpathmoveto{\pgfqpoint{1.378041in}{2.458099in}}%
\pgfpathcurveto{\pgfqpoint{1.386278in}{2.458099in}}{\pgfqpoint{1.394178in}{2.461372in}}{\pgfqpoint{1.400002in}{2.467196in}}%
\pgfpathcurveto{\pgfqpoint{1.405826in}{2.473020in}}{\pgfqpoint{1.409098in}{2.480920in}}{\pgfqpoint{1.409098in}{2.489156in}}%
\pgfpathcurveto{\pgfqpoint{1.409098in}{2.497392in}}{\pgfqpoint{1.405826in}{2.505292in}}{\pgfqpoint{1.400002in}{2.511116in}}%
\pgfpathcurveto{\pgfqpoint{1.394178in}{2.516940in}}{\pgfqpoint{1.386278in}{2.520212in}}{\pgfqpoint{1.378041in}{2.520212in}}%
\pgfpathcurveto{\pgfqpoint{1.369805in}{2.520212in}}{\pgfqpoint{1.361905in}{2.516940in}}{\pgfqpoint{1.356081in}{2.511116in}}%
\pgfpathcurveto{\pgfqpoint{1.350257in}{2.505292in}}{\pgfqpoint{1.346985in}{2.497392in}}{\pgfqpoint{1.346985in}{2.489156in}}%
\pgfpathcurveto{\pgfqpoint{1.346985in}{2.480920in}}{\pgfqpoint{1.350257in}{2.473020in}}{\pgfqpoint{1.356081in}{2.467196in}}%
\pgfpathcurveto{\pgfqpoint{1.361905in}{2.461372in}}{\pgfqpoint{1.369805in}{2.458099in}}{\pgfqpoint{1.378041in}{2.458099in}}%
\pgfpathclose%
\pgfusepath{stroke,fill}%
\end{pgfscope}%
\begin{pgfscope}%
\pgfpathrectangle{\pgfqpoint{0.100000in}{0.212622in}}{\pgfqpoint{3.696000in}{3.696000in}}%
\pgfusepath{clip}%
\pgfsetbuttcap%
\pgfsetroundjoin%
\definecolor{currentfill}{rgb}{0.121569,0.466667,0.705882}%
\pgfsetfillcolor{currentfill}%
\pgfsetfillopacity{0.458729}%
\pgfsetlinewidth{1.003750pt}%
\definecolor{currentstroke}{rgb}{0.121569,0.466667,0.705882}%
\pgfsetstrokecolor{currentstroke}%
\pgfsetstrokeopacity{0.458729}%
\pgfsetdash{}{0pt}%
\pgfpathmoveto{\pgfqpoint{1.375728in}{2.453469in}}%
\pgfpathcurveto{\pgfqpoint{1.383964in}{2.453469in}}{\pgfqpoint{1.391864in}{2.456741in}}{\pgfqpoint{1.397688in}{2.462565in}}%
\pgfpathcurveto{\pgfqpoint{1.403512in}{2.468389in}}{\pgfqpoint{1.406784in}{2.476289in}}{\pgfqpoint{1.406784in}{2.484525in}}%
\pgfpathcurveto{\pgfqpoint{1.406784in}{2.492761in}}{\pgfqpoint{1.403512in}{2.500661in}}{\pgfqpoint{1.397688in}{2.506485in}}%
\pgfpathcurveto{\pgfqpoint{1.391864in}{2.512309in}}{\pgfqpoint{1.383964in}{2.515582in}}{\pgfqpoint{1.375728in}{2.515582in}}%
\pgfpathcurveto{\pgfqpoint{1.367492in}{2.515582in}}{\pgfqpoint{1.359592in}{2.512309in}}{\pgfqpoint{1.353768in}{2.506485in}}%
\pgfpathcurveto{\pgfqpoint{1.347944in}{2.500661in}}{\pgfqpoint{1.344671in}{2.492761in}}{\pgfqpoint{1.344671in}{2.484525in}}%
\pgfpathcurveto{\pgfqpoint{1.344671in}{2.476289in}}{\pgfqpoint{1.347944in}{2.468389in}}{\pgfqpoint{1.353768in}{2.462565in}}%
\pgfpathcurveto{\pgfqpoint{1.359592in}{2.456741in}}{\pgfqpoint{1.367492in}{2.453469in}}{\pgfqpoint{1.375728in}{2.453469in}}%
\pgfpathclose%
\pgfusepath{stroke,fill}%
\end{pgfscope}%
\begin{pgfscope}%
\pgfpathrectangle{\pgfqpoint{0.100000in}{0.212622in}}{\pgfqpoint{3.696000in}{3.696000in}}%
\pgfusepath{clip}%
\pgfsetbuttcap%
\pgfsetroundjoin%
\definecolor{currentfill}{rgb}{0.121569,0.466667,0.705882}%
\pgfsetfillcolor{currentfill}%
\pgfsetfillopacity{0.459326}%
\pgfsetlinewidth{1.003750pt}%
\definecolor{currentstroke}{rgb}{0.121569,0.466667,0.705882}%
\pgfsetstrokecolor{currentstroke}%
\pgfsetstrokeopacity{0.459326}%
\pgfsetdash{}{0pt}%
\pgfpathmoveto{\pgfqpoint{2.014448in}{2.684268in}}%
\pgfpathcurveto{\pgfqpoint{2.022684in}{2.684268in}}{\pgfqpoint{2.030584in}{2.687540in}}{\pgfqpoint{2.036408in}{2.693364in}}%
\pgfpathcurveto{\pgfqpoint{2.042232in}{2.699188in}}{\pgfqpoint{2.045504in}{2.707088in}}{\pgfqpoint{2.045504in}{2.715324in}}%
\pgfpathcurveto{\pgfqpoint{2.045504in}{2.723560in}}{\pgfqpoint{2.042232in}{2.731461in}}{\pgfqpoint{2.036408in}{2.737284in}}%
\pgfpathcurveto{\pgfqpoint{2.030584in}{2.743108in}}{\pgfqpoint{2.022684in}{2.746381in}}{\pgfqpoint{2.014448in}{2.746381in}}%
\pgfpathcurveto{\pgfqpoint{2.006211in}{2.746381in}}{\pgfqpoint{1.998311in}{2.743108in}}{\pgfqpoint{1.992487in}{2.737284in}}%
\pgfpathcurveto{\pgfqpoint{1.986664in}{2.731461in}}{\pgfqpoint{1.983391in}{2.723560in}}{\pgfqpoint{1.983391in}{2.715324in}}%
\pgfpathcurveto{\pgfqpoint{1.983391in}{2.707088in}}{\pgfqpoint{1.986664in}{2.699188in}}{\pgfqpoint{1.992487in}{2.693364in}}%
\pgfpathcurveto{\pgfqpoint{1.998311in}{2.687540in}}{\pgfqpoint{2.006211in}{2.684268in}}{\pgfqpoint{2.014448in}{2.684268in}}%
\pgfpathclose%
\pgfusepath{stroke,fill}%
\end{pgfscope}%
\begin{pgfscope}%
\pgfpathrectangle{\pgfqpoint{0.100000in}{0.212622in}}{\pgfqpoint{3.696000in}{3.696000in}}%
\pgfusepath{clip}%
\pgfsetbuttcap%
\pgfsetroundjoin%
\definecolor{currentfill}{rgb}{0.121569,0.466667,0.705882}%
\pgfsetfillcolor{currentfill}%
\pgfsetfillopacity{0.459425}%
\pgfsetlinewidth{1.003750pt}%
\definecolor{currentstroke}{rgb}{0.121569,0.466667,0.705882}%
\pgfsetstrokecolor{currentstroke}%
\pgfsetstrokeopacity{0.459425}%
\pgfsetdash{}{0pt}%
\pgfpathmoveto{\pgfqpoint{1.373447in}{2.449554in}}%
\pgfpathcurveto{\pgfqpoint{1.381683in}{2.449554in}}{\pgfqpoint{1.389584in}{2.452826in}}{\pgfqpoint{1.395407in}{2.458650in}}%
\pgfpathcurveto{\pgfqpoint{1.401231in}{2.464474in}}{\pgfqpoint{1.404504in}{2.472374in}}{\pgfqpoint{1.404504in}{2.480611in}}%
\pgfpathcurveto{\pgfqpoint{1.404504in}{2.488847in}}{\pgfqpoint{1.401231in}{2.496747in}}{\pgfqpoint{1.395407in}{2.502571in}}%
\pgfpathcurveto{\pgfqpoint{1.389584in}{2.508395in}}{\pgfqpoint{1.381683in}{2.511667in}}{\pgfqpoint{1.373447in}{2.511667in}}%
\pgfpathcurveto{\pgfqpoint{1.365211in}{2.511667in}}{\pgfqpoint{1.357311in}{2.508395in}}{\pgfqpoint{1.351487in}{2.502571in}}%
\pgfpathcurveto{\pgfqpoint{1.345663in}{2.496747in}}{\pgfqpoint{1.342391in}{2.488847in}}{\pgfqpoint{1.342391in}{2.480611in}}%
\pgfpathcurveto{\pgfqpoint{1.342391in}{2.472374in}}{\pgfqpoint{1.345663in}{2.464474in}}{\pgfqpoint{1.351487in}{2.458650in}}%
\pgfpathcurveto{\pgfqpoint{1.357311in}{2.452826in}}{\pgfqpoint{1.365211in}{2.449554in}}{\pgfqpoint{1.373447in}{2.449554in}}%
\pgfpathclose%
\pgfusepath{stroke,fill}%
\end{pgfscope}%
\begin{pgfscope}%
\pgfpathrectangle{\pgfqpoint{0.100000in}{0.212622in}}{\pgfqpoint{3.696000in}{3.696000in}}%
\pgfusepath{clip}%
\pgfsetbuttcap%
\pgfsetroundjoin%
\definecolor{currentfill}{rgb}{0.121569,0.466667,0.705882}%
\pgfsetfillcolor{currentfill}%
\pgfsetfillopacity{0.460607}%
\pgfsetlinewidth{1.003750pt}%
\definecolor{currentstroke}{rgb}{0.121569,0.466667,0.705882}%
\pgfsetstrokecolor{currentstroke}%
\pgfsetstrokeopacity{0.460607}%
\pgfsetdash{}{0pt}%
\pgfpathmoveto{\pgfqpoint{1.369259in}{2.442175in}}%
\pgfpathcurveto{\pgfqpoint{1.377495in}{2.442175in}}{\pgfqpoint{1.385395in}{2.445447in}}{\pgfqpoint{1.391219in}{2.451271in}}%
\pgfpathcurveto{\pgfqpoint{1.397043in}{2.457095in}}{\pgfqpoint{1.400315in}{2.464995in}}{\pgfqpoint{1.400315in}{2.473231in}}%
\pgfpathcurveto{\pgfqpoint{1.400315in}{2.481468in}}{\pgfqpoint{1.397043in}{2.489368in}}{\pgfqpoint{1.391219in}{2.495192in}}%
\pgfpathcurveto{\pgfqpoint{1.385395in}{2.501015in}}{\pgfqpoint{1.377495in}{2.504288in}}{\pgfqpoint{1.369259in}{2.504288in}}%
\pgfpathcurveto{\pgfqpoint{1.361023in}{2.504288in}}{\pgfqpoint{1.353123in}{2.501015in}}{\pgfqpoint{1.347299in}{2.495192in}}%
\pgfpathcurveto{\pgfqpoint{1.341475in}{2.489368in}}{\pgfqpoint{1.338202in}{2.481468in}}{\pgfqpoint{1.338202in}{2.473231in}}%
\pgfpathcurveto{\pgfqpoint{1.338202in}{2.464995in}}{\pgfqpoint{1.341475in}{2.457095in}}{\pgfqpoint{1.347299in}{2.451271in}}%
\pgfpathcurveto{\pgfqpoint{1.353123in}{2.445447in}}{\pgfqpoint{1.361023in}{2.442175in}}{\pgfqpoint{1.369259in}{2.442175in}}%
\pgfpathclose%
\pgfusepath{stroke,fill}%
\end{pgfscope}%
\begin{pgfscope}%
\pgfpathrectangle{\pgfqpoint{0.100000in}{0.212622in}}{\pgfqpoint{3.696000in}{3.696000in}}%
\pgfusepath{clip}%
\pgfsetbuttcap%
\pgfsetroundjoin%
\definecolor{currentfill}{rgb}{0.121569,0.466667,0.705882}%
\pgfsetfillcolor{currentfill}%
\pgfsetfillopacity{0.461643}%
\pgfsetlinewidth{1.003750pt}%
\definecolor{currentstroke}{rgb}{0.121569,0.466667,0.705882}%
\pgfsetstrokecolor{currentstroke}%
\pgfsetstrokeopacity{0.461643}%
\pgfsetdash{}{0pt}%
\pgfpathmoveto{\pgfqpoint{1.366097in}{2.435351in}}%
\pgfpathcurveto{\pgfqpoint{1.374333in}{2.435351in}}{\pgfqpoint{1.382233in}{2.438623in}}{\pgfqpoint{1.388057in}{2.444447in}}%
\pgfpathcurveto{\pgfqpoint{1.393881in}{2.450271in}}{\pgfqpoint{1.397153in}{2.458171in}}{\pgfqpoint{1.397153in}{2.466407in}}%
\pgfpathcurveto{\pgfqpoint{1.397153in}{2.474643in}}{\pgfqpoint{1.393881in}{2.482544in}}{\pgfqpoint{1.388057in}{2.488367in}}%
\pgfpathcurveto{\pgfqpoint{1.382233in}{2.494191in}}{\pgfqpoint{1.374333in}{2.497464in}}{\pgfqpoint{1.366097in}{2.497464in}}%
\pgfpathcurveto{\pgfqpoint{1.357860in}{2.497464in}}{\pgfqpoint{1.349960in}{2.494191in}}{\pgfqpoint{1.344136in}{2.488367in}}%
\pgfpathcurveto{\pgfqpoint{1.338312in}{2.482544in}}{\pgfqpoint{1.335040in}{2.474643in}}{\pgfqpoint{1.335040in}{2.466407in}}%
\pgfpathcurveto{\pgfqpoint{1.335040in}{2.458171in}}{\pgfqpoint{1.338312in}{2.450271in}}{\pgfqpoint{1.344136in}{2.444447in}}%
\pgfpathcurveto{\pgfqpoint{1.349960in}{2.438623in}}{\pgfqpoint{1.357860in}{2.435351in}}{\pgfqpoint{1.366097in}{2.435351in}}%
\pgfpathclose%
\pgfusepath{stroke,fill}%
\end{pgfscope}%
\begin{pgfscope}%
\pgfpathrectangle{\pgfqpoint{0.100000in}{0.212622in}}{\pgfqpoint{3.696000in}{3.696000in}}%
\pgfusepath{clip}%
\pgfsetbuttcap%
\pgfsetroundjoin%
\definecolor{currentfill}{rgb}{0.121569,0.466667,0.705882}%
\pgfsetfillcolor{currentfill}%
\pgfsetfillopacity{0.462373}%
\pgfsetlinewidth{1.003750pt}%
\definecolor{currentstroke}{rgb}{0.121569,0.466667,0.705882}%
\pgfsetstrokecolor{currentstroke}%
\pgfsetstrokeopacity{0.462373}%
\pgfsetdash{}{0pt}%
\pgfpathmoveto{\pgfqpoint{1.363173in}{2.430006in}}%
\pgfpathcurveto{\pgfqpoint{1.371410in}{2.430006in}}{\pgfqpoint{1.379310in}{2.433278in}}{\pgfqpoint{1.385134in}{2.439102in}}%
\pgfpathcurveto{\pgfqpoint{1.390958in}{2.444926in}}{\pgfqpoint{1.394230in}{2.452826in}}{\pgfqpoint{1.394230in}{2.461062in}}%
\pgfpathcurveto{\pgfqpoint{1.394230in}{2.469299in}}{\pgfqpoint{1.390958in}{2.477199in}}{\pgfqpoint{1.385134in}{2.483023in}}%
\pgfpathcurveto{\pgfqpoint{1.379310in}{2.488847in}}{\pgfqpoint{1.371410in}{2.492119in}}{\pgfqpoint{1.363173in}{2.492119in}}%
\pgfpathcurveto{\pgfqpoint{1.354937in}{2.492119in}}{\pgfqpoint{1.347037in}{2.488847in}}{\pgfqpoint{1.341213in}{2.483023in}}%
\pgfpathcurveto{\pgfqpoint{1.335389in}{2.477199in}}{\pgfqpoint{1.332117in}{2.469299in}}{\pgfqpoint{1.332117in}{2.461062in}}%
\pgfpathcurveto{\pgfqpoint{1.332117in}{2.452826in}}{\pgfqpoint{1.335389in}{2.444926in}}{\pgfqpoint{1.341213in}{2.439102in}}%
\pgfpathcurveto{\pgfqpoint{1.347037in}{2.433278in}}{\pgfqpoint{1.354937in}{2.430006in}}{\pgfqpoint{1.363173in}{2.430006in}}%
\pgfpathclose%
\pgfusepath{stroke,fill}%
\end{pgfscope}%
\begin{pgfscope}%
\pgfpathrectangle{\pgfqpoint{0.100000in}{0.212622in}}{\pgfqpoint{3.696000in}{3.696000in}}%
\pgfusepath{clip}%
\pgfsetbuttcap%
\pgfsetroundjoin%
\definecolor{currentfill}{rgb}{0.121569,0.466667,0.705882}%
\pgfsetfillcolor{currentfill}%
\pgfsetfillopacity{0.462829}%
\pgfsetlinewidth{1.003750pt}%
\definecolor{currentstroke}{rgb}{0.121569,0.466667,0.705882}%
\pgfsetstrokecolor{currentstroke}%
\pgfsetstrokeopacity{0.462829}%
\pgfsetdash{}{0pt}%
\pgfpathmoveto{\pgfqpoint{1.361882in}{2.426768in}}%
\pgfpathcurveto{\pgfqpoint{1.370119in}{2.426768in}}{\pgfqpoint{1.378019in}{2.430040in}}{\pgfqpoint{1.383843in}{2.435864in}}%
\pgfpathcurveto{\pgfqpoint{1.389667in}{2.441688in}}{\pgfqpoint{1.392939in}{2.449588in}}{\pgfqpoint{1.392939in}{2.457824in}}%
\pgfpathcurveto{\pgfqpoint{1.392939in}{2.466060in}}{\pgfqpoint{1.389667in}{2.473960in}}{\pgfqpoint{1.383843in}{2.479784in}}%
\pgfpathcurveto{\pgfqpoint{1.378019in}{2.485608in}}{\pgfqpoint{1.370119in}{2.488881in}}{\pgfqpoint{1.361882in}{2.488881in}}%
\pgfpathcurveto{\pgfqpoint{1.353646in}{2.488881in}}{\pgfqpoint{1.345746in}{2.485608in}}{\pgfqpoint{1.339922in}{2.479784in}}%
\pgfpathcurveto{\pgfqpoint{1.334098in}{2.473960in}}{\pgfqpoint{1.330826in}{2.466060in}}{\pgfqpoint{1.330826in}{2.457824in}}%
\pgfpathcurveto{\pgfqpoint{1.330826in}{2.449588in}}{\pgfqpoint{1.334098in}{2.441688in}}{\pgfqpoint{1.339922in}{2.435864in}}%
\pgfpathcurveto{\pgfqpoint{1.345746in}{2.430040in}}{\pgfqpoint{1.353646in}{2.426768in}}{\pgfqpoint{1.361882in}{2.426768in}}%
\pgfpathclose%
\pgfusepath{stroke,fill}%
\end{pgfscope}%
\begin{pgfscope}%
\pgfpathrectangle{\pgfqpoint{0.100000in}{0.212622in}}{\pgfqpoint{3.696000in}{3.696000in}}%
\pgfusepath{clip}%
\pgfsetbuttcap%
\pgfsetroundjoin%
\definecolor{currentfill}{rgb}{0.121569,0.466667,0.705882}%
\pgfsetfillcolor{currentfill}%
\pgfsetfillopacity{0.463077}%
\pgfsetlinewidth{1.003750pt}%
\definecolor{currentstroke}{rgb}{0.121569,0.466667,0.705882}%
\pgfsetstrokecolor{currentstroke}%
\pgfsetstrokeopacity{0.463077}%
\pgfsetdash{}{0pt}%
\pgfpathmoveto{\pgfqpoint{2.017840in}{2.668458in}}%
\pgfpathcurveto{\pgfqpoint{2.026076in}{2.668458in}}{\pgfqpoint{2.033976in}{2.671730in}}{\pgfqpoint{2.039800in}{2.677554in}}%
\pgfpathcurveto{\pgfqpoint{2.045624in}{2.683378in}}{\pgfqpoint{2.048897in}{2.691278in}}{\pgfqpoint{2.048897in}{2.699515in}}%
\pgfpathcurveto{\pgfqpoint{2.048897in}{2.707751in}}{\pgfqpoint{2.045624in}{2.715651in}}{\pgfqpoint{2.039800in}{2.721475in}}%
\pgfpathcurveto{\pgfqpoint{2.033976in}{2.727299in}}{\pgfqpoint{2.026076in}{2.730571in}}{\pgfqpoint{2.017840in}{2.730571in}}%
\pgfpathcurveto{\pgfqpoint{2.009604in}{2.730571in}}{\pgfqpoint{2.001704in}{2.727299in}}{\pgfqpoint{1.995880in}{2.721475in}}%
\pgfpathcurveto{\pgfqpoint{1.990056in}{2.715651in}}{\pgfqpoint{1.986784in}{2.707751in}}{\pgfqpoint{1.986784in}{2.699515in}}%
\pgfpathcurveto{\pgfqpoint{1.986784in}{2.691278in}}{\pgfqpoint{1.990056in}{2.683378in}}{\pgfqpoint{1.995880in}{2.677554in}}%
\pgfpathcurveto{\pgfqpoint{2.001704in}{2.671730in}}{\pgfqpoint{2.009604in}{2.668458in}}{\pgfqpoint{2.017840in}{2.668458in}}%
\pgfpathclose%
\pgfusepath{stroke,fill}%
\end{pgfscope}%
\begin{pgfscope}%
\pgfpathrectangle{\pgfqpoint{0.100000in}{0.212622in}}{\pgfqpoint{3.696000in}{3.696000in}}%
\pgfusepath{clip}%
\pgfsetbuttcap%
\pgfsetroundjoin%
\definecolor{currentfill}{rgb}{0.121569,0.466667,0.705882}%
\pgfsetfillcolor{currentfill}%
\pgfsetfillopacity{0.463200}%
\pgfsetlinewidth{1.003750pt}%
\definecolor{currentstroke}{rgb}{0.121569,0.466667,0.705882}%
\pgfsetstrokecolor{currentstroke}%
\pgfsetstrokeopacity{0.463200}%
\pgfsetdash{}{0pt}%
\pgfpathmoveto{\pgfqpoint{1.360744in}{2.424864in}}%
\pgfpathcurveto{\pgfqpoint{1.368980in}{2.424864in}}{\pgfqpoint{1.376880in}{2.428136in}}{\pgfqpoint{1.382704in}{2.433960in}}%
\pgfpathcurveto{\pgfqpoint{1.388528in}{2.439784in}}{\pgfqpoint{1.391800in}{2.447684in}}{\pgfqpoint{1.391800in}{2.455920in}}%
\pgfpathcurveto{\pgfqpoint{1.391800in}{2.464156in}}{\pgfqpoint{1.388528in}{2.472057in}}{\pgfqpoint{1.382704in}{2.477880in}}%
\pgfpathcurveto{\pgfqpoint{1.376880in}{2.483704in}}{\pgfqpoint{1.368980in}{2.486977in}}{\pgfqpoint{1.360744in}{2.486977in}}%
\pgfpathcurveto{\pgfqpoint{1.352508in}{2.486977in}}{\pgfqpoint{1.344608in}{2.483704in}}{\pgfqpoint{1.338784in}{2.477880in}}%
\pgfpathcurveto{\pgfqpoint{1.332960in}{2.472057in}}{\pgfqpoint{1.329687in}{2.464156in}}{\pgfqpoint{1.329687in}{2.455920in}}%
\pgfpathcurveto{\pgfqpoint{1.329687in}{2.447684in}}{\pgfqpoint{1.332960in}{2.439784in}}{\pgfqpoint{1.338784in}{2.433960in}}%
\pgfpathcurveto{\pgfqpoint{1.344608in}{2.428136in}}{\pgfqpoint{1.352508in}{2.424864in}}{\pgfqpoint{1.360744in}{2.424864in}}%
\pgfpathclose%
\pgfusepath{stroke,fill}%
\end{pgfscope}%
\begin{pgfscope}%
\pgfpathrectangle{\pgfqpoint{0.100000in}{0.212622in}}{\pgfqpoint{3.696000in}{3.696000in}}%
\pgfusepath{clip}%
\pgfsetbuttcap%
\pgfsetroundjoin%
\definecolor{currentfill}{rgb}{0.121569,0.466667,0.705882}%
\pgfsetfillcolor{currentfill}%
\pgfsetfillopacity{0.463913}%
\pgfsetlinewidth{1.003750pt}%
\definecolor{currentstroke}{rgb}{0.121569,0.466667,0.705882}%
\pgfsetstrokecolor{currentstroke}%
\pgfsetstrokeopacity{0.463913}%
\pgfsetdash{}{0pt}%
\pgfpathmoveto{\pgfqpoint{1.358777in}{2.421410in}}%
\pgfpathcurveto{\pgfqpoint{1.367013in}{2.421410in}}{\pgfqpoint{1.374914in}{2.424683in}}{\pgfqpoint{1.380737in}{2.430507in}}%
\pgfpathcurveto{\pgfqpoint{1.386561in}{2.436331in}}{\pgfqpoint{1.389834in}{2.444231in}}{\pgfqpoint{1.389834in}{2.452467in}}%
\pgfpathcurveto{\pgfqpoint{1.389834in}{2.460703in}}{\pgfqpoint{1.386561in}{2.468603in}}{\pgfqpoint{1.380737in}{2.474427in}}%
\pgfpathcurveto{\pgfqpoint{1.374914in}{2.480251in}}{\pgfqpoint{1.367013in}{2.483523in}}{\pgfqpoint{1.358777in}{2.483523in}}%
\pgfpathcurveto{\pgfqpoint{1.350541in}{2.483523in}}{\pgfqpoint{1.342641in}{2.480251in}}{\pgfqpoint{1.336817in}{2.474427in}}%
\pgfpathcurveto{\pgfqpoint{1.330993in}{2.468603in}}{\pgfqpoint{1.327721in}{2.460703in}}{\pgfqpoint{1.327721in}{2.452467in}}%
\pgfpathcurveto{\pgfqpoint{1.327721in}{2.444231in}}{\pgfqpoint{1.330993in}{2.436331in}}{\pgfqpoint{1.336817in}{2.430507in}}%
\pgfpathcurveto{\pgfqpoint{1.342641in}{2.424683in}}{\pgfqpoint{1.350541in}{2.421410in}}{\pgfqpoint{1.358777in}{2.421410in}}%
\pgfpathclose%
\pgfusepath{stroke,fill}%
\end{pgfscope}%
\begin{pgfscope}%
\pgfpathrectangle{\pgfqpoint{0.100000in}{0.212622in}}{\pgfqpoint{3.696000in}{3.696000in}}%
\pgfusepath{clip}%
\pgfsetbuttcap%
\pgfsetroundjoin%
\definecolor{currentfill}{rgb}{0.121569,0.466667,0.705882}%
\pgfsetfillcolor{currentfill}%
\pgfsetfillopacity{0.464529}%
\pgfsetlinewidth{1.003750pt}%
\definecolor{currentstroke}{rgb}{0.121569,0.466667,0.705882}%
\pgfsetstrokecolor{currentstroke}%
\pgfsetstrokeopacity{0.464529}%
\pgfsetdash{}{0pt}%
\pgfpathmoveto{\pgfqpoint{1.357086in}{2.418483in}}%
\pgfpathcurveto{\pgfqpoint{1.365323in}{2.418483in}}{\pgfqpoint{1.373223in}{2.421755in}}{\pgfqpoint{1.379047in}{2.427579in}}%
\pgfpathcurveto{\pgfqpoint{1.384870in}{2.433403in}}{\pgfqpoint{1.388143in}{2.441303in}}{\pgfqpoint{1.388143in}{2.449540in}}%
\pgfpathcurveto{\pgfqpoint{1.388143in}{2.457776in}}{\pgfqpoint{1.384870in}{2.465676in}}{\pgfqpoint{1.379047in}{2.471500in}}%
\pgfpathcurveto{\pgfqpoint{1.373223in}{2.477324in}}{\pgfqpoint{1.365323in}{2.480596in}}{\pgfqpoint{1.357086in}{2.480596in}}%
\pgfpathcurveto{\pgfqpoint{1.348850in}{2.480596in}}{\pgfqpoint{1.340950in}{2.477324in}}{\pgfqpoint{1.335126in}{2.471500in}}%
\pgfpathcurveto{\pgfqpoint{1.329302in}{2.465676in}}{\pgfqpoint{1.326030in}{2.457776in}}{\pgfqpoint{1.326030in}{2.449540in}}%
\pgfpathcurveto{\pgfqpoint{1.326030in}{2.441303in}}{\pgfqpoint{1.329302in}{2.433403in}}{\pgfqpoint{1.335126in}{2.427579in}}%
\pgfpathcurveto{\pgfqpoint{1.340950in}{2.421755in}}{\pgfqpoint{1.348850in}{2.418483in}}{\pgfqpoint{1.357086in}{2.418483in}}%
\pgfpathclose%
\pgfusepath{stroke,fill}%
\end{pgfscope}%
\begin{pgfscope}%
\pgfpathrectangle{\pgfqpoint{0.100000in}{0.212622in}}{\pgfqpoint{3.696000in}{3.696000in}}%
\pgfusepath{clip}%
\pgfsetbuttcap%
\pgfsetroundjoin%
\definecolor{currentfill}{rgb}{0.121569,0.466667,0.705882}%
\pgfsetfillcolor{currentfill}%
\pgfsetfillopacity{0.465035}%
\pgfsetlinewidth{1.003750pt}%
\definecolor{currentstroke}{rgb}{0.121569,0.466667,0.705882}%
\pgfsetstrokecolor{currentstroke}%
\pgfsetstrokeopacity{0.465035}%
\pgfsetdash{}{0pt}%
\pgfpathmoveto{\pgfqpoint{1.355443in}{2.415833in}}%
\pgfpathcurveto{\pgfqpoint{1.363679in}{2.415833in}}{\pgfqpoint{1.371579in}{2.419106in}}{\pgfqpoint{1.377403in}{2.424930in}}%
\pgfpathcurveto{\pgfqpoint{1.383227in}{2.430754in}}{\pgfqpoint{1.386500in}{2.438654in}}{\pgfqpoint{1.386500in}{2.446890in}}%
\pgfpathcurveto{\pgfqpoint{1.386500in}{2.455126in}}{\pgfqpoint{1.383227in}{2.463026in}}{\pgfqpoint{1.377403in}{2.468850in}}%
\pgfpathcurveto{\pgfqpoint{1.371579in}{2.474674in}}{\pgfqpoint{1.363679in}{2.477946in}}{\pgfqpoint{1.355443in}{2.477946in}}%
\pgfpathcurveto{\pgfqpoint{1.347207in}{2.477946in}}{\pgfqpoint{1.339307in}{2.474674in}}{\pgfqpoint{1.333483in}{2.468850in}}%
\pgfpathcurveto{\pgfqpoint{1.327659in}{2.463026in}}{\pgfqpoint{1.324387in}{2.455126in}}{\pgfqpoint{1.324387in}{2.446890in}}%
\pgfpathcurveto{\pgfqpoint{1.324387in}{2.438654in}}{\pgfqpoint{1.327659in}{2.430754in}}{\pgfqpoint{1.333483in}{2.424930in}}%
\pgfpathcurveto{\pgfqpoint{1.339307in}{2.419106in}}{\pgfqpoint{1.347207in}{2.415833in}}{\pgfqpoint{1.355443in}{2.415833in}}%
\pgfpathclose%
\pgfusepath{stroke,fill}%
\end{pgfscope}%
\begin{pgfscope}%
\pgfpathrectangle{\pgfqpoint{0.100000in}{0.212622in}}{\pgfqpoint{3.696000in}{3.696000in}}%
\pgfusepath{clip}%
\pgfsetbuttcap%
\pgfsetroundjoin%
\definecolor{currentfill}{rgb}{0.121569,0.466667,0.705882}%
\pgfsetfillcolor{currentfill}%
\pgfsetfillopacity{0.465212}%
\pgfsetlinewidth{1.003750pt}%
\definecolor{currentstroke}{rgb}{0.121569,0.466667,0.705882}%
\pgfsetstrokecolor{currentstroke}%
\pgfsetstrokeopacity{0.465212}%
\pgfsetdash{}{0pt}%
\pgfpathmoveto{\pgfqpoint{2.019240in}{2.659588in}}%
\pgfpathcurveto{\pgfqpoint{2.027476in}{2.659588in}}{\pgfqpoint{2.035376in}{2.662860in}}{\pgfqpoint{2.041200in}{2.668684in}}%
\pgfpathcurveto{\pgfqpoint{2.047024in}{2.674508in}}{\pgfqpoint{2.050296in}{2.682408in}}{\pgfqpoint{2.050296in}{2.690644in}}%
\pgfpathcurveto{\pgfqpoint{2.050296in}{2.698880in}}{\pgfqpoint{2.047024in}{2.706781in}}{\pgfqpoint{2.041200in}{2.712604in}}%
\pgfpathcurveto{\pgfqpoint{2.035376in}{2.718428in}}{\pgfqpoint{2.027476in}{2.721701in}}{\pgfqpoint{2.019240in}{2.721701in}}%
\pgfpathcurveto{\pgfqpoint{2.011004in}{2.721701in}}{\pgfqpoint{2.003104in}{2.718428in}}{\pgfqpoint{1.997280in}{2.712604in}}%
\pgfpathcurveto{\pgfqpoint{1.991456in}{2.706781in}}{\pgfqpoint{1.988183in}{2.698880in}}{\pgfqpoint{1.988183in}{2.690644in}}%
\pgfpathcurveto{\pgfqpoint{1.988183in}{2.682408in}}{\pgfqpoint{1.991456in}{2.674508in}}{\pgfqpoint{1.997280in}{2.668684in}}%
\pgfpathcurveto{\pgfqpoint{2.003104in}{2.662860in}}{\pgfqpoint{2.011004in}{2.659588in}}{\pgfqpoint{2.019240in}{2.659588in}}%
\pgfpathclose%
\pgfusepath{stroke,fill}%
\end{pgfscope}%
\begin{pgfscope}%
\pgfpathrectangle{\pgfqpoint{0.100000in}{0.212622in}}{\pgfqpoint{3.696000in}{3.696000in}}%
\pgfusepath{clip}%
\pgfsetbuttcap%
\pgfsetroundjoin%
\definecolor{currentfill}{rgb}{0.121569,0.466667,0.705882}%
\pgfsetfillcolor{currentfill}%
\pgfsetfillopacity{0.466006}%
\pgfsetlinewidth{1.003750pt}%
\definecolor{currentstroke}{rgb}{0.121569,0.466667,0.705882}%
\pgfsetstrokecolor{currentstroke}%
\pgfsetstrokeopacity{0.466006}%
\pgfsetdash{}{0pt}%
\pgfpathmoveto{\pgfqpoint{1.352902in}{2.410625in}}%
\pgfpathcurveto{\pgfqpoint{1.361139in}{2.410625in}}{\pgfqpoint{1.369039in}{2.413897in}}{\pgfqpoint{1.374863in}{2.419721in}}%
\pgfpathcurveto{\pgfqpoint{1.380686in}{2.425545in}}{\pgfqpoint{1.383959in}{2.433445in}}{\pgfqpoint{1.383959in}{2.441681in}}%
\pgfpathcurveto{\pgfqpoint{1.383959in}{2.449917in}}{\pgfqpoint{1.380686in}{2.457817in}}{\pgfqpoint{1.374863in}{2.463641in}}%
\pgfpathcurveto{\pgfqpoint{1.369039in}{2.469465in}}{\pgfqpoint{1.361139in}{2.472738in}}{\pgfqpoint{1.352902in}{2.472738in}}%
\pgfpathcurveto{\pgfqpoint{1.344666in}{2.472738in}}{\pgfqpoint{1.336766in}{2.469465in}}{\pgfqpoint{1.330942in}{2.463641in}}%
\pgfpathcurveto{\pgfqpoint{1.325118in}{2.457817in}}{\pgfqpoint{1.321846in}{2.449917in}}{\pgfqpoint{1.321846in}{2.441681in}}%
\pgfpathcurveto{\pgfqpoint{1.321846in}{2.433445in}}{\pgfqpoint{1.325118in}{2.425545in}}{\pgfqpoint{1.330942in}{2.419721in}}%
\pgfpathcurveto{\pgfqpoint{1.336766in}{2.413897in}}{\pgfqpoint{1.344666in}{2.410625in}}{\pgfqpoint{1.352902in}{2.410625in}}%
\pgfpathclose%
\pgfusepath{stroke,fill}%
\end{pgfscope}%
\begin{pgfscope}%
\pgfpathrectangle{\pgfqpoint{0.100000in}{0.212622in}}{\pgfqpoint{3.696000in}{3.696000in}}%
\pgfusepath{clip}%
\pgfsetbuttcap%
\pgfsetroundjoin%
\definecolor{currentfill}{rgb}{0.121569,0.466667,0.705882}%
\pgfsetfillcolor{currentfill}%
\pgfsetfillopacity{0.466453}%
\pgfsetlinewidth{1.003750pt}%
\definecolor{currentstroke}{rgb}{0.121569,0.466667,0.705882}%
\pgfsetstrokecolor{currentstroke}%
\pgfsetstrokeopacity{0.466453}%
\pgfsetdash{}{0pt}%
\pgfpathmoveto{\pgfqpoint{1.351482in}{2.408270in}}%
\pgfpathcurveto{\pgfqpoint{1.359718in}{2.408270in}}{\pgfqpoint{1.367618in}{2.411542in}}{\pgfqpoint{1.373442in}{2.417366in}}%
\pgfpathcurveto{\pgfqpoint{1.379266in}{2.423190in}}{\pgfqpoint{1.382538in}{2.431090in}}{\pgfqpoint{1.382538in}{2.439326in}}%
\pgfpathcurveto{\pgfqpoint{1.382538in}{2.447562in}}{\pgfqpoint{1.379266in}{2.455462in}}{\pgfqpoint{1.373442in}{2.461286in}}%
\pgfpathcurveto{\pgfqpoint{1.367618in}{2.467110in}}{\pgfqpoint{1.359718in}{2.470383in}}{\pgfqpoint{1.351482in}{2.470383in}}%
\pgfpathcurveto{\pgfqpoint{1.343246in}{2.470383in}}{\pgfqpoint{1.335346in}{2.467110in}}{\pgfqpoint{1.329522in}{2.461286in}}%
\pgfpathcurveto{\pgfqpoint{1.323698in}{2.455462in}}{\pgfqpoint{1.320425in}{2.447562in}}{\pgfqpoint{1.320425in}{2.439326in}}%
\pgfpathcurveto{\pgfqpoint{1.320425in}{2.431090in}}{\pgfqpoint{1.323698in}{2.423190in}}{\pgfqpoint{1.329522in}{2.417366in}}%
\pgfpathcurveto{\pgfqpoint{1.335346in}{2.411542in}}{\pgfqpoint{1.343246in}{2.408270in}}{\pgfqpoint{1.351482in}{2.408270in}}%
\pgfpathclose%
\pgfusepath{stroke,fill}%
\end{pgfscope}%
\begin{pgfscope}%
\pgfpathrectangle{\pgfqpoint{0.100000in}{0.212622in}}{\pgfqpoint{3.696000in}{3.696000in}}%
\pgfusepath{clip}%
\pgfsetbuttcap%
\pgfsetroundjoin%
\definecolor{currentfill}{rgb}{0.121569,0.466667,0.705882}%
\pgfsetfillcolor{currentfill}%
\pgfsetfillopacity{0.466817}%
\pgfsetlinewidth{1.003750pt}%
\definecolor{currentstroke}{rgb}{0.121569,0.466667,0.705882}%
\pgfsetstrokecolor{currentstroke}%
\pgfsetstrokeopacity{0.466817}%
\pgfsetdash{}{0pt}%
\pgfpathmoveto{\pgfqpoint{1.350489in}{2.406279in}}%
\pgfpathcurveto{\pgfqpoint{1.358726in}{2.406279in}}{\pgfqpoint{1.366626in}{2.409552in}}{\pgfqpoint{1.372450in}{2.415376in}}%
\pgfpathcurveto{\pgfqpoint{1.378273in}{2.421200in}}{\pgfqpoint{1.381546in}{2.429100in}}{\pgfqpoint{1.381546in}{2.437336in}}%
\pgfpathcurveto{\pgfqpoint{1.381546in}{2.445572in}}{\pgfqpoint{1.378273in}{2.453472in}}{\pgfqpoint{1.372450in}{2.459296in}}%
\pgfpathcurveto{\pgfqpoint{1.366626in}{2.465120in}}{\pgfqpoint{1.358726in}{2.468392in}}{\pgfqpoint{1.350489in}{2.468392in}}%
\pgfpathcurveto{\pgfqpoint{1.342253in}{2.468392in}}{\pgfqpoint{1.334353in}{2.465120in}}{\pgfqpoint{1.328529in}{2.459296in}}%
\pgfpathcurveto{\pgfqpoint{1.322705in}{2.453472in}}{\pgfqpoint{1.319433in}{2.445572in}}{\pgfqpoint{1.319433in}{2.437336in}}%
\pgfpathcurveto{\pgfqpoint{1.319433in}{2.429100in}}{\pgfqpoint{1.322705in}{2.421200in}}{\pgfqpoint{1.328529in}{2.415376in}}%
\pgfpathcurveto{\pgfqpoint{1.334353in}{2.409552in}}{\pgfqpoint{1.342253in}{2.406279in}}{\pgfqpoint{1.350489in}{2.406279in}}%
\pgfpathclose%
\pgfusepath{stroke,fill}%
\end{pgfscope}%
\begin{pgfscope}%
\pgfpathrectangle{\pgfqpoint{0.100000in}{0.212622in}}{\pgfqpoint{3.696000in}{3.696000in}}%
\pgfusepath{clip}%
\pgfsetbuttcap%
\pgfsetroundjoin%
\definecolor{currentfill}{rgb}{0.121569,0.466667,0.705882}%
\pgfsetfillcolor{currentfill}%
\pgfsetfillopacity{0.467500}%
\pgfsetlinewidth{1.003750pt}%
\definecolor{currentstroke}{rgb}{0.121569,0.466667,0.705882}%
\pgfsetstrokecolor{currentstroke}%
\pgfsetstrokeopacity{0.467500}%
\pgfsetdash{}{0pt}%
\pgfpathmoveto{\pgfqpoint{1.348450in}{2.403044in}}%
\pgfpathcurveto{\pgfqpoint{1.356686in}{2.403044in}}{\pgfqpoint{1.364586in}{2.406316in}}{\pgfqpoint{1.370410in}{2.412140in}}%
\pgfpathcurveto{\pgfqpoint{1.376234in}{2.417964in}}{\pgfqpoint{1.379506in}{2.425864in}}{\pgfqpoint{1.379506in}{2.434101in}}%
\pgfpathcurveto{\pgfqpoint{1.379506in}{2.442337in}}{\pgfqpoint{1.376234in}{2.450237in}}{\pgfqpoint{1.370410in}{2.456061in}}%
\pgfpathcurveto{\pgfqpoint{1.364586in}{2.461885in}}{\pgfqpoint{1.356686in}{2.465157in}}{\pgfqpoint{1.348450in}{2.465157in}}%
\pgfpathcurveto{\pgfqpoint{1.340213in}{2.465157in}}{\pgfqpoint{1.332313in}{2.461885in}}{\pgfqpoint{1.326489in}{2.456061in}}%
\pgfpathcurveto{\pgfqpoint{1.320665in}{2.450237in}}{\pgfqpoint{1.317393in}{2.442337in}}{\pgfqpoint{1.317393in}{2.434101in}}%
\pgfpathcurveto{\pgfqpoint{1.317393in}{2.425864in}}{\pgfqpoint{1.320665in}{2.417964in}}{\pgfqpoint{1.326489in}{2.412140in}}%
\pgfpathcurveto{\pgfqpoint{1.332313in}{2.406316in}}{\pgfqpoint{1.340213in}{2.403044in}}{\pgfqpoint{1.348450in}{2.403044in}}%
\pgfpathclose%
\pgfusepath{stroke,fill}%
\end{pgfscope}%
\begin{pgfscope}%
\pgfpathrectangle{\pgfqpoint{0.100000in}{0.212622in}}{\pgfqpoint{3.696000in}{3.696000in}}%
\pgfusepath{clip}%
\pgfsetbuttcap%
\pgfsetroundjoin%
\definecolor{currentfill}{rgb}{0.121569,0.466667,0.705882}%
\pgfsetfillcolor{currentfill}%
\pgfsetfillopacity{0.467751}%
\pgfsetlinewidth{1.003750pt}%
\definecolor{currentstroke}{rgb}{0.121569,0.466667,0.705882}%
\pgfsetstrokecolor{currentstroke}%
\pgfsetstrokeopacity{0.467751}%
\pgfsetdash{}{0pt}%
\pgfpathmoveto{\pgfqpoint{2.020497in}{2.650151in}}%
\pgfpathcurveto{\pgfqpoint{2.028733in}{2.650151in}}{\pgfqpoint{2.036633in}{2.653423in}}{\pgfqpoint{2.042457in}{2.659247in}}%
\pgfpathcurveto{\pgfqpoint{2.048281in}{2.665071in}}{\pgfqpoint{2.051553in}{2.672971in}}{\pgfqpoint{2.051553in}{2.681207in}}%
\pgfpathcurveto{\pgfqpoint{2.051553in}{2.689444in}}{\pgfqpoint{2.048281in}{2.697344in}}{\pgfqpoint{2.042457in}{2.703168in}}%
\pgfpathcurveto{\pgfqpoint{2.036633in}{2.708991in}}{\pgfqpoint{2.028733in}{2.712264in}}{\pgfqpoint{2.020497in}{2.712264in}}%
\pgfpathcurveto{\pgfqpoint{2.012261in}{2.712264in}}{\pgfqpoint{2.004361in}{2.708991in}}{\pgfqpoint{1.998537in}{2.703168in}}%
\pgfpathcurveto{\pgfqpoint{1.992713in}{2.697344in}}{\pgfqpoint{1.989440in}{2.689444in}}{\pgfqpoint{1.989440in}{2.681207in}}%
\pgfpathcurveto{\pgfqpoint{1.989440in}{2.672971in}}{\pgfqpoint{1.992713in}{2.665071in}}{\pgfqpoint{1.998537in}{2.659247in}}%
\pgfpathcurveto{\pgfqpoint{2.004361in}{2.653423in}}{\pgfqpoint{2.012261in}{2.650151in}}{\pgfqpoint{2.020497in}{2.650151in}}%
\pgfpathclose%
\pgfusepath{stroke,fill}%
\end{pgfscope}%
\begin{pgfscope}%
\pgfpathrectangle{\pgfqpoint{0.100000in}{0.212622in}}{\pgfqpoint{3.696000in}{3.696000in}}%
\pgfusepath{clip}%
\pgfsetbuttcap%
\pgfsetroundjoin%
\definecolor{currentfill}{rgb}{0.121569,0.466667,0.705882}%
\pgfsetfillcolor{currentfill}%
\pgfsetfillopacity{0.467880}%
\pgfsetlinewidth{1.003750pt}%
\definecolor{currentstroke}{rgb}{0.121569,0.466667,0.705882}%
\pgfsetstrokecolor{currentstroke}%
\pgfsetstrokeopacity{0.467880}%
\pgfsetdash{}{0pt}%
\pgfpathmoveto{\pgfqpoint{1.347759in}{2.401353in}}%
\pgfpathcurveto{\pgfqpoint{1.355996in}{2.401353in}}{\pgfqpoint{1.363896in}{2.404625in}}{\pgfqpoint{1.369720in}{2.410449in}}%
\pgfpathcurveto{\pgfqpoint{1.375544in}{2.416273in}}{\pgfqpoint{1.378816in}{2.424173in}}{\pgfqpoint{1.378816in}{2.432409in}}%
\pgfpathcurveto{\pgfqpoint{1.378816in}{2.440646in}}{\pgfqpoint{1.375544in}{2.448546in}}{\pgfqpoint{1.369720in}{2.454370in}}%
\pgfpathcurveto{\pgfqpoint{1.363896in}{2.460193in}}{\pgfqpoint{1.355996in}{2.463466in}}{\pgfqpoint{1.347759in}{2.463466in}}%
\pgfpathcurveto{\pgfqpoint{1.339523in}{2.463466in}}{\pgfqpoint{1.331623in}{2.460193in}}{\pgfqpoint{1.325799in}{2.454370in}}%
\pgfpathcurveto{\pgfqpoint{1.319975in}{2.448546in}}{\pgfqpoint{1.316703in}{2.440646in}}{\pgfqpoint{1.316703in}{2.432409in}}%
\pgfpathcurveto{\pgfqpoint{1.316703in}{2.424173in}}{\pgfqpoint{1.319975in}{2.416273in}}{\pgfqpoint{1.325799in}{2.410449in}}%
\pgfpathcurveto{\pgfqpoint{1.331623in}{2.404625in}}{\pgfqpoint{1.339523in}{2.401353in}}{\pgfqpoint{1.347759in}{2.401353in}}%
\pgfpathclose%
\pgfusepath{stroke,fill}%
\end{pgfscope}%
\begin{pgfscope}%
\pgfpathrectangle{\pgfqpoint{0.100000in}{0.212622in}}{\pgfqpoint{3.696000in}{3.696000in}}%
\pgfusepath{clip}%
\pgfsetbuttcap%
\pgfsetroundjoin%
\definecolor{currentfill}{rgb}{0.121569,0.466667,0.705882}%
\pgfsetfillcolor{currentfill}%
\pgfsetfillopacity{0.468486}%
\pgfsetlinewidth{1.003750pt}%
\definecolor{currentstroke}{rgb}{0.121569,0.466667,0.705882}%
\pgfsetstrokecolor{currentstroke}%
\pgfsetstrokeopacity{0.468486}%
\pgfsetdash{}{0pt}%
\pgfpathmoveto{\pgfqpoint{1.346039in}{2.398397in}}%
\pgfpathcurveto{\pgfqpoint{1.354275in}{2.398397in}}{\pgfqpoint{1.362175in}{2.401670in}}{\pgfqpoint{1.367999in}{2.407494in}}%
\pgfpathcurveto{\pgfqpoint{1.373823in}{2.413318in}}{\pgfqpoint{1.377096in}{2.421218in}}{\pgfqpoint{1.377096in}{2.429454in}}%
\pgfpathcurveto{\pgfqpoint{1.377096in}{2.437690in}}{\pgfqpoint{1.373823in}{2.445590in}}{\pgfqpoint{1.367999in}{2.451414in}}%
\pgfpathcurveto{\pgfqpoint{1.362175in}{2.457238in}}{\pgfqpoint{1.354275in}{2.460510in}}{\pgfqpoint{1.346039in}{2.460510in}}%
\pgfpathcurveto{\pgfqpoint{1.337803in}{2.460510in}}{\pgfqpoint{1.329903in}{2.457238in}}{\pgfqpoint{1.324079in}{2.451414in}}%
\pgfpathcurveto{\pgfqpoint{1.318255in}{2.445590in}}{\pgfqpoint{1.314983in}{2.437690in}}{\pgfqpoint{1.314983in}{2.429454in}}%
\pgfpathcurveto{\pgfqpoint{1.314983in}{2.421218in}}{\pgfqpoint{1.318255in}{2.413318in}}{\pgfqpoint{1.324079in}{2.407494in}}%
\pgfpathcurveto{\pgfqpoint{1.329903in}{2.401670in}}{\pgfqpoint{1.337803in}{2.398397in}}{\pgfqpoint{1.346039in}{2.398397in}}%
\pgfpathclose%
\pgfusepath{stroke,fill}%
\end{pgfscope}%
\begin{pgfscope}%
\pgfpathrectangle{\pgfqpoint{0.100000in}{0.212622in}}{\pgfqpoint{3.696000in}{3.696000in}}%
\pgfusepath{clip}%
\pgfsetbuttcap%
\pgfsetroundjoin%
\definecolor{currentfill}{rgb}{0.121569,0.466667,0.705882}%
\pgfsetfillcolor{currentfill}%
\pgfsetfillopacity{0.469640}%
\pgfsetlinewidth{1.003750pt}%
\definecolor{currentstroke}{rgb}{0.121569,0.466667,0.705882}%
\pgfsetstrokecolor{currentstroke}%
\pgfsetstrokeopacity{0.469640}%
\pgfsetdash{}{0pt}%
\pgfpathmoveto{\pgfqpoint{1.343180in}{2.392892in}}%
\pgfpathcurveto{\pgfqpoint{1.351416in}{2.392892in}}{\pgfqpoint{1.359316in}{2.396164in}}{\pgfqpoint{1.365140in}{2.401988in}}%
\pgfpathcurveto{\pgfqpoint{1.370964in}{2.407812in}}{\pgfqpoint{1.374236in}{2.415712in}}{\pgfqpoint{1.374236in}{2.423948in}}%
\pgfpathcurveto{\pgfqpoint{1.374236in}{2.432185in}}{\pgfqpoint{1.370964in}{2.440085in}}{\pgfqpoint{1.365140in}{2.445909in}}%
\pgfpathcurveto{\pgfqpoint{1.359316in}{2.451733in}}{\pgfqpoint{1.351416in}{2.455005in}}{\pgfqpoint{1.343180in}{2.455005in}}%
\pgfpathcurveto{\pgfqpoint{1.334944in}{2.455005in}}{\pgfqpoint{1.327043in}{2.451733in}}{\pgfqpoint{1.321220in}{2.445909in}}%
\pgfpathcurveto{\pgfqpoint{1.315396in}{2.440085in}}{\pgfqpoint{1.312123in}{2.432185in}}{\pgfqpoint{1.312123in}{2.423948in}}%
\pgfpathcurveto{\pgfqpoint{1.312123in}{2.415712in}}{\pgfqpoint{1.315396in}{2.407812in}}{\pgfqpoint{1.321220in}{2.401988in}}%
\pgfpathcurveto{\pgfqpoint{1.327043in}{2.396164in}}{\pgfqpoint{1.334944in}{2.392892in}}{\pgfqpoint{1.343180in}{2.392892in}}%
\pgfpathclose%
\pgfusepath{stroke,fill}%
\end{pgfscope}%
\begin{pgfscope}%
\pgfpathrectangle{\pgfqpoint{0.100000in}{0.212622in}}{\pgfqpoint{3.696000in}{3.696000in}}%
\pgfusepath{clip}%
\pgfsetbuttcap%
\pgfsetroundjoin%
\definecolor{currentfill}{rgb}{0.121569,0.466667,0.705882}%
\pgfsetfillcolor{currentfill}%
\pgfsetfillopacity{0.470376}%
\pgfsetlinewidth{1.003750pt}%
\definecolor{currentstroke}{rgb}{0.121569,0.466667,0.705882}%
\pgfsetstrokecolor{currentstroke}%
\pgfsetstrokeopacity{0.470376}%
\pgfsetdash{}{0pt}%
\pgfpathmoveto{\pgfqpoint{2.022652in}{2.638684in}}%
\pgfpathcurveto{\pgfqpoint{2.030888in}{2.638684in}}{\pgfqpoint{2.038788in}{2.641957in}}{\pgfqpoint{2.044612in}{2.647781in}}%
\pgfpathcurveto{\pgfqpoint{2.050436in}{2.653605in}}{\pgfqpoint{2.053708in}{2.661505in}}{\pgfqpoint{2.053708in}{2.669741in}}%
\pgfpathcurveto{\pgfqpoint{2.053708in}{2.677977in}}{\pgfqpoint{2.050436in}{2.685877in}}{\pgfqpoint{2.044612in}{2.691701in}}%
\pgfpathcurveto{\pgfqpoint{2.038788in}{2.697525in}}{\pgfqpoint{2.030888in}{2.700797in}}{\pgfqpoint{2.022652in}{2.700797in}}%
\pgfpathcurveto{\pgfqpoint{2.014416in}{2.700797in}}{\pgfqpoint{2.006516in}{2.697525in}}{\pgfqpoint{2.000692in}{2.691701in}}%
\pgfpathcurveto{\pgfqpoint{1.994868in}{2.685877in}}{\pgfqpoint{1.991595in}{2.677977in}}{\pgfqpoint{1.991595in}{2.669741in}}%
\pgfpathcurveto{\pgfqpoint{1.991595in}{2.661505in}}{\pgfqpoint{1.994868in}{2.653605in}}{\pgfqpoint{2.000692in}{2.647781in}}%
\pgfpathcurveto{\pgfqpoint{2.006516in}{2.641957in}}{\pgfqpoint{2.014416in}{2.638684in}}{\pgfqpoint{2.022652in}{2.638684in}}%
\pgfpathclose%
\pgfusepath{stroke,fill}%
\end{pgfscope}%
\begin{pgfscope}%
\pgfpathrectangle{\pgfqpoint{0.100000in}{0.212622in}}{\pgfqpoint{3.696000in}{3.696000in}}%
\pgfusepath{clip}%
\pgfsetbuttcap%
\pgfsetroundjoin%
\definecolor{currentfill}{rgb}{0.121569,0.466667,0.705882}%
\pgfsetfillcolor{currentfill}%
\pgfsetfillopacity{0.470709}%
\pgfsetlinewidth{1.003750pt}%
\definecolor{currentstroke}{rgb}{0.121569,0.466667,0.705882}%
\pgfsetstrokecolor{currentstroke}%
\pgfsetstrokeopacity{0.470709}%
\pgfsetdash{}{0pt}%
\pgfpathmoveto{\pgfqpoint{1.340494in}{2.388200in}}%
\pgfpathcurveto{\pgfqpoint{1.348731in}{2.388200in}}{\pgfqpoint{1.356631in}{2.391472in}}{\pgfqpoint{1.362455in}{2.397296in}}%
\pgfpathcurveto{\pgfqpoint{1.368279in}{2.403120in}}{\pgfqpoint{1.371551in}{2.411020in}}{\pgfqpoint{1.371551in}{2.419257in}}%
\pgfpathcurveto{\pgfqpoint{1.371551in}{2.427493in}}{\pgfqpoint{1.368279in}{2.435393in}}{\pgfqpoint{1.362455in}{2.441217in}}%
\pgfpathcurveto{\pgfqpoint{1.356631in}{2.447041in}}{\pgfqpoint{1.348731in}{2.450313in}}{\pgfqpoint{1.340494in}{2.450313in}}%
\pgfpathcurveto{\pgfqpoint{1.332258in}{2.450313in}}{\pgfqpoint{1.324358in}{2.447041in}}{\pgfqpoint{1.318534in}{2.441217in}}%
\pgfpathcurveto{\pgfqpoint{1.312710in}{2.435393in}}{\pgfqpoint{1.309438in}{2.427493in}}{\pgfqpoint{1.309438in}{2.419257in}}%
\pgfpathcurveto{\pgfqpoint{1.309438in}{2.411020in}}{\pgfqpoint{1.312710in}{2.403120in}}{\pgfqpoint{1.318534in}{2.397296in}}%
\pgfpathcurveto{\pgfqpoint{1.324358in}{2.391472in}}{\pgfqpoint{1.332258in}{2.388200in}}{\pgfqpoint{1.340494in}{2.388200in}}%
\pgfpathclose%
\pgfusepath{stroke,fill}%
\end{pgfscope}%
\begin{pgfscope}%
\pgfpathrectangle{\pgfqpoint{0.100000in}{0.212622in}}{\pgfqpoint{3.696000in}{3.696000in}}%
\pgfusepath{clip}%
\pgfsetbuttcap%
\pgfsetroundjoin%
\definecolor{currentfill}{rgb}{0.121569,0.466667,0.705882}%
\pgfsetfillcolor{currentfill}%
\pgfsetfillopacity{0.472653}%
\pgfsetlinewidth{1.003750pt}%
\definecolor{currentstroke}{rgb}{0.121569,0.466667,0.705882}%
\pgfsetstrokecolor{currentstroke}%
\pgfsetstrokeopacity{0.472653}%
\pgfsetdash{}{0pt}%
\pgfpathmoveto{\pgfqpoint{1.335262in}{2.380115in}}%
\pgfpathcurveto{\pgfqpoint{1.343498in}{2.380115in}}{\pgfqpoint{1.351399in}{2.383387in}}{\pgfqpoint{1.357222in}{2.389211in}}%
\pgfpathcurveto{\pgfqpoint{1.363046in}{2.395035in}}{\pgfqpoint{1.366319in}{2.402935in}}{\pgfqpoint{1.366319in}{2.411171in}}%
\pgfpathcurveto{\pgfqpoint{1.366319in}{2.419408in}}{\pgfqpoint{1.363046in}{2.427308in}}{\pgfqpoint{1.357222in}{2.433131in}}%
\pgfpathcurveto{\pgfqpoint{1.351399in}{2.438955in}}{\pgfqpoint{1.343498in}{2.442228in}}{\pgfqpoint{1.335262in}{2.442228in}}%
\pgfpathcurveto{\pgfqpoint{1.327026in}{2.442228in}}{\pgfqpoint{1.319126in}{2.438955in}}{\pgfqpoint{1.313302in}{2.433131in}}%
\pgfpathcurveto{\pgfqpoint{1.307478in}{2.427308in}}{\pgfqpoint{1.304206in}{2.419408in}}{\pgfqpoint{1.304206in}{2.411171in}}%
\pgfpathcurveto{\pgfqpoint{1.304206in}{2.402935in}}{\pgfqpoint{1.307478in}{2.395035in}}{\pgfqpoint{1.313302in}{2.389211in}}%
\pgfpathcurveto{\pgfqpoint{1.319126in}{2.383387in}}{\pgfqpoint{1.327026in}{2.380115in}}{\pgfqpoint{1.335262in}{2.380115in}}%
\pgfpathclose%
\pgfusepath{stroke,fill}%
\end{pgfscope}%
\begin{pgfscope}%
\pgfpathrectangle{\pgfqpoint{0.100000in}{0.212622in}}{\pgfqpoint{3.696000in}{3.696000in}}%
\pgfusepath{clip}%
\pgfsetbuttcap%
\pgfsetroundjoin%
\definecolor{currentfill}{rgb}{0.121569,0.466667,0.705882}%
\pgfsetfillcolor{currentfill}%
\pgfsetfillopacity{0.473325}%
\pgfsetlinewidth{1.003750pt}%
\definecolor{currentstroke}{rgb}{0.121569,0.466667,0.705882}%
\pgfsetstrokecolor{currentstroke}%
\pgfsetstrokeopacity{0.473325}%
\pgfsetdash{}{0pt}%
\pgfpathmoveto{\pgfqpoint{2.023966in}{2.626849in}}%
\pgfpathcurveto{\pgfqpoint{2.032202in}{2.626849in}}{\pgfqpoint{2.040103in}{2.630121in}}{\pgfqpoint{2.045926in}{2.635945in}}%
\pgfpathcurveto{\pgfqpoint{2.051750in}{2.641769in}}{\pgfqpoint{2.055023in}{2.649669in}}{\pgfqpoint{2.055023in}{2.657905in}}%
\pgfpathcurveto{\pgfqpoint{2.055023in}{2.666141in}}{\pgfqpoint{2.051750in}{2.674042in}}{\pgfqpoint{2.045926in}{2.679865in}}%
\pgfpathcurveto{\pgfqpoint{2.040103in}{2.685689in}}{\pgfqpoint{2.032202in}{2.688962in}}{\pgfqpoint{2.023966in}{2.688962in}}%
\pgfpathcurveto{\pgfqpoint{2.015730in}{2.688962in}}{\pgfqpoint{2.007830in}{2.685689in}}{\pgfqpoint{2.002006in}{2.679865in}}%
\pgfpathcurveto{\pgfqpoint{1.996182in}{2.674042in}}{\pgfqpoint{1.992910in}{2.666141in}}{\pgfqpoint{1.992910in}{2.657905in}}%
\pgfpathcurveto{\pgfqpoint{1.992910in}{2.649669in}}{\pgfqpoint{1.996182in}{2.641769in}}{\pgfqpoint{2.002006in}{2.635945in}}%
\pgfpathcurveto{\pgfqpoint{2.007830in}{2.630121in}}{\pgfqpoint{2.015730in}{2.626849in}}{\pgfqpoint{2.023966in}{2.626849in}}%
\pgfpathclose%
\pgfusepath{stroke,fill}%
\end{pgfscope}%
\begin{pgfscope}%
\pgfpathrectangle{\pgfqpoint{0.100000in}{0.212622in}}{\pgfqpoint{3.696000in}{3.696000in}}%
\pgfusepath{clip}%
\pgfsetbuttcap%
\pgfsetroundjoin%
\definecolor{currentfill}{rgb}{0.121569,0.466667,0.705882}%
\pgfsetfillcolor{currentfill}%
\pgfsetfillopacity{0.474483}%
\pgfsetlinewidth{1.003750pt}%
\definecolor{currentstroke}{rgb}{0.121569,0.466667,0.705882}%
\pgfsetstrokecolor{currentstroke}%
\pgfsetstrokeopacity{0.474483}%
\pgfsetdash{}{0pt}%
\pgfpathmoveto{\pgfqpoint{1.331203in}{2.372501in}}%
\pgfpathcurveto{\pgfqpoint{1.339439in}{2.372501in}}{\pgfqpoint{1.347339in}{2.375773in}}{\pgfqpoint{1.353163in}{2.381597in}}%
\pgfpathcurveto{\pgfqpoint{1.358987in}{2.387421in}}{\pgfqpoint{1.362260in}{2.395321in}}{\pgfqpoint{1.362260in}{2.403557in}}%
\pgfpathcurveto{\pgfqpoint{1.362260in}{2.411794in}}{\pgfqpoint{1.358987in}{2.419694in}}{\pgfqpoint{1.353163in}{2.425518in}}%
\pgfpathcurveto{\pgfqpoint{1.347339in}{2.431341in}}{\pgfqpoint{1.339439in}{2.434614in}}{\pgfqpoint{1.331203in}{2.434614in}}%
\pgfpathcurveto{\pgfqpoint{1.322967in}{2.434614in}}{\pgfqpoint{1.315067in}{2.431341in}}{\pgfqpoint{1.309243in}{2.425518in}}%
\pgfpathcurveto{\pgfqpoint{1.303419in}{2.419694in}}{\pgfqpoint{1.300147in}{2.411794in}}{\pgfqpoint{1.300147in}{2.403557in}}%
\pgfpathcurveto{\pgfqpoint{1.300147in}{2.395321in}}{\pgfqpoint{1.303419in}{2.387421in}}{\pgfqpoint{1.309243in}{2.381597in}}%
\pgfpathcurveto{\pgfqpoint{1.315067in}{2.375773in}}{\pgfqpoint{1.322967in}{2.372501in}}{\pgfqpoint{1.331203in}{2.372501in}}%
\pgfpathclose%
\pgfusepath{stroke,fill}%
\end{pgfscope}%
\begin{pgfscope}%
\pgfpathrectangle{\pgfqpoint{0.100000in}{0.212622in}}{\pgfqpoint{3.696000in}{3.696000in}}%
\pgfusepath{clip}%
\pgfsetbuttcap%
\pgfsetroundjoin%
\definecolor{currentfill}{rgb}{0.121569,0.466667,0.705882}%
\pgfsetfillcolor{currentfill}%
\pgfsetfillopacity{0.476056}%
\pgfsetlinewidth{1.003750pt}%
\definecolor{currentstroke}{rgb}{0.121569,0.466667,0.705882}%
\pgfsetstrokecolor{currentstroke}%
\pgfsetstrokeopacity{0.476056}%
\pgfsetdash{}{0pt}%
\pgfpathmoveto{\pgfqpoint{1.327053in}{2.366375in}}%
\pgfpathcurveto{\pgfqpoint{1.335289in}{2.366375in}}{\pgfqpoint{1.343189in}{2.369647in}}{\pgfqpoint{1.349013in}{2.375471in}}%
\pgfpathcurveto{\pgfqpoint{1.354837in}{2.381295in}}{\pgfqpoint{1.358109in}{2.389195in}}{\pgfqpoint{1.358109in}{2.397431in}}%
\pgfpathcurveto{\pgfqpoint{1.358109in}{2.405668in}}{\pgfqpoint{1.354837in}{2.413568in}}{\pgfqpoint{1.349013in}{2.419391in}}%
\pgfpathcurveto{\pgfqpoint{1.343189in}{2.425215in}}{\pgfqpoint{1.335289in}{2.428488in}}{\pgfqpoint{1.327053in}{2.428488in}}%
\pgfpathcurveto{\pgfqpoint{1.318816in}{2.428488in}}{\pgfqpoint{1.310916in}{2.425215in}}{\pgfqpoint{1.305092in}{2.419391in}}%
\pgfpathcurveto{\pgfqpoint{1.299268in}{2.413568in}}{\pgfqpoint{1.295996in}{2.405668in}}{\pgfqpoint{1.295996in}{2.397431in}}%
\pgfpathcurveto{\pgfqpoint{1.295996in}{2.389195in}}{\pgfqpoint{1.299268in}{2.381295in}}{\pgfqpoint{1.305092in}{2.375471in}}%
\pgfpathcurveto{\pgfqpoint{1.310916in}{2.369647in}}{\pgfqpoint{1.318816in}{2.366375in}}{\pgfqpoint{1.327053in}{2.366375in}}%
\pgfpathclose%
\pgfusepath{stroke,fill}%
\end{pgfscope}%
\begin{pgfscope}%
\pgfpathrectangle{\pgfqpoint{0.100000in}{0.212622in}}{\pgfqpoint{3.696000in}{3.696000in}}%
\pgfusepath{clip}%
\pgfsetbuttcap%
\pgfsetroundjoin%
\definecolor{currentfill}{rgb}{0.121569,0.466667,0.705882}%
\pgfsetfillcolor{currentfill}%
\pgfsetfillopacity{0.476483}%
\pgfsetlinewidth{1.003750pt}%
\definecolor{currentstroke}{rgb}{0.121569,0.466667,0.705882}%
\pgfsetstrokecolor{currentstroke}%
\pgfsetstrokeopacity{0.476483}%
\pgfsetdash{}{0pt}%
\pgfpathmoveto{\pgfqpoint{2.025284in}{2.614517in}}%
\pgfpathcurveto{\pgfqpoint{2.033521in}{2.614517in}}{\pgfqpoint{2.041421in}{2.617790in}}{\pgfqpoint{2.047245in}{2.623614in}}%
\pgfpathcurveto{\pgfqpoint{2.053068in}{2.629438in}}{\pgfqpoint{2.056341in}{2.637338in}}{\pgfqpoint{2.056341in}{2.645574in}}%
\pgfpathcurveto{\pgfqpoint{2.056341in}{2.653810in}}{\pgfqpoint{2.053068in}{2.661710in}}{\pgfqpoint{2.047245in}{2.667534in}}%
\pgfpathcurveto{\pgfqpoint{2.041421in}{2.673358in}}{\pgfqpoint{2.033521in}{2.676630in}}{\pgfqpoint{2.025284in}{2.676630in}}%
\pgfpathcurveto{\pgfqpoint{2.017048in}{2.676630in}}{\pgfqpoint{2.009148in}{2.673358in}}{\pgfqpoint{2.003324in}{2.667534in}}%
\pgfpathcurveto{\pgfqpoint{1.997500in}{2.661710in}}{\pgfqpoint{1.994228in}{2.653810in}}{\pgfqpoint{1.994228in}{2.645574in}}%
\pgfpathcurveto{\pgfqpoint{1.994228in}{2.637338in}}{\pgfqpoint{1.997500in}{2.629438in}}{\pgfqpoint{2.003324in}{2.623614in}}%
\pgfpathcurveto{\pgfqpoint{2.009148in}{2.617790in}}{\pgfqpoint{2.017048in}{2.614517in}}{\pgfqpoint{2.025284in}{2.614517in}}%
\pgfpathclose%
\pgfusepath{stroke,fill}%
\end{pgfscope}%
\begin{pgfscope}%
\pgfpathrectangle{\pgfqpoint{0.100000in}{0.212622in}}{\pgfqpoint{3.696000in}{3.696000in}}%
\pgfusepath{clip}%
\pgfsetbuttcap%
\pgfsetroundjoin%
\definecolor{currentfill}{rgb}{0.121569,0.466667,0.705882}%
\pgfsetfillcolor{currentfill}%
\pgfsetfillopacity{0.477300}%
\pgfsetlinewidth{1.003750pt}%
\definecolor{currentstroke}{rgb}{0.121569,0.466667,0.705882}%
\pgfsetstrokecolor{currentstroke}%
\pgfsetstrokeopacity{0.477300}%
\pgfsetdash{}{0pt}%
\pgfpathmoveto{\pgfqpoint{1.323895in}{2.361662in}}%
\pgfpathcurveto{\pgfqpoint{1.332131in}{2.361662in}}{\pgfqpoint{1.340031in}{2.364934in}}{\pgfqpoint{1.345855in}{2.370758in}}%
\pgfpathcurveto{\pgfqpoint{1.351679in}{2.376582in}}{\pgfqpoint{1.354951in}{2.384482in}}{\pgfqpoint{1.354951in}{2.392719in}}%
\pgfpathcurveto{\pgfqpoint{1.354951in}{2.400955in}}{\pgfqpoint{1.351679in}{2.408855in}}{\pgfqpoint{1.345855in}{2.414679in}}%
\pgfpathcurveto{\pgfqpoint{1.340031in}{2.420503in}}{\pgfqpoint{1.332131in}{2.423775in}}{\pgfqpoint{1.323895in}{2.423775in}}%
\pgfpathcurveto{\pgfqpoint{1.315658in}{2.423775in}}{\pgfqpoint{1.307758in}{2.420503in}}{\pgfqpoint{1.301934in}{2.414679in}}%
\pgfpathcurveto{\pgfqpoint{1.296110in}{2.408855in}}{\pgfqpoint{1.292838in}{2.400955in}}{\pgfqpoint{1.292838in}{2.392719in}}%
\pgfpathcurveto{\pgfqpoint{1.292838in}{2.384482in}}{\pgfqpoint{1.296110in}{2.376582in}}{\pgfqpoint{1.301934in}{2.370758in}}%
\pgfpathcurveto{\pgfqpoint{1.307758in}{2.364934in}}{\pgfqpoint{1.315658in}{2.361662in}}{\pgfqpoint{1.323895in}{2.361662in}}%
\pgfpathclose%
\pgfusepath{stroke,fill}%
\end{pgfscope}%
\begin{pgfscope}%
\pgfpathrectangle{\pgfqpoint{0.100000in}{0.212622in}}{\pgfqpoint{3.696000in}{3.696000in}}%
\pgfusepath{clip}%
\pgfsetbuttcap%
\pgfsetroundjoin%
\definecolor{currentfill}{rgb}{0.121569,0.466667,0.705882}%
\pgfsetfillcolor{currentfill}%
\pgfsetfillopacity{0.477808}%
\pgfsetlinewidth{1.003750pt}%
\definecolor{currentstroke}{rgb}{0.121569,0.466667,0.705882}%
\pgfsetstrokecolor{currentstroke}%
\pgfsetstrokeopacity{0.477808}%
\pgfsetdash{}{0pt}%
\pgfpathmoveto{\pgfqpoint{1.322412in}{2.359211in}}%
\pgfpathcurveto{\pgfqpoint{1.330648in}{2.359211in}}{\pgfqpoint{1.338548in}{2.362483in}}{\pgfqpoint{1.344372in}{2.368307in}}%
\pgfpathcurveto{\pgfqpoint{1.350196in}{2.374131in}}{\pgfqpoint{1.353469in}{2.382031in}}{\pgfqpoint{1.353469in}{2.390267in}}%
\pgfpathcurveto{\pgfqpoint{1.353469in}{2.398504in}}{\pgfqpoint{1.350196in}{2.406404in}}{\pgfqpoint{1.344372in}{2.412228in}}%
\pgfpathcurveto{\pgfqpoint{1.338548in}{2.418052in}}{\pgfqpoint{1.330648in}{2.421324in}}{\pgfqpoint{1.322412in}{2.421324in}}%
\pgfpathcurveto{\pgfqpoint{1.314176in}{2.421324in}}{\pgfqpoint{1.306276in}{2.418052in}}{\pgfqpoint{1.300452in}{2.412228in}}%
\pgfpathcurveto{\pgfqpoint{1.294628in}{2.406404in}}{\pgfqpoint{1.291356in}{2.398504in}}{\pgfqpoint{1.291356in}{2.390267in}}%
\pgfpathcurveto{\pgfqpoint{1.291356in}{2.382031in}}{\pgfqpoint{1.294628in}{2.374131in}}{\pgfqpoint{1.300452in}{2.368307in}}%
\pgfpathcurveto{\pgfqpoint{1.306276in}{2.362483in}}{\pgfqpoint{1.314176in}{2.359211in}}{\pgfqpoint{1.322412in}{2.359211in}}%
\pgfpathclose%
\pgfusepath{stroke,fill}%
\end{pgfscope}%
\begin{pgfscope}%
\pgfpathrectangle{\pgfqpoint{0.100000in}{0.212622in}}{\pgfqpoint{3.696000in}{3.696000in}}%
\pgfusepath{clip}%
\pgfsetbuttcap%
\pgfsetroundjoin%
\definecolor{currentfill}{rgb}{0.121569,0.466667,0.705882}%
\pgfsetfillcolor{currentfill}%
\pgfsetfillopacity{0.477901}%
\pgfsetlinewidth{1.003750pt}%
\definecolor{currentstroke}{rgb}{0.121569,0.466667,0.705882}%
\pgfsetstrokecolor{currentstroke}%
\pgfsetstrokeopacity{0.477901}%
\pgfsetdash{}{0pt}%
\pgfpathmoveto{\pgfqpoint{2.026513in}{2.606916in}}%
\pgfpathcurveto{\pgfqpoint{2.034750in}{2.606916in}}{\pgfqpoint{2.042650in}{2.610188in}}{\pgfqpoint{2.048474in}{2.616012in}}%
\pgfpathcurveto{\pgfqpoint{2.054298in}{2.621836in}}{\pgfqpoint{2.057570in}{2.629736in}}{\pgfqpoint{2.057570in}{2.637972in}}%
\pgfpathcurveto{\pgfqpoint{2.057570in}{2.646209in}}{\pgfqpoint{2.054298in}{2.654109in}}{\pgfqpoint{2.048474in}{2.659933in}}%
\pgfpathcurveto{\pgfqpoint{2.042650in}{2.665756in}}{\pgfqpoint{2.034750in}{2.669029in}}{\pgfqpoint{2.026513in}{2.669029in}}%
\pgfpathcurveto{\pgfqpoint{2.018277in}{2.669029in}}{\pgfqpoint{2.010377in}{2.665756in}}{\pgfqpoint{2.004553in}{2.659933in}}%
\pgfpathcurveto{\pgfqpoint{1.998729in}{2.654109in}}{\pgfqpoint{1.995457in}{2.646209in}}{\pgfqpoint{1.995457in}{2.637972in}}%
\pgfpathcurveto{\pgfqpoint{1.995457in}{2.629736in}}{\pgfqpoint{1.998729in}{2.621836in}}{\pgfqpoint{2.004553in}{2.616012in}}%
\pgfpathcurveto{\pgfqpoint{2.010377in}{2.610188in}}{\pgfqpoint{2.018277in}{2.606916in}}{\pgfqpoint{2.026513in}{2.606916in}}%
\pgfpathclose%
\pgfusepath{stroke,fill}%
\end{pgfscope}%
\begin{pgfscope}%
\pgfpathrectangle{\pgfqpoint{0.100000in}{0.212622in}}{\pgfqpoint{3.696000in}{3.696000in}}%
\pgfusepath{clip}%
\pgfsetbuttcap%
\pgfsetroundjoin%
\definecolor{currentfill}{rgb}{0.121569,0.466667,0.705882}%
\pgfsetfillcolor{currentfill}%
\pgfsetfillopacity{0.478770}%
\pgfsetlinewidth{1.003750pt}%
\definecolor{currentstroke}{rgb}{0.121569,0.466667,0.705882}%
\pgfsetstrokecolor{currentstroke}%
\pgfsetstrokeopacity{0.478770}%
\pgfsetdash{}{0pt}%
\pgfpathmoveto{\pgfqpoint{1.319814in}{2.354765in}}%
\pgfpathcurveto{\pgfqpoint{1.328051in}{2.354765in}}{\pgfqpoint{1.335951in}{2.358037in}}{\pgfqpoint{1.341775in}{2.363861in}}%
\pgfpathcurveto{\pgfqpoint{1.347599in}{2.369685in}}{\pgfqpoint{1.350871in}{2.377585in}}{\pgfqpoint{1.350871in}{2.385822in}}%
\pgfpathcurveto{\pgfqpoint{1.350871in}{2.394058in}}{\pgfqpoint{1.347599in}{2.401958in}}{\pgfqpoint{1.341775in}{2.407782in}}%
\pgfpathcurveto{\pgfqpoint{1.335951in}{2.413606in}}{\pgfqpoint{1.328051in}{2.416878in}}{\pgfqpoint{1.319814in}{2.416878in}}%
\pgfpathcurveto{\pgfqpoint{1.311578in}{2.416878in}}{\pgfqpoint{1.303678in}{2.413606in}}{\pgfqpoint{1.297854in}{2.407782in}}%
\pgfpathcurveto{\pgfqpoint{1.292030in}{2.401958in}}{\pgfqpoint{1.288758in}{2.394058in}}{\pgfqpoint{1.288758in}{2.385822in}}%
\pgfpathcurveto{\pgfqpoint{1.288758in}{2.377585in}}{\pgfqpoint{1.292030in}{2.369685in}}{\pgfqpoint{1.297854in}{2.363861in}}%
\pgfpathcurveto{\pgfqpoint{1.303678in}{2.358037in}}{\pgfqpoint{1.311578in}{2.354765in}}{\pgfqpoint{1.319814in}{2.354765in}}%
\pgfpathclose%
\pgfusepath{stroke,fill}%
\end{pgfscope}%
\begin{pgfscope}%
\pgfpathrectangle{\pgfqpoint{0.100000in}{0.212622in}}{\pgfqpoint{3.696000in}{3.696000in}}%
\pgfusepath{clip}%
\pgfsetbuttcap%
\pgfsetroundjoin%
\definecolor{currentfill}{rgb}{0.121569,0.466667,0.705882}%
\pgfsetfillcolor{currentfill}%
\pgfsetfillopacity{0.479588}%
\pgfsetlinewidth{1.003750pt}%
\definecolor{currentstroke}{rgb}{0.121569,0.466667,0.705882}%
\pgfsetstrokecolor{currentstroke}%
\pgfsetstrokeopacity{0.479588}%
\pgfsetdash{}{0pt}%
\pgfpathmoveto{\pgfqpoint{1.317490in}{2.350663in}}%
\pgfpathcurveto{\pgfqpoint{1.325727in}{2.350663in}}{\pgfqpoint{1.333627in}{2.353936in}}{\pgfqpoint{1.339450in}{2.359760in}}%
\pgfpathcurveto{\pgfqpoint{1.345274in}{2.365583in}}{\pgfqpoint{1.348547in}{2.373483in}}{\pgfqpoint{1.348547in}{2.381720in}}%
\pgfpathcurveto{\pgfqpoint{1.348547in}{2.389956in}}{\pgfqpoint{1.345274in}{2.397856in}}{\pgfqpoint{1.339450in}{2.403680in}}%
\pgfpathcurveto{\pgfqpoint{1.333627in}{2.409504in}}{\pgfqpoint{1.325727in}{2.412776in}}{\pgfqpoint{1.317490in}{2.412776in}}%
\pgfpathcurveto{\pgfqpoint{1.309254in}{2.412776in}}{\pgfqpoint{1.301354in}{2.409504in}}{\pgfqpoint{1.295530in}{2.403680in}}%
\pgfpathcurveto{\pgfqpoint{1.289706in}{2.397856in}}{\pgfqpoint{1.286434in}{2.389956in}}{\pgfqpoint{1.286434in}{2.381720in}}%
\pgfpathcurveto{\pgfqpoint{1.286434in}{2.373483in}}{\pgfqpoint{1.289706in}{2.365583in}}{\pgfqpoint{1.295530in}{2.359760in}}%
\pgfpathcurveto{\pgfqpoint{1.301354in}{2.353936in}}{\pgfqpoint{1.309254in}{2.350663in}}{\pgfqpoint{1.317490in}{2.350663in}}%
\pgfpathclose%
\pgfusepath{stroke,fill}%
\end{pgfscope}%
\begin{pgfscope}%
\pgfpathrectangle{\pgfqpoint{0.100000in}{0.212622in}}{\pgfqpoint{3.696000in}{3.696000in}}%
\pgfusepath{clip}%
\pgfsetbuttcap%
\pgfsetroundjoin%
\definecolor{currentfill}{rgb}{0.121569,0.466667,0.705882}%
\pgfsetfillcolor{currentfill}%
\pgfsetfillopacity{0.480000}%
\pgfsetlinewidth{1.003750pt}%
\definecolor{currentstroke}{rgb}{0.121569,0.466667,0.705882}%
\pgfsetstrokecolor{currentstroke}%
\pgfsetstrokeopacity{0.480000}%
\pgfsetdash{}{0pt}%
\pgfpathmoveto{\pgfqpoint{2.027330in}{2.597881in}}%
\pgfpathcurveto{\pgfqpoint{2.035566in}{2.597881in}}{\pgfqpoint{2.043466in}{2.601153in}}{\pgfqpoint{2.049290in}{2.606977in}}%
\pgfpathcurveto{\pgfqpoint{2.055114in}{2.612801in}}{\pgfqpoint{2.058386in}{2.620701in}}{\pgfqpoint{2.058386in}{2.628937in}}%
\pgfpathcurveto{\pgfqpoint{2.058386in}{2.637174in}}{\pgfqpoint{2.055114in}{2.645074in}}{\pgfqpoint{2.049290in}{2.650898in}}%
\pgfpathcurveto{\pgfqpoint{2.043466in}{2.656722in}}{\pgfqpoint{2.035566in}{2.659994in}}{\pgfqpoint{2.027330in}{2.659994in}}%
\pgfpathcurveto{\pgfqpoint{2.019094in}{2.659994in}}{\pgfqpoint{2.011194in}{2.656722in}}{\pgfqpoint{2.005370in}{2.650898in}}%
\pgfpathcurveto{\pgfqpoint{1.999546in}{2.645074in}}{\pgfqpoint{1.996273in}{2.637174in}}{\pgfqpoint{1.996273in}{2.628937in}}%
\pgfpathcurveto{\pgfqpoint{1.996273in}{2.620701in}}{\pgfqpoint{1.999546in}{2.612801in}}{\pgfqpoint{2.005370in}{2.606977in}}%
\pgfpathcurveto{\pgfqpoint{2.011194in}{2.601153in}}{\pgfqpoint{2.019094in}{2.597881in}}{\pgfqpoint{2.027330in}{2.597881in}}%
\pgfpathclose%
\pgfusepath{stroke,fill}%
\end{pgfscope}%
\begin{pgfscope}%
\pgfpathrectangle{\pgfqpoint{0.100000in}{0.212622in}}{\pgfqpoint{3.696000in}{3.696000in}}%
\pgfusepath{clip}%
\pgfsetbuttcap%
\pgfsetroundjoin%
\definecolor{currentfill}{rgb}{0.121569,0.466667,0.705882}%
\pgfsetfillcolor{currentfill}%
\pgfsetfillopacity{0.480217}%
\pgfsetlinewidth{1.003750pt}%
\definecolor{currentstroke}{rgb}{0.121569,0.466667,0.705882}%
\pgfsetstrokecolor{currentstroke}%
\pgfsetstrokeopacity{0.480217}%
\pgfsetdash{}{0pt}%
\pgfpathmoveto{\pgfqpoint{1.315404in}{2.347182in}}%
\pgfpathcurveto{\pgfqpoint{1.323641in}{2.347182in}}{\pgfqpoint{1.331541in}{2.350454in}}{\pgfqpoint{1.337365in}{2.356278in}}%
\pgfpathcurveto{\pgfqpoint{1.343188in}{2.362102in}}{\pgfqpoint{1.346461in}{2.370002in}}{\pgfqpoint{1.346461in}{2.378238in}}%
\pgfpathcurveto{\pgfqpoint{1.346461in}{2.386475in}}{\pgfqpoint{1.343188in}{2.394375in}}{\pgfqpoint{1.337365in}{2.400199in}}%
\pgfpathcurveto{\pgfqpoint{1.331541in}{2.406022in}}{\pgfqpoint{1.323641in}{2.409295in}}{\pgfqpoint{1.315404in}{2.409295in}}%
\pgfpathcurveto{\pgfqpoint{1.307168in}{2.409295in}}{\pgfqpoint{1.299268in}{2.406022in}}{\pgfqpoint{1.293444in}{2.400199in}}%
\pgfpathcurveto{\pgfqpoint{1.287620in}{2.394375in}}{\pgfqpoint{1.284348in}{2.386475in}}{\pgfqpoint{1.284348in}{2.378238in}}%
\pgfpathcurveto{\pgfqpoint{1.284348in}{2.370002in}}{\pgfqpoint{1.287620in}{2.362102in}}{\pgfqpoint{1.293444in}{2.356278in}}%
\pgfpathcurveto{\pgfqpoint{1.299268in}{2.350454in}}{\pgfqpoint{1.307168in}{2.347182in}}{\pgfqpoint{1.315404in}{2.347182in}}%
\pgfpathclose%
\pgfusepath{stroke,fill}%
\end{pgfscope}%
\begin{pgfscope}%
\pgfpathrectangle{\pgfqpoint{0.100000in}{0.212622in}}{\pgfqpoint{3.696000in}{3.696000in}}%
\pgfusepath{clip}%
\pgfsetbuttcap%
\pgfsetroundjoin%
\definecolor{currentfill}{rgb}{0.121569,0.466667,0.705882}%
\pgfsetfillcolor{currentfill}%
\pgfsetfillopacity{0.481166}%
\pgfsetlinewidth{1.003750pt}%
\definecolor{currentstroke}{rgb}{0.121569,0.466667,0.705882}%
\pgfsetstrokecolor{currentstroke}%
\pgfsetstrokeopacity{0.481166}%
\pgfsetdash{}{0pt}%
\pgfpathmoveto{\pgfqpoint{2.027822in}{2.592983in}}%
\pgfpathcurveto{\pgfqpoint{2.036058in}{2.592983in}}{\pgfqpoint{2.043958in}{2.596255in}}{\pgfqpoint{2.049782in}{2.602079in}}%
\pgfpathcurveto{\pgfqpoint{2.055606in}{2.607903in}}{\pgfqpoint{2.058878in}{2.615803in}}{\pgfqpoint{2.058878in}{2.624040in}}%
\pgfpathcurveto{\pgfqpoint{2.058878in}{2.632276in}}{\pgfqpoint{2.055606in}{2.640176in}}{\pgfqpoint{2.049782in}{2.646000in}}%
\pgfpathcurveto{\pgfqpoint{2.043958in}{2.651824in}}{\pgfqpoint{2.036058in}{2.655096in}}{\pgfqpoint{2.027822in}{2.655096in}}%
\pgfpathcurveto{\pgfqpoint{2.019586in}{2.655096in}}{\pgfqpoint{2.011686in}{2.651824in}}{\pgfqpoint{2.005862in}{2.646000in}}%
\pgfpathcurveto{\pgfqpoint{2.000038in}{2.640176in}}{\pgfqpoint{1.996765in}{2.632276in}}{\pgfqpoint{1.996765in}{2.624040in}}%
\pgfpathcurveto{\pgfqpoint{1.996765in}{2.615803in}}{\pgfqpoint{2.000038in}{2.607903in}}{\pgfqpoint{2.005862in}{2.602079in}}%
\pgfpathcurveto{\pgfqpoint{2.011686in}{2.596255in}}{\pgfqpoint{2.019586in}{2.592983in}}{\pgfqpoint{2.027822in}{2.592983in}}%
\pgfpathclose%
\pgfusepath{stroke,fill}%
\end{pgfscope}%
\begin{pgfscope}%
\pgfpathrectangle{\pgfqpoint{0.100000in}{0.212622in}}{\pgfqpoint{3.696000in}{3.696000in}}%
\pgfusepath{clip}%
\pgfsetbuttcap%
\pgfsetroundjoin%
\definecolor{currentfill}{rgb}{0.121569,0.466667,0.705882}%
\pgfsetfillcolor{currentfill}%
\pgfsetfillopacity{0.481422}%
\pgfsetlinewidth{1.003750pt}%
\definecolor{currentstroke}{rgb}{0.121569,0.466667,0.705882}%
\pgfsetstrokecolor{currentstroke}%
\pgfsetstrokeopacity{0.481422}%
\pgfsetdash{}{0pt}%
\pgfpathmoveto{\pgfqpoint{1.312117in}{2.340417in}}%
\pgfpathcurveto{\pgfqpoint{1.320353in}{2.340417in}}{\pgfqpoint{1.328253in}{2.343690in}}{\pgfqpoint{1.334077in}{2.349514in}}%
\pgfpathcurveto{\pgfqpoint{1.339901in}{2.355337in}}{\pgfqpoint{1.343173in}{2.363238in}}{\pgfqpoint{1.343173in}{2.371474in}}%
\pgfpathcurveto{\pgfqpoint{1.343173in}{2.379710in}}{\pgfqpoint{1.339901in}{2.387610in}}{\pgfqpoint{1.334077in}{2.393434in}}%
\pgfpathcurveto{\pgfqpoint{1.328253in}{2.399258in}}{\pgfqpoint{1.320353in}{2.402530in}}{\pgfqpoint{1.312117in}{2.402530in}}%
\pgfpathcurveto{\pgfqpoint{1.303880in}{2.402530in}}{\pgfqpoint{1.295980in}{2.399258in}}{\pgfqpoint{1.290156in}{2.393434in}}%
\pgfpathcurveto{\pgfqpoint{1.284332in}{2.387610in}}{\pgfqpoint{1.281060in}{2.379710in}}{\pgfqpoint{1.281060in}{2.371474in}}%
\pgfpathcurveto{\pgfqpoint{1.281060in}{2.363238in}}{\pgfqpoint{1.284332in}{2.355337in}}{\pgfqpoint{1.290156in}{2.349514in}}%
\pgfpathcurveto{\pgfqpoint{1.295980in}{2.343690in}}{\pgfqpoint{1.303880in}{2.340417in}}{\pgfqpoint{1.312117in}{2.340417in}}%
\pgfpathclose%
\pgfusepath{stroke,fill}%
\end{pgfscope}%
\begin{pgfscope}%
\pgfpathrectangle{\pgfqpoint{0.100000in}{0.212622in}}{\pgfqpoint{3.696000in}{3.696000in}}%
\pgfusepath{clip}%
\pgfsetbuttcap%
\pgfsetroundjoin%
\definecolor{currentfill}{rgb}{0.121569,0.466667,0.705882}%
\pgfsetfillcolor{currentfill}%
\pgfsetfillopacity{0.481691}%
\pgfsetlinewidth{1.003750pt}%
\definecolor{currentstroke}{rgb}{0.121569,0.466667,0.705882}%
\pgfsetstrokecolor{currentstroke}%
\pgfsetstrokeopacity{0.481691}%
\pgfsetdash{}{0pt}%
\pgfpathmoveto{\pgfqpoint{2.028285in}{2.590000in}}%
\pgfpathcurveto{\pgfqpoint{2.036521in}{2.590000in}}{\pgfqpoint{2.044421in}{2.593272in}}{\pgfqpoint{2.050245in}{2.599096in}}%
\pgfpathcurveto{\pgfqpoint{2.056069in}{2.604920in}}{\pgfqpoint{2.059342in}{2.612820in}}{\pgfqpoint{2.059342in}{2.621057in}}%
\pgfpathcurveto{\pgfqpoint{2.059342in}{2.629293in}}{\pgfqpoint{2.056069in}{2.637193in}}{\pgfqpoint{2.050245in}{2.643017in}}%
\pgfpathcurveto{\pgfqpoint{2.044421in}{2.648841in}}{\pgfqpoint{2.036521in}{2.652113in}}{\pgfqpoint{2.028285in}{2.652113in}}%
\pgfpathcurveto{\pgfqpoint{2.020049in}{2.652113in}}{\pgfqpoint{2.012149in}{2.648841in}}{\pgfqpoint{2.006325in}{2.643017in}}%
\pgfpathcurveto{\pgfqpoint{2.000501in}{2.637193in}}{\pgfqpoint{1.997229in}{2.629293in}}{\pgfqpoint{1.997229in}{2.621057in}}%
\pgfpathcurveto{\pgfqpoint{1.997229in}{2.612820in}}{\pgfqpoint{2.000501in}{2.604920in}}{\pgfqpoint{2.006325in}{2.599096in}}%
\pgfpathcurveto{\pgfqpoint{2.012149in}{2.593272in}}{\pgfqpoint{2.020049in}{2.590000in}}{\pgfqpoint{2.028285in}{2.590000in}}%
\pgfpathclose%
\pgfusepath{stroke,fill}%
\end{pgfscope}%
\begin{pgfscope}%
\pgfpathrectangle{\pgfqpoint{0.100000in}{0.212622in}}{\pgfqpoint{3.696000in}{3.696000in}}%
\pgfusepath{clip}%
\pgfsetbuttcap%
\pgfsetroundjoin%
\definecolor{currentfill}{rgb}{0.121569,0.466667,0.705882}%
\pgfsetfillcolor{currentfill}%
\pgfsetfillopacity{0.481953}%
\pgfsetlinewidth{1.003750pt}%
\definecolor{currentstroke}{rgb}{0.121569,0.466667,0.705882}%
\pgfsetstrokecolor{currentstroke}%
\pgfsetstrokeopacity{0.481953}%
\pgfsetdash{}{0pt}%
\pgfpathmoveto{\pgfqpoint{1.310048in}{2.336803in}}%
\pgfpathcurveto{\pgfqpoint{1.318285in}{2.336803in}}{\pgfqpoint{1.326185in}{2.340076in}}{\pgfqpoint{1.332009in}{2.345900in}}%
\pgfpathcurveto{\pgfqpoint{1.337833in}{2.351724in}}{\pgfqpoint{1.341105in}{2.359624in}}{\pgfqpoint{1.341105in}{2.367860in}}%
\pgfpathcurveto{\pgfqpoint{1.341105in}{2.376096in}}{\pgfqpoint{1.337833in}{2.383996in}}{\pgfqpoint{1.332009in}{2.389820in}}%
\pgfpathcurveto{\pgfqpoint{1.326185in}{2.395644in}}{\pgfqpoint{1.318285in}{2.398916in}}{\pgfqpoint{1.310048in}{2.398916in}}%
\pgfpathcurveto{\pgfqpoint{1.301812in}{2.398916in}}{\pgfqpoint{1.293912in}{2.395644in}}{\pgfqpoint{1.288088in}{2.389820in}}%
\pgfpathcurveto{\pgfqpoint{1.282264in}{2.383996in}}{\pgfqpoint{1.278992in}{2.376096in}}{\pgfqpoint{1.278992in}{2.367860in}}%
\pgfpathcurveto{\pgfqpoint{1.278992in}{2.359624in}}{\pgfqpoint{1.282264in}{2.351724in}}{\pgfqpoint{1.288088in}{2.345900in}}%
\pgfpathcurveto{\pgfqpoint{1.293912in}{2.340076in}}{\pgfqpoint{1.301812in}{2.336803in}}{\pgfqpoint{1.310048in}{2.336803in}}%
\pgfpathclose%
\pgfusepath{stroke,fill}%
\end{pgfscope}%
\begin{pgfscope}%
\pgfpathrectangle{\pgfqpoint{0.100000in}{0.212622in}}{\pgfqpoint{3.696000in}{3.696000in}}%
\pgfusepath{clip}%
\pgfsetbuttcap%
\pgfsetroundjoin%
\definecolor{currentfill}{rgb}{0.121569,0.466667,0.705882}%
\pgfsetfillcolor{currentfill}%
\pgfsetfillopacity{0.482426}%
\pgfsetlinewidth{1.003750pt}%
\definecolor{currentstroke}{rgb}{0.121569,0.466667,0.705882}%
\pgfsetstrokecolor{currentstroke}%
\pgfsetstrokeopacity{0.482426}%
\pgfsetdash{}{0pt}%
\pgfpathmoveto{\pgfqpoint{1.308448in}{2.333199in}}%
\pgfpathcurveto{\pgfqpoint{1.316684in}{2.333199in}}{\pgfqpoint{1.324584in}{2.336472in}}{\pgfqpoint{1.330408in}{2.342296in}}%
\pgfpathcurveto{\pgfqpoint{1.336232in}{2.348119in}}{\pgfqpoint{1.339504in}{2.356020in}}{\pgfqpoint{1.339504in}{2.364256in}}%
\pgfpathcurveto{\pgfqpoint{1.339504in}{2.372492in}}{\pgfqpoint{1.336232in}{2.380392in}}{\pgfqpoint{1.330408in}{2.386216in}}%
\pgfpathcurveto{\pgfqpoint{1.324584in}{2.392040in}}{\pgfqpoint{1.316684in}{2.395312in}}{\pgfqpoint{1.308448in}{2.395312in}}%
\pgfpathcurveto{\pgfqpoint{1.300211in}{2.395312in}}{\pgfqpoint{1.292311in}{2.392040in}}{\pgfqpoint{1.286487in}{2.386216in}}%
\pgfpathcurveto{\pgfqpoint{1.280663in}{2.380392in}}{\pgfqpoint{1.277391in}{2.372492in}}{\pgfqpoint{1.277391in}{2.364256in}}%
\pgfpathcurveto{\pgfqpoint{1.277391in}{2.356020in}}{\pgfqpoint{1.280663in}{2.348119in}}{\pgfqpoint{1.286487in}{2.342296in}}%
\pgfpathcurveto{\pgfqpoint{1.292311in}{2.336472in}}{\pgfqpoint{1.300211in}{2.333199in}}{\pgfqpoint{1.308448in}{2.333199in}}%
\pgfpathclose%
\pgfusepath{stroke,fill}%
\end{pgfscope}%
\begin{pgfscope}%
\pgfpathrectangle{\pgfqpoint{0.100000in}{0.212622in}}{\pgfqpoint{3.696000in}{3.696000in}}%
\pgfusepath{clip}%
\pgfsetbuttcap%
\pgfsetroundjoin%
\definecolor{currentfill}{rgb}{0.121569,0.466667,0.705882}%
\pgfsetfillcolor{currentfill}%
\pgfsetfillopacity{0.482601}%
\pgfsetlinewidth{1.003750pt}%
\definecolor{currentstroke}{rgb}{0.121569,0.466667,0.705882}%
\pgfsetstrokecolor{currentstroke}%
\pgfsetstrokeopacity{0.482601}%
\pgfsetdash{}{0pt}%
\pgfpathmoveto{\pgfqpoint{2.028599in}{2.585546in}}%
\pgfpathcurveto{\pgfqpoint{2.036835in}{2.585546in}}{\pgfqpoint{2.044735in}{2.588818in}}{\pgfqpoint{2.050559in}{2.594642in}}%
\pgfpathcurveto{\pgfqpoint{2.056383in}{2.600466in}}{\pgfqpoint{2.059655in}{2.608366in}}{\pgfqpoint{2.059655in}{2.616603in}}%
\pgfpathcurveto{\pgfqpoint{2.059655in}{2.624839in}}{\pgfqpoint{2.056383in}{2.632739in}}{\pgfqpoint{2.050559in}{2.638563in}}%
\pgfpathcurveto{\pgfqpoint{2.044735in}{2.644387in}}{\pgfqpoint{2.036835in}{2.647659in}}{\pgfqpoint{2.028599in}{2.647659in}}%
\pgfpathcurveto{\pgfqpoint{2.020363in}{2.647659in}}{\pgfqpoint{2.012463in}{2.644387in}}{\pgfqpoint{2.006639in}{2.638563in}}%
\pgfpathcurveto{\pgfqpoint{2.000815in}{2.632739in}}{\pgfqpoint{1.997542in}{2.624839in}}{\pgfqpoint{1.997542in}{2.616603in}}%
\pgfpathcurveto{\pgfqpoint{1.997542in}{2.608366in}}{\pgfqpoint{2.000815in}{2.600466in}}{\pgfqpoint{2.006639in}{2.594642in}}%
\pgfpathcurveto{\pgfqpoint{2.012463in}{2.588818in}}{\pgfqpoint{2.020363in}{2.585546in}}{\pgfqpoint{2.028599in}{2.585546in}}%
\pgfpathclose%
\pgfusepath{stroke,fill}%
\end{pgfscope}%
\begin{pgfscope}%
\pgfpathrectangle{\pgfqpoint{0.100000in}{0.212622in}}{\pgfqpoint{3.696000in}{3.696000in}}%
\pgfusepath{clip}%
\pgfsetbuttcap%
\pgfsetroundjoin%
\definecolor{currentfill}{rgb}{0.121569,0.466667,0.705882}%
\pgfsetfillcolor{currentfill}%
\pgfsetfillopacity{0.482761}%
\pgfsetlinewidth{1.003750pt}%
\definecolor{currentstroke}{rgb}{0.121569,0.466667,0.705882}%
\pgfsetstrokecolor{currentstroke}%
\pgfsetstrokeopacity{0.482761}%
\pgfsetdash{}{0pt}%
\pgfpathmoveto{\pgfqpoint{1.307104in}{2.330559in}}%
\pgfpathcurveto{\pgfqpoint{1.315340in}{2.330559in}}{\pgfqpoint{1.323240in}{2.333831in}}{\pgfqpoint{1.329064in}{2.339655in}}%
\pgfpathcurveto{\pgfqpoint{1.334888in}{2.345479in}}{\pgfqpoint{1.338160in}{2.353379in}}{\pgfqpoint{1.338160in}{2.361615in}}%
\pgfpathcurveto{\pgfqpoint{1.338160in}{2.369851in}}{\pgfqpoint{1.334888in}{2.377751in}}{\pgfqpoint{1.329064in}{2.383575in}}%
\pgfpathcurveto{\pgfqpoint{1.323240in}{2.389399in}}{\pgfqpoint{1.315340in}{2.392672in}}{\pgfqpoint{1.307104in}{2.392672in}}%
\pgfpathcurveto{\pgfqpoint{1.298867in}{2.392672in}}{\pgfqpoint{1.290967in}{2.389399in}}{\pgfqpoint{1.285143in}{2.383575in}}%
\pgfpathcurveto{\pgfqpoint{1.279319in}{2.377751in}}{\pgfqpoint{1.276047in}{2.369851in}}{\pgfqpoint{1.276047in}{2.361615in}}%
\pgfpathcurveto{\pgfqpoint{1.276047in}{2.353379in}}{\pgfqpoint{1.279319in}{2.345479in}}{\pgfqpoint{1.285143in}{2.339655in}}%
\pgfpathcurveto{\pgfqpoint{1.290967in}{2.333831in}}{\pgfqpoint{1.298867in}{2.330559in}}{\pgfqpoint{1.307104in}{2.330559in}}%
\pgfpathclose%
\pgfusepath{stroke,fill}%
\end{pgfscope}%
\begin{pgfscope}%
\pgfpathrectangle{\pgfqpoint{0.100000in}{0.212622in}}{\pgfqpoint{3.696000in}{3.696000in}}%
\pgfusepath{clip}%
\pgfsetbuttcap%
\pgfsetroundjoin%
\definecolor{currentfill}{rgb}{0.121569,0.466667,0.705882}%
\pgfsetfillcolor{currentfill}%
\pgfsetfillopacity{0.482943}%
\pgfsetlinewidth{1.003750pt}%
\definecolor{currentstroke}{rgb}{0.121569,0.466667,0.705882}%
\pgfsetstrokecolor{currentstroke}%
\pgfsetstrokeopacity{0.482943}%
\pgfsetdash{}{0pt}%
\pgfpathmoveto{\pgfqpoint{1.306567in}{2.328901in}}%
\pgfpathcurveto{\pgfqpoint{1.314804in}{2.328901in}}{\pgfqpoint{1.322704in}{2.332174in}}{\pgfqpoint{1.328528in}{2.337997in}}%
\pgfpathcurveto{\pgfqpoint{1.334352in}{2.343821in}}{\pgfqpoint{1.337624in}{2.351721in}}{\pgfqpoint{1.337624in}{2.359958in}}%
\pgfpathcurveto{\pgfqpoint{1.337624in}{2.368194in}}{\pgfqpoint{1.334352in}{2.376094in}}{\pgfqpoint{1.328528in}{2.381918in}}%
\pgfpathcurveto{\pgfqpoint{1.322704in}{2.387742in}}{\pgfqpoint{1.314804in}{2.391014in}}{\pgfqpoint{1.306567in}{2.391014in}}%
\pgfpathcurveto{\pgfqpoint{1.298331in}{2.391014in}}{\pgfqpoint{1.290431in}{2.387742in}}{\pgfqpoint{1.284607in}{2.381918in}}%
\pgfpathcurveto{\pgfqpoint{1.278783in}{2.376094in}}{\pgfqpoint{1.275511in}{2.368194in}}{\pgfqpoint{1.275511in}{2.359958in}}%
\pgfpathcurveto{\pgfqpoint{1.275511in}{2.351721in}}{\pgfqpoint{1.278783in}{2.343821in}}{\pgfqpoint{1.284607in}{2.337997in}}%
\pgfpathcurveto{\pgfqpoint{1.290431in}{2.332174in}}{\pgfqpoint{1.298331in}{2.328901in}}{\pgfqpoint{1.306567in}{2.328901in}}%
\pgfpathclose%
\pgfusepath{stroke,fill}%
\end{pgfscope}%
\begin{pgfscope}%
\pgfpathrectangle{\pgfqpoint{0.100000in}{0.212622in}}{\pgfqpoint{3.696000in}{3.696000in}}%
\pgfusepath{clip}%
\pgfsetbuttcap%
\pgfsetroundjoin%
\definecolor{currentfill}{rgb}{0.121569,0.466667,0.705882}%
\pgfsetfillcolor{currentfill}%
\pgfsetfillopacity{0.483130}%
\pgfsetlinewidth{1.003750pt}%
\definecolor{currentstroke}{rgb}{0.121569,0.466667,0.705882}%
\pgfsetstrokecolor{currentstroke}%
\pgfsetstrokeopacity{0.483130}%
\pgfsetdash{}{0pt}%
\pgfpathmoveto{\pgfqpoint{1.306051in}{2.328019in}}%
\pgfpathcurveto{\pgfqpoint{1.314288in}{2.328019in}}{\pgfqpoint{1.322188in}{2.331291in}}{\pgfqpoint{1.328012in}{2.337115in}}%
\pgfpathcurveto{\pgfqpoint{1.333835in}{2.342939in}}{\pgfqpoint{1.337108in}{2.350839in}}{\pgfqpoint{1.337108in}{2.359076in}}%
\pgfpathcurveto{\pgfqpoint{1.337108in}{2.367312in}}{\pgfqpoint{1.333835in}{2.375212in}}{\pgfqpoint{1.328012in}{2.381036in}}%
\pgfpathcurveto{\pgfqpoint{1.322188in}{2.386860in}}{\pgfqpoint{1.314288in}{2.390132in}}{\pgfqpoint{1.306051in}{2.390132in}}%
\pgfpathcurveto{\pgfqpoint{1.297815in}{2.390132in}}{\pgfqpoint{1.289915in}{2.386860in}}{\pgfqpoint{1.284091in}{2.381036in}}%
\pgfpathcurveto{\pgfqpoint{1.278267in}{2.375212in}}{\pgfqpoint{1.274995in}{2.367312in}}{\pgfqpoint{1.274995in}{2.359076in}}%
\pgfpathcurveto{\pgfqpoint{1.274995in}{2.350839in}}{\pgfqpoint{1.278267in}{2.342939in}}{\pgfqpoint{1.284091in}{2.337115in}}%
\pgfpathcurveto{\pgfqpoint{1.289915in}{2.331291in}}{\pgfqpoint{1.297815in}{2.328019in}}{\pgfqpoint{1.306051in}{2.328019in}}%
\pgfpathclose%
\pgfusepath{stroke,fill}%
\end{pgfscope}%
\begin{pgfscope}%
\pgfpathrectangle{\pgfqpoint{0.100000in}{0.212622in}}{\pgfqpoint{3.696000in}{3.696000in}}%
\pgfusepath{clip}%
\pgfsetbuttcap%
\pgfsetroundjoin%
\definecolor{currentfill}{rgb}{0.121569,0.466667,0.705882}%
\pgfsetfillcolor{currentfill}%
\pgfsetfillopacity{0.483466}%
\pgfsetlinewidth{1.003750pt}%
\definecolor{currentstroke}{rgb}{0.121569,0.466667,0.705882}%
\pgfsetstrokecolor{currentstroke}%
\pgfsetstrokeopacity{0.483466}%
\pgfsetdash{}{0pt}%
\pgfpathmoveto{\pgfqpoint{1.305077in}{2.326444in}}%
\pgfpathcurveto{\pgfqpoint{1.313314in}{2.326444in}}{\pgfqpoint{1.321214in}{2.329717in}}{\pgfqpoint{1.327038in}{2.335540in}}%
\pgfpathcurveto{\pgfqpoint{1.332862in}{2.341364in}}{\pgfqpoint{1.336134in}{2.349264in}}{\pgfqpoint{1.336134in}{2.357501in}}%
\pgfpathcurveto{\pgfqpoint{1.336134in}{2.365737in}}{\pgfqpoint{1.332862in}{2.373637in}}{\pgfqpoint{1.327038in}{2.379461in}}%
\pgfpathcurveto{\pgfqpoint{1.321214in}{2.385285in}}{\pgfqpoint{1.313314in}{2.388557in}}{\pgfqpoint{1.305077in}{2.388557in}}%
\pgfpathcurveto{\pgfqpoint{1.296841in}{2.388557in}}{\pgfqpoint{1.288941in}{2.385285in}}{\pgfqpoint{1.283117in}{2.379461in}}%
\pgfpathcurveto{\pgfqpoint{1.277293in}{2.373637in}}{\pgfqpoint{1.274021in}{2.365737in}}{\pgfqpoint{1.274021in}{2.357501in}}%
\pgfpathcurveto{\pgfqpoint{1.274021in}{2.349264in}}{\pgfqpoint{1.277293in}{2.341364in}}{\pgfqpoint{1.283117in}{2.335540in}}%
\pgfpathcurveto{\pgfqpoint{1.288941in}{2.329717in}}{\pgfqpoint{1.296841in}{2.326444in}}{\pgfqpoint{1.305077in}{2.326444in}}%
\pgfpathclose%
\pgfusepath{stroke,fill}%
\end{pgfscope}%
\begin{pgfscope}%
\pgfpathrectangle{\pgfqpoint{0.100000in}{0.212622in}}{\pgfqpoint{3.696000in}{3.696000in}}%
\pgfusepath{clip}%
\pgfsetbuttcap%
\pgfsetroundjoin%
\definecolor{currentfill}{rgb}{0.121569,0.466667,0.705882}%
\pgfsetfillcolor{currentfill}%
\pgfsetfillopacity{0.483611}%
\pgfsetlinewidth{1.003750pt}%
\definecolor{currentstroke}{rgb}{0.121569,0.466667,0.705882}%
\pgfsetstrokecolor{currentstroke}%
\pgfsetstrokeopacity{0.483611}%
\pgfsetdash{}{0pt}%
\pgfpathmoveto{\pgfqpoint{2.029146in}{2.580755in}}%
\pgfpathcurveto{\pgfqpoint{2.037383in}{2.580755in}}{\pgfqpoint{2.045283in}{2.584028in}}{\pgfqpoint{2.051106in}{2.589851in}}%
\pgfpathcurveto{\pgfqpoint{2.056930in}{2.595675in}}{\pgfqpoint{2.060203in}{2.603575in}}{\pgfqpoint{2.060203in}{2.611812in}}%
\pgfpathcurveto{\pgfqpoint{2.060203in}{2.620048in}}{\pgfqpoint{2.056930in}{2.627948in}}{\pgfqpoint{2.051106in}{2.633772in}}%
\pgfpathcurveto{\pgfqpoint{2.045283in}{2.639596in}}{\pgfqpoint{2.037383in}{2.642868in}}{\pgfqpoint{2.029146in}{2.642868in}}%
\pgfpathcurveto{\pgfqpoint{2.020910in}{2.642868in}}{\pgfqpoint{2.013010in}{2.639596in}}{\pgfqpoint{2.007186in}{2.633772in}}%
\pgfpathcurveto{\pgfqpoint{2.001362in}{2.627948in}}{\pgfqpoint{1.998090in}{2.620048in}}{\pgfqpoint{1.998090in}{2.611812in}}%
\pgfpathcurveto{\pgfqpoint{1.998090in}{2.603575in}}{\pgfqpoint{2.001362in}{2.595675in}}{\pgfqpoint{2.007186in}{2.589851in}}%
\pgfpathcurveto{\pgfqpoint{2.013010in}{2.584028in}}{\pgfqpoint{2.020910in}{2.580755in}}{\pgfqpoint{2.029146in}{2.580755in}}%
\pgfpathclose%
\pgfusepath{stroke,fill}%
\end{pgfscope}%
\begin{pgfscope}%
\pgfpathrectangle{\pgfqpoint{0.100000in}{0.212622in}}{\pgfqpoint{3.696000in}{3.696000in}}%
\pgfusepath{clip}%
\pgfsetbuttcap%
\pgfsetroundjoin%
\definecolor{currentfill}{rgb}{0.121569,0.466667,0.705882}%
\pgfsetfillcolor{currentfill}%
\pgfsetfillopacity{0.484081}%
\pgfsetlinewidth{1.003750pt}%
\definecolor{currentstroke}{rgb}{0.121569,0.466667,0.705882}%
\pgfsetstrokecolor{currentstroke}%
\pgfsetstrokeopacity{0.484081}%
\pgfsetdash{}{0pt}%
\pgfpathmoveto{\pgfqpoint{1.303429in}{2.323437in}}%
\pgfpathcurveto{\pgfqpoint{1.311665in}{2.323437in}}{\pgfqpoint{1.319565in}{2.326710in}}{\pgfqpoint{1.325389in}{2.332534in}}%
\pgfpathcurveto{\pgfqpoint{1.331213in}{2.338357in}}{\pgfqpoint{1.334485in}{2.346258in}}{\pgfqpoint{1.334485in}{2.354494in}}%
\pgfpathcurveto{\pgfqpoint{1.334485in}{2.362730in}}{\pgfqpoint{1.331213in}{2.370630in}}{\pgfqpoint{1.325389in}{2.376454in}}%
\pgfpathcurveto{\pgfqpoint{1.319565in}{2.382278in}}{\pgfqpoint{1.311665in}{2.385550in}}{\pgfqpoint{1.303429in}{2.385550in}}%
\pgfpathcurveto{\pgfqpoint{1.295192in}{2.385550in}}{\pgfqpoint{1.287292in}{2.382278in}}{\pgfqpoint{1.281468in}{2.376454in}}%
\pgfpathcurveto{\pgfqpoint{1.275645in}{2.370630in}}{\pgfqpoint{1.272372in}{2.362730in}}{\pgfqpoint{1.272372in}{2.354494in}}%
\pgfpathcurveto{\pgfqpoint{1.272372in}{2.346258in}}{\pgfqpoint{1.275645in}{2.338357in}}{\pgfqpoint{1.281468in}{2.332534in}}%
\pgfpathcurveto{\pgfqpoint{1.287292in}{2.326710in}}{\pgfqpoint{1.295192in}{2.323437in}}{\pgfqpoint{1.303429in}{2.323437in}}%
\pgfpathclose%
\pgfusepath{stroke,fill}%
\end{pgfscope}%
\begin{pgfscope}%
\pgfpathrectangle{\pgfqpoint{0.100000in}{0.212622in}}{\pgfqpoint{3.696000in}{3.696000in}}%
\pgfusepath{clip}%
\pgfsetbuttcap%
\pgfsetroundjoin%
\definecolor{currentfill}{rgb}{0.121569,0.466667,0.705882}%
\pgfsetfillcolor{currentfill}%
\pgfsetfillopacity{0.484615}%
\pgfsetlinewidth{1.003750pt}%
\definecolor{currentstroke}{rgb}{0.121569,0.466667,0.705882}%
\pgfsetstrokecolor{currentstroke}%
\pgfsetstrokeopacity{0.484615}%
\pgfsetdash{}{0pt}%
\pgfpathmoveto{\pgfqpoint{2.029934in}{2.574772in}}%
\pgfpathcurveto{\pgfqpoint{2.038171in}{2.574772in}}{\pgfqpoint{2.046071in}{2.578044in}}{\pgfqpoint{2.051895in}{2.583868in}}%
\pgfpathcurveto{\pgfqpoint{2.057719in}{2.589692in}}{\pgfqpoint{2.060991in}{2.597592in}}{\pgfqpoint{2.060991in}{2.605828in}}%
\pgfpathcurveto{\pgfqpoint{2.060991in}{2.614065in}}{\pgfqpoint{2.057719in}{2.621965in}}{\pgfqpoint{2.051895in}{2.627788in}}%
\pgfpathcurveto{\pgfqpoint{2.046071in}{2.633612in}}{\pgfqpoint{2.038171in}{2.636885in}}{\pgfqpoint{2.029934in}{2.636885in}}%
\pgfpathcurveto{\pgfqpoint{2.021698in}{2.636885in}}{\pgfqpoint{2.013798in}{2.633612in}}{\pgfqpoint{2.007974in}{2.627788in}}%
\pgfpathcurveto{\pgfqpoint{2.002150in}{2.621965in}}{\pgfqpoint{1.998878in}{2.614065in}}{\pgfqpoint{1.998878in}{2.605828in}}%
\pgfpathcurveto{\pgfqpoint{1.998878in}{2.597592in}}{\pgfqpoint{2.002150in}{2.589692in}}{\pgfqpoint{2.007974in}{2.583868in}}%
\pgfpathcurveto{\pgfqpoint{2.013798in}{2.578044in}}{\pgfqpoint{2.021698in}{2.574772in}}{\pgfqpoint{2.029934in}{2.574772in}}%
\pgfpathclose%
\pgfusepath{stroke,fill}%
\end{pgfscope}%
\begin{pgfscope}%
\pgfpathrectangle{\pgfqpoint{0.100000in}{0.212622in}}{\pgfqpoint{3.696000in}{3.696000in}}%
\pgfusepath{clip}%
\pgfsetbuttcap%
\pgfsetroundjoin%
\definecolor{currentfill}{rgb}{0.121569,0.466667,0.705882}%
\pgfsetfillcolor{currentfill}%
\pgfsetfillopacity{0.485139}%
\pgfsetlinewidth{1.003750pt}%
\definecolor{currentstroke}{rgb}{0.121569,0.466667,0.705882}%
\pgfsetstrokecolor{currentstroke}%
\pgfsetstrokeopacity{0.485139}%
\pgfsetdash{}{0pt}%
\pgfpathmoveto{\pgfqpoint{1.300169in}{2.318096in}}%
\pgfpathcurveto{\pgfqpoint{1.308405in}{2.318096in}}{\pgfqpoint{1.316306in}{2.321369in}}{\pgfqpoint{1.322129in}{2.327193in}}%
\pgfpathcurveto{\pgfqpoint{1.327953in}{2.333017in}}{\pgfqpoint{1.331226in}{2.340917in}}{\pgfqpoint{1.331226in}{2.349153in}}%
\pgfpathcurveto{\pgfqpoint{1.331226in}{2.357389in}}{\pgfqpoint{1.327953in}{2.365289in}}{\pgfqpoint{1.322129in}{2.371113in}}%
\pgfpathcurveto{\pgfqpoint{1.316306in}{2.376937in}}{\pgfqpoint{1.308405in}{2.380209in}}{\pgfqpoint{1.300169in}{2.380209in}}%
\pgfpathcurveto{\pgfqpoint{1.291933in}{2.380209in}}{\pgfqpoint{1.284033in}{2.376937in}}{\pgfqpoint{1.278209in}{2.371113in}}%
\pgfpathcurveto{\pgfqpoint{1.272385in}{2.365289in}}{\pgfqpoint{1.269113in}{2.357389in}}{\pgfqpoint{1.269113in}{2.349153in}}%
\pgfpathcurveto{\pgfqpoint{1.269113in}{2.340917in}}{\pgfqpoint{1.272385in}{2.333017in}}{\pgfqpoint{1.278209in}{2.327193in}}%
\pgfpathcurveto{\pgfqpoint{1.284033in}{2.321369in}}{\pgfqpoint{1.291933in}{2.318096in}}{\pgfqpoint{1.300169in}{2.318096in}}%
\pgfpathclose%
\pgfusepath{stroke,fill}%
\end{pgfscope}%
\begin{pgfscope}%
\pgfpathrectangle{\pgfqpoint{0.100000in}{0.212622in}}{\pgfqpoint{3.696000in}{3.696000in}}%
\pgfusepath{clip}%
\pgfsetbuttcap%
\pgfsetroundjoin%
\definecolor{currentfill}{rgb}{0.121569,0.466667,0.705882}%
\pgfsetfillcolor{currentfill}%
\pgfsetfillopacity{0.485933}%
\pgfsetlinewidth{1.003750pt}%
\definecolor{currentstroke}{rgb}{0.121569,0.466667,0.705882}%
\pgfsetstrokecolor{currentstroke}%
\pgfsetstrokeopacity{0.485933}%
\pgfsetdash{}{0pt}%
\pgfpathmoveto{\pgfqpoint{2.030353in}{2.568699in}}%
\pgfpathcurveto{\pgfqpoint{2.038590in}{2.568699in}}{\pgfqpoint{2.046490in}{2.571971in}}{\pgfqpoint{2.052314in}{2.577795in}}%
\pgfpathcurveto{\pgfqpoint{2.058137in}{2.583619in}}{\pgfqpoint{2.061410in}{2.591519in}}{\pgfqpoint{2.061410in}{2.599755in}}%
\pgfpathcurveto{\pgfqpoint{2.061410in}{2.607992in}}{\pgfqpoint{2.058137in}{2.615892in}}{\pgfqpoint{2.052314in}{2.621716in}}%
\pgfpathcurveto{\pgfqpoint{2.046490in}{2.627539in}}{\pgfqpoint{2.038590in}{2.630812in}}{\pgfqpoint{2.030353in}{2.630812in}}%
\pgfpathcurveto{\pgfqpoint{2.022117in}{2.630812in}}{\pgfqpoint{2.014217in}{2.627539in}}{\pgfqpoint{2.008393in}{2.621716in}}%
\pgfpathcurveto{\pgfqpoint{2.002569in}{2.615892in}}{\pgfqpoint{1.999297in}{2.607992in}}{\pgfqpoint{1.999297in}{2.599755in}}%
\pgfpathcurveto{\pgfqpoint{1.999297in}{2.591519in}}{\pgfqpoint{2.002569in}{2.583619in}}{\pgfqpoint{2.008393in}{2.577795in}}%
\pgfpathcurveto{\pgfqpoint{2.014217in}{2.571971in}}{\pgfqpoint{2.022117in}{2.568699in}}{\pgfqpoint{2.030353in}{2.568699in}}%
\pgfpathclose%
\pgfusepath{stroke,fill}%
\end{pgfscope}%
\begin{pgfscope}%
\pgfpathrectangle{\pgfqpoint{0.100000in}{0.212622in}}{\pgfqpoint{3.696000in}{3.696000in}}%
\pgfusepath{clip}%
\pgfsetbuttcap%
\pgfsetroundjoin%
\definecolor{currentfill}{rgb}{0.121569,0.466667,0.705882}%
\pgfsetfillcolor{currentfill}%
\pgfsetfillopacity{0.486088}%
\pgfsetlinewidth{1.003750pt}%
\definecolor{currentstroke}{rgb}{0.121569,0.466667,0.705882}%
\pgfsetstrokecolor{currentstroke}%
\pgfsetstrokeopacity{0.486088}%
\pgfsetdash{}{0pt}%
\pgfpathmoveto{\pgfqpoint{1.297769in}{2.313359in}}%
\pgfpathcurveto{\pgfqpoint{1.306006in}{2.313359in}}{\pgfqpoint{1.313906in}{2.316631in}}{\pgfqpoint{1.319730in}{2.322455in}}%
\pgfpathcurveto{\pgfqpoint{1.325554in}{2.328279in}}{\pgfqpoint{1.328826in}{2.336179in}}{\pgfqpoint{1.328826in}{2.344415in}}%
\pgfpathcurveto{\pgfqpoint{1.328826in}{2.352652in}}{\pgfqpoint{1.325554in}{2.360552in}}{\pgfqpoint{1.319730in}{2.366375in}}%
\pgfpathcurveto{\pgfqpoint{1.313906in}{2.372199in}}{\pgfqpoint{1.306006in}{2.375472in}}{\pgfqpoint{1.297769in}{2.375472in}}%
\pgfpathcurveto{\pgfqpoint{1.289533in}{2.375472in}}{\pgfqpoint{1.281633in}{2.372199in}}{\pgfqpoint{1.275809in}{2.366375in}}%
\pgfpathcurveto{\pgfqpoint{1.269985in}{2.360552in}}{\pgfqpoint{1.266713in}{2.352652in}}{\pgfqpoint{1.266713in}{2.344415in}}%
\pgfpathcurveto{\pgfqpoint{1.266713in}{2.336179in}}{\pgfqpoint{1.269985in}{2.328279in}}{\pgfqpoint{1.275809in}{2.322455in}}%
\pgfpathcurveto{\pgfqpoint{1.281633in}{2.316631in}}{\pgfqpoint{1.289533in}{2.313359in}}{\pgfqpoint{1.297769in}{2.313359in}}%
\pgfpathclose%
\pgfusepath{stroke,fill}%
\end{pgfscope}%
\begin{pgfscope}%
\pgfpathrectangle{\pgfqpoint{0.100000in}{0.212622in}}{\pgfqpoint{3.696000in}{3.696000in}}%
\pgfusepath{clip}%
\pgfsetbuttcap%
\pgfsetroundjoin%
\definecolor{currentfill}{rgb}{0.121569,0.466667,0.705882}%
\pgfsetfillcolor{currentfill}%
\pgfsetfillopacity{0.486692}%
\pgfsetlinewidth{1.003750pt}%
\definecolor{currentstroke}{rgb}{0.121569,0.466667,0.705882}%
\pgfsetstrokecolor{currentstroke}%
\pgfsetstrokeopacity{0.486692}%
\pgfsetdash{}{0pt}%
\pgfpathmoveto{\pgfqpoint{1.295710in}{2.309977in}}%
\pgfpathcurveto{\pgfqpoint{1.303946in}{2.309977in}}{\pgfqpoint{1.311846in}{2.313249in}}{\pgfqpoint{1.317670in}{2.319073in}}%
\pgfpathcurveto{\pgfqpoint{1.323494in}{2.324897in}}{\pgfqpoint{1.326766in}{2.332797in}}{\pgfqpoint{1.326766in}{2.341034in}}%
\pgfpathcurveto{\pgfqpoint{1.326766in}{2.349270in}}{\pgfqpoint{1.323494in}{2.357170in}}{\pgfqpoint{1.317670in}{2.362994in}}%
\pgfpathcurveto{\pgfqpoint{1.311846in}{2.368818in}}{\pgfqpoint{1.303946in}{2.372090in}}{\pgfqpoint{1.295710in}{2.372090in}}%
\pgfpathcurveto{\pgfqpoint{1.287474in}{2.372090in}}{\pgfqpoint{1.279574in}{2.368818in}}{\pgfqpoint{1.273750in}{2.362994in}}%
\pgfpathcurveto{\pgfqpoint{1.267926in}{2.357170in}}{\pgfqpoint{1.264653in}{2.349270in}}{\pgfqpoint{1.264653in}{2.341034in}}%
\pgfpathcurveto{\pgfqpoint{1.264653in}{2.332797in}}{\pgfqpoint{1.267926in}{2.324897in}}{\pgfqpoint{1.273750in}{2.319073in}}%
\pgfpathcurveto{\pgfqpoint{1.279574in}{2.313249in}}{\pgfqpoint{1.287474in}{2.309977in}}{\pgfqpoint{1.295710in}{2.309977in}}%
\pgfpathclose%
\pgfusepath{stroke,fill}%
\end{pgfscope}%
\begin{pgfscope}%
\pgfpathrectangle{\pgfqpoint{0.100000in}{0.212622in}}{\pgfqpoint{3.696000in}{3.696000in}}%
\pgfusepath{clip}%
\pgfsetbuttcap%
\pgfsetroundjoin%
\definecolor{currentfill}{rgb}{0.121569,0.466667,0.705882}%
\pgfsetfillcolor{currentfill}%
\pgfsetfillopacity{0.487105}%
\pgfsetlinewidth{1.003750pt}%
\definecolor{currentstroke}{rgb}{0.121569,0.466667,0.705882}%
\pgfsetstrokecolor{currentstroke}%
\pgfsetstrokeopacity{0.487105}%
\pgfsetdash{}{0pt}%
\pgfpathmoveto{\pgfqpoint{1.294531in}{2.307564in}}%
\pgfpathcurveto{\pgfqpoint{1.302768in}{2.307564in}}{\pgfqpoint{1.310668in}{2.310836in}}{\pgfqpoint{1.316492in}{2.316660in}}%
\pgfpathcurveto{\pgfqpoint{1.322316in}{2.322484in}}{\pgfqpoint{1.325588in}{2.330384in}}{\pgfqpoint{1.325588in}{2.338621in}}%
\pgfpathcurveto{\pgfqpoint{1.325588in}{2.346857in}}{\pgfqpoint{1.322316in}{2.354757in}}{\pgfqpoint{1.316492in}{2.360581in}}%
\pgfpathcurveto{\pgfqpoint{1.310668in}{2.366405in}}{\pgfqpoint{1.302768in}{2.369677in}}{\pgfqpoint{1.294531in}{2.369677in}}%
\pgfpathcurveto{\pgfqpoint{1.286295in}{2.369677in}}{\pgfqpoint{1.278395in}{2.366405in}}{\pgfqpoint{1.272571in}{2.360581in}}%
\pgfpathcurveto{\pgfqpoint{1.266747in}{2.354757in}}{\pgfqpoint{1.263475in}{2.346857in}}{\pgfqpoint{1.263475in}{2.338621in}}%
\pgfpathcurveto{\pgfqpoint{1.263475in}{2.330384in}}{\pgfqpoint{1.266747in}{2.322484in}}{\pgfqpoint{1.272571in}{2.316660in}}%
\pgfpathcurveto{\pgfqpoint{1.278395in}{2.310836in}}{\pgfqpoint{1.286295in}{2.307564in}}{\pgfqpoint{1.294531in}{2.307564in}}%
\pgfpathclose%
\pgfusepath{stroke,fill}%
\end{pgfscope}%
\begin{pgfscope}%
\pgfpathrectangle{\pgfqpoint{0.100000in}{0.212622in}}{\pgfqpoint{3.696000in}{3.696000in}}%
\pgfusepath{clip}%
\pgfsetbuttcap%
\pgfsetroundjoin%
\definecolor{currentfill}{rgb}{0.121569,0.466667,0.705882}%
\pgfsetfillcolor{currentfill}%
\pgfsetfillopacity{0.487127}%
\pgfsetlinewidth{1.003750pt}%
\definecolor{currentstroke}{rgb}{0.121569,0.466667,0.705882}%
\pgfsetstrokecolor{currentstroke}%
\pgfsetstrokeopacity{0.487127}%
\pgfsetdash{}{0pt}%
\pgfpathmoveto{\pgfqpoint{1.294458in}{2.307429in}}%
\pgfpathcurveto{\pgfqpoint{1.302694in}{2.307429in}}{\pgfqpoint{1.310594in}{2.310701in}}{\pgfqpoint{1.316418in}{2.316525in}}%
\pgfpathcurveto{\pgfqpoint{1.322242in}{2.322349in}}{\pgfqpoint{1.325514in}{2.330249in}}{\pgfqpoint{1.325514in}{2.338485in}}%
\pgfpathcurveto{\pgfqpoint{1.325514in}{2.346722in}}{\pgfqpoint{1.322242in}{2.354622in}}{\pgfqpoint{1.316418in}{2.360446in}}%
\pgfpathcurveto{\pgfqpoint{1.310594in}{2.366270in}}{\pgfqpoint{1.302694in}{2.369542in}}{\pgfqpoint{1.294458in}{2.369542in}}%
\pgfpathcurveto{\pgfqpoint{1.286221in}{2.369542in}}{\pgfqpoint{1.278321in}{2.366270in}}{\pgfqpoint{1.272497in}{2.360446in}}%
\pgfpathcurveto{\pgfqpoint{1.266673in}{2.354622in}}{\pgfqpoint{1.263401in}{2.346722in}}{\pgfqpoint{1.263401in}{2.338485in}}%
\pgfpathcurveto{\pgfqpoint{1.263401in}{2.330249in}}{\pgfqpoint{1.266673in}{2.322349in}}{\pgfqpoint{1.272497in}{2.316525in}}%
\pgfpathcurveto{\pgfqpoint{1.278321in}{2.310701in}}{\pgfqpoint{1.286221in}{2.307429in}}{\pgfqpoint{1.294458in}{2.307429in}}%
\pgfpathclose%
\pgfusepath{stroke,fill}%
\end{pgfscope}%
\begin{pgfscope}%
\pgfpathrectangle{\pgfqpoint{0.100000in}{0.212622in}}{\pgfqpoint{3.696000in}{3.696000in}}%
\pgfusepath{clip}%
\pgfsetbuttcap%
\pgfsetroundjoin%
\definecolor{currentfill}{rgb}{0.121569,0.466667,0.705882}%
\pgfsetfillcolor{currentfill}%
\pgfsetfillopacity{0.487170}%
\pgfsetlinewidth{1.003750pt}%
\definecolor{currentstroke}{rgb}{0.121569,0.466667,0.705882}%
\pgfsetstrokecolor{currentstroke}%
\pgfsetstrokeopacity{0.487170}%
\pgfsetdash{}{0pt}%
\pgfpathmoveto{\pgfqpoint{1.294341in}{2.307169in}}%
\pgfpathcurveto{\pgfqpoint{1.302577in}{2.307169in}}{\pgfqpoint{1.310477in}{2.310442in}}{\pgfqpoint{1.316301in}{2.316266in}}%
\pgfpathcurveto{\pgfqpoint{1.322125in}{2.322089in}}{\pgfqpoint{1.325397in}{2.329990in}}{\pgfqpoint{1.325397in}{2.338226in}}%
\pgfpathcurveto{\pgfqpoint{1.325397in}{2.346462in}}{\pgfqpoint{1.322125in}{2.354362in}}{\pgfqpoint{1.316301in}{2.360186in}}%
\pgfpathcurveto{\pgfqpoint{1.310477in}{2.366010in}}{\pgfqpoint{1.302577in}{2.369282in}}{\pgfqpoint{1.294341in}{2.369282in}}%
\pgfpathcurveto{\pgfqpoint{1.286104in}{2.369282in}}{\pgfqpoint{1.278204in}{2.366010in}}{\pgfqpoint{1.272380in}{2.360186in}}%
\pgfpathcurveto{\pgfqpoint{1.266556in}{2.354362in}}{\pgfqpoint{1.263284in}{2.346462in}}{\pgfqpoint{1.263284in}{2.338226in}}%
\pgfpathcurveto{\pgfqpoint{1.263284in}{2.329990in}}{\pgfqpoint{1.266556in}{2.322089in}}{\pgfqpoint{1.272380in}{2.316266in}}%
\pgfpathcurveto{\pgfqpoint{1.278204in}{2.310442in}}{\pgfqpoint{1.286104in}{2.307169in}}{\pgfqpoint{1.294341in}{2.307169in}}%
\pgfpathclose%
\pgfusepath{stroke,fill}%
\end{pgfscope}%
\begin{pgfscope}%
\pgfpathrectangle{\pgfqpoint{0.100000in}{0.212622in}}{\pgfqpoint{3.696000in}{3.696000in}}%
\pgfusepath{clip}%
\pgfsetbuttcap%
\pgfsetroundjoin%
\definecolor{currentfill}{rgb}{0.121569,0.466667,0.705882}%
\pgfsetfillcolor{currentfill}%
\pgfsetfillopacity{0.487254}%
\pgfsetlinewidth{1.003750pt}%
\definecolor{currentstroke}{rgb}{0.121569,0.466667,0.705882}%
\pgfsetstrokecolor{currentstroke}%
\pgfsetstrokeopacity{0.487254}%
\pgfsetdash{}{0pt}%
\pgfpathmoveto{\pgfqpoint{1.294097in}{2.306757in}}%
\pgfpathcurveto{\pgfqpoint{1.302333in}{2.306757in}}{\pgfqpoint{1.310233in}{2.310029in}}{\pgfqpoint{1.316057in}{2.315853in}}%
\pgfpathcurveto{\pgfqpoint{1.321881in}{2.321677in}}{\pgfqpoint{1.325153in}{2.329577in}}{\pgfqpoint{1.325153in}{2.337813in}}%
\pgfpathcurveto{\pgfqpoint{1.325153in}{2.346050in}}{\pgfqpoint{1.321881in}{2.353950in}}{\pgfqpoint{1.316057in}{2.359774in}}%
\pgfpathcurveto{\pgfqpoint{1.310233in}{2.365598in}}{\pgfqpoint{1.302333in}{2.368870in}}{\pgfqpoint{1.294097in}{2.368870in}}%
\pgfpathcurveto{\pgfqpoint{1.285861in}{2.368870in}}{\pgfqpoint{1.277961in}{2.365598in}}{\pgfqpoint{1.272137in}{2.359774in}}%
\pgfpathcurveto{\pgfqpoint{1.266313in}{2.353950in}}{\pgfqpoint{1.263040in}{2.346050in}}{\pgfqpoint{1.263040in}{2.337813in}}%
\pgfpathcurveto{\pgfqpoint{1.263040in}{2.329577in}}{\pgfqpoint{1.266313in}{2.321677in}}{\pgfqpoint{1.272137in}{2.315853in}}%
\pgfpathcurveto{\pgfqpoint{1.277961in}{2.310029in}}{\pgfqpoint{1.285861in}{2.306757in}}{\pgfqpoint{1.294097in}{2.306757in}}%
\pgfpathclose%
\pgfusepath{stroke,fill}%
\end{pgfscope}%
\begin{pgfscope}%
\pgfpathrectangle{\pgfqpoint{0.100000in}{0.212622in}}{\pgfqpoint{3.696000in}{3.696000in}}%
\pgfusepath{clip}%
\pgfsetbuttcap%
\pgfsetroundjoin%
\definecolor{currentfill}{rgb}{0.121569,0.466667,0.705882}%
\pgfsetfillcolor{currentfill}%
\pgfsetfillopacity{0.487369}%
\pgfsetlinewidth{1.003750pt}%
\definecolor{currentstroke}{rgb}{0.121569,0.466667,0.705882}%
\pgfsetstrokecolor{currentstroke}%
\pgfsetstrokeopacity{0.487369}%
\pgfsetdash{}{0pt}%
\pgfpathmoveto{\pgfqpoint{2.031208in}{2.561755in}}%
\pgfpathcurveto{\pgfqpoint{2.039444in}{2.561755in}}{\pgfqpoint{2.047344in}{2.565027in}}{\pgfqpoint{2.053168in}{2.570851in}}%
\pgfpathcurveto{\pgfqpoint{2.058992in}{2.576675in}}{\pgfqpoint{2.062264in}{2.584575in}}{\pgfqpoint{2.062264in}{2.592811in}}%
\pgfpathcurveto{\pgfqpoint{2.062264in}{2.601047in}}{\pgfqpoint{2.058992in}{2.608947in}}{\pgfqpoint{2.053168in}{2.614771in}}%
\pgfpathcurveto{\pgfqpoint{2.047344in}{2.620595in}}{\pgfqpoint{2.039444in}{2.623868in}}{\pgfqpoint{2.031208in}{2.623868in}}%
\pgfpathcurveto{\pgfqpoint{2.022971in}{2.623868in}}{\pgfqpoint{2.015071in}{2.620595in}}{\pgfqpoint{2.009247in}{2.614771in}}%
\pgfpathcurveto{\pgfqpoint{2.003423in}{2.608947in}}{\pgfqpoint{2.000151in}{2.601047in}}{\pgfqpoint{2.000151in}{2.592811in}}%
\pgfpathcurveto{\pgfqpoint{2.000151in}{2.584575in}}{\pgfqpoint{2.003423in}{2.576675in}}{\pgfqpoint{2.009247in}{2.570851in}}%
\pgfpathcurveto{\pgfqpoint{2.015071in}{2.565027in}}{\pgfqpoint{2.022971in}{2.561755in}}{\pgfqpoint{2.031208in}{2.561755in}}%
\pgfpathclose%
\pgfusepath{stroke,fill}%
\end{pgfscope}%
\begin{pgfscope}%
\pgfpathrectangle{\pgfqpoint{0.100000in}{0.212622in}}{\pgfqpoint{3.696000in}{3.696000in}}%
\pgfusepath{clip}%
\pgfsetbuttcap%
\pgfsetroundjoin%
\definecolor{currentfill}{rgb}{0.121569,0.466667,0.705882}%
\pgfsetfillcolor{currentfill}%
\pgfsetfillopacity{0.487419}%
\pgfsetlinewidth{1.003750pt}%
\definecolor{currentstroke}{rgb}{0.121569,0.466667,0.705882}%
\pgfsetstrokecolor{currentstroke}%
\pgfsetstrokeopacity{0.487419}%
\pgfsetdash{}{0pt}%
\pgfpathmoveto{\pgfqpoint{1.293685in}{2.306015in}}%
\pgfpathcurveto{\pgfqpoint{1.301921in}{2.306015in}}{\pgfqpoint{1.309821in}{2.309287in}}{\pgfqpoint{1.315645in}{2.315111in}}%
\pgfpathcurveto{\pgfqpoint{1.321469in}{2.320935in}}{\pgfqpoint{1.324741in}{2.328835in}}{\pgfqpoint{1.324741in}{2.337071in}}%
\pgfpathcurveto{\pgfqpoint{1.324741in}{2.345307in}}{\pgfqpoint{1.321469in}{2.353208in}}{\pgfqpoint{1.315645in}{2.359031in}}%
\pgfpathcurveto{\pgfqpoint{1.309821in}{2.364855in}}{\pgfqpoint{1.301921in}{2.368128in}}{\pgfqpoint{1.293685in}{2.368128in}}%
\pgfpathcurveto{\pgfqpoint{1.285448in}{2.368128in}}{\pgfqpoint{1.277548in}{2.364855in}}{\pgfqpoint{1.271724in}{2.359031in}}%
\pgfpathcurveto{\pgfqpoint{1.265901in}{2.353208in}}{\pgfqpoint{1.262628in}{2.345307in}}{\pgfqpoint{1.262628in}{2.337071in}}%
\pgfpathcurveto{\pgfqpoint{1.262628in}{2.328835in}}{\pgfqpoint{1.265901in}{2.320935in}}{\pgfqpoint{1.271724in}{2.315111in}}%
\pgfpathcurveto{\pgfqpoint{1.277548in}{2.309287in}}{\pgfqpoint{1.285448in}{2.306015in}}{\pgfqpoint{1.293685in}{2.306015in}}%
\pgfpathclose%
\pgfusepath{stroke,fill}%
\end{pgfscope}%
\begin{pgfscope}%
\pgfpathrectangle{\pgfqpoint{0.100000in}{0.212622in}}{\pgfqpoint{3.696000in}{3.696000in}}%
\pgfusepath{clip}%
\pgfsetbuttcap%
\pgfsetroundjoin%
\definecolor{currentfill}{rgb}{0.121569,0.466667,0.705882}%
\pgfsetfillcolor{currentfill}%
\pgfsetfillopacity{0.487697}%
\pgfsetlinewidth{1.003750pt}%
\definecolor{currentstroke}{rgb}{0.121569,0.466667,0.705882}%
\pgfsetstrokecolor{currentstroke}%
\pgfsetstrokeopacity{0.487697}%
\pgfsetdash{}{0pt}%
\pgfpathmoveto{\pgfqpoint{1.292865in}{2.304673in}}%
\pgfpathcurveto{\pgfqpoint{1.301101in}{2.304673in}}{\pgfqpoint{1.309001in}{2.307946in}}{\pgfqpoint{1.314825in}{2.313770in}}%
\pgfpathcurveto{\pgfqpoint{1.320649in}{2.319594in}}{\pgfqpoint{1.323921in}{2.327494in}}{\pgfqpoint{1.323921in}{2.335730in}}%
\pgfpathcurveto{\pgfqpoint{1.323921in}{2.343966in}}{\pgfqpoint{1.320649in}{2.351866in}}{\pgfqpoint{1.314825in}{2.357690in}}%
\pgfpathcurveto{\pgfqpoint{1.309001in}{2.363514in}}{\pgfqpoint{1.301101in}{2.366786in}}{\pgfqpoint{1.292865in}{2.366786in}}%
\pgfpathcurveto{\pgfqpoint{1.284628in}{2.366786in}}{\pgfqpoint{1.276728in}{2.363514in}}{\pgfqpoint{1.270904in}{2.357690in}}%
\pgfpathcurveto{\pgfqpoint{1.265080in}{2.351866in}}{\pgfqpoint{1.261808in}{2.343966in}}{\pgfqpoint{1.261808in}{2.335730in}}%
\pgfpathcurveto{\pgfqpoint{1.261808in}{2.327494in}}{\pgfqpoint{1.265080in}{2.319594in}}{\pgfqpoint{1.270904in}{2.313770in}}%
\pgfpathcurveto{\pgfqpoint{1.276728in}{2.307946in}}{\pgfqpoint{1.284628in}{2.304673in}}{\pgfqpoint{1.292865in}{2.304673in}}%
\pgfpathclose%
\pgfusepath{stroke,fill}%
\end{pgfscope}%
\begin{pgfscope}%
\pgfpathrectangle{\pgfqpoint{0.100000in}{0.212622in}}{\pgfqpoint{3.696000in}{3.696000in}}%
\pgfusepath{clip}%
\pgfsetbuttcap%
\pgfsetroundjoin%
\definecolor{currentfill}{rgb}{0.121569,0.466667,0.705882}%
\pgfsetfillcolor{currentfill}%
\pgfsetfillopacity{0.488212}%
\pgfsetlinewidth{1.003750pt}%
\definecolor{currentstroke}{rgb}{0.121569,0.466667,0.705882}%
\pgfsetstrokecolor{currentstroke}%
\pgfsetstrokeopacity{0.488212}%
\pgfsetdash{}{0pt}%
\pgfpathmoveto{\pgfqpoint{1.291474in}{2.302132in}}%
\pgfpathcurveto{\pgfqpoint{1.299710in}{2.302132in}}{\pgfqpoint{1.307610in}{2.305404in}}{\pgfqpoint{1.313434in}{2.311228in}}%
\pgfpathcurveto{\pgfqpoint{1.319258in}{2.317052in}}{\pgfqpoint{1.322530in}{2.324952in}}{\pgfqpoint{1.322530in}{2.333188in}}%
\pgfpathcurveto{\pgfqpoint{1.322530in}{2.341425in}}{\pgfqpoint{1.319258in}{2.349325in}}{\pgfqpoint{1.313434in}{2.355149in}}%
\pgfpathcurveto{\pgfqpoint{1.307610in}{2.360973in}}{\pgfqpoint{1.299710in}{2.364245in}}{\pgfqpoint{1.291474in}{2.364245in}}%
\pgfpathcurveto{\pgfqpoint{1.283238in}{2.364245in}}{\pgfqpoint{1.275338in}{2.360973in}}{\pgfqpoint{1.269514in}{2.355149in}}%
\pgfpathcurveto{\pgfqpoint{1.263690in}{2.349325in}}{\pgfqpoint{1.260417in}{2.341425in}}{\pgfqpoint{1.260417in}{2.333188in}}%
\pgfpathcurveto{\pgfqpoint{1.260417in}{2.324952in}}{\pgfqpoint{1.263690in}{2.317052in}}{\pgfqpoint{1.269514in}{2.311228in}}%
\pgfpathcurveto{\pgfqpoint{1.275338in}{2.305404in}}{\pgfqpoint{1.283238in}{2.302132in}}{\pgfqpoint{1.291474in}{2.302132in}}%
\pgfpathclose%
\pgfusepath{stroke,fill}%
\end{pgfscope}%
\begin{pgfscope}%
\pgfpathrectangle{\pgfqpoint{0.100000in}{0.212622in}}{\pgfqpoint{3.696000in}{3.696000in}}%
\pgfusepath{clip}%
\pgfsetbuttcap%
\pgfsetroundjoin%
\definecolor{currentfill}{rgb}{0.121569,0.466667,0.705882}%
\pgfsetfillcolor{currentfill}%
\pgfsetfillopacity{0.488597}%
\pgfsetlinewidth{1.003750pt}%
\definecolor{currentstroke}{rgb}{0.121569,0.466667,0.705882}%
\pgfsetstrokecolor{currentstroke}%
\pgfsetstrokeopacity{0.488597}%
\pgfsetdash{}{0pt}%
\pgfpathmoveto{\pgfqpoint{1.290440in}{2.300188in}}%
\pgfpathcurveto{\pgfqpoint{1.298676in}{2.300188in}}{\pgfqpoint{1.306576in}{2.303461in}}{\pgfqpoint{1.312400in}{2.309285in}}%
\pgfpathcurveto{\pgfqpoint{1.318224in}{2.315109in}}{\pgfqpoint{1.321496in}{2.323009in}}{\pgfqpoint{1.321496in}{2.331245in}}%
\pgfpathcurveto{\pgfqpoint{1.321496in}{2.339481in}}{\pgfqpoint{1.318224in}{2.347381in}}{\pgfqpoint{1.312400in}{2.353205in}}%
\pgfpathcurveto{\pgfqpoint{1.306576in}{2.359029in}}{\pgfqpoint{1.298676in}{2.362301in}}{\pgfqpoint{1.290440in}{2.362301in}}%
\pgfpathcurveto{\pgfqpoint{1.282204in}{2.362301in}}{\pgfqpoint{1.274304in}{2.359029in}}{\pgfqpoint{1.268480in}{2.353205in}}%
\pgfpathcurveto{\pgfqpoint{1.262656in}{2.347381in}}{\pgfqpoint{1.259383in}{2.339481in}}{\pgfqpoint{1.259383in}{2.331245in}}%
\pgfpathcurveto{\pgfqpoint{1.259383in}{2.323009in}}{\pgfqpoint{1.262656in}{2.315109in}}{\pgfqpoint{1.268480in}{2.309285in}}%
\pgfpathcurveto{\pgfqpoint{1.274304in}{2.303461in}}{\pgfqpoint{1.282204in}{2.300188in}}{\pgfqpoint{1.290440in}{2.300188in}}%
\pgfpathclose%
\pgfusepath{stroke,fill}%
\end{pgfscope}%
\begin{pgfscope}%
\pgfpathrectangle{\pgfqpoint{0.100000in}{0.212622in}}{\pgfqpoint{3.696000in}{3.696000in}}%
\pgfusepath{clip}%
\pgfsetbuttcap%
\pgfsetroundjoin%
\definecolor{currentfill}{rgb}{0.121569,0.466667,0.705882}%
\pgfsetfillcolor{currentfill}%
\pgfsetfillopacity{0.488841}%
\pgfsetlinewidth{1.003750pt}%
\definecolor{currentstroke}{rgb}{0.121569,0.466667,0.705882}%
\pgfsetstrokecolor{currentstroke}%
\pgfsetstrokeopacity{0.488841}%
\pgfsetdash{}{0pt}%
\pgfpathmoveto{\pgfqpoint{1.289673in}{2.298893in}}%
\pgfpathcurveto{\pgfqpoint{1.297909in}{2.298893in}}{\pgfqpoint{1.305809in}{2.302166in}}{\pgfqpoint{1.311633in}{2.307990in}}%
\pgfpathcurveto{\pgfqpoint{1.317457in}{2.313814in}}{\pgfqpoint{1.320729in}{2.321714in}}{\pgfqpoint{1.320729in}{2.329950in}}%
\pgfpathcurveto{\pgfqpoint{1.320729in}{2.338186in}}{\pgfqpoint{1.317457in}{2.346086in}}{\pgfqpoint{1.311633in}{2.351910in}}%
\pgfpathcurveto{\pgfqpoint{1.305809in}{2.357734in}}{\pgfqpoint{1.297909in}{2.361006in}}{\pgfqpoint{1.289673in}{2.361006in}}%
\pgfpathcurveto{\pgfqpoint{1.281436in}{2.361006in}}{\pgfqpoint{1.273536in}{2.357734in}}{\pgfqpoint{1.267712in}{2.351910in}}%
\pgfpathcurveto{\pgfqpoint{1.261888in}{2.346086in}}{\pgfqpoint{1.258616in}{2.338186in}}{\pgfqpoint{1.258616in}{2.329950in}}%
\pgfpathcurveto{\pgfqpoint{1.258616in}{2.321714in}}{\pgfqpoint{1.261888in}{2.313814in}}{\pgfqpoint{1.267712in}{2.307990in}}%
\pgfpathcurveto{\pgfqpoint{1.273536in}{2.302166in}}{\pgfqpoint{1.281436in}{2.298893in}}{\pgfqpoint{1.289673in}{2.298893in}}%
\pgfpathclose%
\pgfusepath{stroke,fill}%
\end{pgfscope}%
\begin{pgfscope}%
\pgfpathrectangle{\pgfqpoint{0.100000in}{0.212622in}}{\pgfqpoint{3.696000in}{3.696000in}}%
\pgfusepath{clip}%
\pgfsetbuttcap%
\pgfsetroundjoin%
\definecolor{currentfill}{rgb}{0.121569,0.466667,0.705882}%
\pgfsetfillcolor{currentfill}%
\pgfsetfillopacity{0.488841}%
\pgfsetlinewidth{1.003750pt}%
\definecolor{currentstroke}{rgb}{0.121569,0.466667,0.705882}%
\pgfsetstrokecolor{currentstroke}%
\pgfsetstrokeopacity{0.488841}%
\pgfsetdash{}{0pt}%
\pgfpathmoveto{\pgfqpoint{2.032398in}{2.553712in}}%
\pgfpathcurveto{\pgfqpoint{2.040635in}{2.553712in}}{\pgfqpoint{2.048535in}{2.556984in}}{\pgfqpoint{2.054359in}{2.562808in}}%
\pgfpathcurveto{\pgfqpoint{2.060183in}{2.568632in}}{\pgfqpoint{2.063455in}{2.576532in}}{\pgfqpoint{2.063455in}{2.584768in}}%
\pgfpathcurveto{\pgfqpoint{2.063455in}{2.593005in}}{\pgfqpoint{2.060183in}{2.600905in}}{\pgfqpoint{2.054359in}{2.606729in}}%
\pgfpathcurveto{\pgfqpoint{2.048535in}{2.612552in}}{\pgfqpoint{2.040635in}{2.615825in}}{\pgfqpoint{2.032398in}{2.615825in}}%
\pgfpathcurveto{\pgfqpoint{2.024162in}{2.615825in}}{\pgfqpoint{2.016262in}{2.612552in}}{\pgfqpoint{2.010438in}{2.606729in}}%
\pgfpathcurveto{\pgfqpoint{2.004614in}{2.600905in}}{\pgfqpoint{2.001342in}{2.593005in}}{\pgfqpoint{2.001342in}{2.584768in}}%
\pgfpathcurveto{\pgfqpoint{2.001342in}{2.576532in}}{\pgfqpoint{2.004614in}{2.568632in}}{\pgfqpoint{2.010438in}{2.562808in}}%
\pgfpathcurveto{\pgfqpoint{2.016262in}{2.556984in}}{\pgfqpoint{2.024162in}{2.553712in}}{\pgfqpoint{2.032398in}{2.553712in}}%
\pgfpathclose%
\pgfusepath{stroke,fill}%
\end{pgfscope}%
\begin{pgfscope}%
\pgfpathrectangle{\pgfqpoint{0.100000in}{0.212622in}}{\pgfqpoint{3.696000in}{3.696000in}}%
\pgfusepath{clip}%
\pgfsetbuttcap%
\pgfsetroundjoin%
\definecolor{currentfill}{rgb}{0.121569,0.466667,0.705882}%
\pgfsetfillcolor{currentfill}%
\pgfsetfillopacity{0.489287}%
\pgfsetlinewidth{1.003750pt}%
\definecolor{currentstroke}{rgb}{0.121569,0.466667,0.705882}%
\pgfsetstrokecolor{currentstroke}%
\pgfsetstrokeopacity{0.489287}%
\pgfsetdash{}{0pt}%
\pgfpathmoveto{\pgfqpoint{1.288423in}{2.296362in}}%
\pgfpathcurveto{\pgfqpoint{1.296660in}{2.296362in}}{\pgfqpoint{1.304560in}{2.299635in}}{\pgfqpoint{1.310384in}{2.305458in}}%
\pgfpathcurveto{\pgfqpoint{1.316207in}{2.311282in}}{\pgfqpoint{1.319480in}{2.319182in}}{\pgfqpoint{1.319480in}{2.327419in}}%
\pgfpathcurveto{\pgfqpoint{1.319480in}{2.335655in}}{\pgfqpoint{1.316207in}{2.343555in}}{\pgfqpoint{1.310384in}{2.349379in}}%
\pgfpathcurveto{\pgfqpoint{1.304560in}{2.355203in}}{\pgfqpoint{1.296660in}{2.358475in}}{\pgfqpoint{1.288423in}{2.358475in}}%
\pgfpathcurveto{\pgfqpoint{1.280187in}{2.358475in}}{\pgfqpoint{1.272287in}{2.355203in}}{\pgfqpoint{1.266463in}{2.349379in}}%
\pgfpathcurveto{\pgfqpoint{1.260639in}{2.343555in}}{\pgfqpoint{1.257367in}{2.335655in}}{\pgfqpoint{1.257367in}{2.327419in}}%
\pgfpathcurveto{\pgfqpoint{1.257367in}{2.319182in}}{\pgfqpoint{1.260639in}{2.311282in}}{\pgfqpoint{1.266463in}{2.305458in}}%
\pgfpathcurveto{\pgfqpoint{1.272287in}{2.299635in}}{\pgfqpoint{1.280187in}{2.296362in}}{\pgfqpoint{1.288423in}{2.296362in}}%
\pgfpathclose%
\pgfusepath{stroke,fill}%
\end{pgfscope}%
\begin{pgfscope}%
\pgfpathrectangle{\pgfqpoint{0.100000in}{0.212622in}}{\pgfqpoint{3.696000in}{3.696000in}}%
\pgfusepath{clip}%
\pgfsetbuttcap%
\pgfsetroundjoin%
\definecolor{currentfill}{rgb}{0.121569,0.466667,0.705882}%
\pgfsetfillcolor{currentfill}%
\pgfsetfillopacity{0.490051}%
\pgfsetlinewidth{1.003750pt}%
\definecolor{currentstroke}{rgb}{0.121569,0.466667,0.705882}%
\pgfsetstrokecolor{currentstroke}%
\pgfsetstrokeopacity{0.490051}%
\pgfsetdash{}{0pt}%
\pgfpathmoveto{\pgfqpoint{1.285853in}{2.291980in}}%
\pgfpathcurveto{\pgfqpoint{1.294089in}{2.291980in}}{\pgfqpoint{1.301989in}{2.295252in}}{\pgfqpoint{1.307813in}{2.301076in}}%
\pgfpathcurveto{\pgfqpoint{1.313637in}{2.306900in}}{\pgfqpoint{1.316909in}{2.314800in}}{\pgfqpoint{1.316909in}{2.323036in}}%
\pgfpathcurveto{\pgfqpoint{1.316909in}{2.331273in}}{\pgfqpoint{1.313637in}{2.339173in}}{\pgfqpoint{1.307813in}{2.344996in}}%
\pgfpathcurveto{\pgfqpoint{1.301989in}{2.350820in}}{\pgfqpoint{1.294089in}{2.354093in}}{\pgfqpoint{1.285853in}{2.354093in}}%
\pgfpathcurveto{\pgfqpoint{1.277617in}{2.354093in}}{\pgfqpoint{1.269716in}{2.350820in}}{\pgfqpoint{1.263893in}{2.344996in}}%
\pgfpathcurveto{\pgfqpoint{1.258069in}{2.339173in}}{\pgfqpoint{1.254796in}{2.331273in}}{\pgfqpoint{1.254796in}{2.323036in}}%
\pgfpathcurveto{\pgfqpoint{1.254796in}{2.314800in}}{\pgfqpoint{1.258069in}{2.306900in}}{\pgfqpoint{1.263893in}{2.301076in}}%
\pgfpathcurveto{\pgfqpoint{1.269716in}{2.295252in}}{\pgfqpoint{1.277617in}{2.291980in}}{\pgfqpoint{1.285853in}{2.291980in}}%
\pgfpathclose%
\pgfusepath{stroke,fill}%
\end{pgfscope}%
\begin{pgfscope}%
\pgfpathrectangle{\pgfqpoint{0.100000in}{0.212622in}}{\pgfqpoint{3.696000in}{3.696000in}}%
\pgfusepath{clip}%
\pgfsetbuttcap%
\pgfsetroundjoin%
\definecolor{currentfill}{rgb}{0.121569,0.466667,0.705882}%
\pgfsetfillcolor{currentfill}%
\pgfsetfillopacity{0.490768}%
\pgfsetlinewidth{1.003750pt}%
\definecolor{currentstroke}{rgb}{0.121569,0.466667,0.705882}%
\pgfsetstrokecolor{currentstroke}%
\pgfsetstrokeopacity{0.490768}%
\pgfsetdash{}{0pt}%
\pgfpathmoveto{\pgfqpoint{1.283798in}{2.287881in}}%
\pgfpathcurveto{\pgfqpoint{1.292034in}{2.287881in}}{\pgfqpoint{1.299934in}{2.291153in}}{\pgfqpoint{1.305758in}{2.296977in}}%
\pgfpathcurveto{\pgfqpoint{1.311582in}{2.302801in}}{\pgfqpoint{1.314854in}{2.310701in}}{\pgfqpoint{1.314854in}{2.318937in}}%
\pgfpathcurveto{\pgfqpoint{1.314854in}{2.327174in}}{\pgfqpoint{1.311582in}{2.335074in}}{\pgfqpoint{1.305758in}{2.340898in}}%
\pgfpathcurveto{\pgfqpoint{1.299934in}{2.346722in}}{\pgfqpoint{1.292034in}{2.349994in}}{\pgfqpoint{1.283798in}{2.349994in}}%
\pgfpathcurveto{\pgfqpoint{1.275561in}{2.349994in}}{\pgfqpoint{1.267661in}{2.346722in}}{\pgfqpoint{1.261838in}{2.340898in}}%
\pgfpathcurveto{\pgfqpoint{1.256014in}{2.335074in}}{\pgfqpoint{1.252741in}{2.327174in}}{\pgfqpoint{1.252741in}{2.318937in}}%
\pgfpathcurveto{\pgfqpoint{1.252741in}{2.310701in}}{\pgfqpoint{1.256014in}{2.302801in}}{\pgfqpoint{1.261838in}{2.296977in}}%
\pgfpathcurveto{\pgfqpoint{1.267661in}{2.291153in}}{\pgfqpoint{1.275561in}{2.287881in}}{\pgfqpoint{1.283798in}{2.287881in}}%
\pgfpathclose%
\pgfusepath{stroke,fill}%
\end{pgfscope}%
\begin{pgfscope}%
\pgfpathrectangle{\pgfqpoint{0.100000in}{0.212622in}}{\pgfqpoint{3.696000in}{3.696000in}}%
\pgfusepath{clip}%
\pgfsetbuttcap%
\pgfsetroundjoin%
\definecolor{currentfill}{rgb}{0.121569,0.466667,0.705882}%
\pgfsetfillcolor{currentfill}%
\pgfsetfillopacity{0.490856}%
\pgfsetlinewidth{1.003750pt}%
\definecolor{currentstroke}{rgb}{0.121569,0.466667,0.705882}%
\pgfsetstrokecolor{currentstroke}%
\pgfsetstrokeopacity{0.490856}%
\pgfsetdash{}{0pt}%
\pgfpathmoveto{\pgfqpoint{2.033121in}{2.544503in}}%
\pgfpathcurveto{\pgfqpoint{2.041358in}{2.544503in}}{\pgfqpoint{2.049258in}{2.547776in}}{\pgfqpoint{2.055082in}{2.553600in}}%
\pgfpathcurveto{\pgfqpoint{2.060906in}{2.559424in}}{\pgfqpoint{2.064178in}{2.567324in}}{\pgfqpoint{2.064178in}{2.575560in}}%
\pgfpathcurveto{\pgfqpoint{2.064178in}{2.583796in}}{\pgfqpoint{2.060906in}{2.591696in}}{\pgfqpoint{2.055082in}{2.597520in}}%
\pgfpathcurveto{\pgfqpoint{2.049258in}{2.603344in}}{\pgfqpoint{2.041358in}{2.606616in}}{\pgfqpoint{2.033121in}{2.606616in}}%
\pgfpathcurveto{\pgfqpoint{2.024885in}{2.606616in}}{\pgfqpoint{2.016985in}{2.603344in}}{\pgfqpoint{2.011161in}{2.597520in}}%
\pgfpathcurveto{\pgfqpoint{2.005337in}{2.591696in}}{\pgfqpoint{2.002065in}{2.583796in}}{\pgfqpoint{2.002065in}{2.575560in}}%
\pgfpathcurveto{\pgfqpoint{2.002065in}{2.567324in}}{\pgfqpoint{2.005337in}{2.559424in}}{\pgfqpoint{2.011161in}{2.553600in}}%
\pgfpathcurveto{\pgfqpoint{2.016985in}{2.547776in}}{\pgfqpoint{2.024885in}{2.544503in}}{\pgfqpoint{2.033121in}{2.544503in}}%
\pgfpathclose%
\pgfusepath{stroke,fill}%
\end{pgfscope}%
\begin{pgfscope}%
\pgfpathrectangle{\pgfqpoint{0.100000in}{0.212622in}}{\pgfqpoint{3.696000in}{3.696000in}}%
\pgfusepath{clip}%
\pgfsetbuttcap%
\pgfsetroundjoin%
\definecolor{currentfill}{rgb}{0.121569,0.466667,0.705882}%
\pgfsetfillcolor{currentfill}%
\pgfsetfillopacity{0.491047}%
\pgfsetlinewidth{1.003750pt}%
\definecolor{currentstroke}{rgb}{0.121569,0.466667,0.705882}%
\pgfsetstrokecolor{currentstroke}%
\pgfsetstrokeopacity{0.491047}%
\pgfsetdash{}{0pt}%
\pgfpathmoveto{\pgfqpoint{1.282790in}{2.286002in}}%
\pgfpathcurveto{\pgfqpoint{1.291026in}{2.286002in}}{\pgfqpoint{1.298926in}{2.289274in}}{\pgfqpoint{1.304750in}{2.295098in}}%
\pgfpathcurveto{\pgfqpoint{1.310574in}{2.300922in}}{\pgfqpoint{1.313846in}{2.308822in}}{\pgfqpoint{1.313846in}{2.317058in}}%
\pgfpathcurveto{\pgfqpoint{1.313846in}{2.325295in}}{\pgfqpoint{1.310574in}{2.333195in}}{\pgfqpoint{1.304750in}{2.339019in}}%
\pgfpathcurveto{\pgfqpoint{1.298926in}{2.344842in}}{\pgfqpoint{1.291026in}{2.348115in}}{\pgfqpoint{1.282790in}{2.348115in}}%
\pgfpathcurveto{\pgfqpoint{1.274553in}{2.348115in}}{\pgfqpoint{1.266653in}{2.344842in}}{\pgfqpoint{1.260829in}{2.339019in}}%
\pgfpathcurveto{\pgfqpoint{1.255005in}{2.333195in}}{\pgfqpoint{1.251733in}{2.325295in}}{\pgfqpoint{1.251733in}{2.317058in}}%
\pgfpathcurveto{\pgfqpoint{1.251733in}{2.308822in}}{\pgfqpoint{1.255005in}{2.300922in}}{\pgfqpoint{1.260829in}{2.295098in}}%
\pgfpathcurveto{\pgfqpoint{1.266653in}{2.289274in}}{\pgfqpoint{1.274553in}{2.286002in}}{\pgfqpoint{1.282790in}{2.286002in}}%
\pgfpathclose%
\pgfusepath{stroke,fill}%
\end{pgfscope}%
\begin{pgfscope}%
\pgfpathrectangle{\pgfqpoint{0.100000in}{0.212622in}}{\pgfqpoint{3.696000in}{3.696000in}}%
\pgfusepath{clip}%
\pgfsetbuttcap%
\pgfsetroundjoin%
\definecolor{currentfill}{rgb}{0.121569,0.466667,0.705882}%
\pgfsetfillcolor{currentfill}%
\pgfsetfillopacity{0.491553}%
\pgfsetlinewidth{1.003750pt}%
\definecolor{currentstroke}{rgb}{0.121569,0.466667,0.705882}%
\pgfsetstrokecolor{currentstroke}%
\pgfsetstrokeopacity{0.491553}%
\pgfsetdash{}{0pt}%
\pgfpathmoveto{\pgfqpoint{1.281218in}{2.282254in}}%
\pgfpathcurveto{\pgfqpoint{1.289454in}{2.282254in}}{\pgfqpoint{1.297354in}{2.285527in}}{\pgfqpoint{1.303178in}{2.291351in}}%
\pgfpathcurveto{\pgfqpoint{1.309002in}{2.297175in}}{\pgfqpoint{1.312275in}{2.305075in}}{\pgfqpoint{1.312275in}{2.313311in}}%
\pgfpathcurveto{\pgfqpoint{1.312275in}{2.321547in}}{\pgfqpoint{1.309002in}{2.329447in}}{\pgfqpoint{1.303178in}{2.335271in}}%
\pgfpathcurveto{\pgfqpoint{1.297354in}{2.341095in}}{\pgfqpoint{1.289454in}{2.344367in}}{\pgfqpoint{1.281218in}{2.344367in}}%
\pgfpathcurveto{\pgfqpoint{1.272982in}{2.344367in}}{\pgfqpoint{1.265082in}{2.341095in}}{\pgfqpoint{1.259258in}{2.335271in}}%
\pgfpathcurveto{\pgfqpoint{1.253434in}{2.329447in}}{\pgfqpoint{1.250162in}{2.321547in}}{\pgfqpoint{1.250162in}{2.313311in}}%
\pgfpathcurveto{\pgfqpoint{1.250162in}{2.305075in}}{\pgfqpoint{1.253434in}{2.297175in}}{\pgfqpoint{1.259258in}{2.291351in}}%
\pgfpathcurveto{\pgfqpoint{1.265082in}{2.285527in}}{\pgfqpoint{1.272982in}{2.282254in}}{\pgfqpoint{1.281218in}{2.282254in}}%
\pgfpathclose%
\pgfusepath{stroke,fill}%
\end{pgfscope}%
\begin{pgfscope}%
\pgfpathrectangle{\pgfqpoint{0.100000in}{0.212622in}}{\pgfqpoint{3.696000in}{3.696000in}}%
\pgfusepath{clip}%
\pgfsetbuttcap%
\pgfsetroundjoin%
\definecolor{currentfill}{rgb}{0.121569,0.466667,0.705882}%
\pgfsetfillcolor{currentfill}%
\pgfsetfillopacity{0.492486}%
\pgfsetlinewidth{1.003750pt}%
\definecolor{currentstroke}{rgb}{0.121569,0.466667,0.705882}%
\pgfsetstrokecolor{currentstroke}%
\pgfsetstrokeopacity{0.492486}%
\pgfsetdash{}{0pt}%
\pgfpathmoveto{\pgfqpoint{1.277863in}{2.276107in}}%
\pgfpathcurveto{\pgfqpoint{1.286099in}{2.276107in}}{\pgfqpoint{1.293999in}{2.279379in}}{\pgfqpoint{1.299823in}{2.285203in}}%
\pgfpathcurveto{\pgfqpoint{1.305647in}{2.291027in}}{\pgfqpoint{1.308919in}{2.298927in}}{\pgfqpoint{1.308919in}{2.307163in}}%
\pgfpathcurveto{\pgfqpoint{1.308919in}{2.315400in}}{\pgfqpoint{1.305647in}{2.323300in}}{\pgfqpoint{1.299823in}{2.329124in}}%
\pgfpathcurveto{\pgfqpoint{1.293999in}{2.334948in}}{\pgfqpoint{1.286099in}{2.338220in}}{\pgfqpoint{1.277863in}{2.338220in}}%
\pgfpathcurveto{\pgfqpoint{1.269627in}{2.338220in}}{\pgfqpoint{1.261726in}{2.334948in}}{\pgfqpoint{1.255903in}{2.329124in}}%
\pgfpathcurveto{\pgfqpoint{1.250079in}{2.323300in}}{\pgfqpoint{1.246806in}{2.315400in}}{\pgfqpoint{1.246806in}{2.307163in}}%
\pgfpathcurveto{\pgfqpoint{1.246806in}{2.298927in}}{\pgfqpoint{1.250079in}{2.291027in}}{\pgfqpoint{1.255903in}{2.285203in}}%
\pgfpathcurveto{\pgfqpoint{1.261726in}{2.279379in}}{\pgfqpoint{1.269627in}{2.276107in}}{\pgfqpoint{1.277863in}{2.276107in}}%
\pgfpathclose%
\pgfusepath{stroke,fill}%
\end{pgfscope}%
\begin{pgfscope}%
\pgfpathrectangle{\pgfqpoint{0.100000in}{0.212622in}}{\pgfqpoint{3.696000in}{3.696000in}}%
\pgfusepath{clip}%
\pgfsetbuttcap%
\pgfsetroundjoin%
\definecolor{currentfill}{rgb}{0.121569,0.466667,0.705882}%
\pgfsetfillcolor{currentfill}%
\pgfsetfillopacity{0.493074}%
\pgfsetlinewidth{1.003750pt}%
\definecolor{currentstroke}{rgb}{0.121569,0.466667,0.705882}%
\pgfsetstrokecolor{currentstroke}%
\pgfsetstrokeopacity{0.493074}%
\pgfsetdash{}{0pt}%
\pgfpathmoveto{\pgfqpoint{2.034277in}{2.534780in}}%
\pgfpathcurveto{\pgfqpoint{2.042513in}{2.534780in}}{\pgfqpoint{2.050413in}{2.538053in}}{\pgfqpoint{2.056237in}{2.543877in}}%
\pgfpathcurveto{\pgfqpoint{2.062061in}{2.549701in}}{\pgfqpoint{2.065333in}{2.557601in}}{\pgfqpoint{2.065333in}{2.565837in}}%
\pgfpathcurveto{\pgfqpoint{2.065333in}{2.574073in}}{\pgfqpoint{2.062061in}{2.581973in}}{\pgfqpoint{2.056237in}{2.587797in}}%
\pgfpathcurveto{\pgfqpoint{2.050413in}{2.593621in}}{\pgfqpoint{2.042513in}{2.596893in}}{\pgfqpoint{2.034277in}{2.596893in}}%
\pgfpathcurveto{\pgfqpoint{2.026041in}{2.596893in}}{\pgfqpoint{2.018141in}{2.593621in}}{\pgfqpoint{2.012317in}{2.587797in}}%
\pgfpathcurveto{\pgfqpoint{2.006493in}{2.581973in}}{\pgfqpoint{2.003220in}{2.574073in}}{\pgfqpoint{2.003220in}{2.565837in}}%
\pgfpathcurveto{\pgfqpoint{2.003220in}{2.557601in}}{\pgfqpoint{2.006493in}{2.549701in}}{\pgfqpoint{2.012317in}{2.543877in}}%
\pgfpathcurveto{\pgfqpoint{2.018141in}{2.538053in}}{\pgfqpoint{2.026041in}{2.534780in}}{\pgfqpoint{2.034277in}{2.534780in}}%
\pgfpathclose%
\pgfusepath{stroke,fill}%
\end{pgfscope}%
\begin{pgfscope}%
\pgfpathrectangle{\pgfqpoint{0.100000in}{0.212622in}}{\pgfqpoint{3.696000in}{3.696000in}}%
\pgfusepath{clip}%
\pgfsetbuttcap%
\pgfsetroundjoin%
\definecolor{currentfill}{rgb}{0.121569,0.466667,0.705882}%
\pgfsetfillcolor{currentfill}%
\pgfsetfillopacity{0.493185}%
\pgfsetlinewidth{1.003750pt}%
\definecolor{currentstroke}{rgb}{0.121569,0.466667,0.705882}%
\pgfsetstrokecolor{currentstroke}%
\pgfsetstrokeopacity{0.493185}%
\pgfsetdash{}{0pt}%
\pgfpathmoveto{\pgfqpoint{1.275211in}{2.270695in}}%
\pgfpathcurveto{\pgfqpoint{1.283448in}{2.270695in}}{\pgfqpoint{1.291348in}{2.273967in}}{\pgfqpoint{1.297172in}{2.279791in}}%
\pgfpathcurveto{\pgfqpoint{1.302995in}{2.285615in}}{\pgfqpoint{1.306268in}{2.293515in}}{\pgfqpoint{1.306268in}{2.301752in}}%
\pgfpathcurveto{\pgfqpoint{1.306268in}{2.309988in}}{\pgfqpoint{1.302995in}{2.317888in}}{\pgfqpoint{1.297172in}{2.323712in}}%
\pgfpathcurveto{\pgfqpoint{1.291348in}{2.329536in}}{\pgfqpoint{1.283448in}{2.332808in}}{\pgfqpoint{1.275211in}{2.332808in}}%
\pgfpathcurveto{\pgfqpoint{1.266975in}{2.332808in}}{\pgfqpoint{1.259075in}{2.329536in}}{\pgfqpoint{1.253251in}{2.323712in}}%
\pgfpathcurveto{\pgfqpoint{1.247427in}{2.317888in}}{\pgfqpoint{1.244155in}{2.309988in}}{\pgfqpoint{1.244155in}{2.301752in}}%
\pgfpathcurveto{\pgfqpoint{1.244155in}{2.293515in}}{\pgfqpoint{1.247427in}{2.285615in}}{\pgfqpoint{1.253251in}{2.279791in}}%
\pgfpathcurveto{\pgfqpoint{1.259075in}{2.273967in}}{\pgfqpoint{1.266975in}{2.270695in}}{\pgfqpoint{1.275211in}{2.270695in}}%
\pgfpathclose%
\pgfusepath{stroke,fill}%
\end{pgfscope}%
\begin{pgfscope}%
\pgfpathrectangle{\pgfqpoint{0.100000in}{0.212622in}}{\pgfqpoint{3.696000in}{3.696000in}}%
\pgfusepath{clip}%
\pgfsetbuttcap%
\pgfsetroundjoin%
\definecolor{currentfill}{rgb}{0.121569,0.466667,0.705882}%
\pgfsetfillcolor{currentfill}%
\pgfsetfillopacity{0.493225}%
\pgfsetlinewidth{1.003750pt}%
\definecolor{currentstroke}{rgb}{0.121569,0.466667,0.705882}%
\pgfsetstrokecolor{currentstroke}%
\pgfsetstrokeopacity{0.493225}%
\pgfsetdash{}{0pt}%
\pgfpathmoveto{\pgfqpoint{1.275015in}{2.270274in}}%
\pgfpathcurveto{\pgfqpoint{1.283251in}{2.270274in}}{\pgfqpoint{1.291151in}{2.273546in}}{\pgfqpoint{1.296975in}{2.279370in}}%
\pgfpathcurveto{\pgfqpoint{1.302799in}{2.285194in}}{\pgfqpoint{1.306071in}{2.293094in}}{\pgfqpoint{1.306071in}{2.301330in}}%
\pgfpathcurveto{\pgfqpoint{1.306071in}{2.309567in}}{\pgfqpoint{1.302799in}{2.317467in}}{\pgfqpoint{1.296975in}{2.323291in}}%
\pgfpathcurveto{\pgfqpoint{1.291151in}{2.329115in}}{\pgfqpoint{1.283251in}{2.332387in}}{\pgfqpoint{1.275015in}{2.332387in}}%
\pgfpathcurveto{\pgfqpoint{1.266779in}{2.332387in}}{\pgfqpoint{1.258879in}{2.329115in}}{\pgfqpoint{1.253055in}{2.323291in}}%
\pgfpathcurveto{\pgfqpoint{1.247231in}{2.317467in}}{\pgfqpoint{1.243958in}{2.309567in}}{\pgfqpoint{1.243958in}{2.301330in}}%
\pgfpathcurveto{\pgfqpoint{1.243958in}{2.293094in}}{\pgfqpoint{1.247231in}{2.285194in}}{\pgfqpoint{1.253055in}{2.279370in}}%
\pgfpathcurveto{\pgfqpoint{1.258879in}{2.273546in}}{\pgfqpoint{1.266779in}{2.270274in}}{\pgfqpoint{1.275015in}{2.270274in}}%
\pgfpathclose%
\pgfusepath{stroke,fill}%
\end{pgfscope}%
\begin{pgfscope}%
\pgfpathrectangle{\pgfqpoint{0.100000in}{0.212622in}}{\pgfqpoint{3.696000in}{3.696000in}}%
\pgfusepath{clip}%
\pgfsetbuttcap%
\pgfsetroundjoin%
\definecolor{currentfill}{rgb}{0.121569,0.466667,0.705882}%
\pgfsetfillcolor{currentfill}%
\pgfsetfillopacity{0.493277}%
\pgfsetlinewidth{1.003750pt}%
\definecolor{currentstroke}{rgb}{0.121569,0.466667,0.705882}%
\pgfsetstrokecolor{currentstroke}%
\pgfsetstrokeopacity{0.493277}%
\pgfsetdash{}{0pt}%
\pgfpathmoveto{\pgfqpoint{1.274687in}{2.269395in}}%
\pgfpathcurveto{\pgfqpoint{1.282923in}{2.269395in}}{\pgfqpoint{1.290823in}{2.272667in}}{\pgfqpoint{1.296647in}{2.278491in}}%
\pgfpathcurveto{\pgfqpoint{1.302471in}{2.284315in}}{\pgfqpoint{1.305743in}{2.292215in}}{\pgfqpoint{1.305743in}{2.300451in}}%
\pgfpathcurveto{\pgfqpoint{1.305743in}{2.308687in}}{\pgfqpoint{1.302471in}{2.316587in}}{\pgfqpoint{1.296647in}{2.322411in}}%
\pgfpathcurveto{\pgfqpoint{1.290823in}{2.328235in}}{\pgfqpoint{1.282923in}{2.331508in}}{\pgfqpoint{1.274687in}{2.331508in}}%
\pgfpathcurveto{\pgfqpoint{1.266451in}{2.331508in}}{\pgfqpoint{1.258551in}{2.328235in}}{\pgfqpoint{1.252727in}{2.322411in}}%
\pgfpathcurveto{\pgfqpoint{1.246903in}{2.316587in}}{\pgfqpoint{1.243630in}{2.308687in}}{\pgfqpoint{1.243630in}{2.300451in}}%
\pgfpathcurveto{\pgfqpoint{1.243630in}{2.292215in}}{\pgfqpoint{1.246903in}{2.284315in}}{\pgfqpoint{1.252727in}{2.278491in}}%
\pgfpathcurveto{\pgfqpoint{1.258551in}{2.272667in}}{\pgfqpoint{1.266451in}{2.269395in}}{\pgfqpoint{1.274687in}{2.269395in}}%
\pgfpathclose%
\pgfusepath{stroke,fill}%
\end{pgfscope}%
\begin{pgfscope}%
\pgfpathrectangle{\pgfqpoint{0.100000in}{0.212622in}}{\pgfqpoint{3.696000in}{3.696000in}}%
\pgfusepath{clip}%
\pgfsetbuttcap%
\pgfsetroundjoin%
\definecolor{currentfill}{rgb}{0.121569,0.466667,0.705882}%
\pgfsetfillcolor{currentfill}%
\pgfsetfillopacity{0.493488}%
\pgfsetlinewidth{1.003750pt}%
\definecolor{currentstroke}{rgb}{0.121569,0.466667,0.705882}%
\pgfsetstrokecolor{currentstroke}%
\pgfsetstrokeopacity{0.493488}%
\pgfsetdash{}{0pt}%
\pgfpathmoveto{\pgfqpoint{1.274021in}{2.268320in}}%
\pgfpathcurveto{\pgfqpoint{1.282258in}{2.268320in}}{\pgfqpoint{1.290158in}{2.271592in}}{\pgfqpoint{1.295982in}{2.277416in}}%
\pgfpathcurveto{\pgfqpoint{1.301806in}{2.283240in}}{\pgfqpoint{1.305078in}{2.291140in}}{\pgfqpoint{1.305078in}{2.299376in}}%
\pgfpathcurveto{\pgfqpoint{1.305078in}{2.307613in}}{\pgfqpoint{1.301806in}{2.315513in}}{\pgfqpoint{1.295982in}{2.321337in}}%
\pgfpathcurveto{\pgfqpoint{1.290158in}{2.327161in}}{\pgfqpoint{1.282258in}{2.330433in}}{\pgfqpoint{1.274021in}{2.330433in}}%
\pgfpathcurveto{\pgfqpoint{1.265785in}{2.330433in}}{\pgfqpoint{1.257885in}{2.327161in}}{\pgfqpoint{1.252061in}{2.321337in}}%
\pgfpathcurveto{\pgfqpoint{1.246237in}{2.315513in}}{\pgfqpoint{1.242965in}{2.307613in}}{\pgfqpoint{1.242965in}{2.299376in}}%
\pgfpathcurveto{\pgfqpoint{1.242965in}{2.291140in}}{\pgfqpoint{1.246237in}{2.283240in}}{\pgfqpoint{1.252061in}{2.277416in}}%
\pgfpathcurveto{\pgfqpoint{1.257885in}{2.271592in}}{\pgfqpoint{1.265785in}{2.268320in}}{\pgfqpoint{1.274021in}{2.268320in}}%
\pgfpathclose%
\pgfusepath{stroke,fill}%
\end{pgfscope}%
\begin{pgfscope}%
\pgfpathrectangle{\pgfqpoint{0.100000in}{0.212622in}}{\pgfqpoint{3.696000in}{3.696000in}}%
\pgfusepath{clip}%
\pgfsetbuttcap%
\pgfsetroundjoin%
\definecolor{currentfill}{rgb}{0.121569,0.466667,0.705882}%
\pgfsetfillcolor{currentfill}%
\pgfsetfillopacity{0.493903}%
\pgfsetlinewidth{1.003750pt}%
\definecolor{currentstroke}{rgb}{0.121569,0.466667,0.705882}%
\pgfsetstrokecolor{currentstroke}%
\pgfsetstrokeopacity{0.493903}%
\pgfsetdash{}{0pt}%
\pgfpathmoveto{\pgfqpoint{1.272949in}{2.266303in}}%
\pgfpathcurveto{\pgfqpoint{1.281185in}{2.266303in}}{\pgfqpoint{1.289085in}{2.269575in}}{\pgfqpoint{1.294909in}{2.275399in}}%
\pgfpathcurveto{\pgfqpoint{1.300733in}{2.281223in}}{\pgfqpoint{1.304005in}{2.289123in}}{\pgfqpoint{1.304005in}{2.297359in}}%
\pgfpathcurveto{\pgfqpoint{1.304005in}{2.305595in}}{\pgfqpoint{1.300733in}{2.313495in}}{\pgfqpoint{1.294909in}{2.319319in}}%
\pgfpathcurveto{\pgfqpoint{1.289085in}{2.325143in}}{\pgfqpoint{1.281185in}{2.328416in}}{\pgfqpoint{1.272949in}{2.328416in}}%
\pgfpathcurveto{\pgfqpoint{1.264713in}{2.328416in}}{\pgfqpoint{1.256813in}{2.325143in}}{\pgfqpoint{1.250989in}{2.319319in}}%
\pgfpathcurveto{\pgfqpoint{1.245165in}{2.313495in}}{\pgfqpoint{1.241892in}{2.305595in}}{\pgfqpoint{1.241892in}{2.297359in}}%
\pgfpathcurveto{\pgfqpoint{1.241892in}{2.289123in}}{\pgfqpoint{1.245165in}{2.281223in}}{\pgfqpoint{1.250989in}{2.275399in}}%
\pgfpathcurveto{\pgfqpoint{1.256813in}{2.269575in}}{\pgfqpoint{1.264713in}{2.266303in}}{\pgfqpoint{1.272949in}{2.266303in}}%
\pgfpathclose%
\pgfusepath{stroke,fill}%
\end{pgfscope}%
\begin{pgfscope}%
\pgfpathrectangle{\pgfqpoint{0.100000in}{0.212622in}}{\pgfqpoint{3.696000in}{3.696000in}}%
\pgfusepath{clip}%
\pgfsetbuttcap%
\pgfsetroundjoin%
\definecolor{currentfill}{rgb}{0.121569,0.466667,0.705882}%
\pgfsetfillcolor{currentfill}%
\pgfsetfillopacity{0.494252}%
\pgfsetlinewidth{1.003750pt}%
\definecolor{currentstroke}{rgb}{0.121569,0.466667,0.705882}%
\pgfsetstrokecolor{currentstroke}%
\pgfsetstrokeopacity{0.494252}%
\pgfsetdash{}{0pt}%
\pgfpathmoveto{\pgfqpoint{1.272027in}{2.264744in}}%
\pgfpathcurveto{\pgfqpoint{1.280264in}{2.264744in}}{\pgfqpoint{1.288164in}{2.268016in}}{\pgfqpoint{1.293988in}{2.273840in}}%
\pgfpathcurveto{\pgfqpoint{1.299812in}{2.279664in}}{\pgfqpoint{1.303084in}{2.287564in}}{\pgfqpoint{1.303084in}{2.295800in}}%
\pgfpathcurveto{\pgfqpoint{1.303084in}{2.304037in}}{\pgfqpoint{1.299812in}{2.311937in}}{\pgfqpoint{1.293988in}{2.317760in}}%
\pgfpathcurveto{\pgfqpoint{1.288164in}{2.323584in}}{\pgfqpoint{1.280264in}{2.326857in}}{\pgfqpoint{1.272027in}{2.326857in}}%
\pgfpathcurveto{\pgfqpoint{1.263791in}{2.326857in}}{\pgfqpoint{1.255891in}{2.323584in}}{\pgfqpoint{1.250067in}{2.317760in}}%
\pgfpathcurveto{\pgfqpoint{1.244243in}{2.311937in}}{\pgfqpoint{1.240971in}{2.304037in}}{\pgfqpoint{1.240971in}{2.295800in}}%
\pgfpathcurveto{\pgfqpoint{1.240971in}{2.287564in}}{\pgfqpoint{1.244243in}{2.279664in}}{\pgfqpoint{1.250067in}{2.273840in}}%
\pgfpathcurveto{\pgfqpoint{1.255891in}{2.268016in}}{\pgfqpoint{1.263791in}{2.264744in}}{\pgfqpoint{1.272027in}{2.264744in}}%
\pgfpathclose%
\pgfusepath{stroke,fill}%
\end{pgfscope}%
\begin{pgfscope}%
\pgfpathrectangle{\pgfqpoint{0.100000in}{0.212622in}}{\pgfqpoint{3.696000in}{3.696000in}}%
\pgfusepath{clip}%
\pgfsetbuttcap%
\pgfsetroundjoin%
\definecolor{currentfill}{rgb}{0.121569,0.466667,0.705882}%
\pgfsetfillcolor{currentfill}%
\pgfsetfillopacity{0.494887}%
\pgfsetlinewidth{1.003750pt}%
\definecolor{currentstroke}{rgb}{0.121569,0.466667,0.705882}%
\pgfsetstrokecolor{currentstroke}%
\pgfsetstrokeopacity{0.494887}%
\pgfsetdash{}{0pt}%
\pgfpathmoveto{\pgfqpoint{1.270379in}{2.261876in}}%
\pgfpathcurveto{\pgfqpoint{1.278616in}{2.261876in}}{\pgfqpoint{1.286516in}{2.265149in}}{\pgfqpoint{1.292340in}{2.270973in}}%
\pgfpathcurveto{\pgfqpoint{1.298164in}{2.276796in}}{\pgfqpoint{1.301436in}{2.284696in}}{\pgfqpoint{1.301436in}{2.292933in}}%
\pgfpathcurveto{\pgfqpoint{1.301436in}{2.301169in}}{\pgfqpoint{1.298164in}{2.309069in}}{\pgfqpoint{1.292340in}{2.314893in}}%
\pgfpathcurveto{\pgfqpoint{1.286516in}{2.320717in}}{\pgfqpoint{1.278616in}{2.323989in}}{\pgfqpoint{1.270379in}{2.323989in}}%
\pgfpathcurveto{\pgfqpoint{1.262143in}{2.323989in}}{\pgfqpoint{1.254243in}{2.320717in}}{\pgfqpoint{1.248419in}{2.314893in}}%
\pgfpathcurveto{\pgfqpoint{1.242595in}{2.309069in}}{\pgfqpoint{1.239323in}{2.301169in}}{\pgfqpoint{1.239323in}{2.292933in}}%
\pgfpathcurveto{\pgfqpoint{1.239323in}{2.284696in}}{\pgfqpoint{1.242595in}{2.276796in}}{\pgfqpoint{1.248419in}{2.270973in}}%
\pgfpathcurveto{\pgfqpoint{1.254243in}{2.265149in}}{\pgfqpoint{1.262143in}{2.261876in}}{\pgfqpoint{1.270379in}{2.261876in}}%
\pgfpathclose%
\pgfusepath{stroke,fill}%
\end{pgfscope}%
\begin{pgfscope}%
\pgfpathrectangle{\pgfqpoint{0.100000in}{0.212622in}}{\pgfqpoint{3.696000in}{3.696000in}}%
\pgfusepath{clip}%
\pgfsetbuttcap%
\pgfsetroundjoin%
\definecolor{currentfill}{rgb}{0.121569,0.466667,0.705882}%
\pgfsetfillcolor{currentfill}%
\pgfsetfillopacity{0.495079}%
\pgfsetlinewidth{1.003750pt}%
\definecolor{currentstroke}{rgb}{0.121569,0.466667,0.705882}%
\pgfsetstrokecolor{currentstroke}%
\pgfsetstrokeopacity{0.495079}%
\pgfsetdash{}{0pt}%
\pgfpathmoveto{\pgfqpoint{2.036056in}{2.523138in}}%
\pgfpathcurveto{\pgfqpoint{2.044292in}{2.523138in}}{\pgfqpoint{2.052192in}{2.526410in}}{\pgfqpoint{2.058016in}{2.532234in}}%
\pgfpathcurveto{\pgfqpoint{2.063840in}{2.538058in}}{\pgfqpoint{2.067112in}{2.545958in}}{\pgfqpoint{2.067112in}{2.554194in}}%
\pgfpathcurveto{\pgfqpoint{2.067112in}{2.562431in}}{\pgfqpoint{2.063840in}{2.570331in}}{\pgfqpoint{2.058016in}{2.576154in}}%
\pgfpathcurveto{\pgfqpoint{2.052192in}{2.581978in}}{\pgfqpoint{2.044292in}{2.585251in}}{\pgfqpoint{2.036056in}{2.585251in}}%
\pgfpathcurveto{\pgfqpoint{2.027819in}{2.585251in}}{\pgfqpoint{2.019919in}{2.581978in}}{\pgfqpoint{2.014095in}{2.576154in}}%
\pgfpathcurveto{\pgfqpoint{2.008272in}{2.570331in}}{\pgfqpoint{2.004999in}{2.562431in}}{\pgfqpoint{2.004999in}{2.554194in}}%
\pgfpathcurveto{\pgfqpoint{2.004999in}{2.545958in}}{\pgfqpoint{2.008272in}{2.538058in}}{\pgfqpoint{2.014095in}{2.532234in}}%
\pgfpathcurveto{\pgfqpoint{2.019919in}{2.526410in}}{\pgfqpoint{2.027819in}{2.523138in}}{\pgfqpoint{2.036056in}{2.523138in}}%
\pgfpathclose%
\pgfusepath{stroke,fill}%
\end{pgfscope}%
\begin{pgfscope}%
\pgfpathrectangle{\pgfqpoint{0.100000in}{0.212622in}}{\pgfqpoint{3.696000in}{3.696000in}}%
\pgfusepath{clip}%
\pgfsetbuttcap%
\pgfsetroundjoin%
\definecolor{currentfill}{rgb}{0.121569,0.466667,0.705882}%
\pgfsetfillcolor{currentfill}%
\pgfsetfillopacity{0.496016}%
\pgfsetlinewidth{1.003750pt}%
\definecolor{currentstroke}{rgb}{0.121569,0.466667,0.705882}%
\pgfsetstrokecolor{currentstroke}%
\pgfsetstrokeopacity{0.496016}%
\pgfsetdash{}{0pt}%
\pgfpathmoveto{\pgfqpoint{1.267430in}{2.256504in}}%
\pgfpathcurveto{\pgfqpoint{1.275666in}{2.256504in}}{\pgfqpoint{1.283566in}{2.259776in}}{\pgfqpoint{1.289390in}{2.265600in}}%
\pgfpathcurveto{\pgfqpoint{1.295214in}{2.271424in}}{\pgfqpoint{1.298486in}{2.279324in}}{\pgfqpoint{1.298486in}{2.287561in}}%
\pgfpathcurveto{\pgfqpoint{1.298486in}{2.295797in}}{\pgfqpoint{1.295214in}{2.303697in}}{\pgfqpoint{1.289390in}{2.309521in}}%
\pgfpathcurveto{\pgfqpoint{1.283566in}{2.315345in}}{\pgfqpoint{1.275666in}{2.318617in}}{\pgfqpoint{1.267430in}{2.318617in}}%
\pgfpathcurveto{\pgfqpoint{1.259194in}{2.318617in}}{\pgfqpoint{1.251293in}{2.315345in}}{\pgfqpoint{1.245470in}{2.309521in}}%
\pgfpathcurveto{\pgfqpoint{1.239646in}{2.303697in}}{\pgfqpoint{1.236373in}{2.295797in}}{\pgfqpoint{1.236373in}{2.287561in}}%
\pgfpathcurveto{\pgfqpoint{1.236373in}{2.279324in}}{\pgfqpoint{1.239646in}{2.271424in}}{\pgfqpoint{1.245470in}{2.265600in}}%
\pgfpathcurveto{\pgfqpoint{1.251293in}{2.259776in}}{\pgfqpoint{1.259194in}{2.256504in}}{\pgfqpoint{1.267430in}{2.256504in}}%
\pgfpathclose%
\pgfusepath{stroke,fill}%
\end{pgfscope}%
\begin{pgfscope}%
\pgfpathrectangle{\pgfqpoint{0.100000in}{0.212622in}}{\pgfqpoint{3.696000in}{3.696000in}}%
\pgfusepath{clip}%
\pgfsetbuttcap%
\pgfsetroundjoin%
\definecolor{currentfill}{rgb}{0.121569,0.466667,0.705882}%
\pgfsetfillcolor{currentfill}%
\pgfsetfillopacity{0.497844}%
\pgfsetlinewidth{1.003750pt}%
\definecolor{currentstroke}{rgb}{0.121569,0.466667,0.705882}%
\pgfsetstrokecolor{currentstroke}%
\pgfsetstrokeopacity{0.497844}%
\pgfsetdash{}{0pt}%
\pgfpathmoveto{\pgfqpoint{2.037083in}{2.509851in}}%
\pgfpathcurveto{\pgfqpoint{2.045319in}{2.509851in}}{\pgfqpoint{2.053219in}{2.513123in}}{\pgfqpoint{2.059043in}{2.518947in}}%
\pgfpathcurveto{\pgfqpoint{2.064867in}{2.524771in}}{\pgfqpoint{2.068139in}{2.532671in}}{\pgfqpoint{2.068139in}{2.540907in}}%
\pgfpathcurveto{\pgfqpoint{2.068139in}{2.549143in}}{\pgfqpoint{2.064867in}{2.557043in}}{\pgfqpoint{2.059043in}{2.562867in}}%
\pgfpathcurveto{\pgfqpoint{2.053219in}{2.568691in}}{\pgfqpoint{2.045319in}{2.571964in}}{\pgfqpoint{2.037083in}{2.571964in}}%
\pgfpathcurveto{\pgfqpoint{2.028846in}{2.571964in}}{\pgfqpoint{2.020946in}{2.568691in}}{\pgfqpoint{2.015122in}{2.562867in}}%
\pgfpathcurveto{\pgfqpoint{2.009298in}{2.557043in}}{\pgfqpoint{2.006026in}{2.549143in}}{\pgfqpoint{2.006026in}{2.540907in}}%
\pgfpathcurveto{\pgfqpoint{2.006026in}{2.532671in}}{\pgfqpoint{2.009298in}{2.524771in}}{\pgfqpoint{2.015122in}{2.518947in}}%
\pgfpathcurveto{\pgfqpoint{2.020946in}{2.513123in}}{\pgfqpoint{2.028846in}{2.509851in}}{\pgfqpoint{2.037083in}{2.509851in}}%
\pgfpathclose%
\pgfusepath{stroke,fill}%
\end{pgfscope}%
\begin{pgfscope}%
\pgfpathrectangle{\pgfqpoint{0.100000in}{0.212622in}}{\pgfqpoint{3.696000in}{3.696000in}}%
\pgfusepath{clip}%
\pgfsetbuttcap%
\pgfsetroundjoin%
\definecolor{currentfill}{rgb}{0.121569,0.466667,0.705882}%
\pgfsetfillcolor{currentfill}%
\pgfsetfillopacity{0.498013}%
\pgfsetlinewidth{1.003750pt}%
\definecolor{currentstroke}{rgb}{0.121569,0.466667,0.705882}%
\pgfsetstrokecolor{currentstroke}%
\pgfsetstrokeopacity{0.498013}%
\pgfsetdash{}{0pt}%
\pgfpathmoveto{\pgfqpoint{1.261685in}{2.247014in}}%
\pgfpathcurveto{\pgfqpoint{1.269921in}{2.247014in}}{\pgfqpoint{1.277821in}{2.250287in}}{\pgfqpoint{1.283645in}{2.256111in}}%
\pgfpathcurveto{\pgfqpoint{1.289469in}{2.261934in}}{\pgfqpoint{1.292742in}{2.269835in}}{\pgfqpoint{1.292742in}{2.278071in}}%
\pgfpathcurveto{\pgfqpoint{1.292742in}{2.286307in}}{\pgfqpoint{1.289469in}{2.294207in}}{\pgfqpoint{1.283645in}{2.300031in}}%
\pgfpathcurveto{\pgfqpoint{1.277821in}{2.305855in}}{\pgfqpoint{1.269921in}{2.309127in}}{\pgfqpoint{1.261685in}{2.309127in}}%
\pgfpathcurveto{\pgfqpoint{1.253449in}{2.309127in}}{\pgfqpoint{1.245549in}{2.305855in}}{\pgfqpoint{1.239725in}{2.300031in}}%
\pgfpathcurveto{\pgfqpoint{1.233901in}{2.294207in}}{\pgfqpoint{1.230629in}{2.286307in}}{\pgfqpoint{1.230629in}{2.278071in}}%
\pgfpathcurveto{\pgfqpoint{1.230629in}{2.269835in}}{\pgfqpoint{1.233901in}{2.261934in}}{\pgfqpoint{1.239725in}{2.256111in}}%
\pgfpathcurveto{\pgfqpoint{1.245549in}{2.250287in}}{\pgfqpoint{1.253449in}{2.247014in}}{\pgfqpoint{1.261685in}{2.247014in}}%
\pgfpathclose%
\pgfusepath{stroke,fill}%
\end{pgfscope}%
\begin{pgfscope}%
\pgfpathrectangle{\pgfqpoint{0.100000in}{0.212622in}}{\pgfqpoint{3.696000in}{3.696000in}}%
\pgfusepath{clip}%
\pgfsetbuttcap%
\pgfsetroundjoin%
\definecolor{currentfill}{rgb}{0.121569,0.466667,0.705882}%
\pgfsetfillcolor{currentfill}%
\pgfsetfillopacity{0.499860}%
\pgfsetlinewidth{1.003750pt}%
\definecolor{currentstroke}{rgb}{0.121569,0.466667,0.705882}%
\pgfsetstrokecolor{currentstroke}%
\pgfsetstrokeopacity{0.499860}%
\pgfsetdash{}{0pt}%
\pgfpathmoveto{\pgfqpoint{1.256477in}{2.237463in}}%
\pgfpathcurveto{\pgfqpoint{1.264714in}{2.237463in}}{\pgfqpoint{1.272614in}{2.240736in}}{\pgfqpoint{1.278438in}{2.246559in}}%
\pgfpathcurveto{\pgfqpoint{1.284262in}{2.252383in}}{\pgfqpoint{1.287534in}{2.260283in}}{\pgfqpoint{1.287534in}{2.268520in}}%
\pgfpathcurveto{\pgfqpoint{1.287534in}{2.276756in}}{\pgfqpoint{1.284262in}{2.284656in}}{\pgfqpoint{1.278438in}{2.290480in}}%
\pgfpathcurveto{\pgfqpoint{1.272614in}{2.296304in}}{\pgfqpoint{1.264714in}{2.299576in}}{\pgfqpoint{1.256477in}{2.299576in}}%
\pgfpathcurveto{\pgfqpoint{1.248241in}{2.299576in}}{\pgfqpoint{1.240341in}{2.296304in}}{\pgfqpoint{1.234517in}{2.290480in}}%
\pgfpathcurveto{\pgfqpoint{1.228693in}{2.284656in}}{\pgfqpoint{1.225421in}{2.276756in}}{\pgfqpoint{1.225421in}{2.268520in}}%
\pgfpathcurveto{\pgfqpoint{1.225421in}{2.260283in}}{\pgfqpoint{1.228693in}{2.252383in}}{\pgfqpoint{1.234517in}{2.246559in}}%
\pgfpathcurveto{\pgfqpoint{1.240341in}{2.240736in}}{\pgfqpoint{1.248241in}{2.237463in}}{\pgfqpoint{1.256477in}{2.237463in}}%
\pgfpathclose%
\pgfusepath{stroke,fill}%
\end{pgfscope}%
\begin{pgfscope}%
\pgfpathrectangle{\pgfqpoint{0.100000in}{0.212622in}}{\pgfqpoint{3.696000in}{3.696000in}}%
\pgfusepath{clip}%
\pgfsetbuttcap%
\pgfsetroundjoin%
\definecolor{currentfill}{rgb}{0.121569,0.466667,0.705882}%
\pgfsetfillcolor{currentfill}%
\pgfsetfillopacity{0.500787}%
\pgfsetlinewidth{1.003750pt}%
\definecolor{currentstroke}{rgb}{0.121569,0.466667,0.705882}%
\pgfsetstrokecolor{currentstroke}%
\pgfsetstrokeopacity{0.500787}%
\pgfsetdash{}{0pt}%
\pgfpathmoveto{\pgfqpoint{2.038287in}{2.496419in}}%
\pgfpathcurveto{\pgfqpoint{2.046523in}{2.496419in}}{\pgfqpoint{2.054423in}{2.499691in}}{\pgfqpoint{2.060247in}{2.505515in}}%
\pgfpathcurveto{\pgfqpoint{2.066071in}{2.511339in}}{\pgfqpoint{2.069343in}{2.519239in}}{\pgfqpoint{2.069343in}{2.527476in}}%
\pgfpathcurveto{\pgfqpoint{2.069343in}{2.535712in}}{\pgfqpoint{2.066071in}{2.543612in}}{\pgfqpoint{2.060247in}{2.549436in}}%
\pgfpathcurveto{\pgfqpoint{2.054423in}{2.555260in}}{\pgfqpoint{2.046523in}{2.558532in}}{\pgfqpoint{2.038287in}{2.558532in}}%
\pgfpathcurveto{\pgfqpoint{2.030051in}{2.558532in}}{\pgfqpoint{2.022151in}{2.555260in}}{\pgfqpoint{2.016327in}{2.549436in}}%
\pgfpathcurveto{\pgfqpoint{2.010503in}{2.543612in}}{\pgfqpoint{2.007230in}{2.535712in}}{\pgfqpoint{2.007230in}{2.527476in}}%
\pgfpathcurveto{\pgfqpoint{2.007230in}{2.519239in}}{\pgfqpoint{2.010503in}{2.511339in}}{\pgfqpoint{2.016327in}{2.505515in}}%
\pgfpathcurveto{\pgfqpoint{2.022151in}{2.499691in}}{\pgfqpoint{2.030051in}{2.496419in}}{\pgfqpoint{2.038287in}{2.496419in}}%
\pgfpathclose%
\pgfusepath{stroke,fill}%
\end{pgfscope}%
\begin{pgfscope}%
\pgfpathrectangle{\pgfqpoint{0.100000in}{0.212622in}}{\pgfqpoint{3.696000in}{3.696000in}}%
\pgfusepath{clip}%
\pgfsetbuttcap%
\pgfsetroundjoin%
\definecolor{currentfill}{rgb}{0.121569,0.466667,0.705882}%
\pgfsetfillcolor{currentfill}%
\pgfsetfillopacity{0.501385}%
\pgfsetlinewidth{1.003750pt}%
\definecolor{currentstroke}{rgb}{0.121569,0.466667,0.705882}%
\pgfsetstrokecolor{currentstroke}%
\pgfsetstrokeopacity{0.501385}%
\pgfsetdash{}{0pt}%
\pgfpathmoveto{\pgfqpoint{1.251284in}{2.228493in}}%
\pgfpathcurveto{\pgfqpoint{1.259520in}{2.228493in}}{\pgfqpoint{1.267420in}{2.231765in}}{\pgfqpoint{1.273244in}{2.237589in}}%
\pgfpathcurveto{\pgfqpoint{1.279068in}{2.243413in}}{\pgfqpoint{1.282341in}{2.251313in}}{\pgfqpoint{1.282341in}{2.259550in}}%
\pgfpathcurveto{\pgfqpoint{1.282341in}{2.267786in}}{\pgfqpoint{1.279068in}{2.275686in}}{\pgfqpoint{1.273244in}{2.281510in}}%
\pgfpathcurveto{\pgfqpoint{1.267420in}{2.287334in}}{\pgfqpoint{1.259520in}{2.290606in}}{\pgfqpoint{1.251284in}{2.290606in}}%
\pgfpathcurveto{\pgfqpoint{1.243048in}{2.290606in}}{\pgfqpoint{1.235148in}{2.287334in}}{\pgfqpoint{1.229324in}{2.281510in}}%
\pgfpathcurveto{\pgfqpoint{1.223500in}{2.275686in}}{\pgfqpoint{1.220228in}{2.267786in}}{\pgfqpoint{1.220228in}{2.259550in}}%
\pgfpathcurveto{\pgfqpoint{1.220228in}{2.251313in}}{\pgfqpoint{1.223500in}{2.243413in}}{\pgfqpoint{1.229324in}{2.237589in}}%
\pgfpathcurveto{\pgfqpoint{1.235148in}{2.231765in}}{\pgfqpoint{1.243048in}{2.228493in}}{\pgfqpoint{1.251284in}{2.228493in}}%
\pgfpathclose%
\pgfusepath{stroke,fill}%
\end{pgfscope}%
\begin{pgfscope}%
\pgfpathrectangle{\pgfqpoint{0.100000in}{0.212622in}}{\pgfqpoint{3.696000in}{3.696000in}}%
\pgfusepath{clip}%
\pgfsetbuttcap%
\pgfsetroundjoin%
\definecolor{currentfill}{rgb}{0.121569,0.466667,0.705882}%
\pgfsetfillcolor{currentfill}%
\pgfsetfillopacity{0.502174}%
\pgfsetlinewidth{1.003750pt}%
\definecolor{currentstroke}{rgb}{0.121569,0.466667,0.705882}%
\pgfsetstrokecolor{currentstroke}%
\pgfsetstrokeopacity{0.502174}%
\pgfsetdash{}{0pt}%
\pgfpathmoveto{\pgfqpoint{2.039487in}{2.488574in}}%
\pgfpathcurveto{\pgfqpoint{2.047723in}{2.488574in}}{\pgfqpoint{2.055623in}{2.491846in}}{\pgfqpoint{2.061447in}{2.497670in}}%
\pgfpathcurveto{\pgfqpoint{2.067271in}{2.503494in}}{\pgfqpoint{2.070543in}{2.511394in}}{\pgfqpoint{2.070543in}{2.519630in}}%
\pgfpathcurveto{\pgfqpoint{2.070543in}{2.527866in}}{\pgfqpoint{2.067271in}{2.535767in}}{\pgfqpoint{2.061447in}{2.541590in}}%
\pgfpathcurveto{\pgfqpoint{2.055623in}{2.547414in}}{\pgfqpoint{2.047723in}{2.550687in}}{\pgfqpoint{2.039487in}{2.550687in}}%
\pgfpathcurveto{\pgfqpoint{2.031250in}{2.550687in}}{\pgfqpoint{2.023350in}{2.547414in}}{\pgfqpoint{2.017526in}{2.541590in}}%
\pgfpathcurveto{\pgfqpoint{2.011702in}{2.535767in}}{\pgfqpoint{2.008430in}{2.527866in}}{\pgfqpoint{2.008430in}{2.519630in}}%
\pgfpathcurveto{\pgfqpoint{2.008430in}{2.511394in}}{\pgfqpoint{2.011702in}{2.503494in}}{\pgfqpoint{2.017526in}{2.497670in}}%
\pgfpathcurveto{\pgfqpoint{2.023350in}{2.491846in}}{\pgfqpoint{2.031250in}{2.488574in}}{\pgfqpoint{2.039487in}{2.488574in}}%
\pgfpathclose%
\pgfusepath{stroke,fill}%
\end{pgfscope}%
\begin{pgfscope}%
\pgfpathrectangle{\pgfqpoint{0.100000in}{0.212622in}}{\pgfqpoint{3.696000in}{3.696000in}}%
\pgfusepath{clip}%
\pgfsetbuttcap%
\pgfsetroundjoin%
\definecolor{currentfill}{rgb}{0.121569,0.466667,0.705882}%
\pgfsetfillcolor{currentfill}%
\pgfsetfillopacity{0.502817}%
\pgfsetlinewidth{1.003750pt}%
\definecolor{currentstroke}{rgb}{0.121569,0.466667,0.705882}%
\pgfsetstrokecolor{currentstroke}%
\pgfsetstrokeopacity{0.502817}%
\pgfsetdash{}{0pt}%
\pgfpathmoveto{\pgfqpoint{1.247241in}{2.219081in}}%
\pgfpathcurveto{\pgfqpoint{1.255477in}{2.219081in}}{\pgfqpoint{1.263377in}{2.222353in}}{\pgfqpoint{1.269201in}{2.228177in}}%
\pgfpathcurveto{\pgfqpoint{1.275025in}{2.234001in}}{\pgfqpoint{1.278297in}{2.241901in}}{\pgfqpoint{1.278297in}{2.250137in}}%
\pgfpathcurveto{\pgfqpoint{1.278297in}{2.258373in}}{\pgfqpoint{1.275025in}{2.266273in}}{\pgfqpoint{1.269201in}{2.272097in}}%
\pgfpathcurveto{\pgfqpoint{1.263377in}{2.277921in}}{\pgfqpoint{1.255477in}{2.281194in}}{\pgfqpoint{1.247241in}{2.281194in}}%
\pgfpathcurveto{\pgfqpoint{1.239004in}{2.281194in}}{\pgfqpoint{1.231104in}{2.277921in}}{\pgfqpoint{1.225281in}{2.272097in}}%
\pgfpathcurveto{\pgfqpoint{1.219457in}{2.266273in}}{\pgfqpoint{1.216184in}{2.258373in}}{\pgfqpoint{1.216184in}{2.250137in}}%
\pgfpathcurveto{\pgfqpoint{1.216184in}{2.241901in}}{\pgfqpoint{1.219457in}{2.234001in}}{\pgfqpoint{1.225281in}{2.228177in}}%
\pgfpathcurveto{\pgfqpoint{1.231104in}{2.222353in}}{\pgfqpoint{1.239004in}{2.219081in}}{\pgfqpoint{1.247241in}{2.219081in}}%
\pgfpathclose%
\pgfusepath{stroke,fill}%
\end{pgfscope}%
\begin{pgfscope}%
\pgfpathrectangle{\pgfqpoint{0.100000in}{0.212622in}}{\pgfqpoint{3.696000in}{3.696000in}}%
\pgfusepath{clip}%
\pgfsetbuttcap%
\pgfsetroundjoin%
\definecolor{currentfill}{rgb}{0.121569,0.466667,0.705882}%
\pgfsetfillcolor{currentfill}%
\pgfsetfillopacity{0.504047}%
\pgfsetlinewidth{1.003750pt}%
\definecolor{currentstroke}{rgb}{0.121569,0.466667,0.705882}%
\pgfsetstrokecolor{currentstroke}%
\pgfsetstrokeopacity{0.504047}%
\pgfsetdash{}{0pt}%
\pgfpathmoveto{\pgfqpoint{2.040333in}{2.479730in}}%
\pgfpathcurveto{\pgfqpoint{2.048570in}{2.479730in}}{\pgfqpoint{2.056470in}{2.483002in}}{\pgfqpoint{2.062294in}{2.488826in}}%
\pgfpathcurveto{\pgfqpoint{2.068118in}{2.494650in}}{\pgfqpoint{2.071390in}{2.502550in}}{\pgfqpoint{2.071390in}{2.510786in}}%
\pgfpathcurveto{\pgfqpoint{2.071390in}{2.519022in}}{\pgfqpoint{2.068118in}{2.526923in}}{\pgfqpoint{2.062294in}{2.532746in}}%
\pgfpathcurveto{\pgfqpoint{2.056470in}{2.538570in}}{\pgfqpoint{2.048570in}{2.541843in}}{\pgfqpoint{2.040333in}{2.541843in}}%
\pgfpathcurveto{\pgfqpoint{2.032097in}{2.541843in}}{\pgfqpoint{2.024197in}{2.538570in}}{\pgfqpoint{2.018373in}{2.532746in}}%
\pgfpathcurveto{\pgfqpoint{2.012549in}{2.526923in}}{\pgfqpoint{2.009277in}{2.519022in}}{\pgfqpoint{2.009277in}{2.510786in}}%
\pgfpathcurveto{\pgfqpoint{2.009277in}{2.502550in}}{\pgfqpoint{2.012549in}{2.494650in}}{\pgfqpoint{2.018373in}{2.488826in}}%
\pgfpathcurveto{\pgfqpoint{2.024197in}{2.483002in}}{\pgfqpoint{2.032097in}{2.479730in}}{\pgfqpoint{2.040333in}{2.479730in}}%
\pgfpathclose%
\pgfusepath{stroke,fill}%
\end{pgfscope}%
\begin{pgfscope}%
\pgfpathrectangle{\pgfqpoint{0.100000in}{0.212622in}}{\pgfqpoint{3.696000in}{3.696000in}}%
\pgfusepath{clip}%
\pgfsetbuttcap%
\pgfsetroundjoin%
\definecolor{currentfill}{rgb}{0.121569,0.466667,0.705882}%
\pgfsetfillcolor{currentfill}%
\pgfsetfillopacity{0.504312}%
\pgfsetlinewidth{1.003750pt}%
\definecolor{currentstroke}{rgb}{0.121569,0.466667,0.705882}%
\pgfsetstrokecolor{currentstroke}%
\pgfsetstrokeopacity{0.504312}%
\pgfsetdash{}{0pt}%
\pgfpathmoveto{\pgfqpoint{1.242735in}{2.211571in}}%
\pgfpathcurveto{\pgfqpoint{1.250971in}{2.211571in}}{\pgfqpoint{1.258871in}{2.214843in}}{\pgfqpoint{1.264695in}{2.220667in}}%
\pgfpathcurveto{\pgfqpoint{1.270519in}{2.226491in}}{\pgfqpoint{1.273791in}{2.234391in}}{\pgfqpoint{1.273791in}{2.242627in}}%
\pgfpathcurveto{\pgfqpoint{1.273791in}{2.250864in}}{\pgfqpoint{1.270519in}{2.258764in}}{\pgfqpoint{1.264695in}{2.264588in}}%
\pgfpathcurveto{\pgfqpoint{1.258871in}{2.270412in}}{\pgfqpoint{1.250971in}{2.273684in}}{\pgfqpoint{1.242735in}{2.273684in}}%
\pgfpathcurveto{\pgfqpoint{1.234499in}{2.273684in}}{\pgfqpoint{1.226598in}{2.270412in}}{\pgfqpoint{1.220775in}{2.264588in}}%
\pgfpathcurveto{\pgfqpoint{1.214951in}{2.258764in}}{\pgfqpoint{1.211678in}{2.250864in}}{\pgfqpoint{1.211678in}{2.242627in}}%
\pgfpathcurveto{\pgfqpoint{1.211678in}{2.234391in}}{\pgfqpoint{1.214951in}{2.226491in}}{\pgfqpoint{1.220775in}{2.220667in}}%
\pgfpathcurveto{\pgfqpoint{1.226598in}{2.214843in}}{\pgfqpoint{1.234499in}{2.211571in}}{\pgfqpoint{1.242735in}{2.211571in}}%
\pgfpathclose%
\pgfusepath{stroke,fill}%
\end{pgfscope}%
\begin{pgfscope}%
\pgfpathrectangle{\pgfqpoint{0.100000in}{0.212622in}}{\pgfqpoint{3.696000in}{3.696000in}}%
\pgfusepath{clip}%
\pgfsetbuttcap%
\pgfsetroundjoin%
\definecolor{currentfill}{rgb}{0.121569,0.466667,0.705882}%
\pgfsetfillcolor{currentfill}%
\pgfsetfillopacity{0.505098}%
\pgfsetlinewidth{1.003750pt}%
\definecolor{currentstroke}{rgb}{0.121569,0.466667,0.705882}%
\pgfsetstrokecolor{currentstroke}%
\pgfsetstrokeopacity{0.505098}%
\pgfsetdash{}{0pt}%
\pgfpathmoveto{\pgfqpoint{2.040739in}{2.474898in}}%
\pgfpathcurveto{\pgfqpoint{2.048975in}{2.474898in}}{\pgfqpoint{2.056875in}{2.478171in}}{\pgfqpoint{2.062699in}{2.483995in}}%
\pgfpathcurveto{\pgfqpoint{2.068523in}{2.489819in}}{\pgfqpoint{2.071796in}{2.497719in}}{\pgfqpoint{2.071796in}{2.505955in}}%
\pgfpathcurveto{\pgfqpoint{2.071796in}{2.514191in}}{\pgfqpoint{2.068523in}{2.522091in}}{\pgfqpoint{2.062699in}{2.527915in}}%
\pgfpathcurveto{\pgfqpoint{2.056875in}{2.533739in}}{\pgfqpoint{2.048975in}{2.537011in}}{\pgfqpoint{2.040739in}{2.537011in}}%
\pgfpathcurveto{\pgfqpoint{2.032503in}{2.537011in}}{\pgfqpoint{2.024603in}{2.533739in}}{\pgfqpoint{2.018779in}{2.527915in}}%
\pgfpathcurveto{\pgfqpoint{2.012955in}{2.522091in}}{\pgfqpoint{2.009683in}{2.514191in}}{\pgfqpoint{2.009683in}{2.505955in}}%
\pgfpathcurveto{\pgfqpoint{2.009683in}{2.497719in}}{\pgfqpoint{2.012955in}{2.489819in}}{\pgfqpoint{2.018779in}{2.483995in}}%
\pgfpathcurveto{\pgfqpoint{2.024603in}{2.478171in}}{\pgfqpoint{2.032503in}{2.474898in}}{\pgfqpoint{2.040739in}{2.474898in}}%
\pgfpathclose%
\pgfusepath{stroke,fill}%
\end{pgfscope}%
\begin{pgfscope}%
\pgfpathrectangle{\pgfqpoint{0.100000in}{0.212622in}}{\pgfqpoint{3.696000in}{3.696000in}}%
\pgfusepath{clip}%
\pgfsetbuttcap%
\pgfsetroundjoin%
\definecolor{currentfill}{rgb}{0.121569,0.466667,0.705882}%
\pgfsetfillcolor{currentfill}%
\pgfsetfillopacity{0.505590}%
\pgfsetlinewidth{1.003750pt}%
\definecolor{currentstroke}{rgb}{0.121569,0.466667,0.705882}%
\pgfsetstrokecolor{currentstroke}%
\pgfsetstrokeopacity{0.505590}%
\pgfsetdash{}{0pt}%
\pgfpathmoveto{\pgfqpoint{1.238859in}{2.204367in}}%
\pgfpathcurveto{\pgfqpoint{1.247095in}{2.204367in}}{\pgfqpoint{1.254995in}{2.207639in}}{\pgfqpoint{1.260819in}{2.213463in}}%
\pgfpathcurveto{\pgfqpoint{1.266643in}{2.219287in}}{\pgfqpoint{1.269915in}{2.227187in}}{\pgfqpoint{1.269915in}{2.235424in}}%
\pgfpathcurveto{\pgfqpoint{1.269915in}{2.243660in}}{\pgfqpoint{1.266643in}{2.251560in}}{\pgfqpoint{1.260819in}{2.257384in}}%
\pgfpathcurveto{\pgfqpoint{1.254995in}{2.263208in}}{\pgfqpoint{1.247095in}{2.266480in}}{\pgfqpoint{1.238859in}{2.266480in}}%
\pgfpathcurveto{\pgfqpoint{1.230623in}{2.266480in}}{\pgfqpoint{1.222723in}{2.263208in}}{\pgfqpoint{1.216899in}{2.257384in}}%
\pgfpathcurveto{\pgfqpoint{1.211075in}{2.251560in}}{\pgfqpoint{1.207802in}{2.243660in}}{\pgfqpoint{1.207802in}{2.235424in}}%
\pgfpathcurveto{\pgfqpoint{1.207802in}{2.227187in}}{\pgfqpoint{1.211075in}{2.219287in}}{\pgfqpoint{1.216899in}{2.213463in}}%
\pgfpathcurveto{\pgfqpoint{1.222723in}{2.207639in}}{\pgfqpoint{1.230623in}{2.204367in}}{\pgfqpoint{1.238859in}{2.204367in}}%
\pgfpathclose%
\pgfusepath{stroke,fill}%
\end{pgfscope}%
\begin{pgfscope}%
\pgfpathrectangle{\pgfqpoint{0.100000in}{0.212622in}}{\pgfqpoint{3.696000in}{3.696000in}}%
\pgfusepath{clip}%
\pgfsetbuttcap%
\pgfsetroundjoin%
\definecolor{currentfill}{rgb}{0.121569,0.466667,0.705882}%
\pgfsetfillcolor{currentfill}%
\pgfsetfillopacity{0.505990}%
\pgfsetlinewidth{1.003750pt}%
\definecolor{currentstroke}{rgb}{0.121569,0.466667,0.705882}%
\pgfsetstrokecolor{currentstroke}%
\pgfsetstrokeopacity{0.505990}%
\pgfsetdash{}{0pt}%
\pgfpathmoveto{\pgfqpoint{1.237431in}{2.201752in}}%
\pgfpathcurveto{\pgfqpoint{1.245667in}{2.201752in}}{\pgfqpoint{1.253567in}{2.205025in}}{\pgfqpoint{1.259391in}{2.210848in}}%
\pgfpathcurveto{\pgfqpoint{1.265215in}{2.216672in}}{\pgfqpoint{1.268487in}{2.224572in}}{\pgfqpoint{1.268487in}{2.232809in}}%
\pgfpathcurveto{\pgfqpoint{1.268487in}{2.241045in}}{\pgfqpoint{1.265215in}{2.248945in}}{\pgfqpoint{1.259391in}{2.254769in}}%
\pgfpathcurveto{\pgfqpoint{1.253567in}{2.260593in}}{\pgfqpoint{1.245667in}{2.263865in}}{\pgfqpoint{1.237431in}{2.263865in}}%
\pgfpathcurveto{\pgfqpoint{1.229195in}{2.263865in}}{\pgfqpoint{1.221295in}{2.260593in}}{\pgfqpoint{1.215471in}{2.254769in}}%
\pgfpathcurveto{\pgfqpoint{1.209647in}{2.248945in}}{\pgfqpoint{1.206374in}{2.241045in}}{\pgfqpoint{1.206374in}{2.232809in}}%
\pgfpathcurveto{\pgfqpoint{1.206374in}{2.224572in}}{\pgfqpoint{1.209647in}{2.216672in}}{\pgfqpoint{1.215471in}{2.210848in}}%
\pgfpathcurveto{\pgfqpoint{1.221295in}{2.205025in}}{\pgfqpoint{1.229195in}{2.201752in}}{\pgfqpoint{1.237431in}{2.201752in}}%
\pgfpathclose%
\pgfusepath{stroke,fill}%
\end{pgfscope}%
\begin{pgfscope}%
\pgfpathrectangle{\pgfqpoint{0.100000in}{0.212622in}}{\pgfqpoint{3.696000in}{3.696000in}}%
\pgfusepath{clip}%
\pgfsetbuttcap%
\pgfsetroundjoin%
\definecolor{currentfill}{rgb}{0.121569,0.466667,0.705882}%
\pgfsetfillcolor{currentfill}%
\pgfsetfillopacity{0.506089}%
\pgfsetlinewidth{1.003750pt}%
\definecolor{currentstroke}{rgb}{0.121569,0.466667,0.705882}%
\pgfsetstrokecolor{currentstroke}%
\pgfsetstrokeopacity{0.506089}%
\pgfsetdash{}{0pt}%
\pgfpathmoveto{\pgfqpoint{2.041580in}{2.469239in}}%
\pgfpathcurveto{\pgfqpoint{2.049816in}{2.469239in}}{\pgfqpoint{2.057716in}{2.472512in}}{\pgfqpoint{2.063540in}{2.478336in}}%
\pgfpathcurveto{\pgfqpoint{2.069364in}{2.484160in}}{\pgfqpoint{2.072636in}{2.492060in}}{\pgfqpoint{2.072636in}{2.500296in}}%
\pgfpathcurveto{\pgfqpoint{2.072636in}{2.508532in}}{\pgfqpoint{2.069364in}{2.516432in}}{\pgfqpoint{2.063540in}{2.522256in}}%
\pgfpathcurveto{\pgfqpoint{2.057716in}{2.528080in}}{\pgfqpoint{2.049816in}{2.531352in}}{\pgfqpoint{2.041580in}{2.531352in}}%
\pgfpathcurveto{\pgfqpoint{2.033343in}{2.531352in}}{\pgfqpoint{2.025443in}{2.528080in}}{\pgfqpoint{2.019619in}{2.522256in}}%
\pgfpathcurveto{\pgfqpoint{2.013795in}{2.516432in}}{\pgfqpoint{2.010523in}{2.508532in}}{\pgfqpoint{2.010523in}{2.500296in}}%
\pgfpathcurveto{\pgfqpoint{2.010523in}{2.492060in}}{\pgfqpoint{2.013795in}{2.484160in}}{\pgfqpoint{2.019619in}{2.478336in}}%
\pgfpathcurveto{\pgfqpoint{2.025443in}{2.472512in}}{\pgfqpoint{2.033343in}{2.469239in}}{\pgfqpoint{2.041580in}{2.469239in}}%
\pgfpathclose%
\pgfusepath{stroke,fill}%
\end{pgfscope}%
\begin{pgfscope}%
\pgfpathrectangle{\pgfqpoint{0.100000in}{0.212622in}}{\pgfqpoint{3.696000in}{3.696000in}}%
\pgfusepath{clip}%
\pgfsetbuttcap%
\pgfsetroundjoin%
\definecolor{currentfill}{rgb}{0.121569,0.466667,0.705882}%
\pgfsetfillcolor{currentfill}%
\pgfsetfillopacity{0.506278}%
\pgfsetlinewidth{1.003750pt}%
\definecolor{currentstroke}{rgb}{0.121569,0.466667,0.705882}%
\pgfsetstrokecolor{currentstroke}%
\pgfsetstrokeopacity{0.506278}%
\pgfsetdash{}{0pt}%
\pgfpathmoveto{\pgfqpoint{1.236251in}{2.199264in}}%
\pgfpathcurveto{\pgfqpoint{1.244488in}{2.199264in}}{\pgfqpoint{1.252388in}{2.202537in}}{\pgfqpoint{1.258212in}{2.208361in}}%
\pgfpathcurveto{\pgfqpoint{1.264036in}{2.214184in}}{\pgfqpoint{1.267308in}{2.222085in}}{\pgfqpoint{1.267308in}{2.230321in}}%
\pgfpathcurveto{\pgfqpoint{1.267308in}{2.238557in}}{\pgfqpoint{1.264036in}{2.246457in}}{\pgfqpoint{1.258212in}{2.252281in}}%
\pgfpathcurveto{\pgfqpoint{1.252388in}{2.258105in}}{\pgfqpoint{1.244488in}{2.261377in}}{\pgfqpoint{1.236251in}{2.261377in}}%
\pgfpathcurveto{\pgfqpoint{1.228015in}{2.261377in}}{\pgfqpoint{1.220115in}{2.258105in}}{\pgfqpoint{1.214291in}{2.252281in}}%
\pgfpathcurveto{\pgfqpoint{1.208467in}{2.246457in}}{\pgfqpoint{1.205195in}{2.238557in}}{\pgfqpoint{1.205195in}{2.230321in}}%
\pgfpathcurveto{\pgfqpoint{1.205195in}{2.222085in}}{\pgfqpoint{1.208467in}{2.214184in}}{\pgfqpoint{1.214291in}{2.208361in}}%
\pgfpathcurveto{\pgfqpoint{1.220115in}{2.202537in}}{\pgfqpoint{1.228015in}{2.199264in}}{\pgfqpoint{1.236251in}{2.199264in}}%
\pgfpathclose%
\pgfusepath{stroke,fill}%
\end{pgfscope}%
\begin{pgfscope}%
\pgfpathrectangle{\pgfqpoint{0.100000in}{0.212622in}}{\pgfqpoint{3.696000in}{3.696000in}}%
\pgfusepath{clip}%
\pgfsetbuttcap%
\pgfsetroundjoin%
\definecolor{currentfill}{rgb}{0.121569,0.466667,0.705882}%
\pgfsetfillcolor{currentfill}%
\pgfsetfillopacity{0.506629}%
\pgfsetlinewidth{1.003750pt}%
\definecolor{currentstroke}{rgb}{0.121569,0.466667,0.705882}%
\pgfsetstrokecolor{currentstroke}%
\pgfsetstrokeopacity{0.506629}%
\pgfsetdash{}{0pt}%
\pgfpathmoveto{\pgfqpoint{1.235256in}{2.197725in}}%
\pgfpathcurveto{\pgfqpoint{1.243492in}{2.197725in}}{\pgfqpoint{1.251392in}{2.200997in}}{\pgfqpoint{1.257216in}{2.206821in}}%
\pgfpathcurveto{\pgfqpoint{1.263040in}{2.212645in}}{\pgfqpoint{1.266312in}{2.220545in}}{\pgfqpoint{1.266312in}{2.228781in}}%
\pgfpathcurveto{\pgfqpoint{1.266312in}{2.237018in}}{\pgfqpoint{1.263040in}{2.244918in}}{\pgfqpoint{1.257216in}{2.250742in}}%
\pgfpathcurveto{\pgfqpoint{1.251392in}{2.256565in}}{\pgfqpoint{1.243492in}{2.259838in}}{\pgfqpoint{1.235256in}{2.259838in}}%
\pgfpathcurveto{\pgfqpoint{1.227019in}{2.259838in}}{\pgfqpoint{1.219119in}{2.256565in}}{\pgfqpoint{1.213295in}{2.250742in}}%
\pgfpathcurveto{\pgfqpoint{1.207472in}{2.244918in}}{\pgfqpoint{1.204199in}{2.237018in}}{\pgfqpoint{1.204199in}{2.228781in}}%
\pgfpathcurveto{\pgfqpoint{1.204199in}{2.220545in}}{\pgfqpoint{1.207472in}{2.212645in}}{\pgfqpoint{1.213295in}{2.206821in}}%
\pgfpathcurveto{\pgfqpoint{1.219119in}{2.200997in}}{\pgfqpoint{1.227019in}{2.197725in}}{\pgfqpoint{1.235256in}{2.197725in}}%
\pgfpathclose%
\pgfusepath{stroke,fill}%
\end{pgfscope}%
\begin{pgfscope}%
\pgfpathrectangle{\pgfqpoint{0.100000in}{0.212622in}}{\pgfqpoint{3.696000in}{3.696000in}}%
\pgfusepath{clip}%
\pgfsetbuttcap%
\pgfsetroundjoin%
\definecolor{currentfill}{rgb}{0.121569,0.466667,0.705882}%
\pgfsetfillcolor{currentfill}%
\pgfsetfillopacity{0.507284}%
\pgfsetlinewidth{1.003750pt}%
\definecolor{currentstroke}{rgb}{0.121569,0.466667,0.705882}%
\pgfsetstrokecolor{currentstroke}%
\pgfsetstrokeopacity{0.507284}%
\pgfsetdash{}{0pt}%
\pgfpathmoveto{\pgfqpoint{1.233482in}{2.194925in}}%
\pgfpathcurveto{\pgfqpoint{1.241718in}{2.194925in}}{\pgfqpoint{1.249618in}{2.198198in}}{\pgfqpoint{1.255442in}{2.204022in}}%
\pgfpathcurveto{\pgfqpoint{1.261266in}{2.209846in}}{\pgfqpoint{1.264539in}{2.217746in}}{\pgfqpoint{1.264539in}{2.225982in}}%
\pgfpathcurveto{\pgfqpoint{1.264539in}{2.234218in}}{\pgfqpoint{1.261266in}{2.242118in}}{\pgfqpoint{1.255442in}{2.247942in}}%
\pgfpathcurveto{\pgfqpoint{1.249618in}{2.253766in}}{\pgfqpoint{1.241718in}{2.257038in}}{\pgfqpoint{1.233482in}{2.257038in}}%
\pgfpathcurveto{\pgfqpoint{1.225246in}{2.257038in}}{\pgfqpoint{1.217346in}{2.253766in}}{\pgfqpoint{1.211522in}{2.247942in}}%
\pgfpathcurveto{\pgfqpoint{1.205698in}{2.242118in}}{\pgfqpoint{1.202426in}{2.234218in}}{\pgfqpoint{1.202426in}{2.225982in}}%
\pgfpathcurveto{\pgfqpoint{1.202426in}{2.217746in}}{\pgfqpoint{1.205698in}{2.209846in}}{\pgfqpoint{1.211522in}{2.204022in}}%
\pgfpathcurveto{\pgfqpoint{1.217346in}{2.198198in}}{\pgfqpoint{1.225246in}{2.194925in}}{\pgfqpoint{1.233482in}{2.194925in}}%
\pgfpathclose%
\pgfusepath{stroke,fill}%
\end{pgfscope}%
\begin{pgfscope}%
\pgfpathrectangle{\pgfqpoint{0.100000in}{0.212622in}}{\pgfqpoint{3.696000in}{3.696000in}}%
\pgfusepath{clip}%
\pgfsetbuttcap%
\pgfsetroundjoin%
\definecolor{currentfill}{rgb}{0.121569,0.466667,0.705882}%
\pgfsetfillcolor{currentfill}%
\pgfsetfillopacity{0.507387}%
\pgfsetlinewidth{1.003750pt}%
\definecolor{currentstroke}{rgb}{0.121569,0.466667,0.705882}%
\pgfsetstrokecolor{currentstroke}%
\pgfsetstrokeopacity{0.507387}%
\pgfsetdash{}{0pt}%
\pgfpathmoveto{\pgfqpoint{2.042131in}{2.462482in}}%
\pgfpathcurveto{\pgfqpoint{2.050367in}{2.462482in}}{\pgfqpoint{2.058267in}{2.465754in}}{\pgfqpoint{2.064091in}{2.471578in}}%
\pgfpathcurveto{\pgfqpoint{2.069915in}{2.477402in}}{\pgfqpoint{2.073187in}{2.485302in}}{\pgfqpoint{2.073187in}{2.493538in}}%
\pgfpathcurveto{\pgfqpoint{2.073187in}{2.501775in}}{\pgfqpoint{2.069915in}{2.509675in}}{\pgfqpoint{2.064091in}{2.515499in}}%
\pgfpathcurveto{\pgfqpoint{2.058267in}{2.521322in}}{\pgfqpoint{2.050367in}{2.524595in}}{\pgfqpoint{2.042131in}{2.524595in}}%
\pgfpathcurveto{\pgfqpoint{2.033895in}{2.524595in}}{\pgfqpoint{2.025995in}{2.521322in}}{\pgfqpoint{2.020171in}{2.515499in}}%
\pgfpathcurveto{\pgfqpoint{2.014347in}{2.509675in}}{\pgfqpoint{2.011074in}{2.501775in}}{\pgfqpoint{2.011074in}{2.493538in}}%
\pgfpathcurveto{\pgfqpoint{2.011074in}{2.485302in}}{\pgfqpoint{2.014347in}{2.477402in}}{\pgfqpoint{2.020171in}{2.471578in}}%
\pgfpathcurveto{\pgfqpoint{2.025995in}{2.465754in}}{\pgfqpoint{2.033895in}{2.462482in}}{\pgfqpoint{2.042131in}{2.462482in}}%
\pgfpathclose%
\pgfusepath{stroke,fill}%
\end{pgfscope}%
\begin{pgfscope}%
\pgfpathrectangle{\pgfqpoint{0.100000in}{0.212622in}}{\pgfqpoint{3.696000in}{3.696000in}}%
\pgfusepath{clip}%
\pgfsetbuttcap%
\pgfsetroundjoin%
\definecolor{currentfill}{rgb}{0.121569,0.466667,0.705882}%
\pgfsetfillcolor{currentfill}%
\pgfsetfillopacity{0.507855}%
\pgfsetlinewidth{1.003750pt}%
\definecolor{currentstroke}{rgb}{0.121569,0.466667,0.705882}%
\pgfsetstrokecolor{currentstroke}%
\pgfsetstrokeopacity{0.507855}%
\pgfsetdash{}{0pt}%
\pgfpathmoveto{\pgfqpoint{1.232011in}{2.192313in}}%
\pgfpathcurveto{\pgfqpoint{1.240247in}{2.192313in}}{\pgfqpoint{1.248147in}{2.195586in}}{\pgfqpoint{1.253971in}{2.201410in}}%
\pgfpathcurveto{\pgfqpoint{1.259795in}{2.207234in}}{\pgfqpoint{1.263068in}{2.215134in}}{\pgfqpoint{1.263068in}{2.223370in}}%
\pgfpathcurveto{\pgfqpoint{1.263068in}{2.231606in}}{\pgfqpoint{1.259795in}{2.239506in}}{\pgfqpoint{1.253971in}{2.245330in}}%
\pgfpathcurveto{\pgfqpoint{1.248147in}{2.251154in}}{\pgfqpoint{1.240247in}{2.254426in}}{\pgfqpoint{1.232011in}{2.254426in}}%
\pgfpathcurveto{\pgfqpoint{1.223775in}{2.254426in}}{\pgfqpoint{1.215875in}{2.251154in}}{\pgfqpoint{1.210051in}{2.245330in}}%
\pgfpathcurveto{\pgfqpoint{1.204227in}{2.239506in}}{\pgfqpoint{1.200955in}{2.231606in}}{\pgfqpoint{1.200955in}{2.223370in}}%
\pgfpathcurveto{\pgfqpoint{1.200955in}{2.215134in}}{\pgfqpoint{1.204227in}{2.207234in}}{\pgfqpoint{1.210051in}{2.201410in}}%
\pgfpathcurveto{\pgfqpoint{1.215875in}{2.195586in}}{\pgfqpoint{1.223775in}{2.192313in}}{\pgfqpoint{1.232011in}{2.192313in}}%
\pgfpathclose%
\pgfusepath{stroke,fill}%
\end{pgfscope}%
\begin{pgfscope}%
\pgfpathrectangle{\pgfqpoint{0.100000in}{0.212622in}}{\pgfqpoint{3.696000in}{3.696000in}}%
\pgfusepath{clip}%
\pgfsetbuttcap%
\pgfsetroundjoin%
\definecolor{currentfill}{rgb}{0.121569,0.466667,0.705882}%
\pgfsetfillcolor{currentfill}%
\pgfsetfillopacity{0.508830}%
\pgfsetlinewidth{1.003750pt}%
\definecolor{currentstroke}{rgb}{0.121569,0.466667,0.705882}%
\pgfsetstrokecolor{currentstroke}%
\pgfsetstrokeopacity{0.508830}%
\pgfsetdash{}{0pt}%
\pgfpathmoveto{\pgfqpoint{1.229107in}{2.187626in}}%
\pgfpathcurveto{\pgfqpoint{1.237343in}{2.187626in}}{\pgfqpoint{1.245243in}{2.190898in}}{\pgfqpoint{1.251067in}{2.196722in}}%
\pgfpathcurveto{\pgfqpoint{1.256891in}{2.202546in}}{\pgfqpoint{1.260163in}{2.210446in}}{\pgfqpoint{1.260163in}{2.218682in}}%
\pgfpathcurveto{\pgfqpoint{1.260163in}{2.226918in}}{\pgfqpoint{1.256891in}{2.234818in}}{\pgfqpoint{1.251067in}{2.240642in}}%
\pgfpathcurveto{\pgfqpoint{1.245243in}{2.246466in}}{\pgfqpoint{1.237343in}{2.249739in}}{\pgfqpoint{1.229107in}{2.249739in}}%
\pgfpathcurveto{\pgfqpoint{1.220870in}{2.249739in}}{\pgfqpoint{1.212970in}{2.246466in}}{\pgfqpoint{1.207146in}{2.240642in}}%
\pgfpathcurveto{\pgfqpoint{1.201322in}{2.234818in}}{\pgfqpoint{1.198050in}{2.226918in}}{\pgfqpoint{1.198050in}{2.218682in}}%
\pgfpathcurveto{\pgfqpoint{1.198050in}{2.210446in}}{\pgfqpoint{1.201322in}{2.202546in}}{\pgfqpoint{1.207146in}{2.196722in}}%
\pgfpathcurveto{\pgfqpoint{1.212970in}{2.190898in}}{\pgfqpoint{1.220870in}{2.187626in}}{\pgfqpoint{1.229107in}{2.187626in}}%
\pgfpathclose%
\pgfusepath{stroke,fill}%
\end{pgfscope}%
\begin{pgfscope}%
\pgfpathrectangle{\pgfqpoint{0.100000in}{0.212622in}}{\pgfqpoint{3.696000in}{3.696000in}}%
\pgfusepath{clip}%
\pgfsetbuttcap%
\pgfsetroundjoin%
\definecolor{currentfill}{rgb}{0.121569,0.466667,0.705882}%
\pgfsetfillcolor{currentfill}%
\pgfsetfillopacity{0.508915}%
\pgfsetlinewidth{1.003750pt}%
\definecolor{currentstroke}{rgb}{0.121569,0.466667,0.705882}%
\pgfsetstrokecolor{currentstroke}%
\pgfsetstrokeopacity{0.508915}%
\pgfsetdash{}{0pt}%
\pgfpathmoveto{\pgfqpoint{2.042686in}{2.455221in}}%
\pgfpathcurveto{\pgfqpoint{2.050922in}{2.455221in}}{\pgfqpoint{2.058822in}{2.458494in}}{\pgfqpoint{2.064646in}{2.464318in}}%
\pgfpathcurveto{\pgfqpoint{2.070470in}{2.470142in}}{\pgfqpoint{2.073742in}{2.478042in}}{\pgfqpoint{2.073742in}{2.486278in}}%
\pgfpathcurveto{\pgfqpoint{2.073742in}{2.494514in}}{\pgfqpoint{2.070470in}{2.502414in}}{\pgfqpoint{2.064646in}{2.508238in}}%
\pgfpathcurveto{\pgfqpoint{2.058822in}{2.514062in}}{\pgfqpoint{2.050922in}{2.517334in}}{\pgfqpoint{2.042686in}{2.517334in}}%
\pgfpathcurveto{\pgfqpoint{2.034449in}{2.517334in}}{\pgfqpoint{2.026549in}{2.514062in}}{\pgfqpoint{2.020725in}{2.508238in}}%
\pgfpathcurveto{\pgfqpoint{2.014902in}{2.502414in}}{\pgfqpoint{2.011629in}{2.494514in}}{\pgfqpoint{2.011629in}{2.486278in}}%
\pgfpathcurveto{\pgfqpoint{2.011629in}{2.478042in}}{\pgfqpoint{2.014902in}{2.470142in}}{\pgfqpoint{2.020725in}{2.464318in}}%
\pgfpathcurveto{\pgfqpoint{2.026549in}{2.458494in}}{\pgfqpoint{2.034449in}{2.455221in}}{\pgfqpoint{2.042686in}{2.455221in}}%
\pgfpathclose%
\pgfusepath{stroke,fill}%
\end{pgfscope}%
\begin{pgfscope}%
\pgfpathrectangle{\pgfqpoint{0.100000in}{0.212622in}}{\pgfqpoint{3.696000in}{3.696000in}}%
\pgfusepath{clip}%
\pgfsetbuttcap%
\pgfsetroundjoin%
\definecolor{currentfill}{rgb}{0.121569,0.466667,0.705882}%
\pgfsetfillcolor{currentfill}%
\pgfsetfillopacity{0.509730}%
\pgfsetlinewidth{1.003750pt}%
\definecolor{currentstroke}{rgb}{0.121569,0.466667,0.705882}%
\pgfsetstrokecolor{currentstroke}%
\pgfsetstrokeopacity{0.509730}%
\pgfsetdash{}{0pt}%
\pgfpathmoveto{\pgfqpoint{1.226591in}{2.183177in}}%
\pgfpathcurveto{\pgfqpoint{1.234827in}{2.183177in}}{\pgfqpoint{1.242727in}{2.186449in}}{\pgfqpoint{1.248551in}{2.192273in}}%
\pgfpathcurveto{\pgfqpoint{1.254375in}{2.198097in}}{\pgfqpoint{1.257648in}{2.205997in}}{\pgfqpoint{1.257648in}{2.214233in}}%
\pgfpathcurveto{\pgfqpoint{1.257648in}{2.222470in}}{\pgfqpoint{1.254375in}{2.230370in}}{\pgfqpoint{1.248551in}{2.236194in}}%
\pgfpathcurveto{\pgfqpoint{1.242727in}{2.242018in}}{\pgfqpoint{1.234827in}{2.245290in}}{\pgfqpoint{1.226591in}{2.245290in}}%
\pgfpathcurveto{\pgfqpoint{1.218355in}{2.245290in}}{\pgfqpoint{1.210455in}{2.242018in}}{\pgfqpoint{1.204631in}{2.236194in}}%
\pgfpathcurveto{\pgfqpoint{1.198807in}{2.230370in}}{\pgfqpoint{1.195535in}{2.222470in}}{\pgfqpoint{1.195535in}{2.214233in}}%
\pgfpathcurveto{\pgfqpoint{1.195535in}{2.205997in}}{\pgfqpoint{1.198807in}{2.198097in}}{\pgfqpoint{1.204631in}{2.192273in}}%
\pgfpathcurveto{\pgfqpoint{1.210455in}{2.186449in}}{\pgfqpoint{1.218355in}{2.183177in}}{\pgfqpoint{1.226591in}{2.183177in}}%
\pgfpathclose%
\pgfusepath{stroke,fill}%
\end{pgfscope}%
\begin{pgfscope}%
\pgfpathrectangle{\pgfqpoint{0.100000in}{0.212622in}}{\pgfqpoint{3.696000in}{3.696000in}}%
\pgfusepath{clip}%
\pgfsetbuttcap%
\pgfsetroundjoin%
\definecolor{currentfill}{rgb}{0.121569,0.466667,0.705882}%
\pgfsetfillcolor{currentfill}%
\pgfsetfillopacity{0.510386}%
\pgfsetlinewidth{1.003750pt}%
\definecolor{currentstroke}{rgb}{0.121569,0.466667,0.705882}%
\pgfsetstrokecolor{currentstroke}%
\pgfsetstrokeopacity{0.510386}%
\pgfsetdash{}{0pt}%
\pgfpathmoveto{\pgfqpoint{2.044060in}{2.446159in}}%
\pgfpathcurveto{\pgfqpoint{2.052296in}{2.446159in}}{\pgfqpoint{2.060196in}{2.449431in}}{\pgfqpoint{2.066020in}{2.455255in}}%
\pgfpathcurveto{\pgfqpoint{2.071844in}{2.461079in}}{\pgfqpoint{2.075116in}{2.468979in}}{\pgfqpoint{2.075116in}{2.477216in}}%
\pgfpathcurveto{\pgfqpoint{2.075116in}{2.485452in}}{\pgfqpoint{2.071844in}{2.493352in}}{\pgfqpoint{2.066020in}{2.499176in}}%
\pgfpathcurveto{\pgfqpoint{2.060196in}{2.505000in}}{\pgfqpoint{2.052296in}{2.508272in}}{\pgfqpoint{2.044060in}{2.508272in}}%
\pgfpathcurveto{\pgfqpoint{2.035824in}{2.508272in}}{\pgfqpoint{2.027924in}{2.505000in}}{\pgfqpoint{2.022100in}{2.499176in}}%
\pgfpathcurveto{\pgfqpoint{2.016276in}{2.493352in}}{\pgfqpoint{2.013003in}{2.485452in}}{\pgfqpoint{2.013003in}{2.477216in}}%
\pgfpathcurveto{\pgfqpoint{2.013003in}{2.468979in}}{\pgfqpoint{2.016276in}{2.461079in}}{\pgfqpoint{2.022100in}{2.455255in}}%
\pgfpathcurveto{\pgfqpoint{2.027924in}{2.449431in}}{\pgfqpoint{2.035824in}{2.446159in}}{\pgfqpoint{2.044060in}{2.446159in}}%
\pgfpathclose%
\pgfusepath{stroke,fill}%
\end{pgfscope}%
\begin{pgfscope}%
\pgfpathrectangle{\pgfqpoint{0.100000in}{0.212622in}}{\pgfqpoint{3.696000in}{3.696000in}}%
\pgfusepath{clip}%
\pgfsetbuttcap%
\pgfsetroundjoin%
\definecolor{currentfill}{rgb}{0.121569,0.466667,0.705882}%
\pgfsetfillcolor{currentfill}%
\pgfsetfillopacity{0.511235}%
\pgfsetlinewidth{1.003750pt}%
\definecolor{currentstroke}{rgb}{0.121569,0.466667,0.705882}%
\pgfsetstrokecolor{currentstroke}%
\pgfsetstrokeopacity{0.511235}%
\pgfsetdash{}{0pt}%
\pgfpathmoveto{\pgfqpoint{1.221684in}{2.175071in}}%
\pgfpathcurveto{\pgfqpoint{1.229920in}{2.175071in}}{\pgfqpoint{1.237820in}{2.178343in}}{\pgfqpoint{1.243644in}{2.184167in}}%
\pgfpathcurveto{\pgfqpoint{1.249468in}{2.189991in}}{\pgfqpoint{1.252740in}{2.197891in}}{\pgfqpoint{1.252740in}{2.206127in}}%
\pgfpathcurveto{\pgfqpoint{1.252740in}{2.214364in}}{\pgfqpoint{1.249468in}{2.222264in}}{\pgfqpoint{1.243644in}{2.228088in}}%
\pgfpathcurveto{\pgfqpoint{1.237820in}{2.233911in}}{\pgfqpoint{1.229920in}{2.237184in}}{\pgfqpoint{1.221684in}{2.237184in}}%
\pgfpathcurveto{\pgfqpoint{1.213447in}{2.237184in}}{\pgfqpoint{1.205547in}{2.233911in}}{\pgfqpoint{1.199723in}{2.228088in}}%
\pgfpathcurveto{\pgfqpoint{1.193900in}{2.222264in}}{\pgfqpoint{1.190627in}{2.214364in}}{\pgfqpoint{1.190627in}{2.206127in}}%
\pgfpathcurveto{\pgfqpoint{1.190627in}{2.197891in}}{\pgfqpoint{1.193900in}{2.189991in}}{\pgfqpoint{1.199723in}{2.184167in}}%
\pgfpathcurveto{\pgfqpoint{1.205547in}{2.178343in}}{\pgfqpoint{1.213447in}{2.175071in}}{\pgfqpoint{1.221684in}{2.175071in}}%
\pgfpathclose%
\pgfusepath{stroke,fill}%
\end{pgfscope}%
\begin{pgfscope}%
\pgfpathrectangle{\pgfqpoint{0.100000in}{0.212622in}}{\pgfqpoint{3.696000in}{3.696000in}}%
\pgfusepath{clip}%
\pgfsetbuttcap%
\pgfsetroundjoin%
\definecolor{currentfill}{rgb}{0.121569,0.466667,0.705882}%
\pgfsetfillcolor{currentfill}%
\pgfsetfillopacity{0.512194}%
\pgfsetlinewidth{1.003750pt}%
\definecolor{currentstroke}{rgb}{0.121569,0.466667,0.705882}%
\pgfsetstrokecolor{currentstroke}%
\pgfsetstrokeopacity{0.512194}%
\pgfsetdash{}{0pt}%
\pgfpathmoveto{\pgfqpoint{2.045079in}{2.436501in}}%
\pgfpathcurveto{\pgfqpoint{2.053315in}{2.436501in}}{\pgfqpoint{2.061215in}{2.439774in}}{\pgfqpoint{2.067039in}{2.445597in}}%
\pgfpathcurveto{\pgfqpoint{2.072863in}{2.451421in}}{\pgfqpoint{2.076135in}{2.459321in}}{\pgfqpoint{2.076135in}{2.467558in}}%
\pgfpathcurveto{\pgfqpoint{2.076135in}{2.475794in}}{\pgfqpoint{2.072863in}{2.483694in}}{\pgfqpoint{2.067039in}{2.489518in}}%
\pgfpathcurveto{\pgfqpoint{2.061215in}{2.495342in}}{\pgfqpoint{2.053315in}{2.498614in}}{\pgfqpoint{2.045079in}{2.498614in}}%
\pgfpathcurveto{\pgfqpoint{2.036843in}{2.498614in}}{\pgfqpoint{2.028943in}{2.495342in}}{\pgfqpoint{2.023119in}{2.489518in}}%
\pgfpathcurveto{\pgfqpoint{2.017295in}{2.483694in}}{\pgfqpoint{2.014022in}{2.475794in}}{\pgfqpoint{2.014022in}{2.467558in}}%
\pgfpathcurveto{\pgfqpoint{2.014022in}{2.459321in}}{\pgfqpoint{2.017295in}{2.451421in}}{\pgfqpoint{2.023119in}{2.445597in}}%
\pgfpathcurveto{\pgfqpoint{2.028943in}{2.439774in}}{\pgfqpoint{2.036843in}{2.436501in}}{\pgfqpoint{2.045079in}{2.436501in}}%
\pgfpathclose%
\pgfusepath{stroke,fill}%
\end{pgfscope}%
\begin{pgfscope}%
\pgfpathrectangle{\pgfqpoint{0.100000in}{0.212622in}}{\pgfqpoint{3.696000in}{3.696000in}}%
\pgfusepath{clip}%
\pgfsetbuttcap%
\pgfsetroundjoin%
\definecolor{currentfill}{rgb}{0.121569,0.466667,0.705882}%
\pgfsetfillcolor{currentfill}%
\pgfsetfillopacity{0.512651}%
\pgfsetlinewidth{1.003750pt}%
\definecolor{currentstroke}{rgb}{0.121569,0.466667,0.705882}%
\pgfsetstrokecolor{currentstroke}%
\pgfsetstrokeopacity{0.512651}%
\pgfsetdash{}{0pt}%
\pgfpathmoveto{\pgfqpoint{1.218162in}{2.167202in}}%
\pgfpathcurveto{\pgfqpoint{1.226399in}{2.167202in}}{\pgfqpoint{1.234299in}{2.170474in}}{\pgfqpoint{1.240123in}{2.176298in}}%
\pgfpathcurveto{\pgfqpoint{1.245947in}{2.182122in}}{\pgfqpoint{1.249219in}{2.190022in}}{\pgfqpoint{1.249219in}{2.198259in}}%
\pgfpathcurveto{\pgfqpoint{1.249219in}{2.206495in}}{\pgfqpoint{1.245947in}{2.214395in}}{\pgfqpoint{1.240123in}{2.220219in}}%
\pgfpathcurveto{\pgfqpoint{1.234299in}{2.226043in}}{\pgfqpoint{1.226399in}{2.229315in}}{\pgfqpoint{1.218162in}{2.229315in}}%
\pgfpathcurveto{\pgfqpoint{1.209926in}{2.229315in}}{\pgfqpoint{1.202026in}{2.226043in}}{\pgfqpoint{1.196202in}{2.220219in}}%
\pgfpathcurveto{\pgfqpoint{1.190378in}{2.214395in}}{\pgfqpoint{1.187106in}{2.206495in}}{\pgfqpoint{1.187106in}{2.198259in}}%
\pgfpathcurveto{\pgfqpoint{1.187106in}{2.190022in}}{\pgfqpoint{1.190378in}{2.182122in}}{\pgfqpoint{1.196202in}{2.176298in}}%
\pgfpathcurveto{\pgfqpoint{1.202026in}{2.170474in}}{\pgfqpoint{1.209926in}{2.167202in}}{\pgfqpoint{1.218162in}{2.167202in}}%
\pgfpathclose%
\pgfusepath{stroke,fill}%
\end{pgfscope}%
\begin{pgfscope}%
\pgfpathrectangle{\pgfqpoint{0.100000in}{0.212622in}}{\pgfqpoint{3.696000in}{3.696000in}}%
\pgfusepath{clip}%
\pgfsetbuttcap%
\pgfsetroundjoin%
\definecolor{currentfill}{rgb}{0.121569,0.466667,0.705882}%
\pgfsetfillcolor{currentfill}%
\pgfsetfillopacity{0.513903}%
\pgfsetlinewidth{1.003750pt}%
\definecolor{currentstroke}{rgb}{0.121569,0.466667,0.705882}%
\pgfsetstrokecolor{currentstroke}%
\pgfsetstrokeopacity{0.513903}%
\pgfsetdash{}{0pt}%
\pgfpathmoveto{\pgfqpoint{1.214291in}{2.161049in}}%
\pgfpathcurveto{\pgfqpoint{1.222527in}{2.161049in}}{\pgfqpoint{1.230427in}{2.164321in}}{\pgfqpoint{1.236251in}{2.170145in}}%
\pgfpathcurveto{\pgfqpoint{1.242075in}{2.175969in}}{\pgfqpoint{1.245348in}{2.183869in}}{\pgfqpoint{1.245348in}{2.192105in}}%
\pgfpathcurveto{\pgfqpoint{1.245348in}{2.200341in}}{\pgfqpoint{1.242075in}{2.208241in}}{\pgfqpoint{1.236251in}{2.214065in}}%
\pgfpathcurveto{\pgfqpoint{1.230427in}{2.219889in}}{\pgfqpoint{1.222527in}{2.223162in}}{\pgfqpoint{1.214291in}{2.223162in}}%
\pgfpathcurveto{\pgfqpoint{1.206055in}{2.223162in}}{\pgfqpoint{1.198155in}{2.219889in}}{\pgfqpoint{1.192331in}{2.214065in}}%
\pgfpathcurveto{\pgfqpoint{1.186507in}{2.208241in}}{\pgfqpoint{1.183235in}{2.200341in}}{\pgfqpoint{1.183235in}{2.192105in}}%
\pgfpathcurveto{\pgfqpoint{1.183235in}{2.183869in}}{\pgfqpoint{1.186507in}{2.175969in}}{\pgfqpoint{1.192331in}{2.170145in}}%
\pgfpathcurveto{\pgfqpoint{1.198155in}{2.164321in}}{\pgfqpoint{1.206055in}{2.161049in}}{\pgfqpoint{1.214291in}{2.161049in}}%
\pgfpathclose%
\pgfusepath{stroke,fill}%
\end{pgfscope}%
\begin{pgfscope}%
\pgfpathrectangle{\pgfqpoint{0.100000in}{0.212622in}}{\pgfqpoint{3.696000in}{3.696000in}}%
\pgfusepath{clip}%
\pgfsetbuttcap%
\pgfsetroundjoin%
\definecolor{currentfill}{rgb}{0.121569,0.466667,0.705882}%
\pgfsetfillcolor{currentfill}%
\pgfsetfillopacity{0.514426}%
\pgfsetlinewidth{1.003750pt}%
\definecolor{currentstroke}{rgb}{0.121569,0.466667,0.705882}%
\pgfsetstrokecolor{currentstroke}%
\pgfsetstrokeopacity{0.514426}%
\pgfsetdash{}{0pt}%
\pgfpathmoveto{\pgfqpoint{2.045966in}{2.427063in}}%
\pgfpathcurveto{\pgfqpoint{2.054202in}{2.427063in}}{\pgfqpoint{2.062102in}{2.430335in}}{\pgfqpoint{2.067926in}{2.436159in}}%
\pgfpathcurveto{\pgfqpoint{2.073750in}{2.441983in}}{\pgfqpoint{2.077023in}{2.449883in}}{\pgfqpoint{2.077023in}{2.458120in}}%
\pgfpathcurveto{\pgfqpoint{2.077023in}{2.466356in}}{\pgfqpoint{2.073750in}{2.474256in}}{\pgfqpoint{2.067926in}{2.480080in}}%
\pgfpathcurveto{\pgfqpoint{2.062102in}{2.485904in}}{\pgfqpoint{2.054202in}{2.489176in}}{\pgfqpoint{2.045966in}{2.489176in}}%
\pgfpathcurveto{\pgfqpoint{2.037730in}{2.489176in}}{\pgfqpoint{2.029830in}{2.485904in}}{\pgfqpoint{2.024006in}{2.480080in}}%
\pgfpathcurveto{\pgfqpoint{2.018182in}{2.474256in}}{\pgfqpoint{2.014910in}{2.466356in}}{\pgfqpoint{2.014910in}{2.458120in}}%
\pgfpathcurveto{\pgfqpoint{2.014910in}{2.449883in}}{\pgfqpoint{2.018182in}{2.441983in}}{\pgfqpoint{2.024006in}{2.436159in}}%
\pgfpathcurveto{\pgfqpoint{2.029830in}{2.430335in}}{\pgfqpoint{2.037730in}{2.427063in}}{\pgfqpoint{2.045966in}{2.427063in}}%
\pgfpathclose%
\pgfusepath{stroke,fill}%
\end{pgfscope}%
\begin{pgfscope}%
\pgfpathrectangle{\pgfqpoint{0.100000in}{0.212622in}}{\pgfqpoint{3.696000in}{3.696000in}}%
\pgfusepath{clip}%
\pgfsetbuttcap%
\pgfsetroundjoin%
\definecolor{currentfill}{rgb}{0.121569,0.466667,0.705882}%
\pgfsetfillcolor{currentfill}%
\pgfsetfillopacity{0.514914}%
\pgfsetlinewidth{1.003750pt}%
\definecolor{currentstroke}{rgb}{0.121569,0.466667,0.705882}%
\pgfsetstrokecolor{currentstroke}%
\pgfsetstrokeopacity{0.514914}%
\pgfsetdash{}{0pt}%
\pgfpathmoveto{\pgfqpoint{1.211561in}{2.155933in}}%
\pgfpathcurveto{\pgfqpoint{1.219797in}{2.155933in}}{\pgfqpoint{1.227697in}{2.159205in}}{\pgfqpoint{1.233521in}{2.165029in}}%
\pgfpathcurveto{\pgfqpoint{1.239345in}{2.170853in}}{\pgfqpoint{1.242618in}{2.178753in}}{\pgfqpoint{1.242618in}{2.186989in}}%
\pgfpathcurveto{\pgfqpoint{1.242618in}{2.195225in}}{\pgfqpoint{1.239345in}{2.203125in}}{\pgfqpoint{1.233521in}{2.208949in}}%
\pgfpathcurveto{\pgfqpoint{1.227697in}{2.214773in}}{\pgfqpoint{1.219797in}{2.218046in}}{\pgfqpoint{1.211561in}{2.218046in}}%
\pgfpathcurveto{\pgfqpoint{1.203325in}{2.218046in}}{\pgfqpoint{1.195425in}{2.214773in}}{\pgfqpoint{1.189601in}{2.208949in}}%
\pgfpathcurveto{\pgfqpoint{1.183777in}{2.203125in}}{\pgfqpoint{1.180505in}{2.195225in}}{\pgfqpoint{1.180505in}{2.186989in}}%
\pgfpathcurveto{\pgfqpoint{1.180505in}{2.178753in}}{\pgfqpoint{1.183777in}{2.170853in}}{\pgfqpoint{1.189601in}{2.165029in}}%
\pgfpathcurveto{\pgfqpoint{1.195425in}{2.159205in}}{\pgfqpoint{1.203325in}{2.155933in}}{\pgfqpoint{1.211561in}{2.155933in}}%
\pgfpathclose%
\pgfusepath{stroke,fill}%
\end{pgfscope}%
\begin{pgfscope}%
\pgfpathrectangle{\pgfqpoint{0.100000in}{0.212622in}}{\pgfqpoint{3.696000in}{3.696000in}}%
\pgfusepath{clip}%
\pgfsetbuttcap%
\pgfsetroundjoin%
\definecolor{currentfill}{rgb}{0.121569,0.466667,0.705882}%
\pgfsetfillcolor{currentfill}%
\pgfsetfillopacity{0.515213}%
\pgfsetlinewidth{1.003750pt}%
\definecolor{currentstroke}{rgb}{0.121569,0.466667,0.705882}%
\pgfsetstrokecolor{currentstroke}%
\pgfsetstrokeopacity{0.515213}%
\pgfsetdash{}{0pt}%
\pgfpathmoveto{\pgfqpoint{1.210594in}{2.154363in}}%
\pgfpathcurveto{\pgfqpoint{1.218830in}{2.154363in}}{\pgfqpoint{1.226730in}{2.157635in}}{\pgfqpoint{1.232554in}{2.163459in}}%
\pgfpathcurveto{\pgfqpoint{1.238378in}{2.169283in}}{\pgfqpoint{1.241651in}{2.177183in}}{\pgfqpoint{1.241651in}{2.185420in}}%
\pgfpathcurveto{\pgfqpoint{1.241651in}{2.193656in}}{\pgfqpoint{1.238378in}{2.201556in}}{\pgfqpoint{1.232554in}{2.207380in}}%
\pgfpathcurveto{\pgfqpoint{1.226730in}{2.213204in}}{\pgfqpoint{1.218830in}{2.216476in}}{\pgfqpoint{1.210594in}{2.216476in}}%
\pgfpathcurveto{\pgfqpoint{1.202358in}{2.216476in}}{\pgfqpoint{1.194458in}{2.213204in}}{\pgfqpoint{1.188634in}{2.207380in}}%
\pgfpathcurveto{\pgfqpoint{1.182810in}{2.201556in}}{\pgfqpoint{1.179538in}{2.193656in}}{\pgfqpoint{1.179538in}{2.185420in}}%
\pgfpathcurveto{\pgfqpoint{1.179538in}{2.177183in}}{\pgfqpoint{1.182810in}{2.169283in}}{\pgfqpoint{1.188634in}{2.163459in}}%
\pgfpathcurveto{\pgfqpoint{1.194458in}{2.157635in}}{\pgfqpoint{1.202358in}{2.154363in}}{\pgfqpoint{1.210594in}{2.154363in}}%
\pgfpathclose%
\pgfusepath{stroke,fill}%
\end{pgfscope}%
\begin{pgfscope}%
\pgfpathrectangle{\pgfqpoint{0.100000in}{0.212622in}}{\pgfqpoint{3.696000in}{3.696000in}}%
\pgfusepath{clip}%
\pgfsetbuttcap%
\pgfsetroundjoin%
\definecolor{currentfill}{rgb}{0.121569,0.466667,0.705882}%
\pgfsetfillcolor{currentfill}%
\pgfsetfillopacity{0.515343}%
\pgfsetlinewidth{1.003750pt}%
\definecolor{currentstroke}{rgb}{0.121569,0.466667,0.705882}%
\pgfsetstrokecolor{currentstroke}%
\pgfsetstrokeopacity{0.515343}%
\pgfsetdash{}{0pt}%
\pgfpathmoveto{\pgfqpoint{1.210322in}{2.153679in}}%
\pgfpathcurveto{\pgfqpoint{1.218559in}{2.153679in}}{\pgfqpoint{1.226459in}{2.156951in}}{\pgfqpoint{1.232283in}{2.162775in}}%
\pgfpathcurveto{\pgfqpoint{1.238107in}{2.168599in}}{\pgfqpoint{1.241379in}{2.176499in}}{\pgfqpoint{1.241379in}{2.184735in}}%
\pgfpathcurveto{\pgfqpoint{1.241379in}{2.192971in}}{\pgfqpoint{1.238107in}{2.200871in}}{\pgfqpoint{1.232283in}{2.206695in}}%
\pgfpathcurveto{\pgfqpoint{1.226459in}{2.212519in}}{\pgfqpoint{1.218559in}{2.215792in}}{\pgfqpoint{1.210322in}{2.215792in}}%
\pgfpathcurveto{\pgfqpoint{1.202086in}{2.215792in}}{\pgfqpoint{1.194186in}{2.212519in}}{\pgfqpoint{1.188362in}{2.206695in}}%
\pgfpathcurveto{\pgfqpoint{1.182538in}{2.200871in}}{\pgfqpoint{1.179266in}{2.192971in}}{\pgfqpoint{1.179266in}{2.184735in}}%
\pgfpathcurveto{\pgfqpoint{1.179266in}{2.176499in}}{\pgfqpoint{1.182538in}{2.168599in}}{\pgfqpoint{1.188362in}{2.162775in}}%
\pgfpathcurveto{\pgfqpoint{1.194186in}{2.156951in}}{\pgfqpoint{1.202086in}{2.153679in}}{\pgfqpoint{1.210322in}{2.153679in}}%
\pgfpathclose%
\pgfusepath{stroke,fill}%
\end{pgfscope}%
\begin{pgfscope}%
\pgfpathrectangle{\pgfqpoint{0.100000in}{0.212622in}}{\pgfqpoint{3.696000in}{3.696000in}}%
\pgfusepath{clip}%
\pgfsetbuttcap%
\pgfsetroundjoin%
\definecolor{currentfill}{rgb}{0.121569,0.466667,0.705882}%
\pgfsetfillcolor{currentfill}%
\pgfsetfillopacity{0.515502}%
\pgfsetlinewidth{1.003750pt}%
\definecolor{currentstroke}{rgb}{0.121569,0.466667,0.705882}%
\pgfsetstrokecolor{currentstroke}%
\pgfsetstrokeopacity{0.515502}%
\pgfsetdash{}{0pt}%
\pgfpathmoveto{\pgfqpoint{2.046920in}{2.421678in}}%
\pgfpathcurveto{\pgfqpoint{2.055156in}{2.421678in}}{\pgfqpoint{2.063056in}{2.424951in}}{\pgfqpoint{2.068880in}{2.430774in}}%
\pgfpathcurveto{\pgfqpoint{2.074704in}{2.436598in}}{\pgfqpoint{2.077976in}{2.444498in}}{\pgfqpoint{2.077976in}{2.452735in}}%
\pgfpathcurveto{\pgfqpoint{2.077976in}{2.460971in}}{\pgfqpoint{2.074704in}{2.468871in}}{\pgfqpoint{2.068880in}{2.474695in}}%
\pgfpathcurveto{\pgfqpoint{2.063056in}{2.480519in}}{\pgfqpoint{2.055156in}{2.483791in}}{\pgfqpoint{2.046920in}{2.483791in}}%
\pgfpathcurveto{\pgfqpoint{2.038684in}{2.483791in}}{\pgfqpoint{2.030784in}{2.480519in}}{\pgfqpoint{2.024960in}{2.474695in}}%
\pgfpathcurveto{\pgfqpoint{2.019136in}{2.468871in}}{\pgfqpoint{2.015863in}{2.460971in}}{\pgfqpoint{2.015863in}{2.452735in}}%
\pgfpathcurveto{\pgfqpoint{2.015863in}{2.444498in}}{\pgfqpoint{2.019136in}{2.436598in}}{\pgfqpoint{2.024960in}{2.430774in}}%
\pgfpathcurveto{\pgfqpoint{2.030784in}{2.424951in}}{\pgfqpoint{2.038684in}{2.421678in}}{\pgfqpoint{2.046920in}{2.421678in}}%
\pgfpathclose%
\pgfusepath{stroke,fill}%
\end{pgfscope}%
\begin{pgfscope}%
\pgfpathrectangle{\pgfqpoint{0.100000in}{0.212622in}}{\pgfqpoint{3.696000in}{3.696000in}}%
\pgfusepath{clip}%
\pgfsetbuttcap%
\pgfsetroundjoin%
\definecolor{currentfill}{rgb}{0.121569,0.466667,0.705882}%
\pgfsetfillcolor{currentfill}%
\pgfsetfillopacity{0.515579}%
\pgfsetlinewidth{1.003750pt}%
\definecolor{currentstroke}{rgb}{0.121569,0.466667,0.705882}%
\pgfsetstrokecolor{currentstroke}%
\pgfsetstrokeopacity{0.515579}%
\pgfsetdash{}{0pt}%
\pgfpathmoveto{\pgfqpoint{1.209645in}{2.152625in}}%
\pgfpathcurveto{\pgfqpoint{1.217881in}{2.152625in}}{\pgfqpoint{1.225781in}{2.155897in}}{\pgfqpoint{1.231605in}{2.161721in}}%
\pgfpathcurveto{\pgfqpoint{1.237429in}{2.167545in}}{\pgfqpoint{1.240701in}{2.175445in}}{\pgfqpoint{1.240701in}{2.183682in}}%
\pgfpathcurveto{\pgfqpoint{1.240701in}{2.191918in}}{\pgfqpoint{1.237429in}{2.199818in}}{\pgfqpoint{1.231605in}{2.205642in}}%
\pgfpathcurveto{\pgfqpoint{1.225781in}{2.211466in}}{\pgfqpoint{1.217881in}{2.214738in}}{\pgfqpoint{1.209645in}{2.214738in}}%
\pgfpathcurveto{\pgfqpoint{1.201409in}{2.214738in}}{\pgfqpoint{1.193509in}{2.211466in}}{\pgfqpoint{1.187685in}{2.205642in}}%
\pgfpathcurveto{\pgfqpoint{1.181861in}{2.199818in}}{\pgfqpoint{1.178588in}{2.191918in}}{\pgfqpoint{1.178588in}{2.183682in}}%
\pgfpathcurveto{\pgfqpoint{1.178588in}{2.175445in}}{\pgfqpoint{1.181861in}{2.167545in}}{\pgfqpoint{1.187685in}{2.161721in}}%
\pgfpathcurveto{\pgfqpoint{1.193509in}{2.155897in}}{\pgfqpoint{1.201409in}{2.152625in}}{\pgfqpoint{1.209645in}{2.152625in}}%
\pgfpathclose%
\pgfusepath{stroke,fill}%
\end{pgfscope}%
\begin{pgfscope}%
\pgfpathrectangle{\pgfqpoint{0.100000in}{0.212622in}}{\pgfqpoint{3.696000in}{3.696000in}}%
\pgfusepath{clip}%
\pgfsetbuttcap%
\pgfsetroundjoin%
\definecolor{currentfill}{rgb}{0.121569,0.466667,0.705882}%
\pgfsetfillcolor{currentfill}%
\pgfsetfillopacity{0.516033}%
\pgfsetlinewidth{1.003750pt}%
\definecolor{currentstroke}{rgb}{0.121569,0.466667,0.705882}%
\pgfsetstrokecolor{currentstroke}%
\pgfsetstrokeopacity{0.516033}%
\pgfsetdash{}{0pt}%
\pgfpathmoveto{\pgfqpoint{1.208522in}{2.150659in}}%
\pgfpathcurveto{\pgfqpoint{1.216759in}{2.150659in}}{\pgfqpoint{1.224659in}{2.153932in}}{\pgfqpoint{1.230483in}{2.159756in}}%
\pgfpathcurveto{\pgfqpoint{1.236307in}{2.165580in}}{\pgfqpoint{1.239579in}{2.173480in}}{\pgfqpoint{1.239579in}{2.181716in}}%
\pgfpathcurveto{\pgfqpoint{1.239579in}{2.189952in}}{\pgfqpoint{1.236307in}{2.197852in}}{\pgfqpoint{1.230483in}{2.203676in}}%
\pgfpathcurveto{\pgfqpoint{1.224659in}{2.209500in}}{\pgfqpoint{1.216759in}{2.212772in}}{\pgfqpoint{1.208522in}{2.212772in}}%
\pgfpathcurveto{\pgfqpoint{1.200286in}{2.212772in}}{\pgfqpoint{1.192386in}{2.209500in}}{\pgfqpoint{1.186562in}{2.203676in}}%
\pgfpathcurveto{\pgfqpoint{1.180738in}{2.197852in}}{\pgfqpoint{1.177466in}{2.189952in}}{\pgfqpoint{1.177466in}{2.181716in}}%
\pgfpathcurveto{\pgfqpoint{1.177466in}{2.173480in}}{\pgfqpoint{1.180738in}{2.165580in}}{\pgfqpoint{1.186562in}{2.159756in}}%
\pgfpathcurveto{\pgfqpoint{1.192386in}{2.153932in}}{\pgfqpoint{1.200286in}{2.150659in}}{\pgfqpoint{1.208522in}{2.150659in}}%
\pgfpathclose%
\pgfusepath{stroke,fill}%
\end{pgfscope}%
\begin{pgfscope}%
\pgfpathrectangle{\pgfqpoint{0.100000in}{0.212622in}}{\pgfqpoint{3.696000in}{3.696000in}}%
\pgfusepath{clip}%
\pgfsetbuttcap%
\pgfsetroundjoin%
\definecolor{currentfill}{rgb}{0.121569,0.466667,0.705882}%
\pgfsetfillcolor{currentfill}%
\pgfsetfillopacity{0.516390}%
\pgfsetlinewidth{1.003750pt}%
\definecolor{currentstroke}{rgb}{0.121569,0.466667,0.705882}%
\pgfsetstrokecolor{currentstroke}%
\pgfsetstrokeopacity{0.516390}%
\pgfsetdash{}{0pt}%
\pgfpathmoveto{\pgfqpoint{1.207558in}{2.149043in}}%
\pgfpathcurveto{\pgfqpoint{1.215794in}{2.149043in}}{\pgfqpoint{1.223694in}{2.152315in}}{\pgfqpoint{1.229518in}{2.158139in}}%
\pgfpathcurveto{\pgfqpoint{1.235342in}{2.163963in}}{\pgfqpoint{1.238614in}{2.171863in}}{\pgfqpoint{1.238614in}{2.180100in}}%
\pgfpathcurveto{\pgfqpoint{1.238614in}{2.188336in}}{\pgfqpoint{1.235342in}{2.196236in}}{\pgfqpoint{1.229518in}{2.202060in}}%
\pgfpathcurveto{\pgfqpoint{1.223694in}{2.207884in}}{\pgfqpoint{1.215794in}{2.211156in}}{\pgfqpoint{1.207558in}{2.211156in}}%
\pgfpathcurveto{\pgfqpoint{1.199321in}{2.211156in}}{\pgfqpoint{1.191421in}{2.207884in}}{\pgfqpoint{1.185597in}{2.202060in}}%
\pgfpathcurveto{\pgfqpoint{1.179774in}{2.196236in}}{\pgfqpoint{1.176501in}{2.188336in}}{\pgfqpoint{1.176501in}{2.180100in}}%
\pgfpathcurveto{\pgfqpoint{1.176501in}{2.171863in}}{\pgfqpoint{1.179774in}{2.163963in}}{\pgfqpoint{1.185597in}{2.158139in}}%
\pgfpathcurveto{\pgfqpoint{1.191421in}{2.152315in}}{\pgfqpoint{1.199321in}{2.149043in}}{\pgfqpoint{1.207558in}{2.149043in}}%
\pgfpathclose%
\pgfusepath{stroke,fill}%
\end{pgfscope}%
\begin{pgfscope}%
\pgfpathrectangle{\pgfqpoint{0.100000in}{0.212622in}}{\pgfqpoint{3.696000in}{3.696000in}}%
\pgfusepath{clip}%
\pgfsetbuttcap%
\pgfsetroundjoin%
\definecolor{currentfill}{rgb}{0.121569,0.466667,0.705882}%
\pgfsetfillcolor{currentfill}%
\pgfsetfillopacity{0.516916}%
\pgfsetlinewidth{1.003750pt}%
\definecolor{currentstroke}{rgb}{0.121569,0.466667,0.705882}%
\pgfsetstrokecolor{currentstroke}%
\pgfsetstrokeopacity{0.516916}%
\pgfsetdash{}{0pt}%
\pgfpathmoveto{\pgfqpoint{2.047774in}{2.414751in}}%
\pgfpathcurveto{\pgfqpoint{2.056010in}{2.414751in}}{\pgfqpoint{2.063910in}{2.418024in}}{\pgfqpoint{2.069734in}{2.423848in}}%
\pgfpathcurveto{\pgfqpoint{2.075558in}{2.429672in}}{\pgfqpoint{2.078830in}{2.437572in}}{\pgfqpoint{2.078830in}{2.445808in}}%
\pgfpathcurveto{\pgfqpoint{2.078830in}{2.454044in}}{\pgfqpoint{2.075558in}{2.461944in}}{\pgfqpoint{2.069734in}{2.467768in}}%
\pgfpathcurveto{\pgfqpoint{2.063910in}{2.473592in}}{\pgfqpoint{2.056010in}{2.476864in}}{\pgfqpoint{2.047774in}{2.476864in}}%
\pgfpathcurveto{\pgfqpoint{2.039537in}{2.476864in}}{\pgfqpoint{2.031637in}{2.473592in}}{\pgfqpoint{2.025813in}{2.467768in}}%
\pgfpathcurveto{\pgfqpoint{2.019989in}{2.461944in}}{\pgfqpoint{2.016717in}{2.454044in}}{\pgfqpoint{2.016717in}{2.445808in}}%
\pgfpathcurveto{\pgfqpoint{2.016717in}{2.437572in}}{\pgfqpoint{2.019989in}{2.429672in}}{\pgfqpoint{2.025813in}{2.423848in}}%
\pgfpathcurveto{\pgfqpoint{2.031637in}{2.418024in}}{\pgfqpoint{2.039537in}{2.414751in}}{\pgfqpoint{2.047774in}{2.414751in}}%
\pgfpathclose%
\pgfusepath{stroke,fill}%
\end{pgfscope}%
\begin{pgfscope}%
\pgfpathrectangle{\pgfqpoint{0.100000in}{0.212622in}}{\pgfqpoint{3.696000in}{3.696000in}}%
\pgfusepath{clip}%
\pgfsetbuttcap%
\pgfsetroundjoin%
\definecolor{currentfill}{rgb}{0.121569,0.466667,0.705882}%
\pgfsetfillcolor{currentfill}%
\pgfsetfillopacity{0.517034}%
\pgfsetlinewidth{1.003750pt}%
\definecolor{currentstroke}{rgb}{0.121569,0.466667,0.705882}%
\pgfsetstrokecolor{currentstroke}%
\pgfsetstrokeopacity{0.517034}%
\pgfsetdash{}{0pt}%
\pgfpathmoveto{\pgfqpoint{1.205782in}{2.146109in}}%
\pgfpathcurveto{\pgfqpoint{1.214019in}{2.146109in}}{\pgfqpoint{1.221919in}{2.149381in}}{\pgfqpoint{1.227742in}{2.155205in}}%
\pgfpathcurveto{\pgfqpoint{1.233566in}{2.161029in}}{\pgfqpoint{1.236839in}{2.168929in}}{\pgfqpoint{1.236839in}{2.177166in}}%
\pgfpathcurveto{\pgfqpoint{1.236839in}{2.185402in}}{\pgfqpoint{1.233566in}{2.193302in}}{\pgfqpoint{1.227742in}{2.199126in}}%
\pgfpathcurveto{\pgfqpoint{1.221919in}{2.204950in}}{\pgfqpoint{1.214019in}{2.208222in}}{\pgfqpoint{1.205782in}{2.208222in}}%
\pgfpathcurveto{\pgfqpoint{1.197546in}{2.208222in}}{\pgfqpoint{1.189646in}{2.204950in}}{\pgfqpoint{1.183822in}{2.199126in}}%
\pgfpathcurveto{\pgfqpoint{1.177998in}{2.193302in}}{\pgfqpoint{1.174726in}{2.185402in}}{\pgfqpoint{1.174726in}{2.177166in}}%
\pgfpathcurveto{\pgfqpoint{1.174726in}{2.168929in}}{\pgfqpoint{1.177998in}{2.161029in}}{\pgfqpoint{1.183822in}{2.155205in}}%
\pgfpathcurveto{\pgfqpoint{1.189646in}{2.149381in}}{\pgfqpoint{1.197546in}{2.146109in}}{\pgfqpoint{1.205782in}{2.146109in}}%
\pgfpathclose%
\pgfusepath{stroke,fill}%
\end{pgfscope}%
\begin{pgfscope}%
\pgfpathrectangle{\pgfqpoint{0.100000in}{0.212622in}}{\pgfqpoint{3.696000in}{3.696000in}}%
\pgfusepath{clip}%
\pgfsetbuttcap%
\pgfsetroundjoin%
\definecolor{currentfill}{rgb}{0.121569,0.466667,0.705882}%
\pgfsetfillcolor{currentfill}%
\pgfsetfillopacity{0.517575}%
\pgfsetlinewidth{1.003750pt}%
\definecolor{currentstroke}{rgb}{0.121569,0.466667,0.705882}%
\pgfsetstrokecolor{currentstroke}%
\pgfsetstrokeopacity{0.517575}%
\pgfsetdash{}{0pt}%
\pgfpathmoveto{\pgfqpoint{1.204321in}{2.143519in}}%
\pgfpathcurveto{\pgfqpoint{1.212557in}{2.143519in}}{\pgfqpoint{1.220457in}{2.146791in}}{\pgfqpoint{1.226281in}{2.152615in}}%
\pgfpathcurveto{\pgfqpoint{1.232105in}{2.158439in}}{\pgfqpoint{1.235378in}{2.166339in}}{\pgfqpoint{1.235378in}{2.174576in}}%
\pgfpathcurveto{\pgfqpoint{1.235378in}{2.182812in}}{\pgfqpoint{1.232105in}{2.190712in}}{\pgfqpoint{1.226281in}{2.196536in}}%
\pgfpathcurveto{\pgfqpoint{1.220457in}{2.202360in}}{\pgfqpoint{1.212557in}{2.205632in}}{\pgfqpoint{1.204321in}{2.205632in}}%
\pgfpathcurveto{\pgfqpoint{1.196085in}{2.205632in}}{\pgfqpoint{1.188185in}{2.202360in}}{\pgfqpoint{1.182361in}{2.196536in}}%
\pgfpathcurveto{\pgfqpoint{1.176537in}{2.190712in}}{\pgfqpoint{1.173265in}{2.182812in}}{\pgfqpoint{1.173265in}{2.174576in}}%
\pgfpathcurveto{\pgfqpoint{1.173265in}{2.166339in}}{\pgfqpoint{1.176537in}{2.158439in}}{\pgfqpoint{1.182361in}{2.152615in}}%
\pgfpathcurveto{\pgfqpoint{1.188185in}{2.146791in}}{\pgfqpoint{1.196085in}{2.143519in}}{\pgfqpoint{1.204321in}{2.143519in}}%
\pgfpathclose%
\pgfusepath{stroke,fill}%
\end{pgfscope}%
\begin{pgfscope}%
\pgfpathrectangle{\pgfqpoint{0.100000in}{0.212622in}}{\pgfqpoint{3.696000in}{3.696000in}}%
\pgfusepath{clip}%
\pgfsetbuttcap%
\pgfsetroundjoin%
\definecolor{currentfill}{rgb}{0.121569,0.466667,0.705882}%
\pgfsetfillcolor{currentfill}%
\pgfsetfillopacity{0.518559}%
\pgfsetlinewidth{1.003750pt}%
\definecolor{currentstroke}{rgb}{0.121569,0.466667,0.705882}%
\pgfsetstrokecolor{currentstroke}%
\pgfsetstrokeopacity{0.518559}%
\pgfsetdash{}{0pt}%
\pgfpathmoveto{\pgfqpoint{1.201556in}{2.138954in}}%
\pgfpathcurveto{\pgfqpoint{1.209793in}{2.138954in}}{\pgfqpoint{1.217693in}{2.142226in}}{\pgfqpoint{1.223517in}{2.148050in}}%
\pgfpathcurveto{\pgfqpoint{1.229341in}{2.153874in}}{\pgfqpoint{1.232613in}{2.161774in}}{\pgfqpoint{1.232613in}{2.170010in}}%
\pgfpathcurveto{\pgfqpoint{1.232613in}{2.178246in}}{\pgfqpoint{1.229341in}{2.186147in}}{\pgfqpoint{1.223517in}{2.191970in}}%
\pgfpathcurveto{\pgfqpoint{1.217693in}{2.197794in}}{\pgfqpoint{1.209793in}{2.201067in}}{\pgfqpoint{1.201556in}{2.201067in}}%
\pgfpathcurveto{\pgfqpoint{1.193320in}{2.201067in}}{\pgfqpoint{1.185420in}{2.197794in}}{\pgfqpoint{1.179596in}{2.191970in}}%
\pgfpathcurveto{\pgfqpoint{1.173772in}{2.186147in}}{\pgfqpoint{1.170500in}{2.178246in}}{\pgfqpoint{1.170500in}{2.170010in}}%
\pgfpathcurveto{\pgfqpoint{1.170500in}{2.161774in}}{\pgfqpoint{1.173772in}{2.153874in}}{\pgfqpoint{1.179596in}{2.148050in}}%
\pgfpathcurveto{\pgfqpoint{1.185420in}{2.142226in}}{\pgfqpoint{1.193320in}{2.138954in}}{\pgfqpoint{1.201556in}{2.138954in}}%
\pgfpathclose%
\pgfusepath{stroke,fill}%
\end{pgfscope}%
\begin{pgfscope}%
\pgfpathrectangle{\pgfqpoint{0.100000in}{0.212622in}}{\pgfqpoint{3.696000in}{3.696000in}}%
\pgfusepath{clip}%
\pgfsetbuttcap%
\pgfsetroundjoin%
\definecolor{currentfill}{rgb}{0.121569,0.466667,0.705882}%
\pgfsetfillcolor{currentfill}%
\pgfsetfillopacity{0.518646}%
\pgfsetlinewidth{1.003750pt}%
\definecolor{currentstroke}{rgb}{0.121569,0.466667,0.705882}%
\pgfsetstrokecolor{currentstroke}%
\pgfsetstrokeopacity{0.518646}%
\pgfsetdash{}{0pt}%
\pgfpathmoveto{\pgfqpoint{2.048356in}{2.407794in}}%
\pgfpathcurveto{\pgfqpoint{2.056592in}{2.407794in}}{\pgfqpoint{2.064492in}{2.411066in}}{\pgfqpoint{2.070316in}{2.416890in}}%
\pgfpathcurveto{\pgfqpoint{2.076140in}{2.422714in}}{\pgfqpoint{2.079412in}{2.430614in}}{\pgfqpoint{2.079412in}{2.438851in}}%
\pgfpathcurveto{\pgfqpoint{2.079412in}{2.447087in}}{\pgfqpoint{2.076140in}{2.454987in}}{\pgfqpoint{2.070316in}{2.460811in}}%
\pgfpathcurveto{\pgfqpoint{2.064492in}{2.466635in}}{\pgfqpoint{2.056592in}{2.469907in}}{\pgfqpoint{2.048356in}{2.469907in}}%
\pgfpathcurveto{\pgfqpoint{2.040119in}{2.469907in}}{\pgfqpoint{2.032219in}{2.466635in}}{\pgfqpoint{2.026395in}{2.460811in}}%
\pgfpathcurveto{\pgfqpoint{2.020572in}{2.454987in}}{\pgfqpoint{2.017299in}{2.447087in}}{\pgfqpoint{2.017299in}{2.438851in}}%
\pgfpathcurveto{\pgfqpoint{2.017299in}{2.430614in}}{\pgfqpoint{2.020572in}{2.422714in}}{\pgfqpoint{2.026395in}{2.416890in}}%
\pgfpathcurveto{\pgfqpoint{2.032219in}{2.411066in}}{\pgfqpoint{2.040119in}{2.407794in}}{\pgfqpoint{2.048356in}{2.407794in}}%
\pgfpathclose%
\pgfusepath{stroke,fill}%
\end{pgfscope}%
\begin{pgfscope}%
\pgfpathrectangle{\pgfqpoint{0.100000in}{0.212622in}}{\pgfqpoint{3.696000in}{3.696000in}}%
\pgfusepath{clip}%
\pgfsetbuttcap%
\pgfsetroundjoin%
\definecolor{currentfill}{rgb}{0.121569,0.466667,0.705882}%
\pgfsetfillcolor{currentfill}%
\pgfsetfillopacity{0.519426}%
\pgfsetlinewidth{1.003750pt}%
\definecolor{currentstroke}{rgb}{0.121569,0.466667,0.705882}%
\pgfsetstrokecolor{currentstroke}%
\pgfsetstrokeopacity{0.519426}%
\pgfsetdash{}{0pt}%
\pgfpathmoveto{\pgfqpoint{1.199200in}{2.134733in}}%
\pgfpathcurveto{\pgfqpoint{1.207437in}{2.134733in}}{\pgfqpoint{1.215337in}{2.138006in}}{\pgfqpoint{1.221161in}{2.143830in}}%
\pgfpathcurveto{\pgfqpoint{1.226985in}{2.149653in}}{\pgfqpoint{1.230257in}{2.157554in}}{\pgfqpoint{1.230257in}{2.165790in}}%
\pgfpathcurveto{\pgfqpoint{1.230257in}{2.174026in}}{\pgfqpoint{1.226985in}{2.181926in}}{\pgfqpoint{1.221161in}{2.187750in}}%
\pgfpathcurveto{\pgfqpoint{1.215337in}{2.193574in}}{\pgfqpoint{1.207437in}{2.196846in}}{\pgfqpoint{1.199200in}{2.196846in}}%
\pgfpathcurveto{\pgfqpoint{1.190964in}{2.196846in}}{\pgfqpoint{1.183064in}{2.193574in}}{\pgfqpoint{1.177240in}{2.187750in}}%
\pgfpathcurveto{\pgfqpoint{1.171416in}{2.181926in}}{\pgfqpoint{1.168144in}{2.174026in}}{\pgfqpoint{1.168144in}{2.165790in}}%
\pgfpathcurveto{\pgfqpoint{1.168144in}{2.157554in}}{\pgfqpoint{1.171416in}{2.149653in}}{\pgfqpoint{1.177240in}{2.143830in}}%
\pgfpathcurveto{\pgfqpoint{1.183064in}{2.138006in}}{\pgfqpoint{1.190964in}{2.134733in}}{\pgfqpoint{1.199200in}{2.134733in}}%
\pgfpathclose%
\pgfusepath{stroke,fill}%
\end{pgfscope}%
\begin{pgfscope}%
\pgfpathrectangle{\pgfqpoint{0.100000in}{0.212622in}}{\pgfqpoint{3.696000in}{3.696000in}}%
\pgfusepath{clip}%
\pgfsetbuttcap%
\pgfsetroundjoin%
\definecolor{currentfill}{rgb}{0.121569,0.466667,0.705882}%
\pgfsetfillcolor{currentfill}%
\pgfsetfillopacity{0.520548}%
\pgfsetlinewidth{1.003750pt}%
\definecolor{currentstroke}{rgb}{0.121569,0.466667,0.705882}%
\pgfsetstrokecolor{currentstroke}%
\pgfsetstrokeopacity{0.520548}%
\pgfsetdash{}{0pt}%
\pgfpathmoveto{\pgfqpoint{2.049933in}{2.399161in}}%
\pgfpathcurveto{\pgfqpoint{2.058169in}{2.399161in}}{\pgfqpoint{2.066069in}{2.402433in}}{\pgfqpoint{2.071893in}{2.408257in}}%
\pgfpathcurveto{\pgfqpoint{2.077717in}{2.414081in}}{\pgfqpoint{2.080989in}{2.421981in}}{\pgfqpoint{2.080989in}{2.430217in}}%
\pgfpathcurveto{\pgfqpoint{2.080989in}{2.438453in}}{\pgfqpoint{2.077717in}{2.446353in}}{\pgfqpoint{2.071893in}{2.452177in}}%
\pgfpathcurveto{\pgfqpoint{2.066069in}{2.458001in}}{\pgfqpoint{2.058169in}{2.461274in}}{\pgfqpoint{2.049933in}{2.461274in}}%
\pgfpathcurveto{\pgfqpoint{2.041696in}{2.461274in}}{\pgfqpoint{2.033796in}{2.458001in}}{\pgfqpoint{2.027972in}{2.452177in}}%
\pgfpathcurveto{\pgfqpoint{2.022149in}{2.446353in}}{\pgfqpoint{2.018876in}{2.438453in}}{\pgfqpoint{2.018876in}{2.430217in}}%
\pgfpathcurveto{\pgfqpoint{2.018876in}{2.421981in}}{\pgfqpoint{2.022149in}{2.414081in}}{\pgfqpoint{2.027972in}{2.408257in}}%
\pgfpathcurveto{\pgfqpoint{2.033796in}{2.402433in}}{\pgfqpoint{2.041696in}{2.399161in}}{\pgfqpoint{2.049933in}{2.399161in}}%
\pgfpathclose%
\pgfusepath{stroke,fill}%
\end{pgfscope}%
\begin{pgfscope}%
\pgfpathrectangle{\pgfqpoint{0.100000in}{0.212622in}}{\pgfqpoint{3.696000in}{3.696000in}}%
\pgfusepath{clip}%
\pgfsetbuttcap%
\pgfsetroundjoin%
\definecolor{currentfill}{rgb}{0.121569,0.466667,0.705882}%
\pgfsetfillcolor{currentfill}%
\pgfsetfillopacity{0.520930}%
\pgfsetlinewidth{1.003750pt}%
\definecolor{currentstroke}{rgb}{0.121569,0.466667,0.705882}%
\pgfsetstrokecolor{currentstroke}%
\pgfsetstrokeopacity{0.520930}%
\pgfsetdash{}{0pt}%
\pgfpathmoveto{\pgfqpoint{1.194611in}{2.127198in}}%
\pgfpathcurveto{\pgfqpoint{1.202848in}{2.127198in}}{\pgfqpoint{1.210748in}{2.130471in}}{\pgfqpoint{1.216572in}{2.136295in}}%
\pgfpathcurveto{\pgfqpoint{1.222395in}{2.142119in}}{\pgfqpoint{1.225668in}{2.150019in}}{\pgfqpoint{1.225668in}{2.158255in}}%
\pgfpathcurveto{\pgfqpoint{1.225668in}{2.166491in}}{\pgfqpoint{1.222395in}{2.174391in}}{\pgfqpoint{1.216572in}{2.180215in}}%
\pgfpathcurveto{\pgfqpoint{1.210748in}{2.186039in}}{\pgfqpoint{1.202848in}{2.189311in}}{\pgfqpoint{1.194611in}{2.189311in}}%
\pgfpathcurveto{\pgfqpoint{1.186375in}{2.189311in}}{\pgfqpoint{1.178475in}{2.186039in}}{\pgfqpoint{1.172651in}{2.180215in}}%
\pgfpathcurveto{\pgfqpoint{1.166827in}{2.174391in}}{\pgfqpoint{1.163555in}{2.166491in}}{\pgfqpoint{1.163555in}{2.158255in}}%
\pgfpathcurveto{\pgfqpoint{1.163555in}{2.150019in}}{\pgfqpoint{1.166827in}{2.142119in}}{\pgfqpoint{1.172651in}{2.136295in}}%
\pgfpathcurveto{\pgfqpoint{1.178475in}{2.130471in}}{\pgfqpoint{1.186375in}{2.127198in}}{\pgfqpoint{1.194611in}{2.127198in}}%
\pgfpathclose%
\pgfusepath{stroke,fill}%
\end{pgfscope}%
\begin{pgfscope}%
\pgfpathrectangle{\pgfqpoint{0.100000in}{0.212622in}}{\pgfqpoint{3.696000in}{3.696000in}}%
\pgfusepath{clip}%
\pgfsetbuttcap%
\pgfsetroundjoin%
\definecolor{currentfill}{rgb}{0.121569,0.466667,0.705882}%
\pgfsetfillcolor{currentfill}%
\pgfsetfillopacity{0.522760}%
\pgfsetlinewidth{1.003750pt}%
\definecolor{currentstroke}{rgb}{0.121569,0.466667,0.705882}%
\pgfsetstrokecolor{currentstroke}%
\pgfsetstrokeopacity{0.522760}%
\pgfsetdash{}{0pt}%
\pgfpathmoveto{\pgfqpoint{2.051526in}{2.389675in}}%
\pgfpathcurveto{\pgfqpoint{2.059762in}{2.389675in}}{\pgfqpoint{2.067662in}{2.392947in}}{\pgfqpoint{2.073486in}{2.398771in}}%
\pgfpathcurveto{\pgfqpoint{2.079310in}{2.404595in}}{\pgfqpoint{2.082582in}{2.412495in}}{\pgfqpoint{2.082582in}{2.420732in}}%
\pgfpathcurveto{\pgfqpoint{2.082582in}{2.428968in}}{\pgfqpoint{2.079310in}{2.436868in}}{\pgfqpoint{2.073486in}{2.442692in}}%
\pgfpathcurveto{\pgfqpoint{2.067662in}{2.448516in}}{\pgfqpoint{2.059762in}{2.451788in}}{\pgfqpoint{2.051526in}{2.451788in}}%
\pgfpathcurveto{\pgfqpoint{2.043289in}{2.451788in}}{\pgfqpoint{2.035389in}{2.448516in}}{\pgfqpoint{2.029565in}{2.442692in}}%
\pgfpathcurveto{\pgfqpoint{2.023741in}{2.436868in}}{\pgfqpoint{2.020469in}{2.428968in}}{\pgfqpoint{2.020469in}{2.420732in}}%
\pgfpathcurveto{\pgfqpoint{2.020469in}{2.412495in}}{\pgfqpoint{2.023741in}{2.404595in}}{\pgfqpoint{2.029565in}{2.398771in}}%
\pgfpathcurveto{\pgfqpoint{2.035389in}{2.392947in}}{\pgfqpoint{2.043289in}{2.389675in}}{\pgfqpoint{2.051526in}{2.389675in}}%
\pgfpathclose%
\pgfusepath{stroke,fill}%
\end{pgfscope}%
\begin{pgfscope}%
\pgfpathrectangle{\pgfqpoint{0.100000in}{0.212622in}}{\pgfqpoint{3.696000in}{3.696000in}}%
\pgfusepath{clip}%
\pgfsetbuttcap%
\pgfsetroundjoin%
\definecolor{currentfill}{rgb}{0.121569,0.466667,0.705882}%
\pgfsetfillcolor{currentfill}%
\pgfsetfillopacity{0.523687}%
\pgfsetlinewidth{1.003750pt}%
\definecolor{currentstroke}{rgb}{0.121569,0.466667,0.705882}%
\pgfsetstrokecolor{currentstroke}%
\pgfsetstrokeopacity{0.523687}%
\pgfsetdash{}{0pt}%
\pgfpathmoveto{\pgfqpoint{1.186767in}{2.112878in}}%
\pgfpathcurveto{\pgfqpoint{1.195004in}{2.112878in}}{\pgfqpoint{1.202904in}{2.116150in}}{\pgfqpoint{1.208727in}{2.121974in}}%
\pgfpathcurveto{\pgfqpoint{1.214551in}{2.127798in}}{\pgfqpoint{1.217824in}{2.135698in}}{\pgfqpoint{1.217824in}{2.143935in}}%
\pgfpathcurveto{\pgfqpoint{1.217824in}{2.152171in}}{\pgfqpoint{1.214551in}{2.160071in}}{\pgfqpoint{1.208727in}{2.165895in}}%
\pgfpathcurveto{\pgfqpoint{1.202904in}{2.171719in}}{\pgfqpoint{1.195004in}{2.174991in}}{\pgfqpoint{1.186767in}{2.174991in}}%
\pgfpathcurveto{\pgfqpoint{1.178531in}{2.174991in}}{\pgfqpoint{1.170631in}{2.171719in}}{\pgfqpoint{1.164807in}{2.165895in}}%
\pgfpathcurveto{\pgfqpoint{1.158983in}{2.160071in}}{\pgfqpoint{1.155711in}{2.152171in}}{\pgfqpoint{1.155711in}{2.143935in}}%
\pgfpathcurveto{\pgfqpoint{1.155711in}{2.135698in}}{\pgfqpoint{1.158983in}{2.127798in}}{\pgfqpoint{1.164807in}{2.121974in}}%
\pgfpathcurveto{\pgfqpoint{1.170631in}{2.116150in}}{\pgfqpoint{1.178531in}{2.112878in}}{\pgfqpoint{1.186767in}{2.112878in}}%
\pgfpathclose%
\pgfusepath{stroke,fill}%
\end{pgfscope}%
\begin{pgfscope}%
\pgfpathrectangle{\pgfqpoint{0.100000in}{0.212622in}}{\pgfqpoint{3.696000in}{3.696000in}}%
\pgfusepath{clip}%
\pgfsetbuttcap%
\pgfsetroundjoin%
\definecolor{currentfill}{rgb}{0.121569,0.466667,0.705882}%
\pgfsetfillcolor{currentfill}%
\pgfsetfillopacity{0.525535}%
\pgfsetlinewidth{1.003750pt}%
\definecolor{currentstroke}{rgb}{0.121569,0.466667,0.705882}%
\pgfsetstrokecolor{currentstroke}%
\pgfsetstrokeopacity{0.525535}%
\pgfsetdash{}{0pt}%
\pgfpathmoveto{\pgfqpoint{2.052639in}{2.379655in}}%
\pgfpathcurveto{\pgfqpoint{2.060875in}{2.379655in}}{\pgfqpoint{2.068775in}{2.382927in}}{\pgfqpoint{2.074599in}{2.388751in}}%
\pgfpathcurveto{\pgfqpoint{2.080423in}{2.394575in}}{\pgfqpoint{2.083695in}{2.402475in}}{\pgfqpoint{2.083695in}{2.410711in}}%
\pgfpathcurveto{\pgfqpoint{2.083695in}{2.418948in}}{\pgfqpoint{2.080423in}{2.426848in}}{\pgfqpoint{2.074599in}{2.432672in}}%
\pgfpathcurveto{\pgfqpoint{2.068775in}{2.438495in}}{\pgfqpoint{2.060875in}{2.441768in}}{\pgfqpoint{2.052639in}{2.441768in}}%
\pgfpathcurveto{\pgfqpoint{2.044402in}{2.441768in}}{\pgfqpoint{2.036502in}{2.438495in}}{\pgfqpoint{2.030678in}{2.432672in}}%
\pgfpathcurveto{\pgfqpoint{2.024854in}{2.426848in}}{\pgfqpoint{2.021582in}{2.418948in}}{\pgfqpoint{2.021582in}{2.410711in}}%
\pgfpathcurveto{\pgfqpoint{2.021582in}{2.402475in}}{\pgfqpoint{2.024854in}{2.394575in}}{\pgfqpoint{2.030678in}{2.388751in}}%
\pgfpathcurveto{\pgfqpoint{2.036502in}{2.382927in}}{\pgfqpoint{2.044402in}{2.379655in}}{\pgfqpoint{2.052639in}{2.379655in}}%
\pgfpathclose%
\pgfusepath{stroke,fill}%
\end{pgfscope}%
\begin{pgfscope}%
\pgfpathrectangle{\pgfqpoint{0.100000in}{0.212622in}}{\pgfqpoint{3.696000in}{3.696000in}}%
\pgfusepath{clip}%
\pgfsetbuttcap%
\pgfsetroundjoin%
\definecolor{currentfill}{rgb}{0.121569,0.466667,0.705882}%
\pgfsetfillcolor{currentfill}%
\pgfsetfillopacity{0.526028}%
\pgfsetlinewidth{1.003750pt}%
\definecolor{currentstroke}{rgb}{0.121569,0.466667,0.705882}%
\pgfsetstrokecolor{currentstroke}%
\pgfsetstrokeopacity{0.526028}%
\pgfsetdash{}{0pt}%
\pgfpathmoveto{\pgfqpoint{1.178638in}{2.099946in}}%
\pgfpathcurveto{\pgfqpoint{1.186874in}{2.099946in}}{\pgfqpoint{1.194774in}{2.103219in}}{\pgfqpoint{1.200598in}{2.109043in}}%
\pgfpathcurveto{\pgfqpoint{1.206422in}{2.114866in}}{\pgfqpoint{1.209695in}{2.122767in}}{\pgfqpoint{1.209695in}{2.131003in}}%
\pgfpathcurveto{\pgfqpoint{1.209695in}{2.139239in}}{\pgfqpoint{1.206422in}{2.147139in}}{\pgfqpoint{1.200598in}{2.152963in}}%
\pgfpathcurveto{\pgfqpoint{1.194774in}{2.158787in}}{\pgfqpoint{1.186874in}{2.162059in}}{\pgfqpoint{1.178638in}{2.162059in}}%
\pgfpathcurveto{\pgfqpoint{1.170402in}{2.162059in}}{\pgfqpoint{1.162502in}{2.158787in}}{\pgfqpoint{1.156678in}{2.152963in}}%
\pgfpathcurveto{\pgfqpoint{1.150854in}{2.147139in}}{\pgfqpoint{1.147582in}{2.139239in}}{\pgfqpoint{1.147582in}{2.131003in}}%
\pgfpathcurveto{\pgfqpoint{1.147582in}{2.122767in}}{\pgfqpoint{1.150854in}{2.114866in}}{\pgfqpoint{1.156678in}{2.109043in}}%
\pgfpathcurveto{\pgfqpoint{1.162502in}{2.103219in}}{\pgfqpoint{1.170402in}{2.099946in}}{\pgfqpoint{1.178638in}{2.099946in}}%
\pgfpathclose%
\pgfusepath{stroke,fill}%
\end{pgfscope}%
\begin{pgfscope}%
\pgfpathrectangle{\pgfqpoint{0.100000in}{0.212622in}}{\pgfqpoint{3.696000in}{3.696000in}}%
\pgfusepath{clip}%
\pgfsetbuttcap%
\pgfsetroundjoin%
\definecolor{currentfill}{rgb}{0.121569,0.466667,0.705882}%
\pgfsetfillcolor{currentfill}%
\pgfsetfillopacity{0.528365}%
\pgfsetlinewidth{1.003750pt}%
\definecolor{currentstroke}{rgb}{0.121569,0.466667,0.705882}%
\pgfsetstrokecolor{currentstroke}%
\pgfsetstrokeopacity{0.528365}%
\pgfsetdash{}{0pt}%
\pgfpathmoveto{\pgfqpoint{1.172526in}{2.087249in}}%
\pgfpathcurveto{\pgfqpoint{1.180762in}{2.087249in}}{\pgfqpoint{1.188662in}{2.090521in}}{\pgfqpoint{1.194486in}{2.096345in}}%
\pgfpathcurveto{\pgfqpoint{1.200310in}{2.102169in}}{\pgfqpoint{1.203582in}{2.110069in}}{\pgfqpoint{1.203582in}{2.118305in}}%
\pgfpathcurveto{\pgfqpoint{1.203582in}{2.126541in}}{\pgfqpoint{1.200310in}{2.134441in}}{\pgfqpoint{1.194486in}{2.140265in}}%
\pgfpathcurveto{\pgfqpoint{1.188662in}{2.146089in}}{\pgfqpoint{1.180762in}{2.149362in}}{\pgfqpoint{1.172526in}{2.149362in}}%
\pgfpathcurveto{\pgfqpoint{1.164289in}{2.149362in}}{\pgfqpoint{1.156389in}{2.146089in}}{\pgfqpoint{1.150565in}{2.140265in}}%
\pgfpathcurveto{\pgfqpoint{1.144742in}{2.134441in}}{\pgfqpoint{1.141469in}{2.126541in}}{\pgfqpoint{1.141469in}{2.118305in}}%
\pgfpathcurveto{\pgfqpoint{1.141469in}{2.110069in}}{\pgfqpoint{1.144742in}{2.102169in}}{\pgfqpoint{1.150565in}{2.096345in}}%
\pgfpathcurveto{\pgfqpoint{1.156389in}{2.090521in}}{\pgfqpoint{1.164289in}{2.087249in}}{\pgfqpoint{1.172526in}{2.087249in}}%
\pgfpathclose%
\pgfusepath{stroke,fill}%
\end{pgfscope}%
\begin{pgfscope}%
\pgfpathrectangle{\pgfqpoint{0.100000in}{0.212622in}}{\pgfqpoint{3.696000in}{3.696000in}}%
\pgfusepath{clip}%
\pgfsetbuttcap%
\pgfsetroundjoin%
\definecolor{currentfill}{rgb}{0.121569,0.466667,0.705882}%
\pgfsetfillcolor{currentfill}%
\pgfsetfillopacity{0.528613}%
\pgfsetlinewidth{1.003750pt}%
\definecolor{currentstroke}{rgb}{0.121569,0.466667,0.705882}%
\pgfsetstrokecolor{currentstroke}%
\pgfsetstrokeopacity{0.528613}%
\pgfsetdash{}{0pt}%
\pgfpathmoveto{\pgfqpoint{2.055035in}{2.367393in}}%
\pgfpathcurveto{\pgfqpoint{2.063271in}{2.367393in}}{\pgfqpoint{2.071171in}{2.370665in}}{\pgfqpoint{2.076995in}{2.376489in}}%
\pgfpathcurveto{\pgfqpoint{2.082819in}{2.382313in}}{\pgfqpoint{2.086091in}{2.390213in}}{\pgfqpoint{2.086091in}{2.398449in}}%
\pgfpathcurveto{\pgfqpoint{2.086091in}{2.406685in}}{\pgfqpoint{2.082819in}{2.414585in}}{\pgfqpoint{2.076995in}{2.420409in}}%
\pgfpathcurveto{\pgfqpoint{2.071171in}{2.426233in}}{\pgfqpoint{2.063271in}{2.429506in}}{\pgfqpoint{2.055035in}{2.429506in}}%
\pgfpathcurveto{\pgfqpoint{2.046798in}{2.429506in}}{\pgfqpoint{2.038898in}{2.426233in}}{\pgfqpoint{2.033074in}{2.420409in}}%
\pgfpathcurveto{\pgfqpoint{2.027251in}{2.414585in}}{\pgfqpoint{2.023978in}{2.406685in}}{\pgfqpoint{2.023978in}{2.398449in}}%
\pgfpathcurveto{\pgfqpoint{2.023978in}{2.390213in}}{\pgfqpoint{2.027251in}{2.382313in}}{\pgfqpoint{2.033074in}{2.376489in}}%
\pgfpathcurveto{\pgfqpoint{2.038898in}{2.370665in}}{\pgfqpoint{2.046798in}{2.367393in}}{\pgfqpoint{2.055035in}{2.367393in}}%
\pgfpathclose%
\pgfusepath{stroke,fill}%
\end{pgfscope}%
\begin{pgfscope}%
\pgfpathrectangle{\pgfqpoint{0.100000in}{0.212622in}}{\pgfqpoint{3.696000in}{3.696000in}}%
\pgfusepath{clip}%
\pgfsetbuttcap%
\pgfsetroundjoin%
\definecolor{currentfill}{rgb}{0.121569,0.466667,0.705882}%
\pgfsetfillcolor{currentfill}%
\pgfsetfillopacity{0.530130}%
\pgfsetlinewidth{1.003750pt}%
\definecolor{currentstroke}{rgb}{0.121569,0.466667,0.705882}%
\pgfsetstrokecolor{currentstroke}%
\pgfsetstrokeopacity{0.530130}%
\pgfsetdash{}{0pt}%
\pgfpathmoveto{\pgfqpoint{1.166246in}{2.076342in}}%
\pgfpathcurveto{\pgfqpoint{1.174482in}{2.076342in}}{\pgfqpoint{1.182382in}{2.079614in}}{\pgfqpoint{1.188206in}{2.085438in}}%
\pgfpathcurveto{\pgfqpoint{1.194030in}{2.091262in}}{\pgfqpoint{1.197303in}{2.099162in}}{\pgfqpoint{1.197303in}{2.107398in}}%
\pgfpathcurveto{\pgfqpoint{1.197303in}{2.115634in}}{\pgfqpoint{1.194030in}{2.123534in}}{\pgfqpoint{1.188206in}{2.129358in}}%
\pgfpathcurveto{\pgfqpoint{1.182382in}{2.135182in}}{\pgfqpoint{1.174482in}{2.138455in}}{\pgfqpoint{1.166246in}{2.138455in}}%
\pgfpathcurveto{\pgfqpoint{1.158010in}{2.138455in}}{\pgfqpoint{1.150110in}{2.135182in}}{\pgfqpoint{1.144286in}{2.129358in}}%
\pgfpathcurveto{\pgfqpoint{1.138462in}{2.123534in}}{\pgfqpoint{1.135190in}{2.115634in}}{\pgfqpoint{1.135190in}{2.107398in}}%
\pgfpathcurveto{\pgfqpoint{1.135190in}{2.099162in}}{\pgfqpoint{1.138462in}{2.091262in}}{\pgfqpoint{1.144286in}{2.085438in}}%
\pgfpathcurveto{\pgfqpoint{1.150110in}{2.079614in}}{\pgfqpoint{1.158010in}{2.076342in}}{\pgfqpoint{1.166246in}{2.076342in}}%
\pgfpathclose%
\pgfusepath{stroke,fill}%
\end{pgfscope}%
\begin{pgfscope}%
\pgfpathrectangle{\pgfqpoint{0.100000in}{0.212622in}}{\pgfqpoint{3.696000in}{3.696000in}}%
\pgfusepath{clip}%
\pgfsetbuttcap%
\pgfsetroundjoin%
\definecolor{currentfill}{rgb}{0.121569,0.466667,0.705882}%
\pgfsetfillcolor{currentfill}%
\pgfsetfillopacity{0.531711}%
\pgfsetlinewidth{1.003750pt}%
\definecolor{currentstroke}{rgb}{0.121569,0.466667,0.705882}%
\pgfsetstrokecolor{currentstroke}%
\pgfsetstrokeopacity{0.531711}%
\pgfsetdash{}{0pt}%
\pgfpathmoveto{\pgfqpoint{1.161249in}{2.065556in}}%
\pgfpathcurveto{\pgfqpoint{1.169485in}{2.065556in}}{\pgfqpoint{1.177385in}{2.068828in}}{\pgfqpoint{1.183209in}{2.074652in}}%
\pgfpathcurveto{\pgfqpoint{1.189033in}{2.080476in}}{\pgfqpoint{1.192305in}{2.088376in}}{\pgfqpoint{1.192305in}{2.096612in}}%
\pgfpathcurveto{\pgfqpoint{1.192305in}{2.104848in}}{\pgfqpoint{1.189033in}{2.112748in}}{\pgfqpoint{1.183209in}{2.118572in}}%
\pgfpathcurveto{\pgfqpoint{1.177385in}{2.124396in}}{\pgfqpoint{1.169485in}{2.127669in}}{\pgfqpoint{1.161249in}{2.127669in}}%
\pgfpathcurveto{\pgfqpoint{1.153012in}{2.127669in}}{\pgfqpoint{1.145112in}{2.124396in}}{\pgfqpoint{1.139289in}{2.118572in}}%
\pgfpathcurveto{\pgfqpoint{1.133465in}{2.112748in}}{\pgfqpoint{1.130192in}{2.104848in}}{\pgfqpoint{1.130192in}{2.096612in}}%
\pgfpathcurveto{\pgfqpoint{1.130192in}{2.088376in}}{\pgfqpoint{1.133465in}{2.080476in}}{\pgfqpoint{1.139289in}{2.074652in}}%
\pgfpathcurveto{\pgfqpoint{1.145112in}{2.068828in}}{\pgfqpoint{1.153012in}{2.065556in}}{\pgfqpoint{1.161249in}{2.065556in}}%
\pgfpathclose%
\pgfusepath{stroke,fill}%
\end{pgfscope}%
\begin{pgfscope}%
\pgfpathrectangle{\pgfqpoint{0.100000in}{0.212622in}}{\pgfqpoint{3.696000in}{3.696000in}}%
\pgfusepath{clip}%
\pgfsetbuttcap%
\pgfsetroundjoin%
\definecolor{currentfill}{rgb}{0.121569,0.466667,0.705882}%
\pgfsetfillcolor{currentfill}%
\pgfsetfillopacity{0.531719}%
\pgfsetlinewidth{1.003750pt}%
\definecolor{currentstroke}{rgb}{0.121569,0.466667,0.705882}%
\pgfsetstrokecolor{currentstroke}%
\pgfsetstrokeopacity{0.531719}%
\pgfsetdash{}{0pt}%
\pgfpathmoveto{\pgfqpoint{2.057478in}{2.353735in}}%
\pgfpathcurveto{\pgfqpoint{2.065714in}{2.353735in}}{\pgfqpoint{2.073614in}{2.357008in}}{\pgfqpoint{2.079438in}{2.362831in}}%
\pgfpathcurveto{\pgfqpoint{2.085262in}{2.368655in}}{\pgfqpoint{2.088534in}{2.376555in}}{\pgfqpoint{2.088534in}{2.384792in}}%
\pgfpathcurveto{\pgfqpoint{2.088534in}{2.393028in}}{\pgfqpoint{2.085262in}{2.400928in}}{\pgfqpoint{2.079438in}{2.406752in}}%
\pgfpathcurveto{\pgfqpoint{2.073614in}{2.412576in}}{\pgfqpoint{2.065714in}{2.415848in}}{\pgfqpoint{2.057478in}{2.415848in}}%
\pgfpathcurveto{\pgfqpoint{2.049242in}{2.415848in}}{\pgfqpoint{2.041342in}{2.412576in}}{\pgfqpoint{2.035518in}{2.406752in}}%
\pgfpathcurveto{\pgfqpoint{2.029694in}{2.400928in}}{\pgfqpoint{2.026421in}{2.393028in}}{\pgfqpoint{2.026421in}{2.384792in}}%
\pgfpathcurveto{\pgfqpoint{2.026421in}{2.376555in}}{\pgfqpoint{2.029694in}{2.368655in}}{\pgfqpoint{2.035518in}{2.362831in}}%
\pgfpathcurveto{\pgfqpoint{2.041342in}{2.357008in}}{\pgfqpoint{2.049242in}{2.353735in}}{\pgfqpoint{2.057478in}{2.353735in}}%
\pgfpathclose%
\pgfusepath{stroke,fill}%
\end{pgfscope}%
\begin{pgfscope}%
\pgfpathrectangle{\pgfqpoint{0.100000in}{0.212622in}}{\pgfqpoint{3.696000in}{3.696000in}}%
\pgfusepath{clip}%
\pgfsetbuttcap%
\pgfsetroundjoin%
\definecolor{currentfill}{rgb}{0.121569,0.466667,0.705882}%
\pgfsetfillcolor{currentfill}%
\pgfsetfillopacity{0.533252}%
\pgfsetlinewidth{1.003750pt}%
\definecolor{currentstroke}{rgb}{0.121569,0.466667,0.705882}%
\pgfsetstrokecolor{currentstroke}%
\pgfsetstrokeopacity{0.533252}%
\pgfsetdash{}{0pt}%
\pgfpathmoveto{\pgfqpoint{1.156158in}{2.057111in}}%
\pgfpathcurveto{\pgfqpoint{1.164394in}{2.057111in}}{\pgfqpoint{1.172294in}{2.060383in}}{\pgfqpoint{1.178118in}{2.066207in}}%
\pgfpathcurveto{\pgfqpoint{1.183942in}{2.072031in}}{\pgfqpoint{1.187214in}{2.079931in}}{\pgfqpoint{1.187214in}{2.088168in}}%
\pgfpathcurveto{\pgfqpoint{1.187214in}{2.096404in}}{\pgfqpoint{1.183942in}{2.104304in}}{\pgfqpoint{1.178118in}{2.110128in}}%
\pgfpathcurveto{\pgfqpoint{1.172294in}{2.115952in}}{\pgfqpoint{1.164394in}{2.119224in}}{\pgfqpoint{1.156158in}{2.119224in}}%
\pgfpathcurveto{\pgfqpoint{1.147922in}{2.119224in}}{\pgfqpoint{1.140021in}{2.115952in}}{\pgfqpoint{1.134198in}{2.110128in}}%
\pgfpathcurveto{\pgfqpoint{1.128374in}{2.104304in}}{\pgfqpoint{1.125101in}{2.096404in}}{\pgfqpoint{1.125101in}{2.088168in}}%
\pgfpathcurveto{\pgfqpoint{1.125101in}{2.079931in}}{\pgfqpoint{1.128374in}{2.072031in}}{\pgfqpoint{1.134198in}{2.066207in}}%
\pgfpathcurveto{\pgfqpoint{1.140021in}{2.060383in}}{\pgfqpoint{1.147922in}{2.057111in}}{\pgfqpoint{1.156158in}{2.057111in}}%
\pgfpathclose%
\pgfusepath{stroke,fill}%
\end{pgfscope}%
\begin{pgfscope}%
\pgfpathrectangle{\pgfqpoint{0.100000in}{0.212622in}}{\pgfqpoint{3.696000in}{3.696000in}}%
\pgfusepath{clip}%
\pgfsetbuttcap%
\pgfsetroundjoin%
\definecolor{currentfill}{rgb}{0.121569,0.466667,0.705882}%
\pgfsetfillcolor{currentfill}%
\pgfsetfillopacity{0.534641}%
\pgfsetlinewidth{1.003750pt}%
\definecolor{currentstroke}{rgb}{0.121569,0.466667,0.705882}%
\pgfsetstrokecolor{currentstroke}%
\pgfsetstrokeopacity{0.534641}%
\pgfsetdash{}{0pt}%
\pgfpathmoveto{\pgfqpoint{1.151579in}{2.048589in}}%
\pgfpathcurveto{\pgfqpoint{1.159815in}{2.048589in}}{\pgfqpoint{1.167715in}{2.051861in}}{\pgfqpoint{1.173539in}{2.057685in}}%
\pgfpathcurveto{\pgfqpoint{1.179363in}{2.063509in}}{\pgfqpoint{1.182636in}{2.071409in}}{\pgfqpoint{1.182636in}{2.079645in}}%
\pgfpathcurveto{\pgfqpoint{1.182636in}{2.087882in}}{\pgfqpoint{1.179363in}{2.095782in}}{\pgfqpoint{1.173539in}{2.101606in}}%
\pgfpathcurveto{\pgfqpoint{1.167715in}{2.107430in}}{\pgfqpoint{1.159815in}{2.110702in}}{\pgfqpoint{1.151579in}{2.110702in}}%
\pgfpathcurveto{\pgfqpoint{1.143343in}{2.110702in}}{\pgfqpoint{1.135443in}{2.107430in}}{\pgfqpoint{1.129619in}{2.101606in}}%
\pgfpathcurveto{\pgfqpoint{1.123795in}{2.095782in}}{\pgfqpoint{1.120523in}{2.087882in}}{\pgfqpoint{1.120523in}{2.079645in}}%
\pgfpathcurveto{\pgfqpoint{1.120523in}{2.071409in}}{\pgfqpoint{1.123795in}{2.063509in}}{\pgfqpoint{1.129619in}{2.057685in}}%
\pgfpathcurveto{\pgfqpoint{1.135443in}{2.051861in}}{\pgfqpoint{1.143343in}{2.048589in}}{\pgfqpoint{1.151579in}{2.048589in}}%
\pgfpathclose%
\pgfusepath{stroke,fill}%
\end{pgfscope}%
\begin{pgfscope}%
\pgfpathrectangle{\pgfqpoint{0.100000in}{0.212622in}}{\pgfqpoint{3.696000in}{3.696000in}}%
\pgfusepath{clip}%
\pgfsetbuttcap%
\pgfsetroundjoin%
\definecolor{currentfill}{rgb}{0.121569,0.466667,0.705882}%
\pgfsetfillcolor{currentfill}%
\pgfsetfillopacity{0.535146}%
\pgfsetlinewidth{1.003750pt}%
\definecolor{currentstroke}{rgb}{0.121569,0.466667,0.705882}%
\pgfsetstrokecolor{currentstroke}%
\pgfsetstrokeopacity{0.535146}%
\pgfsetdash{}{0pt}%
\pgfpathmoveto{\pgfqpoint{1.149654in}{2.045189in}}%
\pgfpathcurveto{\pgfqpoint{1.157890in}{2.045189in}}{\pgfqpoint{1.165790in}{2.048461in}}{\pgfqpoint{1.171614in}{2.054285in}}%
\pgfpathcurveto{\pgfqpoint{1.177438in}{2.060109in}}{\pgfqpoint{1.180711in}{2.068009in}}{\pgfqpoint{1.180711in}{2.076245in}}%
\pgfpathcurveto{\pgfqpoint{1.180711in}{2.084482in}}{\pgfqpoint{1.177438in}{2.092382in}}{\pgfqpoint{1.171614in}{2.098206in}}%
\pgfpathcurveto{\pgfqpoint{1.165790in}{2.104030in}}{\pgfqpoint{1.157890in}{2.107302in}}{\pgfqpoint{1.149654in}{2.107302in}}%
\pgfpathcurveto{\pgfqpoint{1.141418in}{2.107302in}}{\pgfqpoint{1.133518in}{2.104030in}}{\pgfqpoint{1.127694in}{2.098206in}}%
\pgfpathcurveto{\pgfqpoint{1.121870in}{2.092382in}}{\pgfqpoint{1.118598in}{2.084482in}}{\pgfqpoint{1.118598in}{2.076245in}}%
\pgfpathcurveto{\pgfqpoint{1.118598in}{2.068009in}}{\pgfqpoint{1.121870in}{2.060109in}}{\pgfqpoint{1.127694in}{2.054285in}}%
\pgfpathcurveto{\pgfqpoint{1.133518in}{2.048461in}}{\pgfqpoint{1.141418in}{2.045189in}}{\pgfqpoint{1.149654in}{2.045189in}}%
\pgfpathclose%
\pgfusepath{stroke,fill}%
\end{pgfscope}%
\begin{pgfscope}%
\pgfpathrectangle{\pgfqpoint{0.100000in}{0.212622in}}{\pgfqpoint{3.696000in}{3.696000in}}%
\pgfusepath{clip}%
\pgfsetbuttcap%
\pgfsetroundjoin%
\definecolor{currentfill}{rgb}{0.121569,0.466667,0.705882}%
\pgfsetfillcolor{currentfill}%
\pgfsetfillopacity{0.535499}%
\pgfsetlinewidth{1.003750pt}%
\definecolor{currentstroke}{rgb}{0.121569,0.466667,0.705882}%
\pgfsetstrokecolor{currentstroke}%
\pgfsetstrokeopacity{0.535499}%
\pgfsetdash{}{0pt}%
\pgfpathmoveto{\pgfqpoint{2.058959in}{2.340050in}}%
\pgfpathcurveto{\pgfqpoint{2.067195in}{2.340050in}}{\pgfqpoint{2.075095in}{2.343322in}}{\pgfqpoint{2.080919in}{2.349146in}}%
\pgfpathcurveto{\pgfqpoint{2.086743in}{2.354970in}}{\pgfqpoint{2.090015in}{2.362870in}}{\pgfqpoint{2.090015in}{2.371106in}}%
\pgfpathcurveto{\pgfqpoint{2.090015in}{2.379342in}}{\pgfqpoint{2.086743in}{2.387242in}}{\pgfqpoint{2.080919in}{2.393066in}}%
\pgfpathcurveto{\pgfqpoint{2.075095in}{2.398890in}}{\pgfqpoint{2.067195in}{2.402163in}}{\pgfqpoint{2.058959in}{2.402163in}}%
\pgfpathcurveto{\pgfqpoint{2.050723in}{2.402163in}}{\pgfqpoint{2.042823in}{2.398890in}}{\pgfqpoint{2.036999in}{2.393066in}}%
\pgfpathcurveto{\pgfqpoint{2.031175in}{2.387242in}}{\pgfqpoint{2.027902in}{2.379342in}}{\pgfqpoint{2.027902in}{2.371106in}}%
\pgfpathcurveto{\pgfqpoint{2.027902in}{2.362870in}}{\pgfqpoint{2.031175in}{2.354970in}}{\pgfqpoint{2.036999in}{2.349146in}}%
\pgfpathcurveto{\pgfqpoint{2.042823in}{2.343322in}}{\pgfqpoint{2.050723in}{2.340050in}}{\pgfqpoint{2.058959in}{2.340050in}}%
\pgfpathclose%
\pgfusepath{stroke,fill}%
\end{pgfscope}%
\begin{pgfscope}%
\pgfpathrectangle{\pgfqpoint{0.100000in}{0.212622in}}{\pgfqpoint{3.696000in}{3.696000in}}%
\pgfusepath{clip}%
\pgfsetbuttcap%
\pgfsetroundjoin%
\definecolor{currentfill}{rgb}{0.121569,0.466667,0.705882}%
\pgfsetfillcolor{currentfill}%
\pgfsetfillopacity{0.535637}%
\pgfsetlinewidth{1.003750pt}%
\definecolor{currentstroke}{rgb}{0.121569,0.466667,0.705882}%
\pgfsetstrokecolor{currentstroke}%
\pgfsetstrokeopacity{0.535637}%
\pgfsetdash{}{0pt}%
\pgfpathmoveto{\pgfqpoint{1.148433in}{2.041709in}}%
\pgfpathcurveto{\pgfqpoint{1.156670in}{2.041709in}}{\pgfqpoint{1.164570in}{2.044981in}}{\pgfqpoint{1.170394in}{2.050805in}}%
\pgfpathcurveto{\pgfqpoint{1.176217in}{2.056629in}}{\pgfqpoint{1.179490in}{2.064529in}}{\pgfqpoint{1.179490in}{2.072766in}}%
\pgfpathcurveto{\pgfqpoint{1.179490in}{2.081002in}}{\pgfqpoint{1.176217in}{2.088902in}}{\pgfqpoint{1.170394in}{2.094726in}}%
\pgfpathcurveto{\pgfqpoint{1.164570in}{2.100550in}}{\pgfqpoint{1.156670in}{2.103822in}}{\pgfqpoint{1.148433in}{2.103822in}}%
\pgfpathcurveto{\pgfqpoint{1.140197in}{2.103822in}}{\pgfqpoint{1.132297in}{2.100550in}}{\pgfqpoint{1.126473in}{2.094726in}}%
\pgfpathcurveto{\pgfqpoint{1.120649in}{2.088902in}}{\pgfqpoint{1.117377in}{2.081002in}}{\pgfqpoint{1.117377in}{2.072766in}}%
\pgfpathcurveto{\pgfqpoint{1.117377in}{2.064529in}}{\pgfqpoint{1.120649in}{2.056629in}}{\pgfqpoint{1.126473in}{2.050805in}}%
\pgfpathcurveto{\pgfqpoint{1.132297in}{2.044981in}}{\pgfqpoint{1.140197in}{2.041709in}}{\pgfqpoint{1.148433in}{2.041709in}}%
\pgfpathclose%
\pgfusepath{stroke,fill}%
\end{pgfscope}%
\begin{pgfscope}%
\pgfpathrectangle{\pgfqpoint{0.100000in}{0.212622in}}{\pgfqpoint{3.696000in}{3.696000in}}%
\pgfusepath{clip}%
\pgfsetbuttcap%
\pgfsetroundjoin%
\definecolor{currentfill}{rgb}{0.121569,0.466667,0.705882}%
\pgfsetfillcolor{currentfill}%
\pgfsetfillopacity{0.535899}%
\pgfsetlinewidth{1.003750pt}%
\definecolor{currentstroke}{rgb}{0.121569,0.466667,0.705882}%
\pgfsetstrokecolor{currentstroke}%
\pgfsetstrokeopacity{0.535899}%
\pgfsetdash{}{0pt}%
\pgfpathmoveto{\pgfqpoint{1.147590in}{2.040414in}}%
\pgfpathcurveto{\pgfqpoint{1.155827in}{2.040414in}}{\pgfqpoint{1.163727in}{2.043686in}}{\pgfqpoint{1.169551in}{2.049510in}}%
\pgfpathcurveto{\pgfqpoint{1.175375in}{2.055334in}}{\pgfqpoint{1.178647in}{2.063234in}}{\pgfqpoint{1.178647in}{2.071471in}}%
\pgfpathcurveto{\pgfqpoint{1.178647in}{2.079707in}}{\pgfqpoint{1.175375in}{2.087607in}}{\pgfqpoint{1.169551in}{2.093431in}}%
\pgfpathcurveto{\pgfqpoint{1.163727in}{2.099255in}}{\pgfqpoint{1.155827in}{2.102527in}}{\pgfqpoint{1.147590in}{2.102527in}}%
\pgfpathcurveto{\pgfqpoint{1.139354in}{2.102527in}}{\pgfqpoint{1.131454in}{2.099255in}}{\pgfqpoint{1.125630in}{2.093431in}}%
\pgfpathcurveto{\pgfqpoint{1.119806in}{2.087607in}}{\pgfqpoint{1.116534in}{2.079707in}}{\pgfqpoint{1.116534in}{2.071471in}}%
\pgfpathcurveto{\pgfqpoint{1.116534in}{2.063234in}}{\pgfqpoint{1.119806in}{2.055334in}}{\pgfqpoint{1.125630in}{2.049510in}}%
\pgfpathcurveto{\pgfqpoint{1.131454in}{2.043686in}}{\pgfqpoint{1.139354in}{2.040414in}}{\pgfqpoint{1.147590in}{2.040414in}}%
\pgfpathclose%
\pgfusepath{stroke,fill}%
\end{pgfscope}%
\begin{pgfscope}%
\pgfpathrectangle{\pgfqpoint{0.100000in}{0.212622in}}{\pgfqpoint{3.696000in}{3.696000in}}%
\pgfusepath{clip}%
\pgfsetbuttcap%
\pgfsetroundjoin%
\definecolor{currentfill}{rgb}{0.121569,0.466667,0.705882}%
\pgfsetfillcolor{currentfill}%
\pgfsetfillopacity{0.536042}%
\pgfsetlinewidth{1.003750pt}%
\definecolor{currentstroke}{rgb}{0.121569,0.466667,0.705882}%
\pgfsetstrokecolor{currentstroke}%
\pgfsetstrokeopacity{0.536042}%
\pgfsetdash{}{0pt}%
\pgfpathmoveto{\pgfqpoint{1.147194in}{2.039717in}}%
\pgfpathcurveto{\pgfqpoint{1.155431in}{2.039717in}}{\pgfqpoint{1.163331in}{2.042989in}}{\pgfqpoint{1.169155in}{2.048813in}}%
\pgfpathcurveto{\pgfqpoint{1.174978in}{2.054637in}}{\pgfqpoint{1.178251in}{2.062537in}}{\pgfqpoint{1.178251in}{2.070774in}}%
\pgfpathcurveto{\pgfqpoint{1.178251in}{2.079010in}}{\pgfqpoint{1.174978in}{2.086910in}}{\pgfqpoint{1.169155in}{2.092734in}}%
\pgfpathcurveto{\pgfqpoint{1.163331in}{2.098558in}}{\pgfqpoint{1.155431in}{2.101830in}}{\pgfqpoint{1.147194in}{2.101830in}}%
\pgfpathcurveto{\pgfqpoint{1.138958in}{2.101830in}}{\pgfqpoint{1.131058in}{2.098558in}}{\pgfqpoint{1.125234in}{2.092734in}}%
\pgfpathcurveto{\pgfqpoint{1.119410in}{2.086910in}}{\pgfqpoint{1.116138in}{2.079010in}}{\pgfqpoint{1.116138in}{2.070774in}}%
\pgfpathcurveto{\pgfqpoint{1.116138in}{2.062537in}}{\pgfqpoint{1.119410in}{2.054637in}}{\pgfqpoint{1.125234in}{2.048813in}}%
\pgfpathcurveto{\pgfqpoint{1.131058in}{2.042989in}}{\pgfqpoint{1.138958in}{2.039717in}}{\pgfqpoint{1.147194in}{2.039717in}}%
\pgfpathclose%
\pgfusepath{stroke,fill}%
\end{pgfscope}%
\begin{pgfscope}%
\pgfpathrectangle{\pgfqpoint{0.100000in}{0.212622in}}{\pgfqpoint{3.696000in}{3.696000in}}%
\pgfusepath{clip}%
\pgfsetbuttcap%
\pgfsetroundjoin%
\definecolor{currentfill}{rgb}{0.121569,0.466667,0.705882}%
\pgfsetfillcolor{currentfill}%
\pgfsetfillopacity{0.536051}%
\pgfsetlinewidth{1.003750pt}%
\definecolor{currentstroke}{rgb}{0.121569,0.466667,0.705882}%
\pgfsetstrokecolor{currentstroke}%
\pgfsetstrokeopacity{0.536051}%
\pgfsetdash{}{0pt}%
\pgfpathmoveto{\pgfqpoint{1.147171in}{2.039683in}}%
\pgfpathcurveto{\pgfqpoint{1.155407in}{2.039683in}}{\pgfqpoint{1.163307in}{2.042956in}}{\pgfqpoint{1.169131in}{2.048780in}}%
\pgfpathcurveto{\pgfqpoint{1.174955in}{2.054604in}}{\pgfqpoint{1.178228in}{2.062504in}}{\pgfqpoint{1.178228in}{2.070740in}}%
\pgfpathcurveto{\pgfqpoint{1.178228in}{2.078976in}}{\pgfqpoint{1.174955in}{2.086876in}}{\pgfqpoint{1.169131in}{2.092700in}}%
\pgfpathcurveto{\pgfqpoint{1.163307in}{2.098524in}}{\pgfqpoint{1.155407in}{2.101796in}}{\pgfqpoint{1.147171in}{2.101796in}}%
\pgfpathcurveto{\pgfqpoint{1.138935in}{2.101796in}}{\pgfqpoint{1.131035in}{2.098524in}}{\pgfqpoint{1.125211in}{2.092700in}}%
\pgfpathcurveto{\pgfqpoint{1.119387in}{2.086876in}}{\pgfqpoint{1.116115in}{2.078976in}}{\pgfqpoint{1.116115in}{2.070740in}}%
\pgfpathcurveto{\pgfqpoint{1.116115in}{2.062504in}}{\pgfqpoint{1.119387in}{2.054604in}}{\pgfqpoint{1.125211in}{2.048780in}}%
\pgfpathcurveto{\pgfqpoint{1.131035in}{2.042956in}}{\pgfqpoint{1.138935in}{2.039683in}}{\pgfqpoint{1.147171in}{2.039683in}}%
\pgfpathclose%
\pgfusepath{stroke,fill}%
\end{pgfscope}%
\begin{pgfscope}%
\pgfpathrectangle{\pgfqpoint{0.100000in}{0.212622in}}{\pgfqpoint{3.696000in}{3.696000in}}%
\pgfusepath{clip}%
\pgfsetbuttcap%
\pgfsetroundjoin%
\definecolor{currentfill}{rgb}{0.121569,0.466667,0.705882}%
\pgfsetfillcolor{currentfill}%
\pgfsetfillopacity{0.536066}%
\pgfsetlinewidth{1.003750pt}%
\definecolor{currentstroke}{rgb}{0.121569,0.466667,0.705882}%
\pgfsetstrokecolor{currentstroke}%
\pgfsetstrokeopacity{0.536066}%
\pgfsetdash{}{0pt}%
\pgfpathmoveto{\pgfqpoint{1.147130in}{2.039620in}}%
\pgfpathcurveto{\pgfqpoint{1.155366in}{2.039620in}}{\pgfqpoint{1.163266in}{2.042892in}}{\pgfqpoint{1.169090in}{2.048716in}}%
\pgfpathcurveto{\pgfqpoint{1.174914in}{2.054540in}}{\pgfqpoint{1.178187in}{2.062440in}}{\pgfqpoint{1.178187in}{2.070676in}}%
\pgfpathcurveto{\pgfqpoint{1.178187in}{2.078913in}}{\pgfqpoint{1.174914in}{2.086813in}}{\pgfqpoint{1.169090in}{2.092637in}}%
\pgfpathcurveto{\pgfqpoint{1.163266in}{2.098461in}}{\pgfqpoint{1.155366in}{2.101733in}}{\pgfqpoint{1.147130in}{2.101733in}}%
\pgfpathcurveto{\pgfqpoint{1.138894in}{2.101733in}}{\pgfqpoint{1.130994in}{2.098461in}}{\pgfqpoint{1.125170in}{2.092637in}}%
\pgfpathcurveto{\pgfqpoint{1.119346in}{2.086813in}}{\pgfqpoint{1.116074in}{2.078913in}}{\pgfqpoint{1.116074in}{2.070676in}}%
\pgfpathcurveto{\pgfqpoint{1.116074in}{2.062440in}}{\pgfqpoint{1.119346in}{2.054540in}}{\pgfqpoint{1.125170in}{2.048716in}}%
\pgfpathcurveto{\pgfqpoint{1.130994in}{2.042892in}}{\pgfqpoint{1.138894in}{2.039620in}}{\pgfqpoint{1.147130in}{2.039620in}}%
\pgfpathclose%
\pgfusepath{stroke,fill}%
\end{pgfscope}%
\begin{pgfscope}%
\pgfpathrectangle{\pgfqpoint{0.100000in}{0.212622in}}{\pgfqpoint{3.696000in}{3.696000in}}%
\pgfusepath{clip}%
\pgfsetbuttcap%
\pgfsetroundjoin%
\definecolor{currentfill}{rgb}{0.121569,0.466667,0.705882}%
\pgfsetfillcolor{currentfill}%
\pgfsetfillopacity{0.536092}%
\pgfsetlinewidth{1.003750pt}%
\definecolor{currentstroke}{rgb}{0.121569,0.466667,0.705882}%
\pgfsetstrokecolor{currentstroke}%
\pgfsetstrokeopacity{0.536092}%
\pgfsetdash{}{0pt}%
\pgfpathmoveto{\pgfqpoint{1.147052in}{2.039505in}}%
\pgfpathcurveto{\pgfqpoint{1.155288in}{2.039505in}}{\pgfqpoint{1.163189in}{2.042777in}}{\pgfqpoint{1.169012in}{2.048601in}}%
\pgfpathcurveto{\pgfqpoint{1.174836in}{2.054425in}}{\pgfqpoint{1.178109in}{2.062325in}}{\pgfqpoint{1.178109in}{2.070562in}}%
\pgfpathcurveto{\pgfqpoint{1.178109in}{2.078798in}}{\pgfqpoint{1.174836in}{2.086698in}}{\pgfqpoint{1.169012in}{2.092522in}}%
\pgfpathcurveto{\pgfqpoint{1.163189in}{2.098346in}}{\pgfqpoint{1.155288in}{2.101618in}}{\pgfqpoint{1.147052in}{2.101618in}}%
\pgfpathcurveto{\pgfqpoint{1.138816in}{2.101618in}}{\pgfqpoint{1.130916in}{2.098346in}}{\pgfqpoint{1.125092in}{2.092522in}}%
\pgfpathcurveto{\pgfqpoint{1.119268in}{2.086698in}}{\pgfqpoint{1.115996in}{2.078798in}}{\pgfqpoint{1.115996in}{2.070562in}}%
\pgfpathcurveto{\pgfqpoint{1.115996in}{2.062325in}}{\pgfqpoint{1.119268in}{2.054425in}}{\pgfqpoint{1.125092in}{2.048601in}}%
\pgfpathcurveto{\pgfqpoint{1.130916in}{2.042777in}}{\pgfqpoint{1.138816in}{2.039505in}}{\pgfqpoint{1.147052in}{2.039505in}}%
\pgfpathclose%
\pgfusepath{stroke,fill}%
\end{pgfscope}%
\begin{pgfscope}%
\pgfpathrectangle{\pgfqpoint{0.100000in}{0.212622in}}{\pgfqpoint{3.696000in}{3.696000in}}%
\pgfusepath{clip}%
\pgfsetbuttcap%
\pgfsetroundjoin%
\definecolor{currentfill}{rgb}{0.121569,0.466667,0.705882}%
\pgfsetfillcolor{currentfill}%
\pgfsetfillopacity{0.536139}%
\pgfsetlinewidth{1.003750pt}%
\definecolor{currentstroke}{rgb}{0.121569,0.466667,0.705882}%
\pgfsetstrokecolor{currentstroke}%
\pgfsetstrokeopacity{0.536139}%
\pgfsetdash{}{0pt}%
\pgfpathmoveto{\pgfqpoint{1.146911in}{2.039294in}}%
\pgfpathcurveto{\pgfqpoint{1.155148in}{2.039294in}}{\pgfqpoint{1.163048in}{2.042566in}}{\pgfqpoint{1.168871in}{2.048390in}}%
\pgfpathcurveto{\pgfqpoint{1.174695in}{2.054214in}}{\pgfqpoint{1.177968in}{2.062114in}}{\pgfqpoint{1.177968in}{2.070351in}}%
\pgfpathcurveto{\pgfqpoint{1.177968in}{2.078587in}}{\pgfqpoint{1.174695in}{2.086487in}}{\pgfqpoint{1.168871in}{2.092311in}}%
\pgfpathcurveto{\pgfqpoint{1.163048in}{2.098135in}}{\pgfqpoint{1.155148in}{2.101407in}}{\pgfqpoint{1.146911in}{2.101407in}}%
\pgfpathcurveto{\pgfqpoint{1.138675in}{2.101407in}}{\pgfqpoint{1.130775in}{2.098135in}}{\pgfqpoint{1.124951in}{2.092311in}}%
\pgfpathcurveto{\pgfqpoint{1.119127in}{2.086487in}}{\pgfqpoint{1.115855in}{2.078587in}}{\pgfqpoint{1.115855in}{2.070351in}}%
\pgfpathcurveto{\pgfqpoint{1.115855in}{2.062114in}}{\pgfqpoint{1.119127in}{2.054214in}}{\pgfqpoint{1.124951in}{2.048390in}}%
\pgfpathcurveto{\pgfqpoint{1.130775in}{2.042566in}}{\pgfqpoint{1.138675in}{2.039294in}}{\pgfqpoint{1.146911in}{2.039294in}}%
\pgfpathclose%
\pgfusepath{stroke,fill}%
\end{pgfscope}%
\begin{pgfscope}%
\pgfpathrectangle{\pgfqpoint{0.100000in}{0.212622in}}{\pgfqpoint{3.696000in}{3.696000in}}%
\pgfusepath{clip}%
\pgfsetbuttcap%
\pgfsetroundjoin%
\definecolor{currentfill}{rgb}{0.121569,0.466667,0.705882}%
\pgfsetfillcolor{currentfill}%
\pgfsetfillopacity{0.536234}%
\pgfsetlinewidth{1.003750pt}%
\definecolor{currentstroke}{rgb}{0.121569,0.466667,0.705882}%
\pgfsetstrokecolor{currentstroke}%
\pgfsetstrokeopacity{0.536234}%
\pgfsetdash{}{0pt}%
\pgfpathmoveto{\pgfqpoint{1.146674in}{2.038914in}}%
\pgfpathcurveto{\pgfqpoint{1.154910in}{2.038914in}}{\pgfqpoint{1.162810in}{2.042187in}}{\pgfqpoint{1.168634in}{2.048011in}}%
\pgfpathcurveto{\pgfqpoint{1.174458in}{2.053834in}}{\pgfqpoint{1.177730in}{2.061735in}}{\pgfqpoint{1.177730in}{2.069971in}}%
\pgfpathcurveto{\pgfqpoint{1.177730in}{2.078207in}}{\pgfqpoint{1.174458in}{2.086107in}}{\pgfqpoint{1.168634in}{2.091931in}}%
\pgfpathcurveto{\pgfqpoint{1.162810in}{2.097755in}}{\pgfqpoint{1.154910in}{2.101027in}}{\pgfqpoint{1.146674in}{2.101027in}}%
\pgfpathcurveto{\pgfqpoint{1.138438in}{2.101027in}}{\pgfqpoint{1.130538in}{2.097755in}}{\pgfqpoint{1.124714in}{2.091931in}}%
\pgfpathcurveto{\pgfqpoint{1.118890in}{2.086107in}}{\pgfqpoint{1.115617in}{2.078207in}}{\pgfqpoint{1.115617in}{2.069971in}}%
\pgfpathcurveto{\pgfqpoint{1.115617in}{2.061735in}}{\pgfqpoint{1.118890in}{2.053834in}}{\pgfqpoint{1.124714in}{2.048011in}}%
\pgfpathcurveto{\pgfqpoint{1.130538in}{2.042187in}}{\pgfqpoint{1.138438in}{2.038914in}}{\pgfqpoint{1.146674in}{2.038914in}}%
\pgfpathclose%
\pgfusepath{stroke,fill}%
\end{pgfscope}%
\begin{pgfscope}%
\pgfpathrectangle{\pgfqpoint{0.100000in}{0.212622in}}{\pgfqpoint{3.696000in}{3.696000in}}%
\pgfusepath{clip}%
\pgfsetbuttcap%
\pgfsetroundjoin%
\definecolor{currentfill}{rgb}{0.121569,0.466667,0.705882}%
\pgfsetfillcolor{currentfill}%
\pgfsetfillopacity{0.536398}%
\pgfsetlinewidth{1.003750pt}%
\definecolor{currentstroke}{rgb}{0.121569,0.466667,0.705882}%
\pgfsetstrokecolor{currentstroke}%
\pgfsetstrokeopacity{0.536398}%
\pgfsetdash{}{0pt}%
\pgfpathmoveto{\pgfqpoint{1.146227in}{2.038217in}}%
\pgfpathcurveto{\pgfqpoint{1.154463in}{2.038217in}}{\pgfqpoint{1.162363in}{2.041489in}}{\pgfqpoint{1.168187in}{2.047313in}}%
\pgfpathcurveto{\pgfqpoint{1.174011in}{2.053137in}}{\pgfqpoint{1.177283in}{2.061037in}}{\pgfqpoint{1.177283in}{2.069273in}}%
\pgfpathcurveto{\pgfqpoint{1.177283in}{2.077510in}}{\pgfqpoint{1.174011in}{2.085410in}}{\pgfqpoint{1.168187in}{2.091234in}}%
\pgfpathcurveto{\pgfqpoint{1.162363in}{2.097057in}}{\pgfqpoint{1.154463in}{2.100330in}}{\pgfqpoint{1.146227in}{2.100330in}}%
\pgfpathcurveto{\pgfqpoint{1.137991in}{2.100330in}}{\pgfqpoint{1.130090in}{2.097057in}}{\pgfqpoint{1.124267in}{2.091234in}}%
\pgfpathcurveto{\pgfqpoint{1.118443in}{2.085410in}}{\pgfqpoint{1.115170in}{2.077510in}}{\pgfqpoint{1.115170in}{2.069273in}}%
\pgfpathcurveto{\pgfqpoint{1.115170in}{2.061037in}}{\pgfqpoint{1.118443in}{2.053137in}}{\pgfqpoint{1.124267in}{2.047313in}}%
\pgfpathcurveto{\pgfqpoint{1.130090in}{2.041489in}}{\pgfqpoint{1.137991in}{2.038217in}}{\pgfqpoint{1.146227in}{2.038217in}}%
\pgfpathclose%
\pgfusepath{stroke,fill}%
\end{pgfscope}%
\begin{pgfscope}%
\pgfpathrectangle{\pgfqpoint{0.100000in}{0.212622in}}{\pgfqpoint{3.696000in}{3.696000in}}%
\pgfusepath{clip}%
\pgfsetbuttcap%
\pgfsetroundjoin%
\definecolor{currentfill}{rgb}{0.121569,0.466667,0.705882}%
\pgfsetfillcolor{currentfill}%
\pgfsetfillopacity{0.536689}%
\pgfsetlinewidth{1.003750pt}%
\definecolor{currentstroke}{rgb}{0.121569,0.466667,0.705882}%
\pgfsetstrokecolor{currentstroke}%
\pgfsetstrokeopacity{0.536689}%
\pgfsetdash{}{0pt}%
\pgfpathmoveto{\pgfqpoint{1.145437in}{2.036889in}}%
\pgfpathcurveto{\pgfqpoint{1.153674in}{2.036889in}}{\pgfqpoint{1.161574in}{2.040161in}}{\pgfqpoint{1.167398in}{2.045985in}}%
\pgfpathcurveto{\pgfqpoint{1.173222in}{2.051809in}}{\pgfqpoint{1.176494in}{2.059709in}}{\pgfqpoint{1.176494in}{2.067945in}}%
\pgfpathcurveto{\pgfqpoint{1.176494in}{2.076182in}}{\pgfqpoint{1.173222in}{2.084082in}}{\pgfqpoint{1.167398in}{2.089906in}}%
\pgfpathcurveto{\pgfqpoint{1.161574in}{2.095729in}}{\pgfqpoint{1.153674in}{2.099002in}}{\pgfqpoint{1.145437in}{2.099002in}}%
\pgfpathcurveto{\pgfqpoint{1.137201in}{2.099002in}}{\pgfqpoint{1.129301in}{2.095729in}}{\pgfqpoint{1.123477in}{2.089906in}}%
\pgfpathcurveto{\pgfqpoint{1.117653in}{2.084082in}}{\pgfqpoint{1.114381in}{2.076182in}}{\pgfqpoint{1.114381in}{2.067945in}}%
\pgfpathcurveto{\pgfqpoint{1.114381in}{2.059709in}}{\pgfqpoint{1.117653in}{2.051809in}}{\pgfqpoint{1.123477in}{2.045985in}}%
\pgfpathcurveto{\pgfqpoint{1.129301in}{2.040161in}}{\pgfqpoint{1.137201in}{2.036889in}}{\pgfqpoint{1.145437in}{2.036889in}}%
\pgfpathclose%
\pgfusepath{stroke,fill}%
\end{pgfscope}%
\begin{pgfscope}%
\pgfpathrectangle{\pgfqpoint{0.100000in}{0.212622in}}{\pgfqpoint{3.696000in}{3.696000in}}%
\pgfusepath{clip}%
\pgfsetbuttcap%
\pgfsetroundjoin%
\definecolor{currentfill}{rgb}{0.121569,0.466667,0.705882}%
\pgfsetfillcolor{currentfill}%
\pgfsetfillopacity{0.537223}%
\pgfsetlinewidth{1.003750pt}%
\definecolor{currentstroke}{rgb}{0.121569,0.466667,0.705882}%
\pgfsetstrokecolor{currentstroke}%
\pgfsetstrokeopacity{0.537223}%
\pgfsetdash{}{0pt}%
\pgfpathmoveto{\pgfqpoint{1.143961in}{2.034550in}}%
\pgfpathcurveto{\pgfqpoint{1.152197in}{2.034550in}}{\pgfqpoint{1.160097in}{2.037823in}}{\pgfqpoint{1.165921in}{2.043647in}}%
\pgfpathcurveto{\pgfqpoint{1.171745in}{2.049471in}}{\pgfqpoint{1.175017in}{2.057371in}}{\pgfqpoint{1.175017in}{2.065607in}}%
\pgfpathcurveto{\pgfqpoint{1.175017in}{2.073843in}}{\pgfqpoint{1.171745in}{2.081743in}}{\pgfqpoint{1.165921in}{2.087567in}}%
\pgfpathcurveto{\pgfqpoint{1.160097in}{2.093391in}}{\pgfqpoint{1.152197in}{2.096663in}}{\pgfqpoint{1.143961in}{2.096663in}}%
\pgfpathcurveto{\pgfqpoint{1.135724in}{2.096663in}}{\pgfqpoint{1.127824in}{2.093391in}}{\pgfqpoint{1.122000in}{2.087567in}}%
\pgfpathcurveto{\pgfqpoint{1.116176in}{2.081743in}}{\pgfqpoint{1.112904in}{2.073843in}}{\pgfqpoint{1.112904in}{2.065607in}}%
\pgfpathcurveto{\pgfqpoint{1.112904in}{2.057371in}}{\pgfqpoint{1.116176in}{2.049471in}}{\pgfqpoint{1.122000in}{2.043647in}}%
\pgfpathcurveto{\pgfqpoint{1.127824in}{2.037823in}}{\pgfqpoint{1.135724in}{2.034550in}}{\pgfqpoint{1.143961in}{2.034550in}}%
\pgfpathclose%
\pgfusepath{stroke,fill}%
\end{pgfscope}%
\begin{pgfscope}%
\pgfpathrectangle{\pgfqpoint{0.100000in}{0.212622in}}{\pgfqpoint{3.696000in}{3.696000in}}%
\pgfusepath{clip}%
\pgfsetbuttcap%
\pgfsetroundjoin%
\definecolor{currentfill}{rgb}{0.121569,0.466667,0.705882}%
\pgfsetfillcolor{currentfill}%
\pgfsetfillopacity{0.537619}%
\pgfsetlinewidth{1.003750pt}%
\definecolor{currentstroke}{rgb}{0.121569,0.466667,0.705882}%
\pgfsetstrokecolor{currentstroke}%
\pgfsetstrokeopacity{0.537619}%
\pgfsetdash{}{0pt}%
\pgfpathmoveto{\pgfqpoint{1.142899in}{2.032721in}}%
\pgfpathcurveto{\pgfqpoint{1.151135in}{2.032721in}}{\pgfqpoint{1.159035in}{2.035994in}}{\pgfqpoint{1.164859in}{2.041818in}}%
\pgfpathcurveto{\pgfqpoint{1.170683in}{2.047641in}}{\pgfqpoint{1.173955in}{2.055541in}}{\pgfqpoint{1.173955in}{2.063778in}}%
\pgfpathcurveto{\pgfqpoint{1.173955in}{2.072014in}}{\pgfqpoint{1.170683in}{2.079914in}}{\pgfqpoint{1.164859in}{2.085738in}}%
\pgfpathcurveto{\pgfqpoint{1.159035in}{2.091562in}}{\pgfqpoint{1.151135in}{2.094834in}}{\pgfqpoint{1.142899in}{2.094834in}}%
\pgfpathcurveto{\pgfqpoint{1.134662in}{2.094834in}}{\pgfqpoint{1.126762in}{2.091562in}}{\pgfqpoint{1.120938in}{2.085738in}}%
\pgfpathcurveto{\pgfqpoint{1.115114in}{2.079914in}}{\pgfqpoint{1.111842in}{2.072014in}}{\pgfqpoint{1.111842in}{2.063778in}}%
\pgfpathcurveto{\pgfqpoint{1.111842in}{2.055541in}}{\pgfqpoint{1.115114in}{2.047641in}}{\pgfqpoint{1.120938in}{2.041818in}}%
\pgfpathcurveto{\pgfqpoint{1.126762in}{2.035994in}}{\pgfqpoint{1.134662in}{2.032721in}}{\pgfqpoint{1.142899in}{2.032721in}}%
\pgfpathclose%
\pgfusepath{stroke,fill}%
\end{pgfscope}%
\begin{pgfscope}%
\pgfpathrectangle{\pgfqpoint{0.100000in}{0.212622in}}{\pgfqpoint{3.696000in}{3.696000in}}%
\pgfusepath{clip}%
\pgfsetbuttcap%
\pgfsetroundjoin%
\definecolor{currentfill}{rgb}{0.121569,0.466667,0.705882}%
\pgfsetfillcolor{currentfill}%
\pgfsetfillopacity{0.538285}%
\pgfsetlinewidth{1.003750pt}%
\definecolor{currentstroke}{rgb}{0.121569,0.466667,0.705882}%
\pgfsetstrokecolor{currentstroke}%
\pgfsetstrokeopacity{0.538285}%
\pgfsetdash{}{0pt}%
\pgfpathmoveto{\pgfqpoint{1.140858in}{2.029352in}}%
\pgfpathcurveto{\pgfqpoint{1.149094in}{2.029352in}}{\pgfqpoint{1.156994in}{2.032624in}}{\pgfqpoint{1.162818in}{2.038448in}}%
\pgfpathcurveto{\pgfqpoint{1.168642in}{2.044272in}}{\pgfqpoint{1.171914in}{2.052172in}}{\pgfqpoint{1.171914in}{2.060408in}}%
\pgfpathcurveto{\pgfqpoint{1.171914in}{2.068645in}}{\pgfqpoint{1.168642in}{2.076545in}}{\pgfqpoint{1.162818in}{2.082369in}}%
\pgfpathcurveto{\pgfqpoint{1.156994in}{2.088193in}}{\pgfqpoint{1.149094in}{2.091465in}}{\pgfqpoint{1.140858in}{2.091465in}}%
\pgfpathcurveto{\pgfqpoint{1.132621in}{2.091465in}}{\pgfqpoint{1.124721in}{2.088193in}}{\pgfqpoint{1.118897in}{2.082369in}}%
\pgfpathcurveto{\pgfqpoint{1.113073in}{2.076545in}}{\pgfqpoint{1.109801in}{2.068645in}}{\pgfqpoint{1.109801in}{2.060408in}}%
\pgfpathcurveto{\pgfqpoint{1.109801in}{2.052172in}}{\pgfqpoint{1.113073in}{2.044272in}}{\pgfqpoint{1.118897in}{2.038448in}}%
\pgfpathcurveto{\pgfqpoint{1.124721in}{2.032624in}}{\pgfqpoint{1.132621in}{2.029352in}}{\pgfqpoint{1.140858in}{2.029352in}}%
\pgfpathclose%
\pgfusepath{stroke,fill}%
\end{pgfscope}%
\begin{pgfscope}%
\pgfpathrectangle{\pgfqpoint{0.100000in}{0.212622in}}{\pgfqpoint{3.696000in}{3.696000in}}%
\pgfusepath{clip}%
\pgfsetbuttcap%
\pgfsetroundjoin%
\definecolor{currentfill}{rgb}{0.121569,0.466667,0.705882}%
\pgfsetfillcolor{currentfill}%
\pgfsetfillopacity{0.539529}%
\pgfsetlinewidth{1.003750pt}%
\definecolor{currentstroke}{rgb}{0.121569,0.466667,0.705882}%
\pgfsetstrokecolor{currentstroke}%
\pgfsetstrokeopacity{0.539529}%
\pgfsetdash{}{0pt}%
\pgfpathmoveto{\pgfqpoint{1.137437in}{2.022958in}}%
\pgfpathcurveto{\pgfqpoint{1.145673in}{2.022958in}}{\pgfqpoint{1.153574in}{2.026230in}}{\pgfqpoint{1.159397in}{2.032054in}}%
\pgfpathcurveto{\pgfqpoint{1.165221in}{2.037878in}}{\pgfqpoint{1.168494in}{2.045778in}}{\pgfqpoint{1.168494in}{2.054015in}}%
\pgfpathcurveto{\pgfqpoint{1.168494in}{2.062251in}}{\pgfqpoint{1.165221in}{2.070151in}}{\pgfqpoint{1.159397in}{2.075975in}}%
\pgfpathcurveto{\pgfqpoint{1.153574in}{2.081799in}}{\pgfqpoint{1.145673in}{2.085071in}}{\pgfqpoint{1.137437in}{2.085071in}}%
\pgfpathcurveto{\pgfqpoint{1.129201in}{2.085071in}}{\pgfqpoint{1.121301in}{2.081799in}}{\pgfqpoint{1.115477in}{2.075975in}}%
\pgfpathcurveto{\pgfqpoint{1.109653in}{2.070151in}}{\pgfqpoint{1.106381in}{2.062251in}}{\pgfqpoint{1.106381in}{2.054015in}}%
\pgfpathcurveto{\pgfqpoint{1.106381in}{2.045778in}}{\pgfqpoint{1.109653in}{2.037878in}}{\pgfqpoint{1.115477in}{2.032054in}}%
\pgfpathcurveto{\pgfqpoint{1.121301in}{2.026230in}}{\pgfqpoint{1.129201in}{2.022958in}}{\pgfqpoint{1.137437in}{2.022958in}}%
\pgfpathclose%
\pgfusepath{stroke,fill}%
\end{pgfscope}%
\begin{pgfscope}%
\pgfpathrectangle{\pgfqpoint{0.100000in}{0.212622in}}{\pgfqpoint{3.696000in}{3.696000in}}%
\pgfusepath{clip}%
\pgfsetbuttcap%
\pgfsetroundjoin%
\definecolor{currentfill}{rgb}{0.121569,0.466667,0.705882}%
\pgfsetfillcolor{currentfill}%
\pgfsetfillopacity{0.539605}%
\pgfsetlinewidth{1.003750pt}%
\definecolor{currentstroke}{rgb}{0.121569,0.466667,0.705882}%
\pgfsetstrokecolor{currentstroke}%
\pgfsetstrokeopacity{0.539605}%
\pgfsetdash{}{0pt}%
\pgfpathmoveto{\pgfqpoint{2.060432in}{2.326475in}}%
\pgfpathcurveto{\pgfqpoint{2.068669in}{2.326475in}}{\pgfqpoint{2.076569in}{2.329748in}}{\pgfqpoint{2.082393in}{2.335572in}}%
\pgfpathcurveto{\pgfqpoint{2.088217in}{2.341396in}}{\pgfqpoint{2.091489in}{2.349296in}}{\pgfqpoint{2.091489in}{2.357532in}}%
\pgfpathcurveto{\pgfqpoint{2.091489in}{2.365768in}}{\pgfqpoint{2.088217in}{2.373668in}}{\pgfqpoint{2.082393in}{2.379492in}}%
\pgfpathcurveto{\pgfqpoint{2.076569in}{2.385316in}}{\pgfqpoint{2.068669in}{2.388588in}}{\pgfqpoint{2.060432in}{2.388588in}}%
\pgfpathcurveto{\pgfqpoint{2.052196in}{2.388588in}}{\pgfqpoint{2.044296in}{2.385316in}}{\pgfqpoint{2.038472in}{2.379492in}}%
\pgfpathcurveto{\pgfqpoint{2.032648in}{2.373668in}}{\pgfqpoint{2.029376in}{2.365768in}}{\pgfqpoint{2.029376in}{2.357532in}}%
\pgfpathcurveto{\pgfqpoint{2.029376in}{2.349296in}}{\pgfqpoint{2.032648in}{2.341396in}}{\pgfqpoint{2.038472in}{2.335572in}}%
\pgfpathcurveto{\pgfqpoint{2.044296in}{2.329748in}}{\pgfqpoint{2.052196in}{2.326475in}}{\pgfqpoint{2.060432in}{2.326475in}}%
\pgfpathclose%
\pgfusepath{stroke,fill}%
\end{pgfscope}%
\begin{pgfscope}%
\pgfpathrectangle{\pgfqpoint{0.100000in}{0.212622in}}{\pgfqpoint{3.696000in}{3.696000in}}%
\pgfusepath{clip}%
\pgfsetbuttcap%
\pgfsetroundjoin%
\definecolor{currentfill}{rgb}{0.121569,0.466667,0.705882}%
\pgfsetfillcolor{currentfill}%
\pgfsetfillopacity{0.541720}%
\pgfsetlinewidth{1.003750pt}%
\definecolor{currentstroke}{rgb}{0.121569,0.466667,0.705882}%
\pgfsetstrokecolor{currentstroke}%
\pgfsetstrokeopacity{0.541720}%
\pgfsetdash{}{0pt}%
\pgfpathmoveto{\pgfqpoint{1.130635in}{2.011847in}}%
\pgfpathcurveto{\pgfqpoint{1.138871in}{2.011847in}}{\pgfqpoint{1.146771in}{2.015119in}}{\pgfqpoint{1.152595in}{2.020943in}}%
\pgfpathcurveto{\pgfqpoint{1.158419in}{2.026767in}}{\pgfqpoint{1.161691in}{2.034667in}}{\pgfqpoint{1.161691in}{2.042903in}}%
\pgfpathcurveto{\pgfqpoint{1.161691in}{2.051140in}}{\pgfqpoint{1.158419in}{2.059040in}}{\pgfqpoint{1.152595in}{2.064864in}}%
\pgfpathcurveto{\pgfqpoint{1.146771in}{2.070688in}}{\pgfqpoint{1.138871in}{2.073960in}}{\pgfqpoint{1.130635in}{2.073960in}}%
\pgfpathcurveto{\pgfqpoint{1.122398in}{2.073960in}}{\pgfqpoint{1.114498in}{2.070688in}}{\pgfqpoint{1.108674in}{2.064864in}}%
\pgfpathcurveto{\pgfqpoint{1.102851in}{2.059040in}}{\pgfqpoint{1.099578in}{2.051140in}}{\pgfqpoint{1.099578in}{2.042903in}}%
\pgfpathcurveto{\pgfqpoint{1.099578in}{2.034667in}}{\pgfqpoint{1.102851in}{2.026767in}}{\pgfqpoint{1.108674in}{2.020943in}}%
\pgfpathcurveto{\pgfqpoint{1.114498in}{2.015119in}}{\pgfqpoint{1.122398in}{2.011847in}}{\pgfqpoint{1.130635in}{2.011847in}}%
\pgfpathclose%
\pgfusepath{stroke,fill}%
\end{pgfscope}%
\begin{pgfscope}%
\pgfpathrectangle{\pgfqpoint{0.100000in}{0.212622in}}{\pgfqpoint{3.696000in}{3.696000in}}%
\pgfusepath{clip}%
\pgfsetbuttcap%
\pgfsetroundjoin%
\definecolor{currentfill}{rgb}{0.121569,0.466667,0.705882}%
\pgfsetfillcolor{currentfill}%
\pgfsetfillopacity{0.543548}%
\pgfsetlinewidth{1.003750pt}%
\definecolor{currentstroke}{rgb}{0.121569,0.466667,0.705882}%
\pgfsetstrokecolor{currentstroke}%
\pgfsetstrokeopacity{0.543548}%
\pgfsetdash{}{0pt}%
\pgfpathmoveto{\pgfqpoint{2.063555in}{2.311696in}}%
\pgfpathcurveto{\pgfqpoint{2.071791in}{2.311696in}}{\pgfqpoint{2.079691in}{2.314968in}}{\pgfqpoint{2.085515in}{2.320792in}}%
\pgfpathcurveto{\pgfqpoint{2.091339in}{2.326616in}}{\pgfqpoint{2.094612in}{2.334516in}}{\pgfqpoint{2.094612in}{2.342752in}}%
\pgfpathcurveto{\pgfqpoint{2.094612in}{2.350989in}}{\pgfqpoint{2.091339in}{2.358889in}}{\pgfqpoint{2.085515in}{2.364713in}}%
\pgfpathcurveto{\pgfqpoint{2.079691in}{2.370537in}}{\pgfqpoint{2.071791in}{2.373809in}}{\pgfqpoint{2.063555in}{2.373809in}}%
\pgfpathcurveto{\pgfqpoint{2.055319in}{2.373809in}}{\pgfqpoint{2.047419in}{2.370537in}}{\pgfqpoint{2.041595in}{2.364713in}}%
\pgfpathcurveto{\pgfqpoint{2.035771in}{2.358889in}}{\pgfqpoint{2.032499in}{2.350989in}}{\pgfqpoint{2.032499in}{2.342752in}}%
\pgfpathcurveto{\pgfqpoint{2.032499in}{2.334516in}}{\pgfqpoint{2.035771in}{2.326616in}}{\pgfqpoint{2.041595in}{2.320792in}}%
\pgfpathcurveto{\pgfqpoint{2.047419in}{2.314968in}}{\pgfqpoint{2.055319in}{2.311696in}}{\pgfqpoint{2.063555in}{2.311696in}}%
\pgfpathclose%
\pgfusepath{stroke,fill}%
\end{pgfscope}%
\begin{pgfscope}%
\pgfpathrectangle{\pgfqpoint{0.100000in}{0.212622in}}{\pgfqpoint{3.696000in}{3.696000in}}%
\pgfusepath{clip}%
\pgfsetbuttcap%
\pgfsetroundjoin%
\definecolor{currentfill}{rgb}{0.121569,0.466667,0.705882}%
\pgfsetfillcolor{currentfill}%
\pgfsetfillopacity{0.543923}%
\pgfsetlinewidth{1.003750pt}%
\definecolor{currentstroke}{rgb}{0.121569,0.466667,0.705882}%
\pgfsetstrokecolor{currentstroke}%
\pgfsetstrokeopacity{0.543923}%
\pgfsetdash{}{0pt}%
\pgfpathmoveto{\pgfqpoint{1.124665in}{2.000884in}}%
\pgfpathcurveto{\pgfqpoint{1.132901in}{2.000884in}}{\pgfqpoint{1.140801in}{2.004156in}}{\pgfqpoint{1.146625in}{2.009980in}}%
\pgfpathcurveto{\pgfqpoint{1.152449in}{2.015804in}}{\pgfqpoint{1.155721in}{2.023704in}}{\pgfqpoint{1.155721in}{2.031940in}}%
\pgfpathcurveto{\pgfqpoint{1.155721in}{2.040177in}}{\pgfqpoint{1.152449in}{2.048077in}}{\pgfqpoint{1.146625in}{2.053901in}}%
\pgfpathcurveto{\pgfqpoint{1.140801in}{2.059725in}}{\pgfqpoint{1.132901in}{2.062997in}}{\pgfqpoint{1.124665in}{2.062997in}}%
\pgfpathcurveto{\pgfqpoint{1.116428in}{2.062997in}}{\pgfqpoint{1.108528in}{2.059725in}}{\pgfqpoint{1.102704in}{2.053901in}}%
\pgfpathcurveto{\pgfqpoint{1.096880in}{2.048077in}}{\pgfqpoint{1.093608in}{2.040177in}}{\pgfqpoint{1.093608in}{2.031940in}}%
\pgfpathcurveto{\pgfqpoint{1.093608in}{2.023704in}}{\pgfqpoint{1.096880in}{2.015804in}}{\pgfqpoint{1.102704in}{2.009980in}}%
\pgfpathcurveto{\pgfqpoint{1.108528in}{2.004156in}}{\pgfqpoint{1.116428in}{2.000884in}}{\pgfqpoint{1.124665in}{2.000884in}}%
\pgfpathclose%
\pgfusepath{stroke,fill}%
\end{pgfscope}%
\begin{pgfscope}%
\pgfpathrectangle{\pgfqpoint{0.100000in}{0.212622in}}{\pgfqpoint{3.696000in}{3.696000in}}%
\pgfusepath{clip}%
\pgfsetbuttcap%
\pgfsetroundjoin%
\definecolor{currentfill}{rgb}{0.121569,0.466667,0.705882}%
\pgfsetfillcolor{currentfill}%
\pgfsetfillopacity{0.545289}%
\pgfsetlinewidth{1.003750pt}%
\definecolor{currentstroke}{rgb}{0.121569,0.466667,0.705882}%
\pgfsetstrokecolor{currentstroke}%
\pgfsetstrokeopacity{0.545289}%
\pgfsetdash{}{0pt}%
\pgfpathmoveto{\pgfqpoint{1.119788in}{1.992829in}}%
\pgfpathcurveto{\pgfqpoint{1.128024in}{1.992829in}}{\pgfqpoint{1.135924in}{1.996102in}}{\pgfqpoint{1.141748in}{2.001926in}}%
\pgfpathcurveto{\pgfqpoint{1.147572in}{2.007750in}}{\pgfqpoint{1.150844in}{2.015650in}}{\pgfqpoint{1.150844in}{2.023886in}}%
\pgfpathcurveto{\pgfqpoint{1.150844in}{2.032122in}}{\pgfqpoint{1.147572in}{2.040022in}}{\pgfqpoint{1.141748in}{2.045846in}}%
\pgfpathcurveto{\pgfqpoint{1.135924in}{2.051670in}}{\pgfqpoint{1.128024in}{2.054942in}}{\pgfqpoint{1.119788in}{2.054942in}}%
\pgfpathcurveto{\pgfqpoint{1.111551in}{2.054942in}}{\pgfqpoint{1.103651in}{2.051670in}}{\pgfqpoint{1.097827in}{2.045846in}}%
\pgfpathcurveto{\pgfqpoint{1.092003in}{2.040022in}}{\pgfqpoint{1.088731in}{2.032122in}}{\pgfqpoint{1.088731in}{2.023886in}}%
\pgfpathcurveto{\pgfqpoint{1.088731in}{2.015650in}}{\pgfqpoint{1.092003in}{2.007750in}}{\pgfqpoint{1.097827in}{2.001926in}}%
\pgfpathcurveto{\pgfqpoint{1.103651in}{1.996102in}}{\pgfqpoint{1.111551in}{1.992829in}}{\pgfqpoint{1.119788in}{1.992829in}}%
\pgfpathclose%
\pgfusepath{stroke,fill}%
\end{pgfscope}%
\begin{pgfscope}%
\pgfpathrectangle{\pgfqpoint{0.100000in}{0.212622in}}{\pgfqpoint{3.696000in}{3.696000in}}%
\pgfusepath{clip}%
\pgfsetbuttcap%
\pgfsetroundjoin%
\definecolor{currentfill}{rgb}{0.121569,0.466667,0.705882}%
\pgfsetfillcolor{currentfill}%
\pgfsetfillopacity{0.546697}%
\pgfsetlinewidth{1.003750pt}%
\definecolor{currentstroke}{rgb}{0.121569,0.466667,0.705882}%
\pgfsetstrokecolor{currentstroke}%
\pgfsetstrokeopacity{0.546697}%
\pgfsetdash{}{0pt}%
\pgfpathmoveto{\pgfqpoint{1.116133in}{1.985025in}}%
\pgfpathcurveto{\pgfqpoint{1.124370in}{1.985025in}}{\pgfqpoint{1.132270in}{1.988297in}}{\pgfqpoint{1.138094in}{1.994121in}}%
\pgfpathcurveto{\pgfqpoint{1.143918in}{1.999945in}}{\pgfqpoint{1.147190in}{2.007845in}}{\pgfqpoint{1.147190in}{2.016081in}}%
\pgfpathcurveto{\pgfqpoint{1.147190in}{2.024317in}}{\pgfqpoint{1.143918in}{2.032217in}}{\pgfqpoint{1.138094in}{2.038041in}}%
\pgfpathcurveto{\pgfqpoint{1.132270in}{2.043865in}}{\pgfqpoint{1.124370in}{2.047138in}}{\pgfqpoint{1.116133in}{2.047138in}}%
\pgfpathcurveto{\pgfqpoint{1.107897in}{2.047138in}}{\pgfqpoint{1.099997in}{2.043865in}}{\pgfqpoint{1.094173in}{2.038041in}}%
\pgfpathcurveto{\pgfqpoint{1.088349in}{2.032217in}}{\pgfqpoint{1.085077in}{2.024317in}}{\pgfqpoint{1.085077in}{2.016081in}}%
\pgfpathcurveto{\pgfqpoint{1.085077in}{2.007845in}}{\pgfqpoint{1.088349in}{1.999945in}}{\pgfqpoint{1.094173in}{1.994121in}}%
\pgfpathcurveto{\pgfqpoint{1.099997in}{1.988297in}}{\pgfqpoint{1.107897in}{1.985025in}}{\pgfqpoint{1.116133in}{1.985025in}}%
\pgfpathclose%
\pgfusepath{stroke,fill}%
\end{pgfscope}%
\begin{pgfscope}%
\pgfpathrectangle{\pgfqpoint{0.100000in}{0.212622in}}{\pgfqpoint{3.696000in}{3.696000in}}%
\pgfusepath{clip}%
\pgfsetbuttcap%
\pgfsetroundjoin%
\definecolor{currentfill}{rgb}{0.121569,0.466667,0.705882}%
\pgfsetfillcolor{currentfill}%
\pgfsetfillopacity{0.547550}%
\pgfsetlinewidth{1.003750pt}%
\definecolor{currentstroke}{rgb}{0.121569,0.466667,0.705882}%
\pgfsetstrokecolor{currentstroke}%
\pgfsetstrokeopacity{0.547550}%
\pgfsetdash{}{0pt}%
\pgfpathmoveto{\pgfqpoint{1.113048in}{1.979849in}}%
\pgfpathcurveto{\pgfqpoint{1.121285in}{1.979849in}}{\pgfqpoint{1.129185in}{1.983122in}}{\pgfqpoint{1.135009in}{1.988946in}}%
\pgfpathcurveto{\pgfqpoint{1.140833in}{1.994769in}}{\pgfqpoint{1.144105in}{2.002670in}}{\pgfqpoint{1.144105in}{2.010906in}}%
\pgfpathcurveto{\pgfqpoint{1.144105in}{2.019142in}}{\pgfqpoint{1.140833in}{2.027042in}}{\pgfqpoint{1.135009in}{2.032866in}}%
\pgfpathcurveto{\pgfqpoint{1.129185in}{2.038690in}}{\pgfqpoint{1.121285in}{2.041962in}}{\pgfqpoint{1.113048in}{2.041962in}}%
\pgfpathcurveto{\pgfqpoint{1.104812in}{2.041962in}}{\pgfqpoint{1.096912in}{2.038690in}}{\pgfqpoint{1.091088in}{2.032866in}}%
\pgfpathcurveto{\pgfqpoint{1.085264in}{2.027042in}}{\pgfqpoint{1.081992in}{2.019142in}}{\pgfqpoint{1.081992in}{2.010906in}}%
\pgfpathcurveto{\pgfqpoint{1.081992in}{2.002670in}}{\pgfqpoint{1.085264in}{1.994769in}}{\pgfqpoint{1.091088in}{1.988946in}}%
\pgfpathcurveto{\pgfqpoint{1.096912in}{1.983122in}}{\pgfqpoint{1.104812in}{1.979849in}}{\pgfqpoint{1.113048in}{1.979849in}}%
\pgfpathclose%
\pgfusepath{stroke,fill}%
\end{pgfscope}%
\begin{pgfscope}%
\pgfpathrectangle{\pgfqpoint{0.100000in}{0.212622in}}{\pgfqpoint{3.696000in}{3.696000in}}%
\pgfusepath{clip}%
\pgfsetbuttcap%
\pgfsetroundjoin%
\definecolor{currentfill}{rgb}{0.121569,0.466667,0.705882}%
\pgfsetfillcolor{currentfill}%
\pgfsetfillopacity{0.547634}%
\pgfsetlinewidth{1.003750pt}%
\definecolor{currentstroke}{rgb}{0.121569,0.466667,0.705882}%
\pgfsetstrokecolor{currentstroke}%
\pgfsetstrokeopacity{0.547634}%
\pgfsetdash{}{0pt}%
\pgfpathmoveto{\pgfqpoint{2.066594in}{2.295172in}}%
\pgfpathcurveto{\pgfqpoint{2.074830in}{2.295172in}}{\pgfqpoint{2.082730in}{2.298445in}}{\pgfqpoint{2.088554in}{2.304269in}}%
\pgfpathcurveto{\pgfqpoint{2.094378in}{2.310093in}}{\pgfqpoint{2.097651in}{2.317993in}}{\pgfqpoint{2.097651in}{2.326229in}}%
\pgfpathcurveto{\pgfqpoint{2.097651in}{2.334465in}}{\pgfqpoint{2.094378in}{2.342365in}}{\pgfqpoint{2.088554in}{2.348189in}}%
\pgfpathcurveto{\pgfqpoint{2.082730in}{2.354013in}}{\pgfqpoint{2.074830in}{2.357285in}}{\pgfqpoint{2.066594in}{2.357285in}}%
\pgfpathcurveto{\pgfqpoint{2.058358in}{2.357285in}}{\pgfqpoint{2.050458in}{2.354013in}}{\pgfqpoint{2.044634in}{2.348189in}}%
\pgfpathcurveto{\pgfqpoint{2.038810in}{2.342365in}}{\pgfqpoint{2.035538in}{2.334465in}}{\pgfqpoint{2.035538in}{2.326229in}}%
\pgfpathcurveto{\pgfqpoint{2.035538in}{2.317993in}}{\pgfqpoint{2.038810in}{2.310093in}}{\pgfqpoint{2.044634in}{2.304269in}}%
\pgfpathcurveto{\pgfqpoint{2.050458in}{2.298445in}}{\pgfqpoint{2.058358in}{2.295172in}}{\pgfqpoint{2.066594in}{2.295172in}}%
\pgfpathclose%
\pgfusepath{stroke,fill}%
\end{pgfscope}%
\begin{pgfscope}%
\pgfpathrectangle{\pgfqpoint{0.100000in}{0.212622in}}{\pgfqpoint{3.696000in}{3.696000in}}%
\pgfusepath{clip}%
\pgfsetbuttcap%
\pgfsetroundjoin%
\definecolor{currentfill}{rgb}{0.121569,0.466667,0.705882}%
\pgfsetfillcolor{currentfill}%
\pgfsetfillopacity{0.548394}%
\pgfsetlinewidth{1.003750pt}%
\definecolor{currentstroke}{rgb}{0.121569,0.466667,0.705882}%
\pgfsetstrokecolor{currentstroke}%
\pgfsetstrokeopacity{0.548394}%
\pgfsetdash{}{0pt}%
\pgfpathmoveto{\pgfqpoint{1.110696in}{1.974611in}}%
\pgfpathcurveto{\pgfqpoint{1.118932in}{1.974611in}}{\pgfqpoint{1.126832in}{1.977884in}}{\pgfqpoint{1.132656in}{1.983708in}}%
\pgfpathcurveto{\pgfqpoint{1.138480in}{1.989531in}}{\pgfqpoint{1.141752in}{1.997432in}}{\pgfqpoint{1.141752in}{2.005668in}}%
\pgfpathcurveto{\pgfqpoint{1.141752in}{2.013904in}}{\pgfqpoint{1.138480in}{2.021804in}}{\pgfqpoint{1.132656in}{2.027628in}}%
\pgfpathcurveto{\pgfqpoint{1.126832in}{2.033452in}}{\pgfqpoint{1.118932in}{2.036724in}}{\pgfqpoint{1.110696in}{2.036724in}}%
\pgfpathcurveto{\pgfqpoint{1.102459in}{2.036724in}}{\pgfqpoint{1.094559in}{2.033452in}}{\pgfqpoint{1.088735in}{2.027628in}}%
\pgfpathcurveto{\pgfqpoint{1.082911in}{2.021804in}}{\pgfqpoint{1.079639in}{2.013904in}}{\pgfqpoint{1.079639in}{2.005668in}}%
\pgfpathcurveto{\pgfqpoint{1.079639in}{1.997432in}}{\pgfqpoint{1.082911in}{1.989531in}}{\pgfqpoint{1.088735in}{1.983708in}}%
\pgfpathcurveto{\pgfqpoint{1.094559in}{1.977884in}}{\pgfqpoint{1.102459in}{1.974611in}}{\pgfqpoint{1.110696in}{1.974611in}}%
\pgfpathclose%
\pgfusepath{stroke,fill}%
\end{pgfscope}%
\begin{pgfscope}%
\pgfpathrectangle{\pgfqpoint{0.100000in}{0.212622in}}{\pgfqpoint{3.696000in}{3.696000in}}%
\pgfusepath{clip}%
\pgfsetbuttcap%
\pgfsetroundjoin%
\definecolor{currentfill}{rgb}{0.121569,0.466667,0.705882}%
\pgfsetfillcolor{currentfill}%
\pgfsetfillopacity{0.548837}%
\pgfsetlinewidth{1.003750pt}%
\definecolor{currentstroke}{rgb}{0.121569,0.466667,0.705882}%
\pgfsetstrokecolor{currentstroke}%
\pgfsetstrokeopacity{0.548837}%
\pgfsetdash{}{0pt}%
\pgfpathmoveto{\pgfqpoint{1.109097in}{1.971765in}}%
\pgfpathcurveto{\pgfqpoint{1.117333in}{1.971765in}}{\pgfqpoint{1.125233in}{1.975037in}}{\pgfqpoint{1.131057in}{1.980861in}}%
\pgfpathcurveto{\pgfqpoint{1.136881in}{1.986685in}}{\pgfqpoint{1.140153in}{1.994585in}}{\pgfqpoint{1.140153in}{2.002822in}}%
\pgfpathcurveto{\pgfqpoint{1.140153in}{2.011058in}}{\pgfqpoint{1.136881in}{2.018958in}}{\pgfqpoint{1.131057in}{2.024782in}}%
\pgfpathcurveto{\pgfqpoint{1.125233in}{2.030606in}}{\pgfqpoint{1.117333in}{2.033878in}}{\pgfqpoint{1.109097in}{2.033878in}}%
\pgfpathcurveto{\pgfqpoint{1.100861in}{2.033878in}}{\pgfqpoint{1.092961in}{2.030606in}}{\pgfqpoint{1.087137in}{2.024782in}}%
\pgfpathcurveto{\pgfqpoint{1.081313in}{2.018958in}}{\pgfqpoint{1.078040in}{2.011058in}}{\pgfqpoint{1.078040in}{2.002822in}}%
\pgfpathcurveto{\pgfqpoint{1.078040in}{1.994585in}}{\pgfqpoint{1.081313in}{1.986685in}}{\pgfqpoint{1.087137in}{1.980861in}}%
\pgfpathcurveto{\pgfqpoint{1.092961in}{1.975037in}}{\pgfqpoint{1.100861in}{1.971765in}}{\pgfqpoint{1.109097in}{1.971765in}}%
\pgfpathclose%
\pgfusepath{stroke,fill}%
\end{pgfscope}%
\begin{pgfscope}%
\pgfpathrectangle{\pgfqpoint{0.100000in}{0.212622in}}{\pgfqpoint{3.696000in}{3.696000in}}%
\pgfusepath{clip}%
\pgfsetbuttcap%
\pgfsetroundjoin%
\definecolor{currentfill}{rgb}{0.121569,0.466667,0.705882}%
\pgfsetfillcolor{currentfill}%
\pgfsetfillopacity{0.549711}%
\pgfsetlinewidth{1.003750pt}%
\definecolor{currentstroke}{rgb}{0.121569,0.466667,0.705882}%
\pgfsetstrokecolor{currentstroke}%
\pgfsetstrokeopacity{0.549711}%
\pgfsetdash{}{0pt}%
\pgfpathmoveto{\pgfqpoint{1.106685in}{1.966223in}}%
\pgfpathcurveto{\pgfqpoint{1.114921in}{1.966223in}}{\pgfqpoint{1.122821in}{1.969495in}}{\pgfqpoint{1.128645in}{1.975319in}}%
\pgfpathcurveto{\pgfqpoint{1.134469in}{1.981143in}}{\pgfqpoint{1.137741in}{1.989043in}}{\pgfqpoint{1.137741in}{1.997279in}}%
\pgfpathcurveto{\pgfqpoint{1.137741in}{2.005516in}}{\pgfqpoint{1.134469in}{2.013416in}}{\pgfqpoint{1.128645in}{2.019240in}}%
\pgfpathcurveto{\pgfqpoint{1.122821in}{2.025064in}}{\pgfqpoint{1.114921in}{2.028336in}}{\pgfqpoint{1.106685in}{2.028336in}}%
\pgfpathcurveto{\pgfqpoint{1.098448in}{2.028336in}}{\pgfqpoint{1.090548in}{2.025064in}}{\pgfqpoint{1.084724in}{2.019240in}}%
\pgfpathcurveto{\pgfqpoint{1.078900in}{2.013416in}}{\pgfqpoint{1.075628in}{2.005516in}}{\pgfqpoint{1.075628in}{1.997279in}}%
\pgfpathcurveto{\pgfqpoint{1.075628in}{1.989043in}}{\pgfqpoint{1.078900in}{1.981143in}}{\pgfqpoint{1.084724in}{1.975319in}}%
\pgfpathcurveto{\pgfqpoint{1.090548in}{1.969495in}}{\pgfqpoint{1.098448in}{1.966223in}}{\pgfqpoint{1.106685in}{1.966223in}}%
\pgfpathclose%
\pgfusepath{stroke,fill}%
\end{pgfscope}%
\begin{pgfscope}%
\pgfpathrectangle{\pgfqpoint{0.100000in}{0.212622in}}{\pgfqpoint{3.696000in}{3.696000in}}%
\pgfusepath{clip}%
\pgfsetbuttcap%
\pgfsetroundjoin%
\definecolor{currentfill}{rgb}{0.121569,0.466667,0.705882}%
\pgfsetfillcolor{currentfill}%
\pgfsetfillopacity{0.550141}%
\pgfsetlinewidth{1.003750pt}%
\definecolor{currentstroke}{rgb}{0.121569,0.466667,0.705882}%
\pgfsetstrokecolor{currentstroke}%
\pgfsetstrokeopacity{0.550141}%
\pgfsetdash{}{0pt}%
\pgfpathmoveto{\pgfqpoint{1.105188in}{1.963590in}}%
\pgfpathcurveto{\pgfqpoint{1.113425in}{1.963590in}}{\pgfqpoint{1.121325in}{1.966863in}}{\pgfqpoint{1.127149in}{1.972687in}}%
\pgfpathcurveto{\pgfqpoint{1.132972in}{1.978511in}}{\pgfqpoint{1.136245in}{1.986411in}}{\pgfqpoint{1.136245in}{1.994647in}}%
\pgfpathcurveto{\pgfqpoint{1.136245in}{2.002883in}}{\pgfqpoint{1.132972in}{2.010783in}}{\pgfqpoint{1.127149in}{2.016607in}}%
\pgfpathcurveto{\pgfqpoint{1.121325in}{2.022431in}}{\pgfqpoint{1.113425in}{2.025703in}}{\pgfqpoint{1.105188in}{2.025703in}}%
\pgfpathcurveto{\pgfqpoint{1.096952in}{2.025703in}}{\pgfqpoint{1.089052in}{2.022431in}}{\pgfqpoint{1.083228in}{2.016607in}}%
\pgfpathcurveto{\pgfqpoint{1.077404in}{2.010783in}}{\pgfqpoint{1.074132in}{2.002883in}}{\pgfqpoint{1.074132in}{1.994647in}}%
\pgfpathcurveto{\pgfqpoint{1.074132in}{1.986411in}}{\pgfqpoint{1.077404in}{1.978511in}}{\pgfqpoint{1.083228in}{1.972687in}}%
\pgfpathcurveto{\pgfqpoint{1.089052in}{1.966863in}}{\pgfqpoint{1.096952in}{1.963590in}}{\pgfqpoint{1.105188in}{1.963590in}}%
\pgfpathclose%
\pgfusepath{stroke,fill}%
\end{pgfscope}%
\begin{pgfscope}%
\pgfpathrectangle{\pgfqpoint{0.100000in}{0.212622in}}{\pgfqpoint{3.696000in}{3.696000in}}%
\pgfusepath{clip}%
\pgfsetbuttcap%
\pgfsetroundjoin%
\definecolor{currentfill}{rgb}{0.121569,0.466667,0.705882}%
\pgfsetfillcolor{currentfill}%
\pgfsetfillopacity{0.550954}%
\pgfsetlinewidth{1.003750pt}%
\definecolor{currentstroke}{rgb}{0.121569,0.466667,0.705882}%
\pgfsetstrokecolor{currentstroke}%
\pgfsetstrokeopacity{0.550954}%
\pgfsetdash{}{0pt}%
\pgfpathmoveto{\pgfqpoint{1.102927in}{1.958353in}}%
\pgfpathcurveto{\pgfqpoint{1.111163in}{1.958353in}}{\pgfqpoint{1.119063in}{1.961625in}}{\pgfqpoint{1.124887in}{1.967449in}}%
\pgfpathcurveto{\pgfqpoint{1.130711in}{1.973273in}}{\pgfqpoint{1.133983in}{1.981173in}}{\pgfqpoint{1.133983in}{1.989409in}}%
\pgfpathcurveto{\pgfqpoint{1.133983in}{1.997646in}}{\pgfqpoint{1.130711in}{2.005546in}}{\pgfqpoint{1.124887in}{2.011370in}}%
\pgfpathcurveto{\pgfqpoint{1.119063in}{2.017194in}}{\pgfqpoint{1.111163in}{2.020466in}}{\pgfqpoint{1.102927in}{2.020466in}}%
\pgfpathcurveto{\pgfqpoint{1.094691in}{2.020466in}}{\pgfqpoint{1.086790in}{2.017194in}}{\pgfqpoint{1.080967in}{2.011370in}}%
\pgfpathcurveto{\pgfqpoint{1.075143in}{2.005546in}}{\pgfqpoint{1.071870in}{1.997646in}}{\pgfqpoint{1.071870in}{1.989409in}}%
\pgfpathcurveto{\pgfqpoint{1.071870in}{1.981173in}}{\pgfqpoint{1.075143in}{1.973273in}}{\pgfqpoint{1.080967in}{1.967449in}}%
\pgfpathcurveto{\pgfqpoint{1.086790in}{1.961625in}}{\pgfqpoint{1.094691in}{1.958353in}}{\pgfqpoint{1.102927in}{1.958353in}}%
\pgfpathclose%
\pgfusepath{stroke,fill}%
\end{pgfscope}%
\begin{pgfscope}%
\pgfpathrectangle{\pgfqpoint{0.100000in}{0.212622in}}{\pgfqpoint{3.696000in}{3.696000in}}%
\pgfusepath{clip}%
\pgfsetbuttcap%
\pgfsetroundjoin%
\definecolor{currentfill}{rgb}{0.121569,0.466667,0.705882}%
\pgfsetfillcolor{currentfill}%
\pgfsetfillopacity{0.551187}%
\pgfsetlinewidth{1.003750pt}%
\definecolor{currentstroke}{rgb}{0.121569,0.466667,0.705882}%
\pgfsetstrokecolor{currentstroke}%
\pgfsetstrokeopacity{0.551187}%
\pgfsetdash{}{0pt}%
\pgfpathmoveto{\pgfqpoint{1.102055in}{1.956691in}}%
\pgfpathcurveto{\pgfqpoint{1.110291in}{1.956691in}}{\pgfqpoint{1.118191in}{1.959963in}}{\pgfqpoint{1.124015in}{1.965787in}}%
\pgfpathcurveto{\pgfqpoint{1.129839in}{1.971611in}}{\pgfqpoint{1.133111in}{1.979511in}}{\pgfqpoint{1.133111in}{1.987748in}}%
\pgfpathcurveto{\pgfqpoint{1.133111in}{1.995984in}}{\pgfqpoint{1.129839in}{2.003884in}}{\pgfqpoint{1.124015in}{2.009708in}}%
\pgfpathcurveto{\pgfqpoint{1.118191in}{2.015532in}}{\pgfqpoint{1.110291in}{2.018804in}}{\pgfqpoint{1.102055in}{2.018804in}}%
\pgfpathcurveto{\pgfqpoint{1.093818in}{2.018804in}}{\pgfqpoint{1.085918in}{2.015532in}}{\pgfqpoint{1.080094in}{2.009708in}}%
\pgfpathcurveto{\pgfqpoint{1.074270in}{2.003884in}}{\pgfqpoint{1.070998in}{1.995984in}}{\pgfqpoint{1.070998in}{1.987748in}}%
\pgfpathcurveto{\pgfqpoint{1.070998in}{1.979511in}}{\pgfqpoint{1.074270in}{1.971611in}}{\pgfqpoint{1.080094in}{1.965787in}}%
\pgfpathcurveto{\pgfqpoint{1.085918in}{1.959963in}}{\pgfqpoint{1.093818in}{1.956691in}}{\pgfqpoint{1.102055in}{1.956691in}}%
\pgfpathclose%
\pgfusepath{stroke,fill}%
\end{pgfscope}%
\begin{pgfscope}%
\pgfpathrectangle{\pgfqpoint{0.100000in}{0.212622in}}{\pgfqpoint{3.696000in}{3.696000in}}%
\pgfusepath{clip}%
\pgfsetbuttcap%
\pgfsetroundjoin%
\definecolor{currentfill}{rgb}{0.121569,0.466667,0.705882}%
\pgfsetfillcolor{currentfill}%
\pgfsetfillopacity{0.551305}%
\pgfsetlinewidth{1.003750pt}%
\definecolor{currentstroke}{rgb}{0.121569,0.466667,0.705882}%
\pgfsetstrokecolor{currentstroke}%
\pgfsetstrokeopacity{0.551305}%
\pgfsetdash{}{0pt}%
\pgfpathmoveto{\pgfqpoint{1.101664in}{1.955727in}}%
\pgfpathcurveto{\pgfqpoint{1.109900in}{1.955727in}}{\pgfqpoint{1.117800in}{1.959000in}}{\pgfqpoint{1.123624in}{1.964824in}}%
\pgfpathcurveto{\pgfqpoint{1.129448in}{1.970648in}}{\pgfqpoint{1.132720in}{1.978548in}}{\pgfqpoint{1.132720in}{1.986784in}}%
\pgfpathcurveto{\pgfqpoint{1.132720in}{1.995020in}}{\pgfqpoint{1.129448in}{2.002920in}}{\pgfqpoint{1.123624in}{2.008744in}}%
\pgfpathcurveto{\pgfqpoint{1.117800in}{2.014568in}}{\pgfqpoint{1.109900in}{2.017840in}}{\pgfqpoint{1.101664in}{2.017840in}}%
\pgfpathcurveto{\pgfqpoint{1.093428in}{2.017840in}}{\pgfqpoint{1.085528in}{2.014568in}}{\pgfqpoint{1.079704in}{2.008744in}}%
\pgfpathcurveto{\pgfqpoint{1.073880in}{2.002920in}}{\pgfqpoint{1.070607in}{1.995020in}}{\pgfqpoint{1.070607in}{1.986784in}}%
\pgfpathcurveto{\pgfqpoint{1.070607in}{1.978548in}}{\pgfqpoint{1.073880in}{1.970648in}}{\pgfqpoint{1.079704in}{1.964824in}}%
\pgfpathcurveto{\pgfqpoint{1.085528in}{1.959000in}}{\pgfqpoint{1.093428in}{1.955727in}}{\pgfqpoint{1.101664in}{1.955727in}}%
\pgfpathclose%
\pgfusepath{stroke,fill}%
\end{pgfscope}%
\begin{pgfscope}%
\pgfpathrectangle{\pgfqpoint{0.100000in}{0.212622in}}{\pgfqpoint{3.696000in}{3.696000in}}%
\pgfusepath{clip}%
\pgfsetbuttcap%
\pgfsetroundjoin%
\definecolor{currentfill}{rgb}{0.121569,0.466667,0.705882}%
\pgfsetfillcolor{currentfill}%
\pgfsetfillopacity{0.551545}%
\pgfsetlinewidth{1.003750pt}%
\definecolor{currentstroke}{rgb}{0.121569,0.466667,0.705882}%
\pgfsetstrokecolor{currentstroke}%
\pgfsetstrokeopacity{0.551545}%
\pgfsetdash{}{0pt}%
\pgfpathmoveto{\pgfqpoint{1.100851in}{1.954191in}}%
\pgfpathcurveto{\pgfqpoint{1.109088in}{1.954191in}}{\pgfqpoint{1.116988in}{1.957463in}}{\pgfqpoint{1.122812in}{1.963287in}}%
\pgfpathcurveto{\pgfqpoint{1.128636in}{1.969111in}}{\pgfqpoint{1.131908in}{1.977011in}}{\pgfqpoint{1.131908in}{1.985247in}}%
\pgfpathcurveto{\pgfqpoint{1.131908in}{1.993483in}}{\pgfqpoint{1.128636in}{2.001383in}}{\pgfqpoint{1.122812in}{2.007207in}}%
\pgfpathcurveto{\pgfqpoint{1.116988in}{2.013031in}}{\pgfqpoint{1.109088in}{2.016304in}}{\pgfqpoint{1.100851in}{2.016304in}}%
\pgfpathcurveto{\pgfqpoint{1.092615in}{2.016304in}}{\pgfqpoint{1.084715in}{2.013031in}}{\pgfqpoint{1.078891in}{2.007207in}}%
\pgfpathcurveto{\pgfqpoint{1.073067in}{2.001383in}}{\pgfqpoint{1.069795in}{1.993483in}}{\pgfqpoint{1.069795in}{1.985247in}}%
\pgfpathcurveto{\pgfqpoint{1.069795in}{1.977011in}}{\pgfqpoint{1.073067in}{1.969111in}}{\pgfqpoint{1.078891in}{1.963287in}}%
\pgfpathcurveto{\pgfqpoint{1.084715in}{1.957463in}}{\pgfqpoint{1.092615in}{1.954191in}}{\pgfqpoint{1.100851in}{1.954191in}}%
\pgfpathclose%
\pgfusepath{stroke,fill}%
\end{pgfscope}%
\begin{pgfscope}%
\pgfpathrectangle{\pgfqpoint{0.100000in}{0.212622in}}{\pgfqpoint{3.696000in}{3.696000in}}%
\pgfusepath{clip}%
\pgfsetbuttcap%
\pgfsetroundjoin%
\definecolor{currentfill}{rgb}{0.121569,0.466667,0.705882}%
\pgfsetfillcolor{currentfill}%
\pgfsetfillopacity{0.551731}%
\pgfsetlinewidth{1.003750pt}%
\definecolor{currentstroke}{rgb}{0.121569,0.466667,0.705882}%
\pgfsetstrokecolor{currentstroke}%
\pgfsetstrokeopacity{0.551731}%
\pgfsetdash{}{0pt}%
\pgfpathmoveto{\pgfqpoint{1.100463in}{1.952896in}}%
\pgfpathcurveto{\pgfqpoint{1.108700in}{1.952896in}}{\pgfqpoint{1.116600in}{1.956168in}}{\pgfqpoint{1.122424in}{1.961992in}}%
\pgfpathcurveto{\pgfqpoint{1.128248in}{1.967816in}}{\pgfqpoint{1.131520in}{1.975716in}}{\pgfqpoint{1.131520in}{1.983952in}}%
\pgfpathcurveto{\pgfqpoint{1.131520in}{1.992188in}}{\pgfqpoint{1.128248in}{2.000088in}}{\pgfqpoint{1.122424in}{2.005912in}}%
\pgfpathcurveto{\pgfqpoint{1.116600in}{2.011736in}}{\pgfqpoint{1.108700in}{2.015009in}}{\pgfqpoint{1.100463in}{2.015009in}}%
\pgfpathcurveto{\pgfqpoint{1.092227in}{2.015009in}}{\pgfqpoint{1.084327in}{2.011736in}}{\pgfqpoint{1.078503in}{2.005912in}}%
\pgfpathcurveto{\pgfqpoint{1.072679in}{2.000088in}}{\pgfqpoint{1.069407in}{1.992188in}}{\pgfqpoint{1.069407in}{1.983952in}}%
\pgfpathcurveto{\pgfqpoint{1.069407in}{1.975716in}}{\pgfqpoint{1.072679in}{1.967816in}}{\pgfqpoint{1.078503in}{1.961992in}}%
\pgfpathcurveto{\pgfqpoint{1.084327in}{1.956168in}}{\pgfqpoint{1.092227in}{1.952896in}}{\pgfqpoint{1.100463in}{1.952896in}}%
\pgfpathclose%
\pgfusepath{stroke,fill}%
\end{pgfscope}%
\begin{pgfscope}%
\pgfpathrectangle{\pgfqpoint{0.100000in}{0.212622in}}{\pgfqpoint{3.696000in}{3.696000in}}%
\pgfusepath{clip}%
\pgfsetbuttcap%
\pgfsetroundjoin%
\definecolor{currentfill}{rgb}{0.121569,0.466667,0.705882}%
\pgfsetfillcolor{currentfill}%
\pgfsetfillopacity{0.552125}%
\pgfsetlinewidth{1.003750pt}%
\definecolor{currentstroke}{rgb}{0.121569,0.466667,0.705882}%
\pgfsetstrokecolor{currentstroke}%
\pgfsetstrokeopacity{0.552125}%
\pgfsetdash{}{0pt}%
\pgfpathmoveto{\pgfqpoint{1.099368in}{1.951091in}}%
\pgfpathcurveto{\pgfqpoint{1.107604in}{1.951091in}}{\pgfqpoint{1.115505in}{1.954364in}}{\pgfqpoint{1.121328in}{1.960188in}}%
\pgfpathcurveto{\pgfqpoint{1.127152in}{1.966011in}}{\pgfqpoint{1.130425in}{1.973912in}}{\pgfqpoint{1.130425in}{1.982148in}}%
\pgfpathcurveto{\pgfqpoint{1.130425in}{1.990384in}}{\pgfqpoint{1.127152in}{1.998284in}}{\pgfqpoint{1.121328in}{2.004108in}}%
\pgfpathcurveto{\pgfqpoint{1.115505in}{2.009932in}}{\pgfqpoint{1.107604in}{2.013204in}}{\pgfqpoint{1.099368in}{2.013204in}}%
\pgfpathcurveto{\pgfqpoint{1.091132in}{2.013204in}}{\pgfqpoint{1.083232in}{2.009932in}}{\pgfqpoint{1.077408in}{2.004108in}}%
\pgfpathcurveto{\pgfqpoint{1.071584in}{1.998284in}}{\pgfqpoint{1.068312in}{1.990384in}}{\pgfqpoint{1.068312in}{1.982148in}}%
\pgfpathcurveto{\pgfqpoint{1.068312in}{1.973912in}}{\pgfqpoint{1.071584in}{1.966011in}}{\pgfqpoint{1.077408in}{1.960188in}}%
\pgfpathcurveto{\pgfqpoint{1.083232in}{1.954364in}}{\pgfqpoint{1.091132in}{1.951091in}}{\pgfqpoint{1.099368in}{1.951091in}}%
\pgfpathclose%
\pgfusepath{stroke,fill}%
\end{pgfscope}%
\begin{pgfscope}%
\pgfpathrectangle{\pgfqpoint{0.100000in}{0.212622in}}{\pgfqpoint{3.696000in}{3.696000in}}%
\pgfusepath{clip}%
\pgfsetbuttcap%
\pgfsetroundjoin%
\definecolor{currentfill}{rgb}{0.121569,0.466667,0.705882}%
\pgfsetfillcolor{currentfill}%
\pgfsetfillopacity{0.552565}%
\pgfsetlinewidth{1.003750pt}%
\definecolor{currentstroke}{rgb}{0.121569,0.466667,0.705882}%
\pgfsetstrokecolor{currentstroke}%
\pgfsetstrokeopacity{0.552565}%
\pgfsetdash{}{0pt}%
\pgfpathmoveto{\pgfqpoint{2.068941in}{2.279255in}}%
\pgfpathcurveto{\pgfqpoint{2.077177in}{2.279255in}}{\pgfqpoint{2.085077in}{2.282527in}}{\pgfqpoint{2.090901in}{2.288351in}}%
\pgfpathcurveto{\pgfqpoint{2.096725in}{2.294175in}}{\pgfqpoint{2.099997in}{2.302075in}}{\pgfqpoint{2.099997in}{2.310311in}}%
\pgfpathcurveto{\pgfqpoint{2.099997in}{2.318548in}}{\pgfqpoint{2.096725in}{2.326448in}}{\pgfqpoint{2.090901in}{2.332272in}}%
\pgfpathcurveto{\pgfqpoint{2.085077in}{2.338096in}}{\pgfqpoint{2.077177in}{2.341368in}}{\pgfqpoint{2.068941in}{2.341368in}}%
\pgfpathcurveto{\pgfqpoint{2.060705in}{2.341368in}}{\pgfqpoint{2.052804in}{2.338096in}}{\pgfqpoint{2.046981in}{2.332272in}}%
\pgfpathcurveto{\pgfqpoint{2.041157in}{2.326448in}}{\pgfqpoint{2.037884in}{2.318548in}}{\pgfqpoint{2.037884in}{2.310311in}}%
\pgfpathcurveto{\pgfqpoint{2.037884in}{2.302075in}}{\pgfqpoint{2.041157in}{2.294175in}}{\pgfqpoint{2.046981in}{2.288351in}}%
\pgfpathcurveto{\pgfqpoint{2.052804in}{2.282527in}}{\pgfqpoint{2.060705in}{2.279255in}}{\pgfqpoint{2.068941in}{2.279255in}}%
\pgfpathclose%
\pgfusepath{stroke,fill}%
\end{pgfscope}%
\begin{pgfscope}%
\pgfpathrectangle{\pgfqpoint{0.100000in}{0.212622in}}{\pgfqpoint{3.696000in}{3.696000in}}%
\pgfusepath{clip}%
\pgfsetbuttcap%
\pgfsetroundjoin%
\definecolor{currentfill}{rgb}{0.121569,0.466667,0.705882}%
\pgfsetfillcolor{currentfill}%
\pgfsetfillopacity{0.552860}%
\pgfsetlinewidth{1.003750pt}%
\definecolor{currentstroke}{rgb}{0.121569,0.466667,0.705882}%
\pgfsetstrokecolor{currentstroke}%
\pgfsetstrokeopacity{0.552860}%
\pgfsetdash{}{0pt}%
\pgfpathmoveto{\pgfqpoint{1.097531in}{1.947691in}}%
\pgfpathcurveto{\pgfqpoint{1.105767in}{1.947691in}}{\pgfqpoint{1.113667in}{1.950963in}}{\pgfqpoint{1.119491in}{1.956787in}}%
\pgfpathcurveto{\pgfqpoint{1.125315in}{1.962611in}}{\pgfqpoint{1.128587in}{1.970511in}}{\pgfqpoint{1.128587in}{1.978747in}}%
\pgfpathcurveto{\pgfqpoint{1.128587in}{1.986983in}}{\pgfqpoint{1.125315in}{1.994883in}}{\pgfqpoint{1.119491in}{2.000707in}}%
\pgfpathcurveto{\pgfqpoint{1.113667in}{2.006531in}}{\pgfqpoint{1.105767in}{2.009804in}}{\pgfqpoint{1.097531in}{2.009804in}}%
\pgfpathcurveto{\pgfqpoint{1.089295in}{2.009804in}}{\pgfqpoint{1.081395in}{2.006531in}}{\pgfqpoint{1.075571in}{2.000707in}}%
\pgfpathcurveto{\pgfqpoint{1.069747in}{1.994883in}}{\pgfqpoint{1.066474in}{1.986983in}}{\pgfqpoint{1.066474in}{1.978747in}}%
\pgfpathcurveto{\pgfqpoint{1.066474in}{1.970511in}}{\pgfqpoint{1.069747in}{1.962611in}}{\pgfqpoint{1.075571in}{1.956787in}}%
\pgfpathcurveto{\pgfqpoint{1.081395in}{1.950963in}}{\pgfqpoint{1.089295in}{1.947691in}}{\pgfqpoint{1.097531in}{1.947691in}}%
\pgfpathclose%
\pgfusepath{stroke,fill}%
\end{pgfscope}%
\begin{pgfscope}%
\pgfpathrectangle{\pgfqpoint{0.100000in}{0.212622in}}{\pgfqpoint{3.696000in}{3.696000in}}%
\pgfusepath{clip}%
\pgfsetbuttcap%
\pgfsetroundjoin%
\definecolor{currentfill}{rgb}{0.121569,0.466667,0.705882}%
\pgfsetfillcolor{currentfill}%
\pgfsetfillopacity{0.553464}%
\pgfsetlinewidth{1.003750pt}%
\definecolor{currentstroke}{rgb}{0.121569,0.466667,0.705882}%
\pgfsetstrokecolor{currentstroke}%
\pgfsetstrokeopacity{0.553464}%
\pgfsetdash{}{0pt}%
\pgfpathmoveto{\pgfqpoint{1.095924in}{1.944861in}}%
\pgfpathcurveto{\pgfqpoint{1.104160in}{1.944861in}}{\pgfqpoint{1.112060in}{1.948133in}}{\pgfqpoint{1.117884in}{1.953957in}}%
\pgfpathcurveto{\pgfqpoint{1.123708in}{1.959781in}}{\pgfqpoint{1.126981in}{1.967681in}}{\pgfqpoint{1.126981in}{1.975917in}}%
\pgfpathcurveto{\pgfqpoint{1.126981in}{1.984153in}}{\pgfqpoint{1.123708in}{1.992054in}}{\pgfqpoint{1.117884in}{1.997877in}}%
\pgfpathcurveto{\pgfqpoint{1.112060in}{2.003701in}}{\pgfqpoint{1.104160in}{2.006974in}}{\pgfqpoint{1.095924in}{2.006974in}}%
\pgfpathcurveto{\pgfqpoint{1.087688in}{2.006974in}}{\pgfqpoint{1.079788in}{2.003701in}}{\pgfqpoint{1.073964in}{1.997877in}}%
\pgfpathcurveto{\pgfqpoint{1.068140in}{1.992054in}}{\pgfqpoint{1.064868in}{1.984153in}}{\pgfqpoint{1.064868in}{1.975917in}}%
\pgfpathcurveto{\pgfqpoint{1.064868in}{1.967681in}}{\pgfqpoint{1.068140in}{1.959781in}}{\pgfqpoint{1.073964in}{1.953957in}}%
\pgfpathcurveto{\pgfqpoint{1.079788in}{1.948133in}}{\pgfqpoint{1.087688in}{1.944861in}}{\pgfqpoint{1.095924in}{1.944861in}}%
\pgfpathclose%
\pgfusepath{stroke,fill}%
\end{pgfscope}%
\begin{pgfscope}%
\pgfpathrectangle{\pgfqpoint{0.100000in}{0.212622in}}{\pgfqpoint{3.696000in}{3.696000in}}%
\pgfusepath{clip}%
\pgfsetbuttcap%
\pgfsetroundjoin%
\definecolor{currentfill}{rgb}{0.121569,0.466667,0.705882}%
\pgfsetfillcolor{currentfill}%
\pgfsetfillopacity{0.554584}%
\pgfsetlinewidth{1.003750pt}%
\definecolor{currentstroke}{rgb}{0.121569,0.466667,0.705882}%
\pgfsetstrokecolor{currentstroke}%
\pgfsetstrokeopacity{0.554584}%
\pgfsetdash{}{0pt}%
\pgfpathmoveto{\pgfqpoint{1.093040in}{1.939745in}}%
\pgfpathcurveto{\pgfqpoint{1.101276in}{1.939745in}}{\pgfqpoint{1.109176in}{1.943017in}}{\pgfqpoint{1.115000in}{1.948841in}}%
\pgfpathcurveto{\pgfqpoint{1.120824in}{1.954665in}}{\pgfqpoint{1.124097in}{1.962565in}}{\pgfqpoint{1.124097in}{1.970801in}}%
\pgfpathcurveto{\pgfqpoint{1.124097in}{1.979037in}}{\pgfqpoint{1.120824in}{1.986937in}}{\pgfqpoint{1.115000in}{1.992761in}}%
\pgfpathcurveto{\pgfqpoint{1.109176in}{1.998585in}}{\pgfqpoint{1.101276in}{2.001857in}}{\pgfqpoint{1.093040in}{2.001857in}}%
\pgfpathcurveto{\pgfqpoint{1.084804in}{2.001857in}}{\pgfqpoint{1.076904in}{1.998585in}}{\pgfqpoint{1.071080in}{1.992761in}}%
\pgfpathcurveto{\pgfqpoint{1.065256in}{1.986937in}}{\pgfqpoint{1.061984in}{1.979037in}}{\pgfqpoint{1.061984in}{1.970801in}}%
\pgfpathcurveto{\pgfqpoint{1.061984in}{1.962565in}}{\pgfqpoint{1.065256in}{1.954665in}}{\pgfqpoint{1.071080in}{1.948841in}}%
\pgfpathcurveto{\pgfqpoint{1.076904in}{1.943017in}}{\pgfqpoint{1.084804in}{1.939745in}}{\pgfqpoint{1.093040in}{1.939745in}}%
\pgfpathclose%
\pgfusepath{stroke,fill}%
\end{pgfscope}%
\begin{pgfscope}%
\pgfpathrectangle{\pgfqpoint{0.100000in}{0.212622in}}{\pgfqpoint{3.696000in}{3.696000in}}%
\pgfusepath{clip}%
\pgfsetbuttcap%
\pgfsetroundjoin%
\definecolor{currentfill}{rgb}{0.121569,0.466667,0.705882}%
\pgfsetfillcolor{currentfill}%
\pgfsetfillopacity{0.555545}%
\pgfsetlinewidth{1.003750pt}%
\definecolor{currentstroke}{rgb}{0.121569,0.466667,0.705882}%
\pgfsetstrokecolor{currentstroke}%
\pgfsetstrokeopacity{0.555545}%
\pgfsetdash{}{0pt}%
\pgfpathmoveto{\pgfqpoint{1.090521in}{1.934920in}}%
\pgfpathcurveto{\pgfqpoint{1.098758in}{1.934920in}}{\pgfqpoint{1.106658in}{1.938192in}}{\pgfqpoint{1.112482in}{1.944016in}}%
\pgfpathcurveto{\pgfqpoint{1.118306in}{1.949840in}}{\pgfqpoint{1.121578in}{1.957740in}}{\pgfqpoint{1.121578in}{1.965977in}}%
\pgfpathcurveto{\pgfqpoint{1.121578in}{1.974213in}}{\pgfqpoint{1.118306in}{1.982113in}}{\pgfqpoint{1.112482in}{1.987937in}}%
\pgfpathcurveto{\pgfqpoint{1.106658in}{1.993761in}}{\pgfqpoint{1.098758in}{1.997033in}}{\pgfqpoint{1.090521in}{1.997033in}}%
\pgfpathcurveto{\pgfqpoint{1.082285in}{1.997033in}}{\pgfqpoint{1.074385in}{1.993761in}}{\pgfqpoint{1.068561in}{1.987937in}}%
\pgfpathcurveto{\pgfqpoint{1.062737in}{1.982113in}}{\pgfqpoint{1.059465in}{1.974213in}}{\pgfqpoint{1.059465in}{1.965977in}}%
\pgfpathcurveto{\pgfqpoint{1.059465in}{1.957740in}}{\pgfqpoint{1.062737in}{1.949840in}}{\pgfqpoint{1.068561in}{1.944016in}}%
\pgfpathcurveto{\pgfqpoint{1.074385in}{1.938192in}}{\pgfqpoint{1.082285in}{1.934920in}}{\pgfqpoint{1.090521in}{1.934920in}}%
\pgfpathclose%
\pgfusepath{stroke,fill}%
\end{pgfscope}%
\begin{pgfscope}%
\pgfpathrectangle{\pgfqpoint{0.100000in}{0.212622in}}{\pgfqpoint{3.696000in}{3.696000in}}%
\pgfusepath{clip}%
\pgfsetbuttcap%
\pgfsetroundjoin%
\definecolor{currentfill}{rgb}{0.121569,0.466667,0.705882}%
\pgfsetfillcolor{currentfill}%
\pgfsetfillopacity{0.557240}%
\pgfsetlinewidth{1.003750pt}%
\definecolor{currentstroke}{rgb}{0.121569,0.466667,0.705882}%
\pgfsetstrokecolor{currentstroke}%
\pgfsetstrokeopacity{0.557240}%
\pgfsetdash{}{0pt}%
\pgfpathmoveto{\pgfqpoint{1.085465in}{1.926558in}}%
\pgfpathcurveto{\pgfqpoint{1.093701in}{1.926558in}}{\pgfqpoint{1.101601in}{1.929831in}}{\pgfqpoint{1.107425in}{1.935655in}}%
\pgfpathcurveto{\pgfqpoint{1.113249in}{1.941479in}}{\pgfqpoint{1.116521in}{1.949379in}}{\pgfqpoint{1.116521in}{1.957615in}}%
\pgfpathcurveto{\pgfqpoint{1.116521in}{1.965851in}}{\pgfqpoint{1.113249in}{1.973751in}}{\pgfqpoint{1.107425in}{1.979575in}}%
\pgfpathcurveto{\pgfqpoint{1.101601in}{1.985399in}}{\pgfqpoint{1.093701in}{1.988671in}}{\pgfqpoint{1.085465in}{1.988671in}}%
\pgfpathcurveto{\pgfqpoint{1.077229in}{1.988671in}}{\pgfqpoint{1.069329in}{1.985399in}}{\pgfqpoint{1.063505in}{1.979575in}}%
\pgfpathcurveto{\pgfqpoint{1.057681in}{1.973751in}}{\pgfqpoint{1.054408in}{1.965851in}}{\pgfqpoint{1.054408in}{1.957615in}}%
\pgfpathcurveto{\pgfqpoint{1.054408in}{1.949379in}}{\pgfqpoint{1.057681in}{1.941479in}}{\pgfqpoint{1.063505in}{1.935655in}}%
\pgfpathcurveto{\pgfqpoint{1.069329in}{1.929831in}}{\pgfqpoint{1.077229in}{1.926558in}}{\pgfqpoint{1.085465in}{1.926558in}}%
\pgfpathclose%
\pgfusepath{stroke,fill}%
\end{pgfscope}%
\begin{pgfscope}%
\pgfpathrectangle{\pgfqpoint{0.100000in}{0.212622in}}{\pgfqpoint{3.696000in}{3.696000in}}%
\pgfusepath{clip}%
\pgfsetbuttcap%
\pgfsetroundjoin%
\definecolor{currentfill}{rgb}{0.121569,0.466667,0.705882}%
\pgfsetfillcolor{currentfill}%
\pgfsetfillopacity{0.557442}%
\pgfsetlinewidth{1.003750pt}%
\definecolor{currentstroke}{rgb}{0.121569,0.466667,0.705882}%
\pgfsetstrokecolor{currentstroke}%
\pgfsetstrokeopacity{0.557442}%
\pgfsetdash{}{0pt}%
\pgfpathmoveto{\pgfqpoint{2.072334in}{2.262802in}}%
\pgfpathcurveto{\pgfqpoint{2.080570in}{2.262802in}}{\pgfqpoint{2.088470in}{2.266075in}}{\pgfqpoint{2.094294in}{2.271899in}}%
\pgfpathcurveto{\pgfqpoint{2.100118in}{2.277722in}}{\pgfqpoint{2.103390in}{2.285623in}}{\pgfqpoint{2.103390in}{2.293859in}}%
\pgfpathcurveto{\pgfqpoint{2.103390in}{2.302095in}}{\pgfqpoint{2.100118in}{2.309995in}}{\pgfqpoint{2.094294in}{2.315819in}}%
\pgfpathcurveto{\pgfqpoint{2.088470in}{2.321643in}}{\pgfqpoint{2.080570in}{2.324915in}}{\pgfqpoint{2.072334in}{2.324915in}}%
\pgfpathcurveto{\pgfqpoint{2.064098in}{2.324915in}}{\pgfqpoint{2.056198in}{2.321643in}}{\pgfqpoint{2.050374in}{2.315819in}}%
\pgfpathcurveto{\pgfqpoint{2.044550in}{2.309995in}}{\pgfqpoint{2.041277in}{2.302095in}}{\pgfqpoint{2.041277in}{2.293859in}}%
\pgfpathcurveto{\pgfqpoint{2.041277in}{2.285623in}}{\pgfqpoint{2.044550in}{2.277722in}}{\pgfqpoint{2.050374in}{2.271899in}}%
\pgfpathcurveto{\pgfqpoint{2.056198in}{2.266075in}}{\pgfqpoint{2.064098in}{2.262802in}}{\pgfqpoint{2.072334in}{2.262802in}}%
\pgfpathclose%
\pgfusepath{stroke,fill}%
\end{pgfscope}%
\begin{pgfscope}%
\pgfpathrectangle{\pgfqpoint{0.100000in}{0.212622in}}{\pgfqpoint{3.696000in}{3.696000in}}%
\pgfusepath{clip}%
\pgfsetbuttcap%
\pgfsetroundjoin%
\definecolor{currentfill}{rgb}{0.121569,0.466667,0.705882}%
\pgfsetfillcolor{currentfill}%
\pgfsetfillopacity{0.558826}%
\pgfsetlinewidth{1.003750pt}%
\definecolor{currentstroke}{rgb}{0.121569,0.466667,0.705882}%
\pgfsetstrokecolor{currentstroke}%
\pgfsetstrokeopacity{0.558826}%
\pgfsetdash{}{0pt}%
\pgfpathmoveto{\pgfqpoint{1.081035in}{1.918458in}}%
\pgfpathcurveto{\pgfqpoint{1.089271in}{1.918458in}}{\pgfqpoint{1.097171in}{1.921730in}}{\pgfqpoint{1.102995in}{1.927554in}}%
\pgfpathcurveto{\pgfqpoint{1.108819in}{1.933378in}}{\pgfqpoint{1.112092in}{1.941278in}}{\pgfqpoint{1.112092in}{1.949514in}}%
\pgfpathcurveto{\pgfqpoint{1.112092in}{1.957750in}}{\pgfqpoint{1.108819in}{1.965650in}}{\pgfqpoint{1.102995in}{1.971474in}}%
\pgfpathcurveto{\pgfqpoint{1.097171in}{1.977298in}}{\pgfqpoint{1.089271in}{1.980571in}}{\pgfqpoint{1.081035in}{1.980571in}}%
\pgfpathcurveto{\pgfqpoint{1.072799in}{1.980571in}}{\pgfqpoint{1.064899in}{1.977298in}}{\pgfqpoint{1.059075in}{1.971474in}}%
\pgfpathcurveto{\pgfqpoint{1.053251in}{1.965650in}}{\pgfqpoint{1.049979in}{1.957750in}}{\pgfqpoint{1.049979in}{1.949514in}}%
\pgfpathcurveto{\pgfqpoint{1.049979in}{1.941278in}}{\pgfqpoint{1.053251in}{1.933378in}}{\pgfqpoint{1.059075in}{1.927554in}}%
\pgfpathcurveto{\pgfqpoint{1.064899in}{1.921730in}}{\pgfqpoint{1.072799in}{1.918458in}}{\pgfqpoint{1.081035in}{1.918458in}}%
\pgfpathclose%
\pgfusepath{stroke,fill}%
\end{pgfscope}%
\begin{pgfscope}%
\pgfpathrectangle{\pgfqpoint{0.100000in}{0.212622in}}{\pgfqpoint{3.696000in}{3.696000in}}%
\pgfusepath{clip}%
\pgfsetbuttcap%
\pgfsetroundjoin%
\definecolor{currentfill}{rgb}{0.121569,0.466667,0.705882}%
\pgfsetfillcolor{currentfill}%
\pgfsetfillopacity{0.560093}%
\pgfsetlinewidth{1.003750pt}%
\definecolor{currentstroke}{rgb}{0.121569,0.466667,0.705882}%
\pgfsetstrokecolor{currentstroke}%
\pgfsetstrokeopacity{0.560093}%
\pgfsetdash{}{0pt}%
\pgfpathmoveto{\pgfqpoint{1.076908in}{1.911397in}}%
\pgfpathcurveto{\pgfqpoint{1.085144in}{1.911397in}}{\pgfqpoint{1.093044in}{1.914670in}}{\pgfqpoint{1.098868in}{1.920494in}}%
\pgfpathcurveto{\pgfqpoint{1.104692in}{1.926318in}}{\pgfqpoint{1.107964in}{1.934218in}}{\pgfqpoint{1.107964in}{1.942454in}}%
\pgfpathcurveto{\pgfqpoint{1.107964in}{1.950690in}}{\pgfqpoint{1.104692in}{1.958590in}}{\pgfqpoint{1.098868in}{1.964414in}}%
\pgfpathcurveto{\pgfqpoint{1.093044in}{1.970238in}}{\pgfqpoint{1.085144in}{1.973510in}}{\pgfqpoint{1.076908in}{1.973510in}}%
\pgfpathcurveto{\pgfqpoint{1.068671in}{1.973510in}}{\pgfqpoint{1.060771in}{1.970238in}}{\pgfqpoint{1.054947in}{1.964414in}}%
\pgfpathcurveto{\pgfqpoint{1.049124in}{1.958590in}}{\pgfqpoint{1.045851in}{1.950690in}}{\pgfqpoint{1.045851in}{1.942454in}}%
\pgfpathcurveto{\pgfqpoint{1.045851in}{1.934218in}}{\pgfqpoint{1.049124in}{1.926318in}}{\pgfqpoint{1.054947in}{1.920494in}}%
\pgfpathcurveto{\pgfqpoint{1.060771in}{1.914670in}}{\pgfqpoint{1.068671in}{1.911397in}}{\pgfqpoint{1.076908in}{1.911397in}}%
\pgfpathclose%
\pgfusepath{stroke,fill}%
\end{pgfscope}%
\begin{pgfscope}%
\pgfpathrectangle{\pgfqpoint{0.100000in}{0.212622in}}{\pgfqpoint{3.696000in}{3.696000in}}%
\pgfusepath{clip}%
\pgfsetbuttcap%
\pgfsetroundjoin%
\definecolor{currentfill}{rgb}{0.121569,0.466667,0.705882}%
\pgfsetfillcolor{currentfill}%
\pgfsetfillopacity{0.561259}%
\pgfsetlinewidth{1.003750pt}%
\definecolor{currentstroke}{rgb}{0.121569,0.466667,0.705882}%
\pgfsetstrokecolor{currentstroke}%
\pgfsetstrokeopacity{0.561259}%
\pgfsetdash{}{0pt}%
\pgfpathmoveto{\pgfqpoint{1.073681in}{1.904844in}}%
\pgfpathcurveto{\pgfqpoint{1.081917in}{1.904844in}}{\pgfqpoint{1.089817in}{1.908116in}}{\pgfqpoint{1.095641in}{1.913940in}}%
\pgfpathcurveto{\pgfqpoint{1.101465in}{1.919764in}}{\pgfqpoint{1.104737in}{1.927664in}}{\pgfqpoint{1.104737in}{1.935900in}}%
\pgfpathcurveto{\pgfqpoint{1.104737in}{1.944136in}}{\pgfqpoint{1.101465in}{1.952036in}}{\pgfqpoint{1.095641in}{1.957860in}}%
\pgfpathcurveto{\pgfqpoint{1.089817in}{1.963684in}}{\pgfqpoint{1.081917in}{1.966957in}}{\pgfqpoint{1.073681in}{1.966957in}}%
\pgfpathcurveto{\pgfqpoint{1.065444in}{1.966957in}}{\pgfqpoint{1.057544in}{1.963684in}}{\pgfqpoint{1.051721in}{1.957860in}}%
\pgfpathcurveto{\pgfqpoint{1.045897in}{1.952036in}}{\pgfqpoint{1.042624in}{1.944136in}}{\pgfqpoint{1.042624in}{1.935900in}}%
\pgfpathcurveto{\pgfqpoint{1.042624in}{1.927664in}}{\pgfqpoint{1.045897in}{1.919764in}}{\pgfqpoint{1.051721in}{1.913940in}}%
\pgfpathcurveto{\pgfqpoint{1.057544in}{1.908116in}}{\pgfqpoint{1.065444in}{1.904844in}}{\pgfqpoint{1.073681in}{1.904844in}}%
\pgfpathclose%
\pgfusepath{stroke,fill}%
\end{pgfscope}%
\begin{pgfscope}%
\pgfpathrectangle{\pgfqpoint{0.100000in}{0.212622in}}{\pgfqpoint{3.696000in}{3.696000in}}%
\pgfusepath{clip}%
\pgfsetbuttcap%
\pgfsetroundjoin%
\definecolor{currentfill}{rgb}{0.121569,0.466667,0.705882}%
\pgfsetfillcolor{currentfill}%
\pgfsetfillopacity{0.561774}%
\pgfsetlinewidth{1.003750pt}%
\definecolor{currentstroke}{rgb}{0.121569,0.466667,0.705882}%
\pgfsetstrokecolor{currentstroke}%
\pgfsetstrokeopacity{0.561774}%
\pgfsetdash{}{0pt}%
\pgfpathmoveto{\pgfqpoint{1.071781in}{1.901337in}}%
\pgfpathcurveto{\pgfqpoint{1.080017in}{1.901337in}}{\pgfqpoint{1.087917in}{1.904609in}}{\pgfqpoint{1.093741in}{1.910433in}}%
\pgfpathcurveto{\pgfqpoint{1.099565in}{1.916257in}}{\pgfqpoint{1.102837in}{1.924157in}}{\pgfqpoint{1.102837in}{1.932393in}}%
\pgfpathcurveto{\pgfqpoint{1.102837in}{1.940629in}}{\pgfqpoint{1.099565in}{1.948529in}}{\pgfqpoint{1.093741in}{1.954353in}}%
\pgfpathcurveto{\pgfqpoint{1.087917in}{1.960177in}}{\pgfqpoint{1.080017in}{1.963450in}}{\pgfqpoint{1.071781in}{1.963450in}}%
\pgfpathcurveto{\pgfqpoint{1.063544in}{1.963450in}}{\pgfqpoint{1.055644in}{1.960177in}}{\pgfqpoint{1.049820in}{1.954353in}}%
\pgfpathcurveto{\pgfqpoint{1.043997in}{1.948529in}}{\pgfqpoint{1.040724in}{1.940629in}}{\pgfqpoint{1.040724in}{1.932393in}}%
\pgfpathcurveto{\pgfqpoint{1.040724in}{1.924157in}}{\pgfqpoint{1.043997in}{1.916257in}}{\pgfqpoint{1.049820in}{1.910433in}}%
\pgfpathcurveto{\pgfqpoint{1.055644in}{1.904609in}}{\pgfqpoint{1.063544in}{1.901337in}}{\pgfqpoint{1.071781in}{1.901337in}}%
\pgfpathclose%
\pgfusepath{stroke,fill}%
\end{pgfscope}%
\begin{pgfscope}%
\pgfpathrectangle{\pgfqpoint{0.100000in}{0.212622in}}{\pgfqpoint{3.696000in}{3.696000in}}%
\pgfusepath{clip}%
\pgfsetbuttcap%
\pgfsetroundjoin%
\definecolor{currentfill}{rgb}{0.121569,0.466667,0.705882}%
\pgfsetfillcolor{currentfill}%
\pgfsetfillopacity{0.562122}%
\pgfsetlinewidth{1.003750pt}%
\definecolor{currentstroke}{rgb}{0.121569,0.466667,0.705882}%
\pgfsetstrokecolor{currentstroke}%
\pgfsetstrokeopacity{0.562122}%
\pgfsetdash{}{0pt}%
\pgfpathmoveto{\pgfqpoint{2.076365in}{2.244236in}}%
\pgfpathcurveto{\pgfqpoint{2.084601in}{2.244236in}}{\pgfqpoint{2.092501in}{2.247508in}}{\pgfqpoint{2.098325in}{2.253332in}}%
\pgfpathcurveto{\pgfqpoint{2.104149in}{2.259156in}}{\pgfqpoint{2.107421in}{2.267056in}}{\pgfqpoint{2.107421in}{2.275292in}}%
\pgfpathcurveto{\pgfqpoint{2.107421in}{2.283528in}}{\pgfqpoint{2.104149in}{2.291428in}}{\pgfqpoint{2.098325in}{2.297252in}}%
\pgfpathcurveto{\pgfqpoint{2.092501in}{2.303076in}}{\pgfqpoint{2.084601in}{2.306349in}}{\pgfqpoint{2.076365in}{2.306349in}}%
\pgfpathcurveto{\pgfqpoint{2.068128in}{2.306349in}}{\pgfqpoint{2.060228in}{2.303076in}}{\pgfqpoint{2.054404in}{2.297252in}}%
\pgfpathcurveto{\pgfqpoint{2.048581in}{2.291428in}}{\pgfqpoint{2.045308in}{2.283528in}}{\pgfqpoint{2.045308in}{2.275292in}}%
\pgfpathcurveto{\pgfqpoint{2.045308in}{2.267056in}}{\pgfqpoint{2.048581in}{2.259156in}}{\pgfqpoint{2.054404in}{2.253332in}}%
\pgfpathcurveto{\pgfqpoint{2.060228in}{2.247508in}}{\pgfqpoint{2.068128in}{2.244236in}}{\pgfqpoint{2.076365in}{2.244236in}}%
\pgfpathclose%
\pgfusepath{stroke,fill}%
\end{pgfscope}%
\begin{pgfscope}%
\pgfpathrectangle{\pgfqpoint{0.100000in}{0.212622in}}{\pgfqpoint{3.696000in}{3.696000in}}%
\pgfusepath{clip}%
\pgfsetbuttcap%
\pgfsetroundjoin%
\definecolor{currentfill}{rgb}{0.121569,0.466667,0.705882}%
\pgfsetfillcolor{currentfill}%
\pgfsetfillopacity{0.562737}%
\pgfsetlinewidth{1.003750pt}%
\definecolor{currentstroke}{rgb}{0.121569,0.466667,0.705882}%
\pgfsetstrokecolor{currentstroke}%
\pgfsetstrokeopacity{0.562737}%
\pgfsetdash{}{0pt}%
\pgfpathmoveto{\pgfqpoint{1.069010in}{1.894249in}}%
\pgfpathcurveto{\pgfqpoint{1.077246in}{1.894249in}}{\pgfqpoint{1.085146in}{1.897521in}}{\pgfqpoint{1.090970in}{1.903345in}}%
\pgfpathcurveto{\pgfqpoint{1.096794in}{1.909169in}}{\pgfqpoint{1.100066in}{1.917069in}}{\pgfqpoint{1.100066in}{1.925305in}}%
\pgfpathcurveto{\pgfqpoint{1.100066in}{1.933542in}}{\pgfqpoint{1.096794in}{1.941442in}}{\pgfqpoint{1.090970in}{1.947266in}}%
\pgfpathcurveto{\pgfqpoint{1.085146in}{1.953090in}}{\pgfqpoint{1.077246in}{1.956362in}}{\pgfqpoint{1.069010in}{1.956362in}}%
\pgfpathcurveto{\pgfqpoint{1.060773in}{1.956362in}}{\pgfqpoint{1.052873in}{1.953090in}}{\pgfqpoint{1.047049in}{1.947266in}}%
\pgfpathcurveto{\pgfqpoint{1.041225in}{1.941442in}}{\pgfqpoint{1.037953in}{1.933542in}}{\pgfqpoint{1.037953in}{1.925305in}}%
\pgfpathcurveto{\pgfqpoint{1.037953in}{1.917069in}}{\pgfqpoint{1.041225in}{1.909169in}}{\pgfqpoint{1.047049in}{1.903345in}}%
\pgfpathcurveto{\pgfqpoint{1.052873in}{1.897521in}}{\pgfqpoint{1.060773in}{1.894249in}}{\pgfqpoint{1.069010in}{1.894249in}}%
\pgfpathclose%
\pgfusepath{stroke,fill}%
\end{pgfscope}%
\begin{pgfscope}%
\pgfpathrectangle{\pgfqpoint{0.100000in}{0.212622in}}{\pgfqpoint{3.696000in}{3.696000in}}%
\pgfusepath{clip}%
\pgfsetbuttcap%
\pgfsetroundjoin%
\definecolor{currentfill}{rgb}{0.121569,0.466667,0.705882}%
\pgfsetfillcolor{currentfill}%
\pgfsetfillopacity{0.563359}%
\pgfsetlinewidth{1.003750pt}%
\definecolor{currentstroke}{rgb}{0.121569,0.466667,0.705882}%
\pgfsetstrokecolor{currentstroke}%
\pgfsetstrokeopacity{0.563359}%
\pgfsetdash{}{0pt}%
\pgfpathmoveto{\pgfqpoint{1.066874in}{1.890352in}}%
\pgfpathcurveto{\pgfqpoint{1.075110in}{1.890352in}}{\pgfqpoint{1.083010in}{1.893624in}}{\pgfqpoint{1.088834in}{1.899448in}}%
\pgfpathcurveto{\pgfqpoint{1.094658in}{1.905272in}}{\pgfqpoint{1.097930in}{1.913172in}}{\pgfqpoint{1.097930in}{1.921408in}}%
\pgfpathcurveto{\pgfqpoint{1.097930in}{1.929645in}}{\pgfqpoint{1.094658in}{1.937545in}}{\pgfqpoint{1.088834in}{1.943369in}}%
\pgfpathcurveto{\pgfqpoint{1.083010in}{1.949192in}}{\pgfqpoint{1.075110in}{1.952465in}}{\pgfqpoint{1.066874in}{1.952465in}}%
\pgfpathcurveto{\pgfqpoint{1.058637in}{1.952465in}}{\pgfqpoint{1.050737in}{1.949192in}}{\pgfqpoint{1.044913in}{1.943369in}}%
\pgfpathcurveto{\pgfqpoint{1.039089in}{1.937545in}}{\pgfqpoint{1.035817in}{1.929645in}}{\pgfqpoint{1.035817in}{1.921408in}}%
\pgfpathcurveto{\pgfqpoint{1.035817in}{1.913172in}}{\pgfqpoint{1.039089in}{1.905272in}}{\pgfqpoint{1.044913in}{1.899448in}}%
\pgfpathcurveto{\pgfqpoint{1.050737in}{1.893624in}}{\pgfqpoint{1.058637in}{1.890352in}}{\pgfqpoint{1.066874in}{1.890352in}}%
\pgfpathclose%
\pgfusepath{stroke,fill}%
\end{pgfscope}%
\begin{pgfscope}%
\pgfpathrectangle{\pgfqpoint{0.100000in}{0.212622in}}{\pgfqpoint{3.696000in}{3.696000in}}%
\pgfusepath{clip}%
\pgfsetbuttcap%
\pgfsetroundjoin%
\definecolor{currentfill}{rgb}{0.121569,0.466667,0.705882}%
\pgfsetfillcolor{currentfill}%
\pgfsetfillopacity{0.564500}%
\pgfsetlinewidth{1.003750pt}%
\definecolor{currentstroke}{rgb}{0.121569,0.466667,0.705882}%
\pgfsetstrokecolor{currentstroke}%
\pgfsetstrokeopacity{0.564500}%
\pgfsetdash{}{0pt}%
\pgfpathmoveto{\pgfqpoint{1.063641in}{1.882533in}}%
\pgfpathcurveto{\pgfqpoint{1.071877in}{1.882533in}}{\pgfqpoint{1.079777in}{1.885805in}}{\pgfqpoint{1.085601in}{1.891629in}}%
\pgfpathcurveto{\pgfqpoint{1.091425in}{1.897453in}}{\pgfqpoint{1.094697in}{1.905353in}}{\pgfqpoint{1.094697in}{1.913590in}}%
\pgfpathcurveto{\pgfqpoint{1.094697in}{1.921826in}}{\pgfqpoint{1.091425in}{1.929726in}}{\pgfqpoint{1.085601in}{1.935550in}}%
\pgfpathcurveto{\pgfqpoint{1.079777in}{1.941374in}}{\pgfqpoint{1.071877in}{1.944646in}}{\pgfqpoint{1.063641in}{1.944646in}}%
\pgfpathcurveto{\pgfqpoint{1.055405in}{1.944646in}}{\pgfqpoint{1.047505in}{1.941374in}}{\pgfqpoint{1.041681in}{1.935550in}}%
\pgfpathcurveto{\pgfqpoint{1.035857in}{1.929726in}}{\pgfqpoint{1.032584in}{1.921826in}}{\pgfqpoint{1.032584in}{1.913590in}}%
\pgfpathcurveto{\pgfqpoint{1.032584in}{1.905353in}}{\pgfqpoint{1.035857in}{1.897453in}}{\pgfqpoint{1.041681in}{1.891629in}}%
\pgfpathcurveto{\pgfqpoint{1.047505in}{1.885805in}}{\pgfqpoint{1.055405in}{1.882533in}}{\pgfqpoint{1.063641in}{1.882533in}}%
\pgfpathclose%
\pgfusepath{stroke,fill}%
\end{pgfscope}%
\begin{pgfscope}%
\pgfpathrectangle{\pgfqpoint{0.100000in}{0.212622in}}{\pgfqpoint{3.696000in}{3.696000in}}%
\pgfusepath{clip}%
\pgfsetbuttcap%
\pgfsetroundjoin%
\definecolor{currentfill}{rgb}{0.121569,0.466667,0.705882}%
\pgfsetfillcolor{currentfill}%
\pgfsetfillopacity{0.564986}%
\pgfsetlinewidth{1.003750pt}%
\definecolor{currentstroke}{rgb}{0.121569,0.466667,0.705882}%
\pgfsetstrokecolor{currentstroke}%
\pgfsetstrokeopacity{0.564986}%
\pgfsetdash{}{0pt}%
\pgfpathmoveto{\pgfqpoint{2.077789in}{2.234384in}}%
\pgfpathcurveto{\pgfqpoint{2.086025in}{2.234384in}}{\pgfqpoint{2.093925in}{2.237656in}}{\pgfqpoint{2.099749in}{2.243480in}}%
\pgfpathcurveto{\pgfqpoint{2.105573in}{2.249304in}}{\pgfqpoint{2.108846in}{2.257204in}}{\pgfqpoint{2.108846in}{2.265440in}}%
\pgfpathcurveto{\pgfqpoint{2.108846in}{2.273676in}}{\pgfqpoint{2.105573in}{2.281577in}}{\pgfqpoint{2.099749in}{2.287400in}}%
\pgfpathcurveto{\pgfqpoint{2.093925in}{2.293224in}}{\pgfqpoint{2.086025in}{2.296497in}}{\pgfqpoint{2.077789in}{2.296497in}}%
\pgfpathcurveto{\pgfqpoint{2.069553in}{2.296497in}}{\pgfqpoint{2.061653in}{2.293224in}}{\pgfqpoint{2.055829in}{2.287400in}}%
\pgfpathcurveto{\pgfqpoint{2.050005in}{2.281577in}}{\pgfqpoint{2.046733in}{2.273676in}}{\pgfqpoint{2.046733in}{2.265440in}}%
\pgfpathcurveto{\pgfqpoint{2.046733in}{2.257204in}}{\pgfqpoint{2.050005in}{2.249304in}}{\pgfqpoint{2.055829in}{2.243480in}}%
\pgfpathcurveto{\pgfqpoint{2.061653in}{2.237656in}}{\pgfqpoint{2.069553in}{2.234384in}}{\pgfqpoint{2.077789in}{2.234384in}}%
\pgfpathclose%
\pgfusepath{stroke,fill}%
\end{pgfscope}%
\begin{pgfscope}%
\pgfpathrectangle{\pgfqpoint{0.100000in}{0.212622in}}{\pgfqpoint{3.696000in}{3.696000in}}%
\pgfusepath{clip}%
\pgfsetbuttcap%
\pgfsetroundjoin%
\definecolor{currentfill}{rgb}{0.121569,0.466667,0.705882}%
\pgfsetfillcolor{currentfill}%
\pgfsetfillopacity{0.565479}%
\pgfsetlinewidth{1.003750pt}%
\definecolor{currentstroke}{rgb}{0.121569,0.466667,0.705882}%
\pgfsetstrokecolor{currentstroke}%
\pgfsetstrokeopacity{0.565479}%
\pgfsetdash{}{0pt}%
\pgfpathmoveto{\pgfqpoint{1.060113in}{1.875184in}}%
\pgfpathcurveto{\pgfqpoint{1.068349in}{1.875184in}}{\pgfqpoint{1.076249in}{1.878457in}}{\pgfqpoint{1.082073in}{1.884281in}}%
\pgfpathcurveto{\pgfqpoint{1.087897in}{1.890105in}}{\pgfqpoint{1.091169in}{1.898005in}}{\pgfqpoint{1.091169in}{1.906241in}}%
\pgfpathcurveto{\pgfqpoint{1.091169in}{1.914477in}}{\pgfqpoint{1.087897in}{1.922377in}}{\pgfqpoint{1.082073in}{1.928201in}}%
\pgfpathcurveto{\pgfqpoint{1.076249in}{1.934025in}}{\pgfqpoint{1.068349in}{1.937297in}}{\pgfqpoint{1.060113in}{1.937297in}}%
\pgfpathcurveto{\pgfqpoint{1.051876in}{1.937297in}}{\pgfqpoint{1.043976in}{1.934025in}}{\pgfqpoint{1.038152in}{1.928201in}}%
\pgfpathcurveto{\pgfqpoint{1.032329in}{1.922377in}}{\pgfqpoint{1.029056in}{1.914477in}}{\pgfqpoint{1.029056in}{1.906241in}}%
\pgfpathcurveto{\pgfqpoint{1.029056in}{1.898005in}}{\pgfqpoint{1.032329in}{1.890105in}}{\pgfqpoint{1.038152in}{1.884281in}}%
\pgfpathcurveto{\pgfqpoint{1.043976in}{1.878457in}}{\pgfqpoint{1.051876in}{1.875184in}}{\pgfqpoint{1.060113in}{1.875184in}}%
\pgfpathclose%
\pgfusepath{stroke,fill}%
\end{pgfscope}%
\begin{pgfscope}%
\pgfpathrectangle{\pgfqpoint{0.100000in}{0.212622in}}{\pgfqpoint{3.696000in}{3.696000in}}%
\pgfusepath{clip}%
\pgfsetbuttcap%
\pgfsetroundjoin%
\definecolor{currentfill}{rgb}{0.121569,0.466667,0.705882}%
\pgfsetfillcolor{currentfill}%
\pgfsetfillopacity{0.566151}%
\pgfsetlinewidth{1.003750pt}%
\definecolor{currentstroke}{rgb}{0.121569,0.466667,0.705882}%
\pgfsetstrokecolor{currentstroke}%
\pgfsetstrokeopacity{0.566151}%
\pgfsetdash{}{0pt}%
\pgfpathmoveto{\pgfqpoint{1.056586in}{1.867875in}}%
\pgfpathcurveto{\pgfqpoint{1.064822in}{1.867875in}}{\pgfqpoint{1.072722in}{1.871147in}}{\pgfqpoint{1.078546in}{1.876971in}}%
\pgfpathcurveto{\pgfqpoint{1.084370in}{1.882795in}}{\pgfqpoint{1.087642in}{1.890695in}}{\pgfqpoint{1.087642in}{1.898931in}}%
\pgfpathcurveto{\pgfqpoint{1.087642in}{1.907167in}}{\pgfqpoint{1.084370in}{1.915067in}}{\pgfqpoint{1.078546in}{1.920891in}}%
\pgfpathcurveto{\pgfqpoint{1.072722in}{1.926715in}}{\pgfqpoint{1.064822in}{1.929988in}}{\pgfqpoint{1.056586in}{1.929988in}}%
\pgfpathcurveto{\pgfqpoint{1.048349in}{1.929988in}}{\pgfqpoint{1.040449in}{1.926715in}}{\pgfqpoint{1.034625in}{1.920891in}}%
\pgfpathcurveto{\pgfqpoint{1.028801in}{1.915067in}}{\pgfqpoint{1.025529in}{1.907167in}}{\pgfqpoint{1.025529in}{1.898931in}}%
\pgfpathcurveto{\pgfqpoint{1.025529in}{1.890695in}}{\pgfqpoint{1.028801in}{1.882795in}}{\pgfqpoint{1.034625in}{1.876971in}}%
\pgfpathcurveto{\pgfqpoint{1.040449in}{1.871147in}}{\pgfqpoint{1.048349in}{1.867875in}}{\pgfqpoint{1.056586in}{1.867875in}}%
\pgfpathclose%
\pgfusepath{stroke,fill}%
\end{pgfscope}%
\begin{pgfscope}%
\pgfpathrectangle{\pgfqpoint{0.100000in}{0.212622in}}{\pgfqpoint{3.696000in}{3.696000in}}%
\pgfusepath{clip}%
\pgfsetbuttcap%
\pgfsetroundjoin%
\definecolor{currentfill}{rgb}{0.121569,0.466667,0.705882}%
\pgfsetfillcolor{currentfill}%
\pgfsetfillopacity{0.566305}%
\pgfsetlinewidth{1.003750pt}%
\definecolor{currentstroke}{rgb}{0.121569,0.466667,0.705882}%
\pgfsetstrokecolor{currentstroke}%
\pgfsetstrokeopacity{0.566305}%
\pgfsetdash{}{0pt}%
\pgfpathmoveto{\pgfqpoint{1.055746in}{1.865910in}}%
\pgfpathcurveto{\pgfqpoint{1.063983in}{1.865910in}}{\pgfqpoint{1.071883in}{1.869182in}}{\pgfqpoint{1.077707in}{1.875006in}}%
\pgfpathcurveto{\pgfqpoint{1.083530in}{1.880830in}}{\pgfqpoint{1.086803in}{1.888730in}}{\pgfqpoint{1.086803in}{1.896966in}}%
\pgfpathcurveto{\pgfqpoint{1.086803in}{1.905202in}}{\pgfqpoint{1.083530in}{1.913103in}}{\pgfqpoint{1.077707in}{1.918926in}}%
\pgfpathcurveto{\pgfqpoint{1.071883in}{1.924750in}}{\pgfqpoint{1.063983in}{1.928023in}}{\pgfqpoint{1.055746in}{1.928023in}}%
\pgfpathcurveto{\pgfqpoint{1.047510in}{1.928023in}}{\pgfqpoint{1.039610in}{1.924750in}}{\pgfqpoint{1.033786in}{1.918926in}}%
\pgfpathcurveto{\pgfqpoint{1.027962in}{1.913103in}}{\pgfqpoint{1.024690in}{1.905202in}}{\pgfqpoint{1.024690in}{1.896966in}}%
\pgfpathcurveto{\pgfqpoint{1.024690in}{1.888730in}}{\pgfqpoint{1.027962in}{1.880830in}}{\pgfqpoint{1.033786in}{1.875006in}}%
\pgfpathcurveto{\pgfqpoint{1.039610in}{1.869182in}}{\pgfqpoint{1.047510in}{1.865910in}}{\pgfqpoint{1.055746in}{1.865910in}}%
\pgfpathclose%
\pgfusepath{stroke,fill}%
\end{pgfscope}%
\begin{pgfscope}%
\pgfpathrectangle{\pgfqpoint{0.100000in}{0.212622in}}{\pgfqpoint{3.696000in}{3.696000in}}%
\pgfusepath{clip}%
\pgfsetbuttcap%
\pgfsetroundjoin%
\definecolor{currentfill}{rgb}{0.121569,0.466667,0.705882}%
\pgfsetfillcolor{currentfill}%
\pgfsetfillopacity{0.566353}%
\pgfsetlinewidth{1.003750pt}%
\definecolor{currentstroke}{rgb}{0.121569,0.466667,0.705882}%
\pgfsetstrokecolor{currentstroke}%
\pgfsetstrokeopacity{0.566353}%
\pgfsetdash{}{0pt}%
\pgfpathmoveto{\pgfqpoint{1.055208in}{1.864481in}}%
\pgfpathcurveto{\pgfqpoint{1.063444in}{1.864481in}}{\pgfqpoint{1.071344in}{1.867754in}}{\pgfqpoint{1.077168in}{1.873578in}}%
\pgfpathcurveto{\pgfqpoint{1.082992in}{1.879402in}}{\pgfqpoint{1.086264in}{1.887302in}}{\pgfqpoint{1.086264in}{1.895538in}}%
\pgfpathcurveto{\pgfqpoint{1.086264in}{1.903774in}}{\pgfqpoint{1.082992in}{1.911674in}}{\pgfqpoint{1.077168in}{1.917498in}}%
\pgfpathcurveto{\pgfqpoint{1.071344in}{1.923322in}}{\pgfqpoint{1.063444in}{1.926594in}}{\pgfqpoint{1.055208in}{1.926594in}}%
\pgfpathcurveto{\pgfqpoint{1.046971in}{1.926594in}}{\pgfqpoint{1.039071in}{1.923322in}}{\pgfqpoint{1.033247in}{1.917498in}}%
\pgfpathcurveto{\pgfqpoint{1.027423in}{1.911674in}}{\pgfqpoint{1.024151in}{1.903774in}}{\pgfqpoint{1.024151in}{1.895538in}}%
\pgfpathcurveto{\pgfqpoint{1.024151in}{1.887302in}}{\pgfqpoint{1.027423in}{1.879402in}}{\pgfqpoint{1.033247in}{1.873578in}}%
\pgfpathcurveto{\pgfqpoint{1.039071in}{1.867754in}}{\pgfqpoint{1.046971in}{1.864481in}}{\pgfqpoint{1.055208in}{1.864481in}}%
\pgfpathclose%
\pgfusepath{stroke,fill}%
\end{pgfscope}%
\begin{pgfscope}%
\pgfpathrectangle{\pgfqpoint{0.100000in}{0.212622in}}{\pgfqpoint{3.696000in}{3.696000in}}%
\pgfusepath{clip}%
\pgfsetbuttcap%
\pgfsetroundjoin%
\definecolor{currentfill}{rgb}{0.121569,0.466667,0.705882}%
\pgfsetfillcolor{currentfill}%
\pgfsetfillopacity{0.566463}%
\pgfsetlinewidth{1.003750pt}%
\definecolor{currentstroke}{rgb}{0.121569,0.466667,0.705882}%
\pgfsetstrokecolor{currentstroke}%
\pgfsetstrokeopacity{0.566463}%
\pgfsetdash{}{0pt}%
\pgfpathmoveto{\pgfqpoint{1.054874in}{1.863907in}}%
\pgfpathcurveto{\pgfqpoint{1.063111in}{1.863907in}}{\pgfqpoint{1.071011in}{1.867179in}}{\pgfqpoint{1.076834in}{1.873003in}}%
\pgfpathcurveto{\pgfqpoint{1.082658in}{1.878827in}}{\pgfqpoint{1.085931in}{1.886727in}}{\pgfqpoint{1.085931in}{1.894963in}}%
\pgfpathcurveto{\pgfqpoint{1.085931in}{1.903199in}}{\pgfqpoint{1.082658in}{1.911099in}}{\pgfqpoint{1.076834in}{1.916923in}}%
\pgfpathcurveto{\pgfqpoint{1.071011in}{1.922747in}}{\pgfqpoint{1.063111in}{1.926020in}}{\pgfqpoint{1.054874in}{1.926020in}}%
\pgfpathcurveto{\pgfqpoint{1.046638in}{1.926020in}}{\pgfqpoint{1.038738in}{1.922747in}}{\pgfqpoint{1.032914in}{1.916923in}}%
\pgfpathcurveto{\pgfqpoint{1.027090in}{1.911099in}}{\pgfqpoint{1.023818in}{1.903199in}}{\pgfqpoint{1.023818in}{1.894963in}}%
\pgfpathcurveto{\pgfqpoint{1.023818in}{1.886727in}}{\pgfqpoint{1.027090in}{1.878827in}}{\pgfqpoint{1.032914in}{1.873003in}}%
\pgfpathcurveto{\pgfqpoint{1.038738in}{1.867179in}}{\pgfqpoint{1.046638in}{1.863907in}}{\pgfqpoint{1.054874in}{1.863907in}}%
\pgfpathclose%
\pgfusepath{stroke,fill}%
\end{pgfscope}%
\begin{pgfscope}%
\pgfpathrectangle{\pgfqpoint{0.100000in}{0.212622in}}{\pgfqpoint{3.696000in}{3.696000in}}%
\pgfusepath{clip}%
\pgfsetbuttcap%
\pgfsetroundjoin%
\definecolor{currentfill}{rgb}{0.121569,0.466667,0.705882}%
\pgfsetfillcolor{currentfill}%
\pgfsetfillopacity{0.566670}%
\pgfsetlinewidth{1.003750pt}%
\definecolor{currentstroke}{rgb}{0.121569,0.466667,0.705882}%
\pgfsetstrokecolor{currentstroke}%
\pgfsetstrokeopacity{0.566670}%
\pgfsetdash{}{0pt}%
\pgfpathmoveto{\pgfqpoint{1.054272in}{1.862878in}}%
\pgfpathcurveto{\pgfqpoint{1.062508in}{1.862878in}}{\pgfqpoint{1.070409in}{1.866150in}}{\pgfqpoint{1.076232in}{1.871974in}}%
\pgfpathcurveto{\pgfqpoint{1.082056in}{1.877798in}}{\pgfqpoint{1.085329in}{1.885698in}}{\pgfqpoint{1.085329in}{1.893934in}}%
\pgfpathcurveto{\pgfqpoint{1.085329in}{1.902171in}}{\pgfqpoint{1.082056in}{1.910071in}}{\pgfqpoint{1.076232in}{1.915895in}}%
\pgfpathcurveto{\pgfqpoint{1.070409in}{1.921719in}}{\pgfqpoint{1.062508in}{1.924991in}}{\pgfqpoint{1.054272in}{1.924991in}}%
\pgfpathcurveto{\pgfqpoint{1.046036in}{1.924991in}}{\pgfqpoint{1.038136in}{1.921719in}}{\pgfqpoint{1.032312in}{1.915895in}}%
\pgfpathcurveto{\pgfqpoint{1.026488in}{1.910071in}}{\pgfqpoint{1.023216in}{1.902171in}}{\pgfqpoint{1.023216in}{1.893934in}}%
\pgfpathcurveto{\pgfqpoint{1.023216in}{1.885698in}}{\pgfqpoint{1.026488in}{1.877798in}}{\pgfqpoint{1.032312in}{1.871974in}}%
\pgfpathcurveto{\pgfqpoint{1.038136in}{1.866150in}}{\pgfqpoint{1.046036in}{1.862878in}}{\pgfqpoint{1.054272in}{1.862878in}}%
\pgfpathclose%
\pgfusepath{stroke,fill}%
\end{pgfscope}%
\begin{pgfscope}%
\pgfpathrectangle{\pgfqpoint{0.100000in}{0.212622in}}{\pgfqpoint{3.696000in}{3.696000in}}%
\pgfusepath{clip}%
\pgfsetbuttcap%
\pgfsetroundjoin%
\definecolor{currentfill}{rgb}{0.121569,0.466667,0.705882}%
\pgfsetfillcolor{currentfill}%
\pgfsetfillopacity{0.567034}%
\pgfsetlinewidth{1.003750pt}%
\definecolor{currentstroke}{rgb}{0.121569,0.466667,0.705882}%
\pgfsetstrokecolor{currentstroke}%
\pgfsetstrokeopacity{0.567034}%
\pgfsetdash{}{0pt}%
\pgfpathmoveto{\pgfqpoint{1.053217in}{1.860908in}}%
\pgfpathcurveto{\pgfqpoint{1.061453in}{1.860908in}}{\pgfqpoint{1.069353in}{1.864181in}}{\pgfqpoint{1.075177in}{1.870005in}}%
\pgfpathcurveto{\pgfqpoint{1.081001in}{1.875829in}}{\pgfqpoint{1.084273in}{1.883729in}}{\pgfqpoint{1.084273in}{1.891965in}}%
\pgfpathcurveto{\pgfqpoint{1.084273in}{1.900201in}}{\pgfqpoint{1.081001in}{1.908101in}}{\pgfqpoint{1.075177in}{1.913925in}}%
\pgfpathcurveto{\pgfqpoint{1.069353in}{1.919749in}}{\pgfqpoint{1.061453in}{1.923021in}}{\pgfqpoint{1.053217in}{1.923021in}}%
\pgfpathcurveto{\pgfqpoint{1.044981in}{1.923021in}}{\pgfqpoint{1.037080in}{1.919749in}}{\pgfqpoint{1.031257in}{1.913925in}}%
\pgfpathcurveto{\pgfqpoint{1.025433in}{1.908101in}}{\pgfqpoint{1.022160in}{1.900201in}}{\pgfqpoint{1.022160in}{1.891965in}}%
\pgfpathcurveto{\pgfqpoint{1.022160in}{1.883729in}}{\pgfqpoint{1.025433in}{1.875829in}}{\pgfqpoint{1.031257in}{1.870005in}}%
\pgfpathcurveto{\pgfqpoint{1.037080in}{1.864181in}}{\pgfqpoint{1.044981in}{1.860908in}}{\pgfqpoint{1.053217in}{1.860908in}}%
\pgfpathclose%
\pgfusepath{stroke,fill}%
\end{pgfscope}%
\begin{pgfscope}%
\pgfpathrectangle{\pgfqpoint{0.100000in}{0.212622in}}{\pgfqpoint{3.696000in}{3.696000in}}%
\pgfusepath{clip}%
\pgfsetbuttcap%
\pgfsetroundjoin%
\definecolor{currentfill}{rgb}{0.121569,0.466667,0.705882}%
\pgfsetfillcolor{currentfill}%
\pgfsetfillopacity{0.567644}%
\pgfsetlinewidth{1.003750pt}%
\definecolor{currentstroke}{rgb}{0.121569,0.466667,0.705882}%
\pgfsetstrokecolor{currentstroke}%
\pgfsetstrokeopacity{0.567644}%
\pgfsetdash{}{0pt}%
\pgfpathmoveto{\pgfqpoint{1.051219in}{1.857246in}}%
\pgfpathcurveto{\pgfqpoint{1.059456in}{1.857246in}}{\pgfqpoint{1.067356in}{1.860519in}}{\pgfqpoint{1.073180in}{1.866343in}}%
\pgfpathcurveto{\pgfqpoint{1.079004in}{1.872166in}}{\pgfqpoint{1.082276in}{1.880067in}}{\pgfqpoint{1.082276in}{1.888303in}}%
\pgfpathcurveto{\pgfqpoint{1.082276in}{1.896539in}}{\pgfqpoint{1.079004in}{1.904439in}}{\pgfqpoint{1.073180in}{1.910263in}}%
\pgfpathcurveto{\pgfqpoint{1.067356in}{1.916087in}}{\pgfqpoint{1.059456in}{1.919359in}}{\pgfqpoint{1.051219in}{1.919359in}}%
\pgfpathcurveto{\pgfqpoint{1.042983in}{1.919359in}}{\pgfqpoint{1.035083in}{1.916087in}}{\pgfqpoint{1.029259in}{1.910263in}}%
\pgfpathcurveto{\pgfqpoint{1.023435in}{1.904439in}}{\pgfqpoint{1.020163in}{1.896539in}}{\pgfqpoint{1.020163in}{1.888303in}}%
\pgfpathcurveto{\pgfqpoint{1.020163in}{1.880067in}}{\pgfqpoint{1.023435in}{1.872166in}}{\pgfqpoint{1.029259in}{1.866343in}}%
\pgfpathcurveto{\pgfqpoint{1.035083in}{1.860519in}}{\pgfqpoint{1.042983in}{1.857246in}}{\pgfqpoint{1.051219in}{1.857246in}}%
\pgfpathclose%
\pgfusepath{stroke,fill}%
\end{pgfscope}%
\begin{pgfscope}%
\pgfpathrectangle{\pgfqpoint{0.100000in}{0.212622in}}{\pgfqpoint{3.696000in}{3.696000in}}%
\pgfusepath{clip}%
\pgfsetbuttcap%
\pgfsetroundjoin%
\definecolor{currentfill}{rgb}{0.121569,0.466667,0.705882}%
\pgfsetfillcolor{currentfill}%
\pgfsetfillopacity{0.568014}%
\pgfsetlinewidth{1.003750pt}%
\definecolor{currentstroke}{rgb}{0.121569,0.466667,0.705882}%
\pgfsetstrokecolor{currentstroke}%
\pgfsetstrokeopacity{0.568014}%
\pgfsetdash{}{0pt}%
\pgfpathmoveto{\pgfqpoint{2.080047in}{2.222979in}}%
\pgfpathcurveto{\pgfqpoint{2.088283in}{2.222979in}}{\pgfqpoint{2.096183in}{2.226251in}}{\pgfqpoint{2.102007in}{2.232075in}}%
\pgfpathcurveto{\pgfqpoint{2.107831in}{2.237899in}}{\pgfqpoint{2.111104in}{2.245799in}}{\pgfqpoint{2.111104in}{2.254035in}}%
\pgfpathcurveto{\pgfqpoint{2.111104in}{2.262272in}}{\pgfqpoint{2.107831in}{2.270172in}}{\pgfqpoint{2.102007in}{2.275996in}}%
\pgfpathcurveto{\pgfqpoint{2.096183in}{2.281820in}}{\pgfqpoint{2.088283in}{2.285092in}}{\pgfqpoint{2.080047in}{2.285092in}}%
\pgfpathcurveto{\pgfqpoint{2.071811in}{2.285092in}}{\pgfqpoint{2.063911in}{2.281820in}}{\pgfqpoint{2.058087in}{2.275996in}}%
\pgfpathcurveto{\pgfqpoint{2.052263in}{2.270172in}}{\pgfqpoint{2.048991in}{2.262272in}}{\pgfqpoint{2.048991in}{2.254035in}}%
\pgfpathcurveto{\pgfqpoint{2.048991in}{2.245799in}}{\pgfqpoint{2.052263in}{2.237899in}}{\pgfqpoint{2.058087in}{2.232075in}}%
\pgfpathcurveto{\pgfqpoint{2.063911in}{2.226251in}}{\pgfqpoint{2.071811in}{2.222979in}}{\pgfqpoint{2.080047in}{2.222979in}}%
\pgfpathclose%
\pgfusepath{stroke,fill}%
\end{pgfscope}%
\begin{pgfscope}%
\pgfpathrectangle{\pgfqpoint{0.100000in}{0.212622in}}{\pgfqpoint{3.696000in}{3.696000in}}%
\pgfusepath{clip}%
\pgfsetbuttcap%
\pgfsetroundjoin%
\definecolor{currentfill}{rgb}{0.121569,0.466667,0.705882}%
\pgfsetfillcolor{currentfill}%
\pgfsetfillopacity{0.568788}%
\pgfsetlinewidth{1.003750pt}%
\definecolor{currentstroke}{rgb}{0.121569,0.466667,0.705882}%
\pgfsetstrokecolor{currentstroke}%
\pgfsetstrokeopacity{0.568788}%
\pgfsetdash{}{0pt}%
\pgfpathmoveto{\pgfqpoint{1.047848in}{1.850376in}}%
\pgfpathcurveto{\pgfqpoint{1.056084in}{1.850376in}}{\pgfqpoint{1.063984in}{1.853648in}}{\pgfqpoint{1.069808in}{1.859472in}}%
\pgfpathcurveto{\pgfqpoint{1.075632in}{1.865296in}}{\pgfqpoint{1.078905in}{1.873196in}}{\pgfqpoint{1.078905in}{1.881432in}}%
\pgfpathcurveto{\pgfqpoint{1.078905in}{1.889669in}}{\pgfqpoint{1.075632in}{1.897569in}}{\pgfqpoint{1.069808in}{1.903393in}}%
\pgfpathcurveto{\pgfqpoint{1.063984in}{1.909217in}}{\pgfqpoint{1.056084in}{1.912489in}}{\pgfqpoint{1.047848in}{1.912489in}}%
\pgfpathcurveto{\pgfqpoint{1.039612in}{1.912489in}}{\pgfqpoint{1.031712in}{1.909217in}}{\pgfqpoint{1.025888in}{1.903393in}}%
\pgfpathcurveto{\pgfqpoint{1.020064in}{1.897569in}}{\pgfqpoint{1.016792in}{1.889669in}}{\pgfqpoint{1.016792in}{1.881432in}}%
\pgfpathcurveto{\pgfqpoint{1.016792in}{1.873196in}}{\pgfqpoint{1.020064in}{1.865296in}}{\pgfqpoint{1.025888in}{1.859472in}}%
\pgfpathcurveto{\pgfqpoint{1.031712in}{1.853648in}}{\pgfqpoint{1.039612in}{1.850376in}}{\pgfqpoint{1.047848in}{1.850376in}}%
\pgfpathclose%
\pgfusepath{stroke,fill}%
\end{pgfscope}%
\begin{pgfscope}%
\pgfpathrectangle{\pgfqpoint{0.100000in}{0.212622in}}{\pgfqpoint{3.696000in}{3.696000in}}%
\pgfusepath{clip}%
\pgfsetbuttcap%
\pgfsetroundjoin%
\definecolor{currentfill}{rgb}{0.121569,0.466667,0.705882}%
\pgfsetfillcolor{currentfill}%
\pgfsetfillopacity{0.570688}%
\pgfsetlinewidth{1.003750pt}%
\definecolor{currentstroke}{rgb}{0.121569,0.466667,0.705882}%
\pgfsetstrokecolor{currentstroke}%
\pgfsetstrokeopacity{0.570688}%
\pgfsetdash{}{0pt}%
\pgfpathmoveto{\pgfqpoint{1.041006in}{1.838177in}}%
\pgfpathcurveto{\pgfqpoint{1.049242in}{1.838177in}}{\pgfqpoint{1.057143in}{1.841449in}}{\pgfqpoint{1.062966in}{1.847273in}}%
\pgfpathcurveto{\pgfqpoint{1.068790in}{1.853097in}}{\pgfqpoint{1.072063in}{1.860997in}}{\pgfqpoint{1.072063in}{1.869233in}}%
\pgfpathcurveto{\pgfqpoint{1.072063in}{1.877470in}}{\pgfqpoint{1.068790in}{1.885370in}}{\pgfqpoint{1.062966in}{1.891194in}}%
\pgfpathcurveto{\pgfqpoint{1.057143in}{1.897017in}}{\pgfqpoint{1.049242in}{1.900290in}}{\pgfqpoint{1.041006in}{1.900290in}}%
\pgfpathcurveto{\pgfqpoint{1.032770in}{1.900290in}}{\pgfqpoint{1.024870in}{1.897017in}}{\pgfqpoint{1.019046in}{1.891194in}}%
\pgfpathcurveto{\pgfqpoint{1.013222in}{1.885370in}}{\pgfqpoint{1.009950in}{1.877470in}}{\pgfqpoint{1.009950in}{1.869233in}}%
\pgfpathcurveto{\pgfqpoint{1.009950in}{1.860997in}}{\pgfqpoint{1.013222in}{1.853097in}}{\pgfqpoint{1.019046in}{1.847273in}}%
\pgfpathcurveto{\pgfqpoint{1.024870in}{1.841449in}}{\pgfqpoint{1.032770in}{1.838177in}}{\pgfqpoint{1.041006in}{1.838177in}}%
\pgfpathclose%
\pgfusepath{stroke,fill}%
\end{pgfscope}%
\begin{pgfscope}%
\pgfpathrectangle{\pgfqpoint{0.100000in}{0.212622in}}{\pgfqpoint{3.696000in}{3.696000in}}%
\pgfusepath{clip}%
\pgfsetbuttcap%
\pgfsetroundjoin%
\definecolor{currentfill}{rgb}{0.121569,0.466667,0.705882}%
\pgfsetfillcolor{currentfill}%
\pgfsetfillopacity{0.570951}%
\pgfsetlinewidth{1.003750pt}%
\definecolor{currentstroke}{rgb}{0.121569,0.466667,0.705882}%
\pgfsetstrokecolor{currentstroke}%
\pgfsetstrokeopacity{0.570951}%
\pgfsetdash{}{0pt}%
\pgfpathmoveto{\pgfqpoint{2.082822in}{2.209838in}}%
\pgfpathcurveto{\pgfqpoint{2.091058in}{2.209838in}}{\pgfqpoint{2.098958in}{2.213111in}}{\pgfqpoint{2.104782in}{2.218935in}}%
\pgfpathcurveto{\pgfqpoint{2.110606in}{2.224759in}}{\pgfqpoint{2.113878in}{2.232659in}}{\pgfqpoint{2.113878in}{2.240895in}}%
\pgfpathcurveto{\pgfqpoint{2.113878in}{2.249131in}}{\pgfqpoint{2.110606in}{2.257031in}}{\pgfqpoint{2.104782in}{2.262855in}}%
\pgfpathcurveto{\pgfqpoint{2.098958in}{2.268679in}}{\pgfqpoint{2.091058in}{2.271951in}}{\pgfqpoint{2.082822in}{2.271951in}}%
\pgfpathcurveto{\pgfqpoint{2.074585in}{2.271951in}}{\pgfqpoint{2.066685in}{2.268679in}}{\pgfqpoint{2.060861in}{2.262855in}}%
\pgfpathcurveto{\pgfqpoint{2.055037in}{2.257031in}}{\pgfqpoint{2.051765in}{2.249131in}}{\pgfqpoint{2.051765in}{2.240895in}}%
\pgfpathcurveto{\pgfqpoint{2.051765in}{2.232659in}}{\pgfqpoint{2.055037in}{2.224759in}}{\pgfqpoint{2.060861in}{2.218935in}}%
\pgfpathcurveto{\pgfqpoint{2.066685in}{2.213111in}}{\pgfqpoint{2.074585in}{2.209838in}}{\pgfqpoint{2.082822in}{2.209838in}}%
\pgfpathclose%
\pgfusepath{stroke,fill}%
\end{pgfscope}%
\begin{pgfscope}%
\pgfpathrectangle{\pgfqpoint{0.100000in}{0.212622in}}{\pgfqpoint{3.696000in}{3.696000in}}%
\pgfusepath{clip}%
\pgfsetbuttcap%
\pgfsetroundjoin%
\definecolor{currentfill}{rgb}{0.121569,0.466667,0.705882}%
\pgfsetfillcolor{currentfill}%
\pgfsetfillopacity{0.572490}%
\pgfsetlinewidth{1.003750pt}%
\definecolor{currentstroke}{rgb}{0.121569,0.466667,0.705882}%
\pgfsetstrokecolor{currentstroke}%
\pgfsetstrokeopacity{0.572490}%
\pgfsetdash{}{0pt}%
\pgfpathmoveto{\pgfqpoint{1.036295in}{1.826737in}}%
\pgfpathcurveto{\pgfqpoint{1.044531in}{1.826737in}}{\pgfqpoint{1.052431in}{1.830009in}}{\pgfqpoint{1.058255in}{1.835833in}}%
\pgfpathcurveto{\pgfqpoint{1.064079in}{1.841657in}}{\pgfqpoint{1.067352in}{1.849557in}}{\pgfqpoint{1.067352in}{1.857794in}}%
\pgfpathcurveto{\pgfqpoint{1.067352in}{1.866030in}}{\pgfqpoint{1.064079in}{1.873930in}}{\pgfqpoint{1.058255in}{1.879754in}}%
\pgfpathcurveto{\pgfqpoint{1.052431in}{1.885578in}}{\pgfqpoint{1.044531in}{1.888850in}}{\pgfqpoint{1.036295in}{1.888850in}}%
\pgfpathcurveto{\pgfqpoint{1.028059in}{1.888850in}}{\pgfqpoint{1.020159in}{1.885578in}}{\pgfqpoint{1.014335in}{1.879754in}}%
\pgfpathcurveto{\pgfqpoint{1.008511in}{1.873930in}}{\pgfqpoint{1.005239in}{1.866030in}}{\pgfqpoint{1.005239in}{1.857794in}}%
\pgfpathcurveto{\pgfqpoint{1.005239in}{1.849557in}}{\pgfqpoint{1.008511in}{1.841657in}}{\pgfqpoint{1.014335in}{1.835833in}}%
\pgfpathcurveto{\pgfqpoint{1.020159in}{1.830009in}}{\pgfqpoint{1.028059in}{1.826737in}}{\pgfqpoint{1.036295in}{1.826737in}}%
\pgfpathclose%
\pgfusepath{stroke,fill}%
\end{pgfscope}%
\begin{pgfscope}%
\pgfpathrectangle{\pgfqpoint{0.100000in}{0.212622in}}{\pgfqpoint{3.696000in}{3.696000in}}%
\pgfusepath{clip}%
\pgfsetbuttcap%
\pgfsetroundjoin%
\definecolor{currentfill}{rgb}{0.121569,0.466667,0.705882}%
\pgfsetfillcolor{currentfill}%
\pgfsetfillopacity{0.573791}%
\pgfsetlinewidth{1.003750pt}%
\definecolor{currentstroke}{rgb}{0.121569,0.466667,0.705882}%
\pgfsetstrokecolor{currentstroke}%
\pgfsetstrokeopacity{0.573791}%
\pgfsetdash{}{0pt}%
\pgfpathmoveto{\pgfqpoint{1.031642in}{1.817959in}}%
\pgfpathcurveto{\pgfqpoint{1.039878in}{1.817959in}}{\pgfqpoint{1.047778in}{1.821232in}}{\pgfqpoint{1.053602in}{1.827056in}}%
\pgfpathcurveto{\pgfqpoint{1.059426in}{1.832880in}}{\pgfqpoint{1.062699in}{1.840780in}}{\pgfqpoint{1.062699in}{1.849016in}}%
\pgfpathcurveto{\pgfqpoint{1.062699in}{1.857252in}}{\pgfqpoint{1.059426in}{1.865152in}}{\pgfqpoint{1.053602in}{1.870976in}}%
\pgfpathcurveto{\pgfqpoint{1.047778in}{1.876800in}}{\pgfqpoint{1.039878in}{1.880072in}}{\pgfqpoint{1.031642in}{1.880072in}}%
\pgfpathcurveto{\pgfqpoint{1.023406in}{1.880072in}}{\pgfqpoint{1.015506in}{1.876800in}}{\pgfqpoint{1.009682in}{1.870976in}}%
\pgfpathcurveto{\pgfqpoint{1.003858in}{1.865152in}}{\pgfqpoint{1.000586in}{1.857252in}}{\pgfqpoint{1.000586in}{1.849016in}}%
\pgfpathcurveto{\pgfqpoint{1.000586in}{1.840780in}}{\pgfqpoint{1.003858in}{1.832880in}}{\pgfqpoint{1.009682in}{1.827056in}}%
\pgfpathcurveto{\pgfqpoint{1.015506in}{1.821232in}}{\pgfqpoint{1.023406in}{1.817959in}}{\pgfqpoint{1.031642in}{1.817959in}}%
\pgfpathclose%
\pgfusepath{stroke,fill}%
\end{pgfscope}%
\begin{pgfscope}%
\pgfpathrectangle{\pgfqpoint{0.100000in}{0.212622in}}{\pgfqpoint{3.696000in}{3.696000in}}%
\pgfusepath{clip}%
\pgfsetbuttcap%
\pgfsetroundjoin%
\definecolor{currentfill}{rgb}{0.121569,0.466667,0.705882}%
\pgfsetfillcolor{currentfill}%
\pgfsetfillopacity{0.574779}%
\pgfsetlinewidth{1.003750pt}%
\definecolor{currentstroke}{rgb}{0.121569,0.466667,0.705882}%
\pgfsetstrokecolor{currentstroke}%
\pgfsetstrokeopacity{0.574779}%
\pgfsetdash{}{0pt}%
\pgfpathmoveto{\pgfqpoint{2.084657in}{2.196051in}}%
\pgfpathcurveto{\pgfqpoint{2.092894in}{2.196051in}}{\pgfqpoint{2.100794in}{2.199324in}}{\pgfqpoint{2.106618in}{2.205148in}}%
\pgfpathcurveto{\pgfqpoint{2.112442in}{2.210972in}}{\pgfqpoint{2.115714in}{2.218872in}}{\pgfqpoint{2.115714in}{2.227108in}}%
\pgfpathcurveto{\pgfqpoint{2.115714in}{2.235344in}}{\pgfqpoint{2.112442in}{2.243244in}}{\pgfqpoint{2.106618in}{2.249068in}}%
\pgfpathcurveto{\pgfqpoint{2.100794in}{2.254892in}}{\pgfqpoint{2.092894in}{2.258164in}}{\pgfqpoint{2.084657in}{2.258164in}}%
\pgfpathcurveto{\pgfqpoint{2.076421in}{2.258164in}}{\pgfqpoint{2.068521in}{2.254892in}}{\pgfqpoint{2.062697in}{2.249068in}}%
\pgfpathcurveto{\pgfqpoint{2.056873in}{2.243244in}}{\pgfqpoint{2.053601in}{2.235344in}}{\pgfqpoint{2.053601in}{2.227108in}}%
\pgfpathcurveto{\pgfqpoint{2.053601in}{2.218872in}}{\pgfqpoint{2.056873in}{2.210972in}}{\pgfqpoint{2.062697in}{2.205148in}}%
\pgfpathcurveto{\pgfqpoint{2.068521in}{2.199324in}}{\pgfqpoint{2.076421in}{2.196051in}}{\pgfqpoint{2.084657in}{2.196051in}}%
\pgfpathclose%
\pgfusepath{stroke,fill}%
\end{pgfscope}%
\begin{pgfscope}%
\pgfpathrectangle{\pgfqpoint{0.100000in}{0.212622in}}{\pgfqpoint{3.696000in}{3.696000in}}%
\pgfusepath{clip}%
\pgfsetbuttcap%
\pgfsetroundjoin%
\definecolor{currentfill}{rgb}{0.121569,0.466667,0.705882}%
\pgfsetfillcolor{currentfill}%
\pgfsetfillopacity{0.575007}%
\pgfsetlinewidth{1.003750pt}%
\definecolor{currentstroke}{rgb}{0.121569,0.466667,0.705882}%
\pgfsetstrokecolor{currentstroke}%
\pgfsetstrokeopacity{0.575007}%
\pgfsetdash{}{0pt}%
\pgfpathmoveto{\pgfqpoint{1.028340in}{1.809047in}}%
\pgfpathcurveto{\pgfqpoint{1.036576in}{1.809047in}}{\pgfqpoint{1.044476in}{1.812319in}}{\pgfqpoint{1.050300in}{1.818143in}}%
\pgfpathcurveto{\pgfqpoint{1.056124in}{1.823967in}}{\pgfqpoint{1.059397in}{1.831867in}}{\pgfqpoint{1.059397in}{1.840104in}}%
\pgfpathcurveto{\pgfqpoint{1.059397in}{1.848340in}}{\pgfqpoint{1.056124in}{1.856240in}}{\pgfqpoint{1.050300in}{1.862064in}}%
\pgfpathcurveto{\pgfqpoint{1.044476in}{1.867888in}}{\pgfqpoint{1.036576in}{1.871160in}}{\pgfqpoint{1.028340in}{1.871160in}}%
\pgfpathcurveto{\pgfqpoint{1.020104in}{1.871160in}}{\pgfqpoint{1.012204in}{1.867888in}}{\pgfqpoint{1.006380in}{1.862064in}}%
\pgfpathcurveto{\pgfqpoint{1.000556in}{1.856240in}}{\pgfqpoint{0.997284in}{1.848340in}}{\pgfqpoint{0.997284in}{1.840104in}}%
\pgfpathcurveto{\pgfqpoint{0.997284in}{1.831867in}}{\pgfqpoint{1.000556in}{1.823967in}}{\pgfqpoint{1.006380in}{1.818143in}}%
\pgfpathcurveto{\pgfqpoint{1.012204in}{1.812319in}}{\pgfqpoint{1.020104in}{1.809047in}}{\pgfqpoint{1.028340in}{1.809047in}}%
\pgfpathclose%
\pgfusepath{stroke,fill}%
\end{pgfscope}%
\begin{pgfscope}%
\pgfpathrectangle{\pgfqpoint{0.100000in}{0.212622in}}{\pgfqpoint{3.696000in}{3.696000in}}%
\pgfusepath{clip}%
\pgfsetbuttcap%
\pgfsetroundjoin%
\definecolor{currentfill}{rgb}{0.121569,0.466667,0.705882}%
\pgfsetfillcolor{currentfill}%
\pgfsetfillopacity{0.576210}%
\pgfsetlinewidth{1.003750pt}%
\definecolor{currentstroke}{rgb}{0.121569,0.466667,0.705882}%
\pgfsetstrokecolor{currentstroke}%
\pgfsetstrokeopacity{0.576210}%
\pgfsetdash{}{0pt}%
\pgfpathmoveto{\pgfqpoint{1.024712in}{1.802136in}}%
\pgfpathcurveto{\pgfqpoint{1.032948in}{1.802136in}}{\pgfqpoint{1.040848in}{1.805408in}}{\pgfqpoint{1.046672in}{1.811232in}}%
\pgfpathcurveto{\pgfqpoint{1.052496in}{1.817056in}}{\pgfqpoint{1.055769in}{1.824956in}}{\pgfqpoint{1.055769in}{1.833193in}}%
\pgfpathcurveto{\pgfqpoint{1.055769in}{1.841429in}}{\pgfqpoint{1.052496in}{1.849329in}}{\pgfqpoint{1.046672in}{1.855153in}}%
\pgfpathcurveto{\pgfqpoint{1.040848in}{1.860977in}}{\pgfqpoint{1.032948in}{1.864249in}}{\pgfqpoint{1.024712in}{1.864249in}}%
\pgfpathcurveto{\pgfqpoint{1.016476in}{1.864249in}}{\pgfqpoint{1.008576in}{1.860977in}}{\pgfqpoint{1.002752in}{1.855153in}}%
\pgfpathcurveto{\pgfqpoint{0.996928in}{1.849329in}}{\pgfqpoint{0.993656in}{1.841429in}}{\pgfqpoint{0.993656in}{1.833193in}}%
\pgfpathcurveto{\pgfqpoint{0.993656in}{1.824956in}}{\pgfqpoint{0.996928in}{1.817056in}}{\pgfqpoint{1.002752in}{1.811232in}}%
\pgfpathcurveto{\pgfqpoint{1.008576in}{1.805408in}}{\pgfqpoint{1.016476in}{1.802136in}}{\pgfqpoint{1.024712in}{1.802136in}}%
\pgfpathclose%
\pgfusepath{stroke,fill}%
\end{pgfscope}%
\begin{pgfscope}%
\pgfpathrectangle{\pgfqpoint{0.100000in}{0.212622in}}{\pgfqpoint{3.696000in}{3.696000in}}%
\pgfusepath{clip}%
\pgfsetbuttcap%
\pgfsetroundjoin%
\definecolor{currentfill}{rgb}{0.121569,0.466667,0.705882}%
\pgfsetfillcolor{currentfill}%
\pgfsetfillopacity{0.576865}%
\pgfsetlinewidth{1.003750pt}%
\definecolor{currentstroke}{rgb}{0.121569,0.466667,0.705882}%
\pgfsetstrokecolor{currentstroke}%
\pgfsetstrokeopacity{0.576865}%
\pgfsetdash{}{0pt}%
\pgfpathmoveto{\pgfqpoint{0.897365in}{1.537425in}}%
\pgfpathcurveto{\pgfqpoint{0.905601in}{1.537425in}}{\pgfqpoint{0.913501in}{1.540697in}}{\pgfqpoint{0.919325in}{1.546521in}}%
\pgfpathcurveto{\pgfqpoint{0.925149in}{1.552345in}}{\pgfqpoint{0.928421in}{1.560245in}}{\pgfqpoint{0.928421in}{1.568481in}}%
\pgfpathcurveto{\pgfqpoint{0.928421in}{1.576717in}}{\pgfqpoint{0.925149in}{1.584617in}}{\pgfqpoint{0.919325in}{1.590441in}}%
\pgfpathcurveto{\pgfqpoint{0.913501in}{1.596265in}}{\pgfqpoint{0.905601in}{1.599538in}}{\pgfqpoint{0.897365in}{1.599538in}}%
\pgfpathcurveto{\pgfqpoint{0.889128in}{1.599538in}}{\pgfqpoint{0.881228in}{1.596265in}}{\pgfqpoint{0.875404in}{1.590441in}}%
\pgfpathcurveto{\pgfqpoint{0.869581in}{1.584617in}}{\pgfqpoint{0.866308in}{1.576717in}}{\pgfqpoint{0.866308in}{1.568481in}}%
\pgfpathcurveto{\pgfqpoint{0.866308in}{1.560245in}}{\pgfqpoint{0.869581in}{1.552345in}}{\pgfqpoint{0.875404in}{1.546521in}}%
\pgfpathcurveto{\pgfqpoint{0.881228in}{1.540697in}}{\pgfqpoint{0.889128in}{1.537425in}}{\pgfqpoint{0.897365in}{1.537425in}}%
\pgfpathclose%
\pgfusepath{stroke,fill}%
\end{pgfscope}%
\begin{pgfscope}%
\pgfpathrectangle{\pgfqpoint{0.100000in}{0.212622in}}{\pgfqpoint{3.696000in}{3.696000in}}%
\pgfusepath{clip}%
\pgfsetbuttcap%
\pgfsetroundjoin%
\definecolor{currentfill}{rgb}{0.121569,0.466667,0.705882}%
\pgfsetfillcolor{currentfill}%
\pgfsetfillopacity{0.576875}%
\pgfsetlinewidth{1.003750pt}%
\definecolor{currentstroke}{rgb}{0.121569,0.466667,0.705882}%
\pgfsetstrokecolor{currentstroke}%
\pgfsetstrokeopacity{0.576875}%
\pgfsetdash{}{0pt}%
\pgfpathmoveto{\pgfqpoint{0.896486in}{1.539137in}}%
\pgfpathcurveto{\pgfqpoint{0.904723in}{1.539137in}}{\pgfqpoint{0.912623in}{1.542409in}}{\pgfqpoint{0.918447in}{1.548233in}}%
\pgfpathcurveto{\pgfqpoint{0.924271in}{1.554057in}}{\pgfqpoint{0.927543in}{1.561957in}}{\pgfqpoint{0.927543in}{1.570193in}}%
\pgfpathcurveto{\pgfqpoint{0.927543in}{1.578430in}}{\pgfqpoint{0.924271in}{1.586330in}}{\pgfqpoint{0.918447in}{1.592154in}}%
\pgfpathcurveto{\pgfqpoint{0.912623in}{1.597978in}}{\pgfqpoint{0.904723in}{1.601250in}}{\pgfqpoint{0.896486in}{1.601250in}}%
\pgfpathcurveto{\pgfqpoint{0.888250in}{1.601250in}}{\pgfqpoint{0.880350in}{1.597978in}}{\pgfqpoint{0.874526in}{1.592154in}}%
\pgfpathcurveto{\pgfqpoint{0.868702in}{1.586330in}}{\pgfqpoint{0.865430in}{1.578430in}}{\pgfqpoint{0.865430in}{1.570193in}}%
\pgfpathcurveto{\pgfqpoint{0.865430in}{1.561957in}}{\pgfqpoint{0.868702in}{1.554057in}}{\pgfqpoint{0.874526in}{1.548233in}}%
\pgfpathcurveto{\pgfqpoint{0.880350in}{1.542409in}}{\pgfqpoint{0.888250in}{1.539137in}}{\pgfqpoint{0.896486in}{1.539137in}}%
\pgfpathclose%
\pgfusepath{stroke,fill}%
\end{pgfscope}%
\begin{pgfscope}%
\pgfpathrectangle{\pgfqpoint{0.100000in}{0.212622in}}{\pgfqpoint{3.696000in}{3.696000in}}%
\pgfusepath{clip}%
\pgfsetbuttcap%
\pgfsetroundjoin%
\definecolor{currentfill}{rgb}{0.121569,0.466667,0.705882}%
\pgfsetfillcolor{currentfill}%
\pgfsetfillopacity{0.576892}%
\pgfsetlinewidth{1.003750pt}%
\definecolor{currentstroke}{rgb}{0.121569,0.466667,0.705882}%
\pgfsetstrokecolor{currentstroke}%
\pgfsetstrokeopacity{0.576892}%
\pgfsetdash{}{0pt}%
\pgfpathmoveto{\pgfqpoint{0.897794in}{1.536490in}}%
\pgfpathcurveto{\pgfqpoint{0.906031in}{1.536490in}}{\pgfqpoint{0.913931in}{1.539762in}}{\pgfqpoint{0.919755in}{1.545586in}}%
\pgfpathcurveto{\pgfqpoint{0.925579in}{1.551410in}}{\pgfqpoint{0.928851in}{1.559310in}}{\pgfqpoint{0.928851in}{1.567546in}}%
\pgfpathcurveto{\pgfqpoint{0.928851in}{1.575782in}}{\pgfqpoint{0.925579in}{1.583682in}}{\pgfqpoint{0.919755in}{1.589506in}}%
\pgfpathcurveto{\pgfqpoint{0.913931in}{1.595330in}}{\pgfqpoint{0.906031in}{1.598603in}}{\pgfqpoint{0.897794in}{1.598603in}}%
\pgfpathcurveto{\pgfqpoint{0.889558in}{1.598603in}}{\pgfqpoint{0.881658in}{1.595330in}}{\pgfqpoint{0.875834in}{1.589506in}}%
\pgfpathcurveto{\pgfqpoint{0.870010in}{1.583682in}}{\pgfqpoint{0.866738in}{1.575782in}}{\pgfqpoint{0.866738in}{1.567546in}}%
\pgfpathcurveto{\pgfqpoint{0.866738in}{1.559310in}}{\pgfqpoint{0.870010in}{1.551410in}}{\pgfqpoint{0.875834in}{1.545586in}}%
\pgfpathcurveto{\pgfqpoint{0.881658in}{1.539762in}}{\pgfqpoint{0.889558in}{1.536490in}}{\pgfqpoint{0.897794in}{1.536490in}}%
\pgfpathclose%
\pgfusepath{stroke,fill}%
\end{pgfscope}%
\begin{pgfscope}%
\pgfpathrectangle{\pgfqpoint{0.100000in}{0.212622in}}{\pgfqpoint{3.696000in}{3.696000in}}%
\pgfusepath{clip}%
\pgfsetbuttcap%
\pgfsetroundjoin%
\definecolor{currentfill}{rgb}{0.121569,0.466667,0.705882}%
\pgfsetfillcolor{currentfill}%
\pgfsetfillopacity{0.576933}%
\pgfsetlinewidth{1.003750pt}%
\definecolor{currentstroke}{rgb}{0.121569,0.466667,0.705882}%
\pgfsetstrokecolor{currentstroke}%
\pgfsetstrokeopacity{0.576933}%
\pgfsetdash{}{0pt}%
\pgfpathmoveto{\pgfqpoint{0.895742in}{1.540432in}}%
\pgfpathcurveto{\pgfqpoint{0.903978in}{1.540432in}}{\pgfqpoint{0.911878in}{1.543705in}}{\pgfqpoint{0.917702in}{1.549529in}}%
\pgfpathcurveto{\pgfqpoint{0.923526in}{1.555353in}}{\pgfqpoint{0.926798in}{1.563253in}}{\pgfqpoint{0.926798in}{1.571489in}}%
\pgfpathcurveto{\pgfqpoint{0.926798in}{1.579725in}}{\pgfqpoint{0.923526in}{1.587625in}}{\pgfqpoint{0.917702in}{1.593449in}}%
\pgfpathcurveto{\pgfqpoint{0.911878in}{1.599273in}}{\pgfqpoint{0.903978in}{1.602545in}}{\pgfqpoint{0.895742in}{1.602545in}}%
\pgfpathcurveto{\pgfqpoint{0.887506in}{1.602545in}}{\pgfqpoint{0.879606in}{1.599273in}}{\pgfqpoint{0.873782in}{1.593449in}}%
\pgfpathcurveto{\pgfqpoint{0.867958in}{1.587625in}}{\pgfqpoint{0.864685in}{1.579725in}}{\pgfqpoint{0.864685in}{1.571489in}}%
\pgfpathcurveto{\pgfqpoint{0.864685in}{1.563253in}}{\pgfqpoint{0.867958in}{1.555353in}}{\pgfqpoint{0.873782in}{1.549529in}}%
\pgfpathcurveto{\pgfqpoint{0.879606in}{1.543705in}}{\pgfqpoint{0.887506in}{1.540432in}}{\pgfqpoint{0.895742in}{1.540432in}}%
\pgfpathclose%
\pgfusepath{stroke,fill}%
\end{pgfscope}%
\begin{pgfscope}%
\pgfpathrectangle{\pgfqpoint{0.100000in}{0.212622in}}{\pgfqpoint{3.696000in}{3.696000in}}%
\pgfusepath{clip}%
\pgfsetbuttcap%
\pgfsetroundjoin%
\definecolor{currentfill}{rgb}{0.121569,0.466667,0.705882}%
\pgfsetfillcolor{currentfill}%
\pgfsetfillopacity{0.576995}%
\pgfsetlinewidth{1.003750pt}%
\definecolor{currentstroke}{rgb}{0.121569,0.466667,0.705882}%
\pgfsetstrokecolor{currentstroke}%
\pgfsetstrokeopacity{0.576995}%
\pgfsetdash{}{0pt}%
\pgfpathmoveto{\pgfqpoint{0.898463in}{1.534883in}}%
\pgfpathcurveto{\pgfqpoint{0.906700in}{1.534883in}}{\pgfqpoint{0.914600in}{1.538156in}}{\pgfqpoint{0.920424in}{1.543979in}}%
\pgfpathcurveto{\pgfqpoint{0.926248in}{1.549803in}}{\pgfqpoint{0.929520in}{1.557703in}}{\pgfqpoint{0.929520in}{1.565940in}}%
\pgfpathcurveto{\pgfqpoint{0.929520in}{1.574176in}}{\pgfqpoint{0.926248in}{1.582076in}}{\pgfqpoint{0.920424in}{1.587900in}}%
\pgfpathcurveto{\pgfqpoint{0.914600in}{1.593724in}}{\pgfqpoint{0.906700in}{1.596996in}}{\pgfqpoint{0.898463in}{1.596996in}}%
\pgfpathcurveto{\pgfqpoint{0.890227in}{1.596996in}}{\pgfqpoint{0.882327in}{1.593724in}}{\pgfqpoint{0.876503in}{1.587900in}}%
\pgfpathcurveto{\pgfqpoint{0.870679in}{1.582076in}}{\pgfqpoint{0.867407in}{1.574176in}}{\pgfqpoint{0.867407in}{1.565940in}}%
\pgfpathcurveto{\pgfqpoint{0.867407in}{1.557703in}}{\pgfqpoint{0.870679in}{1.549803in}}{\pgfqpoint{0.876503in}{1.543979in}}%
\pgfpathcurveto{\pgfqpoint{0.882327in}{1.538156in}}{\pgfqpoint{0.890227in}{1.534883in}}{\pgfqpoint{0.898463in}{1.534883in}}%
\pgfpathclose%
\pgfusepath{stroke,fill}%
\end{pgfscope}%
\begin{pgfscope}%
\pgfpathrectangle{\pgfqpoint{0.100000in}{0.212622in}}{\pgfqpoint{3.696000in}{3.696000in}}%
\pgfusepath{clip}%
\pgfsetbuttcap%
\pgfsetroundjoin%
\definecolor{currentfill}{rgb}{0.121569,0.466667,0.705882}%
\pgfsetfillcolor{currentfill}%
\pgfsetfillopacity{0.577078}%
\pgfsetlinewidth{1.003750pt}%
\definecolor{currentstroke}{rgb}{0.121569,0.466667,0.705882}%
\pgfsetstrokecolor{currentstroke}%
\pgfsetstrokeopacity{0.577078}%
\pgfsetdash{}{0pt}%
\pgfpathmoveto{\pgfqpoint{0.898783in}{1.534018in}}%
\pgfpathcurveto{\pgfqpoint{0.907020in}{1.534018in}}{\pgfqpoint{0.914920in}{1.537290in}}{\pgfqpoint{0.920744in}{1.543114in}}%
\pgfpathcurveto{\pgfqpoint{0.926568in}{1.548938in}}{\pgfqpoint{0.929840in}{1.556838in}}{\pgfqpoint{0.929840in}{1.565074in}}%
\pgfpathcurveto{\pgfqpoint{0.929840in}{1.573310in}}{\pgfqpoint{0.926568in}{1.581210in}}{\pgfqpoint{0.920744in}{1.587034in}}%
\pgfpathcurveto{\pgfqpoint{0.914920in}{1.592858in}}{\pgfqpoint{0.907020in}{1.596131in}}{\pgfqpoint{0.898783in}{1.596131in}}%
\pgfpathcurveto{\pgfqpoint{0.890547in}{1.596131in}}{\pgfqpoint{0.882647in}{1.592858in}}{\pgfqpoint{0.876823in}{1.587034in}}%
\pgfpathcurveto{\pgfqpoint{0.870999in}{1.581210in}}{\pgfqpoint{0.867727in}{1.573310in}}{\pgfqpoint{0.867727in}{1.565074in}}%
\pgfpathcurveto{\pgfqpoint{0.867727in}{1.556838in}}{\pgfqpoint{0.870999in}{1.548938in}}{\pgfqpoint{0.876823in}{1.543114in}}%
\pgfpathcurveto{\pgfqpoint{0.882647in}{1.537290in}}{\pgfqpoint{0.890547in}{1.534018in}}{\pgfqpoint{0.898783in}{1.534018in}}%
\pgfpathclose%
\pgfusepath{stroke,fill}%
\end{pgfscope}%
\begin{pgfscope}%
\pgfpathrectangle{\pgfqpoint{0.100000in}{0.212622in}}{\pgfqpoint{3.696000in}{3.696000in}}%
\pgfusepath{clip}%
\pgfsetbuttcap%
\pgfsetroundjoin%
\definecolor{currentfill}{rgb}{0.121569,0.466667,0.705882}%
\pgfsetfillcolor{currentfill}%
\pgfsetfillopacity{0.577132}%
\pgfsetlinewidth{1.003750pt}%
\definecolor{currentstroke}{rgb}{0.121569,0.466667,0.705882}%
\pgfsetstrokecolor{currentstroke}%
\pgfsetstrokeopacity{0.577132}%
\pgfsetdash{}{0pt}%
\pgfpathmoveto{\pgfqpoint{0.898936in}{1.533538in}}%
\pgfpathcurveto{\pgfqpoint{0.907173in}{1.533538in}}{\pgfqpoint{0.915073in}{1.536811in}}{\pgfqpoint{0.920897in}{1.542635in}}%
\pgfpathcurveto{\pgfqpoint{0.926720in}{1.548459in}}{\pgfqpoint{0.929993in}{1.556359in}}{\pgfqpoint{0.929993in}{1.564595in}}%
\pgfpathcurveto{\pgfqpoint{0.929993in}{1.572831in}}{\pgfqpoint{0.926720in}{1.580731in}}{\pgfqpoint{0.920897in}{1.586555in}}%
\pgfpathcurveto{\pgfqpoint{0.915073in}{1.592379in}}{\pgfqpoint{0.907173in}{1.595651in}}{\pgfqpoint{0.898936in}{1.595651in}}%
\pgfpathcurveto{\pgfqpoint{0.890700in}{1.595651in}}{\pgfqpoint{0.882800in}{1.592379in}}{\pgfqpoint{0.876976in}{1.586555in}}%
\pgfpathcurveto{\pgfqpoint{0.871152in}{1.580731in}}{\pgfqpoint{0.867880in}{1.572831in}}{\pgfqpoint{0.867880in}{1.564595in}}%
\pgfpathcurveto{\pgfqpoint{0.867880in}{1.556359in}}{\pgfqpoint{0.871152in}{1.548459in}}{\pgfqpoint{0.876976in}{1.542635in}}%
\pgfpathcurveto{\pgfqpoint{0.882800in}{1.536811in}}{\pgfqpoint{0.890700in}{1.533538in}}{\pgfqpoint{0.898936in}{1.533538in}}%
\pgfpathclose%
\pgfusepath{stroke,fill}%
\end{pgfscope}%
\begin{pgfscope}%
\pgfpathrectangle{\pgfqpoint{0.100000in}{0.212622in}}{\pgfqpoint{3.696000in}{3.696000in}}%
\pgfusepath{clip}%
\pgfsetbuttcap%
\pgfsetroundjoin%
\definecolor{currentfill}{rgb}{0.121569,0.466667,0.705882}%
\pgfsetfillcolor{currentfill}%
\pgfsetfillopacity{0.577143}%
\pgfsetlinewidth{1.003750pt}%
\definecolor{currentstroke}{rgb}{0.121569,0.466667,0.705882}%
\pgfsetstrokecolor{currentstroke}%
\pgfsetstrokeopacity{0.577143}%
\pgfsetdash{}{0pt}%
\pgfpathmoveto{\pgfqpoint{0.894254in}{1.542788in}}%
\pgfpathcurveto{\pgfqpoint{0.902491in}{1.542788in}}{\pgfqpoint{0.910391in}{1.546060in}}{\pgfqpoint{0.916215in}{1.551884in}}%
\pgfpathcurveto{\pgfqpoint{0.922039in}{1.557708in}}{\pgfqpoint{0.925311in}{1.565608in}}{\pgfqpoint{0.925311in}{1.573845in}}%
\pgfpathcurveto{\pgfqpoint{0.925311in}{1.582081in}}{\pgfqpoint{0.922039in}{1.589981in}}{\pgfqpoint{0.916215in}{1.595805in}}%
\pgfpathcurveto{\pgfqpoint{0.910391in}{1.601629in}}{\pgfqpoint{0.902491in}{1.604901in}}{\pgfqpoint{0.894254in}{1.604901in}}%
\pgfpathcurveto{\pgfqpoint{0.886018in}{1.604901in}}{\pgfqpoint{0.878118in}{1.601629in}}{\pgfqpoint{0.872294in}{1.595805in}}%
\pgfpathcurveto{\pgfqpoint{0.866470in}{1.589981in}}{\pgfqpoint{0.863198in}{1.582081in}}{\pgfqpoint{0.863198in}{1.573845in}}%
\pgfpathcurveto{\pgfqpoint{0.863198in}{1.565608in}}{\pgfqpoint{0.866470in}{1.557708in}}{\pgfqpoint{0.872294in}{1.551884in}}%
\pgfpathcurveto{\pgfqpoint{0.878118in}{1.546060in}}{\pgfqpoint{0.886018in}{1.542788in}}{\pgfqpoint{0.894254in}{1.542788in}}%
\pgfpathclose%
\pgfusepath{stroke,fill}%
\end{pgfscope}%
\begin{pgfscope}%
\pgfpathrectangle{\pgfqpoint{0.100000in}{0.212622in}}{\pgfqpoint{3.696000in}{3.696000in}}%
\pgfusepath{clip}%
\pgfsetbuttcap%
\pgfsetroundjoin%
\definecolor{currentfill}{rgb}{0.121569,0.466667,0.705882}%
\pgfsetfillcolor{currentfill}%
\pgfsetfillopacity{0.577168}%
\pgfsetlinewidth{1.003750pt}%
\definecolor{currentstroke}{rgb}{0.121569,0.466667,0.705882}%
\pgfsetstrokecolor{currentstroke}%
\pgfsetstrokeopacity{0.577168}%
\pgfsetdash{}{0pt}%
\pgfpathmoveto{\pgfqpoint{0.899008in}{1.533278in}}%
\pgfpathcurveto{\pgfqpoint{0.907245in}{1.533278in}}{\pgfqpoint{0.915145in}{1.536550in}}{\pgfqpoint{0.920969in}{1.542374in}}%
\pgfpathcurveto{\pgfqpoint{0.926792in}{1.548198in}}{\pgfqpoint{0.930065in}{1.556098in}}{\pgfqpoint{0.930065in}{1.564334in}}%
\pgfpathcurveto{\pgfqpoint{0.930065in}{1.572571in}}{\pgfqpoint{0.926792in}{1.580471in}}{\pgfqpoint{0.920969in}{1.586295in}}%
\pgfpathcurveto{\pgfqpoint{0.915145in}{1.592119in}}{\pgfqpoint{0.907245in}{1.595391in}}{\pgfqpoint{0.899008in}{1.595391in}}%
\pgfpathcurveto{\pgfqpoint{0.890772in}{1.595391in}}{\pgfqpoint{0.882872in}{1.592119in}}{\pgfqpoint{0.877048in}{1.586295in}}%
\pgfpathcurveto{\pgfqpoint{0.871224in}{1.580471in}}{\pgfqpoint{0.867952in}{1.572571in}}{\pgfqpoint{0.867952in}{1.564334in}}%
\pgfpathcurveto{\pgfqpoint{0.867952in}{1.556098in}}{\pgfqpoint{0.871224in}{1.548198in}}{\pgfqpoint{0.877048in}{1.542374in}}%
\pgfpathcurveto{\pgfqpoint{0.882872in}{1.536550in}}{\pgfqpoint{0.890772in}{1.533278in}}{\pgfqpoint{0.899008in}{1.533278in}}%
\pgfpathclose%
\pgfusepath{stroke,fill}%
\end{pgfscope}%
\begin{pgfscope}%
\pgfpathrectangle{\pgfqpoint{0.100000in}{0.212622in}}{\pgfqpoint{3.696000in}{3.696000in}}%
\pgfusepath{clip}%
\pgfsetbuttcap%
\pgfsetroundjoin%
\definecolor{currentfill}{rgb}{0.121569,0.466667,0.705882}%
\pgfsetfillcolor{currentfill}%
\pgfsetfillopacity{0.577189}%
\pgfsetlinewidth{1.003750pt}%
\definecolor{currentstroke}{rgb}{0.121569,0.466667,0.705882}%
\pgfsetstrokecolor{currentstroke}%
\pgfsetstrokeopacity{0.577189}%
\pgfsetdash{}{0pt}%
\pgfpathmoveto{\pgfqpoint{0.899041in}{1.533134in}}%
\pgfpathcurveto{\pgfqpoint{0.907277in}{1.533134in}}{\pgfqpoint{0.915177in}{1.536406in}}{\pgfqpoint{0.921001in}{1.542230in}}%
\pgfpathcurveto{\pgfqpoint{0.926825in}{1.548054in}}{\pgfqpoint{0.930097in}{1.555954in}}{\pgfqpoint{0.930097in}{1.564191in}}%
\pgfpathcurveto{\pgfqpoint{0.930097in}{1.572427in}}{\pgfqpoint{0.926825in}{1.580327in}}{\pgfqpoint{0.921001in}{1.586151in}}%
\pgfpathcurveto{\pgfqpoint{0.915177in}{1.591975in}}{\pgfqpoint{0.907277in}{1.595247in}}{\pgfqpoint{0.899041in}{1.595247in}}%
\pgfpathcurveto{\pgfqpoint{0.890804in}{1.595247in}}{\pgfqpoint{0.882904in}{1.591975in}}{\pgfqpoint{0.877080in}{1.586151in}}%
\pgfpathcurveto{\pgfqpoint{0.871257in}{1.580327in}}{\pgfqpoint{0.867984in}{1.572427in}}{\pgfqpoint{0.867984in}{1.564191in}}%
\pgfpathcurveto{\pgfqpoint{0.867984in}{1.555954in}}{\pgfqpoint{0.871257in}{1.548054in}}{\pgfqpoint{0.877080in}{1.542230in}}%
\pgfpathcurveto{\pgfqpoint{0.882904in}{1.536406in}}{\pgfqpoint{0.890804in}{1.533134in}}{\pgfqpoint{0.899041in}{1.533134in}}%
\pgfpathclose%
\pgfusepath{stroke,fill}%
\end{pgfscope}%
\begin{pgfscope}%
\pgfpathrectangle{\pgfqpoint{0.100000in}{0.212622in}}{\pgfqpoint{3.696000in}{3.696000in}}%
\pgfusepath{clip}%
\pgfsetbuttcap%
\pgfsetroundjoin%
\definecolor{currentfill}{rgb}{0.121569,0.466667,0.705882}%
\pgfsetfillcolor{currentfill}%
\pgfsetfillopacity{0.577202}%
\pgfsetlinewidth{1.003750pt}%
\definecolor{currentstroke}{rgb}{0.121569,0.466667,0.705882}%
\pgfsetstrokecolor{currentstroke}%
\pgfsetstrokeopacity{0.577202}%
\pgfsetdash{}{0pt}%
\pgfpathmoveto{\pgfqpoint{0.899055in}{1.533055in}}%
\pgfpathcurveto{\pgfqpoint{0.907291in}{1.533055in}}{\pgfqpoint{0.915191in}{1.536328in}}{\pgfqpoint{0.921015in}{1.542152in}}%
\pgfpathcurveto{\pgfqpoint{0.926839in}{1.547976in}}{\pgfqpoint{0.930112in}{1.555876in}}{\pgfqpoint{0.930112in}{1.564112in}}%
\pgfpathcurveto{\pgfqpoint{0.930112in}{1.572348in}}{\pgfqpoint{0.926839in}{1.580248in}}{\pgfqpoint{0.921015in}{1.586072in}}%
\pgfpathcurveto{\pgfqpoint{0.915191in}{1.591896in}}{\pgfqpoint{0.907291in}{1.595168in}}{\pgfqpoint{0.899055in}{1.595168in}}%
\pgfpathcurveto{\pgfqpoint{0.890819in}{1.595168in}}{\pgfqpoint{0.882919in}{1.591896in}}{\pgfqpoint{0.877095in}{1.586072in}}%
\pgfpathcurveto{\pgfqpoint{0.871271in}{1.580248in}}{\pgfqpoint{0.867999in}{1.572348in}}{\pgfqpoint{0.867999in}{1.564112in}}%
\pgfpathcurveto{\pgfqpoint{0.867999in}{1.555876in}}{\pgfqpoint{0.871271in}{1.547976in}}{\pgfqpoint{0.877095in}{1.542152in}}%
\pgfpathcurveto{\pgfqpoint{0.882919in}{1.536328in}}{\pgfqpoint{0.890819in}{1.533055in}}{\pgfqpoint{0.899055in}{1.533055in}}%
\pgfpathclose%
\pgfusepath{stroke,fill}%
\end{pgfscope}%
\begin{pgfscope}%
\pgfpathrectangle{\pgfqpoint{0.100000in}{0.212622in}}{\pgfqpoint{3.696000in}{3.696000in}}%
\pgfusepath{clip}%
\pgfsetbuttcap%
\pgfsetroundjoin%
\definecolor{currentfill}{rgb}{0.121569,0.466667,0.705882}%
\pgfsetfillcolor{currentfill}%
\pgfsetfillopacity{0.577205}%
\pgfsetlinewidth{1.003750pt}%
\definecolor{currentstroke}{rgb}{0.121569,0.466667,0.705882}%
\pgfsetstrokecolor{currentstroke}%
\pgfsetstrokeopacity{0.577205}%
\pgfsetdash{}{0pt}%
\pgfpathmoveto{\pgfqpoint{1.021393in}{1.795848in}}%
\pgfpathcurveto{\pgfqpoint{1.029629in}{1.795848in}}{\pgfqpoint{1.037529in}{1.799120in}}{\pgfqpoint{1.043353in}{1.804944in}}%
\pgfpathcurveto{\pgfqpoint{1.049177in}{1.810768in}}{\pgfqpoint{1.052449in}{1.818668in}}{\pgfqpoint{1.052449in}{1.826904in}}%
\pgfpathcurveto{\pgfqpoint{1.052449in}{1.835140in}}{\pgfqpoint{1.049177in}{1.843041in}}{\pgfqpoint{1.043353in}{1.848864in}}%
\pgfpathcurveto{\pgfqpoint{1.037529in}{1.854688in}}{\pgfqpoint{1.029629in}{1.857961in}}{\pgfqpoint{1.021393in}{1.857961in}}%
\pgfpathcurveto{\pgfqpoint{1.013156in}{1.857961in}}{\pgfqpoint{1.005256in}{1.854688in}}{\pgfqpoint{0.999432in}{1.848864in}}%
\pgfpathcurveto{\pgfqpoint{0.993608in}{1.843041in}}{\pgfqpoint{0.990336in}{1.835140in}}{\pgfqpoint{0.990336in}{1.826904in}}%
\pgfpathcurveto{\pgfqpoint{0.990336in}{1.818668in}}{\pgfqpoint{0.993608in}{1.810768in}}{\pgfqpoint{0.999432in}{1.804944in}}%
\pgfpathcurveto{\pgfqpoint{1.005256in}{1.799120in}}{\pgfqpoint{1.013156in}{1.795848in}}{\pgfqpoint{1.021393in}{1.795848in}}%
\pgfpathclose%
\pgfusepath{stroke,fill}%
\end{pgfscope}%
\begin{pgfscope}%
\pgfpathrectangle{\pgfqpoint{0.100000in}{0.212622in}}{\pgfqpoint{3.696000in}{3.696000in}}%
\pgfusepath{clip}%
\pgfsetbuttcap%
\pgfsetroundjoin%
\definecolor{currentfill}{rgb}{0.121569,0.466667,0.705882}%
\pgfsetfillcolor{currentfill}%
\pgfsetfillopacity{0.577210}%
\pgfsetlinewidth{1.003750pt}%
\definecolor{currentstroke}{rgb}{0.121569,0.466667,0.705882}%
\pgfsetstrokecolor{currentstroke}%
\pgfsetstrokeopacity{0.577210}%
\pgfsetdash{}{0pt}%
\pgfpathmoveto{\pgfqpoint{0.899061in}{1.533013in}}%
\pgfpathcurveto{\pgfqpoint{0.907297in}{1.533013in}}{\pgfqpoint{0.915198in}{1.536285in}}{\pgfqpoint{0.921021in}{1.542109in}}%
\pgfpathcurveto{\pgfqpoint{0.926845in}{1.547933in}}{\pgfqpoint{0.930118in}{1.555833in}}{\pgfqpoint{0.930118in}{1.564069in}}%
\pgfpathcurveto{\pgfqpoint{0.930118in}{1.572306in}}{\pgfqpoint{0.926845in}{1.580206in}}{\pgfqpoint{0.921021in}{1.586030in}}%
\pgfpathcurveto{\pgfqpoint{0.915198in}{1.591854in}}{\pgfqpoint{0.907297in}{1.595126in}}{\pgfqpoint{0.899061in}{1.595126in}}%
\pgfpathcurveto{\pgfqpoint{0.890825in}{1.595126in}}{\pgfqpoint{0.882925in}{1.591854in}}{\pgfqpoint{0.877101in}{1.586030in}}%
\pgfpathcurveto{\pgfqpoint{0.871277in}{1.580206in}}{\pgfqpoint{0.868005in}{1.572306in}}{\pgfqpoint{0.868005in}{1.564069in}}%
\pgfpathcurveto{\pgfqpoint{0.868005in}{1.555833in}}{\pgfqpoint{0.871277in}{1.547933in}}{\pgfqpoint{0.877101in}{1.542109in}}%
\pgfpathcurveto{\pgfqpoint{0.882925in}{1.536285in}}{\pgfqpoint{0.890825in}{1.533013in}}{\pgfqpoint{0.899061in}{1.533013in}}%
\pgfpathclose%
\pgfusepath{stroke,fill}%
\end{pgfscope}%
\begin{pgfscope}%
\pgfpathrectangle{\pgfqpoint{0.100000in}{0.212622in}}{\pgfqpoint{3.696000in}{3.696000in}}%
\pgfusepath{clip}%
\pgfsetbuttcap%
\pgfsetroundjoin%
\definecolor{currentfill}{rgb}{0.121569,0.466667,0.705882}%
\pgfsetfillcolor{currentfill}%
\pgfsetfillopacity{0.577215}%
\pgfsetlinewidth{1.003750pt}%
\definecolor{currentstroke}{rgb}{0.121569,0.466667,0.705882}%
\pgfsetstrokecolor{currentstroke}%
\pgfsetstrokeopacity{0.577215}%
\pgfsetdash{}{0pt}%
\pgfpathmoveto{\pgfqpoint{0.899064in}{1.532990in}}%
\pgfpathcurveto{\pgfqpoint{0.907300in}{1.532990in}}{\pgfqpoint{0.915200in}{1.536262in}}{\pgfqpoint{0.921024in}{1.542086in}}%
\pgfpathcurveto{\pgfqpoint{0.926848in}{1.547910in}}{\pgfqpoint{0.930120in}{1.555810in}}{\pgfqpoint{0.930120in}{1.564046in}}%
\pgfpathcurveto{\pgfqpoint{0.930120in}{1.572282in}}{\pgfqpoint{0.926848in}{1.580182in}}{\pgfqpoint{0.921024in}{1.586006in}}%
\pgfpathcurveto{\pgfqpoint{0.915200in}{1.591830in}}{\pgfqpoint{0.907300in}{1.595103in}}{\pgfqpoint{0.899064in}{1.595103in}}%
\pgfpathcurveto{\pgfqpoint{0.890827in}{1.595103in}}{\pgfqpoint{0.882927in}{1.591830in}}{\pgfqpoint{0.877103in}{1.586006in}}%
\pgfpathcurveto{\pgfqpoint{0.871279in}{1.580182in}}{\pgfqpoint{0.868007in}{1.572282in}}{\pgfqpoint{0.868007in}{1.564046in}}%
\pgfpathcurveto{\pgfqpoint{0.868007in}{1.555810in}}{\pgfqpoint{0.871279in}{1.547910in}}{\pgfqpoint{0.877103in}{1.542086in}}%
\pgfpathcurveto{\pgfqpoint{0.882927in}{1.536262in}}{\pgfqpoint{0.890827in}{1.532990in}}{\pgfqpoint{0.899064in}{1.532990in}}%
\pgfpathclose%
\pgfusepath{stroke,fill}%
\end{pgfscope}%
\begin{pgfscope}%
\pgfpathrectangle{\pgfqpoint{0.100000in}{0.212622in}}{\pgfqpoint{3.696000in}{3.696000in}}%
\pgfusepath{clip}%
\pgfsetbuttcap%
\pgfsetroundjoin%
\definecolor{currentfill}{rgb}{0.121569,0.466667,0.705882}%
\pgfsetfillcolor{currentfill}%
\pgfsetfillopacity{0.577217}%
\pgfsetlinewidth{1.003750pt}%
\definecolor{currentstroke}{rgb}{0.121569,0.466667,0.705882}%
\pgfsetstrokecolor{currentstroke}%
\pgfsetstrokeopacity{0.577217}%
\pgfsetdash{}{0pt}%
\pgfpathmoveto{\pgfqpoint{0.899065in}{1.532976in}}%
\pgfpathcurveto{\pgfqpoint{0.907301in}{1.532976in}}{\pgfqpoint{0.915201in}{1.536249in}}{\pgfqpoint{0.921025in}{1.542073in}}%
\pgfpathcurveto{\pgfqpoint{0.926849in}{1.547897in}}{\pgfqpoint{0.930121in}{1.555797in}}{\pgfqpoint{0.930121in}{1.564033in}}%
\pgfpathcurveto{\pgfqpoint{0.930121in}{1.572269in}}{\pgfqpoint{0.926849in}{1.580169in}}{\pgfqpoint{0.921025in}{1.585993in}}%
\pgfpathcurveto{\pgfqpoint{0.915201in}{1.591817in}}{\pgfqpoint{0.907301in}{1.595089in}}{\pgfqpoint{0.899065in}{1.595089in}}%
\pgfpathcurveto{\pgfqpoint{0.890828in}{1.595089in}}{\pgfqpoint{0.882928in}{1.591817in}}{\pgfqpoint{0.877104in}{1.585993in}}%
\pgfpathcurveto{\pgfqpoint{0.871280in}{1.580169in}}{\pgfqpoint{0.868008in}{1.572269in}}{\pgfqpoint{0.868008in}{1.564033in}}%
\pgfpathcurveto{\pgfqpoint{0.868008in}{1.555797in}}{\pgfqpoint{0.871280in}{1.547897in}}{\pgfqpoint{0.877104in}{1.542073in}}%
\pgfpathcurveto{\pgfqpoint{0.882928in}{1.536249in}}{\pgfqpoint{0.890828in}{1.532976in}}{\pgfqpoint{0.899065in}{1.532976in}}%
\pgfpathclose%
\pgfusepath{stroke,fill}%
\end{pgfscope}%
\begin{pgfscope}%
\pgfpathrectangle{\pgfqpoint{0.100000in}{0.212622in}}{\pgfqpoint{3.696000in}{3.696000in}}%
\pgfusepath{clip}%
\pgfsetbuttcap%
\pgfsetroundjoin%
\definecolor{currentfill}{rgb}{0.121569,0.466667,0.705882}%
\pgfsetfillcolor{currentfill}%
\pgfsetfillopacity{0.577339}%
\pgfsetlinewidth{1.003750pt}%
\definecolor{currentstroke}{rgb}{0.121569,0.466667,0.705882}%
\pgfsetstrokecolor{currentstroke}%
\pgfsetstrokeopacity{0.577339}%
\pgfsetdash{}{0pt}%
\pgfpathmoveto{\pgfqpoint{0.899089in}{1.532341in}}%
\pgfpathcurveto{\pgfqpoint{0.907325in}{1.532341in}}{\pgfqpoint{0.915225in}{1.535614in}}{\pgfqpoint{0.921049in}{1.541438in}}%
\pgfpathcurveto{\pgfqpoint{0.926873in}{1.547262in}}{\pgfqpoint{0.930146in}{1.555162in}}{\pgfqpoint{0.930146in}{1.563398in}}%
\pgfpathcurveto{\pgfqpoint{0.930146in}{1.571634in}}{\pgfqpoint{0.926873in}{1.579534in}}{\pgfqpoint{0.921049in}{1.585358in}}%
\pgfpathcurveto{\pgfqpoint{0.915225in}{1.591182in}}{\pgfqpoint{0.907325in}{1.594454in}}{\pgfqpoint{0.899089in}{1.594454in}}%
\pgfpathcurveto{\pgfqpoint{0.890853in}{1.594454in}}{\pgfqpoint{0.882953in}{1.591182in}}{\pgfqpoint{0.877129in}{1.585358in}}%
\pgfpathcurveto{\pgfqpoint{0.871305in}{1.579534in}}{\pgfqpoint{0.868033in}{1.571634in}}{\pgfqpoint{0.868033in}{1.563398in}}%
\pgfpathcurveto{\pgfqpoint{0.868033in}{1.555162in}}{\pgfqpoint{0.871305in}{1.547262in}}{\pgfqpoint{0.877129in}{1.541438in}}%
\pgfpathcurveto{\pgfqpoint{0.882953in}{1.535614in}}{\pgfqpoint{0.890853in}{1.532341in}}{\pgfqpoint{0.899089in}{1.532341in}}%
\pgfpathclose%
\pgfusepath{stroke,fill}%
\end{pgfscope}%
\begin{pgfscope}%
\pgfpathrectangle{\pgfqpoint{0.100000in}{0.212622in}}{\pgfqpoint{3.696000in}{3.696000in}}%
\pgfusepath{clip}%
\pgfsetbuttcap%
\pgfsetroundjoin%
\definecolor{currentfill}{rgb}{0.121569,0.466667,0.705882}%
\pgfsetfillcolor{currentfill}%
\pgfsetfillopacity{0.577379}%
\pgfsetlinewidth{1.003750pt}%
\definecolor{currentstroke}{rgb}{0.121569,0.466667,0.705882}%
\pgfsetstrokecolor{currentstroke}%
\pgfsetstrokeopacity{0.577379}%
\pgfsetdash{}{0pt}%
\pgfpathmoveto{\pgfqpoint{1.020772in}{1.794675in}}%
\pgfpathcurveto{\pgfqpoint{1.029008in}{1.794675in}}{\pgfqpoint{1.036908in}{1.797948in}}{\pgfqpoint{1.042732in}{1.803772in}}%
\pgfpathcurveto{\pgfqpoint{1.048556in}{1.809596in}}{\pgfqpoint{1.051829in}{1.817496in}}{\pgfqpoint{1.051829in}{1.825732in}}%
\pgfpathcurveto{\pgfqpoint{1.051829in}{1.833968in}}{\pgfqpoint{1.048556in}{1.841868in}}{\pgfqpoint{1.042732in}{1.847692in}}%
\pgfpathcurveto{\pgfqpoint{1.036908in}{1.853516in}}{\pgfqpoint{1.029008in}{1.856788in}}{\pgfqpoint{1.020772in}{1.856788in}}%
\pgfpathcurveto{\pgfqpoint{1.012536in}{1.856788in}}{\pgfqpoint{1.004636in}{1.853516in}}{\pgfqpoint{0.998812in}{1.847692in}}%
\pgfpathcurveto{\pgfqpoint{0.992988in}{1.841868in}}{\pgfqpoint{0.989716in}{1.833968in}}{\pgfqpoint{0.989716in}{1.825732in}}%
\pgfpathcurveto{\pgfqpoint{0.989716in}{1.817496in}}{\pgfqpoint{0.992988in}{1.809596in}}{\pgfqpoint{0.998812in}{1.803772in}}%
\pgfpathcurveto{\pgfqpoint{1.004636in}{1.797948in}}{\pgfqpoint{1.012536in}{1.794675in}}{\pgfqpoint{1.020772in}{1.794675in}}%
\pgfpathclose%
\pgfusepath{stroke,fill}%
\end{pgfscope}%
\begin{pgfscope}%
\pgfpathrectangle{\pgfqpoint{0.100000in}{0.212622in}}{\pgfqpoint{3.696000in}{3.696000in}}%
\pgfusepath{clip}%
\pgfsetbuttcap%
\pgfsetroundjoin%
\definecolor{currentfill}{rgb}{0.121569,0.466667,0.705882}%
\pgfsetfillcolor{currentfill}%
\pgfsetfillopacity{0.577408}%
\pgfsetlinewidth{1.003750pt}%
\definecolor{currentstroke}{rgb}{0.121569,0.466667,0.705882}%
\pgfsetstrokecolor{currentstroke}%
\pgfsetstrokeopacity{0.577408}%
\pgfsetdash{}{0pt}%
\pgfpathmoveto{\pgfqpoint{0.899091in}{1.531993in}}%
\pgfpathcurveto{\pgfqpoint{0.907327in}{1.531993in}}{\pgfqpoint{0.915227in}{1.535265in}}{\pgfqpoint{0.921051in}{1.541089in}}%
\pgfpathcurveto{\pgfqpoint{0.926875in}{1.546913in}}{\pgfqpoint{0.930147in}{1.554813in}}{\pgfqpoint{0.930147in}{1.563049in}}%
\pgfpathcurveto{\pgfqpoint{0.930147in}{1.571285in}}{\pgfqpoint{0.926875in}{1.579185in}}{\pgfqpoint{0.921051in}{1.585009in}}%
\pgfpathcurveto{\pgfqpoint{0.915227in}{1.590833in}}{\pgfqpoint{0.907327in}{1.594106in}}{\pgfqpoint{0.899091in}{1.594106in}}%
\pgfpathcurveto{\pgfqpoint{0.890855in}{1.594106in}}{\pgfqpoint{0.882955in}{1.590833in}}{\pgfqpoint{0.877131in}{1.585009in}}%
\pgfpathcurveto{\pgfqpoint{0.871307in}{1.579185in}}{\pgfqpoint{0.868034in}{1.571285in}}{\pgfqpoint{0.868034in}{1.563049in}}%
\pgfpathcurveto{\pgfqpoint{0.868034in}{1.554813in}}{\pgfqpoint{0.871307in}{1.546913in}}{\pgfqpoint{0.877131in}{1.541089in}}%
\pgfpathcurveto{\pgfqpoint{0.882955in}{1.535265in}}{\pgfqpoint{0.890855in}{1.531993in}}{\pgfqpoint{0.899091in}{1.531993in}}%
\pgfpathclose%
\pgfusepath{stroke,fill}%
\end{pgfscope}%
\begin{pgfscope}%
\pgfpathrectangle{\pgfqpoint{0.100000in}{0.212622in}}{\pgfqpoint{3.696000in}{3.696000in}}%
\pgfusepath{clip}%
\pgfsetbuttcap%
\pgfsetroundjoin%
\definecolor{currentfill}{rgb}{0.121569,0.466667,0.705882}%
\pgfsetfillcolor{currentfill}%
\pgfsetfillopacity{0.577458}%
\pgfsetlinewidth{1.003750pt}%
\definecolor{currentstroke}{rgb}{0.121569,0.466667,0.705882}%
\pgfsetstrokecolor{currentstroke}%
\pgfsetstrokeopacity{0.577458}%
\pgfsetdash{}{0pt}%
\pgfpathmoveto{\pgfqpoint{1.020617in}{1.794122in}}%
\pgfpathcurveto{\pgfqpoint{1.028853in}{1.794122in}}{\pgfqpoint{1.036753in}{1.797394in}}{\pgfqpoint{1.042577in}{1.803218in}}%
\pgfpathcurveto{\pgfqpoint{1.048401in}{1.809042in}}{\pgfqpoint{1.051673in}{1.816942in}}{\pgfqpoint{1.051673in}{1.825179in}}%
\pgfpathcurveto{\pgfqpoint{1.051673in}{1.833415in}}{\pgfqpoint{1.048401in}{1.841315in}}{\pgfqpoint{1.042577in}{1.847139in}}%
\pgfpathcurveto{\pgfqpoint{1.036753in}{1.852963in}}{\pgfqpoint{1.028853in}{1.856235in}}{\pgfqpoint{1.020617in}{1.856235in}}%
\pgfpathcurveto{\pgfqpoint{1.012380in}{1.856235in}}{\pgfqpoint{1.004480in}{1.852963in}}{\pgfqpoint{0.998656in}{1.847139in}}%
\pgfpathcurveto{\pgfqpoint{0.992832in}{1.841315in}}{\pgfqpoint{0.989560in}{1.833415in}}{\pgfqpoint{0.989560in}{1.825179in}}%
\pgfpathcurveto{\pgfqpoint{0.989560in}{1.816942in}}{\pgfqpoint{0.992832in}{1.809042in}}{\pgfqpoint{0.998656in}{1.803218in}}%
\pgfpathcurveto{\pgfqpoint{1.004480in}{1.797394in}}{\pgfqpoint{1.012380in}{1.794122in}}{\pgfqpoint{1.020617in}{1.794122in}}%
\pgfpathclose%
\pgfusepath{stroke,fill}%
\end{pgfscope}%
\begin{pgfscope}%
\pgfpathrectangle{\pgfqpoint{0.100000in}{0.212622in}}{\pgfqpoint{3.696000in}{3.696000in}}%
\pgfusepath{clip}%
\pgfsetbuttcap%
\pgfsetroundjoin%
\definecolor{currentfill}{rgb}{0.121569,0.466667,0.705882}%
\pgfsetfillcolor{currentfill}%
\pgfsetfillopacity{0.577563}%
\pgfsetlinewidth{1.003750pt}%
\definecolor{currentstroke}{rgb}{0.121569,0.466667,0.705882}%
\pgfsetstrokecolor{currentstroke}%
\pgfsetstrokeopacity{0.577563}%
\pgfsetdash{}{0pt}%
\pgfpathmoveto{\pgfqpoint{0.899076in}{1.531211in}}%
\pgfpathcurveto{\pgfqpoint{0.907312in}{1.531211in}}{\pgfqpoint{0.915212in}{1.534484in}}{\pgfqpoint{0.921036in}{1.540308in}}%
\pgfpathcurveto{\pgfqpoint{0.926860in}{1.546132in}}{\pgfqpoint{0.930133in}{1.554032in}}{\pgfqpoint{0.930133in}{1.562268in}}%
\pgfpathcurveto{\pgfqpoint{0.930133in}{1.570504in}}{\pgfqpoint{0.926860in}{1.578404in}}{\pgfqpoint{0.921036in}{1.584228in}}%
\pgfpathcurveto{\pgfqpoint{0.915212in}{1.590052in}}{\pgfqpoint{0.907312in}{1.593324in}}{\pgfqpoint{0.899076in}{1.593324in}}%
\pgfpathcurveto{\pgfqpoint{0.890840in}{1.593324in}}{\pgfqpoint{0.882940in}{1.590052in}}{\pgfqpoint{0.877116in}{1.584228in}}%
\pgfpathcurveto{\pgfqpoint{0.871292in}{1.578404in}}{\pgfqpoint{0.868020in}{1.570504in}}{\pgfqpoint{0.868020in}{1.562268in}}%
\pgfpathcurveto{\pgfqpoint{0.868020in}{1.554032in}}{\pgfqpoint{0.871292in}{1.546132in}}{\pgfqpoint{0.877116in}{1.540308in}}%
\pgfpathcurveto{\pgfqpoint{0.882940in}{1.534484in}}{\pgfqpoint{0.890840in}{1.531211in}}{\pgfqpoint{0.899076in}{1.531211in}}%
\pgfpathclose%
\pgfusepath{stroke,fill}%
\end{pgfscope}%
\begin{pgfscope}%
\pgfpathrectangle{\pgfqpoint{0.100000in}{0.212622in}}{\pgfqpoint{3.696000in}{3.696000in}}%
\pgfusepath{clip}%
\pgfsetbuttcap%
\pgfsetroundjoin%
\definecolor{currentfill}{rgb}{0.121569,0.466667,0.705882}%
\pgfsetfillcolor{currentfill}%
\pgfsetfillopacity{0.577618}%
\pgfsetlinewidth{1.003750pt}%
\definecolor{currentstroke}{rgb}{0.121569,0.466667,0.705882}%
\pgfsetstrokecolor{currentstroke}%
\pgfsetstrokeopacity{0.577618}%
\pgfsetdash{}{0pt}%
\pgfpathmoveto{\pgfqpoint{1.020149in}{1.793331in}}%
\pgfpathcurveto{\pgfqpoint{1.028385in}{1.793331in}}{\pgfqpoint{1.036285in}{1.796604in}}{\pgfqpoint{1.042109in}{1.802428in}}%
\pgfpathcurveto{\pgfqpoint{1.047933in}{1.808251in}}{\pgfqpoint{1.051205in}{1.816152in}}{\pgfqpoint{1.051205in}{1.824388in}}%
\pgfpathcurveto{\pgfqpoint{1.051205in}{1.832624in}}{\pgfqpoint{1.047933in}{1.840524in}}{\pgfqpoint{1.042109in}{1.846348in}}%
\pgfpathcurveto{\pgfqpoint{1.036285in}{1.852172in}}{\pgfqpoint{1.028385in}{1.855444in}}{\pgfqpoint{1.020149in}{1.855444in}}%
\pgfpathcurveto{\pgfqpoint{1.011912in}{1.855444in}}{\pgfqpoint{1.004012in}{1.852172in}}{\pgfqpoint{0.998188in}{1.846348in}}%
\pgfpathcurveto{\pgfqpoint{0.992364in}{1.840524in}}{\pgfqpoint{0.989092in}{1.832624in}}{\pgfqpoint{0.989092in}{1.824388in}}%
\pgfpathcurveto{\pgfqpoint{0.989092in}{1.816152in}}{\pgfqpoint{0.992364in}{1.808251in}}{\pgfqpoint{0.998188in}{1.802428in}}%
\pgfpathcurveto{\pgfqpoint{1.004012in}{1.796604in}}{\pgfqpoint{1.011912in}{1.793331in}}{\pgfqpoint{1.020149in}{1.793331in}}%
\pgfpathclose%
\pgfusepath{stroke,fill}%
\end{pgfscope}%
\begin{pgfscope}%
\pgfpathrectangle{\pgfqpoint{0.100000in}{0.212622in}}{\pgfqpoint{3.696000in}{3.696000in}}%
\pgfusepath{clip}%
\pgfsetbuttcap%
\pgfsetroundjoin%
\definecolor{currentfill}{rgb}{0.121569,0.466667,0.705882}%
\pgfsetfillcolor{currentfill}%
\pgfsetfillopacity{0.577730}%
\pgfsetlinewidth{1.003750pt}%
\definecolor{currentstroke}{rgb}{0.121569,0.466667,0.705882}%
\pgfsetstrokecolor{currentstroke}%
\pgfsetstrokeopacity{0.577730}%
\pgfsetdash{}{0pt}%
\pgfpathmoveto{\pgfqpoint{0.891326in}{1.547034in}}%
\pgfpathcurveto{\pgfqpoint{0.899563in}{1.547034in}}{\pgfqpoint{0.907463in}{1.550306in}}{\pgfqpoint{0.913287in}{1.556130in}}%
\pgfpathcurveto{\pgfqpoint{0.919111in}{1.561954in}}{\pgfqpoint{0.922383in}{1.569854in}}{\pgfqpoint{0.922383in}{1.578091in}}%
\pgfpathcurveto{\pgfqpoint{0.922383in}{1.586327in}}{\pgfqpoint{0.919111in}{1.594227in}}{\pgfqpoint{0.913287in}{1.600051in}}%
\pgfpathcurveto{\pgfqpoint{0.907463in}{1.605875in}}{\pgfqpoint{0.899563in}{1.609147in}}{\pgfqpoint{0.891326in}{1.609147in}}%
\pgfpathcurveto{\pgfqpoint{0.883090in}{1.609147in}}{\pgfqpoint{0.875190in}{1.605875in}}{\pgfqpoint{0.869366in}{1.600051in}}%
\pgfpathcurveto{\pgfqpoint{0.863542in}{1.594227in}}{\pgfqpoint{0.860270in}{1.586327in}}{\pgfqpoint{0.860270in}{1.578091in}}%
\pgfpathcurveto{\pgfqpoint{0.860270in}{1.569854in}}{\pgfqpoint{0.863542in}{1.561954in}}{\pgfqpoint{0.869366in}{1.556130in}}%
\pgfpathcurveto{\pgfqpoint{0.875190in}{1.550306in}}{\pgfqpoint{0.883090in}{1.547034in}}{\pgfqpoint{0.891326in}{1.547034in}}%
\pgfpathclose%
\pgfusepath{stroke,fill}%
\end{pgfscope}%
\begin{pgfscope}%
\pgfpathrectangle{\pgfqpoint{0.100000in}{0.212622in}}{\pgfqpoint{3.696000in}{3.696000in}}%
\pgfusepath{clip}%
\pgfsetbuttcap%
\pgfsetroundjoin%
\definecolor{currentfill}{rgb}{0.121569,0.466667,0.705882}%
\pgfsetfillcolor{currentfill}%
\pgfsetfillopacity{0.577812}%
\pgfsetlinewidth{1.003750pt}%
\definecolor{currentstroke}{rgb}{0.121569,0.466667,0.705882}%
\pgfsetstrokecolor{currentstroke}%
\pgfsetstrokeopacity{0.577812}%
\pgfsetdash{}{0pt}%
\pgfpathmoveto{\pgfqpoint{0.899016in}{1.529938in}}%
\pgfpathcurveto{\pgfqpoint{0.907253in}{1.529938in}}{\pgfqpoint{0.915153in}{1.533211in}}{\pgfqpoint{0.920977in}{1.539034in}}%
\pgfpathcurveto{\pgfqpoint{0.926800in}{1.544858in}}{\pgfqpoint{0.930073in}{1.552758in}}{\pgfqpoint{0.930073in}{1.560995in}}%
\pgfpathcurveto{\pgfqpoint{0.930073in}{1.569231in}}{\pgfqpoint{0.926800in}{1.577131in}}{\pgfqpoint{0.920977in}{1.582955in}}%
\pgfpathcurveto{\pgfqpoint{0.915153in}{1.588779in}}{\pgfqpoint{0.907253in}{1.592051in}}{\pgfqpoint{0.899016in}{1.592051in}}%
\pgfpathcurveto{\pgfqpoint{0.890780in}{1.592051in}}{\pgfqpoint{0.882880in}{1.588779in}}{\pgfqpoint{0.877056in}{1.582955in}}%
\pgfpathcurveto{\pgfqpoint{0.871232in}{1.577131in}}{\pgfqpoint{0.867960in}{1.569231in}}{\pgfqpoint{0.867960in}{1.560995in}}%
\pgfpathcurveto{\pgfqpoint{0.867960in}{1.552758in}}{\pgfqpoint{0.871232in}{1.544858in}}{\pgfqpoint{0.877056in}{1.539034in}}%
\pgfpathcurveto{\pgfqpoint{0.882880in}{1.533211in}}{\pgfqpoint{0.890780in}{1.529938in}}{\pgfqpoint{0.899016in}{1.529938in}}%
\pgfpathclose%
\pgfusepath{stroke,fill}%
\end{pgfscope}%
\begin{pgfscope}%
\pgfpathrectangle{\pgfqpoint{0.100000in}{0.212622in}}{\pgfqpoint{3.696000in}{3.696000in}}%
\pgfusepath{clip}%
\pgfsetbuttcap%
\pgfsetroundjoin%
\definecolor{currentfill}{rgb}{0.121569,0.466667,0.705882}%
\pgfsetfillcolor{currentfill}%
\pgfsetfillopacity{0.577927}%
\pgfsetlinewidth{1.003750pt}%
\definecolor{currentstroke}{rgb}{0.121569,0.466667,0.705882}%
\pgfsetstrokecolor{currentstroke}%
\pgfsetstrokeopacity{0.577927}%
\pgfsetdash{}{0pt}%
\pgfpathmoveto{\pgfqpoint{1.019384in}{1.791861in}}%
\pgfpathcurveto{\pgfqpoint{1.027620in}{1.791861in}}{\pgfqpoint{1.035520in}{1.795133in}}{\pgfqpoint{1.041344in}{1.800957in}}%
\pgfpathcurveto{\pgfqpoint{1.047168in}{1.806781in}}{\pgfqpoint{1.050441in}{1.814681in}}{\pgfqpoint{1.050441in}{1.822917in}}%
\pgfpathcurveto{\pgfqpoint{1.050441in}{1.831154in}}{\pgfqpoint{1.047168in}{1.839054in}}{\pgfqpoint{1.041344in}{1.844878in}}%
\pgfpathcurveto{\pgfqpoint{1.035520in}{1.850701in}}{\pgfqpoint{1.027620in}{1.853974in}}{\pgfqpoint{1.019384in}{1.853974in}}%
\pgfpathcurveto{\pgfqpoint{1.011148in}{1.853974in}}{\pgfqpoint{1.003248in}{1.850701in}}{\pgfqpoint{0.997424in}{1.844878in}}%
\pgfpathcurveto{\pgfqpoint{0.991600in}{1.839054in}}{\pgfqpoint{0.988328in}{1.831154in}}{\pgfqpoint{0.988328in}{1.822917in}}%
\pgfpathcurveto{\pgfqpoint{0.988328in}{1.814681in}}{\pgfqpoint{0.991600in}{1.806781in}}{\pgfqpoint{0.997424in}{1.800957in}}%
\pgfpathcurveto{\pgfqpoint{1.003248in}{1.795133in}}{\pgfqpoint{1.011148in}{1.791861in}}{\pgfqpoint{1.019384in}{1.791861in}}%
\pgfpathclose%
\pgfusepath{stroke,fill}%
\end{pgfscope}%
\begin{pgfscope}%
\pgfpathrectangle{\pgfqpoint{0.100000in}{0.212622in}}{\pgfqpoint{3.696000in}{3.696000in}}%
\pgfusepath{clip}%
\pgfsetbuttcap%
\pgfsetroundjoin%
\definecolor{currentfill}{rgb}{0.121569,0.466667,0.705882}%
\pgfsetfillcolor{currentfill}%
\pgfsetfillopacity{0.577950}%
\pgfsetlinewidth{1.003750pt}%
\definecolor{currentstroke}{rgb}{0.121569,0.466667,0.705882}%
\pgfsetstrokecolor{currentstroke}%
\pgfsetstrokeopacity{0.577950}%
\pgfsetdash{}{0pt}%
\pgfpathmoveto{\pgfqpoint{0.898968in}{1.529235in}}%
\pgfpathcurveto{\pgfqpoint{0.907204in}{1.529235in}}{\pgfqpoint{0.915104in}{1.532507in}}{\pgfqpoint{0.920928in}{1.538331in}}%
\pgfpathcurveto{\pgfqpoint{0.926752in}{1.544155in}}{\pgfqpoint{0.930024in}{1.552055in}}{\pgfqpoint{0.930024in}{1.560291in}}%
\pgfpathcurveto{\pgfqpoint{0.930024in}{1.568528in}}{\pgfqpoint{0.926752in}{1.576428in}}{\pgfqpoint{0.920928in}{1.582252in}}%
\pgfpathcurveto{\pgfqpoint{0.915104in}{1.588076in}}{\pgfqpoint{0.907204in}{1.591348in}}{\pgfqpoint{0.898968in}{1.591348in}}%
\pgfpathcurveto{\pgfqpoint{0.890731in}{1.591348in}}{\pgfqpoint{0.882831in}{1.588076in}}{\pgfqpoint{0.877007in}{1.582252in}}%
\pgfpathcurveto{\pgfqpoint{0.871184in}{1.576428in}}{\pgfqpoint{0.867911in}{1.568528in}}{\pgfqpoint{0.867911in}{1.560291in}}%
\pgfpathcurveto{\pgfqpoint{0.867911in}{1.552055in}}{\pgfqpoint{0.871184in}{1.544155in}}{\pgfqpoint{0.877007in}{1.538331in}}%
\pgfpathcurveto{\pgfqpoint{0.882831in}{1.532507in}}{\pgfqpoint{0.890731in}{1.529235in}}{\pgfqpoint{0.898968in}{1.529235in}}%
\pgfpathclose%
\pgfusepath{stroke,fill}%
\end{pgfscope}%
\begin{pgfscope}%
\pgfpathrectangle{\pgfqpoint{0.100000in}{0.212622in}}{\pgfqpoint{3.696000in}{3.696000in}}%
\pgfusepath{clip}%
\pgfsetbuttcap%
\pgfsetroundjoin%
\definecolor{currentfill}{rgb}{0.121569,0.466667,0.705882}%
\pgfsetfillcolor{currentfill}%
\pgfsetfillopacity{0.578176}%
\pgfsetlinewidth{1.003750pt}%
\definecolor{currentstroke}{rgb}{0.121569,0.466667,0.705882}%
\pgfsetstrokecolor{currentstroke}%
\pgfsetstrokeopacity{0.578176}%
\pgfsetdash{}{0pt}%
\pgfpathmoveto{\pgfqpoint{0.898875in}{1.528099in}}%
\pgfpathcurveto{\pgfqpoint{0.907112in}{1.528099in}}{\pgfqpoint{0.915012in}{1.531372in}}{\pgfqpoint{0.920836in}{1.537196in}}%
\pgfpathcurveto{\pgfqpoint{0.926660in}{1.543020in}}{\pgfqpoint{0.929932in}{1.550920in}}{\pgfqpoint{0.929932in}{1.559156in}}%
\pgfpathcurveto{\pgfqpoint{0.929932in}{1.567392in}}{\pgfqpoint{0.926660in}{1.575292in}}{\pgfqpoint{0.920836in}{1.581116in}}%
\pgfpathcurveto{\pgfqpoint{0.915012in}{1.586940in}}{\pgfqpoint{0.907112in}{1.590212in}}{\pgfqpoint{0.898875in}{1.590212in}}%
\pgfpathcurveto{\pgfqpoint{0.890639in}{1.590212in}}{\pgfqpoint{0.882739in}{1.586940in}}{\pgfqpoint{0.876915in}{1.581116in}}%
\pgfpathcurveto{\pgfqpoint{0.871091in}{1.575292in}}{\pgfqpoint{0.867819in}{1.567392in}}{\pgfqpoint{0.867819in}{1.559156in}}%
\pgfpathcurveto{\pgfqpoint{0.867819in}{1.550920in}}{\pgfqpoint{0.871091in}{1.543020in}}{\pgfqpoint{0.876915in}{1.537196in}}%
\pgfpathcurveto{\pgfqpoint{0.882739in}{1.531372in}}{\pgfqpoint{0.890639in}{1.528099in}}{\pgfqpoint{0.898875in}{1.528099in}}%
\pgfpathclose%
\pgfusepath{stroke,fill}%
\end{pgfscope}%
\begin{pgfscope}%
\pgfpathrectangle{\pgfqpoint{0.100000in}{0.212622in}}{\pgfqpoint{3.696000in}{3.696000in}}%
\pgfusepath{clip}%
\pgfsetbuttcap%
\pgfsetroundjoin%
\definecolor{currentfill}{rgb}{0.121569,0.466667,0.705882}%
\pgfsetfillcolor{currentfill}%
\pgfsetfillopacity{0.578301}%
\pgfsetlinewidth{1.003750pt}%
\definecolor{currentstroke}{rgb}{0.121569,0.466667,0.705882}%
\pgfsetstrokecolor{currentstroke}%
\pgfsetstrokeopacity{0.578301}%
\pgfsetdash{}{0pt}%
\pgfpathmoveto{\pgfqpoint{0.898816in}{1.527477in}}%
\pgfpathcurveto{\pgfqpoint{0.907052in}{1.527477in}}{\pgfqpoint{0.914952in}{1.530750in}}{\pgfqpoint{0.920776in}{1.536574in}}%
\pgfpathcurveto{\pgfqpoint{0.926600in}{1.542398in}}{\pgfqpoint{0.929872in}{1.550298in}}{\pgfqpoint{0.929872in}{1.558534in}}%
\pgfpathcurveto{\pgfqpoint{0.929872in}{1.566770in}}{\pgfqpoint{0.926600in}{1.574670in}}{\pgfqpoint{0.920776in}{1.580494in}}%
\pgfpathcurveto{\pgfqpoint{0.914952in}{1.586318in}}{\pgfqpoint{0.907052in}{1.589590in}}{\pgfqpoint{0.898816in}{1.589590in}}%
\pgfpathcurveto{\pgfqpoint{0.890579in}{1.589590in}}{\pgfqpoint{0.882679in}{1.586318in}}{\pgfqpoint{0.876855in}{1.580494in}}%
\pgfpathcurveto{\pgfqpoint{0.871031in}{1.574670in}}{\pgfqpoint{0.867759in}{1.566770in}}{\pgfqpoint{0.867759in}{1.558534in}}%
\pgfpathcurveto{\pgfqpoint{0.867759in}{1.550298in}}{\pgfqpoint{0.871031in}{1.542398in}}{\pgfqpoint{0.876855in}{1.536574in}}%
\pgfpathcurveto{\pgfqpoint{0.882679in}{1.530750in}}{\pgfqpoint{0.890579in}{1.527477in}}{\pgfqpoint{0.898816in}{1.527477in}}%
\pgfpathclose%
\pgfusepath{stroke,fill}%
\end{pgfscope}%
\begin{pgfscope}%
\pgfpathrectangle{\pgfqpoint{0.100000in}{0.212622in}}{\pgfqpoint{3.696000in}{3.696000in}}%
\pgfusepath{clip}%
\pgfsetbuttcap%
\pgfsetroundjoin%
\definecolor{currentfill}{rgb}{0.121569,0.466667,0.705882}%
\pgfsetfillcolor{currentfill}%
\pgfsetfillopacity{0.578369}%
\pgfsetlinewidth{1.003750pt}%
\definecolor{currentstroke}{rgb}{0.121569,0.466667,0.705882}%
\pgfsetstrokecolor{currentstroke}%
\pgfsetstrokeopacity{0.578369}%
\pgfsetdash{}{0pt}%
\pgfpathmoveto{\pgfqpoint{0.898778in}{1.527132in}}%
\pgfpathcurveto{\pgfqpoint{0.907015in}{1.527132in}}{\pgfqpoint{0.914915in}{1.530405in}}{\pgfqpoint{0.920739in}{1.536229in}}%
\pgfpathcurveto{\pgfqpoint{0.926563in}{1.542053in}}{\pgfqpoint{0.929835in}{1.549953in}}{\pgfqpoint{0.929835in}{1.558189in}}%
\pgfpathcurveto{\pgfqpoint{0.929835in}{1.566425in}}{\pgfqpoint{0.926563in}{1.574325in}}{\pgfqpoint{0.920739in}{1.580149in}}%
\pgfpathcurveto{\pgfqpoint{0.914915in}{1.585973in}}{\pgfqpoint{0.907015in}{1.589245in}}{\pgfqpoint{0.898778in}{1.589245in}}%
\pgfpathcurveto{\pgfqpoint{0.890542in}{1.589245in}}{\pgfqpoint{0.882642in}{1.585973in}}{\pgfqpoint{0.876818in}{1.580149in}}%
\pgfpathcurveto{\pgfqpoint{0.870994in}{1.574325in}}{\pgfqpoint{0.867722in}{1.566425in}}{\pgfqpoint{0.867722in}{1.558189in}}%
\pgfpathcurveto{\pgfqpoint{0.867722in}{1.549953in}}{\pgfqpoint{0.870994in}{1.542053in}}{\pgfqpoint{0.876818in}{1.536229in}}%
\pgfpathcurveto{\pgfqpoint{0.882642in}{1.530405in}}{\pgfqpoint{0.890542in}{1.527132in}}{\pgfqpoint{0.898778in}{1.527132in}}%
\pgfpathclose%
\pgfusepath{stroke,fill}%
\end{pgfscope}%
\begin{pgfscope}%
\pgfpathrectangle{\pgfqpoint{0.100000in}{0.212622in}}{\pgfqpoint{3.696000in}{3.696000in}}%
\pgfusepath{clip}%
\pgfsetbuttcap%
\pgfsetroundjoin%
\definecolor{currentfill}{rgb}{0.121569,0.466667,0.705882}%
\pgfsetfillcolor{currentfill}%
\pgfsetfillopacity{0.578407}%
\pgfsetlinewidth{1.003750pt}%
\definecolor{currentstroke}{rgb}{0.121569,0.466667,0.705882}%
\pgfsetstrokecolor{currentstroke}%
\pgfsetstrokeopacity{0.578407}%
\pgfsetdash{}{0pt}%
\pgfpathmoveto{\pgfqpoint{0.898756in}{1.526942in}}%
\pgfpathcurveto{\pgfqpoint{0.906992in}{1.526942in}}{\pgfqpoint{0.914892in}{1.530214in}}{\pgfqpoint{0.920716in}{1.536038in}}%
\pgfpathcurveto{\pgfqpoint{0.926540in}{1.541862in}}{\pgfqpoint{0.929812in}{1.549762in}}{\pgfqpoint{0.929812in}{1.557998in}}%
\pgfpathcurveto{\pgfqpoint{0.929812in}{1.566234in}}{\pgfqpoint{0.926540in}{1.574134in}}{\pgfqpoint{0.920716in}{1.579958in}}%
\pgfpathcurveto{\pgfqpoint{0.914892in}{1.585782in}}{\pgfqpoint{0.906992in}{1.589055in}}{\pgfqpoint{0.898756in}{1.589055in}}%
\pgfpathcurveto{\pgfqpoint{0.890519in}{1.589055in}}{\pgfqpoint{0.882619in}{1.585782in}}{\pgfqpoint{0.876795in}{1.579958in}}%
\pgfpathcurveto{\pgfqpoint{0.870971in}{1.574134in}}{\pgfqpoint{0.867699in}{1.566234in}}{\pgfqpoint{0.867699in}{1.557998in}}%
\pgfpathcurveto{\pgfqpoint{0.867699in}{1.549762in}}{\pgfqpoint{0.870971in}{1.541862in}}{\pgfqpoint{0.876795in}{1.536038in}}%
\pgfpathcurveto{\pgfqpoint{0.882619in}{1.530214in}}{\pgfqpoint{0.890519in}{1.526942in}}{\pgfqpoint{0.898756in}{1.526942in}}%
\pgfpathclose%
\pgfusepath{stroke,fill}%
\end{pgfscope}%
\begin{pgfscope}%
\pgfpathrectangle{\pgfqpoint{0.100000in}{0.212622in}}{\pgfqpoint{3.696000in}{3.696000in}}%
\pgfusepath{clip}%
\pgfsetbuttcap%
\pgfsetroundjoin%
\definecolor{currentfill}{rgb}{0.121569,0.466667,0.705882}%
\pgfsetfillcolor{currentfill}%
\pgfsetfillopacity{0.578498}%
\pgfsetlinewidth{1.003750pt}%
\definecolor{currentstroke}{rgb}{0.121569,0.466667,0.705882}%
\pgfsetstrokecolor{currentstroke}%
\pgfsetstrokeopacity{0.578498}%
\pgfsetdash{}{0pt}%
\pgfpathmoveto{\pgfqpoint{1.017872in}{1.789359in}}%
\pgfpathcurveto{\pgfqpoint{1.026108in}{1.789359in}}{\pgfqpoint{1.034008in}{1.792631in}}{\pgfqpoint{1.039832in}{1.798455in}}%
\pgfpathcurveto{\pgfqpoint{1.045656in}{1.804279in}}{\pgfqpoint{1.048928in}{1.812179in}}{\pgfqpoint{1.048928in}{1.820415in}}%
\pgfpathcurveto{\pgfqpoint{1.048928in}{1.828652in}}{\pgfqpoint{1.045656in}{1.836552in}}{\pgfqpoint{1.039832in}{1.842376in}}%
\pgfpathcurveto{\pgfqpoint{1.034008in}{1.848200in}}{\pgfqpoint{1.026108in}{1.851472in}}{\pgfqpoint{1.017872in}{1.851472in}}%
\pgfpathcurveto{\pgfqpoint{1.009636in}{1.851472in}}{\pgfqpoint{1.001736in}{1.848200in}}{\pgfqpoint{0.995912in}{1.842376in}}%
\pgfpathcurveto{\pgfqpoint{0.990088in}{1.836552in}}{\pgfqpoint{0.986815in}{1.828652in}}{\pgfqpoint{0.986815in}{1.820415in}}%
\pgfpathcurveto{\pgfqpoint{0.986815in}{1.812179in}}{\pgfqpoint{0.990088in}{1.804279in}}{\pgfqpoint{0.995912in}{1.798455in}}%
\pgfpathcurveto{\pgfqpoint{1.001736in}{1.792631in}}{\pgfqpoint{1.009636in}{1.789359in}}{\pgfqpoint{1.017872in}{1.789359in}}%
\pgfpathclose%
\pgfusepath{stroke,fill}%
\end{pgfscope}%
\begin{pgfscope}%
\pgfpathrectangle{\pgfqpoint{0.100000in}{0.212622in}}{\pgfqpoint{3.696000in}{3.696000in}}%
\pgfusepath{clip}%
\pgfsetbuttcap%
\pgfsetroundjoin%
\definecolor{currentfill}{rgb}{0.121569,0.466667,0.705882}%
\pgfsetfillcolor{currentfill}%
\pgfsetfillopacity{0.578501}%
\pgfsetlinewidth{1.003750pt}%
\definecolor{currentstroke}{rgb}{0.121569,0.466667,0.705882}%
\pgfsetstrokecolor{currentstroke}%
\pgfsetstrokeopacity{0.578501}%
\pgfsetdash{}{0pt}%
\pgfpathmoveto{\pgfqpoint{0.888512in}{1.550962in}}%
\pgfpathcurveto{\pgfqpoint{0.896748in}{1.550962in}}{\pgfqpoint{0.904648in}{1.554234in}}{\pgfqpoint{0.910472in}{1.560058in}}%
\pgfpathcurveto{\pgfqpoint{0.916296in}{1.565882in}}{\pgfqpoint{0.919568in}{1.573782in}}{\pgfqpoint{0.919568in}{1.582018in}}%
\pgfpathcurveto{\pgfqpoint{0.919568in}{1.590254in}}{\pgfqpoint{0.916296in}{1.598154in}}{\pgfqpoint{0.910472in}{1.603978in}}%
\pgfpathcurveto{\pgfqpoint{0.904648in}{1.609802in}}{\pgfqpoint{0.896748in}{1.613075in}}{\pgfqpoint{0.888512in}{1.613075in}}%
\pgfpathcurveto{\pgfqpoint{0.880275in}{1.613075in}}{\pgfqpoint{0.872375in}{1.609802in}}{\pgfqpoint{0.866551in}{1.603978in}}%
\pgfpathcurveto{\pgfqpoint{0.860727in}{1.598154in}}{\pgfqpoint{0.857455in}{1.590254in}}{\pgfqpoint{0.857455in}{1.582018in}}%
\pgfpathcurveto{\pgfqpoint{0.857455in}{1.573782in}}{\pgfqpoint{0.860727in}{1.565882in}}{\pgfqpoint{0.866551in}{1.560058in}}%
\pgfpathcurveto{\pgfqpoint{0.872375in}{1.554234in}}{\pgfqpoint{0.880275in}{1.550962in}}{\pgfqpoint{0.888512in}{1.550962in}}%
\pgfpathclose%
\pgfusepath{stroke,fill}%
\end{pgfscope}%
\begin{pgfscope}%
\pgfpathrectangle{\pgfqpoint{0.100000in}{0.212622in}}{\pgfqpoint{3.696000in}{3.696000in}}%
\pgfusepath{clip}%
\pgfsetbuttcap%
\pgfsetroundjoin%
\definecolor{currentfill}{rgb}{0.121569,0.466667,0.705882}%
\pgfsetfillcolor{currentfill}%
\pgfsetfillopacity{0.578534}%
\pgfsetlinewidth{1.003750pt}%
\definecolor{currentstroke}{rgb}{0.121569,0.466667,0.705882}%
\pgfsetstrokecolor{currentstroke}%
\pgfsetstrokeopacity{0.578534}%
\pgfsetdash{}{0pt}%
\pgfpathmoveto{\pgfqpoint{0.898672in}{1.526308in}}%
\pgfpathcurveto{\pgfqpoint{0.906908in}{1.526308in}}{\pgfqpoint{0.914808in}{1.529580in}}{\pgfqpoint{0.920632in}{1.535404in}}%
\pgfpathcurveto{\pgfqpoint{0.926456in}{1.541228in}}{\pgfqpoint{0.929728in}{1.549128in}}{\pgfqpoint{0.929728in}{1.557365in}}%
\pgfpathcurveto{\pgfqpoint{0.929728in}{1.565601in}}{\pgfqpoint{0.926456in}{1.573501in}}{\pgfqpoint{0.920632in}{1.579325in}}%
\pgfpathcurveto{\pgfqpoint{0.914808in}{1.585149in}}{\pgfqpoint{0.906908in}{1.588421in}}{\pgfqpoint{0.898672in}{1.588421in}}%
\pgfpathcurveto{\pgfqpoint{0.890436in}{1.588421in}}{\pgfqpoint{0.882535in}{1.585149in}}{\pgfqpoint{0.876712in}{1.579325in}}%
\pgfpathcurveto{\pgfqpoint{0.870888in}{1.573501in}}{\pgfqpoint{0.867615in}{1.565601in}}{\pgfqpoint{0.867615in}{1.557365in}}%
\pgfpathcurveto{\pgfqpoint{0.867615in}{1.549128in}}{\pgfqpoint{0.870888in}{1.541228in}}{\pgfqpoint{0.876712in}{1.535404in}}%
\pgfpathcurveto{\pgfqpoint{0.882535in}{1.529580in}}{\pgfqpoint{0.890436in}{1.526308in}}{\pgfqpoint{0.898672in}{1.526308in}}%
\pgfpathclose%
\pgfusepath{stroke,fill}%
\end{pgfscope}%
\begin{pgfscope}%
\pgfpathrectangle{\pgfqpoint{0.100000in}{0.212622in}}{\pgfqpoint{3.696000in}{3.696000in}}%
\pgfusepath{clip}%
\pgfsetbuttcap%
\pgfsetroundjoin%
\definecolor{currentfill}{rgb}{0.121569,0.466667,0.705882}%
\pgfsetfillcolor{currentfill}%
\pgfsetfillopacity{0.578680}%
\pgfsetlinewidth{1.003750pt}%
\definecolor{currentstroke}{rgb}{0.121569,0.466667,0.705882}%
\pgfsetstrokecolor{currentstroke}%
\pgfsetstrokeopacity{0.578680}%
\pgfsetdash{}{0pt}%
\pgfpathmoveto{\pgfqpoint{2.087003in}{2.181532in}}%
\pgfpathcurveto{\pgfqpoint{2.095239in}{2.181532in}}{\pgfqpoint{2.103139in}{2.184805in}}{\pgfqpoint{2.108963in}{2.190629in}}%
\pgfpathcurveto{\pgfqpoint{2.114787in}{2.196453in}}{\pgfqpoint{2.118059in}{2.204353in}}{\pgfqpoint{2.118059in}{2.212589in}}%
\pgfpathcurveto{\pgfqpoint{2.118059in}{2.220825in}}{\pgfqpoint{2.114787in}{2.228725in}}{\pgfqpoint{2.108963in}{2.234549in}}%
\pgfpathcurveto{\pgfqpoint{2.103139in}{2.240373in}}{\pgfqpoint{2.095239in}{2.243645in}}{\pgfqpoint{2.087003in}{2.243645in}}%
\pgfpathcurveto{\pgfqpoint{2.078766in}{2.243645in}}{\pgfqpoint{2.070866in}{2.240373in}}{\pgfqpoint{2.065042in}{2.234549in}}%
\pgfpathcurveto{\pgfqpoint{2.059219in}{2.228725in}}{\pgfqpoint{2.055946in}{2.220825in}}{\pgfqpoint{2.055946in}{2.212589in}}%
\pgfpathcurveto{\pgfqpoint{2.055946in}{2.204353in}}{\pgfqpoint{2.059219in}{2.196453in}}{\pgfqpoint{2.065042in}{2.190629in}}%
\pgfpathcurveto{\pgfqpoint{2.070866in}{2.184805in}}{\pgfqpoint{2.078766in}{2.181532in}}{\pgfqpoint{2.087003in}{2.181532in}}%
\pgfpathclose%
\pgfusepath{stroke,fill}%
\end{pgfscope}%
\begin{pgfscope}%
\pgfpathrectangle{\pgfqpoint{0.100000in}{0.212622in}}{\pgfqpoint{3.696000in}{3.696000in}}%
\pgfusepath{clip}%
\pgfsetbuttcap%
\pgfsetroundjoin%
\definecolor{currentfill}{rgb}{0.121569,0.466667,0.705882}%
\pgfsetfillcolor{currentfill}%
\pgfsetfillopacity{0.578765}%
\pgfsetlinewidth{1.003750pt}%
\definecolor{currentstroke}{rgb}{0.121569,0.466667,0.705882}%
\pgfsetstrokecolor{currentstroke}%
\pgfsetstrokeopacity{0.578765}%
\pgfsetdash{}{0pt}%
\pgfpathmoveto{\pgfqpoint{0.898513in}{1.525120in}}%
\pgfpathcurveto{\pgfqpoint{0.906749in}{1.525120in}}{\pgfqpoint{0.914649in}{1.528393in}}{\pgfqpoint{0.920473in}{1.534217in}}%
\pgfpathcurveto{\pgfqpoint{0.926297in}{1.540041in}}{\pgfqpoint{0.929569in}{1.547941in}}{\pgfqpoint{0.929569in}{1.556177in}}%
\pgfpathcurveto{\pgfqpoint{0.929569in}{1.564413in}}{\pgfqpoint{0.926297in}{1.572313in}}{\pgfqpoint{0.920473in}{1.578137in}}%
\pgfpathcurveto{\pgfqpoint{0.914649in}{1.583961in}}{\pgfqpoint{0.906749in}{1.587233in}}{\pgfqpoint{0.898513in}{1.587233in}}%
\pgfpathcurveto{\pgfqpoint{0.890276in}{1.587233in}}{\pgfqpoint{0.882376in}{1.583961in}}{\pgfqpoint{0.876552in}{1.578137in}}%
\pgfpathcurveto{\pgfqpoint{0.870728in}{1.572313in}}{\pgfqpoint{0.867456in}{1.564413in}}{\pgfqpoint{0.867456in}{1.556177in}}%
\pgfpathcurveto{\pgfqpoint{0.867456in}{1.547941in}}{\pgfqpoint{0.870728in}{1.540041in}}{\pgfqpoint{0.876552in}{1.534217in}}%
\pgfpathcurveto{\pgfqpoint{0.882376in}{1.528393in}}{\pgfqpoint{0.890276in}{1.525120in}}{\pgfqpoint{0.898513in}{1.525120in}}%
\pgfpathclose%
\pgfusepath{stroke,fill}%
\end{pgfscope}%
\begin{pgfscope}%
\pgfpathrectangle{\pgfqpoint{0.100000in}{0.212622in}}{\pgfqpoint{3.696000in}{3.696000in}}%
\pgfusepath{clip}%
\pgfsetbuttcap%
\pgfsetroundjoin%
\definecolor{currentfill}{rgb}{0.121569,0.466667,0.705882}%
\pgfsetfillcolor{currentfill}%
\pgfsetfillopacity{0.579051}%
\pgfsetlinewidth{1.003750pt}%
\definecolor{currentstroke}{rgb}{0.121569,0.466667,0.705882}%
\pgfsetstrokecolor{currentstroke}%
\pgfsetstrokeopacity{0.579051}%
\pgfsetdash{}{0pt}%
\pgfpathmoveto{\pgfqpoint{0.898288in}{1.523537in}}%
\pgfpathcurveto{\pgfqpoint{0.906525in}{1.523537in}}{\pgfqpoint{0.914425in}{1.526810in}}{\pgfqpoint{0.920249in}{1.532634in}}%
\pgfpathcurveto{\pgfqpoint{0.926073in}{1.538458in}}{\pgfqpoint{0.929345in}{1.546358in}}{\pgfqpoint{0.929345in}{1.554594in}}%
\pgfpathcurveto{\pgfqpoint{0.929345in}{1.562830in}}{\pgfqpoint{0.926073in}{1.570730in}}{\pgfqpoint{0.920249in}{1.576554in}}%
\pgfpathcurveto{\pgfqpoint{0.914425in}{1.582378in}}{\pgfqpoint{0.906525in}{1.585650in}}{\pgfqpoint{0.898288in}{1.585650in}}%
\pgfpathcurveto{\pgfqpoint{0.890052in}{1.585650in}}{\pgfqpoint{0.882152in}{1.582378in}}{\pgfqpoint{0.876328in}{1.576554in}}%
\pgfpathcurveto{\pgfqpoint{0.870504in}{1.570730in}}{\pgfqpoint{0.867232in}{1.562830in}}{\pgfqpoint{0.867232in}{1.554594in}}%
\pgfpathcurveto{\pgfqpoint{0.867232in}{1.546358in}}{\pgfqpoint{0.870504in}{1.538458in}}{\pgfqpoint{0.876328in}{1.532634in}}%
\pgfpathcurveto{\pgfqpoint{0.882152in}{1.526810in}}{\pgfqpoint{0.890052in}{1.523537in}}{\pgfqpoint{0.898288in}{1.523537in}}%
\pgfpathclose%
\pgfusepath{stroke,fill}%
\end{pgfscope}%
\begin{pgfscope}%
\pgfpathrectangle{\pgfqpoint{0.100000in}{0.212622in}}{\pgfqpoint{3.696000in}{3.696000in}}%
\pgfusepath{clip}%
\pgfsetbuttcap%
\pgfsetroundjoin%
\definecolor{currentfill}{rgb}{0.121569,0.466667,0.705882}%
\pgfsetfillcolor{currentfill}%
\pgfsetfillopacity{0.579209}%
\pgfsetlinewidth{1.003750pt}%
\definecolor{currentstroke}{rgb}{0.121569,0.466667,0.705882}%
\pgfsetstrokecolor{currentstroke}%
\pgfsetstrokeopacity{0.579209}%
\pgfsetdash{}{0pt}%
\pgfpathmoveto{\pgfqpoint{0.898154in}{1.522668in}}%
\pgfpathcurveto{\pgfqpoint{0.906390in}{1.522668in}}{\pgfqpoint{0.914290in}{1.525940in}}{\pgfqpoint{0.920114in}{1.531764in}}%
\pgfpathcurveto{\pgfqpoint{0.925938in}{1.537588in}}{\pgfqpoint{0.929210in}{1.545488in}}{\pgfqpoint{0.929210in}{1.553724in}}%
\pgfpathcurveto{\pgfqpoint{0.929210in}{1.561960in}}{\pgfqpoint{0.925938in}{1.569860in}}{\pgfqpoint{0.920114in}{1.575684in}}%
\pgfpathcurveto{\pgfqpoint{0.914290in}{1.581508in}}{\pgfqpoint{0.906390in}{1.584781in}}{\pgfqpoint{0.898154in}{1.584781in}}%
\pgfpathcurveto{\pgfqpoint{0.889917in}{1.584781in}}{\pgfqpoint{0.882017in}{1.581508in}}{\pgfqpoint{0.876193in}{1.575684in}}%
\pgfpathcurveto{\pgfqpoint{0.870370in}{1.569860in}}{\pgfqpoint{0.867097in}{1.561960in}}{\pgfqpoint{0.867097in}{1.553724in}}%
\pgfpathcurveto{\pgfqpoint{0.867097in}{1.545488in}}{\pgfqpoint{0.870370in}{1.537588in}}{\pgfqpoint{0.876193in}{1.531764in}}%
\pgfpathcurveto{\pgfqpoint{0.882017in}{1.525940in}}{\pgfqpoint{0.889917in}{1.522668in}}{\pgfqpoint{0.898154in}{1.522668in}}%
\pgfpathclose%
\pgfusepath{stroke,fill}%
\end{pgfscope}%
\begin{pgfscope}%
\pgfpathrectangle{\pgfqpoint{0.100000in}{0.212622in}}{\pgfqpoint{3.696000in}{3.696000in}}%
\pgfusepath{clip}%
\pgfsetbuttcap%
\pgfsetroundjoin%
\definecolor{currentfill}{rgb}{0.121569,0.466667,0.705882}%
\pgfsetfillcolor{currentfill}%
\pgfsetfillopacity{0.579398}%
\pgfsetlinewidth{1.003750pt}%
\definecolor{currentstroke}{rgb}{0.121569,0.466667,0.705882}%
\pgfsetstrokecolor{currentstroke}%
\pgfsetstrokeopacity{0.579398}%
\pgfsetdash{}{0pt}%
\pgfpathmoveto{\pgfqpoint{0.885949in}{1.554477in}}%
\pgfpathcurveto{\pgfqpoint{0.894186in}{1.554477in}}{\pgfqpoint{0.902086in}{1.557750in}}{\pgfqpoint{0.907910in}{1.563573in}}%
\pgfpathcurveto{\pgfqpoint{0.913733in}{1.569397in}}{\pgfqpoint{0.917006in}{1.577297in}}{\pgfqpoint{0.917006in}{1.585534in}}%
\pgfpathcurveto{\pgfqpoint{0.917006in}{1.593770in}}{\pgfqpoint{0.913733in}{1.601670in}}{\pgfqpoint{0.907910in}{1.607494in}}%
\pgfpathcurveto{\pgfqpoint{0.902086in}{1.613318in}}{\pgfqpoint{0.894186in}{1.616590in}}{\pgfqpoint{0.885949in}{1.616590in}}%
\pgfpathcurveto{\pgfqpoint{0.877713in}{1.616590in}}{\pgfqpoint{0.869813in}{1.613318in}}{\pgfqpoint{0.863989in}{1.607494in}}%
\pgfpathcurveto{\pgfqpoint{0.858165in}{1.601670in}}{\pgfqpoint{0.854893in}{1.593770in}}{\pgfqpoint{0.854893in}{1.585534in}}%
\pgfpathcurveto{\pgfqpoint{0.854893in}{1.577297in}}{\pgfqpoint{0.858165in}{1.569397in}}{\pgfqpoint{0.863989in}{1.563573in}}%
\pgfpathcurveto{\pgfqpoint{0.869813in}{1.557750in}}{\pgfqpoint{0.877713in}{1.554477in}}{\pgfqpoint{0.885949in}{1.554477in}}%
\pgfpathclose%
\pgfusepath{stroke,fill}%
\end{pgfscope}%
\begin{pgfscope}%
\pgfpathrectangle{\pgfqpoint{0.100000in}{0.212622in}}{\pgfqpoint{3.696000in}{3.696000in}}%
\pgfusepath{clip}%
\pgfsetbuttcap%
\pgfsetroundjoin%
\definecolor{currentfill}{rgb}{0.121569,0.466667,0.705882}%
\pgfsetfillcolor{currentfill}%
\pgfsetfillopacity{0.579479}%
\pgfsetlinewidth{1.003750pt}%
\definecolor{currentstroke}{rgb}{0.121569,0.466667,0.705882}%
\pgfsetstrokecolor{currentstroke}%
\pgfsetstrokeopacity{0.579479}%
\pgfsetdash{}{0pt}%
\pgfpathmoveto{\pgfqpoint{0.897895in}{1.521141in}}%
\pgfpathcurveto{\pgfqpoint{0.906131in}{1.521141in}}{\pgfqpoint{0.914031in}{1.524414in}}{\pgfqpoint{0.919855in}{1.530238in}}%
\pgfpathcurveto{\pgfqpoint{0.925679in}{1.536061in}}{\pgfqpoint{0.928951in}{1.543962in}}{\pgfqpoint{0.928951in}{1.552198in}}%
\pgfpathcurveto{\pgfqpoint{0.928951in}{1.560434in}}{\pgfqpoint{0.925679in}{1.568334in}}{\pgfqpoint{0.919855in}{1.574158in}}%
\pgfpathcurveto{\pgfqpoint{0.914031in}{1.579982in}}{\pgfqpoint{0.906131in}{1.583254in}}{\pgfqpoint{0.897895in}{1.583254in}}%
\pgfpathcurveto{\pgfqpoint{0.889658in}{1.583254in}}{\pgfqpoint{0.881758in}{1.579982in}}{\pgfqpoint{0.875934in}{1.574158in}}%
\pgfpathcurveto{\pgfqpoint{0.870110in}{1.568334in}}{\pgfqpoint{0.866838in}{1.560434in}}{\pgfqpoint{0.866838in}{1.552198in}}%
\pgfpathcurveto{\pgfqpoint{0.866838in}{1.543962in}}{\pgfqpoint{0.870110in}{1.536061in}}{\pgfqpoint{0.875934in}{1.530238in}}%
\pgfpathcurveto{\pgfqpoint{0.881758in}{1.524414in}}{\pgfqpoint{0.889658in}{1.521141in}}{\pgfqpoint{0.897895in}{1.521141in}}%
\pgfpathclose%
\pgfusepath{stroke,fill}%
\end{pgfscope}%
\begin{pgfscope}%
\pgfpathrectangle{\pgfqpoint{0.100000in}{0.212622in}}{\pgfqpoint{3.696000in}{3.696000in}}%
\pgfusepath{clip}%
\pgfsetbuttcap%
\pgfsetroundjoin%
\definecolor{currentfill}{rgb}{0.121569,0.466667,0.705882}%
\pgfsetfillcolor{currentfill}%
\pgfsetfillopacity{0.579573}%
\pgfsetlinewidth{1.003750pt}%
\definecolor{currentstroke}{rgb}{0.121569,0.466667,0.705882}%
\pgfsetstrokecolor{currentstroke}%
\pgfsetstrokeopacity{0.579573}%
\pgfsetdash{}{0pt}%
\pgfpathmoveto{\pgfqpoint{1.015268in}{1.784769in}}%
\pgfpathcurveto{\pgfqpoint{1.023504in}{1.784769in}}{\pgfqpoint{1.031404in}{1.788041in}}{\pgfqpoint{1.037228in}{1.793865in}}%
\pgfpathcurveto{\pgfqpoint{1.043052in}{1.799689in}}{\pgfqpoint{1.046324in}{1.807589in}}{\pgfqpoint{1.046324in}{1.815826in}}%
\pgfpathcurveto{\pgfqpoint{1.046324in}{1.824062in}}{\pgfqpoint{1.043052in}{1.831962in}}{\pgfqpoint{1.037228in}{1.837786in}}%
\pgfpathcurveto{\pgfqpoint{1.031404in}{1.843610in}}{\pgfqpoint{1.023504in}{1.846882in}}{\pgfqpoint{1.015268in}{1.846882in}}%
\pgfpathcurveto{\pgfqpoint{1.007031in}{1.846882in}}{\pgfqpoint{0.999131in}{1.843610in}}{\pgfqpoint{0.993307in}{1.837786in}}%
\pgfpathcurveto{\pgfqpoint{0.987483in}{1.831962in}}{\pgfqpoint{0.984211in}{1.824062in}}{\pgfqpoint{0.984211in}{1.815826in}}%
\pgfpathcurveto{\pgfqpoint{0.984211in}{1.807589in}}{\pgfqpoint{0.987483in}{1.799689in}}{\pgfqpoint{0.993307in}{1.793865in}}%
\pgfpathcurveto{\pgfqpoint{0.999131in}{1.788041in}}{\pgfqpoint{1.007031in}{1.784769in}}{\pgfqpoint{1.015268in}{1.784769in}}%
\pgfpathclose%
\pgfusepath{stroke,fill}%
\end{pgfscope}%
\begin{pgfscope}%
\pgfpathrectangle{\pgfqpoint{0.100000in}{0.212622in}}{\pgfqpoint{3.696000in}{3.696000in}}%
\pgfusepath{clip}%
\pgfsetbuttcap%
\pgfsetroundjoin%
\definecolor{currentfill}{rgb}{0.121569,0.466667,0.705882}%
\pgfsetfillcolor{currentfill}%
\pgfsetfillopacity{0.579627}%
\pgfsetlinewidth{1.003750pt}%
\definecolor{currentstroke}{rgb}{0.121569,0.466667,0.705882}%
\pgfsetstrokecolor{currentstroke}%
\pgfsetstrokeopacity{0.579627}%
\pgfsetdash{}{0pt}%
\pgfpathmoveto{\pgfqpoint{0.897749in}{1.520299in}}%
\pgfpathcurveto{\pgfqpoint{0.905985in}{1.520299in}}{\pgfqpoint{0.913885in}{1.523571in}}{\pgfqpoint{0.919709in}{1.529395in}}%
\pgfpathcurveto{\pgfqpoint{0.925533in}{1.535219in}}{\pgfqpoint{0.928805in}{1.543119in}}{\pgfqpoint{0.928805in}{1.551356in}}%
\pgfpathcurveto{\pgfqpoint{0.928805in}{1.559592in}}{\pgfqpoint{0.925533in}{1.567492in}}{\pgfqpoint{0.919709in}{1.573316in}}%
\pgfpathcurveto{\pgfqpoint{0.913885in}{1.579140in}}{\pgfqpoint{0.905985in}{1.582412in}}{\pgfqpoint{0.897749in}{1.582412in}}%
\pgfpathcurveto{\pgfqpoint{0.889512in}{1.582412in}}{\pgfqpoint{0.881612in}{1.579140in}}{\pgfqpoint{0.875788in}{1.573316in}}%
\pgfpathcurveto{\pgfqpoint{0.869964in}{1.567492in}}{\pgfqpoint{0.866692in}{1.559592in}}{\pgfqpoint{0.866692in}{1.551356in}}%
\pgfpathcurveto{\pgfqpoint{0.866692in}{1.543119in}}{\pgfqpoint{0.869964in}{1.535219in}}{\pgfqpoint{0.875788in}{1.529395in}}%
\pgfpathcurveto{\pgfqpoint{0.881612in}{1.523571in}}{\pgfqpoint{0.889512in}{1.520299in}}{\pgfqpoint{0.897749in}{1.520299in}}%
\pgfpathclose%
\pgfusepath{stroke,fill}%
\end{pgfscope}%
\begin{pgfscope}%
\pgfpathrectangle{\pgfqpoint{0.100000in}{0.212622in}}{\pgfqpoint{3.696000in}{3.696000in}}%
\pgfusepath{clip}%
\pgfsetbuttcap%
\pgfsetroundjoin%
\definecolor{currentfill}{rgb}{0.121569,0.466667,0.705882}%
\pgfsetfillcolor{currentfill}%
\pgfsetfillopacity{0.579706}%
\pgfsetlinewidth{1.003750pt}%
\definecolor{currentstroke}{rgb}{0.121569,0.466667,0.705882}%
\pgfsetstrokecolor{currentstroke}%
\pgfsetstrokeopacity{0.579706}%
\pgfsetdash{}{0pt}%
\pgfpathmoveto{\pgfqpoint{0.897665in}{1.519829in}}%
\pgfpathcurveto{\pgfqpoint{0.905902in}{1.519829in}}{\pgfqpoint{0.913802in}{1.523101in}}{\pgfqpoint{0.919626in}{1.528925in}}%
\pgfpathcurveto{\pgfqpoint{0.925450in}{1.534749in}}{\pgfqpoint{0.928722in}{1.542649in}}{\pgfqpoint{0.928722in}{1.550886in}}%
\pgfpathcurveto{\pgfqpoint{0.928722in}{1.559122in}}{\pgfqpoint{0.925450in}{1.567022in}}{\pgfqpoint{0.919626in}{1.572846in}}%
\pgfpathcurveto{\pgfqpoint{0.913802in}{1.578670in}}{\pgfqpoint{0.905902in}{1.581942in}}{\pgfqpoint{0.897665in}{1.581942in}}%
\pgfpathcurveto{\pgfqpoint{0.889429in}{1.581942in}}{\pgfqpoint{0.881529in}{1.578670in}}{\pgfqpoint{0.875705in}{1.572846in}}%
\pgfpathcurveto{\pgfqpoint{0.869881in}{1.567022in}}{\pgfqpoint{0.866609in}{1.559122in}}{\pgfqpoint{0.866609in}{1.550886in}}%
\pgfpathcurveto{\pgfqpoint{0.866609in}{1.542649in}}{\pgfqpoint{0.869881in}{1.534749in}}{\pgfqpoint{0.875705in}{1.528925in}}%
\pgfpathcurveto{\pgfqpoint{0.881529in}{1.523101in}}{\pgfqpoint{0.889429in}{1.519829in}}{\pgfqpoint{0.897665in}{1.519829in}}%
\pgfpathclose%
\pgfusepath{stroke,fill}%
\end{pgfscope}%
\begin{pgfscope}%
\pgfpathrectangle{\pgfqpoint{0.100000in}{0.212622in}}{\pgfqpoint{3.696000in}{3.696000in}}%
\pgfusepath{clip}%
\pgfsetbuttcap%
\pgfsetroundjoin%
\definecolor{currentfill}{rgb}{0.121569,0.466667,0.705882}%
\pgfsetfillcolor{currentfill}%
\pgfsetfillopacity{0.579911}%
\pgfsetlinewidth{1.003750pt}%
\definecolor{currentstroke}{rgb}{0.121569,0.466667,0.705882}%
\pgfsetstrokecolor{currentstroke}%
\pgfsetstrokeopacity{0.579911}%
\pgfsetdash{}{0pt}%
\pgfpathmoveto{\pgfqpoint{0.897460in}{1.518594in}}%
\pgfpathcurveto{\pgfqpoint{0.905697in}{1.518594in}}{\pgfqpoint{0.913597in}{1.521866in}}{\pgfqpoint{0.919421in}{1.527690in}}%
\pgfpathcurveto{\pgfqpoint{0.925245in}{1.533514in}}{\pgfqpoint{0.928517in}{1.541414in}}{\pgfqpoint{0.928517in}{1.549650in}}%
\pgfpathcurveto{\pgfqpoint{0.928517in}{1.557886in}}{\pgfqpoint{0.925245in}{1.565786in}}{\pgfqpoint{0.919421in}{1.571610in}}%
\pgfpathcurveto{\pgfqpoint{0.913597in}{1.577434in}}{\pgfqpoint{0.905697in}{1.580707in}}{\pgfqpoint{0.897460in}{1.580707in}}%
\pgfpathcurveto{\pgfqpoint{0.889224in}{1.580707in}}{\pgfqpoint{0.881324in}{1.577434in}}{\pgfqpoint{0.875500in}{1.571610in}}%
\pgfpathcurveto{\pgfqpoint{0.869676in}{1.565786in}}{\pgfqpoint{0.866404in}{1.557886in}}{\pgfqpoint{0.866404in}{1.549650in}}%
\pgfpathcurveto{\pgfqpoint{0.866404in}{1.541414in}}{\pgfqpoint{0.869676in}{1.533514in}}{\pgfqpoint{0.875500in}{1.527690in}}%
\pgfpathcurveto{\pgfqpoint{0.881324in}{1.521866in}}{\pgfqpoint{0.889224in}{1.518594in}}{\pgfqpoint{0.897460in}{1.518594in}}%
\pgfpathclose%
\pgfusepath{stroke,fill}%
\end{pgfscope}%
\begin{pgfscope}%
\pgfpathrectangle{\pgfqpoint{0.100000in}{0.212622in}}{\pgfqpoint{3.696000in}{3.696000in}}%
\pgfusepath{clip}%
\pgfsetbuttcap%
\pgfsetroundjoin%
\definecolor{currentfill}{rgb}{0.121569,0.466667,0.705882}%
\pgfsetfillcolor{currentfill}%
\pgfsetfillopacity{0.580023}%
\pgfsetlinewidth{1.003750pt}%
\definecolor{currentstroke}{rgb}{0.121569,0.466667,0.705882}%
\pgfsetstrokecolor{currentstroke}%
\pgfsetstrokeopacity{0.580023}%
\pgfsetdash{}{0pt}%
\pgfpathmoveto{\pgfqpoint{0.897347in}{1.517908in}}%
\pgfpathcurveto{\pgfqpoint{0.905583in}{1.517908in}}{\pgfqpoint{0.913483in}{1.521181in}}{\pgfqpoint{0.919307in}{1.527005in}}%
\pgfpathcurveto{\pgfqpoint{0.925131in}{1.532828in}}{\pgfqpoint{0.928403in}{1.540729in}}{\pgfqpoint{0.928403in}{1.548965in}}%
\pgfpathcurveto{\pgfqpoint{0.928403in}{1.557201in}}{\pgfqpoint{0.925131in}{1.565101in}}{\pgfqpoint{0.919307in}{1.570925in}}%
\pgfpathcurveto{\pgfqpoint{0.913483in}{1.576749in}}{\pgfqpoint{0.905583in}{1.580021in}}{\pgfqpoint{0.897347in}{1.580021in}}%
\pgfpathcurveto{\pgfqpoint{0.889111in}{1.580021in}}{\pgfqpoint{0.881211in}{1.576749in}}{\pgfqpoint{0.875387in}{1.570925in}}%
\pgfpathcurveto{\pgfqpoint{0.869563in}{1.565101in}}{\pgfqpoint{0.866290in}{1.557201in}}{\pgfqpoint{0.866290in}{1.548965in}}%
\pgfpathcurveto{\pgfqpoint{0.866290in}{1.540729in}}{\pgfqpoint{0.869563in}{1.532828in}}{\pgfqpoint{0.875387in}{1.527005in}}%
\pgfpathcurveto{\pgfqpoint{0.881211in}{1.521181in}}{\pgfqpoint{0.889111in}{1.517908in}}{\pgfqpoint{0.897347in}{1.517908in}}%
\pgfpathclose%
\pgfusepath{stroke,fill}%
\end{pgfscope}%
\begin{pgfscope}%
\pgfpathrectangle{\pgfqpoint{0.100000in}{0.212622in}}{\pgfqpoint{3.696000in}{3.696000in}}%
\pgfusepath{clip}%
\pgfsetbuttcap%
\pgfsetroundjoin%
\definecolor{currentfill}{rgb}{0.121569,0.466667,0.705882}%
\pgfsetfillcolor{currentfill}%
\pgfsetfillopacity{0.580204}%
\pgfsetlinewidth{1.003750pt}%
\definecolor{currentstroke}{rgb}{0.121569,0.466667,0.705882}%
\pgfsetstrokecolor{currentstroke}%
\pgfsetstrokeopacity{0.580204}%
\pgfsetdash{}{0pt}%
\pgfpathmoveto{\pgfqpoint{0.897140in}{1.516724in}}%
\pgfpathcurveto{\pgfqpoint{0.905376in}{1.516724in}}{\pgfqpoint{0.913276in}{1.519996in}}{\pgfqpoint{0.919100in}{1.525820in}}%
\pgfpathcurveto{\pgfqpoint{0.924924in}{1.531644in}}{\pgfqpoint{0.928196in}{1.539544in}}{\pgfqpoint{0.928196in}{1.547781in}}%
\pgfpathcurveto{\pgfqpoint{0.928196in}{1.556017in}}{\pgfqpoint{0.924924in}{1.563917in}}{\pgfqpoint{0.919100in}{1.569741in}}%
\pgfpathcurveto{\pgfqpoint{0.913276in}{1.575565in}}{\pgfqpoint{0.905376in}{1.578837in}}{\pgfqpoint{0.897140in}{1.578837in}}%
\pgfpathcurveto{\pgfqpoint{0.888904in}{1.578837in}}{\pgfqpoint{0.881004in}{1.575565in}}{\pgfqpoint{0.875180in}{1.569741in}}%
\pgfpathcurveto{\pgfqpoint{0.869356in}{1.563917in}}{\pgfqpoint{0.866083in}{1.556017in}}{\pgfqpoint{0.866083in}{1.547781in}}%
\pgfpathcurveto{\pgfqpoint{0.866083in}{1.539544in}}{\pgfqpoint{0.869356in}{1.531644in}}{\pgfqpoint{0.875180in}{1.525820in}}%
\pgfpathcurveto{\pgfqpoint{0.881004in}{1.519996in}}{\pgfqpoint{0.888904in}{1.516724in}}{\pgfqpoint{0.897140in}{1.516724in}}%
\pgfpathclose%
\pgfusepath{stroke,fill}%
\end{pgfscope}%
\begin{pgfscope}%
\pgfpathrectangle{\pgfqpoint{0.100000in}{0.212622in}}{\pgfqpoint{3.696000in}{3.696000in}}%
\pgfusepath{clip}%
\pgfsetbuttcap%
\pgfsetroundjoin%
\definecolor{currentfill}{rgb}{0.121569,0.466667,0.705882}%
\pgfsetfillcolor{currentfill}%
\pgfsetfillopacity{0.580309}%
\pgfsetlinewidth{1.003750pt}%
\definecolor{currentstroke}{rgb}{0.121569,0.466667,0.705882}%
\pgfsetstrokecolor{currentstroke}%
\pgfsetstrokeopacity{0.580309}%
\pgfsetdash{}{0pt}%
\pgfpathmoveto{\pgfqpoint{0.897028in}{1.516092in}}%
\pgfpathcurveto{\pgfqpoint{0.905264in}{1.516092in}}{\pgfqpoint{0.913164in}{1.519364in}}{\pgfqpoint{0.918988in}{1.525188in}}%
\pgfpathcurveto{\pgfqpoint{0.924812in}{1.531012in}}{\pgfqpoint{0.928084in}{1.538912in}}{\pgfqpoint{0.928084in}{1.547149in}}%
\pgfpathcurveto{\pgfqpoint{0.928084in}{1.555385in}}{\pgfqpoint{0.924812in}{1.563285in}}{\pgfqpoint{0.918988in}{1.569109in}}%
\pgfpathcurveto{\pgfqpoint{0.913164in}{1.574933in}}{\pgfqpoint{0.905264in}{1.578205in}}{\pgfqpoint{0.897028in}{1.578205in}}%
\pgfpathcurveto{\pgfqpoint{0.888791in}{1.578205in}}{\pgfqpoint{0.880891in}{1.574933in}}{\pgfqpoint{0.875067in}{1.569109in}}%
\pgfpathcurveto{\pgfqpoint{0.869243in}{1.563285in}}{\pgfqpoint{0.865971in}{1.555385in}}{\pgfqpoint{0.865971in}{1.547149in}}%
\pgfpathcurveto{\pgfqpoint{0.865971in}{1.538912in}}{\pgfqpoint{0.869243in}{1.531012in}}{\pgfqpoint{0.875067in}{1.525188in}}%
\pgfpathcurveto{\pgfqpoint{0.880891in}{1.519364in}}{\pgfqpoint{0.888791in}{1.516092in}}{\pgfqpoint{0.897028in}{1.516092in}}%
\pgfpathclose%
\pgfusepath{stroke,fill}%
\end{pgfscope}%
\begin{pgfscope}%
\pgfpathrectangle{\pgfqpoint{0.100000in}{0.212622in}}{\pgfqpoint{3.696000in}{3.696000in}}%
\pgfusepath{clip}%
\pgfsetbuttcap%
\pgfsetroundjoin%
\definecolor{currentfill}{rgb}{0.121569,0.466667,0.705882}%
\pgfsetfillcolor{currentfill}%
\pgfsetfillopacity{0.580364}%
\pgfsetlinewidth{1.003750pt}%
\definecolor{currentstroke}{rgb}{0.121569,0.466667,0.705882}%
\pgfsetstrokecolor{currentstroke}%
\pgfsetstrokeopacity{0.580364}%
\pgfsetdash{}{0pt}%
\pgfpathmoveto{\pgfqpoint{0.896960in}{1.515735in}}%
\pgfpathcurveto{\pgfqpoint{0.905196in}{1.515735in}}{\pgfqpoint{0.913096in}{1.519007in}}{\pgfqpoint{0.918920in}{1.524831in}}%
\pgfpathcurveto{\pgfqpoint{0.924744in}{1.530655in}}{\pgfqpoint{0.928016in}{1.538555in}}{\pgfqpoint{0.928016in}{1.546792in}}%
\pgfpathcurveto{\pgfqpoint{0.928016in}{1.555028in}}{\pgfqpoint{0.924744in}{1.562928in}}{\pgfqpoint{0.918920in}{1.568752in}}%
\pgfpathcurveto{\pgfqpoint{0.913096in}{1.574576in}}{\pgfqpoint{0.905196in}{1.577848in}}{\pgfqpoint{0.896960in}{1.577848in}}%
\pgfpathcurveto{\pgfqpoint{0.888724in}{1.577848in}}{\pgfqpoint{0.880824in}{1.574576in}}{\pgfqpoint{0.875000in}{1.568752in}}%
\pgfpathcurveto{\pgfqpoint{0.869176in}{1.562928in}}{\pgfqpoint{0.865903in}{1.555028in}}{\pgfqpoint{0.865903in}{1.546792in}}%
\pgfpathcurveto{\pgfqpoint{0.865903in}{1.538555in}}{\pgfqpoint{0.869176in}{1.530655in}}{\pgfqpoint{0.875000in}{1.524831in}}%
\pgfpathcurveto{\pgfqpoint{0.880824in}{1.519007in}}{\pgfqpoint{0.888724in}{1.515735in}}{\pgfqpoint{0.896960in}{1.515735in}}%
\pgfpathclose%
\pgfusepath{stroke,fill}%
\end{pgfscope}%
\begin{pgfscope}%
\pgfpathrectangle{\pgfqpoint{0.100000in}{0.212622in}}{\pgfqpoint{3.696000in}{3.696000in}}%
\pgfusepath{clip}%
\pgfsetbuttcap%
\pgfsetroundjoin%
\definecolor{currentfill}{rgb}{0.121569,0.466667,0.705882}%
\pgfsetfillcolor{currentfill}%
\pgfsetfillopacity{0.580412}%
\pgfsetlinewidth{1.003750pt}%
\definecolor{currentstroke}{rgb}{0.121569,0.466667,0.705882}%
\pgfsetstrokecolor{currentstroke}%
\pgfsetstrokeopacity{0.580412}%
\pgfsetdash{}{0pt}%
\pgfpathmoveto{\pgfqpoint{2.088878in}{2.172530in}}%
\pgfpathcurveto{\pgfqpoint{2.097114in}{2.172530in}}{\pgfqpoint{2.105014in}{2.175802in}}{\pgfqpoint{2.110838in}{2.181626in}}%
\pgfpathcurveto{\pgfqpoint{2.116662in}{2.187450in}}{\pgfqpoint{2.119934in}{2.195350in}}{\pgfqpoint{2.119934in}{2.203587in}}%
\pgfpathcurveto{\pgfqpoint{2.119934in}{2.211823in}}{\pgfqpoint{2.116662in}{2.219723in}}{\pgfqpoint{2.110838in}{2.225547in}}%
\pgfpathcurveto{\pgfqpoint{2.105014in}{2.231371in}}{\pgfqpoint{2.097114in}{2.234643in}}{\pgfqpoint{2.088878in}{2.234643in}}%
\pgfpathcurveto{\pgfqpoint{2.080642in}{2.234643in}}{\pgfqpoint{2.072742in}{2.231371in}}{\pgfqpoint{2.066918in}{2.225547in}}%
\pgfpathcurveto{\pgfqpoint{2.061094in}{2.219723in}}{\pgfqpoint{2.057821in}{2.211823in}}{\pgfqpoint{2.057821in}{2.203587in}}%
\pgfpathcurveto{\pgfqpoint{2.057821in}{2.195350in}}{\pgfqpoint{2.061094in}{2.187450in}}{\pgfqpoint{2.066918in}{2.181626in}}%
\pgfpathcurveto{\pgfqpoint{2.072742in}{2.175802in}}{\pgfqpoint{2.080642in}{2.172530in}}{\pgfqpoint{2.088878in}{2.172530in}}%
\pgfpathclose%
\pgfusepath{stroke,fill}%
\end{pgfscope}%
\begin{pgfscope}%
\pgfpathrectangle{\pgfqpoint{0.100000in}{0.212622in}}{\pgfqpoint{3.696000in}{3.696000in}}%
\pgfusepath{clip}%
\pgfsetbuttcap%
\pgfsetroundjoin%
\definecolor{currentfill}{rgb}{0.121569,0.466667,0.705882}%
\pgfsetfillcolor{currentfill}%
\pgfsetfillopacity{0.580454}%
\pgfsetlinewidth{1.003750pt}%
\definecolor{currentstroke}{rgb}{0.121569,0.466667,0.705882}%
\pgfsetstrokecolor{currentstroke}%
\pgfsetstrokeopacity{0.580454}%
\pgfsetdash{}{0pt}%
\pgfpathmoveto{\pgfqpoint{1.012759in}{1.780331in}}%
\pgfpathcurveto{\pgfqpoint{1.020995in}{1.780331in}}{\pgfqpoint{1.028895in}{1.783604in}}{\pgfqpoint{1.034719in}{1.789428in}}%
\pgfpathcurveto{\pgfqpoint{1.040543in}{1.795252in}}{\pgfqpoint{1.043815in}{1.803152in}}{\pgfqpoint{1.043815in}{1.811388in}}%
\pgfpathcurveto{\pgfqpoint{1.043815in}{1.819624in}}{\pgfqpoint{1.040543in}{1.827524in}}{\pgfqpoint{1.034719in}{1.833348in}}%
\pgfpathcurveto{\pgfqpoint{1.028895in}{1.839172in}}{\pgfqpoint{1.020995in}{1.842444in}}{\pgfqpoint{1.012759in}{1.842444in}}%
\pgfpathcurveto{\pgfqpoint{1.004522in}{1.842444in}}{\pgfqpoint{0.996622in}{1.839172in}}{\pgfqpoint{0.990798in}{1.833348in}}%
\pgfpathcurveto{\pgfqpoint{0.984975in}{1.827524in}}{\pgfqpoint{0.981702in}{1.819624in}}{\pgfqpoint{0.981702in}{1.811388in}}%
\pgfpathcurveto{\pgfqpoint{0.981702in}{1.803152in}}{\pgfqpoint{0.984975in}{1.795252in}}{\pgfqpoint{0.990798in}{1.789428in}}%
\pgfpathcurveto{\pgfqpoint{0.996622in}{1.783604in}}{\pgfqpoint{1.004522in}{1.780331in}}{\pgfqpoint{1.012759in}{1.780331in}}%
\pgfpathclose%
\pgfusepath{stroke,fill}%
\end{pgfscope}%
\begin{pgfscope}%
\pgfpathrectangle{\pgfqpoint{0.100000in}{0.212622in}}{\pgfqpoint{3.696000in}{3.696000in}}%
\pgfusepath{clip}%
\pgfsetbuttcap%
\pgfsetroundjoin%
\definecolor{currentfill}{rgb}{0.121569,0.466667,0.705882}%
\pgfsetfillcolor{currentfill}%
\pgfsetfillopacity{0.580479}%
\pgfsetlinewidth{1.003750pt}%
\definecolor{currentstroke}{rgb}{0.121569,0.466667,0.705882}%
\pgfsetstrokecolor{currentstroke}%
\pgfsetstrokeopacity{0.580479}%
\pgfsetdash{}{0pt}%
\pgfpathmoveto{\pgfqpoint{0.896820in}{1.514948in}}%
\pgfpathcurveto{\pgfqpoint{0.905056in}{1.514948in}}{\pgfqpoint{0.912956in}{1.518220in}}{\pgfqpoint{0.918780in}{1.524044in}}%
\pgfpathcurveto{\pgfqpoint{0.924604in}{1.529868in}}{\pgfqpoint{0.927876in}{1.537768in}}{\pgfqpoint{0.927876in}{1.546005in}}%
\pgfpathcurveto{\pgfqpoint{0.927876in}{1.554241in}}{\pgfqpoint{0.924604in}{1.562141in}}{\pgfqpoint{0.918780in}{1.567965in}}%
\pgfpathcurveto{\pgfqpoint{0.912956in}{1.573789in}}{\pgfqpoint{0.905056in}{1.577061in}}{\pgfqpoint{0.896820in}{1.577061in}}%
\pgfpathcurveto{\pgfqpoint{0.888584in}{1.577061in}}{\pgfqpoint{0.880684in}{1.573789in}}{\pgfqpoint{0.874860in}{1.567965in}}%
\pgfpathcurveto{\pgfqpoint{0.869036in}{1.562141in}}{\pgfqpoint{0.865763in}{1.554241in}}{\pgfqpoint{0.865763in}{1.546005in}}%
\pgfpathcurveto{\pgfqpoint{0.865763in}{1.537768in}}{\pgfqpoint{0.869036in}{1.529868in}}{\pgfqpoint{0.874860in}{1.524044in}}%
\pgfpathcurveto{\pgfqpoint{0.880684in}{1.518220in}}{\pgfqpoint{0.888584in}{1.514948in}}{\pgfqpoint{0.896820in}{1.514948in}}%
\pgfpathclose%
\pgfusepath{stroke,fill}%
\end{pgfscope}%
\begin{pgfscope}%
\pgfpathrectangle{\pgfqpoint{0.100000in}{0.212622in}}{\pgfqpoint{3.696000in}{3.696000in}}%
\pgfusepath{clip}%
\pgfsetbuttcap%
\pgfsetroundjoin%
\definecolor{currentfill}{rgb}{0.121569,0.466667,0.705882}%
\pgfsetfillcolor{currentfill}%
\pgfsetfillopacity{0.580543}%
\pgfsetlinewidth{1.003750pt}%
\definecolor{currentstroke}{rgb}{0.121569,0.466667,0.705882}%
\pgfsetstrokecolor{currentstroke}%
\pgfsetstrokeopacity{0.580543}%
\pgfsetdash{}{0pt}%
\pgfpathmoveto{\pgfqpoint{0.896742in}{1.514518in}}%
\pgfpathcurveto{\pgfqpoint{0.904979in}{1.514518in}}{\pgfqpoint{0.912879in}{1.517790in}}{\pgfqpoint{0.918703in}{1.523614in}}%
\pgfpathcurveto{\pgfqpoint{0.924527in}{1.529438in}}{\pgfqpoint{0.927799in}{1.537338in}}{\pgfqpoint{0.927799in}{1.545574in}}%
\pgfpathcurveto{\pgfqpoint{0.927799in}{1.553811in}}{\pgfqpoint{0.924527in}{1.561711in}}{\pgfqpoint{0.918703in}{1.567535in}}%
\pgfpathcurveto{\pgfqpoint{0.912879in}{1.573358in}}{\pgfqpoint{0.904979in}{1.576631in}}{\pgfqpoint{0.896742in}{1.576631in}}%
\pgfpathcurveto{\pgfqpoint{0.888506in}{1.576631in}}{\pgfqpoint{0.880606in}{1.573358in}}{\pgfqpoint{0.874782in}{1.567535in}}%
\pgfpathcurveto{\pgfqpoint{0.868958in}{1.561711in}}{\pgfqpoint{0.865686in}{1.553811in}}{\pgfqpoint{0.865686in}{1.545574in}}%
\pgfpathcurveto{\pgfqpoint{0.865686in}{1.537338in}}{\pgfqpoint{0.868958in}{1.529438in}}{\pgfqpoint{0.874782in}{1.523614in}}%
\pgfpathcurveto{\pgfqpoint{0.880606in}{1.517790in}}{\pgfqpoint{0.888506in}{1.514518in}}{\pgfqpoint{0.896742in}{1.514518in}}%
\pgfpathclose%
\pgfusepath{stroke,fill}%
\end{pgfscope}%
\begin{pgfscope}%
\pgfpathrectangle{\pgfqpoint{0.100000in}{0.212622in}}{\pgfqpoint{3.696000in}{3.696000in}}%
\pgfusepath{clip}%
\pgfsetbuttcap%
\pgfsetroundjoin%
\definecolor{currentfill}{rgb}{0.121569,0.466667,0.705882}%
\pgfsetfillcolor{currentfill}%
\pgfsetfillopacity{0.580672}%
\pgfsetlinewidth{1.003750pt}%
\definecolor{currentstroke}{rgb}{0.121569,0.466667,0.705882}%
\pgfsetstrokecolor{currentstroke}%
\pgfsetstrokeopacity{0.580672}%
\pgfsetdash{}{0pt}%
\pgfpathmoveto{\pgfqpoint{0.896573in}{1.513638in}}%
\pgfpathcurveto{\pgfqpoint{0.904810in}{1.513638in}}{\pgfqpoint{0.912710in}{1.516910in}}{\pgfqpoint{0.918534in}{1.522734in}}%
\pgfpathcurveto{\pgfqpoint{0.924358in}{1.528558in}}{\pgfqpoint{0.927630in}{1.536458in}}{\pgfqpoint{0.927630in}{1.544695in}}%
\pgfpathcurveto{\pgfqpoint{0.927630in}{1.552931in}}{\pgfqpoint{0.924358in}{1.560831in}}{\pgfqpoint{0.918534in}{1.566655in}}%
\pgfpathcurveto{\pgfqpoint{0.912710in}{1.572479in}}{\pgfqpoint{0.904810in}{1.575751in}}{\pgfqpoint{0.896573in}{1.575751in}}%
\pgfpathcurveto{\pgfqpoint{0.888337in}{1.575751in}}{\pgfqpoint{0.880437in}{1.572479in}}{\pgfqpoint{0.874613in}{1.566655in}}%
\pgfpathcurveto{\pgfqpoint{0.868789in}{1.560831in}}{\pgfqpoint{0.865517in}{1.552931in}}{\pgfqpoint{0.865517in}{1.544695in}}%
\pgfpathcurveto{\pgfqpoint{0.865517in}{1.536458in}}{\pgfqpoint{0.868789in}{1.528558in}}{\pgfqpoint{0.874613in}{1.522734in}}%
\pgfpathcurveto{\pgfqpoint{0.880437in}{1.516910in}}{\pgfqpoint{0.888337in}{1.513638in}}{\pgfqpoint{0.896573in}{1.513638in}}%
\pgfpathclose%
\pgfusepath{stroke,fill}%
\end{pgfscope}%
\begin{pgfscope}%
\pgfpathrectangle{\pgfqpoint{0.100000in}{0.212622in}}{\pgfqpoint{3.696000in}{3.696000in}}%
\pgfusepath{clip}%
\pgfsetbuttcap%
\pgfsetroundjoin%
\definecolor{currentfill}{rgb}{0.121569,0.466667,0.705882}%
\pgfsetfillcolor{currentfill}%
\pgfsetfillopacity{0.580745}%
\pgfsetlinewidth{1.003750pt}%
\definecolor{currentstroke}{rgb}{0.121569,0.466667,0.705882}%
\pgfsetstrokecolor{currentstroke}%
\pgfsetstrokeopacity{0.580745}%
\pgfsetdash{}{0pt}%
\pgfpathmoveto{\pgfqpoint{0.896484in}{1.513162in}}%
\pgfpathcurveto{\pgfqpoint{0.904720in}{1.513162in}}{\pgfqpoint{0.912620in}{1.516434in}}{\pgfqpoint{0.918444in}{1.522258in}}%
\pgfpathcurveto{\pgfqpoint{0.924268in}{1.528082in}}{\pgfqpoint{0.927540in}{1.535982in}}{\pgfqpoint{0.927540in}{1.544219in}}%
\pgfpathcurveto{\pgfqpoint{0.927540in}{1.552455in}}{\pgfqpoint{0.924268in}{1.560355in}}{\pgfqpoint{0.918444in}{1.566179in}}%
\pgfpathcurveto{\pgfqpoint{0.912620in}{1.572003in}}{\pgfqpoint{0.904720in}{1.575275in}}{\pgfqpoint{0.896484in}{1.575275in}}%
\pgfpathcurveto{\pgfqpoint{0.888248in}{1.575275in}}{\pgfqpoint{0.880348in}{1.572003in}}{\pgfqpoint{0.874524in}{1.566179in}}%
\pgfpathcurveto{\pgfqpoint{0.868700in}{1.560355in}}{\pgfqpoint{0.865427in}{1.552455in}}{\pgfqpoint{0.865427in}{1.544219in}}%
\pgfpathcurveto{\pgfqpoint{0.865427in}{1.535982in}}{\pgfqpoint{0.868700in}{1.528082in}}{\pgfqpoint{0.874524in}{1.522258in}}%
\pgfpathcurveto{\pgfqpoint{0.880348in}{1.516434in}}{\pgfqpoint{0.888248in}{1.513162in}}{\pgfqpoint{0.896484in}{1.513162in}}%
\pgfpathclose%
\pgfusepath{stroke,fill}%
\end{pgfscope}%
\begin{pgfscope}%
\pgfpathrectangle{\pgfqpoint{0.100000in}{0.212622in}}{\pgfqpoint{3.696000in}{3.696000in}}%
\pgfusepath{clip}%
\pgfsetbuttcap%
\pgfsetroundjoin%
\definecolor{currentfill}{rgb}{0.121569,0.466667,0.705882}%
\pgfsetfillcolor{currentfill}%
\pgfsetfillopacity{0.580878}%
\pgfsetlinewidth{1.003750pt}%
\definecolor{currentstroke}{rgb}{0.121569,0.466667,0.705882}%
\pgfsetstrokecolor{currentstroke}%
\pgfsetstrokeopacity{0.580878}%
\pgfsetdash{}{0pt}%
\pgfpathmoveto{\pgfqpoint{0.896322in}{1.512306in}}%
\pgfpathcurveto{\pgfqpoint{0.904558in}{1.512306in}}{\pgfqpoint{0.912458in}{1.515578in}}{\pgfqpoint{0.918282in}{1.521402in}}%
\pgfpathcurveto{\pgfqpoint{0.924106in}{1.527226in}}{\pgfqpoint{0.927378in}{1.535126in}}{\pgfqpoint{0.927378in}{1.543363in}}%
\pgfpathcurveto{\pgfqpoint{0.927378in}{1.551599in}}{\pgfqpoint{0.924106in}{1.559499in}}{\pgfqpoint{0.918282in}{1.565323in}}%
\pgfpathcurveto{\pgfqpoint{0.912458in}{1.571147in}}{\pgfqpoint{0.904558in}{1.574419in}}{\pgfqpoint{0.896322in}{1.574419in}}%
\pgfpathcurveto{\pgfqpoint{0.888086in}{1.574419in}}{\pgfqpoint{0.880186in}{1.571147in}}{\pgfqpoint{0.874362in}{1.565323in}}%
\pgfpathcurveto{\pgfqpoint{0.868538in}{1.559499in}}{\pgfqpoint{0.865265in}{1.551599in}}{\pgfqpoint{0.865265in}{1.543363in}}%
\pgfpathcurveto{\pgfqpoint{0.865265in}{1.535126in}}{\pgfqpoint{0.868538in}{1.527226in}}{\pgfqpoint{0.874362in}{1.521402in}}%
\pgfpathcurveto{\pgfqpoint{0.880186in}{1.515578in}}{\pgfqpoint{0.888086in}{1.512306in}}{\pgfqpoint{0.896322in}{1.512306in}}%
\pgfpathclose%
\pgfusepath{stroke,fill}%
\end{pgfscope}%
\begin{pgfscope}%
\pgfpathrectangle{\pgfqpoint{0.100000in}{0.212622in}}{\pgfqpoint{3.696000in}{3.696000in}}%
\pgfusepath{clip}%
\pgfsetbuttcap%
\pgfsetroundjoin%
\definecolor{currentfill}{rgb}{0.121569,0.466667,0.705882}%
\pgfsetfillcolor{currentfill}%
\pgfsetfillopacity{0.581072}%
\pgfsetlinewidth{1.003750pt}%
\definecolor{currentstroke}{rgb}{0.121569,0.466667,0.705882}%
\pgfsetstrokecolor{currentstroke}%
\pgfsetstrokeopacity{0.581072}%
\pgfsetdash{}{0pt}%
\pgfpathmoveto{\pgfqpoint{0.896068in}{1.511051in}}%
\pgfpathcurveto{\pgfqpoint{0.904304in}{1.511051in}}{\pgfqpoint{0.912204in}{1.514323in}}{\pgfqpoint{0.918028in}{1.520147in}}%
\pgfpathcurveto{\pgfqpoint{0.923852in}{1.525971in}}{\pgfqpoint{0.927125in}{1.533871in}}{\pgfqpoint{0.927125in}{1.542107in}}%
\pgfpathcurveto{\pgfqpoint{0.927125in}{1.550344in}}{\pgfqpoint{0.923852in}{1.558244in}}{\pgfqpoint{0.918028in}{1.564068in}}%
\pgfpathcurveto{\pgfqpoint{0.912204in}{1.569892in}}{\pgfqpoint{0.904304in}{1.573164in}}{\pgfqpoint{0.896068in}{1.573164in}}%
\pgfpathcurveto{\pgfqpoint{0.887832in}{1.573164in}}{\pgfqpoint{0.879932in}{1.569892in}}{\pgfqpoint{0.874108in}{1.564068in}}%
\pgfpathcurveto{\pgfqpoint{0.868284in}{1.558244in}}{\pgfqpoint{0.865012in}{1.550344in}}{\pgfqpoint{0.865012in}{1.542107in}}%
\pgfpathcurveto{\pgfqpoint{0.865012in}{1.533871in}}{\pgfqpoint{0.868284in}{1.525971in}}{\pgfqpoint{0.874108in}{1.520147in}}%
\pgfpathcurveto{\pgfqpoint{0.879932in}{1.514323in}}{\pgfqpoint{0.887832in}{1.511051in}}{\pgfqpoint{0.896068in}{1.511051in}}%
\pgfpathclose%
\pgfusepath{stroke,fill}%
\end{pgfscope}%
\begin{pgfscope}%
\pgfpathrectangle{\pgfqpoint{0.100000in}{0.212622in}}{\pgfqpoint{3.696000in}{3.696000in}}%
\pgfusepath{clip}%
\pgfsetbuttcap%
\pgfsetroundjoin%
\definecolor{currentfill}{rgb}{0.121569,0.466667,0.705882}%
\pgfsetfillcolor{currentfill}%
\pgfsetfillopacity{0.581179}%
\pgfsetlinewidth{1.003750pt}%
\definecolor{currentstroke}{rgb}{0.121569,0.466667,0.705882}%
\pgfsetstrokecolor{currentstroke}%
\pgfsetstrokeopacity{0.581179}%
\pgfsetdash{}{0pt}%
\pgfpathmoveto{\pgfqpoint{0.895928in}{1.510360in}}%
\pgfpathcurveto{\pgfqpoint{0.904165in}{1.510360in}}{\pgfqpoint{0.912065in}{1.513632in}}{\pgfqpoint{0.917889in}{1.519456in}}%
\pgfpathcurveto{\pgfqpoint{0.923713in}{1.525280in}}{\pgfqpoint{0.926985in}{1.533180in}}{\pgfqpoint{0.926985in}{1.541416in}}%
\pgfpathcurveto{\pgfqpoint{0.926985in}{1.549652in}}{\pgfqpoint{0.923713in}{1.557552in}}{\pgfqpoint{0.917889in}{1.563376in}}%
\pgfpathcurveto{\pgfqpoint{0.912065in}{1.569200in}}{\pgfqpoint{0.904165in}{1.572473in}}{\pgfqpoint{0.895928in}{1.572473in}}%
\pgfpathcurveto{\pgfqpoint{0.887692in}{1.572473in}}{\pgfqpoint{0.879792in}{1.569200in}}{\pgfqpoint{0.873968in}{1.563376in}}%
\pgfpathcurveto{\pgfqpoint{0.868144in}{1.557552in}}{\pgfqpoint{0.864872in}{1.549652in}}{\pgfqpoint{0.864872in}{1.541416in}}%
\pgfpathcurveto{\pgfqpoint{0.864872in}{1.533180in}}{\pgfqpoint{0.868144in}{1.525280in}}{\pgfqpoint{0.873968in}{1.519456in}}%
\pgfpathcurveto{\pgfqpoint{0.879792in}{1.513632in}}{\pgfqpoint{0.887692in}{1.510360in}}{\pgfqpoint{0.895928in}{1.510360in}}%
\pgfpathclose%
\pgfusepath{stroke,fill}%
\end{pgfscope}%
\begin{pgfscope}%
\pgfpathrectangle{\pgfqpoint{0.100000in}{0.212622in}}{\pgfqpoint{3.696000in}{3.696000in}}%
\pgfusepath{clip}%
\pgfsetbuttcap%
\pgfsetroundjoin%
\definecolor{currentfill}{rgb}{0.121569,0.466667,0.705882}%
\pgfsetfillcolor{currentfill}%
\pgfsetfillopacity{0.581346}%
\pgfsetlinewidth{1.003750pt}%
\definecolor{currentstroke}{rgb}{0.121569,0.466667,0.705882}%
\pgfsetstrokecolor{currentstroke}%
\pgfsetstrokeopacity{0.581346}%
\pgfsetdash{}{0pt}%
\pgfpathmoveto{\pgfqpoint{0.895706in}{1.509289in}}%
\pgfpathcurveto{\pgfqpoint{0.903942in}{1.509289in}}{\pgfqpoint{0.911842in}{1.512561in}}{\pgfqpoint{0.917666in}{1.518385in}}%
\pgfpathcurveto{\pgfqpoint{0.923490in}{1.524209in}}{\pgfqpoint{0.926763in}{1.532109in}}{\pgfqpoint{0.926763in}{1.540345in}}%
\pgfpathcurveto{\pgfqpoint{0.926763in}{1.548581in}}{\pgfqpoint{0.923490in}{1.556482in}}{\pgfqpoint{0.917666in}{1.562305in}}%
\pgfpathcurveto{\pgfqpoint{0.911842in}{1.568129in}}{\pgfqpoint{0.903942in}{1.571402in}}{\pgfqpoint{0.895706in}{1.571402in}}%
\pgfpathcurveto{\pgfqpoint{0.887470in}{1.571402in}}{\pgfqpoint{0.879570in}{1.568129in}}{\pgfqpoint{0.873746in}{1.562305in}}%
\pgfpathcurveto{\pgfqpoint{0.867922in}{1.556482in}}{\pgfqpoint{0.864650in}{1.548581in}}{\pgfqpoint{0.864650in}{1.540345in}}%
\pgfpathcurveto{\pgfqpoint{0.864650in}{1.532109in}}{\pgfqpoint{0.867922in}{1.524209in}}{\pgfqpoint{0.873746in}{1.518385in}}%
\pgfpathcurveto{\pgfqpoint{0.879570in}{1.512561in}}{\pgfqpoint{0.887470in}{1.509289in}}{\pgfqpoint{0.895706in}{1.509289in}}%
\pgfpathclose%
\pgfusepath{stroke,fill}%
\end{pgfscope}%
\begin{pgfscope}%
\pgfpathrectangle{\pgfqpoint{0.100000in}{0.212622in}}{\pgfqpoint{3.696000in}{3.696000in}}%
\pgfusepath{clip}%
\pgfsetbuttcap%
\pgfsetroundjoin%
\definecolor{currentfill}{rgb}{0.121569,0.466667,0.705882}%
\pgfsetfillcolor{currentfill}%
\pgfsetfillopacity{0.581581}%
\pgfsetlinewidth{1.003750pt}%
\definecolor{currentstroke}{rgb}{0.121569,0.466667,0.705882}%
\pgfsetstrokecolor{currentstroke}%
\pgfsetstrokeopacity{0.581581}%
\pgfsetdash{}{0pt}%
\pgfpathmoveto{\pgfqpoint{0.895404in}{1.507825in}}%
\pgfpathcurveto{\pgfqpoint{0.903641in}{1.507825in}}{\pgfqpoint{0.911541in}{1.511097in}}{\pgfqpoint{0.917365in}{1.516921in}}%
\pgfpathcurveto{\pgfqpoint{0.923189in}{1.522745in}}{\pgfqpoint{0.926461in}{1.530645in}}{\pgfqpoint{0.926461in}{1.538881in}}%
\pgfpathcurveto{\pgfqpoint{0.926461in}{1.547117in}}{\pgfqpoint{0.923189in}{1.555017in}}{\pgfqpoint{0.917365in}{1.560841in}}%
\pgfpathcurveto{\pgfqpoint{0.911541in}{1.566665in}}{\pgfqpoint{0.903641in}{1.569938in}}{\pgfqpoint{0.895404in}{1.569938in}}%
\pgfpathcurveto{\pgfqpoint{0.887168in}{1.569938in}}{\pgfqpoint{0.879268in}{1.566665in}}{\pgfqpoint{0.873444in}{1.560841in}}%
\pgfpathcurveto{\pgfqpoint{0.867620in}{1.555017in}}{\pgfqpoint{0.864348in}{1.547117in}}{\pgfqpoint{0.864348in}{1.538881in}}%
\pgfpathcurveto{\pgfqpoint{0.864348in}{1.530645in}}{\pgfqpoint{0.867620in}{1.522745in}}{\pgfqpoint{0.873444in}{1.516921in}}%
\pgfpathcurveto{\pgfqpoint{0.879268in}{1.511097in}}{\pgfqpoint{0.887168in}{1.507825in}}{\pgfqpoint{0.895404in}{1.507825in}}%
\pgfpathclose%
\pgfusepath{stroke,fill}%
\end{pgfscope}%
\begin{pgfscope}%
\pgfpathrectangle{\pgfqpoint{0.100000in}{0.212622in}}{\pgfqpoint{3.696000in}{3.696000in}}%
\pgfusepath{clip}%
\pgfsetbuttcap%
\pgfsetroundjoin%
\definecolor{currentfill}{rgb}{0.121569,0.466667,0.705882}%
\pgfsetfillcolor{currentfill}%
\pgfsetfillopacity{0.581594}%
\pgfsetlinewidth{1.003750pt}%
\definecolor{currentstroke}{rgb}{0.121569,0.466667,0.705882}%
\pgfsetstrokecolor{currentstroke}%
\pgfsetstrokeopacity{0.581594}%
\pgfsetdash{}{0pt}%
\pgfpathmoveto{\pgfqpoint{0.881102in}{1.561504in}}%
\pgfpathcurveto{\pgfqpoint{0.889338in}{1.561504in}}{\pgfqpoint{0.897238in}{1.564777in}}{\pgfqpoint{0.903062in}{1.570601in}}%
\pgfpathcurveto{\pgfqpoint{0.908886in}{1.576424in}}{\pgfqpoint{0.912159in}{1.584325in}}{\pgfqpoint{0.912159in}{1.592561in}}%
\pgfpathcurveto{\pgfqpoint{0.912159in}{1.600797in}}{\pgfqpoint{0.908886in}{1.608697in}}{\pgfqpoint{0.903062in}{1.614521in}}%
\pgfpathcurveto{\pgfqpoint{0.897238in}{1.620345in}}{\pgfqpoint{0.889338in}{1.623617in}}{\pgfqpoint{0.881102in}{1.623617in}}%
\pgfpathcurveto{\pgfqpoint{0.872866in}{1.623617in}}{\pgfqpoint{0.864966in}{1.620345in}}{\pgfqpoint{0.859142in}{1.614521in}}%
\pgfpathcurveto{\pgfqpoint{0.853318in}{1.608697in}}{\pgfqpoint{0.850046in}{1.600797in}}{\pgfqpoint{0.850046in}{1.592561in}}%
\pgfpathcurveto{\pgfqpoint{0.850046in}{1.584325in}}{\pgfqpoint{0.853318in}{1.576424in}}{\pgfqpoint{0.859142in}{1.570601in}}%
\pgfpathcurveto{\pgfqpoint{0.864966in}{1.564777in}}{\pgfqpoint{0.872866in}{1.561504in}}{\pgfqpoint{0.881102in}{1.561504in}}%
\pgfpathclose%
\pgfusepath{stroke,fill}%
\end{pgfscope}%
\begin{pgfscope}%
\pgfpathrectangle{\pgfqpoint{0.100000in}{0.212622in}}{\pgfqpoint{3.696000in}{3.696000in}}%
\pgfusepath{clip}%
\pgfsetbuttcap%
\pgfsetroundjoin%
\definecolor{currentfill}{rgb}{0.121569,0.466667,0.705882}%
\pgfsetfillcolor{currentfill}%
\pgfsetfillopacity{0.581717}%
\pgfsetlinewidth{1.003750pt}%
\definecolor{currentstroke}{rgb}{0.121569,0.466667,0.705882}%
\pgfsetstrokecolor{currentstroke}%
\pgfsetstrokeopacity{0.581717}%
\pgfsetdash{}{0pt}%
\pgfpathmoveto{\pgfqpoint{0.895242in}{1.507044in}}%
\pgfpathcurveto{\pgfqpoint{0.903478in}{1.507044in}}{\pgfqpoint{0.911378in}{1.510316in}}{\pgfqpoint{0.917202in}{1.516140in}}%
\pgfpathcurveto{\pgfqpoint{0.923026in}{1.521964in}}{\pgfqpoint{0.926299in}{1.529864in}}{\pgfqpoint{0.926299in}{1.538100in}}%
\pgfpathcurveto{\pgfqpoint{0.926299in}{1.546336in}}{\pgfqpoint{0.923026in}{1.554236in}}{\pgfqpoint{0.917202in}{1.560060in}}%
\pgfpathcurveto{\pgfqpoint{0.911378in}{1.565884in}}{\pgfqpoint{0.903478in}{1.569157in}}{\pgfqpoint{0.895242in}{1.569157in}}%
\pgfpathcurveto{\pgfqpoint{0.887006in}{1.569157in}}{\pgfqpoint{0.879106in}{1.565884in}}{\pgfqpoint{0.873282in}{1.560060in}}%
\pgfpathcurveto{\pgfqpoint{0.867458in}{1.554236in}}{\pgfqpoint{0.864186in}{1.546336in}}{\pgfqpoint{0.864186in}{1.538100in}}%
\pgfpathcurveto{\pgfqpoint{0.864186in}{1.529864in}}{\pgfqpoint{0.867458in}{1.521964in}}{\pgfqpoint{0.873282in}{1.516140in}}%
\pgfpathcurveto{\pgfqpoint{0.879106in}{1.510316in}}{\pgfqpoint{0.887006in}{1.507044in}}{\pgfqpoint{0.895242in}{1.507044in}}%
\pgfpathclose%
\pgfusepath{stroke,fill}%
\end{pgfscope}%
\begin{pgfscope}%
\pgfpathrectangle{\pgfqpoint{0.100000in}{0.212622in}}{\pgfqpoint{3.696000in}{3.696000in}}%
\pgfusepath{clip}%
\pgfsetbuttcap%
\pgfsetroundjoin%
\definecolor{currentfill}{rgb}{0.121569,0.466667,0.705882}%
\pgfsetfillcolor{currentfill}%
\pgfsetfillopacity{0.581796}%
\pgfsetlinewidth{1.003750pt}%
\definecolor{currentstroke}{rgb}{0.121569,0.466667,0.705882}%
\pgfsetstrokecolor{currentstroke}%
\pgfsetstrokeopacity{0.581796}%
\pgfsetdash{}{0pt}%
\pgfpathmoveto{\pgfqpoint{0.895153in}{1.506629in}}%
\pgfpathcurveto{\pgfqpoint{0.903389in}{1.506629in}}{\pgfqpoint{0.911289in}{1.509901in}}{\pgfqpoint{0.917113in}{1.515725in}}%
\pgfpathcurveto{\pgfqpoint{0.922937in}{1.521549in}}{\pgfqpoint{0.926209in}{1.529449in}}{\pgfqpoint{0.926209in}{1.537685in}}%
\pgfpathcurveto{\pgfqpoint{0.926209in}{1.545921in}}{\pgfqpoint{0.922937in}{1.553821in}}{\pgfqpoint{0.917113in}{1.559645in}}%
\pgfpathcurveto{\pgfqpoint{0.911289in}{1.565469in}}{\pgfqpoint{0.903389in}{1.568742in}}{\pgfqpoint{0.895153in}{1.568742in}}%
\pgfpathcurveto{\pgfqpoint{0.886916in}{1.568742in}}{\pgfqpoint{0.879016in}{1.565469in}}{\pgfqpoint{0.873192in}{1.559645in}}%
\pgfpathcurveto{\pgfqpoint{0.867368in}{1.553821in}}{\pgfqpoint{0.864096in}{1.545921in}}{\pgfqpoint{0.864096in}{1.537685in}}%
\pgfpathcurveto{\pgfqpoint{0.864096in}{1.529449in}}{\pgfqpoint{0.867368in}{1.521549in}}{\pgfqpoint{0.873192in}{1.515725in}}%
\pgfpathcurveto{\pgfqpoint{0.879016in}{1.509901in}}{\pgfqpoint{0.886916in}{1.506629in}}{\pgfqpoint{0.895153in}{1.506629in}}%
\pgfpathclose%
\pgfusepath{stroke,fill}%
\end{pgfscope}%
\begin{pgfscope}%
\pgfpathrectangle{\pgfqpoint{0.100000in}{0.212622in}}{\pgfqpoint{3.696000in}{3.696000in}}%
\pgfusepath{clip}%
\pgfsetbuttcap%
\pgfsetroundjoin%
\definecolor{currentfill}{rgb}{0.121569,0.466667,0.705882}%
\pgfsetfillcolor{currentfill}%
\pgfsetfillopacity{0.581840}%
\pgfsetlinewidth{1.003750pt}%
\definecolor{currentstroke}{rgb}{0.121569,0.466667,0.705882}%
\pgfsetstrokecolor{currentstroke}%
\pgfsetstrokeopacity{0.581840}%
\pgfsetdash{}{0pt}%
\pgfpathmoveto{\pgfqpoint{0.895102in}{1.506404in}}%
\pgfpathcurveto{\pgfqpoint{0.903338in}{1.506404in}}{\pgfqpoint{0.911238in}{1.509676in}}{\pgfqpoint{0.917062in}{1.515500in}}%
\pgfpathcurveto{\pgfqpoint{0.922886in}{1.521324in}}{\pgfqpoint{0.926159in}{1.529224in}}{\pgfqpoint{0.926159in}{1.537460in}}%
\pgfpathcurveto{\pgfqpoint{0.926159in}{1.545697in}}{\pgfqpoint{0.922886in}{1.553597in}}{\pgfqpoint{0.917062in}{1.559421in}}%
\pgfpathcurveto{\pgfqpoint{0.911238in}{1.565245in}}{\pgfqpoint{0.903338in}{1.568517in}}{\pgfqpoint{0.895102in}{1.568517in}}%
\pgfpathcurveto{\pgfqpoint{0.886866in}{1.568517in}}{\pgfqpoint{0.878966in}{1.565245in}}{\pgfqpoint{0.873142in}{1.559421in}}%
\pgfpathcurveto{\pgfqpoint{0.867318in}{1.553597in}}{\pgfqpoint{0.864046in}{1.545697in}}{\pgfqpoint{0.864046in}{1.537460in}}%
\pgfpathcurveto{\pgfqpoint{0.864046in}{1.529224in}}{\pgfqpoint{0.867318in}{1.521324in}}{\pgfqpoint{0.873142in}{1.515500in}}%
\pgfpathcurveto{\pgfqpoint{0.878966in}{1.509676in}}{\pgfqpoint{0.886866in}{1.506404in}}{\pgfqpoint{0.895102in}{1.506404in}}%
\pgfpathclose%
\pgfusepath{stroke,fill}%
\end{pgfscope}%
\begin{pgfscope}%
\pgfpathrectangle{\pgfqpoint{0.100000in}{0.212622in}}{\pgfqpoint{3.696000in}{3.696000in}}%
\pgfusepath{clip}%
\pgfsetbuttcap%
\pgfsetroundjoin%
\definecolor{currentfill}{rgb}{0.121569,0.466667,0.705882}%
\pgfsetfillcolor{currentfill}%
\pgfsetfillopacity{0.581963}%
\pgfsetlinewidth{1.003750pt}%
\definecolor{currentstroke}{rgb}{0.121569,0.466667,0.705882}%
\pgfsetstrokecolor{currentstroke}%
\pgfsetstrokeopacity{0.581963}%
\pgfsetdash{}{0pt}%
\pgfpathmoveto{\pgfqpoint{0.894965in}{1.505805in}}%
\pgfpathcurveto{\pgfqpoint{0.903201in}{1.505805in}}{\pgfqpoint{0.911101in}{1.509077in}}{\pgfqpoint{0.916925in}{1.514901in}}%
\pgfpathcurveto{\pgfqpoint{0.922749in}{1.520725in}}{\pgfqpoint{0.926021in}{1.528625in}}{\pgfqpoint{0.926021in}{1.536862in}}%
\pgfpathcurveto{\pgfqpoint{0.926021in}{1.545098in}}{\pgfqpoint{0.922749in}{1.552998in}}{\pgfqpoint{0.916925in}{1.558822in}}%
\pgfpathcurveto{\pgfqpoint{0.911101in}{1.564646in}}{\pgfqpoint{0.903201in}{1.567918in}}{\pgfqpoint{0.894965in}{1.567918in}}%
\pgfpathcurveto{\pgfqpoint{0.886729in}{1.567918in}}{\pgfqpoint{0.878829in}{1.564646in}}{\pgfqpoint{0.873005in}{1.558822in}}%
\pgfpathcurveto{\pgfqpoint{0.867181in}{1.552998in}}{\pgfqpoint{0.863908in}{1.545098in}}{\pgfqpoint{0.863908in}{1.536862in}}%
\pgfpathcurveto{\pgfqpoint{0.863908in}{1.528625in}}{\pgfqpoint{0.867181in}{1.520725in}}{\pgfqpoint{0.873005in}{1.514901in}}%
\pgfpathcurveto{\pgfqpoint{0.878829in}{1.509077in}}{\pgfqpoint{0.886729in}{1.505805in}}{\pgfqpoint{0.894965in}{1.505805in}}%
\pgfpathclose%
\pgfusepath{stroke,fill}%
\end{pgfscope}%
\begin{pgfscope}%
\pgfpathrectangle{\pgfqpoint{0.100000in}{0.212622in}}{\pgfqpoint{3.696000in}{3.696000in}}%
\pgfusepath{clip}%
\pgfsetbuttcap%
\pgfsetroundjoin%
\definecolor{currentfill}{rgb}{0.121569,0.466667,0.705882}%
\pgfsetfillcolor{currentfill}%
\pgfsetfillopacity{0.582033}%
\pgfsetlinewidth{1.003750pt}%
\definecolor{currentstroke}{rgb}{0.121569,0.466667,0.705882}%
\pgfsetstrokecolor{currentstroke}%
\pgfsetstrokeopacity{0.582033}%
\pgfsetdash{}{0pt}%
\pgfpathmoveto{\pgfqpoint{0.894893in}{1.505480in}}%
\pgfpathcurveto{\pgfqpoint{0.903129in}{1.505480in}}{\pgfqpoint{0.911029in}{1.508753in}}{\pgfqpoint{0.916853in}{1.514576in}}%
\pgfpathcurveto{\pgfqpoint{0.922677in}{1.520400in}}{\pgfqpoint{0.925949in}{1.528300in}}{\pgfqpoint{0.925949in}{1.536537in}}%
\pgfpathcurveto{\pgfqpoint{0.925949in}{1.544773in}}{\pgfqpoint{0.922677in}{1.552673in}}{\pgfqpoint{0.916853in}{1.558497in}}%
\pgfpathcurveto{\pgfqpoint{0.911029in}{1.564321in}}{\pgfqpoint{0.903129in}{1.567593in}}{\pgfqpoint{0.894893in}{1.567593in}}%
\pgfpathcurveto{\pgfqpoint{0.886657in}{1.567593in}}{\pgfqpoint{0.878757in}{1.564321in}}{\pgfqpoint{0.872933in}{1.558497in}}%
\pgfpathcurveto{\pgfqpoint{0.867109in}{1.552673in}}{\pgfqpoint{0.863836in}{1.544773in}}{\pgfqpoint{0.863836in}{1.536537in}}%
\pgfpathcurveto{\pgfqpoint{0.863836in}{1.528300in}}{\pgfqpoint{0.867109in}{1.520400in}}{\pgfqpoint{0.872933in}{1.514576in}}%
\pgfpathcurveto{\pgfqpoint{0.878757in}{1.508753in}}{\pgfqpoint{0.886657in}{1.505480in}}{\pgfqpoint{0.894893in}{1.505480in}}%
\pgfpathclose%
\pgfusepath{stroke,fill}%
\end{pgfscope}%
\begin{pgfscope}%
\pgfpathrectangle{\pgfqpoint{0.100000in}{0.212622in}}{\pgfqpoint{3.696000in}{3.696000in}}%
\pgfusepath{clip}%
\pgfsetbuttcap%
\pgfsetroundjoin%
\definecolor{currentfill}{rgb}{0.121569,0.466667,0.705882}%
\pgfsetfillcolor{currentfill}%
\pgfsetfillopacity{0.582080}%
\pgfsetlinewidth{1.003750pt}%
\definecolor{currentstroke}{rgb}{0.121569,0.466667,0.705882}%
\pgfsetstrokecolor{currentstroke}%
\pgfsetstrokeopacity{0.582080}%
\pgfsetdash{}{0pt}%
\pgfpathmoveto{\pgfqpoint{1.008194in}{1.772355in}}%
\pgfpathcurveto{\pgfqpoint{1.016430in}{1.772355in}}{\pgfqpoint{1.024330in}{1.775628in}}{\pgfqpoint{1.030154in}{1.781452in}}%
\pgfpathcurveto{\pgfqpoint{1.035978in}{1.787276in}}{\pgfqpoint{1.039250in}{1.795176in}}{\pgfqpoint{1.039250in}{1.803412in}}%
\pgfpathcurveto{\pgfqpoint{1.039250in}{1.811648in}}{\pgfqpoint{1.035978in}{1.819548in}}{\pgfqpoint{1.030154in}{1.825372in}}%
\pgfpathcurveto{\pgfqpoint{1.024330in}{1.831196in}}{\pgfqpoint{1.016430in}{1.834468in}}{\pgfqpoint{1.008194in}{1.834468in}}%
\pgfpathcurveto{\pgfqpoint{0.999958in}{1.834468in}}{\pgfqpoint{0.992058in}{1.831196in}}{\pgfqpoint{0.986234in}{1.825372in}}%
\pgfpathcurveto{\pgfqpoint{0.980410in}{1.819548in}}{\pgfqpoint{0.977137in}{1.811648in}}{\pgfqpoint{0.977137in}{1.803412in}}%
\pgfpathcurveto{\pgfqpoint{0.977137in}{1.795176in}}{\pgfqpoint{0.980410in}{1.787276in}}{\pgfqpoint{0.986234in}{1.781452in}}%
\pgfpathcurveto{\pgfqpoint{0.992058in}{1.775628in}}{\pgfqpoint{0.999958in}{1.772355in}}{\pgfqpoint{1.008194in}{1.772355in}}%
\pgfpathclose%
\pgfusepath{stroke,fill}%
\end{pgfscope}%
\begin{pgfscope}%
\pgfpathrectangle{\pgfqpoint{0.100000in}{0.212622in}}{\pgfqpoint{3.696000in}{3.696000in}}%
\pgfusepath{clip}%
\pgfsetbuttcap%
\pgfsetroundjoin%
\definecolor{currentfill}{rgb}{0.121569,0.466667,0.705882}%
\pgfsetfillcolor{currentfill}%
\pgfsetfillopacity{0.582212}%
\pgfsetlinewidth{1.003750pt}%
\definecolor{currentstroke}{rgb}{0.121569,0.466667,0.705882}%
\pgfsetstrokecolor{currentstroke}%
\pgfsetstrokeopacity{0.582212}%
\pgfsetdash{}{0pt}%
\pgfpathmoveto{\pgfqpoint{0.894706in}{1.504659in}}%
\pgfpathcurveto{\pgfqpoint{0.902942in}{1.504659in}}{\pgfqpoint{0.910842in}{1.507931in}}{\pgfqpoint{0.916666in}{1.513755in}}%
\pgfpathcurveto{\pgfqpoint{0.922490in}{1.519579in}}{\pgfqpoint{0.925762in}{1.527479in}}{\pgfqpoint{0.925762in}{1.535715in}}%
\pgfpathcurveto{\pgfqpoint{0.925762in}{1.543951in}}{\pgfqpoint{0.922490in}{1.551851in}}{\pgfqpoint{0.916666in}{1.557675in}}%
\pgfpathcurveto{\pgfqpoint{0.910842in}{1.563499in}}{\pgfqpoint{0.902942in}{1.566772in}}{\pgfqpoint{0.894706in}{1.566772in}}%
\pgfpathcurveto{\pgfqpoint{0.886469in}{1.566772in}}{\pgfqpoint{0.878569in}{1.563499in}}{\pgfqpoint{0.872745in}{1.557675in}}%
\pgfpathcurveto{\pgfqpoint{0.866921in}{1.551851in}}{\pgfqpoint{0.863649in}{1.543951in}}{\pgfqpoint{0.863649in}{1.535715in}}%
\pgfpathcurveto{\pgfqpoint{0.863649in}{1.527479in}}{\pgfqpoint{0.866921in}{1.519579in}}{\pgfqpoint{0.872745in}{1.513755in}}%
\pgfpathcurveto{\pgfqpoint{0.878569in}{1.507931in}}{\pgfqpoint{0.886469in}{1.504659in}}{\pgfqpoint{0.894706in}{1.504659in}}%
\pgfpathclose%
\pgfusepath{stroke,fill}%
\end{pgfscope}%
\begin{pgfscope}%
\pgfpathrectangle{\pgfqpoint{0.100000in}{0.212622in}}{\pgfqpoint{3.696000in}{3.696000in}}%
\pgfusepath{clip}%
\pgfsetbuttcap%
\pgfsetroundjoin%
\definecolor{currentfill}{rgb}{0.121569,0.466667,0.705882}%
\pgfsetfillcolor{currentfill}%
\pgfsetfillopacity{0.582310}%
\pgfsetlinewidth{1.003750pt}%
\definecolor{currentstroke}{rgb}{0.121569,0.466667,0.705882}%
\pgfsetstrokecolor{currentstroke}%
\pgfsetstrokeopacity{0.582310}%
\pgfsetdash{}{0pt}%
\pgfpathmoveto{\pgfqpoint{0.894595in}{1.504205in}}%
\pgfpathcurveto{\pgfqpoint{0.902831in}{1.504205in}}{\pgfqpoint{0.910731in}{1.507477in}}{\pgfqpoint{0.916555in}{1.513301in}}%
\pgfpathcurveto{\pgfqpoint{0.922379in}{1.519125in}}{\pgfqpoint{0.925651in}{1.527025in}}{\pgfqpoint{0.925651in}{1.535261in}}%
\pgfpathcurveto{\pgfqpoint{0.925651in}{1.543497in}}{\pgfqpoint{0.922379in}{1.551397in}}{\pgfqpoint{0.916555in}{1.557221in}}%
\pgfpathcurveto{\pgfqpoint{0.910731in}{1.563045in}}{\pgfqpoint{0.902831in}{1.566318in}}{\pgfqpoint{0.894595in}{1.566318in}}%
\pgfpathcurveto{\pgfqpoint{0.886359in}{1.566318in}}{\pgfqpoint{0.878459in}{1.563045in}}{\pgfqpoint{0.872635in}{1.557221in}}%
\pgfpathcurveto{\pgfqpoint{0.866811in}{1.551397in}}{\pgfqpoint{0.863538in}{1.543497in}}{\pgfqpoint{0.863538in}{1.535261in}}%
\pgfpathcurveto{\pgfqpoint{0.863538in}{1.527025in}}{\pgfqpoint{0.866811in}{1.519125in}}{\pgfqpoint{0.872635in}{1.513301in}}%
\pgfpathcurveto{\pgfqpoint{0.878459in}{1.507477in}}{\pgfqpoint{0.886359in}{1.504205in}}{\pgfqpoint{0.894595in}{1.504205in}}%
\pgfpathclose%
\pgfusepath{stroke,fill}%
\end{pgfscope}%
\begin{pgfscope}%
\pgfpathrectangle{\pgfqpoint{0.100000in}{0.212622in}}{\pgfqpoint{3.696000in}{3.696000in}}%
\pgfusepath{clip}%
\pgfsetbuttcap%
\pgfsetroundjoin%
\definecolor{currentfill}{rgb}{0.121569,0.466667,0.705882}%
\pgfsetfillcolor{currentfill}%
\pgfsetfillopacity{0.582502}%
\pgfsetlinewidth{1.003750pt}%
\definecolor{currentstroke}{rgb}{0.121569,0.466667,0.705882}%
\pgfsetstrokecolor{currentstroke}%
\pgfsetstrokeopacity{0.582502}%
\pgfsetdash{}{0pt}%
\pgfpathmoveto{\pgfqpoint{0.894390in}{1.503340in}}%
\pgfpathcurveto{\pgfqpoint{0.902627in}{1.503340in}}{\pgfqpoint{0.910527in}{1.506612in}}{\pgfqpoint{0.916351in}{1.512436in}}%
\pgfpathcurveto{\pgfqpoint{0.922174in}{1.518260in}}{\pgfqpoint{0.925447in}{1.526160in}}{\pgfqpoint{0.925447in}{1.534396in}}%
\pgfpathcurveto{\pgfqpoint{0.925447in}{1.542632in}}{\pgfqpoint{0.922174in}{1.550532in}}{\pgfqpoint{0.916351in}{1.556356in}}%
\pgfpathcurveto{\pgfqpoint{0.910527in}{1.562180in}}{\pgfqpoint{0.902627in}{1.565453in}}{\pgfqpoint{0.894390in}{1.565453in}}%
\pgfpathcurveto{\pgfqpoint{0.886154in}{1.565453in}}{\pgfqpoint{0.878254in}{1.562180in}}{\pgfqpoint{0.872430in}{1.556356in}}%
\pgfpathcurveto{\pgfqpoint{0.866606in}{1.550532in}}{\pgfqpoint{0.863334in}{1.542632in}}{\pgfqpoint{0.863334in}{1.534396in}}%
\pgfpathcurveto{\pgfqpoint{0.863334in}{1.526160in}}{\pgfqpoint{0.866606in}{1.518260in}}{\pgfqpoint{0.872430in}{1.512436in}}%
\pgfpathcurveto{\pgfqpoint{0.878254in}{1.506612in}}{\pgfqpoint{0.886154in}{1.503340in}}{\pgfqpoint{0.894390in}{1.503340in}}%
\pgfpathclose%
\pgfusepath{stroke,fill}%
\end{pgfscope}%
\begin{pgfscope}%
\pgfpathrectangle{\pgfqpoint{0.100000in}{0.212622in}}{\pgfqpoint{3.696000in}{3.696000in}}%
\pgfusepath{clip}%
\pgfsetbuttcap%
\pgfsetroundjoin%
\definecolor{currentfill}{rgb}{0.121569,0.466667,0.705882}%
\pgfsetfillcolor{currentfill}%
\pgfsetfillopacity{0.582809}%
\pgfsetlinewidth{1.003750pt}%
\definecolor{currentstroke}{rgb}{0.121569,0.466667,0.705882}%
\pgfsetstrokecolor{currentstroke}%
\pgfsetstrokeopacity{0.582809}%
\pgfsetdash{}{0pt}%
\pgfpathmoveto{\pgfqpoint{0.894090in}{1.502049in}}%
\pgfpathcurveto{\pgfqpoint{0.902326in}{1.502049in}}{\pgfqpoint{0.910226in}{1.505321in}}{\pgfqpoint{0.916050in}{1.511145in}}%
\pgfpathcurveto{\pgfqpoint{0.921874in}{1.516969in}}{\pgfqpoint{0.925146in}{1.524869in}}{\pgfqpoint{0.925146in}{1.533105in}}%
\pgfpathcurveto{\pgfqpoint{0.925146in}{1.541342in}}{\pgfqpoint{0.921874in}{1.549242in}}{\pgfqpoint{0.916050in}{1.555066in}}%
\pgfpathcurveto{\pgfqpoint{0.910226in}{1.560889in}}{\pgfqpoint{0.902326in}{1.564162in}}{\pgfqpoint{0.894090in}{1.564162in}}%
\pgfpathcurveto{\pgfqpoint{0.885854in}{1.564162in}}{\pgfqpoint{0.877954in}{1.560889in}}{\pgfqpoint{0.872130in}{1.555066in}}%
\pgfpathcurveto{\pgfqpoint{0.866306in}{1.549242in}}{\pgfqpoint{0.863033in}{1.541342in}}{\pgfqpoint{0.863033in}{1.533105in}}%
\pgfpathcurveto{\pgfqpoint{0.863033in}{1.524869in}}{\pgfqpoint{0.866306in}{1.516969in}}{\pgfqpoint{0.872130in}{1.511145in}}%
\pgfpathcurveto{\pgfqpoint{0.877954in}{1.505321in}}{\pgfqpoint{0.885854in}{1.502049in}}{\pgfqpoint{0.894090in}{1.502049in}}%
\pgfpathclose%
\pgfusepath{stroke,fill}%
\end{pgfscope}%
\begin{pgfscope}%
\pgfpathrectangle{\pgfqpoint{0.100000in}{0.212622in}}{\pgfqpoint{3.696000in}{3.696000in}}%
\pgfusepath{clip}%
\pgfsetbuttcap%
\pgfsetroundjoin%
\definecolor{currentfill}{rgb}{0.121569,0.466667,0.705882}%
\pgfsetfillcolor{currentfill}%
\pgfsetfillopacity{0.582863}%
\pgfsetlinewidth{1.003750pt}%
\definecolor{currentstroke}{rgb}{0.121569,0.466667,0.705882}%
\pgfsetstrokecolor{currentstroke}%
\pgfsetstrokeopacity{0.582863}%
\pgfsetdash{}{0pt}%
\pgfpathmoveto{\pgfqpoint{2.090247in}{2.163047in}}%
\pgfpathcurveto{\pgfqpoint{2.098484in}{2.163047in}}{\pgfqpoint{2.106384in}{2.166319in}}{\pgfqpoint{2.112208in}{2.172143in}}%
\pgfpathcurveto{\pgfqpoint{2.118032in}{2.177967in}}{\pgfqpoint{2.121304in}{2.185867in}}{\pgfqpoint{2.121304in}{2.194103in}}%
\pgfpathcurveto{\pgfqpoint{2.121304in}{2.202340in}}{\pgfqpoint{2.118032in}{2.210240in}}{\pgfqpoint{2.112208in}{2.216064in}}%
\pgfpathcurveto{\pgfqpoint{2.106384in}{2.221888in}}{\pgfqpoint{2.098484in}{2.225160in}}{\pgfqpoint{2.090247in}{2.225160in}}%
\pgfpathcurveto{\pgfqpoint{2.082011in}{2.225160in}}{\pgfqpoint{2.074111in}{2.221888in}}{\pgfqpoint{2.068287in}{2.216064in}}%
\pgfpathcurveto{\pgfqpoint{2.062463in}{2.210240in}}{\pgfqpoint{2.059191in}{2.202340in}}{\pgfqpoint{2.059191in}{2.194103in}}%
\pgfpathcurveto{\pgfqpoint{2.059191in}{2.185867in}}{\pgfqpoint{2.062463in}{2.177967in}}{\pgfqpoint{2.068287in}{2.172143in}}%
\pgfpathcurveto{\pgfqpoint{2.074111in}{2.166319in}}{\pgfqpoint{2.082011in}{2.163047in}}{\pgfqpoint{2.090247in}{2.163047in}}%
\pgfpathclose%
\pgfusepath{stroke,fill}%
\end{pgfscope}%
\begin{pgfscope}%
\pgfpathrectangle{\pgfqpoint{0.100000in}{0.212622in}}{\pgfqpoint{3.696000in}{3.696000in}}%
\pgfusepath{clip}%
\pgfsetbuttcap%
\pgfsetroundjoin%
\definecolor{currentfill}{rgb}{0.121569,0.466667,0.705882}%
\pgfsetfillcolor{currentfill}%
\pgfsetfillopacity{0.583261}%
\pgfsetlinewidth{1.003750pt}%
\definecolor{currentstroke}{rgb}{0.121569,0.466667,0.705882}%
\pgfsetstrokecolor{currentstroke}%
\pgfsetstrokeopacity{0.583261}%
\pgfsetdash{}{0pt}%
\pgfpathmoveto{\pgfqpoint{0.893656in}{1.500188in}}%
\pgfpathcurveto{\pgfqpoint{0.901893in}{1.500188in}}{\pgfqpoint{0.909793in}{1.503461in}}{\pgfqpoint{0.915617in}{1.509284in}}%
\pgfpathcurveto{\pgfqpoint{0.921440in}{1.515108in}}{\pgfqpoint{0.924713in}{1.523008in}}{\pgfqpoint{0.924713in}{1.531245in}}%
\pgfpathcurveto{\pgfqpoint{0.924713in}{1.539481in}}{\pgfqpoint{0.921440in}{1.547381in}}{\pgfqpoint{0.915617in}{1.553205in}}%
\pgfpathcurveto{\pgfqpoint{0.909793in}{1.559029in}}{\pgfqpoint{0.901893in}{1.562301in}}{\pgfqpoint{0.893656in}{1.562301in}}%
\pgfpathcurveto{\pgfqpoint{0.885420in}{1.562301in}}{\pgfqpoint{0.877520in}{1.559029in}}{\pgfqpoint{0.871696in}{1.553205in}}%
\pgfpathcurveto{\pgfqpoint{0.865872in}{1.547381in}}{\pgfqpoint{0.862600in}{1.539481in}}{\pgfqpoint{0.862600in}{1.531245in}}%
\pgfpathcurveto{\pgfqpoint{0.862600in}{1.523008in}}{\pgfqpoint{0.865872in}{1.515108in}}{\pgfqpoint{0.871696in}{1.509284in}}%
\pgfpathcurveto{\pgfqpoint{0.877520in}{1.503461in}}{\pgfqpoint{0.885420in}{1.500188in}}{\pgfqpoint{0.893656in}{1.500188in}}%
\pgfpathclose%
\pgfusepath{stroke,fill}%
\end{pgfscope}%
\begin{pgfscope}%
\pgfpathrectangle{\pgfqpoint{0.100000in}{0.212622in}}{\pgfqpoint{3.696000in}{3.696000in}}%
\pgfusepath{clip}%
\pgfsetbuttcap%
\pgfsetroundjoin%
\definecolor{currentfill}{rgb}{0.121569,0.466667,0.705882}%
\pgfsetfillcolor{currentfill}%
\pgfsetfillopacity{0.583874}%
\pgfsetlinewidth{1.003750pt}%
\definecolor{currentstroke}{rgb}{0.121569,0.466667,0.705882}%
\pgfsetstrokecolor{currentstroke}%
\pgfsetstrokeopacity{0.583874}%
\pgfsetdash{}{0pt}%
\pgfpathmoveto{\pgfqpoint{0.893073in}{1.497753in}}%
\pgfpathcurveto{\pgfqpoint{0.901310in}{1.497753in}}{\pgfqpoint{0.909210in}{1.501025in}}{\pgfqpoint{0.915033in}{1.506849in}}%
\pgfpathcurveto{\pgfqpoint{0.920857in}{1.512673in}}{\pgfqpoint{0.924130in}{1.520573in}}{\pgfqpoint{0.924130in}{1.528810in}}%
\pgfpathcurveto{\pgfqpoint{0.924130in}{1.537046in}}{\pgfqpoint{0.920857in}{1.544946in}}{\pgfqpoint{0.915033in}{1.550770in}}%
\pgfpathcurveto{\pgfqpoint{0.909210in}{1.556594in}}{\pgfqpoint{0.901310in}{1.559866in}}{\pgfqpoint{0.893073in}{1.559866in}}%
\pgfpathcurveto{\pgfqpoint{0.884837in}{1.559866in}}{\pgfqpoint{0.876937in}{1.556594in}}{\pgfqpoint{0.871113in}{1.550770in}}%
\pgfpathcurveto{\pgfqpoint{0.865289in}{1.544946in}}{\pgfqpoint{0.862017in}{1.537046in}}{\pgfqpoint{0.862017in}{1.528810in}}%
\pgfpathcurveto{\pgfqpoint{0.862017in}{1.520573in}}{\pgfqpoint{0.865289in}{1.512673in}}{\pgfqpoint{0.871113in}{1.506849in}}%
\pgfpathcurveto{\pgfqpoint{0.876937in}{1.501025in}}{\pgfqpoint{0.884837in}{1.497753in}}{\pgfqpoint{0.893073in}{1.497753in}}%
\pgfpathclose%
\pgfusepath{stroke,fill}%
\end{pgfscope}%
\begin{pgfscope}%
\pgfpathrectangle{\pgfqpoint{0.100000in}{0.212622in}}{\pgfqpoint{3.696000in}{3.696000in}}%
\pgfusepath{clip}%
\pgfsetbuttcap%
\pgfsetroundjoin%
\definecolor{currentfill}{rgb}{0.121569,0.466667,0.705882}%
\pgfsetfillcolor{currentfill}%
\pgfsetfillopacity{0.584188}%
\pgfsetlinewidth{1.003750pt}%
\definecolor{currentstroke}{rgb}{0.121569,0.466667,0.705882}%
\pgfsetstrokecolor{currentstroke}%
\pgfsetstrokeopacity{0.584188}%
\pgfsetdash{}{0pt}%
\pgfpathmoveto{\pgfqpoint{0.892740in}{1.496332in}}%
\pgfpathcurveto{\pgfqpoint{0.900977in}{1.496332in}}{\pgfqpoint{0.908877in}{1.499604in}}{\pgfqpoint{0.914701in}{1.505428in}}%
\pgfpathcurveto{\pgfqpoint{0.920525in}{1.511252in}}{\pgfqpoint{0.923797in}{1.519152in}}{\pgfqpoint{0.923797in}{1.527388in}}%
\pgfpathcurveto{\pgfqpoint{0.923797in}{1.535625in}}{\pgfqpoint{0.920525in}{1.543525in}}{\pgfqpoint{0.914701in}{1.549349in}}%
\pgfpathcurveto{\pgfqpoint{0.908877in}{1.555173in}}{\pgfqpoint{0.900977in}{1.558445in}}{\pgfqpoint{0.892740in}{1.558445in}}%
\pgfpathcurveto{\pgfqpoint{0.884504in}{1.558445in}}{\pgfqpoint{0.876604in}{1.555173in}}{\pgfqpoint{0.870780in}{1.549349in}}%
\pgfpathcurveto{\pgfqpoint{0.864956in}{1.543525in}}{\pgfqpoint{0.861684in}{1.535625in}}{\pgfqpoint{0.861684in}{1.527388in}}%
\pgfpathcurveto{\pgfqpoint{0.861684in}{1.519152in}}{\pgfqpoint{0.864956in}{1.511252in}}{\pgfqpoint{0.870780in}{1.505428in}}%
\pgfpathcurveto{\pgfqpoint{0.876604in}{1.499604in}}{\pgfqpoint{0.884504in}{1.496332in}}{\pgfqpoint{0.892740in}{1.496332in}}%
\pgfpathclose%
\pgfusepath{stroke,fill}%
\end{pgfscope}%
\begin{pgfscope}%
\pgfpathrectangle{\pgfqpoint{0.100000in}{0.212622in}}{\pgfqpoint{3.696000in}{3.696000in}}%
\pgfusepath{clip}%
\pgfsetbuttcap%
\pgfsetroundjoin%
\definecolor{currentfill}{rgb}{0.121569,0.466667,0.705882}%
\pgfsetfillcolor{currentfill}%
\pgfsetfillopacity{0.584296}%
\pgfsetlinewidth{1.003750pt}%
\definecolor{currentstroke}{rgb}{0.121569,0.466667,0.705882}%
\pgfsetstrokecolor{currentstroke}%
\pgfsetstrokeopacity{0.584296}%
\pgfsetdash{}{0pt}%
\pgfpathmoveto{\pgfqpoint{0.876605in}{1.568903in}}%
\pgfpathcurveto{\pgfqpoint{0.884841in}{1.568903in}}{\pgfqpoint{0.892741in}{1.572175in}}{\pgfqpoint{0.898565in}{1.577999in}}%
\pgfpathcurveto{\pgfqpoint{0.904389in}{1.583823in}}{\pgfqpoint{0.907661in}{1.591723in}}{\pgfqpoint{0.907661in}{1.599959in}}%
\pgfpathcurveto{\pgfqpoint{0.907661in}{1.608196in}}{\pgfqpoint{0.904389in}{1.616096in}}{\pgfqpoint{0.898565in}{1.621920in}}%
\pgfpathcurveto{\pgfqpoint{0.892741in}{1.627744in}}{\pgfqpoint{0.884841in}{1.631016in}}{\pgfqpoint{0.876605in}{1.631016in}}%
\pgfpathcurveto{\pgfqpoint{0.868368in}{1.631016in}}{\pgfqpoint{0.860468in}{1.627744in}}{\pgfqpoint{0.854645in}{1.621920in}}%
\pgfpathcurveto{\pgfqpoint{0.848821in}{1.616096in}}{\pgfqpoint{0.845548in}{1.608196in}}{\pgfqpoint{0.845548in}{1.599959in}}%
\pgfpathcurveto{\pgfqpoint{0.845548in}{1.591723in}}{\pgfqpoint{0.848821in}{1.583823in}}{\pgfqpoint{0.854645in}{1.577999in}}%
\pgfpathcurveto{\pgfqpoint{0.860468in}{1.572175in}}{\pgfqpoint{0.868368in}{1.568903in}}{\pgfqpoint{0.876605in}{1.568903in}}%
\pgfpathclose%
\pgfusepath{stroke,fill}%
\end{pgfscope}%
\begin{pgfscope}%
\pgfpathrectangle{\pgfqpoint{0.100000in}{0.212622in}}{\pgfqpoint{3.696000in}{3.696000in}}%
\pgfusepath{clip}%
\pgfsetbuttcap%
\pgfsetroundjoin%
\definecolor{currentfill}{rgb}{0.121569,0.466667,0.705882}%
\pgfsetfillcolor{currentfill}%
\pgfsetfillopacity{0.584354}%
\pgfsetlinewidth{1.003750pt}%
\definecolor{currentstroke}{rgb}{0.121569,0.466667,0.705882}%
\pgfsetstrokecolor{currentstroke}%
\pgfsetstrokeopacity{0.584354}%
\pgfsetdash{}{0pt}%
\pgfpathmoveto{\pgfqpoint{0.892550in}{1.495522in}}%
\pgfpathcurveto{\pgfqpoint{0.900787in}{1.495522in}}{\pgfqpoint{0.908687in}{1.498794in}}{\pgfqpoint{0.914511in}{1.504618in}}%
\pgfpathcurveto{\pgfqpoint{0.920334in}{1.510442in}}{\pgfqpoint{0.923607in}{1.518342in}}{\pgfqpoint{0.923607in}{1.526578in}}%
\pgfpathcurveto{\pgfqpoint{0.923607in}{1.534814in}}{\pgfqpoint{0.920334in}{1.542715in}}{\pgfqpoint{0.914511in}{1.548538in}}%
\pgfpathcurveto{\pgfqpoint{0.908687in}{1.554362in}}{\pgfqpoint{0.900787in}{1.557635in}}{\pgfqpoint{0.892550in}{1.557635in}}%
\pgfpathcurveto{\pgfqpoint{0.884314in}{1.557635in}}{\pgfqpoint{0.876414in}{1.554362in}}{\pgfqpoint{0.870590in}{1.548538in}}%
\pgfpathcurveto{\pgfqpoint{0.864766in}{1.542715in}}{\pgfqpoint{0.861494in}{1.534814in}}{\pgfqpoint{0.861494in}{1.526578in}}%
\pgfpathcurveto{\pgfqpoint{0.861494in}{1.518342in}}{\pgfqpoint{0.864766in}{1.510442in}}{\pgfqpoint{0.870590in}{1.504618in}}%
\pgfpathcurveto{\pgfqpoint{0.876414in}{1.498794in}}{\pgfqpoint{0.884314in}{1.495522in}}{\pgfqpoint{0.892550in}{1.495522in}}%
\pgfpathclose%
\pgfusepath{stroke,fill}%
\end{pgfscope}%
\begin{pgfscope}%
\pgfpathrectangle{\pgfqpoint{0.100000in}{0.212622in}}{\pgfqpoint{3.696000in}{3.696000in}}%
\pgfusepath{clip}%
\pgfsetbuttcap%
\pgfsetroundjoin%
\definecolor{currentfill}{rgb}{0.121569,0.466667,0.705882}%
\pgfsetfillcolor{currentfill}%
\pgfsetfillopacity{0.584441}%
\pgfsetlinewidth{1.003750pt}%
\definecolor{currentstroke}{rgb}{0.121569,0.466667,0.705882}%
\pgfsetstrokecolor{currentstroke}%
\pgfsetstrokeopacity{0.584441}%
\pgfsetdash{}{0pt}%
\pgfpathmoveto{\pgfqpoint{0.892445in}{1.495063in}}%
\pgfpathcurveto{\pgfqpoint{0.900681in}{1.495063in}}{\pgfqpoint{0.908581in}{1.498335in}}{\pgfqpoint{0.914405in}{1.504159in}}%
\pgfpathcurveto{\pgfqpoint{0.920229in}{1.509983in}}{\pgfqpoint{0.923501in}{1.517883in}}{\pgfqpoint{0.923501in}{1.526119in}}%
\pgfpathcurveto{\pgfqpoint{0.923501in}{1.534356in}}{\pgfqpoint{0.920229in}{1.542256in}}{\pgfqpoint{0.914405in}{1.548080in}}%
\pgfpathcurveto{\pgfqpoint{0.908581in}{1.553904in}}{\pgfqpoint{0.900681in}{1.557176in}}{\pgfqpoint{0.892445in}{1.557176in}}%
\pgfpathcurveto{\pgfqpoint{0.884208in}{1.557176in}}{\pgfqpoint{0.876308in}{1.553904in}}{\pgfqpoint{0.870484in}{1.548080in}}%
\pgfpathcurveto{\pgfqpoint{0.864660in}{1.542256in}}{\pgfqpoint{0.861388in}{1.534356in}}{\pgfqpoint{0.861388in}{1.526119in}}%
\pgfpathcurveto{\pgfqpoint{0.861388in}{1.517883in}}{\pgfqpoint{0.864660in}{1.509983in}}{\pgfqpoint{0.870484in}{1.504159in}}%
\pgfpathcurveto{\pgfqpoint{0.876308in}{1.498335in}}{\pgfqpoint{0.884208in}{1.495063in}}{\pgfqpoint{0.892445in}{1.495063in}}%
\pgfpathclose%
\pgfusepath{stroke,fill}%
\end{pgfscope}%
\begin{pgfscope}%
\pgfpathrectangle{\pgfqpoint{0.100000in}{0.212622in}}{\pgfqpoint{3.696000in}{3.696000in}}%
\pgfusepath{clip}%
\pgfsetbuttcap%
\pgfsetroundjoin%
\definecolor{currentfill}{rgb}{0.121569,0.466667,0.705882}%
\pgfsetfillcolor{currentfill}%
\pgfsetfillopacity{0.584488}%
\pgfsetlinewidth{1.003750pt}%
\definecolor{currentstroke}{rgb}{0.121569,0.466667,0.705882}%
\pgfsetstrokecolor{currentstroke}%
\pgfsetstrokeopacity{0.584488}%
\pgfsetdash{}{0pt}%
\pgfpathmoveto{\pgfqpoint{0.892389in}{1.494805in}}%
\pgfpathcurveto{\pgfqpoint{0.900625in}{1.494805in}}{\pgfqpoint{0.908525in}{1.498078in}}{\pgfqpoint{0.914349in}{1.503902in}}%
\pgfpathcurveto{\pgfqpoint{0.920173in}{1.509726in}}{\pgfqpoint{0.923445in}{1.517626in}}{\pgfqpoint{0.923445in}{1.525862in}}%
\pgfpathcurveto{\pgfqpoint{0.923445in}{1.534098in}}{\pgfqpoint{0.920173in}{1.541998in}}{\pgfqpoint{0.914349in}{1.547822in}}%
\pgfpathcurveto{\pgfqpoint{0.908525in}{1.553646in}}{\pgfqpoint{0.900625in}{1.556918in}}{\pgfqpoint{0.892389in}{1.556918in}}%
\pgfpathcurveto{\pgfqpoint{0.884153in}{1.556918in}}{\pgfqpoint{0.876253in}{1.553646in}}{\pgfqpoint{0.870429in}{1.547822in}}%
\pgfpathcurveto{\pgfqpoint{0.864605in}{1.541998in}}{\pgfqpoint{0.861332in}{1.534098in}}{\pgfqpoint{0.861332in}{1.525862in}}%
\pgfpathcurveto{\pgfqpoint{0.861332in}{1.517626in}}{\pgfqpoint{0.864605in}{1.509726in}}{\pgfqpoint{0.870429in}{1.503902in}}%
\pgfpathcurveto{\pgfqpoint{0.876253in}{1.498078in}}{\pgfqpoint{0.884153in}{1.494805in}}{\pgfqpoint{0.892389in}{1.494805in}}%
\pgfpathclose%
\pgfusepath{stroke,fill}%
\end{pgfscope}%
\begin{pgfscope}%
\pgfpathrectangle{\pgfqpoint{0.100000in}{0.212622in}}{\pgfqpoint{3.696000in}{3.696000in}}%
\pgfusepath{clip}%
\pgfsetbuttcap%
\pgfsetroundjoin%
\definecolor{currentfill}{rgb}{0.121569,0.466667,0.705882}%
\pgfsetfillcolor{currentfill}%
\pgfsetfillopacity{0.584513}%
\pgfsetlinewidth{1.003750pt}%
\definecolor{currentstroke}{rgb}{0.121569,0.466667,0.705882}%
\pgfsetstrokecolor{currentstroke}%
\pgfsetstrokeopacity{0.584513}%
\pgfsetdash{}{0pt}%
\pgfpathmoveto{\pgfqpoint{0.892359in}{1.494662in}}%
\pgfpathcurveto{\pgfqpoint{0.900595in}{1.494662in}}{\pgfqpoint{0.908495in}{1.497935in}}{\pgfqpoint{0.914319in}{1.503759in}}%
\pgfpathcurveto{\pgfqpoint{0.920143in}{1.509583in}}{\pgfqpoint{0.923415in}{1.517483in}}{\pgfqpoint{0.923415in}{1.525719in}}%
\pgfpathcurveto{\pgfqpoint{0.923415in}{1.533955in}}{\pgfqpoint{0.920143in}{1.541855in}}{\pgfqpoint{0.914319in}{1.547679in}}%
\pgfpathcurveto{\pgfqpoint{0.908495in}{1.553503in}}{\pgfqpoint{0.900595in}{1.556775in}}{\pgfqpoint{0.892359in}{1.556775in}}%
\pgfpathcurveto{\pgfqpoint{0.884122in}{1.556775in}}{\pgfqpoint{0.876222in}{1.553503in}}{\pgfqpoint{0.870398in}{1.547679in}}%
\pgfpathcurveto{\pgfqpoint{0.864575in}{1.541855in}}{\pgfqpoint{0.861302in}{1.533955in}}{\pgfqpoint{0.861302in}{1.525719in}}%
\pgfpathcurveto{\pgfqpoint{0.861302in}{1.517483in}}{\pgfqpoint{0.864575in}{1.509583in}}{\pgfqpoint{0.870398in}{1.503759in}}%
\pgfpathcurveto{\pgfqpoint{0.876222in}{1.497935in}}{\pgfqpoint{0.884122in}{1.494662in}}{\pgfqpoint{0.892359in}{1.494662in}}%
\pgfpathclose%
\pgfusepath{stroke,fill}%
\end{pgfscope}%
\begin{pgfscope}%
\pgfpathrectangle{\pgfqpoint{0.100000in}{0.212622in}}{\pgfqpoint{3.696000in}{3.696000in}}%
\pgfusepath{clip}%
\pgfsetbuttcap%
\pgfsetroundjoin%
\definecolor{currentfill}{rgb}{0.121569,0.466667,0.705882}%
\pgfsetfillcolor{currentfill}%
\pgfsetfillopacity{0.584527}%
\pgfsetlinewidth{1.003750pt}%
\definecolor{currentstroke}{rgb}{0.121569,0.466667,0.705882}%
\pgfsetstrokecolor{currentstroke}%
\pgfsetstrokeopacity{0.584527}%
\pgfsetdash{}{0pt}%
\pgfpathmoveto{\pgfqpoint{0.892342in}{1.494584in}}%
\pgfpathcurveto{\pgfqpoint{0.900578in}{1.494584in}}{\pgfqpoint{0.908478in}{1.497856in}}{\pgfqpoint{0.914302in}{1.503680in}}%
\pgfpathcurveto{\pgfqpoint{0.920126in}{1.509504in}}{\pgfqpoint{0.923399in}{1.517404in}}{\pgfqpoint{0.923399in}{1.525640in}}%
\pgfpathcurveto{\pgfqpoint{0.923399in}{1.533876in}}{\pgfqpoint{0.920126in}{1.541777in}}{\pgfqpoint{0.914302in}{1.547600in}}%
\pgfpathcurveto{\pgfqpoint{0.908478in}{1.553424in}}{\pgfqpoint{0.900578in}{1.556697in}}{\pgfqpoint{0.892342in}{1.556697in}}%
\pgfpathcurveto{\pgfqpoint{0.884106in}{1.556697in}}{\pgfqpoint{0.876206in}{1.553424in}}{\pgfqpoint{0.870382in}{1.547600in}}%
\pgfpathcurveto{\pgfqpoint{0.864558in}{1.541777in}}{\pgfqpoint{0.861286in}{1.533876in}}{\pgfqpoint{0.861286in}{1.525640in}}%
\pgfpathcurveto{\pgfqpoint{0.861286in}{1.517404in}}{\pgfqpoint{0.864558in}{1.509504in}}{\pgfqpoint{0.870382in}{1.503680in}}%
\pgfpathcurveto{\pgfqpoint{0.876206in}{1.497856in}}{\pgfqpoint{0.884106in}{1.494584in}}{\pgfqpoint{0.892342in}{1.494584in}}%
\pgfpathclose%
\pgfusepath{stroke,fill}%
\end{pgfscope}%
\begin{pgfscope}%
\pgfpathrectangle{\pgfqpoint{0.100000in}{0.212622in}}{\pgfqpoint{3.696000in}{3.696000in}}%
\pgfusepath{clip}%
\pgfsetbuttcap%
\pgfsetroundjoin%
\definecolor{currentfill}{rgb}{0.121569,0.466667,0.705882}%
\pgfsetfillcolor{currentfill}%
\pgfsetfillopacity{0.584535}%
\pgfsetlinewidth{1.003750pt}%
\definecolor{currentstroke}{rgb}{0.121569,0.466667,0.705882}%
\pgfsetstrokecolor{currentstroke}%
\pgfsetstrokeopacity{0.584535}%
\pgfsetdash{}{0pt}%
\pgfpathmoveto{\pgfqpoint{0.892332in}{1.494540in}}%
\pgfpathcurveto{\pgfqpoint{0.900568in}{1.494540in}}{\pgfqpoint{0.908468in}{1.497812in}}{\pgfqpoint{0.914292in}{1.503636in}}%
\pgfpathcurveto{\pgfqpoint{0.920116in}{1.509460in}}{\pgfqpoint{0.923389in}{1.517360in}}{\pgfqpoint{0.923389in}{1.525596in}}%
\pgfpathcurveto{\pgfqpoint{0.923389in}{1.533833in}}{\pgfqpoint{0.920116in}{1.541733in}}{\pgfqpoint{0.914292in}{1.547557in}}%
\pgfpathcurveto{\pgfqpoint{0.908468in}{1.553381in}}{\pgfqpoint{0.900568in}{1.556653in}}{\pgfqpoint{0.892332in}{1.556653in}}%
\pgfpathcurveto{\pgfqpoint{0.884096in}{1.556653in}}{\pgfqpoint{0.876196in}{1.553381in}}{\pgfqpoint{0.870372in}{1.547557in}}%
\pgfpathcurveto{\pgfqpoint{0.864548in}{1.541733in}}{\pgfqpoint{0.861276in}{1.533833in}}{\pgfqpoint{0.861276in}{1.525596in}}%
\pgfpathcurveto{\pgfqpoint{0.861276in}{1.517360in}}{\pgfqpoint{0.864548in}{1.509460in}}{\pgfqpoint{0.870372in}{1.503636in}}%
\pgfpathcurveto{\pgfqpoint{0.876196in}{1.497812in}}{\pgfqpoint{0.884096in}{1.494540in}}{\pgfqpoint{0.892332in}{1.494540in}}%
\pgfpathclose%
\pgfusepath{stroke,fill}%
\end{pgfscope}%
\begin{pgfscope}%
\pgfpathrectangle{\pgfqpoint{0.100000in}{0.212622in}}{\pgfqpoint{3.696000in}{3.696000in}}%
\pgfusepath{clip}%
\pgfsetbuttcap%
\pgfsetroundjoin%
\definecolor{currentfill}{rgb}{0.121569,0.466667,0.705882}%
\pgfsetfillcolor{currentfill}%
\pgfsetfillopacity{0.584539}%
\pgfsetlinewidth{1.003750pt}%
\definecolor{currentstroke}{rgb}{0.121569,0.466667,0.705882}%
\pgfsetstrokecolor{currentstroke}%
\pgfsetstrokeopacity{0.584539}%
\pgfsetdash{}{0pt}%
\pgfpathmoveto{\pgfqpoint{0.892327in}{1.494516in}}%
\pgfpathcurveto{\pgfqpoint{0.900563in}{1.494516in}}{\pgfqpoint{0.908463in}{1.497788in}}{\pgfqpoint{0.914287in}{1.503612in}}%
\pgfpathcurveto{\pgfqpoint{0.920111in}{1.509436in}}{\pgfqpoint{0.923383in}{1.517336in}}{\pgfqpoint{0.923383in}{1.525572in}}%
\pgfpathcurveto{\pgfqpoint{0.923383in}{1.533809in}}{\pgfqpoint{0.920111in}{1.541709in}}{\pgfqpoint{0.914287in}{1.547533in}}%
\pgfpathcurveto{\pgfqpoint{0.908463in}{1.553356in}}{\pgfqpoint{0.900563in}{1.556629in}}{\pgfqpoint{0.892327in}{1.556629in}}%
\pgfpathcurveto{\pgfqpoint{0.884091in}{1.556629in}}{\pgfqpoint{0.876191in}{1.553356in}}{\pgfqpoint{0.870367in}{1.547533in}}%
\pgfpathcurveto{\pgfqpoint{0.864543in}{1.541709in}}{\pgfqpoint{0.861270in}{1.533809in}}{\pgfqpoint{0.861270in}{1.525572in}}%
\pgfpathcurveto{\pgfqpoint{0.861270in}{1.517336in}}{\pgfqpoint{0.864543in}{1.509436in}}{\pgfqpoint{0.870367in}{1.503612in}}%
\pgfpathcurveto{\pgfqpoint{0.876191in}{1.497788in}}{\pgfqpoint{0.884091in}{1.494516in}}{\pgfqpoint{0.892327in}{1.494516in}}%
\pgfpathclose%
\pgfusepath{stroke,fill}%
\end{pgfscope}%
\begin{pgfscope}%
\pgfpathrectangle{\pgfqpoint{0.100000in}{0.212622in}}{\pgfqpoint{3.696000in}{3.696000in}}%
\pgfusepath{clip}%
\pgfsetbuttcap%
\pgfsetroundjoin%
\definecolor{currentfill}{rgb}{0.121569,0.466667,0.705882}%
\pgfsetfillcolor{currentfill}%
\pgfsetfillopacity{0.584541}%
\pgfsetlinewidth{1.003750pt}%
\definecolor{currentstroke}{rgb}{0.121569,0.466667,0.705882}%
\pgfsetstrokecolor{currentstroke}%
\pgfsetstrokeopacity{0.584541}%
\pgfsetdash{}{0pt}%
\pgfpathmoveto{\pgfqpoint{0.892324in}{1.494503in}}%
\pgfpathcurveto{\pgfqpoint{0.900561in}{1.494503in}}{\pgfqpoint{0.908461in}{1.497775in}}{\pgfqpoint{0.914285in}{1.503599in}}%
\pgfpathcurveto{\pgfqpoint{0.920108in}{1.509423in}}{\pgfqpoint{0.923381in}{1.517323in}}{\pgfqpoint{0.923381in}{1.525559in}}%
\pgfpathcurveto{\pgfqpoint{0.923381in}{1.533795in}}{\pgfqpoint{0.920108in}{1.541695in}}{\pgfqpoint{0.914285in}{1.547519in}}%
\pgfpathcurveto{\pgfqpoint{0.908461in}{1.553343in}}{\pgfqpoint{0.900561in}{1.556616in}}{\pgfqpoint{0.892324in}{1.556616in}}%
\pgfpathcurveto{\pgfqpoint{0.884088in}{1.556616in}}{\pgfqpoint{0.876188in}{1.553343in}}{\pgfqpoint{0.870364in}{1.547519in}}%
\pgfpathcurveto{\pgfqpoint{0.864540in}{1.541695in}}{\pgfqpoint{0.861268in}{1.533795in}}{\pgfqpoint{0.861268in}{1.525559in}}%
\pgfpathcurveto{\pgfqpoint{0.861268in}{1.517323in}}{\pgfqpoint{0.864540in}{1.509423in}}{\pgfqpoint{0.870364in}{1.503599in}}%
\pgfpathcurveto{\pgfqpoint{0.876188in}{1.497775in}}{\pgfqpoint{0.884088in}{1.494503in}}{\pgfqpoint{0.892324in}{1.494503in}}%
\pgfpathclose%
\pgfusepath{stroke,fill}%
\end{pgfscope}%
\begin{pgfscope}%
\pgfpathrectangle{\pgfqpoint{0.100000in}{0.212622in}}{\pgfqpoint{3.696000in}{3.696000in}}%
\pgfusepath{clip}%
\pgfsetbuttcap%
\pgfsetroundjoin%
\definecolor{currentfill}{rgb}{0.121569,0.466667,0.705882}%
\pgfsetfillcolor{currentfill}%
\pgfsetfillopacity{0.584542}%
\pgfsetlinewidth{1.003750pt}%
\definecolor{currentstroke}{rgb}{0.121569,0.466667,0.705882}%
\pgfsetstrokecolor{currentstroke}%
\pgfsetstrokeopacity{0.584542}%
\pgfsetdash{}{0pt}%
\pgfpathmoveto{\pgfqpoint{0.892323in}{1.494495in}}%
\pgfpathcurveto{\pgfqpoint{0.900559in}{1.494495in}}{\pgfqpoint{0.908459in}{1.497768in}}{\pgfqpoint{0.914283in}{1.503592in}}%
\pgfpathcurveto{\pgfqpoint{0.920107in}{1.509416in}}{\pgfqpoint{0.923379in}{1.517316in}}{\pgfqpoint{0.923379in}{1.525552in}}%
\pgfpathcurveto{\pgfqpoint{0.923379in}{1.533788in}}{\pgfqpoint{0.920107in}{1.541688in}}{\pgfqpoint{0.914283in}{1.547512in}}%
\pgfpathcurveto{\pgfqpoint{0.908459in}{1.553336in}}{\pgfqpoint{0.900559in}{1.556608in}}{\pgfqpoint{0.892323in}{1.556608in}}%
\pgfpathcurveto{\pgfqpoint{0.884087in}{1.556608in}}{\pgfqpoint{0.876186in}{1.553336in}}{\pgfqpoint{0.870363in}{1.547512in}}%
\pgfpathcurveto{\pgfqpoint{0.864539in}{1.541688in}}{\pgfqpoint{0.861266in}{1.533788in}}{\pgfqpoint{0.861266in}{1.525552in}}%
\pgfpathcurveto{\pgfqpoint{0.861266in}{1.517316in}}{\pgfqpoint{0.864539in}{1.509416in}}{\pgfqpoint{0.870363in}{1.503592in}}%
\pgfpathcurveto{\pgfqpoint{0.876186in}{1.497768in}}{\pgfqpoint{0.884087in}{1.494495in}}{\pgfqpoint{0.892323in}{1.494495in}}%
\pgfpathclose%
\pgfusepath{stroke,fill}%
\end{pgfscope}%
\begin{pgfscope}%
\pgfpathrectangle{\pgfqpoint{0.100000in}{0.212622in}}{\pgfqpoint{3.696000in}{3.696000in}}%
\pgfusepath{clip}%
\pgfsetbuttcap%
\pgfsetroundjoin%
\definecolor{currentfill}{rgb}{0.121569,0.466667,0.705882}%
\pgfsetfillcolor{currentfill}%
\pgfsetfillopacity{0.584543}%
\pgfsetlinewidth{1.003750pt}%
\definecolor{currentstroke}{rgb}{0.121569,0.466667,0.705882}%
\pgfsetstrokecolor{currentstroke}%
\pgfsetstrokeopacity{0.584543}%
\pgfsetdash{}{0pt}%
\pgfpathmoveto{\pgfqpoint{0.892322in}{1.494491in}}%
\pgfpathcurveto{\pgfqpoint{0.900558in}{1.494491in}}{\pgfqpoint{0.908458in}{1.497764in}}{\pgfqpoint{0.914282in}{1.503588in}}%
\pgfpathcurveto{\pgfqpoint{0.920106in}{1.509412in}}{\pgfqpoint{0.923379in}{1.517312in}}{\pgfqpoint{0.923379in}{1.525548in}}%
\pgfpathcurveto{\pgfqpoint{0.923379in}{1.533784in}}{\pgfqpoint{0.920106in}{1.541684in}}{\pgfqpoint{0.914282in}{1.547508in}}%
\pgfpathcurveto{\pgfqpoint{0.908458in}{1.553332in}}{\pgfqpoint{0.900558in}{1.556604in}}{\pgfqpoint{0.892322in}{1.556604in}}%
\pgfpathcurveto{\pgfqpoint{0.884086in}{1.556604in}}{\pgfqpoint{0.876186in}{1.553332in}}{\pgfqpoint{0.870362in}{1.547508in}}%
\pgfpathcurveto{\pgfqpoint{0.864538in}{1.541684in}}{\pgfqpoint{0.861266in}{1.533784in}}{\pgfqpoint{0.861266in}{1.525548in}}%
\pgfpathcurveto{\pgfqpoint{0.861266in}{1.517312in}}{\pgfqpoint{0.864538in}{1.509412in}}{\pgfqpoint{0.870362in}{1.503588in}}%
\pgfpathcurveto{\pgfqpoint{0.876186in}{1.497764in}}{\pgfqpoint{0.884086in}{1.494491in}}{\pgfqpoint{0.892322in}{1.494491in}}%
\pgfpathclose%
\pgfusepath{stroke,fill}%
\end{pgfscope}%
\begin{pgfscope}%
\pgfpathrectangle{\pgfqpoint{0.100000in}{0.212622in}}{\pgfqpoint{3.696000in}{3.696000in}}%
\pgfusepath{clip}%
\pgfsetbuttcap%
\pgfsetroundjoin%
\definecolor{currentfill}{rgb}{0.121569,0.466667,0.705882}%
\pgfsetfillcolor{currentfill}%
\pgfsetfillopacity{0.584543}%
\pgfsetlinewidth{1.003750pt}%
\definecolor{currentstroke}{rgb}{0.121569,0.466667,0.705882}%
\pgfsetstrokecolor{currentstroke}%
\pgfsetstrokeopacity{0.584543}%
\pgfsetdash{}{0pt}%
\pgfpathmoveto{\pgfqpoint{0.892322in}{1.494489in}}%
\pgfpathcurveto{\pgfqpoint{0.900558in}{1.494489in}}{\pgfqpoint{0.908458in}{1.497762in}}{\pgfqpoint{0.914282in}{1.503585in}}%
\pgfpathcurveto{\pgfqpoint{0.920106in}{1.509409in}}{\pgfqpoint{0.923378in}{1.517309in}}{\pgfqpoint{0.923378in}{1.525546in}}%
\pgfpathcurveto{\pgfqpoint{0.923378in}{1.533782in}}{\pgfqpoint{0.920106in}{1.541682in}}{\pgfqpoint{0.914282in}{1.547506in}}%
\pgfpathcurveto{\pgfqpoint{0.908458in}{1.553330in}}{\pgfqpoint{0.900558in}{1.556602in}}{\pgfqpoint{0.892322in}{1.556602in}}%
\pgfpathcurveto{\pgfqpoint{0.884085in}{1.556602in}}{\pgfqpoint{0.876185in}{1.553330in}}{\pgfqpoint{0.870361in}{1.547506in}}%
\pgfpathcurveto{\pgfqpoint{0.864537in}{1.541682in}}{\pgfqpoint{0.861265in}{1.533782in}}{\pgfqpoint{0.861265in}{1.525546in}}%
\pgfpathcurveto{\pgfqpoint{0.861265in}{1.517309in}}{\pgfqpoint{0.864537in}{1.509409in}}{\pgfqpoint{0.870361in}{1.503585in}}%
\pgfpathcurveto{\pgfqpoint{0.876185in}{1.497762in}}{\pgfqpoint{0.884085in}{1.494489in}}{\pgfqpoint{0.892322in}{1.494489in}}%
\pgfpathclose%
\pgfusepath{stroke,fill}%
\end{pgfscope}%
\begin{pgfscope}%
\pgfpathrectangle{\pgfqpoint{0.100000in}{0.212622in}}{\pgfqpoint{3.696000in}{3.696000in}}%
\pgfusepath{clip}%
\pgfsetbuttcap%
\pgfsetroundjoin%
\definecolor{currentfill}{rgb}{0.121569,0.466667,0.705882}%
\pgfsetfillcolor{currentfill}%
\pgfsetfillopacity{0.584619}%
\pgfsetlinewidth{1.003750pt}%
\definecolor{currentstroke}{rgb}{0.121569,0.466667,0.705882}%
\pgfsetstrokecolor{currentstroke}%
\pgfsetstrokeopacity{0.584619}%
\pgfsetdash{}{0pt}%
\pgfpathmoveto{\pgfqpoint{0.892238in}{1.494078in}}%
\pgfpathcurveto{\pgfqpoint{0.900474in}{1.494078in}}{\pgfqpoint{0.908374in}{1.497350in}}{\pgfqpoint{0.914198in}{1.503174in}}%
\pgfpathcurveto{\pgfqpoint{0.920022in}{1.508998in}}{\pgfqpoint{0.923295in}{1.516898in}}{\pgfqpoint{0.923295in}{1.525134in}}%
\pgfpathcurveto{\pgfqpoint{0.923295in}{1.533371in}}{\pgfqpoint{0.920022in}{1.541271in}}{\pgfqpoint{0.914198in}{1.547095in}}%
\pgfpathcurveto{\pgfqpoint{0.908374in}{1.552919in}}{\pgfqpoint{0.900474in}{1.556191in}}{\pgfqpoint{0.892238in}{1.556191in}}%
\pgfpathcurveto{\pgfqpoint{0.884002in}{1.556191in}}{\pgfqpoint{0.876102in}{1.552919in}}{\pgfqpoint{0.870278in}{1.547095in}}%
\pgfpathcurveto{\pgfqpoint{0.864454in}{1.541271in}}{\pgfqpoint{0.861182in}{1.533371in}}{\pgfqpoint{0.861182in}{1.525134in}}%
\pgfpathcurveto{\pgfqpoint{0.861182in}{1.516898in}}{\pgfqpoint{0.864454in}{1.508998in}}{\pgfqpoint{0.870278in}{1.503174in}}%
\pgfpathcurveto{\pgfqpoint{0.876102in}{1.497350in}}{\pgfqpoint{0.884002in}{1.494078in}}{\pgfqpoint{0.892238in}{1.494078in}}%
\pgfpathclose%
\pgfusepath{stroke,fill}%
\end{pgfscope}%
\begin{pgfscope}%
\pgfpathrectangle{\pgfqpoint{0.100000in}{0.212622in}}{\pgfqpoint{3.696000in}{3.696000in}}%
\pgfusepath{clip}%
\pgfsetbuttcap%
\pgfsetroundjoin%
\definecolor{currentfill}{rgb}{0.121569,0.466667,0.705882}%
\pgfsetfillcolor{currentfill}%
\pgfsetfillopacity{0.584661}%
\pgfsetlinewidth{1.003750pt}%
\definecolor{currentstroke}{rgb}{0.121569,0.466667,0.705882}%
\pgfsetstrokecolor{currentstroke}%
\pgfsetstrokeopacity{0.584661}%
\pgfsetdash{}{0pt}%
\pgfpathmoveto{\pgfqpoint{0.892187in}{1.493853in}}%
\pgfpathcurveto{\pgfqpoint{0.900424in}{1.493853in}}{\pgfqpoint{0.908324in}{1.497125in}}{\pgfqpoint{0.914148in}{1.502949in}}%
\pgfpathcurveto{\pgfqpoint{0.919972in}{1.508773in}}{\pgfqpoint{0.923244in}{1.516673in}}{\pgfqpoint{0.923244in}{1.524909in}}%
\pgfpathcurveto{\pgfqpoint{0.923244in}{1.533146in}}{\pgfqpoint{0.919972in}{1.541046in}}{\pgfqpoint{0.914148in}{1.546870in}}%
\pgfpathcurveto{\pgfqpoint{0.908324in}{1.552694in}}{\pgfqpoint{0.900424in}{1.555966in}}{\pgfqpoint{0.892187in}{1.555966in}}%
\pgfpathcurveto{\pgfqpoint{0.883951in}{1.555966in}}{\pgfqpoint{0.876051in}{1.552694in}}{\pgfqpoint{0.870227in}{1.546870in}}%
\pgfpathcurveto{\pgfqpoint{0.864403in}{1.541046in}}{\pgfqpoint{0.861131in}{1.533146in}}{\pgfqpoint{0.861131in}{1.524909in}}%
\pgfpathcurveto{\pgfqpoint{0.861131in}{1.516673in}}{\pgfqpoint{0.864403in}{1.508773in}}{\pgfqpoint{0.870227in}{1.502949in}}%
\pgfpathcurveto{\pgfqpoint{0.876051in}{1.497125in}}{\pgfqpoint{0.883951in}{1.493853in}}{\pgfqpoint{0.892187in}{1.493853in}}%
\pgfpathclose%
\pgfusepath{stroke,fill}%
\end{pgfscope}%
\begin{pgfscope}%
\pgfpathrectangle{\pgfqpoint{0.100000in}{0.212622in}}{\pgfqpoint{3.696000in}{3.696000in}}%
\pgfusepath{clip}%
\pgfsetbuttcap%
\pgfsetroundjoin%
\definecolor{currentfill}{rgb}{0.121569,0.466667,0.705882}%
\pgfsetfillcolor{currentfill}%
\pgfsetfillopacity{0.584777}%
\pgfsetlinewidth{1.003750pt}%
\definecolor{currentstroke}{rgb}{0.121569,0.466667,0.705882}%
\pgfsetstrokecolor{currentstroke}%
\pgfsetstrokeopacity{0.584777}%
\pgfsetdash{}{0pt}%
\pgfpathmoveto{\pgfqpoint{0.892040in}{1.493237in}}%
\pgfpathcurveto{\pgfqpoint{0.900276in}{1.493237in}}{\pgfqpoint{0.908176in}{1.496509in}}{\pgfqpoint{0.914000in}{1.502333in}}%
\pgfpathcurveto{\pgfqpoint{0.919824in}{1.508157in}}{\pgfqpoint{0.923096in}{1.516057in}}{\pgfqpoint{0.923096in}{1.524294in}}%
\pgfpathcurveto{\pgfqpoint{0.923096in}{1.532530in}}{\pgfqpoint{0.919824in}{1.540430in}}{\pgfqpoint{0.914000in}{1.546254in}}%
\pgfpathcurveto{\pgfqpoint{0.908176in}{1.552078in}}{\pgfqpoint{0.900276in}{1.555350in}}{\pgfqpoint{0.892040in}{1.555350in}}%
\pgfpathcurveto{\pgfqpoint{0.883803in}{1.555350in}}{\pgfqpoint{0.875903in}{1.552078in}}{\pgfqpoint{0.870079in}{1.546254in}}%
\pgfpathcurveto{\pgfqpoint{0.864255in}{1.540430in}}{\pgfqpoint{0.860983in}{1.532530in}}{\pgfqpoint{0.860983in}{1.524294in}}%
\pgfpathcurveto{\pgfqpoint{0.860983in}{1.516057in}}{\pgfqpoint{0.864255in}{1.508157in}}{\pgfqpoint{0.870079in}{1.502333in}}%
\pgfpathcurveto{\pgfqpoint{0.875903in}{1.496509in}}{\pgfqpoint{0.883803in}{1.493237in}}{\pgfqpoint{0.892040in}{1.493237in}}%
\pgfpathclose%
\pgfusepath{stroke,fill}%
\end{pgfscope}%
\begin{pgfscope}%
\pgfpathrectangle{\pgfqpoint{0.100000in}{0.212622in}}{\pgfqpoint{3.696000in}{3.696000in}}%
\pgfusepath{clip}%
\pgfsetbuttcap%
\pgfsetroundjoin%
\definecolor{currentfill}{rgb}{0.121569,0.466667,0.705882}%
\pgfsetfillcolor{currentfill}%
\pgfsetfillopacity{0.584842}%
\pgfsetlinewidth{1.003750pt}%
\definecolor{currentstroke}{rgb}{0.121569,0.466667,0.705882}%
\pgfsetstrokecolor{currentstroke}%
\pgfsetstrokeopacity{0.584842}%
\pgfsetdash{}{0pt}%
\pgfpathmoveto{\pgfqpoint{0.891960in}{1.492905in}}%
\pgfpathcurveto{\pgfqpoint{0.900196in}{1.492905in}}{\pgfqpoint{0.908096in}{1.496177in}}{\pgfqpoint{0.913920in}{1.502001in}}%
\pgfpathcurveto{\pgfqpoint{0.919744in}{1.507825in}}{\pgfqpoint{0.923016in}{1.515725in}}{\pgfqpoint{0.923016in}{1.523961in}}%
\pgfpathcurveto{\pgfqpoint{0.923016in}{1.532197in}}{\pgfqpoint{0.919744in}{1.540097in}}{\pgfqpoint{0.913920in}{1.545921in}}%
\pgfpathcurveto{\pgfqpoint{0.908096in}{1.551745in}}{\pgfqpoint{0.900196in}{1.555018in}}{\pgfqpoint{0.891960in}{1.555018in}}%
\pgfpathcurveto{\pgfqpoint{0.883723in}{1.555018in}}{\pgfqpoint{0.875823in}{1.551745in}}{\pgfqpoint{0.869999in}{1.545921in}}%
\pgfpathcurveto{\pgfqpoint{0.864175in}{1.540097in}}{\pgfqpoint{0.860903in}{1.532197in}}{\pgfqpoint{0.860903in}{1.523961in}}%
\pgfpathcurveto{\pgfqpoint{0.860903in}{1.515725in}}{\pgfqpoint{0.864175in}{1.507825in}}{\pgfqpoint{0.869999in}{1.502001in}}%
\pgfpathcurveto{\pgfqpoint{0.875823in}{1.496177in}}{\pgfqpoint{0.883723in}{1.492905in}}{\pgfqpoint{0.891960in}{1.492905in}}%
\pgfpathclose%
\pgfusepath{stroke,fill}%
\end{pgfscope}%
\begin{pgfscope}%
\pgfpathrectangle{\pgfqpoint{0.100000in}{0.212622in}}{\pgfqpoint{3.696000in}{3.696000in}}%
\pgfusepath{clip}%
\pgfsetbuttcap%
\pgfsetroundjoin%
\definecolor{currentfill}{rgb}{0.121569,0.466667,0.705882}%
\pgfsetfillcolor{currentfill}%
\pgfsetfillopacity{0.584877}%
\pgfsetlinewidth{1.003750pt}%
\definecolor{currentstroke}{rgb}{0.121569,0.466667,0.705882}%
\pgfsetstrokecolor{currentstroke}%
\pgfsetstrokeopacity{0.584877}%
\pgfsetdash{}{0pt}%
\pgfpathmoveto{\pgfqpoint{0.891913in}{1.492723in}}%
\pgfpathcurveto{\pgfqpoint{0.900149in}{1.492723in}}{\pgfqpoint{0.908049in}{1.495995in}}{\pgfqpoint{0.913873in}{1.501819in}}%
\pgfpathcurveto{\pgfqpoint{0.919697in}{1.507643in}}{\pgfqpoint{0.922969in}{1.515543in}}{\pgfqpoint{0.922969in}{1.523779in}}%
\pgfpathcurveto{\pgfqpoint{0.922969in}{1.532015in}}{\pgfqpoint{0.919697in}{1.539915in}}{\pgfqpoint{0.913873in}{1.545739in}}%
\pgfpathcurveto{\pgfqpoint{0.908049in}{1.551563in}}{\pgfqpoint{0.900149in}{1.554836in}}{\pgfqpoint{0.891913in}{1.554836in}}%
\pgfpathcurveto{\pgfqpoint{0.883677in}{1.554836in}}{\pgfqpoint{0.875777in}{1.551563in}}{\pgfqpoint{0.869953in}{1.545739in}}%
\pgfpathcurveto{\pgfqpoint{0.864129in}{1.539915in}}{\pgfqpoint{0.860856in}{1.532015in}}{\pgfqpoint{0.860856in}{1.523779in}}%
\pgfpathcurveto{\pgfqpoint{0.860856in}{1.515543in}}{\pgfqpoint{0.864129in}{1.507643in}}{\pgfqpoint{0.869953in}{1.501819in}}%
\pgfpathcurveto{\pgfqpoint{0.875777in}{1.495995in}}{\pgfqpoint{0.883677in}{1.492723in}}{\pgfqpoint{0.891913in}{1.492723in}}%
\pgfpathclose%
\pgfusepath{stroke,fill}%
\end{pgfscope}%
\begin{pgfscope}%
\pgfpathrectangle{\pgfqpoint{0.100000in}{0.212622in}}{\pgfqpoint{3.696000in}{3.696000in}}%
\pgfusepath{clip}%
\pgfsetbuttcap%
\pgfsetroundjoin%
\definecolor{currentfill}{rgb}{0.121569,0.466667,0.705882}%
\pgfsetfillcolor{currentfill}%
\pgfsetfillopacity{0.584897}%
\pgfsetlinewidth{1.003750pt}%
\definecolor{currentstroke}{rgb}{0.121569,0.466667,0.705882}%
\pgfsetstrokecolor{currentstroke}%
\pgfsetstrokeopacity{0.584897}%
\pgfsetdash{}{0pt}%
\pgfpathmoveto{\pgfqpoint{0.891886in}{1.492623in}}%
\pgfpathcurveto{\pgfqpoint{0.900123in}{1.492623in}}{\pgfqpoint{0.908023in}{1.495895in}}{\pgfqpoint{0.913847in}{1.501719in}}%
\pgfpathcurveto{\pgfqpoint{0.919670in}{1.507543in}}{\pgfqpoint{0.922943in}{1.515443in}}{\pgfqpoint{0.922943in}{1.523679in}}%
\pgfpathcurveto{\pgfqpoint{0.922943in}{1.531916in}}{\pgfqpoint{0.919670in}{1.539816in}}{\pgfqpoint{0.913847in}{1.545640in}}%
\pgfpathcurveto{\pgfqpoint{0.908023in}{1.551463in}}{\pgfqpoint{0.900123in}{1.554736in}}{\pgfqpoint{0.891886in}{1.554736in}}%
\pgfpathcurveto{\pgfqpoint{0.883650in}{1.554736in}}{\pgfqpoint{0.875750in}{1.551463in}}{\pgfqpoint{0.869926in}{1.545640in}}%
\pgfpathcurveto{\pgfqpoint{0.864102in}{1.539816in}}{\pgfqpoint{0.860830in}{1.531916in}}{\pgfqpoint{0.860830in}{1.523679in}}%
\pgfpathcurveto{\pgfqpoint{0.860830in}{1.515443in}}{\pgfqpoint{0.864102in}{1.507543in}}{\pgfqpoint{0.869926in}{1.501719in}}%
\pgfpathcurveto{\pgfqpoint{0.875750in}{1.495895in}}{\pgfqpoint{0.883650in}{1.492623in}}{\pgfqpoint{0.891886in}{1.492623in}}%
\pgfpathclose%
\pgfusepath{stroke,fill}%
\end{pgfscope}%
\begin{pgfscope}%
\pgfpathrectangle{\pgfqpoint{0.100000in}{0.212622in}}{\pgfqpoint{3.696000in}{3.696000in}}%
\pgfusepath{clip}%
\pgfsetbuttcap%
\pgfsetroundjoin%
\definecolor{currentfill}{rgb}{0.121569,0.466667,0.705882}%
\pgfsetfillcolor{currentfill}%
\pgfsetfillopacity{0.584908}%
\pgfsetlinewidth{1.003750pt}%
\definecolor{currentstroke}{rgb}{0.121569,0.466667,0.705882}%
\pgfsetstrokecolor{currentstroke}%
\pgfsetstrokeopacity{0.584908}%
\pgfsetdash{}{0pt}%
\pgfpathmoveto{\pgfqpoint{0.891871in}{1.492568in}}%
\pgfpathcurveto{\pgfqpoint{0.900107in}{1.492568in}}{\pgfqpoint{0.908007in}{1.495840in}}{\pgfqpoint{0.913831in}{1.501664in}}%
\pgfpathcurveto{\pgfqpoint{0.919655in}{1.507488in}}{\pgfqpoint{0.922927in}{1.515388in}}{\pgfqpoint{0.922927in}{1.523624in}}%
\pgfpathcurveto{\pgfqpoint{0.922927in}{1.531860in}}{\pgfqpoint{0.919655in}{1.539760in}}{\pgfqpoint{0.913831in}{1.545584in}}%
\pgfpathcurveto{\pgfqpoint{0.908007in}{1.551408in}}{\pgfqpoint{0.900107in}{1.554681in}}{\pgfqpoint{0.891871in}{1.554681in}}%
\pgfpathcurveto{\pgfqpoint{0.883635in}{1.554681in}}{\pgfqpoint{0.875735in}{1.551408in}}{\pgfqpoint{0.869911in}{1.545584in}}%
\pgfpathcurveto{\pgfqpoint{0.864087in}{1.539760in}}{\pgfqpoint{0.860814in}{1.531860in}}{\pgfqpoint{0.860814in}{1.523624in}}%
\pgfpathcurveto{\pgfqpoint{0.860814in}{1.515388in}}{\pgfqpoint{0.864087in}{1.507488in}}{\pgfqpoint{0.869911in}{1.501664in}}%
\pgfpathcurveto{\pgfqpoint{0.875735in}{1.495840in}}{\pgfqpoint{0.883635in}{1.492568in}}{\pgfqpoint{0.891871in}{1.492568in}}%
\pgfpathclose%
\pgfusepath{stroke,fill}%
\end{pgfscope}%
\begin{pgfscope}%
\pgfpathrectangle{\pgfqpoint{0.100000in}{0.212622in}}{\pgfqpoint{3.696000in}{3.696000in}}%
\pgfusepath{clip}%
\pgfsetbuttcap%
\pgfsetroundjoin%
\definecolor{currentfill}{rgb}{0.121569,0.466667,0.705882}%
\pgfsetfillcolor{currentfill}%
\pgfsetfillopacity{0.584914}%
\pgfsetlinewidth{1.003750pt}%
\definecolor{currentstroke}{rgb}{0.121569,0.466667,0.705882}%
\pgfsetstrokecolor{currentstroke}%
\pgfsetstrokeopacity{0.584914}%
\pgfsetdash{}{0pt}%
\pgfpathmoveto{\pgfqpoint{0.891863in}{1.492537in}}%
\pgfpathcurveto{\pgfqpoint{0.900099in}{1.492537in}}{\pgfqpoint{0.907999in}{1.495809in}}{\pgfqpoint{0.913823in}{1.501633in}}%
\pgfpathcurveto{\pgfqpoint{0.919647in}{1.507457in}}{\pgfqpoint{0.922919in}{1.515357in}}{\pgfqpoint{0.922919in}{1.523593in}}%
\pgfpathcurveto{\pgfqpoint{0.922919in}{1.531830in}}{\pgfqpoint{0.919647in}{1.539730in}}{\pgfqpoint{0.913823in}{1.545554in}}%
\pgfpathcurveto{\pgfqpoint{0.907999in}{1.551378in}}{\pgfqpoint{0.900099in}{1.554650in}}{\pgfqpoint{0.891863in}{1.554650in}}%
\pgfpathcurveto{\pgfqpoint{0.883626in}{1.554650in}}{\pgfqpoint{0.875726in}{1.551378in}}{\pgfqpoint{0.869902in}{1.545554in}}%
\pgfpathcurveto{\pgfqpoint{0.864078in}{1.539730in}}{\pgfqpoint{0.860806in}{1.531830in}}{\pgfqpoint{0.860806in}{1.523593in}}%
\pgfpathcurveto{\pgfqpoint{0.860806in}{1.515357in}}{\pgfqpoint{0.864078in}{1.507457in}}{\pgfqpoint{0.869902in}{1.501633in}}%
\pgfpathcurveto{\pgfqpoint{0.875726in}{1.495809in}}{\pgfqpoint{0.883626in}{1.492537in}}{\pgfqpoint{0.891863in}{1.492537in}}%
\pgfpathclose%
\pgfusepath{stroke,fill}%
\end{pgfscope}%
\begin{pgfscope}%
\pgfpathrectangle{\pgfqpoint{0.100000in}{0.212622in}}{\pgfqpoint{3.696000in}{3.696000in}}%
\pgfusepath{clip}%
\pgfsetbuttcap%
\pgfsetroundjoin%
\definecolor{currentfill}{rgb}{0.121569,0.466667,0.705882}%
\pgfsetfillcolor{currentfill}%
\pgfsetfillopacity{0.584917}%
\pgfsetlinewidth{1.003750pt}%
\definecolor{currentstroke}{rgb}{0.121569,0.466667,0.705882}%
\pgfsetstrokecolor{currentstroke}%
\pgfsetstrokeopacity{0.584917}%
\pgfsetdash{}{0pt}%
\pgfpathmoveto{\pgfqpoint{0.891858in}{1.492520in}}%
\pgfpathcurveto{\pgfqpoint{0.900094in}{1.492520in}}{\pgfqpoint{0.907994in}{1.495793in}}{\pgfqpoint{0.913818in}{1.501616in}}%
\pgfpathcurveto{\pgfqpoint{0.919642in}{1.507440in}}{\pgfqpoint{0.922914in}{1.515340in}}{\pgfqpoint{0.922914in}{1.523577in}}%
\pgfpathcurveto{\pgfqpoint{0.922914in}{1.531813in}}{\pgfqpoint{0.919642in}{1.539713in}}{\pgfqpoint{0.913818in}{1.545537in}}%
\pgfpathcurveto{\pgfqpoint{0.907994in}{1.551361in}}{\pgfqpoint{0.900094in}{1.554633in}}{\pgfqpoint{0.891858in}{1.554633in}}%
\pgfpathcurveto{\pgfqpoint{0.883622in}{1.554633in}}{\pgfqpoint{0.875721in}{1.551361in}}{\pgfqpoint{0.869898in}{1.545537in}}%
\pgfpathcurveto{\pgfqpoint{0.864074in}{1.539713in}}{\pgfqpoint{0.860801in}{1.531813in}}{\pgfqpoint{0.860801in}{1.523577in}}%
\pgfpathcurveto{\pgfqpoint{0.860801in}{1.515340in}}{\pgfqpoint{0.864074in}{1.507440in}}{\pgfqpoint{0.869898in}{1.501616in}}%
\pgfpathcurveto{\pgfqpoint{0.875721in}{1.495793in}}{\pgfqpoint{0.883622in}{1.492520in}}{\pgfqpoint{0.891858in}{1.492520in}}%
\pgfpathclose%
\pgfusepath{stroke,fill}%
\end{pgfscope}%
\begin{pgfscope}%
\pgfpathrectangle{\pgfqpoint{0.100000in}{0.212622in}}{\pgfqpoint{3.696000in}{3.696000in}}%
\pgfusepath{clip}%
\pgfsetbuttcap%
\pgfsetroundjoin%
\definecolor{currentfill}{rgb}{0.121569,0.466667,0.705882}%
\pgfsetfillcolor{currentfill}%
\pgfsetfillopacity{0.584919}%
\pgfsetlinewidth{1.003750pt}%
\definecolor{currentstroke}{rgb}{0.121569,0.466667,0.705882}%
\pgfsetstrokecolor{currentstroke}%
\pgfsetstrokeopacity{0.584919}%
\pgfsetdash{}{0pt}%
\pgfpathmoveto{\pgfqpoint{0.891855in}{1.492511in}}%
\pgfpathcurveto{\pgfqpoint{0.900092in}{1.492511in}}{\pgfqpoint{0.907992in}{1.495784in}}{\pgfqpoint{0.913815in}{1.501608in}}%
\pgfpathcurveto{\pgfqpoint{0.919639in}{1.507431in}}{\pgfqpoint{0.922912in}{1.515332in}}{\pgfqpoint{0.922912in}{1.523568in}}%
\pgfpathcurveto{\pgfqpoint{0.922912in}{1.531804in}}{\pgfqpoint{0.919639in}{1.539704in}}{\pgfqpoint{0.913815in}{1.545528in}}%
\pgfpathcurveto{\pgfqpoint{0.907992in}{1.551352in}}{\pgfqpoint{0.900092in}{1.554624in}}{\pgfqpoint{0.891855in}{1.554624in}}%
\pgfpathcurveto{\pgfqpoint{0.883619in}{1.554624in}}{\pgfqpoint{0.875719in}{1.551352in}}{\pgfqpoint{0.869895in}{1.545528in}}%
\pgfpathcurveto{\pgfqpoint{0.864071in}{1.539704in}}{\pgfqpoint{0.860799in}{1.531804in}}{\pgfqpoint{0.860799in}{1.523568in}}%
\pgfpathcurveto{\pgfqpoint{0.860799in}{1.515332in}}{\pgfqpoint{0.864071in}{1.507431in}}{\pgfqpoint{0.869895in}{1.501608in}}%
\pgfpathcurveto{\pgfqpoint{0.875719in}{1.495784in}}{\pgfqpoint{0.883619in}{1.492511in}}{\pgfqpoint{0.891855in}{1.492511in}}%
\pgfpathclose%
\pgfusepath{stroke,fill}%
\end{pgfscope}%
\begin{pgfscope}%
\pgfpathrectangle{\pgfqpoint{0.100000in}{0.212622in}}{\pgfqpoint{3.696000in}{3.696000in}}%
\pgfusepath{clip}%
\pgfsetbuttcap%
\pgfsetroundjoin%
\definecolor{currentfill}{rgb}{0.121569,0.466667,0.705882}%
\pgfsetfillcolor{currentfill}%
\pgfsetfillopacity{0.584920}%
\pgfsetlinewidth{1.003750pt}%
\definecolor{currentstroke}{rgb}{0.121569,0.466667,0.705882}%
\pgfsetstrokecolor{currentstroke}%
\pgfsetstrokeopacity{0.584920}%
\pgfsetdash{}{0pt}%
\pgfpathmoveto{\pgfqpoint{0.891854in}{1.492506in}}%
\pgfpathcurveto{\pgfqpoint{0.900090in}{1.492506in}}{\pgfqpoint{0.907990in}{1.495779in}}{\pgfqpoint{0.913814in}{1.501603in}}%
\pgfpathcurveto{\pgfqpoint{0.919638in}{1.507427in}}{\pgfqpoint{0.922910in}{1.515327in}}{\pgfqpoint{0.922910in}{1.523563in}}%
\pgfpathcurveto{\pgfqpoint{0.922910in}{1.531799in}}{\pgfqpoint{0.919638in}{1.539699in}}{\pgfqpoint{0.913814in}{1.545523in}}%
\pgfpathcurveto{\pgfqpoint{0.907990in}{1.551347in}}{\pgfqpoint{0.900090in}{1.554619in}}{\pgfqpoint{0.891854in}{1.554619in}}%
\pgfpathcurveto{\pgfqpoint{0.883618in}{1.554619in}}{\pgfqpoint{0.875717in}{1.551347in}}{\pgfqpoint{0.869894in}{1.545523in}}%
\pgfpathcurveto{\pgfqpoint{0.864070in}{1.539699in}}{\pgfqpoint{0.860797in}{1.531799in}}{\pgfqpoint{0.860797in}{1.523563in}}%
\pgfpathcurveto{\pgfqpoint{0.860797in}{1.515327in}}{\pgfqpoint{0.864070in}{1.507427in}}{\pgfqpoint{0.869894in}{1.501603in}}%
\pgfpathcurveto{\pgfqpoint{0.875717in}{1.495779in}}{\pgfqpoint{0.883618in}{1.492506in}}{\pgfqpoint{0.891854in}{1.492506in}}%
\pgfpathclose%
\pgfusepath{stroke,fill}%
\end{pgfscope}%
\begin{pgfscope}%
\pgfpathrectangle{\pgfqpoint{0.100000in}{0.212622in}}{\pgfqpoint{3.696000in}{3.696000in}}%
\pgfusepath{clip}%
\pgfsetbuttcap%
\pgfsetroundjoin%
\definecolor{currentfill}{rgb}{0.121569,0.466667,0.705882}%
\pgfsetfillcolor{currentfill}%
\pgfsetfillopacity{0.584996}%
\pgfsetlinewidth{1.003750pt}%
\definecolor{currentstroke}{rgb}{0.121569,0.466667,0.705882}%
\pgfsetstrokecolor{currentstroke}%
\pgfsetstrokeopacity{0.584996}%
\pgfsetdash{}{0pt}%
\pgfpathmoveto{\pgfqpoint{0.891754in}{1.492164in}}%
\pgfpathcurveto{\pgfqpoint{0.899990in}{1.492164in}}{\pgfqpoint{0.907890in}{1.495437in}}{\pgfqpoint{0.913714in}{1.501261in}}%
\pgfpathcurveto{\pgfqpoint{0.919538in}{1.507085in}}{\pgfqpoint{0.922810in}{1.514985in}}{\pgfqpoint{0.922810in}{1.523221in}}%
\pgfpathcurveto{\pgfqpoint{0.922810in}{1.531457in}}{\pgfqpoint{0.919538in}{1.539357in}}{\pgfqpoint{0.913714in}{1.545181in}}%
\pgfpathcurveto{\pgfqpoint{0.907890in}{1.551005in}}{\pgfqpoint{0.899990in}{1.554277in}}{\pgfqpoint{0.891754in}{1.554277in}}%
\pgfpathcurveto{\pgfqpoint{0.883518in}{1.554277in}}{\pgfqpoint{0.875618in}{1.551005in}}{\pgfqpoint{0.869794in}{1.545181in}}%
\pgfpathcurveto{\pgfqpoint{0.863970in}{1.539357in}}{\pgfqpoint{0.860697in}{1.531457in}}{\pgfqpoint{0.860697in}{1.523221in}}%
\pgfpathcurveto{\pgfqpoint{0.860697in}{1.514985in}}{\pgfqpoint{0.863970in}{1.507085in}}{\pgfqpoint{0.869794in}{1.501261in}}%
\pgfpathcurveto{\pgfqpoint{0.875618in}{1.495437in}}{\pgfqpoint{0.883518in}{1.492164in}}{\pgfqpoint{0.891754in}{1.492164in}}%
\pgfpathclose%
\pgfusepath{stroke,fill}%
\end{pgfscope}%
\begin{pgfscope}%
\pgfpathrectangle{\pgfqpoint{0.100000in}{0.212622in}}{\pgfqpoint{3.696000in}{3.696000in}}%
\pgfusepath{clip}%
\pgfsetbuttcap%
\pgfsetroundjoin%
\definecolor{currentfill}{rgb}{0.121569,0.466667,0.705882}%
\pgfsetfillcolor{currentfill}%
\pgfsetfillopacity{0.585037}%
\pgfsetlinewidth{1.003750pt}%
\definecolor{currentstroke}{rgb}{0.121569,0.466667,0.705882}%
\pgfsetstrokecolor{currentstroke}%
\pgfsetstrokeopacity{0.585037}%
\pgfsetdash{}{0pt}%
\pgfpathmoveto{\pgfqpoint{0.891698in}{1.491976in}}%
\pgfpathcurveto{\pgfqpoint{0.899934in}{1.491976in}}{\pgfqpoint{0.907834in}{1.495248in}}{\pgfqpoint{0.913658in}{1.501072in}}%
\pgfpathcurveto{\pgfqpoint{0.919482in}{1.506896in}}{\pgfqpoint{0.922754in}{1.514796in}}{\pgfqpoint{0.922754in}{1.523033in}}%
\pgfpathcurveto{\pgfqpoint{0.922754in}{1.531269in}}{\pgfqpoint{0.919482in}{1.539169in}}{\pgfqpoint{0.913658in}{1.544993in}}%
\pgfpathcurveto{\pgfqpoint{0.907834in}{1.550817in}}{\pgfqpoint{0.899934in}{1.554089in}}{\pgfqpoint{0.891698in}{1.554089in}}%
\pgfpathcurveto{\pgfqpoint{0.883462in}{1.554089in}}{\pgfqpoint{0.875562in}{1.550817in}}{\pgfqpoint{0.869738in}{1.544993in}}%
\pgfpathcurveto{\pgfqpoint{0.863914in}{1.539169in}}{\pgfqpoint{0.860641in}{1.531269in}}{\pgfqpoint{0.860641in}{1.523033in}}%
\pgfpathcurveto{\pgfqpoint{0.860641in}{1.514796in}}{\pgfqpoint{0.863914in}{1.506896in}}{\pgfqpoint{0.869738in}{1.501072in}}%
\pgfpathcurveto{\pgfqpoint{0.875562in}{1.495248in}}{\pgfqpoint{0.883462in}{1.491976in}}{\pgfqpoint{0.891698in}{1.491976in}}%
\pgfpathclose%
\pgfusepath{stroke,fill}%
\end{pgfscope}%
\begin{pgfscope}%
\pgfpathrectangle{\pgfqpoint{0.100000in}{0.212622in}}{\pgfqpoint{3.696000in}{3.696000in}}%
\pgfusepath{clip}%
\pgfsetbuttcap%
\pgfsetroundjoin%
\definecolor{currentfill}{rgb}{0.121569,0.466667,0.705882}%
\pgfsetfillcolor{currentfill}%
\pgfsetfillopacity{0.585059}%
\pgfsetlinewidth{1.003750pt}%
\definecolor{currentstroke}{rgb}{0.121569,0.466667,0.705882}%
\pgfsetstrokecolor{currentstroke}%
\pgfsetstrokeopacity{0.585059}%
\pgfsetdash{}{0pt}%
\pgfpathmoveto{\pgfqpoint{0.891666in}{1.491868in}}%
\pgfpathcurveto{\pgfqpoint{0.899902in}{1.491868in}}{\pgfqpoint{0.907802in}{1.495141in}}{\pgfqpoint{0.913626in}{1.500965in}}%
\pgfpathcurveto{\pgfqpoint{0.919450in}{1.506789in}}{\pgfqpoint{0.922723in}{1.514689in}}{\pgfqpoint{0.922723in}{1.522925in}}%
\pgfpathcurveto{\pgfqpoint{0.922723in}{1.531161in}}{\pgfqpoint{0.919450in}{1.539061in}}{\pgfqpoint{0.913626in}{1.544885in}}%
\pgfpathcurveto{\pgfqpoint{0.907802in}{1.550709in}}{\pgfqpoint{0.899902in}{1.553981in}}{\pgfqpoint{0.891666in}{1.553981in}}%
\pgfpathcurveto{\pgfqpoint{0.883430in}{1.553981in}}{\pgfqpoint{0.875530in}{1.550709in}}{\pgfqpoint{0.869706in}{1.544885in}}%
\pgfpathcurveto{\pgfqpoint{0.863882in}{1.539061in}}{\pgfqpoint{0.860610in}{1.531161in}}{\pgfqpoint{0.860610in}{1.522925in}}%
\pgfpathcurveto{\pgfqpoint{0.860610in}{1.514689in}}{\pgfqpoint{0.863882in}{1.506789in}}{\pgfqpoint{0.869706in}{1.500965in}}%
\pgfpathcurveto{\pgfqpoint{0.875530in}{1.495141in}}{\pgfqpoint{0.883430in}{1.491868in}}{\pgfqpoint{0.891666in}{1.491868in}}%
\pgfpathclose%
\pgfusepath{stroke,fill}%
\end{pgfscope}%
\begin{pgfscope}%
\pgfpathrectangle{\pgfqpoint{0.100000in}{0.212622in}}{\pgfqpoint{3.696000in}{3.696000in}}%
\pgfusepath{clip}%
\pgfsetbuttcap%
\pgfsetroundjoin%
\definecolor{currentfill}{rgb}{0.121569,0.466667,0.705882}%
\pgfsetfillcolor{currentfill}%
\pgfsetfillopacity{0.585071}%
\pgfsetlinewidth{1.003750pt}%
\definecolor{currentstroke}{rgb}{0.121569,0.466667,0.705882}%
\pgfsetstrokecolor{currentstroke}%
\pgfsetstrokeopacity{0.585071}%
\pgfsetdash{}{0pt}%
\pgfpathmoveto{\pgfqpoint{0.891649in}{1.491807in}}%
\pgfpathcurveto{\pgfqpoint{0.899885in}{1.491807in}}{\pgfqpoint{0.907786in}{1.495079in}}{\pgfqpoint{0.913609in}{1.500903in}}%
\pgfpathcurveto{\pgfqpoint{0.919433in}{1.506727in}}{\pgfqpoint{0.922706in}{1.514627in}}{\pgfqpoint{0.922706in}{1.522863in}}%
\pgfpathcurveto{\pgfqpoint{0.922706in}{1.531100in}}{\pgfqpoint{0.919433in}{1.539000in}}{\pgfqpoint{0.913609in}{1.544824in}}%
\pgfpathcurveto{\pgfqpoint{0.907786in}{1.550647in}}{\pgfqpoint{0.899885in}{1.553920in}}{\pgfqpoint{0.891649in}{1.553920in}}%
\pgfpathcurveto{\pgfqpoint{0.883413in}{1.553920in}}{\pgfqpoint{0.875513in}{1.550647in}}{\pgfqpoint{0.869689in}{1.544824in}}%
\pgfpathcurveto{\pgfqpoint{0.863865in}{1.539000in}}{\pgfqpoint{0.860593in}{1.531100in}}{\pgfqpoint{0.860593in}{1.522863in}}%
\pgfpathcurveto{\pgfqpoint{0.860593in}{1.514627in}}{\pgfqpoint{0.863865in}{1.506727in}}{\pgfqpoint{0.869689in}{1.500903in}}%
\pgfpathcurveto{\pgfqpoint{0.875513in}{1.495079in}}{\pgfqpoint{0.883413in}{1.491807in}}{\pgfqpoint{0.891649in}{1.491807in}}%
\pgfpathclose%
\pgfusepath{stroke,fill}%
\end{pgfscope}%
\begin{pgfscope}%
\pgfpathrectangle{\pgfqpoint{0.100000in}{0.212622in}}{\pgfqpoint{3.696000in}{3.696000in}}%
\pgfusepath{clip}%
\pgfsetbuttcap%
\pgfsetroundjoin%
\definecolor{currentfill}{rgb}{0.121569,0.466667,0.705882}%
\pgfsetfillcolor{currentfill}%
\pgfsetfillopacity{0.585076}%
\pgfsetlinewidth{1.003750pt}%
\definecolor{currentstroke}{rgb}{0.121569,0.466667,0.705882}%
\pgfsetstrokecolor{currentstroke}%
\pgfsetstrokeopacity{0.585076}%
\pgfsetdash{}{0pt}%
\pgfpathmoveto{\pgfqpoint{1.000159in}{1.757641in}}%
\pgfpathcurveto{\pgfqpoint{1.008395in}{1.757641in}}{\pgfqpoint{1.016295in}{1.760913in}}{\pgfqpoint{1.022119in}{1.766737in}}%
\pgfpathcurveto{\pgfqpoint{1.027943in}{1.772561in}}{\pgfqpoint{1.031215in}{1.780461in}}{\pgfqpoint{1.031215in}{1.788697in}}%
\pgfpathcurveto{\pgfqpoint{1.031215in}{1.796934in}}{\pgfqpoint{1.027943in}{1.804834in}}{\pgfqpoint{1.022119in}{1.810658in}}%
\pgfpathcurveto{\pgfqpoint{1.016295in}{1.816482in}}{\pgfqpoint{1.008395in}{1.819754in}}{\pgfqpoint{1.000159in}{1.819754in}}%
\pgfpathcurveto{\pgfqpoint{0.991922in}{1.819754in}}{\pgfqpoint{0.984022in}{1.816482in}}{\pgfqpoint{0.978198in}{1.810658in}}%
\pgfpathcurveto{\pgfqpoint{0.972374in}{1.804834in}}{\pgfqpoint{0.969102in}{1.796934in}}{\pgfqpoint{0.969102in}{1.788697in}}%
\pgfpathcurveto{\pgfqpoint{0.969102in}{1.780461in}}{\pgfqpoint{0.972374in}{1.772561in}}{\pgfqpoint{0.978198in}{1.766737in}}%
\pgfpathcurveto{\pgfqpoint{0.984022in}{1.760913in}}{\pgfqpoint{0.991922in}{1.757641in}}{\pgfqpoint{1.000159in}{1.757641in}}%
\pgfpathclose%
\pgfusepath{stroke,fill}%
\end{pgfscope}%
\begin{pgfscope}%
\pgfpathrectangle{\pgfqpoint{0.100000in}{0.212622in}}{\pgfqpoint{3.696000in}{3.696000in}}%
\pgfusepath{clip}%
\pgfsetbuttcap%
\pgfsetroundjoin%
\definecolor{currentfill}{rgb}{0.121569,0.466667,0.705882}%
\pgfsetfillcolor{currentfill}%
\pgfsetfillopacity{0.585077}%
\pgfsetlinewidth{1.003750pt}%
\definecolor{currentstroke}{rgb}{0.121569,0.466667,0.705882}%
\pgfsetstrokecolor{currentstroke}%
\pgfsetstrokeopacity{0.585077}%
\pgfsetdash{}{0pt}%
\pgfpathmoveto{\pgfqpoint{0.891639in}{1.491772in}}%
\pgfpathcurveto{\pgfqpoint{0.899876in}{1.491772in}}{\pgfqpoint{0.907776in}{1.495044in}}{\pgfqpoint{0.913600in}{1.500868in}}%
\pgfpathcurveto{\pgfqpoint{0.919424in}{1.506692in}}{\pgfqpoint{0.922696in}{1.514592in}}{\pgfqpoint{0.922696in}{1.522829in}}%
\pgfpathcurveto{\pgfqpoint{0.922696in}{1.531065in}}{\pgfqpoint{0.919424in}{1.538965in}}{\pgfqpoint{0.913600in}{1.544789in}}%
\pgfpathcurveto{\pgfqpoint{0.907776in}{1.550613in}}{\pgfqpoint{0.899876in}{1.553885in}}{\pgfqpoint{0.891639in}{1.553885in}}%
\pgfpathcurveto{\pgfqpoint{0.883403in}{1.553885in}}{\pgfqpoint{0.875503in}{1.550613in}}{\pgfqpoint{0.869679in}{1.544789in}}%
\pgfpathcurveto{\pgfqpoint{0.863855in}{1.538965in}}{\pgfqpoint{0.860583in}{1.531065in}}{\pgfqpoint{0.860583in}{1.522829in}}%
\pgfpathcurveto{\pgfqpoint{0.860583in}{1.514592in}}{\pgfqpoint{0.863855in}{1.506692in}}{\pgfqpoint{0.869679in}{1.500868in}}%
\pgfpathcurveto{\pgfqpoint{0.875503in}{1.495044in}}{\pgfqpoint{0.883403in}{1.491772in}}{\pgfqpoint{0.891639in}{1.491772in}}%
\pgfpathclose%
\pgfusepath{stroke,fill}%
\end{pgfscope}%
\begin{pgfscope}%
\pgfpathrectangle{\pgfqpoint{0.100000in}{0.212622in}}{\pgfqpoint{3.696000in}{3.696000in}}%
\pgfusepath{clip}%
\pgfsetbuttcap%
\pgfsetroundjoin%
\definecolor{currentfill}{rgb}{0.121569,0.466667,0.705882}%
\pgfsetfillcolor{currentfill}%
\pgfsetfillopacity{0.585080}%
\pgfsetlinewidth{1.003750pt}%
\definecolor{currentstroke}{rgb}{0.121569,0.466667,0.705882}%
\pgfsetstrokecolor{currentstroke}%
\pgfsetstrokeopacity{0.585080}%
\pgfsetdash{}{0pt}%
\pgfpathmoveto{\pgfqpoint{0.891634in}{1.491753in}}%
\pgfpathcurveto{\pgfqpoint{0.899870in}{1.491753in}}{\pgfqpoint{0.907770in}{1.495025in}}{\pgfqpoint{0.913594in}{1.500849in}}%
\pgfpathcurveto{\pgfqpoint{0.919418in}{1.506673in}}{\pgfqpoint{0.922691in}{1.514573in}}{\pgfqpoint{0.922691in}{1.522809in}}%
\pgfpathcurveto{\pgfqpoint{0.922691in}{1.531045in}}{\pgfqpoint{0.919418in}{1.538946in}}{\pgfqpoint{0.913594in}{1.544769in}}%
\pgfpathcurveto{\pgfqpoint{0.907770in}{1.550593in}}{\pgfqpoint{0.899870in}{1.553866in}}{\pgfqpoint{0.891634in}{1.553866in}}%
\pgfpathcurveto{\pgfqpoint{0.883398in}{1.553866in}}{\pgfqpoint{0.875498in}{1.550593in}}{\pgfqpoint{0.869674in}{1.544769in}}%
\pgfpathcurveto{\pgfqpoint{0.863850in}{1.538946in}}{\pgfqpoint{0.860578in}{1.531045in}}{\pgfqpoint{0.860578in}{1.522809in}}%
\pgfpathcurveto{\pgfqpoint{0.860578in}{1.514573in}}{\pgfqpoint{0.863850in}{1.506673in}}{\pgfqpoint{0.869674in}{1.500849in}}%
\pgfpathcurveto{\pgfqpoint{0.875498in}{1.495025in}}{\pgfqpoint{0.883398in}{1.491753in}}{\pgfqpoint{0.891634in}{1.491753in}}%
\pgfpathclose%
\pgfusepath{stroke,fill}%
\end{pgfscope}%
\begin{pgfscope}%
\pgfpathrectangle{\pgfqpoint{0.100000in}{0.212622in}}{\pgfqpoint{3.696000in}{3.696000in}}%
\pgfusepath{clip}%
\pgfsetbuttcap%
\pgfsetroundjoin%
\definecolor{currentfill}{rgb}{0.121569,0.466667,0.705882}%
\pgfsetfillcolor{currentfill}%
\pgfsetfillopacity{0.585081}%
\pgfsetlinewidth{1.003750pt}%
\definecolor{currentstroke}{rgb}{0.121569,0.466667,0.705882}%
\pgfsetstrokecolor{currentstroke}%
\pgfsetstrokeopacity{0.585081}%
\pgfsetdash{}{0pt}%
\pgfpathmoveto{\pgfqpoint{0.891631in}{1.491742in}}%
\pgfpathcurveto{\pgfqpoint{0.899867in}{1.491742in}}{\pgfqpoint{0.907767in}{1.495014in}}{\pgfqpoint{0.913591in}{1.500838in}}%
\pgfpathcurveto{\pgfqpoint{0.919415in}{1.506662in}}{\pgfqpoint{0.922688in}{1.514562in}}{\pgfqpoint{0.922688in}{1.522798in}}%
\pgfpathcurveto{\pgfqpoint{0.922688in}{1.531035in}}{\pgfqpoint{0.919415in}{1.538935in}}{\pgfqpoint{0.913591in}{1.544759in}}%
\pgfpathcurveto{\pgfqpoint{0.907767in}{1.550583in}}{\pgfqpoint{0.899867in}{1.553855in}}{\pgfqpoint{0.891631in}{1.553855in}}%
\pgfpathcurveto{\pgfqpoint{0.883395in}{1.553855in}}{\pgfqpoint{0.875495in}{1.550583in}}{\pgfqpoint{0.869671in}{1.544759in}}%
\pgfpathcurveto{\pgfqpoint{0.863847in}{1.538935in}}{\pgfqpoint{0.860575in}{1.531035in}}{\pgfqpoint{0.860575in}{1.522798in}}%
\pgfpathcurveto{\pgfqpoint{0.860575in}{1.514562in}}{\pgfqpoint{0.863847in}{1.506662in}}{\pgfqpoint{0.869671in}{1.500838in}}%
\pgfpathcurveto{\pgfqpoint{0.875495in}{1.495014in}}{\pgfqpoint{0.883395in}{1.491742in}}{\pgfqpoint{0.891631in}{1.491742in}}%
\pgfpathclose%
\pgfusepath{stroke,fill}%
\end{pgfscope}%
\begin{pgfscope}%
\pgfpathrectangle{\pgfqpoint{0.100000in}{0.212622in}}{\pgfqpoint{3.696000in}{3.696000in}}%
\pgfusepath{clip}%
\pgfsetbuttcap%
\pgfsetroundjoin%
\definecolor{currentfill}{rgb}{0.121569,0.466667,0.705882}%
\pgfsetfillcolor{currentfill}%
\pgfsetfillopacity{0.585174}%
\pgfsetlinewidth{1.003750pt}%
\definecolor{currentstroke}{rgb}{0.121569,0.466667,0.705882}%
\pgfsetstrokecolor{currentstroke}%
\pgfsetstrokeopacity{0.585174}%
\pgfsetdash{}{0pt}%
\pgfpathmoveto{\pgfqpoint{0.891477in}{1.491165in}}%
\pgfpathcurveto{\pgfqpoint{0.899714in}{1.491165in}}{\pgfqpoint{0.907614in}{1.494437in}}{\pgfqpoint{0.913438in}{1.500261in}}%
\pgfpathcurveto{\pgfqpoint{0.919262in}{1.506085in}}{\pgfqpoint{0.922534in}{1.513985in}}{\pgfqpoint{0.922534in}{1.522222in}}%
\pgfpathcurveto{\pgfqpoint{0.922534in}{1.530458in}}{\pgfqpoint{0.919262in}{1.538358in}}{\pgfqpoint{0.913438in}{1.544182in}}%
\pgfpathcurveto{\pgfqpoint{0.907614in}{1.550006in}}{\pgfqpoint{0.899714in}{1.553278in}}{\pgfqpoint{0.891477in}{1.553278in}}%
\pgfpathcurveto{\pgfqpoint{0.883241in}{1.553278in}}{\pgfqpoint{0.875341in}{1.550006in}}{\pgfqpoint{0.869517in}{1.544182in}}%
\pgfpathcurveto{\pgfqpoint{0.863693in}{1.538358in}}{\pgfqpoint{0.860421in}{1.530458in}}{\pgfqpoint{0.860421in}{1.522222in}}%
\pgfpathcurveto{\pgfqpoint{0.860421in}{1.513985in}}{\pgfqpoint{0.863693in}{1.506085in}}{\pgfqpoint{0.869517in}{1.500261in}}%
\pgfpathcurveto{\pgfqpoint{0.875341in}{1.494437in}}{\pgfqpoint{0.883241in}{1.491165in}}{\pgfqpoint{0.891477in}{1.491165in}}%
\pgfpathclose%
\pgfusepath{stroke,fill}%
\end{pgfscope}%
\begin{pgfscope}%
\pgfpathrectangle{\pgfqpoint{0.100000in}{0.212622in}}{\pgfqpoint{3.696000in}{3.696000in}}%
\pgfusepath{clip}%
\pgfsetbuttcap%
\pgfsetroundjoin%
\definecolor{currentfill}{rgb}{0.121569,0.466667,0.705882}%
\pgfsetfillcolor{currentfill}%
\pgfsetfillopacity{0.585225}%
\pgfsetlinewidth{1.003750pt}%
\definecolor{currentstroke}{rgb}{0.121569,0.466667,0.705882}%
\pgfsetstrokecolor{currentstroke}%
\pgfsetstrokeopacity{0.585225}%
\pgfsetdash{}{0pt}%
\pgfpathmoveto{\pgfqpoint{0.891391in}{1.490851in}}%
\pgfpathcurveto{\pgfqpoint{0.899627in}{1.490851in}}{\pgfqpoint{0.907527in}{1.494123in}}{\pgfqpoint{0.913351in}{1.499947in}}%
\pgfpathcurveto{\pgfqpoint{0.919175in}{1.505771in}}{\pgfqpoint{0.922447in}{1.513671in}}{\pgfqpoint{0.922447in}{1.521907in}}%
\pgfpathcurveto{\pgfqpoint{0.922447in}{1.530144in}}{\pgfqpoint{0.919175in}{1.538044in}}{\pgfqpoint{0.913351in}{1.543868in}}%
\pgfpathcurveto{\pgfqpoint{0.907527in}{1.549691in}}{\pgfqpoint{0.899627in}{1.552964in}}{\pgfqpoint{0.891391in}{1.552964in}}%
\pgfpathcurveto{\pgfqpoint{0.883154in}{1.552964in}}{\pgfqpoint{0.875254in}{1.549691in}}{\pgfqpoint{0.869431in}{1.543868in}}%
\pgfpathcurveto{\pgfqpoint{0.863607in}{1.538044in}}{\pgfqpoint{0.860334in}{1.530144in}}{\pgfqpoint{0.860334in}{1.521907in}}%
\pgfpathcurveto{\pgfqpoint{0.860334in}{1.513671in}}{\pgfqpoint{0.863607in}{1.505771in}}{\pgfqpoint{0.869431in}{1.499947in}}%
\pgfpathcurveto{\pgfqpoint{0.875254in}{1.494123in}}{\pgfqpoint{0.883154in}{1.490851in}}{\pgfqpoint{0.891391in}{1.490851in}}%
\pgfpathclose%
\pgfusepath{stroke,fill}%
\end{pgfscope}%
\begin{pgfscope}%
\pgfpathrectangle{\pgfqpoint{0.100000in}{0.212622in}}{\pgfqpoint{3.696000in}{3.696000in}}%
\pgfusepath{clip}%
\pgfsetbuttcap%
\pgfsetroundjoin%
\definecolor{currentfill}{rgb}{0.121569,0.466667,0.705882}%
\pgfsetfillcolor{currentfill}%
\pgfsetfillopacity{0.585349}%
\pgfsetlinewidth{1.003750pt}%
\definecolor{currentstroke}{rgb}{0.121569,0.466667,0.705882}%
\pgfsetstrokecolor{currentstroke}%
\pgfsetstrokeopacity{0.585349}%
\pgfsetdash{}{0pt}%
\pgfpathmoveto{\pgfqpoint{0.891179in}{1.490091in}}%
\pgfpathcurveto{\pgfqpoint{0.899415in}{1.490091in}}{\pgfqpoint{0.907315in}{1.493364in}}{\pgfqpoint{0.913139in}{1.499188in}}%
\pgfpathcurveto{\pgfqpoint{0.918963in}{1.505012in}}{\pgfqpoint{0.922235in}{1.512912in}}{\pgfqpoint{0.922235in}{1.521148in}}%
\pgfpathcurveto{\pgfqpoint{0.922235in}{1.529384in}}{\pgfqpoint{0.918963in}{1.537284in}}{\pgfqpoint{0.913139in}{1.543108in}}%
\pgfpathcurveto{\pgfqpoint{0.907315in}{1.548932in}}{\pgfqpoint{0.899415in}{1.552204in}}{\pgfqpoint{0.891179in}{1.552204in}}%
\pgfpathcurveto{\pgfqpoint{0.882942in}{1.552204in}}{\pgfqpoint{0.875042in}{1.548932in}}{\pgfqpoint{0.869218in}{1.543108in}}%
\pgfpathcurveto{\pgfqpoint{0.863394in}{1.537284in}}{\pgfqpoint{0.860122in}{1.529384in}}{\pgfqpoint{0.860122in}{1.521148in}}%
\pgfpathcurveto{\pgfqpoint{0.860122in}{1.512912in}}{\pgfqpoint{0.863394in}{1.505012in}}{\pgfqpoint{0.869218in}{1.499188in}}%
\pgfpathcurveto{\pgfqpoint{0.875042in}{1.493364in}}{\pgfqpoint{0.882942in}{1.490091in}}{\pgfqpoint{0.891179in}{1.490091in}}%
\pgfpathclose%
\pgfusepath{stroke,fill}%
\end{pgfscope}%
\begin{pgfscope}%
\pgfpathrectangle{\pgfqpoint{0.100000in}{0.212622in}}{\pgfqpoint{3.696000in}{3.696000in}}%
\pgfusepath{clip}%
\pgfsetbuttcap%
\pgfsetroundjoin%
\definecolor{currentfill}{rgb}{0.121569,0.466667,0.705882}%
\pgfsetfillcolor{currentfill}%
\pgfsetfillopacity{0.585417}%
\pgfsetlinewidth{1.003750pt}%
\definecolor{currentstroke}{rgb}{0.121569,0.466667,0.705882}%
\pgfsetstrokecolor{currentstroke}%
\pgfsetstrokeopacity{0.585417}%
\pgfsetdash{}{0pt}%
\pgfpathmoveto{\pgfqpoint{0.891058in}{1.489674in}}%
\pgfpathcurveto{\pgfqpoint{0.899295in}{1.489674in}}{\pgfqpoint{0.907195in}{1.492946in}}{\pgfqpoint{0.913019in}{1.498770in}}%
\pgfpathcurveto{\pgfqpoint{0.918843in}{1.504594in}}{\pgfqpoint{0.922115in}{1.512494in}}{\pgfqpoint{0.922115in}{1.520731in}}%
\pgfpathcurveto{\pgfqpoint{0.922115in}{1.528967in}}{\pgfqpoint{0.918843in}{1.536867in}}{\pgfqpoint{0.913019in}{1.542691in}}%
\pgfpathcurveto{\pgfqpoint{0.907195in}{1.548515in}}{\pgfqpoint{0.899295in}{1.551787in}}{\pgfqpoint{0.891058in}{1.551787in}}%
\pgfpathcurveto{\pgfqpoint{0.882822in}{1.551787in}}{\pgfqpoint{0.874922in}{1.548515in}}{\pgfqpoint{0.869098in}{1.542691in}}%
\pgfpathcurveto{\pgfqpoint{0.863274in}{1.536867in}}{\pgfqpoint{0.860002in}{1.528967in}}{\pgfqpoint{0.860002in}{1.520731in}}%
\pgfpathcurveto{\pgfqpoint{0.860002in}{1.512494in}}{\pgfqpoint{0.863274in}{1.504594in}}{\pgfqpoint{0.869098in}{1.498770in}}%
\pgfpathcurveto{\pgfqpoint{0.874922in}{1.492946in}}{\pgfqpoint{0.882822in}{1.489674in}}{\pgfqpoint{0.891058in}{1.489674in}}%
\pgfpathclose%
\pgfusepath{stroke,fill}%
\end{pgfscope}%
\begin{pgfscope}%
\pgfpathrectangle{\pgfqpoint{0.100000in}{0.212622in}}{\pgfqpoint{3.696000in}{3.696000in}}%
\pgfusepath{clip}%
\pgfsetbuttcap%
\pgfsetroundjoin%
\definecolor{currentfill}{rgb}{0.121569,0.466667,0.705882}%
\pgfsetfillcolor{currentfill}%
\pgfsetfillopacity{0.585454}%
\pgfsetlinewidth{1.003750pt}%
\definecolor{currentstroke}{rgb}{0.121569,0.466667,0.705882}%
\pgfsetstrokecolor{currentstroke}%
\pgfsetstrokeopacity{0.585454}%
\pgfsetdash{}{0pt}%
\pgfpathmoveto{\pgfqpoint{0.890994in}{1.489444in}}%
\pgfpathcurveto{\pgfqpoint{0.899230in}{1.489444in}}{\pgfqpoint{0.907130in}{1.492717in}}{\pgfqpoint{0.912954in}{1.498541in}}%
\pgfpathcurveto{\pgfqpoint{0.918778in}{1.504365in}}{\pgfqpoint{0.922050in}{1.512265in}}{\pgfqpoint{0.922050in}{1.520501in}}%
\pgfpathcurveto{\pgfqpoint{0.922050in}{1.528737in}}{\pgfqpoint{0.918778in}{1.536637in}}{\pgfqpoint{0.912954in}{1.542461in}}%
\pgfpathcurveto{\pgfqpoint{0.907130in}{1.548285in}}{\pgfqpoint{0.899230in}{1.551557in}}{\pgfqpoint{0.890994in}{1.551557in}}%
\pgfpathcurveto{\pgfqpoint{0.882757in}{1.551557in}}{\pgfqpoint{0.874857in}{1.548285in}}{\pgfqpoint{0.869033in}{1.542461in}}%
\pgfpathcurveto{\pgfqpoint{0.863210in}{1.536637in}}{\pgfqpoint{0.859937in}{1.528737in}}{\pgfqpoint{0.859937in}{1.520501in}}%
\pgfpathcurveto{\pgfqpoint{0.859937in}{1.512265in}}{\pgfqpoint{0.863210in}{1.504365in}}{\pgfqpoint{0.869033in}{1.498541in}}%
\pgfpathcurveto{\pgfqpoint{0.874857in}{1.492717in}}{\pgfqpoint{0.882757in}{1.489444in}}{\pgfqpoint{0.890994in}{1.489444in}}%
\pgfpathclose%
\pgfusepath{stroke,fill}%
\end{pgfscope}%
\begin{pgfscope}%
\pgfpathrectangle{\pgfqpoint{0.100000in}{0.212622in}}{\pgfqpoint{3.696000in}{3.696000in}}%
\pgfusepath{clip}%
\pgfsetbuttcap%
\pgfsetroundjoin%
\definecolor{currentfill}{rgb}{0.121569,0.466667,0.705882}%
\pgfsetfillcolor{currentfill}%
\pgfsetfillopacity{0.585474}%
\pgfsetlinewidth{1.003750pt}%
\definecolor{currentstroke}{rgb}{0.121569,0.466667,0.705882}%
\pgfsetstrokecolor{currentstroke}%
\pgfsetstrokeopacity{0.585474}%
\pgfsetdash{}{0pt}%
\pgfpathmoveto{\pgfqpoint{0.890959in}{1.489317in}}%
\pgfpathcurveto{\pgfqpoint{0.899195in}{1.489317in}}{\pgfqpoint{0.907095in}{1.492589in}}{\pgfqpoint{0.912919in}{1.498413in}}%
\pgfpathcurveto{\pgfqpoint{0.918743in}{1.504237in}}{\pgfqpoint{0.922015in}{1.512137in}}{\pgfqpoint{0.922015in}{1.520373in}}%
\pgfpathcurveto{\pgfqpoint{0.922015in}{1.528609in}}{\pgfqpoint{0.918743in}{1.536509in}}{\pgfqpoint{0.912919in}{1.542333in}}%
\pgfpathcurveto{\pgfqpoint{0.907095in}{1.548157in}}{\pgfqpoint{0.899195in}{1.551430in}}{\pgfqpoint{0.890959in}{1.551430in}}%
\pgfpathcurveto{\pgfqpoint{0.882722in}{1.551430in}}{\pgfqpoint{0.874822in}{1.548157in}}{\pgfqpoint{0.868998in}{1.542333in}}%
\pgfpathcurveto{\pgfqpoint{0.863174in}{1.536509in}}{\pgfqpoint{0.859902in}{1.528609in}}{\pgfqpoint{0.859902in}{1.520373in}}%
\pgfpathcurveto{\pgfqpoint{0.859902in}{1.512137in}}{\pgfqpoint{0.863174in}{1.504237in}}{\pgfqpoint{0.868998in}{1.498413in}}%
\pgfpathcurveto{\pgfqpoint{0.874822in}{1.492589in}}{\pgfqpoint{0.882722in}{1.489317in}}{\pgfqpoint{0.890959in}{1.489317in}}%
\pgfpathclose%
\pgfusepath{stroke,fill}%
\end{pgfscope}%
\begin{pgfscope}%
\pgfpathrectangle{\pgfqpoint{0.100000in}{0.212622in}}{\pgfqpoint{3.696000in}{3.696000in}}%
\pgfusepath{clip}%
\pgfsetbuttcap%
\pgfsetroundjoin%
\definecolor{currentfill}{rgb}{0.121569,0.466667,0.705882}%
\pgfsetfillcolor{currentfill}%
\pgfsetfillopacity{0.585486}%
\pgfsetlinewidth{1.003750pt}%
\definecolor{currentstroke}{rgb}{0.121569,0.466667,0.705882}%
\pgfsetstrokecolor{currentstroke}%
\pgfsetstrokeopacity{0.585486}%
\pgfsetdash{}{0pt}%
\pgfpathmoveto{\pgfqpoint{0.890939in}{1.489246in}}%
\pgfpathcurveto{\pgfqpoint{0.899175in}{1.489246in}}{\pgfqpoint{0.907075in}{1.492519in}}{\pgfqpoint{0.912899in}{1.498343in}}%
\pgfpathcurveto{\pgfqpoint{0.918723in}{1.504166in}}{\pgfqpoint{0.921996in}{1.512067in}}{\pgfqpoint{0.921996in}{1.520303in}}%
\pgfpathcurveto{\pgfqpoint{0.921996in}{1.528539in}}{\pgfqpoint{0.918723in}{1.536439in}}{\pgfqpoint{0.912899in}{1.542263in}}%
\pgfpathcurveto{\pgfqpoint{0.907075in}{1.548087in}}{\pgfqpoint{0.899175in}{1.551359in}}{\pgfqpoint{0.890939in}{1.551359in}}%
\pgfpathcurveto{\pgfqpoint{0.882703in}{1.551359in}}{\pgfqpoint{0.874803in}{1.548087in}}{\pgfqpoint{0.868979in}{1.542263in}}%
\pgfpathcurveto{\pgfqpoint{0.863155in}{1.536439in}}{\pgfqpoint{0.859883in}{1.528539in}}{\pgfqpoint{0.859883in}{1.520303in}}%
\pgfpathcurveto{\pgfqpoint{0.859883in}{1.512067in}}{\pgfqpoint{0.863155in}{1.504166in}}{\pgfqpoint{0.868979in}{1.498343in}}%
\pgfpathcurveto{\pgfqpoint{0.874803in}{1.492519in}}{\pgfqpoint{0.882703in}{1.489246in}}{\pgfqpoint{0.890939in}{1.489246in}}%
\pgfpathclose%
\pgfusepath{stroke,fill}%
\end{pgfscope}%
\begin{pgfscope}%
\pgfpathrectangle{\pgfqpoint{0.100000in}{0.212622in}}{\pgfqpoint{3.696000in}{3.696000in}}%
\pgfusepath{clip}%
\pgfsetbuttcap%
\pgfsetroundjoin%
\definecolor{currentfill}{rgb}{0.121569,0.466667,0.705882}%
\pgfsetfillcolor{currentfill}%
\pgfsetfillopacity{0.585492}%
\pgfsetlinewidth{1.003750pt}%
\definecolor{currentstroke}{rgb}{0.121569,0.466667,0.705882}%
\pgfsetstrokecolor{currentstroke}%
\pgfsetstrokeopacity{0.585492}%
\pgfsetdash{}{0pt}%
\pgfpathmoveto{\pgfqpoint{0.890929in}{1.489208in}}%
\pgfpathcurveto{\pgfqpoint{0.899165in}{1.489208in}}{\pgfqpoint{0.907065in}{1.492480in}}{\pgfqpoint{0.912889in}{1.498304in}}%
\pgfpathcurveto{\pgfqpoint{0.918713in}{1.504128in}}{\pgfqpoint{0.921985in}{1.512028in}}{\pgfqpoint{0.921985in}{1.520264in}}%
\pgfpathcurveto{\pgfqpoint{0.921985in}{1.528500in}}{\pgfqpoint{0.918713in}{1.536400in}}{\pgfqpoint{0.912889in}{1.542224in}}%
\pgfpathcurveto{\pgfqpoint{0.907065in}{1.548048in}}{\pgfqpoint{0.899165in}{1.551321in}}{\pgfqpoint{0.890929in}{1.551321in}}%
\pgfpathcurveto{\pgfqpoint{0.882693in}{1.551321in}}{\pgfqpoint{0.874793in}{1.548048in}}{\pgfqpoint{0.868969in}{1.542224in}}%
\pgfpathcurveto{\pgfqpoint{0.863145in}{1.536400in}}{\pgfqpoint{0.859872in}{1.528500in}}{\pgfqpoint{0.859872in}{1.520264in}}%
\pgfpathcurveto{\pgfqpoint{0.859872in}{1.512028in}}{\pgfqpoint{0.863145in}{1.504128in}}{\pgfqpoint{0.868969in}{1.498304in}}%
\pgfpathcurveto{\pgfqpoint{0.874793in}{1.492480in}}{\pgfqpoint{0.882693in}{1.489208in}}{\pgfqpoint{0.890929in}{1.489208in}}%
\pgfpathclose%
\pgfusepath{stroke,fill}%
\end{pgfscope}%
\begin{pgfscope}%
\pgfpathrectangle{\pgfqpoint{0.100000in}{0.212622in}}{\pgfqpoint{3.696000in}{3.696000in}}%
\pgfusepath{clip}%
\pgfsetbuttcap%
\pgfsetroundjoin%
\definecolor{currentfill}{rgb}{0.121569,0.466667,0.705882}%
\pgfsetfillcolor{currentfill}%
\pgfsetfillopacity{0.585495}%
\pgfsetlinewidth{1.003750pt}%
\definecolor{currentstroke}{rgb}{0.121569,0.466667,0.705882}%
\pgfsetstrokecolor{currentstroke}%
\pgfsetstrokeopacity{0.585495}%
\pgfsetdash{}{0pt}%
\pgfpathmoveto{\pgfqpoint{0.890923in}{1.489186in}}%
\pgfpathcurveto{\pgfqpoint{0.899159in}{1.489186in}}{\pgfqpoint{0.907059in}{1.492458in}}{\pgfqpoint{0.912883in}{1.498282in}}%
\pgfpathcurveto{\pgfqpoint{0.918707in}{1.504106in}}{\pgfqpoint{0.921980in}{1.512006in}}{\pgfqpoint{0.921980in}{1.520243in}}%
\pgfpathcurveto{\pgfqpoint{0.921980in}{1.528479in}}{\pgfqpoint{0.918707in}{1.536379in}}{\pgfqpoint{0.912883in}{1.542203in}}%
\pgfpathcurveto{\pgfqpoint{0.907059in}{1.548027in}}{\pgfqpoint{0.899159in}{1.551299in}}{\pgfqpoint{0.890923in}{1.551299in}}%
\pgfpathcurveto{\pgfqpoint{0.882687in}{1.551299in}}{\pgfqpoint{0.874787in}{1.548027in}}{\pgfqpoint{0.868963in}{1.542203in}}%
\pgfpathcurveto{\pgfqpoint{0.863139in}{1.536379in}}{\pgfqpoint{0.859867in}{1.528479in}}{\pgfqpoint{0.859867in}{1.520243in}}%
\pgfpathcurveto{\pgfqpoint{0.859867in}{1.512006in}}{\pgfqpoint{0.863139in}{1.504106in}}{\pgfqpoint{0.868963in}{1.498282in}}%
\pgfpathcurveto{\pgfqpoint{0.874787in}{1.492458in}}{\pgfqpoint{0.882687in}{1.489186in}}{\pgfqpoint{0.890923in}{1.489186in}}%
\pgfpathclose%
\pgfusepath{stroke,fill}%
\end{pgfscope}%
\begin{pgfscope}%
\pgfpathrectangle{\pgfqpoint{0.100000in}{0.212622in}}{\pgfqpoint{3.696000in}{3.696000in}}%
\pgfusepath{clip}%
\pgfsetbuttcap%
\pgfsetroundjoin%
\definecolor{currentfill}{rgb}{0.121569,0.466667,0.705882}%
\pgfsetfillcolor{currentfill}%
\pgfsetfillopacity{0.585497}%
\pgfsetlinewidth{1.003750pt}%
\definecolor{currentstroke}{rgb}{0.121569,0.466667,0.705882}%
\pgfsetstrokecolor{currentstroke}%
\pgfsetstrokeopacity{0.585497}%
\pgfsetdash{}{0pt}%
\pgfpathmoveto{\pgfqpoint{0.890920in}{1.489174in}}%
\pgfpathcurveto{\pgfqpoint{0.899156in}{1.489174in}}{\pgfqpoint{0.907056in}{1.492447in}}{\pgfqpoint{0.912880in}{1.498271in}}%
\pgfpathcurveto{\pgfqpoint{0.918704in}{1.504095in}}{\pgfqpoint{0.921976in}{1.511995in}}{\pgfqpoint{0.921976in}{1.520231in}}%
\pgfpathcurveto{\pgfqpoint{0.921976in}{1.528467in}}{\pgfqpoint{0.918704in}{1.536367in}}{\pgfqpoint{0.912880in}{1.542191in}}%
\pgfpathcurveto{\pgfqpoint{0.907056in}{1.548015in}}{\pgfqpoint{0.899156in}{1.551287in}}{\pgfqpoint{0.890920in}{1.551287in}}%
\pgfpathcurveto{\pgfqpoint{0.882684in}{1.551287in}}{\pgfqpoint{0.874784in}{1.548015in}}{\pgfqpoint{0.868960in}{1.542191in}}%
\pgfpathcurveto{\pgfqpoint{0.863136in}{1.536367in}}{\pgfqpoint{0.859863in}{1.528467in}}{\pgfqpoint{0.859863in}{1.520231in}}%
\pgfpathcurveto{\pgfqpoint{0.859863in}{1.511995in}}{\pgfqpoint{0.863136in}{1.504095in}}{\pgfqpoint{0.868960in}{1.498271in}}%
\pgfpathcurveto{\pgfqpoint{0.874784in}{1.492447in}}{\pgfqpoint{0.882684in}{1.489174in}}{\pgfqpoint{0.890920in}{1.489174in}}%
\pgfpathclose%
\pgfusepath{stroke,fill}%
\end{pgfscope}%
\begin{pgfscope}%
\pgfpathrectangle{\pgfqpoint{0.100000in}{0.212622in}}{\pgfqpoint{3.696000in}{3.696000in}}%
\pgfusepath{clip}%
\pgfsetbuttcap%
\pgfsetroundjoin%
\definecolor{currentfill}{rgb}{0.121569,0.466667,0.705882}%
\pgfsetfillcolor{currentfill}%
\pgfsetfillopacity{0.585498}%
\pgfsetlinewidth{1.003750pt}%
\definecolor{currentstroke}{rgb}{0.121569,0.466667,0.705882}%
\pgfsetstrokecolor{currentstroke}%
\pgfsetstrokeopacity{0.585498}%
\pgfsetdash{}{0pt}%
\pgfpathmoveto{\pgfqpoint{0.890918in}{1.489168in}}%
\pgfpathcurveto{\pgfqpoint{0.899155in}{1.489168in}}{\pgfqpoint{0.907055in}{1.492440in}}{\pgfqpoint{0.912878in}{1.498264in}}%
\pgfpathcurveto{\pgfqpoint{0.918702in}{1.504088in}}{\pgfqpoint{0.921975in}{1.511988in}}{\pgfqpoint{0.921975in}{1.520225in}}%
\pgfpathcurveto{\pgfqpoint{0.921975in}{1.528461in}}{\pgfqpoint{0.918702in}{1.536361in}}{\pgfqpoint{0.912878in}{1.542185in}}%
\pgfpathcurveto{\pgfqpoint{0.907055in}{1.548009in}}{\pgfqpoint{0.899155in}{1.551281in}}{\pgfqpoint{0.890918in}{1.551281in}}%
\pgfpathcurveto{\pgfqpoint{0.882682in}{1.551281in}}{\pgfqpoint{0.874782in}{1.548009in}}{\pgfqpoint{0.868958in}{1.542185in}}%
\pgfpathcurveto{\pgfqpoint{0.863134in}{1.536361in}}{\pgfqpoint{0.859862in}{1.528461in}}{\pgfqpoint{0.859862in}{1.520225in}}%
\pgfpathcurveto{\pgfqpoint{0.859862in}{1.511988in}}{\pgfqpoint{0.863134in}{1.504088in}}{\pgfqpoint{0.868958in}{1.498264in}}%
\pgfpathcurveto{\pgfqpoint{0.874782in}{1.492440in}}{\pgfqpoint{0.882682in}{1.489168in}}{\pgfqpoint{0.890918in}{1.489168in}}%
\pgfpathclose%
\pgfusepath{stroke,fill}%
\end{pgfscope}%
\begin{pgfscope}%
\pgfpathrectangle{\pgfqpoint{0.100000in}{0.212622in}}{\pgfqpoint{3.696000in}{3.696000in}}%
\pgfusepath{clip}%
\pgfsetbuttcap%
\pgfsetroundjoin%
\definecolor{currentfill}{rgb}{0.121569,0.466667,0.705882}%
\pgfsetfillcolor{currentfill}%
\pgfsetfillopacity{0.585498}%
\pgfsetlinewidth{1.003750pt}%
\definecolor{currentstroke}{rgb}{0.121569,0.466667,0.705882}%
\pgfsetstrokecolor{currentstroke}%
\pgfsetstrokeopacity{0.585498}%
\pgfsetdash{}{0pt}%
\pgfpathmoveto{\pgfqpoint{0.890917in}{1.489165in}}%
\pgfpathcurveto{\pgfqpoint{0.899154in}{1.489165in}}{\pgfqpoint{0.907054in}{1.492437in}}{\pgfqpoint{0.912878in}{1.498261in}}%
\pgfpathcurveto{\pgfqpoint{0.918701in}{1.504085in}}{\pgfqpoint{0.921974in}{1.511985in}}{\pgfqpoint{0.921974in}{1.520221in}}%
\pgfpathcurveto{\pgfqpoint{0.921974in}{1.528457in}}{\pgfqpoint{0.918701in}{1.536357in}}{\pgfqpoint{0.912878in}{1.542181in}}%
\pgfpathcurveto{\pgfqpoint{0.907054in}{1.548005in}}{\pgfqpoint{0.899154in}{1.551278in}}{\pgfqpoint{0.890917in}{1.551278in}}%
\pgfpathcurveto{\pgfqpoint{0.882681in}{1.551278in}}{\pgfqpoint{0.874781in}{1.548005in}}{\pgfqpoint{0.868957in}{1.542181in}}%
\pgfpathcurveto{\pgfqpoint{0.863133in}{1.536357in}}{\pgfqpoint{0.859861in}{1.528457in}}{\pgfqpoint{0.859861in}{1.520221in}}%
\pgfpathcurveto{\pgfqpoint{0.859861in}{1.511985in}}{\pgfqpoint{0.863133in}{1.504085in}}{\pgfqpoint{0.868957in}{1.498261in}}%
\pgfpathcurveto{\pgfqpoint{0.874781in}{1.492437in}}{\pgfqpoint{0.882681in}{1.489165in}}{\pgfqpoint{0.890917in}{1.489165in}}%
\pgfpathclose%
\pgfusepath{stroke,fill}%
\end{pgfscope}%
\begin{pgfscope}%
\pgfpathrectangle{\pgfqpoint{0.100000in}{0.212622in}}{\pgfqpoint{3.696000in}{3.696000in}}%
\pgfusepath{clip}%
\pgfsetbuttcap%
\pgfsetroundjoin%
\definecolor{currentfill}{rgb}{0.121569,0.466667,0.705882}%
\pgfsetfillcolor{currentfill}%
\pgfsetfillopacity{0.585499}%
\pgfsetlinewidth{1.003750pt}%
\definecolor{currentstroke}{rgb}{0.121569,0.466667,0.705882}%
\pgfsetstrokecolor{currentstroke}%
\pgfsetstrokeopacity{0.585499}%
\pgfsetdash{}{0pt}%
\pgfpathmoveto{\pgfqpoint{0.890917in}{1.489163in}}%
\pgfpathcurveto{\pgfqpoint{0.899153in}{1.489163in}}{\pgfqpoint{0.907053in}{1.492435in}}{\pgfqpoint{0.912877in}{1.498259in}}%
\pgfpathcurveto{\pgfqpoint{0.918701in}{1.504083in}}{\pgfqpoint{0.921973in}{1.511983in}}{\pgfqpoint{0.921973in}{1.520219in}}%
\pgfpathcurveto{\pgfqpoint{0.921973in}{1.528455in}}{\pgfqpoint{0.918701in}{1.536355in}}{\pgfqpoint{0.912877in}{1.542179in}}%
\pgfpathcurveto{\pgfqpoint{0.907053in}{1.548003in}}{\pgfqpoint{0.899153in}{1.551276in}}{\pgfqpoint{0.890917in}{1.551276in}}%
\pgfpathcurveto{\pgfqpoint{0.882680in}{1.551276in}}{\pgfqpoint{0.874780in}{1.548003in}}{\pgfqpoint{0.868956in}{1.542179in}}%
\pgfpathcurveto{\pgfqpoint{0.863133in}{1.536355in}}{\pgfqpoint{0.859860in}{1.528455in}}{\pgfqpoint{0.859860in}{1.520219in}}%
\pgfpathcurveto{\pgfqpoint{0.859860in}{1.511983in}}{\pgfqpoint{0.863133in}{1.504083in}}{\pgfqpoint{0.868956in}{1.498259in}}%
\pgfpathcurveto{\pgfqpoint{0.874780in}{1.492435in}}{\pgfqpoint{0.882680in}{1.489163in}}{\pgfqpoint{0.890917in}{1.489163in}}%
\pgfpathclose%
\pgfusepath{stroke,fill}%
\end{pgfscope}%
\begin{pgfscope}%
\pgfpathrectangle{\pgfqpoint{0.100000in}{0.212622in}}{\pgfqpoint{3.696000in}{3.696000in}}%
\pgfusepath{clip}%
\pgfsetbuttcap%
\pgfsetroundjoin%
\definecolor{currentfill}{rgb}{0.121569,0.466667,0.705882}%
\pgfsetfillcolor{currentfill}%
\pgfsetfillopacity{0.585499}%
\pgfsetlinewidth{1.003750pt}%
\definecolor{currentstroke}{rgb}{0.121569,0.466667,0.705882}%
\pgfsetstrokecolor{currentstroke}%
\pgfsetstrokeopacity{0.585499}%
\pgfsetdash{}{0pt}%
\pgfpathmoveto{\pgfqpoint{0.890916in}{1.489162in}}%
\pgfpathcurveto{\pgfqpoint{0.899153in}{1.489162in}}{\pgfqpoint{0.907053in}{1.492434in}}{\pgfqpoint{0.912877in}{1.498258in}}%
\pgfpathcurveto{\pgfqpoint{0.918701in}{1.504082in}}{\pgfqpoint{0.921973in}{1.511982in}}{\pgfqpoint{0.921973in}{1.520218in}}%
\pgfpathcurveto{\pgfqpoint{0.921973in}{1.528454in}}{\pgfqpoint{0.918701in}{1.536354in}}{\pgfqpoint{0.912877in}{1.542178in}}%
\pgfpathcurveto{\pgfqpoint{0.907053in}{1.548002in}}{\pgfqpoint{0.899153in}{1.551275in}}{\pgfqpoint{0.890916in}{1.551275in}}%
\pgfpathcurveto{\pgfqpoint{0.882680in}{1.551275in}}{\pgfqpoint{0.874780in}{1.548002in}}{\pgfqpoint{0.868956in}{1.542178in}}%
\pgfpathcurveto{\pgfqpoint{0.863132in}{1.536354in}}{\pgfqpoint{0.859860in}{1.528454in}}{\pgfqpoint{0.859860in}{1.520218in}}%
\pgfpathcurveto{\pgfqpoint{0.859860in}{1.511982in}}{\pgfqpoint{0.863132in}{1.504082in}}{\pgfqpoint{0.868956in}{1.498258in}}%
\pgfpathcurveto{\pgfqpoint{0.874780in}{1.492434in}}{\pgfqpoint{0.882680in}{1.489162in}}{\pgfqpoint{0.890916in}{1.489162in}}%
\pgfpathclose%
\pgfusepath{stroke,fill}%
\end{pgfscope}%
\begin{pgfscope}%
\pgfpathrectangle{\pgfqpoint{0.100000in}{0.212622in}}{\pgfqpoint{3.696000in}{3.696000in}}%
\pgfusepath{clip}%
\pgfsetbuttcap%
\pgfsetroundjoin%
\definecolor{currentfill}{rgb}{0.121569,0.466667,0.705882}%
\pgfsetfillcolor{currentfill}%
\pgfsetfillopacity{0.585499}%
\pgfsetlinewidth{1.003750pt}%
\definecolor{currentstroke}{rgb}{0.121569,0.466667,0.705882}%
\pgfsetstrokecolor{currentstroke}%
\pgfsetstrokeopacity{0.585499}%
\pgfsetdash{}{0pt}%
\pgfpathmoveto{\pgfqpoint{0.890916in}{1.489161in}}%
\pgfpathcurveto{\pgfqpoint{0.899153in}{1.489161in}}{\pgfqpoint{0.907053in}{1.492433in}}{\pgfqpoint{0.912877in}{1.498257in}}%
\pgfpathcurveto{\pgfqpoint{0.918700in}{1.504081in}}{\pgfqpoint{0.921973in}{1.511981in}}{\pgfqpoint{0.921973in}{1.520217in}}%
\pgfpathcurveto{\pgfqpoint{0.921973in}{1.528454in}}{\pgfqpoint{0.918700in}{1.536354in}}{\pgfqpoint{0.912877in}{1.542178in}}%
\pgfpathcurveto{\pgfqpoint{0.907053in}{1.548002in}}{\pgfqpoint{0.899153in}{1.551274in}}{\pgfqpoint{0.890916in}{1.551274in}}%
\pgfpathcurveto{\pgfqpoint{0.882680in}{1.551274in}}{\pgfqpoint{0.874780in}{1.548002in}}{\pgfqpoint{0.868956in}{1.542178in}}%
\pgfpathcurveto{\pgfqpoint{0.863132in}{1.536354in}}{\pgfqpoint{0.859860in}{1.528454in}}{\pgfqpoint{0.859860in}{1.520217in}}%
\pgfpathcurveto{\pgfqpoint{0.859860in}{1.511981in}}{\pgfqpoint{0.863132in}{1.504081in}}{\pgfqpoint{0.868956in}{1.498257in}}%
\pgfpathcurveto{\pgfqpoint{0.874780in}{1.492433in}}{\pgfqpoint{0.882680in}{1.489161in}}{\pgfqpoint{0.890916in}{1.489161in}}%
\pgfpathclose%
\pgfusepath{stroke,fill}%
\end{pgfscope}%
\begin{pgfscope}%
\pgfpathrectangle{\pgfqpoint{0.100000in}{0.212622in}}{\pgfqpoint{3.696000in}{3.696000in}}%
\pgfusepath{clip}%
\pgfsetbuttcap%
\pgfsetroundjoin%
\definecolor{currentfill}{rgb}{0.121569,0.466667,0.705882}%
\pgfsetfillcolor{currentfill}%
\pgfsetfillopacity{0.585499}%
\pgfsetlinewidth{1.003750pt}%
\definecolor{currentstroke}{rgb}{0.121569,0.466667,0.705882}%
\pgfsetstrokecolor{currentstroke}%
\pgfsetstrokeopacity{0.585499}%
\pgfsetdash{}{0pt}%
\pgfpathmoveto{\pgfqpoint{0.890916in}{1.489161in}}%
\pgfpathcurveto{\pgfqpoint{0.899152in}{1.489161in}}{\pgfqpoint{0.907053in}{1.492433in}}{\pgfqpoint{0.912876in}{1.498257in}}%
\pgfpathcurveto{\pgfqpoint{0.918700in}{1.504081in}}{\pgfqpoint{0.921973in}{1.511981in}}{\pgfqpoint{0.921973in}{1.520217in}}%
\pgfpathcurveto{\pgfqpoint{0.921973in}{1.528453in}}{\pgfqpoint{0.918700in}{1.536353in}}{\pgfqpoint{0.912876in}{1.542177in}}%
\pgfpathcurveto{\pgfqpoint{0.907053in}{1.548001in}}{\pgfqpoint{0.899152in}{1.551274in}}{\pgfqpoint{0.890916in}{1.551274in}}%
\pgfpathcurveto{\pgfqpoint{0.882680in}{1.551274in}}{\pgfqpoint{0.874780in}{1.548001in}}{\pgfqpoint{0.868956in}{1.542177in}}%
\pgfpathcurveto{\pgfqpoint{0.863132in}{1.536353in}}{\pgfqpoint{0.859860in}{1.528453in}}{\pgfqpoint{0.859860in}{1.520217in}}%
\pgfpathcurveto{\pgfqpoint{0.859860in}{1.511981in}}{\pgfqpoint{0.863132in}{1.504081in}}{\pgfqpoint{0.868956in}{1.498257in}}%
\pgfpathcurveto{\pgfqpoint{0.874780in}{1.492433in}}{\pgfqpoint{0.882680in}{1.489161in}}{\pgfqpoint{0.890916in}{1.489161in}}%
\pgfpathclose%
\pgfusepath{stroke,fill}%
\end{pgfscope}%
\begin{pgfscope}%
\pgfpathrectangle{\pgfqpoint{0.100000in}{0.212622in}}{\pgfqpoint{3.696000in}{3.696000in}}%
\pgfusepath{clip}%
\pgfsetbuttcap%
\pgfsetroundjoin%
\definecolor{currentfill}{rgb}{0.121569,0.466667,0.705882}%
\pgfsetfillcolor{currentfill}%
\pgfsetfillopacity{0.585499}%
\pgfsetlinewidth{1.003750pt}%
\definecolor{currentstroke}{rgb}{0.121569,0.466667,0.705882}%
\pgfsetstrokecolor{currentstroke}%
\pgfsetstrokeopacity{0.585499}%
\pgfsetdash{}{0pt}%
\pgfpathmoveto{\pgfqpoint{0.890916in}{1.489160in}}%
\pgfpathcurveto{\pgfqpoint{0.899152in}{1.489160in}}{\pgfqpoint{0.907052in}{1.492433in}}{\pgfqpoint{0.912876in}{1.498257in}}%
\pgfpathcurveto{\pgfqpoint{0.918700in}{1.504081in}}{\pgfqpoint{0.921973in}{1.511981in}}{\pgfqpoint{0.921973in}{1.520217in}}%
\pgfpathcurveto{\pgfqpoint{0.921973in}{1.528453in}}{\pgfqpoint{0.918700in}{1.536353in}}{\pgfqpoint{0.912876in}{1.542177in}}%
\pgfpathcurveto{\pgfqpoint{0.907052in}{1.548001in}}{\pgfqpoint{0.899152in}{1.551273in}}{\pgfqpoint{0.890916in}{1.551273in}}%
\pgfpathcurveto{\pgfqpoint{0.882680in}{1.551273in}}{\pgfqpoint{0.874780in}{1.548001in}}{\pgfqpoint{0.868956in}{1.542177in}}%
\pgfpathcurveto{\pgfqpoint{0.863132in}{1.536353in}}{\pgfqpoint{0.859860in}{1.528453in}}{\pgfqpoint{0.859860in}{1.520217in}}%
\pgfpathcurveto{\pgfqpoint{0.859860in}{1.511981in}}{\pgfqpoint{0.863132in}{1.504081in}}{\pgfqpoint{0.868956in}{1.498257in}}%
\pgfpathcurveto{\pgfqpoint{0.874780in}{1.492433in}}{\pgfqpoint{0.882680in}{1.489160in}}{\pgfqpoint{0.890916in}{1.489160in}}%
\pgfpathclose%
\pgfusepath{stroke,fill}%
\end{pgfscope}%
\begin{pgfscope}%
\pgfpathrectangle{\pgfqpoint{0.100000in}{0.212622in}}{\pgfqpoint{3.696000in}{3.696000in}}%
\pgfusepath{clip}%
\pgfsetbuttcap%
\pgfsetroundjoin%
\definecolor{currentfill}{rgb}{0.121569,0.466667,0.705882}%
\pgfsetfillcolor{currentfill}%
\pgfsetfillopacity{0.585499}%
\pgfsetlinewidth{1.003750pt}%
\definecolor{currentstroke}{rgb}{0.121569,0.466667,0.705882}%
\pgfsetstrokecolor{currentstroke}%
\pgfsetstrokeopacity{0.585499}%
\pgfsetdash{}{0pt}%
\pgfpathmoveto{\pgfqpoint{0.890916in}{1.489160in}}%
\pgfpathcurveto{\pgfqpoint{0.899152in}{1.489160in}}{\pgfqpoint{0.907052in}{1.492433in}}{\pgfqpoint{0.912876in}{1.498257in}}%
\pgfpathcurveto{\pgfqpoint{0.918700in}{1.504081in}}{\pgfqpoint{0.921973in}{1.511981in}}{\pgfqpoint{0.921973in}{1.520217in}}%
\pgfpathcurveto{\pgfqpoint{0.921973in}{1.528453in}}{\pgfqpoint{0.918700in}{1.536353in}}{\pgfqpoint{0.912876in}{1.542177in}}%
\pgfpathcurveto{\pgfqpoint{0.907052in}{1.548001in}}{\pgfqpoint{0.899152in}{1.551273in}}{\pgfqpoint{0.890916in}{1.551273in}}%
\pgfpathcurveto{\pgfqpoint{0.882680in}{1.551273in}}{\pgfqpoint{0.874780in}{1.548001in}}{\pgfqpoint{0.868956in}{1.542177in}}%
\pgfpathcurveto{\pgfqpoint{0.863132in}{1.536353in}}{\pgfqpoint{0.859860in}{1.528453in}}{\pgfqpoint{0.859860in}{1.520217in}}%
\pgfpathcurveto{\pgfqpoint{0.859860in}{1.511981in}}{\pgfqpoint{0.863132in}{1.504081in}}{\pgfqpoint{0.868956in}{1.498257in}}%
\pgfpathcurveto{\pgfqpoint{0.874780in}{1.492433in}}{\pgfqpoint{0.882680in}{1.489160in}}{\pgfqpoint{0.890916in}{1.489160in}}%
\pgfpathclose%
\pgfusepath{stroke,fill}%
\end{pgfscope}%
\begin{pgfscope}%
\pgfpathrectangle{\pgfqpoint{0.100000in}{0.212622in}}{\pgfqpoint{3.696000in}{3.696000in}}%
\pgfusepath{clip}%
\pgfsetbuttcap%
\pgfsetroundjoin%
\definecolor{currentfill}{rgb}{0.121569,0.466667,0.705882}%
\pgfsetfillcolor{currentfill}%
\pgfsetfillopacity{0.585499}%
\pgfsetlinewidth{1.003750pt}%
\definecolor{currentstroke}{rgb}{0.121569,0.466667,0.705882}%
\pgfsetstrokecolor{currentstroke}%
\pgfsetstrokeopacity{0.585499}%
\pgfsetdash{}{0pt}%
\pgfpathmoveto{\pgfqpoint{0.890916in}{1.489160in}}%
\pgfpathcurveto{\pgfqpoint{0.899152in}{1.489160in}}{\pgfqpoint{0.907052in}{1.492433in}}{\pgfqpoint{0.912876in}{1.498257in}}%
\pgfpathcurveto{\pgfqpoint{0.918700in}{1.504080in}}{\pgfqpoint{0.921973in}{1.511981in}}{\pgfqpoint{0.921973in}{1.520217in}}%
\pgfpathcurveto{\pgfqpoint{0.921973in}{1.528453in}}{\pgfqpoint{0.918700in}{1.536353in}}{\pgfqpoint{0.912876in}{1.542177in}}%
\pgfpathcurveto{\pgfqpoint{0.907052in}{1.548001in}}{\pgfqpoint{0.899152in}{1.551273in}}{\pgfqpoint{0.890916in}{1.551273in}}%
\pgfpathcurveto{\pgfqpoint{0.882680in}{1.551273in}}{\pgfqpoint{0.874780in}{1.548001in}}{\pgfqpoint{0.868956in}{1.542177in}}%
\pgfpathcurveto{\pgfqpoint{0.863132in}{1.536353in}}{\pgfqpoint{0.859860in}{1.528453in}}{\pgfqpoint{0.859860in}{1.520217in}}%
\pgfpathcurveto{\pgfqpoint{0.859860in}{1.511981in}}{\pgfqpoint{0.863132in}{1.504080in}}{\pgfqpoint{0.868956in}{1.498257in}}%
\pgfpathcurveto{\pgfqpoint{0.874780in}{1.492433in}}{\pgfqpoint{0.882680in}{1.489160in}}{\pgfqpoint{0.890916in}{1.489160in}}%
\pgfpathclose%
\pgfusepath{stroke,fill}%
\end{pgfscope}%
\begin{pgfscope}%
\pgfpathrectangle{\pgfqpoint{0.100000in}{0.212622in}}{\pgfqpoint{3.696000in}{3.696000in}}%
\pgfusepath{clip}%
\pgfsetbuttcap%
\pgfsetroundjoin%
\definecolor{currentfill}{rgb}{0.121569,0.466667,0.705882}%
\pgfsetfillcolor{currentfill}%
\pgfsetfillopacity{0.585499}%
\pgfsetlinewidth{1.003750pt}%
\definecolor{currentstroke}{rgb}{0.121569,0.466667,0.705882}%
\pgfsetstrokecolor{currentstroke}%
\pgfsetstrokeopacity{0.585499}%
\pgfsetdash{}{0pt}%
\pgfpathmoveto{\pgfqpoint{0.890916in}{1.489160in}}%
\pgfpathcurveto{\pgfqpoint{0.899152in}{1.489160in}}{\pgfqpoint{0.907052in}{1.492433in}}{\pgfqpoint{0.912876in}{1.498256in}}%
\pgfpathcurveto{\pgfqpoint{0.918700in}{1.504080in}}{\pgfqpoint{0.921973in}{1.511980in}}{\pgfqpoint{0.921973in}{1.520217in}}%
\pgfpathcurveto{\pgfqpoint{0.921973in}{1.528453in}}{\pgfqpoint{0.918700in}{1.536353in}}{\pgfqpoint{0.912876in}{1.542177in}}%
\pgfpathcurveto{\pgfqpoint{0.907052in}{1.548001in}}{\pgfqpoint{0.899152in}{1.551273in}}{\pgfqpoint{0.890916in}{1.551273in}}%
\pgfpathcurveto{\pgfqpoint{0.882680in}{1.551273in}}{\pgfqpoint{0.874780in}{1.548001in}}{\pgfqpoint{0.868956in}{1.542177in}}%
\pgfpathcurveto{\pgfqpoint{0.863132in}{1.536353in}}{\pgfqpoint{0.859860in}{1.528453in}}{\pgfqpoint{0.859860in}{1.520217in}}%
\pgfpathcurveto{\pgfqpoint{0.859860in}{1.511980in}}{\pgfqpoint{0.863132in}{1.504080in}}{\pgfqpoint{0.868956in}{1.498256in}}%
\pgfpathcurveto{\pgfqpoint{0.874780in}{1.492433in}}{\pgfqpoint{0.882680in}{1.489160in}}{\pgfqpoint{0.890916in}{1.489160in}}%
\pgfpathclose%
\pgfusepath{stroke,fill}%
\end{pgfscope}%
\begin{pgfscope}%
\pgfpathrectangle{\pgfqpoint{0.100000in}{0.212622in}}{\pgfqpoint{3.696000in}{3.696000in}}%
\pgfusepath{clip}%
\pgfsetbuttcap%
\pgfsetroundjoin%
\definecolor{currentfill}{rgb}{0.121569,0.466667,0.705882}%
\pgfsetfillcolor{currentfill}%
\pgfsetfillopacity{0.585499}%
\pgfsetlinewidth{1.003750pt}%
\definecolor{currentstroke}{rgb}{0.121569,0.466667,0.705882}%
\pgfsetstrokecolor{currentstroke}%
\pgfsetstrokeopacity{0.585499}%
\pgfsetdash{}{0pt}%
\pgfpathmoveto{\pgfqpoint{0.890916in}{1.489160in}}%
\pgfpathcurveto{\pgfqpoint{0.899152in}{1.489160in}}{\pgfqpoint{0.907052in}{1.492433in}}{\pgfqpoint{0.912876in}{1.498256in}}%
\pgfpathcurveto{\pgfqpoint{0.918700in}{1.504080in}}{\pgfqpoint{0.921973in}{1.511980in}}{\pgfqpoint{0.921973in}{1.520217in}}%
\pgfpathcurveto{\pgfqpoint{0.921973in}{1.528453in}}{\pgfqpoint{0.918700in}{1.536353in}}{\pgfqpoint{0.912876in}{1.542177in}}%
\pgfpathcurveto{\pgfqpoint{0.907052in}{1.548001in}}{\pgfqpoint{0.899152in}{1.551273in}}{\pgfqpoint{0.890916in}{1.551273in}}%
\pgfpathcurveto{\pgfqpoint{0.882680in}{1.551273in}}{\pgfqpoint{0.874780in}{1.548001in}}{\pgfqpoint{0.868956in}{1.542177in}}%
\pgfpathcurveto{\pgfqpoint{0.863132in}{1.536353in}}{\pgfqpoint{0.859860in}{1.528453in}}{\pgfqpoint{0.859860in}{1.520217in}}%
\pgfpathcurveto{\pgfqpoint{0.859860in}{1.511980in}}{\pgfqpoint{0.863132in}{1.504080in}}{\pgfqpoint{0.868956in}{1.498256in}}%
\pgfpathcurveto{\pgfqpoint{0.874780in}{1.492433in}}{\pgfqpoint{0.882680in}{1.489160in}}{\pgfqpoint{0.890916in}{1.489160in}}%
\pgfpathclose%
\pgfusepath{stroke,fill}%
\end{pgfscope}%
\begin{pgfscope}%
\pgfpathrectangle{\pgfqpoint{0.100000in}{0.212622in}}{\pgfqpoint{3.696000in}{3.696000in}}%
\pgfusepath{clip}%
\pgfsetbuttcap%
\pgfsetroundjoin%
\definecolor{currentfill}{rgb}{0.121569,0.466667,0.705882}%
\pgfsetfillcolor{currentfill}%
\pgfsetfillopacity{0.585499}%
\pgfsetlinewidth{1.003750pt}%
\definecolor{currentstroke}{rgb}{0.121569,0.466667,0.705882}%
\pgfsetstrokecolor{currentstroke}%
\pgfsetstrokeopacity{0.585499}%
\pgfsetdash{}{0pt}%
\pgfpathmoveto{\pgfqpoint{0.890916in}{1.489160in}}%
\pgfpathcurveto{\pgfqpoint{0.899152in}{1.489160in}}{\pgfqpoint{0.907052in}{1.492433in}}{\pgfqpoint{0.912876in}{1.498256in}}%
\pgfpathcurveto{\pgfqpoint{0.918700in}{1.504080in}}{\pgfqpoint{0.921973in}{1.511980in}}{\pgfqpoint{0.921973in}{1.520217in}}%
\pgfpathcurveto{\pgfqpoint{0.921973in}{1.528453in}}{\pgfqpoint{0.918700in}{1.536353in}}{\pgfqpoint{0.912876in}{1.542177in}}%
\pgfpathcurveto{\pgfqpoint{0.907052in}{1.548001in}}{\pgfqpoint{0.899152in}{1.551273in}}{\pgfqpoint{0.890916in}{1.551273in}}%
\pgfpathcurveto{\pgfqpoint{0.882680in}{1.551273in}}{\pgfqpoint{0.874780in}{1.548001in}}{\pgfqpoint{0.868956in}{1.542177in}}%
\pgfpathcurveto{\pgfqpoint{0.863132in}{1.536353in}}{\pgfqpoint{0.859860in}{1.528453in}}{\pgfqpoint{0.859860in}{1.520217in}}%
\pgfpathcurveto{\pgfqpoint{0.859860in}{1.511980in}}{\pgfqpoint{0.863132in}{1.504080in}}{\pgfqpoint{0.868956in}{1.498256in}}%
\pgfpathcurveto{\pgfqpoint{0.874780in}{1.492433in}}{\pgfqpoint{0.882680in}{1.489160in}}{\pgfqpoint{0.890916in}{1.489160in}}%
\pgfpathclose%
\pgfusepath{stroke,fill}%
\end{pgfscope}%
\begin{pgfscope}%
\pgfpathrectangle{\pgfqpoint{0.100000in}{0.212622in}}{\pgfqpoint{3.696000in}{3.696000in}}%
\pgfusepath{clip}%
\pgfsetbuttcap%
\pgfsetroundjoin%
\definecolor{currentfill}{rgb}{0.121569,0.466667,0.705882}%
\pgfsetfillcolor{currentfill}%
\pgfsetfillopacity{0.585499}%
\pgfsetlinewidth{1.003750pt}%
\definecolor{currentstroke}{rgb}{0.121569,0.466667,0.705882}%
\pgfsetstrokecolor{currentstroke}%
\pgfsetstrokeopacity{0.585499}%
\pgfsetdash{}{0pt}%
\pgfpathmoveto{\pgfqpoint{0.890916in}{1.489160in}}%
\pgfpathcurveto{\pgfqpoint{0.899152in}{1.489160in}}{\pgfqpoint{0.907052in}{1.492433in}}{\pgfqpoint{0.912876in}{1.498256in}}%
\pgfpathcurveto{\pgfqpoint{0.918700in}{1.504080in}}{\pgfqpoint{0.921973in}{1.511980in}}{\pgfqpoint{0.921973in}{1.520217in}}%
\pgfpathcurveto{\pgfqpoint{0.921973in}{1.528453in}}{\pgfqpoint{0.918700in}{1.536353in}}{\pgfqpoint{0.912876in}{1.542177in}}%
\pgfpathcurveto{\pgfqpoint{0.907052in}{1.548001in}}{\pgfqpoint{0.899152in}{1.551273in}}{\pgfqpoint{0.890916in}{1.551273in}}%
\pgfpathcurveto{\pgfqpoint{0.882680in}{1.551273in}}{\pgfqpoint{0.874780in}{1.548001in}}{\pgfqpoint{0.868956in}{1.542177in}}%
\pgfpathcurveto{\pgfqpoint{0.863132in}{1.536353in}}{\pgfqpoint{0.859860in}{1.528453in}}{\pgfqpoint{0.859860in}{1.520217in}}%
\pgfpathcurveto{\pgfqpoint{0.859860in}{1.511980in}}{\pgfqpoint{0.863132in}{1.504080in}}{\pgfqpoint{0.868956in}{1.498256in}}%
\pgfpathcurveto{\pgfqpoint{0.874780in}{1.492433in}}{\pgfqpoint{0.882680in}{1.489160in}}{\pgfqpoint{0.890916in}{1.489160in}}%
\pgfpathclose%
\pgfusepath{stroke,fill}%
\end{pgfscope}%
\begin{pgfscope}%
\pgfpathrectangle{\pgfqpoint{0.100000in}{0.212622in}}{\pgfqpoint{3.696000in}{3.696000in}}%
\pgfusepath{clip}%
\pgfsetbuttcap%
\pgfsetroundjoin%
\definecolor{currentfill}{rgb}{0.121569,0.466667,0.705882}%
\pgfsetfillcolor{currentfill}%
\pgfsetfillopacity{0.585499}%
\pgfsetlinewidth{1.003750pt}%
\definecolor{currentstroke}{rgb}{0.121569,0.466667,0.705882}%
\pgfsetstrokecolor{currentstroke}%
\pgfsetstrokeopacity{0.585499}%
\pgfsetdash{}{0pt}%
\pgfpathmoveto{\pgfqpoint{0.890916in}{1.489160in}}%
\pgfpathcurveto{\pgfqpoint{0.899152in}{1.489160in}}{\pgfqpoint{0.907052in}{1.492433in}}{\pgfqpoint{0.912876in}{1.498256in}}%
\pgfpathcurveto{\pgfqpoint{0.918700in}{1.504080in}}{\pgfqpoint{0.921973in}{1.511980in}}{\pgfqpoint{0.921973in}{1.520217in}}%
\pgfpathcurveto{\pgfqpoint{0.921973in}{1.528453in}}{\pgfqpoint{0.918700in}{1.536353in}}{\pgfqpoint{0.912876in}{1.542177in}}%
\pgfpathcurveto{\pgfqpoint{0.907052in}{1.548001in}}{\pgfqpoint{0.899152in}{1.551273in}}{\pgfqpoint{0.890916in}{1.551273in}}%
\pgfpathcurveto{\pgfqpoint{0.882680in}{1.551273in}}{\pgfqpoint{0.874780in}{1.548001in}}{\pgfqpoint{0.868956in}{1.542177in}}%
\pgfpathcurveto{\pgfqpoint{0.863132in}{1.536353in}}{\pgfqpoint{0.859860in}{1.528453in}}{\pgfqpoint{0.859860in}{1.520217in}}%
\pgfpathcurveto{\pgfqpoint{0.859860in}{1.511980in}}{\pgfqpoint{0.863132in}{1.504080in}}{\pgfqpoint{0.868956in}{1.498256in}}%
\pgfpathcurveto{\pgfqpoint{0.874780in}{1.492433in}}{\pgfqpoint{0.882680in}{1.489160in}}{\pgfqpoint{0.890916in}{1.489160in}}%
\pgfpathclose%
\pgfusepath{stroke,fill}%
\end{pgfscope}%
\begin{pgfscope}%
\pgfpathrectangle{\pgfqpoint{0.100000in}{0.212622in}}{\pgfqpoint{3.696000in}{3.696000in}}%
\pgfusepath{clip}%
\pgfsetbuttcap%
\pgfsetroundjoin%
\definecolor{currentfill}{rgb}{0.121569,0.466667,0.705882}%
\pgfsetfillcolor{currentfill}%
\pgfsetfillopacity{0.585499}%
\pgfsetlinewidth{1.003750pt}%
\definecolor{currentstroke}{rgb}{0.121569,0.466667,0.705882}%
\pgfsetstrokecolor{currentstroke}%
\pgfsetstrokeopacity{0.585499}%
\pgfsetdash{}{0pt}%
\pgfpathmoveto{\pgfqpoint{0.890916in}{1.489160in}}%
\pgfpathcurveto{\pgfqpoint{0.899152in}{1.489160in}}{\pgfqpoint{0.907052in}{1.492433in}}{\pgfqpoint{0.912876in}{1.498256in}}%
\pgfpathcurveto{\pgfqpoint{0.918700in}{1.504080in}}{\pgfqpoint{0.921973in}{1.511980in}}{\pgfqpoint{0.921973in}{1.520217in}}%
\pgfpathcurveto{\pgfqpoint{0.921973in}{1.528453in}}{\pgfqpoint{0.918700in}{1.536353in}}{\pgfqpoint{0.912876in}{1.542177in}}%
\pgfpathcurveto{\pgfqpoint{0.907052in}{1.548001in}}{\pgfqpoint{0.899152in}{1.551273in}}{\pgfqpoint{0.890916in}{1.551273in}}%
\pgfpathcurveto{\pgfqpoint{0.882680in}{1.551273in}}{\pgfqpoint{0.874780in}{1.548001in}}{\pgfqpoint{0.868956in}{1.542177in}}%
\pgfpathcurveto{\pgfqpoint{0.863132in}{1.536353in}}{\pgfqpoint{0.859860in}{1.528453in}}{\pgfqpoint{0.859860in}{1.520217in}}%
\pgfpathcurveto{\pgfqpoint{0.859860in}{1.511980in}}{\pgfqpoint{0.863132in}{1.504080in}}{\pgfqpoint{0.868956in}{1.498256in}}%
\pgfpathcurveto{\pgfqpoint{0.874780in}{1.492433in}}{\pgfqpoint{0.882680in}{1.489160in}}{\pgfqpoint{0.890916in}{1.489160in}}%
\pgfpathclose%
\pgfusepath{stroke,fill}%
\end{pgfscope}%
\begin{pgfscope}%
\pgfpathrectangle{\pgfqpoint{0.100000in}{0.212622in}}{\pgfqpoint{3.696000in}{3.696000in}}%
\pgfusepath{clip}%
\pgfsetbuttcap%
\pgfsetroundjoin%
\definecolor{currentfill}{rgb}{0.121569,0.466667,0.705882}%
\pgfsetfillcolor{currentfill}%
\pgfsetfillopacity{0.585499}%
\pgfsetlinewidth{1.003750pt}%
\definecolor{currentstroke}{rgb}{0.121569,0.466667,0.705882}%
\pgfsetstrokecolor{currentstroke}%
\pgfsetstrokeopacity{0.585499}%
\pgfsetdash{}{0pt}%
\pgfpathmoveto{\pgfqpoint{0.890916in}{1.489160in}}%
\pgfpathcurveto{\pgfqpoint{0.899152in}{1.489160in}}{\pgfqpoint{0.907052in}{1.492433in}}{\pgfqpoint{0.912876in}{1.498256in}}%
\pgfpathcurveto{\pgfqpoint{0.918700in}{1.504080in}}{\pgfqpoint{0.921973in}{1.511980in}}{\pgfqpoint{0.921973in}{1.520217in}}%
\pgfpathcurveto{\pgfqpoint{0.921973in}{1.528453in}}{\pgfqpoint{0.918700in}{1.536353in}}{\pgfqpoint{0.912876in}{1.542177in}}%
\pgfpathcurveto{\pgfqpoint{0.907052in}{1.548001in}}{\pgfqpoint{0.899152in}{1.551273in}}{\pgfqpoint{0.890916in}{1.551273in}}%
\pgfpathcurveto{\pgfqpoint{0.882680in}{1.551273in}}{\pgfqpoint{0.874780in}{1.548001in}}{\pgfqpoint{0.868956in}{1.542177in}}%
\pgfpathcurveto{\pgfqpoint{0.863132in}{1.536353in}}{\pgfqpoint{0.859860in}{1.528453in}}{\pgfqpoint{0.859860in}{1.520217in}}%
\pgfpathcurveto{\pgfqpoint{0.859860in}{1.511980in}}{\pgfqpoint{0.863132in}{1.504080in}}{\pgfqpoint{0.868956in}{1.498256in}}%
\pgfpathcurveto{\pgfqpoint{0.874780in}{1.492433in}}{\pgfqpoint{0.882680in}{1.489160in}}{\pgfqpoint{0.890916in}{1.489160in}}%
\pgfpathclose%
\pgfusepath{stroke,fill}%
\end{pgfscope}%
\begin{pgfscope}%
\pgfpathrectangle{\pgfqpoint{0.100000in}{0.212622in}}{\pgfqpoint{3.696000in}{3.696000in}}%
\pgfusepath{clip}%
\pgfsetbuttcap%
\pgfsetroundjoin%
\definecolor{currentfill}{rgb}{0.121569,0.466667,0.705882}%
\pgfsetfillcolor{currentfill}%
\pgfsetfillopacity{0.585499}%
\pgfsetlinewidth{1.003750pt}%
\definecolor{currentstroke}{rgb}{0.121569,0.466667,0.705882}%
\pgfsetstrokecolor{currentstroke}%
\pgfsetstrokeopacity{0.585499}%
\pgfsetdash{}{0pt}%
\pgfpathmoveto{\pgfqpoint{0.890916in}{1.489160in}}%
\pgfpathcurveto{\pgfqpoint{0.899152in}{1.489160in}}{\pgfqpoint{0.907052in}{1.492433in}}{\pgfqpoint{0.912876in}{1.498256in}}%
\pgfpathcurveto{\pgfqpoint{0.918700in}{1.504080in}}{\pgfqpoint{0.921973in}{1.511980in}}{\pgfqpoint{0.921973in}{1.520217in}}%
\pgfpathcurveto{\pgfqpoint{0.921973in}{1.528453in}}{\pgfqpoint{0.918700in}{1.536353in}}{\pgfqpoint{0.912876in}{1.542177in}}%
\pgfpathcurveto{\pgfqpoint{0.907052in}{1.548001in}}{\pgfqpoint{0.899152in}{1.551273in}}{\pgfqpoint{0.890916in}{1.551273in}}%
\pgfpathcurveto{\pgfqpoint{0.882680in}{1.551273in}}{\pgfqpoint{0.874780in}{1.548001in}}{\pgfqpoint{0.868956in}{1.542177in}}%
\pgfpathcurveto{\pgfqpoint{0.863132in}{1.536353in}}{\pgfqpoint{0.859860in}{1.528453in}}{\pgfqpoint{0.859860in}{1.520217in}}%
\pgfpathcurveto{\pgfqpoint{0.859860in}{1.511980in}}{\pgfqpoint{0.863132in}{1.504080in}}{\pgfqpoint{0.868956in}{1.498256in}}%
\pgfpathcurveto{\pgfqpoint{0.874780in}{1.492433in}}{\pgfqpoint{0.882680in}{1.489160in}}{\pgfqpoint{0.890916in}{1.489160in}}%
\pgfpathclose%
\pgfusepath{stroke,fill}%
\end{pgfscope}%
\begin{pgfscope}%
\pgfpathrectangle{\pgfqpoint{0.100000in}{0.212622in}}{\pgfqpoint{3.696000in}{3.696000in}}%
\pgfusepath{clip}%
\pgfsetbuttcap%
\pgfsetroundjoin%
\definecolor{currentfill}{rgb}{0.121569,0.466667,0.705882}%
\pgfsetfillcolor{currentfill}%
\pgfsetfillopacity{0.585499}%
\pgfsetlinewidth{1.003750pt}%
\definecolor{currentstroke}{rgb}{0.121569,0.466667,0.705882}%
\pgfsetstrokecolor{currentstroke}%
\pgfsetstrokeopacity{0.585499}%
\pgfsetdash{}{0pt}%
\pgfpathmoveto{\pgfqpoint{0.890916in}{1.489160in}}%
\pgfpathcurveto{\pgfqpoint{0.899152in}{1.489160in}}{\pgfqpoint{0.907052in}{1.492433in}}{\pgfqpoint{0.912876in}{1.498256in}}%
\pgfpathcurveto{\pgfqpoint{0.918700in}{1.504080in}}{\pgfqpoint{0.921973in}{1.511980in}}{\pgfqpoint{0.921973in}{1.520217in}}%
\pgfpathcurveto{\pgfqpoint{0.921973in}{1.528453in}}{\pgfqpoint{0.918700in}{1.536353in}}{\pgfqpoint{0.912876in}{1.542177in}}%
\pgfpathcurveto{\pgfqpoint{0.907052in}{1.548001in}}{\pgfqpoint{0.899152in}{1.551273in}}{\pgfqpoint{0.890916in}{1.551273in}}%
\pgfpathcurveto{\pgfqpoint{0.882680in}{1.551273in}}{\pgfqpoint{0.874780in}{1.548001in}}{\pgfqpoint{0.868956in}{1.542177in}}%
\pgfpathcurveto{\pgfqpoint{0.863132in}{1.536353in}}{\pgfqpoint{0.859860in}{1.528453in}}{\pgfqpoint{0.859860in}{1.520217in}}%
\pgfpathcurveto{\pgfqpoint{0.859860in}{1.511980in}}{\pgfqpoint{0.863132in}{1.504080in}}{\pgfqpoint{0.868956in}{1.498256in}}%
\pgfpathcurveto{\pgfqpoint{0.874780in}{1.492433in}}{\pgfqpoint{0.882680in}{1.489160in}}{\pgfqpoint{0.890916in}{1.489160in}}%
\pgfpathclose%
\pgfusepath{stroke,fill}%
\end{pgfscope}%
\begin{pgfscope}%
\pgfpathrectangle{\pgfqpoint{0.100000in}{0.212622in}}{\pgfqpoint{3.696000in}{3.696000in}}%
\pgfusepath{clip}%
\pgfsetbuttcap%
\pgfsetroundjoin%
\definecolor{currentfill}{rgb}{0.121569,0.466667,0.705882}%
\pgfsetfillcolor{currentfill}%
\pgfsetfillopacity{0.585499}%
\pgfsetlinewidth{1.003750pt}%
\definecolor{currentstroke}{rgb}{0.121569,0.466667,0.705882}%
\pgfsetstrokecolor{currentstroke}%
\pgfsetstrokeopacity{0.585499}%
\pgfsetdash{}{0pt}%
\pgfpathmoveto{\pgfqpoint{0.890916in}{1.489160in}}%
\pgfpathcurveto{\pgfqpoint{0.899152in}{1.489160in}}{\pgfqpoint{0.907052in}{1.492433in}}{\pgfqpoint{0.912876in}{1.498256in}}%
\pgfpathcurveto{\pgfqpoint{0.918700in}{1.504080in}}{\pgfqpoint{0.921973in}{1.511980in}}{\pgfqpoint{0.921973in}{1.520217in}}%
\pgfpathcurveto{\pgfqpoint{0.921973in}{1.528453in}}{\pgfqpoint{0.918700in}{1.536353in}}{\pgfqpoint{0.912876in}{1.542177in}}%
\pgfpathcurveto{\pgfqpoint{0.907052in}{1.548001in}}{\pgfqpoint{0.899152in}{1.551273in}}{\pgfqpoint{0.890916in}{1.551273in}}%
\pgfpathcurveto{\pgfqpoint{0.882680in}{1.551273in}}{\pgfqpoint{0.874780in}{1.548001in}}{\pgfqpoint{0.868956in}{1.542177in}}%
\pgfpathcurveto{\pgfqpoint{0.863132in}{1.536353in}}{\pgfqpoint{0.859860in}{1.528453in}}{\pgfqpoint{0.859860in}{1.520217in}}%
\pgfpathcurveto{\pgfqpoint{0.859860in}{1.511980in}}{\pgfqpoint{0.863132in}{1.504080in}}{\pgfqpoint{0.868956in}{1.498256in}}%
\pgfpathcurveto{\pgfqpoint{0.874780in}{1.492433in}}{\pgfqpoint{0.882680in}{1.489160in}}{\pgfqpoint{0.890916in}{1.489160in}}%
\pgfpathclose%
\pgfusepath{stroke,fill}%
\end{pgfscope}%
\begin{pgfscope}%
\pgfpathrectangle{\pgfqpoint{0.100000in}{0.212622in}}{\pgfqpoint{3.696000in}{3.696000in}}%
\pgfusepath{clip}%
\pgfsetbuttcap%
\pgfsetroundjoin%
\definecolor{currentfill}{rgb}{0.121569,0.466667,0.705882}%
\pgfsetfillcolor{currentfill}%
\pgfsetfillopacity{0.585499}%
\pgfsetlinewidth{1.003750pt}%
\definecolor{currentstroke}{rgb}{0.121569,0.466667,0.705882}%
\pgfsetstrokecolor{currentstroke}%
\pgfsetstrokeopacity{0.585499}%
\pgfsetdash{}{0pt}%
\pgfpathmoveto{\pgfqpoint{0.890916in}{1.489160in}}%
\pgfpathcurveto{\pgfqpoint{0.899152in}{1.489160in}}{\pgfqpoint{0.907052in}{1.492433in}}{\pgfqpoint{0.912876in}{1.498256in}}%
\pgfpathcurveto{\pgfqpoint{0.918700in}{1.504080in}}{\pgfqpoint{0.921973in}{1.511980in}}{\pgfqpoint{0.921973in}{1.520217in}}%
\pgfpathcurveto{\pgfqpoint{0.921973in}{1.528453in}}{\pgfqpoint{0.918700in}{1.536353in}}{\pgfqpoint{0.912876in}{1.542177in}}%
\pgfpathcurveto{\pgfqpoint{0.907052in}{1.548001in}}{\pgfqpoint{0.899152in}{1.551273in}}{\pgfqpoint{0.890916in}{1.551273in}}%
\pgfpathcurveto{\pgfqpoint{0.882680in}{1.551273in}}{\pgfqpoint{0.874780in}{1.548001in}}{\pgfqpoint{0.868956in}{1.542177in}}%
\pgfpathcurveto{\pgfqpoint{0.863132in}{1.536353in}}{\pgfqpoint{0.859860in}{1.528453in}}{\pgfqpoint{0.859860in}{1.520217in}}%
\pgfpathcurveto{\pgfqpoint{0.859860in}{1.511980in}}{\pgfqpoint{0.863132in}{1.504080in}}{\pgfqpoint{0.868956in}{1.498256in}}%
\pgfpathcurveto{\pgfqpoint{0.874780in}{1.492433in}}{\pgfqpoint{0.882680in}{1.489160in}}{\pgfqpoint{0.890916in}{1.489160in}}%
\pgfpathclose%
\pgfusepath{stroke,fill}%
\end{pgfscope}%
\begin{pgfscope}%
\pgfpathrectangle{\pgfqpoint{0.100000in}{0.212622in}}{\pgfqpoint{3.696000in}{3.696000in}}%
\pgfusepath{clip}%
\pgfsetbuttcap%
\pgfsetroundjoin%
\definecolor{currentfill}{rgb}{0.121569,0.466667,0.705882}%
\pgfsetfillcolor{currentfill}%
\pgfsetfillopacity{0.585499}%
\pgfsetlinewidth{1.003750pt}%
\definecolor{currentstroke}{rgb}{0.121569,0.466667,0.705882}%
\pgfsetstrokecolor{currentstroke}%
\pgfsetstrokeopacity{0.585499}%
\pgfsetdash{}{0pt}%
\pgfpathmoveto{\pgfqpoint{0.890916in}{1.489160in}}%
\pgfpathcurveto{\pgfqpoint{0.899152in}{1.489160in}}{\pgfqpoint{0.907052in}{1.492433in}}{\pgfqpoint{0.912876in}{1.498256in}}%
\pgfpathcurveto{\pgfqpoint{0.918700in}{1.504080in}}{\pgfqpoint{0.921973in}{1.511980in}}{\pgfqpoint{0.921973in}{1.520217in}}%
\pgfpathcurveto{\pgfqpoint{0.921973in}{1.528453in}}{\pgfqpoint{0.918700in}{1.536353in}}{\pgfqpoint{0.912876in}{1.542177in}}%
\pgfpathcurveto{\pgfqpoint{0.907052in}{1.548001in}}{\pgfqpoint{0.899152in}{1.551273in}}{\pgfqpoint{0.890916in}{1.551273in}}%
\pgfpathcurveto{\pgfqpoint{0.882680in}{1.551273in}}{\pgfqpoint{0.874780in}{1.548001in}}{\pgfqpoint{0.868956in}{1.542177in}}%
\pgfpathcurveto{\pgfqpoint{0.863132in}{1.536353in}}{\pgfqpoint{0.859860in}{1.528453in}}{\pgfqpoint{0.859860in}{1.520217in}}%
\pgfpathcurveto{\pgfqpoint{0.859860in}{1.511980in}}{\pgfqpoint{0.863132in}{1.504080in}}{\pgfqpoint{0.868956in}{1.498256in}}%
\pgfpathcurveto{\pgfqpoint{0.874780in}{1.492433in}}{\pgfqpoint{0.882680in}{1.489160in}}{\pgfqpoint{0.890916in}{1.489160in}}%
\pgfpathclose%
\pgfusepath{stroke,fill}%
\end{pgfscope}%
\begin{pgfscope}%
\pgfpathrectangle{\pgfqpoint{0.100000in}{0.212622in}}{\pgfqpoint{3.696000in}{3.696000in}}%
\pgfusepath{clip}%
\pgfsetbuttcap%
\pgfsetroundjoin%
\definecolor{currentfill}{rgb}{0.121569,0.466667,0.705882}%
\pgfsetfillcolor{currentfill}%
\pgfsetfillopacity{0.585499}%
\pgfsetlinewidth{1.003750pt}%
\definecolor{currentstroke}{rgb}{0.121569,0.466667,0.705882}%
\pgfsetstrokecolor{currentstroke}%
\pgfsetstrokeopacity{0.585499}%
\pgfsetdash{}{0pt}%
\pgfpathmoveto{\pgfqpoint{0.890916in}{1.489160in}}%
\pgfpathcurveto{\pgfqpoint{0.899152in}{1.489160in}}{\pgfqpoint{0.907052in}{1.492433in}}{\pgfqpoint{0.912876in}{1.498256in}}%
\pgfpathcurveto{\pgfqpoint{0.918700in}{1.504080in}}{\pgfqpoint{0.921973in}{1.511980in}}{\pgfqpoint{0.921973in}{1.520217in}}%
\pgfpathcurveto{\pgfqpoint{0.921973in}{1.528453in}}{\pgfqpoint{0.918700in}{1.536353in}}{\pgfqpoint{0.912876in}{1.542177in}}%
\pgfpathcurveto{\pgfqpoint{0.907052in}{1.548001in}}{\pgfqpoint{0.899152in}{1.551273in}}{\pgfqpoint{0.890916in}{1.551273in}}%
\pgfpathcurveto{\pgfqpoint{0.882680in}{1.551273in}}{\pgfqpoint{0.874780in}{1.548001in}}{\pgfqpoint{0.868956in}{1.542177in}}%
\pgfpathcurveto{\pgfqpoint{0.863132in}{1.536353in}}{\pgfqpoint{0.859860in}{1.528453in}}{\pgfqpoint{0.859860in}{1.520217in}}%
\pgfpathcurveto{\pgfqpoint{0.859860in}{1.511980in}}{\pgfqpoint{0.863132in}{1.504080in}}{\pgfqpoint{0.868956in}{1.498256in}}%
\pgfpathcurveto{\pgfqpoint{0.874780in}{1.492433in}}{\pgfqpoint{0.882680in}{1.489160in}}{\pgfqpoint{0.890916in}{1.489160in}}%
\pgfpathclose%
\pgfusepath{stroke,fill}%
\end{pgfscope}%
\begin{pgfscope}%
\pgfpathrectangle{\pgfqpoint{0.100000in}{0.212622in}}{\pgfqpoint{3.696000in}{3.696000in}}%
\pgfusepath{clip}%
\pgfsetbuttcap%
\pgfsetroundjoin%
\definecolor{currentfill}{rgb}{0.121569,0.466667,0.705882}%
\pgfsetfillcolor{currentfill}%
\pgfsetfillopacity{0.585499}%
\pgfsetlinewidth{1.003750pt}%
\definecolor{currentstroke}{rgb}{0.121569,0.466667,0.705882}%
\pgfsetstrokecolor{currentstroke}%
\pgfsetstrokeopacity{0.585499}%
\pgfsetdash{}{0pt}%
\pgfpathmoveto{\pgfqpoint{0.890916in}{1.489160in}}%
\pgfpathcurveto{\pgfqpoint{0.899152in}{1.489160in}}{\pgfqpoint{0.907052in}{1.492433in}}{\pgfqpoint{0.912876in}{1.498256in}}%
\pgfpathcurveto{\pgfqpoint{0.918700in}{1.504080in}}{\pgfqpoint{0.921973in}{1.511980in}}{\pgfqpoint{0.921973in}{1.520217in}}%
\pgfpathcurveto{\pgfqpoint{0.921973in}{1.528453in}}{\pgfqpoint{0.918700in}{1.536353in}}{\pgfqpoint{0.912876in}{1.542177in}}%
\pgfpathcurveto{\pgfqpoint{0.907052in}{1.548001in}}{\pgfqpoint{0.899152in}{1.551273in}}{\pgfqpoint{0.890916in}{1.551273in}}%
\pgfpathcurveto{\pgfqpoint{0.882680in}{1.551273in}}{\pgfqpoint{0.874780in}{1.548001in}}{\pgfqpoint{0.868956in}{1.542177in}}%
\pgfpathcurveto{\pgfqpoint{0.863132in}{1.536353in}}{\pgfqpoint{0.859860in}{1.528453in}}{\pgfqpoint{0.859860in}{1.520217in}}%
\pgfpathcurveto{\pgfqpoint{0.859860in}{1.511980in}}{\pgfqpoint{0.863132in}{1.504080in}}{\pgfqpoint{0.868956in}{1.498256in}}%
\pgfpathcurveto{\pgfqpoint{0.874780in}{1.492433in}}{\pgfqpoint{0.882680in}{1.489160in}}{\pgfqpoint{0.890916in}{1.489160in}}%
\pgfpathclose%
\pgfusepath{stroke,fill}%
\end{pgfscope}%
\begin{pgfscope}%
\pgfpathrectangle{\pgfqpoint{0.100000in}{0.212622in}}{\pgfqpoint{3.696000in}{3.696000in}}%
\pgfusepath{clip}%
\pgfsetbuttcap%
\pgfsetroundjoin%
\definecolor{currentfill}{rgb}{0.121569,0.466667,0.705882}%
\pgfsetfillcolor{currentfill}%
\pgfsetfillopacity{0.585499}%
\pgfsetlinewidth{1.003750pt}%
\definecolor{currentstroke}{rgb}{0.121569,0.466667,0.705882}%
\pgfsetstrokecolor{currentstroke}%
\pgfsetstrokeopacity{0.585499}%
\pgfsetdash{}{0pt}%
\pgfpathmoveto{\pgfqpoint{0.890916in}{1.489160in}}%
\pgfpathcurveto{\pgfqpoint{0.899152in}{1.489160in}}{\pgfqpoint{0.907052in}{1.492433in}}{\pgfqpoint{0.912876in}{1.498256in}}%
\pgfpathcurveto{\pgfqpoint{0.918700in}{1.504080in}}{\pgfqpoint{0.921973in}{1.511980in}}{\pgfqpoint{0.921973in}{1.520217in}}%
\pgfpathcurveto{\pgfqpoint{0.921973in}{1.528453in}}{\pgfqpoint{0.918700in}{1.536353in}}{\pgfqpoint{0.912876in}{1.542177in}}%
\pgfpathcurveto{\pgfqpoint{0.907052in}{1.548001in}}{\pgfqpoint{0.899152in}{1.551273in}}{\pgfqpoint{0.890916in}{1.551273in}}%
\pgfpathcurveto{\pgfqpoint{0.882680in}{1.551273in}}{\pgfqpoint{0.874780in}{1.548001in}}{\pgfqpoint{0.868956in}{1.542177in}}%
\pgfpathcurveto{\pgfqpoint{0.863132in}{1.536353in}}{\pgfqpoint{0.859860in}{1.528453in}}{\pgfqpoint{0.859860in}{1.520217in}}%
\pgfpathcurveto{\pgfqpoint{0.859860in}{1.511980in}}{\pgfqpoint{0.863132in}{1.504080in}}{\pgfqpoint{0.868956in}{1.498256in}}%
\pgfpathcurveto{\pgfqpoint{0.874780in}{1.492433in}}{\pgfqpoint{0.882680in}{1.489160in}}{\pgfqpoint{0.890916in}{1.489160in}}%
\pgfpathclose%
\pgfusepath{stroke,fill}%
\end{pgfscope}%
\begin{pgfscope}%
\pgfpathrectangle{\pgfqpoint{0.100000in}{0.212622in}}{\pgfqpoint{3.696000in}{3.696000in}}%
\pgfusepath{clip}%
\pgfsetbuttcap%
\pgfsetroundjoin%
\definecolor{currentfill}{rgb}{0.121569,0.466667,0.705882}%
\pgfsetfillcolor{currentfill}%
\pgfsetfillopacity{0.585499}%
\pgfsetlinewidth{1.003750pt}%
\definecolor{currentstroke}{rgb}{0.121569,0.466667,0.705882}%
\pgfsetstrokecolor{currentstroke}%
\pgfsetstrokeopacity{0.585499}%
\pgfsetdash{}{0pt}%
\pgfpathmoveto{\pgfqpoint{0.890916in}{1.489160in}}%
\pgfpathcurveto{\pgfqpoint{0.899152in}{1.489160in}}{\pgfqpoint{0.907052in}{1.492433in}}{\pgfqpoint{0.912876in}{1.498256in}}%
\pgfpathcurveto{\pgfqpoint{0.918700in}{1.504080in}}{\pgfqpoint{0.921973in}{1.511980in}}{\pgfqpoint{0.921973in}{1.520217in}}%
\pgfpathcurveto{\pgfqpoint{0.921973in}{1.528453in}}{\pgfqpoint{0.918700in}{1.536353in}}{\pgfqpoint{0.912876in}{1.542177in}}%
\pgfpathcurveto{\pgfqpoint{0.907052in}{1.548001in}}{\pgfqpoint{0.899152in}{1.551273in}}{\pgfqpoint{0.890916in}{1.551273in}}%
\pgfpathcurveto{\pgfqpoint{0.882680in}{1.551273in}}{\pgfqpoint{0.874780in}{1.548001in}}{\pgfqpoint{0.868956in}{1.542177in}}%
\pgfpathcurveto{\pgfqpoint{0.863132in}{1.536353in}}{\pgfqpoint{0.859860in}{1.528453in}}{\pgfqpoint{0.859860in}{1.520217in}}%
\pgfpathcurveto{\pgfqpoint{0.859860in}{1.511980in}}{\pgfqpoint{0.863132in}{1.504080in}}{\pgfqpoint{0.868956in}{1.498256in}}%
\pgfpathcurveto{\pgfqpoint{0.874780in}{1.492433in}}{\pgfqpoint{0.882680in}{1.489160in}}{\pgfqpoint{0.890916in}{1.489160in}}%
\pgfpathclose%
\pgfusepath{stroke,fill}%
\end{pgfscope}%
\begin{pgfscope}%
\pgfpathrectangle{\pgfqpoint{0.100000in}{0.212622in}}{\pgfqpoint{3.696000in}{3.696000in}}%
\pgfusepath{clip}%
\pgfsetbuttcap%
\pgfsetroundjoin%
\definecolor{currentfill}{rgb}{0.121569,0.466667,0.705882}%
\pgfsetfillcolor{currentfill}%
\pgfsetfillopacity{0.585499}%
\pgfsetlinewidth{1.003750pt}%
\definecolor{currentstroke}{rgb}{0.121569,0.466667,0.705882}%
\pgfsetstrokecolor{currentstroke}%
\pgfsetstrokeopacity{0.585499}%
\pgfsetdash{}{0pt}%
\pgfpathmoveto{\pgfqpoint{0.890916in}{1.489160in}}%
\pgfpathcurveto{\pgfqpoint{0.899152in}{1.489160in}}{\pgfqpoint{0.907052in}{1.492433in}}{\pgfqpoint{0.912876in}{1.498256in}}%
\pgfpathcurveto{\pgfqpoint{0.918700in}{1.504080in}}{\pgfqpoint{0.921973in}{1.511980in}}{\pgfqpoint{0.921973in}{1.520217in}}%
\pgfpathcurveto{\pgfqpoint{0.921973in}{1.528453in}}{\pgfqpoint{0.918700in}{1.536353in}}{\pgfqpoint{0.912876in}{1.542177in}}%
\pgfpathcurveto{\pgfqpoint{0.907052in}{1.548001in}}{\pgfqpoint{0.899152in}{1.551273in}}{\pgfqpoint{0.890916in}{1.551273in}}%
\pgfpathcurveto{\pgfqpoint{0.882680in}{1.551273in}}{\pgfqpoint{0.874780in}{1.548001in}}{\pgfqpoint{0.868956in}{1.542177in}}%
\pgfpathcurveto{\pgfqpoint{0.863132in}{1.536353in}}{\pgfqpoint{0.859860in}{1.528453in}}{\pgfqpoint{0.859860in}{1.520217in}}%
\pgfpathcurveto{\pgfqpoint{0.859860in}{1.511980in}}{\pgfqpoint{0.863132in}{1.504080in}}{\pgfqpoint{0.868956in}{1.498256in}}%
\pgfpathcurveto{\pgfqpoint{0.874780in}{1.492433in}}{\pgfqpoint{0.882680in}{1.489160in}}{\pgfqpoint{0.890916in}{1.489160in}}%
\pgfpathclose%
\pgfusepath{stroke,fill}%
\end{pgfscope}%
\begin{pgfscope}%
\pgfpathrectangle{\pgfqpoint{0.100000in}{0.212622in}}{\pgfqpoint{3.696000in}{3.696000in}}%
\pgfusepath{clip}%
\pgfsetbuttcap%
\pgfsetroundjoin%
\definecolor{currentfill}{rgb}{0.121569,0.466667,0.705882}%
\pgfsetfillcolor{currentfill}%
\pgfsetfillopacity{0.585499}%
\pgfsetlinewidth{1.003750pt}%
\definecolor{currentstroke}{rgb}{0.121569,0.466667,0.705882}%
\pgfsetstrokecolor{currentstroke}%
\pgfsetstrokeopacity{0.585499}%
\pgfsetdash{}{0pt}%
\pgfpathmoveto{\pgfqpoint{0.890916in}{1.489160in}}%
\pgfpathcurveto{\pgfqpoint{0.899152in}{1.489160in}}{\pgfqpoint{0.907052in}{1.492433in}}{\pgfqpoint{0.912876in}{1.498256in}}%
\pgfpathcurveto{\pgfqpoint{0.918700in}{1.504080in}}{\pgfqpoint{0.921973in}{1.511980in}}{\pgfqpoint{0.921973in}{1.520217in}}%
\pgfpathcurveto{\pgfqpoint{0.921973in}{1.528453in}}{\pgfqpoint{0.918700in}{1.536353in}}{\pgfqpoint{0.912876in}{1.542177in}}%
\pgfpathcurveto{\pgfqpoint{0.907052in}{1.548001in}}{\pgfqpoint{0.899152in}{1.551273in}}{\pgfqpoint{0.890916in}{1.551273in}}%
\pgfpathcurveto{\pgfqpoint{0.882680in}{1.551273in}}{\pgfqpoint{0.874780in}{1.548001in}}{\pgfqpoint{0.868956in}{1.542177in}}%
\pgfpathcurveto{\pgfqpoint{0.863132in}{1.536353in}}{\pgfqpoint{0.859860in}{1.528453in}}{\pgfqpoint{0.859860in}{1.520217in}}%
\pgfpathcurveto{\pgfqpoint{0.859860in}{1.511980in}}{\pgfqpoint{0.863132in}{1.504080in}}{\pgfqpoint{0.868956in}{1.498256in}}%
\pgfpathcurveto{\pgfqpoint{0.874780in}{1.492433in}}{\pgfqpoint{0.882680in}{1.489160in}}{\pgfqpoint{0.890916in}{1.489160in}}%
\pgfpathclose%
\pgfusepath{stroke,fill}%
\end{pgfscope}%
\begin{pgfscope}%
\pgfpathrectangle{\pgfqpoint{0.100000in}{0.212622in}}{\pgfqpoint{3.696000in}{3.696000in}}%
\pgfusepath{clip}%
\pgfsetbuttcap%
\pgfsetroundjoin%
\definecolor{currentfill}{rgb}{0.121569,0.466667,0.705882}%
\pgfsetfillcolor{currentfill}%
\pgfsetfillopacity{0.585499}%
\pgfsetlinewidth{1.003750pt}%
\definecolor{currentstroke}{rgb}{0.121569,0.466667,0.705882}%
\pgfsetstrokecolor{currentstroke}%
\pgfsetstrokeopacity{0.585499}%
\pgfsetdash{}{0pt}%
\pgfpathmoveto{\pgfqpoint{0.890916in}{1.489160in}}%
\pgfpathcurveto{\pgfqpoint{0.899152in}{1.489160in}}{\pgfqpoint{0.907052in}{1.492433in}}{\pgfqpoint{0.912876in}{1.498256in}}%
\pgfpathcurveto{\pgfqpoint{0.918700in}{1.504080in}}{\pgfqpoint{0.921973in}{1.511980in}}{\pgfqpoint{0.921973in}{1.520217in}}%
\pgfpathcurveto{\pgfqpoint{0.921973in}{1.528453in}}{\pgfqpoint{0.918700in}{1.536353in}}{\pgfqpoint{0.912876in}{1.542177in}}%
\pgfpathcurveto{\pgfqpoint{0.907052in}{1.548001in}}{\pgfqpoint{0.899152in}{1.551273in}}{\pgfqpoint{0.890916in}{1.551273in}}%
\pgfpathcurveto{\pgfqpoint{0.882680in}{1.551273in}}{\pgfqpoint{0.874780in}{1.548001in}}{\pgfqpoint{0.868956in}{1.542177in}}%
\pgfpathcurveto{\pgfqpoint{0.863132in}{1.536353in}}{\pgfqpoint{0.859860in}{1.528453in}}{\pgfqpoint{0.859860in}{1.520217in}}%
\pgfpathcurveto{\pgfqpoint{0.859860in}{1.511980in}}{\pgfqpoint{0.863132in}{1.504080in}}{\pgfqpoint{0.868956in}{1.498256in}}%
\pgfpathcurveto{\pgfqpoint{0.874780in}{1.492433in}}{\pgfqpoint{0.882680in}{1.489160in}}{\pgfqpoint{0.890916in}{1.489160in}}%
\pgfpathclose%
\pgfusepath{stroke,fill}%
\end{pgfscope}%
\begin{pgfscope}%
\pgfpathrectangle{\pgfqpoint{0.100000in}{0.212622in}}{\pgfqpoint{3.696000in}{3.696000in}}%
\pgfusepath{clip}%
\pgfsetbuttcap%
\pgfsetroundjoin%
\definecolor{currentfill}{rgb}{0.121569,0.466667,0.705882}%
\pgfsetfillcolor{currentfill}%
\pgfsetfillopacity{0.585499}%
\pgfsetlinewidth{1.003750pt}%
\definecolor{currentstroke}{rgb}{0.121569,0.466667,0.705882}%
\pgfsetstrokecolor{currentstroke}%
\pgfsetstrokeopacity{0.585499}%
\pgfsetdash{}{0pt}%
\pgfpathmoveto{\pgfqpoint{0.890916in}{1.489160in}}%
\pgfpathcurveto{\pgfqpoint{0.899152in}{1.489160in}}{\pgfqpoint{0.907052in}{1.492433in}}{\pgfqpoint{0.912876in}{1.498256in}}%
\pgfpathcurveto{\pgfqpoint{0.918700in}{1.504080in}}{\pgfqpoint{0.921973in}{1.511980in}}{\pgfqpoint{0.921973in}{1.520217in}}%
\pgfpathcurveto{\pgfqpoint{0.921973in}{1.528453in}}{\pgfqpoint{0.918700in}{1.536353in}}{\pgfqpoint{0.912876in}{1.542177in}}%
\pgfpathcurveto{\pgfqpoint{0.907052in}{1.548001in}}{\pgfqpoint{0.899152in}{1.551273in}}{\pgfqpoint{0.890916in}{1.551273in}}%
\pgfpathcurveto{\pgfqpoint{0.882680in}{1.551273in}}{\pgfqpoint{0.874780in}{1.548001in}}{\pgfqpoint{0.868956in}{1.542177in}}%
\pgfpathcurveto{\pgfqpoint{0.863132in}{1.536353in}}{\pgfqpoint{0.859860in}{1.528453in}}{\pgfqpoint{0.859860in}{1.520217in}}%
\pgfpathcurveto{\pgfqpoint{0.859860in}{1.511980in}}{\pgfqpoint{0.863132in}{1.504080in}}{\pgfqpoint{0.868956in}{1.498256in}}%
\pgfpathcurveto{\pgfqpoint{0.874780in}{1.492433in}}{\pgfqpoint{0.882680in}{1.489160in}}{\pgfqpoint{0.890916in}{1.489160in}}%
\pgfpathclose%
\pgfusepath{stroke,fill}%
\end{pgfscope}%
\begin{pgfscope}%
\pgfpathrectangle{\pgfqpoint{0.100000in}{0.212622in}}{\pgfqpoint{3.696000in}{3.696000in}}%
\pgfusepath{clip}%
\pgfsetbuttcap%
\pgfsetroundjoin%
\definecolor{currentfill}{rgb}{0.121569,0.466667,0.705882}%
\pgfsetfillcolor{currentfill}%
\pgfsetfillopacity{0.585499}%
\pgfsetlinewidth{1.003750pt}%
\definecolor{currentstroke}{rgb}{0.121569,0.466667,0.705882}%
\pgfsetstrokecolor{currentstroke}%
\pgfsetstrokeopacity{0.585499}%
\pgfsetdash{}{0pt}%
\pgfpathmoveto{\pgfqpoint{0.890916in}{1.489160in}}%
\pgfpathcurveto{\pgfqpoint{0.899152in}{1.489160in}}{\pgfqpoint{0.907052in}{1.492433in}}{\pgfqpoint{0.912876in}{1.498256in}}%
\pgfpathcurveto{\pgfqpoint{0.918700in}{1.504080in}}{\pgfqpoint{0.921973in}{1.511980in}}{\pgfqpoint{0.921973in}{1.520217in}}%
\pgfpathcurveto{\pgfqpoint{0.921973in}{1.528453in}}{\pgfqpoint{0.918700in}{1.536353in}}{\pgfqpoint{0.912876in}{1.542177in}}%
\pgfpathcurveto{\pgfqpoint{0.907052in}{1.548001in}}{\pgfqpoint{0.899152in}{1.551273in}}{\pgfqpoint{0.890916in}{1.551273in}}%
\pgfpathcurveto{\pgfqpoint{0.882680in}{1.551273in}}{\pgfqpoint{0.874780in}{1.548001in}}{\pgfqpoint{0.868956in}{1.542177in}}%
\pgfpathcurveto{\pgfqpoint{0.863132in}{1.536353in}}{\pgfqpoint{0.859860in}{1.528453in}}{\pgfqpoint{0.859860in}{1.520217in}}%
\pgfpathcurveto{\pgfqpoint{0.859860in}{1.511980in}}{\pgfqpoint{0.863132in}{1.504080in}}{\pgfqpoint{0.868956in}{1.498256in}}%
\pgfpathcurveto{\pgfqpoint{0.874780in}{1.492433in}}{\pgfqpoint{0.882680in}{1.489160in}}{\pgfqpoint{0.890916in}{1.489160in}}%
\pgfpathclose%
\pgfusepath{stroke,fill}%
\end{pgfscope}%
\begin{pgfscope}%
\pgfpathrectangle{\pgfqpoint{0.100000in}{0.212622in}}{\pgfqpoint{3.696000in}{3.696000in}}%
\pgfusepath{clip}%
\pgfsetbuttcap%
\pgfsetroundjoin%
\definecolor{currentfill}{rgb}{0.121569,0.466667,0.705882}%
\pgfsetfillcolor{currentfill}%
\pgfsetfillopacity{0.585499}%
\pgfsetlinewidth{1.003750pt}%
\definecolor{currentstroke}{rgb}{0.121569,0.466667,0.705882}%
\pgfsetstrokecolor{currentstroke}%
\pgfsetstrokeopacity{0.585499}%
\pgfsetdash{}{0pt}%
\pgfpathmoveto{\pgfqpoint{0.890916in}{1.489160in}}%
\pgfpathcurveto{\pgfqpoint{0.899152in}{1.489160in}}{\pgfqpoint{0.907052in}{1.492433in}}{\pgfqpoint{0.912876in}{1.498256in}}%
\pgfpathcurveto{\pgfqpoint{0.918700in}{1.504080in}}{\pgfqpoint{0.921973in}{1.511980in}}{\pgfqpoint{0.921973in}{1.520217in}}%
\pgfpathcurveto{\pgfqpoint{0.921973in}{1.528453in}}{\pgfqpoint{0.918700in}{1.536353in}}{\pgfqpoint{0.912876in}{1.542177in}}%
\pgfpathcurveto{\pgfqpoint{0.907052in}{1.548001in}}{\pgfqpoint{0.899152in}{1.551273in}}{\pgfqpoint{0.890916in}{1.551273in}}%
\pgfpathcurveto{\pgfqpoint{0.882680in}{1.551273in}}{\pgfqpoint{0.874780in}{1.548001in}}{\pgfqpoint{0.868956in}{1.542177in}}%
\pgfpathcurveto{\pgfqpoint{0.863132in}{1.536353in}}{\pgfqpoint{0.859860in}{1.528453in}}{\pgfqpoint{0.859860in}{1.520217in}}%
\pgfpathcurveto{\pgfqpoint{0.859860in}{1.511980in}}{\pgfqpoint{0.863132in}{1.504080in}}{\pgfqpoint{0.868956in}{1.498256in}}%
\pgfpathcurveto{\pgfqpoint{0.874780in}{1.492433in}}{\pgfqpoint{0.882680in}{1.489160in}}{\pgfqpoint{0.890916in}{1.489160in}}%
\pgfpathclose%
\pgfusepath{stroke,fill}%
\end{pgfscope}%
\begin{pgfscope}%
\pgfpathrectangle{\pgfqpoint{0.100000in}{0.212622in}}{\pgfqpoint{3.696000in}{3.696000in}}%
\pgfusepath{clip}%
\pgfsetbuttcap%
\pgfsetroundjoin%
\definecolor{currentfill}{rgb}{0.121569,0.466667,0.705882}%
\pgfsetfillcolor{currentfill}%
\pgfsetfillopacity{0.585499}%
\pgfsetlinewidth{1.003750pt}%
\definecolor{currentstroke}{rgb}{0.121569,0.466667,0.705882}%
\pgfsetstrokecolor{currentstroke}%
\pgfsetstrokeopacity{0.585499}%
\pgfsetdash{}{0pt}%
\pgfpathmoveto{\pgfqpoint{0.890916in}{1.489160in}}%
\pgfpathcurveto{\pgfqpoint{0.899152in}{1.489160in}}{\pgfqpoint{0.907052in}{1.492433in}}{\pgfqpoint{0.912876in}{1.498256in}}%
\pgfpathcurveto{\pgfqpoint{0.918700in}{1.504080in}}{\pgfqpoint{0.921973in}{1.511980in}}{\pgfqpoint{0.921973in}{1.520217in}}%
\pgfpathcurveto{\pgfqpoint{0.921973in}{1.528453in}}{\pgfqpoint{0.918700in}{1.536353in}}{\pgfqpoint{0.912876in}{1.542177in}}%
\pgfpathcurveto{\pgfqpoint{0.907052in}{1.548001in}}{\pgfqpoint{0.899152in}{1.551273in}}{\pgfqpoint{0.890916in}{1.551273in}}%
\pgfpathcurveto{\pgfqpoint{0.882680in}{1.551273in}}{\pgfqpoint{0.874780in}{1.548001in}}{\pgfqpoint{0.868956in}{1.542177in}}%
\pgfpathcurveto{\pgfqpoint{0.863132in}{1.536353in}}{\pgfqpoint{0.859860in}{1.528453in}}{\pgfqpoint{0.859860in}{1.520217in}}%
\pgfpathcurveto{\pgfqpoint{0.859860in}{1.511980in}}{\pgfqpoint{0.863132in}{1.504080in}}{\pgfqpoint{0.868956in}{1.498256in}}%
\pgfpathcurveto{\pgfqpoint{0.874780in}{1.492433in}}{\pgfqpoint{0.882680in}{1.489160in}}{\pgfqpoint{0.890916in}{1.489160in}}%
\pgfpathclose%
\pgfusepath{stroke,fill}%
\end{pgfscope}%
\begin{pgfscope}%
\pgfpathrectangle{\pgfqpoint{0.100000in}{0.212622in}}{\pgfqpoint{3.696000in}{3.696000in}}%
\pgfusepath{clip}%
\pgfsetbuttcap%
\pgfsetroundjoin%
\definecolor{currentfill}{rgb}{0.121569,0.466667,0.705882}%
\pgfsetfillcolor{currentfill}%
\pgfsetfillopacity{0.585499}%
\pgfsetlinewidth{1.003750pt}%
\definecolor{currentstroke}{rgb}{0.121569,0.466667,0.705882}%
\pgfsetstrokecolor{currentstroke}%
\pgfsetstrokeopacity{0.585499}%
\pgfsetdash{}{0pt}%
\pgfpathmoveto{\pgfqpoint{0.890916in}{1.489160in}}%
\pgfpathcurveto{\pgfqpoint{0.899152in}{1.489160in}}{\pgfqpoint{0.907052in}{1.492433in}}{\pgfqpoint{0.912876in}{1.498256in}}%
\pgfpathcurveto{\pgfqpoint{0.918700in}{1.504080in}}{\pgfqpoint{0.921973in}{1.511980in}}{\pgfqpoint{0.921973in}{1.520217in}}%
\pgfpathcurveto{\pgfqpoint{0.921973in}{1.528453in}}{\pgfqpoint{0.918700in}{1.536353in}}{\pgfqpoint{0.912876in}{1.542177in}}%
\pgfpathcurveto{\pgfqpoint{0.907052in}{1.548001in}}{\pgfqpoint{0.899152in}{1.551273in}}{\pgfqpoint{0.890916in}{1.551273in}}%
\pgfpathcurveto{\pgfqpoint{0.882680in}{1.551273in}}{\pgfqpoint{0.874780in}{1.548001in}}{\pgfqpoint{0.868956in}{1.542177in}}%
\pgfpathcurveto{\pgfqpoint{0.863132in}{1.536353in}}{\pgfqpoint{0.859860in}{1.528453in}}{\pgfqpoint{0.859860in}{1.520217in}}%
\pgfpathcurveto{\pgfqpoint{0.859860in}{1.511980in}}{\pgfqpoint{0.863132in}{1.504080in}}{\pgfqpoint{0.868956in}{1.498256in}}%
\pgfpathcurveto{\pgfqpoint{0.874780in}{1.492433in}}{\pgfqpoint{0.882680in}{1.489160in}}{\pgfqpoint{0.890916in}{1.489160in}}%
\pgfpathclose%
\pgfusepath{stroke,fill}%
\end{pgfscope}%
\begin{pgfscope}%
\pgfpathrectangle{\pgfqpoint{0.100000in}{0.212622in}}{\pgfqpoint{3.696000in}{3.696000in}}%
\pgfusepath{clip}%
\pgfsetbuttcap%
\pgfsetroundjoin%
\definecolor{currentfill}{rgb}{0.121569,0.466667,0.705882}%
\pgfsetfillcolor{currentfill}%
\pgfsetfillopacity{0.585499}%
\pgfsetlinewidth{1.003750pt}%
\definecolor{currentstroke}{rgb}{0.121569,0.466667,0.705882}%
\pgfsetstrokecolor{currentstroke}%
\pgfsetstrokeopacity{0.585499}%
\pgfsetdash{}{0pt}%
\pgfpathmoveto{\pgfqpoint{0.890916in}{1.489160in}}%
\pgfpathcurveto{\pgfqpoint{0.899152in}{1.489160in}}{\pgfqpoint{0.907052in}{1.492433in}}{\pgfqpoint{0.912876in}{1.498256in}}%
\pgfpathcurveto{\pgfqpoint{0.918700in}{1.504080in}}{\pgfqpoint{0.921973in}{1.511980in}}{\pgfqpoint{0.921973in}{1.520217in}}%
\pgfpathcurveto{\pgfqpoint{0.921973in}{1.528453in}}{\pgfqpoint{0.918700in}{1.536353in}}{\pgfqpoint{0.912876in}{1.542177in}}%
\pgfpathcurveto{\pgfqpoint{0.907052in}{1.548001in}}{\pgfqpoint{0.899152in}{1.551273in}}{\pgfqpoint{0.890916in}{1.551273in}}%
\pgfpathcurveto{\pgfqpoint{0.882680in}{1.551273in}}{\pgfqpoint{0.874780in}{1.548001in}}{\pgfqpoint{0.868956in}{1.542177in}}%
\pgfpathcurveto{\pgfqpoint{0.863132in}{1.536353in}}{\pgfqpoint{0.859860in}{1.528453in}}{\pgfqpoint{0.859860in}{1.520217in}}%
\pgfpathcurveto{\pgfqpoint{0.859860in}{1.511980in}}{\pgfqpoint{0.863132in}{1.504080in}}{\pgfqpoint{0.868956in}{1.498256in}}%
\pgfpathcurveto{\pgfqpoint{0.874780in}{1.492433in}}{\pgfqpoint{0.882680in}{1.489160in}}{\pgfqpoint{0.890916in}{1.489160in}}%
\pgfpathclose%
\pgfusepath{stroke,fill}%
\end{pgfscope}%
\begin{pgfscope}%
\pgfpathrectangle{\pgfqpoint{0.100000in}{0.212622in}}{\pgfqpoint{3.696000in}{3.696000in}}%
\pgfusepath{clip}%
\pgfsetbuttcap%
\pgfsetroundjoin%
\definecolor{currentfill}{rgb}{0.121569,0.466667,0.705882}%
\pgfsetfillcolor{currentfill}%
\pgfsetfillopacity{0.585499}%
\pgfsetlinewidth{1.003750pt}%
\definecolor{currentstroke}{rgb}{0.121569,0.466667,0.705882}%
\pgfsetstrokecolor{currentstroke}%
\pgfsetstrokeopacity{0.585499}%
\pgfsetdash{}{0pt}%
\pgfpathmoveto{\pgfqpoint{0.890916in}{1.489160in}}%
\pgfpathcurveto{\pgfqpoint{0.899152in}{1.489160in}}{\pgfqpoint{0.907052in}{1.492433in}}{\pgfqpoint{0.912876in}{1.498256in}}%
\pgfpathcurveto{\pgfqpoint{0.918700in}{1.504080in}}{\pgfqpoint{0.921973in}{1.511980in}}{\pgfqpoint{0.921973in}{1.520217in}}%
\pgfpathcurveto{\pgfqpoint{0.921973in}{1.528453in}}{\pgfqpoint{0.918700in}{1.536353in}}{\pgfqpoint{0.912876in}{1.542177in}}%
\pgfpathcurveto{\pgfqpoint{0.907052in}{1.548001in}}{\pgfqpoint{0.899152in}{1.551273in}}{\pgfqpoint{0.890916in}{1.551273in}}%
\pgfpathcurveto{\pgfqpoint{0.882680in}{1.551273in}}{\pgfqpoint{0.874780in}{1.548001in}}{\pgfqpoint{0.868956in}{1.542177in}}%
\pgfpathcurveto{\pgfqpoint{0.863132in}{1.536353in}}{\pgfqpoint{0.859860in}{1.528453in}}{\pgfqpoint{0.859860in}{1.520217in}}%
\pgfpathcurveto{\pgfqpoint{0.859860in}{1.511980in}}{\pgfqpoint{0.863132in}{1.504080in}}{\pgfqpoint{0.868956in}{1.498256in}}%
\pgfpathcurveto{\pgfqpoint{0.874780in}{1.492433in}}{\pgfqpoint{0.882680in}{1.489160in}}{\pgfqpoint{0.890916in}{1.489160in}}%
\pgfpathclose%
\pgfusepath{stroke,fill}%
\end{pgfscope}%
\begin{pgfscope}%
\pgfpathrectangle{\pgfqpoint{0.100000in}{0.212622in}}{\pgfqpoint{3.696000in}{3.696000in}}%
\pgfusepath{clip}%
\pgfsetbuttcap%
\pgfsetroundjoin%
\definecolor{currentfill}{rgb}{0.121569,0.466667,0.705882}%
\pgfsetfillcolor{currentfill}%
\pgfsetfillopacity{0.585499}%
\pgfsetlinewidth{1.003750pt}%
\definecolor{currentstroke}{rgb}{0.121569,0.466667,0.705882}%
\pgfsetstrokecolor{currentstroke}%
\pgfsetstrokeopacity{0.585499}%
\pgfsetdash{}{0pt}%
\pgfpathmoveto{\pgfqpoint{0.890916in}{1.489160in}}%
\pgfpathcurveto{\pgfqpoint{0.899152in}{1.489160in}}{\pgfqpoint{0.907052in}{1.492433in}}{\pgfqpoint{0.912876in}{1.498256in}}%
\pgfpathcurveto{\pgfqpoint{0.918700in}{1.504080in}}{\pgfqpoint{0.921973in}{1.511980in}}{\pgfqpoint{0.921973in}{1.520217in}}%
\pgfpathcurveto{\pgfqpoint{0.921973in}{1.528453in}}{\pgfqpoint{0.918700in}{1.536353in}}{\pgfqpoint{0.912876in}{1.542177in}}%
\pgfpathcurveto{\pgfqpoint{0.907052in}{1.548001in}}{\pgfqpoint{0.899152in}{1.551273in}}{\pgfqpoint{0.890916in}{1.551273in}}%
\pgfpathcurveto{\pgfqpoint{0.882680in}{1.551273in}}{\pgfqpoint{0.874780in}{1.548001in}}{\pgfqpoint{0.868956in}{1.542177in}}%
\pgfpathcurveto{\pgfqpoint{0.863132in}{1.536353in}}{\pgfqpoint{0.859860in}{1.528453in}}{\pgfqpoint{0.859860in}{1.520217in}}%
\pgfpathcurveto{\pgfqpoint{0.859860in}{1.511980in}}{\pgfqpoint{0.863132in}{1.504080in}}{\pgfqpoint{0.868956in}{1.498256in}}%
\pgfpathcurveto{\pgfqpoint{0.874780in}{1.492433in}}{\pgfqpoint{0.882680in}{1.489160in}}{\pgfqpoint{0.890916in}{1.489160in}}%
\pgfpathclose%
\pgfusepath{stroke,fill}%
\end{pgfscope}%
\begin{pgfscope}%
\pgfpathrectangle{\pgfqpoint{0.100000in}{0.212622in}}{\pgfqpoint{3.696000in}{3.696000in}}%
\pgfusepath{clip}%
\pgfsetbuttcap%
\pgfsetroundjoin%
\definecolor{currentfill}{rgb}{0.121569,0.466667,0.705882}%
\pgfsetfillcolor{currentfill}%
\pgfsetfillopacity{0.585499}%
\pgfsetlinewidth{1.003750pt}%
\definecolor{currentstroke}{rgb}{0.121569,0.466667,0.705882}%
\pgfsetstrokecolor{currentstroke}%
\pgfsetstrokeopacity{0.585499}%
\pgfsetdash{}{0pt}%
\pgfpathmoveto{\pgfqpoint{0.890916in}{1.489160in}}%
\pgfpathcurveto{\pgfqpoint{0.899152in}{1.489160in}}{\pgfqpoint{0.907052in}{1.492433in}}{\pgfqpoint{0.912876in}{1.498256in}}%
\pgfpathcurveto{\pgfqpoint{0.918700in}{1.504080in}}{\pgfqpoint{0.921973in}{1.511980in}}{\pgfqpoint{0.921973in}{1.520217in}}%
\pgfpathcurveto{\pgfqpoint{0.921973in}{1.528453in}}{\pgfqpoint{0.918700in}{1.536353in}}{\pgfqpoint{0.912876in}{1.542177in}}%
\pgfpathcurveto{\pgfqpoint{0.907052in}{1.548001in}}{\pgfqpoint{0.899152in}{1.551273in}}{\pgfqpoint{0.890916in}{1.551273in}}%
\pgfpathcurveto{\pgfqpoint{0.882680in}{1.551273in}}{\pgfqpoint{0.874780in}{1.548001in}}{\pgfqpoint{0.868956in}{1.542177in}}%
\pgfpathcurveto{\pgfqpoint{0.863132in}{1.536353in}}{\pgfqpoint{0.859860in}{1.528453in}}{\pgfqpoint{0.859860in}{1.520217in}}%
\pgfpathcurveto{\pgfqpoint{0.859860in}{1.511980in}}{\pgfqpoint{0.863132in}{1.504080in}}{\pgfqpoint{0.868956in}{1.498256in}}%
\pgfpathcurveto{\pgfqpoint{0.874780in}{1.492433in}}{\pgfqpoint{0.882680in}{1.489160in}}{\pgfqpoint{0.890916in}{1.489160in}}%
\pgfpathclose%
\pgfusepath{stroke,fill}%
\end{pgfscope}%
\begin{pgfscope}%
\pgfpathrectangle{\pgfqpoint{0.100000in}{0.212622in}}{\pgfqpoint{3.696000in}{3.696000in}}%
\pgfusepath{clip}%
\pgfsetbuttcap%
\pgfsetroundjoin%
\definecolor{currentfill}{rgb}{0.121569,0.466667,0.705882}%
\pgfsetfillcolor{currentfill}%
\pgfsetfillopacity{0.585549}%
\pgfsetlinewidth{1.003750pt}%
\definecolor{currentstroke}{rgb}{0.121569,0.466667,0.705882}%
\pgfsetstrokecolor{currentstroke}%
\pgfsetstrokeopacity{0.585549}%
\pgfsetdash{}{0pt}%
\pgfpathmoveto{\pgfqpoint{2.091715in}{2.152751in}}%
\pgfpathcurveto{\pgfqpoint{2.099951in}{2.152751in}}{\pgfqpoint{2.107851in}{2.156023in}}{\pgfqpoint{2.113675in}{2.161847in}}%
\pgfpathcurveto{\pgfqpoint{2.119499in}{2.167671in}}{\pgfqpoint{2.122771in}{2.175571in}}{\pgfqpoint{2.122771in}{2.183807in}}%
\pgfpathcurveto{\pgfqpoint{2.122771in}{2.192044in}}{\pgfqpoint{2.119499in}{2.199944in}}{\pgfqpoint{2.113675in}{2.205768in}}%
\pgfpathcurveto{\pgfqpoint{2.107851in}{2.211592in}}{\pgfqpoint{2.099951in}{2.214864in}}{\pgfqpoint{2.091715in}{2.214864in}}%
\pgfpathcurveto{\pgfqpoint{2.083478in}{2.214864in}}{\pgfqpoint{2.075578in}{2.211592in}}{\pgfqpoint{2.069754in}{2.205768in}}%
\pgfpathcurveto{\pgfqpoint{2.063931in}{2.199944in}}{\pgfqpoint{2.060658in}{2.192044in}}{\pgfqpoint{2.060658in}{2.183807in}}%
\pgfpathcurveto{\pgfqpoint{2.060658in}{2.175571in}}{\pgfqpoint{2.063931in}{2.167671in}}{\pgfqpoint{2.069754in}{2.161847in}}%
\pgfpathcurveto{\pgfqpoint{2.075578in}{2.156023in}}{\pgfqpoint{2.083478in}{2.152751in}}{\pgfqpoint{2.091715in}{2.152751in}}%
\pgfpathclose%
\pgfusepath{stroke,fill}%
\end{pgfscope}%
\begin{pgfscope}%
\pgfpathrectangle{\pgfqpoint{0.100000in}{0.212622in}}{\pgfqpoint{3.696000in}{3.696000in}}%
\pgfusepath{clip}%
\pgfsetbuttcap%
\pgfsetroundjoin%
\definecolor{currentfill}{rgb}{0.121569,0.466667,0.705882}%
\pgfsetfillcolor{currentfill}%
\pgfsetfillopacity{0.585550}%
\pgfsetlinewidth{1.003750pt}%
\definecolor{currentstroke}{rgb}{0.121569,0.466667,0.705882}%
\pgfsetstrokecolor{currentstroke}%
\pgfsetstrokeopacity{0.585550}%
\pgfsetdash{}{0pt}%
\pgfpathmoveto{\pgfqpoint{0.890769in}{1.488751in}}%
\pgfpathcurveto{\pgfqpoint{0.899005in}{1.488751in}}{\pgfqpoint{0.906906in}{1.492024in}}{\pgfqpoint{0.912729in}{1.497848in}}%
\pgfpathcurveto{\pgfqpoint{0.918553in}{1.503672in}}{\pgfqpoint{0.921826in}{1.511572in}}{\pgfqpoint{0.921826in}{1.519808in}}%
\pgfpathcurveto{\pgfqpoint{0.921826in}{1.528044in}}{\pgfqpoint{0.918553in}{1.535944in}}{\pgfqpoint{0.912729in}{1.541768in}}%
\pgfpathcurveto{\pgfqpoint{0.906906in}{1.547592in}}{\pgfqpoint{0.899005in}{1.550864in}}{\pgfqpoint{0.890769in}{1.550864in}}%
\pgfpathcurveto{\pgfqpoint{0.882533in}{1.550864in}}{\pgfqpoint{0.874633in}{1.547592in}}{\pgfqpoint{0.868809in}{1.541768in}}%
\pgfpathcurveto{\pgfqpoint{0.862985in}{1.535944in}}{\pgfqpoint{0.859713in}{1.528044in}}{\pgfqpoint{0.859713in}{1.519808in}}%
\pgfpathcurveto{\pgfqpoint{0.859713in}{1.511572in}}{\pgfqpoint{0.862985in}{1.503672in}}{\pgfqpoint{0.868809in}{1.497848in}}%
\pgfpathcurveto{\pgfqpoint{0.874633in}{1.492024in}}{\pgfqpoint{0.882533in}{1.488751in}}{\pgfqpoint{0.890769in}{1.488751in}}%
\pgfpathclose%
\pgfusepath{stroke,fill}%
\end{pgfscope}%
\begin{pgfscope}%
\pgfpathrectangle{\pgfqpoint{0.100000in}{0.212622in}}{\pgfqpoint{3.696000in}{3.696000in}}%
\pgfusepath{clip}%
\pgfsetbuttcap%
\pgfsetroundjoin%
\definecolor{currentfill}{rgb}{0.121569,0.466667,0.705882}%
\pgfsetfillcolor{currentfill}%
\pgfsetfillopacity{0.585577}%
\pgfsetlinewidth{1.003750pt}%
\definecolor{currentstroke}{rgb}{0.121569,0.466667,0.705882}%
\pgfsetstrokecolor{currentstroke}%
\pgfsetstrokeopacity{0.585577}%
\pgfsetdash{}{0pt}%
\pgfpathmoveto{\pgfqpoint{0.890685in}{1.488527in}}%
\pgfpathcurveto{\pgfqpoint{0.898921in}{1.488527in}}{\pgfqpoint{0.906821in}{1.491800in}}{\pgfqpoint{0.912645in}{1.497624in}}%
\pgfpathcurveto{\pgfqpoint{0.918469in}{1.503447in}}{\pgfqpoint{0.921742in}{1.511347in}}{\pgfqpoint{0.921742in}{1.519584in}}%
\pgfpathcurveto{\pgfqpoint{0.921742in}{1.527820in}}{\pgfqpoint{0.918469in}{1.535720in}}{\pgfqpoint{0.912645in}{1.541544in}}%
\pgfpathcurveto{\pgfqpoint{0.906821in}{1.547368in}}{\pgfqpoint{0.898921in}{1.550640in}}{\pgfqpoint{0.890685in}{1.550640in}}%
\pgfpathcurveto{\pgfqpoint{0.882449in}{1.550640in}}{\pgfqpoint{0.874549in}{1.547368in}}{\pgfqpoint{0.868725in}{1.541544in}}%
\pgfpathcurveto{\pgfqpoint{0.862901in}{1.535720in}}{\pgfqpoint{0.859629in}{1.527820in}}{\pgfqpoint{0.859629in}{1.519584in}}%
\pgfpathcurveto{\pgfqpoint{0.859629in}{1.511347in}}{\pgfqpoint{0.862901in}{1.503447in}}{\pgfqpoint{0.868725in}{1.497624in}}%
\pgfpathcurveto{\pgfqpoint{0.874549in}{1.491800in}}{\pgfqpoint{0.882449in}{1.488527in}}{\pgfqpoint{0.890685in}{1.488527in}}%
\pgfpathclose%
\pgfusepath{stroke,fill}%
\end{pgfscope}%
\begin{pgfscope}%
\pgfpathrectangle{\pgfqpoint{0.100000in}{0.212622in}}{\pgfqpoint{3.696000in}{3.696000in}}%
\pgfusepath{clip}%
\pgfsetbuttcap%
\pgfsetroundjoin%
\definecolor{currentfill}{rgb}{0.121569,0.466667,0.705882}%
\pgfsetfillcolor{currentfill}%
\pgfsetfillopacity{0.585850}%
\pgfsetlinewidth{1.003750pt}%
\definecolor{currentstroke}{rgb}{0.121569,0.466667,0.705882}%
\pgfsetstrokecolor{currentstroke}%
\pgfsetstrokeopacity{0.585850}%
\pgfsetdash{}{0pt}%
\pgfpathmoveto{\pgfqpoint{0.889838in}{1.486929in}}%
\pgfpathcurveto{\pgfqpoint{0.898074in}{1.486929in}}{\pgfqpoint{0.905974in}{1.490201in}}{\pgfqpoint{0.911798in}{1.496025in}}%
\pgfpathcurveto{\pgfqpoint{0.917622in}{1.501849in}}{\pgfqpoint{0.920895in}{1.509749in}}{\pgfqpoint{0.920895in}{1.517985in}}%
\pgfpathcurveto{\pgfqpoint{0.920895in}{1.526222in}}{\pgfqpoint{0.917622in}{1.534122in}}{\pgfqpoint{0.911798in}{1.539946in}}%
\pgfpathcurveto{\pgfqpoint{0.905974in}{1.545770in}}{\pgfqpoint{0.898074in}{1.549042in}}{\pgfqpoint{0.889838in}{1.549042in}}%
\pgfpathcurveto{\pgfqpoint{0.881602in}{1.549042in}}{\pgfqpoint{0.873702in}{1.545770in}}{\pgfqpoint{0.867878in}{1.539946in}}%
\pgfpathcurveto{\pgfqpoint{0.862054in}{1.534122in}}{\pgfqpoint{0.858782in}{1.526222in}}{\pgfqpoint{0.858782in}{1.517985in}}%
\pgfpathcurveto{\pgfqpoint{0.858782in}{1.509749in}}{\pgfqpoint{0.862054in}{1.501849in}}{\pgfqpoint{0.867878in}{1.496025in}}%
\pgfpathcurveto{\pgfqpoint{0.873702in}{1.490201in}}{\pgfqpoint{0.881602in}{1.486929in}}{\pgfqpoint{0.889838in}{1.486929in}}%
\pgfpathclose%
\pgfusepath{stroke,fill}%
\end{pgfscope}%
\begin{pgfscope}%
\pgfpathrectangle{\pgfqpoint{0.100000in}{0.212622in}}{\pgfqpoint{3.696000in}{3.696000in}}%
\pgfusepath{clip}%
\pgfsetbuttcap%
\pgfsetroundjoin%
\definecolor{currentfill}{rgb}{0.121569,0.466667,0.705882}%
\pgfsetfillcolor{currentfill}%
\pgfsetfillopacity{0.586013}%
\pgfsetlinewidth{1.003750pt}%
\definecolor{currentstroke}{rgb}{0.121569,0.466667,0.705882}%
\pgfsetstrokecolor{currentstroke}%
\pgfsetstrokeopacity{0.586013}%
\pgfsetdash{}{0pt}%
\pgfpathmoveto{\pgfqpoint{0.889350in}{1.486125in}}%
\pgfpathcurveto{\pgfqpoint{0.897587in}{1.486125in}}{\pgfqpoint{0.905487in}{1.489398in}}{\pgfqpoint{0.911311in}{1.495222in}}%
\pgfpathcurveto{\pgfqpoint{0.917135in}{1.501046in}}{\pgfqpoint{0.920407in}{1.508946in}}{\pgfqpoint{0.920407in}{1.517182in}}%
\pgfpathcurveto{\pgfqpoint{0.920407in}{1.525418in}}{\pgfqpoint{0.917135in}{1.533318in}}{\pgfqpoint{0.911311in}{1.539142in}}%
\pgfpathcurveto{\pgfqpoint{0.905487in}{1.544966in}}{\pgfqpoint{0.897587in}{1.548238in}}{\pgfqpoint{0.889350in}{1.548238in}}%
\pgfpathcurveto{\pgfqpoint{0.881114in}{1.548238in}}{\pgfqpoint{0.873214in}{1.544966in}}{\pgfqpoint{0.867390in}{1.539142in}}%
\pgfpathcurveto{\pgfqpoint{0.861566in}{1.533318in}}{\pgfqpoint{0.858294in}{1.525418in}}{\pgfqpoint{0.858294in}{1.517182in}}%
\pgfpathcurveto{\pgfqpoint{0.858294in}{1.508946in}}{\pgfqpoint{0.861566in}{1.501046in}}{\pgfqpoint{0.867390in}{1.495222in}}%
\pgfpathcurveto{\pgfqpoint{0.873214in}{1.489398in}}{\pgfqpoint{0.881114in}{1.486125in}}{\pgfqpoint{0.889350in}{1.486125in}}%
\pgfpathclose%
\pgfusepath{stroke,fill}%
\end{pgfscope}%
\begin{pgfscope}%
\pgfpathrectangle{\pgfqpoint{0.100000in}{0.212622in}}{\pgfqpoint{3.696000in}{3.696000in}}%
\pgfusepath{clip}%
\pgfsetbuttcap%
\pgfsetroundjoin%
\definecolor{currentfill}{rgb}{0.121569,0.466667,0.705882}%
\pgfsetfillcolor{currentfill}%
\pgfsetfillopacity{0.586119}%
\pgfsetlinewidth{1.003750pt}%
\definecolor{currentstroke}{rgb}{0.121569,0.466667,0.705882}%
\pgfsetstrokecolor{currentstroke}%
\pgfsetstrokeopacity{0.586119}%
\pgfsetdash{}{0pt}%
\pgfpathmoveto{\pgfqpoint{0.889051in}{1.485793in}}%
\pgfpathcurveto{\pgfqpoint{0.897287in}{1.485793in}}{\pgfqpoint{0.905187in}{1.489065in}}{\pgfqpoint{0.911011in}{1.494889in}}%
\pgfpathcurveto{\pgfqpoint{0.916835in}{1.500713in}}{\pgfqpoint{0.920107in}{1.508613in}}{\pgfqpoint{0.920107in}{1.516849in}}%
\pgfpathcurveto{\pgfqpoint{0.920107in}{1.525085in}}{\pgfqpoint{0.916835in}{1.532985in}}{\pgfqpoint{0.911011in}{1.538809in}}%
\pgfpathcurveto{\pgfqpoint{0.905187in}{1.544633in}}{\pgfqpoint{0.897287in}{1.547906in}}{\pgfqpoint{0.889051in}{1.547906in}}%
\pgfpathcurveto{\pgfqpoint{0.880815in}{1.547906in}}{\pgfqpoint{0.872914in}{1.544633in}}{\pgfqpoint{0.867091in}{1.538809in}}%
\pgfpathcurveto{\pgfqpoint{0.861267in}{1.532985in}}{\pgfqpoint{0.857994in}{1.525085in}}{\pgfqpoint{0.857994in}{1.516849in}}%
\pgfpathcurveto{\pgfqpoint{0.857994in}{1.508613in}}{\pgfqpoint{0.861267in}{1.500713in}}{\pgfqpoint{0.867091in}{1.494889in}}%
\pgfpathcurveto{\pgfqpoint{0.872914in}{1.489065in}}{\pgfqpoint{0.880815in}{1.485793in}}{\pgfqpoint{0.889051in}{1.485793in}}%
\pgfpathclose%
\pgfusepath{stroke,fill}%
\end{pgfscope}%
\begin{pgfscope}%
\pgfpathrectangle{\pgfqpoint{0.100000in}{0.212622in}}{\pgfqpoint{3.696000in}{3.696000in}}%
\pgfusepath{clip}%
\pgfsetbuttcap%
\pgfsetroundjoin%
\definecolor{currentfill}{rgb}{0.121569,0.466667,0.705882}%
\pgfsetfillcolor{currentfill}%
\pgfsetfillopacity{0.586196}%
\pgfsetlinewidth{1.003750pt}%
\definecolor{currentstroke}{rgb}{0.121569,0.466667,0.705882}%
\pgfsetstrokecolor{currentstroke}%
\pgfsetstrokeopacity{0.586196}%
\pgfsetdash{}{0pt}%
\pgfpathmoveto{\pgfqpoint{0.888901in}{1.485649in}}%
\pgfpathcurveto{\pgfqpoint{0.897137in}{1.485649in}}{\pgfqpoint{0.905038in}{1.488922in}}{\pgfqpoint{0.910861in}{1.494746in}}%
\pgfpathcurveto{\pgfqpoint{0.916685in}{1.500570in}}{\pgfqpoint{0.919958in}{1.508470in}}{\pgfqpoint{0.919958in}{1.516706in}}%
\pgfpathcurveto{\pgfqpoint{0.919958in}{1.524942in}}{\pgfqpoint{0.916685in}{1.532842in}}{\pgfqpoint{0.910861in}{1.538666in}}%
\pgfpathcurveto{\pgfqpoint{0.905038in}{1.544490in}}{\pgfqpoint{0.897137in}{1.547762in}}{\pgfqpoint{0.888901in}{1.547762in}}%
\pgfpathcurveto{\pgfqpoint{0.880665in}{1.547762in}}{\pgfqpoint{0.872765in}{1.544490in}}{\pgfqpoint{0.866941in}{1.538666in}}%
\pgfpathcurveto{\pgfqpoint{0.861117in}{1.532842in}}{\pgfqpoint{0.857845in}{1.524942in}}{\pgfqpoint{0.857845in}{1.516706in}}%
\pgfpathcurveto{\pgfqpoint{0.857845in}{1.508470in}}{\pgfqpoint{0.861117in}{1.500570in}}{\pgfqpoint{0.866941in}{1.494746in}}%
\pgfpathcurveto{\pgfqpoint{0.872765in}{1.488922in}}{\pgfqpoint{0.880665in}{1.485649in}}{\pgfqpoint{0.888901in}{1.485649in}}%
\pgfpathclose%
\pgfusepath{stroke,fill}%
\end{pgfscope}%
\begin{pgfscope}%
\pgfpathrectangle{\pgfqpoint{0.100000in}{0.212622in}}{\pgfqpoint{3.696000in}{3.696000in}}%
\pgfusepath{clip}%
\pgfsetbuttcap%
\pgfsetroundjoin%
\definecolor{currentfill}{rgb}{0.121569,0.466667,0.705882}%
\pgfsetfillcolor{currentfill}%
\pgfsetfillopacity{0.586462}%
\pgfsetlinewidth{1.003750pt}%
\definecolor{currentstroke}{rgb}{0.121569,0.466667,0.705882}%
\pgfsetstrokecolor{currentstroke}%
\pgfsetstrokeopacity{0.586462}%
\pgfsetdash{}{0pt}%
\pgfpathmoveto{\pgfqpoint{0.888437in}{1.485303in}}%
\pgfpathcurveto{\pgfqpoint{0.896673in}{1.485303in}}{\pgfqpoint{0.904573in}{1.488575in}}{\pgfqpoint{0.910397in}{1.494399in}}%
\pgfpathcurveto{\pgfqpoint{0.916221in}{1.500223in}}{\pgfqpoint{0.919494in}{1.508123in}}{\pgfqpoint{0.919494in}{1.516359in}}%
\pgfpathcurveto{\pgfqpoint{0.919494in}{1.524596in}}{\pgfqpoint{0.916221in}{1.532496in}}{\pgfqpoint{0.910397in}{1.538320in}}%
\pgfpathcurveto{\pgfqpoint{0.904573in}{1.544144in}}{\pgfqpoint{0.896673in}{1.547416in}}{\pgfqpoint{0.888437in}{1.547416in}}%
\pgfpathcurveto{\pgfqpoint{0.880201in}{1.547416in}}{\pgfqpoint{0.872301in}{1.544144in}}{\pgfqpoint{0.866477in}{1.538320in}}%
\pgfpathcurveto{\pgfqpoint{0.860653in}{1.532496in}}{\pgfqpoint{0.857381in}{1.524596in}}{\pgfqpoint{0.857381in}{1.516359in}}%
\pgfpathcurveto{\pgfqpoint{0.857381in}{1.508123in}}{\pgfqpoint{0.860653in}{1.500223in}}{\pgfqpoint{0.866477in}{1.494399in}}%
\pgfpathcurveto{\pgfqpoint{0.872301in}{1.488575in}}{\pgfqpoint{0.880201in}{1.485303in}}{\pgfqpoint{0.888437in}{1.485303in}}%
\pgfpathclose%
\pgfusepath{stroke,fill}%
\end{pgfscope}%
\begin{pgfscope}%
\pgfpathrectangle{\pgfqpoint{0.100000in}{0.212622in}}{\pgfqpoint{3.696000in}{3.696000in}}%
\pgfusepath{clip}%
\pgfsetbuttcap%
\pgfsetroundjoin%
\definecolor{currentfill}{rgb}{0.121569,0.466667,0.705882}%
\pgfsetfillcolor{currentfill}%
\pgfsetfillopacity{0.586653}%
\pgfsetlinewidth{1.003750pt}%
\definecolor{currentstroke}{rgb}{0.121569,0.466667,0.705882}%
\pgfsetstrokecolor{currentstroke}%
\pgfsetstrokeopacity{0.586653}%
\pgfsetdash{}{0pt}%
\pgfpathmoveto{\pgfqpoint{0.888199in}{1.485247in}}%
\pgfpathcurveto{\pgfqpoint{0.896436in}{1.485247in}}{\pgfqpoint{0.904336in}{1.488520in}}{\pgfqpoint{0.910160in}{1.494344in}}%
\pgfpathcurveto{\pgfqpoint{0.915984in}{1.500168in}}{\pgfqpoint{0.919256in}{1.508068in}}{\pgfqpoint{0.919256in}{1.516304in}}%
\pgfpathcurveto{\pgfqpoint{0.919256in}{1.524540in}}{\pgfqpoint{0.915984in}{1.532440in}}{\pgfqpoint{0.910160in}{1.538264in}}%
\pgfpathcurveto{\pgfqpoint{0.904336in}{1.544088in}}{\pgfqpoint{0.896436in}{1.547360in}}{\pgfqpoint{0.888199in}{1.547360in}}%
\pgfpathcurveto{\pgfqpoint{0.879963in}{1.547360in}}{\pgfqpoint{0.872063in}{1.544088in}}{\pgfqpoint{0.866239in}{1.538264in}}%
\pgfpathcurveto{\pgfqpoint{0.860415in}{1.532440in}}{\pgfqpoint{0.857143in}{1.524540in}}{\pgfqpoint{0.857143in}{1.516304in}}%
\pgfpathcurveto{\pgfqpoint{0.857143in}{1.508068in}}{\pgfqpoint{0.860415in}{1.500168in}}{\pgfqpoint{0.866239in}{1.494344in}}%
\pgfpathcurveto{\pgfqpoint{0.872063in}{1.488520in}}{\pgfqpoint{0.879963in}{1.485247in}}{\pgfqpoint{0.888199in}{1.485247in}}%
\pgfpathclose%
\pgfusepath{stroke,fill}%
\end{pgfscope}%
\begin{pgfscope}%
\pgfpathrectangle{\pgfqpoint{0.100000in}{0.212622in}}{\pgfqpoint{3.696000in}{3.696000in}}%
\pgfusepath{clip}%
\pgfsetbuttcap%
\pgfsetroundjoin%
\definecolor{currentfill}{rgb}{0.121569,0.466667,0.705882}%
\pgfsetfillcolor{currentfill}%
\pgfsetfillopacity{0.586797}%
\pgfsetlinewidth{1.003750pt}%
\definecolor{currentstroke}{rgb}{0.121569,0.466667,0.705882}%
\pgfsetstrokecolor{currentstroke}%
\pgfsetstrokeopacity{0.586797}%
\pgfsetdash{}{0pt}%
\pgfpathmoveto{\pgfqpoint{2.092908in}{2.146570in}}%
\pgfpathcurveto{\pgfqpoint{2.101144in}{2.146570in}}{\pgfqpoint{2.109044in}{2.149843in}}{\pgfqpoint{2.114868in}{2.155667in}}%
\pgfpathcurveto{\pgfqpoint{2.120692in}{2.161491in}}{\pgfqpoint{2.123964in}{2.169391in}}{\pgfqpoint{2.123964in}{2.177627in}}%
\pgfpathcurveto{\pgfqpoint{2.123964in}{2.185863in}}{\pgfqpoint{2.120692in}{2.193763in}}{\pgfqpoint{2.114868in}{2.199587in}}%
\pgfpathcurveto{\pgfqpoint{2.109044in}{2.205411in}}{\pgfqpoint{2.101144in}{2.208683in}}{\pgfqpoint{2.092908in}{2.208683in}}%
\pgfpathcurveto{\pgfqpoint{2.084671in}{2.208683in}}{\pgfqpoint{2.076771in}{2.205411in}}{\pgfqpoint{2.070947in}{2.199587in}}%
\pgfpathcurveto{\pgfqpoint{2.065123in}{2.193763in}}{\pgfqpoint{2.061851in}{2.185863in}}{\pgfqpoint{2.061851in}{2.177627in}}%
\pgfpathcurveto{\pgfqpoint{2.061851in}{2.169391in}}{\pgfqpoint{2.065123in}{2.161491in}}{\pgfqpoint{2.070947in}{2.155667in}}%
\pgfpathcurveto{\pgfqpoint{2.076771in}{2.149843in}}{\pgfqpoint{2.084671in}{2.146570in}}{\pgfqpoint{2.092908in}{2.146570in}}%
\pgfpathclose%
\pgfusepath{stroke,fill}%
\end{pgfscope}%
\begin{pgfscope}%
\pgfpathrectangle{\pgfqpoint{0.100000in}{0.212622in}}{\pgfqpoint{3.696000in}{3.696000in}}%
\pgfusepath{clip}%
\pgfsetbuttcap%
\pgfsetroundjoin%
\definecolor{currentfill}{rgb}{0.121569,0.466667,0.705882}%
\pgfsetfillcolor{currentfill}%
\pgfsetfillopacity{0.587027}%
\pgfsetlinewidth{1.003750pt}%
\definecolor{currentstroke}{rgb}{0.121569,0.466667,0.705882}%
\pgfsetstrokecolor{currentstroke}%
\pgfsetstrokeopacity{0.587027}%
\pgfsetdash{}{0pt}%
\pgfpathmoveto{\pgfqpoint{0.585743in}{1.220593in}}%
\pgfpathcurveto{\pgfqpoint{0.593979in}{1.220593in}}{\pgfqpoint{0.601879in}{1.223866in}}{\pgfqpoint{0.607703in}{1.229690in}}%
\pgfpathcurveto{\pgfqpoint{0.613527in}{1.235514in}}{\pgfqpoint{0.616799in}{1.243414in}}{\pgfqpoint{0.616799in}{1.251650in}}%
\pgfpathcurveto{\pgfqpoint{0.616799in}{1.259886in}}{\pgfqpoint{0.613527in}{1.267786in}}{\pgfqpoint{0.607703in}{1.273610in}}%
\pgfpathcurveto{\pgfqpoint{0.601879in}{1.279434in}}{\pgfqpoint{0.593979in}{1.282706in}}{\pgfqpoint{0.585743in}{1.282706in}}%
\pgfpathcurveto{\pgfqpoint{0.577507in}{1.282706in}}{\pgfqpoint{0.569606in}{1.279434in}}{\pgfqpoint{0.563783in}{1.273610in}}%
\pgfpathcurveto{\pgfqpoint{0.557959in}{1.267786in}}{\pgfqpoint{0.554686in}{1.259886in}}{\pgfqpoint{0.554686in}{1.251650in}}%
\pgfpathcurveto{\pgfqpoint{0.554686in}{1.243414in}}{\pgfqpoint{0.557959in}{1.235514in}}{\pgfqpoint{0.563783in}{1.229690in}}%
\pgfpathcurveto{\pgfqpoint{0.569606in}{1.223866in}}{\pgfqpoint{0.577507in}{1.220593in}}{\pgfqpoint{0.585743in}{1.220593in}}%
\pgfpathclose%
\pgfusepath{stroke,fill}%
\end{pgfscope}%
\begin{pgfscope}%
\pgfpathrectangle{\pgfqpoint{0.100000in}{0.212622in}}{\pgfqpoint{3.696000in}{3.696000in}}%
\pgfusepath{clip}%
\pgfsetbuttcap%
\pgfsetroundjoin%
\definecolor{currentfill}{rgb}{0.121569,0.466667,0.705882}%
\pgfsetfillcolor{currentfill}%
\pgfsetfillopacity{0.587440}%
\pgfsetlinewidth{1.003750pt}%
\definecolor{currentstroke}{rgb}{0.121569,0.466667,0.705882}%
\pgfsetstrokecolor{currentstroke}%
\pgfsetstrokeopacity{0.587440}%
\pgfsetdash{}{0pt}%
\pgfpathmoveto{\pgfqpoint{0.887224in}{1.485086in}}%
\pgfpathcurveto{\pgfqpoint{0.895460in}{1.485086in}}{\pgfqpoint{0.903360in}{1.488358in}}{\pgfqpoint{0.909184in}{1.494182in}}%
\pgfpathcurveto{\pgfqpoint{0.915008in}{1.500006in}}{\pgfqpoint{0.918280in}{1.507906in}}{\pgfqpoint{0.918280in}{1.516143in}}%
\pgfpathcurveto{\pgfqpoint{0.918280in}{1.524379in}}{\pgfqpoint{0.915008in}{1.532279in}}{\pgfqpoint{0.909184in}{1.538103in}}%
\pgfpathcurveto{\pgfqpoint{0.903360in}{1.543927in}}{\pgfqpoint{0.895460in}{1.547199in}}{\pgfqpoint{0.887224in}{1.547199in}}%
\pgfpathcurveto{\pgfqpoint{0.878987in}{1.547199in}}{\pgfqpoint{0.871087in}{1.543927in}}{\pgfqpoint{0.865263in}{1.538103in}}%
\pgfpathcurveto{\pgfqpoint{0.859439in}{1.532279in}}{\pgfqpoint{0.856167in}{1.524379in}}{\pgfqpoint{0.856167in}{1.516143in}}%
\pgfpathcurveto{\pgfqpoint{0.856167in}{1.507906in}}{\pgfqpoint{0.859439in}{1.500006in}}{\pgfqpoint{0.865263in}{1.494182in}}%
\pgfpathcurveto{\pgfqpoint{0.871087in}{1.488358in}}{\pgfqpoint{0.878987in}{1.485086in}}{\pgfqpoint{0.887224in}{1.485086in}}%
\pgfpathclose%
\pgfusepath{stroke,fill}%
\end{pgfscope}%
\begin{pgfscope}%
\pgfpathrectangle{\pgfqpoint{0.100000in}{0.212622in}}{\pgfqpoint{3.696000in}{3.696000in}}%
\pgfusepath{clip}%
\pgfsetbuttcap%
\pgfsetroundjoin%
\definecolor{currentfill}{rgb}{0.121569,0.466667,0.705882}%
\pgfsetfillcolor{currentfill}%
\pgfsetfillopacity{0.587569}%
\pgfsetlinewidth{1.003750pt}%
\definecolor{currentstroke}{rgb}{0.121569,0.466667,0.705882}%
\pgfsetstrokecolor{currentstroke}%
\pgfsetstrokeopacity{0.587569}%
\pgfsetdash{}{0pt}%
\pgfpathmoveto{\pgfqpoint{0.588738in}{1.218323in}}%
\pgfpathcurveto{\pgfqpoint{0.596974in}{1.218323in}}{\pgfqpoint{0.604874in}{1.221595in}}{\pgfqpoint{0.610698in}{1.227419in}}%
\pgfpathcurveto{\pgfqpoint{0.616522in}{1.233243in}}{\pgfqpoint{0.619795in}{1.241143in}}{\pgfqpoint{0.619795in}{1.249380in}}%
\pgfpathcurveto{\pgfqpoint{0.619795in}{1.257616in}}{\pgfqpoint{0.616522in}{1.265516in}}{\pgfqpoint{0.610698in}{1.271340in}}%
\pgfpathcurveto{\pgfqpoint{0.604874in}{1.277164in}}{\pgfqpoint{0.596974in}{1.280436in}}{\pgfqpoint{0.588738in}{1.280436in}}%
\pgfpathcurveto{\pgfqpoint{0.580502in}{1.280436in}}{\pgfqpoint{0.572602in}{1.277164in}}{\pgfqpoint{0.566778in}{1.271340in}}%
\pgfpathcurveto{\pgfqpoint{0.560954in}{1.265516in}}{\pgfqpoint{0.557682in}{1.257616in}}{\pgfqpoint{0.557682in}{1.249380in}}%
\pgfpathcurveto{\pgfqpoint{0.557682in}{1.241143in}}{\pgfqpoint{0.560954in}{1.233243in}}{\pgfqpoint{0.566778in}{1.227419in}}%
\pgfpathcurveto{\pgfqpoint{0.572602in}{1.221595in}}{\pgfqpoint{0.580502in}{1.218323in}}{\pgfqpoint{0.588738in}{1.218323in}}%
\pgfpathclose%
\pgfusepath{stroke,fill}%
\end{pgfscope}%
\begin{pgfscope}%
\pgfpathrectangle{\pgfqpoint{0.100000in}{0.212622in}}{\pgfqpoint{3.696000in}{3.696000in}}%
\pgfusepath{clip}%
\pgfsetbuttcap%
\pgfsetroundjoin%
\definecolor{currentfill}{rgb}{0.121569,0.466667,0.705882}%
\pgfsetfillcolor{currentfill}%
\pgfsetfillopacity{0.587774}%
\pgfsetlinewidth{1.003750pt}%
\definecolor{currentstroke}{rgb}{0.121569,0.466667,0.705882}%
\pgfsetstrokecolor{currentstroke}%
\pgfsetstrokeopacity{0.587774}%
\pgfsetdash{}{0pt}%
\pgfpathmoveto{\pgfqpoint{0.991587in}{1.743450in}}%
\pgfpathcurveto{\pgfqpoint{0.999823in}{1.743450in}}{\pgfqpoint{1.007723in}{1.746723in}}{\pgfqpoint{1.013547in}{1.752546in}}%
\pgfpathcurveto{\pgfqpoint{1.019371in}{1.758370in}}{\pgfqpoint{1.022644in}{1.766270in}}{\pgfqpoint{1.022644in}{1.774507in}}%
\pgfpathcurveto{\pgfqpoint{1.022644in}{1.782743in}}{\pgfqpoint{1.019371in}{1.790643in}}{\pgfqpoint{1.013547in}{1.796467in}}%
\pgfpathcurveto{\pgfqpoint{1.007723in}{1.802291in}}{\pgfqpoint{0.999823in}{1.805563in}}{\pgfqpoint{0.991587in}{1.805563in}}%
\pgfpathcurveto{\pgfqpoint{0.983351in}{1.805563in}}{\pgfqpoint{0.975451in}{1.802291in}}{\pgfqpoint{0.969627in}{1.796467in}}%
\pgfpathcurveto{\pgfqpoint{0.963803in}{1.790643in}}{\pgfqpoint{0.960531in}{1.782743in}}{\pgfqpoint{0.960531in}{1.774507in}}%
\pgfpathcurveto{\pgfqpoint{0.960531in}{1.766270in}}{\pgfqpoint{0.963803in}{1.758370in}}{\pgfqpoint{0.969627in}{1.752546in}}%
\pgfpathcurveto{\pgfqpoint{0.975451in}{1.746723in}}{\pgfqpoint{0.983351in}{1.743450in}}{\pgfqpoint{0.991587in}{1.743450in}}%
\pgfpathclose%
\pgfusepath{stroke,fill}%
\end{pgfscope}%
\begin{pgfscope}%
\pgfpathrectangle{\pgfqpoint{0.100000in}{0.212622in}}{\pgfqpoint{3.696000in}{3.696000in}}%
\pgfusepath{clip}%
\pgfsetbuttcap%
\pgfsetroundjoin%
\definecolor{currentfill}{rgb}{0.121569,0.466667,0.705882}%
\pgfsetfillcolor{currentfill}%
\pgfsetfillopacity{0.587952}%
\pgfsetlinewidth{1.003750pt}%
\definecolor{currentstroke}{rgb}{0.121569,0.466667,0.705882}%
\pgfsetstrokecolor{currentstroke}%
\pgfsetstrokeopacity{0.587952}%
\pgfsetdash{}{0pt}%
\pgfpathmoveto{\pgfqpoint{0.873148in}{1.576271in}}%
\pgfpathcurveto{\pgfqpoint{0.881384in}{1.576271in}}{\pgfqpoint{0.889284in}{1.579543in}}{\pgfqpoint{0.895108in}{1.585367in}}%
\pgfpathcurveto{\pgfqpoint{0.900932in}{1.591191in}}{\pgfqpoint{0.904205in}{1.599091in}}{\pgfqpoint{0.904205in}{1.607327in}}%
\pgfpathcurveto{\pgfqpoint{0.904205in}{1.615564in}}{\pgfqpoint{0.900932in}{1.623464in}}{\pgfqpoint{0.895108in}{1.629288in}}%
\pgfpathcurveto{\pgfqpoint{0.889284in}{1.635112in}}{\pgfqpoint{0.881384in}{1.638384in}}{\pgfqpoint{0.873148in}{1.638384in}}%
\pgfpathcurveto{\pgfqpoint{0.864912in}{1.638384in}}{\pgfqpoint{0.857012in}{1.635112in}}{\pgfqpoint{0.851188in}{1.629288in}}%
\pgfpathcurveto{\pgfqpoint{0.845364in}{1.623464in}}{\pgfqpoint{0.842092in}{1.615564in}}{\pgfqpoint{0.842092in}{1.607327in}}%
\pgfpathcurveto{\pgfqpoint{0.842092in}{1.599091in}}{\pgfqpoint{0.845364in}{1.591191in}}{\pgfqpoint{0.851188in}{1.585367in}}%
\pgfpathcurveto{\pgfqpoint{0.857012in}{1.579543in}}{\pgfqpoint{0.864912in}{1.576271in}}{\pgfqpoint{0.873148in}{1.576271in}}%
\pgfpathclose%
\pgfusepath{stroke,fill}%
\end{pgfscope}%
\begin{pgfscope}%
\pgfpathrectangle{\pgfqpoint{0.100000in}{0.212622in}}{\pgfqpoint{3.696000in}{3.696000in}}%
\pgfusepath{clip}%
\pgfsetbuttcap%
\pgfsetroundjoin%
\definecolor{currentfill}{rgb}{0.121569,0.466667,0.705882}%
\pgfsetfillcolor{currentfill}%
\pgfsetfillopacity{0.588243}%
\pgfsetlinewidth{1.003750pt}%
\definecolor{currentstroke}{rgb}{0.121569,0.466667,0.705882}%
\pgfsetstrokecolor{currentstroke}%
\pgfsetstrokeopacity{0.588243}%
\pgfsetdash{}{0pt}%
\pgfpathmoveto{\pgfqpoint{0.591323in}{1.217001in}}%
\pgfpathcurveto{\pgfqpoint{0.599559in}{1.217001in}}{\pgfqpoint{0.607459in}{1.220274in}}{\pgfqpoint{0.613283in}{1.226097in}}%
\pgfpathcurveto{\pgfqpoint{0.619107in}{1.231921in}}{\pgfqpoint{0.622379in}{1.239821in}}{\pgfqpoint{0.622379in}{1.248058in}}%
\pgfpathcurveto{\pgfqpoint{0.622379in}{1.256294in}}{\pgfqpoint{0.619107in}{1.264194in}}{\pgfqpoint{0.613283in}{1.270018in}}%
\pgfpathcurveto{\pgfqpoint{0.607459in}{1.275842in}}{\pgfqpoint{0.599559in}{1.279114in}}{\pgfqpoint{0.591323in}{1.279114in}}%
\pgfpathcurveto{\pgfqpoint{0.583086in}{1.279114in}}{\pgfqpoint{0.575186in}{1.275842in}}{\pgfqpoint{0.569362in}{1.270018in}}%
\pgfpathcurveto{\pgfqpoint{0.563538in}{1.264194in}}{\pgfqpoint{0.560266in}{1.256294in}}{\pgfqpoint{0.560266in}{1.248058in}}%
\pgfpathcurveto{\pgfqpoint{0.560266in}{1.239821in}}{\pgfqpoint{0.563538in}{1.231921in}}{\pgfqpoint{0.569362in}{1.226097in}}%
\pgfpathcurveto{\pgfqpoint{0.575186in}{1.220274in}}{\pgfqpoint{0.583086in}{1.217001in}}{\pgfqpoint{0.591323in}{1.217001in}}%
\pgfpathclose%
\pgfusepath{stroke,fill}%
\end{pgfscope}%
\begin{pgfscope}%
\pgfpathrectangle{\pgfqpoint{0.100000in}{0.212622in}}{\pgfqpoint{3.696000in}{3.696000in}}%
\pgfusepath{clip}%
\pgfsetbuttcap%
\pgfsetroundjoin%
\definecolor{currentfill}{rgb}{0.121569,0.466667,0.705882}%
\pgfsetfillcolor{currentfill}%
\pgfsetfillopacity{0.588661}%
\pgfsetlinewidth{1.003750pt}%
\definecolor{currentstroke}{rgb}{0.121569,0.466667,0.705882}%
\pgfsetstrokecolor{currentstroke}%
\pgfsetstrokeopacity{0.588661}%
\pgfsetdash{}{0pt}%
\pgfpathmoveto{\pgfqpoint{2.094011in}{2.139026in}}%
\pgfpathcurveto{\pgfqpoint{2.102247in}{2.139026in}}{\pgfqpoint{2.110147in}{2.142299in}}{\pgfqpoint{2.115971in}{2.148123in}}%
\pgfpathcurveto{\pgfqpoint{2.121795in}{2.153946in}}{\pgfqpoint{2.125067in}{2.161847in}}{\pgfqpoint{2.125067in}{2.170083in}}%
\pgfpathcurveto{\pgfqpoint{2.125067in}{2.178319in}}{\pgfqpoint{2.121795in}{2.186219in}}{\pgfqpoint{2.115971in}{2.192043in}}%
\pgfpathcurveto{\pgfqpoint{2.110147in}{2.197867in}}{\pgfqpoint{2.102247in}{2.201139in}}{\pgfqpoint{2.094011in}{2.201139in}}%
\pgfpathcurveto{\pgfqpoint{2.085775in}{2.201139in}}{\pgfqpoint{2.077875in}{2.197867in}}{\pgfqpoint{2.072051in}{2.192043in}}%
\pgfpathcurveto{\pgfqpoint{2.066227in}{2.186219in}}{\pgfqpoint{2.062954in}{2.178319in}}{\pgfqpoint{2.062954in}{2.170083in}}%
\pgfpathcurveto{\pgfqpoint{2.062954in}{2.161847in}}{\pgfqpoint{2.066227in}{2.153946in}}{\pgfqpoint{2.072051in}{2.148123in}}%
\pgfpathcurveto{\pgfqpoint{2.077875in}{2.142299in}}{\pgfqpoint{2.085775in}{2.139026in}}{\pgfqpoint{2.094011in}{2.139026in}}%
\pgfpathclose%
\pgfusepath{stroke,fill}%
\end{pgfscope}%
\begin{pgfscope}%
\pgfpathrectangle{\pgfqpoint{0.100000in}{0.212622in}}{\pgfqpoint{3.696000in}{3.696000in}}%
\pgfusepath{clip}%
\pgfsetbuttcap%
\pgfsetroundjoin%
\definecolor{currentfill}{rgb}{0.121569,0.466667,0.705882}%
\pgfsetfillcolor{currentfill}%
\pgfsetfillopacity{0.588849}%
\pgfsetlinewidth{1.003750pt}%
\definecolor{currentstroke}{rgb}{0.121569,0.466667,0.705882}%
\pgfsetstrokecolor{currentstroke}%
\pgfsetstrokeopacity{0.588849}%
\pgfsetdash{}{0pt}%
\pgfpathmoveto{\pgfqpoint{0.885606in}{1.485021in}}%
\pgfpathcurveto{\pgfqpoint{0.893842in}{1.485021in}}{\pgfqpoint{0.901743in}{1.488293in}}{\pgfqpoint{0.907566in}{1.494117in}}%
\pgfpathcurveto{\pgfqpoint{0.913390in}{1.499941in}}{\pgfqpoint{0.916663in}{1.507841in}}{\pgfqpoint{0.916663in}{1.516078in}}%
\pgfpathcurveto{\pgfqpoint{0.916663in}{1.524314in}}{\pgfqpoint{0.913390in}{1.532214in}}{\pgfqpoint{0.907566in}{1.538038in}}%
\pgfpathcurveto{\pgfqpoint{0.901743in}{1.543862in}}{\pgfqpoint{0.893842in}{1.547134in}}{\pgfqpoint{0.885606in}{1.547134in}}%
\pgfpathcurveto{\pgfqpoint{0.877370in}{1.547134in}}{\pgfqpoint{0.869470in}{1.543862in}}{\pgfqpoint{0.863646in}{1.538038in}}%
\pgfpathcurveto{\pgfqpoint{0.857822in}{1.532214in}}{\pgfqpoint{0.854550in}{1.524314in}}{\pgfqpoint{0.854550in}{1.516078in}}%
\pgfpathcurveto{\pgfqpoint{0.854550in}{1.507841in}}{\pgfqpoint{0.857822in}{1.499941in}}{\pgfqpoint{0.863646in}{1.494117in}}%
\pgfpathcurveto{\pgfqpoint{0.869470in}{1.488293in}}{\pgfqpoint{0.877370in}{1.485021in}}{\pgfqpoint{0.885606in}{1.485021in}}%
\pgfpathclose%
\pgfusepath{stroke,fill}%
\end{pgfscope}%
\begin{pgfscope}%
\pgfpathrectangle{\pgfqpoint{0.100000in}{0.212622in}}{\pgfqpoint{3.696000in}{3.696000in}}%
\pgfusepath{clip}%
\pgfsetbuttcap%
\pgfsetroundjoin%
\definecolor{currentfill}{rgb}{0.121569,0.466667,0.705882}%
\pgfsetfillcolor{currentfill}%
\pgfsetfillopacity{0.589434}%
\pgfsetlinewidth{1.003750pt}%
\definecolor{currentstroke}{rgb}{0.121569,0.466667,0.705882}%
\pgfsetstrokecolor{currentstroke}%
\pgfsetstrokeopacity{0.589434}%
\pgfsetdash{}{0pt}%
\pgfpathmoveto{\pgfqpoint{0.596309in}{1.215382in}}%
\pgfpathcurveto{\pgfqpoint{0.604545in}{1.215382in}}{\pgfqpoint{0.612445in}{1.218654in}}{\pgfqpoint{0.618269in}{1.224478in}}%
\pgfpathcurveto{\pgfqpoint{0.624093in}{1.230302in}}{\pgfqpoint{0.627365in}{1.238202in}}{\pgfqpoint{0.627365in}{1.246439in}}%
\pgfpathcurveto{\pgfqpoint{0.627365in}{1.254675in}}{\pgfqpoint{0.624093in}{1.262575in}}{\pgfqpoint{0.618269in}{1.268399in}}%
\pgfpathcurveto{\pgfqpoint{0.612445in}{1.274223in}}{\pgfqpoint{0.604545in}{1.277495in}}{\pgfqpoint{0.596309in}{1.277495in}}%
\pgfpathcurveto{\pgfqpoint{0.588073in}{1.277495in}}{\pgfqpoint{0.580172in}{1.274223in}}{\pgfqpoint{0.574349in}{1.268399in}}%
\pgfpathcurveto{\pgfqpoint{0.568525in}{1.262575in}}{\pgfqpoint{0.565252in}{1.254675in}}{\pgfqpoint{0.565252in}{1.246439in}}%
\pgfpathcurveto{\pgfqpoint{0.565252in}{1.238202in}}{\pgfqpoint{0.568525in}{1.230302in}}{\pgfqpoint{0.574349in}{1.224478in}}%
\pgfpathcurveto{\pgfqpoint{0.580172in}{1.218654in}}{\pgfqpoint{0.588073in}{1.215382in}}{\pgfqpoint{0.596309in}{1.215382in}}%
\pgfpathclose%
\pgfusepath{stroke,fill}%
\end{pgfscope}%
\begin{pgfscope}%
\pgfpathrectangle{\pgfqpoint{0.100000in}{0.212622in}}{\pgfqpoint{3.696000in}{3.696000in}}%
\pgfusepath{clip}%
\pgfsetbuttcap%
\pgfsetroundjoin%
\definecolor{currentfill}{rgb}{0.121569,0.466667,0.705882}%
\pgfsetfillcolor{currentfill}%
\pgfsetfillopacity{0.590431}%
\pgfsetlinewidth{1.003750pt}%
\definecolor{currentstroke}{rgb}{0.121569,0.466667,0.705882}%
\pgfsetstrokecolor{currentstroke}%
\pgfsetstrokeopacity{0.590431}%
\pgfsetdash{}{0pt}%
\pgfpathmoveto{\pgfqpoint{0.984221in}{1.729435in}}%
\pgfpathcurveto{\pgfqpoint{0.992458in}{1.729435in}}{\pgfqpoint{1.000358in}{1.732708in}}{\pgfqpoint{1.006182in}{1.738532in}}%
\pgfpathcurveto{\pgfqpoint{1.012005in}{1.744356in}}{\pgfqpoint{1.015278in}{1.752256in}}{\pgfqpoint{1.015278in}{1.760492in}}%
\pgfpathcurveto{\pgfqpoint{1.015278in}{1.768728in}}{\pgfqpoint{1.012005in}{1.776628in}}{\pgfqpoint{1.006182in}{1.782452in}}%
\pgfpathcurveto{\pgfqpoint{1.000358in}{1.788276in}}{\pgfqpoint{0.992458in}{1.791548in}}{\pgfqpoint{0.984221in}{1.791548in}}%
\pgfpathcurveto{\pgfqpoint{0.975985in}{1.791548in}}{\pgfqpoint{0.968085in}{1.788276in}}{\pgfqpoint{0.962261in}{1.782452in}}%
\pgfpathcurveto{\pgfqpoint{0.956437in}{1.776628in}}{\pgfqpoint{0.953165in}{1.768728in}}{\pgfqpoint{0.953165in}{1.760492in}}%
\pgfpathcurveto{\pgfqpoint{0.953165in}{1.752256in}}{\pgfqpoint{0.956437in}{1.744356in}}{\pgfqpoint{0.962261in}{1.738532in}}%
\pgfpathcurveto{\pgfqpoint{0.968085in}{1.732708in}}{\pgfqpoint{0.975985in}{1.729435in}}{\pgfqpoint{0.984221in}{1.729435in}}%
\pgfpathclose%
\pgfusepath{stroke,fill}%
\end{pgfscope}%
\begin{pgfscope}%
\pgfpathrectangle{\pgfqpoint{0.100000in}{0.212622in}}{\pgfqpoint{3.696000in}{3.696000in}}%
\pgfusepath{clip}%
\pgfsetbuttcap%
\pgfsetroundjoin%
\definecolor{currentfill}{rgb}{0.121569,0.466667,0.705882}%
\pgfsetfillcolor{currentfill}%
\pgfsetfillopacity{0.590564}%
\pgfsetlinewidth{1.003750pt}%
\definecolor{currentstroke}{rgb}{0.121569,0.466667,0.705882}%
\pgfsetstrokecolor{currentstroke}%
\pgfsetstrokeopacity{0.590564}%
\pgfsetdash{}{0pt}%
\pgfpathmoveto{\pgfqpoint{0.870639in}{1.579868in}}%
\pgfpathcurveto{\pgfqpoint{0.878875in}{1.579868in}}{\pgfqpoint{0.886775in}{1.583140in}}{\pgfqpoint{0.892599in}{1.588964in}}%
\pgfpathcurveto{\pgfqpoint{0.898423in}{1.594788in}}{\pgfqpoint{0.901696in}{1.602688in}}{\pgfqpoint{0.901696in}{1.610924in}}%
\pgfpathcurveto{\pgfqpoint{0.901696in}{1.619161in}}{\pgfqpoint{0.898423in}{1.627061in}}{\pgfqpoint{0.892599in}{1.632885in}}%
\pgfpathcurveto{\pgfqpoint{0.886775in}{1.638709in}}{\pgfqpoint{0.878875in}{1.641981in}}{\pgfqpoint{0.870639in}{1.641981in}}%
\pgfpathcurveto{\pgfqpoint{0.862403in}{1.641981in}}{\pgfqpoint{0.854503in}{1.638709in}}{\pgfqpoint{0.848679in}{1.632885in}}%
\pgfpathcurveto{\pgfqpoint{0.842855in}{1.627061in}}{\pgfqpoint{0.839583in}{1.619161in}}{\pgfqpoint{0.839583in}{1.610924in}}%
\pgfpathcurveto{\pgfqpoint{0.839583in}{1.602688in}}{\pgfqpoint{0.842855in}{1.594788in}}{\pgfqpoint{0.848679in}{1.588964in}}%
\pgfpathcurveto{\pgfqpoint{0.854503in}{1.583140in}}{\pgfqpoint{0.862403in}{1.579868in}}{\pgfqpoint{0.870639in}{1.579868in}}%
\pgfpathclose%
\pgfusepath{stroke,fill}%
\end{pgfscope}%
\begin{pgfscope}%
\pgfpathrectangle{\pgfqpoint{0.100000in}{0.212622in}}{\pgfqpoint{3.696000in}{3.696000in}}%
\pgfusepath{clip}%
\pgfsetbuttcap%
\pgfsetroundjoin%
\definecolor{currentfill}{rgb}{0.121569,0.466667,0.705882}%
\pgfsetfillcolor{currentfill}%
\pgfsetfillopacity{0.590821}%
\pgfsetlinewidth{1.003750pt}%
\definecolor{currentstroke}{rgb}{0.121569,0.466667,0.705882}%
\pgfsetstrokecolor{currentstroke}%
\pgfsetstrokeopacity{0.590821}%
\pgfsetdash{}{0pt}%
\pgfpathmoveto{\pgfqpoint{2.095052in}{2.130807in}}%
\pgfpathcurveto{\pgfqpoint{2.103289in}{2.130807in}}{\pgfqpoint{2.111189in}{2.134080in}}{\pgfqpoint{2.117013in}{2.139904in}}%
\pgfpathcurveto{\pgfqpoint{2.122837in}{2.145728in}}{\pgfqpoint{2.126109in}{2.153628in}}{\pgfqpoint{2.126109in}{2.161864in}}%
\pgfpathcurveto{\pgfqpoint{2.126109in}{2.170100in}}{\pgfqpoint{2.122837in}{2.178000in}}{\pgfqpoint{2.117013in}{2.183824in}}%
\pgfpathcurveto{\pgfqpoint{2.111189in}{2.189648in}}{\pgfqpoint{2.103289in}{2.192920in}}{\pgfqpoint{2.095052in}{2.192920in}}%
\pgfpathcurveto{\pgfqpoint{2.086816in}{2.192920in}}{\pgfqpoint{2.078916in}{2.189648in}}{\pgfqpoint{2.073092in}{2.183824in}}%
\pgfpathcurveto{\pgfqpoint{2.067268in}{2.178000in}}{\pgfqpoint{2.063996in}{2.170100in}}{\pgfqpoint{2.063996in}{2.161864in}}%
\pgfpathcurveto{\pgfqpoint{2.063996in}{2.153628in}}{\pgfqpoint{2.067268in}{2.145728in}}{\pgfqpoint{2.073092in}{2.139904in}}%
\pgfpathcurveto{\pgfqpoint{2.078916in}{2.134080in}}{\pgfqpoint{2.086816in}{2.130807in}}{\pgfqpoint{2.095052in}{2.130807in}}%
\pgfpathclose%
\pgfusepath{stroke,fill}%
\end{pgfscope}%
\begin{pgfscope}%
\pgfpathrectangle{\pgfqpoint{0.100000in}{0.212622in}}{\pgfqpoint{3.696000in}{3.696000in}}%
\pgfusepath{clip}%
\pgfsetbuttcap%
\pgfsetroundjoin%
\definecolor{currentfill}{rgb}{0.121569,0.466667,0.705882}%
\pgfsetfillcolor{currentfill}%
\pgfsetfillopacity{0.591000}%
\pgfsetlinewidth{1.003750pt}%
\definecolor{currentstroke}{rgb}{0.121569,0.466667,0.705882}%
\pgfsetstrokecolor{currentstroke}%
\pgfsetstrokeopacity{0.591000}%
\pgfsetdash{}{0pt}%
\pgfpathmoveto{\pgfqpoint{0.883340in}{1.485127in}}%
\pgfpathcurveto{\pgfqpoint{0.891576in}{1.485127in}}{\pgfqpoint{0.899476in}{1.488400in}}{\pgfqpoint{0.905300in}{1.494224in}}%
\pgfpathcurveto{\pgfqpoint{0.911124in}{1.500048in}}{\pgfqpoint{0.914396in}{1.507948in}}{\pgfqpoint{0.914396in}{1.516184in}}%
\pgfpathcurveto{\pgfqpoint{0.914396in}{1.524420in}}{\pgfqpoint{0.911124in}{1.532320in}}{\pgfqpoint{0.905300in}{1.538144in}}%
\pgfpathcurveto{\pgfqpoint{0.899476in}{1.543968in}}{\pgfqpoint{0.891576in}{1.547240in}}{\pgfqpoint{0.883340in}{1.547240in}}%
\pgfpathcurveto{\pgfqpoint{0.875103in}{1.547240in}}{\pgfqpoint{0.867203in}{1.543968in}}{\pgfqpoint{0.861380in}{1.538144in}}%
\pgfpathcurveto{\pgfqpoint{0.855556in}{1.532320in}}{\pgfqpoint{0.852283in}{1.524420in}}{\pgfqpoint{0.852283in}{1.516184in}}%
\pgfpathcurveto{\pgfqpoint{0.852283in}{1.507948in}}{\pgfqpoint{0.855556in}{1.500048in}}{\pgfqpoint{0.861380in}{1.494224in}}%
\pgfpathcurveto{\pgfqpoint{0.867203in}{1.488400in}}{\pgfqpoint{0.875103in}{1.485127in}}{\pgfqpoint{0.883340in}{1.485127in}}%
\pgfpathclose%
\pgfusepath{stroke,fill}%
\end{pgfscope}%
\begin{pgfscope}%
\pgfpathrectangle{\pgfqpoint{0.100000in}{0.212622in}}{\pgfqpoint{3.696000in}{3.696000in}}%
\pgfusepath{clip}%
\pgfsetbuttcap%
\pgfsetroundjoin%
\definecolor{currentfill}{rgb}{0.121569,0.466667,0.705882}%
\pgfsetfillcolor{currentfill}%
\pgfsetfillopacity{0.592236}%
\pgfsetlinewidth{1.003750pt}%
\definecolor{currentstroke}{rgb}{0.121569,0.466667,0.705882}%
\pgfsetstrokecolor{currentstroke}%
\pgfsetstrokeopacity{0.592236}%
\pgfsetdash{}{0pt}%
\pgfpathmoveto{\pgfqpoint{0.868177in}{1.581284in}}%
\pgfpathcurveto{\pgfqpoint{0.876413in}{1.581284in}}{\pgfqpoint{0.884313in}{1.584556in}}{\pgfqpoint{0.890137in}{1.590380in}}%
\pgfpathcurveto{\pgfqpoint{0.895961in}{1.596204in}}{\pgfqpoint{0.899233in}{1.604104in}}{\pgfqpoint{0.899233in}{1.612341in}}%
\pgfpathcurveto{\pgfqpoint{0.899233in}{1.620577in}}{\pgfqpoint{0.895961in}{1.628477in}}{\pgfqpoint{0.890137in}{1.634301in}}%
\pgfpathcurveto{\pgfqpoint{0.884313in}{1.640125in}}{\pgfqpoint{0.876413in}{1.643397in}}{\pgfqpoint{0.868177in}{1.643397in}}%
\pgfpathcurveto{\pgfqpoint{0.859940in}{1.643397in}}{\pgfqpoint{0.852040in}{1.640125in}}{\pgfqpoint{0.846216in}{1.634301in}}%
\pgfpathcurveto{\pgfqpoint{0.840392in}{1.628477in}}{\pgfqpoint{0.837120in}{1.620577in}}{\pgfqpoint{0.837120in}{1.612341in}}%
\pgfpathcurveto{\pgfqpoint{0.837120in}{1.604104in}}{\pgfqpoint{0.840392in}{1.596204in}}{\pgfqpoint{0.846216in}{1.590380in}}%
\pgfpathcurveto{\pgfqpoint{0.852040in}{1.584556in}}{\pgfqpoint{0.859940in}{1.581284in}}{\pgfqpoint{0.868177in}{1.581284in}}%
\pgfpathclose%
\pgfusepath{stroke,fill}%
\end{pgfscope}%
\begin{pgfscope}%
\pgfpathrectangle{\pgfqpoint{0.100000in}{0.212622in}}{\pgfqpoint{3.696000in}{3.696000in}}%
\pgfusepath{clip}%
\pgfsetbuttcap%
\pgfsetroundjoin%
\definecolor{currentfill}{rgb}{0.121569,0.466667,0.705882}%
\pgfsetfillcolor{currentfill}%
\pgfsetfillopacity{0.592607}%
\pgfsetlinewidth{1.003750pt}%
\definecolor{currentstroke}{rgb}{0.121569,0.466667,0.705882}%
\pgfsetstrokecolor{currentstroke}%
\pgfsetstrokeopacity{0.592607}%
\pgfsetdash{}{0pt}%
\pgfpathmoveto{\pgfqpoint{0.976654in}{1.716762in}}%
\pgfpathcurveto{\pgfqpoint{0.984890in}{1.716762in}}{\pgfqpoint{0.992790in}{1.720034in}}{\pgfqpoint{0.998614in}{1.725858in}}%
\pgfpathcurveto{\pgfqpoint{1.004438in}{1.731682in}}{\pgfqpoint{1.007710in}{1.739582in}}{\pgfqpoint{1.007710in}{1.747819in}}%
\pgfpathcurveto{\pgfqpoint{1.007710in}{1.756055in}}{\pgfqpoint{1.004438in}{1.763955in}}{\pgfqpoint{0.998614in}{1.769779in}}%
\pgfpathcurveto{\pgfqpoint{0.992790in}{1.775603in}}{\pgfqpoint{0.984890in}{1.778875in}}{\pgfqpoint{0.976654in}{1.778875in}}%
\pgfpathcurveto{\pgfqpoint{0.968417in}{1.778875in}}{\pgfqpoint{0.960517in}{1.775603in}}{\pgfqpoint{0.954693in}{1.769779in}}%
\pgfpathcurveto{\pgfqpoint{0.948870in}{1.763955in}}{\pgfqpoint{0.945597in}{1.756055in}}{\pgfqpoint{0.945597in}{1.747819in}}%
\pgfpathcurveto{\pgfqpoint{0.945597in}{1.739582in}}{\pgfqpoint{0.948870in}{1.731682in}}{\pgfqpoint{0.954693in}{1.725858in}}%
\pgfpathcurveto{\pgfqpoint{0.960517in}{1.720034in}}{\pgfqpoint{0.968417in}{1.716762in}}{\pgfqpoint{0.976654in}{1.716762in}}%
\pgfpathclose%
\pgfusepath{stroke,fill}%
\end{pgfscope}%
\begin{pgfscope}%
\pgfpathrectangle{\pgfqpoint{0.100000in}{0.212622in}}{\pgfqpoint{3.696000in}{3.696000in}}%
\pgfusepath{clip}%
\pgfsetbuttcap%
\pgfsetroundjoin%
\definecolor{currentfill}{rgb}{0.121569,0.466667,0.705882}%
\pgfsetfillcolor{currentfill}%
\pgfsetfillopacity{0.592953}%
\pgfsetlinewidth{1.003750pt}%
\definecolor{currentstroke}{rgb}{0.121569,0.466667,0.705882}%
\pgfsetstrokecolor{currentstroke}%
\pgfsetstrokeopacity{0.592953}%
\pgfsetdash{}{0pt}%
\pgfpathmoveto{\pgfqpoint{0.604651in}{1.215754in}}%
\pgfpathcurveto{\pgfqpoint{0.612887in}{1.215754in}}{\pgfqpoint{0.620787in}{1.219026in}}{\pgfqpoint{0.626611in}{1.224850in}}%
\pgfpathcurveto{\pgfqpoint{0.632435in}{1.230674in}}{\pgfqpoint{0.635707in}{1.238574in}}{\pgfqpoint{0.635707in}{1.246810in}}%
\pgfpathcurveto{\pgfqpoint{0.635707in}{1.255047in}}{\pgfqpoint{0.632435in}{1.262947in}}{\pgfqpoint{0.626611in}{1.268771in}}%
\pgfpathcurveto{\pgfqpoint{0.620787in}{1.274595in}}{\pgfqpoint{0.612887in}{1.277867in}}{\pgfqpoint{0.604651in}{1.277867in}}%
\pgfpathcurveto{\pgfqpoint{0.596414in}{1.277867in}}{\pgfqpoint{0.588514in}{1.274595in}}{\pgfqpoint{0.582690in}{1.268771in}}%
\pgfpathcurveto{\pgfqpoint{0.576866in}{1.262947in}}{\pgfqpoint{0.573594in}{1.255047in}}{\pgfqpoint{0.573594in}{1.246810in}}%
\pgfpathcurveto{\pgfqpoint{0.573594in}{1.238574in}}{\pgfqpoint{0.576866in}{1.230674in}}{\pgfqpoint{0.582690in}{1.224850in}}%
\pgfpathcurveto{\pgfqpoint{0.588514in}{1.219026in}}{\pgfqpoint{0.596414in}{1.215754in}}{\pgfqpoint{0.604651in}{1.215754in}}%
\pgfpathclose%
\pgfusepath{stroke,fill}%
\end{pgfscope}%
\begin{pgfscope}%
\pgfpathrectangle{\pgfqpoint{0.100000in}{0.212622in}}{\pgfqpoint{3.696000in}{3.696000in}}%
\pgfusepath{clip}%
\pgfsetbuttcap%
\pgfsetroundjoin%
\definecolor{currentfill}{rgb}{0.121569,0.466667,0.705882}%
\pgfsetfillcolor{currentfill}%
\pgfsetfillopacity{0.593162}%
\pgfsetlinewidth{1.003750pt}%
\definecolor{currentstroke}{rgb}{0.121569,0.466667,0.705882}%
\pgfsetstrokecolor{currentstroke}%
\pgfsetstrokeopacity{0.593162}%
\pgfsetdash{}{0pt}%
\pgfpathmoveto{\pgfqpoint{0.866388in}{1.581450in}}%
\pgfpathcurveto{\pgfqpoint{0.874624in}{1.581450in}}{\pgfqpoint{0.882524in}{1.584723in}}{\pgfqpoint{0.888348in}{1.590547in}}%
\pgfpathcurveto{\pgfqpoint{0.894172in}{1.596371in}}{\pgfqpoint{0.897444in}{1.604271in}}{\pgfqpoint{0.897444in}{1.612507in}}%
\pgfpathcurveto{\pgfqpoint{0.897444in}{1.620743in}}{\pgfqpoint{0.894172in}{1.628643in}}{\pgfqpoint{0.888348in}{1.634467in}}%
\pgfpathcurveto{\pgfqpoint{0.882524in}{1.640291in}}{\pgfqpoint{0.874624in}{1.643563in}}{\pgfqpoint{0.866388in}{1.643563in}}%
\pgfpathcurveto{\pgfqpoint{0.858152in}{1.643563in}}{\pgfqpoint{0.850252in}{1.640291in}}{\pgfqpoint{0.844428in}{1.634467in}}%
\pgfpathcurveto{\pgfqpoint{0.838604in}{1.628643in}}{\pgfqpoint{0.835331in}{1.620743in}}{\pgfqpoint{0.835331in}{1.612507in}}%
\pgfpathcurveto{\pgfqpoint{0.835331in}{1.604271in}}{\pgfqpoint{0.838604in}{1.596371in}}{\pgfqpoint{0.844428in}{1.590547in}}%
\pgfpathcurveto{\pgfqpoint{0.850252in}{1.584723in}}{\pgfqpoint{0.858152in}{1.581450in}}{\pgfqpoint{0.866388in}{1.581450in}}%
\pgfpathclose%
\pgfusepath{stroke,fill}%
\end{pgfscope}%
\begin{pgfscope}%
\pgfpathrectangle{\pgfqpoint{0.100000in}{0.212622in}}{\pgfqpoint{3.696000in}{3.696000in}}%
\pgfusepath{clip}%
\pgfsetbuttcap%
\pgfsetroundjoin%
\definecolor{currentfill}{rgb}{0.121569,0.466667,0.705882}%
\pgfsetfillcolor{currentfill}%
\pgfsetfillopacity{0.593166}%
\pgfsetlinewidth{1.003750pt}%
\definecolor{currentstroke}{rgb}{0.121569,0.466667,0.705882}%
\pgfsetstrokecolor{currentstroke}%
\pgfsetstrokeopacity{0.593166}%
\pgfsetdash{}{0pt}%
\pgfpathmoveto{\pgfqpoint{2.097278in}{2.120543in}}%
\pgfpathcurveto{\pgfqpoint{2.105514in}{2.120543in}}{\pgfqpoint{2.113414in}{2.123816in}}{\pgfqpoint{2.119238in}{2.129640in}}%
\pgfpathcurveto{\pgfqpoint{2.125062in}{2.135463in}}{\pgfqpoint{2.128335in}{2.143364in}}{\pgfqpoint{2.128335in}{2.151600in}}%
\pgfpathcurveto{\pgfqpoint{2.128335in}{2.159836in}}{\pgfqpoint{2.125062in}{2.167736in}}{\pgfqpoint{2.119238in}{2.173560in}}%
\pgfpathcurveto{\pgfqpoint{2.113414in}{2.179384in}}{\pgfqpoint{2.105514in}{2.182656in}}{\pgfqpoint{2.097278in}{2.182656in}}%
\pgfpathcurveto{\pgfqpoint{2.089042in}{2.182656in}}{\pgfqpoint{2.081142in}{2.179384in}}{\pgfqpoint{2.075318in}{2.173560in}}%
\pgfpathcurveto{\pgfqpoint{2.069494in}{2.167736in}}{\pgfqpoint{2.066222in}{2.159836in}}{\pgfqpoint{2.066222in}{2.151600in}}%
\pgfpathcurveto{\pgfqpoint{2.066222in}{2.143364in}}{\pgfqpoint{2.069494in}{2.135463in}}{\pgfqpoint{2.075318in}{2.129640in}}%
\pgfpathcurveto{\pgfqpoint{2.081142in}{2.123816in}}{\pgfqpoint{2.089042in}{2.120543in}}{\pgfqpoint{2.097278in}{2.120543in}}%
\pgfpathclose%
\pgfusepath{stroke,fill}%
\end{pgfscope}%
\begin{pgfscope}%
\pgfpathrectangle{\pgfqpoint{0.100000in}{0.212622in}}{\pgfqpoint{3.696000in}{3.696000in}}%
\pgfusepath{clip}%
\pgfsetbuttcap%
\pgfsetroundjoin%
\definecolor{currentfill}{rgb}{0.121569,0.466667,0.705882}%
\pgfsetfillcolor{currentfill}%
\pgfsetfillopacity{0.593795}%
\pgfsetlinewidth{1.003750pt}%
\definecolor{currentstroke}{rgb}{0.121569,0.466667,0.705882}%
\pgfsetstrokecolor{currentstroke}%
\pgfsetstrokeopacity{0.593795}%
\pgfsetdash{}{0pt}%
\pgfpathmoveto{\pgfqpoint{0.864999in}{1.580955in}}%
\pgfpathcurveto{\pgfqpoint{0.873235in}{1.580955in}}{\pgfqpoint{0.881135in}{1.584228in}}{\pgfqpoint{0.886959in}{1.590052in}}%
\pgfpathcurveto{\pgfqpoint{0.892783in}{1.595875in}}{\pgfqpoint{0.896055in}{1.603776in}}{\pgfqpoint{0.896055in}{1.612012in}}%
\pgfpathcurveto{\pgfqpoint{0.896055in}{1.620248in}}{\pgfqpoint{0.892783in}{1.628148in}}{\pgfqpoint{0.886959in}{1.633972in}}%
\pgfpathcurveto{\pgfqpoint{0.881135in}{1.639796in}}{\pgfqpoint{0.873235in}{1.643068in}}{\pgfqpoint{0.864999in}{1.643068in}}%
\pgfpathcurveto{\pgfqpoint{0.856763in}{1.643068in}}{\pgfqpoint{0.848863in}{1.639796in}}{\pgfqpoint{0.843039in}{1.633972in}}%
\pgfpathcurveto{\pgfqpoint{0.837215in}{1.628148in}}{\pgfqpoint{0.833942in}{1.620248in}}{\pgfqpoint{0.833942in}{1.612012in}}%
\pgfpathcurveto{\pgfqpoint{0.833942in}{1.603776in}}{\pgfqpoint{0.837215in}{1.595875in}}{\pgfqpoint{0.843039in}{1.590052in}}%
\pgfpathcurveto{\pgfqpoint{0.848863in}{1.584228in}}{\pgfqpoint{0.856763in}{1.580955in}}{\pgfqpoint{0.864999in}{1.580955in}}%
\pgfpathclose%
\pgfusepath{stroke,fill}%
\end{pgfscope}%
\begin{pgfscope}%
\pgfpathrectangle{\pgfqpoint{0.100000in}{0.212622in}}{\pgfqpoint{3.696000in}{3.696000in}}%
\pgfusepath{clip}%
\pgfsetbuttcap%
\pgfsetroundjoin%
\definecolor{currentfill}{rgb}{0.121569,0.466667,0.705882}%
\pgfsetfillcolor{currentfill}%
\pgfsetfillopacity{0.593807}%
\pgfsetlinewidth{1.003750pt}%
\definecolor{currentstroke}{rgb}{0.121569,0.466667,0.705882}%
\pgfsetstrokecolor{currentstroke}%
\pgfsetstrokeopacity{0.593807}%
\pgfsetdash{}{0pt}%
\pgfpathmoveto{\pgfqpoint{0.880557in}{1.485399in}}%
\pgfpathcurveto{\pgfqpoint{0.888793in}{1.485399in}}{\pgfqpoint{0.896693in}{1.488672in}}{\pgfqpoint{0.902517in}{1.494496in}}%
\pgfpathcurveto{\pgfqpoint{0.908341in}{1.500319in}}{\pgfqpoint{0.911614in}{1.508220in}}{\pgfqpoint{0.911614in}{1.516456in}}%
\pgfpathcurveto{\pgfqpoint{0.911614in}{1.524692in}}{\pgfqpoint{0.908341in}{1.532592in}}{\pgfqpoint{0.902517in}{1.538416in}}%
\pgfpathcurveto{\pgfqpoint{0.896693in}{1.544240in}}{\pgfqpoint{0.888793in}{1.547512in}}{\pgfqpoint{0.880557in}{1.547512in}}%
\pgfpathcurveto{\pgfqpoint{0.872321in}{1.547512in}}{\pgfqpoint{0.864421in}{1.544240in}}{\pgfqpoint{0.858597in}{1.538416in}}%
\pgfpathcurveto{\pgfqpoint{0.852773in}{1.532592in}}{\pgfqpoint{0.849501in}{1.524692in}}{\pgfqpoint{0.849501in}{1.516456in}}%
\pgfpathcurveto{\pgfqpoint{0.849501in}{1.508220in}}{\pgfqpoint{0.852773in}{1.500319in}}{\pgfqpoint{0.858597in}{1.494496in}}%
\pgfpathcurveto{\pgfqpoint{0.864421in}{1.488672in}}{\pgfqpoint{0.872321in}{1.485399in}}{\pgfqpoint{0.880557in}{1.485399in}}%
\pgfpathclose%
\pgfusepath{stroke,fill}%
\end{pgfscope}%
\begin{pgfscope}%
\pgfpathrectangle{\pgfqpoint{0.100000in}{0.212622in}}{\pgfqpoint{3.696000in}{3.696000in}}%
\pgfusepath{clip}%
\pgfsetbuttcap%
\pgfsetroundjoin%
\definecolor{currentfill}{rgb}{0.121569,0.466667,0.705882}%
\pgfsetfillcolor{currentfill}%
\pgfsetfillopacity{0.594461}%
\pgfsetlinewidth{1.003750pt}%
\definecolor{currentstroke}{rgb}{0.121569,0.466667,0.705882}%
\pgfsetstrokecolor{currentstroke}%
\pgfsetstrokeopacity{0.594461}%
\pgfsetdash{}{0pt}%
\pgfpathmoveto{\pgfqpoint{2.098335in}{2.114744in}}%
\pgfpathcurveto{\pgfqpoint{2.106572in}{2.114744in}}{\pgfqpoint{2.114472in}{2.118016in}}{\pgfqpoint{2.120296in}{2.123840in}}%
\pgfpathcurveto{\pgfqpoint{2.126120in}{2.129664in}}{\pgfqpoint{2.129392in}{2.137564in}}{\pgfqpoint{2.129392in}{2.145800in}}%
\pgfpathcurveto{\pgfqpoint{2.129392in}{2.154036in}}{\pgfqpoint{2.126120in}{2.161936in}}{\pgfqpoint{2.120296in}{2.167760in}}%
\pgfpathcurveto{\pgfqpoint{2.114472in}{2.173584in}}{\pgfqpoint{2.106572in}{2.176857in}}{\pgfqpoint{2.098335in}{2.176857in}}%
\pgfpathcurveto{\pgfqpoint{2.090099in}{2.176857in}}{\pgfqpoint{2.082199in}{2.173584in}}{\pgfqpoint{2.076375in}{2.167760in}}%
\pgfpathcurveto{\pgfqpoint{2.070551in}{2.161936in}}{\pgfqpoint{2.067279in}{2.154036in}}{\pgfqpoint{2.067279in}{2.145800in}}%
\pgfpathcurveto{\pgfqpoint{2.067279in}{2.137564in}}{\pgfqpoint{2.070551in}{2.129664in}}{\pgfqpoint{2.076375in}{2.123840in}}%
\pgfpathcurveto{\pgfqpoint{2.082199in}{2.118016in}}{\pgfqpoint{2.090099in}{2.114744in}}{\pgfqpoint{2.098335in}{2.114744in}}%
\pgfpathclose%
\pgfusepath{stroke,fill}%
\end{pgfscope}%
\begin{pgfscope}%
\pgfpathrectangle{\pgfqpoint{0.100000in}{0.212622in}}{\pgfqpoint{3.696000in}{3.696000in}}%
\pgfusepath{clip}%
\pgfsetbuttcap%
\pgfsetroundjoin%
\definecolor{currentfill}{rgb}{0.121569,0.466667,0.705882}%
\pgfsetfillcolor{currentfill}%
\pgfsetfillopacity{0.594764}%
\pgfsetlinewidth{1.003750pt}%
\definecolor{currentstroke}{rgb}{0.121569,0.466667,0.705882}%
\pgfsetstrokecolor{currentstroke}%
\pgfsetstrokeopacity{0.594764}%
\pgfsetdash{}{0pt}%
\pgfpathmoveto{\pgfqpoint{0.970980in}{1.704764in}}%
\pgfpathcurveto{\pgfqpoint{0.979216in}{1.704764in}}{\pgfqpoint{0.987116in}{1.708036in}}{\pgfqpoint{0.992940in}{1.713860in}}%
\pgfpathcurveto{\pgfqpoint{0.998764in}{1.719684in}}{\pgfqpoint{1.002036in}{1.727584in}}{\pgfqpoint{1.002036in}{1.735821in}}%
\pgfpathcurveto{\pgfqpoint{1.002036in}{1.744057in}}{\pgfqpoint{0.998764in}{1.751957in}}{\pgfqpoint{0.992940in}{1.757781in}}%
\pgfpathcurveto{\pgfqpoint{0.987116in}{1.763605in}}{\pgfqpoint{0.979216in}{1.766877in}}{\pgfqpoint{0.970980in}{1.766877in}}%
\pgfpathcurveto{\pgfqpoint{0.962744in}{1.766877in}}{\pgfqpoint{0.954844in}{1.763605in}}{\pgfqpoint{0.949020in}{1.757781in}}%
\pgfpathcurveto{\pgfqpoint{0.943196in}{1.751957in}}{\pgfqpoint{0.939923in}{1.744057in}}{\pgfqpoint{0.939923in}{1.735821in}}%
\pgfpathcurveto{\pgfqpoint{0.939923in}{1.727584in}}{\pgfqpoint{0.943196in}{1.719684in}}{\pgfqpoint{0.949020in}{1.713860in}}%
\pgfpathcurveto{\pgfqpoint{0.954844in}{1.708036in}}{\pgfqpoint{0.962744in}{1.704764in}}{\pgfqpoint{0.970980in}{1.704764in}}%
\pgfpathclose%
\pgfusepath{stroke,fill}%
\end{pgfscope}%
\begin{pgfscope}%
\pgfpathrectangle{\pgfqpoint{0.100000in}{0.212622in}}{\pgfqpoint{3.696000in}{3.696000in}}%
\pgfusepath{clip}%
\pgfsetbuttcap%
\pgfsetroundjoin%
\definecolor{currentfill}{rgb}{0.121569,0.466667,0.705882}%
\pgfsetfillcolor{currentfill}%
\pgfsetfillopacity{0.596202}%
\pgfsetlinewidth{1.003750pt}%
\definecolor{currentstroke}{rgb}{0.121569,0.466667,0.705882}%
\pgfsetstrokecolor{currentstroke}%
\pgfsetstrokeopacity{0.596202}%
\pgfsetdash{}{0pt}%
\pgfpathmoveto{\pgfqpoint{2.099084in}{2.108186in}}%
\pgfpathcurveto{\pgfqpoint{2.107321in}{2.108186in}}{\pgfqpoint{2.115221in}{2.111458in}}{\pgfqpoint{2.121044in}{2.117282in}}%
\pgfpathcurveto{\pgfqpoint{2.126868in}{2.123106in}}{\pgfqpoint{2.130141in}{2.131006in}}{\pgfqpoint{2.130141in}{2.139242in}}%
\pgfpathcurveto{\pgfqpoint{2.130141in}{2.147478in}}{\pgfqpoint{2.126868in}{2.155378in}}{\pgfqpoint{2.121044in}{2.161202in}}%
\pgfpathcurveto{\pgfqpoint{2.115221in}{2.167026in}}{\pgfqpoint{2.107321in}{2.170299in}}{\pgfqpoint{2.099084in}{2.170299in}}%
\pgfpathcurveto{\pgfqpoint{2.090848in}{2.170299in}}{\pgfqpoint{2.082948in}{2.167026in}}{\pgfqpoint{2.077124in}{2.161202in}}%
\pgfpathcurveto{\pgfqpoint{2.071300in}{2.155378in}}{\pgfqpoint{2.068028in}{2.147478in}}{\pgfqpoint{2.068028in}{2.139242in}}%
\pgfpathcurveto{\pgfqpoint{2.068028in}{2.131006in}}{\pgfqpoint{2.071300in}{2.123106in}}{\pgfqpoint{2.077124in}{2.117282in}}%
\pgfpathcurveto{\pgfqpoint{2.082948in}{2.111458in}}{\pgfqpoint{2.090848in}{2.108186in}}{\pgfqpoint{2.099084in}{2.108186in}}%
\pgfpathclose%
\pgfusepath{stroke,fill}%
\end{pgfscope}%
\begin{pgfscope}%
\pgfpathrectangle{\pgfqpoint{0.100000in}{0.212622in}}{\pgfqpoint{3.696000in}{3.696000in}}%
\pgfusepath{clip}%
\pgfsetbuttcap%
\pgfsetroundjoin%
\definecolor{currentfill}{rgb}{0.121569,0.466667,0.705882}%
\pgfsetfillcolor{currentfill}%
\pgfsetfillopacity{0.596318}%
\pgfsetlinewidth{1.003750pt}%
\definecolor{currentstroke}{rgb}{0.121569,0.466667,0.705882}%
\pgfsetstrokecolor{currentstroke}%
\pgfsetstrokeopacity{0.596318}%
\pgfsetdash{}{0pt}%
\pgfpathmoveto{\pgfqpoint{0.965404in}{1.695082in}}%
\pgfpathcurveto{\pgfqpoint{0.973641in}{1.695082in}}{\pgfqpoint{0.981541in}{1.698354in}}{\pgfqpoint{0.987365in}{1.704178in}}%
\pgfpathcurveto{\pgfqpoint{0.993189in}{1.710002in}}{\pgfqpoint{0.996461in}{1.717902in}}{\pgfqpoint{0.996461in}{1.726138in}}%
\pgfpathcurveto{\pgfqpoint{0.996461in}{1.734375in}}{\pgfqpoint{0.993189in}{1.742275in}}{\pgfqpoint{0.987365in}{1.748099in}}%
\pgfpathcurveto{\pgfqpoint{0.981541in}{1.753922in}}{\pgfqpoint{0.973641in}{1.757195in}}{\pgfqpoint{0.965404in}{1.757195in}}%
\pgfpathcurveto{\pgfqpoint{0.957168in}{1.757195in}}{\pgfqpoint{0.949268in}{1.753922in}}{\pgfqpoint{0.943444in}{1.748099in}}%
\pgfpathcurveto{\pgfqpoint{0.937620in}{1.742275in}}{\pgfqpoint{0.934348in}{1.734375in}}{\pgfqpoint{0.934348in}{1.726138in}}%
\pgfpathcurveto{\pgfqpoint{0.934348in}{1.717902in}}{\pgfqpoint{0.937620in}{1.710002in}}{\pgfqpoint{0.943444in}{1.704178in}}%
\pgfpathcurveto{\pgfqpoint{0.949268in}{1.698354in}}{\pgfqpoint{0.957168in}{1.695082in}}{\pgfqpoint{0.965404in}{1.695082in}}%
\pgfpathclose%
\pgfusepath{stroke,fill}%
\end{pgfscope}%
\begin{pgfscope}%
\pgfpathrectangle{\pgfqpoint{0.100000in}{0.212622in}}{\pgfqpoint{3.696000in}{3.696000in}}%
\pgfusepath{clip}%
\pgfsetbuttcap%
\pgfsetroundjoin%
\definecolor{currentfill}{rgb}{0.121569,0.466667,0.705882}%
\pgfsetfillcolor{currentfill}%
\pgfsetfillopacity{0.597441}%
\pgfsetlinewidth{1.003750pt}%
\definecolor{currentstroke}{rgb}{0.121569,0.466667,0.705882}%
\pgfsetstrokecolor{currentstroke}%
\pgfsetstrokeopacity{0.597441}%
\pgfsetdash{}{0pt}%
\pgfpathmoveto{\pgfqpoint{0.620771in}{1.211398in}}%
\pgfpathcurveto{\pgfqpoint{0.629007in}{1.211398in}}{\pgfqpoint{0.636907in}{1.214670in}}{\pgfqpoint{0.642731in}{1.220494in}}%
\pgfpathcurveto{\pgfqpoint{0.648555in}{1.226318in}}{\pgfqpoint{0.651827in}{1.234218in}}{\pgfqpoint{0.651827in}{1.242454in}}%
\pgfpathcurveto{\pgfqpoint{0.651827in}{1.250691in}}{\pgfqpoint{0.648555in}{1.258591in}}{\pgfqpoint{0.642731in}{1.264415in}}%
\pgfpathcurveto{\pgfqpoint{0.636907in}{1.270239in}}{\pgfqpoint{0.629007in}{1.273511in}}{\pgfqpoint{0.620771in}{1.273511in}}%
\pgfpathcurveto{\pgfqpoint{0.612535in}{1.273511in}}{\pgfqpoint{0.604635in}{1.270239in}}{\pgfqpoint{0.598811in}{1.264415in}}%
\pgfpathcurveto{\pgfqpoint{0.592987in}{1.258591in}}{\pgfqpoint{0.589714in}{1.250691in}}{\pgfqpoint{0.589714in}{1.242454in}}%
\pgfpathcurveto{\pgfqpoint{0.589714in}{1.234218in}}{\pgfqpoint{0.592987in}{1.226318in}}{\pgfqpoint{0.598811in}{1.220494in}}%
\pgfpathcurveto{\pgfqpoint{0.604635in}{1.214670in}}{\pgfqpoint{0.612535in}{1.211398in}}{\pgfqpoint{0.620771in}{1.211398in}}%
\pgfpathclose%
\pgfusepath{stroke,fill}%
\end{pgfscope}%
\begin{pgfscope}%
\pgfpathrectangle{\pgfqpoint{0.100000in}{0.212622in}}{\pgfqpoint{3.696000in}{3.696000in}}%
\pgfusepath{clip}%
\pgfsetbuttcap%
\pgfsetroundjoin%
\definecolor{currentfill}{rgb}{0.121569,0.466667,0.705882}%
\pgfsetfillcolor{currentfill}%
\pgfsetfillopacity{0.597478}%
\pgfsetlinewidth{1.003750pt}%
\definecolor{currentstroke}{rgb}{0.121569,0.466667,0.705882}%
\pgfsetstrokecolor{currentstroke}%
\pgfsetstrokeopacity{0.597478}%
\pgfsetdash{}{0pt}%
\pgfpathmoveto{\pgfqpoint{0.877299in}{1.485991in}}%
\pgfpathcurveto{\pgfqpoint{0.885536in}{1.485991in}}{\pgfqpoint{0.893436in}{1.489263in}}{\pgfqpoint{0.899260in}{1.495087in}}%
\pgfpathcurveto{\pgfqpoint{0.905084in}{1.500911in}}{\pgfqpoint{0.908356in}{1.508811in}}{\pgfqpoint{0.908356in}{1.517047in}}%
\pgfpathcurveto{\pgfqpoint{0.908356in}{1.525284in}}{\pgfqpoint{0.905084in}{1.533184in}}{\pgfqpoint{0.899260in}{1.539008in}}%
\pgfpathcurveto{\pgfqpoint{0.893436in}{1.544832in}}{\pgfqpoint{0.885536in}{1.548104in}}{\pgfqpoint{0.877299in}{1.548104in}}%
\pgfpathcurveto{\pgfqpoint{0.869063in}{1.548104in}}{\pgfqpoint{0.861163in}{1.544832in}}{\pgfqpoint{0.855339in}{1.539008in}}%
\pgfpathcurveto{\pgfqpoint{0.849515in}{1.533184in}}{\pgfqpoint{0.846243in}{1.525284in}}{\pgfqpoint{0.846243in}{1.517047in}}%
\pgfpathcurveto{\pgfqpoint{0.846243in}{1.508811in}}{\pgfqpoint{0.849515in}{1.500911in}}{\pgfqpoint{0.855339in}{1.495087in}}%
\pgfpathcurveto{\pgfqpoint{0.861163in}{1.489263in}}{\pgfqpoint{0.869063in}{1.485991in}}{\pgfqpoint{0.877299in}{1.485991in}}%
\pgfpathclose%
\pgfusepath{stroke,fill}%
\end{pgfscope}%
\begin{pgfscope}%
\pgfpathrectangle{\pgfqpoint{0.100000in}{0.212622in}}{\pgfqpoint{3.696000in}{3.696000in}}%
\pgfusepath{clip}%
\pgfsetbuttcap%
\pgfsetroundjoin%
\definecolor{currentfill}{rgb}{0.121569,0.466667,0.705882}%
\pgfsetfillcolor{currentfill}%
\pgfsetfillopacity{0.597742}%
\pgfsetlinewidth{1.003750pt}%
\definecolor{currentstroke}{rgb}{0.121569,0.466667,0.705882}%
\pgfsetstrokecolor{currentstroke}%
\pgfsetstrokeopacity{0.597742}%
\pgfsetdash{}{0pt}%
\pgfpathmoveto{\pgfqpoint{0.961284in}{1.686511in}}%
\pgfpathcurveto{\pgfqpoint{0.969520in}{1.686511in}}{\pgfqpoint{0.977420in}{1.689784in}}{\pgfqpoint{0.983244in}{1.695607in}}%
\pgfpathcurveto{\pgfqpoint{0.989068in}{1.701431in}}{\pgfqpoint{0.992340in}{1.709331in}}{\pgfqpoint{0.992340in}{1.717568in}}%
\pgfpathcurveto{\pgfqpoint{0.992340in}{1.725804in}}{\pgfqpoint{0.989068in}{1.733704in}}{\pgfqpoint{0.983244in}{1.739528in}}%
\pgfpathcurveto{\pgfqpoint{0.977420in}{1.745352in}}{\pgfqpoint{0.969520in}{1.748624in}}{\pgfqpoint{0.961284in}{1.748624in}}%
\pgfpathcurveto{\pgfqpoint{0.953047in}{1.748624in}}{\pgfqpoint{0.945147in}{1.745352in}}{\pgfqpoint{0.939323in}{1.739528in}}%
\pgfpathcurveto{\pgfqpoint{0.933499in}{1.733704in}}{\pgfqpoint{0.930227in}{1.725804in}}{\pgfqpoint{0.930227in}{1.717568in}}%
\pgfpathcurveto{\pgfqpoint{0.930227in}{1.709331in}}{\pgfqpoint{0.933499in}{1.701431in}}{\pgfqpoint{0.939323in}{1.695607in}}%
\pgfpathcurveto{\pgfqpoint{0.945147in}{1.689784in}}{\pgfqpoint{0.953047in}{1.686511in}}{\pgfqpoint{0.961284in}{1.686511in}}%
\pgfpathclose%
\pgfusepath{stroke,fill}%
\end{pgfscope}%
\begin{pgfscope}%
\pgfpathrectangle{\pgfqpoint{0.100000in}{0.212622in}}{\pgfqpoint{3.696000in}{3.696000in}}%
\pgfusepath{clip}%
\pgfsetbuttcap%
\pgfsetroundjoin%
\definecolor{currentfill}{rgb}{0.121569,0.466667,0.705882}%
\pgfsetfillcolor{currentfill}%
\pgfsetfillopacity{0.598465}%
\pgfsetlinewidth{1.003750pt}%
\definecolor{currentstroke}{rgb}{0.121569,0.466667,0.705882}%
\pgfsetstrokecolor{currentstroke}%
\pgfsetstrokeopacity{0.598465}%
\pgfsetdash{}{0pt}%
\pgfpathmoveto{\pgfqpoint{2.101109in}{2.098783in}}%
\pgfpathcurveto{\pgfqpoint{2.109345in}{2.098783in}}{\pgfqpoint{2.117245in}{2.102055in}}{\pgfqpoint{2.123069in}{2.107879in}}%
\pgfpathcurveto{\pgfqpoint{2.128893in}{2.113703in}}{\pgfqpoint{2.132165in}{2.121603in}}{\pgfqpoint{2.132165in}{2.129839in}}%
\pgfpathcurveto{\pgfqpoint{2.132165in}{2.138075in}}{\pgfqpoint{2.128893in}{2.145976in}}{\pgfqpoint{2.123069in}{2.151799in}}%
\pgfpathcurveto{\pgfqpoint{2.117245in}{2.157623in}}{\pgfqpoint{2.109345in}{2.160896in}}{\pgfqpoint{2.101109in}{2.160896in}}%
\pgfpathcurveto{\pgfqpoint{2.092873in}{2.160896in}}{\pgfqpoint{2.084973in}{2.157623in}}{\pgfqpoint{2.079149in}{2.151799in}}%
\pgfpathcurveto{\pgfqpoint{2.073325in}{2.145976in}}{\pgfqpoint{2.070052in}{2.138075in}}{\pgfqpoint{2.070052in}{2.129839in}}%
\pgfpathcurveto{\pgfqpoint{2.070052in}{2.121603in}}{\pgfqpoint{2.073325in}{2.113703in}}{\pgfqpoint{2.079149in}{2.107879in}}%
\pgfpathcurveto{\pgfqpoint{2.084973in}{2.102055in}}{\pgfqpoint{2.092873in}{2.098783in}}{\pgfqpoint{2.101109in}{2.098783in}}%
\pgfpathclose%
\pgfusepath{stroke,fill}%
\end{pgfscope}%
\begin{pgfscope}%
\pgfpathrectangle{\pgfqpoint{0.100000in}{0.212622in}}{\pgfqpoint{3.696000in}{3.696000in}}%
\pgfusepath{clip}%
\pgfsetbuttcap%
\pgfsetroundjoin%
\definecolor{currentfill}{rgb}{0.121569,0.466667,0.705882}%
\pgfsetfillcolor{currentfill}%
\pgfsetfillopacity{0.598632}%
\pgfsetlinewidth{1.003750pt}%
\definecolor{currentstroke}{rgb}{0.121569,0.466667,0.705882}%
\pgfsetstrokecolor{currentstroke}%
\pgfsetstrokeopacity{0.598632}%
\pgfsetdash{}{0pt}%
\pgfpathmoveto{\pgfqpoint{0.957958in}{1.680535in}}%
\pgfpathcurveto{\pgfqpoint{0.966195in}{1.680535in}}{\pgfqpoint{0.974095in}{1.683807in}}{\pgfqpoint{0.979919in}{1.689631in}}%
\pgfpathcurveto{\pgfqpoint{0.985742in}{1.695455in}}{\pgfqpoint{0.989015in}{1.703355in}}{\pgfqpoint{0.989015in}{1.711592in}}%
\pgfpathcurveto{\pgfqpoint{0.989015in}{1.719828in}}{\pgfqpoint{0.985742in}{1.727728in}}{\pgfqpoint{0.979919in}{1.733552in}}%
\pgfpathcurveto{\pgfqpoint{0.974095in}{1.739376in}}{\pgfqpoint{0.966195in}{1.742648in}}{\pgfqpoint{0.957958in}{1.742648in}}%
\pgfpathcurveto{\pgfqpoint{0.949722in}{1.742648in}}{\pgfqpoint{0.941822in}{1.739376in}}{\pgfqpoint{0.935998in}{1.733552in}}%
\pgfpathcurveto{\pgfqpoint{0.930174in}{1.727728in}}{\pgfqpoint{0.926902in}{1.719828in}}{\pgfqpoint{0.926902in}{1.711592in}}%
\pgfpathcurveto{\pgfqpoint{0.926902in}{1.703355in}}{\pgfqpoint{0.930174in}{1.695455in}}{\pgfqpoint{0.935998in}{1.689631in}}%
\pgfpathcurveto{\pgfqpoint{0.941822in}{1.683807in}}{\pgfqpoint{0.949722in}{1.680535in}}{\pgfqpoint{0.957958in}{1.680535in}}%
\pgfpathclose%
\pgfusepath{stroke,fill}%
\end{pgfscope}%
\begin{pgfscope}%
\pgfpathrectangle{\pgfqpoint{0.100000in}{0.212622in}}{\pgfqpoint{3.696000in}{3.696000in}}%
\pgfusepath{clip}%
\pgfsetbuttcap%
\pgfsetroundjoin%
\definecolor{currentfill}{rgb}{0.121569,0.466667,0.705882}%
\pgfsetfillcolor{currentfill}%
\pgfsetfillopacity{0.599384}%
\pgfsetlinewidth{1.003750pt}%
\definecolor{currentstroke}{rgb}{0.121569,0.466667,0.705882}%
\pgfsetstrokecolor{currentstroke}%
\pgfsetstrokeopacity{0.599384}%
\pgfsetdash{}{0pt}%
\pgfpathmoveto{\pgfqpoint{0.955971in}{1.675791in}}%
\pgfpathcurveto{\pgfqpoint{0.964207in}{1.675791in}}{\pgfqpoint{0.972107in}{1.679063in}}{\pgfqpoint{0.977931in}{1.684887in}}%
\pgfpathcurveto{\pgfqpoint{0.983755in}{1.690711in}}{\pgfqpoint{0.987028in}{1.698611in}}{\pgfqpoint{0.987028in}{1.706848in}}%
\pgfpathcurveto{\pgfqpoint{0.987028in}{1.715084in}}{\pgfqpoint{0.983755in}{1.722984in}}{\pgfqpoint{0.977931in}{1.728808in}}%
\pgfpathcurveto{\pgfqpoint{0.972107in}{1.734632in}}{\pgfqpoint{0.964207in}{1.737904in}}{\pgfqpoint{0.955971in}{1.737904in}}%
\pgfpathcurveto{\pgfqpoint{0.947735in}{1.737904in}}{\pgfqpoint{0.939835in}{1.734632in}}{\pgfqpoint{0.934011in}{1.728808in}}%
\pgfpathcurveto{\pgfqpoint{0.928187in}{1.722984in}}{\pgfqpoint{0.924915in}{1.715084in}}{\pgfqpoint{0.924915in}{1.706848in}}%
\pgfpathcurveto{\pgfqpoint{0.924915in}{1.698611in}}{\pgfqpoint{0.928187in}{1.690711in}}{\pgfqpoint{0.934011in}{1.684887in}}%
\pgfpathcurveto{\pgfqpoint{0.939835in}{1.679063in}}{\pgfqpoint{0.947735in}{1.675791in}}{\pgfqpoint{0.955971in}{1.675791in}}%
\pgfpathclose%
\pgfusepath{stroke,fill}%
\end{pgfscope}%
\begin{pgfscope}%
\pgfpathrectangle{\pgfqpoint{0.100000in}{0.212622in}}{\pgfqpoint{3.696000in}{3.696000in}}%
\pgfusepath{clip}%
\pgfsetbuttcap%
\pgfsetroundjoin%
\definecolor{currentfill}{rgb}{0.121569,0.466667,0.705882}%
\pgfsetfillcolor{currentfill}%
\pgfsetfillopacity{0.599699}%
\pgfsetlinewidth{1.003750pt}%
\definecolor{currentstroke}{rgb}{0.121569,0.466667,0.705882}%
\pgfsetstrokecolor{currentstroke}%
\pgfsetstrokeopacity{0.599699}%
\pgfsetdash{}{0pt}%
\pgfpathmoveto{\pgfqpoint{0.954842in}{1.673818in}}%
\pgfpathcurveto{\pgfqpoint{0.963078in}{1.673818in}}{\pgfqpoint{0.970978in}{1.677090in}}{\pgfqpoint{0.976802in}{1.682914in}}%
\pgfpathcurveto{\pgfqpoint{0.982626in}{1.688738in}}{\pgfqpoint{0.985899in}{1.696638in}}{\pgfqpoint{0.985899in}{1.704874in}}%
\pgfpathcurveto{\pgfqpoint{0.985899in}{1.713111in}}{\pgfqpoint{0.982626in}{1.721011in}}{\pgfqpoint{0.976802in}{1.726835in}}%
\pgfpathcurveto{\pgfqpoint{0.970978in}{1.732659in}}{\pgfqpoint{0.963078in}{1.735931in}}{\pgfqpoint{0.954842in}{1.735931in}}%
\pgfpathcurveto{\pgfqpoint{0.946606in}{1.735931in}}{\pgfqpoint{0.938706in}{1.732659in}}{\pgfqpoint{0.932882in}{1.726835in}}%
\pgfpathcurveto{\pgfqpoint{0.927058in}{1.721011in}}{\pgfqpoint{0.923786in}{1.713111in}}{\pgfqpoint{0.923786in}{1.704874in}}%
\pgfpathcurveto{\pgfqpoint{0.923786in}{1.696638in}}{\pgfqpoint{0.927058in}{1.688738in}}{\pgfqpoint{0.932882in}{1.682914in}}%
\pgfpathcurveto{\pgfqpoint{0.938706in}{1.677090in}}{\pgfqpoint{0.946606in}{1.673818in}}{\pgfqpoint{0.954842in}{1.673818in}}%
\pgfpathclose%
\pgfusepath{stroke,fill}%
\end{pgfscope}%
\begin{pgfscope}%
\pgfpathrectangle{\pgfqpoint{0.100000in}{0.212622in}}{\pgfqpoint{3.696000in}{3.696000in}}%
\pgfusepath{clip}%
\pgfsetbuttcap%
\pgfsetroundjoin%
\definecolor{currentfill}{rgb}{0.121569,0.466667,0.705882}%
\pgfsetfillcolor{currentfill}%
\pgfsetfillopacity{0.599768}%
\pgfsetlinewidth{1.003750pt}%
\definecolor{currentstroke}{rgb}{0.121569,0.466667,0.705882}%
\pgfsetstrokecolor{currentstroke}%
\pgfsetstrokeopacity{0.599768}%
\pgfsetdash{}{0pt}%
\pgfpathmoveto{\pgfqpoint{2.102087in}{2.093692in}}%
\pgfpathcurveto{\pgfqpoint{2.110323in}{2.093692in}}{\pgfqpoint{2.118223in}{2.096964in}}{\pgfqpoint{2.124047in}{2.102788in}}%
\pgfpathcurveto{\pgfqpoint{2.129871in}{2.108612in}}{\pgfqpoint{2.133144in}{2.116512in}}{\pgfqpoint{2.133144in}{2.124748in}}%
\pgfpathcurveto{\pgfqpoint{2.133144in}{2.132985in}}{\pgfqpoint{2.129871in}{2.140885in}}{\pgfqpoint{2.124047in}{2.146709in}}%
\pgfpathcurveto{\pgfqpoint{2.118223in}{2.152532in}}{\pgfqpoint{2.110323in}{2.155805in}}{\pgfqpoint{2.102087in}{2.155805in}}%
\pgfpathcurveto{\pgfqpoint{2.093851in}{2.155805in}}{\pgfqpoint{2.085951in}{2.152532in}}{\pgfqpoint{2.080127in}{2.146709in}}%
\pgfpathcurveto{\pgfqpoint{2.074303in}{2.140885in}}{\pgfqpoint{2.071031in}{2.132985in}}{\pgfqpoint{2.071031in}{2.124748in}}%
\pgfpathcurveto{\pgfqpoint{2.071031in}{2.116512in}}{\pgfqpoint{2.074303in}{2.108612in}}{\pgfqpoint{2.080127in}{2.102788in}}%
\pgfpathcurveto{\pgfqpoint{2.085951in}{2.096964in}}{\pgfqpoint{2.093851in}{2.093692in}}{\pgfqpoint{2.102087in}{2.093692in}}%
\pgfpathclose%
\pgfusepath{stroke,fill}%
\end{pgfscope}%
\begin{pgfscope}%
\pgfpathrectangle{\pgfqpoint{0.100000in}{0.212622in}}{\pgfqpoint{3.696000in}{3.696000in}}%
\pgfusepath{clip}%
\pgfsetbuttcap%
\pgfsetroundjoin%
\definecolor{currentfill}{rgb}{0.121569,0.466667,0.705882}%
\pgfsetfillcolor{currentfill}%
\pgfsetfillopacity{0.599831}%
\pgfsetlinewidth{1.003750pt}%
\definecolor{currentstroke}{rgb}{0.121569,0.466667,0.705882}%
\pgfsetstrokecolor{currentstroke}%
\pgfsetstrokeopacity{0.599831}%
\pgfsetdash{}{0pt}%
\pgfpathmoveto{\pgfqpoint{0.954443in}{1.672953in}}%
\pgfpathcurveto{\pgfqpoint{0.962680in}{1.672953in}}{\pgfqpoint{0.970580in}{1.676226in}}{\pgfqpoint{0.976404in}{1.682050in}}%
\pgfpathcurveto{\pgfqpoint{0.982228in}{1.687874in}}{\pgfqpoint{0.985500in}{1.695774in}}{\pgfqpoint{0.985500in}{1.704010in}}%
\pgfpathcurveto{\pgfqpoint{0.985500in}{1.712246in}}{\pgfqpoint{0.982228in}{1.720146in}}{\pgfqpoint{0.976404in}{1.725970in}}%
\pgfpathcurveto{\pgfqpoint{0.970580in}{1.731794in}}{\pgfqpoint{0.962680in}{1.735066in}}{\pgfqpoint{0.954443in}{1.735066in}}%
\pgfpathcurveto{\pgfqpoint{0.946207in}{1.735066in}}{\pgfqpoint{0.938307in}{1.731794in}}{\pgfqpoint{0.932483in}{1.725970in}}%
\pgfpathcurveto{\pgfqpoint{0.926659in}{1.720146in}}{\pgfqpoint{0.923387in}{1.712246in}}{\pgfqpoint{0.923387in}{1.704010in}}%
\pgfpathcurveto{\pgfqpoint{0.923387in}{1.695774in}}{\pgfqpoint{0.926659in}{1.687874in}}{\pgfqpoint{0.932483in}{1.682050in}}%
\pgfpathcurveto{\pgfqpoint{0.938307in}{1.676226in}}{\pgfqpoint{0.946207in}{1.672953in}}{\pgfqpoint{0.954443in}{1.672953in}}%
\pgfpathclose%
\pgfusepath{stroke,fill}%
\end{pgfscope}%
\begin{pgfscope}%
\pgfpathrectangle{\pgfqpoint{0.100000in}{0.212622in}}{\pgfqpoint{3.696000in}{3.696000in}}%
\pgfusepath{clip}%
\pgfsetbuttcap%
\pgfsetroundjoin%
\definecolor{currentfill}{rgb}{0.121569,0.466667,0.705882}%
\pgfsetfillcolor{currentfill}%
\pgfsetfillopacity{0.600050}%
\pgfsetlinewidth{1.003750pt}%
\definecolor{currentstroke}{rgb}{0.121569,0.466667,0.705882}%
\pgfsetstrokecolor{currentstroke}%
\pgfsetstrokeopacity{0.600050}%
\pgfsetdash{}{0pt}%
\pgfpathmoveto{\pgfqpoint{0.953629in}{1.671422in}}%
\pgfpathcurveto{\pgfqpoint{0.961865in}{1.671422in}}{\pgfqpoint{0.969765in}{1.674695in}}{\pgfqpoint{0.975589in}{1.680519in}}%
\pgfpathcurveto{\pgfqpoint{0.981413in}{1.686343in}}{\pgfqpoint{0.984686in}{1.694243in}}{\pgfqpoint{0.984686in}{1.702479in}}%
\pgfpathcurveto{\pgfqpoint{0.984686in}{1.710715in}}{\pgfqpoint{0.981413in}{1.718615in}}{\pgfqpoint{0.975589in}{1.724439in}}%
\pgfpathcurveto{\pgfqpoint{0.969765in}{1.730263in}}{\pgfqpoint{0.961865in}{1.733535in}}{\pgfqpoint{0.953629in}{1.733535in}}%
\pgfpathcurveto{\pgfqpoint{0.945393in}{1.733535in}}{\pgfqpoint{0.937493in}{1.730263in}}{\pgfqpoint{0.931669in}{1.724439in}}%
\pgfpathcurveto{\pgfqpoint{0.925845in}{1.718615in}}{\pgfqpoint{0.922573in}{1.710715in}}{\pgfqpoint{0.922573in}{1.702479in}}%
\pgfpathcurveto{\pgfqpoint{0.922573in}{1.694243in}}{\pgfqpoint{0.925845in}{1.686343in}}{\pgfqpoint{0.931669in}{1.680519in}}%
\pgfpathcurveto{\pgfqpoint{0.937493in}{1.674695in}}{\pgfqpoint{0.945393in}{1.671422in}}{\pgfqpoint{0.953629in}{1.671422in}}%
\pgfpathclose%
\pgfusepath{stroke,fill}%
\end{pgfscope}%
\begin{pgfscope}%
\pgfpathrectangle{\pgfqpoint{0.100000in}{0.212622in}}{\pgfqpoint{3.696000in}{3.696000in}}%
\pgfusepath{clip}%
\pgfsetbuttcap%
\pgfsetroundjoin%
\definecolor{currentfill}{rgb}{0.121569,0.466667,0.705882}%
\pgfsetfillcolor{currentfill}%
\pgfsetfillopacity{0.600204}%
\pgfsetlinewidth{1.003750pt}%
\definecolor{currentstroke}{rgb}{0.121569,0.466667,0.705882}%
\pgfsetstrokecolor{currentstroke}%
\pgfsetstrokeopacity{0.600204}%
\pgfsetdash{}{0pt}%
\pgfpathmoveto{\pgfqpoint{0.953275in}{1.670317in}}%
\pgfpathcurveto{\pgfqpoint{0.961512in}{1.670317in}}{\pgfqpoint{0.969412in}{1.673589in}}{\pgfqpoint{0.975236in}{1.679413in}}%
\pgfpathcurveto{\pgfqpoint{0.981060in}{1.685237in}}{\pgfqpoint{0.984332in}{1.693137in}}{\pgfqpoint{0.984332in}{1.701373in}}%
\pgfpathcurveto{\pgfqpoint{0.984332in}{1.709610in}}{\pgfqpoint{0.981060in}{1.717510in}}{\pgfqpoint{0.975236in}{1.723334in}}%
\pgfpathcurveto{\pgfqpoint{0.969412in}{1.729157in}}{\pgfqpoint{0.961512in}{1.732430in}}{\pgfqpoint{0.953275in}{1.732430in}}%
\pgfpathcurveto{\pgfqpoint{0.945039in}{1.732430in}}{\pgfqpoint{0.937139in}{1.729157in}}{\pgfqpoint{0.931315in}{1.723334in}}%
\pgfpathcurveto{\pgfqpoint{0.925491in}{1.717510in}}{\pgfqpoint{0.922219in}{1.709610in}}{\pgfqpoint{0.922219in}{1.701373in}}%
\pgfpathcurveto{\pgfqpoint{0.922219in}{1.693137in}}{\pgfqpoint{0.925491in}{1.685237in}}{\pgfqpoint{0.931315in}{1.679413in}}%
\pgfpathcurveto{\pgfqpoint{0.937139in}{1.673589in}}{\pgfqpoint{0.945039in}{1.670317in}}{\pgfqpoint{0.953275in}{1.670317in}}%
\pgfpathclose%
\pgfusepath{stroke,fill}%
\end{pgfscope}%
\begin{pgfscope}%
\pgfpathrectangle{\pgfqpoint{0.100000in}{0.212622in}}{\pgfqpoint{3.696000in}{3.696000in}}%
\pgfusepath{clip}%
\pgfsetbuttcap%
\pgfsetroundjoin%
\definecolor{currentfill}{rgb}{0.121569,0.466667,0.705882}%
\pgfsetfillcolor{currentfill}%
\pgfsetfillopacity{0.600541}%
\pgfsetlinewidth{1.003750pt}%
\definecolor{currentstroke}{rgb}{0.121569,0.466667,0.705882}%
\pgfsetstrokecolor{currentstroke}%
\pgfsetstrokeopacity{0.600541}%
\pgfsetdash{}{0pt}%
\pgfpathmoveto{\pgfqpoint{2.102511in}{2.091002in}}%
\pgfpathcurveto{\pgfqpoint{2.110747in}{2.091002in}}{\pgfqpoint{2.118647in}{2.094275in}}{\pgfqpoint{2.124471in}{2.100099in}}%
\pgfpathcurveto{\pgfqpoint{2.130295in}{2.105923in}}{\pgfqpoint{2.133567in}{2.113823in}}{\pgfqpoint{2.133567in}{2.122059in}}%
\pgfpathcurveto{\pgfqpoint{2.133567in}{2.130295in}}{\pgfqpoint{2.130295in}{2.138195in}}{\pgfqpoint{2.124471in}{2.144019in}}%
\pgfpathcurveto{\pgfqpoint{2.118647in}{2.149843in}}{\pgfqpoint{2.110747in}{2.153115in}}{\pgfqpoint{2.102511in}{2.153115in}}%
\pgfpathcurveto{\pgfqpoint{2.094274in}{2.153115in}}{\pgfqpoint{2.086374in}{2.149843in}}{\pgfqpoint{2.080550in}{2.144019in}}%
\pgfpathcurveto{\pgfqpoint{2.074726in}{2.138195in}}{\pgfqpoint{2.071454in}{2.130295in}}{\pgfqpoint{2.071454in}{2.122059in}}%
\pgfpathcurveto{\pgfqpoint{2.071454in}{2.113823in}}{\pgfqpoint{2.074726in}{2.105923in}}{\pgfqpoint{2.080550in}{2.100099in}}%
\pgfpathcurveto{\pgfqpoint{2.086374in}{2.094275in}}{\pgfqpoint{2.094274in}{2.091002in}}{\pgfqpoint{2.102511in}{2.091002in}}%
\pgfpathclose%
\pgfusepath{stroke,fill}%
\end{pgfscope}%
\begin{pgfscope}%
\pgfpathrectangle{\pgfqpoint{0.100000in}{0.212622in}}{\pgfqpoint{3.696000in}{3.696000in}}%
\pgfusepath{clip}%
\pgfsetbuttcap%
\pgfsetroundjoin%
\definecolor{currentfill}{rgb}{0.121569,0.466667,0.705882}%
\pgfsetfillcolor{currentfill}%
\pgfsetfillopacity{0.600547}%
\pgfsetlinewidth{1.003750pt}%
\definecolor{currentstroke}{rgb}{0.121569,0.466667,0.705882}%
\pgfsetstrokecolor{currentstroke}%
\pgfsetstrokeopacity{0.600547}%
\pgfsetdash{}{0pt}%
\pgfpathmoveto{\pgfqpoint{0.952332in}{1.668824in}}%
\pgfpathcurveto{\pgfqpoint{0.960569in}{1.668824in}}{\pgfqpoint{0.968469in}{1.672096in}}{\pgfqpoint{0.974293in}{1.677920in}}%
\pgfpathcurveto{\pgfqpoint{0.980116in}{1.683744in}}{\pgfqpoint{0.983389in}{1.691644in}}{\pgfqpoint{0.983389in}{1.699880in}}%
\pgfpathcurveto{\pgfqpoint{0.983389in}{1.708117in}}{\pgfqpoint{0.980116in}{1.716017in}}{\pgfqpoint{0.974293in}{1.721841in}}%
\pgfpathcurveto{\pgfqpoint{0.968469in}{1.727664in}}{\pgfqpoint{0.960569in}{1.730937in}}{\pgfqpoint{0.952332in}{1.730937in}}%
\pgfpathcurveto{\pgfqpoint{0.944096in}{1.730937in}}{\pgfqpoint{0.936196in}{1.727664in}}{\pgfqpoint{0.930372in}{1.721841in}}%
\pgfpathcurveto{\pgfqpoint{0.924548in}{1.716017in}}{\pgfqpoint{0.921276in}{1.708117in}}{\pgfqpoint{0.921276in}{1.699880in}}%
\pgfpathcurveto{\pgfqpoint{0.921276in}{1.691644in}}{\pgfqpoint{0.924548in}{1.683744in}}{\pgfqpoint{0.930372in}{1.677920in}}%
\pgfpathcurveto{\pgfqpoint{0.936196in}{1.672096in}}{\pgfqpoint{0.944096in}{1.668824in}}{\pgfqpoint{0.952332in}{1.668824in}}%
\pgfpathclose%
\pgfusepath{stroke,fill}%
\end{pgfscope}%
\begin{pgfscope}%
\pgfpathrectangle{\pgfqpoint{0.100000in}{0.212622in}}{\pgfqpoint{3.696000in}{3.696000in}}%
\pgfusepath{clip}%
\pgfsetbuttcap%
\pgfsetroundjoin%
\definecolor{currentfill}{rgb}{0.121569,0.466667,0.705882}%
\pgfsetfillcolor{currentfill}%
\pgfsetfillopacity{0.601214}%
\pgfsetlinewidth{1.003750pt}%
\definecolor{currentstroke}{rgb}{0.121569,0.466667,0.705882}%
\pgfsetstrokecolor{currentstroke}%
\pgfsetstrokeopacity{0.601214}%
\pgfsetdash{}{0pt}%
\pgfpathmoveto{\pgfqpoint{0.950727in}{1.666126in}}%
\pgfpathcurveto{\pgfqpoint{0.958963in}{1.666126in}}{\pgfqpoint{0.966863in}{1.669399in}}{\pgfqpoint{0.972687in}{1.675222in}}%
\pgfpathcurveto{\pgfqpoint{0.978511in}{1.681046in}}{\pgfqpoint{0.981783in}{1.688946in}}{\pgfqpoint{0.981783in}{1.697183in}}%
\pgfpathcurveto{\pgfqpoint{0.981783in}{1.705419in}}{\pgfqpoint{0.978511in}{1.713319in}}{\pgfqpoint{0.972687in}{1.719143in}}%
\pgfpathcurveto{\pgfqpoint{0.966863in}{1.724967in}}{\pgfqpoint{0.958963in}{1.728239in}}{\pgfqpoint{0.950727in}{1.728239in}}%
\pgfpathcurveto{\pgfqpoint{0.942490in}{1.728239in}}{\pgfqpoint{0.934590in}{1.724967in}}{\pgfqpoint{0.928766in}{1.719143in}}%
\pgfpathcurveto{\pgfqpoint{0.922943in}{1.713319in}}{\pgfqpoint{0.919670in}{1.705419in}}{\pgfqpoint{0.919670in}{1.697183in}}%
\pgfpathcurveto{\pgfqpoint{0.919670in}{1.688946in}}{\pgfqpoint{0.922943in}{1.681046in}}{\pgfqpoint{0.928766in}{1.675222in}}%
\pgfpathcurveto{\pgfqpoint{0.934590in}{1.669399in}}{\pgfqpoint{0.942490in}{1.666126in}}{\pgfqpoint{0.950727in}{1.666126in}}%
\pgfpathclose%
\pgfusepath{stroke,fill}%
\end{pgfscope}%
\begin{pgfscope}%
\pgfpathrectangle{\pgfqpoint{0.100000in}{0.212622in}}{\pgfqpoint{3.696000in}{3.696000in}}%
\pgfusepath{clip}%
\pgfsetbuttcap%
\pgfsetroundjoin%
\definecolor{currentfill}{rgb}{0.121569,0.466667,0.705882}%
\pgfsetfillcolor{currentfill}%
\pgfsetfillopacity{0.601660}%
\pgfsetlinewidth{1.003750pt}%
\definecolor{currentstroke}{rgb}{0.121569,0.466667,0.705882}%
\pgfsetstrokecolor{currentstroke}%
\pgfsetstrokeopacity{0.601660}%
\pgfsetdash{}{0pt}%
\pgfpathmoveto{\pgfqpoint{0.949555in}{1.664201in}}%
\pgfpathcurveto{\pgfqpoint{0.957791in}{1.664201in}}{\pgfqpoint{0.965691in}{1.667473in}}{\pgfqpoint{0.971515in}{1.673297in}}%
\pgfpathcurveto{\pgfqpoint{0.977339in}{1.679121in}}{\pgfqpoint{0.980611in}{1.687021in}}{\pgfqpoint{0.980611in}{1.695258in}}%
\pgfpathcurveto{\pgfqpoint{0.980611in}{1.703494in}}{\pgfqpoint{0.977339in}{1.711394in}}{\pgfqpoint{0.971515in}{1.717218in}}%
\pgfpathcurveto{\pgfqpoint{0.965691in}{1.723042in}}{\pgfqpoint{0.957791in}{1.726314in}}{\pgfqpoint{0.949555in}{1.726314in}}%
\pgfpathcurveto{\pgfqpoint{0.941319in}{1.726314in}}{\pgfqpoint{0.933419in}{1.723042in}}{\pgfqpoint{0.927595in}{1.717218in}}%
\pgfpathcurveto{\pgfqpoint{0.921771in}{1.711394in}}{\pgfqpoint{0.918498in}{1.703494in}}{\pgfqpoint{0.918498in}{1.695258in}}%
\pgfpathcurveto{\pgfqpoint{0.918498in}{1.687021in}}{\pgfqpoint{0.921771in}{1.679121in}}{\pgfqpoint{0.927595in}{1.673297in}}%
\pgfpathcurveto{\pgfqpoint{0.933419in}{1.667473in}}{\pgfqpoint{0.941319in}{1.664201in}}{\pgfqpoint{0.949555in}{1.664201in}}%
\pgfpathclose%
\pgfusepath{stroke,fill}%
\end{pgfscope}%
\begin{pgfscope}%
\pgfpathrectangle{\pgfqpoint{0.100000in}{0.212622in}}{\pgfqpoint{3.696000in}{3.696000in}}%
\pgfusepath{clip}%
\pgfsetbuttcap%
\pgfsetroundjoin%
\definecolor{currentfill}{rgb}{0.121569,0.466667,0.705882}%
\pgfsetfillcolor{currentfill}%
\pgfsetfillopacity{0.601691}%
\pgfsetlinewidth{1.003750pt}%
\definecolor{currentstroke}{rgb}{0.121569,0.466667,0.705882}%
\pgfsetstrokecolor{currentstroke}%
\pgfsetstrokeopacity{0.601691}%
\pgfsetdash{}{0pt}%
\pgfpathmoveto{\pgfqpoint{2.103637in}{2.086515in}}%
\pgfpathcurveto{\pgfqpoint{2.111874in}{2.086515in}}{\pgfqpoint{2.119774in}{2.089787in}}{\pgfqpoint{2.125598in}{2.095611in}}%
\pgfpathcurveto{\pgfqpoint{2.131422in}{2.101435in}}{\pgfqpoint{2.134694in}{2.109335in}}{\pgfqpoint{2.134694in}{2.117571in}}%
\pgfpathcurveto{\pgfqpoint{2.134694in}{2.125808in}}{\pgfqpoint{2.131422in}{2.133708in}}{\pgfqpoint{2.125598in}{2.139532in}}%
\pgfpathcurveto{\pgfqpoint{2.119774in}{2.145355in}}{\pgfqpoint{2.111874in}{2.148628in}}{\pgfqpoint{2.103637in}{2.148628in}}%
\pgfpathcurveto{\pgfqpoint{2.095401in}{2.148628in}}{\pgfqpoint{2.087501in}{2.145355in}}{\pgfqpoint{2.081677in}{2.139532in}}%
\pgfpathcurveto{\pgfqpoint{2.075853in}{2.133708in}}{\pgfqpoint{2.072581in}{2.125808in}}{\pgfqpoint{2.072581in}{2.117571in}}%
\pgfpathcurveto{\pgfqpoint{2.072581in}{2.109335in}}{\pgfqpoint{2.075853in}{2.101435in}}{\pgfqpoint{2.081677in}{2.095611in}}%
\pgfpathcurveto{\pgfqpoint{2.087501in}{2.089787in}}{\pgfqpoint{2.095401in}{2.086515in}}{\pgfqpoint{2.103637in}{2.086515in}}%
\pgfpathclose%
\pgfusepath{stroke,fill}%
\end{pgfscope}%
\begin{pgfscope}%
\pgfpathrectangle{\pgfqpoint{0.100000in}{0.212622in}}{\pgfqpoint{3.696000in}{3.696000in}}%
\pgfusepath{clip}%
\pgfsetbuttcap%
\pgfsetroundjoin%
\definecolor{currentfill}{rgb}{0.121569,0.466667,0.705882}%
\pgfsetfillcolor{currentfill}%
\pgfsetfillopacity{0.601872}%
\pgfsetlinewidth{1.003750pt}%
\definecolor{currentstroke}{rgb}{0.121569,0.466667,0.705882}%
\pgfsetstrokecolor{currentstroke}%
\pgfsetstrokeopacity{0.601872}%
\pgfsetdash{}{0pt}%
\pgfpathmoveto{\pgfqpoint{0.637190in}{1.208762in}}%
\pgfpathcurveto{\pgfqpoint{0.645426in}{1.208762in}}{\pgfqpoint{0.653326in}{1.212035in}}{\pgfqpoint{0.659150in}{1.217859in}}%
\pgfpathcurveto{\pgfqpoint{0.664974in}{1.223683in}}{\pgfqpoint{0.668246in}{1.231583in}}{\pgfqpoint{0.668246in}{1.239819in}}%
\pgfpathcurveto{\pgfqpoint{0.668246in}{1.248055in}}{\pgfqpoint{0.664974in}{1.255955in}}{\pgfqpoint{0.659150in}{1.261779in}}%
\pgfpathcurveto{\pgfqpoint{0.653326in}{1.267603in}}{\pgfqpoint{0.645426in}{1.270875in}}{\pgfqpoint{0.637190in}{1.270875in}}%
\pgfpathcurveto{\pgfqpoint{0.628953in}{1.270875in}}{\pgfqpoint{0.621053in}{1.267603in}}{\pgfqpoint{0.615230in}{1.261779in}}%
\pgfpathcurveto{\pgfqpoint{0.609406in}{1.255955in}}{\pgfqpoint{0.606133in}{1.248055in}}{\pgfqpoint{0.606133in}{1.239819in}}%
\pgfpathcurveto{\pgfqpoint{0.606133in}{1.231583in}}{\pgfqpoint{0.609406in}{1.223683in}}{\pgfqpoint{0.615230in}{1.217859in}}%
\pgfpathcurveto{\pgfqpoint{0.621053in}{1.212035in}}{\pgfqpoint{0.628953in}{1.208762in}}{\pgfqpoint{0.637190in}{1.208762in}}%
\pgfpathclose%
\pgfusepath{stroke,fill}%
\end{pgfscope}%
\begin{pgfscope}%
\pgfpathrectangle{\pgfqpoint{0.100000in}{0.212622in}}{\pgfqpoint{3.696000in}{3.696000in}}%
\pgfusepath{clip}%
\pgfsetbuttcap%
\pgfsetroundjoin%
\definecolor{currentfill}{rgb}{0.121569,0.466667,0.705882}%
\pgfsetfillcolor{currentfill}%
\pgfsetfillopacity{0.601910}%
\pgfsetlinewidth{1.003750pt}%
\definecolor{currentstroke}{rgb}{0.121569,0.466667,0.705882}%
\pgfsetstrokecolor{currentstroke}%
\pgfsetstrokeopacity{0.601910}%
\pgfsetdash{}{0pt}%
\pgfpathmoveto{\pgfqpoint{0.873744in}{1.486842in}}%
\pgfpathcurveto{\pgfqpoint{0.881980in}{1.486842in}}{\pgfqpoint{0.889880in}{1.490114in}}{\pgfqpoint{0.895704in}{1.495938in}}%
\pgfpathcurveto{\pgfqpoint{0.901528in}{1.501762in}}{\pgfqpoint{0.904801in}{1.509662in}}{\pgfqpoint{0.904801in}{1.517899in}}%
\pgfpathcurveto{\pgfqpoint{0.904801in}{1.526135in}}{\pgfqpoint{0.901528in}{1.534035in}}{\pgfqpoint{0.895704in}{1.539859in}}%
\pgfpathcurveto{\pgfqpoint{0.889880in}{1.545683in}}{\pgfqpoint{0.881980in}{1.548955in}}{\pgfqpoint{0.873744in}{1.548955in}}%
\pgfpathcurveto{\pgfqpoint{0.865508in}{1.548955in}}{\pgfqpoint{0.857608in}{1.545683in}}{\pgfqpoint{0.851784in}{1.539859in}}%
\pgfpathcurveto{\pgfqpoint{0.845960in}{1.534035in}}{\pgfqpoint{0.842688in}{1.526135in}}{\pgfqpoint{0.842688in}{1.517899in}}%
\pgfpathcurveto{\pgfqpoint{0.842688in}{1.509662in}}{\pgfqpoint{0.845960in}{1.501762in}}{\pgfqpoint{0.851784in}{1.495938in}}%
\pgfpathcurveto{\pgfqpoint{0.857608in}{1.490114in}}{\pgfqpoint{0.865508in}{1.486842in}}{\pgfqpoint{0.873744in}{1.486842in}}%
\pgfpathclose%
\pgfusepath{stroke,fill}%
\end{pgfscope}%
\begin{pgfscope}%
\pgfpathrectangle{\pgfqpoint{0.100000in}{0.212622in}}{\pgfqpoint{3.696000in}{3.696000in}}%
\pgfusepath{clip}%
\pgfsetbuttcap%
\pgfsetroundjoin%
\definecolor{currentfill}{rgb}{0.121569,0.466667,0.705882}%
\pgfsetfillcolor{currentfill}%
\pgfsetfillopacity{0.602474}%
\pgfsetlinewidth{1.003750pt}%
\definecolor{currentstroke}{rgb}{0.121569,0.466667,0.705882}%
\pgfsetstrokecolor{currentstroke}%
\pgfsetstrokeopacity{0.602474}%
\pgfsetdash{}{0pt}%
\pgfpathmoveto{\pgfqpoint{0.947367in}{1.660792in}}%
\pgfpathcurveto{\pgfqpoint{0.955604in}{1.660792in}}{\pgfqpoint{0.963504in}{1.664064in}}{\pgfqpoint{0.969328in}{1.669888in}}%
\pgfpathcurveto{\pgfqpoint{0.975151in}{1.675712in}}{\pgfqpoint{0.978424in}{1.683612in}}{\pgfqpoint{0.978424in}{1.691848in}}%
\pgfpathcurveto{\pgfqpoint{0.978424in}{1.700085in}}{\pgfqpoint{0.975151in}{1.707985in}}{\pgfqpoint{0.969328in}{1.713809in}}%
\pgfpathcurveto{\pgfqpoint{0.963504in}{1.719633in}}{\pgfqpoint{0.955604in}{1.722905in}}{\pgfqpoint{0.947367in}{1.722905in}}%
\pgfpathcurveto{\pgfqpoint{0.939131in}{1.722905in}}{\pgfqpoint{0.931231in}{1.719633in}}{\pgfqpoint{0.925407in}{1.713809in}}%
\pgfpathcurveto{\pgfqpoint{0.919583in}{1.707985in}}{\pgfqpoint{0.916311in}{1.700085in}}{\pgfqpoint{0.916311in}{1.691848in}}%
\pgfpathcurveto{\pgfqpoint{0.916311in}{1.683612in}}{\pgfqpoint{0.919583in}{1.675712in}}{\pgfqpoint{0.925407in}{1.669888in}}%
\pgfpathcurveto{\pgfqpoint{0.931231in}{1.664064in}}{\pgfqpoint{0.939131in}{1.660792in}}{\pgfqpoint{0.947367in}{1.660792in}}%
\pgfpathclose%
\pgfusepath{stroke,fill}%
\end{pgfscope}%
\begin{pgfscope}%
\pgfpathrectangle{\pgfqpoint{0.100000in}{0.212622in}}{\pgfqpoint{3.696000in}{3.696000in}}%
\pgfusepath{clip}%
\pgfsetbuttcap%
\pgfsetroundjoin%
\definecolor{currentfill}{rgb}{0.121569,0.466667,0.705882}%
\pgfsetfillcolor{currentfill}%
\pgfsetfillopacity{0.603005}%
\pgfsetlinewidth{1.003750pt}%
\definecolor{currentstroke}{rgb}{0.121569,0.466667,0.705882}%
\pgfsetstrokecolor{currentstroke}%
\pgfsetstrokeopacity{0.603005}%
\pgfsetdash{}{0pt}%
\pgfpathmoveto{\pgfqpoint{0.945969in}{1.658485in}}%
\pgfpathcurveto{\pgfqpoint{0.954205in}{1.658485in}}{\pgfqpoint{0.962105in}{1.661757in}}{\pgfqpoint{0.967929in}{1.667581in}}%
\pgfpathcurveto{\pgfqpoint{0.973753in}{1.673405in}}{\pgfqpoint{0.977025in}{1.681305in}}{\pgfqpoint{0.977025in}{1.689541in}}%
\pgfpathcurveto{\pgfqpoint{0.977025in}{1.697778in}}{\pgfqpoint{0.973753in}{1.705678in}}{\pgfqpoint{0.967929in}{1.711502in}}%
\pgfpathcurveto{\pgfqpoint{0.962105in}{1.717325in}}{\pgfqpoint{0.954205in}{1.720598in}}{\pgfqpoint{0.945969in}{1.720598in}}%
\pgfpathcurveto{\pgfqpoint{0.937733in}{1.720598in}}{\pgfqpoint{0.929833in}{1.717325in}}{\pgfqpoint{0.924009in}{1.711502in}}%
\pgfpathcurveto{\pgfqpoint{0.918185in}{1.705678in}}{\pgfqpoint{0.914912in}{1.697778in}}{\pgfqpoint{0.914912in}{1.689541in}}%
\pgfpathcurveto{\pgfqpoint{0.914912in}{1.681305in}}{\pgfqpoint{0.918185in}{1.673405in}}{\pgfqpoint{0.924009in}{1.667581in}}%
\pgfpathcurveto{\pgfqpoint{0.929833in}{1.661757in}}{\pgfqpoint{0.937733in}{1.658485in}}{\pgfqpoint{0.945969in}{1.658485in}}%
\pgfpathclose%
\pgfusepath{stroke,fill}%
\end{pgfscope}%
\begin{pgfscope}%
\pgfpathrectangle{\pgfqpoint{0.100000in}{0.212622in}}{\pgfqpoint{3.696000in}{3.696000in}}%
\pgfusepath{clip}%
\pgfsetbuttcap%
\pgfsetroundjoin%
\definecolor{currentfill}{rgb}{0.121569,0.466667,0.705882}%
\pgfsetfillcolor{currentfill}%
\pgfsetfillopacity{0.603075}%
\pgfsetlinewidth{1.003750pt}%
\definecolor{currentstroke}{rgb}{0.121569,0.466667,0.705882}%
\pgfsetstrokecolor{currentstroke}%
\pgfsetstrokeopacity{0.603075}%
\pgfsetdash{}{0pt}%
\pgfpathmoveto{\pgfqpoint{2.104588in}{2.081938in}}%
\pgfpathcurveto{\pgfqpoint{2.112824in}{2.081938in}}{\pgfqpoint{2.120724in}{2.085210in}}{\pgfqpoint{2.126548in}{2.091034in}}%
\pgfpathcurveto{\pgfqpoint{2.132372in}{2.096858in}}{\pgfqpoint{2.135644in}{2.104758in}}{\pgfqpoint{2.135644in}{2.112994in}}%
\pgfpathcurveto{\pgfqpoint{2.135644in}{2.121230in}}{\pgfqpoint{2.132372in}{2.129131in}}{\pgfqpoint{2.126548in}{2.134954in}}%
\pgfpathcurveto{\pgfqpoint{2.120724in}{2.140778in}}{\pgfqpoint{2.112824in}{2.144051in}}{\pgfqpoint{2.104588in}{2.144051in}}%
\pgfpathcurveto{\pgfqpoint{2.096351in}{2.144051in}}{\pgfqpoint{2.088451in}{2.140778in}}{\pgfqpoint{2.082627in}{2.134954in}}%
\pgfpathcurveto{\pgfqpoint{2.076803in}{2.129131in}}{\pgfqpoint{2.073531in}{2.121230in}}{\pgfqpoint{2.073531in}{2.112994in}}%
\pgfpathcurveto{\pgfqpoint{2.073531in}{2.104758in}}{\pgfqpoint{2.076803in}{2.096858in}}{\pgfqpoint{2.082627in}{2.091034in}}%
\pgfpathcurveto{\pgfqpoint{2.088451in}{2.085210in}}{\pgfqpoint{2.096351in}{2.081938in}}{\pgfqpoint{2.104588in}{2.081938in}}%
\pgfpathclose%
\pgfusepath{stroke,fill}%
\end{pgfscope}%
\begin{pgfscope}%
\pgfpathrectangle{\pgfqpoint{0.100000in}{0.212622in}}{\pgfqpoint{3.696000in}{3.696000in}}%
\pgfusepath{clip}%
\pgfsetbuttcap%
\pgfsetroundjoin%
\definecolor{currentfill}{rgb}{0.121569,0.466667,0.705882}%
\pgfsetfillcolor{currentfill}%
\pgfsetfillopacity{0.603861}%
\pgfsetlinewidth{1.003750pt}%
\definecolor{currentstroke}{rgb}{0.121569,0.466667,0.705882}%
\pgfsetstrokecolor{currentstroke}%
\pgfsetstrokeopacity{0.603861}%
\pgfsetdash{}{0pt}%
\pgfpathmoveto{\pgfqpoint{2.104998in}{2.079411in}}%
\pgfpathcurveto{\pgfqpoint{2.113235in}{2.079411in}}{\pgfqpoint{2.121135in}{2.082684in}}{\pgfqpoint{2.126959in}{2.088507in}}%
\pgfpathcurveto{\pgfqpoint{2.132783in}{2.094331in}}{\pgfqpoint{2.136055in}{2.102231in}}{\pgfqpoint{2.136055in}{2.110468in}}%
\pgfpathcurveto{\pgfqpoint{2.136055in}{2.118704in}}{\pgfqpoint{2.132783in}{2.126604in}}{\pgfqpoint{2.126959in}{2.132428in}}%
\pgfpathcurveto{\pgfqpoint{2.121135in}{2.138252in}}{\pgfqpoint{2.113235in}{2.141524in}}{\pgfqpoint{2.104998in}{2.141524in}}%
\pgfpathcurveto{\pgfqpoint{2.096762in}{2.141524in}}{\pgfqpoint{2.088862in}{2.138252in}}{\pgfqpoint{2.083038in}{2.132428in}}%
\pgfpathcurveto{\pgfqpoint{2.077214in}{2.126604in}}{\pgfqpoint{2.073942in}{2.118704in}}{\pgfqpoint{2.073942in}{2.110468in}}%
\pgfpathcurveto{\pgfqpoint{2.073942in}{2.102231in}}{\pgfqpoint{2.077214in}{2.094331in}}{\pgfqpoint{2.083038in}{2.088507in}}%
\pgfpathcurveto{\pgfqpoint{2.088862in}{2.082684in}}{\pgfqpoint{2.096762in}{2.079411in}}{\pgfqpoint{2.104998in}{2.079411in}}%
\pgfpathclose%
\pgfusepath{stroke,fill}%
\end{pgfscope}%
\begin{pgfscope}%
\pgfpathrectangle{\pgfqpoint{0.100000in}{0.212622in}}{\pgfqpoint{3.696000in}{3.696000in}}%
\pgfusepath{clip}%
\pgfsetbuttcap%
\pgfsetroundjoin%
\definecolor{currentfill}{rgb}{0.121569,0.466667,0.705882}%
\pgfsetfillcolor{currentfill}%
\pgfsetfillopacity{0.604011}%
\pgfsetlinewidth{1.003750pt}%
\definecolor{currentstroke}{rgb}{0.121569,0.466667,0.705882}%
\pgfsetstrokecolor{currentstroke}%
\pgfsetstrokeopacity{0.604011}%
\pgfsetdash{}{0pt}%
\pgfpathmoveto{\pgfqpoint{0.943307in}{1.654597in}}%
\pgfpathcurveto{\pgfqpoint{0.951543in}{1.654597in}}{\pgfqpoint{0.959443in}{1.657869in}}{\pgfqpoint{0.965267in}{1.663693in}}%
\pgfpathcurveto{\pgfqpoint{0.971091in}{1.669517in}}{\pgfqpoint{0.974363in}{1.677417in}}{\pgfqpoint{0.974363in}{1.685653in}}%
\pgfpathcurveto{\pgfqpoint{0.974363in}{1.693890in}}{\pgfqpoint{0.971091in}{1.701790in}}{\pgfqpoint{0.965267in}{1.707614in}}%
\pgfpathcurveto{\pgfqpoint{0.959443in}{1.713438in}}{\pgfqpoint{0.951543in}{1.716710in}}{\pgfqpoint{0.943307in}{1.716710in}}%
\pgfpathcurveto{\pgfqpoint{0.935070in}{1.716710in}}{\pgfqpoint{0.927170in}{1.713438in}}{\pgfqpoint{0.921346in}{1.707614in}}%
\pgfpathcurveto{\pgfqpoint{0.915522in}{1.701790in}}{\pgfqpoint{0.912250in}{1.693890in}}{\pgfqpoint{0.912250in}{1.685653in}}%
\pgfpathcurveto{\pgfqpoint{0.912250in}{1.677417in}}{\pgfqpoint{0.915522in}{1.669517in}}{\pgfqpoint{0.921346in}{1.663693in}}%
\pgfpathcurveto{\pgfqpoint{0.927170in}{1.657869in}}{\pgfqpoint{0.935070in}{1.654597in}}{\pgfqpoint{0.943307in}{1.654597in}}%
\pgfpathclose%
\pgfusepath{stroke,fill}%
\end{pgfscope}%
\begin{pgfscope}%
\pgfpathrectangle{\pgfqpoint{0.100000in}{0.212622in}}{\pgfqpoint{3.696000in}{3.696000in}}%
\pgfusepath{clip}%
\pgfsetbuttcap%
\pgfsetroundjoin%
\definecolor{currentfill}{rgb}{0.121569,0.466667,0.705882}%
\pgfsetfillcolor{currentfill}%
\pgfsetfillopacity{0.604828}%
\pgfsetlinewidth{1.003750pt}%
\definecolor{currentstroke}{rgb}{0.121569,0.466667,0.705882}%
\pgfsetstrokecolor{currentstroke}%
\pgfsetstrokeopacity{0.604828}%
\pgfsetdash{}{0pt}%
\pgfpathmoveto{\pgfqpoint{0.941390in}{1.651524in}}%
\pgfpathcurveto{\pgfqpoint{0.949626in}{1.651524in}}{\pgfqpoint{0.957526in}{1.654796in}}{\pgfqpoint{0.963350in}{1.660620in}}%
\pgfpathcurveto{\pgfqpoint{0.969174in}{1.666444in}}{\pgfqpoint{0.972447in}{1.674344in}}{\pgfqpoint{0.972447in}{1.682581in}}%
\pgfpathcurveto{\pgfqpoint{0.972447in}{1.690817in}}{\pgfqpoint{0.969174in}{1.698717in}}{\pgfqpoint{0.963350in}{1.704541in}}%
\pgfpathcurveto{\pgfqpoint{0.957526in}{1.710365in}}{\pgfqpoint{0.949626in}{1.713637in}}{\pgfqpoint{0.941390in}{1.713637in}}%
\pgfpathcurveto{\pgfqpoint{0.933154in}{1.713637in}}{\pgfqpoint{0.925254in}{1.710365in}}{\pgfqpoint{0.919430in}{1.704541in}}%
\pgfpathcurveto{\pgfqpoint{0.913606in}{1.698717in}}{\pgfqpoint{0.910334in}{1.690817in}}{\pgfqpoint{0.910334in}{1.682581in}}%
\pgfpathcurveto{\pgfqpoint{0.910334in}{1.674344in}}{\pgfqpoint{0.913606in}{1.666444in}}{\pgfqpoint{0.919430in}{1.660620in}}%
\pgfpathcurveto{\pgfqpoint{0.925254in}{1.654796in}}{\pgfqpoint{0.933154in}{1.651524in}}{\pgfqpoint{0.941390in}{1.651524in}}%
\pgfpathclose%
\pgfusepath{stroke,fill}%
\end{pgfscope}%
\begin{pgfscope}%
\pgfpathrectangle{\pgfqpoint{0.100000in}{0.212622in}}{\pgfqpoint{3.696000in}{3.696000in}}%
\pgfusepath{clip}%
\pgfsetbuttcap%
\pgfsetroundjoin%
\definecolor{currentfill}{rgb}{0.121569,0.466667,0.705882}%
\pgfsetfillcolor{currentfill}%
\pgfsetfillopacity{0.604909}%
\pgfsetlinewidth{1.003750pt}%
\definecolor{currentstroke}{rgb}{0.121569,0.466667,0.705882}%
\pgfsetstrokecolor{currentstroke}%
\pgfsetstrokeopacity{0.604909}%
\pgfsetdash{}{0pt}%
\pgfpathmoveto{\pgfqpoint{2.105862in}{2.075552in}}%
\pgfpathcurveto{\pgfqpoint{2.114098in}{2.075552in}}{\pgfqpoint{2.121998in}{2.078824in}}{\pgfqpoint{2.127822in}{2.084648in}}%
\pgfpathcurveto{\pgfqpoint{2.133646in}{2.090472in}}{\pgfqpoint{2.136918in}{2.098372in}}{\pgfqpoint{2.136918in}{2.106608in}}%
\pgfpathcurveto{\pgfqpoint{2.136918in}{2.114844in}}{\pgfqpoint{2.133646in}{2.122745in}}{\pgfqpoint{2.127822in}{2.128568in}}%
\pgfpathcurveto{\pgfqpoint{2.121998in}{2.134392in}}{\pgfqpoint{2.114098in}{2.137665in}}{\pgfqpoint{2.105862in}{2.137665in}}%
\pgfpathcurveto{\pgfqpoint{2.097626in}{2.137665in}}{\pgfqpoint{2.089725in}{2.134392in}}{\pgfqpoint{2.083902in}{2.128568in}}%
\pgfpathcurveto{\pgfqpoint{2.078078in}{2.122745in}}{\pgfqpoint{2.074805in}{2.114844in}}{\pgfqpoint{2.074805in}{2.106608in}}%
\pgfpathcurveto{\pgfqpoint{2.074805in}{2.098372in}}{\pgfqpoint{2.078078in}{2.090472in}}{\pgfqpoint{2.083902in}{2.084648in}}%
\pgfpathcurveto{\pgfqpoint{2.089725in}{2.078824in}}{\pgfqpoint{2.097626in}{2.075552in}}{\pgfqpoint{2.105862in}{2.075552in}}%
\pgfpathclose%
\pgfusepath{stroke,fill}%
\end{pgfscope}%
\begin{pgfscope}%
\pgfpathrectangle{\pgfqpoint{0.100000in}{0.212622in}}{\pgfqpoint{3.696000in}{3.696000in}}%
\pgfusepath{clip}%
\pgfsetbuttcap%
\pgfsetroundjoin%
\definecolor{currentfill}{rgb}{0.121569,0.466667,0.705882}%
\pgfsetfillcolor{currentfill}%
\pgfsetfillopacity{0.605502}%
\pgfsetlinewidth{1.003750pt}%
\definecolor{currentstroke}{rgb}{0.121569,0.466667,0.705882}%
\pgfsetstrokecolor{currentstroke}%
\pgfsetstrokeopacity{0.605502}%
\pgfsetdash{}{0pt}%
\pgfpathmoveto{\pgfqpoint{2.106294in}{2.073445in}}%
\pgfpathcurveto{\pgfqpoint{2.114530in}{2.073445in}}{\pgfqpoint{2.122430in}{2.076717in}}{\pgfqpoint{2.128254in}{2.082541in}}%
\pgfpathcurveto{\pgfqpoint{2.134078in}{2.088365in}}{\pgfqpoint{2.137351in}{2.096265in}}{\pgfqpoint{2.137351in}{2.104502in}}%
\pgfpathcurveto{\pgfqpoint{2.137351in}{2.112738in}}{\pgfqpoint{2.134078in}{2.120638in}}{\pgfqpoint{2.128254in}{2.126462in}}%
\pgfpathcurveto{\pgfqpoint{2.122430in}{2.132286in}}{\pgfqpoint{2.114530in}{2.135558in}}{\pgfqpoint{2.106294in}{2.135558in}}%
\pgfpathcurveto{\pgfqpoint{2.098058in}{2.135558in}}{\pgfqpoint{2.090158in}{2.132286in}}{\pgfqpoint{2.084334in}{2.126462in}}%
\pgfpathcurveto{\pgfqpoint{2.078510in}{2.120638in}}{\pgfqpoint{2.075238in}{2.112738in}}{\pgfqpoint{2.075238in}{2.104502in}}%
\pgfpathcurveto{\pgfqpoint{2.075238in}{2.096265in}}{\pgfqpoint{2.078510in}{2.088365in}}{\pgfqpoint{2.084334in}{2.082541in}}%
\pgfpathcurveto{\pgfqpoint{2.090158in}{2.076717in}}{\pgfqpoint{2.098058in}{2.073445in}}{\pgfqpoint{2.106294in}{2.073445in}}%
\pgfpathclose%
\pgfusepath{stroke,fill}%
\end{pgfscope}%
\begin{pgfscope}%
\pgfpathrectangle{\pgfqpoint{0.100000in}{0.212622in}}{\pgfqpoint{3.696000in}{3.696000in}}%
\pgfusepath{clip}%
\pgfsetbuttcap%
\pgfsetroundjoin%
\definecolor{currentfill}{rgb}{0.121569,0.466667,0.705882}%
\pgfsetfillcolor{currentfill}%
\pgfsetfillopacity{0.605523}%
\pgfsetlinewidth{1.003750pt}%
\definecolor{currentstroke}{rgb}{0.121569,0.466667,0.705882}%
\pgfsetstrokecolor{currentstroke}%
\pgfsetstrokeopacity{0.605523}%
\pgfsetdash{}{0pt}%
\pgfpathmoveto{\pgfqpoint{0.939611in}{1.648738in}}%
\pgfpathcurveto{\pgfqpoint{0.947847in}{1.648738in}}{\pgfqpoint{0.955747in}{1.652011in}}{\pgfqpoint{0.961571in}{1.657835in}}%
\pgfpathcurveto{\pgfqpoint{0.967395in}{1.663659in}}{\pgfqpoint{0.970667in}{1.671559in}}{\pgfqpoint{0.970667in}{1.679795in}}%
\pgfpathcurveto{\pgfqpoint{0.970667in}{1.688031in}}{\pgfqpoint{0.967395in}{1.695931in}}{\pgfqpoint{0.961571in}{1.701755in}}%
\pgfpathcurveto{\pgfqpoint{0.955747in}{1.707579in}}{\pgfqpoint{0.947847in}{1.710851in}}{\pgfqpoint{0.939611in}{1.710851in}}%
\pgfpathcurveto{\pgfqpoint{0.931374in}{1.710851in}}{\pgfqpoint{0.923474in}{1.707579in}}{\pgfqpoint{0.917650in}{1.701755in}}%
\pgfpathcurveto{\pgfqpoint{0.911826in}{1.695931in}}{\pgfqpoint{0.908554in}{1.688031in}}{\pgfqpoint{0.908554in}{1.679795in}}%
\pgfpathcurveto{\pgfqpoint{0.908554in}{1.671559in}}{\pgfqpoint{0.911826in}{1.663659in}}{\pgfqpoint{0.917650in}{1.657835in}}%
\pgfpathcurveto{\pgfqpoint{0.923474in}{1.652011in}}{\pgfqpoint{0.931374in}{1.648738in}}{\pgfqpoint{0.939611in}{1.648738in}}%
\pgfpathclose%
\pgfusepath{stroke,fill}%
\end{pgfscope}%
\begin{pgfscope}%
\pgfpathrectangle{\pgfqpoint{0.100000in}{0.212622in}}{\pgfqpoint{3.696000in}{3.696000in}}%
\pgfusepath{clip}%
\pgfsetbuttcap%
\pgfsetroundjoin%
\definecolor{currentfill}{rgb}{0.121569,0.466667,0.705882}%
\pgfsetfillcolor{currentfill}%
\pgfsetfillopacity{0.605616}%
\pgfsetlinewidth{1.003750pt}%
\definecolor{currentstroke}{rgb}{0.121569,0.466667,0.705882}%
\pgfsetstrokecolor{currentstroke}%
\pgfsetstrokeopacity{0.605616}%
\pgfsetdash{}{0pt}%
\pgfpathmoveto{\pgfqpoint{0.653848in}{1.205825in}}%
\pgfpathcurveto{\pgfqpoint{0.662084in}{1.205825in}}{\pgfqpoint{0.669984in}{1.209097in}}{\pgfqpoint{0.675808in}{1.214921in}}%
\pgfpathcurveto{\pgfqpoint{0.681632in}{1.220745in}}{\pgfqpoint{0.684904in}{1.228645in}}{\pgfqpoint{0.684904in}{1.236881in}}%
\pgfpathcurveto{\pgfqpoint{0.684904in}{1.245117in}}{\pgfqpoint{0.681632in}{1.253017in}}{\pgfqpoint{0.675808in}{1.258841in}}%
\pgfpathcurveto{\pgfqpoint{0.669984in}{1.264665in}}{\pgfqpoint{0.662084in}{1.267938in}}{\pgfqpoint{0.653848in}{1.267938in}}%
\pgfpathcurveto{\pgfqpoint{0.645612in}{1.267938in}}{\pgfqpoint{0.637712in}{1.264665in}}{\pgfqpoint{0.631888in}{1.258841in}}%
\pgfpathcurveto{\pgfqpoint{0.626064in}{1.253017in}}{\pgfqpoint{0.622791in}{1.245117in}}{\pgfqpoint{0.622791in}{1.236881in}}%
\pgfpathcurveto{\pgfqpoint{0.622791in}{1.228645in}}{\pgfqpoint{0.626064in}{1.220745in}}{\pgfqpoint{0.631888in}{1.214921in}}%
\pgfpathcurveto{\pgfqpoint{0.637712in}{1.209097in}}{\pgfqpoint{0.645612in}{1.205825in}}{\pgfqpoint{0.653848in}{1.205825in}}%
\pgfpathclose%
\pgfusepath{stroke,fill}%
\end{pgfscope}%
\begin{pgfscope}%
\pgfpathrectangle{\pgfqpoint{0.100000in}{0.212622in}}{\pgfqpoint{3.696000in}{3.696000in}}%
\pgfusepath{clip}%
\pgfsetbuttcap%
\pgfsetroundjoin%
\definecolor{currentfill}{rgb}{0.121569,0.466667,0.705882}%
\pgfsetfillcolor{currentfill}%
\pgfsetfillopacity{0.606148}%
\pgfsetlinewidth{1.003750pt}%
\definecolor{currentstroke}{rgb}{0.121569,0.466667,0.705882}%
\pgfsetstrokecolor{currentstroke}%
\pgfsetstrokeopacity{0.606148}%
\pgfsetdash{}{0pt}%
\pgfpathmoveto{\pgfqpoint{0.938208in}{1.646250in}}%
\pgfpathcurveto{\pgfqpoint{0.946444in}{1.646250in}}{\pgfqpoint{0.954344in}{1.649523in}}{\pgfqpoint{0.960168in}{1.655346in}}%
\pgfpathcurveto{\pgfqpoint{0.965992in}{1.661170in}}{\pgfqpoint{0.969264in}{1.669070in}}{\pgfqpoint{0.969264in}{1.677307in}}%
\pgfpathcurveto{\pgfqpoint{0.969264in}{1.685543in}}{\pgfqpoint{0.965992in}{1.693443in}}{\pgfqpoint{0.960168in}{1.699267in}}%
\pgfpathcurveto{\pgfqpoint{0.954344in}{1.705091in}}{\pgfqpoint{0.946444in}{1.708363in}}{\pgfqpoint{0.938208in}{1.708363in}}%
\pgfpathcurveto{\pgfqpoint{0.929971in}{1.708363in}}{\pgfqpoint{0.922071in}{1.705091in}}{\pgfqpoint{0.916247in}{1.699267in}}%
\pgfpathcurveto{\pgfqpoint{0.910423in}{1.693443in}}{\pgfqpoint{0.907151in}{1.685543in}}{\pgfqpoint{0.907151in}{1.677307in}}%
\pgfpathcurveto{\pgfqpoint{0.907151in}{1.669070in}}{\pgfqpoint{0.910423in}{1.661170in}}{\pgfqpoint{0.916247in}{1.655346in}}%
\pgfpathcurveto{\pgfqpoint{0.922071in}{1.649523in}}{\pgfqpoint{0.929971in}{1.646250in}}{\pgfqpoint{0.938208in}{1.646250in}}%
\pgfpathclose%
\pgfusepath{stroke,fill}%
\end{pgfscope}%
\begin{pgfscope}%
\pgfpathrectangle{\pgfqpoint{0.100000in}{0.212622in}}{\pgfqpoint{3.696000in}{3.696000in}}%
\pgfusepath{clip}%
\pgfsetbuttcap%
\pgfsetroundjoin%
\definecolor{currentfill}{rgb}{0.121569,0.466667,0.705882}%
\pgfsetfillcolor{currentfill}%
\pgfsetfillopacity{0.606433}%
\pgfsetlinewidth{1.003750pt}%
\definecolor{currentstroke}{rgb}{0.121569,0.466667,0.705882}%
\pgfsetstrokecolor{currentstroke}%
\pgfsetstrokeopacity{0.606433}%
\pgfsetdash{}{0pt}%
\pgfpathmoveto{\pgfqpoint{2.106736in}{2.070664in}}%
\pgfpathcurveto{\pgfqpoint{2.114973in}{2.070664in}}{\pgfqpoint{2.122873in}{2.073937in}}{\pgfqpoint{2.128697in}{2.079760in}}%
\pgfpathcurveto{\pgfqpoint{2.134521in}{2.085584in}}{\pgfqpoint{2.137793in}{2.093484in}}{\pgfqpoint{2.137793in}{2.101721in}}%
\pgfpathcurveto{\pgfqpoint{2.137793in}{2.109957in}}{\pgfqpoint{2.134521in}{2.117857in}}{\pgfqpoint{2.128697in}{2.123681in}}%
\pgfpathcurveto{\pgfqpoint{2.122873in}{2.129505in}}{\pgfqpoint{2.114973in}{2.132777in}}{\pgfqpoint{2.106736in}{2.132777in}}%
\pgfpathcurveto{\pgfqpoint{2.098500in}{2.132777in}}{\pgfqpoint{2.090600in}{2.129505in}}{\pgfqpoint{2.084776in}{2.123681in}}%
\pgfpathcurveto{\pgfqpoint{2.078952in}{2.117857in}}{\pgfqpoint{2.075680in}{2.109957in}}{\pgfqpoint{2.075680in}{2.101721in}}%
\pgfpathcurveto{\pgfqpoint{2.075680in}{2.093484in}}{\pgfqpoint{2.078952in}{2.085584in}}{\pgfqpoint{2.084776in}{2.079760in}}%
\pgfpathcurveto{\pgfqpoint{2.090600in}{2.073937in}}{\pgfqpoint{2.098500in}{2.070664in}}{\pgfqpoint{2.106736in}{2.070664in}}%
\pgfpathclose%
\pgfusepath{stroke,fill}%
\end{pgfscope}%
\begin{pgfscope}%
\pgfpathrectangle{\pgfqpoint{0.100000in}{0.212622in}}{\pgfqpoint{3.696000in}{3.696000in}}%
\pgfusepath{clip}%
\pgfsetbuttcap%
\pgfsetroundjoin%
\definecolor{currentfill}{rgb}{0.121569,0.466667,0.705882}%
\pgfsetfillcolor{currentfill}%
\pgfsetfillopacity{0.606460}%
\pgfsetlinewidth{1.003750pt}%
\definecolor{currentstroke}{rgb}{0.121569,0.466667,0.705882}%
\pgfsetstrokecolor{currentstroke}%
\pgfsetstrokeopacity{0.606460}%
\pgfsetdash{}{0pt}%
\pgfpathmoveto{\pgfqpoint{0.937322in}{1.644827in}}%
\pgfpathcurveto{\pgfqpoint{0.945558in}{1.644827in}}{\pgfqpoint{0.953458in}{1.648099in}}{\pgfqpoint{0.959282in}{1.653923in}}%
\pgfpathcurveto{\pgfqpoint{0.965106in}{1.659747in}}{\pgfqpoint{0.968378in}{1.667647in}}{\pgfqpoint{0.968378in}{1.675883in}}%
\pgfpathcurveto{\pgfqpoint{0.968378in}{1.684119in}}{\pgfqpoint{0.965106in}{1.692019in}}{\pgfqpoint{0.959282in}{1.697843in}}%
\pgfpathcurveto{\pgfqpoint{0.953458in}{1.703667in}}{\pgfqpoint{0.945558in}{1.706940in}}{\pgfqpoint{0.937322in}{1.706940in}}%
\pgfpathcurveto{\pgfqpoint{0.929086in}{1.706940in}}{\pgfqpoint{0.921186in}{1.703667in}}{\pgfqpoint{0.915362in}{1.697843in}}%
\pgfpathcurveto{\pgfqpoint{0.909538in}{1.692019in}}{\pgfqpoint{0.906265in}{1.684119in}}{\pgfqpoint{0.906265in}{1.675883in}}%
\pgfpathcurveto{\pgfqpoint{0.906265in}{1.667647in}}{\pgfqpoint{0.909538in}{1.659747in}}{\pgfqpoint{0.915362in}{1.653923in}}%
\pgfpathcurveto{\pgfqpoint{0.921186in}{1.648099in}}{\pgfqpoint{0.929086in}{1.644827in}}{\pgfqpoint{0.937322in}{1.644827in}}%
\pgfpathclose%
\pgfusepath{stroke,fill}%
\end{pgfscope}%
\begin{pgfscope}%
\pgfpathrectangle{\pgfqpoint{0.100000in}{0.212622in}}{\pgfqpoint{3.696000in}{3.696000in}}%
\pgfusepath{clip}%
\pgfsetbuttcap%
\pgfsetroundjoin%
\definecolor{currentfill}{rgb}{0.121569,0.466667,0.705882}%
\pgfsetfillcolor{currentfill}%
\pgfsetfillopacity{0.606888}%
\pgfsetlinewidth{1.003750pt}%
\definecolor{currentstroke}{rgb}{0.121569,0.466667,0.705882}%
\pgfsetstrokecolor{currentstroke}%
\pgfsetstrokeopacity{0.606888}%
\pgfsetdash{}{0pt}%
\pgfpathmoveto{\pgfqpoint{0.870062in}{1.487791in}}%
\pgfpathcurveto{\pgfqpoint{0.878298in}{1.487791in}}{\pgfqpoint{0.886198in}{1.491063in}}{\pgfqpoint{0.892022in}{1.496887in}}%
\pgfpathcurveto{\pgfqpoint{0.897846in}{1.502711in}}{\pgfqpoint{0.901118in}{1.510611in}}{\pgfqpoint{0.901118in}{1.518848in}}%
\pgfpathcurveto{\pgfqpoint{0.901118in}{1.527084in}}{\pgfqpoint{0.897846in}{1.534984in}}{\pgfqpoint{0.892022in}{1.540808in}}%
\pgfpathcurveto{\pgfqpoint{0.886198in}{1.546632in}}{\pgfqpoint{0.878298in}{1.549904in}}{\pgfqpoint{0.870062in}{1.549904in}}%
\pgfpathcurveto{\pgfqpoint{0.861826in}{1.549904in}}{\pgfqpoint{0.853926in}{1.546632in}}{\pgfqpoint{0.848102in}{1.540808in}}%
\pgfpathcurveto{\pgfqpoint{0.842278in}{1.534984in}}{\pgfqpoint{0.839005in}{1.527084in}}{\pgfqpoint{0.839005in}{1.518848in}}%
\pgfpathcurveto{\pgfqpoint{0.839005in}{1.510611in}}{\pgfqpoint{0.842278in}{1.502711in}}{\pgfqpoint{0.848102in}{1.496887in}}%
\pgfpathcurveto{\pgfqpoint{0.853926in}{1.491063in}}{\pgfqpoint{0.861826in}{1.487791in}}{\pgfqpoint{0.870062in}{1.487791in}}%
\pgfpathclose%
\pgfusepath{stroke,fill}%
\end{pgfscope}%
\begin{pgfscope}%
\pgfpathrectangle{\pgfqpoint{0.100000in}{0.212622in}}{\pgfqpoint{3.696000in}{3.696000in}}%
\pgfusepath{clip}%
\pgfsetbuttcap%
\pgfsetroundjoin%
\definecolor{currentfill}{rgb}{0.121569,0.466667,0.705882}%
\pgfsetfillcolor{currentfill}%
\pgfsetfillopacity{0.607034}%
\pgfsetlinewidth{1.003750pt}%
\definecolor{currentstroke}{rgb}{0.121569,0.466667,0.705882}%
\pgfsetstrokecolor{currentstroke}%
\pgfsetstrokeopacity{0.607034}%
\pgfsetdash{}{0pt}%
\pgfpathmoveto{\pgfqpoint{0.935833in}{1.642104in}}%
\pgfpathcurveto{\pgfqpoint{0.944069in}{1.642104in}}{\pgfqpoint{0.951969in}{1.645377in}}{\pgfqpoint{0.957793in}{1.651201in}}%
\pgfpathcurveto{\pgfqpoint{0.963617in}{1.657025in}}{\pgfqpoint{0.966889in}{1.664925in}}{\pgfqpoint{0.966889in}{1.673161in}}%
\pgfpathcurveto{\pgfqpoint{0.966889in}{1.681397in}}{\pgfqpoint{0.963617in}{1.689297in}}{\pgfqpoint{0.957793in}{1.695121in}}%
\pgfpathcurveto{\pgfqpoint{0.951969in}{1.700945in}}{\pgfqpoint{0.944069in}{1.704217in}}{\pgfqpoint{0.935833in}{1.704217in}}%
\pgfpathcurveto{\pgfqpoint{0.927597in}{1.704217in}}{\pgfqpoint{0.919696in}{1.700945in}}{\pgfqpoint{0.913873in}{1.695121in}}%
\pgfpathcurveto{\pgfqpoint{0.908049in}{1.689297in}}{\pgfqpoint{0.904776in}{1.681397in}}{\pgfqpoint{0.904776in}{1.673161in}}%
\pgfpathcurveto{\pgfqpoint{0.904776in}{1.664925in}}{\pgfqpoint{0.908049in}{1.657025in}}{\pgfqpoint{0.913873in}{1.651201in}}%
\pgfpathcurveto{\pgfqpoint{0.919696in}{1.645377in}}{\pgfqpoint{0.927597in}{1.642104in}}{\pgfqpoint{0.935833in}{1.642104in}}%
\pgfpathclose%
\pgfusepath{stroke,fill}%
\end{pgfscope}%
\begin{pgfscope}%
\pgfpathrectangle{\pgfqpoint{0.100000in}{0.212622in}}{\pgfqpoint{3.696000in}{3.696000in}}%
\pgfusepath{clip}%
\pgfsetbuttcap%
\pgfsetroundjoin%
\definecolor{currentfill}{rgb}{0.121569,0.466667,0.705882}%
\pgfsetfillcolor{currentfill}%
\pgfsetfillopacity{0.607422}%
\pgfsetlinewidth{1.003750pt}%
\definecolor{currentstroke}{rgb}{0.121569,0.466667,0.705882}%
\pgfsetstrokecolor{currentstroke}%
\pgfsetstrokeopacity{0.607422}%
\pgfsetdash{}{0pt}%
\pgfpathmoveto{\pgfqpoint{0.934465in}{1.639705in}}%
\pgfpathcurveto{\pgfqpoint{0.942701in}{1.639705in}}{\pgfqpoint{0.950601in}{1.642977in}}{\pgfqpoint{0.956425in}{1.648801in}}%
\pgfpathcurveto{\pgfqpoint{0.962249in}{1.654625in}}{\pgfqpoint{0.965521in}{1.662525in}}{\pgfqpoint{0.965521in}{1.670761in}}%
\pgfpathcurveto{\pgfqpoint{0.965521in}{1.678997in}}{\pgfqpoint{0.962249in}{1.686897in}}{\pgfqpoint{0.956425in}{1.692721in}}%
\pgfpathcurveto{\pgfqpoint{0.950601in}{1.698545in}}{\pgfqpoint{0.942701in}{1.701818in}}{\pgfqpoint{0.934465in}{1.701818in}}%
\pgfpathcurveto{\pgfqpoint{0.926228in}{1.701818in}}{\pgfqpoint{0.918328in}{1.698545in}}{\pgfqpoint{0.912504in}{1.692721in}}%
\pgfpathcurveto{\pgfqpoint{0.906680in}{1.686897in}}{\pgfqpoint{0.903408in}{1.678997in}}{\pgfqpoint{0.903408in}{1.670761in}}%
\pgfpathcurveto{\pgfqpoint{0.903408in}{1.662525in}}{\pgfqpoint{0.906680in}{1.654625in}}{\pgfqpoint{0.912504in}{1.648801in}}%
\pgfpathcurveto{\pgfqpoint{0.918328in}{1.642977in}}{\pgfqpoint{0.926228in}{1.639705in}}{\pgfqpoint{0.934465in}{1.639705in}}%
\pgfpathclose%
\pgfusepath{stroke,fill}%
\end{pgfscope}%
\begin{pgfscope}%
\pgfpathrectangle{\pgfqpoint{0.100000in}{0.212622in}}{\pgfqpoint{3.696000in}{3.696000in}}%
\pgfusepath{clip}%
\pgfsetbuttcap%
\pgfsetroundjoin%
\definecolor{currentfill}{rgb}{0.121569,0.466667,0.705882}%
\pgfsetfillcolor{currentfill}%
\pgfsetfillopacity{0.607776}%
\pgfsetlinewidth{1.003750pt}%
\definecolor{currentstroke}{rgb}{0.121569,0.466667,0.705882}%
\pgfsetstrokecolor{currentstroke}%
\pgfsetstrokeopacity{0.607776}%
\pgfsetdash{}{0pt}%
\pgfpathmoveto{\pgfqpoint{0.933657in}{1.637440in}}%
\pgfpathcurveto{\pgfqpoint{0.941893in}{1.637440in}}{\pgfqpoint{0.949793in}{1.640712in}}{\pgfqpoint{0.955617in}{1.646536in}}%
\pgfpathcurveto{\pgfqpoint{0.961441in}{1.652360in}}{\pgfqpoint{0.964713in}{1.660260in}}{\pgfqpoint{0.964713in}{1.668496in}}%
\pgfpathcurveto{\pgfqpoint{0.964713in}{1.676733in}}{\pgfqpoint{0.961441in}{1.684633in}}{\pgfqpoint{0.955617in}{1.690457in}}%
\pgfpathcurveto{\pgfqpoint{0.949793in}{1.696280in}}{\pgfqpoint{0.941893in}{1.699553in}}{\pgfqpoint{0.933657in}{1.699553in}}%
\pgfpathcurveto{\pgfqpoint{0.925421in}{1.699553in}}{\pgfqpoint{0.917521in}{1.696280in}}{\pgfqpoint{0.911697in}{1.690457in}}%
\pgfpathcurveto{\pgfqpoint{0.905873in}{1.684633in}}{\pgfqpoint{0.902600in}{1.676733in}}{\pgfqpoint{0.902600in}{1.668496in}}%
\pgfpathcurveto{\pgfqpoint{0.902600in}{1.660260in}}{\pgfqpoint{0.905873in}{1.652360in}}{\pgfqpoint{0.911697in}{1.646536in}}%
\pgfpathcurveto{\pgfqpoint{0.917521in}{1.640712in}}{\pgfqpoint{0.925421in}{1.637440in}}{\pgfqpoint{0.933657in}{1.637440in}}%
\pgfpathclose%
\pgfusepath{stroke,fill}%
\end{pgfscope}%
\begin{pgfscope}%
\pgfpathrectangle{\pgfqpoint{0.100000in}{0.212622in}}{\pgfqpoint{3.696000in}{3.696000in}}%
\pgfusepath{clip}%
\pgfsetbuttcap%
\pgfsetroundjoin%
\definecolor{currentfill}{rgb}{0.121569,0.466667,0.705882}%
\pgfsetfillcolor{currentfill}%
\pgfsetfillopacity{0.607889}%
\pgfsetlinewidth{1.003750pt}%
\definecolor{currentstroke}{rgb}{0.121569,0.466667,0.705882}%
\pgfsetstrokecolor{currentstroke}%
\pgfsetstrokeopacity{0.607889}%
\pgfsetdash{}{0pt}%
\pgfpathmoveto{\pgfqpoint{2.107971in}{2.065821in}}%
\pgfpathcurveto{\pgfqpoint{2.116208in}{2.065821in}}{\pgfqpoint{2.124108in}{2.069093in}}{\pgfqpoint{2.129932in}{2.074917in}}%
\pgfpathcurveto{\pgfqpoint{2.135756in}{2.080741in}}{\pgfqpoint{2.139028in}{2.088641in}}{\pgfqpoint{2.139028in}{2.096877in}}%
\pgfpathcurveto{\pgfqpoint{2.139028in}{2.105114in}}{\pgfqpoint{2.135756in}{2.113014in}}{\pgfqpoint{2.129932in}{2.118838in}}%
\pgfpathcurveto{\pgfqpoint{2.124108in}{2.124661in}}{\pgfqpoint{2.116208in}{2.127934in}}{\pgfqpoint{2.107971in}{2.127934in}}%
\pgfpathcurveto{\pgfqpoint{2.099735in}{2.127934in}}{\pgfqpoint{2.091835in}{2.124661in}}{\pgfqpoint{2.086011in}{2.118838in}}%
\pgfpathcurveto{\pgfqpoint{2.080187in}{2.113014in}}{\pgfqpoint{2.076915in}{2.105114in}}{\pgfqpoint{2.076915in}{2.096877in}}%
\pgfpathcurveto{\pgfqpoint{2.076915in}{2.088641in}}{\pgfqpoint{2.080187in}{2.080741in}}{\pgfqpoint{2.086011in}{2.074917in}}%
\pgfpathcurveto{\pgfqpoint{2.091835in}{2.069093in}}{\pgfqpoint{2.099735in}{2.065821in}}{\pgfqpoint{2.107971in}{2.065821in}}%
\pgfpathclose%
\pgfusepath{stroke,fill}%
\end{pgfscope}%
\begin{pgfscope}%
\pgfpathrectangle{\pgfqpoint{0.100000in}{0.212622in}}{\pgfqpoint{3.696000in}{3.696000in}}%
\pgfusepath{clip}%
\pgfsetbuttcap%
\pgfsetroundjoin%
\definecolor{currentfill}{rgb}{0.121569,0.466667,0.705882}%
\pgfsetfillcolor{currentfill}%
\pgfsetfillopacity{0.608418}%
\pgfsetlinewidth{1.003750pt}%
\definecolor{currentstroke}{rgb}{0.121569,0.466667,0.705882}%
\pgfsetstrokecolor{currentstroke}%
\pgfsetstrokeopacity{0.608418}%
\pgfsetdash{}{0pt}%
\pgfpathmoveto{\pgfqpoint{0.931672in}{1.633826in}}%
\pgfpathcurveto{\pgfqpoint{0.939908in}{1.633826in}}{\pgfqpoint{0.947808in}{1.637099in}}{\pgfqpoint{0.953632in}{1.642923in}}%
\pgfpathcurveto{\pgfqpoint{0.959456in}{1.648747in}}{\pgfqpoint{0.962728in}{1.656647in}}{\pgfqpoint{0.962728in}{1.664883in}}%
\pgfpathcurveto{\pgfqpoint{0.962728in}{1.673119in}}{\pgfqpoint{0.959456in}{1.681019in}}{\pgfqpoint{0.953632in}{1.686843in}}%
\pgfpathcurveto{\pgfqpoint{0.947808in}{1.692667in}}{\pgfqpoint{0.939908in}{1.695939in}}{\pgfqpoint{0.931672in}{1.695939in}}%
\pgfpathcurveto{\pgfqpoint{0.923435in}{1.695939in}}{\pgfqpoint{0.915535in}{1.692667in}}{\pgfqpoint{0.909711in}{1.686843in}}%
\pgfpathcurveto{\pgfqpoint{0.903887in}{1.681019in}}{\pgfqpoint{0.900615in}{1.673119in}}{\pgfqpoint{0.900615in}{1.664883in}}%
\pgfpathcurveto{\pgfqpoint{0.900615in}{1.656647in}}{\pgfqpoint{0.903887in}{1.648747in}}{\pgfqpoint{0.909711in}{1.642923in}}%
\pgfpathcurveto{\pgfqpoint{0.915535in}{1.637099in}}{\pgfqpoint{0.923435in}{1.633826in}}{\pgfqpoint{0.931672in}{1.633826in}}%
\pgfpathclose%
\pgfusepath{stroke,fill}%
\end{pgfscope}%
\begin{pgfscope}%
\pgfpathrectangle{\pgfqpoint{0.100000in}{0.212622in}}{\pgfqpoint{3.696000in}{3.696000in}}%
\pgfusepath{clip}%
\pgfsetbuttcap%
\pgfsetroundjoin%
\definecolor{currentfill}{rgb}{0.121569,0.466667,0.705882}%
\pgfsetfillcolor{currentfill}%
\pgfsetfillopacity{0.608834}%
\pgfsetlinewidth{1.003750pt}%
\definecolor{currentstroke}{rgb}{0.121569,0.466667,0.705882}%
\pgfsetstrokecolor{currentstroke}%
\pgfsetstrokeopacity{0.608834}%
\pgfsetdash{}{0pt}%
\pgfpathmoveto{\pgfqpoint{0.669070in}{1.204320in}}%
\pgfpathcurveto{\pgfqpoint{0.677306in}{1.204320in}}{\pgfqpoint{0.685206in}{1.207592in}}{\pgfqpoint{0.691030in}{1.213416in}}%
\pgfpathcurveto{\pgfqpoint{0.696854in}{1.219240in}}{\pgfqpoint{0.700127in}{1.227140in}}{\pgfqpoint{0.700127in}{1.235377in}}%
\pgfpathcurveto{\pgfqpoint{0.700127in}{1.243613in}}{\pgfqpoint{0.696854in}{1.251513in}}{\pgfqpoint{0.691030in}{1.257337in}}%
\pgfpathcurveto{\pgfqpoint{0.685206in}{1.263161in}}{\pgfqpoint{0.677306in}{1.266433in}}{\pgfqpoint{0.669070in}{1.266433in}}%
\pgfpathcurveto{\pgfqpoint{0.660834in}{1.266433in}}{\pgfqpoint{0.652934in}{1.263161in}}{\pgfqpoint{0.647110in}{1.257337in}}%
\pgfpathcurveto{\pgfqpoint{0.641286in}{1.251513in}}{\pgfqpoint{0.638014in}{1.243613in}}{\pgfqpoint{0.638014in}{1.235377in}}%
\pgfpathcurveto{\pgfqpoint{0.638014in}{1.227140in}}{\pgfqpoint{0.641286in}{1.219240in}}{\pgfqpoint{0.647110in}{1.213416in}}%
\pgfpathcurveto{\pgfqpoint{0.652934in}{1.207592in}}{\pgfqpoint{0.660834in}{1.204320in}}{\pgfqpoint{0.669070in}{1.204320in}}%
\pgfpathclose%
\pgfusepath{stroke,fill}%
\end{pgfscope}%
\begin{pgfscope}%
\pgfpathrectangle{\pgfqpoint{0.100000in}{0.212622in}}{\pgfqpoint{3.696000in}{3.696000in}}%
\pgfusepath{clip}%
\pgfsetbuttcap%
\pgfsetroundjoin%
\definecolor{currentfill}{rgb}{0.121569,0.466667,0.705882}%
\pgfsetfillcolor{currentfill}%
\pgfsetfillopacity{0.609524}%
\pgfsetlinewidth{1.003750pt}%
\definecolor{currentstroke}{rgb}{0.121569,0.466667,0.705882}%
\pgfsetstrokecolor{currentstroke}%
\pgfsetstrokeopacity{0.609524}%
\pgfsetdash{}{0pt}%
\pgfpathmoveto{\pgfqpoint{0.928019in}{1.627090in}}%
\pgfpathcurveto{\pgfqpoint{0.936255in}{1.627090in}}{\pgfqpoint{0.944155in}{1.630362in}}{\pgfqpoint{0.949979in}{1.636186in}}%
\pgfpathcurveto{\pgfqpoint{0.955803in}{1.642010in}}{\pgfqpoint{0.959075in}{1.649910in}}{\pgfqpoint{0.959075in}{1.658146in}}%
\pgfpathcurveto{\pgfqpoint{0.959075in}{1.666383in}}{\pgfqpoint{0.955803in}{1.674283in}}{\pgfqpoint{0.949979in}{1.680107in}}%
\pgfpathcurveto{\pgfqpoint{0.944155in}{1.685931in}}{\pgfqpoint{0.936255in}{1.689203in}}{\pgfqpoint{0.928019in}{1.689203in}}%
\pgfpathcurveto{\pgfqpoint{0.919783in}{1.689203in}}{\pgfqpoint{0.911883in}{1.685931in}}{\pgfqpoint{0.906059in}{1.680107in}}%
\pgfpathcurveto{\pgfqpoint{0.900235in}{1.674283in}}{\pgfqpoint{0.896962in}{1.666383in}}{\pgfqpoint{0.896962in}{1.658146in}}%
\pgfpathcurveto{\pgfqpoint{0.896962in}{1.649910in}}{\pgfqpoint{0.900235in}{1.642010in}}{\pgfqpoint{0.906059in}{1.636186in}}%
\pgfpathcurveto{\pgfqpoint{0.911883in}{1.630362in}}{\pgfqpoint{0.919783in}{1.627090in}}{\pgfqpoint{0.928019in}{1.627090in}}%
\pgfpathclose%
\pgfusepath{stroke,fill}%
\end{pgfscope}%
\begin{pgfscope}%
\pgfpathrectangle{\pgfqpoint{0.100000in}{0.212622in}}{\pgfqpoint{3.696000in}{3.696000in}}%
\pgfusepath{clip}%
\pgfsetbuttcap%
\pgfsetroundjoin%
\definecolor{currentfill}{rgb}{0.121569,0.466667,0.705882}%
\pgfsetfillcolor{currentfill}%
\pgfsetfillopacity{0.609793}%
\pgfsetlinewidth{1.003750pt}%
\definecolor{currentstroke}{rgb}{0.121569,0.466667,0.705882}%
\pgfsetstrokecolor{currentstroke}%
\pgfsetstrokeopacity{0.609793}%
\pgfsetdash{}{0pt}%
\pgfpathmoveto{\pgfqpoint{2.109164in}{2.060881in}}%
\pgfpathcurveto{\pgfqpoint{2.117400in}{2.060881in}}{\pgfqpoint{2.125300in}{2.064154in}}{\pgfqpoint{2.131124in}{2.069978in}}%
\pgfpathcurveto{\pgfqpoint{2.136948in}{2.075802in}}{\pgfqpoint{2.140220in}{2.083702in}}{\pgfqpoint{2.140220in}{2.091938in}}%
\pgfpathcurveto{\pgfqpoint{2.140220in}{2.100174in}}{\pgfqpoint{2.136948in}{2.108074in}}{\pgfqpoint{2.131124in}{2.113898in}}%
\pgfpathcurveto{\pgfqpoint{2.125300in}{2.119722in}}{\pgfqpoint{2.117400in}{2.122994in}}{\pgfqpoint{2.109164in}{2.122994in}}%
\pgfpathcurveto{\pgfqpoint{2.100928in}{2.122994in}}{\pgfqpoint{2.093028in}{2.119722in}}{\pgfqpoint{2.087204in}{2.113898in}}%
\pgfpathcurveto{\pgfqpoint{2.081380in}{2.108074in}}{\pgfqpoint{2.078107in}{2.100174in}}{\pgfqpoint{2.078107in}{2.091938in}}%
\pgfpathcurveto{\pgfqpoint{2.078107in}{2.083702in}}{\pgfqpoint{2.081380in}{2.075802in}}{\pgfqpoint{2.087204in}{2.069978in}}%
\pgfpathcurveto{\pgfqpoint{2.093028in}{2.064154in}}{\pgfqpoint{2.100928in}{2.060881in}}{\pgfqpoint{2.109164in}{2.060881in}}%
\pgfpathclose%
\pgfusepath{stroke,fill}%
\end{pgfscope}%
\begin{pgfscope}%
\pgfpathrectangle{\pgfqpoint{0.100000in}{0.212622in}}{\pgfqpoint{3.696000in}{3.696000in}}%
\pgfusepath{clip}%
\pgfsetbuttcap%
\pgfsetroundjoin%
\definecolor{currentfill}{rgb}{0.121569,0.466667,0.705882}%
\pgfsetfillcolor{currentfill}%
\pgfsetfillopacity{0.610455}%
\pgfsetlinewidth{1.003750pt}%
\definecolor{currentstroke}{rgb}{0.121569,0.466667,0.705882}%
\pgfsetstrokecolor{currentstroke}%
\pgfsetstrokeopacity{0.610455}%
\pgfsetdash{}{0pt}%
\pgfpathmoveto{\pgfqpoint{0.924846in}{1.620738in}}%
\pgfpathcurveto{\pgfqpoint{0.933083in}{1.620738in}}{\pgfqpoint{0.940983in}{1.624011in}}{\pgfqpoint{0.946807in}{1.629835in}}%
\pgfpathcurveto{\pgfqpoint{0.952631in}{1.635659in}}{\pgfqpoint{0.955903in}{1.643559in}}{\pgfqpoint{0.955903in}{1.651795in}}%
\pgfpathcurveto{\pgfqpoint{0.955903in}{1.660031in}}{\pgfqpoint{0.952631in}{1.667931in}}{\pgfqpoint{0.946807in}{1.673755in}}%
\pgfpathcurveto{\pgfqpoint{0.940983in}{1.679579in}}{\pgfqpoint{0.933083in}{1.682851in}}{\pgfqpoint{0.924846in}{1.682851in}}%
\pgfpathcurveto{\pgfqpoint{0.916610in}{1.682851in}}{\pgfqpoint{0.908710in}{1.679579in}}{\pgfqpoint{0.902886in}{1.673755in}}%
\pgfpathcurveto{\pgfqpoint{0.897062in}{1.667931in}}{\pgfqpoint{0.893790in}{1.660031in}}{\pgfqpoint{0.893790in}{1.651795in}}%
\pgfpathcurveto{\pgfqpoint{0.893790in}{1.643559in}}{\pgfqpoint{0.897062in}{1.635659in}}{\pgfqpoint{0.902886in}{1.629835in}}%
\pgfpathcurveto{\pgfqpoint{0.908710in}{1.624011in}}{\pgfqpoint{0.916610in}{1.620738in}}{\pgfqpoint{0.924846in}{1.620738in}}%
\pgfpathclose%
\pgfusepath{stroke,fill}%
\end{pgfscope}%
\begin{pgfscope}%
\pgfpathrectangle{\pgfqpoint{0.100000in}{0.212622in}}{\pgfqpoint{3.696000in}{3.696000in}}%
\pgfusepath{clip}%
\pgfsetbuttcap%
\pgfsetroundjoin%
\definecolor{currentfill}{rgb}{0.121569,0.466667,0.705882}%
\pgfsetfillcolor{currentfill}%
\pgfsetfillopacity{0.611159}%
\pgfsetlinewidth{1.003750pt}%
\definecolor{currentstroke}{rgb}{0.121569,0.466667,0.705882}%
\pgfsetstrokecolor{currentstroke}%
\pgfsetstrokeopacity{0.611159}%
\pgfsetdash{}{0pt}%
\pgfpathmoveto{\pgfqpoint{0.922142in}{1.615834in}}%
\pgfpathcurveto{\pgfqpoint{0.930378in}{1.615834in}}{\pgfqpoint{0.938278in}{1.619106in}}{\pgfqpoint{0.944102in}{1.624930in}}%
\pgfpathcurveto{\pgfqpoint{0.949926in}{1.630754in}}{\pgfqpoint{0.953198in}{1.638654in}}{\pgfqpoint{0.953198in}{1.646890in}}%
\pgfpathcurveto{\pgfqpoint{0.953198in}{1.655127in}}{\pgfqpoint{0.949926in}{1.663027in}}{\pgfqpoint{0.944102in}{1.668851in}}%
\pgfpathcurveto{\pgfqpoint{0.938278in}{1.674675in}}{\pgfqpoint{0.930378in}{1.677947in}}{\pgfqpoint{0.922142in}{1.677947in}}%
\pgfpathcurveto{\pgfqpoint{0.913906in}{1.677947in}}{\pgfqpoint{0.906006in}{1.674675in}}{\pgfqpoint{0.900182in}{1.668851in}}%
\pgfpathcurveto{\pgfqpoint{0.894358in}{1.663027in}}{\pgfqpoint{0.891085in}{1.655127in}}{\pgfqpoint{0.891085in}{1.646890in}}%
\pgfpathcurveto{\pgfqpoint{0.891085in}{1.638654in}}{\pgfqpoint{0.894358in}{1.630754in}}{\pgfqpoint{0.900182in}{1.624930in}}%
\pgfpathcurveto{\pgfqpoint{0.906006in}{1.619106in}}{\pgfqpoint{0.913906in}{1.615834in}}{\pgfqpoint{0.922142in}{1.615834in}}%
\pgfpathclose%
\pgfusepath{stroke,fill}%
\end{pgfscope}%
\begin{pgfscope}%
\pgfpathrectangle{\pgfqpoint{0.100000in}{0.212622in}}{\pgfqpoint{3.696000in}{3.696000in}}%
\pgfusepath{clip}%
\pgfsetbuttcap%
\pgfsetroundjoin%
\definecolor{currentfill}{rgb}{0.121569,0.466667,0.705882}%
\pgfsetfillcolor{currentfill}%
\pgfsetfillopacity{0.611848}%
\pgfsetlinewidth{1.003750pt}%
\definecolor{currentstroke}{rgb}{0.121569,0.466667,0.705882}%
\pgfsetstrokecolor{currentstroke}%
\pgfsetstrokeopacity{0.611848}%
\pgfsetdash{}{0pt}%
\pgfpathmoveto{\pgfqpoint{0.919931in}{1.611514in}}%
\pgfpathcurveto{\pgfqpoint{0.928167in}{1.611514in}}{\pgfqpoint{0.936067in}{1.614787in}}{\pgfqpoint{0.941891in}{1.620611in}}%
\pgfpathcurveto{\pgfqpoint{0.947715in}{1.626435in}}{\pgfqpoint{0.950987in}{1.634335in}}{\pgfqpoint{0.950987in}{1.642571in}}%
\pgfpathcurveto{\pgfqpoint{0.950987in}{1.650807in}}{\pgfqpoint{0.947715in}{1.658707in}}{\pgfqpoint{0.941891in}{1.664531in}}%
\pgfpathcurveto{\pgfqpoint{0.936067in}{1.670355in}}{\pgfqpoint{0.928167in}{1.673627in}}{\pgfqpoint{0.919931in}{1.673627in}}%
\pgfpathcurveto{\pgfqpoint{0.911694in}{1.673627in}}{\pgfqpoint{0.903794in}{1.670355in}}{\pgfqpoint{0.897970in}{1.664531in}}%
\pgfpathcurveto{\pgfqpoint{0.892146in}{1.658707in}}{\pgfqpoint{0.888874in}{1.650807in}}{\pgfqpoint{0.888874in}{1.642571in}}%
\pgfpathcurveto{\pgfqpoint{0.888874in}{1.634335in}}{\pgfqpoint{0.892146in}{1.626435in}}{\pgfqpoint{0.897970in}{1.620611in}}%
\pgfpathcurveto{\pgfqpoint{0.903794in}{1.614787in}}{\pgfqpoint{0.911694in}{1.611514in}}{\pgfqpoint{0.919931in}{1.611514in}}%
\pgfpathclose%
\pgfusepath{stroke,fill}%
\end{pgfscope}%
\begin{pgfscope}%
\pgfpathrectangle{\pgfqpoint{0.100000in}{0.212622in}}{\pgfqpoint{3.696000in}{3.696000in}}%
\pgfusepath{clip}%
\pgfsetbuttcap%
\pgfsetroundjoin%
\definecolor{currentfill}{rgb}{0.121569,0.466667,0.705882}%
\pgfsetfillcolor{currentfill}%
\pgfsetfillopacity{0.611899}%
\pgfsetlinewidth{1.003750pt}%
\definecolor{currentstroke}{rgb}{0.121569,0.466667,0.705882}%
\pgfsetstrokecolor{currentstroke}%
\pgfsetstrokeopacity{0.611899}%
\pgfsetdash{}{0pt}%
\pgfpathmoveto{\pgfqpoint{2.110008in}{2.055721in}}%
\pgfpathcurveto{\pgfqpoint{2.118245in}{2.055721in}}{\pgfqpoint{2.126145in}{2.058993in}}{\pgfqpoint{2.131969in}{2.064817in}}%
\pgfpathcurveto{\pgfqpoint{2.137793in}{2.070641in}}{\pgfqpoint{2.141065in}{2.078541in}}{\pgfqpoint{2.141065in}{2.086777in}}%
\pgfpathcurveto{\pgfqpoint{2.141065in}{2.095014in}}{\pgfqpoint{2.137793in}{2.102914in}}{\pgfqpoint{2.131969in}{2.108738in}}%
\pgfpathcurveto{\pgfqpoint{2.126145in}{2.114562in}}{\pgfqpoint{2.118245in}{2.117834in}}{\pgfqpoint{2.110008in}{2.117834in}}%
\pgfpathcurveto{\pgfqpoint{2.101772in}{2.117834in}}{\pgfqpoint{2.093872in}{2.114562in}}{\pgfqpoint{2.088048in}{2.108738in}}%
\pgfpathcurveto{\pgfqpoint{2.082224in}{2.102914in}}{\pgfqpoint{2.078952in}{2.095014in}}{\pgfqpoint{2.078952in}{2.086777in}}%
\pgfpathcurveto{\pgfqpoint{2.078952in}{2.078541in}}{\pgfqpoint{2.082224in}{2.070641in}}{\pgfqpoint{2.088048in}{2.064817in}}%
\pgfpathcurveto{\pgfqpoint{2.093872in}{2.058993in}}{\pgfqpoint{2.101772in}{2.055721in}}{\pgfqpoint{2.110008in}{2.055721in}}%
\pgfpathclose%
\pgfusepath{stroke,fill}%
\end{pgfscope}%
\begin{pgfscope}%
\pgfpathrectangle{\pgfqpoint{0.100000in}{0.212622in}}{\pgfqpoint{3.696000in}{3.696000in}}%
\pgfusepath{clip}%
\pgfsetbuttcap%
\pgfsetroundjoin%
\definecolor{currentfill}{rgb}{0.121569,0.466667,0.705882}%
\pgfsetfillcolor{currentfill}%
\pgfsetfillopacity{0.612086}%
\pgfsetlinewidth{1.003750pt}%
\definecolor{currentstroke}{rgb}{0.121569,0.466667,0.705882}%
\pgfsetstrokecolor{currentstroke}%
\pgfsetstrokeopacity{0.612086}%
\pgfsetdash{}{0pt}%
\pgfpathmoveto{\pgfqpoint{0.681636in}{1.204953in}}%
\pgfpathcurveto{\pgfqpoint{0.689872in}{1.204953in}}{\pgfqpoint{0.697772in}{1.208225in}}{\pgfqpoint{0.703596in}{1.214049in}}%
\pgfpathcurveto{\pgfqpoint{0.709420in}{1.219873in}}{\pgfqpoint{0.712692in}{1.227773in}}{\pgfqpoint{0.712692in}{1.236009in}}%
\pgfpathcurveto{\pgfqpoint{0.712692in}{1.244246in}}{\pgfqpoint{0.709420in}{1.252146in}}{\pgfqpoint{0.703596in}{1.257970in}}%
\pgfpathcurveto{\pgfqpoint{0.697772in}{1.263793in}}{\pgfqpoint{0.689872in}{1.267066in}}{\pgfqpoint{0.681636in}{1.267066in}}%
\pgfpathcurveto{\pgfqpoint{0.673399in}{1.267066in}}{\pgfqpoint{0.665499in}{1.263793in}}{\pgfqpoint{0.659675in}{1.257970in}}%
\pgfpathcurveto{\pgfqpoint{0.653851in}{1.252146in}}{\pgfqpoint{0.650579in}{1.244246in}}{\pgfqpoint{0.650579in}{1.236009in}}%
\pgfpathcurveto{\pgfqpoint{0.650579in}{1.227773in}}{\pgfqpoint{0.653851in}{1.219873in}}{\pgfqpoint{0.659675in}{1.214049in}}%
\pgfpathcurveto{\pgfqpoint{0.665499in}{1.208225in}}{\pgfqpoint{0.673399in}{1.204953in}}{\pgfqpoint{0.681636in}{1.204953in}}%
\pgfpathclose%
\pgfusepath{stroke,fill}%
\end{pgfscope}%
\begin{pgfscope}%
\pgfpathrectangle{\pgfqpoint{0.100000in}{0.212622in}}{\pgfqpoint{3.696000in}{3.696000in}}%
\pgfusepath{clip}%
\pgfsetbuttcap%
\pgfsetroundjoin%
\definecolor{currentfill}{rgb}{0.121569,0.466667,0.705882}%
\pgfsetfillcolor{currentfill}%
\pgfsetfillopacity{0.612297}%
\pgfsetlinewidth{1.003750pt}%
\definecolor{currentstroke}{rgb}{0.121569,0.466667,0.705882}%
\pgfsetstrokecolor{currentstroke}%
\pgfsetstrokeopacity{0.612297}%
\pgfsetdash{}{0pt}%
\pgfpathmoveto{\pgfqpoint{0.866428in}{1.489010in}}%
\pgfpathcurveto{\pgfqpoint{0.874665in}{1.489010in}}{\pgfqpoint{0.882565in}{1.492283in}}{\pgfqpoint{0.888389in}{1.498107in}}%
\pgfpathcurveto{\pgfqpoint{0.894212in}{1.503931in}}{\pgfqpoint{0.897485in}{1.511831in}}{\pgfqpoint{0.897485in}{1.520067in}}%
\pgfpathcurveto{\pgfqpoint{0.897485in}{1.528303in}}{\pgfqpoint{0.894212in}{1.536203in}}{\pgfqpoint{0.888389in}{1.542027in}}%
\pgfpathcurveto{\pgfqpoint{0.882565in}{1.547851in}}{\pgfqpoint{0.874665in}{1.551123in}}{\pgfqpoint{0.866428in}{1.551123in}}%
\pgfpathcurveto{\pgfqpoint{0.858192in}{1.551123in}}{\pgfqpoint{0.850292in}{1.547851in}}{\pgfqpoint{0.844468in}{1.542027in}}%
\pgfpathcurveto{\pgfqpoint{0.838644in}{1.536203in}}{\pgfqpoint{0.835372in}{1.528303in}}{\pgfqpoint{0.835372in}{1.520067in}}%
\pgfpathcurveto{\pgfqpoint{0.835372in}{1.511831in}}{\pgfqpoint{0.838644in}{1.503931in}}{\pgfqpoint{0.844468in}{1.498107in}}%
\pgfpathcurveto{\pgfqpoint{0.850292in}{1.492283in}}{\pgfqpoint{0.858192in}{1.489010in}}{\pgfqpoint{0.866428in}{1.489010in}}%
\pgfpathclose%
\pgfusepath{stroke,fill}%
\end{pgfscope}%
\begin{pgfscope}%
\pgfpathrectangle{\pgfqpoint{0.100000in}{0.212622in}}{\pgfqpoint{3.696000in}{3.696000in}}%
\pgfusepath{clip}%
\pgfsetbuttcap%
\pgfsetroundjoin%
\definecolor{currentfill}{rgb}{0.121569,0.466667,0.705882}%
\pgfsetfillcolor{currentfill}%
\pgfsetfillopacity{0.612389}%
\pgfsetlinewidth{1.003750pt}%
\definecolor{currentstroke}{rgb}{0.121569,0.466667,0.705882}%
\pgfsetstrokecolor{currentstroke}%
\pgfsetstrokeopacity{0.612389}%
\pgfsetdash{}{0pt}%
\pgfpathmoveto{\pgfqpoint{0.918224in}{1.608576in}}%
\pgfpathcurveto{\pgfqpoint{0.926460in}{1.608576in}}{\pgfqpoint{0.934360in}{1.611848in}}{\pgfqpoint{0.940184in}{1.617672in}}%
\pgfpathcurveto{\pgfqpoint{0.946008in}{1.623496in}}{\pgfqpoint{0.949281in}{1.631396in}}{\pgfqpoint{0.949281in}{1.639632in}}%
\pgfpathcurveto{\pgfqpoint{0.949281in}{1.647869in}}{\pgfqpoint{0.946008in}{1.655769in}}{\pgfqpoint{0.940184in}{1.661593in}}%
\pgfpathcurveto{\pgfqpoint{0.934360in}{1.667416in}}{\pgfqpoint{0.926460in}{1.670689in}}{\pgfqpoint{0.918224in}{1.670689in}}%
\pgfpathcurveto{\pgfqpoint{0.909988in}{1.670689in}}{\pgfqpoint{0.902088in}{1.667416in}}{\pgfqpoint{0.896264in}{1.661593in}}%
\pgfpathcurveto{\pgfqpoint{0.890440in}{1.655769in}}{\pgfqpoint{0.887168in}{1.647869in}}{\pgfqpoint{0.887168in}{1.639632in}}%
\pgfpathcurveto{\pgfqpoint{0.887168in}{1.631396in}}{\pgfqpoint{0.890440in}{1.623496in}}{\pgfqpoint{0.896264in}{1.617672in}}%
\pgfpathcurveto{\pgfqpoint{0.902088in}{1.611848in}}{\pgfqpoint{0.909988in}{1.608576in}}{\pgfqpoint{0.918224in}{1.608576in}}%
\pgfpathclose%
\pgfusepath{stroke,fill}%
\end{pgfscope}%
\begin{pgfscope}%
\pgfpathrectangle{\pgfqpoint{0.100000in}{0.212622in}}{\pgfqpoint{3.696000in}{3.696000in}}%
\pgfusepath{clip}%
\pgfsetbuttcap%
\pgfsetroundjoin%
\definecolor{currentfill}{rgb}{0.121569,0.466667,0.705882}%
\pgfsetfillcolor{currentfill}%
\pgfsetfillopacity{0.613529}%
\pgfsetlinewidth{1.003750pt}%
\definecolor{currentstroke}{rgb}{0.121569,0.466667,0.705882}%
\pgfsetstrokecolor{currentstroke}%
\pgfsetstrokeopacity{0.613529}%
\pgfsetdash{}{0pt}%
\pgfpathmoveto{\pgfqpoint{0.915934in}{1.602945in}}%
\pgfpathcurveto{\pgfqpoint{0.924170in}{1.602945in}}{\pgfqpoint{0.932070in}{1.606217in}}{\pgfqpoint{0.937894in}{1.612041in}}%
\pgfpathcurveto{\pgfqpoint{0.943718in}{1.617865in}}{\pgfqpoint{0.946991in}{1.625765in}}{\pgfqpoint{0.946991in}{1.634001in}}%
\pgfpathcurveto{\pgfqpoint{0.946991in}{1.642237in}}{\pgfqpoint{0.943718in}{1.650138in}}{\pgfqpoint{0.937894in}{1.655961in}}%
\pgfpathcurveto{\pgfqpoint{0.932070in}{1.661785in}}{\pgfqpoint{0.924170in}{1.665058in}}{\pgfqpoint{0.915934in}{1.665058in}}%
\pgfpathcurveto{\pgfqpoint{0.907698in}{1.665058in}}{\pgfqpoint{0.899798in}{1.661785in}}{\pgfqpoint{0.893974in}{1.655961in}}%
\pgfpathcurveto{\pgfqpoint{0.888150in}{1.650138in}}{\pgfqpoint{0.884878in}{1.642237in}}{\pgfqpoint{0.884878in}{1.634001in}}%
\pgfpathcurveto{\pgfqpoint{0.884878in}{1.625765in}}{\pgfqpoint{0.888150in}{1.617865in}}{\pgfqpoint{0.893974in}{1.612041in}}%
\pgfpathcurveto{\pgfqpoint{0.899798in}{1.606217in}}{\pgfqpoint{0.907698in}{1.602945in}}{\pgfqpoint{0.915934in}{1.602945in}}%
\pgfpathclose%
\pgfusepath{stroke,fill}%
\end{pgfscope}%
\begin{pgfscope}%
\pgfpathrectangle{\pgfqpoint{0.100000in}{0.212622in}}{\pgfqpoint{3.696000in}{3.696000in}}%
\pgfusepath{clip}%
\pgfsetbuttcap%
\pgfsetroundjoin%
\definecolor{currentfill}{rgb}{0.121569,0.466667,0.705882}%
\pgfsetfillcolor{currentfill}%
\pgfsetfillopacity{0.614350}%
\pgfsetlinewidth{1.003750pt}%
\definecolor{currentstroke}{rgb}{0.121569,0.466667,0.705882}%
\pgfsetstrokecolor{currentstroke}%
\pgfsetstrokeopacity{0.614350}%
\pgfsetdash{}{0pt}%
\pgfpathmoveto{\pgfqpoint{2.111929in}{2.048638in}}%
\pgfpathcurveto{\pgfqpoint{2.120165in}{2.048638in}}{\pgfqpoint{2.128065in}{2.051910in}}{\pgfqpoint{2.133889in}{2.057734in}}%
\pgfpathcurveto{\pgfqpoint{2.139713in}{2.063558in}}{\pgfqpoint{2.142985in}{2.071458in}}{\pgfqpoint{2.142985in}{2.079694in}}%
\pgfpathcurveto{\pgfqpoint{2.142985in}{2.087931in}}{\pgfqpoint{2.139713in}{2.095831in}}{\pgfqpoint{2.133889in}{2.101655in}}%
\pgfpathcurveto{\pgfqpoint{2.128065in}{2.107479in}}{\pgfqpoint{2.120165in}{2.110751in}}{\pgfqpoint{2.111929in}{2.110751in}}%
\pgfpathcurveto{\pgfqpoint{2.103692in}{2.110751in}}{\pgfqpoint{2.095792in}{2.107479in}}{\pgfqpoint{2.089968in}{2.101655in}}%
\pgfpathcurveto{\pgfqpoint{2.084144in}{2.095831in}}{\pgfqpoint{2.080872in}{2.087931in}}{\pgfqpoint{2.080872in}{2.079694in}}%
\pgfpathcurveto{\pgfqpoint{2.080872in}{2.071458in}}{\pgfqpoint{2.084144in}{2.063558in}}{\pgfqpoint{2.089968in}{2.057734in}}%
\pgfpathcurveto{\pgfqpoint{2.095792in}{2.051910in}}{\pgfqpoint{2.103692in}{2.048638in}}{\pgfqpoint{2.111929in}{2.048638in}}%
\pgfpathclose%
\pgfusepath{stroke,fill}%
\end{pgfscope}%
\begin{pgfscope}%
\pgfpathrectangle{\pgfqpoint{0.100000in}{0.212622in}}{\pgfqpoint{3.696000in}{3.696000in}}%
\pgfusepath{clip}%
\pgfsetbuttcap%
\pgfsetroundjoin%
\definecolor{currentfill}{rgb}{0.121569,0.466667,0.705882}%
\pgfsetfillcolor{currentfill}%
\pgfsetfillopacity{0.614709}%
\pgfsetlinewidth{1.003750pt}%
\definecolor{currentstroke}{rgb}{0.121569,0.466667,0.705882}%
\pgfsetstrokecolor{currentstroke}%
\pgfsetstrokeopacity{0.614709}%
\pgfsetdash{}{0pt}%
\pgfpathmoveto{\pgfqpoint{0.913621in}{1.599486in}}%
\pgfpathcurveto{\pgfqpoint{0.921857in}{1.599486in}}{\pgfqpoint{0.929757in}{1.602758in}}{\pgfqpoint{0.935581in}{1.608582in}}%
\pgfpathcurveto{\pgfqpoint{0.941405in}{1.614406in}}{\pgfqpoint{0.944677in}{1.622306in}}{\pgfqpoint{0.944677in}{1.630542in}}%
\pgfpathcurveto{\pgfqpoint{0.944677in}{1.638778in}}{\pgfqpoint{0.941405in}{1.646678in}}{\pgfqpoint{0.935581in}{1.652502in}}%
\pgfpathcurveto{\pgfqpoint{0.929757in}{1.658326in}}{\pgfqpoint{0.921857in}{1.661599in}}{\pgfqpoint{0.913621in}{1.661599in}}%
\pgfpathcurveto{\pgfqpoint{0.905385in}{1.661599in}}{\pgfqpoint{0.897485in}{1.658326in}}{\pgfqpoint{0.891661in}{1.652502in}}%
\pgfpathcurveto{\pgfqpoint{0.885837in}{1.646678in}}{\pgfqpoint{0.882564in}{1.638778in}}{\pgfqpoint{0.882564in}{1.630542in}}%
\pgfpathcurveto{\pgfqpoint{0.882564in}{1.622306in}}{\pgfqpoint{0.885837in}{1.614406in}}{\pgfqpoint{0.891661in}{1.608582in}}%
\pgfpathcurveto{\pgfqpoint{0.897485in}{1.602758in}}{\pgfqpoint{0.905385in}{1.599486in}}{\pgfqpoint{0.913621in}{1.599486in}}%
\pgfpathclose%
\pgfusepath{stroke,fill}%
\end{pgfscope}%
\begin{pgfscope}%
\pgfpathrectangle{\pgfqpoint{0.100000in}{0.212622in}}{\pgfqpoint{3.696000in}{3.696000in}}%
\pgfusepath{clip}%
\pgfsetbuttcap%
\pgfsetroundjoin%
\definecolor{currentfill}{rgb}{0.121569,0.466667,0.705882}%
\pgfsetfillcolor{currentfill}%
\pgfsetfillopacity{0.615615}%
\pgfsetlinewidth{1.003750pt}%
\definecolor{currentstroke}{rgb}{0.121569,0.466667,0.705882}%
\pgfsetstrokecolor{currentstroke}%
\pgfsetstrokeopacity{0.615615}%
\pgfsetdash{}{0pt}%
\pgfpathmoveto{\pgfqpoint{2.113107in}{2.044557in}}%
\pgfpathcurveto{\pgfqpoint{2.121343in}{2.044557in}}{\pgfqpoint{2.129243in}{2.047830in}}{\pgfqpoint{2.135067in}{2.053654in}}%
\pgfpathcurveto{\pgfqpoint{2.140891in}{2.059478in}}{\pgfqpoint{2.144164in}{2.067378in}}{\pgfqpoint{2.144164in}{2.075614in}}%
\pgfpathcurveto{\pgfqpoint{2.144164in}{2.083850in}}{\pgfqpoint{2.140891in}{2.091750in}}{\pgfqpoint{2.135067in}{2.097574in}}%
\pgfpathcurveto{\pgfqpoint{2.129243in}{2.103398in}}{\pgfqpoint{2.121343in}{2.106670in}}{\pgfqpoint{2.113107in}{2.106670in}}%
\pgfpathcurveto{\pgfqpoint{2.104871in}{2.106670in}}{\pgfqpoint{2.096971in}{2.103398in}}{\pgfqpoint{2.091147in}{2.097574in}}%
\pgfpathcurveto{\pgfqpoint{2.085323in}{2.091750in}}{\pgfqpoint{2.082051in}{2.083850in}}{\pgfqpoint{2.082051in}{2.075614in}}%
\pgfpathcurveto{\pgfqpoint{2.082051in}{2.067378in}}{\pgfqpoint{2.085323in}{2.059478in}}{\pgfqpoint{2.091147in}{2.053654in}}%
\pgfpathcurveto{\pgfqpoint{2.096971in}{2.047830in}}{\pgfqpoint{2.104871in}{2.044557in}}{\pgfqpoint{2.113107in}{2.044557in}}%
\pgfpathclose%
\pgfusepath{stroke,fill}%
\end{pgfscope}%
\begin{pgfscope}%
\pgfpathrectangle{\pgfqpoint{0.100000in}{0.212622in}}{\pgfqpoint{3.696000in}{3.696000in}}%
\pgfusepath{clip}%
\pgfsetbuttcap%
\pgfsetroundjoin%
\definecolor{currentfill}{rgb}{0.121569,0.466667,0.705882}%
\pgfsetfillcolor{currentfill}%
\pgfsetfillopacity{0.615629}%
\pgfsetlinewidth{1.003750pt}%
\definecolor{currentstroke}{rgb}{0.121569,0.466667,0.705882}%
\pgfsetstrokecolor{currentstroke}%
\pgfsetstrokeopacity{0.615629}%
\pgfsetdash{}{0pt}%
\pgfpathmoveto{\pgfqpoint{0.911369in}{1.596392in}}%
\pgfpathcurveto{\pgfqpoint{0.919605in}{1.596392in}}{\pgfqpoint{0.927505in}{1.599664in}}{\pgfqpoint{0.933329in}{1.605488in}}%
\pgfpathcurveto{\pgfqpoint{0.939153in}{1.611312in}}{\pgfqpoint{0.942426in}{1.619212in}}{\pgfqpoint{0.942426in}{1.627449in}}%
\pgfpathcurveto{\pgfqpoint{0.942426in}{1.635685in}}{\pgfqpoint{0.939153in}{1.643585in}}{\pgfqpoint{0.933329in}{1.649409in}}%
\pgfpathcurveto{\pgfqpoint{0.927505in}{1.655233in}}{\pgfqpoint{0.919605in}{1.658505in}}{\pgfqpoint{0.911369in}{1.658505in}}%
\pgfpathcurveto{\pgfqpoint{0.903133in}{1.658505in}}{\pgfqpoint{0.895233in}{1.655233in}}{\pgfqpoint{0.889409in}{1.649409in}}%
\pgfpathcurveto{\pgfqpoint{0.883585in}{1.643585in}}{\pgfqpoint{0.880313in}{1.635685in}}{\pgfqpoint{0.880313in}{1.627449in}}%
\pgfpathcurveto{\pgfqpoint{0.880313in}{1.619212in}}{\pgfqpoint{0.883585in}{1.611312in}}{\pgfqpoint{0.889409in}{1.605488in}}%
\pgfpathcurveto{\pgfqpoint{0.895233in}{1.599664in}}{\pgfqpoint{0.903133in}{1.596392in}}{\pgfqpoint{0.911369in}{1.596392in}}%
\pgfpathclose%
\pgfusepath{stroke,fill}%
\end{pgfscope}%
\begin{pgfscope}%
\pgfpathrectangle{\pgfqpoint{0.100000in}{0.212622in}}{\pgfqpoint{3.696000in}{3.696000in}}%
\pgfusepath{clip}%
\pgfsetbuttcap%
\pgfsetroundjoin%
\definecolor{currentfill}{rgb}{0.121569,0.466667,0.705882}%
\pgfsetfillcolor{currentfill}%
\pgfsetfillopacity{0.615875}%
\pgfsetlinewidth{1.003750pt}%
\definecolor{currentstroke}{rgb}{0.121569,0.466667,0.705882}%
\pgfsetstrokecolor{currentstroke}%
\pgfsetstrokeopacity{0.615875}%
\pgfsetdash{}{0pt}%
\pgfpathmoveto{\pgfqpoint{0.693410in}{1.207581in}}%
\pgfpathcurveto{\pgfqpoint{0.701646in}{1.207581in}}{\pgfqpoint{0.709547in}{1.210853in}}{\pgfqpoint{0.715370in}{1.216677in}}%
\pgfpathcurveto{\pgfqpoint{0.721194in}{1.222501in}}{\pgfqpoint{0.724467in}{1.230401in}}{\pgfqpoint{0.724467in}{1.238638in}}%
\pgfpathcurveto{\pgfqpoint{0.724467in}{1.246874in}}{\pgfqpoint{0.721194in}{1.254774in}}{\pgfqpoint{0.715370in}{1.260598in}}%
\pgfpathcurveto{\pgfqpoint{0.709547in}{1.266422in}}{\pgfqpoint{0.701646in}{1.269694in}}{\pgfqpoint{0.693410in}{1.269694in}}%
\pgfpathcurveto{\pgfqpoint{0.685174in}{1.269694in}}{\pgfqpoint{0.677274in}{1.266422in}}{\pgfqpoint{0.671450in}{1.260598in}}%
\pgfpathcurveto{\pgfqpoint{0.665626in}{1.254774in}}{\pgfqpoint{0.662354in}{1.246874in}}{\pgfqpoint{0.662354in}{1.238638in}}%
\pgfpathcurveto{\pgfqpoint{0.662354in}{1.230401in}}{\pgfqpoint{0.665626in}{1.222501in}}{\pgfqpoint{0.671450in}{1.216677in}}%
\pgfpathcurveto{\pgfqpoint{0.677274in}{1.210853in}}{\pgfqpoint{0.685174in}{1.207581in}}{\pgfqpoint{0.693410in}{1.207581in}}%
\pgfpathclose%
\pgfusepath{stroke,fill}%
\end{pgfscope}%
\begin{pgfscope}%
\pgfpathrectangle{\pgfqpoint{0.100000in}{0.212622in}}{\pgfqpoint{3.696000in}{3.696000in}}%
\pgfusepath{clip}%
\pgfsetbuttcap%
\pgfsetroundjoin%
\definecolor{currentfill}{rgb}{0.121569,0.466667,0.705882}%
\pgfsetfillcolor{currentfill}%
\pgfsetfillopacity{0.616363}%
\pgfsetlinewidth{1.003750pt}%
\definecolor{currentstroke}{rgb}{0.121569,0.466667,0.705882}%
\pgfsetstrokecolor{currentstroke}%
\pgfsetstrokeopacity{0.616363}%
\pgfsetdash{}{0pt}%
\pgfpathmoveto{\pgfqpoint{0.909941in}{1.593836in}}%
\pgfpathcurveto{\pgfqpoint{0.918177in}{1.593836in}}{\pgfqpoint{0.926077in}{1.597109in}}{\pgfqpoint{0.931901in}{1.602933in}}%
\pgfpathcurveto{\pgfqpoint{0.937725in}{1.608756in}}{\pgfqpoint{0.940997in}{1.616657in}}{\pgfqpoint{0.940997in}{1.624893in}}%
\pgfpathcurveto{\pgfqpoint{0.940997in}{1.633129in}}{\pgfqpoint{0.937725in}{1.641029in}}{\pgfqpoint{0.931901in}{1.646853in}}%
\pgfpathcurveto{\pgfqpoint{0.926077in}{1.652677in}}{\pgfqpoint{0.918177in}{1.655949in}}{\pgfqpoint{0.909941in}{1.655949in}}%
\pgfpathcurveto{\pgfqpoint{0.901704in}{1.655949in}}{\pgfqpoint{0.893804in}{1.652677in}}{\pgfqpoint{0.887980in}{1.646853in}}%
\pgfpathcurveto{\pgfqpoint{0.882156in}{1.641029in}}{\pgfqpoint{0.878884in}{1.633129in}}{\pgfqpoint{0.878884in}{1.624893in}}%
\pgfpathcurveto{\pgfqpoint{0.878884in}{1.616657in}}{\pgfqpoint{0.882156in}{1.608756in}}{\pgfqpoint{0.887980in}{1.602933in}}%
\pgfpathcurveto{\pgfqpoint{0.893804in}{1.597109in}}{\pgfqpoint{0.901704in}{1.593836in}}{\pgfqpoint{0.909941in}{1.593836in}}%
\pgfpathclose%
\pgfusepath{stroke,fill}%
\end{pgfscope}%
\begin{pgfscope}%
\pgfpathrectangle{\pgfqpoint{0.100000in}{0.212622in}}{\pgfqpoint{3.696000in}{3.696000in}}%
\pgfusepath{clip}%
\pgfsetbuttcap%
\pgfsetroundjoin%
\definecolor{currentfill}{rgb}{0.121569,0.466667,0.705882}%
\pgfsetfillcolor{currentfill}%
\pgfsetfillopacity{0.616725}%
\pgfsetlinewidth{1.003750pt}%
\definecolor{currentstroke}{rgb}{0.121569,0.466667,0.705882}%
\pgfsetstrokecolor{currentstroke}%
\pgfsetstrokeopacity{0.616725}%
\pgfsetdash{}{0pt}%
\pgfpathmoveto{\pgfqpoint{0.908989in}{1.592460in}}%
\pgfpathcurveto{\pgfqpoint{0.917225in}{1.592460in}}{\pgfqpoint{0.925125in}{1.595733in}}{\pgfqpoint{0.930949in}{1.601557in}}%
\pgfpathcurveto{\pgfqpoint{0.936773in}{1.607381in}}{\pgfqpoint{0.940045in}{1.615281in}}{\pgfqpoint{0.940045in}{1.623517in}}%
\pgfpathcurveto{\pgfqpoint{0.940045in}{1.631753in}}{\pgfqpoint{0.936773in}{1.639653in}}{\pgfqpoint{0.930949in}{1.645477in}}%
\pgfpathcurveto{\pgfqpoint{0.925125in}{1.651301in}}{\pgfqpoint{0.917225in}{1.654573in}}{\pgfqpoint{0.908989in}{1.654573in}}%
\pgfpathcurveto{\pgfqpoint{0.900752in}{1.654573in}}{\pgfqpoint{0.892852in}{1.651301in}}{\pgfqpoint{0.887028in}{1.645477in}}%
\pgfpathcurveto{\pgfqpoint{0.881204in}{1.639653in}}{\pgfqpoint{0.877932in}{1.631753in}}{\pgfqpoint{0.877932in}{1.623517in}}%
\pgfpathcurveto{\pgfqpoint{0.877932in}{1.615281in}}{\pgfqpoint{0.881204in}{1.607381in}}{\pgfqpoint{0.887028in}{1.601557in}}%
\pgfpathcurveto{\pgfqpoint{0.892852in}{1.595733in}}{\pgfqpoint{0.900752in}{1.592460in}}{\pgfqpoint{0.908989in}{1.592460in}}%
\pgfpathclose%
\pgfusepath{stroke,fill}%
\end{pgfscope}%
\begin{pgfscope}%
\pgfpathrectangle{\pgfqpoint{0.100000in}{0.212622in}}{\pgfqpoint{3.696000in}{3.696000in}}%
\pgfusepath{clip}%
\pgfsetbuttcap%
\pgfsetroundjoin%
\definecolor{currentfill}{rgb}{0.121569,0.466667,0.705882}%
\pgfsetfillcolor{currentfill}%
\pgfsetfillopacity{0.616853}%
\pgfsetlinewidth{1.003750pt}%
\definecolor{currentstroke}{rgb}{0.121569,0.466667,0.705882}%
\pgfsetstrokecolor{currentstroke}%
\pgfsetstrokeopacity{0.616853}%
\pgfsetdash{}{0pt}%
\pgfpathmoveto{\pgfqpoint{0.908738in}{1.591994in}}%
\pgfpathcurveto{\pgfqpoint{0.916975in}{1.591994in}}{\pgfqpoint{0.924875in}{1.595266in}}{\pgfqpoint{0.930699in}{1.601090in}}%
\pgfpathcurveto{\pgfqpoint{0.936523in}{1.606914in}}{\pgfqpoint{0.939795in}{1.614814in}}{\pgfqpoint{0.939795in}{1.623051in}}%
\pgfpathcurveto{\pgfqpoint{0.939795in}{1.631287in}}{\pgfqpoint{0.936523in}{1.639187in}}{\pgfqpoint{0.930699in}{1.645011in}}%
\pgfpathcurveto{\pgfqpoint{0.924875in}{1.650835in}}{\pgfqpoint{0.916975in}{1.654107in}}{\pgfqpoint{0.908738in}{1.654107in}}%
\pgfpathcurveto{\pgfqpoint{0.900502in}{1.654107in}}{\pgfqpoint{0.892602in}{1.650835in}}{\pgfqpoint{0.886778in}{1.645011in}}%
\pgfpathcurveto{\pgfqpoint{0.880954in}{1.639187in}}{\pgfqpoint{0.877682in}{1.631287in}}{\pgfqpoint{0.877682in}{1.623051in}}%
\pgfpathcurveto{\pgfqpoint{0.877682in}{1.614814in}}{\pgfqpoint{0.880954in}{1.606914in}}{\pgfqpoint{0.886778in}{1.601090in}}%
\pgfpathcurveto{\pgfqpoint{0.892602in}{1.595266in}}{\pgfqpoint{0.900502in}{1.591994in}}{\pgfqpoint{0.908738in}{1.591994in}}%
\pgfpathclose%
\pgfusepath{stroke,fill}%
\end{pgfscope}%
\begin{pgfscope}%
\pgfpathrectangle{\pgfqpoint{0.100000in}{0.212622in}}{\pgfqpoint{3.696000in}{3.696000in}}%
\pgfusepath{clip}%
\pgfsetbuttcap%
\pgfsetroundjoin%
\definecolor{currentfill}{rgb}{0.121569,0.466667,0.705882}%
\pgfsetfillcolor{currentfill}%
\pgfsetfillopacity{0.617045}%
\pgfsetlinewidth{1.003750pt}%
\definecolor{currentstroke}{rgb}{0.121569,0.466667,0.705882}%
\pgfsetstrokecolor{currentstroke}%
\pgfsetstrokeopacity{0.617045}%
\pgfsetdash{}{0pt}%
\pgfpathmoveto{\pgfqpoint{0.908196in}{1.591103in}}%
\pgfpathcurveto{\pgfqpoint{0.916432in}{1.591103in}}{\pgfqpoint{0.924332in}{1.594376in}}{\pgfqpoint{0.930156in}{1.600200in}}%
\pgfpathcurveto{\pgfqpoint{0.935980in}{1.606023in}}{\pgfqpoint{0.939252in}{1.613924in}}{\pgfqpoint{0.939252in}{1.622160in}}%
\pgfpathcurveto{\pgfqpoint{0.939252in}{1.630396in}}{\pgfqpoint{0.935980in}{1.638296in}}{\pgfqpoint{0.930156in}{1.644120in}}%
\pgfpathcurveto{\pgfqpoint{0.924332in}{1.649944in}}{\pgfqpoint{0.916432in}{1.653216in}}{\pgfqpoint{0.908196in}{1.653216in}}%
\pgfpathcurveto{\pgfqpoint{0.899960in}{1.653216in}}{\pgfqpoint{0.892059in}{1.649944in}}{\pgfqpoint{0.886236in}{1.644120in}}%
\pgfpathcurveto{\pgfqpoint{0.880412in}{1.638296in}}{\pgfqpoint{0.877139in}{1.630396in}}{\pgfqpoint{0.877139in}{1.622160in}}%
\pgfpathcurveto{\pgfqpoint{0.877139in}{1.613924in}}{\pgfqpoint{0.880412in}{1.606023in}}{\pgfqpoint{0.886236in}{1.600200in}}%
\pgfpathcurveto{\pgfqpoint{0.892059in}{1.594376in}}{\pgfqpoint{0.899960in}{1.591103in}}{\pgfqpoint{0.908196in}{1.591103in}}%
\pgfpathclose%
\pgfusepath{stroke,fill}%
\end{pgfscope}%
\begin{pgfscope}%
\pgfpathrectangle{\pgfqpoint{0.100000in}{0.212622in}}{\pgfqpoint{3.696000in}{3.696000in}}%
\pgfusepath{clip}%
\pgfsetbuttcap%
\pgfsetroundjoin%
\definecolor{currentfill}{rgb}{0.121569,0.466667,0.705882}%
\pgfsetfillcolor{currentfill}%
\pgfsetfillopacity{0.617275}%
\pgfsetlinewidth{1.003750pt}%
\definecolor{currentstroke}{rgb}{0.121569,0.466667,0.705882}%
\pgfsetstrokecolor{currentstroke}%
\pgfsetstrokeopacity{0.617275}%
\pgfsetdash{}{0pt}%
\pgfpathmoveto{\pgfqpoint{2.113950in}{2.040129in}}%
\pgfpathcurveto{\pgfqpoint{2.122186in}{2.040129in}}{\pgfqpoint{2.130087in}{2.043401in}}{\pgfqpoint{2.135910in}{2.049225in}}%
\pgfpathcurveto{\pgfqpoint{2.141734in}{2.055049in}}{\pgfqpoint{2.145007in}{2.062949in}}{\pgfqpoint{2.145007in}{2.071185in}}%
\pgfpathcurveto{\pgfqpoint{2.145007in}{2.079422in}}{\pgfqpoint{2.141734in}{2.087322in}}{\pgfqpoint{2.135910in}{2.093146in}}%
\pgfpathcurveto{\pgfqpoint{2.130087in}{2.098970in}}{\pgfqpoint{2.122186in}{2.102242in}}{\pgfqpoint{2.113950in}{2.102242in}}%
\pgfpathcurveto{\pgfqpoint{2.105714in}{2.102242in}}{\pgfqpoint{2.097814in}{2.098970in}}{\pgfqpoint{2.091990in}{2.093146in}}%
\pgfpathcurveto{\pgfqpoint{2.086166in}{2.087322in}}{\pgfqpoint{2.082894in}{2.079422in}}{\pgfqpoint{2.082894in}{2.071185in}}%
\pgfpathcurveto{\pgfqpoint{2.082894in}{2.062949in}}{\pgfqpoint{2.086166in}{2.055049in}}{\pgfqpoint{2.091990in}{2.049225in}}%
\pgfpathcurveto{\pgfqpoint{2.097814in}{2.043401in}}{\pgfqpoint{2.105714in}{2.040129in}}{\pgfqpoint{2.113950in}{2.040129in}}%
\pgfpathclose%
\pgfusepath{stroke,fill}%
\end{pgfscope}%
\begin{pgfscope}%
\pgfpathrectangle{\pgfqpoint{0.100000in}{0.212622in}}{\pgfqpoint{3.696000in}{3.696000in}}%
\pgfusepath{clip}%
\pgfsetbuttcap%
\pgfsetroundjoin%
\definecolor{currentfill}{rgb}{0.121569,0.466667,0.705882}%
\pgfsetfillcolor{currentfill}%
\pgfsetfillopacity{0.617389}%
\pgfsetlinewidth{1.003750pt}%
\definecolor{currentstroke}{rgb}{0.121569,0.466667,0.705882}%
\pgfsetstrokecolor{currentstroke}%
\pgfsetstrokeopacity{0.617389}%
\pgfsetdash{}{0pt}%
\pgfpathmoveto{\pgfqpoint{0.907250in}{1.589408in}}%
\pgfpathcurveto{\pgfqpoint{0.915486in}{1.589408in}}{\pgfqpoint{0.923386in}{1.592680in}}{\pgfqpoint{0.929210in}{1.598504in}}%
\pgfpathcurveto{\pgfqpoint{0.935034in}{1.604328in}}{\pgfqpoint{0.938306in}{1.612228in}}{\pgfqpoint{0.938306in}{1.620464in}}%
\pgfpathcurveto{\pgfqpoint{0.938306in}{1.628700in}}{\pgfqpoint{0.935034in}{1.636600in}}{\pgfqpoint{0.929210in}{1.642424in}}%
\pgfpathcurveto{\pgfqpoint{0.923386in}{1.648248in}}{\pgfqpoint{0.915486in}{1.651521in}}{\pgfqpoint{0.907250in}{1.651521in}}%
\pgfpathcurveto{\pgfqpoint{0.899014in}{1.651521in}}{\pgfqpoint{0.891114in}{1.648248in}}{\pgfqpoint{0.885290in}{1.642424in}}%
\pgfpathcurveto{\pgfqpoint{0.879466in}{1.636600in}}{\pgfqpoint{0.876193in}{1.628700in}}{\pgfqpoint{0.876193in}{1.620464in}}%
\pgfpathcurveto{\pgfqpoint{0.876193in}{1.612228in}}{\pgfqpoint{0.879466in}{1.604328in}}{\pgfqpoint{0.885290in}{1.598504in}}%
\pgfpathcurveto{\pgfqpoint{0.891114in}{1.592680in}}{\pgfqpoint{0.899014in}{1.589408in}}{\pgfqpoint{0.907250in}{1.589408in}}%
\pgfpathclose%
\pgfusepath{stroke,fill}%
\end{pgfscope}%
\begin{pgfscope}%
\pgfpathrectangle{\pgfqpoint{0.100000in}{0.212622in}}{\pgfqpoint{3.696000in}{3.696000in}}%
\pgfusepath{clip}%
\pgfsetbuttcap%
\pgfsetroundjoin%
\definecolor{currentfill}{rgb}{0.121569,0.466667,0.705882}%
\pgfsetfillcolor{currentfill}%
\pgfsetfillopacity{0.617933}%
\pgfsetlinewidth{1.003750pt}%
\definecolor{currentstroke}{rgb}{0.121569,0.466667,0.705882}%
\pgfsetstrokecolor{currentstroke}%
\pgfsetstrokeopacity{0.617933}%
\pgfsetdash{}{0pt}%
\pgfpathmoveto{\pgfqpoint{0.905543in}{1.586014in}}%
\pgfpathcurveto{\pgfqpoint{0.913780in}{1.586014in}}{\pgfqpoint{0.921680in}{1.589287in}}{\pgfqpoint{0.927504in}{1.595111in}}%
\pgfpathcurveto{\pgfqpoint{0.933328in}{1.600935in}}{\pgfqpoint{0.936600in}{1.608835in}}{\pgfqpoint{0.936600in}{1.617071in}}%
\pgfpathcurveto{\pgfqpoint{0.936600in}{1.625307in}}{\pgfqpoint{0.933328in}{1.633207in}}{\pgfqpoint{0.927504in}{1.639031in}}%
\pgfpathcurveto{\pgfqpoint{0.921680in}{1.644855in}}{\pgfqpoint{0.913780in}{1.648127in}}{\pgfqpoint{0.905543in}{1.648127in}}%
\pgfpathcurveto{\pgfqpoint{0.897307in}{1.648127in}}{\pgfqpoint{0.889407in}{1.644855in}}{\pgfqpoint{0.883583in}{1.639031in}}%
\pgfpathcurveto{\pgfqpoint{0.877759in}{1.633207in}}{\pgfqpoint{0.874487in}{1.625307in}}{\pgfqpoint{0.874487in}{1.617071in}}%
\pgfpathcurveto{\pgfqpoint{0.874487in}{1.608835in}}{\pgfqpoint{0.877759in}{1.600935in}}{\pgfqpoint{0.883583in}{1.595111in}}%
\pgfpathcurveto{\pgfqpoint{0.889407in}{1.589287in}}{\pgfqpoint{0.897307in}{1.586014in}}{\pgfqpoint{0.905543in}{1.586014in}}%
\pgfpathclose%
\pgfusepath{stroke,fill}%
\end{pgfscope}%
\begin{pgfscope}%
\pgfpathrectangle{\pgfqpoint{0.100000in}{0.212622in}}{\pgfqpoint{3.696000in}{3.696000in}}%
\pgfusepath{clip}%
\pgfsetbuttcap%
\pgfsetroundjoin%
\definecolor{currentfill}{rgb}{0.121569,0.466667,0.705882}%
\pgfsetfillcolor{currentfill}%
\pgfsetfillopacity{0.617944}%
\pgfsetlinewidth{1.003750pt}%
\definecolor{currentstroke}{rgb}{0.121569,0.466667,0.705882}%
\pgfsetstrokecolor{currentstroke}%
\pgfsetstrokeopacity{0.617944}%
\pgfsetdash{}{0pt}%
\pgfpathmoveto{\pgfqpoint{0.862968in}{1.490380in}}%
\pgfpathcurveto{\pgfqpoint{0.871204in}{1.490380in}}{\pgfqpoint{0.879105in}{1.493652in}}{\pgfqpoint{0.884928in}{1.499476in}}%
\pgfpathcurveto{\pgfqpoint{0.890752in}{1.505300in}}{\pgfqpoint{0.894025in}{1.513200in}}{\pgfqpoint{0.894025in}{1.521436in}}%
\pgfpathcurveto{\pgfqpoint{0.894025in}{1.529672in}}{\pgfqpoint{0.890752in}{1.537573in}}{\pgfqpoint{0.884928in}{1.543396in}}%
\pgfpathcurveto{\pgfqpoint{0.879105in}{1.549220in}}{\pgfqpoint{0.871204in}{1.552493in}}{\pgfqpoint{0.862968in}{1.552493in}}%
\pgfpathcurveto{\pgfqpoint{0.854732in}{1.552493in}}{\pgfqpoint{0.846832in}{1.549220in}}{\pgfqpoint{0.841008in}{1.543396in}}%
\pgfpathcurveto{\pgfqpoint{0.835184in}{1.537573in}}{\pgfqpoint{0.831912in}{1.529672in}}{\pgfqpoint{0.831912in}{1.521436in}}%
\pgfpathcurveto{\pgfqpoint{0.831912in}{1.513200in}}{\pgfqpoint{0.835184in}{1.505300in}}{\pgfqpoint{0.841008in}{1.499476in}}%
\pgfpathcurveto{\pgfqpoint{0.846832in}{1.493652in}}{\pgfqpoint{0.854732in}{1.490380in}}{\pgfqpoint{0.862968in}{1.490380in}}%
\pgfpathclose%
\pgfusepath{stroke,fill}%
\end{pgfscope}%
\begin{pgfscope}%
\pgfpathrectangle{\pgfqpoint{0.100000in}{0.212622in}}{\pgfqpoint{3.696000in}{3.696000in}}%
\pgfusepath{clip}%
\pgfsetbuttcap%
\pgfsetroundjoin%
\definecolor{currentfill}{rgb}{0.121569,0.466667,0.705882}%
\pgfsetfillcolor{currentfill}%
\pgfsetfillopacity{0.618203}%
\pgfsetlinewidth{1.003750pt}%
\definecolor{currentstroke}{rgb}{0.121569,0.466667,0.705882}%
\pgfsetstrokecolor{currentstroke}%
\pgfsetstrokeopacity{0.618203}%
\pgfsetdash{}{0pt}%
\pgfpathmoveto{\pgfqpoint{0.904094in}{1.583298in}}%
\pgfpathcurveto{\pgfqpoint{0.912330in}{1.583298in}}{\pgfqpoint{0.920230in}{1.586570in}}{\pgfqpoint{0.926054in}{1.592394in}}%
\pgfpathcurveto{\pgfqpoint{0.931878in}{1.598218in}}{\pgfqpoint{0.935151in}{1.606118in}}{\pgfqpoint{0.935151in}{1.614355in}}%
\pgfpathcurveto{\pgfqpoint{0.935151in}{1.622591in}}{\pgfqpoint{0.931878in}{1.630491in}}{\pgfqpoint{0.926054in}{1.636315in}}%
\pgfpathcurveto{\pgfqpoint{0.920230in}{1.642139in}}{\pgfqpoint{0.912330in}{1.645411in}}{\pgfqpoint{0.904094in}{1.645411in}}%
\pgfpathcurveto{\pgfqpoint{0.895858in}{1.645411in}}{\pgfqpoint{0.887958in}{1.642139in}}{\pgfqpoint{0.882134in}{1.636315in}}%
\pgfpathcurveto{\pgfqpoint{0.876310in}{1.630491in}}{\pgfqpoint{0.873038in}{1.622591in}}{\pgfqpoint{0.873038in}{1.614355in}}%
\pgfpathcurveto{\pgfqpoint{0.873038in}{1.606118in}}{\pgfqpoint{0.876310in}{1.598218in}}{\pgfqpoint{0.882134in}{1.592394in}}%
\pgfpathcurveto{\pgfqpoint{0.887958in}{1.586570in}}{\pgfqpoint{0.895858in}{1.583298in}}{\pgfqpoint{0.904094in}{1.583298in}}%
\pgfpathclose%
\pgfusepath{stroke,fill}%
\end{pgfscope}%
\begin{pgfscope}%
\pgfpathrectangle{\pgfqpoint{0.100000in}{0.212622in}}{\pgfqpoint{3.696000in}{3.696000in}}%
\pgfusepath{clip}%
\pgfsetbuttcap%
\pgfsetroundjoin%
\definecolor{currentfill}{rgb}{0.121569,0.466667,0.705882}%
\pgfsetfillcolor{currentfill}%
\pgfsetfillopacity{0.618439}%
\pgfsetlinewidth{1.003750pt}%
\definecolor{currentstroke}{rgb}{0.121569,0.466667,0.705882}%
\pgfsetstrokecolor{currentstroke}%
\pgfsetstrokeopacity{0.618439}%
\pgfsetdash{}{0pt}%
\pgfpathmoveto{\pgfqpoint{0.903244in}{1.581149in}}%
\pgfpathcurveto{\pgfqpoint{0.911480in}{1.581149in}}{\pgfqpoint{0.919380in}{1.584422in}}{\pgfqpoint{0.925204in}{1.590246in}}%
\pgfpathcurveto{\pgfqpoint{0.931028in}{1.596069in}}{\pgfqpoint{0.934300in}{1.603970in}}{\pgfqpoint{0.934300in}{1.612206in}}%
\pgfpathcurveto{\pgfqpoint{0.934300in}{1.620442in}}{\pgfqpoint{0.931028in}{1.628342in}}{\pgfqpoint{0.925204in}{1.634166in}}%
\pgfpathcurveto{\pgfqpoint{0.919380in}{1.639990in}}{\pgfqpoint{0.911480in}{1.643262in}}{\pgfqpoint{0.903244in}{1.643262in}}%
\pgfpathcurveto{\pgfqpoint{0.895007in}{1.643262in}}{\pgfqpoint{0.887107in}{1.639990in}}{\pgfqpoint{0.881284in}{1.634166in}}%
\pgfpathcurveto{\pgfqpoint{0.875460in}{1.628342in}}{\pgfqpoint{0.872187in}{1.620442in}}{\pgfqpoint{0.872187in}{1.612206in}}%
\pgfpathcurveto{\pgfqpoint{0.872187in}{1.603970in}}{\pgfqpoint{0.875460in}{1.596069in}}{\pgfqpoint{0.881284in}{1.590246in}}%
\pgfpathcurveto{\pgfqpoint{0.887107in}{1.584422in}}{\pgfqpoint{0.895007in}{1.581149in}}{\pgfqpoint{0.903244in}{1.581149in}}%
\pgfpathclose%
\pgfusepath{stroke,fill}%
\end{pgfscope}%
\begin{pgfscope}%
\pgfpathrectangle{\pgfqpoint{0.100000in}{0.212622in}}{\pgfqpoint{3.696000in}{3.696000in}}%
\pgfusepath{clip}%
\pgfsetbuttcap%
\pgfsetroundjoin%
\definecolor{currentfill}{rgb}{0.121569,0.466667,0.705882}%
\pgfsetfillcolor{currentfill}%
\pgfsetfillopacity{0.618519}%
\pgfsetlinewidth{1.003750pt}%
\definecolor{currentstroke}{rgb}{0.121569,0.466667,0.705882}%
\pgfsetstrokecolor{currentstroke}%
\pgfsetstrokeopacity{0.618519}%
\pgfsetdash{}{0pt}%
\pgfpathmoveto{\pgfqpoint{0.902816in}{1.580289in}}%
\pgfpathcurveto{\pgfqpoint{0.911052in}{1.580289in}}{\pgfqpoint{0.918952in}{1.583561in}}{\pgfqpoint{0.924776in}{1.589385in}}%
\pgfpathcurveto{\pgfqpoint{0.930600in}{1.595209in}}{\pgfqpoint{0.933873in}{1.603109in}}{\pgfqpoint{0.933873in}{1.611346in}}%
\pgfpathcurveto{\pgfqpoint{0.933873in}{1.619582in}}{\pgfqpoint{0.930600in}{1.627482in}}{\pgfqpoint{0.924776in}{1.633306in}}%
\pgfpathcurveto{\pgfqpoint{0.918952in}{1.639130in}}{\pgfqpoint{0.911052in}{1.642402in}}{\pgfqpoint{0.902816in}{1.642402in}}%
\pgfpathcurveto{\pgfqpoint{0.894580in}{1.642402in}}{\pgfqpoint{0.886680in}{1.639130in}}{\pgfqpoint{0.880856in}{1.633306in}}%
\pgfpathcurveto{\pgfqpoint{0.875032in}{1.627482in}}{\pgfqpoint{0.871760in}{1.619582in}}{\pgfqpoint{0.871760in}{1.611346in}}%
\pgfpathcurveto{\pgfqpoint{0.871760in}{1.603109in}}{\pgfqpoint{0.875032in}{1.595209in}}{\pgfqpoint{0.880856in}{1.589385in}}%
\pgfpathcurveto{\pgfqpoint{0.886680in}{1.583561in}}{\pgfqpoint{0.894580in}{1.580289in}}{\pgfqpoint{0.902816in}{1.580289in}}%
\pgfpathclose%
\pgfusepath{stroke,fill}%
\end{pgfscope}%
\begin{pgfscope}%
\pgfpathrectangle{\pgfqpoint{0.100000in}{0.212622in}}{\pgfqpoint{3.696000in}{3.696000in}}%
\pgfusepath{clip}%
\pgfsetbuttcap%
\pgfsetroundjoin%
\definecolor{currentfill}{rgb}{0.121569,0.466667,0.705882}%
\pgfsetfillcolor{currentfill}%
\pgfsetfillopacity{0.618672}%
\pgfsetlinewidth{1.003750pt}%
\definecolor{currentstroke}{rgb}{0.121569,0.466667,0.705882}%
\pgfsetstrokecolor{currentstroke}%
\pgfsetstrokeopacity{0.618672}%
\pgfsetdash{}{0pt}%
\pgfpathmoveto{\pgfqpoint{0.902127in}{1.578631in}}%
\pgfpathcurveto{\pgfqpoint{0.910363in}{1.578631in}}{\pgfqpoint{0.918263in}{1.581903in}}{\pgfqpoint{0.924087in}{1.587727in}}%
\pgfpathcurveto{\pgfqpoint{0.929911in}{1.593551in}}{\pgfqpoint{0.933183in}{1.601451in}}{\pgfqpoint{0.933183in}{1.609687in}}%
\pgfpathcurveto{\pgfqpoint{0.933183in}{1.617924in}}{\pgfqpoint{0.929911in}{1.625824in}}{\pgfqpoint{0.924087in}{1.631648in}}%
\pgfpathcurveto{\pgfqpoint{0.918263in}{1.637472in}}{\pgfqpoint{0.910363in}{1.640744in}}{\pgfqpoint{0.902127in}{1.640744in}}%
\pgfpathcurveto{\pgfqpoint{0.893890in}{1.640744in}}{\pgfqpoint{0.885990in}{1.637472in}}{\pgfqpoint{0.880167in}{1.631648in}}%
\pgfpathcurveto{\pgfqpoint{0.874343in}{1.625824in}}{\pgfqpoint{0.871070in}{1.617924in}}{\pgfqpoint{0.871070in}{1.609687in}}%
\pgfpathcurveto{\pgfqpoint{0.871070in}{1.601451in}}{\pgfqpoint{0.874343in}{1.593551in}}{\pgfqpoint{0.880167in}{1.587727in}}%
\pgfpathcurveto{\pgfqpoint{0.885990in}{1.581903in}}{\pgfqpoint{0.893890in}{1.578631in}}{\pgfqpoint{0.902127in}{1.578631in}}%
\pgfpathclose%
\pgfusepath{stroke,fill}%
\end{pgfscope}%
\begin{pgfscope}%
\pgfpathrectangle{\pgfqpoint{0.100000in}{0.212622in}}{\pgfqpoint{3.696000in}{3.696000in}}%
\pgfusepath{clip}%
\pgfsetbuttcap%
\pgfsetroundjoin%
\definecolor{currentfill}{rgb}{0.121569,0.466667,0.705882}%
\pgfsetfillcolor{currentfill}%
\pgfsetfillopacity{0.618817}%
\pgfsetlinewidth{1.003750pt}%
\definecolor{currentstroke}{rgb}{0.121569,0.466667,0.705882}%
\pgfsetstrokecolor{currentstroke}%
\pgfsetstrokeopacity{0.618817}%
\pgfsetdash{}{0pt}%
\pgfpathmoveto{\pgfqpoint{0.900747in}{1.575313in}}%
\pgfpathcurveto{\pgfqpoint{0.908983in}{1.575313in}}{\pgfqpoint{0.916883in}{1.578585in}}{\pgfqpoint{0.922707in}{1.584409in}}%
\pgfpathcurveto{\pgfqpoint{0.928531in}{1.590233in}}{\pgfqpoint{0.931803in}{1.598133in}}{\pgfqpoint{0.931803in}{1.606370in}}%
\pgfpathcurveto{\pgfqpoint{0.931803in}{1.614606in}}{\pgfqpoint{0.928531in}{1.622506in}}{\pgfqpoint{0.922707in}{1.628330in}}%
\pgfpathcurveto{\pgfqpoint{0.916883in}{1.634154in}}{\pgfqpoint{0.908983in}{1.637426in}}{\pgfqpoint{0.900747in}{1.637426in}}%
\pgfpathcurveto{\pgfqpoint{0.892511in}{1.637426in}}{\pgfqpoint{0.884611in}{1.634154in}}{\pgfqpoint{0.878787in}{1.628330in}}%
\pgfpathcurveto{\pgfqpoint{0.872963in}{1.622506in}}{\pgfqpoint{0.869690in}{1.614606in}}{\pgfqpoint{0.869690in}{1.606370in}}%
\pgfpathcurveto{\pgfqpoint{0.869690in}{1.598133in}}{\pgfqpoint{0.872963in}{1.590233in}}{\pgfqpoint{0.878787in}{1.584409in}}%
\pgfpathcurveto{\pgfqpoint{0.884611in}{1.578585in}}{\pgfqpoint{0.892511in}{1.575313in}}{\pgfqpoint{0.900747in}{1.575313in}}%
\pgfpathclose%
\pgfusepath{stroke,fill}%
\end{pgfscope}%
\begin{pgfscope}%
\pgfpathrectangle{\pgfqpoint{0.100000in}{0.212622in}}{\pgfqpoint{3.696000in}{3.696000in}}%
\pgfusepath{clip}%
\pgfsetbuttcap%
\pgfsetroundjoin%
\definecolor{currentfill}{rgb}{0.121569,0.466667,0.705882}%
\pgfsetfillcolor{currentfill}%
\pgfsetfillopacity{0.618871}%
\pgfsetlinewidth{1.003750pt}%
\definecolor{currentstroke}{rgb}{0.121569,0.466667,0.705882}%
\pgfsetstrokecolor{currentstroke}%
\pgfsetstrokeopacity{0.618871}%
\pgfsetdash{}{0pt}%
\pgfpathmoveto{\pgfqpoint{0.899622in}{1.572364in}}%
\pgfpathcurveto{\pgfqpoint{0.907859in}{1.572364in}}{\pgfqpoint{0.915759in}{1.575636in}}{\pgfqpoint{0.921583in}{1.581460in}}%
\pgfpathcurveto{\pgfqpoint{0.927406in}{1.587284in}}{\pgfqpoint{0.930679in}{1.595184in}}{\pgfqpoint{0.930679in}{1.603420in}}%
\pgfpathcurveto{\pgfqpoint{0.930679in}{1.611657in}}{\pgfqpoint{0.927406in}{1.619557in}}{\pgfqpoint{0.921583in}{1.625381in}}%
\pgfpathcurveto{\pgfqpoint{0.915759in}{1.631205in}}{\pgfqpoint{0.907859in}{1.634477in}}{\pgfqpoint{0.899622in}{1.634477in}}%
\pgfpathcurveto{\pgfqpoint{0.891386in}{1.634477in}}{\pgfqpoint{0.883486in}{1.631205in}}{\pgfqpoint{0.877662in}{1.625381in}}%
\pgfpathcurveto{\pgfqpoint{0.871838in}{1.619557in}}{\pgfqpoint{0.868566in}{1.611657in}}{\pgfqpoint{0.868566in}{1.603420in}}%
\pgfpathcurveto{\pgfqpoint{0.868566in}{1.595184in}}{\pgfqpoint{0.871838in}{1.587284in}}{\pgfqpoint{0.877662in}{1.581460in}}%
\pgfpathcurveto{\pgfqpoint{0.883486in}{1.575636in}}{\pgfqpoint{0.891386in}{1.572364in}}{\pgfqpoint{0.899622in}{1.572364in}}%
\pgfpathclose%
\pgfusepath{stroke,fill}%
\end{pgfscope}%
\begin{pgfscope}%
\pgfpathrectangle{\pgfqpoint{0.100000in}{0.212622in}}{\pgfqpoint{3.696000in}{3.696000in}}%
\pgfusepath{clip}%
\pgfsetbuttcap%
\pgfsetroundjoin%
\definecolor{currentfill}{rgb}{0.121569,0.466667,0.705882}%
\pgfsetfillcolor{currentfill}%
\pgfsetfillopacity{0.619157}%
\pgfsetlinewidth{1.003750pt}%
\definecolor{currentstroke}{rgb}{0.121569,0.466667,0.705882}%
\pgfsetstrokecolor{currentstroke}%
\pgfsetstrokeopacity{0.619157}%
\pgfsetdash{}{0pt}%
\pgfpathmoveto{\pgfqpoint{2.115372in}{2.034854in}}%
\pgfpathcurveto{\pgfqpoint{2.123608in}{2.034854in}}{\pgfqpoint{2.131508in}{2.038126in}}{\pgfqpoint{2.137332in}{2.043950in}}%
\pgfpathcurveto{\pgfqpoint{2.143156in}{2.049774in}}{\pgfqpoint{2.146429in}{2.057674in}}{\pgfqpoint{2.146429in}{2.065910in}}%
\pgfpathcurveto{\pgfqpoint{2.146429in}{2.074146in}}{\pgfqpoint{2.143156in}{2.082046in}}{\pgfqpoint{2.137332in}{2.087870in}}%
\pgfpathcurveto{\pgfqpoint{2.131508in}{2.093694in}}{\pgfqpoint{2.123608in}{2.096967in}}{\pgfqpoint{2.115372in}{2.096967in}}%
\pgfpathcurveto{\pgfqpoint{2.107136in}{2.096967in}}{\pgfqpoint{2.099236in}{2.093694in}}{\pgfqpoint{2.093412in}{2.087870in}}%
\pgfpathcurveto{\pgfqpoint{2.087588in}{2.082046in}}{\pgfqpoint{2.084316in}{2.074146in}}{\pgfqpoint{2.084316in}{2.065910in}}%
\pgfpathcurveto{\pgfqpoint{2.084316in}{2.057674in}}{\pgfqpoint{2.087588in}{2.049774in}}{\pgfqpoint{2.093412in}{2.043950in}}%
\pgfpathcurveto{\pgfqpoint{2.099236in}{2.038126in}}{\pgfqpoint{2.107136in}{2.034854in}}{\pgfqpoint{2.115372in}{2.034854in}}%
\pgfpathclose%
\pgfusepath{stroke,fill}%
\end{pgfscope}%
\begin{pgfscope}%
\pgfpathrectangle{\pgfqpoint{0.100000in}{0.212622in}}{\pgfqpoint{3.696000in}{3.696000in}}%
\pgfusepath{clip}%
\pgfsetbuttcap%
\pgfsetroundjoin%
\definecolor{currentfill}{rgb}{0.121569,0.466667,0.705882}%
\pgfsetfillcolor{currentfill}%
\pgfsetfillopacity{0.619272}%
\pgfsetlinewidth{1.003750pt}%
\definecolor{currentstroke}{rgb}{0.121569,0.466667,0.705882}%
\pgfsetstrokecolor{currentstroke}%
\pgfsetstrokeopacity{0.619272}%
\pgfsetdash{}{0pt}%
\pgfpathmoveto{\pgfqpoint{0.897515in}{1.568175in}}%
\pgfpathcurveto{\pgfqpoint{0.905751in}{1.568175in}}{\pgfqpoint{0.913651in}{1.571448in}}{\pgfqpoint{0.919475in}{1.577272in}}%
\pgfpathcurveto{\pgfqpoint{0.925299in}{1.583095in}}{\pgfqpoint{0.928572in}{1.590996in}}{\pgfqpoint{0.928572in}{1.599232in}}%
\pgfpathcurveto{\pgfqpoint{0.928572in}{1.607468in}}{\pgfqpoint{0.925299in}{1.615368in}}{\pgfqpoint{0.919475in}{1.621192in}}%
\pgfpathcurveto{\pgfqpoint{0.913651in}{1.627016in}}{\pgfqpoint{0.905751in}{1.630288in}}{\pgfqpoint{0.897515in}{1.630288in}}%
\pgfpathcurveto{\pgfqpoint{0.889279in}{1.630288in}}{\pgfqpoint{0.881379in}{1.627016in}}{\pgfqpoint{0.875555in}{1.621192in}}%
\pgfpathcurveto{\pgfqpoint{0.869731in}{1.615368in}}{\pgfqpoint{0.866459in}{1.607468in}}{\pgfqpoint{0.866459in}{1.599232in}}%
\pgfpathcurveto{\pgfqpoint{0.866459in}{1.590996in}}{\pgfqpoint{0.869731in}{1.583095in}}{\pgfqpoint{0.875555in}{1.577272in}}%
\pgfpathcurveto{\pgfqpoint{0.881379in}{1.571448in}}{\pgfqpoint{0.889279in}{1.568175in}}{\pgfqpoint{0.897515in}{1.568175in}}%
\pgfpathclose%
\pgfusepath{stroke,fill}%
\end{pgfscope}%
\begin{pgfscope}%
\pgfpathrectangle{\pgfqpoint{0.100000in}{0.212622in}}{\pgfqpoint{3.696000in}{3.696000in}}%
\pgfusepath{clip}%
\pgfsetbuttcap%
\pgfsetroundjoin%
\definecolor{currentfill}{rgb}{0.121569,0.466667,0.705882}%
\pgfsetfillcolor{currentfill}%
\pgfsetfillopacity{0.619629}%
\pgfsetlinewidth{1.003750pt}%
\definecolor{currentstroke}{rgb}{0.121569,0.466667,0.705882}%
\pgfsetstrokecolor{currentstroke}%
\pgfsetstrokeopacity{0.619629}%
\pgfsetdash{}{0pt}%
\pgfpathmoveto{\pgfqpoint{0.895533in}{1.564590in}}%
\pgfpathcurveto{\pgfqpoint{0.903770in}{1.564590in}}{\pgfqpoint{0.911670in}{1.567863in}}{\pgfqpoint{0.917494in}{1.573687in}}%
\pgfpathcurveto{\pgfqpoint{0.923317in}{1.579510in}}{\pgfqpoint{0.926590in}{1.587411in}}{\pgfqpoint{0.926590in}{1.595647in}}%
\pgfpathcurveto{\pgfqpoint{0.926590in}{1.603883in}}{\pgfqpoint{0.923317in}{1.611783in}}{\pgfqpoint{0.917494in}{1.617607in}}%
\pgfpathcurveto{\pgfqpoint{0.911670in}{1.623431in}}{\pgfqpoint{0.903770in}{1.626703in}}{\pgfqpoint{0.895533in}{1.626703in}}%
\pgfpathcurveto{\pgfqpoint{0.887297in}{1.626703in}}{\pgfqpoint{0.879397in}{1.623431in}}{\pgfqpoint{0.873573in}{1.617607in}}%
\pgfpathcurveto{\pgfqpoint{0.867749in}{1.611783in}}{\pgfqpoint{0.864477in}{1.603883in}}{\pgfqpoint{0.864477in}{1.595647in}}%
\pgfpathcurveto{\pgfqpoint{0.864477in}{1.587411in}}{\pgfqpoint{0.867749in}{1.579510in}}{\pgfqpoint{0.873573in}{1.573687in}}%
\pgfpathcurveto{\pgfqpoint{0.879397in}{1.567863in}}{\pgfqpoint{0.887297in}{1.564590in}}{\pgfqpoint{0.895533in}{1.564590in}}%
\pgfpathclose%
\pgfusepath{stroke,fill}%
\end{pgfscope}%
\begin{pgfscope}%
\pgfpathrectangle{\pgfqpoint{0.100000in}{0.212622in}}{\pgfqpoint{3.696000in}{3.696000in}}%
\pgfusepath{clip}%
\pgfsetbuttcap%
\pgfsetroundjoin%
\definecolor{currentfill}{rgb}{0.121569,0.466667,0.705882}%
\pgfsetfillcolor{currentfill}%
\pgfsetfillopacity{0.619897}%
\pgfsetlinewidth{1.003750pt}%
\definecolor{currentstroke}{rgb}{0.121569,0.466667,0.705882}%
\pgfsetstrokecolor{currentstroke}%
\pgfsetstrokeopacity{0.619897}%
\pgfsetdash{}{0pt}%
\pgfpathmoveto{\pgfqpoint{0.894404in}{1.561947in}}%
\pgfpathcurveto{\pgfqpoint{0.902641in}{1.561947in}}{\pgfqpoint{0.910541in}{1.565220in}}{\pgfqpoint{0.916365in}{1.571044in}}%
\pgfpathcurveto{\pgfqpoint{0.922189in}{1.576868in}}{\pgfqpoint{0.925461in}{1.584768in}}{\pgfqpoint{0.925461in}{1.593004in}}%
\pgfpathcurveto{\pgfqpoint{0.925461in}{1.601240in}}{\pgfqpoint{0.922189in}{1.609140in}}{\pgfqpoint{0.916365in}{1.614964in}}%
\pgfpathcurveto{\pgfqpoint{0.910541in}{1.620788in}}{\pgfqpoint{0.902641in}{1.624060in}}{\pgfqpoint{0.894404in}{1.624060in}}%
\pgfpathcurveto{\pgfqpoint{0.886168in}{1.624060in}}{\pgfqpoint{0.878268in}{1.620788in}}{\pgfqpoint{0.872444in}{1.614964in}}%
\pgfpathcurveto{\pgfqpoint{0.866620in}{1.609140in}}{\pgfqpoint{0.863348in}{1.601240in}}{\pgfqpoint{0.863348in}{1.593004in}}%
\pgfpathcurveto{\pgfqpoint{0.863348in}{1.584768in}}{\pgfqpoint{0.866620in}{1.576868in}}{\pgfqpoint{0.872444in}{1.571044in}}%
\pgfpathcurveto{\pgfqpoint{0.878268in}{1.565220in}}{\pgfqpoint{0.886168in}{1.561947in}}{\pgfqpoint{0.894404in}{1.561947in}}%
\pgfpathclose%
\pgfusepath{stroke,fill}%
\end{pgfscope}%
\begin{pgfscope}%
\pgfpathrectangle{\pgfqpoint{0.100000in}{0.212622in}}{\pgfqpoint{3.696000in}{3.696000in}}%
\pgfusepath{clip}%
\pgfsetbuttcap%
\pgfsetroundjoin%
\definecolor{currentfill}{rgb}{0.121569,0.466667,0.705882}%
\pgfsetfillcolor{currentfill}%
\pgfsetfillopacity{0.620310}%
\pgfsetlinewidth{1.003750pt}%
\definecolor{currentstroke}{rgb}{0.121569,0.466667,0.705882}%
\pgfsetstrokecolor{currentstroke}%
\pgfsetstrokeopacity{0.620310}%
\pgfsetdash{}{0pt}%
\pgfpathmoveto{\pgfqpoint{0.892046in}{1.557333in}}%
\pgfpathcurveto{\pgfqpoint{0.900282in}{1.557333in}}{\pgfqpoint{0.908182in}{1.560605in}}{\pgfqpoint{0.914006in}{1.566429in}}%
\pgfpathcurveto{\pgfqpoint{0.919830in}{1.572253in}}{\pgfqpoint{0.923102in}{1.580153in}}{\pgfqpoint{0.923102in}{1.588389in}}%
\pgfpathcurveto{\pgfqpoint{0.923102in}{1.596626in}}{\pgfqpoint{0.919830in}{1.604526in}}{\pgfqpoint{0.914006in}{1.610350in}}%
\pgfpathcurveto{\pgfqpoint{0.908182in}{1.616174in}}{\pgfqpoint{0.900282in}{1.619446in}}{\pgfqpoint{0.892046in}{1.619446in}}%
\pgfpathcurveto{\pgfqpoint{0.883809in}{1.619446in}}{\pgfqpoint{0.875909in}{1.616174in}}{\pgfqpoint{0.870085in}{1.610350in}}%
\pgfpathcurveto{\pgfqpoint{0.864261in}{1.604526in}}{\pgfqpoint{0.860989in}{1.596626in}}{\pgfqpoint{0.860989in}{1.588389in}}%
\pgfpathcurveto{\pgfqpoint{0.860989in}{1.580153in}}{\pgfqpoint{0.864261in}{1.572253in}}{\pgfqpoint{0.870085in}{1.566429in}}%
\pgfpathcurveto{\pgfqpoint{0.875909in}{1.560605in}}{\pgfqpoint{0.883809in}{1.557333in}}{\pgfqpoint{0.892046in}{1.557333in}}%
\pgfpathclose%
\pgfusepath{stroke,fill}%
\end{pgfscope}%
\begin{pgfscope}%
\pgfpathrectangle{\pgfqpoint{0.100000in}{0.212622in}}{\pgfqpoint{3.696000in}{3.696000in}}%
\pgfusepath{clip}%
\pgfsetbuttcap%
\pgfsetroundjoin%
\definecolor{currentfill}{rgb}{0.121569,0.466667,0.705882}%
\pgfsetfillcolor{currentfill}%
\pgfsetfillopacity{0.620587}%
\pgfsetlinewidth{1.003750pt}%
\definecolor{currentstroke}{rgb}{0.121569,0.466667,0.705882}%
\pgfsetstrokecolor{currentstroke}%
\pgfsetstrokeopacity{0.620587}%
\pgfsetdash{}{0pt}%
\pgfpathmoveto{\pgfqpoint{0.891074in}{1.554597in}}%
\pgfpathcurveto{\pgfqpoint{0.899310in}{1.554597in}}{\pgfqpoint{0.907210in}{1.557870in}}{\pgfqpoint{0.913034in}{1.563694in}}%
\pgfpathcurveto{\pgfqpoint{0.918858in}{1.569517in}}{\pgfqpoint{0.922131in}{1.577417in}}{\pgfqpoint{0.922131in}{1.585654in}}%
\pgfpathcurveto{\pgfqpoint{0.922131in}{1.593890in}}{\pgfqpoint{0.918858in}{1.601790in}}{\pgfqpoint{0.913034in}{1.607614in}}%
\pgfpathcurveto{\pgfqpoint{0.907210in}{1.613438in}}{\pgfqpoint{0.899310in}{1.616710in}}{\pgfqpoint{0.891074in}{1.616710in}}%
\pgfpathcurveto{\pgfqpoint{0.882838in}{1.616710in}}{\pgfqpoint{0.874938in}{1.613438in}}{\pgfqpoint{0.869114in}{1.607614in}}%
\pgfpathcurveto{\pgfqpoint{0.863290in}{1.601790in}}{\pgfqpoint{0.860018in}{1.593890in}}{\pgfqpoint{0.860018in}{1.585654in}}%
\pgfpathcurveto{\pgfqpoint{0.860018in}{1.577417in}}{\pgfqpoint{0.863290in}{1.569517in}}{\pgfqpoint{0.869114in}{1.563694in}}%
\pgfpathcurveto{\pgfqpoint{0.874938in}{1.557870in}}{\pgfqpoint{0.882838in}{1.554597in}}{\pgfqpoint{0.891074in}{1.554597in}}%
\pgfpathclose%
\pgfusepath{stroke,fill}%
\end{pgfscope}%
\begin{pgfscope}%
\pgfpathrectangle{\pgfqpoint{0.100000in}{0.212622in}}{\pgfqpoint{3.696000in}{3.696000in}}%
\pgfusepath{clip}%
\pgfsetbuttcap%
\pgfsetroundjoin%
\definecolor{currentfill}{rgb}{0.121569,0.466667,0.705882}%
\pgfsetfillcolor{currentfill}%
\pgfsetfillopacity{0.620732}%
\pgfsetlinewidth{1.003750pt}%
\definecolor{currentstroke}{rgb}{0.121569,0.466667,0.705882}%
\pgfsetstrokecolor{currentstroke}%
\pgfsetstrokeopacity{0.620732}%
\pgfsetdash{}{0pt}%
\pgfpathmoveto{\pgfqpoint{0.715394in}{1.205895in}}%
\pgfpathcurveto{\pgfqpoint{0.723630in}{1.205895in}}{\pgfqpoint{0.731530in}{1.209168in}}{\pgfqpoint{0.737354in}{1.214992in}}%
\pgfpathcurveto{\pgfqpoint{0.743178in}{1.220816in}}{\pgfqpoint{0.746451in}{1.228716in}}{\pgfqpoint{0.746451in}{1.236952in}}%
\pgfpathcurveto{\pgfqpoint{0.746451in}{1.245188in}}{\pgfqpoint{0.743178in}{1.253088in}}{\pgfqpoint{0.737354in}{1.258912in}}%
\pgfpathcurveto{\pgfqpoint{0.731530in}{1.264736in}}{\pgfqpoint{0.723630in}{1.268008in}}{\pgfqpoint{0.715394in}{1.268008in}}%
\pgfpathcurveto{\pgfqpoint{0.707158in}{1.268008in}}{\pgfqpoint{0.699258in}{1.264736in}}{\pgfqpoint{0.693434in}{1.258912in}}%
\pgfpathcurveto{\pgfqpoint{0.687610in}{1.253088in}}{\pgfqpoint{0.684338in}{1.245188in}}{\pgfqpoint{0.684338in}{1.236952in}}%
\pgfpathcurveto{\pgfqpoint{0.684338in}{1.228716in}}{\pgfqpoint{0.687610in}{1.220816in}}{\pgfqpoint{0.693434in}{1.214992in}}%
\pgfpathcurveto{\pgfqpoint{0.699258in}{1.209168in}}{\pgfqpoint{0.707158in}{1.205895in}}{\pgfqpoint{0.715394in}{1.205895in}}%
\pgfpathclose%
\pgfusepath{stroke,fill}%
\end{pgfscope}%
\begin{pgfscope}%
\pgfpathrectangle{\pgfqpoint{0.100000in}{0.212622in}}{\pgfqpoint{3.696000in}{3.696000in}}%
\pgfusepath{clip}%
\pgfsetbuttcap%
\pgfsetroundjoin%
\definecolor{currentfill}{rgb}{0.121569,0.466667,0.705882}%
\pgfsetfillcolor{currentfill}%
\pgfsetfillopacity{0.620927}%
\pgfsetlinewidth{1.003750pt}%
\definecolor{currentstroke}{rgb}{0.121569,0.466667,0.705882}%
\pgfsetstrokecolor{currentstroke}%
\pgfsetstrokeopacity{0.620927}%
\pgfsetdash{}{0pt}%
\pgfpathmoveto{\pgfqpoint{2.117143in}{2.028511in}}%
\pgfpathcurveto{\pgfqpoint{2.125379in}{2.028511in}}{\pgfqpoint{2.133279in}{2.031783in}}{\pgfqpoint{2.139103in}{2.037607in}}%
\pgfpathcurveto{\pgfqpoint{2.144927in}{2.043431in}}{\pgfqpoint{2.148199in}{2.051331in}}{\pgfqpoint{2.148199in}{2.059567in}}%
\pgfpathcurveto{\pgfqpoint{2.148199in}{2.067803in}}{\pgfqpoint{2.144927in}{2.075703in}}{\pgfqpoint{2.139103in}{2.081527in}}%
\pgfpathcurveto{\pgfqpoint{2.133279in}{2.087351in}}{\pgfqpoint{2.125379in}{2.090624in}}{\pgfqpoint{2.117143in}{2.090624in}}%
\pgfpathcurveto{\pgfqpoint{2.108906in}{2.090624in}}{\pgfqpoint{2.101006in}{2.087351in}}{\pgfqpoint{2.095182in}{2.081527in}}%
\pgfpathcurveto{\pgfqpoint{2.089358in}{2.075703in}}{\pgfqpoint{2.086086in}{2.067803in}}{\pgfqpoint{2.086086in}{2.059567in}}%
\pgfpathcurveto{\pgfqpoint{2.086086in}{2.051331in}}{\pgfqpoint{2.089358in}{2.043431in}}{\pgfqpoint{2.095182in}{2.037607in}}%
\pgfpathcurveto{\pgfqpoint{2.101006in}{2.031783in}}{\pgfqpoint{2.108906in}{2.028511in}}{\pgfqpoint{2.117143in}{2.028511in}}%
\pgfpathclose%
\pgfusepath{stroke,fill}%
\end{pgfscope}%
\begin{pgfscope}%
\pgfpathrectangle{\pgfqpoint{0.100000in}{0.212622in}}{\pgfqpoint{3.696000in}{3.696000in}}%
\pgfusepath{clip}%
\pgfsetbuttcap%
\pgfsetroundjoin%
\definecolor{currentfill}{rgb}{0.121569,0.466667,0.705882}%
\pgfsetfillcolor{currentfill}%
\pgfsetfillopacity{0.621053}%
\pgfsetlinewidth{1.003750pt}%
\definecolor{currentstroke}{rgb}{0.121569,0.466667,0.705882}%
\pgfsetstrokecolor{currentstroke}%
\pgfsetstrokeopacity{0.621053}%
\pgfsetdash{}{0pt}%
\pgfpathmoveto{\pgfqpoint{0.861282in}{1.491095in}}%
\pgfpathcurveto{\pgfqpoint{0.869518in}{1.491095in}}{\pgfqpoint{0.877418in}{1.494367in}}{\pgfqpoint{0.883242in}{1.500191in}}%
\pgfpathcurveto{\pgfqpoint{0.889066in}{1.506015in}}{\pgfqpoint{0.892338in}{1.513915in}}{\pgfqpoint{0.892338in}{1.522152in}}%
\pgfpathcurveto{\pgfqpoint{0.892338in}{1.530388in}}{\pgfqpoint{0.889066in}{1.538288in}}{\pgfqpoint{0.883242in}{1.544112in}}%
\pgfpathcurveto{\pgfqpoint{0.877418in}{1.549936in}}{\pgfqpoint{0.869518in}{1.553208in}}{\pgfqpoint{0.861282in}{1.553208in}}%
\pgfpathcurveto{\pgfqpoint{0.853045in}{1.553208in}}{\pgfqpoint{0.845145in}{1.549936in}}{\pgfqpoint{0.839321in}{1.544112in}}%
\pgfpathcurveto{\pgfqpoint{0.833497in}{1.538288in}}{\pgfqpoint{0.830225in}{1.530388in}}{\pgfqpoint{0.830225in}{1.522152in}}%
\pgfpathcurveto{\pgfqpoint{0.830225in}{1.513915in}}{\pgfqpoint{0.833497in}{1.506015in}}{\pgfqpoint{0.839321in}{1.500191in}}%
\pgfpathcurveto{\pgfqpoint{0.845145in}{1.494367in}}{\pgfqpoint{0.853045in}{1.491095in}}{\pgfqpoint{0.861282in}{1.491095in}}%
\pgfpathclose%
\pgfusepath{stroke,fill}%
\end{pgfscope}%
\begin{pgfscope}%
\pgfpathrectangle{\pgfqpoint{0.100000in}{0.212622in}}{\pgfqpoint{3.696000in}{3.696000in}}%
\pgfusepath{clip}%
\pgfsetbuttcap%
\pgfsetroundjoin%
\definecolor{currentfill}{rgb}{0.121569,0.466667,0.705882}%
\pgfsetfillcolor{currentfill}%
\pgfsetfillopacity{0.621057}%
\pgfsetlinewidth{1.003750pt}%
\definecolor{currentstroke}{rgb}{0.121569,0.466667,0.705882}%
\pgfsetstrokecolor{currentstroke}%
\pgfsetstrokeopacity{0.621057}%
\pgfsetdash{}{0pt}%
\pgfpathmoveto{\pgfqpoint{0.888728in}{1.550254in}}%
\pgfpathcurveto{\pgfqpoint{0.896964in}{1.550254in}}{\pgfqpoint{0.904864in}{1.553526in}}{\pgfqpoint{0.910688in}{1.559350in}}%
\pgfpathcurveto{\pgfqpoint{0.916512in}{1.565174in}}{\pgfqpoint{0.919784in}{1.573074in}}{\pgfqpoint{0.919784in}{1.581310in}}%
\pgfpathcurveto{\pgfqpoint{0.919784in}{1.589547in}}{\pgfqpoint{0.916512in}{1.597447in}}{\pgfqpoint{0.910688in}{1.603271in}}%
\pgfpathcurveto{\pgfqpoint{0.904864in}{1.609094in}}{\pgfqpoint{0.896964in}{1.612367in}}{\pgfqpoint{0.888728in}{1.612367in}}%
\pgfpathcurveto{\pgfqpoint{0.880492in}{1.612367in}}{\pgfqpoint{0.872592in}{1.609094in}}{\pgfqpoint{0.866768in}{1.603271in}}%
\pgfpathcurveto{\pgfqpoint{0.860944in}{1.597447in}}{\pgfqpoint{0.857671in}{1.589547in}}{\pgfqpoint{0.857671in}{1.581310in}}%
\pgfpathcurveto{\pgfqpoint{0.857671in}{1.573074in}}{\pgfqpoint{0.860944in}{1.565174in}}{\pgfqpoint{0.866768in}{1.559350in}}%
\pgfpathcurveto{\pgfqpoint{0.872592in}{1.553526in}}{\pgfqpoint{0.880492in}{1.550254in}}{\pgfqpoint{0.888728in}{1.550254in}}%
\pgfpathclose%
\pgfusepath{stroke,fill}%
\end{pgfscope}%
\begin{pgfscope}%
\pgfpathrectangle{\pgfqpoint{0.100000in}{0.212622in}}{\pgfqpoint{3.696000in}{3.696000in}}%
\pgfusepath{clip}%
\pgfsetbuttcap%
\pgfsetroundjoin%
\definecolor{currentfill}{rgb}{0.121569,0.466667,0.705882}%
\pgfsetfillcolor{currentfill}%
\pgfsetfillopacity{0.621984}%
\pgfsetlinewidth{1.003750pt}%
\definecolor{currentstroke}{rgb}{0.121569,0.466667,0.705882}%
\pgfsetstrokecolor{currentstroke}%
\pgfsetstrokeopacity{0.621984}%
\pgfsetdash{}{0pt}%
\pgfpathmoveto{\pgfqpoint{0.884578in}{1.542406in}}%
\pgfpathcurveto{\pgfqpoint{0.892815in}{1.542406in}}{\pgfqpoint{0.900715in}{1.545678in}}{\pgfqpoint{0.906539in}{1.551502in}}%
\pgfpathcurveto{\pgfqpoint{0.912363in}{1.557326in}}{\pgfqpoint{0.915635in}{1.565226in}}{\pgfqpoint{0.915635in}{1.573462in}}%
\pgfpathcurveto{\pgfqpoint{0.915635in}{1.581698in}}{\pgfqpoint{0.912363in}{1.589598in}}{\pgfqpoint{0.906539in}{1.595422in}}%
\pgfpathcurveto{\pgfqpoint{0.900715in}{1.601246in}}{\pgfqpoint{0.892815in}{1.604519in}}{\pgfqpoint{0.884578in}{1.604519in}}%
\pgfpathcurveto{\pgfqpoint{0.876342in}{1.604519in}}{\pgfqpoint{0.868442in}{1.601246in}}{\pgfqpoint{0.862618in}{1.595422in}}%
\pgfpathcurveto{\pgfqpoint{0.856794in}{1.589598in}}{\pgfqpoint{0.853522in}{1.581698in}}{\pgfqpoint{0.853522in}{1.573462in}}%
\pgfpathcurveto{\pgfqpoint{0.853522in}{1.565226in}}{\pgfqpoint{0.856794in}{1.557326in}}{\pgfqpoint{0.862618in}{1.551502in}}%
\pgfpathcurveto{\pgfqpoint{0.868442in}{1.545678in}}{\pgfqpoint{0.876342in}{1.542406in}}{\pgfqpoint{0.884578in}{1.542406in}}%
\pgfpathclose%
\pgfusepath{stroke,fill}%
\end{pgfscope}%
\begin{pgfscope}%
\pgfpathrectangle{\pgfqpoint{0.100000in}{0.212622in}}{\pgfqpoint{3.696000in}{3.696000in}}%
\pgfusepath{clip}%
\pgfsetbuttcap%
\pgfsetroundjoin%
\definecolor{currentfill}{rgb}{0.121569,0.466667,0.705882}%
\pgfsetfillcolor{currentfill}%
\pgfsetfillopacity{0.622759}%
\pgfsetlinewidth{1.003750pt}%
\definecolor{currentstroke}{rgb}{0.121569,0.466667,0.705882}%
\pgfsetstrokecolor{currentstroke}%
\pgfsetstrokeopacity{0.622759}%
\pgfsetdash{}{0pt}%
\pgfpathmoveto{\pgfqpoint{0.860483in}{1.491464in}}%
\pgfpathcurveto{\pgfqpoint{0.868720in}{1.491464in}}{\pgfqpoint{0.876620in}{1.494736in}}{\pgfqpoint{0.882444in}{1.500560in}}%
\pgfpathcurveto{\pgfqpoint{0.888268in}{1.506384in}}{\pgfqpoint{0.891540in}{1.514284in}}{\pgfqpoint{0.891540in}{1.522520in}}%
\pgfpathcurveto{\pgfqpoint{0.891540in}{1.530757in}}{\pgfqpoint{0.888268in}{1.538657in}}{\pgfqpoint{0.882444in}{1.544481in}}%
\pgfpathcurveto{\pgfqpoint{0.876620in}{1.550305in}}{\pgfqpoint{0.868720in}{1.553577in}}{\pgfqpoint{0.860483in}{1.553577in}}%
\pgfpathcurveto{\pgfqpoint{0.852247in}{1.553577in}}{\pgfqpoint{0.844347in}{1.550305in}}{\pgfqpoint{0.838523in}{1.544481in}}%
\pgfpathcurveto{\pgfqpoint{0.832699in}{1.538657in}}{\pgfqpoint{0.829427in}{1.530757in}}{\pgfqpoint{0.829427in}{1.522520in}}%
\pgfpathcurveto{\pgfqpoint{0.829427in}{1.514284in}}{\pgfqpoint{0.832699in}{1.506384in}}{\pgfqpoint{0.838523in}{1.500560in}}%
\pgfpathcurveto{\pgfqpoint{0.844347in}{1.494736in}}{\pgfqpoint{0.852247in}{1.491464in}}{\pgfqpoint{0.860483in}{1.491464in}}%
\pgfpathclose%
\pgfusepath{stroke,fill}%
\end{pgfscope}%
\begin{pgfscope}%
\pgfpathrectangle{\pgfqpoint{0.100000in}{0.212622in}}{\pgfqpoint{3.696000in}{3.696000in}}%
\pgfusepath{clip}%
\pgfsetbuttcap%
\pgfsetroundjoin%
\definecolor{currentfill}{rgb}{0.121569,0.466667,0.705882}%
\pgfsetfillcolor{currentfill}%
\pgfsetfillopacity{0.622832}%
\pgfsetlinewidth{1.003750pt}%
\definecolor{currentstroke}{rgb}{0.121569,0.466667,0.705882}%
\pgfsetstrokecolor{currentstroke}%
\pgfsetstrokeopacity{0.622832}%
\pgfsetdash{}{0pt}%
\pgfpathmoveto{\pgfqpoint{0.881239in}{1.535428in}}%
\pgfpathcurveto{\pgfqpoint{0.889475in}{1.535428in}}{\pgfqpoint{0.897375in}{1.538700in}}{\pgfqpoint{0.903199in}{1.544524in}}%
\pgfpathcurveto{\pgfqpoint{0.909023in}{1.550348in}}{\pgfqpoint{0.912295in}{1.558248in}}{\pgfqpoint{0.912295in}{1.566484in}}%
\pgfpathcurveto{\pgfqpoint{0.912295in}{1.574721in}}{\pgfqpoint{0.909023in}{1.582621in}}{\pgfqpoint{0.903199in}{1.588445in}}%
\pgfpathcurveto{\pgfqpoint{0.897375in}{1.594269in}}{\pgfqpoint{0.889475in}{1.597541in}}{\pgfqpoint{0.881239in}{1.597541in}}%
\pgfpathcurveto{\pgfqpoint{0.873002in}{1.597541in}}{\pgfqpoint{0.865102in}{1.594269in}}{\pgfqpoint{0.859278in}{1.588445in}}%
\pgfpathcurveto{\pgfqpoint{0.853454in}{1.582621in}}{\pgfqpoint{0.850182in}{1.574721in}}{\pgfqpoint{0.850182in}{1.566484in}}%
\pgfpathcurveto{\pgfqpoint{0.850182in}{1.558248in}}{\pgfqpoint{0.853454in}{1.550348in}}{\pgfqpoint{0.859278in}{1.544524in}}%
\pgfpathcurveto{\pgfqpoint{0.865102in}{1.538700in}}{\pgfqpoint{0.873002in}{1.535428in}}{\pgfqpoint{0.881239in}{1.535428in}}%
\pgfpathclose%
\pgfusepath{stroke,fill}%
\end{pgfscope}%
\begin{pgfscope}%
\pgfpathrectangle{\pgfqpoint{0.100000in}{0.212622in}}{\pgfqpoint{3.696000in}{3.696000in}}%
\pgfusepath{clip}%
\pgfsetbuttcap%
\pgfsetroundjoin%
\definecolor{currentfill}{rgb}{0.121569,0.466667,0.705882}%
\pgfsetfillcolor{currentfill}%
\pgfsetfillopacity{0.623294}%
\pgfsetlinewidth{1.003750pt}%
\definecolor{currentstroke}{rgb}{0.121569,0.466667,0.705882}%
\pgfsetstrokecolor{currentstroke}%
\pgfsetstrokeopacity{0.623294}%
\pgfsetdash{}{0pt}%
\pgfpathmoveto{\pgfqpoint{2.118541in}{2.021335in}}%
\pgfpathcurveto{\pgfqpoint{2.126778in}{2.021335in}}{\pgfqpoint{2.134678in}{2.024607in}}{\pgfqpoint{2.140502in}{2.030431in}}%
\pgfpathcurveto{\pgfqpoint{2.146326in}{2.036255in}}{\pgfqpoint{2.149598in}{2.044155in}}{\pgfqpoint{2.149598in}{2.052391in}}%
\pgfpathcurveto{\pgfqpoint{2.149598in}{2.060628in}}{\pgfqpoint{2.146326in}{2.068528in}}{\pgfqpoint{2.140502in}{2.074352in}}%
\pgfpathcurveto{\pgfqpoint{2.134678in}{2.080175in}}{\pgfqpoint{2.126778in}{2.083448in}}{\pgfqpoint{2.118541in}{2.083448in}}%
\pgfpathcurveto{\pgfqpoint{2.110305in}{2.083448in}}{\pgfqpoint{2.102405in}{2.080175in}}{\pgfqpoint{2.096581in}{2.074352in}}%
\pgfpathcurveto{\pgfqpoint{2.090757in}{2.068528in}}{\pgfqpoint{2.087485in}{2.060628in}}{\pgfqpoint{2.087485in}{2.052391in}}%
\pgfpathcurveto{\pgfqpoint{2.087485in}{2.044155in}}{\pgfqpoint{2.090757in}{2.036255in}}{\pgfqpoint{2.096581in}{2.030431in}}%
\pgfpathcurveto{\pgfqpoint{2.102405in}{2.024607in}}{\pgfqpoint{2.110305in}{2.021335in}}{\pgfqpoint{2.118541in}{2.021335in}}%
\pgfpathclose%
\pgfusepath{stroke,fill}%
\end{pgfscope}%
\begin{pgfscope}%
\pgfpathrectangle{\pgfqpoint{0.100000in}{0.212622in}}{\pgfqpoint{3.696000in}{3.696000in}}%
\pgfusepath{clip}%
\pgfsetbuttcap%
\pgfsetroundjoin%
\definecolor{currentfill}{rgb}{0.121569,0.466667,0.705882}%
\pgfsetfillcolor{currentfill}%
\pgfsetfillopacity{0.623696}%
\pgfsetlinewidth{1.003750pt}%
\definecolor{currentstroke}{rgb}{0.121569,0.466667,0.705882}%
\pgfsetstrokecolor{currentstroke}%
\pgfsetstrokeopacity{0.623696}%
\pgfsetdash{}{0pt}%
\pgfpathmoveto{\pgfqpoint{0.860116in}{1.491665in}}%
\pgfpathcurveto{\pgfqpoint{0.868352in}{1.491665in}}{\pgfqpoint{0.876252in}{1.494938in}}{\pgfqpoint{0.882076in}{1.500762in}}%
\pgfpathcurveto{\pgfqpoint{0.887900in}{1.506586in}}{\pgfqpoint{0.891172in}{1.514486in}}{\pgfqpoint{0.891172in}{1.522722in}}%
\pgfpathcurveto{\pgfqpoint{0.891172in}{1.530958in}}{\pgfqpoint{0.887900in}{1.538858in}}{\pgfqpoint{0.882076in}{1.544682in}}%
\pgfpathcurveto{\pgfqpoint{0.876252in}{1.550506in}}{\pgfqpoint{0.868352in}{1.553778in}}{\pgfqpoint{0.860116in}{1.553778in}}%
\pgfpathcurveto{\pgfqpoint{0.851879in}{1.553778in}}{\pgfqpoint{0.843979in}{1.550506in}}{\pgfqpoint{0.838155in}{1.544682in}}%
\pgfpathcurveto{\pgfqpoint{0.832331in}{1.538858in}}{\pgfqpoint{0.829059in}{1.530958in}}{\pgfqpoint{0.829059in}{1.522722in}}%
\pgfpathcurveto{\pgfqpoint{0.829059in}{1.514486in}}{\pgfqpoint{0.832331in}{1.506586in}}{\pgfqpoint{0.838155in}{1.500762in}}%
\pgfpathcurveto{\pgfqpoint{0.843979in}{1.494938in}}{\pgfqpoint{0.851879in}{1.491665in}}{\pgfqpoint{0.860116in}{1.491665in}}%
\pgfpathclose%
\pgfusepath{stroke,fill}%
\end{pgfscope}%
\begin{pgfscope}%
\pgfpathrectangle{\pgfqpoint{0.100000in}{0.212622in}}{\pgfqpoint{3.696000in}{3.696000in}}%
\pgfusepath{clip}%
\pgfsetbuttcap%
\pgfsetroundjoin%
\definecolor{currentfill}{rgb}{0.121569,0.466667,0.705882}%
\pgfsetfillcolor{currentfill}%
\pgfsetfillopacity{0.624198}%
\pgfsetlinewidth{1.003750pt}%
\definecolor{currentstroke}{rgb}{0.121569,0.466667,0.705882}%
\pgfsetstrokecolor{currentstroke}%
\pgfsetstrokeopacity{0.624198}%
\pgfsetdash{}{0pt}%
\pgfpathmoveto{\pgfqpoint{0.874576in}{1.523018in}}%
\pgfpathcurveto{\pgfqpoint{0.882812in}{1.523018in}}{\pgfqpoint{0.890713in}{1.526291in}}{\pgfqpoint{0.896536in}{1.532115in}}%
\pgfpathcurveto{\pgfqpoint{0.902360in}{1.537939in}}{\pgfqpoint{0.905633in}{1.545839in}}{\pgfqpoint{0.905633in}{1.554075in}}%
\pgfpathcurveto{\pgfqpoint{0.905633in}{1.562311in}}{\pgfqpoint{0.902360in}{1.570211in}}{\pgfqpoint{0.896536in}{1.576035in}}%
\pgfpathcurveto{\pgfqpoint{0.890713in}{1.581859in}}{\pgfqpoint{0.882812in}{1.585131in}}{\pgfqpoint{0.874576in}{1.585131in}}%
\pgfpathcurveto{\pgfqpoint{0.866340in}{1.585131in}}{\pgfqpoint{0.858440in}{1.581859in}}{\pgfqpoint{0.852616in}{1.576035in}}%
\pgfpathcurveto{\pgfqpoint{0.846792in}{1.570211in}}{\pgfqpoint{0.843520in}{1.562311in}}{\pgfqpoint{0.843520in}{1.554075in}}%
\pgfpathcurveto{\pgfqpoint{0.843520in}{1.545839in}}{\pgfqpoint{0.846792in}{1.537939in}}{\pgfqpoint{0.852616in}{1.532115in}}%
\pgfpathcurveto{\pgfqpoint{0.858440in}{1.526291in}}{\pgfqpoint{0.866340in}{1.523018in}}{\pgfqpoint{0.874576in}{1.523018in}}%
\pgfpathclose%
\pgfusepath{stroke,fill}%
\end{pgfscope}%
\begin{pgfscope}%
\pgfpathrectangle{\pgfqpoint{0.100000in}{0.212622in}}{\pgfqpoint{3.696000in}{3.696000in}}%
\pgfusepath{clip}%
\pgfsetbuttcap%
\pgfsetroundjoin%
\definecolor{currentfill}{rgb}{0.121569,0.466667,0.705882}%
\pgfsetfillcolor{currentfill}%
\pgfsetfillopacity{0.624209}%
\pgfsetlinewidth{1.003750pt}%
\definecolor{currentstroke}{rgb}{0.121569,0.466667,0.705882}%
\pgfsetstrokecolor{currentstroke}%
\pgfsetstrokeopacity{0.624209}%
\pgfsetdash{}{0pt}%
\pgfpathmoveto{\pgfqpoint{0.859945in}{1.491771in}}%
\pgfpathcurveto{\pgfqpoint{0.868181in}{1.491771in}}{\pgfqpoint{0.876082in}{1.495044in}}{\pgfqpoint{0.881905in}{1.500868in}}%
\pgfpathcurveto{\pgfqpoint{0.887729in}{1.506692in}}{\pgfqpoint{0.891002in}{1.514592in}}{\pgfqpoint{0.891002in}{1.522828in}}%
\pgfpathcurveto{\pgfqpoint{0.891002in}{1.531064in}}{\pgfqpoint{0.887729in}{1.538964in}}{\pgfqpoint{0.881905in}{1.544788in}}%
\pgfpathcurveto{\pgfqpoint{0.876082in}{1.550612in}}{\pgfqpoint{0.868181in}{1.553884in}}{\pgfqpoint{0.859945in}{1.553884in}}%
\pgfpathcurveto{\pgfqpoint{0.851709in}{1.553884in}}{\pgfqpoint{0.843809in}{1.550612in}}{\pgfqpoint{0.837985in}{1.544788in}}%
\pgfpathcurveto{\pgfqpoint{0.832161in}{1.538964in}}{\pgfqpoint{0.828889in}{1.531064in}}{\pgfqpoint{0.828889in}{1.522828in}}%
\pgfpathcurveto{\pgfqpoint{0.828889in}{1.514592in}}{\pgfqpoint{0.832161in}{1.506692in}}{\pgfqpoint{0.837985in}{1.500868in}}%
\pgfpathcurveto{\pgfqpoint{0.843809in}{1.495044in}}{\pgfqpoint{0.851709in}{1.491771in}}{\pgfqpoint{0.859945in}{1.491771in}}%
\pgfpathclose%
\pgfusepath{stroke,fill}%
\end{pgfscope}%
\begin{pgfscope}%
\pgfpathrectangle{\pgfqpoint{0.100000in}{0.212622in}}{\pgfqpoint{3.696000in}{3.696000in}}%
\pgfusepath{clip}%
\pgfsetbuttcap%
\pgfsetroundjoin%
\definecolor{currentfill}{rgb}{0.121569,0.466667,0.705882}%
\pgfsetfillcolor{currentfill}%
\pgfsetfillopacity{0.624556}%
\pgfsetlinewidth{1.003750pt}%
\definecolor{currentstroke}{rgb}{0.121569,0.466667,0.705882}%
\pgfsetstrokecolor{currentstroke}%
\pgfsetstrokeopacity{0.624556}%
\pgfsetdash{}{0pt}%
\pgfpathmoveto{\pgfqpoint{2.119465in}{2.017383in}}%
\pgfpathcurveto{\pgfqpoint{2.127701in}{2.017383in}}{\pgfqpoint{2.135601in}{2.020655in}}{\pgfqpoint{2.141425in}{2.026479in}}%
\pgfpathcurveto{\pgfqpoint{2.147249in}{2.032303in}}{\pgfqpoint{2.150521in}{2.040203in}}{\pgfqpoint{2.150521in}{2.048439in}}%
\pgfpathcurveto{\pgfqpoint{2.150521in}{2.056675in}}{\pgfqpoint{2.147249in}{2.064575in}}{\pgfqpoint{2.141425in}{2.070399in}}%
\pgfpathcurveto{\pgfqpoint{2.135601in}{2.076223in}}{\pgfqpoint{2.127701in}{2.079496in}}{\pgfqpoint{2.119465in}{2.079496in}}%
\pgfpathcurveto{\pgfqpoint{2.111229in}{2.079496in}}{\pgfqpoint{2.103329in}{2.076223in}}{\pgfqpoint{2.097505in}{2.070399in}}%
\pgfpathcurveto{\pgfqpoint{2.091681in}{2.064575in}}{\pgfqpoint{2.088408in}{2.056675in}}{\pgfqpoint{2.088408in}{2.048439in}}%
\pgfpathcurveto{\pgfqpoint{2.088408in}{2.040203in}}{\pgfqpoint{2.091681in}{2.032303in}}{\pgfqpoint{2.097505in}{2.026479in}}%
\pgfpathcurveto{\pgfqpoint{2.103329in}{2.020655in}}{\pgfqpoint{2.111229in}{2.017383in}}{\pgfqpoint{2.119465in}{2.017383in}}%
\pgfpathclose%
\pgfusepath{stroke,fill}%
\end{pgfscope}%
\begin{pgfscope}%
\pgfpathrectangle{\pgfqpoint{0.100000in}{0.212622in}}{\pgfqpoint{3.696000in}{3.696000in}}%
\pgfusepath{clip}%
\pgfsetbuttcap%
\pgfsetroundjoin%
\definecolor{currentfill}{rgb}{0.121569,0.466667,0.705882}%
\pgfsetfillcolor{currentfill}%
\pgfsetfillopacity{0.624911}%
\pgfsetlinewidth{1.003750pt}%
\definecolor{currentstroke}{rgb}{0.121569,0.466667,0.705882}%
\pgfsetstrokecolor{currentstroke}%
\pgfsetstrokeopacity{0.624911}%
\pgfsetdash{}{0pt}%
\pgfpathmoveto{\pgfqpoint{0.859771in}{1.491931in}}%
\pgfpathcurveto{\pgfqpoint{0.868007in}{1.491931in}}{\pgfqpoint{0.875907in}{1.495204in}}{\pgfqpoint{0.881731in}{1.501028in}}%
\pgfpathcurveto{\pgfqpoint{0.887555in}{1.506851in}}{\pgfqpoint{0.890828in}{1.514752in}}{\pgfqpoint{0.890828in}{1.522988in}}%
\pgfpathcurveto{\pgfqpoint{0.890828in}{1.531224in}}{\pgfqpoint{0.887555in}{1.539124in}}{\pgfqpoint{0.881731in}{1.544948in}}%
\pgfpathcurveto{\pgfqpoint{0.875907in}{1.550772in}}{\pgfqpoint{0.868007in}{1.554044in}}{\pgfqpoint{0.859771in}{1.554044in}}%
\pgfpathcurveto{\pgfqpoint{0.851535in}{1.554044in}}{\pgfqpoint{0.843635in}{1.550772in}}{\pgfqpoint{0.837811in}{1.544948in}}%
\pgfpathcurveto{\pgfqpoint{0.831987in}{1.539124in}}{\pgfqpoint{0.828715in}{1.531224in}}{\pgfqpoint{0.828715in}{1.522988in}}%
\pgfpathcurveto{\pgfqpoint{0.828715in}{1.514752in}}{\pgfqpoint{0.831987in}{1.506851in}}{\pgfqpoint{0.837811in}{1.501028in}}%
\pgfpathcurveto{\pgfqpoint{0.843635in}{1.495204in}}{\pgfqpoint{0.851535in}{1.491931in}}{\pgfqpoint{0.859771in}{1.491931in}}%
\pgfpathclose%
\pgfusepath{stroke,fill}%
\end{pgfscope}%
\begin{pgfscope}%
\pgfpathrectangle{\pgfqpoint{0.100000in}{0.212622in}}{\pgfqpoint{3.696000in}{3.696000in}}%
\pgfusepath{clip}%
\pgfsetbuttcap%
\pgfsetroundjoin%
\definecolor{currentfill}{rgb}{0.121569,0.466667,0.705882}%
\pgfsetfillcolor{currentfill}%
\pgfsetfillopacity{0.625256}%
\pgfsetlinewidth{1.003750pt}%
\definecolor{currentstroke}{rgb}{0.121569,0.466667,0.705882}%
\pgfsetstrokecolor{currentstroke}%
\pgfsetstrokeopacity{0.625256}%
\pgfsetdash{}{0pt}%
\pgfpathmoveto{\pgfqpoint{0.737458in}{1.204942in}}%
\pgfpathcurveto{\pgfqpoint{0.745694in}{1.204942in}}{\pgfqpoint{0.753594in}{1.208214in}}{\pgfqpoint{0.759418in}{1.214038in}}%
\pgfpathcurveto{\pgfqpoint{0.765242in}{1.219862in}}{\pgfqpoint{0.768515in}{1.227762in}}{\pgfqpoint{0.768515in}{1.235999in}}%
\pgfpathcurveto{\pgfqpoint{0.768515in}{1.244235in}}{\pgfqpoint{0.765242in}{1.252135in}}{\pgfqpoint{0.759418in}{1.257959in}}%
\pgfpathcurveto{\pgfqpoint{0.753594in}{1.263783in}}{\pgfqpoint{0.745694in}{1.267055in}}{\pgfqpoint{0.737458in}{1.267055in}}%
\pgfpathcurveto{\pgfqpoint{0.729222in}{1.267055in}}{\pgfqpoint{0.721322in}{1.263783in}}{\pgfqpoint{0.715498in}{1.257959in}}%
\pgfpathcurveto{\pgfqpoint{0.709674in}{1.252135in}}{\pgfqpoint{0.706402in}{1.244235in}}{\pgfqpoint{0.706402in}{1.235999in}}%
\pgfpathcurveto{\pgfqpoint{0.706402in}{1.227762in}}{\pgfqpoint{0.709674in}{1.219862in}}{\pgfqpoint{0.715498in}{1.214038in}}%
\pgfpathcurveto{\pgfqpoint{0.721322in}{1.208214in}}{\pgfqpoint{0.729222in}{1.204942in}}{\pgfqpoint{0.737458in}{1.204942in}}%
\pgfpathclose%
\pgfusepath{stroke,fill}%
\end{pgfscope}%
\begin{pgfscope}%
\pgfpathrectangle{\pgfqpoint{0.100000in}{0.212622in}}{\pgfqpoint{3.696000in}{3.696000in}}%
\pgfusepath{clip}%
\pgfsetbuttcap%
\pgfsetroundjoin%
\definecolor{currentfill}{rgb}{0.121569,0.466667,0.705882}%
\pgfsetfillcolor{currentfill}%
\pgfsetfillopacity{0.625290}%
\pgfsetlinewidth{1.003750pt}%
\definecolor{currentstroke}{rgb}{0.121569,0.466667,0.705882}%
\pgfsetstrokecolor{currentstroke}%
\pgfsetstrokeopacity{0.625290}%
\pgfsetdash{}{0pt}%
\pgfpathmoveto{\pgfqpoint{0.859701in}{1.492000in}}%
\pgfpathcurveto{\pgfqpoint{0.867937in}{1.492000in}}{\pgfqpoint{0.875837in}{1.495273in}}{\pgfqpoint{0.881661in}{1.501097in}}%
\pgfpathcurveto{\pgfqpoint{0.887485in}{1.506921in}}{\pgfqpoint{0.890757in}{1.514821in}}{\pgfqpoint{0.890757in}{1.523057in}}%
\pgfpathcurveto{\pgfqpoint{0.890757in}{1.531293in}}{\pgfqpoint{0.887485in}{1.539193in}}{\pgfqpoint{0.881661in}{1.545017in}}%
\pgfpathcurveto{\pgfqpoint{0.875837in}{1.550841in}}{\pgfqpoint{0.867937in}{1.554113in}}{\pgfqpoint{0.859701in}{1.554113in}}%
\pgfpathcurveto{\pgfqpoint{0.851464in}{1.554113in}}{\pgfqpoint{0.843564in}{1.550841in}}{\pgfqpoint{0.837740in}{1.545017in}}%
\pgfpathcurveto{\pgfqpoint{0.831916in}{1.539193in}}{\pgfqpoint{0.828644in}{1.531293in}}{\pgfqpoint{0.828644in}{1.523057in}}%
\pgfpathcurveto{\pgfqpoint{0.828644in}{1.514821in}}{\pgfqpoint{0.831916in}{1.506921in}}{\pgfqpoint{0.837740in}{1.501097in}}%
\pgfpathcurveto{\pgfqpoint{0.843564in}{1.495273in}}{\pgfqpoint{0.851464in}{1.492000in}}{\pgfqpoint{0.859701in}{1.492000in}}%
\pgfpathclose%
\pgfusepath{stroke,fill}%
\end{pgfscope}%
\begin{pgfscope}%
\pgfpathrectangle{\pgfqpoint{0.100000in}{0.212622in}}{\pgfqpoint{3.696000in}{3.696000in}}%
\pgfusepath{clip}%
\pgfsetbuttcap%
\pgfsetroundjoin%
\definecolor{currentfill}{rgb}{0.121569,0.466667,0.705882}%
\pgfsetfillcolor{currentfill}%
\pgfsetfillopacity{0.625601}%
\pgfsetlinewidth{1.003750pt}%
\definecolor{currentstroke}{rgb}{0.121569,0.466667,0.705882}%
\pgfsetstrokecolor{currentstroke}%
\pgfsetstrokeopacity{0.625601}%
\pgfsetdash{}{0pt}%
\pgfpathmoveto{\pgfqpoint{0.869649in}{1.511870in}}%
\pgfpathcurveto{\pgfqpoint{0.877885in}{1.511870in}}{\pgfqpoint{0.885785in}{1.515142in}}{\pgfqpoint{0.891609in}{1.520966in}}%
\pgfpathcurveto{\pgfqpoint{0.897433in}{1.526790in}}{\pgfqpoint{0.900706in}{1.534690in}}{\pgfqpoint{0.900706in}{1.542926in}}%
\pgfpathcurveto{\pgfqpoint{0.900706in}{1.551163in}}{\pgfqpoint{0.897433in}{1.559063in}}{\pgfqpoint{0.891609in}{1.564887in}}%
\pgfpathcurveto{\pgfqpoint{0.885785in}{1.570711in}}{\pgfqpoint{0.877885in}{1.573983in}}{\pgfqpoint{0.869649in}{1.573983in}}%
\pgfpathcurveto{\pgfqpoint{0.861413in}{1.573983in}}{\pgfqpoint{0.853513in}{1.570711in}}{\pgfqpoint{0.847689in}{1.564887in}}%
\pgfpathcurveto{\pgfqpoint{0.841865in}{1.559063in}}{\pgfqpoint{0.838593in}{1.551163in}}{\pgfqpoint{0.838593in}{1.542926in}}%
\pgfpathcurveto{\pgfqpoint{0.838593in}{1.534690in}}{\pgfqpoint{0.841865in}{1.526790in}}{\pgfqpoint{0.847689in}{1.520966in}}%
\pgfpathcurveto{\pgfqpoint{0.853513in}{1.515142in}}{\pgfqpoint{0.861413in}{1.511870in}}{\pgfqpoint{0.869649in}{1.511870in}}%
\pgfpathclose%
\pgfusepath{stroke,fill}%
\end{pgfscope}%
\begin{pgfscope}%
\pgfpathrectangle{\pgfqpoint{0.100000in}{0.212622in}}{\pgfqpoint{3.696000in}{3.696000in}}%
\pgfusepath{clip}%
\pgfsetbuttcap%
\pgfsetroundjoin%
\definecolor{currentfill}{rgb}{0.121569,0.466667,0.705882}%
\pgfsetfillcolor{currentfill}%
\pgfsetfillopacity{0.625912}%
\pgfsetlinewidth{1.003750pt}%
\definecolor{currentstroke}{rgb}{0.121569,0.466667,0.705882}%
\pgfsetstrokecolor{currentstroke}%
\pgfsetstrokeopacity{0.625912}%
\pgfsetdash{}{0pt}%
\pgfpathmoveto{\pgfqpoint{0.859654in}{1.492123in}}%
\pgfpathcurveto{\pgfqpoint{0.867890in}{1.492123in}}{\pgfqpoint{0.875790in}{1.495395in}}{\pgfqpoint{0.881614in}{1.501219in}}%
\pgfpathcurveto{\pgfqpoint{0.887438in}{1.507043in}}{\pgfqpoint{0.890711in}{1.514943in}}{\pgfqpoint{0.890711in}{1.523180in}}%
\pgfpathcurveto{\pgfqpoint{0.890711in}{1.531416in}}{\pgfqpoint{0.887438in}{1.539316in}}{\pgfqpoint{0.881614in}{1.545140in}}%
\pgfpathcurveto{\pgfqpoint{0.875790in}{1.550964in}}{\pgfqpoint{0.867890in}{1.554236in}}{\pgfqpoint{0.859654in}{1.554236in}}%
\pgfpathcurveto{\pgfqpoint{0.851418in}{1.554236in}}{\pgfqpoint{0.843518in}{1.550964in}}{\pgfqpoint{0.837694in}{1.545140in}}%
\pgfpathcurveto{\pgfqpoint{0.831870in}{1.539316in}}{\pgfqpoint{0.828598in}{1.531416in}}{\pgfqpoint{0.828598in}{1.523180in}}%
\pgfpathcurveto{\pgfqpoint{0.828598in}{1.514943in}}{\pgfqpoint{0.831870in}{1.507043in}}{\pgfqpoint{0.837694in}{1.501219in}}%
\pgfpathcurveto{\pgfqpoint{0.843518in}{1.495395in}}{\pgfqpoint{0.851418in}{1.492123in}}{\pgfqpoint{0.859654in}{1.492123in}}%
\pgfpathclose%
\pgfusepath{stroke,fill}%
\end{pgfscope}%
\begin{pgfscope}%
\pgfpathrectangle{\pgfqpoint{0.100000in}{0.212622in}}{\pgfqpoint{3.696000in}{3.696000in}}%
\pgfusepath{clip}%
\pgfsetbuttcap%
\pgfsetroundjoin%
\definecolor{currentfill}{rgb}{0.121569,0.466667,0.705882}%
\pgfsetfillcolor{currentfill}%
\pgfsetfillopacity{0.625923}%
\pgfsetlinewidth{1.003750pt}%
\definecolor{currentstroke}{rgb}{0.121569,0.466667,0.705882}%
\pgfsetstrokecolor{currentstroke}%
\pgfsetstrokeopacity{0.625923}%
\pgfsetdash{}{0pt}%
\pgfpathmoveto{\pgfqpoint{2.120837in}{2.011986in}}%
\pgfpathcurveto{\pgfqpoint{2.129073in}{2.011986in}}{\pgfqpoint{2.136973in}{2.015259in}}{\pgfqpoint{2.142797in}{2.021083in}}%
\pgfpathcurveto{\pgfqpoint{2.148621in}{2.026906in}}{\pgfqpoint{2.151894in}{2.034807in}}{\pgfqpoint{2.151894in}{2.043043in}}%
\pgfpathcurveto{\pgfqpoint{2.151894in}{2.051279in}}{\pgfqpoint{2.148621in}{2.059179in}}{\pgfqpoint{2.142797in}{2.065003in}}%
\pgfpathcurveto{\pgfqpoint{2.136973in}{2.070827in}}{\pgfqpoint{2.129073in}{2.074099in}}{\pgfqpoint{2.120837in}{2.074099in}}%
\pgfpathcurveto{\pgfqpoint{2.112601in}{2.074099in}}{\pgfqpoint{2.104701in}{2.070827in}}{\pgfqpoint{2.098877in}{2.065003in}}%
\pgfpathcurveto{\pgfqpoint{2.093053in}{2.059179in}}{\pgfqpoint{2.089781in}{2.051279in}}{\pgfqpoint{2.089781in}{2.043043in}}%
\pgfpathcurveto{\pgfqpoint{2.089781in}{2.034807in}}{\pgfqpoint{2.093053in}{2.026906in}}{\pgfqpoint{2.098877in}{2.021083in}}%
\pgfpathcurveto{\pgfqpoint{2.104701in}{2.015259in}}{\pgfqpoint{2.112601in}{2.011986in}}{\pgfqpoint{2.120837in}{2.011986in}}%
\pgfpathclose%
\pgfusepath{stroke,fill}%
\end{pgfscope}%
\begin{pgfscope}%
\pgfpathrectangle{\pgfqpoint{0.100000in}{0.212622in}}{\pgfqpoint{3.696000in}{3.696000in}}%
\pgfusepath{clip}%
\pgfsetbuttcap%
\pgfsetroundjoin%
\definecolor{currentfill}{rgb}{0.121569,0.466667,0.705882}%
\pgfsetfillcolor{currentfill}%
\pgfsetfillopacity{0.626252}%
\pgfsetlinewidth{1.003750pt}%
\definecolor{currentstroke}{rgb}{0.121569,0.466667,0.705882}%
\pgfsetstrokecolor{currentstroke}%
\pgfsetstrokeopacity{0.626252}%
\pgfsetdash{}{0pt}%
\pgfpathmoveto{\pgfqpoint{0.859660in}{1.492204in}}%
\pgfpathcurveto{\pgfqpoint{0.867897in}{1.492204in}}{\pgfqpoint{0.875797in}{1.495477in}}{\pgfqpoint{0.881621in}{1.501301in}}%
\pgfpathcurveto{\pgfqpoint{0.887445in}{1.507125in}}{\pgfqpoint{0.890717in}{1.515025in}}{\pgfqpoint{0.890717in}{1.523261in}}%
\pgfpathcurveto{\pgfqpoint{0.890717in}{1.531497in}}{\pgfqpoint{0.887445in}{1.539397in}}{\pgfqpoint{0.881621in}{1.545221in}}%
\pgfpathcurveto{\pgfqpoint{0.875797in}{1.551045in}}{\pgfqpoint{0.867897in}{1.554317in}}{\pgfqpoint{0.859660in}{1.554317in}}%
\pgfpathcurveto{\pgfqpoint{0.851424in}{1.554317in}}{\pgfqpoint{0.843524in}{1.551045in}}{\pgfqpoint{0.837700in}{1.545221in}}%
\pgfpathcurveto{\pgfqpoint{0.831876in}{1.539397in}}{\pgfqpoint{0.828604in}{1.531497in}}{\pgfqpoint{0.828604in}{1.523261in}}%
\pgfpathcurveto{\pgfqpoint{0.828604in}{1.515025in}}{\pgfqpoint{0.831876in}{1.507125in}}{\pgfqpoint{0.837700in}{1.501301in}}%
\pgfpathcurveto{\pgfqpoint{0.843524in}{1.495477in}}{\pgfqpoint{0.851424in}{1.492204in}}{\pgfqpoint{0.859660in}{1.492204in}}%
\pgfpathclose%
\pgfusepath{stroke,fill}%
\end{pgfscope}%
\begin{pgfscope}%
\pgfpathrectangle{\pgfqpoint{0.100000in}{0.212622in}}{\pgfqpoint{3.696000in}{3.696000in}}%
\pgfusepath{clip}%
\pgfsetbuttcap%
\pgfsetroundjoin%
\definecolor{currentfill}{rgb}{0.121569,0.466667,0.705882}%
\pgfsetfillcolor{currentfill}%
\pgfsetfillopacity{0.626443}%
\pgfsetlinewidth{1.003750pt}%
\definecolor{currentstroke}{rgb}{0.121569,0.466667,0.705882}%
\pgfsetstrokecolor{currentstroke}%
\pgfsetstrokeopacity{0.626443}%
\pgfsetdash{}{0pt}%
\pgfpathmoveto{\pgfqpoint{0.859680in}{1.492275in}}%
\pgfpathcurveto{\pgfqpoint{0.867916in}{1.492275in}}{\pgfqpoint{0.875816in}{1.495547in}}{\pgfqpoint{0.881640in}{1.501371in}}%
\pgfpathcurveto{\pgfqpoint{0.887464in}{1.507195in}}{\pgfqpoint{0.890736in}{1.515095in}}{\pgfqpoint{0.890736in}{1.523331in}}%
\pgfpathcurveto{\pgfqpoint{0.890736in}{1.531567in}}{\pgfqpoint{0.887464in}{1.539468in}}{\pgfqpoint{0.881640in}{1.545291in}}%
\pgfpathcurveto{\pgfqpoint{0.875816in}{1.551115in}}{\pgfqpoint{0.867916in}{1.554388in}}{\pgfqpoint{0.859680in}{1.554388in}}%
\pgfpathcurveto{\pgfqpoint{0.851444in}{1.554388in}}{\pgfqpoint{0.843544in}{1.551115in}}{\pgfqpoint{0.837720in}{1.545291in}}%
\pgfpathcurveto{\pgfqpoint{0.831896in}{1.539468in}}{\pgfqpoint{0.828623in}{1.531567in}}{\pgfqpoint{0.828623in}{1.523331in}}%
\pgfpathcurveto{\pgfqpoint{0.828623in}{1.515095in}}{\pgfqpoint{0.831896in}{1.507195in}}{\pgfqpoint{0.837720in}{1.501371in}}%
\pgfpathcurveto{\pgfqpoint{0.843544in}{1.495547in}}{\pgfqpoint{0.851444in}{1.492275in}}{\pgfqpoint{0.859680in}{1.492275in}}%
\pgfpathclose%
\pgfusepath{stroke,fill}%
\end{pgfscope}%
\begin{pgfscope}%
\pgfpathrectangle{\pgfqpoint{0.100000in}{0.212622in}}{\pgfqpoint{3.696000in}{3.696000in}}%
\pgfusepath{clip}%
\pgfsetbuttcap%
\pgfsetroundjoin%
\definecolor{currentfill}{rgb}{0.121569,0.466667,0.705882}%
\pgfsetfillcolor{currentfill}%
\pgfsetfillopacity{0.626494}%
\pgfsetlinewidth{1.003750pt}%
\definecolor{currentstroke}{rgb}{0.121569,0.466667,0.705882}%
\pgfsetstrokecolor{currentstroke}%
\pgfsetstrokeopacity{0.626494}%
\pgfsetdash{}{0pt}%
\pgfpathmoveto{\pgfqpoint{0.864918in}{1.502680in}}%
\pgfpathcurveto{\pgfqpoint{0.873155in}{1.502680in}}{\pgfqpoint{0.881055in}{1.505952in}}{\pgfqpoint{0.886879in}{1.511776in}}%
\pgfpathcurveto{\pgfqpoint{0.892703in}{1.517600in}}{\pgfqpoint{0.895975in}{1.525500in}}{\pgfqpoint{0.895975in}{1.533737in}}%
\pgfpathcurveto{\pgfqpoint{0.895975in}{1.541973in}}{\pgfqpoint{0.892703in}{1.549873in}}{\pgfqpoint{0.886879in}{1.555697in}}%
\pgfpathcurveto{\pgfqpoint{0.881055in}{1.561521in}}{\pgfqpoint{0.873155in}{1.564793in}}{\pgfqpoint{0.864918in}{1.564793in}}%
\pgfpathcurveto{\pgfqpoint{0.856682in}{1.564793in}}{\pgfqpoint{0.848782in}{1.561521in}}{\pgfqpoint{0.842958in}{1.555697in}}%
\pgfpathcurveto{\pgfqpoint{0.837134in}{1.549873in}}{\pgfqpoint{0.833862in}{1.541973in}}{\pgfqpoint{0.833862in}{1.533737in}}%
\pgfpathcurveto{\pgfqpoint{0.833862in}{1.525500in}}{\pgfqpoint{0.837134in}{1.517600in}}{\pgfqpoint{0.842958in}{1.511776in}}%
\pgfpathcurveto{\pgfqpoint{0.848782in}{1.505952in}}{\pgfqpoint{0.856682in}{1.502680in}}{\pgfqpoint{0.864918in}{1.502680in}}%
\pgfpathclose%
\pgfusepath{stroke,fill}%
\end{pgfscope}%
\begin{pgfscope}%
\pgfpathrectangle{\pgfqpoint{0.100000in}{0.212622in}}{\pgfqpoint{3.696000in}{3.696000in}}%
\pgfusepath{clip}%
\pgfsetbuttcap%
\pgfsetroundjoin%
\definecolor{currentfill}{rgb}{0.121569,0.466667,0.705882}%
\pgfsetfillcolor{currentfill}%
\pgfsetfillopacity{0.626546}%
\pgfsetlinewidth{1.003750pt}%
\definecolor{currentstroke}{rgb}{0.121569,0.466667,0.705882}%
\pgfsetstrokecolor{currentstroke}%
\pgfsetstrokeopacity{0.626546}%
\pgfsetdash{}{0pt}%
\pgfpathmoveto{\pgfqpoint{0.859702in}{1.492317in}}%
\pgfpathcurveto{\pgfqpoint{0.867938in}{1.492317in}}{\pgfqpoint{0.875838in}{1.495589in}}{\pgfqpoint{0.881662in}{1.501413in}}%
\pgfpathcurveto{\pgfqpoint{0.887486in}{1.507237in}}{\pgfqpoint{0.890758in}{1.515137in}}{\pgfqpoint{0.890758in}{1.523373in}}%
\pgfpathcurveto{\pgfqpoint{0.890758in}{1.531610in}}{\pgfqpoint{0.887486in}{1.539510in}}{\pgfqpoint{0.881662in}{1.545334in}}%
\pgfpathcurveto{\pgfqpoint{0.875838in}{1.551158in}}{\pgfqpoint{0.867938in}{1.554430in}}{\pgfqpoint{0.859702in}{1.554430in}}%
\pgfpathcurveto{\pgfqpoint{0.851465in}{1.554430in}}{\pgfqpoint{0.843565in}{1.551158in}}{\pgfqpoint{0.837741in}{1.545334in}}%
\pgfpathcurveto{\pgfqpoint{0.831917in}{1.539510in}}{\pgfqpoint{0.828645in}{1.531610in}}{\pgfqpoint{0.828645in}{1.523373in}}%
\pgfpathcurveto{\pgfqpoint{0.828645in}{1.515137in}}{\pgfqpoint{0.831917in}{1.507237in}}{\pgfqpoint{0.837741in}{1.501413in}}%
\pgfpathcurveto{\pgfqpoint{0.843565in}{1.495589in}}{\pgfqpoint{0.851465in}{1.492317in}}{\pgfqpoint{0.859702in}{1.492317in}}%
\pgfpathclose%
\pgfusepath{stroke,fill}%
\end{pgfscope}%
\begin{pgfscope}%
\pgfpathrectangle{\pgfqpoint{0.100000in}{0.212622in}}{\pgfqpoint{3.696000in}{3.696000in}}%
\pgfusepath{clip}%
\pgfsetbuttcap%
\pgfsetroundjoin%
\definecolor{currentfill}{rgb}{0.121569,0.466667,0.705882}%
\pgfsetfillcolor{currentfill}%
\pgfsetfillopacity{0.626601}%
\pgfsetlinewidth{1.003750pt}%
\definecolor{currentstroke}{rgb}{0.121569,0.466667,0.705882}%
\pgfsetstrokecolor{currentstroke}%
\pgfsetstrokeopacity{0.626601}%
\pgfsetdash{}{0pt}%
\pgfpathmoveto{\pgfqpoint{0.859721in}{1.492341in}}%
\pgfpathcurveto{\pgfqpoint{0.867957in}{1.492341in}}{\pgfqpoint{0.875857in}{1.495614in}}{\pgfqpoint{0.881681in}{1.501437in}}%
\pgfpathcurveto{\pgfqpoint{0.887505in}{1.507261in}}{\pgfqpoint{0.890777in}{1.515161in}}{\pgfqpoint{0.890777in}{1.523398in}}%
\pgfpathcurveto{\pgfqpoint{0.890777in}{1.531634in}}{\pgfqpoint{0.887505in}{1.539534in}}{\pgfqpoint{0.881681in}{1.545358in}}%
\pgfpathcurveto{\pgfqpoint{0.875857in}{1.551182in}}{\pgfqpoint{0.867957in}{1.554454in}}{\pgfqpoint{0.859721in}{1.554454in}}%
\pgfpathcurveto{\pgfqpoint{0.851484in}{1.554454in}}{\pgfqpoint{0.843584in}{1.551182in}}{\pgfqpoint{0.837760in}{1.545358in}}%
\pgfpathcurveto{\pgfqpoint{0.831936in}{1.539534in}}{\pgfqpoint{0.828664in}{1.531634in}}{\pgfqpoint{0.828664in}{1.523398in}}%
\pgfpathcurveto{\pgfqpoint{0.828664in}{1.515161in}}{\pgfqpoint{0.831936in}{1.507261in}}{\pgfqpoint{0.837760in}{1.501437in}}%
\pgfpathcurveto{\pgfqpoint{0.843584in}{1.495614in}}{\pgfqpoint{0.851484in}{1.492341in}}{\pgfqpoint{0.859721in}{1.492341in}}%
\pgfpathclose%
\pgfusepath{stroke,fill}%
\end{pgfscope}%
\begin{pgfscope}%
\pgfpathrectangle{\pgfqpoint{0.100000in}{0.212622in}}{\pgfqpoint{3.696000in}{3.696000in}}%
\pgfusepath{clip}%
\pgfsetbuttcap%
\pgfsetroundjoin%
\definecolor{currentfill}{rgb}{0.121569,0.466667,0.705882}%
\pgfsetfillcolor{currentfill}%
\pgfsetfillopacity{0.626821}%
\pgfsetlinewidth{1.003750pt}%
\definecolor{currentstroke}{rgb}{0.121569,0.466667,0.705882}%
\pgfsetstrokecolor{currentstroke}%
\pgfsetstrokeopacity{0.626821}%
\pgfsetdash{}{0pt}%
\pgfpathmoveto{\pgfqpoint{0.859819in}{1.492465in}}%
\pgfpathcurveto{\pgfqpoint{0.868055in}{1.492465in}}{\pgfqpoint{0.875955in}{1.495737in}}{\pgfqpoint{0.881779in}{1.501561in}}%
\pgfpathcurveto{\pgfqpoint{0.887603in}{1.507385in}}{\pgfqpoint{0.890875in}{1.515285in}}{\pgfqpoint{0.890875in}{1.523521in}}%
\pgfpathcurveto{\pgfqpoint{0.890875in}{1.531757in}}{\pgfqpoint{0.887603in}{1.539657in}}{\pgfqpoint{0.881779in}{1.545481in}}%
\pgfpathcurveto{\pgfqpoint{0.875955in}{1.551305in}}{\pgfqpoint{0.868055in}{1.554578in}}{\pgfqpoint{0.859819in}{1.554578in}}%
\pgfpathcurveto{\pgfqpoint{0.851582in}{1.554578in}}{\pgfqpoint{0.843682in}{1.551305in}}{\pgfqpoint{0.837858in}{1.545481in}}%
\pgfpathcurveto{\pgfqpoint{0.832034in}{1.539657in}}{\pgfqpoint{0.828762in}{1.531757in}}{\pgfqpoint{0.828762in}{1.523521in}}%
\pgfpathcurveto{\pgfqpoint{0.828762in}{1.515285in}}{\pgfqpoint{0.832034in}{1.507385in}}{\pgfqpoint{0.837858in}{1.501561in}}%
\pgfpathcurveto{\pgfqpoint{0.843682in}{1.495737in}}{\pgfqpoint{0.851582in}{1.492465in}}{\pgfqpoint{0.859819in}{1.492465in}}%
\pgfpathclose%
\pgfusepath{stroke,fill}%
\end{pgfscope}%
\begin{pgfscope}%
\pgfpathrectangle{\pgfqpoint{0.100000in}{0.212622in}}{\pgfqpoint{3.696000in}{3.696000in}}%
\pgfusepath{clip}%
\pgfsetbuttcap%
\pgfsetroundjoin%
\definecolor{currentfill}{rgb}{0.121569,0.466667,0.705882}%
\pgfsetfillcolor{currentfill}%
\pgfsetfillopacity{0.626942}%
\pgfsetlinewidth{1.003750pt}%
\definecolor{currentstroke}{rgb}{0.121569,0.466667,0.705882}%
\pgfsetstrokecolor{currentstroke}%
\pgfsetstrokeopacity{0.626942}%
\pgfsetdash{}{0pt}%
\pgfpathmoveto{\pgfqpoint{0.859883in}{1.492547in}}%
\pgfpathcurveto{\pgfqpoint{0.868119in}{1.492547in}}{\pgfqpoint{0.876019in}{1.495820in}}{\pgfqpoint{0.881843in}{1.501644in}}%
\pgfpathcurveto{\pgfqpoint{0.887667in}{1.507468in}}{\pgfqpoint{0.890939in}{1.515368in}}{\pgfqpoint{0.890939in}{1.523604in}}%
\pgfpathcurveto{\pgfqpoint{0.890939in}{1.531840in}}{\pgfqpoint{0.887667in}{1.539740in}}{\pgfqpoint{0.881843in}{1.545564in}}%
\pgfpathcurveto{\pgfqpoint{0.876019in}{1.551388in}}{\pgfqpoint{0.868119in}{1.554660in}}{\pgfqpoint{0.859883in}{1.554660in}}%
\pgfpathcurveto{\pgfqpoint{0.851647in}{1.554660in}}{\pgfqpoint{0.843747in}{1.551388in}}{\pgfqpoint{0.837923in}{1.545564in}}%
\pgfpathcurveto{\pgfqpoint{0.832099in}{1.539740in}}{\pgfqpoint{0.828826in}{1.531840in}}{\pgfqpoint{0.828826in}{1.523604in}}%
\pgfpathcurveto{\pgfqpoint{0.828826in}{1.515368in}}{\pgfqpoint{0.832099in}{1.507468in}}{\pgfqpoint{0.837923in}{1.501644in}}%
\pgfpathcurveto{\pgfqpoint{0.843747in}{1.495820in}}{\pgfqpoint{0.851647in}{1.492547in}}{\pgfqpoint{0.859883in}{1.492547in}}%
\pgfpathclose%
\pgfusepath{stroke,fill}%
\end{pgfscope}%
\begin{pgfscope}%
\pgfpathrectangle{\pgfqpoint{0.100000in}{0.212622in}}{\pgfqpoint{3.696000in}{3.696000in}}%
\pgfusepath{clip}%
\pgfsetbuttcap%
\pgfsetroundjoin%
\definecolor{currentfill}{rgb}{0.121569,0.466667,0.705882}%
\pgfsetfillcolor{currentfill}%
\pgfsetfillopacity{0.627009}%
\pgfsetlinewidth{1.003750pt}%
\definecolor{currentstroke}{rgb}{0.121569,0.466667,0.705882}%
\pgfsetstrokecolor{currentstroke}%
\pgfsetstrokeopacity{0.627009}%
\pgfsetdash{}{0pt}%
\pgfpathmoveto{\pgfqpoint{0.859925in}{1.492605in}}%
\pgfpathcurveto{\pgfqpoint{0.868161in}{1.492605in}}{\pgfqpoint{0.876061in}{1.495877in}}{\pgfqpoint{0.881885in}{1.501701in}}%
\pgfpathcurveto{\pgfqpoint{0.887709in}{1.507525in}}{\pgfqpoint{0.890981in}{1.515425in}}{\pgfqpoint{0.890981in}{1.523662in}}%
\pgfpathcurveto{\pgfqpoint{0.890981in}{1.531898in}}{\pgfqpoint{0.887709in}{1.539798in}}{\pgfqpoint{0.881885in}{1.545622in}}%
\pgfpathcurveto{\pgfqpoint{0.876061in}{1.551446in}}{\pgfqpoint{0.868161in}{1.554718in}}{\pgfqpoint{0.859925in}{1.554718in}}%
\pgfpathcurveto{\pgfqpoint{0.851688in}{1.554718in}}{\pgfqpoint{0.843788in}{1.551446in}}{\pgfqpoint{0.837965in}{1.545622in}}%
\pgfpathcurveto{\pgfqpoint{0.832141in}{1.539798in}}{\pgfqpoint{0.828868in}{1.531898in}}{\pgfqpoint{0.828868in}{1.523662in}}%
\pgfpathcurveto{\pgfqpoint{0.828868in}{1.515425in}}{\pgfqpoint{0.832141in}{1.507525in}}{\pgfqpoint{0.837965in}{1.501701in}}%
\pgfpathcurveto{\pgfqpoint{0.843788in}{1.495877in}}{\pgfqpoint{0.851688in}{1.492605in}}{\pgfqpoint{0.859925in}{1.492605in}}%
\pgfpathclose%
\pgfusepath{stroke,fill}%
\end{pgfscope}%
\begin{pgfscope}%
\pgfpathrectangle{\pgfqpoint{0.100000in}{0.212622in}}{\pgfqpoint{3.696000in}{3.696000in}}%
\pgfusepath{clip}%
\pgfsetbuttcap%
\pgfsetroundjoin%
\definecolor{currentfill}{rgb}{0.121569,0.466667,0.705882}%
\pgfsetfillcolor{currentfill}%
\pgfsetfillopacity{0.627045}%
\pgfsetlinewidth{1.003750pt}%
\definecolor{currentstroke}{rgb}{0.121569,0.466667,0.705882}%
\pgfsetstrokecolor{currentstroke}%
\pgfsetstrokeopacity{0.627045}%
\pgfsetdash{}{0pt}%
\pgfpathmoveto{\pgfqpoint{0.859952in}{1.492641in}}%
\pgfpathcurveto{\pgfqpoint{0.868189in}{1.492641in}}{\pgfqpoint{0.876089in}{1.495914in}}{\pgfqpoint{0.881913in}{1.501738in}}%
\pgfpathcurveto{\pgfqpoint{0.887736in}{1.507562in}}{\pgfqpoint{0.891009in}{1.515462in}}{\pgfqpoint{0.891009in}{1.523698in}}%
\pgfpathcurveto{\pgfqpoint{0.891009in}{1.531934in}}{\pgfqpoint{0.887736in}{1.539834in}}{\pgfqpoint{0.881913in}{1.545658in}}%
\pgfpathcurveto{\pgfqpoint{0.876089in}{1.551482in}}{\pgfqpoint{0.868189in}{1.554754in}}{\pgfqpoint{0.859952in}{1.554754in}}%
\pgfpathcurveto{\pgfqpoint{0.851716in}{1.554754in}}{\pgfqpoint{0.843816in}{1.551482in}}{\pgfqpoint{0.837992in}{1.545658in}}%
\pgfpathcurveto{\pgfqpoint{0.832168in}{1.539834in}}{\pgfqpoint{0.828896in}{1.531934in}}{\pgfqpoint{0.828896in}{1.523698in}}%
\pgfpathcurveto{\pgfqpoint{0.828896in}{1.515462in}}{\pgfqpoint{0.832168in}{1.507562in}}{\pgfqpoint{0.837992in}{1.501738in}}%
\pgfpathcurveto{\pgfqpoint{0.843816in}{1.495914in}}{\pgfqpoint{0.851716in}{1.492641in}}{\pgfqpoint{0.859952in}{1.492641in}}%
\pgfpathclose%
\pgfusepath{stroke,fill}%
\end{pgfscope}%
\begin{pgfscope}%
\pgfpathrectangle{\pgfqpoint{0.100000in}{0.212622in}}{\pgfqpoint{3.696000in}{3.696000in}}%
\pgfusepath{clip}%
\pgfsetbuttcap%
\pgfsetroundjoin%
\definecolor{currentfill}{rgb}{0.121569,0.466667,0.705882}%
\pgfsetfillcolor{currentfill}%
\pgfsetfillopacity{0.627144}%
\pgfsetlinewidth{1.003750pt}%
\definecolor{currentstroke}{rgb}{0.121569,0.466667,0.705882}%
\pgfsetstrokecolor{currentstroke}%
\pgfsetstrokeopacity{0.627144}%
\pgfsetdash{}{0pt}%
\pgfpathmoveto{\pgfqpoint{0.861917in}{1.496639in}}%
\pgfpathcurveto{\pgfqpoint{0.870154in}{1.496639in}}{\pgfqpoint{0.878054in}{1.499912in}}{\pgfqpoint{0.883878in}{1.505736in}}%
\pgfpathcurveto{\pgfqpoint{0.889702in}{1.511560in}}{\pgfqpoint{0.892974in}{1.519460in}}{\pgfqpoint{0.892974in}{1.527696in}}%
\pgfpathcurveto{\pgfqpoint{0.892974in}{1.535932in}}{\pgfqpoint{0.889702in}{1.543832in}}{\pgfqpoint{0.883878in}{1.549656in}}%
\pgfpathcurveto{\pgfqpoint{0.878054in}{1.555480in}}{\pgfqpoint{0.870154in}{1.558752in}}{\pgfqpoint{0.861917in}{1.558752in}}%
\pgfpathcurveto{\pgfqpoint{0.853681in}{1.558752in}}{\pgfqpoint{0.845781in}{1.555480in}}{\pgfqpoint{0.839957in}{1.549656in}}%
\pgfpathcurveto{\pgfqpoint{0.834133in}{1.543832in}}{\pgfqpoint{0.830861in}{1.535932in}}{\pgfqpoint{0.830861in}{1.527696in}}%
\pgfpathcurveto{\pgfqpoint{0.830861in}{1.519460in}}{\pgfqpoint{0.834133in}{1.511560in}}{\pgfqpoint{0.839957in}{1.505736in}}%
\pgfpathcurveto{\pgfqpoint{0.845781in}{1.499912in}}{\pgfqpoint{0.853681in}{1.496639in}}{\pgfqpoint{0.861917in}{1.496639in}}%
\pgfpathclose%
\pgfusepath{stroke,fill}%
\end{pgfscope}%
\begin{pgfscope}%
\pgfpathrectangle{\pgfqpoint{0.100000in}{0.212622in}}{\pgfqpoint{3.696000in}{3.696000in}}%
\pgfusepath{clip}%
\pgfsetbuttcap%
\pgfsetroundjoin%
\definecolor{currentfill}{rgb}{0.121569,0.466667,0.705882}%
\pgfsetfillcolor{currentfill}%
\pgfsetfillopacity{0.627238}%
\pgfsetlinewidth{1.003750pt}%
\definecolor{currentstroke}{rgb}{0.121569,0.466667,0.705882}%
\pgfsetstrokecolor{currentstroke}%
\pgfsetstrokeopacity{0.627238}%
\pgfsetdash{}{0pt}%
\pgfpathmoveto{\pgfqpoint{0.860133in}{1.492854in}}%
\pgfpathcurveto{\pgfqpoint{0.868369in}{1.492854in}}{\pgfqpoint{0.876269in}{1.496126in}}{\pgfqpoint{0.882093in}{1.501950in}}%
\pgfpathcurveto{\pgfqpoint{0.887917in}{1.507774in}}{\pgfqpoint{0.891189in}{1.515674in}}{\pgfqpoint{0.891189in}{1.523911in}}%
\pgfpathcurveto{\pgfqpoint{0.891189in}{1.532147in}}{\pgfqpoint{0.887917in}{1.540047in}}{\pgfqpoint{0.882093in}{1.545871in}}%
\pgfpathcurveto{\pgfqpoint{0.876269in}{1.551695in}}{\pgfqpoint{0.868369in}{1.554967in}}{\pgfqpoint{0.860133in}{1.554967in}}%
\pgfpathcurveto{\pgfqpoint{0.851896in}{1.554967in}}{\pgfqpoint{0.843996in}{1.551695in}}{\pgfqpoint{0.838172in}{1.545871in}}%
\pgfpathcurveto{\pgfqpoint{0.832348in}{1.540047in}}{\pgfqpoint{0.829076in}{1.532147in}}{\pgfqpoint{0.829076in}{1.523911in}}%
\pgfpathcurveto{\pgfqpoint{0.829076in}{1.515674in}}{\pgfqpoint{0.832348in}{1.507774in}}{\pgfqpoint{0.838172in}{1.501950in}}%
\pgfpathcurveto{\pgfqpoint{0.843996in}{1.496126in}}{\pgfqpoint{0.851896in}{1.492854in}}{\pgfqpoint{0.860133in}{1.492854in}}%
\pgfpathclose%
\pgfusepath{stroke,fill}%
\end{pgfscope}%
\begin{pgfscope}%
\pgfpathrectangle{\pgfqpoint{0.100000in}{0.212622in}}{\pgfqpoint{3.696000in}{3.696000in}}%
\pgfusepath{clip}%
\pgfsetbuttcap%
\pgfsetroundjoin%
\definecolor{currentfill}{rgb}{0.121569,0.466667,0.705882}%
\pgfsetfillcolor{currentfill}%
\pgfsetfillopacity{0.627333}%
\pgfsetlinewidth{1.003750pt}%
\definecolor{currentstroke}{rgb}{0.121569,0.466667,0.705882}%
\pgfsetstrokecolor{currentstroke}%
\pgfsetstrokeopacity{0.627333}%
\pgfsetdash{}{0pt}%
\pgfpathmoveto{\pgfqpoint{0.860807in}{1.494116in}}%
\pgfpathcurveto{\pgfqpoint{0.869043in}{1.494116in}}{\pgfqpoint{0.876943in}{1.497388in}}{\pgfqpoint{0.882767in}{1.503212in}}%
\pgfpathcurveto{\pgfqpoint{0.888591in}{1.509036in}}{\pgfqpoint{0.891863in}{1.516936in}}{\pgfqpoint{0.891863in}{1.525172in}}%
\pgfpathcurveto{\pgfqpoint{0.891863in}{1.533408in}}{\pgfqpoint{0.888591in}{1.541308in}}{\pgfqpoint{0.882767in}{1.547132in}}%
\pgfpathcurveto{\pgfqpoint{0.876943in}{1.552956in}}{\pgfqpoint{0.869043in}{1.556229in}}{\pgfqpoint{0.860807in}{1.556229in}}%
\pgfpathcurveto{\pgfqpoint{0.852571in}{1.556229in}}{\pgfqpoint{0.844670in}{1.552956in}}{\pgfqpoint{0.838847in}{1.547132in}}%
\pgfpathcurveto{\pgfqpoint{0.833023in}{1.541308in}}{\pgfqpoint{0.829750in}{1.533408in}}{\pgfqpoint{0.829750in}{1.525172in}}%
\pgfpathcurveto{\pgfqpoint{0.829750in}{1.516936in}}{\pgfqpoint{0.833023in}{1.509036in}}{\pgfqpoint{0.838847in}{1.503212in}}%
\pgfpathcurveto{\pgfqpoint{0.844670in}{1.497388in}}{\pgfqpoint{0.852571in}{1.494116in}}{\pgfqpoint{0.860807in}{1.494116in}}%
\pgfpathclose%
\pgfusepath{stroke,fill}%
\end{pgfscope}%
\begin{pgfscope}%
\pgfpathrectangle{\pgfqpoint{0.100000in}{0.212622in}}{\pgfqpoint{3.696000in}{3.696000in}}%
\pgfusepath{clip}%
\pgfsetbuttcap%
\pgfsetroundjoin%
\definecolor{currentfill}{rgb}{0.121569,0.466667,0.705882}%
\pgfsetfillcolor{currentfill}%
\pgfsetfillopacity{0.627335}%
\pgfsetlinewidth{1.003750pt}%
\definecolor{currentstroke}{rgb}{0.121569,0.466667,0.705882}%
\pgfsetstrokecolor{currentstroke}%
\pgfsetstrokeopacity{0.627335}%
\pgfsetdash{}{0pt}%
\pgfpathmoveto{\pgfqpoint{0.860246in}{1.492965in}}%
\pgfpathcurveto{\pgfqpoint{0.868482in}{1.492965in}}{\pgfqpoint{0.876382in}{1.496238in}}{\pgfqpoint{0.882206in}{1.502062in}}%
\pgfpathcurveto{\pgfqpoint{0.888030in}{1.507886in}}{\pgfqpoint{0.891302in}{1.515786in}}{\pgfqpoint{0.891302in}{1.524022in}}%
\pgfpathcurveto{\pgfqpoint{0.891302in}{1.532258in}}{\pgfqpoint{0.888030in}{1.540158in}}{\pgfqpoint{0.882206in}{1.545982in}}%
\pgfpathcurveto{\pgfqpoint{0.876382in}{1.551806in}}{\pgfqpoint{0.868482in}{1.555078in}}{\pgfqpoint{0.860246in}{1.555078in}}%
\pgfpathcurveto{\pgfqpoint{0.852010in}{1.555078in}}{\pgfqpoint{0.844110in}{1.551806in}}{\pgfqpoint{0.838286in}{1.545982in}}%
\pgfpathcurveto{\pgfqpoint{0.832462in}{1.540158in}}{\pgfqpoint{0.829189in}{1.532258in}}{\pgfqpoint{0.829189in}{1.524022in}}%
\pgfpathcurveto{\pgfqpoint{0.829189in}{1.515786in}}{\pgfqpoint{0.832462in}{1.507886in}}{\pgfqpoint{0.838286in}{1.502062in}}%
\pgfpathcurveto{\pgfqpoint{0.844110in}{1.496238in}}{\pgfqpoint{0.852010in}{1.492965in}}{\pgfqpoint{0.860246in}{1.492965in}}%
\pgfpathclose%
\pgfusepath{stroke,fill}%
\end{pgfscope}%
\begin{pgfscope}%
\pgfpathrectangle{\pgfqpoint{0.100000in}{0.212622in}}{\pgfqpoint{3.696000in}{3.696000in}}%
\pgfusepath{clip}%
\pgfsetbuttcap%
\pgfsetroundjoin%
\definecolor{currentfill}{rgb}{0.121569,0.466667,0.705882}%
\pgfsetfillcolor{currentfill}%
\pgfsetfillopacity{0.627348}%
\pgfsetlinewidth{1.003750pt}%
\definecolor{currentstroke}{rgb}{0.121569,0.466667,0.705882}%
\pgfsetstrokecolor{currentstroke}%
\pgfsetstrokeopacity{0.627348}%
\pgfsetdash{}{0pt}%
\pgfpathmoveto{\pgfqpoint{0.860756in}{1.493935in}}%
\pgfpathcurveto{\pgfqpoint{0.868993in}{1.493935in}}{\pgfqpoint{0.876893in}{1.497207in}}{\pgfqpoint{0.882717in}{1.503031in}}%
\pgfpathcurveto{\pgfqpoint{0.888541in}{1.508855in}}{\pgfqpoint{0.891813in}{1.516755in}}{\pgfqpoint{0.891813in}{1.524992in}}%
\pgfpathcurveto{\pgfqpoint{0.891813in}{1.533228in}}{\pgfqpoint{0.888541in}{1.541128in}}{\pgfqpoint{0.882717in}{1.546952in}}%
\pgfpathcurveto{\pgfqpoint{0.876893in}{1.552776in}}{\pgfqpoint{0.868993in}{1.556048in}}{\pgfqpoint{0.860756in}{1.556048in}}%
\pgfpathcurveto{\pgfqpoint{0.852520in}{1.556048in}}{\pgfqpoint{0.844620in}{1.552776in}}{\pgfqpoint{0.838796in}{1.546952in}}%
\pgfpathcurveto{\pgfqpoint{0.832972in}{1.541128in}}{\pgfqpoint{0.829700in}{1.533228in}}{\pgfqpoint{0.829700in}{1.524992in}}%
\pgfpathcurveto{\pgfqpoint{0.829700in}{1.516755in}}{\pgfqpoint{0.832972in}{1.508855in}}{\pgfqpoint{0.838796in}{1.503031in}}%
\pgfpathcurveto{\pgfqpoint{0.844620in}{1.497207in}}{\pgfqpoint{0.852520in}{1.493935in}}{\pgfqpoint{0.860756in}{1.493935in}}%
\pgfpathclose%
\pgfusepath{stroke,fill}%
\end{pgfscope}%
\begin{pgfscope}%
\pgfpathrectangle{\pgfqpoint{0.100000in}{0.212622in}}{\pgfqpoint{3.696000in}{3.696000in}}%
\pgfusepath{clip}%
\pgfsetbuttcap%
\pgfsetroundjoin%
\definecolor{currentfill}{rgb}{0.121569,0.466667,0.705882}%
\pgfsetfillcolor{currentfill}%
\pgfsetfillopacity{0.627375}%
\pgfsetlinewidth{1.003750pt}%
\definecolor{currentstroke}{rgb}{0.121569,0.466667,0.705882}%
\pgfsetstrokecolor{currentstroke}%
\pgfsetstrokeopacity{0.627375}%
\pgfsetdash{}{0pt}%
\pgfpathmoveto{\pgfqpoint{0.860622in}{1.493650in}}%
\pgfpathcurveto{\pgfqpoint{0.868858in}{1.493650in}}{\pgfqpoint{0.876758in}{1.496922in}}{\pgfqpoint{0.882582in}{1.502746in}}%
\pgfpathcurveto{\pgfqpoint{0.888406in}{1.508570in}}{\pgfqpoint{0.891678in}{1.516470in}}{\pgfqpoint{0.891678in}{1.524706in}}%
\pgfpathcurveto{\pgfqpoint{0.891678in}{1.532943in}}{\pgfqpoint{0.888406in}{1.540843in}}{\pgfqpoint{0.882582in}{1.546667in}}%
\pgfpathcurveto{\pgfqpoint{0.876758in}{1.552491in}}{\pgfqpoint{0.868858in}{1.555763in}}{\pgfqpoint{0.860622in}{1.555763in}}%
\pgfpathcurveto{\pgfqpoint{0.852385in}{1.555763in}}{\pgfqpoint{0.844485in}{1.552491in}}{\pgfqpoint{0.838661in}{1.546667in}}%
\pgfpathcurveto{\pgfqpoint{0.832837in}{1.540843in}}{\pgfqpoint{0.829565in}{1.532943in}}{\pgfqpoint{0.829565in}{1.524706in}}%
\pgfpathcurveto{\pgfqpoint{0.829565in}{1.516470in}}{\pgfqpoint{0.832837in}{1.508570in}}{\pgfqpoint{0.838661in}{1.502746in}}%
\pgfpathcurveto{\pgfqpoint{0.844485in}{1.496922in}}{\pgfqpoint{0.852385in}{1.493650in}}{\pgfqpoint{0.860622in}{1.493650in}}%
\pgfpathclose%
\pgfusepath{stroke,fill}%
\end{pgfscope}%
\begin{pgfscope}%
\pgfpathrectangle{\pgfqpoint{0.100000in}{0.212622in}}{\pgfqpoint{3.696000in}{3.696000in}}%
\pgfusepath{clip}%
\pgfsetbuttcap%
\pgfsetroundjoin%
\definecolor{currentfill}{rgb}{0.121569,0.466667,0.705882}%
\pgfsetfillcolor{currentfill}%
\pgfsetfillopacity{0.627383}%
\pgfsetlinewidth{1.003750pt}%
\definecolor{currentstroke}{rgb}{0.121569,0.466667,0.705882}%
\pgfsetstrokecolor{currentstroke}%
\pgfsetstrokeopacity{0.627383}%
\pgfsetdash{}{0pt}%
\pgfpathmoveto{\pgfqpoint{0.860315in}{1.493027in}}%
\pgfpathcurveto{\pgfqpoint{0.868551in}{1.493027in}}{\pgfqpoint{0.876451in}{1.496299in}}{\pgfqpoint{0.882275in}{1.502123in}}%
\pgfpathcurveto{\pgfqpoint{0.888099in}{1.507947in}}{\pgfqpoint{0.891372in}{1.515847in}}{\pgfqpoint{0.891372in}{1.524084in}}%
\pgfpathcurveto{\pgfqpoint{0.891372in}{1.532320in}}{\pgfqpoint{0.888099in}{1.540220in}}{\pgfqpoint{0.882275in}{1.546044in}}%
\pgfpathcurveto{\pgfqpoint{0.876451in}{1.551868in}}{\pgfqpoint{0.868551in}{1.555140in}}{\pgfqpoint{0.860315in}{1.555140in}}%
\pgfpathcurveto{\pgfqpoint{0.852079in}{1.555140in}}{\pgfqpoint{0.844179in}{1.551868in}}{\pgfqpoint{0.838355in}{1.546044in}}%
\pgfpathcurveto{\pgfqpoint{0.832531in}{1.540220in}}{\pgfqpoint{0.829259in}{1.532320in}}{\pgfqpoint{0.829259in}{1.524084in}}%
\pgfpathcurveto{\pgfqpoint{0.829259in}{1.515847in}}{\pgfqpoint{0.832531in}{1.507947in}}{\pgfqpoint{0.838355in}{1.502123in}}%
\pgfpathcurveto{\pgfqpoint{0.844179in}{1.496299in}}{\pgfqpoint{0.852079in}{1.493027in}}{\pgfqpoint{0.860315in}{1.493027in}}%
\pgfpathclose%
\pgfusepath{stroke,fill}%
\end{pgfscope}%
\begin{pgfscope}%
\pgfpathrectangle{\pgfqpoint{0.100000in}{0.212622in}}{\pgfqpoint{3.696000in}{3.696000in}}%
\pgfusepath{clip}%
\pgfsetbuttcap%
\pgfsetroundjoin%
\definecolor{currentfill}{rgb}{0.121569,0.466667,0.705882}%
\pgfsetfillcolor{currentfill}%
\pgfsetfillopacity{0.627409}%
\pgfsetlinewidth{1.003750pt}%
\definecolor{currentstroke}{rgb}{0.121569,0.466667,0.705882}%
\pgfsetstrokecolor{currentstroke}%
\pgfsetstrokeopacity{0.627409}%
\pgfsetdash{}{0pt}%
\pgfpathmoveto{\pgfqpoint{0.860356in}{1.493065in}}%
\pgfpathcurveto{\pgfqpoint{0.868593in}{1.493065in}}{\pgfqpoint{0.876493in}{1.496337in}}{\pgfqpoint{0.882317in}{1.502161in}}%
\pgfpathcurveto{\pgfqpoint{0.888141in}{1.507985in}}{\pgfqpoint{0.891413in}{1.515885in}}{\pgfqpoint{0.891413in}{1.524122in}}%
\pgfpathcurveto{\pgfqpoint{0.891413in}{1.532358in}}{\pgfqpoint{0.888141in}{1.540258in}}{\pgfqpoint{0.882317in}{1.546082in}}%
\pgfpathcurveto{\pgfqpoint{0.876493in}{1.551906in}}{\pgfqpoint{0.868593in}{1.555178in}}{\pgfqpoint{0.860356in}{1.555178in}}%
\pgfpathcurveto{\pgfqpoint{0.852120in}{1.555178in}}{\pgfqpoint{0.844220in}{1.551906in}}{\pgfqpoint{0.838396in}{1.546082in}}%
\pgfpathcurveto{\pgfqpoint{0.832572in}{1.540258in}}{\pgfqpoint{0.829300in}{1.532358in}}{\pgfqpoint{0.829300in}{1.524122in}}%
\pgfpathcurveto{\pgfqpoint{0.829300in}{1.515885in}}{\pgfqpoint{0.832572in}{1.507985in}}{\pgfqpoint{0.838396in}{1.502161in}}%
\pgfpathcurveto{\pgfqpoint{0.844220in}{1.496337in}}{\pgfqpoint{0.852120in}{1.493065in}}{\pgfqpoint{0.860356in}{1.493065in}}%
\pgfpathclose%
\pgfusepath{stroke,fill}%
\end{pgfscope}%
\begin{pgfscope}%
\pgfpathrectangle{\pgfqpoint{0.100000in}{0.212622in}}{\pgfqpoint{3.696000in}{3.696000in}}%
\pgfusepath{clip}%
\pgfsetbuttcap%
\pgfsetroundjoin%
\definecolor{currentfill}{rgb}{0.121569,0.466667,0.705882}%
\pgfsetfillcolor{currentfill}%
\pgfsetfillopacity{0.627421}%
\pgfsetlinewidth{1.003750pt}%
\definecolor{currentstroke}{rgb}{0.121569,0.466667,0.705882}%
\pgfsetstrokecolor{currentstroke}%
\pgfsetstrokeopacity{0.627421}%
\pgfsetdash{}{0pt}%
\pgfpathmoveto{\pgfqpoint{0.860381in}{1.493087in}}%
\pgfpathcurveto{\pgfqpoint{0.868617in}{1.493087in}}{\pgfqpoint{0.876517in}{1.496359in}}{\pgfqpoint{0.882341in}{1.502183in}}%
\pgfpathcurveto{\pgfqpoint{0.888165in}{1.508007in}}{\pgfqpoint{0.891438in}{1.515907in}}{\pgfqpoint{0.891438in}{1.524143in}}%
\pgfpathcurveto{\pgfqpoint{0.891438in}{1.532379in}}{\pgfqpoint{0.888165in}{1.540279in}}{\pgfqpoint{0.882341in}{1.546103in}}%
\pgfpathcurveto{\pgfqpoint{0.876517in}{1.551927in}}{\pgfqpoint{0.868617in}{1.555200in}}{\pgfqpoint{0.860381in}{1.555200in}}%
\pgfpathcurveto{\pgfqpoint{0.852145in}{1.555200in}}{\pgfqpoint{0.844245in}{1.551927in}}{\pgfqpoint{0.838421in}{1.546103in}}%
\pgfpathcurveto{\pgfqpoint{0.832597in}{1.540279in}}{\pgfqpoint{0.829325in}{1.532379in}}{\pgfqpoint{0.829325in}{1.524143in}}%
\pgfpathcurveto{\pgfqpoint{0.829325in}{1.515907in}}{\pgfqpoint{0.832597in}{1.508007in}}{\pgfqpoint{0.838421in}{1.502183in}}%
\pgfpathcurveto{\pgfqpoint{0.844245in}{1.496359in}}{\pgfqpoint{0.852145in}{1.493087in}}{\pgfqpoint{0.860381in}{1.493087in}}%
\pgfpathclose%
\pgfusepath{stroke,fill}%
\end{pgfscope}%
\begin{pgfscope}%
\pgfpathrectangle{\pgfqpoint{0.100000in}{0.212622in}}{\pgfqpoint{3.696000in}{3.696000in}}%
\pgfusepath{clip}%
\pgfsetbuttcap%
\pgfsetroundjoin%
\definecolor{currentfill}{rgb}{0.121569,0.466667,0.705882}%
\pgfsetfillcolor{currentfill}%
\pgfsetfillopacity{0.627427}%
\pgfsetlinewidth{1.003750pt}%
\definecolor{currentstroke}{rgb}{0.121569,0.466667,0.705882}%
\pgfsetstrokecolor{currentstroke}%
\pgfsetstrokeopacity{0.627427}%
\pgfsetdash{}{0pt}%
\pgfpathmoveto{\pgfqpoint{0.860395in}{1.493099in}}%
\pgfpathcurveto{\pgfqpoint{0.868632in}{1.493099in}}{\pgfqpoint{0.876532in}{1.496372in}}{\pgfqpoint{0.882356in}{1.502196in}}%
\pgfpathcurveto{\pgfqpoint{0.888180in}{1.508020in}}{\pgfqpoint{0.891452in}{1.515920in}}{\pgfqpoint{0.891452in}{1.524156in}}%
\pgfpathcurveto{\pgfqpoint{0.891452in}{1.532392in}}{\pgfqpoint{0.888180in}{1.540292in}}{\pgfqpoint{0.882356in}{1.546116in}}%
\pgfpathcurveto{\pgfqpoint{0.876532in}{1.551940in}}{\pgfqpoint{0.868632in}{1.555212in}}{\pgfqpoint{0.860395in}{1.555212in}}%
\pgfpathcurveto{\pgfqpoint{0.852159in}{1.555212in}}{\pgfqpoint{0.844259in}{1.551940in}}{\pgfqpoint{0.838435in}{1.546116in}}%
\pgfpathcurveto{\pgfqpoint{0.832611in}{1.540292in}}{\pgfqpoint{0.829339in}{1.532392in}}{\pgfqpoint{0.829339in}{1.524156in}}%
\pgfpathcurveto{\pgfqpoint{0.829339in}{1.515920in}}{\pgfqpoint{0.832611in}{1.508020in}}{\pgfqpoint{0.838435in}{1.502196in}}%
\pgfpathcurveto{\pgfqpoint{0.844259in}{1.496372in}}{\pgfqpoint{0.852159in}{1.493099in}}{\pgfqpoint{0.860395in}{1.493099in}}%
\pgfpathclose%
\pgfusepath{stroke,fill}%
\end{pgfscope}%
\begin{pgfscope}%
\pgfpathrectangle{\pgfqpoint{0.100000in}{0.212622in}}{\pgfqpoint{3.696000in}{3.696000in}}%
\pgfusepath{clip}%
\pgfsetbuttcap%
\pgfsetroundjoin%
\definecolor{currentfill}{rgb}{0.121569,0.466667,0.705882}%
\pgfsetfillcolor{currentfill}%
\pgfsetfillopacity{0.627430}%
\pgfsetlinewidth{1.003750pt}%
\definecolor{currentstroke}{rgb}{0.121569,0.466667,0.705882}%
\pgfsetstrokecolor{currentstroke}%
\pgfsetstrokeopacity{0.627430}%
\pgfsetdash{}{0pt}%
\pgfpathmoveto{\pgfqpoint{0.860404in}{1.493107in}}%
\pgfpathcurveto{\pgfqpoint{0.868640in}{1.493107in}}{\pgfqpoint{0.876540in}{1.496379in}}{\pgfqpoint{0.882364in}{1.502203in}}%
\pgfpathcurveto{\pgfqpoint{0.888188in}{1.508027in}}{\pgfqpoint{0.891460in}{1.515927in}}{\pgfqpoint{0.891460in}{1.524163in}}%
\pgfpathcurveto{\pgfqpoint{0.891460in}{1.532399in}}{\pgfqpoint{0.888188in}{1.540299in}}{\pgfqpoint{0.882364in}{1.546123in}}%
\pgfpathcurveto{\pgfqpoint{0.876540in}{1.551947in}}{\pgfqpoint{0.868640in}{1.555220in}}{\pgfqpoint{0.860404in}{1.555220in}}%
\pgfpathcurveto{\pgfqpoint{0.852168in}{1.555220in}}{\pgfqpoint{0.844268in}{1.551947in}}{\pgfqpoint{0.838444in}{1.546123in}}%
\pgfpathcurveto{\pgfqpoint{0.832620in}{1.540299in}}{\pgfqpoint{0.829347in}{1.532399in}}{\pgfqpoint{0.829347in}{1.524163in}}%
\pgfpathcurveto{\pgfqpoint{0.829347in}{1.515927in}}{\pgfqpoint{0.832620in}{1.508027in}}{\pgfqpoint{0.838444in}{1.502203in}}%
\pgfpathcurveto{\pgfqpoint{0.844268in}{1.496379in}}{\pgfqpoint{0.852168in}{1.493107in}}{\pgfqpoint{0.860404in}{1.493107in}}%
\pgfpathclose%
\pgfusepath{stroke,fill}%
\end{pgfscope}%
\begin{pgfscope}%
\pgfpathrectangle{\pgfqpoint{0.100000in}{0.212622in}}{\pgfqpoint{3.696000in}{3.696000in}}%
\pgfusepath{clip}%
\pgfsetbuttcap%
\pgfsetroundjoin%
\definecolor{currentfill}{rgb}{0.121569,0.466667,0.705882}%
\pgfsetfillcolor{currentfill}%
\pgfsetfillopacity{0.627432}%
\pgfsetlinewidth{1.003750pt}%
\definecolor{currentstroke}{rgb}{0.121569,0.466667,0.705882}%
\pgfsetstrokecolor{currentstroke}%
\pgfsetstrokeopacity{0.627432}%
\pgfsetdash{}{0pt}%
\pgfpathmoveto{\pgfqpoint{0.860409in}{1.493111in}}%
\pgfpathcurveto{\pgfqpoint{0.868645in}{1.493111in}}{\pgfqpoint{0.876545in}{1.496383in}}{\pgfqpoint{0.882369in}{1.502207in}}%
\pgfpathcurveto{\pgfqpoint{0.888193in}{1.508031in}}{\pgfqpoint{0.891465in}{1.515931in}}{\pgfqpoint{0.891465in}{1.524168in}}%
\pgfpathcurveto{\pgfqpoint{0.891465in}{1.532404in}}{\pgfqpoint{0.888193in}{1.540304in}}{\pgfqpoint{0.882369in}{1.546128in}}%
\pgfpathcurveto{\pgfqpoint{0.876545in}{1.551952in}}{\pgfqpoint{0.868645in}{1.555224in}}{\pgfqpoint{0.860409in}{1.555224in}}%
\pgfpathcurveto{\pgfqpoint{0.852172in}{1.555224in}}{\pgfqpoint{0.844272in}{1.551952in}}{\pgfqpoint{0.838448in}{1.546128in}}%
\pgfpathcurveto{\pgfqpoint{0.832624in}{1.540304in}}{\pgfqpoint{0.829352in}{1.532404in}}{\pgfqpoint{0.829352in}{1.524168in}}%
\pgfpathcurveto{\pgfqpoint{0.829352in}{1.515931in}}{\pgfqpoint{0.832624in}{1.508031in}}{\pgfqpoint{0.838448in}{1.502207in}}%
\pgfpathcurveto{\pgfqpoint{0.844272in}{1.496383in}}{\pgfqpoint{0.852172in}{1.493111in}}{\pgfqpoint{0.860409in}{1.493111in}}%
\pgfpathclose%
\pgfusepath{stroke,fill}%
\end{pgfscope}%
\begin{pgfscope}%
\pgfpathrectangle{\pgfqpoint{0.100000in}{0.212622in}}{\pgfqpoint{3.696000in}{3.696000in}}%
\pgfusepath{clip}%
\pgfsetbuttcap%
\pgfsetroundjoin%
\definecolor{currentfill}{rgb}{0.121569,0.466667,0.705882}%
\pgfsetfillcolor{currentfill}%
\pgfsetfillopacity{0.627433}%
\pgfsetlinewidth{1.003750pt}%
\definecolor{currentstroke}{rgb}{0.121569,0.466667,0.705882}%
\pgfsetstrokecolor{currentstroke}%
\pgfsetstrokeopacity{0.627433}%
\pgfsetdash{}{0pt}%
\pgfpathmoveto{\pgfqpoint{0.860411in}{1.493114in}}%
\pgfpathcurveto{\pgfqpoint{0.868648in}{1.493114in}}{\pgfqpoint{0.876548in}{1.496386in}}{\pgfqpoint{0.882372in}{1.502210in}}%
\pgfpathcurveto{\pgfqpoint{0.888196in}{1.508034in}}{\pgfqpoint{0.891468in}{1.515934in}}{\pgfqpoint{0.891468in}{1.524170in}}%
\pgfpathcurveto{\pgfqpoint{0.891468in}{1.532406in}}{\pgfqpoint{0.888196in}{1.540306in}}{\pgfqpoint{0.882372in}{1.546130in}}%
\pgfpathcurveto{\pgfqpoint{0.876548in}{1.551954in}}{\pgfqpoint{0.868648in}{1.555227in}}{\pgfqpoint{0.860411in}{1.555227in}}%
\pgfpathcurveto{\pgfqpoint{0.852175in}{1.555227in}}{\pgfqpoint{0.844275in}{1.551954in}}{\pgfqpoint{0.838451in}{1.546130in}}%
\pgfpathcurveto{\pgfqpoint{0.832627in}{1.540306in}}{\pgfqpoint{0.829355in}{1.532406in}}{\pgfqpoint{0.829355in}{1.524170in}}%
\pgfpathcurveto{\pgfqpoint{0.829355in}{1.515934in}}{\pgfqpoint{0.832627in}{1.508034in}}{\pgfqpoint{0.838451in}{1.502210in}}%
\pgfpathcurveto{\pgfqpoint{0.844275in}{1.496386in}}{\pgfqpoint{0.852175in}{1.493114in}}{\pgfqpoint{0.860411in}{1.493114in}}%
\pgfpathclose%
\pgfusepath{stroke,fill}%
\end{pgfscope}%
\begin{pgfscope}%
\pgfpathrectangle{\pgfqpoint{0.100000in}{0.212622in}}{\pgfqpoint{3.696000in}{3.696000in}}%
\pgfusepath{clip}%
\pgfsetbuttcap%
\pgfsetroundjoin%
\definecolor{currentfill}{rgb}{0.121569,0.466667,0.705882}%
\pgfsetfillcolor{currentfill}%
\pgfsetfillopacity{0.627433}%
\pgfsetlinewidth{1.003750pt}%
\definecolor{currentstroke}{rgb}{0.121569,0.466667,0.705882}%
\pgfsetstrokecolor{currentstroke}%
\pgfsetstrokeopacity{0.627433}%
\pgfsetdash{}{0pt}%
\pgfpathmoveto{\pgfqpoint{0.860413in}{1.493115in}}%
\pgfpathcurveto{\pgfqpoint{0.868649in}{1.493115in}}{\pgfqpoint{0.876549in}{1.496387in}}{\pgfqpoint{0.882373in}{1.502211in}}%
\pgfpathcurveto{\pgfqpoint{0.888197in}{1.508035in}}{\pgfqpoint{0.891469in}{1.515935in}}{\pgfqpoint{0.891469in}{1.524172in}}%
\pgfpathcurveto{\pgfqpoint{0.891469in}{1.532408in}}{\pgfqpoint{0.888197in}{1.540308in}}{\pgfqpoint{0.882373in}{1.546132in}}%
\pgfpathcurveto{\pgfqpoint{0.876549in}{1.551956in}}{\pgfqpoint{0.868649in}{1.555228in}}{\pgfqpoint{0.860413in}{1.555228in}}%
\pgfpathcurveto{\pgfqpoint{0.852177in}{1.555228in}}{\pgfqpoint{0.844277in}{1.551956in}}{\pgfqpoint{0.838453in}{1.546132in}}%
\pgfpathcurveto{\pgfqpoint{0.832629in}{1.540308in}}{\pgfqpoint{0.829356in}{1.532408in}}{\pgfqpoint{0.829356in}{1.524172in}}%
\pgfpathcurveto{\pgfqpoint{0.829356in}{1.515935in}}{\pgfqpoint{0.832629in}{1.508035in}}{\pgfqpoint{0.838453in}{1.502211in}}%
\pgfpathcurveto{\pgfqpoint{0.844277in}{1.496387in}}{\pgfqpoint{0.852177in}{1.493115in}}{\pgfqpoint{0.860413in}{1.493115in}}%
\pgfpathclose%
\pgfusepath{stroke,fill}%
\end{pgfscope}%
\begin{pgfscope}%
\pgfpathrectangle{\pgfqpoint{0.100000in}{0.212622in}}{\pgfqpoint{3.696000in}{3.696000in}}%
\pgfusepath{clip}%
\pgfsetbuttcap%
\pgfsetroundjoin%
\definecolor{currentfill}{rgb}{0.121569,0.466667,0.705882}%
\pgfsetfillcolor{currentfill}%
\pgfsetfillopacity{0.627433}%
\pgfsetlinewidth{1.003750pt}%
\definecolor{currentstroke}{rgb}{0.121569,0.466667,0.705882}%
\pgfsetstrokecolor{currentstroke}%
\pgfsetstrokeopacity{0.627433}%
\pgfsetdash{}{0pt}%
\pgfpathmoveto{\pgfqpoint{0.860414in}{1.493116in}}%
\pgfpathcurveto{\pgfqpoint{0.868650in}{1.493116in}}{\pgfqpoint{0.876550in}{1.496388in}}{\pgfqpoint{0.882374in}{1.502212in}}%
\pgfpathcurveto{\pgfqpoint{0.888198in}{1.508036in}}{\pgfqpoint{0.891470in}{1.515936in}}{\pgfqpoint{0.891470in}{1.524172in}}%
\pgfpathcurveto{\pgfqpoint{0.891470in}{1.532409in}}{\pgfqpoint{0.888198in}{1.540309in}}{\pgfqpoint{0.882374in}{1.546133in}}%
\pgfpathcurveto{\pgfqpoint{0.876550in}{1.551957in}}{\pgfqpoint{0.868650in}{1.555229in}}{\pgfqpoint{0.860414in}{1.555229in}}%
\pgfpathcurveto{\pgfqpoint{0.852177in}{1.555229in}}{\pgfqpoint{0.844277in}{1.551957in}}{\pgfqpoint{0.838453in}{1.546133in}}%
\pgfpathcurveto{\pgfqpoint{0.832630in}{1.540309in}}{\pgfqpoint{0.829357in}{1.532409in}}{\pgfqpoint{0.829357in}{1.524172in}}%
\pgfpathcurveto{\pgfqpoint{0.829357in}{1.515936in}}{\pgfqpoint{0.832630in}{1.508036in}}{\pgfqpoint{0.838453in}{1.502212in}}%
\pgfpathcurveto{\pgfqpoint{0.844277in}{1.496388in}}{\pgfqpoint{0.852177in}{1.493116in}}{\pgfqpoint{0.860414in}{1.493116in}}%
\pgfpathclose%
\pgfusepath{stroke,fill}%
\end{pgfscope}%
\begin{pgfscope}%
\pgfpathrectangle{\pgfqpoint{0.100000in}{0.212622in}}{\pgfqpoint{3.696000in}{3.696000in}}%
\pgfusepath{clip}%
\pgfsetbuttcap%
\pgfsetroundjoin%
\definecolor{currentfill}{rgb}{0.121569,0.466667,0.705882}%
\pgfsetfillcolor{currentfill}%
\pgfsetfillopacity{0.627433}%
\pgfsetlinewidth{1.003750pt}%
\definecolor{currentstroke}{rgb}{0.121569,0.466667,0.705882}%
\pgfsetstrokecolor{currentstroke}%
\pgfsetstrokeopacity{0.627433}%
\pgfsetdash{}{0pt}%
\pgfpathmoveto{\pgfqpoint{0.860414in}{1.493116in}}%
\pgfpathcurveto{\pgfqpoint{0.868650in}{1.493116in}}{\pgfqpoint{0.876551in}{1.496389in}}{\pgfqpoint{0.882374in}{1.502213in}}%
\pgfpathcurveto{\pgfqpoint{0.888198in}{1.508036in}}{\pgfqpoint{0.891471in}{1.515937in}}{\pgfqpoint{0.891471in}{1.524173in}}%
\pgfpathcurveto{\pgfqpoint{0.891471in}{1.532409in}}{\pgfqpoint{0.888198in}{1.540309in}}{\pgfqpoint{0.882374in}{1.546133in}}%
\pgfpathcurveto{\pgfqpoint{0.876551in}{1.551957in}}{\pgfqpoint{0.868650in}{1.555229in}}{\pgfqpoint{0.860414in}{1.555229in}}%
\pgfpathcurveto{\pgfqpoint{0.852178in}{1.555229in}}{\pgfqpoint{0.844278in}{1.551957in}}{\pgfqpoint{0.838454in}{1.546133in}}%
\pgfpathcurveto{\pgfqpoint{0.832630in}{1.540309in}}{\pgfqpoint{0.829358in}{1.532409in}}{\pgfqpoint{0.829358in}{1.524173in}}%
\pgfpathcurveto{\pgfqpoint{0.829358in}{1.515937in}}{\pgfqpoint{0.832630in}{1.508036in}}{\pgfqpoint{0.838454in}{1.502213in}}%
\pgfpathcurveto{\pgfqpoint{0.844278in}{1.496389in}}{\pgfqpoint{0.852178in}{1.493116in}}{\pgfqpoint{0.860414in}{1.493116in}}%
\pgfpathclose%
\pgfusepath{stroke,fill}%
\end{pgfscope}%
\begin{pgfscope}%
\pgfpathrectangle{\pgfqpoint{0.100000in}{0.212622in}}{\pgfqpoint{3.696000in}{3.696000in}}%
\pgfusepath{clip}%
\pgfsetbuttcap%
\pgfsetroundjoin%
\definecolor{currentfill}{rgb}{0.121569,0.466667,0.705882}%
\pgfsetfillcolor{currentfill}%
\pgfsetfillopacity{0.627433}%
\pgfsetlinewidth{1.003750pt}%
\definecolor{currentstroke}{rgb}{0.121569,0.466667,0.705882}%
\pgfsetstrokecolor{currentstroke}%
\pgfsetstrokeopacity{0.627433}%
\pgfsetdash{}{0pt}%
\pgfpathmoveto{\pgfqpoint{0.860414in}{1.493117in}}%
\pgfpathcurveto{\pgfqpoint{0.868651in}{1.493117in}}{\pgfqpoint{0.876551in}{1.496389in}}{\pgfqpoint{0.882375in}{1.502213in}}%
\pgfpathcurveto{\pgfqpoint{0.888199in}{1.508037in}}{\pgfqpoint{0.891471in}{1.515937in}}{\pgfqpoint{0.891471in}{1.524173in}}%
\pgfpathcurveto{\pgfqpoint{0.891471in}{1.532409in}}{\pgfqpoint{0.888199in}{1.540309in}}{\pgfqpoint{0.882375in}{1.546133in}}%
\pgfpathcurveto{\pgfqpoint{0.876551in}{1.551957in}}{\pgfqpoint{0.868651in}{1.555230in}}{\pgfqpoint{0.860414in}{1.555230in}}%
\pgfpathcurveto{\pgfqpoint{0.852178in}{1.555230in}}{\pgfqpoint{0.844278in}{1.551957in}}{\pgfqpoint{0.838454in}{1.546133in}}%
\pgfpathcurveto{\pgfqpoint{0.832630in}{1.540309in}}{\pgfqpoint{0.829358in}{1.532409in}}{\pgfqpoint{0.829358in}{1.524173in}}%
\pgfpathcurveto{\pgfqpoint{0.829358in}{1.515937in}}{\pgfqpoint{0.832630in}{1.508037in}}{\pgfqpoint{0.838454in}{1.502213in}}%
\pgfpathcurveto{\pgfqpoint{0.844278in}{1.496389in}}{\pgfqpoint{0.852178in}{1.493117in}}{\pgfqpoint{0.860414in}{1.493117in}}%
\pgfpathclose%
\pgfusepath{stroke,fill}%
\end{pgfscope}%
\begin{pgfscope}%
\pgfpathrectangle{\pgfqpoint{0.100000in}{0.212622in}}{\pgfqpoint{3.696000in}{3.696000in}}%
\pgfusepath{clip}%
\pgfsetbuttcap%
\pgfsetroundjoin%
\definecolor{currentfill}{rgb}{0.121569,0.466667,0.705882}%
\pgfsetfillcolor{currentfill}%
\pgfsetfillopacity{0.627433}%
\pgfsetlinewidth{1.003750pt}%
\definecolor{currentstroke}{rgb}{0.121569,0.466667,0.705882}%
\pgfsetstrokecolor{currentstroke}%
\pgfsetstrokeopacity{0.627433}%
\pgfsetdash{}{0pt}%
\pgfpathmoveto{\pgfqpoint{0.860415in}{1.493117in}}%
\pgfpathcurveto{\pgfqpoint{0.868651in}{1.493117in}}{\pgfqpoint{0.876551in}{1.496389in}}{\pgfqpoint{0.882375in}{1.502213in}}%
\pgfpathcurveto{\pgfqpoint{0.888199in}{1.508037in}}{\pgfqpoint{0.891471in}{1.515937in}}{\pgfqpoint{0.891471in}{1.524173in}}%
\pgfpathcurveto{\pgfqpoint{0.891471in}{1.532410in}}{\pgfqpoint{0.888199in}{1.540310in}}{\pgfqpoint{0.882375in}{1.546134in}}%
\pgfpathcurveto{\pgfqpoint{0.876551in}{1.551957in}}{\pgfqpoint{0.868651in}{1.555230in}}{\pgfqpoint{0.860415in}{1.555230in}}%
\pgfpathcurveto{\pgfqpoint{0.852178in}{1.555230in}}{\pgfqpoint{0.844278in}{1.551957in}}{\pgfqpoint{0.838454in}{1.546134in}}%
\pgfpathcurveto{\pgfqpoint{0.832630in}{1.540310in}}{\pgfqpoint{0.829358in}{1.532410in}}{\pgfqpoint{0.829358in}{1.524173in}}%
\pgfpathcurveto{\pgfqpoint{0.829358in}{1.515937in}}{\pgfqpoint{0.832630in}{1.508037in}}{\pgfqpoint{0.838454in}{1.502213in}}%
\pgfpathcurveto{\pgfqpoint{0.844278in}{1.496389in}}{\pgfqpoint{0.852178in}{1.493117in}}{\pgfqpoint{0.860415in}{1.493117in}}%
\pgfpathclose%
\pgfusepath{stroke,fill}%
\end{pgfscope}%
\begin{pgfscope}%
\pgfpathrectangle{\pgfqpoint{0.100000in}{0.212622in}}{\pgfqpoint{3.696000in}{3.696000in}}%
\pgfusepath{clip}%
\pgfsetbuttcap%
\pgfsetroundjoin%
\definecolor{currentfill}{rgb}{0.121569,0.466667,0.705882}%
\pgfsetfillcolor{currentfill}%
\pgfsetfillopacity{0.627433}%
\pgfsetlinewidth{1.003750pt}%
\definecolor{currentstroke}{rgb}{0.121569,0.466667,0.705882}%
\pgfsetstrokecolor{currentstroke}%
\pgfsetstrokeopacity{0.627433}%
\pgfsetdash{}{0pt}%
\pgfpathmoveto{\pgfqpoint{0.860415in}{1.493117in}}%
\pgfpathcurveto{\pgfqpoint{0.868651in}{1.493117in}}{\pgfqpoint{0.876551in}{1.496389in}}{\pgfqpoint{0.882375in}{1.502213in}}%
\pgfpathcurveto{\pgfqpoint{0.888199in}{1.508037in}}{\pgfqpoint{0.891471in}{1.515937in}}{\pgfqpoint{0.891471in}{1.524173in}}%
\pgfpathcurveto{\pgfqpoint{0.891471in}{1.532410in}}{\pgfqpoint{0.888199in}{1.540310in}}{\pgfqpoint{0.882375in}{1.546134in}}%
\pgfpathcurveto{\pgfqpoint{0.876551in}{1.551958in}}{\pgfqpoint{0.868651in}{1.555230in}}{\pgfqpoint{0.860415in}{1.555230in}}%
\pgfpathcurveto{\pgfqpoint{0.852178in}{1.555230in}}{\pgfqpoint{0.844278in}{1.551958in}}{\pgfqpoint{0.838454in}{1.546134in}}%
\pgfpathcurveto{\pgfqpoint{0.832631in}{1.540310in}}{\pgfqpoint{0.829358in}{1.532410in}}{\pgfqpoint{0.829358in}{1.524173in}}%
\pgfpathcurveto{\pgfqpoint{0.829358in}{1.515937in}}{\pgfqpoint{0.832631in}{1.508037in}}{\pgfqpoint{0.838454in}{1.502213in}}%
\pgfpathcurveto{\pgfqpoint{0.844278in}{1.496389in}}{\pgfqpoint{0.852178in}{1.493117in}}{\pgfqpoint{0.860415in}{1.493117in}}%
\pgfpathclose%
\pgfusepath{stroke,fill}%
\end{pgfscope}%
\begin{pgfscope}%
\pgfpathrectangle{\pgfqpoint{0.100000in}{0.212622in}}{\pgfqpoint{3.696000in}{3.696000in}}%
\pgfusepath{clip}%
\pgfsetbuttcap%
\pgfsetroundjoin%
\definecolor{currentfill}{rgb}{0.121569,0.466667,0.705882}%
\pgfsetfillcolor{currentfill}%
\pgfsetfillopacity{0.627433}%
\pgfsetlinewidth{1.003750pt}%
\definecolor{currentstroke}{rgb}{0.121569,0.466667,0.705882}%
\pgfsetstrokecolor{currentstroke}%
\pgfsetstrokeopacity{0.627433}%
\pgfsetdash{}{0pt}%
\pgfpathmoveto{\pgfqpoint{0.860415in}{1.493117in}}%
\pgfpathcurveto{\pgfqpoint{0.868651in}{1.493117in}}{\pgfqpoint{0.876551in}{1.496389in}}{\pgfqpoint{0.882375in}{1.502213in}}%
\pgfpathcurveto{\pgfqpoint{0.888199in}{1.508037in}}{\pgfqpoint{0.891471in}{1.515937in}}{\pgfqpoint{0.891471in}{1.524173in}}%
\pgfpathcurveto{\pgfqpoint{0.891471in}{1.532410in}}{\pgfqpoint{0.888199in}{1.540310in}}{\pgfqpoint{0.882375in}{1.546134in}}%
\pgfpathcurveto{\pgfqpoint{0.876551in}{1.551958in}}{\pgfqpoint{0.868651in}{1.555230in}}{\pgfqpoint{0.860415in}{1.555230in}}%
\pgfpathcurveto{\pgfqpoint{0.852178in}{1.555230in}}{\pgfqpoint{0.844278in}{1.551958in}}{\pgfqpoint{0.838454in}{1.546134in}}%
\pgfpathcurveto{\pgfqpoint{0.832631in}{1.540310in}}{\pgfqpoint{0.829358in}{1.532410in}}{\pgfqpoint{0.829358in}{1.524173in}}%
\pgfpathcurveto{\pgfqpoint{0.829358in}{1.515937in}}{\pgfqpoint{0.832631in}{1.508037in}}{\pgfqpoint{0.838454in}{1.502213in}}%
\pgfpathcurveto{\pgfqpoint{0.844278in}{1.496389in}}{\pgfqpoint{0.852178in}{1.493117in}}{\pgfqpoint{0.860415in}{1.493117in}}%
\pgfpathclose%
\pgfusepath{stroke,fill}%
\end{pgfscope}%
\begin{pgfscope}%
\pgfpathrectangle{\pgfqpoint{0.100000in}{0.212622in}}{\pgfqpoint{3.696000in}{3.696000in}}%
\pgfusepath{clip}%
\pgfsetbuttcap%
\pgfsetroundjoin%
\definecolor{currentfill}{rgb}{0.121569,0.466667,0.705882}%
\pgfsetfillcolor{currentfill}%
\pgfsetfillopacity{0.627433}%
\pgfsetlinewidth{1.003750pt}%
\definecolor{currentstroke}{rgb}{0.121569,0.466667,0.705882}%
\pgfsetstrokecolor{currentstroke}%
\pgfsetstrokeopacity{0.627433}%
\pgfsetdash{}{0pt}%
\pgfpathmoveto{\pgfqpoint{0.860415in}{1.493117in}}%
\pgfpathcurveto{\pgfqpoint{0.868651in}{1.493117in}}{\pgfqpoint{0.876551in}{1.496389in}}{\pgfqpoint{0.882375in}{1.502213in}}%
\pgfpathcurveto{\pgfqpoint{0.888199in}{1.508037in}}{\pgfqpoint{0.891471in}{1.515937in}}{\pgfqpoint{0.891471in}{1.524173in}}%
\pgfpathcurveto{\pgfqpoint{0.891471in}{1.532410in}}{\pgfqpoint{0.888199in}{1.540310in}}{\pgfqpoint{0.882375in}{1.546134in}}%
\pgfpathcurveto{\pgfqpoint{0.876551in}{1.551958in}}{\pgfqpoint{0.868651in}{1.555230in}}{\pgfqpoint{0.860415in}{1.555230in}}%
\pgfpathcurveto{\pgfqpoint{0.852178in}{1.555230in}}{\pgfqpoint{0.844278in}{1.551958in}}{\pgfqpoint{0.838454in}{1.546134in}}%
\pgfpathcurveto{\pgfqpoint{0.832631in}{1.540310in}}{\pgfqpoint{0.829358in}{1.532410in}}{\pgfqpoint{0.829358in}{1.524173in}}%
\pgfpathcurveto{\pgfqpoint{0.829358in}{1.515937in}}{\pgfqpoint{0.832631in}{1.508037in}}{\pgfqpoint{0.838454in}{1.502213in}}%
\pgfpathcurveto{\pgfqpoint{0.844278in}{1.496389in}}{\pgfqpoint{0.852178in}{1.493117in}}{\pgfqpoint{0.860415in}{1.493117in}}%
\pgfpathclose%
\pgfusepath{stroke,fill}%
\end{pgfscope}%
\begin{pgfscope}%
\pgfpathrectangle{\pgfqpoint{0.100000in}{0.212622in}}{\pgfqpoint{3.696000in}{3.696000in}}%
\pgfusepath{clip}%
\pgfsetbuttcap%
\pgfsetroundjoin%
\definecolor{currentfill}{rgb}{0.121569,0.466667,0.705882}%
\pgfsetfillcolor{currentfill}%
\pgfsetfillopacity{0.627433}%
\pgfsetlinewidth{1.003750pt}%
\definecolor{currentstroke}{rgb}{0.121569,0.466667,0.705882}%
\pgfsetstrokecolor{currentstroke}%
\pgfsetstrokeopacity{0.627433}%
\pgfsetdash{}{0pt}%
\pgfpathmoveto{\pgfqpoint{0.860415in}{1.493117in}}%
\pgfpathcurveto{\pgfqpoint{0.868651in}{1.493117in}}{\pgfqpoint{0.876551in}{1.496389in}}{\pgfqpoint{0.882375in}{1.502213in}}%
\pgfpathcurveto{\pgfqpoint{0.888199in}{1.508037in}}{\pgfqpoint{0.891471in}{1.515937in}}{\pgfqpoint{0.891471in}{1.524173in}}%
\pgfpathcurveto{\pgfqpoint{0.891471in}{1.532410in}}{\pgfqpoint{0.888199in}{1.540310in}}{\pgfqpoint{0.882375in}{1.546134in}}%
\pgfpathcurveto{\pgfqpoint{0.876551in}{1.551958in}}{\pgfqpoint{0.868651in}{1.555230in}}{\pgfqpoint{0.860415in}{1.555230in}}%
\pgfpathcurveto{\pgfqpoint{0.852178in}{1.555230in}}{\pgfqpoint{0.844278in}{1.551958in}}{\pgfqpoint{0.838454in}{1.546134in}}%
\pgfpathcurveto{\pgfqpoint{0.832631in}{1.540310in}}{\pgfqpoint{0.829358in}{1.532410in}}{\pgfqpoint{0.829358in}{1.524173in}}%
\pgfpathcurveto{\pgfqpoint{0.829358in}{1.515937in}}{\pgfqpoint{0.832631in}{1.508037in}}{\pgfqpoint{0.838454in}{1.502213in}}%
\pgfpathcurveto{\pgfqpoint{0.844278in}{1.496389in}}{\pgfqpoint{0.852178in}{1.493117in}}{\pgfqpoint{0.860415in}{1.493117in}}%
\pgfpathclose%
\pgfusepath{stroke,fill}%
\end{pgfscope}%
\begin{pgfscope}%
\pgfpathrectangle{\pgfqpoint{0.100000in}{0.212622in}}{\pgfqpoint{3.696000in}{3.696000in}}%
\pgfusepath{clip}%
\pgfsetbuttcap%
\pgfsetroundjoin%
\definecolor{currentfill}{rgb}{0.121569,0.466667,0.705882}%
\pgfsetfillcolor{currentfill}%
\pgfsetfillopacity{0.627433}%
\pgfsetlinewidth{1.003750pt}%
\definecolor{currentstroke}{rgb}{0.121569,0.466667,0.705882}%
\pgfsetstrokecolor{currentstroke}%
\pgfsetstrokeopacity{0.627433}%
\pgfsetdash{}{0pt}%
\pgfpathmoveto{\pgfqpoint{0.860415in}{1.493117in}}%
\pgfpathcurveto{\pgfqpoint{0.868651in}{1.493117in}}{\pgfqpoint{0.876551in}{1.496389in}}{\pgfqpoint{0.882375in}{1.502213in}}%
\pgfpathcurveto{\pgfqpoint{0.888199in}{1.508037in}}{\pgfqpoint{0.891471in}{1.515937in}}{\pgfqpoint{0.891471in}{1.524173in}}%
\pgfpathcurveto{\pgfqpoint{0.891471in}{1.532410in}}{\pgfqpoint{0.888199in}{1.540310in}}{\pgfqpoint{0.882375in}{1.546134in}}%
\pgfpathcurveto{\pgfqpoint{0.876551in}{1.551958in}}{\pgfqpoint{0.868651in}{1.555230in}}{\pgfqpoint{0.860415in}{1.555230in}}%
\pgfpathcurveto{\pgfqpoint{0.852178in}{1.555230in}}{\pgfqpoint{0.844278in}{1.551958in}}{\pgfqpoint{0.838454in}{1.546134in}}%
\pgfpathcurveto{\pgfqpoint{0.832631in}{1.540310in}}{\pgfqpoint{0.829358in}{1.532410in}}{\pgfqpoint{0.829358in}{1.524173in}}%
\pgfpathcurveto{\pgfqpoint{0.829358in}{1.515937in}}{\pgfqpoint{0.832631in}{1.508037in}}{\pgfqpoint{0.838454in}{1.502213in}}%
\pgfpathcurveto{\pgfqpoint{0.844278in}{1.496389in}}{\pgfqpoint{0.852178in}{1.493117in}}{\pgfqpoint{0.860415in}{1.493117in}}%
\pgfpathclose%
\pgfusepath{stroke,fill}%
\end{pgfscope}%
\begin{pgfscope}%
\pgfpathrectangle{\pgfqpoint{0.100000in}{0.212622in}}{\pgfqpoint{3.696000in}{3.696000in}}%
\pgfusepath{clip}%
\pgfsetbuttcap%
\pgfsetroundjoin%
\definecolor{currentfill}{rgb}{0.121569,0.466667,0.705882}%
\pgfsetfillcolor{currentfill}%
\pgfsetfillopacity{0.627433}%
\pgfsetlinewidth{1.003750pt}%
\definecolor{currentstroke}{rgb}{0.121569,0.466667,0.705882}%
\pgfsetstrokecolor{currentstroke}%
\pgfsetstrokeopacity{0.627433}%
\pgfsetdash{}{0pt}%
\pgfpathmoveto{\pgfqpoint{0.860415in}{1.493117in}}%
\pgfpathcurveto{\pgfqpoint{0.868651in}{1.493117in}}{\pgfqpoint{0.876551in}{1.496389in}}{\pgfqpoint{0.882375in}{1.502213in}}%
\pgfpathcurveto{\pgfqpoint{0.888199in}{1.508037in}}{\pgfqpoint{0.891471in}{1.515937in}}{\pgfqpoint{0.891471in}{1.524173in}}%
\pgfpathcurveto{\pgfqpoint{0.891471in}{1.532410in}}{\pgfqpoint{0.888199in}{1.540310in}}{\pgfqpoint{0.882375in}{1.546134in}}%
\pgfpathcurveto{\pgfqpoint{0.876551in}{1.551958in}}{\pgfqpoint{0.868651in}{1.555230in}}{\pgfqpoint{0.860415in}{1.555230in}}%
\pgfpathcurveto{\pgfqpoint{0.852178in}{1.555230in}}{\pgfqpoint{0.844278in}{1.551958in}}{\pgfqpoint{0.838454in}{1.546134in}}%
\pgfpathcurveto{\pgfqpoint{0.832631in}{1.540310in}}{\pgfqpoint{0.829358in}{1.532410in}}{\pgfqpoint{0.829358in}{1.524173in}}%
\pgfpathcurveto{\pgfqpoint{0.829358in}{1.515937in}}{\pgfqpoint{0.832631in}{1.508037in}}{\pgfqpoint{0.838454in}{1.502213in}}%
\pgfpathcurveto{\pgfqpoint{0.844278in}{1.496389in}}{\pgfqpoint{0.852178in}{1.493117in}}{\pgfqpoint{0.860415in}{1.493117in}}%
\pgfpathclose%
\pgfusepath{stroke,fill}%
\end{pgfscope}%
\begin{pgfscope}%
\pgfpathrectangle{\pgfqpoint{0.100000in}{0.212622in}}{\pgfqpoint{3.696000in}{3.696000in}}%
\pgfusepath{clip}%
\pgfsetbuttcap%
\pgfsetroundjoin%
\definecolor{currentfill}{rgb}{0.121569,0.466667,0.705882}%
\pgfsetfillcolor{currentfill}%
\pgfsetfillopacity{0.627433}%
\pgfsetlinewidth{1.003750pt}%
\definecolor{currentstroke}{rgb}{0.121569,0.466667,0.705882}%
\pgfsetstrokecolor{currentstroke}%
\pgfsetstrokeopacity{0.627433}%
\pgfsetdash{}{0pt}%
\pgfpathmoveto{\pgfqpoint{0.860415in}{1.493117in}}%
\pgfpathcurveto{\pgfqpoint{0.868651in}{1.493117in}}{\pgfqpoint{0.876551in}{1.496389in}}{\pgfqpoint{0.882375in}{1.502213in}}%
\pgfpathcurveto{\pgfqpoint{0.888199in}{1.508037in}}{\pgfqpoint{0.891471in}{1.515937in}}{\pgfqpoint{0.891471in}{1.524173in}}%
\pgfpathcurveto{\pgfqpoint{0.891471in}{1.532410in}}{\pgfqpoint{0.888199in}{1.540310in}}{\pgfqpoint{0.882375in}{1.546134in}}%
\pgfpathcurveto{\pgfqpoint{0.876551in}{1.551958in}}{\pgfqpoint{0.868651in}{1.555230in}}{\pgfqpoint{0.860415in}{1.555230in}}%
\pgfpathcurveto{\pgfqpoint{0.852178in}{1.555230in}}{\pgfqpoint{0.844278in}{1.551958in}}{\pgfqpoint{0.838454in}{1.546134in}}%
\pgfpathcurveto{\pgfqpoint{0.832631in}{1.540310in}}{\pgfqpoint{0.829358in}{1.532410in}}{\pgfqpoint{0.829358in}{1.524173in}}%
\pgfpathcurveto{\pgfqpoint{0.829358in}{1.515937in}}{\pgfqpoint{0.832631in}{1.508037in}}{\pgfqpoint{0.838454in}{1.502213in}}%
\pgfpathcurveto{\pgfqpoint{0.844278in}{1.496389in}}{\pgfqpoint{0.852178in}{1.493117in}}{\pgfqpoint{0.860415in}{1.493117in}}%
\pgfpathclose%
\pgfusepath{stroke,fill}%
\end{pgfscope}%
\begin{pgfscope}%
\pgfpathrectangle{\pgfqpoint{0.100000in}{0.212622in}}{\pgfqpoint{3.696000in}{3.696000in}}%
\pgfusepath{clip}%
\pgfsetbuttcap%
\pgfsetroundjoin%
\definecolor{currentfill}{rgb}{0.121569,0.466667,0.705882}%
\pgfsetfillcolor{currentfill}%
\pgfsetfillopacity{0.627433}%
\pgfsetlinewidth{1.003750pt}%
\definecolor{currentstroke}{rgb}{0.121569,0.466667,0.705882}%
\pgfsetstrokecolor{currentstroke}%
\pgfsetstrokeopacity{0.627433}%
\pgfsetdash{}{0pt}%
\pgfpathmoveto{\pgfqpoint{0.860415in}{1.493117in}}%
\pgfpathcurveto{\pgfqpoint{0.868651in}{1.493117in}}{\pgfqpoint{0.876551in}{1.496389in}}{\pgfqpoint{0.882375in}{1.502213in}}%
\pgfpathcurveto{\pgfqpoint{0.888199in}{1.508037in}}{\pgfqpoint{0.891471in}{1.515937in}}{\pgfqpoint{0.891471in}{1.524173in}}%
\pgfpathcurveto{\pgfqpoint{0.891471in}{1.532410in}}{\pgfqpoint{0.888199in}{1.540310in}}{\pgfqpoint{0.882375in}{1.546134in}}%
\pgfpathcurveto{\pgfqpoint{0.876551in}{1.551958in}}{\pgfqpoint{0.868651in}{1.555230in}}{\pgfqpoint{0.860415in}{1.555230in}}%
\pgfpathcurveto{\pgfqpoint{0.852178in}{1.555230in}}{\pgfqpoint{0.844278in}{1.551958in}}{\pgfqpoint{0.838454in}{1.546134in}}%
\pgfpathcurveto{\pgfqpoint{0.832631in}{1.540310in}}{\pgfqpoint{0.829358in}{1.532410in}}{\pgfqpoint{0.829358in}{1.524173in}}%
\pgfpathcurveto{\pgfqpoint{0.829358in}{1.515937in}}{\pgfqpoint{0.832631in}{1.508037in}}{\pgfqpoint{0.838454in}{1.502213in}}%
\pgfpathcurveto{\pgfqpoint{0.844278in}{1.496389in}}{\pgfqpoint{0.852178in}{1.493117in}}{\pgfqpoint{0.860415in}{1.493117in}}%
\pgfpathclose%
\pgfusepath{stroke,fill}%
\end{pgfscope}%
\begin{pgfscope}%
\pgfpathrectangle{\pgfqpoint{0.100000in}{0.212622in}}{\pgfqpoint{3.696000in}{3.696000in}}%
\pgfusepath{clip}%
\pgfsetbuttcap%
\pgfsetroundjoin%
\definecolor{currentfill}{rgb}{0.121569,0.466667,0.705882}%
\pgfsetfillcolor{currentfill}%
\pgfsetfillopacity{0.627433}%
\pgfsetlinewidth{1.003750pt}%
\definecolor{currentstroke}{rgb}{0.121569,0.466667,0.705882}%
\pgfsetstrokecolor{currentstroke}%
\pgfsetstrokeopacity{0.627433}%
\pgfsetdash{}{0pt}%
\pgfpathmoveto{\pgfqpoint{0.860415in}{1.493117in}}%
\pgfpathcurveto{\pgfqpoint{0.868651in}{1.493117in}}{\pgfqpoint{0.876551in}{1.496389in}}{\pgfqpoint{0.882375in}{1.502213in}}%
\pgfpathcurveto{\pgfqpoint{0.888199in}{1.508037in}}{\pgfqpoint{0.891471in}{1.515937in}}{\pgfqpoint{0.891471in}{1.524173in}}%
\pgfpathcurveto{\pgfqpoint{0.891471in}{1.532410in}}{\pgfqpoint{0.888199in}{1.540310in}}{\pgfqpoint{0.882375in}{1.546134in}}%
\pgfpathcurveto{\pgfqpoint{0.876551in}{1.551958in}}{\pgfqpoint{0.868651in}{1.555230in}}{\pgfqpoint{0.860415in}{1.555230in}}%
\pgfpathcurveto{\pgfqpoint{0.852178in}{1.555230in}}{\pgfqpoint{0.844278in}{1.551958in}}{\pgfqpoint{0.838454in}{1.546134in}}%
\pgfpathcurveto{\pgfqpoint{0.832631in}{1.540310in}}{\pgfqpoint{0.829358in}{1.532410in}}{\pgfqpoint{0.829358in}{1.524173in}}%
\pgfpathcurveto{\pgfqpoint{0.829358in}{1.515937in}}{\pgfqpoint{0.832631in}{1.508037in}}{\pgfqpoint{0.838454in}{1.502213in}}%
\pgfpathcurveto{\pgfqpoint{0.844278in}{1.496389in}}{\pgfqpoint{0.852178in}{1.493117in}}{\pgfqpoint{0.860415in}{1.493117in}}%
\pgfpathclose%
\pgfusepath{stroke,fill}%
\end{pgfscope}%
\begin{pgfscope}%
\pgfpathrectangle{\pgfqpoint{0.100000in}{0.212622in}}{\pgfqpoint{3.696000in}{3.696000in}}%
\pgfusepath{clip}%
\pgfsetbuttcap%
\pgfsetroundjoin%
\definecolor{currentfill}{rgb}{0.121569,0.466667,0.705882}%
\pgfsetfillcolor{currentfill}%
\pgfsetfillopacity{0.627433}%
\pgfsetlinewidth{1.003750pt}%
\definecolor{currentstroke}{rgb}{0.121569,0.466667,0.705882}%
\pgfsetstrokecolor{currentstroke}%
\pgfsetstrokeopacity{0.627433}%
\pgfsetdash{}{0pt}%
\pgfpathmoveto{\pgfqpoint{0.860415in}{1.493117in}}%
\pgfpathcurveto{\pgfqpoint{0.868651in}{1.493117in}}{\pgfqpoint{0.876551in}{1.496389in}}{\pgfqpoint{0.882375in}{1.502213in}}%
\pgfpathcurveto{\pgfqpoint{0.888199in}{1.508037in}}{\pgfqpoint{0.891471in}{1.515937in}}{\pgfqpoint{0.891471in}{1.524173in}}%
\pgfpathcurveto{\pgfqpoint{0.891471in}{1.532410in}}{\pgfqpoint{0.888199in}{1.540310in}}{\pgfqpoint{0.882375in}{1.546134in}}%
\pgfpathcurveto{\pgfqpoint{0.876551in}{1.551958in}}{\pgfqpoint{0.868651in}{1.555230in}}{\pgfqpoint{0.860415in}{1.555230in}}%
\pgfpathcurveto{\pgfqpoint{0.852178in}{1.555230in}}{\pgfqpoint{0.844278in}{1.551958in}}{\pgfqpoint{0.838454in}{1.546134in}}%
\pgfpathcurveto{\pgfqpoint{0.832631in}{1.540310in}}{\pgfqpoint{0.829358in}{1.532410in}}{\pgfqpoint{0.829358in}{1.524173in}}%
\pgfpathcurveto{\pgfqpoint{0.829358in}{1.515937in}}{\pgfqpoint{0.832631in}{1.508037in}}{\pgfqpoint{0.838454in}{1.502213in}}%
\pgfpathcurveto{\pgfqpoint{0.844278in}{1.496389in}}{\pgfqpoint{0.852178in}{1.493117in}}{\pgfqpoint{0.860415in}{1.493117in}}%
\pgfpathclose%
\pgfusepath{stroke,fill}%
\end{pgfscope}%
\begin{pgfscope}%
\pgfpathrectangle{\pgfqpoint{0.100000in}{0.212622in}}{\pgfqpoint{3.696000in}{3.696000in}}%
\pgfusepath{clip}%
\pgfsetbuttcap%
\pgfsetroundjoin%
\definecolor{currentfill}{rgb}{0.121569,0.466667,0.705882}%
\pgfsetfillcolor{currentfill}%
\pgfsetfillopacity{0.627433}%
\pgfsetlinewidth{1.003750pt}%
\definecolor{currentstroke}{rgb}{0.121569,0.466667,0.705882}%
\pgfsetstrokecolor{currentstroke}%
\pgfsetstrokeopacity{0.627433}%
\pgfsetdash{}{0pt}%
\pgfpathmoveto{\pgfqpoint{0.860415in}{1.493117in}}%
\pgfpathcurveto{\pgfqpoint{0.868651in}{1.493117in}}{\pgfqpoint{0.876551in}{1.496389in}}{\pgfqpoint{0.882375in}{1.502213in}}%
\pgfpathcurveto{\pgfqpoint{0.888199in}{1.508037in}}{\pgfqpoint{0.891471in}{1.515937in}}{\pgfqpoint{0.891471in}{1.524173in}}%
\pgfpathcurveto{\pgfqpoint{0.891471in}{1.532410in}}{\pgfqpoint{0.888199in}{1.540310in}}{\pgfqpoint{0.882375in}{1.546134in}}%
\pgfpathcurveto{\pgfqpoint{0.876551in}{1.551958in}}{\pgfqpoint{0.868651in}{1.555230in}}{\pgfqpoint{0.860415in}{1.555230in}}%
\pgfpathcurveto{\pgfqpoint{0.852178in}{1.555230in}}{\pgfqpoint{0.844278in}{1.551958in}}{\pgfqpoint{0.838454in}{1.546134in}}%
\pgfpathcurveto{\pgfqpoint{0.832631in}{1.540310in}}{\pgfqpoint{0.829358in}{1.532410in}}{\pgfqpoint{0.829358in}{1.524173in}}%
\pgfpathcurveto{\pgfqpoint{0.829358in}{1.515937in}}{\pgfqpoint{0.832631in}{1.508037in}}{\pgfqpoint{0.838454in}{1.502213in}}%
\pgfpathcurveto{\pgfqpoint{0.844278in}{1.496389in}}{\pgfqpoint{0.852178in}{1.493117in}}{\pgfqpoint{0.860415in}{1.493117in}}%
\pgfpathclose%
\pgfusepath{stroke,fill}%
\end{pgfscope}%
\begin{pgfscope}%
\pgfpathrectangle{\pgfqpoint{0.100000in}{0.212622in}}{\pgfqpoint{3.696000in}{3.696000in}}%
\pgfusepath{clip}%
\pgfsetbuttcap%
\pgfsetroundjoin%
\definecolor{currentfill}{rgb}{0.121569,0.466667,0.705882}%
\pgfsetfillcolor{currentfill}%
\pgfsetfillopacity{0.627433}%
\pgfsetlinewidth{1.003750pt}%
\definecolor{currentstroke}{rgb}{0.121569,0.466667,0.705882}%
\pgfsetstrokecolor{currentstroke}%
\pgfsetstrokeopacity{0.627433}%
\pgfsetdash{}{0pt}%
\pgfpathmoveto{\pgfqpoint{0.860415in}{1.493117in}}%
\pgfpathcurveto{\pgfqpoint{0.868651in}{1.493117in}}{\pgfqpoint{0.876551in}{1.496389in}}{\pgfqpoint{0.882375in}{1.502213in}}%
\pgfpathcurveto{\pgfqpoint{0.888199in}{1.508037in}}{\pgfqpoint{0.891471in}{1.515937in}}{\pgfqpoint{0.891471in}{1.524173in}}%
\pgfpathcurveto{\pgfqpoint{0.891471in}{1.532410in}}{\pgfqpoint{0.888199in}{1.540310in}}{\pgfqpoint{0.882375in}{1.546134in}}%
\pgfpathcurveto{\pgfqpoint{0.876551in}{1.551958in}}{\pgfqpoint{0.868651in}{1.555230in}}{\pgfqpoint{0.860415in}{1.555230in}}%
\pgfpathcurveto{\pgfqpoint{0.852178in}{1.555230in}}{\pgfqpoint{0.844278in}{1.551958in}}{\pgfqpoint{0.838454in}{1.546134in}}%
\pgfpathcurveto{\pgfqpoint{0.832631in}{1.540310in}}{\pgfqpoint{0.829358in}{1.532410in}}{\pgfqpoint{0.829358in}{1.524173in}}%
\pgfpathcurveto{\pgfqpoint{0.829358in}{1.515937in}}{\pgfqpoint{0.832631in}{1.508037in}}{\pgfqpoint{0.838454in}{1.502213in}}%
\pgfpathcurveto{\pgfqpoint{0.844278in}{1.496389in}}{\pgfqpoint{0.852178in}{1.493117in}}{\pgfqpoint{0.860415in}{1.493117in}}%
\pgfpathclose%
\pgfusepath{stroke,fill}%
\end{pgfscope}%
\begin{pgfscope}%
\pgfpathrectangle{\pgfqpoint{0.100000in}{0.212622in}}{\pgfqpoint{3.696000in}{3.696000in}}%
\pgfusepath{clip}%
\pgfsetbuttcap%
\pgfsetroundjoin%
\definecolor{currentfill}{rgb}{0.121569,0.466667,0.705882}%
\pgfsetfillcolor{currentfill}%
\pgfsetfillopacity{0.627433}%
\pgfsetlinewidth{1.003750pt}%
\definecolor{currentstroke}{rgb}{0.121569,0.466667,0.705882}%
\pgfsetstrokecolor{currentstroke}%
\pgfsetstrokeopacity{0.627433}%
\pgfsetdash{}{0pt}%
\pgfpathmoveto{\pgfqpoint{0.860415in}{1.493117in}}%
\pgfpathcurveto{\pgfqpoint{0.868651in}{1.493117in}}{\pgfqpoint{0.876551in}{1.496389in}}{\pgfqpoint{0.882375in}{1.502213in}}%
\pgfpathcurveto{\pgfqpoint{0.888199in}{1.508037in}}{\pgfqpoint{0.891471in}{1.515937in}}{\pgfqpoint{0.891471in}{1.524173in}}%
\pgfpathcurveto{\pgfqpoint{0.891471in}{1.532410in}}{\pgfqpoint{0.888199in}{1.540310in}}{\pgfqpoint{0.882375in}{1.546134in}}%
\pgfpathcurveto{\pgfqpoint{0.876551in}{1.551958in}}{\pgfqpoint{0.868651in}{1.555230in}}{\pgfqpoint{0.860415in}{1.555230in}}%
\pgfpathcurveto{\pgfqpoint{0.852178in}{1.555230in}}{\pgfqpoint{0.844278in}{1.551958in}}{\pgfqpoint{0.838454in}{1.546134in}}%
\pgfpathcurveto{\pgfqpoint{0.832631in}{1.540310in}}{\pgfqpoint{0.829358in}{1.532410in}}{\pgfqpoint{0.829358in}{1.524173in}}%
\pgfpathcurveto{\pgfqpoint{0.829358in}{1.515937in}}{\pgfqpoint{0.832631in}{1.508037in}}{\pgfqpoint{0.838454in}{1.502213in}}%
\pgfpathcurveto{\pgfqpoint{0.844278in}{1.496389in}}{\pgfqpoint{0.852178in}{1.493117in}}{\pgfqpoint{0.860415in}{1.493117in}}%
\pgfpathclose%
\pgfusepath{stroke,fill}%
\end{pgfscope}%
\begin{pgfscope}%
\pgfpathrectangle{\pgfqpoint{0.100000in}{0.212622in}}{\pgfqpoint{3.696000in}{3.696000in}}%
\pgfusepath{clip}%
\pgfsetbuttcap%
\pgfsetroundjoin%
\definecolor{currentfill}{rgb}{0.121569,0.466667,0.705882}%
\pgfsetfillcolor{currentfill}%
\pgfsetfillopacity{0.627433}%
\pgfsetlinewidth{1.003750pt}%
\definecolor{currentstroke}{rgb}{0.121569,0.466667,0.705882}%
\pgfsetstrokecolor{currentstroke}%
\pgfsetstrokeopacity{0.627433}%
\pgfsetdash{}{0pt}%
\pgfpathmoveto{\pgfqpoint{0.860415in}{1.493117in}}%
\pgfpathcurveto{\pgfqpoint{0.868651in}{1.493117in}}{\pgfqpoint{0.876551in}{1.496389in}}{\pgfqpoint{0.882375in}{1.502213in}}%
\pgfpathcurveto{\pgfqpoint{0.888199in}{1.508037in}}{\pgfqpoint{0.891471in}{1.515937in}}{\pgfqpoint{0.891471in}{1.524173in}}%
\pgfpathcurveto{\pgfqpoint{0.891471in}{1.532410in}}{\pgfqpoint{0.888199in}{1.540310in}}{\pgfqpoint{0.882375in}{1.546134in}}%
\pgfpathcurveto{\pgfqpoint{0.876551in}{1.551958in}}{\pgfqpoint{0.868651in}{1.555230in}}{\pgfqpoint{0.860415in}{1.555230in}}%
\pgfpathcurveto{\pgfqpoint{0.852178in}{1.555230in}}{\pgfqpoint{0.844278in}{1.551958in}}{\pgfqpoint{0.838454in}{1.546134in}}%
\pgfpathcurveto{\pgfqpoint{0.832631in}{1.540310in}}{\pgfqpoint{0.829358in}{1.532410in}}{\pgfqpoint{0.829358in}{1.524173in}}%
\pgfpathcurveto{\pgfqpoint{0.829358in}{1.515937in}}{\pgfqpoint{0.832631in}{1.508037in}}{\pgfqpoint{0.838454in}{1.502213in}}%
\pgfpathcurveto{\pgfqpoint{0.844278in}{1.496389in}}{\pgfqpoint{0.852178in}{1.493117in}}{\pgfqpoint{0.860415in}{1.493117in}}%
\pgfpathclose%
\pgfusepath{stroke,fill}%
\end{pgfscope}%
\begin{pgfscope}%
\pgfpathrectangle{\pgfqpoint{0.100000in}{0.212622in}}{\pgfqpoint{3.696000in}{3.696000in}}%
\pgfusepath{clip}%
\pgfsetbuttcap%
\pgfsetroundjoin%
\definecolor{currentfill}{rgb}{0.121569,0.466667,0.705882}%
\pgfsetfillcolor{currentfill}%
\pgfsetfillopacity{0.627433}%
\pgfsetlinewidth{1.003750pt}%
\definecolor{currentstroke}{rgb}{0.121569,0.466667,0.705882}%
\pgfsetstrokecolor{currentstroke}%
\pgfsetstrokeopacity{0.627433}%
\pgfsetdash{}{0pt}%
\pgfpathmoveto{\pgfqpoint{0.860415in}{1.493117in}}%
\pgfpathcurveto{\pgfqpoint{0.868651in}{1.493117in}}{\pgfqpoint{0.876551in}{1.496389in}}{\pgfqpoint{0.882375in}{1.502213in}}%
\pgfpathcurveto{\pgfqpoint{0.888199in}{1.508037in}}{\pgfqpoint{0.891471in}{1.515937in}}{\pgfqpoint{0.891471in}{1.524173in}}%
\pgfpathcurveto{\pgfqpoint{0.891471in}{1.532410in}}{\pgfqpoint{0.888199in}{1.540310in}}{\pgfqpoint{0.882375in}{1.546134in}}%
\pgfpathcurveto{\pgfqpoint{0.876551in}{1.551958in}}{\pgfqpoint{0.868651in}{1.555230in}}{\pgfqpoint{0.860415in}{1.555230in}}%
\pgfpathcurveto{\pgfqpoint{0.852178in}{1.555230in}}{\pgfqpoint{0.844278in}{1.551958in}}{\pgfqpoint{0.838454in}{1.546134in}}%
\pgfpathcurveto{\pgfqpoint{0.832631in}{1.540310in}}{\pgfqpoint{0.829358in}{1.532410in}}{\pgfqpoint{0.829358in}{1.524173in}}%
\pgfpathcurveto{\pgfqpoint{0.829358in}{1.515937in}}{\pgfqpoint{0.832631in}{1.508037in}}{\pgfqpoint{0.838454in}{1.502213in}}%
\pgfpathcurveto{\pgfqpoint{0.844278in}{1.496389in}}{\pgfqpoint{0.852178in}{1.493117in}}{\pgfqpoint{0.860415in}{1.493117in}}%
\pgfpathclose%
\pgfusepath{stroke,fill}%
\end{pgfscope}%
\begin{pgfscope}%
\pgfpathrectangle{\pgfqpoint{0.100000in}{0.212622in}}{\pgfqpoint{3.696000in}{3.696000in}}%
\pgfusepath{clip}%
\pgfsetbuttcap%
\pgfsetroundjoin%
\definecolor{currentfill}{rgb}{0.121569,0.466667,0.705882}%
\pgfsetfillcolor{currentfill}%
\pgfsetfillopacity{0.627433}%
\pgfsetlinewidth{1.003750pt}%
\definecolor{currentstroke}{rgb}{0.121569,0.466667,0.705882}%
\pgfsetstrokecolor{currentstroke}%
\pgfsetstrokeopacity{0.627433}%
\pgfsetdash{}{0pt}%
\pgfpathmoveto{\pgfqpoint{0.860415in}{1.493117in}}%
\pgfpathcurveto{\pgfqpoint{0.868651in}{1.493117in}}{\pgfqpoint{0.876551in}{1.496389in}}{\pgfqpoint{0.882375in}{1.502213in}}%
\pgfpathcurveto{\pgfqpoint{0.888199in}{1.508037in}}{\pgfqpoint{0.891471in}{1.515937in}}{\pgfqpoint{0.891471in}{1.524173in}}%
\pgfpathcurveto{\pgfqpoint{0.891471in}{1.532410in}}{\pgfqpoint{0.888199in}{1.540310in}}{\pgfqpoint{0.882375in}{1.546134in}}%
\pgfpathcurveto{\pgfqpoint{0.876551in}{1.551958in}}{\pgfqpoint{0.868651in}{1.555230in}}{\pgfqpoint{0.860415in}{1.555230in}}%
\pgfpathcurveto{\pgfqpoint{0.852178in}{1.555230in}}{\pgfqpoint{0.844278in}{1.551958in}}{\pgfqpoint{0.838454in}{1.546134in}}%
\pgfpathcurveto{\pgfqpoint{0.832631in}{1.540310in}}{\pgfqpoint{0.829358in}{1.532410in}}{\pgfqpoint{0.829358in}{1.524173in}}%
\pgfpathcurveto{\pgfqpoint{0.829358in}{1.515937in}}{\pgfqpoint{0.832631in}{1.508037in}}{\pgfqpoint{0.838454in}{1.502213in}}%
\pgfpathcurveto{\pgfqpoint{0.844278in}{1.496389in}}{\pgfqpoint{0.852178in}{1.493117in}}{\pgfqpoint{0.860415in}{1.493117in}}%
\pgfpathclose%
\pgfusepath{stroke,fill}%
\end{pgfscope}%
\begin{pgfscope}%
\pgfpathrectangle{\pgfqpoint{0.100000in}{0.212622in}}{\pgfqpoint{3.696000in}{3.696000in}}%
\pgfusepath{clip}%
\pgfsetbuttcap%
\pgfsetroundjoin%
\definecolor{currentfill}{rgb}{0.121569,0.466667,0.705882}%
\pgfsetfillcolor{currentfill}%
\pgfsetfillopacity{0.627433}%
\pgfsetlinewidth{1.003750pt}%
\definecolor{currentstroke}{rgb}{0.121569,0.466667,0.705882}%
\pgfsetstrokecolor{currentstroke}%
\pgfsetstrokeopacity{0.627433}%
\pgfsetdash{}{0pt}%
\pgfpathmoveto{\pgfqpoint{0.860415in}{1.493117in}}%
\pgfpathcurveto{\pgfqpoint{0.868651in}{1.493117in}}{\pgfqpoint{0.876551in}{1.496389in}}{\pgfqpoint{0.882375in}{1.502213in}}%
\pgfpathcurveto{\pgfqpoint{0.888199in}{1.508037in}}{\pgfqpoint{0.891471in}{1.515937in}}{\pgfqpoint{0.891471in}{1.524173in}}%
\pgfpathcurveto{\pgfqpoint{0.891471in}{1.532410in}}{\pgfqpoint{0.888199in}{1.540310in}}{\pgfqpoint{0.882375in}{1.546134in}}%
\pgfpathcurveto{\pgfqpoint{0.876551in}{1.551958in}}{\pgfqpoint{0.868651in}{1.555230in}}{\pgfqpoint{0.860415in}{1.555230in}}%
\pgfpathcurveto{\pgfqpoint{0.852178in}{1.555230in}}{\pgfqpoint{0.844278in}{1.551958in}}{\pgfqpoint{0.838454in}{1.546134in}}%
\pgfpathcurveto{\pgfqpoint{0.832631in}{1.540310in}}{\pgfqpoint{0.829358in}{1.532410in}}{\pgfqpoint{0.829358in}{1.524173in}}%
\pgfpathcurveto{\pgfqpoint{0.829358in}{1.515937in}}{\pgfqpoint{0.832631in}{1.508037in}}{\pgfqpoint{0.838454in}{1.502213in}}%
\pgfpathcurveto{\pgfqpoint{0.844278in}{1.496389in}}{\pgfqpoint{0.852178in}{1.493117in}}{\pgfqpoint{0.860415in}{1.493117in}}%
\pgfpathclose%
\pgfusepath{stroke,fill}%
\end{pgfscope}%
\begin{pgfscope}%
\pgfpathrectangle{\pgfqpoint{0.100000in}{0.212622in}}{\pgfqpoint{3.696000in}{3.696000in}}%
\pgfusepath{clip}%
\pgfsetbuttcap%
\pgfsetroundjoin%
\definecolor{currentfill}{rgb}{0.121569,0.466667,0.705882}%
\pgfsetfillcolor{currentfill}%
\pgfsetfillopacity{0.627433}%
\pgfsetlinewidth{1.003750pt}%
\definecolor{currentstroke}{rgb}{0.121569,0.466667,0.705882}%
\pgfsetstrokecolor{currentstroke}%
\pgfsetstrokeopacity{0.627433}%
\pgfsetdash{}{0pt}%
\pgfpathmoveto{\pgfqpoint{0.860415in}{1.493117in}}%
\pgfpathcurveto{\pgfqpoint{0.868651in}{1.493117in}}{\pgfqpoint{0.876551in}{1.496389in}}{\pgfqpoint{0.882375in}{1.502213in}}%
\pgfpathcurveto{\pgfqpoint{0.888199in}{1.508037in}}{\pgfqpoint{0.891471in}{1.515937in}}{\pgfqpoint{0.891471in}{1.524173in}}%
\pgfpathcurveto{\pgfqpoint{0.891471in}{1.532410in}}{\pgfqpoint{0.888199in}{1.540310in}}{\pgfqpoint{0.882375in}{1.546134in}}%
\pgfpathcurveto{\pgfqpoint{0.876551in}{1.551958in}}{\pgfqpoint{0.868651in}{1.555230in}}{\pgfqpoint{0.860415in}{1.555230in}}%
\pgfpathcurveto{\pgfqpoint{0.852178in}{1.555230in}}{\pgfqpoint{0.844278in}{1.551958in}}{\pgfqpoint{0.838454in}{1.546134in}}%
\pgfpathcurveto{\pgfqpoint{0.832631in}{1.540310in}}{\pgfqpoint{0.829358in}{1.532410in}}{\pgfqpoint{0.829358in}{1.524173in}}%
\pgfpathcurveto{\pgfqpoint{0.829358in}{1.515937in}}{\pgfqpoint{0.832631in}{1.508037in}}{\pgfqpoint{0.838454in}{1.502213in}}%
\pgfpathcurveto{\pgfqpoint{0.844278in}{1.496389in}}{\pgfqpoint{0.852178in}{1.493117in}}{\pgfqpoint{0.860415in}{1.493117in}}%
\pgfpathclose%
\pgfusepath{stroke,fill}%
\end{pgfscope}%
\begin{pgfscope}%
\pgfpathrectangle{\pgfqpoint{0.100000in}{0.212622in}}{\pgfqpoint{3.696000in}{3.696000in}}%
\pgfusepath{clip}%
\pgfsetbuttcap%
\pgfsetroundjoin%
\definecolor{currentfill}{rgb}{0.121569,0.466667,0.705882}%
\pgfsetfillcolor{currentfill}%
\pgfsetfillopacity{0.627433}%
\pgfsetlinewidth{1.003750pt}%
\definecolor{currentstroke}{rgb}{0.121569,0.466667,0.705882}%
\pgfsetstrokecolor{currentstroke}%
\pgfsetstrokeopacity{0.627433}%
\pgfsetdash{}{0pt}%
\pgfpathmoveto{\pgfqpoint{0.860415in}{1.493117in}}%
\pgfpathcurveto{\pgfqpoint{0.868651in}{1.493117in}}{\pgfqpoint{0.876551in}{1.496389in}}{\pgfqpoint{0.882375in}{1.502213in}}%
\pgfpathcurveto{\pgfqpoint{0.888199in}{1.508037in}}{\pgfqpoint{0.891471in}{1.515937in}}{\pgfqpoint{0.891471in}{1.524173in}}%
\pgfpathcurveto{\pgfqpoint{0.891471in}{1.532410in}}{\pgfqpoint{0.888199in}{1.540310in}}{\pgfqpoint{0.882375in}{1.546134in}}%
\pgfpathcurveto{\pgfqpoint{0.876551in}{1.551958in}}{\pgfqpoint{0.868651in}{1.555230in}}{\pgfqpoint{0.860415in}{1.555230in}}%
\pgfpathcurveto{\pgfqpoint{0.852178in}{1.555230in}}{\pgfqpoint{0.844278in}{1.551958in}}{\pgfqpoint{0.838454in}{1.546134in}}%
\pgfpathcurveto{\pgfqpoint{0.832631in}{1.540310in}}{\pgfqpoint{0.829358in}{1.532410in}}{\pgfqpoint{0.829358in}{1.524173in}}%
\pgfpathcurveto{\pgfqpoint{0.829358in}{1.515937in}}{\pgfqpoint{0.832631in}{1.508037in}}{\pgfqpoint{0.838454in}{1.502213in}}%
\pgfpathcurveto{\pgfqpoint{0.844278in}{1.496389in}}{\pgfqpoint{0.852178in}{1.493117in}}{\pgfqpoint{0.860415in}{1.493117in}}%
\pgfpathclose%
\pgfusepath{stroke,fill}%
\end{pgfscope}%
\begin{pgfscope}%
\pgfpathrectangle{\pgfqpoint{0.100000in}{0.212622in}}{\pgfqpoint{3.696000in}{3.696000in}}%
\pgfusepath{clip}%
\pgfsetbuttcap%
\pgfsetroundjoin%
\definecolor{currentfill}{rgb}{0.121569,0.466667,0.705882}%
\pgfsetfillcolor{currentfill}%
\pgfsetfillopacity{0.627433}%
\pgfsetlinewidth{1.003750pt}%
\definecolor{currentstroke}{rgb}{0.121569,0.466667,0.705882}%
\pgfsetstrokecolor{currentstroke}%
\pgfsetstrokeopacity{0.627433}%
\pgfsetdash{}{0pt}%
\pgfpathmoveto{\pgfqpoint{0.860415in}{1.493117in}}%
\pgfpathcurveto{\pgfqpoint{0.868651in}{1.493117in}}{\pgfqpoint{0.876551in}{1.496389in}}{\pgfqpoint{0.882375in}{1.502213in}}%
\pgfpathcurveto{\pgfqpoint{0.888199in}{1.508037in}}{\pgfqpoint{0.891471in}{1.515937in}}{\pgfqpoint{0.891471in}{1.524173in}}%
\pgfpathcurveto{\pgfqpoint{0.891471in}{1.532410in}}{\pgfqpoint{0.888199in}{1.540310in}}{\pgfqpoint{0.882375in}{1.546134in}}%
\pgfpathcurveto{\pgfqpoint{0.876551in}{1.551958in}}{\pgfqpoint{0.868651in}{1.555230in}}{\pgfqpoint{0.860415in}{1.555230in}}%
\pgfpathcurveto{\pgfqpoint{0.852178in}{1.555230in}}{\pgfqpoint{0.844278in}{1.551958in}}{\pgfqpoint{0.838454in}{1.546134in}}%
\pgfpathcurveto{\pgfqpoint{0.832631in}{1.540310in}}{\pgfqpoint{0.829358in}{1.532410in}}{\pgfqpoint{0.829358in}{1.524173in}}%
\pgfpathcurveto{\pgfqpoint{0.829358in}{1.515937in}}{\pgfqpoint{0.832631in}{1.508037in}}{\pgfqpoint{0.838454in}{1.502213in}}%
\pgfpathcurveto{\pgfqpoint{0.844278in}{1.496389in}}{\pgfqpoint{0.852178in}{1.493117in}}{\pgfqpoint{0.860415in}{1.493117in}}%
\pgfpathclose%
\pgfusepath{stroke,fill}%
\end{pgfscope}%
\begin{pgfscope}%
\pgfpathrectangle{\pgfqpoint{0.100000in}{0.212622in}}{\pgfqpoint{3.696000in}{3.696000in}}%
\pgfusepath{clip}%
\pgfsetbuttcap%
\pgfsetroundjoin%
\definecolor{currentfill}{rgb}{0.121569,0.466667,0.705882}%
\pgfsetfillcolor{currentfill}%
\pgfsetfillopacity{0.627433}%
\pgfsetlinewidth{1.003750pt}%
\definecolor{currentstroke}{rgb}{0.121569,0.466667,0.705882}%
\pgfsetstrokecolor{currentstroke}%
\pgfsetstrokeopacity{0.627433}%
\pgfsetdash{}{0pt}%
\pgfpathmoveto{\pgfqpoint{0.860415in}{1.493117in}}%
\pgfpathcurveto{\pgfqpoint{0.868651in}{1.493117in}}{\pgfqpoint{0.876551in}{1.496389in}}{\pgfqpoint{0.882375in}{1.502213in}}%
\pgfpathcurveto{\pgfqpoint{0.888199in}{1.508037in}}{\pgfqpoint{0.891471in}{1.515937in}}{\pgfqpoint{0.891471in}{1.524173in}}%
\pgfpathcurveto{\pgfqpoint{0.891471in}{1.532410in}}{\pgfqpoint{0.888199in}{1.540310in}}{\pgfqpoint{0.882375in}{1.546134in}}%
\pgfpathcurveto{\pgfqpoint{0.876551in}{1.551958in}}{\pgfqpoint{0.868651in}{1.555230in}}{\pgfqpoint{0.860415in}{1.555230in}}%
\pgfpathcurveto{\pgfqpoint{0.852178in}{1.555230in}}{\pgfqpoint{0.844278in}{1.551958in}}{\pgfqpoint{0.838454in}{1.546134in}}%
\pgfpathcurveto{\pgfqpoint{0.832631in}{1.540310in}}{\pgfqpoint{0.829358in}{1.532410in}}{\pgfqpoint{0.829358in}{1.524173in}}%
\pgfpathcurveto{\pgfqpoint{0.829358in}{1.515937in}}{\pgfqpoint{0.832631in}{1.508037in}}{\pgfqpoint{0.838454in}{1.502213in}}%
\pgfpathcurveto{\pgfqpoint{0.844278in}{1.496389in}}{\pgfqpoint{0.852178in}{1.493117in}}{\pgfqpoint{0.860415in}{1.493117in}}%
\pgfpathclose%
\pgfusepath{stroke,fill}%
\end{pgfscope}%
\begin{pgfscope}%
\pgfpathrectangle{\pgfqpoint{0.100000in}{0.212622in}}{\pgfqpoint{3.696000in}{3.696000in}}%
\pgfusepath{clip}%
\pgfsetbuttcap%
\pgfsetroundjoin%
\definecolor{currentfill}{rgb}{0.121569,0.466667,0.705882}%
\pgfsetfillcolor{currentfill}%
\pgfsetfillopacity{0.627433}%
\pgfsetlinewidth{1.003750pt}%
\definecolor{currentstroke}{rgb}{0.121569,0.466667,0.705882}%
\pgfsetstrokecolor{currentstroke}%
\pgfsetstrokeopacity{0.627433}%
\pgfsetdash{}{0pt}%
\pgfpathmoveto{\pgfqpoint{0.860415in}{1.493117in}}%
\pgfpathcurveto{\pgfqpoint{0.868651in}{1.493117in}}{\pgfqpoint{0.876551in}{1.496389in}}{\pgfqpoint{0.882375in}{1.502213in}}%
\pgfpathcurveto{\pgfqpoint{0.888199in}{1.508037in}}{\pgfqpoint{0.891471in}{1.515937in}}{\pgfqpoint{0.891471in}{1.524173in}}%
\pgfpathcurveto{\pgfqpoint{0.891471in}{1.532410in}}{\pgfqpoint{0.888199in}{1.540310in}}{\pgfqpoint{0.882375in}{1.546134in}}%
\pgfpathcurveto{\pgfqpoint{0.876551in}{1.551958in}}{\pgfqpoint{0.868651in}{1.555230in}}{\pgfqpoint{0.860415in}{1.555230in}}%
\pgfpathcurveto{\pgfqpoint{0.852178in}{1.555230in}}{\pgfqpoint{0.844278in}{1.551958in}}{\pgfqpoint{0.838454in}{1.546134in}}%
\pgfpathcurveto{\pgfqpoint{0.832631in}{1.540310in}}{\pgfqpoint{0.829358in}{1.532410in}}{\pgfqpoint{0.829358in}{1.524173in}}%
\pgfpathcurveto{\pgfqpoint{0.829358in}{1.515937in}}{\pgfqpoint{0.832631in}{1.508037in}}{\pgfqpoint{0.838454in}{1.502213in}}%
\pgfpathcurveto{\pgfqpoint{0.844278in}{1.496389in}}{\pgfqpoint{0.852178in}{1.493117in}}{\pgfqpoint{0.860415in}{1.493117in}}%
\pgfpathclose%
\pgfusepath{stroke,fill}%
\end{pgfscope}%
\begin{pgfscope}%
\pgfpathrectangle{\pgfqpoint{0.100000in}{0.212622in}}{\pgfqpoint{3.696000in}{3.696000in}}%
\pgfusepath{clip}%
\pgfsetbuttcap%
\pgfsetroundjoin%
\definecolor{currentfill}{rgb}{0.121569,0.466667,0.705882}%
\pgfsetfillcolor{currentfill}%
\pgfsetfillopacity{0.627433}%
\pgfsetlinewidth{1.003750pt}%
\definecolor{currentstroke}{rgb}{0.121569,0.466667,0.705882}%
\pgfsetstrokecolor{currentstroke}%
\pgfsetstrokeopacity{0.627433}%
\pgfsetdash{}{0pt}%
\pgfpathmoveto{\pgfqpoint{0.860415in}{1.493117in}}%
\pgfpathcurveto{\pgfqpoint{0.868651in}{1.493117in}}{\pgfqpoint{0.876551in}{1.496389in}}{\pgfqpoint{0.882375in}{1.502213in}}%
\pgfpathcurveto{\pgfqpoint{0.888199in}{1.508037in}}{\pgfqpoint{0.891471in}{1.515937in}}{\pgfqpoint{0.891471in}{1.524173in}}%
\pgfpathcurveto{\pgfqpoint{0.891471in}{1.532410in}}{\pgfqpoint{0.888199in}{1.540310in}}{\pgfqpoint{0.882375in}{1.546134in}}%
\pgfpathcurveto{\pgfqpoint{0.876551in}{1.551958in}}{\pgfqpoint{0.868651in}{1.555230in}}{\pgfqpoint{0.860415in}{1.555230in}}%
\pgfpathcurveto{\pgfqpoint{0.852178in}{1.555230in}}{\pgfqpoint{0.844278in}{1.551958in}}{\pgfqpoint{0.838454in}{1.546134in}}%
\pgfpathcurveto{\pgfqpoint{0.832631in}{1.540310in}}{\pgfqpoint{0.829358in}{1.532410in}}{\pgfqpoint{0.829358in}{1.524173in}}%
\pgfpathcurveto{\pgfqpoint{0.829358in}{1.515937in}}{\pgfqpoint{0.832631in}{1.508037in}}{\pgfqpoint{0.838454in}{1.502213in}}%
\pgfpathcurveto{\pgfqpoint{0.844278in}{1.496389in}}{\pgfqpoint{0.852178in}{1.493117in}}{\pgfqpoint{0.860415in}{1.493117in}}%
\pgfpathclose%
\pgfusepath{stroke,fill}%
\end{pgfscope}%
\begin{pgfscope}%
\pgfpathrectangle{\pgfqpoint{0.100000in}{0.212622in}}{\pgfqpoint{3.696000in}{3.696000in}}%
\pgfusepath{clip}%
\pgfsetbuttcap%
\pgfsetroundjoin%
\definecolor{currentfill}{rgb}{0.121569,0.466667,0.705882}%
\pgfsetfillcolor{currentfill}%
\pgfsetfillopacity{0.627433}%
\pgfsetlinewidth{1.003750pt}%
\definecolor{currentstroke}{rgb}{0.121569,0.466667,0.705882}%
\pgfsetstrokecolor{currentstroke}%
\pgfsetstrokeopacity{0.627433}%
\pgfsetdash{}{0pt}%
\pgfpathmoveto{\pgfqpoint{0.860415in}{1.493117in}}%
\pgfpathcurveto{\pgfqpoint{0.868651in}{1.493117in}}{\pgfqpoint{0.876551in}{1.496389in}}{\pgfqpoint{0.882375in}{1.502213in}}%
\pgfpathcurveto{\pgfqpoint{0.888199in}{1.508037in}}{\pgfqpoint{0.891471in}{1.515937in}}{\pgfqpoint{0.891471in}{1.524173in}}%
\pgfpathcurveto{\pgfqpoint{0.891471in}{1.532410in}}{\pgfqpoint{0.888199in}{1.540310in}}{\pgfqpoint{0.882375in}{1.546134in}}%
\pgfpathcurveto{\pgfqpoint{0.876551in}{1.551958in}}{\pgfqpoint{0.868651in}{1.555230in}}{\pgfqpoint{0.860415in}{1.555230in}}%
\pgfpathcurveto{\pgfqpoint{0.852178in}{1.555230in}}{\pgfqpoint{0.844278in}{1.551958in}}{\pgfqpoint{0.838454in}{1.546134in}}%
\pgfpathcurveto{\pgfqpoint{0.832631in}{1.540310in}}{\pgfqpoint{0.829358in}{1.532410in}}{\pgfqpoint{0.829358in}{1.524173in}}%
\pgfpathcurveto{\pgfqpoint{0.829358in}{1.515937in}}{\pgfqpoint{0.832631in}{1.508037in}}{\pgfqpoint{0.838454in}{1.502213in}}%
\pgfpathcurveto{\pgfqpoint{0.844278in}{1.496389in}}{\pgfqpoint{0.852178in}{1.493117in}}{\pgfqpoint{0.860415in}{1.493117in}}%
\pgfpathclose%
\pgfusepath{stroke,fill}%
\end{pgfscope}%
\begin{pgfscope}%
\pgfpathrectangle{\pgfqpoint{0.100000in}{0.212622in}}{\pgfqpoint{3.696000in}{3.696000in}}%
\pgfusepath{clip}%
\pgfsetbuttcap%
\pgfsetroundjoin%
\definecolor{currentfill}{rgb}{0.121569,0.466667,0.705882}%
\pgfsetfillcolor{currentfill}%
\pgfsetfillopacity{0.627433}%
\pgfsetlinewidth{1.003750pt}%
\definecolor{currentstroke}{rgb}{0.121569,0.466667,0.705882}%
\pgfsetstrokecolor{currentstroke}%
\pgfsetstrokeopacity{0.627433}%
\pgfsetdash{}{0pt}%
\pgfpathmoveto{\pgfqpoint{0.860415in}{1.493117in}}%
\pgfpathcurveto{\pgfqpoint{0.868651in}{1.493117in}}{\pgfqpoint{0.876551in}{1.496389in}}{\pgfqpoint{0.882375in}{1.502213in}}%
\pgfpathcurveto{\pgfqpoint{0.888199in}{1.508037in}}{\pgfqpoint{0.891471in}{1.515937in}}{\pgfqpoint{0.891471in}{1.524173in}}%
\pgfpathcurveto{\pgfqpoint{0.891471in}{1.532410in}}{\pgfqpoint{0.888199in}{1.540310in}}{\pgfqpoint{0.882375in}{1.546134in}}%
\pgfpathcurveto{\pgfqpoint{0.876551in}{1.551958in}}{\pgfqpoint{0.868651in}{1.555230in}}{\pgfqpoint{0.860415in}{1.555230in}}%
\pgfpathcurveto{\pgfqpoint{0.852178in}{1.555230in}}{\pgfqpoint{0.844278in}{1.551958in}}{\pgfqpoint{0.838454in}{1.546134in}}%
\pgfpathcurveto{\pgfqpoint{0.832631in}{1.540310in}}{\pgfqpoint{0.829358in}{1.532410in}}{\pgfqpoint{0.829358in}{1.524173in}}%
\pgfpathcurveto{\pgfqpoint{0.829358in}{1.515937in}}{\pgfqpoint{0.832631in}{1.508037in}}{\pgfqpoint{0.838454in}{1.502213in}}%
\pgfpathcurveto{\pgfqpoint{0.844278in}{1.496389in}}{\pgfqpoint{0.852178in}{1.493117in}}{\pgfqpoint{0.860415in}{1.493117in}}%
\pgfpathclose%
\pgfusepath{stroke,fill}%
\end{pgfscope}%
\begin{pgfscope}%
\pgfpathrectangle{\pgfqpoint{0.100000in}{0.212622in}}{\pgfqpoint{3.696000in}{3.696000in}}%
\pgfusepath{clip}%
\pgfsetbuttcap%
\pgfsetroundjoin%
\definecolor{currentfill}{rgb}{0.121569,0.466667,0.705882}%
\pgfsetfillcolor{currentfill}%
\pgfsetfillopacity{0.627433}%
\pgfsetlinewidth{1.003750pt}%
\definecolor{currentstroke}{rgb}{0.121569,0.466667,0.705882}%
\pgfsetstrokecolor{currentstroke}%
\pgfsetstrokeopacity{0.627433}%
\pgfsetdash{}{0pt}%
\pgfpathmoveto{\pgfqpoint{0.860415in}{1.493117in}}%
\pgfpathcurveto{\pgfqpoint{0.868651in}{1.493117in}}{\pgfqpoint{0.876551in}{1.496389in}}{\pgfqpoint{0.882375in}{1.502213in}}%
\pgfpathcurveto{\pgfqpoint{0.888199in}{1.508037in}}{\pgfqpoint{0.891471in}{1.515937in}}{\pgfqpoint{0.891471in}{1.524173in}}%
\pgfpathcurveto{\pgfqpoint{0.891471in}{1.532410in}}{\pgfqpoint{0.888199in}{1.540310in}}{\pgfqpoint{0.882375in}{1.546134in}}%
\pgfpathcurveto{\pgfqpoint{0.876551in}{1.551958in}}{\pgfqpoint{0.868651in}{1.555230in}}{\pgfqpoint{0.860415in}{1.555230in}}%
\pgfpathcurveto{\pgfqpoint{0.852178in}{1.555230in}}{\pgfqpoint{0.844278in}{1.551958in}}{\pgfqpoint{0.838454in}{1.546134in}}%
\pgfpathcurveto{\pgfqpoint{0.832631in}{1.540310in}}{\pgfqpoint{0.829358in}{1.532410in}}{\pgfqpoint{0.829358in}{1.524173in}}%
\pgfpathcurveto{\pgfqpoint{0.829358in}{1.515937in}}{\pgfqpoint{0.832631in}{1.508037in}}{\pgfqpoint{0.838454in}{1.502213in}}%
\pgfpathcurveto{\pgfqpoint{0.844278in}{1.496389in}}{\pgfqpoint{0.852178in}{1.493117in}}{\pgfqpoint{0.860415in}{1.493117in}}%
\pgfpathclose%
\pgfusepath{stroke,fill}%
\end{pgfscope}%
\begin{pgfscope}%
\pgfpathrectangle{\pgfqpoint{0.100000in}{0.212622in}}{\pgfqpoint{3.696000in}{3.696000in}}%
\pgfusepath{clip}%
\pgfsetbuttcap%
\pgfsetroundjoin%
\definecolor{currentfill}{rgb}{0.121569,0.466667,0.705882}%
\pgfsetfillcolor{currentfill}%
\pgfsetfillopacity{0.627433}%
\pgfsetlinewidth{1.003750pt}%
\definecolor{currentstroke}{rgb}{0.121569,0.466667,0.705882}%
\pgfsetstrokecolor{currentstroke}%
\pgfsetstrokeopacity{0.627433}%
\pgfsetdash{}{0pt}%
\pgfpathmoveto{\pgfqpoint{0.860415in}{1.493117in}}%
\pgfpathcurveto{\pgfqpoint{0.868651in}{1.493117in}}{\pgfqpoint{0.876551in}{1.496389in}}{\pgfqpoint{0.882375in}{1.502213in}}%
\pgfpathcurveto{\pgfqpoint{0.888199in}{1.508037in}}{\pgfqpoint{0.891471in}{1.515937in}}{\pgfqpoint{0.891471in}{1.524173in}}%
\pgfpathcurveto{\pgfqpoint{0.891471in}{1.532410in}}{\pgfqpoint{0.888199in}{1.540310in}}{\pgfqpoint{0.882375in}{1.546134in}}%
\pgfpathcurveto{\pgfqpoint{0.876551in}{1.551958in}}{\pgfqpoint{0.868651in}{1.555230in}}{\pgfqpoint{0.860415in}{1.555230in}}%
\pgfpathcurveto{\pgfqpoint{0.852178in}{1.555230in}}{\pgfqpoint{0.844278in}{1.551958in}}{\pgfqpoint{0.838454in}{1.546134in}}%
\pgfpathcurveto{\pgfqpoint{0.832631in}{1.540310in}}{\pgfqpoint{0.829358in}{1.532410in}}{\pgfqpoint{0.829358in}{1.524173in}}%
\pgfpathcurveto{\pgfqpoint{0.829358in}{1.515937in}}{\pgfqpoint{0.832631in}{1.508037in}}{\pgfqpoint{0.838454in}{1.502213in}}%
\pgfpathcurveto{\pgfqpoint{0.844278in}{1.496389in}}{\pgfqpoint{0.852178in}{1.493117in}}{\pgfqpoint{0.860415in}{1.493117in}}%
\pgfpathclose%
\pgfusepath{stroke,fill}%
\end{pgfscope}%
\begin{pgfscope}%
\pgfpathrectangle{\pgfqpoint{0.100000in}{0.212622in}}{\pgfqpoint{3.696000in}{3.696000in}}%
\pgfusepath{clip}%
\pgfsetbuttcap%
\pgfsetroundjoin%
\definecolor{currentfill}{rgb}{0.121569,0.466667,0.705882}%
\pgfsetfillcolor{currentfill}%
\pgfsetfillopacity{0.627433}%
\pgfsetlinewidth{1.003750pt}%
\definecolor{currentstroke}{rgb}{0.121569,0.466667,0.705882}%
\pgfsetstrokecolor{currentstroke}%
\pgfsetstrokeopacity{0.627433}%
\pgfsetdash{}{0pt}%
\pgfpathmoveto{\pgfqpoint{0.860415in}{1.493117in}}%
\pgfpathcurveto{\pgfqpoint{0.868651in}{1.493117in}}{\pgfqpoint{0.876551in}{1.496389in}}{\pgfqpoint{0.882375in}{1.502213in}}%
\pgfpathcurveto{\pgfqpoint{0.888199in}{1.508037in}}{\pgfqpoint{0.891471in}{1.515937in}}{\pgfqpoint{0.891471in}{1.524173in}}%
\pgfpathcurveto{\pgfqpoint{0.891471in}{1.532410in}}{\pgfqpoint{0.888199in}{1.540310in}}{\pgfqpoint{0.882375in}{1.546134in}}%
\pgfpathcurveto{\pgfqpoint{0.876551in}{1.551958in}}{\pgfqpoint{0.868651in}{1.555230in}}{\pgfqpoint{0.860415in}{1.555230in}}%
\pgfpathcurveto{\pgfqpoint{0.852178in}{1.555230in}}{\pgfqpoint{0.844278in}{1.551958in}}{\pgfqpoint{0.838454in}{1.546134in}}%
\pgfpathcurveto{\pgfqpoint{0.832631in}{1.540310in}}{\pgfqpoint{0.829358in}{1.532410in}}{\pgfqpoint{0.829358in}{1.524173in}}%
\pgfpathcurveto{\pgfqpoint{0.829358in}{1.515937in}}{\pgfqpoint{0.832631in}{1.508037in}}{\pgfqpoint{0.838454in}{1.502213in}}%
\pgfpathcurveto{\pgfqpoint{0.844278in}{1.496389in}}{\pgfqpoint{0.852178in}{1.493117in}}{\pgfqpoint{0.860415in}{1.493117in}}%
\pgfpathclose%
\pgfusepath{stroke,fill}%
\end{pgfscope}%
\begin{pgfscope}%
\pgfpathrectangle{\pgfqpoint{0.100000in}{0.212622in}}{\pgfqpoint{3.696000in}{3.696000in}}%
\pgfusepath{clip}%
\pgfsetbuttcap%
\pgfsetroundjoin%
\definecolor{currentfill}{rgb}{0.121569,0.466667,0.705882}%
\pgfsetfillcolor{currentfill}%
\pgfsetfillopacity{0.627433}%
\pgfsetlinewidth{1.003750pt}%
\definecolor{currentstroke}{rgb}{0.121569,0.466667,0.705882}%
\pgfsetstrokecolor{currentstroke}%
\pgfsetstrokeopacity{0.627433}%
\pgfsetdash{}{0pt}%
\pgfpathmoveto{\pgfqpoint{0.860415in}{1.493117in}}%
\pgfpathcurveto{\pgfqpoint{0.868651in}{1.493117in}}{\pgfqpoint{0.876551in}{1.496389in}}{\pgfqpoint{0.882375in}{1.502213in}}%
\pgfpathcurveto{\pgfqpoint{0.888199in}{1.508037in}}{\pgfqpoint{0.891471in}{1.515937in}}{\pgfqpoint{0.891471in}{1.524173in}}%
\pgfpathcurveto{\pgfqpoint{0.891471in}{1.532410in}}{\pgfqpoint{0.888199in}{1.540310in}}{\pgfqpoint{0.882375in}{1.546134in}}%
\pgfpathcurveto{\pgfqpoint{0.876551in}{1.551958in}}{\pgfqpoint{0.868651in}{1.555230in}}{\pgfqpoint{0.860415in}{1.555230in}}%
\pgfpathcurveto{\pgfqpoint{0.852178in}{1.555230in}}{\pgfqpoint{0.844278in}{1.551958in}}{\pgfqpoint{0.838454in}{1.546134in}}%
\pgfpathcurveto{\pgfqpoint{0.832631in}{1.540310in}}{\pgfqpoint{0.829358in}{1.532410in}}{\pgfqpoint{0.829358in}{1.524173in}}%
\pgfpathcurveto{\pgfqpoint{0.829358in}{1.515937in}}{\pgfqpoint{0.832631in}{1.508037in}}{\pgfqpoint{0.838454in}{1.502213in}}%
\pgfpathcurveto{\pgfqpoint{0.844278in}{1.496389in}}{\pgfqpoint{0.852178in}{1.493117in}}{\pgfqpoint{0.860415in}{1.493117in}}%
\pgfpathclose%
\pgfusepath{stroke,fill}%
\end{pgfscope}%
\begin{pgfscope}%
\pgfpathrectangle{\pgfqpoint{0.100000in}{0.212622in}}{\pgfqpoint{3.696000in}{3.696000in}}%
\pgfusepath{clip}%
\pgfsetbuttcap%
\pgfsetroundjoin%
\definecolor{currentfill}{rgb}{0.121569,0.466667,0.705882}%
\pgfsetfillcolor{currentfill}%
\pgfsetfillopacity{0.627433}%
\pgfsetlinewidth{1.003750pt}%
\definecolor{currentstroke}{rgb}{0.121569,0.466667,0.705882}%
\pgfsetstrokecolor{currentstroke}%
\pgfsetstrokeopacity{0.627433}%
\pgfsetdash{}{0pt}%
\pgfpathmoveto{\pgfqpoint{0.860415in}{1.493117in}}%
\pgfpathcurveto{\pgfqpoint{0.868651in}{1.493117in}}{\pgfqpoint{0.876551in}{1.496389in}}{\pgfqpoint{0.882375in}{1.502213in}}%
\pgfpathcurveto{\pgfqpoint{0.888199in}{1.508037in}}{\pgfqpoint{0.891471in}{1.515937in}}{\pgfqpoint{0.891471in}{1.524173in}}%
\pgfpathcurveto{\pgfqpoint{0.891471in}{1.532410in}}{\pgfqpoint{0.888199in}{1.540310in}}{\pgfqpoint{0.882375in}{1.546134in}}%
\pgfpathcurveto{\pgfqpoint{0.876551in}{1.551958in}}{\pgfqpoint{0.868651in}{1.555230in}}{\pgfqpoint{0.860415in}{1.555230in}}%
\pgfpathcurveto{\pgfqpoint{0.852178in}{1.555230in}}{\pgfqpoint{0.844278in}{1.551958in}}{\pgfqpoint{0.838454in}{1.546134in}}%
\pgfpathcurveto{\pgfqpoint{0.832631in}{1.540310in}}{\pgfqpoint{0.829358in}{1.532410in}}{\pgfqpoint{0.829358in}{1.524173in}}%
\pgfpathcurveto{\pgfqpoint{0.829358in}{1.515937in}}{\pgfqpoint{0.832631in}{1.508037in}}{\pgfqpoint{0.838454in}{1.502213in}}%
\pgfpathcurveto{\pgfqpoint{0.844278in}{1.496389in}}{\pgfqpoint{0.852178in}{1.493117in}}{\pgfqpoint{0.860415in}{1.493117in}}%
\pgfpathclose%
\pgfusepath{stroke,fill}%
\end{pgfscope}%
\begin{pgfscope}%
\pgfpathrectangle{\pgfqpoint{0.100000in}{0.212622in}}{\pgfqpoint{3.696000in}{3.696000in}}%
\pgfusepath{clip}%
\pgfsetbuttcap%
\pgfsetroundjoin%
\definecolor{currentfill}{rgb}{0.121569,0.466667,0.705882}%
\pgfsetfillcolor{currentfill}%
\pgfsetfillopacity{0.627433}%
\pgfsetlinewidth{1.003750pt}%
\definecolor{currentstroke}{rgb}{0.121569,0.466667,0.705882}%
\pgfsetstrokecolor{currentstroke}%
\pgfsetstrokeopacity{0.627433}%
\pgfsetdash{}{0pt}%
\pgfpathmoveto{\pgfqpoint{0.860415in}{1.493117in}}%
\pgfpathcurveto{\pgfqpoint{0.868651in}{1.493117in}}{\pgfqpoint{0.876551in}{1.496389in}}{\pgfqpoint{0.882375in}{1.502213in}}%
\pgfpathcurveto{\pgfqpoint{0.888199in}{1.508037in}}{\pgfqpoint{0.891471in}{1.515937in}}{\pgfqpoint{0.891471in}{1.524173in}}%
\pgfpathcurveto{\pgfqpoint{0.891471in}{1.532410in}}{\pgfqpoint{0.888199in}{1.540310in}}{\pgfqpoint{0.882375in}{1.546134in}}%
\pgfpathcurveto{\pgfqpoint{0.876551in}{1.551958in}}{\pgfqpoint{0.868651in}{1.555230in}}{\pgfqpoint{0.860415in}{1.555230in}}%
\pgfpathcurveto{\pgfqpoint{0.852178in}{1.555230in}}{\pgfqpoint{0.844278in}{1.551958in}}{\pgfqpoint{0.838454in}{1.546134in}}%
\pgfpathcurveto{\pgfqpoint{0.832631in}{1.540310in}}{\pgfqpoint{0.829358in}{1.532410in}}{\pgfqpoint{0.829358in}{1.524173in}}%
\pgfpathcurveto{\pgfqpoint{0.829358in}{1.515937in}}{\pgfqpoint{0.832631in}{1.508037in}}{\pgfqpoint{0.838454in}{1.502213in}}%
\pgfpathcurveto{\pgfqpoint{0.844278in}{1.496389in}}{\pgfqpoint{0.852178in}{1.493117in}}{\pgfqpoint{0.860415in}{1.493117in}}%
\pgfpathclose%
\pgfusepath{stroke,fill}%
\end{pgfscope}%
\begin{pgfscope}%
\pgfpathrectangle{\pgfqpoint{0.100000in}{0.212622in}}{\pgfqpoint{3.696000in}{3.696000in}}%
\pgfusepath{clip}%
\pgfsetbuttcap%
\pgfsetroundjoin%
\definecolor{currentfill}{rgb}{0.121569,0.466667,0.705882}%
\pgfsetfillcolor{currentfill}%
\pgfsetfillopacity{0.627433}%
\pgfsetlinewidth{1.003750pt}%
\definecolor{currentstroke}{rgb}{0.121569,0.466667,0.705882}%
\pgfsetstrokecolor{currentstroke}%
\pgfsetstrokeopacity{0.627433}%
\pgfsetdash{}{0pt}%
\pgfpathmoveto{\pgfqpoint{0.860415in}{1.493117in}}%
\pgfpathcurveto{\pgfqpoint{0.868651in}{1.493117in}}{\pgfqpoint{0.876551in}{1.496389in}}{\pgfqpoint{0.882375in}{1.502213in}}%
\pgfpathcurveto{\pgfqpoint{0.888199in}{1.508037in}}{\pgfqpoint{0.891471in}{1.515937in}}{\pgfqpoint{0.891471in}{1.524173in}}%
\pgfpathcurveto{\pgfqpoint{0.891471in}{1.532410in}}{\pgfqpoint{0.888199in}{1.540310in}}{\pgfqpoint{0.882375in}{1.546134in}}%
\pgfpathcurveto{\pgfqpoint{0.876551in}{1.551958in}}{\pgfqpoint{0.868651in}{1.555230in}}{\pgfqpoint{0.860415in}{1.555230in}}%
\pgfpathcurveto{\pgfqpoint{0.852178in}{1.555230in}}{\pgfqpoint{0.844278in}{1.551958in}}{\pgfqpoint{0.838454in}{1.546134in}}%
\pgfpathcurveto{\pgfqpoint{0.832631in}{1.540310in}}{\pgfqpoint{0.829358in}{1.532410in}}{\pgfqpoint{0.829358in}{1.524173in}}%
\pgfpathcurveto{\pgfqpoint{0.829358in}{1.515937in}}{\pgfqpoint{0.832631in}{1.508037in}}{\pgfqpoint{0.838454in}{1.502213in}}%
\pgfpathcurveto{\pgfqpoint{0.844278in}{1.496389in}}{\pgfqpoint{0.852178in}{1.493117in}}{\pgfqpoint{0.860415in}{1.493117in}}%
\pgfpathclose%
\pgfusepath{stroke,fill}%
\end{pgfscope}%
\begin{pgfscope}%
\pgfpathrectangle{\pgfqpoint{0.100000in}{0.212622in}}{\pgfqpoint{3.696000in}{3.696000in}}%
\pgfusepath{clip}%
\pgfsetbuttcap%
\pgfsetroundjoin%
\definecolor{currentfill}{rgb}{0.121569,0.466667,0.705882}%
\pgfsetfillcolor{currentfill}%
\pgfsetfillopacity{0.627433}%
\pgfsetlinewidth{1.003750pt}%
\definecolor{currentstroke}{rgb}{0.121569,0.466667,0.705882}%
\pgfsetstrokecolor{currentstroke}%
\pgfsetstrokeopacity{0.627433}%
\pgfsetdash{}{0pt}%
\pgfpathmoveto{\pgfqpoint{0.860415in}{1.493117in}}%
\pgfpathcurveto{\pgfqpoint{0.868651in}{1.493117in}}{\pgfqpoint{0.876551in}{1.496389in}}{\pgfqpoint{0.882375in}{1.502213in}}%
\pgfpathcurveto{\pgfqpoint{0.888199in}{1.508037in}}{\pgfqpoint{0.891471in}{1.515937in}}{\pgfqpoint{0.891471in}{1.524173in}}%
\pgfpathcurveto{\pgfqpoint{0.891471in}{1.532410in}}{\pgfqpoint{0.888199in}{1.540310in}}{\pgfqpoint{0.882375in}{1.546134in}}%
\pgfpathcurveto{\pgfqpoint{0.876551in}{1.551958in}}{\pgfqpoint{0.868651in}{1.555230in}}{\pgfqpoint{0.860415in}{1.555230in}}%
\pgfpathcurveto{\pgfqpoint{0.852178in}{1.555230in}}{\pgfqpoint{0.844278in}{1.551958in}}{\pgfqpoint{0.838454in}{1.546134in}}%
\pgfpathcurveto{\pgfqpoint{0.832631in}{1.540310in}}{\pgfqpoint{0.829358in}{1.532410in}}{\pgfqpoint{0.829358in}{1.524173in}}%
\pgfpathcurveto{\pgfqpoint{0.829358in}{1.515937in}}{\pgfqpoint{0.832631in}{1.508037in}}{\pgfqpoint{0.838454in}{1.502213in}}%
\pgfpathcurveto{\pgfqpoint{0.844278in}{1.496389in}}{\pgfqpoint{0.852178in}{1.493117in}}{\pgfqpoint{0.860415in}{1.493117in}}%
\pgfpathclose%
\pgfusepath{stroke,fill}%
\end{pgfscope}%
\begin{pgfscope}%
\pgfpathrectangle{\pgfqpoint{0.100000in}{0.212622in}}{\pgfqpoint{3.696000in}{3.696000in}}%
\pgfusepath{clip}%
\pgfsetbuttcap%
\pgfsetroundjoin%
\definecolor{currentfill}{rgb}{0.121569,0.466667,0.705882}%
\pgfsetfillcolor{currentfill}%
\pgfsetfillopacity{0.627433}%
\pgfsetlinewidth{1.003750pt}%
\definecolor{currentstroke}{rgb}{0.121569,0.466667,0.705882}%
\pgfsetstrokecolor{currentstroke}%
\pgfsetstrokeopacity{0.627433}%
\pgfsetdash{}{0pt}%
\pgfpathmoveto{\pgfqpoint{0.860415in}{1.493117in}}%
\pgfpathcurveto{\pgfqpoint{0.868651in}{1.493117in}}{\pgfqpoint{0.876551in}{1.496389in}}{\pgfqpoint{0.882375in}{1.502213in}}%
\pgfpathcurveto{\pgfqpoint{0.888199in}{1.508037in}}{\pgfqpoint{0.891471in}{1.515937in}}{\pgfqpoint{0.891471in}{1.524173in}}%
\pgfpathcurveto{\pgfqpoint{0.891471in}{1.532410in}}{\pgfqpoint{0.888199in}{1.540310in}}{\pgfqpoint{0.882375in}{1.546134in}}%
\pgfpathcurveto{\pgfqpoint{0.876551in}{1.551958in}}{\pgfqpoint{0.868651in}{1.555230in}}{\pgfqpoint{0.860415in}{1.555230in}}%
\pgfpathcurveto{\pgfqpoint{0.852178in}{1.555230in}}{\pgfqpoint{0.844278in}{1.551958in}}{\pgfqpoint{0.838454in}{1.546134in}}%
\pgfpathcurveto{\pgfqpoint{0.832631in}{1.540310in}}{\pgfqpoint{0.829358in}{1.532410in}}{\pgfqpoint{0.829358in}{1.524173in}}%
\pgfpathcurveto{\pgfqpoint{0.829358in}{1.515937in}}{\pgfqpoint{0.832631in}{1.508037in}}{\pgfqpoint{0.838454in}{1.502213in}}%
\pgfpathcurveto{\pgfqpoint{0.844278in}{1.496389in}}{\pgfqpoint{0.852178in}{1.493117in}}{\pgfqpoint{0.860415in}{1.493117in}}%
\pgfpathclose%
\pgfusepath{stroke,fill}%
\end{pgfscope}%
\begin{pgfscope}%
\pgfpathrectangle{\pgfqpoint{0.100000in}{0.212622in}}{\pgfqpoint{3.696000in}{3.696000in}}%
\pgfusepath{clip}%
\pgfsetbuttcap%
\pgfsetroundjoin%
\definecolor{currentfill}{rgb}{0.121569,0.466667,0.705882}%
\pgfsetfillcolor{currentfill}%
\pgfsetfillopacity{0.627433}%
\pgfsetlinewidth{1.003750pt}%
\definecolor{currentstroke}{rgb}{0.121569,0.466667,0.705882}%
\pgfsetstrokecolor{currentstroke}%
\pgfsetstrokeopacity{0.627433}%
\pgfsetdash{}{0pt}%
\pgfpathmoveto{\pgfqpoint{0.860415in}{1.493117in}}%
\pgfpathcurveto{\pgfqpoint{0.868651in}{1.493117in}}{\pgfqpoint{0.876551in}{1.496389in}}{\pgfqpoint{0.882375in}{1.502213in}}%
\pgfpathcurveto{\pgfqpoint{0.888199in}{1.508037in}}{\pgfqpoint{0.891471in}{1.515937in}}{\pgfqpoint{0.891471in}{1.524173in}}%
\pgfpathcurveto{\pgfqpoint{0.891471in}{1.532410in}}{\pgfqpoint{0.888199in}{1.540310in}}{\pgfqpoint{0.882375in}{1.546134in}}%
\pgfpathcurveto{\pgfqpoint{0.876551in}{1.551958in}}{\pgfqpoint{0.868651in}{1.555230in}}{\pgfqpoint{0.860415in}{1.555230in}}%
\pgfpathcurveto{\pgfqpoint{0.852178in}{1.555230in}}{\pgfqpoint{0.844278in}{1.551958in}}{\pgfqpoint{0.838454in}{1.546134in}}%
\pgfpathcurveto{\pgfqpoint{0.832631in}{1.540310in}}{\pgfqpoint{0.829358in}{1.532410in}}{\pgfqpoint{0.829358in}{1.524173in}}%
\pgfpathcurveto{\pgfqpoint{0.829358in}{1.515937in}}{\pgfqpoint{0.832631in}{1.508037in}}{\pgfqpoint{0.838454in}{1.502213in}}%
\pgfpathcurveto{\pgfqpoint{0.844278in}{1.496389in}}{\pgfqpoint{0.852178in}{1.493117in}}{\pgfqpoint{0.860415in}{1.493117in}}%
\pgfpathclose%
\pgfusepath{stroke,fill}%
\end{pgfscope}%
\begin{pgfscope}%
\pgfpathrectangle{\pgfqpoint{0.100000in}{0.212622in}}{\pgfqpoint{3.696000in}{3.696000in}}%
\pgfusepath{clip}%
\pgfsetbuttcap%
\pgfsetroundjoin%
\definecolor{currentfill}{rgb}{0.121569,0.466667,0.705882}%
\pgfsetfillcolor{currentfill}%
\pgfsetfillopacity{0.627433}%
\pgfsetlinewidth{1.003750pt}%
\definecolor{currentstroke}{rgb}{0.121569,0.466667,0.705882}%
\pgfsetstrokecolor{currentstroke}%
\pgfsetstrokeopacity{0.627433}%
\pgfsetdash{}{0pt}%
\pgfpathmoveto{\pgfqpoint{0.860415in}{1.493117in}}%
\pgfpathcurveto{\pgfqpoint{0.868651in}{1.493117in}}{\pgfqpoint{0.876551in}{1.496389in}}{\pgfqpoint{0.882375in}{1.502213in}}%
\pgfpathcurveto{\pgfqpoint{0.888199in}{1.508037in}}{\pgfqpoint{0.891471in}{1.515937in}}{\pgfqpoint{0.891471in}{1.524173in}}%
\pgfpathcurveto{\pgfqpoint{0.891471in}{1.532410in}}{\pgfqpoint{0.888199in}{1.540310in}}{\pgfqpoint{0.882375in}{1.546134in}}%
\pgfpathcurveto{\pgfqpoint{0.876551in}{1.551958in}}{\pgfqpoint{0.868651in}{1.555230in}}{\pgfqpoint{0.860415in}{1.555230in}}%
\pgfpathcurveto{\pgfqpoint{0.852178in}{1.555230in}}{\pgfqpoint{0.844278in}{1.551958in}}{\pgfqpoint{0.838454in}{1.546134in}}%
\pgfpathcurveto{\pgfqpoint{0.832631in}{1.540310in}}{\pgfqpoint{0.829358in}{1.532410in}}{\pgfqpoint{0.829358in}{1.524173in}}%
\pgfpathcurveto{\pgfqpoint{0.829358in}{1.515937in}}{\pgfqpoint{0.832631in}{1.508037in}}{\pgfqpoint{0.838454in}{1.502213in}}%
\pgfpathcurveto{\pgfqpoint{0.844278in}{1.496389in}}{\pgfqpoint{0.852178in}{1.493117in}}{\pgfqpoint{0.860415in}{1.493117in}}%
\pgfpathclose%
\pgfusepath{stroke,fill}%
\end{pgfscope}%
\begin{pgfscope}%
\pgfpathrectangle{\pgfqpoint{0.100000in}{0.212622in}}{\pgfqpoint{3.696000in}{3.696000in}}%
\pgfusepath{clip}%
\pgfsetbuttcap%
\pgfsetroundjoin%
\definecolor{currentfill}{rgb}{0.121569,0.466667,0.705882}%
\pgfsetfillcolor{currentfill}%
\pgfsetfillopacity{0.627433}%
\pgfsetlinewidth{1.003750pt}%
\definecolor{currentstroke}{rgb}{0.121569,0.466667,0.705882}%
\pgfsetstrokecolor{currentstroke}%
\pgfsetstrokeopacity{0.627433}%
\pgfsetdash{}{0pt}%
\pgfpathmoveto{\pgfqpoint{0.860415in}{1.493117in}}%
\pgfpathcurveto{\pgfqpoint{0.868651in}{1.493117in}}{\pgfqpoint{0.876551in}{1.496389in}}{\pgfqpoint{0.882375in}{1.502213in}}%
\pgfpathcurveto{\pgfqpoint{0.888199in}{1.508037in}}{\pgfqpoint{0.891471in}{1.515937in}}{\pgfqpoint{0.891471in}{1.524173in}}%
\pgfpathcurveto{\pgfqpoint{0.891471in}{1.532410in}}{\pgfqpoint{0.888199in}{1.540310in}}{\pgfqpoint{0.882375in}{1.546134in}}%
\pgfpathcurveto{\pgfqpoint{0.876551in}{1.551958in}}{\pgfqpoint{0.868651in}{1.555230in}}{\pgfqpoint{0.860415in}{1.555230in}}%
\pgfpathcurveto{\pgfqpoint{0.852178in}{1.555230in}}{\pgfqpoint{0.844278in}{1.551958in}}{\pgfqpoint{0.838454in}{1.546134in}}%
\pgfpathcurveto{\pgfqpoint{0.832631in}{1.540310in}}{\pgfqpoint{0.829358in}{1.532410in}}{\pgfqpoint{0.829358in}{1.524173in}}%
\pgfpathcurveto{\pgfqpoint{0.829358in}{1.515937in}}{\pgfqpoint{0.832631in}{1.508037in}}{\pgfqpoint{0.838454in}{1.502213in}}%
\pgfpathcurveto{\pgfqpoint{0.844278in}{1.496389in}}{\pgfqpoint{0.852178in}{1.493117in}}{\pgfqpoint{0.860415in}{1.493117in}}%
\pgfpathclose%
\pgfusepath{stroke,fill}%
\end{pgfscope}%
\begin{pgfscope}%
\pgfpathrectangle{\pgfqpoint{0.100000in}{0.212622in}}{\pgfqpoint{3.696000in}{3.696000in}}%
\pgfusepath{clip}%
\pgfsetbuttcap%
\pgfsetroundjoin%
\definecolor{currentfill}{rgb}{0.121569,0.466667,0.705882}%
\pgfsetfillcolor{currentfill}%
\pgfsetfillopacity{0.627433}%
\pgfsetlinewidth{1.003750pt}%
\definecolor{currentstroke}{rgb}{0.121569,0.466667,0.705882}%
\pgfsetstrokecolor{currentstroke}%
\pgfsetstrokeopacity{0.627433}%
\pgfsetdash{}{0pt}%
\pgfpathmoveto{\pgfqpoint{0.860415in}{1.493117in}}%
\pgfpathcurveto{\pgfqpoint{0.868651in}{1.493117in}}{\pgfqpoint{0.876551in}{1.496389in}}{\pgfqpoint{0.882375in}{1.502213in}}%
\pgfpathcurveto{\pgfqpoint{0.888199in}{1.508037in}}{\pgfqpoint{0.891471in}{1.515937in}}{\pgfqpoint{0.891471in}{1.524173in}}%
\pgfpathcurveto{\pgfqpoint{0.891471in}{1.532410in}}{\pgfqpoint{0.888199in}{1.540310in}}{\pgfqpoint{0.882375in}{1.546134in}}%
\pgfpathcurveto{\pgfqpoint{0.876551in}{1.551958in}}{\pgfqpoint{0.868651in}{1.555230in}}{\pgfqpoint{0.860415in}{1.555230in}}%
\pgfpathcurveto{\pgfqpoint{0.852178in}{1.555230in}}{\pgfqpoint{0.844278in}{1.551958in}}{\pgfqpoint{0.838454in}{1.546134in}}%
\pgfpathcurveto{\pgfqpoint{0.832631in}{1.540310in}}{\pgfqpoint{0.829358in}{1.532410in}}{\pgfqpoint{0.829358in}{1.524173in}}%
\pgfpathcurveto{\pgfqpoint{0.829358in}{1.515937in}}{\pgfqpoint{0.832631in}{1.508037in}}{\pgfqpoint{0.838454in}{1.502213in}}%
\pgfpathcurveto{\pgfqpoint{0.844278in}{1.496389in}}{\pgfqpoint{0.852178in}{1.493117in}}{\pgfqpoint{0.860415in}{1.493117in}}%
\pgfpathclose%
\pgfusepath{stroke,fill}%
\end{pgfscope}%
\begin{pgfscope}%
\pgfpathrectangle{\pgfqpoint{0.100000in}{0.212622in}}{\pgfqpoint{3.696000in}{3.696000in}}%
\pgfusepath{clip}%
\pgfsetbuttcap%
\pgfsetroundjoin%
\definecolor{currentfill}{rgb}{0.121569,0.466667,0.705882}%
\pgfsetfillcolor{currentfill}%
\pgfsetfillopacity{0.627433}%
\pgfsetlinewidth{1.003750pt}%
\definecolor{currentstroke}{rgb}{0.121569,0.466667,0.705882}%
\pgfsetstrokecolor{currentstroke}%
\pgfsetstrokeopacity{0.627433}%
\pgfsetdash{}{0pt}%
\pgfpathmoveto{\pgfqpoint{0.860415in}{1.493117in}}%
\pgfpathcurveto{\pgfqpoint{0.868651in}{1.493117in}}{\pgfqpoint{0.876551in}{1.496389in}}{\pgfqpoint{0.882375in}{1.502213in}}%
\pgfpathcurveto{\pgfqpoint{0.888199in}{1.508037in}}{\pgfqpoint{0.891471in}{1.515937in}}{\pgfqpoint{0.891471in}{1.524173in}}%
\pgfpathcurveto{\pgfqpoint{0.891471in}{1.532410in}}{\pgfqpoint{0.888199in}{1.540310in}}{\pgfqpoint{0.882375in}{1.546134in}}%
\pgfpathcurveto{\pgfqpoint{0.876551in}{1.551958in}}{\pgfqpoint{0.868651in}{1.555230in}}{\pgfqpoint{0.860415in}{1.555230in}}%
\pgfpathcurveto{\pgfqpoint{0.852178in}{1.555230in}}{\pgfqpoint{0.844278in}{1.551958in}}{\pgfqpoint{0.838454in}{1.546134in}}%
\pgfpathcurveto{\pgfqpoint{0.832631in}{1.540310in}}{\pgfqpoint{0.829358in}{1.532410in}}{\pgfqpoint{0.829358in}{1.524173in}}%
\pgfpathcurveto{\pgfqpoint{0.829358in}{1.515937in}}{\pgfqpoint{0.832631in}{1.508037in}}{\pgfqpoint{0.838454in}{1.502213in}}%
\pgfpathcurveto{\pgfqpoint{0.844278in}{1.496389in}}{\pgfqpoint{0.852178in}{1.493117in}}{\pgfqpoint{0.860415in}{1.493117in}}%
\pgfpathclose%
\pgfusepath{stroke,fill}%
\end{pgfscope}%
\begin{pgfscope}%
\pgfpathrectangle{\pgfqpoint{0.100000in}{0.212622in}}{\pgfqpoint{3.696000in}{3.696000in}}%
\pgfusepath{clip}%
\pgfsetbuttcap%
\pgfsetroundjoin%
\definecolor{currentfill}{rgb}{0.121569,0.466667,0.705882}%
\pgfsetfillcolor{currentfill}%
\pgfsetfillopacity{0.627433}%
\pgfsetlinewidth{1.003750pt}%
\definecolor{currentstroke}{rgb}{0.121569,0.466667,0.705882}%
\pgfsetstrokecolor{currentstroke}%
\pgfsetstrokeopacity{0.627433}%
\pgfsetdash{}{0pt}%
\pgfpathmoveto{\pgfqpoint{0.860415in}{1.493117in}}%
\pgfpathcurveto{\pgfqpoint{0.868651in}{1.493117in}}{\pgfqpoint{0.876551in}{1.496389in}}{\pgfqpoint{0.882375in}{1.502213in}}%
\pgfpathcurveto{\pgfqpoint{0.888199in}{1.508037in}}{\pgfqpoint{0.891471in}{1.515937in}}{\pgfqpoint{0.891471in}{1.524173in}}%
\pgfpathcurveto{\pgfqpoint{0.891471in}{1.532410in}}{\pgfqpoint{0.888199in}{1.540310in}}{\pgfqpoint{0.882375in}{1.546134in}}%
\pgfpathcurveto{\pgfqpoint{0.876551in}{1.551958in}}{\pgfqpoint{0.868651in}{1.555230in}}{\pgfqpoint{0.860415in}{1.555230in}}%
\pgfpathcurveto{\pgfqpoint{0.852178in}{1.555230in}}{\pgfqpoint{0.844278in}{1.551958in}}{\pgfqpoint{0.838454in}{1.546134in}}%
\pgfpathcurveto{\pgfqpoint{0.832631in}{1.540310in}}{\pgfqpoint{0.829358in}{1.532410in}}{\pgfqpoint{0.829358in}{1.524173in}}%
\pgfpathcurveto{\pgfqpoint{0.829358in}{1.515937in}}{\pgfqpoint{0.832631in}{1.508037in}}{\pgfqpoint{0.838454in}{1.502213in}}%
\pgfpathcurveto{\pgfqpoint{0.844278in}{1.496389in}}{\pgfqpoint{0.852178in}{1.493117in}}{\pgfqpoint{0.860415in}{1.493117in}}%
\pgfpathclose%
\pgfusepath{stroke,fill}%
\end{pgfscope}%
\begin{pgfscope}%
\pgfpathrectangle{\pgfqpoint{0.100000in}{0.212622in}}{\pgfqpoint{3.696000in}{3.696000in}}%
\pgfusepath{clip}%
\pgfsetbuttcap%
\pgfsetroundjoin%
\definecolor{currentfill}{rgb}{0.121569,0.466667,0.705882}%
\pgfsetfillcolor{currentfill}%
\pgfsetfillopacity{0.627433}%
\pgfsetlinewidth{1.003750pt}%
\definecolor{currentstroke}{rgb}{0.121569,0.466667,0.705882}%
\pgfsetstrokecolor{currentstroke}%
\pgfsetstrokeopacity{0.627433}%
\pgfsetdash{}{0pt}%
\pgfpathmoveto{\pgfqpoint{0.860415in}{1.493117in}}%
\pgfpathcurveto{\pgfqpoint{0.868651in}{1.493117in}}{\pgfqpoint{0.876551in}{1.496389in}}{\pgfqpoint{0.882375in}{1.502213in}}%
\pgfpathcurveto{\pgfqpoint{0.888199in}{1.508037in}}{\pgfqpoint{0.891471in}{1.515937in}}{\pgfqpoint{0.891471in}{1.524173in}}%
\pgfpathcurveto{\pgfqpoint{0.891471in}{1.532410in}}{\pgfqpoint{0.888199in}{1.540310in}}{\pgfqpoint{0.882375in}{1.546134in}}%
\pgfpathcurveto{\pgfqpoint{0.876551in}{1.551958in}}{\pgfqpoint{0.868651in}{1.555230in}}{\pgfqpoint{0.860415in}{1.555230in}}%
\pgfpathcurveto{\pgfqpoint{0.852178in}{1.555230in}}{\pgfqpoint{0.844278in}{1.551958in}}{\pgfqpoint{0.838454in}{1.546134in}}%
\pgfpathcurveto{\pgfqpoint{0.832631in}{1.540310in}}{\pgfqpoint{0.829358in}{1.532410in}}{\pgfqpoint{0.829358in}{1.524173in}}%
\pgfpathcurveto{\pgfqpoint{0.829358in}{1.515937in}}{\pgfqpoint{0.832631in}{1.508037in}}{\pgfqpoint{0.838454in}{1.502213in}}%
\pgfpathcurveto{\pgfqpoint{0.844278in}{1.496389in}}{\pgfqpoint{0.852178in}{1.493117in}}{\pgfqpoint{0.860415in}{1.493117in}}%
\pgfpathclose%
\pgfusepath{stroke,fill}%
\end{pgfscope}%
\begin{pgfscope}%
\pgfpathrectangle{\pgfqpoint{0.100000in}{0.212622in}}{\pgfqpoint{3.696000in}{3.696000in}}%
\pgfusepath{clip}%
\pgfsetbuttcap%
\pgfsetroundjoin%
\definecolor{currentfill}{rgb}{0.121569,0.466667,0.705882}%
\pgfsetfillcolor{currentfill}%
\pgfsetfillopacity{0.627433}%
\pgfsetlinewidth{1.003750pt}%
\definecolor{currentstroke}{rgb}{0.121569,0.466667,0.705882}%
\pgfsetstrokecolor{currentstroke}%
\pgfsetstrokeopacity{0.627433}%
\pgfsetdash{}{0pt}%
\pgfpathmoveto{\pgfqpoint{0.860415in}{1.493117in}}%
\pgfpathcurveto{\pgfqpoint{0.868651in}{1.493117in}}{\pgfqpoint{0.876551in}{1.496389in}}{\pgfqpoint{0.882375in}{1.502213in}}%
\pgfpathcurveto{\pgfqpoint{0.888199in}{1.508037in}}{\pgfqpoint{0.891471in}{1.515937in}}{\pgfqpoint{0.891471in}{1.524173in}}%
\pgfpathcurveto{\pgfqpoint{0.891471in}{1.532410in}}{\pgfqpoint{0.888199in}{1.540310in}}{\pgfqpoint{0.882375in}{1.546134in}}%
\pgfpathcurveto{\pgfqpoint{0.876551in}{1.551958in}}{\pgfqpoint{0.868651in}{1.555230in}}{\pgfqpoint{0.860415in}{1.555230in}}%
\pgfpathcurveto{\pgfqpoint{0.852178in}{1.555230in}}{\pgfqpoint{0.844278in}{1.551958in}}{\pgfqpoint{0.838454in}{1.546134in}}%
\pgfpathcurveto{\pgfqpoint{0.832631in}{1.540310in}}{\pgfqpoint{0.829358in}{1.532410in}}{\pgfqpoint{0.829358in}{1.524173in}}%
\pgfpathcurveto{\pgfqpoint{0.829358in}{1.515937in}}{\pgfqpoint{0.832631in}{1.508037in}}{\pgfqpoint{0.838454in}{1.502213in}}%
\pgfpathcurveto{\pgfqpoint{0.844278in}{1.496389in}}{\pgfqpoint{0.852178in}{1.493117in}}{\pgfqpoint{0.860415in}{1.493117in}}%
\pgfpathclose%
\pgfusepath{stroke,fill}%
\end{pgfscope}%
\begin{pgfscope}%
\pgfpathrectangle{\pgfqpoint{0.100000in}{0.212622in}}{\pgfqpoint{3.696000in}{3.696000in}}%
\pgfusepath{clip}%
\pgfsetbuttcap%
\pgfsetroundjoin%
\definecolor{currentfill}{rgb}{0.121569,0.466667,0.705882}%
\pgfsetfillcolor{currentfill}%
\pgfsetfillopacity{0.627433}%
\pgfsetlinewidth{1.003750pt}%
\definecolor{currentstroke}{rgb}{0.121569,0.466667,0.705882}%
\pgfsetstrokecolor{currentstroke}%
\pgfsetstrokeopacity{0.627433}%
\pgfsetdash{}{0pt}%
\pgfpathmoveto{\pgfqpoint{0.860415in}{1.493117in}}%
\pgfpathcurveto{\pgfqpoint{0.868651in}{1.493117in}}{\pgfqpoint{0.876551in}{1.496389in}}{\pgfqpoint{0.882375in}{1.502213in}}%
\pgfpathcurveto{\pgfqpoint{0.888199in}{1.508037in}}{\pgfqpoint{0.891471in}{1.515937in}}{\pgfqpoint{0.891471in}{1.524173in}}%
\pgfpathcurveto{\pgfqpoint{0.891471in}{1.532410in}}{\pgfqpoint{0.888199in}{1.540310in}}{\pgfqpoint{0.882375in}{1.546134in}}%
\pgfpathcurveto{\pgfqpoint{0.876551in}{1.551958in}}{\pgfqpoint{0.868651in}{1.555230in}}{\pgfqpoint{0.860415in}{1.555230in}}%
\pgfpathcurveto{\pgfqpoint{0.852178in}{1.555230in}}{\pgfqpoint{0.844278in}{1.551958in}}{\pgfqpoint{0.838454in}{1.546134in}}%
\pgfpathcurveto{\pgfqpoint{0.832631in}{1.540310in}}{\pgfqpoint{0.829358in}{1.532410in}}{\pgfqpoint{0.829358in}{1.524173in}}%
\pgfpathcurveto{\pgfqpoint{0.829358in}{1.515937in}}{\pgfqpoint{0.832631in}{1.508037in}}{\pgfqpoint{0.838454in}{1.502213in}}%
\pgfpathcurveto{\pgfqpoint{0.844278in}{1.496389in}}{\pgfqpoint{0.852178in}{1.493117in}}{\pgfqpoint{0.860415in}{1.493117in}}%
\pgfpathclose%
\pgfusepath{stroke,fill}%
\end{pgfscope}%
\begin{pgfscope}%
\pgfpathrectangle{\pgfqpoint{0.100000in}{0.212622in}}{\pgfqpoint{3.696000in}{3.696000in}}%
\pgfusepath{clip}%
\pgfsetbuttcap%
\pgfsetroundjoin%
\definecolor{currentfill}{rgb}{0.121569,0.466667,0.705882}%
\pgfsetfillcolor{currentfill}%
\pgfsetfillopacity{0.627433}%
\pgfsetlinewidth{1.003750pt}%
\definecolor{currentstroke}{rgb}{0.121569,0.466667,0.705882}%
\pgfsetstrokecolor{currentstroke}%
\pgfsetstrokeopacity{0.627433}%
\pgfsetdash{}{0pt}%
\pgfpathmoveto{\pgfqpoint{0.860415in}{1.493117in}}%
\pgfpathcurveto{\pgfqpoint{0.868651in}{1.493117in}}{\pgfqpoint{0.876551in}{1.496389in}}{\pgfqpoint{0.882375in}{1.502213in}}%
\pgfpathcurveto{\pgfqpoint{0.888199in}{1.508037in}}{\pgfqpoint{0.891471in}{1.515937in}}{\pgfqpoint{0.891471in}{1.524173in}}%
\pgfpathcurveto{\pgfqpoint{0.891471in}{1.532410in}}{\pgfqpoint{0.888199in}{1.540310in}}{\pgfqpoint{0.882375in}{1.546134in}}%
\pgfpathcurveto{\pgfqpoint{0.876551in}{1.551958in}}{\pgfqpoint{0.868651in}{1.555230in}}{\pgfqpoint{0.860415in}{1.555230in}}%
\pgfpathcurveto{\pgfqpoint{0.852178in}{1.555230in}}{\pgfqpoint{0.844278in}{1.551958in}}{\pgfqpoint{0.838454in}{1.546134in}}%
\pgfpathcurveto{\pgfqpoint{0.832631in}{1.540310in}}{\pgfqpoint{0.829358in}{1.532410in}}{\pgfqpoint{0.829358in}{1.524173in}}%
\pgfpathcurveto{\pgfqpoint{0.829358in}{1.515937in}}{\pgfqpoint{0.832631in}{1.508037in}}{\pgfqpoint{0.838454in}{1.502213in}}%
\pgfpathcurveto{\pgfqpoint{0.844278in}{1.496389in}}{\pgfqpoint{0.852178in}{1.493117in}}{\pgfqpoint{0.860415in}{1.493117in}}%
\pgfpathclose%
\pgfusepath{stroke,fill}%
\end{pgfscope}%
\begin{pgfscope}%
\pgfpathrectangle{\pgfqpoint{0.100000in}{0.212622in}}{\pgfqpoint{3.696000in}{3.696000in}}%
\pgfusepath{clip}%
\pgfsetbuttcap%
\pgfsetroundjoin%
\definecolor{currentfill}{rgb}{0.121569,0.466667,0.705882}%
\pgfsetfillcolor{currentfill}%
\pgfsetfillopacity{0.627433}%
\pgfsetlinewidth{1.003750pt}%
\definecolor{currentstroke}{rgb}{0.121569,0.466667,0.705882}%
\pgfsetstrokecolor{currentstroke}%
\pgfsetstrokeopacity{0.627433}%
\pgfsetdash{}{0pt}%
\pgfpathmoveto{\pgfqpoint{0.860415in}{1.493117in}}%
\pgfpathcurveto{\pgfqpoint{0.868651in}{1.493117in}}{\pgfqpoint{0.876551in}{1.496389in}}{\pgfqpoint{0.882375in}{1.502213in}}%
\pgfpathcurveto{\pgfqpoint{0.888199in}{1.508037in}}{\pgfqpoint{0.891471in}{1.515937in}}{\pgfqpoint{0.891471in}{1.524173in}}%
\pgfpathcurveto{\pgfqpoint{0.891471in}{1.532410in}}{\pgfqpoint{0.888199in}{1.540310in}}{\pgfqpoint{0.882375in}{1.546134in}}%
\pgfpathcurveto{\pgfqpoint{0.876551in}{1.551958in}}{\pgfqpoint{0.868651in}{1.555230in}}{\pgfqpoint{0.860415in}{1.555230in}}%
\pgfpathcurveto{\pgfqpoint{0.852178in}{1.555230in}}{\pgfqpoint{0.844278in}{1.551958in}}{\pgfqpoint{0.838454in}{1.546134in}}%
\pgfpathcurveto{\pgfqpoint{0.832631in}{1.540310in}}{\pgfqpoint{0.829358in}{1.532410in}}{\pgfqpoint{0.829358in}{1.524173in}}%
\pgfpathcurveto{\pgfqpoint{0.829358in}{1.515937in}}{\pgfqpoint{0.832631in}{1.508037in}}{\pgfqpoint{0.838454in}{1.502213in}}%
\pgfpathcurveto{\pgfqpoint{0.844278in}{1.496389in}}{\pgfqpoint{0.852178in}{1.493117in}}{\pgfqpoint{0.860415in}{1.493117in}}%
\pgfpathclose%
\pgfusepath{stroke,fill}%
\end{pgfscope}%
\begin{pgfscope}%
\pgfpathrectangle{\pgfqpoint{0.100000in}{0.212622in}}{\pgfqpoint{3.696000in}{3.696000in}}%
\pgfusepath{clip}%
\pgfsetbuttcap%
\pgfsetroundjoin%
\definecolor{currentfill}{rgb}{0.121569,0.466667,0.705882}%
\pgfsetfillcolor{currentfill}%
\pgfsetfillopacity{0.627433}%
\pgfsetlinewidth{1.003750pt}%
\definecolor{currentstroke}{rgb}{0.121569,0.466667,0.705882}%
\pgfsetstrokecolor{currentstroke}%
\pgfsetstrokeopacity{0.627433}%
\pgfsetdash{}{0pt}%
\pgfpathmoveto{\pgfqpoint{0.860415in}{1.493117in}}%
\pgfpathcurveto{\pgfqpoint{0.868651in}{1.493117in}}{\pgfqpoint{0.876551in}{1.496389in}}{\pgfqpoint{0.882375in}{1.502213in}}%
\pgfpathcurveto{\pgfqpoint{0.888199in}{1.508037in}}{\pgfqpoint{0.891471in}{1.515937in}}{\pgfqpoint{0.891471in}{1.524173in}}%
\pgfpathcurveto{\pgfqpoint{0.891471in}{1.532410in}}{\pgfqpoint{0.888199in}{1.540310in}}{\pgfqpoint{0.882375in}{1.546134in}}%
\pgfpathcurveto{\pgfqpoint{0.876551in}{1.551958in}}{\pgfqpoint{0.868651in}{1.555230in}}{\pgfqpoint{0.860415in}{1.555230in}}%
\pgfpathcurveto{\pgfqpoint{0.852178in}{1.555230in}}{\pgfqpoint{0.844278in}{1.551958in}}{\pgfqpoint{0.838454in}{1.546134in}}%
\pgfpathcurveto{\pgfqpoint{0.832631in}{1.540310in}}{\pgfqpoint{0.829358in}{1.532410in}}{\pgfqpoint{0.829358in}{1.524173in}}%
\pgfpathcurveto{\pgfqpoint{0.829358in}{1.515937in}}{\pgfqpoint{0.832631in}{1.508037in}}{\pgfqpoint{0.838454in}{1.502213in}}%
\pgfpathcurveto{\pgfqpoint{0.844278in}{1.496389in}}{\pgfqpoint{0.852178in}{1.493117in}}{\pgfqpoint{0.860415in}{1.493117in}}%
\pgfpathclose%
\pgfusepath{stroke,fill}%
\end{pgfscope}%
\begin{pgfscope}%
\pgfpathrectangle{\pgfqpoint{0.100000in}{0.212622in}}{\pgfqpoint{3.696000in}{3.696000in}}%
\pgfusepath{clip}%
\pgfsetbuttcap%
\pgfsetroundjoin%
\definecolor{currentfill}{rgb}{0.121569,0.466667,0.705882}%
\pgfsetfillcolor{currentfill}%
\pgfsetfillopacity{0.627433}%
\pgfsetlinewidth{1.003750pt}%
\definecolor{currentstroke}{rgb}{0.121569,0.466667,0.705882}%
\pgfsetstrokecolor{currentstroke}%
\pgfsetstrokeopacity{0.627433}%
\pgfsetdash{}{0pt}%
\pgfpathmoveto{\pgfqpoint{0.860415in}{1.493117in}}%
\pgfpathcurveto{\pgfqpoint{0.868651in}{1.493117in}}{\pgfqpoint{0.876551in}{1.496389in}}{\pgfqpoint{0.882375in}{1.502213in}}%
\pgfpathcurveto{\pgfqpoint{0.888199in}{1.508037in}}{\pgfqpoint{0.891471in}{1.515937in}}{\pgfqpoint{0.891471in}{1.524173in}}%
\pgfpathcurveto{\pgfqpoint{0.891471in}{1.532410in}}{\pgfqpoint{0.888199in}{1.540310in}}{\pgfqpoint{0.882375in}{1.546134in}}%
\pgfpathcurveto{\pgfqpoint{0.876551in}{1.551958in}}{\pgfqpoint{0.868651in}{1.555230in}}{\pgfqpoint{0.860415in}{1.555230in}}%
\pgfpathcurveto{\pgfqpoint{0.852178in}{1.555230in}}{\pgfqpoint{0.844278in}{1.551958in}}{\pgfqpoint{0.838454in}{1.546134in}}%
\pgfpathcurveto{\pgfqpoint{0.832631in}{1.540310in}}{\pgfqpoint{0.829358in}{1.532410in}}{\pgfqpoint{0.829358in}{1.524173in}}%
\pgfpathcurveto{\pgfqpoint{0.829358in}{1.515937in}}{\pgfqpoint{0.832631in}{1.508037in}}{\pgfqpoint{0.838454in}{1.502213in}}%
\pgfpathcurveto{\pgfqpoint{0.844278in}{1.496389in}}{\pgfqpoint{0.852178in}{1.493117in}}{\pgfqpoint{0.860415in}{1.493117in}}%
\pgfpathclose%
\pgfusepath{stroke,fill}%
\end{pgfscope}%
\begin{pgfscope}%
\pgfpathrectangle{\pgfqpoint{0.100000in}{0.212622in}}{\pgfqpoint{3.696000in}{3.696000in}}%
\pgfusepath{clip}%
\pgfsetbuttcap%
\pgfsetroundjoin%
\definecolor{currentfill}{rgb}{0.121569,0.466667,0.705882}%
\pgfsetfillcolor{currentfill}%
\pgfsetfillopacity{0.627433}%
\pgfsetlinewidth{1.003750pt}%
\definecolor{currentstroke}{rgb}{0.121569,0.466667,0.705882}%
\pgfsetstrokecolor{currentstroke}%
\pgfsetstrokeopacity{0.627433}%
\pgfsetdash{}{0pt}%
\pgfpathmoveto{\pgfqpoint{0.860415in}{1.493117in}}%
\pgfpathcurveto{\pgfqpoint{0.868651in}{1.493117in}}{\pgfqpoint{0.876551in}{1.496389in}}{\pgfqpoint{0.882375in}{1.502213in}}%
\pgfpathcurveto{\pgfqpoint{0.888199in}{1.508037in}}{\pgfqpoint{0.891471in}{1.515937in}}{\pgfqpoint{0.891471in}{1.524173in}}%
\pgfpathcurveto{\pgfqpoint{0.891471in}{1.532410in}}{\pgfqpoint{0.888199in}{1.540310in}}{\pgfqpoint{0.882375in}{1.546134in}}%
\pgfpathcurveto{\pgfqpoint{0.876551in}{1.551958in}}{\pgfqpoint{0.868651in}{1.555230in}}{\pgfqpoint{0.860415in}{1.555230in}}%
\pgfpathcurveto{\pgfqpoint{0.852178in}{1.555230in}}{\pgfqpoint{0.844278in}{1.551958in}}{\pgfqpoint{0.838454in}{1.546134in}}%
\pgfpathcurveto{\pgfqpoint{0.832631in}{1.540310in}}{\pgfqpoint{0.829358in}{1.532410in}}{\pgfqpoint{0.829358in}{1.524173in}}%
\pgfpathcurveto{\pgfqpoint{0.829358in}{1.515937in}}{\pgfqpoint{0.832631in}{1.508037in}}{\pgfqpoint{0.838454in}{1.502213in}}%
\pgfpathcurveto{\pgfqpoint{0.844278in}{1.496389in}}{\pgfqpoint{0.852178in}{1.493117in}}{\pgfqpoint{0.860415in}{1.493117in}}%
\pgfpathclose%
\pgfusepath{stroke,fill}%
\end{pgfscope}%
\begin{pgfscope}%
\pgfpathrectangle{\pgfqpoint{0.100000in}{0.212622in}}{\pgfqpoint{3.696000in}{3.696000in}}%
\pgfusepath{clip}%
\pgfsetbuttcap%
\pgfsetroundjoin%
\definecolor{currentfill}{rgb}{0.121569,0.466667,0.705882}%
\pgfsetfillcolor{currentfill}%
\pgfsetfillopacity{0.627433}%
\pgfsetlinewidth{1.003750pt}%
\definecolor{currentstroke}{rgb}{0.121569,0.466667,0.705882}%
\pgfsetstrokecolor{currentstroke}%
\pgfsetstrokeopacity{0.627433}%
\pgfsetdash{}{0pt}%
\pgfpathmoveto{\pgfqpoint{0.860415in}{1.493117in}}%
\pgfpathcurveto{\pgfqpoint{0.868651in}{1.493117in}}{\pgfqpoint{0.876551in}{1.496389in}}{\pgfqpoint{0.882375in}{1.502213in}}%
\pgfpathcurveto{\pgfqpoint{0.888199in}{1.508037in}}{\pgfqpoint{0.891471in}{1.515937in}}{\pgfqpoint{0.891471in}{1.524173in}}%
\pgfpathcurveto{\pgfqpoint{0.891471in}{1.532410in}}{\pgfqpoint{0.888199in}{1.540310in}}{\pgfqpoint{0.882375in}{1.546134in}}%
\pgfpathcurveto{\pgfqpoint{0.876551in}{1.551958in}}{\pgfqpoint{0.868651in}{1.555230in}}{\pgfqpoint{0.860415in}{1.555230in}}%
\pgfpathcurveto{\pgfqpoint{0.852178in}{1.555230in}}{\pgfqpoint{0.844278in}{1.551958in}}{\pgfqpoint{0.838454in}{1.546134in}}%
\pgfpathcurveto{\pgfqpoint{0.832631in}{1.540310in}}{\pgfqpoint{0.829358in}{1.532410in}}{\pgfqpoint{0.829358in}{1.524173in}}%
\pgfpathcurveto{\pgfqpoint{0.829358in}{1.515937in}}{\pgfqpoint{0.832631in}{1.508037in}}{\pgfqpoint{0.838454in}{1.502213in}}%
\pgfpathcurveto{\pgfqpoint{0.844278in}{1.496389in}}{\pgfqpoint{0.852178in}{1.493117in}}{\pgfqpoint{0.860415in}{1.493117in}}%
\pgfpathclose%
\pgfusepath{stroke,fill}%
\end{pgfscope}%
\begin{pgfscope}%
\pgfpathrectangle{\pgfqpoint{0.100000in}{0.212622in}}{\pgfqpoint{3.696000in}{3.696000in}}%
\pgfusepath{clip}%
\pgfsetbuttcap%
\pgfsetroundjoin%
\definecolor{currentfill}{rgb}{0.121569,0.466667,0.705882}%
\pgfsetfillcolor{currentfill}%
\pgfsetfillopacity{0.627433}%
\pgfsetlinewidth{1.003750pt}%
\definecolor{currentstroke}{rgb}{0.121569,0.466667,0.705882}%
\pgfsetstrokecolor{currentstroke}%
\pgfsetstrokeopacity{0.627433}%
\pgfsetdash{}{0pt}%
\pgfpathmoveto{\pgfqpoint{0.860415in}{1.493117in}}%
\pgfpathcurveto{\pgfqpoint{0.868651in}{1.493117in}}{\pgfqpoint{0.876551in}{1.496389in}}{\pgfqpoint{0.882375in}{1.502213in}}%
\pgfpathcurveto{\pgfqpoint{0.888199in}{1.508037in}}{\pgfqpoint{0.891471in}{1.515937in}}{\pgfqpoint{0.891471in}{1.524173in}}%
\pgfpathcurveto{\pgfqpoint{0.891471in}{1.532410in}}{\pgfqpoint{0.888199in}{1.540310in}}{\pgfqpoint{0.882375in}{1.546134in}}%
\pgfpathcurveto{\pgfqpoint{0.876551in}{1.551958in}}{\pgfqpoint{0.868651in}{1.555230in}}{\pgfqpoint{0.860415in}{1.555230in}}%
\pgfpathcurveto{\pgfqpoint{0.852178in}{1.555230in}}{\pgfqpoint{0.844278in}{1.551958in}}{\pgfqpoint{0.838454in}{1.546134in}}%
\pgfpathcurveto{\pgfqpoint{0.832631in}{1.540310in}}{\pgfqpoint{0.829358in}{1.532410in}}{\pgfqpoint{0.829358in}{1.524173in}}%
\pgfpathcurveto{\pgfqpoint{0.829358in}{1.515937in}}{\pgfqpoint{0.832631in}{1.508037in}}{\pgfqpoint{0.838454in}{1.502213in}}%
\pgfpathcurveto{\pgfqpoint{0.844278in}{1.496389in}}{\pgfqpoint{0.852178in}{1.493117in}}{\pgfqpoint{0.860415in}{1.493117in}}%
\pgfpathclose%
\pgfusepath{stroke,fill}%
\end{pgfscope}%
\begin{pgfscope}%
\pgfpathrectangle{\pgfqpoint{0.100000in}{0.212622in}}{\pgfqpoint{3.696000in}{3.696000in}}%
\pgfusepath{clip}%
\pgfsetbuttcap%
\pgfsetroundjoin%
\definecolor{currentfill}{rgb}{0.121569,0.466667,0.705882}%
\pgfsetfillcolor{currentfill}%
\pgfsetfillopacity{0.627433}%
\pgfsetlinewidth{1.003750pt}%
\definecolor{currentstroke}{rgb}{0.121569,0.466667,0.705882}%
\pgfsetstrokecolor{currentstroke}%
\pgfsetstrokeopacity{0.627433}%
\pgfsetdash{}{0pt}%
\pgfpathmoveto{\pgfqpoint{0.860415in}{1.493117in}}%
\pgfpathcurveto{\pgfqpoint{0.868651in}{1.493117in}}{\pgfqpoint{0.876551in}{1.496389in}}{\pgfqpoint{0.882375in}{1.502213in}}%
\pgfpathcurveto{\pgfqpoint{0.888199in}{1.508037in}}{\pgfqpoint{0.891471in}{1.515937in}}{\pgfqpoint{0.891471in}{1.524173in}}%
\pgfpathcurveto{\pgfqpoint{0.891471in}{1.532410in}}{\pgfqpoint{0.888199in}{1.540310in}}{\pgfqpoint{0.882375in}{1.546134in}}%
\pgfpathcurveto{\pgfqpoint{0.876551in}{1.551958in}}{\pgfqpoint{0.868651in}{1.555230in}}{\pgfqpoint{0.860415in}{1.555230in}}%
\pgfpathcurveto{\pgfqpoint{0.852178in}{1.555230in}}{\pgfqpoint{0.844278in}{1.551958in}}{\pgfqpoint{0.838454in}{1.546134in}}%
\pgfpathcurveto{\pgfqpoint{0.832631in}{1.540310in}}{\pgfqpoint{0.829358in}{1.532410in}}{\pgfqpoint{0.829358in}{1.524173in}}%
\pgfpathcurveto{\pgfqpoint{0.829358in}{1.515937in}}{\pgfqpoint{0.832631in}{1.508037in}}{\pgfqpoint{0.838454in}{1.502213in}}%
\pgfpathcurveto{\pgfqpoint{0.844278in}{1.496389in}}{\pgfqpoint{0.852178in}{1.493117in}}{\pgfqpoint{0.860415in}{1.493117in}}%
\pgfpathclose%
\pgfusepath{stroke,fill}%
\end{pgfscope}%
\begin{pgfscope}%
\pgfpathrectangle{\pgfqpoint{0.100000in}{0.212622in}}{\pgfqpoint{3.696000in}{3.696000in}}%
\pgfusepath{clip}%
\pgfsetbuttcap%
\pgfsetroundjoin%
\definecolor{currentfill}{rgb}{0.121569,0.466667,0.705882}%
\pgfsetfillcolor{currentfill}%
\pgfsetfillopacity{0.627433}%
\pgfsetlinewidth{1.003750pt}%
\definecolor{currentstroke}{rgb}{0.121569,0.466667,0.705882}%
\pgfsetstrokecolor{currentstroke}%
\pgfsetstrokeopacity{0.627433}%
\pgfsetdash{}{0pt}%
\pgfpathmoveto{\pgfqpoint{0.860415in}{1.493117in}}%
\pgfpathcurveto{\pgfqpoint{0.868651in}{1.493117in}}{\pgfqpoint{0.876551in}{1.496389in}}{\pgfqpoint{0.882375in}{1.502213in}}%
\pgfpathcurveto{\pgfqpoint{0.888199in}{1.508037in}}{\pgfqpoint{0.891471in}{1.515937in}}{\pgfqpoint{0.891471in}{1.524173in}}%
\pgfpathcurveto{\pgfqpoint{0.891471in}{1.532410in}}{\pgfqpoint{0.888199in}{1.540310in}}{\pgfqpoint{0.882375in}{1.546134in}}%
\pgfpathcurveto{\pgfqpoint{0.876551in}{1.551958in}}{\pgfqpoint{0.868651in}{1.555230in}}{\pgfqpoint{0.860415in}{1.555230in}}%
\pgfpathcurveto{\pgfqpoint{0.852178in}{1.555230in}}{\pgfqpoint{0.844278in}{1.551958in}}{\pgfqpoint{0.838454in}{1.546134in}}%
\pgfpathcurveto{\pgfqpoint{0.832631in}{1.540310in}}{\pgfqpoint{0.829358in}{1.532410in}}{\pgfqpoint{0.829358in}{1.524173in}}%
\pgfpathcurveto{\pgfqpoint{0.829358in}{1.515937in}}{\pgfqpoint{0.832631in}{1.508037in}}{\pgfqpoint{0.838454in}{1.502213in}}%
\pgfpathcurveto{\pgfqpoint{0.844278in}{1.496389in}}{\pgfqpoint{0.852178in}{1.493117in}}{\pgfqpoint{0.860415in}{1.493117in}}%
\pgfpathclose%
\pgfusepath{stroke,fill}%
\end{pgfscope}%
\begin{pgfscope}%
\pgfpathrectangle{\pgfqpoint{0.100000in}{0.212622in}}{\pgfqpoint{3.696000in}{3.696000in}}%
\pgfusepath{clip}%
\pgfsetbuttcap%
\pgfsetroundjoin%
\definecolor{currentfill}{rgb}{0.121569,0.466667,0.705882}%
\pgfsetfillcolor{currentfill}%
\pgfsetfillopacity{0.627433}%
\pgfsetlinewidth{1.003750pt}%
\definecolor{currentstroke}{rgb}{0.121569,0.466667,0.705882}%
\pgfsetstrokecolor{currentstroke}%
\pgfsetstrokeopacity{0.627433}%
\pgfsetdash{}{0pt}%
\pgfpathmoveto{\pgfqpoint{0.860415in}{1.493117in}}%
\pgfpathcurveto{\pgfqpoint{0.868651in}{1.493117in}}{\pgfqpoint{0.876551in}{1.496389in}}{\pgfqpoint{0.882375in}{1.502213in}}%
\pgfpathcurveto{\pgfqpoint{0.888199in}{1.508037in}}{\pgfqpoint{0.891471in}{1.515937in}}{\pgfqpoint{0.891471in}{1.524173in}}%
\pgfpathcurveto{\pgfqpoint{0.891471in}{1.532410in}}{\pgfqpoint{0.888199in}{1.540310in}}{\pgfqpoint{0.882375in}{1.546134in}}%
\pgfpathcurveto{\pgfqpoint{0.876551in}{1.551958in}}{\pgfqpoint{0.868651in}{1.555230in}}{\pgfqpoint{0.860415in}{1.555230in}}%
\pgfpathcurveto{\pgfqpoint{0.852178in}{1.555230in}}{\pgfqpoint{0.844278in}{1.551958in}}{\pgfqpoint{0.838454in}{1.546134in}}%
\pgfpathcurveto{\pgfqpoint{0.832631in}{1.540310in}}{\pgfqpoint{0.829358in}{1.532410in}}{\pgfqpoint{0.829358in}{1.524173in}}%
\pgfpathcurveto{\pgfqpoint{0.829358in}{1.515937in}}{\pgfqpoint{0.832631in}{1.508037in}}{\pgfqpoint{0.838454in}{1.502213in}}%
\pgfpathcurveto{\pgfqpoint{0.844278in}{1.496389in}}{\pgfqpoint{0.852178in}{1.493117in}}{\pgfqpoint{0.860415in}{1.493117in}}%
\pgfpathclose%
\pgfusepath{stroke,fill}%
\end{pgfscope}%
\begin{pgfscope}%
\pgfpathrectangle{\pgfqpoint{0.100000in}{0.212622in}}{\pgfqpoint{3.696000in}{3.696000in}}%
\pgfusepath{clip}%
\pgfsetbuttcap%
\pgfsetroundjoin%
\definecolor{currentfill}{rgb}{0.121569,0.466667,0.705882}%
\pgfsetfillcolor{currentfill}%
\pgfsetfillopacity{0.627433}%
\pgfsetlinewidth{1.003750pt}%
\definecolor{currentstroke}{rgb}{0.121569,0.466667,0.705882}%
\pgfsetstrokecolor{currentstroke}%
\pgfsetstrokeopacity{0.627433}%
\pgfsetdash{}{0pt}%
\pgfpathmoveto{\pgfqpoint{0.860415in}{1.493117in}}%
\pgfpathcurveto{\pgfqpoint{0.868651in}{1.493117in}}{\pgfqpoint{0.876551in}{1.496389in}}{\pgfqpoint{0.882375in}{1.502213in}}%
\pgfpathcurveto{\pgfqpoint{0.888199in}{1.508037in}}{\pgfqpoint{0.891471in}{1.515937in}}{\pgfqpoint{0.891471in}{1.524173in}}%
\pgfpathcurveto{\pgfqpoint{0.891471in}{1.532410in}}{\pgfqpoint{0.888199in}{1.540310in}}{\pgfqpoint{0.882375in}{1.546134in}}%
\pgfpathcurveto{\pgfqpoint{0.876551in}{1.551958in}}{\pgfqpoint{0.868651in}{1.555230in}}{\pgfqpoint{0.860415in}{1.555230in}}%
\pgfpathcurveto{\pgfqpoint{0.852178in}{1.555230in}}{\pgfqpoint{0.844278in}{1.551958in}}{\pgfqpoint{0.838454in}{1.546134in}}%
\pgfpathcurveto{\pgfqpoint{0.832631in}{1.540310in}}{\pgfqpoint{0.829358in}{1.532410in}}{\pgfqpoint{0.829358in}{1.524173in}}%
\pgfpathcurveto{\pgfqpoint{0.829358in}{1.515937in}}{\pgfqpoint{0.832631in}{1.508037in}}{\pgfqpoint{0.838454in}{1.502213in}}%
\pgfpathcurveto{\pgfqpoint{0.844278in}{1.496389in}}{\pgfqpoint{0.852178in}{1.493117in}}{\pgfqpoint{0.860415in}{1.493117in}}%
\pgfpathclose%
\pgfusepath{stroke,fill}%
\end{pgfscope}%
\begin{pgfscope}%
\pgfpathrectangle{\pgfqpoint{0.100000in}{0.212622in}}{\pgfqpoint{3.696000in}{3.696000in}}%
\pgfusepath{clip}%
\pgfsetbuttcap%
\pgfsetroundjoin%
\definecolor{currentfill}{rgb}{0.121569,0.466667,0.705882}%
\pgfsetfillcolor{currentfill}%
\pgfsetfillopacity{0.627433}%
\pgfsetlinewidth{1.003750pt}%
\definecolor{currentstroke}{rgb}{0.121569,0.466667,0.705882}%
\pgfsetstrokecolor{currentstroke}%
\pgfsetstrokeopacity{0.627433}%
\pgfsetdash{}{0pt}%
\pgfpathmoveto{\pgfqpoint{0.860415in}{1.493117in}}%
\pgfpathcurveto{\pgfqpoint{0.868651in}{1.493117in}}{\pgfqpoint{0.876551in}{1.496389in}}{\pgfqpoint{0.882375in}{1.502213in}}%
\pgfpathcurveto{\pgfqpoint{0.888199in}{1.508037in}}{\pgfqpoint{0.891471in}{1.515937in}}{\pgfqpoint{0.891471in}{1.524173in}}%
\pgfpathcurveto{\pgfqpoint{0.891471in}{1.532410in}}{\pgfqpoint{0.888199in}{1.540310in}}{\pgfqpoint{0.882375in}{1.546134in}}%
\pgfpathcurveto{\pgfqpoint{0.876551in}{1.551958in}}{\pgfqpoint{0.868651in}{1.555230in}}{\pgfqpoint{0.860415in}{1.555230in}}%
\pgfpathcurveto{\pgfqpoint{0.852178in}{1.555230in}}{\pgfqpoint{0.844278in}{1.551958in}}{\pgfqpoint{0.838454in}{1.546134in}}%
\pgfpathcurveto{\pgfqpoint{0.832631in}{1.540310in}}{\pgfqpoint{0.829358in}{1.532410in}}{\pgfqpoint{0.829358in}{1.524173in}}%
\pgfpathcurveto{\pgfqpoint{0.829358in}{1.515937in}}{\pgfqpoint{0.832631in}{1.508037in}}{\pgfqpoint{0.838454in}{1.502213in}}%
\pgfpathcurveto{\pgfqpoint{0.844278in}{1.496389in}}{\pgfqpoint{0.852178in}{1.493117in}}{\pgfqpoint{0.860415in}{1.493117in}}%
\pgfpathclose%
\pgfusepath{stroke,fill}%
\end{pgfscope}%
\begin{pgfscope}%
\pgfpathrectangle{\pgfqpoint{0.100000in}{0.212622in}}{\pgfqpoint{3.696000in}{3.696000in}}%
\pgfusepath{clip}%
\pgfsetbuttcap%
\pgfsetroundjoin%
\definecolor{currentfill}{rgb}{0.121569,0.466667,0.705882}%
\pgfsetfillcolor{currentfill}%
\pgfsetfillopacity{0.627433}%
\pgfsetlinewidth{1.003750pt}%
\definecolor{currentstroke}{rgb}{0.121569,0.466667,0.705882}%
\pgfsetstrokecolor{currentstroke}%
\pgfsetstrokeopacity{0.627433}%
\pgfsetdash{}{0pt}%
\pgfpathmoveto{\pgfqpoint{0.860415in}{1.493117in}}%
\pgfpathcurveto{\pgfqpoint{0.868651in}{1.493117in}}{\pgfqpoint{0.876551in}{1.496389in}}{\pgfqpoint{0.882375in}{1.502213in}}%
\pgfpathcurveto{\pgfqpoint{0.888199in}{1.508037in}}{\pgfqpoint{0.891471in}{1.515937in}}{\pgfqpoint{0.891471in}{1.524173in}}%
\pgfpathcurveto{\pgfqpoint{0.891471in}{1.532410in}}{\pgfqpoint{0.888199in}{1.540310in}}{\pgfqpoint{0.882375in}{1.546134in}}%
\pgfpathcurveto{\pgfqpoint{0.876551in}{1.551958in}}{\pgfqpoint{0.868651in}{1.555230in}}{\pgfqpoint{0.860415in}{1.555230in}}%
\pgfpathcurveto{\pgfqpoint{0.852178in}{1.555230in}}{\pgfqpoint{0.844278in}{1.551958in}}{\pgfqpoint{0.838454in}{1.546134in}}%
\pgfpathcurveto{\pgfqpoint{0.832631in}{1.540310in}}{\pgfqpoint{0.829358in}{1.532410in}}{\pgfqpoint{0.829358in}{1.524173in}}%
\pgfpathcurveto{\pgfqpoint{0.829358in}{1.515937in}}{\pgfqpoint{0.832631in}{1.508037in}}{\pgfqpoint{0.838454in}{1.502213in}}%
\pgfpathcurveto{\pgfqpoint{0.844278in}{1.496389in}}{\pgfqpoint{0.852178in}{1.493117in}}{\pgfqpoint{0.860415in}{1.493117in}}%
\pgfpathclose%
\pgfusepath{stroke,fill}%
\end{pgfscope}%
\begin{pgfscope}%
\pgfpathrectangle{\pgfqpoint{0.100000in}{0.212622in}}{\pgfqpoint{3.696000in}{3.696000in}}%
\pgfusepath{clip}%
\pgfsetbuttcap%
\pgfsetroundjoin%
\definecolor{currentfill}{rgb}{0.121569,0.466667,0.705882}%
\pgfsetfillcolor{currentfill}%
\pgfsetfillopacity{0.627433}%
\pgfsetlinewidth{1.003750pt}%
\definecolor{currentstroke}{rgb}{0.121569,0.466667,0.705882}%
\pgfsetstrokecolor{currentstroke}%
\pgfsetstrokeopacity{0.627433}%
\pgfsetdash{}{0pt}%
\pgfpathmoveto{\pgfqpoint{0.860415in}{1.493117in}}%
\pgfpathcurveto{\pgfqpoint{0.868651in}{1.493117in}}{\pgfqpoint{0.876551in}{1.496389in}}{\pgfqpoint{0.882375in}{1.502213in}}%
\pgfpathcurveto{\pgfqpoint{0.888199in}{1.508037in}}{\pgfqpoint{0.891471in}{1.515937in}}{\pgfqpoint{0.891471in}{1.524173in}}%
\pgfpathcurveto{\pgfqpoint{0.891471in}{1.532410in}}{\pgfqpoint{0.888199in}{1.540310in}}{\pgfqpoint{0.882375in}{1.546134in}}%
\pgfpathcurveto{\pgfqpoint{0.876551in}{1.551958in}}{\pgfqpoint{0.868651in}{1.555230in}}{\pgfqpoint{0.860415in}{1.555230in}}%
\pgfpathcurveto{\pgfqpoint{0.852178in}{1.555230in}}{\pgfqpoint{0.844278in}{1.551958in}}{\pgfqpoint{0.838454in}{1.546134in}}%
\pgfpathcurveto{\pgfqpoint{0.832631in}{1.540310in}}{\pgfqpoint{0.829358in}{1.532410in}}{\pgfqpoint{0.829358in}{1.524173in}}%
\pgfpathcurveto{\pgfqpoint{0.829358in}{1.515937in}}{\pgfqpoint{0.832631in}{1.508037in}}{\pgfqpoint{0.838454in}{1.502213in}}%
\pgfpathcurveto{\pgfqpoint{0.844278in}{1.496389in}}{\pgfqpoint{0.852178in}{1.493117in}}{\pgfqpoint{0.860415in}{1.493117in}}%
\pgfpathclose%
\pgfusepath{stroke,fill}%
\end{pgfscope}%
\begin{pgfscope}%
\pgfpathrectangle{\pgfqpoint{0.100000in}{0.212622in}}{\pgfqpoint{3.696000in}{3.696000in}}%
\pgfusepath{clip}%
\pgfsetbuttcap%
\pgfsetroundjoin%
\definecolor{currentfill}{rgb}{0.121569,0.466667,0.705882}%
\pgfsetfillcolor{currentfill}%
\pgfsetfillopacity{0.627433}%
\pgfsetlinewidth{1.003750pt}%
\definecolor{currentstroke}{rgb}{0.121569,0.466667,0.705882}%
\pgfsetstrokecolor{currentstroke}%
\pgfsetstrokeopacity{0.627433}%
\pgfsetdash{}{0pt}%
\pgfpathmoveto{\pgfqpoint{0.860415in}{1.493117in}}%
\pgfpathcurveto{\pgfqpoint{0.868651in}{1.493117in}}{\pgfqpoint{0.876551in}{1.496389in}}{\pgfqpoint{0.882375in}{1.502213in}}%
\pgfpathcurveto{\pgfqpoint{0.888199in}{1.508037in}}{\pgfqpoint{0.891471in}{1.515937in}}{\pgfqpoint{0.891471in}{1.524173in}}%
\pgfpathcurveto{\pgfqpoint{0.891471in}{1.532410in}}{\pgfqpoint{0.888199in}{1.540310in}}{\pgfqpoint{0.882375in}{1.546134in}}%
\pgfpathcurveto{\pgfqpoint{0.876551in}{1.551958in}}{\pgfqpoint{0.868651in}{1.555230in}}{\pgfqpoint{0.860415in}{1.555230in}}%
\pgfpathcurveto{\pgfqpoint{0.852178in}{1.555230in}}{\pgfqpoint{0.844278in}{1.551958in}}{\pgfqpoint{0.838454in}{1.546134in}}%
\pgfpathcurveto{\pgfqpoint{0.832631in}{1.540310in}}{\pgfqpoint{0.829358in}{1.532410in}}{\pgfqpoint{0.829358in}{1.524173in}}%
\pgfpathcurveto{\pgfqpoint{0.829358in}{1.515937in}}{\pgfqpoint{0.832631in}{1.508037in}}{\pgfqpoint{0.838454in}{1.502213in}}%
\pgfpathcurveto{\pgfqpoint{0.844278in}{1.496389in}}{\pgfqpoint{0.852178in}{1.493117in}}{\pgfqpoint{0.860415in}{1.493117in}}%
\pgfpathclose%
\pgfusepath{stroke,fill}%
\end{pgfscope}%
\begin{pgfscope}%
\pgfpathrectangle{\pgfqpoint{0.100000in}{0.212622in}}{\pgfqpoint{3.696000in}{3.696000in}}%
\pgfusepath{clip}%
\pgfsetbuttcap%
\pgfsetroundjoin%
\definecolor{currentfill}{rgb}{0.121569,0.466667,0.705882}%
\pgfsetfillcolor{currentfill}%
\pgfsetfillopacity{0.627433}%
\pgfsetlinewidth{1.003750pt}%
\definecolor{currentstroke}{rgb}{0.121569,0.466667,0.705882}%
\pgfsetstrokecolor{currentstroke}%
\pgfsetstrokeopacity{0.627433}%
\pgfsetdash{}{0pt}%
\pgfpathmoveto{\pgfqpoint{0.860415in}{1.493117in}}%
\pgfpathcurveto{\pgfqpoint{0.868651in}{1.493117in}}{\pgfqpoint{0.876551in}{1.496389in}}{\pgfqpoint{0.882375in}{1.502213in}}%
\pgfpathcurveto{\pgfqpoint{0.888199in}{1.508037in}}{\pgfqpoint{0.891471in}{1.515937in}}{\pgfqpoint{0.891471in}{1.524173in}}%
\pgfpathcurveto{\pgfqpoint{0.891471in}{1.532410in}}{\pgfqpoint{0.888199in}{1.540310in}}{\pgfqpoint{0.882375in}{1.546134in}}%
\pgfpathcurveto{\pgfqpoint{0.876551in}{1.551958in}}{\pgfqpoint{0.868651in}{1.555230in}}{\pgfqpoint{0.860415in}{1.555230in}}%
\pgfpathcurveto{\pgfqpoint{0.852178in}{1.555230in}}{\pgfqpoint{0.844278in}{1.551958in}}{\pgfqpoint{0.838454in}{1.546134in}}%
\pgfpathcurveto{\pgfqpoint{0.832631in}{1.540310in}}{\pgfqpoint{0.829358in}{1.532410in}}{\pgfqpoint{0.829358in}{1.524173in}}%
\pgfpathcurveto{\pgfqpoint{0.829358in}{1.515937in}}{\pgfqpoint{0.832631in}{1.508037in}}{\pgfqpoint{0.838454in}{1.502213in}}%
\pgfpathcurveto{\pgfqpoint{0.844278in}{1.496389in}}{\pgfqpoint{0.852178in}{1.493117in}}{\pgfqpoint{0.860415in}{1.493117in}}%
\pgfpathclose%
\pgfusepath{stroke,fill}%
\end{pgfscope}%
\begin{pgfscope}%
\pgfpathrectangle{\pgfqpoint{0.100000in}{0.212622in}}{\pgfqpoint{3.696000in}{3.696000in}}%
\pgfusepath{clip}%
\pgfsetbuttcap%
\pgfsetroundjoin%
\definecolor{currentfill}{rgb}{0.121569,0.466667,0.705882}%
\pgfsetfillcolor{currentfill}%
\pgfsetfillopacity{0.627433}%
\pgfsetlinewidth{1.003750pt}%
\definecolor{currentstroke}{rgb}{0.121569,0.466667,0.705882}%
\pgfsetstrokecolor{currentstroke}%
\pgfsetstrokeopacity{0.627433}%
\pgfsetdash{}{0pt}%
\pgfpathmoveto{\pgfqpoint{0.860415in}{1.493117in}}%
\pgfpathcurveto{\pgfqpoint{0.868651in}{1.493117in}}{\pgfqpoint{0.876551in}{1.496389in}}{\pgfqpoint{0.882375in}{1.502213in}}%
\pgfpathcurveto{\pgfqpoint{0.888199in}{1.508037in}}{\pgfqpoint{0.891471in}{1.515937in}}{\pgfqpoint{0.891471in}{1.524173in}}%
\pgfpathcurveto{\pgfqpoint{0.891471in}{1.532410in}}{\pgfqpoint{0.888199in}{1.540310in}}{\pgfqpoint{0.882375in}{1.546134in}}%
\pgfpathcurveto{\pgfqpoint{0.876551in}{1.551958in}}{\pgfqpoint{0.868651in}{1.555230in}}{\pgfqpoint{0.860415in}{1.555230in}}%
\pgfpathcurveto{\pgfqpoint{0.852178in}{1.555230in}}{\pgfqpoint{0.844278in}{1.551958in}}{\pgfqpoint{0.838454in}{1.546134in}}%
\pgfpathcurveto{\pgfqpoint{0.832631in}{1.540310in}}{\pgfqpoint{0.829358in}{1.532410in}}{\pgfqpoint{0.829358in}{1.524173in}}%
\pgfpathcurveto{\pgfqpoint{0.829358in}{1.515937in}}{\pgfqpoint{0.832631in}{1.508037in}}{\pgfqpoint{0.838454in}{1.502213in}}%
\pgfpathcurveto{\pgfqpoint{0.844278in}{1.496389in}}{\pgfqpoint{0.852178in}{1.493117in}}{\pgfqpoint{0.860415in}{1.493117in}}%
\pgfpathclose%
\pgfusepath{stroke,fill}%
\end{pgfscope}%
\begin{pgfscope}%
\pgfpathrectangle{\pgfqpoint{0.100000in}{0.212622in}}{\pgfqpoint{3.696000in}{3.696000in}}%
\pgfusepath{clip}%
\pgfsetbuttcap%
\pgfsetroundjoin%
\definecolor{currentfill}{rgb}{0.121569,0.466667,0.705882}%
\pgfsetfillcolor{currentfill}%
\pgfsetfillopacity{0.627433}%
\pgfsetlinewidth{1.003750pt}%
\definecolor{currentstroke}{rgb}{0.121569,0.466667,0.705882}%
\pgfsetstrokecolor{currentstroke}%
\pgfsetstrokeopacity{0.627433}%
\pgfsetdash{}{0pt}%
\pgfpathmoveto{\pgfqpoint{0.860415in}{1.493117in}}%
\pgfpathcurveto{\pgfqpoint{0.868651in}{1.493117in}}{\pgfqpoint{0.876551in}{1.496389in}}{\pgfqpoint{0.882375in}{1.502213in}}%
\pgfpathcurveto{\pgfqpoint{0.888199in}{1.508037in}}{\pgfqpoint{0.891471in}{1.515937in}}{\pgfqpoint{0.891471in}{1.524173in}}%
\pgfpathcurveto{\pgfqpoint{0.891471in}{1.532410in}}{\pgfqpoint{0.888199in}{1.540310in}}{\pgfqpoint{0.882375in}{1.546134in}}%
\pgfpathcurveto{\pgfqpoint{0.876551in}{1.551958in}}{\pgfqpoint{0.868651in}{1.555230in}}{\pgfqpoint{0.860415in}{1.555230in}}%
\pgfpathcurveto{\pgfqpoint{0.852178in}{1.555230in}}{\pgfqpoint{0.844278in}{1.551958in}}{\pgfqpoint{0.838454in}{1.546134in}}%
\pgfpathcurveto{\pgfqpoint{0.832631in}{1.540310in}}{\pgfqpoint{0.829358in}{1.532410in}}{\pgfqpoint{0.829358in}{1.524173in}}%
\pgfpathcurveto{\pgfqpoint{0.829358in}{1.515937in}}{\pgfqpoint{0.832631in}{1.508037in}}{\pgfqpoint{0.838454in}{1.502213in}}%
\pgfpathcurveto{\pgfqpoint{0.844278in}{1.496389in}}{\pgfqpoint{0.852178in}{1.493117in}}{\pgfqpoint{0.860415in}{1.493117in}}%
\pgfpathclose%
\pgfusepath{stroke,fill}%
\end{pgfscope}%
\begin{pgfscope}%
\pgfpathrectangle{\pgfqpoint{0.100000in}{0.212622in}}{\pgfqpoint{3.696000in}{3.696000in}}%
\pgfusepath{clip}%
\pgfsetbuttcap%
\pgfsetroundjoin%
\definecolor{currentfill}{rgb}{0.121569,0.466667,0.705882}%
\pgfsetfillcolor{currentfill}%
\pgfsetfillopacity{0.627433}%
\pgfsetlinewidth{1.003750pt}%
\definecolor{currentstroke}{rgb}{0.121569,0.466667,0.705882}%
\pgfsetstrokecolor{currentstroke}%
\pgfsetstrokeopacity{0.627433}%
\pgfsetdash{}{0pt}%
\pgfpathmoveto{\pgfqpoint{0.860415in}{1.493117in}}%
\pgfpathcurveto{\pgfqpoint{0.868651in}{1.493117in}}{\pgfqpoint{0.876551in}{1.496389in}}{\pgfqpoint{0.882375in}{1.502213in}}%
\pgfpathcurveto{\pgfqpoint{0.888199in}{1.508037in}}{\pgfqpoint{0.891471in}{1.515937in}}{\pgfqpoint{0.891471in}{1.524173in}}%
\pgfpathcurveto{\pgfqpoint{0.891471in}{1.532410in}}{\pgfqpoint{0.888199in}{1.540310in}}{\pgfqpoint{0.882375in}{1.546134in}}%
\pgfpathcurveto{\pgfqpoint{0.876551in}{1.551958in}}{\pgfqpoint{0.868651in}{1.555230in}}{\pgfqpoint{0.860415in}{1.555230in}}%
\pgfpathcurveto{\pgfqpoint{0.852178in}{1.555230in}}{\pgfqpoint{0.844278in}{1.551958in}}{\pgfqpoint{0.838454in}{1.546134in}}%
\pgfpathcurveto{\pgfqpoint{0.832631in}{1.540310in}}{\pgfqpoint{0.829358in}{1.532410in}}{\pgfqpoint{0.829358in}{1.524173in}}%
\pgfpathcurveto{\pgfqpoint{0.829358in}{1.515937in}}{\pgfqpoint{0.832631in}{1.508037in}}{\pgfqpoint{0.838454in}{1.502213in}}%
\pgfpathcurveto{\pgfqpoint{0.844278in}{1.496389in}}{\pgfqpoint{0.852178in}{1.493117in}}{\pgfqpoint{0.860415in}{1.493117in}}%
\pgfpathclose%
\pgfusepath{stroke,fill}%
\end{pgfscope}%
\begin{pgfscope}%
\pgfpathrectangle{\pgfqpoint{0.100000in}{0.212622in}}{\pgfqpoint{3.696000in}{3.696000in}}%
\pgfusepath{clip}%
\pgfsetbuttcap%
\pgfsetroundjoin%
\definecolor{currentfill}{rgb}{0.121569,0.466667,0.705882}%
\pgfsetfillcolor{currentfill}%
\pgfsetfillopacity{0.627433}%
\pgfsetlinewidth{1.003750pt}%
\definecolor{currentstroke}{rgb}{0.121569,0.466667,0.705882}%
\pgfsetstrokecolor{currentstroke}%
\pgfsetstrokeopacity{0.627433}%
\pgfsetdash{}{0pt}%
\pgfpathmoveto{\pgfqpoint{0.860415in}{1.493117in}}%
\pgfpathcurveto{\pgfqpoint{0.868651in}{1.493117in}}{\pgfqpoint{0.876551in}{1.496389in}}{\pgfqpoint{0.882375in}{1.502213in}}%
\pgfpathcurveto{\pgfqpoint{0.888199in}{1.508037in}}{\pgfqpoint{0.891471in}{1.515937in}}{\pgfqpoint{0.891471in}{1.524173in}}%
\pgfpathcurveto{\pgfqpoint{0.891471in}{1.532410in}}{\pgfqpoint{0.888199in}{1.540310in}}{\pgfqpoint{0.882375in}{1.546134in}}%
\pgfpathcurveto{\pgfqpoint{0.876551in}{1.551958in}}{\pgfqpoint{0.868651in}{1.555230in}}{\pgfqpoint{0.860415in}{1.555230in}}%
\pgfpathcurveto{\pgfqpoint{0.852178in}{1.555230in}}{\pgfqpoint{0.844278in}{1.551958in}}{\pgfqpoint{0.838454in}{1.546134in}}%
\pgfpathcurveto{\pgfqpoint{0.832631in}{1.540310in}}{\pgfqpoint{0.829358in}{1.532410in}}{\pgfqpoint{0.829358in}{1.524173in}}%
\pgfpathcurveto{\pgfqpoint{0.829358in}{1.515937in}}{\pgfqpoint{0.832631in}{1.508037in}}{\pgfqpoint{0.838454in}{1.502213in}}%
\pgfpathcurveto{\pgfqpoint{0.844278in}{1.496389in}}{\pgfqpoint{0.852178in}{1.493117in}}{\pgfqpoint{0.860415in}{1.493117in}}%
\pgfpathclose%
\pgfusepath{stroke,fill}%
\end{pgfscope}%
\begin{pgfscope}%
\pgfpathrectangle{\pgfqpoint{0.100000in}{0.212622in}}{\pgfqpoint{3.696000in}{3.696000in}}%
\pgfusepath{clip}%
\pgfsetbuttcap%
\pgfsetroundjoin%
\definecolor{currentfill}{rgb}{0.121569,0.466667,0.705882}%
\pgfsetfillcolor{currentfill}%
\pgfsetfillopacity{0.627433}%
\pgfsetlinewidth{1.003750pt}%
\definecolor{currentstroke}{rgb}{0.121569,0.466667,0.705882}%
\pgfsetstrokecolor{currentstroke}%
\pgfsetstrokeopacity{0.627433}%
\pgfsetdash{}{0pt}%
\pgfpathmoveto{\pgfqpoint{0.860415in}{1.493117in}}%
\pgfpathcurveto{\pgfqpoint{0.868651in}{1.493117in}}{\pgfqpoint{0.876551in}{1.496389in}}{\pgfqpoint{0.882375in}{1.502213in}}%
\pgfpathcurveto{\pgfqpoint{0.888199in}{1.508037in}}{\pgfqpoint{0.891471in}{1.515937in}}{\pgfqpoint{0.891471in}{1.524173in}}%
\pgfpathcurveto{\pgfqpoint{0.891471in}{1.532410in}}{\pgfqpoint{0.888199in}{1.540310in}}{\pgfqpoint{0.882375in}{1.546134in}}%
\pgfpathcurveto{\pgfqpoint{0.876551in}{1.551958in}}{\pgfqpoint{0.868651in}{1.555230in}}{\pgfqpoint{0.860415in}{1.555230in}}%
\pgfpathcurveto{\pgfqpoint{0.852178in}{1.555230in}}{\pgfqpoint{0.844278in}{1.551958in}}{\pgfqpoint{0.838454in}{1.546134in}}%
\pgfpathcurveto{\pgfqpoint{0.832631in}{1.540310in}}{\pgfqpoint{0.829358in}{1.532410in}}{\pgfqpoint{0.829358in}{1.524173in}}%
\pgfpathcurveto{\pgfqpoint{0.829358in}{1.515937in}}{\pgfqpoint{0.832631in}{1.508037in}}{\pgfqpoint{0.838454in}{1.502213in}}%
\pgfpathcurveto{\pgfqpoint{0.844278in}{1.496389in}}{\pgfqpoint{0.852178in}{1.493117in}}{\pgfqpoint{0.860415in}{1.493117in}}%
\pgfpathclose%
\pgfusepath{stroke,fill}%
\end{pgfscope}%
\begin{pgfscope}%
\pgfpathrectangle{\pgfqpoint{0.100000in}{0.212622in}}{\pgfqpoint{3.696000in}{3.696000in}}%
\pgfusepath{clip}%
\pgfsetbuttcap%
\pgfsetroundjoin%
\definecolor{currentfill}{rgb}{0.121569,0.466667,0.705882}%
\pgfsetfillcolor{currentfill}%
\pgfsetfillopacity{0.627433}%
\pgfsetlinewidth{1.003750pt}%
\definecolor{currentstroke}{rgb}{0.121569,0.466667,0.705882}%
\pgfsetstrokecolor{currentstroke}%
\pgfsetstrokeopacity{0.627433}%
\pgfsetdash{}{0pt}%
\pgfpathmoveto{\pgfqpoint{0.860415in}{1.493117in}}%
\pgfpathcurveto{\pgfqpoint{0.868651in}{1.493117in}}{\pgfqpoint{0.876551in}{1.496389in}}{\pgfqpoint{0.882375in}{1.502213in}}%
\pgfpathcurveto{\pgfqpoint{0.888199in}{1.508037in}}{\pgfqpoint{0.891471in}{1.515937in}}{\pgfqpoint{0.891471in}{1.524173in}}%
\pgfpathcurveto{\pgfqpoint{0.891471in}{1.532410in}}{\pgfqpoint{0.888199in}{1.540310in}}{\pgfqpoint{0.882375in}{1.546134in}}%
\pgfpathcurveto{\pgfqpoint{0.876551in}{1.551958in}}{\pgfqpoint{0.868651in}{1.555230in}}{\pgfqpoint{0.860415in}{1.555230in}}%
\pgfpathcurveto{\pgfqpoint{0.852178in}{1.555230in}}{\pgfqpoint{0.844278in}{1.551958in}}{\pgfqpoint{0.838454in}{1.546134in}}%
\pgfpathcurveto{\pgfqpoint{0.832631in}{1.540310in}}{\pgfqpoint{0.829358in}{1.532410in}}{\pgfqpoint{0.829358in}{1.524173in}}%
\pgfpathcurveto{\pgfqpoint{0.829358in}{1.515937in}}{\pgfqpoint{0.832631in}{1.508037in}}{\pgfqpoint{0.838454in}{1.502213in}}%
\pgfpathcurveto{\pgfqpoint{0.844278in}{1.496389in}}{\pgfqpoint{0.852178in}{1.493117in}}{\pgfqpoint{0.860415in}{1.493117in}}%
\pgfpathclose%
\pgfusepath{stroke,fill}%
\end{pgfscope}%
\begin{pgfscope}%
\pgfpathrectangle{\pgfqpoint{0.100000in}{0.212622in}}{\pgfqpoint{3.696000in}{3.696000in}}%
\pgfusepath{clip}%
\pgfsetbuttcap%
\pgfsetroundjoin%
\definecolor{currentfill}{rgb}{0.121569,0.466667,0.705882}%
\pgfsetfillcolor{currentfill}%
\pgfsetfillopacity{0.627433}%
\pgfsetlinewidth{1.003750pt}%
\definecolor{currentstroke}{rgb}{0.121569,0.466667,0.705882}%
\pgfsetstrokecolor{currentstroke}%
\pgfsetstrokeopacity{0.627433}%
\pgfsetdash{}{0pt}%
\pgfpathmoveto{\pgfqpoint{0.860415in}{1.493117in}}%
\pgfpathcurveto{\pgfqpoint{0.868651in}{1.493117in}}{\pgfqpoint{0.876551in}{1.496389in}}{\pgfqpoint{0.882375in}{1.502213in}}%
\pgfpathcurveto{\pgfqpoint{0.888199in}{1.508037in}}{\pgfqpoint{0.891471in}{1.515937in}}{\pgfqpoint{0.891471in}{1.524173in}}%
\pgfpathcurveto{\pgfqpoint{0.891471in}{1.532410in}}{\pgfqpoint{0.888199in}{1.540310in}}{\pgfqpoint{0.882375in}{1.546134in}}%
\pgfpathcurveto{\pgfqpoint{0.876551in}{1.551958in}}{\pgfqpoint{0.868651in}{1.555230in}}{\pgfqpoint{0.860415in}{1.555230in}}%
\pgfpathcurveto{\pgfqpoint{0.852178in}{1.555230in}}{\pgfqpoint{0.844278in}{1.551958in}}{\pgfqpoint{0.838454in}{1.546134in}}%
\pgfpathcurveto{\pgfqpoint{0.832631in}{1.540310in}}{\pgfqpoint{0.829358in}{1.532410in}}{\pgfqpoint{0.829358in}{1.524173in}}%
\pgfpathcurveto{\pgfqpoint{0.829358in}{1.515937in}}{\pgfqpoint{0.832631in}{1.508037in}}{\pgfqpoint{0.838454in}{1.502213in}}%
\pgfpathcurveto{\pgfqpoint{0.844278in}{1.496389in}}{\pgfqpoint{0.852178in}{1.493117in}}{\pgfqpoint{0.860415in}{1.493117in}}%
\pgfpathclose%
\pgfusepath{stroke,fill}%
\end{pgfscope}%
\begin{pgfscope}%
\pgfpathrectangle{\pgfqpoint{0.100000in}{0.212622in}}{\pgfqpoint{3.696000in}{3.696000in}}%
\pgfusepath{clip}%
\pgfsetbuttcap%
\pgfsetroundjoin%
\definecolor{currentfill}{rgb}{0.121569,0.466667,0.705882}%
\pgfsetfillcolor{currentfill}%
\pgfsetfillopacity{0.627433}%
\pgfsetlinewidth{1.003750pt}%
\definecolor{currentstroke}{rgb}{0.121569,0.466667,0.705882}%
\pgfsetstrokecolor{currentstroke}%
\pgfsetstrokeopacity{0.627433}%
\pgfsetdash{}{0pt}%
\pgfpathmoveto{\pgfqpoint{0.860415in}{1.493117in}}%
\pgfpathcurveto{\pgfqpoint{0.868651in}{1.493117in}}{\pgfqpoint{0.876551in}{1.496389in}}{\pgfqpoint{0.882375in}{1.502213in}}%
\pgfpathcurveto{\pgfqpoint{0.888199in}{1.508037in}}{\pgfqpoint{0.891471in}{1.515937in}}{\pgfqpoint{0.891471in}{1.524173in}}%
\pgfpathcurveto{\pgfqpoint{0.891471in}{1.532410in}}{\pgfqpoint{0.888199in}{1.540310in}}{\pgfqpoint{0.882375in}{1.546134in}}%
\pgfpathcurveto{\pgfqpoint{0.876551in}{1.551958in}}{\pgfqpoint{0.868651in}{1.555230in}}{\pgfqpoint{0.860415in}{1.555230in}}%
\pgfpathcurveto{\pgfqpoint{0.852178in}{1.555230in}}{\pgfqpoint{0.844278in}{1.551958in}}{\pgfqpoint{0.838454in}{1.546134in}}%
\pgfpathcurveto{\pgfqpoint{0.832631in}{1.540310in}}{\pgfqpoint{0.829358in}{1.532410in}}{\pgfqpoint{0.829358in}{1.524173in}}%
\pgfpathcurveto{\pgfqpoint{0.829358in}{1.515937in}}{\pgfqpoint{0.832631in}{1.508037in}}{\pgfqpoint{0.838454in}{1.502213in}}%
\pgfpathcurveto{\pgfqpoint{0.844278in}{1.496389in}}{\pgfqpoint{0.852178in}{1.493117in}}{\pgfqpoint{0.860415in}{1.493117in}}%
\pgfpathclose%
\pgfusepath{stroke,fill}%
\end{pgfscope}%
\begin{pgfscope}%
\pgfpathrectangle{\pgfqpoint{0.100000in}{0.212622in}}{\pgfqpoint{3.696000in}{3.696000in}}%
\pgfusepath{clip}%
\pgfsetbuttcap%
\pgfsetroundjoin%
\definecolor{currentfill}{rgb}{0.121569,0.466667,0.705882}%
\pgfsetfillcolor{currentfill}%
\pgfsetfillopacity{0.627433}%
\pgfsetlinewidth{1.003750pt}%
\definecolor{currentstroke}{rgb}{0.121569,0.466667,0.705882}%
\pgfsetstrokecolor{currentstroke}%
\pgfsetstrokeopacity{0.627433}%
\pgfsetdash{}{0pt}%
\pgfpathmoveto{\pgfqpoint{0.860415in}{1.493117in}}%
\pgfpathcurveto{\pgfqpoint{0.868651in}{1.493117in}}{\pgfqpoint{0.876551in}{1.496389in}}{\pgfqpoint{0.882375in}{1.502213in}}%
\pgfpathcurveto{\pgfqpoint{0.888199in}{1.508037in}}{\pgfqpoint{0.891471in}{1.515937in}}{\pgfqpoint{0.891471in}{1.524173in}}%
\pgfpathcurveto{\pgfqpoint{0.891471in}{1.532410in}}{\pgfqpoint{0.888199in}{1.540310in}}{\pgfqpoint{0.882375in}{1.546134in}}%
\pgfpathcurveto{\pgfqpoint{0.876551in}{1.551958in}}{\pgfqpoint{0.868651in}{1.555230in}}{\pgfqpoint{0.860415in}{1.555230in}}%
\pgfpathcurveto{\pgfqpoint{0.852178in}{1.555230in}}{\pgfqpoint{0.844278in}{1.551958in}}{\pgfqpoint{0.838454in}{1.546134in}}%
\pgfpathcurveto{\pgfqpoint{0.832631in}{1.540310in}}{\pgfqpoint{0.829358in}{1.532410in}}{\pgfqpoint{0.829358in}{1.524173in}}%
\pgfpathcurveto{\pgfqpoint{0.829358in}{1.515937in}}{\pgfqpoint{0.832631in}{1.508037in}}{\pgfqpoint{0.838454in}{1.502213in}}%
\pgfpathcurveto{\pgfqpoint{0.844278in}{1.496389in}}{\pgfqpoint{0.852178in}{1.493117in}}{\pgfqpoint{0.860415in}{1.493117in}}%
\pgfpathclose%
\pgfusepath{stroke,fill}%
\end{pgfscope}%
\begin{pgfscope}%
\pgfpathrectangle{\pgfqpoint{0.100000in}{0.212622in}}{\pgfqpoint{3.696000in}{3.696000in}}%
\pgfusepath{clip}%
\pgfsetbuttcap%
\pgfsetroundjoin%
\definecolor{currentfill}{rgb}{0.121569,0.466667,0.705882}%
\pgfsetfillcolor{currentfill}%
\pgfsetfillopacity{0.627433}%
\pgfsetlinewidth{1.003750pt}%
\definecolor{currentstroke}{rgb}{0.121569,0.466667,0.705882}%
\pgfsetstrokecolor{currentstroke}%
\pgfsetstrokeopacity{0.627433}%
\pgfsetdash{}{0pt}%
\pgfpathmoveto{\pgfqpoint{0.860415in}{1.493117in}}%
\pgfpathcurveto{\pgfqpoint{0.868651in}{1.493117in}}{\pgfqpoint{0.876551in}{1.496389in}}{\pgfqpoint{0.882375in}{1.502213in}}%
\pgfpathcurveto{\pgfqpoint{0.888199in}{1.508037in}}{\pgfqpoint{0.891471in}{1.515937in}}{\pgfqpoint{0.891471in}{1.524173in}}%
\pgfpathcurveto{\pgfqpoint{0.891471in}{1.532410in}}{\pgfqpoint{0.888199in}{1.540310in}}{\pgfqpoint{0.882375in}{1.546134in}}%
\pgfpathcurveto{\pgfqpoint{0.876551in}{1.551958in}}{\pgfqpoint{0.868651in}{1.555230in}}{\pgfqpoint{0.860415in}{1.555230in}}%
\pgfpathcurveto{\pgfqpoint{0.852178in}{1.555230in}}{\pgfqpoint{0.844278in}{1.551958in}}{\pgfqpoint{0.838454in}{1.546134in}}%
\pgfpathcurveto{\pgfqpoint{0.832631in}{1.540310in}}{\pgfqpoint{0.829358in}{1.532410in}}{\pgfqpoint{0.829358in}{1.524173in}}%
\pgfpathcurveto{\pgfqpoint{0.829358in}{1.515937in}}{\pgfqpoint{0.832631in}{1.508037in}}{\pgfqpoint{0.838454in}{1.502213in}}%
\pgfpathcurveto{\pgfqpoint{0.844278in}{1.496389in}}{\pgfqpoint{0.852178in}{1.493117in}}{\pgfqpoint{0.860415in}{1.493117in}}%
\pgfpathclose%
\pgfusepath{stroke,fill}%
\end{pgfscope}%
\begin{pgfscope}%
\pgfpathrectangle{\pgfqpoint{0.100000in}{0.212622in}}{\pgfqpoint{3.696000in}{3.696000in}}%
\pgfusepath{clip}%
\pgfsetbuttcap%
\pgfsetroundjoin%
\definecolor{currentfill}{rgb}{0.121569,0.466667,0.705882}%
\pgfsetfillcolor{currentfill}%
\pgfsetfillopacity{0.627433}%
\pgfsetlinewidth{1.003750pt}%
\definecolor{currentstroke}{rgb}{0.121569,0.466667,0.705882}%
\pgfsetstrokecolor{currentstroke}%
\pgfsetstrokeopacity{0.627433}%
\pgfsetdash{}{0pt}%
\pgfpathmoveto{\pgfqpoint{0.860415in}{1.493117in}}%
\pgfpathcurveto{\pgfqpoint{0.868651in}{1.493117in}}{\pgfqpoint{0.876551in}{1.496389in}}{\pgfqpoint{0.882375in}{1.502213in}}%
\pgfpathcurveto{\pgfqpoint{0.888199in}{1.508037in}}{\pgfqpoint{0.891471in}{1.515937in}}{\pgfqpoint{0.891471in}{1.524173in}}%
\pgfpathcurveto{\pgfqpoint{0.891471in}{1.532410in}}{\pgfqpoint{0.888199in}{1.540310in}}{\pgfqpoint{0.882375in}{1.546134in}}%
\pgfpathcurveto{\pgfqpoint{0.876551in}{1.551958in}}{\pgfqpoint{0.868651in}{1.555230in}}{\pgfqpoint{0.860415in}{1.555230in}}%
\pgfpathcurveto{\pgfqpoint{0.852178in}{1.555230in}}{\pgfqpoint{0.844278in}{1.551958in}}{\pgfqpoint{0.838454in}{1.546134in}}%
\pgfpathcurveto{\pgfqpoint{0.832631in}{1.540310in}}{\pgfqpoint{0.829358in}{1.532410in}}{\pgfqpoint{0.829358in}{1.524173in}}%
\pgfpathcurveto{\pgfqpoint{0.829358in}{1.515937in}}{\pgfqpoint{0.832631in}{1.508037in}}{\pgfqpoint{0.838454in}{1.502213in}}%
\pgfpathcurveto{\pgfqpoint{0.844278in}{1.496389in}}{\pgfqpoint{0.852178in}{1.493117in}}{\pgfqpoint{0.860415in}{1.493117in}}%
\pgfpathclose%
\pgfusepath{stroke,fill}%
\end{pgfscope}%
\begin{pgfscope}%
\pgfpathrectangle{\pgfqpoint{0.100000in}{0.212622in}}{\pgfqpoint{3.696000in}{3.696000in}}%
\pgfusepath{clip}%
\pgfsetbuttcap%
\pgfsetroundjoin%
\definecolor{currentfill}{rgb}{0.121569,0.466667,0.705882}%
\pgfsetfillcolor{currentfill}%
\pgfsetfillopacity{0.627433}%
\pgfsetlinewidth{1.003750pt}%
\definecolor{currentstroke}{rgb}{0.121569,0.466667,0.705882}%
\pgfsetstrokecolor{currentstroke}%
\pgfsetstrokeopacity{0.627433}%
\pgfsetdash{}{0pt}%
\pgfpathmoveto{\pgfqpoint{0.860415in}{1.493117in}}%
\pgfpathcurveto{\pgfqpoint{0.868651in}{1.493117in}}{\pgfqpoint{0.876551in}{1.496389in}}{\pgfqpoint{0.882375in}{1.502213in}}%
\pgfpathcurveto{\pgfqpoint{0.888199in}{1.508037in}}{\pgfqpoint{0.891471in}{1.515937in}}{\pgfqpoint{0.891471in}{1.524173in}}%
\pgfpathcurveto{\pgfqpoint{0.891471in}{1.532410in}}{\pgfqpoint{0.888199in}{1.540310in}}{\pgfqpoint{0.882375in}{1.546134in}}%
\pgfpathcurveto{\pgfqpoint{0.876551in}{1.551958in}}{\pgfqpoint{0.868651in}{1.555230in}}{\pgfqpoint{0.860415in}{1.555230in}}%
\pgfpathcurveto{\pgfqpoint{0.852178in}{1.555230in}}{\pgfqpoint{0.844278in}{1.551958in}}{\pgfqpoint{0.838454in}{1.546134in}}%
\pgfpathcurveto{\pgfqpoint{0.832631in}{1.540310in}}{\pgfqpoint{0.829358in}{1.532410in}}{\pgfqpoint{0.829358in}{1.524173in}}%
\pgfpathcurveto{\pgfqpoint{0.829358in}{1.515937in}}{\pgfqpoint{0.832631in}{1.508037in}}{\pgfqpoint{0.838454in}{1.502213in}}%
\pgfpathcurveto{\pgfqpoint{0.844278in}{1.496389in}}{\pgfqpoint{0.852178in}{1.493117in}}{\pgfqpoint{0.860415in}{1.493117in}}%
\pgfpathclose%
\pgfusepath{stroke,fill}%
\end{pgfscope}%
\begin{pgfscope}%
\pgfpathrectangle{\pgfqpoint{0.100000in}{0.212622in}}{\pgfqpoint{3.696000in}{3.696000in}}%
\pgfusepath{clip}%
\pgfsetbuttcap%
\pgfsetroundjoin%
\definecolor{currentfill}{rgb}{0.121569,0.466667,0.705882}%
\pgfsetfillcolor{currentfill}%
\pgfsetfillopacity{0.627433}%
\pgfsetlinewidth{1.003750pt}%
\definecolor{currentstroke}{rgb}{0.121569,0.466667,0.705882}%
\pgfsetstrokecolor{currentstroke}%
\pgfsetstrokeopacity{0.627433}%
\pgfsetdash{}{0pt}%
\pgfpathmoveto{\pgfqpoint{0.860415in}{1.493117in}}%
\pgfpathcurveto{\pgfqpoint{0.868651in}{1.493117in}}{\pgfqpoint{0.876551in}{1.496389in}}{\pgfqpoint{0.882375in}{1.502213in}}%
\pgfpathcurveto{\pgfqpoint{0.888199in}{1.508037in}}{\pgfqpoint{0.891471in}{1.515937in}}{\pgfqpoint{0.891471in}{1.524173in}}%
\pgfpathcurveto{\pgfqpoint{0.891471in}{1.532410in}}{\pgfqpoint{0.888199in}{1.540310in}}{\pgfqpoint{0.882375in}{1.546134in}}%
\pgfpathcurveto{\pgfqpoint{0.876551in}{1.551958in}}{\pgfqpoint{0.868651in}{1.555230in}}{\pgfqpoint{0.860415in}{1.555230in}}%
\pgfpathcurveto{\pgfqpoint{0.852178in}{1.555230in}}{\pgfqpoint{0.844278in}{1.551958in}}{\pgfqpoint{0.838454in}{1.546134in}}%
\pgfpathcurveto{\pgfqpoint{0.832631in}{1.540310in}}{\pgfqpoint{0.829358in}{1.532410in}}{\pgfqpoint{0.829358in}{1.524173in}}%
\pgfpathcurveto{\pgfqpoint{0.829358in}{1.515937in}}{\pgfqpoint{0.832631in}{1.508037in}}{\pgfqpoint{0.838454in}{1.502213in}}%
\pgfpathcurveto{\pgfqpoint{0.844278in}{1.496389in}}{\pgfqpoint{0.852178in}{1.493117in}}{\pgfqpoint{0.860415in}{1.493117in}}%
\pgfpathclose%
\pgfusepath{stroke,fill}%
\end{pgfscope}%
\begin{pgfscope}%
\pgfpathrectangle{\pgfqpoint{0.100000in}{0.212622in}}{\pgfqpoint{3.696000in}{3.696000in}}%
\pgfusepath{clip}%
\pgfsetbuttcap%
\pgfsetroundjoin%
\definecolor{currentfill}{rgb}{0.121569,0.466667,0.705882}%
\pgfsetfillcolor{currentfill}%
\pgfsetfillopacity{0.627433}%
\pgfsetlinewidth{1.003750pt}%
\definecolor{currentstroke}{rgb}{0.121569,0.466667,0.705882}%
\pgfsetstrokecolor{currentstroke}%
\pgfsetstrokeopacity{0.627433}%
\pgfsetdash{}{0pt}%
\pgfpathmoveto{\pgfqpoint{0.860415in}{1.493117in}}%
\pgfpathcurveto{\pgfqpoint{0.868651in}{1.493117in}}{\pgfqpoint{0.876551in}{1.496389in}}{\pgfqpoint{0.882375in}{1.502213in}}%
\pgfpathcurveto{\pgfqpoint{0.888199in}{1.508037in}}{\pgfqpoint{0.891471in}{1.515937in}}{\pgfqpoint{0.891471in}{1.524173in}}%
\pgfpathcurveto{\pgfqpoint{0.891471in}{1.532410in}}{\pgfqpoint{0.888199in}{1.540310in}}{\pgfqpoint{0.882375in}{1.546134in}}%
\pgfpathcurveto{\pgfqpoint{0.876551in}{1.551958in}}{\pgfqpoint{0.868651in}{1.555230in}}{\pgfqpoint{0.860415in}{1.555230in}}%
\pgfpathcurveto{\pgfqpoint{0.852178in}{1.555230in}}{\pgfqpoint{0.844278in}{1.551958in}}{\pgfqpoint{0.838454in}{1.546134in}}%
\pgfpathcurveto{\pgfqpoint{0.832631in}{1.540310in}}{\pgfqpoint{0.829358in}{1.532410in}}{\pgfqpoint{0.829358in}{1.524173in}}%
\pgfpathcurveto{\pgfqpoint{0.829358in}{1.515937in}}{\pgfqpoint{0.832631in}{1.508037in}}{\pgfqpoint{0.838454in}{1.502213in}}%
\pgfpathcurveto{\pgfqpoint{0.844278in}{1.496389in}}{\pgfqpoint{0.852178in}{1.493117in}}{\pgfqpoint{0.860415in}{1.493117in}}%
\pgfpathclose%
\pgfusepath{stroke,fill}%
\end{pgfscope}%
\begin{pgfscope}%
\pgfpathrectangle{\pgfqpoint{0.100000in}{0.212622in}}{\pgfqpoint{3.696000in}{3.696000in}}%
\pgfusepath{clip}%
\pgfsetbuttcap%
\pgfsetroundjoin%
\definecolor{currentfill}{rgb}{0.121569,0.466667,0.705882}%
\pgfsetfillcolor{currentfill}%
\pgfsetfillopacity{0.627433}%
\pgfsetlinewidth{1.003750pt}%
\definecolor{currentstroke}{rgb}{0.121569,0.466667,0.705882}%
\pgfsetstrokecolor{currentstroke}%
\pgfsetstrokeopacity{0.627433}%
\pgfsetdash{}{0pt}%
\pgfpathmoveto{\pgfqpoint{0.860415in}{1.493117in}}%
\pgfpathcurveto{\pgfqpoint{0.868651in}{1.493117in}}{\pgfqpoint{0.876551in}{1.496389in}}{\pgfqpoint{0.882375in}{1.502213in}}%
\pgfpathcurveto{\pgfqpoint{0.888199in}{1.508037in}}{\pgfqpoint{0.891471in}{1.515937in}}{\pgfqpoint{0.891471in}{1.524173in}}%
\pgfpathcurveto{\pgfqpoint{0.891471in}{1.532410in}}{\pgfqpoint{0.888199in}{1.540310in}}{\pgfqpoint{0.882375in}{1.546134in}}%
\pgfpathcurveto{\pgfqpoint{0.876551in}{1.551958in}}{\pgfqpoint{0.868651in}{1.555230in}}{\pgfqpoint{0.860415in}{1.555230in}}%
\pgfpathcurveto{\pgfqpoint{0.852178in}{1.555230in}}{\pgfqpoint{0.844278in}{1.551958in}}{\pgfqpoint{0.838454in}{1.546134in}}%
\pgfpathcurveto{\pgfqpoint{0.832631in}{1.540310in}}{\pgfqpoint{0.829358in}{1.532410in}}{\pgfqpoint{0.829358in}{1.524173in}}%
\pgfpathcurveto{\pgfqpoint{0.829358in}{1.515937in}}{\pgfqpoint{0.832631in}{1.508037in}}{\pgfqpoint{0.838454in}{1.502213in}}%
\pgfpathcurveto{\pgfqpoint{0.844278in}{1.496389in}}{\pgfqpoint{0.852178in}{1.493117in}}{\pgfqpoint{0.860415in}{1.493117in}}%
\pgfpathclose%
\pgfusepath{stroke,fill}%
\end{pgfscope}%
\begin{pgfscope}%
\pgfpathrectangle{\pgfqpoint{0.100000in}{0.212622in}}{\pgfqpoint{3.696000in}{3.696000in}}%
\pgfusepath{clip}%
\pgfsetbuttcap%
\pgfsetroundjoin%
\definecolor{currentfill}{rgb}{0.121569,0.466667,0.705882}%
\pgfsetfillcolor{currentfill}%
\pgfsetfillopacity{0.627433}%
\pgfsetlinewidth{1.003750pt}%
\definecolor{currentstroke}{rgb}{0.121569,0.466667,0.705882}%
\pgfsetstrokecolor{currentstroke}%
\pgfsetstrokeopacity{0.627433}%
\pgfsetdash{}{0pt}%
\pgfpathmoveto{\pgfqpoint{0.860415in}{1.493117in}}%
\pgfpathcurveto{\pgfqpoint{0.868651in}{1.493117in}}{\pgfqpoint{0.876551in}{1.496389in}}{\pgfqpoint{0.882375in}{1.502213in}}%
\pgfpathcurveto{\pgfqpoint{0.888199in}{1.508037in}}{\pgfqpoint{0.891471in}{1.515937in}}{\pgfqpoint{0.891471in}{1.524173in}}%
\pgfpathcurveto{\pgfqpoint{0.891471in}{1.532410in}}{\pgfqpoint{0.888199in}{1.540310in}}{\pgfqpoint{0.882375in}{1.546134in}}%
\pgfpathcurveto{\pgfqpoint{0.876551in}{1.551958in}}{\pgfqpoint{0.868651in}{1.555230in}}{\pgfqpoint{0.860415in}{1.555230in}}%
\pgfpathcurveto{\pgfqpoint{0.852178in}{1.555230in}}{\pgfqpoint{0.844278in}{1.551958in}}{\pgfqpoint{0.838454in}{1.546134in}}%
\pgfpathcurveto{\pgfqpoint{0.832631in}{1.540310in}}{\pgfqpoint{0.829358in}{1.532410in}}{\pgfqpoint{0.829358in}{1.524173in}}%
\pgfpathcurveto{\pgfqpoint{0.829358in}{1.515937in}}{\pgfqpoint{0.832631in}{1.508037in}}{\pgfqpoint{0.838454in}{1.502213in}}%
\pgfpathcurveto{\pgfqpoint{0.844278in}{1.496389in}}{\pgfqpoint{0.852178in}{1.493117in}}{\pgfqpoint{0.860415in}{1.493117in}}%
\pgfpathclose%
\pgfusepath{stroke,fill}%
\end{pgfscope}%
\begin{pgfscope}%
\pgfpathrectangle{\pgfqpoint{0.100000in}{0.212622in}}{\pgfqpoint{3.696000in}{3.696000in}}%
\pgfusepath{clip}%
\pgfsetbuttcap%
\pgfsetroundjoin%
\definecolor{currentfill}{rgb}{0.121569,0.466667,0.705882}%
\pgfsetfillcolor{currentfill}%
\pgfsetfillopacity{0.627433}%
\pgfsetlinewidth{1.003750pt}%
\definecolor{currentstroke}{rgb}{0.121569,0.466667,0.705882}%
\pgfsetstrokecolor{currentstroke}%
\pgfsetstrokeopacity{0.627433}%
\pgfsetdash{}{0pt}%
\pgfpathmoveto{\pgfqpoint{0.860415in}{1.493117in}}%
\pgfpathcurveto{\pgfqpoint{0.868651in}{1.493117in}}{\pgfqpoint{0.876551in}{1.496389in}}{\pgfqpoint{0.882375in}{1.502213in}}%
\pgfpathcurveto{\pgfqpoint{0.888199in}{1.508037in}}{\pgfqpoint{0.891471in}{1.515937in}}{\pgfqpoint{0.891471in}{1.524173in}}%
\pgfpathcurveto{\pgfqpoint{0.891471in}{1.532410in}}{\pgfqpoint{0.888199in}{1.540310in}}{\pgfqpoint{0.882375in}{1.546134in}}%
\pgfpathcurveto{\pgfqpoint{0.876551in}{1.551958in}}{\pgfqpoint{0.868651in}{1.555230in}}{\pgfqpoint{0.860415in}{1.555230in}}%
\pgfpathcurveto{\pgfqpoint{0.852178in}{1.555230in}}{\pgfqpoint{0.844278in}{1.551958in}}{\pgfqpoint{0.838454in}{1.546134in}}%
\pgfpathcurveto{\pgfqpoint{0.832631in}{1.540310in}}{\pgfqpoint{0.829358in}{1.532410in}}{\pgfqpoint{0.829358in}{1.524173in}}%
\pgfpathcurveto{\pgfqpoint{0.829358in}{1.515937in}}{\pgfqpoint{0.832631in}{1.508037in}}{\pgfqpoint{0.838454in}{1.502213in}}%
\pgfpathcurveto{\pgfqpoint{0.844278in}{1.496389in}}{\pgfqpoint{0.852178in}{1.493117in}}{\pgfqpoint{0.860415in}{1.493117in}}%
\pgfpathclose%
\pgfusepath{stroke,fill}%
\end{pgfscope}%
\begin{pgfscope}%
\pgfpathrectangle{\pgfqpoint{0.100000in}{0.212622in}}{\pgfqpoint{3.696000in}{3.696000in}}%
\pgfusepath{clip}%
\pgfsetbuttcap%
\pgfsetroundjoin%
\definecolor{currentfill}{rgb}{0.121569,0.466667,0.705882}%
\pgfsetfillcolor{currentfill}%
\pgfsetfillopacity{0.627433}%
\pgfsetlinewidth{1.003750pt}%
\definecolor{currentstroke}{rgb}{0.121569,0.466667,0.705882}%
\pgfsetstrokecolor{currentstroke}%
\pgfsetstrokeopacity{0.627433}%
\pgfsetdash{}{0pt}%
\pgfpathmoveto{\pgfqpoint{0.860415in}{1.493117in}}%
\pgfpathcurveto{\pgfqpoint{0.868651in}{1.493117in}}{\pgfqpoint{0.876551in}{1.496389in}}{\pgfqpoint{0.882375in}{1.502213in}}%
\pgfpathcurveto{\pgfqpoint{0.888199in}{1.508037in}}{\pgfqpoint{0.891471in}{1.515937in}}{\pgfqpoint{0.891471in}{1.524173in}}%
\pgfpathcurveto{\pgfqpoint{0.891471in}{1.532410in}}{\pgfqpoint{0.888199in}{1.540310in}}{\pgfqpoint{0.882375in}{1.546134in}}%
\pgfpathcurveto{\pgfqpoint{0.876551in}{1.551958in}}{\pgfqpoint{0.868651in}{1.555230in}}{\pgfqpoint{0.860415in}{1.555230in}}%
\pgfpathcurveto{\pgfqpoint{0.852178in}{1.555230in}}{\pgfqpoint{0.844278in}{1.551958in}}{\pgfqpoint{0.838454in}{1.546134in}}%
\pgfpathcurveto{\pgfqpoint{0.832631in}{1.540310in}}{\pgfqpoint{0.829358in}{1.532410in}}{\pgfqpoint{0.829358in}{1.524173in}}%
\pgfpathcurveto{\pgfqpoint{0.829358in}{1.515937in}}{\pgfqpoint{0.832631in}{1.508037in}}{\pgfqpoint{0.838454in}{1.502213in}}%
\pgfpathcurveto{\pgfqpoint{0.844278in}{1.496389in}}{\pgfqpoint{0.852178in}{1.493117in}}{\pgfqpoint{0.860415in}{1.493117in}}%
\pgfpathclose%
\pgfusepath{stroke,fill}%
\end{pgfscope}%
\begin{pgfscope}%
\pgfpathrectangle{\pgfqpoint{0.100000in}{0.212622in}}{\pgfqpoint{3.696000in}{3.696000in}}%
\pgfusepath{clip}%
\pgfsetbuttcap%
\pgfsetroundjoin%
\definecolor{currentfill}{rgb}{0.121569,0.466667,0.705882}%
\pgfsetfillcolor{currentfill}%
\pgfsetfillopacity{0.627433}%
\pgfsetlinewidth{1.003750pt}%
\definecolor{currentstroke}{rgb}{0.121569,0.466667,0.705882}%
\pgfsetstrokecolor{currentstroke}%
\pgfsetstrokeopacity{0.627433}%
\pgfsetdash{}{0pt}%
\pgfpathmoveto{\pgfqpoint{0.860415in}{1.493117in}}%
\pgfpathcurveto{\pgfqpoint{0.868651in}{1.493117in}}{\pgfqpoint{0.876551in}{1.496389in}}{\pgfqpoint{0.882375in}{1.502213in}}%
\pgfpathcurveto{\pgfqpoint{0.888199in}{1.508037in}}{\pgfqpoint{0.891471in}{1.515937in}}{\pgfqpoint{0.891471in}{1.524173in}}%
\pgfpathcurveto{\pgfqpoint{0.891471in}{1.532410in}}{\pgfqpoint{0.888199in}{1.540310in}}{\pgfqpoint{0.882375in}{1.546134in}}%
\pgfpathcurveto{\pgfqpoint{0.876551in}{1.551958in}}{\pgfqpoint{0.868651in}{1.555230in}}{\pgfqpoint{0.860415in}{1.555230in}}%
\pgfpathcurveto{\pgfqpoint{0.852178in}{1.555230in}}{\pgfqpoint{0.844278in}{1.551958in}}{\pgfqpoint{0.838454in}{1.546134in}}%
\pgfpathcurveto{\pgfqpoint{0.832631in}{1.540310in}}{\pgfqpoint{0.829358in}{1.532410in}}{\pgfqpoint{0.829358in}{1.524173in}}%
\pgfpathcurveto{\pgfqpoint{0.829358in}{1.515937in}}{\pgfqpoint{0.832631in}{1.508037in}}{\pgfqpoint{0.838454in}{1.502213in}}%
\pgfpathcurveto{\pgfqpoint{0.844278in}{1.496389in}}{\pgfqpoint{0.852178in}{1.493117in}}{\pgfqpoint{0.860415in}{1.493117in}}%
\pgfpathclose%
\pgfusepath{stroke,fill}%
\end{pgfscope}%
\begin{pgfscope}%
\pgfpathrectangle{\pgfqpoint{0.100000in}{0.212622in}}{\pgfqpoint{3.696000in}{3.696000in}}%
\pgfusepath{clip}%
\pgfsetbuttcap%
\pgfsetroundjoin%
\definecolor{currentfill}{rgb}{0.121569,0.466667,0.705882}%
\pgfsetfillcolor{currentfill}%
\pgfsetfillopacity{0.627433}%
\pgfsetlinewidth{1.003750pt}%
\definecolor{currentstroke}{rgb}{0.121569,0.466667,0.705882}%
\pgfsetstrokecolor{currentstroke}%
\pgfsetstrokeopacity{0.627433}%
\pgfsetdash{}{0pt}%
\pgfpathmoveto{\pgfqpoint{0.860415in}{1.493117in}}%
\pgfpathcurveto{\pgfqpoint{0.868651in}{1.493117in}}{\pgfqpoint{0.876551in}{1.496389in}}{\pgfqpoint{0.882375in}{1.502213in}}%
\pgfpathcurveto{\pgfqpoint{0.888199in}{1.508037in}}{\pgfqpoint{0.891471in}{1.515937in}}{\pgfqpoint{0.891471in}{1.524173in}}%
\pgfpathcurveto{\pgfqpoint{0.891471in}{1.532410in}}{\pgfqpoint{0.888199in}{1.540310in}}{\pgfqpoint{0.882375in}{1.546134in}}%
\pgfpathcurveto{\pgfqpoint{0.876551in}{1.551958in}}{\pgfqpoint{0.868651in}{1.555230in}}{\pgfqpoint{0.860415in}{1.555230in}}%
\pgfpathcurveto{\pgfqpoint{0.852178in}{1.555230in}}{\pgfqpoint{0.844278in}{1.551958in}}{\pgfqpoint{0.838454in}{1.546134in}}%
\pgfpathcurveto{\pgfqpoint{0.832631in}{1.540310in}}{\pgfqpoint{0.829358in}{1.532410in}}{\pgfqpoint{0.829358in}{1.524173in}}%
\pgfpathcurveto{\pgfqpoint{0.829358in}{1.515937in}}{\pgfqpoint{0.832631in}{1.508037in}}{\pgfqpoint{0.838454in}{1.502213in}}%
\pgfpathcurveto{\pgfqpoint{0.844278in}{1.496389in}}{\pgfqpoint{0.852178in}{1.493117in}}{\pgfqpoint{0.860415in}{1.493117in}}%
\pgfpathclose%
\pgfusepath{stroke,fill}%
\end{pgfscope}%
\begin{pgfscope}%
\pgfpathrectangle{\pgfqpoint{0.100000in}{0.212622in}}{\pgfqpoint{3.696000in}{3.696000in}}%
\pgfusepath{clip}%
\pgfsetbuttcap%
\pgfsetroundjoin%
\definecolor{currentfill}{rgb}{0.121569,0.466667,0.705882}%
\pgfsetfillcolor{currentfill}%
\pgfsetfillopacity{0.627433}%
\pgfsetlinewidth{1.003750pt}%
\definecolor{currentstroke}{rgb}{0.121569,0.466667,0.705882}%
\pgfsetstrokecolor{currentstroke}%
\pgfsetstrokeopacity{0.627433}%
\pgfsetdash{}{0pt}%
\pgfpathmoveto{\pgfqpoint{0.860415in}{1.493117in}}%
\pgfpathcurveto{\pgfqpoint{0.868651in}{1.493117in}}{\pgfqpoint{0.876551in}{1.496389in}}{\pgfqpoint{0.882375in}{1.502213in}}%
\pgfpathcurveto{\pgfqpoint{0.888199in}{1.508037in}}{\pgfqpoint{0.891471in}{1.515937in}}{\pgfqpoint{0.891471in}{1.524173in}}%
\pgfpathcurveto{\pgfqpoint{0.891471in}{1.532410in}}{\pgfqpoint{0.888199in}{1.540310in}}{\pgfqpoint{0.882375in}{1.546134in}}%
\pgfpathcurveto{\pgfqpoint{0.876551in}{1.551958in}}{\pgfqpoint{0.868651in}{1.555230in}}{\pgfqpoint{0.860415in}{1.555230in}}%
\pgfpathcurveto{\pgfqpoint{0.852178in}{1.555230in}}{\pgfqpoint{0.844278in}{1.551958in}}{\pgfqpoint{0.838454in}{1.546134in}}%
\pgfpathcurveto{\pgfqpoint{0.832631in}{1.540310in}}{\pgfqpoint{0.829358in}{1.532410in}}{\pgfqpoint{0.829358in}{1.524173in}}%
\pgfpathcurveto{\pgfqpoint{0.829358in}{1.515937in}}{\pgfqpoint{0.832631in}{1.508037in}}{\pgfqpoint{0.838454in}{1.502213in}}%
\pgfpathcurveto{\pgfqpoint{0.844278in}{1.496389in}}{\pgfqpoint{0.852178in}{1.493117in}}{\pgfqpoint{0.860415in}{1.493117in}}%
\pgfpathclose%
\pgfusepath{stroke,fill}%
\end{pgfscope}%
\begin{pgfscope}%
\pgfpathrectangle{\pgfqpoint{0.100000in}{0.212622in}}{\pgfqpoint{3.696000in}{3.696000in}}%
\pgfusepath{clip}%
\pgfsetbuttcap%
\pgfsetroundjoin%
\definecolor{currentfill}{rgb}{0.121569,0.466667,0.705882}%
\pgfsetfillcolor{currentfill}%
\pgfsetfillopacity{0.627433}%
\pgfsetlinewidth{1.003750pt}%
\definecolor{currentstroke}{rgb}{0.121569,0.466667,0.705882}%
\pgfsetstrokecolor{currentstroke}%
\pgfsetstrokeopacity{0.627433}%
\pgfsetdash{}{0pt}%
\pgfpathmoveto{\pgfqpoint{0.860415in}{1.493117in}}%
\pgfpathcurveto{\pgfqpoint{0.868651in}{1.493117in}}{\pgfqpoint{0.876551in}{1.496389in}}{\pgfqpoint{0.882375in}{1.502213in}}%
\pgfpathcurveto{\pgfqpoint{0.888199in}{1.508037in}}{\pgfqpoint{0.891471in}{1.515937in}}{\pgfqpoint{0.891471in}{1.524173in}}%
\pgfpathcurveto{\pgfqpoint{0.891471in}{1.532410in}}{\pgfqpoint{0.888199in}{1.540310in}}{\pgfqpoint{0.882375in}{1.546134in}}%
\pgfpathcurveto{\pgfqpoint{0.876551in}{1.551958in}}{\pgfqpoint{0.868651in}{1.555230in}}{\pgfqpoint{0.860415in}{1.555230in}}%
\pgfpathcurveto{\pgfqpoint{0.852178in}{1.555230in}}{\pgfqpoint{0.844278in}{1.551958in}}{\pgfqpoint{0.838454in}{1.546134in}}%
\pgfpathcurveto{\pgfqpoint{0.832631in}{1.540310in}}{\pgfqpoint{0.829358in}{1.532410in}}{\pgfqpoint{0.829358in}{1.524173in}}%
\pgfpathcurveto{\pgfqpoint{0.829358in}{1.515937in}}{\pgfqpoint{0.832631in}{1.508037in}}{\pgfqpoint{0.838454in}{1.502213in}}%
\pgfpathcurveto{\pgfqpoint{0.844278in}{1.496389in}}{\pgfqpoint{0.852178in}{1.493117in}}{\pgfqpoint{0.860415in}{1.493117in}}%
\pgfpathclose%
\pgfusepath{stroke,fill}%
\end{pgfscope}%
\begin{pgfscope}%
\pgfpathrectangle{\pgfqpoint{0.100000in}{0.212622in}}{\pgfqpoint{3.696000in}{3.696000in}}%
\pgfusepath{clip}%
\pgfsetbuttcap%
\pgfsetroundjoin%
\definecolor{currentfill}{rgb}{0.121569,0.466667,0.705882}%
\pgfsetfillcolor{currentfill}%
\pgfsetfillopacity{0.627433}%
\pgfsetlinewidth{1.003750pt}%
\definecolor{currentstroke}{rgb}{0.121569,0.466667,0.705882}%
\pgfsetstrokecolor{currentstroke}%
\pgfsetstrokeopacity{0.627433}%
\pgfsetdash{}{0pt}%
\pgfpathmoveto{\pgfqpoint{0.860415in}{1.493117in}}%
\pgfpathcurveto{\pgfqpoint{0.868651in}{1.493117in}}{\pgfqpoint{0.876551in}{1.496389in}}{\pgfqpoint{0.882375in}{1.502213in}}%
\pgfpathcurveto{\pgfqpoint{0.888199in}{1.508037in}}{\pgfqpoint{0.891471in}{1.515937in}}{\pgfqpoint{0.891471in}{1.524173in}}%
\pgfpathcurveto{\pgfqpoint{0.891471in}{1.532410in}}{\pgfqpoint{0.888199in}{1.540310in}}{\pgfqpoint{0.882375in}{1.546134in}}%
\pgfpathcurveto{\pgfqpoint{0.876551in}{1.551958in}}{\pgfqpoint{0.868651in}{1.555230in}}{\pgfqpoint{0.860415in}{1.555230in}}%
\pgfpathcurveto{\pgfqpoint{0.852178in}{1.555230in}}{\pgfqpoint{0.844278in}{1.551958in}}{\pgfqpoint{0.838454in}{1.546134in}}%
\pgfpathcurveto{\pgfqpoint{0.832631in}{1.540310in}}{\pgfqpoint{0.829358in}{1.532410in}}{\pgfqpoint{0.829358in}{1.524173in}}%
\pgfpathcurveto{\pgfqpoint{0.829358in}{1.515937in}}{\pgfqpoint{0.832631in}{1.508037in}}{\pgfqpoint{0.838454in}{1.502213in}}%
\pgfpathcurveto{\pgfqpoint{0.844278in}{1.496389in}}{\pgfqpoint{0.852178in}{1.493117in}}{\pgfqpoint{0.860415in}{1.493117in}}%
\pgfpathclose%
\pgfusepath{stroke,fill}%
\end{pgfscope}%
\begin{pgfscope}%
\pgfpathrectangle{\pgfqpoint{0.100000in}{0.212622in}}{\pgfqpoint{3.696000in}{3.696000in}}%
\pgfusepath{clip}%
\pgfsetbuttcap%
\pgfsetroundjoin%
\definecolor{currentfill}{rgb}{0.121569,0.466667,0.705882}%
\pgfsetfillcolor{currentfill}%
\pgfsetfillopacity{0.627433}%
\pgfsetlinewidth{1.003750pt}%
\definecolor{currentstroke}{rgb}{0.121569,0.466667,0.705882}%
\pgfsetstrokecolor{currentstroke}%
\pgfsetstrokeopacity{0.627433}%
\pgfsetdash{}{0pt}%
\pgfpathmoveto{\pgfqpoint{0.860415in}{1.493117in}}%
\pgfpathcurveto{\pgfqpoint{0.868651in}{1.493117in}}{\pgfqpoint{0.876551in}{1.496389in}}{\pgfqpoint{0.882375in}{1.502213in}}%
\pgfpathcurveto{\pgfqpoint{0.888199in}{1.508037in}}{\pgfqpoint{0.891471in}{1.515937in}}{\pgfqpoint{0.891471in}{1.524173in}}%
\pgfpathcurveto{\pgfqpoint{0.891471in}{1.532410in}}{\pgfqpoint{0.888199in}{1.540310in}}{\pgfqpoint{0.882375in}{1.546134in}}%
\pgfpathcurveto{\pgfqpoint{0.876551in}{1.551958in}}{\pgfqpoint{0.868651in}{1.555230in}}{\pgfqpoint{0.860415in}{1.555230in}}%
\pgfpathcurveto{\pgfqpoint{0.852178in}{1.555230in}}{\pgfqpoint{0.844278in}{1.551958in}}{\pgfqpoint{0.838454in}{1.546134in}}%
\pgfpathcurveto{\pgfqpoint{0.832631in}{1.540310in}}{\pgfqpoint{0.829358in}{1.532410in}}{\pgfqpoint{0.829358in}{1.524173in}}%
\pgfpathcurveto{\pgfqpoint{0.829358in}{1.515937in}}{\pgfqpoint{0.832631in}{1.508037in}}{\pgfqpoint{0.838454in}{1.502213in}}%
\pgfpathcurveto{\pgfqpoint{0.844278in}{1.496389in}}{\pgfqpoint{0.852178in}{1.493117in}}{\pgfqpoint{0.860415in}{1.493117in}}%
\pgfpathclose%
\pgfusepath{stroke,fill}%
\end{pgfscope}%
\begin{pgfscope}%
\pgfpathrectangle{\pgfqpoint{0.100000in}{0.212622in}}{\pgfqpoint{3.696000in}{3.696000in}}%
\pgfusepath{clip}%
\pgfsetbuttcap%
\pgfsetroundjoin%
\definecolor{currentfill}{rgb}{0.121569,0.466667,0.705882}%
\pgfsetfillcolor{currentfill}%
\pgfsetfillopacity{0.627433}%
\pgfsetlinewidth{1.003750pt}%
\definecolor{currentstroke}{rgb}{0.121569,0.466667,0.705882}%
\pgfsetstrokecolor{currentstroke}%
\pgfsetstrokeopacity{0.627433}%
\pgfsetdash{}{0pt}%
\pgfpathmoveto{\pgfqpoint{0.860415in}{1.493117in}}%
\pgfpathcurveto{\pgfqpoint{0.868651in}{1.493117in}}{\pgfqpoint{0.876551in}{1.496389in}}{\pgfqpoint{0.882375in}{1.502213in}}%
\pgfpathcurveto{\pgfqpoint{0.888199in}{1.508037in}}{\pgfqpoint{0.891471in}{1.515937in}}{\pgfqpoint{0.891471in}{1.524173in}}%
\pgfpathcurveto{\pgfqpoint{0.891471in}{1.532410in}}{\pgfqpoint{0.888199in}{1.540310in}}{\pgfqpoint{0.882375in}{1.546134in}}%
\pgfpathcurveto{\pgfqpoint{0.876551in}{1.551958in}}{\pgfqpoint{0.868651in}{1.555230in}}{\pgfqpoint{0.860415in}{1.555230in}}%
\pgfpathcurveto{\pgfqpoint{0.852178in}{1.555230in}}{\pgfqpoint{0.844278in}{1.551958in}}{\pgfqpoint{0.838454in}{1.546134in}}%
\pgfpathcurveto{\pgfqpoint{0.832631in}{1.540310in}}{\pgfqpoint{0.829358in}{1.532410in}}{\pgfqpoint{0.829358in}{1.524173in}}%
\pgfpathcurveto{\pgfqpoint{0.829358in}{1.515937in}}{\pgfqpoint{0.832631in}{1.508037in}}{\pgfqpoint{0.838454in}{1.502213in}}%
\pgfpathcurveto{\pgfqpoint{0.844278in}{1.496389in}}{\pgfqpoint{0.852178in}{1.493117in}}{\pgfqpoint{0.860415in}{1.493117in}}%
\pgfpathclose%
\pgfusepath{stroke,fill}%
\end{pgfscope}%
\begin{pgfscope}%
\pgfpathrectangle{\pgfqpoint{0.100000in}{0.212622in}}{\pgfqpoint{3.696000in}{3.696000in}}%
\pgfusepath{clip}%
\pgfsetbuttcap%
\pgfsetroundjoin%
\definecolor{currentfill}{rgb}{0.121569,0.466667,0.705882}%
\pgfsetfillcolor{currentfill}%
\pgfsetfillopacity{0.627433}%
\pgfsetlinewidth{1.003750pt}%
\definecolor{currentstroke}{rgb}{0.121569,0.466667,0.705882}%
\pgfsetstrokecolor{currentstroke}%
\pgfsetstrokeopacity{0.627433}%
\pgfsetdash{}{0pt}%
\pgfpathmoveto{\pgfqpoint{0.860415in}{1.493117in}}%
\pgfpathcurveto{\pgfqpoint{0.868651in}{1.493117in}}{\pgfqpoint{0.876551in}{1.496389in}}{\pgfqpoint{0.882375in}{1.502213in}}%
\pgfpathcurveto{\pgfqpoint{0.888199in}{1.508037in}}{\pgfqpoint{0.891471in}{1.515937in}}{\pgfqpoint{0.891471in}{1.524173in}}%
\pgfpathcurveto{\pgfqpoint{0.891471in}{1.532410in}}{\pgfqpoint{0.888199in}{1.540310in}}{\pgfqpoint{0.882375in}{1.546134in}}%
\pgfpathcurveto{\pgfqpoint{0.876551in}{1.551958in}}{\pgfqpoint{0.868651in}{1.555230in}}{\pgfqpoint{0.860415in}{1.555230in}}%
\pgfpathcurveto{\pgfqpoint{0.852178in}{1.555230in}}{\pgfqpoint{0.844278in}{1.551958in}}{\pgfqpoint{0.838454in}{1.546134in}}%
\pgfpathcurveto{\pgfqpoint{0.832631in}{1.540310in}}{\pgfqpoint{0.829358in}{1.532410in}}{\pgfqpoint{0.829358in}{1.524173in}}%
\pgfpathcurveto{\pgfqpoint{0.829358in}{1.515937in}}{\pgfqpoint{0.832631in}{1.508037in}}{\pgfqpoint{0.838454in}{1.502213in}}%
\pgfpathcurveto{\pgfqpoint{0.844278in}{1.496389in}}{\pgfqpoint{0.852178in}{1.493117in}}{\pgfqpoint{0.860415in}{1.493117in}}%
\pgfpathclose%
\pgfusepath{stroke,fill}%
\end{pgfscope}%
\begin{pgfscope}%
\pgfpathrectangle{\pgfqpoint{0.100000in}{0.212622in}}{\pgfqpoint{3.696000in}{3.696000in}}%
\pgfusepath{clip}%
\pgfsetbuttcap%
\pgfsetroundjoin%
\definecolor{currentfill}{rgb}{0.121569,0.466667,0.705882}%
\pgfsetfillcolor{currentfill}%
\pgfsetfillopacity{0.627433}%
\pgfsetlinewidth{1.003750pt}%
\definecolor{currentstroke}{rgb}{0.121569,0.466667,0.705882}%
\pgfsetstrokecolor{currentstroke}%
\pgfsetstrokeopacity{0.627433}%
\pgfsetdash{}{0pt}%
\pgfpathmoveto{\pgfqpoint{0.860415in}{1.493117in}}%
\pgfpathcurveto{\pgfqpoint{0.868651in}{1.493117in}}{\pgfqpoint{0.876551in}{1.496389in}}{\pgfqpoint{0.882375in}{1.502213in}}%
\pgfpathcurveto{\pgfqpoint{0.888199in}{1.508037in}}{\pgfqpoint{0.891471in}{1.515937in}}{\pgfqpoint{0.891471in}{1.524173in}}%
\pgfpathcurveto{\pgfqpoint{0.891471in}{1.532410in}}{\pgfqpoint{0.888199in}{1.540310in}}{\pgfqpoint{0.882375in}{1.546134in}}%
\pgfpathcurveto{\pgfqpoint{0.876551in}{1.551958in}}{\pgfqpoint{0.868651in}{1.555230in}}{\pgfqpoint{0.860415in}{1.555230in}}%
\pgfpathcurveto{\pgfqpoint{0.852178in}{1.555230in}}{\pgfqpoint{0.844278in}{1.551958in}}{\pgfqpoint{0.838454in}{1.546134in}}%
\pgfpathcurveto{\pgfqpoint{0.832631in}{1.540310in}}{\pgfqpoint{0.829358in}{1.532410in}}{\pgfqpoint{0.829358in}{1.524173in}}%
\pgfpathcurveto{\pgfqpoint{0.829358in}{1.515937in}}{\pgfqpoint{0.832631in}{1.508037in}}{\pgfqpoint{0.838454in}{1.502213in}}%
\pgfpathcurveto{\pgfqpoint{0.844278in}{1.496389in}}{\pgfqpoint{0.852178in}{1.493117in}}{\pgfqpoint{0.860415in}{1.493117in}}%
\pgfpathclose%
\pgfusepath{stroke,fill}%
\end{pgfscope}%
\begin{pgfscope}%
\pgfpathrectangle{\pgfqpoint{0.100000in}{0.212622in}}{\pgfqpoint{3.696000in}{3.696000in}}%
\pgfusepath{clip}%
\pgfsetbuttcap%
\pgfsetroundjoin%
\definecolor{currentfill}{rgb}{0.121569,0.466667,0.705882}%
\pgfsetfillcolor{currentfill}%
\pgfsetfillopacity{0.627433}%
\pgfsetlinewidth{1.003750pt}%
\definecolor{currentstroke}{rgb}{0.121569,0.466667,0.705882}%
\pgfsetstrokecolor{currentstroke}%
\pgfsetstrokeopacity{0.627433}%
\pgfsetdash{}{0pt}%
\pgfpathmoveto{\pgfqpoint{0.860415in}{1.493117in}}%
\pgfpathcurveto{\pgfqpoint{0.868651in}{1.493117in}}{\pgfqpoint{0.876551in}{1.496389in}}{\pgfqpoint{0.882375in}{1.502213in}}%
\pgfpathcurveto{\pgfqpoint{0.888199in}{1.508037in}}{\pgfqpoint{0.891471in}{1.515937in}}{\pgfqpoint{0.891471in}{1.524173in}}%
\pgfpathcurveto{\pgfqpoint{0.891471in}{1.532410in}}{\pgfqpoint{0.888199in}{1.540310in}}{\pgfqpoint{0.882375in}{1.546134in}}%
\pgfpathcurveto{\pgfqpoint{0.876551in}{1.551958in}}{\pgfqpoint{0.868651in}{1.555230in}}{\pgfqpoint{0.860415in}{1.555230in}}%
\pgfpathcurveto{\pgfqpoint{0.852178in}{1.555230in}}{\pgfqpoint{0.844278in}{1.551958in}}{\pgfqpoint{0.838454in}{1.546134in}}%
\pgfpathcurveto{\pgfqpoint{0.832631in}{1.540310in}}{\pgfqpoint{0.829358in}{1.532410in}}{\pgfqpoint{0.829358in}{1.524173in}}%
\pgfpathcurveto{\pgfqpoint{0.829358in}{1.515937in}}{\pgfqpoint{0.832631in}{1.508037in}}{\pgfqpoint{0.838454in}{1.502213in}}%
\pgfpathcurveto{\pgfqpoint{0.844278in}{1.496389in}}{\pgfqpoint{0.852178in}{1.493117in}}{\pgfqpoint{0.860415in}{1.493117in}}%
\pgfpathclose%
\pgfusepath{stroke,fill}%
\end{pgfscope}%
\begin{pgfscope}%
\pgfpathrectangle{\pgfqpoint{0.100000in}{0.212622in}}{\pgfqpoint{3.696000in}{3.696000in}}%
\pgfusepath{clip}%
\pgfsetbuttcap%
\pgfsetroundjoin%
\definecolor{currentfill}{rgb}{0.121569,0.466667,0.705882}%
\pgfsetfillcolor{currentfill}%
\pgfsetfillopacity{0.627433}%
\pgfsetlinewidth{1.003750pt}%
\definecolor{currentstroke}{rgb}{0.121569,0.466667,0.705882}%
\pgfsetstrokecolor{currentstroke}%
\pgfsetstrokeopacity{0.627433}%
\pgfsetdash{}{0pt}%
\pgfpathmoveto{\pgfqpoint{0.860415in}{1.493117in}}%
\pgfpathcurveto{\pgfqpoint{0.868651in}{1.493117in}}{\pgfqpoint{0.876551in}{1.496389in}}{\pgfqpoint{0.882375in}{1.502213in}}%
\pgfpathcurveto{\pgfqpoint{0.888199in}{1.508037in}}{\pgfqpoint{0.891471in}{1.515937in}}{\pgfqpoint{0.891471in}{1.524173in}}%
\pgfpathcurveto{\pgfqpoint{0.891471in}{1.532410in}}{\pgfqpoint{0.888199in}{1.540310in}}{\pgfqpoint{0.882375in}{1.546134in}}%
\pgfpathcurveto{\pgfqpoint{0.876551in}{1.551958in}}{\pgfqpoint{0.868651in}{1.555230in}}{\pgfqpoint{0.860415in}{1.555230in}}%
\pgfpathcurveto{\pgfqpoint{0.852178in}{1.555230in}}{\pgfqpoint{0.844278in}{1.551958in}}{\pgfqpoint{0.838454in}{1.546134in}}%
\pgfpathcurveto{\pgfqpoint{0.832631in}{1.540310in}}{\pgfqpoint{0.829358in}{1.532410in}}{\pgfqpoint{0.829358in}{1.524173in}}%
\pgfpathcurveto{\pgfqpoint{0.829358in}{1.515937in}}{\pgfqpoint{0.832631in}{1.508037in}}{\pgfqpoint{0.838454in}{1.502213in}}%
\pgfpathcurveto{\pgfqpoint{0.844278in}{1.496389in}}{\pgfqpoint{0.852178in}{1.493117in}}{\pgfqpoint{0.860415in}{1.493117in}}%
\pgfpathclose%
\pgfusepath{stroke,fill}%
\end{pgfscope}%
\begin{pgfscope}%
\pgfpathrectangle{\pgfqpoint{0.100000in}{0.212622in}}{\pgfqpoint{3.696000in}{3.696000in}}%
\pgfusepath{clip}%
\pgfsetbuttcap%
\pgfsetroundjoin%
\definecolor{currentfill}{rgb}{0.121569,0.466667,0.705882}%
\pgfsetfillcolor{currentfill}%
\pgfsetfillopacity{0.627433}%
\pgfsetlinewidth{1.003750pt}%
\definecolor{currentstroke}{rgb}{0.121569,0.466667,0.705882}%
\pgfsetstrokecolor{currentstroke}%
\pgfsetstrokeopacity{0.627433}%
\pgfsetdash{}{0pt}%
\pgfpathmoveto{\pgfqpoint{0.860415in}{1.493117in}}%
\pgfpathcurveto{\pgfqpoint{0.868651in}{1.493117in}}{\pgfqpoint{0.876551in}{1.496389in}}{\pgfqpoint{0.882375in}{1.502213in}}%
\pgfpathcurveto{\pgfqpoint{0.888199in}{1.508037in}}{\pgfqpoint{0.891471in}{1.515937in}}{\pgfqpoint{0.891471in}{1.524173in}}%
\pgfpathcurveto{\pgfqpoint{0.891471in}{1.532410in}}{\pgfqpoint{0.888199in}{1.540310in}}{\pgfqpoint{0.882375in}{1.546134in}}%
\pgfpathcurveto{\pgfqpoint{0.876551in}{1.551958in}}{\pgfqpoint{0.868651in}{1.555230in}}{\pgfqpoint{0.860415in}{1.555230in}}%
\pgfpathcurveto{\pgfqpoint{0.852178in}{1.555230in}}{\pgfqpoint{0.844278in}{1.551958in}}{\pgfqpoint{0.838454in}{1.546134in}}%
\pgfpathcurveto{\pgfqpoint{0.832631in}{1.540310in}}{\pgfqpoint{0.829358in}{1.532410in}}{\pgfqpoint{0.829358in}{1.524173in}}%
\pgfpathcurveto{\pgfqpoint{0.829358in}{1.515937in}}{\pgfqpoint{0.832631in}{1.508037in}}{\pgfqpoint{0.838454in}{1.502213in}}%
\pgfpathcurveto{\pgfqpoint{0.844278in}{1.496389in}}{\pgfqpoint{0.852178in}{1.493117in}}{\pgfqpoint{0.860415in}{1.493117in}}%
\pgfpathclose%
\pgfusepath{stroke,fill}%
\end{pgfscope}%
\begin{pgfscope}%
\pgfpathrectangle{\pgfqpoint{0.100000in}{0.212622in}}{\pgfqpoint{3.696000in}{3.696000in}}%
\pgfusepath{clip}%
\pgfsetbuttcap%
\pgfsetroundjoin%
\definecolor{currentfill}{rgb}{0.121569,0.466667,0.705882}%
\pgfsetfillcolor{currentfill}%
\pgfsetfillopacity{0.627433}%
\pgfsetlinewidth{1.003750pt}%
\definecolor{currentstroke}{rgb}{0.121569,0.466667,0.705882}%
\pgfsetstrokecolor{currentstroke}%
\pgfsetstrokeopacity{0.627433}%
\pgfsetdash{}{0pt}%
\pgfpathmoveto{\pgfqpoint{0.860415in}{1.493117in}}%
\pgfpathcurveto{\pgfqpoint{0.868651in}{1.493117in}}{\pgfqpoint{0.876551in}{1.496389in}}{\pgfqpoint{0.882375in}{1.502213in}}%
\pgfpathcurveto{\pgfqpoint{0.888199in}{1.508037in}}{\pgfqpoint{0.891471in}{1.515937in}}{\pgfqpoint{0.891471in}{1.524173in}}%
\pgfpathcurveto{\pgfqpoint{0.891471in}{1.532410in}}{\pgfqpoint{0.888199in}{1.540310in}}{\pgfqpoint{0.882375in}{1.546134in}}%
\pgfpathcurveto{\pgfqpoint{0.876551in}{1.551958in}}{\pgfqpoint{0.868651in}{1.555230in}}{\pgfqpoint{0.860415in}{1.555230in}}%
\pgfpathcurveto{\pgfqpoint{0.852178in}{1.555230in}}{\pgfqpoint{0.844278in}{1.551958in}}{\pgfqpoint{0.838454in}{1.546134in}}%
\pgfpathcurveto{\pgfqpoint{0.832630in}{1.540310in}}{\pgfqpoint{0.829358in}{1.532410in}}{\pgfqpoint{0.829358in}{1.524173in}}%
\pgfpathcurveto{\pgfqpoint{0.829358in}{1.515937in}}{\pgfqpoint{0.832630in}{1.508037in}}{\pgfqpoint{0.838454in}{1.502213in}}%
\pgfpathcurveto{\pgfqpoint{0.844278in}{1.496389in}}{\pgfqpoint{0.852178in}{1.493117in}}{\pgfqpoint{0.860415in}{1.493117in}}%
\pgfpathclose%
\pgfusepath{stroke,fill}%
\end{pgfscope}%
\begin{pgfscope}%
\pgfpathrectangle{\pgfqpoint{0.100000in}{0.212622in}}{\pgfqpoint{3.696000in}{3.696000in}}%
\pgfusepath{clip}%
\pgfsetbuttcap%
\pgfsetroundjoin%
\definecolor{currentfill}{rgb}{0.121569,0.466667,0.705882}%
\pgfsetfillcolor{currentfill}%
\pgfsetfillopacity{0.627433}%
\pgfsetlinewidth{1.003750pt}%
\definecolor{currentstroke}{rgb}{0.121569,0.466667,0.705882}%
\pgfsetstrokecolor{currentstroke}%
\pgfsetstrokeopacity{0.627433}%
\pgfsetdash{}{0pt}%
\pgfpathmoveto{\pgfqpoint{0.860415in}{1.493117in}}%
\pgfpathcurveto{\pgfqpoint{0.868651in}{1.493117in}}{\pgfqpoint{0.876551in}{1.496389in}}{\pgfqpoint{0.882375in}{1.502213in}}%
\pgfpathcurveto{\pgfqpoint{0.888199in}{1.508037in}}{\pgfqpoint{0.891471in}{1.515937in}}{\pgfqpoint{0.891471in}{1.524173in}}%
\pgfpathcurveto{\pgfqpoint{0.891471in}{1.532410in}}{\pgfqpoint{0.888199in}{1.540310in}}{\pgfqpoint{0.882375in}{1.546134in}}%
\pgfpathcurveto{\pgfqpoint{0.876551in}{1.551958in}}{\pgfqpoint{0.868651in}{1.555230in}}{\pgfqpoint{0.860415in}{1.555230in}}%
\pgfpathcurveto{\pgfqpoint{0.852178in}{1.555230in}}{\pgfqpoint{0.844278in}{1.551958in}}{\pgfqpoint{0.838454in}{1.546134in}}%
\pgfpathcurveto{\pgfqpoint{0.832631in}{1.540310in}}{\pgfqpoint{0.829358in}{1.532410in}}{\pgfqpoint{0.829358in}{1.524173in}}%
\pgfpathcurveto{\pgfqpoint{0.829358in}{1.515937in}}{\pgfqpoint{0.832631in}{1.508037in}}{\pgfqpoint{0.838454in}{1.502213in}}%
\pgfpathcurveto{\pgfqpoint{0.844278in}{1.496389in}}{\pgfqpoint{0.852178in}{1.493117in}}{\pgfqpoint{0.860415in}{1.493117in}}%
\pgfpathclose%
\pgfusepath{stroke,fill}%
\end{pgfscope}%
\begin{pgfscope}%
\pgfpathrectangle{\pgfqpoint{0.100000in}{0.212622in}}{\pgfqpoint{3.696000in}{3.696000in}}%
\pgfusepath{clip}%
\pgfsetbuttcap%
\pgfsetroundjoin%
\definecolor{currentfill}{rgb}{0.121569,0.466667,0.705882}%
\pgfsetfillcolor{currentfill}%
\pgfsetfillopacity{0.627433}%
\pgfsetlinewidth{1.003750pt}%
\definecolor{currentstroke}{rgb}{0.121569,0.466667,0.705882}%
\pgfsetstrokecolor{currentstroke}%
\pgfsetstrokeopacity{0.627433}%
\pgfsetdash{}{0pt}%
\pgfpathmoveto{\pgfqpoint{0.860415in}{1.493117in}}%
\pgfpathcurveto{\pgfqpoint{0.868651in}{1.493117in}}{\pgfqpoint{0.876551in}{1.496389in}}{\pgfqpoint{0.882375in}{1.502213in}}%
\pgfpathcurveto{\pgfqpoint{0.888199in}{1.508037in}}{\pgfqpoint{0.891471in}{1.515937in}}{\pgfqpoint{0.891471in}{1.524173in}}%
\pgfpathcurveto{\pgfqpoint{0.891471in}{1.532410in}}{\pgfqpoint{0.888199in}{1.540310in}}{\pgfqpoint{0.882375in}{1.546134in}}%
\pgfpathcurveto{\pgfqpoint{0.876551in}{1.551958in}}{\pgfqpoint{0.868651in}{1.555230in}}{\pgfqpoint{0.860415in}{1.555230in}}%
\pgfpathcurveto{\pgfqpoint{0.852178in}{1.555230in}}{\pgfqpoint{0.844278in}{1.551958in}}{\pgfqpoint{0.838454in}{1.546134in}}%
\pgfpathcurveto{\pgfqpoint{0.832631in}{1.540310in}}{\pgfqpoint{0.829358in}{1.532410in}}{\pgfqpoint{0.829358in}{1.524173in}}%
\pgfpathcurveto{\pgfqpoint{0.829358in}{1.515937in}}{\pgfqpoint{0.832631in}{1.508037in}}{\pgfqpoint{0.838454in}{1.502213in}}%
\pgfpathcurveto{\pgfqpoint{0.844278in}{1.496389in}}{\pgfqpoint{0.852178in}{1.493117in}}{\pgfqpoint{0.860415in}{1.493117in}}%
\pgfpathclose%
\pgfusepath{stroke,fill}%
\end{pgfscope}%
\begin{pgfscope}%
\pgfpathrectangle{\pgfqpoint{0.100000in}{0.212622in}}{\pgfqpoint{3.696000in}{3.696000in}}%
\pgfusepath{clip}%
\pgfsetbuttcap%
\pgfsetroundjoin%
\definecolor{currentfill}{rgb}{0.121569,0.466667,0.705882}%
\pgfsetfillcolor{currentfill}%
\pgfsetfillopacity{0.628136}%
\pgfsetlinewidth{1.003750pt}%
\definecolor{currentstroke}{rgb}{0.121569,0.466667,0.705882}%
\pgfsetstrokecolor{currentstroke}%
\pgfsetstrokeopacity{0.628136}%
\pgfsetdash{}{0pt}%
\pgfpathmoveto{\pgfqpoint{2.122061in}{2.004903in}}%
\pgfpathcurveto{\pgfqpoint{2.130297in}{2.004903in}}{\pgfqpoint{2.138197in}{2.008176in}}{\pgfqpoint{2.144021in}{2.014000in}}%
\pgfpathcurveto{\pgfqpoint{2.149845in}{2.019824in}}{\pgfqpoint{2.153117in}{2.027724in}}{\pgfqpoint{2.153117in}{2.035960in}}%
\pgfpathcurveto{\pgfqpoint{2.153117in}{2.044196in}}{\pgfqpoint{2.149845in}{2.052096in}}{\pgfqpoint{2.144021in}{2.057920in}}%
\pgfpathcurveto{\pgfqpoint{2.138197in}{2.063744in}}{\pgfqpoint{2.130297in}{2.067016in}}{\pgfqpoint{2.122061in}{2.067016in}}%
\pgfpathcurveto{\pgfqpoint{2.113824in}{2.067016in}}{\pgfqpoint{2.105924in}{2.063744in}}{\pgfqpoint{2.100100in}{2.057920in}}%
\pgfpathcurveto{\pgfqpoint{2.094276in}{2.052096in}}{\pgfqpoint{2.091004in}{2.044196in}}{\pgfqpoint{2.091004in}{2.035960in}}%
\pgfpathcurveto{\pgfqpoint{2.091004in}{2.027724in}}{\pgfqpoint{2.094276in}{2.019824in}}{\pgfqpoint{2.100100in}{2.014000in}}%
\pgfpathcurveto{\pgfqpoint{2.105924in}{2.008176in}}{\pgfqpoint{2.113824in}{2.004903in}}{\pgfqpoint{2.122061in}{2.004903in}}%
\pgfpathclose%
\pgfusepath{stroke,fill}%
\end{pgfscope}%
\begin{pgfscope}%
\pgfpathrectangle{\pgfqpoint{0.100000in}{0.212622in}}{\pgfqpoint{3.696000in}{3.696000in}}%
\pgfusepath{clip}%
\pgfsetbuttcap%
\pgfsetroundjoin%
\definecolor{currentfill}{rgb}{0.121569,0.466667,0.705882}%
\pgfsetfillcolor{currentfill}%
\pgfsetfillopacity{0.630318}%
\pgfsetlinewidth{1.003750pt}%
\definecolor{currentstroke}{rgb}{0.121569,0.466667,0.705882}%
\pgfsetstrokecolor{currentstroke}%
\pgfsetstrokeopacity{0.630318}%
\pgfsetdash{}{0pt}%
\pgfpathmoveto{\pgfqpoint{0.757388in}{1.205105in}}%
\pgfpathcurveto{\pgfqpoint{0.765625in}{1.205105in}}{\pgfqpoint{0.773525in}{1.208377in}}{\pgfqpoint{0.779349in}{1.214201in}}%
\pgfpathcurveto{\pgfqpoint{0.785173in}{1.220025in}}{\pgfqpoint{0.788445in}{1.227925in}}{\pgfqpoint{0.788445in}{1.236162in}}%
\pgfpathcurveto{\pgfqpoint{0.788445in}{1.244398in}}{\pgfqpoint{0.785173in}{1.252298in}}{\pgfqpoint{0.779349in}{1.258122in}}%
\pgfpathcurveto{\pgfqpoint{0.773525in}{1.263946in}}{\pgfqpoint{0.765625in}{1.267218in}}{\pgfqpoint{0.757388in}{1.267218in}}%
\pgfpathcurveto{\pgfqpoint{0.749152in}{1.267218in}}{\pgfqpoint{0.741252in}{1.263946in}}{\pgfqpoint{0.735428in}{1.258122in}}%
\pgfpathcurveto{\pgfqpoint{0.729604in}{1.252298in}}{\pgfqpoint{0.726332in}{1.244398in}}{\pgfqpoint{0.726332in}{1.236162in}}%
\pgfpathcurveto{\pgfqpoint{0.726332in}{1.227925in}}{\pgfqpoint{0.729604in}{1.220025in}}{\pgfqpoint{0.735428in}{1.214201in}}%
\pgfpathcurveto{\pgfqpoint{0.741252in}{1.208377in}}{\pgfqpoint{0.749152in}{1.205105in}}{\pgfqpoint{0.757388in}{1.205105in}}%
\pgfpathclose%
\pgfusepath{stroke,fill}%
\end{pgfscope}%
\begin{pgfscope}%
\pgfpathrectangle{\pgfqpoint{0.100000in}{0.212622in}}{\pgfqpoint{3.696000in}{3.696000in}}%
\pgfusepath{clip}%
\pgfsetbuttcap%
\pgfsetroundjoin%
\definecolor{currentfill}{rgb}{0.121569,0.466667,0.705882}%
\pgfsetfillcolor{currentfill}%
\pgfsetfillopacity{0.630468}%
\pgfsetlinewidth{1.003750pt}%
\definecolor{currentstroke}{rgb}{0.121569,0.466667,0.705882}%
\pgfsetstrokecolor{currentstroke}%
\pgfsetstrokeopacity{0.630468}%
\pgfsetdash{}{0pt}%
\pgfpathmoveto{\pgfqpoint{2.123735in}{1.997077in}}%
\pgfpathcurveto{\pgfqpoint{2.131971in}{1.997077in}}{\pgfqpoint{2.139871in}{2.000349in}}{\pgfqpoint{2.145695in}{2.006173in}}%
\pgfpathcurveto{\pgfqpoint{2.151519in}{2.011997in}}{\pgfqpoint{2.154791in}{2.019897in}}{\pgfqpoint{2.154791in}{2.028133in}}%
\pgfpathcurveto{\pgfqpoint{2.154791in}{2.036369in}}{\pgfqpoint{2.151519in}{2.044269in}}{\pgfqpoint{2.145695in}{2.050093in}}%
\pgfpathcurveto{\pgfqpoint{2.139871in}{2.055917in}}{\pgfqpoint{2.131971in}{2.059190in}}{\pgfqpoint{2.123735in}{2.059190in}}%
\pgfpathcurveto{\pgfqpoint{2.115499in}{2.059190in}}{\pgfqpoint{2.107598in}{2.055917in}}{\pgfqpoint{2.101775in}{2.050093in}}%
\pgfpathcurveto{\pgfqpoint{2.095951in}{2.044269in}}{\pgfqpoint{2.092678in}{2.036369in}}{\pgfqpoint{2.092678in}{2.028133in}}%
\pgfpathcurveto{\pgfqpoint{2.092678in}{2.019897in}}{\pgfqpoint{2.095951in}{2.011997in}}{\pgfqpoint{2.101775in}{2.006173in}}%
\pgfpathcurveto{\pgfqpoint{2.107598in}{2.000349in}}{\pgfqpoint{2.115499in}{1.997077in}}{\pgfqpoint{2.123735in}{1.997077in}}%
\pgfpathclose%
\pgfusepath{stroke,fill}%
\end{pgfscope}%
\begin{pgfscope}%
\pgfpathrectangle{\pgfqpoint{0.100000in}{0.212622in}}{\pgfqpoint{3.696000in}{3.696000in}}%
\pgfusepath{clip}%
\pgfsetbuttcap%
\pgfsetroundjoin%
\definecolor{currentfill}{rgb}{0.121569,0.466667,0.705882}%
\pgfsetfillcolor{currentfill}%
\pgfsetfillopacity{0.632619}%
\pgfsetlinewidth{1.003750pt}%
\definecolor{currentstroke}{rgb}{0.121569,0.466667,0.705882}%
\pgfsetstrokecolor{currentstroke}%
\pgfsetstrokeopacity{0.632619}%
\pgfsetdash{}{0pt}%
\pgfpathmoveto{\pgfqpoint{2.126009in}{1.988580in}}%
\pgfpathcurveto{\pgfqpoint{2.134246in}{1.988580in}}{\pgfqpoint{2.142146in}{1.991852in}}{\pgfqpoint{2.147970in}{1.997676in}}%
\pgfpathcurveto{\pgfqpoint{2.153794in}{2.003500in}}{\pgfqpoint{2.157066in}{2.011400in}}{\pgfqpoint{2.157066in}{2.019636in}}%
\pgfpathcurveto{\pgfqpoint{2.157066in}{2.027873in}}{\pgfqpoint{2.153794in}{2.035773in}}{\pgfqpoint{2.147970in}{2.041597in}}%
\pgfpathcurveto{\pgfqpoint{2.142146in}{2.047421in}}{\pgfqpoint{2.134246in}{2.050693in}}{\pgfqpoint{2.126009in}{2.050693in}}%
\pgfpathcurveto{\pgfqpoint{2.117773in}{2.050693in}}{\pgfqpoint{2.109873in}{2.047421in}}{\pgfqpoint{2.104049in}{2.041597in}}%
\pgfpathcurveto{\pgfqpoint{2.098225in}{2.035773in}}{\pgfqpoint{2.094953in}{2.027873in}}{\pgfqpoint{2.094953in}{2.019636in}}%
\pgfpathcurveto{\pgfqpoint{2.094953in}{2.011400in}}{\pgfqpoint{2.098225in}{2.003500in}}{\pgfqpoint{2.104049in}{1.997676in}}%
\pgfpathcurveto{\pgfqpoint{2.109873in}{1.991852in}}{\pgfqpoint{2.117773in}{1.988580in}}{\pgfqpoint{2.126009in}{1.988580in}}%
\pgfpathclose%
\pgfusepath{stroke,fill}%
\end{pgfscope}%
\begin{pgfscope}%
\pgfpathrectangle{\pgfqpoint{0.100000in}{0.212622in}}{\pgfqpoint{3.696000in}{3.696000in}}%
\pgfusepath{clip}%
\pgfsetbuttcap%
\pgfsetroundjoin%
\definecolor{currentfill}{rgb}{0.121569,0.466667,0.705882}%
\pgfsetfillcolor{currentfill}%
\pgfsetfillopacity{0.635072}%
\pgfsetlinewidth{1.003750pt}%
\definecolor{currentstroke}{rgb}{0.121569,0.466667,0.705882}%
\pgfsetstrokecolor{currentstroke}%
\pgfsetstrokeopacity{0.635072}%
\pgfsetdash{}{0pt}%
\pgfpathmoveto{\pgfqpoint{0.776862in}{1.206321in}}%
\pgfpathcurveto{\pgfqpoint{0.785099in}{1.206321in}}{\pgfqpoint{0.792999in}{1.209594in}}{\pgfqpoint{0.798823in}{1.215418in}}%
\pgfpathcurveto{\pgfqpoint{0.804647in}{1.221241in}}{\pgfqpoint{0.807919in}{1.229142in}}{\pgfqpoint{0.807919in}{1.237378in}}%
\pgfpathcurveto{\pgfqpoint{0.807919in}{1.245614in}}{\pgfqpoint{0.804647in}{1.253514in}}{\pgfqpoint{0.798823in}{1.259338in}}%
\pgfpathcurveto{\pgfqpoint{0.792999in}{1.265162in}}{\pgfqpoint{0.785099in}{1.268434in}}{\pgfqpoint{0.776862in}{1.268434in}}%
\pgfpathcurveto{\pgfqpoint{0.768626in}{1.268434in}}{\pgfqpoint{0.760726in}{1.265162in}}{\pgfqpoint{0.754902in}{1.259338in}}%
\pgfpathcurveto{\pgfqpoint{0.749078in}{1.253514in}}{\pgfqpoint{0.745806in}{1.245614in}}{\pgfqpoint{0.745806in}{1.237378in}}%
\pgfpathcurveto{\pgfqpoint{0.745806in}{1.229142in}}{\pgfqpoint{0.749078in}{1.221241in}}{\pgfqpoint{0.754902in}{1.215418in}}%
\pgfpathcurveto{\pgfqpoint{0.760726in}{1.209594in}}{\pgfqpoint{0.768626in}{1.206321in}}{\pgfqpoint{0.776862in}{1.206321in}}%
\pgfpathclose%
\pgfusepath{stroke,fill}%
\end{pgfscope}%
\begin{pgfscope}%
\pgfpathrectangle{\pgfqpoint{0.100000in}{0.212622in}}{\pgfqpoint{3.696000in}{3.696000in}}%
\pgfusepath{clip}%
\pgfsetbuttcap%
\pgfsetroundjoin%
\definecolor{currentfill}{rgb}{0.121569,0.466667,0.705882}%
\pgfsetfillcolor{currentfill}%
\pgfsetfillopacity{0.635330}%
\pgfsetlinewidth{1.003750pt}%
\definecolor{currentstroke}{rgb}{0.121569,0.466667,0.705882}%
\pgfsetstrokecolor{currentstroke}%
\pgfsetstrokeopacity{0.635330}%
\pgfsetdash{}{0pt}%
\pgfpathmoveto{\pgfqpoint{2.127634in}{1.979274in}}%
\pgfpathcurveto{\pgfqpoint{2.135870in}{1.979274in}}{\pgfqpoint{2.143770in}{1.982546in}}{\pgfqpoint{2.149594in}{1.988370in}}%
\pgfpathcurveto{\pgfqpoint{2.155418in}{1.994194in}}{\pgfqpoint{2.158691in}{2.002094in}}{\pgfqpoint{2.158691in}{2.010330in}}%
\pgfpathcurveto{\pgfqpoint{2.158691in}{2.018567in}}{\pgfqpoint{2.155418in}{2.026467in}}{\pgfqpoint{2.149594in}{2.032291in}}%
\pgfpathcurveto{\pgfqpoint{2.143770in}{2.038115in}}{\pgfqpoint{2.135870in}{2.041387in}}{\pgfqpoint{2.127634in}{2.041387in}}%
\pgfpathcurveto{\pgfqpoint{2.119398in}{2.041387in}}{\pgfqpoint{2.111498in}{2.038115in}}{\pgfqpoint{2.105674in}{2.032291in}}%
\pgfpathcurveto{\pgfqpoint{2.099850in}{2.026467in}}{\pgfqpoint{2.096578in}{2.018567in}}{\pgfqpoint{2.096578in}{2.010330in}}%
\pgfpathcurveto{\pgfqpoint{2.096578in}{2.002094in}}{\pgfqpoint{2.099850in}{1.994194in}}{\pgfqpoint{2.105674in}{1.988370in}}%
\pgfpathcurveto{\pgfqpoint{2.111498in}{1.982546in}}{\pgfqpoint{2.119398in}{1.979274in}}{\pgfqpoint{2.127634in}{1.979274in}}%
\pgfpathclose%
\pgfusepath{stroke,fill}%
\end{pgfscope}%
\begin{pgfscope}%
\pgfpathrectangle{\pgfqpoint{0.100000in}{0.212622in}}{\pgfqpoint{3.696000in}{3.696000in}}%
\pgfusepath{clip}%
\pgfsetbuttcap%
\pgfsetroundjoin%
\definecolor{currentfill}{rgb}{0.121569,0.466667,0.705882}%
\pgfsetfillcolor{currentfill}%
\pgfsetfillopacity{0.638215}%
\pgfsetlinewidth{1.003750pt}%
\definecolor{currentstroke}{rgb}{0.121569,0.466667,0.705882}%
\pgfsetstrokecolor{currentstroke}%
\pgfsetstrokeopacity{0.638215}%
\pgfsetdash{}{0pt}%
\pgfpathmoveto{\pgfqpoint{2.129619in}{1.969697in}}%
\pgfpathcurveto{\pgfqpoint{2.137855in}{1.969697in}}{\pgfqpoint{2.145755in}{1.972969in}}{\pgfqpoint{2.151579in}{1.978793in}}%
\pgfpathcurveto{\pgfqpoint{2.157403in}{1.984617in}}{\pgfqpoint{2.160675in}{1.992517in}}{\pgfqpoint{2.160675in}{2.000753in}}%
\pgfpathcurveto{\pgfqpoint{2.160675in}{2.008989in}}{\pgfqpoint{2.157403in}{2.016889in}}{\pgfqpoint{2.151579in}{2.022713in}}%
\pgfpathcurveto{\pgfqpoint{2.145755in}{2.028537in}}{\pgfqpoint{2.137855in}{2.031810in}}{\pgfqpoint{2.129619in}{2.031810in}}%
\pgfpathcurveto{\pgfqpoint{2.121382in}{2.031810in}}{\pgfqpoint{2.113482in}{2.028537in}}{\pgfqpoint{2.107658in}{2.022713in}}%
\pgfpathcurveto{\pgfqpoint{2.101834in}{2.016889in}}{\pgfqpoint{2.098562in}{2.008989in}}{\pgfqpoint{2.098562in}{2.000753in}}%
\pgfpathcurveto{\pgfqpoint{2.098562in}{1.992517in}}{\pgfqpoint{2.101834in}{1.984617in}}{\pgfqpoint{2.107658in}{1.978793in}}%
\pgfpathcurveto{\pgfqpoint{2.113482in}{1.972969in}}{\pgfqpoint{2.121382in}{1.969697in}}{\pgfqpoint{2.129619in}{1.969697in}}%
\pgfpathclose%
\pgfusepath{stroke,fill}%
\end{pgfscope}%
\begin{pgfscope}%
\pgfpathrectangle{\pgfqpoint{0.100000in}{0.212622in}}{\pgfqpoint{3.696000in}{3.696000in}}%
\pgfusepath{clip}%
\pgfsetbuttcap%
\pgfsetroundjoin%
\definecolor{currentfill}{rgb}{0.121569,0.466667,0.705882}%
\pgfsetfillcolor{currentfill}%
\pgfsetfillopacity{0.639512}%
\pgfsetlinewidth{1.003750pt}%
\definecolor{currentstroke}{rgb}{0.121569,0.466667,0.705882}%
\pgfsetstrokecolor{currentstroke}%
\pgfsetstrokeopacity{0.639512}%
\pgfsetdash{}{0pt}%
\pgfpathmoveto{\pgfqpoint{0.795991in}{1.207054in}}%
\pgfpathcurveto{\pgfqpoint{0.804227in}{1.207054in}}{\pgfqpoint{0.812127in}{1.210327in}}{\pgfqpoint{0.817951in}{1.216151in}}%
\pgfpathcurveto{\pgfqpoint{0.823775in}{1.221974in}}{\pgfqpoint{0.827048in}{1.229875in}}{\pgfqpoint{0.827048in}{1.238111in}}%
\pgfpathcurveto{\pgfqpoint{0.827048in}{1.246347in}}{\pgfqpoint{0.823775in}{1.254247in}}{\pgfqpoint{0.817951in}{1.260071in}}%
\pgfpathcurveto{\pgfqpoint{0.812127in}{1.265895in}}{\pgfqpoint{0.804227in}{1.269167in}}{\pgfqpoint{0.795991in}{1.269167in}}%
\pgfpathcurveto{\pgfqpoint{0.787755in}{1.269167in}}{\pgfqpoint{0.779855in}{1.265895in}}{\pgfqpoint{0.774031in}{1.260071in}}%
\pgfpathcurveto{\pgfqpoint{0.768207in}{1.254247in}}{\pgfqpoint{0.764935in}{1.246347in}}{\pgfqpoint{0.764935in}{1.238111in}}%
\pgfpathcurveto{\pgfqpoint{0.764935in}{1.229875in}}{\pgfqpoint{0.768207in}{1.221974in}}{\pgfqpoint{0.774031in}{1.216151in}}%
\pgfpathcurveto{\pgfqpoint{0.779855in}{1.210327in}}{\pgfqpoint{0.787755in}{1.207054in}}{\pgfqpoint{0.795991in}{1.207054in}}%
\pgfpathclose%
\pgfusepath{stroke,fill}%
\end{pgfscope}%
\begin{pgfscope}%
\pgfpathrectangle{\pgfqpoint{0.100000in}{0.212622in}}{\pgfqpoint{3.696000in}{3.696000in}}%
\pgfusepath{clip}%
\pgfsetbuttcap%
\pgfsetroundjoin%
\definecolor{currentfill}{rgb}{0.121569,0.466667,0.705882}%
\pgfsetfillcolor{currentfill}%
\pgfsetfillopacity{0.640813}%
\pgfsetlinewidth{1.003750pt}%
\definecolor{currentstroke}{rgb}{0.121569,0.466667,0.705882}%
\pgfsetstrokecolor{currentstroke}%
\pgfsetstrokeopacity{0.640813}%
\pgfsetdash{}{0pt}%
\pgfpathmoveto{\pgfqpoint{2.132438in}{1.958888in}}%
\pgfpathcurveto{\pgfqpoint{2.140674in}{1.958888in}}{\pgfqpoint{2.148574in}{1.962160in}}{\pgfqpoint{2.154398in}{1.967984in}}%
\pgfpathcurveto{\pgfqpoint{2.160222in}{1.973808in}}{\pgfqpoint{2.163495in}{1.981708in}}{\pgfqpoint{2.163495in}{1.989944in}}%
\pgfpathcurveto{\pgfqpoint{2.163495in}{1.998181in}}{\pgfqpoint{2.160222in}{2.006081in}}{\pgfqpoint{2.154398in}{2.011905in}}%
\pgfpathcurveto{\pgfqpoint{2.148574in}{2.017729in}}{\pgfqpoint{2.140674in}{2.021001in}}{\pgfqpoint{2.132438in}{2.021001in}}%
\pgfpathcurveto{\pgfqpoint{2.124202in}{2.021001in}}{\pgfqpoint{2.116302in}{2.017729in}}{\pgfqpoint{2.110478in}{2.011905in}}%
\pgfpathcurveto{\pgfqpoint{2.104654in}{2.006081in}}{\pgfqpoint{2.101382in}{1.998181in}}{\pgfqpoint{2.101382in}{1.989944in}}%
\pgfpathcurveto{\pgfqpoint{2.101382in}{1.981708in}}{\pgfqpoint{2.104654in}{1.973808in}}{\pgfqpoint{2.110478in}{1.967984in}}%
\pgfpathcurveto{\pgfqpoint{2.116302in}{1.962160in}}{\pgfqpoint{2.124202in}{1.958888in}}{\pgfqpoint{2.132438in}{1.958888in}}%
\pgfpathclose%
\pgfusepath{stroke,fill}%
\end{pgfscope}%
\begin{pgfscope}%
\pgfpathrectangle{\pgfqpoint{0.100000in}{0.212622in}}{\pgfqpoint{3.696000in}{3.696000in}}%
\pgfusepath{clip}%
\pgfsetbuttcap%
\pgfsetroundjoin%
\definecolor{currentfill}{rgb}{0.121569,0.466667,0.705882}%
\pgfsetfillcolor{currentfill}%
\pgfsetfillopacity{0.643654}%
\pgfsetlinewidth{1.003750pt}%
\definecolor{currentstroke}{rgb}{0.121569,0.466667,0.705882}%
\pgfsetstrokecolor{currentstroke}%
\pgfsetstrokeopacity{0.643654}%
\pgfsetdash{}{0pt}%
\pgfpathmoveto{\pgfqpoint{0.813414in}{1.207609in}}%
\pgfpathcurveto{\pgfqpoint{0.821650in}{1.207609in}}{\pgfqpoint{0.829550in}{1.210881in}}{\pgfqpoint{0.835374in}{1.216705in}}%
\pgfpathcurveto{\pgfqpoint{0.841198in}{1.222529in}}{\pgfqpoint{0.844470in}{1.230429in}}{\pgfqpoint{0.844470in}{1.238666in}}%
\pgfpathcurveto{\pgfqpoint{0.844470in}{1.246902in}}{\pgfqpoint{0.841198in}{1.254802in}}{\pgfqpoint{0.835374in}{1.260626in}}%
\pgfpathcurveto{\pgfqpoint{0.829550in}{1.266450in}}{\pgfqpoint{0.821650in}{1.269722in}}{\pgfqpoint{0.813414in}{1.269722in}}%
\pgfpathcurveto{\pgfqpoint{0.805177in}{1.269722in}}{\pgfqpoint{0.797277in}{1.266450in}}{\pgfqpoint{0.791453in}{1.260626in}}%
\pgfpathcurveto{\pgfqpoint{0.785630in}{1.254802in}}{\pgfqpoint{0.782357in}{1.246902in}}{\pgfqpoint{0.782357in}{1.238666in}}%
\pgfpathcurveto{\pgfqpoint{0.782357in}{1.230429in}}{\pgfqpoint{0.785630in}{1.222529in}}{\pgfqpoint{0.791453in}{1.216705in}}%
\pgfpathcurveto{\pgfqpoint{0.797277in}{1.210881in}}{\pgfqpoint{0.805177in}{1.207609in}}{\pgfqpoint{0.813414in}{1.207609in}}%
\pgfpathclose%
\pgfusepath{stroke,fill}%
\end{pgfscope}%
\begin{pgfscope}%
\pgfpathrectangle{\pgfqpoint{0.100000in}{0.212622in}}{\pgfqpoint{3.696000in}{3.696000in}}%
\pgfusepath{clip}%
\pgfsetbuttcap%
\pgfsetroundjoin%
\definecolor{currentfill}{rgb}{0.121569,0.466667,0.705882}%
\pgfsetfillcolor{currentfill}%
\pgfsetfillopacity{0.644312}%
\pgfsetlinewidth{1.003750pt}%
\definecolor{currentstroke}{rgb}{0.121569,0.466667,0.705882}%
\pgfsetstrokecolor{currentstroke}%
\pgfsetstrokeopacity{0.644312}%
\pgfsetdash{}{0pt}%
\pgfpathmoveto{\pgfqpoint{2.134846in}{1.946473in}}%
\pgfpathcurveto{\pgfqpoint{2.143082in}{1.946473in}}{\pgfqpoint{2.150982in}{1.949745in}}{\pgfqpoint{2.156806in}{1.955569in}}%
\pgfpathcurveto{\pgfqpoint{2.162630in}{1.961393in}}{\pgfqpoint{2.165902in}{1.969293in}}{\pgfqpoint{2.165902in}{1.977530in}}%
\pgfpathcurveto{\pgfqpoint{2.165902in}{1.985766in}}{\pgfqpoint{2.162630in}{1.993666in}}{\pgfqpoint{2.156806in}{1.999490in}}%
\pgfpathcurveto{\pgfqpoint{2.150982in}{2.005314in}}{\pgfqpoint{2.143082in}{2.008586in}}{\pgfqpoint{2.134846in}{2.008586in}}%
\pgfpathcurveto{\pgfqpoint{2.126610in}{2.008586in}}{\pgfqpoint{2.118710in}{2.005314in}}{\pgfqpoint{2.112886in}{1.999490in}}%
\pgfpathcurveto{\pgfqpoint{2.107062in}{1.993666in}}{\pgfqpoint{2.103789in}{1.985766in}}{\pgfqpoint{2.103789in}{1.977530in}}%
\pgfpathcurveto{\pgfqpoint{2.103789in}{1.969293in}}{\pgfqpoint{2.107062in}{1.961393in}}{\pgfqpoint{2.112886in}{1.955569in}}%
\pgfpathcurveto{\pgfqpoint{2.118710in}{1.949745in}}{\pgfqpoint{2.126610in}{1.946473in}}{\pgfqpoint{2.134846in}{1.946473in}}%
\pgfpathclose%
\pgfusepath{stroke,fill}%
\end{pgfscope}%
\begin{pgfscope}%
\pgfpathrectangle{\pgfqpoint{0.100000in}{0.212622in}}{\pgfqpoint{3.696000in}{3.696000in}}%
\pgfusepath{clip}%
\pgfsetbuttcap%
\pgfsetroundjoin%
\definecolor{currentfill}{rgb}{0.121569,0.466667,0.705882}%
\pgfsetfillcolor{currentfill}%
\pgfsetfillopacity{0.647952}%
\pgfsetlinewidth{1.003750pt}%
\definecolor{currentstroke}{rgb}{0.121569,0.466667,0.705882}%
\pgfsetstrokecolor{currentstroke}%
\pgfsetstrokeopacity{0.647952}%
\pgfsetdash{}{0pt}%
\pgfpathmoveto{\pgfqpoint{0.830271in}{1.207872in}}%
\pgfpathcurveto{\pgfqpoint{0.838507in}{1.207872in}}{\pgfqpoint{0.846407in}{1.211144in}}{\pgfqpoint{0.852231in}{1.216968in}}%
\pgfpathcurveto{\pgfqpoint{0.858055in}{1.222792in}}{\pgfqpoint{0.861327in}{1.230692in}}{\pgfqpoint{0.861327in}{1.238929in}}%
\pgfpathcurveto{\pgfqpoint{0.861327in}{1.247165in}}{\pgfqpoint{0.858055in}{1.255065in}}{\pgfqpoint{0.852231in}{1.260889in}}%
\pgfpathcurveto{\pgfqpoint{0.846407in}{1.266713in}}{\pgfqpoint{0.838507in}{1.269985in}}{\pgfqpoint{0.830271in}{1.269985in}}%
\pgfpathcurveto{\pgfqpoint{0.822035in}{1.269985in}}{\pgfqpoint{0.814135in}{1.266713in}}{\pgfqpoint{0.808311in}{1.260889in}}%
\pgfpathcurveto{\pgfqpoint{0.802487in}{1.255065in}}{\pgfqpoint{0.799214in}{1.247165in}}{\pgfqpoint{0.799214in}{1.238929in}}%
\pgfpathcurveto{\pgfqpoint{0.799214in}{1.230692in}}{\pgfqpoint{0.802487in}{1.222792in}}{\pgfqpoint{0.808311in}{1.216968in}}%
\pgfpathcurveto{\pgfqpoint{0.814135in}{1.211144in}}{\pgfqpoint{0.822035in}{1.207872in}}{\pgfqpoint{0.830271in}{1.207872in}}%
\pgfpathclose%
\pgfusepath{stroke,fill}%
\end{pgfscope}%
\begin{pgfscope}%
\pgfpathrectangle{\pgfqpoint{0.100000in}{0.212622in}}{\pgfqpoint{3.696000in}{3.696000in}}%
\pgfusepath{clip}%
\pgfsetbuttcap%
\pgfsetroundjoin%
\definecolor{currentfill}{rgb}{0.121569,0.466667,0.705882}%
\pgfsetfillcolor{currentfill}%
\pgfsetfillopacity{0.648222}%
\pgfsetlinewidth{1.003750pt}%
\definecolor{currentstroke}{rgb}{0.121569,0.466667,0.705882}%
\pgfsetstrokecolor{currentstroke}%
\pgfsetstrokeopacity{0.648222}%
\pgfsetdash{}{0pt}%
\pgfpathmoveto{\pgfqpoint{2.137133in}{1.934212in}}%
\pgfpathcurveto{\pgfqpoint{2.145369in}{1.934212in}}{\pgfqpoint{2.153269in}{1.937484in}}{\pgfqpoint{2.159093in}{1.943308in}}%
\pgfpathcurveto{\pgfqpoint{2.164917in}{1.949132in}}{\pgfqpoint{2.168189in}{1.957032in}}{\pgfqpoint{2.168189in}{1.965268in}}%
\pgfpathcurveto{\pgfqpoint{2.168189in}{1.973504in}}{\pgfqpoint{2.164917in}{1.981405in}}{\pgfqpoint{2.159093in}{1.987228in}}%
\pgfpathcurveto{\pgfqpoint{2.153269in}{1.993052in}}{\pgfqpoint{2.145369in}{1.996325in}}{\pgfqpoint{2.137133in}{1.996325in}}%
\pgfpathcurveto{\pgfqpoint{2.128897in}{1.996325in}}{\pgfqpoint{2.120996in}{1.993052in}}{\pgfqpoint{2.115173in}{1.987228in}}%
\pgfpathcurveto{\pgfqpoint{2.109349in}{1.981405in}}{\pgfqpoint{2.106076in}{1.973504in}}{\pgfqpoint{2.106076in}{1.965268in}}%
\pgfpathcurveto{\pgfqpoint{2.106076in}{1.957032in}}{\pgfqpoint{2.109349in}{1.949132in}}{\pgfqpoint{2.115173in}{1.943308in}}%
\pgfpathcurveto{\pgfqpoint{2.120996in}{1.937484in}}{\pgfqpoint{2.128897in}{1.934212in}}{\pgfqpoint{2.137133in}{1.934212in}}%
\pgfpathclose%
\pgfusepath{stroke,fill}%
\end{pgfscope}%
\begin{pgfscope}%
\pgfpathrectangle{\pgfqpoint{0.100000in}{0.212622in}}{\pgfqpoint{3.696000in}{3.696000in}}%
\pgfusepath{clip}%
\pgfsetbuttcap%
\pgfsetroundjoin%
\definecolor{currentfill}{rgb}{0.121569,0.466667,0.705882}%
\pgfsetfillcolor{currentfill}%
\pgfsetfillopacity{0.651280}%
\pgfsetlinewidth{1.003750pt}%
\definecolor{currentstroke}{rgb}{0.121569,0.466667,0.705882}%
\pgfsetstrokecolor{currentstroke}%
\pgfsetstrokeopacity{0.651280}%
\pgfsetdash{}{0pt}%
\pgfpathmoveto{\pgfqpoint{0.845324in}{1.207064in}}%
\pgfpathcurveto{\pgfqpoint{0.853560in}{1.207064in}}{\pgfqpoint{0.861460in}{1.210337in}}{\pgfqpoint{0.867284in}{1.216161in}}%
\pgfpathcurveto{\pgfqpoint{0.873108in}{1.221985in}}{\pgfqpoint{0.876380in}{1.229885in}}{\pgfqpoint{0.876380in}{1.238121in}}%
\pgfpathcurveto{\pgfqpoint{0.876380in}{1.246357in}}{\pgfqpoint{0.873108in}{1.254257in}}{\pgfqpoint{0.867284in}{1.260081in}}%
\pgfpathcurveto{\pgfqpoint{0.861460in}{1.265905in}}{\pgfqpoint{0.853560in}{1.269177in}}{\pgfqpoint{0.845324in}{1.269177in}}%
\pgfpathcurveto{\pgfqpoint{0.837087in}{1.269177in}}{\pgfqpoint{0.829187in}{1.265905in}}{\pgfqpoint{0.823363in}{1.260081in}}%
\pgfpathcurveto{\pgfqpoint{0.817540in}{1.254257in}}{\pgfqpoint{0.814267in}{1.246357in}}{\pgfqpoint{0.814267in}{1.238121in}}%
\pgfpathcurveto{\pgfqpoint{0.814267in}{1.229885in}}{\pgfqpoint{0.817540in}{1.221985in}}{\pgfqpoint{0.823363in}{1.216161in}}%
\pgfpathcurveto{\pgfqpoint{0.829187in}{1.210337in}}{\pgfqpoint{0.837087in}{1.207064in}}{\pgfqpoint{0.845324in}{1.207064in}}%
\pgfpathclose%
\pgfusepath{stroke,fill}%
\end{pgfscope}%
\begin{pgfscope}%
\pgfpathrectangle{\pgfqpoint{0.100000in}{0.212622in}}{\pgfqpoint{3.696000in}{3.696000in}}%
\pgfusepath{clip}%
\pgfsetbuttcap%
\pgfsetroundjoin%
\definecolor{currentfill}{rgb}{0.121569,0.466667,0.705882}%
\pgfsetfillcolor{currentfill}%
\pgfsetfillopacity{0.651975}%
\pgfsetlinewidth{1.003750pt}%
\definecolor{currentstroke}{rgb}{0.121569,0.466667,0.705882}%
\pgfsetstrokecolor{currentstroke}%
\pgfsetstrokeopacity{0.651975}%
\pgfsetdash{}{0pt}%
\pgfpathmoveto{\pgfqpoint{2.141042in}{1.920082in}}%
\pgfpathcurveto{\pgfqpoint{2.149279in}{1.920082in}}{\pgfqpoint{2.157179in}{1.923354in}}{\pgfqpoint{2.163003in}{1.929178in}}%
\pgfpathcurveto{\pgfqpoint{2.168827in}{1.935002in}}{\pgfqpoint{2.172099in}{1.942902in}}{\pgfqpoint{2.172099in}{1.951138in}}%
\pgfpathcurveto{\pgfqpoint{2.172099in}{1.959375in}}{\pgfqpoint{2.168827in}{1.967275in}}{\pgfqpoint{2.163003in}{1.973098in}}%
\pgfpathcurveto{\pgfqpoint{2.157179in}{1.978922in}}{\pgfqpoint{2.149279in}{1.982195in}}{\pgfqpoint{2.141042in}{1.982195in}}%
\pgfpathcurveto{\pgfqpoint{2.132806in}{1.982195in}}{\pgfqpoint{2.124906in}{1.978922in}}{\pgfqpoint{2.119082in}{1.973098in}}%
\pgfpathcurveto{\pgfqpoint{2.113258in}{1.967275in}}{\pgfqpoint{2.109986in}{1.959375in}}{\pgfqpoint{2.109986in}{1.951138in}}%
\pgfpathcurveto{\pgfqpoint{2.109986in}{1.942902in}}{\pgfqpoint{2.113258in}{1.935002in}}{\pgfqpoint{2.119082in}{1.929178in}}%
\pgfpathcurveto{\pgfqpoint{2.124906in}{1.923354in}}{\pgfqpoint{2.132806in}{1.920082in}}{\pgfqpoint{2.141042in}{1.920082in}}%
\pgfpathclose%
\pgfusepath{stroke,fill}%
\end{pgfscope}%
\begin{pgfscope}%
\pgfpathrectangle{\pgfqpoint{0.100000in}{0.212622in}}{\pgfqpoint{3.696000in}{3.696000in}}%
\pgfusepath{clip}%
\pgfsetbuttcap%
\pgfsetroundjoin%
\definecolor{currentfill}{rgb}{0.121569,0.466667,0.705882}%
\pgfsetfillcolor{currentfill}%
\pgfsetfillopacity{0.654651}%
\pgfsetlinewidth{1.003750pt}%
\definecolor{currentstroke}{rgb}{0.121569,0.466667,0.705882}%
\pgfsetstrokecolor{currentstroke}%
\pgfsetstrokeopacity{0.654651}%
\pgfsetdash{}{0pt}%
\pgfpathmoveto{\pgfqpoint{0.859395in}{1.206069in}}%
\pgfpathcurveto{\pgfqpoint{0.867631in}{1.206069in}}{\pgfqpoint{0.875531in}{1.209341in}}{\pgfqpoint{0.881355in}{1.215165in}}%
\pgfpathcurveto{\pgfqpoint{0.887179in}{1.220989in}}{\pgfqpoint{0.890451in}{1.228889in}}{\pgfqpoint{0.890451in}{1.237125in}}%
\pgfpathcurveto{\pgfqpoint{0.890451in}{1.245361in}}{\pgfqpoint{0.887179in}{1.253261in}}{\pgfqpoint{0.881355in}{1.259085in}}%
\pgfpathcurveto{\pgfqpoint{0.875531in}{1.264909in}}{\pgfqpoint{0.867631in}{1.268182in}}{\pgfqpoint{0.859395in}{1.268182in}}%
\pgfpathcurveto{\pgfqpoint{0.851158in}{1.268182in}}{\pgfqpoint{0.843258in}{1.264909in}}{\pgfqpoint{0.837434in}{1.259085in}}%
\pgfpathcurveto{\pgfqpoint{0.831610in}{1.253261in}}{\pgfqpoint{0.828338in}{1.245361in}}{\pgfqpoint{0.828338in}{1.237125in}}%
\pgfpathcurveto{\pgfqpoint{0.828338in}{1.228889in}}{\pgfqpoint{0.831610in}{1.220989in}}{\pgfqpoint{0.837434in}{1.215165in}}%
\pgfpathcurveto{\pgfqpoint{0.843258in}{1.209341in}}{\pgfqpoint{0.851158in}{1.206069in}}{\pgfqpoint{0.859395in}{1.206069in}}%
\pgfpathclose%
\pgfusepath{stroke,fill}%
\end{pgfscope}%
\begin{pgfscope}%
\pgfpathrectangle{\pgfqpoint{0.100000in}{0.212622in}}{\pgfqpoint{3.696000in}{3.696000in}}%
\pgfusepath{clip}%
\pgfsetbuttcap%
\pgfsetroundjoin%
\definecolor{currentfill}{rgb}{0.121569,0.466667,0.705882}%
\pgfsetfillcolor{currentfill}%
\pgfsetfillopacity{0.656320}%
\pgfsetlinewidth{1.003750pt}%
\definecolor{currentstroke}{rgb}{0.121569,0.466667,0.705882}%
\pgfsetstrokecolor{currentstroke}%
\pgfsetstrokeopacity{0.656320}%
\pgfsetdash{}{0pt}%
\pgfpathmoveto{\pgfqpoint{2.144133in}{1.906538in}}%
\pgfpathcurveto{\pgfqpoint{2.152370in}{1.906538in}}{\pgfqpoint{2.160270in}{1.909811in}}{\pgfqpoint{2.166094in}{1.915635in}}%
\pgfpathcurveto{\pgfqpoint{2.171918in}{1.921459in}}{\pgfqpoint{2.175190in}{1.929359in}}{\pgfqpoint{2.175190in}{1.937595in}}%
\pgfpathcurveto{\pgfqpoint{2.175190in}{1.945831in}}{\pgfqpoint{2.171918in}{1.953731in}}{\pgfqpoint{2.166094in}{1.959555in}}%
\pgfpathcurveto{\pgfqpoint{2.160270in}{1.965379in}}{\pgfqpoint{2.152370in}{1.968651in}}{\pgfqpoint{2.144133in}{1.968651in}}%
\pgfpathcurveto{\pgfqpoint{2.135897in}{1.968651in}}{\pgfqpoint{2.127997in}{1.965379in}}{\pgfqpoint{2.122173in}{1.959555in}}%
\pgfpathcurveto{\pgfqpoint{2.116349in}{1.953731in}}{\pgfqpoint{2.113077in}{1.945831in}}{\pgfqpoint{2.113077in}{1.937595in}}%
\pgfpathcurveto{\pgfqpoint{2.113077in}{1.929359in}}{\pgfqpoint{2.116349in}{1.921459in}}{\pgfqpoint{2.122173in}{1.915635in}}%
\pgfpathcurveto{\pgfqpoint{2.127997in}{1.909811in}}{\pgfqpoint{2.135897in}{1.906538in}}{\pgfqpoint{2.144133in}{1.906538in}}%
\pgfpathclose%
\pgfusepath{stroke,fill}%
\end{pgfscope}%
\begin{pgfscope}%
\pgfpathrectangle{\pgfqpoint{0.100000in}{0.212622in}}{\pgfqpoint{3.696000in}{3.696000in}}%
\pgfusepath{clip}%
\pgfsetbuttcap%
\pgfsetroundjoin%
\definecolor{currentfill}{rgb}{0.121569,0.466667,0.705882}%
\pgfsetfillcolor{currentfill}%
\pgfsetfillopacity{0.657979}%
\pgfsetlinewidth{1.003750pt}%
\definecolor{currentstroke}{rgb}{0.121569,0.466667,0.705882}%
\pgfsetstrokecolor{currentstroke}%
\pgfsetstrokeopacity{0.657979}%
\pgfsetdash{}{0pt}%
\pgfpathmoveto{\pgfqpoint{0.873297in}{1.205977in}}%
\pgfpathcurveto{\pgfqpoint{0.881533in}{1.205977in}}{\pgfqpoint{0.889433in}{1.209249in}}{\pgfqpoint{0.895257in}{1.215073in}}%
\pgfpathcurveto{\pgfqpoint{0.901081in}{1.220897in}}{\pgfqpoint{0.904353in}{1.228797in}}{\pgfqpoint{0.904353in}{1.237033in}}%
\pgfpathcurveto{\pgfqpoint{0.904353in}{1.245269in}}{\pgfqpoint{0.901081in}{1.253169in}}{\pgfqpoint{0.895257in}{1.258993in}}%
\pgfpathcurveto{\pgfqpoint{0.889433in}{1.264817in}}{\pgfqpoint{0.881533in}{1.268090in}}{\pgfqpoint{0.873297in}{1.268090in}}%
\pgfpathcurveto{\pgfqpoint{0.865060in}{1.268090in}}{\pgfqpoint{0.857160in}{1.264817in}}{\pgfqpoint{0.851336in}{1.258993in}}%
\pgfpathcurveto{\pgfqpoint{0.845512in}{1.253169in}}{\pgfqpoint{0.842240in}{1.245269in}}{\pgfqpoint{0.842240in}{1.237033in}}%
\pgfpathcurveto{\pgfqpoint{0.842240in}{1.228797in}}{\pgfqpoint{0.845512in}{1.220897in}}{\pgfqpoint{0.851336in}{1.215073in}}%
\pgfpathcurveto{\pgfqpoint{0.857160in}{1.209249in}}{\pgfqpoint{0.865060in}{1.205977in}}{\pgfqpoint{0.873297in}{1.205977in}}%
\pgfpathclose%
\pgfusepath{stroke,fill}%
\end{pgfscope}%
\begin{pgfscope}%
\pgfpathrectangle{\pgfqpoint{0.100000in}{0.212622in}}{\pgfqpoint{3.696000in}{3.696000in}}%
\pgfusepath{clip}%
\pgfsetbuttcap%
\pgfsetroundjoin%
\definecolor{currentfill}{rgb}{0.121569,0.466667,0.705882}%
\pgfsetfillcolor{currentfill}%
\pgfsetfillopacity{0.660970}%
\pgfsetlinewidth{1.003750pt}%
\definecolor{currentstroke}{rgb}{0.121569,0.466667,0.705882}%
\pgfsetstrokecolor{currentstroke}%
\pgfsetstrokeopacity{0.660970}%
\pgfsetdash{}{0pt}%
\pgfpathmoveto{\pgfqpoint{0.886813in}{1.205599in}}%
\pgfpathcurveto{\pgfqpoint{0.895049in}{1.205599in}}{\pgfqpoint{0.902949in}{1.208871in}}{\pgfqpoint{0.908773in}{1.214695in}}%
\pgfpathcurveto{\pgfqpoint{0.914597in}{1.220519in}}{\pgfqpoint{0.917869in}{1.228419in}}{\pgfqpoint{0.917869in}{1.236655in}}%
\pgfpathcurveto{\pgfqpoint{0.917869in}{1.244891in}}{\pgfqpoint{0.914597in}{1.252791in}}{\pgfqpoint{0.908773in}{1.258615in}}%
\pgfpathcurveto{\pgfqpoint{0.902949in}{1.264439in}}{\pgfqpoint{0.895049in}{1.267712in}}{\pgfqpoint{0.886813in}{1.267712in}}%
\pgfpathcurveto{\pgfqpoint{0.878577in}{1.267712in}}{\pgfqpoint{0.870676in}{1.264439in}}{\pgfqpoint{0.864853in}{1.258615in}}%
\pgfpathcurveto{\pgfqpoint{0.859029in}{1.252791in}}{\pgfqpoint{0.855756in}{1.244891in}}{\pgfqpoint{0.855756in}{1.236655in}}%
\pgfpathcurveto{\pgfqpoint{0.855756in}{1.228419in}}{\pgfqpoint{0.859029in}{1.220519in}}{\pgfqpoint{0.864853in}{1.214695in}}%
\pgfpathcurveto{\pgfqpoint{0.870676in}{1.208871in}}{\pgfqpoint{0.878577in}{1.205599in}}{\pgfqpoint{0.886813in}{1.205599in}}%
\pgfpathclose%
\pgfusepath{stroke,fill}%
\end{pgfscope}%
\begin{pgfscope}%
\pgfpathrectangle{\pgfqpoint{0.100000in}{0.212622in}}{\pgfqpoint{3.696000in}{3.696000in}}%
\pgfusepath{clip}%
\pgfsetbuttcap%
\pgfsetroundjoin%
\definecolor{currentfill}{rgb}{0.121569,0.466667,0.705882}%
\pgfsetfillcolor{currentfill}%
\pgfsetfillopacity{0.661320}%
\pgfsetlinewidth{1.003750pt}%
\definecolor{currentstroke}{rgb}{0.121569,0.466667,0.705882}%
\pgfsetstrokecolor{currentstroke}%
\pgfsetstrokeopacity{0.661320}%
\pgfsetdash{}{0pt}%
\pgfpathmoveto{\pgfqpoint{2.146972in}{1.892810in}}%
\pgfpathcurveto{\pgfqpoint{2.155208in}{1.892810in}}{\pgfqpoint{2.163109in}{1.896082in}}{\pgfqpoint{2.168932in}{1.901906in}}%
\pgfpathcurveto{\pgfqpoint{2.174756in}{1.907730in}}{\pgfqpoint{2.178029in}{1.915630in}}{\pgfqpoint{2.178029in}{1.923866in}}%
\pgfpathcurveto{\pgfqpoint{2.178029in}{1.932102in}}{\pgfqpoint{2.174756in}{1.940002in}}{\pgfqpoint{2.168932in}{1.945826in}}%
\pgfpathcurveto{\pgfqpoint{2.163109in}{1.951650in}}{\pgfqpoint{2.155208in}{1.954923in}}{\pgfqpoint{2.146972in}{1.954923in}}%
\pgfpathcurveto{\pgfqpoint{2.138736in}{1.954923in}}{\pgfqpoint{2.130836in}{1.951650in}}{\pgfqpoint{2.125012in}{1.945826in}}%
\pgfpathcurveto{\pgfqpoint{2.119188in}{1.940002in}}{\pgfqpoint{2.115916in}{1.932102in}}{\pgfqpoint{2.115916in}{1.923866in}}%
\pgfpathcurveto{\pgfqpoint{2.115916in}{1.915630in}}{\pgfqpoint{2.119188in}{1.907730in}}{\pgfqpoint{2.125012in}{1.901906in}}%
\pgfpathcurveto{\pgfqpoint{2.130836in}{1.896082in}}{\pgfqpoint{2.138736in}{1.892810in}}{\pgfqpoint{2.146972in}{1.892810in}}%
\pgfpathclose%
\pgfusepath{stroke,fill}%
\end{pgfscope}%
\begin{pgfscope}%
\pgfpathrectangle{\pgfqpoint{0.100000in}{0.212622in}}{\pgfqpoint{3.696000in}{3.696000in}}%
\pgfusepath{clip}%
\pgfsetbuttcap%
\pgfsetroundjoin%
\definecolor{currentfill}{rgb}{0.121569,0.466667,0.705882}%
\pgfsetfillcolor{currentfill}%
\pgfsetfillopacity{0.663711}%
\pgfsetlinewidth{1.003750pt}%
\definecolor{currentstroke}{rgb}{0.121569,0.466667,0.705882}%
\pgfsetstrokecolor{currentstroke}%
\pgfsetstrokeopacity{0.663711}%
\pgfsetdash{}{0pt}%
\pgfpathmoveto{\pgfqpoint{0.899048in}{1.205639in}}%
\pgfpathcurveto{\pgfqpoint{0.907284in}{1.205639in}}{\pgfqpoint{0.915184in}{1.208912in}}{\pgfqpoint{0.921008in}{1.214736in}}%
\pgfpathcurveto{\pgfqpoint{0.926832in}{1.220560in}}{\pgfqpoint{0.930105in}{1.228460in}}{\pgfqpoint{0.930105in}{1.236696in}}%
\pgfpathcurveto{\pgfqpoint{0.930105in}{1.244932in}}{\pgfqpoint{0.926832in}{1.252832in}}{\pgfqpoint{0.921008in}{1.258656in}}%
\pgfpathcurveto{\pgfqpoint{0.915184in}{1.264480in}}{\pgfqpoint{0.907284in}{1.267752in}}{\pgfqpoint{0.899048in}{1.267752in}}%
\pgfpathcurveto{\pgfqpoint{0.890812in}{1.267752in}}{\pgfqpoint{0.882912in}{1.264480in}}{\pgfqpoint{0.877088in}{1.258656in}}%
\pgfpathcurveto{\pgfqpoint{0.871264in}{1.252832in}}{\pgfqpoint{0.867992in}{1.244932in}}{\pgfqpoint{0.867992in}{1.236696in}}%
\pgfpathcurveto{\pgfqpoint{0.867992in}{1.228460in}}{\pgfqpoint{0.871264in}{1.220560in}}{\pgfqpoint{0.877088in}{1.214736in}}%
\pgfpathcurveto{\pgfqpoint{0.882912in}{1.208912in}}{\pgfqpoint{0.890812in}{1.205639in}}{\pgfqpoint{0.899048in}{1.205639in}}%
\pgfpathclose%
\pgfusepath{stroke,fill}%
\end{pgfscope}%
\begin{pgfscope}%
\pgfpathrectangle{\pgfqpoint{0.100000in}{0.212622in}}{\pgfqpoint{3.696000in}{3.696000in}}%
\pgfusepath{clip}%
\pgfsetbuttcap%
\pgfsetroundjoin%
\definecolor{currentfill}{rgb}{0.121569,0.466667,0.705882}%
\pgfsetfillcolor{currentfill}%
\pgfsetfillopacity{0.666375}%
\pgfsetlinewidth{1.003750pt}%
\definecolor{currentstroke}{rgb}{0.121569,0.466667,0.705882}%
\pgfsetstrokecolor{currentstroke}%
\pgfsetstrokeopacity{0.666375}%
\pgfsetdash{}{0pt}%
\pgfpathmoveto{\pgfqpoint{2.151875in}{1.876932in}}%
\pgfpathcurveto{\pgfqpoint{2.160111in}{1.876932in}}{\pgfqpoint{2.168011in}{1.880205in}}{\pgfqpoint{2.173835in}{1.886028in}}%
\pgfpathcurveto{\pgfqpoint{2.179659in}{1.891852in}}{\pgfqpoint{2.182931in}{1.899752in}}{\pgfqpoint{2.182931in}{1.907989in}}%
\pgfpathcurveto{\pgfqpoint{2.182931in}{1.916225in}}{\pgfqpoint{2.179659in}{1.924125in}}{\pgfqpoint{2.173835in}{1.929949in}}%
\pgfpathcurveto{\pgfqpoint{2.168011in}{1.935773in}}{\pgfqpoint{2.160111in}{1.939045in}}{\pgfqpoint{2.151875in}{1.939045in}}%
\pgfpathcurveto{\pgfqpoint{2.143638in}{1.939045in}}{\pgfqpoint{2.135738in}{1.935773in}}{\pgfqpoint{2.129914in}{1.929949in}}%
\pgfpathcurveto{\pgfqpoint{2.124090in}{1.924125in}}{\pgfqpoint{2.120818in}{1.916225in}}{\pgfqpoint{2.120818in}{1.907989in}}%
\pgfpathcurveto{\pgfqpoint{2.120818in}{1.899752in}}{\pgfqpoint{2.124090in}{1.891852in}}{\pgfqpoint{2.129914in}{1.886028in}}%
\pgfpathcurveto{\pgfqpoint{2.135738in}{1.880205in}}{\pgfqpoint{2.143638in}{1.876932in}}{\pgfqpoint{2.151875in}{1.876932in}}%
\pgfpathclose%
\pgfusepath{stroke,fill}%
\end{pgfscope}%
\begin{pgfscope}%
\pgfpathrectangle{\pgfqpoint{0.100000in}{0.212622in}}{\pgfqpoint{3.696000in}{3.696000in}}%
\pgfusepath{clip}%
\pgfsetbuttcap%
\pgfsetroundjoin%
\definecolor{currentfill}{rgb}{0.121569,0.466667,0.705882}%
\pgfsetfillcolor{currentfill}%
\pgfsetfillopacity{0.666494}%
\pgfsetlinewidth{1.003750pt}%
\definecolor{currentstroke}{rgb}{0.121569,0.466667,0.705882}%
\pgfsetstrokecolor{currentstroke}%
\pgfsetstrokeopacity{0.666494}%
\pgfsetdash{}{0pt}%
\pgfpathmoveto{\pgfqpoint{0.910968in}{1.206153in}}%
\pgfpathcurveto{\pgfqpoint{0.919205in}{1.206153in}}{\pgfqpoint{0.927105in}{1.209425in}}{\pgfqpoint{0.932928in}{1.215249in}}%
\pgfpathcurveto{\pgfqpoint{0.938752in}{1.221073in}}{\pgfqpoint{0.942025in}{1.228973in}}{\pgfqpoint{0.942025in}{1.237209in}}%
\pgfpathcurveto{\pgfqpoint{0.942025in}{1.245445in}}{\pgfqpoint{0.938752in}{1.253345in}}{\pgfqpoint{0.932928in}{1.259169in}}%
\pgfpathcurveto{\pgfqpoint{0.927105in}{1.264993in}}{\pgfqpoint{0.919205in}{1.268266in}}{\pgfqpoint{0.910968in}{1.268266in}}%
\pgfpathcurveto{\pgfqpoint{0.902732in}{1.268266in}}{\pgfqpoint{0.894832in}{1.264993in}}{\pgfqpoint{0.889008in}{1.259169in}}%
\pgfpathcurveto{\pgfqpoint{0.883184in}{1.253345in}}{\pgfqpoint{0.879912in}{1.245445in}}{\pgfqpoint{0.879912in}{1.237209in}}%
\pgfpathcurveto{\pgfqpoint{0.879912in}{1.228973in}}{\pgfqpoint{0.883184in}{1.221073in}}{\pgfqpoint{0.889008in}{1.215249in}}%
\pgfpathcurveto{\pgfqpoint{0.894832in}{1.209425in}}{\pgfqpoint{0.902732in}{1.206153in}}{\pgfqpoint{0.910968in}{1.206153in}}%
\pgfpathclose%
\pgfusepath{stroke,fill}%
\end{pgfscope}%
\begin{pgfscope}%
\pgfpathrectangle{\pgfqpoint{0.100000in}{0.212622in}}{\pgfqpoint{3.696000in}{3.696000in}}%
\pgfusepath{clip}%
\pgfsetbuttcap%
\pgfsetroundjoin%
\definecolor{currentfill}{rgb}{0.121569,0.466667,0.705882}%
\pgfsetfillcolor{currentfill}%
\pgfsetfillopacity{0.668438}%
\pgfsetlinewidth{1.003750pt}%
\definecolor{currentstroke}{rgb}{0.121569,0.466667,0.705882}%
\pgfsetstrokecolor{currentstroke}%
\pgfsetstrokeopacity{0.668438}%
\pgfsetdash{}{0pt}%
\pgfpathmoveto{\pgfqpoint{0.922012in}{1.205249in}}%
\pgfpathcurveto{\pgfqpoint{0.930248in}{1.205249in}}{\pgfqpoint{0.938148in}{1.208521in}}{\pgfqpoint{0.943972in}{1.214345in}}%
\pgfpathcurveto{\pgfqpoint{0.949796in}{1.220169in}}{\pgfqpoint{0.953068in}{1.228069in}}{\pgfqpoint{0.953068in}{1.236305in}}%
\pgfpathcurveto{\pgfqpoint{0.953068in}{1.244541in}}{\pgfqpoint{0.949796in}{1.252441in}}{\pgfqpoint{0.943972in}{1.258265in}}%
\pgfpathcurveto{\pgfqpoint{0.938148in}{1.264089in}}{\pgfqpoint{0.930248in}{1.267362in}}{\pgfqpoint{0.922012in}{1.267362in}}%
\pgfpathcurveto{\pgfqpoint{0.913776in}{1.267362in}}{\pgfqpoint{0.905876in}{1.264089in}}{\pgfqpoint{0.900052in}{1.258265in}}%
\pgfpathcurveto{\pgfqpoint{0.894228in}{1.252441in}}{\pgfqpoint{0.890955in}{1.244541in}}{\pgfqpoint{0.890955in}{1.236305in}}%
\pgfpathcurveto{\pgfqpoint{0.890955in}{1.228069in}}{\pgfqpoint{0.894228in}{1.220169in}}{\pgfqpoint{0.900052in}{1.214345in}}%
\pgfpathcurveto{\pgfqpoint{0.905876in}{1.208521in}}{\pgfqpoint{0.913776in}{1.205249in}}{\pgfqpoint{0.922012in}{1.205249in}}%
\pgfpathclose%
\pgfusepath{stroke,fill}%
\end{pgfscope}%
\begin{pgfscope}%
\pgfpathrectangle{\pgfqpoint{0.100000in}{0.212622in}}{\pgfqpoint{3.696000in}{3.696000in}}%
\pgfusepath{clip}%
\pgfsetbuttcap%
\pgfsetroundjoin%
\definecolor{currentfill}{rgb}{0.121569,0.466667,0.705882}%
\pgfsetfillcolor{currentfill}%
\pgfsetfillopacity{0.670380}%
\pgfsetlinewidth{1.003750pt}%
\definecolor{currentstroke}{rgb}{0.121569,0.466667,0.705882}%
\pgfsetstrokecolor{currentstroke}%
\pgfsetstrokeopacity{0.670380}%
\pgfsetdash{}{0pt}%
\pgfpathmoveto{\pgfqpoint{0.931783in}{1.204344in}}%
\pgfpathcurveto{\pgfqpoint{0.940019in}{1.204344in}}{\pgfqpoint{0.947919in}{1.207616in}}{\pgfqpoint{0.953743in}{1.213440in}}%
\pgfpathcurveto{\pgfqpoint{0.959567in}{1.219264in}}{\pgfqpoint{0.962840in}{1.227164in}}{\pgfqpoint{0.962840in}{1.235400in}}%
\pgfpathcurveto{\pgfqpoint{0.962840in}{1.243637in}}{\pgfqpoint{0.959567in}{1.251537in}}{\pgfqpoint{0.953743in}{1.257361in}}%
\pgfpathcurveto{\pgfqpoint{0.947919in}{1.263184in}}{\pgfqpoint{0.940019in}{1.266457in}}{\pgfqpoint{0.931783in}{1.266457in}}%
\pgfpathcurveto{\pgfqpoint{0.923547in}{1.266457in}}{\pgfqpoint{0.915647in}{1.263184in}}{\pgfqpoint{0.909823in}{1.257361in}}%
\pgfpathcurveto{\pgfqpoint{0.903999in}{1.251537in}}{\pgfqpoint{0.900727in}{1.243637in}}{\pgfqpoint{0.900727in}{1.235400in}}%
\pgfpathcurveto{\pgfqpoint{0.900727in}{1.227164in}}{\pgfqpoint{0.903999in}{1.219264in}}{\pgfqpoint{0.909823in}{1.213440in}}%
\pgfpathcurveto{\pgfqpoint{0.915647in}{1.207616in}}{\pgfqpoint{0.923547in}{1.204344in}}{\pgfqpoint{0.931783in}{1.204344in}}%
\pgfpathclose%
\pgfusepath{stroke,fill}%
\end{pgfscope}%
\begin{pgfscope}%
\pgfpathrectangle{\pgfqpoint{0.100000in}{0.212622in}}{\pgfqpoint{3.696000in}{3.696000in}}%
\pgfusepath{clip}%
\pgfsetbuttcap%
\pgfsetroundjoin%
\definecolor{currentfill}{rgb}{0.121569,0.466667,0.705882}%
\pgfsetfillcolor{currentfill}%
\pgfsetfillopacity{0.672141}%
\pgfsetlinewidth{1.003750pt}%
\definecolor{currentstroke}{rgb}{0.121569,0.466667,0.705882}%
\pgfsetstrokecolor{currentstroke}%
\pgfsetstrokeopacity{0.672141}%
\pgfsetdash{}{0pt}%
\pgfpathmoveto{\pgfqpoint{2.156477in}{1.860995in}}%
\pgfpathcurveto{\pgfqpoint{2.164713in}{1.860995in}}{\pgfqpoint{2.172613in}{1.864267in}}{\pgfqpoint{2.178437in}{1.870091in}}%
\pgfpathcurveto{\pgfqpoint{2.184261in}{1.875915in}}{\pgfqpoint{2.187534in}{1.883815in}}{\pgfqpoint{2.187534in}{1.892051in}}%
\pgfpathcurveto{\pgfqpoint{2.187534in}{1.900288in}}{\pgfqpoint{2.184261in}{1.908188in}}{\pgfqpoint{2.178437in}{1.914012in}}%
\pgfpathcurveto{\pgfqpoint{2.172613in}{1.919835in}}{\pgfqpoint{2.164713in}{1.923108in}}{\pgfqpoint{2.156477in}{1.923108in}}%
\pgfpathcurveto{\pgfqpoint{2.148241in}{1.923108in}}{\pgfqpoint{2.140341in}{1.919835in}}{\pgfqpoint{2.134517in}{1.914012in}}%
\pgfpathcurveto{\pgfqpoint{2.128693in}{1.908188in}}{\pgfqpoint{2.125421in}{1.900288in}}{\pgfqpoint{2.125421in}{1.892051in}}%
\pgfpathcurveto{\pgfqpoint{2.125421in}{1.883815in}}{\pgfqpoint{2.128693in}{1.875915in}}{\pgfqpoint{2.134517in}{1.870091in}}%
\pgfpathcurveto{\pgfqpoint{2.140341in}{1.864267in}}{\pgfqpoint{2.148241in}{1.860995in}}{\pgfqpoint{2.156477in}{1.860995in}}%
\pgfpathclose%
\pgfusepath{stroke,fill}%
\end{pgfscope}%
\begin{pgfscope}%
\pgfpathrectangle{\pgfqpoint{0.100000in}{0.212622in}}{\pgfqpoint{3.696000in}{3.696000in}}%
\pgfusepath{clip}%
\pgfsetbuttcap%
\pgfsetroundjoin%
\definecolor{currentfill}{rgb}{0.121569,0.466667,0.705882}%
\pgfsetfillcolor{currentfill}%
\pgfsetfillopacity{0.672376}%
\pgfsetlinewidth{1.003750pt}%
\definecolor{currentstroke}{rgb}{0.121569,0.466667,0.705882}%
\pgfsetstrokecolor{currentstroke}%
\pgfsetstrokeopacity{0.672376}%
\pgfsetdash{}{0pt}%
\pgfpathmoveto{\pgfqpoint{0.941319in}{1.203863in}}%
\pgfpathcurveto{\pgfqpoint{0.949555in}{1.203863in}}{\pgfqpoint{0.957455in}{1.207136in}}{\pgfqpoint{0.963279in}{1.212960in}}%
\pgfpathcurveto{\pgfqpoint{0.969103in}{1.218784in}}{\pgfqpoint{0.972375in}{1.226684in}}{\pgfqpoint{0.972375in}{1.234920in}}%
\pgfpathcurveto{\pgfqpoint{0.972375in}{1.243156in}}{\pgfqpoint{0.969103in}{1.251056in}}{\pgfqpoint{0.963279in}{1.256880in}}%
\pgfpathcurveto{\pgfqpoint{0.957455in}{1.262704in}}{\pgfqpoint{0.949555in}{1.265976in}}{\pgfqpoint{0.941319in}{1.265976in}}%
\pgfpathcurveto{\pgfqpoint{0.933082in}{1.265976in}}{\pgfqpoint{0.925182in}{1.262704in}}{\pgfqpoint{0.919358in}{1.256880in}}%
\pgfpathcurveto{\pgfqpoint{0.913534in}{1.251056in}}{\pgfqpoint{0.910262in}{1.243156in}}{\pgfqpoint{0.910262in}{1.234920in}}%
\pgfpathcurveto{\pgfqpoint{0.910262in}{1.226684in}}{\pgfqpoint{0.913534in}{1.218784in}}{\pgfqpoint{0.919358in}{1.212960in}}%
\pgfpathcurveto{\pgfqpoint{0.925182in}{1.207136in}}{\pgfqpoint{0.933082in}{1.203863in}}{\pgfqpoint{0.941319in}{1.203863in}}%
\pgfpathclose%
\pgfusepath{stroke,fill}%
\end{pgfscope}%
\begin{pgfscope}%
\pgfpathrectangle{\pgfqpoint{0.100000in}{0.212622in}}{\pgfqpoint{3.696000in}{3.696000in}}%
\pgfusepath{clip}%
\pgfsetbuttcap%
\pgfsetroundjoin%
\definecolor{currentfill}{rgb}{0.121569,0.466667,0.705882}%
\pgfsetfillcolor{currentfill}%
\pgfsetfillopacity{0.674140}%
\pgfsetlinewidth{1.003750pt}%
\definecolor{currentstroke}{rgb}{0.121569,0.466667,0.705882}%
\pgfsetstrokecolor{currentstroke}%
\pgfsetstrokeopacity{0.674140}%
\pgfsetdash{}{0pt}%
\pgfpathmoveto{\pgfqpoint{0.950460in}{1.203272in}}%
\pgfpathcurveto{\pgfqpoint{0.958697in}{1.203272in}}{\pgfqpoint{0.966597in}{1.206544in}}{\pgfqpoint{0.972421in}{1.212368in}}%
\pgfpathcurveto{\pgfqpoint{0.978245in}{1.218192in}}{\pgfqpoint{0.981517in}{1.226092in}}{\pgfqpoint{0.981517in}{1.234329in}}%
\pgfpathcurveto{\pgfqpoint{0.981517in}{1.242565in}}{\pgfqpoint{0.978245in}{1.250465in}}{\pgfqpoint{0.972421in}{1.256289in}}%
\pgfpathcurveto{\pgfqpoint{0.966597in}{1.262113in}}{\pgfqpoint{0.958697in}{1.265385in}}{\pgfqpoint{0.950460in}{1.265385in}}%
\pgfpathcurveto{\pgfqpoint{0.942224in}{1.265385in}}{\pgfqpoint{0.934324in}{1.262113in}}{\pgfqpoint{0.928500in}{1.256289in}}%
\pgfpathcurveto{\pgfqpoint{0.922676in}{1.250465in}}{\pgfqpoint{0.919404in}{1.242565in}}{\pgfqpoint{0.919404in}{1.234329in}}%
\pgfpathcurveto{\pgfqpoint{0.919404in}{1.226092in}}{\pgfqpoint{0.922676in}{1.218192in}}{\pgfqpoint{0.928500in}{1.212368in}}%
\pgfpathcurveto{\pgfqpoint{0.934324in}{1.206544in}}{\pgfqpoint{0.942224in}{1.203272in}}{\pgfqpoint{0.950460in}{1.203272in}}%
\pgfpathclose%
\pgfusepath{stroke,fill}%
\end{pgfscope}%
\begin{pgfscope}%
\pgfpathrectangle{\pgfqpoint{0.100000in}{0.212622in}}{\pgfqpoint{3.696000in}{3.696000in}}%
\pgfusepath{clip}%
\pgfsetbuttcap%
\pgfsetroundjoin%
\definecolor{currentfill}{rgb}{0.121569,0.466667,0.705882}%
\pgfsetfillcolor{currentfill}%
\pgfsetfillopacity{0.675792}%
\pgfsetlinewidth{1.003750pt}%
\definecolor{currentstroke}{rgb}{0.121569,0.466667,0.705882}%
\pgfsetstrokecolor{currentstroke}%
\pgfsetstrokeopacity{0.675792}%
\pgfsetdash{}{0pt}%
\pgfpathmoveto{\pgfqpoint{0.959279in}{1.202750in}}%
\pgfpathcurveto{\pgfqpoint{0.967515in}{1.202750in}}{\pgfqpoint{0.975415in}{1.206023in}}{\pgfqpoint{0.981239in}{1.211847in}}%
\pgfpathcurveto{\pgfqpoint{0.987063in}{1.217670in}}{\pgfqpoint{0.990336in}{1.225570in}}{\pgfqpoint{0.990336in}{1.233807in}}%
\pgfpathcurveto{\pgfqpoint{0.990336in}{1.242043in}}{\pgfqpoint{0.987063in}{1.249943in}}{\pgfqpoint{0.981239in}{1.255767in}}%
\pgfpathcurveto{\pgfqpoint{0.975415in}{1.261591in}}{\pgfqpoint{0.967515in}{1.264863in}}{\pgfqpoint{0.959279in}{1.264863in}}%
\pgfpathcurveto{\pgfqpoint{0.951043in}{1.264863in}}{\pgfqpoint{0.943143in}{1.261591in}}{\pgfqpoint{0.937319in}{1.255767in}}%
\pgfpathcurveto{\pgfqpoint{0.931495in}{1.249943in}}{\pgfqpoint{0.928223in}{1.242043in}}{\pgfqpoint{0.928223in}{1.233807in}}%
\pgfpathcurveto{\pgfqpoint{0.928223in}{1.225570in}}{\pgfqpoint{0.931495in}{1.217670in}}{\pgfqpoint{0.937319in}{1.211847in}}%
\pgfpathcurveto{\pgfqpoint{0.943143in}{1.206023in}}{\pgfqpoint{0.951043in}{1.202750in}}{\pgfqpoint{0.959279in}{1.202750in}}%
\pgfpathclose%
\pgfusepath{stroke,fill}%
\end{pgfscope}%
\begin{pgfscope}%
\pgfpathrectangle{\pgfqpoint{0.100000in}{0.212622in}}{\pgfqpoint{3.696000in}{3.696000in}}%
\pgfusepath{clip}%
\pgfsetbuttcap%
\pgfsetroundjoin%
\definecolor{currentfill}{rgb}{0.121569,0.466667,0.705882}%
\pgfsetfillcolor{currentfill}%
\pgfsetfillopacity{0.677342}%
\pgfsetlinewidth{1.003750pt}%
\definecolor{currentstroke}{rgb}{0.121569,0.466667,0.705882}%
\pgfsetstrokecolor{currentstroke}%
\pgfsetstrokeopacity{0.677342}%
\pgfsetdash{}{0pt}%
\pgfpathmoveto{\pgfqpoint{0.966501in}{1.202492in}}%
\pgfpathcurveto{\pgfqpoint{0.974737in}{1.202492in}}{\pgfqpoint{0.982638in}{1.205765in}}{\pgfqpoint{0.988461in}{1.211589in}}%
\pgfpathcurveto{\pgfqpoint{0.994285in}{1.217413in}}{\pgfqpoint{0.997558in}{1.225313in}}{\pgfqpoint{0.997558in}{1.233549in}}%
\pgfpathcurveto{\pgfqpoint{0.997558in}{1.241785in}}{\pgfqpoint{0.994285in}{1.249685in}}{\pgfqpoint{0.988461in}{1.255509in}}%
\pgfpathcurveto{\pgfqpoint{0.982638in}{1.261333in}}{\pgfqpoint{0.974737in}{1.264605in}}{\pgfqpoint{0.966501in}{1.264605in}}%
\pgfpathcurveto{\pgfqpoint{0.958265in}{1.264605in}}{\pgfqpoint{0.950365in}{1.261333in}}{\pgfqpoint{0.944541in}{1.255509in}}%
\pgfpathcurveto{\pgfqpoint{0.938717in}{1.249685in}}{\pgfqpoint{0.935445in}{1.241785in}}{\pgfqpoint{0.935445in}{1.233549in}}%
\pgfpathcurveto{\pgfqpoint{0.935445in}{1.225313in}}{\pgfqpoint{0.938717in}{1.217413in}}{\pgfqpoint{0.944541in}{1.211589in}}%
\pgfpathcurveto{\pgfqpoint{0.950365in}{1.205765in}}{\pgfqpoint{0.958265in}{1.202492in}}{\pgfqpoint{0.966501in}{1.202492in}}%
\pgfpathclose%
\pgfusepath{stroke,fill}%
\end{pgfscope}%
\begin{pgfscope}%
\pgfpathrectangle{\pgfqpoint{0.100000in}{0.212622in}}{\pgfqpoint{3.696000in}{3.696000in}}%
\pgfusepath{clip}%
\pgfsetbuttcap%
\pgfsetroundjoin%
\definecolor{currentfill}{rgb}{0.121569,0.466667,0.705882}%
\pgfsetfillcolor{currentfill}%
\pgfsetfillopacity{0.678430}%
\pgfsetlinewidth{1.003750pt}%
\definecolor{currentstroke}{rgb}{0.121569,0.466667,0.705882}%
\pgfsetstrokecolor{currentstroke}%
\pgfsetstrokeopacity{0.678430}%
\pgfsetdash{}{0pt}%
\pgfpathmoveto{\pgfqpoint{2.160048in}{1.844477in}}%
\pgfpathcurveto{\pgfqpoint{2.168284in}{1.844477in}}{\pgfqpoint{2.176184in}{1.847750in}}{\pgfqpoint{2.182008in}{1.853574in}}%
\pgfpathcurveto{\pgfqpoint{2.187832in}{1.859398in}}{\pgfqpoint{2.191105in}{1.867298in}}{\pgfqpoint{2.191105in}{1.875534in}}%
\pgfpathcurveto{\pgfqpoint{2.191105in}{1.883770in}}{\pgfqpoint{2.187832in}{1.891670in}}{\pgfqpoint{2.182008in}{1.897494in}}%
\pgfpathcurveto{\pgfqpoint{2.176184in}{1.903318in}}{\pgfqpoint{2.168284in}{1.906590in}}{\pgfqpoint{2.160048in}{1.906590in}}%
\pgfpathcurveto{\pgfqpoint{2.151812in}{1.906590in}}{\pgfqpoint{2.143912in}{1.903318in}}{\pgfqpoint{2.138088in}{1.897494in}}%
\pgfpathcurveto{\pgfqpoint{2.132264in}{1.891670in}}{\pgfqpoint{2.128992in}{1.883770in}}{\pgfqpoint{2.128992in}{1.875534in}}%
\pgfpathcurveto{\pgfqpoint{2.128992in}{1.867298in}}{\pgfqpoint{2.132264in}{1.859398in}}{\pgfqpoint{2.138088in}{1.853574in}}%
\pgfpathcurveto{\pgfqpoint{2.143912in}{1.847750in}}{\pgfqpoint{2.151812in}{1.844477in}}{\pgfqpoint{2.160048in}{1.844477in}}%
\pgfpathclose%
\pgfusepath{stroke,fill}%
\end{pgfscope}%
\begin{pgfscope}%
\pgfpathrectangle{\pgfqpoint{0.100000in}{0.212622in}}{\pgfqpoint{3.696000in}{3.696000in}}%
\pgfusepath{clip}%
\pgfsetbuttcap%
\pgfsetroundjoin%
\definecolor{currentfill}{rgb}{0.121569,0.466667,0.705882}%
\pgfsetfillcolor{currentfill}%
\pgfsetfillopacity{0.678762}%
\pgfsetlinewidth{1.003750pt}%
\definecolor{currentstroke}{rgb}{0.121569,0.466667,0.705882}%
\pgfsetstrokecolor{currentstroke}%
\pgfsetstrokeopacity{0.678762}%
\pgfsetdash{}{0pt}%
\pgfpathmoveto{\pgfqpoint{0.973617in}{1.202407in}}%
\pgfpathcurveto{\pgfqpoint{0.981853in}{1.202407in}}{\pgfqpoint{0.989753in}{1.205680in}}{\pgfqpoint{0.995577in}{1.211504in}}%
\pgfpathcurveto{\pgfqpoint{1.001401in}{1.217328in}}{\pgfqpoint{1.004673in}{1.225228in}}{\pgfqpoint{1.004673in}{1.233464in}}%
\pgfpathcurveto{\pgfqpoint{1.004673in}{1.241700in}}{\pgfqpoint{1.001401in}{1.249600in}}{\pgfqpoint{0.995577in}{1.255424in}}%
\pgfpathcurveto{\pgfqpoint{0.989753in}{1.261248in}}{\pgfqpoint{0.981853in}{1.264520in}}{\pgfqpoint{0.973617in}{1.264520in}}%
\pgfpathcurveto{\pgfqpoint{0.965380in}{1.264520in}}{\pgfqpoint{0.957480in}{1.261248in}}{\pgfqpoint{0.951656in}{1.255424in}}%
\pgfpathcurveto{\pgfqpoint{0.945833in}{1.249600in}}{\pgfqpoint{0.942560in}{1.241700in}}{\pgfqpoint{0.942560in}{1.233464in}}%
\pgfpathcurveto{\pgfqpoint{0.942560in}{1.225228in}}{\pgfqpoint{0.945833in}{1.217328in}}{\pgfqpoint{0.951656in}{1.211504in}}%
\pgfpathcurveto{\pgfqpoint{0.957480in}{1.205680in}}{\pgfqpoint{0.965380in}{1.202407in}}{\pgfqpoint{0.973617in}{1.202407in}}%
\pgfpathclose%
\pgfusepath{stroke,fill}%
\end{pgfscope}%
\begin{pgfscope}%
\pgfpathrectangle{\pgfqpoint{0.100000in}{0.212622in}}{\pgfqpoint{3.696000in}{3.696000in}}%
\pgfusepath{clip}%
\pgfsetbuttcap%
\pgfsetroundjoin%
\definecolor{currentfill}{rgb}{0.121569,0.466667,0.705882}%
\pgfsetfillcolor{currentfill}%
\pgfsetfillopacity{0.679957}%
\pgfsetlinewidth{1.003750pt}%
\definecolor{currentstroke}{rgb}{0.121569,0.466667,0.705882}%
\pgfsetstrokecolor{currentstroke}%
\pgfsetstrokeopacity{0.679957}%
\pgfsetdash{}{0pt}%
\pgfpathmoveto{\pgfqpoint{0.980056in}{1.202276in}}%
\pgfpathcurveto{\pgfqpoint{0.988292in}{1.202276in}}{\pgfqpoint{0.996192in}{1.205548in}}{\pgfqpoint{1.002016in}{1.211372in}}%
\pgfpathcurveto{\pgfqpoint{1.007840in}{1.217196in}}{\pgfqpoint{1.011112in}{1.225096in}}{\pgfqpoint{1.011112in}{1.233332in}}%
\pgfpathcurveto{\pgfqpoint{1.011112in}{1.241568in}}{\pgfqpoint{1.007840in}{1.249468in}}{\pgfqpoint{1.002016in}{1.255292in}}%
\pgfpathcurveto{\pgfqpoint{0.996192in}{1.261116in}}{\pgfqpoint{0.988292in}{1.264389in}}{\pgfqpoint{0.980056in}{1.264389in}}%
\pgfpathcurveto{\pgfqpoint{0.971820in}{1.264389in}}{\pgfqpoint{0.963920in}{1.261116in}}{\pgfqpoint{0.958096in}{1.255292in}}%
\pgfpathcurveto{\pgfqpoint{0.952272in}{1.249468in}}{\pgfqpoint{0.948999in}{1.241568in}}{\pgfqpoint{0.948999in}{1.233332in}}%
\pgfpathcurveto{\pgfqpoint{0.948999in}{1.225096in}}{\pgfqpoint{0.952272in}{1.217196in}}{\pgfqpoint{0.958096in}{1.211372in}}%
\pgfpathcurveto{\pgfqpoint{0.963920in}{1.205548in}}{\pgfqpoint{0.971820in}{1.202276in}}{\pgfqpoint{0.980056in}{1.202276in}}%
\pgfpathclose%
\pgfusepath{stroke,fill}%
\end{pgfscope}%
\begin{pgfscope}%
\pgfpathrectangle{\pgfqpoint{0.100000in}{0.212622in}}{\pgfqpoint{3.696000in}{3.696000in}}%
\pgfusepath{clip}%
\pgfsetbuttcap%
\pgfsetroundjoin%
\definecolor{currentfill}{rgb}{0.121569,0.466667,0.705882}%
\pgfsetfillcolor{currentfill}%
\pgfsetfillopacity{0.680975}%
\pgfsetlinewidth{1.003750pt}%
\definecolor{currentstroke}{rgb}{0.121569,0.466667,0.705882}%
\pgfsetstrokecolor{currentstroke}%
\pgfsetstrokeopacity{0.680975}%
\pgfsetdash{}{0pt}%
\pgfpathmoveto{\pgfqpoint{0.984913in}{1.202154in}}%
\pgfpathcurveto{\pgfqpoint{0.993149in}{1.202154in}}{\pgfqpoint{1.001049in}{1.205427in}}{\pgfqpoint{1.006873in}{1.211251in}}%
\pgfpathcurveto{\pgfqpoint{1.012697in}{1.217075in}}{\pgfqpoint{1.015969in}{1.224975in}}{\pgfqpoint{1.015969in}{1.233211in}}%
\pgfpathcurveto{\pgfqpoint{1.015969in}{1.241447in}}{\pgfqpoint{1.012697in}{1.249347in}}{\pgfqpoint{1.006873in}{1.255171in}}%
\pgfpathcurveto{\pgfqpoint{1.001049in}{1.260995in}}{\pgfqpoint{0.993149in}{1.264267in}}{\pgfqpoint{0.984913in}{1.264267in}}%
\pgfpathcurveto{\pgfqpoint{0.976677in}{1.264267in}}{\pgfqpoint{0.968777in}{1.260995in}}{\pgfqpoint{0.962953in}{1.255171in}}%
\pgfpathcurveto{\pgfqpoint{0.957129in}{1.249347in}}{\pgfqpoint{0.953856in}{1.241447in}}{\pgfqpoint{0.953856in}{1.233211in}}%
\pgfpathcurveto{\pgfqpoint{0.953856in}{1.224975in}}{\pgfqpoint{0.957129in}{1.217075in}}{\pgfqpoint{0.962953in}{1.211251in}}%
\pgfpathcurveto{\pgfqpoint{0.968777in}{1.205427in}}{\pgfqpoint{0.976677in}{1.202154in}}{\pgfqpoint{0.984913in}{1.202154in}}%
\pgfpathclose%
\pgfusepath{stroke,fill}%
\end{pgfscope}%
\begin{pgfscope}%
\pgfpathrectangle{\pgfqpoint{0.100000in}{0.212622in}}{\pgfqpoint{3.696000in}{3.696000in}}%
\pgfusepath{clip}%
\pgfsetbuttcap%
\pgfsetroundjoin%
\definecolor{currentfill}{rgb}{0.121569,0.466667,0.705882}%
\pgfsetfillcolor{currentfill}%
\pgfsetfillopacity{0.682990}%
\pgfsetlinewidth{1.003750pt}%
\definecolor{currentstroke}{rgb}{0.121569,0.466667,0.705882}%
\pgfsetstrokecolor{currentstroke}%
\pgfsetstrokeopacity{0.682990}%
\pgfsetdash{}{0pt}%
\pgfpathmoveto{\pgfqpoint{0.993716in}{1.202478in}}%
\pgfpathcurveto{\pgfqpoint{1.001952in}{1.202478in}}{\pgfqpoint{1.009852in}{1.205750in}}{\pgfqpoint{1.015676in}{1.211574in}}%
\pgfpathcurveto{\pgfqpoint{1.021500in}{1.217398in}}{\pgfqpoint{1.024773in}{1.225298in}}{\pgfqpoint{1.024773in}{1.233535in}}%
\pgfpathcurveto{\pgfqpoint{1.024773in}{1.241771in}}{\pgfqpoint{1.021500in}{1.249671in}}{\pgfqpoint{1.015676in}{1.255495in}}%
\pgfpathcurveto{\pgfqpoint{1.009852in}{1.261319in}}{\pgfqpoint{1.001952in}{1.264591in}}{\pgfqpoint{0.993716in}{1.264591in}}%
\pgfpathcurveto{\pgfqpoint{0.985480in}{1.264591in}}{\pgfqpoint{0.977580in}{1.261319in}}{\pgfqpoint{0.971756in}{1.255495in}}%
\pgfpathcurveto{\pgfqpoint{0.965932in}{1.249671in}}{\pgfqpoint{0.962660in}{1.241771in}}{\pgfqpoint{0.962660in}{1.233535in}}%
\pgfpathcurveto{\pgfqpoint{0.962660in}{1.225298in}}{\pgfqpoint{0.965932in}{1.217398in}}{\pgfqpoint{0.971756in}{1.211574in}}%
\pgfpathcurveto{\pgfqpoint{0.977580in}{1.205750in}}{\pgfqpoint{0.985480in}{1.202478in}}{\pgfqpoint{0.993716in}{1.202478in}}%
\pgfpathclose%
\pgfusepath{stroke,fill}%
\end{pgfscope}%
\begin{pgfscope}%
\pgfpathrectangle{\pgfqpoint{0.100000in}{0.212622in}}{\pgfqpoint{3.696000in}{3.696000in}}%
\pgfusepath{clip}%
\pgfsetbuttcap%
\pgfsetroundjoin%
\definecolor{currentfill}{rgb}{0.121569,0.466667,0.705882}%
\pgfsetfillcolor{currentfill}%
\pgfsetfillopacity{0.684424}%
\pgfsetlinewidth{1.003750pt}%
\definecolor{currentstroke}{rgb}{0.121569,0.466667,0.705882}%
\pgfsetstrokecolor{currentstroke}%
\pgfsetstrokeopacity{0.684424}%
\pgfsetdash{}{0pt}%
\pgfpathmoveto{\pgfqpoint{1.001094in}{1.202867in}}%
\pgfpathcurveto{\pgfqpoint{1.009330in}{1.202867in}}{\pgfqpoint{1.017230in}{1.206140in}}{\pgfqpoint{1.023054in}{1.211964in}}%
\pgfpathcurveto{\pgfqpoint{1.028878in}{1.217788in}}{\pgfqpoint{1.032151in}{1.225688in}}{\pgfqpoint{1.032151in}{1.233924in}}%
\pgfpathcurveto{\pgfqpoint{1.032151in}{1.242160in}}{\pgfqpoint{1.028878in}{1.250060in}}{\pgfqpoint{1.023054in}{1.255884in}}%
\pgfpathcurveto{\pgfqpoint{1.017230in}{1.261708in}}{\pgfqpoint{1.009330in}{1.264980in}}{\pgfqpoint{1.001094in}{1.264980in}}%
\pgfpathcurveto{\pgfqpoint{0.992858in}{1.264980in}}{\pgfqpoint{0.984958in}{1.261708in}}{\pgfqpoint{0.979134in}{1.255884in}}%
\pgfpathcurveto{\pgfqpoint{0.973310in}{1.250060in}}{\pgfqpoint{0.970038in}{1.242160in}}{\pgfqpoint{0.970038in}{1.233924in}}%
\pgfpathcurveto{\pgfqpoint{0.970038in}{1.225688in}}{\pgfqpoint{0.973310in}{1.217788in}}{\pgfqpoint{0.979134in}{1.211964in}}%
\pgfpathcurveto{\pgfqpoint{0.984958in}{1.206140in}}{\pgfqpoint{0.992858in}{1.202867in}}{\pgfqpoint{1.001094in}{1.202867in}}%
\pgfpathclose%
\pgfusepath{stroke,fill}%
\end{pgfscope}%
\begin{pgfscope}%
\pgfpathrectangle{\pgfqpoint{0.100000in}{0.212622in}}{\pgfqpoint{3.696000in}{3.696000in}}%
\pgfusepath{clip}%
\pgfsetbuttcap%
\pgfsetroundjoin%
\definecolor{currentfill}{rgb}{0.121569,0.466667,0.705882}%
\pgfsetfillcolor{currentfill}%
\pgfsetfillopacity{0.684915}%
\pgfsetlinewidth{1.003750pt}%
\definecolor{currentstroke}{rgb}{0.121569,0.466667,0.705882}%
\pgfsetstrokecolor{currentstroke}%
\pgfsetstrokeopacity{0.684915}%
\pgfsetdash{}{0pt}%
\pgfpathmoveto{\pgfqpoint{2.166055in}{1.826284in}}%
\pgfpathcurveto{\pgfqpoint{2.174291in}{1.826284in}}{\pgfqpoint{2.182191in}{1.829556in}}{\pgfqpoint{2.188015in}{1.835380in}}%
\pgfpathcurveto{\pgfqpoint{2.193839in}{1.841204in}}{\pgfqpoint{2.197111in}{1.849104in}}{\pgfqpoint{2.197111in}{1.857341in}}%
\pgfpathcurveto{\pgfqpoint{2.197111in}{1.865577in}}{\pgfqpoint{2.193839in}{1.873477in}}{\pgfqpoint{2.188015in}{1.879301in}}%
\pgfpathcurveto{\pgfqpoint{2.182191in}{1.885125in}}{\pgfqpoint{2.174291in}{1.888397in}}{\pgfqpoint{2.166055in}{1.888397in}}%
\pgfpathcurveto{\pgfqpoint{2.157819in}{1.888397in}}{\pgfqpoint{2.149919in}{1.885125in}}{\pgfqpoint{2.144095in}{1.879301in}}%
\pgfpathcurveto{\pgfqpoint{2.138271in}{1.873477in}}{\pgfqpoint{2.134998in}{1.865577in}}{\pgfqpoint{2.134998in}{1.857341in}}%
\pgfpathcurveto{\pgfqpoint{2.134998in}{1.849104in}}{\pgfqpoint{2.138271in}{1.841204in}}{\pgfqpoint{2.144095in}{1.835380in}}%
\pgfpathcurveto{\pgfqpoint{2.149919in}{1.829556in}}{\pgfqpoint{2.157819in}{1.826284in}}{\pgfqpoint{2.166055in}{1.826284in}}%
\pgfpathclose%
\pgfusepath{stroke,fill}%
\end{pgfscope}%
\begin{pgfscope}%
\pgfpathrectangle{\pgfqpoint{0.100000in}{0.212622in}}{\pgfqpoint{3.696000in}{3.696000in}}%
\pgfusepath{clip}%
\pgfsetbuttcap%
\pgfsetroundjoin%
\definecolor{currentfill}{rgb}{0.121569,0.466667,0.705882}%
\pgfsetfillcolor{currentfill}%
\pgfsetfillopacity{0.686006}%
\pgfsetlinewidth{1.003750pt}%
\definecolor{currentstroke}{rgb}{0.121569,0.466667,0.705882}%
\pgfsetstrokecolor{currentstroke}%
\pgfsetstrokeopacity{0.686006}%
\pgfsetdash{}{0pt}%
\pgfpathmoveto{\pgfqpoint{1.007963in}{1.203567in}}%
\pgfpathcurveto{\pgfqpoint{1.016200in}{1.203567in}}{\pgfqpoint{1.024100in}{1.206840in}}{\pgfqpoint{1.029924in}{1.212664in}}%
\pgfpathcurveto{\pgfqpoint{1.035748in}{1.218488in}}{\pgfqpoint{1.039020in}{1.226388in}}{\pgfqpoint{1.039020in}{1.234624in}}%
\pgfpathcurveto{\pgfqpoint{1.039020in}{1.242860in}}{\pgfqpoint{1.035748in}{1.250760in}}{\pgfqpoint{1.029924in}{1.256584in}}%
\pgfpathcurveto{\pgfqpoint{1.024100in}{1.262408in}}{\pgfqpoint{1.016200in}{1.265680in}}{\pgfqpoint{1.007963in}{1.265680in}}%
\pgfpathcurveto{\pgfqpoint{0.999727in}{1.265680in}}{\pgfqpoint{0.991827in}{1.262408in}}{\pgfqpoint{0.986003in}{1.256584in}}%
\pgfpathcurveto{\pgfqpoint{0.980179in}{1.250760in}}{\pgfqpoint{0.976907in}{1.242860in}}{\pgfqpoint{0.976907in}{1.234624in}}%
\pgfpathcurveto{\pgfqpoint{0.976907in}{1.226388in}}{\pgfqpoint{0.980179in}{1.218488in}}{\pgfqpoint{0.986003in}{1.212664in}}%
\pgfpathcurveto{\pgfqpoint{0.991827in}{1.206840in}}{\pgfqpoint{0.999727in}{1.203567in}}{\pgfqpoint{1.007963in}{1.203567in}}%
\pgfpathclose%
\pgfusepath{stroke,fill}%
\end{pgfscope}%
\begin{pgfscope}%
\pgfpathrectangle{\pgfqpoint{0.100000in}{0.212622in}}{\pgfqpoint{3.696000in}{3.696000in}}%
\pgfusepath{clip}%
\pgfsetbuttcap%
\pgfsetroundjoin%
\definecolor{currentfill}{rgb}{0.121569,0.466667,0.705882}%
\pgfsetfillcolor{currentfill}%
\pgfsetfillopacity{0.687404}%
\pgfsetlinewidth{1.003750pt}%
\definecolor{currentstroke}{rgb}{0.121569,0.466667,0.705882}%
\pgfsetstrokecolor{currentstroke}%
\pgfsetstrokeopacity{0.687404}%
\pgfsetdash{}{0pt}%
\pgfpathmoveto{\pgfqpoint{1.014000in}{1.203795in}}%
\pgfpathcurveto{\pgfqpoint{1.022236in}{1.203795in}}{\pgfqpoint{1.030136in}{1.207068in}}{\pgfqpoint{1.035960in}{1.212892in}}%
\pgfpathcurveto{\pgfqpoint{1.041784in}{1.218716in}}{\pgfqpoint{1.045057in}{1.226616in}}{\pgfqpoint{1.045057in}{1.234852in}}%
\pgfpathcurveto{\pgfqpoint{1.045057in}{1.243088in}}{\pgfqpoint{1.041784in}{1.250988in}}{\pgfqpoint{1.035960in}{1.256812in}}%
\pgfpathcurveto{\pgfqpoint{1.030136in}{1.262636in}}{\pgfqpoint{1.022236in}{1.265908in}}{\pgfqpoint{1.014000in}{1.265908in}}%
\pgfpathcurveto{\pgfqpoint{1.005764in}{1.265908in}}{\pgfqpoint{0.997864in}{1.262636in}}{\pgfqpoint{0.992040in}{1.256812in}}%
\pgfpathcurveto{\pgfqpoint{0.986216in}{1.250988in}}{\pgfqpoint{0.982944in}{1.243088in}}{\pgfqpoint{0.982944in}{1.234852in}}%
\pgfpathcurveto{\pgfqpoint{0.982944in}{1.226616in}}{\pgfqpoint{0.986216in}{1.218716in}}{\pgfqpoint{0.992040in}{1.212892in}}%
\pgfpathcurveto{\pgfqpoint{0.997864in}{1.207068in}}{\pgfqpoint{1.005764in}{1.203795in}}{\pgfqpoint{1.014000in}{1.203795in}}%
\pgfpathclose%
\pgfusepath{stroke,fill}%
\end{pgfscope}%
\begin{pgfscope}%
\pgfpathrectangle{\pgfqpoint{0.100000in}{0.212622in}}{\pgfqpoint{3.696000in}{3.696000in}}%
\pgfusepath{clip}%
\pgfsetbuttcap%
\pgfsetroundjoin%
\definecolor{currentfill}{rgb}{0.121569,0.466667,0.705882}%
\pgfsetfillcolor{currentfill}%
\pgfsetfillopacity{0.688388}%
\pgfsetlinewidth{1.003750pt}%
\definecolor{currentstroke}{rgb}{0.121569,0.466667,0.705882}%
\pgfsetstrokecolor{currentstroke}%
\pgfsetstrokeopacity{0.688388}%
\pgfsetdash{}{0pt}%
\pgfpathmoveto{\pgfqpoint{2.169239in}{1.815762in}}%
\pgfpathcurveto{\pgfqpoint{2.177475in}{1.815762in}}{\pgfqpoint{2.185375in}{1.819034in}}{\pgfqpoint{2.191199in}{1.824858in}}%
\pgfpathcurveto{\pgfqpoint{2.197023in}{1.830682in}}{\pgfqpoint{2.200296in}{1.838582in}}{\pgfqpoint{2.200296in}{1.846818in}}%
\pgfpathcurveto{\pgfqpoint{2.200296in}{1.855054in}}{\pgfqpoint{2.197023in}{1.862955in}}{\pgfqpoint{2.191199in}{1.868778in}}%
\pgfpathcurveto{\pgfqpoint{2.185375in}{1.874602in}}{\pgfqpoint{2.177475in}{1.877875in}}{\pgfqpoint{2.169239in}{1.877875in}}%
\pgfpathcurveto{\pgfqpoint{2.161003in}{1.877875in}}{\pgfqpoint{2.153103in}{1.874602in}}{\pgfqpoint{2.147279in}{1.868778in}}%
\pgfpathcurveto{\pgfqpoint{2.141455in}{1.862955in}}{\pgfqpoint{2.138183in}{1.855054in}}{\pgfqpoint{2.138183in}{1.846818in}}%
\pgfpathcurveto{\pgfqpoint{2.138183in}{1.838582in}}{\pgfqpoint{2.141455in}{1.830682in}}{\pgfqpoint{2.147279in}{1.824858in}}%
\pgfpathcurveto{\pgfqpoint{2.153103in}{1.819034in}}{\pgfqpoint{2.161003in}{1.815762in}}{\pgfqpoint{2.169239in}{1.815762in}}%
\pgfpathclose%
\pgfusepath{stroke,fill}%
\end{pgfscope}%
\begin{pgfscope}%
\pgfpathrectangle{\pgfqpoint{0.100000in}{0.212622in}}{\pgfqpoint{3.696000in}{3.696000in}}%
\pgfusepath{clip}%
\pgfsetbuttcap%
\pgfsetroundjoin%
\definecolor{currentfill}{rgb}{0.121569,0.466667,0.705882}%
\pgfsetfillcolor{currentfill}%
\pgfsetfillopacity{0.688440}%
\pgfsetlinewidth{1.003750pt}%
\definecolor{currentstroke}{rgb}{0.121569,0.466667,0.705882}%
\pgfsetstrokecolor{currentstroke}%
\pgfsetstrokeopacity{0.688440}%
\pgfsetdash{}{0pt}%
\pgfpathmoveto{\pgfqpoint{1.018762in}{1.204042in}}%
\pgfpathcurveto{\pgfqpoint{1.026998in}{1.204042in}}{\pgfqpoint{1.034898in}{1.207314in}}{\pgfqpoint{1.040722in}{1.213138in}}%
\pgfpathcurveto{\pgfqpoint{1.046546in}{1.218962in}}{\pgfqpoint{1.049819in}{1.226862in}}{\pgfqpoint{1.049819in}{1.235098in}}%
\pgfpathcurveto{\pgfqpoint{1.049819in}{1.243334in}}{\pgfqpoint{1.046546in}{1.251235in}}{\pgfqpoint{1.040722in}{1.257058in}}%
\pgfpathcurveto{\pgfqpoint{1.034898in}{1.262882in}}{\pgfqpoint{1.026998in}{1.266155in}}{\pgfqpoint{1.018762in}{1.266155in}}%
\pgfpathcurveto{\pgfqpoint{1.010526in}{1.266155in}}{\pgfqpoint{1.002626in}{1.262882in}}{\pgfqpoint{0.996802in}{1.257058in}}%
\pgfpathcurveto{\pgfqpoint{0.990978in}{1.251235in}}{\pgfqpoint{0.987706in}{1.243334in}}{\pgfqpoint{0.987706in}{1.235098in}}%
\pgfpathcurveto{\pgfqpoint{0.987706in}{1.226862in}}{\pgfqpoint{0.990978in}{1.218962in}}{\pgfqpoint{0.996802in}{1.213138in}}%
\pgfpathcurveto{\pgfqpoint{1.002626in}{1.207314in}}{\pgfqpoint{1.010526in}{1.204042in}}{\pgfqpoint{1.018762in}{1.204042in}}%
\pgfpathclose%
\pgfusepath{stroke,fill}%
\end{pgfscope}%
\begin{pgfscope}%
\pgfpathrectangle{\pgfqpoint{0.100000in}{0.212622in}}{\pgfqpoint{3.696000in}{3.696000in}}%
\pgfusepath{clip}%
\pgfsetbuttcap%
\pgfsetroundjoin%
\definecolor{currentfill}{rgb}{0.121569,0.466667,0.705882}%
\pgfsetfillcolor{currentfill}%
\pgfsetfillopacity{0.689306}%
\pgfsetlinewidth{1.003750pt}%
\definecolor{currentstroke}{rgb}{0.121569,0.466667,0.705882}%
\pgfsetstrokecolor{currentstroke}%
\pgfsetstrokeopacity{0.689306}%
\pgfsetdash{}{0pt}%
\pgfpathmoveto{\pgfqpoint{1.023125in}{1.204047in}}%
\pgfpathcurveto{\pgfqpoint{1.031361in}{1.204047in}}{\pgfqpoint{1.039261in}{1.207319in}}{\pgfqpoint{1.045085in}{1.213143in}}%
\pgfpathcurveto{\pgfqpoint{1.050909in}{1.218967in}}{\pgfqpoint{1.054181in}{1.226867in}}{\pgfqpoint{1.054181in}{1.235103in}}%
\pgfpathcurveto{\pgfqpoint{1.054181in}{1.243340in}}{\pgfqpoint{1.050909in}{1.251240in}}{\pgfqpoint{1.045085in}{1.257064in}}%
\pgfpathcurveto{\pgfqpoint{1.039261in}{1.262888in}}{\pgfqpoint{1.031361in}{1.266160in}}{\pgfqpoint{1.023125in}{1.266160in}}%
\pgfpathcurveto{\pgfqpoint{1.014888in}{1.266160in}}{\pgfqpoint{1.006988in}{1.262888in}}{\pgfqpoint{1.001164in}{1.257064in}}%
\pgfpathcurveto{\pgfqpoint{0.995341in}{1.251240in}}{\pgfqpoint{0.992068in}{1.243340in}}{\pgfqpoint{0.992068in}{1.235103in}}%
\pgfpathcurveto{\pgfqpoint{0.992068in}{1.226867in}}{\pgfqpoint{0.995341in}{1.218967in}}{\pgfqpoint{1.001164in}{1.213143in}}%
\pgfpathcurveto{\pgfqpoint{1.006988in}{1.207319in}}{\pgfqpoint{1.014888in}{1.204047in}}{\pgfqpoint{1.023125in}{1.204047in}}%
\pgfpathclose%
\pgfusepath{stroke,fill}%
\end{pgfscope}%
\begin{pgfscope}%
\pgfpathrectangle{\pgfqpoint{0.100000in}{0.212622in}}{\pgfqpoint{3.696000in}{3.696000in}}%
\pgfusepath{clip}%
\pgfsetbuttcap%
\pgfsetroundjoin%
\definecolor{currentfill}{rgb}{0.121569,0.466667,0.705882}%
\pgfsetfillcolor{currentfill}%
\pgfsetfillopacity{0.689898}%
\pgfsetlinewidth{1.003750pt}%
\definecolor{currentstroke}{rgb}{0.121569,0.466667,0.705882}%
\pgfsetstrokecolor{currentstroke}%
\pgfsetstrokeopacity{0.689898}%
\pgfsetdash{}{0pt}%
\pgfpathmoveto{\pgfqpoint{1.025845in}{1.203971in}}%
\pgfpathcurveto{\pgfqpoint{1.034081in}{1.203971in}}{\pgfqpoint{1.041981in}{1.207243in}}{\pgfqpoint{1.047805in}{1.213067in}}%
\pgfpathcurveto{\pgfqpoint{1.053629in}{1.218891in}}{\pgfqpoint{1.056901in}{1.226791in}}{\pgfqpoint{1.056901in}{1.235027in}}%
\pgfpathcurveto{\pgfqpoint{1.056901in}{1.243264in}}{\pgfqpoint{1.053629in}{1.251164in}}{\pgfqpoint{1.047805in}{1.256988in}}%
\pgfpathcurveto{\pgfqpoint{1.041981in}{1.262812in}}{\pgfqpoint{1.034081in}{1.266084in}}{\pgfqpoint{1.025845in}{1.266084in}}%
\pgfpathcurveto{\pgfqpoint{1.017608in}{1.266084in}}{\pgfqpoint{1.009708in}{1.262812in}}{\pgfqpoint{1.003884in}{1.256988in}}%
\pgfpathcurveto{\pgfqpoint{0.998061in}{1.251164in}}{\pgfqpoint{0.994788in}{1.243264in}}{\pgfqpoint{0.994788in}{1.235027in}}%
\pgfpathcurveto{\pgfqpoint{0.994788in}{1.226791in}}{\pgfqpoint{0.998061in}{1.218891in}}{\pgfqpoint{1.003884in}{1.213067in}}%
\pgfpathcurveto{\pgfqpoint{1.009708in}{1.207243in}}{\pgfqpoint{1.017608in}{1.203971in}}{\pgfqpoint{1.025845in}{1.203971in}}%
\pgfpathclose%
\pgfusepath{stroke,fill}%
\end{pgfscope}%
\begin{pgfscope}%
\pgfpathrectangle{\pgfqpoint{0.100000in}{0.212622in}}{\pgfqpoint{3.696000in}{3.696000in}}%
\pgfusepath{clip}%
\pgfsetbuttcap%
\pgfsetroundjoin%
\definecolor{currentfill}{rgb}{0.121569,0.466667,0.705882}%
\pgfsetfillcolor{currentfill}%
\pgfsetfillopacity{0.691035}%
\pgfsetlinewidth{1.003750pt}%
\definecolor{currentstroke}{rgb}{0.121569,0.466667,0.705882}%
\pgfsetstrokecolor{currentstroke}%
\pgfsetstrokeopacity{0.691035}%
\pgfsetdash{}{0pt}%
\pgfpathmoveto{\pgfqpoint{1.030744in}{1.203886in}}%
\pgfpathcurveto{\pgfqpoint{1.038981in}{1.203886in}}{\pgfqpoint{1.046881in}{1.207159in}}{\pgfqpoint{1.052705in}{1.212983in}}%
\pgfpathcurveto{\pgfqpoint{1.058528in}{1.218807in}}{\pgfqpoint{1.061801in}{1.226707in}}{\pgfqpoint{1.061801in}{1.234943in}}%
\pgfpathcurveto{\pgfqpoint{1.061801in}{1.243179in}}{\pgfqpoint{1.058528in}{1.251079in}}{\pgfqpoint{1.052705in}{1.256903in}}%
\pgfpathcurveto{\pgfqpoint{1.046881in}{1.262727in}}{\pgfqpoint{1.038981in}{1.265999in}}{\pgfqpoint{1.030744in}{1.265999in}}%
\pgfpathcurveto{\pgfqpoint{1.022508in}{1.265999in}}{\pgfqpoint{1.014608in}{1.262727in}}{\pgfqpoint{1.008784in}{1.256903in}}%
\pgfpathcurveto{\pgfqpoint{1.002960in}{1.251079in}}{\pgfqpoint{0.999688in}{1.243179in}}{\pgfqpoint{0.999688in}{1.234943in}}%
\pgfpathcurveto{\pgfqpoint{0.999688in}{1.226707in}}{\pgfqpoint{1.002960in}{1.218807in}}{\pgfqpoint{1.008784in}{1.212983in}}%
\pgfpathcurveto{\pgfqpoint{1.014608in}{1.207159in}}{\pgfqpoint{1.022508in}{1.203886in}}{\pgfqpoint{1.030744in}{1.203886in}}%
\pgfpathclose%
\pgfusepath{stroke,fill}%
\end{pgfscope}%
\begin{pgfscope}%
\pgfpathrectangle{\pgfqpoint{0.100000in}{0.212622in}}{\pgfqpoint{3.696000in}{3.696000in}}%
\pgfusepath{clip}%
\pgfsetbuttcap%
\pgfsetroundjoin%
\definecolor{currentfill}{rgb}{0.121569,0.466667,0.705882}%
\pgfsetfillcolor{currentfill}%
\pgfsetfillopacity{0.691720}%
\pgfsetlinewidth{1.003750pt}%
\definecolor{currentstroke}{rgb}{0.121569,0.466667,0.705882}%
\pgfsetstrokecolor{currentstroke}%
\pgfsetstrokeopacity{0.691720}%
\pgfsetdash{}{0pt}%
\pgfpathmoveto{\pgfqpoint{1.033812in}{1.203978in}}%
\pgfpathcurveto{\pgfqpoint{1.042049in}{1.203978in}}{\pgfqpoint{1.049949in}{1.207251in}}{\pgfqpoint{1.055773in}{1.213074in}}%
\pgfpathcurveto{\pgfqpoint{1.061597in}{1.218898in}}{\pgfqpoint{1.064869in}{1.226798in}}{\pgfqpoint{1.064869in}{1.235035in}}%
\pgfpathcurveto{\pgfqpoint{1.064869in}{1.243271in}}{\pgfqpoint{1.061597in}{1.251171in}}{\pgfqpoint{1.055773in}{1.256995in}}%
\pgfpathcurveto{\pgfqpoint{1.049949in}{1.262819in}}{\pgfqpoint{1.042049in}{1.266091in}}{\pgfqpoint{1.033812in}{1.266091in}}%
\pgfpathcurveto{\pgfqpoint{1.025576in}{1.266091in}}{\pgfqpoint{1.017676in}{1.262819in}}{\pgfqpoint{1.011852in}{1.256995in}}%
\pgfpathcurveto{\pgfqpoint{1.006028in}{1.251171in}}{\pgfqpoint{1.002756in}{1.243271in}}{\pgfqpoint{1.002756in}{1.235035in}}%
\pgfpathcurveto{\pgfqpoint{1.002756in}{1.226798in}}{\pgfqpoint{1.006028in}{1.218898in}}{\pgfqpoint{1.011852in}{1.213074in}}%
\pgfpathcurveto{\pgfqpoint{1.017676in}{1.207251in}}{\pgfqpoint{1.025576in}{1.203978in}}{\pgfqpoint{1.033812in}{1.203978in}}%
\pgfpathclose%
\pgfusepath{stroke,fill}%
\end{pgfscope}%
\begin{pgfscope}%
\pgfpathrectangle{\pgfqpoint{0.100000in}{0.212622in}}{\pgfqpoint{3.696000in}{3.696000in}}%
\pgfusepath{clip}%
\pgfsetbuttcap%
\pgfsetroundjoin%
\definecolor{currentfill}{rgb}{0.121569,0.466667,0.705882}%
\pgfsetfillcolor{currentfill}%
\pgfsetfillopacity{0.692632}%
\pgfsetlinewidth{1.003750pt}%
\definecolor{currentstroke}{rgb}{0.121569,0.466667,0.705882}%
\pgfsetstrokecolor{currentstroke}%
\pgfsetstrokeopacity{0.692632}%
\pgfsetdash{}{0pt}%
\pgfpathmoveto{\pgfqpoint{2.171665in}{1.804568in}}%
\pgfpathcurveto{\pgfqpoint{2.179901in}{1.804568in}}{\pgfqpoint{2.187801in}{1.807841in}}{\pgfqpoint{2.193625in}{1.813665in}}%
\pgfpathcurveto{\pgfqpoint{2.199449in}{1.819489in}}{\pgfqpoint{2.202721in}{1.827389in}}{\pgfqpoint{2.202721in}{1.835625in}}%
\pgfpathcurveto{\pgfqpoint{2.202721in}{1.843861in}}{\pgfqpoint{2.199449in}{1.851761in}}{\pgfqpoint{2.193625in}{1.857585in}}%
\pgfpathcurveto{\pgfqpoint{2.187801in}{1.863409in}}{\pgfqpoint{2.179901in}{1.866681in}}{\pgfqpoint{2.171665in}{1.866681in}}%
\pgfpathcurveto{\pgfqpoint{2.163428in}{1.866681in}}{\pgfqpoint{2.155528in}{1.863409in}}{\pgfqpoint{2.149704in}{1.857585in}}%
\pgfpathcurveto{\pgfqpoint{2.143880in}{1.851761in}}{\pgfqpoint{2.140608in}{1.843861in}}{\pgfqpoint{2.140608in}{1.835625in}}%
\pgfpathcurveto{\pgfqpoint{2.140608in}{1.827389in}}{\pgfqpoint{2.143880in}{1.819489in}}{\pgfqpoint{2.149704in}{1.813665in}}%
\pgfpathcurveto{\pgfqpoint{2.155528in}{1.807841in}}{\pgfqpoint{2.163428in}{1.804568in}}{\pgfqpoint{2.171665in}{1.804568in}}%
\pgfpathclose%
\pgfusepath{stroke,fill}%
\end{pgfscope}%
\begin{pgfscope}%
\pgfpathrectangle{\pgfqpoint{0.100000in}{0.212622in}}{\pgfqpoint{3.696000in}{3.696000in}}%
\pgfusepath{clip}%
\pgfsetbuttcap%
\pgfsetroundjoin%
\definecolor{currentfill}{rgb}{0.121569,0.466667,0.705882}%
\pgfsetfillcolor{currentfill}%
\pgfsetfillopacity{0.693146}%
\pgfsetlinewidth{1.003750pt}%
\definecolor{currentstroke}{rgb}{0.121569,0.466667,0.705882}%
\pgfsetstrokecolor{currentstroke}%
\pgfsetstrokeopacity{0.693146}%
\pgfsetdash{}{0pt}%
\pgfpathmoveto{\pgfqpoint{1.039299in}{1.204495in}}%
\pgfpathcurveto{\pgfqpoint{1.047536in}{1.204495in}}{\pgfqpoint{1.055436in}{1.207767in}}{\pgfqpoint{1.061260in}{1.213591in}}%
\pgfpathcurveto{\pgfqpoint{1.067083in}{1.219415in}}{\pgfqpoint{1.070356in}{1.227315in}}{\pgfqpoint{1.070356in}{1.235551in}}%
\pgfpathcurveto{\pgfqpoint{1.070356in}{1.243787in}}{\pgfqpoint{1.067083in}{1.251687in}}{\pgfqpoint{1.061260in}{1.257511in}}%
\pgfpathcurveto{\pgfqpoint{1.055436in}{1.263335in}}{\pgfqpoint{1.047536in}{1.266608in}}{\pgfqpoint{1.039299in}{1.266608in}}%
\pgfpathcurveto{\pgfqpoint{1.031063in}{1.266608in}}{\pgfqpoint{1.023163in}{1.263335in}}{\pgfqpoint{1.017339in}{1.257511in}}%
\pgfpathcurveto{\pgfqpoint{1.011515in}{1.251687in}}{\pgfqpoint{1.008243in}{1.243787in}}{\pgfqpoint{1.008243in}{1.235551in}}%
\pgfpathcurveto{\pgfqpoint{1.008243in}{1.227315in}}{\pgfqpoint{1.011515in}{1.219415in}}{\pgfqpoint{1.017339in}{1.213591in}}%
\pgfpathcurveto{\pgfqpoint{1.023163in}{1.207767in}}{\pgfqpoint{1.031063in}{1.204495in}}{\pgfqpoint{1.039299in}{1.204495in}}%
\pgfpathclose%
\pgfusepath{stroke,fill}%
\end{pgfscope}%
\begin{pgfscope}%
\pgfpathrectangle{\pgfqpoint{0.100000in}{0.212622in}}{\pgfqpoint{3.696000in}{3.696000in}}%
\pgfusepath{clip}%
\pgfsetbuttcap%
\pgfsetroundjoin%
\definecolor{currentfill}{rgb}{0.121569,0.466667,0.705882}%
\pgfsetfillcolor{currentfill}%
\pgfsetfillopacity{0.694289}%
\pgfsetlinewidth{1.003750pt}%
\definecolor{currentstroke}{rgb}{0.121569,0.466667,0.705882}%
\pgfsetstrokecolor{currentstroke}%
\pgfsetstrokeopacity{0.694289}%
\pgfsetdash{}{0pt}%
\pgfpathmoveto{\pgfqpoint{1.044050in}{1.204514in}}%
\pgfpathcurveto{\pgfqpoint{1.052286in}{1.204514in}}{\pgfqpoint{1.060186in}{1.207787in}}{\pgfqpoint{1.066010in}{1.213611in}}%
\pgfpathcurveto{\pgfqpoint{1.071834in}{1.219435in}}{\pgfqpoint{1.075106in}{1.227335in}}{\pgfqpoint{1.075106in}{1.235571in}}%
\pgfpathcurveto{\pgfqpoint{1.075106in}{1.243807in}}{\pgfqpoint{1.071834in}{1.251707in}}{\pgfqpoint{1.066010in}{1.257531in}}%
\pgfpathcurveto{\pgfqpoint{1.060186in}{1.263355in}}{\pgfqpoint{1.052286in}{1.266627in}}{\pgfqpoint{1.044050in}{1.266627in}}%
\pgfpathcurveto{\pgfqpoint{1.035813in}{1.266627in}}{\pgfqpoint{1.027913in}{1.263355in}}{\pgfqpoint{1.022089in}{1.257531in}}%
\pgfpathcurveto{\pgfqpoint{1.016265in}{1.251707in}}{\pgfqpoint{1.012993in}{1.243807in}}{\pgfqpoint{1.012993in}{1.235571in}}%
\pgfpathcurveto{\pgfqpoint{1.012993in}{1.227335in}}{\pgfqpoint{1.016265in}{1.219435in}}{\pgfqpoint{1.022089in}{1.213611in}}%
\pgfpathcurveto{\pgfqpoint{1.027913in}{1.207787in}}{\pgfqpoint{1.035813in}{1.204514in}}{\pgfqpoint{1.044050in}{1.204514in}}%
\pgfpathclose%
\pgfusepath{stroke,fill}%
\end{pgfscope}%
\begin{pgfscope}%
\pgfpathrectangle{\pgfqpoint{0.100000in}{0.212622in}}{\pgfqpoint{3.696000in}{3.696000in}}%
\pgfusepath{clip}%
\pgfsetbuttcap%
\pgfsetroundjoin%
\definecolor{currentfill}{rgb}{0.121569,0.466667,0.705882}%
\pgfsetfillcolor{currentfill}%
\pgfsetfillopacity{0.695081}%
\pgfsetlinewidth{1.003750pt}%
\definecolor{currentstroke}{rgb}{0.121569,0.466667,0.705882}%
\pgfsetstrokecolor{currentstroke}%
\pgfsetstrokeopacity{0.695081}%
\pgfsetdash{}{0pt}%
\pgfpathmoveto{\pgfqpoint{1.047909in}{1.204498in}}%
\pgfpathcurveto{\pgfqpoint{1.056145in}{1.204498in}}{\pgfqpoint{1.064045in}{1.207771in}}{\pgfqpoint{1.069869in}{1.213595in}}%
\pgfpathcurveto{\pgfqpoint{1.075693in}{1.219419in}}{\pgfqpoint{1.078965in}{1.227319in}}{\pgfqpoint{1.078965in}{1.235555in}}%
\pgfpathcurveto{\pgfqpoint{1.078965in}{1.243791in}}{\pgfqpoint{1.075693in}{1.251691in}}{\pgfqpoint{1.069869in}{1.257515in}}%
\pgfpathcurveto{\pgfqpoint{1.064045in}{1.263339in}}{\pgfqpoint{1.056145in}{1.266611in}}{\pgfqpoint{1.047909in}{1.266611in}}%
\pgfpathcurveto{\pgfqpoint{1.039673in}{1.266611in}}{\pgfqpoint{1.031773in}{1.263339in}}{\pgfqpoint{1.025949in}{1.257515in}}%
\pgfpathcurveto{\pgfqpoint{1.020125in}{1.251691in}}{\pgfqpoint{1.016852in}{1.243791in}}{\pgfqpoint{1.016852in}{1.235555in}}%
\pgfpathcurveto{\pgfqpoint{1.016852in}{1.227319in}}{\pgfqpoint{1.020125in}{1.219419in}}{\pgfqpoint{1.025949in}{1.213595in}}%
\pgfpathcurveto{\pgfqpoint{1.031773in}{1.207771in}}{\pgfqpoint{1.039673in}{1.204498in}}{\pgfqpoint{1.047909in}{1.204498in}}%
\pgfpathclose%
\pgfusepath{stroke,fill}%
\end{pgfscope}%
\begin{pgfscope}%
\pgfpathrectangle{\pgfqpoint{0.100000in}{0.212622in}}{\pgfqpoint{3.696000in}{3.696000in}}%
\pgfusepath{clip}%
\pgfsetbuttcap%
\pgfsetroundjoin%
\definecolor{currentfill}{rgb}{0.121569,0.466667,0.705882}%
\pgfsetfillcolor{currentfill}%
\pgfsetfillopacity{0.696694}%
\pgfsetlinewidth{1.003750pt}%
\definecolor{currentstroke}{rgb}{0.121569,0.466667,0.705882}%
\pgfsetstrokecolor{currentstroke}%
\pgfsetstrokeopacity{0.696694}%
\pgfsetdash{}{0pt}%
\pgfpathmoveto{\pgfqpoint{1.054902in}{1.205069in}}%
\pgfpathcurveto{\pgfqpoint{1.063139in}{1.205069in}}{\pgfqpoint{1.071039in}{1.208341in}}{\pgfqpoint{1.076863in}{1.214165in}}%
\pgfpathcurveto{\pgfqpoint{1.082687in}{1.219989in}}{\pgfqpoint{1.085959in}{1.227889in}}{\pgfqpoint{1.085959in}{1.236125in}}%
\pgfpathcurveto{\pgfqpoint{1.085959in}{1.244362in}}{\pgfqpoint{1.082687in}{1.252262in}}{\pgfqpoint{1.076863in}{1.258086in}}%
\pgfpathcurveto{\pgfqpoint{1.071039in}{1.263909in}}{\pgfqpoint{1.063139in}{1.267182in}}{\pgfqpoint{1.054902in}{1.267182in}}%
\pgfpathcurveto{\pgfqpoint{1.046666in}{1.267182in}}{\pgfqpoint{1.038766in}{1.263909in}}{\pgfqpoint{1.032942in}{1.258086in}}%
\pgfpathcurveto{\pgfqpoint{1.027118in}{1.252262in}}{\pgfqpoint{1.023846in}{1.244362in}}{\pgfqpoint{1.023846in}{1.236125in}}%
\pgfpathcurveto{\pgfqpoint{1.023846in}{1.227889in}}{\pgfqpoint{1.027118in}{1.219989in}}{\pgfqpoint{1.032942in}{1.214165in}}%
\pgfpathcurveto{\pgfqpoint{1.038766in}{1.208341in}}{\pgfqpoint{1.046666in}{1.205069in}}{\pgfqpoint{1.054902in}{1.205069in}}%
\pgfpathclose%
\pgfusepath{stroke,fill}%
\end{pgfscope}%
\begin{pgfscope}%
\pgfpathrectangle{\pgfqpoint{0.100000in}{0.212622in}}{\pgfqpoint{3.696000in}{3.696000in}}%
\pgfusepath{clip}%
\pgfsetbuttcap%
\pgfsetroundjoin%
\definecolor{currentfill}{rgb}{0.121569,0.466667,0.705882}%
\pgfsetfillcolor{currentfill}%
\pgfsetfillopacity{0.696812}%
\pgfsetlinewidth{1.003750pt}%
\definecolor{currentstroke}{rgb}{0.121569,0.466667,0.705882}%
\pgfsetstrokecolor{currentstroke}%
\pgfsetstrokeopacity{0.696812}%
\pgfsetdash{}{0pt}%
\pgfpathmoveto{\pgfqpoint{2.175034in}{1.792341in}}%
\pgfpathcurveto{\pgfqpoint{2.183270in}{1.792341in}}{\pgfqpoint{2.191170in}{1.795614in}}{\pgfqpoint{2.196994in}{1.801438in}}%
\pgfpathcurveto{\pgfqpoint{2.202818in}{1.807261in}}{\pgfqpoint{2.206090in}{1.815161in}}{\pgfqpoint{2.206090in}{1.823398in}}%
\pgfpathcurveto{\pgfqpoint{2.206090in}{1.831634in}}{\pgfqpoint{2.202818in}{1.839534in}}{\pgfqpoint{2.196994in}{1.845358in}}%
\pgfpathcurveto{\pgfqpoint{2.191170in}{1.851182in}}{\pgfqpoint{2.183270in}{1.854454in}}{\pgfqpoint{2.175034in}{1.854454in}}%
\pgfpathcurveto{\pgfqpoint{2.166797in}{1.854454in}}{\pgfqpoint{2.158897in}{1.851182in}}{\pgfqpoint{2.153073in}{1.845358in}}%
\pgfpathcurveto{\pgfqpoint{2.147249in}{1.839534in}}{\pgfqpoint{2.143977in}{1.831634in}}{\pgfqpoint{2.143977in}{1.823398in}}%
\pgfpathcurveto{\pgfqpoint{2.143977in}{1.815161in}}{\pgfqpoint{2.147249in}{1.807261in}}{\pgfqpoint{2.153073in}{1.801438in}}%
\pgfpathcurveto{\pgfqpoint{2.158897in}{1.795614in}}{\pgfqpoint{2.166797in}{1.792341in}}{\pgfqpoint{2.175034in}{1.792341in}}%
\pgfpathclose%
\pgfusepath{stroke,fill}%
\end{pgfscope}%
\begin{pgfscope}%
\pgfpathrectangle{\pgfqpoint{0.100000in}{0.212622in}}{\pgfqpoint{3.696000in}{3.696000in}}%
\pgfusepath{clip}%
\pgfsetbuttcap%
\pgfsetroundjoin%
\definecolor{currentfill}{rgb}{0.121569,0.466667,0.705882}%
\pgfsetfillcolor{currentfill}%
\pgfsetfillopacity{0.697968}%
\pgfsetlinewidth{1.003750pt}%
\definecolor{currentstroke}{rgb}{0.121569,0.466667,0.705882}%
\pgfsetstrokecolor{currentstroke}%
\pgfsetstrokeopacity{0.697968}%
\pgfsetdash{}{0pt}%
\pgfpathmoveto{\pgfqpoint{1.060464in}{1.205058in}}%
\pgfpathcurveto{\pgfqpoint{1.068700in}{1.205058in}}{\pgfqpoint{1.076600in}{1.208330in}}{\pgfqpoint{1.082424in}{1.214154in}}%
\pgfpathcurveto{\pgfqpoint{1.088248in}{1.219978in}}{\pgfqpoint{1.091520in}{1.227878in}}{\pgfqpoint{1.091520in}{1.236114in}}%
\pgfpathcurveto{\pgfqpoint{1.091520in}{1.244351in}}{\pgfqpoint{1.088248in}{1.252251in}}{\pgfqpoint{1.082424in}{1.258075in}}%
\pgfpathcurveto{\pgfqpoint{1.076600in}{1.263899in}}{\pgfqpoint{1.068700in}{1.267171in}}{\pgfqpoint{1.060464in}{1.267171in}}%
\pgfpathcurveto{\pgfqpoint{1.052228in}{1.267171in}}{\pgfqpoint{1.044328in}{1.263899in}}{\pgfqpoint{1.038504in}{1.258075in}}%
\pgfpathcurveto{\pgfqpoint{1.032680in}{1.252251in}}{\pgfqpoint{1.029407in}{1.244351in}}{\pgfqpoint{1.029407in}{1.236114in}}%
\pgfpathcurveto{\pgfqpoint{1.029407in}{1.227878in}}{\pgfqpoint{1.032680in}{1.219978in}}{\pgfqpoint{1.038504in}{1.214154in}}%
\pgfpathcurveto{\pgfqpoint{1.044328in}{1.208330in}}{\pgfqpoint{1.052228in}{1.205058in}}{\pgfqpoint{1.060464in}{1.205058in}}%
\pgfpathclose%
\pgfusepath{stroke,fill}%
\end{pgfscope}%
\begin{pgfscope}%
\pgfpathrectangle{\pgfqpoint{0.100000in}{0.212622in}}{\pgfqpoint{3.696000in}{3.696000in}}%
\pgfusepath{clip}%
\pgfsetbuttcap%
\pgfsetroundjoin%
\definecolor{currentfill}{rgb}{0.121569,0.466667,0.705882}%
\pgfsetfillcolor{currentfill}%
\pgfsetfillopacity{0.699031}%
\pgfsetlinewidth{1.003750pt}%
\definecolor{currentstroke}{rgb}{0.121569,0.466667,0.705882}%
\pgfsetstrokecolor{currentstroke}%
\pgfsetstrokeopacity{0.699031}%
\pgfsetdash{}{0pt}%
\pgfpathmoveto{\pgfqpoint{1.065510in}{1.204993in}}%
\pgfpathcurveto{\pgfqpoint{1.073746in}{1.204993in}}{\pgfqpoint{1.081646in}{1.208265in}}{\pgfqpoint{1.087470in}{1.214089in}}%
\pgfpathcurveto{\pgfqpoint{1.093294in}{1.219913in}}{\pgfqpoint{1.096566in}{1.227813in}}{\pgfqpoint{1.096566in}{1.236049in}}%
\pgfpathcurveto{\pgfqpoint{1.096566in}{1.244286in}}{\pgfqpoint{1.093294in}{1.252186in}}{\pgfqpoint{1.087470in}{1.258010in}}%
\pgfpathcurveto{\pgfqpoint{1.081646in}{1.263834in}}{\pgfqpoint{1.073746in}{1.267106in}}{\pgfqpoint{1.065510in}{1.267106in}}%
\pgfpathcurveto{\pgfqpoint{1.057274in}{1.267106in}}{\pgfqpoint{1.049374in}{1.263834in}}{\pgfqpoint{1.043550in}{1.258010in}}%
\pgfpathcurveto{\pgfqpoint{1.037726in}{1.252186in}}{\pgfqpoint{1.034453in}{1.244286in}}{\pgfqpoint{1.034453in}{1.236049in}}%
\pgfpathcurveto{\pgfqpoint{1.034453in}{1.227813in}}{\pgfqpoint{1.037726in}{1.219913in}}{\pgfqpoint{1.043550in}{1.214089in}}%
\pgfpathcurveto{\pgfqpoint{1.049374in}{1.208265in}}{\pgfqpoint{1.057274in}{1.204993in}}{\pgfqpoint{1.065510in}{1.204993in}}%
\pgfpathclose%
\pgfusepath{stroke,fill}%
\end{pgfscope}%
\begin{pgfscope}%
\pgfpathrectangle{\pgfqpoint{0.100000in}{0.212622in}}{\pgfqpoint{3.696000in}{3.696000in}}%
\pgfusepath{clip}%
\pgfsetbuttcap%
\pgfsetroundjoin%
\definecolor{currentfill}{rgb}{0.121569,0.466667,0.705882}%
\pgfsetfillcolor{currentfill}%
\pgfsetfillopacity{0.699825}%
\pgfsetlinewidth{1.003750pt}%
\definecolor{currentstroke}{rgb}{0.121569,0.466667,0.705882}%
\pgfsetstrokecolor{currentstroke}%
\pgfsetstrokeopacity{0.699825}%
\pgfsetdash{}{0pt}%
\pgfpathmoveto{\pgfqpoint{1.069551in}{1.204762in}}%
\pgfpathcurveto{\pgfqpoint{1.077788in}{1.204762in}}{\pgfqpoint{1.085688in}{1.208034in}}{\pgfqpoint{1.091512in}{1.213858in}}%
\pgfpathcurveto{\pgfqpoint{1.097336in}{1.219682in}}{\pgfqpoint{1.100608in}{1.227582in}}{\pgfqpoint{1.100608in}{1.235818in}}%
\pgfpathcurveto{\pgfqpoint{1.100608in}{1.244054in}}{\pgfqpoint{1.097336in}{1.251955in}}{\pgfqpoint{1.091512in}{1.257778in}}%
\pgfpathcurveto{\pgfqpoint{1.085688in}{1.263602in}}{\pgfqpoint{1.077788in}{1.266875in}}{\pgfqpoint{1.069551in}{1.266875in}}%
\pgfpathcurveto{\pgfqpoint{1.061315in}{1.266875in}}{\pgfqpoint{1.053415in}{1.263602in}}{\pgfqpoint{1.047591in}{1.257778in}}%
\pgfpathcurveto{\pgfqpoint{1.041767in}{1.251955in}}{\pgfqpoint{1.038495in}{1.244054in}}{\pgfqpoint{1.038495in}{1.235818in}}%
\pgfpathcurveto{\pgfqpoint{1.038495in}{1.227582in}}{\pgfqpoint{1.041767in}{1.219682in}}{\pgfqpoint{1.047591in}{1.213858in}}%
\pgfpathcurveto{\pgfqpoint{1.053415in}{1.208034in}}{\pgfqpoint{1.061315in}{1.204762in}}{\pgfqpoint{1.069551in}{1.204762in}}%
\pgfpathclose%
\pgfusepath{stroke,fill}%
\end{pgfscope}%
\begin{pgfscope}%
\pgfpathrectangle{\pgfqpoint{0.100000in}{0.212622in}}{\pgfqpoint{3.696000in}{3.696000in}}%
\pgfusepath{clip}%
\pgfsetbuttcap%
\pgfsetroundjoin%
\definecolor{currentfill}{rgb}{0.121569,0.466667,0.705882}%
\pgfsetfillcolor{currentfill}%
\pgfsetfillopacity{0.700495}%
\pgfsetlinewidth{1.003750pt}%
\definecolor{currentstroke}{rgb}{0.121569,0.466667,0.705882}%
\pgfsetstrokecolor{currentstroke}%
\pgfsetstrokeopacity{0.700495}%
\pgfsetdash{}{0pt}%
\pgfpathmoveto{\pgfqpoint{1.072593in}{1.204503in}}%
\pgfpathcurveto{\pgfqpoint{1.080829in}{1.204503in}}{\pgfqpoint{1.088729in}{1.207775in}}{\pgfqpoint{1.094553in}{1.213599in}}%
\pgfpathcurveto{\pgfqpoint{1.100377in}{1.219423in}}{\pgfqpoint{1.103649in}{1.227323in}}{\pgfqpoint{1.103649in}{1.235559in}}%
\pgfpathcurveto{\pgfqpoint{1.103649in}{1.243796in}}{\pgfqpoint{1.100377in}{1.251696in}}{\pgfqpoint{1.094553in}{1.257520in}}%
\pgfpathcurveto{\pgfqpoint{1.088729in}{1.263344in}}{\pgfqpoint{1.080829in}{1.266616in}}{\pgfqpoint{1.072593in}{1.266616in}}%
\pgfpathcurveto{\pgfqpoint{1.064356in}{1.266616in}}{\pgfqpoint{1.056456in}{1.263344in}}{\pgfqpoint{1.050632in}{1.257520in}}%
\pgfpathcurveto{\pgfqpoint{1.044808in}{1.251696in}}{\pgfqpoint{1.041536in}{1.243796in}}{\pgfqpoint{1.041536in}{1.235559in}}%
\pgfpathcurveto{\pgfqpoint{1.041536in}{1.227323in}}{\pgfqpoint{1.044808in}{1.219423in}}{\pgfqpoint{1.050632in}{1.213599in}}%
\pgfpathcurveto{\pgfqpoint{1.056456in}{1.207775in}}{\pgfqpoint{1.064356in}{1.204503in}}{\pgfqpoint{1.072593in}{1.204503in}}%
\pgfpathclose%
\pgfusepath{stroke,fill}%
\end{pgfscope}%
\begin{pgfscope}%
\pgfpathrectangle{\pgfqpoint{0.100000in}{0.212622in}}{\pgfqpoint{3.696000in}{3.696000in}}%
\pgfusepath{clip}%
\pgfsetbuttcap%
\pgfsetroundjoin%
\definecolor{currentfill}{rgb}{0.121569,0.466667,0.705882}%
\pgfsetfillcolor{currentfill}%
\pgfsetfillopacity{0.700842}%
\pgfsetlinewidth{1.003750pt}%
\definecolor{currentstroke}{rgb}{0.121569,0.466667,0.705882}%
\pgfsetstrokecolor{currentstroke}%
\pgfsetstrokeopacity{0.700842}%
\pgfsetdash{}{0pt}%
\pgfpathmoveto{\pgfqpoint{2.178990in}{1.778971in}}%
\pgfpathcurveto{\pgfqpoint{2.187226in}{1.778971in}}{\pgfqpoint{2.195126in}{1.782244in}}{\pgfqpoint{2.200950in}{1.788068in}}%
\pgfpathcurveto{\pgfqpoint{2.206774in}{1.793892in}}{\pgfqpoint{2.210047in}{1.801792in}}{\pgfqpoint{2.210047in}{1.810028in}}%
\pgfpathcurveto{\pgfqpoint{2.210047in}{1.818264in}}{\pgfqpoint{2.206774in}{1.826164in}}{\pgfqpoint{2.200950in}{1.831988in}}%
\pgfpathcurveto{\pgfqpoint{2.195126in}{1.837812in}}{\pgfqpoint{2.187226in}{1.841084in}}{\pgfqpoint{2.178990in}{1.841084in}}%
\pgfpathcurveto{\pgfqpoint{2.170754in}{1.841084in}}{\pgfqpoint{2.162854in}{1.837812in}}{\pgfqpoint{2.157030in}{1.831988in}}%
\pgfpathcurveto{\pgfqpoint{2.151206in}{1.826164in}}{\pgfqpoint{2.147934in}{1.818264in}}{\pgfqpoint{2.147934in}{1.810028in}}%
\pgfpathcurveto{\pgfqpoint{2.147934in}{1.801792in}}{\pgfqpoint{2.151206in}{1.793892in}}{\pgfqpoint{2.157030in}{1.788068in}}%
\pgfpathcurveto{\pgfqpoint{2.162854in}{1.782244in}}{\pgfqpoint{2.170754in}{1.778971in}}{\pgfqpoint{2.178990in}{1.778971in}}%
\pgfpathclose%
\pgfusepath{stroke,fill}%
\end{pgfscope}%
\begin{pgfscope}%
\pgfpathrectangle{\pgfqpoint{0.100000in}{0.212622in}}{\pgfqpoint{3.696000in}{3.696000in}}%
\pgfusepath{clip}%
\pgfsetbuttcap%
\pgfsetroundjoin%
\definecolor{currentfill}{rgb}{0.121569,0.466667,0.705882}%
\pgfsetfillcolor{currentfill}%
\pgfsetfillopacity{0.701707}%
\pgfsetlinewidth{1.003750pt}%
\definecolor{currentstroke}{rgb}{0.121569,0.466667,0.705882}%
\pgfsetstrokecolor{currentstroke}%
\pgfsetstrokeopacity{0.701707}%
\pgfsetdash{}{0pt}%
\pgfpathmoveto{\pgfqpoint{1.078139in}{1.204076in}}%
\pgfpathcurveto{\pgfqpoint{1.086375in}{1.204076in}}{\pgfqpoint{1.094275in}{1.207348in}}{\pgfqpoint{1.100099in}{1.213172in}}%
\pgfpathcurveto{\pgfqpoint{1.105923in}{1.218996in}}{\pgfqpoint{1.109195in}{1.226896in}}{\pgfqpoint{1.109195in}{1.235133in}}%
\pgfpathcurveto{\pgfqpoint{1.109195in}{1.243369in}}{\pgfqpoint{1.105923in}{1.251269in}}{\pgfqpoint{1.100099in}{1.257093in}}%
\pgfpathcurveto{\pgfqpoint{1.094275in}{1.262917in}}{\pgfqpoint{1.086375in}{1.266189in}}{\pgfqpoint{1.078139in}{1.266189in}}%
\pgfpathcurveto{\pgfqpoint{1.069902in}{1.266189in}}{\pgfqpoint{1.062002in}{1.262917in}}{\pgfqpoint{1.056178in}{1.257093in}}%
\pgfpathcurveto{\pgfqpoint{1.050355in}{1.251269in}}{\pgfqpoint{1.047082in}{1.243369in}}{\pgfqpoint{1.047082in}{1.235133in}}%
\pgfpathcurveto{\pgfqpoint{1.047082in}{1.226896in}}{\pgfqpoint{1.050355in}{1.218996in}}{\pgfqpoint{1.056178in}{1.213172in}}%
\pgfpathcurveto{\pgfqpoint{1.062002in}{1.207348in}}{\pgfqpoint{1.069902in}{1.204076in}}{\pgfqpoint{1.078139in}{1.204076in}}%
\pgfpathclose%
\pgfusepath{stroke,fill}%
\end{pgfscope}%
\begin{pgfscope}%
\pgfpathrectangle{\pgfqpoint{0.100000in}{0.212622in}}{\pgfqpoint{3.696000in}{3.696000in}}%
\pgfusepath{clip}%
\pgfsetbuttcap%
\pgfsetroundjoin%
\definecolor{currentfill}{rgb}{0.121569,0.466667,0.705882}%
\pgfsetfillcolor{currentfill}%
\pgfsetfillopacity{0.702730}%
\pgfsetlinewidth{1.003750pt}%
\definecolor{currentstroke}{rgb}{0.121569,0.466667,0.705882}%
\pgfsetstrokecolor{currentstroke}%
\pgfsetstrokeopacity{0.702730}%
\pgfsetdash{}{0pt}%
\pgfpathmoveto{\pgfqpoint{1.083003in}{1.203991in}}%
\pgfpathcurveto{\pgfqpoint{1.091239in}{1.203991in}}{\pgfqpoint{1.099139in}{1.207264in}}{\pgfqpoint{1.104963in}{1.213087in}}%
\pgfpathcurveto{\pgfqpoint{1.110787in}{1.218911in}}{\pgfqpoint{1.114060in}{1.226811in}}{\pgfqpoint{1.114060in}{1.235048in}}%
\pgfpathcurveto{\pgfqpoint{1.114060in}{1.243284in}}{\pgfqpoint{1.110787in}{1.251184in}}{\pgfqpoint{1.104963in}{1.257008in}}%
\pgfpathcurveto{\pgfqpoint{1.099139in}{1.262832in}}{\pgfqpoint{1.091239in}{1.266104in}}{\pgfqpoint{1.083003in}{1.266104in}}%
\pgfpathcurveto{\pgfqpoint{1.074767in}{1.266104in}}{\pgfqpoint{1.066867in}{1.262832in}}{\pgfqpoint{1.061043in}{1.257008in}}%
\pgfpathcurveto{\pgfqpoint{1.055219in}{1.251184in}}{\pgfqpoint{1.051947in}{1.243284in}}{\pgfqpoint{1.051947in}{1.235048in}}%
\pgfpathcurveto{\pgfqpoint{1.051947in}{1.226811in}}{\pgfqpoint{1.055219in}{1.218911in}}{\pgfqpoint{1.061043in}{1.213087in}}%
\pgfpathcurveto{\pgfqpoint{1.066867in}{1.207264in}}{\pgfqpoint{1.074767in}{1.203991in}}{\pgfqpoint{1.083003in}{1.203991in}}%
\pgfpathclose%
\pgfusepath{stroke,fill}%
\end{pgfscope}%
\begin{pgfscope}%
\pgfpathrectangle{\pgfqpoint{0.100000in}{0.212622in}}{\pgfqpoint{3.696000in}{3.696000in}}%
\pgfusepath{clip}%
\pgfsetbuttcap%
\pgfsetroundjoin%
\definecolor{currentfill}{rgb}{0.121569,0.466667,0.705882}%
\pgfsetfillcolor{currentfill}%
\pgfsetfillopacity{0.703296}%
\pgfsetlinewidth{1.003750pt}%
\definecolor{currentstroke}{rgb}{0.121569,0.466667,0.705882}%
\pgfsetstrokecolor{currentstroke}%
\pgfsetstrokeopacity{0.703296}%
\pgfsetdash{}{0pt}%
\pgfpathmoveto{\pgfqpoint{2.180519in}{1.771818in}}%
\pgfpathcurveto{\pgfqpoint{2.188756in}{1.771818in}}{\pgfqpoint{2.196656in}{1.775090in}}{\pgfqpoint{2.202480in}{1.780914in}}%
\pgfpathcurveto{\pgfqpoint{2.208304in}{1.786738in}}{\pgfqpoint{2.211576in}{1.794638in}}{\pgfqpoint{2.211576in}{1.802874in}}%
\pgfpathcurveto{\pgfqpoint{2.211576in}{1.811110in}}{\pgfqpoint{2.208304in}{1.819010in}}{\pgfqpoint{2.202480in}{1.824834in}}%
\pgfpathcurveto{\pgfqpoint{2.196656in}{1.830658in}}{\pgfqpoint{2.188756in}{1.833931in}}{\pgfqpoint{2.180519in}{1.833931in}}%
\pgfpathcurveto{\pgfqpoint{2.172283in}{1.833931in}}{\pgfqpoint{2.164383in}{1.830658in}}{\pgfqpoint{2.158559in}{1.824834in}}%
\pgfpathcurveto{\pgfqpoint{2.152735in}{1.819010in}}{\pgfqpoint{2.149463in}{1.811110in}}{\pgfqpoint{2.149463in}{1.802874in}}%
\pgfpathcurveto{\pgfqpoint{2.149463in}{1.794638in}}{\pgfqpoint{2.152735in}{1.786738in}}{\pgfqpoint{2.158559in}{1.780914in}}%
\pgfpathcurveto{\pgfqpoint{2.164383in}{1.775090in}}{\pgfqpoint{2.172283in}{1.771818in}}{\pgfqpoint{2.180519in}{1.771818in}}%
\pgfpathclose%
\pgfusepath{stroke,fill}%
\end{pgfscope}%
\begin{pgfscope}%
\pgfpathrectangle{\pgfqpoint{0.100000in}{0.212622in}}{\pgfqpoint{3.696000in}{3.696000in}}%
\pgfusepath{clip}%
\pgfsetbuttcap%
\pgfsetroundjoin%
\definecolor{currentfill}{rgb}{0.121569,0.466667,0.705882}%
\pgfsetfillcolor{currentfill}%
\pgfsetfillopacity{0.704853}%
\pgfsetlinewidth{1.003750pt}%
\definecolor{currentstroke}{rgb}{0.121569,0.466667,0.705882}%
\pgfsetstrokecolor{currentstroke}%
\pgfsetstrokeopacity{0.704853}%
\pgfsetdash{}{0pt}%
\pgfpathmoveto{\pgfqpoint{1.091706in}{1.204288in}}%
\pgfpathcurveto{\pgfqpoint{1.099942in}{1.204288in}}{\pgfqpoint{1.107842in}{1.207561in}}{\pgfqpoint{1.113666in}{1.213385in}}%
\pgfpathcurveto{\pgfqpoint{1.119490in}{1.219209in}}{\pgfqpoint{1.122763in}{1.227109in}}{\pgfqpoint{1.122763in}{1.235345in}}%
\pgfpathcurveto{\pgfqpoint{1.122763in}{1.243581in}}{\pgfqpoint{1.119490in}{1.251481in}}{\pgfqpoint{1.113666in}{1.257305in}}%
\pgfpathcurveto{\pgfqpoint{1.107842in}{1.263129in}}{\pgfqpoint{1.099942in}{1.266401in}}{\pgfqpoint{1.091706in}{1.266401in}}%
\pgfpathcurveto{\pgfqpoint{1.083470in}{1.266401in}}{\pgfqpoint{1.075570in}{1.263129in}}{\pgfqpoint{1.069746in}{1.257305in}}%
\pgfpathcurveto{\pgfqpoint{1.063922in}{1.251481in}}{\pgfqpoint{1.060650in}{1.243581in}}{\pgfqpoint{1.060650in}{1.235345in}}%
\pgfpathcurveto{\pgfqpoint{1.060650in}{1.227109in}}{\pgfqpoint{1.063922in}{1.219209in}}{\pgfqpoint{1.069746in}{1.213385in}}%
\pgfpathcurveto{\pgfqpoint{1.075570in}{1.207561in}}{\pgfqpoint{1.083470in}{1.204288in}}{\pgfqpoint{1.091706in}{1.204288in}}%
\pgfpathclose%
\pgfusepath{stroke,fill}%
\end{pgfscope}%
\begin{pgfscope}%
\pgfpathrectangle{\pgfqpoint{0.100000in}{0.212622in}}{\pgfqpoint{3.696000in}{3.696000in}}%
\pgfusepath{clip}%
\pgfsetbuttcap%
\pgfsetroundjoin%
\definecolor{currentfill}{rgb}{0.121569,0.466667,0.705882}%
\pgfsetfillcolor{currentfill}%
\pgfsetfillopacity{0.705885}%
\pgfsetlinewidth{1.003750pt}%
\definecolor{currentstroke}{rgb}{0.121569,0.466667,0.705882}%
\pgfsetstrokecolor{currentstroke}%
\pgfsetstrokeopacity{0.705885}%
\pgfsetdash{}{0pt}%
\pgfpathmoveto{\pgfqpoint{2.182433in}{1.764053in}}%
\pgfpathcurveto{\pgfqpoint{2.190670in}{1.764053in}}{\pgfqpoint{2.198570in}{1.767326in}}{\pgfqpoint{2.204394in}{1.773149in}}%
\pgfpathcurveto{\pgfqpoint{2.210218in}{1.778973in}}{\pgfqpoint{2.213490in}{1.786873in}}{\pgfqpoint{2.213490in}{1.795110in}}%
\pgfpathcurveto{\pgfqpoint{2.213490in}{1.803346in}}{\pgfqpoint{2.210218in}{1.811246in}}{\pgfqpoint{2.204394in}{1.817070in}}%
\pgfpathcurveto{\pgfqpoint{2.198570in}{1.822894in}}{\pgfqpoint{2.190670in}{1.826166in}}{\pgfqpoint{2.182433in}{1.826166in}}%
\pgfpathcurveto{\pgfqpoint{2.174197in}{1.826166in}}{\pgfqpoint{2.166297in}{1.822894in}}{\pgfqpoint{2.160473in}{1.817070in}}%
\pgfpathcurveto{\pgfqpoint{2.154649in}{1.811246in}}{\pgfqpoint{2.151377in}{1.803346in}}{\pgfqpoint{2.151377in}{1.795110in}}%
\pgfpathcurveto{\pgfqpoint{2.151377in}{1.786873in}}{\pgfqpoint{2.154649in}{1.778973in}}{\pgfqpoint{2.160473in}{1.773149in}}%
\pgfpathcurveto{\pgfqpoint{2.166297in}{1.767326in}}{\pgfqpoint{2.174197in}{1.764053in}}{\pgfqpoint{2.182433in}{1.764053in}}%
\pgfpathclose%
\pgfusepath{stroke,fill}%
\end{pgfscope}%
\begin{pgfscope}%
\pgfpathrectangle{\pgfqpoint{0.100000in}{0.212622in}}{\pgfqpoint{3.696000in}{3.696000in}}%
\pgfusepath{clip}%
\pgfsetbuttcap%
\pgfsetroundjoin%
\definecolor{currentfill}{rgb}{0.121569,0.466667,0.705882}%
\pgfsetfillcolor{currentfill}%
\pgfsetfillopacity{0.706595}%
\pgfsetlinewidth{1.003750pt}%
\definecolor{currentstroke}{rgb}{0.121569,0.466667,0.705882}%
\pgfsetstrokecolor{currentstroke}%
\pgfsetstrokeopacity{0.706595}%
\pgfsetdash{}{0pt}%
\pgfpathmoveto{\pgfqpoint{1.098991in}{1.204049in}}%
\pgfpathcurveto{\pgfqpoint{1.107227in}{1.204049in}}{\pgfqpoint{1.115127in}{1.207322in}}{\pgfqpoint{1.120951in}{1.213146in}}%
\pgfpathcurveto{\pgfqpoint{1.126775in}{1.218970in}}{\pgfqpoint{1.130047in}{1.226870in}}{\pgfqpoint{1.130047in}{1.235106in}}%
\pgfpathcurveto{\pgfqpoint{1.130047in}{1.243342in}}{\pgfqpoint{1.126775in}{1.251242in}}{\pgfqpoint{1.120951in}{1.257066in}}%
\pgfpathcurveto{\pgfqpoint{1.115127in}{1.262890in}}{\pgfqpoint{1.107227in}{1.266162in}}{\pgfqpoint{1.098991in}{1.266162in}}%
\pgfpathcurveto{\pgfqpoint{1.090755in}{1.266162in}}{\pgfqpoint{1.082855in}{1.262890in}}{\pgfqpoint{1.077031in}{1.257066in}}%
\pgfpathcurveto{\pgfqpoint{1.071207in}{1.251242in}}{\pgfqpoint{1.067934in}{1.243342in}}{\pgfqpoint{1.067934in}{1.235106in}}%
\pgfpathcurveto{\pgfqpoint{1.067934in}{1.226870in}}{\pgfqpoint{1.071207in}{1.218970in}}{\pgfqpoint{1.077031in}{1.213146in}}%
\pgfpathcurveto{\pgfqpoint{1.082855in}{1.207322in}}{\pgfqpoint{1.090755in}{1.204049in}}{\pgfqpoint{1.098991in}{1.204049in}}%
\pgfpathclose%
\pgfusepath{stroke,fill}%
\end{pgfscope}%
\begin{pgfscope}%
\pgfpathrectangle{\pgfqpoint{0.100000in}{0.212622in}}{\pgfqpoint{3.696000in}{3.696000in}}%
\pgfusepath{clip}%
\pgfsetbuttcap%
\pgfsetroundjoin%
\definecolor{currentfill}{rgb}{0.121569,0.466667,0.705882}%
\pgfsetfillcolor{currentfill}%
\pgfsetfillopacity{0.708125}%
\pgfsetlinewidth{1.003750pt}%
\definecolor{currentstroke}{rgb}{0.121569,0.466667,0.705882}%
\pgfsetstrokecolor{currentstroke}%
\pgfsetstrokeopacity{0.708125}%
\pgfsetdash{}{0pt}%
\pgfpathmoveto{\pgfqpoint{2.184752in}{1.754713in}}%
\pgfpathcurveto{\pgfqpoint{2.192988in}{1.754713in}}{\pgfqpoint{2.200888in}{1.757985in}}{\pgfqpoint{2.206712in}{1.763809in}}%
\pgfpathcurveto{\pgfqpoint{2.212536in}{1.769633in}}{\pgfqpoint{2.215809in}{1.777533in}}{\pgfqpoint{2.215809in}{1.785769in}}%
\pgfpathcurveto{\pgfqpoint{2.215809in}{1.794005in}}{\pgfqpoint{2.212536in}{1.801906in}}{\pgfqpoint{2.206712in}{1.807729in}}%
\pgfpathcurveto{\pgfqpoint{2.200888in}{1.813553in}}{\pgfqpoint{2.192988in}{1.816826in}}{\pgfqpoint{2.184752in}{1.816826in}}%
\pgfpathcurveto{\pgfqpoint{2.176516in}{1.816826in}}{\pgfqpoint{2.168616in}{1.813553in}}{\pgfqpoint{2.162792in}{1.807729in}}%
\pgfpathcurveto{\pgfqpoint{2.156968in}{1.801906in}}{\pgfqpoint{2.153696in}{1.794005in}}{\pgfqpoint{2.153696in}{1.785769in}}%
\pgfpathcurveto{\pgfqpoint{2.153696in}{1.777533in}}{\pgfqpoint{2.156968in}{1.769633in}}{\pgfqpoint{2.162792in}{1.763809in}}%
\pgfpathcurveto{\pgfqpoint{2.168616in}{1.757985in}}{\pgfqpoint{2.176516in}{1.754713in}}{\pgfqpoint{2.184752in}{1.754713in}}%
\pgfpathclose%
\pgfusepath{stroke,fill}%
\end{pgfscope}%
\begin{pgfscope}%
\pgfpathrectangle{\pgfqpoint{0.100000in}{0.212622in}}{\pgfqpoint{3.696000in}{3.696000in}}%
\pgfusepath{clip}%
\pgfsetbuttcap%
\pgfsetroundjoin%
\definecolor{currentfill}{rgb}{0.121569,0.466667,0.705882}%
\pgfsetfillcolor{currentfill}%
\pgfsetfillopacity{0.708133}%
\pgfsetlinewidth{1.003750pt}%
\definecolor{currentstroke}{rgb}{0.121569,0.466667,0.705882}%
\pgfsetstrokecolor{currentstroke}%
\pgfsetstrokeopacity{0.708133}%
\pgfsetdash{}{0pt}%
\pgfpathmoveto{\pgfqpoint{1.106009in}{1.203757in}}%
\pgfpathcurveto{\pgfqpoint{1.114246in}{1.203757in}}{\pgfqpoint{1.122146in}{1.207029in}}{\pgfqpoint{1.127970in}{1.212853in}}%
\pgfpathcurveto{\pgfqpoint{1.133794in}{1.218677in}}{\pgfqpoint{1.137066in}{1.226577in}}{\pgfqpoint{1.137066in}{1.234813in}}%
\pgfpathcurveto{\pgfqpoint{1.137066in}{1.243049in}}{\pgfqpoint{1.133794in}{1.250949in}}{\pgfqpoint{1.127970in}{1.256773in}}%
\pgfpathcurveto{\pgfqpoint{1.122146in}{1.262597in}}{\pgfqpoint{1.114246in}{1.265870in}}{\pgfqpoint{1.106009in}{1.265870in}}%
\pgfpathcurveto{\pgfqpoint{1.097773in}{1.265870in}}{\pgfqpoint{1.089873in}{1.262597in}}{\pgfqpoint{1.084049in}{1.256773in}}%
\pgfpathcurveto{\pgfqpoint{1.078225in}{1.250949in}}{\pgfqpoint{1.074953in}{1.243049in}}{\pgfqpoint{1.074953in}{1.234813in}}%
\pgfpathcurveto{\pgfqpoint{1.074953in}{1.226577in}}{\pgfqpoint{1.078225in}{1.218677in}}{\pgfqpoint{1.084049in}{1.212853in}}%
\pgfpathcurveto{\pgfqpoint{1.089873in}{1.207029in}}{\pgfqpoint{1.097773in}{1.203757in}}{\pgfqpoint{1.106009in}{1.203757in}}%
\pgfpathclose%
\pgfusepath{stroke,fill}%
\end{pgfscope}%
\begin{pgfscope}%
\pgfpathrectangle{\pgfqpoint{0.100000in}{0.212622in}}{\pgfqpoint{3.696000in}{3.696000in}}%
\pgfusepath{clip}%
\pgfsetbuttcap%
\pgfsetroundjoin%
\definecolor{currentfill}{rgb}{0.121569,0.466667,0.705882}%
\pgfsetfillcolor{currentfill}%
\pgfsetfillopacity{0.709384}%
\pgfsetlinewidth{1.003750pt}%
\definecolor{currentstroke}{rgb}{0.121569,0.466667,0.705882}%
\pgfsetstrokecolor{currentstroke}%
\pgfsetstrokeopacity{0.709384}%
\pgfsetdash{}{0pt}%
\pgfpathmoveto{\pgfqpoint{1.112290in}{1.203272in}}%
\pgfpathcurveto{\pgfqpoint{1.120526in}{1.203272in}}{\pgfqpoint{1.128426in}{1.206544in}}{\pgfqpoint{1.134250in}{1.212368in}}%
\pgfpathcurveto{\pgfqpoint{1.140074in}{1.218192in}}{\pgfqpoint{1.143346in}{1.226092in}}{\pgfqpoint{1.143346in}{1.234328in}}%
\pgfpathcurveto{\pgfqpoint{1.143346in}{1.242565in}}{\pgfqpoint{1.140074in}{1.250465in}}{\pgfqpoint{1.134250in}{1.256289in}}%
\pgfpathcurveto{\pgfqpoint{1.128426in}{1.262112in}}{\pgfqpoint{1.120526in}{1.265385in}}{\pgfqpoint{1.112290in}{1.265385in}}%
\pgfpathcurveto{\pgfqpoint{1.104054in}{1.265385in}}{\pgfqpoint{1.096153in}{1.262112in}}{\pgfqpoint{1.090330in}{1.256289in}}%
\pgfpathcurveto{\pgfqpoint{1.084506in}{1.250465in}}{\pgfqpoint{1.081233in}{1.242565in}}{\pgfqpoint{1.081233in}{1.234328in}}%
\pgfpathcurveto{\pgfqpoint{1.081233in}{1.226092in}}{\pgfqpoint{1.084506in}{1.218192in}}{\pgfqpoint{1.090330in}{1.212368in}}%
\pgfpathcurveto{\pgfqpoint{1.096153in}{1.206544in}}{\pgfqpoint{1.104054in}{1.203272in}}{\pgfqpoint{1.112290in}{1.203272in}}%
\pgfpathclose%
\pgfusepath{stroke,fill}%
\end{pgfscope}%
\begin{pgfscope}%
\pgfpathrectangle{\pgfqpoint{0.100000in}{0.212622in}}{\pgfqpoint{3.696000in}{3.696000in}}%
\pgfusepath{clip}%
\pgfsetbuttcap%
\pgfsetroundjoin%
\definecolor{currentfill}{rgb}{0.121569,0.466667,0.705882}%
\pgfsetfillcolor{currentfill}%
\pgfsetfillopacity{0.710377}%
\pgfsetlinewidth{1.003750pt}%
\definecolor{currentstroke}{rgb}{0.121569,0.466667,0.705882}%
\pgfsetstrokecolor{currentstroke}%
\pgfsetstrokeopacity{0.710377}%
\pgfsetdash{}{0pt}%
\pgfpathmoveto{\pgfqpoint{1.117003in}{1.202598in}}%
\pgfpathcurveto{\pgfqpoint{1.125239in}{1.202598in}}{\pgfqpoint{1.133139in}{1.205870in}}{\pgfqpoint{1.138963in}{1.211694in}}%
\pgfpathcurveto{\pgfqpoint{1.144787in}{1.217518in}}{\pgfqpoint{1.148059in}{1.225418in}}{\pgfqpoint{1.148059in}{1.233654in}}%
\pgfpathcurveto{\pgfqpoint{1.148059in}{1.241891in}}{\pgfqpoint{1.144787in}{1.249791in}}{\pgfqpoint{1.138963in}{1.255615in}}%
\pgfpathcurveto{\pgfqpoint{1.133139in}{1.261439in}}{\pgfqpoint{1.125239in}{1.264711in}}{\pgfqpoint{1.117003in}{1.264711in}}%
\pgfpathcurveto{\pgfqpoint{1.108767in}{1.264711in}}{\pgfqpoint{1.100867in}{1.261439in}}{\pgfqpoint{1.095043in}{1.255615in}}%
\pgfpathcurveto{\pgfqpoint{1.089219in}{1.249791in}}{\pgfqpoint{1.085946in}{1.241891in}}{\pgfqpoint{1.085946in}{1.233654in}}%
\pgfpathcurveto{\pgfqpoint{1.085946in}{1.225418in}}{\pgfqpoint{1.089219in}{1.217518in}}{\pgfqpoint{1.095043in}{1.211694in}}%
\pgfpathcurveto{\pgfqpoint{1.100867in}{1.205870in}}{\pgfqpoint{1.108767in}{1.202598in}}{\pgfqpoint{1.117003in}{1.202598in}}%
\pgfpathclose%
\pgfusepath{stroke,fill}%
\end{pgfscope}%
\begin{pgfscope}%
\pgfpathrectangle{\pgfqpoint{0.100000in}{0.212622in}}{\pgfqpoint{3.696000in}{3.696000in}}%
\pgfusepath{clip}%
\pgfsetbuttcap%
\pgfsetroundjoin%
\definecolor{currentfill}{rgb}{0.121569,0.466667,0.705882}%
\pgfsetfillcolor{currentfill}%
\pgfsetfillopacity{0.711015}%
\pgfsetlinewidth{1.003750pt}%
\definecolor{currentstroke}{rgb}{0.121569,0.466667,0.705882}%
\pgfsetstrokecolor{currentstroke}%
\pgfsetstrokeopacity{0.711015}%
\pgfsetdash{}{0pt}%
\pgfpathmoveto{\pgfqpoint{2.186685in}{1.744333in}}%
\pgfpathcurveto{\pgfqpoint{2.194921in}{1.744333in}}{\pgfqpoint{2.202821in}{1.747606in}}{\pgfqpoint{2.208645in}{1.753429in}}%
\pgfpathcurveto{\pgfqpoint{2.214469in}{1.759253in}}{\pgfqpoint{2.217741in}{1.767153in}}{\pgfqpoint{2.217741in}{1.775390in}}%
\pgfpathcurveto{\pgfqpoint{2.217741in}{1.783626in}}{\pgfqpoint{2.214469in}{1.791526in}}{\pgfqpoint{2.208645in}{1.797350in}}%
\pgfpathcurveto{\pgfqpoint{2.202821in}{1.803174in}}{\pgfqpoint{2.194921in}{1.806446in}}{\pgfqpoint{2.186685in}{1.806446in}}%
\pgfpathcurveto{\pgfqpoint{2.178448in}{1.806446in}}{\pgfqpoint{2.170548in}{1.803174in}}{\pgfqpoint{2.164724in}{1.797350in}}%
\pgfpathcurveto{\pgfqpoint{2.158901in}{1.791526in}}{\pgfqpoint{2.155628in}{1.783626in}}{\pgfqpoint{2.155628in}{1.775390in}}%
\pgfpathcurveto{\pgfqpoint{2.155628in}{1.767153in}}{\pgfqpoint{2.158901in}{1.759253in}}{\pgfqpoint{2.164724in}{1.753429in}}%
\pgfpathcurveto{\pgfqpoint{2.170548in}{1.747606in}}{\pgfqpoint{2.178448in}{1.744333in}}{\pgfqpoint{2.186685in}{1.744333in}}%
\pgfpathclose%
\pgfusepath{stroke,fill}%
\end{pgfscope}%
\begin{pgfscope}%
\pgfpathrectangle{\pgfqpoint{0.100000in}{0.212622in}}{\pgfqpoint{3.696000in}{3.696000in}}%
\pgfusepath{clip}%
\pgfsetbuttcap%
\pgfsetroundjoin%
\definecolor{currentfill}{rgb}{0.121569,0.466667,0.705882}%
\pgfsetfillcolor{currentfill}%
\pgfsetfillopacity{0.712270}%
\pgfsetlinewidth{1.003750pt}%
\definecolor{currentstroke}{rgb}{0.121569,0.466667,0.705882}%
\pgfsetstrokecolor{currentstroke}%
\pgfsetstrokeopacity{0.712270}%
\pgfsetdash{}{0pt}%
\pgfpathmoveto{\pgfqpoint{1.125644in}{1.202008in}}%
\pgfpathcurveto{\pgfqpoint{1.133880in}{1.202008in}}{\pgfqpoint{1.141780in}{1.205280in}}{\pgfqpoint{1.147604in}{1.211104in}}%
\pgfpathcurveto{\pgfqpoint{1.153428in}{1.216928in}}{\pgfqpoint{1.156701in}{1.224828in}}{\pgfqpoint{1.156701in}{1.233064in}}%
\pgfpathcurveto{\pgfqpoint{1.156701in}{1.241300in}}{\pgfqpoint{1.153428in}{1.249200in}}{\pgfqpoint{1.147604in}{1.255024in}}%
\pgfpathcurveto{\pgfqpoint{1.141780in}{1.260848in}}{\pgfqpoint{1.133880in}{1.264121in}}{\pgfqpoint{1.125644in}{1.264121in}}%
\pgfpathcurveto{\pgfqpoint{1.117408in}{1.264121in}}{\pgfqpoint{1.109508in}{1.260848in}}{\pgfqpoint{1.103684in}{1.255024in}}%
\pgfpathcurveto{\pgfqpoint{1.097860in}{1.249200in}}{\pgfqpoint{1.094588in}{1.241300in}}{\pgfqpoint{1.094588in}{1.233064in}}%
\pgfpathcurveto{\pgfqpoint{1.094588in}{1.224828in}}{\pgfqpoint{1.097860in}{1.216928in}}{\pgfqpoint{1.103684in}{1.211104in}}%
\pgfpathcurveto{\pgfqpoint{1.109508in}{1.205280in}}{\pgfqpoint{1.117408in}{1.202008in}}{\pgfqpoint{1.125644in}{1.202008in}}%
\pgfpathclose%
\pgfusepath{stroke,fill}%
\end{pgfscope}%
\begin{pgfscope}%
\pgfpathrectangle{\pgfqpoint{0.100000in}{0.212622in}}{\pgfqpoint{3.696000in}{3.696000in}}%
\pgfusepath{clip}%
\pgfsetbuttcap%
\pgfsetroundjoin%
\definecolor{currentfill}{rgb}{0.121569,0.466667,0.705882}%
\pgfsetfillcolor{currentfill}%
\pgfsetfillopacity{0.713939}%
\pgfsetlinewidth{1.003750pt}%
\definecolor{currentstroke}{rgb}{0.121569,0.466667,0.705882}%
\pgfsetstrokecolor{currentstroke}%
\pgfsetstrokeopacity{0.713939}%
\pgfsetdash{}{0pt}%
\pgfpathmoveto{\pgfqpoint{1.133590in}{1.201565in}}%
\pgfpathcurveto{\pgfqpoint{1.141826in}{1.201565in}}{\pgfqpoint{1.149726in}{1.204837in}}{\pgfqpoint{1.155550in}{1.210661in}}%
\pgfpathcurveto{\pgfqpoint{1.161374in}{1.216485in}}{\pgfqpoint{1.164646in}{1.224385in}}{\pgfqpoint{1.164646in}{1.232621in}}%
\pgfpathcurveto{\pgfqpoint{1.164646in}{1.240858in}}{\pgfqpoint{1.161374in}{1.248758in}}{\pgfqpoint{1.155550in}{1.254582in}}%
\pgfpathcurveto{\pgfqpoint{1.149726in}{1.260406in}}{\pgfqpoint{1.141826in}{1.263678in}}{\pgfqpoint{1.133590in}{1.263678in}}%
\pgfpathcurveto{\pgfqpoint{1.125354in}{1.263678in}}{\pgfqpoint{1.117454in}{1.260406in}}{\pgfqpoint{1.111630in}{1.254582in}}%
\pgfpathcurveto{\pgfqpoint{1.105806in}{1.248758in}}{\pgfqpoint{1.102533in}{1.240858in}}{\pgfqpoint{1.102533in}{1.232621in}}%
\pgfpathcurveto{\pgfqpoint{1.102533in}{1.224385in}}{\pgfqpoint{1.105806in}{1.216485in}}{\pgfqpoint{1.111630in}{1.210661in}}%
\pgfpathcurveto{\pgfqpoint{1.117454in}{1.204837in}}{\pgfqpoint{1.125354in}{1.201565in}}{\pgfqpoint{1.133590in}{1.201565in}}%
\pgfpathclose%
\pgfusepath{stroke,fill}%
\end{pgfscope}%
\begin{pgfscope}%
\pgfpathrectangle{\pgfqpoint{0.100000in}{0.212622in}}{\pgfqpoint{3.696000in}{3.696000in}}%
\pgfusepath{clip}%
\pgfsetbuttcap%
\pgfsetroundjoin%
\definecolor{currentfill}{rgb}{0.121569,0.466667,0.705882}%
\pgfsetfillcolor{currentfill}%
\pgfsetfillopacity{0.714129}%
\pgfsetlinewidth{1.003750pt}%
\definecolor{currentstroke}{rgb}{0.121569,0.466667,0.705882}%
\pgfsetstrokecolor{currentstroke}%
\pgfsetstrokeopacity{0.714129}%
\pgfsetdash{}{0pt}%
\pgfpathmoveto{\pgfqpoint{2.188902in}{1.733664in}}%
\pgfpathcurveto{\pgfqpoint{2.197138in}{1.733664in}}{\pgfqpoint{2.205038in}{1.736937in}}{\pgfqpoint{2.210862in}{1.742761in}}%
\pgfpathcurveto{\pgfqpoint{2.216686in}{1.748584in}}{\pgfqpoint{2.219958in}{1.756485in}}{\pgfqpoint{2.219958in}{1.764721in}}%
\pgfpathcurveto{\pgfqpoint{2.219958in}{1.772957in}}{\pgfqpoint{2.216686in}{1.780857in}}{\pgfqpoint{2.210862in}{1.786681in}}%
\pgfpathcurveto{\pgfqpoint{2.205038in}{1.792505in}}{\pgfqpoint{2.197138in}{1.795777in}}{\pgfqpoint{2.188902in}{1.795777in}}%
\pgfpathcurveto{\pgfqpoint{2.180665in}{1.795777in}}{\pgfqpoint{2.172765in}{1.792505in}}{\pgfqpoint{2.166941in}{1.786681in}}%
\pgfpathcurveto{\pgfqpoint{2.161117in}{1.780857in}}{\pgfqpoint{2.157845in}{1.772957in}}{\pgfqpoint{2.157845in}{1.764721in}}%
\pgfpathcurveto{\pgfqpoint{2.157845in}{1.756485in}}{\pgfqpoint{2.161117in}{1.748584in}}{\pgfqpoint{2.166941in}{1.742761in}}%
\pgfpathcurveto{\pgfqpoint{2.172765in}{1.736937in}}{\pgfqpoint{2.180665in}{1.733664in}}{\pgfqpoint{2.188902in}{1.733664in}}%
\pgfpathclose%
\pgfusepath{stroke,fill}%
\end{pgfscope}%
\begin{pgfscope}%
\pgfpathrectangle{\pgfqpoint{0.100000in}{0.212622in}}{\pgfqpoint{3.696000in}{3.696000in}}%
\pgfusepath{clip}%
\pgfsetbuttcap%
\pgfsetroundjoin%
\definecolor{currentfill}{rgb}{0.121569,0.466667,0.705882}%
\pgfsetfillcolor{currentfill}%
\pgfsetfillopacity{0.717117}%
\pgfsetlinewidth{1.003750pt}%
\definecolor{currentstroke}{rgb}{0.121569,0.466667,0.705882}%
\pgfsetstrokecolor{currentstroke}%
\pgfsetstrokeopacity{0.717117}%
\pgfsetdash{}{0pt}%
\pgfpathmoveto{\pgfqpoint{2.192041in}{1.722248in}}%
\pgfpathcurveto{\pgfqpoint{2.200277in}{1.722248in}}{\pgfqpoint{2.208177in}{1.725520in}}{\pgfqpoint{2.214001in}{1.731344in}}%
\pgfpathcurveto{\pgfqpoint{2.219825in}{1.737168in}}{\pgfqpoint{2.223097in}{1.745068in}}{\pgfqpoint{2.223097in}{1.753304in}}%
\pgfpathcurveto{\pgfqpoint{2.223097in}{1.761541in}}{\pgfqpoint{2.219825in}{1.769441in}}{\pgfqpoint{2.214001in}{1.775265in}}%
\pgfpathcurveto{\pgfqpoint{2.208177in}{1.781089in}}{\pgfqpoint{2.200277in}{1.784361in}}{\pgfqpoint{2.192041in}{1.784361in}}%
\pgfpathcurveto{\pgfqpoint{2.183805in}{1.784361in}}{\pgfqpoint{2.175905in}{1.781089in}}{\pgfqpoint{2.170081in}{1.775265in}}%
\pgfpathcurveto{\pgfqpoint{2.164257in}{1.769441in}}{\pgfqpoint{2.160984in}{1.761541in}}{\pgfqpoint{2.160984in}{1.753304in}}%
\pgfpathcurveto{\pgfqpoint{2.160984in}{1.745068in}}{\pgfqpoint{2.164257in}{1.737168in}}{\pgfqpoint{2.170081in}{1.731344in}}%
\pgfpathcurveto{\pgfqpoint{2.175905in}{1.725520in}}{\pgfqpoint{2.183805in}{1.722248in}}{\pgfqpoint{2.192041in}{1.722248in}}%
\pgfpathclose%
\pgfusepath{stroke,fill}%
\end{pgfscope}%
\begin{pgfscope}%
\pgfpathrectangle{\pgfqpoint{0.100000in}{0.212622in}}{\pgfqpoint{3.696000in}{3.696000in}}%
\pgfusepath{clip}%
\pgfsetbuttcap%
\pgfsetroundjoin%
\definecolor{currentfill}{rgb}{0.121569,0.466667,0.705882}%
\pgfsetfillcolor{currentfill}%
\pgfsetfillopacity{0.717188}%
\pgfsetlinewidth{1.003750pt}%
\definecolor{currentstroke}{rgb}{0.121569,0.466667,0.705882}%
\pgfsetstrokecolor{currentstroke}%
\pgfsetstrokeopacity{0.717188}%
\pgfsetdash{}{0pt}%
\pgfpathmoveto{\pgfqpoint{1.148020in}{1.201512in}}%
\pgfpathcurveto{\pgfqpoint{1.156256in}{1.201512in}}{\pgfqpoint{1.164156in}{1.204785in}}{\pgfqpoint{1.169980in}{1.210608in}}%
\pgfpathcurveto{\pgfqpoint{1.175804in}{1.216432in}}{\pgfqpoint{1.179076in}{1.224332in}}{\pgfqpoint{1.179076in}{1.232569in}}%
\pgfpathcurveto{\pgfqpoint{1.179076in}{1.240805in}}{\pgfqpoint{1.175804in}{1.248705in}}{\pgfqpoint{1.169980in}{1.254529in}}%
\pgfpathcurveto{\pgfqpoint{1.164156in}{1.260353in}}{\pgfqpoint{1.156256in}{1.263625in}}{\pgfqpoint{1.148020in}{1.263625in}}%
\pgfpathcurveto{\pgfqpoint{1.139783in}{1.263625in}}{\pgfqpoint{1.131883in}{1.260353in}}{\pgfqpoint{1.126059in}{1.254529in}}%
\pgfpathcurveto{\pgfqpoint{1.120235in}{1.248705in}}{\pgfqpoint{1.116963in}{1.240805in}}{\pgfqpoint{1.116963in}{1.232569in}}%
\pgfpathcurveto{\pgfqpoint{1.116963in}{1.224332in}}{\pgfqpoint{1.120235in}{1.216432in}}{\pgfqpoint{1.126059in}{1.210608in}}%
\pgfpathcurveto{\pgfqpoint{1.131883in}{1.204785in}}{\pgfqpoint{1.139783in}{1.201512in}}{\pgfqpoint{1.148020in}{1.201512in}}%
\pgfpathclose%
\pgfusepath{stroke,fill}%
\end{pgfscope}%
\begin{pgfscope}%
\pgfpathrectangle{\pgfqpoint{0.100000in}{0.212622in}}{\pgfqpoint{3.696000in}{3.696000in}}%
\pgfusepath{clip}%
\pgfsetbuttcap%
\pgfsetroundjoin%
\definecolor{currentfill}{rgb}{0.121569,0.466667,0.705882}%
\pgfsetfillcolor{currentfill}%
\pgfsetfillopacity{0.720302}%
\pgfsetlinewidth{1.003750pt}%
\definecolor{currentstroke}{rgb}{0.121569,0.466667,0.705882}%
\pgfsetstrokecolor{currentstroke}%
\pgfsetstrokeopacity{0.720302}%
\pgfsetdash{}{0pt}%
\pgfpathmoveto{\pgfqpoint{1.161315in}{1.200777in}}%
\pgfpathcurveto{\pgfqpoint{1.169551in}{1.200777in}}{\pgfqpoint{1.177451in}{1.204049in}}{\pgfqpoint{1.183275in}{1.209873in}}%
\pgfpathcurveto{\pgfqpoint{1.189099in}{1.215697in}}{\pgfqpoint{1.192371in}{1.223597in}}{\pgfqpoint{1.192371in}{1.231833in}}%
\pgfpathcurveto{\pgfqpoint{1.192371in}{1.240070in}}{\pgfqpoint{1.189099in}{1.247970in}}{\pgfqpoint{1.183275in}{1.253794in}}%
\pgfpathcurveto{\pgfqpoint{1.177451in}{1.259618in}}{\pgfqpoint{1.169551in}{1.262890in}}{\pgfqpoint{1.161315in}{1.262890in}}%
\pgfpathcurveto{\pgfqpoint{1.153079in}{1.262890in}}{\pgfqpoint{1.145179in}{1.259618in}}{\pgfqpoint{1.139355in}{1.253794in}}%
\pgfpathcurveto{\pgfqpoint{1.133531in}{1.247970in}}{\pgfqpoint{1.130258in}{1.240070in}}{\pgfqpoint{1.130258in}{1.231833in}}%
\pgfpathcurveto{\pgfqpoint{1.130258in}{1.223597in}}{\pgfqpoint{1.133531in}{1.215697in}}{\pgfqpoint{1.139355in}{1.209873in}}%
\pgfpathcurveto{\pgfqpoint{1.145179in}{1.204049in}}{\pgfqpoint{1.153079in}{1.200777in}}{\pgfqpoint{1.161315in}{1.200777in}}%
\pgfpathclose%
\pgfusepath{stroke,fill}%
\end{pgfscope}%
\begin{pgfscope}%
\pgfpathrectangle{\pgfqpoint{0.100000in}{0.212622in}}{\pgfqpoint{3.696000in}{3.696000in}}%
\pgfusepath{clip}%
\pgfsetbuttcap%
\pgfsetroundjoin%
\definecolor{currentfill}{rgb}{0.121569,0.466667,0.705882}%
\pgfsetfillcolor{currentfill}%
\pgfsetfillopacity{0.720664}%
\pgfsetlinewidth{1.003750pt}%
\definecolor{currentstroke}{rgb}{0.121569,0.466667,0.705882}%
\pgfsetstrokecolor{currentstroke}%
\pgfsetstrokeopacity{0.720664}%
\pgfsetdash{}{0pt}%
\pgfpathmoveto{\pgfqpoint{2.194796in}{1.710660in}}%
\pgfpathcurveto{\pgfqpoint{2.203032in}{1.710660in}}{\pgfqpoint{2.210932in}{1.713933in}}{\pgfqpoint{2.216756in}{1.719757in}}%
\pgfpathcurveto{\pgfqpoint{2.222580in}{1.725580in}}{\pgfqpoint{2.225852in}{1.733481in}}{\pgfqpoint{2.225852in}{1.741717in}}%
\pgfpathcurveto{\pgfqpoint{2.225852in}{1.749953in}}{\pgfqpoint{2.222580in}{1.757853in}}{\pgfqpoint{2.216756in}{1.763677in}}%
\pgfpathcurveto{\pgfqpoint{2.210932in}{1.769501in}}{\pgfqpoint{2.203032in}{1.772773in}}{\pgfqpoint{2.194796in}{1.772773in}}%
\pgfpathcurveto{\pgfqpoint{2.186559in}{1.772773in}}{\pgfqpoint{2.178659in}{1.769501in}}{\pgfqpoint{2.172835in}{1.763677in}}%
\pgfpathcurveto{\pgfqpoint{2.167011in}{1.757853in}}{\pgfqpoint{2.163739in}{1.749953in}}{\pgfqpoint{2.163739in}{1.741717in}}%
\pgfpathcurveto{\pgfqpoint{2.163739in}{1.733481in}}{\pgfqpoint{2.167011in}{1.725580in}}{\pgfqpoint{2.172835in}{1.719757in}}%
\pgfpathcurveto{\pgfqpoint{2.178659in}{1.713933in}}{\pgfqpoint{2.186559in}{1.710660in}}{\pgfqpoint{2.194796in}{1.710660in}}%
\pgfpathclose%
\pgfusepath{stroke,fill}%
\end{pgfscope}%
\begin{pgfscope}%
\pgfpathrectangle{\pgfqpoint{0.100000in}{0.212622in}}{\pgfqpoint{3.696000in}{3.696000in}}%
\pgfusepath{clip}%
\pgfsetbuttcap%
\pgfsetroundjoin%
\definecolor{currentfill}{rgb}{0.121569,0.466667,0.705882}%
\pgfsetfillcolor{currentfill}%
\pgfsetfillopacity{0.722903}%
\pgfsetlinewidth{1.003750pt}%
\definecolor{currentstroke}{rgb}{0.121569,0.466667,0.705882}%
\pgfsetstrokecolor{currentstroke}%
\pgfsetstrokeopacity{0.722903}%
\pgfsetdash{}{0pt}%
\pgfpathmoveto{\pgfqpoint{1.173821in}{1.199850in}}%
\pgfpathcurveto{\pgfqpoint{1.182057in}{1.199850in}}{\pgfqpoint{1.189957in}{1.203122in}}{\pgfqpoint{1.195781in}{1.208946in}}%
\pgfpathcurveto{\pgfqpoint{1.201605in}{1.214770in}}{\pgfqpoint{1.204878in}{1.222670in}}{\pgfqpoint{1.204878in}{1.230906in}}%
\pgfpathcurveto{\pgfqpoint{1.204878in}{1.239143in}}{\pgfqpoint{1.201605in}{1.247043in}}{\pgfqpoint{1.195781in}{1.252867in}}%
\pgfpathcurveto{\pgfqpoint{1.189957in}{1.258691in}}{\pgfqpoint{1.182057in}{1.261963in}}{\pgfqpoint{1.173821in}{1.261963in}}%
\pgfpathcurveto{\pgfqpoint{1.165585in}{1.261963in}}{\pgfqpoint{1.157685in}{1.258691in}}{\pgfqpoint{1.151861in}{1.252867in}}%
\pgfpathcurveto{\pgfqpoint{1.146037in}{1.247043in}}{\pgfqpoint{1.142765in}{1.239143in}}{\pgfqpoint{1.142765in}{1.230906in}}%
\pgfpathcurveto{\pgfqpoint{1.142765in}{1.222670in}}{\pgfqpoint{1.146037in}{1.214770in}}{\pgfqpoint{1.151861in}{1.208946in}}%
\pgfpathcurveto{\pgfqpoint{1.157685in}{1.203122in}}{\pgfqpoint{1.165585in}{1.199850in}}{\pgfqpoint{1.173821in}{1.199850in}}%
\pgfpathclose%
\pgfusepath{stroke,fill}%
\end{pgfscope}%
\begin{pgfscope}%
\pgfpathrectangle{\pgfqpoint{0.100000in}{0.212622in}}{\pgfqpoint{3.696000in}{3.696000in}}%
\pgfusepath{clip}%
\pgfsetbuttcap%
\pgfsetroundjoin%
\definecolor{currentfill}{rgb}{0.121569,0.466667,0.705882}%
\pgfsetfillcolor{currentfill}%
\pgfsetfillopacity{0.724756}%
\pgfsetlinewidth{1.003750pt}%
\definecolor{currentstroke}{rgb}{0.121569,0.466667,0.705882}%
\pgfsetstrokecolor{currentstroke}%
\pgfsetstrokeopacity{0.724756}%
\pgfsetdash{}{0pt}%
\pgfpathmoveto{\pgfqpoint{2.197416in}{1.698429in}}%
\pgfpathcurveto{\pgfqpoint{2.205652in}{1.698429in}}{\pgfqpoint{2.213552in}{1.701701in}}{\pgfqpoint{2.219376in}{1.707525in}}%
\pgfpathcurveto{\pgfqpoint{2.225200in}{1.713349in}}{\pgfqpoint{2.228472in}{1.721249in}}{\pgfqpoint{2.228472in}{1.729485in}}%
\pgfpathcurveto{\pgfqpoint{2.228472in}{1.737721in}}{\pgfqpoint{2.225200in}{1.745622in}}{\pgfqpoint{2.219376in}{1.751445in}}%
\pgfpathcurveto{\pgfqpoint{2.213552in}{1.757269in}}{\pgfqpoint{2.205652in}{1.760542in}}{\pgfqpoint{2.197416in}{1.760542in}}%
\pgfpathcurveto{\pgfqpoint{2.189179in}{1.760542in}}{\pgfqpoint{2.181279in}{1.757269in}}{\pgfqpoint{2.175455in}{1.751445in}}%
\pgfpathcurveto{\pgfqpoint{2.169632in}{1.745622in}}{\pgfqpoint{2.166359in}{1.737721in}}{\pgfqpoint{2.166359in}{1.729485in}}%
\pgfpathcurveto{\pgfqpoint{2.166359in}{1.721249in}}{\pgfqpoint{2.169632in}{1.713349in}}{\pgfqpoint{2.175455in}{1.707525in}}%
\pgfpathcurveto{\pgfqpoint{2.181279in}{1.701701in}}{\pgfqpoint{2.189179in}{1.698429in}}{\pgfqpoint{2.197416in}{1.698429in}}%
\pgfpathclose%
\pgfusepath{stroke,fill}%
\end{pgfscope}%
\begin{pgfscope}%
\pgfpathrectangle{\pgfqpoint{0.100000in}{0.212622in}}{\pgfqpoint{3.696000in}{3.696000in}}%
\pgfusepath{clip}%
\pgfsetbuttcap%
\pgfsetroundjoin%
\definecolor{currentfill}{rgb}{0.121569,0.466667,0.705882}%
\pgfsetfillcolor{currentfill}%
\pgfsetfillopacity{0.725474}%
\pgfsetlinewidth{1.003750pt}%
\definecolor{currentstroke}{rgb}{0.121569,0.466667,0.705882}%
\pgfsetstrokecolor{currentstroke}%
\pgfsetstrokeopacity{0.725474}%
\pgfsetdash{}{0pt}%
\pgfpathmoveto{\pgfqpoint{1.186008in}{1.198832in}}%
\pgfpathcurveto{\pgfqpoint{1.194244in}{1.198832in}}{\pgfqpoint{1.202144in}{1.202104in}}{\pgfqpoint{1.207968in}{1.207928in}}%
\pgfpathcurveto{\pgfqpoint{1.213792in}{1.213752in}}{\pgfqpoint{1.217064in}{1.221652in}}{\pgfqpoint{1.217064in}{1.229888in}}%
\pgfpathcurveto{\pgfqpoint{1.217064in}{1.238124in}}{\pgfqpoint{1.213792in}{1.246024in}}{\pgfqpoint{1.207968in}{1.251848in}}%
\pgfpathcurveto{\pgfqpoint{1.202144in}{1.257672in}}{\pgfqpoint{1.194244in}{1.260945in}}{\pgfqpoint{1.186008in}{1.260945in}}%
\pgfpathcurveto{\pgfqpoint{1.177771in}{1.260945in}}{\pgfqpoint{1.169871in}{1.257672in}}{\pgfqpoint{1.164047in}{1.251848in}}%
\pgfpathcurveto{\pgfqpoint{1.158223in}{1.246024in}}{\pgfqpoint{1.154951in}{1.238124in}}{\pgfqpoint{1.154951in}{1.229888in}}%
\pgfpathcurveto{\pgfqpoint{1.154951in}{1.221652in}}{\pgfqpoint{1.158223in}{1.213752in}}{\pgfqpoint{1.164047in}{1.207928in}}%
\pgfpathcurveto{\pgfqpoint{1.169871in}{1.202104in}}{\pgfqpoint{1.177771in}{1.198832in}}{\pgfqpoint{1.186008in}{1.198832in}}%
\pgfpathclose%
\pgfusepath{stroke,fill}%
\end{pgfscope}%
\begin{pgfscope}%
\pgfpathrectangle{\pgfqpoint{0.100000in}{0.212622in}}{\pgfqpoint{3.696000in}{3.696000in}}%
\pgfusepath{clip}%
\pgfsetbuttcap%
\pgfsetroundjoin%
\definecolor{currentfill}{rgb}{0.121569,0.466667,0.705882}%
\pgfsetfillcolor{currentfill}%
\pgfsetfillopacity{0.727857}%
\pgfsetlinewidth{1.003750pt}%
\definecolor{currentstroke}{rgb}{0.121569,0.466667,0.705882}%
\pgfsetstrokecolor{currentstroke}%
\pgfsetstrokeopacity{0.727857}%
\pgfsetdash{}{0pt}%
\pgfpathmoveto{\pgfqpoint{1.196114in}{1.197771in}}%
\pgfpathcurveto{\pgfqpoint{1.204350in}{1.197771in}}{\pgfqpoint{1.212250in}{1.201043in}}{\pgfqpoint{1.218074in}{1.206867in}}%
\pgfpathcurveto{\pgfqpoint{1.223898in}{1.212691in}}{\pgfqpoint{1.227170in}{1.220591in}}{\pgfqpoint{1.227170in}{1.228828in}}%
\pgfpathcurveto{\pgfqpoint{1.227170in}{1.237064in}}{\pgfqpoint{1.223898in}{1.244964in}}{\pgfqpoint{1.218074in}{1.250788in}}%
\pgfpathcurveto{\pgfqpoint{1.212250in}{1.256612in}}{\pgfqpoint{1.204350in}{1.259884in}}{\pgfqpoint{1.196114in}{1.259884in}}%
\pgfpathcurveto{\pgfqpoint{1.187878in}{1.259884in}}{\pgfqpoint{1.179977in}{1.256612in}}{\pgfqpoint{1.174154in}{1.250788in}}%
\pgfpathcurveto{\pgfqpoint{1.168330in}{1.244964in}}{\pgfqpoint{1.165057in}{1.237064in}}{\pgfqpoint{1.165057in}{1.228828in}}%
\pgfpathcurveto{\pgfqpoint{1.165057in}{1.220591in}}{\pgfqpoint{1.168330in}{1.212691in}}{\pgfqpoint{1.174154in}{1.206867in}}%
\pgfpathcurveto{\pgfqpoint{1.179977in}{1.201043in}}{\pgfqpoint{1.187878in}{1.197771in}}{\pgfqpoint{1.196114in}{1.197771in}}%
\pgfpathclose%
\pgfusepath{stroke,fill}%
\end{pgfscope}%
\begin{pgfscope}%
\pgfpathrectangle{\pgfqpoint{0.100000in}{0.212622in}}{\pgfqpoint{3.696000in}{3.696000in}}%
\pgfusepath{clip}%
\pgfsetbuttcap%
\pgfsetroundjoin%
\definecolor{currentfill}{rgb}{0.121569,0.466667,0.705882}%
\pgfsetfillcolor{currentfill}%
\pgfsetfillopacity{0.728950}%
\pgfsetlinewidth{1.003750pt}%
\definecolor{currentstroke}{rgb}{0.121569,0.466667,0.705882}%
\pgfsetstrokecolor{currentstroke}%
\pgfsetstrokeopacity{0.728950}%
\pgfsetdash{}{0pt}%
\pgfpathmoveto{\pgfqpoint{2.201585in}{1.684440in}}%
\pgfpathcurveto{\pgfqpoint{2.209822in}{1.684440in}}{\pgfqpoint{2.217722in}{1.687712in}}{\pgfqpoint{2.223546in}{1.693536in}}%
\pgfpathcurveto{\pgfqpoint{2.229370in}{1.699360in}}{\pgfqpoint{2.232642in}{1.707260in}}{\pgfqpoint{2.232642in}{1.715497in}}%
\pgfpathcurveto{\pgfqpoint{2.232642in}{1.723733in}}{\pgfqpoint{2.229370in}{1.731633in}}{\pgfqpoint{2.223546in}{1.737457in}}%
\pgfpathcurveto{\pgfqpoint{2.217722in}{1.743281in}}{\pgfqpoint{2.209822in}{1.746553in}}{\pgfqpoint{2.201585in}{1.746553in}}%
\pgfpathcurveto{\pgfqpoint{2.193349in}{1.746553in}}{\pgfqpoint{2.185449in}{1.743281in}}{\pgfqpoint{2.179625in}{1.737457in}}%
\pgfpathcurveto{\pgfqpoint{2.173801in}{1.731633in}}{\pgfqpoint{2.170529in}{1.723733in}}{\pgfqpoint{2.170529in}{1.715497in}}%
\pgfpathcurveto{\pgfqpoint{2.170529in}{1.707260in}}{\pgfqpoint{2.173801in}{1.699360in}}{\pgfqpoint{2.179625in}{1.693536in}}%
\pgfpathcurveto{\pgfqpoint{2.185449in}{1.687712in}}{\pgfqpoint{2.193349in}{1.684440in}}{\pgfqpoint{2.201585in}{1.684440in}}%
\pgfpathclose%
\pgfusepath{stroke,fill}%
\end{pgfscope}%
\begin{pgfscope}%
\pgfpathrectangle{\pgfqpoint{0.100000in}{0.212622in}}{\pgfqpoint{3.696000in}{3.696000in}}%
\pgfusepath{clip}%
\pgfsetbuttcap%
\pgfsetroundjoin%
\definecolor{currentfill}{rgb}{0.121569,0.466667,0.705882}%
\pgfsetfillcolor{currentfill}%
\pgfsetfillopacity{0.730092}%
\pgfsetlinewidth{1.003750pt}%
\definecolor{currentstroke}{rgb}{0.121569,0.466667,0.705882}%
\pgfsetstrokecolor{currentstroke}%
\pgfsetstrokeopacity{0.730092}%
\pgfsetdash{}{0pt}%
\pgfpathmoveto{\pgfqpoint{1.205610in}{1.196871in}}%
\pgfpathcurveto{\pgfqpoint{1.213846in}{1.196871in}}{\pgfqpoint{1.221747in}{1.200143in}}{\pgfqpoint{1.227570in}{1.205967in}}%
\pgfpathcurveto{\pgfqpoint{1.233394in}{1.211791in}}{\pgfqpoint{1.236667in}{1.219691in}}{\pgfqpoint{1.236667in}{1.227927in}}%
\pgfpathcurveto{\pgfqpoint{1.236667in}{1.236163in}}{\pgfqpoint{1.233394in}{1.244064in}}{\pgfqpoint{1.227570in}{1.249887in}}%
\pgfpathcurveto{\pgfqpoint{1.221747in}{1.255711in}}{\pgfqpoint{1.213846in}{1.258984in}}{\pgfqpoint{1.205610in}{1.258984in}}%
\pgfpathcurveto{\pgfqpoint{1.197374in}{1.258984in}}{\pgfqpoint{1.189474in}{1.255711in}}{\pgfqpoint{1.183650in}{1.249887in}}%
\pgfpathcurveto{\pgfqpoint{1.177826in}{1.244064in}}{\pgfqpoint{1.174554in}{1.236163in}}{\pgfqpoint{1.174554in}{1.227927in}}%
\pgfpathcurveto{\pgfqpoint{1.174554in}{1.219691in}}{\pgfqpoint{1.177826in}{1.211791in}}{\pgfqpoint{1.183650in}{1.205967in}}%
\pgfpathcurveto{\pgfqpoint{1.189474in}{1.200143in}}{\pgfqpoint{1.197374in}{1.196871in}}{\pgfqpoint{1.205610in}{1.196871in}}%
\pgfpathclose%
\pgfusepath{stroke,fill}%
\end{pgfscope}%
\begin{pgfscope}%
\pgfpathrectangle{\pgfqpoint{0.100000in}{0.212622in}}{\pgfqpoint{3.696000in}{3.696000in}}%
\pgfusepath{clip}%
\pgfsetbuttcap%
\pgfsetroundjoin%
\definecolor{currentfill}{rgb}{0.121569,0.466667,0.705882}%
\pgfsetfillcolor{currentfill}%
\pgfsetfillopacity{0.732085}%
\pgfsetlinewidth{1.003750pt}%
\definecolor{currentstroke}{rgb}{0.121569,0.466667,0.705882}%
\pgfsetstrokecolor{currentstroke}%
\pgfsetstrokeopacity{0.732085}%
\pgfsetdash{}{0pt}%
\pgfpathmoveto{\pgfqpoint{1.214166in}{1.196379in}}%
\pgfpathcurveto{\pgfqpoint{1.222402in}{1.196379in}}{\pgfqpoint{1.230302in}{1.199652in}}{\pgfqpoint{1.236126in}{1.205476in}}%
\pgfpathcurveto{\pgfqpoint{1.241950in}{1.211300in}}{\pgfqpoint{1.245222in}{1.219200in}}{\pgfqpoint{1.245222in}{1.227436in}}%
\pgfpathcurveto{\pgfqpoint{1.245222in}{1.235672in}}{\pgfqpoint{1.241950in}{1.243572in}}{\pgfqpoint{1.236126in}{1.249396in}}%
\pgfpathcurveto{\pgfqpoint{1.230302in}{1.255220in}}{\pgfqpoint{1.222402in}{1.258492in}}{\pgfqpoint{1.214166in}{1.258492in}}%
\pgfpathcurveto{\pgfqpoint{1.205929in}{1.258492in}}{\pgfqpoint{1.198029in}{1.255220in}}{\pgfqpoint{1.192206in}{1.249396in}}%
\pgfpathcurveto{\pgfqpoint{1.186382in}{1.243572in}}{\pgfqpoint{1.183109in}{1.235672in}}{\pgfqpoint{1.183109in}{1.227436in}}%
\pgfpathcurveto{\pgfqpoint{1.183109in}{1.219200in}}{\pgfqpoint{1.186382in}{1.211300in}}{\pgfqpoint{1.192206in}{1.205476in}}%
\pgfpathcurveto{\pgfqpoint{1.198029in}{1.199652in}}{\pgfqpoint{1.205929in}{1.196379in}}{\pgfqpoint{1.214166in}{1.196379in}}%
\pgfpathclose%
\pgfusepath{stroke,fill}%
\end{pgfscope}%
\begin{pgfscope}%
\pgfpathrectangle{\pgfqpoint{0.100000in}{0.212622in}}{\pgfqpoint{3.696000in}{3.696000in}}%
\pgfusepath{clip}%
\pgfsetbuttcap%
\pgfsetroundjoin%
\definecolor{currentfill}{rgb}{0.121569,0.466667,0.705882}%
\pgfsetfillcolor{currentfill}%
\pgfsetfillopacity{0.733325}%
\pgfsetlinewidth{1.003750pt}%
\definecolor{currentstroke}{rgb}{0.121569,0.466667,0.705882}%
\pgfsetstrokecolor{currentstroke}%
\pgfsetstrokeopacity{0.733325}%
\pgfsetdash{}{0pt}%
\pgfpathmoveto{\pgfqpoint{2.205439in}{1.669924in}}%
\pgfpathcurveto{\pgfqpoint{2.213675in}{1.669924in}}{\pgfqpoint{2.221575in}{1.673196in}}{\pgfqpoint{2.227399in}{1.679020in}}%
\pgfpathcurveto{\pgfqpoint{2.233223in}{1.684844in}}{\pgfqpoint{2.236495in}{1.692744in}}{\pgfqpoint{2.236495in}{1.700980in}}%
\pgfpathcurveto{\pgfqpoint{2.236495in}{1.709217in}}{\pgfqpoint{2.233223in}{1.717117in}}{\pgfqpoint{2.227399in}{1.722941in}}%
\pgfpathcurveto{\pgfqpoint{2.221575in}{1.728765in}}{\pgfqpoint{2.213675in}{1.732037in}}{\pgfqpoint{2.205439in}{1.732037in}}%
\pgfpathcurveto{\pgfqpoint{2.197202in}{1.732037in}}{\pgfqpoint{2.189302in}{1.728765in}}{\pgfqpoint{2.183478in}{1.722941in}}%
\pgfpathcurveto{\pgfqpoint{2.177654in}{1.717117in}}{\pgfqpoint{2.174382in}{1.709217in}}{\pgfqpoint{2.174382in}{1.700980in}}%
\pgfpathcurveto{\pgfqpoint{2.174382in}{1.692744in}}{\pgfqpoint{2.177654in}{1.684844in}}{\pgfqpoint{2.183478in}{1.679020in}}%
\pgfpathcurveto{\pgfqpoint{2.189302in}{1.673196in}}{\pgfqpoint{2.197202in}{1.669924in}}{\pgfqpoint{2.205439in}{1.669924in}}%
\pgfpathclose%
\pgfusepath{stroke,fill}%
\end{pgfscope}%
\begin{pgfscope}%
\pgfpathrectangle{\pgfqpoint{0.100000in}{0.212622in}}{\pgfqpoint{3.696000in}{3.696000in}}%
\pgfusepath{clip}%
\pgfsetbuttcap%
\pgfsetroundjoin%
\definecolor{currentfill}{rgb}{0.121569,0.466667,0.705882}%
\pgfsetfillcolor{currentfill}%
\pgfsetfillopacity{0.733901}%
\pgfsetlinewidth{1.003750pt}%
\definecolor{currentstroke}{rgb}{0.121569,0.466667,0.705882}%
\pgfsetstrokecolor{currentstroke}%
\pgfsetstrokeopacity{0.733901}%
\pgfsetdash{}{0pt}%
\pgfpathmoveto{\pgfqpoint{1.221409in}{1.195551in}}%
\pgfpathcurveto{\pgfqpoint{1.229645in}{1.195551in}}{\pgfqpoint{1.237545in}{1.198823in}}{\pgfqpoint{1.243369in}{1.204647in}}%
\pgfpathcurveto{\pgfqpoint{1.249193in}{1.210471in}}{\pgfqpoint{1.252466in}{1.218371in}}{\pgfqpoint{1.252466in}{1.226608in}}%
\pgfpathcurveto{\pgfqpoint{1.252466in}{1.234844in}}{\pgfqpoint{1.249193in}{1.242744in}}{\pgfqpoint{1.243369in}{1.248568in}}%
\pgfpathcurveto{\pgfqpoint{1.237545in}{1.254392in}}{\pgfqpoint{1.229645in}{1.257664in}}{\pgfqpoint{1.221409in}{1.257664in}}%
\pgfpathcurveto{\pgfqpoint{1.213173in}{1.257664in}}{\pgfqpoint{1.205273in}{1.254392in}}{\pgfqpoint{1.199449in}{1.248568in}}%
\pgfpathcurveto{\pgfqpoint{1.193625in}{1.242744in}}{\pgfqpoint{1.190353in}{1.234844in}}{\pgfqpoint{1.190353in}{1.226608in}}%
\pgfpathcurveto{\pgfqpoint{1.190353in}{1.218371in}}{\pgfqpoint{1.193625in}{1.210471in}}{\pgfqpoint{1.199449in}{1.204647in}}%
\pgfpathcurveto{\pgfqpoint{1.205273in}{1.198823in}}{\pgfqpoint{1.213173in}{1.195551in}}{\pgfqpoint{1.221409in}{1.195551in}}%
\pgfpathclose%
\pgfusepath{stroke,fill}%
\end{pgfscope}%
\begin{pgfscope}%
\pgfpathrectangle{\pgfqpoint{0.100000in}{0.212622in}}{\pgfqpoint{3.696000in}{3.696000in}}%
\pgfusepath{clip}%
\pgfsetbuttcap%
\pgfsetroundjoin%
\definecolor{currentfill}{rgb}{0.121569,0.466667,0.705882}%
\pgfsetfillcolor{currentfill}%
\pgfsetfillopacity{0.737562}%
\pgfsetlinewidth{1.003750pt}%
\definecolor{currentstroke}{rgb}{0.121569,0.466667,0.705882}%
\pgfsetstrokecolor{currentstroke}%
\pgfsetstrokeopacity{0.737562}%
\pgfsetdash{}{0pt}%
\pgfpathmoveto{\pgfqpoint{1.234508in}{1.195163in}}%
\pgfpathcurveto{\pgfqpoint{1.242744in}{1.195163in}}{\pgfqpoint{1.250644in}{1.198435in}}{\pgfqpoint{1.256468in}{1.204259in}}%
\pgfpathcurveto{\pgfqpoint{1.262292in}{1.210083in}}{\pgfqpoint{1.265564in}{1.217983in}}{\pgfqpoint{1.265564in}{1.226220in}}%
\pgfpathcurveto{\pgfqpoint{1.265564in}{1.234456in}}{\pgfqpoint{1.262292in}{1.242356in}}{\pgfqpoint{1.256468in}{1.248180in}}%
\pgfpathcurveto{\pgfqpoint{1.250644in}{1.254004in}}{\pgfqpoint{1.242744in}{1.257276in}}{\pgfqpoint{1.234508in}{1.257276in}}%
\pgfpathcurveto{\pgfqpoint{1.226272in}{1.257276in}}{\pgfqpoint{1.218371in}{1.254004in}}{\pgfqpoint{1.212548in}{1.248180in}}%
\pgfpathcurveto{\pgfqpoint{1.206724in}{1.242356in}}{\pgfqpoint{1.203451in}{1.234456in}}{\pgfqpoint{1.203451in}{1.226220in}}%
\pgfpathcurveto{\pgfqpoint{1.203451in}{1.217983in}}{\pgfqpoint{1.206724in}{1.210083in}}{\pgfqpoint{1.212548in}{1.204259in}}%
\pgfpathcurveto{\pgfqpoint{1.218371in}{1.198435in}}{\pgfqpoint{1.226272in}{1.195163in}}{\pgfqpoint{1.234508in}{1.195163in}}%
\pgfpathclose%
\pgfusepath{stroke,fill}%
\end{pgfscope}%
\begin{pgfscope}%
\pgfpathrectangle{\pgfqpoint{0.100000in}{0.212622in}}{\pgfqpoint{3.696000in}{3.696000in}}%
\pgfusepath{clip}%
\pgfsetbuttcap%
\pgfsetroundjoin%
\definecolor{currentfill}{rgb}{0.121569,0.466667,0.705882}%
\pgfsetfillcolor{currentfill}%
\pgfsetfillopacity{0.738216}%
\pgfsetlinewidth{1.003750pt}%
\definecolor{currentstroke}{rgb}{0.121569,0.466667,0.705882}%
\pgfsetstrokecolor{currentstroke}%
\pgfsetstrokeopacity{0.738216}%
\pgfsetdash{}{0pt}%
\pgfpathmoveto{\pgfqpoint{2.208686in}{1.655630in}}%
\pgfpathcurveto{\pgfqpoint{2.216922in}{1.655630in}}{\pgfqpoint{2.224822in}{1.658902in}}{\pgfqpoint{2.230646in}{1.664726in}}%
\pgfpathcurveto{\pgfqpoint{2.236470in}{1.670550in}}{\pgfqpoint{2.239743in}{1.678450in}}{\pgfqpoint{2.239743in}{1.686686in}}%
\pgfpathcurveto{\pgfqpoint{2.239743in}{1.694922in}}{\pgfqpoint{2.236470in}{1.702822in}}{\pgfqpoint{2.230646in}{1.708646in}}%
\pgfpathcurveto{\pgfqpoint{2.224822in}{1.714470in}}{\pgfqpoint{2.216922in}{1.717743in}}{\pgfqpoint{2.208686in}{1.717743in}}%
\pgfpathcurveto{\pgfqpoint{2.200450in}{1.717743in}}{\pgfqpoint{2.192550in}{1.714470in}}{\pgfqpoint{2.186726in}{1.708646in}}%
\pgfpathcurveto{\pgfqpoint{2.180902in}{1.702822in}}{\pgfqpoint{2.177630in}{1.694922in}}{\pgfqpoint{2.177630in}{1.686686in}}%
\pgfpathcurveto{\pgfqpoint{2.177630in}{1.678450in}}{\pgfqpoint{2.180902in}{1.670550in}}{\pgfqpoint{2.186726in}{1.664726in}}%
\pgfpathcurveto{\pgfqpoint{2.192550in}{1.658902in}}{\pgfqpoint{2.200450in}{1.655630in}}{\pgfqpoint{2.208686in}{1.655630in}}%
\pgfpathclose%
\pgfusepath{stroke,fill}%
\end{pgfscope}%
\begin{pgfscope}%
\pgfpathrectangle{\pgfqpoint{0.100000in}{0.212622in}}{\pgfqpoint{3.696000in}{3.696000in}}%
\pgfusepath{clip}%
\pgfsetbuttcap%
\pgfsetroundjoin%
\definecolor{currentfill}{rgb}{0.121569,0.466667,0.705882}%
\pgfsetfillcolor{currentfill}%
\pgfsetfillopacity{0.740343}%
\pgfsetlinewidth{1.003750pt}%
\definecolor{currentstroke}{rgb}{0.121569,0.466667,0.705882}%
\pgfsetstrokecolor{currentstroke}%
\pgfsetstrokeopacity{0.740343}%
\pgfsetdash{}{0pt}%
\pgfpathmoveto{\pgfqpoint{1.245449in}{1.195404in}}%
\pgfpathcurveto{\pgfqpoint{1.253685in}{1.195404in}}{\pgfqpoint{1.261585in}{1.198676in}}{\pgfqpoint{1.267409in}{1.204500in}}%
\pgfpathcurveto{\pgfqpoint{1.273233in}{1.210324in}}{\pgfqpoint{1.276505in}{1.218224in}}{\pgfqpoint{1.276505in}{1.226460in}}%
\pgfpathcurveto{\pgfqpoint{1.276505in}{1.234697in}}{\pgfqpoint{1.273233in}{1.242597in}}{\pgfqpoint{1.267409in}{1.248421in}}%
\pgfpathcurveto{\pgfqpoint{1.261585in}{1.254245in}}{\pgfqpoint{1.253685in}{1.257517in}}{\pgfqpoint{1.245449in}{1.257517in}}%
\pgfpathcurveto{\pgfqpoint{1.237212in}{1.257517in}}{\pgfqpoint{1.229312in}{1.254245in}}{\pgfqpoint{1.223488in}{1.248421in}}%
\pgfpathcurveto{\pgfqpoint{1.217665in}{1.242597in}}{\pgfqpoint{1.214392in}{1.234697in}}{\pgfqpoint{1.214392in}{1.226460in}}%
\pgfpathcurveto{\pgfqpoint{1.214392in}{1.218224in}}{\pgfqpoint{1.217665in}{1.210324in}}{\pgfqpoint{1.223488in}{1.204500in}}%
\pgfpathcurveto{\pgfqpoint{1.229312in}{1.198676in}}{\pgfqpoint{1.237212in}{1.195404in}}{\pgfqpoint{1.245449in}{1.195404in}}%
\pgfpathclose%
\pgfusepath{stroke,fill}%
\end{pgfscope}%
\begin{pgfscope}%
\pgfpathrectangle{\pgfqpoint{0.100000in}{0.212622in}}{\pgfqpoint{3.696000in}{3.696000in}}%
\pgfusepath{clip}%
\pgfsetbuttcap%
\pgfsetroundjoin%
\definecolor{currentfill}{rgb}{0.121569,0.466667,0.705882}%
\pgfsetfillcolor{currentfill}%
\pgfsetfillopacity{0.742980}%
\pgfsetlinewidth{1.003750pt}%
\definecolor{currentstroke}{rgb}{0.121569,0.466667,0.705882}%
\pgfsetstrokecolor{currentstroke}%
\pgfsetstrokeopacity{0.742980}%
\pgfsetdash{}{0pt}%
\pgfpathmoveto{\pgfqpoint{1.255780in}{1.195869in}}%
\pgfpathcurveto{\pgfqpoint{1.264016in}{1.195869in}}{\pgfqpoint{1.271917in}{1.199141in}}{\pgfqpoint{1.277740in}{1.204965in}}%
\pgfpathcurveto{\pgfqpoint{1.283564in}{1.210789in}}{\pgfqpoint{1.286837in}{1.218689in}}{\pgfqpoint{1.286837in}{1.226925in}}%
\pgfpathcurveto{\pgfqpoint{1.286837in}{1.235162in}}{\pgfqpoint{1.283564in}{1.243062in}}{\pgfqpoint{1.277740in}{1.248886in}}%
\pgfpathcurveto{\pgfqpoint{1.271917in}{1.254710in}}{\pgfqpoint{1.264016in}{1.257982in}}{\pgfqpoint{1.255780in}{1.257982in}}%
\pgfpathcurveto{\pgfqpoint{1.247544in}{1.257982in}}{\pgfqpoint{1.239644in}{1.254710in}}{\pgfqpoint{1.233820in}{1.248886in}}%
\pgfpathcurveto{\pgfqpoint{1.227996in}{1.243062in}}{\pgfqpoint{1.224724in}{1.235162in}}{\pgfqpoint{1.224724in}{1.226925in}}%
\pgfpathcurveto{\pgfqpoint{1.224724in}{1.218689in}}{\pgfqpoint{1.227996in}{1.210789in}}{\pgfqpoint{1.233820in}{1.204965in}}%
\pgfpathcurveto{\pgfqpoint{1.239644in}{1.199141in}}{\pgfqpoint{1.247544in}{1.195869in}}{\pgfqpoint{1.255780in}{1.195869in}}%
\pgfpathclose%
\pgfusepath{stroke,fill}%
\end{pgfscope}%
\begin{pgfscope}%
\pgfpathrectangle{\pgfqpoint{0.100000in}{0.212622in}}{\pgfqpoint{3.696000in}{3.696000in}}%
\pgfusepath{clip}%
\pgfsetbuttcap%
\pgfsetroundjoin%
\definecolor{currentfill}{rgb}{0.121569,0.466667,0.705882}%
\pgfsetfillcolor{currentfill}%
\pgfsetfillopacity{0.743222}%
\pgfsetlinewidth{1.003750pt}%
\definecolor{currentstroke}{rgb}{0.121569,0.466667,0.705882}%
\pgfsetstrokecolor{currentstroke}%
\pgfsetstrokeopacity{0.743222}%
\pgfsetdash{}{0pt}%
\pgfpathmoveto{\pgfqpoint{2.213273in}{1.639627in}}%
\pgfpathcurveto{\pgfqpoint{2.221510in}{1.639627in}}{\pgfqpoint{2.229410in}{1.642899in}}{\pgfqpoint{2.235234in}{1.648723in}}%
\pgfpathcurveto{\pgfqpoint{2.241058in}{1.654547in}}{\pgfqpoint{2.244330in}{1.662447in}}{\pgfqpoint{2.244330in}{1.670683in}}%
\pgfpathcurveto{\pgfqpoint{2.244330in}{1.678920in}}{\pgfqpoint{2.241058in}{1.686820in}}{\pgfqpoint{2.235234in}{1.692644in}}%
\pgfpathcurveto{\pgfqpoint{2.229410in}{1.698468in}}{\pgfqpoint{2.221510in}{1.701740in}}{\pgfqpoint{2.213273in}{1.701740in}}%
\pgfpathcurveto{\pgfqpoint{2.205037in}{1.701740in}}{\pgfqpoint{2.197137in}{1.698468in}}{\pgfqpoint{2.191313in}{1.692644in}}%
\pgfpathcurveto{\pgfqpoint{2.185489in}{1.686820in}}{\pgfqpoint{2.182217in}{1.678920in}}{\pgfqpoint{2.182217in}{1.670683in}}%
\pgfpathcurveto{\pgfqpoint{2.182217in}{1.662447in}}{\pgfqpoint{2.185489in}{1.654547in}}{\pgfqpoint{2.191313in}{1.648723in}}%
\pgfpathcurveto{\pgfqpoint{2.197137in}{1.642899in}}{\pgfqpoint{2.205037in}{1.639627in}}{\pgfqpoint{2.213273in}{1.639627in}}%
\pgfpathclose%
\pgfusepath{stroke,fill}%
\end{pgfscope}%
\begin{pgfscope}%
\pgfpathrectangle{\pgfqpoint{0.100000in}{0.212622in}}{\pgfqpoint{3.696000in}{3.696000in}}%
\pgfusepath{clip}%
\pgfsetbuttcap%
\pgfsetroundjoin%
\definecolor{currentfill}{rgb}{0.121569,0.466667,0.705882}%
\pgfsetfillcolor{currentfill}%
\pgfsetfillopacity{0.745351}%
\pgfsetlinewidth{1.003750pt}%
\definecolor{currentstroke}{rgb}{0.121569,0.466667,0.705882}%
\pgfsetstrokecolor{currentstroke}%
\pgfsetstrokeopacity{0.745351}%
\pgfsetdash{}{0pt}%
\pgfpathmoveto{\pgfqpoint{1.264987in}{1.196609in}}%
\pgfpathcurveto{\pgfqpoint{1.273223in}{1.196609in}}{\pgfqpoint{1.281123in}{1.199881in}}{\pgfqpoint{1.286947in}{1.205705in}}%
\pgfpathcurveto{\pgfqpoint{1.292771in}{1.211529in}}{\pgfqpoint{1.296044in}{1.219429in}}{\pgfqpoint{1.296044in}{1.227666in}}%
\pgfpathcurveto{\pgfqpoint{1.296044in}{1.235902in}}{\pgfqpoint{1.292771in}{1.243802in}}{\pgfqpoint{1.286947in}{1.249626in}}%
\pgfpathcurveto{\pgfqpoint{1.281123in}{1.255450in}}{\pgfqpoint{1.273223in}{1.258722in}}{\pgfqpoint{1.264987in}{1.258722in}}%
\pgfpathcurveto{\pgfqpoint{1.256751in}{1.258722in}}{\pgfqpoint{1.248851in}{1.255450in}}{\pgfqpoint{1.243027in}{1.249626in}}%
\pgfpathcurveto{\pgfqpoint{1.237203in}{1.243802in}}{\pgfqpoint{1.233931in}{1.235902in}}{\pgfqpoint{1.233931in}{1.227666in}}%
\pgfpathcurveto{\pgfqpoint{1.233931in}{1.219429in}}{\pgfqpoint{1.237203in}{1.211529in}}{\pgfqpoint{1.243027in}{1.205705in}}%
\pgfpathcurveto{\pgfqpoint{1.248851in}{1.199881in}}{\pgfqpoint{1.256751in}{1.196609in}}{\pgfqpoint{1.264987in}{1.196609in}}%
\pgfpathclose%
\pgfusepath{stroke,fill}%
\end{pgfscope}%
\begin{pgfscope}%
\pgfpathrectangle{\pgfqpoint{0.100000in}{0.212622in}}{\pgfqpoint{3.696000in}{3.696000in}}%
\pgfusepath{clip}%
\pgfsetbuttcap%
\pgfsetroundjoin%
\definecolor{currentfill}{rgb}{0.121569,0.466667,0.705882}%
\pgfsetfillcolor{currentfill}%
\pgfsetfillopacity{0.747345}%
\pgfsetlinewidth{1.003750pt}%
\definecolor{currentstroke}{rgb}{0.121569,0.466667,0.705882}%
\pgfsetstrokecolor{currentstroke}%
\pgfsetstrokeopacity{0.747345}%
\pgfsetdash{}{0pt}%
\pgfpathmoveto{\pgfqpoint{1.272103in}{1.198150in}}%
\pgfpathcurveto{\pgfqpoint{1.280340in}{1.198150in}}{\pgfqpoint{1.288240in}{1.201423in}}{\pgfqpoint{1.294064in}{1.207247in}}%
\pgfpathcurveto{\pgfqpoint{1.299888in}{1.213071in}}{\pgfqpoint{1.303160in}{1.220971in}}{\pgfqpoint{1.303160in}{1.229207in}}%
\pgfpathcurveto{\pgfqpoint{1.303160in}{1.237443in}}{\pgfqpoint{1.299888in}{1.245343in}}{\pgfqpoint{1.294064in}{1.251167in}}%
\pgfpathcurveto{\pgfqpoint{1.288240in}{1.256991in}}{\pgfqpoint{1.280340in}{1.260263in}}{\pgfqpoint{1.272103in}{1.260263in}}%
\pgfpathcurveto{\pgfqpoint{1.263867in}{1.260263in}}{\pgfqpoint{1.255967in}{1.256991in}}{\pgfqpoint{1.250143in}{1.251167in}}%
\pgfpathcurveto{\pgfqpoint{1.244319in}{1.245343in}}{\pgfqpoint{1.241047in}{1.237443in}}{\pgfqpoint{1.241047in}{1.229207in}}%
\pgfpathcurveto{\pgfqpoint{1.241047in}{1.220971in}}{\pgfqpoint{1.244319in}{1.213071in}}{\pgfqpoint{1.250143in}{1.207247in}}%
\pgfpathcurveto{\pgfqpoint{1.255967in}{1.201423in}}{\pgfqpoint{1.263867in}{1.198150in}}{\pgfqpoint{1.272103in}{1.198150in}}%
\pgfpathclose%
\pgfusepath{stroke,fill}%
\end{pgfscope}%
\begin{pgfscope}%
\pgfpathrectangle{\pgfqpoint{0.100000in}{0.212622in}}{\pgfqpoint{3.696000in}{3.696000in}}%
\pgfusepath{clip}%
\pgfsetbuttcap%
\pgfsetroundjoin%
\definecolor{currentfill}{rgb}{0.121569,0.466667,0.705882}%
\pgfsetfillcolor{currentfill}%
\pgfsetfillopacity{0.748034}%
\pgfsetlinewidth{1.003750pt}%
\definecolor{currentstroke}{rgb}{0.121569,0.466667,0.705882}%
\pgfsetstrokecolor{currentstroke}%
\pgfsetstrokeopacity{0.748034}%
\pgfsetdash{}{0pt}%
\pgfpathmoveto{\pgfqpoint{2.218363in}{1.621909in}}%
\pgfpathcurveto{\pgfqpoint{2.226599in}{1.621909in}}{\pgfqpoint{2.234499in}{1.625182in}}{\pgfqpoint{2.240323in}{1.631005in}}%
\pgfpathcurveto{\pgfqpoint{2.246147in}{1.636829in}}{\pgfqpoint{2.249419in}{1.644729in}}{\pgfqpoint{2.249419in}{1.652966in}}%
\pgfpathcurveto{\pgfqpoint{2.249419in}{1.661202in}}{\pgfqpoint{2.246147in}{1.669102in}}{\pgfqpoint{2.240323in}{1.674926in}}%
\pgfpathcurveto{\pgfqpoint{2.234499in}{1.680750in}}{\pgfqpoint{2.226599in}{1.684022in}}{\pgfqpoint{2.218363in}{1.684022in}}%
\pgfpathcurveto{\pgfqpoint{2.210127in}{1.684022in}}{\pgfqpoint{2.202227in}{1.680750in}}{\pgfqpoint{2.196403in}{1.674926in}}%
\pgfpathcurveto{\pgfqpoint{2.190579in}{1.669102in}}{\pgfqpoint{2.187306in}{1.661202in}}{\pgfqpoint{2.187306in}{1.652966in}}%
\pgfpathcurveto{\pgfqpoint{2.187306in}{1.644729in}}{\pgfqpoint{2.190579in}{1.636829in}}{\pgfqpoint{2.196403in}{1.631005in}}%
\pgfpathcurveto{\pgfqpoint{2.202227in}{1.625182in}}{\pgfqpoint{2.210127in}{1.621909in}}{\pgfqpoint{2.218363in}{1.621909in}}%
\pgfpathclose%
\pgfusepath{stroke,fill}%
\end{pgfscope}%
\begin{pgfscope}%
\pgfpathrectangle{\pgfqpoint{0.100000in}{0.212622in}}{\pgfqpoint{3.696000in}{3.696000in}}%
\pgfusepath{clip}%
\pgfsetbuttcap%
\pgfsetroundjoin%
\definecolor{currentfill}{rgb}{0.121569,0.466667,0.705882}%
\pgfsetfillcolor{currentfill}%
\pgfsetfillopacity{0.749243}%
\pgfsetlinewidth{1.003750pt}%
\definecolor{currentstroke}{rgb}{0.121569,0.466667,0.705882}%
\pgfsetstrokecolor{currentstroke}%
\pgfsetstrokeopacity{0.749243}%
\pgfsetdash{}{0pt}%
\pgfpathmoveto{\pgfqpoint{1.278849in}{1.199806in}}%
\pgfpathcurveto{\pgfqpoint{1.287085in}{1.199806in}}{\pgfqpoint{1.294985in}{1.203078in}}{\pgfqpoint{1.300809in}{1.208902in}}%
\pgfpathcurveto{\pgfqpoint{1.306633in}{1.214726in}}{\pgfqpoint{1.309906in}{1.222626in}}{\pgfqpoint{1.309906in}{1.230862in}}%
\pgfpathcurveto{\pgfqpoint{1.309906in}{1.239098in}}{\pgfqpoint{1.306633in}{1.246998in}}{\pgfqpoint{1.300809in}{1.252822in}}%
\pgfpathcurveto{\pgfqpoint{1.294985in}{1.258646in}}{\pgfqpoint{1.287085in}{1.261919in}}{\pgfqpoint{1.278849in}{1.261919in}}%
\pgfpathcurveto{\pgfqpoint{1.270613in}{1.261919in}}{\pgfqpoint{1.262713in}{1.258646in}}{\pgfqpoint{1.256889in}{1.252822in}}%
\pgfpathcurveto{\pgfqpoint{1.251065in}{1.246998in}}{\pgfqpoint{1.247793in}{1.239098in}}{\pgfqpoint{1.247793in}{1.230862in}}%
\pgfpathcurveto{\pgfqpoint{1.247793in}{1.222626in}}{\pgfqpoint{1.251065in}{1.214726in}}{\pgfqpoint{1.256889in}{1.208902in}}%
\pgfpathcurveto{\pgfqpoint{1.262713in}{1.203078in}}{\pgfqpoint{1.270613in}{1.199806in}}{\pgfqpoint{1.278849in}{1.199806in}}%
\pgfpathclose%
\pgfusepath{stroke,fill}%
\end{pgfscope}%
\begin{pgfscope}%
\pgfpathrectangle{\pgfqpoint{0.100000in}{0.212622in}}{\pgfqpoint{3.696000in}{3.696000in}}%
\pgfusepath{clip}%
\pgfsetbuttcap%
\pgfsetroundjoin%
\definecolor{currentfill}{rgb}{0.121569,0.466667,0.705882}%
\pgfsetfillcolor{currentfill}%
\pgfsetfillopacity{0.751162}%
\pgfsetlinewidth{1.003750pt}%
\definecolor{currentstroke}{rgb}{0.121569,0.466667,0.705882}%
\pgfsetstrokecolor{currentstroke}%
\pgfsetstrokeopacity{0.751162}%
\pgfsetdash{}{0pt}%
\pgfpathmoveto{\pgfqpoint{1.284629in}{1.201758in}}%
\pgfpathcurveto{\pgfqpoint{1.292866in}{1.201758in}}{\pgfqpoint{1.300766in}{1.205031in}}{\pgfqpoint{1.306590in}{1.210855in}}%
\pgfpathcurveto{\pgfqpoint{1.312414in}{1.216679in}}{\pgfqpoint{1.315686in}{1.224579in}}{\pgfqpoint{1.315686in}{1.232815in}}%
\pgfpathcurveto{\pgfqpoint{1.315686in}{1.241051in}}{\pgfqpoint{1.312414in}{1.248951in}}{\pgfqpoint{1.306590in}{1.254775in}}%
\pgfpathcurveto{\pgfqpoint{1.300766in}{1.260599in}}{\pgfqpoint{1.292866in}{1.263871in}}{\pgfqpoint{1.284629in}{1.263871in}}%
\pgfpathcurveto{\pgfqpoint{1.276393in}{1.263871in}}{\pgfqpoint{1.268493in}{1.260599in}}{\pgfqpoint{1.262669in}{1.254775in}}%
\pgfpathcurveto{\pgfqpoint{1.256845in}{1.248951in}}{\pgfqpoint{1.253573in}{1.241051in}}{\pgfqpoint{1.253573in}{1.232815in}}%
\pgfpathcurveto{\pgfqpoint{1.253573in}{1.224579in}}{\pgfqpoint{1.256845in}{1.216679in}}{\pgfqpoint{1.262669in}{1.210855in}}%
\pgfpathcurveto{\pgfqpoint{1.268493in}{1.205031in}}{\pgfqpoint{1.276393in}{1.201758in}}{\pgfqpoint{1.284629in}{1.201758in}}%
\pgfpathclose%
\pgfusepath{stroke,fill}%
\end{pgfscope}%
\begin{pgfscope}%
\pgfpathrectangle{\pgfqpoint{0.100000in}{0.212622in}}{\pgfqpoint{3.696000in}{3.696000in}}%
\pgfusepath{clip}%
\pgfsetbuttcap%
\pgfsetroundjoin%
\definecolor{currentfill}{rgb}{0.121569,0.466667,0.705882}%
\pgfsetfillcolor{currentfill}%
\pgfsetfillopacity{0.753393}%
\pgfsetlinewidth{1.003750pt}%
\definecolor{currentstroke}{rgb}{0.121569,0.466667,0.705882}%
\pgfsetstrokecolor{currentstroke}%
\pgfsetstrokeopacity{0.753393}%
\pgfsetdash{}{0pt}%
\pgfpathmoveto{\pgfqpoint{1.294975in}{1.199668in}}%
\pgfpathcurveto{\pgfqpoint{1.303211in}{1.199668in}}{\pgfqpoint{1.311111in}{1.202940in}}{\pgfqpoint{1.316935in}{1.208764in}}%
\pgfpathcurveto{\pgfqpoint{1.322759in}{1.214588in}}{\pgfqpoint{1.326031in}{1.222488in}}{\pgfqpoint{1.326031in}{1.230725in}}%
\pgfpathcurveto{\pgfqpoint{1.326031in}{1.238961in}}{\pgfqpoint{1.322759in}{1.246861in}}{\pgfqpoint{1.316935in}{1.252685in}}%
\pgfpathcurveto{\pgfqpoint{1.311111in}{1.258509in}}{\pgfqpoint{1.303211in}{1.261781in}}{\pgfqpoint{1.294975in}{1.261781in}}%
\pgfpathcurveto{\pgfqpoint{1.286739in}{1.261781in}}{\pgfqpoint{1.278839in}{1.258509in}}{\pgfqpoint{1.273015in}{1.252685in}}%
\pgfpathcurveto{\pgfqpoint{1.267191in}{1.246861in}}{\pgfqpoint{1.263918in}{1.238961in}}{\pgfqpoint{1.263918in}{1.230725in}}%
\pgfpathcurveto{\pgfqpoint{1.263918in}{1.222488in}}{\pgfqpoint{1.267191in}{1.214588in}}{\pgfqpoint{1.273015in}{1.208764in}}%
\pgfpathcurveto{\pgfqpoint{1.278839in}{1.202940in}}{\pgfqpoint{1.286739in}{1.199668in}}{\pgfqpoint{1.294975in}{1.199668in}}%
\pgfpathclose%
\pgfusepath{stroke,fill}%
\end{pgfscope}%
\begin{pgfscope}%
\pgfpathrectangle{\pgfqpoint{0.100000in}{0.212622in}}{\pgfqpoint{3.696000in}{3.696000in}}%
\pgfusepath{clip}%
\pgfsetbuttcap%
\pgfsetroundjoin%
\definecolor{currentfill}{rgb}{0.121569,0.466667,0.705882}%
\pgfsetfillcolor{currentfill}%
\pgfsetfillopacity{0.753738}%
\pgfsetlinewidth{1.003750pt}%
\definecolor{currentstroke}{rgb}{0.121569,0.466667,0.705882}%
\pgfsetstrokecolor{currentstroke}%
\pgfsetstrokeopacity{0.753738}%
\pgfsetdash{}{0pt}%
\pgfpathmoveto{\pgfqpoint{2.222352in}{1.603953in}}%
\pgfpathcurveto{\pgfqpoint{2.230588in}{1.603953in}}{\pgfqpoint{2.238488in}{1.607225in}}{\pgfqpoint{2.244312in}{1.613049in}}%
\pgfpathcurveto{\pgfqpoint{2.250136in}{1.618873in}}{\pgfqpoint{2.253409in}{1.626773in}}{\pgfqpoint{2.253409in}{1.635009in}}%
\pgfpathcurveto{\pgfqpoint{2.253409in}{1.643245in}}{\pgfqpoint{2.250136in}{1.651145in}}{\pgfqpoint{2.244312in}{1.656969in}}%
\pgfpathcurveto{\pgfqpoint{2.238488in}{1.662793in}}{\pgfqpoint{2.230588in}{1.666066in}}{\pgfqpoint{2.222352in}{1.666066in}}%
\pgfpathcurveto{\pgfqpoint{2.214116in}{1.666066in}}{\pgfqpoint{2.206216in}{1.662793in}}{\pgfqpoint{2.200392in}{1.656969in}}%
\pgfpathcurveto{\pgfqpoint{2.194568in}{1.651145in}}{\pgfqpoint{2.191296in}{1.643245in}}{\pgfqpoint{2.191296in}{1.635009in}}%
\pgfpathcurveto{\pgfqpoint{2.191296in}{1.626773in}}{\pgfqpoint{2.194568in}{1.618873in}}{\pgfqpoint{2.200392in}{1.613049in}}%
\pgfpathcurveto{\pgfqpoint{2.206216in}{1.607225in}}{\pgfqpoint{2.214116in}{1.603953in}}{\pgfqpoint{2.222352in}{1.603953in}}%
\pgfpathclose%
\pgfusepath{stroke,fill}%
\end{pgfscope}%
\begin{pgfscope}%
\pgfpathrectangle{\pgfqpoint{0.100000in}{0.212622in}}{\pgfqpoint{3.696000in}{3.696000in}}%
\pgfusepath{clip}%
\pgfsetbuttcap%
\pgfsetroundjoin%
\definecolor{currentfill}{rgb}{0.121569,0.466667,0.705882}%
\pgfsetfillcolor{currentfill}%
\pgfsetfillopacity{0.755465}%
\pgfsetlinewidth{1.003750pt}%
\definecolor{currentstroke}{rgb}{0.121569,0.466667,0.705882}%
\pgfsetstrokecolor{currentstroke}%
\pgfsetstrokeopacity{0.755465}%
\pgfsetdash{}{0pt}%
\pgfpathmoveto{\pgfqpoint{1.304471in}{1.198132in}}%
\pgfpathcurveto{\pgfqpoint{1.312707in}{1.198132in}}{\pgfqpoint{1.320607in}{1.201404in}}{\pgfqpoint{1.326431in}{1.207228in}}%
\pgfpathcurveto{\pgfqpoint{1.332255in}{1.213052in}}{\pgfqpoint{1.335527in}{1.220952in}}{\pgfqpoint{1.335527in}{1.229189in}}%
\pgfpathcurveto{\pgfqpoint{1.335527in}{1.237425in}}{\pgfqpoint{1.332255in}{1.245325in}}{\pgfqpoint{1.326431in}{1.251149in}}%
\pgfpathcurveto{\pgfqpoint{1.320607in}{1.256973in}}{\pgfqpoint{1.312707in}{1.260245in}}{\pgfqpoint{1.304471in}{1.260245in}}%
\pgfpathcurveto{\pgfqpoint{1.296235in}{1.260245in}}{\pgfqpoint{1.288335in}{1.256973in}}{\pgfqpoint{1.282511in}{1.251149in}}%
\pgfpathcurveto{\pgfqpoint{1.276687in}{1.245325in}}{\pgfqpoint{1.273414in}{1.237425in}}{\pgfqpoint{1.273414in}{1.229189in}}%
\pgfpathcurveto{\pgfqpoint{1.273414in}{1.220952in}}{\pgfqpoint{1.276687in}{1.213052in}}{\pgfqpoint{1.282511in}{1.207228in}}%
\pgfpathcurveto{\pgfqpoint{1.288335in}{1.201404in}}{\pgfqpoint{1.296235in}{1.198132in}}{\pgfqpoint{1.304471in}{1.198132in}}%
\pgfpathclose%
\pgfusepath{stroke,fill}%
\end{pgfscope}%
\begin{pgfscope}%
\pgfpathrectangle{\pgfqpoint{0.100000in}{0.212622in}}{\pgfqpoint{3.696000in}{3.696000in}}%
\pgfusepath{clip}%
\pgfsetbuttcap%
\pgfsetroundjoin%
\definecolor{currentfill}{rgb}{0.121569,0.466667,0.705882}%
\pgfsetfillcolor{currentfill}%
\pgfsetfillopacity{0.757153}%
\pgfsetlinewidth{1.003750pt}%
\definecolor{currentstroke}{rgb}{0.121569,0.466667,0.705882}%
\pgfsetstrokecolor{currentstroke}%
\pgfsetstrokeopacity{0.757153}%
\pgfsetdash{}{0pt}%
\pgfpathmoveto{\pgfqpoint{1.312124in}{1.196532in}}%
\pgfpathcurveto{\pgfqpoint{1.320360in}{1.196532in}}{\pgfqpoint{1.328260in}{1.199805in}}{\pgfqpoint{1.334084in}{1.205628in}}%
\pgfpathcurveto{\pgfqpoint{1.339908in}{1.211452in}}{\pgfqpoint{1.343181in}{1.219352in}}{\pgfqpoint{1.343181in}{1.227589in}}%
\pgfpathcurveto{\pgfqpoint{1.343181in}{1.235825in}}{\pgfqpoint{1.339908in}{1.243725in}}{\pgfqpoint{1.334084in}{1.249549in}}%
\pgfpathcurveto{\pgfqpoint{1.328260in}{1.255373in}}{\pgfqpoint{1.320360in}{1.258645in}}{\pgfqpoint{1.312124in}{1.258645in}}%
\pgfpathcurveto{\pgfqpoint{1.303888in}{1.258645in}}{\pgfqpoint{1.295988in}{1.255373in}}{\pgfqpoint{1.290164in}{1.249549in}}%
\pgfpathcurveto{\pgfqpoint{1.284340in}{1.243725in}}{\pgfqpoint{1.281068in}{1.235825in}}{\pgfqpoint{1.281068in}{1.227589in}}%
\pgfpathcurveto{\pgfqpoint{1.281068in}{1.219352in}}{\pgfqpoint{1.284340in}{1.211452in}}{\pgfqpoint{1.290164in}{1.205628in}}%
\pgfpathcurveto{\pgfqpoint{1.295988in}{1.199805in}}{\pgfqpoint{1.303888in}{1.196532in}}{\pgfqpoint{1.312124in}{1.196532in}}%
\pgfpathclose%
\pgfusepath{stroke,fill}%
\end{pgfscope}%
\begin{pgfscope}%
\pgfpathrectangle{\pgfqpoint{0.100000in}{0.212622in}}{\pgfqpoint{3.696000in}{3.696000in}}%
\pgfusepath{clip}%
\pgfsetbuttcap%
\pgfsetroundjoin%
\definecolor{currentfill}{rgb}{0.121569,0.466667,0.705882}%
\pgfsetfillcolor{currentfill}%
\pgfsetfillopacity{0.759372}%
\pgfsetlinewidth{1.003750pt}%
\definecolor{currentstroke}{rgb}{0.121569,0.466667,0.705882}%
\pgfsetstrokecolor{currentstroke}%
\pgfsetstrokeopacity{0.759372}%
\pgfsetdash{}{0pt}%
\pgfpathmoveto{\pgfqpoint{2.227212in}{1.585693in}}%
\pgfpathcurveto{\pgfqpoint{2.235449in}{1.585693in}}{\pgfqpoint{2.243349in}{1.588965in}}{\pgfqpoint{2.249173in}{1.594789in}}%
\pgfpathcurveto{\pgfqpoint{2.254997in}{1.600613in}}{\pgfqpoint{2.258269in}{1.608513in}}{\pgfqpoint{2.258269in}{1.616750in}}%
\pgfpathcurveto{\pgfqpoint{2.258269in}{1.624986in}}{\pgfqpoint{2.254997in}{1.632886in}}{\pgfqpoint{2.249173in}{1.638710in}}%
\pgfpathcurveto{\pgfqpoint{2.243349in}{1.644534in}}{\pgfqpoint{2.235449in}{1.647806in}}{\pgfqpoint{2.227212in}{1.647806in}}%
\pgfpathcurveto{\pgfqpoint{2.218976in}{1.647806in}}{\pgfqpoint{2.211076in}{1.644534in}}{\pgfqpoint{2.205252in}{1.638710in}}%
\pgfpathcurveto{\pgfqpoint{2.199428in}{1.632886in}}{\pgfqpoint{2.196156in}{1.624986in}}{\pgfqpoint{2.196156in}{1.616750in}}%
\pgfpathcurveto{\pgfqpoint{2.196156in}{1.608513in}}{\pgfqpoint{2.199428in}{1.600613in}}{\pgfqpoint{2.205252in}{1.594789in}}%
\pgfpathcurveto{\pgfqpoint{2.211076in}{1.588965in}}{\pgfqpoint{2.218976in}{1.585693in}}{\pgfqpoint{2.227212in}{1.585693in}}%
\pgfpathclose%
\pgfusepath{stroke,fill}%
\end{pgfscope}%
\begin{pgfscope}%
\pgfpathrectangle{\pgfqpoint{0.100000in}{0.212622in}}{\pgfqpoint{3.696000in}{3.696000in}}%
\pgfusepath{clip}%
\pgfsetbuttcap%
\pgfsetroundjoin%
\definecolor{currentfill}{rgb}{0.121569,0.466667,0.705882}%
\pgfsetfillcolor{currentfill}%
\pgfsetfillopacity{0.760396}%
\pgfsetlinewidth{1.003750pt}%
\definecolor{currentstroke}{rgb}{0.121569,0.466667,0.705882}%
\pgfsetstrokecolor{currentstroke}%
\pgfsetstrokeopacity{0.760396}%
\pgfsetdash{}{0pt}%
\pgfpathmoveto{\pgfqpoint{1.325965in}{1.193996in}}%
\pgfpathcurveto{\pgfqpoint{1.334201in}{1.193996in}}{\pgfqpoint{1.342101in}{1.197268in}}{\pgfqpoint{1.347925in}{1.203092in}}%
\pgfpathcurveto{\pgfqpoint{1.353749in}{1.208916in}}{\pgfqpoint{1.357021in}{1.216816in}}{\pgfqpoint{1.357021in}{1.225052in}}%
\pgfpathcurveto{\pgfqpoint{1.357021in}{1.233289in}}{\pgfqpoint{1.353749in}{1.241189in}}{\pgfqpoint{1.347925in}{1.247013in}}%
\pgfpathcurveto{\pgfqpoint{1.342101in}{1.252837in}}{\pgfqpoint{1.334201in}{1.256109in}}{\pgfqpoint{1.325965in}{1.256109in}}%
\pgfpathcurveto{\pgfqpoint{1.317728in}{1.256109in}}{\pgfqpoint{1.309828in}{1.252837in}}{\pgfqpoint{1.304004in}{1.247013in}}%
\pgfpathcurveto{\pgfqpoint{1.298181in}{1.241189in}}{\pgfqpoint{1.294908in}{1.233289in}}{\pgfqpoint{1.294908in}{1.225052in}}%
\pgfpathcurveto{\pgfqpoint{1.294908in}{1.216816in}}{\pgfqpoint{1.298181in}{1.208916in}}{\pgfqpoint{1.304004in}{1.203092in}}%
\pgfpathcurveto{\pgfqpoint{1.309828in}{1.197268in}}{\pgfqpoint{1.317728in}{1.193996in}}{\pgfqpoint{1.325965in}{1.193996in}}%
\pgfpathclose%
\pgfusepath{stroke,fill}%
\end{pgfscope}%
\begin{pgfscope}%
\pgfpathrectangle{\pgfqpoint{0.100000in}{0.212622in}}{\pgfqpoint{3.696000in}{3.696000in}}%
\pgfusepath{clip}%
\pgfsetbuttcap%
\pgfsetroundjoin%
\definecolor{currentfill}{rgb}{0.121569,0.466667,0.705882}%
\pgfsetfillcolor{currentfill}%
\pgfsetfillopacity{0.762208}%
\pgfsetlinewidth{1.003750pt}%
\definecolor{currentstroke}{rgb}{0.121569,0.466667,0.705882}%
\pgfsetstrokecolor{currentstroke}%
\pgfsetstrokeopacity{0.762208}%
\pgfsetdash{}{0pt}%
\pgfpathmoveto{\pgfqpoint{2.230278in}{1.575107in}}%
\pgfpathcurveto{\pgfqpoint{2.238514in}{1.575107in}}{\pgfqpoint{2.246414in}{1.578379in}}{\pgfqpoint{2.252238in}{1.584203in}}%
\pgfpathcurveto{\pgfqpoint{2.258062in}{1.590027in}}{\pgfqpoint{2.261334in}{1.597927in}}{\pgfqpoint{2.261334in}{1.606163in}}%
\pgfpathcurveto{\pgfqpoint{2.261334in}{1.614400in}}{\pgfqpoint{2.258062in}{1.622300in}}{\pgfqpoint{2.252238in}{1.628124in}}%
\pgfpathcurveto{\pgfqpoint{2.246414in}{1.633948in}}{\pgfqpoint{2.238514in}{1.637220in}}{\pgfqpoint{2.230278in}{1.637220in}}%
\pgfpathcurveto{\pgfqpoint{2.222041in}{1.637220in}}{\pgfqpoint{2.214141in}{1.633948in}}{\pgfqpoint{2.208317in}{1.628124in}}%
\pgfpathcurveto{\pgfqpoint{2.202494in}{1.622300in}}{\pgfqpoint{2.199221in}{1.614400in}}{\pgfqpoint{2.199221in}{1.606163in}}%
\pgfpathcurveto{\pgfqpoint{2.199221in}{1.597927in}}{\pgfqpoint{2.202494in}{1.590027in}}{\pgfqpoint{2.208317in}{1.584203in}}%
\pgfpathcurveto{\pgfqpoint{2.214141in}{1.578379in}}{\pgfqpoint{2.222041in}{1.575107in}}{\pgfqpoint{2.230278in}{1.575107in}}%
\pgfpathclose%
\pgfusepath{stroke,fill}%
\end{pgfscope}%
\begin{pgfscope}%
\pgfpathrectangle{\pgfqpoint{0.100000in}{0.212622in}}{\pgfqpoint{3.696000in}{3.696000in}}%
\pgfusepath{clip}%
\pgfsetbuttcap%
\pgfsetroundjoin%
\definecolor{currentfill}{rgb}{0.121569,0.466667,0.705882}%
\pgfsetfillcolor{currentfill}%
\pgfsetfillopacity{0.763030}%
\pgfsetlinewidth{1.003750pt}%
\definecolor{currentstroke}{rgb}{0.121569,0.466667,0.705882}%
\pgfsetstrokecolor{currentstroke}%
\pgfsetstrokeopacity{0.763030}%
\pgfsetdash{}{0pt}%
\pgfpathmoveto{\pgfqpoint{1.338157in}{1.192310in}}%
\pgfpathcurveto{\pgfqpoint{1.346393in}{1.192310in}}{\pgfqpoint{1.354293in}{1.195582in}}{\pgfqpoint{1.360117in}{1.201406in}}%
\pgfpathcurveto{\pgfqpoint{1.365941in}{1.207230in}}{\pgfqpoint{1.369214in}{1.215130in}}{\pgfqpoint{1.369214in}{1.223367in}}%
\pgfpathcurveto{\pgfqpoint{1.369214in}{1.231603in}}{\pgfqpoint{1.365941in}{1.239503in}}{\pgfqpoint{1.360117in}{1.245327in}}%
\pgfpathcurveto{\pgfqpoint{1.354293in}{1.251151in}}{\pgfqpoint{1.346393in}{1.254423in}}{\pgfqpoint{1.338157in}{1.254423in}}%
\pgfpathcurveto{\pgfqpoint{1.329921in}{1.254423in}}{\pgfqpoint{1.322021in}{1.251151in}}{\pgfqpoint{1.316197in}{1.245327in}}%
\pgfpathcurveto{\pgfqpoint{1.310373in}{1.239503in}}{\pgfqpoint{1.307101in}{1.231603in}}{\pgfqpoint{1.307101in}{1.223367in}}%
\pgfpathcurveto{\pgfqpoint{1.307101in}{1.215130in}}{\pgfqpoint{1.310373in}{1.207230in}}{\pgfqpoint{1.316197in}{1.201406in}}%
\pgfpathcurveto{\pgfqpoint{1.322021in}{1.195582in}}{\pgfqpoint{1.329921in}{1.192310in}}{\pgfqpoint{1.338157in}{1.192310in}}%
\pgfpathclose%
\pgfusepath{stroke,fill}%
\end{pgfscope}%
\begin{pgfscope}%
\pgfpathrectangle{\pgfqpoint{0.100000in}{0.212622in}}{\pgfqpoint{3.696000in}{3.696000in}}%
\pgfusepath{clip}%
\pgfsetbuttcap%
\pgfsetroundjoin%
\definecolor{currentfill}{rgb}{0.121569,0.466667,0.705882}%
\pgfsetfillcolor{currentfill}%
\pgfsetfillopacity{0.765566}%
\pgfsetlinewidth{1.003750pt}%
\definecolor{currentstroke}{rgb}{0.121569,0.466667,0.705882}%
\pgfsetstrokecolor{currentstroke}%
\pgfsetstrokeopacity{0.765566}%
\pgfsetdash{}{0pt}%
\pgfpathmoveto{\pgfqpoint{2.232659in}{1.563944in}}%
\pgfpathcurveto{\pgfqpoint{2.240895in}{1.563944in}}{\pgfqpoint{2.248795in}{1.567216in}}{\pgfqpoint{2.254619in}{1.573040in}}%
\pgfpathcurveto{\pgfqpoint{2.260443in}{1.578864in}}{\pgfqpoint{2.263716in}{1.586764in}}{\pgfqpoint{2.263716in}{1.595000in}}%
\pgfpathcurveto{\pgfqpoint{2.263716in}{1.603236in}}{\pgfqpoint{2.260443in}{1.611137in}}{\pgfqpoint{2.254619in}{1.616960in}}%
\pgfpathcurveto{\pgfqpoint{2.248795in}{1.622784in}}{\pgfqpoint{2.240895in}{1.626057in}}{\pgfqpoint{2.232659in}{1.626057in}}%
\pgfpathcurveto{\pgfqpoint{2.224423in}{1.626057in}}{\pgfqpoint{2.216523in}{1.622784in}}{\pgfqpoint{2.210699in}{1.616960in}}%
\pgfpathcurveto{\pgfqpoint{2.204875in}{1.611137in}}{\pgfqpoint{2.201603in}{1.603236in}}{\pgfqpoint{2.201603in}{1.595000in}}%
\pgfpathcurveto{\pgfqpoint{2.201603in}{1.586764in}}{\pgfqpoint{2.204875in}{1.578864in}}{\pgfqpoint{2.210699in}{1.573040in}}%
\pgfpathcurveto{\pgfqpoint{2.216523in}{1.567216in}}{\pgfqpoint{2.224423in}{1.563944in}}{\pgfqpoint{2.232659in}{1.563944in}}%
\pgfpathclose%
\pgfusepath{stroke,fill}%
\end{pgfscope}%
\begin{pgfscope}%
\pgfpathrectangle{\pgfqpoint{0.100000in}{0.212622in}}{\pgfqpoint{3.696000in}{3.696000in}}%
\pgfusepath{clip}%
\pgfsetbuttcap%
\pgfsetroundjoin%
\definecolor{currentfill}{rgb}{0.121569,0.466667,0.705882}%
\pgfsetfillcolor{currentfill}%
\pgfsetfillopacity{0.765733}%
\pgfsetlinewidth{1.003750pt}%
\definecolor{currentstroke}{rgb}{0.121569,0.466667,0.705882}%
\pgfsetstrokecolor{currentstroke}%
\pgfsetstrokeopacity{0.765733}%
\pgfsetdash{}{0pt}%
\pgfpathmoveto{\pgfqpoint{1.349216in}{1.191105in}}%
\pgfpathcurveto{\pgfqpoint{1.357452in}{1.191105in}}{\pgfqpoint{1.365352in}{1.194377in}}{\pgfqpoint{1.371176in}{1.200201in}}%
\pgfpathcurveto{\pgfqpoint{1.377000in}{1.206025in}}{\pgfqpoint{1.380272in}{1.213925in}}{\pgfqpoint{1.380272in}{1.222162in}}%
\pgfpathcurveto{\pgfqpoint{1.380272in}{1.230398in}}{\pgfqpoint{1.377000in}{1.238298in}}{\pgfqpoint{1.371176in}{1.244122in}}%
\pgfpathcurveto{\pgfqpoint{1.365352in}{1.249946in}}{\pgfqpoint{1.357452in}{1.253218in}}{\pgfqpoint{1.349216in}{1.253218in}}%
\pgfpathcurveto{\pgfqpoint{1.340980in}{1.253218in}}{\pgfqpoint{1.333080in}{1.249946in}}{\pgfqpoint{1.327256in}{1.244122in}}%
\pgfpathcurveto{\pgfqpoint{1.321432in}{1.238298in}}{\pgfqpoint{1.318159in}{1.230398in}}{\pgfqpoint{1.318159in}{1.222162in}}%
\pgfpathcurveto{\pgfqpoint{1.318159in}{1.213925in}}{\pgfqpoint{1.321432in}{1.206025in}}{\pgfqpoint{1.327256in}{1.200201in}}%
\pgfpathcurveto{\pgfqpoint{1.333080in}{1.194377in}}{\pgfqpoint{1.340980in}{1.191105in}}{\pgfqpoint{1.349216in}{1.191105in}}%
\pgfpathclose%
\pgfusepath{stroke,fill}%
\end{pgfscope}%
\begin{pgfscope}%
\pgfpathrectangle{\pgfqpoint{0.100000in}{0.212622in}}{\pgfqpoint{3.696000in}{3.696000in}}%
\pgfusepath{clip}%
\pgfsetbuttcap%
\pgfsetroundjoin%
\definecolor{currentfill}{rgb}{0.121569,0.466667,0.705882}%
\pgfsetfillcolor{currentfill}%
\pgfsetfillopacity{0.768264}%
\pgfsetlinewidth{1.003750pt}%
\definecolor{currentstroke}{rgb}{0.121569,0.466667,0.705882}%
\pgfsetstrokecolor{currentstroke}%
\pgfsetstrokeopacity{0.768264}%
\pgfsetdash{}{0pt}%
\pgfpathmoveto{\pgfqpoint{1.359549in}{1.189977in}}%
\pgfpathcurveto{\pgfqpoint{1.367785in}{1.189977in}}{\pgfqpoint{1.375685in}{1.193249in}}{\pgfqpoint{1.381509in}{1.199073in}}%
\pgfpathcurveto{\pgfqpoint{1.387333in}{1.204897in}}{\pgfqpoint{1.390605in}{1.212797in}}{\pgfqpoint{1.390605in}{1.221033in}}%
\pgfpathcurveto{\pgfqpoint{1.390605in}{1.229269in}}{\pgfqpoint{1.387333in}{1.237169in}}{\pgfqpoint{1.381509in}{1.242993in}}%
\pgfpathcurveto{\pgfqpoint{1.375685in}{1.248817in}}{\pgfqpoint{1.367785in}{1.252090in}}{\pgfqpoint{1.359549in}{1.252090in}}%
\pgfpathcurveto{\pgfqpoint{1.351313in}{1.252090in}}{\pgfqpoint{1.343413in}{1.248817in}}{\pgfqpoint{1.337589in}{1.242993in}}%
\pgfpathcurveto{\pgfqpoint{1.331765in}{1.237169in}}{\pgfqpoint{1.328492in}{1.229269in}}{\pgfqpoint{1.328492in}{1.221033in}}%
\pgfpathcurveto{\pgfqpoint{1.328492in}{1.212797in}}{\pgfqpoint{1.331765in}{1.204897in}}{\pgfqpoint{1.337589in}{1.199073in}}%
\pgfpathcurveto{\pgfqpoint{1.343413in}{1.193249in}}{\pgfqpoint{1.351313in}{1.189977in}}{\pgfqpoint{1.359549in}{1.189977in}}%
\pgfpathclose%
\pgfusepath{stroke,fill}%
\end{pgfscope}%
\begin{pgfscope}%
\pgfpathrectangle{\pgfqpoint{0.100000in}{0.212622in}}{\pgfqpoint{3.696000in}{3.696000in}}%
\pgfusepath{clip}%
\pgfsetbuttcap%
\pgfsetroundjoin%
\definecolor{currentfill}{rgb}{0.121569,0.466667,0.705882}%
\pgfsetfillcolor{currentfill}%
\pgfsetfillopacity{0.769079}%
\pgfsetlinewidth{1.003750pt}%
\definecolor{currentstroke}{rgb}{0.121569,0.466667,0.705882}%
\pgfsetstrokecolor{currentstroke}%
\pgfsetstrokeopacity{0.769079}%
\pgfsetdash{}{0pt}%
\pgfpathmoveto{\pgfqpoint{2.235480in}{1.551935in}}%
\pgfpathcurveto{\pgfqpoint{2.243716in}{1.551935in}}{\pgfqpoint{2.251616in}{1.555208in}}{\pgfqpoint{2.257440in}{1.561032in}}%
\pgfpathcurveto{\pgfqpoint{2.263264in}{1.566855in}}{\pgfqpoint{2.266536in}{1.574756in}}{\pgfqpoint{2.266536in}{1.582992in}}%
\pgfpathcurveto{\pgfqpoint{2.266536in}{1.591228in}}{\pgfqpoint{2.263264in}{1.599128in}}{\pgfqpoint{2.257440in}{1.604952in}}%
\pgfpathcurveto{\pgfqpoint{2.251616in}{1.610776in}}{\pgfqpoint{2.243716in}{1.614048in}}{\pgfqpoint{2.235480in}{1.614048in}}%
\pgfpathcurveto{\pgfqpoint{2.227243in}{1.614048in}}{\pgfqpoint{2.219343in}{1.610776in}}{\pgfqpoint{2.213519in}{1.604952in}}%
\pgfpathcurveto{\pgfqpoint{2.207696in}{1.599128in}}{\pgfqpoint{2.204423in}{1.591228in}}{\pgfqpoint{2.204423in}{1.582992in}}%
\pgfpathcurveto{\pgfqpoint{2.204423in}{1.574756in}}{\pgfqpoint{2.207696in}{1.566855in}}{\pgfqpoint{2.213519in}{1.561032in}}%
\pgfpathcurveto{\pgfqpoint{2.219343in}{1.555208in}}{\pgfqpoint{2.227243in}{1.551935in}}{\pgfqpoint{2.235480in}{1.551935in}}%
\pgfpathclose%
\pgfusepath{stroke,fill}%
\end{pgfscope}%
\begin{pgfscope}%
\pgfpathrectangle{\pgfqpoint{0.100000in}{0.212622in}}{\pgfqpoint{3.696000in}{3.696000in}}%
\pgfusepath{clip}%
\pgfsetbuttcap%
\pgfsetroundjoin%
\definecolor{currentfill}{rgb}{0.121569,0.466667,0.705882}%
\pgfsetfillcolor{currentfill}%
\pgfsetfillopacity{0.770503}%
\pgfsetlinewidth{1.003750pt}%
\definecolor{currentstroke}{rgb}{0.121569,0.466667,0.705882}%
\pgfsetstrokecolor{currentstroke}%
\pgfsetstrokeopacity{0.770503}%
\pgfsetdash{}{0pt}%
\pgfpathmoveto{\pgfqpoint{1.369478in}{1.188519in}}%
\pgfpathcurveto{\pgfqpoint{1.377714in}{1.188519in}}{\pgfqpoint{1.385614in}{1.191791in}}{\pgfqpoint{1.391438in}{1.197615in}}%
\pgfpathcurveto{\pgfqpoint{1.397262in}{1.203439in}}{\pgfqpoint{1.400534in}{1.211339in}}{\pgfqpoint{1.400534in}{1.219575in}}%
\pgfpathcurveto{\pgfqpoint{1.400534in}{1.227812in}}{\pgfqpoint{1.397262in}{1.235712in}}{\pgfqpoint{1.391438in}{1.241535in}}%
\pgfpathcurveto{\pgfqpoint{1.385614in}{1.247359in}}{\pgfqpoint{1.377714in}{1.250632in}}{\pgfqpoint{1.369478in}{1.250632in}}%
\pgfpathcurveto{\pgfqpoint{1.361242in}{1.250632in}}{\pgfqpoint{1.353342in}{1.247359in}}{\pgfqpoint{1.347518in}{1.241535in}}%
\pgfpathcurveto{\pgfqpoint{1.341694in}{1.235712in}}{\pgfqpoint{1.338421in}{1.227812in}}{\pgfqpoint{1.338421in}{1.219575in}}%
\pgfpathcurveto{\pgfqpoint{1.338421in}{1.211339in}}{\pgfqpoint{1.341694in}{1.203439in}}{\pgfqpoint{1.347518in}{1.197615in}}%
\pgfpathcurveto{\pgfqpoint{1.353342in}{1.191791in}}{\pgfqpoint{1.361242in}{1.188519in}}{\pgfqpoint{1.369478in}{1.188519in}}%
\pgfpathclose%
\pgfusepath{stroke,fill}%
\end{pgfscope}%
\begin{pgfscope}%
\pgfpathrectangle{\pgfqpoint{0.100000in}{0.212622in}}{\pgfqpoint{3.696000in}{3.696000in}}%
\pgfusepath{clip}%
\pgfsetbuttcap%
\pgfsetroundjoin%
\definecolor{currentfill}{rgb}{0.121569,0.466667,0.705882}%
\pgfsetfillcolor{currentfill}%
\pgfsetfillopacity{0.772466}%
\pgfsetlinewidth{1.003750pt}%
\definecolor{currentstroke}{rgb}{0.121569,0.466667,0.705882}%
\pgfsetstrokecolor{currentstroke}%
\pgfsetstrokeopacity{0.772466}%
\pgfsetdash{}{0pt}%
\pgfpathmoveto{\pgfqpoint{2.239239in}{1.538818in}}%
\pgfpathcurveto{\pgfqpoint{2.247475in}{1.538818in}}{\pgfqpoint{2.255375in}{1.542090in}}{\pgfqpoint{2.261199in}{1.547914in}}%
\pgfpathcurveto{\pgfqpoint{2.267023in}{1.553738in}}{\pgfqpoint{2.270296in}{1.561638in}}{\pgfqpoint{2.270296in}{1.569874in}}%
\pgfpathcurveto{\pgfqpoint{2.270296in}{1.578111in}}{\pgfqpoint{2.267023in}{1.586011in}}{\pgfqpoint{2.261199in}{1.591835in}}%
\pgfpathcurveto{\pgfqpoint{2.255375in}{1.597659in}}{\pgfqpoint{2.247475in}{1.600931in}}{\pgfqpoint{2.239239in}{1.600931in}}%
\pgfpathcurveto{\pgfqpoint{2.231003in}{1.600931in}}{\pgfqpoint{2.223103in}{1.597659in}}{\pgfqpoint{2.217279in}{1.591835in}}%
\pgfpathcurveto{\pgfqpoint{2.211455in}{1.586011in}}{\pgfqpoint{2.208183in}{1.578111in}}{\pgfqpoint{2.208183in}{1.569874in}}%
\pgfpathcurveto{\pgfqpoint{2.208183in}{1.561638in}}{\pgfqpoint{2.211455in}{1.553738in}}{\pgfqpoint{2.217279in}{1.547914in}}%
\pgfpathcurveto{\pgfqpoint{2.223103in}{1.542090in}}{\pgfqpoint{2.231003in}{1.538818in}}{\pgfqpoint{2.239239in}{1.538818in}}%
\pgfpathclose%
\pgfusepath{stroke,fill}%
\end{pgfscope}%
\begin{pgfscope}%
\pgfpathrectangle{\pgfqpoint{0.100000in}{0.212622in}}{\pgfqpoint{3.696000in}{3.696000in}}%
\pgfusepath{clip}%
\pgfsetbuttcap%
\pgfsetroundjoin%
\definecolor{currentfill}{rgb}{0.121569,0.466667,0.705882}%
\pgfsetfillcolor{currentfill}%
\pgfsetfillopacity{0.772783}%
\pgfsetlinewidth{1.003750pt}%
\definecolor{currentstroke}{rgb}{0.121569,0.466667,0.705882}%
\pgfsetstrokecolor{currentstroke}%
\pgfsetstrokeopacity{0.772783}%
\pgfsetdash{}{0pt}%
\pgfpathmoveto{\pgfqpoint{1.378761in}{1.187786in}}%
\pgfpathcurveto{\pgfqpoint{1.386998in}{1.187786in}}{\pgfqpoint{1.394898in}{1.191058in}}{\pgfqpoint{1.400722in}{1.196882in}}%
\pgfpathcurveto{\pgfqpoint{1.406545in}{1.202706in}}{\pgfqpoint{1.409818in}{1.210606in}}{\pgfqpoint{1.409818in}{1.218843in}}%
\pgfpathcurveto{\pgfqpoint{1.409818in}{1.227079in}}{\pgfqpoint{1.406545in}{1.234979in}}{\pgfqpoint{1.400722in}{1.240803in}}%
\pgfpathcurveto{\pgfqpoint{1.394898in}{1.246627in}}{\pgfqpoint{1.386998in}{1.249899in}}{\pgfqpoint{1.378761in}{1.249899in}}%
\pgfpathcurveto{\pgfqpoint{1.370525in}{1.249899in}}{\pgfqpoint{1.362625in}{1.246627in}}{\pgfqpoint{1.356801in}{1.240803in}}%
\pgfpathcurveto{\pgfqpoint{1.350977in}{1.234979in}}{\pgfqpoint{1.347705in}{1.227079in}}{\pgfqpoint{1.347705in}{1.218843in}}%
\pgfpathcurveto{\pgfqpoint{1.347705in}{1.210606in}}{\pgfqpoint{1.350977in}{1.202706in}}{\pgfqpoint{1.356801in}{1.196882in}}%
\pgfpathcurveto{\pgfqpoint{1.362625in}{1.191058in}}{\pgfqpoint{1.370525in}{1.187786in}}{\pgfqpoint{1.378761in}{1.187786in}}%
\pgfpathclose%
\pgfusepath{stroke,fill}%
\end{pgfscope}%
\begin{pgfscope}%
\pgfpathrectangle{\pgfqpoint{0.100000in}{0.212622in}}{\pgfqpoint{3.696000in}{3.696000in}}%
\pgfusepath{clip}%
\pgfsetbuttcap%
\pgfsetroundjoin%
\definecolor{currentfill}{rgb}{0.121569,0.466667,0.705882}%
\pgfsetfillcolor{currentfill}%
\pgfsetfillopacity{0.774762}%
\pgfsetlinewidth{1.003750pt}%
\definecolor{currentstroke}{rgb}{0.121569,0.466667,0.705882}%
\pgfsetstrokecolor{currentstroke}%
\pgfsetstrokeopacity{0.774762}%
\pgfsetdash{}{0pt}%
\pgfpathmoveto{\pgfqpoint{1.386417in}{1.187104in}}%
\pgfpathcurveto{\pgfqpoint{1.394653in}{1.187104in}}{\pgfqpoint{1.402553in}{1.190376in}}{\pgfqpoint{1.408377in}{1.196200in}}%
\pgfpathcurveto{\pgfqpoint{1.414201in}{1.202024in}}{\pgfqpoint{1.417473in}{1.209924in}}{\pgfqpoint{1.417473in}{1.218161in}}%
\pgfpathcurveto{\pgfqpoint{1.417473in}{1.226397in}}{\pgfqpoint{1.414201in}{1.234297in}}{\pgfqpoint{1.408377in}{1.240121in}}%
\pgfpathcurveto{\pgfqpoint{1.402553in}{1.245945in}}{\pgfqpoint{1.394653in}{1.249217in}}{\pgfqpoint{1.386417in}{1.249217in}}%
\pgfpathcurveto{\pgfqpoint{1.378180in}{1.249217in}}{\pgfqpoint{1.370280in}{1.245945in}}{\pgfqpoint{1.364456in}{1.240121in}}%
\pgfpathcurveto{\pgfqpoint{1.358633in}{1.234297in}}{\pgfqpoint{1.355360in}{1.226397in}}{\pgfqpoint{1.355360in}{1.218161in}}%
\pgfpathcurveto{\pgfqpoint{1.355360in}{1.209924in}}{\pgfqpoint{1.358633in}{1.202024in}}{\pgfqpoint{1.364456in}{1.196200in}}%
\pgfpathcurveto{\pgfqpoint{1.370280in}{1.190376in}}{\pgfqpoint{1.378180in}{1.187104in}}{\pgfqpoint{1.386417in}{1.187104in}}%
\pgfpathclose%
\pgfusepath{stroke,fill}%
\end{pgfscope}%
\begin{pgfscope}%
\pgfpathrectangle{\pgfqpoint{0.100000in}{0.212622in}}{\pgfqpoint{3.696000in}{3.696000in}}%
\pgfusepath{clip}%
\pgfsetbuttcap%
\pgfsetroundjoin%
\definecolor{currentfill}{rgb}{0.121569,0.466667,0.705882}%
\pgfsetfillcolor{currentfill}%
\pgfsetfillopacity{0.776307}%
\pgfsetlinewidth{1.003750pt}%
\definecolor{currentstroke}{rgb}{0.121569,0.466667,0.705882}%
\pgfsetstrokecolor{currentstroke}%
\pgfsetstrokeopacity{0.776307}%
\pgfsetdash{}{0pt}%
\pgfpathmoveto{\pgfqpoint{1.393621in}{1.185848in}}%
\pgfpathcurveto{\pgfqpoint{1.401858in}{1.185848in}}{\pgfqpoint{1.409758in}{1.189120in}}{\pgfqpoint{1.415582in}{1.194944in}}%
\pgfpathcurveto{\pgfqpoint{1.421405in}{1.200768in}}{\pgfqpoint{1.424678in}{1.208668in}}{\pgfqpoint{1.424678in}{1.216904in}}%
\pgfpathcurveto{\pgfqpoint{1.424678in}{1.225141in}}{\pgfqpoint{1.421405in}{1.233041in}}{\pgfqpoint{1.415582in}{1.238865in}}%
\pgfpathcurveto{\pgfqpoint{1.409758in}{1.244688in}}{\pgfqpoint{1.401858in}{1.247961in}}{\pgfqpoint{1.393621in}{1.247961in}}%
\pgfpathcurveto{\pgfqpoint{1.385385in}{1.247961in}}{\pgfqpoint{1.377485in}{1.244688in}}{\pgfqpoint{1.371661in}{1.238865in}}%
\pgfpathcurveto{\pgfqpoint{1.365837in}{1.233041in}}{\pgfqpoint{1.362565in}{1.225141in}}{\pgfqpoint{1.362565in}{1.216904in}}%
\pgfpathcurveto{\pgfqpoint{1.362565in}{1.208668in}}{\pgfqpoint{1.365837in}{1.200768in}}{\pgfqpoint{1.371661in}{1.194944in}}%
\pgfpathcurveto{\pgfqpoint{1.377485in}{1.189120in}}{\pgfqpoint{1.385385in}{1.185848in}}{\pgfqpoint{1.393621in}{1.185848in}}%
\pgfpathclose%
\pgfusepath{stroke,fill}%
\end{pgfscope}%
\begin{pgfscope}%
\pgfpathrectangle{\pgfqpoint{0.100000in}{0.212622in}}{\pgfqpoint{3.696000in}{3.696000in}}%
\pgfusepath{clip}%
\pgfsetbuttcap%
\pgfsetroundjoin%
\definecolor{currentfill}{rgb}{0.121569,0.466667,0.705882}%
\pgfsetfillcolor{currentfill}%
\pgfsetfillopacity{0.776647}%
\pgfsetlinewidth{1.003750pt}%
\definecolor{currentstroke}{rgb}{0.121569,0.466667,0.705882}%
\pgfsetstrokecolor{currentstroke}%
\pgfsetstrokeopacity{0.776647}%
\pgfsetdash{}{0pt}%
\pgfpathmoveto{\pgfqpoint{2.242705in}{1.524434in}}%
\pgfpathcurveto{\pgfqpoint{2.250942in}{1.524434in}}{\pgfqpoint{2.258842in}{1.527707in}}{\pgfqpoint{2.264666in}{1.533530in}}%
\pgfpathcurveto{\pgfqpoint{2.270490in}{1.539354in}}{\pgfqpoint{2.273762in}{1.547254in}}{\pgfqpoint{2.273762in}{1.555491in}}%
\pgfpathcurveto{\pgfqpoint{2.273762in}{1.563727in}}{\pgfqpoint{2.270490in}{1.571627in}}{\pgfqpoint{2.264666in}{1.577451in}}%
\pgfpathcurveto{\pgfqpoint{2.258842in}{1.583275in}}{\pgfqpoint{2.250942in}{1.586547in}}{\pgfqpoint{2.242705in}{1.586547in}}%
\pgfpathcurveto{\pgfqpoint{2.234469in}{1.586547in}}{\pgfqpoint{2.226569in}{1.583275in}}{\pgfqpoint{2.220745in}{1.577451in}}%
\pgfpathcurveto{\pgfqpoint{2.214921in}{1.571627in}}{\pgfqpoint{2.211649in}{1.563727in}}{\pgfqpoint{2.211649in}{1.555491in}}%
\pgfpathcurveto{\pgfqpoint{2.211649in}{1.547254in}}{\pgfqpoint{2.214921in}{1.539354in}}{\pgfqpoint{2.220745in}{1.533530in}}%
\pgfpathcurveto{\pgfqpoint{2.226569in}{1.527707in}}{\pgfqpoint{2.234469in}{1.524434in}}{\pgfqpoint{2.242705in}{1.524434in}}%
\pgfpathclose%
\pgfusepath{stroke,fill}%
\end{pgfscope}%
\begin{pgfscope}%
\pgfpathrectangle{\pgfqpoint{0.100000in}{0.212622in}}{\pgfqpoint{3.696000in}{3.696000in}}%
\pgfusepath{clip}%
\pgfsetbuttcap%
\pgfsetroundjoin%
\definecolor{currentfill}{rgb}{0.121569,0.466667,0.705882}%
\pgfsetfillcolor{currentfill}%
\pgfsetfillopacity{0.777856}%
\pgfsetlinewidth{1.003750pt}%
\definecolor{currentstroke}{rgb}{0.121569,0.466667,0.705882}%
\pgfsetstrokecolor{currentstroke}%
\pgfsetstrokeopacity{0.777856}%
\pgfsetdash{}{0pt}%
\pgfpathmoveto{\pgfqpoint{1.400390in}{1.184368in}}%
\pgfpathcurveto{\pgfqpoint{1.408626in}{1.184368in}}{\pgfqpoint{1.416527in}{1.187641in}}{\pgfqpoint{1.422350in}{1.193464in}}%
\pgfpathcurveto{\pgfqpoint{1.428174in}{1.199288in}}{\pgfqpoint{1.431447in}{1.207188in}}{\pgfqpoint{1.431447in}{1.215425in}}%
\pgfpathcurveto{\pgfqpoint{1.431447in}{1.223661in}}{\pgfqpoint{1.428174in}{1.231561in}}{\pgfqpoint{1.422350in}{1.237385in}}%
\pgfpathcurveto{\pgfqpoint{1.416527in}{1.243209in}}{\pgfqpoint{1.408626in}{1.246481in}}{\pgfqpoint{1.400390in}{1.246481in}}%
\pgfpathcurveto{\pgfqpoint{1.392154in}{1.246481in}}{\pgfqpoint{1.384254in}{1.243209in}}{\pgfqpoint{1.378430in}{1.237385in}}%
\pgfpathcurveto{\pgfqpoint{1.372606in}{1.231561in}}{\pgfqpoint{1.369334in}{1.223661in}}{\pgfqpoint{1.369334in}{1.215425in}}%
\pgfpathcurveto{\pgfqpoint{1.369334in}{1.207188in}}{\pgfqpoint{1.372606in}{1.199288in}}{\pgfqpoint{1.378430in}{1.193464in}}%
\pgfpathcurveto{\pgfqpoint{1.384254in}{1.187641in}}{\pgfqpoint{1.392154in}{1.184368in}}{\pgfqpoint{1.400390in}{1.184368in}}%
\pgfpathclose%
\pgfusepath{stroke,fill}%
\end{pgfscope}%
\begin{pgfscope}%
\pgfpathrectangle{\pgfqpoint{0.100000in}{0.212622in}}{\pgfqpoint{3.696000in}{3.696000in}}%
\pgfusepath{clip}%
\pgfsetbuttcap%
\pgfsetroundjoin%
\definecolor{currentfill}{rgb}{0.121569,0.466667,0.705882}%
\pgfsetfillcolor{currentfill}%
\pgfsetfillopacity{0.779018}%
\pgfsetlinewidth{1.003750pt}%
\definecolor{currentstroke}{rgb}{0.121569,0.466667,0.705882}%
\pgfsetstrokecolor{currentstroke}%
\pgfsetstrokeopacity{0.779018}%
\pgfsetdash{}{0pt}%
\pgfpathmoveto{\pgfqpoint{1.405509in}{1.183113in}}%
\pgfpathcurveto{\pgfqpoint{1.413745in}{1.183113in}}{\pgfqpoint{1.421645in}{1.186385in}}{\pgfqpoint{1.427469in}{1.192209in}}%
\pgfpathcurveto{\pgfqpoint{1.433293in}{1.198033in}}{\pgfqpoint{1.436565in}{1.205933in}}{\pgfqpoint{1.436565in}{1.214170in}}%
\pgfpathcurveto{\pgfqpoint{1.436565in}{1.222406in}}{\pgfqpoint{1.433293in}{1.230306in}}{\pgfqpoint{1.427469in}{1.236130in}}%
\pgfpathcurveto{\pgfqpoint{1.421645in}{1.241954in}}{\pgfqpoint{1.413745in}{1.245226in}}{\pgfqpoint{1.405509in}{1.245226in}}%
\pgfpathcurveto{\pgfqpoint{1.397273in}{1.245226in}}{\pgfqpoint{1.389373in}{1.241954in}}{\pgfqpoint{1.383549in}{1.236130in}}%
\pgfpathcurveto{\pgfqpoint{1.377725in}{1.230306in}}{\pgfqpoint{1.374452in}{1.222406in}}{\pgfqpoint{1.374452in}{1.214170in}}%
\pgfpathcurveto{\pgfqpoint{1.374452in}{1.205933in}}{\pgfqpoint{1.377725in}{1.198033in}}{\pgfqpoint{1.383549in}{1.192209in}}%
\pgfpathcurveto{\pgfqpoint{1.389373in}{1.186385in}}{\pgfqpoint{1.397273in}{1.183113in}}{\pgfqpoint{1.405509in}{1.183113in}}%
\pgfpathclose%
\pgfusepath{stroke,fill}%
\end{pgfscope}%
\begin{pgfscope}%
\pgfpathrectangle{\pgfqpoint{0.100000in}{0.212622in}}{\pgfqpoint{3.696000in}{3.696000in}}%
\pgfusepath{clip}%
\pgfsetbuttcap%
\pgfsetroundjoin%
\definecolor{currentfill}{rgb}{0.121569,0.466667,0.705882}%
\pgfsetfillcolor{currentfill}%
\pgfsetfillopacity{0.781001}%
\pgfsetlinewidth{1.003750pt}%
\definecolor{currentstroke}{rgb}{0.121569,0.466667,0.705882}%
\pgfsetstrokecolor{currentstroke}%
\pgfsetstrokeopacity{0.781001}%
\pgfsetdash{}{0pt}%
\pgfpathmoveto{\pgfqpoint{1.414744in}{1.180050in}}%
\pgfpathcurveto{\pgfqpoint{1.422980in}{1.180050in}}{\pgfqpoint{1.430880in}{1.183322in}}{\pgfqpoint{1.436704in}{1.189146in}}%
\pgfpathcurveto{\pgfqpoint{1.442528in}{1.194970in}}{\pgfqpoint{1.445801in}{1.202870in}}{\pgfqpoint{1.445801in}{1.211106in}}%
\pgfpathcurveto{\pgfqpoint{1.445801in}{1.219342in}}{\pgfqpoint{1.442528in}{1.227242in}}{\pgfqpoint{1.436704in}{1.233066in}}%
\pgfpathcurveto{\pgfqpoint{1.430880in}{1.238890in}}{\pgfqpoint{1.422980in}{1.242163in}}{\pgfqpoint{1.414744in}{1.242163in}}%
\pgfpathcurveto{\pgfqpoint{1.406508in}{1.242163in}}{\pgfqpoint{1.398608in}{1.238890in}}{\pgfqpoint{1.392784in}{1.233066in}}%
\pgfpathcurveto{\pgfqpoint{1.386960in}{1.227242in}}{\pgfqpoint{1.383688in}{1.219342in}}{\pgfqpoint{1.383688in}{1.211106in}}%
\pgfpathcurveto{\pgfqpoint{1.383688in}{1.202870in}}{\pgfqpoint{1.386960in}{1.194970in}}{\pgfqpoint{1.392784in}{1.189146in}}%
\pgfpathcurveto{\pgfqpoint{1.398608in}{1.183322in}}{\pgfqpoint{1.406508in}{1.180050in}}{\pgfqpoint{1.414744in}{1.180050in}}%
\pgfpathclose%
\pgfusepath{stroke,fill}%
\end{pgfscope}%
\begin{pgfscope}%
\pgfpathrectangle{\pgfqpoint{0.100000in}{0.212622in}}{\pgfqpoint{3.696000in}{3.696000in}}%
\pgfusepath{clip}%
\pgfsetbuttcap%
\pgfsetroundjoin%
\definecolor{currentfill}{rgb}{0.121569,0.466667,0.705882}%
\pgfsetfillcolor{currentfill}%
\pgfsetfillopacity{0.781144}%
\pgfsetlinewidth{1.003750pt}%
\definecolor{currentstroke}{rgb}{0.121569,0.466667,0.705882}%
\pgfsetstrokecolor{currentstroke}%
\pgfsetstrokeopacity{0.781144}%
\pgfsetdash{}{0pt}%
\pgfpathmoveto{\pgfqpoint{2.246151in}{1.509296in}}%
\pgfpathcurveto{\pgfqpoint{2.254387in}{1.509296in}}{\pgfqpoint{2.262287in}{1.512569in}}{\pgfqpoint{2.268111in}{1.518392in}}%
\pgfpathcurveto{\pgfqpoint{2.273935in}{1.524216in}}{\pgfqpoint{2.277207in}{1.532116in}}{\pgfqpoint{2.277207in}{1.540353in}}%
\pgfpathcurveto{\pgfqpoint{2.277207in}{1.548589in}}{\pgfqpoint{2.273935in}{1.556489in}}{\pgfqpoint{2.268111in}{1.562313in}}%
\pgfpathcurveto{\pgfqpoint{2.262287in}{1.568137in}}{\pgfqpoint{2.254387in}{1.571409in}}{\pgfqpoint{2.246151in}{1.571409in}}%
\pgfpathcurveto{\pgfqpoint{2.237915in}{1.571409in}}{\pgfqpoint{2.230015in}{1.568137in}}{\pgfqpoint{2.224191in}{1.562313in}}%
\pgfpathcurveto{\pgfqpoint{2.218367in}{1.556489in}}{\pgfqpoint{2.215094in}{1.548589in}}{\pgfqpoint{2.215094in}{1.540353in}}%
\pgfpathcurveto{\pgfqpoint{2.215094in}{1.532116in}}{\pgfqpoint{2.218367in}{1.524216in}}{\pgfqpoint{2.224191in}{1.518392in}}%
\pgfpathcurveto{\pgfqpoint{2.230015in}{1.512569in}}{\pgfqpoint{2.237915in}{1.509296in}}{\pgfqpoint{2.246151in}{1.509296in}}%
\pgfpathclose%
\pgfusepath{stroke,fill}%
\end{pgfscope}%
\begin{pgfscope}%
\pgfpathrectangle{\pgfqpoint{0.100000in}{0.212622in}}{\pgfqpoint{3.696000in}{3.696000in}}%
\pgfusepath{clip}%
\pgfsetbuttcap%
\pgfsetroundjoin%
\definecolor{currentfill}{rgb}{0.121569,0.466667,0.705882}%
\pgfsetfillcolor{currentfill}%
\pgfsetfillopacity{0.782688}%
\pgfsetlinewidth{1.003750pt}%
\definecolor{currentstroke}{rgb}{0.121569,0.466667,0.705882}%
\pgfsetstrokecolor{currentstroke}%
\pgfsetstrokeopacity{0.782688}%
\pgfsetdash{}{0pt}%
\pgfpathmoveto{\pgfqpoint{1.422330in}{1.177236in}}%
\pgfpathcurveto{\pgfqpoint{1.430566in}{1.177236in}}{\pgfqpoint{1.438466in}{1.180508in}}{\pgfqpoint{1.444290in}{1.186332in}}%
\pgfpathcurveto{\pgfqpoint{1.450114in}{1.192156in}}{\pgfqpoint{1.453386in}{1.200056in}}{\pgfqpoint{1.453386in}{1.208293in}}%
\pgfpathcurveto{\pgfqpoint{1.453386in}{1.216529in}}{\pgfqpoint{1.450114in}{1.224429in}}{\pgfqpoint{1.444290in}{1.230253in}}%
\pgfpathcurveto{\pgfqpoint{1.438466in}{1.236077in}}{\pgfqpoint{1.430566in}{1.239349in}}{\pgfqpoint{1.422330in}{1.239349in}}%
\pgfpathcurveto{\pgfqpoint{1.414093in}{1.239349in}}{\pgfqpoint{1.406193in}{1.236077in}}{\pgfqpoint{1.400369in}{1.230253in}}%
\pgfpathcurveto{\pgfqpoint{1.394545in}{1.224429in}}{\pgfqpoint{1.391273in}{1.216529in}}{\pgfqpoint{1.391273in}{1.208293in}}%
\pgfpathcurveto{\pgfqpoint{1.391273in}{1.200056in}}{\pgfqpoint{1.394545in}{1.192156in}}{\pgfqpoint{1.400369in}{1.186332in}}%
\pgfpathcurveto{\pgfqpoint{1.406193in}{1.180508in}}{\pgfqpoint{1.414093in}{1.177236in}}{\pgfqpoint{1.422330in}{1.177236in}}%
\pgfpathclose%
\pgfusepath{stroke,fill}%
\end{pgfscope}%
\begin{pgfscope}%
\pgfpathrectangle{\pgfqpoint{0.100000in}{0.212622in}}{\pgfqpoint{3.696000in}{3.696000in}}%
\pgfusepath{clip}%
\pgfsetbuttcap%
\pgfsetroundjoin%
\definecolor{currentfill}{rgb}{0.121569,0.466667,0.705882}%
\pgfsetfillcolor{currentfill}%
\pgfsetfillopacity{0.784025}%
\pgfsetlinewidth{1.003750pt}%
\definecolor{currentstroke}{rgb}{0.121569,0.466667,0.705882}%
\pgfsetstrokecolor{currentstroke}%
\pgfsetstrokeopacity{0.784025}%
\pgfsetdash{}{0pt}%
\pgfpathmoveto{\pgfqpoint{1.428580in}{1.174533in}}%
\pgfpathcurveto{\pgfqpoint{1.436817in}{1.174533in}}{\pgfqpoint{1.444717in}{1.177806in}}{\pgfqpoint{1.450541in}{1.183630in}}%
\pgfpathcurveto{\pgfqpoint{1.456364in}{1.189454in}}{\pgfqpoint{1.459637in}{1.197354in}}{\pgfqpoint{1.459637in}{1.205590in}}%
\pgfpathcurveto{\pgfqpoint{1.459637in}{1.213826in}}{\pgfqpoint{1.456364in}{1.221726in}}{\pgfqpoint{1.450541in}{1.227550in}}%
\pgfpathcurveto{\pgfqpoint{1.444717in}{1.233374in}}{\pgfqpoint{1.436817in}{1.236646in}}{\pgfqpoint{1.428580in}{1.236646in}}%
\pgfpathcurveto{\pgfqpoint{1.420344in}{1.236646in}}{\pgfqpoint{1.412444in}{1.233374in}}{\pgfqpoint{1.406620in}{1.227550in}}%
\pgfpathcurveto{\pgfqpoint{1.400796in}{1.221726in}}{\pgfqpoint{1.397524in}{1.213826in}}{\pgfqpoint{1.397524in}{1.205590in}}%
\pgfpathcurveto{\pgfqpoint{1.397524in}{1.197354in}}{\pgfqpoint{1.400796in}{1.189454in}}{\pgfqpoint{1.406620in}{1.183630in}}%
\pgfpathcurveto{\pgfqpoint{1.412444in}{1.177806in}}{\pgfqpoint{1.420344in}{1.174533in}}{\pgfqpoint{1.428580in}{1.174533in}}%
\pgfpathclose%
\pgfusepath{stroke,fill}%
\end{pgfscope}%
\begin{pgfscope}%
\pgfpathrectangle{\pgfqpoint{0.100000in}{0.212622in}}{\pgfqpoint{3.696000in}{3.696000in}}%
\pgfusepath{clip}%
\pgfsetbuttcap%
\pgfsetroundjoin%
\definecolor{currentfill}{rgb}{0.121569,0.466667,0.705882}%
\pgfsetfillcolor{currentfill}%
\pgfsetfillopacity{0.785686}%
\pgfsetlinewidth{1.003750pt}%
\definecolor{currentstroke}{rgb}{0.121569,0.466667,0.705882}%
\pgfsetstrokecolor{currentstroke}%
\pgfsetstrokeopacity{0.785686}%
\pgfsetdash{}{0pt}%
\pgfpathmoveto{\pgfqpoint{2.251202in}{1.493562in}}%
\pgfpathcurveto{\pgfqpoint{2.259439in}{1.493562in}}{\pgfqpoint{2.267339in}{1.496834in}}{\pgfqpoint{2.273163in}{1.502658in}}%
\pgfpathcurveto{\pgfqpoint{2.278987in}{1.508482in}}{\pgfqpoint{2.282259in}{1.516382in}}{\pgfqpoint{2.282259in}{1.524618in}}%
\pgfpathcurveto{\pgfqpoint{2.282259in}{1.532854in}}{\pgfqpoint{2.278987in}{1.540754in}}{\pgfqpoint{2.273163in}{1.546578in}}%
\pgfpathcurveto{\pgfqpoint{2.267339in}{1.552402in}}{\pgfqpoint{2.259439in}{1.555675in}}{\pgfqpoint{2.251202in}{1.555675in}}%
\pgfpathcurveto{\pgfqpoint{2.242966in}{1.555675in}}{\pgfqpoint{2.235066in}{1.552402in}}{\pgfqpoint{2.229242in}{1.546578in}}%
\pgfpathcurveto{\pgfqpoint{2.223418in}{1.540754in}}{\pgfqpoint{2.220146in}{1.532854in}}{\pgfqpoint{2.220146in}{1.524618in}}%
\pgfpathcurveto{\pgfqpoint{2.220146in}{1.516382in}}{\pgfqpoint{2.223418in}{1.508482in}}{\pgfqpoint{2.229242in}{1.502658in}}%
\pgfpathcurveto{\pgfqpoint{2.235066in}{1.496834in}}{\pgfqpoint{2.242966in}{1.493562in}}{\pgfqpoint{2.251202in}{1.493562in}}%
\pgfpathclose%
\pgfusepath{stroke,fill}%
\end{pgfscope}%
\begin{pgfscope}%
\pgfpathrectangle{\pgfqpoint{0.100000in}{0.212622in}}{\pgfqpoint{3.696000in}{3.696000in}}%
\pgfusepath{clip}%
\pgfsetbuttcap%
\pgfsetroundjoin%
\definecolor{currentfill}{rgb}{0.121569,0.466667,0.705882}%
\pgfsetfillcolor{currentfill}%
\pgfsetfillopacity{0.786509}%
\pgfsetlinewidth{1.003750pt}%
\definecolor{currentstroke}{rgb}{0.121569,0.466667,0.705882}%
\pgfsetstrokecolor{currentstroke}%
\pgfsetstrokeopacity{0.786509}%
\pgfsetdash{}{0pt}%
\pgfpathmoveto{\pgfqpoint{1.439972in}{1.169892in}}%
\pgfpathcurveto{\pgfqpoint{1.448208in}{1.169892in}}{\pgfqpoint{1.456108in}{1.173165in}}{\pgfqpoint{1.461932in}{1.178988in}}%
\pgfpathcurveto{\pgfqpoint{1.467756in}{1.184812in}}{\pgfqpoint{1.471028in}{1.192712in}}{\pgfqpoint{1.471028in}{1.200949in}}%
\pgfpathcurveto{\pgfqpoint{1.471028in}{1.209185in}}{\pgfqpoint{1.467756in}{1.217085in}}{\pgfqpoint{1.461932in}{1.222909in}}%
\pgfpathcurveto{\pgfqpoint{1.456108in}{1.228733in}}{\pgfqpoint{1.448208in}{1.232005in}}{\pgfqpoint{1.439972in}{1.232005in}}%
\pgfpathcurveto{\pgfqpoint{1.431736in}{1.232005in}}{\pgfqpoint{1.423835in}{1.228733in}}{\pgfqpoint{1.418012in}{1.222909in}}%
\pgfpathcurveto{\pgfqpoint{1.412188in}{1.217085in}}{\pgfqpoint{1.408915in}{1.209185in}}{\pgfqpoint{1.408915in}{1.200949in}}%
\pgfpathcurveto{\pgfqpoint{1.408915in}{1.192712in}}{\pgfqpoint{1.412188in}{1.184812in}}{\pgfqpoint{1.418012in}{1.178988in}}%
\pgfpathcurveto{\pgfqpoint{1.423835in}{1.173165in}}{\pgfqpoint{1.431736in}{1.169892in}}{\pgfqpoint{1.439972in}{1.169892in}}%
\pgfpathclose%
\pgfusepath{stroke,fill}%
\end{pgfscope}%
\begin{pgfscope}%
\pgfpathrectangle{\pgfqpoint{0.100000in}{0.212622in}}{\pgfqpoint{3.696000in}{3.696000in}}%
\pgfusepath{clip}%
\pgfsetbuttcap%
\pgfsetroundjoin%
\definecolor{currentfill}{rgb}{0.121569,0.466667,0.705882}%
\pgfsetfillcolor{currentfill}%
\pgfsetfillopacity{0.788806}%
\pgfsetlinewidth{1.003750pt}%
\definecolor{currentstroke}{rgb}{0.121569,0.466667,0.705882}%
\pgfsetstrokecolor{currentstroke}%
\pgfsetstrokeopacity{0.788806}%
\pgfsetdash{}{0pt}%
\pgfpathmoveto{\pgfqpoint{1.449815in}{1.166000in}}%
\pgfpathcurveto{\pgfqpoint{1.458051in}{1.166000in}}{\pgfqpoint{1.465951in}{1.169272in}}{\pgfqpoint{1.471775in}{1.175096in}}%
\pgfpathcurveto{\pgfqpoint{1.477599in}{1.180920in}}{\pgfqpoint{1.480871in}{1.188820in}}{\pgfqpoint{1.480871in}{1.197056in}}%
\pgfpathcurveto{\pgfqpoint{1.480871in}{1.205292in}}{\pgfqpoint{1.477599in}{1.213192in}}{\pgfqpoint{1.471775in}{1.219016in}}%
\pgfpathcurveto{\pgfqpoint{1.465951in}{1.224840in}}{\pgfqpoint{1.458051in}{1.228113in}}{\pgfqpoint{1.449815in}{1.228113in}}%
\pgfpathcurveto{\pgfqpoint{1.441578in}{1.228113in}}{\pgfqpoint{1.433678in}{1.224840in}}{\pgfqpoint{1.427854in}{1.219016in}}%
\pgfpathcurveto{\pgfqpoint{1.422030in}{1.213192in}}{\pgfqpoint{1.418758in}{1.205292in}}{\pgfqpoint{1.418758in}{1.197056in}}%
\pgfpathcurveto{\pgfqpoint{1.418758in}{1.188820in}}{\pgfqpoint{1.422030in}{1.180920in}}{\pgfqpoint{1.427854in}{1.175096in}}%
\pgfpathcurveto{\pgfqpoint{1.433678in}{1.169272in}}{\pgfqpoint{1.441578in}{1.166000in}}{\pgfqpoint{1.449815in}{1.166000in}}%
\pgfpathclose%
\pgfusepath{stroke,fill}%
\end{pgfscope}%
\begin{pgfscope}%
\pgfpathrectangle{\pgfqpoint{0.100000in}{0.212622in}}{\pgfqpoint{3.696000in}{3.696000in}}%
\pgfusepath{clip}%
\pgfsetbuttcap%
\pgfsetroundjoin%
\definecolor{currentfill}{rgb}{0.121569,0.466667,0.705882}%
\pgfsetfillcolor{currentfill}%
\pgfsetfillopacity{0.790582}%
\pgfsetlinewidth{1.003750pt}%
\definecolor{currentstroke}{rgb}{0.121569,0.466667,0.705882}%
\pgfsetstrokecolor{currentstroke}%
\pgfsetstrokeopacity{0.790582}%
\pgfsetdash{}{0pt}%
\pgfpathmoveto{\pgfqpoint{2.255697in}{1.477758in}}%
\pgfpathcurveto{\pgfqpoint{2.263933in}{1.477758in}}{\pgfqpoint{2.271833in}{1.481031in}}{\pgfqpoint{2.277657in}{1.486855in}}%
\pgfpathcurveto{\pgfqpoint{2.283481in}{1.492678in}}{\pgfqpoint{2.286753in}{1.500578in}}{\pgfqpoint{2.286753in}{1.508815in}}%
\pgfpathcurveto{\pgfqpoint{2.286753in}{1.517051in}}{\pgfqpoint{2.283481in}{1.524951in}}{\pgfqpoint{2.277657in}{1.530775in}}%
\pgfpathcurveto{\pgfqpoint{2.271833in}{1.536599in}}{\pgfqpoint{2.263933in}{1.539871in}}{\pgfqpoint{2.255697in}{1.539871in}}%
\pgfpathcurveto{\pgfqpoint{2.247461in}{1.539871in}}{\pgfqpoint{2.239561in}{1.536599in}}{\pgfqpoint{2.233737in}{1.530775in}}%
\pgfpathcurveto{\pgfqpoint{2.227913in}{1.524951in}}{\pgfqpoint{2.224640in}{1.517051in}}{\pgfqpoint{2.224640in}{1.508815in}}%
\pgfpathcurveto{\pgfqpoint{2.224640in}{1.500578in}}{\pgfqpoint{2.227913in}{1.492678in}}{\pgfqpoint{2.233737in}{1.486855in}}%
\pgfpathcurveto{\pgfqpoint{2.239561in}{1.481031in}}{\pgfqpoint{2.247461in}{1.477758in}}{\pgfqpoint{2.255697in}{1.477758in}}%
\pgfpathclose%
\pgfusepath{stroke,fill}%
\end{pgfscope}%
\begin{pgfscope}%
\pgfpathrectangle{\pgfqpoint{0.100000in}{0.212622in}}{\pgfqpoint{3.696000in}{3.696000in}}%
\pgfusepath{clip}%
\pgfsetbuttcap%
\pgfsetroundjoin%
\definecolor{currentfill}{rgb}{0.121569,0.466667,0.705882}%
\pgfsetfillcolor{currentfill}%
\pgfsetfillopacity{0.790957}%
\pgfsetlinewidth{1.003750pt}%
\definecolor{currentstroke}{rgb}{0.121569,0.466667,0.705882}%
\pgfsetstrokecolor{currentstroke}%
\pgfsetstrokeopacity{0.790957}%
\pgfsetdash{}{0pt}%
\pgfpathmoveto{\pgfqpoint{1.459176in}{1.161882in}}%
\pgfpathcurveto{\pgfqpoint{1.467412in}{1.161882in}}{\pgfqpoint{1.475312in}{1.165154in}}{\pgfqpoint{1.481136in}{1.170978in}}%
\pgfpathcurveto{\pgfqpoint{1.486960in}{1.176802in}}{\pgfqpoint{1.490232in}{1.184702in}}{\pgfqpoint{1.490232in}{1.192938in}}%
\pgfpathcurveto{\pgfqpoint{1.490232in}{1.201175in}}{\pgfqpoint{1.486960in}{1.209075in}}{\pgfqpoint{1.481136in}{1.214899in}}%
\pgfpathcurveto{\pgfqpoint{1.475312in}{1.220723in}}{\pgfqpoint{1.467412in}{1.223995in}}{\pgfqpoint{1.459176in}{1.223995in}}%
\pgfpathcurveto{\pgfqpoint{1.450939in}{1.223995in}}{\pgfqpoint{1.443039in}{1.220723in}}{\pgfqpoint{1.437215in}{1.214899in}}%
\pgfpathcurveto{\pgfqpoint{1.431391in}{1.209075in}}{\pgfqpoint{1.428119in}{1.201175in}}{\pgfqpoint{1.428119in}{1.192938in}}%
\pgfpathcurveto{\pgfqpoint{1.428119in}{1.184702in}}{\pgfqpoint{1.431391in}{1.176802in}}{\pgfqpoint{1.437215in}{1.170978in}}%
\pgfpathcurveto{\pgfqpoint{1.443039in}{1.165154in}}{\pgfqpoint{1.450939in}{1.161882in}}{\pgfqpoint{1.459176in}{1.161882in}}%
\pgfpathclose%
\pgfusepath{stroke,fill}%
\end{pgfscope}%
\begin{pgfscope}%
\pgfpathrectangle{\pgfqpoint{0.100000in}{0.212622in}}{\pgfqpoint{3.696000in}{3.696000in}}%
\pgfusepath{clip}%
\pgfsetbuttcap%
\pgfsetroundjoin%
\definecolor{currentfill}{rgb}{0.121569,0.466667,0.705882}%
\pgfsetfillcolor{currentfill}%
\pgfsetfillopacity{0.792936}%
\pgfsetlinewidth{1.003750pt}%
\definecolor{currentstroke}{rgb}{0.121569,0.466667,0.705882}%
\pgfsetstrokecolor{currentstroke}%
\pgfsetstrokeopacity{0.792936}%
\pgfsetdash{}{0pt}%
\pgfpathmoveto{\pgfqpoint{1.467916in}{1.158940in}}%
\pgfpathcurveto{\pgfqpoint{1.476152in}{1.158940in}}{\pgfqpoint{1.484052in}{1.162213in}}{\pgfqpoint{1.489876in}{1.168037in}}%
\pgfpathcurveto{\pgfqpoint{1.495700in}{1.173861in}}{\pgfqpoint{1.498972in}{1.181761in}}{\pgfqpoint{1.498972in}{1.189997in}}%
\pgfpathcurveto{\pgfqpoint{1.498972in}{1.198233in}}{\pgfqpoint{1.495700in}{1.206133in}}{\pgfqpoint{1.489876in}{1.211957in}}%
\pgfpathcurveto{\pgfqpoint{1.484052in}{1.217781in}}{\pgfqpoint{1.476152in}{1.221053in}}{\pgfqpoint{1.467916in}{1.221053in}}%
\pgfpathcurveto{\pgfqpoint{1.459680in}{1.221053in}}{\pgfqpoint{1.451780in}{1.217781in}}{\pgfqpoint{1.445956in}{1.211957in}}%
\pgfpathcurveto{\pgfqpoint{1.440132in}{1.206133in}}{\pgfqpoint{1.436859in}{1.198233in}}{\pgfqpoint{1.436859in}{1.189997in}}%
\pgfpathcurveto{\pgfqpoint{1.436859in}{1.181761in}}{\pgfqpoint{1.440132in}{1.173861in}}{\pgfqpoint{1.445956in}{1.168037in}}%
\pgfpathcurveto{\pgfqpoint{1.451780in}{1.162213in}}{\pgfqpoint{1.459680in}{1.158940in}}{\pgfqpoint{1.467916in}{1.158940in}}%
\pgfpathclose%
\pgfusepath{stroke,fill}%
\end{pgfscope}%
\begin{pgfscope}%
\pgfpathrectangle{\pgfqpoint{0.100000in}{0.212622in}}{\pgfqpoint{3.696000in}{3.696000in}}%
\pgfusepath{clip}%
\pgfsetbuttcap%
\pgfsetroundjoin%
\definecolor{currentfill}{rgb}{0.121569,0.466667,0.705882}%
\pgfsetfillcolor{currentfill}%
\pgfsetfillopacity{0.794948}%
\pgfsetlinewidth{1.003750pt}%
\definecolor{currentstroke}{rgb}{0.121569,0.466667,0.705882}%
\pgfsetstrokecolor{currentstroke}%
\pgfsetstrokeopacity{0.794948}%
\pgfsetdash{}{0pt}%
\pgfpathmoveto{\pgfqpoint{1.475634in}{1.156593in}}%
\pgfpathcurveto{\pgfqpoint{1.483870in}{1.156593in}}{\pgfqpoint{1.491770in}{1.159865in}}{\pgfqpoint{1.497594in}{1.165689in}}%
\pgfpathcurveto{\pgfqpoint{1.503418in}{1.171513in}}{\pgfqpoint{1.506690in}{1.179413in}}{\pgfqpoint{1.506690in}{1.187650in}}%
\pgfpathcurveto{\pgfqpoint{1.506690in}{1.195886in}}{\pgfqpoint{1.503418in}{1.203786in}}{\pgfqpoint{1.497594in}{1.209610in}}%
\pgfpathcurveto{\pgfqpoint{1.491770in}{1.215434in}}{\pgfqpoint{1.483870in}{1.218706in}}{\pgfqpoint{1.475634in}{1.218706in}}%
\pgfpathcurveto{\pgfqpoint{1.467397in}{1.218706in}}{\pgfqpoint{1.459497in}{1.215434in}}{\pgfqpoint{1.453673in}{1.209610in}}%
\pgfpathcurveto{\pgfqpoint{1.447850in}{1.203786in}}{\pgfqpoint{1.444577in}{1.195886in}}{\pgfqpoint{1.444577in}{1.187650in}}%
\pgfpathcurveto{\pgfqpoint{1.444577in}{1.179413in}}{\pgfqpoint{1.447850in}{1.171513in}}{\pgfqpoint{1.453673in}{1.165689in}}%
\pgfpathcurveto{\pgfqpoint{1.459497in}{1.159865in}}{\pgfqpoint{1.467397in}{1.156593in}}{\pgfqpoint{1.475634in}{1.156593in}}%
\pgfpathclose%
\pgfusepath{stroke,fill}%
\end{pgfscope}%
\begin{pgfscope}%
\pgfpathrectangle{\pgfqpoint{0.100000in}{0.212622in}}{\pgfqpoint{3.696000in}{3.696000in}}%
\pgfusepath{clip}%
\pgfsetbuttcap%
\pgfsetroundjoin%
\definecolor{currentfill}{rgb}{0.121569,0.466667,0.705882}%
\pgfsetfillcolor{currentfill}%
\pgfsetfillopacity{0.795912}%
\pgfsetlinewidth{1.003750pt}%
\definecolor{currentstroke}{rgb}{0.121569,0.466667,0.705882}%
\pgfsetstrokecolor{currentstroke}%
\pgfsetstrokeopacity{0.795912}%
\pgfsetdash{}{0pt}%
\pgfpathmoveto{\pgfqpoint{2.259505in}{1.461166in}}%
\pgfpathcurveto{\pgfqpoint{2.267742in}{1.461166in}}{\pgfqpoint{2.275642in}{1.464439in}}{\pgfqpoint{2.281466in}{1.470263in}}%
\pgfpathcurveto{\pgfqpoint{2.287290in}{1.476086in}}{\pgfqpoint{2.290562in}{1.483987in}}{\pgfqpoint{2.290562in}{1.492223in}}%
\pgfpathcurveto{\pgfqpoint{2.290562in}{1.500459in}}{\pgfqpoint{2.287290in}{1.508359in}}{\pgfqpoint{2.281466in}{1.514183in}}%
\pgfpathcurveto{\pgfqpoint{2.275642in}{1.520007in}}{\pgfqpoint{2.267742in}{1.523279in}}{\pgfqpoint{2.259505in}{1.523279in}}%
\pgfpathcurveto{\pgfqpoint{2.251269in}{1.523279in}}{\pgfqpoint{2.243369in}{1.520007in}}{\pgfqpoint{2.237545in}{1.514183in}}%
\pgfpathcurveto{\pgfqpoint{2.231721in}{1.508359in}}{\pgfqpoint{2.228449in}{1.500459in}}{\pgfqpoint{2.228449in}{1.492223in}}%
\pgfpathcurveto{\pgfqpoint{2.228449in}{1.483987in}}{\pgfqpoint{2.231721in}{1.476086in}}{\pgfqpoint{2.237545in}{1.470263in}}%
\pgfpathcurveto{\pgfqpoint{2.243369in}{1.464439in}}{\pgfqpoint{2.251269in}{1.461166in}}{\pgfqpoint{2.259505in}{1.461166in}}%
\pgfpathclose%
\pgfusepath{stroke,fill}%
\end{pgfscope}%
\begin{pgfscope}%
\pgfpathrectangle{\pgfqpoint{0.100000in}{0.212622in}}{\pgfqpoint{3.696000in}{3.696000in}}%
\pgfusepath{clip}%
\pgfsetbuttcap%
\pgfsetroundjoin%
\definecolor{currentfill}{rgb}{0.121569,0.466667,0.705882}%
\pgfsetfillcolor{currentfill}%
\pgfsetfillopacity{0.796724}%
\pgfsetlinewidth{1.003750pt}%
\definecolor{currentstroke}{rgb}{0.121569,0.466667,0.705882}%
\pgfsetstrokecolor{currentstroke}%
\pgfsetstrokeopacity{0.796724}%
\pgfsetdash{}{0pt}%
\pgfpathmoveto{\pgfqpoint{1.482981in}{1.154354in}}%
\pgfpathcurveto{\pgfqpoint{1.491218in}{1.154354in}}{\pgfqpoint{1.499118in}{1.157626in}}{\pgfqpoint{1.504942in}{1.163450in}}%
\pgfpathcurveto{\pgfqpoint{1.510766in}{1.169274in}}{\pgfqpoint{1.514038in}{1.177174in}}{\pgfqpoint{1.514038in}{1.185410in}}%
\pgfpathcurveto{\pgfqpoint{1.514038in}{1.193646in}}{\pgfqpoint{1.510766in}{1.201546in}}{\pgfqpoint{1.504942in}{1.207370in}}%
\pgfpathcurveto{\pgfqpoint{1.499118in}{1.213194in}}{\pgfqpoint{1.491218in}{1.216467in}}{\pgfqpoint{1.482981in}{1.216467in}}%
\pgfpathcurveto{\pgfqpoint{1.474745in}{1.216467in}}{\pgfqpoint{1.466845in}{1.213194in}}{\pgfqpoint{1.461021in}{1.207370in}}%
\pgfpathcurveto{\pgfqpoint{1.455197in}{1.201546in}}{\pgfqpoint{1.451925in}{1.193646in}}{\pgfqpoint{1.451925in}{1.185410in}}%
\pgfpathcurveto{\pgfqpoint{1.451925in}{1.177174in}}{\pgfqpoint{1.455197in}{1.169274in}}{\pgfqpoint{1.461021in}{1.163450in}}%
\pgfpathcurveto{\pgfqpoint{1.466845in}{1.157626in}}{\pgfqpoint{1.474745in}{1.154354in}}{\pgfqpoint{1.482981in}{1.154354in}}%
\pgfpathclose%
\pgfusepath{stroke,fill}%
\end{pgfscope}%
\begin{pgfscope}%
\pgfpathrectangle{\pgfqpoint{0.100000in}{0.212622in}}{\pgfqpoint{3.696000in}{3.696000in}}%
\pgfusepath{clip}%
\pgfsetbuttcap%
\pgfsetroundjoin%
\definecolor{currentfill}{rgb}{0.121569,0.466667,0.705882}%
\pgfsetfillcolor{currentfill}%
\pgfsetfillopacity{0.797944}%
\pgfsetlinewidth{1.003750pt}%
\definecolor{currentstroke}{rgb}{0.121569,0.466667,0.705882}%
\pgfsetstrokecolor{currentstroke}%
\pgfsetstrokeopacity{0.797944}%
\pgfsetdash{}{0pt}%
\pgfpathmoveto{\pgfqpoint{1.488515in}{1.152157in}}%
\pgfpathcurveto{\pgfqpoint{1.496751in}{1.152157in}}{\pgfqpoint{1.504651in}{1.155430in}}{\pgfqpoint{1.510475in}{1.161254in}}%
\pgfpathcurveto{\pgfqpoint{1.516299in}{1.167077in}}{\pgfqpoint{1.519571in}{1.174978in}}{\pgfqpoint{1.519571in}{1.183214in}}%
\pgfpathcurveto{\pgfqpoint{1.519571in}{1.191450in}}{\pgfqpoint{1.516299in}{1.199350in}}{\pgfqpoint{1.510475in}{1.205174in}}%
\pgfpathcurveto{\pgfqpoint{1.504651in}{1.210998in}}{\pgfqpoint{1.496751in}{1.214270in}}{\pgfqpoint{1.488515in}{1.214270in}}%
\pgfpathcurveto{\pgfqpoint{1.480278in}{1.214270in}}{\pgfqpoint{1.472378in}{1.210998in}}{\pgfqpoint{1.466554in}{1.205174in}}%
\pgfpathcurveto{\pgfqpoint{1.460730in}{1.199350in}}{\pgfqpoint{1.457458in}{1.191450in}}{\pgfqpoint{1.457458in}{1.183214in}}%
\pgfpathcurveto{\pgfqpoint{1.457458in}{1.174978in}}{\pgfqpoint{1.460730in}{1.167077in}}{\pgfqpoint{1.466554in}{1.161254in}}%
\pgfpathcurveto{\pgfqpoint{1.472378in}{1.155430in}}{\pgfqpoint{1.480278in}{1.152157in}}{\pgfqpoint{1.488515in}{1.152157in}}%
\pgfpathclose%
\pgfusepath{stroke,fill}%
\end{pgfscope}%
\begin{pgfscope}%
\pgfpathrectangle{\pgfqpoint{0.100000in}{0.212622in}}{\pgfqpoint{3.696000in}{3.696000in}}%
\pgfusepath{clip}%
\pgfsetbuttcap%
\pgfsetroundjoin%
\definecolor{currentfill}{rgb}{0.121569,0.466667,0.705882}%
\pgfsetfillcolor{currentfill}%
\pgfsetfillopacity{0.800146}%
\pgfsetlinewidth{1.003750pt}%
\definecolor{currentstroke}{rgb}{0.121569,0.466667,0.705882}%
\pgfsetstrokecolor{currentstroke}%
\pgfsetstrokeopacity{0.800146}%
\pgfsetdash{}{0pt}%
\pgfpathmoveto{\pgfqpoint{1.498765in}{1.148750in}}%
\pgfpathcurveto{\pgfqpoint{1.507001in}{1.148750in}}{\pgfqpoint{1.514901in}{1.152022in}}{\pgfqpoint{1.520725in}{1.157846in}}%
\pgfpathcurveto{\pgfqpoint{1.526549in}{1.163670in}}{\pgfqpoint{1.529822in}{1.171570in}}{\pgfqpoint{1.529822in}{1.179806in}}%
\pgfpathcurveto{\pgfqpoint{1.529822in}{1.188043in}}{\pgfqpoint{1.526549in}{1.195943in}}{\pgfqpoint{1.520725in}{1.201767in}}%
\pgfpathcurveto{\pgfqpoint{1.514901in}{1.207590in}}{\pgfqpoint{1.507001in}{1.210863in}}{\pgfqpoint{1.498765in}{1.210863in}}%
\pgfpathcurveto{\pgfqpoint{1.490529in}{1.210863in}}{\pgfqpoint{1.482629in}{1.207590in}}{\pgfqpoint{1.476805in}{1.201767in}}%
\pgfpathcurveto{\pgfqpoint{1.470981in}{1.195943in}}{\pgfqpoint{1.467709in}{1.188043in}}{\pgfqpoint{1.467709in}{1.179806in}}%
\pgfpathcurveto{\pgfqpoint{1.467709in}{1.171570in}}{\pgfqpoint{1.470981in}{1.163670in}}{\pgfqpoint{1.476805in}{1.157846in}}%
\pgfpathcurveto{\pgfqpoint{1.482629in}{1.152022in}}{\pgfqpoint{1.490529in}{1.148750in}}{\pgfqpoint{1.498765in}{1.148750in}}%
\pgfpathclose%
\pgfusepath{stroke,fill}%
\end{pgfscope}%
\begin{pgfscope}%
\pgfpathrectangle{\pgfqpoint{0.100000in}{0.212622in}}{\pgfqpoint{3.696000in}{3.696000in}}%
\pgfusepath{clip}%
\pgfsetbuttcap%
\pgfsetroundjoin%
\definecolor{currentfill}{rgb}{0.121569,0.466667,0.705882}%
\pgfsetfillcolor{currentfill}%
\pgfsetfillopacity{0.801293}%
\pgfsetlinewidth{1.003750pt}%
\definecolor{currentstroke}{rgb}{0.121569,0.466667,0.705882}%
\pgfsetstrokecolor{currentstroke}%
\pgfsetstrokeopacity{0.801293}%
\pgfsetdash{}{0pt}%
\pgfpathmoveto{\pgfqpoint{2.265310in}{1.443202in}}%
\pgfpathcurveto{\pgfqpoint{2.273546in}{1.443202in}}{\pgfqpoint{2.281447in}{1.446474in}}{\pgfqpoint{2.287270in}{1.452298in}}%
\pgfpathcurveto{\pgfqpoint{2.293094in}{1.458122in}}{\pgfqpoint{2.296367in}{1.466022in}}{\pgfqpoint{2.296367in}{1.474258in}}%
\pgfpathcurveto{\pgfqpoint{2.296367in}{1.482495in}}{\pgfqpoint{2.293094in}{1.490395in}}{\pgfqpoint{2.287270in}{1.496219in}}%
\pgfpathcurveto{\pgfqpoint{2.281447in}{1.502043in}}{\pgfqpoint{2.273546in}{1.505315in}}{\pgfqpoint{2.265310in}{1.505315in}}%
\pgfpathcurveto{\pgfqpoint{2.257074in}{1.505315in}}{\pgfqpoint{2.249174in}{1.502043in}}{\pgfqpoint{2.243350in}{1.496219in}}%
\pgfpathcurveto{\pgfqpoint{2.237526in}{1.490395in}}{\pgfqpoint{2.234254in}{1.482495in}}{\pgfqpoint{2.234254in}{1.474258in}}%
\pgfpathcurveto{\pgfqpoint{2.234254in}{1.466022in}}{\pgfqpoint{2.237526in}{1.458122in}}{\pgfqpoint{2.243350in}{1.452298in}}%
\pgfpathcurveto{\pgfqpoint{2.249174in}{1.446474in}}{\pgfqpoint{2.257074in}{1.443202in}}{\pgfqpoint{2.265310in}{1.443202in}}%
\pgfpathclose%
\pgfusepath{stroke,fill}%
\end{pgfscope}%
\begin{pgfscope}%
\pgfpathrectangle{\pgfqpoint{0.100000in}{0.212622in}}{\pgfqpoint{3.696000in}{3.696000in}}%
\pgfusepath{clip}%
\pgfsetbuttcap%
\pgfsetroundjoin%
\definecolor{currentfill}{rgb}{0.121569,0.466667,0.705882}%
\pgfsetfillcolor{currentfill}%
\pgfsetfillopacity{0.802010}%
\pgfsetlinewidth{1.003750pt}%
\definecolor{currentstroke}{rgb}{0.121569,0.466667,0.705882}%
\pgfsetstrokecolor{currentstroke}%
\pgfsetstrokeopacity{0.802010}%
\pgfsetdash{}{0pt}%
\pgfpathmoveto{\pgfqpoint{1.506835in}{1.146057in}}%
\pgfpathcurveto{\pgfqpoint{1.515071in}{1.146057in}}{\pgfqpoint{1.522971in}{1.149330in}}{\pgfqpoint{1.528795in}{1.155153in}}%
\pgfpathcurveto{\pgfqpoint{1.534619in}{1.160977in}}{\pgfqpoint{1.537891in}{1.168877in}}{\pgfqpoint{1.537891in}{1.177114in}}%
\pgfpathcurveto{\pgfqpoint{1.537891in}{1.185350in}}{\pgfqpoint{1.534619in}{1.193250in}}{\pgfqpoint{1.528795in}{1.199074in}}%
\pgfpathcurveto{\pgfqpoint{1.522971in}{1.204898in}}{\pgfqpoint{1.515071in}{1.208170in}}{\pgfqpoint{1.506835in}{1.208170in}}%
\pgfpathcurveto{\pgfqpoint{1.498599in}{1.208170in}}{\pgfqpoint{1.490699in}{1.204898in}}{\pgfqpoint{1.484875in}{1.199074in}}%
\pgfpathcurveto{\pgfqpoint{1.479051in}{1.193250in}}{\pgfqpoint{1.475778in}{1.185350in}}{\pgfqpoint{1.475778in}{1.177114in}}%
\pgfpathcurveto{\pgfqpoint{1.475778in}{1.168877in}}{\pgfqpoint{1.479051in}{1.160977in}}{\pgfqpoint{1.484875in}{1.155153in}}%
\pgfpathcurveto{\pgfqpoint{1.490699in}{1.149330in}}{\pgfqpoint{1.498599in}{1.146057in}}{\pgfqpoint{1.506835in}{1.146057in}}%
\pgfpathclose%
\pgfusepath{stroke,fill}%
\end{pgfscope}%
\begin{pgfscope}%
\pgfpathrectangle{\pgfqpoint{0.100000in}{0.212622in}}{\pgfqpoint{3.696000in}{3.696000in}}%
\pgfusepath{clip}%
\pgfsetbuttcap%
\pgfsetroundjoin%
\definecolor{currentfill}{rgb}{0.121569,0.466667,0.705882}%
\pgfsetfillcolor{currentfill}%
\pgfsetfillopacity{0.803694}%
\pgfsetlinewidth{1.003750pt}%
\definecolor{currentstroke}{rgb}{0.121569,0.466667,0.705882}%
\pgfsetstrokecolor{currentstroke}%
\pgfsetstrokeopacity{0.803694}%
\pgfsetdash{}{0pt}%
\pgfpathmoveto{\pgfqpoint{1.514281in}{1.143287in}}%
\pgfpathcurveto{\pgfqpoint{1.522517in}{1.143287in}}{\pgfqpoint{1.530417in}{1.146559in}}{\pgfqpoint{1.536241in}{1.152383in}}%
\pgfpathcurveto{\pgfqpoint{1.542065in}{1.158207in}}{\pgfqpoint{1.545337in}{1.166107in}}{\pgfqpoint{1.545337in}{1.174344in}}%
\pgfpathcurveto{\pgfqpoint{1.545337in}{1.182580in}}{\pgfqpoint{1.542065in}{1.190480in}}{\pgfqpoint{1.536241in}{1.196304in}}%
\pgfpathcurveto{\pgfqpoint{1.530417in}{1.202128in}}{\pgfqpoint{1.522517in}{1.205400in}}{\pgfqpoint{1.514281in}{1.205400in}}%
\pgfpathcurveto{\pgfqpoint{1.506045in}{1.205400in}}{\pgfqpoint{1.498144in}{1.202128in}}{\pgfqpoint{1.492321in}{1.196304in}}%
\pgfpathcurveto{\pgfqpoint{1.486497in}{1.190480in}}{\pgfqpoint{1.483224in}{1.182580in}}{\pgfqpoint{1.483224in}{1.174344in}}%
\pgfpathcurveto{\pgfqpoint{1.483224in}{1.166107in}}{\pgfqpoint{1.486497in}{1.158207in}}{\pgfqpoint{1.492321in}{1.152383in}}%
\pgfpathcurveto{\pgfqpoint{1.498144in}{1.146559in}}{\pgfqpoint{1.506045in}{1.143287in}}{\pgfqpoint{1.514281in}{1.143287in}}%
\pgfpathclose%
\pgfusepath{stroke,fill}%
\end{pgfscope}%
\begin{pgfscope}%
\pgfpathrectangle{\pgfqpoint{0.100000in}{0.212622in}}{\pgfqpoint{3.696000in}{3.696000in}}%
\pgfusepath{clip}%
\pgfsetbuttcap%
\pgfsetroundjoin%
\definecolor{currentfill}{rgb}{0.121569,0.466667,0.705882}%
\pgfsetfillcolor{currentfill}%
\pgfsetfillopacity{0.805138}%
\pgfsetlinewidth{1.003750pt}%
\definecolor{currentstroke}{rgb}{0.121569,0.466667,0.705882}%
\pgfsetstrokecolor{currentstroke}%
\pgfsetstrokeopacity{0.805138}%
\pgfsetdash{}{0pt}%
\pgfpathmoveto{\pgfqpoint{1.521114in}{1.140742in}}%
\pgfpathcurveto{\pgfqpoint{1.529350in}{1.140742in}}{\pgfqpoint{1.537250in}{1.144014in}}{\pgfqpoint{1.543074in}{1.149838in}}%
\pgfpathcurveto{\pgfqpoint{1.548898in}{1.155662in}}{\pgfqpoint{1.552170in}{1.163562in}}{\pgfqpoint{1.552170in}{1.171798in}}%
\pgfpathcurveto{\pgfqpoint{1.552170in}{1.180035in}}{\pgfqpoint{1.548898in}{1.187935in}}{\pgfqpoint{1.543074in}{1.193759in}}%
\pgfpathcurveto{\pgfqpoint{1.537250in}{1.199583in}}{\pgfqpoint{1.529350in}{1.202855in}}{\pgfqpoint{1.521114in}{1.202855in}}%
\pgfpathcurveto{\pgfqpoint{1.512877in}{1.202855in}}{\pgfqpoint{1.504977in}{1.199583in}}{\pgfqpoint{1.499153in}{1.193759in}}%
\pgfpathcurveto{\pgfqpoint{1.493329in}{1.187935in}}{\pgfqpoint{1.490057in}{1.180035in}}{\pgfqpoint{1.490057in}{1.171798in}}%
\pgfpathcurveto{\pgfqpoint{1.490057in}{1.163562in}}{\pgfqpoint{1.493329in}{1.155662in}}{\pgfqpoint{1.499153in}{1.149838in}}%
\pgfpathcurveto{\pgfqpoint{1.504977in}{1.144014in}}{\pgfqpoint{1.512877in}{1.140742in}}{\pgfqpoint{1.521114in}{1.140742in}}%
\pgfpathclose%
\pgfusepath{stroke,fill}%
\end{pgfscope}%
\begin{pgfscope}%
\pgfpathrectangle{\pgfqpoint{0.100000in}{0.212622in}}{\pgfqpoint{3.696000in}{3.696000in}}%
\pgfusepath{clip}%
\pgfsetbuttcap%
\pgfsetroundjoin%
\definecolor{currentfill}{rgb}{0.121569,0.466667,0.705882}%
\pgfsetfillcolor{currentfill}%
\pgfsetfillopacity{0.806374}%
\pgfsetlinewidth{1.003750pt}%
\definecolor{currentstroke}{rgb}{0.121569,0.466667,0.705882}%
\pgfsetstrokecolor{currentstroke}%
\pgfsetstrokeopacity{0.806374}%
\pgfsetdash{}{0pt}%
\pgfpathmoveto{\pgfqpoint{1.526682in}{1.138574in}}%
\pgfpathcurveto{\pgfqpoint{1.534918in}{1.138574in}}{\pgfqpoint{1.542818in}{1.141847in}}{\pgfqpoint{1.548642in}{1.147671in}}%
\pgfpathcurveto{\pgfqpoint{1.554466in}{1.153495in}}{\pgfqpoint{1.557738in}{1.161395in}}{\pgfqpoint{1.557738in}{1.169631in}}%
\pgfpathcurveto{\pgfqpoint{1.557738in}{1.177867in}}{\pgfqpoint{1.554466in}{1.185767in}}{\pgfqpoint{1.548642in}{1.191591in}}%
\pgfpathcurveto{\pgfqpoint{1.542818in}{1.197415in}}{\pgfqpoint{1.534918in}{1.200687in}}{\pgfqpoint{1.526682in}{1.200687in}}%
\pgfpathcurveto{\pgfqpoint{1.518446in}{1.200687in}}{\pgfqpoint{1.510546in}{1.197415in}}{\pgfqpoint{1.504722in}{1.191591in}}%
\pgfpathcurveto{\pgfqpoint{1.498898in}{1.185767in}}{\pgfqpoint{1.495625in}{1.177867in}}{\pgfqpoint{1.495625in}{1.169631in}}%
\pgfpathcurveto{\pgfqpoint{1.495625in}{1.161395in}}{\pgfqpoint{1.498898in}{1.153495in}}{\pgfqpoint{1.504722in}{1.147671in}}%
\pgfpathcurveto{\pgfqpoint{1.510546in}{1.141847in}}{\pgfqpoint{1.518446in}{1.138574in}}{\pgfqpoint{1.526682in}{1.138574in}}%
\pgfpathclose%
\pgfusepath{stroke,fill}%
\end{pgfscope}%
\begin{pgfscope}%
\pgfpathrectangle{\pgfqpoint{0.100000in}{0.212622in}}{\pgfqpoint{3.696000in}{3.696000in}}%
\pgfusepath{clip}%
\pgfsetbuttcap%
\pgfsetroundjoin%
\definecolor{currentfill}{rgb}{0.121569,0.466667,0.705882}%
\pgfsetfillcolor{currentfill}%
\pgfsetfillopacity{0.806756}%
\pgfsetlinewidth{1.003750pt}%
\definecolor{currentstroke}{rgb}{0.121569,0.466667,0.705882}%
\pgfsetstrokecolor{currentstroke}%
\pgfsetstrokeopacity{0.806756}%
\pgfsetdash{}{0pt}%
\pgfpathmoveto{\pgfqpoint{2.270881in}{1.424518in}}%
\pgfpathcurveto{\pgfqpoint{2.279117in}{1.424518in}}{\pgfqpoint{2.287018in}{1.427790in}}{\pgfqpoint{2.292841in}{1.433614in}}%
\pgfpathcurveto{\pgfqpoint{2.298665in}{1.439438in}}{\pgfqpoint{2.301938in}{1.447338in}}{\pgfqpoint{2.301938in}{1.455575in}}%
\pgfpathcurveto{\pgfqpoint{2.301938in}{1.463811in}}{\pgfqpoint{2.298665in}{1.471711in}}{\pgfqpoint{2.292841in}{1.477535in}}%
\pgfpathcurveto{\pgfqpoint{2.287018in}{1.483359in}}{\pgfqpoint{2.279117in}{1.486631in}}{\pgfqpoint{2.270881in}{1.486631in}}%
\pgfpathcurveto{\pgfqpoint{2.262645in}{1.486631in}}{\pgfqpoint{2.254745in}{1.483359in}}{\pgfqpoint{2.248921in}{1.477535in}}%
\pgfpathcurveto{\pgfqpoint{2.243097in}{1.471711in}}{\pgfqpoint{2.239825in}{1.463811in}}{\pgfqpoint{2.239825in}{1.455575in}}%
\pgfpathcurveto{\pgfqpoint{2.239825in}{1.447338in}}{\pgfqpoint{2.243097in}{1.439438in}}{\pgfqpoint{2.248921in}{1.433614in}}%
\pgfpathcurveto{\pgfqpoint{2.254745in}{1.427790in}}{\pgfqpoint{2.262645in}{1.424518in}}{\pgfqpoint{2.270881in}{1.424518in}}%
\pgfpathclose%
\pgfusepath{stroke,fill}%
\end{pgfscope}%
\begin{pgfscope}%
\pgfpathrectangle{\pgfqpoint{0.100000in}{0.212622in}}{\pgfqpoint{3.696000in}{3.696000in}}%
\pgfusepath{clip}%
\pgfsetbuttcap%
\pgfsetroundjoin%
\definecolor{currentfill}{rgb}{0.121569,0.466667,0.705882}%
\pgfsetfillcolor{currentfill}%
\pgfsetfillopacity{0.808518}%
\pgfsetlinewidth{1.003750pt}%
\definecolor{currentstroke}{rgb}{0.121569,0.466667,0.705882}%
\pgfsetstrokecolor{currentstroke}%
\pgfsetstrokeopacity{0.808518}%
\pgfsetdash{}{0pt}%
\pgfpathmoveto{\pgfqpoint{1.536929in}{1.134648in}}%
\pgfpathcurveto{\pgfqpoint{1.545165in}{1.134648in}}{\pgfqpoint{1.553065in}{1.137920in}}{\pgfqpoint{1.558889in}{1.143744in}}%
\pgfpathcurveto{\pgfqpoint{1.564713in}{1.149568in}}{\pgfqpoint{1.567986in}{1.157468in}}{\pgfqpoint{1.567986in}{1.165705in}}%
\pgfpathcurveto{\pgfqpoint{1.567986in}{1.173941in}}{\pgfqpoint{1.564713in}{1.181841in}}{\pgfqpoint{1.558889in}{1.187665in}}%
\pgfpathcurveto{\pgfqpoint{1.553065in}{1.193489in}}{\pgfqpoint{1.545165in}{1.196761in}}{\pgfqpoint{1.536929in}{1.196761in}}%
\pgfpathcurveto{\pgfqpoint{1.528693in}{1.196761in}}{\pgfqpoint{1.520793in}{1.193489in}}{\pgfqpoint{1.514969in}{1.187665in}}%
\pgfpathcurveto{\pgfqpoint{1.509145in}{1.181841in}}{\pgfqpoint{1.505873in}{1.173941in}}{\pgfqpoint{1.505873in}{1.165705in}}%
\pgfpathcurveto{\pgfqpoint{1.505873in}{1.157468in}}{\pgfqpoint{1.509145in}{1.149568in}}{\pgfqpoint{1.514969in}{1.143744in}}%
\pgfpathcurveto{\pgfqpoint{1.520793in}{1.137920in}}{\pgfqpoint{1.528693in}{1.134648in}}{\pgfqpoint{1.536929in}{1.134648in}}%
\pgfpathclose%
\pgfusepath{stroke,fill}%
\end{pgfscope}%
\begin{pgfscope}%
\pgfpathrectangle{\pgfqpoint{0.100000in}{0.212622in}}{\pgfqpoint{3.696000in}{3.696000in}}%
\pgfusepath{clip}%
\pgfsetbuttcap%
\pgfsetroundjoin%
\definecolor{currentfill}{rgb}{0.121569,0.466667,0.705882}%
\pgfsetfillcolor{currentfill}%
\pgfsetfillopacity{0.810125}%
\pgfsetlinewidth{1.003750pt}%
\definecolor{currentstroke}{rgb}{0.121569,0.466667,0.705882}%
\pgfsetstrokecolor{currentstroke}%
\pgfsetstrokeopacity{0.810125}%
\pgfsetdash{}{0pt}%
\pgfpathmoveto{\pgfqpoint{1.545246in}{1.131291in}}%
\pgfpathcurveto{\pgfqpoint{1.553482in}{1.131291in}}{\pgfqpoint{1.561382in}{1.134563in}}{\pgfqpoint{1.567206in}{1.140387in}}%
\pgfpathcurveto{\pgfqpoint{1.573030in}{1.146211in}}{\pgfqpoint{1.576302in}{1.154111in}}{\pgfqpoint{1.576302in}{1.162347in}}%
\pgfpathcurveto{\pgfqpoint{1.576302in}{1.170584in}}{\pgfqpoint{1.573030in}{1.178484in}}{\pgfqpoint{1.567206in}{1.184308in}}%
\pgfpathcurveto{\pgfqpoint{1.561382in}{1.190132in}}{\pgfqpoint{1.553482in}{1.193404in}}{\pgfqpoint{1.545246in}{1.193404in}}%
\pgfpathcurveto{\pgfqpoint{1.537009in}{1.193404in}}{\pgfqpoint{1.529109in}{1.190132in}}{\pgfqpoint{1.523285in}{1.184308in}}%
\pgfpathcurveto{\pgfqpoint{1.517462in}{1.178484in}}{\pgfqpoint{1.514189in}{1.170584in}}{\pgfqpoint{1.514189in}{1.162347in}}%
\pgfpathcurveto{\pgfqpoint{1.514189in}{1.154111in}}{\pgfqpoint{1.517462in}{1.146211in}}{\pgfqpoint{1.523285in}{1.140387in}}%
\pgfpathcurveto{\pgfqpoint{1.529109in}{1.134563in}}{\pgfqpoint{1.537009in}{1.131291in}}{\pgfqpoint{1.545246in}{1.131291in}}%
\pgfpathclose%
\pgfusepath{stroke,fill}%
\end{pgfscope}%
\begin{pgfscope}%
\pgfpathrectangle{\pgfqpoint{0.100000in}{0.212622in}}{\pgfqpoint{3.696000in}{3.696000in}}%
\pgfusepath{clip}%
\pgfsetbuttcap%
\pgfsetroundjoin%
\definecolor{currentfill}{rgb}{0.121569,0.466667,0.705882}%
\pgfsetfillcolor{currentfill}%
\pgfsetfillopacity{0.811884}%
\pgfsetlinewidth{1.003750pt}%
\definecolor{currentstroke}{rgb}{0.121569,0.466667,0.705882}%
\pgfsetstrokecolor{currentstroke}%
\pgfsetstrokeopacity{0.811884}%
\pgfsetdash{}{0pt}%
\pgfpathmoveto{\pgfqpoint{1.553020in}{1.128941in}}%
\pgfpathcurveto{\pgfqpoint{1.561256in}{1.128941in}}{\pgfqpoint{1.569156in}{1.132213in}}{\pgfqpoint{1.574980in}{1.138037in}}%
\pgfpathcurveto{\pgfqpoint{1.580804in}{1.143861in}}{\pgfqpoint{1.584076in}{1.151761in}}{\pgfqpoint{1.584076in}{1.159998in}}%
\pgfpathcurveto{\pgfqpoint{1.584076in}{1.168234in}}{\pgfqpoint{1.580804in}{1.176134in}}{\pgfqpoint{1.574980in}{1.181958in}}%
\pgfpathcurveto{\pgfqpoint{1.569156in}{1.187782in}}{\pgfqpoint{1.561256in}{1.191054in}}{\pgfqpoint{1.553020in}{1.191054in}}%
\pgfpathcurveto{\pgfqpoint{1.544783in}{1.191054in}}{\pgfqpoint{1.536883in}{1.187782in}}{\pgfqpoint{1.531059in}{1.181958in}}%
\pgfpathcurveto{\pgfqpoint{1.525235in}{1.176134in}}{\pgfqpoint{1.521963in}{1.168234in}}{\pgfqpoint{1.521963in}{1.159998in}}%
\pgfpathcurveto{\pgfqpoint{1.521963in}{1.151761in}}{\pgfqpoint{1.525235in}{1.143861in}}{\pgfqpoint{1.531059in}{1.138037in}}%
\pgfpathcurveto{\pgfqpoint{1.536883in}{1.132213in}}{\pgfqpoint{1.544783in}{1.128941in}}{\pgfqpoint{1.553020in}{1.128941in}}%
\pgfpathclose%
\pgfusepath{stroke,fill}%
\end{pgfscope}%
\begin{pgfscope}%
\pgfpathrectangle{\pgfqpoint{0.100000in}{0.212622in}}{\pgfqpoint{3.696000in}{3.696000in}}%
\pgfusepath{clip}%
\pgfsetbuttcap%
\pgfsetroundjoin%
\definecolor{currentfill}{rgb}{0.121569,0.466667,0.705882}%
\pgfsetfillcolor{currentfill}%
\pgfsetfillopacity{0.812933}%
\pgfsetlinewidth{1.003750pt}%
\definecolor{currentstroke}{rgb}{0.121569,0.466667,0.705882}%
\pgfsetstrokecolor{currentstroke}%
\pgfsetstrokeopacity{0.812933}%
\pgfsetdash{}{0pt}%
\pgfpathmoveto{\pgfqpoint{2.275297in}{1.405400in}}%
\pgfpathcurveto{\pgfqpoint{2.283533in}{1.405400in}}{\pgfqpoint{2.291433in}{1.408673in}}{\pgfqpoint{2.297257in}{1.414496in}}%
\pgfpathcurveto{\pgfqpoint{2.303081in}{1.420320in}}{\pgfqpoint{2.306353in}{1.428220in}}{\pgfqpoint{2.306353in}{1.436457in}}%
\pgfpathcurveto{\pgfqpoint{2.306353in}{1.444693in}}{\pgfqpoint{2.303081in}{1.452593in}}{\pgfqpoint{2.297257in}{1.458417in}}%
\pgfpathcurveto{\pgfqpoint{2.291433in}{1.464241in}}{\pgfqpoint{2.283533in}{1.467513in}}{\pgfqpoint{2.275297in}{1.467513in}}%
\pgfpathcurveto{\pgfqpoint{2.267061in}{1.467513in}}{\pgfqpoint{2.259161in}{1.464241in}}{\pgfqpoint{2.253337in}{1.458417in}}%
\pgfpathcurveto{\pgfqpoint{2.247513in}{1.452593in}}{\pgfqpoint{2.244240in}{1.444693in}}{\pgfqpoint{2.244240in}{1.436457in}}%
\pgfpathcurveto{\pgfqpoint{2.244240in}{1.428220in}}{\pgfqpoint{2.247513in}{1.420320in}}{\pgfqpoint{2.253337in}{1.414496in}}%
\pgfpathcurveto{\pgfqpoint{2.259161in}{1.408673in}}{\pgfqpoint{2.267061in}{1.405400in}}{\pgfqpoint{2.275297in}{1.405400in}}%
\pgfpathclose%
\pgfusepath{stroke,fill}%
\end{pgfscope}%
\begin{pgfscope}%
\pgfpathrectangle{\pgfqpoint{0.100000in}{0.212622in}}{\pgfqpoint{3.696000in}{3.696000in}}%
\pgfusepath{clip}%
\pgfsetbuttcap%
\pgfsetroundjoin%
\definecolor{currentfill}{rgb}{0.121569,0.466667,0.705882}%
\pgfsetfillcolor{currentfill}%
\pgfsetfillopacity{0.813201}%
\pgfsetlinewidth{1.003750pt}%
\definecolor{currentstroke}{rgb}{0.121569,0.466667,0.705882}%
\pgfsetstrokecolor{currentstroke}%
\pgfsetstrokeopacity{0.813201}%
\pgfsetdash{}{0pt}%
\pgfpathmoveto{\pgfqpoint{1.558712in}{1.127154in}}%
\pgfpathcurveto{\pgfqpoint{1.566948in}{1.127154in}}{\pgfqpoint{1.574848in}{1.130427in}}{\pgfqpoint{1.580672in}{1.136251in}}%
\pgfpathcurveto{\pgfqpoint{1.586496in}{1.142074in}}{\pgfqpoint{1.589768in}{1.149975in}}{\pgfqpoint{1.589768in}{1.158211in}}%
\pgfpathcurveto{\pgfqpoint{1.589768in}{1.166447in}}{\pgfqpoint{1.586496in}{1.174347in}}{\pgfqpoint{1.580672in}{1.180171in}}%
\pgfpathcurveto{\pgfqpoint{1.574848in}{1.185995in}}{\pgfqpoint{1.566948in}{1.189267in}}{\pgfqpoint{1.558712in}{1.189267in}}%
\pgfpathcurveto{\pgfqpoint{1.550476in}{1.189267in}}{\pgfqpoint{1.542575in}{1.185995in}}{\pgfqpoint{1.536752in}{1.180171in}}%
\pgfpathcurveto{\pgfqpoint{1.530928in}{1.174347in}}{\pgfqpoint{1.527655in}{1.166447in}}{\pgfqpoint{1.527655in}{1.158211in}}%
\pgfpathcurveto{\pgfqpoint{1.527655in}{1.149975in}}{\pgfqpoint{1.530928in}{1.142074in}}{\pgfqpoint{1.536752in}{1.136251in}}%
\pgfpathcurveto{\pgfqpoint{1.542575in}{1.130427in}}{\pgfqpoint{1.550476in}{1.127154in}}{\pgfqpoint{1.558712in}{1.127154in}}%
\pgfpathclose%
\pgfusepath{stroke,fill}%
\end{pgfscope}%
\begin{pgfscope}%
\pgfpathrectangle{\pgfqpoint{0.100000in}{0.212622in}}{\pgfqpoint{3.696000in}{3.696000in}}%
\pgfusepath{clip}%
\pgfsetbuttcap%
\pgfsetroundjoin%
\definecolor{currentfill}{rgb}{0.121569,0.466667,0.705882}%
\pgfsetfillcolor{currentfill}%
\pgfsetfillopacity{0.814324}%
\pgfsetlinewidth{1.003750pt}%
\definecolor{currentstroke}{rgb}{0.121569,0.466667,0.705882}%
\pgfsetstrokecolor{currentstroke}%
\pgfsetstrokeopacity{0.814324}%
\pgfsetdash{}{0pt}%
\pgfpathmoveto{\pgfqpoint{1.563732in}{1.125489in}}%
\pgfpathcurveto{\pgfqpoint{1.571968in}{1.125489in}}{\pgfqpoint{1.579868in}{1.128761in}}{\pgfqpoint{1.585692in}{1.134585in}}%
\pgfpathcurveto{\pgfqpoint{1.591516in}{1.140409in}}{\pgfqpoint{1.594789in}{1.148309in}}{\pgfqpoint{1.594789in}{1.156546in}}%
\pgfpathcurveto{\pgfqpoint{1.594789in}{1.164782in}}{\pgfqpoint{1.591516in}{1.172682in}}{\pgfqpoint{1.585692in}{1.178506in}}%
\pgfpathcurveto{\pgfqpoint{1.579868in}{1.184330in}}{\pgfqpoint{1.571968in}{1.187602in}}{\pgfqpoint{1.563732in}{1.187602in}}%
\pgfpathcurveto{\pgfqpoint{1.555496in}{1.187602in}}{\pgfqpoint{1.547596in}{1.184330in}}{\pgfqpoint{1.541772in}{1.178506in}}%
\pgfpathcurveto{\pgfqpoint{1.535948in}{1.172682in}}{\pgfqpoint{1.532676in}{1.164782in}}{\pgfqpoint{1.532676in}{1.156546in}}%
\pgfpathcurveto{\pgfqpoint{1.532676in}{1.148309in}}{\pgfqpoint{1.535948in}{1.140409in}}{\pgfqpoint{1.541772in}{1.134585in}}%
\pgfpathcurveto{\pgfqpoint{1.547596in}{1.128761in}}{\pgfqpoint{1.555496in}{1.125489in}}{\pgfqpoint{1.563732in}{1.125489in}}%
\pgfpathclose%
\pgfusepath{stroke,fill}%
\end{pgfscope}%
\begin{pgfscope}%
\pgfpathrectangle{\pgfqpoint{0.100000in}{0.212622in}}{\pgfqpoint{3.696000in}{3.696000in}}%
\pgfusepath{clip}%
\pgfsetbuttcap%
\pgfsetroundjoin%
\definecolor{currentfill}{rgb}{0.121569,0.466667,0.705882}%
\pgfsetfillcolor{currentfill}%
\pgfsetfillopacity{0.816374}%
\pgfsetlinewidth{1.003750pt}%
\definecolor{currentstroke}{rgb}{0.121569,0.466667,0.705882}%
\pgfsetstrokecolor{currentstroke}%
\pgfsetstrokeopacity{0.816374}%
\pgfsetdash{}{0pt}%
\pgfpathmoveto{\pgfqpoint{1.572929in}{1.122727in}}%
\pgfpathcurveto{\pgfqpoint{1.581165in}{1.122727in}}{\pgfqpoint{1.589065in}{1.126000in}}{\pgfqpoint{1.594889in}{1.131824in}}%
\pgfpathcurveto{\pgfqpoint{1.600713in}{1.137647in}}{\pgfqpoint{1.603986in}{1.145548in}}{\pgfqpoint{1.603986in}{1.153784in}}%
\pgfpathcurveto{\pgfqpoint{1.603986in}{1.162020in}}{\pgfqpoint{1.600713in}{1.169920in}}{\pgfqpoint{1.594889in}{1.175744in}}%
\pgfpathcurveto{\pgfqpoint{1.589065in}{1.181568in}}{\pgfqpoint{1.581165in}{1.184840in}}{\pgfqpoint{1.572929in}{1.184840in}}%
\pgfpathcurveto{\pgfqpoint{1.564693in}{1.184840in}}{\pgfqpoint{1.556793in}{1.181568in}}{\pgfqpoint{1.550969in}{1.175744in}}%
\pgfpathcurveto{\pgfqpoint{1.545145in}{1.169920in}}{\pgfqpoint{1.541873in}{1.162020in}}{\pgfqpoint{1.541873in}{1.153784in}}%
\pgfpathcurveto{\pgfqpoint{1.541873in}{1.145548in}}{\pgfqpoint{1.545145in}{1.137647in}}{\pgfqpoint{1.550969in}{1.131824in}}%
\pgfpathcurveto{\pgfqpoint{1.556793in}{1.126000in}}{\pgfqpoint{1.564693in}{1.122727in}}{\pgfqpoint{1.572929in}{1.122727in}}%
\pgfpathclose%
\pgfusepath{stroke,fill}%
\end{pgfscope}%
\begin{pgfscope}%
\pgfpathrectangle{\pgfqpoint{0.100000in}{0.212622in}}{\pgfqpoint{3.696000in}{3.696000in}}%
\pgfusepath{clip}%
\pgfsetbuttcap%
\pgfsetroundjoin%
\definecolor{currentfill}{rgb}{0.121569,0.466667,0.705882}%
\pgfsetfillcolor{currentfill}%
\pgfsetfillopacity{0.818239}%
\pgfsetlinewidth{1.003750pt}%
\definecolor{currentstroke}{rgb}{0.121569,0.466667,0.705882}%
\pgfsetstrokecolor{currentstroke}%
\pgfsetstrokeopacity{0.818239}%
\pgfsetdash{}{0pt}%
\pgfpathmoveto{\pgfqpoint{1.580579in}{1.120478in}}%
\pgfpathcurveto{\pgfqpoint{1.588815in}{1.120478in}}{\pgfqpoint{1.596715in}{1.123751in}}{\pgfqpoint{1.602539in}{1.129574in}}%
\pgfpathcurveto{\pgfqpoint{1.608363in}{1.135398in}}{\pgfqpoint{1.611636in}{1.143298in}}{\pgfqpoint{1.611636in}{1.151535in}}%
\pgfpathcurveto{\pgfqpoint{1.611636in}{1.159771in}}{\pgfqpoint{1.608363in}{1.167671in}}{\pgfqpoint{1.602539in}{1.173495in}}%
\pgfpathcurveto{\pgfqpoint{1.596715in}{1.179319in}}{\pgfqpoint{1.588815in}{1.182591in}}{\pgfqpoint{1.580579in}{1.182591in}}%
\pgfpathcurveto{\pgfqpoint{1.572343in}{1.182591in}}{\pgfqpoint{1.564443in}{1.179319in}}{\pgfqpoint{1.558619in}{1.173495in}}%
\pgfpathcurveto{\pgfqpoint{1.552795in}{1.167671in}}{\pgfqpoint{1.549523in}{1.159771in}}{\pgfqpoint{1.549523in}{1.151535in}}%
\pgfpathcurveto{\pgfqpoint{1.549523in}{1.143298in}}{\pgfqpoint{1.552795in}{1.135398in}}{\pgfqpoint{1.558619in}{1.129574in}}%
\pgfpathcurveto{\pgfqpoint{1.564443in}{1.123751in}}{\pgfqpoint{1.572343in}{1.120478in}}{\pgfqpoint{1.580579in}{1.120478in}}%
\pgfpathclose%
\pgfusepath{stroke,fill}%
\end{pgfscope}%
\begin{pgfscope}%
\pgfpathrectangle{\pgfqpoint{0.100000in}{0.212622in}}{\pgfqpoint{3.696000in}{3.696000in}}%
\pgfusepath{clip}%
\pgfsetbuttcap%
\pgfsetroundjoin%
\definecolor{currentfill}{rgb}{0.121569,0.466667,0.705882}%
\pgfsetfillcolor{currentfill}%
\pgfsetfillopacity{0.819282}%
\pgfsetlinewidth{1.003750pt}%
\definecolor{currentstroke}{rgb}{0.121569,0.466667,0.705882}%
\pgfsetstrokecolor{currentstroke}%
\pgfsetstrokeopacity{0.819282}%
\pgfsetdash{}{0pt}%
\pgfpathmoveto{\pgfqpoint{2.281812in}{1.384887in}}%
\pgfpathcurveto{\pgfqpoint{2.290048in}{1.384887in}}{\pgfqpoint{2.297948in}{1.388159in}}{\pgfqpoint{2.303772in}{1.393983in}}%
\pgfpathcurveto{\pgfqpoint{2.309596in}{1.399807in}}{\pgfqpoint{2.312868in}{1.407707in}}{\pgfqpoint{2.312868in}{1.415943in}}%
\pgfpathcurveto{\pgfqpoint{2.312868in}{1.424180in}}{\pgfqpoint{2.309596in}{1.432080in}}{\pgfqpoint{2.303772in}{1.437904in}}%
\pgfpathcurveto{\pgfqpoint{2.297948in}{1.443727in}}{\pgfqpoint{2.290048in}{1.447000in}}{\pgfqpoint{2.281812in}{1.447000in}}%
\pgfpathcurveto{\pgfqpoint{2.273575in}{1.447000in}}{\pgfqpoint{2.265675in}{1.443727in}}{\pgfqpoint{2.259852in}{1.437904in}}%
\pgfpathcurveto{\pgfqpoint{2.254028in}{1.432080in}}{\pgfqpoint{2.250755in}{1.424180in}}{\pgfqpoint{2.250755in}{1.415943in}}%
\pgfpathcurveto{\pgfqpoint{2.250755in}{1.407707in}}{\pgfqpoint{2.254028in}{1.399807in}}{\pgfqpoint{2.259852in}{1.393983in}}%
\pgfpathcurveto{\pgfqpoint{2.265675in}{1.388159in}}{\pgfqpoint{2.273575in}{1.384887in}}{\pgfqpoint{2.281812in}{1.384887in}}%
\pgfpathclose%
\pgfusepath{stroke,fill}%
\end{pgfscope}%
\begin{pgfscope}%
\pgfpathrectangle{\pgfqpoint{0.100000in}{0.212622in}}{\pgfqpoint{3.696000in}{3.696000in}}%
\pgfusepath{clip}%
\pgfsetbuttcap%
\pgfsetroundjoin%
\definecolor{currentfill}{rgb}{0.121569,0.466667,0.705882}%
\pgfsetfillcolor{currentfill}%
\pgfsetfillopacity{0.821476}%
\pgfsetlinewidth{1.003750pt}%
\definecolor{currentstroke}{rgb}{0.121569,0.466667,0.705882}%
\pgfsetstrokecolor{currentstroke}%
\pgfsetstrokeopacity{0.821476}%
\pgfsetdash{}{0pt}%
\pgfpathmoveto{\pgfqpoint{1.594505in}{1.115831in}}%
\pgfpathcurveto{\pgfqpoint{1.602741in}{1.115831in}}{\pgfqpoint{1.610641in}{1.119104in}}{\pgfqpoint{1.616465in}{1.124928in}}%
\pgfpathcurveto{\pgfqpoint{1.622289in}{1.130751in}}{\pgfqpoint{1.625561in}{1.138652in}}{\pgfqpoint{1.625561in}{1.146888in}}%
\pgfpathcurveto{\pgfqpoint{1.625561in}{1.155124in}}{\pgfqpoint{1.622289in}{1.163024in}}{\pgfqpoint{1.616465in}{1.168848in}}%
\pgfpathcurveto{\pgfqpoint{1.610641in}{1.174672in}}{\pgfqpoint{1.602741in}{1.177944in}}{\pgfqpoint{1.594505in}{1.177944in}}%
\pgfpathcurveto{\pgfqpoint{1.586268in}{1.177944in}}{\pgfqpoint{1.578368in}{1.174672in}}{\pgfqpoint{1.572544in}{1.168848in}}%
\pgfpathcurveto{\pgfqpoint{1.566720in}{1.163024in}}{\pgfqpoint{1.563448in}{1.155124in}}{\pgfqpoint{1.563448in}{1.146888in}}%
\pgfpathcurveto{\pgfqpoint{1.563448in}{1.138652in}}{\pgfqpoint{1.566720in}{1.130751in}}{\pgfqpoint{1.572544in}{1.124928in}}%
\pgfpathcurveto{\pgfqpoint{1.578368in}{1.119104in}}{\pgfqpoint{1.586268in}{1.115831in}}{\pgfqpoint{1.594505in}{1.115831in}}%
\pgfpathclose%
\pgfusepath{stroke,fill}%
\end{pgfscope}%
\begin{pgfscope}%
\pgfpathrectangle{\pgfqpoint{0.100000in}{0.212622in}}{\pgfqpoint{3.696000in}{3.696000in}}%
\pgfusepath{clip}%
\pgfsetbuttcap%
\pgfsetroundjoin%
\definecolor{currentfill}{rgb}{0.121569,0.466667,0.705882}%
\pgfsetfillcolor{currentfill}%
\pgfsetfillopacity{0.824625}%
\pgfsetlinewidth{1.003750pt}%
\definecolor{currentstroke}{rgb}{0.121569,0.466667,0.705882}%
\pgfsetstrokecolor{currentstroke}%
\pgfsetstrokeopacity{0.824625}%
\pgfsetdash{}{0pt}%
\pgfpathmoveto{\pgfqpoint{1.607691in}{1.110362in}}%
\pgfpathcurveto{\pgfqpoint{1.615927in}{1.110362in}}{\pgfqpoint{1.623827in}{1.113635in}}{\pgfqpoint{1.629651in}{1.119459in}}%
\pgfpathcurveto{\pgfqpoint{1.635475in}{1.125283in}}{\pgfqpoint{1.638747in}{1.133183in}}{\pgfqpoint{1.638747in}{1.141419in}}%
\pgfpathcurveto{\pgfqpoint{1.638747in}{1.149655in}}{\pgfqpoint{1.635475in}{1.157555in}}{\pgfqpoint{1.629651in}{1.163379in}}%
\pgfpathcurveto{\pgfqpoint{1.623827in}{1.169203in}}{\pgfqpoint{1.615927in}{1.172475in}}{\pgfqpoint{1.607691in}{1.172475in}}%
\pgfpathcurveto{\pgfqpoint{1.599454in}{1.172475in}}{\pgfqpoint{1.591554in}{1.169203in}}{\pgfqpoint{1.585730in}{1.163379in}}%
\pgfpathcurveto{\pgfqpoint{1.579906in}{1.157555in}}{\pgfqpoint{1.576634in}{1.149655in}}{\pgfqpoint{1.576634in}{1.141419in}}%
\pgfpathcurveto{\pgfqpoint{1.576634in}{1.133183in}}{\pgfqpoint{1.579906in}{1.125283in}}{\pgfqpoint{1.585730in}{1.119459in}}%
\pgfpathcurveto{\pgfqpoint{1.591554in}{1.113635in}}{\pgfqpoint{1.599454in}{1.110362in}}{\pgfqpoint{1.607691in}{1.110362in}}%
\pgfpathclose%
\pgfusepath{stroke,fill}%
\end{pgfscope}%
\begin{pgfscope}%
\pgfpathrectangle{\pgfqpoint{0.100000in}{0.212622in}}{\pgfqpoint{3.696000in}{3.696000in}}%
\pgfusepath{clip}%
\pgfsetbuttcap%
\pgfsetroundjoin%
\definecolor{currentfill}{rgb}{0.121569,0.466667,0.705882}%
\pgfsetfillcolor{currentfill}%
\pgfsetfillopacity{0.825320}%
\pgfsetlinewidth{1.003750pt}%
\definecolor{currentstroke}{rgb}{0.121569,0.466667,0.705882}%
\pgfsetstrokecolor{currentstroke}%
\pgfsetstrokeopacity{0.825320}%
\pgfsetdash{}{0pt}%
\pgfpathmoveto{\pgfqpoint{2.288869in}{1.362235in}}%
\pgfpathcurveto{\pgfqpoint{2.297105in}{1.362235in}}{\pgfqpoint{2.305005in}{1.365507in}}{\pgfqpoint{2.310829in}{1.371331in}}%
\pgfpathcurveto{\pgfqpoint{2.316653in}{1.377155in}}{\pgfqpoint{2.319925in}{1.385055in}}{\pgfqpoint{2.319925in}{1.393292in}}%
\pgfpathcurveto{\pgfqpoint{2.319925in}{1.401528in}}{\pgfqpoint{2.316653in}{1.409428in}}{\pgfqpoint{2.310829in}{1.415252in}}%
\pgfpathcurveto{\pgfqpoint{2.305005in}{1.421076in}}{\pgfqpoint{2.297105in}{1.424348in}}{\pgfqpoint{2.288869in}{1.424348in}}%
\pgfpathcurveto{\pgfqpoint{2.280632in}{1.424348in}}{\pgfqpoint{2.272732in}{1.421076in}}{\pgfqpoint{2.266908in}{1.415252in}}%
\pgfpathcurveto{\pgfqpoint{2.261084in}{1.409428in}}{\pgfqpoint{2.257812in}{1.401528in}}{\pgfqpoint{2.257812in}{1.393292in}}%
\pgfpathcurveto{\pgfqpoint{2.257812in}{1.385055in}}{\pgfqpoint{2.261084in}{1.377155in}}{\pgfqpoint{2.266908in}{1.371331in}}%
\pgfpathcurveto{\pgfqpoint{2.272732in}{1.365507in}}{\pgfqpoint{2.280632in}{1.362235in}}{\pgfqpoint{2.288869in}{1.362235in}}%
\pgfpathclose%
\pgfusepath{stroke,fill}%
\end{pgfscope}%
\begin{pgfscope}%
\pgfpathrectangle{\pgfqpoint{0.100000in}{0.212622in}}{\pgfqpoint{3.696000in}{3.696000in}}%
\pgfusepath{clip}%
\pgfsetbuttcap%
\pgfsetroundjoin%
\definecolor{currentfill}{rgb}{0.121569,0.466667,0.705882}%
\pgfsetfillcolor{currentfill}%
\pgfsetfillopacity{0.827303}%
\pgfsetlinewidth{1.003750pt}%
\definecolor{currentstroke}{rgb}{0.121569,0.466667,0.705882}%
\pgfsetstrokecolor{currentstroke}%
\pgfsetstrokeopacity{0.827303}%
\pgfsetdash{}{0pt}%
\pgfpathmoveto{\pgfqpoint{1.619747in}{1.105464in}}%
\pgfpathcurveto{\pgfqpoint{1.627983in}{1.105464in}}{\pgfqpoint{1.635883in}{1.108736in}}{\pgfqpoint{1.641707in}{1.114560in}}%
\pgfpathcurveto{\pgfqpoint{1.647531in}{1.120384in}}{\pgfqpoint{1.650803in}{1.128284in}}{\pgfqpoint{1.650803in}{1.136521in}}%
\pgfpathcurveto{\pgfqpoint{1.650803in}{1.144757in}}{\pgfqpoint{1.647531in}{1.152657in}}{\pgfqpoint{1.641707in}{1.158481in}}%
\pgfpathcurveto{\pgfqpoint{1.635883in}{1.164305in}}{\pgfqpoint{1.627983in}{1.167577in}}{\pgfqpoint{1.619747in}{1.167577in}}%
\pgfpathcurveto{\pgfqpoint{1.611511in}{1.167577in}}{\pgfqpoint{1.603610in}{1.164305in}}{\pgfqpoint{1.597787in}{1.158481in}}%
\pgfpathcurveto{\pgfqpoint{1.591963in}{1.152657in}}{\pgfqpoint{1.588690in}{1.144757in}}{\pgfqpoint{1.588690in}{1.136521in}}%
\pgfpathcurveto{\pgfqpoint{1.588690in}{1.128284in}}{\pgfqpoint{1.591963in}{1.120384in}}{\pgfqpoint{1.597787in}{1.114560in}}%
\pgfpathcurveto{\pgfqpoint{1.603610in}{1.108736in}}{\pgfqpoint{1.611511in}{1.105464in}}{\pgfqpoint{1.619747in}{1.105464in}}%
\pgfpathclose%
\pgfusepath{stroke,fill}%
\end{pgfscope}%
\begin{pgfscope}%
\pgfpathrectangle{\pgfqpoint{0.100000in}{0.212622in}}{\pgfqpoint{3.696000in}{3.696000in}}%
\pgfusepath{clip}%
\pgfsetbuttcap%
\pgfsetroundjoin%
\definecolor{currentfill}{rgb}{0.121569,0.466667,0.705882}%
\pgfsetfillcolor{currentfill}%
\pgfsetfillopacity{0.829951}%
\pgfsetlinewidth{1.003750pt}%
\definecolor{currentstroke}{rgb}{0.121569,0.466667,0.705882}%
\pgfsetstrokecolor{currentstroke}%
\pgfsetstrokeopacity{0.829951}%
\pgfsetdash{}{0pt}%
\pgfpathmoveto{\pgfqpoint{1.631557in}{1.100739in}}%
\pgfpathcurveto{\pgfqpoint{1.639793in}{1.100739in}}{\pgfqpoint{1.647693in}{1.104012in}}{\pgfqpoint{1.653517in}{1.109836in}}%
\pgfpathcurveto{\pgfqpoint{1.659341in}{1.115659in}}{\pgfqpoint{1.662614in}{1.123560in}}{\pgfqpoint{1.662614in}{1.131796in}}%
\pgfpathcurveto{\pgfqpoint{1.662614in}{1.140032in}}{\pgfqpoint{1.659341in}{1.147932in}}{\pgfqpoint{1.653517in}{1.153756in}}%
\pgfpathcurveto{\pgfqpoint{1.647693in}{1.159580in}}{\pgfqpoint{1.639793in}{1.162852in}}{\pgfqpoint{1.631557in}{1.162852in}}%
\pgfpathcurveto{\pgfqpoint{1.623321in}{1.162852in}}{\pgfqpoint{1.615421in}{1.159580in}}{\pgfqpoint{1.609597in}{1.153756in}}%
\pgfpathcurveto{\pgfqpoint{1.603773in}{1.147932in}}{\pgfqpoint{1.600501in}{1.140032in}}{\pgfqpoint{1.600501in}{1.131796in}}%
\pgfpathcurveto{\pgfqpoint{1.600501in}{1.123560in}}{\pgfqpoint{1.603773in}{1.115659in}}{\pgfqpoint{1.609597in}{1.109836in}}%
\pgfpathcurveto{\pgfqpoint{1.615421in}{1.104012in}}{\pgfqpoint{1.623321in}{1.100739in}}{\pgfqpoint{1.631557in}{1.100739in}}%
\pgfpathclose%
\pgfusepath{stroke,fill}%
\end{pgfscope}%
\begin{pgfscope}%
\pgfpathrectangle{\pgfqpoint{0.100000in}{0.212622in}}{\pgfqpoint{3.696000in}{3.696000in}}%
\pgfusepath{clip}%
\pgfsetbuttcap%
\pgfsetroundjoin%
\definecolor{currentfill}{rgb}{0.121569,0.466667,0.705882}%
\pgfsetfillcolor{currentfill}%
\pgfsetfillopacity{0.832228}%
\pgfsetlinewidth{1.003750pt}%
\definecolor{currentstroke}{rgb}{0.121569,0.466667,0.705882}%
\pgfsetstrokecolor{currentstroke}%
\pgfsetstrokeopacity{0.832228}%
\pgfsetdash{}{0pt}%
\pgfpathmoveto{\pgfqpoint{1.642247in}{1.096723in}}%
\pgfpathcurveto{\pgfqpoint{1.650483in}{1.096723in}}{\pgfqpoint{1.658383in}{1.099996in}}{\pgfqpoint{1.664207in}{1.105819in}}%
\pgfpathcurveto{\pgfqpoint{1.670031in}{1.111643in}}{\pgfqpoint{1.673303in}{1.119543in}}{\pgfqpoint{1.673303in}{1.127780in}}%
\pgfpathcurveto{\pgfqpoint{1.673303in}{1.136016in}}{\pgfqpoint{1.670031in}{1.143916in}}{\pgfqpoint{1.664207in}{1.149740in}}%
\pgfpathcurveto{\pgfqpoint{1.658383in}{1.155564in}}{\pgfqpoint{1.650483in}{1.158836in}}{\pgfqpoint{1.642247in}{1.158836in}}%
\pgfpathcurveto{\pgfqpoint{1.634011in}{1.158836in}}{\pgfqpoint{1.626111in}{1.155564in}}{\pgfqpoint{1.620287in}{1.149740in}}%
\pgfpathcurveto{\pgfqpoint{1.614463in}{1.143916in}}{\pgfqpoint{1.611190in}{1.136016in}}{\pgfqpoint{1.611190in}{1.127780in}}%
\pgfpathcurveto{\pgfqpoint{1.611190in}{1.119543in}}{\pgfqpoint{1.614463in}{1.111643in}}{\pgfqpoint{1.620287in}{1.105819in}}%
\pgfpathcurveto{\pgfqpoint{1.626111in}{1.099996in}}{\pgfqpoint{1.634011in}{1.096723in}}{\pgfqpoint{1.642247in}{1.096723in}}%
\pgfpathclose%
\pgfusepath{stroke,fill}%
\end{pgfscope}%
\begin{pgfscope}%
\pgfpathrectangle{\pgfqpoint{0.100000in}{0.212622in}}{\pgfqpoint{3.696000in}{3.696000in}}%
\pgfusepath{clip}%
\pgfsetbuttcap%
\pgfsetroundjoin%
\definecolor{currentfill}{rgb}{0.121569,0.466667,0.705882}%
\pgfsetfillcolor{currentfill}%
\pgfsetfillopacity{0.832337}%
\pgfsetlinewidth{1.003750pt}%
\definecolor{currentstroke}{rgb}{0.121569,0.466667,0.705882}%
\pgfsetstrokecolor{currentstroke}%
\pgfsetstrokeopacity{0.832337}%
\pgfsetdash{}{0pt}%
\pgfpathmoveto{\pgfqpoint{2.294440in}{1.339369in}}%
\pgfpathcurveto{\pgfqpoint{2.302676in}{1.339369in}}{\pgfqpoint{2.310576in}{1.342641in}}{\pgfqpoint{2.316400in}{1.348465in}}%
\pgfpathcurveto{\pgfqpoint{2.322224in}{1.354289in}}{\pgfqpoint{2.325496in}{1.362189in}}{\pgfqpoint{2.325496in}{1.370426in}}%
\pgfpathcurveto{\pgfqpoint{2.325496in}{1.378662in}}{\pgfqpoint{2.322224in}{1.386562in}}{\pgfqpoint{2.316400in}{1.392386in}}%
\pgfpathcurveto{\pgfqpoint{2.310576in}{1.398210in}}{\pgfqpoint{2.302676in}{1.401482in}}{\pgfqpoint{2.294440in}{1.401482in}}%
\pgfpathcurveto{\pgfqpoint{2.286204in}{1.401482in}}{\pgfqpoint{2.278304in}{1.398210in}}{\pgfqpoint{2.272480in}{1.392386in}}%
\pgfpathcurveto{\pgfqpoint{2.266656in}{1.386562in}}{\pgfqpoint{2.263383in}{1.378662in}}{\pgfqpoint{2.263383in}{1.370426in}}%
\pgfpathcurveto{\pgfqpoint{2.263383in}{1.362189in}}{\pgfqpoint{2.266656in}{1.354289in}}{\pgfqpoint{2.272480in}{1.348465in}}%
\pgfpathcurveto{\pgfqpoint{2.278304in}{1.342641in}}{\pgfqpoint{2.286204in}{1.339369in}}{\pgfqpoint{2.294440in}{1.339369in}}%
\pgfpathclose%
\pgfusepath{stroke,fill}%
\end{pgfscope}%
\begin{pgfscope}%
\pgfpathrectangle{\pgfqpoint{0.100000in}{0.212622in}}{\pgfqpoint{3.696000in}{3.696000in}}%
\pgfusepath{clip}%
\pgfsetbuttcap%
\pgfsetroundjoin%
\definecolor{currentfill}{rgb}{0.121569,0.466667,0.705882}%
\pgfsetfillcolor{currentfill}%
\pgfsetfillopacity{0.834437}%
\pgfsetlinewidth{1.003750pt}%
\definecolor{currentstroke}{rgb}{0.121569,0.466667,0.705882}%
\pgfsetstrokecolor{currentstroke}%
\pgfsetstrokeopacity{0.834437}%
\pgfsetdash{}{0pt}%
\pgfpathmoveto{\pgfqpoint{1.651936in}{1.093699in}}%
\pgfpathcurveto{\pgfqpoint{1.660173in}{1.093699in}}{\pgfqpoint{1.668073in}{1.096971in}}{\pgfqpoint{1.673897in}{1.102795in}}%
\pgfpathcurveto{\pgfqpoint{1.679721in}{1.108619in}}{\pgfqpoint{1.682993in}{1.116519in}}{\pgfqpoint{1.682993in}{1.124755in}}%
\pgfpathcurveto{\pgfqpoint{1.682993in}{1.132992in}}{\pgfqpoint{1.679721in}{1.140892in}}{\pgfqpoint{1.673897in}{1.146716in}}%
\pgfpathcurveto{\pgfqpoint{1.668073in}{1.152540in}}{\pgfqpoint{1.660173in}{1.155812in}}{\pgfqpoint{1.651936in}{1.155812in}}%
\pgfpathcurveto{\pgfqpoint{1.643700in}{1.155812in}}{\pgfqpoint{1.635800in}{1.152540in}}{\pgfqpoint{1.629976in}{1.146716in}}%
\pgfpathcurveto{\pgfqpoint{1.624152in}{1.140892in}}{\pgfqpoint{1.620880in}{1.132992in}}{\pgfqpoint{1.620880in}{1.124755in}}%
\pgfpathcurveto{\pgfqpoint{1.620880in}{1.116519in}}{\pgfqpoint{1.624152in}{1.108619in}}{\pgfqpoint{1.629976in}{1.102795in}}%
\pgfpathcurveto{\pgfqpoint{1.635800in}{1.096971in}}{\pgfqpoint{1.643700in}{1.093699in}}{\pgfqpoint{1.651936in}{1.093699in}}%
\pgfpathclose%
\pgfusepath{stroke,fill}%
\end{pgfscope}%
\begin{pgfscope}%
\pgfpathrectangle{\pgfqpoint{0.100000in}{0.212622in}}{\pgfqpoint{3.696000in}{3.696000in}}%
\pgfusepath{clip}%
\pgfsetbuttcap%
\pgfsetroundjoin%
\definecolor{currentfill}{rgb}{0.121569,0.466667,0.705882}%
\pgfsetfillcolor{currentfill}%
\pgfsetfillopacity{0.838407}%
\pgfsetlinewidth{1.003750pt}%
\definecolor{currentstroke}{rgb}{0.121569,0.466667,0.705882}%
\pgfsetstrokecolor{currentstroke}%
\pgfsetstrokeopacity{0.838407}%
\pgfsetdash{}{0pt}%
\pgfpathmoveto{\pgfqpoint{1.669778in}{1.088775in}}%
\pgfpathcurveto{\pgfqpoint{1.678014in}{1.088775in}}{\pgfqpoint{1.685914in}{1.092048in}}{\pgfqpoint{1.691738in}{1.097872in}}%
\pgfpathcurveto{\pgfqpoint{1.697562in}{1.103696in}}{\pgfqpoint{1.700834in}{1.111596in}}{\pgfqpoint{1.700834in}{1.119832in}}%
\pgfpathcurveto{\pgfqpoint{1.700834in}{1.128068in}}{\pgfqpoint{1.697562in}{1.135968in}}{\pgfqpoint{1.691738in}{1.141792in}}%
\pgfpathcurveto{\pgfqpoint{1.685914in}{1.147616in}}{\pgfqpoint{1.678014in}{1.150888in}}{\pgfqpoint{1.669778in}{1.150888in}}%
\pgfpathcurveto{\pgfqpoint{1.661542in}{1.150888in}}{\pgfqpoint{1.653641in}{1.147616in}}{\pgfqpoint{1.647818in}{1.141792in}}%
\pgfpathcurveto{\pgfqpoint{1.641994in}{1.135968in}}{\pgfqpoint{1.638721in}{1.128068in}}{\pgfqpoint{1.638721in}{1.119832in}}%
\pgfpathcurveto{\pgfqpoint{1.638721in}{1.111596in}}{\pgfqpoint{1.641994in}{1.103696in}}{\pgfqpoint{1.647818in}{1.097872in}}%
\pgfpathcurveto{\pgfqpoint{1.653641in}{1.092048in}}{\pgfqpoint{1.661542in}{1.088775in}}{\pgfqpoint{1.669778in}{1.088775in}}%
\pgfpathclose%
\pgfusepath{stroke,fill}%
\end{pgfscope}%
\begin{pgfscope}%
\pgfpathrectangle{\pgfqpoint{0.100000in}{0.212622in}}{\pgfqpoint{3.696000in}{3.696000in}}%
\pgfusepath{clip}%
\pgfsetbuttcap%
\pgfsetroundjoin%
\definecolor{currentfill}{rgb}{0.121569,0.466667,0.705882}%
\pgfsetfillcolor{currentfill}%
\pgfsetfillopacity{0.839193}%
\pgfsetlinewidth{1.003750pt}%
\definecolor{currentstroke}{rgb}{0.121569,0.466667,0.705882}%
\pgfsetstrokecolor{currentstroke}%
\pgfsetstrokeopacity{0.839193}%
\pgfsetdash{}{0pt}%
\pgfpathmoveto{\pgfqpoint{2.300775in}{1.315213in}}%
\pgfpathcurveto{\pgfqpoint{2.309011in}{1.315213in}}{\pgfqpoint{2.316911in}{1.318485in}}{\pgfqpoint{2.322735in}{1.324309in}}%
\pgfpathcurveto{\pgfqpoint{2.328559in}{1.330133in}}{\pgfqpoint{2.331831in}{1.338033in}}{\pgfqpoint{2.331831in}{1.346270in}}%
\pgfpathcurveto{\pgfqpoint{2.331831in}{1.354506in}}{\pgfqpoint{2.328559in}{1.362406in}}{\pgfqpoint{2.322735in}{1.368230in}}%
\pgfpathcurveto{\pgfqpoint{2.316911in}{1.374054in}}{\pgfqpoint{2.309011in}{1.377326in}}{\pgfqpoint{2.300775in}{1.377326in}}%
\pgfpathcurveto{\pgfqpoint{2.292539in}{1.377326in}}{\pgfqpoint{2.284639in}{1.374054in}}{\pgfqpoint{2.278815in}{1.368230in}}%
\pgfpathcurveto{\pgfqpoint{2.272991in}{1.362406in}}{\pgfqpoint{2.269718in}{1.354506in}}{\pgfqpoint{2.269718in}{1.346270in}}%
\pgfpathcurveto{\pgfqpoint{2.269718in}{1.338033in}}{\pgfqpoint{2.272991in}{1.330133in}}{\pgfqpoint{2.278815in}{1.324309in}}%
\pgfpathcurveto{\pgfqpoint{2.284639in}{1.318485in}}{\pgfqpoint{2.292539in}{1.315213in}}{\pgfqpoint{2.300775in}{1.315213in}}%
\pgfpathclose%
\pgfusepath{stroke,fill}%
\end{pgfscope}%
\begin{pgfscope}%
\pgfpathrectangle{\pgfqpoint{0.100000in}{0.212622in}}{\pgfqpoint{3.696000in}{3.696000in}}%
\pgfusepath{clip}%
\pgfsetbuttcap%
\pgfsetroundjoin%
\definecolor{currentfill}{rgb}{0.121569,0.466667,0.705882}%
\pgfsetfillcolor{currentfill}%
\pgfsetfillopacity{0.841861}%
\pgfsetlinewidth{1.003750pt}%
\definecolor{currentstroke}{rgb}{0.121569,0.466667,0.705882}%
\pgfsetstrokecolor{currentstroke}%
\pgfsetstrokeopacity{0.841861}%
\pgfsetdash{}{0pt}%
\pgfpathmoveto{\pgfqpoint{1.685912in}{1.082975in}}%
\pgfpathcurveto{\pgfqpoint{1.694148in}{1.082975in}}{\pgfqpoint{1.702048in}{1.086247in}}{\pgfqpoint{1.707872in}{1.092071in}}%
\pgfpathcurveto{\pgfqpoint{1.713696in}{1.097895in}}{\pgfqpoint{1.716968in}{1.105795in}}{\pgfqpoint{1.716968in}{1.114032in}}%
\pgfpathcurveto{\pgfqpoint{1.716968in}{1.122268in}}{\pgfqpoint{1.713696in}{1.130168in}}{\pgfqpoint{1.707872in}{1.135992in}}%
\pgfpathcurveto{\pgfqpoint{1.702048in}{1.141816in}}{\pgfqpoint{1.694148in}{1.145088in}}{\pgfqpoint{1.685912in}{1.145088in}}%
\pgfpathcurveto{\pgfqpoint{1.677675in}{1.145088in}}{\pgfqpoint{1.669775in}{1.141816in}}{\pgfqpoint{1.663951in}{1.135992in}}%
\pgfpathcurveto{\pgfqpoint{1.658127in}{1.130168in}}{\pgfqpoint{1.654855in}{1.122268in}}{\pgfqpoint{1.654855in}{1.114032in}}%
\pgfpathcurveto{\pgfqpoint{1.654855in}{1.105795in}}{\pgfqpoint{1.658127in}{1.097895in}}{\pgfqpoint{1.663951in}{1.092071in}}%
\pgfpathcurveto{\pgfqpoint{1.669775in}{1.086247in}}{\pgfqpoint{1.677675in}{1.082975in}}{\pgfqpoint{1.685912in}{1.082975in}}%
\pgfpathclose%
\pgfusepath{stroke,fill}%
\end{pgfscope}%
\begin{pgfscope}%
\pgfpathrectangle{\pgfqpoint{0.100000in}{0.212622in}}{\pgfqpoint{3.696000in}{3.696000in}}%
\pgfusepath{clip}%
\pgfsetbuttcap%
\pgfsetroundjoin%
\definecolor{currentfill}{rgb}{0.121569,0.466667,0.705882}%
\pgfsetfillcolor{currentfill}%
\pgfsetfillopacity{0.845255}%
\pgfsetlinewidth{1.003750pt}%
\definecolor{currentstroke}{rgb}{0.121569,0.466667,0.705882}%
\pgfsetstrokecolor{currentstroke}%
\pgfsetstrokeopacity{0.845255}%
\pgfsetdash{}{0pt}%
\pgfpathmoveto{\pgfqpoint{1.701708in}{1.076960in}}%
\pgfpathcurveto{\pgfqpoint{1.709944in}{1.076960in}}{\pgfqpoint{1.717844in}{1.080233in}}{\pgfqpoint{1.723668in}{1.086057in}}%
\pgfpathcurveto{\pgfqpoint{1.729492in}{1.091881in}}{\pgfqpoint{1.732764in}{1.099781in}}{\pgfqpoint{1.732764in}{1.108017in}}%
\pgfpathcurveto{\pgfqpoint{1.732764in}{1.116253in}}{\pgfqpoint{1.729492in}{1.124153in}}{\pgfqpoint{1.723668in}{1.129977in}}%
\pgfpathcurveto{\pgfqpoint{1.717844in}{1.135801in}}{\pgfqpoint{1.709944in}{1.139073in}}{\pgfqpoint{1.701708in}{1.139073in}}%
\pgfpathcurveto{\pgfqpoint{1.693472in}{1.139073in}}{\pgfqpoint{1.685572in}{1.135801in}}{\pgfqpoint{1.679748in}{1.129977in}}%
\pgfpathcurveto{\pgfqpoint{1.673924in}{1.124153in}}{\pgfqpoint{1.670651in}{1.116253in}}{\pgfqpoint{1.670651in}{1.108017in}}%
\pgfpathcurveto{\pgfqpoint{1.670651in}{1.099781in}}{\pgfqpoint{1.673924in}{1.091881in}}{\pgfqpoint{1.679748in}{1.086057in}}%
\pgfpathcurveto{\pgfqpoint{1.685572in}{1.080233in}}{\pgfqpoint{1.693472in}{1.076960in}}{\pgfqpoint{1.701708in}{1.076960in}}%
\pgfpathclose%
\pgfusepath{stroke,fill}%
\end{pgfscope}%
\begin{pgfscope}%
\pgfpathrectangle{\pgfqpoint{0.100000in}{0.212622in}}{\pgfqpoint{3.696000in}{3.696000in}}%
\pgfusepath{clip}%
\pgfsetbuttcap%
\pgfsetroundjoin%
\definecolor{currentfill}{rgb}{0.121569,0.466667,0.705882}%
\pgfsetfillcolor{currentfill}%
\pgfsetfillopacity{0.845575}%
\pgfsetlinewidth{1.003750pt}%
\definecolor{currentstroke}{rgb}{0.121569,0.466667,0.705882}%
\pgfsetstrokecolor{currentstroke}%
\pgfsetstrokeopacity{0.845575}%
\pgfsetdash{}{0pt}%
\pgfpathmoveto{\pgfqpoint{2.308324in}{1.289722in}}%
\pgfpathcurveto{\pgfqpoint{2.316560in}{1.289722in}}{\pgfqpoint{2.324460in}{1.292995in}}{\pgfqpoint{2.330284in}{1.298819in}}%
\pgfpathcurveto{\pgfqpoint{2.336108in}{1.304643in}}{\pgfqpoint{2.339381in}{1.312543in}}{\pgfqpoint{2.339381in}{1.320779in}}%
\pgfpathcurveto{\pgfqpoint{2.339381in}{1.329015in}}{\pgfqpoint{2.336108in}{1.336915in}}{\pgfqpoint{2.330284in}{1.342739in}}%
\pgfpathcurveto{\pgfqpoint{2.324460in}{1.348563in}}{\pgfqpoint{2.316560in}{1.351835in}}{\pgfqpoint{2.308324in}{1.351835in}}%
\pgfpathcurveto{\pgfqpoint{2.300088in}{1.351835in}}{\pgfqpoint{2.292188in}{1.348563in}}{\pgfqpoint{2.286364in}{1.342739in}}%
\pgfpathcurveto{\pgfqpoint{2.280540in}{1.336915in}}{\pgfqpoint{2.277268in}{1.329015in}}{\pgfqpoint{2.277268in}{1.320779in}}%
\pgfpathcurveto{\pgfqpoint{2.277268in}{1.312543in}}{\pgfqpoint{2.280540in}{1.304643in}}{\pgfqpoint{2.286364in}{1.298819in}}%
\pgfpathcurveto{\pgfqpoint{2.292188in}{1.292995in}}{\pgfqpoint{2.300088in}{1.289722in}}{\pgfqpoint{2.308324in}{1.289722in}}%
\pgfpathclose%
\pgfusepath{stroke,fill}%
\end{pgfscope}%
\begin{pgfscope}%
\pgfpathrectangle{\pgfqpoint{0.100000in}{0.212622in}}{\pgfqpoint{3.696000in}{3.696000in}}%
\pgfusepath{clip}%
\pgfsetbuttcap%
\pgfsetroundjoin%
\definecolor{currentfill}{rgb}{0.121569,0.466667,0.705882}%
\pgfsetfillcolor{currentfill}%
\pgfsetfillopacity{0.848412}%
\pgfsetlinewidth{1.003750pt}%
\definecolor{currentstroke}{rgb}{0.121569,0.466667,0.705882}%
\pgfsetstrokecolor{currentstroke}%
\pgfsetstrokeopacity{0.848412}%
\pgfsetdash{}{0pt}%
\pgfpathmoveto{\pgfqpoint{1.715851in}{1.072419in}}%
\pgfpathcurveto{\pgfqpoint{1.724087in}{1.072419in}}{\pgfqpoint{1.731987in}{1.075691in}}{\pgfqpoint{1.737811in}{1.081515in}}%
\pgfpathcurveto{\pgfqpoint{1.743635in}{1.087339in}}{\pgfqpoint{1.746907in}{1.095239in}}{\pgfqpoint{1.746907in}{1.103475in}}%
\pgfpathcurveto{\pgfqpoint{1.746907in}{1.111712in}}{\pgfqpoint{1.743635in}{1.119612in}}{\pgfqpoint{1.737811in}{1.125436in}}%
\pgfpathcurveto{\pgfqpoint{1.731987in}{1.131260in}}{\pgfqpoint{1.724087in}{1.134532in}}{\pgfqpoint{1.715851in}{1.134532in}}%
\pgfpathcurveto{\pgfqpoint{1.707615in}{1.134532in}}{\pgfqpoint{1.699715in}{1.131260in}}{\pgfqpoint{1.693891in}{1.125436in}}%
\pgfpathcurveto{\pgfqpoint{1.688067in}{1.119612in}}{\pgfqpoint{1.684794in}{1.111712in}}{\pgfqpoint{1.684794in}{1.103475in}}%
\pgfpathcurveto{\pgfqpoint{1.684794in}{1.095239in}}{\pgfqpoint{1.688067in}{1.087339in}}{\pgfqpoint{1.693891in}{1.081515in}}%
\pgfpathcurveto{\pgfqpoint{1.699715in}{1.075691in}}{\pgfqpoint{1.707615in}{1.072419in}}{\pgfqpoint{1.715851in}{1.072419in}}%
\pgfpathclose%
\pgfusepath{stroke,fill}%
\end{pgfscope}%
\begin{pgfscope}%
\pgfpathrectangle{\pgfqpoint{0.100000in}{0.212622in}}{\pgfqpoint{3.696000in}{3.696000in}}%
\pgfusepath{clip}%
\pgfsetbuttcap%
\pgfsetroundjoin%
\definecolor{currentfill}{rgb}{0.121569,0.466667,0.705882}%
\pgfsetfillcolor{currentfill}%
\pgfsetfillopacity{0.851267}%
\pgfsetlinewidth{1.003750pt}%
\definecolor{currentstroke}{rgb}{0.121569,0.466667,0.705882}%
\pgfsetstrokecolor{currentstroke}%
\pgfsetstrokeopacity{0.851267}%
\pgfsetdash{}{0pt}%
\pgfpathmoveto{\pgfqpoint{1.729567in}{1.068098in}}%
\pgfpathcurveto{\pgfqpoint{1.737803in}{1.068098in}}{\pgfqpoint{1.745703in}{1.071371in}}{\pgfqpoint{1.751527in}{1.077195in}}%
\pgfpathcurveto{\pgfqpoint{1.757351in}{1.083018in}}{\pgfqpoint{1.760623in}{1.090919in}}{\pgfqpoint{1.760623in}{1.099155in}}%
\pgfpathcurveto{\pgfqpoint{1.760623in}{1.107391in}}{\pgfqpoint{1.757351in}{1.115291in}}{\pgfqpoint{1.751527in}{1.121115in}}%
\pgfpathcurveto{\pgfqpoint{1.745703in}{1.126939in}}{\pgfqpoint{1.737803in}{1.130211in}}{\pgfqpoint{1.729567in}{1.130211in}}%
\pgfpathcurveto{\pgfqpoint{1.721330in}{1.130211in}}{\pgfqpoint{1.713430in}{1.126939in}}{\pgfqpoint{1.707606in}{1.121115in}}%
\pgfpathcurveto{\pgfqpoint{1.701782in}{1.115291in}}{\pgfqpoint{1.698510in}{1.107391in}}{\pgfqpoint{1.698510in}{1.099155in}}%
\pgfpathcurveto{\pgfqpoint{1.698510in}{1.090919in}}{\pgfqpoint{1.701782in}{1.083018in}}{\pgfqpoint{1.707606in}{1.077195in}}%
\pgfpathcurveto{\pgfqpoint{1.713430in}{1.071371in}}{\pgfqpoint{1.721330in}{1.068098in}}{\pgfqpoint{1.729567in}{1.068098in}}%
\pgfpathclose%
\pgfusepath{stroke,fill}%
\end{pgfscope}%
\begin{pgfscope}%
\pgfpathrectangle{\pgfqpoint{0.100000in}{0.212622in}}{\pgfqpoint{3.696000in}{3.696000in}}%
\pgfusepath{clip}%
\pgfsetbuttcap%
\pgfsetroundjoin%
\definecolor{currentfill}{rgb}{0.121569,0.466667,0.705882}%
\pgfsetfillcolor{currentfill}%
\pgfsetfillopacity{0.853031}%
\pgfsetlinewidth{1.003750pt}%
\definecolor{currentstroke}{rgb}{0.121569,0.466667,0.705882}%
\pgfsetstrokecolor{currentstroke}%
\pgfsetstrokeopacity{0.853031}%
\pgfsetdash{}{0pt}%
\pgfpathmoveto{\pgfqpoint{2.314129in}{1.264505in}}%
\pgfpathcurveto{\pgfqpoint{2.322366in}{1.264505in}}{\pgfqpoint{2.330266in}{1.267777in}}{\pgfqpoint{2.336090in}{1.273601in}}%
\pgfpathcurveto{\pgfqpoint{2.341914in}{1.279425in}}{\pgfqpoint{2.345186in}{1.287325in}}{\pgfqpoint{2.345186in}{1.295562in}}%
\pgfpathcurveto{\pgfqpoint{2.345186in}{1.303798in}}{\pgfqpoint{2.341914in}{1.311698in}}{\pgfqpoint{2.336090in}{1.317522in}}%
\pgfpathcurveto{\pgfqpoint{2.330266in}{1.323346in}}{\pgfqpoint{2.322366in}{1.326618in}}{\pgfqpoint{2.314129in}{1.326618in}}%
\pgfpathcurveto{\pgfqpoint{2.305893in}{1.326618in}}{\pgfqpoint{2.297993in}{1.323346in}}{\pgfqpoint{2.292169in}{1.317522in}}%
\pgfpathcurveto{\pgfqpoint{2.286345in}{1.311698in}}{\pgfqpoint{2.283073in}{1.303798in}}{\pgfqpoint{2.283073in}{1.295562in}}%
\pgfpathcurveto{\pgfqpoint{2.283073in}{1.287325in}}{\pgfqpoint{2.286345in}{1.279425in}}{\pgfqpoint{2.292169in}{1.273601in}}%
\pgfpathcurveto{\pgfqpoint{2.297993in}{1.267777in}}{\pgfqpoint{2.305893in}{1.264505in}}{\pgfqpoint{2.314129in}{1.264505in}}%
\pgfpathclose%
\pgfusepath{stroke,fill}%
\end{pgfscope}%
\begin{pgfscope}%
\pgfpathrectangle{\pgfqpoint{0.100000in}{0.212622in}}{\pgfqpoint{3.696000in}{3.696000in}}%
\pgfusepath{clip}%
\pgfsetbuttcap%
\pgfsetroundjoin%
\definecolor{currentfill}{rgb}{0.121569,0.466667,0.705882}%
\pgfsetfillcolor{currentfill}%
\pgfsetfillopacity{0.854106}%
\pgfsetlinewidth{1.003750pt}%
\definecolor{currentstroke}{rgb}{0.121569,0.466667,0.705882}%
\pgfsetstrokecolor{currentstroke}%
\pgfsetstrokeopacity{0.854106}%
\pgfsetdash{}{0pt}%
\pgfpathmoveto{\pgfqpoint{1.742974in}{1.063901in}}%
\pgfpathcurveto{\pgfqpoint{1.751210in}{1.063901in}}{\pgfqpoint{1.759110in}{1.067173in}}{\pgfqpoint{1.764934in}{1.072997in}}%
\pgfpathcurveto{\pgfqpoint{1.770758in}{1.078821in}}{\pgfqpoint{1.774030in}{1.086721in}}{\pgfqpoint{1.774030in}{1.094957in}}%
\pgfpathcurveto{\pgfqpoint{1.774030in}{1.103194in}}{\pgfqpoint{1.770758in}{1.111094in}}{\pgfqpoint{1.764934in}{1.116918in}}%
\pgfpathcurveto{\pgfqpoint{1.759110in}{1.122741in}}{\pgfqpoint{1.751210in}{1.126014in}}{\pgfqpoint{1.742974in}{1.126014in}}%
\pgfpathcurveto{\pgfqpoint{1.734738in}{1.126014in}}{\pgfqpoint{1.726838in}{1.122741in}}{\pgfqpoint{1.721014in}{1.116918in}}%
\pgfpathcurveto{\pgfqpoint{1.715190in}{1.111094in}}{\pgfqpoint{1.711917in}{1.103194in}}{\pgfqpoint{1.711917in}{1.094957in}}%
\pgfpathcurveto{\pgfqpoint{1.711917in}{1.086721in}}{\pgfqpoint{1.715190in}{1.078821in}}{\pgfqpoint{1.721014in}{1.072997in}}%
\pgfpathcurveto{\pgfqpoint{1.726838in}{1.067173in}}{\pgfqpoint{1.734738in}{1.063901in}}{\pgfqpoint{1.742974in}{1.063901in}}%
\pgfpathclose%
\pgfusepath{stroke,fill}%
\end{pgfscope}%
\begin{pgfscope}%
\pgfpathrectangle{\pgfqpoint{0.100000in}{0.212622in}}{\pgfqpoint{3.696000in}{3.696000in}}%
\pgfusepath{clip}%
\pgfsetbuttcap%
\pgfsetroundjoin%
\definecolor{currentfill}{rgb}{0.121569,0.466667,0.705882}%
\pgfsetfillcolor{currentfill}%
\pgfsetfillopacity{0.856644}%
\pgfsetlinewidth{1.003750pt}%
\definecolor{currentstroke}{rgb}{0.121569,0.466667,0.705882}%
\pgfsetstrokecolor{currentstroke}%
\pgfsetstrokeopacity{0.856644}%
\pgfsetdash{}{0pt}%
\pgfpathmoveto{\pgfqpoint{1.754419in}{1.060556in}}%
\pgfpathcurveto{\pgfqpoint{1.762655in}{1.060556in}}{\pgfqpoint{1.770555in}{1.063828in}}{\pgfqpoint{1.776379in}{1.069652in}}%
\pgfpathcurveto{\pgfqpoint{1.782203in}{1.075476in}}{\pgfqpoint{1.785475in}{1.083376in}}{\pgfqpoint{1.785475in}{1.091612in}}%
\pgfpathcurveto{\pgfqpoint{1.785475in}{1.099848in}}{\pgfqpoint{1.782203in}{1.107748in}}{\pgfqpoint{1.776379in}{1.113572in}}%
\pgfpathcurveto{\pgfqpoint{1.770555in}{1.119396in}}{\pgfqpoint{1.762655in}{1.122669in}}{\pgfqpoint{1.754419in}{1.122669in}}%
\pgfpathcurveto{\pgfqpoint{1.746182in}{1.122669in}}{\pgfqpoint{1.738282in}{1.119396in}}{\pgfqpoint{1.732458in}{1.113572in}}%
\pgfpathcurveto{\pgfqpoint{1.726634in}{1.107748in}}{\pgfqpoint{1.723362in}{1.099848in}}{\pgfqpoint{1.723362in}{1.091612in}}%
\pgfpathcurveto{\pgfqpoint{1.723362in}{1.083376in}}{\pgfqpoint{1.726634in}{1.075476in}}{\pgfqpoint{1.732458in}{1.069652in}}%
\pgfpathcurveto{\pgfqpoint{1.738282in}{1.063828in}}{\pgfqpoint{1.746182in}{1.060556in}}{\pgfqpoint{1.754419in}{1.060556in}}%
\pgfpathclose%
\pgfusepath{stroke,fill}%
\end{pgfscope}%
\begin{pgfscope}%
\pgfpathrectangle{\pgfqpoint{0.100000in}{0.212622in}}{\pgfqpoint{3.696000in}{3.696000in}}%
\pgfusepath{clip}%
\pgfsetbuttcap%
\pgfsetroundjoin%
\definecolor{currentfill}{rgb}{0.121569,0.466667,0.705882}%
\pgfsetfillcolor{currentfill}%
\pgfsetfillopacity{0.859006}%
\pgfsetlinewidth{1.003750pt}%
\definecolor{currentstroke}{rgb}{0.121569,0.466667,0.705882}%
\pgfsetstrokecolor{currentstroke}%
\pgfsetstrokeopacity{0.859006}%
\pgfsetdash{}{0pt}%
\pgfpathmoveto{\pgfqpoint{1.765604in}{1.057092in}}%
\pgfpathcurveto{\pgfqpoint{1.773840in}{1.057092in}}{\pgfqpoint{1.781740in}{1.060364in}}{\pgfqpoint{1.787564in}{1.066188in}}%
\pgfpathcurveto{\pgfqpoint{1.793388in}{1.072012in}}{\pgfqpoint{1.796660in}{1.079912in}}{\pgfqpoint{1.796660in}{1.088148in}}%
\pgfpathcurveto{\pgfqpoint{1.796660in}{1.096384in}}{\pgfqpoint{1.793388in}{1.104284in}}{\pgfqpoint{1.787564in}{1.110108in}}%
\pgfpathcurveto{\pgfqpoint{1.781740in}{1.115932in}}{\pgfqpoint{1.773840in}{1.119205in}}{\pgfqpoint{1.765604in}{1.119205in}}%
\pgfpathcurveto{\pgfqpoint{1.757368in}{1.119205in}}{\pgfqpoint{1.749468in}{1.115932in}}{\pgfqpoint{1.743644in}{1.110108in}}%
\pgfpathcurveto{\pgfqpoint{1.737820in}{1.104284in}}{\pgfqpoint{1.734547in}{1.096384in}}{\pgfqpoint{1.734547in}{1.088148in}}%
\pgfpathcurveto{\pgfqpoint{1.734547in}{1.079912in}}{\pgfqpoint{1.737820in}{1.072012in}}{\pgfqpoint{1.743644in}{1.066188in}}%
\pgfpathcurveto{\pgfqpoint{1.749468in}{1.060364in}}{\pgfqpoint{1.757368in}{1.057092in}}{\pgfqpoint{1.765604in}{1.057092in}}%
\pgfpathclose%
\pgfusepath{stroke,fill}%
\end{pgfscope}%
\begin{pgfscope}%
\pgfpathrectangle{\pgfqpoint{0.100000in}{0.212622in}}{\pgfqpoint{3.696000in}{3.696000in}}%
\pgfusepath{clip}%
\pgfsetbuttcap%
\pgfsetroundjoin%
\definecolor{currentfill}{rgb}{0.121569,0.466667,0.705882}%
\pgfsetfillcolor{currentfill}%
\pgfsetfillopacity{0.860669}%
\pgfsetlinewidth{1.003750pt}%
\definecolor{currentstroke}{rgb}{0.121569,0.466667,0.705882}%
\pgfsetstrokecolor{currentstroke}%
\pgfsetstrokeopacity{0.860669}%
\pgfsetdash{}{0pt}%
\pgfpathmoveto{\pgfqpoint{2.320609in}{1.238862in}}%
\pgfpathcurveto{\pgfqpoint{2.328845in}{1.238862in}}{\pgfqpoint{2.336745in}{1.242134in}}{\pgfqpoint{2.342569in}{1.247958in}}%
\pgfpathcurveto{\pgfqpoint{2.348393in}{1.253782in}}{\pgfqpoint{2.351665in}{1.261682in}}{\pgfqpoint{2.351665in}{1.269918in}}%
\pgfpathcurveto{\pgfqpoint{2.351665in}{1.278154in}}{\pgfqpoint{2.348393in}{1.286054in}}{\pgfqpoint{2.342569in}{1.291878in}}%
\pgfpathcurveto{\pgfqpoint{2.336745in}{1.297702in}}{\pgfqpoint{2.328845in}{1.300975in}}{\pgfqpoint{2.320609in}{1.300975in}}%
\pgfpathcurveto{\pgfqpoint{2.312373in}{1.300975in}}{\pgfqpoint{2.304473in}{1.297702in}}{\pgfqpoint{2.298649in}{1.291878in}}%
\pgfpathcurveto{\pgfqpoint{2.292825in}{1.286054in}}{\pgfqpoint{2.289552in}{1.278154in}}{\pgfqpoint{2.289552in}{1.269918in}}%
\pgfpathcurveto{\pgfqpoint{2.289552in}{1.261682in}}{\pgfqpoint{2.292825in}{1.253782in}}{\pgfqpoint{2.298649in}{1.247958in}}%
\pgfpathcurveto{\pgfqpoint{2.304473in}{1.242134in}}{\pgfqpoint{2.312373in}{1.238862in}}{\pgfqpoint{2.320609in}{1.238862in}}%
\pgfpathclose%
\pgfusepath{stroke,fill}%
\end{pgfscope}%
\begin{pgfscope}%
\pgfpathrectangle{\pgfqpoint{0.100000in}{0.212622in}}{\pgfqpoint{3.696000in}{3.696000in}}%
\pgfusepath{clip}%
\pgfsetbuttcap%
\pgfsetroundjoin%
\definecolor{currentfill}{rgb}{0.121569,0.466667,0.705882}%
\pgfsetfillcolor{currentfill}%
\pgfsetfillopacity{0.861125}%
\pgfsetlinewidth{1.003750pt}%
\definecolor{currentstroke}{rgb}{0.121569,0.466667,0.705882}%
\pgfsetstrokecolor{currentstroke}%
\pgfsetstrokeopacity{0.861125}%
\pgfsetdash{}{0pt}%
\pgfpathmoveto{\pgfqpoint{1.775242in}{1.054312in}}%
\pgfpathcurveto{\pgfqpoint{1.783478in}{1.054312in}}{\pgfqpoint{1.791378in}{1.057584in}}{\pgfqpoint{1.797202in}{1.063408in}}%
\pgfpathcurveto{\pgfqpoint{1.803026in}{1.069232in}}{\pgfqpoint{1.806298in}{1.077132in}}{\pgfqpoint{1.806298in}{1.085368in}}%
\pgfpathcurveto{\pgfqpoint{1.806298in}{1.093605in}}{\pgfqpoint{1.803026in}{1.101505in}}{\pgfqpoint{1.797202in}{1.107329in}}%
\pgfpathcurveto{\pgfqpoint{1.791378in}{1.113153in}}{\pgfqpoint{1.783478in}{1.116425in}}{\pgfqpoint{1.775242in}{1.116425in}}%
\pgfpathcurveto{\pgfqpoint{1.767006in}{1.116425in}}{\pgfqpoint{1.759105in}{1.113153in}}{\pgfqpoint{1.753282in}{1.107329in}}%
\pgfpathcurveto{\pgfqpoint{1.747458in}{1.101505in}}{\pgfqpoint{1.744185in}{1.093605in}}{\pgfqpoint{1.744185in}{1.085368in}}%
\pgfpathcurveto{\pgfqpoint{1.744185in}{1.077132in}}{\pgfqpoint{1.747458in}{1.069232in}}{\pgfqpoint{1.753282in}{1.063408in}}%
\pgfpathcurveto{\pgfqpoint{1.759105in}{1.057584in}}{\pgfqpoint{1.767006in}{1.054312in}}{\pgfqpoint{1.775242in}{1.054312in}}%
\pgfpathclose%
\pgfusepath{stroke,fill}%
\end{pgfscope}%
\begin{pgfscope}%
\pgfpathrectangle{\pgfqpoint{0.100000in}{0.212622in}}{\pgfqpoint{3.696000in}{3.696000in}}%
\pgfusepath{clip}%
\pgfsetbuttcap%
\pgfsetroundjoin%
\definecolor{currentfill}{rgb}{0.121569,0.466667,0.705882}%
\pgfsetfillcolor{currentfill}%
\pgfsetfillopacity{0.862970}%
\pgfsetlinewidth{1.003750pt}%
\definecolor{currentstroke}{rgb}{0.121569,0.466667,0.705882}%
\pgfsetstrokecolor{currentstroke}%
\pgfsetstrokeopacity{0.862970}%
\pgfsetdash{}{0pt}%
\pgfpathmoveto{\pgfqpoint{1.784139in}{1.051677in}}%
\pgfpathcurveto{\pgfqpoint{1.792375in}{1.051677in}}{\pgfqpoint{1.800276in}{1.054949in}}{\pgfqpoint{1.806099in}{1.060773in}}%
\pgfpathcurveto{\pgfqpoint{1.811923in}{1.066597in}}{\pgfqpoint{1.815196in}{1.074497in}}{\pgfqpoint{1.815196in}{1.082733in}}%
\pgfpathcurveto{\pgfqpoint{1.815196in}{1.090970in}}{\pgfqpoint{1.811923in}{1.098870in}}{\pgfqpoint{1.806099in}{1.104694in}}%
\pgfpathcurveto{\pgfqpoint{1.800276in}{1.110517in}}{\pgfqpoint{1.792375in}{1.113790in}}{\pgfqpoint{1.784139in}{1.113790in}}%
\pgfpathcurveto{\pgfqpoint{1.775903in}{1.113790in}}{\pgfqpoint{1.768003in}{1.110517in}}{\pgfqpoint{1.762179in}{1.104694in}}%
\pgfpathcurveto{\pgfqpoint{1.756355in}{1.098870in}}{\pgfqpoint{1.753083in}{1.090970in}}{\pgfqpoint{1.753083in}{1.082733in}}%
\pgfpathcurveto{\pgfqpoint{1.753083in}{1.074497in}}{\pgfqpoint{1.756355in}{1.066597in}}{\pgfqpoint{1.762179in}{1.060773in}}%
\pgfpathcurveto{\pgfqpoint{1.768003in}{1.054949in}}{\pgfqpoint{1.775903in}{1.051677in}}{\pgfqpoint{1.784139in}{1.051677in}}%
\pgfpathclose%
\pgfusepath{stroke,fill}%
\end{pgfscope}%
\begin{pgfscope}%
\pgfpathrectangle{\pgfqpoint{0.100000in}{0.212622in}}{\pgfqpoint{3.696000in}{3.696000in}}%
\pgfusepath{clip}%
\pgfsetbuttcap%
\pgfsetroundjoin%
\definecolor{currentfill}{rgb}{0.121569,0.466667,0.705882}%
\pgfsetfillcolor{currentfill}%
\pgfsetfillopacity{0.864814}%
\pgfsetlinewidth{1.003750pt}%
\definecolor{currentstroke}{rgb}{0.121569,0.466667,0.705882}%
\pgfsetstrokecolor{currentstroke}%
\pgfsetstrokeopacity{0.864814}%
\pgfsetdash{}{0pt}%
\pgfpathmoveto{\pgfqpoint{1.792719in}{1.049148in}}%
\pgfpathcurveto{\pgfqpoint{1.800955in}{1.049148in}}{\pgfqpoint{1.808856in}{1.052420in}}{\pgfqpoint{1.814679in}{1.058244in}}%
\pgfpathcurveto{\pgfqpoint{1.820503in}{1.064068in}}{\pgfqpoint{1.823776in}{1.071968in}}{\pgfqpoint{1.823776in}{1.080205in}}%
\pgfpathcurveto{\pgfqpoint{1.823776in}{1.088441in}}{\pgfqpoint{1.820503in}{1.096341in}}{\pgfqpoint{1.814679in}{1.102165in}}%
\pgfpathcurveto{\pgfqpoint{1.808856in}{1.107989in}}{\pgfqpoint{1.800955in}{1.111261in}}{\pgfqpoint{1.792719in}{1.111261in}}%
\pgfpathcurveto{\pgfqpoint{1.784483in}{1.111261in}}{\pgfqpoint{1.776583in}{1.107989in}}{\pgfqpoint{1.770759in}{1.102165in}}%
\pgfpathcurveto{\pgfqpoint{1.764935in}{1.096341in}}{\pgfqpoint{1.761663in}{1.088441in}}{\pgfqpoint{1.761663in}{1.080205in}}%
\pgfpathcurveto{\pgfqpoint{1.761663in}{1.071968in}}{\pgfqpoint{1.764935in}{1.064068in}}{\pgfqpoint{1.770759in}{1.058244in}}%
\pgfpathcurveto{\pgfqpoint{1.776583in}{1.052420in}}{\pgfqpoint{1.784483in}{1.049148in}}{\pgfqpoint{1.792719in}{1.049148in}}%
\pgfpathclose%
\pgfusepath{stroke,fill}%
\end{pgfscope}%
\begin{pgfscope}%
\pgfpathrectangle{\pgfqpoint{0.100000in}{0.212622in}}{\pgfqpoint{3.696000in}{3.696000in}}%
\pgfusepath{clip}%
\pgfsetbuttcap%
\pgfsetroundjoin%
\definecolor{currentfill}{rgb}{0.121569,0.466667,0.705882}%
\pgfsetfillcolor{currentfill}%
\pgfsetfillopacity{0.866382}%
\pgfsetlinewidth{1.003750pt}%
\definecolor{currentstroke}{rgb}{0.121569,0.466667,0.705882}%
\pgfsetstrokecolor{currentstroke}%
\pgfsetstrokeopacity{0.866382}%
\pgfsetdash{}{0pt}%
\pgfpathmoveto{\pgfqpoint{1.799422in}{1.047472in}}%
\pgfpathcurveto{\pgfqpoint{1.807658in}{1.047472in}}{\pgfqpoint{1.815559in}{1.050744in}}{\pgfqpoint{1.821382in}{1.056568in}}%
\pgfpathcurveto{\pgfqpoint{1.827206in}{1.062392in}}{\pgfqpoint{1.830479in}{1.070292in}}{\pgfqpoint{1.830479in}{1.078528in}}%
\pgfpathcurveto{\pgfqpoint{1.830479in}{1.086764in}}{\pgfqpoint{1.827206in}{1.094664in}}{\pgfqpoint{1.821382in}{1.100488in}}%
\pgfpathcurveto{\pgfqpoint{1.815559in}{1.106312in}}{\pgfqpoint{1.807658in}{1.109585in}}{\pgfqpoint{1.799422in}{1.109585in}}%
\pgfpathcurveto{\pgfqpoint{1.791186in}{1.109585in}}{\pgfqpoint{1.783286in}{1.106312in}}{\pgfqpoint{1.777462in}{1.100488in}}%
\pgfpathcurveto{\pgfqpoint{1.771638in}{1.094664in}}{\pgfqpoint{1.768366in}{1.086764in}}{\pgfqpoint{1.768366in}{1.078528in}}%
\pgfpathcurveto{\pgfqpoint{1.768366in}{1.070292in}}{\pgfqpoint{1.771638in}{1.062392in}}{\pgfqpoint{1.777462in}{1.056568in}}%
\pgfpathcurveto{\pgfqpoint{1.783286in}{1.050744in}}{\pgfqpoint{1.791186in}{1.047472in}}{\pgfqpoint{1.799422in}{1.047472in}}%
\pgfpathclose%
\pgfusepath{stroke,fill}%
\end{pgfscope}%
\begin{pgfscope}%
\pgfpathrectangle{\pgfqpoint{0.100000in}{0.212622in}}{\pgfqpoint{3.696000in}{3.696000in}}%
\pgfusepath{clip}%
\pgfsetbuttcap%
\pgfsetroundjoin%
\definecolor{currentfill}{rgb}{0.121569,0.466667,0.705882}%
\pgfsetfillcolor{currentfill}%
\pgfsetfillopacity{0.867845}%
\pgfsetlinewidth{1.003750pt}%
\definecolor{currentstroke}{rgb}{0.121569,0.466667,0.705882}%
\pgfsetstrokecolor{currentstroke}%
\pgfsetstrokeopacity{0.867845}%
\pgfsetdash{}{0pt}%
\pgfpathmoveto{\pgfqpoint{1.805761in}{1.045213in}}%
\pgfpathcurveto{\pgfqpoint{1.813998in}{1.045213in}}{\pgfqpoint{1.821898in}{1.048486in}}{\pgfqpoint{1.827722in}{1.054309in}}%
\pgfpathcurveto{\pgfqpoint{1.833545in}{1.060133in}}{\pgfqpoint{1.836818in}{1.068033in}}{\pgfqpoint{1.836818in}{1.076270in}}%
\pgfpathcurveto{\pgfqpoint{1.836818in}{1.084506in}}{\pgfqpoint{1.833545in}{1.092406in}}{\pgfqpoint{1.827722in}{1.098230in}}%
\pgfpathcurveto{\pgfqpoint{1.821898in}{1.104054in}}{\pgfqpoint{1.813998in}{1.107326in}}{\pgfqpoint{1.805761in}{1.107326in}}%
\pgfpathcurveto{\pgfqpoint{1.797525in}{1.107326in}}{\pgfqpoint{1.789625in}{1.104054in}}{\pgfqpoint{1.783801in}{1.098230in}}%
\pgfpathcurveto{\pgfqpoint{1.777977in}{1.092406in}}{\pgfqpoint{1.774705in}{1.084506in}}{\pgfqpoint{1.774705in}{1.076270in}}%
\pgfpathcurveto{\pgfqpoint{1.774705in}{1.068033in}}{\pgfqpoint{1.777977in}{1.060133in}}{\pgfqpoint{1.783801in}{1.054309in}}%
\pgfpathcurveto{\pgfqpoint{1.789625in}{1.048486in}}{\pgfqpoint{1.797525in}{1.045213in}}{\pgfqpoint{1.805761in}{1.045213in}}%
\pgfpathclose%
\pgfusepath{stroke,fill}%
\end{pgfscope}%
\begin{pgfscope}%
\pgfpathrectangle{\pgfqpoint{0.100000in}{0.212622in}}{\pgfqpoint{3.696000in}{3.696000in}}%
\pgfusepath{clip}%
\pgfsetbuttcap%
\pgfsetroundjoin%
\definecolor{currentfill}{rgb}{0.121569,0.466667,0.705882}%
\pgfsetfillcolor{currentfill}%
\pgfsetfillopacity{0.867984}%
\pgfsetlinewidth{1.003750pt}%
\definecolor{currentstroke}{rgb}{0.121569,0.466667,0.705882}%
\pgfsetstrokecolor{currentstroke}%
\pgfsetstrokeopacity{0.867984}%
\pgfsetdash{}{0pt}%
\pgfpathmoveto{\pgfqpoint{2.329155in}{1.212663in}}%
\pgfpathcurveto{\pgfqpoint{2.337391in}{1.212663in}}{\pgfqpoint{2.345291in}{1.215935in}}{\pgfqpoint{2.351115in}{1.221759in}}%
\pgfpathcurveto{\pgfqpoint{2.356939in}{1.227583in}}{\pgfqpoint{2.360211in}{1.235483in}}{\pgfqpoint{2.360211in}{1.243719in}}%
\pgfpathcurveto{\pgfqpoint{2.360211in}{1.251955in}}{\pgfqpoint{2.356939in}{1.259855in}}{\pgfqpoint{2.351115in}{1.265679in}}%
\pgfpathcurveto{\pgfqpoint{2.345291in}{1.271503in}}{\pgfqpoint{2.337391in}{1.274776in}}{\pgfqpoint{2.329155in}{1.274776in}}%
\pgfpathcurveto{\pgfqpoint{2.320919in}{1.274776in}}{\pgfqpoint{2.313018in}{1.271503in}}{\pgfqpoint{2.307195in}{1.265679in}}%
\pgfpathcurveto{\pgfqpoint{2.301371in}{1.259855in}}{\pgfqpoint{2.298098in}{1.251955in}}{\pgfqpoint{2.298098in}{1.243719in}}%
\pgfpathcurveto{\pgfqpoint{2.298098in}{1.235483in}}{\pgfqpoint{2.301371in}{1.227583in}}{\pgfqpoint{2.307195in}{1.221759in}}%
\pgfpathcurveto{\pgfqpoint{2.313018in}{1.215935in}}{\pgfqpoint{2.320919in}{1.212663in}}{\pgfqpoint{2.329155in}{1.212663in}}%
\pgfpathclose%
\pgfusepath{stroke,fill}%
\end{pgfscope}%
\begin{pgfscope}%
\pgfpathrectangle{\pgfqpoint{0.100000in}{0.212622in}}{\pgfqpoint{3.696000in}{3.696000in}}%
\pgfusepath{clip}%
\pgfsetbuttcap%
\pgfsetroundjoin%
\definecolor{currentfill}{rgb}{0.121569,0.466667,0.705882}%
\pgfsetfillcolor{currentfill}%
\pgfsetfillopacity{0.868941}%
\pgfsetlinewidth{1.003750pt}%
\definecolor{currentstroke}{rgb}{0.121569,0.466667,0.705882}%
\pgfsetstrokecolor{currentstroke}%
\pgfsetstrokeopacity{0.868941}%
\pgfsetdash{}{0pt}%
\pgfpathmoveto{\pgfqpoint{1.811365in}{1.043150in}}%
\pgfpathcurveto{\pgfqpoint{1.819602in}{1.043150in}}{\pgfqpoint{1.827502in}{1.046423in}}{\pgfqpoint{1.833326in}{1.052247in}}%
\pgfpathcurveto{\pgfqpoint{1.839150in}{1.058071in}}{\pgfqpoint{1.842422in}{1.065971in}}{\pgfqpoint{1.842422in}{1.074207in}}%
\pgfpathcurveto{\pgfqpoint{1.842422in}{1.082443in}}{\pgfqpoint{1.839150in}{1.090343in}}{\pgfqpoint{1.833326in}{1.096167in}}%
\pgfpathcurveto{\pgfqpoint{1.827502in}{1.101991in}}{\pgfqpoint{1.819602in}{1.105263in}}{\pgfqpoint{1.811365in}{1.105263in}}%
\pgfpathcurveto{\pgfqpoint{1.803129in}{1.105263in}}{\pgfqpoint{1.795229in}{1.101991in}}{\pgfqpoint{1.789405in}{1.096167in}}%
\pgfpathcurveto{\pgfqpoint{1.783581in}{1.090343in}}{\pgfqpoint{1.780309in}{1.082443in}}{\pgfqpoint{1.780309in}{1.074207in}}%
\pgfpathcurveto{\pgfqpoint{1.780309in}{1.065971in}}{\pgfqpoint{1.783581in}{1.058071in}}{\pgfqpoint{1.789405in}{1.052247in}}%
\pgfpathcurveto{\pgfqpoint{1.795229in}{1.046423in}}{\pgfqpoint{1.803129in}{1.043150in}}{\pgfqpoint{1.811365in}{1.043150in}}%
\pgfpathclose%
\pgfusepath{stroke,fill}%
\end{pgfscope}%
\begin{pgfscope}%
\pgfpathrectangle{\pgfqpoint{0.100000in}{0.212622in}}{\pgfqpoint{3.696000in}{3.696000in}}%
\pgfusepath{clip}%
\pgfsetbuttcap%
\pgfsetroundjoin%
\definecolor{currentfill}{rgb}{0.121569,0.466667,0.705882}%
\pgfsetfillcolor{currentfill}%
\pgfsetfillopacity{0.869810}%
\pgfsetlinewidth{1.003750pt}%
\definecolor{currentstroke}{rgb}{0.121569,0.466667,0.705882}%
\pgfsetstrokecolor{currentstroke}%
\pgfsetstrokeopacity{0.869810}%
\pgfsetdash{}{0pt}%
\pgfpathmoveto{\pgfqpoint{1.815344in}{1.041690in}}%
\pgfpathcurveto{\pgfqpoint{1.823581in}{1.041690in}}{\pgfqpoint{1.831481in}{1.044962in}}{\pgfqpoint{1.837305in}{1.050786in}}%
\pgfpathcurveto{\pgfqpoint{1.843129in}{1.056610in}}{\pgfqpoint{1.846401in}{1.064510in}}{\pgfqpoint{1.846401in}{1.072746in}}%
\pgfpathcurveto{\pgfqpoint{1.846401in}{1.080983in}}{\pgfqpoint{1.843129in}{1.088883in}}{\pgfqpoint{1.837305in}{1.094706in}}%
\pgfpathcurveto{\pgfqpoint{1.831481in}{1.100530in}}{\pgfqpoint{1.823581in}{1.103803in}}{\pgfqpoint{1.815344in}{1.103803in}}%
\pgfpathcurveto{\pgfqpoint{1.807108in}{1.103803in}}{\pgfqpoint{1.799208in}{1.100530in}}{\pgfqpoint{1.793384in}{1.094706in}}%
\pgfpathcurveto{\pgfqpoint{1.787560in}{1.088883in}}{\pgfqpoint{1.784288in}{1.080983in}}{\pgfqpoint{1.784288in}{1.072746in}}%
\pgfpathcurveto{\pgfqpoint{1.784288in}{1.064510in}}{\pgfqpoint{1.787560in}{1.056610in}}{\pgfqpoint{1.793384in}{1.050786in}}%
\pgfpathcurveto{\pgfqpoint{1.799208in}{1.044962in}}{\pgfqpoint{1.807108in}{1.041690in}}{\pgfqpoint{1.815344in}{1.041690in}}%
\pgfpathclose%
\pgfusepath{stroke,fill}%
\end{pgfscope}%
\begin{pgfscope}%
\pgfpathrectangle{\pgfqpoint{0.100000in}{0.212622in}}{\pgfqpoint{3.696000in}{3.696000in}}%
\pgfusepath{clip}%
\pgfsetbuttcap%
\pgfsetroundjoin%
\definecolor{currentfill}{rgb}{0.121569,0.466667,0.705882}%
\pgfsetfillcolor{currentfill}%
\pgfsetfillopacity{0.871188}%
\pgfsetlinewidth{1.003750pt}%
\definecolor{currentstroke}{rgb}{0.121569,0.466667,0.705882}%
\pgfsetstrokecolor{currentstroke}%
\pgfsetstrokeopacity{0.871188}%
\pgfsetdash{}{0pt}%
\pgfpathmoveto{\pgfqpoint{1.822759in}{1.038935in}}%
\pgfpathcurveto{\pgfqpoint{1.830995in}{1.038935in}}{\pgfqpoint{1.838895in}{1.042207in}}{\pgfqpoint{1.844719in}{1.048031in}}%
\pgfpathcurveto{\pgfqpoint{1.850543in}{1.053855in}}{\pgfqpoint{1.853816in}{1.061755in}}{\pgfqpoint{1.853816in}{1.069991in}}%
\pgfpathcurveto{\pgfqpoint{1.853816in}{1.078228in}}{\pgfqpoint{1.850543in}{1.086128in}}{\pgfqpoint{1.844719in}{1.091952in}}%
\pgfpathcurveto{\pgfqpoint{1.838895in}{1.097776in}}{\pgfqpoint{1.830995in}{1.101048in}}{\pgfqpoint{1.822759in}{1.101048in}}%
\pgfpathcurveto{\pgfqpoint{1.814523in}{1.101048in}}{\pgfqpoint{1.806623in}{1.097776in}}{\pgfqpoint{1.800799in}{1.091952in}}%
\pgfpathcurveto{\pgfqpoint{1.794975in}{1.086128in}}{\pgfqpoint{1.791703in}{1.078228in}}{\pgfqpoint{1.791703in}{1.069991in}}%
\pgfpathcurveto{\pgfqpoint{1.791703in}{1.061755in}}{\pgfqpoint{1.794975in}{1.053855in}}{\pgfqpoint{1.800799in}{1.048031in}}%
\pgfpathcurveto{\pgfqpoint{1.806623in}{1.042207in}}{\pgfqpoint{1.814523in}{1.038935in}}{\pgfqpoint{1.822759in}{1.038935in}}%
\pgfpathclose%
\pgfusepath{stroke,fill}%
\end{pgfscope}%
\begin{pgfscope}%
\pgfpathrectangle{\pgfqpoint{0.100000in}{0.212622in}}{\pgfqpoint{3.696000in}{3.696000in}}%
\pgfusepath{clip}%
\pgfsetbuttcap%
\pgfsetroundjoin%
\definecolor{currentfill}{rgb}{0.121569,0.466667,0.705882}%
\pgfsetfillcolor{currentfill}%
\pgfsetfillopacity{0.872226}%
\pgfsetlinewidth{1.003750pt}%
\definecolor{currentstroke}{rgb}{0.121569,0.466667,0.705882}%
\pgfsetstrokecolor{currentstroke}%
\pgfsetstrokeopacity{0.872226}%
\pgfsetdash{}{0pt}%
\pgfpathmoveto{\pgfqpoint{1.828165in}{1.036862in}}%
\pgfpathcurveto{\pgfqpoint{1.836402in}{1.036862in}}{\pgfqpoint{1.844302in}{1.040134in}}{\pgfqpoint{1.850126in}{1.045958in}}%
\pgfpathcurveto{\pgfqpoint{1.855950in}{1.051782in}}{\pgfqpoint{1.859222in}{1.059682in}}{\pgfqpoint{1.859222in}{1.067918in}}%
\pgfpathcurveto{\pgfqpoint{1.859222in}{1.076155in}}{\pgfqpoint{1.855950in}{1.084055in}}{\pgfqpoint{1.850126in}{1.089879in}}%
\pgfpathcurveto{\pgfqpoint{1.844302in}{1.095702in}}{\pgfqpoint{1.836402in}{1.098975in}}{\pgfqpoint{1.828165in}{1.098975in}}%
\pgfpathcurveto{\pgfqpoint{1.819929in}{1.098975in}}{\pgfqpoint{1.812029in}{1.095702in}}{\pgfqpoint{1.806205in}{1.089879in}}%
\pgfpathcurveto{\pgfqpoint{1.800381in}{1.084055in}}{\pgfqpoint{1.797109in}{1.076155in}}{\pgfqpoint{1.797109in}{1.067918in}}%
\pgfpathcurveto{\pgfqpoint{1.797109in}{1.059682in}}{\pgfqpoint{1.800381in}{1.051782in}}{\pgfqpoint{1.806205in}{1.045958in}}%
\pgfpathcurveto{\pgfqpoint{1.812029in}{1.040134in}}{\pgfqpoint{1.819929in}{1.036862in}}{\pgfqpoint{1.828165in}{1.036862in}}%
\pgfpathclose%
\pgfusepath{stroke,fill}%
\end{pgfscope}%
\begin{pgfscope}%
\pgfpathrectangle{\pgfqpoint{0.100000in}{0.212622in}}{\pgfqpoint{3.696000in}{3.696000in}}%
\pgfusepath{clip}%
\pgfsetbuttcap%
\pgfsetroundjoin%
\definecolor{currentfill}{rgb}{0.121569,0.466667,0.705882}%
\pgfsetfillcolor{currentfill}%
\pgfsetfillopacity{0.873039}%
\pgfsetlinewidth{1.003750pt}%
\definecolor{currentstroke}{rgb}{0.121569,0.466667,0.705882}%
\pgfsetstrokecolor{currentstroke}%
\pgfsetstrokeopacity{0.873039}%
\pgfsetdash{}{0pt}%
\pgfpathmoveto{\pgfqpoint{1.832537in}{1.035147in}}%
\pgfpathcurveto{\pgfqpoint{1.840773in}{1.035147in}}{\pgfqpoint{1.848673in}{1.038419in}}{\pgfqpoint{1.854497in}{1.044243in}}%
\pgfpathcurveto{\pgfqpoint{1.860321in}{1.050067in}}{\pgfqpoint{1.863593in}{1.057967in}}{\pgfqpoint{1.863593in}{1.066204in}}%
\pgfpathcurveto{\pgfqpoint{1.863593in}{1.074440in}}{\pgfqpoint{1.860321in}{1.082340in}}{\pgfqpoint{1.854497in}{1.088164in}}%
\pgfpathcurveto{\pgfqpoint{1.848673in}{1.093988in}}{\pgfqpoint{1.840773in}{1.097260in}}{\pgfqpoint{1.832537in}{1.097260in}}%
\pgfpathcurveto{\pgfqpoint{1.824301in}{1.097260in}}{\pgfqpoint{1.816400in}{1.093988in}}{\pgfqpoint{1.810577in}{1.088164in}}%
\pgfpathcurveto{\pgfqpoint{1.804753in}{1.082340in}}{\pgfqpoint{1.801480in}{1.074440in}}{\pgfqpoint{1.801480in}{1.066204in}}%
\pgfpathcurveto{\pgfqpoint{1.801480in}{1.057967in}}{\pgfqpoint{1.804753in}{1.050067in}}{\pgfqpoint{1.810577in}{1.044243in}}%
\pgfpathcurveto{\pgfqpoint{1.816400in}{1.038419in}}{\pgfqpoint{1.824301in}{1.035147in}}{\pgfqpoint{1.832537in}{1.035147in}}%
\pgfpathclose%
\pgfusepath{stroke,fill}%
\end{pgfscope}%
\begin{pgfscope}%
\pgfpathrectangle{\pgfqpoint{0.100000in}{0.212622in}}{\pgfqpoint{3.696000in}{3.696000in}}%
\pgfusepath{clip}%
\pgfsetbuttcap%
\pgfsetroundjoin%
\definecolor{currentfill}{rgb}{0.121569,0.466667,0.705882}%
\pgfsetfillcolor{currentfill}%
\pgfsetfillopacity{0.874615}%
\pgfsetlinewidth{1.003750pt}%
\definecolor{currentstroke}{rgb}{0.121569,0.466667,0.705882}%
\pgfsetstrokecolor{currentstroke}%
\pgfsetstrokeopacity{0.874615}%
\pgfsetdash{}{0pt}%
\pgfpathmoveto{\pgfqpoint{1.840407in}{1.032096in}}%
\pgfpathcurveto{\pgfqpoint{1.848643in}{1.032096in}}{\pgfqpoint{1.856544in}{1.035369in}}{\pgfqpoint{1.862367in}{1.041193in}}%
\pgfpathcurveto{\pgfqpoint{1.868191in}{1.047017in}}{\pgfqpoint{1.871464in}{1.054917in}}{\pgfqpoint{1.871464in}{1.063153in}}%
\pgfpathcurveto{\pgfqpoint{1.871464in}{1.071389in}}{\pgfqpoint{1.868191in}{1.079289in}}{\pgfqpoint{1.862367in}{1.085113in}}%
\pgfpathcurveto{\pgfqpoint{1.856544in}{1.090937in}}{\pgfqpoint{1.848643in}{1.094209in}}{\pgfqpoint{1.840407in}{1.094209in}}%
\pgfpathcurveto{\pgfqpoint{1.832171in}{1.094209in}}{\pgfqpoint{1.824271in}{1.090937in}}{\pgfqpoint{1.818447in}{1.085113in}}%
\pgfpathcurveto{\pgfqpoint{1.812623in}{1.079289in}}{\pgfqpoint{1.809351in}{1.071389in}}{\pgfqpoint{1.809351in}{1.063153in}}%
\pgfpathcurveto{\pgfqpoint{1.809351in}{1.054917in}}{\pgfqpoint{1.812623in}{1.047017in}}{\pgfqpoint{1.818447in}{1.041193in}}%
\pgfpathcurveto{\pgfqpoint{1.824271in}{1.035369in}}{\pgfqpoint{1.832171in}{1.032096in}}{\pgfqpoint{1.840407in}{1.032096in}}%
\pgfpathclose%
\pgfusepath{stroke,fill}%
\end{pgfscope}%
\begin{pgfscope}%
\pgfpathrectangle{\pgfqpoint{0.100000in}{0.212622in}}{\pgfqpoint{3.696000in}{3.696000in}}%
\pgfusepath{clip}%
\pgfsetbuttcap%
\pgfsetroundjoin%
\definecolor{currentfill}{rgb}{0.121569,0.466667,0.705882}%
\pgfsetfillcolor{currentfill}%
\pgfsetfillopacity{0.875900}%
\pgfsetlinewidth{1.003750pt}%
\definecolor{currentstroke}{rgb}{0.121569,0.466667,0.705882}%
\pgfsetstrokecolor{currentstroke}%
\pgfsetstrokeopacity{0.875900}%
\pgfsetdash{}{0pt}%
\pgfpathmoveto{\pgfqpoint{1.846587in}{1.030134in}}%
\pgfpathcurveto{\pgfqpoint{1.854823in}{1.030134in}}{\pgfqpoint{1.862723in}{1.033407in}}{\pgfqpoint{1.868547in}{1.039231in}}%
\pgfpathcurveto{\pgfqpoint{1.874371in}{1.045054in}}{\pgfqpoint{1.877644in}{1.052954in}}{\pgfqpoint{1.877644in}{1.061191in}}%
\pgfpathcurveto{\pgfqpoint{1.877644in}{1.069427in}}{\pgfqpoint{1.874371in}{1.077327in}}{\pgfqpoint{1.868547in}{1.083151in}}%
\pgfpathcurveto{\pgfqpoint{1.862723in}{1.088975in}}{\pgfqpoint{1.854823in}{1.092247in}}{\pgfqpoint{1.846587in}{1.092247in}}%
\pgfpathcurveto{\pgfqpoint{1.838351in}{1.092247in}}{\pgfqpoint{1.830451in}{1.088975in}}{\pgfqpoint{1.824627in}{1.083151in}}%
\pgfpathcurveto{\pgfqpoint{1.818803in}{1.077327in}}{\pgfqpoint{1.815531in}{1.069427in}}{\pgfqpoint{1.815531in}{1.061191in}}%
\pgfpathcurveto{\pgfqpoint{1.815531in}{1.052954in}}{\pgfqpoint{1.818803in}{1.045054in}}{\pgfqpoint{1.824627in}{1.039231in}}%
\pgfpathcurveto{\pgfqpoint{1.830451in}{1.033407in}}{\pgfqpoint{1.838351in}{1.030134in}}{\pgfqpoint{1.846587in}{1.030134in}}%
\pgfpathclose%
\pgfusepath{stroke,fill}%
\end{pgfscope}%
\begin{pgfscope}%
\pgfpathrectangle{\pgfqpoint{0.100000in}{0.212622in}}{\pgfqpoint{3.696000in}{3.696000in}}%
\pgfusepath{clip}%
\pgfsetbuttcap%
\pgfsetroundjoin%
\definecolor{currentfill}{rgb}{0.121569,0.466667,0.705882}%
\pgfsetfillcolor{currentfill}%
\pgfsetfillopacity{0.876442}%
\pgfsetlinewidth{1.003750pt}%
\definecolor{currentstroke}{rgb}{0.121569,0.466667,0.705882}%
\pgfsetstrokecolor{currentstroke}%
\pgfsetstrokeopacity{0.876442}%
\pgfsetdash{}{0pt}%
\pgfpathmoveto{\pgfqpoint{2.336502in}{1.186055in}}%
\pgfpathcurveto{\pgfqpoint{2.344739in}{1.186055in}}{\pgfqpoint{2.352639in}{1.189328in}}{\pgfqpoint{2.358463in}{1.195152in}}%
\pgfpathcurveto{\pgfqpoint{2.364286in}{1.200976in}}{\pgfqpoint{2.367559in}{1.208876in}}{\pgfqpoint{2.367559in}{1.217112in}}%
\pgfpathcurveto{\pgfqpoint{2.367559in}{1.225348in}}{\pgfqpoint{2.364286in}{1.233248in}}{\pgfqpoint{2.358463in}{1.239072in}}%
\pgfpathcurveto{\pgfqpoint{2.352639in}{1.244896in}}{\pgfqpoint{2.344739in}{1.248168in}}{\pgfqpoint{2.336502in}{1.248168in}}%
\pgfpathcurveto{\pgfqpoint{2.328266in}{1.248168in}}{\pgfqpoint{2.320366in}{1.244896in}}{\pgfqpoint{2.314542in}{1.239072in}}%
\pgfpathcurveto{\pgfqpoint{2.308718in}{1.233248in}}{\pgfqpoint{2.305446in}{1.225348in}}{\pgfqpoint{2.305446in}{1.217112in}}%
\pgfpathcurveto{\pgfqpoint{2.305446in}{1.208876in}}{\pgfqpoint{2.308718in}{1.200976in}}{\pgfqpoint{2.314542in}{1.195152in}}%
\pgfpathcurveto{\pgfqpoint{2.320366in}{1.189328in}}{\pgfqpoint{2.328266in}{1.186055in}}{\pgfqpoint{2.336502in}{1.186055in}}%
\pgfpathclose%
\pgfusepath{stroke,fill}%
\end{pgfscope}%
\begin{pgfscope}%
\pgfpathrectangle{\pgfqpoint{0.100000in}{0.212622in}}{\pgfqpoint{3.696000in}{3.696000in}}%
\pgfusepath{clip}%
\pgfsetbuttcap%
\pgfsetroundjoin%
\definecolor{currentfill}{rgb}{0.121569,0.466667,0.705882}%
\pgfsetfillcolor{currentfill}%
\pgfsetfillopacity{0.877014}%
\pgfsetlinewidth{1.003750pt}%
\definecolor{currentstroke}{rgb}{0.121569,0.466667,0.705882}%
\pgfsetstrokecolor{currentstroke}%
\pgfsetstrokeopacity{0.877014}%
\pgfsetdash{}{0pt}%
\pgfpathmoveto{\pgfqpoint{1.851837in}{1.028696in}}%
\pgfpathcurveto{\pgfqpoint{1.860073in}{1.028696in}}{\pgfqpoint{1.867973in}{1.031968in}}{\pgfqpoint{1.873797in}{1.037792in}}%
\pgfpathcurveto{\pgfqpoint{1.879621in}{1.043616in}}{\pgfqpoint{1.882893in}{1.051516in}}{\pgfqpoint{1.882893in}{1.059752in}}%
\pgfpathcurveto{\pgfqpoint{1.882893in}{1.067988in}}{\pgfqpoint{1.879621in}{1.075888in}}{\pgfqpoint{1.873797in}{1.081712in}}%
\pgfpathcurveto{\pgfqpoint{1.867973in}{1.087536in}}{\pgfqpoint{1.860073in}{1.090809in}}{\pgfqpoint{1.851837in}{1.090809in}}%
\pgfpathcurveto{\pgfqpoint{1.843600in}{1.090809in}}{\pgfqpoint{1.835700in}{1.087536in}}{\pgfqpoint{1.829877in}{1.081712in}}%
\pgfpathcurveto{\pgfqpoint{1.824053in}{1.075888in}}{\pgfqpoint{1.820780in}{1.067988in}}{\pgfqpoint{1.820780in}{1.059752in}}%
\pgfpathcurveto{\pgfqpoint{1.820780in}{1.051516in}}{\pgfqpoint{1.824053in}{1.043616in}}{\pgfqpoint{1.829877in}{1.037792in}}%
\pgfpathcurveto{\pgfqpoint{1.835700in}{1.031968in}}{\pgfqpoint{1.843600in}{1.028696in}}{\pgfqpoint{1.851837in}{1.028696in}}%
\pgfpathclose%
\pgfusepath{stroke,fill}%
\end{pgfscope}%
\begin{pgfscope}%
\pgfpathrectangle{\pgfqpoint{0.100000in}{0.212622in}}{\pgfqpoint{3.696000in}{3.696000in}}%
\pgfusepath{clip}%
\pgfsetbuttcap%
\pgfsetroundjoin%
\definecolor{currentfill}{rgb}{0.121569,0.466667,0.705882}%
\pgfsetfillcolor{currentfill}%
\pgfsetfillopacity{0.878814}%
\pgfsetlinewidth{1.003750pt}%
\definecolor{currentstroke}{rgb}{0.121569,0.466667,0.705882}%
\pgfsetstrokecolor{currentstroke}%
\pgfsetstrokeopacity{0.878814}%
\pgfsetdash{}{0pt}%
\pgfpathmoveto{\pgfqpoint{1.861391in}{1.025288in}}%
\pgfpathcurveto{\pgfqpoint{1.869627in}{1.025288in}}{\pgfqpoint{1.877527in}{1.028561in}}{\pgfqpoint{1.883351in}{1.034384in}}%
\pgfpathcurveto{\pgfqpoint{1.889175in}{1.040208in}}{\pgfqpoint{1.892448in}{1.048108in}}{\pgfqpoint{1.892448in}{1.056345in}}%
\pgfpathcurveto{\pgfqpoint{1.892448in}{1.064581in}}{\pgfqpoint{1.889175in}{1.072481in}}{\pgfqpoint{1.883351in}{1.078305in}}%
\pgfpathcurveto{\pgfqpoint{1.877527in}{1.084129in}}{\pgfqpoint{1.869627in}{1.087401in}}{\pgfqpoint{1.861391in}{1.087401in}}%
\pgfpathcurveto{\pgfqpoint{1.853155in}{1.087401in}}{\pgfqpoint{1.845255in}{1.084129in}}{\pgfqpoint{1.839431in}{1.078305in}}%
\pgfpathcurveto{\pgfqpoint{1.833607in}{1.072481in}}{\pgfqpoint{1.830335in}{1.064581in}}{\pgfqpoint{1.830335in}{1.056345in}}%
\pgfpathcurveto{\pgfqpoint{1.830335in}{1.048108in}}{\pgfqpoint{1.833607in}{1.040208in}}{\pgfqpoint{1.839431in}{1.034384in}}%
\pgfpathcurveto{\pgfqpoint{1.845255in}{1.028561in}}{\pgfqpoint{1.853155in}{1.025288in}}{\pgfqpoint{1.861391in}{1.025288in}}%
\pgfpathclose%
\pgfusepath{stroke,fill}%
\end{pgfscope}%
\begin{pgfscope}%
\pgfpathrectangle{\pgfqpoint{0.100000in}{0.212622in}}{\pgfqpoint{3.696000in}{3.696000in}}%
\pgfusepath{clip}%
\pgfsetbuttcap%
\pgfsetroundjoin%
\definecolor{currentfill}{rgb}{0.121569,0.466667,0.705882}%
\pgfsetfillcolor{currentfill}%
\pgfsetfillopacity{0.880249}%
\pgfsetlinewidth{1.003750pt}%
\definecolor{currentstroke}{rgb}{0.121569,0.466667,0.705882}%
\pgfsetstrokecolor{currentstroke}%
\pgfsetstrokeopacity{0.880249}%
\pgfsetdash{}{0pt}%
\pgfpathmoveto{\pgfqpoint{1.868763in}{1.022618in}}%
\pgfpathcurveto{\pgfqpoint{1.876999in}{1.022618in}}{\pgfqpoint{1.884899in}{1.025891in}}{\pgfqpoint{1.890723in}{1.031715in}}%
\pgfpathcurveto{\pgfqpoint{1.896547in}{1.037538in}}{\pgfqpoint{1.899819in}{1.045439in}}{\pgfqpoint{1.899819in}{1.053675in}}%
\pgfpathcurveto{\pgfqpoint{1.899819in}{1.061911in}}{\pgfqpoint{1.896547in}{1.069811in}}{\pgfqpoint{1.890723in}{1.075635in}}%
\pgfpathcurveto{\pgfqpoint{1.884899in}{1.081459in}}{\pgfqpoint{1.876999in}{1.084731in}}{\pgfqpoint{1.868763in}{1.084731in}}%
\pgfpathcurveto{\pgfqpoint{1.860527in}{1.084731in}}{\pgfqpoint{1.852627in}{1.081459in}}{\pgfqpoint{1.846803in}{1.075635in}}%
\pgfpathcurveto{\pgfqpoint{1.840979in}{1.069811in}}{\pgfqpoint{1.837706in}{1.061911in}}{\pgfqpoint{1.837706in}{1.053675in}}%
\pgfpathcurveto{\pgfqpoint{1.837706in}{1.045439in}}{\pgfqpoint{1.840979in}{1.037538in}}{\pgfqpoint{1.846803in}{1.031715in}}%
\pgfpathcurveto{\pgfqpoint{1.852627in}{1.025891in}}{\pgfqpoint{1.860527in}{1.022618in}}{\pgfqpoint{1.868763in}{1.022618in}}%
\pgfpathclose%
\pgfusepath{stroke,fill}%
\end{pgfscope}%
\begin{pgfscope}%
\pgfpathrectangle{\pgfqpoint{0.100000in}{0.212622in}}{\pgfqpoint{3.696000in}{3.696000in}}%
\pgfusepath{clip}%
\pgfsetbuttcap%
\pgfsetroundjoin%
\definecolor{currentfill}{rgb}{0.121569,0.466667,0.705882}%
\pgfsetfillcolor{currentfill}%
\pgfsetfillopacity{0.881337}%
\pgfsetlinewidth{1.003750pt}%
\definecolor{currentstroke}{rgb}{0.121569,0.466667,0.705882}%
\pgfsetstrokecolor{currentstroke}%
\pgfsetstrokeopacity{0.881337}%
\pgfsetdash{}{0pt}%
\pgfpathmoveto{\pgfqpoint{2.340050in}{1.171720in}}%
\pgfpathcurveto{\pgfqpoint{2.348287in}{1.171720in}}{\pgfqpoint{2.356187in}{1.174993in}}{\pgfqpoint{2.362011in}{1.180817in}}%
\pgfpathcurveto{\pgfqpoint{2.367835in}{1.186641in}}{\pgfqpoint{2.371107in}{1.194541in}}{\pgfqpoint{2.371107in}{1.202777in}}%
\pgfpathcurveto{\pgfqpoint{2.371107in}{1.211013in}}{\pgfqpoint{2.367835in}{1.218913in}}{\pgfqpoint{2.362011in}{1.224737in}}%
\pgfpathcurveto{\pgfqpoint{2.356187in}{1.230561in}}{\pgfqpoint{2.348287in}{1.233833in}}{\pgfqpoint{2.340050in}{1.233833in}}%
\pgfpathcurveto{\pgfqpoint{2.331814in}{1.233833in}}{\pgfqpoint{2.323914in}{1.230561in}}{\pgfqpoint{2.318090in}{1.224737in}}%
\pgfpathcurveto{\pgfqpoint{2.312266in}{1.218913in}}{\pgfqpoint{2.308994in}{1.211013in}}{\pgfqpoint{2.308994in}{1.202777in}}%
\pgfpathcurveto{\pgfqpoint{2.308994in}{1.194541in}}{\pgfqpoint{2.312266in}{1.186641in}}{\pgfqpoint{2.318090in}{1.180817in}}%
\pgfpathcurveto{\pgfqpoint{2.323914in}{1.174993in}}{\pgfqpoint{2.331814in}{1.171720in}}{\pgfqpoint{2.340050in}{1.171720in}}%
\pgfpathclose%
\pgfusepath{stroke,fill}%
\end{pgfscope}%
\begin{pgfscope}%
\pgfpathrectangle{\pgfqpoint{0.100000in}{0.212622in}}{\pgfqpoint{3.696000in}{3.696000in}}%
\pgfusepath{clip}%
\pgfsetbuttcap%
\pgfsetroundjoin%
\definecolor{currentfill}{rgb}{0.121569,0.466667,0.705882}%
\pgfsetfillcolor{currentfill}%
\pgfsetfillopacity{0.881593}%
\pgfsetlinewidth{1.003750pt}%
\definecolor{currentstroke}{rgb}{0.121569,0.466667,0.705882}%
\pgfsetstrokecolor{currentstroke}%
\pgfsetstrokeopacity{0.881593}%
\pgfsetdash{}{0pt}%
\pgfpathmoveto{\pgfqpoint{1.875708in}{1.020443in}}%
\pgfpathcurveto{\pgfqpoint{1.883944in}{1.020443in}}{\pgfqpoint{1.891844in}{1.023715in}}{\pgfqpoint{1.897668in}{1.029539in}}%
\pgfpathcurveto{\pgfqpoint{1.903492in}{1.035363in}}{\pgfqpoint{1.906765in}{1.043263in}}{\pgfqpoint{1.906765in}{1.051499in}}%
\pgfpathcurveto{\pgfqpoint{1.906765in}{1.059735in}}{\pgfqpoint{1.903492in}{1.067635in}}{\pgfqpoint{1.897668in}{1.073459in}}%
\pgfpathcurveto{\pgfqpoint{1.891844in}{1.079283in}}{\pgfqpoint{1.883944in}{1.082556in}}{\pgfqpoint{1.875708in}{1.082556in}}%
\pgfpathcurveto{\pgfqpoint{1.867472in}{1.082556in}}{\pgfqpoint{1.859572in}{1.079283in}}{\pgfqpoint{1.853748in}{1.073459in}}%
\pgfpathcurveto{\pgfqpoint{1.847924in}{1.067635in}}{\pgfqpoint{1.844652in}{1.059735in}}{\pgfqpoint{1.844652in}{1.051499in}}%
\pgfpathcurveto{\pgfqpoint{1.844652in}{1.043263in}}{\pgfqpoint{1.847924in}{1.035363in}}{\pgfqpoint{1.853748in}{1.029539in}}%
\pgfpathcurveto{\pgfqpoint{1.859572in}{1.023715in}}{\pgfqpoint{1.867472in}{1.020443in}}{\pgfqpoint{1.875708in}{1.020443in}}%
\pgfpathclose%
\pgfusepath{stroke,fill}%
\end{pgfscope}%
\begin{pgfscope}%
\pgfpathrectangle{\pgfqpoint{0.100000in}{0.212622in}}{\pgfqpoint{3.696000in}{3.696000in}}%
\pgfusepath{clip}%
\pgfsetbuttcap%
\pgfsetroundjoin%
\definecolor{currentfill}{rgb}{0.121569,0.466667,0.705882}%
\pgfsetfillcolor{currentfill}%
\pgfsetfillopacity{0.882784}%
\pgfsetlinewidth{1.003750pt}%
\definecolor{currentstroke}{rgb}{0.121569,0.466667,0.705882}%
\pgfsetstrokecolor{currentstroke}%
\pgfsetstrokeopacity{0.882784}%
\pgfsetdash{}{0pt}%
\pgfpathmoveto{\pgfqpoint{1.881675in}{1.018597in}}%
\pgfpathcurveto{\pgfqpoint{1.889912in}{1.018597in}}{\pgfqpoint{1.897812in}{1.021870in}}{\pgfqpoint{1.903636in}{1.027694in}}%
\pgfpathcurveto{\pgfqpoint{1.909460in}{1.033517in}}{\pgfqpoint{1.912732in}{1.041418in}}{\pgfqpoint{1.912732in}{1.049654in}}%
\pgfpathcurveto{\pgfqpoint{1.912732in}{1.057890in}}{\pgfqpoint{1.909460in}{1.065790in}}{\pgfqpoint{1.903636in}{1.071614in}}%
\pgfpathcurveto{\pgfqpoint{1.897812in}{1.077438in}}{\pgfqpoint{1.889912in}{1.080710in}}{\pgfqpoint{1.881675in}{1.080710in}}%
\pgfpathcurveto{\pgfqpoint{1.873439in}{1.080710in}}{\pgfqpoint{1.865539in}{1.077438in}}{\pgfqpoint{1.859715in}{1.071614in}}%
\pgfpathcurveto{\pgfqpoint{1.853891in}{1.065790in}}{\pgfqpoint{1.850619in}{1.057890in}}{\pgfqpoint{1.850619in}{1.049654in}}%
\pgfpathcurveto{\pgfqpoint{1.850619in}{1.041418in}}{\pgfqpoint{1.853891in}{1.033517in}}{\pgfqpoint{1.859715in}{1.027694in}}%
\pgfpathcurveto{\pgfqpoint{1.865539in}{1.021870in}}{\pgfqpoint{1.873439in}{1.018597in}}{\pgfqpoint{1.881675in}{1.018597in}}%
\pgfpathclose%
\pgfusepath{stroke,fill}%
\end{pgfscope}%
\begin{pgfscope}%
\pgfpathrectangle{\pgfqpoint{0.100000in}{0.212622in}}{\pgfqpoint{3.696000in}{3.696000in}}%
\pgfusepath{clip}%
\pgfsetbuttcap%
\pgfsetroundjoin%
\definecolor{currentfill}{rgb}{0.121569,0.466667,0.705882}%
\pgfsetfillcolor{currentfill}%
\pgfsetfillopacity{0.883702}%
\pgfsetlinewidth{1.003750pt}%
\definecolor{currentstroke}{rgb}{0.121569,0.466667,0.705882}%
\pgfsetstrokecolor{currentstroke}%
\pgfsetstrokeopacity{0.883702}%
\pgfsetdash{}{0pt}%
\pgfpathmoveto{\pgfqpoint{1.886533in}{1.017051in}}%
\pgfpathcurveto{\pgfqpoint{1.894769in}{1.017051in}}{\pgfqpoint{1.902669in}{1.020323in}}{\pgfqpoint{1.908493in}{1.026147in}}%
\pgfpathcurveto{\pgfqpoint{1.914317in}{1.031971in}}{\pgfqpoint{1.917589in}{1.039871in}}{\pgfqpoint{1.917589in}{1.048107in}}%
\pgfpathcurveto{\pgfqpoint{1.917589in}{1.056344in}}{\pgfqpoint{1.914317in}{1.064244in}}{\pgfqpoint{1.908493in}{1.070068in}}%
\pgfpathcurveto{\pgfqpoint{1.902669in}{1.075891in}}{\pgfqpoint{1.894769in}{1.079164in}}{\pgfqpoint{1.886533in}{1.079164in}}%
\pgfpathcurveto{\pgfqpoint{1.878297in}{1.079164in}}{\pgfqpoint{1.870397in}{1.075891in}}{\pgfqpoint{1.864573in}{1.070068in}}%
\pgfpathcurveto{\pgfqpoint{1.858749in}{1.064244in}}{\pgfqpoint{1.855476in}{1.056344in}}{\pgfqpoint{1.855476in}{1.048107in}}%
\pgfpathcurveto{\pgfqpoint{1.855476in}{1.039871in}}{\pgfqpoint{1.858749in}{1.031971in}}{\pgfqpoint{1.864573in}{1.026147in}}%
\pgfpathcurveto{\pgfqpoint{1.870397in}{1.020323in}}{\pgfqpoint{1.878297in}{1.017051in}}{\pgfqpoint{1.886533in}{1.017051in}}%
\pgfpathclose%
\pgfusepath{stroke,fill}%
\end{pgfscope}%
\begin{pgfscope}%
\pgfpathrectangle{\pgfqpoint{0.100000in}{0.212622in}}{\pgfqpoint{3.696000in}{3.696000in}}%
\pgfusepath{clip}%
\pgfsetbuttcap%
\pgfsetroundjoin%
\definecolor{currentfill}{rgb}{0.121569,0.466667,0.705882}%
\pgfsetfillcolor{currentfill}%
\pgfsetfillopacity{0.885444}%
\pgfsetlinewidth{1.003750pt}%
\definecolor{currentstroke}{rgb}{0.121569,0.466667,0.705882}%
\pgfsetstrokecolor{currentstroke}%
\pgfsetstrokeopacity{0.885444}%
\pgfsetdash{}{0pt}%
\pgfpathmoveto{\pgfqpoint{1.895371in}{1.014516in}}%
\pgfpathcurveto{\pgfqpoint{1.903608in}{1.014516in}}{\pgfqpoint{1.911508in}{1.017789in}}{\pgfqpoint{1.917332in}{1.023612in}}%
\pgfpathcurveto{\pgfqpoint{1.923155in}{1.029436in}}{\pgfqpoint{1.926428in}{1.037336in}}{\pgfqpoint{1.926428in}{1.045573in}}%
\pgfpathcurveto{\pgfqpoint{1.926428in}{1.053809in}}{\pgfqpoint{1.923155in}{1.061709in}}{\pgfqpoint{1.917332in}{1.067533in}}%
\pgfpathcurveto{\pgfqpoint{1.911508in}{1.073357in}}{\pgfqpoint{1.903608in}{1.076629in}}{\pgfqpoint{1.895371in}{1.076629in}}%
\pgfpathcurveto{\pgfqpoint{1.887135in}{1.076629in}}{\pgfqpoint{1.879235in}{1.073357in}}{\pgfqpoint{1.873411in}{1.067533in}}%
\pgfpathcurveto{\pgfqpoint{1.867587in}{1.061709in}}{\pgfqpoint{1.864315in}{1.053809in}}{\pgfqpoint{1.864315in}{1.045573in}}%
\pgfpathcurveto{\pgfqpoint{1.864315in}{1.037336in}}{\pgfqpoint{1.867587in}{1.029436in}}{\pgfqpoint{1.873411in}{1.023612in}}%
\pgfpathcurveto{\pgfqpoint{1.879235in}{1.017789in}}{\pgfqpoint{1.887135in}{1.014516in}}{\pgfqpoint{1.895371in}{1.014516in}}%
\pgfpathclose%
\pgfusepath{stroke,fill}%
\end{pgfscope}%
\begin{pgfscope}%
\pgfpathrectangle{\pgfqpoint{0.100000in}{0.212622in}}{\pgfqpoint{3.696000in}{3.696000in}}%
\pgfusepath{clip}%
\pgfsetbuttcap%
\pgfsetroundjoin%
\definecolor{currentfill}{rgb}{0.121569,0.466667,0.705882}%
\pgfsetfillcolor{currentfill}%
\pgfsetfillopacity{0.886154}%
\pgfsetlinewidth{1.003750pt}%
\definecolor{currentstroke}{rgb}{0.121569,0.466667,0.705882}%
\pgfsetstrokecolor{currentstroke}%
\pgfsetstrokeopacity{0.886154}%
\pgfsetdash{}{0pt}%
\pgfpathmoveto{\pgfqpoint{2.345049in}{1.156764in}}%
\pgfpathcurveto{\pgfqpoint{2.353285in}{1.156764in}}{\pgfqpoint{2.361185in}{1.160036in}}{\pgfqpoint{2.367009in}{1.165860in}}%
\pgfpathcurveto{\pgfqpoint{2.372833in}{1.171684in}}{\pgfqpoint{2.376105in}{1.179584in}}{\pgfqpoint{2.376105in}{1.187820in}}%
\pgfpathcurveto{\pgfqpoint{2.376105in}{1.196056in}}{\pgfqpoint{2.372833in}{1.203956in}}{\pgfqpoint{2.367009in}{1.209780in}}%
\pgfpathcurveto{\pgfqpoint{2.361185in}{1.215604in}}{\pgfqpoint{2.353285in}{1.218877in}}{\pgfqpoint{2.345049in}{1.218877in}}%
\pgfpathcurveto{\pgfqpoint{2.336812in}{1.218877in}}{\pgfqpoint{2.328912in}{1.215604in}}{\pgfqpoint{2.323088in}{1.209780in}}%
\pgfpathcurveto{\pgfqpoint{2.317264in}{1.203956in}}{\pgfqpoint{2.313992in}{1.196056in}}{\pgfqpoint{2.313992in}{1.187820in}}%
\pgfpathcurveto{\pgfqpoint{2.313992in}{1.179584in}}{\pgfqpoint{2.317264in}{1.171684in}}{\pgfqpoint{2.323088in}{1.165860in}}%
\pgfpathcurveto{\pgfqpoint{2.328912in}{1.160036in}}{\pgfqpoint{2.336812in}{1.156764in}}{\pgfqpoint{2.345049in}{1.156764in}}%
\pgfpathclose%
\pgfusepath{stroke,fill}%
\end{pgfscope}%
\begin{pgfscope}%
\pgfpathrectangle{\pgfqpoint{0.100000in}{0.212622in}}{\pgfqpoint{3.696000in}{3.696000in}}%
\pgfusepath{clip}%
\pgfsetbuttcap%
\pgfsetroundjoin%
\definecolor{currentfill}{rgb}{0.121569,0.466667,0.705882}%
\pgfsetfillcolor{currentfill}%
\pgfsetfillopacity{0.886829}%
\pgfsetlinewidth{1.003750pt}%
\definecolor{currentstroke}{rgb}{0.121569,0.466667,0.705882}%
\pgfsetstrokecolor{currentstroke}%
\pgfsetstrokeopacity{0.886829}%
\pgfsetdash{}{0pt}%
\pgfpathmoveto{\pgfqpoint{1.902050in}{1.012387in}}%
\pgfpathcurveto{\pgfqpoint{1.910286in}{1.012387in}}{\pgfqpoint{1.918186in}{1.015659in}}{\pgfqpoint{1.924010in}{1.021483in}}%
\pgfpathcurveto{\pgfqpoint{1.929834in}{1.027307in}}{\pgfqpoint{1.933106in}{1.035207in}}{\pgfqpoint{1.933106in}{1.043443in}}%
\pgfpathcurveto{\pgfqpoint{1.933106in}{1.051680in}}{\pgfqpoint{1.929834in}{1.059580in}}{\pgfqpoint{1.924010in}{1.065404in}}%
\pgfpathcurveto{\pgfqpoint{1.918186in}{1.071227in}}{\pgfqpoint{1.910286in}{1.074500in}}{\pgfqpoint{1.902050in}{1.074500in}}%
\pgfpathcurveto{\pgfqpoint{1.893813in}{1.074500in}}{\pgfqpoint{1.885913in}{1.071227in}}{\pgfqpoint{1.880089in}{1.065404in}}%
\pgfpathcurveto{\pgfqpoint{1.874265in}{1.059580in}}{\pgfqpoint{1.870993in}{1.051680in}}{\pgfqpoint{1.870993in}{1.043443in}}%
\pgfpathcurveto{\pgfqpoint{1.870993in}{1.035207in}}{\pgfqpoint{1.874265in}{1.027307in}}{\pgfqpoint{1.880089in}{1.021483in}}%
\pgfpathcurveto{\pgfqpoint{1.885913in}{1.015659in}}{\pgfqpoint{1.893813in}{1.012387in}}{\pgfqpoint{1.902050in}{1.012387in}}%
\pgfpathclose%
\pgfusepath{stroke,fill}%
\end{pgfscope}%
\begin{pgfscope}%
\pgfpathrectangle{\pgfqpoint{0.100000in}{0.212622in}}{\pgfqpoint{3.696000in}{3.696000in}}%
\pgfusepath{clip}%
\pgfsetbuttcap%
\pgfsetroundjoin%
\definecolor{currentfill}{rgb}{0.121569,0.466667,0.705882}%
\pgfsetfillcolor{currentfill}%
\pgfsetfillopacity{0.888062}%
\pgfsetlinewidth{1.003750pt}%
\definecolor{currentstroke}{rgb}{0.121569,0.466667,0.705882}%
\pgfsetstrokecolor{currentstroke}%
\pgfsetstrokeopacity{0.888062}%
\pgfsetdash{}{0pt}%
\pgfpathmoveto{\pgfqpoint{1.908211in}{1.010324in}}%
\pgfpathcurveto{\pgfqpoint{1.916447in}{1.010324in}}{\pgfqpoint{1.924347in}{1.013596in}}{\pgfqpoint{1.930171in}{1.019420in}}%
\pgfpathcurveto{\pgfqpoint{1.935995in}{1.025244in}}{\pgfqpoint{1.939267in}{1.033144in}}{\pgfqpoint{1.939267in}{1.041380in}}%
\pgfpathcurveto{\pgfqpoint{1.939267in}{1.049617in}}{\pgfqpoint{1.935995in}{1.057517in}}{\pgfqpoint{1.930171in}{1.063341in}}%
\pgfpathcurveto{\pgfqpoint{1.924347in}{1.069164in}}{\pgfqpoint{1.916447in}{1.072437in}}{\pgfqpoint{1.908211in}{1.072437in}}%
\pgfpathcurveto{\pgfqpoint{1.899975in}{1.072437in}}{\pgfqpoint{1.892074in}{1.069164in}}{\pgfqpoint{1.886251in}{1.063341in}}%
\pgfpathcurveto{\pgfqpoint{1.880427in}{1.057517in}}{\pgfqpoint{1.877154in}{1.049617in}}{\pgfqpoint{1.877154in}{1.041380in}}%
\pgfpathcurveto{\pgfqpoint{1.877154in}{1.033144in}}{\pgfqpoint{1.880427in}{1.025244in}}{\pgfqpoint{1.886251in}{1.019420in}}%
\pgfpathcurveto{\pgfqpoint{1.892074in}{1.013596in}}{\pgfqpoint{1.899975in}{1.010324in}}{\pgfqpoint{1.908211in}{1.010324in}}%
\pgfpathclose%
\pgfusepath{stroke,fill}%
\end{pgfscope}%
\begin{pgfscope}%
\pgfpathrectangle{\pgfqpoint{0.100000in}{0.212622in}}{\pgfqpoint{3.696000in}{3.696000in}}%
\pgfusepath{clip}%
\pgfsetbuttcap%
\pgfsetroundjoin%
\definecolor{currentfill}{rgb}{0.121569,0.466667,0.705882}%
\pgfsetfillcolor{currentfill}%
\pgfsetfillopacity{0.888995}%
\pgfsetlinewidth{1.003750pt}%
\definecolor{currentstroke}{rgb}{0.121569,0.466667,0.705882}%
\pgfsetstrokecolor{currentstroke}%
\pgfsetstrokeopacity{0.888995}%
\pgfsetdash{}{0pt}%
\pgfpathmoveto{\pgfqpoint{1.913073in}{1.008581in}}%
\pgfpathcurveto{\pgfqpoint{1.921309in}{1.008581in}}{\pgfqpoint{1.929209in}{1.011854in}}{\pgfqpoint{1.935033in}{1.017678in}}%
\pgfpathcurveto{\pgfqpoint{1.940857in}{1.023502in}}{\pgfqpoint{1.944130in}{1.031402in}}{\pgfqpoint{1.944130in}{1.039638in}}%
\pgfpathcurveto{\pgfqpoint{1.944130in}{1.047874in}}{\pgfqpoint{1.940857in}{1.055774in}}{\pgfqpoint{1.935033in}{1.061598in}}%
\pgfpathcurveto{\pgfqpoint{1.929209in}{1.067422in}}{\pgfqpoint{1.921309in}{1.070694in}}{\pgfqpoint{1.913073in}{1.070694in}}%
\pgfpathcurveto{\pgfqpoint{1.904837in}{1.070694in}}{\pgfqpoint{1.896937in}{1.067422in}}{\pgfqpoint{1.891113in}{1.061598in}}%
\pgfpathcurveto{\pgfqpoint{1.885289in}{1.055774in}}{\pgfqpoint{1.882017in}{1.047874in}}{\pgfqpoint{1.882017in}{1.039638in}}%
\pgfpathcurveto{\pgfqpoint{1.882017in}{1.031402in}}{\pgfqpoint{1.885289in}{1.023502in}}{\pgfqpoint{1.891113in}{1.017678in}}%
\pgfpathcurveto{\pgfqpoint{1.896937in}{1.011854in}}{\pgfqpoint{1.904837in}{1.008581in}}{\pgfqpoint{1.913073in}{1.008581in}}%
\pgfpathclose%
\pgfusepath{stroke,fill}%
\end{pgfscope}%
\begin{pgfscope}%
\pgfpathrectangle{\pgfqpoint{0.100000in}{0.212622in}}{\pgfqpoint{3.696000in}{3.696000in}}%
\pgfusepath{clip}%
\pgfsetbuttcap%
\pgfsetroundjoin%
\definecolor{currentfill}{rgb}{0.121569,0.466667,0.705882}%
\pgfsetfillcolor{currentfill}%
\pgfsetfillopacity{0.889698}%
\pgfsetlinewidth{1.003750pt}%
\definecolor{currentstroke}{rgb}{0.121569,0.466667,0.705882}%
\pgfsetstrokecolor{currentstroke}%
\pgfsetstrokeopacity{0.889698}%
\pgfsetdash{}{0pt}%
\pgfpathmoveto{\pgfqpoint{1.916537in}{1.007472in}}%
\pgfpathcurveto{\pgfqpoint{1.924774in}{1.007472in}}{\pgfqpoint{1.932674in}{1.010745in}}{\pgfqpoint{1.938498in}{1.016569in}}%
\pgfpathcurveto{\pgfqpoint{1.944321in}{1.022393in}}{\pgfqpoint{1.947594in}{1.030293in}}{\pgfqpoint{1.947594in}{1.038529in}}%
\pgfpathcurveto{\pgfqpoint{1.947594in}{1.046765in}}{\pgfqpoint{1.944321in}{1.054665in}}{\pgfqpoint{1.938498in}{1.060489in}}%
\pgfpathcurveto{\pgfqpoint{1.932674in}{1.066313in}}{\pgfqpoint{1.924774in}{1.069585in}}{\pgfqpoint{1.916537in}{1.069585in}}%
\pgfpathcurveto{\pgfqpoint{1.908301in}{1.069585in}}{\pgfqpoint{1.900401in}{1.066313in}}{\pgfqpoint{1.894577in}{1.060489in}}%
\pgfpathcurveto{\pgfqpoint{1.888753in}{1.054665in}}{\pgfqpoint{1.885481in}{1.046765in}}{\pgfqpoint{1.885481in}{1.038529in}}%
\pgfpathcurveto{\pgfqpoint{1.885481in}{1.030293in}}{\pgfqpoint{1.888753in}{1.022393in}}{\pgfqpoint{1.894577in}{1.016569in}}%
\pgfpathcurveto{\pgfqpoint{1.900401in}{1.010745in}}{\pgfqpoint{1.908301in}{1.007472in}}{\pgfqpoint{1.916537in}{1.007472in}}%
\pgfpathclose%
\pgfusepath{stroke,fill}%
\end{pgfscope}%
\begin{pgfscope}%
\pgfpathrectangle{\pgfqpoint{0.100000in}{0.212622in}}{\pgfqpoint{3.696000in}{3.696000in}}%
\pgfusepath{clip}%
\pgfsetbuttcap%
\pgfsetroundjoin%
\definecolor{currentfill}{rgb}{0.121569,0.466667,0.705882}%
\pgfsetfillcolor{currentfill}%
\pgfsetfillopacity{0.890730}%
\pgfsetlinewidth{1.003750pt}%
\definecolor{currentstroke}{rgb}{0.121569,0.466667,0.705882}%
\pgfsetstrokecolor{currentstroke}%
\pgfsetstrokeopacity{0.890730}%
\pgfsetdash{}{0pt}%
\pgfpathmoveto{\pgfqpoint{2.350131in}{1.139959in}}%
\pgfpathcurveto{\pgfqpoint{2.358367in}{1.139959in}}{\pgfqpoint{2.366267in}{1.143231in}}{\pgfqpoint{2.372091in}{1.149055in}}%
\pgfpathcurveto{\pgfqpoint{2.377915in}{1.154879in}}{\pgfqpoint{2.381187in}{1.162779in}}{\pgfqpoint{2.381187in}{1.171015in}}%
\pgfpathcurveto{\pgfqpoint{2.381187in}{1.179252in}}{\pgfqpoint{2.377915in}{1.187152in}}{\pgfqpoint{2.372091in}{1.192976in}}%
\pgfpathcurveto{\pgfqpoint{2.366267in}{1.198799in}}{\pgfqpoint{2.358367in}{1.202072in}}{\pgfqpoint{2.350131in}{1.202072in}}%
\pgfpathcurveto{\pgfqpoint{2.341895in}{1.202072in}}{\pgfqpoint{2.333995in}{1.198799in}}{\pgfqpoint{2.328171in}{1.192976in}}%
\pgfpathcurveto{\pgfqpoint{2.322347in}{1.187152in}}{\pgfqpoint{2.319074in}{1.179252in}}{\pgfqpoint{2.319074in}{1.171015in}}%
\pgfpathcurveto{\pgfqpoint{2.319074in}{1.162779in}}{\pgfqpoint{2.322347in}{1.154879in}}{\pgfqpoint{2.328171in}{1.149055in}}%
\pgfpathcurveto{\pgfqpoint{2.333995in}{1.143231in}}{\pgfqpoint{2.341895in}{1.139959in}}{\pgfqpoint{2.350131in}{1.139959in}}%
\pgfpathclose%
\pgfusepath{stroke,fill}%
\end{pgfscope}%
\begin{pgfscope}%
\pgfpathrectangle{\pgfqpoint{0.100000in}{0.212622in}}{\pgfqpoint{3.696000in}{3.696000in}}%
\pgfusepath{clip}%
\pgfsetbuttcap%
\pgfsetroundjoin%
\definecolor{currentfill}{rgb}{0.121569,0.466667,0.705882}%
\pgfsetfillcolor{currentfill}%
\pgfsetfillopacity{0.890957}%
\pgfsetlinewidth{1.003750pt}%
\definecolor{currentstroke}{rgb}{0.121569,0.466667,0.705882}%
\pgfsetstrokecolor{currentstroke}%
\pgfsetstrokeopacity{0.890957}%
\pgfsetdash{}{0pt}%
\pgfpathmoveto{\pgfqpoint{1.922871in}{1.005512in}}%
\pgfpathcurveto{\pgfqpoint{1.931108in}{1.005512in}}{\pgfqpoint{1.939008in}{1.008785in}}{\pgfqpoint{1.944832in}{1.014608in}}%
\pgfpathcurveto{\pgfqpoint{1.950655in}{1.020432in}}{\pgfqpoint{1.953928in}{1.028332in}}{\pgfqpoint{1.953928in}{1.036569in}}%
\pgfpathcurveto{\pgfqpoint{1.953928in}{1.044805in}}{\pgfqpoint{1.950655in}{1.052705in}}{\pgfqpoint{1.944832in}{1.058529in}}%
\pgfpathcurveto{\pgfqpoint{1.939008in}{1.064353in}}{\pgfqpoint{1.931108in}{1.067625in}}{\pgfqpoint{1.922871in}{1.067625in}}%
\pgfpathcurveto{\pgfqpoint{1.914635in}{1.067625in}}{\pgfqpoint{1.906735in}{1.064353in}}{\pgfqpoint{1.900911in}{1.058529in}}%
\pgfpathcurveto{\pgfqpoint{1.895087in}{1.052705in}}{\pgfqpoint{1.891815in}{1.044805in}}{\pgfqpoint{1.891815in}{1.036569in}}%
\pgfpathcurveto{\pgfqpoint{1.891815in}{1.028332in}}{\pgfqpoint{1.895087in}{1.020432in}}{\pgfqpoint{1.900911in}{1.014608in}}%
\pgfpathcurveto{\pgfqpoint{1.906735in}{1.008785in}}{\pgfqpoint{1.914635in}{1.005512in}}{\pgfqpoint{1.922871in}{1.005512in}}%
\pgfpathclose%
\pgfusepath{stroke,fill}%
\end{pgfscope}%
\begin{pgfscope}%
\pgfpathrectangle{\pgfqpoint{0.100000in}{0.212622in}}{\pgfqpoint{3.696000in}{3.696000in}}%
\pgfusepath{clip}%
\pgfsetbuttcap%
\pgfsetroundjoin%
\definecolor{currentfill}{rgb}{0.121569,0.466667,0.705882}%
\pgfsetfillcolor{currentfill}%
\pgfsetfillopacity{0.891902}%
\pgfsetlinewidth{1.003750pt}%
\definecolor{currentstroke}{rgb}{0.121569,0.466667,0.705882}%
\pgfsetstrokecolor{currentstroke}%
\pgfsetstrokeopacity{0.891902}%
\pgfsetdash{}{0pt}%
\pgfpathmoveto{\pgfqpoint{1.927908in}{1.004057in}}%
\pgfpathcurveto{\pgfqpoint{1.936144in}{1.004057in}}{\pgfqpoint{1.944044in}{1.007329in}}{\pgfqpoint{1.949868in}{1.013153in}}%
\pgfpathcurveto{\pgfqpoint{1.955692in}{1.018977in}}{\pgfqpoint{1.958964in}{1.026877in}}{\pgfqpoint{1.958964in}{1.035113in}}%
\pgfpathcurveto{\pgfqpoint{1.958964in}{1.043350in}}{\pgfqpoint{1.955692in}{1.051250in}}{\pgfqpoint{1.949868in}{1.057074in}}%
\pgfpathcurveto{\pgfqpoint{1.944044in}{1.062898in}}{\pgfqpoint{1.936144in}{1.066170in}}{\pgfqpoint{1.927908in}{1.066170in}}%
\pgfpathcurveto{\pgfqpoint{1.919672in}{1.066170in}}{\pgfqpoint{1.911772in}{1.062898in}}{\pgfqpoint{1.905948in}{1.057074in}}%
\pgfpathcurveto{\pgfqpoint{1.900124in}{1.051250in}}{\pgfqpoint{1.896851in}{1.043350in}}{\pgfqpoint{1.896851in}{1.035113in}}%
\pgfpathcurveto{\pgfqpoint{1.896851in}{1.026877in}}{\pgfqpoint{1.900124in}{1.018977in}}{\pgfqpoint{1.905948in}{1.013153in}}%
\pgfpathcurveto{\pgfqpoint{1.911772in}{1.007329in}}{\pgfqpoint{1.919672in}{1.004057in}}{\pgfqpoint{1.927908in}{1.004057in}}%
\pgfpathclose%
\pgfusepath{stroke,fill}%
\end{pgfscope}%
\begin{pgfscope}%
\pgfpathrectangle{\pgfqpoint{0.100000in}{0.212622in}}{\pgfqpoint{3.696000in}{3.696000in}}%
\pgfusepath{clip}%
\pgfsetbuttcap%
\pgfsetroundjoin%
\definecolor{currentfill}{rgb}{0.121569,0.466667,0.705882}%
\pgfsetfillcolor{currentfill}%
\pgfsetfillopacity{0.892727}%
\pgfsetlinewidth{1.003750pt}%
\definecolor{currentstroke}{rgb}{0.121569,0.466667,0.705882}%
\pgfsetstrokecolor{currentstroke}%
\pgfsetstrokeopacity{0.892727}%
\pgfsetdash{}{0pt}%
\pgfpathmoveto{\pgfqpoint{1.931876in}{1.003060in}}%
\pgfpathcurveto{\pgfqpoint{1.940112in}{1.003060in}}{\pgfqpoint{1.948012in}{1.006333in}}{\pgfqpoint{1.953836in}{1.012156in}}%
\pgfpathcurveto{\pgfqpoint{1.959660in}{1.017980in}}{\pgfqpoint{1.962932in}{1.025880in}}{\pgfqpoint{1.962932in}{1.034117in}}%
\pgfpathcurveto{\pgfqpoint{1.962932in}{1.042353in}}{\pgfqpoint{1.959660in}{1.050253in}}{\pgfqpoint{1.953836in}{1.056077in}}%
\pgfpathcurveto{\pgfqpoint{1.948012in}{1.061901in}}{\pgfqpoint{1.940112in}{1.065173in}}{\pgfqpoint{1.931876in}{1.065173in}}%
\pgfpathcurveto{\pgfqpoint{1.923639in}{1.065173in}}{\pgfqpoint{1.915739in}{1.061901in}}{\pgfqpoint{1.909915in}{1.056077in}}%
\pgfpathcurveto{\pgfqpoint{1.904092in}{1.050253in}}{\pgfqpoint{1.900819in}{1.042353in}}{\pgfqpoint{1.900819in}{1.034117in}}%
\pgfpathcurveto{\pgfqpoint{1.900819in}{1.025880in}}{\pgfqpoint{1.904092in}{1.017980in}}{\pgfqpoint{1.909915in}{1.012156in}}%
\pgfpathcurveto{\pgfqpoint{1.915739in}{1.006333in}}{\pgfqpoint{1.923639in}{1.003060in}}{\pgfqpoint{1.931876in}{1.003060in}}%
\pgfpathclose%
\pgfusepath{stroke,fill}%
\end{pgfscope}%
\begin{pgfscope}%
\pgfpathrectangle{\pgfqpoint{0.100000in}{0.212622in}}{\pgfqpoint{3.696000in}{3.696000in}}%
\pgfusepath{clip}%
\pgfsetbuttcap%
\pgfsetroundjoin%
\definecolor{currentfill}{rgb}{0.121569,0.466667,0.705882}%
\pgfsetfillcolor{currentfill}%
\pgfsetfillopacity{0.893545}%
\pgfsetlinewidth{1.003750pt}%
\definecolor{currentstroke}{rgb}{0.121569,0.466667,0.705882}%
\pgfsetstrokecolor{currentstroke}%
\pgfsetstrokeopacity{0.893545}%
\pgfsetdash{}{0pt}%
\pgfpathmoveto{\pgfqpoint{2.352433in}{1.131206in}}%
\pgfpathcurveto{\pgfqpoint{2.360669in}{1.131206in}}{\pgfqpoint{2.368569in}{1.134479in}}{\pgfqpoint{2.374393in}{1.140303in}}%
\pgfpathcurveto{\pgfqpoint{2.380217in}{1.146127in}}{\pgfqpoint{2.383489in}{1.154027in}}{\pgfqpoint{2.383489in}{1.162263in}}%
\pgfpathcurveto{\pgfqpoint{2.383489in}{1.170499in}}{\pgfqpoint{2.380217in}{1.178399in}}{\pgfqpoint{2.374393in}{1.184223in}}%
\pgfpathcurveto{\pgfqpoint{2.368569in}{1.190047in}}{\pgfqpoint{2.360669in}{1.193319in}}{\pgfqpoint{2.352433in}{1.193319in}}%
\pgfpathcurveto{\pgfqpoint{2.344196in}{1.193319in}}{\pgfqpoint{2.336296in}{1.190047in}}{\pgfqpoint{2.330472in}{1.184223in}}%
\pgfpathcurveto{\pgfqpoint{2.324648in}{1.178399in}}{\pgfqpoint{2.321376in}{1.170499in}}{\pgfqpoint{2.321376in}{1.162263in}}%
\pgfpathcurveto{\pgfqpoint{2.321376in}{1.154027in}}{\pgfqpoint{2.324648in}{1.146127in}}{\pgfqpoint{2.330472in}{1.140303in}}%
\pgfpathcurveto{\pgfqpoint{2.336296in}{1.134479in}}{\pgfqpoint{2.344196in}{1.131206in}}{\pgfqpoint{2.352433in}{1.131206in}}%
\pgfpathclose%
\pgfusepath{stroke,fill}%
\end{pgfscope}%
\begin{pgfscope}%
\pgfpathrectangle{\pgfqpoint{0.100000in}{0.212622in}}{\pgfqpoint{3.696000in}{3.696000in}}%
\pgfusepath{clip}%
\pgfsetbuttcap%
\pgfsetroundjoin%
\definecolor{currentfill}{rgb}{0.121569,0.466667,0.705882}%
\pgfsetfillcolor{currentfill}%
\pgfsetfillopacity{0.894184}%
\pgfsetlinewidth{1.003750pt}%
\definecolor{currentstroke}{rgb}{0.121569,0.466667,0.705882}%
\pgfsetstrokecolor{currentstroke}%
\pgfsetstrokeopacity{0.894184}%
\pgfsetdash{}{0pt}%
\pgfpathmoveto{\pgfqpoint{1.939057in}{1.000965in}}%
\pgfpathcurveto{\pgfqpoint{1.947293in}{1.000965in}}{\pgfqpoint{1.955193in}{1.004237in}}{\pgfqpoint{1.961017in}{1.010061in}}%
\pgfpathcurveto{\pgfqpoint{1.966841in}{1.015885in}}{\pgfqpoint{1.970114in}{1.023785in}}{\pgfqpoint{1.970114in}{1.032021in}}%
\pgfpathcurveto{\pgfqpoint{1.970114in}{1.040258in}}{\pgfqpoint{1.966841in}{1.048158in}}{\pgfqpoint{1.961017in}{1.053982in}}%
\pgfpathcurveto{\pgfqpoint{1.955193in}{1.059806in}}{\pgfqpoint{1.947293in}{1.063078in}}{\pgfqpoint{1.939057in}{1.063078in}}%
\pgfpathcurveto{\pgfqpoint{1.930821in}{1.063078in}}{\pgfqpoint{1.922921in}{1.059806in}}{\pgfqpoint{1.917097in}{1.053982in}}%
\pgfpathcurveto{\pgfqpoint{1.911273in}{1.048158in}}{\pgfqpoint{1.908001in}{1.040258in}}{\pgfqpoint{1.908001in}{1.032021in}}%
\pgfpathcurveto{\pgfqpoint{1.908001in}{1.023785in}}{\pgfqpoint{1.911273in}{1.015885in}}{\pgfqpoint{1.917097in}{1.010061in}}%
\pgfpathcurveto{\pgfqpoint{1.922921in}{1.004237in}}{\pgfqpoint{1.930821in}{1.000965in}}{\pgfqpoint{1.939057in}{1.000965in}}%
\pgfpathclose%
\pgfusepath{stroke,fill}%
\end{pgfscope}%
\begin{pgfscope}%
\pgfpathrectangle{\pgfqpoint{0.100000in}{0.212622in}}{\pgfqpoint{3.696000in}{3.696000in}}%
\pgfusepath{clip}%
\pgfsetbuttcap%
\pgfsetroundjoin%
\definecolor{currentfill}{rgb}{0.121569,0.466667,0.705882}%
\pgfsetfillcolor{currentfill}%
\pgfsetfillopacity{0.895296}%
\pgfsetlinewidth{1.003750pt}%
\definecolor{currentstroke}{rgb}{0.121569,0.466667,0.705882}%
\pgfsetstrokecolor{currentstroke}%
\pgfsetstrokeopacity{0.895296}%
\pgfsetdash{}{0pt}%
\pgfpathmoveto{\pgfqpoint{1.944951in}{0.999371in}}%
\pgfpathcurveto{\pgfqpoint{1.953187in}{0.999371in}}{\pgfqpoint{1.961087in}{1.002644in}}{\pgfqpoint{1.966911in}{1.008468in}}%
\pgfpathcurveto{\pgfqpoint{1.972735in}{1.014291in}}{\pgfqpoint{1.976008in}{1.022192in}}{\pgfqpoint{1.976008in}{1.030428in}}%
\pgfpathcurveto{\pgfqpoint{1.976008in}{1.038664in}}{\pgfqpoint{1.972735in}{1.046564in}}{\pgfqpoint{1.966911in}{1.052388in}}%
\pgfpathcurveto{\pgfqpoint{1.961087in}{1.058212in}}{\pgfqpoint{1.953187in}{1.061484in}}{\pgfqpoint{1.944951in}{1.061484in}}%
\pgfpathcurveto{\pgfqpoint{1.936715in}{1.061484in}}{\pgfqpoint{1.928815in}{1.058212in}}{\pgfqpoint{1.922991in}{1.052388in}}%
\pgfpathcurveto{\pgfqpoint{1.917167in}{1.046564in}}{\pgfqpoint{1.913895in}{1.038664in}}{\pgfqpoint{1.913895in}{1.030428in}}%
\pgfpathcurveto{\pgfqpoint{1.913895in}{1.022192in}}{\pgfqpoint{1.917167in}{1.014291in}}{\pgfqpoint{1.922991in}{1.008468in}}%
\pgfpathcurveto{\pgfqpoint{1.928815in}{1.002644in}}{\pgfqpoint{1.936715in}{0.999371in}}{\pgfqpoint{1.944951in}{0.999371in}}%
\pgfpathclose%
\pgfusepath{stroke,fill}%
\end{pgfscope}%
\begin{pgfscope}%
\pgfpathrectangle{\pgfqpoint{0.100000in}{0.212622in}}{\pgfqpoint{3.696000in}{3.696000in}}%
\pgfusepath{clip}%
\pgfsetbuttcap%
\pgfsetroundjoin%
\definecolor{currentfill}{rgb}{0.121569,0.466667,0.705882}%
\pgfsetfillcolor{currentfill}%
\pgfsetfillopacity{0.896386}%
\pgfsetlinewidth{1.003750pt}%
\definecolor{currentstroke}{rgb}{0.121569,0.466667,0.705882}%
\pgfsetstrokecolor{currentstroke}%
\pgfsetstrokeopacity{0.896386}%
\pgfsetdash{}{0pt}%
\pgfpathmoveto{\pgfqpoint{2.355054in}{1.121723in}}%
\pgfpathcurveto{\pgfqpoint{2.363291in}{1.121723in}}{\pgfqpoint{2.371191in}{1.124995in}}{\pgfqpoint{2.377015in}{1.130819in}}%
\pgfpathcurveto{\pgfqpoint{2.382838in}{1.136643in}}{\pgfqpoint{2.386111in}{1.144543in}}{\pgfqpoint{2.386111in}{1.152780in}}%
\pgfpathcurveto{\pgfqpoint{2.386111in}{1.161016in}}{\pgfqpoint{2.382838in}{1.168916in}}{\pgfqpoint{2.377015in}{1.174740in}}%
\pgfpathcurveto{\pgfqpoint{2.371191in}{1.180564in}}{\pgfqpoint{2.363291in}{1.183836in}}{\pgfqpoint{2.355054in}{1.183836in}}%
\pgfpathcurveto{\pgfqpoint{2.346818in}{1.183836in}}{\pgfqpoint{2.338918in}{1.180564in}}{\pgfqpoint{2.333094in}{1.174740in}}%
\pgfpathcurveto{\pgfqpoint{2.327270in}{1.168916in}}{\pgfqpoint{2.323998in}{1.161016in}}{\pgfqpoint{2.323998in}{1.152780in}}%
\pgfpathcurveto{\pgfqpoint{2.323998in}{1.144543in}}{\pgfqpoint{2.327270in}{1.136643in}}{\pgfqpoint{2.333094in}{1.130819in}}%
\pgfpathcurveto{\pgfqpoint{2.338918in}{1.124995in}}{\pgfqpoint{2.346818in}{1.121723in}}{\pgfqpoint{2.355054in}{1.121723in}}%
\pgfpathclose%
\pgfusepath{stroke,fill}%
\end{pgfscope}%
\begin{pgfscope}%
\pgfpathrectangle{\pgfqpoint{0.100000in}{0.212622in}}{\pgfqpoint{3.696000in}{3.696000in}}%
\pgfusepath{clip}%
\pgfsetbuttcap%
\pgfsetroundjoin%
\definecolor{currentfill}{rgb}{0.121569,0.466667,0.705882}%
\pgfsetfillcolor{currentfill}%
\pgfsetfillopacity{0.896502}%
\pgfsetlinewidth{1.003750pt}%
\definecolor{currentstroke}{rgb}{0.121569,0.466667,0.705882}%
\pgfsetstrokecolor{currentstroke}%
\pgfsetstrokeopacity{0.896502}%
\pgfsetdash{}{0pt}%
\pgfpathmoveto{\pgfqpoint{1.950356in}{0.998224in}}%
\pgfpathcurveto{\pgfqpoint{1.958592in}{0.998224in}}{\pgfqpoint{1.966492in}{1.001497in}}{\pgfqpoint{1.972316in}{1.007321in}}%
\pgfpathcurveto{\pgfqpoint{1.978140in}{1.013144in}}{\pgfqpoint{1.981413in}{1.021045in}}{\pgfqpoint{1.981413in}{1.029281in}}%
\pgfpathcurveto{\pgfqpoint{1.981413in}{1.037517in}}{\pgfqpoint{1.978140in}{1.045417in}}{\pgfqpoint{1.972316in}{1.051241in}}%
\pgfpathcurveto{\pgfqpoint{1.966492in}{1.057065in}}{\pgfqpoint{1.958592in}{1.060337in}}{\pgfqpoint{1.950356in}{1.060337in}}%
\pgfpathcurveto{\pgfqpoint{1.942120in}{1.060337in}}{\pgfqpoint{1.934220in}{1.057065in}}{\pgfqpoint{1.928396in}{1.051241in}}%
\pgfpathcurveto{\pgfqpoint{1.922572in}{1.045417in}}{\pgfqpoint{1.919300in}{1.037517in}}{\pgfqpoint{1.919300in}{1.029281in}}%
\pgfpathcurveto{\pgfqpoint{1.919300in}{1.021045in}}{\pgfqpoint{1.922572in}{1.013144in}}{\pgfqpoint{1.928396in}{1.007321in}}%
\pgfpathcurveto{\pgfqpoint{1.934220in}{1.001497in}}{\pgfqpoint{1.942120in}{0.998224in}}{\pgfqpoint{1.950356in}{0.998224in}}%
\pgfpathclose%
\pgfusepath{stroke,fill}%
\end{pgfscope}%
\begin{pgfscope}%
\pgfpathrectangle{\pgfqpoint{0.100000in}{0.212622in}}{\pgfqpoint{3.696000in}{3.696000in}}%
\pgfusepath{clip}%
\pgfsetbuttcap%
\pgfsetroundjoin%
\definecolor{currentfill}{rgb}{0.121569,0.466667,0.705882}%
\pgfsetfillcolor{currentfill}%
\pgfsetfillopacity{0.897533}%
\pgfsetlinewidth{1.003750pt}%
\definecolor{currentstroke}{rgb}{0.121569,0.466667,0.705882}%
\pgfsetstrokecolor{currentstroke}%
\pgfsetstrokeopacity{0.897533}%
\pgfsetdash{}{0pt}%
\pgfpathmoveto{\pgfqpoint{1.955556in}{0.996934in}}%
\pgfpathcurveto{\pgfqpoint{1.963793in}{0.996934in}}{\pgfqpoint{1.971693in}{1.000206in}}{\pgfqpoint{1.977517in}{1.006030in}}%
\pgfpathcurveto{\pgfqpoint{1.983341in}{1.011854in}}{\pgfqpoint{1.986613in}{1.019754in}}{\pgfqpoint{1.986613in}{1.027991in}}%
\pgfpathcurveto{\pgfqpoint{1.986613in}{1.036227in}}{\pgfqpoint{1.983341in}{1.044127in}}{\pgfqpoint{1.977517in}{1.049951in}}%
\pgfpathcurveto{\pgfqpoint{1.971693in}{1.055775in}}{\pgfqpoint{1.963793in}{1.059047in}}{\pgfqpoint{1.955556in}{1.059047in}}%
\pgfpathcurveto{\pgfqpoint{1.947320in}{1.059047in}}{\pgfqpoint{1.939420in}{1.055775in}}{\pgfqpoint{1.933596in}{1.049951in}}%
\pgfpathcurveto{\pgfqpoint{1.927772in}{1.044127in}}{\pgfqpoint{1.924500in}{1.036227in}}{\pgfqpoint{1.924500in}{1.027991in}}%
\pgfpathcurveto{\pgfqpoint{1.924500in}{1.019754in}}{\pgfqpoint{1.927772in}{1.011854in}}{\pgfqpoint{1.933596in}{1.006030in}}%
\pgfpathcurveto{\pgfqpoint{1.939420in}{1.000206in}}{\pgfqpoint{1.947320in}{0.996934in}}{\pgfqpoint{1.955556in}{0.996934in}}%
\pgfpathclose%
\pgfusepath{stroke,fill}%
\end{pgfscope}%
\begin{pgfscope}%
\pgfpathrectangle{\pgfqpoint{0.100000in}{0.212622in}}{\pgfqpoint{3.696000in}{3.696000in}}%
\pgfusepath{clip}%
\pgfsetbuttcap%
\pgfsetroundjoin%
\definecolor{currentfill}{rgb}{0.121569,0.466667,0.705882}%
\pgfsetfillcolor{currentfill}%
\pgfsetfillopacity{0.898342}%
\pgfsetlinewidth{1.003750pt}%
\definecolor{currentstroke}{rgb}{0.121569,0.466667,0.705882}%
\pgfsetstrokecolor{currentstroke}%
\pgfsetstrokeopacity{0.898342}%
\pgfsetdash{}{0pt}%
\pgfpathmoveto{\pgfqpoint{1.959767in}{0.995841in}}%
\pgfpathcurveto{\pgfqpoint{1.968003in}{0.995841in}}{\pgfqpoint{1.975903in}{0.999114in}}{\pgfqpoint{1.981727in}{1.004938in}}%
\pgfpathcurveto{\pgfqpoint{1.987551in}{1.010762in}}{\pgfqpoint{1.990823in}{1.018662in}}{\pgfqpoint{1.990823in}{1.026898in}}%
\pgfpathcurveto{\pgfqpoint{1.990823in}{1.035134in}}{\pgfqpoint{1.987551in}{1.043034in}}{\pgfqpoint{1.981727in}{1.048858in}}%
\pgfpathcurveto{\pgfqpoint{1.975903in}{1.054682in}}{\pgfqpoint{1.968003in}{1.057954in}}{\pgfqpoint{1.959767in}{1.057954in}}%
\pgfpathcurveto{\pgfqpoint{1.951530in}{1.057954in}}{\pgfqpoint{1.943630in}{1.054682in}}{\pgfqpoint{1.937806in}{1.048858in}}%
\pgfpathcurveto{\pgfqpoint{1.931983in}{1.043034in}}{\pgfqpoint{1.928710in}{1.035134in}}{\pgfqpoint{1.928710in}{1.026898in}}%
\pgfpathcurveto{\pgfqpoint{1.928710in}{1.018662in}}{\pgfqpoint{1.931983in}{1.010762in}}{\pgfqpoint{1.937806in}{1.004938in}}%
\pgfpathcurveto{\pgfqpoint{1.943630in}{0.999114in}}{\pgfqpoint{1.951530in}{0.995841in}}{\pgfqpoint{1.959767in}{0.995841in}}%
\pgfpathclose%
\pgfusepath{stroke,fill}%
\end{pgfscope}%
\begin{pgfscope}%
\pgfpathrectangle{\pgfqpoint{0.100000in}{0.212622in}}{\pgfqpoint{3.696000in}{3.696000in}}%
\pgfusepath{clip}%
\pgfsetbuttcap%
\pgfsetroundjoin%
\definecolor{currentfill}{rgb}{0.121569,0.466667,0.705882}%
\pgfsetfillcolor{currentfill}%
\pgfsetfillopacity{0.899168}%
\pgfsetlinewidth{1.003750pt}%
\definecolor{currentstroke}{rgb}{0.121569,0.466667,0.705882}%
\pgfsetstrokecolor{currentstroke}%
\pgfsetstrokeopacity{0.899168}%
\pgfsetdash{}{0pt}%
\pgfpathmoveto{\pgfqpoint{1.963454in}{0.994960in}}%
\pgfpathcurveto{\pgfqpoint{1.971691in}{0.994960in}}{\pgfqpoint{1.979591in}{0.998232in}}{\pgfqpoint{1.985415in}{1.004056in}}%
\pgfpathcurveto{\pgfqpoint{1.991239in}{1.009880in}}{\pgfqpoint{1.994511in}{1.017780in}}{\pgfqpoint{1.994511in}{1.026016in}}%
\pgfpathcurveto{\pgfqpoint{1.994511in}{1.034253in}}{\pgfqpoint{1.991239in}{1.042153in}}{\pgfqpoint{1.985415in}{1.047977in}}%
\pgfpathcurveto{\pgfqpoint{1.979591in}{1.053801in}}{\pgfqpoint{1.971691in}{1.057073in}}{\pgfqpoint{1.963454in}{1.057073in}}%
\pgfpathcurveto{\pgfqpoint{1.955218in}{1.057073in}}{\pgfqpoint{1.947318in}{1.053801in}}{\pgfqpoint{1.941494in}{1.047977in}}%
\pgfpathcurveto{\pgfqpoint{1.935670in}{1.042153in}}{\pgfqpoint{1.932398in}{1.034253in}}{\pgfqpoint{1.932398in}{1.026016in}}%
\pgfpathcurveto{\pgfqpoint{1.932398in}{1.017780in}}{\pgfqpoint{1.935670in}{1.009880in}}{\pgfqpoint{1.941494in}{1.004056in}}%
\pgfpathcurveto{\pgfqpoint{1.947318in}{0.998232in}}{\pgfqpoint{1.955218in}{0.994960in}}{\pgfqpoint{1.963454in}{0.994960in}}%
\pgfpathclose%
\pgfusepath{stroke,fill}%
\end{pgfscope}%
\begin{pgfscope}%
\pgfpathrectangle{\pgfqpoint{0.100000in}{0.212622in}}{\pgfqpoint{3.696000in}{3.696000in}}%
\pgfusepath{clip}%
\pgfsetbuttcap%
\pgfsetroundjoin%
\definecolor{currentfill}{rgb}{0.121569,0.466667,0.705882}%
\pgfsetfillcolor{currentfill}%
\pgfsetfillopacity{0.899212}%
\pgfsetlinewidth{1.003750pt}%
\definecolor{currentstroke}{rgb}{0.121569,0.466667,0.705882}%
\pgfsetstrokecolor{currentstroke}%
\pgfsetstrokeopacity{0.899212}%
\pgfsetdash{}{0pt}%
\pgfpathmoveto{\pgfqpoint{2.358268in}{1.111558in}}%
\pgfpathcurveto{\pgfqpoint{2.366505in}{1.111558in}}{\pgfqpoint{2.374405in}{1.114830in}}{\pgfqpoint{2.380229in}{1.120654in}}%
\pgfpathcurveto{\pgfqpoint{2.386053in}{1.126478in}}{\pgfqpoint{2.389325in}{1.134378in}}{\pgfqpoint{2.389325in}{1.142615in}}%
\pgfpathcurveto{\pgfqpoint{2.389325in}{1.150851in}}{\pgfqpoint{2.386053in}{1.158751in}}{\pgfqpoint{2.380229in}{1.164575in}}%
\pgfpathcurveto{\pgfqpoint{2.374405in}{1.170399in}}{\pgfqpoint{2.366505in}{1.173671in}}{\pgfqpoint{2.358268in}{1.173671in}}%
\pgfpathcurveto{\pgfqpoint{2.350032in}{1.173671in}}{\pgfqpoint{2.342132in}{1.170399in}}{\pgfqpoint{2.336308in}{1.164575in}}%
\pgfpathcurveto{\pgfqpoint{2.330484in}{1.158751in}}{\pgfqpoint{2.327212in}{1.150851in}}{\pgfqpoint{2.327212in}{1.142615in}}%
\pgfpathcurveto{\pgfqpoint{2.327212in}{1.134378in}}{\pgfqpoint{2.330484in}{1.126478in}}{\pgfqpoint{2.336308in}{1.120654in}}%
\pgfpathcurveto{\pgfqpoint{2.342132in}{1.114830in}}{\pgfqpoint{2.350032in}{1.111558in}}{\pgfqpoint{2.358268in}{1.111558in}}%
\pgfpathclose%
\pgfusepath{stroke,fill}%
\end{pgfscope}%
\begin{pgfscope}%
\pgfpathrectangle{\pgfqpoint{0.100000in}{0.212622in}}{\pgfqpoint{3.696000in}{3.696000in}}%
\pgfusepath{clip}%
\pgfsetbuttcap%
\pgfsetroundjoin%
\definecolor{currentfill}{rgb}{0.121569,0.466667,0.705882}%
\pgfsetfillcolor{currentfill}%
\pgfsetfillopacity{0.900537}%
\pgfsetlinewidth{1.003750pt}%
\definecolor{currentstroke}{rgb}{0.121569,0.466667,0.705882}%
\pgfsetstrokecolor{currentstroke}%
\pgfsetstrokeopacity{0.900537}%
\pgfsetdash{}{0pt}%
\pgfpathmoveto{\pgfqpoint{1.970237in}{0.993170in}}%
\pgfpathcurveto{\pgfqpoint{1.978473in}{0.993170in}}{\pgfqpoint{1.986373in}{0.996442in}}{\pgfqpoint{1.992197in}{1.002266in}}%
\pgfpathcurveto{\pgfqpoint{1.998021in}{1.008090in}}{\pgfqpoint{2.001294in}{1.015990in}}{\pgfqpoint{2.001294in}{1.024226in}}%
\pgfpathcurveto{\pgfqpoint{2.001294in}{1.032463in}}{\pgfqpoint{1.998021in}{1.040363in}}{\pgfqpoint{1.992197in}{1.046187in}}%
\pgfpathcurveto{\pgfqpoint{1.986373in}{1.052010in}}{\pgfqpoint{1.978473in}{1.055283in}}{\pgfqpoint{1.970237in}{1.055283in}}%
\pgfpathcurveto{\pgfqpoint{1.962001in}{1.055283in}}{\pgfqpoint{1.954101in}{1.052010in}}{\pgfqpoint{1.948277in}{1.046187in}}%
\pgfpathcurveto{\pgfqpoint{1.942453in}{1.040363in}}{\pgfqpoint{1.939181in}{1.032463in}}{\pgfqpoint{1.939181in}{1.024226in}}%
\pgfpathcurveto{\pgfqpoint{1.939181in}{1.015990in}}{\pgfqpoint{1.942453in}{1.008090in}}{\pgfqpoint{1.948277in}{1.002266in}}%
\pgfpathcurveto{\pgfqpoint{1.954101in}{0.996442in}}{\pgfqpoint{1.962001in}{0.993170in}}{\pgfqpoint{1.970237in}{0.993170in}}%
\pgfpathclose%
\pgfusepath{stroke,fill}%
\end{pgfscope}%
\begin{pgfscope}%
\pgfpathrectangle{\pgfqpoint{0.100000in}{0.212622in}}{\pgfqpoint{3.696000in}{3.696000in}}%
\pgfusepath{clip}%
\pgfsetbuttcap%
\pgfsetroundjoin%
\definecolor{currentfill}{rgb}{0.121569,0.466667,0.705882}%
\pgfsetfillcolor{currentfill}%
\pgfsetfillopacity{0.901542}%
\pgfsetlinewidth{1.003750pt}%
\definecolor{currentstroke}{rgb}{0.121569,0.466667,0.705882}%
\pgfsetstrokecolor{currentstroke}%
\pgfsetstrokeopacity{0.901542}%
\pgfsetdash{}{0pt}%
\pgfpathmoveto{\pgfqpoint{1.975658in}{0.991611in}}%
\pgfpathcurveto{\pgfqpoint{1.983895in}{0.991611in}}{\pgfqpoint{1.991795in}{0.994883in}}{\pgfqpoint{1.997619in}{1.000707in}}%
\pgfpathcurveto{\pgfqpoint{2.003442in}{1.006531in}}{\pgfqpoint{2.006715in}{1.014431in}}{\pgfqpoint{2.006715in}{1.022668in}}%
\pgfpathcurveto{\pgfqpoint{2.006715in}{1.030904in}}{\pgfqpoint{2.003442in}{1.038804in}}{\pgfqpoint{1.997619in}{1.044628in}}%
\pgfpathcurveto{\pgfqpoint{1.991795in}{1.050452in}}{\pgfqpoint{1.983895in}{1.053724in}}{\pgfqpoint{1.975658in}{1.053724in}}%
\pgfpathcurveto{\pgfqpoint{1.967422in}{1.053724in}}{\pgfqpoint{1.959522in}{1.050452in}}{\pgfqpoint{1.953698in}{1.044628in}}%
\pgfpathcurveto{\pgfqpoint{1.947874in}{1.038804in}}{\pgfqpoint{1.944602in}{1.030904in}}{\pgfqpoint{1.944602in}{1.022668in}}%
\pgfpathcurveto{\pgfqpoint{1.944602in}{1.014431in}}{\pgfqpoint{1.947874in}{1.006531in}}{\pgfqpoint{1.953698in}{1.000707in}}%
\pgfpathcurveto{\pgfqpoint{1.959522in}{0.994883in}}{\pgfqpoint{1.967422in}{0.991611in}}{\pgfqpoint{1.975658in}{0.991611in}}%
\pgfpathclose%
\pgfusepath{stroke,fill}%
\end{pgfscope}%
\begin{pgfscope}%
\pgfpathrectangle{\pgfqpoint{0.100000in}{0.212622in}}{\pgfqpoint{3.696000in}{3.696000in}}%
\pgfusepath{clip}%
\pgfsetbuttcap%
\pgfsetroundjoin%
\definecolor{currentfill}{rgb}{0.121569,0.466667,0.705882}%
\pgfsetfillcolor{currentfill}%
\pgfsetfillopacity{0.902620}%
\pgfsetlinewidth{1.003750pt}%
\definecolor{currentstroke}{rgb}{0.121569,0.466667,0.705882}%
\pgfsetstrokecolor{currentstroke}%
\pgfsetstrokeopacity{0.902620}%
\pgfsetdash{}{0pt}%
\pgfpathmoveto{\pgfqpoint{1.980609in}{0.990604in}}%
\pgfpathcurveto{\pgfqpoint{1.988845in}{0.990604in}}{\pgfqpoint{1.996745in}{0.993877in}}{\pgfqpoint{2.002569in}{0.999701in}}%
\pgfpathcurveto{\pgfqpoint{2.008393in}{1.005524in}}{\pgfqpoint{2.011665in}{1.013425in}}{\pgfqpoint{2.011665in}{1.021661in}}%
\pgfpathcurveto{\pgfqpoint{2.011665in}{1.029897in}}{\pgfqpoint{2.008393in}{1.037797in}}{\pgfqpoint{2.002569in}{1.043621in}}%
\pgfpathcurveto{\pgfqpoint{1.996745in}{1.049445in}}{\pgfqpoint{1.988845in}{1.052717in}}{\pgfqpoint{1.980609in}{1.052717in}}%
\pgfpathcurveto{\pgfqpoint{1.972373in}{1.052717in}}{\pgfqpoint{1.964473in}{1.049445in}}{\pgfqpoint{1.958649in}{1.043621in}}%
\pgfpathcurveto{\pgfqpoint{1.952825in}{1.037797in}}{\pgfqpoint{1.949552in}{1.029897in}}{\pgfqpoint{1.949552in}{1.021661in}}%
\pgfpathcurveto{\pgfqpoint{1.949552in}{1.013425in}}{\pgfqpoint{1.952825in}{1.005524in}}{\pgfqpoint{1.958649in}{0.999701in}}%
\pgfpathcurveto{\pgfqpoint{1.964473in}{0.993877in}}{\pgfqpoint{1.972373in}{0.990604in}}{\pgfqpoint{1.980609in}{0.990604in}}%
\pgfpathclose%
\pgfusepath{stroke,fill}%
\end{pgfscope}%
\begin{pgfscope}%
\pgfpathrectangle{\pgfqpoint{0.100000in}{0.212622in}}{\pgfqpoint{3.696000in}{3.696000in}}%
\pgfusepath{clip}%
\pgfsetbuttcap%
\pgfsetroundjoin%
\definecolor{currentfill}{rgb}{0.121569,0.466667,0.705882}%
\pgfsetfillcolor{currentfill}%
\pgfsetfillopacity{0.902674}%
\pgfsetlinewidth{1.003750pt}%
\definecolor{currentstroke}{rgb}{0.121569,0.466667,0.705882}%
\pgfsetstrokecolor{currentstroke}%
\pgfsetstrokeopacity{0.902674}%
\pgfsetdash{}{0pt}%
\pgfpathmoveto{\pgfqpoint{2.361246in}{1.100252in}}%
\pgfpathcurveto{\pgfqpoint{2.369482in}{1.100252in}}{\pgfqpoint{2.377382in}{1.103524in}}{\pgfqpoint{2.383206in}{1.109348in}}%
\pgfpathcurveto{\pgfqpoint{2.389030in}{1.115172in}}{\pgfqpoint{2.392302in}{1.123072in}}{\pgfqpoint{2.392302in}{1.131309in}}%
\pgfpathcurveto{\pgfqpoint{2.392302in}{1.139545in}}{\pgfqpoint{2.389030in}{1.147445in}}{\pgfqpoint{2.383206in}{1.153269in}}%
\pgfpathcurveto{\pgfqpoint{2.377382in}{1.159093in}}{\pgfqpoint{2.369482in}{1.162365in}}{\pgfqpoint{2.361246in}{1.162365in}}%
\pgfpathcurveto{\pgfqpoint{2.353010in}{1.162365in}}{\pgfqpoint{2.345110in}{1.159093in}}{\pgfqpoint{2.339286in}{1.153269in}}%
\pgfpathcurveto{\pgfqpoint{2.333462in}{1.147445in}}{\pgfqpoint{2.330189in}{1.139545in}}{\pgfqpoint{2.330189in}{1.131309in}}%
\pgfpathcurveto{\pgfqpoint{2.330189in}{1.123072in}}{\pgfqpoint{2.333462in}{1.115172in}}{\pgfqpoint{2.339286in}{1.109348in}}%
\pgfpathcurveto{\pgfqpoint{2.345110in}{1.103524in}}{\pgfqpoint{2.353010in}{1.100252in}}{\pgfqpoint{2.361246in}{1.100252in}}%
\pgfpathclose%
\pgfusepath{stroke,fill}%
\end{pgfscope}%
\begin{pgfscope}%
\pgfpathrectangle{\pgfqpoint{0.100000in}{0.212622in}}{\pgfqpoint{3.696000in}{3.696000in}}%
\pgfusepath{clip}%
\pgfsetbuttcap%
\pgfsetroundjoin%
\definecolor{currentfill}{rgb}{0.121569,0.466667,0.705882}%
\pgfsetfillcolor{currentfill}%
\pgfsetfillopacity{0.903640}%
\pgfsetlinewidth{1.003750pt}%
\definecolor{currentstroke}{rgb}{0.121569,0.466667,0.705882}%
\pgfsetstrokecolor{currentstroke}%
\pgfsetstrokeopacity{0.903640}%
\pgfsetdash{}{0pt}%
\pgfpathmoveto{\pgfqpoint{1.985102in}{0.989646in}}%
\pgfpathcurveto{\pgfqpoint{1.993339in}{0.989646in}}{\pgfqpoint{2.001239in}{0.992918in}}{\pgfqpoint{2.007063in}{0.998742in}}%
\pgfpathcurveto{\pgfqpoint{2.012886in}{1.004566in}}{\pgfqpoint{2.016159in}{1.012466in}}{\pgfqpoint{2.016159in}{1.020703in}}%
\pgfpathcurveto{\pgfqpoint{2.016159in}{1.028939in}}{\pgfqpoint{2.012886in}{1.036839in}}{\pgfqpoint{2.007063in}{1.042663in}}%
\pgfpathcurveto{\pgfqpoint{2.001239in}{1.048487in}}{\pgfqpoint{1.993339in}{1.051759in}}{\pgfqpoint{1.985102in}{1.051759in}}%
\pgfpathcurveto{\pgfqpoint{1.976866in}{1.051759in}}{\pgfqpoint{1.968966in}{1.048487in}}{\pgfqpoint{1.963142in}{1.042663in}}%
\pgfpathcurveto{\pgfqpoint{1.957318in}{1.036839in}}{\pgfqpoint{1.954046in}{1.028939in}}{\pgfqpoint{1.954046in}{1.020703in}}%
\pgfpathcurveto{\pgfqpoint{1.954046in}{1.012466in}}{\pgfqpoint{1.957318in}{1.004566in}}{\pgfqpoint{1.963142in}{0.998742in}}%
\pgfpathcurveto{\pgfqpoint{1.968966in}{0.992918in}}{\pgfqpoint{1.976866in}{0.989646in}}{\pgfqpoint{1.985102in}{0.989646in}}%
\pgfpathclose%
\pgfusepath{stroke,fill}%
\end{pgfscope}%
\begin{pgfscope}%
\pgfpathrectangle{\pgfqpoint{0.100000in}{0.212622in}}{\pgfqpoint{3.696000in}{3.696000in}}%
\pgfusepath{clip}%
\pgfsetbuttcap%
\pgfsetroundjoin%
\definecolor{currentfill}{rgb}{0.121569,0.466667,0.705882}%
\pgfsetfillcolor{currentfill}%
\pgfsetfillopacity{0.904259}%
\pgfsetlinewidth{1.003750pt}%
\definecolor{currentstroke}{rgb}{0.121569,0.466667,0.705882}%
\pgfsetstrokecolor{currentstroke}%
\pgfsetstrokeopacity{0.904259}%
\pgfsetdash{}{0pt}%
\pgfpathmoveto{\pgfqpoint{1.988261in}{0.988853in}}%
\pgfpathcurveto{\pgfqpoint{1.996497in}{0.988853in}}{\pgfqpoint{2.004397in}{0.992125in}}{\pgfqpoint{2.010221in}{0.997949in}}%
\pgfpathcurveto{\pgfqpoint{2.016045in}{1.003773in}}{\pgfqpoint{2.019317in}{1.011673in}}{\pgfqpoint{2.019317in}{1.019909in}}%
\pgfpathcurveto{\pgfqpoint{2.019317in}{1.028146in}}{\pgfqpoint{2.016045in}{1.036046in}}{\pgfqpoint{2.010221in}{1.041870in}}%
\pgfpathcurveto{\pgfqpoint{2.004397in}{1.047694in}}{\pgfqpoint{1.996497in}{1.050966in}}{\pgfqpoint{1.988261in}{1.050966in}}%
\pgfpathcurveto{\pgfqpoint{1.980024in}{1.050966in}}{\pgfqpoint{1.972124in}{1.047694in}}{\pgfqpoint{1.966300in}{1.041870in}}%
\pgfpathcurveto{\pgfqpoint{1.960476in}{1.036046in}}{\pgfqpoint{1.957204in}{1.028146in}}{\pgfqpoint{1.957204in}{1.019909in}}%
\pgfpathcurveto{\pgfqpoint{1.957204in}{1.011673in}}{\pgfqpoint{1.960476in}{1.003773in}}{\pgfqpoint{1.966300in}{0.997949in}}%
\pgfpathcurveto{\pgfqpoint{1.972124in}{0.992125in}}{\pgfqpoint{1.980024in}{0.988853in}}{\pgfqpoint{1.988261in}{0.988853in}}%
\pgfpathclose%
\pgfusepath{stroke,fill}%
\end{pgfscope}%
\begin{pgfscope}%
\pgfpathrectangle{\pgfqpoint{0.100000in}{0.212622in}}{\pgfqpoint{3.696000in}{3.696000in}}%
\pgfusepath{clip}%
\pgfsetbuttcap%
\pgfsetroundjoin%
\definecolor{currentfill}{rgb}{0.121569,0.466667,0.705882}%
\pgfsetfillcolor{currentfill}%
\pgfsetfillopacity{0.905472}%
\pgfsetlinewidth{1.003750pt}%
\definecolor{currentstroke}{rgb}{0.121569,0.466667,0.705882}%
\pgfsetstrokecolor{currentstroke}%
\pgfsetstrokeopacity{0.905472}%
\pgfsetdash{}{0pt}%
\pgfpathmoveto{\pgfqpoint{1.993931in}{0.987443in}}%
\pgfpathcurveto{\pgfqpoint{2.002168in}{0.987443in}}{\pgfqpoint{2.010068in}{0.990715in}}{\pgfqpoint{2.015892in}{0.996539in}}%
\pgfpathcurveto{\pgfqpoint{2.021715in}{1.002363in}}{\pgfqpoint{2.024988in}{1.010263in}}{\pgfqpoint{2.024988in}{1.018499in}}%
\pgfpathcurveto{\pgfqpoint{2.024988in}{1.026736in}}{\pgfqpoint{2.021715in}{1.034636in}}{\pgfqpoint{2.015892in}{1.040460in}}%
\pgfpathcurveto{\pgfqpoint{2.010068in}{1.046283in}}{\pgfqpoint{2.002168in}{1.049556in}}{\pgfqpoint{1.993931in}{1.049556in}}%
\pgfpathcurveto{\pgfqpoint{1.985695in}{1.049556in}}{\pgfqpoint{1.977795in}{1.046283in}}{\pgfqpoint{1.971971in}{1.040460in}}%
\pgfpathcurveto{\pgfqpoint{1.966147in}{1.034636in}}{\pgfqpoint{1.962875in}{1.026736in}}{\pgfqpoint{1.962875in}{1.018499in}}%
\pgfpathcurveto{\pgfqpoint{1.962875in}{1.010263in}}{\pgfqpoint{1.966147in}{1.002363in}}{\pgfqpoint{1.971971in}{0.996539in}}%
\pgfpathcurveto{\pgfqpoint{1.977795in}{0.990715in}}{\pgfqpoint{1.985695in}{0.987443in}}{\pgfqpoint{1.993931in}{0.987443in}}%
\pgfpathclose%
\pgfusepath{stroke,fill}%
\end{pgfscope}%
\begin{pgfscope}%
\pgfpathrectangle{\pgfqpoint{0.100000in}{0.212622in}}{\pgfqpoint{3.696000in}{3.696000in}}%
\pgfusepath{clip}%
\pgfsetbuttcap%
\pgfsetroundjoin%
\definecolor{currentfill}{rgb}{0.121569,0.466667,0.705882}%
\pgfsetfillcolor{currentfill}%
\pgfsetfillopacity{0.906398}%
\pgfsetlinewidth{1.003750pt}%
\definecolor{currentstroke}{rgb}{0.121569,0.466667,0.705882}%
\pgfsetstrokecolor{currentstroke}%
\pgfsetstrokeopacity{0.906398}%
\pgfsetdash{}{0pt}%
\pgfpathmoveto{\pgfqpoint{2.364357in}{1.087949in}}%
\pgfpathcurveto{\pgfqpoint{2.372593in}{1.087949in}}{\pgfqpoint{2.380493in}{1.091221in}}{\pgfqpoint{2.386317in}{1.097045in}}%
\pgfpathcurveto{\pgfqpoint{2.392141in}{1.102869in}}{\pgfqpoint{2.395413in}{1.110769in}}{\pgfqpoint{2.395413in}{1.119005in}}%
\pgfpathcurveto{\pgfqpoint{2.395413in}{1.127241in}}{\pgfqpoint{2.392141in}{1.135141in}}{\pgfqpoint{2.386317in}{1.140965in}}%
\pgfpathcurveto{\pgfqpoint{2.380493in}{1.146789in}}{\pgfqpoint{2.372593in}{1.150062in}}{\pgfqpoint{2.364357in}{1.150062in}}%
\pgfpathcurveto{\pgfqpoint{2.356121in}{1.150062in}}{\pgfqpoint{2.348221in}{1.146789in}}{\pgfqpoint{2.342397in}{1.140965in}}%
\pgfpathcurveto{\pgfqpoint{2.336573in}{1.135141in}}{\pgfqpoint{2.333300in}{1.127241in}}{\pgfqpoint{2.333300in}{1.119005in}}%
\pgfpathcurveto{\pgfqpoint{2.333300in}{1.110769in}}{\pgfqpoint{2.336573in}{1.102869in}}{\pgfqpoint{2.342397in}{1.097045in}}%
\pgfpathcurveto{\pgfqpoint{2.348221in}{1.091221in}}{\pgfqpoint{2.356121in}{1.087949in}}{\pgfqpoint{2.364357in}{1.087949in}}%
\pgfpathclose%
\pgfusepath{stroke,fill}%
\end{pgfscope}%
\begin{pgfscope}%
\pgfpathrectangle{\pgfqpoint{0.100000in}{0.212622in}}{\pgfqpoint{3.696000in}{3.696000in}}%
\pgfusepath{clip}%
\pgfsetbuttcap%
\pgfsetroundjoin%
\definecolor{currentfill}{rgb}{0.121569,0.466667,0.705882}%
\pgfsetfillcolor{currentfill}%
\pgfsetfillopacity{0.906565}%
\pgfsetlinewidth{1.003750pt}%
\definecolor{currentstroke}{rgb}{0.121569,0.466667,0.705882}%
\pgfsetstrokecolor{currentstroke}%
\pgfsetstrokeopacity{0.906565}%
\pgfsetdash{}{0pt}%
\pgfpathmoveto{\pgfqpoint{1.999003in}{0.986319in}}%
\pgfpathcurveto{\pgfqpoint{2.007239in}{0.986319in}}{\pgfqpoint{2.015139in}{0.989591in}}{\pgfqpoint{2.020963in}{0.995415in}}%
\pgfpathcurveto{\pgfqpoint{2.026787in}{1.001239in}}{\pgfqpoint{2.030060in}{1.009139in}}{\pgfqpoint{2.030060in}{1.017375in}}%
\pgfpathcurveto{\pgfqpoint{2.030060in}{1.025612in}}{\pgfqpoint{2.026787in}{1.033512in}}{\pgfqpoint{2.020963in}{1.039336in}}%
\pgfpathcurveto{\pgfqpoint{2.015139in}{1.045160in}}{\pgfqpoint{2.007239in}{1.048432in}}{\pgfqpoint{1.999003in}{1.048432in}}%
\pgfpathcurveto{\pgfqpoint{1.990767in}{1.048432in}}{\pgfqpoint{1.982867in}{1.045160in}}{\pgfqpoint{1.977043in}{1.039336in}}%
\pgfpathcurveto{\pgfqpoint{1.971219in}{1.033512in}}{\pgfqpoint{1.967947in}{1.025612in}}{\pgfqpoint{1.967947in}{1.017375in}}%
\pgfpathcurveto{\pgfqpoint{1.967947in}{1.009139in}}{\pgfqpoint{1.971219in}{1.001239in}}{\pgfqpoint{1.977043in}{0.995415in}}%
\pgfpathcurveto{\pgfqpoint{1.982867in}{0.989591in}}{\pgfqpoint{1.990767in}{0.986319in}}{\pgfqpoint{1.999003in}{0.986319in}}%
\pgfpathclose%
\pgfusepath{stroke,fill}%
\end{pgfscope}%
\begin{pgfscope}%
\pgfpathrectangle{\pgfqpoint{0.100000in}{0.212622in}}{\pgfqpoint{3.696000in}{3.696000in}}%
\pgfusepath{clip}%
\pgfsetbuttcap%
\pgfsetroundjoin%
\definecolor{currentfill}{rgb}{0.121569,0.466667,0.705882}%
\pgfsetfillcolor{currentfill}%
\pgfsetfillopacity{0.907322}%
\pgfsetlinewidth{1.003750pt}%
\definecolor{currentstroke}{rgb}{0.121569,0.466667,0.705882}%
\pgfsetstrokecolor{currentstroke}%
\pgfsetstrokeopacity{0.907322}%
\pgfsetdash{}{0pt}%
\pgfpathmoveto{\pgfqpoint{2.002924in}{0.985293in}}%
\pgfpathcurveto{\pgfqpoint{2.011161in}{0.985293in}}{\pgfqpoint{2.019061in}{0.988565in}}{\pgfqpoint{2.024885in}{0.994389in}}%
\pgfpathcurveto{\pgfqpoint{2.030709in}{1.000213in}}{\pgfqpoint{2.033981in}{1.008113in}}{\pgfqpoint{2.033981in}{1.016349in}}%
\pgfpathcurveto{\pgfqpoint{2.033981in}{1.024586in}}{\pgfqpoint{2.030709in}{1.032486in}}{\pgfqpoint{2.024885in}{1.038310in}}%
\pgfpathcurveto{\pgfqpoint{2.019061in}{1.044134in}}{\pgfqpoint{2.011161in}{1.047406in}}{\pgfqpoint{2.002924in}{1.047406in}}%
\pgfpathcurveto{\pgfqpoint{1.994688in}{1.047406in}}{\pgfqpoint{1.986788in}{1.044134in}}{\pgfqpoint{1.980964in}{1.038310in}}%
\pgfpathcurveto{\pgfqpoint{1.975140in}{1.032486in}}{\pgfqpoint{1.971868in}{1.024586in}}{\pgfqpoint{1.971868in}{1.016349in}}%
\pgfpathcurveto{\pgfqpoint{1.971868in}{1.008113in}}{\pgfqpoint{1.975140in}{1.000213in}}{\pgfqpoint{1.980964in}{0.994389in}}%
\pgfpathcurveto{\pgfqpoint{1.986788in}{0.988565in}}{\pgfqpoint{1.994688in}{0.985293in}}{\pgfqpoint{2.002924in}{0.985293in}}%
\pgfpathclose%
\pgfusepath{stroke,fill}%
\end{pgfscope}%
\begin{pgfscope}%
\pgfpathrectangle{\pgfqpoint{0.100000in}{0.212622in}}{\pgfqpoint{3.696000in}{3.696000in}}%
\pgfusepath{clip}%
\pgfsetbuttcap%
\pgfsetroundjoin%
\definecolor{currentfill}{rgb}{0.121569,0.466667,0.705882}%
\pgfsetfillcolor{currentfill}%
\pgfsetfillopacity{0.908820}%
\pgfsetlinewidth{1.003750pt}%
\definecolor{currentstroke}{rgb}{0.121569,0.466667,0.705882}%
\pgfsetstrokecolor{currentstroke}%
\pgfsetstrokeopacity{0.908820}%
\pgfsetdash{}{0pt}%
\pgfpathmoveto{\pgfqpoint{2.010031in}{0.983764in}}%
\pgfpathcurveto{\pgfqpoint{2.018267in}{0.983764in}}{\pgfqpoint{2.026167in}{0.987036in}}{\pgfqpoint{2.031991in}{0.992860in}}%
\pgfpathcurveto{\pgfqpoint{2.037815in}{0.998684in}}{\pgfqpoint{2.041087in}{1.006584in}}{\pgfqpoint{2.041087in}{1.014820in}}%
\pgfpathcurveto{\pgfqpoint{2.041087in}{1.023057in}}{\pgfqpoint{2.037815in}{1.030957in}}{\pgfqpoint{2.031991in}{1.036781in}}%
\pgfpathcurveto{\pgfqpoint{2.026167in}{1.042605in}}{\pgfqpoint{2.018267in}{1.045877in}}{\pgfqpoint{2.010031in}{1.045877in}}%
\pgfpathcurveto{\pgfqpoint{2.001794in}{1.045877in}}{\pgfqpoint{1.993894in}{1.042605in}}{\pgfqpoint{1.988070in}{1.036781in}}%
\pgfpathcurveto{\pgfqpoint{1.982246in}{1.030957in}}{\pgfqpoint{1.978974in}{1.023057in}}{\pgfqpoint{1.978974in}{1.014820in}}%
\pgfpathcurveto{\pgfqpoint{1.978974in}{1.006584in}}{\pgfqpoint{1.982246in}{0.998684in}}{\pgfqpoint{1.988070in}{0.992860in}}%
\pgfpathcurveto{\pgfqpoint{1.993894in}{0.987036in}}{\pgfqpoint{2.001794in}{0.983764in}}{\pgfqpoint{2.010031in}{0.983764in}}%
\pgfpathclose%
\pgfusepath{stroke,fill}%
\end{pgfscope}%
\begin{pgfscope}%
\pgfpathrectangle{\pgfqpoint{0.100000in}{0.212622in}}{\pgfqpoint{3.696000in}{3.696000in}}%
\pgfusepath{clip}%
\pgfsetbuttcap%
\pgfsetroundjoin%
\definecolor{currentfill}{rgb}{0.121569,0.466667,0.705882}%
\pgfsetfillcolor{currentfill}%
\pgfsetfillopacity{0.910180}%
\pgfsetlinewidth{1.003750pt}%
\definecolor{currentstroke}{rgb}{0.121569,0.466667,0.705882}%
\pgfsetstrokecolor{currentstroke}%
\pgfsetstrokeopacity{0.910180}%
\pgfsetdash{}{0pt}%
\pgfpathmoveto{\pgfqpoint{2.368197in}{1.075035in}}%
\pgfpathcurveto{\pgfqpoint{2.376434in}{1.075035in}}{\pgfqpoint{2.384334in}{1.078308in}}{\pgfqpoint{2.390158in}{1.084132in}}%
\pgfpathcurveto{\pgfqpoint{2.395982in}{1.089955in}}{\pgfqpoint{2.399254in}{1.097856in}}{\pgfqpoint{2.399254in}{1.106092in}}%
\pgfpathcurveto{\pgfqpoint{2.399254in}{1.114328in}}{\pgfqpoint{2.395982in}{1.122228in}}{\pgfqpoint{2.390158in}{1.128052in}}%
\pgfpathcurveto{\pgfqpoint{2.384334in}{1.133876in}}{\pgfqpoint{2.376434in}{1.137148in}}{\pgfqpoint{2.368197in}{1.137148in}}%
\pgfpathcurveto{\pgfqpoint{2.359961in}{1.137148in}}{\pgfqpoint{2.352061in}{1.133876in}}{\pgfqpoint{2.346237in}{1.128052in}}%
\pgfpathcurveto{\pgfqpoint{2.340413in}{1.122228in}}{\pgfqpoint{2.337141in}{1.114328in}}{\pgfqpoint{2.337141in}{1.106092in}}%
\pgfpathcurveto{\pgfqpoint{2.337141in}{1.097856in}}{\pgfqpoint{2.340413in}{1.089955in}}{\pgfqpoint{2.346237in}{1.084132in}}%
\pgfpathcurveto{\pgfqpoint{2.352061in}{1.078308in}}{\pgfqpoint{2.359961in}{1.075035in}}{\pgfqpoint{2.368197in}{1.075035in}}%
\pgfpathclose%
\pgfusepath{stroke,fill}%
\end{pgfscope}%
\begin{pgfscope}%
\pgfpathrectangle{\pgfqpoint{0.100000in}{0.212622in}}{\pgfqpoint{3.696000in}{3.696000in}}%
\pgfusepath{clip}%
\pgfsetbuttcap%
\pgfsetroundjoin%
\definecolor{currentfill}{rgb}{0.121569,0.466667,0.705882}%
\pgfsetfillcolor{currentfill}%
\pgfsetfillopacity{0.910248}%
\pgfsetlinewidth{1.003750pt}%
\definecolor{currentstroke}{rgb}{0.121569,0.466667,0.705882}%
\pgfsetstrokecolor{currentstroke}%
\pgfsetstrokeopacity{0.910248}%
\pgfsetdash{}{0pt}%
\pgfpathmoveto{\pgfqpoint{2.016367in}{0.982370in}}%
\pgfpathcurveto{\pgfqpoint{2.024603in}{0.982370in}}{\pgfqpoint{2.032503in}{0.985642in}}{\pgfqpoint{2.038327in}{0.991466in}}%
\pgfpathcurveto{\pgfqpoint{2.044151in}{0.997290in}}{\pgfqpoint{2.047423in}{1.005190in}}{\pgfqpoint{2.047423in}{1.013427in}}%
\pgfpathcurveto{\pgfqpoint{2.047423in}{1.021663in}}{\pgfqpoint{2.044151in}{1.029563in}}{\pgfqpoint{2.038327in}{1.035387in}}%
\pgfpathcurveto{\pgfqpoint{2.032503in}{1.041211in}}{\pgfqpoint{2.024603in}{1.044483in}}{\pgfqpoint{2.016367in}{1.044483in}}%
\pgfpathcurveto{\pgfqpoint{2.008130in}{1.044483in}}{\pgfqpoint{2.000230in}{1.041211in}}{\pgfqpoint{1.994406in}{1.035387in}}%
\pgfpathcurveto{\pgfqpoint{1.988583in}{1.029563in}}{\pgfqpoint{1.985310in}{1.021663in}}{\pgfqpoint{1.985310in}{1.013427in}}%
\pgfpathcurveto{\pgfqpoint{1.985310in}{1.005190in}}{\pgfqpoint{1.988583in}{0.997290in}}{\pgfqpoint{1.994406in}{0.991466in}}%
\pgfpathcurveto{\pgfqpoint{2.000230in}{0.985642in}}{\pgfqpoint{2.008130in}{0.982370in}}{\pgfqpoint{2.016367in}{0.982370in}}%
\pgfpathclose%
\pgfusepath{stroke,fill}%
\end{pgfscope}%
\begin{pgfscope}%
\pgfpathrectangle{\pgfqpoint{0.100000in}{0.212622in}}{\pgfqpoint{3.696000in}{3.696000in}}%
\pgfusepath{clip}%
\pgfsetbuttcap%
\pgfsetroundjoin%
\definecolor{currentfill}{rgb}{0.121569,0.466667,0.705882}%
\pgfsetfillcolor{currentfill}%
\pgfsetfillopacity{0.911294}%
\pgfsetlinewidth{1.003750pt}%
\definecolor{currentstroke}{rgb}{0.121569,0.466667,0.705882}%
\pgfsetstrokecolor{currentstroke}%
\pgfsetstrokeopacity{0.911294}%
\pgfsetdash{}{0pt}%
\pgfpathmoveto{\pgfqpoint{2.021490in}{0.980858in}}%
\pgfpathcurveto{\pgfqpoint{2.029726in}{0.980858in}}{\pgfqpoint{2.037626in}{0.984130in}}{\pgfqpoint{2.043450in}{0.989954in}}%
\pgfpathcurveto{\pgfqpoint{2.049274in}{0.995778in}}{\pgfqpoint{2.052546in}{1.003678in}}{\pgfqpoint{2.052546in}{1.011914in}}%
\pgfpathcurveto{\pgfqpoint{2.052546in}{1.020151in}}{\pgfqpoint{2.049274in}{1.028051in}}{\pgfqpoint{2.043450in}{1.033875in}}%
\pgfpathcurveto{\pgfqpoint{2.037626in}{1.039699in}}{\pgfqpoint{2.029726in}{1.042971in}}{\pgfqpoint{2.021490in}{1.042971in}}%
\pgfpathcurveto{\pgfqpoint{2.013253in}{1.042971in}}{\pgfqpoint{2.005353in}{1.039699in}}{\pgfqpoint{1.999529in}{1.033875in}}%
\pgfpathcurveto{\pgfqpoint{1.993705in}{1.028051in}}{\pgfqpoint{1.990433in}{1.020151in}}{\pgfqpoint{1.990433in}{1.011914in}}%
\pgfpathcurveto{\pgfqpoint{1.990433in}{1.003678in}}{\pgfqpoint{1.993705in}{0.995778in}}{\pgfqpoint{1.999529in}{0.989954in}}%
\pgfpathcurveto{\pgfqpoint{2.005353in}{0.984130in}}{\pgfqpoint{2.013253in}{0.980858in}}{\pgfqpoint{2.021490in}{0.980858in}}%
\pgfpathclose%
\pgfusepath{stroke,fill}%
\end{pgfscope}%
\begin{pgfscope}%
\pgfpathrectangle{\pgfqpoint{0.100000in}{0.212622in}}{\pgfqpoint{3.696000in}{3.696000in}}%
\pgfusepath{clip}%
\pgfsetbuttcap%
\pgfsetroundjoin%
\definecolor{currentfill}{rgb}{0.121569,0.466667,0.705882}%
\pgfsetfillcolor{currentfill}%
\pgfsetfillopacity{0.913365}%
\pgfsetlinewidth{1.003750pt}%
\definecolor{currentstroke}{rgb}{0.121569,0.466667,0.705882}%
\pgfsetstrokecolor{currentstroke}%
\pgfsetstrokeopacity{0.913365}%
\pgfsetdash{}{0pt}%
\pgfpathmoveto{\pgfqpoint{2.030798in}{0.978672in}}%
\pgfpathcurveto{\pgfqpoint{2.039034in}{0.978672in}}{\pgfqpoint{2.046934in}{0.981944in}}{\pgfqpoint{2.052758in}{0.987768in}}%
\pgfpathcurveto{\pgfqpoint{2.058582in}{0.993592in}}{\pgfqpoint{2.061854in}{1.001492in}}{\pgfqpoint{2.061854in}{1.009728in}}%
\pgfpathcurveto{\pgfqpoint{2.061854in}{1.017964in}}{\pgfqpoint{2.058582in}{1.025865in}}{\pgfqpoint{2.052758in}{1.031688in}}%
\pgfpathcurveto{\pgfqpoint{2.046934in}{1.037512in}}{\pgfqpoint{2.039034in}{1.040785in}}{\pgfqpoint{2.030798in}{1.040785in}}%
\pgfpathcurveto{\pgfqpoint{2.022562in}{1.040785in}}{\pgfqpoint{2.014661in}{1.037512in}}{\pgfqpoint{2.008838in}{1.031688in}}%
\pgfpathcurveto{\pgfqpoint{2.003014in}{1.025865in}}{\pgfqpoint{1.999741in}{1.017964in}}{\pgfqpoint{1.999741in}{1.009728in}}%
\pgfpathcurveto{\pgfqpoint{1.999741in}{1.001492in}}{\pgfqpoint{2.003014in}{0.993592in}}{\pgfqpoint{2.008838in}{0.987768in}}%
\pgfpathcurveto{\pgfqpoint{2.014661in}{0.981944in}}{\pgfqpoint{2.022562in}{0.978672in}}{\pgfqpoint{2.030798in}{0.978672in}}%
\pgfpathclose%
\pgfusepath{stroke,fill}%
\end{pgfscope}%
\begin{pgfscope}%
\pgfpathrectangle{\pgfqpoint{0.100000in}{0.212622in}}{\pgfqpoint{3.696000in}{3.696000in}}%
\pgfusepath{clip}%
\pgfsetbuttcap%
\pgfsetroundjoin%
\definecolor{currentfill}{rgb}{0.121569,0.466667,0.705882}%
\pgfsetfillcolor{currentfill}%
\pgfsetfillopacity{0.914288}%
\pgfsetlinewidth{1.003750pt}%
\definecolor{currentstroke}{rgb}{0.121569,0.466667,0.705882}%
\pgfsetstrokecolor{currentstroke}%
\pgfsetstrokeopacity{0.914288}%
\pgfsetdash{}{0pt}%
\pgfpathmoveto{\pgfqpoint{2.372298in}{1.061014in}}%
\pgfpathcurveto{\pgfqpoint{2.380534in}{1.061014in}}{\pgfqpoint{2.388435in}{1.064287in}}{\pgfqpoint{2.394258in}{1.070111in}}%
\pgfpathcurveto{\pgfqpoint{2.400082in}{1.075935in}}{\pgfqpoint{2.403355in}{1.083835in}}{\pgfqpoint{2.403355in}{1.092071in}}%
\pgfpathcurveto{\pgfqpoint{2.403355in}{1.100307in}}{\pgfqpoint{2.400082in}{1.108207in}}{\pgfqpoint{2.394258in}{1.114031in}}%
\pgfpathcurveto{\pgfqpoint{2.388435in}{1.119855in}}{\pgfqpoint{2.380534in}{1.123127in}}{\pgfqpoint{2.372298in}{1.123127in}}%
\pgfpathcurveto{\pgfqpoint{2.364062in}{1.123127in}}{\pgfqpoint{2.356162in}{1.119855in}}{\pgfqpoint{2.350338in}{1.114031in}}%
\pgfpathcurveto{\pgfqpoint{2.344514in}{1.108207in}}{\pgfqpoint{2.341242in}{1.100307in}}{\pgfqpoint{2.341242in}{1.092071in}}%
\pgfpathcurveto{\pgfqpoint{2.341242in}{1.083835in}}{\pgfqpoint{2.344514in}{1.075935in}}{\pgfqpoint{2.350338in}{1.070111in}}%
\pgfpathcurveto{\pgfqpoint{2.356162in}{1.064287in}}{\pgfqpoint{2.364062in}{1.061014in}}{\pgfqpoint{2.372298in}{1.061014in}}%
\pgfpathclose%
\pgfusepath{stroke,fill}%
\end{pgfscope}%
\begin{pgfscope}%
\pgfpathrectangle{\pgfqpoint{0.100000in}{0.212622in}}{\pgfqpoint{3.696000in}{3.696000in}}%
\pgfusepath{clip}%
\pgfsetbuttcap%
\pgfsetroundjoin%
\definecolor{currentfill}{rgb}{0.121569,0.466667,0.705882}%
\pgfsetfillcolor{currentfill}%
\pgfsetfillopacity{0.915323}%
\pgfsetlinewidth{1.003750pt}%
\definecolor{currentstroke}{rgb}{0.121569,0.466667,0.705882}%
\pgfsetstrokecolor{currentstroke}%
\pgfsetstrokeopacity{0.915323}%
\pgfsetdash{}{0pt}%
\pgfpathmoveto{\pgfqpoint{2.039131in}{0.976680in}}%
\pgfpathcurveto{\pgfqpoint{2.047367in}{0.976680in}}{\pgfqpoint{2.055267in}{0.979953in}}{\pgfqpoint{2.061091in}{0.985776in}}%
\pgfpathcurveto{\pgfqpoint{2.066915in}{0.991600in}}{\pgfqpoint{2.070187in}{0.999500in}}{\pgfqpoint{2.070187in}{1.007737in}}%
\pgfpathcurveto{\pgfqpoint{2.070187in}{1.015973in}}{\pgfqpoint{2.066915in}{1.023873in}}{\pgfqpoint{2.061091in}{1.029697in}}%
\pgfpathcurveto{\pgfqpoint{2.055267in}{1.035521in}}{\pgfqpoint{2.047367in}{1.038793in}}{\pgfqpoint{2.039131in}{1.038793in}}%
\pgfpathcurveto{\pgfqpoint{2.030894in}{1.038793in}}{\pgfqpoint{2.022994in}{1.035521in}}{\pgfqpoint{2.017170in}{1.029697in}}%
\pgfpathcurveto{\pgfqpoint{2.011346in}{1.023873in}}{\pgfqpoint{2.008074in}{1.015973in}}{\pgfqpoint{2.008074in}{1.007737in}}%
\pgfpathcurveto{\pgfqpoint{2.008074in}{0.999500in}}{\pgfqpoint{2.011346in}{0.991600in}}{\pgfqpoint{2.017170in}{0.985776in}}%
\pgfpathcurveto{\pgfqpoint{2.022994in}{0.979953in}}{\pgfqpoint{2.030894in}{0.976680in}}{\pgfqpoint{2.039131in}{0.976680in}}%
\pgfpathclose%
\pgfusepath{stroke,fill}%
\end{pgfscope}%
\begin{pgfscope}%
\pgfpathrectangle{\pgfqpoint{0.100000in}{0.212622in}}{\pgfqpoint{3.696000in}{3.696000in}}%
\pgfusepath{clip}%
\pgfsetbuttcap%
\pgfsetroundjoin%
\definecolor{currentfill}{rgb}{0.121569,0.466667,0.705882}%
\pgfsetfillcolor{currentfill}%
\pgfsetfillopacity{0.916864}%
\pgfsetlinewidth{1.003750pt}%
\definecolor{currentstroke}{rgb}{0.121569,0.466667,0.705882}%
\pgfsetstrokecolor{currentstroke}%
\pgfsetstrokeopacity{0.916864}%
\pgfsetdash{}{0pt}%
\pgfpathmoveto{\pgfqpoint{2.046649in}{0.974603in}}%
\pgfpathcurveto{\pgfqpoint{2.054885in}{0.974603in}}{\pgfqpoint{2.062785in}{0.977875in}}{\pgfqpoint{2.068609in}{0.983699in}}%
\pgfpathcurveto{\pgfqpoint{2.074433in}{0.989523in}}{\pgfqpoint{2.077705in}{0.997423in}}{\pgfqpoint{2.077705in}{1.005659in}}%
\pgfpathcurveto{\pgfqpoint{2.077705in}{1.013895in}}{\pgfqpoint{2.074433in}{1.021795in}}{\pgfqpoint{2.068609in}{1.027619in}}%
\pgfpathcurveto{\pgfqpoint{2.062785in}{1.033443in}}{\pgfqpoint{2.054885in}{1.036716in}}{\pgfqpoint{2.046649in}{1.036716in}}%
\pgfpathcurveto{\pgfqpoint{2.038413in}{1.036716in}}{\pgfqpoint{2.030513in}{1.033443in}}{\pgfqpoint{2.024689in}{1.027619in}}%
\pgfpathcurveto{\pgfqpoint{2.018865in}{1.021795in}}{\pgfqpoint{2.015592in}{1.013895in}}{\pgfqpoint{2.015592in}{1.005659in}}%
\pgfpathcurveto{\pgfqpoint{2.015592in}{0.997423in}}{\pgfqpoint{2.018865in}{0.989523in}}{\pgfqpoint{2.024689in}{0.983699in}}%
\pgfpathcurveto{\pgfqpoint{2.030513in}{0.977875in}}{\pgfqpoint{2.038413in}{0.974603in}}{\pgfqpoint{2.046649in}{0.974603in}}%
\pgfpathclose%
\pgfusepath{stroke,fill}%
\end{pgfscope}%
\begin{pgfscope}%
\pgfpathrectangle{\pgfqpoint{0.100000in}{0.212622in}}{\pgfqpoint{3.696000in}{3.696000in}}%
\pgfusepath{clip}%
\pgfsetbuttcap%
\pgfsetroundjoin%
\definecolor{currentfill}{rgb}{0.121569,0.466667,0.705882}%
\pgfsetfillcolor{currentfill}%
\pgfsetfillopacity{0.918367}%
\pgfsetlinewidth{1.003750pt}%
\definecolor{currentstroke}{rgb}{0.121569,0.466667,0.705882}%
\pgfsetstrokecolor{currentstroke}%
\pgfsetstrokeopacity{0.918367}%
\pgfsetdash{}{0pt}%
\pgfpathmoveto{\pgfqpoint{2.053094in}{0.972887in}}%
\pgfpathcurveto{\pgfqpoint{2.061330in}{0.972887in}}{\pgfqpoint{2.069230in}{0.976160in}}{\pgfqpoint{2.075054in}{0.981984in}}%
\pgfpathcurveto{\pgfqpoint{2.080878in}{0.987808in}}{\pgfqpoint{2.084150in}{0.995708in}}{\pgfqpoint{2.084150in}{1.003944in}}%
\pgfpathcurveto{\pgfqpoint{2.084150in}{1.012180in}}{\pgfqpoint{2.080878in}{1.020080in}}{\pgfqpoint{2.075054in}{1.025904in}}%
\pgfpathcurveto{\pgfqpoint{2.069230in}{1.031728in}}{\pgfqpoint{2.061330in}{1.035000in}}{\pgfqpoint{2.053094in}{1.035000in}}%
\pgfpathcurveto{\pgfqpoint{2.044858in}{1.035000in}}{\pgfqpoint{2.036957in}{1.031728in}}{\pgfqpoint{2.031134in}{1.025904in}}%
\pgfpathcurveto{\pgfqpoint{2.025310in}{1.020080in}}{\pgfqpoint{2.022037in}{1.012180in}}{\pgfqpoint{2.022037in}{1.003944in}}%
\pgfpathcurveto{\pgfqpoint{2.022037in}{0.995708in}}{\pgfqpoint{2.025310in}{0.987808in}}{\pgfqpoint{2.031134in}{0.981984in}}%
\pgfpathcurveto{\pgfqpoint{2.036957in}{0.976160in}}{\pgfqpoint{2.044858in}{0.972887in}}{\pgfqpoint{2.053094in}{0.972887in}}%
\pgfpathclose%
\pgfusepath{stroke,fill}%
\end{pgfscope}%
\begin{pgfscope}%
\pgfpathrectangle{\pgfqpoint{0.100000in}{0.212622in}}{\pgfqpoint{3.696000in}{3.696000in}}%
\pgfusepath{clip}%
\pgfsetbuttcap%
\pgfsetroundjoin%
\definecolor{currentfill}{rgb}{0.121569,0.466667,0.705882}%
\pgfsetfillcolor{currentfill}%
\pgfsetfillopacity{0.918542}%
\pgfsetlinewidth{1.003750pt}%
\definecolor{currentstroke}{rgb}{0.121569,0.466667,0.705882}%
\pgfsetstrokecolor{currentstroke}%
\pgfsetstrokeopacity{0.918542}%
\pgfsetdash{}{0pt}%
\pgfpathmoveto{\pgfqpoint{2.376667in}{1.045838in}}%
\pgfpathcurveto{\pgfqpoint{2.384903in}{1.045838in}}{\pgfqpoint{2.392803in}{1.049110in}}{\pgfqpoint{2.398627in}{1.054934in}}%
\pgfpathcurveto{\pgfqpoint{2.404451in}{1.060758in}}{\pgfqpoint{2.407724in}{1.068658in}}{\pgfqpoint{2.407724in}{1.076894in}}%
\pgfpathcurveto{\pgfqpoint{2.407724in}{1.085130in}}{\pgfqpoint{2.404451in}{1.093030in}}{\pgfqpoint{2.398627in}{1.098854in}}%
\pgfpathcurveto{\pgfqpoint{2.392803in}{1.104678in}}{\pgfqpoint{2.384903in}{1.107951in}}{\pgfqpoint{2.376667in}{1.107951in}}%
\pgfpathcurveto{\pgfqpoint{2.368431in}{1.107951in}}{\pgfqpoint{2.360531in}{1.104678in}}{\pgfqpoint{2.354707in}{1.098854in}}%
\pgfpathcurveto{\pgfqpoint{2.348883in}{1.093030in}}{\pgfqpoint{2.345611in}{1.085130in}}{\pgfqpoint{2.345611in}{1.076894in}}%
\pgfpathcurveto{\pgfqpoint{2.345611in}{1.068658in}}{\pgfqpoint{2.348883in}{1.060758in}}{\pgfqpoint{2.354707in}{1.054934in}}%
\pgfpathcurveto{\pgfqpoint{2.360531in}{1.049110in}}{\pgfqpoint{2.368431in}{1.045838in}}{\pgfqpoint{2.376667in}{1.045838in}}%
\pgfpathclose%
\pgfusepath{stroke,fill}%
\end{pgfscope}%
\begin{pgfscope}%
\pgfpathrectangle{\pgfqpoint{0.100000in}{0.212622in}}{\pgfqpoint{3.696000in}{3.696000in}}%
\pgfusepath{clip}%
\pgfsetbuttcap%
\pgfsetroundjoin%
\definecolor{currentfill}{rgb}{0.121569,0.466667,0.705882}%
\pgfsetfillcolor{currentfill}%
\pgfsetfillopacity{0.920916}%
\pgfsetlinewidth{1.003750pt}%
\definecolor{currentstroke}{rgb}{0.121569,0.466667,0.705882}%
\pgfsetstrokecolor{currentstroke}%
\pgfsetstrokeopacity{0.920916}%
\pgfsetdash{}{0pt}%
\pgfpathmoveto{\pgfqpoint{2.064871in}{0.969310in}}%
\pgfpathcurveto{\pgfqpoint{2.073107in}{0.969310in}}{\pgfqpoint{2.081007in}{0.972582in}}{\pgfqpoint{2.086831in}{0.978406in}}%
\pgfpathcurveto{\pgfqpoint{2.092655in}{0.984230in}}{\pgfqpoint{2.095927in}{0.992130in}}{\pgfqpoint{2.095927in}{1.000366in}}%
\pgfpathcurveto{\pgfqpoint{2.095927in}{1.008603in}}{\pgfqpoint{2.092655in}{1.016503in}}{\pgfqpoint{2.086831in}{1.022327in}}%
\pgfpathcurveto{\pgfqpoint{2.081007in}{1.028151in}}{\pgfqpoint{2.073107in}{1.031423in}}{\pgfqpoint{2.064871in}{1.031423in}}%
\pgfpathcurveto{\pgfqpoint{2.056635in}{1.031423in}}{\pgfqpoint{2.048735in}{1.028151in}}{\pgfqpoint{2.042911in}{1.022327in}}%
\pgfpathcurveto{\pgfqpoint{2.037087in}{1.016503in}}{\pgfqpoint{2.033814in}{1.008603in}}{\pgfqpoint{2.033814in}{1.000366in}}%
\pgfpathcurveto{\pgfqpoint{2.033814in}{0.992130in}}{\pgfqpoint{2.037087in}{0.984230in}}{\pgfqpoint{2.042911in}{0.978406in}}%
\pgfpathcurveto{\pgfqpoint{2.048735in}{0.972582in}}{\pgfqpoint{2.056635in}{0.969310in}}{\pgfqpoint{2.064871in}{0.969310in}}%
\pgfpathclose%
\pgfusepath{stroke,fill}%
\end{pgfscope}%
\begin{pgfscope}%
\pgfpathrectangle{\pgfqpoint{0.100000in}{0.212622in}}{\pgfqpoint{3.696000in}{3.696000in}}%
\pgfusepath{clip}%
\pgfsetbuttcap%
\pgfsetroundjoin%
\definecolor{currentfill}{rgb}{0.121569,0.466667,0.705882}%
\pgfsetfillcolor{currentfill}%
\pgfsetfillopacity{0.923123}%
\pgfsetlinewidth{1.003750pt}%
\definecolor{currentstroke}{rgb}{0.121569,0.466667,0.705882}%
\pgfsetstrokecolor{currentstroke}%
\pgfsetstrokeopacity{0.923123}%
\pgfsetdash{}{0pt}%
\pgfpathmoveto{\pgfqpoint{2.075447in}{0.965535in}}%
\pgfpathcurveto{\pgfqpoint{2.083684in}{0.965535in}}{\pgfqpoint{2.091584in}{0.968808in}}{\pgfqpoint{2.097408in}{0.974632in}}%
\pgfpathcurveto{\pgfqpoint{2.103232in}{0.980456in}}{\pgfqpoint{2.106504in}{0.988356in}}{\pgfqpoint{2.106504in}{0.996592in}}%
\pgfpathcurveto{\pgfqpoint{2.106504in}{1.004828in}}{\pgfqpoint{2.103232in}{1.012728in}}{\pgfqpoint{2.097408in}{1.018552in}}%
\pgfpathcurveto{\pgfqpoint{2.091584in}{1.024376in}}{\pgfqpoint{2.083684in}{1.027648in}}{\pgfqpoint{2.075447in}{1.027648in}}%
\pgfpathcurveto{\pgfqpoint{2.067211in}{1.027648in}}{\pgfqpoint{2.059311in}{1.024376in}}{\pgfqpoint{2.053487in}{1.018552in}}%
\pgfpathcurveto{\pgfqpoint{2.047663in}{1.012728in}}{\pgfqpoint{2.044391in}{1.004828in}}{\pgfqpoint{2.044391in}{0.996592in}}%
\pgfpathcurveto{\pgfqpoint{2.044391in}{0.988356in}}{\pgfqpoint{2.047663in}{0.980456in}}{\pgfqpoint{2.053487in}{0.974632in}}%
\pgfpathcurveto{\pgfqpoint{2.059311in}{0.968808in}}{\pgfqpoint{2.067211in}{0.965535in}}{\pgfqpoint{2.075447in}{0.965535in}}%
\pgfpathclose%
\pgfusepath{stroke,fill}%
\end{pgfscope}%
\begin{pgfscope}%
\pgfpathrectangle{\pgfqpoint{0.100000in}{0.212622in}}{\pgfqpoint{3.696000in}{3.696000in}}%
\pgfusepath{clip}%
\pgfsetbuttcap%
\pgfsetroundjoin%
\definecolor{currentfill}{rgb}{0.121569,0.466667,0.705882}%
\pgfsetfillcolor{currentfill}%
\pgfsetfillopacity{0.923139}%
\pgfsetlinewidth{1.003750pt}%
\definecolor{currentstroke}{rgb}{0.121569,0.466667,0.705882}%
\pgfsetstrokecolor{currentstroke}%
\pgfsetstrokeopacity{0.923139}%
\pgfsetdash{}{0pt}%
\pgfpathmoveto{\pgfqpoint{2.381097in}{1.030114in}}%
\pgfpathcurveto{\pgfqpoint{2.389334in}{1.030114in}}{\pgfqpoint{2.397234in}{1.033386in}}{\pgfqpoint{2.403058in}{1.039210in}}%
\pgfpathcurveto{\pgfqpoint{2.408882in}{1.045034in}}{\pgfqpoint{2.412154in}{1.052934in}}{\pgfqpoint{2.412154in}{1.061170in}}%
\pgfpathcurveto{\pgfqpoint{2.412154in}{1.069407in}}{\pgfqpoint{2.408882in}{1.077307in}}{\pgfqpoint{2.403058in}{1.083131in}}%
\pgfpathcurveto{\pgfqpoint{2.397234in}{1.088955in}}{\pgfqpoint{2.389334in}{1.092227in}}{\pgfqpoint{2.381097in}{1.092227in}}%
\pgfpathcurveto{\pgfqpoint{2.372861in}{1.092227in}}{\pgfqpoint{2.364961in}{1.088955in}}{\pgfqpoint{2.359137in}{1.083131in}}%
\pgfpathcurveto{\pgfqpoint{2.353313in}{1.077307in}}{\pgfqpoint{2.350041in}{1.069407in}}{\pgfqpoint{2.350041in}{1.061170in}}%
\pgfpathcurveto{\pgfqpoint{2.350041in}{1.052934in}}{\pgfqpoint{2.353313in}{1.045034in}}{\pgfqpoint{2.359137in}{1.039210in}}%
\pgfpathcurveto{\pgfqpoint{2.364961in}{1.033386in}}{\pgfqpoint{2.372861in}{1.030114in}}{\pgfqpoint{2.381097in}{1.030114in}}%
\pgfpathclose%
\pgfusepath{stroke,fill}%
\end{pgfscope}%
\begin{pgfscope}%
\pgfpathrectangle{\pgfqpoint{0.100000in}{0.212622in}}{\pgfqpoint{3.696000in}{3.696000in}}%
\pgfusepath{clip}%
\pgfsetbuttcap%
\pgfsetroundjoin%
\definecolor{currentfill}{rgb}{0.121569,0.466667,0.705882}%
\pgfsetfillcolor{currentfill}%
\pgfsetfillopacity{0.925426}%
\pgfsetlinewidth{1.003750pt}%
\definecolor{currentstroke}{rgb}{0.121569,0.466667,0.705882}%
\pgfsetstrokecolor{currentstroke}%
\pgfsetstrokeopacity{0.925426}%
\pgfsetdash{}{0pt}%
\pgfpathmoveto{\pgfqpoint{2.084811in}{0.962275in}}%
\pgfpathcurveto{\pgfqpoint{2.093047in}{0.962275in}}{\pgfqpoint{2.100947in}{0.965547in}}{\pgfqpoint{2.106771in}{0.971371in}}%
\pgfpathcurveto{\pgfqpoint{2.112595in}{0.977195in}}{\pgfqpoint{2.115867in}{0.985095in}}{\pgfqpoint{2.115867in}{0.993331in}}%
\pgfpathcurveto{\pgfqpoint{2.115867in}{1.001567in}}{\pgfqpoint{2.112595in}{1.009468in}}{\pgfqpoint{2.106771in}{1.015291in}}%
\pgfpathcurveto{\pgfqpoint{2.100947in}{1.021115in}}{\pgfqpoint{2.093047in}{1.024388in}}{\pgfqpoint{2.084811in}{1.024388in}}%
\pgfpathcurveto{\pgfqpoint{2.076575in}{1.024388in}}{\pgfqpoint{2.068674in}{1.021115in}}{\pgfqpoint{2.062851in}{1.015291in}}%
\pgfpathcurveto{\pgfqpoint{2.057027in}{1.009468in}}{\pgfqpoint{2.053754in}{1.001567in}}{\pgfqpoint{2.053754in}{0.993331in}}%
\pgfpathcurveto{\pgfqpoint{2.053754in}{0.985095in}}{\pgfqpoint{2.057027in}{0.977195in}}{\pgfqpoint{2.062851in}{0.971371in}}%
\pgfpathcurveto{\pgfqpoint{2.068674in}{0.965547in}}{\pgfqpoint{2.076575in}{0.962275in}}{\pgfqpoint{2.084811in}{0.962275in}}%
\pgfpathclose%
\pgfusepath{stroke,fill}%
\end{pgfscope}%
\begin{pgfscope}%
\pgfpathrectangle{\pgfqpoint{0.100000in}{0.212622in}}{\pgfqpoint{3.696000in}{3.696000in}}%
\pgfusepath{clip}%
\pgfsetbuttcap%
\pgfsetroundjoin%
\definecolor{currentfill}{rgb}{0.121569,0.466667,0.705882}%
\pgfsetfillcolor{currentfill}%
\pgfsetfillopacity{0.927478}%
\pgfsetlinewidth{1.003750pt}%
\definecolor{currentstroke}{rgb}{0.121569,0.466667,0.705882}%
\pgfsetstrokecolor{currentstroke}%
\pgfsetstrokeopacity{0.927478}%
\pgfsetdash{}{0pt}%
\pgfpathmoveto{\pgfqpoint{2.093997in}{0.958628in}}%
\pgfpathcurveto{\pgfqpoint{2.102233in}{0.958628in}}{\pgfqpoint{2.110133in}{0.961900in}}{\pgfqpoint{2.115957in}{0.967724in}}%
\pgfpathcurveto{\pgfqpoint{2.121781in}{0.973548in}}{\pgfqpoint{2.125053in}{0.981448in}}{\pgfqpoint{2.125053in}{0.989684in}}%
\pgfpathcurveto{\pgfqpoint{2.125053in}{0.997920in}}{\pgfqpoint{2.121781in}{1.005820in}}{\pgfqpoint{2.115957in}{1.011644in}}%
\pgfpathcurveto{\pgfqpoint{2.110133in}{1.017468in}}{\pgfqpoint{2.102233in}{1.020741in}}{\pgfqpoint{2.093997in}{1.020741in}}%
\pgfpathcurveto{\pgfqpoint{2.085760in}{1.020741in}}{\pgfqpoint{2.077860in}{1.017468in}}{\pgfqpoint{2.072036in}{1.011644in}}%
\pgfpathcurveto{\pgfqpoint{2.066213in}{1.005820in}}{\pgfqpoint{2.062940in}{0.997920in}}{\pgfqpoint{2.062940in}{0.989684in}}%
\pgfpathcurveto{\pgfqpoint{2.062940in}{0.981448in}}{\pgfqpoint{2.066213in}{0.973548in}}{\pgfqpoint{2.072036in}{0.967724in}}%
\pgfpathcurveto{\pgfqpoint{2.077860in}{0.961900in}}{\pgfqpoint{2.085760in}{0.958628in}}{\pgfqpoint{2.093997in}{0.958628in}}%
\pgfpathclose%
\pgfusepath{stroke,fill}%
\end{pgfscope}%
\begin{pgfscope}%
\pgfpathrectangle{\pgfqpoint{0.100000in}{0.212622in}}{\pgfqpoint{3.696000in}{3.696000in}}%
\pgfusepath{clip}%
\pgfsetbuttcap%
\pgfsetroundjoin%
\definecolor{currentfill}{rgb}{0.121569,0.466667,0.705882}%
\pgfsetfillcolor{currentfill}%
\pgfsetfillopacity{0.927937}%
\pgfsetlinewidth{1.003750pt}%
\definecolor{currentstroke}{rgb}{0.121569,0.466667,0.705882}%
\pgfsetstrokecolor{currentstroke}%
\pgfsetstrokeopacity{0.927937}%
\pgfsetdash{}{0pt}%
\pgfpathmoveto{\pgfqpoint{2.385870in}{1.013422in}}%
\pgfpathcurveto{\pgfqpoint{2.394106in}{1.013422in}}{\pgfqpoint{2.402006in}{1.016694in}}{\pgfqpoint{2.407830in}{1.022518in}}%
\pgfpathcurveto{\pgfqpoint{2.413654in}{1.028342in}}{\pgfqpoint{2.416926in}{1.036242in}}{\pgfqpoint{2.416926in}{1.044479in}}%
\pgfpathcurveto{\pgfqpoint{2.416926in}{1.052715in}}{\pgfqpoint{2.413654in}{1.060615in}}{\pgfqpoint{2.407830in}{1.066439in}}%
\pgfpathcurveto{\pgfqpoint{2.402006in}{1.072263in}}{\pgfqpoint{2.394106in}{1.075535in}}{\pgfqpoint{2.385870in}{1.075535in}}%
\pgfpathcurveto{\pgfqpoint{2.377634in}{1.075535in}}{\pgfqpoint{2.369734in}{1.072263in}}{\pgfqpoint{2.363910in}{1.066439in}}%
\pgfpathcurveto{\pgfqpoint{2.358086in}{1.060615in}}{\pgfqpoint{2.354813in}{1.052715in}}{\pgfqpoint{2.354813in}{1.044479in}}%
\pgfpathcurveto{\pgfqpoint{2.354813in}{1.036242in}}{\pgfqpoint{2.358086in}{1.028342in}}{\pgfqpoint{2.363910in}{1.022518in}}%
\pgfpathcurveto{\pgfqpoint{2.369734in}{1.016694in}}{\pgfqpoint{2.377634in}{1.013422in}}{\pgfqpoint{2.385870in}{1.013422in}}%
\pgfpathclose%
\pgfusepath{stroke,fill}%
\end{pgfscope}%
\begin{pgfscope}%
\pgfpathrectangle{\pgfqpoint{0.100000in}{0.212622in}}{\pgfqpoint{3.696000in}{3.696000in}}%
\pgfusepath{clip}%
\pgfsetbuttcap%
\pgfsetroundjoin%
\definecolor{currentfill}{rgb}{0.121569,0.466667,0.705882}%
\pgfsetfillcolor{currentfill}%
\pgfsetfillopacity{0.929472}%
\pgfsetlinewidth{1.003750pt}%
\definecolor{currentstroke}{rgb}{0.121569,0.466667,0.705882}%
\pgfsetstrokecolor{currentstroke}%
\pgfsetstrokeopacity{0.929472}%
\pgfsetdash{}{0pt}%
\pgfpathmoveto{\pgfqpoint{2.102360in}{0.954425in}}%
\pgfpathcurveto{\pgfqpoint{2.110596in}{0.954425in}}{\pgfqpoint{2.118496in}{0.957698in}}{\pgfqpoint{2.124320in}{0.963521in}}%
\pgfpathcurveto{\pgfqpoint{2.130144in}{0.969345in}}{\pgfqpoint{2.133417in}{0.977245in}}{\pgfqpoint{2.133417in}{0.985482in}}%
\pgfpathcurveto{\pgfqpoint{2.133417in}{0.993718in}}{\pgfqpoint{2.130144in}{1.001618in}}{\pgfqpoint{2.124320in}{1.007442in}}%
\pgfpathcurveto{\pgfqpoint{2.118496in}{1.013266in}}{\pgfqpoint{2.110596in}{1.016538in}}{\pgfqpoint{2.102360in}{1.016538in}}%
\pgfpathcurveto{\pgfqpoint{2.094124in}{1.016538in}}{\pgfqpoint{2.086224in}{1.013266in}}{\pgfqpoint{2.080400in}{1.007442in}}%
\pgfpathcurveto{\pgfqpoint{2.074576in}{1.001618in}}{\pgfqpoint{2.071304in}{0.993718in}}{\pgfqpoint{2.071304in}{0.985482in}}%
\pgfpathcurveto{\pgfqpoint{2.071304in}{0.977245in}}{\pgfqpoint{2.074576in}{0.969345in}}{\pgfqpoint{2.080400in}{0.963521in}}%
\pgfpathcurveto{\pgfqpoint{2.086224in}{0.957698in}}{\pgfqpoint{2.094124in}{0.954425in}}{\pgfqpoint{2.102360in}{0.954425in}}%
\pgfpathclose%
\pgfusepath{stroke,fill}%
\end{pgfscope}%
\begin{pgfscope}%
\pgfpathrectangle{\pgfqpoint{0.100000in}{0.212622in}}{\pgfqpoint{3.696000in}{3.696000in}}%
\pgfusepath{clip}%
\pgfsetbuttcap%
\pgfsetroundjoin%
\definecolor{currentfill}{rgb}{0.121569,0.466667,0.705882}%
\pgfsetfillcolor{currentfill}%
\pgfsetfillopacity{0.931298}%
\pgfsetlinewidth{1.003750pt}%
\definecolor{currentstroke}{rgb}{0.121569,0.466667,0.705882}%
\pgfsetstrokecolor{currentstroke}%
\pgfsetstrokeopacity{0.931298}%
\pgfsetdash{}{0pt}%
\pgfpathmoveto{\pgfqpoint{2.109525in}{0.950839in}}%
\pgfpathcurveto{\pgfqpoint{2.117761in}{0.950839in}}{\pgfqpoint{2.125661in}{0.954111in}}{\pgfqpoint{2.131485in}{0.959935in}}%
\pgfpathcurveto{\pgfqpoint{2.137309in}{0.965759in}}{\pgfqpoint{2.140581in}{0.973659in}}{\pgfqpoint{2.140581in}{0.981895in}}%
\pgfpathcurveto{\pgfqpoint{2.140581in}{0.990131in}}{\pgfqpoint{2.137309in}{0.998032in}}{\pgfqpoint{2.131485in}{1.003855in}}%
\pgfpathcurveto{\pgfqpoint{2.125661in}{1.009679in}}{\pgfqpoint{2.117761in}{1.012952in}}{\pgfqpoint{2.109525in}{1.012952in}}%
\pgfpathcurveto{\pgfqpoint{2.101288in}{1.012952in}}{\pgfqpoint{2.093388in}{1.009679in}}{\pgfqpoint{2.087564in}{1.003855in}}%
\pgfpathcurveto{\pgfqpoint{2.081740in}{0.998032in}}{\pgfqpoint{2.078468in}{0.990131in}}{\pgfqpoint{2.078468in}{0.981895in}}%
\pgfpathcurveto{\pgfqpoint{2.078468in}{0.973659in}}{\pgfqpoint{2.081740in}{0.965759in}}{\pgfqpoint{2.087564in}{0.959935in}}%
\pgfpathcurveto{\pgfqpoint{2.093388in}{0.954111in}}{\pgfqpoint{2.101288in}{0.950839in}}{\pgfqpoint{2.109525in}{0.950839in}}%
\pgfpathclose%
\pgfusepath{stroke,fill}%
\end{pgfscope}%
\begin{pgfscope}%
\pgfpathrectangle{\pgfqpoint{0.100000in}{0.212622in}}{\pgfqpoint{3.696000in}{3.696000in}}%
\pgfusepath{clip}%
\pgfsetbuttcap%
\pgfsetroundjoin%
\definecolor{currentfill}{rgb}{0.121569,0.466667,0.705882}%
\pgfsetfillcolor{currentfill}%
\pgfsetfillopacity{0.933018}%
\pgfsetlinewidth{1.003750pt}%
\definecolor{currentstroke}{rgb}{0.121569,0.466667,0.705882}%
\pgfsetstrokecolor{currentstroke}%
\pgfsetstrokeopacity{0.933018}%
\pgfsetdash{}{0pt}%
\pgfpathmoveto{\pgfqpoint{2.390638in}{0.995816in}}%
\pgfpathcurveto{\pgfqpoint{2.398875in}{0.995816in}}{\pgfqpoint{2.406775in}{0.999088in}}{\pgfqpoint{2.412599in}{1.004912in}}%
\pgfpathcurveto{\pgfqpoint{2.418423in}{1.010736in}}{\pgfqpoint{2.421695in}{1.018636in}}{\pgfqpoint{2.421695in}{1.026872in}}%
\pgfpathcurveto{\pgfqpoint{2.421695in}{1.035108in}}{\pgfqpoint{2.418423in}{1.043008in}}{\pgfqpoint{2.412599in}{1.048832in}}%
\pgfpathcurveto{\pgfqpoint{2.406775in}{1.054656in}}{\pgfqpoint{2.398875in}{1.057929in}}{\pgfqpoint{2.390638in}{1.057929in}}%
\pgfpathcurveto{\pgfqpoint{2.382402in}{1.057929in}}{\pgfqpoint{2.374502in}{1.054656in}}{\pgfqpoint{2.368678in}{1.048832in}}%
\pgfpathcurveto{\pgfqpoint{2.362854in}{1.043008in}}{\pgfqpoint{2.359582in}{1.035108in}}{\pgfqpoint{2.359582in}{1.026872in}}%
\pgfpathcurveto{\pgfqpoint{2.359582in}{1.018636in}}{\pgfqpoint{2.362854in}{1.010736in}}{\pgfqpoint{2.368678in}{1.004912in}}%
\pgfpathcurveto{\pgfqpoint{2.374502in}{0.999088in}}{\pgfqpoint{2.382402in}{0.995816in}}{\pgfqpoint{2.390638in}{0.995816in}}%
\pgfpathclose%
\pgfusepath{stroke,fill}%
\end{pgfscope}%
\begin{pgfscope}%
\pgfpathrectangle{\pgfqpoint{0.100000in}{0.212622in}}{\pgfqpoint{3.696000in}{3.696000in}}%
\pgfusepath{clip}%
\pgfsetbuttcap%
\pgfsetroundjoin%
\definecolor{currentfill}{rgb}{0.121569,0.466667,0.705882}%
\pgfsetfillcolor{currentfill}%
\pgfsetfillopacity{0.934277}%
\pgfsetlinewidth{1.003750pt}%
\definecolor{currentstroke}{rgb}{0.121569,0.466667,0.705882}%
\pgfsetstrokecolor{currentstroke}%
\pgfsetstrokeopacity{0.934277}%
\pgfsetdash{}{0pt}%
\pgfpathmoveto{\pgfqpoint{2.122394in}{0.942617in}}%
\pgfpathcurveto{\pgfqpoint{2.130630in}{0.942617in}}{\pgfqpoint{2.138530in}{0.945890in}}{\pgfqpoint{2.144354in}{0.951714in}}%
\pgfpathcurveto{\pgfqpoint{2.150178in}{0.957538in}}{\pgfqpoint{2.153450in}{0.965438in}}{\pgfqpoint{2.153450in}{0.973674in}}%
\pgfpathcurveto{\pgfqpoint{2.153450in}{0.981910in}}{\pgfqpoint{2.150178in}{0.989810in}}{\pgfqpoint{2.144354in}{0.995634in}}%
\pgfpathcurveto{\pgfqpoint{2.138530in}{1.001458in}}{\pgfqpoint{2.130630in}{1.004730in}}{\pgfqpoint{2.122394in}{1.004730in}}%
\pgfpathcurveto{\pgfqpoint{2.114157in}{1.004730in}}{\pgfqpoint{2.106257in}{1.001458in}}{\pgfqpoint{2.100433in}{0.995634in}}%
\pgfpathcurveto{\pgfqpoint{2.094609in}{0.989810in}}{\pgfqpoint{2.091337in}{0.981910in}}{\pgfqpoint{2.091337in}{0.973674in}}%
\pgfpathcurveto{\pgfqpoint{2.091337in}{0.965438in}}{\pgfqpoint{2.094609in}{0.957538in}}{\pgfqpoint{2.100433in}{0.951714in}}%
\pgfpathcurveto{\pgfqpoint{2.106257in}{0.945890in}}{\pgfqpoint{2.114157in}{0.942617in}}{\pgfqpoint{2.122394in}{0.942617in}}%
\pgfpathclose%
\pgfusepath{stroke,fill}%
\end{pgfscope}%
\begin{pgfscope}%
\pgfpathrectangle{\pgfqpoint{0.100000in}{0.212622in}}{\pgfqpoint{3.696000in}{3.696000in}}%
\pgfusepath{clip}%
\pgfsetbuttcap%
\pgfsetroundjoin%
\definecolor{currentfill}{rgb}{0.121569,0.466667,0.705882}%
\pgfsetfillcolor{currentfill}%
\pgfsetfillopacity{0.937251}%
\pgfsetlinewidth{1.003750pt}%
\definecolor{currentstroke}{rgb}{0.121569,0.466667,0.705882}%
\pgfsetstrokecolor{currentstroke}%
\pgfsetstrokeopacity{0.937251}%
\pgfsetdash{}{0pt}%
\pgfpathmoveto{\pgfqpoint{2.134790in}{0.934476in}}%
\pgfpathcurveto{\pgfqpoint{2.143026in}{0.934476in}}{\pgfqpoint{2.150926in}{0.937749in}}{\pgfqpoint{2.156750in}{0.943572in}}%
\pgfpathcurveto{\pgfqpoint{2.162574in}{0.949396in}}{\pgfqpoint{2.165847in}{0.957296in}}{\pgfqpoint{2.165847in}{0.965533in}}%
\pgfpathcurveto{\pgfqpoint{2.165847in}{0.973769in}}{\pgfqpoint{2.162574in}{0.981669in}}{\pgfqpoint{2.156750in}{0.987493in}}%
\pgfpathcurveto{\pgfqpoint{2.150926in}{0.993317in}}{\pgfqpoint{2.143026in}{0.996589in}}{\pgfqpoint{2.134790in}{0.996589in}}%
\pgfpathcurveto{\pgfqpoint{2.126554in}{0.996589in}}{\pgfqpoint{2.118654in}{0.993317in}}{\pgfqpoint{2.112830in}{0.987493in}}%
\pgfpathcurveto{\pgfqpoint{2.107006in}{0.981669in}}{\pgfqpoint{2.103734in}{0.973769in}}{\pgfqpoint{2.103734in}{0.965533in}}%
\pgfpathcurveto{\pgfqpoint{2.103734in}{0.957296in}}{\pgfqpoint{2.107006in}{0.949396in}}{\pgfqpoint{2.112830in}{0.943572in}}%
\pgfpathcurveto{\pgfqpoint{2.118654in}{0.937749in}}{\pgfqpoint{2.126554in}{0.934476in}}{\pgfqpoint{2.134790in}{0.934476in}}%
\pgfpathclose%
\pgfusepath{stroke,fill}%
\end{pgfscope}%
\begin{pgfscope}%
\pgfpathrectangle{\pgfqpoint{0.100000in}{0.212622in}}{\pgfqpoint{3.696000in}{3.696000in}}%
\pgfusepath{clip}%
\pgfsetbuttcap%
\pgfsetroundjoin%
\definecolor{currentfill}{rgb}{0.121569,0.466667,0.705882}%
\pgfsetfillcolor{currentfill}%
\pgfsetfillopacity{0.938192}%
\pgfsetlinewidth{1.003750pt}%
\definecolor{currentstroke}{rgb}{0.121569,0.466667,0.705882}%
\pgfsetstrokecolor{currentstroke}%
\pgfsetstrokeopacity{0.938192}%
\pgfsetdash{}{0pt}%
\pgfpathmoveto{\pgfqpoint{2.395506in}{0.977147in}}%
\pgfpathcurveto{\pgfqpoint{2.403742in}{0.977147in}}{\pgfqpoint{2.411642in}{0.980419in}}{\pgfqpoint{2.417466in}{0.986243in}}%
\pgfpathcurveto{\pgfqpoint{2.423290in}{0.992067in}}{\pgfqpoint{2.426562in}{0.999967in}}{\pgfqpoint{2.426562in}{1.008203in}}%
\pgfpathcurveto{\pgfqpoint{2.426562in}{1.016440in}}{\pgfqpoint{2.423290in}{1.024340in}}{\pgfqpoint{2.417466in}{1.030164in}}%
\pgfpathcurveto{\pgfqpoint{2.411642in}{1.035987in}}{\pgfqpoint{2.403742in}{1.039260in}}{\pgfqpoint{2.395506in}{1.039260in}}%
\pgfpathcurveto{\pgfqpoint{2.387269in}{1.039260in}}{\pgfqpoint{2.379369in}{1.035987in}}{\pgfqpoint{2.373545in}{1.030164in}}%
\pgfpathcurveto{\pgfqpoint{2.367722in}{1.024340in}}{\pgfqpoint{2.364449in}{1.016440in}}{\pgfqpoint{2.364449in}{1.008203in}}%
\pgfpathcurveto{\pgfqpoint{2.364449in}{0.999967in}}{\pgfqpoint{2.367722in}{0.992067in}}{\pgfqpoint{2.373545in}{0.986243in}}%
\pgfpathcurveto{\pgfqpoint{2.379369in}{0.980419in}}{\pgfqpoint{2.387269in}{0.977147in}}{\pgfqpoint{2.395506in}{0.977147in}}%
\pgfpathclose%
\pgfusepath{stroke,fill}%
\end{pgfscope}%
\begin{pgfscope}%
\pgfpathrectangle{\pgfqpoint{0.100000in}{0.212622in}}{\pgfqpoint{3.696000in}{3.696000in}}%
\pgfusepath{clip}%
\pgfsetbuttcap%
\pgfsetroundjoin%
\definecolor{currentfill}{rgb}{0.121569,0.466667,0.705882}%
\pgfsetfillcolor{currentfill}%
\pgfsetfillopacity{0.939995}%
\pgfsetlinewidth{1.003750pt}%
\definecolor{currentstroke}{rgb}{0.121569,0.466667,0.705882}%
\pgfsetstrokecolor{currentstroke}%
\pgfsetstrokeopacity{0.939995}%
\pgfsetdash{}{0pt}%
\pgfpathmoveto{\pgfqpoint{2.146830in}{0.925935in}}%
\pgfpathcurveto{\pgfqpoint{2.155067in}{0.925935in}}{\pgfqpoint{2.162967in}{0.929208in}}{\pgfqpoint{2.168791in}{0.935032in}}%
\pgfpathcurveto{\pgfqpoint{2.174614in}{0.940856in}}{\pgfqpoint{2.177887in}{0.948756in}}{\pgfqpoint{2.177887in}{0.956992in}}%
\pgfpathcurveto{\pgfqpoint{2.177887in}{0.965228in}}{\pgfqpoint{2.174614in}{0.973128in}}{\pgfqpoint{2.168791in}{0.978952in}}%
\pgfpathcurveto{\pgfqpoint{2.162967in}{0.984776in}}{\pgfqpoint{2.155067in}{0.988048in}}{\pgfqpoint{2.146830in}{0.988048in}}%
\pgfpathcurveto{\pgfqpoint{2.138594in}{0.988048in}}{\pgfqpoint{2.130694in}{0.984776in}}{\pgfqpoint{2.124870in}{0.978952in}}%
\pgfpathcurveto{\pgfqpoint{2.119046in}{0.973128in}}{\pgfqpoint{2.115774in}{0.965228in}}{\pgfqpoint{2.115774in}{0.956992in}}%
\pgfpathcurveto{\pgfqpoint{2.115774in}{0.948756in}}{\pgfqpoint{2.119046in}{0.940856in}}{\pgfqpoint{2.124870in}{0.935032in}}%
\pgfpathcurveto{\pgfqpoint{2.130694in}{0.929208in}}{\pgfqpoint{2.138594in}{0.925935in}}{\pgfqpoint{2.146830in}{0.925935in}}%
\pgfpathclose%
\pgfusepath{stroke,fill}%
\end{pgfscope}%
\begin{pgfscope}%
\pgfpathrectangle{\pgfqpoint{0.100000in}{0.212622in}}{\pgfqpoint{3.696000in}{3.696000in}}%
\pgfusepath{clip}%
\pgfsetbuttcap%
\pgfsetroundjoin%
\definecolor{currentfill}{rgb}{0.121569,0.466667,0.705882}%
\pgfsetfillcolor{currentfill}%
\pgfsetfillopacity{0.942742}%
\pgfsetlinewidth{1.003750pt}%
\definecolor{currentstroke}{rgb}{0.121569,0.466667,0.705882}%
\pgfsetstrokecolor{currentstroke}%
\pgfsetstrokeopacity{0.942742}%
\pgfsetdash{}{0pt}%
\pgfpathmoveto{\pgfqpoint{2.158385in}{0.917603in}}%
\pgfpathcurveto{\pgfqpoint{2.166621in}{0.917603in}}{\pgfqpoint{2.174521in}{0.920876in}}{\pgfqpoint{2.180345in}{0.926700in}}%
\pgfpathcurveto{\pgfqpoint{2.186169in}{0.932524in}}{\pgfqpoint{2.189441in}{0.940424in}}{\pgfqpoint{2.189441in}{0.948660in}}%
\pgfpathcurveto{\pgfqpoint{2.189441in}{0.956896in}}{\pgfqpoint{2.186169in}{0.964796in}}{\pgfqpoint{2.180345in}{0.970620in}}%
\pgfpathcurveto{\pgfqpoint{2.174521in}{0.976444in}}{\pgfqpoint{2.166621in}{0.979716in}}{\pgfqpoint{2.158385in}{0.979716in}}%
\pgfpathcurveto{\pgfqpoint{2.150148in}{0.979716in}}{\pgfqpoint{2.142248in}{0.976444in}}{\pgfqpoint{2.136424in}{0.970620in}}%
\pgfpathcurveto{\pgfqpoint{2.130600in}{0.964796in}}{\pgfqpoint{2.127328in}{0.956896in}}{\pgfqpoint{2.127328in}{0.948660in}}%
\pgfpathcurveto{\pgfqpoint{2.127328in}{0.940424in}}{\pgfqpoint{2.130600in}{0.932524in}}{\pgfqpoint{2.136424in}{0.926700in}}%
\pgfpathcurveto{\pgfqpoint{2.142248in}{0.920876in}}{\pgfqpoint{2.150148in}{0.917603in}}{\pgfqpoint{2.158385in}{0.917603in}}%
\pgfpathclose%
\pgfusepath{stroke,fill}%
\end{pgfscope}%
\begin{pgfscope}%
\pgfpathrectangle{\pgfqpoint{0.100000in}{0.212622in}}{\pgfqpoint{3.696000in}{3.696000in}}%
\pgfusepath{clip}%
\pgfsetbuttcap%
\pgfsetroundjoin%
\definecolor{currentfill}{rgb}{0.121569,0.466667,0.705882}%
\pgfsetfillcolor{currentfill}%
\pgfsetfillopacity{0.943575}%
\pgfsetlinewidth{1.003750pt}%
\definecolor{currentstroke}{rgb}{0.121569,0.466667,0.705882}%
\pgfsetstrokecolor{currentstroke}%
\pgfsetstrokeopacity{0.943575}%
\pgfsetdash{}{0pt}%
\pgfpathmoveto{\pgfqpoint{2.400514in}{0.957689in}}%
\pgfpathcurveto{\pgfqpoint{2.408750in}{0.957689in}}{\pgfqpoint{2.416650in}{0.960961in}}{\pgfqpoint{2.422474in}{0.966785in}}%
\pgfpathcurveto{\pgfqpoint{2.428298in}{0.972609in}}{\pgfqpoint{2.431571in}{0.980509in}}{\pgfqpoint{2.431571in}{0.988745in}}%
\pgfpathcurveto{\pgfqpoint{2.431571in}{0.996982in}}{\pgfqpoint{2.428298in}{1.004882in}}{\pgfqpoint{2.422474in}{1.010706in}}%
\pgfpathcurveto{\pgfqpoint{2.416650in}{1.016529in}}{\pgfqpoint{2.408750in}{1.019802in}}{\pgfqpoint{2.400514in}{1.019802in}}%
\pgfpathcurveto{\pgfqpoint{2.392278in}{1.019802in}}{\pgfqpoint{2.384378in}{1.016529in}}{\pgfqpoint{2.378554in}{1.010706in}}%
\pgfpathcurveto{\pgfqpoint{2.372730in}{1.004882in}}{\pgfqpoint{2.369458in}{0.996982in}}{\pgfqpoint{2.369458in}{0.988745in}}%
\pgfpathcurveto{\pgfqpoint{2.369458in}{0.980509in}}{\pgfqpoint{2.372730in}{0.972609in}}{\pgfqpoint{2.378554in}{0.966785in}}%
\pgfpathcurveto{\pgfqpoint{2.384378in}{0.960961in}}{\pgfqpoint{2.392278in}{0.957689in}}{\pgfqpoint{2.400514in}{0.957689in}}%
\pgfpathclose%
\pgfusepath{stroke,fill}%
\end{pgfscope}%
\begin{pgfscope}%
\pgfpathrectangle{\pgfqpoint{0.100000in}{0.212622in}}{\pgfqpoint{3.696000in}{3.696000in}}%
\pgfusepath{clip}%
\pgfsetbuttcap%
\pgfsetroundjoin%
\definecolor{currentfill}{rgb}{0.121569,0.466667,0.705882}%
\pgfsetfillcolor{currentfill}%
\pgfsetfillopacity{0.945378}%
\pgfsetlinewidth{1.003750pt}%
\definecolor{currentstroke}{rgb}{0.121569,0.466667,0.705882}%
\pgfsetstrokecolor{currentstroke}%
\pgfsetstrokeopacity{0.945378}%
\pgfsetdash{}{0pt}%
\pgfpathmoveto{\pgfqpoint{2.169570in}{0.909508in}}%
\pgfpathcurveto{\pgfqpoint{2.177806in}{0.909508in}}{\pgfqpoint{2.185707in}{0.912781in}}{\pgfqpoint{2.191530in}{0.918605in}}%
\pgfpathcurveto{\pgfqpoint{2.197354in}{0.924429in}}{\pgfqpoint{2.200627in}{0.932329in}}{\pgfqpoint{2.200627in}{0.940565in}}%
\pgfpathcurveto{\pgfqpoint{2.200627in}{0.948801in}}{\pgfqpoint{2.197354in}{0.956701in}}{\pgfqpoint{2.191530in}{0.962525in}}%
\pgfpathcurveto{\pgfqpoint{2.185707in}{0.968349in}}{\pgfqpoint{2.177806in}{0.971621in}}{\pgfqpoint{2.169570in}{0.971621in}}%
\pgfpathcurveto{\pgfqpoint{2.161334in}{0.971621in}}{\pgfqpoint{2.153434in}{0.968349in}}{\pgfqpoint{2.147610in}{0.962525in}}%
\pgfpathcurveto{\pgfqpoint{2.141786in}{0.956701in}}{\pgfqpoint{2.138514in}{0.948801in}}{\pgfqpoint{2.138514in}{0.940565in}}%
\pgfpathcurveto{\pgfqpoint{2.138514in}{0.932329in}}{\pgfqpoint{2.141786in}{0.924429in}}{\pgfqpoint{2.147610in}{0.918605in}}%
\pgfpathcurveto{\pgfqpoint{2.153434in}{0.912781in}}{\pgfqpoint{2.161334in}{0.909508in}}{\pgfqpoint{2.169570in}{0.909508in}}%
\pgfpathclose%
\pgfusepath{stroke,fill}%
\end{pgfscope}%
\begin{pgfscope}%
\pgfpathrectangle{\pgfqpoint{0.100000in}{0.212622in}}{\pgfqpoint{3.696000in}{3.696000in}}%
\pgfusepath{clip}%
\pgfsetbuttcap%
\pgfsetroundjoin%
\definecolor{currentfill}{rgb}{0.121569,0.466667,0.705882}%
\pgfsetfillcolor{currentfill}%
\pgfsetfillopacity{0.947960}%
\pgfsetlinewidth{1.003750pt}%
\definecolor{currentstroke}{rgb}{0.121569,0.466667,0.705882}%
\pgfsetstrokecolor{currentstroke}%
\pgfsetstrokeopacity{0.947960}%
\pgfsetdash{}{0pt}%
\pgfpathmoveto{\pgfqpoint{2.180468in}{0.901937in}}%
\pgfpathcurveto{\pgfqpoint{2.188704in}{0.901937in}}{\pgfqpoint{2.196604in}{0.905210in}}{\pgfqpoint{2.202428in}{0.911034in}}%
\pgfpathcurveto{\pgfqpoint{2.208252in}{0.916857in}}{\pgfqpoint{2.211524in}{0.924757in}}{\pgfqpoint{2.211524in}{0.932994in}}%
\pgfpathcurveto{\pgfqpoint{2.211524in}{0.941230in}}{\pgfqpoint{2.208252in}{0.949130in}}{\pgfqpoint{2.202428in}{0.954954in}}%
\pgfpathcurveto{\pgfqpoint{2.196604in}{0.960778in}}{\pgfqpoint{2.188704in}{0.964050in}}{\pgfqpoint{2.180468in}{0.964050in}}%
\pgfpathcurveto{\pgfqpoint{2.172231in}{0.964050in}}{\pgfqpoint{2.164331in}{0.960778in}}{\pgfqpoint{2.158507in}{0.954954in}}%
\pgfpathcurveto{\pgfqpoint{2.152683in}{0.949130in}}{\pgfqpoint{2.149411in}{0.941230in}}{\pgfqpoint{2.149411in}{0.932994in}}%
\pgfpathcurveto{\pgfqpoint{2.149411in}{0.924757in}}{\pgfqpoint{2.152683in}{0.916857in}}{\pgfqpoint{2.158507in}{0.911034in}}%
\pgfpathcurveto{\pgfqpoint{2.164331in}{0.905210in}}{\pgfqpoint{2.172231in}{0.901937in}}{\pgfqpoint{2.180468in}{0.901937in}}%
\pgfpathclose%
\pgfusepath{stroke,fill}%
\end{pgfscope}%
\begin{pgfscope}%
\pgfpathrectangle{\pgfqpoint{0.100000in}{0.212622in}}{\pgfqpoint{3.696000in}{3.696000in}}%
\pgfusepath{clip}%
\pgfsetbuttcap%
\pgfsetroundjoin%
\definecolor{currentfill}{rgb}{0.121569,0.466667,0.705882}%
\pgfsetfillcolor{currentfill}%
\pgfsetfillopacity{0.949340}%
\pgfsetlinewidth{1.003750pt}%
\definecolor{currentstroke}{rgb}{0.121569,0.466667,0.705882}%
\pgfsetstrokecolor{currentstroke}%
\pgfsetstrokeopacity{0.949340}%
\pgfsetdash{}{0pt}%
\pgfpathmoveto{\pgfqpoint{2.405615in}{0.937573in}}%
\pgfpathcurveto{\pgfqpoint{2.413851in}{0.937573in}}{\pgfqpoint{2.421751in}{0.940846in}}{\pgfqpoint{2.427575in}{0.946669in}}%
\pgfpathcurveto{\pgfqpoint{2.433399in}{0.952493in}}{\pgfqpoint{2.436671in}{0.960393in}}{\pgfqpoint{2.436671in}{0.968630in}}%
\pgfpathcurveto{\pgfqpoint{2.436671in}{0.976866in}}{\pgfqpoint{2.433399in}{0.984766in}}{\pgfqpoint{2.427575in}{0.990590in}}%
\pgfpathcurveto{\pgfqpoint{2.421751in}{0.996414in}}{\pgfqpoint{2.413851in}{0.999686in}}{\pgfqpoint{2.405615in}{0.999686in}}%
\pgfpathcurveto{\pgfqpoint{2.397378in}{0.999686in}}{\pgfqpoint{2.389478in}{0.996414in}}{\pgfqpoint{2.383654in}{0.990590in}}%
\pgfpathcurveto{\pgfqpoint{2.377831in}{0.984766in}}{\pgfqpoint{2.374558in}{0.976866in}}{\pgfqpoint{2.374558in}{0.968630in}}%
\pgfpathcurveto{\pgfqpoint{2.374558in}{0.960393in}}{\pgfqpoint{2.377831in}{0.952493in}}{\pgfqpoint{2.383654in}{0.946669in}}%
\pgfpathcurveto{\pgfqpoint{2.389478in}{0.940846in}}{\pgfqpoint{2.397378in}{0.937573in}}{\pgfqpoint{2.405615in}{0.937573in}}%
\pgfpathclose%
\pgfusepath{stroke,fill}%
\end{pgfscope}%
\begin{pgfscope}%
\pgfpathrectangle{\pgfqpoint{0.100000in}{0.212622in}}{\pgfqpoint{3.696000in}{3.696000in}}%
\pgfusepath{clip}%
\pgfsetbuttcap%
\pgfsetroundjoin%
\definecolor{currentfill}{rgb}{0.121569,0.466667,0.705882}%
\pgfsetfillcolor{currentfill}%
\pgfsetfillopacity{0.950449}%
\pgfsetlinewidth{1.003750pt}%
\definecolor{currentstroke}{rgb}{0.121569,0.466667,0.705882}%
\pgfsetstrokecolor{currentstroke}%
\pgfsetstrokeopacity{0.950449}%
\pgfsetdash{}{0pt}%
\pgfpathmoveto{\pgfqpoint{2.191084in}{0.894759in}}%
\pgfpathcurveto{\pgfqpoint{2.199320in}{0.894759in}}{\pgfqpoint{2.207220in}{0.898032in}}{\pgfqpoint{2.213044in}{0.903855in}}%
\pgfpathcurveto{\pgfqpoint{2.218868in}{0.909679in}}{\pgfqpoint{2.222141in}{0.917579in}}{\pgfqpoint{2.222141in}{0.925816in}}%
\pgfpathcurveto{\pgfqpoint{2.222141in}{0.934052in}}{\pgfqpoint{2.218868in}{0.941952in}}{\pgfqpoint{2.213044in}{0.947776in}}%
\pgfpathcurveto{\pgfqpoint{2.207220in}{0.953600in}}{\pgfqpoint{2.199320in}{0.956872in}}{\pgfqpoint{2.191084in}{0.956872in}}%
\pgfpathcurveto{\pgfqpoint{2.182848in}{0.956872in}}{\pgfqpoint{2.174948in}{0.953600in}}{\pgfqpoint{2.169124in}{0.947776in}}%
\pgfpathcurveto{\pgfqpoint{2.163300in}{0.941952in}}{\pgfqpoint{2.160028in}{0.934052in}}{\pgfqpoint{2.160028in}{0.925816in}}%
\pgfpathcurveto{\pgfqpoint{2.160028in}{0.917579in}}{\pgfqpoint{2.163300in}{0.909679in}}{\pgfqpoint{2.169124in}{0.903855in}}%
\pgfpathcurveto{\pgfqpoint{2.174948in}{0.898032in}}{\pgfqpoint{2.182848in}{0.894759in}}{\pgfqpoint{2.191084in}{0.894759in}}%
\pgfpathclose%
\pgfusepath{stroke,fill}%
\end{pgfscope}%
\begin{pgfscope}%
\pgfpathrectangle{\pgfqpoint{0.100000in}{0.212622in}}{\pgfqpoint{3.696000in}{3.696000in}}%
\pgfusepath{clip}%
\pgfsetbuttcap%
\pgfsetroundjoin%
\definecolor{currentfill}{rgb}{0.121569,0.466667,0.705882}%
\pgfsetfillcolor{currentfill}%
\pgfsetfillopacity{0.952819}%
\pgfsetlinewidth{1.003750pt}%
\definecolor{currentstroke}{rgb}{0.121569,0.466667,0.705882}%
\pgfsetstrokecolor{currentstroke}%
\pgfsetstrokeopacity{0.952819}%
\pgfsetdash{}{0pt}%
\pgfpathmoveto{\pgfqpoint{2.201416in}{0.887891in}}%
\pgfpathcurveto{\pgfqpoint{2.209652in}{0.887891in}}{\pgfqpoint{2.217552in}{0.891163in}}{\pgfqpoint{2.223376in}{0.896987in}}%
\pgfpathcurveto{\pgfqpoint{2.229200in}{0.902811in}}{\pgfqpoint{2.232472in}{0.910711in}}{\pgfqpoint{2.232472in}{0.918948in}}%
\pgfpathcurveto{\pgfqpoint{2.232472in}{0.927184in}}{\pgfqpoint{2.229200in}{0.935084in}}{\pgfqpoint{2.223376in}{0.940908in}}%
\pgfpathcurveto{\pgfqpoint{2.217552in}{0.946732in}}{\pgfqpoint{2.209652in}{0.950004in}}{\pgfqpoint{2.201416in}{0.950004in}}%
\pgfpathcurveto{\pgfqpoint{2.193180in}{0.950004in}}{\pgfqpoint{2.185280in}{0.946732in}}{\pgfqpoint{2.179456in}{0.940908in}}%
\pgfpathcurveto{\pgfqpoint{2.173632in}{0.935084in}}{\pgfqpoint{2.170359in}{0.927184in}}{\pgfqpoint{2.170359in}{0.918948in}}%
\pgfpathcurveto{\pgfqpoint{2.170359in}{0.910711in}}{\pgfqpoint{2.173632in}{0.902811in}}{\pgfqpoint{2.179456in}{0.896987in}}%
\pgfpathcurveto{\pgfqpoint{2.185280in}{0.891163in}}{\pgfqpoint{2.193180in}{0.887891in}}{\pgfqpoint{2.201416in}{0.887891in}}%
\pgfpathclose%
\pgfusepath{stroke,fill}%
\end{pgfscope}%
\begin{pgfscope}%
\pgfpathrectangle{\pgfqpoint{0.100000in}{0.212622in}}{\pgfqpoint{3.696000in}{3.696000in}}%
\pgfusepath{clip}%
\pgfsetbuttcap%
\pgfsetroundjoin%
\definecolor{currentfill}{rgb}{0.121569,0.466667,0.705882}%
\pgfsetfillcolor{currentfill}%
\pgfsetfillopacity{0.955120}%
\pgfsetlinewidth{1.003750pt}%
\definecolor{currentstroke}{rgb}{0.121569,0.466667,0.705882}%
\pgfsetstrokecolor{currentstroke}%
\pgfsetstrokeopacity{0.955120}%
\pgfsetdash{}{0pt}%
\pgfpathmoveto{\pgfqpoint{2.211360in}{0.881231in}}%
\pgfpathcurveto{\pgfqpoint{2.219596in}{0.881231in}}{\pgfqpoint{2.227496in}{0.884503in}}{\pgfqpoint{2.233320in}{0.890327in}}%
\pgfpathcurveto{\pgfqpoint{2.239144in}{0.896151in}}{\pgfqpoint{2.242416in}{0.904051in}}{\pgfqpoint{2.242416in}{0.912287in}}%
\pgfpathcurveto{\pgfqpoint{2.242416in}{0.920523in}}{\pgfqpoint{2.239144in}{0.928423in}}{\pgfqpoint{2.233320in}{0.934247in}}%
\pgfpathcurveto{\pgfqpoint{2.227496in}{0.940071in}}{\pgfqpoint{2.219596in}{0.943344in}}{\pgfqpoint{2.211360in}{0.943344in}}%
\pgfpathcurveto{\pgfqpoint{2.203123in}{0.943344in}}{\pgfqpoint{2.195223in}{0.940071in}}{\pgfqpoint{2.189400in}{0.934247in}}%
\pgfpathcurveto{\pgfqpoint{2.183576in}{0.928423in}}{\pgfqpoint{2.180303in}{0.920523in}}{\pgfqpoint{2.180303in}{0.912287in}}%
\pgfpathcurveto{\pgfqpoint{2.180303in}{0.904051in}}{\pgfqpoint{2.183576in}{0.896151in}}{\pgfqpoint{2.189400in}{0.890327in}}%
\pgfpathcurveto{\pgfqpoint{2.195223in}{0.884503in}}{\pgfqpoint{2.203123in}{0.881231in}}{\pgfqpoint{2.211360in}{0.881231in}}%
\pgfpathclose%
\pgfusepath{stroke,fill}%
\end{pgfscope}%
\begin{pgfscope}%
\pgfpathrectangle{\pgfqpoint{0.100000in}{0.212622in}}{\pgfqpoint{3.696000in}{3.696000in}}%
\pgfusepath{clip}%
\pgfsetbuttcap%
\pgfsetroundjoin%
\definecolor{currentfill}{rgb}{0.121569,0.466667,0.705882}%
\pgfsetfillcolor{currentfill}%
\pgfsetfillopacity{0.955237}%
\pgfsetlinewidth{1.003750pt}%
\definecolor{currentstroke}{rgb}{0.121569,0.466667,0.705882}%
\pgfsetstrokecolor{currentstroke}%
\pgfsetstrokeopacity{0.955237}%
\pgfsetdash{}{0pt}%
\pgfpathmoveto{\pgfqpoint{2.410885in}{0.915716in}}%
\pgfpathcurveto{\pgfqpoint{2.419121in}{0.915716in}}{\pgfqpoint{2.427021in}{0.918988in}}{\pgfqpoint{2.432845in}{0.924812in}}%
\pgfpathcurveto{\pgfqpoint{2.438669in}{0.930636in}}{\pgfqpoint{2.441941in}{0.938536in}}{\pgfqpoint{2.441941in}{0.946773in}}%
\pgfpathcurveto{\pgfqpoint{2.441941in}{0.955009in}}{\pgfqpoint{2.438669in}{0.962909in}}{\pgfqpoint{2.432845in}{0.968733in}}%
\pgfpathcurveto{\pgfqpoint{2.427021in}{0.974557in}}{\pgfqpoint{2.419121in}{0.977829in}}{\pgfqpoint{2.410885in}{0.977829in}}%
\pgfpathcurveto{\pgfqpoint{2.402648in}{0.977829in}}{\pgfqpoint{2.394748in}{0.974557in}}{\pgfqpoint{2.388924in}{0.968733in}}%
\pgfpathcurveto{\pgfqpoint{2.383101in}{0.962909in}}{\pgfqpoint{2.379828in}{0.955009in}}{\pgfqpoint{2.379828in}{0.946773in}}%
\pgfpathcurveto{\pgfqpoint{2.379828in}{0.938536in}}{\pgfqpoint{2.383101in}{0.930636in}}{\pgfqpoint{2.388924in}{0.924812in}}%
\pgfpathcurveto{\pgfqpoint{2.394748in}{0.918988in}}{\pgfqpoint{2.402648in}{0.915716in}}{\pgfqpoint{2.410885in}{0.915716in}}%
\pgfpathclose%
\pgfusepath{stroke,fill}%
\end{pgfscope}%
\begin{pgfscope}%
\pgfpathrectangle{\pgfqpoint{0.100000in}{0.212622in}}{\pgfqpoint{3.696000in}{3.696000in}}%
\pgfusepath{clip}%
\pgfsetbuttcap%
\pgfsetroundjoin%
\definecolor{currentfill}{rgb}{0.121569,0.466667,0.705882}%
\pgfsetfillcolor{currentfill}%
\pgfsetfillopacity{0.957397}%
\pgfsetlinewidth{1.003750pt}%
\definecolor{currentstroke}{rgb}{0.121569,0.466667,0.705882}%
\pgfsetstrokecolor{currentstroke}%
\pgfsetstrokeopacity{0.957397}%
\pgfsetdash{}{0pt}%
\pgfpathmoveto{\pgfqpoint{2.220864in}{0.874683in}}%
\pgfpathcurveto{\pgfqpoint{2.229100in}{0.874683in}}{\pgfqpoint{2.237000in}{0.877955in}}{\pgfqpoint{2.242824in}{0.883779in}}%
\pgfpathcurveto{\pgfqpoint{2.248648in}{0.889603in}}{\pgfqpoint{2.251920in}{0.897503in}}{\pgfqpoint{2.251920in}{0.905740in}}%
\pgfpathcurveto{\pgfqpoint{2.251920in}{0.913976in}}{\pgfqpoint{2.248648in}{0.921876in}}{\pgfqpoint{2.242824in}{0.927700in}}%
\pgfpathcurveto{\pgfqpoint{2.237000in}{0.933524in}}{\pgfqpoint{2.229100in}{0.936796in}}{\pgfqpoint{2.220864in}{0.936796in}}%
\pgfpathcurveto{\pgfqpoint{2.212628in}{0.936796in}}{\pgfqpoint{2.204728in}{0.933524in}}{\pgfqpoint{2.198904in}{0.927700in}}%
\pgfpathcurveto{\pgfqpoint{2.193080in}{0.921876in}}{\pgfqpoint{2.189807in}{0.913976in}}{\pgfqpoint{2.189807in}{0.905740in}}%
\pgfpathcurveto{\pgfqpoint{2.189807in}{0.897503in}}{\pgfqpoint{2.193080in}{0.889603in}}{\pgfqpoint{2.198904in}{0.883779in}}%
\pgfpathcurveto{\pgfqpoint{2.204728in}{0.877955in}}{\pgfqpoint{2.212628in}{0.874683in}}{\pgfqpoint{2.220864in}{0.874683in}}%
\pgfpathclose%
\pgfusepath{stroke,fill}%
\end{pgfscope}%
\begin{pgfscope}%
\pgfpathrectangle{\pgfqpoint{0.100000in}{0.212622in}}{\pgfqpoint{3.696000in}{3.696000in}}%
\pgfusepath{clip}%
\pgfsetbuttcap%
\pgfsetroundjoin%
\definecolor{currentfill}{rgb}{0.121569,0.466667,0.705882}%
\pgfsetfillcolor{currentfill}%
\pgfsetfillopacity{0.959502}%
\pgfsetlinewidth{1.003750pt}%
\definecolor{currentstroke}{rgb}{0.121569,0.466667,0.705882}%
\pgfsetstrokecolor{currentstroke}%
\pgfsetstrokeopacity{0.959502}%
\pgfsetdash{}{0pt}%
\pgfpathmoveto{\pgfqpoint{2.230117in}{0.868191in}}%
\pgfpathcurveto{\pgfqpoint{2.238353in}{0.868191in}}{\pgfqpoint{2.246253in}{0.871464in}}{\pgfqpoint{2.252077in}{0.877288in}}%
\pgfpathcurveto{\pgfqpoint{2.257901in}{0.883111in}}{\pgfqpoint{2.261173in}{0.891012in}}{\pgfqpoint{2.261173in}{0.899248in}}%
\pgfpathcurveto{\pgfqpoint{2.261173in}{0.907484in}}{\pgfqpoint{2.257901in}{0.915384in}}{\pgfqpoint{2.252077in}{0.921208in}}%
\pgfpathcurveto{\pgfqpoint{2.246253in}{0.927032in}}{\pgfqpoint{2.238353in}{0.930304in}}{\pgfqpoint{2.230117in}{0.930304in}}%
\pgfpathcurveto{\pgfqpoint{2.221880in}{0.930304in}}{\pgfqpoint{2.213980in}{0.927032in}}{\pgfqpoint{2.208157in}{0.921208in}}%
\pgfpathcurveto{\pgfqpoint{2.202333in}{0.915384in}}{\pgfqpoint{2.199060in}{0.907484in}}{\pgfqpoint{2.199060in}{0.899248in}}%
\pgfpathcurveto{\pgfqpoint{2.199060in}{0.891012in}}{\pgfqpoint{2.202333in}{0.883111in}}{\pgfqpoint{2.208157in}{0.877288in}}%
\pgfpathcurveto{\pgfqpoint{2.213980in}{0.871464in}}{\pgfqpoint{2.221880in}{0.868191in}}{\pgfqpoint{2.230117in}{0.868191in}}%
\pgfpathclose%
\pgfusepath{stroke,fill}%
\end{pgfscope}%
\begin{pgfscope}%
\pgfpathrectangle{\pgfqpoint{0.100000in}{0.212622in}}{\pgfqpoint{3.696000in}{3.696000in}}%
\pgfusepath{clip}%
\pgfsetbuttcap%
\pgfsetroundjoin%
\definecolor{currentfill}{rgb}{0.121569,0.466667,0.705882}%
\pgfsetfillcolor{currentfill}%
\pgfsetfillopacity{0.961351}%
\pgfsetlinewidth{1.003750pt}%
\definecolor{currentstroke}{rgb}{0.121569,0.466667,0.705882}%
\pgfsetstrokecolor{currentstroke}%
\pgfsetstrokeopacity{0.961351}%
\pgfsetdash{}{0pt}%
\pgfpathmoveto{\pgfqpoint{2.416615in}{0.893089in}}%
\pgfpathcurveto{\pgfqpoint{2.424851in}{0.893089in}}{\pgfqpoint{2.432751in}{0.896361in}}{\pgfqpoint{2.438575in}{0.902185in}}%
\pgfpathcurveto{\pgfqpoint{2.444399in}{0.908009in}}{\pgfqpoint{2.447671in}{0.915909in}}{\pgfqpoint{2.447671in}{0.924145in}}%
\pgfpathcurveto{\pgfqpoint{2.447671in}{0.932381in}}{\pgfqpoint{2.444399in}{0.940281in}}{\pgfqpoint{2.438575in}{0.946105in}}%
\pgfpathcurveto{\pgfqpoint{2.432751in}{0.951929in}}{\pgfqpoint{2.424851in}{0.955202in}}{\pgfqpoint{2.416615in}{0.955202in}}%
\pgfpathcurveto{\pgfqpoint{2.408378in}{0.955202in}}{\pgfqpoint{2.400478in}{0.951929in}}{\pgfqpoint{2.394654in}{0.946105in}}%
\pgfpathcurveto{\pgfqpoint{2.388831in}{0.940281in}}{\pgfqpoint{2.385558in}{0.932381in}}{\pgfqpoint{2.385558in}{0.924145in}}%
\pgfpathcurveto{\pgfqpoint{2.385558in}{0.915909in}}{\pgfqpoint{2.388831in}{0.908009in}}{\pgfqpoint{2.394654in}{0.902185in}}%
\pgfpathcurveto{\pgfqpoint{2.400478in}{0.896361in}}{\pgfqpoint{2.408378in}{0.893089in}}{\pgfqpoint{2.416615in}{0.893089in}}%
\pgfpathclose%
\pgfusepath{stroke,fill}%
\end{pgfscope}%
\begin{pgfscope}%
\pgfpathrectangle{\pgfqpoint{0.100000in}{0.212622in}}{\pgfqpoint{3.696000in}{3.696000in}}%
\pgfusepath{clip}%
\pgfsetbuttcap%
\pgfsetroundjoin%
\definecolor{currentfill}{rgb}{0.121569,0.466667,0.705882}%
\pgfsetfillcolor{currentfill}%
\pgfsetfillopacity{0.961595}%
\pgfsetlinewidth{1.003750pt}%
\definecolor{currentstroke}{rgb}{0.121569,0.466667,0.705882}%
\pgfsetstrokecolor{currentstroke}%
\pgfsetstrokeopacity{0.961595}%
\pgfsetdash{}{0pt}%
\pgfpathmoveto{\pgfqpoint{2.239032in}{0.862224in}}%
\pgfpathcurveto{\pgfqpoint{2.247268in}{0.862224in}}{\pgfqpoint{2.255168in}{0.865497in}}{\pgfqpoint{2.260992in}{0.871320in}}%
\pgfpathcurveto{\pgfqpoint{2.266816in}{0.877144in}}{\pgfqpoint{2.270089in}{0.885044in}}{\pgfqpoint{2.270089in}{0.893281in}}%
\pgfpathcurveto{\pgfqpoint{2.270089in}{0.901517in}}{\pgfqpoint{2.266816in}{0.909417in}}{\pgfqpoint{2.260992in}{0.915241in}}%
\pgfpathcurveto{\pgfqpoint{2.255168in}{0.921065in}}{\pgfqpoint{2.247268in}{0.924337in}}{\pgfqpoint{2.239032in}{0.924337in}}%
\pgfpathcurveto{\pgfqpoint{2.230796in}{0.924337in}}{\pgfqpoint{2.222896in}{0.921065in}}{\pgfqpoint{2.217072in}{0.915241in}}%
\pgfpathcurveto{\pgfqpoint{2.211248in}{0.909417in}}{\pgfqpoint{2.207976in}{0.901517in}}{\pgfqpoint{2.207976in}{0.893281in}}%
\pgfpathcurveto{\pgfqpoint{2.207976in}{0.885044in}}{\pgfqpoint{2.211248in}{0.877144in}}{\pgfqpoint{2.217072in}{0.871320in}}%
\pgfpathcurveto{\pgfqpoint{2.222896in}{0.865497in}}{\pgfqpoint{2.230796in}{0.862224in}}{\pgfqpoint{2.239032in}{0.862224in}}%
\pgfpathclose%
\pgfusepath{stroke,fill}%
\end{pgfscope}%
\begin{pgfscope}%
\pgfpathrectangle{\pgfqpoint{0.100000in}{0.212622in}}{\pgfqpoint{3.696000in}{3.696000in}}%
\pgfusepath{clip}%
\pgfsetbuttcap%
\pgfsetroundjoin%
\definecolor{currentfill}{rgb}{0.121569,0.466667,0.705882}%
\pgfsetfillcolor{currentfill}%
\pgfsetfillopacity{0.963553}%
\pgfsetlinewidth{1.003750pt}%
\definecolor{currentstroke}{rgb}{0.121569,0.466667,0.705882}%
\pgfsetstrokecolor{currentstroke}%
\pgfsetstrokeopacity{0.963553}%
\pgfsetdash{}{0pt}%
\pgfpathmoveto{\pgfqpoint{2.247629in}{0.856445in}}%
\pgfpathcurveto{\pgfqpoint{2.255865in}{0.856445in}}{\pgfqpoint{2.263765in}{0.859717in}}{\pgfqpoint{2.269589in}{0.865541in}}%
\pgfpathcurveto{\pgfqpoint{2.275413in}{0.871365in}}{\pgfqpoint{2.278685in}{0.879265in}}{\pgfqpoint{2.278685in}{0.887501in}}%
\pgfpathcurveto{\pgfqpoint{2.278685in}{0.895738in}}{\pgfqpoint{2.275413in}{0.903638in}}{\pgfqpoint{2.269589in}{0.909462in}}%
\pgfpathcurveto{\pgfqpoint{2.263765in}{0.915286in}}{\pgfqpoint{2.255865in}{0.918558in}}{\pgfqpoint{2.247629in}{0.918558in}}%
\pgfpathcurveto{\pgfqpoint{2.239392in}{0.918558in}}{\pgfqpoint{2.231492in}{0.915286in}}{\pgfqpoint{2.225668in}{0.909462in}}%
\pgfpathcurveto{\pgfqpoint{2.219844in}{0.903638in}}{\pgfqpoint{2.216572in}{0.895738in}}{\pgfqpoint{2.216572in}{0.887501in}}%
\pgfpathcurveto{\pgfqpoint{2.216572in}{0.879265in}}{\pgfqpoint{2.219844in}{0.871365in}}{\pgfqpoint{2.225668in}{0.865541in}}%
\pgfpathcurveto{\pgfqpoint{2.231492in}{0.859717in}}{\pgfqpoint{2.239392in}{0.856445in}}{\pgfqpoint{2.247629in}{0.856445in}}%
\pgfpathclose%
\pgfusepath{stroke,fill}%
\end{pgfscope}%
\begin{pgfscope}%
\pgfpathrectangle{\pgfqpoint{0.100000in}{0.212622in}}{\pgfqpoint{3.696000in}{3.696000in}}%
\pgfusepath{clip}%
\pgfsetbuttcap%
\pgfsetroundjoin%
\definecolor{currentfill}{rgb}{0.121569,0.466667,0.705882}%
\pgfsetfillcolor{currentfill}%
\pgfsetfillopacity{0.965405}%
\pgfsetlinewidth{1.003750pt}%
\definecolor{currentstroke}{rgb}{0.121569,0.466667,0.705882}%
\pgfsetstrokecolor{currentstroke}%
\pgfsetstrokeopacity{0.965405}%
\pgfsetdash{}{0pt}%
\pgfpathmoveto{\pgfqpoint{2.255915in}{0.850802in}}%
\pgfpathcurveto{\pgfqpoint{2.264152in}{0.850802in}}{\pgfqpoint{2.272052in}{0.854075in}}{\pgfqpoint{2.277876in}{0.859898in}}%
\pgfpathcurveto{\pgfqpoint{2.283700in}{0.865722in}}{\pgfqpoint{2.286972in}{0.873622in}}{\pgfqpoint{2.286972in}{0.881859in}}%
\pgfpathcurveto{\pgfqpoint{2.286972in}{0.890095in}}{\pgfqpoint{2.283700in}{0.897995in}}{\pgfqpoint{2.277876in}{0.903819in}}%
\pgfpathcurveto{\pgfqpoint{2.272052in}{0.909643in}}{\pgfqpoint{2.264152in}{0.912915in}}{\pgfqpoint{2.255915in}{0.912915in}}%
\pgfpathcurveto{\pgfqpoint{2.247679in}{0.912915in}}{\pgfqpoint{2.239779in}{0.909643in}}{\pgfqpoint{2.233955in}{0.903819in}}%
\pgfpathcurveto{\pgfqpoint{2.228131in}{0.897995in}}{\pgfqpoint{2.224859in}{0.890095in}}{\pgfqpoint{2.224859in}{0.881859in}}%
\pgfpathcurveto{\pgfqpoint{2.224859in}{0.873622in}}{\pgfqpoint{2.228131in}{0.865722in}}{\pgfqpoint{2.233955in}{0.859898in}}%
\pgfpathcurveto{\pgfqpoint{2.239779in}{0.854075in}}{\pgfqpoint{2.247679in}{0.850802in}}{\pgfqpoint{2.255915in}{0.850802in}}%
\pgfpathclose%
\pgfusepath{stroke,fill}%
\end{pgfscope}%
\begin{pgfscope}%
\pgfpathrectangle{\pgfqpoint{0.100000in}{0.212622in}}{\pgfqpoint{3.696000in}{3.696000in}}%
\pgfusepath{clip}%
\pgfsetbuttcap%
\pgfsetroundjoin%
\definecolor{currentfill}{rgb}{0.121569,0.466667,0.705882}%
\pgfsetfillcolor{currentfill}%
\pgfsetfillopacity{0.967149}%
\pgfsetlinewidth{1.003750pt}%
\definecolor{currentstroke}{rgb}{0.121569,0.466667,0.705882}%
\pgfsetstrokecolor{currentstroke}%
\pgfsetstrokeopacity{0.967149}%
\pgfsetdash{}{0pt}%
\pgfpathmoveto{\pgfqpoint{2.263928in}{0.845359in}}%
\pgfpathcurveto{\pgfqpoint{2.272165in}{0.845359in}}{\pgfqpoint{2.280065in}{0.848631in}}{\pgfqpoint{2.285889in}{0.854455in}}%
\pgfpathcurveto{\pgfqpoint{2.291713in}{0.860279in}}{\pgfqpoint{2.294985in}{0.868179in}}{\pgfqpoint{2.294985in}{0.876416in}}%
\pgfpathcurveto{\pgfqpoint{2.294985in}{0.884652in}}{\pgfqpoint{2.291713in}{0.892552in}}{\pgfqpoint{2.285889in}{0.898376in}}%
\pgfpathcurveto{\pgfqpoint{2.280065in}{0.904200in}}{\pgfqpoint{2.272165in}{0.907472in}}{\pgfqpoint{2.263928in}{0.907472in}}%
\pgfpathcurveto{\pgfqpoint{2.255692in}{0.907472in}}{\pgfqpoint{2.247792in}{0.904200in}}{\pgfqpoint{2.241968in}{0.898376in}}%
\pgfpathcurveto{\pgfqpoint{2.236144in}{0.892552in}}{\pgfqpoint{2.232872in}{0.884652in}}{\pgfqpoint{2.232872in}{0.876416in}}%
\pgfpathcurveto{\pgfqpoint{2.232872in}{0.868179in}}{\pgfqpoint{2.236144in}{0.860279in}}{\pgfqpoint{2.241968in}{0.854455in}}%
\pgfpathcurveto{\pgfqpoint{2.247792in}{0.848631in}}{\pgfqpoint{2.255692in}{0.845359in}}{\pgfqpoint{2.263928in}{0.845359in}}%
\pgfpathclose%
\pgfusepath{stroke,fill}%
\end{pgfscope}%
\begin{pgfscope}%
\pgfpathrectangle{\pgfqpoint{0.100000in}{0.212622in}}{\pgfqpoint{3.696000in}{3.696000in}}%
\pgfusepath{clip}%
\pgfsetbuttcap%
\pgfsetroundjoin%
\definecolor{currentfill}{rgb}{0.121569,0.466667,0.705882}%
\pgfsetfillcolor{currentfill}%
\pgfsetfillopacity{0.967716}%
\pgfsetlinewidth{1.003750pt}%
\definecolor{currentstroke}{rgb}{0.121569,0.466667,0.705882}%
\pgfsetstrokecolor{currentstroke}%
\pgfsetstrokeopacity{0.967716}%
\pgfsetdash{}{0pt}%
\pgfpathmoveto{\pgfqpoint{2.422557in}{0.869581in}}%
\pgfpathcurveto{\pgfqpoint{2.430794in}{0.869581in}}{\pgfqpoint{2.438694in}{0.872853in}}{\pgfqpoint{2.444518in}{0.878677in}}%
\pgfpathcurveto{\pgfqpoint{2.450341in}{0.884501in}}{\pgfqpoint{2.453614in}{0.892401in}}{\pgfqpoint{2.453614in}{0.900637in}}%
\pgfpathcurveto{\pgfqpoint{2.453614in}{0.908873in}}{\pgfqpoint{2.450341in}{0.916773in}}{\pgfqpoint{2.444518in}{0.922597in}}%
\pgfpathcurveto{\pgfqpoint{2.438694in}{0.928421in}}{\pgfqpoint{2.430794in}{0.931694in}}{\pgfqpoint{2.422557in}{0.931694in}}%
\pgfpathcurveto{\pgfqpoint{2.414321in}{0.931694in}}{\pgfqpoint{2.406421in}{0.928421in}}{\pgfqpoint{2.400597in}{0.922597in}}%
\pgfpathcurveto{\pgfqpoint{2.394773in}{0.916773in}}{\pgfqpoint{2.391501in}{0.908873in}}{\pgfqpoint{2.391501in}{0.900637in}}%
\pgfpathcurveto{\pgfqpoint{2.391501in}{0.892401in}}{\pgfqpoint{2.394773in}{0.884501in}}{\pgfqpoint{2.400597in}{0.878677in}}%
\pgfpathcurveto{\pgfqpoint{2.406421in}{0.872853in}}{\pgfqpoint{2.414321in}{0.869581in}}{\pgfqpoint{2.422557in}{0.869581in}}%
\pgfpathclose%
\pgfusepath{stroke,fill}%
\end{pgfscope}%
\begin{pgfscope}%
\pgfpathrectangle{\pgfqpoint{0.100000in}{0.212622in}}{\pgfqpoint{3.696000in}{3.696000in}}%
\pgfusepath{clip}%
\pgfsetbuttcap%
\pgfsetroundjoin%
\definecolor{currentfill}{rgb}{0.121569,0.466667,0.705882}%
\pgfsetfillcolor{currentfill}%
\pgfsetfillopacity{0.968772}%
\pgfsetlinewidth{1.003750pt}%
\definecolor{currentstroke}{rgb}{0.121569,0.466667,0.705882}%
\pgfsetstrokecolor{currentstroke}%
\pgfsetstrokeopacity{0.968772}%
\pgfsetdash{}{0pt}%
\pgfpathmoveto{\pgfqpoint{2.271715in}{0.840240in}}%
\pgfpathcurveto{\pgfqpoint{2.279951in}{0.840240in}}{\pgfqpoint{2.287851in}{0.843513in}}{\pgfqpoint{2.293675in}{0.849336in}}%
\pgfpathcurveto{\pgfqpoint{2.299499in}{0.855160in}}{\pgfqpoint{2.302772in}{0.863060in}}{\pgfqpoint{2.302772in}{0.871297in}}%
\pgfpathcurveto{\pgfqpoint{2.302772in}{0.879533in}}{\pgfqpoint{2.299499in}{0.887433in}}{\pgfqpoint{2.293675in}{0.893257in}}%
\pgfpathcurveto{\pgfqpoint{2.287851in}{0.899081in}}{\pgfqpoint{2.279951in}{0.902353in}}{\pgfqpoint{2.271715in}{0.902353in}}%
\pgfpathcurveto{\pgfqpoint{2.263479in}{0.902353in}}{\pgfqpoint{2.255579in}{0.899081in}}{\pgfqpoint{2.249755in}{0.893257in}}%
\pgfpathcurveto{\pgfqpoint{2.243931in}{0.887433in}}{\pgfqpoint{2.240659in}{0.879533in}}{\pgfqpoint{2.240659in}{0.871297in}}%
\pgfpathcurveto{\pgfqpoint{2.240659in}{0.863060in}}{\pgfqpoint{2.243931in}{0.855160in}}{\pgfqpoint{2.249755in}{0.849336in}}%
\pgfpathcurveto{\pgfqpoint{2.255579in}{0.843513in}}{\pgfqpoint{2.263479in}{0.840240in}}{\pgfqpoint{2.271715in}{0.840240in}}%
\pgfpathclose%
\pgfusepath{stroke,fill}%
\end{pgfscope}%
\begin{pgfscope}%
\pgfpathrectangle{\pgfqpoint{0.100000in}{0.212622in}}{\pgfqpoint{3.696000in}{3.696000in}}%
\pgfusepath{clip}%
\pgfsetbuttcap%
\pgfsetroundjoin%
\definecolor{currentfill}{rgb}{0.121569,0.466667,0.705882}%
\pgfsetfillcolor{currentfill}%
\pgfsetfillopacity{0.970288}%
\pgfsetlinewidth{1.003750pt}%
\definecolor{currentstroke}{rgb}{0.121569,0.466667,0.705882}%
\pgfsetstrokecolor{currentstroke}%
\pgfsetstrokeopacity{0.970288}%
\pgfsetdash{}{0pt}%
\pgfpathmoveto{\pgfqpoint{2.279190in}{0.835328in}}%
\pgfpathcurveto{\pgfqpoint{2.287426in}{0.835328in}}{\pgfqpoint{2.295326in}{0.838600in}}{\pgfqpoint{2.301150in}{0.844424in}}%
\pgfpathcurveto{\pgfqpoint{2.306974in}{0.850248in}}{\pgfqpoint{2.310247in}{0.858148in}}{\pgfqpoint{2.310247in}{0.866384in}}%
\pgfpathcurveto{\pgfqpoint{2.310247in}{0.874620in}}{\pgfqpoint{2.306974in}{0.882521in}}{\pgfqpoint{2.301150in}{0.888344in}}%
\pgfpathcurveto{\pgfqpoint{2.295326in}{0.894168in}}{\pgfqpoint{2.287426in}{0.897441in}}{\pgfqpoint{2.279190in}{0.897441in}}%
\pgfpathcurveto{\pgfqpoint{2.270954in}{0.897441in}}{\pgfqpoint{2.263054in}{0.894168in}}{\pgfqpoint{2.257230in}{0.888344in}}%
\pgfpathcurveto{\pgfqpoint{2.251406in}{0.882521in}}{\pgfqpoint{2.248134in}{0.874620in}}{\pgfqpoint{2.248134in}{0.866384in}}%
\pgfpathcurveto{\pgfqpoint{2.248134in}{0.858148in}}{\pgfqpoint{2.251406in}{0.850248in}}{\pgfqpoint{2.257230in}{0.844424in}}%
\pgfpathcurveto{\pgfqpoint{2.263054in}{0.838600in}}{\pgfqpoint{2.270954in}{0.835328in}}{\pgfqpoint{2.279190in}{0.835328in}}%
\pgfpathclose%
\pgfusepath{stroke,fill}%
\end{pgfscope}%
\begin{pgfscope}%
\pgfpathrectangle{\pgfqpoint{0.100000in}{0.212622in}}{\pgfqpoint{3.696000in}{3.696000in}}%
\pgfusepath{clip}%
\pgfsetbuttcap%
\pgfsetroundjoin%
\definecolor{currentfill}{rgb}{0.121569,0.466667,0.705882}%
\pgfsetfillcolor{currentfill}%
\pgfsetfillopacity{0.971728}%
\pgfsetlinewidth{1.003750pt}%
\definecolor{currentstroke}{rgb}{0.121569,0.466667,0.705882}%
\pgfsetstrokecolor{currentstroke}%
\pgfsetstrokeopacity{0.971728}%
\pgfsetdash{}{0pt}%
\pgfpathmoveto{\pgfqpoint{2.286329in}{0.830572in}}%
\pgfpathcurveto{\pgfqpoint{2.294565in}{0.830572in}}{\pgfqpoint{2.302465in}{0.833844in}}{\pgfqpoint{2.308289in}{0.839668in}}%
\pgfpathcurveto{\pgfqpoint{2.314113in}{0.845492in}}{\pgfqpoint{2.317386in}{0.853392in}}{\pgfqpoint{2.317386in}{0.861629in}}%
\pgfpathcurveto{\pgfqpoint{2.317386in}{0.869865in}}{\pgfqpoint{2.314113in}{0.877765in}}{\pgfqpoint{2.308289in}{0.883589in}}%
\pgfpathcurveto{\pgfqpoint{2.302465in}{0.889413in}}{\pgfqpoint{2.294565in}{0.892685in}}{\pgfqpoint{2.286329in}{0.892685in}}%
\pgfpathcurveto{\pgfqpoint{2.278093in}{0.892685in}}{\pgfqpoint{2.270193in}{0.889413in}}{\pgfqpoint{2.264369in}{0.883589in}}%
\pgfpathcurveto{\pgfqpoint{2.258545in}{0.877765in}}{\pgfqpoint{2.255273in}{0.869865in}}{\pgfqpoint{2.255273in}{0.861629in}}%
\pgfpathcurveto{\pgfqpoint{2.255273in}{0.853392in}}{\pgfqpoint{2.258545in}{0.845492in}}{\pgfqpoint{2.264369in}{0.839668in}}%
\pgfpathcurveto{\pgfqpoint{2.270193in}{0.833844in}}{\pgfqpoint{2.278093in}{0.830572in}}{\pgfqpoint{2.286329in}{0.830572in}}%
\pgfpathclose%
\pgfusepath{stroke,fill}%
\end{pgfscope}%
\begin{pgfscope}%
\pgfpathrectangle{\pgfqpoint{0.100000in}{0.212622in}}{\pgfqpoint{3.696000in}{3.696000in}}%
\pgfusepath{clip}%
\pgfsetbuttcap%
\pgfsetroundjoin%
\definecolor{currentfill}{rgb}{0.121569,0.466667,0.705882}%
\pgfsetfillcolor{currentfill}%
\pgfsetfillopacity{0.973146}%
\pgfsetlinewidth{1.003750pt}%
\definecolor{currentstroke}{rgb}{0.121569,0.466667,0.705882}%
\pgfsetstrokecolor{currentstroke}%
\pgfsetstrokeopacity{0.973146}%
\pgfsetdash{}{0pt}%
\pgfpathmoveto{\pgfqpoint{2.293081in}{0.826006in}}%
\pgfpathcurveto{\pgfqpoint{2.301318in}{0.826006in}}{\pgfqpoint{2.309218in}{0.829278in}}{\pgfqpoint{2.315042in}{0.835102in}}%
\pgfpathcurveto{\pgfqpoint{2.320866in}{0.840926in}}{\pgfqpoint{2.324138in}{0.848826in}}{\pgfqpoint{2.324138in}{0.857062in}}%
\pgfpathcurveto{\pgfqpoint{2.324138in}{0.865298in}}{\pgfqpoint{2.320866in}{0.873198in}}{\pgfqpoint{2.315042in}{0.879022in}}%
\pgfpathcurveto{\pgfqpoint{2.309218in}{0.884846in}}{\pgfqpoint{2.301318in}{0.888119in}}{\pgfqpoint{2.293081in}{0.888119in}}%
\pgfpathcurveto{\pgfqpoint{2.284845in}{0.888119in}}{\pgfqpoint{2.276945in}{0.884846in}}{\pgfqpoint{2.271121in}{0.879022in}}%
\pgfpathcurveto{\pgfqpoint{2.265297in}{0.873198in}}{\pgfqpoint{2.262025in}{0.865298in}}{\pgfqpoint{2.262025in}{0.857062in}}%
\pgfpathcurveto{\pgfqpoint{2.262025in}{0.848826in}}{\pgfqpoint{2.265297in}{0.840926in}}{\pgfqpoint{2.271121in}{0.835102in}}%
\pgfpathcurveto{\pgfqpoint{2.276945in}{0.829278in}}{\pgfqpoint{2.284845in}{0.826006in}}{\pgfqpoint{2.293081in}{0.826006in}}%
\pgfpathclose%
\pgfusepath{stroke,fill}%
\end{pgfscope}%
\begin{pgfscope}%
\pgfpathrectangle{\pgfqpoint{0.100000in}{0.212622in}}{\pgfqpoint{3.696000in}{3.696000in}}%
\pgfusepath{clip}%
\pgfsetbuttcap%
\pgfsetroundjoin%
\definecolor{currentfill}{rgb}{0.121569,0.466667,0.705882}%
\pgfsetfillcolor{currentfill}%
\pgfsetfillopacity{0.974499}%
\pgfsetlinewidth{1.003750pt}%
\definecolor{currentstroke}{rgb}{0.121569,0.466667,0.705882}%
\pgfsetstrokecolor{currentstroke}%
\pgfsetstrokeopacity{0.974499}%
\pgfsetdash{}{0pt}%
\pgfpathmoveto{\pgfqpoint{2.299480in}{0.821756in}}%
\pgfpathcurveto{\pgfqpoint{2.307716in}{0.821756in}}{\pgfqpoint{2.315616in}{0.825028in}}{\pgfqpoint{2.321440in}{0.830852in}}%
\pgfpathcurveto{\pgfqpoint{2.327264in}{0.836676in}}{\pgfqpoint{2.330537in}{0.844576in}}{\pgfqpoint{2.330537in}{0.852812in}}%
\pgfpathcurveto{\pgfqpoint{2.330537in}{0.861048in}}{\pgfqpoint{2.327264in}{0.868948in}}{\pgfqpoint{2.321440in}{0.874772in}}%
\pgfpathcurveto{\pgfqpoint{2.315616in}{0.880596in}}{\pgfqpoint{2.307716in}{0.883869in}}{\pgfqpoint{2.299480in}{0.883869in}}%
\pgfpathcurveto{\pgfqpoint{2.291244in}{0.883869in}}{\pgfqpoint{2.283344in}{0.880596in}}{\pgfqpoint{2.277520in}{0.874772in}}%
\pgfpathcurveto{\pgfqpoint{2.271696in}{0.868948in}}{\pgfqpoint{2.268424in}{0.861048in}}{\pgfqpoint{2.268424in}{0.852812in}}%
\pgfpathcurveto{\pgfqpoint{2.268424in}{0.844576in}}{\pgfqpoint{2.271696in}{0.836676in}}{\pgfqpoint{2.277520in}{0.830852in}}%
\pgfpathcurveto{\pgfqpoint{2.283344in}{0.825028in}}{\pgfqpoint{2.291244in}{0.821756in}}{\pgfqpoint{2.299480in}{0.821756in}}%
\pgfpathclose%
\pgfusepath{stroke,fill}%
\end{pgfscope}%
\begin{pgfscope}%
\pgfpathrectangle{\pgfqpoint{0.100000in}{0.212622in}}{\pgfqpoint{3.696000in}{3.696000in}}%
\pgfusepath{clip}%
\pgfsetbuttcap%
\pgfsetroundjoin%
\definecolor{currentfill}{rgb}{0.121569,0.466667,0.705882}%
\pgfsetfillcolor{currentfill}%
\pgfsetfillopacity{0.974652}%
\pgfsetlinewidth{1.003750pt}%
\definecolor{currentstroke}{rgb}{0.121569,0.466667,0.705882}%
\pgfsetstrokecolor{currentstroke}%
\pgfsetstrokeopacity{0.974652}%
\pgfsetdash{}{0pt}%
\pgfpathmoveto{\pgfqpoint{2.428877in}{0.846494in}}%
\pgfpathcurveto{\pgfqpoint{2.437114in}{0.846494in}}{\pgfqpoint{2.445014in}{0.849766in}}{\pgfqpoint{2.450838in}{0.855590in}}%
\pgfpathcurveto{\pgfqpoint{2.456661in}{0.861414in}}{\pgfqpoint{2.459934in}{0.869314in}}{\pgfqpoint{2.459934in}{0.877550in}}%
\pgfpathcurveto{\pgfqpoint{2.459934in}{0.885786in}}{\pgfqpoint{2.456661in}{0.893687in}}{\pgfqpoint{2.450838in}{0.899510in}}%
\pgfpathcurveto{\pgfqpoint{2.445014in}{0.905334in}}{\pgfqpoint{2.437114in}{0.908607in}}{\pgfqpoint{2.428877in}{0.908607in}}%
\pgfpathcurveto{\pgfqpoint{2.420641in}{0.908607in}}{\pgfqpoint{2.412741in}{0.905334in}}{\pgfqpoint{2.406917in}{0.899510in}}%
\pgfpathcurveto{\pgfqpoint{2.401093in}{0.893687in}}{\pgfqpoint{2.397821in}{0.885786in}}{\pgfqpoint{2.397821in}{0.877550in}}%
\pgfpathcurveto{\pgfqpoint{2.397821in}{0.869314in}}{\pgfqpoint{2.401093in}{0.861414in}}{\pgfqpoint{2.406917in}{0.855590in}}%
\pgfpathcurveto{\pgfqpoint{2.412741in}{0.849766in}}{\pgfqpoint{2.420641in}{0.846494in}}{\pgfqpoint{2.428877in}{0.846494in}}%
\pgfpathclose%
\pgfusepath{stroke,fill}%
\end{pgfscope}%
\begin{pgfscope}%
\pgfpathrectangle{\pgfqpoint{0.100000in}{0.212622in}}{\pgfqpoint{3.696000in}{3.696000in}}%
\pgfusepath{clip}%
\pgfsetbuttcap%
\pgfsetroundjoin%
\definecolor{currentfill}{rgb}{0.121569,0.466667,0.705882}%
\pgfsetfillcolor{currentfill}%
\pgfsetfillopacity{0.975769}%
\pgfsetlinewidth{1.003750pt}%
\definecolor{currentstroke}{rgb}{0.121569,0.466667,0.705882}%
\pgfsetstrokecolor{currentstroke}%
\pgfsetstrokeopacity{0.975769}%
\pgfsetdash{}{0pt}%
\pgfpathmoveto{\pgfqpoint{2.305561in}{0.817739in}}%
\pgfpathcurveto{\pgfqpoint{2.313797in}{0.817739in}}{\pgfqpoint{2.321697in}{0.821012in}}{\pgfqpoint{2.327521in}{0.826835in}}%
\pgfpathcurveto{\pgfqpoint{2.333345in}{0.832659in}}{\pgfqpoint{2.336617in}{0.840559in}}{\pgfqpoint{2.336617in}{0.848796in}}%
\pgfpathcurveto{\pgfqpoint{2.336617in}{0.857032in}}{\pgfqpoint{2.333345in}{0.864932in}}{\pgfqpoint{2.327521in}{0.870756in}}%
\pgfpathcurveto{\pgfqpoint{2.321697in}{0.876580in}}{\pgfqpoint{2.313797in}{0.879852in}}{\pgfqpoint{2.305561in}{0.879852in}}%
\pgfpathcurveto{\pgfqpoint{2.297324in}{0.879852in}}{\pgfqpoint{2.289424in}{0.876580in}}{\pgfqpoint{2.283600in}{0.870756in}}%
\pgfpathcurveto{\pgfqpoint{2.277776in}{0.864932in}}{\pgfqpoint{2.274504in}{0.857032in}}{\pgfqpoint{2.274504in}{0.848796in}}%
\pgfpathcurveto{\pgfqpoint{2.274504in}{0.840559in}}{\pgfqpoint{2.277776in}{0.832659in}}{\pgfqpoint{2.283600in}{0.826835in}}%
\pgfpathcurveto{\pgfqpoint{2.289424in}{0.821012in}}{\pgfqpoint{2.297324in}{0.817739in}}{\pgfqpoint{2.305561in}{0.817739in}}%
\pgfpathclose%
\pgfusepath{stroke,fill}%
\end{pgfscope}%
\begin{pgfscope}%
\pgfpathrectangle{\pgfqpoint{0.100000in}{0.212622in}}{\pgfqpoint{3.696000in}{3.696000in}}%
\pgfusepath{clip}%
\pgfsetbuttcap%
\pgfsetroundjoin%
\definecolor{currentfill}{rgb}{0.121569,0.466667,0.705882}%
\pgfsetfillcolor{currentfill}%
\pgfsetfillopacity{0.976963}%
\pgfsetlinewidth{1.003750pt}%
\definecolor{currentstroke}{rgb}{0.121569,0.466667,0.705882}%
\pgfsetstrokecolor{currentstroke}%
\pgfsetstrokeopacity{0.976963}%
\pgfsetdash{}{0pt}%
\pgfpathmoveto{\pgfqpoint{2.311357in}{0.814137in}}%
\pgfpathcurveto{\pgfqpoint{2.319593in}{0.814137in}}{\pgfqpoint{2.327493in}{0.817410in}}{\pgfqpoint{2.333317in}{0.823234in}}%
\pgfpathcurveto{\pgfqpoint{2.339141in}{0.829058in}}{\pgfqpoint{2.342413in}{0.836958in}}{\pgfqpoint{2.342413in}{0.845194in}}%
\pgfpathcurveto{\pgfqpoint{2.342413in}{0.853430in}}{\pgfqpoint{2.339141in}{0.861330in}}{\pgfqpoint{2.333317in}{0.867154in}}%
\pgfpathcurveto{\pgfqpoint{2.327493in}{0.872978in}}{\pgfqpoint{2.319593in}{0.876250in}}{\pgfqpoint{2.311357in}{0.876250in}}%
\pgfpathcurveto{\pgfqpoint{2.303121in}{0.876250in}}{\pgfqpoint{2.295221in}{0.872978in}}{\pgfqpoint{2.289397in}{0.867154in}}%
\pgfpathcurveto{\pgfqpoint{2.283573in}{0.861330in}}{\pgfqpoint{2.280300in}{0.853430in}}{\pgfqpoint{2.280300in}{0.845194in}}%
\pgfpathcurveto{\pgfqpoint{2.280300in}{0.836958in}}{\pgfqpoint{2.283573in}{0.829058in}}{\pgfqpoint{2.289397in}{0.823234in}}%
\pgfpathcurveto{\pgfqpoint{2.295221in}{0.817410in}}{\pgfqpoint{2.303121in}{0.814137in}}{\pgfqpoint{2.311357in}{0.814137in}}%
\pgfpathclose%
\pgfusepath{stroke,fill}%
\end{pgfscope}%
\begin{pgfscope}%
\pgfpathrectangle{\pgfqpoint{0.100000in}{0.212622in}}{\pgfqpoint{3.696000in}{3.696000in}}%
\pgfusepath{clip}%
\pgfsetbuttcap%
\pgfsetroundjoin%
\definecolor{currentfill}{rgb}{0.121569,0.466667,0.705882}%
\pgfsetfillcolor{currentfill}%
\pgfsetfillopacity{0.978108}%
\pgfsetlinewidth{1.003750pt}%
\definecolor{currentstroke}{rgb}{0.121569,0.466667,0.705882}%
\pgfsetstrokecolor{currentstroke}%
\pgfsetstrokeopacity{0.978108}%
\pgfsetdash{}{0pt}%
\pgfpathmoveto{\pgfqpoint{2.316787in}{0.810749in}}%
\pgfpathcurveto{\pgfqpoint{2.325024in}{0.810749in}}{\pgfqpoint{2.332924in}{0.814021in}}{\pgfqpoint{2.338748in}{0.819845in}}%
\pgfpathcurveto{\pgfqpoint{2.344572in}{0.825669in}}{\pgfqpoint{2.347844in}{0.833569in}}{\pgfqpoint{2.347844in}{0.841805in}}%
\pgfpathcurveto{\pgfqpoint{2.347844in}{0.850041in}}{\pgfqpoint{2.344572in}{0.857941in}}{\pgfqpoint{2.338748in}{0.863765in}}%
\pgfpathcurveto{\pgfqpoint{2.332924in}{0.869589in}}{\pgfqpoint{2.325024in}{0.872862in}}{\pgfqpoint{2.316787in}{0.872862in}}%
\pgfpathcurveto{\pgfqpoint{2.308551in}{0.872862in}}{\pgfqpoint{2.300651in}{0.869589in}}{\pgfqpoint{2.294827in}{0.863765in}}%
\pgfpathcurveto{\pgfqpoint{2.289003in}{0.857941in}}{\pgfqpoint{2.285731in}{0.850041in}}{\pgfqpoint{2.285731in}{0.841805in}}%
\pgfpathcurveto{\pgfqpoint{2.285731in}{0.833569in}}{\pgfqpoint{2.289003in}{0.825669in}}{\pgfqpoint{2.294827in}{0.819845in}}%
\pgfpathcurveto{\pgfqpoint{2.300651in}{0.814021in}}{\pgfqpoint{2.308551in}{0.810749in}}{\pgfqpoint{2.316787in}{0.810749in}}%
\pgfpathclose%
\pgfusepath{stroke,fill}%
\end{pgfscope}%
\begin{pgfscope}%
\pgfpathrectangle{\pgfqpoint{0.100000in}{0.212622in}}{\pgfqpoint{3.696000in}{3.696000in}}%
\pgfusepath{clip}%
\pgfsetbuttcap%
\pgfsetroundjoin%
\definecolor{currentfill}{rgb}{0.121569,0.466667,0.705882}%
\pgfsetfillcolor{currentfill}%
\pgfsetfillopacity{0.979200}%
\pgfsetlinewidth{1.003750pt}%
\definecolor{currentstroke}{rgb}{0.121569,0.466667,0.705882}%
\pgfsetstrokecolor{currentstroke}%
\pgfsetstrokeopacity{0.979200}%
\pgfsetdash{}{0pt}%
\pgfpathmoveto{\pgfqpoint{2.321927in}{0.807758in}}%
\pgfpathcurveto{\pgfqpoint{2.330163in}{0.807758in}}{\pgfqpoint{2.338063in}{0.811030in}}{\pgfqpoint{2.343887in}{0.816854in}}%
\pgfpathcurveto{\pgfqpoint{2.349711in}{0.822678in}}{\pgfqpoint{2.352983in}{0.830578in}}{\pgfqpoint{2.352983in}{0.838814in}}%
\pgfpathcurveto{\pgfqpoint{2.352983in}{0.847051in}}{\pgfqpoint{2.349711in}{0.854951in}}{\pgfqpoint{2.343887in}{0.860775in}}%
\pgfpathcurveto{\pgfqpoint{2.338063in}{0.866599in}}{\pgfqpoint{2.330163in}{0.869871in}}{\pgfqpoint{2.321927in}{0.869871in}}%
\pgfpathcurveto{\pgfqpoint{2.313690in}{0.869871in}}{\pgfqpoint{2.305790in}{0.866599in}}{\pgfqpoint{2.299966in}{0.860775in}}%
\pgfpathcurveto{\pgfqpoint{2.294142in}{0.854951in}}{\pgfqpoint{2.290870in}{0.847051in}}{\pgfqpoint{2.290870in}{0.838814in}}%
\pgfpathcurveto{\pgfqpoint{2.290870in}{0.830578in}}{\pgfqpoint{2.294142in}{0.822678in}}{\pgfqpoint{2.299966in}{0.816854in}}%
\pgfpathcurveto{\pgfqpoint{2.305790in}{0.811030in}}{\pgfqpoint{2.313690in}{0.807758in}}{\pgfqpoint{2.321927in}{0.807758in}}%
\pgfpathclose%
\pgfusepath{stroke,fill}%
\end{pgfscope}%
\begin{pgfscope}%
\pgfpathrectangle{\pgfqpoint{0.100000in}{0.212622in}}{\pgfqpoint{3.696000in}{3.696000in}}%
\pgfusepath{clip}%
\pgfsetbuttcap%
\pgfsetroundjoin%
\definecolor{currentfill}{rgb}{0.121569,0.466667,0.705882}%
\pgfsetfillcolor{currentfill}%
\pgfsetfillopacity{0.980240}%
\pgfsetlinewidth{1.003750pt}%
\definecolor{currentstroke}{rgb}{0.121569,0.466667,0.705882}%
\pgfsetstrokecolor{currentstroke}%
\pgfsetstrokeopacity{0.980240}%
\pgfsetdash{}{0pt}%
\pgfpathmoveto{\pgfqpoint{2.326768in}{0.805114in}}%
\pgfpathcurveto{\pgfqpoint{2.335004in}{0.805114in}}{\pgfqpoint{2.342904in}{0.808386in}}{\pgfqpoint{2.348728in}{0.814210in}}%
\pgfpathcurveto{\pgfqpoint{2.354552in}{0.820034in}}{\pgfqpoint{2.357824in}{0.827934in}}{\pgfqpoint{2.357824in}{0.836170in}}%
\pgfpathcurveto{\pgfqpoint{2.357824in}{0.844407in}}{\pgfqpoint{2.354552in}{0.852307in}}{\pgfqpoint{2.348728in}{0.858131in}}%
\pgfpathcurveto{\pgfqpoint{2.342904in}{0.863955in}}{\pgfqpoint{2.335004in}{0.867227in}}{\pgfqpoint{2.326768in}{0.867227in}}%
\pgfpathcurveto{\pgfqpoint{2.318532in}{0.867227in}}{\pgfqpoint{2.310631in}{0.863955in}}{\pgfqpoint{2.304808in}{0.858131in}}%
\pgfpathcurveto{\pgfqpoint{2.298984in}{0.852307in}}{\pgfqpoint{2.295711in}{0.844407in}}{\pgfqpoint{2.295711in}{0.836170in}}%
\pgfpathcurveto{\pgfqpoint{2.295711in}{0.827934in}}{\pgfqpoint{2.298984in}{0.820034in}}{\pgfqpoint{2.304808in}{0.814210in}}%
\pgfpathcurveto{\pgfqpoint{2.310631in}{0.808386in}}{\pgfqpoint{2.318532in}{0.805114in}}{\pgfqpoint{2.326768in}{0.805114in}}%
\pgfpathclose%
\pgfusepath{stroke,fill}%
\end{pgfscope}%
\begin{pgfscope}%
\pgfpathrectangle{\pgfqpoint{0.100000in}{0.212622in}}{\pgfqpoint{3.696000in}{3.696000in}}%
\pgfusepath{clip}%
\pgfsetbuttcap%
\pgfsetroundjoin%
\definecolor{currentfill}{rgb}{0.121569,0.466667,0.705882}%
\pgfsetfillcolor{currentfill}%
\pgfsetfillopacity{0.981198}%
\pgfsetlinewidth{1.003750pt}%
\definecolor{currentstroke}{rgb}{0.121569,0.466667,0.705882}%
\pgfsetstrokecolor{currentstroke}%
\pgfsetstrokeopacity{0.981198}%
\pgfsetdash{}{0pt}%
\pgfpathmoveto{\pgfqpoint{2.331297in}{0.802743in}}%
\pgfpathcurveto{\pgfqpoint{2.339534in}{0.802743in}}{\pgfqpoint{2.347434in}{0.806015in}}{\pgfqpoint{2.353258in}{0.811839in}}%
\pgfpathcurveto{\pgfqpoint{2.359082in}{0.817663in}}{\pgfqpoint{2.362354in}{0.825563in}}{\pgfqpoint{2.362354in}{0.833799in}}%
\pgfpathcurveto{\pgfqpoint{2.362354in}{0.842035in}}{\pgfqpoint{2.359082in}{0.849936in}}{\pgfqpoint{2.353258in}{0.855759in}}%
\pgfpathcurveto{\pgfqpoint{2.347434in}{0.861583in}}{\pgfqpoint{2.339534in}{0.864856in}}{\pgfqpoint{2.331297in}{0.864856in}}%
\pgfpathcurveto{\pgfqpoint{2.323061in}{0.864856in}}{\pgfqpoint{2.315161in}{0.861583in}}{\pgfqpoint{2.309337in}{0.855759in}}%
\pgfpathcurveto{\pgfqpoint{2.303513in}{0.849936in}}{\pgfqpoint{2.300241in}{0.842035in}}{\pgfqpoint{2.300241in}{0.833799in}}%
\pgfpathcurveto{\pgfqpoint{2.300241in}{0.825563in}}{\pgfqpoint{2.303513in}{0.817663in}}{\pgfqpoint{2.309337in}{0.811839in}}%
\pgfpathcurveto{\pgfqpoint{2.315161in}{0.806015in}}{\pgfqpoint{2.323061in}{0.802743in}}{\pgfqpoint{2.331297in}{0.802743in}}%
\pgfpathclose%
\pgfusepath{stroke,fill}%
\end{pgfscope}%
\begin{pgfscope}%
\pgfpathrectangle{\pgfqpoint{0.100000in}{0.212622in}}{\pgfqpoint{3.696000in}{3.696000in}}%
\pgfusepath{clip}%
\pgfsetbuttcap%
\pgfsetroundjoin%
\definecolor{currentfill}{rgb}{0.121569,0.466667,0.705882}%
\pgfsetfillcolor{currentfill}%
\pgfsetfillopacity{0.981654}%
\pgfsetlinewidth{1.003750pt}%
\definecolor{currentstroke}{rgb}{0.121569,0.466667,0.705882}%
\pgfsetstrokecolor{currentstroke}%
\pgfsetstrokeopacity{0.981654}%
\pgfsetdash{}{0pt}%
\pgfpathmoveto{\pgfqpoint{2.435095in}{0.821830in}}%
\pgfpathcurveto{\pgfqpoint{2.443331in}{0.821830in}}{\pgfqpoint{2.451231in}{0.825103in}}{\pgfqpoint{2.457055in}{0.830926in}}%
\pgfpathcurveto{\pgfqpoint{2.462879in}{0.836750in}}{\pgfqpoint{2.466151in}{0.844650in}}{\pgfqpoint{2.466151in}{0.852887in}}%
\pgfpathcurveto{\pgfqpoint{2.466151in}{0.861123in}}{\pgfqpoint{2.462879in}{0.869023in}}{\pgfqpoint{2.457055in}{0.874847in}}%
\pgfpathcurveto{\pgfqpoint{2.451231in}{0.880671in}}{\pgfqpoint{2.443331in}{0.883943in}}{\pgfqpoint{2.435095in}{0.883943in}}%
\pgfpathcurveto{\pgfqpoint{2.426859in}{0.883943in}}{\pgfqpoint{2.418959in}{0.880671in}}{\pgfqpoint{2.413135in}{0.874847in}}%
\pgfpathcurveto{\pgfqpoint{2.407311in}{0.869023in}}{\pgfqpoint{2.404038in}{0.861123in}}{\pgfqpoint{2.404038in}{0.852887in}}%
\pgfpathcurveto{\pgfqpoint{2.404038in}{0.844650in}}{\pgfqpoint{2.407311in}{0.836750in}}{\pgfqpoint{2.413135in}{0.830926in}}%
\pgfpathcurveto{\pgfqpoint{2.418959in}{0.825103in}}{\pgfqpoint{2.426859in}{0.821830in}}{\pgfqpoint{2.435095in}{0.821830in}}%
\pgfpathclose%
\pgfusepath{stroke,fill}%
\end{pgfscope}%
\begin{pgfscope}%
\pgfpathrectangle{\pgfqpoint{0.100000in}{0.212622in}}{\pgfqpoint{3.696000in}{3.696000in}}%
\pgfusepath{clip}%
\pgfsetbuttcap%
\pgfsetroundjoin%
\definecolor{currentfill}{rgb}{0.121569,0.466667,0.705882}%
\pgfsetfillcolor{currentfill}%
\pgfsetfillopacity{0.982123}%
\pgfsetlinewidth{1.003750pt}%
\definecolor{currentstroke}{rgb}{0.121569,0.466667,0.705882}%
\pgfsetstrokecolor{currentstroke}%
\pgfsetstrokeopacity{0.982123}%
\pgfsetdash{}{0pt}%
\pgfpathmoveto{\pgfqpoint{2.335600in}{0.800662in}}%
\pgfpathcurveto{\pgfqpoint{2.343836in}{0.800662in}}{\pgfqpoint{2.351736in}{0.803934in}}{\pgfqpoint{2.357560in}{0.809758in}}%
\pgfpathcurveto{\pgfqpoint{2.363384in}{0.815582in}}{\pgfqpoint{2.366656in}{0.823482in}}{\pgfqpoint{2.366656in}{0.831718in}}%
\pgfpathcurveto{\pgfqpoint{2.366656in}{0.839955in}}{\pgfqpoint{2.363384in}{0.847855in}}{\pgfqpoint{2.357560in}{0.853679in}}%
\pgfpathcurveto{\pgfqpoint{2.351736in}{0.859503in}}{\pgfqpoint{2.343836in}{0.862775in}}{\pgfqpoint{2.335600in}{0.862775in}}%
\pgfpathcurveto{\pgfqpoint{2.327364in}{0.862775in}}{\pgfqpoint{2.319464in}{0.859503in}}{\pgfqpoint{2.313640in}{0.853679in}}%
\pgfpathcurveto{\pgfqpoint{2.307816in}{0.847855in}}{\pgfqpoint{2.304543in}{0.839955in}}{\pgfqpoint{2.304543in}{0.831718in}}%
\pgfpathcurveto{\pgfqpoint{2.304543in}{0.823482in}}{\pgfqpoint{2.307816in}{0.815582in}}{\pgfqpoint{2.313640in}{0.809758in}}%
\pgfpathcurveto{\pgfqpoint{2.319464in}{0.803934in}}{\pgfqpoint{2.327364in}{0.800662in}}{\pgfqpoint{2.335600in}{0.800662in}}%
\pgfpathclose%
\pgfusepath{stroke,fill}%
\end{pgfscope}%
\begin{pgfscope}%
\pgfpathrectangle{\pgfqpoint{0.100000in}{0.212622in}}{\pgfqpoint{3.696000in}{3.696000in}}%
\pgfusepath{clip}%
\pgfsetbuttcap%
\pgfsetroundjoin%
\definecolor{currentfill}{rgb}{0.121569,0.466667,0.705882}%
\pgfsetfillcolor{currentfill}%
\pgfsetfillopacity{0.982981}%
\pgfsetlinewidth{1.003750pt}%
\definecolor{currentstroke}{rgb}{0.121569,0.466667,0.705882}%
\pgfsetstrokecolor{currentstroke}%
\pgfsetstrokeopacity{0.982981}%
\pgfsetdash{}{0pt}%
\pgfpathmoveto{\pgfqpoint{2.339511in}{0.798929in}}%
\pgfpathcurveto{\pgfqpoint{2.347747in}{0.798929in}}{\pgfqpoint{2.355647in}{0.802202in}}{\pgfqpoint{2.361471in}{0.808025in}}%
\pgfpathcurveto{\pgfqpoint{2.367295in}{0.813849in}}{\pgfqpoint{2.370568in}{0.821749in}}{\pgfqpoint{2.370568in}{0.829986in}}%
\pgfpathcurveto{\pgfqpoint{2.370568in}{0.838222in}}{\pgfqpoint{2.367295in}{0.846122in}}{\pgfqpoint{2.361471in}{0.851946in}}%
\pgfpathcurveto{\pgfqpoint{2.355647in}{0.857770in}}{\pgfqpoint{2.347747in}{0.861042in}}{\pgfqpoint{2.339511in}{0.861042in}}%
\pgfpathcurveto{\pgfqpoint{2.331275in}{0.861042in}}{\pgfqpoint{2.323375in}{0.857770in}}{\pgfqpoint{2.317551in}{0.851946in}}%
\pgfpathcurveto{\pgfqpoint{2.311727in}{0.846122in}}{\pgfqpoint{2.308455in}{0.838222in}}{\pgfqpoint{2.308455in}{0.829986in}}%
\pgfpathcurveto{\pgfqpoint{2.308455in}{0.821749in}}{\pgfqpoint{2.311727in}{0.813849in}}{\pgfqpoint{2.317551in}{0.808025in}}%
\pgfpathcurveto{\pgfqpoint{2.323375in}{0.802202in}}{\pgfqpoint{2.331275in}{0.798929in}}{\pgfqpoint{2.339511in}{0.798929in}}%
\pgfpathclose%
\pgfusepath{stroke,fill}%
\end{pgfscope}%
\begin{pgfscope}%
\pgfpathrectangle{\pgfqpoint{0.100000in}{0.212622in}}{\pgfqpoint{3.696000in}{3.696000in}}%
\pgfusepath{clip}%
\pgfsetbuttcap%
\pgfsetroundjoin%
\definecolor{currentfill}{rgb}{0.121569,0.466667,0.705882}%
\pgfsetfillcolor{currentfill}%
\pgfsetfillopacity{0.984549}%
\pgfsetlinewidth{1.003750pt}%
\definecolor{currentstroke}{rgb}{0.121569,0.466667,0.705882}%
\pgfsetstrokecolor{currentstroke}%
\pgfsetstrokeopacity{0.984549}%
\pgfsetdash{}{0pt}%
\pgfpathmoveto{\pgfqpoint{2.346628in}{0.795816in}}%
\pgfpathcurveto{\pgfqpoint{2.354864in}{0.795816in}}{\pgfqpoint{2.362765in}{0.799088in}}{\pgfqpoint{2.368588in}{0.804912in}}%
\pgfpathcurveto{\pgfqpoint{2.374412in}{0.810736in}}{\pgfqpoint{2.377685in}{0.818636in}}{\pgfqpoint{2.377685in}{0.826872in}}%
\pgfpathcurveto{\pgfqpoint{2.377685in}{0.835109in}}{\pgfqpoint{2.374412in}{0.843009in}}{\pgfqpoint{2.368588in}{0.848833in}}%
\pgfpathcurveto{\pgfqpoint{2.362765in}{0.854656in}}{\pgfqpoint{2.354864in}{0.857929in}}{\pgfqpoint{2.346628in}{0.857929in}}%
\pgfpathcurveto{\pgfqpoint{2.338392in}{0.857929in}}{\pgfqpoint{2.330492in}{0.854656in}}{\pgfqpoint{2.324668in}{0.848833in}}%
\pgfpathcurveto{\pgfqpoint{2.318844in}{0.843009in}}{\pgfqpoint{2.315572in}{0.835109in}}{\pgfqpoint{2.315572in}{0.826872in}}%
\pgfpathcurveto{\pgfqpoint{2.315572in}{0.818636in}}{\pgfqpoint{2.318844in}{0.810736in}}{\pgfqpoint{2.324668in}{0.804912in}}%
\pgfpathcurveto{\pgfqpoint{2.330492in}{0.799088in}}{\pgfqpoint{2.338392in}{0.795816in}}{\pgfqpoint{2.346628in}{0.795816in}}%
\pgfpathclose%
\pgfusepath{stroke,fill}%
\end{pgfscope}%
\begin{pgfscope}%
\pgfpathrectangle{\pgfqpoint{0.100000in}{0.212622in}}{\pgfqpoint{3.696000in}{3.696000in}}%
\pgfusepath{clip}%
\pgfsetbuttcap%
\pgfsetroundjoin%
\definecolor{currentfill}{rgb}{0.121569,0.466667,0.705882}%
\pgfsetfillcolor{currentfill}%
\pgfsetfillopacity{0.985914}%
\pgfsetlinewidth{1.003750pt}%
\definecolor{currentstroke}{rgb}{0.121569,0.466667,0.705882}%
\pgfsetstrokecolor{currentstroke}%
\pgfsetstrokeopacity{0.985914}%
\pgfsetdash{}{0pt}%
\pgfpathmoveto{\pgfqpoint{2.353496in}{0.792860in}}%
\pgfpathcurveto{\pgfqpoint{2.361732in}{0.792860in}}{\pgfqpoint{2.369632in}{0.796132in}}{\pgfqpoint{2.375456in}{0.801956in}}%
\pgfpathcurveto{\pgfqpoint{2.381280in}{0.807780in}}{\pgfqpoint{2.384552in}{0.815680in}}{\pgfqpoint{2.384552in}{0.823916in}}%
\pgfpathcurveto{\pgfqpoint{2.384552in}{0.832153in}}{\pgfqpoint{2.381280in}{0.840053in}}{\pgfqpoint{2.375456in}{0.845877in}}%
\pgfpathcurveto{\pgfqpoint{2.369632in}{0.851700in}}{\pgfqpoint{2.361732in}{0.854973in}}{\pgfqpoint{2.353496in}{0.854973in}}%
\pgfpathcurveto{\pgfqpoint{2.345259in}{0.854973in}}{\pgfqpoint{2.337359in}{0.851700in}}{\pgfqpoint{2.331535in}{0.845877in}}%
\pgfpathcurveto{\pgfqpoint{2.325711in}{0.840053in}}{\pgfqpoint{2.322439in}{0.832153in}}{\pgfqpoint{2.322439in}{0.823916in}}%
\pgfpathcurveto{\pgfqpoint{2.322439in}{0.815680in}}{\pgfqpoint{2.325711in}{0.807780in}}{\pgfqpoint{2.331535in}{0.801956in}}%
\pgfpathcurveto{\pgfqpoint{2.337359in}{0.796132in}}{\pgfqpoint{2.345259in}{0.792860in}}{\pgfqpoint{2.353496in}{0.792860in}}%
\pgfpathclose%
\pgfusepath{stroke,fill}%
\end{pgfscope}%
\begin{pgfscope}%
\pgfpathrectangle{\pgfqpoint{0.100000in}{0.212622in}}{\pgfqpoint{3.696000in}{3.696000in}}%
\pgfusepath{clip}%
\pgfsetbuttcap%
\pgfsetroundjoin%
\definecolor{currentfill}{rgb}{0.121569,0.466667,0.705882}%
\pgfsetfillcolor{currentfill}%
\pgfsetfillopacity{0.987254}%
\pgfsetlinewidth{1.003750pt}%
\definecolor{currentstroke}{rgb}{0.121569,0.466667,0.705882}%
\pgfsetstrokecolor{currentstroke}%
\pgfsetstrokeopacity{0.987254}%
\pgfsetdash{}{0pt}%
\pgfpathmoveto{\pgfqpoint{2.360062in}{0.790242in}}%
\pgfpathcurveto{\pgfqpoint{2.368299in}{0.790242in}}{\pgfqpoint{2.376199in}{0.793514in}}{\pgfqpoint{2.382023in}{0.799338in}}%
\pgfpathcurveto{\pgfqpoint{2.387847in}{0.805162in}}{\pgfqpoint{2.391119in}{0.813062in}}{\pgfqpoint{2.391119in}{0.821299in}}%
\pgfpathcurveto{\pgfqpoint{2.391119in}{0.829535in}}{\pgfqpoint{2.387847in}{0.837435in}}{\pgfqpoint{2.382023in}{0.843259in}}%
\pgfpathcurveto{\pgfqpoint{2.376199in}{0.849083in}}{\pgfqpoint{2.368299in}{0.852355in}}{\pgfqpoint{2.360062in}{0.852355in}}%
\pgfpathcurveto{\pgfqpoint{2.351826in}{0.852355in}}{\pgfqpoint{2.343926in}{0.849083in}}{\pgfqpoint{2.338102in}{0.843259in}}%
\pgfpathcurveto{\pgfqpoint{2.332278in}{0.837435in}}{\pgfqpoint{2.329006in}{0.829535in}}{\pgfqpoint{2.329006in}{0.821299in}}%
\pgfpathcurveto{\pgfqpoint{2.329006in}{0.813062in}}{\pgfqpoint{2.332278in}{0.805162in}}{\pgfqpoint{2.338102in}{0.799338in}}%
\pgfpathcurveto{\pgfqpoint{2.343926in}{0.793514in}}{\pgfqpoint{2.351826in}{0.790242in}}{\pgfqpoint{2.360062in}{0.790242in}}%
\pgfpathclose%
\pgfusepath{stroke,fill}%
\end{pgfscope}%
\begin{pgfscope}%
\pgfpathrectangle{\pgfqpoint{0.100000in}{0.212622in}}{\pgfqpoint{3.696000in}{3.696000in}}%
\pgfusepath{clip}%
\pgfsetbuttcap%
\pgfsetroundjoin%
\definecolor{currentfill}{rgb}{0.121569,0.466667,0.705882}%
\pgfsetfillcolor{currentfill}%
\pgfsetfillopacity{0.988545}%
\pgfsetlinewidth{1.003750pt}%
\definecolor{currentstroke}{rgb}{0.121569,0.466667,0.705882}%
\pgfsetstrokecolor{currentstroke}%
\pgfsetstrokeopacity{0.988545}%
\pgfsetdash{}{0pt}%
\pgfpathmoveto{\pgfqpoint{2.366404in}{0.787824in}}%
\pgfpathcurveto{\pgfqpoint{2.374640in}{0.787824in}}{\pgfqpoint{2.382540in}{0.791096in}}{\pgfqpoint{2.388364in}{0.796920in}}%
\pgfpathcurveto{\pgfqpoint{2.394188in}{0.802744in}}{\pgfqpoint{2.397460in}{0.810644in}}{\pgfqpoint{2.397460in}{0.818881in}}%
\pgfpathcurveto{\pgfqpoint{2.397460in}{0.827117in}}{\pgfqpoint{2.394188in}{0.835017in}}{\pgfqpoint{2.388364in}{0.840841in}}%
\pgfpathcurveto{\pgfqpoint{2.382540in}{0.846665in}}{\pgfqpoint{2.374640in}{0.849937in}}{\pgfqpoint{2.366404in}{0.849937in}}%
\pgfpathcurveto{\pgfqpoint{2.358167in}{0.849937in}}{\pgfqpoint{2.350267in}{0.846665in}}{\pgfqpoint{2.344443in}{0.840841in}}%
\pgfpathcurveto{\pgfqpoint{2.338619in}{0.835017in}}{\pgfqpoint{2.335347in}{0.827117in}}{\pgfqpoint{2.335347in}{0.818881in}}%
\pgfpathcurveto{\pgfqpoint{2.335347in}{0.810644in}}{\pgfqpoint{2.338619in}{0.802744in}}{\pgfqpoint{2.344443in}{0.796920in}}%
\pgfpathcurveto{\pgfqpoint{2.350267in}{0.791096in}}{\pgfqpoint{2.358167in}{0.787824in}}{\pgfqpoint{2.366404in}{0.787824in}}%
\pgfpathclose%
\pgfusepath{stroke,fill}%
\end{pgfscope}%
\begin{pgfscope}%
\pgfpathrectangle{\pgfqpoint{0.100000in}{0.212622in}}{\pgfqpoint{3.696000in}{3.696000in}}%
\pgfusepath{clip}%
\pgfsetbuttcap%
\pgfsetroundjoin%
\definecolor{currentfill}{rgb}{0.121569,0.466667,0.705882}%
\pgfsetfillcolor{currentfill}%
\pgfsetfillopacity{0.988943}%
\pgfsetlinewidth{1.003750pt}%
\definecolor{currentstroke}{rgb}{0.121569,0.466667,0.705882}%
\pgfsetstrokecolor{currentstroke}%
\pgfsetstrokeopacity{0.988943}%
\pgfsetdash{}{0pt}%
\pgfpathmoveto{\pgfqpoint{2.441306in}{0.796156in}}%
\pgfpathcurveto{\pgfqpoint{2.449543in}{0.796156in}}{\pgfqpoint{2.457443in}{0.799428in}}{\pgfqpoint{2.463267in}{0.805252in}}%
\pgfpathcurveto{\pgfqpoint{2.469091in}{0.811076in}}{\pgfqpoint{2.472363in}{0.818976in}}{\pgfqpoint{2.472363in}{0.827212in}}%
\pgfpathcurveto{\pgfqpoint{2.472363in}{0.835448in}}{\pgfqpoint{2.469091in}{0.843348in}}{\pgfqpoint{2.463267in}{0.849172in}}%
\pgfpathcurveto{\pgfqpoint{2.457443in}{0.854996in}}{\pgfqpoint{2.449543in}{0.858269in}}{\pgfqpoint{2.441306in}{0.858269in}}%
\pgfpathcurveto{\pgfqpoint{2.433070in}{0.858269in}}{\pgfqpoint{2.425170in}{0.854996in}}{\pgfqpoint{2.419346in}{0.849172in}}%
\pgfpathcurveto{\pgfqpoint{2.413522in}{0.843348in}}{\pgfqpoint{2.410250in}{0.835448in}}{\pgfqpoint{2.410250in}{0.827212in}}%
\pgfpathcurveto{\pgfqpoint{2.410250in}{0.818976in}}{\pgfqpoint{2.413522in}{0.811076in}}{\pgfqpoint{2.419346in}{0.805252in}}%
\pgfpathcurveto{\pgfqpoint{2.425170in}{0.799428in}}{\pgfqpoint{2.433070in}{0.796156in}}{\pgfqpoint{2.441306in}{0.796156in}}%
\pgfpathclose%
\pgfusepath{stroke,fill}%
\end{pgfscope}%
\begin{pgfscope}%
\pgfpathrectangle{\pgfqpoint{0.100000in}{0.212622in}}{\pgfqpoint{3.696000in}{3.696000in}}%
\pgfusepath{clip}%
\pgfsetbuttcap%
\pgfsetroundjoin%
\definecolor{currentfill}{rgb}{0.121569,0.466667,0.705882}%
\pgfsetfillcolor{currentfill}%
\pgfsetfillopacity{0.989793}%
\pgfsetlinewidth{1.003750pt}%
\definecolor{currentstroke}{rgb}{0.121569,0.466667,0.705882}%
\pgfsetstrokecolor{currentstroke}%
\pgfsetstrokeopacity{0.989793}%
\pgfsetdash{}{0pt}%
\pgfpathmoveto{\pgfqpoint{2.372467in}{0.785488in}}%
\pgfpathcurveto{\pgfqpoint{2.380703in}{0.785488in}}{\pgfqpoint{2.388603in}{0.788760in}}{\pgfqpoint{2.394427in}{0.794584in}}%
\pgfpathcurveto{\pgfqpoint{2.400251in}{0.800408in}}{\pgfqpoint{2.403523in}{0.808308in}}{\pgfqpoint{2.403523in}{0.816545in}}%
\pgfpathcurveto{\pgfqpoint{2.403523in}{0.824781in}}{\pgfqpoint{2.400251in}{0.832681in}}{\pgfqpoint{2.394427in}{0.838505in}}%
\pgfpathcurveto{\pgfqpoint{2.388603in}{0.844329in}}{\pgfqpoint{2.380703in}{0.847601in}}{\pgfqpoint{2.372467in}{0.847601in}}%
\pgfpathcurveto{\pgfqpoint{2.364231in}{0.847601in}}{\pgfqpoint{2.356331in}{0.844329in}}{\pgfqpoint{2.350507in}{0.838505in}}%
\pgfpathcurveto{\pgfqpoint{2.344683in}{0.832681in}}{\pgfqpoint{2.341410in}{0.824781in}}{\pgfqpoint{2.341410in}{0.816545in}}%
\pgfpathcurveto{\pgfqpoint{2.341410in}{0.808308in}}{\pgfqpoint{2.344683in}{0.800408in}}{\pgfqpoint{2.350507in}{0.794584in}}%
\pgfpathcurveto{\pgfqpoint{2.356331in}{0.788760in}}{\pgfqpoint{2.364231in}{0.785488in}}{\pgfqpoint{2.372467in}{0.785488in}}%
\pgfpathclose%
\pgfusepath{stroke,fill}%
\end{pgfscope}%
\begin{pgfscope}%
\pgfpathrectangle{\pgfqpoint{0.100000in}{0.212622in}}{\pgfqpoint{3.696000in}{3.696000in}}%
\pgfusepath{clip}%
\pgfsetbuttcap%
\pgfsetroundjoin%
\definecolor{currentfill}{rgb}{0.121569,0.466667,0.705882}%
\pgfsetfillcolor{currentfill}%
\pgfsetfillopacity{0.990948}%
\pgfsetlinewidth{1.003750pt}%
\definecolor{currentstroke}{rgb}{0.121569,0.466667,0.705882}%
\pgfsetstrokecolor{currentstroke}%
\pgfsetstrokeopacity{0.990948}%
\pgfsetdash{}{0pt}%
\pgfpathmoveto{\pgfqpoint{2.378288in}{0.783315in}}%
\pgfpathcurveto{\pgfqpoint{2.386524in}{0.783315in}}{\pgfqpoint{2.394424in}{0.786588in}}{\pgfqpoint{2.400248in}{0.792412in}}%
\pgfpathcurveto{\pgfqpoint{2.406072in}{0.798235in}}{\pgfqpoint{2.409344in}{0.806135in}}{\pgfqpoint{2.409344in}{0.814372in}}%
\pgfpathcurveto{\pgfqpoint{2.409344in}{0.822608in}}{\pgfqpoint{2.406072in}{0.830508in}}{\pgfqpoint{2.400248in}{0.836332in}}%
\pgfpathcurveto{\pgfqpoint{2.394424in}{0.842156in}}{\pgfqpoint{2.386524in}{0.845428in}}{\pgfqpoint{2.378288in}{0.845428in}}%
\pgfpathcurveto{\pgfqpoint{2.370052in}{0.845428in}}{\pgfqpoint{2.362152in}{0.842156in}}{\pgfqpoint{2.356328in}{0.836332in}}%
\pgfpathcurveto{\pgfqpoint{2.350504in}{0.830508in}}{\pgfqpoint{2.347231in}{0.822608in}}{\pgfqpoint{2.347231in}{0.814372in}}%
\pgfpathcurveto{\pgfqpoint{2.347231in}{0.806135in}}{\pgfqpoint{2.350504in}{0.798235in}}{\pgfqpoint{2.356328in}{0.792412in}}%
\pgfpathcurveto{\pgfqpoint{2.362152in}{0.786588in}}{\pgfqpoint{2.370052in}{0.783315in}}{\pgfqpoint{2.378288in}{0.783315in}}%
\pgfpathclose%
\pgfusepath{stroke,fill}%
\end{pgfscope}%
\begin{pgfscope}%
\pgfpathrectangle{\pgfqpoint{0.100000in}{0.212622in}}{\pgfqpoint{3.696000in}{3.696000in}}%
\pgfusepath{clip}%
\pgfsetbuttcap%
\pgfsetroundjoin%
\definecolor{currentfill}{rgb}{0.121569,0.466667,0.705882}%
\pgfsetfillcolor{currentfill}%
\pgfsetfillopacity{0.991992}%
\pgfsetlinewidth{1.003750pt}%
\definecolor{currentstroke}{rgb}{0.121569,0.466667,0.705882}%
\pgfsetstrokecolor{currentstroke}%
\pgfsetstrokeopacity{0.991992}%
\pgfsetdash{}{0pt}%
\pgfpathmoveto{\pgfqpoint{2.383846in}{0.781280in}}%
\pgfpathcurveto{\pgfqpoint{2.392083in}{0.781280in}}{\pgfqpoint{2.399983in}{0.784552in}}{\pgfqpoint{2.405807in}{0.790376in}}%
\pgfpathcurveto{\pgfqpoint{2.411630in}{0.796200in}}{\pgfqpoint{2.414903in}{0.804100in}}{\pgfqpoint{2.414903in}{0.812337in}}%
\pgfpathcurveto{\pgfqpoint{2.414903in}{0.820573in}}{\pgfqpoint{2.411630in}{0.828473in}}{\pgfqpoint{2.405807in}{0.834297in}}%
\pgfpathcurveto{\pgfqpoint{2.399983in}{0.840121in}}{\pgfqpoint{2.392083in}{0.843393in}}{\pgfqpoint{2.383846in}{0.843393in}}%
\pgfpathcurveto{\pgfqpoint{2.375610in}{0.843393in}}{\pgfqpoint{2.367710in}{0.840121in}}{\pgfqpoint{2.361886in}{0.834297in}}%
\pgfpathcurveto{\pgfqpoint{2.356062in}{0.828473in}}{\pgfqpoint{2.352790in}{0.820573in}}{\pgfqpoint{2.352790in}{0.812337in}}%
\pgfpathcurveto{\pgfqpoint{2.352790in}{0.804100in}}{\pgfqpoint{2.356062in}{0.796200in}}{\pgfqpoint{2.361886in}{0.790376in}}%
\pgfpathcurveto{\pgfqpoint{2.367710in}{0.784552in}}{\pgfqpoint{2.375610in}{0.781280in}}{\pgfqpoint{2.383846in}{0.781280in}}%
\pgfpathclose%
\pgfusepath{stroke,fill}%
\end{pgfscope}%
\begin{pgfscope}%
\pgfpathrectangle{\pgfqpoint{0.100000in}{0.212622in}}{\pgfqpoint{3.696000in}{3.696000in}}%
\pgfusepath{clip}%
\pgfsetbuttcap%
\pgfsetroundjoin%
\definecolor{currentfill}{rgb}{0.121569,0.466667,0.705882}%
\pgfsetfillcolor{currentfill}%
\pgfsetfillopacity{0.992814}%
\pgfsetlinewidth{1.003750pt}%
\definecolor{currentstroke}{rgb}{0.121569,0.466667,0.705882}%
\pgfsetstrokecolor{currentstroke}%
\pgfsetstrokeopacity{0.992814}%
\pgfsetdash{}{0pt}%
\pgfpathmoveto{\pgfqpoint{2.444547in}{0.781342in}}%
\pgfpathcurveto{\pgfqpoint{2.452784in}{0.781342in}}{\pgfqpoint{2.460684in}{0.784614in}}{\pgfqpoint{2.466508in}{0.790438in}}%
\pgfpathcurveto{\pgfqpoint{2.472332in}{0.796262in}}{\pgfqpoint{2.475604in}{0.804162in}}{\pgfqpoint{2.475604in}{0.812399in}}%
\pgfpathcurveto{\pgfqpoint{2.475604in}{0.820635in}}{\pgfqpoint{2.472332in}{0.828535in}}{\pgfqpoint{2.466508in}{0.834359in}}%
\pgfpathcurveto{\pgfqpoint{2.460684in}{0.840183in}}{\pgfqpoint{2.452784in}{0.843455in}}{\pgfqpoint{2.444547in}{0.843455in}}%
\pgfpathcurveto{\pgfqpoint{2.436311in}{0.843455in}}{\pgfqpoint{2.428411in}{0.840183in}}{\pgfqpoint{2.422587in}{0.834359in}}%
\pgfpathcurveto{\pgfqpoint{2.416763in}{0.828535in}}{\pgfqpoint{2.413491in}{0.820635in}}{\pgfqpoint{2.413491in}{0.812399in}}%
\pgfpathcurveto{\pgfqpoint{2.413491in}{0.804162in}}{\pgfqpoint{2.416763in}{0.796262in}}{\pgfqpoint{2.422587in}{0.790438in}}%
\pgfpathcurveto{\pgfqpoint{2.428411in}{0.784614in}}{\pgfqpoint{2.436311in}{0.781342in}}{\pgfqpoint{2.444547in}{0.781342in}}%
\pgfpathclose%
\pgfusepath{stroke,fill}%
\end{pgfscope}%
\begin{pgfscope}%
\pgfpathrectangle{\pgfqpoint{0.100000in}{0.212622in}}{\pgfqpoint{3.696000in}{3.696000in}}%
\pgfusepath{clip}%
\pgfsetbuttcap%
\pgfsetroundjoin%
\definecolor{currentfill}{rgb}{0.121569,0.466667,0.705882}%
\pgfsetfillcolor{currentfill}%
\pgfsetfillopacity{0.992987}%
\pgfsetlinewidth{1.003750pt}%
\definecolor{currentstroke}{rgb}{0.121569,0.466667,0.705882}%
\pgfsetstrokecolor{currentstroke}%
\pgfsetstrokeopacity{0.992987}%
\pgfsetdash{}{0pt}%
\pgfpathmoveto{\pgfqpoint{2.389050in}{0.779213in}}%
\pgfpathcurveto{\pgfqpoint{2.397286in}{0.779213in}}{\pgfqpoint{2.405186in}{0.782485in}}{\pgfqpoint{2.411010in}{0.788309in}}%
\pgfpathcurveto{\pgfqpoint{2.416834in}{0.794133in}}{\pgfqpoint{2.420106in}{0.802033in}}{\pgfqpoint{2.420106in}{0.810269in}}%
\pgfpathcurveto{\pgfqpoint{2.420106in}{0.818505in}}{\pgfqpoint{2.416834in}{0.826405in}}{\pgfqpoint{2.411010in}{0.832229in}}%
\pgfpathcurveto{\pgfqpoint{2.405186in}{0.838053in}}{\pgfqpoint{2.397286in}{0.841326in}}{\pgfqpoint{2.389050in}{0.841326in}}%
\pgfpathcurveto{\pgfqpoint{2.380814in}{0.841326in}}{\pgfqpoint{2.372914in}{0.838053in}}{\pgfqpoint{2.367090in}{0.832229in}}%
\pgfpathcurveto{\pgfqpoint{2.361266in}{0.826405in}}{\pgfqpoint{2.357993in}{0.818505in}}{\pgfqpoint{2.357993in}{0.810269in}}%
\pgfpathcurveto{\pgfqpoint{2.357993in}{0.802033in}}{\pgfqpoint{2.361266in}{0.794133in}}{\pgfqpoint{2.367090in}{0.788309in}}%
\pgfpathcurveto{\pgfqpoint{2.372914in}{0.782485in}}{\pgfqpoint{2.380814in}{0.779213in}}{\pgfqpoint{2.389050in}{0.779213in}}%
\pgfpathclose%
\pgfusepath{stroke,fill}%
\end{pgfscope}%
\begin{pgfscope}%
\pgfpathrectangle{\pgfqpoint{0.100000in}{0.212622in}}{\pgfqpoint{3.696000in}{3.696000in}}%
\pgfusepath{clip}%
\pgfsetbuttcap%
\pgfsetroundjoin%
\definecolor{currentfill}{rgb}{0.121569,0.466667,0.705882}%
\pgfsetfillcolor{currentfill}%
\pgfsetfillopacity{0.993893}%
\pgfsetlinewidth{1.003750pt}%
\definecolor{currentstroke}{rgb}{0.121569,0.466667,0.705882}%
\pgfsetstrokecolor{currentstroke}%
\pgfsetstrokeopacity{0.993893}%
\pgfsetdash{}{0pt}%
\pgfpathmoveto{\pgfqpoint{2.393901in}{0.777161in}}%
\pgfpathcurveto{\pgfqpoint{2.402137in}{0.777161in}}{\pgfqpoint{2.410037in}{0.780433in}}{\pgfqpoint{2.415861in}{0.786257in}}%
\pgfpathcurveto{\pgfqpoint{2.421685in}{0.792081in}}{\pgfqpoint{2.424957in}{0.799981in}}{\pgfqpoint{2.424957in}{0.808218in}}%
\pgfpathcurveto{\pgfqpoint{2.424957in}{0.816454in}}{\pgfqpoint{2.421685in}{0.824354in}}{\pgfqpoint{2.415861in}{0.830178in}}%
\pgfpathcurveto{\pgfqpoint{2.410037in}{0.836002in}}{\pgfqpoint{2.402137in}{0.839274in}}{\pgfqpoint{2.393901in}{0.839274in}}%
\pgfpathcurveto{\pgfqpoint{2.385665in}{0.839274in}}{\pgfqpoint{2.377764in}{0.836002in}}{\pgfqpoint{2.371941in}{0.830178in}}%
\pgfpathcurveto{\pgfqpoint{2.366117in}{0.824354in}}{\pgfqpoint{2.362844in}{0.816454in}}{\pgfqpoint{2.362844in}{0.808218in}}%
\pgfpathcurveto{\pgfqpoint{2.362844in}{0.799981in}}{\pgfqpoint{2.366117in}{0.792081in}}{\pgfqpoint{2.371941in}{0.786257in}}%
\pgfpathcurveto{\pgfqpoint{2.377764in}{0.780433in}}{\pgfqpoint{2.385665in}{0.777161in}}{\pgfqpoint{2.393901in}{0.777161in}}%
\pgfpathclose%
\pgfusepath{stroke,fill}%
\end{pgfscope}%
\begin{pgfscope}%
\pgfpathrectangle{\pgfqpoint{0.100000in}{0.212622in}}{\pgfqpoint{3.696000in}{3.696000in}}%
\pgfusepath{clip}%
\pgfsetbuttcap%
\pgfsetroundjoin%
\definecolor{currentfill}{rgb}{0.121569,0.466667,0.705882}%
\pgfsetfillcolor{currentfill}%
\pgfsetfillopacity{0.994728}%
\pgfsetlinewidth{1.003750pt}%
\definecolor{currentstroke}{rgb}{0.121569,0.466667,0.705882}%
\pgfsetstrokecolor{currentstroke}%
\pgfsetstrokeopacity{0.994728}%
\pgfsetdash{}{0pt}%
\pgfpathmoveto{\pgfqpoint{2.398435in}{0.775152in}}%
\pgfpathcurveto{\pgfqpoint{2.406671in}{0.775152in}}{\pgfqpoint{2.414571in}{0.778424in}}{\pgfqpoint{2.420395in}{0.784248in}}%
\pgfpathcurveto{\pgfqpoint{2.426219in}{0.790072in}}{\pgfqpoint{2.429492in}{0.797972in}}{\pgfqpoint{2.429492in}{0.806209in}}%
\pgfpathcurveto{\pgfqpoint{2.429492in}{0.814445in}}{\pgfqpoint{2.426219in}{0.822345in}}{\pgfqpoint{2.420395in}{0.828169in}}%
\pgfpathcurveto{\pgfqpoint{2.414571in}{0.833993in}}{\pgfqpoint{2.406671in}{0.837265in}}{\pgfqpoint{2.398435in}{0.837265in}}%
\pgfpathcurveto{\pgfqpoint{2.390199in}{0.837265in}}{\pgfqpoint{2.382299in}{0.833993in}}{\pgfqpoint{2.376475in}{0.828169in}}%
\pgfpathcurveto{\pgfqpoint{2.370651in}{0.822345in}}{\pgfqpoint{2.367379in}{0.814445in}}{\pgfqpoint{2.367379in}{0.806209in}}%
\pgfpathcurveto{\pgfqpoint{2.367379in}{0.797972in}}{\pgfqpoint{2.370651in}{0.790072in}}{\pgfqpoint{2.376475in}{0.784248in}}%
\pgfpathcurveto{\pgfqpoint{2.382299in}{0.778424in}}{\pgfqpoint{2.390199in}{0.775152in}}{\pgfqpoint{2.398435in}{0.775152in}}%
\pgfpathclose%
\pgfusepath{stroke,fill}%
\end{pgfscope}%
\begin{pgfscope}%
\pgfpathrectangle{\pgfqpoint{0.100000in}{0.212622in}}{\pgfqpoint{3.696000in}{3.696000in}}%
\pgfusepath{clip}%
\pgfsetbuttcap%
\pgfsetroundjoin%
\definecolor{currentfill}{rgb}{0.121569,0.466667,0.705882}%
\pgfsetfillcolor{currentfill}%
\pgfsetfillopacity{0.995049}%
\pgfsetlinewidth{1.003750pt}%
\definecolor{currentstroke}{rgb}{0.121569,0.466667,0.705882}%
\pgfsetstrokecolor{currentstroke}%
\pgfsetstrokeopacity{0.995049}%
\pgfsetdash{}{0pt}%
\pgfpathmoveto{\pgfqpoint{2.446247in}{0.773466in}}%
\pgfpathcurveto{\pgfqpoint{2.454483in}{0.773466in}}{\pgfqpoint{2.462383in}{0.776738in}}{\pgfqpoint{2.468207in}{0.782562in}}%
\pgfpathcurveto{\pgfqpoint{2.474031in}{0.788386in}}{\pgfqpoint{2.477303in}{0.796286in}}{\pgfqpoint{2.477303in}{0.804522in}}%
\pgfpathcurveto{\pgfqpoint{2.477303in}{0.812759in}}{\pgfqpoint{2.474031in}{0.820659in}}{\pgfqpoint{2.468207in}{0.826483in}}%
\pgfpathcurveto{\pgfqpoint{2.462383in}{0.832306in}}{\pgfqpoint{2.454483in}{0.835579in}}{\pgfqpoint{2.446247in}{0.835579in}}%
\pgfpathcurveto{\pgfqpoint{2.438011in}{0.835579in}}{\pgfqpoint{2.430111in}{0.832306in}}{\pgfqpoint{2.424287in}{0.826483in}}%
\pgfpathcurveto{\pgfqpoint{2.418463in}{0.820659in}}{\pgfqpoint{2.415190in}{0.812759in}}{\pgfqpoint{2.415190in}{0.804522in}}%
\pgfpathcurveto{\pgfqpoint{2.415190in}{0.796286in}}{\pgfqpoint{2.418463in}{0.788386in}}{\pgfqpoint{2.424287in}{0.782562in}}%
\pgfpathcurveto{\pgfqpoint{2.430111in}{0.776738in}}{\pgfqpoint{2.438011in}{0.773466in}}{\pgfqpoint{2.446247in}{0.773466in}}%
\pgfpathclose%
\pgfusepath{stroke,fill}%
\end{pgfscope}%
\begin{pgfscope}%
\pgfpathrectangle{\pgfqpoint{0.100000in}{0.212622in}}{\pgfqpoint{3.696000in}{3.696000in}}%
\pgfusepath{clip}%
\pgfsetbuttcap%
\pgfsetroundjoin%
\definecolor{currentfill}{rgb}{0.121569,0.466667,0.705882}%
\pgfsetfillcolor{currentfill}%
\pgfsetfillopacity{0.995509}%
\pgfsetlinewidth{1.003750pt}%
\definecolor{currentstroke}{rgb}{0.121569,0.466667,0.705882}%
\pgfsetstrokecolor{currentstroke}%
\pgfsetstrokeopacity{0.995509}%
\pgfsetdash{}{0pt}%
\pgfpathmoveto{\pgfqpoint{2.402629in}{0.773197in}}%
\pgfpathcurveto{\pgfqpoint{2.410865in}{0.773197in}}{\pgfqpoint{2.418765in}{0.776469in}}{\pgfqpoint{2.424589in}{0.782293in}}%
\pgfpathcurveto{\pgfqpoint{2.430413in}{0.788117in}}{\pgfqpoint{2.433685in}{0.796017in}}{\pgfqpoint{2.433685in}{0.804253in}}%
\pgfpathcurveto{\pgfqpoint{2.433685in}{0.812490in}}{\pgfqpoint{2.430413in}{0.820390in}}{\pgfqpoint{2.424589in}{0.826214in}}%
\pgfpathcurveto{\pgfqpoint{2.418765in}{0.832037in}}{\pgfqpoint{2.410865in}{0.835310in}}{\pgfqpoint{2.402629in}{0.835310in}}%
\pgfpathcurveto{\pgfqpoint{2.394392in}{0.835310in}}{\pgfqpoint{2.386492in}{0.832037in}}{\pgfqpoint{2.380668in}{0.826214in}}%
\pgfpathcurveto{\pgfqpoint{2.374844in}{0.820390in}}{\pgfqpoint{2.371572in}{0.812490in}}{\pgfqpoint{2.371572in}{0.804253in}}%
\pgfpathcurveto{\pgfqpoint{2.371572in}{0.796017in}}{\pgfqpoint{2.374844in}{0.788117in}}{\pgfqpoint{2.380668in}{0.782293in}}%
\pgfpathcurveto{\pgfqpoint{2.386492in}{0.776469in}}{\pgfqpoint{2.394392in}{0.773197in}}{\pgfqpoint{2.402629in}{0.773197in}}%
\pgfpathclose%
\pgfusepath{stroke,fill}%
\end{pgfscope}%
\begin{pgfscope}%
\pgfpathrectangle{\pgfqpoint{0.100000in}{0.212622in}}{\pgfqpoint{3.696000in}{3.696000in}}%
\pgfusepath{clip}%
\pgfsetbuttcap%
\pgfsetroundjoin%
\definecolor{currentfill}{rgb}{0.121569,0.466667,0.705882}%
\pgfsetfillcolor{currentfill}%
\pgfsetfillopacity{0.996209}%
\pgfsetlinewidth{1.003750pt}%
\definecolor{currentstroke}{rgb}{0.121569,0.466667,0.705882}%
\pgfsetstrokecolor{currentstroke}%
\pgfsetstrokeopacity{0.996209}%
\pgfsetdash{}{0pt}%
\pgfpathmoveto{\pgfqpoint{2.406527in}{0.771395in}}%
\pgfpathcurveto{\pgfqpoint{2.414764in}{0.771395in}}{\pgfqpoint{2.422664in}{0.774667in}}{\pgfqpoint{2.428488in}{0.780491in}}%
\pgfpathcurveto{\pgfqpoint{2.434311in}{0.786315in}}{\pgfqpoint{2.437584in}{0.794215in}}{\pgfqpoint{2.437584in}{0.802451in}}%
\pgfpathcurveto{\pgfqpoint{2.437584in}{0.810687in}}{\pgfqpoint{2.434311in}{0.818588in}}{\pgfqpoint{2.428488in}{0.824411in}}%
\pgfpathcurveto{\pgfqpoint{2.422664in}{0.830235in}}{\pgfqpoint{2.414764in}{0.833508in}}{\pgfqpoint{2.406527in}{0.833508in}}%
\pgfpathcurveto{\pgfqpoint{2.398291in}{0.833508in}}{\pgfqpoint{2.390391in}{0.830235in}}{\pgfqpoint{2.384567in}{0.824411in}}%
\pgfpathcurveto{\pgfqpoint{2.378743in}{0.818588in}}{\pgfqpoint{2.375471in}{0.810687in}}{\pgfqpoint{2.375471in}{0.802451in}}%
\pgfpathcurveto{\pgfqpoint{2.375471in}{0.794215in}}{\pgfqpoint{2.378743in}{0.786315in}}{\pgfqpoint{2.384567in}{0.780491in}}%
\pgfpathcurveto{\pgfqpoint{2.390391in}{0.774667in}}{\pgfqpoint{2.398291in}{0.771395in}}{\pgfqpoint{2.406527in}{0.771395in}}%
\pgfpathclose%
\pgfusepath{stroke,fill}%
\end{pgfscope}%
\begin{pgfscope}%
\pgfpathrectangle{\pgfqpoint{0.100000in}{0.212622in}}{\pgfqpoint{3.696000in}{3.696000in}}%
\pgfusepath{clip}%
\pgfsetbuttcap%
\pgfsetroundjoin%
\definecolor{currentfill}{rgb}{0.121569,0.466667,0.705882}%
\pgfsetfillcolor{currentfill}%
\pgfsetfillopacity{0.996280}%
\pgfsetlinewidth{1.003750pt}%
\definecolor{currentstroke}{rgb}{0.121569,0.466667,0.705882}%
\pgfsetstrokecolor{currentstroke}%
\pgfsetstrokeopacity{0.996280}%
\pgfsetdash{}{0pt}%
\pgfpathmoveto{\pgfqpoint{2.447060in}{0.769024in}}%
\pgfpathcurveto{\pgfqpoint{2.455297in}{0.769024in}}{\pgfqpoint{2.463197in}{0.772296in}}{\pgfqpoint{2.469021in}{0.778120in}}%
\pgfpathcurveto{\pgfqpoint{2.474844in}{0.783944in}}{\pgfqpoint{2.478117in}{0.791844in}}{\pgfqpoint{2.478117in}{0.800080in}}%
\pgfpathcurveto{\pgfqpoint{2.478117in}{0.808316in}}{\pgfqpoint{2.474844in}{0.816216in}}{\pgfqpoint{2.469021in}{0.822040in}}%
\pgfpathcurveto{\pgfqpoint{2.463197in}{0.827864in}}{\pgfqpoint{2.455297in}{0.831137in}}{\pgfqpoint{2.447060in}{0.831137in}}%
\pgfpathcurveto{\pgfqpoint{2.438824in}{0.831137in}}{\pgfqpoint{2.430924in}{0.827864in}}{\pgfqpoint{2.425100in}{0.822040in}}%
\pgfpathcurveto{\pgfqpoint{2.419276in}{0.816216in}}{\pgfqpoint{2.416004in}{0.808316in}}{\pgfqpoint{2.416004in}{0.800080in}}%
\pgfpathcurveto{\pgfqpoint{2.416004in}{0.791844in}}{\pgfqpoint{2.419276in}{0.783944in}}{\pgfqpoint{2.425100in}{0.778120in}}%
\pgfpathcurveto{\pgfqpoint{2.430924in}{0.772296in}}{\pgfqpoint{2.438824in}{0.769024in}}{\pgfqpoint{2.447060in}{0.769024in}}%
\pgfpathclose%
\pgfusepath{stroke,fill}%
\end{pgfscope}%
\begin{pgfscope}%
\pgfpathrectangle{\pgfqpoint{0.100000in}{0.212622in}}{\pgfqpoint{3.696000in}{3.696000in}}%
\pgfusepath{clip}%
\pgfsetbuttcap%
\pgfsetroundjoin%
\definecolor{currentfill}{rgb}{0.121569,0.466667,0.705882}%
\pgfsetfillcolor{currentfill}%
\pgfsetfillopacity{0.996827}%
\pgfsetlinewidth{1.003750pt}%
\definecolor{currentstroke}{rgb}{0.121569,0.466667,0.705882}%
\pgfsetstrokecolor{currentstroke}%
\pgfsetstrokeopacity{0.996827}%
\pgfsetdash{}{0pt}%
\pgfpathmoveto{\pgfqpoint{2.410166in}{0.769680in}}%
\pgfpathcurveto{\pgfqpoint{2.418402in}{0.769680in}}{\pgfqpoint{2.426302in}{0.772952in}}{\pgfqpoint{2.432126in}{0.778776in}}%
\pgfpathcurveto{\pgfqpoint{2.437950in}{0.784600in}}{\pgfqpoint{2.441222in}{0.792500in}}{\pgfqpoint{2.441222in}{0.800737in}}%
\pgfpathcurveto{\pgfqpoint{2.441222in}{0.808973in}}{\pgfqpoint{2.437950in}{0.816873in}}{\pgfqpoint{2.432126in}{0.822697in}}%
\pgfpathcurveto{\pgfqpoint{2.426302in}{0.828521in}}{\pgfqpoint{2.418402in}{0.831793in}}{\pgfqpoint{2.410166in}{0.831793in}}%
\pgfpathcurveto{\pgfqpoint{2.401930in}{0.831793in}}{\pgfqpoint{2.394030in}{0.828521in}}{\pgfqpoint{2.388206in}{0.822697in}}%
\pgfpathcurveto{\pgfqpoint{2.382382in}{0.816873in}}{\pgfqpoint{2.379109in}{0.808973in}}{\pgfqpoint{2.379109in}{0.800737in}}%
\pgfpathcurveto{\pgfqpoint{2.379109in}{0.792500in}}{\pgfqpoint{2.382382in}{0.784600in}}{\pgfqpoint{2.388206in}{0.778776in}}%
\pgfpathcurveto{\pgfqpoint{2.394030in}{0.772952in}}{\pgfqpoint{2.401930in}{0.769680in}}{\pgfqpoint{2.410166in}{0.769680in}}%
\pgfpathclose%
\pgfusepath{stroke,fill}%
\end{pgfscope}%
\begin{pgfscope}%
\pgfpathrectangle{\pgfqpoint{0.100000in}{0.212622in}}{\pgfqpoint{3.696000in}{3.696000in}}%
\pgfusepath{clip}%
\pgfsetbuttcap%
\pgfsetroundjoin%
\definecolor{currentfill}{rgb}{0.121569,0.466667,0.705882}%
\pgfsetfillcolor{currentfill}%
\pgfsetfillopacity{0.996864}%
\pgfsetlinewidth{1.003750pt}%
\definecolor{currentstroke}{rgb}{0.121569,0.466667,0.705882}%
\pgfsetstrokecolor{currentstroke}%
\pgfsetstrokeopacity{0.996864}%
\pgfsetdash{}{0pt}%
\pgfpathmoveto{\pgfqpoint{2.447395in}{0.766159in}}%
\pgfpathcurveto{\pgfqpoint{2.455631in}{0.766159in}}{\pgfqpoint{2.463531in}{0.769431in}}{\pgfqpoint{2.469355in}{0.775255in}}%
\pgfpathcurveto{\pgfqpoint{2.475179in}{0.781079in}}{\pgfqpoint{2.478451in}{0.788979in}}{\pgfqpoint{2.478451in}{0.797215in}}%
\pgfpathcurveto{\pgfqpoint{2.478451in}{0.805452in}}{\pgfqpoint{2.475179in}{0.813352in}}{\pgfqpoint{2.469355in}{0.819176in}}%
\pgfpathcurveto{\pgfqpoint{2.463531in}{0.825000in}}{\pgfqpoint{2.455631in}{0.828272in}}{\pgfqpoint{2.447395in}{0.828272in}}%
\pgfpathcurveto{\pgfqpoint{2.439158in}{0.828272in}}{\pgfqpoint{2.431258in}{0.825000in}}{\pgfqpoint{2.425435in}{0.819176in}}%
\pgfpathcurveto{\pgfqpoint{2.419611in}{0.813352in}}{\pgfqpoint{2.416338in}{0.805452in}}{\pgfqpoint{2.416338in}{0.797215in}}%
\pgfpathcurveto{\pgfqpoint{2.416338in}{0.788979in}}{\pgfqpoint{2.419611in}{0.781079in}}{\pgfqpoint{2.425435in}{0.775255in}}%
\pgfpathcurveto{\pgfqpoint{2.431258in}{0.769431in}}{\pgfqpoint{2.439158in}{0.766159in}}{\pgfqpoint{2.447395in}{0.766159in}}%
\pgfpathclose%
\pgfusepath{stroke,fill}%
\end{pgfscope}%
\begin{pgfscope}%
\pgfpathrectangle{\pgfqpoint{0.100000in}{0.212622in}}{\pgfqpoint{3.696000in}{3.696000in}}%
\pgfusepath{clip}%
\pgfsetbuttcap%
\pgfsetroundjoin%
\definecolor{currentfill}{rgb}{0.121569,0.466667,0.705882}%
\pgfsetfillcolor{currentfill}%
\pgfsetfillopacity{0.997204}%
\pgfsetlinewidth{1.003750pt}%
\definecolor{currentstroke}{rgb}{0.121569,0.466667,0.705882}%
\pgfsetstrokecolor{currentstroke}%
\pgfsetstrokeopacity{0.997204}%
\pgfsetdash{}{0pt}%
\pgfpathmoveto{\pgfqpoint{2.447463in}{0.764568in}}%
\pgfpathcurveto{\pgfqpoint{2.455699in}{0.764568in}}{\pgfqpoint{2.463599in}{0.767841in}}{\pgfqpoint{2.469423in}{0.773665in}}%
\pgfpathcurveto{\pgfqpoint{2.475247in}{0.779489in}}{\pgfqpoint{2.478519in}{0.787389in}}{\pgfqpoint{2.478519in}{0.795625in}}%
\pgfpathcurveto{\pgfqpoint{2.478519in}{0.803861in}}{\pgfqpoint{2.475247in}{0.811761in}}{\pgfqpoint{2.469423in}{0.817585in}}%
\pgfpathcurveto{\pgfqpoint{2.463599in}{0.823409in}}{\pgfqpoint{2.455699in}{0.826681in}}{\pgfqpoint{2.447463in}{0.826681in}}%
\pgfpathcurveto{\pgfqpoint{2.439227in}{0.826681in}}{\pgfqpoint{2.431327in}{0.823409in}}{\pgfqpoint{2.425503in}{0.817585in}}%
\pgfpathcurveto{\pgfqpoint{2.419679in}{0.811761in}}{\pgfqpoint{2.416406in}{0.803861in}}{\pgfqpoint{2.416406in}{0.795625in}}%
\pgfpathcurveto{\pgfqpoint{2.416406in}{0.787389in}}{\pgfqpoint{2.419679in}{0.779489in}}{\pgfqpoint{2.425503in}{0.773665in}}%
\pgfpathcurveto{\pgfqpoint{2.431327in}{0.767841in}}{\pgfqpoint{2.439227in}{0.764568in}}{\pgfqpoint{2.447463in}{0.764568in}}%
\pgfpathclose%
\pgfusepath{stroke,fill}%
\end{pgfscope}%
\begin{pgfscope}%
\pgfpathrectangle{\pgfqpoint{0.100000in}{0.212622in}}{\pgfqpoint{3.696000in}{3.696000in}}%
\pgfusepath{clip}%
\pgfsetbuttcap%
\pgfsetroundjoin%
\definecolor{currentfill}{rgb}{0.121569,0.466667,0.705882}%
\pgfsetfillcolor{currentfill}%
\pgfsetfillopacity{0.997384}%
\pgfsetlinewidth{1.003750pt}%
\definecolor{currentstroke}{rgb}{0.121569,0.466667,0.705882}%
\pgfsetstrokecolor{currentstroke}%
\pgfsetstrokeopacity{0.997384}%
\pgfsetdash{}{0pt}%
\pgfpathmoveto{\pgfqpoint{2.447451in}{0.763641in}}%
\pgfpathcurveto{\pgfqpoint{2.455687in}{0.763641in}}{\pgfqpoint{2.463587in}{0.766914in}}{\pgfqpoint{2.469411in}{0.772738in}}%
\pgfpathcurveto{\pgfqpoint{2.475235in}{0.778562in}}{\pgfqpoint{2.478507in}{0.786462in}}{\pgfqpoint{2.478507in}{0.794698in}}%
\pgfpathcurveto{\pgfqpoint{2.478507in}{0.802934in}}{\pgfqpoint{2.475235in}{0.810834in}}{\pgfqpoint{2.469411in}{0.816658in}}%
\pgfpathcurveto{\pgfqpoint{2.463587in}{0.822482in}}{\pgfqpoint{2.455687in}{0.825754in}}{\pgfqpoint{2.447451in}{0.825754in}}%
\pgfpathcurveto{\pgfqpoint{2.439215in}{0.825754in}}{\pgfqpoint{2.431315in}{0.822482in}}{\pgfqpoint{2.425491in}{0.816658in}}%
\pgfpathcurveto{\pgfqpoint{2.419667in}{0.810834in}}{\pgfqpoint{2.416394in}{0.802934in}}{\pgfqpoint{2.416394in}{0.794698in}}%
\pgfpathcurveto{\pgfqpoint{2.416394in}{0.786462in}}{\pgfqpoint{2.419667in}{0.778562in}}{\pgfqpoint{2.425491in}{0.772738in}}%
\pgfpathcurveto{\pgfqpoint{2.431315in}{0.766914in}}{\pgfqpoint{2.439215in}{0.763641in}}{\pgfqpoint{2.447451in}{0.763641in}}%
\pgfpathclose%
\pgfusepath{stroke,fill}%
\end{pgfscope}%
\begin{pgfscope}%
\pgfpathrectangle{\pgfqpoint{0.100000in}{0.212622in}}{\pgfqpoint{3.696000in}{3.696000in}}%
\pgfusepath{clip}%
\pgfsetbuttcap%
\pgfsetroundjoin%
\definecolor{currentfill}{rgb}{0.121569,0.466667,0.705882}%
\pgfsetfillcolor{currentfill}%
\pgfsetfillopacity{0.997388}%
\pgfsetlinewidth{1.003750pt}%
\definecolor{currentstroke}{rgb}{0.121569,0.466667,0.705882}%
\pgfsetstrokecolor{currentstroke}%
\pgfsetstrokeopacity{0.997388}%
\pgfsetdash{}{0pt}%
\pgfpathmoveto{\pgfqpoint{2.413545in}{0.768115in}}%
\pgfpathcurveto{\pgfqpoint{2.421782in}{0.768115in}}{\pgfqpoint{2.429682in}{0.771387in}}{\pgfqpoint{2.435506in}{0.777211in}}%
\pgfpathcurveto{\pgfqpoint{2.441329in}{0.783035in}}{\pgfqpoint{2.444602in}{0.790935in}}{\pgfqpoint{2.444602in}{0.799171in}}%
\pgfpathcurveto{\pgfqpoint{2.444602in}{0.807408in}}{\pgfqpoint{2.441329in}{0.815308in}}{\pgfqpoint{2.435506in}{0.821132in}}%
\pgfpathcurveto{\pgfqpoint{2.429682in}{0.826956in}}{\pgfqpoint{2.421782in}{0.830228in}}{\pgfqpoint{2.413545in}{0.830228in}}%
\pgfpathcurveto{\pgfqpoint{2.405309in}{0.830228in}}{\pgfqpoint{2.397409in}{0.826956in}}{\pgfqpoint{2.391585in}{0.821132in}}%
\pgfpathcurveto{\pgfqpoint{2.385761in}{0.815308in}}{\pgfqpoint{2.382489in}{0.807408in}}{\pgfqpoint{2.382489in}{0.799171in}}%
\pgfpathcurveto{\pgfqpoint{2.382489in}{0.790935in}}{\pgfqpoint{2.385761in}{0.783035in}}{\pgfqpoint{2.391585in}{0.777211in}}%
\pgfpathcurveto{\pgfqpoint{2.397409in}{0.771387in}}{\pgfqpoint{2.405309in}{0.768115in}}{\pgfqpoint{2.413545in}{0.768115in}}%
\pgfpathclose%
\pgfusepath{stroke,fill}%
\end{pgfscope}%
\begin{pgfscope}%
\pgfpathrectangle{\pgfqpoint{0.100000in}{0.212622in}}{\pgfqpoint{3.696000in}{3.696000in}}%
\pgfusepath{clip}%
\pgfsetbuttcap%
\pgfsetroundjoin%
\definecolor{currentfill}{rgb}{0.121569,0.466667,0.705882}%
\pgfsetfillcolor{currentfill}%
\pgfsetfillopacity{0.997484}%
\pgfsetlinewidth{1.003750pt}%
\definecolor{currentstroke}{rgb}{0.121569,0.466667,0.705882}%
\pgfsetstrokecolor{currentstroke}%
\pgfsetstrokeopacity{0.997484}%
\pgfsetdash{}{0pt}%
\pgfpathmoveto{\pgfqpoint{2.447424in}{0.763130in}}%
\pgfpathcurveto{\pgfqpoint{2.455660in}{0.763130in}}{\pgfqpoint{2.463560in}{0.766402in}}{\pgfqpoint{2.469384in}{0.772226in}}%
\pgfpathcurveto{\pgfqpoint{2.475208in}{0.778050in}}{\pgfqpoint{2.478480in}{0.785950in}}{\pgfqpoint{2.478480in}{0.794186in}}%
\pgfpathcurveto{\pgfqpoint{2.478480in}{0.802423in}}{\pgfqpoint{2.475208in}{0.810323in}}{\pgfqpoint{2.469384in}{0.816147in}}%
\pgfpathcurveto{\pgfqpoint{2.463560in}{0.821970in}}{\pgfqpoint{2.455660in}{0.825243in}}{\pgfqpoint{2.447424in}{0.825243in}}%
\pgfpathcurveto{\pgfqpoint{2.439187in}{0.825243in}}{\pgfqpoint{2.431287in}{0.821970in}}{\pgfqpoint{2.425463in}{0.816147in}}%
\pgfpathcurveto{\pgfqpoint{2.419639in}{0.810323in}}{\pgfqpoint{2.416367in}{0.802423in}}{\pgfqpoint{2.416367in}{0.794186in}}%
\pgfpathcurveto{\pgfqpoint{2.416367in}{0.785950in}}{\pgfqpoint{2.419639in}{0.778050in}}{\pgfqpoint{2.425463in}{0.772226in}}%
\pgfpathcurveto{\pgfqpoint{2.431287in}{0.766402in}}{\pgfqpoint{2.439187in}{0.763130in}}{\pgfqpoint{2.447424in}{0.763130in}}%
\pgfpathclose%
\pgfusepath{stroke,fill}%
\end{pgfscope}%
\begin{pgfscope}%
\pgfpathrectangle{\pgfqpoint{0.100000in}{0.212622in}}{\pgfqpoint{3.696000in}{3.696000in}}%
\pgfusepath{clip}%
\pgfsetbuttcap%
\pgfsetroundjoin%
\definecolor{currentfill}{rgb}{0.121569,0.466667,0.705882}%
\pgfsetfillcolor{currentfill}%
\pgfsetfillopacity{0.997542}%
\pgfsetlinewidth{1.003750pt}%
\definecolor{currentstroke}{rgb}{0.121569,0.466667,0.705882}%
\pgfsetstrokecolor{currentstroke}%
\pgfsetstrokeopacity{0.997542}%
\pgfsetdash{}{0pt}%
\pgfpathmoveto{\pgfqpoint{2.447393in}{0.762855in}}%
\pgfpathcurveto{\pgfqpoint{2.455630in}{0.762855in}}{\pgfqpoint{2.463530in}{0.766127in}}{\pgfqpoint{2.469353in}{0.771951in}}%
\pgfpathcurveto{\pgfqpoint{2.475177in}{0.777775in}}{\pgfqpoint{2.478450in}{0.785675in}}{\pgfqpoint{2.478450in}{0.793911in}}%
\pgfpathcurveto{\pgfqpoint{2.478450in}{0.802147in}}{\pgfqpoint{2.475177in}{0.810047in}}{\pgfqpoint{2.469353in}{0.815871in}}%
\pgfpathcurveto{\pgfqpoint{2.463530in}{0.821695in}}{\pgfqpoint{2.455630in}{0.824968in}}{\pgfqpoint{2.447393in}{0.824968in}}%
\pgfpathcurveto{\pgfqpoint{2.439157in}{0.824968in}}{\pgfqpoint{2.431257in}{0.821695in}}{\pgfqpoint{2.425433in}{0.815871in}}%
\pgfpathcurveto{\pgfqpoint{2.419609in}{0.810047in}}{\pgfqpoint{2.416337in}{0.802147in}}{\pgfqpoint{2.416337in}{0.793911in}}%
\pgfpathcurveto{\pgfqpoint{2.416337in}{0.785675in}}{\pgfqpoint{2.419609in}{0.777775in}}{\pgfqpoint{2.425433in}{0.771951in}}%
\pgfpathcurveto{\pgfqpoint{2.431257in}{0.766127in}}{\pgfqpoint{2.439157in}{0.762855in}}{\pgfqpoint{2.447393in}{0.762855in}}%
\pgfpathclose%
\pgfusepath{stroke,fill}%
\end{pgfscope}%
\begin{pgfscope}%
\pgfpathrectangle{\pgfqpoint{0.100000in}{0.212622in}}{\pgfqpoint{3.696000in}{3.696000in}}%
\pgfusepath{clip}%
\pgfsetbuttcap%
\pgfsetroundjoin%
\definecolor{currentfill}{rgb}{0.121569,0.466667,0.705882}%
\pgfsetfillcolor{currentfill}%
\pgfsetfillopacity{0.997712}%
\pgfsetlinewidth{1.003750pt}%
\definecolor{currentstroke}{rgb}{0.121569,0.466667,0.705882}%
\pgfsetstrokecolor{currentstroke}%
\pgfsetstrokeopacity{0.997712}%
\pgfsetdash{}{0pt}%
\pgfpathmoveto{\pgfqpoint{2.447270in}{0.762104in}}%
\pgfpathcurveto{\pgfqpoint{2.455507in}{0.762104in}}{\pgfqpoint{2.463407in}{0.765376in}}{\pgfqpoint{2.469231in}{0.771200in}}%
\pgfpathcurveto{\pgfqpoint{2.475055in}{0.777024in}}{\pgfqpoint{2.478327in}{0.784924in}}{\pgfqpoint{2.478327in}{0.793160in}}%
\pgfpathcurveto{\pgfqpoint{2.478327in}{0.801396in}}{\pgfqpoint{2.475055in}{0.809296in}}{\pgfqpoint{2.469231in}{0.815120in}}%
\pgfpathcurveto{\pgfqpoint{2.463407in}{0.820944in}}{\pgfqpoint{2.455507in}{0.824217in}}{\pgfqpoint{2.447270in}{0.824217in}}%
\pgfpathcurveto{\pgfqpoint{2.439034in}{0.824217in}}{\pgfqpoint{2.431134in}{0.820944in}}{\pgfqpoint{2.425310in}{0.815120in}}%
\pgfpathcurveto{\pgfqpoint{2.419486in}{0.809296in}}{\pgfqpoint{2.416214in}{0.801396in}}{\pgfqpoint{2.416214in}{0.793160in}}%
\pgfpathcurveto{\pgfqpoint{2.416214in}{0.784924in}}{\pgfqpoint{2.419486in}{0.777024in}}{\pgfqpoint{2.425310in}{0.771200in}}%
\pgfpathcurveto{\pgfqpoint{2.431134in}{0.765376in}}{\pgfqpoint{2.439034in}{0.762104in}}{\pgfqpoint{2.447270in}{0.762104in}}%
\pgfpathclose%
\pgfusepath{stroke,fill}%
\end{pgfscope}%
\begin{pgfscope}%
\pgfpathrectangle{\pgfqpoint{0.100000in}{0.212622in}}{\pgfqpoint{3.696000in}{3.696000in}}%
\pgfusepath{clip}%
\pgfsetbuttcap%
\pgfsetroundjoin%
\definecolor{currentfill}{rgb}{0.121569,0.466667,0.705882}%
\pgfsetfillcolor{currentfill}%
\pgfsetfillopacity{0.997804}%
\pgfsetlinewidth{1.003750pt}%
\definecolor{currentstroke}{rgb}{0.121569,0.466667,0.705882}%
\pgfsetstrokecolor{currentstroke}%
\pgfsetstrokeopacity{0.997804}%
\pgfsetdash{}{0pt}%
\pgfpathmoveto{\pgfqpoint{2.447183in}{0.761688in}}%
\pgfpathcurveto{\pgfqpoint{2.455420in}{0.761688in}}{\pgfqpoint{2.463320in}{0.764961in}}{\pgfqpoint{2.469144in}{0.770785in}}%
\pgfpathcurveto{\pgfqpoint{2.474967in}{0.776608in}}{\pgfqpoint{2.478240in}{0.784509in}}{\pgfqpoint{2.478240in}{0.792745in}}%
\pgfpathcurveto{\pgfqpoint{2.478240in}{0.800981in}}{\pgfqpoint{2.474967in}{0.808881in}}{\pgfqpoint{2.469144in}{0.814705in}}%
\pgfpathcurveto{\pgfqpoint{2.463320in}{0.820529in}}{\pgfqpoint{2.455420in}{0.823801in}}{\pgfqpoint{2.447183in}{0.823801in}}%
\pgfpathcurveto{\pgfqpoint{2.438947in}{0.823801in}}{\pgfqpoint{2.431047in}{0.820529in}}{\pgfqpoint{2.425223in}{0.814705in}}%
\pgfpathcurveto{\pgfqpoint{2.419399in}{0.808881in}}{\pgfqpoint{2.416127in}{0.800981in}}{\pgfqpoint{2.416127in}{0.792745in}}%
\pgfpathcurveto{\pgfqpoint{2.416127in}{0.784509in}}{\pgfqpoint{2.419399in}{0.776608in}}{\pgfqpoint{2.425223in}{0.770785in}}%
\pgfpathcurveto{\pgfqpoint{2.431047in}{0.764961in}}{\pgfqpoint{2.438947in}{0.761688in}}{\pgfqpoint{2.447183in}{0.761688in}}%
\pgfpathclose%
\pgfusepath{stroke,fill}%
\end{pgfscope}%
\begin{pgfscope}%
\pgfpathrectangle{\pgfqpoint{0.100000in}{0.212622in}}{\pgfqpoint{3.696000in}{3.696000in}}%
\pgfusepath{clip}%
\pgfsetbuttcap%
\pgfsetroundjoin%
\definecolor{currentfill}{rgb}{0.121569,0.466667,0.705882}%
\pgfsetfillcolor{currentfill}%
\pgfsetfillopacity{0.997854}%
\pgfsetlinewidth{1.003750pt}%
\definecolor{currentstroke}{rgb}{0.121569,0.466667,0.705882}%
\pgfsetstrokecolor{currentstroke}%
\pgfsetstrokeopacity{0.997854}%
\pgfsetdash{}{0pt}%
\pgfpathmoveto{\pgfqpoint{2.447119in}{0.761464in}}%
\pgfpathcurveto{\pgfqpoint{2.455355in}{0.761464in}}{\pgfqpoint{2.463255in}{0.764737in}}{\pgfqpoint{2.469079in}{0.770561in}}%
\pgfpathcurveto{\pgfqpoint{2.474903in}{0.776384in}}{\pgfqpoint{2.478175in}{0.784285in}}{\pgfqpoint{2.478175in}{0.792521in}}%
\pgfpathcurveto{\pgfqpoint{2.478175in}{0.800757in}}{\pgfqpoint{2.474903in}{0.808657in}}{\pgfqpoint{2.469079in}{0.814481in}}%
\pgfpathcurveto{\pgfqpoint{2.463255in}{0.820305in}}{\pgfqpoint{2.455355in}{0.823577in}}{\pgfqpoint{2.447119in}{0.823577in}}%
\pgfpathcurveto{\pgfqpoint{2.438882in}{0.823577in}}{\pgfqpoint{2.430982in}{0.820305in}}{\pgfqpoint{2.425158in}{0.814481in}}%
\pgfpathcurveto{\pgfqpoint{2.419334in}{0.808657in}}{\pgfqpoint{2.416062in}{0.800757in}}{\pgfqpoint{2.416062in}{0.792521in}}%
\pgfpathcurveto{\pgfqpoint{2.416062in}{0.784285in}}{\pgfqpoint{2.419334in}{0.776384in}}{\pgfqpoint{2.425158in}{0.770561in}}%
\pgfpathcurveto{\pgfqpoint{2.430982in}{0.764737in}}{\pgfqpoint{2.438882in}{0.761464in}}{\pgfqpoint{2.447119in}{0.761464in}}%
\pgfpathclose%
\pgfusepath{stroke,fill}%
\end{pgfscope}%
\begin{pgfscope}%
\pgfpathrectangle{\pgfqpoint{0.100000in}{0.212622in}}{\pgfqpoint{3.696000in}{3.696000in}}%
\pgfusepath{clip}%
\pgfsetbuttcap%
\pgfsetroundjoin%
\definecolor{currentfill}{rgb}{0.121569,0.466667,0.705882}%
\pgfsetfillcolor{currentfill}%
\pgfsetfillopacity{0.997881}%
\pgfsetlinewidth{1.003750pt}%
\definecolor{currentstroke}{rgb}{0.121569,0.466667,0.705882}%
\pgfsetstrokecolor{currentstroke}%
\pgfsetstrokeopacity{0.997881}%
\pgfsetdash{}{0pt}%
\pgfpathmoveto{\pgfqpoint{2.416674in}{0.766660in}}%
\pgfpathcurveto{\pgfqpoint{2.424910in}{0.766660in}}{\pgfqpoint{2.432810in}{0.769932in}}{\pgfqpoint{2.438634in}{0.775756in}}%
\pgfpathcurveto{\pgfqpoint{2.444458in}{0.781580in}}{\pgfqpoint{2.447731in}{0.789480in}}{\pgfqpoint{2.447731in}{0.797717in}}%
\pgfpathcurveto{\pgfqpoint{2.447731in}{0.805953in}}{\pgfqpoint{2.444458in}{0.813853in}}{\pgfqpoint{2.438634in}{0.819677in}}%
\pgfpathcurveto{\pgfqpoint{2.432810in}{0.825501in}}{\pgfqpoint{2.424910in}{0.828773in}}{\pgfqpoint{2.416674in}{0.828773in}}%
\pgfpathcurveto{\pgfqpoint{2.408438in}{0.828773in}}{\pgfqpoint{2.400538in}{0.825501in}}{\pgfqpoint{2.394714in}{0.819677in}}%
\pgfpathcurveto{\pgfqpoint{2.388890in}{0.813853in}}{\pgfqpoint{2.385618in}{0.805953in}}{\pgfqpoint{2.385618in}{0.797717in}}%
\pgfpathcurveto{\pgfqpoint{2.385618in}{0.789480in}}{\pgfqpoint{2.388890in}{0.781580in}}{\pgfqpoint{2.394714in}{0.775756in}}%
\pgfpathcurveto{\pgfqpoint{2.400538in}{0.769932in}}{\pgfqpoint{2.408438in}{0.766660in}}{\pgfqpoint{2.416674in}{0.766660in}}%
\pgfpathclose%
\pgfusepath{stroke,fill}%
\end{pgfscope}%
\begin{pgfscope}%
\pgfpathrectangle{\pgfqpoint{0.100000in}{0.212622in}}{\pgfqpoint{3.696000in}{3.696000in}}%
\pgfusepath{clip}%
\pgfsetbuttcap%
\pgfsetroundjoin%
\definecolor{currentfill}{rgb}{0.121569,0.466667,0.705882}%
\pgfsetfillcolor{currentfill}%
\pgfsetfillopacity{0.998043}%
\pgfsetlinewidth{1.003750pt}%
\definecolor{currentstroke}{rgb}{0.121569,0.466667,0.705882}%
\pgfsetstrokecolor{currentstroke}%
\pgfsetstrokeopacity{0.998043}%
\pgfsetdash{}{0pt}%
\pgfpathmoveto{\pgfqpoint{2.446830in}{0.760679in}}%
\pgfpathcurveto{\pgfqpoint{2.455066in}{0.760679in}}{\pgfqpoint{2.462966in}{0.763952in}}{\pgfqpoint{2.468790in}{0.769776in}}%
\pgfpathcurveto{\pgfqpoint{2.474614in}{0.775600in}}{\pgfqpoint{2.477887in}{0.783500in}}{\pgfqpoint{2.477887in}{0.791736in}}%
\pgfpathcurveto{\pgfqpoint{2.477887in}{0.799972in}}{\pgfqpoint{2.474614in}{0.807872in}}{\pgfqpoint{2.468790in}{0.813696in}}%
\pgfpathcurveto{\pgfqpoint{2.462966in}{0.819520in}}{\pgfqpoint{2.455066in}{0.822792in}}{\pgfqpoint{2.446830in}{0.822792in}}%
\pgfpathcurveto{\pgfqpoint{2.438594in}{0.822792in}}{\pgfqpoint{2.430694in}{0.819520in}}{\pgfqpoint{2.424870in}{0.813696in}}%
\pgfpathcurveto{\pgfqpoint{2.419046in}{0.807872in}}{\pgfqpoint{2.415774in}{0.799972in}}{\pgfqpoint{2.415774in}{0.791736in}}%
\pgfpathcurveto{\pgfqpoint{2.415774in}{0.783500in}}{\pgfqpoint{2.419046in}{0.775600in}}{\pgfqpoint{2.424870in}{0.769776in}}%
\pgfpathcurveto{\pgfqpoint{2.430694in}{0.763952in}}{\pgfqpoint{2.438594in}{0.760679in}}{\pgfqpoint{2.446830in}{0.760679in}}%
\pgfpathclose%
\pgfusepath{stroke,fill}%
\end{pgfscope}%
\begin{pgfscope}%
\pgfpathrectangle{\pgfqpoint{0.100000in}{0.212622in}}{\pgfqpoint{3.696000in}{3.696000in}}%
\pgfusepath{clip}%
\pgfsetbuttcap%
\pgfsetroundjoin%
\definecolor{currentfill}{rgb}{0.121569,0.466667,0.705882}%
\pgfsetfillcolor{currentfill}%
\pgfsetfillopacity{0.998315}%
\pgfsetlinewidth{1.003750pt}%
\definecolor{currentstroke}{rgb}{0.121569,0.466667,0.705882}%
\pgfsetstrokecolor{currentstroke}%
\pgfsetstrokeopacity{0.998315}%
\pgfsetdash{}{0pt}%
\pgfpathmoveto{\pgfqpoint{2.419508in}{0.765293in}}%
\pgfpathcurveto{\pgfqpoint{2.427745in}{0.765293in}}{\pgfqpoint{2.435645in}{0.768565in}}{\pgfqpoint{2.441469in}{0.774389in}}%
\pgfpathcurveto{\pgfqpoint{2.447292in}{0.780213in}}{\pgfqpoint{2.450565in}{0.788113in}}{\pgfqpoint{2.450565in}{0.796350in}}%
\pgfpathcurveto{\pgfqpoint{2.450565in}{0.804586in}}{\pgfqpoint{2.447292in}{0.812486in}}{\pgfqpoint{2.441469in}{0.818310in}}%
\pgfpathcurveto{\pgfqpoint{2.435645in}{0.824134in}}{\pgfqpoint{2.427745in}{0.827406in}}{\pgfqpoint{2.419508in}{0.827406in}}%
\pgfpathcurveto{\pgfqpoint{2.411272in}{0.827406in}}{\pgfqpoint{2.403372in}{0.824134in}}{\pgfqpoint{2.397548in}{0.818310in}}%
\pgfpathcurveto{\pgfqpoint{2.391724in}{0.812486in}}{\pgfqpoint{2.388452in}{0.804586in}}{\pgfqpoint{2.388452in}{0.796350in}}%
\pgfpathcurveto{\pgfqpoint{2.388452in}{0.788113in}}{\pgfqpoint{2.391724in}{0.780213in}}{\pgfqpoint{2.397548in}{0.774389in}}%
\pgfpathcurveto{\pgfqpoint{2.403372in}{0.768565in}}{\pgfqpoint{2.411272in}{0.765293in}}{\pgfqpoint{2.419508in}{0.765293in}}%
\pgfpathclose%
\pgfusepath{stroke,fill}%
\end{pgfscope}%
\begin{pgfscope}%
\pgfpathrectangle{\pgfqpoint{0.100000in}{0.212622in}}{\pgfqpoint{3.696000in}{3.696000in}}%
\pgfusepath{clip}%
\pgfsetbuttcap%
\pgfsetroundjoin%
\definecolor{currentfill}{rgb}{0.121569,0.466667,0.705882}%
\pgfsetfillcolor{currentfill}%
\pgfsetfillopacity{0.998397}%
\pgfsetlinewidth{1.003750pt}%
\definecolor{currentstroke}{rgb}{0.121569,0.466667,0.705882}%
\pgfsetstrokecolor{currentstroke}%
\pgfsetstrokeopacity{0.998397}%
\pgfsetdash{}{0pt}%
\pgfpathmoveto{\pgfqpoint{2.446193in}{0.759361in}}%
\pgfpathcurveto{\pgfqpoint{2.454430in}{0.759361in}}{\pgfqpoint{2.462330in}{0.762633in}}{\pgfqpoint{2.468154in}{0.768457in}}%
\pgfpathcurveto{\pgfqpoint{2.473978in}{0.774281in}}{\pgfqpoint{2.477250in}{0.782181in}}{\pgfqpoint{2.477250in}{0.790417in}}%
\pgfpathcurveto{\pgfqpoint{2.477250in}{0.798654in}}{\pgfqpoint{2.473978in}{0.806554in}}{\pgfqpoint{2.468154in}{0.812378in}}%
\pgfpathcurveto{\pgfqpoint{2.462330in}{0.818202in}}{\pgfqpoint{2.454430in}{0.821474in}}{\pgfqpoint{2.446193in}{0.821474in}}%
\pgfpathcurveto{\pgfqpoint{2.437957in}{0.821474in}}{\pgfqpoint{2.430057in}{0.818202in}}{\pgfqpoint{2.424233in}{0.812378in}}%
\pgfpathcurveto{\pgfqpoint{2.418409in}{0.806554in}}{\pgfqpoint{2.415137in}{0.798654in}}{\pgfqpoint{2.415137in}{0.790417in}}%
\pgfpathcurveto{\pgfqpoint{2.415137in}{0.782181in}}{\pgfqpoint{2.418409in}{0.774281in}}{\pgfqpoint{2.424233in}{0.768457in}}%
\pgfpathcurveto{\pgfqpoint{2.430057in}{0.762633in}}{\pgfqpoint{2.437957in}{0.759361in}}{\pgfqpoint{2.446193in}{0.759361in}}%
\pgfpathclose%
\pgfusepath{stroke,fill}%
\end{pgfscope}%
\begin{pgfscope}%
\pgfpathrectangle{\pgfqpoint{0.100000in}{0.212622in}}{\pgfqpoint{3.696000in}{3.696000in}}%
\pgfusepath{clip}%
\pgfsetbuttcap%
\pgfsetroundjoin%
\definecolor{currentfill}{rgb}{0.121569,0.466667,0.705882}%
\pgfsetfillcolor{currentfill}%
\pgfsetfillopacity{0.998673}%
\pgfsetlinewidth{1.003750pt}%
\definecolor{currentstroke}{rgb}{0.121569,0.466667,0.705882}%
\pgfsetstrokecolor{currentstroke}%
\pgfsetstrokeopacity{0.998673}%
\pgfsetdash{}{0pt}%
\pgfpathmoveto{\pgfqpoint{2.422047in}{0.764098in}}%
\pgfpathcurveto{\pgfqpoint{2.430283in}{0.764098in}}{\pgfqpoint{2.438183in}{0.767370in}}{\pgfqpoint{2.444007in}{0.773194in}}%
\pgfpathcurveto{\pgfqpoint{2.449831in}{0.779018in}}{\pgfqpoint{2.453103in}{0.786918in}}{\pgfqpoint{2.453103in}{0.795154in}}%
\pgfpathcurveto{\pgfqpoint{2.453103in}{0.803390in}}{\pgfqpoint{2.449831in}{0.811291in}}{\pgfqpoint{2.444007in}{0.817114in}}%
\pgfpathcurveto{\pgfqpoint{2.438183in}{0.822938in}}{\pgfqpoint{2.430283in}{0.826211in}}{\pgfqpoint{2.422047in}{0.826211in}}%
\pgfpathcurveto{\pgfqpoint{2.413810in}{0.826211in}}{\pgfqpoint{2.405910in}{0.822938in}}{\pgfqpoint{2.400086in}{0.817114in}}%
\pgfpathcurveto{\pgfqpoint{2.394262in}{0.811291in}}{\pgfqpoint{2.390990in}{0.803390in}}{\pgfqpoint{2.390990in}{0.795154in}}%
\pgfpathcurveto{\pgfqpoint{2.390990in}{0.786918in}}{\pgfqpoint{2.394262in}{0.779018in}}{\pgfqpoint{2.400086in}{0.773194in}}%
\pgfpathcurveto{\pgfqpoint{2.405910in}{0.767370in}}{\pgfqpoint{2.413810in}{0.764098in}}{\pgfqpoint{2.422047in}{0.764098in}}%
\pgfpathclose%
\pgfusepath{stroke,fill}%
\end{pgfscope}%
\begin{pgfscope}%
\pgfpathrectangle{\pgfqpoint{0.100000in}{0.212622in}}{\pgfqpoint{3.696000in}{3.696000in}}%
\pgfusepath{clip}%
\pgfsetbuttcap%
\pgfsetroundjoin%
\definecolor{currentfill}{rgb}{0.121569,0.466667,0.705882}%
\pgfsetfillcolor{currentfill}%
\pgfsetfillopacity{0.998940}%
\pgfsetlinewidth{1.003750pt}%
\definecolor{currentstroke}{rgb}{0.121569,0.466667,0.705882}%
\pgfsetstrokecolor{currentstroke}%
\pgfsetstrokeopacity{0.998940}%
\pgfsetdash{}{0pt}%
\pgfpathmoveto{\pgfqpoint{2.445165in}{0.757736in}}%
\pgfpathcurveto{\pgfqpoint{2.453401in}{0.757736in}}{\pgfqpoint{2.461301in}{0.761008in}}{\pgfqpoint{2.467125in}{0.766832in}}%
\pgfpathcurveto{\pgfqpoint{2.472949in}{0.772656in}}{\pgfqpoint{2.476221in}{0.780556in}}{\pgfqpoint{2.476221in}{0.788792in}}%
\pgfpathcurveto{\pgfqpoint{2.476221in}{0.797029in}}{\pgfqpoint{2.472949in}{0.804929in}}{\pgfqpoint{2.467125in}{0.810753in}}%
\pgfpathcurveto{\pgfqpoint{2.461301in}{0.816577in}}{\pgfqpoint{2.453401in}{0.819849in}}{\pgfqpoint{2.445165in}{0.819849in}}%
\pgfpathcurveto{\pgfqpoint{2.436929in}{0.819849in}}{\pgfqpoint{2.429029in}{0.816577in}}{\pgfqpoint{2.423205in}{0.810753in}}%
\pgfpathcurveto{\pgfqpoint{2.417381in}{0.804929in}}{\pgfqpoint{2.414108in}{0.797029in}}{\pgfqpoint{2.414108in}{0.788792in}}%
\pgfpathcurveto{\pgfqpoint{2.414108in}{0.780556in}}{\pgfqpoint{2.417381in}{0.772656in}}{\pgfqpoint{2.423205in}{0.766832in}}%
\pgfpathcurveto{\pgfqpoint{2.429029in}{0.761008in}}{\pgfqpoint{2.436929in}{0.757736in}}{\pgfqpoint{2.445165in}{0.757736in}}%
\pgfpathclose%
\pgfusepath{stroke,fill}%
\end{pgfscope}%
\begin{pgfscope}%
\pgfpathrectangle{\pgfqpoint{0.100000in}{0.212622in}}{\pgfqpoint{3.696000in}{3.696000in}}%
\pgfusepath{clip}%
\pgfsetbuttcap%
\pgfsetroundjoin%
\definecolor{currentfill}{rgb}{0.121569,0.466667,0.705882}%
\pgfsetfillcolor{currentfill}%
\pgfsetfillopacity{0.998972}%
\pgfsetlinewidth{1.003750pt}%
\definecolor{currentstroke}{rgb}{0.121569,0.466667,0.705882}%
\pgfsetstrokecolor{currentstroke}%
\pgfsetstrokeopacity{0.998972}%
\pgfsetdash{}{0pt}%
\pgfpathmoveto{\pgfqpoint{2.424293in}{0.762984in}}%
\pgfpathcurveto{\pgfqpoint{2.432530in}{0.762984in}}{\pgfqpoint{2.440430in}{0.766256in}}{\pgfqpoint{2.446254in}{0.772080in}}%
\pgfpathcurveto{\pgfqpoint{2.452078in}{0.777904in}}{\pgfqpoint{2.455350in}{0.785804in}}{\pgfqpoint{2.455350in}{0.794040in}}%
\pgfpathcurveto{\pgfqpoint{2.455350in}{0.802277in}}{\pgfqpoint{2.452078in}{0.810177in}}{\pgfqpoint{2.446254in}{0.816001in}}%
\pgfpathcurveto{\pgfqpoint{2.440430in}{0.821824in}}{\pgfqpoint{2.432530in}{0.825097in}}{\pgfqpoint{2.424293in}{0.825097in}}%
\pgfpathcurveto{\pgfqpoint{2.416057in}{0.825097in}}{\pgfqpoint{2.408157in}{0.821824in}}{\pgfqpoint{2.402333in}{0.816001in}}%
\pgfpathcurveto{\pgfqpoint{2.396509in}{0.810177in}}{\pgfqpoint{2.393237in}{0.802277in}}{\pgfqpoint{2.393237in}{0.794040in}}%
\pgfpathcurveto{\pgfqpoint{2.393237in}{0.785804in}}{\pgfqpoint{2.396509in}{0.777904in}}{\pgfqpoint{2.402333in}{0.772080in}}%
\pgfpathcurveto{\pgfqpoint{2.408157in}{0.766256in}}{\pgfqpoint{2.416057in}{0.762984in}}{\pgfqpoint{2.424293in}{0.762984in}}%
\pgfpathclose%
\pgfusepath{stroke,fill}%
\end{pgfscope}%
\begin{pgfscope}%
\pgfpathrectangle{\pgfqpoint{0.100000in}{0.212622in}}{\pgfqpoint{3.696000in}{3.696000in}}%
\pgfusepath{clip}%
\pgfsetbuttcap%
\pgfsetroundjoin%
\definecolor{currentfill}{rgb}{0.121569,0.466667,0.705882}%
\pgfsetfillcolor{currentfill}%
\pgfsetfillopacity{0.999223}%
\pgfsetlinewidth{1.003750pt}%
\definecolor{currentstroke}{rgb}{0.121569,0.466667,0.705882}%
\pgfsetstrokecolor{currentstroke}%
\pgfsetstrokeopacity{0.999223}%
\pgfsetdash{}{0pt}%
\pgfpathmoveto{\pgfqpoint{2.426226in}{0.761988in}}%
\pgfpathcurveto{\pgfqpoint{2.434463in}{0.761988in}}{\pgfqpoint{2.442363in}{0.765261in}}{\pgfqpoint{2.448187in}{0.771085in}}%
\pgfpathcurveto{\pgfqpoint{2.454011in}{0.776909in}}{\pgfqpoint{2.457283in}{0.784809in}}{\pgfqpoint{2.457283in}{0.793045in}}%
\pgfpathcurveto{\pgfqpoint{2.457283in}{0.801281in}}{\pgfqpoint{2.454011in}{0.809181in}}{\pgfqpoint{2.448187in}{0.815005in}}%
\pgfpathcurveto{\pgfqpoint{2.442363in}{0.820829in}}{\pgfqpoint{2.434463in}{0.824101in}}{\pgfqpoint{2.426226in}{0.824101in}}%
\pgfpathcurveto{\pgfqpoint{2.417990in}{0.824101in}}{\pgfqpoint{2.410090in}{0.820829in}}{\pgfqpoint{2.404266in}{0.815005in}}%
\pgfpathcurveto{\pgfqpoint{2.398442in}{0.809181in}}{\pgfqpoint{2.395170in}{0.801281in}}{\pgfqpoint{2.395170in}{0.793045in}}%
\pgfpathcurveto{\pgfqpoint{2.395170in}{0.784809in}}{\pgfqpoint{2.398442in}{0.776909in}}{\pgfqpoint{2.404266in}{0.771085in}}%
\pgfpathcurveto{\pgfqpoint{2.410090in}{0.765261in}}{\pgfqpoint{2.417990in}{0.761988in}}{\pgfqpoint{2.426226in}{0.761988in}}%
\pgfpathclose%
\pgfusepath{stroke,fill}%
\end{pgfscope}%
\begin{pgfscope}%
\pgfpathrectangle{\pgfqpoint{0.100000in}{0.212622in}}{\pgfqpoint{3.696000in}{3.696000in}}%
\pgfusepath{clip}%
\pgfsetbuttcap%
\pgfsetroundjoin%
\definecolor{currentfill}{rgb}{0.121569,0.466667,0.705882}%
\pgfsetfillcolor{currentfill}%
\pgfsetfillopacity{0.999245}%
\pgfsetlinewidth{1.003750pt}%
\definecolor{currentstroke}{rgb}{0.121569,0.466667,0.705882}%
\pgfsetstrokecolor{currentstroke}%
\pgfsetstrokeopacity{0.999245}%
\pgfsetdash{}{0pt}%
\pgfpathmoveto{\pgfqpoint{2.444525in}{0.756972in}}%
\pgfpathcurveto{\pgfqpoint{2.452762in}{0.756972in}}{\pgfqpoint{2.460662in}{0.760244in}}{\pgfqpoint{2.466486in}{0.766068in}}%
\pgfpathcurveto{\pgfqpoint{2.472310in}{0.771892in}}{\pgfqpoint{2.475582in}{0.779792in}}{\pgfqpoint{2.475582in}{0.788028in}}%
\pgfpathcurveto{\pgfqpoint{2.475582in}{0.796264in}}{\pgfqpoint{2.472310in}{0.804165in}}{\pgfqpoint{2.466486in}{0.809988in}}%
\pgfpathcurveto{\pgfqpoint{2.460662in}{0.815812in}}{\pgfqpoint{2.452762in}{0.819085in}}{\pgfqpoint{2.444525in}{0.819085in}}%
\pgfpathcurveto{\pgfqpoint{2.436289in}{0.819085in}}{\pgfqpoint{2.428389in}{0.815812in}}{\pgfqpoint{2.422565in}{0.809988in}}%
\pgfpathcurveto{\pgfqpoint{2.416741in}{0.804165in}}{\pgfqpoint{2.413469in}{0.796264in}}{\pgfqpoint{2.413469in}{0.788028in}}%
\pgfpathcurveto{\pgfqpoint{2.413469in}{0.779792in}}{\pgfqpoint{2.416741in}{0.771892in}}{\pgfqpoint{2.422565in}{0.766068in}}%
\pgfpathcurveto{\pgfqpoint{2.428389in}{0.760244in}}{\pgfqpoint{2.436289in}{0.756972in}}{\pgfqpoint{2.444525in}{0.756972in}}%
\pgfpathclose%
\pgfusepath{stroke,fill}%
\end{pgfscope}%
\begin{pgfscope}%
\pgfpathrectangle{\pgfqpoint{0.100000in}{0.212622in}}{\pgfqpoint{3.696000in}{3.696000in}}%
\pgfusepath{clip}%
\pgfsetbuttcap%
\pgfsetroundjoin%
\definecolor{currentfill}{rgb}{0.121569,0.466667,0.705882}%
\pgfsetfillcolor{currentfill}%
\pgfsetfillopacity{0.999412}%
\pgfsetlinewidth{1.003750pt}%
\definecolor{currentstroke}{rgb}{0.121569,0.466667,0.705882}%
\pgfsetstrokecolor{currentstroke}%
\pgfsetstrokeopacity{0.999412}%
\pgfsetdash{}{0pt}%
\pgfpathmoveto{\pgfqpoint{2.444136in}{0.756615in}}%
\pgfpathcurveto{\pgfqpoint{2.452372in}{0.756615in}}{\pgfqpoint{2.460272in}{0.759888in}}{\pgfqpoint{2.466096in}{0.765712in}}%
\pgfpathcurveto{\pgfqpoint{2.471920in}{0.771536in}}{\pgfqpoint{2.475192in}{0.779436in}}{\pgfqpoint{2.475192in}{0.787672in}}%
\pgfpathcurveto{\pgfqpoint{2.475192in}{0.795908in}}{\pgfqpoint{2.471920in}{0.803808in}}{\pgfqpoint{2.466096in}{0.809632in}}%
\pgfpathcurveto{\pgfqpoint{2.460272in}{0.815456in}}{\pgfqpoint{2.452372in}{0.818728in}}{\pgfqpoint{2.444136in}{0.818728in}}%
\pgfpathcurveto{\pgfqpoint{2.435899in}{0.818728in}}{\pgfqpoint{2.427999in}{0.815456in}}{\pgfqpoint{2.422175in}{0.809632in}}%
\pgfpathcurveto{\pgfqpoint{2.416352in}{0.803808in}}{\pgfqpoint{2.413079in}{0.795908in}}{\pgfqpoint{2.413079in}{0.787672in}}%
\pgfpathcurveto{\pgfqpoint{2.413079in}{0.779436in}}{\pgfqpoint{2.416352in}{0.771536in}}{\pgfqpoint{2.422175in}{0.765712in}}%
\pgfpathcurveto{\pgfqpoint{2.427999in}{0.759888in}}{\pgfqpoint{2.435899in}{0.756615in}}{\pgfqpoint{2.444136in}{0.756615in}}%
\pgfpathclose%
\pgfusepath{stroke,fill}%
\end{pgfscope}%
\begin{pgfscope}%
\pgfpathrectangle{\pgfqpoint{0.100000in}{0.212622in}}{\pgfqpoint{3.696000in}{3.696000in}}%
\pgfusepath{clip}%
\pgfsetbuttcap%
\pgfsetroundjoin%
\definecolor{currentfill}{rgb}{0.121569,0.466667,0.705882}%
\pgfsetfillcolor{currentfill}%
\pgfsetfillopacity{0.999414}%
\pgfsetlinewidth{1.003750pt}%
\definecolor{currentstroke}{rgb}{0.121569,0.466667,0.705882}%
\pgfsetstrokecolor{currentstroke}%
\pgfsetstrokeopacity{0.999414}%
\pgfsetdash{}{0pt}%
\pgfpathmoveto{\pgfqpoint{2.427888in}{0.761112in}}%
\pgfpathcurveto{\pgfqpoint{2.436125in}{0.761112in}}{\pgfqpoint{2.444025in}{0.764385in}}{\pgfqpoint{2.449849in}{0.770209in}}%
\pgfpathcurveto{\pgfqpoint{2.455673in}{0.776033in}}{\pgfqpoint{2.458945in}{0.783933in}}{\pgfqpoint{2.458945in}{0.792169in}}%
\pgfpathcurveto{\pgfqpoint{2.458945in}{0.800405in}}{\pgfqpoint{2.455673in}{0.808305in}}{\pgfqpoint{2.449849in}{0.814129in}}%
\pgfpathcurveto{\pgfqpoint{2.444025in}{0.819953in}}{\pgfqpoint{2.436125in}{0.823225in}}{\pgfqpoint{2.427888in}{0.823225in}}%
\pgfpathcurveto{\pgfqpoint{2.419652in}{0.823225in}}{\pgfqpoint{2.411752in}{0.819953in}}{\pgfqpoint{2.405928in}{0.814129in}}%
\pgfpathcurveto{\pgfqpoint{2.400104in}{0.808305in}}{\pgfqpoint{2.396832in}{0.800405in}}{\pgfqpoint{2.396832in}{0.792169in}}%
\pgfpathcurveto{\pgfqpoint{2.396832in}{0.783933in}}{\pgfqpoint{2.400104in}{0.776033in}}{\pgfqpoint{2.405928in}{0.770209in}}%
\pgfpathcurveto{\pgfqpoint{2.411752in}{0.764385in}}{\pgfqpoint{2.419652in}{0.761112in}}{\pgfqpoint{2.427888in}{0.761112in}}%
\pgfpathclose%
\pgfusepath{stroke,fill}%
\end{pgfscope}%
\begin{pgfscope}%
\pgfpathrectangle{\pgfqpoint{0.100000in}{0.212622in}}{\pgfqpoint{3.696000in}{3.696000in}}%
\pgfusepath{clip}%
\pgfsetbuttcap%
\pgfsetroundjoin%
\definecolor{currentfill}{rgb}{0.121569,0.466667,0.705882}%
\pgfsetfillcolor{currentfill}%
\pgfsetfillopacity{0.999498}%
\pgfsetlinewidth{1.003750pt}%
\definecolor{currentstroke}{rgb}{0.121569,0.466667,0.705882}%
\pgfsetstrokecolor{currentstroke}%
\pgfsetstrokeopacity{0.999498}%
\pgfsetdash{}{0pt}%
\pgfpathmoveto{\pgfqpoint{2.443904in}{0.756446in}}%
\pgfpathcurveto{\pgfqpoint{2.452140in}{0.756446in}}{\pgfqpoint{2.460040in}{0.759719in}}{\pgfqpoint{2.465864in}{0.765543in}}%
\pgfpathcurveto{\pgfqpoint{2.471688in}{0.771367in}}{\pgfqpoint{2.474960in}{0.779267in}}{\pgfqpoint{2.474960in}{0.787503in}}%
\pgfpathcurveto{\pgfqpoint{2.474960in}{0.795739in}}{\pgfqpoint{2.471688in}{0.803639in}}{\pgfqpoint{2.465864in}{0.809463in}}%
\pgfpathcurveto{\pgfqpoint{2.460040in}{0.815287in}}{\pgfqpoint{2.452140in}{0.818559in}}{\pgfqpoint{2.443904in}{0.818559in}}%
\pgfpathcurveto{\pgfqpoint{2.435667in}{0.818559in}}{\pgfqpoint{2.427767in}{0.815287in}}{\pgfqpoint{2.421943in}{0.809463in}}%
\pgfpathcurveto{\pgfqpoint{2.416119in}{0.803639in}}{\pgfqpoint{2.412847in}{0.795739in}}{\pgfqpoint{2.412847in}{0.787503in}}%
\pgfpathcurveto{\pgfqpoint{2.412847in}{0.779267in}}{\pgfqpoint{2.416119in}{0.771367in}}{\pgfqpoint{2.421943in}{0.765543in}}%
\pgfpathcurveto{\pgfqpoint{2.427767in}{0.759719in}}{\pgfqpoint{2.435667in}{0.756446in}}{\pgfqpoint{2.443904in}{0.756446in}}%
\pgfpathclose%
\pgfusepath{stroke,fill}%
\end{pgfscope}%
\begin{pgfscope}%
\pgfpathrectangle{\pgfqpoint{0.100000in}{0.212622in}}{\pgfqpoint{3.696000in}{3.696000in}}%
\pgfusepath{clip}%
\pgfsetbuttcap%
\pgfsetroundjoin%
\definecolor{currentfill}{rgb}{0.121569,0.466667,0.705882}%
\pgfsetfillcolor{currentfill}%
\pgfsetfillopacity{0.999544}%
\pgfsetlinewidth{1.003750pt}%
\definecolor{currentstroke}{rgb}{0.121569,0.466667,0.705882}%
\pgfsetstrokecolor{currentstroke}%
\pgfsetstrokeopacity{0.999544}%
\pgfsetdash{}{0pt}%
\pgfpathmoveto{\pgfqpoint{2.443768in}{0.756369in}}%
\pgfpathcurveto{\pgfqpoint{2.452005in}{0.756369in}}{\pgfqpoint{2.459905in}{0.759641in}}{\pgfqpoint{2.465729in}{0.765465in}}%
\pgfpathcurveto{\pgfqpoint{2.471553in}{0.771289in}}{\pgfqpoint{2.474825in}{0.779189in}}{\pgfqpoint{2.474825in}{0.787426in}}%
\pgfpathcurveto{\pgfqpoint{2.474825in}{0.795662in}}{\pgfqpoint{2.471553in}{0.803562in}}{\pgfqpoint{2.465729in}{0.809386in}}%
\pgfpathcurveto{\pgfqpoint{2.459905in}{0.815210in}}{\pgfqpoint{2.452005in}{0.818482in}}{\pgfqpoint{2.443768in}{0.818482in}}%
\pgfpathcurveto{\pgfqpoint{2.435532in}{0.818482in}}{\pgfqpoint{2.427632in}{0.815210in}}{\pgfqpoint{2.421808in}{0.809386in}}%
\pgfpathcurveto{\pgfqpoint{2.415984in}{0.803562in}}{\pgfqpoint{2.412712in}{0.795662in}}{\pgfqpoint{2.412712in}{0.787426in}}%
\pgfpathcurveto{\pgfqpoint{2.412712in}{0.779189in}}{\pgfqpoint{2.415984in}{0.771289in}}{\pgfqpoint{2.421808in}{0.765465in}}%
\pgfpathcurveto{\pgfqpoint{2.427632in}{0.759641in}}{\pgfqpoint{2.435532in}{0.756369in}}{\pgfqpoint{2.443768in}{0.756369in}}%
\pgfpathclose%
\pgfusepath{stroke,fill}%
\end{pgfscope}%
\begin{pgfscope}%
\pgfpathrectangle{\pgfqpoint{0.100000in}{0.212622in}}{\pgfqpoint{3.696000in}{3.696000in}}%
\pgfusepath{clip}%
\pgfsetbuttcap%
\pgfsetroundjoin%
\definecolor{currentfill}{rgb}{0.121569,0.466667,0.705882}%
\pgfsetfillcolor{currentfill}%
\pgfsetfillopacity{0.999562}%
\pgfsetlinewidth{1.003750pt}%
\definecolor{currentstroke}{rgb}{0.121569,0.466667,0.705882}%
\pgfsetstrokecolor{currentstroke}%
\pgfsetstrokeopacity{0.999562}%
\pgfsetdash{}{0pt}%
\pgfpathmoveto{\pgfqpoint{2.429266in}{0.760417in}}%
\pgfpathcurveto{\pgfqpoint{2.437502in}{0.760417in}}{\pgfqpoint{2.445402in}{0.763690in}}{\pgfqpoint{2.451226in}{0.769514in}}%
\pgfpathcurveto{\pgfqpoint{2.457050in}{0.775338in}}{\pgfqpoint{2.460322in}{0.783238in}}{\pgfqpoint{2.460322in}{0.791474in}}%
\pgfpathcurveto{\pgfqpoint{2.460322in}{0.799710in}}{\pgfqpoint{2.457050in}{0.807610in}}{\pgfqpoint{2.451226in}{0.813434in}}%
\pgfpathcurveto{\pgfqpoint{2.445402in}{0.819258in}}{\pgfqpoint{2.437502in}{0.822530in}}{\pgfqpoint{2.429266in}{0.822530in}}%
\pgfpathcurveto{\pgfqpoint{2.421030in}{0.822530in}}{\pgfqpoint{2.413130in}{0.819258in}}{\pgfqpoint{2.407306in}{0.813434in}}%
\pgfpathcurveto{\pgfqpoint{2.401482in}{0.807610in}}{\pgfqpoint{2.398209in}{0.799710in}}{\pgfqpoint{2.398209in}{0.791474in}}%
\pgfpathcurveto{\pgfqpoint{2.398209in}{0.783238in}}{\pgfqpoint{2.401482in}{0.775338in}}{\pgfqpoint{2.407306in}{0.769514in}}%
\pgfpathcurveto{\pgfqpoint{2.413130in}{0.763690in}}{\pgfqpoint{2.421030in}{0.760417in}}{\pgfqpoint{2.429266in}{0.760417in}}%
\pgfpathclose%
\pgfusepath{stroke,fill}%
\end{pgfscope}%
\begin{pgfscope}%
\pgfpathrectangle{\pgfqpoint{0.100000in}{0.212622in}}{\pgfqpoint{3.696000in}{3.696000in}}%
\pgfusepath{clip}%
\pgfsetbuttcap%
\pgfsetroundjoin%
\definecolor{currentfill}{rgb}{0.121569,0.466667,0.705882}%
\pgfsetfillcolor{currentfill}%
\pgfsetfillopacity{0.999567}%
\pgfsetlinewidth{1.003750pt}%
\definecolor{currentstroke}{rgb}{0.121569,0.466667,0.705882}%
\pgfsetstrokecolor{currentstroke}%
\pgfsetstrokeopacity{0.999567}%
\pgfsetdash{}{0pt}%
\pgfpathmoveto{\pgfqpoint{2.443691in}{0.756333in}}%
\pgfpathcurveto{\pgfqpoint{2.451927in}{0.756333in}}{\pgfqpoint{2.459827in}{0.759605in}}{\pgfqpoint{2.465651in}{0.765429in}}%
\pgfpathcurveto{\pgfqpoint{2.471475in}{0.771253in}}{\pgfqpoint{2.474747in}{0.779153in}}{\pgfqpoint{2.474747in}{0.787390in}}%
\pgfpathcurveto{\pgfqpoint{2.474747in}{0.795626in}}{\pgfqpoint{2.471475in}{0.803526in}}{\pgfqpoint{2.465651in}{0.809350in}}%
\pgfpathcurveto{\pgfqpoint{2.459827in}{0.815174in}}{\pgfqpoint{2.451927in}{0.818446in}}{\pgfqpoint{2.443691in}{0.818446in}}%
\pgfpathcurveto{\pgfqpoint{2.435454in}{0.818446in}}{\pgfqpoint{2.427554in}{0.815174in}}{\pgfqpoint{2.421730in}{0.809350in}}%
\pgfpathcurveto{\pgfqpoint{2.415906in}{0.803526in}}{\pgfqpoint{2.412634in}{0.795626in}}{\pgfqpoint{2.412634in}{0.787390in}}%
\pgfpathcurveto{\pgfqpoint{2.412634in}{0.779153in}}{\pgfqpoint{2.415906in}{0.771253in}}{\pgfqpoint{2.421730in}{0.765429in}}%
\pgfpathcurveto{\pgfqpoint{2.427554in}{0.759605in}}{\pgfqpoint{2.435454in}{0.756333in}}{\pgfqpoint{2.443691in}{0.756333in}}%
\pgfpathclose%
\pgfusepath{stroke,fill}%
\end{pgfscope}%
\begin{pgfscope}%
\pgfpathrectangle{\pgfqpoint{0.100000in}{0.212622in}}{\pgfqpoint{3.696000in}{3.696000in}}%
\pgfusepath{clip}%
\pgfsetbuttcap%
\pgfsetroundjoin%
\definecolor{currentfill}{rgb}{0.121569,0.466667,0.705882}%
\pgfsetfillcolor{currentfill}%
\pgfsetfillopacity{0.999579}%
\pgfsetlinewidth{1.003750pt}%
\definecolor{currentstroke}{rgb}{0.121569,0.466667,0.705882}%
\pgfsetstrokecolor{currentstroke}%
\pgfsetstrokeopacity{0.999579}%
\pgfsetdash{}{0pt}%
\pgfpathmoveto{\pgfqpoint{2.443646in}{0.756318in}}%
\pgfpathcurveto{\pgfqpoint{2.451882in}{0.756318in}}{\pgfqpoint{2.459782in}{0.759590in}}{\pgfqpoint{2.465606in}{0.765414in}}%
\pgfpathcurveto{\pgfqpoint{2.471430in}{0.771238in}}{\pgfqpoint{2.474703in}{0.779138in}}{\pgfqpoint{2.474703in}{0.787375in}}%
\pgfpathcurveto{\pgfqpoint{2.474703in}{0.795611in}}{\pgfqpoint{2.471430in}{0.803511in}}{\pgfqpoint{2.465606in}{0.809335in}}%
\pgfpathcurveto{\pgfqpoint{2.459782in}{0.815159in}}{\pgfqpoint{2.451882in}{0.818431in}}{\pgfqpoint{2.443646in}{0.818431in}}%
\pgfpathcurveto{\pgfqpoint{2.435410in}{0.818431in}}{\pgfqpoint{2.427510in}{0.815159in}}{\pgfqpoint{2.421686in}{0.809335in}}%
\pgfpathcurveto{\pgfqpoint{2.415862in}{0.803511in}}{\pgfqpoint{2.412590in}{0.795611in}}{\pgfqpoint{2.412590in}{0.787375in}}%
\pgfpathcurveto{\pgfqpoint{2.412590in}{0.779138in}}{\pgfqpoint{2.415862in}{0.771238in}}{\pgfqpoint{2.421686in}{0.765414in}}%
\pgfpathcurveto{\pgfqpoint{2.427510in}{0.759590in}}{\pgfqpoint{2.435410in}{0.756318in}}{\pgfqpoint{2.443646in}{0.756318in}}%
\pgfpathclose%
\pgfusepath{stroke,fill}%
\end{pgfscope}%
\begin{pgfscope}%
\pgfpathrectangle{\pgfqpoint{0.100000in}{0.212622in}}{\pgfqpoint{3.696000in}{3.696000in}}%
\pgfusepath{clip}%
\pgfsetbuttcap%
\pgfsetroundjoin%
\definecolor{currentfill}{rgb}{0.121569,0.466667,0.705882}%
\pgfsetfillcolor{currentfill}%
\pgfsetfillopacity{0.999584}%
\pgfsetlinewidth{1.003750pt}%
\definecolor{currentstroke}{rgb}{0.121569,0.466667,0.705882}%
\pgfsetstrokecolor{currentstroke}%
\pgfsetstrokeopacity{0.999584}%
\pgfsetdash{}{0pt}%
\pgfpathmoveto{\pgfqpoint{2.443621in}{0.756312in}}%
\pgfpathcurveto{\pgfqpoint{2.451857in}{0.756312in}}{\pgfqpoint{2.459757in}{0.759584in}}{\pgfqpoint{2.465581in}{0.765408in}}%
\pgfpathcurveto{\pgfqpoint{2.471405in}{0.771232in}}{\pgfqpoint{2.474678in}{0.779132in}}{\pgfqpoint{2.474678in}{0.787368in}}%
\pgfpathcurveto{\pgfqpoint{2.474678in}{0.795604in}}{\pgfqpoint{2.471405in}{0.803505in}}{\pgfqpoint{2.465581in}{0.809328in}}%
\pgfpathcurveto{\pgfqpoint{2.459757in}{0.815152in}}{\pgfqpoint{2.451857in}{0.818425in}}{\pgfqpoint{2.443621in}{0.818425in}}%
\pgfpathcurveto{\pgfqpoint{2.435385in}{0.818425in}}{\pgfqpoint{2.427485in}{0.815152in}}{\pgfqpoint{2.421661in}{0.809328in}}%
\pgfpathcurveto{\pgfqpoint{2.415837in}{0.803505in}}{\pgfqpoint{2.412565in}{0.795604in}}{\pgfqpoint{2.412565in}{0.787368in}}%
\pgfpathcurveto{\pgfqpoint{2.412565in}{0.779132in}}{\pgfqpoint{2.415837in}{0.771232in}}{\pgfqpoint{2.421661in}{0.765408in}}%
\pgfpathcurveto{\pgfqpoint{2.427485in}{0.759584in}}{\pgfqpoint{2.435385in}{0.756312in}}{\pgfqpoint{2.443621in}{0.756312in}}%
\pgfpathclose%
\pgfusepath{stroke,fill}%
\end{pgfscope}%
\begin{pgfscope}%
\pgfpathrectangle{\pgfqpoint{0.100000in}{0.212622in}}{\pgfqpoint{3.696000in}{3.696000in}}%
\pgfusepath{clip}%
\pgfsetbuttcap%
\pgfsetroundjoin%
\definecolor{currentfill}{rgb}{0.121569,0.466667,0.705882}%
\pgfsetfillcolor{currentfill}%
\pgfsetfillopacity{0.999587}%
\pgfsetlinewidth{1.003750pt}%
\definecolor{currentstroke}{rgb}{0.121569,0.466667,0.705882}%
\pgfsetstrokecolor{currentstroke}%
\pgfsetstrokeopacity{0.999587}%
\pgfsetdash{}{0pt}%
\pgfpathmoveto{\pgfqpoint{2.443607in}{0.756309in}}%
\pgfpathcurveto{\pgfqpoint{2.451843in}{0.756309in}}{\pgfqpoint{2.459743in}{0.759582in}}{\pgfqpoint{2.465567in}{0.765406in}}%
\pgfpathcurveto{\pgfqpoint{2.471391in}{0.771230in}}{\pgfqpoint{2.474663in}{0.779130in}}{\pgfqpoint{2.474663in}{0.787366in}}%
\pgfpathcurveto{\pgfqpoint{2.474663in}{0.795602in}}{\pgfqpoint{2.471391in}{0.803502in}}{\pgfqpoint{2.465567in}{0.809326in}}%
\pgfpathcurveto{\pgfqpoint{2.459743in}{0.815150in}}{\pgfqpoint{2.451843in}{0.818422in}}{\pgfqpoint{2.443607in}{0.818422in}}%
\pgfpathcurveto{\pgfqpoint{2.435371in}{0.818422in}}{\pgfqpoint{2.427471in}{0.815150in}}{\pgfqpoint{2.421647in}{0.809326in}}%
\pgfpathcurveto{\pgfqpoint{2.415823in}{0.803502in}}{\pgfqpoint{2.412550in}{0.795602in}}{\pgfqpoint{2.412550in}{0.787366in}}%
\pgfpathcurveto{\pgfqpoint{2.412550in}{0.779130in}}{\pgfqpoint{2.415823in}{0.771230in}}{\pgfqpoint{2.421647in}{0.765406in}}%
\pgfpathcurveto{\pgfqpoint{2.427471in}{0.759582in}}{\pgfqpoint{2.435371in}{0.756309in}}{\pgfqpoint{2.443607in}{0.756309in}}%
\pgfpathclose%
\pgfusepath{stroke,fill}%
\end{pgfscope}%
\begin{pgfscope}%
\pgfpathrectangle{\pgfqpoint{0.100000in}{0.212622in}}{\pgfqpoint{3.696000in}{3.696000in}}%
\pgfusepath{clip}%
\pgfsetbuttcap%
\pgfsetroundjoin%
\definecolor{currentfill}{rgb}{0.121569,0.466667,0.705882}%
\pgfsetfillcolor{currentfill}%
\pgfsetfillopacity{0.999588}%
\pgfsetlinewidth{1.003750pt}%
\definecolor{currentstroke}{rgb}{0.121569,0.466667,0.705882}%
\pgfsetstrokecolor{currentstroke}%
\pgfsetstrokeopacity{0.999588}%
\pgfsetdash{}{0pt}%
\pgfpathmoveto{\pgfqpoint{2.443599in}{0.756309in}}%
\pgfpathcurveto{\pgfqpoint{2.451835in}{0.756309in}}{\pgfqpoint{2.459735in}{0.759581in}}{\pgfqpoint{2.465559in}{0.765405in}}%
\pgfpathcurveto{\pgfqpoint{2.471383in}{0.771229in}}{\pgfqpoint{2.474656in}{0.779129in}}{\pgfqpoint{2.474656in}{0.787365in}}%
\pgfpathcurveto{\pgfqpoint{2.474656in}{0.795601in}}{\pgfqpoint{2.471383in}{0.803501in}}{\pgfqpoint{2.465559in}{0.809325in}}%
\pgfpathcurveto{\pgfqpoint{2.459735in}{0.815149in}}{\pgfqpoint{2.451835in}{0.818422in}}{\pgfqpoint{2.443599in}{0.818422in}}%
\pgfpathcurveto{\pgfqpoint{2.435363in}{0.818422in}}{\pgfqpoint{2.427463in}{0.815149in}}{\pgfqpoint{2.421639in}{0.809325in}}%
\pgfpathcurveto{\pgfqpoint{2.415815in}{0.803501in}}{\pgfqpoint{2.412543in}{0.795601in}}{\pgfqpoint{2.412543in}{0.787365in}}%
\pgfpathcurveto{\pgfqpoint{2.412543in}{0.779129in}}{\pgfqpoint{2.415815in}{0.771229in}}{\pgfqpoint{2.421639in}{0.765405in}}%
\pgfpathcurveto{\pgfqpoint{2.427463in}{0.759581in}}{\pgfqpoint{2.435363in}{0.756309in}}{\pgfqpoint{2.443599in}{0.756309in}}%
\pgfpathclose%
\pgfusepath{stroke,fill}%
\end{pgfscope}%
\begin{pgfscope}%
\pgfpathrectangle{\pgfqpoint{0.100000in}{0.212622in}}{\pgfqpoint{3.696000in}{3.696000in}}%
\pgfusepath{clip}%
\pgfsetbuttcap%
\pgfsetroundjoin%
\definecolor{currentfill}{rgb}{0.121569,0.466667,0.705882}%
\pgfsetfillcolor{currentfill}%
\pgfsetfillopacity{0.999633}%
\pgfsetlinewidth{1.003750pt}%
\definecolor{currentstroke}{rgb}{0.121569,0.466667,0.705882}%
\pgfsetstrokecolor{currentstroke}%
\pgfsetstrokeopacity{0.999633}%
\pgfsetdash{}{0pt}%
\pgfpathmoveto{\pgfqpoint{2.443304in}{0.756301in}}%
\pgfpathcurveto{\pgfqpoint{2.451541in}{0.756301in}}{\pgfqpoint{2.459441in}{0.759574in}}{\pgfqpoint{2.465265in}{0.765397in}}%
\pgfpathcurveto{\pgfqpoint{2.471089in}{0.771221in}}{\pgfqpoint{2.474361in}{0.779121in}}{\pgfqpoint{2.474361in}{0.787358in}}%
\pgfpathcurveto{\pgfqpoint{2.474361in}{0.795594in}}{\pgfqpoint{2.471089in}{0.803494in}}{\pgfqpoint{2.465265in}{0.809318in}}%
\pgfpathcurveto{\pgfqpoint{2.459441in}{0.815142in}}{\pgfqpoint{2.451541in}{0.818414in}}{\pgfqpoint{2.443304in}{0.818414in}}%
\pgfpathcurveto{\pgfqpoint{2.435068in}{0.818414in}}{\pgfqpoint{2.427168in}{0.815142in}}{\pgfqpoint{2.421344in}{0.809318in}}%
\pgfpathcurveto{\pgfqpoint{2.415520in}{0.803494in}}{\pgfqpoint{2.412248in}{0.795594in}}{\pgfqpoint{2.412248in}{0.787358in}}%
\pgfpathcurveto{\pgfqpoint{2.412248in}{0.779121in}}{\pgfqpoint{2.415520in}{0.771221in}}{\pgfqpoint{2.421344in}{0.765397in}}%
\pgfpathcurveto{\pgfqpoint{2.427168in}{0.759574in}}{\pgfqpoint{2.435068in}{0.756301in}}{\pgfqpoint{2.443304in}{0.756301in}}%
\pgfpathclose%
\pgfusepath{stroke,fill}%
\end{pgfscope}%
\begin{pgfscope}%
\pgfpathrectangle{\pgfqpoint{0.100000in}{0.212622in}}{\pgfqpoint{3.696000in}{3.696000in}}%
\pgfusepath{clip}%
\pgfsetbuttcap%
\pgfsetroundjoin%
\definecolor{currentfill}{rgb}{0.121569,0.466667,0.705882}%
\pgfsetfillcolor{currentfill}%
\pgfsetfillopacity{0.999678}%
\pgfsetlinewidth{1.003750pt}%
\definecolor{currentstroke}{rgb}{0.121569,0.466667,0.705882}%
\pgfsetstrokecolor{currentstroke}%
\pgfsetstrokeopacity{0.999678}%
\pgfsetdash{}{0pt}%
\pgfpathmoveto{\pgfqpoint{2.430405in}{0.759866in}}%
\pgfpathcurveto{\pgfqpoint{2.438642in}{0.759866in}}{\pgfqpoint{2.446542in}{0.763138in}}{\pgfqpoint{2.452366in}{0.768962in}}%
\pgfpathcurveto{\pgfqpoint{2.458190in}{0.774786in}}{\pgfqpoint{2.461462in}{0.782686in}}{\pgfqpoint{2.461462in}{0.790922in}}%
\pgfpathcurveto{\pgfqpoint{2.461462in}{0.799159in}}{\pgfqpoint{2.458190in}{0.807059in}}{\pgfqpoint{2.452366in}{0.812882in}}%
\pgfpathcurveto{\pgfqpoint{2.446542in}{0.818706in}}{\pgfqpoint{2.438642in}{0.821979in}}{\pgfqpoint{2.430405in}{0.821979in}}%
\pgfpathcurveto{\pgfqpoint{2.422169in}{0.821979in}}{\pgfqpoint{2.414269in}{0.818706in}}{\pgfqpoint{2.408445in}{0.812882in}}%
\pgfpathcurveto{\pgfqpoint{2.402621in}{0.807059in}}{\pgfqpoint{2.399349in}{0.799159in}}{\pgfqpoint{2.399349in}{0.790922in}}%
\pgfpathcurveto{\pgfqpoint{2.399349in}{0.782686in}}{\pgfqpoint{2.402621in}{0.774786in}}{\pgfqpoint{2.408445in}{0.768962in}}%
\pgfpathcurveto{\pgfqpoint{2.414269in}{0.763138in}}{\pgfqpoint{2.422169in}{0.759866in}}{\pgfqpoint{2.430405in}{0.759866in}}%
\pgfpathclose%
\pgfusepath{stroke,fill}%
\end{pgfscope}%
\begin{pgfscope}%
\pgfpathrectangle{\pgfqpoint{0.100000in}{0.212622in}}{\pgfqpoint{3.696000in}{3.696000in}}%
\pgfusepath{clip}%
\pgfsetbuttcap%
\pgfsetroundjoin%
\definecolor{currentfill}{rgb}{0.121569,0.466667,0.705882}%
\pgfsetfillcolor{currentfill}%
\pgfsetfillopacity{0.999713}%
\pgfsetlinewidth{1.003750pt}%
\definecolor{currentstroke}{rgb}{0.121569,0.466667,0.705882}%
\pgfsetstrokecolor{currentstroke}%
\pgfsetstrokeopacity{0.999713}%
\pgfsetdash{}{0pt}%
\pgfpathmoveto{\pgfqpoint{2.442713in}{0.756336in}}%
\pgfpathcurveto{\pgfqpoint{2.450949in}{0.756336in}}{\pgfqpoint{2.458849in}{0.759609in}}{\pgfqpoint{2.464673in}{0.765433in}}%
\pgfpathcurveto{\pgfqpoint{2.470497in}{0.771256in}}{\pgfqpoint{2.473769in}{0.779157in}}{\pgfqpoint{2.473769in}{0.787393in}}%
\pgfpathcurveto{\pgfqpoint{2.473769in}{0.795629in}}{\pgfqpoint{2.470497in}{0.803529in}}{\pgfqpoint{2.464673in}{0.809353in}}%
\pgfpathcurveto{\pgfqpoint{2.458849in}{0.815177in}}{\pgfqpoint{2.450949in}{0.818449in}}{\pgfqpoint{2.442713in}{0.818449in}}%
\pgfpathcurveto{\pgfqpoint{2.434477in}{0.818449in}}{\pgfqpoint{2.426577in}{0.815177in}}{\pgfqpoint{2.420753in}{0.809353in}}%
\pgfpathcurveto{\pgfqpoint{2.414929in}{0.803529in}}{\pgfqpoint{2.411656in}{0.795629in}}{\pgfqpoint{2.411656in}{0.787393in}}%
\pgfpathcurveto{\pgfqpoint{2.411656in}{0.779157in}}{\pgfqpoint{2.414929in}{0.771256in}}{\pgfqpoint{2.420753in}{0.765433in}}%
\pgfpathcurveto{\pgfqpoint{2.426577in}{0.759609in}}{\pgfqpoint{2.434477in}{0.756336in}}{\pgfqpoint{2.442713in}{0.756336in}}%
\pgfpathclose%
\pgfusepath{stroke,fill}%
\end{pgfscope}%
\begin{pgfscope}%
\pgfpathrectangle{\pgfqpoint{0.100000in}{0.212622in}}{\pgfqpoint{3.696000in}{3.696000in}}%
\pgfusepath{clip}%
\pgfsetbuttcap%
\pgfsetroundjoin%
\definecolor{currentfill}{rgb}{0.121569,0.466667,0.705882}%
\pgfsetfillcolor{currentfill}%
\pgfsetfillopacity{0.999758}%
\pgfsetlinewidth{1.003750pt}%
\definecolor{currentstroke}{rgb}{0.121569,0.466667,0.705882}%
\pgfsetstrokecolor{currentstroke}%
\pgfsetstrokeopacity{0.999758}%
\pgfsetdash{}{0pt}%
\pgfpathmoveto{\pgfqpoint{2.431291in}{0.759460in}}%
\pgfpathcurveto{\pgfqpoint{2.439527in}{0.759460in}}{\pgfqpoint{2.447427in}{0.762733in}}{\pgfqpoint{2.453251in}{0.768556in}}%
\pgfpathcurveto{\pgfqpoint{2.459075in}{0.774380in}}{\pgfqpoint{2.462347in}{0.782280in}}{\pgfqpoint{2.462347in}{0.790517in}}%
\pgfpathcurveto{\pgfqpoint{2.462347in}{0.798753in}}{\pgfqpoint{2.459075in}{0.806653in}}{\pgfqpoint{2.453251in}{0.812477in}}%
\pgfpathcurveto{\pgfqpoint{2.447427in}{0.818301in}}{\pgfqpoint{2.439527in}{0.821573in}}{\pgfqpoint{2.431291in}{0.821573in}}%
\pgfpathcurveto{\pgfqpoint{2.423054in}{0.821573in}}{\pgfqpoint{2.415154in}{0.818301in}}{\pgfqpoint{2.409330in}{0.812477in}}%
\pgfpathcurveto{\pgfqpoint{2.403506in}{0.806653in}}{\pgfqpoint{2.400234in}{0.798753in}}{\pgfqpoint{2.400234in}{0.790517in}}%
\pgfpathcurveto{\pgfqpoint{2.400234in}{0.782280in}}{\pgfqpoint{2.403506in}{0.774380in}}{\pgfqpoint{2.409330in}{0.768556in}}%
\pgfpathcurveto{\pgfqpoint{2.415154in}{0.762733in}}{\pgfqpoint{2.423054in}{0.759460in}}{\pgfqpoint{2.431291in}{0.759460in}}%
\pgfpathclose%
\pgfusepath{stroke,fill}%
\end{pgfscope}%
\begin{pgfscope}%
\pgfpathrectangle{\pgfqpoint{0.100000in}{0.212622in}}{\pgfqpoint{3.696000in}{3.696000in}}%
\pgfusepath{clip}%
\pgfsetbuttcap%
\pgfsetroundjoin%
\definecolor{currentfill}{rgb}{0.121569,0.466667,0.705882}%
\pgfsetfillcolor{currentfill}%
\pgfsetfillopacity{0.999809}%
\pgfsetlinewidth{1.003750pt}%
\definecolor{currentstroke}{rgb}{0.121569,0.466667,0.705882}%
\pgfsetstrokecolor{currentstroke}%
\pgfsetstrokeopacity{0.999809}%
\pgfsetdash{}{0pt}%
\pgfpathmoveto{\pgfqpoint{2.431943in}{0.759195in}}%
\pgfpathcurveto{\pgfqpoint{2.440179in}{0.759195in}}{\pgfqpoint{2.448079in}{0.762467in}}{\pgfqpoint{2.453903in}{0.768291in}}%
\pgfpathcurveto{\pgfqpoint{2.459727in}{0.774115in}}{\pgfqpoint{2.462999in}{0.782015in}}{\pgfqpoint{2.462999in}{0.790251in}}%
\pgfpathcurveto{\pgfqpoint{2.462999in}{0.798487in}}{\pgfqpoint{2.459727in}{0.806387in}}{\pgfqpoint{2.453903in}{0.812211in}}%
\pgfpathcurveto{\pgfqpoint{2.448079in}{0.818035in}}{\pgfqpoint{2.440179in}{0.821308in}}{\pgfqpoint{2.431943in}{0.821308in}}%
\pgfpathcurveto{\pgfqpoint{2.423706in}{0.821308in}}{\pgfqpoint{2.415806in}{0.818035in}}{\pgfqpoint{2.409982in}{0.812211in}}%
\pgfpathcurveto{\pgfqpoint{2.404158in}{0.806387in}}{\pgfqpoint{2.400886in}{0.798487in}}{\pgfqpoint{2.400886in}{0.790251in}}%
\pgfpathcurveto{\pgfqpoint{2.400886in}{0.782015in}}{\pgfqpoint{2.404158in}{0.774115in}}{\pgfqpoint{2.409982in}{0.768291in}}%
\pgfpathcurveto{\pgfqpoint{2.415806in}{0.762467in}}{\pgfqpoint{2.423706in}{0.759195in}}{\pgfqpoint{2.431943in}{0.759195in}}%
\pgfpathclose%
\pgfusepath{stroke,fill}%
\end{pgfscope}%
\begin{pgfscope}%
\pgfpathrectangle{\pgfqpoint{0.100000in}{0.212622in}}{\pgfqpoint{3.696000in}{3.696000in}}%
\pgfusepath{clip}%
\pgfsetbuttcap%
\pgfsetroundjoin%
\definecolor{currentfill}{rgb}{0.121569,0.466667,0.705882}%
\pgfsetfillcolor{currentfill}%
\pgfsetfillopacity{0.999816}%
\pgfsetlinewidth{1.003750pt}%
\definecolor{currentstroke}{rgb}{0.121569,0.466667,0.705882}%
\pgfsetstrokecolor{currentstroke}%
\pgfsetstrokeopacity{0.999816}%
\pgfsetdash{}{0pt}%
\pgfpathmoveto{\pgfqpoint{2.441703in}{0.756434in}}%
\pgfpathcurveto{\pgfqpoint{2.449939in}{0.756434in}}{\pgfqpoint{2.457839in}{0.759706in}}{\pgfqpoint{2.463663in}{0.765530in}}%
\pgfpathcurveto{\pgfqpoint{2.469487in}{0.771354in}}{\pgfqpoint{2.472760in}{0.779254in}}{\pgfqpoint{2.472760in}{0.787490in}}%
\pgfpathcurveto{\pgfqpoint{2.472760in}{0.795727in}}{\pgfqpoint{2.469487in}{0.803627in}}{\pgfqpoint{2.463663in}{0.809451in}}%
\pgfpathcurveto{\pgfqpoint{2.457839in}{0.815275in}}{\pgfqpoint{2.449939in}{0.818547in}}{\pgfqpoint{2.441703in}{0.818547in}}%
\pgfpathcurveto{\pgfqpoint{2.433467in}{0.818547in}}{\pgfqpoint{2.425567in}{0.815275in}}{\pgfqpoint{2.419743in}{0.809451in}}%
\pgfpathcurveto{\pgfqpoint{2.413919in}{0.803627in}}{\pgfqpoint{2.410647in}{0.795727in}}{\pgfqpoint{2.410647in}{0.787490in}}%
\pgfpathcurveto{\pgfqpoint{2.410647in}{0.779254in}}{\pgfqpoint{2.413919in}{0.771354in}}{\pgfqpoint{2.419743in}{0.765530in}}%
\pgfpathcurveto{\pgfqpoint{2.425567in}{0.759706in}}{\pgfqpoint{2.433467in}{0.756434in}}{\pgfqpoint{2.441703in}{0.756434in}}%
\pgfpathclose%
\pgfusepath{stroke,fill}%
\end{pgfscope}%
\begin{pgfscope}%
\pgfpathrectangle{\pgfqpoint{0.100000in}{0.212622in}}{\pgfqpoint{3.696000in}{3.696000in}}%
\pgfusepath{clip}%
\pgfsetbuttcap%
\pgfsetroundjoin%
\definecolor{currentfill}{rgb}{0.121569,0.466667,0.705882}%
\pgfsetfillcolor{currentfill}%
\pgfsetfillopacity{0.999884}%
\pgfsetlinewidth{1.003750pt}%
\definecolor{currentstroke}{rgb}{0.121569,0.466667,0.705882}%
\pgfsetstrokecolor{currentstroke}%
\pgfsetstrokeopacity{0.999884}%
\pgfsetdash{}{0pt}%
\pgfpathmoveto{\pgfqpoint{2.433146in}{0.758739in}}%
\pgfpathcurveto{\pgfqpoint{2.441382in}{0.758739in}}{\pgfqpoint{2.449282in}{0.762012in}}{\pgfqpoint{2.455106in}{0.767836in}}%
\pgfpathcurveto{\pgfqpoint{2.460930in}{0.773660in}}{\pgfqpoint{2.464203in}{0.781560in}}{\pgfqpoint{2.464203in}{0.789796in}}%
\pgfpathcurveto{\pgfqpoint{2.464203in}{0.798032in}}{\pgfqpoint{2.460930in}{0.805932in}}{\pgfqpoint{2.455106in}{0.811756in}}%
\pgfpathcurveto{\pgfqpoint{2.449282in}{0.817580in}}{\pgfqpoint{2.441382in}{0.820852in}}{\pgfqpoint{2.433146in}{0.820852in}}%
\pgfpathcurveto{\pgfqpoint{2.424910in}{0.820852in}}{\pgfqpoint{2.417010in}{0.817580in}}{\pgfqpoint{2.411186in}{0.811756in}}%
\pgfpathcurveto{\pgfqpoint{2.405362in}{0.805932in}}{\pgfqpoint{2.402090in}{0.798032in}}{\pgfqpoint{2.402090in}{0.789796in}}%
\pgfpathcurveto{\pgfqpoint{2.402090in}{0.781560in}}{\pgfqpoint{2.405362in}{0.773660in}}{\pgfqpoint{2.411186in}{0.767836in}}%
\pgfpathcurveto{\pgfqpoint{2.417010in}{0.762012in}}{\pgfqpoint{2.424910in}{0.758739in}}{\pgfqpoint{2.433146in}{0.758739in}}%
\pgfpathclose%
\pgfusepath{stroke,fill}%
\end{pgfscope}%
\begin{pgfscope}%
\pgfpathrectangle{\pgfqpoint{0.100000in}{0.212622in}}{\pgfqpoint{3.696000in}{3.696000in}}%
\pgfusepath{clip}%
\pgfsetbuttcap%
\pgfsetroundjoin%
\definecolor{currentfill}{rgb}{0.121569,0.466667,0.705882}%
\pgfsetfillcolor{currentfill}%
\pgfsetfillopacity{0.999921}%
\pgfsetlinewidth{1.003750pt}%
\definecolor{currentstroke}{rgb}{0.121569,0.466667,0.705882}%
\pgfsetstrokecolor{currentstroke}%
\pgfsetstrokeopacity{0.999921}%
\pgfsetdash{}{0pt}%
\pgfpathmoveto{\pgfqpoint{2.440352in}{0.756621in}}%
\pgfpathcurveto{\pgfqpoint{2.448589in}{0.756621in}}{\pgfqpoint{2.456489in}{0.759894in}}{\pgfqpoint{2.462313in}{0.765717in}}%
\pgfpathcurveto{\pgfqpoint{2.468137in}{0.771541in}}{\pgfqpoint{2.471409in}{0.779441in}}{\pgfqpoint{2.471409in}{0.787678in}}%
\pgfpathcurveto{\pgfqpoint{2.471409in}{0.795914in}}{\pgfqpoint{2.468137in}{0.803814in}}{\pgfqpoint{2.462313in}{0.809638in}}%
\pgfpathcurveto{\pgfqpoint{2.456489in}{0.815462in}}{\pgfqpoint{2.448589in}{0.818734in}}{\pgfqpoint{2.440352in}{0.818734in}}%
\pgfpathcurveto{\pgfqpoint{2.432116in}{0.818734in}}{\pgfqpoint{2.424216in}{0.815462in}}{\pgfqpoint{2.418392in}{0.809638in}}%
\pgfpathcurveto{\pgfqpoint{2.412568in}{0.803814in}}{\pgfqpoint{2.409296in}{0.795914in}}{\pgfqpoint{2.409296in}{0.787678in}}%
\pgfpathcurveto{\pgfqpoint{2.409296in}{0.779441in}}{\pgfqpoint{2.412568in}{0.771541in}}{\pgfqpoint{2.418392in}{0.765717in}}%
\pgfpathcurveto{\pgfqpoint{2.424216in}{0.759894in}}{\pgfqpoint{2.432116in}{0.756621in}}{\pgfqpoint{2.440352in}{0.756621in}}%
\pgfpathclose%
\pgfusepath{stroke,fill}%
\end{pgfscope}%
\begin{pgfscope}%
\pgfpathrectangle{\pgfqpoint{0.100000in}{0.212622in}}{\pgfqpoint{3.696000in}{3.696000in}}%
\pgfusepath{clip}%
\pgfsetbuttcap%
\pgfsetroundjoin%
\definecolor{currentfill}{rgb}{0.121569,0.466667,0.705882}%
\pgfsetfillcolor{currentfill}%
\pgfsetfillopacity{0.999928}%
\pgfsetlinewidth{1.003750pt}%
\definecolor{currentstroke}{rgb}{0.121569,0.466667,0.705882}%
\pgfsetstrokecolor{currentstroke}%
\pgfsetstrokeopacity{0.999928}%
\pgfsetdash{}{0pt}%
\pgfpathmoveto{\pgfqpoint{2.434102in}{0.758385in}}%
\pgfpathcurveto{\pgfqpoint{2.442338in}{0.758385in}}{\pgfqpoint{2.450238in}{0.761658in}}{\pgfqpoint{2.456062in}{0.767482in}}%
\pgfpathcurveto{\pgfqpoint{2.461886in}{0.773306in}}{\pgfqpoint{2.465159in}{0.781206in}}{\pgfqpoint{2.465159in}{0.789442in}}%
\pgfpathcurveto{\pgfqpoint{2.465159in}{0.797678in}}{\pgfqpoint{2.461886in}{0.805578in}}{\pgfqpoint{2.456062in}{0.811402in}}%
\pgfpathcurveto{\pgfqpoint{2.450238in}{0.817226in}}{\pgfqpoint{2.442338in}{0.820498in}}{\pgfqpoint{2.434102in}{0.820498in}}%
\pgfpathcurveto{\pgfqpoint{2.425866in}{0.820498in}}{\pgfqpoint{2.417966in}{0.817226in}}{\pgfqpoint{2.412142in}{0.811402in}}%
\pgfpathcurveto{\pgfqpoint{2.406318in}{0.805578in}}{\pgfqpoint{2.403046in}{0.797678in}}{\pgfqpoint{2.403046in}{0.789442in}}%
\pgfpathcurveto{\pgfqpoint{2.403046in}{0.781206in}}{\pgfqpoint{2.406318in}{0.773306in}}{\pgfqpoint{2.412142in}{0.767482in}}%
\pgfpathcurveto{\pgfqpoint{2.417966in}{0.761658in}}{\pgfqpoint{2.425866in}{0.758385in}}{\pgfqpoint{2.434102in}{0.758385in}}%
\pgfpathclose%
\pgfusepath{stroke,fill}%
\end{pgfscope}%
\begin{pgfscope}%
\pgfpathrectangle{\pgfqpoint{0.100000in}{0.212622in}}{\pgfqpoint{3.696000in}{3.696000in}}%
\pgfusepath{clip}%
\pgfsetbuttcap%
\pgfsetroundjoin%
\definecolor{currentfill}{rgb}{0.121569,0.466667,0.705882}%
\pgfsetfillcolor{currentfill}%
\pgfsetfillopacity{0.999959}%
\pgfsetlinewidth{1.003750pt}%
\definecolor{currentstroke}{rgb}{0.121569,0.466667,0.705882}%
\pgfsetstrokecolor{currentstroke}%
\pgfsetstrokeopacity{0.999959}%
\pgfsetdash{}{0pt}%
\pgfpathmoveto{\pgfqpoint{2.439617in}{0.756750in}}%
\pgfpathcurveto{\pgfqpoint{2.447853in}{0.756750in}}{\pgfqpoint{2.455753in}{0.760023in}}{\pgfqpoint{2.461577in}{0.765847in}}%
\pgfpathcurveto{\pgfqpoint{2.467401in}{0.771671in}}{\pgfqpoint{2.470673in}{0.779571in}}{\pgfqpoint{2.470673in}{0.787807in}}%
\pgfpathcurveto{\pgfqpoint{2.470673in}{0.796043in}}{\pgfqpoint{2.467401in}{0.803943in}}{\pgfqpoint{2.461577in}{0.809767in}}%
\pgfpathcurveto{\pgfqpoint{2.455753in}{0.815591in}}{\pgfqpoint{2.447853in}{0.818863in}}{\pgfqpoint{2.439617in}{0.818863in}}%
\pgfpathcurveto{\pgfqpoint{2.431380in}{0.818863in}}{\pgfqpoint{2.423480in}{0.815591in}}{\pgfqpoint{2.417656in}{0.809767in}}%
\pgfpathcurveto{\pgfqpoint{2.411832in}{0.803943in}}{\pgfqpoint{2.408560in}{0.796043in}}{\pgfqpoint{2.408560in}{0.787807in}}%
\pgfpathcurveto{\pgfqpoint{2.408560in}{0.779571in}}{\pgfqpoint{2.411832in}{0.771671in}}{\pgfqpoint{2.417656in}{0.765847in}}%
\pgfpathcurveto{\pgfqpoint{2.423480in}{0.760023in}}{\pgfqpoint{2.431380in}{0.756750in}}{\pgfqpoint{2.439617in}{0.756750in}}%
\pgfpathclose%
\pgfusepath{stroke,fill}%
\end{pgfscope}%
\begin{pgfscope}%
\pgfpathrectangle{\pgfqpoint{0.100000in}{0.212622in}}{\pgfqpoint{3.696000in}{3.696000in}}%
\pgfusepath{clip}%
\pgfsetbuttcap%
\pgfsetroundjoin%
\definecolor{currentfill}{rgb}{0.121569,0.466667,0.705882}%
\pgfsetfillcolor{currentfill}%
\pgfsetfillopacity{0.999983}%
\pgfsetlinewidth{1.003750pt}%
\definecolor{currentstroke}{rgb}{0.121569,0.466667,0.705882}%
\pgfsetstrokecolor{currentstroke}%
\pgfsetstrokeopacity{0.999983}%
\pgfsetdash{}{0pt}%
\pgfpathmoveto{\pgfqpoint{2.435859in}{0.757764in}}%
\pgfpathcurveto{\pgfqpoint{2.444095in}{0.757764in}}{\pgfqpoint{2.451995in}{0.761036in}}{\pgfqpoint{2.457819in}{0.766860in}}%
\pgfpathcurveto{\pgfqpoint{2.463643in}{0.772684in}}{\pgfqpoint{2.466916in}{0.780584in}}{\pgfqpoint{2.466916in}{0.788820in}}%
\pgfpathcurveto{\pgfqpoint{2.466916in}{0.797056in}}{\pgfqpoint{2.463643in}{0.804956in}}{\pgfqpoint{2.457819in}{0.810780in}}%
\pgfpathcurveto{\pgfqpoint{2.451995in}{0.816604in}}{\pgfqpoint{2.444095in}{0.819877in}}{\pgfqpoint{2.435859in}{0.819877in}}%
\pgfpathcurveto{\pgfqpoint{2.427623in}{0.819877in}}{\pgfqpoint{2.419723in}{0.816604in}}{\pgfqpoint{2.413899in}{0.810780in}}%
\pgfpathcurveto{\pgfqpoint{2.408075in}{0.804956in}}{\pgfqpoint{2.404803in}{0.797056in}}{\pgfqpoint{2.404803in}{0.788820in}}%
\pgfpathcurveto{\pgfqpoint{2.404803in}{0.780584in}}{\pgfqpoint{2.408075in}{0.772684in}}{\pgfqpoint{2.413899in}{0.766860in}}%
\pgfpathcurveto{\pgfqpoint{2.419723in}{0.761036in}}{\pgfqpoint{2.427623in}{0.757764in}}{\pgfqpoint{2.435859in}{0.757764in}}%
\pgfpathclose%
\pgfusepath{stroke,fill}%
\end{pgfscope}%
\begin{pgfscope}%
\pgfpathrectangle{\pgfqpoint{0.100000in}{0.212622in}}{\pgfqpoint{3.696000in}{3.696000in}}%
\pgfusepath{clip}%
\pgfsetbuttcap%
\pgfsetroundjoin%
\definecolor{currentfill}{rgb}{0.121569,0.466667,0.705882}%
\pgfsetfillcolor{currentfill}%
\pgfsetfillopacity{0.999989}%
\pgfsetlinewidth{1.003750pt}%
\definecolor{currentstroke}{rgb}{0.121569,0.466667,0.705882}%
\pgfsetstrokecolor{currentstroke}%
\pgfsetstrokeopacity{0.999989}%
\pgfsetdash{}{0pt}%
\pgfpathmoveto{\pgfqpoint{2.438625in}{0.756982in}}%
\pgfpathcurveto{\pgfqpoint{2.446861in}{0.756982in}}{\pgfqpoint{2.454761in}{0.760254in}}{\pgfqpoint{2.460585in}{0.766078in}}%
\pgfpathcurveto{\pgfqpoint{2.466409in}{0.771902in}}{\pgfqpoint{2.469681in}{0.779802in}}{\pgfqpoint{2.469681in}{0.788038in}}%
\pgfpathcurveto{\pgfqpoint{2.469681in}{0.796275in}}{\pgfqpoint{2.466409in}{0.804175in}}{\pgfqpoint{2.460585in}{0.809999in}}%
\pgfpathcurveto{\pgfqpoint{2.454761in}{0.815823in}}{\pgfqpoint{2.446861in}{0.819095in}}{\pgfqpoint{2.438625in}{0.819095in}}%
\pgfpathcurveto{\pgfqpoint{2.430389in}{0.819095in}}{\pgfqpoint{2.422488in}{0.815823in}}{\pgfqpoint{2.416665in}{0.809999in}}%
\pgfpathcurveto{\pgfqpoint{2.410841in}{0.804175in}}{\pgfqpoint{2.407568in}{0.796275in}}{\pgfqpoint{2.407568in}{0.788038in}}%
\pgfpathcurveto{\pgfqpoint{2.407568in}{0.779802in}}{\pgfqpoint{2.410841in}{0.771902in}}{\pgfqpoint{2.416665in}{0.766078in}}%
\pgfpathcurveto{\pgfqpoint{2.422488in}{0.760254in}}{\pgfqpoint{2.430389in}{0.756982in}}{\pgfqpoint{2.438625in}{0.756982in}}%
\pgfpathclose%
\pgfusepath{stroke,fill}%
\end{pgfscope}%
\begin{pgfscope}%
\pgfpathrectangle{\pgfqpoint{0.100000in}{0.212622in}}{\pgfqpoint{3.696000in}{3.696000in}}%
\pgfusepath{clip}%
\pgfsetbuttcap%
\pgfsetroundjoin%
\definecolor{currentfill}{rgb}{0.121569,0.466667,0.705882}%
\pgfsetfillcolor{currentfill}%
\pgfsetlinewidth{1.003750pt}%
\definecolor{currentstroke}{rgb}{0.121569,0.466667,0.705882}%
\pgfsetstrokecolor{currentstroke}%
\pgfsetdash{}{0pt}%
\pgfpathmoveto{\pgfqpoint{2.437375in}{0.757319in}}%
\pgfpathcurveto{\pgfqpoint{2.445612in}{0.757319in}}{\pgfqpoint{2.453512in}{0.760591in}}{\pgfqpoint{2.459336in}{0.766415in}}%
\pgfpathcurveto{\pgfqpoint{2.465160in}{0.772239in}}{\pgfqpoint{2.468432in}{0.780139in}}{\pgfqpoint{2.468432in}{0.788375in}}%
\pgfpathcurveto{\pgfqpoint{2.468432in}{0.796611in}}{\pgfqpoint{2.465160in}{0.804511in}}{\pgfqpoint{2.459336in}{0.810335in}}%
\pgfpathcurveto{\pgfqpoint{2.453512in}{0.816159in}}{\pgfqpoint{2.445612in}{0.819432in}}{\pgfqpoint{2.437375in}{0.819432in}}%
\pgfpathcurveto{\pgfqpoint{2.429139in}{0.819432in}}{\pgfqpoint{2.421239in}{0.816159in}}{\pgfqpoint{2.415415in}{0.810335in}}%
\pgfpathcurveto{\pgfqpoint{2.409591in}{0.804511in}}{\pgfqpoint{2.406319in}{0.796611in}}{\pgfqpoint{2.406319in}{0.788375in}}%
\pgfpathcurveto{\pgfqpoint{2.406319in}{0.780139in}}{\pgfqpoint{2.409591in}{0.772239in}}{\pgfqpoint{2.415415in}{0.766415in}}%
\pgfpathcurveto{\pgfqpoint{2.421239in}{0.760591in}}{\pgfqpoint{2.429139in}{0.757319in}}{\pgfqpoint{2.437375in}{0.757319in}}%
\pgfpathclose%
\pgfusepath{stroke,fill}%
\end{pgfscope}%
\begin{pgfscope}%
\pgfsetbuttcap%
\pgfsetmiterjoin%
\definecolor{currentfill}{rgb}{1.000000,1.000000,1.000000}%
\pgfsetfillcolor{currentfill}%
\pgfsetfillopacity{0.800000}%
\pgfsetlinewidth{1.003750pt}%
\definecolor{currentstroke}{rgb}{0.800000,0.800000,0.800000}%
\pgfsetstrokecolor{currentstroke}%
\pgfsetstrokeopacity{0.800000}%
\pgfsetdash{}{0pt}%
\pgfpathmoveto{\pgfqpoint{2.104889in}{3.216678in}}%
\pgfpathlineto{\pgfqpoint{3.698778in}{3.216678in}}%
\pgfpathquadraticcurveto{\pgfqpoint{3.726556in}{3.216678in}}{\pgfqpoint{3.726556in}{3.244456in}}%
\pgfpathlineto{\pgfqpoint{3.726556in}{3.811400in}}%
\pgfpathquadraticcurveto{\pgfqpoint{3.726556in}{3.839178in}}{\pgfqpoint{3.698778in}{3.839178in}}%
\pgfpathlineto{\pgfqpoint{2.104889in}{3.839178in}}%
\pgfpathquadraticcurveto{\pgfqpoint{2.077111in}{3.839178in}}{\pgfqpoint{2.077111in}{3.811400in}}%
\pgfpathlineto{\pgfqpoint{2.077111in}{3.244456in}}%
\pgfpathquadraticcurveto{\pgfqpoint{2.077111in}{3.216678in}}{\pgfqpoint{2.104889in}{3.216678in}}%
\pgfpathclose%
\pgfusepath{stroke,fill}%
\end{pgfscope}%
\begin{pgfscope}%
\pgfsetrectcap%
\pgfsetroundjoin%
\pgfsetlinewidth{1.505625pt}%
\definecolor{currentstroke}{rgb}{0.121569,0.466667,0.705882}%
\pgfsetstrokecolor{currentstroke}%
\pgfsetdash{}{0pt}%
\pgfpathmoveto{\pgfqpoint{2.132667in}{3.735011in}}%
\pgfpathlineto{\pgfqpoint{2.410444in}{3.735011in}}%
\pgfusepath{stroke}%
\end{pgfscope}%
\begin{pgfscope}%
\definecolor{textcolor}{rgb}{0.000000,0.000000,0.000000}%
\pgfsetstrokecolor{textcolor}%
\pgfsetfillcolor{textcolor}%
\pgftext[x=2.521555in,y=3.686400in,left,base]{\color{textcolor}\rmfamily\fontsize{10.000000}{12.000000}\selectfont Ground truth}%
\end{pgfscope}%
\begin{pgfscope}%
\pgfsetbuttcap%
\pgfsetroundjoin%
\definecolor{currentfill}{rgb}{0.121569,0.466667,0.705882}%
\pgfsetfillcolor{currentfill}%
\pgfsetlinewidth{1.003750pt}%
\definecolor{currentstroke}{rgb}{0.121569,0.466667,0.705882}%
\pgfsetstrokecolor{currentstroke}%
\pgfsetdash{}{0pt}%
\pgfsys@defobject{currentmarker}{\pgfqpoint{-0.031056in}{-0.031056in}}{\pgfqpoint{0.031056in}{0.031056in}}{%
\pgfpathmoveto{\pgfqpoint{0.000000in}{-0.031056in}}%
\pgfpathcurveto{\pgfqpoint{0.008236in}{-0.031056in}}{\pgfqpoint{0.016136in}{-0.027784in}}{\pgfqpoint{0.021960in}{-0.021960in}}%
\pgfpathcurveto{\pgfqpoint{0.027784in}{-0.016136in}}{\pgfqpoint{0.031056in}{-0.008236in}}{\pgfqpoint{0.031056in}{0.000000in}}%
\pgfpathcurveto{\pgfqpoint{0.031056in}{0.008236in}}{\pgfqpoint{0.027784in}{0.016136in}}{\pgfqpoint{0.021960in}{0.021960in}}%
\pgfpathcurveto{\pgfqpoint{0.016136in}{0.027784in}}{\pgfqpoint{0.008236in}{0.031056in}}{\pgfqpoint{0.000000in}{0.031056in}}%
\pgfpathcurveto{\pgfqpoint{-0.008236in}{0.031056in}}{\pgfqpoint{-0.016136in}{0.027784in}}{\pgfqpoint{-0.021960in}{0.021960in}}%
\pgfpathcurveto{\pgfqpoint{-0.027784in}{0.016136in}}{\pgfqpoint{-0.031056in}{0.008236in}}{\pgfqpoint{-0.031056in}{0.000000in}}%
\pgfpathcurveto{\pgfqpoint{-0.031056in}{-0.008236in}}{\pgfqpoint{-0.027784in}{-0.016136in}}{\pgfqpoint{-0.021960in}{-0.021960in}}%
\pgfpathcurveto{\pgfqpoint{-0.016136in}{-0.027784in}}{\pgfqpoint{-0.008236in}{-0.031056in}}{\pgfqpoint{0.000000in}{-0.031056in}}%
\pgfpathclose%
\pgfusepath{stroke,fill}%
}%
\begin{pgfscope}%
\pgfsys@transformshift{2.271555in}{3.529248in}%
\pgfsys@useobject{currentmarker}{}%
\end{pgfscope}%
\end{pgfscope}%
\begin{pgfscope}%
\definecolor{textcolor}{rgb}{0.000000,0.000000,0.000000}%
\pgfsetstrokecolor{textcolor}%
\pgfsetfillcolor{textcolor}%
\pgftext[x=2.521555in,y=3.492789in,left,base]{\color{textcolor}\rmfamily\fontsize{10.000000}{12.000000}\selectfont Estimated position}%
\end{pgfscope}%
\begin{pgfscope}%
\pgfsetbuttcap%
\pgfsetroundjoin%
\definecolor{currentfill}{rgb}{1.000000,0.498039,0.054902}%
\pgfsetfillcolor{currentfill}%
\pgfsetlinewidth{1.003750pt}%
\definecolor{currentstroke}{rgb}{1.000000,0.498039,0.054902}%
\pgfsetstrokecolor{currentstroke}%
\pgfsetdash{}{0pt}%
\pgfsys@defobject{currentmarker}{\pgfqpoint{-0.031056in}{-0.031056in}}{\pgfqpoint{0.031056in}{0.031056in}}{%
\pgfpathmoveto{\pgfqpoint{0.000000in}{-0.031056in}}%
\pgfpathcurveto{\pgfqpoint{0.008236in}{-0.031056in}}{\pgfqpoint{0.016136in}{-0.027784in}}{\pgfqpoint{0.021960in}{-0.021960in}}%
\pgfpathcurveto{\pgfqpoint{0.027784in}{-0.016136in}}{\pgfqpoint{0.031056in}{-0.008236in}}{\pgfqpoint{0.031056in}{0.000000in}}%
\pgfpathcurveto{\pgfqpoint{0.031056in}{0.008236in}}{\pgfqpoint{0.027784in}{0.016136in}}{\pgfqpoint{0.021960in}{0.021960in}}%
\pgfpathcurveto{\pgfqpoint{0.016136in}{0.027784in}}{\pgfqpoint{0.008236in}{0.031056in}}{\pgfqpoint{0.000000in}{0.031056in}}%
\pgfpathcurveto{\pgfqpoint{-0.008236in}{0.031056in}}{\pgfqpoint{-0.016136in}{0.027784in}}{\pgfqpoint{-0.021960in}{0.021960in}}%
\pgfpathcurveto{\pgfqpoint{-0.027784in}{0.016136in}}{\pgfqpoint{-0.031056in}{0.008236in}}{\pgfqpoint{-0.031056in}{0.000000in}}%
\pgfpathcurveto{\pgfqpoint{-0.031056in}{-0.008236in}}{\pgfqpoint{-0.027784in}{-0.016136in}}{\pgfqpoint{-0.021960in}{-0.021960in}}%
\pgfpathcurveto{\pgfqpoint{-0.016136in}{-0.027784in}}{\pgfqpoint{-0.008236in}{-0.031056in}}{\pgfqpoint{0.000000in}{-0.031056in}}%
\pgfpathclose%
\pgfusepath{stroke,fill}%
}%
\begin{pgfscope}%
\pgfsys@transformshift{2.271555in}{3.335637in}%
\pgfsys@useobject{currentmarker}{}%
\end{pgfscope}%
\end{pgfscope}%
\begin{pgfscope}%
\definecolor{textcolor}{rgb}{0.000000,0.000000,0.000000}%
\pgfsetstrokecolor{textcolor}%
\pgfsetfillcolor{textcolor}%
\pgftext[x=2.521555in,y=3.299178in,left,base]{\color{textcolor}\rmfamily\fontsize{10.000000}{12.000000}\selectfont Estimated turn}%
\end{pgfscope}%
\end{pgfpicture}%
\makeatother%
\endgroup%
}
% %         \caption{MPU-9250 Breakout}
% %         \label{fig:triangle28_3D}
% %     \end{subfigure}
% %     \caption{Position estimation by the best performing algorithms in the 4-meter line experiment.}
% %     \label{fig:triangle28}
% % \end{figure}

% \subsection{Square}

% The line shape consisted of moving the inertial system in a straight line for a determined distance. 3 line distances were tested: 4, 16, and 28 meter. The results are shown below:

% \subsubsection{4 meter}

% For the 16-meter line experiment, the Mahony algorithm which had the lowest displacement error with an average of 0.48 meters (16\% of error margin), and ROLEQ with an average of 0.24 meters of turn error (7\% of error margin).


% \begin{figure}[!h]
%     \centering
%     \begin{table}[H]
    \begin{center}
        \resizebox{1\linewidth}{!}{

            \begin{tabular}[t]{lcccc}
                \hline
                Algorithm        & Displacement Error[$m$] & Displacement Error[\%] & Turn Error[$m$] & Turn Error[\%] \\
                \hline
                AngularRate      & 5.97                    & 37.29                  & 7.29            & 45.58          \\
                \acrshort{aqua}  & 4.12                    & 25.75                  & 4.20            & 26.23          \\
                Complementary    & 1.20                    & 7.50                   & 1.46            & 9.13           \\
                Davenport        & 0.69                    & 4.31                   & 0.43            & 2.70           \\
                \acrshort{ekf}   & 1.09                    & 6.79                   & 1.53            & 9.56           \\
                \acrshort{famc}  & 6.01                    & 37.57                  & 6.98            & 43.61          \\
                \acrshort{flae}  & 0.69                    & 4.32                   & 0.38            & 2.37           \\
                Fourati          & 9.08                    & 56.78                  & 9.74            & 60.87          \\
                Madgwick         & 1.02                    & 6.40                   & 1.18            & 7.36           \\
                Mahony           & 0.53                    & 3.34                   & 0.37            & 2.29           \\
                \acrshort{oleq}  & 0.60                    & 3.74                   & 0.52            & 3.23           \\
                \acrshort{quest} & 5.07                    & 31.71                  & 5.57            & 34.79          \\
                \acrshort{roleq} & 0.78                    & 4.90                   & 0.88            & 5.48           \\
                \acrshort{saam}  & 0.59                    & 3.69                   & 0.37            & 2.33           \\
                Tilt             & 0.59                    & 3.69                   & 0.37            & 2.33           \\

                \hline
                Average          & 2.54                    & 15.85                  & 2.75            & 17.19
            \end{tabular}
        }
        \caption{4 meter square position estimation error (displacement and turn) of the sensor fusion algorithms. }
        \label{tab:4_square}
    \end{center}
\end{table}
% \end{figure}

% \begin{figure}[!h]
%     \centering
%     \begin{subfigure}{0.49\textwidth}
%         \centering
%         \resizebox{1\linewidth}{!}{%% Creator: Matplotlib, PGF backend
%%
%% To include the figure in your LaTeX document, write
%%   \input{<filename>.pgf}
%%
%% Make sure the required packages are loaded in your preamble
%%   \usepackage{pgf}
%%
%% and, on pdftex
%%   \usepackage[utf8]{inputenc}\DeclareUnicodeCharacter{2212}{-}
%%
%% or, on luatex and xetex
%%   \usepackage{unicode-math}
%%
%% Figures using additional raster images can only be included by \input if
%% they are in the same directory as the main LaTeX file. For loading figures
%% from other directories you can use the `import` package
%%   \usepackage{import}
%%
%% and then include the figures with
%%   \import{<path to file>}{<filename>.pgf}
%%
%% Matplotlib used the following preamble
%%   \usepackage{fontspec}
%%
\begingroup%
\makeatletter%
\begin{pgfpicture}%
\pgfpathrectangle{\pgfpointorigin}{\pgfqpoint{4.342355in}{4.207622in}}%
\pgfusepath{use as bounding box, clip}%
\begin{pgfscope}%
\pgfsetbuttcap%
\pgfsetmiterjoin%
\definecolor{currentfill}{rgb}{1.000000,1.000000,1.000000}%
\pgfsetfillcolor{currentfill}%
\pgfsetlinewidth{0.000000pt}%
\definecolor{currentstroke}{rgb}{1.000000,1.000000,1.000000}%
\pgfsetstrokecolor{currentstroke}%
\pgfsetdash{}{0pt}%
\pgfpathmoveto{\pgfqpoint{0.000000in}{0.000000in}}%
\pgfpathlineto{\pgfqpoint{4.342355in}{0.000000in}}%
\pgfpathlineto{\pgfqpoint{4.342355in}{4.207622in}}%
\pgfpathlineto{\pgfqpoint{0.000000in}{4.207622in}}%
\pgfpathclose%
\pgfusepath{fill}%
\end{pgfscope}%
\begin{pgfscope}%
\pgfsetbuttcap%
\pgfsetmiterjoin%
\definecolor{currentfill}{rgb}{1.000000,1.000000,1.000000}%
\pgfsetfillcolor{currentfill}%
\pgfsetlinewidth{0.000000pt}%
\definecolor{currentstroke}{rgb}{0.000000,0.000000,0.000000}%
\pgfsetstrokecolor{currentstroke}%
\pgfsetstrokeopacity{0.000000}%
\pgfsetdash{}{0pt}%
\pgfpathmoveto{\pgfqpoint{0.100000in}{0.212622in}}%
\pgfpathlineto{\pgfqpoint{3.796000in}{0.212622in}}%
\pgfpathlineto{\pgfqpoint{3.796000in}{3.908622in}}%
\pgfpathlineto{\pgfqpoint{0.100000in}{3.908622in}}%
\pgfpathclose%
\pgfusepath{fill}%
\end{pgfscope}%
\begin{pgfscope}%
\pgfsetbuttcap%
\pgfsetmiterjoin%
\definecolor{currentfill}{rgb}{0.950000,0.950000,0.950000}%
\pgfsetfillcolor{currentfill}%
\pgfsetfillopacity{0.500000}%
\pgfsetlinewidth{1.003750pt}%
\definecolor{currentstroke}{rgb}{0.950000,0.950000,0.950000}%
\pgfsetstrokecolor{currentstroke}%
\pgfsetstrokeopacity{0.500000}%
\pgfsetdash{}{0pt}%
\pgfpathmoveto{\pgfqpoint{0.379073in}{1.123938in}}%
\pgfpathlineto{\pgfqpoint{1.599613in}{2.147018in}}%
\pgfpathlineto{\pgfqpoint{1.582647in}{3.622484in}}%
\pgfpathlineto{\pgfqpoint{0.303698in}{2.689165in}}%
\pgfusepath{stroke,fill}%
\end{pgfscope}%
\begin{pgfscope}%
\pgfsetbuttcap%
\pgfsetmiterjoin%
\definecolor{currentfill}{rgb}{0.900000,0.900000,0.900000}%
\pgfsetfillcolor{currentfill}%
\pgfsetfillopacity{0.500000}%
\pgfsetlinewidth{1.003750pt}%
\definecolor{currentstroke}{rgb}{0.900000,0.900000,0.900000}%
\pgfsetstrokecolor{currentstroke}%
\pgfsetstrokeopacity{0.500000}%
\pgfsetdash{}{0pt}%
\pgfpathmoveto{\pgfqpoint{1.599613in}{2.147018in}}%
\pgfpathlineto{\pgfqpoint{3.558144in}{1.577751in}}%
\pgfpathlineto{\pgfqpoint{3.628038in}{3.104037in}}%
\pgfpathlineto{\pgfqpoint{1.582647in}{3.622484in}}%
\pgfusepath{stroke,fill}%
\end{pgfscope}%
\begin{pgfscope}%
\pgfsetbuttcap%
\pgfsetmiterjoin%
\definecolor{currentfill}{rgb}{0.925000,0.925000,0.925000}%
\pgfsetfillcolor{currentfill}%
\pgfsetfillopacity{0.500000}%
\pgfsetlinewidth{1.003750pt}%
\definecolor{currentstroke}{rgb}{0.925000,0.925000,0.925000}%
\pgfsetstrokecolor{currentstroke}%
\pgfsetstrokeopacity{0.500000}%
\pgfsetdash{}{0pt}%
\pgfpathmoveto{\pgfqpoint{0.379073in}{1.123938in}}%
\pgfpathlineto{\pgfqpoint{2.455212in}{0.445871in}}%
\pgfpathlineto{\pgfqpoint{3.558144in}{1.577751in}}%
\pgfpathlineto{\pgfqpoint{1.599613in}{2.147018in}}%
\pgfusepath{stroke,fill}%
\end{pgfscope}%
\begin{pgfscope}%
\pgfsetrectcap%
\pgfsetroundjoin%
\pgfsetlinewidth{0.803000pt}%
\definecolor{currentstroke}{rgb}{0.000000,0.000000,0.000000}%
\pgfsetstrokecolor{currentstroke}%
\pgfsetdash{}{0pt}%
\pgfpathmoveto{\pgfqpoint{0.379073in}{1.123938in}}%
\pgfpathlineto{\pgfqpoint{2.455212in}{0.445871in}}%
\pgfusepath{stroke}%
\end{pgfscope}%
\begin{pgfscope}%
\definecolor{textcolor}{rgb}{0.000000,0.000000,0.000000}%
\pgfsetstrokecolor{textcolor}%
\pgfsetfillcolor{textcolor}%
\pgftext[x=0.730374in, y=0.408886in, left, base,rotate=341.912962]{\color{textcolor}\rmfamily\fontsize{10.000000}{12.000000}\selectfont Position X [\(\displaystyle m\)]}%
\end{pgfscope}%
\begin{pgfscope}%
\pgfsetbuttcap%
\pgfsetroundjoin%
\pgfsetlinewidth{0.803000pt}%
\definecolor{currentstroke}{rgb}{0.690196,0.690196,0.690196}%
\pgfsetstrokecolor{currentstroke}%
\pgfsetdash{}{0pt}%
\pgfpathmoveto{\pgfqpoint{0.710603in}{1.015660in}}%
\pgfpathlineto{\pgfqpoint{1.913525in}{2.055777in}}%
\pgfpathlineto{\pgfqpoint{1.909899in}{3.539535in}}%
\pgfusepath{stroke}%
\end{pgfscope}%
\begin{pgfscope}%
\pgfsetbuttcap%
\pgfsetroundjoin%
\pgfsetlinewidth{0.803000pt}%
\definecolor{currentstroke}{rgb}{0.690196,0.690196,0.690196}%
\pgfsetstrokecolor{currentstroke}%
\pgfsetdash{}{0pt}%
\pgfpathmoveto{\pgfqpoint{1.057043in}{0.902513in}}%
\pgfpathlineto{\pgfqpoint{2.241081in}{1.960569in}}%
\pgfpathlineto{\pgfqpoint{2.251612in}{3.452920in}}%
\pgfusepath{stroke}%
\end{pgfscope}%
\begin{pgfscope}%
\pgfsetbuttcap%
\pgfsetroundjoin%
\pgfsetlinewidth{0.803000pt}%
\definecolor{currentstroke}{rgb}{0.690196,0.690196,0.690196}%
\pgfsetstrokecolor{currentstroke}%
\pgfsetdash{}{0pt}%
\pgfpathmoveto{\pgfqpoint{1.409746in}{0.787320in}}%
\pgfpathlineto{\pgfqpoint{2.574063in}{1.863784in}}%
\pgfpathlineto{\pgfqpoint{2.599233in}{3.364809in}}%
\pgfusepath{stroke}%
\end{pgfscope}%
\begin{pgfscope}%
\pgfsetbuttcap%
\pgfsetroundjoin%
\pgfsetlinewidth{0.803000pt}%
\definecolor{currentstroke}{rgb}{0.690196,0.690196,0.690196}%
\pgfsetstrokecolor{currentstroke}%
\pgfsetdash{}{0pt}%
\pgfpathmoveto{\pgfqpoint{1.768885in}{0.670025in}}%
\pgfpathlineto{\pgfqpoint{2.912608in}{1.765383in}}%
\pgfpathlineto{\pgfqpoint{2.952917in}{3.275160in}}%
\pgfusepath{stroke}%
\end{pgfscope}%
\begin{pgfscope}%
\pgfsetbuttcap%
\pgfsetroundjoin%
\pgfsetlinewidth{0.803000pt}%
\definecolor{currentstroke}{rgb}{0.690196,0.690196,0.690196}%
\pgfsetstrokecolor{currentstroke}%
\pgfsetdash{}{0pt}%
\pgfpathmoveto{\pgfqpoint{2.134636in}{0.550571in}}%
\pgfpathlineto{\pgfqpoint{3.256855in}{1.665324in}}%
\pgfpathlineto{\pgfqpoint{3.312824in}{3.183934in}}%
\pgfusepath{stroke}%
\end{pgfscope}%
\begin{pgfscope}%
\pgfsetrectcap%
\pgfsetroundjoin%
\pgfsetlinewidth{0.803000pt}%
\definecolor{currentstroke}{rgb}{0.000000,0.000000,0.000000}%
\pgfsetstrokecolor{currentstroke}%
\pgfsetdash{}{0pt}%
\pgfpathmoveto{\pgfqpoint{0.721083in}{1.024721in}}%
\pgfpathlineto{\pgfqpoint{0.689599in}{0.997499in}}%
\pgfusepath{stroke}%
\end{pgfscope}%
\begin{pgfscope}%
\definecolor{textcolor}{rgb}{0.000000,0.000000,0.000000}%
\pgfsetstrokecolor{textcolor}%
\pgfsetfillcolor{textcolor}%
\pgftext[x=0.606240in,y=0.796033in,,top]{\color{textcolor}\rmfamily\fontsize{10.000000}{12.000000}\selectfont \(\displaystyle {0}\)}%
\end{pgfscope}%
\begin{pgfscope}%
\pgfsetrectcap%
\pgfsetroundjoin%
\pgfsetlinewidth{0.803000pt}%
\definecolor{currentstroke}{rgb}{0.000000,0.000000,0.000000}%
\pgfsetstrokecolor{currentstroke}%
\pgfsetdash{}{0pt}%
\pgfpathmoveto{\pgfqpoint{1.067365in}{0.911737in}}%
\pgfpathlineto{\pgfqpoint{1.036353in}{0.884024in}}%
\pgfusepath{stroke}%
\end{pgfscope}%
\begin{pgfscope}%
\definecolor{textcolor}{rgb}{0.000000,0.000000,0.000000}%
\pgfsetstrokecolor{textcolor}%
\pgfsetfillcolor{textcolor}%
\pgftext[x=0.953048in,y=0.680484in,,top]{\color{textcolor}\rmfamily\fontsize{10.000000}{12.000000}\selectfont \(\displaystyle {1}\)}%
\end{pgfscope}%
\begin{pgfscope}%
\pgfsetrectcap%
\pgfsetroundjoin%
\pgfsetlinewidth{0.803000pt}%
\definecolor{currentstroke}{rgb}{0.000000,0.000000,0.000000}%
\pgfsetstrokecolor{currentstroke}%
\pgfsetdash{}{0pt}%
\pgfpathmoveto{\pgfqpoint{1.419904in}{0.796712in}}%
\pgfpathlineto{\pgfqpoint{1.389385in}{0.768495in}}%
\pgfusepath{stroke}%
\end{pgfscope}%
\begin{pgfscope}%
\definecolor{textcolor}{rgb}{0.000000,0.000000,0.000000}%
\pgfsetstrokecolor{textcolor}%
\pgfsetfillcolor{textcolor}%
\pgftext[x=1.306150in,y=0.562838in,,top]{\color{textcolor}\rmfamily\fontsize{10.000000}{12.000000}\selectfont \(\displaystyle {2}\)}%
\end{pgfscope}%
\begin{pgfscope}%
\pgfsetrectcap%
\pgfsetroundjoin%
\pgfsetlinewidth{0.803000pt}%
\definecolor{currentstroke}{rgb}{0.000000,0.000000,0.000000}%
\pgfsetstrokecolor{currentstroke}%
\pgfsetdash{}{0pt}%
\pgfpathmoveto{\pgfqpoint{1.778871in}{0.679589in}}%
\pgfpathlineto{\pgfqpoint{1.748868in}{0.650855in}}%
\pgfusepath{stroke}%
\end{pgfscope}%
\begin{pgfscope}%
\definecolor{textcolor}{rgb}{0.000000,0.000000,0.000000}%
\pgfsetstrokecolor{textcolor}%
\pgfsetfillcolor{textcolor}%
\pgftext[x=1.665718in,y=0.443037in,,top]{\color{textcolor}\rmfamily\fontsize{10.000000}{12.000000}\selectfont \(\displaystyle {3}\)}%
\end{pgfscope}%
\begin{pgfscope}%
\pgfsetrectcap%
\pgfsetroundjoin%
\pgfsetlinewidth{0.803000pt}%
\definecolor{currentstroke}{rgb}{0.000000,0.000000,0.000000}%
\pgfsetstrokecolor{currentstroke}%
\pgfsetdash{}{0pt}%
\pgfpathmoveto{\pgfqpoint{2.144442in}{0.560312in}}%
\pgfpathlineto{\pgfqpoint{2.114980in}{0.531046in}}%
\pgfusepath{stroke}%
\end{pgfscope}%
\begin{pgfscope}%
\definecolor{textcolor}{rgb}{0.000000,0.000000,0.000000}%
\pgfsetstrokecolor{textcolor}%
\pgfsetfillcolor{textcolor}%
\pgftext[x=2.031934in,y=0.321022in,,top]{\color{textcolor}\rmfamily\fontsize{10.000000}{12.000000}\selectfont \(\displaystyle {4}\)}%
\end{pgfscope}%
\begin{pgfscope}%
\pgfsetrectcap%
\pgfsetroundjoin%
\pgfsetlinewidth{0.803000pt}%
\definecolor{currentstroke}{rgb}{0.000000,0.000000,0.000000}%
\pgfsetstrokecolor{currentstroke}%
\pgfsetdash{}{0pt}%
\pgfpathmoveto{\pgfqpoint{3.558144in}{1.577751in}}%
\pgfpathlineto{\pgfqpoint{2.455212in}{0.445871in}}%
\pgfusepath{stroke}%
\end{pgfscope}%
\begin{pgfscope}%
\definecolor{textcolor}{rgb}{0.000000,0.000000,0.000000}%
\pgfsetstrokecolor{textcolor}%
\pgfsetfillcolor{textcolor}%
\pgftext[x=3.120747in, y=0.305657in, left, base,rotate=45.742112]{\color{textcolor}\rmfamily\fontsize{10.000000}{12.000000}\selectfont Position Y [\(\displaystyle m\)]}%
\end{pgfscope}%
\begin{pgfscope}%
\pgfsetbuttcap%
\pgfsetroundjoin%
\pgfsetlinewidth{0.803000pt}%
\definecolor{currentstroke}{rgb}{0.690196,0.690196,0.690196}%
\pgfsetstrokecolor{currentstroke}%
\pgfsetdash{}{0pt}%
\pgfpathmoveto{\pgfqpoint{0.526775in}{2.851957in}}%
\pgfpathlineto{\pgfqpoint{0.591297in}{1.301827in}}%
\pgfpathlineto{\pgfqpoint{2.647687in}{0.643397in}}%
\pgfusepath{stroke}%
\end{pgfscope}%
\begin{pgfscope}%
\pgfsetbuttcap%
\pgfsetroundjoin%
\pgfsetlinewidth{0.803000pt}%
\definecolor{currentstroke}{rgb}{0.690196,0.690196,0.690196}%
\pgfsetstrokecolor{currentstroke}%
\pgfsetdash{}{0pt}%
\pgfpathmoveto{\pgfqpoint{0.771236in}{3.030353in}}%
\pgfpathlineto{\pgfqpoint{0.824185in}{1.497039in}}%
\pgfpathlineto{\pgfqpoint{2.858563in}{0.859808in}}%
\pgfusepath{stroke}%
\end{pgfscope}%
\begin{pgfscope}%
\pgfsetbuttcap%
\pgfsetroundjoin%
\pgfsetlinewidth{0.803000pt}%
\definecolor{currentstroke}{rgb}{0.690196,0.690196,0.690196}%
\pgfsetstrokecolor{currentstroke}%
\pgfsetdash{}{0pt}%
\pgfpathmoveto{\pgfqpoint{1.007746in}{3.202947in}}%
\pgfpathlineto{\pgfqpoint{1.049821in}{1.686172in}}%
\pgfpathlineto{\pgfqpoint{3.062534in}{1.069132in}}%
\pgfusepath{stroke}%
\end{pgfscope}%
\begin{pgfscope}%
\pgfsetbuttcap%
\pgfsetroundjoin%
\pgfsetlinewidth{0.803000pt}%
\definecolor{currentstroke}{rgb}{0.690196,0.690196,0.690196}%
\pgfsetstrokecolor{currentstroke}%
\pgfsetdash{}{0pt}%
\pgfpathmoveto{\pgfqpoint{1.236688in}{3.370019in}}%
\pgfpathlineto{\pgfqpoint{1.268540in}{1.869506in}}%
\pgfpathlineto{\pgfqpoint{3.259934in}{1.271713in}}%
\pgfusepath{stroke}%
\end{pgfscope}%
\begin{pgfscope}%
\pgfsetbuttcap%
\pgfsetroundjoin%
\pgfsetlinewidth{0.803000pt}%
\definecolor{currentstroke}{rgb}{0.690196,0.690196,0.690196}%
\pgfsetstrokecolor{currentstroke}%
\pgfsetdash{}{0pt}%
\pgfpathmoveto{\pgfqpoint{1.458420in}{3.531829in}}%
\pgfpathlineto{\pgfqpoint{1.480653in}{2.047303in}}%
\pgfpathlineto{\pgfqpoint{3.451074in}{1.467870in}}%
\pgfusepath{stroke}%
\end{pgfscope}%
\begin{pgfscope}%
\pgfsetrectcap%
\pgfsetroundjoin%
\pgfsetlinewidth{0.803000pt}%
\definecolor{currentstroke}{rgb}{0.000000,0.000000,0.000000}%
\pgfsetstrokecolor{currentstroke}%
\pgfsetdash{}{0pt}%
\pgfpathmoveto{\pgfqpoint{2.630365in}{0.648943in}}%
\pgfpathlineto{\pgfqpoint{2.682374in}{0.632291in}}%
\pgfusepath{stroke}%
\end{pgfscope}%
\begin{pgfscope}%
\definecolor{textcolor}{rgb}{0.000000,0.000000,0.000000}%
\pgfsetstrokecolor{textcolor}%
\pgfsetfillcolor{textcolor}%
\pgftext[x=2.825027in,y=0.458728in,,top]{\color{textcolor}\rmfamily\fontsize{10.000000}{12.000000}\selectfont \(\displaystyle {0}\)}%
\end{pgfscope}%
\begin{pgfscope}%
\pgfsetrectcap%
\pgfsetroundjoin%
\pgfsetlinewidth{0.803000pt}%
\definecolor{currentstroke}{rgb}{0.000000,0.000000,0.000000}%
\pgfsetstrokecolor{currentstroke}%
\pgfsetdash{}{0pt}%
\pgfpathmoveto{\pgfqpoint{2.841441in}{0.865171in}}%
\pgfpathlineto{\pgfqpoint{2.892849in}{0.849068in}}%
\pgfusepath{stroke}%
\end{pgfscope}%
\begin{pgfscope}%
\definecolor{textcolor}{rgb}{0.000000,0.000000,0.000000}%
\pgfsetstrokecolor{textcolor}%
\pgfsetfillcolor{textcolor}%
\pgftext[x=3.033073in,y=0.678341in,,top]{\color{textcolor}\rmfamily\fontsize{10.000000}{12.000000}\selectfont \(\displaystyle {1}\)}%
\end{pgfscope}%
\begin{pgfscope}%
\pgfsetrectcap%
\pgfsetroundjoin%
\pgfsetlinewidth{0.803000pt}%
\definecolor{currentstroke}{rgb}{0.000000,0.000000,0.000000}%
\pgfsetstrokecolor{currentstroke}%
\pgfsetdash{}{0pt}%
\pgfpathmoveto{\pgfqpoint{3.045608in}{1.074321in}}%
\pgfpathlineto{\pgfqpoint{3.096427in}{1.058742in}}%
\pgfusepath{stroke}%
\end{pgfscope}%
\begin{pgfscope}%
\definecolor{textcolor}{rgb}{0.000000,0.000000,0.000000}%
\pgfsetstrokecolor{textcolor}%
\pgfsetfillcolor{textcolor}%
\pgftext[x=3.234302in,y=0.890758in,,top]{\color{textcolor}\rmfamily\fontsize{10.000000}{12.000000}\selectfont \(\displaystyle {2}\)}%
\end{pgfscope}%
\begin{pgfscope}%
\pgfsetrectcap%
\pgfsetroundjoin%
\pgfsetlinewidth{0.803000pt}%
\definecolor{currentstroke}{rgb}{0.000000,0.000000,0.000000}%
\pgfsetstrokecolor{currentstroke}%
\pgfsetdash{}{0pt}%
\pgfpathmoveto{\pgfqpoint{3.243200in}{1.276736in}}%
\pgfpathlineto{\pgfqpoint{3.293441in}{1.261654in}}%
\pgfusepath{stroke}%
\end{pgfscope}%
\begin{pgfscope}%
\definecolor{textcolor}{rgb}{0.000000,0.000000,0.000000}%
\pgfsetstrokecolor{textcolor}%
\pgfsetfillcolor{textcolor}%
\pgftext[x=3.429045in,y=1.096328in,,top]{\color{textcolor}\rmfamily\fontsize{10.000000}{12.000000}\selectfont \(\displaystyle {3}\)}%
\end{pgfscope}%
\begin{pgfscope}%
\pgfsetrectcap%
\pgfsetroundjoin%
\pgfsetlinewidth{0.803000pt}%
\definecolor{currentstroke}{rgb}{0.000000,0.000000,0.000000}%
\pgfsetstrokecolor{currentstroke}%
\pgfsetdash{}{0pt}%
\pgfpathmoveto{\pgfqpoint{3.434530in}{1.472736in}}%
\pgfpathlineto{\pgfqpoint{3.484203in}{1.458129in}}%
\pgfusepath{stroke}%
\end{pgfscope}%
\begin{pgfscope}%
\definecolor{textcolor}{rgb}{0.000000,0.000000,0.000000}%
\pgfsetstrokecolor{textcolor}%
\pgfsetfillcolor{textcolor}%
\pgftext[x=3.617609in,y=1.295376in,,top]{\color{textcolor}\rmfamily\fontsize{10.000000}{12.000000}\selectfont \(\displaystyle {4}\)}%
\end{pgfscope}%
\begin{pgfscope}%
\pgfsetrectcap%
\pgfsetroundjoin%
\pgfsetlinewidth{0.803000pt}%
\definecolor{currentstroke}{rgb}{0.000000,0.000000,0.000000}%
\pgfsetstrokecolor{currentstroke}%
\pgfsetdash{}{0pt}%
\pgfpathmoveto{\pgfqpoint{3.558144in}{1.577751in}}%
\pgfpathlineto{\pgfqpoint{3.628038in}{3.104037in}}%
\pgfusepath{stroke}%
\end{pgfscope}%
\begin{pgfscope}%
\definecolor{textcolor}{rgb}{0.000000,0.000000,0.000000}%
\pgfsetstrokecolor{textcolor}%
\pgfsetfillcolor{textcolor}%
\pgftext[x=4.167903in, y=1.963517in, left, base,rotate=87.378092]{\color{textcolor}\rmfamily\fontsize{10.000000}{12.000000}\selectfont Position Z [\(\displaystyle m\)]}%
\end{pgfscope}%
\begin{pgfscope}%
\pgfsetbuttcap%
\pgfsetroundjoin%
\pgfsetlinewidth{0.803000pt}%
\definecolor{currentstroke}{rgb}{0.690196,0.690196,0.690196}%
\pgfsetstrokecolor{currentstroke}%
\pgfsetdash{}{0pt}%
\pgfpathmoveto{\pgfqpoint{3.562413in}{1.670968in}}%
\pgfpathlineto{\pgfqpoint{1.598575in}{2.237310in}}%
\pgfpathlineto{\pgfqpoint{0.374477in}{1.219382in}}%
\pgfusepath{stroke}%
\end{pgfscope}%
\begin{pgfscope}%
\pgfsetbuttcap%
\pgfsetroundjoin%
\pgfsetlinewidth{0.803000pt}%
\definecolor{currentstroke}{rgb}{0.690196,0.690196,0.690196}%
\pgfsetstrokecolor{currentstroke}%
\pgfsetdash{}{0pt}%
\pgfpathmoveto{\pgfqpoint{3.573719in}{1.917855in}}%
\pgfpathlineto{\pgfqpoint{1.595826in}{2.476338in}}%
\pgfpathlineto{\pgfqpoint{0.362299in}{1.472263in}}%
\pgfusepath{stroke}%
\end{pgfscope}%
\begin{pgfscope}%
\pgfsetbuttcap%
\pgfsetroundjoin%
\pgfsetlinewidth{0.803000pt}%
\definecolor{currentstroke}{rgb}{0.690196,0.690196,0.690196}%
\pgfsetstrokecolor{currentstroke}%
\pgfsetdash{}{0pt}%
\pgfpathmoveto{\pgfqpoint{3.585189in}{2.168337in}}%
\pgfpathlineto{\pgfqpoint{1.593040in}{2.718679in}}%
\pgfpathlineto{\pgfqpoint{0.349937in}{1.728966in}}%
\pgfusepath{stroke}%
\end{pgfscope}%
\begin{pgfscope}%
\pgfsetbuttcap%
\pgfsetroundjoin%
\pgfsetlinewidth{0.803000pt}%
\definecolor{currentstroke}{rgb}{0.690196,0.690196,0.690196}%
\pgfsetstrokecolor{currentstroke}%
\pgfsetdash{}{0pt}%
\pgfpathmoveto{\pgfqpoint{3.596828in}{2.422492in}}%
\pgfpathlineto{\pgfqpoint{1.590214in}{2.964402in}}%
\pgfpathlineto{\pgfqpoint{0.337387in}{1.989580in}}%
\pgfusepath{stroke}%
\end{pgfscope}%
\begin{pgfscope}%
\pgfsetbuttcap%
\pgfsetroundjoin%
\pgfsetlinewidth{0.803000pt}%
\definecolor{currentstroke}{rgb}{0.690196,0.690196,0.690196}%
\pgfsetstrokecolor{currentstroke}%
\pgfsetdash{}{0pt}%
\pgfpathmoveto{\pgfqpoint{3.608638in}{2.680402in}}%
\pgfpathlineto{\pgfqpoint{1.587349in}{3.213579in}}%
\pgfpathlineto{\pgfqpoint{0.324644in}{2.254193in}}%
\pgfusepath{stroke}%
\end{pgfscope}%
\begin{pgfscope}%
\pgfsetbuttcap%
\pgfsetroundjoin%
\pgfsetlinewidth{0.803000pt}%
\definecolor{currentstroke}{rgb}{0.690196,0.690196,0.690196}%
\pgfsetstrokecolor{currentstroke}%
\pgfsetdash{}{0pt}%
\pgfpathmoveto{\pgfqpoint{3.620624in}{2.942151in}}%
\pgfpathlineto{\pgfqpoint{1.584443in}{3.466283in}}%
\pgfpathlineto{\pgfqpoint{0.311704in}{2.522898in}}%
\pgfusepath{stroke}%
\end{pgfscope}%
\begin{pgfscope}%
\pgfsetrectcap%
\pgfsetroundjoin%
\pgfsetlinewidth{0.803000pt}%
\definecolor{currentstroke}{rgb}{0.000000,0.000000,0.000000}%
\pgfsetstrokecolor{currentstroke}%
\pgfsetdash{}{0pt}%
\pgfpathmoveto{\pgfqpoint{3.545929in}{1.675722in}}%
\pgfpathlineto{\pgfqpoint{3.595421in}{1.661449in}}%
\pgfusepath{stroke}%
\end{pgfscope}%
\begin{pgfscope}%
\definecolor{textcolor}{rgb}{0.000000,0.000000,0.000000}%
\pgfsetstrokecolor{textcolor}%
\pgfsetfillcolor{textcolor}%
\pgftext[x=3.816545in,y=1.706967in,,top]{\color{textcolor}\rmfamily\fontsize{10.000000}{12.000000}\selectfont \(\displaystyle {0.0}\)}%
\end{pgfscope}%
\begin{pgfscope}%
\pgfsetrectcap%
\pgfsetroundjoin%
\pgfsetlinewidth{0.803000pt}%
\definecolor{currentstroke}{rgb}{0.000000,0.000000,0.000000}%
\pgfsetstrokecolor{currentstroke}%
\pgfsetdash{}{0pt}%
\pgfpathmoveto{\pgfqpoint{3.557111in}{1.922545in}}%
\pgfpathlineto{\pgfqpoint{3.606974in}{1.908465in}}%
\pgfusepath{stroke}%
\end{pgfscope}%
\begin{pgfscope}%
\definecolor{textcolor}{rgb}{0.000000,0.000000,0.000000}%
\pgfsetstrokecolor{textcolor}%
\pgfsetfillcolor{textcolor}%
\pgftext[x=3.829647in,y=1.953366in,,top]{\color{textcolor}\rmfamily\fontsize{10.000000}{12.000000}\selectfont \(\displaystyle {0.1}\)}%
\end{pgfscope}%
\begin{pgfscope}%
\pgfsetrectcap%
\pgfsetroundjoin%
\pgfsetlinewidth{0.803000pt}%
\definecolor{currentstroke}{rgb}{0.000000,0.000000,0.000000}%
\pgfsetstrokecolor{currentstroke}%
\pgfsetdash{}{0pt}%
\pgfpathmoveto{\pgfqpoint{3.568456in}{2.172960in}}%
\pgfpathlineto{\pgfqpoint{3.618696in}{2.159081in}}%
\pgfusepath{stroke}%
\end{pgfscope}%
\begin{pgfscope}%
\definecolor{textcolor}{rgb}{0.000000,0.000000,0.000000}%
\pgfsetstrokecolor{textcolor}%
\pgfsetfillcolor{textcolor}%
\pgftext[x=3.842940in,y=2.203342in,,top]{\color{textcolor}\rmfamily\fontsize{10.000000}{12.000000}\selectfont \(\displaystyle {0.2}\)}%
\end{pgfscope}%
\begin{pgfscope}%
\pgfsetrectcap%
\pgfsetroundjoin%
\pgfsetlinewidth{0.803000pt}%
\definecolor{currentstroke}{rgb}{0.000000,0.000000,0.000000}%
\pgfsetstrokecolor{currentstroke}%
\pgfsetdash{}{0pt}%
\pgfpathmoveto{\pgfqpoint{3.579967in}{2.427046in}}%
\pgfpathlineto{\pgfqpoint{3.630590in}{2.413375in}}%
\pgfusepath{stroke}%
\end{pgfscope}%
\begin{pgfscope}%
\definecolor{textcolor}{rgb}{0.000000,0.000000,0.000000}%
\pgfsetstrokecolor{textcolor}%
\pgfsetfillcolor{textcolor}%
\pgftext[x=3.856427in,y=2.456972in,,top]{\color{textcolor}\rmfamily\fontsize{10.000000}{12.000000}\selectfont \(\displaystyle {0.3}\)}%
\end{pgfscope}%
\begin{pgfscope}%
\pgfsetrectcap%
\pgfsetroundjoin%
\pgfsetlinewidth{0.803000pt}%
\definecolor{currentstroke}{rgb}{0.000000,0.000000,0.000000}%
\pgfsetstrokecolor{currentstroke}%
\pgfsetdash{}{0pt}%
\pgfpathmoveto{\pgfqpoint{3.591648in}{2.684884in}}%
\pgfpathlineto{\pgfqpoint{3.642660in}{2.671428in}}%
\pgfusepath{stroke}%
\end{pgfscope}%
\begin{pgfscope}%
\definecolor{textcolor}{rgb}{0.000000,0.000000,0.000000}%
\pgfsetstrokecolor{textcolor}%
\pgfsetfillcolor{textcolor}%
\pgftext[x=3.870112in,y=2.714338in,,top]{\color{textcolor}\rmfamily\fontsize{10.000000}{12.000000}\selectfont \(\displaystyle {0.4}\)}%
\end{pgfscope}%
\begin{pgfscope}%
\pgfsetrectcap%
\pgfsetroundjoin%
\pgfsetlinewidth{0.803000pt}%
\definecolor{currentstroke}{rgb}{0.000000,0.000000,0.000000}%
\pgfsetstrokecolor{currentstroke}%
\pgfsetdash{}{0pt}%
\pgfpathmoveto{\pgfqpoint{3.603503in}{2.946558in}}%
\pgfpathlineto{\pgfqpoint{3.654909in}{2.933326in}}%
\pgfusepath{stroke}%
\end{pgfscope}%
\begin{pgfscope}%
\definecolor{textcolor}{rgb}{0.000000,0.000000,0.000000}%
\pgfsetstrokecolor{textcolor}%
\pgfsetfillcolor{textcolor}%
\pgftext[x=3.884001in,y=2.975522in,,top]{\color{textcolor}\rmfamily\fontsize{10.000000}{12.000000}\selectfont \(\displaystyle {0.5}\)}%
\end{pgfscope}%
\begin{pgfscope}%
\pgfpathrectangle{\pgfqpoint{0.100000in}{0.212622in}}{\pgfqpoint{3.696000in}{3.696000in}}%
\pgfusepath{clip}%
\pgfsetrectcap%
\pgfsetroundjoin%
\pgfsetlinewidth{1.505625pt}%
\definecolor{currentstroke}{rgb}{0.121569,0.466667,0.705882}%
\pgfsetstrokecolor{currentstroke}%
\pgfsetdash{}{0pt}%
\pgfpathmoveto{\pgfqpoint{0.916842in}{1.291711in}}%
\pgfpathlineto{\pgfqpoint{1.795822in}{2.045764in}}%
\pgfpathlineto{\pgfqpoint{3.150997in}{1.650360in}}%
\pgfpathlineto{\pgfqpoint{2.331339in}{0.842232in}}%
\pgfpathlineto{\pgfqpoint{0.916842in}{1.291711in}}%
\pgfusepath{stroke}%
\end{pgfscope}%
\begin{pgfscope}%
\pgfpathrectangle{\pgfqpoint{0.100000in}{0.212622in}}{\pgfqpoint{3.696000in}{3.696000in}}%
\pgfusepath{clip}%
\pgfsetrectcap%
\pgfsetroundjoin%
\pgfsetlinewidth{1.505625pt}%
\definecolor{currentstroke}{rgb}{1.000000,0.000000,0.000000}%
\pgfsetstrokecolor{currentstroke}%
\pgfsetdash{}{0pt}%
\pgfpathmoveto{\pgfqpoint{0.916842in}{1.291711in}}%
\pgfpathlineto{\pgfqpoint{0.916842in}{1.291711in}}%
\pgfusepath{stroke}%
\end{pgfscope}%
\begin{pgfscope}%
\pgfpathrectangle{\pgfqpoint{0.100000in}{0.212622in}}{\pgfqpoint{3.696000in}{3.696000in}}%
\pgfusepath{clip}%
\pgfsetrectcap%
\pgfsetroundjoin%
\pgfsetlinewidth{1.505625pt}%
\definecolor{currentstroke}{rgb}{1.000000,0.000000,0.000000}%
\pgfsetstrokecolor{currentstroke}%
\pgfsetdash{}{0pt}%
\pgfpathmoveto{\pgfqpoint{0.913953in}{1.291267in}}%
\pgfpathlineto{\pgfqpoint{0.916842in}{1.291711in}}%
\pgfusepath{stroke}%
\end{pgfscope}%
\begin{pgfscope}%
\pgfpathrectangle{\pgfqpoint{0.100000in}{0.212622in}}{\pgfqpoint{3.696000in}{3.696000in}}%
\pgfusepath{clip}%
\pgfsetrectcap%
\pgfsetroundjoin%
\pgfsetlinewidth{1.505625pt}%
\definecolor{currentstroke}{rgb}{1.000000,0.000000,0.000000}%
\pgfsetstrokecolor{currentstroke}%
\pgfsetdash{}{0pt}%
\pgfpathmoveto{\pgfqpoint{0.906395in}{1.289089in}}%
\pgfpathlineto{\pgfqpoint{0.916842in}{1.291711in}}%
\pgfusepath{stroke}%
\end{pgfscope}%
\begin{pgfscope}%
\pgfpathrectangle{\pgfqpoint{0.100000in}{0.212622in}}{\pgfqpoint{3.696000in}{3.696000in}}%
\pgfusepath{clip}%
\pgfsetrectcap%
\pgfsetroundjoin%
\pgfsetlinewidth{1.505625pt}%
\definecolor{currentstroke}{rgb}{1.000000,0.000000,0.000000}%
\pgfsetstrokecolor{currentstroke}%
\pgfsetdash{}{0pt}%
\pgfpathmoveto{\pgfqpoint{0.892987in}{1.285240in}}%
\pgfpathlineto{\pgfqpoint{0.916842in}{1.291711in}}%
\pgfusepath{stroke}%
\end{pgfscope}%
\begin{pgfscope}%
\pgfpathrectangle{\pgfqpoint{0.100000in}{0.212622in}}{\pgfqpoint{3.696000in}{3.696000in}}%
\pgfusepath{clip}%
\pgfsetrectcap%
\pgfsetroundjoin%
\pgfsetlinewidth{1.505625pt}%
\definecolor{currentstroke}{rgb}{1.000000,0.000000,0.000000}%
\pgfsetstrokecolor{currentstroke}%
\pgfsetdash{}{0pt}%
\pgfpathmoveto{\pgfqpoint{0.875380in}{1.280545in}}%
\pgfpathlineto{\pgfqpoint{0.916842in}{1.291711in}}%
\pgfusepath{stroke}%
\end{pgfscope}%
\begin{pgfscope}%
\pgfpathrectangle{\pgfqpoint{0.100000in}{0.212622in}}{\pgfqpoint{3.696000in}{3.696000in}}%
\pgfusepath{clip}%
\pgfsetrectcap%
\pgfsetroundjoin%
\pgfsetlinewidth{1.505625pt}%
\definecolor{currentstroke}{rgb}{1.000000,0.000000,0.000000}%
\pgfsetstrokecolor{currentstroke}%
\pgfsetdash{}{0pt}%
\pgfpathmoveto{\pgfqpoint{0.851587in}{1.275014in}}%
\pgfpathlineto{\pgfqpoint{0.916842in}{1.291711in}}%
\pgfusepath{stroke}%
\end{pgfscope}%
\begin{pgfscope}%
\pgfpathrectangle{\pgfqpoint{0.100000in}{0.212622in}}{\pgfqpoint{3.696000in}{3.696000in}}%
\pgfusepath{clip}%
\pgfsetrectcap%
\pgfsetroundjoin%
\pgfsetlinewidth{1.505625pt}%
\definecolor{currentstroke}{rgb}{1.000000,0.000000,0.000000}%
\pgfsetstrokecolor{currentstroke}%
\pgfsetdash{}{0pt}%
\pgfpathmoveto{\pgfqpoint{0.824958in}{1.269252in}}%
\pgfpathlineto{\pgfqpoint{0.916842in}{1.291711in}}%
\pgfusepath{stroke}%
\end{pgfscope}%
\begin{pgfscope}%
\pgfpathrectangle{\pgfqpoint{0.100000in}{0.212622in}}{\pgfqpoint{3.696000in}{3.696000in}}%
\pgfusepath{clip}%
\pgfsetrectcap%
\pgfsetroundjoin%
\pgfsetlinewidth{1.505625pt}%
\definecolor{currentstroke}{rgb}{1.000000,0.000000,0.000000}%
\pgfsetstrokecolor{currentstroke}%
\pgfsetdash{}{0pt}%
\pgfpathmoveto{\pgfqpoint{0.796032in}{1.261512in}}%
\pgfpathlineto{\pgfqpoint{0.916842in}{1.291711in}}%
\pgfusepath{stroke}%
\end{pgfscope}%
\begin{pgfscope}%
\pgfpathrectangle{\pgfqpoint{0.100000in}{0.212622in}}{\pgfqpoint{3.696000in}{3.696000in}}%
\pgfusepath{clip}%
\pgfsetrectcap%
\pgfsetroundjoin%
\pgfsetlinewidth{1.505625pt}%
\definecolor{currentstroke}{rgb}{1.000000,0.000000,0.000000}%
\pgfsetstrokecolor{currentstroke}%
\pgfsetdash{}{0pt}%
\pgfpathmoveto{\pgfqpoint{0.764760in}{1.253804in}}%
\pgfpathlineto{\pgfqpoint{0.916842in}{1.291711in}}%
\pgfusepath{stroke}%
\end{pgfscope}%
\begin{pgfscope}%
\pgfpathrectangle{\pgfqpoint{0.100000in}{0.212622in}}{\pgfqpoint{3.696000in}{3.696000in}}%
\pgfusepath{clip}%
\pgfsetrectcap%
\pgfsetroundjoin%
\pgfsetlinewidth{1.505625pt}%
\definecolor{currentstroke}{rgb}{1.000000,0.000000,0.000000}%
\pgfsetstrokecolor{currentstroke}%
\pgfsetdash{}{0pt}%
\pgfpathmoveto{\pgfqpoint{0.747797in}{1.249252in}}%
\pgfpathlineto{\pgfqpoint{0.916842in}{1.291711in}}%
\pgfusepath{stroke}%
\end{pgfscope}%
\begin{pgfscope}%
\pgfpathrectangle{\pgfqpoint{0.100000in}{0.212622in}}{\pgfqpoint{3.696000in}{3.696000in}}%
\pgfusepath{clip}%
\pgfsetrectcap%
\pgfsetroundjoin%
\pgfsetlinewidth{1.505625pt}%
\definecolor{currentstroke}{rgb}{1.000000,0.000000,0.000000}%
\pgfsetstrokecolor{currentstroke}%
\pgfsetdash{}{0pt}%
\pgfpathmoveto{\pgfqpoint{0.738388in}{1.246685in}}%
\pgfpathlineto{\pgfqpoint{0.916842in}{1.291711in}}%
\pgfusepath{stroke}%
\end{pgfscope}%
\begin{pgfscope}%
\pgfpathrectangle{\pgfqpoint{0.100000in}{0.212622in}}{\pgfqpoint{3.696000in}{3.696000in}}%
\pgfusepath{clip}%
\pgfsetrectcap%
\pgfsetroundjoin%
\pgfsetlinewidth{1.505625pt}%
\definecolor{currentstroke}{rgb}{1.000000,0.000000,0.000000}%
\pgfsetstrokecolor{currentstroke}%
\pgfsetdash{}{0pt}%
\pgfpathmoveto{\pgfqpoint{0.733294in}{1.245314in}}%
\pgfpathlineto{\pgfqpoint{0.916842in}{1.291711in}}%
\pgfusepath{stroke}%
\end{pgfscope}%
\begin{pgfscope}%
\pgfpathrectangle{\pgfqpoint{0.100000in}{0.212622in}}{\pgfqpoint{3.696000in}{3.696000in}}%
\pgfusepath{clip}%
\pgfsetrectcap%
\pgfsetroundjoin%
\pgfsetlinewidth{1.505625pt}%
\definecolor{currentstroke}{rgb}{1.000000,0.000000,0.000000}%
\pgfsetstrokecolor{currentstroke}%
\pgfsetdash{}{0pt}%
\pgfpathmoveto{\pgfqpoint{0.730443in}{1.244530in}}%
\pgfpathlineto{\pgfqpoint{0.916842in}{1.291711in}}%
\pgfusepath{stroke}%
\end{pgfscope}%
\begin{pgfscope}%
\pgfpathrectangle{\pgfqpoint{0.100000in}{0.212622in}}{\pgfqpoint{3.696000in}{3.696000in}}%
\pgfusepath{clip}%
\pgfsetrectcap%
\pgfsetroundjoin%
\pgfsetlinewidth{1.505625pt}%
\definecolor{currentstroke}{rgb}{1.000000,0.000000,0.000000}%
\pgfsetstrokecolor{currentstroke}%
\pgfsetdash{}{0pt}%
\pgfpathmoveto{\pgfqpoint{0.728826in}{1.244140in}}%
\pgfpathlineto{\pgfqpoint{0.916842in}{1.291711in}}%
\pgfusepath{stroke}%
\end{pgfscope}%
\begin{pgfscope}%
\pgfpathrectangle{\pgfqpoint{0.100000in}{0.212622in}}{\pgfqpoint{3.696000in}{3.696000in}}%
\pgfusepath{clip}%
\pgfsetrectcap%
\pgfsetroundjoin%
\pgfsetlinewidth{1.505625pt}%
\definecolor{currentstroke}{rgb}{1.000000,0.000000,0.000000}%
\pgfsetstrokecolor{currentstroke}%
\pgfsetdash{}{0pt}%
\pgfpathmoveto{\pgfqpoint{0.727967in}{1.243923in}}%
\pgfpathlineto{\pgfqpoint{0.916842in}{1.291711in}}%
\pgfusepath{stroke}%
\end{pgfscope}%
\begin{pgfscope}%
\pgfpathrectangle{\pgfqpoint{0.100000in}{0.212622in}}{\pgfqpoint{3.696000in}{3.696000in}}%
\pgfusepath{clip}%
\pgfsetrectcap%
\pgfsetroundjoin%
\pgfsetlinewidth{1.505625pt}%
\definecolor{currentstroke}{rgb}{1.000000,0.000000,0.000000}%
\pgfsetstrokecolor{currentstroke}%
\pgfsetdash{}{0pt}%
\pgfpathmoveto{\pgfqpoint{0.727473in}{1.243823in}}%
\pgfpathlineto{\pgfqpoint{0.916842in}{1.291711in}}%
\pgfusepath{stroke}%
\end{pgfscope}%
\begin{pgfscope}%
\pgfpathrectangle{\pgfqpoint{0.100000in}{0.212622in}}{\pgfqpoint{3.696000in}{3.696000in}}%
\pgfusepath{clip}%
\pgfsetrectcap%
\pgfsetroundjoin%
\pgfsetlinewidth{1.505625pt}%
\definecolor{currentstroke}{rgb}{1.000000,0.000000,0.000000}%
\pgfsetstrokecolor{currentstroke}%
\pgfsetdash{}{0pt}%
\pgfpathmoveto{\pgfqpoint{0.727214in}{1.243780in}}%
\pgfpathlineto{\pgfqpoint{0.916842in}{1.291711in}}%
\pgfusepath{stroke}%
\end{pgfscope}%
\begin{pgfscope}%
\pgfpathrectangle{\pgfqpoint{0.100000in}{0.212622in}}{\pgfqpoint{3.696000in}{3.696000in}}%
\pgfusepath{clip}%
\pgfsetrectcap%
\pgfsetroundjoin%
\pgfsetlinewidth{1.505625pt}%
\definecolor{currentstroke}{rgb}{1.000000,0.000000,0.000000}%
\pgfsetstrokecolor{currentstroke}%
\pgfsetdash{}{0pt}%
\pgfpathmoveto{\pgfqpoint{0.727075in}{1.243760in}}%
\pgfpathlineto{\pgfqpoint{0.916842in}{1.291711in}}%
\pgfusepath{stroke}%
\end{pgfscope}%
\begin{pgfscope}%
\pgfpathrectangle{\pgfqpoint{0.100000in}{0.212622in}}{\pgfqpoint{3.696000in}{3.696000in}}%
\pgfusepath{clip}%
\pgfsetrectcap%
\pgfsetroundjoin%
\pgfsetlinewidth{1.505625pt}%
\definecolor{currentstroke}{rgb}{1.000000,0.000000,0.000000}%
\pgfsetstrokecolor{currentstroke}%
\pgfsetdash{}{0pt}%
\pgfpathmoveto{\pgfqpoint{0.726998in}{1.243749in}}%
\pgfpathlineto{\pgfqpoint{0.916842in}{1.291711in}}%
\pgfusepath{stroke}%
\end{pgfscope}%
\begin{pgfscope}%
\pgfpathrectangle{\pgfqpoint{0.100000in}{0.212622in}}{\pgfqpoint{3.696000in}{3.696000in}}%
\pgfusepath{clip}%
\pgfsetrectcap%
\pgfsetroundjoin%
\pgfsetlinewidth{1.505625pt}%
\definecolor{currentstroke}{rgb}{1.000000,0.000000,0.000000}%
\pgfsetstrokecolor{currentstroke}%
\pgfsetdash{}{0pt}%
\pgfpathmoveto{\pgfqpoint{0.726955in}{1.243743in}}%
\pgfpathlineto{\pgfqpoint{0.916842in}{1.291711in}}%
\pgfusepath{stroke}%
\end{pgfscope}%
\begin{pgfscope}%
\pgfpathrectangle{\pgfqpoint{0.100000in}{0.212622in}}{\pgfqpoint{3.696000in}{3.696000in}}%
\pgfusepath{clip}%
\pgfsetrectcap%
\pgfsetroundjoin%
\pgfsetlinewidth{1.505625pt}%
\definecolor{currentstroke}{rgb}{1.000000,0.000000,0.000000}%
\pgfsetstrokecolor{currentstroke}%
\pgfsetdash{}{0pt}%
\pgfpathmoveto{\pgfqpoint{0.726932in}{1.243739in}}%
\pgfpathlineto{\pgfqpoint{0.916842in}{1.291711in}}%
\pgfusepath{stroke}%
\end{pgfscope}%
\begin{pgfscope}%
\pgfpathrectangle{\pgfqpoint{0.100000in}{0.212622in}}{\pgfqpoint{3.696000in}{3.696000in}}%
\pgfusepath{clip}%
\pgfsetrectcap%
\pgfsetroundjoin%
\pgfsetlinewidth{1.505625pt}%
\definecolor{currentstroke}{rgb}{1.000000,0.000000,0.000000}%
\pgfsetstrokecolor{currentstroke}%
\pgfsetdash{}{0pt}%
\pgfpathmoveto{\pgfqpoint{0.726919in}{1.243737in}}%
\pgfpathlineto{\pgfqpoint{0.916842in}{1.291711in}}%
\pgfusepath{stroke}%
\end{pgfscope}%
\begin{pgfscope}%
\pgfpathrectangle{\pgfqpoint{0.100000in}{0.212622in}}{\pgfqpoint{3.696000in}{3.696000in}}%
\pgfusepath{clip}%
\pgfsetrectcap%
\pgfsetroundjoin%
\pgfsetlinewidth{1.505625pt}%
\definecolor{currentstroke}{rgb}{1.000000,0.000000,0.000000}%
\pgfsetstrokecolor{currentstroke}%
\pgfsetdash{}{0pt}%
\pgfpathmoveto{\pgfqpoint{0.726912in}{1.243737in}}%
\pgfpathlineto{\pgfqpoint{0.916842in}{1.291711in}}%
\pgfusepath{stroke}%
\end{pgfscope}%
\begin{pgfscope}%
\pgfpathrectangle{\pgfqpoint{0.100000in}{0.212622in}}{\pgfqpoint{3.696000in}{3.696000in}}%
\pgfusepath{clip}%
\pgfsetrectcap%
\pgfsetroundjoin%
\pgfsetlinewidth{1.505625pt}%
\definecolor{currentstroke}{rgb}{1.000000,0.000000,0.000000}%
\pgfsetstrokecolor{currentstroke}%
\pgfsetdash{}{0pt}%
\pgfpathmoveto{\pgfqpoint{0.726908in}{1.243736in}}%
\pgfpathlineto{\pgfqpoint{0.916842in}{1.291711in}}%
\pgfusepath{stroke}%
\end{pgfscope}%
\begin{pgfscope}%
\pgfpathrectangle{\pgfqpoint{0.100000in}{0.212622in}}{\pgfqpoint{3.696000in}{3.696000in}}%
\pgfusepath{clip}%
\pgfsetrectcap%
\pgfsetroundjoin%
\pgfsetlinewidth{1.505625pt}%
\definecolor{currentstroke}{rgb}{1.000000,0.000000,0.000000}%
\pgfsetstrokecolor{currentstroke}%
\pgfsetdash{}{0pt}%
\pgfpathmoveto{\pgfqpoint{0.726906in}{1.243736in}}%
\pgfpathlineto{\pgfqpoint{0.916842in}{1.291711in}}%
\pgfusepath{stroke}%
\end{pgfscope}%
\begin{pgfscope}%
\pgfpathrectangle{\pgfqpoint{0.100000in}{0.212622in}}{\pgfqpoint{3.696000in}{3.696000in}}%
\pgfusepath{clip}%
\pgfsetrectcap%
\pgfsetroundjoin%
\pgfsetlinewidth{1.505625pt}%
\definecolor{currentstroke}{rgb}{1.000000,0.000000,0.000000}%
\pgfsetstrokecolor{currentstroke}%
\pgfsetdash{}{0pt}%
\pgfpathmoveto{\pgfqpoint{0.726905in}{1.243736in}}%
\pgfpathlineto{\pgfqpoint{0.916842in}{1.291711in}}%
\pgfusepath{stroke}%
\end{pgfscope}%
\begin{pgfscope}%
\pgfpathrectangle{\pgfqpoint{0.100000in}{0.212622in}}{\pgfqpoint{3.696000in}{3.696000in}}%
\pgfusepath{clip}%
\pgfsetrectcap%
\pgfsetroundjoin%
\pgfsetlinewidth{1.505625pt}%
\definecolor{currentstroke}{rgb}{1.000000,0.000000,0.000000}%
\pgfsetstrokecolor{currentstroke}%
\pgfsetdash{}{0pt}%
\pgfpathmoveto{\pgfqpoint{0.726904in}{1.243736in}}%
\pgfpathlineto{\pgfqpoint{0.916842in}{1.291711in}}%
\pgfusepath{stroke}%
\end{pgfscope}%
\begin{pgfscope}%
\pgfpathrectangle{\pgfqpoint{0.100000in}{0.212622in}}{\pgfqpoint{3.696000in}{3.696000in}}%
\pgfusepath{clip}%
\pgfsetrectcap%
\pgfsetroundjoin%
\pgfsetlinewidth{1.505625pt}%
\definecolor{currentstroke}{rgb}{1.000000,0.000000,0.000000}%
\pgfsetstrokecolor{currentstroke}%
\pgfsetdash{}{0pt}%
\pgfpathmoveto{\pgfqpoint{0.726904in}{1.243736in}}%
\pgfpathlineto{\pgfqpoint{0.916842in}{1.291711in}}%
\pgfusepath{stroke}%
\end{pgfscope}%
\begin{pgfscope}%
\pgfpathrectangle{\pgfqpoint{0.100000in}{0.212622in}}{\pgfqpoint{3.696000in}{3.696000in}}%
\pgfusepath{clip}%
\pgfsetrectcap%
\pgfsetroundjoin%
\pgfsetlinewidth{1.505625pt}%
\definecolor{currentstroke}{rgb}{1.000000,0.000000,0.000000}%
\pgfsetstrokecolor{currentstroke}%
\pgfsetdash{}{0pt}%
\pgfpathmoveto{\pgfqpoint{0.726904in}{1.243736in}}%
\pgfpathlineto{\pgfqpoint{0.916842in}{1.291711in}}%
\pgfusepath{stroke}%
\end{pgfscope}%
\begin{pgfscope}%
\pgfpathrectangle{\pgfqpoint{0.100000in}{0.212622in}}{\pgfqpoint{3.696000in}{3.696000in}}%
\pgfusepath{clip}%
\pgfsetrectcap%
\pgfsetroundjoin%
\pgfsetlinewidth{1.505625pt}%
\definecolor{currentstroke}{rgb}{1.000000,0.000000,0.000000}%
\pgfsetstrokecolor{currentstroke}%
\pgfsetdash{}{0pt}%
\pgfpathmoveto{\pgfqpoint{0.726904in}{1.243736in}}%
\pgfpathlineto{\pgfqpoint{0.916842in}{1.291711in}}%
\pgfusepath{stroke}%
\end{pgfscope}%
\begin{pgfscope}%
\pgfpathrectangle{\pgfqpoint{0.100000in}{0.212622in}}{\pgfqpoint{3.696000in}{3.696000in}}%
\pgfusepath{clip}%
\pgfsetrectcap%
\pgfsetroundjoin%
\pgfsetlinewidth{1.505625pt}%
\definecolor{currentstroke}{rgb}{1.000000,0.000000,0.000000}%
\pgfsetstrokecolor{currentstroke}%
\pgfsetdash{}{0pt}%
\pgfpathmoveto{\pgfqpoint{0.726904in}{1.243736in}}%
\pgfpathlineto{\pgfqpoint{0.916842in}{1.291711in}}%
\pgfusepath{stroke}%
\end{pgfscope}%
\begin{pgfscope}%
\pgfpathrectangle{\pgfqpoint{0.100000in}{0.212622in}}{\pgfqpoint{3.696000in}{3.696000in}}%
\pgfusepath{clip}%
\pgfsetrectcap%
\pgfsetroundjoin%
\pgfsetlinewidth{1.505625pt}%
\definecolor{currentstroke}{rgb}{1.000000,0.000000,0.000000}%
\pgfsetstrokecolor{currentstroke}%
\pgfsetdash{}{0pt}%
\pgfpathmoveto{\pgfqpoint{0.726903in}{1.243736in}}%
\pgfpathlineto{\pgfqpoint{0.916842in}{1.291711in}}%
\pgfusepath{stroke}%
\end{pgfscope}%
\begin{pgfscope}%
\pgfpathrectangle{\pgfqpoint{0.100000in}{0.212622in}}{\pgfqpoint{3.696000in}{3.696000in}}%
\pgfusepath{clip}%
\pgfsetrectcap%
\pgfsetroundjoin%
\pgfsetlinewidth{1.505625pt}%
\definecolor{currentstroke}{rgb}{1.000000,0.000000,0.000000}%
\pgfsetstrokecolor{currentstroke}%
\pgfsetdash{}{0pt}%
\pgfpathmoveto{\pgfqpoint{0.726903in}{1.243736in}}%
\pgfpathlineto{\pgfqpoint{0.916842in}{1.291711in}}%
\pgfusepath{stroke}%
\end{pgfscope}%
\begin{pgfscope}%
\pgfpathrectangle{\pgfqpoint{0.100000in}{0.212622in}}{\pgfqpoint{3.696000in}{3.696000in}}%
\pgfusepath{clip}%
\pgfsetrectcap%
\pgfsetroundjoin%
\pgfsetlinewidth{1.505625pt}%
\definecolor{currentstroke}{rgb}{1.000000,0.000000,0.000000}%
\pgfsetstrokecolor{currentstroke}%
\pgfsetdash{}{0pt}%
\pgfpathmoveto{\pgfqpoint{0.726903in}{1.243736in}}%
\pgfpathlineto{\pgfqpoint{0.916842in}{1.291711in}}%
\pgfusepath{stroke}%
\end{pgfscope}%
\begin{pgfscope}%
\pgfpathrectangle{\pgfqpoint{0.100000in}{0.212622in}}{\pgfqpoint{3.696000in}{3.696000in}}%
\pgfusepath{clip}%
\pgfsetrectcap%
\pgfsetroundjoin%
\pgfsetlinewidth{1.505625pt}%
\definecolor{currentstroke}{rgb}{1.000000,0.000000,0.000000}%
\pgfsetstrokecolor{currentstroke}%
\pgfsetdash{}{0pt}%
\pgfpathmoveto{\pgfqpoint{0.726903in}{1.243736in}}%
\pgfpathlineto{\pgfqpoint{0.916842in}{1.291711in}}%
\pgfusepath{stroke}%
\end{pgfscope}%
\begin{pgfscope}%
\pgfpathrectangle{\pgfqpoint{0.100000in}{0.212622in}}{\pgfqpoint{3.696000in}{3.696000in}}%
\pgfusepath{clip}%
\pgfsetrectcap%
\pgfsetroundjoin%
\pgfsetlinewidth{1.505625pt}%
\definecolor{currentstroke}{rgb}{1.000000,0.000000,0.000000}%
\pgfsetstrokecolor{currentstroke}%
\pgfsetdash{}{0pt}%
\pgfpathmoveto{\pgfqpoint{0.726903in}{1.243736in}}%
\pgfpathlineto{\pgfqpoint{0.916842in}{1.291711in}}%
\pgfusepath{stroke}%
\end{pgfscope}%
\begin{pgfscope}%
\pgfpathrectangle{\pgfqpoint{0.100000in}{0.212622in}}{\pgfqpoint{3.696000in}{3.696000in}}%
\pgfusepath{clip}%
\pgfsetrectcap%
\pgfsetroundjoin%
\pgfsetlinewidth{1.505625pt}%
\definecolor{currentstroke}{rgb}{1.000000,0.000000,0.000000}%
\pgfsetstrokecolor{currentstroke}%
\pgfsetdash{}{0pt}%
\pgfpathmoveto{\pgfqpoint{0.726903in}{1.243736in}}%
\pgfpathlineto{\pgfqpoint{0.916842in}{1.291711in}}%
\pgfusepath{stroke}%
\end{pgfscope}%
\begin{pgfscope}%
\pgfpathrectangle{\pgfqpoint{0.100000in}{0.212622in}}{\pgfqpoint{3.696000in}{3.696000in}}%
\pgfusepath{clip}%
\pgfsetrectcap%
\pgfsetroundjoin%
\pgfsetlinewidth{1.505625pt}%
\definecolor{currentstroke}{rgb}{1.000000,0.000000,0.000000}%
\pgfsetstrokecolor{currentstroke}%
\pgfsetdash{}{0pt}%
\pgfpathmoveto{\pgfqpoint{0.726903in}{1.243736in}}%
\pgfpathlineto{\pgfqpoint{0.916842in}{1.291711in}}%
\pgfusepath{stroke}%
\end{pgfscope}%
\begin{pgfscope}%
\pgfpathrectangle{\pgfqpoint{0.100000in}{0.212622in}}{\pgfqpoint{3.696000in}{3.696000in}}%
\pgfusepath{clip}%
\pgfsetrectcap%
\pgfsetroundjoin%
\pgfsetlinewidth{1.505625pt}%
\definecolor{currentstroke}{rgb}{1.000000,0.000000,0.000000}%
\pgfsetstrokecolor{currentstroke}%
\pgfsetdash{}{0pt}%
\pgfpathmoveto{\pgfqpoint{0.726903in}{1.243736in}}%
\pgfpathlineto{\pgfqpoint{0.916842in}{1.291711in}}%
\pgfusepath{stroke}%
\end{pgfscope}%
\begin{pgfscope}%
\pgfpathrectangle{\pgfqpoint{0.100000in}{0.212622in}}{\pgfqpoint{3.696000in}{3.696000in}}%
\pgfusepath{clip}%
\pgfsetrectcap%
\pgfsetroundjoin%
\pgfsetlinewidth{1.505625pt}%
\definecolor{currentstroke}{rgb}{1.000000,0.000000,0.000000}%
\pgfsetstrokecolor{currentstroke}%
\pgfsetdash{}{0pt}%
\pgfpathmoveto{\pgfqpoint{0.726903in}{1.243736in}}%
\pgfpathlineto{\pgfqpoint{0.916842in}{1.291711in}}%
\pgfusepath{stroke}%
\end{pgfscope}%
\begin{pgfscope}%
\pgfpathrectangle{\pgfqpoint{0.100000in}{0.212622in}}{\pgfqpoint{3.696000in}{3.696000in}}%
\pgfusepath{clip}%
\pgfsetrectcap%
\pgfsetroundjoin%
\pgfsetlinewidth{1.505625pt}%
\definecolor{currentstroke}{rgb}{1.000000,0.000000,0.000000}%
\pgfsetstrokecolor{currentstroke}%
\pgfsetdash{}{0pt}%
\pgfpathmoveto{\pgfqpoint{0.726903in}{1.243736in}}%
\pgfpathlineto{\pgfqpoint{0.916842in}{1.291711in}}%
\pgfusepath{stroke}%
\end{pgfscope}%
\begin{pgfscope}%
\pgfpathrectangle{\pgfqpoint{0.100000in}{0.212622in}}{\pgfqpoint{3.696000in}{3.696000in}}%
\pgfusepath{clip}%
\pgfsetrectcap%
\pgfsetroundjoin%
\pgfsetlinewidth{1.505625pt}%
\definecolor{currentstroke}{rgb}{1.000000,0.000000,0.000000}%
\pgfsetstrokecolor{currentstroke}%
\pgfsetdash{}{0pt}%
\pgfpathmoveto{\pgfqpoint{0.726903in}{1.243736in}}%
\pgfpathlineto{\pgfqpoint{0.916842in}{1.291711in}}%
\pgfusepath{stroke}%
\end{pgfscope}%
\begin{pgfscope}%
\pgfpathrectangle{\pgfqpoint{0.100000in}{0.212622in}}{\pgfqpoint{3.696000in}{3.696000in}}%
\pgfusepath{clip}%
\pgfsetrectcap%
\pgfsetroundjoin%
\pgfsetlinewidth{1.505625pt}%
\definecolor{currentstroke}{rgb}{1.000000,0.000000,0.000000}%
\pgfsetstrokecolor{currentstroke}%
\pgfsetdash{}{0pt}%
\pgfpathmoveto{\pgfqpoint{0.726903in}{1.243736in}}%
\pgfpathlineto{\pgfqpoint{0.916842in}{1.291711in}}%
\pgfusepath{stroke}%
\end{pgfscope}%
\begin{pgfscope}%
\pgfpathrectangle{\pgfqpoint{0.100000in}{0.212622in}}{\pgfqpoint{3.696000in}{3.696000in}}%
\pgfusepath{clip}%
\pgfsetrectcap%
\pgfsetroundjoin%
\pgfsetlinewidth{1.505625pt}%
\definecolor{currentstroke}{rgb}{1.000000,0.000000,0.000000}%
\pgfsetstrokecolor{currentstroke}%
\pgfsetdash{}{0pt}%
\pgfpathmoveto{\pgfqpoint{0.726903in}{1.243736in}}%
\pgfpathlineto{\pgfqpoint{0.916842in}{1.291711in}}%
\pgfusepath{stroke}%
\end{pgfscope}%
\begin{pgfscope}%
\pgfpathrectangle{\pgfqpoint{0.100000in}{0.212622in}}{\pgfqpoint{3.696000in}{3.696000in}}%
\pgfusepath{clip}%
\pgfsetrectcap%
\pgfsetroundjoin%
\pgfsetlinewidth{1.505625pt}%
\definecolor{currentstroke}{rgb}{1.000000,0.000000,0.000000}%
\pgfsetstrokecolor{currentstroke}%
\pgfsetdash{}{0pt}%
\pgfpathmoveto{\pgfqpoint{0.726903in}{1.243736in}}%
\pgfpathlineto{\pgfqpoint{0.916842in}{1.291711in}}%
\pgfusepath{stroke}%
\end{pgfscope}%
\begin{pgfscope}%
\pgfpathrectangle{\pgfqpoint{0.100000in}{0.212622in}}{\pgfqpoint{3.696000in}{3.696000in}}%
\pgfusepath{clip}%
\pgfsetrectcap%
\pgfsetroundjoin%
\pgfsetlinewidth{1.505625pt}%
\definecolor{currentstroke}{rgb}{1.000000,0.000000,0.000000}%
\pgfsetstrokecolor{currentstroke}%
\pgfsetdash{}{0pt}%
\pgfpathmoveto{\pgfqpoint{0.726903in}{1.243736in}}%
\pgfpathlineto{\pgfqpoint{0.916842in}{1.291711in}}%
\pgfusepath{stroke}%
\end{pgfscope}%
\begin{pgfscope}%
\pgfpathrectangle{\pgfqpoint{0.100000in}{0.212622in}}{\pgfqpoint{3.696000in}{3.696000in}}%
\pgfusepath{clip}%
\pgfsetrectcap%
\pgfsetroundjoin%
\pgfsetlinewidth{1.505625pt}%
\definecolor{currentstroke}{rgb}{1.000000,0.000000,0.000000}%
\pgfsetstrokecolor{currentstroke}%
\pgfsetdash{}{0pt}%
\pgfpathmoveto{\pgfqpoint{0.726903in}{1.243736in}}%
\pgfpathlineto{\pgfqpoint{0.916842in}{1.291711in}}%
\pgfusepath{stroke}%
\end{pgfscope}%
\begin{pgfscope}%
\pgfpathrectangle{\pgfqpoint{0.100000in}{0.212622in}}{\pgfqpoint{3.696000in}{3.696000in}}%
\pgfusepath{clip}%
\pgfsetrectcap%
\pgfsetroundjoin%
\pgfsetlinewidth{1.505625pt}%
\definecolor{currentstroke}{rgb}{1.000000,0.000000,0.000000}%
\pgfsetstrokecolor{currentstroke}%
\pgfsetdash{}{0pt}%
\pgfpathmoveto{\pgfqpoint{0.726903in}{1.243736in}}%
\pgfpathlineto{\pgfqpoint{0.916842in}{1.291711in}}%
\pgfusepath{stroke}%
\end{pgfscope}%
\begin{pgfscope}%
\pgfpathrectangle{\pgfqpoint{0.100000in}{0.212622in}}{\pgfqpoint{3.696000in}{3.696000in}}%
\pgfusepath{clip}%
\pgfsetrectcap%
\pgfsetroundjoin%
\pgfsetlinewidth{1.505625pt}%
\definecolor{currentstroke}{rgb}{1.000000,0.000000,0.000000}%
\pgfsetstrokecolor{currentstroke}%
\pgfsetdash{}{0pt}%
\pgfpathmoveto{\pgfqpoint{0.726903in}{1.243736in}}%
\pgfpathlineto{\pgfqpoint{0.916842in}{1.291711in}}%
\pgfusepath{stroke}%
\end{pgfscope}%
\begin{pgfscope}%
\pgfpathrectangle{\pgfqpoint{0.100000in}{0.212622in}}{\pgfqpoint{3.696000in}{3.696000in}}%
\pgfusepath{clip}%
\pgfsetrectcap%
\pgfsetroundjoin%
\pgfsetlinewidth{1.505625pt}%
\definecolor{currentstroke}{rgb}{1.000000,0.000000,0.000000}%
\pgfsetstrokecolor{currentstroke}%
\pgfsetdash{}{0pt}%
\pgfpathmoveto{\pgfqpoint{0.726903in}{1.243736in}}%
\pgfpathlineto{\pgfqpoint{0.916842in}{1.291711in}}%
\pgfusepath{stroke}%
\end{pgfscope}%
\begin{pgfscope}%
\pgfpathrectangle{\pgfqpoint{0.100000in}{0.212622in}}{\pgfqpoint{3.696000in}{3.696000in}}%
\pgfusepath{clip}%
\pgfsetrectcap%
\pgfsetroundjoin%
\pgfsetlinewidth{1.505625pt}%
\definecolor{currentstroke}{rgb}{1.000000,0.000000,0.000000}%
\pgfsetstrokecolor{currentstroke}%
\pgfsetdash{}{0pt}%
\pgfpathmoveto{\pgfqpoint{0.726903in}{1.243736in}}%
\pgfpathlineto{\pgfqpoint{0.916842in}{1.291711in}}%
\pgfusepath{stroke}%
\end{pgfscope}%
\begin{pgfscope}%
\pgfpathrectangle{\pgfqpoint{0.100000in}{0.212622in}}{\pgfqpoint{3.696000in}{3.696000in}}%
\pgfusepath{clip}%
\pgfsetrectcap%
\pgfsetroundjoin%
\pgfsetlinewidth{1.505625pt}%
\definecolor{currentstroke}{rgb}{1.000000,0.000000,0.000000}%
\pgfsetstrokecolor{currentstroke}%
\pgfsetdash{}{0pt}%
\pgfpathmoveto{\pgfqpoint{0.726903in}{1.243736in}}%
\pgfpathlineto{\pgfqpoint{0.916842in}{1.291711in}}%
\pgfusepath{stroke}%
\end{pgfscope}%
\begin{pgfscope}%
\pgfpathrectangle{\pgfqpoint{0.100000in}{0.212622in}}{\pgfqpoint{3.696000in}{3.696000in}}%
\pgfusepath{clip}%
\pgfsetrectcap%
\pgfsetroundjoin%
\pgfsetlinewidth{1.505625pt}%
\definecolor{currentstroke}{rgb}{1.000000,0.000000,0.000000}%
\pgfsetstrokecolor{currentstroke}%
\pgfsetdash{}{0pt}%
\pgfpathmoveto{\pgfqpoint{0.726903in}{1.243736in}}%
\pgfpathlineto{\pgfqpoint{0.916842in}{1.291711in}}%
\pgfusepath{stroke}%
\end{pgfscope}%
\begin{pgfscope}%
\pgfpathrectangle{\pgfqpoint{0.100000in}{0.212622in}}{\pgfqpoint{3.696000in}{3.696000in}}%
\pgfusepath{clip}%
\pgfsetrectcap%
\pgfsetroundjoin%
\pgfsetlinewidth{1.505625pt}%
\definecolor{currentstroke}{rgb}{1.000000,0.000000,0.000000}%
\pgfsetstrokecolor{currentstroke}%
\pgfsetdash{}{0pt}%
\pgfpathmoveto{\pgfqpoint{0.726903in}{1.243736in}}%
\pgfpathlineto{\pgfqpoint{0.916842in}{1.291711in}}%
\pgfusepath{stroke}%
\end{pgfscope}%
\begin{pgfscope}%
\pgfpathrectangle{\pgfqpoint{0.100000in}{0.212622in}}{\pgfqpoint{3.696000in}{3.696000in}}%
\pgfusepath{clip}%
\pgfsetrectcap%
\pgfsetroundjoin%
\pgfsetlinewidth{1.505625pt}%
\definecolor{currentstroke}{rgb}{1.000000,0.000000,0.000000}%
\pgfsetstrokecolor{currentstroke}%
\pgfsetdash{}{0pt}%
\pgfpathmoveto{\pgfqpoint{0.726903in}{1.243736in}}%
\pgfpathlineto{\pgfqpoint{0.916842in}{1.291711in}}%
\pgfusepath{stroke}%
\end{pgfscope}%
\begin{pgfscope}%
\pgfpathrectangle{\pgfqpoint{0.100000in}{0.212622in}}{\pgfqpoint{3.696000in}{3.696000in}}%
\pgfusepath{clip}%
\pgfsetrectcap%
\pgfsetroundjoin%
\pgfsetlinewidth{1.505625pt}%
\definecolor{currentstroke}{rgb}{1.000000,0.000000,0.000000}%
\pgfsetstrokecolor{currentstroke}%
\pgfsetdash{}{0pt}%
\pgfpathmoveto{\pgfqpoint{0.726903in}{1.243736in}}%
\pgfpathlineto{\pgfqpoint{0.916842in}{1.291711in}}%
\pgfusepath{stroke}%
\end{pgfscope}%
\begin{pgfscope}%
\pgfpathrectangle{\pgfqpoint{0.100000in}{0.212622in}}{\pgfqpoint{3.696000in}{3.696000in}}%
\pgfusepath{clip}%
\pgfsetrectcap%
\pgfsetroundjoin%
\pgfsetlinewidth{1.505625pt}%
\definecolor{currentstroke}{rgb}{1.000000,0.000000,0.000000}%
\pgfsetstrokecolor{currentstroke}%
\pgfsetdash{}{0pt}%
\pgfpathmoveto{\pgfqpoint{0.726903in}{1.243736in}}%
\pgfpathlineto{\pgfqpoint{0.916842in}{1.291711in}}%
\pgfusepath{stroke}%
\end{pgfscope}%
\begin{pgfscope}%
\pgfpathrectangle{\pgfqpoint{0.100000in}{0.212622in}}{\pgfqpoint{3.696000in}{3.696000in}}%
\pgfusepath{clip}%
\pgfsetrectcap%
\pgfsetroundjoin%
\pgfsetlinewidth{1.505625pt}%
\definecolor{currentstroke}{rgb}{1.000000,0.000000,0.000000}%
\pgfsetstrokecolor{currentstroke}%
\pgfsetdash{}{0pt}%
\pgfpathmoveto{\pgfqpoint{0.726903in}{1.243736in}}%
\pgfpathlineto{\pgfqpoint{0.916842in}{1.291711in}}%
\pgfusepath{stroke}%
\end{pgfscope}%
\begin{pgfscope}%
\pgfpathrectangle{\pgfqpoint{0.100000in}{0.212622in}}{\pgfqpoint{3.696000in}{3.696000in}}%
\pgfusepath{clip}%
\pgfsetrectcap%
\pgfsetroundjoin%
\pgfsetlinewidth{1.505625pt}%
\definecolor{currentstroke}{rgb}{1.000000,0.000000,0.000000}%
\pgfsetstrokecolor{currentstroke}%
\pgfsetdash{}{0pt}%
\pgfpathmoveto{\pgfqpoint{0.726903in}{1.243736in}}%
\pgfpathlineto{\pgfqpoint{0.916842in}{1.291711in}}%
\pgfusepath{stroke}%
\end{pgfscope}%
\begin{pgfscope}%
\pgfpathrectangle{\pgfqpoint{0.100000in}{0.212622in}}{\pgfqpoint{3.696000in}{3.696000in}}%
\pgfusepath{clip}%
\pgfsetrectcap%
\pgfsetroundjoin%
\pgfsetlinewidth{1.505625pt}%
\definecolor{currentstroke}{rgb}{1.000000,0.000000,0.000000}%
\pgfsetstrokecolor{currentstroke}%
\pgfsetdash{}{0pt}%
\pgfpathmoveto{\pgfqpoint{0.726903in}{1.243736in}}%
\pgfpathlineto{\pgfqpoint{0.916842in}{1.291711in}}%
\pgfusepath{stroke}%
\end{pgfscope}%
\begin{pgfscope}%
\pgfpathrectangle{\pgfqpoint{0.100000in}{0.212622in}}{\pgfqpoint{3.696000in}{3.696000in}}%
\pgfusepath{clip}%
\pgfsetrectcap%
\pgfsetroundjoin%
\pgfsetlinewidth{1.505625pt}%
\definecolor{currentstroke}{rgb}{1.000000,0.000000,0.000000}%
\pgfsetstrokecolor{currentstroke}%
\pgfsetdash{}{0pt}%
\pgfpathmoveto{\pgfqpoint{0.726903in}{1.243736in}}%
\pgfpathlineto{\pgfqpoint{0.916842in}{1.291711in}}%
\pgfusepath{stroke}%
\end{pgfscope}%
\begin{pgfscope}%
\pgfpathrectangle{\pgfqpoint{0.100000in}{0.212622in}}{\pgfqpoint{3.696000in}{3.696000in}}%
\pgfusepath{clip}%
\pgfsetrectcap%
\pgfsetroundjoin%
\pgfsetlinewidth{1.505625pt}%
\definecolor{currentstroke}{rgb}{1.000000,0.000000,0.000000}%
\pgfsetstrokecolor{currentstroke}%
\pgfsetdash{}{0pt}%
\pgfpathmoveto{\pgfqpoint{0.726903in}{1.243736in}}%
\pgfpathlineto{\pgfqpoint{0.916842in}{1.291711in}}%
\pgfusepath{stroke}%
\end{pgfscope}%
\begin{pgfscope}%
\pgfpathrectangle{\pgfqpoint{0.100000in}{0.212622in}}{\pgfqpoint{3.696000in}{3.696000in}}%
\pgfusepath{clip}%
\pgfsetrectcap%
\pgfsetroundjoin%
\pgfsetlinewidth{1.505625pt}%
\definecolor{currentstroke}{rgb}{1.000000,0.000000,0.000000}%
\pgfsetstrokecolor{currentstroke}%
\pgfsetdash{}{0pt}%
\pgfpathmoveto{\pgfqpoint{0.726903in}{1.243736in}}%
\pgfpathlineto{\pgfqpoint{0.916842in}{1.291711in}}%
\pgfusepath{stroke}%
\end{pgfscope}%
\begin{pgfscope}%
\pgfpathrectangle{\pgfqpoint{0.100000in}{0.212622in}}{\pgfqpoint{3.696000in}{3.696000in}}%
\pgfusepath{clip}%
\pgfsetrectcap%
\pgfsetroundjoin%
\pgfsetlinewidth{1.505625pt}%
\definecolor{currentstroke}{rgb}{1.000000,0.000000,0.000000}%
\pgfsetstrokecolor{currentstroke}%
\pgfsetdash{}{0pt}%
\pgfpathmoveto{\pgfqpoint{0.726903in}{1.243736in}}%
\pgfpathlineto{\pgfqpoint{0.916842in}{1.291711in}}%
\pgfusepath{stroke}%
\end{pgfscope}%
\begin{pgfscope}%
\pgfpathrectangle{\pgfqpoint{0.100000in}{0.212622in}}{\pgfqpoint{3.696000in}{3.696000in}}%
\pgfusepath{clip}%
\pgfsetrectcap%
\pgfsetroundjoin%
\pgfsetlinewidth{1.505625pt}%
\definecolor{currentstroke}{rgb}{1.000000,0.000000,0.000000}%
\pgfsetstrokecolor{currentstroke}%
\pgfsetdash{}{0pt}%
\pgfpathmoveto{\pgfqpoint{0.726903in}{1.243736in}}%
\pgfpathlineto{\pgfqpoint{0.916842in}{1.291711in}}%
\pgfusepath{stroke}%
\end{pgfscope}%
\begin{pgfscope}%
\pgfpathrectangle{\pgfqpoint{0.100000in}{0.212622in}}{\pgfqpoint{3.696000in}{3.696000in}}%
\pgfusepath{clip}%
\pgfsetrectcap%
\pgfsetroundjoin%
\pgfsetlinewidth{1.505625pt}%
\definecolor{currentstroke}{rgb}{1.000000,0.000000,0.000000}%
\pgfsetstrokecolor{currentstroke}%
\pgfsetdash{}{0pt}%
\pgfpathmoveto{\pgfqpoint{0.726903in}{1.243736in}}%
\pgfpathlineto{\pgfqpoint{0.916842in}{1.291711in}}%
\pgfusepath{stroke}%
\end{pgfscope}%
\begin{pgfscope}%
\pgfpathrectangle{\pgfqpoint{0.100000in}{0.212622in}}{\pgfqpoint{3.696000in}{3.696000in}}%
\pgfusepath{clip}%
\pgfsetrectcap%
\pgfsetroundjoin%
\pgfsetlinewidth{1.505625pt}%
\definecolor{currentstroke}{rgb}{1.000000,0.000000,0.000000}%
\pgfsetstrokecolor{currentstroke}%
\pgfsetdash{}{0pt}%
\pgfpathmoveto{\pgfqpoint{0.726903in}{1.243736in}}%
\pgfpathlineto{\pgfqpoint{0.916842in}{1.291711in}}%
\pgfusepath{stroke}%
\end{pgfscope}%
\begin{pgfscope}%
\pgfpathrectangle{\pgfqpoint{0.100000in}{0.212622in}}{\pgfqpoint{3.696000in}{3.696000in}}%
\pgfusepath{clip}%
\pgfsetrectcap%
\pgfsetroundjoin%
\pgfsetlinewidth{1.505625pt}%
\definecolor{currentstroke}{rgb}{1.000000,0.000000,0.000000}%
\pgfsetstrokecolor{currentstroke}%
\pgfsetdash{}{0pt}%
\pgfpathmoveto{\pgfqpoint{0.726903in}{1.243736in}}%
\pgfpathlineto{\pgfqpoint{0.916842in}{1.291711in}}%
\pgfusepath{stroke}%
\end{pgfscope}%
\begin{pgfscope}%
\pgfpathrectangle{\pgfqpoint{0.100000in}{0.212622in}}{\pgfqpoint{3.696000in}{3.696000in}}%
\pgfusepath{clip}%
\pgfsetrectcap%
\pgfsetroundjoin%
\pgfsetlinewidth{1.505625pt}%
\definecolor{currentstroke}{rgb}{1.000000,0.000000,0.000000}%
\pgfsetstrokecolor{currentstroke}%
\pgfsetdash{}{0pt}%
\pgfpathmoveto{\pgfqpoint{0.726903in}{1.243736in}}%
\pgfpathlineto{\pgfqpoint{0.916842in}{1.291711in}}%
\pgfusepath{stroke}%
\end{pgfscope}%
\begin{pgfscope}%
\pgfpathrectangle{\pgfqpoint{0.100000in}{0.212622in}}{\pgfqpoint{3.696000in}{3.696000in}}%
\pgfusepath{clip}%
\pgfsetrectcap%
\pgfsetroundjoin%
\pgfsetlinewidth{1.505625pt}%
\definecolor{currentstroke}{rgb}{1.000000,0.000000,0.000000}%
\pgfsetstrokecolor{currentstroke}%
\pgfsetdash{}{0pt}%
\pgfpathmoveto{\pgfqpoint{0.726903in}{1.243736in}}%
\pgfpathlineto{\pgfqpoint{0.916842in}{1.291711in}}%
\pgfusepath{stroke}%
\end{pgfscope}%
\begin{pgfscope}%
\pgfpathrectangle{\pgfqpoint{0.100000in}{0.212622in}}{\pgfqpoint{3.696000in}{3.696000in}}%
\pgfusepath{clip}%
\pgfsetrectcap%
\pgfsetroundjoin%
\pgfsetlinewidth{1.505625pt}%
\definecolor{currentstroke}{rgb}{1.000000,0.000000,0.000000}%
\pgfsetstrokecolor{currentstroke}%
\pgfsetdash{}{0pt}%
\pgfpathmoveto{\pgfqpoint{0.726903in}{1.243736in}}%
\pgfpathlineto{\pgfqpoint{0.916842in}{1.291711in}}%
\pgfusepath{stroke}%
\end{pgfscope}%
\begin{pgfscope}%
\pgfpathrectangle{\pgfqpoint{0.100000in}{0.212622in}}{\pgfqpoint{3.696000in}{3.696000in}}%
\pgfusepath{clip}%
\pgfsetrectcap%
\pgfsetroundjoin%
\pgfsetlinewidth{1.505625pt}%
\definecolor{currentstroke}{rgb}{1.000000,0.000000,0.000000}%
\pgfsetstrokecolor{currentstroke}%
\pgfsetdash{}{0pt}%
\pgfpathmoveto{\pgfqpoint{0.726903in}{1.243736in}}%
\pgfpathlineto{\pgfqpoint{0.916842in}{1.291711in}}%
\pgfusepath{stroke}%
\end{pgfscope}%
\begin{pgfscope}%
\pgfpathrectangle{\pgfqpoint{0.100000in}{0.212622in}}{\pgfqpoint{3.696000in}{3.696000in}}%
\pgfusepath{clip}%
\pgfsetrectcap%
\pgfsetroundjoin%
\pgfsetlinewidth{1.505625pt}%
\definecolor{currentstroke}{rgb}{1.000000,0.000000,0.000000}%
\pgfsetstrokecolor{currentstroke}%
\pgfsetdash{}{0pt}%
\pgfpathmoveto{\pgfqpoint{0.726903in}{1.243736in}}%
\pgfpathlineto{\pgfqpoint{0.916842in}{1.291711in}}%
\pgfusepath{stroke}%
\end{pgfscope}%
\begin{pgfscope}%
\pgfpathrectangle{\pgfqpoint{0.100000in}{0.212622in}}{\pgfqpoint{3.696000in}{3.696000in}}%
\pgfusepath{clip}%
\pgfsetrectcap%
\pgfsetroundjoin%
\pgfsetlinewidth{1.505625pt}%
\definecolor{currentstroke}{rgb}{1.000000,0.000000,0.000000}%
\pgfsetstrokecolor{currentstroke}%
\pgfsetdash{}{0pt}%
\pgfpathmoveto{\pgfqpoint{0.726903in}{1.243736in}}%
\pgfpathlineto{\pgfqpoint{0.916842in}{1.291711in}}%
\pgfusepath{stroke}%
\end{pgfscope}%
\begin{pgfscope}%
\pgfpathrectangle{\pgfqpoint{0.100000in}{0.212622in}}{\pgfqpoint{3.696000in}{3.696000in}}%
\pgfusepath{clip}%
\pgfsetrectcap%
\pgfsetroundjoin%
\pgfsetlinewidth{1.505625pt}%
\definecolor{currentstroke}{rgb}{1.000000,0.000000,0.000000}%
\pgfsetstrokecolor{currentstroke}%
\pgfsetdash{}{0pt}%
\pgfpathmoveto{\pgfqpoint{0.726903in}{1.243736in}}%
\pgfpathlineto{\pgfqpoint{0.916842in}{1.291711in}}%
\pgfusepath{stroke}%
\end{pgfscope}%
\begin{pgfscope}%
\pgfpathrectangle{\pgfqpoint{0.100000in}{0.212622in}}{\pgfqpoint{3.696000in}{3.696000in}}%
\pgfusepath{clip}%
\pgfsetrectcap%
\pgfsetroundjoin%
\pgfsetlinewidth{1.505625pt}%
\definecolor{currentstroke}{rgb}{1.000000,0.000000,0.000000}%
\pgfsetstrokecolor{currentstroke}%
\pgfsetdash{}{0pt}%
\pgfpathmoveto{\pgfqpoint{0.726903in}{1.243736in}}%
\pgfpathlineto{\pgfqpoint{0.916842in}{1.291711in}}%
\pgfusepath{stroke}%
\end{pgfscope}%
\begin{pgfscope}%
\pgfpathrectangle{\pgfqpoint{0.100000in}{0.212622in}}{\pgfqpoint{3.696000in}{3.696000in}}%
\pgfusepath{clip}%
\pgfsetrectcap%
\pgfsetroundjoin%
\pgfsetlinewidth{1.505625pt}%
\definecolor{currentstroke}{rgb}{1.000000,0.000000,0.000000}%
\pgfsetstrokecolor{currentstroke}%
\pgfsetdash{}{0pt}%
\pgfpathmoveto{\pgfqpoint{0.724743in}{1.243364in}}%
\pgfpathlineto{\pgfqpoint{0.916842in}{1.291711in}}%
\pgfusepath{stroke}%
\end{pgfscope}%
\begin{pgfscope}%
\pgfpathrectangle{\pgfqpoint{0.100000in}{0.212622in}}{\pgfqpoint{3.696000in}{3.696000in}}%
\pgfusepath{clip}%
\pgfsetrectcap%
\pgfsetroundjoin%
\pgfsetlinewidth{1.505625pt}%
\definecolor{currentstroke}{rgb}{1.000000,0.000000,0.000000}%
\pgfsetstrokecolor{currentstroke}%
\pgfsetdash{}{0pt}%
\pgfpathmoveto{\pgfqpoint{0.723456in}{1.243873in}}%
\pgfpathlineto{\pgfqpoint{0.916842in}{1.291711in}}%
\pgfusepath{stroke}%
\end{pgfscope}%
\begin{pgfscope}%
\pgfpathrectangle{\pgfqpoint{0.100000in}{0.212622in}}{\pgfqpoint{3.696000in}{3.696000in}}%
\pgfusepath{clip}%
\pgfsetrectcap%
\pgfsetroundjoin%
\pgfsetlinewidth{1.505625pt}%
\definecolor{currentstroke}{rgb}{1.000000,0.000000,0.000000}%
\pgfsetstrokecolor{currentstroke}%
\pgfsetdash{}{0pt}%
\pgfpathmoveto{\pgfqpoint{0.722800in}{1.243850in}}%
\pgfpathlineto{\pgfqpoint{0.916842in}{1.291711in}}%
\pgfusepath{stroke}%
\end{pgfscope}%
\begin{pgfscope}%
\pgfpathrectangle{\pgfqpoint{0.100000in}{0.212622in}}{\pgfqpoint{3.696000in}{3.696000in}}%
\pgfusepath{clip}%
\pgfsetrectcap%
\pgfsetroundjoin%
\pgfsetlinewidth{1.505625pt}%
\definecolor{currentstroke}{rgb}{1.000000,0.000000,0.000000}%
\pgfsetstrokecolor{currentstroke}%
\pgfsetdash{}{0pt}%
\pgfpathmoveto{\pgfqpoint{0.722412in}{1.244032in}}%
\pgfpathlineto{\pgfqpoint{0.916842in}{1.291711in}}%
\pgfusepath{stroke}%
\end{pgfscope}%
\begin{pgfscope}%
\pgfpathrectangle{\pgfqpoint{0.100000in}{0.212622in}}{\pgfqpoint{3.696000in}{3.696000in}}%
\pgfusepath{clip}%
\pgfsetrectcap%
\pgfsetroundjoin%
\pgfsetlinewidth{1.505625pt}%
\definecolor{currentstroke}{rgb}{1.000000,0.000000,0.000000}%
\pgfsetstrokecolor{currentstroke}%
\pgfsetdash{}{0pt}%
\pgfpathmoveto{\pgfqpoint{0.722221in}{1.244031in}}%
\pgfpathlineto{\pgfqpoint{0.916842in}{1.291711in}}%
\pgfusepath{stroke}%
\end{pgfscope}%
\begin{pgfscope}%
\pgfpathrectangle{\pgfqpoint{0.100000in}{0.212622in}}{\pgfqpoint{3.696000in}{3.696000in}}%
\pgfusepath{clip}%
\pgfsetrectcap%
\pgfsetroundjoin%
\pgfsetlinewidth{1.505625pt}%
\definecolor{currentstroke}{rgb}{1.000000,0.000000,0.000000}%
\pgfsetstrokecolor{currentstroke}%
\pgfsetdash{}{0pt}%
\pgfpathmoveto{\pgfqpoint{0.722105in}{1.244102in}}%
\pgfpathlineto{\pgfqpoint{0.916842in}{1.291711in}}%
\pgfusepath{stroke}%
\end{pgfscope}%
\begin{pgfscope}%
\pgfpathrectangle{\pgfqpoint{0.100000in}{0.212622in}}{\pgfqpoint{3.696000in}{3.696000in}}%
\pgfusepath{clip}%
\pgfsetrectcap%
\pgfsetroundjoin%
\pgfsetlinewidth{1.505625pt}%
\definecolor{currentstroke}{rgb}{1.000000,0.000000,0.000000}%
\pgfsetstrokecolor{currentstroke}%
\pgfsetdash{}{0pt}%
\pgfpathmoveto{\pgfqpoint{0.720189in}{1.244071in}}%
\pgfpathlineto{\pgfqpoint{0.916842in}{1.291711in}}%
\pgfusepath{stroke}%
\end{pgfscope}%
\begin{pgfscope}%
\pgfpathrectangle{\pgfqpoint{0.100000in}{0.212622in}}{\pgfqpoint{3.696000in}{3.696000in}}%
\pgfusepath{clip}%
\pgfsetrectcap%
\pgfsetroundjoin%
\pgfsetlinewidth{1.505625pt}%
\definecolor{currentstroke}{rgb}{1.000000,0.000000,0.000000}%
\pgfsetstrokecolor{currentstroke}%
\pgfsetdash{}{0pt}%
\pgfpathmoveto{\pgfqpoint{0.718872in}{1.244675in}}%
\pgfpathlineto{\pgfqpoint{0.916842in}{1.291711in}}%
\pgfusepath{stroke}%
\end{pgfscope}%
\begin{pgfscope}%
\pgfpathrectangle{\pgfqpoint{0.100000in}{0.212622in}}{\pgfqpoint{3.696000in}{3.696000in}}%
\pgfusepath{clip}%
\pgfsetrectcap%
\pgfsetroundjoin%
\pgfsetlinewidth{1.505625pt}%
\definecolor{currentstroke}{rgb}{1.000000,0.000000,0.000000}%
\pgfsetstrokecolor{currentstroke}%
\pgfsetdash{}{0pt}%
\pgfpathmoveto{\pgfqpoint{0.718219in}{1.244668in}}%
\pgfpathlineto{\pgfqpoint{0.916842in}{1.291711in}}%
\pgfusepath{stroke}%
\end{pgfscope}%
\begin{pgfscope}%
\pgfpathrectangle{\pgfqpoint{0.100000in}{0.212622in}}{\pgfqpoint{3.696000in}{3.696000in}}%
\pgfusepath{clip}%
\pgfsetrectcap%
\pgfsetroundjoin%
\pgfsetlinewidth{1.505625pt}%
\definecolor{currentstroke}{rgb}{1.000000,0.000000,0.000000}%
\pgfsetstrokecolor{currentstroke}%
\pgfsetdash{}{0pt}%
\pgfpathmoveto{\pgfqpoint{0.717963in}{1.244850in}}%
\pgfpathlineto{\pgfqpoint{0.916842in}{1.291711in}}%
\pgfusepath{stroke}%
\end{pgfscope}%
\begin{pgfscope}%
\pgfpathrectangle{\pgfqpoint{0.100000in}{0.212622in}}{\pgfqpoint{3.696000in}{3.696000in}}%
\pgfusepath{clip}%
\pgfsetrectcap%
\pgfsetroundjoin%
\pgfsetlinewidth{1.505625pt}%
\definecolor{currentstroke}{rgb}{1.000000,0.000000,0.000000}%
\pgfsetstrokecolor{currentstroke}%
\pgfsetdash{}{0pt}%
\pgfpathmoveto{\pgfqpoint{0.717860in}{1.244971in}}%
\pgfpathlineto{\pgfqpoint{0.916842in}{1.291711in}}%
\pgfusepath{stroke}%
\end{pgfscope}%
\begin{pgfscope}%
\pgfpathrectangle{\pgfqpoint{0.100000in}{0.212622in}}{\pgfqpoint{3.696000in}{3.696000in}}%
\pgfusepath{clip}%
\pgfsetrectcap%
\pgfsetroundjoin%
\pgfsetlinewidth{1.505625pt}%
\definecolor{currentstroke}{rgb}{1.000000,0.000000,0.000000}%
\pgfsetstrokecolor{currentstroke}%
\pgfsetdash{}{0pt}%
\pgfpathmoveto{\pgfqpoint{0.717845in}{1.245040in}}%
\pgfpathlineto{\pgfqpoint{0.916842in}{1.291711in}}%
\pgfusepath{stroke}%
\end{pgfscope}%
\begin{pgfscope}%
\pgfpathrectangle{\pgfqpoint{0.100000in}{0.212622in}}{\pgfqpoint{3.696000in}{3.696000in}}%
\pgfusepath{clip}%
\pgfsetrectcap%
\pgfsetroundjoin%
\pgfsetlinewidth{1.505625pt}%
\definecolor{currentstroke}{rgb}{1.000000,0.000000,0.000000}%
\pgfsetstrokecolor{currentstroke}%
\pgfsetdash{}{0pt}%
\pgfpathmoveto{\pgfqpoint{0.717858in}{1.245091in}}%
\pgfpathlineto{\pgfqpoint{0.916842in}{1.291711in}}%
\pgfusepath{stroke}%
\end{pgfscope}%
\begin{pgfscope}%
\pgfpathrectangle{\pgfqpoint{0.100000in}{0.212622in}}{\pgfqpoint{3.696000in}{3.696000in}}%
\pgfusepath{clip}%
\pgfsetrectcap%
\pgfsetroundjoin%
\pgfsetlinewidth{1.505625pt}%
\definecolor{currentstroke}{rgb}{1.000000,0.000000,0.000000}%
\pgfsetstrokecolor{currentstroke}%
\pgfsetdash{}{0pt}%
\pgfpathmoveto{\pgfqpoint{0.717875in}{1.245116in}}%
\pgfpathlineto{\pgfqpoint{0.916842in}{1.291711in}}%
\pgfusepath{stroke}%
\end{pgfscope}%
\begin{pgfscope}%
\pgfpathrectangle{\pgfqpoint{0.100000in}{0.212622in}}{\pgfqpoint{3.696000in}{3.696000in}}%
\pgfusepath{clip}%
\pgfsetrectcap%
\pgfsetroundjoin%
\pgfsetlinewidth{1.505625pt}%
\definecolor{currentstroke}{rgb}{1.000000,0.000000,0.000000}%
\pgfsetstrokecolor{currentstroke}%
\pgfsetdash{}{0pt}%
\pgfpathmoveto{\pgfqpoint{0.717884in}{1.245132in}}%
\pgfpathlineto{\pgfqpoint{0.916842in}{1.291711in}}%
\pgfusepath{stroke}%
\end{pgfscope}%
\begin{pgfscope}%
\pgfpathrectangle{\pgfqpoint{0.100000in}{0.212622in}}{\pgfqpoint{3.696000in}{3.696000in}}%
\pgfusepath{clip}%
\pgfsetrectcap%
\pgfsetroundjoin%
\pgfsetlinewidth{1.505625pt}%
\definecolor{currentstroke}{rgb}{1.000000,0.000000,0.000000}%
\pgfsetstrokecolor{currentstroke}%
\pgfsetdash{}{0pt}%
\pgfpathmoveto{\pgfqpoint{0.719143in}{1.247070in}}%
\pgfpathlineto{\pgfqpoint{0.916842in}{1.291711in}}%
\pgfusepath{stroke}%
\end{pgfscope}%
\begin{pgfscope}%
\pgfpathrectangle{\pgfqpoint{0.100000in}{0.212622in}}{\pgfqpoint{3.696000in}{3.696000in}}%
\pgfusepath{clip}%
\pgfsetrectcap%
\pgfsetroundjoin%
\pgfsetlinewidth{1.505625pt}%
\definecolor{currentstroke}{rgb}{1.000000,0.000000,0.000000}%
\pgfsetstrokecolor{currentstroke}%
\pgfsetdash{}{0pt}%
\pgfpathmoveto{\pgfqpoint{0.719935in}{1.248111in}}%
\pgfpathlineto{\pgfqpoint{0.916842in}{1.291711in}}%
\pgfusepath{stroke}%
\end{pgfscope}%
\begin{pgfscope}%
\pgfpathrectangle{\pgfqpoint{0.100000in}{0.212622in}}{\pgfqpoint{3.696000in}{3.696000in}}%
\pgfusepath{clip}%
\pgfsetrectcap%
\pgfsetroundjoin%
\pgfsetlinewidth{1.505625pt}%
\definecolor{currentstroke}{rgb}{1.000000,0.000000,0.000000}%
\pgfsetstrokecolor{currentstroke}%
\pgfsetdash{}{0pt}%
\pgfpathmoveto{\pgfqpoint{0.720371in}{1.248668in}}%
\pgfpathlineto{\pgfqpoint{0.916842in}{1.291711in}}%
\pgfusepath{stroke}%
\end{pgfscope}%
\begin{pgfscope}%
\pgfpathrectangle{\pgfqpoint{0.100000in}{0.212622in}}{\pgfqpoint{3.696000in}{3.696000in}}%
\pgfusepath{clip}%
\pgfsetrectcap%
\pgfsetroundjoin%
\pgfsetlinewidth{1.505625pt}%
\definecolor{currentstroke}{rgb}{1.000000,0.000000,0.000000}%
\pgfsetstrokecolor{currentstroke}%
\pgfsetdash{}{0pt}%
\pgfpathmoveto{\pgfqpoint{0.720574in}{1.249018in}}%
\pgfpathlineto{\pgfqpoint{0.916842in}{1.291711in}}%
\pgfusepath{stroke}%
\end{pgfscope}%
\begin{pgfscope}%
\pgfpathrectangle{\pgfqpoint{0.100000in}{0.212622in}}{\pgfqpoint{3.696000in}{3.696000in}}%
\pgfusepath{clip}%
\pgfsetrectcap%
\pgfsetroundjoin%
\pgfsetlinewidth{1.505625pt}%
\definecolor{currentstroke}{rgb}{1.000000,0.000000,0.000000}%
\pgfsetstrokecolor{currentstroke}%
\pgfsetdash{}{0pt}%
\pgfpathmoveto{\pgfqpoint{0.720708in}{1.249201in}}%
\pgfpathlineto{\pgfqpoint{0.916842in}{1.291711in}}%
\pgfusepath{stroke}%
\end{pgfscope}%
\begin{pgfscope}%
\pgfpathrectangle{\pgfqpoint{0.100000in}{0.212622in}}{\pgfqpoint{3.696000in}{3.696000in}}%
\pgfusepath{clip}%
\pgfsetrectcap%
\pgfsetroundjoin%
\pgfsetlinewidth{1.505625pt}%
\definecolor{currentstroke}{rgb}{1.000000,0.000000,0.000000}%
\pgfsetstrokecolor{currentstroke}%
\pgfsetdash{}{0pt}%
\pgfpathmoveto{\pgfqpoint{0.720771in}{1.249301in}}%
\pgfpathlineto{\pgfqpoint{0.916842in}{1.291711in}}%
\pgfusepath{stroke}%
\end{pgfscope}%
\begin{pgfscope}%
\pgfpathrectangle{\pgfqpoint{0.100000in}{0.212622in}}{\pgfqpoint{3.696000in}{3.696000in}}%
\pgfusepath{clip}%
\pgfsetrectcap%
\pgfsetroundjoin%
\pgfsetlinewidth{1.505625pt}%
\definecolor{currentstroke}{rgb}{1.000000,0.000000,0.000000}%
\pgfsetstrokecolor{currentstroke}%
\pgfsetdash{}{0pt}%
\pgfpathmoveto{\pgfqpoint{0.720804in}{1.249351in}}%
\pgfpathlineto{\pgfqpoint{0.916842in}{1.291711in}}%
\pgfusepath{stroke}%
\end{pgfscope}%
\begin{pgfscope}%
\pgfpathrectangle{\pgfqpoint{0.100000in}{0.212622in}}{\pgfqpoint{3.696000in}{3.696000in}}%
\pgfusepath{clip}%
\pgfsetrectcap%
\pgfsetroundjoin%
\pgfsetlinewidth{1.505625pt}%
\definecolor{currentstroke}{rgb}{1.000000,0.000000,0.000000}%
\pgfsetstrokecolor{currentstroke}%
\pgfsetdash{}{0pt}%
\pgfpathmoveto{\pgfqpoint{0.720826in}{1.249384in}}%
\pgfpathlineto{\pgfqpoint{0.916842in}{1.291711in}}%
\pgfusepath{stroke}%
\end{pgfscope}%
\begin{pgfscope}%
\pgfpathrectangle{\pgfqpoint{0.100000in}{0.212622in}}{\pgfqpoint{3.696000in}{3.696000in}}%
\pgfusepath{clip}%
\pgfsetrectcap%
\pgfsetroundjoin%
\pgfsetlinewidth{1.505625pt}%
\definecolor{currentstroke}{rgb}{1.000000,0.000000,0.000000}%
\pgfsetstrokecolor{currentstroke}%
\pgfsetdash{}{0pt}%
\pgfpathmoveto{\pgfqpoint{0.720838in}{1.249400in}}%
\pgfpathlineto{\pgfqpoint{0.916842in}{1.291711in}}%
\pgfusepath{stroke}%
\end{pgfscope}%
\begin{pgfscope}%
\pgfpathrectangle{\pgfqpoint{0.100000in}{0.212622in}}{\pgfqpoint{3.696000in}{3.696000in}}%
\pgfusepath{clip}%
\pgfsetrectcap%
\pgfsetroundjoin%
\pgfsetlinewidth{1.505625pt}%
\definecolor{currentstroke}{rgb}{1.000000,0.000000,0.000000}%
\pgfsetstrokecolor{currentstroke}%
\pgfsetdash{}{0pt}%
\pgfpathmoveto{\pgfqpoint{0.720844in}{1.249408in}}%
\pgfpathlineto{\pgfqpoint{0.916842in}{1.291711in}}%
\pgfusepath{stroke}%
\end{pgfscope}%
\begin{pgfscope}%
\pgfpathrectangle{\pgfqpoint{0.100000in}{0.212622in}}{\pgfqpoint{3.696000in}{3.696000in}}%
\pgfusepath{clip}%
\pgfsetrectcap%
\pgfsetroundjoin%
\pgfsetlinewidth{1.505625pt}%
\definecolor{currentstroke}{rgb}{1.000000,0.000000,0.000000}%
\pgfsetstrokecolor{currentstroke}%
\pgfsetdash{}{0pt}%
\pgfpathmoveto{\pgfqpoint{0.720848in}{1.249413in}}%
\pgfpathlineto{\pgfqpoint{0.916842in}{1.291711in}}%
\pgfusepath{stroke}%
\end{pgfscope}%
\begin{pgfscope}%
\pgfpathrectangle{\pgfqpoint{0.100000in}{0.212622in}}{\pgfqpoint{3.696000in}{3.696000in}}%
\pgfusepath{clip}%
\pgfsetrectcap%
\pgfsetroundjoin%
\pgfsetlinewidth{1.505625pt}%
\definecolor{currentstroke}{rgb}{1.000000,0.000000,0.000000}%
\pgfsetstrokecolor{currentstroke}%
\pgfsetdash{}{0pt}%
\pgfpathmoveto{\pgfqpoint{0.720850in}{1.249415in}}%
\pgfpathlineto{\pgfqpoint{0.916842in}{1.291711in}}%
\pgfusepath{stroke}%
\end{pgfscope}%
\begin{pgfscope}%
\pgfpathrectangle{\pgfqpoint{0.100000in}{0.212622in}}{\pgfqpoint{3.696000in}{3.696000in}}%
\pgfusepath{clip}%
\pgfsetrectcap%
\pgfsetroundjoin%
\pgfsetlinewidth{1.505625pt}%
\definecolor{currentstroke}{rgb}{1.000000,0.000000,0.000000}%
\pgfsetstrokecolor{currentstroke}%
\pgfsetdash{}{0pt}%
\pgfpathmoveto{\pgfqpoint{0.720851in}{1.249417in}}%
\pgfpathlineto{\pgfqpoint{0.916842in}{1.291711in}}%
\pgfusepath{stroke}%
\end{pgfscope}%
\begin{pgfscope}%
\pgfpathrectangle{\pgfqpoint{0.100000in}{0.212622in}}{\pgfqpoint{3.696000in}{3.696000in}}%
\pgfusepath{clip}%
\pgfsetrectcap%
\pgfsetroundjoin%
\pgfsetlinewidth{1.505625pt}%
\definecolor{currentstroke}{rgb}{1.000000,0.000000,0.000000}%
\pgfsetstrokecolor{currentstroke}%
\pgfsetdash{}{0pt}%
\pgfpathmoveto{\pgfqpoint{0.720851in}{1.249417in}}%
\pgfpathlineto{\pgfqpoint{0.916842in}{1.291711in}}%
\pgfusepath{stroke}%
\end{pgfscope}%
\begin{pgfscope}%
\pgfpathrectangle{\pgfqpoint{0.100000in}{0.212622in}}{\pgfqpoint{3.696000in}{3.696000in}}%
\pgfusepath{clip}%
\pgfsetrectcap%
\pgfsetroundjoin%
\pgfsetlinewidth{1.505625pt}%
\definecolor{currentstroke}{rgb}{1.000000,0.000000,0.000000}%
\pgfsetstrokecolor{currentstroke}%
\pgfsetdash{}{0pt}%
\pgfpathmoveto{\pgfqpoint{0.722047in}{1.250896in}}%
\pgfpathlineto{\pgfqpoint{0.916842in}{1.291711in}}%
\pgfusepath{stroke}%
\end{pgfscope}%
\begin{pgfscope}%
\pgfpathrectangle{\pgfqpoint{0.100000in}{0.212622in}}{\pgfqpoint{3.696000in}{3.696000in}}%
\pgfusepath{clip}%
\pgfsetrectcap%
\pgfsetroundjoin%
\pgfsetlinewidth{1.505625pt}%
\definecolor{currentstroke}{rgb}{1.000000,0.000000,0.000000}%
\pgfsetstrokecolor{currentstroke}%
\pgfsetdash{}{0pt}%
\pgfpathmoveto{\pgfqpoint{0.722413in}{1.251829in}}%
\pgfpathlineto{\pgfqpoint{0.916842in}{1.291711in}}%
\pgfusepath{stroke}%
\end{pgfscope}%
\begin{pgfscope}%
\pgfpathrectangle{\pgfqpoint{0.100000in}{0.212622in}}{\pgfqpoint{3.696000in}{3.696000in}}%
\pgfusepath{clip}%
\pgfsetrectcap%
\pgfsetroundjoin%
\pgfsetlinewidth{1.505625pt}%
\definecolor{currentstroke}{rgb}{1.000000,0.000000,0.000000}%
\pgfsetstrokecolor{currentstroke}%
\pgfsetdash{}{0pt}%
\pgfpathmoveto{\pgfqpoint{0.722726in}{1.252441in}}%
\pgfpathlineto{\pgfqpoint{0.916842in}{1.291711in}}%
\pgfusepath{stroke}%
\end{pgfscope}%
\begin{pgfscope}%
\pgfpathrectangle{\pgfqpoint{0.100000in}{0.212622in}}{\pgfqpoint{3.696000in}{3.696000in}}%
\pgfusepath{clip}%
\pgfsetrectcap%
\pgfsetroundjoin%
\pgfsetlinewidth{1.505625pt}%
\definecolor{currentstroke}{rgb}{1.000000,0.000000,0.000000}%
\pgfsetstrokecolor{currentstroke}%
\pgfsetdash{}{0pt}%
\pgfpathmoveto{\pgfqpoint{0.722868in}{1.252791in}}%
\pgfpathlineto{\pgfqpoint{0.916842in}{1.291711in}}%
\pgfusepath{stroke}%
\end{pgfscope}%
\begin{pgfscope}%
\pgfpathrectangle{\pgfqpoint{0.100000in}{0.212622in}}{\pgfqpoint{3.696000in}{3.696000in}}%
\pgfusepath{clip}%
\pgfsetrectcap%
\pgfsetroundjoin%
\pgfsetlinewidth{1.505625pt}%
\definecolor{currentstroke}{rgb}{1.000000,0.000000,0.000000}%
\pgfsetstrokecolor{currentstroke}%
\pgfsetdash{}{0pt}%
\pgfpathmoveto{\pgfqpoint{0.722954in}{1.252981in}}%
\pgfpathlineto{\pgfqpoint{0.916842in}{1.291711in}}%
\pgfusepath{stroke}%
\end{pgfscope}%
\begin{pgfscope}%
\pgfpathrectangle{\pgfqpoint{0.100000in}{0.212622in}}{\pgfqpoint{3.696000in}{3.696000in}}%
\pgfusepath{clip}%
\pgfsetrectcap%
\pgfsetroundjoin%
\pgfsetlinewidth{1.505625pt}%
\definecolor{currentstroke}{rgb}{1.000000,0.000000,0.000000}%
\pgfsetstrokecolor{currentstroke}%
\pgfsetdash{}{0pt}%
\pgfpathmoveto{\pgfqpoint{0.723011in}{1.253042in}}%
\pgfpathlineto{\pgfqpoint{0.916842in}{1.291711in}}%
\pgfusepath{stroke}%
\end{pgfscope}%
\begin{pgfscope}%
\pgfpathrectangle{\pgfqpoint{0.100000in}{0.212622in}}{\pgfqpoint{3.696000in}{3.696000in}}%
\pgfusepath{clip}%
\pgfsetrectcap%
\pgfsetroundjoin%
\pgfsetlinewidth{1.505625pt}%
\definecolor{currentstroke}{rgb}{1.000000,0.000000,0.000000}%
\pgfsetstrokecolor{currentstroke}%
\pgfsetdash{}{0pt}%
\pgfpathmoveto{\pgfqpoint{0.723040in}{1.253082in}}%
\pgfpathlineto{\pgfqpoint{0.916842in}{1.291711in}}%
\pgfusepath{stroke}%
\end{pgfscope}%
\begin{pgfscope}%
\pgfpathrectangle{\pgfqpoint{0.100000in}{0.212622in}}{\pgfqpoint{3.696000in}{3.696000in}}%
\pgfusepath{clip}%
\pgfsetrectcap%
\pgfsetroundjoin%
\pgfsetlinewidth{1.505625pt}%
\definecolor{currentstroke}{rgb}{1.000000,0.000000,0.000000}%
\pgfsetstrokecolor{currentstroke}%
\pgfsetdash{}{0pt}%
\pgfpathmoveto{\pgfqpoint{0.723060in}{1.253106in}}%
\pgfpathlineto{\pgfqpoint{0.916842in}{1.291711in}}%
\pgfusepath{stroke}%
\end{pgfscope}%
\begin{pgfscope}%
\pgfpathrectangle{\pgfqpoint{0.100000in}{0.212622in}}{\pgfqpoint{3.696000in}{3.696000in}}%
\pgfusepath{clip}%
\pgfsetrectcap%
\pgfsetroundjoin%
\pgfsetlinewidth{1.505625pt}%
\definecolor{currentstroke}{rgb}{1.000000,0.000000,0.000000}%
\pgfsetstrokecolor{currentstroke}%
\pgfsetdash{}{0pt}%
\pgfpathmoveto{\pgfqpoint{0.723069in}{1.253115in}}%
\pgfpathlineto{\pgfqpoint{0.916842in}{1.291711in}}%
\pgfusepath{stroke}%
\end{pgfscope}%
\begin{pgfscope}%
\pgfpathrectangle{\pgfqpoint{0.100000in}{0.212622in}}{\pgfqpoint{3.696000in}{3.696000in}}%
\pgfusepath{clip}%
\pgfsetrectcap%
\pgfsetroundjoin%
\pgfsetlinewidth{1.505625pt}%
\definecolor{currentstroke}{rgb}{1.000000,0.000000,0.000000}%
\pgfsetstrokecolor{currentstroke}%
\pgfsetdash{}{0pt}%
\pgfpathmoveto{\pgfqpoint{0.723075in}{1.253122in}}%
\pgfpathlineto{\pgfqpoint{0.916842in}{1.291711in}}%
\pgfusepath{stroke}%
\end{pgfscope}%
\begin{pgfscope}%
\pgfpathrectangle{\pgfqpoint{0.100000in}{0.212622in}}{\pgfqpoint{3.696000in}{3.696000in}}%
\pgfusepath{clip}%
\pgfsetrectcap%
\pgfsetroundjoin%
\pgfsetlinewidth{1.505625pt}%
\definecolor{currentstroke}{rgb}{1.000000,0.000000,0.000000}%
\pgfsetstrokecolor{currentstroke}%
\pgfsetdash{}{0pt}%
\pgfpathmoveto{\pgfqpoint{0.723078in}{1.253125in}}%
\pgfpathlineto{\pgfqpoint{0.916842in}{1.291711in}}%
\pgfusepath{stroke}%
\end{pgfscope}%
\begin{pgfscope}%
\pgfpathrectangle{\pgfqpoint{0.100000in}{0.212622in}}{\pgfqpoint{3.696000in}{3.696000in}}%
\pgfusepath{clip}%
\pgfsetrectcap%
\pgfsetroundjoin%
\pgfsetlinewidth{1.505625pt}%
\definecolor{currentstroke}{rgb}{1.000000,0.000000,0.000000}%
\pgfsetstrokecolor{currentstroke}%
\pgfsetdash{}{0pt}%
\pgfpathmoveto{\pgfqpoint{0.723079in}{1.253126in}}%
\pgfpathlineto{\pgfqpoint{0.916842in}{1.291711in}}%
\pgfusepath{stroke}%
\end{pgfscope}%
\begin{pgfscope}%
\pgfpathrectangle{\pgfqpoint{0.100000in}{0.212622in}}{\pgfqpoint{3.696000in}{3.696000in}}%
\pgfusepath{clip}%
\pgfsetrectcap%
\pgfsetroundjoin%
\pgfsetlinewidth{1.505625pt}%
\definecolor{currentstroke}{rgb}{1.000000,0.000000,0.000000}%
\pgfsetstrokecolor{currentstroke}%
\pgfsetdash{}{0pt}%
\pgfpathmoveto{\pgfqpoint{0.723080in}{1.253127in}}%
\pgfpathlineto{\pgfqpoint{0.916842in}{1.291711in}}%
\pgfusepath{stroke}%
\end{pgfscope}%
\begin{pgfscope}%
\pgfpathrectangle{\pgfqpoint{0.100000in}{0.212622in}}{\pgfqpoint{3.696000in}{3.696000in}}%
\pgfusepath{clip}%
\pgfsetrectcap%
\pgfsetroundjoin%
\pgfsetlinewidth{1.505625pt}%
\definecolor{currentstroke}{rgb}{1.000000,0.000000,0.000000}%
\pgfsetstrokecolor{currentstroke}%
\pgfsetdash{}{0pt}%
\pgfpathmoveto{\pgfqpoint{0.723081in}{1.253128in}}%
\pgfpathlineto{\pgfqpoint{0.916842in}{1.291711in}}%
\pgfusepath{stroke}%
\end{pgfscope}%
\begin{pgfscope}%
\pgfpathrectangle{\pgfqpoint{0.100000in}{0.212622in}}{\pgfqpoint{3.696000in}{3.696000in}}%
\pgfusepath{clip}%
\pgfsetrectcap%
\pgfsetroundjoin%
\pgfsetlinewidth{1.505625pt}%
\definecolor{currentstroke}{rgb}{1.000000,0.000000,0.000000}%
\pgfsetstrokecolor{currentstroke}%
\pgfsetdash{}{0pt}%
\pgfpathmoveto{\pgfqpoint{0.724348in}{1.254292in}}%
\pgfpathlineto{\pgfqpoint{0.916842in}{1.291711in}}%
\pgfusepath{stroke}%
\end{pgfscope}%
\begin{pgfscope}%
\pgfpathrectangle{\pgfqpoint{0.100000in}{0.212622in}}{\pgfqpoint{3.696000in}{3.696000in}}%
\pgfusepath{clip}%
\pgfsetrectcap%
\pgfsetroundjoin%
\pgfsetlinewidth{1.505625pt}%
\definecolor{currentstroke}{rgb}{1.000000,0.000000,0.000000}%
\pgfsetstrokecolor{currentstroke}%
\pgfsetdash{}{0pt}%
\pgfpathmoveto{\pgfqpoint{0.725009in}{1.254856in}}%
\pgfpathlineto{\pgfqpoint{0.916842in}{1.291711in}}%
\pgfusepath{stroke}%
\end{pgfscope}%
\begin{pgfscope}%
\pgfpathrectangle{\pgfqpoint{0.100000in}{0.212622in}}{\pgfqpoint{3.696000in}{3.696000in}}%
\pgfusepath{clip}%
\pgfsetrectcap%
\pgfsetroundjoin%
\pgfsetlinewidth{1.505625pt}%
\definecolor{currentstroke}{rgb}{1.000000,0.000000,0.000000}%
\pgfsetstrokecolor{currentstroke}%
\pgfsetdash{}{0pt}%
\pgfpathmoveto{\pgfqpoint{0.725401in}{1.255192in}}%
\pgfpathlineto{\pgfqpoint{0.916842in}{1.291711in}}%
\pgfusepath{stroke}%
\end{pgfscope}%
\begin{pgfscope}%
\pgfpathrectangle{\pgfqpoint{0.100000in}{0.212622in}}{\pgfqpoint{3.696000in}{3.696000in}}%
\pgfusepath{clip}%
\pgfsetrectcap%
\pgfsetroundjoin%
\pgfsetlinewidth{1.505625pt}%
\definecolor{currentstroke}{rgb}{1.000000,0.000000,0.000000}%
\pgfsetstrokecolor{currentstroke}%
\pgfsetdash{}{0pt}%
\pgfpathmoveto{\pgfqpoint{0.725587in}{1.255390in}}%
\pgfpathlineto{\pgfqpoint{0.916842in}{1.291711in}}%
\pgfusepath{stroke}%
\end{pgfscope}%
\begin{pgfscope}%
\pgfpathrectangle{\pgfqpoint{0.100000in}{0.212622in}}{\pgfqpoint{3.696000in}{3.696000in}}%
\pgfusepath{clip}%
\pgfsetrectcap%
\pgfsetroundjoin%
\pgfsetlinewidth{1.505625pt}%
\definecolor{currentstroke}{rgb}{1.000000,0.000000,0.000000}%
\pgfsetstrokecolor{currentstroke}%
\pgfsetdash{}{0pt}%
\pgfpathmoveto{\pgfqpoint{0.725697in}{1.255497in}}%
\pgfpathlineto{\pgfqpoint{0.916842in}{1.291711in}}%
\pgfusepath{stroke}%
\end{pgfscope}%
\begin{pgfscope}%
\pgfpathrectangle{\pgfqpoint{0.100000in}{0.212622in}}{\pgfqpoint{3.696000in}{3.696000in}}%
\pgfusepath{clip}%
\pgfsetrectcap%
\pgfsetroundjoin%
\pgfsetlinewidth{1.505625pt}%
\definecolor{currentstroke}{rgb}{1.000000,0.000000,0.000000}%
\pgfsetstrokecolor{currentstroke}%
\pgfsetdash{}{0pt}%
\pgfpathmoveto{\pgfqpoint{0.725753in}{1.255558in}}%
\pgfpathlineto{\pgfqpoint{0.916842in}{1.291711in}}%
\pgfusepath{stroke}%
\end{pgfscope}%
\begin{pgfscope}%
\pgfpathrectangle{\pgfqpoint{0.100000in}{0.212622in}}{\pgfqpoint{3.696000in}{3.696000in}}%
\pgfusepath{clip}%
\pgfsetrectcap%
\pgfsetroundjoin%
\pgfsetlinewidth{1.505625pt}%
\definecolor{currentstroke}{rgb}{1.000000,0.000000,0.000000}%
\pgfsetstrokecolor{currentstroke}%
\pgfsetdash{}{0pt}%
\pgfpathmoveto{\pgfqpoint{0.725783in}{1.255597in}}%
\pgfpathlineto{\pgfqpoint{0.916842in}{1.291711in}}%
\pgfusepath{stroke}%
\end{pgfscope}%
\begin{pgfscope}%
\pgfpathrectangle{\pgfqpoint{0.100000in}{0.212622in}}{\pgfqpoint{3.696000in}{3.696000in}}%
\pgfusepath{clip}%
\pgfsetrectcap%
\pgfsetroundjoin%
\pgfsetlinewidth{1.505625pt}%
\definecolor{currentstroke}{rgb}{1.000000,0.000000,0.000000}%
\pgfsetstrokecolor{currentstroke}%
\pgfsetdash{}{0pt}%
\pgfpathmoveto{\pgfqpoint{0.725803in}{1.255613in}}%
\pgfpathlineto{\pgfqpoint{0.916842in}{1.291711in}}%
\pgfusepath{stroke}%
\end{pgfscope}%
\begin{pgfscope}%
\pgfpathrectangle{\pgfqpoint{0.100000in}{0.212622in}}{\pgfqpoint{3.696000in}{3.696000in}}%
\pgfusepath{clip}%
\pgfsetrectcap%
\pgfsetroundjoin%
\pgfsetlinewidth{1.505625pt}%
\definecolor{currentstroke}{rgb}{1.000000,0.000000,0.000000}%
\pgfsetstrokecolor{currentstroke}%
\pgfsetdash{}{0pt}%
\pgfpathmoveto{\pgfqpoint{0.725810in}{1.255623in}}%
\pgfpathlineto{\pgfqpoint{0.916842in}{1.291711in}}%
\pgfusepath{stroke}%
\end{pgfscope}%
\begin{pgfscope}%
\pgfpathrectangle{\pgfqpoint{0.100000in}{0.212622in}}{\pgfqpoint{3.696000in}{3.696000in}}%
\pgfusepath{clip}%
\pgfsetrectcap%
\pgfsetroundjoin%
\pgfsetlinewidth{1.505625pt}%
\definecolor{currentstroke}{rgb}{1.000000,0.000000,0.000000}%
\pgfsetstrokecolor{currentstroke}%
\pgfsetdash{}{0pt}%
\pgfpathmoveto{\pgfqpoint{0.731305in}{1.259461in}}%
\pgfpathlineto{\pgfqpoint{0.916842in}{1.291711in}}%
\pgfusepath{stroke}%
\end{pgfscope}%
\begin{pgfscope}%
\pgfpathrectangle{\pgfqpoint{0.100000in}{0.212622in}}{\pgfqpoint{3.696000in}{3.696000in}}%
\pgfusepath{clip}%
\pgfsetrectcap%
\pgfsetroundjoin%
\pgfsetlinewidth{1.505625pt}%
\definecolor{currentstroke}{rgb}{1.000000,0.000000,0.000000}%
\pgfsetstrokecolor{currentstroke}%
\pgfsetdash{}{0pt}%
\pgfpathmoveto{\pgfqpoint{0.733659in}{1.261909in}}%
\pgfpathlineto{\pgfqpoint{0.916842in}{1.291711in}}%
\pgfusepath{stroke}%
\end{pgfscope}%
\begin{pgfscope}%
\pgfpathrectangle{\pgfqpoint{0.100000in}{0.212622in}}{\pgfqpoint{3.696000in}{3.696000in}}%
\pgfusepath{clip}%
\pgfsetrectcap%
\pgfsetroundjoin%
\pgfsetlinewidth{1.505625pt}%
\definecolor{currentstroke}{rgb}{1.000000,0.000000,0.000000}%
\pgfsetstrokecolor{currentstroke}%
\pgfsetdash{}{0pt}%
\pgfpathmoveto{\pgfqpoint{0.737769in}{1.267531in}}%
\pgfpathlineto{\pgfqpoint{0.916842in}{1.291711in}}%
\pgfusepath{stroke}%
\end{pgfscope}%
\begin{pgfscope}%
\pgfpathrectangle{\pgfqpoint{0.100000in}{0.212622in}}{\pgfqpoint{3.696000in}{3.696000in}}%
\pgfusepath{clip}%
\pgfsetrectcap%
\pgfsetroundjoin%
\pgfsetlinewidth{1.505625pt}%
\definecolor{currentstroke}{rgb}{1.000000,0.000000,0.000000}%
\pgfsetstrokecolor{currentstroke}%
\pgfsetdash{}{0pt}%
\pgfpathmoveto{\pgfqpoint{0.751127in}{1.275524in}}%
\pgfpathlineto{\pgfqpoint{0.916842in}{1.291711in}}%
\pgfusepath{stroke}%
\end{pgfscope}%
\begin{pgfscope}%
\pgfpathrectangle{\pgfqpoint{0.100000in}{0.212622in}}{\pgfqpoint{3.696000in}{3.696000in}}%
\pgfusepath{clip}%
\pgfsetrectcap%
\pgfsetroundjoin%
\pgfsetlinewidth{1.505625pt}%
\definecolor{currentstroke}{rgb}{1.000000,0.000000,0.000000}%
\pgfsetstrokecolor{currentstroke}%
\pgfsetdash{}{0pt}%
\pgfpathmoveto{\pgfqpoint{0.762702in}{1.287923in}}%
\pgfpathlineto{\pgfqpoint{0.916842in}{1.291711in}}%
\pgfusepath{stroke}%
\end{pgfscope}%
\begin{pgfscope}%
\pgfpathrectangle{\pgfqpoint{0.100000in}{0.212622in}}{\pgfqpoint{3.696000in}{3.696000in}}%
\pgfusepath{clip}%
\pgfsetrectcap%
\pgfsetroundjoin%
\pgfsetlinewidth{1.505625pt}%
\definecolor{currentstroke}{rgb}{1.000000,0.000000,0.000000}%
\pgfsetstrokecolor{currentstroke}%
\pgfsetdash{}{0pt}%
\pgfpathmoveto{\pgfqpoint{0.786323in}{1.300686in}}%
\pgfpathlineto{\pgfqpoint{0.916842in}{1.291711in}}%
\pgfusepath{stroke}%
\end{pgfscope}%
\begin{pgfscope}%
\pgfpathrectangle{\pgfqpoint{0.100000in}{0.212622in}}{\pgfqpoint{3.696000in}{3.696000in}}%
\pgfusepath{clip}%
\pgfsetrectcap%
\pgfsetroundjoin%
\pgfsetlinewidth{1.505625pt}%
\definecolor{currentstroke}{rgb}{1.000000,0.000000,0.000000}%
\pgfsetstrokecolor{currentstroke}%
\pgfsetdash{}{0pt}%
\pgfpathmoveto{\pgfqpoint{0.793122in}{1.311976in}}%
\pgfpathlineto{\pgfqpoint{0.916842in}{1.291711in}}%
\pgfusepath{stroke}%
\end{pgfscope}%
\begin{pgfscope}%
\pgfpathrectangle{\pgfqpoint{0.100000in}{0.212622in}}{\pgfqpoint{3.696000in}{3.696000in}}%
\pgfusepath{clip}%
\pgfsetrectcap%
\pgfsetroundjoin%
\pgfsetlinewidth{1.505625pt}%
\definecolor{currentstroke}{rgb}{1.000000,0.000000,0.000000}%
\pgfsetstrokecolor{currentstroke}%
\pgfsetdash{}{0pt}%
\pgfpathmoveto{\pgfqpoint{0.799605in}{1.318223in}}%
\pgfpathlineto{\pgfqpoint{0.916842in}{1.291711in}}%
\pgfusepath{stroke}%
\end{pgfscope}%
\begin{pgfscope}%
\pgfpathrectangle{\pgfqpoint{0.100000in}{0.212622in}}{\pgfqpoint{3.696000in}{3.696000in}}%
\pgfusepath{clip}%
\pgfsetrectcap%
\pgfsetroundjoin%
\pgfsetlinewidth{1.505625pt}%
\definecolor{currentstroke}{rgb}{1.000000,0.000000,0.000000}%
\pgfsetstrokecolor{currentstroke}%
\pgfsetdash{}{0pt}%
\pgfpathmoveto{\pgfqpoint{0.802360in}{1.321315in}}%
\pgfpathlineto{\pgfqpoint{0.916842in}{1.291711in}}%
\pgfusepath{stroke}%
\end{pgfscope}%
\begin{pgfscope}%
\pgfpathrectangle{\pgfqpoint{0.100000in}{0.212622in}}{\pgfqpoint{3.696000in}{3.696000in}}%
\pgfusepath{clip}%
\pgfsetrectcap%
\pgfsetroundjoin%
\pgfsetlinewidth{1.505625pt}%
\definecolor{currentstroke}{rgb}{1.000000,0.000000,0.000000}%
\pgfsetstrokecolor{currentstroke}%
\pgfsetdash{}{0pt}%
\pgfpathmoveto{\pgfqpoint{0.807483in}{1.330105in}}%
\pgfpathlineto{\pgfqpoint{0.916842in}{1.291711in}}%
\pgfusepath{stroke}%
\end{pgfscope}%
\begin{pgfscope}%
\pgfpathrectangle{\pgfqpoint{0.100000in}{0.212622in}}{\pgfqpoint{3.696000in}{3.696000in}}%
\pgfusepath{clip}%
\pgfsetrectcap%
\pgfsetroundjoin%
\pgfsetlinewidth{1.505625pt}%
\definecolor{currentstroke}{rgb}{1.000000,0.000000,0.000000}%
\pgfsetstrokecolor{currentstroke}%
\pgfsetdash{}{0pt}%
\pgfpathmoveto{\pgfqpoint{0.824991in}{1.335319in}}%
\pgfpathlineto{\pgfqpoint{0.916842in}{1.291711in}}%
\pgfusepath{stroke}%
\end{pgfscope}%
\begin{pgfscope}%
\pgfpathrectangle{\pgfqpoint{0.100000in}{0.212622in}}{\pgfqpoint{3.696000in}{3.696000in}}%
\pgfusepath{clip}%
\pgfsetrectcap%
\pgfsetroundjoin%
\pgfsetlinewidth{1.505625pt}%
\definecolor{currentstroke}{rgb}{1.000000,0.000000,0.000000}%
\pgfsetstrokecolor{currentstroke}%
\pgfsetdash{}{0pt}%
\pgfpathmoveto{\pgfqpoint{0.837149in}{1.350082in}}%
\pgfpathlineto{\pgfqpoint{0.916842in}{1.291711in}}%
\pgfusepath{stroke}%
\end{pgfscope}%
\begin{pgfscope}%
\pgfpathrectangle{\pgfqpoint{0.100000in}{0.212622in}}{\pgfqpoint{3.696000in}{3.696000in}}%
\pgfusepath{clip}%
\pgfsetrectcap%
\pgfsetroundjoin%
\pgfsetlinewidth{1.505625pt}%
\definecolor{currentstroke}{rgb}{1.000000,0.000000,0.000000}%
\pgfsetstrokecolor{currentstroke}%
\pgfsetdash{}{0pt}%
\pgfpathmoveto{\pgfqpoint{0.865979in}{1.365128in}}%
\pgfpathlineto{\pgfqpoint{0.935729in}{1.307914in}}%
\pgfusepath{stroke}%
\end{pgfscope}%
\begin{pgfscope}%
\pgfpathrectangle{\pgfqpoint{0.100000in}{0.212622in}}{\pgfqpoint{3.696000in}{3.696000in}}%
\pgfusepath{clip}%
\pgfsetrectcap%
\pgfsetroundjoin%
\pgfsetlinewidth{1.505625pt}%
\definecolor{currentstroke}{rgb}{1.000000,0.000000,0.000000}%
\pgfsetstrokecolor{currentstroke}%
\pgfsetdash{}{0pt}%
\pgfpathmoveto{\pgfqpoint{0.886413in}{1.393411in}}%
\pgfpathlineto{\pgfqpoint{0.963966in}{1.332138in}}%
\pgfusepath{stroke}%
\end{pgfscope}%
\begin{pgfscope}%
\pgfpathrectangle{\pgfqpoint{0.100000in}{0.212622in}}{\pgfqpoint{3.696000in}{3.696000in}}%
\pgfusepath{clip}%
\pgfsetrectcap%
\pgfsetroundjoin%
\pgfsetlinewidth{1.505625pt}%
\definecolor{currentstroke}{rgb}{1.000000,0.000000,0.000000}%
\pgfsetstrokecolor{currentstroke}%
\pgfsetdash{}{0pt}%
\pgfpathmoveto{\pgfqpoint{0.904840in}{1.406328in}}%
\pgfpathlineto{\pgfqpoint{0.973354in}{1.340191in}}%
\pgfusepath{stroke}%
\end{pgfscope}%
\begin{pgfscope}%
\pgfpathrectangle{\pgfqpoint{0.100000in}{0.212622in}}{\pgfqpoint{3.696000in}{3.696000in}}%
\pgfusepath{clip}%
\pgfsetrectcap%
\pgfsetroundjoin%
\pgfsetlinewidth{1.505625pt}%
\definecolor{currentstroke}{rgb}{1.000000,0.000000,0.000000}%
\pgfsetstrokecolor{currentstroke}%
\pgfsetdash{}{0pt}%
\pgfpathmoveto{\pgfqpoint{0.910448in}{1.413626in}}%
\pgfpathlineto{\pgfqpoint{0.982730in}{1.348234in}}%
\pgfusepath{stroke}%
\end{pgfscope}%
\begin{pgfscope}%
\pgfpathrectangle{\pgfqpoint{0.100000in}{0.212622in}}{\pgfqpoint{3.696000in}{3.696000in}}%
\pgfusepath{clip}%
\pgfsetrectcap%
\pgfsetroundjoin%
\pgfsetlinewidth{1.505625pt}%
\definecolor{currentstroke}{rgb}{1.000000,0.000000,0.000000}%
\pgfsetstrokecolor{currentstroke}%
\pgfsetdash{}{0pt}%
\pgfpathmoveto{\pgfqpoint{0.914399in}{1.419857in}}%
\pgfpathlineto{\pgfqpoint{0.992093in}{1.356267in}}%
\pgfusepath{stroke}%
\end{pgfscope}%
\begin{pgfscope}%
\pgfpathrectangle{\pgfqpoint{0.100000in}{0.212622in}}{\pgfqpoint{3.696000in}{3.696000in}}%
\pgfusepath{clip}%
\pgfsetrectcap%
\pgfsetroundjoin%
\pgfsetlinewidth{1.505625pt}%
\definecolor{currentstroke}{rgb}{1.000000,0.000000,0.000000}%
\pgfsetstrokecolor{currentstroke}%
\pgfsetdash{}{0pt}%
\pgfpathmoveto{\pgfqpoint{0.929392in}{1.428418in}}%
\pgfpathlineto{\pgfqpoint{1.001444in}{1.364289in}}%
\pgfusepath{stroke}%
\end{pgfscope}%
\begin{pgfscope}%
\pgfpathrectangle{\pgfqpoint{0.100000in}{0.212622in}}{\pgfqpoint{3.696000in}{3.696000in}}%
\pgfusepath{clip}%
\pgfsetrectcap%
\pgfsetroundjoin%
\pgfsetlinewidth{1.505625pt}%
\definecolor{currentstroke}{rgb}{1.000000,0.000000,0.000000}%
\pgfsetstrokecolor{currentstroke}%
\pgfsetdash{}{0pt}%
\pgfpathmoveto{\pgfqpoint{0.937391in}{1.440413in}}%
\pgfpathlineto{\pgfqpoint{1.010783in}{1.372301in}}%
\pgfusepath{stroke}%
\end{pgfscope}%
\begin{pgfscope}%
\pgfpathrectangle{\pgfqpoint{0.100000in}{0.212622in}}{\pgfqpoint{3.696000in}{3.696000in}}%
\pgfusepath{clip}%
\pgfsetrectcap%
\pgfsetroundjoin%
\pgfsetlinewidth{1.505625pt}%
\definecolor{currentstroke}{rgb}{1.000000,0.000000,0.000000}%
\pgfsetstrokecolor{currentstroke}%
\pgfsetdash{}{0pt}%
\pgfpathmoveto{\pgfqpoint{0.957619in}{1.457192in}}%
\pgfpathlineto{\pgfqpoint{1.029424in}{1.388292in}}%
\pgfusepath{stroke}%
\end{pgfscope}%
\begin{pgfscope}%
\pgfpathrectangle{\pgfqpoint{0.100000in}{0.212622in}}{\pgfqpoint{3.696000in}{3.696000in}}%
\pgfusepath{clip}%
\pgfsetrectcap%
\pgfsetroundjoin%
\pgfsetlinewidth{1.505625pt}%
\definecolor{currentstroke}{rgb}{1.000000,0.000000,0.000000}%
\pgfsetstrokecolor{currentstroke}%
\pgfsetdash{}{0pt}%
\pgfpathmoveto{\pgfqpoint{0.965894in}{1.471065in}}%
\pgfpathlineto{\pgfqpoint{1.038727in}{1.396273in}}%
\pgfusepath{stroke}%
\end{pgfscope}%
\begin{pgfscope}%
\pgfpathrectangle{\pgfqpoint{0.100000in}{0.212622in}}{\pgfqpoint{3.696000in}{3.696000in}}%
\pgfusepath{clip}%
\pgfsetrectcap%
\pgfsetroundjoin%
\pgfsetlinewidth{1.505625pt}%
\definecolor{currentstroke}{rgb}{1.000000,0.000000,0.000000}%
\pgfsetstrokecolor{currentstroke}%
\pgfsetdash{}{0pt}%
\pgfpathmoveto{\pgfqpoint{0.971437in}{1.477777in}}%
\pgfpathlineto{\pgfqpoint{1.048017in}{1.404243in}}%
\pgfusepath{stroke}%
\end{pgfscope}%
\begin{pgfscope}%
\pgfpathrectangle{\pgfqpoint{0.100000in}{0.212622in}}{\pgfqpoint{3.696000in}{3.696000in}}%
\pgfusepath{clip}%
\pgfsetrectcap%
\pgfsetroundjoin%
\pgfsetlinewidth{1.505625pt}%
\definecolor{currentstroke}{rgb}{1.000000,0.000000,0.000000}%
\pgfsetstrokecolor{currentstroke}%
\pgfsetdash{}{0pt}%
\pgfpathmoveto{\pgfqpoint{0.974384in}{1.481275in}}%
\pgfpathlineto{\pgfqpoint{1.048017in}{1.404243in}}%
\pgfusepath{stroke}%
\end{pgfscope}%
\begin{pgfscope}%
\pgfpathrectangle{\pgfqpoint{0.100000in}{0.212622in}}{\pgfqpoint{3.696000in}{3.696000in}}%
\pgfusepath{clip}%
\pgfsetrectcap%
\pgfsetroundjoin%
\pgfsetlinewidth{1.505625pt}%
\definecolor{currentstroke}{rgb}{1.000000,0.000000,0.000000}%
\pgfsetstrokecolor{currentstroke}%
\pgfsetdash{}{0pt}%
\pgfpathmoveto{\pgfqpoint{0.979176in}{1.488163in}}%
\pgfpathlineto{\pgfqpoint{1.057295in}{1.412202in}}%
\pgfusepath{stroke}%
\end{pgfscope}%
\begin{pgfscope}%
\pgfpathrectangle{\pgfqpoint{0.100000in}{0.212622in}}{\pgfqpoint{3.696000in}{3.696000in}}%
\pgfusepath{clip}%
\pgfsetrectcap%
\pgfsetroundjoin%
\pgfsetlinewidth{1.505625pt}%
\definecolor{currentstroke}{rgb}{1.000000,0.000000,0.000000}%
\pgfsetstrokecolor{currentstroke}%
\pgfsetdash{}{0pt}%
\pgfpathmoveto{\pgfqpoint{0.994537in}{1.496652in}}%
\pgfpathlineto{\pgfqpoint{1.066561in}{1.420151in}}%
\pgfusepath{stroke}%
\end{pgfscope}%
\begin{pgfscope}%
\pgfpathrectangle{\pgfqpoint{0.100000in}{0.212622in}}{\pgfqpoint{3.696000in}{3.696000in}}%
\pgfusepath{clip}%
\pgfsetrectcap%
\pgfsetroundjoin%
\pgfsetlinewidth{1.505625pt}%
\definecolor{currentstroke}{rgb}{1.000000,0.000000,0.000000}%
\pgfsetstrokecolor{currentstroke}%
\pgfsetdash{}{0pt}%
\pgfpathmoveto{\pgfqpoint{1.006223in}{1.514565in}}%
\pgfpathlineto{\pgfqpoint{1.085058in}{1.436019in}}%
\pgfusepath{stroke}%
\end{pgfscope}%
\begin{pgfscope}%
\pgfpathrectangle{\pgfqpoint{0.100000in}{0.212622in}}{\pgfqpoint{3.696000in}{3.696000in}}%
\pgfusepath{clip}%
\pgfsetrectcap%
\pgfsetroundjoin%
\pgfsetlinewidth{1.505625pt}%
\definecolor{currentstroke}{rgb}{1.000000,0.000000,0.000000}%
\pgfsetstrokecolor{currentstroke}%
\pgfsetdash{}{0pt}%
\pgfpathmoveto{\pgfqpoint{1.031730in}{1.533504in}}%
\pgfpathlineto{\pgfqpoint{1.112712in}{1.459743in}}%
\pgfusepath{stroke}%
\end{pgfscope}%
\begin{pgfscope}%
\pgfpathrectangle{\pgfqpoint{0.100000in}{0.212622in}}{\pgfqpoint{3.696000in}{3.696000in}}%
\pgfusepath{clip}%
\pgfsetrectcap%
\pgfsetroundjoin%
\pgfsetlinewidth{1.505625pt}%
\definecolor{currentstroke}{rgb}{1.000000,0.000000,0.000000}%
\pgfsetstrokecolor{currentstroke}%
\pgfsetdash{}{0pt}%
\pgfpathmoveto{\pgfqpoint{1.050409in}{1.563542in}}%
\pgfpathlineto{\pgfqpoint{1.140260in}{1.483375in}}%
\pgfusepath{stroke}%
\end{pgfscope}%
\begin{pgfscope}%
\pgfpathrectangle{\pgfqpoint{0.100000in}{0.212622in}}{\pgfqpoint{3.696000in}{3.696000in}}%
\pgfusepath{clip}%
\pgfsetrectcap%
\pgfsetroundjoin%
\pgfsetlinewidth{1.505625pt}%
\definecolor{currentstroke}{rgb}{1.000000,0.000000,0.000000}%
\pgfsetstrokecolor{currentstroke}%
\pgfsetdash{}{0pt}%
\pgfpathmoveto{\pgfqpoint{1.072678in}{1.597130in}}%
\pgfpathlineto{\pgfqpoint{1.167700in}{1.506916in}}%
\pgfusepath{stroke}%
\end{pgfscope}%
\begin{pgfscope}%
\pgfpathrectangle{\pgfqpoint{0.100000in}{0.212622in}}{\pgfqpoint{3.696000in}{3.696000in}}%
\pgfusepath{clip}%
\pgfsetrectcap%
\pgfsetroundjoin%
\pgfsetlinewidth{1.505625pt}%
\definecolor{currentstroke}{rgb}{1.000000,0.000000,0.000000}%
\pgfsetstrokecolor{currentstroke}%
\pgfsetdash{}{0pt}%
\pgfpathmoveto{\pgfqpoint{1.099420in}{1.643610in}}%
\pgfpathlineto{\pgfqpoint{1.195035in}{1.530365in}}%
\pgfusepath{stroke}%
\end{pgfscope}%
\begin{pgfscope}%
\pgfpathrectangle{\pgfqpoint{0.100000in}{0.212622in}}{\pgfqpoint{3.696000in}{3.696000in}}%
\pgfusepath{clip}%
\pgfsetrectcap%
\pgfsetroundjoin%
\pgfsetlinewidth{1.505625pt}%
\definecolor{currentstroke}{rgb}{1.000000,0.000000,0.000000}%
\pgfsetstrokecolor{currentstroke}%
\pgfsetdash{}{0pt}%
\pgfpathmoveto{\pgfqpoint{1.117823in}{1.661595in}}%
\pgfpathlineto{\pgfqpoint{1.213199in}{1.545948in}}%
\pgfusepath{stroke}%
\end{pgfscope}%
\begin{pgfscope}%
\pgfpathrectangle{\pgfqpoint{0.100000in}{0.212622in}}{\pgfqpoint{3.696000in}{3.696000in}}%
\pgfusepath{clip}%
\pgfsetrectcap%
\pgfsetroundjoin%
\pgfsetlinewidth{1.505625pt}%
\definecolor{currentstroke}{rgb}{1.000000,0.000000,0.000000}%
\pgfsetstrokecolor{currentstroke}%
\pgfsetdash{}{0pt}%
\pgfpathmoveto{\pgfqpoint{1.125702in}{1.676269in}}%
\pgfpathlineto{\pgfqpoint{1.231317in}{1.561491in}}%
\pgfusepath{stroke}%
\end{pgfscope}%
\begin{pgfscope}%
\pgfpathrectangle{\pgfqpoint{0.100000in}{0.212622in}}{\pgfqpoint{3.696000in}{3.696000in}}%
\pgfusepath{clip}%
\pgfsetrectcap%
\pgfsetroundjoin%
\pgfsetlinewidth{1.505625pt}%
\definecolor{currentstroke}{rgb}{1.000000,0.000000,0.000000}%
\pgfsetstrokecolor{currentstroke}%
\pgfsetdash{}{0pt}%
\pgfpathmoveto{\pgfqpoint{1.130400in}{1.683256in}}%
\pgfpathlineto{\pgfqpoint{1.231317in}{1.561491in}}%
\pgfusepath{stroke}%
\end{pgfscope}%
\begin{pgfscope}%
\pgfpathrectangle{\pgfqpoint{0.100000in}{0.212622in}}{\pgfqpoint{3.696000in}{3.696000in}}%
\pgfusepath{clip}%
\pgfsetrectcap%
\pgfsetroundjoin%
\pgfsetlinewidth{1.505625pt}%
\definecolor{currentstroke}{rgb}{1.000000,0.000000,0.000000}%
\pgfsetstrokecolor{currentstroke}%
\pgfsetdash{}{0pt}%
\pgfpathmoveto{\pgfqpoint{1.133108in}{1.686880in}}%
\pgfpathlineto{\pgfqpoint{1.240359in}{1.569247in}}%
\pgfusepath{stroke}%
\end{pgfscope}%
\begin{pgfscope}%
\pgfpathrectangle{\pgfqpoint{0.100000in}{0.212622in}}{\pgfqpoint{3.696000in}{3.696000in}}%
\pgfusepath{clip}%
\pgfsetrectcap%
\pgfsetroundjoin%
\pgfsetlinewidth{1.505625pt}%
\definecolor{currentstroke}{rgb}{1.000000,0.000000,0.000000}%
\pgfsetstrokecolor{currentstroke}%
\pgfsetdash{}{0pt}%
\pgfpathmoveto{\pgfqpoint{1.134404in}{1.689193in}}%
\pgfpathlineto{\pgfqpoint{1.240359in}{1.569247in}}%
\pgfusepath{stroke}%
\end{pgfscope}%
\begin{pgfscope}%
\pgfpathrectangle{\pgfqpoint{0.100000in}{0.212622in}}{\pgfqpoint{3.696000in}{3.696000in}}%
\pgfusepath{clip}%
\pgfsetrectcap%
\pgfsetroundjoin%
\pgfsetlinewidth{1.505625pt}%
\definecolor{currentstroke}{rgb}{1.000000,0.000000,0.000000}%
\pgfsetstrokecolor{currentstroke}%
\pgfsetdash{}{0pt}%
\pgfpathmoveto{\pgfqpoint{1.138963in}{1.693179in}}%
\pgfpathlineto{\pgfqpoint{1.240359in}{1.569247in}}%
\pgfusepath{stroke}%
\end{pgfscope}%
\begin{pgfscope}%
\pgfpathrectangle{\pgfqpoint{0.100000in}{0.212622in}}{\pgfqpoint{3.696000in}{3.696000in}}%
\pgfusepath{clip}%
\pgfsetrectcap%
\pgfsetroundjoin%
\pgfsetlinewidth{1.505625pt}%
\definecolor{currentstroke}{rgb}{1.000000,0.000000,0.000000}%
\pgfsetstrokecolor{currentstroke}%
\pgfsetdash{}{0pt}%
\pgfpathmoveto{\pgfqpoint{1.146426in}{1.705269in}}%
\pgfpathlineto{\pgfqpoint{1.249388in}{1.576994in}}%
\pgfusepath{stroke}%
\end{pgfscope}%
\begin{pgfscope}%
\pgfpathrectangle{\pgfqpoint{0.100000in}{0.212622in}}{\pgfqpoint{3.696000in}{3.696000in}}%
\pgfusepath{clip}%
\pgfsetrectcap%
\pgfsetroundjoin%
\pgfsetlinewidth{1.505625pt}%
\definecolor{currentstroke}{rgb}{1.000000,0.000000,0.000000}%
\pgfsetstrokecolor{currentstroke}%
\pgfsetdash{}{0pt}%
\pgfpathmoveto{\pgfqpoint{1.159444in}{1.723285in}}%
\pgfpathlineto{\pgfqpoint{1.267414in}{1.592457in}}%
\pgfusepath{stroke}%
\end{pgfscope}%
\begin{pgfscope}%
\pgfpathrectangle{\pgfqpoint{0.100000in}{0.212622in}}{\pgfqpoint{3.696000in}{3.696000in}}%
\pgfusepath{clip}%
\pgfsetrectcap%
\pgfsetroundjoin%
\pgfsetlinewidth{1.505625pt}%
\definecolor{currentstroke}{rgb}{1.000000,0.000000,0.000000}%
\pgfsetstrokecolor{currentstroke}%
\pgfsetdash{}{0pt}%
\pgfpathmoveto{\pgfqpoint{1.170038in}{1.744949in}}%
\pgfpathlineto{\pgfqpoint{1.285392in}{1.607880in}}%
\pgfusepath{stroke}%
\end{pgfscope}%
\begin{pgfscope}%
\pgfpathrectangle{\pgfqpoint{0.100000in}{0.212622in}}{\pgfqpoint{3.696000in}{3.696000in}}%
\pgfusepath{clip}%
\pgfsetrectcap%
\pgfsetroundjoin%
\pgfsetlinewidth{1.505625pt}%
\definecolor{currentstroke}{rgb}{1.000000,0.000000,0.000000}%
\pgfsetstrokecolor{currentstroke}%
\pgfsetdash{}{0pt}%
\pgfpathmoveto{\pgfqpoint{1.194157in}{1.766083in}}%
\pgfpathlineto{\pgfqpoint{1.312275in}{1.630942in}}%
\pgfusepath{stroke}%
\end{pgfscope}%
\begin{pgfscope}%
\pgfpathrectangle{\pgfqpoint{0.100000in}{0.212622in}}{\pgfqpoint{3.696000in}{3.696000in}}%
\pgfusepath{clip}%
\pgfsetrectcap%
\pgfsetroundjoin%
\pgfsetlinewidth{1.505625pt}%
\definecolor{currentstroke}{rgb}{1.000000,0.000000,0.000000}%
\pgfsetstrokecolor{currentstroke}%
\pgfsetdash{}{0pt}%
\pgfpathmoveto{\pgfqpoint{1.205755in}{1.783218in}}%
\pgfpathlineto{\pgfqpoint{1.321212in}{1.638610in}}%
\pgfusepath{stroke}%
\end{pgfscope}%
\begin{pgfscope}%
\pgfpathrectangle{\pgfqpoint{0.100000in}{0.212622in}}{\pgfqpoint{3.696000in}{3.696000in}}%
\pgfusepath{clip}%
\pgfsetrectcap%
\pgfsetroundjoin%
\pgfsetlinewidth{1.505625pt}%
\definecolor{currentstroke}{rgb}{1.000000,0.000000,0.000000}%
\pgfsetstrokecolor{currentstroke}%
\pgfsetdash{}{0pt}%
\pgfpathmoveto{\pgfqpoint{1.211774in}{1.792157in}}%
\pgfpathlineto{\pgfqpoint{1.330139in}{1.646267in}}%
\pgfusepath{stroke}%
\end{pgfscope}%
\begin{pgfscope}%
\pgfpathrectangle{\pgfqpoint{0.100000in}{0.212622in}}{\pgfqpoint{3.696000in}{3.696000in}}%
\pgfusepath{clip}%
\pgfsetrectcap%
\pgfsetroundjoin%
\pgfsetlinewidth{1.505625pt}%
\definecolor{currentstroke}{rgb}{1.000000,0.000000,0.000000}%
\pgfsetstrokecolor{currentstroke}%
\pgfsetdash{}{0pt}%
\pgfpathmoveto{\pgfqpoint{1.214570in}{1.796730in}}%
\pgfpathlineto{\pgfqpoint{1.330139in}{1.646267in}}%
\pgfusepath{stroke}%
\end{pgfscope}%
\begin{pgfscope}%
\pgfpathrectangle{\pgfqpoint{0.100000in}{0.212622in}}{\pgfqpoint{3.696000in}{3.696000in}}%
\pgfusepath{clip}%
\pgfsetrectcap%
\pgfsetroundjoin%
\pgfsetlinewidth{1.505625pt}%
\definecolor{currentstroke}{rgb}{1.000000,0.000000,0.000000}%
\pgfsetstrokecolor{currentstroke}%
\pgfsetdash{}{0pt}%
\pgfpathmoveto{\pgfqpoint{1.216007in}{1.799773in}}%
\pgfpathlineto{\pgfqpoint{1.330139in}{1.646267in}}%
\pgfusepath{stroke}%
\end{pgfscope}%
\begin{pgfscope}%
\pgfpathrectangle{\pgfqpoint{0.100000in}{0.212622in}}{\pgfqpoint{3.696000in}{3.696000in}}%
\pgfusepath{clip}%
\pgfsetrectcap%
\pgfsetroundjoin%
\pgfsetlinewidth{1.505625pt}%
\definecolor{currentstroke}{rgb}{1.000000,0.000000,0.000000}%
\pgfsetstrokecolor{currentstroke}%
\pgfsetdash{}{0pt}%
\pgfpathmoveto{\pgfqpoint{1.217369in}{1.800860in}}%
\pgfpathlineto{\pgfqpoint{1.339054in}{1.653915in}}%
\pgfusepath{stroke}%
\end{pgfscope}%
\begin{pgfscope}%
\pgfpathrectangle{\pgfqpoint{0.100000in}{0.212622in}}{\pgfqpoint{3.696000in}{3.696000in}}%
\pgfusepath{clip}%
\pgfsetrectcap%
\pgfsetroundjoin%
\pgfsetlinewidth{1.505625pt}%
\definecolor{currentstroke}{rgb}{1.000000,0.000000,0.000000}%
\pgfsetstrokecolor{currentstroke}%
\pgfsetdash{}{0pt}%
\pgfpathmoveto{\pgfqpoint{1.221217in}{1.806727in}}%
\pgfpathlineto{\pgfqpoint{1.339054in}{1.653915in}}%
\pgfusepath{stroke}%
\end{pgfscope}%
\begin{pgfscope}%
\pgfpathrectangle{\pgfqpoint{0.100000in}{0.212622in}}{\pgfqpoint{3.696000in}{3.696000in}}%
\pgfusepath{clip}%
\pgfsetrectcap%
\pgfsetroundjoin%
\pgfsetlinewidth{1.505625pt}%
\definecolor{currentstroke}{rgb}{1.000000,0.000000,0.000000}%
\pgfsetstrokecolor{currentstroke}%
\pgfsetdash{}{0pt}%
\pgfpathmoveto{\pgfqpoint{1.231141in}{1.815567in}}%
\pgfpathlineto{\pgfqpoint{1.347958in}{1.661554in}}%
\pgfusepath{stroke}%
\end{pgfscope}%
\begin{pgfscope}%
\pgfpathrectangle{\pgfqpoint{0.100000in}{0.212622in}}{\pgfqpoint{3.696000in}{3.696000in}}%
\pgfusepath{clip}%
\pgfsetrectcap%
\pgfsetroundjoin%
\pgfsetlinewidth{1.505625pt}%
\definecolor{currentstroke}{rgb}{1.000000,0.000000,0.000000}%
\pgfsetstrokecolor{currentstroke}%
\pgfsetdash{}{0pt}%
\pgfpathmoveto{\pgfqpoint{1.239892in}{1.832569in}}%
\pgfpathlineto{\pgfqpoint{1.365731in}{1.676801in}}%
\pgfusepath{stroke}%
\end{pgfscope}%
\begin{pgfscope}%
\pgfpathrectangle{\pgfqpoint{0.100000in}{0.212622in}}{\pgfqpoint{3.696000in}{3.696000in}}%
\pgfusepath{clip}%
\pgfsetrectcap%
\pgfsetroundjoin%
\pgfsetlinewidth{1.505625pt}%
\definecolor{currentstroke}{rgb}{1.000000,0.000000,0.000000}%
\pgfsetstrokecolor{currentstroke}%
\pgfsetdash{}{0pt}%
\pgfpathmoveto{\pgfqpoint{1.259613in}{1.851270in}}%
\pgfpathlineto{\pgfqpoint{1.383460in}{1.692010in}}%
\pgfusepath{stroke}%
\end{pgfscope}%
\begin{pgfscope}%
\pgfpathrectangle{\pgfqpoint{0.100000in}{0.212622in}}{\pgfqpoint{3.696000in}{3.696000in}}%
\pgfusepath{clip}%
\pgfsetrectcap%
\pgfsetroundjoin%
\pgfsetlinewidth{1.505625pt}%
\definecolor{currentstroke}{rgb}{1.000000,0.000000,0.000000}%
\pgfsetstrokecolor{currentstroke}%
\pgfsetdash{}{0pt}%
\pgfpathmoveto{\pgfqpoint{1.283087in}{1.884256in}}%
\pgfpathlineto{\pgfqpoint{1.409968in}{1.714750in}}%
\pgfusepath{stroke}%
\end{pgfscope}%
\begin{pgfscope}%
\pgfpathrectangle{\pgfqpoint{0.100000in}{0.212622in}}{\pgfqpoint{3.696000in}{3.696000in}}%
\pgfusepath{clip}%
\pgfsetrectcap%
\pgfsetroundjoin%
\pgfsetlinewidth{1.505625pt}%
\definecolor{currentstroke}{rgb}{1.000000,0.000000,0.000000}%
\pgfsetstrokecolor{currentstroke}%
\pgfsetdash{}{0pt}%
\pgfpathmoveto{\pgfqpoint{1.312856in}{1.924766in}}%
\pgfpathlineto{\pgfqpoint{1.445155in}{1.744937in}}%
\pgfusepath{stroke}%
\end{pgfscope}%
\begin{pgfscope}%
\pgfpathrectangle{\pgfqpoint{0.100000in}{0.212622in}}{\pgfqpoint{3.696000in}{3.696000in}}%
\pgfusepath{clip}%
\pgfsetrectcap%
\pgfsetroundjoin%
\pgfsetlinewidth{1.505625pt}%
\definecolor{currentstroke}{rgb}{1.000000,0.000000,0.000000}%
\pgfsetstrokecolor{currentstroke}%
\pgfsetdash{}{0pt}%
\pgfpathmoveto{\pgfqpoint{1.341704in}{1.972195in}}%
\pgfpathlineto{\pgfqpoint{1.480165in}{1.774971in}}%
\pgfusepath{stroke}%
\end{pgfscope}%
\begin{pgfscope}%
\pgfpathrectangle{\pgfqpoint{0.100000in}{0.212622in}}{\pgfqpoint{3.696000in}{3.696000in}}%
\pgfusepath{clip}%
\pgfsetrectcap%
\pgfsetroundjoin%
\pgfsetlinewidth{1.505625pt}%
\definecolor{currentstroke}{rgb}{1.000000,0.000000,0.000000}%
\pgfsetstrokecolor{currentstroke}%
\pgfsetdash{}{0pt}%
\pgfpathmoveto{\pgfqpoint{1.379277in}{2.006665in}}%
\pgfpathlineto{\pgfqpoint{1.514999in}{1.804854in}}%
\pgfusepath{stroke}%
\end{pgfscope}%
\begin{pgfscope}%
\pgfpathrectangle{\pgfqpoint{0.100000in}{0.212622in}}{\pgfqpoint{3.696000in}{3.696000in}}%
\pgfusepath{clip}%
\pgfsetrectcap%
\pgfsetroundjoin%
\pgfsetlinewidth{1.505625pt}%
\definecolor{currentstroke}{rgb}{1.000000,0.000000,0.000000}%
\pgfsetstrokecolor{currentstroke}%
\pgfsetdash{}{0pt}%
\pgfpathmoveto{\pgfqpoint{1.394484in}{2.029991in}}%
\pgfpathlineto{\pgfqpoint{1.532350in}{1.819739in}}%
\pgfusepath{stroke}%
\end{pgfscope}%
\begin{pgfscope}%
\pgfpathrectangle{\pgfqpoint{0.100000in}{0.212622in}}{\pgfqpoint{3.696000in}{3.696000in}}%
\pgfusepath{clip}%
\pgfsetrectcap%
\pgfsetroundjoin%
\pgfsetlinewidth{1.505625pt}%
\definecolor{currentstroke}{rgb}{1.000000,0.000000,0.000000}%
\pgfsetstrokecolor{currentstroke}%
\pgfsetdash{}{0pt}%
\pgfpathmoveto{\pgfqpoint{1.403713in}{2.043121in}}%
\pgfpathlineto{\pgfqpoint{1.541010in}{1.827167in}}%
\pgfusepath{stroke}%
\end{pgfscope}%
\begin{pgfscope}%
\pgfpathrectangle{\pgfqpoint{0.100000in}{0.212622in}}{\pgfqpoint{3.696000in}{3.696000in}}%
\pgfusepath{clip}%
\pgfsetrectcap%
\pgfsetroundjoin%
\pgfsetlinewidth{1.505625pt}%
\definecolor{currentstroke}{rgb}{1.000000,0.000000,0.000000}%
\pgfsetstrokecolor{currentstroke}%
\pgfsetdash{}{0pt}%
\pgfpathmoveto{\pgfqpoint{1.407306in}{2.050121in}}%
\pgfpathlineto{\pgfqpoint{1.549658in}{1.834587in}}%
\pgfusepath{stroke}%
\end{pgfscope}%
\begin{pgfscope}%
\pgfpathrectangle{\pgfqpoint{0.100000in}{0.212622in}}{\pgfqpoint{3.696000in}{3.696000in}}%
\pgfusepath{clip}%
\pgfsetrectcap%
\pgfsetroundjoin%
\pgfsetlinewidth{1.505625pt}%
\definecolor{currentstroke}{rgb}{1.000000,0.000000,0.000000}%
\pgfsetstrokecolor{currentstroke}%
\pgfsetdash{}{0pt}%
\pgfpathmoveto{\pgfqpoint{1.410240in}{2.054669in}}%
\pgfpathlineto{\pgfqpoint{1.549658in}{1.834587in}}%
\pgfusepath{stroke}%
\end{pgfscope}%
\begin{pgfscope}%
\pgfpathrectangle{\pgfqpoint{0.100000in}{0.212622in}}{\pgfqpoint{3.696000in}{3.696000in}}%
\pgfusepath{clip}%
\pgfsetrectcap%
\pgfsetroundjoin%
\pgfsetlinewidth{1.505625pt}%
\definecolor{currentstroke}{rgb}{1.000000,0.000000,0.000000}%
\pgfsetstrokecolor{currentstroke}%
\pgfsetdash{}{0pt}%
\pgfpathmoveto{\pgfqpoint{1.411679in}{2.056813in}}%
\pgfpathlineto{\pgfqpoint{1.558296in}{1.841997in}}%
\pgfusepath{stroke}%
\end{pgfscope}%
\begin{pgfscope}%
\pgfpathrectangle{\pgfqpoint{0.100000in}{0.212622in}}{\pgfqpoint{3.696000in}{3.696000in}}%
\pgfusepath{clip}%
\pgfsetrectcap%
\pgfsetroundjoin%
\pgfsetlinewidth{1.505625pt}%
\definecolor{currentstroke}{rgb}{1.000000,0.000000,0.000000}%
\pgfsetstrokecolor{currentstroke}%
\pgfsetdash{}{0pt}%
\pgfpathmoveto{\pgfqpoint{1.414851in}{2.062317in}}%
\pgfpathlineto{\pgfqpoint{1.558296in}{1.841997in}}%
\pgfusepath{stroke}%
\end{pgfscope}%
\begin{pgfscope}%
\pgfpathrectangle{\pgfqpoint{0.100000in}{0.212622in}}{\pgfqpoint{3.696000in}{3.696000in}}%
\pgfusepath{clip}%
\pgfsetrectcap%
\pgfsetroundjoin%
\pgfsetlinewidth{1.505625pt}%
\definecolor{currentstroke}{rgb}{1.000000,0.000000,0.000000}%
\pgfsetstrokecolor{currentstroke}%
\pgfsetdash{}{0pt}%
\pgfpathmoveto{\pgfqpoint{1.422380in}{2.071197in}}%
\pgfpathlineto{\pgfqpoint{1.566922in}{1.849397in}}%
\pgfusepath{stroke}%
\end{pgfscope}%
\begin{pgfscope}%
\pgfpathrectangle{\pgfqpoint{0.100000in}{0.212622in}}{\pgfqpoint{3.696000in}{3.696000in}}%
\pgfusepath{clip}%
\pgfsetrectcap%
\pgfsetroundjoin%
\pgfsetlinewidth{1.505625pt}%
\definecolor{currentstroke}{rgb}{1.000000,0.000000,0.000000}%
\pgfsetstrokecolor{currentstroke}%
\pgfsetdash{}{0pt}%
\pgfpathmoveto{\pgfqpoint{1.428736in}{2.084012in}}%
\pgfpathlineto{\pgfqpoint{1.575538in}{1.856788in}}%
\pgfusepath{stroke}%
\end{pgfscope}%
\begin{pgfscope}%
\pgfpathrectangle{\pgfqpoint{0.100000in}{0.212622in}}{\pgfqpoint{3.696000in}{3.696000in}}%
\pgfusepath{clip}%
\pgfsetrectcap%
\pgfsetroundjoin%
\pgfsetlinewidth{1.505625pt}%
\definecolor{currentstroke}{rgb}{1.000000,0.000000,0.000000}%
\pgfsetstrokecolor{currentstroke}%
\pgfsetdash{}{0pt}%
\pgfpathmoveto{\pgfqpoint{1.443062in}{2.095446in}}%
\pgfpathlineto{\pgfqpoint{1.592738in}{1.871543in}}%
\pgfusepath{stroke}%
\end{pgfscope}%
\begin{pgfscope}%
\pgfpathrectangle{\pgfqpoint{0.100000in}{0.212622in}}{\pgfqpoint{3.696000in}{3.696000in}}%
\pgfusepath{clip}%
\pgfsetrectcap%
\pgfsetroundjoin%
\pgfsetlinewidth{1.505625pt}%
\definecolor{currentstroke}{rgb}{1.000000,0.000000,0.000000}%
\pgfsetstrokecolor{currentstroke}%
\pgfsetdash{}{0pt}%
\pgfpathmoveto{\pgfqpoint{1.456090in}{2.116154in}}%
\pgfpathlineto{\pgfqpoint{1.609894in}{1.886261in}}%
\pgfusepath{stroke}%
\end{pgfscope}%
\begin{pgfscope}%
\pgfpathrectangle{\pgfqpoint{0.100000in}{0.212622in}}{\pgfqpoint{3.696000in}{3.696000in}}%
\pgfusepath{clip}%
\pgfsetrectcap%
\pgfsetroundjoin%
\pgfsetlinewidth{1.505625pt}%
\definecolor{currentstroke}{rgb}{1.000000,0.000000,0.000000}%
\pgfsetstrokecolor{currentstroke}%
\pgfsetdash{}{0pt}%
\pgfpathmoveto{\pgfqpoint{1.478360in}{2.137110in}}%
\pgfpathlineto{\pgfqpoint{1.627008in}{1.900943in}}%
\pgfusepath{stroke}%
\end{pgfscope}%
\begin{pgfscope}%
\pgfpathrectangle{\pgfqpoint{0.100000in}{0.212622in}}{\pgfqpoint{3.696000in}{3.696000in}}%
\pgfusepath{clip}%
\pgfsetrectcap%
\pgfsetroundjoin%
\pgfsetlinewidth{1.505625pt}%
\definecolor{currentstroke}{rgb}{1.000000,0.000000,0.000000}%
\pgfsetstrokecolor{currentstroke}%
\pgfsetdash{}{0pt}%
\pgfpathmoveto{\pgfqpoint{1.497048in}{2.166672in}}%
\pgfpathlineto{\pgfqpoint{1.652598in}{1.922896in}}%
\pgfusepath{stroke}%
\end{pgfscope}%
\begin{pgfscope}%
\pgfpathrectangle{\pgfqpoint{0.100000in}{0.212622in}}{\pgfqpoint{3.696000in}{3.696000in}}%
\pgfusepath{clip}%
\pgfsetrectcap%
\pgfsetroundjoin%
\pgfsetlinewidth{1.505625pt}%
\definecolor{currentstroke}{rgb}{1.000000,0.000000,0.000000}%
\pgfsetstrokecolor{currentstroke}%
\pgfsetdash{}{0pt}%
\pgfpathmoveto{\pgfqpoint{1.517958in}{2.200180in}}%
\pgfpathlineto{\pgfqpoint{1.678093in}{1.944767in}}%
\pgfusepath{stroke}%
\end{pgfscope}%
\begin{pgfscope}%
\pgfpathrectangle{\pgfqpoint{0.100000in}{0.212622in}}{\pgfqpoint{3.696000in}{3.696000in}}%
\pgfusepath{clip}%
\pgfsetrectcap%
\pgfsetroundjoin%
\pgfsetlinewidth{1.505625pt}%
\definecolor{currentstroke}{rgb}{1.000000,0.000000,0.000000}%
\pgfsetstrokecolor{currentstroke}%
\pgfsetdash{}{0pt}%
\pgfpathmoveto{\pgfqpoint{1.540360in}{2.239412in}}%
\pgfpathlineto{\pgfqpoint{1.711938in}{1.973802in}}%
\pgfusepath{stroke}%
\end{pgfscope}%
\begin{pgfscope}%
\pgfpathrectangle{\pgfqpoint{0.100000in}{0.212622in}}{\pgfqpoint{3.696000in}{3.696000in}}%
\pgfusepath{clip}%
\pgfsetrectcap%
\pgfsetroundjoin%
\pgfsetlinewidth{1.505625pt}%
\definecolor{currentstroke}{rgb}{1.000000,0.000000,0.000000}%
\pgfsetstrokecolor{currentstroke}%
\pgfsetdash{}{0pt}%
\pgfpathmoveto{\pgfqpoint{1.554180in}{2.257041in}}%
\pgfpathlineto{\pgfqpoint{1.728798in}{1.988266in}}%
\pgfusepath{stroke}%
\end{pgfscope}%
\begin{pgfscope}%
\pgfpathrectangle{\pgfqpoint{0.100000in}{0.212622in}}{\pgfqpoint{3.696000in}{3.696000in}}%
\pgfusepath{clip}%
\pgfsetrectcap%
\pgfsetroundjoin%
\pgfsetlinewidth{1.505625pt}%
\definecolor{currentstroke}{rgb}{1.000000,0.000000,0.000000}%
\pgfsetstrokecolor{currentstroke}%
\pgfsetdash{}{0pt}%
\pgfpathmoveto{\pgfqpoint{1.561122in}{2.267913in}}%
\pgfpathlineto{\pgfqpoint{1.737213in}{1.995484in}}%
\pgfusepath{stroke}%
\end{pgfscope}%
\begin{pgfscope}%
\pgfpathrectangle{\pgfqpoint{0.100000in}{0.212622in}}{\pgfqpoint{3.696000in}{3.696000in}}%
\pgfusepath{clip}%
\pgfsetrectcap%
\pgfsetroundjoin%
\pgfsetlinewidth{1.505625pt}%
\definecolor{currentstroke}{rgb}{1.000000,0.000000,0.000000}%
\pgfsetstrokecolor{currentstroke}%
\pgfsetdash{}{0pt}%
\pgfpathmoveto{\pgfqpoint{1.565025in}{2.273575in}}%
\pgfpathlineto{\pgfqpoint{1.737213in}{1.995484in}}%
\pgfusepath{stroke}%
\end{pgfscope}%
\begin{pgfscope}%
\pgfpathrectangle{\pgfqpoint{0.100000in}{0.212622in}}{\pgfqpoint{3.696000in}{3.696000in}}%
\pgfusepath{clip}%
\pgfsetrectcap%
\pgfsetroundjoin%
\pgfsetlinewidth{1.505625pt}%
\definecolor{currentstroke}{rgb}{1.000000,0.000000,0.000000}%
\pgfsetstrokecolor{currentstroke}%
\pgfsetdash{}{0pt}%
\pgfpathmoveto{\pgfqpoint{1.566917in}{2.276560in}}%
\pgfpathlineto{\pgfqpoint{1.745616in}{2.002694in}}%
\pgfusepath{stroke}%
\end{pgfscope}%
\begin{pgfscope}%
\pgfpathrectangle{\pgfqpoint{0.100000in}{0.212622in}}{\pgfqpoint{3.696000in}{3.696000in}}%
\pgfusepath{clip}%
\pgfsetrectcap%
\pgfsetroundjoin%
\pgfsetlinewidth{1.505625pt}%
\definecolor{currentstroke}{rgb}{1.000000,0.000000,0.000000}%
\pgfsetstrokecolor{currentstroke}%
\pgfsetdash{}{0pt}%
\pgfpathmoveto{\pgfqpoint{1.568468in}{2.277807in}}%
\pgfpathlineto{\pgfqpoint{1.745616in}{2.002694in}}%
\pgfusepath{stroke}%
\end{pgfscope}%
\begin{pgfscope}%
\pgfpathrectangle{\pgfqpoint{0.100000in}{0.212622in}}{\pgfqpoint{3.696000in}{3.696000in}}%
\pgfusepath{clip}%
\pgfsetrectcap%
\pgfsetroundjoin%
\pgfsetlinewidth{1.505625pt}%
\definecolor{currentstroke}{rgb}{1.000000,0.000000,0.000000}%
\pgfsetstrokecolor{currentstroke}%
\pgfsetdash{}{0pt}%
\pgfpathmoveto{\pgfqpoint{1.569069in}{2.278887in}}%
\pgfpathlineto{\pgfqpoint{1.745616in}{2.002694in}}%
\pgfusepath{stroke}%
\end{pgfscope}%
\begin{pgfscope}%
\pgfpathrectangle{\pgfqpoint{0.100000in}{0.212622in}}{\pgfqpoint{3.696000in}{3.696000in}}%
\pgfusepath{clip}%
\pgfsetrectcap%
\pgfsetroundjoin%
\pgfsetlinewidth{1.505625pt}%
\definecolor{currentstroke}{rgb}{1.000000,0.000000,0.000000}%
\pgfsetstrokecolor{currentstroke}%
\pgfsetdash{}{0pt}%
\pgfpathmoveto{\pgfqpoint{1.574186in}{2.283079in}}%
\pgfpathlineto{\pgfqpoint{1.754010in}{2.009894in}}%
\pgfusepath{stroke}%
\end{pgfscope}%
\begin{pgfscope}%
\pgfpathrectangle{\pgfqpoint{0.100000in}{0.212622in}}{\pgfqpoint{3.696000in}{3.696000in}}%
\pgfusepath{clip}%
\pgfsetrectcap%
\pgfsetroundjoin%
\pgfsetlinewidth{1.505625pt}%
\definecolor{currentstroke}{rgb}{1.000000,0.000000,0.000000}%
\pgfsetstrokecolor{currentstroke}%
\pgfsetdash{}{0pt}%
\pgfpathmoveto{\pgfqpoint{1.582011in}{2.293152in}}%
\pgfpathlineto{\pgfqpoint{1.762393in}{2.017086in}}%
\pgfusepath{stroke}%
\end{pgfscope}%
\begin{pgfscope}%
\pgfpathrectangle{\pgfqpoint{0.100000in}{0.212622in}}{\pgfqpoint{3.696000in}{3.696000in}}%
\pgfusepath{clip}%
\pgfsetrectcap%
\pgfsetroundjoin%
\pgfsetlinewidth{1.505625pt}%
\definecolor{currentstroke}{rgb}{1.000000,0.000000,0.000000}%
\pgfsetstrokecolor{currentstroke}%
\pgfsetdash{}{0pt}%
\pgfpathmoveto{\pgfqpoint{1.594926in}{2.305251in}}%
\pgfpathlineto{\pgfqpoint{1.770766in}{2.024269in}}%
\pgfusepath{stroke}%
\end{pgfscope}%
\begin{pgfscope}%
\pgfpathrectangle{\pgfqpoint{0.100000in}{0.212622in}}{\pgfqpoint{3.696000in}{3.696000in}}%
\pgfusepath{clip}%
\pgfsetrectcap%
\pgfsetroundjoin%
\pgfsetlinewidth{1.505625pt}%
\definecolor{currentstroke}{rgb}{1.000000,0.000000,0.000000}%
\pgfsetstrokecolor{currentstroke}%
\pgfsetdash{}{0pt}%
\pgfpathmoveto{\pgfqpoint{1.609202in}{2.325194in}}%
\pgfpathlineto{\pgfqpoint{1.787480in}{2.038608in}}%
\pgfusepath{stroke}%
\end{pgfscope}%
\begin{pgfscope}%
\pgfpathrectangle{\pgfqpoint{0.100000in}{0.212622in}}{\pgfqpoint{3.696000in}{3.696000in}}%
\pgfusepath{clip}%
\pgfsetrectcap%
\pgfsetroundjoin%
\pgfsetlinewidth{1.505625pt}%
\definecolor{currentstroke}{rgb}{1.000000,0.000000,0.000000}%
\pgfsetstrokecolor{currentstroke}%
\pgfsetdash{}{0pt}%
\pgfpathmoveto{\pgfqpoint{1.627511in}{2.344082in}}%
\pgfpathlineto{\pgfqpoint{1.795822in}{2.045764in}}%
\pgfusepath{stroke}%
\end{pgfscope}%
\begin{pgfscope}%
\pgfpathrectangle{\pgfqpoint{0.100000in}{0.212622in}}{\pgfqpoint{3.696000in}{3.696000in}}%
\pgfusepath{clip}%
\pgfsetrectcap%
\pgfsetroundjoin%
\pgfsetlinewidth{1.505625pt}%
\definecolor{currentstroke}{rgb}{1.000000,0.000000,0.000000}%
\pgfsetstrokecolor{currentstroke}%
\pgfsetdash{}{0pt}%
\pgfpathmoveto{\pgfqpoint{1.636061in}{2.357154in}}%
\pgfpathlineto{\pgfqpoint{1.795822in}{2.045764in}}%
\pgfusepath{stroke}%
\end{pgfscope}%
\begin{pgfscope}%
\pgfpathrectangle{\pgfqpoint{0.100000in}{0.212622in}}{\pgfqpoint{3.696000in}{3.696000in}}%
\pgfusepath{clip}%
\pgfsetrectcap%
\pgfsetroundjoin%
\pgfsetlinewidth{1.505625pt}%
\definecolor{currentstroke}{rgb}{1.000000,0.000000,0.000000}%
\pgfsetstrokecolor{currentstroke}%
\pgfsetdash{}{0pt}%
\pgfpathmoveto{\pgfqpoint{1.640755in}{2.363684in}}%
\pgfpathlineto{\pgfqpoint{1.795822in}{2.045764in}}%
\pgfusepath{stroke}%
\end{pgfscope}%
\begin{pgfscope}%
\pgfpathrectangle{\pgfqpoint{0.100000in}{0.212622in}}{\pgfqpoint{3.696000in}{3.696000in}}%
\pgfusepath{clip}%
\pgfsetrectcap%
\pgfsetroundjoin%
\pgfsetlinewidth{1.505625pt}%
\definecolor{currentstroke}{rgb}{1.000000,0.000000,0.000000}%
\pgfsetstrokecolor{currentstroke}%
\pgfsetdash{}{0pt}%
\pgfpathmoveto{\pgfqpoint{1.642958in}{2.367119in}}%
\pgfpathlineto{\pgfqpoint{1.795822in}{2.045764in}}%
\pgfusepath{stroke}%
\end{pgfscope}%
\begin{pgfscope}%
\pgfpathrectangle{\pgfqpoint{0.100000in}{0.212622in}}{\pgfqpoint{3.696000in}{3.696000in}}%
\pgfusepath{clip}%
\pgfsetrectcap%
\pgfsetroundjoin%
\pgfsetlinewidth{1.505625pt}%
\definecolor{currentstroke}{rgb}{1.000000,0.000000,0.000000}%
\pgfsetstrokecolor{currentstroke}%
\pgfsetdash{}{0pt}%
\pgfpathmoveto{\pgfqpoint{1.644355in}{2.369364in}}%
\pgfpathlineto{\pgfqpoint{1.795822in}{2.045764in}}%
\pgfusepath{stroke}%
\end{pgfscope}%
\begin{pgfscope}%
\pgfpathrectangle{\pgfqpoint{0.100000in}{0.212622in}}{\pgfqpoint{3.696000in}{3.696000in}}%
\pgfusepath{clip}%
\pgfsetrectcap%
\pgfsetroundjoin%
\pgfsetlinewidth{1.505625pt}%
\definecolor{currentstroke}{rgb}{1.000000,0.000000,0.000000}%
\pgfsetstrokecolor{currentstroke}%
\pgfsetdash{}{0pt}%
\pgfpathmoveto{\pgfqpoint{1.645011in}{2.370480in}}%
\pgfpathlineto{\pgfqpoint{1.795822in}{2.045764in}}%
\pgfusepath{stroke}%
\end{pgfscope}%
\begin{pgfscope}%
\pgfpathrectangle{\pgfqpoint{0.100000in}{0.212622in}}{\pgfqpoint{3.696000in}{3.696000in}}%
\pgfusepath{clip}%
\pgfsetrectcap%
\pgfsetroundjoin%
\pgfsetlinewidth{1.505625pt}%
\definecolor{currentstroke}{rgb}{1.000000,0.000000,0.000000}%
\pgfsetstrokecolor{currentstroke}%
\pgfsetdash{}{0pt}%
\pgfpathmoveto{\pgfqpoint{1.645419in}{2.371061in}}%
\pgfpathlineto{\pgfqpoint{1.795822in}{2.045764in}}%
\pgfusepath{stroke}%
\end{pgfscope}%
\begin{pgfscope}%
\pgfpathrectangle{\pgfqpoint{0.100000in}{0.212622in}}{\pgfqpoint{3.696000in}{3.696000in}}%
\pgfusepath{clip}%
\pgfsetrectcap%
\pgfsetroundjoin%
\pgfsetlinewidth{1.505625pt}%
\definecolor{currentstroke}{rgb}{1.000000,0.000000,0.000000}%
\pgfsetstrokecolor{currentstroke}%
\pgfsetdash{}{0pt}%
\pgfpathmoveto{\pgfqpoint{1.645646in}{2.371425in}}%
\pgfpathlineto{\pgfqpoint{1.795822in}{2.045764in}}%
\pgfusepath{stroke}%
\end{pgfscope}%
\begin{pgfscope}%
\pgfpathrectangle{\pgfqpoint{0.100000in}{0.212622in}}{\pgfqpoint{3.696000in}{3.696000in}}%
\pgfusepath{clip}%
\pgfsetrectcap%
\pgfsetroundjoin%
\pgfsetlinewidth{1.505625pt}%
\definecolor{currentstroke}{rgb}{1.000000,0.000000,0.000000}%
\pgfsetstrokecolor{currentstroke}%
\pgfsetdash{}{0pt}%
\pgfpathmoveto{\pgfqpoint{1.645772in}{2.371614in}}%
\pgfpathlineto{\pgfqpoint{1.795822in}{2.045764in}}%
\pgfusepath{stroke}%
\end{pgfscope}%
\begin{pgfscope}%
\pgfpathrectangle{\pgfqpoint{0.100000in}{0.212622in}}{\pgfqpoint{3.696000in}{3.696000in}}%
\pgfusepath{clip}%
\pgfsetrectcap%
\pgfsetroundjoin%
\pgfsetlinewidth{1.505625pt}%
\definecolor{currentstroke}{rgb}{1.000000,0.000000,0.000000}%
\pgfsetstrokecolor{currentstroke}%
\pgfsetdash{}{0pt}%
\pgfpathmoveto{\pgfqpoint{1.645841in}{2.371717in}}%
\pgfpathlineto{\pgfqpoint{1.795822in}{2.045764in}}%
\pgfusepath{stroke}%
\end{pgfscope}%
\begin{pgfscope}%
\pgfpathrectangle{\pgfqpoint{0.100000in}{0.212622in}}{\pgfqpoint{3.696000in}{3.696000in}}%
\pgfusepath{clip}%
\pgfsetrectcap%
\pgfsetroundjoin%
\pgfsetlinewidth{1.505625pt}%
\definecolor{currentstroke}{rgb}{1.000000,0.000000,0.000000}%
\pgfsetstrokecolor{currentstroke}%
\pgfsetdash{}{0pt}%
\pgfpathmoveto{\pgfqpoint{1.645878in}{2.371773in}}%
\pgfpathlineto{\pgfqpoint{1.795822in}{2.045764in}}%
\pgfusepath{stroke}%
\end{pgfscope}%
\begin{pgfscope}%
\pgfpathrectangle{\pgfqpoint{0.100000in}{0.212622in}}{\pgfqpoint{3.696000in}{3.696000in}}%
\pgfusepath{clip}%
\pgfsetrectcap%
\pgfsetroundjoin%
\pgfsetlinewidth{1.505625pt}%
\definecolor{currentstroke}{rgb}{1.000000,0.000000,0.000000}%
\pgfsetstrokecolor{currentstroke}%
\pgfsetdash{}{0pt}%
\pgfpathmoveto{\pgfqpoint{1.645899in}{2.371804in}}%
\pgfpathlineto{\pgfqpoint{1.795822in}{2.045764in}}%
\pgfusepath{stroke}%
\end{pgfscope}%
\begin{pgfscope}%
\pgfpathrectangle{\pgfqpoint{0.100000in}{0.212622in}}{\pgfqpoint{3.696000in}{3.696000in}}%
\pgfusepath{clip}%
\pgfsetrectcap%
\pgfsetroundjoin%
\pgfsetlinewidth{1.505625pt}%
\definecolor{currentstroke}{rgb}{1.000000,0.000000,0.000000}%
\pgfsetstrokecolor{currentstroke}%
\pgfsetdash{}{0pt}%
\pgfpathmoveto{\pgfqpoint{1.645909in}{2.371821in}}%
\pgfpathlineto{\pgfqpoint{1.795822in}{2.045764in}}%
\pgfusepath{stroke}%
\end{pgfscope}%
\begin{pgfscope}%
\pgfpathrectangle{\pgfqpoint{0.100000in}{0.212622in}}{\pgfqpoint{3.696000in}{3.696000in}}%
\pgfusepath{clip}%
\pgfsetrectcap%
\pgfsetroundjoin%
\pgfsetlinewidth{1.505625pt}%
\definecolor{currentstroke}{rgb}{1.000000,0.000000,0.000000}%
\pgfsetstrokecolor{currentstroke}%
\pgfsetdash{}{0pt}%
\pgfpathmoveto{\pgfqpoint{1.645917in}{2.371830in}}%
\pgfpathlineto{\pgfqpoint{1.795822in}{2.045764in}}%
\pgfusepath{stroke}%
\end{pgfscope}%
\begin{pgfscope}%
\pgfpathrectangle{\pgfqpoint{0.100000in}{0.212622in}}{\pgfqpoint{3.696000in}{3.696000in}}%
\pgfusepath{clip}%
\pgfsetrectcap%
\pgfsetroundjoin%
\pgfsetlinewidth{1.505625pt}%
\definecolor{currentstroke}{rgb}{1.000000,0.000000,0.000000}%
\pgfsetstrokecolor{currentstroke}%
\pgfsetdash{}{0pt}%
\pgfpathmoveto{\pgfqpoint{1.645920in}{2.371836in}}%
\pgfpathlineto{\pgfqpoint{1.795822in}{2.045764in}}%
\pgfusepath{stroke}%
\end{pgfscope}%
\begin{pgfscope}%
\pgfpathrectangle{\pgfqpoint{0.100000in}{0.212622in}}{\pgfqpoint{3.696000in}{3.696000in}}%
\pgfusepath{clip}%
\pgfsetrectcap%
\pgfsetroundjoin%
\pgfsetlinewidth{1.505625pt}%
\definecolor{currentstroke}{rgb}{1.000000,0.000000,0.000000}%
\pgfsetstrokecolor{currentstroke}%
\pgfsetdash{}{0pt}%
\pgfpathmoveto{\pgfqpoint{1.645923in}{2.371839in}}%
\pgfpathlineto{\pgfqpoint{1.795822in}{2.045764in}}%
\pgfusepath{stroke}%
\end{pgfscope}%
\begin{pgfscope}%
\pgfpathrectangle{\pgfqpoint{0.100000in}{0.212622in}}{\pgfqpoint{3.696000in}{3.696000in}}%
\pgfusepath{clip}%
\pgfsetrectcap%
\pgfsetroundjoin%
\pgfsetlinewidth{1.505625pt}%
\definecolor{currentstroke}{rgb}{1.000000,0.000000,0.000000}%
\pgfsetstrokecolor{currentstroke}%
\pgfsetdash{}{0pt}%
\pgfpathmoveto{\pgfqpoint{1.645924in}{2.371840in}}%
\pgfpathlineto{\pgfqpoint{1.795822in}{2.045764in}}%
\pgfusepath{stroke}%
\end{pgfscope}%
\begin{pgfscope}%
\pgfpathrectangle{\pgfqpoint{0.100000in}{0.212622in}}{\pgfqpoint{3.696000in}{3.696000in}}%
\pgfusepath{clip}%
\pgfsetrectcap%
\pgfsetroundjoin%
\pgfsetlinewidth{1.505625pt}%
\definecolor{currentstroke}{rgb}{1.000000,0.000000,0.000000}%
\pgfsetstrokecolor{currentstroke}%
\pgfsetdash{}{0pt}%
\pgfpathmoveto{\pgfqpoint{1.648292in}{2.373429in}}%
\pgfpathlineto{\pgfqpoint{1.795822in}{2.045764in}}%
\pgfusepath{stroke}%
\end{pgfscope}%
\begin{pgfscope}%
\pgfpathrectangle{\pgfqpoint{0.100000in}{0.212622in}}{\pgfqpoint{3.696000in}{3.696000in}}%
\pgfusepath{clip}%
\pgfsetrectcap%
\pgfsetroundjoin%
\pgfsetlinewidth{1.505625pt}%
\definecolor{currentstroke}{rgb}{1.000000,0.000000,0.000000}%
\pgfsetstrokecolor{currentstroke}%
\pgfsetdash{}{0pt}%
\pgfpathmoveto{\pgfqpoint{1.649688in}{2.374513in}}%
\pgfpathlineto{\pgfqpoint{1.795822in}{2.045764in}}%
\pgfusepath{stroke}%
\end{pgfscope}%
\begin{pgfscope}%
\pgfpathrectangle{\pgfqpoint{0.100000in}{0.212622in}}{\pgfqpoint{3.696000in}{3.696000in}}%
\pgfusepath{clip}%
\pgfsetrectcap%
\pgfsetroundjoin%
\pgfsetlinewidth{1.505625pt}%
\definecolor{currentstroke}{rgb}{1.000000,0.000000,0.000000}%
\pgfsetstrokecolor{currentstroke}%
\pgfsetdash{}{0pt}%
\pgfpathmoveto{\pgfqpoint{1.654583in}{2.376030in}}%
\pgfpathlineto{\pgfqpoint{1.795822in}{2.045764in}}%
\pgfusepath{stroke}%
\end{pgfscope}%
\begin{pgfscope}%
\pgfpathrectangle{\pgfqpoint{0.100000in}{0.212622in}}{\pgfqpoint{3.696000in}{3.696000in}}%
\pgfusepath{clip}%
\pgfsetrectcap%
\pgfsetroundjoin%
\pgfsetlinewidth{1.505625pt}%
\definecolor{currentstroke}{rgb}{1.000000,0.000000,0.000000}%
\pgfsetstrokecolor{currentstroke}%
\pgfsetdash{}{0pt}%
\pgfpathmoveto{\pgfqpoint{1.661802in}{2.378586in}}%
\pgfpathlineto{\pgfqpoint{1.795822in}{2.045764in}}%
\pgfusepath{stroke}%
\end{pgfscope}%
\begin{pgfscope}%
\pgfpathrectangle{\pgfqpoint{0.100000in}{0.212622in}}{\pgfqpoint{3.696000in}{3.696000in}}%
\pgfusepath{clip}%
\pgfsetrectcap%
\pgfsetroundjoin%
\pgfsetlinewidth{1.505625pt}%
\definecolor{currentstroke}{rgb}{1.000000,0.000000,0.000000}%
\pgfsetstrokecolor{currentstroke}%
\pgfsetdash{}{0pt}%
\pgfpathmoveto{\pgfqpoint{1.672363in}{2.380011in}}%
\pgfpathlineto{\pgfqpoint{1.795822in}{2.045764in}}%
\pgfusepath{stroke}%
\end{pgfscope}%
\begin{pgfscope}%
\pgfpathrectangle{\pgfqpoint{0.100000in}{0.212622in}}{\pgfqpoint{3.696000in}{3.696000in}}%
\pgfusepath{clip}%
\pgfsetrectcap%
\pgfsetroundjoin%
\pgfsetlinewidth{1.505625pt}%
\definecolor{currentstroke}{rgb}{1.000000,0.000000,0.000000}%
\pgfsetstrokecolor{currentstroke}%
\pgfsetdash{}{0pt}%
\pgfpathmoveto{\pgfqpoint{1.684693in}{2.380609in}}%
\pgfpathlineto{\pgfqpoint{1.795822in}{2.045764in}}%
\pgfusepath{stroke}%
\end{pgfscope}%
\begin{pgfscope}%
\pgfpathrectangle{\pgfqpoint{0.100000in}{0.212622in}}{\pgfqpoint{3.696000in}{3.696000in}}%
\pgfusepath{clip}%
\pgfsetrectcap%
\pgfsetroundjoin%
\pgfsetlinewidth{1.505625pt}%
\definecolor{currentstroke}{rgb}{1.000000,0.000000,0.000000}%
\pgfsetstrokecolor{currentstroke}%
\pgfsetdash{}{0pt}%
\pgfpathmoveto{\pgfqpoint{1.698426in}{2.381170in}}%
\pgfpathlineto{\pgfqpoint{1.795822in}{2.045764in}}%
\pgfusepath{stroke}%
\end{pgfscope}%
\begin{pgfscope}%
\pgfpathrectangle{\pgfqpoint{0.100000in}{0.212622in}}{\pgfqpoint{3.696000in}{3.696000in}}%
\pgfusepath{clip}%
\pgfsetrectcap%
\pgfsetroundjoin%
\pgfsetlinewidth{1.505625pt}%
\definecolor{currentstroke}{rgb}{1.000000,0.000000,0.000000}%
\pgfsetstrokecolor{currentstroke}%
\pgfsetdash{}{0pt}%
\pgfpathmoveto{\pgfqpoint{1.714482in}{2.379732in}}%
\pgfpathlineto{\pgfqpoint{1.795822in}{2.045764in}}%
\pgfusepath{stroke}%
\end{pgfscope}%
\begin{pgfscope}%
\pgfpathrectangle{\pgfqpoint{0.100000in}{0.212622in}}{\pgfqpoint{3.696000in}{3.696000in}}%
\pgfusepath{clip}%
\pgfsetrectcap%
\pgfsetroundjoin%
\pgfsetlinewidth{1.505625pt}%
\definecolor{currentstroke}{rgb}{1.000000,0.000000,0.000000}%
\pgfsetstrokecolor{currentstroke}%
\pgfsetdash{}{0pt}%
\pgfpathmoveto{\pgfqpoint{1.733905in}{2.380460in}}%
\pgfpathlineto{\pgfqpoint{1.795822in}{2.045764in}}%
\pgfusepath{stroke}%
\end{pgfscope}%
\begin{pgfscope}%
\pgfpathrectangle{\pgfqpoint{0.100000in}{0.212622in}}{\pgfqpoint{3.696000in}{3.696000in}}%
\pgfusepath{clip}%
\pgfsetrectcap%
\pgfsetroundjoin%
\pgfsetlinewidth{1.505625pt}%
\definecolor{currentstroke}{rgb}{1.000000,0.000000,0.000000}%
\pgfsetstrokecolor{currentstroke}%
\pgfsetdash{}{0pt}%
\pgfpathmoveto{\pgfqpoint{1.755907in}{2.375849in}}%
\pgfpathlineto{\pgfqpoint{1.795822in}{2.045764in}}%
\pgfusepath{stroke}%
\end{pgfscope}%
\begin{pgfscope}%
\pgfpathrectangle{\pgfqpoint{0.100000in}{0.212622in}}{\pgfqpoint{3.696000in}{3.696000in}}%
\pgfusepath{clip}%
\pgfsetrectcap%
\pgfsetroundjoin%
\pgfsetlinewidth{1.505625pt}%
\definecolor{currentstroke}{rgb}{1.000000,0.000000,0.000000}%
\pgfsetstrokecolor{currentstroke}%
\pgfsetdash{}{0pt}%
\pgfpathmoveto{\pgfqpoint{1.767961in}{2.373870in}}%
\pgfpathlineto{\pgfqpoint{1.795822in}{2.045764in}}%
\pgfusepath{stroke}%
\end{pgfscope}%
\begin{pgfscope}%
\pgfpathrectangle{\pgfqpoint{0.100000in}{0.212622in}}{\pgfqpoint{3.696000in}{3.696000in}}%
\pgfusepath{clip}%
\pgfsetrectcap%
\pgfsetroundjoin%
\pgfsetlinewidth{1.505625pt}%
\definecolor{currentstroke}{rgb}{1.000000,0.000000,0.000000}%
\pgfsetstrokecolor{currentstroke}%
\pgfsetdash{}{0pt}%
\pgfpathmoveto{\pgfqpoint{1.782353in}{2.372794in}}%
\pgfpathlineto{\pgfqpoint{1.795822in}{2.045764in}}%
\pgfusepath{stroke}%
\end{pgfscope}%
\begin{pgfscope}%
\pgfpathrectangle{\pgfqpoint{0.100000in}{0.212622in}}{\pgfqpoint{3.696000in}{3.696000in}}%
\pgfusepath{clip}%
\pgfsetrectcap%
\pgfsetroundjoin%
\pgfsetlinewidth{1.505625pt}%
\definecolor{currentstroke}{rgb}{1.000000,0.000000,0.000000}%
\pgfsetstrokecolor{currentstroke}%
\pgfsetdash{}{0pt}%
\pgfpathmoveto{\pgfqpoint{1.790160in}{2.373233in}}%
\pgfpathlineto{\pgfqpoint{1.795822in}{2.045764in}}%
\pgfusepath{stroke}%
\end{pgfscope}%
\begin{pgfscope}%
\pgfpathrectangle{\pgfqpoint{0.100000in}{0.212622in}}{\pgfqpoint{3.696000in}{3.696000in}}%
\pgfusepath{clip}%
\pgfsetrectcap%
\pgfsetroundjoin%
\pgfsetlinewidth{1.505625pt}%
\definecolor{currentstroke}{rgb}{1.000000,0.000000,0.000000}%
\pgfsetstrokecolor{currentstroke}%
\pgfsetdash{}{0pt}%
\pgfpathmoveto{\pgfqpoint{1.800055in}{2.371832in}}%
\pgfpathlineto{\pgfqpoint{1.795822in}{2.045764in}}%
\pgfusepath{stroke}%
\end{pgfscope}%
\begin{pgfscope}%
\pgfpathrectangle{\pgfqpoint{0.100000in}{0.212622in}}{\pgfqpoint{3.696000in}{3.696000in}}%
\pgfusepath{clip}%
\pgfsetrectcap%
\pgfsetroundjoin%
\pgfsetlinewidth{1.505625pt}%
\definecolor{currentstroke}{rgb}{1.000000,0.000000,0.000000}%
\pgfsetstrokecolor{currentstroke}%
\pgfsetdash{}{0pt}%
\pgfpathmoveto{\pgfqpoint{1.818161in}{2.370564in}}%
\pgfpathlineto{\pgfqpoint{1.795822in}{2.045764in}}%
\pgfusepath{stroke}%
\end{pgfscope}%
\begin{pgfscope}%
\pgfpathrectangle{\pgfqpoint{0.100000in}{0.212622in}}{\pgfqpoint{3.696000in}{3.696000in}}%
\pgfusepath{clip}%
\pgfsetrectcap%
\pgfsetroundjoin%
\pgfsetlinewidth{1.505625pt}%
\definecolor{currentstroke}{rgb}{1.000000,0.000000,0.000000}%
\pgfsetstrokecolor{currentstroke}%
\pgfsetdash{}{0pt}%
\pgfpathmoveto{\pgfqpoint{1.842617in}{2.363516in}}%
\pgfpathlineto{\pgfqpoint{1.795822in}{2.045764in}}%
\pgfusepath{stroke}%
\end{pgfscope}%
\begin{pgfscope}%
\pgfpathrectangle{\pgfqpoint{0.100000in}{0.212622in}}{\pgfqpoint{3.696000in}{3.696000in}}%
\pgfusepath{clip}%
\pgfsetrectcap%
\pgfsetroundjoin%
\pgfsetlinewidth{1.505625pt}%
\definecolor{currentstroke}{rgb}{1.000000,0.000000,0.000000}%
\pgfsetstrokecolor{currentstroke}%
\pgfsetdash{}{0pt}%
\pgfpathmoveto{\pgfqpoint{1.873323in}{2.362067in}}%
\pgfpathlineto{\pgfqpoint{1.835574in}{2.034165in}}%
\pgfusepath{stroke}%
\end{pgfscope}%
\begin{pgfscope}%
\pgfpathrectangle{\pgfqpoint{0.100000in}{0.212622in}}{\pgfqpoint{3.696000in}{3.696000in}}%
\pgfusepath{clip}%
\pgfsetrectcap%
\pgfsetroundjoin%
\pgfsetlinewidth{1.505625pt}%
\definecolor{currentstroke}{rgb}{1.000000,0.000000,0.000000}%
\pgfsetstrokecolor{currentstroke}%
\pgfsetdash{}{0pt}%
\pgfpathmoveto{\pgfqpoint{1.905601in}{2.363422in}}%
\pgfpathlineto{\pgfqpoint{1.862119in}{2.026420in}}%
\pgfusepath{stroke}%
\end{pgfscope}%
\begin{pgfscope}%
\pgfpathrectangle{\pgfqpoint{0.100000in}{0.212622in}}{\pgfqpoint{3.696000in}{3.696000in}}%
\pgfusepath{clip}%
\pgfsetrectcap%
\pgfsetroundjoin%
\pgfsetlinewidth{1.505625pt}%
\definecolor{currentstroke}{rgb}{1.000000,0.000000,0.000000}%
\pgfsetstrokecolor{currentstroke}%
\pgfsetdash{}{0pt}%
\pgfpathmoveto{\pgfqpoint{1.945921in}{2.364690in}}%
\pgfpathlineto{\pgfqpoint{1.915317in}{2.010898in}}%
\pgfusepath{stroke}%
\end{pgfscope}%
\begin{pgfscope}%
\pgfpathrectangle{\pgfqpoint{0.100000in}{0.212622in}}{\pgfqpoint{3.696000in}{3.696000in}}%
\pgfusepath{clip}%
\pgfsetrectcap%
\pgfsetroundjoin%
\pgfsetlinewidth{1.505625pt}%
\definecolor{currentstroke}{rgb}{1.000000,0.000000,0.000000}%
\pgfsetstrokecolor{currentstroke}%
\pgfsetdash{}{0pt}%
\pgfpathmoveto{\pgfqpoint{1.968858in}{2.363314in}}%
\pgfpathlineto{\pgfqpoint{1.928638in}{2.007012in}}%
\pgfusepath{stroke}%
\end{pgfscope}%
\begin{pgfscope}%
\pgfpathrectangle{\pgfqpoint{0.100000in}{0.212622in}}{\pgfqpoint{3.696000in}{3.696000in}}%
\pgfusepath{clip}%
\pgfsetrectcap%
\pgfsetroundjoin%
\pgfsetlinewidth{1.505625pt}%
\definecolor{currentstroke}{rgb}{1.000000,0.000000,0.000000}%
\pgfsetstrokecolor{currentstroke}%
\pgfsetdash{}{0pt}%
\pgfpathmoveto{\pgfqpoint{1.999923in}{2.361574in}}%
\pgfpathlineto{\pgfqpoint{1.968656in}{1.995335in}}%
\pgfusepath{stroke}%
\end{pgfscope}%
\begin{pgfscope}%
\pgfpathrectangle{\pgfqpoint{0.100000in}{0.212622in}}{\pgfqpoint{3.696000in}{3.696000in}}%
\pgfusepath{clip}%
\pgfsetrectcap%
\pgfsetroundjoin%
\pgfsetlinewidth{1.505625pt}%
\definecolor{currentstroke}{rgb}{1.000000,0.000000,0.000000}%
\pgfsetstrokecolor{currentstroke}%
\pgfsetdash{}{0pt}%
\pgfpathmoveto{\pgfqpoint{2.015633in}{2.359314in}}%
\pgfpathlineto{\pgfqpoint{1.982013in}{1.991438in}}%
\pgfusepath{stroke}%
\end{pgfscope}%
\begin{pgfscope}%
\pgfpathrectangle{\pgfqpoint{0.100000in}{0.212622in}}{\pgfqpoint{3.696000in}{3.696000in}}%
\pgfusepath{clip}%
\pgfsetrectcap%
\pgfsetroundjoin%
\pgfsetlinewidth{1.505625pt}%
\definecolor{currentstroke}{rgb}{1.000000,0.000000,0.000000}%
\pgfsetstrokecolor{currentstroke}%
\pgfsetdash{}{0pt}%
\pgfpathmoveto{\pgfqpoint{2.039161in}{2.354632in}}%
\pgfpathlineto{\pgfqpoint{2.008755in}{1.983636in}}%
\pgfusepath{stroke}%
\end{pgfscope}%
\begin{pgfscope}%
\pgfpathrectangle{\pgfqpoint{0.100000in}{0.212622in}}{\pgfqpoint{3.696000in}{3.696000in}}%
\pgfusepath{clip}%
\pgfsetrectcap%
\pgfsetroundjoin%
\pgfsetlinewidth{1.505625pt}%
\definecolor{currentstroke}{rgb}{1.000000,0.000000,0.000000}%
\pgfsetstrokecolor{currentstroke}%
\pgfsetdash{}{0pt}%
\pgfpathmoveto{\pgfqpoint{2.052721in}{2.354903in}}%
\pgfpathlineto{\pgfqpoint{2.022139in}{1.979731in}}%
\pgfusepath{stroke}%
\end{pgfscope}%
\begin{pgfscope}%
\pgfpathrectangle{\pgfqpoint{0.100000in}{0.212622in}}{\pgfqpoint{3.696000in}{3.696000in}}%
\pgfusepath{clip}%
\pgfsetrectcap%
\pgfsetroundjoin%
\pgfsetlinewidth{1.505625pt}%
\definecolor{currentstroke}{rgb}{1.000000,0.000000,0.000000}%
\pgfsetstrokecolor{currentstroke}%
\pgfsetdash{}{0pt}%
\pgfpathmoveto{\pgfqpoint{2.069877in}{2.351267in}}%
\pgfpathlineto{\pgfqpoint{2.035532in}{1.975823in}}%
\pgfusepath{stroke}%
\end{pgfscope}%
\begin{pgfscope}%
\pgfpathrectangle{\pgfqpoint{0.100000in}{0.212622in}}{\pgfqpoint{3.696000in}{3.696000in}}%
\pgfusepath{clip}%
\pgfsetrectcap%
\pgfsetroundjoin%
\pgfsetlinewidth{1.505625pt}%
\definecolor{currentstroke}{rgb}{1.000000,0.000000,0.000000}%
\pgfsetstrokecolor{currentstroke}%
\pgfsetdash{}{0pt}%
\pgfpathmoveto{\pgfqpoint{2.092679in}{2.350523in}}%
\pgfpathlineto{\pgfqpoint{2.062345in}{1.968000in}}%
\pgfusepath{stroke}%
\end{pgfscope}%
\begin{pgfscope}%
\pgfpathrectangle{\pgfqpoint{0.100000in}{0.212622in}}{\pgfqpoint{3.696000in}{3.696000in}}%
\pgfusepath{clip}%
\pgfsetrectcap%
\pgfsetroundjoin%
\pgfsetlinewidth{1.505625pt}%
\definecolor{currentstroke}{rgb}{1.000000,0.000000,0.000000}%
\pgfsetstrokecolor{currentstroke}%
\pgfsetdash{}{0pt}%
\pgfpathmoveto{\pgfqpoint{2.121682in}{2.357047in}}%
\pgfpathlineto{\pgfqpoint{2.089194in}{1.960166in}}%
\pgfusepath{stroke}%
\end{pgfscope}%
\begin{pgfscope}%
\pgfpathrectangle{\pgfqpoint{0.100000in}{0.212622in}}{\pgfqpoint{3.696000in}{3.696000in}}%
\pgfusepath{clip}%
\pgfsetrectcap%
\pgfsetroundjoin%
\pgfsetlinewidth{1.505625pt}%
\definecolor{currentstroke}{rgb}{1.000000,0.000000,0.000000}%
\pgfsetstrokecolor{currentstroke}%
\pgfsetdash{}{0pt}%
\pgfpathmoveto{\pgfqpoint{2.150230in}{2.348622in}}%
\pgfpathlineto{\pgfqpoint{2.129535in}{1.948395in}}%
\pgfusepath{stroke}%
\end{pgfscope}%
\begin{pgfscope}%
\pgfpathrectangle{\pgfqpoint{0.100000in}{0.212622in}}{\pgfqpoint{3.696000in}{3.696000in}}%
\pgfusepath{clip}%
\pgfsetrectcap%
\pgfsetroundjoin%
\pgfsetlinewidth{1.505625pt}%
\definecolor{currentstroke}{rgb}{1.000000,0.000000,0.000000}%
\pgfsetstrokecolor{currentstroke}%
\pgfsetdash{}{0pt}%
\pgfpathmoveto{\pgfqpoint{2.189221in}{2.354429in}}%
\pgfpathlineto{\pgfqpoint{2.156474in}{1.940535in}}%
\pgfusepath{stroke}%
\end{pgfscope}%
\begin{pgfscope}%
\pgfpathrectangle{\pgfqpoint{0.100000in}{0.212622in}}{\pgfqpoint{3.696000in}{3.696000in}}%
\pgfusepath{clip}%
\pgfsetrectcap%
\pgfsetroundjoin%
\pgfsetlinewidth{1.505625pt}%
\definecolor{currentstroke}{rgb}{1.000000,0.000000,0.000000}%
\pgfsetstrokecolor{currentstroke}%
\pgfsetdash{}{0pt}%
\pgfpathmoveto{\pgfqpoint{2.229220in}{2.333922in}}%
\pgfpathlineto{\pgfqpoint{2.210461in}{1.924783in}}%
\pgfusepath{stroke}%
\end{pgfscope}%
\begin{pgfscope}%
\pgfpathrectangle{\pgfqpoint{0.100000in}{0.212622in}}{\pgfqpoint{3.696000in}{3.696000in}}%
\pgfusepath{clip}%
\pgfsetrectcap%
\pgfsetroundjoin%
\pgfsetlinewidth{1.505625pt}%
\definecolor{currentstroke}{rgb}{1.000000,0.000000,0.000000}%
\pgfsetstrokecolor{currentstroke}%
\pgfsetdash{}{0pt}%
\pgfpathmoveto{\pgfqpoint{2.254208in}{2.340503in}}%
\pgfpathlineto{\pgfqpoint{2.223981in}{1.920839in}}%
\pgfusepath{stroke}%
\end{pgfscope}%
\begin{pgfscope}%
\pgfpathrectangle{\pgfqpoint{0.100000in}{0.212622in}}{\pgfqpoint{3.696000in}{3.696000in}}%
\pgfusepath{clip}%
\pgfsetrectcap%
\pgfsetroundjoin%
\pgfsetlinewidth{1.505625pt}%
\definecolor{currentstroke}{rgb}{1.000000,0.000000,0.000000}%
\pgfsetstrokecolor{currentstroke}%
\pgfsetdash{}{0pt}%
\pgfpathmoveto{\pgfqpoint{2.284698in}{2.329634in}}%
\pgfpathlineto{\pgfqpoint{2.264594in}{1.908989in}}%
\pgfusepath{stroke}%
\end{pgfscope}%
\begin{pgfscope}%
\pgfpathrectangle{\pgfqpoint{0.100000in}{0.212622in}}{\pgfqpoint{3.696000in}{3.696000in}}%
\pgfusepath{clip}%
\pgfsetrectcap%
\pgfsetroundjoin%
\pgfsetlinewidth{1.505625pt}%
\definecolor{currentstroke}{rgb}{1.000000,0.000000,0.000000}%
\pgfsetstrokecolor{currentstroke}%
\pgfsetdash{}{0pt}%
\pgfpathmoveto{\pgfqpoint{2.322059in}{2.342759in}}%
\pgfpathlineto{\pgfqpoint{2.291715in}{1.901076in}}%
\pgfusepath{stroke}%
\end{pgfscope}%
\begin{pgfscope}%
\pgfpathrectangle{\pgfqpoint{0.100000in}{0.212622in}}{\pgfqpoint{3.696000in}{3.696000in}}%
\pgfusepath{clip}%
\pgfsetrectcap%
\pgfsetroundjoin%
\pgfsetlinewidth{1.505625pt}%
\definecolor{currentstroke}{rgb}{1.000000,0.000000,0.000000}%
\pgfsetstrokecolor{currentstroke}%
\pgfsetdash{}{0pt}%
\pgfpathmoveto{\pgfqpoint{2.339211in}{2.332368in}}%
\pgfpathlineto{\pgfqpoint{2.318873in}{1.893152in}}%
\pgfusepath{stroke}%
\end{pgfscope}%
\begin{pgfscope}%
\pgfpathrectangle{\pgfqpoint{0.100000in}{0.212622in}}{\pgfqpoint{3.696000in}{3.696000in}}%
\pgfusepath{clip}%
\pgfsetrectcap%
\pgfsetroundjoin%
\pgfsetlinewidth{1.505625pt}%
\definecolor{currentstroke}{rgb}{1.000000,0.000000,0.000000}%
\pgfsetstrokecolor{currentstroke}%
\pgfsetdash{}{0pt}%
\pgfpathmoveto{\pgfqpoint{2.350630in}{2.335807in}}%
\pgfpathlineto{\pgfqpoint{2.332465in}{1.889186in}}%
\pgfusepath{stroke}%
\end{pgfscope}%
\begin{pgfscope}%
\pgfpathrectangle{\pgfqpoint{0.100000in}{0.212622in}}{\pgfqpoint{3.696000in}{3.696000in}}%
\pgfusepath{clip}%
\pgfsetrectcap%
\pgfsetroundjoin%
\pgfsetlinewidth{1.505625pt}%
\definecolor{currentstroke}{rgb}{1.000000,0.000000,0.000000}%
\pgfsetstrokecolor{currentstroke}%
\pgfsetdash{}{0pt}%
\pgfpathmoveto{\pgfqpoint{2.355962in}{2.333303in}}%
\pgfpathlineto{\pgfqpoint{2.332465in}{1.889186in}}%
\pgfusepath{stroke}%
\end{pgfscope}%
\begin{pgfscope}%
\pgfpathrectangle{\pgfqpoint{0.100000in}{0.212622in}}{\pgfqpoint{3.696000in}{3.696000in}}%
\pgfusepath{clip}%
\pgfsetrectcap%
\pgfsetroundjoin%
\pgfsetlinewidth{1.505625pt}%
\definecolor{currentstroke}{rgb}{1.000000,0.000000,0.000000}%
\pgfsetstrokecolor{currentstroke}%
\pgfsetdash{}{0pt}%
\pgfpathmoveto{\pgfqpoint{2.359352in}{2.334187in}}%
\pgfpathlineto{\pgfqpoint{2.332465in}{1.889186in}}%
\pgfusepath{stroke}%
\end{pgfscope}%
\begin{pgfscope}%
\pgfpathrectangle{\pgfqpoint{0.100000in}{0.212622in}}{\pgfqpoint{3.696000in}{3.696000in}}%
\pgfusepath{clip}%
\pgfsetrectcap%
\pgfsetroundjoin%
\pgfsetlinewidth{1.505625pt}%
\definecolor{currentstroke}{rgb}{1.000000,0.000000,0.000000}%
\pgfsetstrokecolor{currentstroke}%
\pgfsetdash{}{0pt}%
\pgfpathmoveto{\pgfqpoint{2.364209in}{2.331866in}}%
\pgfpathlineto{\pgfqpoint{2.346067in}{1.885217in}}%
\pgfusepath{stroke}%
\end{pgfscope}%
\begin{pgfscope}%
\pgfpathrectangle{\pgfqpoint{0.100000in}{0.212622in}}{\pgfqpoint{3.696000in}{3.696000in}}%
\pgfusepath{clip}%
\pgfsetrectcap%
\pgfsetroundjoin%
\pgfsetlinewidth{1.505625pt}%
\definecolor{currentstroke}{rgb}{1.000000,0.000000,0.000000}%
\pgfsetstrokecolor{currentstroke}%
\pgfsetdash{}{0pt}%
\pgfpathmoveto{\pgfqpoint{2.374386in}{2.335356in}}%
\pgfpathlineto{\pgfqpoint{2.346067in}{1.885217in}}%
\pgfusepath{stroke}%
\end{pgfscope}%
\begin{pgfscope}%
\pgfpathrectangle{\pgfqpoint{0.100000in}{0.212622in}}{\pgfqpoint{3.696000in}{3.696000in}}%
\pgfusepath{clip}%
\pgfsetrectcap%
\pgfsetroundjoin%
\pgfsetlinewidth{1.505625pt}%
\definecolor{currentstroke}{rgb}{1.000000,0.000000,0.000000}%
\pgfsetstrokecolor{currentstroke}%
\pgfsetdash{}{0pt}%
\pgfpathmoveto{\pgfqpoint{2.385007in}{2.332710in}}%
\pgfpathlineto{\pgfqpoint{2.359678in}{1.881246in}}%
\pgfusepath{stroke}%
\end{pgfscope}%
\begin{pgfscope}%
\pgfpathrectangle{\pgfqpoint{0.100000in}{0.212622in}}{\pgfqpoint{3.696000in}{3.696000in}}%
\pgfusepath{clip}%
\pgfsetrectcap%
\pgfsetroundjoin%
\pgfsetlinewidth{1.505625pt}%
\definecolor{currentstroke}{rgb}{1.000000,0.000000,0.000000}%
\pgfsetstrokecolor{currentstroke}%
\pgfsetdash{}{0pt}%
\pgfpathmoveto{\pgfqpoint{2.400930in}{2.334964in}}%
\pgfpathlineto{\pgfqpoint{2.373298in}{1.877272in}}%
\pgfusepath{stroke}%
\end{pgfscope}%
\begin{pgfscope}%
\pgfpathrectangle{\pgfqpoint{0.100000in}{0.212622in}}{\pgfqpoint{3.696000in}{3.696000in}}%
\pgfusepath{clip}%
\pgfsetrectcap%
\pgfsetroundjoin%
\pgfsetlinewidth{1.505625pt}%
\definecolor{currentstroke}{rgb}{1.000000,0.000000,0.000000}%
\pgfsetstrokecolor{currentstroke}%
\pgfsetdash{}{0pt}%
\pgfpathmoveto{\pgfqpoint{2.409444in}{2.333765in}}%
\pgfpathlineto{\pgfqpoint{2.386928in}{1.873295in}}%
\pgfusepath{stroke}%
\end{pgfscope}%
\begin{pgfscope}%
\pgfpathrectangle{\pgfqpoint{0.100000in}{0.212622in}}{\pgfqpoint{3.696000in}{3.696000in}}%
\pgfusepath{clip}%
\pgfsetrectcap%
\pgfsetroundjoin%
\pgfsetlinewidth{1.505625pt}%
\definecolor{currentstroke}{rgb}{1.000000,0.000000,0.000000}%
\pgfsetstrokecolor{currentstroke}%
\pgfsetdash{}{0pt}%
\pgfpathmoveto{\pgfqpoint{2.423509in}{2.334798in}}%
\pgfpathlineto{\pgfqpoint{2.400566in}{1.869316in}}%
\pgfusepath{stroke}%
\end{pgfscope}%
\begin{pgfscope}%
\pgfpathrectangle{\pgfqpoint{0.100000in}{0.212622in}}{\pgfqpoint{3.696000in}{3.696000in}}%
\pgfusepath{clip}%
\pgfsetrectcap%
\pgfsetroundjoin%
\pgfsetlinewidth{1.505625pt}%
\definecolor{currentstroke}{rgb}{1.000000,0.000000,0.000000}%
\pgfsetstrokecolor{currentstroke}%
\pgfsetdash{}{0pt}%
\pgfpathmoveto{\pgfqpoint{2.438650in}{2.330474in}}%
\pgfpathlineto{\pgfqpoint{2.414214in}{1.865334in}}%
\pgfusepath{stroke}%
\end{pgfscope}%
\begin{pgfscope}%
\pgfpathrectangle{\pgfqpoint{0.100000in}{0.212622in}}{\pgfqpoint{3.696000in}{3.696000in}}%
\pgfusepath{clip}%
\pgfsetrectcap%
\pgfsetroundjoin%
\pgfsetlinewidth{1.505625pt}%
\definecolor{currentstroke}{rgb}{1.000000,0.000000,0.000000}%
\pgfsetstrokecolor{currentstroke}%
\pgfsetdash{}{0pt}%
\pgfpathmoveto{\pgfqpoint{2.463517in}{2.332776in}}%
\pgfpathlineto{\pgfqpoint{2.441537in}{1.857361in}}%
\pgfusepath{stroke}%
\end{pgfscope}%
\begin{pgfscope}%
\pgfpathrectangle{\pgfqpoint{0.100000in}{0.212622in}}{\pgfqpoint{3.696000in}{3.696000in}}%
\pgfusepath{clip}%
\pgfsetrectcap%
\pgfsetroundjoin%
\pgfsetlinewidth{1.505625pt}%
\definecolor{currentstroke}{rgb}{1.000000,0.000000,0.000000}%
\pgfsetstrokecolor{currentstroke}%
\pgfsetdash{}{0pt}%
\pgfpathmoveto{\pgfqpoint{2.476034in}{2.330449in}}%
\pgfpathlineto{\pgfqpoint{2.455213in}{1.853371in}}%
\pgfusepath{stroke}%
\end{pgfscope}%
\begin{pgfscope}%
\pgfpathrectangle{\pgfqpoint{0.100000in}{0.212622in}}{\pgfqpoint{3.696000in}{3.696000in}}%
\pgfusepath{clip}%
\pgfsetrectcap%
\pgfsetroundjoin%
\pgfsetlinewidth{1.505625pt}%
\definecolor{currentstroke}{rgb}{1.000000,0.000000,0.000000}%
\pgfsetstrokecolor{currentstroke}%
\pgfsetdash{}{0pt}%
\pgfpathmoveto{\pgfqpoint{2.493633in}{2.332720in}}%
\pgfpathlineto{\pgfqpoint{2.468898in}{1.849378in}}%
\pgfusepath{stroke}%
\end{pgfscope}%
\begin{pgfscope}%
\pgfpathrectangle{\pgfqpoint{0.100000in}{0.212622in}}{\pgfqpoint{3.696000in}{3.696000in}}%
\pgfusepath{clip}%
\pgfsetrectcap%
\pgfsetroundjoin%
\pgfsetlinewidth{1.505625pt}%
\definecolor{currentstroke}{rgb}{1.000000,0.000000,0.000000}%
\pgfsetstrokecolor{currentstroke}%
\pgfsetdash{}{0pt}%
\pgfpathmoveto{\pgfqpoint{2.509696in}{2.324093in}}%
\pgfpathlineto{\pgfqpoint{2.496295in}{1.841385in}}%
\pgfusepath{stroke}%
\end{pgfscope}%
\begin{pgfscope}%
\pgfpathrectangle{\pgfqpoint{0.100000in}{0.212622in}}{\pgfqpoint{3.696000in}{3.696000in}}%
\pgfusepath{clip}%
\pgfsetrectcap%
\pgfsetroundjoin%
\pgfsetlinewidth{1.505625pt}%
\definecolor{currentstroke}{rgb}{1.000000,0.000000,0.000000}%
\pgfsetstrokecolor{currentstroke}%
\pgfsetdash{}{0pt}%
\pgfpathmoveto{\pgfqpoint{2.533270in}{2.327371in}}%
\pgfpathlineto{\pgfqpoint{2.523730in}{1.833380in}}%
\pgfusepath{stroke}%
\end{pgfscope}%
\begin{pgfscope}%
\pgfpathrectangle{\pgfqpoint{0.100000in}{0.212622in}}{\pgfqpoint{3.696000in}{3.696000in}}%
\pgfusepath{clip}%
\pgfsetrectcap%
\pgfsetroundjoin%
\pgfsetlinewidth{1.505625pt}%
\definecolor{currentstroke}{rgb}{1.000000,0.000000,0.000000}%
\pgfsetstrokecolor{currentstroke}%
\pgfsetdash{}{0pt}%
\pgfpathmoveto{\pgfqpoint{2.554501in}{2.321949in}}%
\pgfpathlineto{\pgfqpoint{2.537461in}{1.829374in}}%
\pgfusepath{stroke}%
\end{pgfscope}%
\begin{pgfscope}%
\pgfpathrectangle{\pgfqpoint{0.100000in}{0.212622in}}{\pgfqpoint{3.696000in}{3.696000in}}%
\pgfusepath{clip}%
\pgfsetrectcap%
\pgfsetroundjoin%
\pgfsetlinewidth{1.505625pt}%
\definecolor{currentstroke}{rgb}{1.000000,0.000000,0.000000}%
\pgfsetstrokecolor{currentstroke}%
\pgfsetdash{}{0pt}%
\pgfpathmoveto{\pgfqpoint{2.567954in}{2.323266in}}%
\pgfpathlineto{\pgfqpoint{2.551202in}{1.825364in}}%
\pgfusepath{stroke}%
\end{pgfscope}%
\begin{pgfscope}%
\pgfpathrectangle{\pgfqpoint{0.100000in}{0.212622in}}{\pgfqpoint{3.696000in}{3.696000in}}%
\pgfusepath{clip}%
\pgfsetrectcap%
\pgfsetroundjoin%
\pgfsetlinewidth{1.505625pt}%
\definecolor{currentstroke}{rgb}{1.000000,0.000000,0.000000}%
\pgfsetstrokecolor{currentstroke}%
\pgfsetdash{}{0pt}%
\pgfpathmoveto{\pgfqpoint{2.583359in}{2.321428in}}%
\pgfpathlineto{\pgfqpoint{2.578711in}{1.817338in}}%
\pgfusepath{stroke}%
\end{pgfscope}%
\begin{pgfscope}%
\pgfpathrectangle{\pgfqpoint{0.100000in}{0.212622in}}{\pgfqpoint{3.696000in}{3.696000in}}%
\pgfusepath{clip}%
\pgfsetrectcap%
\pgfsetroundjoin%
\pgfsetlinewidth{1.505625pt}%
\definecolor{currentstroke}{rgb}{1.000000,0.000000,0.000000}%
\pgfsetstrokecolor{currentstroke}%
\pgfsetdash{}{0pt}%
\pgfpathmoveto{\pgfqpoint{2.602977in}{2.322608in}}%
\pgfpathlineto{\pgfqpoint{2.592479in}{1.813321in}}%
\pgfusepath{stroke}%
\end{pgfscope}%
\begin{pgfscope}%
\pgfpathrectangle{\pgfqpoint{0.100000in}{0.212622in}}{\pgfqpoint{3.696000in}{3.696000in}}%
\pgfusepath{clip}%
\pgfsetrectcap%
\pgfsetroundjoin%
\pgfsetlinewidth{1.505625pt}%
\definecolor{currentstroke}{rgb}{1.000000,0.000000,0.000000}%
\pgfsetstrokecolor{currentstroke}%
\pgfsetdash{}{0pt}%
\pgfpathmoveto{\pgfqpoint{2.613995in}{2.322554in}}%
\pgfpathlineto{\pgfqpoint{2.606257in}{1.809301in}}%
\pgfusepath{stroke}%
\end{pgfscope}%
\begin{pgfscope}%
\pgfpathrectangle{\pgfqpoint{0.100000in}{0.212622in}}{\pgfqpoint{3.696000in}{3.696000in}}%
\pgfusepath{clip}%
\pgfsetrectcap%
\pgfsetroundjoin%
\pgfsetlinewidth{1.505625pt}%
\definecolor{currentstroke}{rgb}{1.000000,0.000000,0.000000}%
\pgfsetstrokecolor{currentstroke}%
\pgfsetdash{}{0pt}%
\pgfpathmoveto{\pgfqpoint{2.620426in}{2.321983in}}%
\pgfpathlineto{\pgfqpoint{2.606257in}{1.809301in}}%
\pgfusepath{stroke}%
\end{pgfscope}%
\begin{pgfscope}%
\pgfpathrectangle{\pgfqpoint{0.100000in}{0.212622in}}{\pgfqpoint{3.696000in}{3.696000in}}%
\pgfusepath{clip}%
\pgfsetrectcap%
\pgfsetroundjoin%
\pgfsetlinewidth{1.505625pt}%
\definecolor{currentstroke}{rgb}{1.000000,0.000000,0.000000}%
\pgfsetstrokecolor{currentstroke}%
\pgfsetdash{}{0pt}%
\pgfpathmoveto{\pgfqpoint{2.628931in}{2.323740in}}%
\pgfpathlineto{\pgfqpoint{2.620044in}{1.805278in}}%
\pgfusepath{stroke}%
\end{pgfscope}%
\begin{pgfscope}%
\pgfpathrectangle{\pgfqpoint{0.100000in}{0.212622in}}{\pgfqpoint{3.696000in}{3.696000in}}%
\pgfusepath{clip}%
\pgfsetrectcap%
\pgfsetroundjoin%
\pgfsetlinewidth{1.505625pt}%
\definecolor{currentstroke}{rgb}{1.000000,0.000000,0.000000}%
\pgfsetstrokecolor{currentstroke}%
\pgfsetdash{}{0pt}%
\pgfpathmoveto{\pgfqpoint{2.643425in}{2.321297in}}%
\pgfpathlineto{\pgfqpoint{2.633841in}{1.801252in}}%
\pgfusepath{stroke}%
\end{pgfscope}%
\begin{pgfscope}%
\pgfpathrectangle{\pgfqpoint{0.100000in}{0.212622in}}{\pgfqpoint{3.696000in}{3.696000in}}%
\pgfusepath{clip}%
\pgfsetrectcap%
\pgfsetroundjoin%
\pgfsetlinewidth{1.505625pt}%
\definecolor{currentstroke}{rgb}{1.000000,0.000000,0.000000}%
\pgfsetstrokecolor{currentstroke}%
\pgfsetdash{}{0pt}%
\pgfpathmoveto{\pgfqpoint{2.660963in}{2.324575in}}%
\pgfpathlineto{\pgfqpoint{2.647647in}{1.797224in}}%
\pgfusepath{stroke}%
\end{pgfscope}%
\begin{pgfscope}%
\pgfpathrectangle{\pgfqpoint{0.100000in}{0.212622in}}{\pgfqpoint{3.696000in}{3.696000in}}%
\pgfusepath{clip}%
\pgfsetrectcap%
\pgfsetroundjoin%
\pgfsetlinewidth{1.505625pt}%
\definecolor{currentstroke}{rgb}{1.000000,0.000000,0.000000}%
\pgfsetstrokecolor{currentstroke}%
\pgfsetdash{}{0pt}%
\pgfpathmoveto{\pgfqpoint{2.680969in}{2.320377in}}%
\pgfpathlineto{\pgfqpoint{2.675287in}{1.789160in}}%
\pgfusepath{stroke}%
\end{pgfscope}%
\begin{pgfscope}%
\pgfpathrectangle{\pgfqpoint{0.100000in}{0.212622in}}{\pgfqpoint{3.696000in}{3.696000in}}%
\pgfusepath{clip}%
\pgfsetrectcap%
\pgfsetroundjoin%
\pgfsetlinewidth{1.505625pt}%
\definecolor{currentstroke}{rgb}{1.000000,0.000000,0.000000}%
\pgfsetstrokecolor{currentstroke}%
\pgfsetdash{}{0pt}%
\pgfpathmoveto{\pgfqpoint{2.693582in}{2.323380in}}%
\pgfpathlineto{\pgfqpoint{2.689122in}{1.785123in}}%
\pgfusepath{stroke}%
\end{pgfscope}%
\begin{pgfscope}%
\pgfpathrectangle{\pgfqpoint{0.100000in}{0.212622in}}{\pgfqpoint{3.696000in}{3.696000in}}%
\pgfusepath{clip}%
\pgfsetrectcap%
\pgfsetroundjoin%
\pgfsetlinewidth{1.505625pt}%
\definecolor{currentstroke}{rgb}{1.000000,0.000000,0.000000}%
\pgfsetstrokecolor{currentstroke}%
\pgfsetdash{}{0pt}%
\pgfpathmoveto{\pgfqpoint{2.707530in}{2.318415in}}%
\pgfpathlineto{\pgfqpoint{2.702965in}{1.781084in}}%
\pgfusepath{stroke}%
\end{pgfscope}%
\begin{pgfscope}%
\pgfpathrectangle{\pgfqpoint{0.100000in}{0.212622in}}{\pgfqpoint{3.696000in}{3.696000in}}%
\pgfusepath{clip}%
\pgfsetrectcap%
\pgfsetroundjoin%
\pgfsetlinewidth{1.505625pt}%
\definecolor{currentstroke}{rgb}{1.000000,0.000000,0.000000}%
\pgfsetstrokecolor{currentstroke}%
\pgfsetdash{}{0pt}%
\pgfpathmoveto{\pgfqpoint{2.726375in}{2.322237in}}%
\pgfpathlineto{\pgfqpoint{2.716819in}{1.777042in}}%
\pgfusepath{stroke}%
\end{pgfscope}%
\begin{pgfscope}%
\pgfpathrectangle{\pgfqpoint{0.100000in}{0.212622in}}{\pgfqpoint{3.696000in}{3.696000in}}%
\pgfusepath{clip}%
\pgfsetrectcap%
\pgfsetroundjoin%
\pgfsetlinewidth{1.505625pt}%
\definecolor{currentstroke}{rgb}{1.000000,0.000000,0.000000}%
\pgfsetstrokecolor{currentstroke}%
\pgfsetdash{}{0pt}%
\pgfpathmoveto{\pgfqpoint{2.735648in}{2.319942in}}%
\pgfpathlineto{\pgfqpoint{2.730681in}{1.772997in}}%
\pgfusepath{stroke}%
\end{pgfscope}%
\begin{pgfscope}%
\pgfpathrectangle{\pgfqpoint{0.100000in}{0.212622in}}{\pgfqpoint{3.696000in}{3.696000in}}%
\pgfusepath{clip}%
\pgfsetrectcap%
\pgfsetroundjoin%
\pgfsetlinewidth{1.505625pt}%
\definecolor{currentstroke}{rgb}{1.000000,0.000000,0.000000}%
\pgfsetstrokecolor{currentstroke}%
\pgfsetdash{}{0pt}%
\pgfpathmoveto{\pgfqpoint{2.750327in}{2.316736in}}%
\pgfpathlineto{\pgfqpoint{2.744553in}{1.768950in}}%
\pgfusepath{stroke}%
\end{pgfscope}%
\begin{pgfscope}%
\pgfpathrectangle{\pgfqpoint{0.100000in}{0.212622in}}{\pgfqpoint{3.696000in}{3.696000in}}%
\pgfusepath{clip}%
\pgfsetrectcap%
\pgfsetroundjoin%
\pgfsetlinewidth{1.505625pt}%
\definecolor{currentstroke}{rgb}{1.000000,0.000000,0.000000}%
\pgfsetstrokecolor{currentstroke}%
\pgfsetdash{}{0pt}%
\pgfpathmoveto{\pgfqpoint{2.759416in}{2.319709in}}%
\pgfpathlineto{\pgfqpoint{2.758435in}{1.764899in}}%
\pgfusepath{stroke}%
\end{pgfscope}%
\begin{pgfscope}%
\pgfpathrectangle{\pgfqpoint{0.100000in}{0.212622in}}{\pgfqpoint{3.696000in}{3.696000in}}%
\pgfusepath{clip}%
\pgfsetrectcap%
\pgfsetroundjoin%
\pgfsetlinewidth{1.505625pt}%
\definecolor{currentstroke}{rgb}{1.000000,0.000000,0.000000}%
\pgfsetstrokecolor{currentstroke}%
\pgfsetdash{}{0pt}%
\pgfpathmoveto{\pgfqpoint{2.763765in}{2.318552in}}%
\pgfpathlineto{\pgfqpoint{2.758435in}{1.764899in}}%
\pgfusepath{stroke}%
\end{pgfscope}%
\begin{pgfscope}%
\pgfpathrectangle{\pgfqpoint{0.100000in}{0.212622in}}{\pgfqpoint{3.696000in}{3.696000in}}%
\pgfusepath{clip}%
\pgfsetrectcap%
\pgfsetroundjoin%
\pgfsetlinewidth{1.505625pt}%
\definecolor{currentstroke}{rgb}{1.000000,0.000000,0.000000}%
\pgfsetstrokecolor{currentstroke}%
\pgfsetdash{}{0pt}%
\pgfpathmoveto{\pgfqpoint{2.770632in}{2.321077in}}%
\pgfpathlineto{\pgfqpoint{2.758435in}{1.764899in}}%
\pgfusepath{stroke}%
\end{pgfscope}%
\begin{pgfscope}%
\pgfpathrectangle{\pgfqpoint{0.100000in}{0.212622in}}{\pgfqpoint{3.696000in}{3.696000in}}%
\pgfusepath{clip}%
\pgfsetrectcap%
\pgfsetroundjoin%
\pgfsetlinewidth{1.505625pt}%
\definecolor{currentstroke}{rgb}{1.000000,0.000000,0.000000}%
\pgfsetstrokecolor{currentstroke}%
\pgfsetdash{}{0pt}%
\pgfpathmoveto{\pgfqpoint{2.778375in}{2.317259in}}%
\pgfpathlineto{\pgfqpoint{2.772326in}{1.760846in}}%
\pgfusepath{stroke}%
\end{pgfscope}%
\begin{pgfscope}%
\pgfpathrectangle{\pgfqpoint{0.100000in}{0.212622in}}{\pgfqpoint{3.696000in}{3.696000in}}%
\pgfusepath{clip}%
\pgfsetrectcap%
\pgfsetroundjoin%
\pgfsetlinewidth{1.505625pt}%
\definecolor{currentstroke}{rgb}{1.000000,0.000000,0.000000}%
\pgfsetstrokecolor{currentstroke}%
\pgfsetdash{}{0pt}%
\pgfpathmoveto{\pgfqpoint{2.791109in}{2.319552in}}%
\pgfpathlineto{\pgfqpoint{2.786226in}{1.756791in}}%
\pgfusepath{stroke}%
\end{pgfscope}%
\begin{pgfscope}%
\pgfpathrectangle{\pgfqpoint{0.100000in}{0.212622in}}{\pgfqpoint{3.696000in}{3.696000in}}%
\pgfusepath{clip}%
\pgfsetrectcap%
\pgfsetroundjoin%
\pgfsetlinewidth{1.505625pt}%
\definecolor{currentstroke}{rgb}{1.000000,0.000000,0.000000}%
\pgfsetstrokecolor{currentstroke}%
\pgfsetdash{}{0pt}%
\pgfpathmoveto{\pgfqpoint{2.797493in}{2.317899in}}%
\pgfpathlineto{\pgfqpoint{2.786226in}{1.756791in}}%
\pgfusepath{stroke}%
\end{pgfscope}%
\begin{pgfscope}%
\pgfpathrectangle{\pgfqpoint{0.100000in}{0.212622in}}{\pgfqpoint{3.696000in}{3.696000in}}%
\pgfusepath{clip}%
\pgfsetrectcap%
\pgfsetroundjoin%
\pgfsetlinewidth{1.505625pt}%
\definecolor{currentstroke}{rgb}{1.000000,0.000000,0.000000}%
\pgfsetstrokecolor{currentstroke}%
\pgfsetdash{}{0pt}%
\pgfpathmoveto{\pgfqpoint{2.801364in}{2.318725in}}%
\pgfpathlineto{\pgfqpoint{2.800136in}{1.752732in}}%
\pgfusepath{stroke}%
\end{pgfscope}%
\begin{pgfscope}%
\pgfpathrectangle{\pgfqpoint{0.100000in}{0.212622in}}{\pgfqpoint{3.696000in}{3.696000in}}%
\pgfusepath{clip}%
\pgfsetrectcap%
\pgfsetroundjoin%
\pgfsetlinewidth{1.505625pt}%
\definecolor{currentstroke}{rgb}{1.000000,0.000000,0.000000}%
\pgfsetstrokecolor{currentstroke}%
\pgfsetdash{}{0pt}%
\pgfpathmoveto{\pgfqpoint{2.803199in}{2.318179in}}%
\pgfpathlineto{\pgfqpoint{2.800136in}{1.752732in}}%
\pgfusepath{stroke}%
\end{pgfscope}%
\begin{pgfscope}%
\pgfpathrectangle{\pgfqpoint{0.100000in}{0.212622in}}{\pgfqpoint{3.696000in}{3.696000in}}%
\pgfusepath{clip}%
\pgfsetrectcap%
\pgfsetroundjoin%
\pgfsetlinewidth{1.505625pt}%
\definecolor{currentstroke}{rgb}{1.000000,0.000000,0.000000}%
\pgfsetstrokecolor{currentstroke}%
\pgfsetdash{}{0pt}%
\pgfpathmoveto{\pgfqpoint{2.810371in}{2.318337in}}%
\pgfpathlineto{\pgfqpoint{2.800136in}{1.752732in}}%
\pgfusepath{stroke}%
\end{pgfscope}%
\begin{pgfscope}%
\pgfpathrectangle{\pgfqpoint{0.100000in}{0.212622in}}{\pgfqpoint{3.696000in}{3.696000in}}%
\pgfusepath{clip}%
\pgfsetrectcap%
\pgfsetroundjoin%
\pgfsetlinewidth{1.505625pt}%
\definecolor{currentstroke}{rgb}{1.000000,0.000000,0.000000}%
\pgfsetstrokecolor{currentstroke}%
\pgfsetdash{}{0pt}%
\pgfpathmoveto{\pgfqpoint{2.823270in}{2.318659in}}%
\pgfpathlineto{\pgfqpoint{2.814055in}{1.748671in}}%
\pgfusepath{stroke}%
\end{pgfscope}%
\begin{pgfscope}%
\pgfpathrectangle{\pgfqpoint{0.100000in}{0.212622in}}{\pgfqpoint{3.696000in}{3.696000in}}%
\pgfusepath{clip}%
\pgfsetrectcap%
\pgfsetroundjoin%
\pgfsetlinewidth{1.505625pt}%
\definecolor{currentstroke}{rgb}{1.000000,0.000000,0.000000}%
\pgfsetstrokecolor{currentstroke}%
\pgfsetdash{}{0pt}%
\pgfpathmoveto{\pgfqpoint{2.842238in}{2.314966in}}%
\pgfpathlineto{\pgfqpoint{2.841923in}{1.740540in}}%
\pgfusepath{stroke}%
\end{pgfscope}%
\begin{pgfscope}%
\pgfpathrectangle{\pgfqpoint{0.100000in}{0.212622in}}{\pgfqpoint{3.696000in}{3.696000in}}%
\pgfusepath{clip}%
\pgfsetrectcap%
\pgfsetroundjoin%
\pgfsetlinewidth{1.505625pt}%
\definecolor{currentstroke}{rgb}{1.000000,0.000000,0.000000}%
\pgfsetstrokecolor{currentstroke}%
\pgfsetdash{}{0pt}%
\pgfpathmoveto{\pgfqpoint{2.864753in}{2.317627in}}%
\pgfpathlineto{\pgfqpoint{2.855871in}{1.736470in}}%
\pgfusepath{stroke}%
\end{pgfscope}%
\begin{pgfscope}%
\pgfpathrectangle{\pgfqpoint{0.100000in}{0.212622in}}{\pgfqpoint{3.696000in}{3.696000in}}%
\pgfusepath{clip}%
\pgfsetrectcap%
\pgfsetroundjoin%
\pgfsetlinewidth{1.505625pt}%
\definecolor{currentstroke}{rgb}{1.000000,0.000000,0.000000}%
\pgfsetstrokecolor{currentstroke}%
\pgfsetdash{}{0pt}%
\pgfpathmoveto{\pgfqpoint{2.876255in}{2.315960in}}%
\pgfpathlineto{\pgfqpoint{2.869828in}{1.732398in}}%
\pgfusepath{stroke}%
\end{pgfscope}%
\begin{pgfscope}%
\pgfpathrectangle{\pgfqpoint{0.100000in}{0.212622in}}{\pgfqpoint{3.696000in}{3.696000in}}%
\pgfusepath{clip}%
\pgfsetrectcap%
\pgfsetroundjoin%
\pgfsetlinewidth{1.505625pt}%
\definecolor{currentstroke}{rgb}{1.000000,0.000000,0.000000}%
\pgfsetstrokecolor{currentstroke}%
\pgfsetdash{}{0pt}%
\pgfpathmoveto{\pgfqpoint{2.893089in}{2.319673in}}%
\pgfpathlineto{\pgfqpoint{2.883795in}{1.728323in}}%
\pgfusepath{stroke}%
\end{pgfscope}%
\begin{pgfscope}%
\pgfpathrectangle{\pgfqpoint{0.100000in}{0.212622in}}{\pgfqpoint{3.696000in}{3.696000in}}%
\pgfusepath{clip}%
\pgfsetrectcap%
\pgfsetroundjoin%
\pgfsetlinewidth{1.505625pt}%
\definecolor{currentstroke}{rgb}{1.000000,0.000000,0.000000}%
\pgfsetstrokecolor{currentstroke}%
\pgfsetdash{}{0pt}%
\pgfpathmoveto{\pgfqpoint{2.901661in}{2.317732in}}%
\pgfpathlineto{\pgfqpoint{2.897772in}{1.724245in}}%
\pgfusepath{stroke}%
\end{pgfscope}%
\begin{pgfscope}%
\pgfpathrectangle{\pgfqpoint{0.100000in}{0.212622in}}{\pgfqpoint{3.696000in}{3.696000in}}%
\pgfusepath{clip}%
\pgfsetrectcap%
\pgfsetroundjoin%
\pgfsetlinewidth{1.505625pt}%
\definecolor{currentstroke}{rgb}{1.000000,0.000000,0.000000}%
\pgfsetstrokecolor{currentstroke}%
\pgfsetdash{}{0pt}%
\pgfpathmoveto{\pgfqpoint{2.915327in}{2.320604in}}%
\pgfpathlineto{\pgfqpoint{2.911758in}{1.720164in}}%
\pgfusepath{stroke}%
\end{pgfscope}%
\begin{pgfscope}%
\pgfpathrectangle{\pgfqpoint{0.100000in}{0.212622in}}{\pgfqpoint{3.696000in}{3.696000in}}%
\pgfusepath{clip}%
\pgfsetrectcap%
\pgfsetroundjoin%
\pgfsetlinewidth{1.505625pt}%
\definecolor{currentstroke}{rgb}{1.000000,0.000000,0.000000}%
\pgfsetstrokecolor{currentstroke}%
\pgfsetdash{}{0pt}%
\pgfpathmoveto{\pgfqpoint{2.930813in}{2.316505in}}%
\pgfpathlineto{\pgfqpoint{2.925754in}{1.716080in}}%
\pgfusepath{stroke}%
\end{pgfscope}%
\begin{pgfscope}%
\pgfpathrectangle{\pgfqpoint{0.100000in}{0.212622in}}{\pgfqpoint{3.696000in}{3.696000in}}%
\pgfusepath{clip}%
\pgfsetrectcap%
\pgfsetroundjoin%
\pgfsetlinewidth{1.505625pt}%
\definecolor{currentstroke}{rgb}{1.000000,0.000000,0.000000}%
\pgfsetstrokecolor{currentstroke}%
\pgfsetdash{}{0pt}%
\pgfpathmoveto{\pgfqpoint{2.939837in}{2.315458in}}%
\pgfpathlineto{\pgfqpoint{2.939759in}{1.711994in}}%
\pgfusepath{stroke}%
\end{pgfscope}%
\begin{pgfscope}%
\pgfpathrectangle{\pgfqpoint{0.100000in}{0.212622in}}{\pgfqpoint{3.696000in}{3.696000in}}%
\pgfusepath{clip}%
\pgfsetrectcap%
\pgfsetroundjoin%
\pgfsetlinewidth{1.505625pt}%
\definecolor{currentstroke}{rgb}{1.000000,0.000000,0.000000}%
\pgfsetstrokecolor{currentstroke}%
\pgfsetdash{}{0pt}%
\pgfpathmoveto{\pgfqpoint{2.953835in}{2.315934in}}%
\pgfpathlineto{\pgfqpoint{2.953774in}{1.707905in}}%
\pgfusepath{stroke}%
\end{pgfscope}%
\begin{pgfscope}%
\pgfpathrectangle{\pgfqpoint{0.100000in}{0.212622in}}{\pgfqpoint{3.696000in}{3.696000in}}%
\pgfusepath{clip}%
\pgfsetrectcap%
\pgfsetroundjoin%
\pgfsetlinewidth{1.505625pt}%
\definecolor{currentstroke}{rgb}{1.000000,0.000000,0.000000}%
\pgfsetstrokecolor{currentstroke}%
\pgfsetdash{}{0pt}%
\pgfpathmoveto{\pgfqpoint{2.961323in}{2.315159in}}%
\pgfpathlineto{\pgfqpoint{2.953774in}{1.707905in}}%
\pgfusepath{stroke}%
\end{pgfscope}%
\begin{pgfscope}%
\pgfpathrectangle{\pgfqpoint{0.100000in}{0.212622in}}{\pgfqpoint{3.696000in}{3.696000in}}%
\pgfusepath{clip}%
\pgfsetrectcap%
\pgfsetroundjoin%
\pgfsetlinewidth{1.505625pt}%
\definecolor{currentstroke}{rgb}{1.000000,0.000000,0.000000}%
\pgfsetstrokecolor{currentstroke}%
\pgfsetdash{}{0pt}%
\pgfpathmoveto{\pgfqpoint{2.972486in}{2.317223in}}%
\pgfpathlineto{\pgfqpoint{2.967799in}{1.703813in}}%
\pgfusepath{stroke}%
\end{pgfscope}%
\begin{pgfscope}%
\pgfpathrectangle{\pgfqpoint{0.100000in}{0.212622in}}{\pgfqpoint{3.696000in}{3.696000in}}%
\pgfusepath{clip}%
\pgfsetrectcap%
\pgfsetroundjoin%
\pgfsetlinewidth{1.505625pt}%
\definecolor{currentstroke}{rgb}{1.000000,0.000000,0.000000}%
\pgfsetstrokecolor{currentstroke}%
\pgfsetdash{}{0pt}%
\pgfpathmoveto{\pgfqpoint{2.989859in}{2.314265in}}%
\pgfpathlineto{\pgfqpoint{2.981833in}{1.699718in}}%
\pgfusepath{stroke}%
\end{pgfscope}%
\begin{pgfscope}%
\pgfpathrectangle{\pgfqpoint{0.100000in}{0.212622in}}{\pgfqpoint{3.696000in}{3.696000in}}%
\pgfusepath{clip}%
\pgfsetrectcap%
\pgfsetroundjoin%
\pgfsetlinewidth{1.505625pt}%
\definecolor{currentstroke}{rgb}{1.000000,0.000000,0.000000}%
\pgfsetstrokecolor{currentstroke}%
\pgfsetdash{}{0pt}%
\pgfpathmoveto{\pgfqpoint{3.009801in}{2.315248in}}%
\pgfpathlineto{\pgfqpoint{3.009930in}{1.691520in}}%
\pgfusepath{stroke}%
\end{pgfscope}%
\begin{pgfscope}%
\pgfpathrectangle{\pgfqpoint{0.100000in}{0.212622in}}{\pgfqpoint{3.696000in}{3.696000in}}%
\pgfusepath{clip}%
\pgfsetrectcap%
\pgfsetroundjoin%
\pgfsetlinewidth{1.505625pt}%
\definecolor{currentstroke}{rgb}{1.000000,0.000000,0.000000}%
\pgfsetstrokecolor{currentstroke}%
\pgfsetdash{}{0pt}%
\pgfpathmoveto{\pgfqpoint{3.032017in}{2.311989in}}%
\pgfpathlineto{\pgfqpoint{3.023993in}{1.687417in}}%
\pgfusepath{stroke}%
\end{pgfscope}%
\begin{pgfscope}%
\pgfpathrectangle{\pgfqpoint{0.100000in}{0.212622in}}{\pgfqpoint{3.696000in}{3.696000in}}%
\pgfusepath{clip}%
\pgfsetrectcap%
\pgfsetroundjoin%
\pgfsetlinewidth{1.505625pt}%
\definecolor{currentstroke}{rgb}{1.000000,0.000000,0.000000}%
\pgfsetstrokecolor{currentstroke}%
\pgfsetdash{}{0pt}%
\pgfpathmoveto{\pgfqpoint{3.060012in}{2.315919in}}%
\pgfpathlineto{\pgfqpoint{3.052149in}{1.679202in}}%
\pgfusepath{stroke}%
\end{pgfscope}%
\begin{pgfscope}%
\pgfpathrectangle{\pgfqpoint{0.100000in}{0.212622in}}{\pgfqpoint{3.696000in}{3.696000in}}%
\pgfusepath{clip}%
\pgfsetrectcap%
\pgfsetroundjoin%
\pgfsetlinewidth{1.505625pt}%
\definecolor{currentstroke}{rgb}{1.000000,0.000000,0.000000}%
\pgfsetstrokecolor{currentstroke}%
\pgfsetdash{}{0pt}%
\pgfpathmoveto{\pgfqpoint{3.088619in}{2.309748in}}%
\pgfpathlineto{\pgfqpoint{3.080343in}{1.670975in}}%
\pgfusepath{stroke}%
\end{pgfscope}%
\begin{pgfscope}%
\pgfpathrectangle{\pgfqpoint{0.100000in}{0.212622in}}{\pgfqpoint{3.696000in}{3.696000in}}%
\pgfusepath{clip}%
\pgfsetrectcap%
\pgfsetroundjoin%
\pgfsetlinewidth{1.505625pt}%
\definecolor{currentstroke}{rgb}{1.000000,0.000000,0.000000}%
\pgfsetstrokecolor{currentstroke}%
\pgfsetdash{}{0pt}%
\pgfpathmoveto{\pgfqpoint{3.126861in}{2.306567in}}%
\pgfpathlineto{\pgfqpoint{3.127682in}{1.627373in}}%
\pgfusepath{stroke}%
\end{pgfscope}%
\begin{pgfscope}%
\pgfpathrectangle{\pgfqpoint{0.100000in}{0.212622in}}{\pgfqpoint{3.696000in}{3.696000in}}%
\pgfusepath{clip}%
\pgfsetrectcap%
\pgfsetroundjoin%
\pgfsetlinewidth{1.505625pt}%
\definecolor{currentstroke}{rgb}{1.000000,0.000000,0.000000}%
\pgfsetstrokecolor{currentstroke}%
\pgfsetdash{}{0pt}%
\pgfpathmoveto{\pgfqpoint{3.166746in}{2.306824in}}%
\pgfpathlineto{\pgfqpoint{3.127682in}{1.627373in}}%
\pgfusepath{stroke}%
\end{pgfscope}%
\begin{pgfscope}%
\pgfpathrectangle{\pgfqpoint{0.100000in}{0.212622in}}{\pgfqpoint{3.696000in}{3.696000in}}%
\pgfusepath{clip}%
\pgfsetrectcap%
\pgfsetroundjoin%
\pgfsetlinewidth{1.505625pt}%
\definecolor{currentstroke}{rgb}{1.000000,0.000000,0.000000}%
\pgfsetstrokecolor{currentstroke}%
\pgfsetdash{}{0pt}%
\pgfpathmoveto{\pgfqpoint{3.210538in}{2.307332in}}%
\pgfpathlineto{\pgfqpoint{3.127682in}{1.627373in}}%
\pgfusepath{stroke}%
\end{pgfscope}%
\begin{pgfscope}%
\pgfpathrectangle{\pgfqpoint{0.100000in}{0.212622in}}{\pgfqpoint{3.696000in}{3.696000in}}%
\pgfusepath{clip}%
\pgfsetrectcap%
\pgfsetroundjoin%
\pgfsetlinewidth{1.505625pt}%
\definecolor{currentstroke}{rgb}{1.000000,0.000000,0.000000}%
\pgfsetstrokecolor{currentstroke}%
\pgfsetdash{}{0pt}%
\pgfpathmoveto{\pgfqpoint{3.233608in}{2.304526in}}%
\pgfpathlineto{\pgfqpoint{3.127682in}{1.627373in}}%
\pgfusepath{stroke}%
\end{pgfscope}%
\begin{pgfscope}%
\pgfpathrectangle{\pgfqpoint{0.100000in}{0.212622in}}{\pgfqpoint{3.696000in}{3.696000in}}%
\pgfusepath{clip}%
\pgfsetrectcap%
\pgfsetroundjoin%
\pgfsetlinewidth{1.505625pt}%
\definecolor{currentstroke}{rgb}{1.000000,0.000000,0.000000}%
\pgfsetstrokecolor{currentstroke}%
\pgfsetdash{}{0pt}%
\pgfpathmoveto{\pgfqpoint{3.258866in}{2.302542in}}%
\pgfpathlineto{\pgfqpoint{3.119891in}{1.619691in}}%
\pgfusepath{stroke}%
\end{pgfscope}%
\begin{pgfscope}%
\pgfpathrectangle{\pgfqpoint{0.100000in}{0.212622in}}{\pgfqpoint{3.696000in}{3.696000in}}%
\pgfusepath{clip}%
\pgfsetrectcap%
\pgfsetroundjoin%
\pgfsetlinewidth{1.505625pt}%
\definecolor{currentstroke}{rgb}{1.000000,0.000000,0.000000}%
\pgfsetstrokecolor{currentstroke}%
\pgfsetdash{}{0pt}%
\pgfpathmoveto{\pgfqpoint{3.273190in}{2.302308in}}%
\pgfpathlineto{\pgfqpoint{3.119891in}{1.619691in}}%
\pgfusepath{stroke}%
\end{pgfscope}%
\begin{pgfscope}%
\pgfpathrectangle{\pgfqpoint{0.100000in}{0.212622in}}{\pgfqpoint{3.696000in}{3.696000in}}%
\pgfusepath{clip}%
\pgfsetrectcap%
\pgfsetroundjoin%
\pgfsetlinewidth{1.505625pt}%
\definecolor{currentstroke}{rgb}{1.000000,0.000000,0.000000}%
\pgfsetstrokecolor{currentstroke}%
\pgfsetdash{}{0pt}%
\pgfpathmoveto{\pgfqpoint{3.291242in}{2.298587in}}%
\pgfpathlineto{\pgfqpoint{3.119891in}{1.619691in}}%
\pgfusepath{stroke}%
\end{pgfscope}%
\begin{pgfscope}%
\pgfpathrectangle{\pgfqpoint{0.100000in}{0.212622in}}{\pgfqpoint{3.696000in}{3.696000in}}%
\pgfusepath{clip}%
\pgfsetrectcap%
\pgfsetroundjoin%
\pgfsetlinewidth{1.505625pt}%
\definecolor{currentstroke}{rgb}{1.000000,0.000000,0.000000}%
\pgfsetstrokecolor{currentstroke}%
\pgfsetdash{}{0pt}%
\pgfpathmoveto{\pgfqpoint{3.299873in}{2.296551in}}%
\pgfpathlineto{\pgfqpoint{3.119891in}{1.619691in}}%
\pgfusepath{stroke}%
\end{pgfscope}%
\begin{pgfscope}%
\pgfpathrectangle{\pgfqpoint{0.100000in}{0.212622in}}{\pgfqpoint{3.696000in}{3.696000in}}%
\pgfusepath{clip}%
\pgfsetrectcap%
\pgfsetroundjoin%
\pgfsetlinewidth{1.505625pt}%
\definecolor{currentstroke}{rgb}{1.000000,0.000000,0.000000}%
\pgfsetstrokecolor{currentstroke}%
\pgfsetdash{}{0pt}%
\pgfpathmoveto{\pgfqpoint{3.302797in}{2.292263in}}%
\pgfpathlineto{\pgfqpoint{3.119891in}{1.619691in}}%
\pgfusepath{stroke}%
\end{pgfscope}%
\begin{pgfscope}%
\pgfpathrectangle{\pgfqpoint{0.100000in}{0.212622in}}{\pgfqpoint{3.696000in}{3.696000in}}%
\pgfusepath{clip}%
\pgfsetrectcap%
\pgfsetroundjoin%
\pgfsetlinewidth{1.505625pt}%
\definecolor{currentstroke}{rgb}{1.000000,0.000000,0.000000}%
\pgfsetstrokecolor{currentstroke}%
\pgfsetdash{}{0pt}%
\pgfpathmoveto{\pgfqpoint{3.306089in}{2.287826in}}%
\pgfpathlineto{\pgfqpoint{3.112089in}{1.612000in}}%
\pgfusepath{stroke}%
\end{pgfscope}%
\begin{pgfscope}%
\pgfpathrectangle{\pgfqpoint{0.100000in}{0.212622in}}{\pgfqpoint{3.696000in}{3.696000in}}%
\pgfusepath{clip}%
\pgfsetrectcap%
\pgfsetroundjoin%
\pgfsetlinewidth{1.505625pt}%
\definecolor{currentstroke}{rgb}{1.000000,0.000000,0.000000}%
\pgfsetstrokecolor{currentstroke}%
\pgfsetdash{}{0pt}%
\pgfpathmoveto{\pgfqpoint{3.306915in}{2.283615in}}%
\pgfpathlineto{\pgfqpoint{3.112089in}{1.612000in}}%
\pgfusepath{stroke}%
\end{pgfscope}%
\begin{pgfscope}%
\pgfpathrectangle{\pgfqpoint{0.100000in}{0.212622in}}{\pgfqpoint{3.696000in}{3.696000in}}%
\pgfusepath{clip}%
\pgfsetrectcap%
\pgfsetroundjoin%
\pgfsetlinewidth{1.505625pt}%
\definecolor{currentstroke}{rgb}{1.000000,0.000000,0.000000}%
\pgfsetstrokecolor{currentstroke}%
\pgfsetdash{}{0pt}%
\pgfpathmoveto{\pgfqpoint{3.305075in}{2.282414in}}%
\pgfpathlineto{\pgfqpoint{3.104278in}{1.604298in}}%
\pgfusepath{stroke}%
\end{pgfscope}%
\begin{pgfscope}%
\pgfpathrectangle{\pgfqpoint{0.100000in}{0.212622in}}{\pgfqpoint{3.696000in}{3.696000in}}%
\pgfusepath{clip}%
\pgfsetrectcap%
\pgfsetroundjoin%
\pgfsetlinewidth{1.505625pt}%
\definecolor{currentstroke}{rgb}{1.000000,0.000000,0.000000}%
\pgfsetstrokecolor{currentstroke}%
\pgfsetdash{}{0pt}%
\pgfpathmoveto{\pgfqpoint{3.300833in}{2.279814in}}%
\pgfpathlineto{\pgfqpoint{3.104278in}{1.604298in}}%
\pgfusepath{stroke}%
\end{pgfscope}%
\begin{pgfscope}%
\pgfpathrectangle{\pgfqpoint{0.100000in}{0.212622in}}{\pgfqpoint{3.696000in}{3.696000in}}%
\pgfusepath{clip}%
\pgfsetrectcap%
\pgfsetroundjoin%
\pgfsetlinewidth{1.505625pt}%
\definecolor{currentstroke}{rgb}{1.000000,0.000000,0.000000}%
\pgfsetstrokecolor{currentstroke}%
\pgfsetdash{}{0pt}%
\pgfpathmoveto{\pgfqpoint{3.295651in}{2.277738in}}%
\pgfpathlineto{\pgfqpoint{3.096456in}{1.596587in}}%
\pgfusepath{stroke}%
\end{pgfscope}%
\begin{pgfscope}%
\pgfpathrectangle{\pgfqpoint{0.100000in}{0.212622in}}{\pgfqpoint{3.696000in}{3.696000in}}%
\pgfusepath{clip}%
\pgfsetrectcap%
\pgfsetroundjoin%
\pgfsetlinewidth{1.505625pt}%
\definecolor{currentstroke}{rgb}{1.000000,0.000000,0.000000}%
\pgfsetstrokecolor{currentstroke}%
\pgfsetdash{}{0pt}%
\pgfpathmoveto{\pgfqpoint{3.287113in}{2.274365in}}%
\pgfpathlineto{\pgfqpoint{3.088625in}{1.588865in}}%
\pgfusepath{stroke}%
\end{pgfscope}%
\begin{pgfscope}%
\pgfpathrectangle{\pgfqpoint{0.100000in}{0.212622in}}{\pgfqpoint{3.696000in}{3.696000in}}%
\pgfusepath{clip}%
\pgfsetrectcap%
\pgfsetroundjoin%
\pgfsetlinewidth{1.505625pt}%
\definecolor{currentstroke}{rgb}{1.000000,0.000000,0.000000}%
\pgfsetstrokecolor{currentstroke}%
\pgfsetdash{}{0pt}%
\pgfpathmoveto{\pgfqpoint{3.277214in}{2.270541in}}%
\pgfpathlineto{\pgfqpoint{3.080783in}{1.581134in}}%
\pgfusepath{stroke}%
\end{pgfscope}%
\begin{pgfscope}%
\pgfpathrectangle{\pgfqpoint{0.100000in}{0.212622in}}{\pgfqpoint{3.696000in}{3.696000in}}%
\pgfusepath{clip}%
\pgfsetrectcap%
\pgfsetroundjoin%
\pgfsetlinewidth{1.505625pt}%
\definecolor{currentstroke}{rgb}{1.000000,0.000000,0.000000}%
\pgfsetstrokecolor{currentstroke}%
\pgfsetdash{}{0pt}%
\pgfpathmoveto{\pgfqpoint{3.271036in}{2.267848in}}%
\pgfpathlineto{\pgfqpoint{3.072932in}{1.573393in}}%
\pgfusepath{stroke}%
\end{pgfscope}%
\begin{pgfscope}%
\pgfpathrectangle{\pgfqpoint{0.100000in}{0.212622in}}{\pgfqpoint{3.696000in}{3.696000in}}%
\pgfusepath{clip}%
\pgfsetrectcap%
\pgfsetroundjoin%
\pgfsetlinewidth{1.505625pt}%
\definecolor{currentstroke}{rgb}{1.000000,0.000000,0.000000}%
\pgfsetstrokecolor{currentstroke}%
\pgfsetdash{}{0pt}%
\pgfpathmoveto{\pgfqpoint{3.267853in}{2.266918in}}%
\pgfpathlineto{\pgfqpoint{3.072932in}{1.573393in}}%
\pgfusepath{stroke}%
\end{pgfscope}%
\begin{pgfscope}%
\pgfpathrectangle{\pgfqpoint{0.100000in}{0.212622in}}{\pgfqpoint{3.696000in}{3.696000in}}%
\pgfusepath{clip}%
\pgfsetrectcap%
\pgfsetroundjoin%
\pgfsetlinewidth{1.505625pt}%
\definecolor{currentstroke}{rgb}{1.000000,0.000000,0.000000}%
\pgfsetstrokecolor{currentstroke}%
\pgfsetdash{}{0pt}%
\pgfpathmoveto{\pgfqpoint{3.263091in}{2.265854in}}%
\pgfpathlineto{\pgfqpoint{3.065070in}{1.565642in}}%
\pgfusepath{stroke}%
\end{pgfscope}%
\begin{pgfscope}%
\pgfpathrectangle{\pgfqpoint{0.100000in}{0.212622in}}{\pgfqpoint{3.696000in}{3.696000in}}%
\pgfusepath{clip}%
\pgfsetrectcap%
\pgfsetroundjoin%
\pgfsetlinewidth{1.505625pt}%
\definecolor{currentstroke}{rgb}{1.000000,0.000000,0.000000}%
\pgfsetstrokecolor{currentstroke}%
\pgfsetdash{}{0pt}%
\pgfpathmoveto{\pgfqpoint{3.260143in}{2.264675in}}%
\pgfpathlineto{\pgfqpoint{3.065070in}{1.565642in}}%
\pgfusepath{stroke}%
\end{pgfscope}%
\begin{pgfscope}%
\pgfpathrectangle{\pgfqpoint{0.100000in}{0.212622in}}{\pgfqpoint{3.696000in}{3.696000in}}%
\pgfusepath{clip}%
\pgfsetrectcap%
\pgfsetroundjoin%
\pgfsetlinewidth{1.505625pt}%
\definecolor{currentstroke}{rgb}{1.000000,0.000000,0.000000}%
\pgfsetstrokecolor{currentstroke}%
\pgfsetdash{}{0pt}%
\pgfpathmoveto{\pgfqpoint{3.255978in}{2.262394in}}%
\pgfpathlineto{\pgfqpoint{3.057198in}{1.557881in}}%
\pgfusepath{stroke}%
\end{pgfscope}%
\begin{pgfscope}%
\pgfpathrectangle{\pgfqpoint{0.100000in}{0.212622in}}{\pgfqpoint{3.696000in}{3.696000in}}%
\pgfusepath{clip}%
\pgfsetrectcap%
\pgfsetroundjoin%
\pgfsetlinewidth{1.505625pt}%
\definecolor{currentstroke}{rgb}{1.000000,0.000000,0.000000}%
\pgfsetstrokecolor{currentstroke}%
\pgfsetdash{}{0pt}%
\pgfpathmoveto{\pgfqpoint{3.247332in}{2.259555in}}%
\pgfpathlineto{\pgfqpoint{3.057198in}{1.557881in}}%
\pgfusepath{stroke}%
\end{pgfscope}%
\begin{pgfscope}%
\pgfpathrectangle{\pgfqpoint{0.100000in}{0.212622in}}{\pgfqpoint{3.696000in}{3.696000in}}%
\pgfusepath{clip}%
\pgfsetrectcap%
\pgfsetroundjoin%
\pgfsetlinewidth{1.505625pt}%
\definecolor{currentstroke}{rgb}{1.000000,0.000000,0.000000}%
\pgfsetstrokecolor{currentstroke}%
\pgfsetdash{}{0pt}%
\pgfpathmoveto{\pgfqpoint{3.243650in}{2.258372in}}%
\pgfpathlineto{\pgfqpoint{3.049316in}{1.550110in}}%
\pgfusepath{stroke}%
\end{pgfscope}%
\begin{pgfscope}%
\pgfpathrectangle{\pgfqpoint{0.100000in}{0.212622in}}{\pgfqpoint{3.696000in}{3.696000in}}%
\pgfusepath{clip}%
\pgfsetrectcap%
\pgfsetroundjoin%
\pgfsetlinewidth{1.505625pt}%
\definecolor{currentstroke}{rgb}{1.000000,0.000000,0.000000}%
\pgfsetstrokecolor{currentstroke}%
\pgfsetdash{}{0pt}%
\pgfpathmoveto{\pgfqpoint{3.241034in}{2.257506in}}%
\pgfpathlineto{\pgfqpoint{3.049316in}{1.550110in}}%
\pgfusepath{stroke}%
\end{pgfscope}%
\begin{pgfscope}%
\pgfpathrectangle{\pgfqpoint{0.100000in}{0.212622in}}{\pgfqpoint{3.696000in}{3.696000in}}%
\pgfusepath{clip}%
\pgfsetrectcap%
\pgfsetroundjoin%
\pgfsetlinewidth{1.505625pt}%
\definecolor{currentstroke}{rgb}{1.000000,0.000000,0.000000}%
\pgfsetstrokecolor{currentstroke}%
\pgfsetdash{}{0pt}%
\pgfpathmoveto{\pgfqpoint{3.239863in}{2.257294in}}%
\pgfpathlineto{\pgfqpoint{3.049316in}{1.550110in}}%
\pgfusepath{stroke}%
\end{pgfscope}%
\begin{pgfscope}%
\pgfpathrectangle{\pgfqpoint{0.100000in}{0.212622in}}{\pgfqpoint{3.696000in}{3.696000in}}%
\pgfusepath{clip}%
\pgfsetrectcap%
\pgfsetroundjoin%
\pgfsetlinewidth{1.505625pt}%
\definecolor{currentstroke}{rgb}{1.000000,0.000000,0.000000}%
\pgfsetstrokecolor{currentstroke}%
\pgfsetdash{}{0pt}%
\pgfpathmoveto{\pgfqpoint{3.239092in}{2.257001in}}%
\pgfpathlineto{\pgfqpoint{3.049316in}{1.550110in}}%
\pgfusepath{stroke}%
\end{pgfscope}%
\begin{pgfscope}%
\pgfpathrectangle{\pgfqpoint{0.100000in}{0.212622in}}{\pgfqpoint{3.696000in}{3.696000in}}%
\pgfusepath{clip}%
\pgfsetrectcap%
\pgfsetroundjoin%
\pgfsetlinewidth{1.505625pt}%
\definecolor{currentstroke}{rgb}{1.000000,0.000000,0.000000}%
\pgfsetstrokecolor{currentstroke}%
\pgfsetdash{}{0pt}%
\pgfpathmoveto{\pgfqpoint{3.238746in}{2.256893in}}%
\pgfpathlineto{\pgfqpoint{3.049316in}{1.550110in}}%
\pgfusepath{stroke}%
\end{pgfscope}%
\begin{pgfscope}%
\pgfpathrectangle{\pgfqpoint{0.100000in}{0.212622in}}{\pgfqpoint{3.696000in}{3.696000in}}%
\pgfusepath{clip}%
\pgfsetrectcap%
\pgfsetroundjoin%
\pgfsetlinewidth{1.505625pt}%
\definecolor{currentstroke}{rgb}{1.000000,0.000000,0.000000}%
\pgfsetstrokecolor{currentstroke}%
\pgfsetdash{}{0pt}%
\pgfpathmoveto{\pgfqpoint{3.238584in}{2.256832in}}%
\pgfpathlineto{\pgfqpoint{3.049316in}{1.550110in}}%
\pgfusepath{stroke}%
\end{pgfscope}%
\begin{pgfscope}%
\pgfpathrectangle{\pgfqpoint{0.100000in}{0.212622in}}{\pgfqpoint{3.696000in}{3.696000in}}%
\pgfusepath{clip}%
\pgfsetrectcap%
\pgfsetroundjoin%
\pgfsetlinewidth{1.505625pt}%
\definecolor{currentstroke}{rgb}{1.000000,0.000000,0.000000}%
\pgfsetstrokecolor{currentstroke}%
\pgfsetdash{}{0pt}%
\pgfpathmoveto{\pgfqpoint{3.236871in}{2.256688in}}%
\pgfpathlineto{\pgfqpoint{3.049316in}{1.550110in}}%
\pgfusepath{stroke}%
\end{pgfscope}%
\begin{pgfscope}%
\pgfpathrectangle{\pgfqpoint{0.100000in}{0.212622in}}{\pgfqpoint{3.696000in}{3.696000in}}%
\pgfusepath{clip}%
\pgfsetrectcap%
\pgfsetroundjoin%
\pgfsetlinewidth{1.505625pt}%
\definecolor{currentstroke}{rgb}{1.000000,0.000000,0.000000}%
\pgfsetstrokecolor{currentstroke}%
\pgfsetdash{}{0pt}%
\pgfpathmoveto{\pgfqpoint{3.233596in}{2.252836in}}%
\pgfpathlineto{\pgfqpoint{3.041424in}{1.542329in}}%
\pgfusepath{stroke}%
\end{pgfscope}%
\begin{pgfscope}%
\pgfpathrectangle{\pgfqpoint{0.100000in}{0.212622in}}{\pgfqpoint{3.696000in}{3.696000in}}%
\pgfusepath{clip}%
\pgfsetrectcap%
\pgfsetroundjoin%
\pgfsetlinewidth{1.505625pt}%
\definecolor{currentstroke}{rgb}{1.000000,0.000000,0.000000}%
\pgfsetstrokecolor{currentstroke}%
\pgfsetdash{}{0pt}%
\pgfpathmoveto{\pgfqpoint{3.225840in}{2.252520in}}%
\pgfpathlineto{\pgfqpoint{3.033522in}{1.534538in}}%
\pgfusepath{stroke}%
\end{pgfscope}%
\begin{pgfscope}%
\pgfpathrectangle{\pgfqpoint{0.100000in}{0.212622in}}{\pgfqpoint{3.696000in}{3.696000in}}%
\pgfusepath{clip}%
\pgfsetrectcap%
\pgfsetroundjoin%
\pgfsetlinewidth{1.505625pt}%
\definecolor{currentstroke}{rgb}{1.000000,0.000000,0.000000}%
\pgfsetstrokecolor{currentstroke}%
\pgfsetdash{}{0pt}%
\pgfpathmoveto{\pgfqpoint{3.216928in}{2.246784in}}%
\pgfpathlineto{\pgfqpoint{3.025610in}{1.526737in}}%
\pgfusepath{stroke}%
\end{pgfscope}%
\begin{pgfscope}%
\pgfpathrectangle{\pgfqpoint{0.100000in}{0.212622in}}{\pgfqpoint{3.696000in}{3.696000in}}%
\pgfusepath{clip}%
\pgfsetrectcap%
\pgfsetroundjoin%
\pgfsetlinewidth{1.505625pt}%
\definecolor{currentstroke}{rgb}{1.000000,0.000000,0.000000}%
\pgfsetstrokecolor{currentstroke}%
\pgfsetdash{}{0pt}%
\pgfpathmoveto{\pgfqpoint{3.200978in}{2.243154in}}%
\pgfpathlineto{\pgfqpoint{3.017687in}{1.518925in}}%
\pgfusepath{stroke}%
\end{pgfscope}%
\begin{pgfscope}%
\pgfpathrectangle{\pgfqpoint{0.100000in}{0.212622in}}{\pgfqpoint{3.696000in}{3.696000in}}%
\pgfusepath{clip}%
\pgfsetrectcap%
\pgfsetroundjoin%
\pgfsetlinewidth{1.505625pt}%
\definecolor{currentstroke}{rgb}{1.000000,0.000000,0.000000}%
\pgfsetstrokecolor{currentstroke}%
\pgfsetdash{}{0pt}%
\pgfpathmoveto{\pgfqpoint{3.189546in}{2.234460in}}%
\pgfpathlineto{\pgfqpoint{3.001811in}{1.503273in}}%
\pgfusepath{stroke}%
\end{pgfscope}%
\begin{pgfscope}%
\pgfpathrectangle{\pgfqpoint{0.100000in}{0.212622in}}{\pgfqpoint{3.696000in}{3.696000in}}%
\pgfusepath{clip}%
\pgfsetrectcap%
\pgfsetroundjoin%
\pgfsetlinewidth{1.505625pt}%
\definecolor{currentstroke}{rgb}{1.000000,0.000000,0.000000}%
\pgfsetstrokecolor{currentstroke}%
\pgfsetdash{}{0pt}%
\pgfpathmoveto{\pgfqpoint{3.167423in}{2.227283in}}%
\pgfpathlineto{\pgfqpoint{2.993858in}{1.495431in}}%
\pgfusepath{stroke}%
\end{pgfscope}%
\begin{pgfscope}%
\pgfpathrectangle{\pgfqpoint{0.100000in}{0.212622in}}{\pgfqpoint{3.696000in}{3.696000in}}%
\pgfusepath{clip}%
\pgfsetrectcap%
\pgfsetroundjoin%
\pgfsetlinewidth{1.505625pt}%
\definecolor{currentstroke}{rgb}{1.000000,0.000000,0.000000}%
\pgfsetstrokecolor{currentstroke}%
\pgfsetdash{}{0pt}%
\pgfpathmoveto{\pgfqpoint{3.149094in}{2.214548in}}%
\pgfpathlineto{\pgfqpoint{2.969936in}{1.471846in}}%
\pgfusepath{stroke}%
\end{pgfscope}%
\begin{pgfscope}%
\pgfpathrectangle{\pgfqpoint{0.100000in}{0.212622in}}{\pgfqpoint{3.696000in}{3.696000in}}%
\pgfusepath{clip}%
\pgfsetrectcap%
\pgfsetroundjoin%
\pgfsetlinewidth{1.505625pt}%
\definecolor{currentstroke}{rgb}{1.000000,0.000000,0.000000}%
\pgfsetstrokecolor{currentstroke}%
\pgfsetdash{}{0pt}%
\pgfpathmoveto{\pgfqpoint{3.121159in}{2.207515in}}%
\pgfpathlineto{\pgfqpoint{2.953937in}{1.456072in}}%
\pgfusepath{stroke}%
\end{pgfscope}%
\begin{pgfscope}%
\pgfpathrectangle{\pgfqpoint{0.100000in}{0.212622in}}{\pgfqpoint{3.696000in}{3.696000in}}%
\pgfusepath{clip}%
\pgfsetrectcap%
\pgfsetroundjoin%
\pgfsetlinewidth{1.505625pt}%
\definecolor{currentstroke}{rgb}{1.000000,0.000000,0.000000}%
\pgfsetstrokecolor{currentstroke}%
\pgfsetdash{}{0pt}%
\pgfpathmoveto{\pgfqpoint{3.109794in}{2.205463in}}%
\pgfpathlineto{\pgfqpoint{2.945921in}{1.448169in}}%
\pgfusepath{stroke}%
\end{pgfscope}%
\begin{pgfscope}%
\pgfpathrectangle{\pgfqpoint{0.100000in}{0.212622in}}{\pgfqpoint{3.696000in}{3.696000in}}%
\pgfusepath{clip}%
\pgfsetrectcap%
\pgfsetroundjoin%
\pgfsetlinewidth{1.505625pt}%
\definecolor{currentstroke}{rgb}{1.000000,0.000000,0.000000}%
\pgfsetstrokecolor{currentstroke}%
\pgfsetdash{}{0pt}%
\pgfpathmoveto{\pgfqpoint{3.101195in}{2.202204in}}%
\pgfpathlineto{\pgfqpoint{2.937896in}{1.440256in}}%
\pgfusepath{stroke}%
\end{pgfscope}%
\begin{pgfscope}%
\pgfpathrectangle{\pgfqpoint{0.100000in}{0.212622in}}{\pgfqpoint{3.696000in}{3.696000in}}%
\pgfusepath{clip}%
\pgfsetrectcap%
\pgfsetroundjoin%
\pgfsetlinewidth{1.505625pt}%
\definecolor{currentstroke}{rgb}{1.000000,0.000000,0.000000}%
\pgfsetstrokecolor{currentstroke}%
\pgfsetdash{}{0pt}%
\pgfpathmoveto{\pgfqpoint{3.097933in}{2.201624in}}%
\pgfpathlineto{\pgfqpoint{2.937896in}{1.440256in}}%
\pgfusepath{stroke}%
\end{pgfscope}%
\begin{pgfscope}%
\pgfpathrectangle{\pgfqpoint{0.100000in}{0.212622in}}{\pgfqpoint{3.696000in}{3.696000in}}%
\pgfusepath{clip}%
\pgfsetrectcap%
\pgfsetroundjoin%
\pgfsetlinewidth{1.505625pt}%
\definecolor{currentstroke}{rgb}{1.000000,0.000000,0.000000}%
\pgfsetstrokecolor{currentstroke}%
\pgfsetdash{}{0pt}%
\pgfpathmoveto{\pgfqpoint{3.095579in}{2.201261in}}%
\pgfpathlineto{\pgfqpoint{2.929860in}{1.432333in}}%
\pgfusepath{stroke}%
\end{pgfscope}%
\begin{pgfscope}%
\pgfpathrectangle{\pgfqpoint{0.100000in}{0.212622in}}{\pgfqpoint{3.696000in}{3.696000in}}%
\pgfusepath{clip}%
\pgfsetrectcap%
\pgfsetroundjoin%
\pgfsetlinewidth{1.505625pt}%
\definecolor{currentstroke}{rgb}{1.000000,0.000000,0.000000}%
\pgfsetstrokecolor{currentstroke}%
\pgfsetdash{}{0pt}%
\pgfpathmoveto{\pgfqpoint{3.094279in}{2.201083in}}%
\pgfpathlineto{\pgfqpoint{2.929860in}{1.432333in}}%
\pgfusepath{stroke}%
\end{pgfscope}%
\begin{pgfscope}%
\pgfpathrectangle{\pgfqpoint{0.100000in}{0.212622in}}{\pgfqpoint{3.696000in}{3.696000in}}%
\pgfusepath{clip}%
\pgfsetrectcap%
\pgfsetroundjoin%
\pgfsetlinewidth{1.505625pt}%
\definecolor{currentstroke}{rgb}{1.000000,0.000000,0.000000}%
\pgfsetstrokecolor{currentstroke}%
\pgfsetdash{}{0pt}%
\pgfpathmoveto{\pgfqpoint{3.093702in}{2.200811in}}%
\pgfpathlineto{\pgfqpoint{2.929860in}{1.432333in}}%
\pgfusepath{stroke}%
\end{pgfscope}%
\begin{pgfscope}%
\pgfpathrectangle{\pgfqpoint{0.100000in}{0.212622in}}{\pgfqpoint{3.696000in}{3.696000in}}%
\pgfusepath{clip}%
\pgfsetrectcap%
\pgfsetroundjoin%
\pgfsetlinewidth{1.505625pt}%
\definecolor{currentstroke}{rgb}{1.000000,0.000000,0.000000}%
\pgfsetstrokecolor{currentstroke}%
\pgfsetdash{}{0pt}%
\pgfpathmoveto{\pgfqpoint{3.093276in}{2.200840in}}%
\pgfpathlineto{\pgfqpoint{2.929860in}{1.432333in}}%
\pgfusepath{stroke}%
\end{pgfscope}%
\begin{pgfscope}%
\pgfpathrectangle{\pgfqpoint{0.100000in}{0.212622in}}{\pgfqpoint{3.696000in}{3.696000in}}%
\pgfusepath{clip}%
\pgfsetrectcap%
\pgfsetroundjoin%
\pgfsetlinewidth{1.505625pt}%
\definecolor{currentstroke}{rgb}{1.000000,0.000000,0.000000}%
\pgfsetstrokecolor{currentstroke}%
\pgfsetdash{}{0pt}%
\pgfpathmoveto{\pgfqpoint{3.093123in}{2.200747in}}%
\pgfpathlineto{\pgfqpoint{2.929860in}{1.432333in}}%
\pgfusepath{stroke}%
\end{pgfscope}%
\begin{pgfscope}%
\pgfpathrectangle{\pgfqpoint{0.100000in}{0.212622in}}{\pgfqpoint{3.696000in}{3.696000in}}%
\pgfusepath{clip}%
\pgfsetrectcap%
\pgfsetroundjoin%
\pgfsetlinewidth{1.505625pt}%
\definecolor{currentstroke}{rgb}{1.000000,0.000000,0.000000}%
\pgfsetstrokecolor{currentstroke}%
\pgfsetdash{}{0pt}%
\pgfpathmoveto{\pgfqpoint{3.090362in}{2.199531in}}%
\pgfpathlineto{\pgfqpoint{2.929860in}{1.432333in}}%
\pgfusepath{stroke}%
\end{pgfscope}%
\begin{pgfscope}%
\pgfpathrectangle{\pgfqpoint{0.100000in}{0.212622in}}{\pgfqpoint{3.696000in}{3.696000in}}%
\pgfusepath{clip}%
\pgfsetrectcap%
\pgfsetroundjoin%
\pgfsetlinewidth{1.505625pt}%
\definecolor{currentstroke}{rgb}{1.000000,0.000000,0.000000}%
\pgfsetstrokecolor{currentstroke}%
\pgfsetdash{}{0pt}%
\pgfpathmoveto{\pgfqpoint{3.085849in}{2.195821in}}%
\pgfpathlineto{\pgfqpoint{2.921813in}{1.424400in}}%
\pgfusepath{stroke}%
\end{pgfscope}%
\begin{pgfscope}%
\pgfpathrectangle{\pgfqpoint{0.100000in}{0.212622in}}{\pgfqpoint{3.696000in}{3.696000in}}%
\pgfusepath{clip}%
\pgfsetrectcap%
\pgfsetroundjoin%
\pgfsetlinewidth{1.505625pt}%
\definecolor{currentstroke}{rgb}{1.000000,0.000000,0.000000}%
\pgfsetstrokecolor{currentstroke}%
\pgfsetdash{}{0pt}%
\pgfpathmoveto{\pgfqpoint{3.081963in}{2.195287in}}%
\pgfpathlineto{\pgfqpoint{2.921813in}{1.424400in}}%
\pgfusepath{stroke}%
\end{pgfscope}%
\begin{pgfscope}%
\pgfpathrectangle{\pgfqpoint{0.100000in}{0.212622in}}{\pgfqpoint{3.696000in}{3.696000in}}%
\pgfusepath{clip}%
\pgfsetrectcap%
\pgfsetroundjoin%
\pgfsetlinewidth{1.505625pt}%
\definecolor{currentstroke}{rgb}{1.000000,0.000000,0.000000}%
\pgfsetstrokecolor{currentstroke}%
\pgfsetdash{}{0pt}%
\pgfpathmoveto{\pgfqpoint{3.077137in}{2.190915in}}%
\pgfpathlineto{\pgfqpoint{2.913756in}{1.416456in}}%
\pgfusepath{stroke}%
\end{pgfscope}%
\begin{pgfscope}%
\pgfpathrectangle{\pgfqpoint{0.100000in}{0.212622in}}{\pgfqpoint{3.696000in}{3.696000in}}%
\pgfusepath{clip}%
\pgfsetrectcap%
\pgfsetroundjoin%
\pgfsetlinewidth{1.505625pt}%
\definecolor{currentstroke}{rgb}{1.000000,0.000000,0.000000}%
\pgfsetstrokecolor{currentstroke}%
\pgfsetdash{}{0pt}%
\pgfpathmoveto{\pgfqpoint{3.069423in}{2.189273in}}%
\pgfpathlineto{\pgfqpoint{2.913756in}{1.416456in}}%
\pgfusepath{stroke}%
\end{pgfscope}%
\begin{pgfscope}%
\pgfpathrectangle{\pgfqpoint{0.100000in}{0.212622in}}{\pgfqpoint{3.696000in}{3.696000in}}%
\pgfusepath{clip}%
\pgfsetrectcap%
\pgfsetroundjoin%
\pgfsetlinewidth{1.505625pt}%
\definecolor{currentstroke}{rgb}{1.000000,0.000000,0.000000}%
\pgfsetstrokecolor{currentstroke}%
\pgfsetdash{}{0pt}%
\pgfpathmoveto{\pgfqpoint{3.059546in}{2.184394in}}%
\pgfpathlineto{\pgfqpoint{2.897611in}{1.400538in}}%
\pgfusepath{stroke}%
\end{pgfscope}%
\begin{pgfscope}%
\pgfpathrectangle{\pgfqpoint{0.100000in}{0.212622in}}{\pgfqpoint{3.696000in}{3.696000in}}%
\pgfusepath{clip}%
\pgfsetrectcap%
\pgfsetroundjoin%
\pgfsetlinewidth{1.505625pt}%
\definecolor{currentstroke}{rgb}{1.000000,0.000000,0.000000}%
\pgfsetstrokecolor{currentstroke}%
\pgfsetdash{}{0pt}%
\pgfpathmoveto{\pgfqpoint{3.045755in}{2.179895in}}%
\pgfpathlineto{\pgfqpoint{2.889522in}{1.392563in}}%
\pgfusepath{stroke}%
\end{pgfscope}%
\begin{pgfscope}%
\pgfpathrectangle{\pgfqpoint{0.100000in}{0.212622in}}{\pgfqpoint{3.696000in}{3.696000in}}%
\pgfusepath{clip}%
\pgfsetrectcap%
\pgfsetroundjoin%
\pgfsetlinewidth{1.505625pt}%
\definecolor{currentstroke}{rgb}{1.000000,0.000000,0.000000}%
\pgfsetstrokecolor{currentstroke}%
\pgfsetdash{}{0pt}%
\pgfpathmoveto{\pgfqpoint{3.039477in}{2.177744in}}%
\pgfpathlineto{\pgfqpoint{2.881423in}{1.384578in}}%
\pgfusepath{stroke}%
\end{pgfscope}%
\begin{pgfscope}%
\pgfpathrectangle{\pgfqpoint{0.100000in}{0.212622in}}{\pgfqpoint{3.696000in}{3.696000in}}%
\pgfusepath{clip}%
\pgfsetrectcap%
\pgfsetroundjoin%
\pgfsetlinewidth{1.505625pt}%
\definecolor{currentstroke}{rgb}{1.000000,0.000000,0.000000}%
\pgfsetstrokecolor{currentstroke}%
\pgfsetdash{}{0pt}%
\pgfpathmoveto{\pgfqpoint{3.035990in}{2.175980in}}%
\pgfpathlineto{\pgfqpoint{2.881423in}{1.384578in}}%
\pgfusepath{stroke}%
\end{pgfscope}%
\begin{pgfscope}%
\pgfpathrectangle{\pgfqpoint{0.100000in}{0.212622in}}{\pgfqpoint{3.696000in}{3.696000in}}%
\pgfusepath{clip}%
\pgfsetrectcap%
\pgfsetroundjoin%
\pgfsetlinewidth{1.505625pt}%
\definecolor{currentstroke}{rgb}{1.000000,0.000000,0.000000}%
\pgfsetstrokecolor{currentstroke}%
\pgfsetdash{}{0pt}%
\pgfpathmoveto{\pgfqpoint{3.034087in}{2.176006in}}%
\pgfpathlineto{\pgfqpoint{2.881423in}{1.384578in}}%
\pgfusepath{stroke}%
\end{pgfscope}%
\begin{pgfscope}%
\pgfpathrectangle{\pgfqpoint{0.100000in}{0.212622in}}{\pgfqpoint{3.696000in}{3.696000in}}%
\pgfusepath{clip}%
\pgfsetrectcap%
\pgfsetroundjoin%
\pgfsetlinewidth{1.505625pt}%
\definecolor{currentstroke}{rgb}{1.000000,0.000000,0.000000}%
\pgfsetstrokecolor{currentstroke}%
\pgfsetdash{}{0pt}%
\pgfpathmoveto{\pgfqpoint{3.033106in}{2.175585in}}%
\pgfpathlineto{\pgfqpoint{2.881423in}{1.384578in}}%
\pgfusepath{stroke}%
\end{pgfscope}%
\begin{pgfscope}%
\pgfpathrectangle{\pgfqpoint{0.100000in}{0.212622in}}{\pgfqpoint{3.696000in}{3.696000in}}%
\pgfusepath{clip}%
\pgfsetrectcap%
\pgfsetroundjoin%
\pgfsetlinewidth{1.505625pt}%
\definecolor{currentstroke}{rgb}{1.000000,0.000000,0.000000}%
\pgfsetstrokecolor{currentstroke}%
\pgfsetdash{}{0pt}%
\pgfpathmoveto{\pgfqpoint{3.032435in}{2.175536in}}%
\pgfpathlineto{\pgfqpoint{2.881423in}{1.384578in}}%
\pgfusepath{stroke}%
\end{pgfscope}%
\begin{pgfscope}%
\pgfpathrectangle{\pgfqpoint{0.100000in}{0.212622in}}{\pgfqpoint{3.696000in}{3.696000in}}%
\pgfusepath{clip}%
\pgfsetrectcap%
\pgfsetroundjoin%
\pgfsetlinewidth{1.505625pt}%
\definecolor{currentstroke}{rgb}{1.000000,0.000000,0.000000}%
\pgfsetstrokecolor{currentstroke}%
\pgfsetdash{}{0pt}%
\pgfpathmoveto{\pgfqpoint{3.030421in}{2.173364in}}%
\pgfpathlineto{\pgfqpoint{2.873313in}{1.376583in}}%
\pgfusepath{stroke}%
\end{pgfscope}%
\begin{pgfscope}%
\pgfpathrectangle{\pgfqpoint{0.100000in}{0.212622in}}{\pgfqpoint{3.696000in}{3.696000in}}%
\pgfusepath{clip}%
\pgfsetrectcap%
\pgfsetroundjoin%
\pgfsetlinewidth{1.505625pt}%
\definecolor{currentstroke}{rgb}{1.000000,0.000000,0.000000}%
\pgfsetstrokecolor{currentstroke}%
\pgfsetdash{}{0pt}%
\pgfpathmoveto{\pgfqpoint{3.028801in}{2.173171in}}%
\pgfpathlineto{\pgfqpoint{2.873313in}{1.376583in}}%
\pgfusepath{stroke}%
\end{pgfscope}%
\begin{pgfscope}%
\pgfpathrectangle{\pgfqpoint{0.100000in}{0.212622in}}{\pgfqpoint{3.696000in}{3.696000in}}%
\pgfusepath{clip}%
\pgfsetrectcap%
\pgfsetroundjoin%
\pgfsetlinewidth{1.505625pt}%
\definecolor{currentstroke}{rgb}{1.000000,0.000000,0.000000}%
\pgfsetstrokecolor{currentstroke}%
\pgfsetdash{}{0pt}%
\pgfpathmoveto{\pgfqpoint{3.025859in}{2.169276in}}%
\pgfpathlineto{\pgfqpoint{2.873313in}{1.376583in}}%
\pgfusepath{stroke}%
\end{pgfscope}%
\begin{pgfscope}%
\pgfpathrectangle{\pgfqpoint{0.100000in}{0.212622in}}{\pgfqpoint{3.696000in}{3.696000in}}%
\pgfusepath{clip}%
\pgfsetrectcap%
\pgfsetroundjoin%
\pgfsetlinewidth{1.505625pt}%
\definecolor{currentstroke}{rgb}{1.000000,0.000000,0.000000}%
\pgfsetstrokecolor{currentstroke}%
\pgfsetdash{}{0pt}%
\pgfpathmoveto{\pgfqpoint{3.018903in}{2.167173in}}%
\pgfpathlineto{\pgfqpoint{2.865193in}{1.368577in}}%
\pgfusepath{stroke}%
\end{pgfscope}%
\begin{pgfscope}%
\pgfpathrectangle{\pgfqpoint{0.100000in}{0.212622in}}{\pgfqpoint{3.696000in}{3.696000in}}%
\pgfusepath{clip}%
\pgfsetrectcap%
\pgfsetroundjoin%
\pgfsetlinewidth{1.505625pt}%
\definecolor{currentstroke}{rgb}{1.000000,0.000000,0.000000}%
\pgfsetstrokecolor{currentstroke}%
\pgfsetdash{}{0pt}%
\pgfpathmoveto{\pgfqpoint{3.012668in}{2.160572in}}%
\pgfpathlineto{\pgfqpoint{2.857062in}{1.360560in}}%
\pgfusepath{stroke}%
\end{pgfscope}%
\begin{pgfscope}%
\pgfpathrectangle{\pgfqpoint{0.100000in}{0.212622in}}{\pgfqpoint{3.696000in}{3.696000in}}%
\pgfusepath{clip}%
\pgfsetrectcap%
\pgfsetroundjoin%
\pgfsetlinewidth{1.505625pt}%
\definecolor{currentstroke}{rgb}{1.000000,0.000000,0.000000}%
\pgfsetstrokecolor{currentstroke}%
\pgfsetdash{}{0pt}%
\pgfpathmoveto{\pgfqpoint{2.999911in}{2.156279in}}%
\pgfpathlineto{\pgfqpoint{2.848921in}{1.352533in}}%
\pgfusepath{stroke}%
\end{pgfscope}%
\begin{pgfscope}%
\pgfpathrectangle{\pgfqpoint{0.100000in}{0.212622in}}{\pgfqpoint{3.696000in}{3.696000in}}%
\pgfusepath{clip}%
\pgfsetrectcap%
\pgfsetroundjoin%
\pgfsetlinewidth{1.505625pt}%
\definecolor{currentstroke}{rgb}{1.000000,0.000000,0.000000}%
\pgfsetstrokecolor{currentstroke}%
\pgfsetdash{}{0pt}%
\pgfpathmoveto{\pgfqpoint{2.990124in}{2.149960in}}%
\pgfpathlineto{\pgfqpoint{2.840769in}{1.344496in}}%
\pgfusepath{stroke}%
\end{pgfscope}%
\begin{pgfscope}%
\pgfpathrectangle{\pgfqpoint{0.100000in}{0.212622in}}{\pgfqpoint{3.696000in}{3.696000in}}%
\pgfusepath{clip}%
\pgfsetrectcap%
\pgfsetroundjoin%
\pgfsetlinewidth{1.505625pt}%
\definecolor{currentstroke}{rgb}{1.000000,0.000000,0.000000}%
\pgfsetstrokecolor{currentstroke}%
\pgfsetdash{}{0pt}%
\pgfpathmoveto{\pgfqpoint{2.982595in}{2.146379in}}%
\pgfpathlineto{\pgfqpoint{2.832606in}{1.336448in}}%
\pgfusepath{stroke}%
\end{pgfscope}%
\begin{pgfscope}%
\pgfpathrectangle{\pgfqpoint{0.100000in}{0.212622in}}{\pgfqpoint{3.696000in}{3.696000in}}%
\pgfusepath{clip}%
\pgfsetrectcap%
\pgfsetroundjoin%
\pgfsetlinewidth{1.505625pt}%
\definecolor{currentstroke}{rgb}{1.000000,0.000000,0.000000}%
\pgfsetstrokecolor{currentstroke}%
\pgfsetdash{}{0pt}%
\pgfpathmoveto{\pgfqpoint{2.979083in}{2.145306in}}%
\pgfpathlineto{\pgfqpoint{2.832606in}{1.336448in}}%
\pgfusepath{stroke}%
\end{pgfscope}%
\begin{pgfscope}%
\pgfpathrectangle{\pgfqpoint{0.100000in}{0.212622in}}{\pgfqpoint{3.696000in}{3.696000in}}%
\pgfusepath{clip}%
\pgfsetrectcap%
\pgfsetroundjoin%
\pgfsetlinewidth{1.505625pt}%
\definecolor{currentstroke}{rgb}{1.000000,0.000000,0.000000}%
\pgfsetstrokecolor{currentstroke}%
\pgfsetdash{}{0pt}%
\pgfpathmoveto{\pgfqpoint{2.977162in}{2.144702in}}%
\pgfpathlineto{\pgfqpoint{2.832606in}{1.336448in}}%
\pgfusepath{stroke}%
\end{pgfscope}%
\begin{pgfscope}%
\pgfpathrectangle{\pgfqpoint{0.100000in}{0.212622in}}{\pgfqpoint{3.696000in}{3.696000in}}%
\pgfusepath{clip}%
\pgfsetrectcap%
\pgfsetroundjoin%
\pgfsetlinewidth{1.505625pt}%
\definecolor{currentstroke}{rgb}{1.000000,0.000000,0.000000}%
\pgfsetstrokecolor{currentstroke}%
\pgfsetdash{}{0pt}%
\pgfpathmoveto{\pgfqpoint{2.975953in}{2.144492in}}%
\pgfpathlineto{\pgfqpoint{2.824433in}{1.328389in}}%
\pgfusepath{stroke}%
\end{pgfscope}%
\begin{pgfscope}%
\pgfpathrectangle{\pgfqpoint{0.100000in}{0.212622in}}{\pgfqpoint{3.696000in}{3.696000in}}%
\pgfusepath{clip}%
\pgfsetrectcap%
\pgfsetroundjoin%
\pgfsetlinewidth{1.505625pt}%
\definecolor{currentstroke}{rgb}{1.000000,0.000000,0.000000}%
\pgfsetstrokecolor{currentstroke}%
\pgfsetdash{}{0pt}%
\pgfpathmoveto{\pgfqpoint{2.973921in}{2.142537in}}%
\pgfpathlineto{\pgfqpoint{2.824433in}{1.328389in}}%
\pgfusepath{stroke}%
\end{pgfscope}%
\begin{pgfscope}%
\pgfpathrectangle{\pgfqpoint{0.100000in}{0.212622in}}{\pgfqpoint{3.696000in}{3.696000in}}%
\pgfusepath{clip}%
\pgfsetrectcap%
\pgfsetroundjoin%
\pgfsetlinewidth{1.505625pt}%
\definecolor{currentstroke}{rgb}{1.000000,0.000000,0.000000}%
\pgfsetstrokecolor{currentstroke}%
\pgfsetdash{}{0pt}%
\pgfpathmoveto{\pgfqpoint{2.969857in}{2.141447in}}%
\pgfpathlineto{\pgfqpoint{2.824433in}{1.328389in}}%
\pgfusepath{stroke}%
\end{pgfscope}%
\begin{pgfscope}%
\pgfpathrectangle{\pgfqpoint{0.100000in}{0.212622in}}{\pgfqpoint{3.696000in}{3.696000in}}%
\pgfusepath{clip}%
\pgfsetrectcap%
\pgfsetroundjoin%
\pgfsetlinewidth{1.505625pt}%
\definecolor{currentstroke}{rgb}{1.000000,0.000000,0.000000}%
\pgfsetstrokecolor{currentstroke}%
\pgfsetdash{}{0pt}%
\pgfpathmoveto{\pgfqpoint{2.964766in}{2.136056in}}%
\pgfpathlineto{\pgfqpoint{2.816248in}{1.320320in}}%
\pgfusepath{stroke}%
\end{pgfscope}%
\begin{pgfscope}%
\pgfpathrectangle{\pgfqpoint{0.100000in}{0.212622in}}{\pgfqpoint{3.696000in}{3.696000in}}%
\pgfusepath{clip}%
\pgfsetrectcap%
\pgfsetroundjoin%
\pgfsetlinewidth{1.505625pt}%
\definecolor{currentstroke}{rgb}{1.000000,0.000000,0.000000}%
\pgfsetstrokecolor{currentstroke}%
\pgfsetdash{}{0pt}%
\pgfpathmoveto{\pgfqpoint{2.960517in}{2.134572in}}%
\pgfpathlineto{\pgfqpoint{2.816248in}{1.320320in}}%
\pgfusepath{stroke}%
\end{pgfscope}%
\begin{pgfscope}%
\pgfpathrectangle{\pgfqpoint{0.100000in}{0.212622in}}{\pgfqpoint{3.696000in}{3.696000in}}%
\pgfusepath{clip}%
\pgfsetrectcap%
\pgfsetroundjoin%
\pgfsetlinewidth{1.505625pt}%
\definecolor{currentstroke}{rgb}{1.000000,0.000000,0.000000}%
\pgfsetstrokecolor{currentstroke}%
\pgfsetdash{}{0pt}%
\pgfpathmoveto{\pgfqpoint{2.955825in}{2.132297in}}%
\pgfpathlineto{\pgfqpoint{2.808054in}{1.312241in}}%
\pgfusepath{stroke}%
\end{pgfscope}%
\begin{pgfscope}%
\pgfpathrectangle{\pgfqpoint{0.100000in}{0.212622in}}{\pgfqpoint{3.696000in}{3.696000in}}%
\pgfusepath{clip}%
\pgfsetrectcap%
\pgfsetroundjoin%
\pgfsetlinewidth{1.505625pt}%
\definecolor{currentstroke}{rgb}{1.000000,0.000000,0.000000}%
\pgfsetstrokecolor{currentstroke}%
\pgfsetdash{}{0pt}%
\pgfpathmoveto{\pgfqpoint{2.952598in}{2.131068in}}%
\pgfpathlineto{\pgfqpoint{2.808054in}{1.312241in}}%
\pgfusepath{stroke}%
\end{pgfscope}%
\begin{pgfscope}%
\pgfpathrectangle{\pgfqpoint{0.100000in}{0.212622in}}{\pgfqpoint{3.696000in}{3.696000in}}%
\pgfusepath{clip}%
\pgfsetrectcap%
\pgfsetroundjoin%
\pgfsetlinewidth{1.505625pt}%
\definecolor{currentstroke}{rgb}{1.000000,0.000000,0.000000}%
\pgfsetstrokecolor{currentstroke}%
\pgfsetdash{}{0pt}%
\pgfpathmoveto{\pgfqpoint{2.951073in}{2.130025in}}%
\pgfpathlineto{\pgfqpoint{2.808054in}{1.312241in}}%
\pgfusepath{stroke}%
\end{pgfscope}%
\begin{pgfscope}%
\pgfpathrectangle{\pgfqpoint{0.100000in}{0.212622in}}{\pgfqpoint{3.696000in}{3.696000in}}%
\pgfusepath{clip}%
\pgfsetrectcap%
\pgfsetroundjoin%
\pgfsetlinewidth{1.505625pt}%
\definecolor{currentstroke}{rgb}{1.000000,0.000000,0.000000}%
\pgfsetstrokecolor{currentstroke}%
\pgfsetdash{}{0pt}%
\pgfpathmoveto{\pgfqpoint{2.950075in}{2.129876in}}%
\pgfpathlineto{\pgfqpoint{2.799848in}{1.304151in}}%
\pgfusepath{stroke}%
\end{pgfscope}%
\begin{pgfscope}%
\pgfpathrectangle{\pgfqpoint{0.100000in}{0.212622in}}{\pgfqpoint{3.696000in}{3.696000in}}%
\pgfusepath{clip}%
\pgfsetrectcap%
\pgfsetroundjoin%
\pgfsetlinewidth{1.505625pt}%
\definecolor{currentstroke}{rgb}{1.000000,0.000000,0.000000}%
\pgfsetstrokecolor{currentstroke}%
\pgfsetdash{}{0pt}%
\pgfpathmoveto{\pgfqpoint{2.949520in}{2.129666in}}%
\pgfpathlineto{\pgfqpoint{2.799848in}{1.304151in}}%
\pgfusepath{stroke}%
\end{pgfscope}%
\begin{pgfscope}%
\pgfpathrectangle{\pgfqpoint{0.100000in}{0.212622in}}{\pgfqpoint{3.696000in}{3.696000in}}%
\pgfusepath{clip}%
\pgfsetrectcap%
\pgfsetroundjoin%
\pgfsetlinewidth{1.505625pt}%
\definecolor{currentstroke}{rgb}{1.000000,0.000000,0.000000}%
\pgfsetstrokecolor{currentstroke}%
\pgfsetdash{}{0pt}%
\pgfpathmoveto{\pgfqpoint{2.949287in}{2.129603in}}%
\pgfpathlineto{\pgfqpoint{2.799848in}{1.304151in}}%
\pgfusepath{stroke}%
\end{pgfscope}%
\begin{pgfscope}%
\pgfpathrectangle{\pgfqpoint{0.100000in}{0.212622in}}{\pgfqpoint{3.696000in}{3.696000in}}%
\pgfusepath{clip}%
\pgfsetrectcap%
\pgfsetroundjoin%
\pgfsetlinewidth{1.505625pt}%
\definecolor{currentstroke}{rgb}{1.000000,0.000000,0.000000}%
\pgfsetstrokecolor{currentstroke}%
\pgfsetdash{}{0pt}%
\pgfpathmoveto{\pgfqpoint{2.947326in}{2.129049in}}%
\pgfpathlineto{\pgfqpoint{2.799848in}{1.304151in}}%
\pgfusepath{stroke}%
\end{pgfscope}%
\begin{pgfscope}%
\pgfpathrectangle{\pgfqpoint{0.100000in}{0.212622in}}{\pgfqpoint{3.696000in}{3.696000in}}%
\pgfusepath{clip}%
\pgfsetrectcap%
\pgfsetroundjoin%
\pgfsetlinewidth{1.505625pt}%
\definecolor{currentstroke}{rgb}{1.000000,0.000000,0.000000}%
\pgfsetstrokecolor{currentstroke}%
\pgfsetdash{}{0pt}%
\pgfpathmoveto{\pgfqpoint{2.944452in}{2.126125in}}%
\pgfpathlineto{\pgfqpoint{2.799848in}{1.304151in}}%
\pgfusepath{stroke}%
\end{pgfscope}%
\begin{pgfscope}%
\pgfpathrectangle{\pgfqpoint{0.100000in}{0.212622in}}{\pgfqpoint{3.696000in}{3.696000in}}%
\pgfusepath{clip}%
\pgfsetrectcap%
\pgfsetroundjoin%
\pgfsetlinewidth{1.505625pt}%
\definecolor{currentstroke}{rgb}{1.000000,0.000000,0.000000}%
\pgfsetstrokecolor{currentstroke}%
\pgfsetdash{}{0pt}%
\pgfpathmoveto{\pgfqpoint{2.936809in}{2.123666in}}%
\pgfpathlineto{\pgfqpoint{2.791632in}{1.296050in}}%
\pgfusepath{stroke}%
\end{pgfscope}%
\begin{pgfscope}%
\pgfpathrectangle{\pgfqpoint{0.100000in}{0.212622in}}{\pgfqpoint{3.696000in}{3.696000in}}%
\pgfusepath{clip}%
\pgfsetrectcap%
\pgfsetroundjoin%
\pgfsetlinewidth{1.505625pt}%
\definecolor{currentstroke}{rgb}{1.000000,0.000000,0.000000}%
\pgfsetstrokecolor{currentstroke}%
\pgfsetdash{}{0pt}%
\pgfpathmoveto{\pgfqpoint{2.928235in}{2.118249in}}%
\pgfpathlineto{\pgfqpoint{2.783404in}{1.287938in}}%
\pgfusepath{stroke}%
\end{pgfscope}%
\begin{pgfscope}%
\pgfpathrectangle{\pgfqpoint{0.100000in}{0.212622in}}{\pgfqpoint{3.696000in}{3.696000in}}%
\pgfusepath{clip}%
\pgfsetrectcap%
\pgfsetroundjoin%
\pgfsetlinewidth{1.505625pt}%
\definecolor{currentstroke}{rgb}{1.000000,0.000000,0.000000}%
\pgfsetstrokecolor{currentstroke}%
\pgfsetdash{}{0pt}%
\pgfpathmoveto{\pgfqpoint{2.913346in}{2.113265in}}%
\pgfpathlineto{\pgfqpoint{2.775166in}{1.279816in}}%
\pgfusepath{stroke}%
\end{pgfscope}%
\begin{pgfscope}%
\pgfpathrectangle{\pgfqpoint{0.100000in}{0.212622in}}{\pgfqpoint{3.696000in}{3.696000in}}%
\pgfusepath{clip}%
\pgfsetrectcap%
\pgfsetroundjoin%
\pgfsetlinewidth{1.505625pt}%
\definecolor{currentstroke}{rgb}{1.000000,0.000000,0.000000}%
\pgfsetstrokecolor{currentstroke}%
\pgfsetdash{}{0pt}%
\pgfpathmoveto{\pgfqpoint{2.900326in}{2.105090in}}%
\pgfpathlineto{\pgfqpoint{2.758658in}{1.263540in}}%
\pgfusepath{stroke}%
\end{pgfscope}%
\begin{pgfscope}%
\pgfpathrectangle{\pgfqpoint{0.100000in}{0.212622in}}{\pgfqpoint{3.696000in}{3.696000in}}%
\pgfusepath{clip}%
\pgfsetrectcap%
\pgfsetroundjoin%
\pgfsetlinewidth{1.505625pt}%
\definecolor{currentstroke}{rgb}{1.000000,0.000000,0.000000}%
\pgfsetstrokecolor{currentstroke}%
\pgfsetdash{}{0pt}%
\pgfpathmoveto{\pgfqpoint{2.880711in}{2.098071in}}%
\pgfpathlineto{\pgfqpoint{2.742106in}{1.247221in}}%
\pgfusepath{stroke}%
\end{pgfscope}%
\begin{pgfscope}%
\pgfpathrectangle{\pgfqpoint{0.100000in}{0.212622in}}{\pgfqpoint{3.696000in}{3.696000in}}%
\pgfusepath{clip}%
\pgfsetrectcap%
\pgfsetroundjoin%
\pgfsetlinewidth{1.505625pt}%
\definecolor{currentstroke}{rgb}{1.000000,0.000000,0.000000}%
\pgfsetstrokecolor{currentstroke}%
\pgfsetdash{}{0pt}%
\pgfpathmoveto{\pgfqpoint{2.872468in}{2.093807in}}%
\pgfpathlineto{\pgfqpoint{2.733813in}{1.239045in}}%
\pgfusepath{stroke}%
\end{pgfscope}%
\begin{pgfscope}%
\pgfpathrectangle{\pgfqpoint{0.100000in}{0.212622in}}{\pgfqpoint{3.696000in}{3.696000in}}%
\pgfusepath{clip}%
\pgfsetrectcap%
\pgfsetroundjoin%
\pgfsetlinewidth{1.505625pt}%
\definecolor{currentstroke}{rgb}{1.000000,0.000000,0.000000}%
\pgfsetstrokecolor{currentstroke}%
\pgfsetdash{}{0pt}%
\pgfpathmoveto{\pgfqpoint{2.867300in}{2.093163in}}%
\pgfpathlineto{\pgfqpoint{2.733813in}{1.239045in}}%
\pgfusepath{stroke}%
\end{pgfscope}%
\begin{pgfscope}%
\pgfpathrectangle{\pgfqpoint{0.100000in}{0.212622in}}{\pgfqpoint{3.696000in}{3.696000in}}%
\pgfusepath{clip}%
\pgfsetrectcap%
\pgfsetroundjoin%
\pgfsetlinewidth{1.505625pt}%
\definecolor{currentstroke}{rgb}{1.000000,0.000000,0.000000}%
\pgfsetstrokecolor{currentstroke}%
\pgfsetdash{}{0pt}%
\pgfpathmoveto{\pgfqpoint{2.864104in}{2.092542in}}%
\pgfpathlineto{\pgfqpoint{2.733813in}{1.239045in}}%
\pgfusepath{stroke}%
\end{pgfscope}%
\begin{pgfscope}%
\pgfpathrectangle{\pgfqpoint{0.100000in}{0.212622in}}{\pgfqpoint{3.696000in}{3.696000in}}%
\pgfusepath{clip}%
\pgfsetrectcap%
\pgfsetroundjoin%
\pgfsetlinewidth{1.505625pt}%
\definecolor{currentstroke}{rgb}{1.000000,0.000000,0.000000}%
\pgfsetstrokecolor{currentstroke}%
\pgfsetdash{}{0pt}%
\pgfpathmoveto{\pgfqpoint{2.862894in}{2.091500in}}%
\pgfpathlineto{\pgfqpoint{2.733813in}{1.239045in}}%
\pgfusepath{stroke}%
\end{pgfscope}%
\begin{pgfscope}%
\pgfpathrectangle{\pgfqpoint{0.100000in}{0.212622in}}{\pgfqpoint{3.696000in}{3.696000in}}%
\pgfusepath{clip}%
\pgfsetrectcap%
\pgfsetroundjoin%
\pgfsetlinewidth{1.505625pt}%
\definecolor{currentstroke}{rgb}{1.000000,0.000000,0.000000}%
\pgfsetstrokecolor{currentstroke}%
\pgfsetdash{}{0pt}%
\pgfpathmoveto{\pgfqpoint{2.862064in}{2.091137in}}%
\pgfpathlineto{\pgfqpoint{2.725510in}{1.230858in}}%
\pgfusepath{stroke}%
\end{pgfscope}%
\begin{pgfscope}%
\pgfpathrectangle{\pgfqpoint{0.100000in}{0.212622in}}{\pgfqpoint{3.696000in}{3.696000in}}%
\pgfusepath{clip}%
\pgfsetrectcap%
\pgfsetroundjoin%
\pgfsetlinewidth{1.505625pt}%
\definecolor{currentstroke}{rgb}{1.000000,0.000000,0.000000}%
\pgfsetstrokecolor{currentstroke}%
\pgfsetdash{}{0pt}%
\pgfpathmoveto{\pgfqpoint{2.859849in}{2.088737in}}%
\pgfpathlineto{\pgfqpoint{2.725510in}{1.230858in}}%
\pgfusepath{stroke}%
\end{pgfscope}%
\begin{pgfscope}%
\pgfpathrectangle{\pgfqpoint{0.100000in}{0.212622in}}{\pgfqpoint{3.696000in}{3.696000in}}%
\pgfusepath{clip}%
\pgfsetrectcap%
\pgfsetroundjoin%
\pgfsetlinewidth{1.505625pt}%
\definecolor{currentstroke}{rgb}{1.000000,0.000000,0.000000}%
\pgfsetstrokecolor{currentstroke}%
\pgfsetdash{}{0pt}%
\pgfpathmoveto{\pgfqpoint{2.855005in}{2.086741in}}%
\pgfpathlineto{\pgfqpoint{2.725510in}{1.230858in}}%
\pgfusepath{stroke}%
\end{pgfscope}%
\begin{pgfscope}%
\pgfpathrectangle{\pgfqpoint{0.100000in}{0.212622in}}{\pgfqpoint{3.696000in}{3.696000in}}%
\pgfusepath{clip}%
\pgfsetrectcap%
\pgfsetroundjoin%
\pgfsetlinewidth{1.505625pt}%
\definecolor{currentstroke}{rgb}{1.000000,0.000000,0.000000}%
\pgfsetstrokecolor{currentstroke}%
\pgfsetdash{}{0pt}%
\pgfpathmoveto{\pgfqpoint{2.849418in}{2.082820in}}%
\pgfpathlineto{\pgfqpoint{2.717196in}{1.222661in}}%
\pgfusepath{stroke}%
\end{pgfscope}%
\begin{pgfscope}%
\pgfpathrectangle{\pgfqpoint{0.100000in}{0.212622in}}{\pgfqpoint{3.696000in}{3.696000in}}%
\pgfusepath{clip}%
\pgfsetrectcap%
\pgfsetroundjoin%
\pgfsetlinewidth{1.505625pt}%
\definecolor{currentstroke}{rgb}{1.000000,0.000000,0.000000}%
\pgfsetstrokecolor{currentstroke}%
\pgfsetdash{}{0pt}%
\pgfpathmoveto{\pgfqpoint{2.839405in}{2.079006in}}%
\pgfpathlineto{\pgfqpoint{2.708870in}{1.214453in}}%
\pgfusepath{stroke}%
\end{pgfscope}%
\begin{pgfscope}%
\pgfpathrectangle{\pgfqpoint{0.100000in}{0.212622in}}{\pgfqpoint{3.696000in}{3.696000in}}%
\pgfusepath{clip}%
\pgfsetrectcap%
\pgfsetroundjoin%
\pgfsetlinewidth{1.505625pt}%
\definecolor{currentstroke}{rgb}{1.000000,0.000000,0.000000}%
\pgfsetstrokecolor{currentstroke}%
\pgfsetdash{}{0pt}%
\pgfpathmoveto{\pgfqpoint{2.829916in}{2.074616in}}%
\pgfpathlineto{\pgfqpoint{2.700534in}{1.206234in}}%
\pgfusepath{stroke}%
\end{pgfscope}%
\begin{pgfscope}%
\pgfpathrectangle{\pgfqpoint{0.100000in}{0.212622in}}{\pgfqpoint{3.696000in}{3.696000in}}%
\pgfusepath{clip}%
\pgfsetrectcap%
\pgfsetroundjoin%
\pgfsetlinewidth{1.505625pt}%
\definecolor{currentstroke}{rgb}{1.000000,0.000000,0.000000}%
\pgfsetstrokecolor{currentstroke}%
\pgfsetdash{}{0pt}%
\pgfpathmoveto{\pgfqpoint{2.817734in}{2.068806in}}%
\pgfpathlineto{\pgfqpoint{2.692187in}{1.198004in}}%
\pgfusepath{stroke}%
\end{pgfscope}%
\begin{pgfscope}%
\pgfpathrectangle{\pgfqpoint{0.100000in}{0.212622in}}{\pgfqpoint{3.696000in}{3.696000in}}%
\pgfusepath{clip}%
\pgfsetrectcap%
\pgfsetroundjoin%
\pgfsetlinewidth{1.505625pt}%
\definecolor{currentstroke}{rgb}{1.000000,0.000000,0.000000}%
\pgfsetstrokecolor{currentstroke}%
\pgfsetdash{}{0pt}%
\pgfpathmoveto{\pgfqpoint{2.812014in}{2.066701in}}%
\pgfpathlineto{\pgfqpoint{2.683828in}{1.189763in}}%
\pgfusepath{stroke}%
\end{pgfscope}%
\begin{pgfscope}%
\pgfpathrectangle{\pgfqpoint{0.100000in}{0.212622in}}{\pgfqpoint{3.696000in}{3.696000in}}%
\pgfusepath{clip}%
\pgfsetrectcap%
\pgfsetroundjoin%
\pgfsetlinewidth{1.505625pt}%
\definecolor{currentstroke}{rgb}{1.000000,0.000000,0.000000}%
\pgfsetstrokecolor{currentstroke}%
\pgfsetdash{}{0pt}%
\pgfpathmoveto{\pgfqpoint{2.808488in}{2.065502in}}%
\pgfpathlineto{\pgfqpoint{2.683828in}{1.189763in}}%
\pgfusepath{stroke}%
\end{pgfscope}%
\begin{pgfscope}%
\pgfpathrectangle{\pgfqpoint{0.100000in}{0.212622in}}{\pgfqpoint{3.696000in}{3.696000in}}%
\pgfusepath{clip}%
\pgfsetrectcap%
\pgfsetroundjoin%
\pgfsetlinewidth{1.505625pt}%
\definecolor{currentstroke}{rgb}{1.000000,0.000000,0.000000}%
\pgfsetstrokecolor{currentstroke}%
\pgfsetdash{}{0pt}%
\pgfpathmoveto{\pgfqpoint{2.806427in}{2.064691in}}%
\pgfpathlineto{\pgfqpoint{2.675459in}{1.181511in}}%
\pgfusepath{stroke}%
\end{pgfscope}%
\begin{pgfscope}%
\pgfpathrectangle{\pgfqpoint{0.100000in}{0.212622in}}{\pgfqpoint{3.696000in}{3.696000in}}%
\pgfusepath{clip}%
\pgfsetrectcap%
\pgfsetroundjoin%
\pgfsetlinewidth{1.505625pt}%
\definecolor{currentstroke}{rgb}{1.000000,0.000000,0.000000}%
\pgfsetstrokecolor{currentstroke}%
\pgfsetdash{}{0pt}%
\pgfpathmoveto{\pgfqpoint{2.805715in}{2.063906in}}%
\pgfpathlineto{\pgfqpoint{2.675459in}{1.181511in}}%
\pgfusepath{stroke}%
\end{pgfscope}%
\begin{pgfscope}%
\pgfpathrectangle{\pgfqpoint{0.100000in}{0.212622in}}{\pgfqpoint{3.696000in}{3.696000in}}%
\pgfusepath{clip}%
\pgfsetrectcap%
\pgfsetroundjoin%
\pgfsetlinewidth{1.505625pt}%
\definecolor{currentstroke}{rgb}{1.000000,0.000000,0.000000}%
\pgfsetstrokecolor{currentstroke}%
\pgfsetdash{}{0pt}%
\pgfpathmoveto{\pgfqpoint{2.800412in}{2.062049in}}%
\pgfpathlineto{\pgfqpoint{2.675459in}{1.181511in}}%
\pgfusepath{stroke}%
\end{pgfscope}%
\begin{pgfscope}%
\pgfpathrectangle{\pgfqpoint{0.100000in}{0.212622in}}{\pgfqpoint{3.696000in}{3.696000in}}%
\pgfusepath{clip}%
\pgfsetrectcap%
\pgfsetroundjoin%
\pgfsetlinewidth{1.505625pt}%
\definecolor{currentstroke}{rgb}{1.000000,0.000000,0.000000}%
\pgfsetstrokecolor{currentstroke}%
\pgfsetdash{}{0pt}%
\pgfpathmoveto{\pgfqpoint{2.794047in}{2.055411in}}%
\pgfpathlineto{\pgfqpoint{2.667078in}{1.173249in}}%
\pgfusepath{stroke}%
\end{pgfscope}%
\begin{pgfscope}%
\pgfpathrectangle{\pgfqpoint{0.100000in}{0.212622in}}{\pgfqpoint{3.696000in}{3.696000in}}%
\pgfusepath{clip}%
\pgfsetrectcap%
\pgfsetroundjoin%
\pgfsetlinewidth{1.505625pt}%
\definecolor{currentstroke}{rgb}{1.000000,0.000000,0.000000}%
\pgfsetstrokecolor{currentstroke}%
\pgfsetdash{}{0pt}%
\pgfpathmoveto{\pgfqpoint{2.782218in}{2.051724in}}%
\pgfpathlineto{\pgfqpoint{2.658687in}{1.164975in}}%
\pgfusepath{stroke}%
\end{pgfscope}%
\begin{pgfscope}%
\pgfpathrectangle{\pgfqpoint{0.100000in}{0.212622in}}{\pgfqpoint{3.696000in}{3.696000in}}%
\pgfusepath{clip}%
\pgfsetrectcap%
\pgfsetroundjoin%
\pgfsetlinewidth{1.505625pt}%
\definecolor{currentstroke}{rgb}{1.000000,0.000000,0.000000}%
\pgfsetstrokecolor{currentstroke}%
\pgfsetdash{}{0pt}%
\pgfpathmoveto{\pgfqpoint{2.771357in}{2.043761in}}%
\pgfpathlineto{\pgfqpoint{2.650284in}{1.156690in}}%
\pgfusepath{stroke}%
\end{pgfscope}%
\begin{pgfscope}%
\pgfpathrectangle{\pgfqpoint{0.100000in}{0.212622in}}{\pgfqpoint{3.696000in}{3.696000in}}%
\pgfusepath{clip}%
\pgfsetrectcap%
\pgfsetroundjoin%
\pgfsetlinewidth{1.505625pt}%
\definecolor{currentstroke}{rgb}{1.000000,0.000000,0.000000}%
\pgfsetstrokecolor{currentstroke}%
\pgfsetdash{}{0pt}%
\pgfpathmoveto{\pgfqpoint{2.752887in}{2.038146in}}%
\pgfpathlineto{\pgfqpoint{2.633444in}{1.140088in}}%
\pgfusepath{stroke}%
\end{pgfscope}%
\begin{pgfscope}%
\pgfpathrectangle{\pgfqpoint{0.100000in}{0.212622in}}{\pgfqpoint{3.696000in}{3.696000in}}%
\pgfusepath{clip}%
\pgfsetrectcap%
\pgfsetroundjoin%
\pgfsetlinewidth{1.505625pt}%
\definecolor{currentstroke}{rgb}{1.000000,0.000000,0.000000}%
\pgfsetstrokecolor{currentstroke}%
\pgfsetdash{}{0pt}%
\pgfpathmoveto{\pgfqpoint{2.738655in}{2.027941in}}%
\pgfpathlineto{\pgfqpoint{2.616560in}{1.123441in}}%
\pgfusepath{stroke}%
\end{pgfscope}%
\begin{pgfscope}%
\pgfpathrectangle{\pgfqpoint{0.100000in}{0.212622in}}{\pgfqpoint{3.696000in}{3.696000in}}%
\pgfusepath{clip}%
\pgfsetrectcap%
\pgfsetroundjoin%
\pgfsetlinewidth{1.505625pt}%
\definecolor{currentstroke}{rgb}{1.000000,0.000000,0.000000}%
\pgfsetstrokecolor{currentstroke}%
\pgfsetdash{}{0pt}%
\pgfpathmoveto{\pgfqpoint{2.728781in}{2.023455in}}%
\pgfpathlineto{\pgfqpoint{2.608102in}{1.115102in}}%
\pgfusepath{stroke}%
\end{pgfscope}%
\begin{pgfscope}%
\pgfpathrectangle{\pgfqpoint{0.100000in}{0.212622in}}{\pgfqpoint{3.696000in}{3.696000in}}%
\pgfusepath{clip}%
\pgfsetrectcap%
\pgfsetroundjoin%
\pgfsetlinewidth{1.505625pt}%
\definecolor{currentstroke}{rgb}{1.000000,0.000000,0.000000}%
\pgfsetstrokecolor{currentstroke}%
\pgfsetdash{}{0pt}%
\pgfpathmoveto{\pgfqpoint{2.723485in}{2.021516in}}%
\pgfpathlineto{\pgfqpoint{2.608102in}{1.115102in}}%
\pgfusepath{stroke}%
\end{pgfscope}%
\begin{pgfscope}%
\pgfpathrectangle{\pgfqpoint{0.100000in}{0.212622in}}{\pgfqpoint{3.696000in}{3.696000in}}%
\pgfusepath{clip}%
\pgfsetrectcap%
\pgfsetroundjoin%
\pgfsetlinewidth{1.505625pt}%
\definecolor{currentstroke}{rgb}{1.000000,0.000000,0.000000}%
\pgfsetstrokecolor{currentstroke}%
\pgfsetdash{}{0pt}%
\pgfpathmoveto{\pgfqpoint{2.720689in}{2.020259in}}%
\pgfpathlineto{\pgfqpoint{2.608102in}{1.115102in}}%
\pgfusepath{stroke}%
\end{pgfscope}%
\begin{pgfscope}%
\pgfpathrectangle{\pgfqpoint{0.100000in}{0.212622in}}{\pgfqpoint{3.696000in}{3.696000in}}%
\pgfusepath{clip}%
\pgfsetrectcap%
\pgfsetroundjoin%
\pgfsetlinewidth{1.505625pt}%
\definecolor{currentstroke}{rgb}{1.000000,0.000000,0.000000}%
\pgfsetstrokecolor{currentstroke}%
\pgfsetdash{}{0pt}%
\pgfpathmoveto{\pgfqpoint{2.719155in}{2.020161in}}%
\pgfpathlineto{\pgfqpoint{2.599631in}{1.106751in}}%
\pgfusepath{stroke}%
\end{pgfscope}%
\begin{pgfscope}%
\pgfpathrectangle{\pgfqpoint{0.100000in}{0.212622in}}{\pgfqpoint{3.696000in}{3.696000in}}%
\pgfusepath{clip}%
\pgfsetrectcap%
\pgfsetroundjoin%
\pgfsetlinewidth{1.505625pt}%
\definecolor{currentstroke}{rgb}{1.000000,0.000000,0.000000}%
\pgfsetstrokecolor{currentstroke}%
\pgfsetdash{}{0pt}%
\pgfpathmoveto{\pgfqpoint{2.718516in}{2.019461in}}%
\pgfpathlineto{\pgfqpoint{2.599631in}{1.106751in}}%
\pgfusepath{stroke}%
\end{pgfscope}%
\begin{pgfscope}%
\pgfpathrectangle{\pgfqpoint{0.100000in}{0.212622in}}{\pgfqpoint{3.696000in}{3.696000in}}%
\pgfusepath{clip}%
\pgfsetrectcap%
\pgfsetroundjoin%
\pgfsetlinewidth{1.505625pt}%
\definecolor{currentstroke}{rgb}{1.000000,0.000000,0.000000}%
\pgfsetstrokecolor{currentstroke}%
\pgfsetdash{}{0pt}%
\pgfpathmoveto{\pgfqpoint{2.718074in}{2.019308in}}%
\pgfpathlineto{\pgfqpoint{2.599631in}{1.106751in}}%
\pgfusepath{stroke}%
\end{pgfscope}%
\begin{pgfscope}%
\pgfpathrectangle{\pgfqpoint{0.100000in}{0.212622in}}{\pgfqpoint{3.696000in}{3.696000in}}%
\pgfusepath{clip}%
\pgfsetrectcap%
\pgfsetroundjoin%
\pgfsetlinewidth{1.505625pt}%
\definecolor{currentstroke}{rgb}{1.000000,0.000000,0.000000}%
\pgfsetstrokecolor{currentstroke}%
\pgfsetdash{}{0pt}%
\pgfpathmoveto{\pgfqpoint{2.715126in}{2.015650in}}%
\pgfpathlineto{\pgfqpoint{2.599631in}{1.106751in}}%
\pgfusepath{stroke}%
\end{pgfscope}%
\begin{pgfscope}%
\pgfpathrectangle{\pgfqpoint{0.100000in}{0.212622in}}{\pgfqpoint{3.696000in}{3.696000in}}%
\pgfusepath{clip}%
\pgfsetrectcap%
\pgfsetroundjoin%
\pgfsetlinewidth{1.505625pt}%
\definecolor{currentstroke}{rgb}{1.000000,0.000000,0.000000}%
\pgfsetstrokecolor{currentstroke}%
\pgfsetdash{}{0pt}%
\pgfpathmoveto{\pgfqpoint{2.710125in}{2.015924in}}%
\pgfpathlineto{\pgfqpoint{2.591150in}{1.098388in}}%
\pgfusepath{stroke}%
\end{pgfscope}%
\begin{pgfscope}%
\pgfpathrectangle{\pgfqpoint{0.100000in}{0.212622in}}{\pgfqpoint{3.696000in}{3.696000in}}%
\pgfusepath{clip}%
\pgfsetrectcap%
\pgfsetroundjoin%
\pgfsetlinewidth{1.505625pt}%
\definecolor{currentstroke}{rgb}{1.000000,0.000000,0.000000}%
\pgfsetstrokecolor{currentstroke}%
\pgfsetdash{}{0pt}%
\pgfpathmoveto{\pgfqpoint{2.701946in}{2.004843in}}%
\pgfpathlineto{\pgfqpoint{2.582657in}{1.090015in}}%
\pgfusepath{stroke}%
\end{pgfscope}%
\begin{pgfscope}%
\pgfpathrectangle{\pgfqpoint{0.100000in}{0.212622in}}{\pgfqpoint{3.696000in}{3.696000in}}%
\pgfusepath{clip}%
\pgfsetrectcap%
\pgfsetroundjoin%
\pgfsetlinewidth{1.505625pt}%
\definecolor{currentstroke}{rgb}{1.000000,0.000000,0.000000}%
\pgfsetstrokecolor{currentstroke}%
\pgfsetdash{}{0pt}%
\pgfpathmoveto{\pgfqpoint{2.689869in}{2.001199in}}%
\pgfpathlineto{\pgfqpoint{2.574153in}{1.081631in}}%
\pgfusepath{stroke}%
\end{pgfscope}%
\begin{pgfscope}%
\pgfpathrectangle{\pgfqpoint{0.100000in}{0.212622in}}{\pgfqpoint{3.696000in}{3.696000in}}%
\pgfusepath{clip}%
\pgfsetrectcap%
\pgfsetroundjoin%
\pgfsetlinewidth{1.505625pt}%
\definecolor{currentstroke}{rgb}{1.000000,0.000000,0.000000}%
\pgfsetstrokecolor{currentstroke}%
\pgfsetdash{}{0pt}%
\pgfpathmoveto{\pgfqpoint{2.678161in}{1.989933in}}%
\pgfpathlineto{\pgfqpoint{2.557111in}{1.064828in}}%
\pgfusepath{stroke}%
\end{pgfscope}%
\begin{pgfscope}%
\pgfpathrectangle{\pgfqpoint{0.100000in}{0.212622in}}{\pgfqpoint{3.696000in}{3.696000in}}%
\pgfusepath{clip}%
\pgfsetrectcap%
\pgfsetroundjoin%
\pgfsetlinewidth{1.505625pt}%
\definecolor{currentstroke}{rgb}{1.000000,0.000000,0.000000}%
\pgfsetstrokecolor{currentstroke}%
\pgfsetdash{}{0pt}%
\pgfpathmoveto{\pgfqpoint{2.662128in}{1.981202in}}%
\pgfpathlineto{\pgfqpoint{2.548573in}{1.056410in}}%
\pgfusepath{stroke}%
\end{pgfscope}%
\begin{pgfscope}%
\pgfpathrectangle{\pgfqpoint{0.100000in}{0.212622in}}{\pgfqpoint{3.696000in}{3.696000in}}%
\pgfusepath{clip}%
\pgfsetrectcap%
\pgfsetroundjoin%
\pgfsetlinewidth{1.505625pt}%
\definecolor{currentstroke}{rgb}{1.000000,0.000000,0.000000}%
\pgfsetstrokecolor{currentstroke}%
\pgfsetdash{}{0pt}%
\pgfpathmoveto{\pgfqpoint{2.653374in}{1.976731in}}%
\pgfpathlineto{\pgfqpoint{2.540023in}{1.047980in}}%
\pgfusepath{stroke}%
\end{pgfscope}%
\begin{pgfscope}%
\pgfpathrectangle{\pgfqpoint{0.100000in}{0.212622in}}{\pgfqpoint{3.696000in}{3.696000in}}%
\pgfusepath{clip}%
\pgfsetrectcap%
\pgfsetroundjoin%
\pgfsetlinewidth{1.505625pt}%
\definecolor{currentstroke}{rgb}{1.000000,0.000000,0.000000}%
\pgfsetstrokecolor{currentstroke}%
\pgfsetdash{}{0pt}%
\pgfpathmoveto{\pgfqpoint{2.648447in}{1.974258in}}%
\pgfpathlineto{\pgfqpoint{2.531462in}{1.039540in}}%
\pgfusepath{stroke}%
\end{pgfscope}%
\begin{pgfscope}%
\pgfpathrectangle{\pgfqpoint{0.100000in}{0.212622in}}{\pgfqpoint{3.696000in}{3.696000in}}%
\pgfusepath{clip}%
\pgfsetrectcap%
\pgfsetroundjoin%
\pgfsetlinewidth{1.505625pt}%
\definecolor{currentstroke}{rgb}{1.000000,0.000000,0.000000}%
\pgfsetstrokecolor{currentstroke}%
\pgfsetdash{}{0pt}%
\pgfpathmoveto{\pgfqpoint{2.645333in}{1.972773in}}%
\pgfpathlineto{\pgfqpoint{2.531462in}{1.039540in}}%
\pgfusepath{stroke}%
\end{pgfscope}%
\begin{pgfscope}%
\pgfpathrectangle{\pgfqpoint{0.100000in}{0.212622in}}{\pgfqpoint{3.696000in}{3.696000in}}%
\pgfusepath{clip}%
\pgfsetrectcap%
\pgfsetroundjoin%
\pgfsetlinewidth{1.505625pt}%
\definecolor{currentstroke}{rgb}{1.000000,0.000000,0.000000}%
\pgfsetstrokecolor{currentstroke}%
\pgfsetdash{}{0pt}%
\pgfpathmoveto{\pgfqpoint{2.643691in}{1.971976in}}%
\pgfpathlineto{\pgfqpoint{2.531462in}{1.039540in}}%
\pgfusepath{stroke}%
\end{pgfscope}%
\begin{pgfscope}%
\pgfpathrectangle{\pgfqpoint{0.100000in}{0.212622in}}{\pgfqpoint{3.696000in}{3.696000in}}%
\pgfusepath{clip}%
\pgfsetrectcap%
\pgfsetroundjoin%
\pgfsetlinewidth{1.505625pt}%
\definecolor{currentstroke}{rgb}{1.000000,0.000000,0.000000}%
\pgfsetstrokecolor{currentstroke}%
\pgfsetdash{}{0pt}%
\pgfpathmoveto{\pgfqpoint{2.642883in}{1.971858in}}%
\pgfpathlineto{\pgfqpoint{2.531462in}{1.039540in}}%
\pgfusepath{stroke}%
\end{pgfscope}%
\begin{pgfscope}%
\pgfpathrectangle{\pgfqpoint{0.100000in}{0.212622in}}{\pgfqpoint{3.696000in}{3.696000in}}%
\pgfusepath{clip}%
\pgfsetrectcap%
\pgfsetroundjoin%
\pgfsetlinewidth{1.505625pt}%
\definecolor{currentstroke}{rgb}{1.000000,0.000000,0.000000}%
\pgfsetstrokecolor{currentstroke}%
\pgfsetdash{}{0pt}%
\pgfpathmoveto{\pgfqpoint{2.640882in}{1.969250in}}%
\pgfpathlineto{\pgfqpoint{2.531462in}{1.039540in}}%
\pgfusepath{stroke}%
\end{pgfscope}%
\begin{pgfscope}%
\pgfpathrectangle{\pgfqpoint{0.100000in}{0.212622in}}{\pgfqpoint{3.696000in}{3.696000in}}%
\pgfusepath{clip}%
\pgfsetrectcap%
\pgfsetroundjoin%
\pgfsetlinewidth{1.505625pt}%
\definecolor{currentstroke}{rgb}{1.000000,0.000000,0.000000}%
\pgfsetstrokecolor{currentstroke}%
\pgfsetdash{}{0pt}%
\pgfpathmoveto{\pgfqpoint{2.636345in}{1.968078in}}%
\pgfpathlineto{\pgfqpoint{2.522889in}{1.031088in}}%
\pgfusepath{stroke}%
\end{pgfscope}%
\begin{pgfscope}%
\pgfpathrectangle{\pgfqpoint{0.100000in}{0.212622in}}{\pgfqpoint{3.696000in}{3.696000in}}%
\pgfusepath{clip}%
\pgfsetrectcap%
\pgfsetroundjoin%
\pgfsetlinewidth{1.505625pt}%
\definecolor{currentstroke}{rgb}{1.000000,0.000000,0.000000}%
\pgfsetstrokecolor{currentstroke}%
\pgfsetdash{}{0pt}%
\pgfpathmoveto{\pgfqpoint{2.630133in}{1.964574in}}%
\pgfpathlineto{\pgfqpoint{2.514305in}{1.022624in}}%
\pgfusepath{stroke}%
\end{pgfscope}%
\begin{pgfscope}%
\pgfpathrectangle{\pgfqpoint{0.100000in}{0.212622in}}{\pgfqpoint{3.696000in}{3.696000in}}%
\pgfusepath{clip}%
\pgfsetrectcap%
\pgfsetroundjoin%
\pgfsetlinewidth{1.505625pt}%
\definecolor{currentstroke}{rgb}{1.000000,0.000000,0.000000}%
\pgfsetstrokecolor{currentstroke}%
\pgfsetdash{}{0pt}%
\pgfpathmoveto{\pgfqpoint{2.620329in}{1.961676in}}%
\pgfpathlineto{\pgfqpoint{2.514305in}{1.022624in}}%
\pgfusepath{stroke}%
\end{pgfscope}%
\begin{pgfscope}%
\pgfpathrectangle{\pgfqpoint{0.100000in}{0.212622in}}{\pgfqpoint{3.696000in}{3.696000in}}%
\pgfusepath{clip}%
\pgfsetrectcap%
\pgfsetroundjoin%
\pgfsetlinewidth{1.505625pt}%
\definecolor{currentstroke}{rgb}{1.000000,0.000000,0.000000}%
\pgfsetstrokecolor{currentstroke}%
\pgfsetdash{}{0pt}%
\pgfpathmoveto{\pgfqpoint{2.609915in}{1.956116in}}%
\pgfpathlineto{\pgfqpoint{2.497102in}{1.005663in}}%
\pgfusepath{stroke}%
\end{pgfscope}%
\begin{pgfscope}%
\pgfpathrectangle{\pgfqpoint{0.100000in}{0.212622in}}{\pgfqpoint{3.696000in}{3.696000in}}%
\pgfusepath{clip}%
\pgfsetrectcap%
\pgfsetroundjoin%
\pgfsetlinewidth{1.505625pt}%
\definecolor{currentstroke}{rgb}{1.000000,0.000000,0.000000}%
\pgfsetstrokecolor{currentstroke}%
\pgfsetdash{}{0pt}%
\pgfpathmoveto{\pgfqpoint{2.596745in}{1.950138in}}%
\pgfpathlineto{\pgfqpoint{2.488483in}{0.997166in}}%
\pgfusepath{stroke}%
\end{pgfscope}%
\begin{pgfscope}%
\pgfpathrectangle{\pgfqpoint{0.100000in}{0.212622in}}{\pgfqpoint{3.696000in}{3.696000in}}%
\pgfusepath{clip}%
\pgfsetrectcap%
\pgfsetroundjoin%
\pgfsetlinewidth{1.505625pt}%
\definecolor{currentstroke}{rgb}{1.000000,0.000000,0.000000}%
\pgfsetstrokecolor{currentstroke}%
\pgfsetdash{}{0pt}%
\pgfpathmoveto{\pgfqpoint{2.590690in}{1.948383in}}%
\pgfpathlineto{\pgfqpoint{2.488483in}{0.997166in}}%
\pgfusepath{stroke}%
\end{pgfscope}%
\begin{pgfscope}%
\pgfpathrectangle{\pgfqpoint{0.100000in}{0.212622in}}{\pgfqpoint{3.696000in}{3.696000in}}%
\pgfusepath{clip}%
\pgfsetrectcap%
\pgfsetroundjoin%
\pgfsetlinewidth{1.505625pt}%
\definecolor{currentstroke}{rgb}{1.000000,0.000000,0.000000}%
\pgfsetstrokecolor{currentstroke}%
\pgfsetdash{}{0pt}%
\pgfpathmoveto{\pgfqpoint{2.586525in}{1.946460in}}%
\pgfpathlineto{\pgfqpoint{2.479853in}{0.988657in}}%
\pgfusepath{stroke}%
\end{pgfscope}%
\begin{pgfscope}%
\pgfpathrectangle{\pgfqpoint{0.100000in}{0.212622in}}{\pgfqpoint{3.696000in}{3.696000in}}%
\pgfusepath{clip}%
\pgfsetrectcap%
\pgfsetroundjoin%
\pgfsetlinewidth{1.505625pt}%
\definecolor{currentstroke}{rgb}{1.000000,0.000000,0.000000}%
\pgfsetstrokecolor{currentstroke}%
\pgfsetdash{}{0pt}%
\pgfpathmoveto{\pgfqpoint{2.584653in}{1.946094in}}%
\pgfpathlineto{\pgfqpoint{2.479853in}{0.988657in}}%
\pgfusepath{stroke}%
\end{pgfscope}%
\begin{pgfscope}%
\pgfpathrectangle{\pgfqpoint{0.100000in}{0.212622in}}{\pgfqpoint{3.696000in}{3.696000in}}%
\pgfusepath{clip}%
\pgfsetrectcap%
\pgfsetroundjoin%
\pgfsetlinewidth{1.505625pt}%
\definecolor{currentstroke}{rgb}{1.000000,0.000000,0.000000}%
\pgfsetstrokecolor{currentstroke}%
\pgfsetdash{}{0pt}%
\pgfpathmoveto{\pgfqpoint{2.583624in}{1.945433in}}%
\pgfpathlineto{\pgfqpoint{2.479853in}{0.988657in}}%
\pgfusepath{stroke}%
\end{pgfscope}%
\begin{pgfscope}%
\pgfpathrectangle{\pgfqpoint{0.100000in}{0.212622in}}{\pgfqpoint{3.696000in}{3.696000in}}%
\pgfusepath{clip}%
\pgfsetrectcap%
\pgfsetroundjoin%
\pgfsetlinewidth{1.505625pt}%
\definecolor{currentstroke}{rgb}{1.000000,0.000000,0.000000}%
\pgfsetstrokecolor{currentstroke}%
\pgfsetdash{}{0pt}%
\pgfpathmoveto{\pgfqpoint{2.583013in}{1.945277in}}%
\pgfpathlineto{\pgfqpoint{2.479853in}{0.988657in}}%
\pgfusepath{stroke}%
\end{pgfscope}%
\begin{pgfscope}%
\pgfpathrectangle{\pgfqpoint{0.100000in}{0.212622in}}{\pgfqpoint{3.696000in}{3.696000in}}%
\pgfusepath{clip}%
\pgfsetrectcap%
\pgfsetroundjoin%
\pgfsetlinewidth{1.505625pt}%
\definecolor{currentstroke}{rgb}{1.000000,0.000000,0.000000}%
\pgfsetstrokecolor{currentstroke}%
\pgfsetdash{}{0pt}%
\pgfpathmoveto{\pgfqpoint{2.579480in}{1.941753in}}%
\pgfpathlineto{\pgfqpoint{2.471211in}{0.980136in}}%
\pgfusepath{stroke}%
\end{pgfscope}%
\begin{pgfscope}%
\pgfpathrectangle{\pgfqpoint{0.100000in}{0.212622in}}{\pgfqpoint{3.696000in}{3.696000in}}%
\pgfusepath{clip}%
\pgfsetrectcap%
\pgfsetroundjoin%
\pgfsetlinewidth{1.505625pt}%
\definecolor{currentstroke}{rgb}{1.000000,0.000000,0.000000}%
\pgfsetstrokecolor{currentstroke}%
\pgfsetdash{}{0pt}%
\pgfpathmoveto{\pgfqpoint{2.573391in}{1.940029in}}%
\pgfpathlineto{\pgfqpoint{2.471211in}{0.980136in}}%
\pgfusepath{stroke}%
\end{pgfscope}%
\begin{pgfscope}%
\pgfpathrectangle{\pgfqpoint{0.100000in}{0.212622in}}{\pgfqpoint{3.696000in}{3.696000in}}%
\pgfusepath{clip}%
\pgfsetrectcap%
\pgfsetroundjoin%
\pgfsetlinewidth{1.505625pt}%
\definecolor{currentstroke}{rgb}{1.000000,0.000000,0.000000}%
\pgfsetstrokecolor{currentstroke}%
\pgfsetdash{}{0pt}%
\pgfpathmoveto{\pgfqpoint{2.566111in}{1.933413in}}%
\pgfpathlineto{\pgfqpoint{2.462557in}{0.971604in}}%
\pgfusepath{stroke}%
\end{pgfscope}%
\begin{pgfscope}%
\pgfpathrectangle{\pgfqpoint{0.100000in}{0.212622in}}{\pgfqpoint{3.696000in}{3.696000in}}%
\pgfusepath{clip}%
\pgfsetrectcap%
\pgfsetroundjoin%
\pgfsetlinewidth{1.505625pt}%
\definecolor{currentstroke}{rgb}{1.000000,0.000000,0.000000}%
\pgfsetstrokecolor{currentstroke}%
\pgfsetdash{}{0pt}%
\pgfpathmoveto{\pgfqpoint{2.561468in}{1.931013in}}%
\pgfpathlineto{\pgfqpoint{2.462557in}{0.971604in}}%
\pgfusepath{stroke}%
\end{pgfscope}%
\begin{pgfscope}%
\pgfpathrectangle{\pgfqpoint{0.100000in}{0.212622in}}{\pgfqpoint{3.696000in}{3.696000in}}%
\pgfusepath{clip}%
\pgfsetrectcap%
\pgfsetroundjoin%
\pgfsetlinewidth{1.505625pt}%
\definecolor{currentstroke}{rgb}{1.000000,0.000000,0.000000}%
\pgfsetstrokecolor{currentstroke}%
\pgfsetdash{}{0pt}%
\pgfpathmoveto{\pgfqpoint{2.558921in}{1.929920in}}%
\pgfpathlineto{\pgfqpoint{2.453892in}{0.963061in}}%
\pgfusepath{stroke}%
\end{pgfscope}%
\begin{pgfscope}%
\pgfpathrectangle{\pgfqpoint{0.100000in}{0.212622in}}{\pgfqpoint{3.696000in}{3.696000in}}%
\pgfusepath{clip}%
\pgfsetrectcap%
\pgfsetroundjoin%
\pgfsetlinewidth{1.505625pt}%
\definecolor{currentstroke}{rgb}{1.000000,0.000000,0.000000}%
\pgfsetstrokecolor{currentstroke}%
\pgfsetdash{}{0pt}%
\pgfpathmoveto{\pgfqpoint{2.557552in}{1.929327in}}%
\pgfpathlineto{\pgfqpoint{2.453892in}{0.963061in}}%
\pgfusepath{stroke}%
\end{pgfscope}%
\begin{pgfscope}%
\pgfpathrectangle{\pgfqpoint{0.100000in}{0.212622in}}{\pgfqpoint{3.696000in}{3.696000in}}%
\pgfusepath{clip}%
\pgfsetrectcap%
\pgfsetroundjoin%
\pgfsetlinewidth{1.505625pt}%
\definecolor{currentstroke}{rgb}{1.000000,0.000000,0.000000}%
\pgfsetstrokecolor{currentstroke}%
\pgfsetdash{}{0pt}%
\pgfpathmoveto{\pgfqpoint{2.556648in}{1.929202in}}%
\pgfpathlineto{\pgfqpoint{2.453892in}{0.963061in}}%
\pgfusepath{stroke}%
\end{pgfscope}%
\begin{pgfscope}%
\pgfpathrectangle{\pgfqpoint{0.100000in}{0.212622in}}{\pgfqpoint{3.696000in}{3.696000in}}%
\pgfusepath{clip}%
\pgfsetrectcap%
\pgfsetroundjoin%
\pgfsetlinewidth{1.505625pt}%
\definecolor{currentstroke}{rgb}{1.000000,0.000000,0.000000}%
\pgfsetstrokecolor{currentstroke}%
\pgfsetdash{}{0pt}%
\pgfpathmoveto{\pgfqpoint{2.556274in}{1.928810in}}%
\pgfpathlineto{\pgfqpoint{2.453892in}{0.963061in}}%
\pgfusepath{stroke}%
\end{pgfscope}%
\begin{pgfscope}%
\pgfpathrectangle{\pgfqpoint{0.100000in}{0.212622in}}{\pgfqpoint{3.696000in}{3.696000in}}%
\pgfusepath{clip}%
\pgfsetrectcap%
\pgfsetroundjoin%
\pgfsetlinewidth{1.505625pt}%
\definecolor{currentstroke}{rgb}{1.000000,0.000000,0.000000}%
\pgfsetstrokecolor{currentstroke}%
\pgfsetdash{}{0pt}%
\pgfpathmoveto{\pgfqpoint{2.556089in}{1.928813in}}%
\pgfpathlineto{\pgfqpoint{2.453892in}{0.963061in}}%
\pgfusepath{stroke}%
\end{pgfscope}%
\begin{pgfscope}%
\pgfpathrectangle{\pgfqpoint{0.100000in}{0.212622in}}{\pgfqpoint{3.696000in}{3.696000in}}%
\pgfusepath{clip}%
\pgfsetrectcap%
\pgfsetroundjoin%
\pgfsetlinewidth{1.505625pt}%
\definecolor{currentstroke}{rgb}{1.000000,0.000000,0.000000}%
\pgfsetstrokecolor{currentstroke}%
\pgfsetdash{}{0pt}%
\pgfpathmoveto{\pgfqpoint{2.554277in}{1.926619in}}%
\pgfpathlineto{\pgfqpoint{2.453892in}{0.963061in}}%
\pgfusepath{stroke}%
\end{pgfscope}%
\begin{pgfscope}%
\pgfpathrectangle{\pgfqpoint{0.100000in}{0.212622in}}{\pgfqpoint{3.696000in}{3.696000in}}%
\pgfusepath{clip}%
\pgfsetrectcap%
\pgfsetroundjoin%
\pgfsetlinewidth{1.505625pt}%
\definecolor{currentstroke}{rgb}{1.000000,0.000000,0.000000}%
\pgfsetstrokecolor{currentstroke}%
\pgfsetdash{}{0pt}%
\pgfpathmoveto{\pgfqpoint{2.549799in}{1.925112in}}%
\pgfpathlineto{\pgfqpoint{2.445215in}{0.954506in}}%
\pgfusepath{stroke}%
\end{pgfscope}%
\begin{pgfscope}%
\pgfpathrectangle{\pgfqpoint{0.100000in}{0.212622in}}{\pgfqpoint{3.696000in}{3.696000in}}%
\pgfusepath{clip}%
\pgfsetrectcap%
\pgfsetroundjoin%
\pgfsetlinewidth{1.505625pt}%
\definecolor{currentstroke}{rgb}{1.000000,0.000000,0.000000}%
\pgfsetstrokecolor{currentstroke}%
\pgfsetdash{}{0pt}%
\pgfpathmoveto{\pgfqpoint{2.542731in}{1.918839in}}%
\pgfpathlineto{\pgfqpoint{2.445215in}{0.954506in}}%
\pgfusepath{stroke}%
\end{pgfscope}%
\begin{pgfscope}%
\pgfpathrectangle{\pgfqpoint{0.100000in}{0.212622in}}{\pgfqpoint{3.696000in}{3.696000in}}%
\pgfusepath{clip}%
\pgfsetrectcap%
\pgfsetroundjoin%
\pgfsetlinewidth{1.505625pt}%
\definecolor{currentstroke}{rgb}{1.000000,0.000000,0.000000}%
\pgfsetstrokecolor{currentstroke}%
\pgfsetdash{}{0pt}%
\pgfpathmoveto{\pgfqpoint{2.529836in}{1.913764in}}%
\pgfpathlineto{\pgfqpoint{2.427825in}{0.937361in}}%
\pgfusepath{stroke}%
\end{pgfscope}%
\begin{pgfscope}%
\pgfpathrectangle{\pgfqpoint{0.100000in}{0.212622in}}{\pgfqpoint{3.696000in}{3.696000in}}%
\pgfusepath{clip}%
\pgfsetrectcap%
\pgfsetroundjoin%
\pgfsetlinewidth{1.505625pt}%
\definecolor{currentstroke}{rgb}{1.000000,0.000000,0.000000}%
\pgfsetstrokecolor{currentstroke}%
\pgfsetdash{}{0pt}%
\pgfpathmoveto{\pgfqpoint{2.519174in}{1.903529in}}%
\pgfpathlineto{\pgfqpoint{2.419113in}{0.928772in}}%
\pgfusepath{stroke}%
\end{pgfscope}%
\begin{pgfscope}%
\pgfpathrectangle{\pgfqpoint{0.100000in}{0.212622in}}{\pgfqpoint{3.696000in}{3.696000in}}%
\pgfusepath{clip}%
\pgfsetrectcap%
\pgfsetroundjoin%
\pgfsetlinewidth{1.505625pt}%
\definecolor{currentstroke}{rgb}{1.000000,0.000000,0.000000}%
\pgfsetstrokecolor{currentstroke}%
\pgfsetdash{}{0pt}%
\pgfpathmoveto{\pgfqpoint{2.501084in}{1.894786in}}%
\pgfpathlineto{\pgfqpoint{2.410389in}{0.920170in}}%
\pgfusepath{stroke}%
\end{pgfscope}%
\begin{pgfscope}%
\pgfpathrectangle{\pgfqpoint{0.100000in}{0.212622in}}{\pgfqpoint{3.696000in}{3.696000in}}%
\pgfusepath{clip}%
\pgfsetrectcap%
\pgfsetroundjoin%
\pgfsetlinewidth{1.505625pt}%
\definecolor{currentstroke}{rgb}{1.000000,0.000000,0.000000}%
\pgfsetstrokecolor{currentstroke}%
\pgfsetdash{}{0pt}%
\pgfpathmoveto{\pgfqpoint{2.493448in}{1.890634in}}%
\pgfpathlineto{\pgfqpoint{2.401653in}{0.911557in}}%
\pgfusepath{stroke}%
\end{pgfscope}%
\begin{pgfscope}%
\pgfpathrectangle{\pgfqpoint{0.100000in}{0.212622in}}{\pgfqpoint{3.696000in}{3.696000in}}%
\pgfusepath{clip}%
\pgfsetrectcap%
\pgfsetroundjoin%
\pgfsetlinewidth{1.505625pt}%
\definecolor{currentstroke}{rgb}{1.000000,0.000000,0.000000}%
\pgfsetstrokecolor{currentstroke}%
\pgfsetdash{}{0pt}%
\pgfpathmoveto{\pgfqpoint{2.488975in}{1.888104in}}%
\pgfpathlineto{\pgfqpoint{2.392906in}{0.902933in}}%
\pgfusepath{stroke}%
\end{pgfscope}%
\begin{pgfscope}%
\pgfpathrectangle{\pgfqpoint{0.100000in}{0.212622in}}{\pgfqpoint{3.696000in}{3.696000in}}%
\pgfusepath{clip}%
\pgfsetrectcap%
\pgfsetroundjoin%
\pgfsetlinewidth{1.505625pt}%
\definecolor{currentstroke}{rgb}{1.000000,0.000000,0.000000}%
\pgfsetstrokecolor{currentstroke}%
\pgfsetdash{}{0pt}%
\pgfpathmoveto{\pgfqpoint{2.486606in}{1.887445in}}%
\pgfpathlineto{\pgfqpoint{2.392906in}{0.902933in}}%
\pgfusepath{stroke}%
\end{pgfscope}%
\begin{pgfscope}%
\pgfpathrectangle{\pgfqpoint{0.100000in}{0.212622in}}{\pgfqpoint{3.696000in}{3.696000in}}%
\pgfusepath{clip}%
\pgfsetrectcap%
\pgfsetroundjoin%
\pgfsetlinewidth{1.505625pt}%
\definecolor{currentstroke}{rgb}{1.000000,0.000000,0.000000}%
\pgfsetstrokecolor{currentstroke}%
\pgfsetdash{}{0pt}%
\pgfpathmoveto{\pgfqpoint{2.485305in}{1.886389in}}%
\pgfpathlineto{\pgfqpoint{2.392906in}{0.902933in}}%
\pgfusepath{stroke}%
\end{pgfscope}%
\begin{pgfscope}%
\pgfpathrectangle{\pgfqpoint{0.100000in}{0.212622in}}{\pgfqpoint{3.696000in}{3.696000in}}%
\pgfusepath{clip}%
\pgfsetrectcap%
\pgfsetroundjoin%
\pgfsetlinewidth{1.505625pt}%
\definecolor{currentstroke}{rgb}{1.000000,0.000000,0.000000}%
\pgfsetstrokecolor{currentstroke}%
\pgfsetdash{}{0pt}%
\pgfpathmoveto{\pgfqpoint{2.484536in}{1.886124in}}%
\pgfpathlineto{\pgfqpoint{2.392906in}{0.902933in}}%
\pgfusepath{stroke}%
\end{pgfscope}%
\begin{pgfscope}%
\pgfpathrectangle{\pgfqpoint{0.100000in}{0.212622in}}{\pgfqpoint{3.696000in}{3.696000in}}%
\pgfusepath{clip}%
\pgfsetrectcap%
\pgfsetroundjoin%
\pgfsetlinewidth{1.505625pt}%
\definecolor{currentstroke}{rgb}{1.000000,0.000000,0.000000}%
\pgfsetstrokecolor{currentstroke}%
\pgfsetdash{}{0pt}%
\pgfpathmoveto{\pgfqpoint{2.481419in}{1.883269in}}%
\pgfpathlineto{\pgfqpoint{2.384146in}{0.894296in}}%
\pgfusepath{stroke}%
\end{pgfscope}%
\begin{pgfscope}%
\pgfpathrectangle{\pgfqpoint{0.100000in}{0.212622in}}{\pgfqpoint{3.696000in}{3.696000in}}%
\pgfusepath{clip}%
\pgfsetrectcap%
\pgfsetroundjoin%
\pgfsetlinewidth{1.505625pt}%
\definecolor{currentstroke}{rgb}{1.000000,0.000000,0.000000}%
\pgfsetstrokecolor{currentstroke}%
\pgfsetdash{}{0pt}%
\pgfpathmoveto{\pgfqpoint{2.476110in}{1.881145in}}%
\pgfpathlineto{\pgfqpoint{2.384146in}{0.894296in}}%
\pgfusepath{stroke}%
\end{pgfscope}%
\begin{pgfscope}%
\pgfpathrectangle{\pgfqpoint{0.100000in}{0.212622in}}{\pgfqpoint{3.696000in}{3.696000in}}%
\pgfusepath{clip}%
\pgfsetrectcap%
\pgfsetroundjoin%
\pgfsetlinewidth{1.505625pt}%
\definecolor{currentstroke}{rgb}{1.000000,0.000000,0.000000}%
\pgfsetstrokecolor{currentstroke}%
\pgfsetdash{}{0pt}%
\pgfpathmoveto{\pgfqpoint{2.469091in}{1.876212in}}%
\pgfpathlineto{\pgfqpoint{2.375375in}{0.885648in}}%
\pgfusepath{stroke}%
\end{pgfscope}%
\begin{pgfscope}%
\pgfpathrectangle{\pgfqpoint{0.100000in}{0.212622in}}{\pgfqpoint{3.696000in}{3.696000in}}%
\pgfusepath{clip}%
\pgfsetrectcap%
\pgfsetroundjoin%
\pgfsetlinewidth{1.505625pt}%
\definecolor{currentstroke}{rgb}{1.000000,0.000000,0.000000}%
\pgfsetstrokecolor{currentstroke}%
\pgfsetdash{}{0pt}%
\pgfpathmoveto{\pgfqpoint{2.464762in}{1.874126in}}%
\pgfpathlineto{\pgfqpoint{2.375375in}{0.885648in}}%
\pgfusepath{stroke}%
\end{pgfscope}%
\begin{pgfscope}%
\pgfpathrectangle{\pgfqpoint{0.100000in}{0.212622in}}{\pgfqpoint{3.696000in}{3.696000in}}%
\pgfusepath{clip}%
\pgfsetrectcap%
\pgfsetroundjoin%
\pgfsetlinewidth{1.505625pt}%
\definecolor{currentstroke}{rgb}{1.000000,0.000000,0.000000}%
\pgfsetstrokecolor{currentstroke}%
\pgfsetdash{}{0pt}%
\pgfpathmoveto{\pgfqpoint{2.462289in}{1.873586in}}%
\pgfpathlineto{\pgfqpoint{2.375375in}{0.885648in}}%
\pgfusepath{stroke}%
\end{pgfscope}%
\begin{pgfscope}%
\pgfpathrectangle{\pgfqpoint{0.100000in}{0.212622in}}{\pgfqpoint{3.696000in}{3.696000in}}%
\pgfusepath{clip}%
\pgfsetrectcap%
\pgfsetroundjoin%
\pgfsetlinewidth{1.505625pt}%
\definecolor{currentstroke}{rgb}{1.000000,0.000000,0.000000}%
\pgfsetstrokecolor{currentstroke}%
\pgfsetdash{}{0pt}%
\pgfpathmoveto{\pgfqpoint{2.461125in}{1.872753in}}%
\pgfpathlineto{\pgfqpoint{2.375375in}{0.885648in}}%
\pgfusepath{stroke}%
\end{pgfscope}%
\begin{pgfscope}%
\pgfpathrectangle{\pgfqpoint{0.100000in}{0.212622in}}{\pgfqpoint{3.696000in}{3.696000in}}%
\pgfusepath{clip}%
\pgfsetrectcap%
\pgfsetroundjoin%
\pgfsetlinewidth{1.505625pt}%
\definecolor{currentstroke}{rgb}{1.000000,0.000000,0.000000}%
\pgfsetstrokecolor{currentstroke}%
\pgfsetdash{}{0pt}%
\pgfpathmoveto{\pgfqpoint{2.460421in}{1.872563in}}%
\pgfpathlineto{\pgfqpoint{2.366591in}{0.876989in}}%
\pgfusepath{stroke}%
\end{pgfscope}%
\begin{pgfscope}%
\pgfpathrectangle{\pgfqpoint{0.100000in}{0.212622in}}{\pgfqpoint{3.696000in}{3.696000in}}%
\pgfusepath{clip}%
\pgfsetrectcap%
\pgfsetroundjoin%
\pgfsetlinewidth{1.505625pt}%
\definecolor{currentstroke}{rgb}{1.000000,0.000000,0.000000}%
\pgfsetstrokecolor{currentstroke}%
\pgfsetdash{}{0pt}%
\pgfpathmoveto{\pgfqpoint{2.460096in}{1.872316in}}%
\pgfpathlineto{\pgfqpoint{2.366591in}{0.876989in}}%
\pgfusepath{stroke}%
\end{pgfscope}%
\begin{pgfscope}%
\pgfpathrectangle{\pgfqpoint{0.100000in}{0.212622in}}{\pgfqpoint{3.696000in}{3.696000in}}%
\pgfusepath{clip}%
\pgfsetrectcap%
\pgfsetroundjoin%
\pgfsetlinewidth{1.505625pt}%
\definecolor{currentstroke}{rgb}{1.000000,0.000000,0.000000}%
\pgfsetstrokecolor{currentstroke}%
\pgfsetdash{}{0pt}%
\pgfpathmoveto{\pgfqpoint{2.459903in}{1.872236in}}%
\pgfpathlineto{\pgfqpoint{2.366591in}{0.876989in}}%
\pgfusepath{stroke}%
\end{pgfscope}%
\begin{pgfscope}%
\pgfpathrectangle{\pgfqpoint{0.100000in}{0.212622in}}{\pgfqpoint{3.696000in}{3.696000in}}%
\pgfusepath{clip}%
\pgfsetrectcap%
\pgfsetroundjoin%
\pgfsetlinewidth{1.505625pt}%
\definecolor{currentstroke}{rgb}{1.000000,0.000000,0.000000}%
\pgfsetstrokecolor{currentstroke}%
\pgfsetdash{}{0pt}%
\pgfpathmoveto{\pgfqpoint{2.457980in}{1.870395in}}%
\pgfpathlineto{\pgfqpoint{2.366591in}{0.876989in}}%
\pgfusepath{stroke}%
\end{pgfscope}%
\begin{pgfscope}%
\pgfpathrectangle{\pgfqpoint{0.100000in}{0.212622in}}{\pgfqpoint{3.696000in}{3.696000in}}%
\pgfusepath{clip}%
\pgfsetrectcap%
\pgfsetroundjoin%
\pgfsetlinewidth{1.505625pt}%
\definecolor{currentstroke}{rgb}{1.000000,0.000000,0.000000}%
\pgfsetstrokecolor{currentstroke}%
\pgfsetdash{}{0pt}%
\pgfpathmoveto{\pgfqpoint{2.454047in}{1.868536in}}%
\pgfpathlineto{\pgfqpoint{2.366591in}{0.876989in}}%
\pgfusepath{stroke}%
\end{pgfscope}%
\begin{pgfscope}%
\pgfpathrectangle{\pgfqpoint{0.100000in}{0.212622in}}{\pgfqpoint{3.696000in}{3.696000in}}%
\pgfusepath{clip}%
\pgfsetrectcap%
\pgfsetroundjoin%
\pgfsetlinewidth{1.505625pt}%
\definecolor{currentstroke}{rgb}{1.000000,0.000000,0.000000}%
\pgfsetstrokecolor{currentstroke}%
\pgfsetdash{}{0pt}%
\pgfpathmoveto{\pgfqpoint{2.448119in}{1.865715in}}%
\pgfpathlineto{\pgfqpoint{2.357796in}{0.868317in}}%
\pgfusepath{stroke}%
\end{pgfscope}%
\begin{pgfscope}%
\pgfpathrectangle{\pgfqpoint{0.100000in}{0.212622in}}{\pgfqpoint{3.696000in}{3.696000in}}%
\pgfusepath{clip}%
\pgfsetrectcap%
\pgfsetroundjoin%
\pgfsetlinewidth{1.505625pt}%
\definecolor{currentstroke}{rgb}{1.000000,0.000000,0.000000}%
\pgfsetstrokecolor{currentstroke}%
\pgfsetdash{}{0pt}%
\pgfpathmoveto{\pgfqpoint{2.439689in}{1.861640in}}%
\pgfpathlineto{\pgfqpoint{2.348989in}{0.859634in}}%
\pgfusepath{stroke}%
\end{pgfscope}%
\begin{pgfscope}%
\pgfpathrectangle{\pgfqpoint{0.100000in}{0.212622in}}{\pgfqpoint{3.696000in}{3.696000in}}%
\pgfusepath{clip}%
\pgfsetrectcap%
\pgfsetroundjoin%
\pgfsetlinewidth{1.505625pt}%
\definecolor{currentstroke}{rgb}{1.000000,0.000000,0.000000}%
\pgfsetstrokecolor{currentstroke}%
\pgfsetdash{}{0pt}%
\pgfpathmoveto{\pgfqpoint{2.435957in}{1.859321in}}%
\pgfpathlineto{\pgfqpoint{2.348989in}{0.859634in}}%
\pgfusepath{stroke}%
\end{pgfscope}%
\begin{pgfscope}%
\pgfpathrectangle{\pgfqpoint{0.100000in}{0.212622in}}{\pgfqpoint{3.696000in}{3.696000in}}%
\pgfusepath{clip}%
\pgfsetrectcap%
\pgfsetroundjoin%
\pgfsetlinewidth{1.505625pt}%
\definecolor{currentstroke}{rgb}{1.000000,0.000000,0.000000}%
\pgfsetstrokecolor{currentstroke}%
\pgfsetdash{}{0pt}%
\pgfpathmoveto{\pgfqpoint{2.433216in}{1.858149in}}%
\pgfpathlineto{\pgfqpoint{2.348989in}{0.859634in}}%
\pgfusepath{stroke}%
\end{pgfscope}%
\begin{pgfscope}%
\pgfpathrectangle{\pgfqpoint{0.100000in}{0.212622in}}{\pgfqpoint{3.696000in}{3.696000in}}%
\pgfusepath{clip}%
\pgfsetrectcap%
\pgfsetroundjoin%
\pgfsetlinewidth{1.505625pt}%
\definecolor{currentstroke}{rgb}{1.000000,0.000000,0.000000}%
\pgfsetstrokecolor{currentstroke}%
\pgfsetdash{}{0pt}%
\pgfpathmoveto{\pgfqpoint{2.432073in}{1.857922in}}%
\pgfpathlineto{\pgfqpoint{2.348989in}{0.859634in}}%
\pgfusepath{stroke}%
\end{pgfscope}%
\begin{pgfscope}%
\pgfpathrectangle{\pgfqpoint{0.100000in}{0.212622in}}{\pgfqpoint{3.696000in}{3.696000in}}%
\pgfusepath{clip}%
\pgfsetrectcap%
\pgfsetroundjoin%
\pgfsetlinewidth{1.505625pt}%
\definecolor{currentstroke}{rgb}{1.000000,0.000000,0.000000}%
\pgfsetstrokecolor{currentstroke}%
\pgfsetdash{}{0pt}%
\pgfpathmoveto{\pgfqpoint{2.431457in}{1.857393in}}%
\pgfpathlineto{\pgfqpoint{2.348989in}{0.859634in}}%
\pgfusepath{stroke}%
\end{pgfscope}%
\begin{pgfscope}%
\pgfpathrectangle{\pgfqpoint{0.100000in}{0.212622in}}{\pgfqpoint{3.696000in}{3.696000in}}%
\pgfusepath{clip}%
\pgfsetrectcap%
\pgfsetroundjoin%
\pgfsetlinewidth{1.505625pt}%
\definecolor{currentstroke}{rgb}{1.000000,0.000000,0.000000}%
\pgfsetstrokecolor{currentstroke}%
\pgfsetdash{}{0pt}%
\pgfpathmoveto{\pgfqpoint{2.431081in}{1.857278in}}%
\pgfpathlineto{\pgfqpoint{2.348989in}{0.859634in}}%
\pgfusepath{stroke}%
\end{pgfscope}%
\begin{pgfscope}%
\pgfpathrectangle{\pgfqpoint{0.100000in}{0.212622in}}{\pgfqpoint{3.696000in}{3.696000in}}%
\pgfusepath{clip}%
\pgfsetrectcap%
\pgfsetroundjoin%
\pgfsetlinewidth{1.505625pt}%
\definecolor{currentstroke}{rgb}{1.000000,0.000000,0.000000}%
\pgfsetstrokecolor{currentstroke}%
\pgfsetdash{}{0pt}%
\pgfpathmoveto{\pgfqpoint{2.428622in}{1.855066in}}%
\pgfpathlineto{\pgfqpoint{2.340170in}{0.850939in}}%
\pgfusepath{stroke}%
\end{pgfscope}%
\begin{pgfscope}%
\pgfpathrectangle{\pgfqpoint{0.100000in}{0.212622in}}{\pgfqpoint{3.696000in}{3.696000in}}%
\pgfusepath{clip}%
\pgfsetrectcap%
\pgfsetroundjoin%
\pgfsetlinewidth{1.505625pt}%
\definecolor{currentstroke}{rgb}{1.000000,0.000000,0.000000}%
\pgfsetstrokecolor{currentstroke}%
\pgfsetdash{}{0pt}%
\pgfpathmoveto{\pgfqpoint{2.426848in}{1.854279in}}%
\pgfpathlineto{\pgfqpoint{2.340170in}{0.850939in}}%
\pgfusepath{stroke}%
\end{pgfscope}%
\begin{pgfscope}%
\pgfpathrectangle{\pgfqpoint{0.100000in}{0.212622in}}{\pgfqpoint{3.696000in}{3.696000in}}%
\pgfusepath{clip}%
\pgfsetrectcap%
\pgfsetroundjoin%
\pgfsetlinewidth{1.505625pt}%
\definecolor{currentstroke}{rgb}{1.000000,0.000000,0.000000}%
\pgfsetstrokecolor{currentstroke}%
\pgfsetdash{}{0pt}%
\pgfpathmoveto{\pgfqpoint{2.423420in}{1.852144in}}%
\pgfpathlineto{\pgfqpoint{2.340170in}{0.850939in}}%
\pgfusepath{stroke}%
\end{pgfscope}%
\begin{pgfscope}%
\pgfpathrectangle{\pgfqpoint{0.100000in}{0.212622in}}{\pgfqpoint{3.696000in}{3.696000in}}%
\pgfusepath{clip}%
\pgfsetrectcap%
\pgfsetroundjoin%
\pgfsetlinewidth{1.505625pt}%
\definecolor{currentstroke}{rgb}{1.000000,0.000000,0.000000}%
\pgfsetstrokecolor{currentstroke}%
\pgfsetdash{}{0pt}%
\pgfpathmoveto{\pgfqpoint{2.420954in}{1.851021in}}%
\pgfpathlineto{\pgfqpoint{2.331339in}{0.842232in}}%
\pgfusepath{stroke}%
\end{pgfscope}%
\begin{pgfscope}%
\pgfpathrectangle{\pgfqpoint{0.100000in}{0.212622in}}{\pgfqpoint{3.696000in}{3.696000in}}%
\pgfusepath{clip}%
\pgfsetrectcap%
\pgfsetroundjoin%
\pgfsetlinewidth{1.505625pt}%
\definecolor{currentstroke}{rgb}{1.000000,0.000000,0.000000}%
\pgfsetstrokecolor{currentstroke}%
\pgfsetdash{}{0pt}%
\pgfpathmoveto{\pgfqpoint{2.419889in}{1.850879in}}%
\pgfpathlineto{\pgfqpoint{2.331339in}{0.842232in}}%
\pgfusepath{stroke}%
\end{pgfscope}%
\begin{pgfscope}%
\pgfpathrectangle{\pgfqpoint{0.100000in}{0.212622in}}{\pgfqpoint{3.696000in}{3.696000in}}%
\pgfusepath{clip}%
\pgfsetrectcap%
\pgfsetroundjoin%
\pgfsetlinewidth{1.505625pt}%
\definecolor{currentstroke}{rgb}{1.000000,0.000000,0.000000}%
\pgfsetstrokecolor{currentstroke}%
\pgfsetdash{}{0pt}%
\pgfpathmoveto{\pgfqpoint{2.419255in}{1.850547in}}%
\pgfpathlineto{\pgfqpoint{2.331339in}{0.842232in}}%
\pgfusepath{stroke}%
\end{pgfscope}%
\begin{pgfscope}%
\pgfpathrectangle{\pgfqpoint{0.100000in}{0.212622in}}{\pgfqpoint{3.696000in}{3.696000in}}%
\pgfusepath{clip}%
\pgfsetrectcap%
\pgfsetroundjoin%
\pgfsetlinewidth{1.505625pt}%
\definecolor{currentstroke}{rgb}{1.000000,0.000000,0.000000}%
\pgfsetstrokecolor{currentstroke}%
\pgfsetdash{}{0pt}%
\pgfpathmoveto{\pgfqpoint{2.418912in}{1.850476in}}%
\pgfpathlineto{\pgfqpoint{2.331339in}{0.842232in}}%
\pgfusepath{stroke}%
\end{pgfscope}%
\begin{pgfscope}%
\pgfpathrectangle{\pgfqpoint{0.100000in}{0.212622in}}{\pgfqpoint{3.696000in}{3.696000in}}%
\pgfusepath{clip}%
\pgfsetrectcap%
\pgfsetroundjoin%
\pgfsetlinewidth{1.505625pt}%
\definecolor{currentstroke}{rgb}{1.000000,0.000000,0.000000}%
\pgfsetstrokecolor{currentstroke}%
\pgfsetdash{}{0pt}%
\pgfpathmoveto{\pgfqpoint{2.418749in}{1.850313in}}%
\pgfpathlineto{\pgfqpoint{2.331339in}{0.842232in}}%
\pgfusepath{stroke}%
\end{pgfscope}%
\begin{pgfscope}%
\pgfpathrectangle{\pgfqpoint{0.100000in}{0.212622in}}{\pgfqpoint{3.696000in}{3.696000in}}%
\pgfusepath{clip}%
\pgfsetrectcap%
\pgfsetroundjoin%
\pgfsetlinewidth{1.505625pt}%
\definecolor{currentstroke}{rgb}{1.000000,0.000000,0.000000}%
\pgfsetstrokecolor{currentstroke}%
\pgfsetdash{}{0pt}%
\pgfpathmoveto{\pgfqpoint{2.418641in}{1.850267in}}%
\pgfpathlineto{\pgfqpoint{2.331339in}{0.842232in}}%
\pgfusepath{stroke}%
\end{pgfscope}%
\begin{pgfscope}%
\pgfpathrectangle{\pgfqpoint{0.100000in}{0.212622in}}{\pgfqpoint{3.696000in}{3.696000in}}%
\pgfusepath{clip}%
\pgfsetrectcap%
\pgfsetroundjoin%
\pgfsetlinewidth{1.505625pt}%
\definecolor{currentstroke}{rgb}{1.000000,0.000000,0.000000}%
\pgfsetstrokecolor{currentstroke}%
\pgfsetdash{}{0pt}%
\pgfpathmoveto{\pgfqpoint{2.417329in}{1.849342in}}%
\pgfpathlineto{\pgfqpoint{2.331339in}{0.842232in}}%
\pgfusepath{stroke}%
\end{pgfscope}%
\begin{pgfscope}%
\pgfpathrectangle{\pgfqpoint{0.100000in}{0.212622in}}{\pgfqpoint{3.696000in}{3.696000in}}%
\pgfusepath{clip}%
\pgfsetrectcap%
\pgfsetroundjoin%
\pgfsetlinewidth{1.505625pt}%
\definecolor{currentstroke}{rgb}{1.000000,0.000000,0.000000}%
\pgfsetstrokecolor{currentstroke}%
\pgfsetdash{}{0pt}%
\pgfpathmoveto{\pgfqpoint{2.416376in}{1.848897in}}%
\pgfpathlineto{\pgfqpoint{2.331339in}{0.842232in}}%
\pgfusepath{stroke}%
\end{pgfscope}%
\begin{pgfscope}%
\pgfpathrectangle{\pgfqpoint{0.100000in}{0.212622in}}{\pgfqpoint{3.696000in}{3.696000in}}%
\pgfusepath{clip}%
\pgfsetrectcap%
\pgfsetroundjoin%
\pgfsetlinewidth{1.505625pt}%
\definecolor{currentstroke}{rgb}{1.000000,0.000000,0.000000}%
\pgfsetstrokecolor{currentstroke}%
\pgfsetdash{}{0pt}%
\pgfpathmoveto{\pgfqpoint{2.416011in}{1.848692in}}%
\pgfpathlineto{\pgfqpoint{2.331339in}{0.842232in}}%
\pgfusepath{stroke}%
\end{pgfscope}%
\begin{pgfscope}%
\pgfpathrectangle{\pgfqpoint{0.100000in}{0.212622in}}{\pgfqpoint{3.696000in}{3.696000in}}%
\pgfusepath{clip}%
\pgfsetrectcap%
\pgfsetroundjoin%
\pgfsetlinewidth{1.505625pt}%
\definecolor{currentstroke}{rgb}{1.000000,0.000000,0.000000}%
\pgfsetstrokecolor{currentstroke}%
\pgfsetdash{}{0pt}%
\pgfpathmoveto{\pgfqpoint{2.415729in}{1.848566in}}%
\pgfpathlineto{\pgfqpoint{2.331339in}{0.842232in}}%
\pgfusepath{stroke}%
\end{pgfscope}%
\begin{pgfscope}%
\pgfpathrectangle{\pgfqpoint{0.100000in}{0.212622in}}{\pgfqpoint{3.696000in}{3.696000in}}%
\pgfusepath{clip}%
\pgfsetrectcap%
\pgfsetroundjoin%
\pgfsetlinewidth{1.505625pt}%
\definecolor{currentstroke}{rgb}{1.000000,0.000000,0.000000}%
\pgfsetstrokecolor{currentstroke}%
\pgfsetdash{}{0pt}%
\pgfpathmoveto{\pgfqpoint{2.415609in}{1.848542in}}%
\pgfpathlineto{\pgfqpoint{2.331339in}{0.842232in}}%
\pgfusepath{stroke}%
\end{pgfscope}%
\begin{pgfscope}%
\pgfpathrectangle{\pgfqpoint{0.100000in}{0.212622in}}{\pgfqpoint{3.696000in}{3.696000in}}%
\pgfusepath{clip}%
\pgfsetrectcap%
\pgfsetroundjoin%
\pgfsetlinewidth{1.505625pt}%
\definecolor{currentstroke}{rgb}{1.000000,0.000000,0.000000}%
\pgfsetstrokecolor{currentstroke}%
\pgfsetdash{}{0pt}%
\pgfpathmoveto{\pgfqpoint{2.415542in}{1.848484in}}%
\pgfpathlineto{\pgfqpoint{2.331339in}{0.842232in}}%
\pgfusepath{stroke}%
\end{pgfscope}%
\begin{pgfscope}%
\pgfpathrectangle{\pgfqpoint{0.100000in}{0.212622in}}{\pgfqpoint{3.696000in}{3.696000in}}%
\pgfusepath{clip}%
\pgfsetrectcap%
\pgfsetroundjoin%
\pgfsetlinewidth{1.505625pt}%
\definecolor{currentstroke}{rgb}{1.000000,0.000000,0.000000}%
\pgfsetstrokecolor{currentstroke}%
\pgfsetdash{}{0pt}%
\pgfpathmoveto{\pgfqpoint{2.413905in}{1.847793in}}%
\pgfpathlineto{\pgfqpoint{2.331339in}{0.842232in}}%
\pgfusepath{stroke}%
\end{pgfscope}%
\begin{pgfscope}%
\pgfpathrectangle{\pgfqpoint{0.100000in}{0.212622in}}{\pgfqpoint{3.696000in}{3.696000in}}%
\pgfusepath{clip}%
\pgfsetrectcap%
\pgfsetroundjoin%
\pgfsetlinewidth{1.505625pt}%
\definecolor{currentstroke}{rgb}{1.000000,0.000000,0.000000}%
\pgfsetstrokecolor{currentstroke}%
\pgfsetdash{}{0pt}%
\pgfpathmoveto{\pgfqpoint{2.410481in}{1.845465in}}%
\pgfpathlineto{\pgfqpoint{2.331339in}{0.842232in}}%
\pgfusepath{stroke}%
\end{pgfscope}%
\begin{pgfscope}%
\pgfpathrectangle{\pgfqpoint{0.100000in}{0.212622in}}{\pgfqpoint{3.696000in}{3.696000in}}%
\pgfusepath{clip}%
\pgfsetrectcap%
\pgfsetroundjoin%
\pgfsetlinewidth{1.505625pt}%
\definecolor{currentstroke}{rgb}{1.000000,0.000000,0.000000}%
\pgfsetstrokecolor{currentstroke}%
\pgfsetdash{}{0pt}%
\pgfpathmoveto{\pgfqpoint{2.403343in}{1.842232in}}%
\pgfpathlineto{\pgfqpoint{2.331339in}{0.842232in}}%
\pgfusepath{stroke}%
\end{pgfscope}%
\begin{pgfscope}%
\pgfpathrectangle{\pgfqpoint{0.100000in}{0.212622in}}{\pgfqpoint{3.696000in}{3.696000in}}%
\pgfusepath{clip}%
\pgfsetrectcap%
\pgfsetroundjoin%
\pgfsetlinewidth{1.505625pt}%
\definecolor{currentstroke}{rgb}{1.000000,0.000000,0.000000}%
\pgfsetstrokecolor{currentstroke}%
\pgfsetdash{}{0pt}%
\pgfpathmoveto{\pgfqpoint{2.400035in}{1.840997in}}%
\pgfpathlineto{\pgfqpoint{2.331339in}{0.842232in}}%
\pgfusepath{stroke}%
\end{pgfscope}%
\begin{pgfscope}%
\pgfpathrectangle{\pgfqpoint{0.100000in}{0.212622in}}{\pgfqpoint{3.696000in}{3.696000in}}%
\pgfusepath{clip}%
\pgfsetrectcap%
\pgfsetroundjoin%
\pgfsetlinewidth{1.505625pt}%
\definecolor{currentstroke}{rgb}{1.000000,0.000000,0.000000}%
\pgfsetstrokecolor{currentstroke}%
\pgfsetdash{}{0pt}%
\pgfpathmoveto{\pgfqpoint{2.397947in}{1.839550in}}%
\pgfpathlineto{\pgfqpoint{2.331339in}{0.842232in}}%
\pgfusepath{stroke}%
\end{pgfscope}%
\begin{pgfscope}%
\pgfpathrectangle{\pgfqpoint{0.100000in}{0.212622in}}{\pgfqpoint{3.696000in}{3.696000in}}%
\pgfusepath{clip}%
\pgfsetrectcap%
\pgfsetroundjoin%
\pgfsetlinewidth{1.505625pt}%
\definecolor{currentstroke}{rgb}{1.000000,0.000000,0.000000}%
\pgfsetstrokecolor{currentstroke}%
\pgfsetdash{}{0pt}%
\pgfpathmoveto{\pgfqpoint{2.397132in}{1.839246in}}%
\pgfpathlineto{\pgfqpoint{2.331339in}{0.842232in}}%
\pgfusepath{stroke}%
\end{pgfscope}%
\begin{pgfscope}%
\pgfpathrectangle{\pgfqpoint{0.100000in}{0.212622in}}{\pgfqpoint{3.696000in}{3.696000in}}%
\pgfusepath{clip}%
\pgfsetrectcap%
\pgfsetroundjoin%
\pgfsetlinewidth{1.505625pt}%
\definecolor{currentstroke}{rgb}{1.000000,0.000000,0.000000}%
\pgfsetstrokecolor{currentstroke}%
\pgfsetdash{}{0pt}%
\pgfpathmoveto{\pgfqpoint{2.396539in}{1.838927in}}%
\pgfpathlineto{\pgfqpoint{2.331339in}{0.842232in}}%
\pgfusepath{stroke}%
\end{pgfscope}%
\begin{pgfscope}%
\pgfpathrectangle{\pgfqpoint{0.100000in}{0.212622in}}{\pgfqpoint{3.696000in}{3.696000in}}%
\pgfusepath{clip}%
\pgfsetrectcap%
\pgfsetroundjoin%
\pgfsetlinewidth{1.505625pt}%
\definecolor{currentstroke}{rgb}{1.000000,0.000000,0.000000}%
\pgfsetstrokecolor{currentstroke}%
\pgfsetdash{}{0pt}%
\pgfpathmoveto{\pgfqpoint{2.396201in}{1.838813in}}%
\pgfpathlineto{\pgfqpoint{2.331339in}{0.842232in}}%
\pgfusepath{stroke}%
\end{pgfscope}%
\begin{pgfscope}%
\pgfpathrectangle{\pgfqpoint{0.100000in}{0.212622in}}{\pgfqpoint{3.696000in}{3.696000in}}%
\pgfusepath{clip}%
\pgfsetrectcap%
\pgfsetroundjoin%
\pgfsetlinewidth{1.505625pt}%
\definecolor{currentstroke}{rgb}{1.000000,0.000000,0.000000}%
\pgfsetstrokecolor{currentstroke}%
\pgfsetdash{}{0pt}%
\pgfpathmoveto{\pgfqpoint{2.396049in}{1.838697in}}%
\pgfpathlineto{\pgfqpoint{2.331339in}{0.842232in}}%
\pgfusepath{stroke}%
\end{pgfscope}%
\begin{pgfscope}%
\pgfpathrectangle{\pgfqpoint{0.100000in}{0.212622in}}{\pgfqpoint{3.696000in}{3.696000in}}%
\pgfusepath{clip}%
\pgfsetrectcap%
\pgfsetroundjoin%
\pgfsetlinewidth{1.505625pt}%
\definecolor{currentstroke}{rgb}{1.000000,0.000000,0.000000}%
\pgfsetstrokecolor{currentstroke}%
\pgfsetdash{}{0pt}%
\pgfpathmoveto{\pgfqpoint{2.395944in}{1.838668in}}%
\pgfpathlineto{\pgfqpoint{2.331339in}{0.842232in}}%
\pgfusepath{stroke}%
\end{pgfscope}%
\begin{pgfscope}%
\pgfpathrectangle{\pgfqpoint{0.100000in}{0.212622in}}{\pgfqpoint{3.696000in}{3.696000in}}%
\pgfusepath{clip}%
\pgfsetrectcap%
\pgfsetroundjoin%
\pgfsetlinewidth{1.505625pt}%
\definecolor{currentstroke}{rgb}{1.000000,0.000000,0.000000}%
\pgfsetstrokecolor{currentstroke}%
\pgfsetdash{}{0pt}%
\pgfpathmoveto{\pgfqpoint{2.395893in}{1.838643in}}%
\pgfpathlineto{\pgfqpoint{2.331339in}{0.842232in}}%
\pgfusepath{stroke}%
\end{pgfscope}%
\begin{pgfscope}%
\pgfpathrectangle{\pgfqpoint{0.100000in}{0.212622in}}{\pgfqpoint{3.696000in}{3.696000in}}%
\pgfusepath{clip}%
\pgfsetrectcap%
\pgfsetroundjoin%
\pgfsetlinewidth{1.505625pt}%
\definecolor{currentstroke}{rgb}{1.000000,0.000000,0.000000}%
\pgfsetstrokecolor{currentstroke}%
\pgfsetdash{}{0pt}%
\pgfpathmoveto{\pgfqpoint{2.395864in}{1.838631in}}%
\pgfpathlineto{\pgfqpoint{2.331339in}{0.842232in}}%
\pgfusepath{stroke}%
\end{pgfscope}%
\begin{pgfscope}%
\pgfpathrectangle{\pgfqpoint{0.100000in}{0.212622in}}{\pgfqpoint{3.696000in}{3.696000in}}%
\pgfusepath{clip}%
\pgfsetrectcap%
\pgfsetroundjoin%
\pgfsetlinewidth{1.505625pt}%
\definecolor{currentstroke}{rgb}{1.000000,0.000000,0.000000}%
\pgfsetstrokecolor{currentstroke}%
\pgfsetdash{}{0pt}%
\pgfpathmoveto{\pgfqpoint{2.395847in}{1.838625in}}%
\pgfpathlineto{\pgfqpoint{2.331339in}{0.842232in}}%
\pgfusepath{stroke}%
\end{pgfscope}%
\begin{pgfscope}%
\pgfpathrectangle{\pgfqpoint{0.100000in}{0.212622in}}{\pgfqpoint{3.696000in}{3.696000in}}%
\pgfusepath{clip}%
\pgfsetrectcap%
\pgfsetroundjoin%
\pgfsetlinewidth{1.505625pt}%
\definecolor{currentstroke}{rgb}{1.000000,0.000000,0.000000}%
\pgfsetstrokecolor{currentstroke}%
\pgfsetdash{}{0pt}%
\pgfpathmoveto{\pgfqpoint{2.395837in}{1.838622in}}%
\pgfpathlineto{\pgfqpoint{2.331339in}{0.842232in}}%
\pgfusepath{stroke}%
\end{pgfscope}%
\begin{pgfscope}%
\pgfpathrectangle{\pgfqpoint{0.100000in}{0.212622in}}{\pgfqpoint{3.696000in}{3.696000in}}%
\pgfusepath{clip}%
\pgfsetrectcap%
\pgfsetroundjoin%
\pgfsetlinewidth{1.505625pt}%
\definecolor{currentstroke}{rgb}{1.000000,0.000000,0.000000}%
\pgfsetstrokecolor{currentstroke}%
\pgfsetdash{}{0pt}%
\pgfpathmoveto{\pgfqpoint{2.395831in}{1.838622in}}%
\pgfpathlineto{\pgfqpoint{2.331339in}{0.842232in}}%
\pgfusepath{stroke}%
\end{pgfscope}%
\begin{pgfscope}%
\pgfpathrectangle{\pgfqpoint{0.100000in}{0.212622in}}{\pgfqpoint{3.696000in}{3.696000in}}%
\pgfusepath{clip}%
\pgfsetrectcap%
\pgfsetroundjoin%
\pgfsetlinewidth{1.505625pt}%
\definecolor{currentstroke}{rgb}{1.000000,0.000000,0.000000}%
\pgfsetstrokecolor{currentstroke}%
\pgfsetdash{}{0pt}%
\pgfpathmoveto{\pgfqpoint{2.393989in}{1.838141in}}%
\pgfpathlineto{\pgfqpoint{2.331339in}{0.842232in}}%
\pgfusepath{stroke}%
\end{pgfscope}%
\begin{pgfscope}%
\pgfpathrectangle{\pgfqpoint{0.100000in}{0.212622in}}{\pgfqpoint{3.696000in}{3.696000in}}%
\pgfusepath{clip}%
\pgfsetrectcap%
\pgfsetroundjoin%
\pgfsetlinewidth{1.505625pt}%
\definecolor{currentstroke}{rgb}{1.000000,0.000000,0.000000}%
\pgfsetstrokecolor{currentstroke}%
\pgfsetdash{}{0pt}%
\pgfpathmoveto{\pgfqpoint{2.389510in}{1.837929in}}%
\pgfpathlineto{\pgfqpoint{2.331339in}{0.842232in}}%
\pgfusepath{stroke}%
\end{pgfscope}%
\begin{pgfscope}%
\pgfpathrectangle{\pgfqpoint{0.100000in}{0.212622in}}{\pgfqpoint{3.696000in}{3.696000in}}%
\pgfusepath{clip}%
\pgfsetrectcap%
\pgfsetroundjoin%
\pgfsetlinewidth{1.505625pt}%
\definecolor{currentstroke}{rgb}{1.000000,0.000000,0.000000}%
\pgfsetstrokecolor{currentstroke}%
\pgfsetdash{}{0pt}%
\pgfpathmoveto{\pgfqpoint{2.381163in}{1.836950in}}%
\pgfpathlineto{\pgfqpoint{2.331339in}{0.842232in}}%
\pgfusepath{stroke}%
\end{pgfscope}%
\begin{pgfscope}%
\pgfpathrectangle{\pgfqpoint{0.100000in}{0.212622in}}{\pgfqpoint{3.696000in}{3.696000in}}%
\pgfusepath{clip}%
\pgfsetrectcap%
\pgfsetroundjoin%
\pgfsetlinewidth{1.505625pt}%
\definecolor{currentstroke}{rgb}{1.000000,0.000000,0.000000}%
\pgfsetstrokecolor{currentstroke}%
\pgfsetdash{}{0pt}%
\pgfpathmoveto{\pgfqpoint{2.376496in}{1.837417in}}%
\pgfpathlineto{\pgfqpoint{2.331339in}{0.842232in}}%
\pgfusepath{stroke}%
\end{pgfscope}%
\begin{pgfscope}%
\pgfpathrectangle{\pgfqpoint{0.100000in}{0.212622in}}{\pgfqpoint{3.696000in}{3.696000in}}%
\pgfusepath{clip}%
\pgfsetrectcap%
\pgfsetroundjoin%
\pgfsetlinewidth{1.505625pt}%
\definecolor{currentstroke}{rgb}{1.000000,0.000000,0.000000}%
\pgfsetstrokecolor{currentstroke}%
\pgfsetdash{}{0pt}%
\pgfpathmoveto{\pgfqpoint{2.366282in}{1.839525in}}%
\pgfpathlineto{\pgfqpoint{2.331339in}{0.842232in}}%
\pgfusepath{stroke}%
\end{pgfscope}%
\begin{pgfscope}%
\pgfpathrectangle{\pgfqpoint{0.100000in}{0.212622in}}{\pgfqpoint{3.696000in}{3.696000in}}%
\pgfusepath{clip}%
\pgfsetrectcap%
\pgfsetroundjoin%
\pgfsetlinewidth{1.505625pt}%
\definecolor{currentstroke}{rgb}{1.000000,0.000000,0.000000}%
\pgfsetstrokecolor{currentstroke}%
\pgfsetdash{}{0pt}%
\pgfpathmoveto{\pgfqpoint{2.350975in}{1.845768in}}%
\pgfpathlineto{\pgfqpoint{2.331339in}{0.842232in}}%
\pgfusepath{stroke}%
\end{pgfscope}%
\begin{pgfscope}%
\pgfpathrectangle{\pgfqpoint{0.100000in}{0.212622in}}{\pgfqpoint{3.696000in}{3.696000in}}%
\pgfusepath{clip}%
\pgfsetrectcap%
\pgfsetroundjoin%
\pgfsetlinewidth{1.505625pt}%
\definecolor{currentstroke}{rgb}{1.000000,0.000000,0.000000}%
\pgfsetstrokecolor{currentstroke}%
\pgfsetdash{}{0pt}%
\pgfpathmoveto{\pgfqpoint{2.332392in}{1.860789in}}%
\pgfpathlineto{\pgfqpoint{2.331339in}{0.842232in}}%
\pgfusepath{stroke}%
\end{pgfscope}%
\begin{pgfscope}%
\pgfpathrectangle{\pgfqpoint{0.100000in}{0.212622in}}{\pgfqpoint{3.696000in}{3.696000in}}%
\pgfusepath{clip}%
\pgfsetrectcap%
\pgfsetroundjoin%
\pgfsetlinewidth{1.505625pt}%
\definecolor{currentstroke}{rgb}{1.000000,0.000000,0.000000}%
\pgfsetstrokecolor{currentstroke}%
\pgfsetdash{}{0pt}%
\pgfpathmoveto{\pgfqpoint{2.323888in}{1.870060in}}%
\pgfpathlineto{\pgfqpoint{2.331339in}{0.842232in}}%
\pgfusepath{stroke}%
\end{pgfscope}%
\begin{pgfscope}%
\pgfpathrectangle{\pgfqpoint{0.100000in}{0.212622in}}{\pgfqpoint{3.696000in}{3.696000in}}%
\pgfusepath{clip}%
\pgfsetrectcap%
\pgfsetroundjoin%
\pgfsetlinewidth{1.505625pt}%
\definecolor{currentstroke}{rgb}{1.000000,0.000000,0.000000}%
\pgfsetstrokecolor{currentstroke}%
\pgfsetdash{}{0pt}%
\pgfpathmoveto{\pgfqpoint{2.319176in}{1.874832in}}%
\pgfpathlineto{\pgfqpoint{2.331339in}{0.842232in}}%
\pgfusepath{stroke}%
\end{pgfscope}%
\begin{pgfscope}%
\pgfpathrectangle{\pgfqpoint{0.100000in}{0.212622in}}{\pgfqpoint{3.696000in}{3.696000in}}%
\pgfusepath{clip}%
\pgfsetrectcap%
\pgfsetroundjoin%
\pgfsetlinewidth{1.505625pt}%
\definecolor{currentstroke}{rgb}{1.000000,0.000000,0.000000}%
\pgfsetstrokecolor{currentstroke}%
\pgfsetdash{}{0pt}%
\pgfpathmoveto{\pgfqpoint{2.316733in}{1.877727in}}%
\pgfpathlineto{\pgfqpoint{2.331339in}{0.842232in}}%
\pgfusepath{stroke}%
\end{pgfscope}%
\begin{pgfscope}%
\pgfpathrectangle{\pgfqpoint{0.100000in}{0.212622in}}{\pgfqpoint{3.696000in}{3.696000in}}%
\pgfusepath{clip}%
\pgfsetrectcap%
\pgfsetroundjoin%
\pgfsetlinewidth{1.505625pt}%
\definecolor{currentstroke}{rgb}{1.000000,0.000000,0.000000}%
\pgfsetstrokecolor{currentstroke}%
\pgfsetdash{}{0pt}%
\pgfpathmoveto{\pgfqpoint{2.310735in}{1.883273in}}%
\pgfpathlineto{\pgfqpoint{2.316536in}{0.846935in}}%
\pgfusepath{stroke}%
\end{pgfscope}%
\begin{pgfscope}%
\pgfpathrectangle{\pgfqpoint{0.100000in}{0.212622in}}{\pgfqpoint{3.696000in}{3.696000in}}%
\pgfusepath{clip}%
\pgfsetrectcap%
\pgfsetroundjoin%
\pgfsetlinewidth{1.505625pt}%
\definecolor{currentstroke}{rgb}{1.000000,0.000000,0.000000}%
\pgfsetstrokecolor{currentstroke}%
\pgfsetdash{}{0pt}%
\pgfpathmoveto{\pgfqpoint{2.306833in}{1.884961in}}%
\pgfpathlineto{\pgfqpoint{2.316536in}{0.846935in}}%
\pgfusepath{stroke}%
\end{pgfscope}%
\begin{pgfscope}%
\pgfpathrectangle{\pgfqpoint{0.100000in}{0.212622in}}{\pgfqpoint{3.696000in}{3.696000in}}%
\pgfusepath{clip}%
\pgfsetrectcap%
\pgfsetroundjoin%
\pgfsetlinewidth{1.505625pt}%
\definecolor{currentstroke}{rgb}{1.000000,0.000000,0.000000}%
\pgfsetstrokecolor{currentstroke}%
\pgfsetdash{}{0pt}%
\pgfpathmoveto{\pgfqpoint{2.304804in}{1.886362in}}%
\pgfpathlineto{\pgfqpoint{2.316536in}{0.846935in}}%
\pgfusepath{stroke}%
\end{pgfscope}%
\begin{pgfscope}%
\pgfpathrectangle{\pgfqpoint{0.100000in}{0.212622in}}{\pgfqpoint{3.696000in}{3.696000in}}%
\pgfusepath{clip}%
\pgfsetrectcap%
\pgfsetroundjoin%
\pgfsetlinewidth{1.505625pt}%
\definecolor{currentstroke}{rgb}{1.000000,0.000000,0.000000}%
\pgfsetstrokecolor{currentstroke}%
\pgfsetdash{}{0pt}%
\pgfpathmoveto{\pgfqpoint{2.296007in}{1.889448in}}%
\pgfpathlineto{\pgfqpoint{2.301745in}{0.851636in}}%
\pgfusepath{stroke}%
\end{pgfscope}%
\begin{pgfscope}%
\pgfpathrectangle{\pgfqpoint{0.100000in}{0.212622in}}{\pgfqpoint{3.696000in}{3.696000in}}%
\pgfusepath{clip}%
\pgfsetrectcap%
\pgfsetroundjoin%
\pgfsetlinewidth{1.505625pt}%
\definecolor{currentstroke}{rgb}{1.000000,0.000000,0.000000}%
\pgfsetstrokecolor{currentstroke}%
\pgfsetdash{}{0pt}%
\pgfpathmoveto{\pgfqpoint{2.283786in}{1.895577in}}%
\pgfpathlineto{\pgfqpoint{2.301745in}{0.851636in}}%
\pgfusepath{stroke}%
\end{pgfscope}%
\begin{pgfscope}%
\pgfpathrectangle{\pgfqpoint{0.100000in}{0.212622in}}{\pgfqpoint{3.696000in}{3.696000in}}%
\pgfusepath{clip}%
\pgfsetrectcap%
\pgfsetroundjoin%
\pgfsetlinewidth{1.505625pt}%
\definecolor{currentstroke}{rgb}{1.000000,0.000000,0.000000}%
\pgfsetstrokecolor{currentstroke}%
\pgfsetdash{}{0pt}%
\pgfpathmoveto{\pgfqpoint{2.262824in}{1.905248in}}%
\pgfpathlineto{\pgfqpoint{2.272195in}{0.861026in}}%
\pgfusepath{stroke}%
\end{pgfscope}%
\begin{pgfscope}%
\pgfpathrectangle{\pgfqpoint{0.100000in}{0.212622in}}{\pgfqpoint{3.696000in}{3.696000in}}%
\pgfusepath{clip}%
\pgfsetrectcap%
\pgfsetroundjoin%
\pgfsetlinewidth{1.505625pt}%
\definecolor{currentstroke}{rgb}{1.000000,0.000000,0.000000}%
\pgfsetstrokecolor{currentstroke}%
\pgfsetdash{}{0pt}%
\pgfpathmoveto{\pgfqpoint{2.236776in}{1.920208in}}%
\pgfpathlineto{\pgfqpoint{2.242688in}{0.870402in}}%
\pgfusepath{stroke}%
\end{pgfscope}%
\begin{pgfscope}%
\pgfpathrectangle{\pgfqpoint{0.100000in}{0.212622in}}{\pgfqpoint{3.696000in}{3.696000in}}%
\pgfusepath{clip}%
\pgfsetrectcap%
\pgfsetroundjoin%
\pgfsetlinewidth{1.505625pt}%
\definecolor{currentstroke}{rgb}{1.000000,0.000000,0.000000}%
\pgfsetstrokecolor{currentstroke}%
\pgfsetdash{}{0pt}%
\pgfpathmoveto{\pgfqpoint{2.207074in}{1.933096in}}%
\pgfpathlineto{\pgfqpoint{2.213224in}{0.879765in}}%
\pgfusepath{stroke}%
\end{pgfscope}%
\begin{pgfscope}%
\pgfpathrectangle{\pgfqpoint{0.100000in}{0.212622in}}{\pgfqpoint{3.696000in}{3.696000in}}%
\pgfusepath{clip}%
\pgfsetrectcap%
\pgfsetroundjoin%
\pgfsetlinewidth{1.505625pt}%
\definecolor{currentstroke}{rgb}{1.000000,0.000000,0.000000}%
\pgfsetstrokecolor{currentstroke}%
\pgfsetdash{}{0pt}%
\pgfpathmoveto{\pgfqpoint{2.188848in}{1.939449in}}%
\pgfpathlineto{\pgfqpoint{2.198509in}{0.884441in}}%
\pgfusepath{stroke}%
\end{pgfscope}%
\begin{pgfscope}%
\pgfpathrectangle{\pgfqpoint{0.100000in}{0.212622in}}{\pgfqpoint{3.696000in}{3.696000in}}%
\pgfusepath{clip}%
\pgfsetrectcap%
\pgfsetroundjoin%
\pgfsetlinewidth{1.505625pt}%
\definecolor{currentstroke}{rgb}{1.000000,0.000000,0.000000}%
\pgfsetstrokecolor{currentstroke}%
\pgfsetdash{}{0pt}%
\pgfpathmoveto{\pgfqpoint{2.166009in}{1.954595in}}%
\pgfpathlineto{\pgfqpoint{2.183804in}{0.889113in}}%
\pgfusepath{stroke}%
\end{pgfscope}%
\begin{pgfscope}%
\pgfpathrectangle{\pgfqpoint{0.100000in}{0.212622in}}{\pgfqpoint{3.696000in}{3.696000in}}%
\pgfusepath{clip}%
\pgfsetrectcap%
\pgfsetroundjoin%
\pgfsetlinewidth{1.505625pt}%
\definecolor{currentstroke}{rgb}{1.000000,0.000000,0.000000}%
\pgfsetstrokecolor{currentstroke}%
\pgfsetdash{}{0pt}%
\pgfpathmoveto{\pgfqpoint{2.142868in}{1.958735in}}%
\pgfpathlineto{\pgfqpoint{2.154426in}{0.898449in}}%
\pgfusepath{stroke}%
\end{pgfscope}%
\begin{pgfscope}%
\pgfpathrectangle{\pgfqpoint{0.100000in}{0.212622in}}{\pgfqpoint{3.696000in}{3.696000in}}%
\pgfusepath{clip}%
\pgfsetrectcap%
\pgfsetroundjoin%
\pgfsetlinewidth{1.505625pt}%
\definecolor{currentstroke}{rgb}{1.000000,0.000000,0.000000}%
\pgfsetstrokecolor{currentstroke}%
\pgfsetdash{}{0pt}%
\pgfpathmoveto{\pgfqpoint{2.128507in}{1.968681in}}%
\pgfpathlineto{\pgfqpoint{2.139754in}{0.903111in}}%
\pgfusepath{stroke}%
\end{pgfscope}%
\begin{pgfscope}%
\pgfpathrectangle{\pgfqpoint{0.100000in}{0.212622in}}{\pgfqpoint{3.696000in}{3.696000in}}%
\pgfusepath{clip}%
\pgfsetrectcap%
\pgfsetroundjoin%
\pgfsetlinewidth{1.505625pt}%
\definecolor{currentstroke}{rgb}{1.000000,0.000000,0.000000}%
\pgfsetstrokecolor{currentstroke}%
\pgfsetdash{}{0pt}%
\pgfpathmoveto{\pgfqpoint{2.121227in}{1.970050in}}%
\pgfpathlineto{\pgfqpoint{2.125092in}{0.907770in}}%
\pgfusepath{stroke}%
\end{pgfscope}%
\begin{pgfscope}%
\pgfpathrectangle{\pgfqpoint{0.100000in}{0.212622in}}{\pgfqpoint{3.696000in}{3.696000in}}%
\pgfusepath{clip}%
\pgfsetrectcap%
\pgfsetroundjoin%
\pgfsetlinewidth{1.505625pt}%
\definecolor{currentstroke}{rgb}{1.000000,0.000000,0.000000}%
\pgfsetstrokecolor{currentstroke}%
\pgfsetdash{}{0pt}%
\pgfpathmoveto{\pgfqpoint{2.111102in}{1.974194in}}%
\pgfpathlineto{\pgfqpoint{2.125092in}{0.907770in}}%
\pgfusepath{stroke}%
\end{pgfscope}%
\begin{pgfscope}%
\pgfpathrectangle{\pgfqpoint{0.100000in}{0.212622in}}{\pgfqpoint{3.696000in}{3.696000in}}%
\pgfusepath{clip}%
\pgfsetrectcap%
\pgfsetroundjoin%
\pgfsetlinewidth{1.505625pt}%
\definecolor{currentstroke}{rgb}{1.000000,0.000000,0.000000}%
\pgfsetstrokecolor{currentstroke}%
\pgfsetdash{}{0pt}%
\pgfpathmoveto{\pgfqpoint{2.097616in}{1.978420in}}%
\pgfpathlineto{\pgfqpoint{2.110441in}{0.912426in}}%
\pgfusepath{stroke}%
\end{pgfscope}%
\begin{pgfscope}%
\pgfpathrectangle{\pgfqpoint{0.100000in}{0.212622in}}{\pgfqpoint{3.696000in}{3.696000in}}%
\pgfusepath{clip}%
\pgfsetrectcap%
\pgfsetroundjoin%
\pgfsetlinewidth{1.505625pt}%
\definecolor{currentstroke}{rgb}{1.000000,0.000000,0.000000}%
\pgfsetstrokecolor{currentstroke}%
\pgfsetdash{}{0pt}%
\pgfpathmoveto{\pgfqpoint{2.079999in}{1.985564in}}%
\pgfpathlineto{\pgfqpoint{2.095800in}{0.917078in}}%
\pgfusepath{stroke}%
\end{pgfscope}%
\begin{pgfscope}%
\pgfpathrectangle{\pgfqpoint{0.100000in}{0.212622in}}{\pgfqpoint{3.696000in}{3.696000in}}%
\pgfusepath{clip}%
\pgfsetrectcap%
\pgfsetroundjoin%
\pgfsetlinewidth{1.505625pt}%
\definecolor{currentstroke}{rgb}{1.000000,0.000000,0.000000}%
\pgfsetstrokecolor{currentstroke}%
\pgfsetdash{}{0pt}%
\pgfpathmoveto{\pgfqpoint{2.054452in}{1.996968in}}%
\pgfpathlineto{\pgfqpoint{2.066552in}{0.926372in}}%
\pgfusepath{stroke}%
\end{pgfscope}%
\begin{pgfscope}%
\pgfpathrectangle{\pgfqpoint{0.100000in}{0.212622in}}{\pgfqpoint{3.696000in}{3.696000in}}%
\pgfusepath{clip}%
\pgfsetrectcap%
\pgfsetroundjoin%
\pgfsetlinewidth{1.505625pt}%
\definecolor{currentstroke}{rgb}{1.000000,0.000000,0.000000}%
\pgfsetstrokecolor{currentstroke}%
\pgfsetdash{}{0pt}%
\pgfpathmoveto{\pgfqpoint{2.020791in}{2.010327in}}%
\pgfpathlineto{\pgfqpoint{2.037345in}{0.935653in}}%
\pgfusepath{stroke}%
\end{pgfscope}%
\begin{pgfscope}%
\pgfpathrectangle{\pgfqpoint{0.100000in}{0.212622in}}{\pgfqpoint{3.696000in}{3.696000in}}%
\pgfusepath{clip}%
\pgfsetrectcap%
\pgfsetroundjoin%
\pgfsetlinewidth{1.505625pt}%
\definecolor{currentstroke}{rgb}{1.000000,0.000000,0.000000}%
\pgfsetstrokecolor{currentstroke}%
\pgfsetdash{}{0pt}%
\pgfpathmoveto{\pgfqpoint{1.982786in}{2.035519in}}%
\pgfpathlineto{\pgfqpoint{1.993616in}{0.949549in}}%
\pgfusepath{stroke}%
\end{pgfscope}%
\begin{pgfscope}%
\pgfpathrectangle{\pgfqpoint{0.100000in}{0.212622in}}{\pgfqpoint{3.696000in}{3.696000in}}%
\pgfusepath{clip}%
\pgfsetrectcap%
\pgfsetroundjoin%
\pgfsetlinewidth{1.505625pt}%
\definecolor{currentstroke}{rgb}{1.000000,0.000000,0.000000}%
\pgfsetstrokecolor{currentstroke}%
\pgfsetdash{}{0pt}%
\pgfpathmoveto{\pgfqpoint{1.947006in}{2.052233in}}%
\pgfpathlineto{\pgfqpoint{1.964516in}{0.958796in}}%
\pgfusepath{stroke}%
\end{pgfscope}%
\begin{pgfscope}%
\pgfpathrectangle{\pgfqpoint{0.100000in}{0.212622in}}{\pgfqpoint{3.696000in}{3.696000in}}%
\pgfusepath{clip}%
\pgfsetrectcap%
\pgfsetroundjoin%
\pgfsetlinewidth{1.505625pt}%
\definecolor{currentstroke}{rgb}{1.000000,0.000000,0.000000}%
\pgfsetstrokecolor{currentstroke}%
\pgfsetdash{}{0pt}%
\pgfpathmoveto{\pgfqpoint{1.924767in}{2.067610in}}%
\pgfpathlineto{\pgfqpoint{1.935459in}{0.968029in}}%
\pgfusepath{stroke}%
\end{pgfscope}%
\begin{pgfscope}%
\pgfpathrectangle{\pgfqpoint{0.100000in}{0.212622in}}{\pgfqpoint{3.696000in}{3.696000in}}%
\pgfusepath{clip}%
\pgfsetrectcap%
\pgfsetroundjoin%
\pgfsetlinewidth{1.505625pt}%
\definecolor{currentstroke}{rgb}{1.000000,0.000000,0.000000}%
\pgfsetstrokecolor{currentstroke}%
\pgfsetdash{}{0pt}%
\pgfpathmoveto{\pgfqpoint{1.913105in}{2.071860in}}%
\pgfpathlineto{\pgfqpoint{1.920946in}{0.972641in}}%
\pgfusepath{stroke}%
\end{pgfscope}%
\begin{pgfscope}%
\pgfpathrectangle{\pgfqpoint{0.100000in}{0.212622in}}{\pgfqpoint{3.696000in}{3.696000in}}%
\pgfusepath{clip}%
\pgfsetrectcap%
\pgfsetroundjoin%
\pgfsetlinewidth{1.505625pt}%
\definecolor{currentstroke}{rgb}{1.000000,0.000000,0.000000}%
\pgfsetstrokecolor{currentstroke}%
\pgfsetdash{}{0pt}%
\pgfpathmoveto{\pgfqpoint{1.906427in}{2.075747in}}%
\pgfpathlineto{\pgfqpoint{1.920946in}{0.972641in}}%
\pgfusepath{stroke}%
\end{pgfscope}%
\begin{pgfscope}%
\pgfpathrectangle{\pgfqpoint{0.100000in}{0.212622in}}{\pgfqpoint{3.696000in}{3.696000in}}%
\pgfusepath{clip}%
\pgfsetrectcap%
\pgfsetroundjoin%
\pgfsetlinewidth{1.505625pt}%
\definecolor{currentstroke}{rgb}{1.000000,0.000000,0.000000}%
\pgfsetstrokecolor{currentstroke}%
\pgfsetdash{}{0pt}%
\pgfpathmoveto{\pgfqpoint{1.895117in}{2.080024in}}%
\pgfpathlineto{\pgfqpoint{1.906444in}{0.977249in}}%
\pgfusepath{stroke}%
\end{pgfscope}%
\begin{pgfscope}%
\pgfpathrectangle{\pgfqpoint{0.100000in}{0.212622in}}{\pgfqpoint{3.696000in}{3.696000in}}%
\pgfusepath{clip}%
\pgfsetrectcap%
\pgfsetroundjoin%
\pgfsetlinewidth{1.505625pt}%
\definecolor{currentstroke}{rgb}{1.000000,0.000000,0.000000}%
\pgfsetstrokecolor{currentstroke}%
\pgfsetdash{}{0pt}%
\pgfpathmoveto{\pgfqpoint{1.879480in}{2.090704in}}%
\pgfpathlineto{\pgfqpoint{1.891952in}{0.981854in}}%
\pgfusepath{stroke}%
\end{pgfscope}%
\begin{pgfscope}%
\pgfpathrectangle{\pgfqpoint{0.100000in}{0.212622in}}{\pgfqpoint{3.696000in}{3.696000in}}%
\pgfusepath{clip}%
\pgfsetrectcap%
\pgfsetroundjoin%
\pgfsetlinewidth{1.505625pt}%
\definecolor{currentstroke}{rgb}{1.000000,0.000000,0.000000}%
\pgfsetstrokecolor{currentstroke}%
\pgfsetdash{}{0pt}%
\pgfpathmoveto{\pgfqpoint{1.861053in}{2.095401in}}%
\pgfpathlineto{\pgfqpoint{1.877471in}{0.986456in}}%
\pgfusepath{stroke}%
\end{pgfscope}%
\begin{pgfscope}%
\pgfpathrectangle{\pgfqpoint{0.100000in}{0.212622in}}{\pgfqpoint{3.696000in}{3.696000in}}%
\pgfusepath{clip}%
\pgfsetrectcap%
\pgfsetroundjoin%
\pgfsetlinewidth{1.505625pt}%
\definecolor{currentstroke}{rgb}{1.000000,0.000000,0.000000}%
\pgfsetstrokecolor{currentstroke}%
\pgfsetdash{}{0pt}%
\pgfpathmoveto{\pgfqpoint{1.833729in}{2.115909in}}%
\pgfpathlineto{\pgfqpoint{1.848540in}{0.995649in}}%
\pgfusepath{stroke}%
\end{pgfscope}%
\begin{pgfscope}%
\pgfpathrectangle{\pgfqpoint{0.100000in}{0.212622in}}{\pgfqpoint{3.696000in}{3.696000in}}%
\pgfusepath{clip}%
\pgfsetrectcap%
\pgfsetroundjoin%
\pgfsetlinewidth{1.505625pt}%
\definecolor{currentstroke}{rgb}{1.000000,0.000000,0.000000}%
\pgfsetstrokecolor{currentstroke}%
\pgfsetdash{}{0pt}%
\pgfpathmoveto{\pgfqpoint{1.806982in}{2.120212in}}%
\pgfpathlineto{\pgfqpoint{1.819651in}{1.004829in}}%
\pgfusepath{stroke}%
\end{pgfscope}%
\begin{pgfscope}%
\pgfpathrectangle{\pgfqpoint{0.100000in}{0.212622in}}{\pgfqpoint{3.696000in}{3.696000in}}%
\pgfusepath{clip}%
\pgfsetrectcap%
\pgfsetroundjoin%
\pgfsetlinewidth{1.505625pt}%
\definecolor{currentstroke}{rgb}{1.000000,0.000000,0.000000}%
\pgfsetstrokecolor{currentstroke}%
\pgfsetdash{}{0pt}%
\pgfpathmoveto{\pgfqpoint{1.788430in}{2.132043in}}%
\pgfpathlineto{\pgfqpoint{1.805222in}{1.009414in}}%
\pgfusepath{stroke}%
\end{pgfscope}%
\begin{pgfscope}%
\pgfpathrectangle{\pgfqpoint{0.100000in}{0.212622in}}{\pgfqpoint{3.696000in}{3.696000in}}%
\pgfusepath{clip}%
\pgfsetrectcap%
\pgfsetroundjoin%
\pgfsetlinewidth{1.505625pt}%
\definecolor{currentstroke}{rgb}{1.000000,0.000000,0.000000}%
\pgfsetstrokecolor{currentstroke}%
\pgfsetdash{}{0pt}%
\pgfpathmoveto{\pgfqpoint{1.780026in}{2.133867in}}%
\pgfpathlineto{\pgfqpoint{1.790804in}{1.013996in}}%
\pgfusepath{stroke}%
\end{pgfscope}%
\begin{pgfscope}%
\pgfpathrectangle{\pgfqpoint{0.100000in}{0.212622in}}{\pgfqpoint{3.696000in}{3.696000in}}%
\pgfusepath{clip}%
\pgfsetrectcap%
\pgfsetroundjoin%
\pgfsetlinewidth{1.505625pt}%
\definecolor{currentstroke}{rgb}{1.000000,0.000000,0.000000}%
\pgfsetstrokecolor{currentstroke}%
\pgfsetdash{}{0pt}%
\pgfpathmoveto{\pgfqpoint{1.767935in}{2.141397in}}%
\pgfpathlineto{\pgfqpoint{1.790804in}{1.013996in}}%
\pgfusepath{stroke}%
\end{pgfscope}%
\begin{pgfscope}%
\pgfpathrectangle{\pgfqpoint{0.100000in}{0.212622in}}{\pgfqpoint{3.696000in}{3.696000in}}%
\pgfusepath{clip}%
\pgfsetrectcap%
\pgfsetroundjoin%
\pgfsetlinewidth{1.505625pt}%
\definecolor{currentstroke}{rgb}{1.000000,0.000000,0.000000}%
\pgfsetstrokecolor{currentstroke}%
\pgfsetdash{}{0pt}%
\pgfpathmoveto{\pgfqpoint{1.755069in}{2.149398in}}%
\pgfpathlineto{\pgfqpoint{1.776396in}{1.018574in}}%
\pgfusepath{stroke}%
\end{pgfscope}%
\begin{pgfscope}%
\pgfpathrectangle{\pgfqpoint{0.100000in}{0.212622in}}{\pgfqpoint{3.696000in}{3.696000in}}%
\pgfusepath{clip}%
\pgfsetrectcap%
\pgfsetroundjoin%
\pgfsetlinewidth{1.505625pt}%
\definecolor{currentstroke}{rgb}{1.000000,0.000000,0.000000}%
\pgfsetstrokecolor{currentstroke}%
\pgfsetdash{}{0pt}%
\pgfpathmoveto{\pgfqpoint{1.740093in}{2.157378in}}%
\pgfpathlineto{\pgfqpoint{1.761999in}{1.023149in}}%
\pgfusepath{stroke}%
\end{pgfscope}%
\begin{pgfscope}%
\pgfpathrectangle{\pgfqpoint{0.100000in}{0.212622in}}{\pgfqpoint{3.696000in}{3.696000in}}%
\pgfusepath{clip}%
\pgfsetrectcap%
\pgfsetroundjoin%
\pgfsetlinewidth{1.505625pt}%
\definecolor{currentstroke}{rgb}{1.000000,0.000000,0.000000}%
\pgfsetstrokecolor{currentstroke}%
\pgfsetdash{}{0pt}%
\pgfpathmoveto{\pgfqpoint{1.720057in}{2.167601in}}%
\pgfpathlineto{\pgfqpoint{1.733235in}{1.032289in}}%
\pgfusepath{stroke}%
\end{pgfscope}%
\begin{pgfscope}%
\pgfpathrectangle{\pgfqpoint{0.100000in}{0.212622in}}{\pgfqpoint{3.696000in}{3.696000in}}%
\pgfusepath{clip}%
\pgfsetrectcap%
\pgfsetroundjoin%
\pgfsetlinewidth{1.505625pt}%
\definecolor{currentstroke}{rgb}{1.000000,0.000000,0.000000}%
\pgfsetstrokecolor{currentstroke}%
\pgfsetdash{}{0pt}%
\pgfpathmoveto{\pgfqpoint{1.696836in}{2.178847in}}%
\pgfpathlineto{\pgfqpoint{1.718869in}{1.036854in}}%
\pgfusepath{stroke}%
\end{pgfscope}%
\begin{pgfscope}%
\pgfpathrectangle{\pgfqpoint{0.100000in}{0.212622in}}{\pgfqpoint{3.696000in}{3.696000in}}%
\pgfusepath{clip}%
\pgfsetrectcap%
\pgfsetroundjoin%
\pgfsetlinewidth{1.505625pt}%
\definecolor{currentstroke}{rgb}{1.000000,0.000000,0.000000}%
\pgfsetstrokecolor{currentstroke}%
\pgfsetdash{}{0pt}%
\pgfpathmoveto{\pgfqpoint{1.666674in}{2.198950in}}%
\pgfpathlineto{\pgfqpoint{1.690168in}{1.045974in}}%
\pgfusepath{stroke}%
\end{pgfscope}%
\begin{pgfscope}%
\pgfpathrectangle{\pgfqpoint{0.100000in}{0.212622in}}{\pgfqpoint{3.696000in}{3.696000in}}%
\pgfusepath{clip}%
\pgfsetrectcap%
\pgfsetroundjoin%
\pgfsetlinewidth{1.505625pt}%
\definecolor{currentstroke}{rgb}{1.000000,0.000000,0.000000}%
\pgfsetstrokecolor{currentstroke}%
\pgfsetdash{}{0pt}%
\pgfpathmoveto{\pgfqpoint{1.633678in}{2.213323in}}%
\pgfpathlineto{\pgfqpoint{1.647194in}{1.059630in}}%
\pgfusepath{stroke}%
\end{pgfscope}%
\begin{pgfscope}%
\pgfpathrectangle{\pgfqpoint{0.100000in}{0.212622in}}{\pgfqpoint{3.696000in}{3.696000in}}%
\pgfusepath{clip}%
\pgfsetrectcap%
\pgfsetroundjoin%
\pgfsetlinewidth{1.505625pt}%
\definecolor{currentstroke}{rgb}{1.000000,0.000000,0.000000}%
\pgfsetstrokecolor{currentstroke}%
\pgfsetdash{}{0pt}%
\pgfpathmoveto{\pgfqpoint{1.592452in}{2.236842in}}%
\pgfpathlineto{\pgfqpoint{1.618596in}{1.068717in}}%
\pgfusepath{stroke}%
\end{pgfscope}%
\begin{pgfscope}%
\pgfpathrectangle{\pgfqpoint{0.100000in}{0.212622in}}{\pgfqpoint{3.696000in}{3.696000in}}%
\pgfusepath{clip}%
\pgfsetrectcap%
\pgfsetroundjoin%
\pgfsetlinewidth{1.505625pt}%
\definecolor{currentstroke}{rgb}{1.000000,0.000000,0.000000}%
\pgfsetstrokecolor{currentstroke}%
\pgfsetdash{}{0pt}%
\pgfpathmoveto{\pgfqpoint{1.551380in}{2.248832in}}%
\pgfpathlineto{\pgfqpoint{1.575778in}{1.082324in}}%
\pgfusepath{stroke}%
\end{pgfscope}%
\begin{pgfscope}%
\pgfpathrectangle{\pgfqpoint{0.100000in}{0.212622in}}{\pgfqpoint{3.696000in}{3.696000in}}%
\pgfusepath{clip}%
\pgfsetrectcap%
\pgfsetroundjoin%
\pgfsetlinewidth{1.505625pt}%
\definecolor{currentstroke}{rgb}{1.000000,0.000000,0.000000}%
\pgfsetstrokecolor{currentstroke}%
\pgfsetdash{}{0pt}%
\pgfpathmoveto{\pgfqpoint{1.502832in}{2.281767in}}%
\pgfpathlineto{\pgfqpoint{1.518830in}{1.100420in}}%
\pgfusepath{stroke}%
\end{pgfscope}%
\begin{pgfscope}%
\pgfpathrectangle{\pgfqpoint{0.100000in}{0.212622in}}{\pgfqpoint{3.696000in}{3.696000in}}%
\pgfusepath{clip}%
\pgfsetrectcap%
\pgfsetroundjoin%
\pgfsetlinewidth{1.505625pt}%
\definecolor{currentstroke}{rgb}{1.000000,0.000000,0.000000}%
\pgfsetstrokecolor{currentstroke}%
\pgfsetdash{}{0pt}%
\pgfpathmoveto{\pgfqpoint{1.479158in}{2.293545in}}%
\pgfpathlineto{\pgfqpoint{1.504618in}{1.104936in}}%
\pgfusepath{stroke}%
\end{pgfscope}%
\begin{pgfscope}%
\pgfpathrectangle{\pgfqpoint{0.100000in}{0.212622in}}{\pgfqpoint{3.696000in}{3.696000in}}%
\pgfusepath{clip}%
\pgfsetrectcap%
\pgfsetroundjoin%
\pgfsetlinewidth{1.505625pt}%
\definecolor{currentstroke}{rgb}{1.000000,0.000000,0.000000}%
\pgfsetstrokecolor{currentstroke}%
\pgfsetdash{}{0pt}%
\pgfpathmoveto{\pgfqpoint{1.465250in}{2.298229in}}%
\pgfpathlineto{\pgfqpoint{1.490417in}{1.109448in}}%
\pgfusepath{stroke}%
\end{pgfscope}%
\begin{pgfscope}%
\pgfpathrectangle{\pgfqpoint{0.100000in}{0.212622in}}{\pgfqpoint{3.696000in}{3.696000in}}%
\pgfusepath{clip}%
\pgfsetrectcap%
\pgfsetroundjoin%
\pgfsetlinewidth{1.505625pt}%
\definecolor{currentstroke}{rgb}{1.000000,0.000000,0.000000}%
\pgfsetstrokecolor{currentstroke}%
\pgfsetdash{}{0pt}%
\pgfpathmoveto{\pgfqpoint{1.448551in}{2.305796in}}%
\pgfpathlineto{\pgfqpoint{1.476226in}{1.113958in}}%
\pgfusepath{stroke}%
\end{pgfscope}%
\begin{pgfscope}%
\pgfpathrectangle{\pgfqpoint{0.100000in}{0.212622in}}{\pgfqpoint{3.696000in}{3.696000in}}%
\pgfusepath{clip}%
\pgfsetrectcap%
\pgfsetroundjoin%
\pgfsetlinewidth{1.505625pt}%
\definecolor{currentstroke}{rgb}{1.000000,0.000000,0.000000}%
\pgfsetstrokecolor{currentstroke}%
\pgfsetdash{}{0pt}%
\pgfpathmoveto{\pgfqpoint{1.428873in}{2.314579in}}%
\pgfpathlineto{\pgfqpoint{1.447875in}{1.122967in}}%
\pgfusepath{stroke}%
\end{pgfscope}%
\begin{pgfscope}%
\pgfpathrectangle{\pgfqpoint{0.100000in}{0.212622in}}{\pgfqpoint{3.696000in}{3.696000in}}%
\pgfusepath{clip}%
\pgfsetrectcap%
\pgfsetroundjoin%
\pgfsetlinewidth{1.505625pt}%
\definecolor{currentstroke}{rgb}{1.000000,0.000000,0.000000}%
\pgfsetstrokecolor{currentstroke}%
\pgfsetdash{}{0pt}%
\pgfpathmoveto{\pgfqpoint{1.405994in}{2.330454in}}%
\pgfpathlineto{\pgfqpoint{1.433714in}{1.127467in}}%
\pgfusepath{stroke}%
\end{pgfscope}%
\begin{pgfscope}%
\pgfpathrectangle{\pgfqpoint{0.100000in}{0.212622in}}{\pgfqpoint{3.696000in}{3.696000in}}%
\pgfusepath{clip}%
\pgfsetrectcap%
\pgfsetroundjoin%
\pgfsetlinewidth{1.505625pt}%
\definecolor{currentstroke}{rgb}{1.000000,0.000000,0.000000}%
\pgfsetstrokecolor{currentstroke}%
\pgfsetdash{}{0pt}%
\pgfpathmoveto{\pgfqpoint{1.375672in}{2.345826in}}%
\pgfpathlineto{\pgfqpoint{1.405424in}{1.136456in}}%
\pgfusepath{stroke}%
\end{pgfscope}%
\begin{pgfscope}%
\pgfpathrectangle{\pgfqpoint{0.100000in}{0.212622in}}{\pgfqpoint{3.696000in}{3.696000in}}%
\pgfusepath{clip}%
\pgfsetrectcap%
\pgfsetroundjoin%
\pgfsetlinewidth{1.505625pt}%
\definecolor{currentstroke}{rgb}{1.000000,0.000000,0.000000}%
\pgfsetstrokecolor{currentstroke}%
\pgfsetdash{}{0pt}%
\pgfpathmoveto{\pgfqpoint{1.340107in}{2.368208in}}%
\pgfpathlineto{\pgfqpoint{1.363065in}{1.149917in}}%
\pgfusepath{stroke}%
\end{pgfscope}%
\begin{pgfscope}%
\pgfpathrectangle{\pgfqpoint{0.100000in}{0.212622in}}{\pgfqpoint{3.696000in}{3.696000in}}%
\pgfusepath{clip}%
\pgfsetrectcap%
\pgfsetroundjoin%
\pgfsetlinewidth{1.505625pt}%
\definecolor{currentstroke}{rgb}{1.000000,0.000000,0.000000}%
\pgfsetstrokecolor{currentstroke}%
\pgfsetdash{}{0pt}%
\pgfpathmoveto{\pgfqpoint{1.303222in}{2.390575in}}%
\pgfpathlineto{\pgfqpoint{1.334876in}{1.158874in}}%
\pgfusepath{stroke}%
\end{pgfscope}%
\begin{pgfscope}%
\pgfpathrectangle{\pgfqpoint{0.100000in}{0.212622in}}{\pgfqpoint{3.696000in}{3.696000in}}%
\pgfusepath{clip}%
\pgfsetrectcap%
\pgfsetroundjoin%
\pgfsetlinewidth{1.505625pt}%
\definecolor{currentstroke}{rgb}{1.000000,0.000000,0.000000}%
\pgfsetstrokecolor{currentstroke}%
\pgfsetdash{}{0pt}%
\pgfpathmoveto{\pgfqpoint{1.282281in}{2.401600in}}%
\pgfpathlineto{\pgfqpoint{1.306728in}{1.167819in}}%
\pgfusepath{stroke}%
\end{pgfscope}%
\begin{pgfscope}%
\pgfpathrectangle{\pgfqpoint{0.100000in}{0.212622in}}{\pgfqpoint{3.696000in}{3.696000in}}%
\pgfusepath{clip}%
\pgfsetrectcap%
\pgfsetroundjoin%
\pgfsetlinewidth{1.505625pt}%
\definecolor{currentstroke}{rgb}{1.000000,0.000000,0.000000}%
\pgfsetstrokecolor{currentstroke}%
\pgfsetdash{}{0pt}%
\pgfpathmoveto{\pgfqpoint{1.271026in}{2.408322in}}%
\pgfpathlineto{\pgfqpoint{1.292669in}{1.172286in}}%
\pgfusepath{stroke}%
\end{pgfscope}%
\begin{pgfscope}%
\pgfpathrectangle{\pgfqpoint{0.100000in}{0.212622in}}{\pgfqpoint{3.696000in}{3.696000in}}%
\pgfusepath{clip}%
\pgfsetrectcap%
\pgfsetroundjoin%
\pgfsetlinewidth{1.505625pt}%
\definecolor{currentstroke}{rgb}{1.000000,0.000000,0.000000}%
\pgfsetstrokecolor{currentstroke}%
\pgfsetdash{}{0pt}%
\pgfpathmoveto{\pgfqpoint{1.264901in}{2.409821in}}%
\pgfpathlineto{\pgfqpoint{1.292669in}{1.172286in}}%
\pgfusepath{stroke}%
\end{pgfscope}%
\begin{pgfscope}%
\pgfpathrectangle{\pgfqpoint{0.100000in}{0.212622in}}{\pgfqpoint{3.696000in}{3.696000in}}%
\pgfusepath{clip}%
\pgfsetrectcap%
\pgfsetroundjoin%
\pgfsetlinewidth{1.505625pt}%
\definecolor{currentstroke}{rgb}{1.000000,0.000000,0.000000}%
\pgfsetstrokecolor{currentstroke}%
\pgfsetdash{}{0pt}%
\pgfpathmoveto{\pgfqpoint{1.261205in}{2.412618in}}%
\pgfpathlineto{\pgfqpoint{1.292669in}{1.172286in}}%
\pgfusepath{stroke}%
\end{pgfscope}%
\begin{pgfscope}%
\pgfpathrectangle{\pgfqpoint{0.100000in}{0.212622in}}{\pgfqpoint{3.696000in}{3.696000in}}%
\pgfusepath{clip}%
\pgfsetrectcap%
\pgfsetroundjoin%
\pgfsetlinewidth{1.505625pt}%
\definecolor{currentstroke}{rgb}{1.000000,0.000000,0.000000}%
\pgfsetstrokecolor{currentstroke}%
\pgfsetdash{}{0pt}%
\pgfpathmoveto{\pgfqpoint{1.259359in}{2.412889in}}%
\pgfpathlineto{\pgfqpoint{1.292669in}{1.172286in}}%
\pgfusepath{stroke}%
\end{pgfscope}%
\begin{pgfscope}%
\pgfpathrectangle{\pgfqpoint{0.100000in}{0.212622in}}{\pgfqpoint{3.696000in}{3.696000in}}%
\pgfusepath{clip}%
\pgfsetrectcap%
\pgfsetroundjoin%
\pgfsetlinewidth{1.505625pt}%
\definecolor{currentstroke}{rgb}{1.000000,0.000000,0.000000}%
\pgfsetstrokecolor{currentstroke}%
\pgfsetdash{}{0pt}%
\pgfpathmoveto{\pgfqpoint{1.258229in}{2.413468in}}%
\pgfpathlineto{\pgfqpoint{1.292669in}{1.172286in}}%
\pgfusepath{stroke}%
\end{pgfscope}%
\begin{pgfscope}%
\pgfpathrectangle{\pgfqpoint{0.100000in}{0.212622in}}{\pgfqpoint{3.696000in}{3.696000in}}%
\pgfusepath{clip}%
\pgfsetrectcap%
\pgfsetroundjoin%
\pgfsetlinewidth{1.505625pt}%
\definecolor{currentstroke}{rgb}{1.000000,0.000000,0.000000}%
\pgfsetstrokecolor{currentstroke}%
\pgfsetdash{}{0pt}%
\pgfpathmoveto{\pgfqpoint{1.253137in}{2.414922in}}%
\pgfpathlineto{\pgfqpoint{1.278620in}{1.176751in}}%
\pgfusepath{stroke}%
\end{pgfscope}%
\begin{pgfscope}%
\pgfpathrectangle{\pgfqpoint{0.100000in}{0.212622in}}{\pgfqpoint{3.696000in}{3.696000in}}%
\pgfusepath{clip}%
\pgfsetrectcap%
\pgfsetroundjoin%
\pgfsetlinewidth{1.505625pt}%
\definecolor{currentstroke}{rgb}{1.000000,0.000000,0.000000}%
\pgfsetstrokecolor{currentstroke}%
\pgfsetdash{}{0pt}%
\pgfpathmoveto{\pgfqpoint{1.241896in}{2.420590in}}%
\pgfpathlineto{\pgfqpoint{1.278620in}{1.176751in}}%
\pgfusepath{stroke}%
\end{pgfscope}%
\begin{pgfscope}%
\pgfpathrectangle{\pgfqpoint{0.100000in}{0.212622in}}{\pgfqpoint{3.696000in}{3.696000in}}%
\pgfusepath{clip}%
\pgfsetrectcap%
\pgfsetroundjoin%
\pgfsetlinewidth{1.505625pt}%
\definecolor{currentstroke}{rgb}{1.000000,0.000000,0.000000}%
\pgfsetstrokecolor{currentstroke}%
\pgfsetdash{}{0pt}%
\pgfpathmoveto{\pgfqpoint{1.225639in}{2.426094in}}%
\pgfpathlineto{\pgfqpoint{1.250552in}{1.185670in}}%
\pgfusepath{stroke}%
\end{pgfscope}%
\begin{pgfscope}%
\pgfpathrectangle{\pgfqpoint{0.100000in}{0.212622in}}{\pgfqpoint{3.696000in}{3.696000in}}%
\pgfusepath{clip}%
\pgfsetrectcap%
\pgfsetroundjoin%
\pgfsetlinewidth{1.505625pt}%
\definecolor{currentstroke}{rgb}{1.000000,0.000000,0.000000}%
\pgfsetstrokecolor{currentstroke}%
\pgfsetdash{}{0pt}%
\pgfpathmoveto{\pgfqpoint{1.204895in}{2.437182in}}%
\pgfpathlineto{\pgfqpoint{1.236533in}{1.190124in}}%
\pgfusepath{stroke}%
\end{pgfscope}%
\begin{pgfscope}%
\pgfpathrectangle{\pgfqpoint{0.100000in}{0.212622in}}{\pgfqpoint{3.696000in}{3.696000in}}%
\pgfusepath{clip}%
\pgfsetrectcap%
\pgfsetroundjoin%
\pgfsetlinewidth{1.505625pt}%
\definecolor{currentstroke}{rgb}{1.000000,0.000000,0.000000}%
\pgfsetstrokecolor{currentstroke}%
\pgfsetdash{}{0pt}%
\pgfpathmoveto{\pgfqpoint{1.183636in}{2.446261in}}%
\pgfpathlineto{\pgfqpoint{1.222524in}{1.194576in}}%
\pgfusepath{stroke}%
\end{pgfscope}%
\begin{pgfscope}%
\pgfpathrectangle{\pgfqpoint{0.100000in}{0.212622in}}{\pgfqpoint{3.696000in}{3.696000in}}%
\pgfusepath{clip}%
\pgfsetrectcap%
\pgfsetroundjoin%
\pgfsetlinewidth{1.505625pt}%
\definecolor{currentstroke}{rgb}{1.000000,0.000000,0.000000}%
\pgfsetstrokecolor{currentstroke}%
\pgfsetdash{}{0pt}%
\pgfpathmoveto{\pgfqpoint{1.159633in}{2.458690in}}%
\pgfpathlineto{\pgfqpoint{1.194536in}{1.203469in}}%
\pgfusepath{stroke}%
\end{pgfscope}%
\begin{pgfscope}%
\pgfpathrectangle{\pgfqpoint{0.100000in}{0.212622in}}{\pgfqpoint{3.696000in}{3.696000in}}%
\pgfusepath{clip}%
\pgfsetrectcap%
\pgfsetroundjoin%
\pgfsetlinewidth{1.505625pt}%
\definecolor{currentstroke}{rgb}{1.000000,0.000000,0.000000}%
\pgfsetstrokecolor{currentstroke}%
\pgfsetdash{}{0pt}%
\pgfpathmoveto{\pgfqpoint{1.147167in}{2.465878in}}%
\pgfpathlineto{\pgfqpoint{1.180557in}{1.207912in}}%
\pgfusepath{stroke}%
\end{pgfscope}%
\begin{pgfscope}%
\pgfpathrectangle{\pgfqpoint{0.100000in}{0.212622in}}{\pgfqpoint{3.696000in}{3.696000in}}%
\pgfusepath{clip}%
\pgfsetrectcap%
\pgfsetroundjoin%
\pgfsetlinewidth{1.505625pt}%
\definecolor{currentstroke}{rgb}{1.000000,0.000000,0.000000}%
\pgfsetstrokecolor{currentstroke}%
\pgfsetdash{}{0pt}%
\pgfpathmoveto{\pgfqpoint{1.139823in}{2.469054in}}%
\pgfpathlineto{\pgfqpoint{1.180557in}{1.207912in}}%
\pgfusepath{stroke}%
\end{pgfscope}%
\begin{pgfscope}%
\pgfpathrectangle{\pgfqpoint{0.100000in}{0.212622in}}{\pgfqpoint{3.696000in}{3.696000in}}%
\pgfusepath{clip}%
\pgfsetrectcap%
\pgfsetroundjoin%
\pgfsetlinewidth{1.505625pt}%
\definecolor{currentstroke}{rgb}{1.000000,0.000000,0.000000}%
\pgfsetstrokecolor{currentstroke}%
\pgfsetdash{}{0pt}%
\pgfpathmoveto{\pgfqpoint{1.129259in}{2.472262in}}%
\pgfpathlineto{\pgfqpoint{1.166588in}{1.212350in}}%
\pgfusepath{stroke}%
\end{pgfscope}%
\begin{pgfscope}%
\pgfpathrectangle{\pgfqpoint{0.100000in}{0.212622in}}{\pgfqpoint{3.696000in}{3.696000in}}%
\pgfusepath{clip}%
\pgfsetrectcap%
\pgfsetroundjoin%
\pgfsetlinewidth{1.505625pt}%
\definecolor{currentstroke}{rgb}{1.000000,0.000000,0.000000}%
\pgfsetstrokecolor{currentstroke}%
\pgfsetdash{}{0pt}%
\pgfpathmoveto{\pgfqpoint{1.112046in}{2.481907in}}%
\pgfpathlineto{\pgfqpoint{1.152629in}{1.216786in}}%
\pgfusepath{stroke}%
\end{pgfscope}%
\begin{pgfscope}%
\pgfpathrectangle{\pgfqpoint{0.100000in}{0.212622in}}{\pgfqpoint{3.696000in}{3.696000in}}%
\pgfusepath{clip}%
\pgfsetrectcap%
\pgfsetroundjoin%
\pgfsetlinewidth{1.505625pt}%
\definecolor{currentstroke}{rgb}{1.000000,0.000000,0.000000}%
\pgfsetstrokecolor{currentstroke}%
\pgfsetdash{}{0pt}%
\pgfpathmoveto{\pgfqpoint{1.092741in}{2.491948in}}%
\pgfpathlineto{\pgfqpoint{1.124741in}{1.225648in}}%
\pgfusepath{stroke}%
\end{pgfscope}%
\begin{pgfscope}%
\pgfpathrectangle{\pgfqpoint{0.100000in}{0.212622in}}{\pgfqpoint{3.696000in}{3.696000in}}%
\pgfusepath{clip}%
\pgfsetrectcap%
\pgfsetroundjoin%
\pgfsetlinewidth{1.505625pt}%
\definecolor{currentstroke}{rgb}{1.000000,0.000000,0.000000}%
\pgfsetstrokecolor{currentstroke}%
\pgfsetdash{}{0pt}%
\pgfpathmoveto{\pgfqpoint{1.069656in}{2.503877in}}%
\pgfpathlineto{\pgfqpoint{1.110812in}{1.230074in}}%
\pgfusepath{stroke}%
\end{pgfscope}%
\begin{pgfscope}%
\pgfpathrectangle{\pgfqpoint{0.100000in}{0.212622in}}{\pgfqpoint{3.696000in}{3.696000in}}%
\pgfusepath{clip}%
\pgfsetrectcap%
\pgfsetroundjoin%
\pgfsetlinewidth{1.505625pt}%
\definecolor{currentstroke}{rgb}{1.000000,0.000000,0.000000}%
\pgfsetstrokecolor{currentstroke}%
\pgfsetdash{}{0pt}%
\pgfpathmoveto{\pgfqpoint{1.057717in}{2.506458in}}%
\pgfpathlineto{\pgfqpoint{1.096893in}{1.234497in}}%
\pgfusepath{stroke}%
\end{pgfscope}%
\begin{pgfscope}%
\pgfpathrectangle{\pgfqpoint{0.100000in}{0.212622in}}{\pgfqpoint{3.696000in}{3.696000in}}%
\pgfusepath{clip}%
\pgfsetrectcap%
\pgfsetroundjoin%
\pgfsetlinewidth{1.505625pt}%
\definecolor{currentstroke}{rgb}{1.000000,0.000000,0.000000}%
\pgfsetstrokecolor{currentstroke}%
\pgfsetdash{}{0pt}%
\pgfpathmoveto{\pgfqpoint{1.050512in}{2.509772in}}%
\pgfpathlineto{\pgfqpoint{1.082984in}{1.238917in}}%
\pgfusepath{stroke}%
\end{pgfscope}%
\begin{pgfscope}%
\pgfpathrectangle{\pgfqpoint{0.100000in}{0.212622in}}{\pgfqpoint{3.696000in}{3.696000in}}%
\pgfusepath{clip}%
\pgfsetrectcap%
\pgfsetroundjoin%
\pgfsetlinewidth{1.505625pt}%
\definecolor{currentstroke}{rgb}{1.000000,0.000000,0.000000}%
\pgfsetstrokecolor{currentstroke}%
\pgfsetdash{}{0pt}%
\pgfpathmoveto{\pgfqpoint{1.046932in}{2.511268in}}%
\pgfpathlineto{\pgfqpoint{1.082984in}{1.238917in}}%
\pgfusepath{stroke}%
\end{pgfscope}%
\begin{pgfscope}%
\pgfpathrectangle{\pgfqpoint{0.100000in}{0.212622in}}{\pgfqpoint{3.696000in}{3.696000in}}%
\pgfusepath{clip}%
\pgfsetrectcap%
\pgfsetroundjoin%
\pgfsetlinewidth{1.505625pt}%
\definecolor{currentstroke}{rgb}{1.000000,0.000000,0.000000}%
\pgfsetstrokecolor{currentstroke}%
\pgfsetdash{}{0pt}%
\pgfpathmoveto{\pgfqpoint{1.038550in}{2.516092in}}%
\pgfpathlineto{\pgfqpoint{1.069084in}{1.243334in}}%
\pgfusepath{stroke}%
\end{pgfscope}%
\begin{pgfscope}%
\pgfpathrectangle{\pgfqpoint{0.100000in}{0.212622in}}{\pgfqpoint{3.696000in}{3.696000in}}%
\pgfusepath{clip}%
\pgfsetrectcap%
\pgfsetroundjoin%
\pgfsetlinewidth{1.505625pt}%
\definecolor{currentstroke}{rgb}{1.000000,0.000000,0.000000}%
\pgfsetstrokecolor{currentstroke}%
\pgfsetdash{}{0pt}%
\pgfpathmoveto{\pgfqpoint{1.025719in}{2.519746in}}%
\pgfpathlineto{\pgfqpoint{1.069084in}{1.243334in}}%
\pgfusepath{stroke}%
\end{pgfscope}%
\begin{pgfscope}%
\pgfpathrectangle{\pgfqpoint{0.100000in}{0.212622in}}{\pgfqpoint{3.696000in}{3.696000in}}%
\pgfusepath{clip}%
\pgfsetrectcap%
\pgfsetroundjoin%
\pgfsetlinewidth{1.505625pt}%
\definecolor{currentstroke}{rgb}{1.000000,0.000000,0.000000}%
\pgfsetstrokecolor{currentstroke}%
\pgfsetdash{}{0pt}%
\pgfpathmoveto{\pgfqpoint{1.006031in}{2.530981in}}%
\pgfpathlineto{\pgfqpoint{1.041315in}{1.252158in}}%
\pgfusepath{stroke}%
\end{pgfscope}%
\begin{pgfscope}%
\pgfpathrectangle{\pgfqpoint{0.100000in}{0.212622in}}{\pgfqpoint{3.696000in}{3.696000in}}%
\pgfusepath{clip}%
\pgfsetrectcap%
\pgfsetroundjoin%
\pgfsetlinewidth{1.505625pt}%
\definecolor{currentstroke}{rgb}{1.000000,0.000000,0.000000}%
\pgfsetstrokecolor{currentstroke}%
\pgfsetdash{}{0pt}%
\pgfpathmoveto{\pgfqpoint{0.982843in}{2.543464in}}%
\pgfpathlineto{\pgfqpoint{1.027446in}{1.256565in}}%
\pgfusepath{stroke}%
\end{pgfscope}%
\begin{pgfscope}%
\pgfpathrectangle{\pgfqpoint{0.100000in}{0.212622in}}{\pgfqpoint{3.696000in}{3.696000in}}%
\pgfusepath{clip}%
\pgfsetrectcap%
\pgfsetroundjoin%
\pgfsetlinewidth{1.505625pt}%
\definecolor{currentstroke}{rgb}{1.000000,0.000000,0.000000}%
\pgfsetstrokecolor{currentstroke}%
\pgfsetdash{}{0pt}%
\pgfpathmoveto{\pgfqpoint{0.969254in}{2.553983in}}%
\pgfpathlineto{\pgfqpoint{1.013586in}{1.260969in}}%
\pgfusepath{stroke}%
\end{pgfscope}%
\begin{pgfscope}%
\pgfpathrectangle{\pgfqpoint{0.100000in}{0.212622in}}{\pgfqpoint{3.696000in}{3.696000in}}%
\pgfusepath{clip}%
\pgfsetrectcap%
\pgfsetroundjoin%
\pgfsetlinewidth{1.505625pt}%
\definecolor{currentstroke}{rgb}{1.000000,0.000000,0.000000}%
\pgfsetstrokecolor{currentstroke}%
\pgfsetdash{}{0pt}%
\pgfpathmoveto{\pgfqpoint{0.963187in}{2.557802in}}%
\pgfpathlineto{\pgfqpoint{0.999736in}{1.265371in}}%
\pgfusepath{stroke}%
\end{pgfscope}%
\begin{pgfscope}%
\pgfpathrectangle{\pgfqpoint{0.100000in}{0.212622in}}{\pgfqpoint{3.696000in}{3.696000in}}%
\pgfusepath{clip}%
\pgfsetrectcap%
\pgfsetroundjoin%
\pgfsetlinewidth{1.505625pt}%
\definecolor{currentstroke}{rgb}{1.000000,0.000000,0.000000}%
\pgfsetstrokecolor{currentstroke}%
\pgfsetdash{}{0pt}%
\pgfpathmoveto{\pgfqpoint{0.959419in}{2.561194in}}%
\pgfpathlineto{\pgfqpoint{0.999736in}{1.265371in}}%
\pgfusepath{stroke}%
\end{pgfscope}%
\begin{pgfscope}%
\pgfpathrectangle{\pgfqpoint{0.100000in}{0.212622in}}{\pgfqpoint{3.696000in}{3.696000in}}%
\pgfusepath{clip}%
\pgfsetrectcap%
\pgfsetroundjoin%
\pgfsetlinewidth{1.505625pt}%
\definecolor{currentstroke}{rgb}{1.000000,0.000000,0.000000}%
\pgfsetstrokecolor{currentstroke}%
\pgfsetdash{}{0pt}%
\pgfpathmoveto{\pgfqpoint{0.957570in}{2.562510in}}%
\pgfpathlineto{\pgfqpoint{0.999736in}{1.265371in}}%
\pgfusepath{stroke}%
\end{pgfscope}%
\begin{pgfscope}%
\pgfpathrectangle{\pgfqpoint{0.100000in}{0.212622in}}{\pgfqpoint{3.696000in}{3.696000in}}%
\pgfusepath{clip}%
\pgfsetrectcap%
\pgfsetroundjoin%
\pgfsetlinewidth{1.505625pt}%
\definecolor{currentstroke}{rgb}{1.000000,0.000000,0.000000}%
\pgfsetstrokecolor{currentstroke}%
\pgfsetdash{}{0pt}%
\pgfpathmoveto{\pgfqpoint{0.954018in}{2.565493in}}%
\pgfpathlineto{\pgfqpoint{0.985896in}{1.269768in}}%
\pgfusepath{stroke}%
\end{pgfscope}%
\begin{pgfscope}%
\pgfpathrectangle{\pgfqpoint{0.100000in}{0.212622in}}{\pgfqpoint{3.696000in}{3.696000in}}%
\pgfusepath{clip}%
\pgfsetrectcap%
\pgfsetroundjoin%
\pgfsetlinewidth{1.505625pt}%
\definecolor{currentstroke}{rgb}{1.000000,0.000000,0.000000}%
\pgfsetstrokecolor{currentstroke}%
\pgfsetdash{}{0pt}%
\pgfpathmoveto{\pgfqpoint{0.946902in}{2.568664in}}%
\pgfpathlineto{\pgfqpoint{0.985896in}{1.269768in}}%
\pgfusepath{stroke}%
\end{pgfscope}%
\begin{pgfscope}%
\pgfpathrectangle{\pgfqpoint{0.100000in}{0.212622in}}{\pgfqpoint{3.696000in}{3.696000in}}%
\pgfusepath{clip}%
\pgfsetrectcap%
\pgfsetroundjoin%
\pgfsetlinewidth{1.505625pt}%
\definecolor{currentstroke}{rgb}{1.000000,0.000000,0.000000}%
\pgfsetstrokecolor{currentstroke}%
\pgfsetdash{}{0pt}%
\pgfpathmoveto{\pgfqpoint{0.936273in}{2.575289in}}%
\pgfpathlineto{\pgfqpoint{0.972065in}{1.274163in}}%
\pgfusepath{stroke}%
\end{pgfscope}%
\begin{pgfscope}%
\pgfpathrectangle{\pgfqpoint{0.100000in}{0.212622in}}{\pgfqpoint{3.696000in}{3.696000in}}%
\pgfusepath{clip}%
\pgfsetrectcap%
\pgfsetroundjoin%
\pgfsetlinewidth{1.505625pt}%
\definecolor{currentstroke}{rgb}{1.000000,0.000000,0.000000}%
\pgfsetstrokecolor{currentstroke}%
\pgfsetdash{}{0pt}%
\pgfpathmoveto{\pgfqpoint{0.923220in}{2.585337in}}%
\pgfpathlineto{\pgfqpoint{0.958245in}{1.278555in}}%
\pgfusepath{stroke}%
\end{pgfscope}%
\begin{pgfscope}%
\pgfpathrectangle{\pgfqpoint{0.100000in}{0.212622in}}{\pgfqpoint{3.696000in}{3.696000in}}%
\pgfusepath{clip}%
\pgfsetrectcap%
\pgfsetroundjoin%
\pgfsetlinewidth{1.505625pt}%
\definecolor{currentstroke}{rgb}{1.000000,0.000000,0.000000}%
\pgfsetstrokecolor{currentstroke}%
\pgfsetdash{}{0pt}%
\pgfpathmoveto{\pgfqpoint{0.907671in}{2.594636in}}%
\pgfpathlineto{\pgfqpoint{0.944434in}{1.282944in}}%
\pgfusepath{stroke}%
\end{pgfscope}%
\begin{pgfscope}%
\pgfpathrectangle{\pgfqpoint{0.100000in}{0.212622in}}{\pgfqpoint{3.696000in}{3.696000in}}%
\pgfusepath{clip}%
\pgfsetrectcap%
\pgfsetroundjoin%
\pgfsetlinewidth{1.505625pt}%
\definecolor{currentstroke}{rgb}{1.000000,0.000000,0.000000}%
\pgfsetstrokecolor{currentstroke}%
\pgfsetdash{}{0pt}%
\pgfpathmoveto{\pgfqpoint{0.898638in}{2.601059in}}%
\pgfpathlineto{\pgfqpoint{0.930633in}{1.287329in}}%
\pgfusepath{stroke}%
\end{pgfscope}%
\begin{pgfscope}%
\pgfpathrectangle{\pgfqpoint{0.100000in}{0.212622in}}{\pgfqpoint{3.696000in}{3.696000in}}%
\pgfusepath{clip}%
\pgfsetrectcap%
\pgfsetroundjoin%
\pgfsetlinewidth{1.505625pt}%
\definecolor{currentstroke}{rgb}{1.000000,0.000000,0.000000}%
\pgfsetstrokecolor{currentstroke}%
\pgfsetdash{}{0pt}%
\pgfpathmoveto{\pgfqpoint{0.893712in}{2.603246in}}%
\pgfpathlineto{\pgfqpoint{0.930633in}{1.287329in}}%
\pgfusepath{stroke}%
\end{pgfscope}%
\begin{pgfscope}%
\pgfpathrectangle{\pgfqpoint{0.100000in}{0.212622in}}{\pgfqpoint{3.696000in}{3.696000in}}%
\pgfusepath{clip}%
\pgfsetrectcap%
\pgfsetroundjoin%
\pgfsetlinewidth{1.505625pt}%
\definecolor{currentstroke}{rgb}{1.000000,0.000000,0.000000}%
\pgfsetstrokecolor{currentstroke}%
\pgfsetdash{}{0pt}%
\pgfpathmoveto{\pgfqpoint{0.884801in}{2.609154in}}%
\pgfpathlineto{\pgfqpoint{0.926292in}{1.299818in}}%
\pgfusepath{stroke}%
\end{pgfscope}%
\begin{pgfscope}%
\pgfpathrectangle{\pgfqpoint{0.100000in}{0.212622in}}{\pgfqpoint{3.696000in}{3.696000in}}%
\pgfusepath{clip}%
\pgfsetrectcap%
\pgfsetroundjoin%
\pgfsetlinewidth{1.505625pt}%
\definecolor{currentstroke}{rgb}{1.000000,0.000000,0.000000}%
\pgfsetstrokecolor{currentstroke}%
\pgfsetdash{}{0pt}%
\pgfpathmoveto{\pgfqpoint{0.873621in}{2.613784in}}%
\pgfpathlineto{\pgfqpoint{0.926292in}{1.299818in}}%
\pgfusepath{stroke}%
\end{pgfscope}%
\begin{pgfscope}%
\pgfpathrectangle{\pgfqpoint{0.100000in}{0.212622in}}{\pgfqpoint{3.696000in}{3.696000in}}%
\pgfusepath{clip}%
\pgfsetrectcap%
\pgfsetroundjoin%
\pgfsetlinewidth{1.505625pt}%
\definecolor{currentstroke}{rgb}{1.000000,0.000000,0.000000}%
\pgfsetstrokecolor{currentstroke}%
\pgfsetdash{}{0pt}%
\pgfpathmoveto{\pgfqpoint{0.858322in}{2.623683in}}%
\pgfpathlineto{\pgfqpoint{0.926292in}{1.299818in}}%
\pgfusepath{stroke}%
\end{pgfscope}%
\begin{pgfscope}%
\pgfpathrectangle{\pgfqpoint{0.100000in}{0.212622in}}{\pgfqpoint{3.696000in}{3.696000in}}%
\pgfusepath{clip}%
\pgfsetrectcap%
\pgfsetroundjoin%
\pgfsetlinewidth{1.505625pt}%
\definecolor{currentstroke}{rgb}{1.000000,0.000000,0.000000}%
\pgfsetstrokecolor{currentstroke}%
\pgfsetdash{}{0pt}%
\pgfpathmoveto{\pgfqpoint{0.842016in}{2.632033in}}%
\pgfpathlineto{\pgfqpoint{0.926292in}{1.299818in}}%
\pgfusepath{stroke}%
\end{pgfscope}%
\begin{pgfscope}%
\pgfpathrectangle{\pgfqpoint{0.100000in}{0.212622in}}{\pgfqpoint{3.696000in}{3.696000in}}%
\pgfusepath{clip}%
\pgfsetrectcap%
\pgfsetroundjoin%
\pgfsetlinewidth{1.505625pt}%
\definecolor{currentstroke}{rgb}{1.000000,0.000000,0.000000}%
\pgfsetstrokecolor{currentstroke}%
\pgfsetdash{}{0pt}%
\pgfpathmoveto{\pgfqpoint{0.822793in}{2.643856in}}%
\pgfpathlineto{\pgfqpoint{0.926292in}{1.299818in}}%
\pgfusepath{stroke}%
\end{pgfscope}%
\begin{pgfscope}%
\pgfpathrectangle{\pgfqpoint{0.100000in}{0.212622in}}{\pgfqpoint{3.696000in}{3.696000in}}%
\pgfusepath{clip}%
\pgfsetrectcap%
\pgfsetroundjoin%
\pgfsetlinewidth{1.505625pt}%
\definecolor{currentstroke}{rgb}{1.000000,0.000000,0.000000}%
\pgfsetstrokecolor{currentstroke}%
\pgfsetdash{}{0pt}%
\pgfpathmoveto{\pgfqpoint{0.813048in}{2.650307in}}%
\pgfpathlineto{\pgfqpoint{0.926292in}{1.299818in}}%
\pgfusepath{stroke}%
\end{pgfscope}%
\begin{pgfscope}%
\pgfpathrectangle{\pgfqpoint{0.100000in}{0.212622in}}{\pgfqpoint{3.696000in}{3.696000in}}%
\pgfusepath{clip}%
\pgfsetrectcap%
\pgfsetroundjoin%
\pgfsetlinewidth{1.505625pt}%
\definecolor{currentstroke}{rgb}{1.000000,0.000000,0.000000}%
\pgfsetstrokecolor{currentstroke}%
\pgfsetdash{}{0pt}%
\pgfpathmoveto{\pgfqpoint{0.807586in}{2.654794in}}%
\pgfpathlineto{\pgfqpoint{0.926292in}{1.299818in}}%
\pgfusepath{stroke}%
\end{pgfscope}%
\begin{pgfscope}%
\pgfpathrectangle{\pgfqpoint{0.100000in}{0.212622in}}{\pgfqpoint{3.696000in}{3.696000in}}%
\pgfusepath{clip}%
\pgfsetrectcap%
\pgfsetroundjoin%
\pgfsetlinewidth{1.505625pt}%
\definecolor{currentstroke}{rgb}{1.000000,0.000000,0.000000}%
\pgfsetstrokecolor{currentstroke}%
\pgfsetdash{}{0pt}%
\pgfpathmoveto{\pgfqpoint{0.804874in}{2.656435in}}%
\pgfpathlineto{\pgfqpoint{0.926292in}{1.299818in}}%
\pgfusepath{stroke}%
\end{pgfscope}%
\begin{pgfscope}%
\pgfpathrectangle{\pgfqpoint{0.100000in}{0.212622in}}{\pgfqpoint{3.696000in}{3.696000in}}%
\pgfusepath{clip}%
\pgfsetrectcap%
\pgfsetroundjoin%
\pgfsetlinewidth{1.505625pt}%
\definecolor{currentstroke}{rgb}{1.000000,0.000000,0.000000}%
\pgfsetstrokecolor{currentstroke}%
\pgfsetdash{}{0pt}%
\pgfpathmoveto{\pgfqpoint{0.803139in}{2.657988in}}%
\pgfpathlineto{\pgfqpoint{0.926292in}{1.299818in}}%
\pgfusepath{stroke}%
\end{pgfscope}%
\begin{pgfscope}%
\pgfpathrectangle{\pgfqpoint{0.100000in}{0.212622in}}{\pgfqpoint{3.696000in}{3.696000in}}%
\pgfusepath{clip}%
\pgfsetrectcap%
\pgfsetroundjoin%
\pgfsetlinewidth{1.505625pt}%
\definecolor{currentstroke}{rgb}{1.000000,0.000000,0.000000}%
\pgfsetstrokecolor{currentstroke}%
\pgfsetdash{}{0pt}%
\pgfpathmoveto{\pgfqpoint{0.802304in}{2.658734in}}%
\pgfpathlineto{\pgfqpoint{0.926292in}{1.299818in}}%
\pgfusepath{stroke}%
\end{pgfscope}%
\begin{pgfscope}%
\pgfpathrectangle{\pgfqpoint{0.100000in}{0.212622in}}{\pgfqpoint{3.696000in}{3.696000in}}%
\pgfusepath{clip}%
\pgfsetrectcap%
\pgfsetroundjoin%
\pgfsetlinewidth{1.505625pt}%
\definecolor{currentstroke}{rgb}{1.000000,0.000000,0.000000}%
\pgfsetstrokecolor{currentstroke}%
\pgfsetdash{}{0pt}%
\pgfpathmoveto{\pgfqpoint{0.796931in}{2.662279in}}%
\pgfpathlineto{\pgfqpoint{0.926292in}{1.299818in}}%
\pgfusepath{stroke}%
\end{pgfscope}%
\begin{pgfscope}%
\pgfpathrectangle{\pgfqpoint{0.100000in}{0.212622in}}{\pgfqpoint{3.696000in}{3.696000in}}%
\pgfusepath{clip}%
\pgfsetrectcap%
\pgfsetroundjoin%
\pgfsetlinewidth{1.505625pt}%
\definecolor{currentstroke}{rgb}{1.000000,0.000000,0.000000}%
\pgfsetstrokecolor{currentstroke}%
\pgfsetdash{}{0pt}%
\pgfpathmoveto{\pgfqpoint{0.790396in}{2.668025in}}%
\pgfpathlineto{\pgfqpoint{0.926292in}{1.299818in}}%
\pgfusepath{stroke}%
\end{pgfscope}%
\begin{pgfscope}%
\pgfpathrectangle{\pgfqpoint{0.100000in}{0.212622in}}{\pgfqpoint{3.696000in}{3.696000in}}%
\pgfusepath{clip}%
\pgfsetrectcap%
\pgfsetroundjoin%
\pgfsetlinewidth{1.505625pt}%
\definecolor{currentstroke}{rgb}{1.000000,0.000000,0.000000}%
\pgfsetstrokecolor{currentstroke}%
\pgfsetdash{}{0pt}%
\pgfpathmoveto{\pgfqpoint{0.786008in}{2.670435in}}%
\pgfpathlineto{\pgfqpoint{0.926292in}{1.299818in}}%
\pgfusepath{stroke}%
\end{pgfscope}%
\begin{pgfscope}%
\pgfpathrectangle{\pgfqpoint{0.100000in}{0.212622in}}{\pgfqpoint{3.696000in}{3.696000in}}%
\pgfusepath{clip}%
\pgfsetrectcap%
\pgfsetroundjoin%
\pgfsetlinewidth{1.505625pt}%
\definecolor{currentstroke}{rgb}{1.000000,0.000000,0.000000}%
\pgfsetstrokecolor{currentstroke}%
\pgfsetdash{}{0pt}%
\pgfpathmoveto{\pgfqpoint{0.776362in}{2.676748in}}%
\pgfpathlineto{\pgfqpoint{0.926292in}{1.299818in}}%
\pgfusepath{stroke}%
\end{pgfscope}%
\begin{pgfscope}%
\pgfpathrectangle{\pgfqpoint{0.100000in}{0.212622in}}{\pgfqpoint{3.696000in}{3.696000in}}%
\pgfusepath{clip}%
\pgfsetrectcap%
\pgfsetroundjoin%
\pgfsetlinewidth{1.505625pt}%
\definecolor{currentstroke}{rgb}{1.000000,0.000000,0.000000}%
\pgfsetstrokecolor{currentstroke}%
\pgfsetdash{}{0pt}%
\pgfpathmoveto{\pgfqpoint{0.760543in}{2.687532in}}%
\pgfpathlineto{\pgfqpoint{0.926292in}{1.299818in}}%
\pgfusepath{stroke}%
\end{pgfscope}%
\begin{pgfscope}%
\pgfpathrectangle{\pgfqpoint{0.100000in}{0.212622in}}{\pgfqpoint{3.696000in}{3.696000in}}%
\pgfusepath{clip}%
\pgfsetrectcap%
\pgfsetroundjoin%
\pgfsetlinewidth{1.505625pt}%
\definecolor{currentstroke}{rgb}{1.000000,0.000000,0.000000}%
\pgfsetstrokecolor{currentstroke}%
\pgfsetdash{}{0pt}%
\pgfpathmoveto{\pgfqpoint{0.751516in}{2.684448in}}%
\pgfpathlineto{\pgfqpoint{0.926292in}{1.299818in}}%
\pgfusepath{stroke}%
\end{pgfscope}%
\begin{pgfscope}%
\pgfpathrectangle{\pgfqpoint{0.100000in}{0.212622in}}{\pgfqpoint{3.696000in}{3.696000in}}%
\pgfusepath{clip}%
\pgfsetrectcap%
\pgfsetroundjoin%
\pgfsetlinewidth{1.505625pt}%
\definecolor{currentstroke}{rgb}{1.000000,0.000000,0.000000}%
\pgfsetstrokecolor{currentstroke}%
\pgfsetdash{}{0pt}%
\pgfpathmoveto{\pgfqpoint{0.737995in}{2.693019in}}%
\pgfpathlineto{\pgfqpoint{0.926292in}{1.299818in}}%
\pgfusepath{stroke}%
\end{pgfscope}%
\begin{pgfscope}%
\pgfpathrectangle{\pgfqpoint{0.100000in}{0.212622in}}{\pgfqpoint{3.696000in}{3.696000in}}%
\pgfusepath{clip}%
\pgfsetrectcap%
\pgfsetroundjoin%
\pgfsetlinewidth{1.505625pt}%
\definecolor{currentstroke}{rgb}{1.000000,0.000000,0.000000}%
\pgfsetstrokecolor{currentstroke}%
\pgfsetdash{}{0pt}%
\pgfpathmoveto{\pgfqpoint{0.724919in}{2.691199in}}%
\pgfpathlineto{\pgfqpoint{0.926292in}{1.299818in}}%
\pgfusepath{stroke}%
\end{pgfscope}%
\begin{pgfscope}%
\pgfpathrectangle{\pgfqpoint{0.100000in}{0.212622in}}{\pgfqpoint{3.696000in}{3.696000in}}%
\pgfusepath{clip}%
\pgfsetrectcap%
\pgfsetroundjoin%
\pgfsetlinewidth{1.505625pt}%
\definecolor{currentstroke}{rgb}{1.000000,0.000000,0.000000}%
\pgfsetstrokecolor{currentstroke}%
\pgfsetdash{}{0pt}%
\pgfpathmoveto{\pgfqpoint{0.705888in}{2.706198in}}%
\pgfpathlineto{\pgfqpoint{0.926292in}{1.299818in}}%
\pgfusepath{stroke}%
\end{pgfscope}%
\begin{pgfscope}%
\pgfpathrectangle{\pgfqpoint{0.100000in}{0.212622in}}{\pgfqpoint{3.696000in}{3.696000in}}%
\pgfusepath{clip}%
\pgfsetrectcap%
\pgfsetroundjoin%
\pgfsetlinewidth{1.505625pt}%
\definecolor{currentstroke}{rgb}{1.000000,0.000000,0.000000}%
\pgfsetstrokecolor{currentstroke}%
\pgfsetdash{}{0pt}%
\pgfpathmoveto{\pgfqpoint{0.688958in}{2.703228in}}%
\pgfpathlineto{\pgfqpoint{0.926292in}{1.299818in}}%
\pgfusepath{stroke}%
\end{pgfscope}%
\begin{pgfscope}%
\pgfpathrectangle{\pgfqpoint{0.100000in}{0.212622in}}{\pgfqpoint{3.696000in}{3.696000in}}%
\pgfusepath{clip}%
\pgfsetrectcap%
\pgfsetroundjoin%
\pgfsetlinewidth{1.505625pt}%
\definecolor{currentstroke}{rgb}{1.000000,0.000000,0.000000}%
\pgfsetstrokecolor{currentstroke}%
\pgfsetdash{}{0pt}%
\pgfpathmoveto{\pgfqpoint{0.668713in}{2.699677in}}%
\pgfpathlineto{\pgfqpoint{0.926292in}{1.299818in}}%
\pgfusepath{stroke}%
\end{pgfscope}%
\begin{pgfscope}%
\pgfpathrectangle{\pgfqpoint{0.100000in}{0.212622in}}{\pgfqpoint{3.696000in}{3.696000in}}%
\pgfusepath{clip}%
\pgfsetbuttcap%
\pgfsetroundjoin%
\definecolor{currentfill}{rgb}{0.121569,0.466667,0.705882}%
\pgfsetfillcolor{currentfill}%
\pgfsetfillopacity{0.300000}%
\pgfsetlinewidth{1.003750pt}%
\definecolor{currentstroke}{rgb}{0.121569,0.466667,0.705882}%
\pgfsetstrokecolor{currentstroke}%
\pgfsetstrokeopacity{0.300000}%
\pgfsetdash{}{0pt}%
\pgfpathmoveto{\pgfqpoint{1.672363in}{2.348955in}}%
\pgfpathcurveto{\pgfqpoint{1.680599in}{2.348955in}}{\pgfqpoint{1.688499in}{2.352227in}}{\pgfqpoint{1.694323in}{2.358051in}}%
\pgfpathcurveto{\pgfqpoint{1.700147in}{2.363875in}}{\pgfqpoint{1.703419in}{2.371775in}}{\pgfqpoint{1.703419in}{2.380011in}}%
\pgfpathcurveto{\pgfqpoint{1.703419in}{2.388248in}}{\pgfqpoint{1.700147in}{2.396148in}}{\pgfqpoint{1.694323in}{2.401972in}}%
\pgfpathcurveto{\pgfqpoint{1.688499in}{2.407795in}}{\pgfqpoint{1.680599in}{2.411068in}}{\pgfqpoint{1.672363in}{2.411068in}}%
\pgfpathcurveto{\pgfqpoint{1.664127in}{2.411068in}}{\pgfqpoint{1.656227in}{2.407795in}}{\pgfqpoint{1.650403in}{2.401972in}}%
\pgfpathcurveto{\pgfqpoint{1.644579in}{2.396148in}}{\pgfqpoint{1.641306in}{2.388248in}}{\pgfqpoint{1.641306in}{2.380011in}}%
\pgfpathcurveto{\pgfqpoint{1.641306in}{2.371775in}}{\pgfqpoint{1.644579in}{2.363875in}}{\pgfqpoint{1.650403in}{2.358051in}}%
\pgfpathcurveto{\pgfqpoint{1.656227in}{2.352227in}}{\pgfqpoint{1.664127in}{2.348955in}}{\pgfqpoint{1.672363in}{2.348955in}}%
\pgfpathclose%
\pgfusepath{stroke,fill}%
\end{pgfscope}%
\begin{pgfscope}%
\pgfpathrectangle{\pgfqpoint{0.100000in}{0.212622in}}{\pgfqpoint{3.696000in}{3.696000in}}%
\pgfusepath{clip}%
\pgfsetbuttcap%
\pgfsetroundjoin%
\definecolor{currentfill}{rgb}{0.121569,0.466667,0.705882}%
\pgfsetfillcolor{currentfill}%
\pgfsetfillopacity{0.300103}%
\pgfsetlinewidth{1.003750pt}%
\definecolor{currentstroke}{rgb}{0.121569,0.466667,0.705882}%
\pgfsetstrokecolor{currentstroke}%
\pgfsetstrokeopacity{0.300103}%
\pgfsetdash{}{0pt}%
\pgfpathmoveto{\pgfqpoint{1.661802in}{2.347529in}}%
\pgfpathcurveto{\pgfqpoint{1.670038in}{2.347529in}}{\pgfqpoint{1.677938in}{2.350802in}}{\pgfqpoint{1.683762in}{2.356626in}}%
\pgfpathcurveto{\pgfqpoint{1.689586in}{2.362450in}}{\pgfqpoint{1.692858in}{2.370350in}}{\pgfqpoint{1.692858in}{2.378586in}}%
\pgfpathcurveto{\pgfqpoint{1.692858in}{2.386822in}}{\pgfqpoint{1.689586in}{2.394722in}}{\pgfqpoint{1.683762in}{2.400546in}}%
\pgfpathcurveto{\pgfqpoint{1.677938in}{2.406370in}}{\pgfqpoint{1.670038in}{2.409642in}}{\pgfqpoint{1.661802in}{2.409642in}}%
\pgfpathcurveto{\pgfqpoint{1.653566in}{2.409642in}}{\pgfqpoint{1.645666in}{2.406370in}}{\pgfqpoint{1.639842in}{2.400546in}}%
\pgfpathcurveto{\pgfqpoint{1.634018in}{2.394722in}}{\pgfqpoint{1.630745in}{2.386822in}}{\pgfqpoint{1.630745in}{2.378586in}}%
\pgfpathcurveto{\pgfqpoint{1.630745in}{2.370350in}}{\pgfqpoint{1.634018in}{2.362450in}}{\pgfqpoint{1.639842in}{2.356626in}}%
\pgfpathcurveto{\pgfqpoint{1.645666in}{2.350802in}}{\pgfqpoint{1.653566in}{2.347529in}}{\pgfqpoint{1.661802in}{2.347529in}}%
\pgfpathclose%
\pgfusepath{stroke,fill}%
\end{pgfscope}%
\begin{pgfscope}%
\pgfpathrectangle{\pgfqpoint{0.100000in}{0.212622in}}{\pgfqpoint{3.696000in}{3.696000in}}%
\pgfusepath{clip}%
\pgfsetbuttcap%
\pgfsetroundjoin%
\definecolor{currentfill}{rgb}{0.121569,0.466667,0.705882}%
\pgfsetfillcolor{currentfill}%
\pgfsetfillopacity{0.300287}%
\pgfsetlinewidth{1.003750pt}%
\definecolor{currentstroke}{rgb}{0.121569,0.466667,0.705882}%
\pgfsetstrokecolor{currentstroke}%
\pgfsetstrokeopacity{0.300287}%
\pgfsetdash{}{0pt}%
\pgfpathmoveto{\pgfqpoint{1.654583in}{2.344973in}}%
\pgfpathcurveto{\pgfqpoint{1.662819in}{2.344973in}}{\pgfqpoint{1.670719in}{2.348246in}}{\pgfqpoint{1.676543in}{2.354070in}}%
\pgfpathcurveto{\pgfqpoint{1.682367in}{2.359894in}}{\pgfqpoint{1.685639in}{2.367794in}}{\pgfqpoint{1.685639in}{2.376030in}}%
\pgfpathcurveto{\pgfqpoint{1.685639in}{2.384266in}}{\pgfqpoint{1.682367in}{2.392166in}}{\pgfqpoint{1.676543in}{2.397990in}}%
\pgfpathcurveto{\pgfqpoint{1.670719in}{2.403814in}}{\pgfqpoint{1.662819in}{2.407086in}}{\pgfqpoint{1.654583in}{2.407086in}}%
\pgfpathcurveto{\pgfqpoint{1.646347in}{2.407086in}}{\pgfqpoint{1.638447in}{2.403814in}}{\pgfqpoint{1.632623in}{2.397990in}}%
\pgfpathcurveto{\pgfqpoint{1.626799in}{2.392166in}}{\pgfqpoint{1.623526in}{2.384266in}}{\pgfqpoint{1.623526in}{2.376030in}}%
\pgfpathcurveto{\pgfqpoint{1.623526in}{2.367794in}}{\pgfqpoint{1.626799in}{2.359894in}}{\pgfqpoint{1.632623in}{2.354070in}}%
\pgfpathcurveto{\pgfqpoint{1.638447in}{2.348246in}}{\pgfqpoint{1.646347in}{2.344973in}}{\pgfqpoint{1.654583in}{2.344973in}}%
\pgfpathclose%
\pgfusepath{stroke,fill}%
\end{pgfscope}%
\begin{pgfscope}%
\pgfpathrectangle{\pgfqpoint{0.100000in}{0.212622in}}{\pgfqpoint{3.696000in}{3.696000in}}%
\pgfusepath{clip}%
\pgfsetbuttcap%
\pgfsetroundjoin%
\definecolor{currentfill}{rgb}{0.121569,0.466667,0.705882}%
\pgfsetfillcolor{currentfill}%
\pgfsetfillopacity{0.300602}%
\pgfsetlinewidth{1.003750pt}%
\definecolor{currentstroke}{rgb}{0.121569,0.466667,0.705882}%
\pgfsetstrokecolor{currentstroke}%
\pgfsetstrokeopacity{0.300602}%
\pgfsetdash{}{0pt}%
\pgfpathmoveto{\pgfqpoint{1.649688in}{2.343457in}}%
\pgfpathcurveto{\pgfqpoint{1.657924in}{2.343457in}}{\pgfqpoint{1.665824in}{2.346729in}}{\pgfqpoint{1.671648in}{2.352553in}}%
\pgfpathcurveto{\pgfqpoint{1.677472in}{2.358377in}}{\pgfqpoint{1.680744in}{2.366277in}}{\pgfqpoint{1.680744in}{2.374513in}}%
\pgfpathcurveto{\pgfqpoint{1.680744in}{2.382749in}}{\pgfqpoint{1.677472in}{2.390649in}}{\pgfqpoint{1.671648in}{2.396473in}}%
\pgfpathcurveto{\pgfqpoint{1.665824in}{2.402297in}}{\pgfqpoint{1.657924in}{2.405570in}}{\pgfqpoint{1.649688in}{2.405570in}}%
\pgfpathcurveto{\pgfqpoint{1.641451in}{2.405570in}}{\pgfqpoint{1.633551in}{2.402297in}}{\pgfqpoint{1.627727in}{2.396473in}}%
\pgfpathcurveto{\pgfqpoint{1.621904in}{2.390649in}}{\pgfqpoint{1.618631in}{2.382749in}}{\pgfqpoint{1.618631in}{2.374513in}}%
\pgfpathcurveto{\pgfqpoint{1.618631in}{2.366277in}}{\pgfqpoint{1.621904in}{2.358377in}}{\pgfqpoint{1.627727in}{2.352553in}}%
\pgfpathcurveto{\pgfqpoint{1.633551in}{2.346729in}}{\pgfqpoint{1.641451in}{2.343457in}}{\pgfqpoint{1.649688in}{2.343457in}}%
\pgfpathclose%
\pgfusepath{stroke,fill}%
\end{pgfscope}%
\begin{pgfscope}%
\pgfpathrectangle{\pgfqpoint{0.100000in}{0.212622in}}{\pgfqpoint{3.696000in}{3.696000in}}%
\pgfusepath{clip}%
\pgfsetbuttcap%
\pgfsetroundjoin%
\definecolor{currentfill}{rgb}{0.121569,0.466667,0.705882}%
\pgfsetfillcolor{currentfill}%
\pgfsetfillopacity{0.300747}%
\pgfsetlinewidth{1.003750pt}%
\definecolor{currentstroke}{rgb}{0.121569,0.466667,0.705882}%
\pgfsetstrokecolor{currentstroke}%
\pgfsetstrokeopacity{0.300747}%
\pgfsetdash{}{0pt}%
\pgfpathmoveto{\pgfqpoint{1.648292in}{2.342373in}}%
\pgfpathcurveto{\pgfqpoint{1.656529in}{2.342373in}}{\pgfqpoint{1.664429in}{2.345645in}}{\pgfqpoint{1.670253in}{2.351469in}}%
\pgfpathcurveto{\pgfqpoint{1.676077in}{2.357293in}}{\pgfqpoint{1.679349in}{2.365193in}}{\pgfqpoint{1.679349in}{2.373429in}}%
\pgfpathcurveto{\pgfqpoint{1.679349in}{2.381666in}}{\pgfqpoint{1.676077in}{2.389566in}}{\pgfqpoint{1.670253in}{2.395390in}}%
\pgfpathcurveto{\pgfqpoint{1.664429in}{2.401214in}}{\pgfqpoint{1.656529in}{2.404486in}}{\pgfqpoint{1.648292in}{2.404486in}}%
\pgfpathcurveto{\pgfqpoint{1.640056in}{2.404486in}}{\pgfqpoint{1.632156in}{2.401214in}}{\pgfqpoint{1.626332in}{2.395390in}}%
\pgfpathcurveto{\pgfqpoint{1.620508in}{2.389566in}}{\pgfqpoint{1.617236in}{2.381666in}}{\pgfqpoint{1.617236in}{2.373429in}}%
\pgfpathcurveto{\pgfqpoint{1.617236in}{2.365193in}}{\pgfqpoint{1.620508in}{2.357293in}}{\pgfqpoint{1.626332in}{2.351469in}}%
\pgfpathcurveto{\pgfqpoint{1.632156in}{2.345645in}}{\pgfqpoint{1.640056in}{2.342373in}}{\pgfqpoint{1.648292in}{2.342373in}}%
\pgfpathclose%
\pgfusepath{stroke,fill}%
\end{pgfscope}%
\begin{pgfscope}%
\pgfpathrectangle{\pgfqpoint{0.100000in}{0.212622in}}{\pgfqpoint{3.696000in}{3.696000in}}%
\pgfusepath{clip}%
\pgfsetbuttcap%
\pgfsetroundjoin%
\definecolor{currentfill}{rgb}{0.121569,0.466667,0.705882}%
\pgfsetfillcolor{currentfill}%
\pgfsetfillopacity{0.300828}%
\pgfsetlinewidth{1.003750pt}%
\definecolor{currentstroke}{rgb}{0.121569,0.466667,0.705882}%
\pgfsetstrokecolor{currentstroke}%
\pgfsetstrokeopacity{0.300828}%
\pgfsetdash{}{0pt}%
\pgfpathmoveto{\pgfqpoint{1.684693in}{2.349553in}}%
\pgfpathcurveto{\pgfqpoint{1.692930in}{2.349553in}}{\pgfqpoint{1.700830in}{2.352825in}}{\pgfqpoint{1.706654in}{2.358649in}}%
\pgfpathcurveto{\pgfqpoint{1.712478in}{2.364473in}}{\pgfqpoint{1.715750in}{2.372373in}}{\pgfqpoint{1.715750in}{2.380609in}}%
\pgfpathcurveto{\pgfqpoint{1.715750in}{2.388845in}}{\pgfqpoint{1.712478in}{2.396745in}}{\pgfqpoint{1.706654in}{2.402569in}}%
\pgfpathcurveto{\pgfqpoint{1.700830in}{2.408393in}}{\pgfqpoint{1.692930in}{2.411666in}}{\pgfqpoint{1.684693in}{2.411666in}}%
\pgfpathcurveto{\pgfqpoint{1.676457in}{2.411666in}}{\pgfqpoint{1.668557in}{2.408393in}}{\pgfqpoint{1.662733in}{2.402569in}}%
\pgfpathcurveto{\pgfqpoint{1.656909in}{2.396745in}}{\pgfqpoint{1.653637in}{2.388845in}}{\pgfqpoint{1.653637in}{2.380609in}}%
\pgfpathcurveto{\pgfqpoint{1.653637in}{2.372373in}}{\pgfqpoint{1.656909in}{2.364473in}}{\pgfqpoint{1.662733in}{2.358649in}}%
\pgfpathcurveto{\pgfqpoint{1.668557in}{2.352825in}}{\pgfqpoint{1.676457in}{2.349553in}}{\pgfqpoint{1.684693in}{2.349553in}}%
\pgfpathclose%
\pgfusepath{stroke,fill}%
\end{pgfscope}%
\begin{pgfscope}%
\pgfpathrectangle{\pgfqpoint{0.100000in}{0.212622in}}{\pgfqpoint{3.696000in}{3.696000in}}%
\pgfusepath{clip}%
\pgfsetbuttcap%
\pgfsetroundjoin%
\definecolor{currentfill}{rgb}{0.121569,0.466667,0.705882}%
\pgfsetfillcolor{currentfill}%
\pgfsetfillopacity{0.301130}%
\pgfsetlinewidth{1.003750pt}%
\definecolor{currentstroke}{rgb}{0.121569,0.466667,0.705882}%
\pgfsetstrokecolor{currentstroke}%
\pgfsetstrokeopacity{0.301130}%
\pgfsetdash{}{0pt}%
\pgfpathmoveto{\pgfqpoint{1.645924in}{2.340784in}}%
\pgfpathcurveto{\pgfqpoint{1.654161in}{2.340784in}}{\pgfqpoint{1.662061in}{2.344056in}}{\pgfqpoint{1.667885in}{2.349880in}}%
\pgfpathcurveto{\pgfqpoint{1.673709in}{2.355704in}}{\pgfqpoint{1.676981in}{2.363604in}}{\pgfqpoint{1.676981in}{2.371840in}}%
\pgfpathcurveto{\pgfqpoint{1.676981in}{2.380076in}}{\pgfqpoint{1.673709in}{2.387976in}}{\pgfqpoint{1.667885in}{2.393800in}}%
\pgfpathcurveto{\pgfqpoint{1.662061in}{2.399624in}}{\pgfqpoint{1.654161in}{2.402897in}}{\pgfqpoint{1.645924in}{2.402897in}}%
\pgfpathcurveto{\pgfqpoint{1.637688in}{2.402897in}}{\pgfqpoint{1.629788in}{2.399624in}}{\pgfqpoint{1.623964in}{2.393800in}}%
\pgfpathcurveto{\pgfqpoint{1.618140in}{2.387976in}}{\pgfqpoint{1.614868in}{2.380076in}}{\pgfqpoint{1.614868in}{2.371840in}}%
\pgfpathcurveto{\pgfqpoint{1.614868in}{2.363604in}}{\pgfqpoint{1.618140in}{2.355704in}}{\pgfqpoint{1.623964in}{2.349880in}}%
\pgfpathcurveto{\pgfqpoint{1.629788in}{2.344056in}}{\pgfqpoint{1.637688in}{2.340784in}}{\pgfqpoint{1.645924in}{2.340784in}}%
\pgfpathclose%
\pgfusepath{stroke,fill}%
\end{pgfscope}%
\begin{pgfscope}%
\pgfpathrectangle{\pgfqpoint{0.100000in}{0.212622in}}{\pgfqpoint{3.696000in}{3.696000in}}%
\pgfusepath{clip}%
\pgfsetbuttcap%
\pgfsetroundjoin%
\definecolor{currentfill}{rgb}{0.121569,0.466667,0.705882}%
\pgfsetfillcolor{currentfill}%
\pgfsetfillopacity{0.301130}%
\pgfsetlinewidth{1.003750pt}%
\definecolor{currentstroke}{rgb}{0.121569,0.466667,0.705882}%
\pgfsetstrokecolor{currentstroke}%
\pgfsetstrokeopacity{0.301130}%
\pgfsetdash{}{0pt}%
\pgfpathmoveto{\pgfqpoint{1.645923in}{2.340782in}}%
\pgfpathcurveto{\pgfqpoint{1.654159in}{2.340782in}}{\pgfqpoint{1.662059in}{2.344054in}}{\pgfqpoint{1.667883in}{2.349878in}}%
\pgfpathcurveto{\pgfqpoint{1.673707in}{2.355702in}}{\pgfqpoint{1.676979in}{2.363602in}}{\pgfqpoint{1.676979in}{2.371839in}}%
\pgfpathcurveto{\pgfqpoint{1.676979in}{2.380075in}}{\pgfqpoint{1.673707in}{2.387975in}}{\pgfqpoint{1.667883in}{2.393799in}}%
\pgfpathcurveto{\pgfqpoint{1.662059in}{2.399623in}}{\pgfqpoint{1.654159in}{2.402895in}}{\pgfqpoint{1.645923in}{2.402895in}}%
\pgfpathcurveto{\pgfqpoint{1.637687in}{2.402895in}}{\pgfqpoint{1.629787in}{2.399623in}}{\pgfqpoint{1.623963in}{2.393799in}}%
\pgfpathcurveto{\pgfqpoint{1.618139in}{2.387975in}}{\pgfqpoint{1.614866in}{2.380075in}}{\pgfqpoint{1.614866in}{2.371839in}}%
\pgfpathcurveto{\pgfqpoint{1.614866in}{2.363602in}}{\pgfqpoint{1.618139in}{2.355702in}}{\pgfqpoint{1.623963in}{2.349878in}}%
\pgfpathcurveto{\pgfqpoint{1.629787in}{2.344054in}}{\pgfqpoint{1.637687in}{2.340782in}}{\pgfqpoint{1.645923in}{2.340782in}}%
\pgfpathclose%
\pgfusepath{stroke,fill}%
\end{pgfscope}%
\begin{pgfscope}%
\pgfpathrectangle{\pgfqpoint{0.100000in}{0.212622in}}{\pgfqpoint{3.696000in}{3.696000in}}%
\pgfusepath{clip}%
\pgfsetbuttcap%
\pgfsetroundjoin%
\definecolor{currentfill}{rgb}{0.121569,0.466667,0.705882}%
\pgfsetfillcolor{currentfill}%
\pgfsetfillopacity{0.301131}%
\pgfsetlinewidth{1.003750pt}%
\definecolor{currentstroke}{rgb}{0.121569,0.466667,0.705882}%
\pgfsetstrokecolor{currentstroke}%
\pgfsetstrokeopacity{0.301131}%
\pgfsetdash{}{0pt}%
\pgfpathmoveto{\pgfqpoint{1.645920in}{2.340779in}}%
\pgfpathcurveto{\pgfqpoint{1.654156in}{2.340779in}}{\pgfqpoint{1.662056in}{2.344051in}}{\pgfqpoint{1.667880in}{2.349875in}}%
\pgfpathcurveto{\pgfqpoint{1.673704in}{2.355699in}}{\pgfqpoint{1.676977in}{2.363599in}}{\pgfqpoint{1.676977in}{2.371836in}}%
\pgfpathcurveto{\pgfqpoint{1.676977in}{2.380072in}}{\pgfqpoint{1.673704in}{2.387972in}}{\pgfqpoint{1.667880in}{2.393796in}}%
\pgfpathcurveto{\pgfqpoint{1.662056in}{2.399620in}}{\pgfqpoint{1.654156in}{2.402892in}}{\pgfqpoint{1.645920in}{2.402892in}}%
\pgfpathcurveto{\pgfqpoint{1.637684in}{2.402892in}}{\pgfqpoint{1.629784in}{2.399620in}}{\pgfqpoint{1.623960in}{2.393796in}}%
\pgfpathcurveto{\pgfqpoint{1.618136in}{2.387972in}}{\pgfqpoint{1.614864in}{2.380072in}}{\pgfqpoint{1.614864in}{2.371836in}}%
\pgfpathcurveto{\pgfqpoint{1.614864in}{2.363599in}}{\pgfqpoint{1.618136in}{2.355699in}}{\pgfqpoint{1.623960in}{2.349875in}}%
\pgfpathcurveto{\pgfqpoint{1.629784in}{2.344051in}}{\pgfqpoint{1.637684in}{2.340779in}}{\pgfqpoint{1.645920in}{2.340779in}}%
\pgfpathclose%
\pgfusepath{stroke,fill}%
\end{pgfscope}%
\begin{pgfscope}%
\pgfpathrectangle{\pgfqpoint{0.100000in}{0.212622in}}{\pgfqpoint{3.696000in}{3.696000in}}%
\pgfusepath{clip}%
\pgfsetbuttcap%
\pgfsetroundjoin%
\definecolor{currentfill}{rgb}{0.121569,0.466667,0.705882}%
\pgfsetfillcolor{currentfill}%
\pgfsetfillopacity{0.301133}%
\pgfsetlinewidth{1.003750pt}%
\definecolor{currentstroke}{rgb}{0.121569,0.466667,0.705882}%
\pgfsetstrokecolor{currentstroke}%
\pgfsetstrokeopacity{0.301133}%
\pgfsetdash{}{0pt}%
\pgfpathmoveto{\pgfqpoint{1.645917in}{2.340774in}}%
\pgfpathcurveto{\pgfqpoint{1.654153in}{2.340774in}}{\pgfqpoint{1.662053in}{2.344046in}}{\pgfqpoint{1.667877in}{2.349870in}}%
\pgfpathcurveto{\pgfqpoint{1.673701in}{2.355694in}}{\pgfqpoint{1.676973in}{2.363594in}}{\pgfqpoint{1.676973in}{2.371830in}}%
\pgfpathcurveto{\pgfqpoint{1.676973in}{2.380067in}}{\pgfqpoint{1.673701in}{2.387967in}}{\pgfqpoint{1.667877in}{2.393791in}}%
\pgfpathcurveto{\pgfqpoint{1.662053in}{2.399614in}}{\pgfqpoint{1.654153in}{2.402887in}}{\pgfqpoint{1.645917in}{2.402887in}}%
\pgfpathcurveto{\pgfqpoint{1.637680in}{2.402887in}}{\pgfqpoint{1.629780in}{2.399614in}}{\pgfqpoint{1.623956in}{2.393791in}}%
\pgfpathcurveto{\pgfqpoint{1.618132in}{2.387967in}}{\pgfqpoint{1.614860in}{2.380067in}}{\pgfqpoint{1.614860in}{2.371830in}}%
\pgfpathcurveto{\pgfqpoint{1.614860in}{2.363594in}}{\pgfqpoint{1.618132in}{2.355694in}}{\pgfqpoint{1.623956in}{2.349870in}}%
\pgfpathcurveto{\pgfqpoint{1.629780in}{2.344046in}}{\pgfqpoint{1.637680in}{2.340774in}}{\pgfqpoint{1.645917in}{2.340774in}}%
\pgfpathclose%
\pgfusepath{stroke,fill}%
\end{pgfscope}%
\begin{pgfscope}%
\pgfpathrectangle{\pgfqpoint{0.100000in}{0.212622in}}{\pgfqpoint{3.696000in}{3.696000in}}%
\pgfusepath{clip}%
\pgfsetbuttcap%
\pgfsetroundjoin%
\definecolor{currentfill}{rgb}{0.121569,0.466667,0.705882}%
\pgfsetfillcolor{currentfill}%
\pgfsetfillopacity{0.301135}%
\pgfsetlinewidth{1.003750pt}%
\definecolor{currentstroke}{rgb}{0.121569,0.466667,0.705882}%
\pgfsetstrokecolor{currentstroke}%
\pgfsetstrokeopacity{0.301135}%
\pgfsetdash{}{0pt}%
\pgfpathmoveto{\pgfqpoint{1.645909in}{2.340764in}}%
\pgfpathcurveto{\pgfqpoint{1.654145in}{2.340764in}}{\pgfqpoint{1.662045in}{2.344037in}}{\pgfqpoint{1.667869in}{2.349861in}}%
\pgfpathcurveto{\pgfqpoint{1.673693in}{2.355685in}}{\pgfqpoint{1.676965in}{2.363585in}}{\pgfqpoint{1.676965in}{2.371821in}}%
\pgfpathcurveto{\pgfqpoint{1.676965in}{2.380057in}}{\pgfqpoint{1.673693in}{2.387957in}}{\pgfqpoint{1.667869in}{2.393781in}}%
\pgfpathcurveto{\pgfqpoint{1.662045in}{2.399605in}}{\pgfqpoint{1.654145in}{2.402877in}}{\pgfqpoint{1.645909in}{2.402877in}}%
\pgfpathcurveto{\pgfqpoint{1.637673in}{2.402877in}}{\pgfqpoint{1.629773in}{2.399605in}}{\pgfqpoint{1.623949in}{2.393781in}}%
\pgfpathcurveto{\pgfqpoint{1.618125in}{2.387957in}}{\pgfqpoint{1.614852in}{2.380057in}}{\pgfqpoint{1.614852in}{2.371821in}}%
\pgfpathcurveto{\pgfqpoint{1.614852in}{2.363585in}}{\pgfqpoint{1.618125in}{2.355685in}}{\pgfqpoint{1.623949in}{2.349861in}}%
\pgfpathcurveto{\pgfqpoint{1.629773in}{2.344037in}}{\pgfqpoint{1.637673in}{2.340764in}}{\pgfqpoint{1.645909in}{2.340764in}}%
\pgfpathclose%
\pgfusepath{stroke,fill}%
\end{pgfscope}%
\begin{pgfscope}%
\pgfpathrectangle{\pgfqpoint{0.100000in}{0.212622in}}{\pgfqpoint{3.696000in}{3.696000in}}%
\pgfusepath{clip}%
\pgfsetbuttcap%
\pgfsetroundjoin%
\definecolor{currentfill}{rgb}{0.121569,0.466667,0.705882}%
\pgfsetfillcolor{currentfill}%
\pgfsetfillopacity{0.301140}%
\pgfsetlinewidth{1.003750pt}%
\definecolor{currentstroke}{rgb}{0.121569,0.466667,0.705882}%
\pgfsetstrokecolor{currentstroke}%
\pgfsetstrokeopacity{0.301140}%
\pgfsetdash{}{0pt}%
\pgfpathmoveto{\pgfqpoint{1.645899in}{2.340747in}}%
\pgfpathcurveto{\pgfqpoint{1.654135in}{2.340747in}}{\pgfqpoint{1.662035in}{2.344020in}}{\pgfqpoint{1.667859in}{2.349844in}}%
\pgfpathcurveto{\pgfqpoint{1.673683in}{2.355668in}}{\pgfqpoint{1.676955in}{2.363568in}}{\pgfqpoint{1.676955in}{2.371804in}}%
\pgfpathcurveto{\pgfqpoint{1.676955in}{2.380040in}}{\pgfqpoint{1.673683in}{2.387940in}}{\pgfqpoint{1.667859in}{2.393764in}}%
\pgfpathcurveto{\pgfqpoint{1.662035in}{2.399588in}}{\pgfqpoint{1.654135in}{2.402860in}}{\pgfqpoint{1.645899in}{2.402860in}}%
\pgfpathcurveto{\pgfqpoint{1.637662in}{2.402860in}}{\pgfqpoint{1.629762in}{2.399588in}}{\pgfqpoint{1.623938in}{2.393764in}}%
\pgfpathcurveto{\pgfqpoint{1.618114in}{2.387940in}}{\pgfqpoint{1.614842in}{2.380040in}}{\pgfqpoint{1.614842in}{2.371804in}}%
\pgfpathcurveto{\pgfqpoint{1.614842in}{2.363568in}}{\pgfqpoint{1.618114in}{2.355668in}}{\pgfqpoint{1.623938in}{2.349844in}}%
\pgfpathcurveto{\pgfqpoint{1.629762in}{2.344020in}}{\pgfqpoint{1.637662in}{2.340747in}}{\pgfqpoint{1.645899in}{2.340747in}}%
\pgfpathclose%
\pgfusepath{stroke,fill}%
\end{pgfscope}%
\begin{pgfscope}%
\pgfpathrectangle{\pgfqpoint{0.100000in}{0.212622in}}{\pgfqpoint{3.696000in}{3.696000in}}%
\pgfusepath{clip}%
\pgfsetbuttcap%
\pgfsetroundjoin%
\definecolor{currentfill}{rgb}{0.121569,0.466667,0.705882}%
\pgfsetfillcolor{currentfill}%
\pgfsetfillopacity{0.301149}%
\pgfsetlinewidth{1.003750pt}%
\definecolor{currentstroke}{rgb}{0.121569,0.466667,0.705882}%
\pgfsetstrokecolor{currentstroke}%
\pgfsetstrokeopacity{0.301149}%
\pgfsetdash{}{0pt}%
\pgfpathmoveto{\pgfqpoint{1.645878in}{2.340716in}}%
\pgfpathcurveto{\pgfqpoint{1.654115in}{2.340716in}}{\pgfqpoint{1.662015in}{2.343989in}}{\pgfqpoint{1.667839in}{2.349812in}}%
\pgfpathcurveto{\pgfqpoint{1.673663in}{2.355636in}}{\pgfqpoint{1.676935in}{2.363536in}}{\pgfqpoint{1.676935in}{2.371773in}}%
\pgfpathcurveto{\pgfqpoint{1.676935in}{2.380009in}}{\pgfqpoint{1.673663in}{2.387909in}}{\pgfqpoint{1.667839in}{2.393733in}}%
\pgfpathcurveto{\pgfqpoint{1.662015in}{2.399557in}}{\pgfqpoint{1.654115in}{2.402829in}}{\pgfqpoint{1.645878in}{2.402829in}}%
\pgfpathcurveto{\pgfqpoint{1.637642in}{2.402829in}}{\pgfqpoint{1.629742in}{2.399557in}}{\pgfqpoint{1.623918in}{2.393733in}}%
\pgfpathcurveto{\pgfqpoint{1.618094in}{2.387909in}}{\pgfqpoint{1.614822in}{2.380009in}}{\pgfqpoint{1.614822in}{2.371773in}}%
\pgfpathcurveto{\pgfqpoint{1.614822in}{2.363536in}}{\pgfqpoint{1.618094in}{2.355636in}}{\pgfqpoint{1.623918in}{2.349812in}}%
\pgfpathcurveto{\pgfqpoint{1.629742in}{2.343989in}}{\pgfqpoint{1.637642in}{2.340716in}}{\pgfqpoint{1.645878in}{2.340716in}}%
\pgfpathclose%
\pgfusepath{stroke,fill}%
\end{pgfscope}%
\begin{pgfscope}%
\pgfpathrectangle{\pgfqpoint{0.100000in}{0.212622in}}{\pgfqpoint{3.696000in}{3.696000in}}%
\pgfusepath{clip}%
\pgfsetbuttcap%
\pgfsetroundjoin%
\definecolor{currentfill}{rgb}{0.121569,0.466667,0.705882}%
\pgfsetfillcolor{currentfill}%
\pgfsetfillopacity{0.301165}%
\pgfsetlinewidth{1.003750pt}%
\definecolor{currentstroke}{rgb}{0.121569,0.466667,0.705882}%
\pgfsetstrokecolor{currentstroke}%
\pgfsetstrokeopacity{0.301165}%
\pgfsetdash{}{0pt}%
\pgfpathmoveto{\pgfqpoint{1.645841in}{2.340660in}}%
\pgfpathcurveto{\pgfqpoint{1.654077in}{2.340660in}}{\pgfqpoint{1.661977in}{2.343932in}}{\pgfqpoint{1.667801in}{2.349756in}}%
\pgfpathcurveto{\pgfqpoint{1.673625in}{2.355580in}}{\pgfqpoint{1.676898in}{2.363480in}}{\pgfqpoint{1.676898in}{2.371717in}}%
\pgfpathcurveto{\pgfqpoint{1.676898in}{2.379953in}}{\pgfqpoint{1.673625in}{2.387853in}}{\pgfqpoint{1.667801in}{2.393677in}}%
\pgfpathcurveto{\pgfqpoint{1.661977in}{2.399501in}}{\pgfqpoint{1.654077in}{2.402773in}}{\pgfqpoint{1.645841in}{2.402773in}}%
\pgfpathcurveto{\pgfqpoint{1.637605in}{2.402773in}}{\pgfqpoint{1.629705in}{2.399501in}}{\pgfqpoint{1.623881in}{2.393677in}}%
\pgfpathcurveto{\pgfqpoint{1.618057in}{2.387853in}}{\pgfqpoint{1.614785in}{2.379953in}}{\pgfqpoint{1.614785in}{2.371717in}}%
\pgfpathcurveto{\pgfqpoint{1.614785in}{2.363480in}}{\pgfqpoint{1.618057in}{2.355580in}}{\pgfqpoint{1.623881in}{2.349756in}}%
\pgfpathcurveto{\pgfqpoint{1.629705in}{2.343932in}}{\pgfqpoint{1.637605in}{2.340660in}}{\pgfqpoint{1.645841in}{2.340660in}}%
\pgfpathclose%
\pgfusepath{stroke,fill}%
\end{pgfscope}%
\begin{pgfscope}%
\pgfpathrectangle{\pgfqpoint{0.100000in}{0.212622in}}{\pgfqpoint{3.696000in}{3.696000in}}%
\pgfusepath{clip}%
\pgfsetbuttcap%
\pgfsetroundjoin%
\definecolor{currentfill}{rgb}{0.121569,0.466667,0.705882}%
\pgfsetfillcolor{currentfill}%
\pgfsetfillopacity{0.301194}%
\pgfsetlinewidth{1.003750pt}%
\definecolor{currentstroke}{rgb}{0.121569,0.466667,0.705882}%
\pgfsetstrokecolor{currentstroke}%
\pgfsetstrokeopacity{0.301194}%
\pgfsetdash{}{0pt}%
\pgfpathmoveto{\pgfqpoint{1.645772in}{2.340558in}}%
\pgfpathcurveto{\pgfqpoint{1.654009in}{2.340558in}}{\pgfqpoint{1.661909in}{2.343830in}}{\pgfqpoint{1.667733in}{2.349654in}}%
\pgfpathcurveto{\pgfqpoint{1.673557in}{2.355478in}}{\pgfqpoint{1.676829in}{2.363378in}}{\pgfqpoint{1.676829in}{2.371614in}}%
\pgfpathcurveto{\pgfqpoint{1.676829in}{2.379851in}}{\pgfqpoint{1.673557in}{2.387751in}}{\pgfqpoint{1.667733in}{2.393575in}}%
\pgfpathcurveto{\pgfqpoint{1.661909in}{2.399399in}}{\pgfqpoint{1.654009in}{2.402671in}}{\pgfqpoint{1.645772in}{2.402671in}}%
\pgfpathcurveto{\pgfqpoint{1.637536in}{2.402671in}}{\pgfqpoint{1.629636in}{2.399399in}}{\pgfqpoint{1.623812in}{2.393575in}}%
\pgfpathcurveto{\pgfqpoint{1.617988in}{2.387751in}}{\pgfqpoint{1.614716in}{2.379851in}}{\pgfqpoint{1.614716in}{2.371614in}}%
\pgfpathcurveto{\pgfqpoint{1.614716in}{2.363378in}}{\pgfqpoint{1.617988in}{2.355478in}}{\pgfqpoint{1.623812in}{2.349654in}}%
\pgfpathcurveto{\pgfqpoint{1.629636in}{2.343830in}}{\pgfqpoint{1.637536in}{2.340558in}}{\pgfqpoint{1.645772in}{2.340558in}}%
\pgfpathclose%
\pgfusepath{stroke,fill}%
\end{pgfscope}%
\begin{pgfscope}%
\pgfpathrectangle{\pgfqpoint{0.100000in}{0.212622in}}{\pgfqpoint{3.696000in}{3.696000in}}%
\pgfusepath{clip}%
\pgfsetbuttcap%
\pgfsetroundjoin%
\definecolor{currentfill}{rgb}{0.121569,0.466667,0.705882}%
\pgfsetfillcolor{currentfill}%
\pgfsetfillopacity{0.301246}%
\pgfsetlinewidth{1.003750pt}%
\definecolor{currentstroke}{rgb}{0.121569,0.466667,0.705882}%
\pgfsetstrokecolor{currentstroke}%
\pgfsetstrokeopacity{0.301246}%
\pgfsetdash{}{0pt}%
\pgfpathmoveto{\pgfqpoint{1.645646in}{2.340369in}}%
\pgfpathcurveto{\pgfqpoint{1.653882in}{2.340369in}}{\pgfqpoint{1.661782in}{2.343641in}}{\pgfqpoint{1.667606in}{2.349465in}}%
\pgfpathcurveto{\pgfqpoint{1.673430in}{2.355289in}}{\pgfqpoint{1.676703in}{2.363189in}}{\pgfqpoint{1.676703in}{2.371425in}}%
\pgfpathcurveto{\pgfqpoint{1.676703in}{2.379662in}}{\pgfqpoint{1.673430in}{2.387562in}}{\pgfqpoint{1.667606in}{2.393386in}}%
\pgfpathcurveto{\pgfqpoint{1.661782in}{2.399210in}}{\pgfqpoint{1.653882in}{2.402482in}}{\pgfqpoint{1.645646in}{2.402482in}}%
\pgfpathcurveto{\pgfqpoint{1.637410in}{2.402482in}}{\pgfqpoint{1.629510in}{2.399210in}}{\pgfqpoint{1.623686in}{2.393386in}}%
\pgfpathcurveto{\pgfqpoint{1.617862in}{2.387562in}}{\pgfqpoint{1.614590in}{2.379662in}}{\pgfqpoint{1.614590in}{2.371425in}}%
\pgfpathcurveto{\pgfqpoint{1.614590in}{2.363189in}}{\pgfqpoint{1.617862in}{2.355289in}}{\pgfqpoint{1.623686in}{2.349465in}}%
\pgfpathcurveto{\pgfqpoint{1.629510in}{2.343641in}}{\pgfqpoint{1.637410in}{2.340369in}}{\pgfqpoint{1.645646in}{2.340369in}}%
\pgfpathclose%
\pgfusepath{stroke,fill}%
\end{pgfscope}%
\begin{pgfscope}%
\pgfpathrectangle{\pgfqpoint{0.100000in}{0.212622in}}{\pgfqpoint{3.696000in}{3.696000in}}%
\pgfusepath{clip}%
\pgfsetbuttcap%
\pgfsetroundjoin%
\definecolor{currentfill}{rgb}{0.121569,0.466667,0.705882}%
\pgfsetfillcolor{currentfill}%
\pgfsetfillopacity{0.301338}%
\pgfsetlinewidth{1.003750pt}%
\definecolor{currentstroke}{rgb}{0.121569,0.466667,0.705882}%
\pgfsetstrokecolor{currentstroke}%
\pgfsetstrokeopacity{0.301338}%
\pgfsetdash{}{0pt}%
\pgfpathmoveto{\pgfqpoint{1.645419in}{2.340005in}}%
\pgfpathcurveto{\pgfqpoint{1.653655in}{2.340005in}}{\pgfqpoint{1.661555in}{2.343277in}}{\pgfqpoint{1.667379in}{2.349101in}}%
\pgfpathcurveto{\pgfqpoint{1.673203in}{2.354925in}}{\pgfqpoint{1.676475in}{2.362825in}}{\pgfqpoint{1.676475in}{2.371061in}}%
\pgfpathcurveto{\pgfqpoint{1.676475in}{2.379298in}}{\pgfqpoint{1.673203in}{2.387198in}}{\pgfqpoint{1.667379in}{2.393022in}}%
\pgfpathcurveto{\pgfqpoint{1.661555in}{2.398846in}}{\pgfqpoint{1.653655in}{2.402118in}}{\pgfqpoint{1.645419in}{2.402118in}}%
\pgfpathcurveto{\pgfqpoint{1.637182in}{2.402118in}}{\pgfqpoint{1.629282in}{2.398846in}}{\pgfqpoint{1.623458in}{2.393022in}}%
\pgfpathcurveto{\pgfqpoint{1.617634in}{2.387198in}}{\pgfqpoint{1.614362in}{2.379298in}}{\pgfqpoint{1.614362in}{2.371061in}}%
\pgfpathcurveto{\pgfqpoint{1.614362in}{2.362825in}}{\pgfqpoint{1.617634in}{2.354925in}}{\pgfqpoint{1.623458in}{2.349101in}}%
\pgfpathcurveto{\pgfqpoint{1.629282in}{2.343277in}}{\pgfqpoint{1.637182in}{2.340005in}}{\pgfqpoint{1.645419in}{2.340005in}}%
\pgfpathclose%
\pgfusepath{stroke,fill}%
\end{pgfscope}%
\begin{pgfscope}%
\pgfpathrectangle{\pgfqpoint{0.100000in}{0.212622in}}{\pgfqpoint{3.696000in}{3.696000in}}%
\pgfusepath{clip}%
\pgfsetbuttcap%
\pgfsetroundjoin%
\definecolor{currentfill}{rgb}{0.121569,0.466667,0.705882}%
\pgfsetfillcolor{currentfill}%
\pgfsetfillopacity{0.301517}%
\pgfsetlinewidth{1.003750pt}%
\definecolor{currentstroke}{rgb}{0.121569,0.466667,0.705882}%
\pgfsetstrokecolor{currentstroke}%
\pgfsetstrokeopacity{0.301517}%
\pgfsetdash{}{0pt}%
\pgfpathmoveto{\pgfqpoint{1.645011in}{2.339423in}}%
\pgfpathcurveto{\pgfqpoint{1.653247in}{2.339423in}}{\pgfqpoint{1.661147in}{2.342695in}}{\pgfqpoint{1.666971in}{2.348519in}}%
\pgfpathcurveto{\pgfqpoint{1.672795in}{2.354343in}}{\pgfqpoint{1.676067in}{2.362243in}}{\pgfqpoint{1.676067in}{2.370480in}}%
\pgfpathcurveto{\pgfqpoint{1.676067in}{2.378716in}}{\pgfqpoint{1.672795in}{2.386616in}}{\pgfqpoint{1.666971in}{2.392440in}}%
\pgfpathcurveto{\pgfqpoint{1.661147in}{2.398264in}}{\pgfqpoint{1.653247in}{2.401536in}}{\pgfqpoint{1.645011in}{2.401536in}}%
\pgfpathcurveto{\pgfqpoint{1.636774in}{2.401536in}}{\pgfqpoint{1.628874in}{2.398264in}}{\pgfqpoint{1.623050in}{2.392440in}}%
\pgfpathcurveto{\pgfqpoint{1.617226in}{2.386616in}}{\pgfqpoint{1.613954in}{2.378716in}}{\pgfqpoint{1.613954in}{2.370480in}}%
\pgfpathcurveto{\pgfqpoint{1.613954in}{2.362243in}}{\pgfqpoint{1.617226in}{2.354343in}}{\pgfqpoint{1.623050in}{2.348519in}}%
\pgfpathcurveto{\pgfqpoint{1.628874in}{2.342695in}}{\pgfqpoint{1.636774in}{2.339423in}}{\pgfqpoint{1.645011in}{2.339423in}}%
\pgfpathclose%
\pgfusepath{stroke,fill}%
\end{pgfscope}%
\begin{pgfscope}%
\pgfpathrectangle{\pgfqpoint{0.100000in}{0.212622in}}{\pgfqpoint{3.696000in}{3.696000in}}%
\pgfusepath{clip}%
\pgfsetbuttcap%
\pgfsetroundjoin%
\definecolor{currentfill}{rgb}{0.121569,0.466667,0.705882}%
\pgfsetfillcolor{currentfill}%
\pgfsetfillopacity{0.301845}%
\pgfsetlinewidth{1.003750pt}%
\definecolor{currentstroke}{rgb}{0.121569,0.466667,0.705882}%
\pgfsetstrokecolor{currentstroke}%
\pgfsetstrokeopacity{0.301845}%
\pgfsetdash{}{0pt}%
\pgfpathmoveto{\pgfqpoint{1.644355in}{2.338308in}}%
\pgfpathcurveto{\pgfqpoint{1.652592in}{2.338308in}}{\pgfqpoint{1.660492in}{2.341580in}}{\pgfqpoint{1.666316in}{2.347404in}}%
\pgfpathcurveto{\pgfqpoint{1.672140in}{2.353228in}}{\pgfqpoint{1.675412in}{2.361128in}}{\pgfqpoint{1.675412in}{2.369364in}}%
\pgfpathcurveto{\pgfqpoint{1.675412in}{2.377601in}}{\pgfqpoint{1.672140in}{2.385501in}}{\pgfqpoint{1.666316in}{2.391325in}}%
\pgfpathcurveto{\pgfqpoint{1.660492in}{2.397149in}}{\pgfqpoint{1.652592in}{2.400421in}}{\pgfqpoint{1.644355in}{2.400421in}}%
\pgfpathcurveto{\pgfqpoint{1.636119in}{2.400421in}}{\pgfqpoint{1.628219in}{2.397149in}}{\pgfqpoint{1.622395in}{2.391325in}}%
\pgfpathcurveto{\pgfqpoint{1.616571in}{2.385501in}}{\pgfqpoint{1.613299in}{2.377601in}}{\pgfqpoint{1.613299in}{2.369364in}}%
\pgfpathcurveto{\pgfqpoint{1.613299in}{2.361128in}}{\pgfqpoint{1.616571in}{2.353228in}}{\pgfqpoint{1.622395in}{2.347404in}}%
\pgfpathcurveto{\pgfqpoint{1.628219in}{2.341580in}}{\pgfqpoint{1.636119in}{2.338308in}}{\pgfqpoint{1.644355in}{2.338308in}}%
\pgfpathclose%
\pgfusepath{stroke,fill}%
\end{pgfscope}%
\begin{pgfscope}%
\pgfpathrectangle{\pgfqpoint{0.100000in}{0.212622in}}{\pgfqpoint{3.696000in}{3.696000in}}%
\pgfusepath{clip}%
\pgfsetbuttcap%
\pgfsetroundjoin%
\definecolor{currentfill}{rgb}{0.121569,0.466667,0.705882}%
\pgfsetfillcolor{currentfill}%
\pgfsetfillopacity{0.302253}%
\pgfsetlinewidth{1.003750pt}%
\definecolor{currentstroke}{rgb}{0.121569,0.466667,0.705882}%
\pgfsetstrokecolor{currentstroke}%
\pgfsetstrokeopacity{0.302253}%
\pgfsetdash{}{0pt}%
\pgfpathmoveto{\pgfqpoint{1.698426in}{2.350113in}}%
\pgfpathcurveto{\pgfqpoint{1.706662in}{2.350113in}}{\pgfqpoint{1.714562in}{2.353386in}}{\pgfqpoint{1.720386in}{2.359209in}}%
\pgfpathcurveto{\pgfqpoint{1.726210in}{2.365033in}}{\pgfqpoint{1.729483in}{2.372933in}}{\pgfqpoint{1.729483in}{2.381170in}}%
\pgfpathcurveto{\pgfqpoint{1.729483in}{2.389406in}}{\pgfqpoint{1.726210in}{2.397306in}}{\pgfqpoint{1.720386in}{2.403130in}}%
\pgfpathcurveto{\pgfqpoint{1.714562in}{2.408954in}}{\pgfqpoint{1.706662in}{2.412226in}}{\pgfqpoint{1.698426in}{2.412226in}}%
\pgfpathcurveto{\pgfqpoint{1.690190in}{2.412226in}}{\pgfqpoint{1.682290in}{2.408954in}}{\pgfqpoint{1.676466in}{2.403130in}}%
\pgfpathcurveto{\pgfqpoint{1.670642in}{2.397306in}}{\pgfqpoint{1.667370in}{2.389406in}}{\pgfqpoint{1.667370in}{2.381170in}}%
\pgfpathcurveto{\pgfqpoint{1.667370in}{2.372933in}}{\pgfqpoint{1.670642in}{2.365033in}}{\pgfqpoint{1.676466in}{2.359209in}}%
\pgfpathcurveto{\pgfqpoint{1.682290in}{2.353386in}}{\pgfqpoint{1.690190in}{2.350113in}}{\pgfqpoint{1.698426in}{2.350113in}}%
\pgfpathclose%
\pgfusepath{stroke,fill}%
\end{pgfscope}%
\begin{pgfscope}%
\pgfpathrectangle{\pgfqpoint{0.100000in}{0.212622in}}{\pgfqpoint{3.696000in}{3.696000in}}%
\pgfusepath{clip}%
\pgfsetbuttcap%
\pgfsetroundjoin%
\definecolor{currentfill}{rgb}{0.121569,0.466667,0.705882}%
\pgfsetfillcolor{currentfill}%
\pgfsetfillopacity{0.302387}%
\pgfsetlinewidth{1.003750pt}%
\definecolor{currentstroke}{rgb}{0.121569,0.466667,0.705882}%
\pgfsetstrokecolor{currentstroke}%
\pgfsetstrokeopacity{0.302387}%
\pgfsetdash{}{0pt}%
\pgfpathmoveto{\pgfqpoint{1.642958in}{2.336062in}}%
\pgfpathcurveto{\pgfqpoint{1.651194in}{2.336062in}}{\pgfqpoint{1.659094in}{2.339335in}}{\pgfqpoint{1.664918in}{2.345158in}}%
\pgfpathcurveto{\pgfqpoint{1.670742in}{2.350982in}}{\pgfqpoint{1.674015in}{2.358882in}}{\pgfqpoint{1.674015in}{2.367119in}}%
\pgfpathcurveto{\pgfqpoint{1.674015in}{2.375355in}}{\pgfqpoint{1.670742in}{2.383255in}}{\pgfqpoint{1.664918in}{2.389079in}}%
\pgfpathcurveto{\pgfqpoint{1.659094in}{2.394903in}}{\pgfqpoint{1.651194in}{2.398175in}}{\pgfqpoint{1.642958in}{2.398175in}}%
\pgfpathcurveto{\pgfqpoint{1.634722in}{2.398175in}}{\pgfqpoint{1.626822in}{2.394903in}}{\pgfqpoint{1.620998in}{2.389079in}}%
\pgfpathcurveto{\pgfqpoint{1.615174in}{2.383255in}}{\pgfqpoint{1.611902in}{2.375355in}}{\pgfqpoint{1.611902in}{2.367119in}}%
\pgfpathcurveto{\pgfqpoint{1.611902in}{2.358882in}}{\pgfqpoint{1.615174in}{2.350982in}}{\pgfqpoint{1.620998in}{2.345158in}}%
\pgfpathcurveto{\pgfqpoint{1.626822in}{2.339335in}}{\pgfqpoint{1.634722in}{2.336062in}}{\pgfqpoint{1.642958in}{2.336062in}}%
\pgfpathclose%
\pgfusepath{stroke,fill}%
\end{pgfscope}%
\begin{pgfscope}%
\pgfpathrectangle{\pgfqpoint{0.100000in}{0.212622in}}{\pgfqpoint{3.696000in}{3.696000in}}%
\pgfusepath{clip}%
\pgfsetbuttcap%
\pgfsetroundjoin%
\definecolor{currentfill}{rgb}{0.121569,0.466667,0.705882}%
\pgfsetfillcolor{currentfill}%
\pgfsetfillopacity{0.303503}%
\pgfsetlinewidth{1.003750pt}%
\definecolor{currentstroke}{rgb}{0.121569,0.466667,0.705882}%
\pgfsetstrokecolor{currentstroke}%
\pgfsetstrokeopacity{0.303503}%
\pgfsetdash{}{0pt}%
\pgfpathmoveto{\pgfqpoint{1.640755in}{2.332628in}}%
\pgfpathcurveto{\pgfqpoint{1.648991in}{2.332628in}}{\pgfqpoint{1.656891in}{2.335900in}}{\pgfqpoint{1.662715in}{2.341724in}}%
\pgfpathcurveto{\pgfqpoint{1.668539in}{2.347548in}}{\pgfqpoint{1.671811in}{2.355448in}}{\pgfqpoint{1.671811in}{2.363684in}}%
\pgfpathcurveto{\pgfqpoint{1.671811in}{2.371920in}}{\pgfqpoint{1.668539in}{2.379820in}}{\pgfqpoint{1.662715in}{2.385644in}}%
\pgfpathcurveto{\pgfqpoint{1.656891in}{2.391468in}}{\pgfqpoint{1.648991in}{2.394741in}}{\pgfqpoint{1.640755in}{2.394741in}}%
\pgfpathcurveto{\pgfqpoint{1.632519in}{2.394741in}}{\pgfqpoint{1.624619in}{2.391468in}}{\pgfqpoint{1.618795in}{2.385644in}}%
\pgfpathcurveto{\pgfqpoint{1.612971in}{2.379820in}}{\pgfqpoint{1.609698in}{2.371920in}}{\pgfqpoint{1.609698in}{2.363684in}}%
\pgfpathcurveto{\pgfqpoint{1.609698in}{2.355448in}}{\pgfqpoint{1.612971in}{2.347548in}}{\pgfqpoint{1.618795in}{2.341724in}}%
\pgfpathcurveto{\pgfqpoint{1.624619in}{2.335900in}}{\pgfqpoint{1.632519in}{2.332628in}}{\pgfqpoint{1.640755in}{2.332628in}}%
\pgfpathclose%
\pgfusepath{stroke,fill}%
\end{pgfscope}%
\begin{pgfscope}%
\pgfpathrectangle{\pgfqpoint{0.100000in}{0.212622in}}{\pgfqpoint{3.696000in}{3.696000in}}%
\pgfusepath{clip}%
\pgfsetbuttcap%
\pgfsetroundjoin%
\definecolor{currentfill}{rgb}{0.121569,0.466667,0.705882}%
\pgfsetfillcolor{currentfill}%
\pgfsetfillopacity{0.304887}%
\pgfsetlinewidth{1.003750pt}%
\definecolor{currentstroke}{rgb}{0.121569,0.466667,0.705882}%
\pgfsetstrokecolor{currentstroke}%
\pgfsetstrokeopacity{0.304887}%
\pgfsetdash{}{0pt}%
\pgfpathmoveto{\pgfqpoint{1.714482in}{2.348675in}}%
\pgfpathcurveto{\pgfqpoint{1.722718in}{2.348675in}}{\pgfqpoint{1.730618in}{2.351947in}}{\pgfqpoint{1.736442in}{2.357771in}}%
\pgfpathcurveto{\pgfqpoint{1.742266in}{2.363595in}}{\pgfqpoint{1.745538in}{2.371495in}}{\pgfqpoint{1.745538in}{2.379732in}}%
\pgfpathcurveto{\pgfqpoint{1.745538in}{2.387968in}}{\pgfqpoint{1.742266in}{2.395868in}}{\pgfqpoint{1.736442in}{2.401692in}}%
\pgfpathcurveto{\pgfqpoint{1.730618in}{2.407516in}}{\pgfqpoint{1.722718in}{2.410788in}}{\pgfqpoint{1.714482in}{2.410788in}}%
\pgfpathcurveto{\pgfqpoint{1.706245in}{2.410788in}}{\pgfqpoint{1.698345in}{2.407516in}}{\pgfqpoint{1.692522in}{2.401692in}}%
\pgfpathcurveto{\pgfqpoint{1.686698in}{2.395868in}}{\pgfqpoint{1.683425in}{2.387968in}}{\pgfqpoint{1.683425in}{2.379732in}}%
\pgfpathcurveto{\pgfqpoint{1.683425in}{2.371495in}}{\pgfqpoint{1.686698in}{2.363595in}}{\pgfqpoint{1.692522in}{2.357771in}}%
\pgfpathcurveto{\pgfqpoint{1.698345in}{2.351947in}}{\pgfqpoint{1.706245in}{2.348675in}}{\pgfqpoint{1.714482in}{2.348675in}}%
\pgfpathclose%
\pgfusepath{stroke,fill}%
\end{pgfscope}%
\begin{pgfscope}%
\pgfpathrectangle{\pgfqpoint{0.100000in}{0.212622in}}{\pgfqpoint{3.696000in}{3.696000in}}%
\pgfusepath{clip}%
\pgfsetbuttcap%
\pgfsetroundjoin%
\definecolor{currentfill}{rgb}{0.121569,0.466667,0.705882}%
\pgfsetfillcolor{currentfill}%
\pgfsetfillopacity{0.305417}%
\pgfsetlinewidth{1.003750pt}%
\definecolor{currentstroke}{rgb}{0.121569,0.466667,0.705882}%
\pgfsetstrokecolor{currentstroke}%
\pgfsetstrokeopacity{0.305417}%
\pgfsetdash{}{0pt}%
\pgfpathmoveto{\pgfqpoint{1.636061in}{2.326097in}}%
\pgfpathcurveto{\pgfqpoint{1.644298in}{2.326097in}}{\pgfqpoint{1.652198in}{2.329370in}}{\pgfqpoint{1.658022in}{2.335193in}}%
\pgfpathcurveto{\pgfqpoint{1.663845in}{2.341017in}}{\pgfqpoint{1.667118in}{2.348917in}}{\pgfqpoint{1.667118in}{2.357154in}}%
\pgfpathcurveto{\pgfqpoint{1.667118in}{2.365390in}}{\pgfqpoint{1.663845in}{2.373290in}}{\pgfqpoint{1.658022in}{2.379114in}}%
\pgfpathcurveto{\pgfqpoint{1.652198in}{2.384938in}}{\pgfqpoint{1.644298in}{2.388210in}}{\pgfqpoint{1.636061in}{2.388210in}}%
\pgfpathcurveto{\pgfqpoint{1.627825in}{2.388210in}}{\pgfqpoint{1.619925in}{2.384938in}}{\pgfqpoint{1.614101in}{2.379114in}}%
\pgfpathcurveto{\pgfqpoint{1.608277in}{2.373290in}}{\pgfqpoint{1.605005in}{2.365390in}}{\pgfqpoint{1.605005in}{2.357154in}}%
\pgfpathcurveto{\pgfqpoint{1.605005in}{2.348917in}}{\pgfqpoint{1.608277in}{2.341017in}}{\pgfqpoint{1.614101in}{2.335193in}}%
\pgfpathcurveto{\pgfqpoint{1.619925in}{2.329370in}}{\pgfqpoint{1.627825in}{2.326097in}}{\pgfqpoint{1.636061in}{2.326097in}}%
\pgfpathclose%
\pgfusepath{stroke,fill}%
\end{pgfscope}%
\begin{pgfscope}%
\pgfpathrectangle{\pgfqpoint{0.100000in}{0.212622in}}{\pgfqpoint{3.696000in}{3.696000in}}%
\pgfusepath{clip}%
\pgfsetbuttcap%
\pgfsetroundjoin%
\definecolor{currentfill}{rgb}{0.121569,0.466667,0.705882}%
\pgfsetfillcolor{currentfill}%
\pgfsetfillopacity{0.307800}%
\pgfsetlinewidth{1.003750pt}%
\definecolor{currentstroke}{rgb}{0.121569,0.466667,0.705882}%
\pgfsetstrokecolor{currentstroke}%
\pgfsetstrokeopacity{0.307800}%
\pgfsetdash{}{0pt}%
\pgfpathmoveto{\pgfqpoint{1.733905in}{2.349403in}}%
\pgfpathcurveto{\pgfqpoint{1.742141in}{2.349403in}}{\pgfqpoint{1.750041in}{2.352676in}}{\pgfqpoint{1.755865in}{2.358500in}}%
\pgfpathcurveto{\pgfqpoint{1.761689in}{2.364324in}}{\pgfqpoint{1.764961in}{2.372224in}}{\pgfqpoint{1.764961in}{2.380460in}}%
\pgfpathcurveto{\pgfqpoint{1.764961in}{2.388696in}}{\pgfqpoint{1.761689in}{2.396596in}}{\pgfqpoint{1.755865in}{2.402420in}}%
\pgfpathcurveto{\pgfqpoint{1.750041in}{2.408244in}}{\pgfqpoint{1.742141in}{2.411516in}}{\pgfqpoint{1.733905in}{2.411516in}}%
\pgfpathcurveto{\pgfqpoint{1.725668in}{2.411516in}}{\pgfqpoint{1.717768in}{2.408244in}}{\pgfqpoint{1.711944in}{2.402420in}}%
\pgfpathcurveto{\pgfqpoint{1.706120in}{2.396596in}}{\pgfqpoint{1.702848in}{2.388696in}}{\pgfqpoint{1.702848in}{2.380460in}}%
\pgfpathcurveto{\pgfqpoint{1.702848in}{2.372224in}}{\pgfqpoint{1.706120in}{2.364324in}}{\pgfqpoint{1.711944in}{2.358500in}}%
\pgfpathcurveto{\pgfqpoint{1.717768in}{2.352676in}}{\pgfqpoint{1.725668in}{2.349403in}}{\pgfqpoint{1.733905in}{2.349403in}}%
\pgfpathclose%
\pgfusepath{stroke,fill}%
\end{pgfscope}%
\begin{pgfscope}%
\pgfpathrectangle{\pgfqpoint{0.100000in}{0.212622in}}{\pgfqpoint{3.696000in}{3.696000in}}%
\pgfusepath{clip}%
\pgfsetbuttcap%
\pgfsetroundjoin%
\definecolor{currentfill}{rgb}{0.121569,0.466667,0.705882}%
\pgfsetfillcolor{currentfill}%
\pgfsetfillopacity{0.308733}%
\pgfsetlinewidth{1.003750pt}%
\definecolor{currentstroke}{rgb}{0.121569,0.466667,0.705882}%
\pgfsetstrokecolor{currentstroke}%
\pgfsetstrokeopacity{0.308733}%
\pgfsetdash{}{0pt}%
\pgfpathmoveto{\pgfqpoint{1.627511in}{2.313025in}}%
\pgfpathcurveto{\pgfqpoint{1.635747in}{2.313025in}}{\pgfqpoint{1.643647in}{2.316298in}}{\pgfqpoint{1.649471in}{2.322122in}}%
\pgfpathcurveto{\pgfqpoint{1.655295in}{2.327946in}}{\pgfqpoint{1.658567in}{2.335846in}}{\pgfqpoint{1.658567in}{2.344082in}}%
\pgfpathcurveto{\pgfqpoint{1.658567in}{2.352318in}}{\pgfqpoint{1.655295in}{2.360218in}}{\pgfqpoint{1.649471in}{2.366042in}}%
\pgfpathcurveto{\pgfqpoint{1.643647in}{2.371866in}}{\pgfqpoint{1.635747in}{2.375138in}}{\pgfqpoint{1.627511in}{2.375138in}}%
\pgfpathcurveto{\pgfqpoint{1.619275in}{2.375138in}}{\pgfqpoint{1.611374in}{2.371866in}}{\pgfqpoint{1.605551in}{2.366042in}}%
\pgfpathcurveto{\pgfqpoint{1.599727in}{2.360218in}}{\pgfqpoint{1.596454in}{2.352318in}}{\pgfqpoint{1.596454in}{2.344082in}}%
\pgfpathcurveto{\pgfqpoint{1.596454in}{2.335846in}}{\pgfqpoint{1.599727in}{2.327946in}}{\pgfqpoint{1.605551in}{2.322122in}}%
\pgfpathcurveto{\pgfqpoint{1.611374in}{2.316298in}}{\pgfqpoint{1.619275in}{2.313025in}}{\pgfqpoint{1.627511in}{2.313025in}}%
\pgfpathclose%
\pgfusepath{stroke,fill}%
\end{pgfscope}%
\begin{pgfscope}%
\pgfpathrectangle{\pgfqpoint{0.100000in}{0.212622in}}{\pgfqpoint{3.696000in}{3.696000in}}%
\pgfusepath{clip}%
\pgfsetbuttcap%
\pgfsetroundjoin%
\definecolor{currentfill}{rgb}{0.121569,0.466667,0.705882}%
\pgfsetfillcolor{currentfill}%
\pgfsetfillopacity{0.310918}%
\pgfsetlinewidth{1.003750pt}%
\definecolor{currentstroke}{rgb}{0.121569,0.466667,0.705882}%
\pgfsetstrokecolor{currentstroke}%
\pgfsetstrokeopacity{0.310918}%
\pgfsetdash{}{0pt}%
\pgfpathmoveto{\pgfqpoint{1.755907in}{2.344792in}}%
\pgfpathcurveto{\pgfqpoint{1.764143in}{2.344792in}}{\pgfqpoint{1.772043in}{2.348065in}}{\pgfqpoint{1.777867in}{2.353889in}}%
\pgfpathcurveto{\pgfqpoint{1.783691in}{2.359712in}}{\pgfqpoint{1.786963in}{2.367613in}}{\pgfqpoint{1.786963in}{2.375849in}}%
\pgfpathcurveto{\pgfqpoint{1.786963in}{2.384085in}}{\pgfqpoint{1.783691in}{2.391985in}}{\pgfqpoint{1.777867in}{2.397809in}}%
\pgfpathcurveto{\pgfqpoint{1.772043in}{2.403633in}}{\pgfqpoint{1.764143in}{2.406905in}}{\pgfqpoint{1.755907in}{2.406905in}}%
\pgfpathcurveto{\pgfqpoint{1.747671in}{2.406905in}}{\pgfqpoint{1.739771in}{2.403633in}}{\pgfqpoint{1.733947in}{2.397809in}}%
\pgfpathcurveto{\pgfqpoint{1.728123in}{2.391985in}}{\pgfqpoint{1.724850in}{2.384085in}}{\pgfqpoint{1.724850in}{2.375849in}}%
\pgfpathcurveto{\pgfqpoint{1.724850in}{2.367613in}}{\pgfqpoint{1.728123in}{2.359712in}}{\pgfqpoint{1.733947in}{2.353889in}}%
\pgfpathcurveto{\pgfqpoint{1.739771in}{2.348065in}}{\pgfqpoint{1.747671in}{2.344792in}}{\pgfqpoint{1.755907in}{2.344792in}}%
\pgfpathclose%
\pgfusepath{stroke,fill}%
\end{pgfscope}%
\begin{pgfscope}%
\pgfpathrectangle{\pgfqpoint{0.100000in}{0.212622in}}{\pgfqpoint{3.696000in}{3.696000in}}%
\pgfusepath{clip}%
\pgfsetbuttcap%
\pgfsetroundjoin%
\definecolor{currentfill}{rgb}{0.121569,0.466667,0.705882}%
\pgfsetfillcolor{currentfill}%
\pgfsetfillopacity{0.312739}%
\pgfsetlinewidth{1.003750pt}%
\definecolor{currentstroke}{rgb}{0.121569,0.466667,0.705882}%
\pgfsetstrokecolor{currentstroke}%
\pgfsetstrokeopacity{0.312739}%
\pgfsetdash{}{0pt}%
\pgfpathmoveto{\pgfqpoint{1.767961in}{2.342813in}}%
\pgfpathcurveto{\pgfqpoint{1.776198in}{2.342813in}}{\pgfqpoint{1.784098in}{2.346085in}}{\pgfqpoint{1.789921in}{2.351909in}}%
\pgfpathcurveto{\pgfqpoint{1.795745in}{2.357733in}}{\pgfqpoint{1.799018in}{2.365633in}}{\pgfqpoint{1.799018in}{2.373870in}}%
\pgfpathcurveto{\pgfqpoint{1.799018in}{2.382106in}}{\pgfqpoint{1.795745in}{2.390006in}}{\pgfqpoint{1.789921in}{2.395830in}}%
\pgfpathcurveto{\pgfqpoint{1.784098in}{2.401654in}}{\pgfqpoint{1.776198in}{2.404926in}}{\pgfqpoint{1.767961in}{2.404926in}}%
\pgfpathcurveto{\pgfqpoint{1.759725in}{2.404926in}}{\pgfqpoint{1.751825in}{2.401654in}}{\pgfqpoint{1.746001in}{2.395830in}}%
\pgfpathcurveto{\pgfqpoint{1.740177in}{2.390006in}}{\pgfqpoint{1.736905in}{2.382106in}}{\pgfqpoint{1.736905in}{2.373870in}}%
\pgfpathcurveto{\pgfqpoint{1.736905in}{2.365633in}}{\pgfqpoint{1.740177in}{2.357733in}}{\pgfqpoint{1.746001in}{2.351909in}}%
\pgfpathcurveto{\pgfqpoint{1.751825in}{2.346085in}}{\pgfqpoint{1.759725in}{2.342813in}}{\pgfqpoint{1.767961in}{2.342813in}}%
\pgfpathclose%
\pgfusepath{stroke,fill}%
\end{pgfscope}%
\begin{pgfscope}%
\pgfpathrectangle{\pgfqpoint{0.100000in}{0.212622in}}{\pgfqpoint{3.696000in}{3.696000in}}%
\pgfusepath{clip}%
\pgfsetbuttcap%
\pgfsetroundjoin%
\definecolor{currentfill}{rgb}{0.121569,0.466667,0.705882}%
\pgfsetfillcolor{currentfill}%
\pgfsetfillopacity{0.315087}%
\pgfsetlinewidth{1.003750pt}%
\definecolor{currentstroke}{rgb}{0.121569,0.466667,0.705882}%
\pgfsetstrokecolor{currentstroke}%
\pgfsetstrokeopacity{0.315087}%
\pgfsetdash{}{0pt}%
\pgfpathmoveto{\pgfqpoint{1.609202in}{2.294138in}}%
\pgfpathcurveto{\pgfqpoint{1.617438in}{2.294138in}}{\pgfqpoint{1.625338in}{2.297410in}}{\pgfqpoint{1.631162in}{2.303234in}}%
\pgfpathcurveto{\pgfqpoint{1.636986in}{2.309058in}}{\pgfqpoint{1.640258in}{2.316958in}}{\pgfqpoint{1.640258in}{2.325194in}}%
\pgfpathcurveto{\pgfqpoint{1.640258in}{2.333431in}}{\pgfqpoint{1.636986in}{2.341331in}}{\pgfqpoint{1.631162in}{2.347155in}}%
\pgfpathcurveto{\pgfqpoint{1.625338in}{2.352979in}}{\pgfqpoint{1.617438in}{2.356251in}}{\pgfqpoint{1.609202in}{2.356251in}}%
\pgfpathcurveto{\pgfqpoint{1.600965in}{2.356251in}}{\pgfqpoint{1.593065in}{2.352979in}}{\pgfqpoint{1.587241in}{2.347155in}}%
\pgfpathcurveto{\pgfqpoint{1.581417in}{2.341331in}}{\pgfqpoint{1.578145in}{2.333431in}}{\pgfqpoint{1.578145in}{2.325194in}}%
\pgfpathcurveto{\pgfqpoint{1.578145in}{2.316958in}}{\pgfqpoint{1.581417in}{2.309058in}}{\pgfqpoint{1.587241in}{2.303234in}}%
\pgfpathcurveto{\pgfqpoint{1.593065in}{2.297410in}}{\pgfqpoint{1.600965in}{2.294138in}}{\pgfqpoint{1.609202in}{2.294138in}}%
\pgfpathclose%
\pgfusepath{stroke,fill}%
\end{pgfscope}%
\begin{pgfscope}%
\pgfpathrectangle{\pgfqpoint{0.100000in}{0.212622in}}{\pgfqpoint{3.696000in}{3.696000in}}%
\pgfusepath{clip}%
\pgfsetbuttcap%
\pgfsetroundjoin%
\definecolor{currentfill}{rgb}{0.121569,0.466667,0.705882}%
\pgfsetfillcolor{currentfill}%
\pgfsetfillopacity{0.315432}%
\pgfsetlinewidth{1.003750pt}%
\definecolor{currentstroke}{rgb}{0.121569,0.466667,0.705882}%
\pgfsetstrokecolor{currentstroke}%
\pgfsetstrokeopacity{0.315432}%
\pgfsetdash{}{0pt}%
\pgfpathmoveto{\pgfqpoint{1.782353in}{2.341738in}}%
\pgfpathcurveto{\pgfqpoint{1.790590in}{2.341738in}}{\pgfqpoint{1.798490in}{2.345010in}}{\pgfqpoint{1.804314in}{2.350834in}}%
\pgfpathcurveto{\pgfqpoint{1.810137in}{2.356658in}}{\pgfqpoint{1.813410in}{2.364558in}}{\pgfqpoint{1.813410in}{2.372794in}}%
\pgfpathcurveto{\pgfqpoint{1.813410in}{2.381030in}}{\pgfqpoint{1.810137in}{2.388930in}}{\pgfqpoint{1.804314in}{2.394754in}}%
\pgfpathcurveto{\pgfqpoint{1.798490in}{2.400578in}}{\pgfqpoint{1.790590in}{2.403851in}}{\pgfqpoint{1.782353in}{2.403851in}}%
\pgfpathcurveto{\pgfqpoint{1.774117in}{2.403851in}}{\pgfqpoint{1.766217in}{2.400578in}}{\pgfqpoint{1.760393in}{2.394754in}}%
\pgfpathcurveto{\pgfqpoint{1.754569in}{2.388930in}}{\pgfqpoint{1.751297in}{2.381030in}}{\pgfqpoint{1.751297in}{2.372794in}}%
\pgfpathcurveto{\pgfqpoint{1.751297in}{2.364558in}}{\pgfqpoint{1.754569in}{2.356658in}}{\pgfqpoint{1.760393in}{2.350834in}}%
\pgfpathcurveto{\pgfqpoint{1.766217in}{2.345010in}}{\pgfqpoint{1.774117in}{2.341738in}}{\pgfqpoint{1.782353in}{2.341738in}}%
\pgfpathclose%
\pgfusepath{stroke,fill}%
\end{pgfscope}%
\begin{pgfscope}%
\pgfpathrectangle{\pgfqpoint{0.100000in}{0.212622in}}{\pgfqpoint{3.696000in}{3.696000in}}%
\pgfusepath{clip}%
\pgfsetbuttcap%
\pgfsetroundjoin%
\definecolor{currentfill}{rgb}{0.121569,0.466667,0.705882}%
\pgfsetfillcolor{currentfill}%
\pgfsetfillopacity{0.317105}%
\pgfsetlinewidth{1.003750pt}%
\definecolor{currentstroke}{rgb}{0.121569,0.466667,0.705882}%
\pgfsetstrokecolor{currentstroke}%
\pgfsetstrokeopacity{0.317105}%
\pgfsetdash{}{0pt}%
\pgfpathmoveto{\pgfqpoint{1.790160in}{2.342177in}}%
\pgfpathcurveto{\pgfqpoint{1.798396in}{2.342177in}}{\pgfqpoint{1.806296in}{2.345449in}}{\pgfqpoint{1.812120in}{2.351273in}}%
\pgfpathcurveto{\pgfqpoint{1.817944in}{2.357097in}}{\pgfqpoint{1.821216in}{2.364997in}}{\pgfqpoint{1.821216in}{2.373233in}}%
\pgfpathcurveto{\pgfqpoint{1.821216in}{2.381469in}}{\pgfqpoint{1.817944in}{2.389369in}}{\pgfqpoint{1.812120in}{2.395193in}}%
\pgfpathcurveto{\pgfqpoint{1.806296in}{2.401017in}}{\pgfqpoint{1.798396in}{2.404290in}}{\pgfqpoint{1.790160in}{2.404290in}}%
\pgfpathcurveto{\pgfqpoint{1.781924in}{2.404290in}}{\pgfqpoint{1.774024in}{2.401017in}}{\pgfqpoint{1.768200in}{2.395193in}}%
\pgfpathcurveto{\pgfqpoint{1.762376in}{2.389369in}}{\pgfqpoint{1.759103in}{2.381469in}}{\pgfqpoint{1.759103in}{2.373233in}}%
\pgfpathcurveto{\pgfqpoint{1.759103in}{2.364997in}}{\pgfqpoint{1.762376in}{2.357097in}}{\pgfqpoint{1.768200in}{2.351273in}}%
\pgfpathcurveto{\pgfqpoint{1.774024in}{2.345449in}}{\pgfqpoint{1.781924in}{2.342177in}}{\pgfqpoint{1.790160in}{2.342177in}}%
\pgfpathclose%
\pgfusepath{stroke,fill}%
\end{pgfscope}%
\begin{pgfscope}%
\pgfpathrectangle{\pgfqpoint{0.100000in}{0.212622in}}{\pgfqpoint{3.696000in}{3.696000in}}%
\pgfusepath{clip}%
\pgfsetbuttcap%
\pgfsetroundjoin%
\definecolor{currentfill}{rgb}{0.121569,0.466667,0.705882}%
\pgfsetfillcolor{currentfill}%
\pgfsetfillopacity{0.318823}%
\pgfsetlinewidth{1.003750pt}%
\definecolor{currentstroke}{rgb}{0.121569,0.466667,0.705882}%
\pgfsetstrokecolor{currentstroke}%
\pgfsetstrokeopacity{0.318823}%
\pgfsetdash{}{0pt}%
\pgfpathmoveto{\pgfqpoint{1.800055in}{2.340775in}}%
\pgfpathcurveto{\pgfqpoint{1.808291in}{2.340775in}}{\pgfqpoint{1.816191in}{2.344047in}}{\pgfqpoint{1.822015in}{2.349871in}}%
\pgfpathcurveto{\pgfqpoint{1.827839in}{2.355695in}}{\pgfqpoint{1.831111in}{2.363595in}}{\pgfqpoint{1.831111in}{2.371832in}}%
\pgfpathcurveto{\pgfqpoint{1.831111in}{2.380068in}}{\pgfqpoint{1.827839in}{2.387968in}}{\pgfqpoint{1.822015in}{2.393792in}}%
\pgfpathcurveto{\pgfqpoint{1.816191in}{2.399616in}}{\pgfqpoint{1.808291in}{2.402888in}}{\pgfqpoint{1.800055in}{2.402888in}}%
\pgfpathcurveto{\pgfqpoint{1.791818in}{2.402888in}}{\pgfqpoint{1.783918in}{2.399616in}}{\pgfqpoint{1.778094in}{2.393792in}}%
\pgfpathcurveto{\pgfqpoint{1.772270in}{2.387968in}}{\pgfqpoint{1.768998in}{2.380068in}}{\pgfqpoint{1.768998in}{2.371832in}}%
\pgfpathcurveto{\pgfqpoint{1.768998in}{2.363595in}}{\pgfqpoint{1.772270in}{2.355695in}}{\pgfqpoint{1.778094in}{2.349871in}}%
\pgfpathcurveto{\pgfqpoint{1.783918in}{2.344047in}}{\pgfqpoint{1.791818in}{2.340775in}}{\pgfqpoint{1.800055in}{2.340775in}}%
\pgfpathclose%
\pgfusepath{stroke,fill}%
\end{pgfscope}%
\begin{pgfscope}%
\pgfpathrectangle{\pgfqpoint{0.100000in}{0.212622in}}{\pgfqpoint{3.696000in}{3.696000in}}%
\pgfusepath{clip}%
\pgfsetbuttcap%
\pgfsetroundjoin%
\definecolor{currentfill}{rgb}{0.121569,0.466667,0.705882}%
\pgfsetfillcolor{currentfill}%
\pgfsetfillopacity{0.320230}%
\pgfsetlinewidth{1.003750pt}%
\definecolor{currentstroke}{rgb}{0.121569,0.466667,0.705882}%
\pgfsetstrokecolor{currentstroke}%
\pgfsetstrokeopacity{0.320230}%
\pgfsetdash{}{0pt}%
\pgfpathmoveto{\pgfqpoint{1.594926in}{2.274195in}}%
\pgfpathcurveto{\pgfqpoint{1.603163in}{2.274195in}}{\pgfqpoint{1.611063in}{2.277467in}}{\pgfqpoint{1.616887in}{2.283291in}}%
\pgfpathcurveto{\pgfqpoint{1.622711in}{2.289115in}}{\pgfqpoint{1.625983in}{2.297015in}}{\pgfqpoint{1.625983in}{2.305251in}}%
\pgfpathcurveto{\pgfqpoint{1.625983in}{2.313487in}}{\pgfqpoint{1.622711in}{2.321387in}}{\pgfqpoint{1.616887in}{2.327211in}}%
\pgfpathcurveto{\pgfqpoint{1.611063in}{2.333035in}}{\pgfqpoint{1.603163in}{2.336308in}}{\pgfqpoint{1.594926in}{2.336308in}}%
\pgfpathcurveto{\pgfqpoint{1.586690in}{2.336308in}}{\pgfqpoint{1.578790in}{2.333035in}}{\pgfqpoint{1.572966in}{2.327211in}}%
\pgfpathcurveto{\pgfqpoint{1.567142in}{2.321387in}}{\pgfqpoint{1.563870in}{2.313487in}}{\pgfqpoint{1.563870in}{2.305251in}}%
\pgfpathcurveto{\pgfqpoint{1.563870in}{2.297015in}}{\pgfqpoint{1.567142in}{2.289115in}}{\pgfqpoint{1.572966in}{2.283291in}}%
\pgfpathcurveto{\pgfqpoint{1.578790in}{2.277467in}}{\pgfqpoint{1.586690in}{2.274195in}}{\pgfqpoint{1.594926in}{2.274195in}}%
\pgfpathclose%
\pgfusepath{stroke,fill}%
\end{pgfscope}%
\begin{pgfscope}%
\pgfpathrectangle{\pgfqpoint{0.100000in}{0.212622in}}{\pgfqpoint{3.696000in}{3.696000in}}%
\pgfusepath{clip}%
\pgfsetbuttcap%
\pgfsetroundjoin%
\definecolor{currentfill}{rgb}{0.121569,0.466667,0.705882}%
\pgfsetfillcolor{currentfill}%
\pgfsetfillopacity{0.322550}%
\pgfsetlinewidth{1.003750pt}%
\definecolor{currentstroke}{rgb}{0.121569,0.466667,0.705882}%
\pgfsetstrokecolor{currentstroke}%
\pgfsetstrokeopacity{0.322550}%
\pgfsetdash{}{0pt}%
\pgfpathmoveto{\pgfqpoint{1.818161in}{2.339507in}}%
\pgfpathcurveto{\pgfqpoint{1.826397in}{2.339507in}}{\pgfqpoint{1.834297in}{2.342780in}}{\pgfqpoint{1.840121in}{2.348604in}}%
\pgfpathcurveto{\pgfqpoint{1.845945in}{2.354427in}}{\pgfqpoint{1.849217in}{2.362328in}}{\pgfqpoint{1.849217in}{2.370564in}}%
\pgfpathcurveto{\pgfqpoint{1.849217in}{2.378800in}}{\pgfqpoint{1.845945in}{2.386700in}}{\pgfqpoint{1.840121in}{2.392524in}}%
\pgfpathcurveto{\pgfqpoint{1.834297in}{2.398348in}}{\pgfqpoint{1.826397in}{2.401620in}}{\pgfqpoint{1.818161in}{2.401620in}}%
\pgfpathcurveto{\pgfqpoint{1.809925in}{2.401620in}}{\pgfqpoint{1.802025in}{2.398348in}}{\pgfqpoint{1.796201in}{2.392524in}}%
\pgfpathcurveto{\pgfqpoint{1.790377in}{2.386700in}}{\pgfqpoint{1.787104in}{2.378800in}}{\pgfqpoint{1.787104in}{2.370564in}}%
\pgfpathcurveto{\pgfqpoint{1.787104in}{2.362328in}}{\pgfqpoint{1.790377in}{2.354427in}}{\pgfqpoint{1.796201in}{2.348604in}}%
\pgfpathcurveto{\pgfqpoint{1.802025in}{2.342780in}}{\pgfqpoint{1.809925in}{2.339507in}}{\pgfqpoint{1.818161in}{2.339507in}}%
\pgfpathclose%
\pgfusepath{stroke,fill}%
\end{pgfscope}%
\begin{pgfscope}%
\pgfpathrectangle{\pgfqpoint{0.100000in}{0.212622in}}{\pgfqpoint{3.696000in}{3.696000in}}%
\pgfusepath{clip}%
\pgfsetbuttcap%
\pgfsetroundjoin%
\definecolor{currentfill}{rgb}{0.121569,0.466667,0.705882}%
\pgfsetfillcolor{currentfill}%
\pgfsetfillopacity{0.325086}%
\pgfsetlinewidth{1.003750pt}%
\definecolor{currentstroke}{rgb}{0.121569,0.466667,0.705882}%
\pgfsetstrokecolor{currentstroke}%
\pgfsetstrokeopacity{0.325086}%
\pgfsetdash{}{0pt}%
\pgfpathmoveto{\pgfqpoint{1.582011in}{2.262096in}}%
\pgfpathcurveto{\pgfqpoint{1.590248in}{2.262096in}}{\pgfqpoint{1.598148in}{2.265368in}}{\pgfqpoint{1.603972in}{2.271192in}}%
\pgfpathcurveto{\pgfqpoint{1.609796in}{2.277016in}}{\pgfqpoint{1.613068in}{2.284916in}}{\pgfqpoint{1.613068in}{2.293152in}}%
\pgfpathcurveto{\pgfqpoint{1.613068in}{2.301389in}}{\pgfqpoint{1.609796in}{2.309289in}}{\pgfqpoint{1.603972in}{2.315113in}}%
\pgfpathcurveto{\pgfqpoint{1.598148in}{2.320937in}}{\pgfqpoint{1.590248in}{2.324209in}}{\pgfqpoint{1.582011in}{2.324209in}}%
\pgfpathcurveto{\pgfqpoint{1.573775in}{2.324209in}}{\pgfqpoint{1.565875in}{2.320937in}}{\pgfqpoint{1.560051in}{2.315113in}}%
\pgfpathcurveto{\pgfqpoint{1.554227in}{2.309289in}}{\pgfqpoint{1.550955in}{2.301389in}}{\pgfqpoint{1.550955in}{2.293152in}}%
\pgfpathcurveto{\pgfqpoint{1.550955in}{2.284916in}}{\pgfqpoint{1.554227in}{2.277016in}}{\pgfqpoint{1.560051in}{2.271192in}}%
\pgfpathcurveto{\pgfqpoint{1.565875in}{2.265368in}}{\pgfqpoint{1.573775in}{2.262096in}}{\pgfqpoint{1.582011in}{2.262096in}}%
\pgfpathclose%
\pgfusepath{stroke,fill}%
\end{pgfscope}%
\begin{pgfscope}%
\pgfpathrectangle{\pgfqpoint{0.100000in}{0.212622in}}{\pgfqpoint{3.696000in}{3.696000in}}%
\pgfusepath{clip}%
\pgfsetbuttcap%
\pgfsetroundjoin%
\definecolor{currentfill}{rgb}{0.121569,0.466667,0.705882}%
\pgfsetfillcolor{currentfill}%
\pgfsetfillopacity{0.326157}%
\pgfsetlinewidth{1.003750pt}%
\definecolor{currentstroke}{rgb}{0.121569,0.466667,0.705882}%
\pgfsetstrokecolor{currentstroke}%
\pgfsetstrokeopacity{0.326157}%
\pgfsetdash{}{0pt}%
\pgfpathmoveto{\pgfqpoint{1.842617in}{2.332459in}}%
\pgfpathcurveto{\pgfqpoint{1.850853in}{2.332459in}}{\pgfqpoint{1.858753in}{2.335731in}}{\pgfqpoint{1.864577in}{2.341555in}}%
\pgfpathcurveto{\pgfqpoint{1.870401in}{2.347379in}}{\pgfqpoint{1.873674in}{2.355279in}}{\pgfqpoint{1.873674in}{2.363516in}}%
\pgfpathcurveto{\pgfqpoint{1.873674in}{2.371752in}}{\pgfqpoint{1.870401in}{2.379652in}}{\pgfqpoint{1.864577in}{2.385476in}}%
\pgfpathcurveto{\pgfqpoint{1.858753in}{2.391300in}}{\pgfqpoint{1.850853in}{2.394572in}}{\pgfqpoint{1.842617in}{2.394572in}}%
\pgfpathcurveto{\pgfqpoint{1.834381in}{2.394572in}}{\pgfqpoint{1.826481in}{2.391300in}}{\pgfqpoint{1.820657in}{2.385476in}}%
\pgfpathcurveto{\pgfqpoint{1.814833in}{2.379652in}}{\pgfqpoint{1.811561in}{2.371752in}}{\pgfqpoint{1.811561in}{2.363516in}}%
\pgfpathcurveto{\pgfqpoint{1.811561in}{2.355279in}}{\pgfqpoint{1.814833in}{2.347379in}}{\pgfqpoint{1.820657in}{2.341555in}}%
\pgfpathcurveto{\pgfqpoint{1.826481in}{2.335731in}}{\pgfqpoint{1.834381in}{2.332459in}}{\pgfqpoint{1.842617in}{2.332459in}}%
\pgfpathclose%
\pgfusepath{stroke,fill}%
\end{pgfscope}%
\begin{pgfscope}%
\pgfpathrectangle{\pgfqpoint{0.100000in}{0.212622in}}{\pgfqpoint{3.696000in}{3.696000in}}%
\pgfusepath{clip}%
\pgfsetbuttcap%
\pgfsetroundjoin%
\definecolor{currentfill}{rgb}{0.121569,0.466667,0.705882}%
\pgfsetfillcolor{currentfill}%
\pgfsetfillopacity{0.328086}%
\pgfsetlinewidth{1.003750pt}%
\definecolor{currentstroke}{rgb}{0.121569,0.466667,0.705882}%
\pgfsetstrokecolor{currentstroke}%
\pgfsetstrokeopacity{0.328086}%
\pgfsetdash{}{0pt}%
\pgfpathmoveto{\pgfqpoint{1.574186in}{2.252022in}}%
\pgfpathcurveto{\pgfqpoint{1.582423in}{2.252022in}}{\pgfqpoint{1.590323in}{2.255294in}}{\pgfqpoint{1.596147in}{2.261118in}}%
\pgfpathcurveto{\pgfqpoint{1.601970in}{2.266942in}}{\pgfqpoint{1.605243in}{2.274842in}}{\pgfqpoint{1.605243in}{2.283079in}}%
\pgfpathcurveto{\pgfqpoint{1.605243in}{2.291315in}}{\pgfqpoint{1.601970in}{2.299215in}}{\pgfqpoint{1.596147in}{2.305039in}}%
\pgfpathcurveto{\pgfqpoint{1.590323in}{2.310863in}}{\pgfqpoint{1.582423in}{2.314135in}}{\pgfqpoint{1.574186in}{2.314135in}}%
\pgfpathcurveto{\pgfqpoint{1.565950in}{2.314135in}}{\pgfqpoint{1.558050in}{2.310863in}}{\pgfqpoint{1.552226in}{2.305039in}}%
\pgfpathcurveto{\pgfqpoint{1.546402in}{2.299215in}}{\pgfqpoint{1.543130in}{2.291315in}}{\pgfqpoint{1.543130in}{2.283079in}}%
\pgfpathcurveto{\pgfqpoint{1.543130in}{2.274842in}}{\pgfqpoint{1.546402in}{2.266942in}}{\pgfqpoint{1.552226in}{2.261118in}}%
\pgfpathcurveto{\pgfqpoint{1.558050in}{2.255294in}}{\pgfqpoint{1.565950in}{2.252022in}}{\pgfqpoint{1.574186in}{2.252022in}}%
\pgfpathclose%
\pgfusepath{stroke,fill}%
\end{pgfscope}%
\begin{pgfscope}%
\pgfpathrectangle{\pgfqpoint{0.100000in}{0.212622in}}{\pgfqpoint{3.696000in}{3.696000in}}%
\pgfusepath{clip}%
\pgfsetbuttcap%
\pgfsetroundjoin%
\definecolor{currentfill}{rgb}{0.121569,0.466667,0.705882}%
\pgfsetfillcolor{currentfill}%
\pgfsetfillopacity{0.329762}%
\pgfsetlinewidth{1.003750pt}%
\definecolor{currentstroke}{rgb}{0.121569,0.466667,0.705882}%
\pgfsetstrokecolor{currentstroke}%
\pgfsetstrokeopacity{0.329762}%
\pgfsetdash{}{0pt}%
\pgfpathmoveto{\pgfqpoint{1.569069in}{2.247830in}}%
\pgfpathcurveto{\pgfqpoint{1.577305in}{2.247830in}}{\pgfqpoint{1.585205in}{2.251102in}}{\pgfqpoint{1.591029in}{2.256926in}}%
\pgfpathcurveto{\pgfqpoint{1.596853in}{2.262750in}}{\pgfqpoint{1.600125in}{2.270650in}}{\pgfqpoint{1.600125in}{2.278887in}}%
\pgfpathcurveto{\pgfqpoint{1.600125in}{2.287123in}}{\pgfqpoint{1.596853in}{2.295023in}}{\pgfqpoint{1.591029in}{2.300847in}}%
\pgfpathcurveto{\pgfqpoint{1.585205in}{2.306671in}}{\pgfqpoint{1.577305in}{2.309943in}}{\pgfqpoint{1.569069in}{2.309943in}}%
\pgfpathcurveto{\pgfqpoint{1.560832in}{2.309943in}}{\pgfqpoint{1.552932in}{2.306671in}}{\pgfqpoint{1.547108in}{2.300847in}}%
\pgfpathcurveto{\pgfqpoint{1.541285in}{2.295023in}}{\pgfqpoint{1.538012in}{2.287123in}}{\pgfqpoint{1.538012in}{2.278887in}}%
\pgfpathcurveto{\pgfqpoint{1.538012in}{2.270650in}}{\pgfqpoint{1.541285in}{2.262750in}}{\pgfqpoint{1.547108in}{2.256926in}}%
\pgfpathcurveto{\pgfqpoint{1.552932in}{2.251102in}}{\pgfqpoint{1.560832in}{2.247830in}}{\pgfqpoint{1.569069in}{2.247830in}}%
\pgfpathclose%
\pgfusepath{stroke,fill}%
\end{pgfscope}%
\begin{pgfscope}%
\pgfpathrectangle{\pgfqpoint{0.100000in}{0.212622in}}{\pgfqpoint{3.696000in}{3.696000in}}%
\pgfusepath{clip}%
\pgfsetbuttcap%
\pgfsetroundjoin%
\definecolor{currentfill}{rgb}{0.121569,0.466667,0.705882}%
\pgfsetfillcolor{currentfill}%
\pgfsetfillopacity{0.330025}%
\pgfsetlinewidth{1.003750pt}%
\definecolor{currentstroke}{rgb}{0.121569,0.466667,0.705882}%
\pgfsetstrokecolor{currentstroke}%
\pgfsetstrokeopacity{0.330025}%
\pgfsetdash{}{0pt}%
\pgfpathmoveto{\pgfqpoint{1.568468in}{2.246750in}}%
\pgfpathcurveto{\pgfqpoint{1.576704in}{2.246750in}}{\pgfqpoint{1.584604in}{2.250022in}}{\pgfqpoint{1.590428in}{2.255846in}}%
\pgfpathcurveto{\pgfqpoint{1.596252in}{2.261670in}}{\pgfqpoint{1.599525in}{2.269570in}}{\pgfqpoint{1.599525in}{2.277807in}}%
\pgfpathcurveto{\pgfqpoint{1.599525in}{2.286043in}}{\pgfqpoint{1.596252in}{2.293943in}}{\pgfqpoint{1.590428in}{2.299767in}}%
\pgfpathcurveto{\pgfqpoint{1.584604in}{2.305591in}}{\pgfqpoint{1.576704in}{2.308863in}}{\pgfqpoint{1.568468in}{2.308863in}}%
\pgfpathcurveto{\pgfqpoint{1.560232in}{2.308863in}}{\pgfqpoint{1.552332in}{2.305591in}}{\pgfqpoint{1.546508in}{2.299767in}}%
\pgfpathcurveto{\pgfqpoint{1.540684in}{2.293943in}}{\pgfqpoint{1.537412in}{2.286043in}}{\pgfqpoint{1.537412in}{2.277807in}}%
\pgfpathcurveto{\pgfqpoint{1.537412in}{2.269570in}}{\pgfqpoint{1.540684in}{2.261670in}}{\pgfqpoint{1.546508in}{2.255846in}}%
\pgfpathcurveto{\pgfqpoint{1.552332in}{2.250022in}}{\pgfqpoint{1.560232in}{2.246750in}}{\pgfqpoint{1.568468in}{2.246750in}}%
\pgfpathclose%
\pgfusepath{stroke,fill}%
\end{pgfscope}%
\begin{pgfscope}%
\pgfpathrectangle{\pgfqpoint{0.100000in}{0.212622in}}{\pgfqpoint{3.696000in}{3.696000in}}%
\pgfusepath{clip}%
\pgfsetbuttcap%
\pgfsetroundjoin%
\definecolor{currentfill}{rgb}{0.121569,0.466667,0.705882}%
\pgfsetfillcolor{currentfill}%
\pgfsetfillopacity{0.330540}%
\pgfsetlinewidth{1.003750pt}%
\definecolor{currentstroke}{rgb}{0.121569,0.466667,0.705882}%
\pgfsetstrokecolor{currentstroke}%
\pgfsetstrokeopacity{0.330540}%
\pgfsetdash{}{0pt}%
\pgfpathmoveto{\pgfqpoint{1.566917in}{2.245504in}}%
\pgfpathcurveto{\pgfqpoint{1.575154in}{2.245504in}}{\pgfqpoint{1.583054in}{2.248776in}}{\pgfqpoint{1.588878in}{2.254600in}}%
\pgfpathcurveto{\pgfqpoint{1.594702in}{2.260424in}}{\pgfqpoint{1.597974in}{2.268324in}}{\pgfqpoint{1.597974in}{2.276560in}}%
\pgfpathcurveto{\pgfqpoint{1.597974in}{2.284796in}}{\pgfqpoint{1.594702in}{2.292696in}}{\pgfqpoint{1.588878in}{2.298520in}}%
\pgfpathcurveto{\pgfqpoint{1.583054in}{2.304344in}}{\pgfqpoint{1.575154in}{2.307617in}}{\pgfqpoint{1.566917in}{2.307617in}}%
\pgfpathcurveto{\pgfqpoint{1.558681in}{2.307617in}}{\pgfqpoint{1.550781in}{2.304344in}}{\pgfqpoint{1.544957in}{2.298520in}}%
\pgfpathcurveto{\pgfqpoint{1.539133in}{2.292696in}}{\pgfqpoint{1.535861in}{2.284796in}}{\pgfqpoint{1.535861in}{2.276560in}}%
\pgfpathcurveto{\pgfqpoint{1.535861in}{2.268324in}}{\pgfqpoint{1.539133in}{2.260424in}}{\pgfqpoint{1.544957in}{2.254600in}}%
\pgfpathcurveto{\pgfqpoint{1.550781in}{2.248776in}}{\pgfqpoint{1.558681in}{2.245504in}}{\pgfqpoint{1.566917in}{2.245504in}}%
\pgfpathclose%
\pgfusepath{stroke,fill}%
\end{pgfscope}%
\begin{pgfscope}%
\pgfpathrectangle{\pgfqpoint{0.100000in}{0.212622in}}{\pgfqpoint{3.696000in}{3.696000in}}%
\pgfusepath{clip}%
\pgfsetbuttcap%
\pgfsetroundjoin%
\definecolor{currentfill}{rgb}{0.121569,0.466667,0.705882}%
\pgfsetfillcolor{currentfill}%
\pgfsetfillopacity{0.331499}%
\pgfsetlinewidth{1.003750pt}%
\definecolor{currentstroke}{rgb}{0.121569,0.466667,0.705882}%
\pgfsetstrokecolor{currentstroke}%
\pgfsetstrokeopacity{0.331499}%
\pgfsetdash{}{0pt}%
\pgfpathmoveto{\pgfqpoint{1.565025in}{2.242518in}}%
\pgfpathcurveto{\pgfqpoint{1.573261in}{2.242518in}}{\pgfqpoint{1.581161in}{2.245791in}}{\pgfqpoint{1.586985in}{2.251615in}}%
\pgfpathcurveto{\pgfqpoint{1.592809in}{2.257439in}}{\pgfqpoint{1.596082in}{2.265339in}}{\pgfqpoint{1.596082in}{2.273575in}}%
\pgfpathcurveto{\pgfqpoint{1.596082in}{2.281811in}}{\pgfqpoint{1.592809in}{2.289711in}}{\pgfqpoint{1.586985in}{2.295535in}}%
\pgfpathcurveto{\pgfqpoint{1.581161in}{2.301359in}}{\pgfqpoint{1.573261in}{2.304631in}}{\pgfqpoint{1.565025in}{2.304631in}}%
\pgfpathcurveto{\pgfqpoint{1.556789in}{2.304631in}}{\pgfqpoint{1.548889in}{2.301359in}}{\pgfqpoint{1.543065in}{2.295535in}}%
\pgfpathcurveto{\pgfqpoint{1.537241in}{2.289711in}}{\pgfqpoint{1.533969in}{2.281811in}}{\pgfqpoint{1.533969in}{2.273575in}}%
\pgfpathcurveto{\pgfqpoint{1.533969in}{2.265339in}}{\pgfqpoint{1.537241in}{2.257439in}}{\pgfqpoint{1.543065in}{2.251615in}}%
\pgfpathcurveto{\pgfqpoint{1.548889in}{2.245791in}}{\pgfqpoint{1.556789in}{2.242518in}}{\pgfqpoint{1.565025in}{2.242518in}}%
\pgfpathclose%
\pgfusepath{stroke,fill}%
\end{pgfscope}%
\begin{pgfscope}%
\pgfpathrectangle{\pgfqpoint{0.100000in}{0.212622in}}{\pgfqpoint{3.696000in}{3.696000in}}%
\pgfusepath{clip}%
\pgfsetbuttcap%
\pgfsetroundjoin%
\definecolor{currentfill}{rgb}{0.121569,0.466667,0.705882}%
\pgfsetfillcolor{currentfill}%
\pgfsetfillopacity{0.331745}%
\pgfsetlinewidth{1.003750pt}%
\definecolor{currentstroke}{rgb}{0.121569,0.466667,0.705882}%
\pgfsetstrokecolor{currentstroke}%
\pgfsetstrokeopacity{0.331745}%
\pgfsetdash{}{0pt}%
\pgfpathmoveto{\pgfqpoint{1.873323in}{2.331010in}}%
\pgfpathcurveto{\pgfqpoint{1.881560in}{2.331010in}}{\pgfqpoint{1.889460in}{2.334282in}}{\pgfqpoint{1.895284in}{2.340106in}}%
\pgfpathcurveto{\pgfqpoint{1.901108in}{2.345930in}}{\pgfqpoint{1.904380in}{2.353830in}}{\pgfqpoint{1.904380in}{2.362067in}}%
\pgfpathcurveto{\pgfqpoint{1.904380in}{2.370303in}}{\pgfqpoint{1.901108in}{2.378203in}}{\pgfqpoint{1.895284in}{2.384027in}}%
\pgfpathcurveto{\pgfqpoint{1.889460in}{2.389851in}}{\pgfqpoint{1.881560in}{2.393123in}}{\pgfqpoint{1.873323in}{2.393123in}}%
\pgfpathcurveto{\pgfqpoint{1.865087in}{2.393123in}}{\pgfqpoint{1.857187in}{2.389851in}}{\pgfqpoint{1.851363in}{2.384027in}}%
\pgfpathcurveto{\pgfqpoint{1.845539in}{2.378203in}}{\pgfqpoint{1.842267in}{2.370303in}}{\pgfqpoint{1.842267in}{2.362067in}}%
\pgfpathcurveto{\pgfqpoint{1.842267in}{2.353830in}}{\pgfqpoint{1.845539in}{2.345930in}}{\pgfqpoint{1.851363in}{2.340106in}}%
\pgfpathcurveto{\pgfqpoint{1.857187in}{2.334282in}}{\pgfqpoint{1.865087in}{2.331010in}}{\pgfqpoint{1.873323in}{2.331010in}}%
\pgfpathclose%
\pgfusepath{stroke,fill}%
\end{pgfscope}%
\begin{pgfscope}%
\pgfpathrectangle{\pgfqpoint{0.100000in}{0.212622in}}{\pgfqpoint{3.696000in}{3.696000in}}%
\pgfusepath{clip}%
\pgfsetbuttcap%
\pgfsetroundjoin%
\definecolor{currentfill}{rgb}{0.121569,0.466667,0.705882}%
\pgfsetfillcolor{currentfill}%
\pgfsetfillopacity{0.333162}%
\pgfsetlinewidth{1.003750pt}%
\definecolor{currentstroke}{rgb}{0.121569,0.466667,0.705882}%
\pgfsetstrokecolor{currentstroke}%
\pgfsetstrokeopacity{0.333162}%
\pgfsetdash{}{0pt}%
\pgfpathmoveto{\pgfqpoint{1.561122in}{2.236856in}}%
\pgfpathcurveto{\pgfqpoint{1.569358in}{2.236856in}}{\pgfqpoint{1.577258in}{2.240129in}}{\pgfqpoint{1.583082in}{2.245952in}}%
\pgfpathcurveto{\pgfqpoint{1.588906in}{2.251776in}}{\pgfqpoint{1.592178in}{2.259676in}}{\pgfqpoint{1.592178in}{2.267913in}}%
\pgfpathcurveto{\pgfqpoint{1.592178in}{2.276149in}}{\pgfqpoint{1.588906in}{2.284049in}}{\pgfqpoint{1.583082in}{2.289873in}}%
\pgfpathcurveto{\pgfqpoint{1.577258in}{2.295697in}}{\pgfqpoint{1.569358in}{2.298969in}}{\pgfqpoint{1.561122in}{2.298969in}}%
\pgfpathcurveto{\pgfqpoint{1.552885in}{2.298969in}}{\pgfqpoint{1.544985in}{2.295697in}}{\pgfqpoint{1.539161in}{2.289873in}}%
\pgfpathcurveto{\pgfqpoint{1.533337in}{2.284049in}}{\pgfqpoint{1.530065in}{2.276149in}}{\pgfqpoint{1.530065in}{2.267913in}}%
\pgfpathcurveto{\pgfqpoint{1.530065in}{2.259676in}}{\pgfqpoint{1.533337in}{2.251776in}}{\pgfqpoint{1.539161in}{2.245952in}}%
\pgfpathcurveto{\pgfqpoint{1.544985in}{2.240129in}}{\pgfqpoint{1.552885in}{2.236856in}}{\pgfqpoint{1.561122in}{2.236856in}}%
\pgfpathclose%
\pgfusepath{stroke,fill}%
\end{pgfscope}%
\begin{pgfscope}%
\pgfpathrectangle{\pgfqpoint{0.100000in}{0.212622in}}{\pgfqpoint{3.696000in}{3.696000in}}%
\pgfusepath{clip}%
\pgfsetbuttcap%
\pgfsetroundjoin%
\definecolor{currentfill}{rgb}{0.121569,0.466667,0.705882}%
\pgfsetfillcolor{currentfill}%
\pgfsetfillopacity{0.336125}%
\pgfsetlinewidth{1.003750pt}%
\definecolor{currentstroke}{rgb}{0.121569,0.466667,0.705882}%
\pgfsetstrokecolor{currentstroke}%
\pgfsetstrokeopacity{0.336125}%
\pgfsetdash{}{0pt}%
\pgfpathmoveto{\pgfqpoint{1.554180in}{2.225985in}}%
\pgfpathcurveto{\pgfqpoint{1.562417in}{2.225985in}}{\pgfqpoint{1.570317in}{2.229257in}}{\pgfqpoint{1.576141in}{2.235081in}}%
\pgfpathcurveto{\pgfqpoint{1.581964in}{2.240905in}}{\pgfqpoint{1.585237in}{2.248805in}}{\pgfqpoint{1.585237in}{2.257041in}}%
\pgfpathcurveto{\pgfqpoint{1.585237in}{2.265278in}}{\pgfqpoint{1.581964in}{2.273178in}}{\pgfqpoint{1.576141in}{2.279002in}}%
\pgfpathcurveto{\pgfqpoint{1.570317in}{2.284826in}}{\pgfqpoint{1.562417in}{2.288098in}}{\pgfqpoint{1.554180in}{2.288098in}}%
\pgfpathcurveto{\pgfqpoint{1.545944in}{2.288098in}}{\pgfqpoint{1.538044in}{2.284826in}}{\pgfqpoint{1.532220in}{2.279002in}}%
\pgfpathcurveto{\pgfqpoint{1.526396in}{2.273178in}}{\pgfqpoint{1.523124in}{2.265278in}}{\pgfqpoint{1.523124in}{2.257041in}}%
\pgfpathcurveto{\pgfqpoint{1.523124in}{2.248805in}}{\pgfqpoint{1.526396in}{2.240905in}}{\pgfqpoint{1.532220in}{2.235081in}}%
\pgfpathcurveto{\pgfqpoint{1.538044in}{2.229257in}}{\pgfqpoint{1.545944in}{2.225985in}}{\pgfqpoint{1.554180in}{2.225985in}}%
\pgfpathclose%
\pgfusepath{stroke,fill}%
\end{pgfscope}%
\begin{pgfscope}%
\pgfpathrectangle{\pgfqpoint{0.100000in}{0.212622in}}{\pgfqpoint{3.696000in}{3.696000in}}%
\pgfusepath{clip}%
\pgfsetbuttcap%
\pgfsetroundjoin%
\definecolor{currentfill}{rgb}{0.121569,0.466667,0.705882}%
\pgfsetfillcolor{currentfill}%
\pgfsetfillopacity{0.338338}%
\pgfsetlinewidth{1.003750pt}%
\definecolor{currentstroke}{rgb}{0.121569,0.466667,0.705882}%
\pgfsetstrokecolor{currentstroke}%
\pgfsetstrokeopacity{0.338338}%
\pgfsetdash{}{0pt}%
\pgfpathmoveto{\pgfqpoint{1.905601in}{2.332366in}}%
\pgfpathcurveto{\pgfqpoint{1.913837in}{2.332366in}}{\pgfqpoint{1.921737in}{2.335638in}}{\pgfqpoint{1.927561in}{2.341462in}}%
\pgfpathcurveto{\pgfqpoint{1.933385in}{2.347286in}}{\pgfqpoint{1.936657in}{2.355186in}}{\pgfqpoint{1.936657in}{2.363422in}}%
\pgfpathcurveto{\pgfqpoint{1.936657in}{2.371659in}}{\pgfqpoint{1.933385in}{2.379559in}}{\pgfqpoint{1.927561in}{2.385383in}}%
\pgfpathcurveto{\pgfqpoint{1.921737in}{2.391206in}}{\pgfqpoint{1.913837in}{2.394479in}}{\pgfqpoint{1.905601in}{2.394479in}}%
\pgfpathcurveto{\pgfqpoint{1.897364in}{2.394479in}}{\pgfqpoint{1.889464in}{2.391206in}}{\pgfqpoint{1.883640in}{2.385383in}}%
\pgfpathcurveto{\pgfqpoint{1.877816in}{2.379559in}}{\pgfqpoint{1.874544in}{2.371659in}}{\pgfqpoint{1.874544in}{2.363422in}}%
\pgfpathcurveto{\pgfqpoint{1.874544in}{2.355186in}}{\pgfqpoint{1.877816in}{2.347286in}}{\pgfqpoint{1.883640in}{2.341462in}}%
\pgfpathcurveto{\pgfqpoint{1.889464in}{2.335638in}}{\pgfqpoint{1.897364in}{2.332366in}}{\pgfqpoint{1.905601in}{2.332366in}}%
\pgfpathclose%
\pgfusepath{stroke,fill}%
\end{pgfscope}%
\begin{pgfscope}%
\pgfpathrectangle{\pgfqpoint{0.100000in}{0.212622in}}{\pgfqpoint{3.696000in}{3.696000in}}%
\pgfusepath{clip}%
\pgfsetbuttcap%
\pgfsetroundjoin%
\definecolor{currentfill}{rgb}{0.121569,0.466667,0.705882}%
\pgfsetfillcolor{currentfill}%
\pgfsetfillopacity{0.341694}%
\pgfsetlinewidth{1.003750pt}%
\definecolor{currentstroke}{rgb}{0.121569,0.466667,0.705882}%
\pgfsetstrokecolor{currentstroke}%
\pgfsetstrokeopacity{0.341694}%
\pgfsetdash{}{0pt}%
\pgfpathmoveto{\pgfqpoint{1.540360in}{2.208355in}}%
\pgfpathcurveto{\pgfqpoint{1.548596in}{2.208355in}}{\pgfqpoint{1.556496in}{2.211628in}}{\pgfqpoint{1.562320in}{2.217452in}}%
\pgfpathcurveto{\pgfqpoint{1.568144in}{2.223275in}}{\pgfqpoint{1.571416in}{2.231175in}}{\pgfqpoint{1.571416in}{2.239412in}}%
\pgfpathcurveto{\pgfqpoint{1.571416in}{2.247648in}}{\pgfqpoint{1.568144in}{2.255548in}}{\pgfqpoint{1.562320in}{2.261372in}}%
\pgfpathcurveto{\pgfqpoint{1.556496in}{2.267196in}}{\pgfqpoint{1.548596in}{2.270468in}}{\pgfqpoint{1.540360in}{2.270468in}}%
\pgfpathcurveto{\pgfqpoint{1.532123in}{2.270468in}}{\pgfqpoint{1.524223in}{2.267196in}}{\pgfqpoint{1.518399in}{2.261372in}}%
\pgfpathcurveto{\pgfqpoint{1.512576in}{2.255548in}}{\pgfqpoint{1.509303in}{2.247648in}}{\pgfqpoint{1.509303in}{2.239412in}}%
\pgfpathcurveto{\pgfqpoint{1.509303in}{2.231175in}}{\pgfqpoint{1.512576in}{2.223275in}}{\pgfqpoint{1.518399in}{2.217452in}}%
\pgfpathcurveto{\pgfqpoint{1.524223in}{2.211628in}}{\pgfqpoint{1.532123in}{2.208355in}}{\pgfqpoint{1.540360in}{2.208355in}}%
\pgfpathclose%
\pgfusepath{stroke,fill}%
\end{pgfscope}%
\begin{pgfscope}%
\pgfpathrectangle{\pgfqpoint{0.100000in}{0.212622in}}{\pgfqpoint{3.696000in}{3.696000in}}%
\pgfusepath{clip}%
\pgfsetbuttcap%
\pgfsetroundjoin%
\definecolor{currentfill}{rgb}{0.121569,0.466667,0.705882}%
\pgfsetfillcolor{currentfill}%
\pgfsetfillopacity{0.346834}%
\pgfsetlinewidth{1.003750pt}%
\definecolor{currentstroke}{rgb}{0.121569,0.466667,0.705882}%
\pgfsetstrokecolor{currentstroke}%
\pgfsetstrokeopacity{0.346834}%
\pgfsetdash{}{0pt}%
\pgfpathmoveto{\pgfqpoint{1.945921in}{2.333634in}}%
\pgfpathcurveto{\pgfqpoint{1.954158in}{2.333634in}}{\pgfqpoint{1.962058in}{2.336906in}}{\pgfqpoint{1.967882in}{2.342730in}}%
\pgfpathcurveto{\pgfqpoint{1.973705in}{2.348554in}}{\pgfqpoint{1.976978in}{2.356454in}}{\pgfqpoint{1.976978in}{2.364690in}}%
\pgfpathcurveto{\pgfqpoint{1.976978in}{2.372926in}}{\pgfqpoint{1.973705in}{2.380826in}}{\pgfqpoint{1.967882in}{2.386650in}}%
\pgfpathcurveto{\pgfqpoint{1.962058in}{2.392474in}}{\pgfqpoint{1.954158in}{2.395747in}}{\pgfqpoint{1.945921in}{2.395747in}}%
\pgfpathcurveto{\pgfqpoint{1.937685in}{2.395747in}}{\pgfqpoint{1.929785in}{2.392474in}}{\pgfqpoint{1.923961in}{2.386650in}}%
\pgfpathcurveto{\pgfqpoint{1.918137in}{2.380826in}}{\pgfqpoint{1.914865in}{2.372926in}}{\pgfqpoint{1.914865in}{2.364690in}}%
\pgfpathcurveto{\pgfqpoint{1.914865in}{2.356454in}}{\pgfqpoint{1.918137in}{2.348554in}}{\pgfqpoint{1.923961in}{2.342730in}}%
\pgfpathcurveto{\pgfqpoint{1.929785in}{2.336906in}}{\pgfqpoint{1.937685in}{2.333634in}}{\pgfqpoint{1.945921in}{2.333634in}}%
\pgfpathclose%
\pgfusepath{stroke,fill}%
\end{pgfscope}%
\begin{pgfscope}%
\pgfpathrectangle{\pgfqpoint{0.100000in}{0.212622in}}{\pgfqpoint{3.696000in}{3.696000in}}%
\pgfusepath{clip}%
\pgfsetbuttcap%
\pgfsetroundjoin%
\definecolor{currentfill}{rgb}{0.121569,0.466667,0.705882}%
\pgfsetfillcolor{currentfill}%
\pgfsetfillopacity{0.350927}%
\pgfsetlinewidth{1.003750pt}%
\definecolor{currentstroke}{rgb}{0.121569,0.466667,0.705882}%
\pgfsetstrokecolor{currentstroke}%
\pgfsetstrokeopacity{0.350927}%
\pgfsetdash{}{0pt}%
\pgfpathmoveto{\pgfqpoint{1.968858in}{2.332258in}}%
\pgfpathcurveto{\pgfqpoint{1.977094in}{2.332258in}}{\pgfqpoint{1.984994in}{2.335530in}}{\pgfqpoint{1.990818in}{2.341354in}}%
\pgfpathcurveto{\pgfqpoint{1.996642in}{2.347178in}}{\pgfqpoint{1.999914in}{2.355078in}}{\pgfqpoint{1.999914in}{2.363314in}}%
\pgfpathcurveto{\pgfqpoint{1.999914in}{2.371550in}}{\pgfqpoint{1.996642in}{2.379450in}}{\pgfqpoint{1.990818in}{2.385274in}}%
\pgfpathcurveto{\pgfqpoint{1.984994in}{2.391098in}}{\pgfqpoint{1.977094in}{2.394371in}}{\pgfqpoint{1.968858in}{2.394371in}}%
\pgfpathcurveto{\pgfqpoint{1.960621in}{2.394371in}}{\pgfqpoint{1.952721in}{2.391098in}}{\pgfqpoint{1.946897in}{2.385274in}}%
\pgfpathcurveto{\pgfqpoint{1.941073in}{2.379450in}}{\pgfqpoint{1.937801in}{2.371550in}}{\pgfqpoint{1.937801in}{2.363314in}}%
\pgfpathcurveto{\pgfqpoint{1.937801in}{2.355078in}}{\pgfqpoint{1.941073in}{2.347178in}}{\pgfqpoint{1.946897in}{2.341354in}}%
\pgfpathcurveto{\pgfqpoint{1.952721in}{2.335530in}}{\pgfqpoint{1.960621in}{2.332258in}}{\pgfqpoint{1.968858in}{2.332258in}}%
\pgfpathclose%
\pgfusepath{stroke,fill}%
\end{pgfscope}%
\begin{pgfscope}%
\pgfpathrectangle{\pgfqpoint{0.100000in}{0.212622in}}{\pgfqpoint{3.696000in}{3.696000in}}%
\pgfusepath{clip}%
\pgfsetbuttcap%
\pgfsetroundjoin%
\definecolor{currentfill}{rgb}{0.121569,0.466667,0.705882}%
\pgfsetfillcolor{currentfill}%
\pgfsetfillopacity{0.351180}%
\pgfsetlinewidth{1.003750pt}%
\definecolor{currentstroke}{rgb}{0.121569,0.466667,0.705882}%
\pgfsetstrokecolor{currentstroke}%
\pgfsetstrokeopacity{0.351180}%
\pgfsetdash{}{0pt}%
\pgfpathmoveto{\pgfqpoint{1.517958in}{2.169123in}}%
\pgfpathcurveto{\pgfqpoint{1.526194in}{2.169123in}}{\pgfqpoint{1.534094in}{2.172396in}}{\pgfqpoint{1.539918in}{2.178220in}}%
\pgfpathcurveto{\pgfqpoint{1.545742in}{2.184044in}}{\pgfqpoint{1.549014in}{2.191944in}}{\pgfqpoint{1.549014in}{2.200180in}}%
\pgfpathcurveto{\pgfqpoint{1.549014in}{2.208416in}}{\pgfqpoint{1.545742in}{2.216316in}}{\pgfqpoint{1.539918in}{2.222140in}}%
\pgfpathcurveto{\pgfqpoint{1.534094in}{2.227964in}}{\pgfqpoint{1.526194in}{2.231236in}}{\pgfqpoint{1.517958in}{2.231236in}}%
\pgfpathcurveto{\pgfqpoint{1.509721in}{2.231236in}}{\pgfqpoint{1.501821in}{2.227964in}}{\pgfqpoint{1.495997in}{2.222140in}}%
\pgfpathcurveto{\pgfqpoint{1.490173in}{2.216316in}}{\pgfqpoint{1.486901in}{2.208416in}}{\pgfqpoint{1.486901in}{2.200180in}}%
\pgfpathcurveto{\pgfqpoint{1.486901in}{2.191944in}}{\pgfqpoint{1.490173in}{2.184044in}}{\pgfqpoint{1.495997in}{2.178220in}}%
\pgfpathcurveto{\pgfqpoint{1.501821in}{2.172396in}}{\pgfqpoint{1.509721in}{2.169123in}}{\pgfqpoint{1.517958in}{2.169123in}}%
\pgfpathclose%
\pgfusepath{stroke,fill}%
\end{pgfscope}%
\begin{pgfscope}%
\pgfpathrectangle{\pgfqpoint{0.100000in}{0.212622in}}{\pgfqpoint{3.696000in}{3.696000in}}%
\pgfusepath{clip}%
\pgfsetbuttcap%
\pgfsetroundjoin%
\definecolor{currentfill}{rgb}{0.121569,0.466667,0.705882}%
\pgfsetfillcolor{currentfill}%
\pgfsetfillopacity{0.356416}%
\pgfsetlinewidth{1.003750pt}%
\definecolor{currentstroke}{rgb}{0.121569,0.466667,0.705882}%
\pgfsetstrokecolor{currentstroke}%
\pgfsetstrokeopacity{0.356416}%
\pgfsetdash{}{0pt}%
\pgfpathmoveto{\pgfqpoint{1.999923in}{2.330517in}}%
\pgfpathcurveto{\pgfqpoint{2.008160in}{2.330517in}}{\pgfqpoint{2.016060in}{2.333790in}}{\pgfqpoint{2.021884in}{2.339614in}}%
\pgfpathcurveto{\pgfqpoint{2.027708in}{2.345438in}}{\pgfqpoint{2.030980in}{2.353338in}}{\pgfqpoint{2.030980in}{2.361574in}}%
\pgfpathcurveto{\pgfqpoint{2.030980in}{2.369810in}}{\pgfqpoint{2.027708in}{2.377710in}}{\pgfqpoint{2.021884in}{2.383534in}}%
\pgfpathcurveto{\pgfqpoint{2.016060in}{2.389358in}}{\pgfqpoint{2.008160in}{2.392630in}}{\pgfqpoint{1.999923in}{2.392630in}}%
\pgfpathcurveto{\pgfqpoint{1.991687in}{2.392630in}}{\pgfqpoint{1.983787in}{2.389358in}}{\pgfqpoint{1.977963in}{2.383534in}}%
\pgfpathcurveto{\pgfqpoint{1.972139in}{2.377710in}}{\pgfqpoint{1.968867in}{2.369810in}}{\pgfqpoint{1.968867in}{2.361574in}}%
\pgfpathcurveto{\pgfqpoint{1.968867in}{2.353338in}}{\pgfqpoint{1.972139in}{2.345438in}}{\pgfqpoint{1.977963in}{2.339614in}}%
\pgfpathcurveto{\pgfqpoint{1.983787in}{2.333790in}}{\pgfqpoint{1.991687in}{2.330517in}}{\pgfqpoint{1.999923in}{2.330517in}}%
\pgfpathclose%
\pgfusepath{stroke,fill}%
\end{pgfscope}%
\begin{pgfscope}%
\pgfpathrectangle{\pgfqpoint{0.100000in}{0.212622in}}{\pgfqpoint{3.696000in}{3.696000in}}%
\pgfusepath{clip}%
\pgfsetbuttcap%
\pgfsetroundjoin%
\definecolor{currentfill}{rgb}{0.121569,0.466667,0.705882}%
\pgfsetfillcolor{currentfill}%
\pgfsetfillopacity{0.359819}%
\pgfsetlinewidth{1.003750pt}%
\definecolor{currentstroke}{rgb}{0.121569,0.466667,0.705882}%
\pgfsetstrokecolor{currentstroke}%
\pgfsetstrokeopacity{0.359819}%
\pgfsetdash{}{0pt}%
\pgfpathmoveto{\pgfqpoint{2.015633in}{2.328258in}}%
\pgfpathcurveto{\pgfqpoint{2.023869in}{2.328258in}}{\pgfqpoint{2.031769in}{2.331530in}}{\pgfqpoint{2.037593in}{2.337354in}}%
\pgfpathcurveto{\pgfqpoint{2.043417in}{2.343178in}}{\pgfqpoint{2.046689in}{2.351078in}}{\pgfqpoint{2.046689in}{2.359314in}}%
\pgfpathcurveto{\pgfqpoint{2.046689in}{2.367551in}}{\pgfqpoint{2.043417in}{2.375451in}}{\pgfqpoint{2.037593in}{2.381275in}}%
\pgfpathcurveto{\pgfqpoint{2.031769in}{2.387099in}}{\pgfqpoint{2.023869in}{2.390371in}}{\pgfqpoint{2.015633in}{2.390371in}}%
\pgfpathcurveto{\pgfqpoint{2.007396in}{2.390371in}}{\pgfqpoint{1.999496in}{2.387099in}}{\pgfqpoint{1.993672in}{2.381275in}}%
\pgfpathcurveto{\pgfqpoint{1.987848in}{2.375451in}}{\pgfqpoint{1.984576in}{2.367551in}}{\pgfqpoint{1.984576in}{2.359314in}}%
\pgfpathcurveto{\pgfqpoint{1.984576in}{2.351078in}}{\pgfqpoint{1.987848in}{2.343178in}}{\pgfqpoint{1.993672in}{2.337354in}}%
\pgfpathcurveto{\pgfqpoint{1.999496in}{2.331530in}}{\pgfqpoint{2.007396in}{2.328258in}}{\pgfqpoint{2.015633in}{2.328258in}}%
\pgfpathclose%
\pgfusepath{stroke,fill}%
\end{pgfscope}%
\begin{pgfscope}%
\pgfpathrectangle{\pgfqpoint{0.100000in}{0.212622in}}{\pgfqpoint{3.696000in}{3.696000in}}%
\pgfusepath{clip}%
\pgfsetbuttcap%
\pgfsetroundjoin%
\definecolor{currentfill}{rgb}{0.121569,0.466667,0.705882}%
\pgfsetfillcolor{currentfill}%
\pgfsetfillopacity{0.360707}%
\pgfsetlinewidth{1.003750pt}%
\definecolor{currentstroke}{rgb}{0.121569,0.466667,0.705882}%
\pgfsetstrokecolor{currentstroke}%
\pgfsetstrokeopacity{0.360707}%
\pgfsetdash{}{0pt}%
\pgfpathmoveto{\pgfqpoint{1.497048in}{2.135616in}}%
\pgfpathcurveto{\pgfqpoint{1.505284in}{2.135616in}}{\pgfqpoint{1.513184in}{2.138888in}}{\pgfqpoint{1.519008in}{2.144712in}}%
\pgfpathcurveto{\pgfqpoint{1.524832in}{2.150536in}}{\pgfqpoint{1.528104in}{2.158436in}}{\pgfqpoint{1.528104in}{2.166672in}}%
\pgfpathcurveto{\pgfqpoint{1.528104in}{2.174909in}}{\pgfqpoint{1.524832in}{2.182809in}}{\pgfqpoint{1.519008in}{2.188633in}}%
\pgfpathcurveto{\pgfqpoint{1.513184in}{2.194457in}}{\pgfqpoint{1.505284in}{2.197729in}}{\pgfqpoint{1.497048in}{2.197729in}}%
\pgfpathcurveto{\pgfqpoint{1.488811in}{2.197729in}}{\pgfqpoint{1.480911in}{2.194457in}}{\pgfqpoint{1.475087in}{2.188633in}}%
\pgfpathcurveto{\pgfqpoint{1.469263in}{2.182809in}}{\pgfqpoint{1.465991in}{2.174909in}}{\pgfqpoint{1.465991in}{2.166672in}}%
\pgfpathcurveto{\pgfqpoint{1.465991in}{2.158436in}}{\pgfqpoint{1.469263in}{2.150536in}}{\pgfqpoint{1.475087in}{2.144712in}}%
\pgfpathcurveto{\pgfqpoint{1.480911in}{2.138888in}}{\pgfqpoint{1.488811in}{2.135616in}}{\pgfqpoint{1.497048in}{2.135616in}}%
\pgfpathclose%
\pgfusepath{stroke,fill}%
\end{pgfscope}%
\begin{pgfscope}%
\pgfpathrectangle{\pgfqpoint{0.100000in}{0.212622in}}{\pgfqpoint{3.696000in}{3.696000in}}%
\pgfusepath{clip}%
\pgfsetbuttcap%
\pgfsetroundjoin%
\definecolor{currentfill}{rgb}{0.121569,0.466667,0.705882}%
\pgfsetfillcolor{currentfill}%
\pgfsetfillopacity{0.363944}%
\pgfsetlinewidth{1.003750pt}%
\definecolor{currentstroke}{rgb}{0.121569,0.466667,0.705882}%
\pgfsetstrokecolor{currentstroke}%
\pgfsetstrokeopacity{0.363944}%
\pgfsetdash{}{0pt}%
\pgfpathmoveto{\pgfqpoint{2.039161in}{2.323576in}}%
\pgfpathcurveto{\pgfqpoint{2.047398in}{2.323576in}}{\pgfqpoint{2.055298in}{2.326848in}}{\pgfqpoint{2.061122in}{2.332672in}}%
\pgfpathcurveto{\pgfqpoint{2.066945in}{2.338496in}}{\pgfqpoint{2.070218in}{2.346396in}}{\pgfqpoint{2.070218in}{2.354632in}}%
\pgfpathcurveto{\pgfqpoint{2.070218in}{2.362868in}}{\pgfqpoint{2.066945in}{2.370768in}}{\pgfqpoint{2.061122in}{2.376592in}}%
\pgfpathcurveto{\pgfqpoint{2.055298in}{2.382416in}}{\pgfqpoint{2.047398in}{2.385689in}}{\pgfqpoint{2.039161in}{2.385689in}}%
\pgfpathcurveto{\pgfqpoint{2.030925in}{2.385689in}}{\pgfqpoint{2.023025in}{2.382416in}}{\pgfqpoint{2.017201in}{2.376592in}}%
\pgfpathcurveto{\pgfqpoint{2.011377in}{2.370768in}}{\pgfqpoint{2.008105in}{2.362868in}}{\pgfqpoint{2.008105in}{2.354632in}}%
\pgfpathcurveto{\pgfqpoint{2.008105in}{2.346396in}}{\pgfqpoint{2.011377in}{2.338496in}}{\pgfqpoint{2.017201in}{2.332672in}}%
\pgfpathcurveto{\pgfqpoint{2.023025in}{2.326848in}}{\pgfqpoint{2.030925in}{2.323576in}}{\pgfqpoint{2.039161in}{2.323576in}}%
\pgfpathclose%
\pgfusepath{stroke,fill}%
\end{pgfscope}%
\begin{pgfscope}%
\pgfpathrectangle{\pgfqpoint{0.100000in}{0.212622in}}{\pgfqpoint{3.696000in}{3.696000in}}%
\pgfusepath{clip}%
\pgfsetbuttcap%
\pgfsetroundjoin%
\definecolor{currentfill}{rgb}{0.121569,0.466667,0.705882}%
\pgfsetfillcolor{currentfill}%
\pgfsetfillopacity{0.366360}%
\pgfsetlinewidth{1.003750pt}%
\definecolor{currentstroke}{rgb}{0.121569,0.466667,0.705882}%
\pgfsetstrokecolor{currentstroke}%
\pgfsetstrokeopacity{0.366360}%
\pgfsetdash{}{0pt}%
\pgfpathmoveto{\pgfqpoint{2.052721in}{2.323846in}}%
\pgfpathcurveto{\pgfqpoint{2.060957in}{2.323846in}}{\pgfqpoint{2.068858in}{2.327119in}}{\pgfqpoint{2.074681in}{2.332943in}}%
\pgfpathcurveto{\pgfqpoint{2.080505in}{2.338767in}}{\pgfqpoint{2.083778in}{2.346667in}}{\pgfqpoint{2.083778in}{2.354903in}}%
\pgfpathcurveto{\pgfqpoint{2.083778in}{2.363139in}}{\pgfqpoint{2.080505in}{2.371039in}}{\pgfqpoint{2.074681in}{2.376863in}}%
\pgfpathcurveto{\pgfqpoint{2.068858in}{2.382687in}}{\pgfqpoint{2.060957in}{2.385959in}}{\pgfqpoint{2.052721in}{2.385959in}}%
\pgfpathcurveto{\pgfqpoint{2.044485in}{2.385959in}}{\pgfqpoint{2.036585in}{2.382687in}}{\pgfqpoint{2.030761in}{2.376863in}}%
\pgfpathcurveto{\pgfqpoint{2.024937in}{2.371039in}}{\pgfqpoint{2.021665in}{2.363139in}}{\pgfqpoint{2.021665in}{2.354903in}}%
\pgfpathcurveto{\pgfqpoint{2.021665in}{2.346667in}}{\pgfqpoint{2.024937in}{2.338767in}}{\pgfqpoint{2.030761in}{2.332943in}}%
\pgfpathcurveto{\pgfqpoint{2.036585in}{2.327119in}}{\pgfqpoint{2.044485in}{2.323846in}}{\pgfqpoint{2.052721in}{2.323846in}}%
\pgfpathclose%
\pgfusepath{stroke,fill}%
\end{pgfscope}%
\begin{pgfscope}%
\pgfpathrectangle{\pgfqpoint{0.100000in}{0.212622in}}{\pgfqpoint{3.696000in}{3.696000in}}%
\pgfusepath{clip}%
\pgfsetbuttcap%
\pgfsetroundjoin%
\definecolor{currentfill}{rgb}{0.121569,0.466667,0.705882}%
\pgfsetfillcolor{currentfill}%
\pgfsetfillopacity{0.369106}%
\pgfsetlinewidth{1.003750pt}%
\definecolor{currentstroke}{rgb}{0.121569,0.466667,0.705882}%
\pgfsetstrokecolor{currentstroke}%
\pgfsetstrokeopacity{0.369106}%
\pgfsetdash{}{0pt}%
\pgfpathmoveto{\pgfqpoint{2.069877in}{2.320210in}}%
\pgfpathcurveto{\pgfqpoint{2.078114in}{2.320210in}}{\pgfqpoint{2.086014in}{2.323483in}}{\pgfqpoint{2.091838in}{2.329307in}}%
\pgfpathcurveto{\pgfqpoint{2.097662in}{2.335130in}}{\pgfqpoint{2.100934in}{2.343031in}}{\pgfqpoint{2.100934in}{2.351267in}}%
\pgfpathcurveto{\pgfqpoint{2.100934in}{2.359503in}}{\pgfqpoint{2.097662in}{2.367403in}}{\pgfqpoint{2.091838in}{2.373227in}}%
\pgfpathcurveto{\pgfqpoint{2.086014in}{2.379051in}}{\pgfqpoint{2.078114in}{2.382323in}}{\pgfqpoint{2.069877in}{2.382323in}}%
\pgfpathcurveto{\pgfqpoint{2.061641in}{2.382323in}}{\pgfqpoint{2.053741in}{2.379051in}}{\pgfqpoint{2.047917in}{2.373227in}}%
\pgfpathcurveto{\pgfqpoint{2.042093in}{2.367403in}}{\pgfqpoint{2.038821in}{2.359503in}}{\pgfqpoint{2.038821in}{2.351267in}}%
\pgfpathcurveto{\pgfqpoint{2.038821in}{2.343031in}}{\pgfqpoint{2.042093in}{2.335130in}}{\pgfqpoint{2.047917in}{2.329307in}}%
\pgfpathcurveto{\pgfqpoint{2.053741in}{2.323483in}}{\pgfqpoint{2.061641in}{2.320210in}}{\pgfqpoint{2.069877in}{2.320210in}}%
\pgfpathclose%
\pgfusepath{stroke,fill}%
\end{pgfscope}%
\begin{pgfscope}%
\pgfpathrectangle{\pgfqpoint{0.100000in}{0.212622in}}{\pgfqpoint{3.696000in}{3.696000in}}%
\pgfusepath{clip}%
\pgfsetbuttcap%
\pgfsetroundjoin%
\definecolor{currentfill}{rgb}{0.121569,0.466667,0.705882}%
\pgfsetfillcolor{currentfill}%
\pgfsetfillopacity{0.369323}%
\pgfsetlinewidth{1.003750pt}%
\definecolor{currentstroke}{rgb}{0.121569,0.466667,0.705882}%
\pgfsetstrokecolor{currentstroke}%
\pgfsetstrokeopacity{0.369323}%
\pgfsetdash{}{0pt}%
\pgfpathmoveto{\pgfqpoint{1.478360in}{2.106053in}}%
\pgfpathcurveto{\pgfqpoint{1.486596in}{2.106053in}}{\pgfqpoint{1.494496in}{2.109325in}}{\pgfqpoint{1.500320in}{2.115149in}}%
\pgfpathcurveto{\pgfqpoint{1.506144in}{2.120973in}}{\pgfqpoint{1.509417in}{2.128873in}}{\pgfqpoint{1.509417in}{2.137110in}}%
\pgfpathcurveto{\pgfqpoint{1.509417in}{2.145346in}}{\pgfqpoint{1.506144in}{2.153246in}}{\pgfqpoint{1.500320in}{2.159070in}}%
\pgfpathcurveto{\pgfqpoint{1.494496in}{2.164894in}}{\pgfqpoint{1.486596in}{2.168166in}}{\pgfqpoint{1.478360in}{2.168166in}}%
\pgfpathcurveto{\pgfqpoint{1.470124in}{2.168166in}}{\pgfqpoint{1.462224in}{2.164894in}}{\pgfqpoint{1.456400in}{2.159070in}}%
\pgfpathcurveto{\pgfqpoint{1.450576in}{2.153246in}}{\pgfqpoint{1.447304in}{2.145346in}}{\pgfqpoint{1.447304in}{2.137110in}}%
\pgfpathcurveto{\pgfqpoint{1.447304in}{2.128873in}}{\pgfqpoint{1.450576in}{2.120973in}}{\pgfqpoint{1.456400in}{2.115149in}}%
\pgfpathcurveto{\pgfqpoint{1.462224in}{2.109325in}}{\pgfqpoint{1.470124in}{2.106053in}}{\pgfqpoint{1.478360in}{2.106053in}}%
\pgfpathclose%
\pgfusepath{stroke,fill}%
\end{pgfscope}%
\begin{pgfscope}%
\pgfpathrectangle{\pgfqpoint{0.100000in}{0.212622in}}{\pgfqpoint{3.696000in}{3.696000in}}%
\pgfusepath{clip}%
\pgfsetbuttcap%
\pgfsetroundjoin%
\definecolor{currentfill}{rgb}{0.121569,0.466667,0.705882}%
\pgfsetfillcolor{currentfill}%
\pgfsetfillopacity{0.373110}%
\pgfsetlinewidth{1.003750pt}%
\definecolor{currentstroke}{rgb}{0.121569,0.466667,0.705882}%
\pgfsetstrokecolor{currentstroke}%
\pgfsetstrokeopacity{0.373110}%
\pgfsetdash{}{0pt}%
\pgfpathmoveto{\pgfqpoint{2.092679in}{2.319467in}}%
\pgfpathcurveto{\pgfqpoint{2.100915in}{2.319467in}}{\pgfqpoint{2.108815in}{2.322739in}}{\pgfqpoint{2.114639in}{2.328563in}}%
\pgfpathcurveto{\pgfqpoint{2.120463in}{2.334387in}}{\pgfqpoint{2.123735in}{2.342287in}}{\pgfqpoint{2.123735in}{2.350523in}}%
\pgfpathcurveto{\pgfqpoint{2.123735in}{2.358760in}}{\pgfqpoint{2.120463in}{2.366660in}}{\pgfqpoint{2.114639in}{2.372484in}}%
\pgfpathcurveto{\pgfqpoint{2.108815in}{2.378308in}}{\pgfqpoint{2.100915in}{2.381580in}}{\pgfqpoint{2.092679in}{2.381580in}}%
\pgfpathcurveto{\pgfqpoint{2.084442in}{2.381580in}}{\pgfqpoint{2.076542in}{2.378308in}}{\pgfqpoint{2.070719in}{2.372484in}}%
\pgfpathcurveto{\pgfqpoint{2.064895in}{2.366660in}}{\pgfqpoint{2.061622in}{2.358760in}}{\pgfqpoint{2.061622in}{2.350523in}}%
\pgfpathcurveto{\pgfqpoint{2.061622in}{2.342287in}}{\pgfqpoint{2.064895in}{2.334387in}}{\pgfqpoint{2.070719in}{2.328563in}}%
\pgfpathcurveto{\pgfqpoint{2.076542in}{2.322739in}}{\pgfqpoint{2.084442in}{2.319467in}}{\pgfqpoint{2.092679in}{2.319467in}}%
\pgfpathclose%
\pgfusepath{stroke,fill}%
\end{pgfscope}%
\begin{pgfscope}%
\pgfpathrectangle{\pgfqpoint{0.100000in}{0.212622in}}{\pgfqpoint{3.696000in}{3.696000in}}%
\pgfusepath{clip}%
\pgfsetbuttcap%
\pgfsetroundjoin%
\definecolor{currentfill}{rgb}{0.121569,0.466667,0.705882}%
\pgfsetfillcolor{currentfill}%
\pgfsetfillopacity{0.376741}%
\pgfsetlinewidth{1.003750pt}%
\definecolor{currentstroke}{rgb}{0.121569,0.466667,0.705882}%
\pgfsetstrokecolor{currentstroke}%
\pgfsetstrokeopacity{0.376741}%
\pgfsetdash{}{0pt}%
\pgfpathmoveto{\pgfqpoint{1.456090in}{2.085098in}}%
\pgfpathcurveto{\pgfqpoint{1.464327in}{2.085098in}}{\pgfqpoint{1.472227in}{2.088370in}}{\pgfqpoint{1.478051in}{2.094194in}}%
\pgfpathcurveto{\pgfqpoint{1.483875in}{2.100018in}}{\pgfqpoint{1.487147in}{2.107918in}}{\pgfqpoint{1.487147in}{2.116154in}}%
\pgfpathcurveto{\pgfqpoint{1.487147in}{2.124391in}}{\pgfqpoint{1.483875in}{2.132291in}}{\pgfqpoint{1.478051in}{2.138114in}}%
\pgfpathcurveto{\pgfqpoint{1.472227in}{2.143938in}}{\pgfqpoint{1.464327in}{2.147211in}}{\pgfqpoint{1.456090in}{2.147211in}}%
\pgfpathcurveto{\pgfqpoint{1.447854in}{2.147211in}}{\pgfqpoint{1.439954in}{2.143938in}}{\pgfqpoint{1.434130in}{2.138114in}}%
\pgfpathcurveto{\pgfqpoint{1.428306in}{2.132291in}}{\pgfqpoint{1.425034in}{2.124391in}}{\pgfqpoint{1.425034in}{2.116154in}}%
\pgfpathcurveto{\pgfqpoint{1.425034in}{2.107918in}}{\pgfqpoint{1.428306in}{2.100018in}}{\pgfqpoint{1.434130in}{2.094194in}}%
\pgfpathcurveto{\pgfqpoint{1.439954in}{2.088370in}}{\pgfqpoint{1.447854in}{2.085098in}}{\pgfqpoint{1.456090in}{2.085098in}}%
\pgfpathclose%
\pgfusepath{stroke,fill}%
\end{pgfscope}%
\begin{pgfscope}%
\pgfpathrectangle{\pgfqpoint{0.100000in}{0.212622in}}{\pgfqpoint{3.696000in}{3.696000in}}%
\pgfusepath{clip}%
\pgfsetbuttcap%
\pgfsetroundjoin%
\definecolor{currentfill}{rgb}{0.121569,0.466667,0.705882}%
\pgfsetfillcolor{currentfill}%
\pgfsetfillopacity{0.379446}%
\pgfsetlinewidth{1.003750pt}%
\definecolor{currentstroke}{rgb}{0.121569,0.466667,0.705882}%
\pgfsetstrokecolor{currentstroke}%
\pgfsetstrokeopacity{0.379446}%
\pgfsetdash{}{0pt}%
\pgfpathmoveto{\pgfqpoint{2.121682in}{2.325991in}}%
\pgfpathcurveto{\pgfqpoint{2.129918in}{2.325991in}}{\pgfqpoint{2.137818in}{2.329263in}}{\pgfqpoint{2.143642in}{2.335087in}}%
\pgfpathcurveto{\pgfqpoint{2.149466in}{2.340911in}}{\pgfqpoint{2.152738in}{2.348811in}}{\pgfqpoint{2.152738in}{2.357047in}}%
\pgfpathcurveto{\pgfqpoint{2.152738in}{2.365284in}}{\pgfqpoint{2.149466in}{2.373184in}}{\pgfqpoint{2.143642in}{2.379008in}}%
\pgfpathcurveto{\pgfqpoint{2.137818in}{2.384832in}}{\pgfqpoint{2.129918in}{2.388104in}}{\pgfqpoint{2.121682in}{2.388104in}}%
\pgfpathcurveto{\pgfqpoint{2.113445in}{2.388104in}}{\pgfqpoint{2.105545in}{2.384832in}}{\pgfqpoint{2.099721in}{2.379008in}}%
\pgfpathcurveto{\pgfqpoint{2.093897in}{2.373184in}}{\pgfqpoint{2.090625in}{2.365284in}}{\pgfqpoint{2.090625in}{2.357047in}}%
\pgfpathcurveto{\pgfqpoint{2.090625in}{2.348811in}}{\pgfqpoint{2.093897in}{2.340911in}}{\pgfqpoint{2.099721in}{2.335087in}}%
\pgfpathcurveto{\pgfqpoint{2.105545in}{2.329263in}}{\pgfqpoint{2.113445in}{2.325991in}}{\pgfqpoint{2.121682in}{2.325991in}}%
\pgfpathclose%
\pgfusepath{stroke,fill}%
\end{pgfscope}%
\begin{pgfscope}%
\pgfpathrectangle{\pgfqpoint{0.100000in}{0.212622in}}{\pgfqpoint{3.696000in}{3.696000in}}%
\pgfusepath{clip}%
\pgfsetbuttcap%
\pgfsetroundjoin%
\definecolor{currentfill}{rgb}{0.121569,0.466667,0.705882}%
\pgfsetfillcolor{currentfill}%
\pgfsetfillopacity{0.383618}%
\pgfsetlinewidth{1.003750pt}%
\definecolor{currentstroke}{rgb}{0.121569,0.466667,0.705882}%
\pgfsetstrokecolor{currentstroke}%
\pgfsetstrokeopacity{0.383618}%
\pgfsetdash{}{0pt}%
\pgfpathmoveto{\pgfqpoint{1.443062in}{2.064389in}}%
\pgfpathcurveto{\pgfqpoint{1.451298in}{2.064389in}}{\pgfqpoint{1.459198in}{2.067661in}}{\pgfqpoint{1.465022in}{2.073485in}}%
\pgfpathcurveto{\pgfqpoint{1.470846in}{2.079309in}}{\pgfqpoint{1.474118in}{2.087209in}}{\pgfqpoint{1.474118in}{2.095446in}}%
\pgfpathcurveto{\pgfqpoint{1.474118in}{2.103682in}}{\pgfqpoint{1.470846in}{2.111582in}}{\pgfqpoint{1.465022in}{2.117406in}}%
\pgfpathcurveto{\pgfqpoint{1.459198in}{2.123230in}}{\pgfqpoint{1.451298in}{2.126502in}}{\pgfqpoint{1.443062in}{2.126502in}}%
\pgfpathcurveto{\pgfqpoint{1.434825in}{2.126502in}}{\pgfqpoint{1.426925in}{2.123230in}}{\pgfqpoint{1.421101in}{2.117406in}}%
\pgfpathcurveto{\pgfqpoint{1.415278in}{2.111582in}}{\pgfqpoint{1.412005in}{2.103682in}}{\pgfqpoint{1.412005in}{2.095446in}}%
\pgfpathcurveto{\pgfqpoint{1.412005in}{2.087209in}}{\pgfqpoint{1.415278in}{2.079309in}}{\pgfqpoint{1.421101in}{2.073485in}}%
\pgfpathcurveto{\pgfqpoint{1.426925in}{2.067661in}}{\pgfqpoint{1.434825in}{2.064389in}}{\pgfqpoint{1.443062in}{2.064389in}}%
\pgfpathclose%
\pgfusepath{stroke,fill}%
\end{pgfscope}%
\begin{pgfscope}%
\pgfpathrectangle{\pgfqpoint{0.100000in}{0.212622in}}{\pgfqpoint{3.696000in}{3.696000in}}%
\pgfusepath{clip}%
\pgfsetbuttcap%
\pgfsetroundjoin%
\definecolor{currentfill}{rgb}{0.121569,0.466667,0.705882}%
\pgfsetfillcolor{currentfill}%
\pgfsetfillopacity{0.386045}%
\pgfsetlinewidth{1.003750pt}%
\definecolor{currentstroke}{rgb}{0.121569,0.466667,0.705882}%
\pgfsetstrokecolor{currentstroke}%
\pgfsetstrokeopacity{0.386045}%
\pgfsetdash{}{0pt}%
\pgfpathmoveto{\pgfqpoint{2.150230in}{2.317565in}}%
\pgfpathcurveto{\pgfqpoint{2.158466in}{2.317565in}}{\pgfqpoint{2.166366in}{2.320838in}}{\pgfqpoint{2.172190in}{2.326662in}}%
\pgfpathcurveto{\pgfqpoint{2.178014in}{2.332486in}}{\pgfqpoint{2.181286in}{2.340386in}}{\pgfqpoint{2.181286in}{2.348622in}}%
\pgfpathcurveto{\pgfqpoint{2.181286in}{2.356858in}}{\pgfqpoint{2.178014in}{2.364758in}}{\pgfqpoint{2.172190in}{2.370582in}}%
\pgfpathcurveto{\pgfqpoint{2.166366in}{2.376406in}}{\pgfqpoint{2.158466in}{2.379678in}}{\pgfqpoint{2.150230in}{2.379678in}}%
\pgfpathcurveto{\pgfqpoint{2.141994in}{2.379678in}}{\pgfqpoint{2.134094in}{2.376406in}}{\pgfqpoint{2.128270in}{2.370582in}}%
\pgfpathcurveto{\pgfqpoint{2.122446in}{2.364758in}}{\pgfqpoint{2.119173in}{2.356858in}}{\pgfqpoint{2.119173in}{2.348622in}}%
\pgfpathcurveto{\pgfqpoint{2.119173in}{2.340386in}}{\pgfqpoint{2.122446in}{2.332486in}}{\pgfqpoint{2.128270in}{2.326662in}}%
\pgfpathcurveto{\pgfqpoint{2.134094in}{2.320838in}}{\pgfqpoint{2.141994in}{2.317565in}}{\pgfqpoint{2.150230in}{2.317565in}}%
\pgfpathclose%
\pgfusepath{stroke,fill}%
\end{pgfscope}%
\begin{pgfscope}%
\pgfpathrectangle{\pgfqpoint{0.100000in}{0.212622in}}{\pgfqpoint{3.696000in}{3.696000in}}%
\pgfusepath{clip}%
\pgfsetbuttcap%
\pgfsetroundjoin%
\definecolor{currentfill}{rgb}{0.121569,0.466667,0.705882}%
\pgfsetfillcolor{currentfill}%
\pgfsetfillopacity{0.388249}%
\pgfsetlinewidth{1.003750pt}%
\definecolor{currentstroke}{rgb}{0.121569,0.466667,0.705882}%
\pgfsetstrokecolor{currentstroke}%
\pgfsetstrokeopacity{0.388249}%
\pgfsetdash{}{0pt}%
\pgfpathmoveto{\pgfqpoint{1.428736in}{2.052956in}}%
\pgfpathcurveto{\pgfqpoint{1.436972in}{2.052956in}}{\pgfqpoint{1.444872in}{2.056228in}}{\pgfqpoint{1.450696in}{2.062052in}}%
\pgfpathcurveto{\pgfqpoint{1.456520in}{2.067876in}}{\pgfqpoint{1.459792in}{2.075776in}}{\pgfqpoint{1.459792in}{2.084012in}}%
\pgfpathcurveto{\pgfqpoint{1.459792in}{2.092248in}}{\pgfqpoint{1.456520in}{2.100148in}}{\pgfqpoint{1.450696in}{2.105972in}}%
\pgfpathcurveto{\pgfqpoint{1.444872in}{2.111796in}}{\pgfqpoint{1.436972in}{2.115069in}}{\pgfqpoint{1.428736in}{2.115069in}}%
\pgfpathcurveto{\pgfqpoint{1.420499in}{2.115069in}}{\pgfqpoint{1.412599in}{2.111796in}}{\pgfqpoint{1.406775in}{2.105972in}}%
\pgfpathcurveto{\pgfqpoint{1.400951in}{2.100148in}}{\pgfqpoint{1.397679in}{2.092248in}}{\pgfqpoint{1.397679in}{2.084012in}}%
\pgfpathcurveto{\pgfqpoint{1.397679in}{2.075776in}}{\pgfqpoint{1.400951in}{2.067876in}}{\pgfqpoint{1.406775in}{2.062052in}}%
\pgfpathcurveto{\pgfqpoint{1.412599in}{2.056228in}}{\pgfqpoint{1.420499in}{2.052956in}}{\pgfqpoint{1.428736in}{2.052956in}}%
\pgfpathclose%
\pgfusepath{stroke,fill}%
\end{pgfscope}%
\begin{pgfscope}%
\pgfpathrectangle{\pgfqpoint{0.100000in}{0.212622in}}{\pgfqpoint{3.696000in}{3.696000in}}%
\pgfusepath{clip}%
\pgfsetbuttcap%
\pgfsetroundjoin%
\definecolor{currentfill}{rgb}{0.121569,0.466667,0.705882}%
\pgfsetfillcolor{currentfill}%
\pgfsetfillopacity{0.391743}%
\pgfsetlinewidth{1.003750pt}%
\definecolor{currentstroke}{rgb}{0.121569,0.466667,0.705882}%
\pgfsetstrokecolor{currentstroke}%
\pgfsetstrokeopacity{0.391743}%
\pgfsetdash{}{0pt}%
\pgfpathmoveto{\pgfqpoint{1.422380in}{2.040141in}}%
\pgfpathcurveto{\pgfqpoint{1.430616in}{2.040141in}}{\pgfqpoint{1.438516in}{2.043413in}}{\pgfqpoint{1.444340in}{2.049237in}}%
\pgfpathcurveto{\pgfqpoint{1.450164in}{2.055061in}}{\pgfqpoint{1.453437in}{2.062961in}}{\pgfqpoint{1.453437in}{2.071197in}}%
\pgfpathcurveto{\pgfqpoint{1.453437in}{2.079434in}}{\pgfqpoint{1.450164in}{2.087334in}}{\pgfqpoint{1.444340in}{2.093158in}}%
\pgfpathcurveto{\pgfqpoint{1.438516in}{2.098982in}}{\pgfqpoint{1.430616in}{2.102254in}}{\pgfqpoint{1.422380in}{2.102254in}}%
\pgfpathcurveto{\pgfqpoint{1.414144in}{2.102254in}}{\pgfqpoint{1.406244in}{2.098982in}}{\pgfqpoint{1.400420in}{2.093158in}}%
\pgfpathcurveto{\pgfqpoint{1.394596in}{2.087334in}}{\pgfqpoint{1.391324in}{2.079434in}}{\pgfqpoint{1.391324in}{2.071197in}}%
\pgfpathcurveto{\pgfqpoint{1.391324in}{2.062961in}}{\pgfqpoint{1.394596in}{2.055061in}}{\pgfqpoint{1.400420in}{2.049237in}}%
\pgfpathcurveto{\pgfqpoint{1.406244in}{2.043413in}}{\pgfqpoint{1.414144in}{2.040141in}}{\pgfqpoint{1.422380in}{2.040141in}}%
\pgfpathclose%
\pgfusepath{stroke,fill}%
\end{pgfscope}%
\begin{pgfscope}%
\pgfpathrectangle{\pgfqpoint{0.100000in}{0.212622in}}{\pgfqpoint{3.696000in}{3.696000in}}%
\pgfusepath{clip}%
\pgfsetbuttcap%
\pgfsetroundjoin%
\definecolor{currentfill}{rgb}{0.121569,0.466667,0.705882}%
\pgfsetfillcolor{currentfill}%
\pgfsetfillopacity{0.393465}%
\pgfsetlinewidth{1.003750pt}%
\definecolor{currentstroke}{rgb}{0.121569,0.466667,0.705882}%
\pgfsetstrokecolor{currentstroke}%
\pgfsetstrokeopacity{0.393465}%
\pgfsetdash{}{0pt}%
\pgfpathmoveto{\pgfqpoint{2.189221in}{2.323373in}}%
\pgfpathcurveto{\pgfqpoint{2.197457in}{2.323373in}}{\pgfqpoint{2.205358in}{2.326645in}}{\pgfqpoint{2.211181in}{2.332469in}}%
\pgfpathcurveto{\pgfqpoint{2.217005in}{2.338293in}}{\pgfqpoint{2.220278in}{2.346193in}}{\pgfqpoint{2.220278in}{2.354429in}}%
\pgfpathcurveto{\pgfqpoint{2.220278in}{2.362666in}}{\pgfqpoint{2.217005in}{2.370566in}}{\pgfqpoint{2.211181in}{2.376390in}}%
\pgfpathcurveto{\pgfqpoint{2.205358in}{2.382214in}}{\pgfqpoint{2.197457in}{2.385486in}}{\pgfqpoint{2.189221in}{2.385486in}}%
\pgfpathcurveto{\pgfqpoint{2.180985in}{2.385486in}}{\pgfqpoint{2.173085in}{2.382214in}}{\pgfqpoint{2.167261in}{2.376390in}}%
\pgfpathcurveto{\pgfqpoint{2.161437in}{2.370566in}}{\pgfqpoint{2.158165in}{2.362666in}}{\pgfqpoint{2.158165in}{2.354429in}}%
\pgfpathcurveto{\pgfqpoint{2.158165in}{2.346193in}}{\pgfqpoint{2.161437in}{2.338293in}}{\pgfqpoint{2.167261in}{2.332469in}}%
\pgfpathcurveto{\pgfqpoint{2.173085in}{2.326645in}}{\pgfqpoint{2.180985in}{2.323373in}}{\pgfqpoint{2.189221in}{2.323373in}}%
\pgfpathclose%
\pgfusepath{stroke,fill}%
\end{pgfscope}%
\begin{pgfscope}%
\pgfpathrectangle{\pgfqpoint{0.100000in}{0.212622in}}{\pgfqpoint{3.696000in}{3.696000in}}%
\pgfusepath{clip}%
\pgfsetbuttcap%
\pgfsetroundjoin%
\definecolor{currentfill}{rgb}{0.121569,0.466667,0.705882}%
\pgfsetfillcolor{currentfill}%
\pgfsetfillopacity{0.394720}%
\pgfsetlinewidth{1.003750pt}%
\definecolor{currentstroke}{rgb}{0.121569,0.466667,0.705882}%
\pgfsetstrokecolor{currentstroke}%
\pgfsetstrokeopacity{0.394720}%
\pgfsetdash{}{0pt}%
\pgfpathmoveto{\pgfqpoint{1.414851in}{2.031260in}}%
\pgfpathcurveto{\pgfqpoint{1.423088in}{2.031260in}}{\pgfqpoint{1.430988in}{2.034533in}}{\pgfqpoint{1.436811in}{2.040357in}}%
\pgfpathcurveto{\pgfqpoint{1.442635in}{2.046181in}}{\pgfqpoint{1.445908in}{2.054081in}}{\pgfqpoint{1.445908in}{2.062317in}}%
\pgfpathcurveto{\pgfqpoint{1.445908in}{2.070553in}}{\pgfqpoint{1.442635in}{2.078453in}}{\pgfqpoint{1.436811in}{2.084277in}}%
\pgfpathcurveto{\pgfqpoint{1.430988in}{2.090101in}}{\pgfqpoint{1.423088in}{2.093373in}}{\pgfqpoint{1.414851in}{2.093373in}}%
\pgfpathcurveto{\pgfqpoint{1.406615in}{2.093373in}}{\pgfqpoint{1.398715in}{2.090101in}}{\pgfqpoint{1.392891in}{2.084277in}}%
\pgfpathcurveto{\pgfqpoint{1.387067in}{2.078453in}}{\pgfqpoint{1.383795in}{2.070553in}}{\pgfqpoint{1.383795in}{2.062317in}}%
\pgfpathcurveto{\pgfqpoint{1.383795in}{2.054081in}}{\pgfqpoint{1.387067in}{2.046181in}}{\pgfqpoint{1.392891in}{2.040357in}}%
\pgfpathcurveto{\pgfqpoint{1.398715in}{2.034533in}}{\pgfqpoint{1.406615in}{2.031260in}}{\pgfqpoint{1.414851in}{2.031260in}}%
\pgfpathclose%
\pgfusepath{stroke,fill}%
\end{pgfscope}%
\begin{pgfscope}%
\pgfpathrectangle{\pgfqpoint{0.100000in}{0.212622in}}{\pgfqpoint{3.696000in}{3.696000in}}%
\pgfusepath{clip}%
\pgfsetbuttcap%
\pgfsetroundjoin%
\definecolor{currentfill}{rgb}{0.121569,0.466667,0.705882}%
\pgfsetfillcolor{currentfill}%
\pgfsetfillopacity{0.395977}%
\pgfsetlinewidth{1.003750pt}%
\definecolor{currentstroke}{rgb}{0.121569,0.466667,0.705882}%
\pgfsetstrokecolor{currentstroke}%
\pgfsetstrokeopacity{0.395977}%
\pgfsetdash{}{0pt}%
\pgfpathmoveto{\pgfqpoint{1.411679in}{2.025756in}}%
\pgfpathcurveto{\pgfqpoint{1.419915in}{2.025756in}}{\pgfqpoint{1.427815in}{2.029028in}}{\pgfqpoint{1.433639in}{2.034852in}}%
\pgfpathcurveto{\pgfqpoint{1.439463in}{2.040676in}}{\pgfqpoint{1.442735in}{2.048576in}}{\pgfqpoint{1.442735in}{2.056813in}}%
\pgfpathcurveto{\pgfqpoint{1.442735in}{2.065049in}}{\pgfqpoint{1.439463in}{2.072949in}}{\pgfqpoint{1.433639in}{2.078773in}}%
\pgfpathcurveto{\pgfqpoint{1.427815in}{2.084597in}}{\pgfqpoint{1.419915in}{2.087869in}}{\pgfqpoint{1.411679in}{2.087869in}}%
\pgfpathcurveto{\pgfqpoint{1.403442in}{2.087869in}}{\pgfqpoint{1.395542in}{2.084597in}}{\pgfqpoint{1.389718in}{2.078773in}}%
\pgfpathcurveto{\pgfqpoint{1.383894in}{2.072949in}}{\pgfqpoint{1.380622in}{2.065049in}}{\pgfqpoint{1.380622in}{2.056813in}}%
\pgfpathcurveto{\pgfqpoint{1.380622in}{2.048576in}}{\pgfqpoint{1.383894in}{2.040676in}}{\pgfqpoint{1.389718in}{2.034852in}}%
\pgfpathcurveto{\pgfqpoint{1.395542in}{2.029028in}}{\pgfqpoint{1.403442in}{2.025756in}}{\pgfqpoint{1.411679in}{2.025756in}}%
\pgfpathclose%
\pgfusepath{stroke,fill}%
\end{pgfscope}%
\begin{pgfscope}%
\pgfpathrectangle{\pgfqpoint{0.100000in}{0.212622in}}{\pgfqpoint{3.696000in}{3.696000in}}%
\pgfusepath{clip}%
\pgfsetbuttcap%
\pgfsetroundjoin%
\definecolor{currentfill}{rgb}{0.121569,0.466667,0.705882}%
\pgfsetfillcolor{currentfill}%
\pgfsetfillopacity{0.396600}%
\pgfsetlinewidth{1.003750pt}%
\definecolor{currentstroke}{rgb}{0.121569,0.466667,0.705882}%
\pgfsetstrokecolor{currentstroke}%
\pgfsetstrokeopacity{0.396600}%
\pgfsetdash{}{0pt}%
\pgfpathmoveto{\pgfqpoint{1.410240in}{2.023612in}}%
\pgfpathcurveto{\pgfqpoint{1.418476in}{2.023612in}}{\pgfqpoint{1.426376in}{2.026885in}}{\pgfqpoint{1.432200in}{2.032709in}}%
\pgfpathcurveto{\pgfqpoint{1.438024in}{2.038533in}}{\pgfqpoint{1.441296in}{2.046433in}}{\pgfqpoint{1.441296in}{2.054669in}}%
\pgfpathcurveto{\pgfqpoint{1.441296in}{2.062905in}}{\pgfqpoint{1.438024in}{2.070805in}}{\pgfqpoint{1.432200in}{2.076629in}}%
\pgfpathcurveto{\pgfqpoint{1.426376in}{2.082453in}}{\pgfqpoint{1.418476in}{2.085725in}}{\pgfqpoint{1.410240in}{2.085725in}}%
\pgfpathcurveto{\pgfqpoint{1.402004in}{2.085725in}}{\pgfqpoint{1.394104in}{2.082453in}}{\pgfqpoint{1.388280in}{2.076629in}}%
\pgfpathcurveto{\pgfqpoint{1.382456in}{2.070805in}}{\pgfqpoint{1.379183in}{2.062905in}}{\pgfqpoint{1.379183in}{2.054669in}}%
\pgfpathcurveto{\pgfqpoint{1.379183in}{2.046433in}}{\pgfqpoint{1.382456in}{2.038533in}}{\pgfqpoint{1.388280in}{2.032709in}}%
\pgfpathcurveto{\pgfqpoint{1.394104in}{2.026885in}}{\pgfqpoint{1.402004in}{2.023612in}}{\pgfqpoint{1.410240in}{2.023612in}}%
\pgfpathclose%
\pgfusepath{stroke,fill}%
\end{pgfscope}%
\begin{pgfscope}%
\pgfpathrectangle{\pgfqpoint{0.100000in}{0.212622in}}{\pgfqpoint{3.696000in}{3.696000in}}%
\pgfusepath{clip}%
\pgfsetbuttcap%
\pgfsetroundjoin%
\definecolor{currentfill}{rgb}{0.121569,0.466667,0.705882}%
\pgfsetfillcolor{currentfill}%
\pgfsetfillopacity{0.397589}%
\pgfsetlinewidth{1.003750pt}%
\definecolor{currentstroke}{rgb}{0.121569,0.466667,0.705882}%
\pgfsetstrokecolor{currentstroke}%
\pgfsetstrokeopacity{0.397589}%
\pgfsetdash{}{0pt}%
\pgfpathmoveto{\pgfqpoint{1.407306in}{2.019064in}}%
\pgfpathcurveto{\pgfqpoint{1.415542in}{2.019064in}}{\pgfqpoint{1.423442in}{2.022336in}}{\pgfqpoint{1.429266in}{2.028160in}}%
\pgfpathcurveto{\pgfqpoint{1.435090in}{2.033984in}}{\pgfqpoint{1.438362in}{2.041884in}}{\pgfqpoint{1.438362in}{2.050121in}}%
\pgfpathcurveto{\pgfqpoint{1.438362in}{2.058357in}}{\pgfqpoint{1.435090in}{2.066257in}}{\pgfqpoint{1.429266in}{2.072081in}}%
\pgfpathcurveto{\pgfqpoint{1.423442in}{2.077905in}}{\pgfqpoint{1.415542in}{2.081177in}}{\pgfqpoint{1.407306in}{2.081177in}}%
\pgfpathcurveto{\pgfqpoint{1.399070in}{2.081177in}}{\pgfqpoint{1.391169in}{2.077905in}}{\pgfqpoint{1.385346in}{2.072081in}}%
\pgfpathcurveto{\pgfqpoint{1.379522in}{2.066257in}}{\pgfqpoint{1.376249in}{2.058357in}}{\pgfqpoint{1.376249in}{2.050121in}}%
\pgfpathcurveto{\pgfqpoint{1.376249in}{2.041884in}}{\pgfqpoint{1.379522in}{2.033984in}}{\pgfqpoint{1.385346in}{2.028160in}}%
\pgfpathcurveto{\pgfqpoint{1.391169in}{2.022336in}}{\pgfqpoint{1.399070in}{2.019064in}}{\pgfqpoint{1.407306in}{2.019064in}}%
\pgfpathclose%
\pgfusepath{stroke,fill}%
\end{pgfscope}%
\begin{pgfscope}%
\pgfpathrectangle{\pgfqpoint{0.100000in}{0.212622in}}{\pgfqpoint{3.696000in}{3.696000in}}%
\pgfusepath{clip}%
\pgfsetbuttcap%
\pgfsetroundjoin%
\definecolor{currentfill}{rgb}{0.121569,0.466667,0.705882}%
\pgfsetfillcolor{currentfill}%
\pgfsetfillopacity{0.399780}%
\pgfsetlinewidth{1.003750pt}%
\definecolor{currentstroke}{rgb}{0.121569,0.466667,0.705882}%
\pgfsetstrokecolor{currentstroke}%
\pgfsetstrokeopacity{0.399780}%
\pgfsetdash{}{0pt}%
\pgfpathmoveto{\pgfqpoint{1.403713in}{2.012064in}}%
\pgfpathcurveto{\pgfqpoint{1.411950in}{2.012064in}}{\pgfqpoint{1.419850in}{2.015336in}}{\pgfqpoint{1.425674in}{2.021160in}}%
\pgfpathcurveto{\pgfqpoint{1.431498in}{2.026984in}}{\pgfqpoint{1.434770in}{2.034884in}}{\pgfqpoint{1.434770in}{2.043121in}}%
\pgfpathcurveto{\pgfqpoint{1.434770in}{2.051357in}}{\pgfqpoint{1.431498in}{2.059257in}}{\pgfqpoint{1.425674in}{2.065081in}}%
\pgfpathcurveto{\pgfqpoint{1.419850in}{2.070905in}}{\pgfqpoint{1.411950in}{2.074177in}}{\pgfqpoint{1.403713in}{2.074177in}}%
\pgfpathcurveto{\pgfqpoint{1.395477in}{2.074177in}}{\pgfqpoint{1.387577in}{2.070905in}}{\pgfqpoint{1.381753in}{2.065081in}}%
\pgfpathcurveto{\pgfqpoint{1.375929in}{2.059257in}}{\pgfqpoint{1.372657in}{2.051357in}}{\pgfqpoint{1.372657in}{2.043121in}}%
\pgfpathcurveto{\pgfqpoint{1.372657in}{2.034884in}}{\pgfqpoint{1.375929in}{2.026984in}}{\pgfqpoint{1.381753in}{2.021160in}}%
\pgfpathcurveto{\pgfqpoint{1.387577in}{2.015336in}}{\pgfqpoint{1.395477in}{2.012064in}}{\pgfqpoint{1.403713in}{2.012064in}}%
\pgfpathclose%
\pgfusepath{stroke,fill}%
\end{pgfscope}%
\begin{pgfscope}%
\pgfpathrectangle{\pgfqpoint{0.100000in}{0.212622in}}{\pgfqpoint{3.696000in}{3.696000in}}%
\pgfusepath{clip}%
\pgfsetbuttcap%
\pgfsetroundjoin%
\definecolor{currentfill}{rgb}{0.121569,0.466667,0.705882}%
\pgfsetfillcolor{currentfill}%
\pgfsetfillopacity{0.399872}%
\pgfsetlinewidth{1.003750pt}%
\definecolor{currentstroke}{rgb}{0.121569,0.466667,0.705882}%
\pgfsetstrokecolor{currentstroke}%
\pgfsetstrokeopacity{0.399872}%
\pgfsetdash{}{0pt}%
\pgfpathmoveto{\pgfqpoint{2.229220in}{2.302866in}}%
\pgfpathcurveto{\pgfqpoint{2.237456in}{2.302866in}}{\pgfqpoint{2.245356in}{2.306138in}}{\pgfqpoint{2.251180in}{2.311962in}}%
\pgfpathcurveto{\pgfqpoint{2.257004in}{2.317786in}}{\pgfqpoint{2.260276in}{2.325686in}}{\pgfqpoint{2.260276in}{2.333922in}}%
\pgfpathcurveto{\pgfqpoint{2.260276in}{2.342158in}}{\pgfqpoint{2.257004in}{2.350058in}}{\pgfqpoint{2.251180in}{2.355882in}}%
\pgfpathcurveto{\pgfqpoint{2.245356in}{2.361706in}}{\pgfqpoint{2.237456in}{2.364979in}}{\pgfqpoint{2.229220in}{2.364979in}}%
\pgfpathcurveto{\pgfqpoint{2.220983in}{2.364979in}}{\pgfqpoint{2.213083in}{2.361706in}}{\pgfqpoint{2.207259in}{2.355882in}}%
\pgfpathcurveto{\pgfqpoint{2.201435in}{2.350058in}}{\pgfqpoint{2.198163in}{2.342158in}}{\pgfqpoint{2.198163in}{2.333922in}}%
\pgfpathcurveto{\pgfqpoint{2.198163in}{2.325686in}}{\pgfqpoint{2.201435in}{2.317786in}}{\pgfqpoint{2.207259in}{2.311962in}}%
\pgfpathcurveto{\pgfqpoint{2.213083in}{2.306138in}}{\pgfqpoint{2.220983in}{2.302866in}}{\pgfqpoint{2.229220in}{2.302866in}}%
\pgfpathclose%
\pgfusepath{stroke,fill}%
\end{pgfscope}%
\begin{pgfscope}%
\pgfpathrectangle{\pgfqpoint{0.100000in}{0.212622in}}{\pgfqpoint{3.696000in}{3.696000in}}%
\pgfusepath{clip}%
\pgfsetbuttcap%
\pgfsetroundjoin%
\definecolor{currentfill}{rgb}{0.121569,0.466667,0.705882}%
\pgfsetfillcolor{currentfill}%
\pgfsetfillopacity{0.403426}%
\pgfsetlinewidth{1.003750pt}%
\definecolor{currentstroke}{rgb}{0.121569,0.466667,0.705882}%
\pgfsetstrokecolor{currentstroke}%
\pgfsetstrokeopacity{0.403426}%
\pgfsetdash{}{0pt}%
\pgfpathmoveto{\pgfqpoint{1.394484in}{1.998935in}}%
\pgfpathcurveto{\pgfqpoint{1.402721in}{1.998935in}}{\pgfqpoint{1.410621in}{2.002207in}}{\pgfqpoint{1.416445in}{2.008031in}}%
\pgfpathcurveto{\pgfqpoint{1.422268in}{2.013855in}}{\pgfqpoint{1.425541in}{2.021755in}}{\pgfqpoint{1.425541in}{2.029991in}}%
\pgfpathcurveto{\pgfqpoint{1.425541in}{2.038228in}}{\pgfqpoint{1.422268in}{2.046128in}}{\pgfqpoint{1.416445in}{2.051952in}}%
\pgfpathcurveto{\pgfqpoint{1.410621in}{2.057775in}}{\pgfqpoint{1.402721in}{2.061048in}}{\pgfqpoint{1.394484in}{2.061048in}}%
\pgfpathcurveto{\pgfqpoint{1.386248in}{2.061048in}}{\pgfqpoint{1.378348in}{2.057775in}}{\pgfqpoint{1.372524in}{2.051952in}}%
\pgfpathcurveto{\pgfqpoint{1.366700in}{2.046128in}}{\pgfqpoint{1.363428in}{2.038228in}}{\pgfqpoint{1.363428in}{2.029991in}}%
\pgfpathcurveto{\pgfqpoint{1.363428in}{2.021755in}}{\pgfqpoint{1.366700in}{2.013855in}}{\pgfqpoint{1.372524in}{2.008031in}}%
\pgfpathcurveto{\pgfqpoint{1.378348in}{2.002207in}}{\pgfqpoint{1.386248in}{1.998935in}}{\pgfqpoint{1.394484in}{1.998935in}}%
\pgfpathclose%
\pgfusepath{stroke,fill}%
\end{pgfscope}%
\begin{pgfscope}%
\pgfpathrectangle{\pgfqpoint{0.100000in}{0.212622in}}{\pgfqpoint{3.696000in}{3.696000in}}%
\pgfusepath{clip}%
\pgfsetbuttcap%
\pgfsetroundjoin%
\definecolor{currentfill}{rgb}{0.121569,0.466667,0.705882}%
\pgfsetfillcolor{currentfill}%
\pgfsetfillopacity{0.404727}%
\pgfsetlinewidth{1.003750pt}%
\definecolor{currentstroke}{rgb}{0.121569,0.466667,0.705882}%
\pgfsetstrokecolor{currentstroke}%
\pgfsetstrokeopacity{0.404727}%
\pgfsetdash{}{0pt}%
\pgfpathmoveto{\pgfqpoint{2.254208in}{2.309446in}}%
\pgfpathcurveto{\pgfqpoint{2.262444in}{2.309446in}}{\pgfqpoint{2.270345in}{2.312719in}}{\pgfqpoint{2.276168in}{2.318543in}}%
\pgfpathcurveto{\pgfqpoint{2.281992in}{2.324366in}}{\pgfqpoint{2.285265in}{2.332267in}}{\pgfqpoint{2.285265in}{2.340503in}}%
\pgfpathcurveto{\pgfqpoint{2.285265in}{2.348739in}}{\pgfqpoint{2.281992in}{2.356639in}}{\pgfqpoint{2.276168in}{2.362463in}}%
\pgfpathcurveto{\pgfqpoint{2.270345in}{2.368287in}}{\pgfqpoint{2.262444in}{2.371559in}}{\pgfqpoint{2.254208in}{2.371559in}}%
\pgfpathcurveto{\pgfqpoint{2.245972in}{2.371559in}}{\pgfqpoint{2.238072in}{2.368287in}}{\pgfqpoint{2.232248in}{2.362463in}}%
\pgfpathcurveto{\pgfqpoint{2.226424in}{2.356639in}}{\pgfqpoint{2.223152in}{2.348739in}}{\pgfqpoint{2.223152in}{2.340503in}}%
\pgfpathcurveto{\pgfqpoint{2.223152in}{2.332267in}}{\pgfqpoint{2.226424in}{2.324366in}}{\pgfqpoint{2.232248in}{2.318543in}}%
\pgfpathcurveto{\pgfqpoint{2.238072in}{2.312719in}}{\pgfqpoint{2.245972in}{2.309446in}}{\pgfqpoint{2.254208in}{2.309446in}}%
\pgfpathclose%
\pgfusepath{stroke,fill}%
\end{pgfscope}%
\begin{pgfscope}%
\pgfpathrectangle{\pgfqpoint{0.100000in}{0.212622in}}{\pgfqpoint{3.696000in}{3.696000in}}%
\pgfusepath{clip}%
\pgfsetbuttcap%
\pgfsetroundjoin%
\definecolor{currentfill}{rgb}{0.121569,0.466667,0.705882}%
\pgfsetfillcolor{currentfill}%
\pgfsetfillopacity{0.409687}%
\pgfsetlinewidth{1.003750pt}%
\definecolor{currentstroke}{rgb}{0.121569,0.466667,0.705882}%
\pgfsetstrokecolor{currentstroke}%
\pgfsetstrokeopacity{0.409687}%
\pgfsetdash{}{0pt}%
\pgfpathmoveto{\pgfqpoint{2.284698in}{2.298577in}}%
\pgfpathcurveto{\pgfqpoint{2.292934in}{2.298577in}}{\pgfqpoint{2.300834in}{2.301850in}}{\pgfqpoint{2.306658in}{2.307674in}}%
\pgfpathcurveto{\pgfqpoint{2.312482in}{2.313498in}}{\pgfqpoint{2.315754in}{2.321398in}}{\pgfqpoint{2.315754in}{2.329634in}}%
\pgfpathcurveto{\pgfqpoint{2.315754in}{2.337870in}}{\pgfqpoint{2.312482in}{2.345770in}}{\pgfqpoint{2.306658in}{2.351594in}}%
\pgfpathcurveto{\pgfqpoint{2.300834in}{2.357418in}}{\pgfqpoint{2.292934in}{2.360690in}}{\pgfqpoint{2.284698in}{2.360690in}}%
\pgfpathcurveto{\pgfqpoint{2.276462in}{2.360690in}}{\pgfqpoint{2.268562in}{2.357418in}}{\pgfqpoint{2.262738in}{2.351594in}}%
\pgfpathcurveto{\pgfqpoint{2.256914in}{2.345770in}}{\pgfqpoint{2.253641in}{2.337870in}}{\pgfqpoint{2.253641in}{2.329634in}}%
\pgfpathcurveto{\pgfqpoint{2.253641in}{2.321398in}}{\pgfqpoint{2.256914in}{2.313498in}}{\pgfqpoint{2.262738in}{2.307674in}}%
\pgfpathcurveto{\pgfqpoint{2.268562in}{2.301850in}}{\pgfqpoint{2.276462in}{2.298577in}}{\pgfqpoint{2.284698in}{2.298577in}}%
\pgfpathclose%
\pgfusepath{stroke,fill}%
\end{pgfscope}%
\begin{pgfscope}%
\pgfpathrectangle{\pgfqpoint{0.100000in}{0.212622in}}{\pgfqpoint{3.696000in}{3.696000in}}%
\pgfusepath{clip}%
\pgfsetbuttcap%
\pgfsetroundjoin%
\definecolor{currentfill}{rgb}{0.121569,0.466667,0.705882}%
\pgfsetfillcolor{currentfill}%
\pgfsetfillopacity{0.410357}%
\pgfsetlinewidth{1.003750pt}%
\definecolor{currentstroke}{rgb}{0.121569,0.466667,0.705882}%
\pgfsetstrokecolor{currentstroke}%
\pgfsetstrokeopacity{0.410357}%
\pgfsetdash{}{0pt}%
\pgfpathmoveto{\pgfqpoint{1.379277in}{1.975608in}}%
\pgfpathcurveto{\pgfqpoint{1.387514in}{1.975608in}}{\pgfqpoint{1.395414in}{1.978881in}}{\pgfqpoint{1.401238in}{1.984705in}}%
\pgfpathcurveto{\pgfqpoint{1.407062in}{1.990528in}}{\pgfqpoint{1.410334in}{1.998429in}}{\pgfqpoint{1.410334in}{2.006665in}}%
\pgfpathcurveto{\pgfqpoint{1.410334in}{2.014901in}}{\pgfqpoint{1.407062in}{2.022801in}}{\pgfqpoint{1.401238in}{2.028625in}}%
\pgfpathcurveto{\pgfqpoint{1.395414in}{2.034449in}}{\pgfqpoint{1.387514in}{2.037721in}}{\pgfqpoint{1.379277in}{2.037721in}}%
\pgfpathcurveto{\pgfqpoint{1.371041in}{2.037721in}}{\pgfqpoint{1.363141in}{2.034449in}}{\pgfqpoint{1.357317in}{2.028625in}}%
\pgfpathcurveto{\pgfqpoint{1.351493in}{2.022801in}}{\pgfqpoint{1.348221in}{2.014901in}}{\pgfqpoint{1.348221in}{2.006665in}}%
\pgfpathcurveto{\pgfqpoint{1.348221in}{1.998429in}}{\pgfqpoint{1.351493in}{1.990528in}}{\pgfqpoint{1.357317in}{1.984705in}}%
\pgfpathcurveto{\pgfqpoint{1.363141in}{1.978881in}}{\pgfqpoint{1.371041in}{1.975608in}}{\pgfqpoint{1.379277in}{1.975608in}}%
\pgfpathclose%
\pgfusepath{stroke,fill}%
\end{pgfscope}%
\begin{pgfscope}%
\pgfpathrectangle{\pgfqpoint{0.100000in}{0.212622in}}{\pgfqpoint{3.696000in}{3.696000in}}%
\pgfusepath{clip}%
\pgfsetbuttcap%
\pgfsetroundjoin%
\definecolor{currentfill}{rgb}{0.121569,0.466667,0.705882}%
\pgfsetfillcolor{currentfill}%
\pgfsetfillopacity{0.416007}%
\pgfsetlinewidth{1.003750pt}%
\definecolor{currentstroke}{rgb}{0.121569,0.466667,0.705882}%
\pgfsetstrokecolor{currentstroke}%
\pgfsetstrokeopacity{0.416007}%
\pgfsetdash{}{0pt}%
\pgfpathmoveto{\pgfqpoint{2.322059in}{2.311702in}}%
\pgfpathcurveto{\pgfqpoint{2.330295in}{2.311702in}}{\pgfqpoint{2.338195in}{2.314975in}}{\pgfqpoint{2.344019in}{2.320799in}}%
\pgfpathcurveto{\pgfqpoint{2.349843in}{2.326623in}}{\pgfqpoint{2.353115in}{2.334523in}}{\pgfqpoint{2.353115in}{2.342759in}}%
\pgfpathcurveto{\pgfqpoint{2.353115in}{2.350995in}}{\pgfqpoint{2.349843in}{2.358895in}}{\pgfqpoint{2.344019in}{2.364719in}}%
\pgfpathcurveto{\pgfqpoint{2.338195in}{2.370543in}}{\pgfqpoint{2.330295in}{2.373815in}}{\pgfqpoint{2.322059in}{2.373815in}}%
\pgfpathcurveto{\pgfqpoint{2.313822in}{2.373815in}}{\pgfqpoint{2.305922in}{2.370543in}}{\pgfqpoint{2.300098in}{2.364719in}}%
\pgfpathcurveto{\pgfqpoint{2.294274in}{2.358895in}}{\pgfqpoint{2.291002in}{2.350995in}}{\pgfqpoint{2.291002in}{2.342759in}}%
\pgfpathcurveto{\pgfqpoint{2.291002in}{2.334523in}}{\pgfqpoint{2.294274in}{2.326623in}}{\pgfqpoint{2.300098in}{2.320799in}}%
\pgfpathcurveto{\pgfqpoint{2.305922in}{2.314975in}}{\pgfqpoint{2.313822in}{2.311702in}}{\pgfqpoint{2.322059in}{2.311702in}}%
\pgfpathclose%
\pgfusepath{stroke,fill}%
\end{pgfscope}%
\begin{pgfscope}%
\pgfpathrectangle{\pgfqpoint{0.100000in}{0.212622in}}{\pgfqpoint{3.696000in}{3.696000in}}%
\pgfusepath{clip}%
\pgfsetbuttcap%
\pgfsetroundjoin%
\definecolor{currentfill}{rgb}{0.121569,0.466667,0.705882}%
\pgfsetfillcolor{currentfill}%
\pgfsetfillopacity{0.418570}%
\pgfsetlinewidth{1.003750pt}%
\definecolor{currentstroke}{rgb}{0.121569,0.466667,0.705882}%
\pgfsetstrokecolor{currentstroke}%
\pgfsetstrokeopacity{0.418570}%
\pgfsetdash{}{0pt}%
\pgfpathmoveto{\pgfqpoint{2.339211in}{2.301311in}}%
\pgfpathcurveto{\pgfqpoint{2.347447in}{2.301311in}}{\pgfqpoint{2.355347in}{2.304583in}}{\pgfqpoint{2.361171in}{2.310407in}}%
\pgfpathcurveto{\pgfqpoint{2.366995in}{2.316231in}}{\pgfqpoint{2.370267in}{2.324131in}}{\pgfqpoint{2.370267in}{2.332368in}}%
\pgfpathcurveto{\pgfqpoint{2.370267in}{2.340604in}}{\pgfqpoint{2.366995in}{2.348504in}}{\pgfqpoint{2.361171in}{2.354328in}}%
\pgfpathcurveto{\pgfqpoint{2.355347in}{2.360152in}}{\pgfqpoint{2.347447in}{2.363424in}}{\pgfqpoint{2.339211in}{2.363424in}}%
\pgfpathcurveto{\pgfqpoint{2.330974in}{2.363424in}}{\pgfqpoint{2.323074in}{2.360152in}}{\pgfqpoint{2.317250in}{2.354328in}}%
\pgfpathcurveto{\pgfqpoint{2.311426in}{2.348504in}}{\pgfqpoint{2.308154in}{2.340604in}}{\pgfqpoint{2.308154in}{2.332368in}}%
\pgfpathcurveto{\pgfqpoint{2.308154in}{2.324131in}}{\pgfqpoint{2.311426in}{2.316231in}}{\pgfqpoint{2.317250in}{2.310407in}}%
\pgfpathcurveto{\pgfqpoint{2.323074in}{2.304583in}}{\pgfqpoint{2.330974in}{2.301311in}}{\pgfqpoint{2.339211in}{2.301311in}}%
\pgfpathclose%
\pgfusepath{stroke,fill}%
\end{pgfscope}%
\begin{pgfscope}%
\pgfpathrectangle{\pgfqpoint{0.100000in}{0.212622in}}{\pgfqpoint{3.696000in}{3.696000in}}%
\pgfusepath{clip}%
\pgfsetbuttcap%
\pgfsetroundjoin%
\definecolor{currentfill}{rgb}{0.121569,0.466667,0.705882}%
\pgfsetfillcolor{currentfill}%
\pgfsetfillopacity{0.420342}%
\pgfsetlinewidth{1.003750pt}%
\definecolor{currentstroke}{rgb}{0.121569,0.466667,0.705882}%
\pgfsetstrokecolor{currentstroke}%
\pgfsetstrokeopacity{0.420342}%
\pgfsetdash{}{0pt}%
\pgfpathmoveto{\pgfqpoint{2.350630in}{2.304750in}}%
\pgfpathcurveto{\pgfqpoint{2.358866in}{2.304750in}}{\pgfqpoint{2.366766in}{2.308023in}}{\pgfqpoint{2.372590in}{2.313847in}}%
\pgfpathcurveto{\pgfqpoint{2.378414in}{2.319671in}}{\pgfqpoint{2.381687in}{2.327571in}}{\pgfqpoint{2.381687in}{2.335807in}}%
\pgfpathcurveto{\pgfqpoint{2.381687in}{2.344043in}}{\pgfqpoint{2.378414in}{2.351943in}}{\pgfqpoint{2.372590in}{2.357767in}}%
\pgfpathcurveto{\pgfqpoint{2.366766in}{2.363591in}}{\pgfqpoint{2.358866in}{2.366863in}}{\pgfqpoint{2.350630in}{2.366863in}}%
\pgfpathcurveto{\pgfqpoint{2.342394in}{2.366863in}}{\pgfqpoint{2.334494in}{2.363591in}}{\pgfqpoint{2.328670in}{2.357767in}}%
\pgfpathcurveto{\pgfqpoint{2.322846in}{2.351943in}}{\pgfqpoint{2.319574in}{2.344043in}}{\pgfqpoint{2.319574in}{2.335807in}}%
\pgfpathcurveto{\pgfqpoint{2.319574in}{2.327571in}}{\pgfqpoint{2.322846in}{2.319671in}}{\pgfqpoint{2.328670in}{2.313847in}}%
\pgfpathcurveto{\pgfqpoint{2.334494in}{2.308023in}}{\pgfqpoint{2.342394in}{2.304750in}}{\pgfqpoint{2.350630in}{2.304750in}}%
\pgfpathclose%
\pgfusepath{stroke,fill}%
\end{pgfscope}%
\begin{pgfscope}%
\pgfpathrectangle{\pgfqpoint{0.100000in}{0.212622in}}{\pgfqpoint{3.696000in}{3.696000in}}%
\pgfusepath{clip}%
\pgfsetbuttcap%
\pgfsetroundjoin%
\definecolor{currentfill}{rgb}{0.121569,0.466667,0.705882}%
\pgfsetfillcolor{currentfill}%
\pgfsetfillopacity{0.421161}%
\pgfsetlinewidth{1.003750pt}%
\definecolor{currentstroke}{rgb}{0.121569,0.466667,0.705882}%
\pgfsetstrokecolor{currentstroke}%
\pgfsetstrokeopacity{0.421161}%
\pgfsetdash{}{0pt}%
\pgfpathmoveto{\pgfqpoint{2.355962in}{2.302247in}}%
\pgfpathcurveto{\pgfqpoint{2.364199in}{2.302247in}}{\pgfqpoint{2.372099in}{2.305519in}}{\pgfqpoint{2.377923in}{2.311343in}}%
\pgfpathcurveto{\pgfqpoint{2.383746in}{2.317167in}}{\pgfqpoint{2.387019in}{2.325067in}}{\pgfqpoint{2.387019in}{2.333303in}}%
\pgfpathcurveto{\pgfqpoint{2.387019in}{2.341539in}}{\pgfqpoint{2.383746in}{2.349439in}}{\pgfqpoint{2.377923in}{2.355263in}}%
\pgfpathcurveto{\pgfqpoint{2.372099in}{2.361087in}}{\pgfqpoint{2.364199in}{2.364360in}}{\pgfqpoint{2.355962in}{2.364360in}}%
\pgfpathcurveto{\pgfqpoint{2.347726in}{2.364360in}}{\pgfqpoint{2.339826in}{2.361087in}}{\pgfqpoint{2.334002in}{2.355263in}}%
\pgfpathcurveto{\pgfqpoint{2.328178in}{2.349439in}}{\pgfqpoint{2.324906in}{2.341539in}}{\pgfqpoint{2.324906in}{2.333303in}}%
\pgfpathcurveto{\pgfqpoint{2.324906in}{2.325067in}}{\pgfqpoint{2.328178in}{2.317167in}}{\pgfqpoint{2.334002in}{2.311343in}}%
\pgfpathcurveto{\pgfqpoint{2.339826in}{2.305519in}}{\pgfqpoint{2.347726in}{2.302247in}}{\pgfqpoint{2.355962in}{2.302247in}}%
\pgfpathclose%
\pgfusepath{stroke,fill}%
\end{pgfscope}%
\begin{pgfscope}%
\pgfpathrectangle{\pgfqpoint{0.100000in}{0.212622in}}{\pgfqpoint{3.696000in}{3.696000in}}%
\pgfusepath{clip}%
\pgfsetbuttcap%
\pgfsetroundjoin%
\definecolor{currentfill}{rgb}{0.121569,0.466667,0.705882}%
\pgfsetfillcolor{currentfill}%
\pgfsetfillopacity{0.421720}%
\pgfsetlinewidth{1.003750pt}%
\definecolor{currentstroke}{rgb}{0.121569,0.466667,0.705882}%
\pgfsetstrokecolor{currentstroke}%
\pgfsetstrokeopacity{0.421720}%
\pgfsetdash{}{0pt}%
\pgfpathmoveto{\pgfqpoint{2.359352in}{2.303130in}}%
\pgfpathcurveto{\pgfqpoint{2.367589in}{2.303130in}}{\pgfqpoint{2.375489in}{2.306402in}}{\pgfqpoint{2.381313in}{2.312226in}}%
\pgfpathcurveto{\pgfqpoint{2.387136in}{2.318050in}}{\pgfqpoint{2.390409in}{2.325950in}}{\pgfqpoint{2.390409in}{2.334187in}}%
\pgfpathcurveto{\pgfqpoint{2.390409in}{2.342423in}}{\pgfqpoint{2.387136in}{2.350323in}}{\pgfqpoint{2.381313in}{2.356147in}}%
\pgfpathcurveto{\pgfqpoint{2.375489in}{2.361971in}}{\pgfqpoint{2.367589in}{2.365243in}}{\pgfqpoint{2.359352in}{2.365243in}}%
\pgfpathcurveto{\pgfqpoint{2.351116in}{2.365243in}}{\pgfqpoint{2.343216in}{2.361971in}}{\pgfqpoint{2.337392in}{2.356147in}}%
\pgfpathcurveto{\pgfqpoint{2.331568in}{2.350323in}}{\pgfqpoint{2.328296in}{2.342423in}}{\pgfqpoint{2.328296in}{2.334187in}}%
\pgfpathcurveto{\pgfqpoint{2.328296in}{2.325950in}}{\pgfqpoint{2.331568in}{2.318050in}}{\pgfqpoint{2.337392in}{2.312226in}}%
\pgfpathcurveto{\pgfqpoint{2.343216in}{2.306402in}}{\pgfqpoint{2.351116in}{2.303130in}}{\pgfqpoint{2.359352in}{2.303130in}}%
\pgfpathclose%
\pgfusepath{stroke,fill}%
\end{pgfscope}%
\begin{pgfscope}%
\pgfpathrectangle{\pgfqpoint{0.100000in}{0.212622in}}{\pgfqpoint{3.696000in}{3.696000in}}%
\pgfusepath{clip}%
\pgfsetbuttcap%
\pgfsetroundjoin%
\definecolor{currentfill}{rgb}{0.121569,0.466667,0.705882}%
\pgfsetfillcolor{currentfill}%
\pgfsetfillopacity{0.422651}%
\pgfsetlinewidth{1.003750pt}%
\definecolor{currentstroke}{rgb}{0.121569,0.466667,0.705882}%
\pgfsetstrokecolor{currentstroke}%
\pgfsetstrokeopacity{0.422651}%
\pgfsetdash{}{0pt}%
\pgfpathmoveto{\pgfqpoint{2.364209in}{2.300810in}}%
\pgfpathcurveto{\pgfqpoint{2.372445in}{2.300810in}}{\pgfqpoint{2.380345in}{2.304082in}}{\pgfqpoint{2.386169in}{2.309906in}}%
\pgfpathcurveto{\pgfqpoint{2.391993in}{2.315730in}}{\pgfqpoint{2.395265in}{2.323630in}}{\pgfqpoint{2.395265in}{2.331866in}}%
\pgfpathcurveto{\pgfqpoint{2.395265in}{2.340102in}}{\pgfqpoint{2.391993in}{2.348002in}}{\pgfqpoint{2.386169in}{2.353826in}}%
\pgfpathcurveto{\pgfqpoint{2.380345in}{2.359650in}}{\pgfqpoint{2.372445in}{2.362923in}}{\pgfqpoint{2.364209in}{2.362923in}}%
\pgfpathcurveto{\pgfqpoint{2.355972in}{2.362923in}}{\pgfqpoint{2.348072in}{2.359650in}}{\pgfqpoint{2.342248in}{2.353826in}}%
\pgfpathcurveto{\pgfqpoint{2.336425in}{2.348002in}}{\pgfqpoint{2.333152in}{2.340102in}}{\pgfqpoint{2.333152in}{2.331866in}}%
\pgfpathcurveto{\pgfqpoint{2.333152in}{2.323630in}}{\pgfqpoint{2.336425in}{2.315730in}}{\pgfqpoint{2.342248in}{2.309906in}}%
\pgfpathcurveto{\pgfqpoint{2.348072in}{2.304082in}}{\pgfqpoint{2.355972in}{2.300810in}}{\pgfqpoint{2.364209in}{2.300810in}}%
\pgfpathclose%
\pgfusepath{stroke,fill}%
\end{pgfscope}%
\begin{pgfscope}%
\pgfpathrectangle{\pgfqpoint{0.100000in}{0.212622in}}{\pgfqpoint{3.696000in}{3.696000in}}%
\pgfusepath{clip}%
\pgfsetbuttcap%
\pgfsetroundjoin%
\definecolor{currentfill}{rgb}{0.121569,0.466667,0.705882}%
\pgfsetfillcolor{currentfill}%
\pgfsetfillopacity{0.422836}%
\pgfsetlinewidth{1.003750pt}%
\definecolor{currentstroke}{rgb}{0.121569,0.466667,0.705882}%
\pgfsetstrokecolor{currentstroke}%
\pgfsetstrokeopacity{0.422836}%
\pgfsetdash{}{0pt}%
\pgfpathmoveto{\pgfqpoint{1.341704in}{1.941139in}}%
\pgfpathcurveto{\pgfqpoint{1.349940in}{1.941139in}}{\pgfqpoint{1.357840in}{1.944411in}}{\pgfqpoint{1.363664in}{1.950235in}}%
\pgfpathcurveto{\pgfqpoint{1.369488in}{1.956059in}}{\pgfqpoint{1.372760in}{1.963959in}}{\pgfqpoint{1.372760in}{1.972195in}}%
\pgfpathcurveto{\pgfqpoint{1.372760in}{1.980431in}}{\pgfqpoint{1.369488in}{1.988331in}}{\pgfqpoint{1.363664in}{1.994155in}}%
\pgfpathcurveto{\pgfqpoint{1.357840in}{1.999979in}}{\pgfqpoint{1.349940in}{2.003252in}}{\pgfqpoint{1.341704in}{2.003252in}}%
\pgfpathcurveto{\pgfqpoint{1.333467in}{2.003252in}}{\pgfqpoint{1.325567in}{1.999979in}}{\pgfqpoint{1.319744in}{1.994155in}}%
\pgfpathcurveto{\pgfqpoint{1.313920in}{1.988331in}}{\pgfqpoint{1.310647in}{1.980431in}}{\pgfqpoint{1.310647in}{1.972195in}}%
\pgfpathcurveto{\pgfqpoint{1.310647in}{1.963959in}}{\pgfqpoint{1.313920in}{1.956059in}}{\pgfqpoint{1.319744in}{1.950235in}}%
\pgfpathcurveto{\pgfqpoint{1.325567in}{1.944411in}}{\pgfqpoint{1.333467in}{1.941139in}}{\pgfqpoint{1.341704in}{1.941139in}}%
\pgfpathclose%
\pgfusepath{stroke,fill}%
\end{pgfscope}%
\begin{pgfscope}%
\pgfpathrectangle{\pgfqpoint{0.100000in}{0.212622in}}{\pgfqpoint{3.696000in}{3.696000in}}%
\pgfusepath{clip}%
\pgfsetbuttcap%
\pgfsetroundjoin%
\definecolor{currentfill}{rgb}{0.121569,0.466667,0.705882}%
\pgfsetfillcolor{currentfill}%
\pgfsetfillopacity{0.424657}%
\pgfsetlinewidth{1.003750pt}%
\definecolor{currentstroke}{rgb}{0.121569,0.466667,0.705882}%
\pgfsetstrokecolor{currentstroke}%
\pgfsetstrokeopacity{0.424657}%
\pgfsetdash{}{0pt}%
\pgfpathmoveto{\pgfqpoint{2.374386in}{2.304299in}}%
\pgfpathcurveto{\pgfqpoint{2.382622in}{2.304299in}}{\pgfqpoint{2.390522in}{2.307572in}}{\pgfqpoint{2.396346in}{2.313396in}}%
\pgfpathcurveto{\pgfqpoint{2.402170in}{2.319219in}}{\pgfqpoint{2.405443in}{2.327119in}}{\pgfqpoint{2.405443in}{2.335356in}}%
\pgfpathcurveto{\pgfqpoint{2.405443in}{2.343592in}}{\pgfqpoint{2.402170in}{2.351492in}}{\pgfqpoint{2.396346in}{2.357316in}}%
\pgfpathcurveto{\pgfqpoint{2.390522in}{2.363140in}}{\pgfqpoint{2.382622in}{2.366412in}}{\pgfqpoint{2.374386in}{2.366412in}}%
\pgfpathcurveto{\pgfqpoint{2.366150in}{2.366412in}}{\pgfqpoint{2.358250in}{2.363140in}}{\pgfqpoint{2.352426in}{2.357316in}}%
\pgfpathcurveto{\pgfqpoint{2.346602in}{2.351492in}}{\pgfqpoint{2.343330in}{2.343592in}}{\pgfqpoint{2.343330in}{2.335356in}}%
\pgfpathcurveto{\pgfqpoint{2.343330in}{2.327119in}}{\pgfqpoint{2.346602in}{2.319219in}}{\pgfqpoint{2.352426in}{2.313396in}}%
\pgfpathcurveto{\pgfqpoint{2.358250in}{2.307572in}}{\pgfqpoint{2.366150in}{2.304299in}}{\pgfqpoint{2.374386in}{2.304299in}}%
\pgfpathclose%
\pgfusepath{stroke,fill}%
\end{pgfscope}%
\begin{pgfscope}%
\pgfpathrectangle{\pgfqpoint{0.100000in}{0.212622in}}{\pgfqpoint{3.696000in}{3.696000in}}%
\pgfusepath{clip}%
\pgfsetbuttcap%
\pgfsetroundjoin%
\definecolor{currentfill}{rgb}{0.121569,0.466667,0.705882}%
\pgfsetfillcolor{currentfill}%
\pgfsetfillopacity{0.426965}%
\pgfsetlinewidth{1.003750pt}%
\definecolor{currentstroke}{rgb}{0.121569,0.466667,0.705882}%
\pgfsetstrokecolor{currentstroke}%
\pgfsetstrokeopacity{0.426965}%
\pgfsetdash{}{0pt}%
\pgfpathmoveto{\pgfqpoint{2.385007in}{2.301654in}}%
\pgfpathcurveto{\pgfqpoint{2.393243in}{2.301654in}}{\pgfqpoint{2.401143in}{2.304926in}}{\pgfqpoint{2.406967in}{2.310750in}}%
\pgfpathcurveto{\pgfqpoint{2.412791in}{2.316574in}}{\pgfqpoint{2.416064in}{2.324474in}}{\pgfqpoint{2.416064in}{2.332710in}}%
\pgfpathcurveto{\pgfqpoint{2.416064in}{2.340946in}}{\pgfqpoint{2.412791in}{2.348846in}}{\pgfqpoint{2.406967in}{2.354670in}}%
\pgfpathcurveto{\pgfqpoint{2.401143in}{2.360494in}}{\pgfqpoint{2.393243in}{2.363767in}}{\pgfqpoint{2.385007in}{2.363767in}}%
\pgfpathcurveto{\pgfqpoint{2.376771in}{2.363767in}}{\pgfqpoint{2.368871in}{2.360494in}}{\pgfqpoint{2.363047in}{2.354670in}}%
\pgfpathcurveto{\pgfqpoint{2.357223in}{2.348846in}}{\pgfqpoint{2.353951in}{2.340946in}}{\pgfqpoint{2.353951in}{2.332710in}}%
\pgfpathcurveto{\pgfqpoint{2.353951in}{2.324474in}}{\pgfqpoint{2.357223in}{2.316574in}}{\pgfqpoint{2.363047in}{2.310750in}}%
\pgfpathcurveto{\pgfqpoint{2.368871in}{2.304926in}}{\pgfqpoint{2.376771in}{2.301654in}}{\pgfqpoint{2.385007in}{2.301654in}}%
\pgfpathclose%
\pgfusepath{stroke,fill}%
\end{pgfscope}%
\begin{pgfscope}%
\pgfpathrectangle{\pgfqpoint{0.100000in}{0.212622in}}{\pgfqpoint{3.696000in}{3.696000in}}%
\pgfusepath{clip}%
\pgfsetbuttcap%
\pgfsetroundjoin%
\definecolor{currentfill}{rgb}{0.121569,0.466667,0.705882}%
\pgfsetfillcolor{currentfill}%
\pgfsetfillopacity{0.430220}%
\pgfsetlinewidth{1.003750pt}%
\definecolor{currentstroke}{rgb}{0.121569,0.466667,0.705882}%
\pgfsetstrokecolor{currentstroke}%
\pgfsetstrokeopacity{0.430220}%
\pgfsetdash{}{0pt}%
\pgfpathmoveto{\pgfqpoint{2.400930in}{2.303908in}}%
\pgfpathcurveto{\pgfqpoint{2.409166in}{2.303908in}}{\pgfqpoint{2.417066in}{2.307180in}}{\pgfqpoint{2.422890in}{2.313004in}}%
\pgfpathcurveto{\pgfqpoint{2.428714in}{2.318828in}}{\pgfqpoint{2.431987in}{2.326728in}}{\pgfqpoint{2.431987in}{2.334964in}}%
\pgfpathcurveto{\pgfqpoint{2.431987in}{2.343200in}}{\pgfqpoint{2.428714in}{2.351100in}}{\pgfqpoint{2.422890in}{2.356924in}}%
\pgfpathcurveto{\pgfqpoint{2.417066in}{2.362748in}}{\pgfqpoint{2.409166in}{2.366021in}}{\pgfqpoint{2.400930in}{2.366021in}}%
\pgfpathcurveto{\pgfqpoint{2.392694in}{2.366021in}}{\pgfqpoint{2.384794in}{2.362748in}}{\pgfqpoint{2.378970in}{2.356924in}}%
\pgfpathcurveto{\pgfqpoint{2.373146in}{2.351100in}}{\pgfqpoint{2.369874in}{2.343200in}}{\pgfqpoint{2.369874in}{2.334964in}}%
\pgfpathcurveto{\pgfqpoint{2.369874in}{2.326728in}}{\pgfqpoint{2.373146in}{2.318828in}}{\pgfqpoint{2.378970in}{2.313004in}}%
\pgfpathcurveto{\pgfqpoint{2.384794in}{2.307180in}}{\pgfqpoint{2.392694in}{2.303908in}}{\pgfqpoint{2.400930in}{2.303908in}}%
\pgfpathclose%
\pgfusepath{stroke,fill}%
\end{pgfscope}%
\begin{pgfscope}%
\pgfpathrectangle{\pgfqpoint{0.100000in}{0.212622in}}{\pgfqpoint{3.696000in}{3.696000in}}%
\pgfusepath{clip}%
\pgfsetbuttcap%
\pgfsetroundjoin%
\definecolor{currentfill}{rgb}{0.121569,0.466667,0.705882}%
\pgfsetfillcolor{currentfill}%
\pgfsetfillopacity{0.431760}%
\pgfsetlinewidth{1.003750pt}%
\definecolor{currentstroke}{rgb}{0.121569,0.466667,0.705882}%
\pgfsetstrokecolor{currentstroke}%
\pgfsetstrokeopacity{0.431760}%
\pgfsetdash{}{0pt}%
\pgfpathmoveto{\pgfqpoint{2.409444in}{2.302708in}}%
\pgfpathcurveto{\pgfqpoint{2.417680in}{2.302708in}}{\pgfqpoint{2.425580in}{2.305980in}}{\pgfqpoint{2.431404in}{2.311804in}}%
\pgfpathcurveto{\pgfqpoint{2.437228in}{2.317628in}}{\pgfqpoint{2.440500in}{2.325528in}}{\pgfqpoint{2.440500in}{2.333765in}}%
\pgfpathcurveto{\pgfqpoint{2.440500in}{2.342001in}}{\pgfqpoint{2.437228in}{2.349901in}}{\pgfqpoint{2.431404in}{2.355725in}}%
\pgfpathcurveto{\pgfqpoint{2.425580in}{2.361549in}}{\pgfqpoint{2.417680in}{2.364821in}}{\pgfqpoint{2.409444in}{2.364821in}}%
\pgfpathcurveto{\pgfqpoint{2.401207in}{2.364821in}}{\pgfqpoint{2.393307in}{2.361549in}}{\pgfqpoint{2.387484in}{2.355725in}}%
\pgfpathcurveto{\pgfqpoint{2.381660in}{2.349901in}}{\pgfqpoint{2.378387in}{2.342001in}}{\pgfqpoint{2.378387in}{2.333765in}}%
\pgfpathcurveto{\pgfqpoint{2.378387in}{2.325528in}}{\pgfqpoint{2.381660in}{2.317628in}}{\pgfqpoint{2.387484in}{2.311804in}}%
\pgfpathcurveto{\pgfqpoint{2.393307in}{2.305980in}}{\pgfqpoint{2.401207in}{2.302708in}}{\pgfqpoint{2.409444in}{2.302708in}}%
\pgfpathclose%
\pgfusepath{stroke,fill}%
\end{pgfscope}%
\begin{pgfscope}%
\pgfpathrectangle{\pgfqpoint{0.100000in}{0.212622in}}{\pgfqpoint{3.696000in}{3.696000in}}%
\pgfusepath{clip}%
\pgfsetbuttcap%
\pgfsetroundjoin%
\definecolor{currentfill}{rgb}{0.121569,0.466667,0.705882}%
\pgfsetfillcolor{currentfill}%
\pgfsetfillopacity{0.433848}%
\pgfsetlinewidth{1.003750pt}%
\definecolor{currentstroke}{rgb}{0.121569,0.466667,0.705882}%
\pgfsetstrokecolor{currentstroke}%
\pgfsetstrokeopacity{0.433848}%
\pgfsetdash{}{0pt}%
\pgfpathmoveto{\pgfqpoint{1.312856in}{1.893710in}}%
\pgfpathcurveto{\pgfqpoint{1.321092in}{1.893710in}}{\pgfqpoint{1.328992in}{1.896982in}}{\pgfqpoint{1.334816in}{1.902806in}}%
\pgfpathcurveto{\pgfqpoint{1.340640in}{1.908630in}}{\pgfqpoint{1.343912in}{1.916530in}}{\pgfqpoint{1.343912in}{1.924766in}}%
\pgfpathcurveto{\pgfqpoint{1.343912in}{1.933003in}}{\pgfqpoint{1.340640in}{1.940903in}}{\pgfqpoint{1.334816in}{1.946727in}}%
\pgfpathcurveto{\pgfqpoint{1.328992in}{1.952551in}}{\pgfqpoint{1.321092in}{1.955823in}}{\pgfqpoint{1.312856in}{1.955823in}}%
\pgfpathcurveto{\pgfqpoint{1.304619in}{1.955823in}}{\pgfqpoint{1.296719in}{1.952551in}}{\pgfqpoint{1.290895in}{1.946727in}}%
\pgfpathcurveto{\pgfqpoint{1.285071in}{1.940903in}}{\pgfqpoint{1.281799in}{1.933003in}}{\pgfqpoint{1.281799in}{1.924766in}}%
\pgfpathcurveto{\pgfqpoint{1.281799in}{1.916530in}}{\pgfqpoint{1.285071in}{1.908630in}}{\pgfqpoint{1.290895in}{1.902806in}}%
\pgfpathcurveto{\pgfqpoint{1.296719in}{1.896982in}}{\pgfqpoint{1.304619in}{1.893710in}}{\pgfqpoint{1.312856in}{1.893710in}}%
\pgfpathclose%
\pgfusepath{stroke,fill}%
\end{pgfscope}%
\begin{pgfscope}%
\pgfpathrectangle{\pgfqpoint{0.100000in}{0.212622in}}{\pgfqpoint{3.696000in}{3.696000in}}%
\pgfusepath{clip}%
\pgfsetbuttcap%
\pgfsetroundjoin%
\definecolor{currentfill}{rgb}{0.121569,0.466667,0.705882}%
\pgfsetfillcolor{currentfill}%
\pgfsetfillopacity{0.434806}%
\pgfsetlinewidth{1.003750pt}%
\definecolor{currentstroke}{rgb}{0.121569,0.466667,0.705882}%
\pgfsetstrokecolor{currentstroke}%
\pgfsetstrokeopacity{0.434806}%
\pgfsetdash{}{0pt}%
\pgfpathmoveto{\pgfqpoint{2.423509in}{2.303741in}}%
\pgfpathcurveto{\pgfqpoint{2.431745in}{2.303741in}}{\pgfqpoint{2.439645in}{2.307014in}}{\pgfqpoint{2.445469in}{2.312837in}}%
\pgfpathcurveto{\pgfqpoint{2.451293in}{2.318661in}}{\pgfqpoint{2.454565in}{2.326561in}}{\pgfqpoint{2.454565in}{2.334798in}}%
\pgfpathcurveto{\pgfqpoint{2.454565in}{2.343034in}}{\pgfqpoint{2.451293in}{2.350934in}}{\pgfqpoint{2.445469in}{2.356758in}}%
\pgfpathcurveto{\pgfqpoint{2.439645in}{2.362582in}}{\pgfqpoint{2.431745in}{2.365854in}}{\pgfqpoint{2.423509in}{2.365854in}}%
\pgfpathcurveto{\pgfqpoint{2.415273in}{2.365854in}}{\pgfqpoint{2.407372in}{2.362582in}}{\pgfqpoint{2.401549in}{2.356758in}}%
\pgfpathcurveto{\pgfqpoint{2.395725in}{2.350934in}}{\pgfqpoint{2.392452in}{2.343034in}}{\pgfqpoint{2.392452in}{2.334798in}}%
\pgfpathcurveto{\pgfqpoint{2.392452in}{2.326561in}}{\pgfqpoint{2.395725in}{2.318661in}}{\pgfqpoint{2.401549in}{2.312837in}}%
\pgfpathcurveto{\pgfqpoint{2.407372in}{2.307014in}}{\pgfqpoint{2.415273in}{2.303741in}}{\pgfqpoint{2.423509in}{2.303741in}}%
\pgfpathclose%
\pgfusepath{stroke,fill}%
\end{pgfscope}%
\begin{pgfscope}%
\pgfpathrectangle{\pgfqpoint{0.100000in}{0.212622in}}{\pgfqpoint{3.696000in}{3.696000in}}%
\pgfusepath{clip}%
\pgfsetbuttcap%
\pgfsetroundjoin%
\definecolor{currentfill}{rgb}{0.121569,0.466667,0.705882}%
\pgfsetfillcolor{currentfill}%
\pgfsetfillopacity{0.438373}%
\pgfsetlinewidth{1.003750pt}%
\definecolor{currentstroke}{rgb}{0.121569,0.466667,0.705882}%
\pgfsetstrokecolor{currentstroke}%
\pgfsetstrokeopacity{0.438373}%
\pgfsetdash{}{0pt}%
\pgfpathmoveto{\pgfqpoint{2.438650in}{2.299418in}}%
\pgfpathcurveto{\pgfqpoint{2.446886in}{2.299418in}}{\pgfqpoint{2.454786in}{2.302690in}}{\pgfqpoint{2.460610in}{2.308514in}}%
\pgfpathcurveto{\pgfqpoint{2.466434in}{2.314338in}}{\pgfqpoint{2.469707in}{2.322238in}}{\pgfqpoint{2.469707in}{2.330474in}}%
\pgfpathcurveto{\pgfqpoint{2.469707in}{2.338711in}}{\pgfqpoint{2.466434in}{2.346611in}}{\pgfqpoint{2.460610in}{2.352435in}}%
\pgfpathcurveto{\pgfqpoint{2.454786in}{2.358259in}}{\pgfqpoint{2.446886in}{2.361531in}}{\pgfqpoint{2.438650in}{2.361531in}}%
\pgfpathcurveto{\pgfqpoint{2.430414in}{2.361531in}}{\pgfqpoint{2.422514in}{2.358259in}}{\pgfqpoint{2.416690in}{2.352435in}}%
\pgfpathcurveto{\pgfqpoint{2.410866in}{2.346611in}}{\pgfqpoint{2.407594in}{2.338711in}}{\pgfqpoint{2.407594in}{2.330474in}}%
\pgfpathcurveto{\pgfqpoint{2.407594in}{2.322238in}}{\pgfqpoint{2.410866in}{2.314338in}}{\pgfqpoint{2.416690in}{2.308514in}}%
\pgfpathcurveto{\pgfqpoint{2.422514in}{2.302690in}}{\pgfqpoint{2.430414in}{2.299418in}}{\pgfqpoint{2.438650in}{2.299418in}}%
\pgfpathclose%
\pgfusepath{stroke,fill}%
\end{pgfscope}%
\begin{pgfscope}%
\pgfpathrectangle{\pgfqpoint{0.100000in}{0.212622in}}{\pgfqpoint{3.696000in}{3.696000in}}%
\pgfusepath{clip}%
\pgfsetbuttcap%
\pgfsetroundjoin%
\definecolor{currentfill}{rgb}{0.121569,0.466667,0.705882}%
\pgfsetfillcolor{currentfill}%
\pgfsetfillopacity{0.442987}%
\pgfsetlinewidth{1.003750pt}%
\definecolor{currentstroke}{rgb}{0.121569,0.466667,0.705882}%
\pgfsetstrokecolor{currentstroke}%
\pgfsetstrokeopacity{0.442987}%
\pgfsetdash{}{0pt}%
\pgfpathmoveto{\pgfqpoint{2.463517in}{2.301719in}}%
\pgfpathcurveto{\pgfqpoint{2.471753in}{2.301719in}}{\pgfqpoint{2.479653in}{2.304992in}}{\pgfqpoint{2.485477in}{2.310816in}}%
\pgfpathcurveto{\pgfqpoint{2.491301in}{2.316640in}}{\pgfqpoint{2.494573in}{2.324540in}}{\pgfqpoint{2.494573in}{2.332776in}}%
\pgfpathcurveto{\pgfqpoint{2.494573in}{2.341012in}}{\pgfqpoint{2.491301in}{2.348912in}}{\pgfqpoint{2.485477in}{2.354736in}}%
\pgfpathcurveto{\pgfqpoint{2.479653in}{2.360560in}}{\pgfqpoint{2.471753in}{2.363832in}}{\pgfqpoint{2.463517in}{2.363832in}}%
\pgfpathcurveto{\pgfqpoint{2.455281in}{2.363832in}}{\pgfqpoint{2.447381in}{2.360560in}}{\pgfqpoint{2.441557in}{2.354736in}}%
\pgfpathcurveto{\pgfqpoint{2.435733in}{2.348912in}}{\pgfqpoint{2.432460in}{2.341012in}}{\pgfqpoint{2.432460in}{2.332776in}}%
\pgfpathcurveto{\pgfqpoint{2.432460in}{2.324540in}}{\pgfqpoint{2.435733in}{2.316640in}}{\pgfqpoint{2.441557in}{2.310816in}}%
\pgfpathcurveto{\pgfqpoint{2.447381in}{2.304992in}}{\pgfqpoint{2.455281in}{2.301719in}}{\pgfqpoint{2.463517in}{2.301719in}}%
\pgfpathclose%
\pgfusepath{stroke,fill}%
\end{pgfscope}%
\begin{pgfscope}%
\pgfpathrectangle{\pgfqpoint{0.100000in}{0.212622in}}{\pgfqpoint{3.696000in}{3.696000in}}%
\pgfusepath{clip}%
\pgfsetbuttcap%
\pgfsetroundjoin%
\definecolor{currentfill}{rgb}{0.121569,0.466667,0.705882}%
\pgfsetfillcolor{currentfill}%
\pgfsetfillopacity{0.444487}%
\pgfsetlinewidth{1.003750pt}%
\definecolor{currentstroke}{rgb}{0.121569,0.466667,0.705882}%
\pgfsetstrokecolor{currentstroke}%
\pgfsetstrokeopacity{0.444487}%
\pgfsetdash{}{0pt}%
\pgfpathmoveto{\pgfqpoint{1.283087in}{1.853200in}}%
\pgfpathcurveto{\pgfqpoint{1.291324in}{1.853200in}}{\pgfqpoint{1.299224in}{1.856472in}}{\pgfqpoint{1.305048in}{1.862296in}}%
\pgfpathcurveto{\pgfqpoint{1.310872in}{1.868120in}}{\pgfqpoint{1.314144in}{1.876020in}}{\pgfqpoint{1.314144in}{1.884256in}}%
\pgfpathcurveto{\pgfqpoint{1.314144in}{1.892493in}}{\pgfqpoint{1.310872in}{1.900393in}}{\pgfqpoint{1.305048in}{1.906216in}}%
\pgfpathcurveto{\pgfqpoint{1.299224in}{1.912040in}}{\pgfqpoint{1.291324in}{1.915313in}}{\pgfqpoint{1.283087in}{1.915313in}}%
\pgfpathcurveto{\pgfqpoint{1.274851in}{1.915313in}}{\pgfqpoint{1.266951in}{1.912040in}}{\pgfqpoint{1.261127in}{1.906216in}}%
\pgfpathcurveto{\pgfqpoint{1.255303in}{1.900393in}}{\pgfqpoint{1.252031in}{1.892493in}}{\pgfqpoint{1.252031in}{1.884256in}}%
\pgfpathcurveto{\pgfqpoint{1.252031in}{1.876020in}}{\pgfqpoint{1.255303in}{1.868120in}}{\pgfqpoint{1.261127in}{1.862296in}}%
\pgfpathcurveto{\pgfqpoint{1.266951in}{1.856472in}}{\pgfqpoint{1.274851in}{1.853200in}}{\pgfqpoint{1.283087in}{1.853200in}}%
\pgfpathclose%
\pgfusepath{stroke,fill}%
\end{pgfscope}%
\begin{pgfscope}%
\pgfpathrectangle{\pgfqpoint{0.100000in}{0.212622in}}{\pgfqpoint{3.696000in}{3.696000in}}%
\pgfusepath{clip}%
\pgfsetbuttcap%
\pgfsetroundjoin%
\definecolor{currentfill}{rgb}{0.121569,0.466667,0.705882}%
\pgfsetfillcolor{currentfill}%
\pgfsetfillopacity{0.445508}%
\pgfsetlinewidth{1.003750pt}%
\definecolor{currentstroke}{rgb}{0.121569,0.466667,0.705882}%
\pgfsetstrokecolor{currentstroke}%
\pgfsetstrokeopacity{0.445508}%
\pgfsetdash{}{0pt}%
\pgfpathmoveto{\pgfqpoint{2.476034in}{2.299393in}}%
\pgfpathcurveto{\pgfqpoint{2.484271in}{2.299393in}}{\pgfqpoint{2.492171in}{2.302665in}}{\pgfqpoint{2.497995in}{2.308489in}}%
\pgfpathcurveto{\pgfqpoint{2.503819in}{2.314313in}}{\pgfqpoint{2.507091in}{2.322213in}}{\pgfqpoint{2.507091in}{2.330449in}}%
\pgfpathcurveto{\pgfqpoint{2.507091in}{2.338686in}}{\pgfqpoint{2.503819in}{2.346586in}}{\pgfqpoint{2.497995in}{2.352410in}}%
\pgfpathcurveto{\pgfqpoint{2.492171in}{2.358234in}}{\pgfqpoint{2.484271in}{2.361506in}}{\pgfqpoint{2.476034in}{2.361506in}}%
\pgfpathcurveto{\pgfqpoint{2.467798in}{2.361506in}}{\pgfqpoint{2.459898in}{2.358234in}}{\pgfqpoint{2.454074in}{2.352410in}}%
\pgfpathcurveto{\pgfqpoint{2.448250in}{2.346586in}}{\pgfqpoint{2.444978in}{2.338686in}}{\pgfqpoint{2.444978in}{2.330449in}}%
\pgfpathcurveto{\pgfqpoint{2.444978in}{2.322213in}}{\pgfqpoint{2.448250in}{2.314313in}}{\pgfqpoint{2.454074in}{2.308489in}}%
\pgfpathcurveto{\pgfqpoint{2.459898in}{2.302665in}}{\pgfqpoint{2.467798in}{2.299393in}}{\pgfqpoint{2.476034in}{2.299393in}}%
\pgfpathclose%
\pgfusepath{stroke,fill}%
\end{pgfscope}%
\begin{pgfscope}%
\pgfpathrectangle{\pgfqpoint{0.100000in}{0.212622in}}{\pgfqpoint{3.696000in}{3.696000in}}%
\pgfusepath{clip}%
\pgfsetbuttcap%
\pgfsetroundjoin%
\definecolor{currentfill}{rgb}{0.121569,0.466667,0.705882}%
\pgfsetfillcolor{currentfill}%
\pgfsetfillopacity{0.448953}%
\pgfsetlinewidth{1.003750pt}%
\definecolor{currentstroke}{rgb}{0.121569,0.466667,0.705882}%
\pgfsetstrokecolor{currentstroke}%
\pgfsetstrokeopacity{0.448953}%
\pgfsetdash{}{0pt}%
\pgfpathmoveto{\pgfqpoint{2.493633in}{2.301663in}}%
\pgfpathcurveto{\pgfqpoint{2.501869in}{2.301663in}}{\pgfqpoint{2.509769in}{2.304936in}}{\pgfqpoint{2.515593in}{2.310760in}}%
\pgfpathcurveto{\pgfqpoint{2.521417in}{2.316584in}}{\pgfqpoint{2.524689in}{2.324484in}}{\pgfqpoint{2.524689in}{2.332720in}}%
\pgfpathcurveto{\pgfqpoint{2.524689in}{2.340956in}}{\pgfqpoint{2.521417in}{2.348856in}}{\pgfqpoint{2.515593in}{2.354680in}}%
\pgfpathcurveto{\pgfqpoint{2.509769in}{2.360504in}}{\pgfqpoint{2.501869in}{2.363776in}}{\pgfqpoint{2.493633in}{2.363776in}}%
\pgfpathcurveto{\pgfqpoint{2.485397in}{2.363776in}}{\pgfqpoint{2.477496in}{2.360504in}}{\pgfqpoint{2.471673in}{2.354680in}}%
\pgfpathcurveto{\pgfqpoint{2.465849in}{2.348856in}}{\pgfqpoint{2.462576in}{2.340956in}}{\pgfqpoint{2.462576in}{2.332720in}}%
\pgfpathcurveto{\pgfqpoint{2.462576in}{2.324484in}}{\pgfqpoint{2.465849in}{2.316584in}}{\pgfqpoint{2.471673in}{2.310760in}}%
\pgfpathcurveto{\pgfqpoint{2.477496in}{2.304936in}}{\pgfqpoint{2.485397in}{2.301663in}}{\pgfqpoint{2.493633in}{2.301663in}}%
\pgfpathclose%
\pgfusepath{stroke,fill}%
\end{pgfscope}%
\begin{pgfscope}%
\pgfpathrectangle{\pgfqpoint{0.100000in}{0.212622in}}{\pgfqpoint{3.696000in}{3.696000in}}%
\pgfusepath{clip}%
\pgfsetbuttcap%
\pgfsetroundjoin%
\definecolor{currentfill}{rgb}{0.121569,0.466667,0.705882}%
\pgfsetfillcolor{currentfill}%
\pgfsetfillopacity{0.452607}%
\pgfsetlinewidth{1.003750pt}%
\definecolor{currentstroke}{rgb}{0.121569,0.466667,0.705882}%
\pgfsetstrokecolor{currentstroke}%
\pgfsetstrokeopacity{0.452607}%
\pgfsetdash{}{0pt}%
\pgfpathmoveto{\pgfqpoint{2.509696in}{2.293036in}}%
\pgfpathcurveto{\pgfqpoint{2.517932in}{2.293036in}}{\pgfqpoint{2.525832in}{2.296308in}}{\pgfqpoint{2.531656in}{2.302132in}}%
\pgfpathcurveto{\pgfqpoint{2.537480in}{2.307956in}}{\pgfqpoint{2.540753in}{2.315856in}}{\pgfqpoint{2.540753in}{2.324093in}}%
\pgfpathcurveto{\pgfqpoint{2.540753in}{2.332329in}}{\pgfqpoint{2.537480in}{2.340229in}}{\pgfqpoint{2.531656in}{2.346053in}}%
\pgfpathcurveto{\pgfqpoint{2.525832in}{2.351877in}}{\pgfqpoint{2.517932in}{2.355149in}}{\pgfqpoint{2.509696in}{2.355149in}}%
\pgfpathcurveto{\pgfqpoint{2.501460in}{2.355149in}}{\pgfqpoint{2.493560in}{2.351877in}}{\pgfqpoint{2.487736in}{2.346053in}}%
\pgfpathcurveto{\pgfqpoint{2.481912in}{2.340229in}}{\pgfqpoint{2.478640in}{2.332329in}}{\pgfqpoint{2.478640in}{2.324093in}}%
\pgfpathcurveto{\pgfqpoint{2.478640in}{2.315856in}}{\pgfqpoint{2.481912in}{2.307956in}}{\pgfqpoint{2.487736in}{2.302132in}}%
\pgfpathcurveto{\pgfqpoint{2.493560in}{2.296308in}}{\pgfqpoint{2.501460in}{2.293036in}}{\pgfqpoint{2.509696in}{2.293036in}}%
\pgfpathclose%
\pgfusepath{stroke,fill}%
\end{pgfscope}%
\begin{pgfscope}%
\pgfpathrectangle{\pgfqpoint{0.100000in}{0.212622in}}{\pgfqpoint{3.696000in}{3.696000in}}%
\pgfusepath{clip}%
\pgfsetbuttcap%
\pgfsetroundjoin%
\definecolor{currentfill}{rgb}{0.121569,0.466667,0.705882}%
\pgfsetfillcolor{currentfill}%
\pgfsetfillopacity{0.453192}%
\pgfsetlinewidth{1.003750pt}%
\definecolor{currentstroke}{rgb}{0.121569,0.466667,0.705882}%
\pgfsetstrokecolor{currentstroke}%
\pgfsetstrokeopacity{0.453192}%
\pgfsetdash{}{0pt}%
\pgfpathmoveto{\pgfqpoint{1.259613in}{1.820214in}}%
\pgfpathcurveto{\pgfqpoint{1.267850in}{1.820214in}}{\pgfqpoint{1.275750in}{1.823486in}}{\pgfqpoint{1.281574in}{1.829310in}}%
\pgfpathcurveto{\pgfqpoint{1.287398in}{1.835134in}}{\pgfqpoint{1.290670in}{1.843034in}}{\pgfqpoint{1.290670in}{1.851270in}}%
\pgfpathcurveto{\pgfqpoint{1.290670in}{1.859506in}}{\pgfqpoint{1.287398in}{1.867406in}}{\pgfqpoint{1.281574in}{1.873230in}}%
\pgfpathcurveto{\pgfqpoint{1.275750in}{1.879054in}}{\pgfqpoint{1.267850in}{1.882327in}}{\pgfqpoint{1.259613in}{1.882327in}}%
\pgfpathcurveto{\pgfqpoint{1.251377in}{1.882327in}}{\pgfqpoint{1.243477in}{1.879054in}}{\pgfqpoint{1.237653in}{1.873230in}}%
\pgfpathcurveto{\pgfqpoint{1.231829in}{1.867406in}}{\pgfqpoint{1.228557in}{1.859506in}}{\pgfqpoint{1.228557in}{1.851270in}}%
\pgfpathcurveto{\pgfqpoint{1.228557in}{1.843034in}}{\pgfqpoint{1.231829in}{1.835134in}}{\pgfqpoint{1.237653in}{1.829310in}}%
\pgfpathcurveto{\pgfqpoint{1.243477in}{1.823486in}}{\pgfqpoint{1.251377in}{1.820214in}}{\pgfqpoint{1.259613in}{1.820214in}}%
\pgfpathclose%
\pgfusepath{stroke,fill}%
\end{pgfscope}%
\begin{pgfscope}%
\pgfpathrectangle{\pgfqpoint{0.100000in}{0.212622in}}{\pgfqpoint{3.696000in}{3.696000in}}%
\pgfusepath{clip}%
\pgfsetbuttcap%
\pgfsetroundjoin%
\definecolor{currentfill}{rgb}{0.121569,0.466667,0.705882}%
\pgfsetfillcolor{currentfill}%
\pgfsetfillopacity{0.457290}%
\pgfsetlinewidth{1.003750pt}%
\definecolor{currentstroke}{rgb}{0.121569,0.466667,0.705882}%
\pgfsetstrokecolor{currentstroke}%
\pgfsetstrokeopacity{0.457290}%
\pgfsetdash{}{0pt}%
\pgfpathmoveto{\pgfqpoint{2.533270in}{2.296315in}}%
\pgfpathcurveto{\pgfqpoint{2.541506in}{2.296315in}}{\pgfqpoint{2.549406in}{2.299587in}}{\pgfqpoint{2.555230in}{2.305411in}}%
\pgfpathcurveto{\pgfqpoint{2.561054in}{2.311235in}}{\pgfqpoint{2.564326in}{2.319135in}}{\pgfqpoint{2.564326in}{2.327371in}}%
\pgfpathcurveto{\pgfqpoint{2.564326in}{2.335608in}}{\pgfqpoint{2.561054in}{2.343508in}}{\pgfqpoint{2.555230in}{2.349332in}}%
\pgfpathcurveto{\pgfqpoint{2.549406in}{2.355156in}}{\pgfqpoint{2.541506in}{2.358428in}}{\pgfqpoint{2.533270in}{2.358428in}}%
\pgfpathcurveto{\pgfqpoint{2.525033in}{2.358428in}}{\pgfqpoint{2.517133in}{2.355156in}}{\pgfqpoint{2.511309in}{2.349332in}}%
\pgfpathcurveto{\pgfqpoint{2.505485in}{2.343508in}}{\pgfqpoint{2.502213in}{2.335608in}}{\pgfqpoint{2.502213in}{2.327371in}}%
\pgfpathcurveto{\pgfqpoint{2.502213in}{2.319135in}}{\pgfqpoint{2.505485in}{2.311235in}}{\pgfqpoint{2.511309in}{2.305411in}}%
\pgfpathcurveto{\pgfqpoint{2.517133in}{2.299587in}}{\pgfqpoint{2.525033in}{2.296315in}}{\pgfqpoint{2.533270in}{2.296315in}}%
\pgfpathclose%
\pgfusepath{stroke,fill}%
\end{pgfscope}%
\begin{pgfscope}%
\pgfpathrectangle{\pgfqpoint{0.100000in}{0.212622in}}{\pgfqpoint{3.696000in}{3.696000in}}%
\pgfusepath{clip}%
\pgfsetbuttcap%
\pgfsetroundjoin%
\definecolor{currentfill}{rgb}{0.121569,0.466667,0.705882}%
\pgfsetfillcolor{currentfill}%
\pgfsetfillopacity{0.460478}%
\pgfsetlinewidth{1.003750pt}%
\definecolor{currentstroke}{rgb}{0.121569,0.466667,0.705882}%
\pgfsetstrokecolor{currentstroke}%
\pgfsetstrokeopacity{0.460478}%
\pgfsetdash{}{0pt}%
\pgfpathmoveto{\pgfqpoint{1.239892in}{1.801513in}}%
\pgfpathcurveto{\pgfqpoint{1.248128in}{1.801513in}}{\pgfqpoint{1.256028in}{1.804785in}}{\pgfqpoint{1.261852in}{1.810609in}}%
\pgfpathcurveto{\pgfqpoint{1.267676in}{1.816433in}}{\pgfqpoint{1.270948in}{1.824333in}}{\pgfqpoint{1.270948in}{1.832569in}}%
\pgfpathcurveto{\pgfqpoint{1.270948in}{1.840806in}}{\pgfqpoint{1.267676in}{1.848706in}}{\pgfqpoint{1.261852in}{1.854530in}}%
\pgfpathcurveto{\pgfqpoint{1.256028in}{1.860354in}}{\pgfqpoint{1.248128in}{1.863626in}}{\pgfqpoint{1.239892in}{1.863626in}}%
\pgfpathcurveto{\pgfqpoint{1.231655in}{1.863626in}}{\pgfqpoint{1.223755in}{1.860354in}}{\pgfqpoint{1.217932in}{1.854530in}}%
\pgfpathcurveto{\pgfqpoint{1.212108in}{1.848706in}}{\pgfqpoint{1.208835in}{1.840806in}}{\pgfqpoint{1.208835in}{1.832569in}}%
\pgfpathcurveto{\pgfqpoint{1.208835in}{1.824333in}}{\pgfqpoint{1.212108in}{1.816433in}}{\pgfqpoint{1.217932in}{1.810609in}}%
\pgfpathcurveto{\pgfqpoint{1.223755in}{1.804785in}}{\pgfqpoint{1.231655in}{1.801513in}}{\pgfqpoint{1.239892in}{1.801513in}}%
\pgfpathclose%
\pgfusepath{stroke,fill}%
\end{pgfscope}%
\begin{pgfscope}%
\pgfpathrectangle{\pgfqpoint{0.100000in}{0.212622in}}{\pgfqpoint{3.696000in}{3.696000in}}%
\pgfusepath{clip}%
\pgfsetbuttcap%
\pgfsetroundjoin%
\definecolor{currentfill}{rgb}{0.121569,0.466667,0.705882}%
\pgfsetfillcolor{currentfill}%
\pgfsetfillopacity{0.462852}%
\pgfsetlinewidth{1.003750pt}%
\definecolor{currentstroke}{rgb}{0.121569,0.466667,0.705882}%
\pgfsetstrokecolor{currentstroke}%
\pgfsetstrokeopacity{0.462852}%
\pgfsetdash{}{0pt}%
\pgfpathmoveto{\pgfqpoint{2.554501in}{2.290893in}}%
\pgfpathcurveto{\pgfqpoint{2.562737in}{2.290893in}}{\pgfqpoint{2.570637in}{2.294165in}}{\pgfqpoint{2.576461in}{2.299989in}}%
\pgfpathcurveto{\pgfqpoint{2.582285in}{2.305813in}}{\pgfqpoint{2.585558in}{2.313713in}}{\pgfqpoint{2.585558in}{2.321949in}}%
\pgfpathcurveto{\pgfqpoint{2.585558in}{2.330185in}}{\pgfqpoint{2.582285in}{2.338085in}}{\pgfqpoint{2.576461in}{2.343909in}}%
\pgfpathcurveto{\pgfqpoint{2.570637in}{2.349733in}}{\pgfqpoint{2.562737in}{2.353006in}}{\pgfqpoint{2.554501in}{2.353006in}}%
\pgfpathcurveto{\pgfqpoint{2.546265in}{2.353006in}}{\pgfqpoint{2.538365in}{2.349733in}}{\pgfqpoint{2.532541in}{2.343909in}}%
\pgfpathcurveto{\pgfqpoint{2.526717in}{2.338085in}}{\pgfqpoint{2.523445in}{2.330185in}}{\pgfqpoint{2.523445in}{2.321949in}}%
\pgfpathcurveto{\pgfqpoint{2.523445in}{2.313713in}}{\pgfqpoint{2.526717in}{2.305813in}}{\pgfqpoint{2.532541in}{2.299989in}}%
\pgfpathcurveto{\pgfqpoint{2.538365in}{2.294165in}}{\pgfqpoint{2.546265in}{2.290893in}}{\pgfqpoint{2.554501in}{2.290893in}}%
\pgfpathclose%
\pgfusepath{stroke,fill}%
\end{pgfscope}%
\begin{pgfscope}%
\pgfpathrectangle{\pgfqpoint{0.100000in}{0.212622in}}{\pgfqpoint{3.696000in}{3.696000in}}%
\pgfusepath{clip}%
\pgfsetbuttcap%
\pgfsetroundjoin%
\definecolor{currentfill}{rgb}{0.121569,0.466667,0.705882}%
\pgfsetfillcolor{currentfill}%
\pgfsetfillopacity{0.465942}%
\pgfsetlinewidth{1.003750pt}%
\definecolor{currentstroke}{rgb}{0.121569,0.466667,0.705882}%
\pgfsetstrokecolor{currentstroke}%
\pgfsetstrokeopacity{0.465942}%
\pgfsetdash{}{0pt}%
\pgfpathmoveto{\pgfqpoint{2.567954in}{2.292209in}}%
\pgfpathcurveto{\pgfqpoint{2.576191in}{2.292209in}}{\pgfqpoint{2.584091in}{2.295482in}}{\pgfqpoint{2.589914in}{2.301306in}}%
\pgfpathcurveto{\pgfqpoint{2.595738in}{2.307130in}}{\pgfqpoint{2.599011in}{2.315030in}}{\pgfqpoint{2.599011in}{2.323266in}}%
\pgfpathcurveto{\pgfqpoint{2.599011in}{2.331502in}}{\pgfqpoint{2.595738in}{2.339402in}}{\pgfqpoint{2.589914in}{2.345226in}}%
\pgfpathcurveto{\pgfqpoint{2.584091in}{2.351050in}}{\pgfqpoint{2.576191in}{2.354322in}}{\pgfqpoint{2.567954in}{2.354322in}}%
\pgfpathcurveto{\pgfqpoint{2.559718in}{2.354322in}}{\pgfqpoint{2.551818in}{2.351050in}}{\pgfqpoint{2.545994in}{2.345226in}}%
\pgfpathcurveto{\pgfqpoint{2.540170in}{2.339402in}}{\pgfqpoint{2.536898in}{2.331502in}}{\pgfqpoint{2.536898in}{2.323266in}}%
\pgfpathcurveto{\pgfqpoint{2.536898in}{2.315030in}}{\pgfqpoint{2.540170in}{2.307130in}}{\pgfqpoint{2.545994in}{2.301306in}}%
\pgfpathcurveto{\pgfqpoint{2.551818in}{2.295482in}}{\pgfqpoint{2.559718in}{2.292209in}}{\pgfqpoint{2.567954in}{2.292209in}}%
\pgfpathclose%
\pgfusepath{stroke,fill}%
\end{pgfscope}%
\begin{pgfscope}%
\pgfpathrectangle{\pgfqpoint{0.100000in}{0.212622in}}{\pgfqpoint{3.696000in}{3.696000in}}%
\pgfusepath{clip}%
\pgfsetbuttcap%
\pgfsetroundjoin%
\definecolor{currentfill}{rgb}{0.121569,0.466667,0.705882}%
\pgfsetfillcolor{currentfill}%
\pgfsetfillopacity{0.466082}%
\pgfsetlinewidth{1.003750pt}%
\definecolor{currentstroke}{rgb}{0.121569,0.466667,0.705882}%
\pgfsetstrokecolor{currentstroke}%
\pgfsetstrokeopacity{0.466082}%
\pgfsetdash{}{0pt}%
\pgfpathmoveto{\pgfqpoint{1.231141in}{1.784511in}}%
\pgfpathcurveto{\pgfqpoint{1.239377in}{1.784511in}}{\pgfqpoint{1.247277in}{1.787783in}}{\pgfqpoint{1.253101in}{1.793607in}}%
\pgfpathcurveto{\pgfqpoint{1.258925in}{1.799431in}}{\pgfqpoint{1.262197in}{1.807331in}}{\pgfqpoint{1.262197in}{1.815567in}}%
\pgfpathcurveto{\pgfqpoint{1.262197in}{1.823803in}}{\pgfqpoint{1.258925in}{1.831703in}}{\pgfqpoint{1.253101in}{1.837527in}}%
\pgfpathcurveto{\pgfqpoint{1.247277in}{1.843351in}}{\pgfqpoint{1.239377in}{1.846624in}}{\pgfqpoint{1.231141in}{1.846624in}}%
\pgfpathcurveto{\pgfqpoint{1.222905in}{1.846624in}}{\pgfqpoint{1.215005in}{1.843351in}}{\pgfqpoint{1.209181in}{1.837527in}}%
\pgfpathcurveto{\pgfqpoint{1.203357in}{1.831703in}}{\pgfqpoint{1.200084in}{1.823803in}}{\pgfqpoint{1.200084in}{1.815567in}}%
\pgfpathcurveto{\pgfqpoint{1.200084in}{1.807331in}}{\pgfqpoint{1.203357in}{1.799431in}}{\pgfqpoint{1.209181in}{1.793607in}}%
\pgfpathcurveto{\pgfqpoint{1.215005in}{1.787783in}}{\pgfqpoint{1.222905in}{1.784511in}}{\pgfqpoint{1.231141in}{1.784511in}}%
\pgfpathclose%
\pgfusepath{stroke,fill}%
\end{pgfscope}%
\begin{pgfscope}%
\pgfpathrectangle{\pgfqpoint{0.100000in}{0.212622in}}{\pgfqpoint{3.696000in}{3.696000in}}%
\pgfusepath{clip}%
\pgfsetbuttcap%
\pgfsetroundjoin%
\definecolor{currentfill}{rgb}{0.121569,0.466667,0.705882}%
\pgfsetfillcolor{currentfill}%
\pgfsetfillopacity{0.469179}%
\pgfsetlinewidth{1.003750pt}%
\definecolor{currentstroke}{rgb}{0.121569,0.466667,0.705882}%
\pgfsetstrokecolor{currentstroke}%
\pgfsetstrokeopacity{0.469179}%
\pgfsetdash{}{0pt}%
\pgfpathmoveto{\pgfqpoint{1.221217in}{1.775670in}}%
\pgfpathcurveto{\pgfqpoint{1.229454in}{1.775670in}}{\pgfqpoint{1.237354in}{1.778942in}}{\pgfqpoint{1.243178in}{1.784766in}}%
\pgfpathcurveto{\pgfqpoint{1.249002in}{1.790590in}}{\pgfqpoint{1.252274in}{1.798490in}}{\pgfqpoint{1.252274in}{1.806727in}}%
\pgfpathcurveto{\pgfqpoint{1.252274in}{1.814963in}}{\pgfqpoint{1.249002in}{1.822863in}}{\pgfqpoint{1.243178in}{1.828687in}}%
\pgfpathcurveto{\pgfqpoint{1.237354in}{1.834511in}}{\pgfqpoint{1.229454in}{1.837783in}}{\pgfqpoint{1.221217in}{1.837783in}}%
\pgfpathcurveto{\pgfqpoint{1.212981in}{1.837783in}}{\pgfqpoint{1.205081in}{1.834511in}}{\pgfqpoint{1.199257in}{1.828687in}}%
\pgfpathcurveto{\pgfqpoint{1.193433in}{1.822863in}}{\pgfqpoint{1.190161in}{1.814963in}}{\pgfqpoint{1.190161in}{1.806727in}}%
\pgfpathcurveto{\pgfqpoint{1.190161in}{1.798490in}}{\pgfqpoint{1.193433in}{1.790590in}}{\pgfqpoint{1.199257in}{1.784766in}}%
\pgfpathcurveto{\pgfqpoint{1.205081in}{1.778942in}}{\pgfqpoint{1.212981in}{1.775670in}}{\pgfqpoint{1.221217in}{1.775670in}}%
\pgfpathclose%
\pgfusepath{stroke,fill}%
\end{pgfscope}%
\begin{pgfscope}%
\pgfpathrectangle{\pgfqpoint{0.100000in}{0.212622in}}{\pgfqpoint{3.696000in}{3.696000in}}%
\pgfusepath{clip}%
\pgfsetbuttcap%
\pgfsetroundjoin%
\definecolor{currentfill}{rgb}{0.121569,0.466667,0.705882}%
\pgfsetfillcolor{currentfill}%
\pgfsetfillopacity{0.469275}%
\pgfsetlinewidth{1.003750pt}%
\definecolor{currentstroke}{rgb}{0.121569,0.466667,0.705882}%
\pgfsetstrokecolor{currentstroke}%
\pgfsetstrokeopacity{0.469275}%
\pgfsetdash{}{0pt}%
\pgfpathmoveto{\pgfqpoint{2.583359in}{2.290372in}}%
\pgfpathcurveto{\pgfqpoint{2.591595in}{2.290372in}}{\pgfqpoint{2.599495in}{2.293644in}}{\pgfqpoint{2.605319in}{2.299468in}}%
\pgfpathcurveto{\pgfqpoint{2.611143in}{2.305292in}}{\pgfqpoint{2.614416in}{2.313192in}}{\pgfqpoint{2.614416in}{2.321428in}}%
\pgfpathcurveto{\pgfqpoint{2.614416in}{2.329664in}}{\pgfqpoint{2.611143in}{2.337564in}}{\pgfqpoint{2.605319in}{2.343388in}}%
\pgfpathcurveto{\pgfqpoint{2.599495in}{2.349212in}}{\pgfqpoint{2.591595in}{2.352485in}}{\pgfqpoint{2.583359in}{2.352485in}}%
\pgfpathcurveto{\pgfqpoint{2.575123in}{2.352485in}}{\pgfqpoint{2.567223in}{2.349212in}}{\pgfqpoint{2.561399in}{2.343388in}}%
\pgfpathcurveto{\pgfqpoint{2.555575in}{2.337564in}}{\pgfqpoint{2.552303in}{2.329664in}}{\pgfqpoint{2.552303in}{2.321428in}}%
\pgfpathcurveto{\pgfqpoint{2.552303in}{2.313192in}}{\pgfqpoint{2.555575in}{2.305292in}}{\pgfqpoint{2.561399in}{2.299468in}}%
\pgfpathcurveto{\pgfqpoint{2.567223in}{2.293644in}}{\pgfqpoint{2.575123in}{2.290372in}}{\pgfqpoint{2.583359in}{2.290372in}}%
\pgfpathclose%
\pgfusepath{stroke,fill}%
\end{pgfscope}%
\begin{pgfscope}%
\pgfpathrectangle{\pgfqpoint{0.100000in}{0.212622in}}{\pgfqpoint{3.696000in}{3.696000in}}%
\pgfusepath{clip}%
\pgfsetbuttcap%
\pgfsetroundjoin%
\definecolor{currentfill}{rgb}{0.121569,0.466667,0.705882}%
\pgfsetfillcolor{currentfill}%
\pgfsetfillopacity{0.470573}%
\pgfsetlinewidth{1.003750pt}%
\definecolor{currentstroke}{rgb}{0.121569,0.466667,0.705882}%
\pgfsetstrokecolor{currentstroke}%
\pgfsetstrokeopacity{0.470573}%
\pgfsetdash{}{0pt}%
\pgfpathmoveto{\pgfqpoint{1.217369in}{1.769803in}}%
\pgfpathcurveto{\pgfqpoint{1.225605in}{1.769803in}}{\pgfqpoint{1.233505in}{1.773076in}}{\pgfqpoint{1.239329in}{1.778900in}}%
\pgfpathcurveto{\pgfqpoint{1.245153in}{1.784724in}}{\pgfqpoint{1.248426in}{1.792624in}}{\pgfqpoint{1.248426in}{1.800860in}}%
\pgfpathcurveto{\pgfqpoint{1.248426in}{1.809096in}}{\pgfqpoint{1.245153in}{1.816996in}}{\pgfqpoint{1.239329in}{1.822820in}}%
\pgfpathcurveto{\pgfqpoint{1.233505in}{1.828644in}}{\pgfqpoint{1.225605in}{1.831916in}}{\pgfqpoint{1.217369in}{1.831916in}}%
\pgfpathcurveto{\pgfqpoint{1.209133in}{1.831916in}}{\pgfqpoint{1.201233in}{1.828644in}}{\pgfqpoint{1.195409in}{1.822820in}}%
\pgfpathcurveto{\pgfqpoint{1.189585in}{1.816996in}}{\pgfqpoint{1.186313in}{1.809096in}}{\pgfqpoint{1.186313in}{1.800860in}}%
\pgfpathcurveto{\pgfqpoint{1.186313in}{1.792624in}}{\pgfqpoint{1.189585in}{1.784724in}}{\pgfqpoint{1.195409in}{1.778900in}}%
\pgfpathcurveto{\pgfqpoint{1.201233in}{1.773076in}}{\pgfqpoint{1.209133in}{1.769803in}}{\pgfqpoint{1.217369in}{1.769803in}}%
\pgfpathclose%
\pgfusepath{stroke,fill}%
\end{pgfscope}%
\begin{pgfscope}%
\pgfpathrectangle{\pgfqpoint{0.100000in}{0.212622in}}{\pgfqpoint{3.696000in}{3.696000in}}%
\pgfusepath{clip}%
\pgfsetbuttcap%
\pgfsetroundjoin%
\definecolor{currentfill}{rgb}{0.121569,0.466667,0.705882}%
\pgfsetfillcolor{currentfill}%
\pgfsetfillopacity{0.470966}%
\pgfsetlinewidth{1.003750pt}%
\definecolor{currentstroke}{rgb}{0.121569,0.466667,0.705882}%
\pgfsetstrokecolor{currentstroke}%
\pgfsetstrokeopacity{0.470966}%
\pgfsetdash{}{0pt}%
\pgfpathmoveto{\pgfqpoint{1.216007in}{1.768716in}}%
\pgfpathcurveto{\pgfqpoint{1.224243in}{1.768716in}}{\pgfqpoint{1.232143in}{1.771988in}}{\pgfqpoint{1.237967in}{1.777812in}}%
\pgfpathcurveto{\pgfqpoint{1.243791in}{1.783636in}}{\pgfqpoint{1.247063in}{1.791536in}}{\pgfqpoint{1.247063in}{1.799773in}}%
\pgfpathcurveto{\pgfqpoint{1.247063in}{1.808009in}}{\pgfqpoint{1.243791in}{1.815909in}}{\pgfqpoint{1.237967in}{1.821733in}}%
\pgfpathcurveto{\pgfqpoint{1.232143in}{1.827557in}}{\pgfqpoint{1.224243in}{1.830829in}}{\pgfqpoint{1.216007in}{1.830829in}}%
\pgfpathcurveto{\pgfqpoint{1.207771in}{1.830829in}}{\pgfqpoint{1.199871in}{1.827557in}}{\pgfqpoint{1.194047in}{1.821733in}}%
\pgfpathcurveto{\pgfqpoint{1.188223in}{1.815909in}}{\pgfqpoint{1.184950in}{1.808009in}}{\pgfqpoint{1.184950in}{1.799773in}}%
\pgfpathcurveto{\pgfqpoint{1.184950in}{1.791536in}}{\pgfqpoint{1.188223in}{1.783636in}}{\pgfqpoint{1.194047in}{1.777812in}}%
\pgfpathcurveto{\pgfqpoint{1.199871in}{1.771988in}}{\pgfqpoint{1.207771in}{1.768716in}}{\pgfqpoint{1.216007in}{1.768716in}}%
\pgfpathclose%
\pgfusepath{stroke,fill}%
\end{pgfscope}%
\begin{pgfscope}%
\pgfpathrectangle{\pgfqpoint{0.100000in}{0.212622in}}{\pgfqpoint{3.696000in}{3.696000in}}%
\pgfusepath{clip}%
\pgfsetbuttcap%
\pgfsetroundjoin%
\definecolor{currentfill}{rgb}{0.121569,0.466667,0.705882}%
\pgfsetfillcolor{currentfill}%
\pgfsetfillopacity{0.471675}%
\pgfsetlinewidth{1.003750pt}%
\definecolor{currentstroke}{rgb}{0.121569,0.466667,0.705882}%
\pgfsetstrokecolor{currentstroke}%
\pgfsetstrokeopacity{0.471675}%
\pgfsetdash{}{0pt}%
\pgfpathmoveto{\pgfqpoint{1.214570in}{1.765674in}}%
\pgfpathcurveto{\pgfqpoint{1.222807in}{1.765674in}}{\pgfqpoint{1.230707in}{1.768946in}}{\pgfqpoint{1.236531in}{1.774770in}}%
\pgfpathcurveto{\pgfqpoint{1.242355in}{1.780594in}}{\pgfqpoint{1.245627in}{1.788494in}}{\pgfqpoint{1.245627in}{1.796730in}}%
\pgfpathcurveto{\pgfqpoint{1.245627in}{1.804967in}}{\pgfqpoint{1.242355in}{1.812867in}}{\pgfqpoint{1.236531in}{1.818691in}}%
\pgfpathcurveto{\pgfqpoint{1.230707in}{1.824515in}}{\pgfqpoint{1.222807in}{1.827787in}}{\pgfqpoint{1.214570in}{1.827787in}}%
\pgfpathcurveto{\pgfqpoint{1.206334in}{1.827787in}}{\pgfqpoint{1.198434in}{1.824515in}}{\pgfqpoint{1.192610in}{1.818691in}}%
\pgfpathcurveto{\pgfqpoint{1.186786in}{1.812867in}}{\pgfqpoint{1.183514in}{1.804967in}}{\pgfqpoint{1.183514in}{1.796730in}}%
\pgfpathcurveto{\pgfqpoint{1.183514in}{1.788494in}}{\pgfqpoint{1.186786in}{1.780594in}}{\pgfqpoint{1.192610in}{1.774770in}}%
\pgfpathcurveto{\pgfqpoint{1.198434in}{1.768946in}}{\pgfqpoint{1.206334in}{1.765674in}}{\pgfqpoint{1.214570in}{1.765674in}}%
\pgfpathclose%
\pgfusepath{stroke,fill}%
\end{pgfscope}%
\begin{pgfscope}%
\pgfpathrectangle{\pgfqpoint{0.100000in}{0.212622in}}{\pgfqpoint{3.696000in}{3.696000in}}%
\pgfusepath{clip}%
\pgfsetbuttcap%
\pgfsetroundjoin%
\definecolor{currentfill}{rgb}{0.121569,0.466667,0.705882}%
\pgfsetfillcolor{currentfill}%
\pgfsetfillopacity{0.473093}%
\pgfsetlinewidth{1.003750pt}%
\definecolor{currentstroke}{rgb}{0.121569,0.466667,0.705882}%
\pgfsetstrokecolor{currentstroke}%
\pgfsetstrokeopacity{0.473093}%
\pgfsetdash{}{0pt}%
\pgfpathmoveto{\pgfqpoint{1.211774in}{1.761100in}}%
\pgfpathcurveto{\pgfqpoint{1.220010in}{1.761100in}}{\pgfqpoint{1.227910in}{1.764373in}}{\pgfqpoint{1.233734in}{1.770197in}}%
\pgfpathcurveto{\pgfqpoint{1.239558in}{1.776021in}}{\pgfqpoint{1.242831in}{1.783921in}}{\pgfqpoint{1.242831in}{1.792157in}}%
\pgfpathcurveto{\pgfqpoint{1.242831in}{1.800393in}}{\pgfqpoint{1.239558in}{1.808293in}}{\pgfqpoint{1.233734in}{1.814117in}}%
\pgfpathcurveto{\pgfqpoint{1.227910in}{1.819941in}}{\pgfqpoint{1.220010in}{1.823213in}}{\pgfqpoint{1.211774in}{1.823213in}}%
\pgfpathcurveto{\pgfqpoint{1.203538in}{1.823213in}}{\pgfqpoint{1.195638in}{1.819941in}}{\pgfqpoint{1.189814in}{1.814117in}}%
\pgfpathcurveto{\pgfqpoint{1.183990in}{1.808293in}}{\pgfqpoint{1.180718in}{1.800393in}}{\pgfqpoint{1.180718in}{1.792157in}}%
\pgfpathcurveto{\pgfqpoint{1.180718in}{1.783921in}}{\pgfqpoint{1.183990in}{1.776021in}}{\pgfqpoint{1.189814in}{1.770197in}}%
\pgfpathcurveto{\pgfqpoint{1.195638in}{1.764373in}}{\pgfqpoint{1.203538in}{1.761100in}}{\pgfqpoint{1.211774in}{1.761100in}}%
\pgfpathclose%
\pgfusepath{stroke,fill}%
\end{pgfscope}%
\begin{pgfscope}%
\pgfpathrectangle{\pgfqpoint{0.100000in}{0.212622in}}{\pgfqpoint{3.696000in}{3.696000in}}%
\pgfusepath{clip}%
\pgfsetbuttcap%
\pgfsetroundjoin%
\definecolor{currentfill}{rgb}{0.121569,0.466667,0.705882}%
\pgfsetfillcolor{currentfill}%
\pgfsetfillopacity{0.473663}%
\pgfsetlinewidth{1.003750pt}%
\definecolor{currentstroke}{rgb}{0.121569,0.466667,0.705882}%
\pgfsetstrokecolor{currentstroke}%
\pgfsetstrokeopacity{0.473663}%
\pgfsetdash{}{0pt}%
\pgfpathmoveto{\pgfqpoint{2.602977in}{2.291551in}}%
\pgfpathcurveto{\pgfqpoint{2.611213in}{2.291551in}}{\pgfqpoint{2.619113in}{2.294823in}}{\pgfqpoint{2.624937in}{2.300647in}}%
\pgfpathcurveto{\pgfqpoint{2.630761in}{2.306471in}}{\pgfqpoint{2.634034in}{2.314371in}}{\pgfqpoint{2.634034in}{2.322608in}}%
\pgfpathcurveto{\pgfqpoint{2.634034in}{2.330844in}}{\pgfqpoint{2.630761in}{2.338744in}}{\pgfqpoint{2.624937in}{2.344568in}}%
\pgfpathcurveto{\pgfqpoint{2.619113in}{2.350392in}}{\pgfqpoint{2.611213in}{2.353664in}}{\pgfqpoint{2.602977in}{2.353664in}}%
\pgfpathcurveto{\pgfqpoint{2.594741in}{2.353664in}}{\pgfqpoint{2.586841in}{2.350392in}}{\pgfqpoint{2.581017in}{2.344568in}}%
\pgfpathcurveto{\pgfqpoint{2.575193in}{2.338744in}}{\pgfqpoint{2.571921in}{2.330844in}}{\pgfqpoint{2.571921in}{2.322608in}}%
\pgfpathcurveto{\pgfqpoint{2.571921in}{2.314371in}}{\pgfqpoint{2.575193in}{2.306471in}}{\pgfqpoint{2.581017in}{2.300647in}}%
\pgfpathcurveto{\pgfqpoint{2.586841in}{2.294823in}}{\pgfqpoint{2.594741in}{2.291551in}}{\pgfqpoint{2.602977in}{2.291551in}}%
\pgfpathclose%
\pgfusepath{stroke,fill}%
\end{pgfscope}%
\begin{pgfscope}%
\pgfpathrectangle{\pgfqpoint{0.100000in}{0.212622in}}{\pgfqpoint{3.696000in}{3.696000in}}%
\pgfusepath{clip}%
\pgfsetbuttcap%
\pgfsetroundjoin%
\definecolor{currentfill}{rgb}{0.121569,0.466667,0.705882}%
\pgfsetfillcolor{currentfill}%
\pgfsetfillopacity{0.475466}%
\pgfsetlinewidth{1.003750pt}%
\definecolor{currentstroke}{rgb}{0.121569,0.466667,0.705882}%
\pgfsetstrokecolor{currentstroke}%
\pgfsetstrokeopacity{0.475466}%
\pgfsetdash{}{0pt}%
\pgfpathmoveto{\pgfqpoint{1.205755in}{1.752162in}}%
\pgfpathcurveto{\pgfqpoint{1.213992in}{1.752162in}}{\pgfqpoint{1.221892in}{1.755434in}}{\pgfqpoint{1.227716in}{1.761258in}}%
\pgfpathcurveto{\pgfqpoint{1.233540in}{1.767082in}}{\pgfqpoint{1.236812in}{1.774982in}}{\pgfqpoint{1.236812in}{1.783218in}}%
\pgfpathcurveto{\pgfqpoint{1.236812in}{1.791454in}}{\pgfqpoint{1.233540in}{1.799354in}}{\pgfqpoint{1.227716in}{1.805178in}}%
\pgfpathcurveto{\pgfqpoint{1.221892in}{1.811002in}}{\pgfqpoint{1.213992in}{1.814275in}}{\pgfqpoint{1.205755in}{1.814275in}}%
\pgfpathcurveto{\pgfqpoint{1.197519in}{1.814275in}}{\pgfqpoint{1.189619in}{1.811002in}}{\pgfqpoint{1.183795in}{1.805178in}}%
\pgfpathcurveto{\pgfqpoint{1.177971in}{1.799354in}}{\pgfqpoint{1.174699in}{1.791454in}}{\pgfqpoint{1.174699in}{1.783218in}}%
\pgfpathcurveto{\pgfqpoint{1.174699in}{1.774982in}}{\pgfqpoint{1.177971in}{1.767082in}}{\pgfqpoint{1.183795in}{1.761258in}}%
\pgfpathcurveto{\pgfqpoint{1.189619in}{1.755434in}}{\pgfqpoint{1.197519in}{1.752162in}}{\pgfqpoint{1.205755in}{1.752162in}}%
\pgfpathclose%
\pgfusepath{stroke,fill}%
\end{pgfscope}%
\begin{pgfscope}%
\pgfpathrectangle{\pgfqpoint{0.100000in}{0.212622in}}{\pgfqpoint{3.696000in}{3.696000in}}%
\pgfusepath{clip}%
\pgfsetbuttcap%
\pgfsetroundjoin%
\definecolor{currentfill}{rgb}{0.121569,0.466667,0.705882}%
\pgfsetfillcolor{currentfill}%
\pgfsetfillopacity{0.475879}%
\pgfsetlinewidth{1.003750pt}%
\definecolor{currentstroke}{rgb}{0.121569,0.466667,0.705882}%
\pgfsetstrokecolor{currentstroke}%
\pgfsetstrokeopacity{0.475879}%
\pgfsetdash{}{0pt}%
\pgfpathmoveto{\pgfqpoint{2.613995in}{2.291497in}}%
\pgfpathcurveto{\pgfqpoint{2.622232in}{2.291497in}}{\pgfqpoint{2.630132in}{2.294770in}}{\pgfqpoint{2.635956in}{2.300593in}}%
\pgfpathcurveto{\pgfqpoint{2.641779in}{2.306417in}}{\pgfqpoint{2.645052in}{2.314317in}}{\pgfqpoint{2.645052in}{2.322554in}}%
\pgfpathcurveto{\pgfqpoint{2.645052in}{2.330790in}}{\pgfqpoint{2.641779in}{2.338690in}}{\pgfqpoint{2.635956in}{2.344514in}}%
\pgfpathcurveto{\pgfqpoint{2.630132in}{2.350338in}}{\pgfqpoint{2.622232in}{2.353610in}}{\pgfqpoint{2.613995in}{2.353610in}}%
\pgfpathcurveto{\pgfqpoint{2.605759in}{2.353610in}}{\pgfqpoint{2.597859in}{2.350338in}}{\pgfqpoint{2.592035in}{2.344514in}}%
\pgfpathcurveto{\pgfqpoint{2.586211in}{2.338690in}}{\pgfqpoint{2.582939in}{2.330790in}}{\pgfqpoint{2.582939in}{2.322554in}}%
\pgfpathcurveto{\pgfqpoint{2.582939in}{2.314317in}}{\pgfqpoint{2.586211in}{2.306417in}}{\pgfqpoint{2.592035in}{2.300593in}}%
\pgfpathcurveto{\pgfqpoint{2.597859in}{2.294770in}}{\pgfqpoint{2.605759in}{2.291497in}}{\pgfqpoint{2.613995in}{2.291497in}}%
\pgfpathclose%
\pgfusepath{stroke,fill}%
\end{pgfscope}%
\begin{pgfscope}%
\pgfpathrectangle{\pgfqpoint{0.100000in}{0.212622in}}{\pgfqpoint{3.696000in}{3.696000in}}%
\pgfusepath{clip}%
\pgfsetbuttcap%
\pgfsetroundjoin%
\definecolor{currentfill}{rgb}{0.121569,0.466667,0.705882}%
\pgfsetfillcolor{currentfill}%
\pgfsetfillopacity{0.476831}%
\pgfsetlinewidth{1.003750pt}%
\definecolor{currentstroke}{rgb}{0.121569,0.466667,0.705882}%
\pgfsetstrokecolor{currentstroke}%
\pgfsetstrokeopacity{0.476831}%
\pgfsetdash{}{0pt}%
\pgfpathmoveto{\pgfqpoint{2.620426in}{2.290927in}}%
\pgfpathcurveto{\pgfqpoint{2.628663in}{2.290927in}}{\pgfqpoint{2.636563in}{2.294199in}}{\pgfqpoint{2.642387in}{2.300023in}}%
\pgfpathcurveto{\pgfqpoint{2.648210in}{2.305847in}}{\pgfqpoint{2.651483in}{2.313747in}}{\pgfqpoint{2.651483in}{2.321983in}}%
\pgfpathcurveto{\pgfqpoint{2.651483in}{2.330219in}}{\pgfqpoint{2.648210in}{2.338119in}}{\pgfqpoint{2.642387in}{2.343943in}}%
\pgfpathcurveto{\pgfqpoint{2.636563in}{2.349767in}}{\pgfqpoint{2.628663in}{2.353040in}}{\pgfqpoint{2.620426in}{2.353040in}}%
\pgfpathcurveto{\pgfqpoint{2.612190in}{2.353040in}}{\pgfqpoint{2.604290in}{2.349767in}}{\pgfqpoint{2.598466in}{2.343943in}}%
\pgfpathcurveto{\pgfqpoint{2.592642in}{2.338119in}}{\pgfqpoint{2.589370in}{2.330219in}}{\pgfqpoint{2.589370in}{2.321983in}}%
\pgfpathcurveto{\pgfqpoint{2.589370in}{2.313747in}}{\pgfqpoint{2.592642in}{2.305847in}}{\pgfqpoint{2.598466in}{2.300023in}}%
\pgfpathcurveto{\pgfqpoint{2.604290in}{2.294199in}}{\pgfqpoint{2.612190in}{2.290927in}}{\pgfqpoint{2.620426in}{2.290927in}}%
\pgfpathclose%
\pgfusepath{stroke,fill}%
\end{pgfscope}%
\begin{pgfscope}%
\pgfpathrectangle{\pgfqpoint{0.100000in}{0.212622in}}{\pgfqpoint{3.696000in}{3.696000in}}%
\pgfusepath{clip}%
\pgfsetbuttcap%
\pgfsetroundjoin%
\definecolor{currentfill}{rgb}{0.121569,0.466667,0.705882}%
\pgfsetfillcolor{currentfill}%
\pgfsetfillopacity{0.478467}%
\pgfsetlinewidth{1.003750pt}%
\definecolor{currentstroke}{rgb}{0.121569,0.466667,0.705882}%
\pgfsetstrokecolor{currentstroke}%
\pgfsetstrokeopacity{0.478467}%
\pgfsetdash{}{0pt}%
\pgfpathmoveto{\pgfqpoint{2.628931in}{2.292683in}}%
\pgfpathcurveto{\pgfqpoint{2.637168in}{2.292683in}}{\pgfqpoint{2.645068in}{2.295956in}}{\pgfqpoint{2.650892in}{2.301780in}}%
\pgfpathcurveto{\pgfqpoint{2.656716in}{2.307604in}}{\pgfqpoint{2.659988in}{2.315504in}}{\pgfqpoint{2.659988in}{2.323740in}}%
\pgfpathcurveto{\pgfqpoint{2.659988in}{2.331976in}}{\pgfqpoint{2.656716in}{2.339876in}}{\pgfqpoint{2.650892in}{2.345700in}}%
\pgfpathcurveto{\pgfqpoint{2.645068in}{2.351524in}}{\pgfqpoint{2.637168in}{2.354796in}}{\pgfqpoint{2.628931in}{2.354796in}}%
\pgfpathcurveto{\pgfqpoint{2.620695in}{2.354796in}}{\pgfqpoint{2.612795in}{2.351524in}}{\pgfqpoint{2.606971in}{2.345700in}}%
\pgfpathcurveto{\pgfqpoint{2.601147in}{2.339876in}}{\pgfqpoint{2.597875in}{2.331976in}}{\pgfqpoint{2.597875in}{2.323740in}}%
\pgfpathcurveto{\pgfqpoint{2.597875in}{2.315504in}}{\pgfqpoint{2.601147in}{2.307604in}}{\pgfqpoint{2.606971in}{2.301780in}}%
\pgfpathcurveto{\pgfqpoint{2.612795in}{2.295956in}}{\pgfqpoint{2.620695in}{2.292683in}}{\pgfqpoint{2.628931in}{2.292683in}}%
\pgfpathclose%
\pgfusepath{stroke,fill}%
\end{pgfscope}%
\begin{pgfscope}%
\pgfpathrectangle{\pgfqpoint{0.100000in}{0.212622in}}{\pgfqpoint{3.696000in}{3.696000in}}%
\pgfusepath{clip}%
\pgfsetbuttcap%
\pgfsetroundjoin%
\definecolor{currentfill}{rgb}{0.121569,0.466667,0.705882}%
\pgfsetfillcolor{currentfill}%
\pgfsetfillopacity{0.479565}%
\pgfsetlinewidth{1.003750pt}%
\definecolor{currentstroke}{rgb}{0.121569,0.466667,0.705882}%
\pgfsetstrokecolor{currentstroke}%
\pgfsetstrokeopacity{0.479565}%
\pgfsetdash{}{0pt}%
\pgfpathmoveto{\pgfqpoint{1.194157in}{1.735027in}}%
\pgfpathcurveto{\pgfqpoint{1.202393in}{1.735027in}}{\pgfqpoint{1.210293in}{1.738299in}}{\pgfqpoint{1.216117in}{1.744123in}}%
\pgfpathcurveto{\pgfqpoint{1.221941in}{1.749947in}}{\pgfqpoint{1.225213in}{1.757847in}}{\pgfqpoint{1.225213in}{1.766083in}}%
\pgfpathcurveto{\pgfqpoint{1.225213in}{1.774320in}}{\pgfqpoint{1.221941in}{1.782220in}}{\pgfqpoint{1.216117in}{1.788044in}}%
\pgfpathcurveto{\pgfqpoint{1.210293in}{1.793868in}}{\pgfqpoint{1.202393in}{1.797140in}}{\pgfqpoint{1.194157in}{1.797140in}}%
\pgfpathcurveto{\pgfqpoint{1.185921in}{1.797140in}}{\pgfqpoint{1.178020in}{1.793868in}}{\pgfqpoint{1.172197in}{1.788044in}}%
\pgfpathcurveto{\pgfqpoint{1.166373in}{1.782220in}}{\pgfqpoint{1.163100in}{1.774320in}}{\pgfqpoint{1.163100in}{1.766083in}}%
\pgfpathcurveto{\pgfqpoint{1.163100in}{1.757847in}}{\pgfqpoint{1.166373in}{1.749947in}}{\pgfqpoint{1.172197in}{1.744123in}}%
\pgfpathcurveto{\pgfqpoint{1.178020in}{1.738299in}}{\pgfqpoint{1.185921in}{1.735027in}}{\pgfqpoint{1.194157in}{1.735027in}}%
\pgfpathclose%
\pgfusepath{stroke,fill}%
\end{pgfscope}%
\begin{pgfscope}%
\pgfpathrectangle{\pgfqpoint{0.100000in}{0.212622in}}{\pgfqpoint{3.696000in}{3.696000in}}%
\pgfusepath{clip}%
\pgfsetbuttcap%
\pgfsetroundjoin%
\definecolor{currentfill}{rgb}{0.121569,0.466667,0.705882}%
\pgfsetfillcolor{currentfill}%
\pgfsetfillopacity{0.480995}%
\pgfsetlinewidth{1.003750pt}%
\definecolor{currentstroke}{rgb}{0.121569,0.466667,0.705882}%
\pgfsetstrokecolor{currentstroke}%
\pgfsetstrokeopacity{0.480995}%
\pgfsetdash{}{0pt}%
\pgfpathmoveto{\pgfqpoint{2.643425in}{2.290241in}}%
\pgfpathcurveto{\pgfqpoint{2.651661in}{2.290241in}}{\pgfqpoint{2.659561in}{2.293513in}}{\pgfqpoint{2.665385in}{2.299337in}}%
\pgfpathcurveto{\pgfqpoint{2.671209in}{2.305161in}}{\pgfqpoint{2.674481in}{2.313061in}}{\pgfqpoint{2.674481in}{2.321297in}}%
\pgfpathcurveto{\pgfqpoint{2.674481in}{2.329534in}}{\pgfqpoint{2.671209in}{2.337434in}}{\pgfqpoint{2.665385in}{2.343258in}}%
\pgfpathcurveto{\pgfqpoint{2.659561in}{2.349082in}}{\pgfqpoint{2.651661in}{2.352354in}}{\pgfqpoint{2.643425in}{2.352354in}}%
\pgfpathcurveto{\pgfqpoint{2.635188in}{2.352354in}}{\pgfqpoint{2.627288in}{2.349082in}}{\pgfqpoint{2.621464in}{2.343258in}}%
\pgfpathcurveto{\pgfqpoint{2.615640in}{2.337434in}}{\pgfqpoint{2.612368in}{2.329534in}}{\pgfqpoint{2.612368in}{2.321297in}}%
\pgfpathcurveto{\pgfqpoint{2.612368in}{2.313061in}}{\pgfqpoint{2.615640in}{2.305161in}}{\pgfqpoint{2.621464in}{2.299337in}}%
\pgfpathcurveto{\pgfqpoint{2.627288in}{2.293513in}}{\pgfqpoint{2.635188in}{2.290241in}}{\pgfqpoint{2.643425in}{2.290241in}}%
\pgfpathclose%
\pgfusepath{stroke,fill}%
\end{pgfscope}%
\begin{pgfscope}%
\pgfpathrectangle{\pgfqpoint{0.100000in}{0.212622in}}{\pgfqpoint{3.696000in}{3.696000in}}%
\pgfusepath{clip}%
\pgfsetbuttcap%
\pgfsetroundjoin%
\definecolor{currentfill}{rgb}{0.121569,0.466667,0.705882}%
\pgfsetfillcolor{currentfill}%
\pgfsetfillopacity{0.484786}%
\pgfsetlinewidth{1.003750pt}%
\definecolor{currentstroke}{rgb}{0.121569,0.466667,0.705882}%
\pgfsetstrokecolor{currentstroke}%
\pgfsetstrokeopacity{0.484786}%
\pgfsetdash{}{0pt}%
\pgfpathmoveto{\pgfqpoint{2.660963in}{2.293518in}}%
\pgfpathcurveto{\pgfqpoint{2.669199in}{2.293518in}}{\pgfqpoint{2.677099in}{2.296791in}}{\pgfqpoint{2.682923in}{2.302614in}}%
\pgfpathcurveto{\pgfqpoint{2.688747in}{2.308438in}}{\pgfqpoint{2.692019in}{2.316338in}}{\pgfqpoint{2.692019in}{2.324575in}}%
\pgfpathcurveto{\pgfqpoint{2.692019in}{2.332811in}}{\pgfqpoint{2.688747in}{2.340711in}}{\pgfqpoint{2.682923in}{2.346535in}}%
\pgfpathcurveto{\pgfqpoint{2.677099in}{2.352359in}}{\pgfqpoint{2.669199in}{2.355631in}}{\pgfqpoint{2.660963in}{2.355631in}}%
\pgfpathcurveto{\pgfqpoint{2.652727in}{2.355631in}}{\pgfqpoint{2.644827in}{2.352359in}}{\pgfqpoint{2.639003in}{2.346535in}}%
\pgfpathcurveto{\pgfqpoint{2.633179in}{2.340711in}}{\pgfqpoint{2.629906in}{2.332811in}}{\pgfqpoint{2.629906in}{2.324575in}}%
\pgfpathcurveto{\pgfqpoint{2.629906in}{2.316338in}}{\pgfqpoint{2.633179in}{2.308438in}}{\pgfqpoint{2.639003in}{2.302614in}}%
\pgfpathcurveto{\pgfqpoint{2.644827in}{2.296791in}}{\pgfqpoint{2.652727in}{2.293518in}}{\pgfqpoint{2.660963in}{2.293518in}}%
\pgfpathclose%
\pgfusepath{stroke,fill}%
\end{pgfscope}%
\begin{pgfscope}%
\pgfpathrectangle{\pgfqpoint{0.100000in}{0.212622in}}{\pgfqpoint{3.696000in}{3.696000in}}%
\pgfusepath{clip}%
\pgfsetbuttcap%
\pgfsetroundjoin%
\definecolor{currentfill}{rgb}{0.121569,0.466667,0.705882}%
\pgfsetfillcolor{currentfill}%
\pgfsetfillopacity{0.488180}%
\pgfsetlinewidth{1.003750pt}%
\definecolor{currentstroke}{rgb}{0.121569,0.466667,0.705882}%
\pgfsetstrokecolor{currentstroke}%
\pgfsetstrokeopacity{0.488180}%
\pgfsetdash{}{0pt}%
\pgfpathmoveto{\pgfqpoint{1.170038in}{1.713893in}}%
\pgfpathcurveto{\pgfqpoint{1.178274in}{1.713893in}}{\pgfqpoint{1.186174in}{1.717165in}}{\pgfqpoint{1.191998in}{1.722989in}}%
\pgfpathcurveto{\pgfqpoint{1.197822in}{1.728813in}}{\pgfqpoint{1.201094in}{1.736713in}}{\pgfqpoint{1.201094in}{1.744949in}}%
\pgfpathcurveto{\pgfqpoint{1.201094in}{1.753186in}}{\pgfqpoint{1.197822in}{1.761086in}}{\pgfqpoint{1.191998in}{1.766910in}}%
\pgfpathcurveto{\pgfqpoint{1.186174in}{1.772733in}}{\pgfqpoint{1.178274in}{1.776006in}}{\pgfqpoint{1.170038in}{1.776006in}}%
\pgfpathcurveto{\pgfqpoint{1.161801in}{1.776006in}}{\pgfqpoint{1.153901in}{1.772733in}}{\pgfqpoint{1.148077in}{1.766910in}}%
\pgfpathcurveto{\pgfqpoint{1.142254in}{1.761086in}}{\pgfqpoint{1.138981in}{1.753186in}}{\pgfqpoint{1.138981in}{1.744949in}}%
\pgfpathcurveto{\pgfqpoint{1.138981in}{1.736713in}}{\pgfqpoint{1.142254in}{1.728813in}}{\pgfqpoint{1.148077in}{1.722989in}}%
\pgfpathcurveto{\pgfqpoint{1.153901in}{1.717165in}}{\pgfqpoint{1.161801in}{1.713893in}}{\pgfqpoint{1.170038in}{1.713893in}}%
\pgfpathclose%
\pgfusepath{stroke,fill}%
\end{pgfscope}%
\begin{pgfscope}%
\pgfpathrectangle{\pgfqpoint{0.100000in}{0.212622in}}{\pgfqpoint{3.696000in}{3.696000in}}%
\pgfusepath{clip}%
\pgfsetbuttcap%
\pgfsetroundjoin%
\definecolor{currentfill}{rgb}{0.121569,0.466667,0.705882}%
\pgfsetfillcolor{currentfill}%
\pgfsetfillopacity{0.488805}%
\pgfsetlinewidth{1.003750pt}%
\definecolor{currentstroke}{rgb}{0.121569,0.466667,0.705882}%
\pgfsetstrokecolor{currentstroke}%
\pgfsetstrokeopacity{0.488805}%
\pgfsetdash{}{0pt}%
\pgfpathmoveto{\pgfqpoint{2.680969in}{2.289321in}}%
\pgfpathcurveto{\pgfqpoint{2.689206in}{2.289321in}}{\pgfqpoint{2.697106in}{2.292593in}}{\pgfqpoint{2.702930in}{2.298417in}}%
\pgfpathcurveto{\pgfqpoint{2.708754in}{2.304241in}}{\pgfqpoint{2.712026in}{2.312141in}}{\pgfqpoint{2.712026in}{2.320377in}}%
\pgfpathcurveto{\pgfqpoint{2.712026in}{2.328614in}}{\pgfqpoint{2.708754in}{2.336514in}}{\pgfqpoint{2.702930in}{2.342337in}}%
\pgfpathcurveto{\pgfqpoint{2.697106in}{2.348161in}}{\pgfqpoint{2.689206in}{2.351434in}}{\pgfqpoint{2.680969in}{2.351434in}}%
\pgfpathcurveto{\pgfqpoint{2.672733in}{2.351434in}}{\pgfqpoint{2.664833in}{2.348161in}}{\pgfqpoint{2.659009in}{2.342337in}}%
\pgfpathcurveto{\pgfqpoint{2.653185in}{2.336514in}}{\pgfqpoint{2.649913in}{2.328614in}}{\pgfqpoint{2.649913in}{2.320377in}}%
\pgfpathcurveto{\pgfqpoint{2.649913in}{2.312141in}}{\pgfqpoint{2.653185in}{2.304241in}}{\pgfqpoint{2.659009in}{2.298417in}}%
\pgfpathcurveto{\pgfqpoint{2.664833in}{2.292593in}}{\pgfqpoint{2.672733in}{2.289321in}}{\pgfqpoint{2.680969in}{2.289321in}}%
\pgfpathclose%
\pgfusepath{stroke,fill}%
\end{pgfscope}%
\begin{pgfscope}%
\pgfpathrectangle{\pgfqpoint{0.100000in}{0.212622in}}{\pgfqpoint{3.696000in}{3.696000in}}%
\pgfusepath{clip}%
\pgfsetbuttcap%
\pgfsetroundjoin%
\definecolor{currentfill}{rgb}{0.121569,0.466667,0.705882}%
\pgfsetfillcolor{currentfill}%
\pgfsetfillopacity{0.491067}%
\pgfsetlinewidth{1.003750pt}%
\definecolor{currentstroke}{rgb}{0.121569,0.466667,0.705882}%
\pgfsetstrokecolor{currentstroke}%
\pgfsetstrokeopacity{0.491067}%
\pgfsetdash{}{0pt}%
\pgfpathmoveto{\pgfqpoint{2.693582in}{2.292323in}}%
\pgfpathcurveto{\pgfqpoint{2.701818in}{2.292323in}}{\pgfqpoint{2.709719in}{2.295596in}}{\pgfqpoint{2.715542in}{2.301420in}}%
\pgfpathcurveto{\pgfqpoint{2.721366in}{2.307243in}}{\pgfqpoint{2.724639in}{2.315144in}}{\pgfqpoint{2.724639in}{2.323380in}}%
\pgfpathcurveto{\pgfqpoint{2.724639in}{2.331616in}}{\pgfqpoint{2.721366in}{2.339516in}}{\pgfqpoint{2.715542in}{2.345340in}}%
\pgfpathcurveto{\pgfqpoint{2.709719in}{2.351164in}}{\pgfqpoint{2.701818in}{2.354436in}}{\pgfqpoint{2.693582in}{2.354436in}}%
\pgfpathcurveto{\pgfqpoint{2.685346in}{2.354436in}}{\pgfqpoint{2.677446in}{2.351164in}}{\pgfqpoint{2.671622in}{2.345340in}}%
\pgfpathcurveto{\pgfqpoint{2.665798in}{2.339516in}}{\pgfqpoint{2.662526in}{2.331616in}}{\pgfqpoint{2.662526in}{2.323380in}}%
\pgfpathcurveto{\pgfqpoint{2.662526in}{2.315144in}}{\pgfqpoint{2.665798in}{2.307243in}}{\pgfqpoint{2.671622in}{2.301420in}}%
\pgfpathcurveto{\pgfqpoint{2.677446in}{2.295596in}}{\pgfqpoint{2.685346in}{2.292323in}}{\pgfqpoint{2.693582in}{2.292323in}}%
\pgfpathclose%
\pgfusepath{stroke,fill}%
\end{pgfscope}%
\begin{pgfscope}%
\pgfpathrectangle{\pgfqpoint{0.100000in}{0.212622in}}{\pgfqpoint{3.696000in}{3.696000in}}%
\pgfusepath{clip}%
\pgfsetbuttcap%
\pgfsetroundjoin%
\definecolor{currentfill}{rgb}{0.121569,0.466667,0.705882}%
\pgfsetfillcolor{currentfill}%
\pgfsetfillopacity{0.493490}%
\pgfsetlinewidth{1.003750pt}%
\definecolor{currentstroke}{rgb}{0.121569,0.466667,0.705882}%
\pgfsetstrokecolor{currentstroke}%
\pgfsetstrokeopacity{0.493490}%
\pgfsetdash{}{0pt}%
\pgfpathmoveto{\pgfqpoint{2.707530in}{2.287359in}}%
\pgfpathcurveto{\pgfqpoint{2.715766in}{2.287359in}}{\pgfqpoint{2.723666in}{2.290631in}}{\pgfqpoint{2.729490in}{2.296455in}}%
\pgfpathcurveto{\pgfqpoint{2.735314in}{2.302279in}}{\pgfqpoint{2.738586in}{2.310179in}}{\pgfqpoint{2.738586in}{2.318415in}}%
\pgfpathcurveto{\pgfqpoint{2.738586in}{2.326651in}}{\pgfqpoint{2.735314in}{2.334551in}}{\pgfqpoint{2.729490in}{2.340375in}}%
\pgfpathcurveto{\pgfqpoint{2.723666in}{2.346199in}}{\pgfqpoint{2.715766in}{2.349472in}}{\pgfqpoint{2.707530in}{2.349472in}}%
\pgfpathcurveto{\pgfqpoint{2.699294in}{2.349472in}}{\pgfqpoint{2.691394in}{2.346199in}}{\pgfqpoint{2.685570in}{2.340375in}}%
\pgfpathcurveto{\pgfqpoint{2.679746in}{2.334551in}}{\pgfqpoint{2.676473in}{2.326651in}}{\pgfqpoint{2.676473in}{2.318415in}}%
\pgfpathcurveto{\pgfqpoint{2.676473in}{2.310179in}}{\pgfqpoint{2.679746in}{2.302279in}}{\pgfqpoint{2.685570in}{2.296455in}}%
\pgfpathcurveto{\pgfqpoint{2.691394in}{2.290631in}}{\pgfqpoint{2.699294in}{2.287359in}}{\pgfqpoint{2.707530in}{2.287359in}}%
\pgfpathclose%
\pgfusepath{stroke,fill}%
\end{pgfscope}%
\begin{pgfscope}%
\pgfpathrectangle{\pgfqpoint{0.100000in}{0.212622in}}{\pgfqpoint{3.696000in}{3.696000in}}%
\pgfusepath{clip}%
\pgfsetbuttcap%
\pgfsetroundjoin%
\definecolor{currentfill}{rgb}{0.121569,0.466667,0.705882}%
\pgfsetfillcolor{currentfill}%
\pgfsetfillopacity{0.495609}%
\pgfsetlinewidth{1.003750pt}%
\definecolor{currentstroke}{rgb}{0.121569,0.466667,0.705882}%
\pgfsetstrokecolor{currentstroke}%
\pgfsetstrokeopacity{0.495609}%
\pgfsetdash{}{0pt}%
\pgfpathmoveto{\pgfqpoint{1.159444in}{1.692229in}}%
\pgfpathcurveto{\pgfqpoint{1.167680in}{1.692229in}}{\pgfqpoint{1.175581in}{1.695501in}}{\pgfqpoint{1.181404in}{1.701325in}}%
\pgfpathcurveto{\pgfqpoint{1.187228in}{1.707149in}}{\pgfqpoint{1.190501in}{1.715049in}}{\pgfqpoint{1.190501in}{1.723285in}}%
\pgfpathcurveto{\pgfqpoint{1.190501in}{1.731522in}}{\pgfqpoint{1.187228in}{1.739422in}}{\pgfqpoint{1.181404in}{1.745246in}}%
\pgfpathcurveto{\pgfqpoint{1.175581in}{1.751070in}}{\pgfqpoint{1.167680in}{1.754342in}}{\pgfqpoint{1.159444in}{1.754342in}}%
\pgfpathcurveto{\pgfqpoint{1.151208in}{1.754342in}}{\pgfqpoint{1.143308in}{1.751070in}}{\pgfqpoint{1.137484in}{1.745246in}}%
\pgfpathcurveto{\pgfqpoint{1.131660in}{1.739422in}}{\pgfqpoint{1.128388in}{1.731522in}}{\pgfqpoint{1.128388in}{1.723285in}}%
\pgfpathcurveto{\pgfqpoint{1.128388in}{1.715049in}}{\pgfqpoint{1.131660in}{1.707149in}}{\pgfqpoint{1.137484in}{1.701325in}}%
\pgfpathcurveto{\pgfqpoint{1.143308in}{1.695501in}}{\pgfqpoint{1.151208in}{1.692229in}}{\pgfqpoint{1.159444in}{1.692229in}}%
\pgfpathclose%
\pgfusepath{stroke,fill}%
\end{pgfscope}%
\begin{pgfscope}%
\pgfpathrectangle{\pgfqpoint{0.100000in}{0.212622in}}{\pgfqpoint{3.696000in}{3.696000in}}%
\pgfusepath{clip}%
\pgfsetbuttcap%
\pgfsetroundjoin%
\definecolor{currentfill}{rgb}{0.121569,0.466667,0.705882}%
\pgfsetfillcolor{currentfill}%
\pgfsetfillopacity{0.496840}%
\pgfsetlinewidth{1.003750pt}%
\definecolor{currentstroke}{rgb}{0.121569,0.466667,0.705882}%
\pgfsetstrokecolor{currentstroke}%
\pgfsetstrokeopacity{0.496840}%
\pgfsetdash{}{0pt}%
\pgfpathmoveto{\pgfqpoint{2.726375in}{2.291181in}}%
\pgfpathcurveto{\pgfqpoint{2.734611in}{2.291181in}}{\pgfqpoint{2.742512in}{2.294453in}}{\pgfqpoint{2.748335in}{2.300277in}}%
\pgfpathcurveto{\pgfqpoint{2.754159in}{2.306101in}}{\pgfqpoint{2.757432in}{2.314001in}}{\pgfqpoint{2.757432in}{2.322237in}}%
\pgfpathcurveto{\pgfqpoint{2.757432in}{2.330473in}}{\pgfqpoint{2.754159in}{2.338373in}}{\pgfqpoint{2.748335in}{2.344197in}}%
\pgfpathcurveto{\pgfqpoint{2.742512in}{2.350021in}}{\pgfqpoint{2.734611in}{2.353294in}}{\pgfqpoint{2.726375in}{2.353294in}}%
\pgfpathcurveto{\pgfqpoint{2.718139in}{2.353294in}}{\pgfqpoint{2.710239in}{2.350021in}}{\pgfqpoint{2.704415in}{2.344197in}}%
\pgfpathcurveto{\pgfqpoint{2.698591in}{2.338373in}}{\pgfqpoint{2.695319in}{2.330473in}}{\pgfqpoint{2.695319in}{2.322237in}}%
\pgfpathcurveto{\pgfqpoint{2.695319in}{2.314001in}}{\pgfqpoint{2.698591in}{2.306101in}}{\pgfqpoint{2.704415in}{2.300277in}}%
\pgfpathcurveto{\pgfqpoint{2.710239in}{2.294453in}}{\pgfqpoint{2.718139in}{2.291181in}}{\pgfqpoint{2.726375in}{2.291181in}}%
\pgfpathclose%
\pgfusepath{stroke,fill}%
\end{pgfscope}%
\begin{pgfscope}%
\pgfpathrectangle{\pgfqpoint{0.100000in}{0.212622in}}{\pgfqpoint{3.696000in}{3.696000in}}%
\pgfusepath{clip}%
\pgfsetbuttcap%
\pgfsetroundjoin%
\definecolor{currentfill}{rgb}{0.121569,0.466667,0.705882}%
\pgfsetfillcolor{currentfill}%
\pgfsetfillopacity{0.498544}%
\pgfsetlinewidth{1.003750pt}%
\definecolor{currentstroke}{rgb}{0.121569,0.466667,0.705882}%
\pgfsetstrokecolor{currentstroke}%
\pgfsetstrokeopacity{0.498544}%
\pgfsetdash{}{0pt}%
\pgfpathmoveto{\pgfqpoint{2.735648in}{2.288886in}}%
\pgfpathcurveto{\pgfqpoint{2.743884in}{2.288886in}}{\pgfqpoint{2.751784in}{2.292158in}}{\pgfqpoint{2.757608in}{2.297982in}}%
\pgfpathcurveto{\pgfqpoint{2.763432in}{2.303806in}}{\pgfqpoint{2.766705in}{2.311706in}}{\pgfqpoint{2.766705in}{2.319942in}}%
\pgfpathcurveto{\pgfqpoint{2.766705in}{2.328179in}}{\pgfqpoint{2.763432in}{2.336079in}}{\pgfqpoint{2.757608in}{2.341903in}}%
\pgfpathcurveto{\pgfqpoint{2.751784in}{2.347726in}}{\pgfqpoint{2.743884in}{2.350999in}}{\pgfqpoint{2.735648in}{2.350999in}}%
\pgfpathcurveto{\pgfqpoint{2.727412in}{2.350999in}}{\pgfqpoint{2.719512in}{2.347726in}}{\pgfqpoint{2.713688in}{2.341903in}}%
\pgfpathcurveto{\pgfqpoint{2.707864in}{2.336079in}}{\pgfqpoint{2.704592in}{2.328179in}}{\pgfqpoint{2.704592in}{2.319942in}}%
\pgfpathcurveto{\pgfqpoint{2.704592in}{2.311706in}}{\pgfqpoint{2.707864in}{2.303806in}}{\pgfqpoint{2.713688in}{2.297982in}}%
\pgfpathcurveto{\pgfqpoint{2.719512in}{2.292158in}}{\pgfqpoint{2.727412in}{2.288886in}}{\pgfqpoint{2.735648in}{2.288886in}}%
\pgfpathclose%
\pgfusepath{stroke,fill}%
\end{pgfscope}%
\begin{pgfscope}%
\pgfpathrectangle{\pgfqpoint{0.100000in}{0.212622in}}{\pgfqpoint{3.696000in}{3.696000in}}%
\pgfusepath{clip}%
\pgfsetbuttcap%
\pgfsetroundjoin%
\definecolor{currentfill}{rgb}{0.121569,0.466667,0.705882}%
\pgfsetfillcolor{currentfill}%
\pgfsetfillopacity{0.500117}%
\pgfsetlinewidth{1.003750pt}%
\definecolor{currentstroke}{rgb}{0.121569,0.466667,0.705882}%
\pgfsetstrokecolor{currentstroke}%
\pgfsetstrokeopacity{0.500117}%
\pgfsetdash{}{0pt}%
\pgfpathmoveto{\pgfqpoint{1.146426in}{1.674212in}}%
\pgfpathcurveto{\pgfqpoint{1.154662in}{1.674212in}}{\pgfqpoint{1.162562in}{1.677485in}}{\pgfqpoint{1.168386in}{1.683308in}}%
\pgfpathcurveto{\pgfqpoint{1.174210in}{1.689132in}}{\pgfqpoint{1.177482in}{1.697032in}}{\pgfqpoint{1.177482in}{1.705269in}}%
\pgfpathcurveto{\pgfqpoint{1.177482in}{1.713505in}}{\pgfqpoint{1.174210in}{1.721405in}}{\pgfqpoint{1.168386in}{1.727229in}}%
\pgfpathcurveto{\pgfqpoint{1.162562in}{1.733053in}}{\pgfqpoint{1.154662in}{1.736325in}}{\pgfqpoint{1.146426in}{1.736325in}}%
\pgfpathcurveto{\pgfqpoint{1.138189in}{1.736325in}}{\pgfqpoint{1.130289in}{1.733053in}}{\pgfqpoint{1.124465in}{1.727229in}}%
\pgfpathcurveto{\pgfqpoint{1.118641in}{1.721405in}}{\pgfqpoint{1.115369in}{1.713505in}}{\pgfqpoint{1.115369in}{1.705269in}}%
\pgfpathcurveto{\pgfqpoint{1.115369in}{1.697032in}}{\pgfqpoint{1.118641in}{1.689132in}}{\pgfqpoint{1.124465in}{1.683308in}}%
\pgfpathcurveto{\pgfqpoint{1.130289in}{1.677485in}}{\pgfqpoint{1.138189in}{1.674212in}}{\pgfqpoint{1.146426in}{1.674212in}}%
\pgfpathclose%
\pgfusepath{stroke,fill}%
\end{pgfscope}%
\begin{pgfscope}%
\pgfpathrectangle{\pgfqpoint{0.100000in}{0.212622in}}{\pgfqpoint{3.696000in}{3.696000in}}%
\pgfusepath{clip}%
\pgfsetbuttcap%
\pgfsetroundjoin%
\definecolor{currentfill}{rgb}{0.121569,0.466667,0.705882}%
\pgfsetfillcolor{currentfill}%
\pgfsetfillopacity{0.501243}%
\pgfsetlinewidth{1.003750pt}%
\definecolor{currentstroke}{rgb}{0.121569,0.466667,0.705882}%
\pgfsetstrokecolor{currentstroke}%
\pgfsetstrokeopacity{0.501243}%
\pgfsetdash{}{0pt}%
\pgfpathmoveto{\pgfqpoint{2.750327in}{2.285680in}}%
\pgfpathcurveto{\pgfqpoint{2.758564in}{2.285680in}}{\pgfqpoint{2.766464in}{2.288952in}}{\pgfqpoint{2.772288in}{2.294776in}}%
\pgfpathcurveto{\pgfqpoint{2.778111in}{2.300600in}}{\pgfqpoint{2.781384in}{2.308500in}}{\pgfqpoint{2.781384in}{2.316736in}}%
\pgfpathcurveto{\pgfqpoint{2.781384in}{2.324972in}}{\pgfqpoint{2.778111in}{2.332872in}}{\pgfqpoint{2.772288in}{2.338696in}}%
\pgfpathcurveto{\pgfqpoint{2.766464in}{2.344520in}}{\pgfqpoint{2.758564in}{2.347793in}}{\pgfqpoint{2.750327in}{2.347793in}}%
\pgfpathcurveto{\pgfqpoint{2.742091in}{2.347793in}}{\pgfqpoint{2.734191in}{2.344520in}}{\pgfqpoint{2.728367in}{2.338696in}}%
\pgfpathcurveto{\pgfqpoint{2.722543in}{2.332872in}}{\pgfqpoint{2.719271in}{2.324972in}}{\pgfqpoint{2.719271in}{2.316736in}}%
\pgfpathcurveto{\pgfqpoint{2.719271in}{2.308500in}}{\pgfqpoint{2.722543in}{2.300600in}}{\pgfqpoint{2.728367in}{2.294776in}}%
\pgfpathcurveto{\pgfqpoint{2.734191in}{2.288952in}}{\pgfqpoint{2.742091in}{2.285680in}}{\pgfqpoint{2.750327in}{2.285680in}}%
\pgfpathclose%
\pgfusepath{stroke,fill}%
\end{pgfscope}%
\begin{pgfscope}%
\pgfpathrectangle{\pgfqpoint{0.100000in}{0.212622in}}{\pgfqpoint{3.696000in}{3.696000in}}%
\pgfusepath{clip}%
\pgfsetbuttcap%
\pgfsetroundjoin%
\definecolor{currentfill}{rgb}{0.121569,0.466667,0.705882}%
\pgfsetfillcolor{currentfill}%
\pgfsetfillopacity{0.502906}%
\pgfsetlinewidth{1.003750pt}%
\definecolor{currentstroke}{rgb}{0.121569,0.466667,0.705882}%
\pgfsetstrokecolor{currentstroke}%
\pgfsetstrokeopacity{0.502906}%
\pgfsetdash{}{0pt}%
\pgfpathmoveto{\pgfqpoint{1.138963in}{1.662122in}}%
\pgfpathcurveto{\pgfqpoint{1.147199in}{1.662122in}}{\pgfqpoint{1.155099in}{1.665395in}}{\pgfqpoint{1.160923in}{1.671219in}}%
\pgfpathcurveto{\pgfqpoint{1.166747in}{1.677043in}}{\pgfqpoint{1.170020in}{1.684943in}}{\pgfqpoint{1.170020in}{1.693179in}}%
\pgfpathcurveto{\pgfqpoint{1.170020in}{1.701415in}}{\pgfqpoint{1.166747in}{1.709315in}}{\pgfqpoint{1.160923in}{1.715139in}}%
\pgfpathcurveto{\pgfqpoint{1.155099in}{1.720963in}}{\pgfqpoint{1.147199in}{1.724235in}}{\pgfqpoint{1.138963in}{1.724235in}}%
\pgfpathcurveto{\pgfqpoint{1.130727in}{1.724235in}}{\pgfqpoint{1.122827in}{1.720963in}}{\pgfqpoint{1.117003in}{1.715139in}}%
\pgfpathcurveto{\pgfqpoint{1.111179in}{1.709315in}}{\pgfqpoint{1.107907in}{1.701415in}}{\pgfqpoint{1.107907in}{1.693179in}}%
\pgfpathcurveto{\pgfqpoint{1.107907in}{1.684943in}}{\pgfqpoint{1.111179in}{1.677043in}}{\pgfqpoint{1.117003in}{1.671219in}}%
\pgfpathcurveto{\pgfqpoint{1.122827in}{1.665395in}}{\pgfqpoint{1.130727in}{1.662122in}}{\pgfqpoint{1.138963in}{1.662122in}}%
\pgfpathclose%
\pgfusepath{stroke,fill}%
\end{pgfscope}%
\begin{pgfscope}%
\pgfpathrectangle{\pgfqpoint{0.100000in}{0.212622in}}{\pgfqpoint{3.696000in}{3.696000in}}%
\pgfusepath{clip}%
\pgfsetbuttcap%
\pgfsetroundjoin%
\definecolor{currentfill}{rgb}{0.121569,0.466667,0.705882}%
\pgfsetfillcolor{currentfill}%
\pgfsetfillopacity{0.502956}%
\pgfsetlinewidth{1.003750pt}%
\definecolor{currentstroke}{rgb}{0.121569,0.466667,0.705882}%
\pgfsetstrokecolor{currentstroke}%
\pgfsetstrokeopacity{0.502956}%
\pgfsetdash{}{0pt}%
\pgfpathmoveto{\pgfqpoint{2.759416in}{2.288652in}}%
\pgfpathcurveto{\pgfqpoint{2.767652in}{2.288652in}}{\pgfqpoint{2.775552in}{2.291924in}}{\pgfqpoint{2.781376in}{2.297748in}}%
\pgfpathcurveto{\pgfqpoint{2.787200in}{2.303572in}}{\pgfqpoint{2.790472in}{2.311472in}}{\pgfqpoint{2.790472in}{2.319709in}}%
\pgfpathcurveto{\pgfqpoint{2.790472in}{2.327945in}}{\pgfqpoint{2.787200in}{2.335845in}}{\pgfqpoint{2.781376in}{2.341669in}}%
\pgfpathcurveto{\pgfqpoint{2.775552in}{2.347493in}}{\pgfqpoint{2.767652in}{2.350765in}}{\pgfqpoint{2.759416in}{2.350765in}}%
\pgfpathcurveto{\pgfqpoint{2.751180in}{2.350765in}}{\pgfqpoint{2.743280in}{2.347493in}}{\pgfqpoint{2.737456in}{2.341669in}}%
\pgfpathcurveto{\pgfqpoint{2.731632in}{2.335845in}}{\pgfqpoint{2.728359in}{2.327945in}}{\pgfqpoint{2.728359in}{2.319709in}}%
\pgfpathcurveto{\pgfqpoint{2.728359in}{2.311472in}}{\pgfqpoint{2.731632in}{2.303572in}}{\pgfqpoint{2.737456in}{2.297748in}}%
\pgfpathcurveto{\pgfqpoint{2.743280in}{2.291924in}}{\pgfqpoint{2.751180in}{2.288652in}}{\pgfqpoint{2.759416in}{2.288652in}}%
\pgfpathclose%
\pgfusepath{stroke,fill}%
\end{pgfscope}%
\begin{pgfscope}%
\pgfpathrectangle{\pgfqpoint{0.100000in}{0.212622in}}{\pgfqpoint{3.696000in}{3.696000in}}%
\pgfusepath{clip}%
\pgfsetbuttcap%
\pgfsetroundjoin%
\definecolor{currentfill}{rgb}{0.121569,0.466667,0.705882}%
\pgfsetfillcolor{currentfill}%
\pgfsetfillopacity{0.503785}%
\pgfsetlinewidth{1.003750pt}%
\definecolor{currentstroke}{rgb}{0.121569,0.466667,0.705882}%
\pgfsetstrokecolor{currentstroke}%
\pgfsetstrokeopacity{0.503785}%
\pgfsetdash{}{0pt}%
\pgfpathmoveto{\pgfqpoint{2.763765in}{2.287495in}}%
\pgfpathcurveto{\pgfqpoint{2.772001in}{2.287495in}}{\pgfqpoint{2.779901in}{2.290768in}}{\pgfqpoint{2.785725in}{2.296592in}}%
\pgfpathcurveto{\pgfqpoint{2.791549in}{2.302415in}}{\pgfqpoint{2.794821in}{2.310315in}}{\pgfqpoint{2.794821in}{2.318552in}}%
\pgfpathcurveto{\pgfqpoint{2.794821in}{2.326788in}}{\pgfqpoint{2.791549in}{2.334688in}}{\pgfqpoint{2.785725in}{2.340512in}}%
\pgfpathcurveto{\pgfqpoint{2.779901in}{2.346336in}}{\pgfqpoint{2.772001in}{2.349608in}}{\pgfqpoint{2.763765in}{2.349608in}}%
\pgfpathcurveto{\pgfqpoint{2.755528in}{2.349608in}}{\pgfqpoint{2.747628in}{2.346336in}}{\pgfqpoint{2.741804in}{2.340512in}}%
\pgfpathcurveto{\pgfqpoint{2.735980in}{2.334688in}}{\pgfqpoint{2.732708in}{2.326788in}}{\pgfqpoint{2.732708in}{2.318552in}}%
\pgfpathcurveto{\pgfqpoint{2.732708in}{2.310315in}}{\pgfqpoint{2.735980in}{2.302415in}}{\pgfqpoint{2.741804in}{2.296592in}}%
\pgfpathcurveto{\pgfqpoint{2.747628in}{2.290768in}}{\pgfqpoint{2.755528in}{2.287495in}}{\pgfqpoint{2.763765in}{2.287495in}}%
\pgfpathclose%
\pgfusepath{stroke,fill}%
\end{pgfscope}%
\begin{pgfscope}%
\pgfpathrectangle{\pgfqpoint{0.100000in}{0.212622in}}{\pgfqpoint{3.696000in}{3.696000in}}%
\pgfusepath{clip}%
\pgfsetbuttcap%
\pgfsetroundjoin%
\definecolor{currentfill}{rgb}{0.121569,0.466667,0.705882}%
\pgfsetfillcolor{currentfill}%
\pgfsetfillopacity{0.504296}%
\pgfsetlinewidth{1.003750pt}%
\definecolor{currentstroke}{rgb}{0.121569,0.466667,0.705882}%
\pgfsetstrokecolor{currentstroke}%
\pgfsetstrokeopacity{0.504296}%
\pgfsetdash{}{0pt}%
\pgfpathmoveto{\pgfqpoint{1.134404in}{1.658137in}}%
\pgfpathcurveto{\pgfqpoint{1.142640in}{1.658137in}}{\pgfqpoint{1.150540in}{1.661409in}}{\pgfqpoint{1.156364in}{1.667233in}}%
\pgfpathcurveto{\pgfqpoint{1.162188in}{1.673057in}}{\pgfqpoint{1.165460in}{1.680957in}}{\pgfqpoint{1.165460in}{1.689193in}}%
\pgfpathcurveto{\pgfqpoint{1.165460in}{1.697430in}}{\pgfqpoint{1.162188in}{1.705330in}}{\pgfqpoint{1.156364in}{1.711154in}}%
\pgfpathcurveto{\pgfqpoint{1.150540in}{1.716978in}}{\pgfqpoint{1.142640in}{1.720250in}}{\pgfqpoint{1.134404in}{1.720250in}}%
\pgfpathcurveto{\pgfqpoint{1.126167in}{1.720250in}}{\pgfqpoint{1.118267in}{1.716978in}}{\pgfqpoint{1.112443in}{1.711154in}}%
\pgfpathcurveto{\pgfqpoint{1.106619in}{1.705330in}}{\pgfqpoint{1.103347in}{1.697430in}}{\pgfqpoint{1.103347in}{1.689193in}}%
\pgfpathcurveto{\pgfqpoint{1.103347in}{1.680957in}}{\pgfqpoint{1.106619in}{1.673057in}}{\pgfqpoint{1.112443in}{1.667233in}}%
\pgfpathcurveto{\pgfqpoint{1.118267in}{1.661409in}}{\pgfqpoint{1.126167in}{1.658137in}}{\pgfqpoint{1.134404in}{1.658137in}}%
\pgfpathclose%
\pgfusepath{stroke,fill}%
\end{pgfscope}%
\begin{pgfscope}%
\pgfpathrectangle{\pgfqpoint{0.100000in}{0.212622in}}{\pgfqpoint{3.696000in}{3.696000in}}%
\pgfusepath{clip}%
\pgfsetbuttcap%
\pgfsetroundjoin%
\definecolor{currentfill}{rgb}{0.121569,0.466667,0.705882}%
\pgfsetfillcolor{currentfill}%
\pgfsetfillopacity{0.504824}%
\pgfsetlinewidth{1.003750pt}%
\definecolor{currentstroke}{rgb}{0.121569,0.466667,0.705882}%
\pgfsetstrokecolor{currentstroke}%
\pgfsetstrokeopacity{0.504824}%
\pgfsetdash{}{0pt}%
\pgfpathmoveto{\pgfqpoint{1.133108in}{1.655823in}}%
\pgfpathcurveto{\pgfqpoint{1.141344in}{1.655823in}}{\pgfqpoint{1.149244in}{1.659096in}}{\pgfqpoint{1.155068in}{1.664919in}}%
\pgfpathcurveto{\pgfqpoint{1.160892in}{1.670743in}}{\pgfqpoint{1.164164in}{1.678643in}}{\pgfqpoint{1.164164in}{1.686880in}}%
\pgfpathcurveto{\pgfqpoint{1.164164in}{1.695116in}}{\pgfqpoint{1.160892in}{1.703016in}}{\pgfqpoint{1.155068in}{1.708840in}}%
\pgfpathcurveto{\pgfqpoint{1.149244in}{1.714664in}}{\pgfqpoint{1.141344in}{1.717936in}}{\pgfqpoint{1.133108in}{1.717936in}}%
\pgfpathcurveto{\pgfqpoint{1.124871in}{1.717936in}}{\pgfqpoint{1.116971in}{1.714664in}}{\pgfqpoint{1.111147in}{1.708840in}}%
\pgfpathcurveto{\pgfqpoint{1.105324in}{1.703016in}}{\pgfqpoint{1.102051in}{1.695116in}}{\pgfqpoint{1.102051in}{1.686880in}}%
\pgfpathcurveto{\pgfqpoint{1.102051in}{1.678643in}}{\pgfqpoint{1.105324in}{1.670743in}}{\pgfqpoint{1.111147in}{1.664919in}}%
\pgfpathcurveto{\pgfqpoint{1.116971in}{1.659096in}}{\pgfqpoint{1.124871in}{1.655823in}}{\pgfqpoint{1.133108in}{1.655823in}}%
\pgfpathclose%
\pgfusepath{stroke,fill}%
\end{pgfscope}%
\begin{pgfscope}%
\pgfpathrectangle{\pgfqpoint{0.100000in}{0.212622in}}{\pgfqpoint{3.696000in}{3.696000in}}%
\pgfusepath{clip}%
\pgfsetbuttcap%
\pgfsetroundjoin%
\definecolor{currentfill}{rgb}{0.121569,0.466667,0.705882}%
\pgfsetfillcolor{currentfill}%
\pgfsetfillopacity{0.505136}%
\pgfsetlinewidth{1.003750pt}%
\definecolor{currentstroke}{rgb}{0.121569,0.466667,0.705882}%
\pgfsetstrokecolor{currentstroke}%
\pgfsetstrokeopacity{0.505136}%
\pgfsetdash{}{0pt}%
\pgfpathmoveto{\pgfqpoint{2.770632in}{2.290021in}}%
\pgfpathcurveto{\pgfqpoint{2.778868in}{2.290021in}}{\pgfqpoint{2.786768in}{2.293293in}}{\pgfqpoint{2.792592in}{2.299117in}}%
\pgfpathcurveto{\pgfqpoint{2.798416in}{2.304941in}}{\pgfqpoint{2.801688in}{2.312841in}}{\pgfqpoint{2.801688in}{2.321077in}}%
\pgfpathcurveto{\pgfqpoint{2.801688in}{2.329313in}}{\pgfqpoint{2.798416in}{2.337213in}}{\pgfqpoint{2.792592in}{2.343037in}}%
\pgfpathcurveto{\pgfqpoint{2.786768in}{2.348861in}}{\pgfqpoint{2.778868in}{2.352133in}}{\pgfqpoint{2.770632in}{2.352133in}}%
\pgfpathcurveto{\pgfqpoint{2.762395in}{2.352133in}}{\pgfqpoint{2.754495in}{2.348861in}}{\pgfqpoint{2.748671in}{2.343037in}}%
\pgfpathcurveto{\pgfqpoint{2.742847in}{2.337213in}}{\pgfqpoint{2.739575in}{2.329313in}}{\pgfqpoint{2.739575in}{2.321077in}}%
\pgfpathcurveto{\pgfqpoint{2.739575in}{2.312841in}}{\pgfqpoint{2.742847in}{2.304941in}}{\pgfqpoint{2.748671in}{2.299117in}}%
\pgfpathcurveto{\pgfqpoint{2.754495in}{2.293293in}}{\pgfqpoint{2.762395in}{2.290021in}}{\pgfqpoint{2.770632in}{2.290021in}}%
\pgfpathclose%
\pgfusepath{stroke,fill}%
\end{pgfscope}%
\begin{pgfscope}%
\pgfpathrectangle{\pgfqpoint{0.100000in}{0.212622in}}{\pgfqpoint{3.696000in}{3.696000in}}%
\pgfusepath{clip}%
\pgfsetbuttcap%
\pgfsetroundjoin%
\definecolor{currentfill}{rgb}{0.121569,0.466667,0.705882}%
\pgfsetfillcolor{currentfill}%
\pgfsetfillopacity{0.505831}%
\pgfsetlinewidth{1.003750pt}%
\definecolor{currentstroke}{rgb}{0.121569,0.466667,0.705882}%
\pgfsetstrokecolor{currentstroke}%
\pgfsetstrokeopacity{0.505831}%
\pgfsetdash{}{0pt}%
\pgfpathmoveto{\pgfqpoint{1.130400in}{1.652200in}}%
\pgfpathcurveto{\pgfqpoint{1.138636in}{1.652200in}}{\pgfqpoint{1.146536in}{1.655472in}}{\pgfqpoint{1.152360in}{1.661296in}}%
\pgfpathcurveto{\pgfqpoint{1.158184in}{1.667120in}}{\pgfqpoint{1.161456in}{1.675020in}}{\pgfqpoint{1.161456in}{1.683256in}}%
\pgfpathcurveto{\pgfqpoint{1.161456in}{1.691492in}}{\pgfqpoint{1.158184in}{1.699392in}}{\pgfqpoint{1.152360in}{1.705216in}}%
\pgfpathcurveto{\pgfqpoint{1.146536in}{1.711040in}}{\pgfqpoint{1.138636in}{1.714313in}}{\pgfqpoint{1.130400in}{1.714313in}}%
\pgfpathcurveto{\pgfqpoint{1.122163in}{1.714313in}}{\pgfqpoint{1.114263in}{1.711040in}}{\pgfqpoint{1.108439in}{1.705216in}}%
\pgfpathcurveto{\pgfqpoint{1.102615in}{1.699392in}}{\pgfqpoint{1.099343in}{1.691492in}}{\pgfqpoint{1.099343in}{1.683256in}}%
\pgfpathcurveto{\pgfqpoint{1.099343in}{1.675020in}}{\pgfqpoint{1.102615in}{1.667120in}}{\pgfqpoint{1.108439in}{1.661296in}}%
\pgfpathcurveto{\pgfqpoint{1.114263in}{1.655472in}}{\pgfqpoint{1.122163in}{1.652200in}}{\pgfqpoint{1.130400in}{1.652200in}}%
\pgfpathclose%
\pgfusepath{stroke,fill}%
\end{pgfscope}%
\begin{pgfscope}%
\pgfpathrectangle{\pgfqpoint{0.100000in}{0.212622in}}{\pgfqpoint{3.696000in}{3.696000in}}%
\pgfusepath{clip}%
\pgfsetbuttcap%
\pgfsetroundjoin%
\definecolor{currentfill}{rgb}{0.121569,0.466667,0.705882}%
\pgfsetfillcolor{currentfill}%
\pgfsetfillopacity{0.506402}%
\pgfsetlinewidth{1.003750pt}%
\definecolor{currentstroke}{rgb}{0.121569,0.466667,0.705882}%
\pgfsetstrokecolor{currentstroke}%
\pgfsetstrokeopacity{0.506402}%
\pgfsetdash{}{0pt}%
\pgfpathmoveto{\pgfqpoint{2.778375in}{2.286202in}}%
\pgfpathcurveto{\pgfqpoint{2.786612in}{2.286202in}}{\pgfqpoint{2.794512in}{2.289475in}}{\pgfqpoint{2.800336in}{2.295299in}}%
\pgfpathcurveto{\pgfqpoint{2.806160in}{2.301123in}}{\pgfqpoint{2.809432in}{2.309023in}}{\pgfqpoint{2.809432in}{2.317259in}}%
\pgfpathcurveto{\pgfqpoint{2.809432in}{2.325495in}}{\pgfqpoint{2.806160in}{2.333395in}}{\pgfqpoint{2.800336in}{2.339219in}}%
\pgfpathcurveto{\pgfqpoint{2.794512in}{2.345043in}}{\pgfqpoint{2.786612in}{2.348315in}}{\pgfqpoint{2.778375in}{2.348315in}}%
\pgfpathcurveto{\pgfqpoint{2.770139in}{2.348315in}}{\pgfqpoint{2.762239in}{2.345043in}}{\pgfqpoint{2.756415in}{2.339219in}}%
\pgfpathcurveto{\pgfqpoint{2.750591in}{2.333395in}}{\pgfqpoint{2.747319in}{2.325495in}}{\pgfqpoint{2.747319in}{2.317259in}}%
\pgfpathcurveto{\pgfqpoint{2.747319in}{2.309023in}}{\pgfqpoint{2.750591in}{2.301123in}}{\pgfqpoint{2.756415in}{2.295299in}}%
\pgfpathcurveto{\pgfqpoint{2.762239in}{2.289475in}}{\pgfqpoint{2.770139in}{2.286202in}}{\pgfqpoint{2.778375in}{2.286202in}}%
\pgfpathclose%
\pgfusepath{stroke,fill}%
\end{pgfscope}%
\begin{pgfscope}%
\pgfpathrectangle{\pgfqpoint{0.100000in}{0.212622in}}{\pgfqpoint{3.696000in}{3.696000in}}%
\pgfusepath{clip}%
\pgfsetbuttcap%
\pgfsetroundjoin%
\definecolor{currentfill}{rgb}{0.121569,0.466667,0.705882}%
\pgfsetfillcolor{currentfill}%
\pgfsetfillopacity{0.507631}%
\pgfsetlinewidth{1.003750pt}%
\definecolor{currentstroke}{rgb}{0.121569,0.466667,0.705882}%
\pgfsetstrokecolor{currentstroke}%
\pgfsetstrokeopacity{0.507631}%
\pgfsetdash{}{0pt}%
\pgfpathmoveto{\pgfqpoint{1.125702in}{1.645213in}}%
\pgfpathcurveto{\pgfqpoint{1.133938in}{1.645213in}}{\pgfqpoint{1.141838in}{1.648485in}}{\pgfqpoint{1.147662in}{1.654309in}}%
\pgfpathcurveto{\pgfqpoint{1.153486in}{1.660133in}}{\pgfqpoint{1.156758in}{1.668033in}}{\pgfqpoint{1.156758in}{1.676269in}}%
\pgfpathcurveto{\pgfqpoint{1.156758in}{1.684505in}}{\pgfqpoint{1.153486in}{1.692406in}}{\pgfqpoint{1.147662in}{1.698229in}}%
\pgfpathcurveto{\pgfqpoint{1.141838in}{1.704053in}}{\pgfqpoint{1.133938in}{1.707326in}}{\pgfqpoint{1.125702in}{1.707326in}}%
\pgfpathcurveto{\pgfqpoint{1.117466in}{1.707326in}}{\pgfqpoint{1.109566in}{1.704053in}}{\pgfqpoint{1.103742in}{1.698229in}}%
\pgfpathcurveto{\pgfqpoint{1.097918in}{1.692406in}}{\pgfqpoint{1.094645in}{1.684505in}}{\pgfqpoint{1.094645in}{1.676269in}}%
\pgfpathcurveto{\pgfqpoint{1.094645in}{1.668033in}}{\pgfqpoint{1.097918in}{1.660133in}}{\pgfqpoint{1.103742in}{1.654309in}}%
\pgfpathcurveto{\pgfqpoint{1.109566in}{1.648485in}}{\pgfqpoint{1.117466in}{1.645213in}}{\pgfqpoint{1.125702in}{1.645213in}}%
\pgfpathclose%
\pgfusepath{stroke,fill}%
\end{pgfscope}%
\begin{pgfscope}%
\pgfpathrectangle{\pgfqpoint{0.100000in}{0.212622in}}{\pgfqpoint{3.696000in}{3.696000in}}%
\pgfusepath{clip}%
\pgfsetbuttcap%
\pgfsetroundjoin%
\definecolor{currentfill}{rgb}{0.121569,0.466667,0.705882}%
\pgfsetfillcolor{currentfill}%
\pgfsetfillopacity{0.508817}%
\pgfsetlinewidth{1.003750pt}%
\definecolor{currentstroke}{rgb}{0.121569,0.466667,0.705882}%
\pgfsetstrokecolor{currentstroke}%
\pgfsetstrokeopacity{0.508817}%
\pgfsetdash{}{0pt}%
\pgfpathmoveto{\pgfqpoint{2.791109in}{2.288496in}}%
\pgfpathcurveto{\pgfqpoint{2.799346in}{2.288496in}}{\pgfqpoint{2.807246in}{2.291768in}}{\pgfqpoint{2.813070in}{2.297592in}}%
\pgfpathcurveto{\pgfqpoint{2.818894in}{2.303416in}}{\pgfqpoint{2.822166in}{2.311316in}}{\pgfqpoint{2.822166in}{2.319552in}}%
\pgfpathcurveto{\pgfqpoint{2.822166in}{2.327789in}}{\pgfqpoint{2.818894in}{2.335689in}}{\pgfqpoint{2.813070in}{2.341513in}}%
\pgfpathcurveto{\pgfqpoint{2.807246in}{2.347337in}}{\pgfqpoint{2.799346in}{2.350609in}}{\pgfqpoint{2.791109in}{2.350609in}}%
\pgfpathcurveto{\pgfqpoint{2.782873in}{2.350609in}}{\pgfqpoint{2.774973in}{2.347337in}}{\pgfqpoint{2.769149in}{2.341513in}}%
\pgfpathcurveto{\pgfqpoint{2.763325in}{2.335689in}}{\pgfqpoint{2.760053in}{2.327789in}}{\pgfqpoint{2.760053in}{2.319552in}}%
\pgfpathcurveto{\pgfqpoint{2.760053in}{2.311316in}}{\pgfqpoint{2.763325in}{2.303416in}}{\pgfqpoint{2.769149in}{2.297592in}}%
\pgfpathcurveto{\pgfqpoint{2.774973in}{2.291768in}}{\pgfqpoint{2.782873in}{2.288496in}}{\pgfqpoint{2.791109in}{2.288496in}}%
\pgfpathclose%
\pgfusepath{stroke,fill}%
\end{pgfscope}%
\begin{pgfscope}%
\pgfpathrectangle{\pgfqpoint{0.100000in}{0.212622in}}{\pgfqpoint{3.696000in}{3.696000in}}%
\pgfusepath{clip}%
\pgfsetbuttcap%
\pgfsetroundjoin%
\definecolor{currentfill}{rgb}{0.121569,0.466667,0.705882}%
\pgfsetfillcolor{currentfill}%
\pgfsetfillopacity{0.509986}%
\pgfsetlinewidth{1.003750pt}%
\definecolor{currentstroke}{rgb}{0.121569,0.466667,0.705882}%
\pgfsetstrokecolor{currentstroke}%
\pgfsetstrokeopacity{0.509986}%
\pgfsetdash{}{0pt}%
\pgfpathmoveto{\pgfqpoint{2.797493in}{2.286843in}}%
\pgfpathcurveto{\pgfqpoint{2.805729in}{2.286843in}}{\pgfqpoint{2.813629in}{2.290115in}}{\pgfqpoint{2.819453in}{2.295939in}}%
\pgfpathcurveto{\pgfqpoint{2.825277in}{2.301763in}}{\pgfqpoint{2.828549in}{2.309663in}}{\pgfqpoint{2.828549in}{2.317899in}}%
\pgfpathcurveto{\pgfqpoint{2.828549in}{2.326136in}}{\pgfqpoint{2.825277in}{2.334036in}}{\pgfqpoint{2.819453in}{2.339860in}}%
\pgfpathcurveto{\pgfqpoint{2.813629in}{2.345684in}}{\pgfqpoint{2.805729in}{2.348956in}}{\pgfqpoint{2.797493in}{2.348956in}}%
\pgfpathcurveto{\pgfqpoint{2.789257in}{2.348956in}}{\pgfqpoint{2.781356in}{2.345684in}}{\pgfqpoint{2.775533in}{2.339860in}}%
\pgfpathcurveto{\pgfqpoint{2.769709in}{2.334036in}}{\pgfqpoint{2.766436in}{2.326136in}}{\pgfqpoint{2.766436in}{2.317899in}}%
\pgfpathcurveto{\pgfqpoint{2.766436in}{2.309663in}}{\pgfqpoint{2.769709in}{2.301763in}}{\pgfqpoint{2.775533in}{2.295939in}}%
\pgfpathcurveto{\pgfqpoint{2.781356in}{2.290115in}}{\pgfqpoint{2.789257in}{2.286843in}}{\pgfqpoint{2.797493in}{2.286843in}}%
\pgfpathclose%
\pgfusepath{stroke,fill}%
\end{pgfscope}%
\begin{pgfscope}%
\pgfpathrectangle{\pgfqpoint{0.100000in}{0.212622in}}{\pgfqpoint{3.696000in}{3.696000in}}%
\pgfusepath{clip}%
\pgfsetbuttcap%
\pgfsetroundjoin%
\definecolor{currentfill}{rgb}{0.121569,0.466667,0.705882}%
\pgfsetfillcolor{currentfill}%
\pgfsetfillopacity{0.510691}%
\pgfsetlinewidth{1.003750pt}%
\definecolor{currentstroke}{rgb}{0.121569,0.466667,0.705882}%
\pgfsetstrokecolor{currentstroke}%
\pgfsetstrokeopacity{0.510691}%
\pgfsetdash{}{0pt}%
\pgfpathmoveto{\pgfqpoint{1.117823in}{1.630539in}}%
\pgfpathcurveto{\pgfqpoint{1.126059in}{1.630539in}}{\pgfqpoint{1.133959in}{1.633811in}}{\pgfqpoint{1.139783in}{1.639635in}}%
\pgfpathcurveto{\pgfqpoint{1.145607in}{1.645459in}}{\pgfqpoint{1.148879in}{1.653359in}}{\pgfqpoint{1.148879in}{1.661595in}}%
\pgfpathcurveto{\pgfqpoint{1.148879in}{1.669832in}}{\pgfqpoint{1.145607in}{1.677732in}}{\pgfqpoint{1.139783in}{1.683556in}}%
\pgfpathcurveto{\pgfqpoint{1.133959in}{1.689379in}}{\pgfqpoint{1.126059in}{1.692652in}}{\pgfqpoint{1.117823in}{1.692652in}}%
\pgfpathcurveto{\pgfqpoint{1.109586in}{1.692652in}}{\pgfqpoint{1.101686in}{1.689379in}}{\pgfqpoint{1.095862in}{1.683556in}}%
\pgfpathcurveto{\pgfqpoint{1.090038in}{1.677732in}}{\pgfqpoint{1.086766in}{1.669832in}}{\pgfqpoint{1.086766in}{1.661595in}}%
\pgfpathcurveto{\pgfqpoint{1.086766in}{1.653359in}}{\pgfqpoint{1.090038in}{1.645459in}}{\pgfqpoint{1.095862in}{1.639635in}}%
\pgfpathcurveto{\pgfqpoint{1.101686in}{1.633811in}}{\pgfqpoint{1.109586in}{1.630539in}}{\pgfqpoint{1.117823in}{1.630539in}}%
\pgfpathclose%
\pgfusepath{stroke,fill}%
\end{pgfscope}%
\begin{pgfscope}%
\pgfpathrectangle{\pgfqpoint{0.100000in}{0.212622in}}{\pgfqpoint{3.696000in}{3.696000in}}%
\pgfusepath{clip}%
\pgfsetbuttcap%
\pgfsetroundjoin%
\definecolor{currentfill}{rgb}{0.121569,0.466667,0.705882}%
\pgfsetfillcolor{currentfill}%
\pgfsetfillopacity{0.510730}%
\pgfsetlinewidth{1.003750pt}%
\definecolor{currentstroke}{rgb}{0.121569,0.466667,0.705882}%
\pgfsetstrokecolor{currentstroke}%
\pgfsetstrokeopacity{0.510730}%
\pgfsetdash{}{0pt}%
\pgfpathmoveto{\pgfqpoint{2.801364in}{2.287669in}}%
\pgfpathcurveto{\pgfqpoint{2.809601in}{2.287669in}}{\pgfqpoint{2.817501in}{2.290941in}}{\pgfqpoint{2.823325in}{2.296765in}}%
\pgfpathcurveto{\pgfqpoint{2.829149in}{2.302589in}}{\pgfqpoint{2.832421in}{2.310489in}}{\pgfqpoint{2.832421in}{2.318725in}}%
\pgfpathcurveto{\pgfqpoint{2.832421in}{2.326962in}}{\pgfqpoint{2.829149in}{2.334862in}}{\pgfqpoint{2.823325in}{2.340686in}}%
\pgfpathcurveto{\pgfqpoint{2.817501in}{2.346510in}}{\pgfqpoint{2.809601in}{2.349782in}}{\pgfqpoint{2.801364in}{2.349782in}}%
\pgfpathcurveto{\pgfqpoint{2.793128in}{2.349782in}}{\pgfqpoint{2.785228in}{2.346510in}}{\pgfqpoint{2.779404in}{2.340686in}}%
\pgfpathcurveto{\pgfqpoint{2.773580in}{2.334862in}}{\pgfqpoint{2.770308in}{2.326962in}}{\pgfqpoint{2.770308in}{2.318725in}}%
\pgfpathcurveto{\pgfqpoint{2.770308in}{2.310489in}}{\pgfqpoint{2.773580in}{2.302589in}}{\pgfqpoint{2.779404in}{2.296765in}}%
\pgfpathcurveto{\pgfqpoint{2.785228in}{2.290941in}}{\pgfqpoint{2.793128in}{2.287669in}}{\pgfqpoint{2.801364in}{2.287669in}}%
\pgfpathclose%
\pgfusepath{stroke,fill}%
\end{pgfscope}%
\begin{pgfscope}%
\pgfpathrectangle{\pgfqpoint{0.100000in}{0.212622in}}{\pgfqpoint{3.696000in}{3.696000in}}%
\pgfusepath{clip}%
\pgfsetbuttcap%
\pgfsetroundjoin%
\definecolor{currentfill}{rgb}{0.121569,0.466667,0.705882}%
\pgfsetfillcolor{currentfill}%
\pgfsetfillopacity{0.511115}%
\pgfsetlinewidth{1.003750pt}%
\definecolor{currentstroke}{rgb}{0.121569,0.466667,0.705882}%
\pgfsetstrokecolor{currentstroke}%
\pgfsetstrokeopacity{0.511115}%
\pgfsetdash{}{0pt}%
\pgfpathmoveto{\pgfqpoint{2.803199in}{2.287122in}}%
\pgfpathcurveto{\pgfqpoint{2.811435in}{2.287122in}}{\pgfqpoint{2.819335in}{2.290395in}}{\pgfqpoint{2.825159in}{2.296219in}}%
\pgfpathcurveto{\pgfqpoint{2.830983in}{2.302042in}}{\pgfqpoint{2.834255in}{2.309943in}}{\pgfqpoint{2.834255in}{2.318179in}}%
\pgfpathcurveto{\pgfqpoint{2.834255in}{2.326415in}}{\pgfqpoint{2.830983in}{2.334315in}}{\pgfqpoint{2.825159in}{2.340139in}}%
\pgfpathcurveto{\pgfqpoint{2.819335in}{2.345963in}}{\pgfqpoint{2.811435in}{2.349235in}}{\pgfqpoint{2.803199in}{2.349235in}}%
\pgfpathcurveto{\pgfqpoint{2.794963in}{2.349235in}}{\pgfqpoint{2.787062in}{2.345963in}}{\pgfqpoint{2.781239in}{2.340139in}}%
\pgfpathcurveto{\pgfqpoint{2.775415in}{2.334315in}}{\pgfqpoint{2.772142in}{2.326415in}}{\pgfqpoint{2.772142in}{2.318179in}}%
\pgfpathcurveto{\pgfqpoint{2.772142in}{2.309943in}}{\pgfqpoint{2.775415in}{2.302042in}}{\pgfqpoint{2.781239in}{2.296219in}}%
\pgfpathcurveto{\pgfqpoint{2.787062in}{2.290395in}}{\pgfqpoint{2.794963in}{2.287122in}}{\pgfqpoint{2.803199in}{2.287122in}}%
\pgfpathclose%
\pgfusepath{stroke,fill}%
\end{pgfscope}%
\begin{pgfscope}%
\pgfpathrectangle{\pgfqpoint{0.100000in}{0.212622in}}{\pgfqpoint{3.696000in}{3.696000in}}%
\pgfusepath{clip}%
\pgfsetbuttcap%
\pgfsetroundjoin%
\definecolor{currentfill}{rgb}{0.121569,0.466667,0.705882}%
\pgfsetfillcolor{currentfill}%
\pgfsetfillopacity{0.512315}%
\pgfsetlinewidth{1.003750pt}%
\definecolor{currentstroke}{rgb}{0.121569,0.466667,0.705882}%
\pgfsetstrokecolor{currentstroke}%
\pgfsetstrokeopacity{0.512315}%
\pgfsetdash{}{0pt}%
\pgfpathmoveto{\pgfqpoint{2.810371in}{2.287281in}}%
\pgfpathcurveto{\pgfqpoint{2.818607in}{2.287281in}}{\pgfqpoint{2.826507in}{2.290553in}}{\pgfqpoint{2.832331in}{2.296377in}}%
\pgfpathcurveto{\pgfqpoint{2.838155in}{2.302201in}}{\pgfqpoint{2.841427in}{2.310101in}}{\pgfqpoint{2.841427in}{2.318337in}}%
\pgfpathcurveto{\pgfqpoint{2.841427in}{2.326573in}}{\pgfqpoint{2.838155in}{2.334473in}}{\pgfqpoint{2.832331in}{2.340297in}}%
\pgfpathcurveto{\pgfqpoint{2.826507in}{2.346121in}}{\pgfqpoint{2.818607in}{2.349394in}}{\pgfqpoint{2.810371in}{2.349394in}}%
\pgfpathcurveto{\pgfqpoint{2.802135in}{2.349394in}}{\pgfqpoint{2.794234in}{2.346121in}}{\pgfqpoint{2.788411in}{2.340297in}}%
\pgfpathcurveto{\pgfqpoint{2.782587in}{2.334473in}}{\pgfqpoint{2.779314in}{2.326573in}}{\pgfqpoint{2.779314in}{2.318337in}}%
\pgfpathcurveto{\pgfqpoint{2.779314in}{2.310101in}}{\pgfqpoint{2.782587in}{2.302201in}}{\pgfqpoint{2.788411in}{2.296377in}}%
\pgfpathcurveto{\pgfqpoint{2.794234in}{2.290553in}}{\pgfqpoint{2.802135in}{2.287281in}}{\pgfqpoint{2.810371in}{2.287281in}}%
\pgfpathclose%
\pgfusepath{stroke,fill}%
\end{pgfscope}%
\begin{pgfscope}%
\pgfpathrectangle{\pgfqpoint{0.100000in}{0.212622in}}{\pgfqpoint{3.696000in}{3.696000in}}%
\pgfusepath{clip}%
\pgfsetbuttcap%
\pgfsetroundjoin%
\definecolor{currentfill}{rgb}{0.121569,0.466667,0.705882}%
\pgfsetfillcolor{currentfill}%
\pgfsetfillopacity{0.514858}%
\pgfsetlinewidth{1.003750pt}%
\definecolor{currentstroke}{rgb}{0.121569,0.466667,0.705882}%
\pgfsetstrokecolor{currentstroke}%
\pgfsetstrokeopacity{0.514858}%
\pgfsetdash{}{0pt}%
\pgfpathmoveto{\pgfqpoint{2.823270in}{2.287602in}}%
\pgfpathcurveto{\pgfqpoint{2.831507in}{2.287602in}}{\pgfqpoint{2.839407in}{2.290874in}}{\pgfqpoint{2.845231in}{2.296698in}}%
\pgfpathcurveto{\pgfqpoint{2.851055in}{2.302522in}}{\pgfqpoint{2.854327in}{2.310422in}}{\pgfqpoint{2.854327in}{2.318659in}}%
\pgfpathcurveto{\pgfqpoint{2.854327in}{2.326895in}}{\pgfqpoint{2.851055in}{2.334795in}}{\pgfqpoint{2.845231in}{2.340619in}}%
\pgfpathcurveto{\pgfqpoint{2.839407in}{2.346443in}}{\pgfqpoint{2.831507in}{2.349715in}}{\pgfqpoint{2.823270in}{2.349715in}}%
\pgfpathcurveto{\pgfqpoint{2.815034in}{2.349715in}}{\pgfqpoint{2.807134in}{2.346443in}}{\pgfqpoint{2.801310in}{2.340619in}}%
\pgfpathcurveto{\pgfqpoint{2.795486in}{2.334795in}}{\pgfqpoint{2.792214in}{2.326895in}}{\pgfqpoint{2.792214in}{2.318659in}}%
\pgfpathcurveto{\pgfqpoint{2.792214in}{2.310422in}}{\pgfqpoint{2.795486in}{2.302522in}}{\pgfqpoint{2.801310in}{2.296698in}}%
\pgfpathcurveto{\pgfqpoint{2.807134in}{2.290874in}}{\pgfqpoint{2.815034in}{2.287602in}}{\pgfqpoint{2.823270in}{2.287602in}}%
\pgfpathclose%
\pgfusepath{stroke,fill}%
\end{pgfscope}%
\begin{pgfscope}%
\pgfpathrectangle{\pgfqpoint{0.100000in}{0.212622in}}{\pgfqpoint{3.696000in}{3.696000in}}%
\pgfusepath{clip}%
\pgfsetbuttcap%
\pgfsetroundjoin%
\definecolor{currentfill}{rgb}{0.121569,0.466667,0.705882}%
\pgfsetfillcolor{currentfill}%
\pgfsetfillopacity{0.517003}%
\pgfsetlinewidth{1.003750pt}%
\definecolor{currentstroke}{rgb}{0.121569,0.466667,0.705882}%
\pgfsetstrokecolor{currentstroke}%
\pgfsetstrokeopacity{0.517003}%
\pgfsetdash{}{0pt}%
\pgfpathmoveto{\pgfqpoint{1.099420in}{1.612554in}}%
\pgfpathcurveto{\pgfqpoint{1.107657in}{1.612554in}}{\pgfqpoint{1.115557in}{1.615826in}}{\pgfqpoint{1.121381in}{1.621650in}}%
\pgfpathcurveto{\pgfqpoint{1.127205in}{1.627474in}}{\pgfqpoint{1.130477in}{1.635374in}}{\pgfqpoint{1.130477in}{1.643610in}}%
\pgfpathcurveto{\pgfqpoint{1.130477in}{1.651847in}}{\pgfqpoint{1.127205in}{1.659747in}}{\pgfqpoint{1.121381in}{1.665571in}}%
\pgfpathcurveto{\pgfqpoint{1.115557in}{1.671394in}}{\pgfqpoint{1.107657in}{1.674667in}}{\pgfqpoint{1.099420in}{1.674667in}}%
\pgfpathcurveto{\pgfqpoint{1.091184in}{1.674667in}}{\pgfqpoint{1.083284in}{1.671394in}}{\pgfqpoint{1.077460in}{1.665571in}}%
\pgfpathcurveto{\pgfqpoint{1.071636in}{1.659747in}}{\pgfqpoint{1.068364in}{1.651847in}}{\pgfqpoint{1.068364in}{1.643610in}}%
\pgfpathcurveto{\pgfqpoint{1.068364in}{1.635374in}}{\pgfqpoint{1.071636in}{1.627474in}}{\pgfqpoint{1.077460in}{1.621650in}}%
\pgfpathcurveto{\pgfqpoint{1.083284in}{1.615826in}}{\pgfqpoint{1.091184in}{1.612554in}}{\pgfqpoint{1.099420in}{1.612554in}}%
\pgfpathclose%
\pgfusepath{stroke,fill}%
\end{pgfscope}%
\begin{pgfscope}%
\pgfpathrectangle{\pgfqpoint{0.100000in}{0.212622in}}{\pgfqpoint{3.696000in}{3.696000in}}%
\pgfusepath{clip}%
\pgfsetbuttcap%
\pgfsetroundjoin%
\definecolor{currentfill}{rgb}{0.121569,0.466667,0.705882}%
\pgfsetfillcolor{currentfill}%
\pgfsetfillopacity{0.518163}%
\pgfsetlinewidth{1.003750pt}%
\definecolor{currentstroke}{rgb}{0.121569,0.466667,0.705882}%
\pgfsetstrokecolor{currentstroke}%
\pgfsetstrokeopacity{0.518163}%
\pgfsetdash{}{0pt}%
\pgfpathmoveto{\pgfqpoint{2.842238in}{2.283909in}}%
\pgfpathcurveto{\pgfqpoint{2.850475in}{2.283909in}}{\pgfqpoint{2.858375in}{2.287181in}}{\pgfqpoint{2.864199in}{2.293005in}}%
\pgfpathcurveto{\pgfqpoint{2.870022in}{2.298829in}}{\pgfqpoint{2.873295in}{2.306729in}}{\pgfqpoint{2.873295in}{2.314966in}}%
\pgfpathcurveto{\pgfqpoint{2.873295in}{2.323202in}}{\pgfqpoint{2.870022in}{2.331102in}}{\pgfqpoint{2.864199in}{2.336926in}}%
\pgfpathcurveto{\pgfqpoint{2.858375in}{2.342750in}}{\pgfqpoint{2.850475in}{2.346022in}}{\pgfqpoint{2.842238in}{2.346022in}}%
\pgfpathcurveto{\pgfqpoint{2.834002in}{2.346022in}}{\pgfqpoint{2.826102in}{2.342750in}}{\pgfqpoint{2.820278in}{2.336926in}}%
\pgfpathcurveto{\pgfqpoint{2.814454in}{2.331102in}}{\pgfqpoint{2.811182in}{2.323202in}}{\pgfqpoint{2.811182in}{2.314966in}}%
\pgfpathcurveto{\pgfqpoint{2.811182in}{2.306729in}}{\pgfqpoint{2.814454in}{2.298829in}}{\pgfqpoint{2.820278in}{2.293005in}}%
\pgfpathcurveto{\pgfqpoint{2.826102in}{2.287181in}}{\pgfqpoint{2.834002in}{2.283909in}}{\pgfqpoint{2.842238in}{2.283909in}}%
\pgfpathclose%
\pgfusepath{stroke,fill}%
\end{pgfscope}%
\begin{pgfscope}%
\pgfpathrectangle{\pgfqpoint{0.100000in}{0.212622in}}{\pgfqpoint{3.696000in}{3.696000in}}%
\pgfusepath{clip}%
\pgfsetbuttcap%
\pgfsetroundjoin%
\definecolor{currentfill}{rgb}{0.121569,0.466667,0.705882}%
\pgfsetfillcolor{currentfill}%
\pgfsetfillopacity{0.522807}%
\pgfsetlinewidth{1.003750pt}%
\definecolor{currentstroke}{rgb}{0.121569,0.466667,0.705882}%
\pgfsetstrokecolor{currentstroke}%
\pgfsetstrokeopacity{0.522807}%
\pgfsetdash{}{0pt}%
\pgfpathmoveto{\pgfqpoint{2.864753in}{2.286571in}}%
\pgfpathcurveto{\pgfqpoint{2.872989in}{2.286571in}}{\pgfqpoint{2.880890in}{2.289843in}}{\pgfqpoint{2.886713in}{2.295667in}}%
\pgfpathcurveto{\pgfqpoint{2.892537in}{2.301491in}}{\pgfqpoint{2.895810in}{2.309391in}}{\pgfqpoint{2.895810in}{2.317627in}}%
\pgfpathcurveto{\pgfqpoint{2.895810in}{2.325863in}}{\pgfqpoint{2.892537in}{2.333763in}}{\pgfqpoint{2.886713in}{2.339587in}}%
\pgfpathcurveto{\pgfqpoint{2.880890in}{2.345411in}}{\pgfqpoint{2.872989in}{2.348684in}}{\pgfqpoint{2.864753in}{2.348684in}}%
\pgfpathcurveto{\pgfqpoint{2.856517in}{2.348684in}}{\pgfqpoint{2.848617in}{2.345411in}}{\pgfqpoint{2.842793in}{2.339587in}}%
\pgfpathcurveto{\pgfqpoint{2.836969in}{2.333763in}}{\pgfqpoint{2.833697in}{2.325863in}}{\pgfqpoint{2.833697in}{2.317627in}}%
\pgfpathcurveto{\pgfqpoint{2.833697in}{2.309391in}}{\pgfqpoint{2.836969in}{2.301491in}}{\pgfqpoint{2.842793in}{2.295667in}}%
\pgfpathcurveto{\pgfqpoint{2.848617in}{2.289843in}}{\pgfqpoint{2.856517in}{2.286571in}}{\pgfqpoint{2.864753in}{2.286571in}}%
\pgfpathclose%
\pgfusepath{stroke,fill}%
\end{pgfscope}%
\begin{pgfscope}%
\pgfpathrectangle{\pgfqpoint{0.100000in}{0.212622in}}{\pgfqpoint{3.696000in}{3.696000in}}%
\pgfusepath{clip}%
\pgfsetbuttcap%
\pgfsetroundjoin%
\definecolor{currentfill}{rgb}{0.121569,0.466667,0.705882}%
\pgfsetfillcolor{currentfill}%
\pgfsetfillopacity{0.525265}%
\pgfsetlinewidth{1.003750pt}%
\definecolor{currentstroke}{rgb}{0.121569,0.466667,0.705882}%
\pgfsetstrokecolor{currentstroke}%
\pgfsetstrokeopacity{0.525265}%
\pgfsetdash{}{0pt}%
\pgfpathmoveto{\pgfqpoint{2.876255in}{2.284904in}}%
\pgfpathcurveto{\pgfqpoint{2.884492in}{2.284904in}}{\pgfqpoint{2.892392in}{2.288176in}}{\pgfqpoint{2.898216in}{2.294000in}}%
\pgfpathcurveto{\pgfqpoint{2.904040in}{2.299824in}}{\pgfqpoint{2.907312in}{2.307724in}}{\pgfqpoint{2.907312in}{2.315960in}}%
\pgfpathcurveto{\pgfqpoint{2.907312in}{2.324197in}}{\pgfqpoint{2.904040in}{2.332097in}}{\pgfqpoint{2.898216in}{2.337921in}}%
\pgfpathcurveto{\pgfqpoint{2.892392in}{2.343745in}}{\pgfqpoint{2.884492in}{2.347017in}}{\pgfqpoint{2.876255in}{2.347017in}}%
\pgfpathcurveto{\pgfqpoint{2.868019in}{2.347017in}}{\pgfqpoint{2.860119in}{2.343745in}}{\pgfqpoint{2.854295in}{2.337921in}}%
\pgfpathcurveto{\pgfqpoint{2.848471in}{2.332097in}}{\pgfqpoint{2.845199in}{2.324197in}}{\pgfqpoint{2.845199in}{2.315960in}}%
\pgfpathcurveto{\pgfqpoint{2.845199in}{2.307724in}}{\pgfqpoint{2.848471in}{2.299824in}}{\pgfqpoint{2.854295in}{2.294000in}}%
\pgfpathcurveto{\pgfqpoint{2.860119in}{2.288176in}}{\pgfqpoint{2.868019in}{2.284904in}}{\pgfqpoint{2.876255in}{2.284904in}}%
\pgfpathclose%
\pgfusepath{stroke,fill}%
\end{pgfscope}%
\begin{pgfscope}%
\pgfpathrectangle{\pgfqpoint{0.100000in}{0.212622in}}{\pgfqpoint{3.696000in}{3.696000in}}%
\pgfusepath{clip}%
\pgfsetbuttcap%
\pgfsetroundjoin%
\definecolor{currentfill}{rgb}{0.121569,0.466667,0.705882}%
\pgfsetfillcolor{currentfill}%
\pgfsetfillopacity{0.527389}%
\pgfsetlinewidth{1.003750pt}%
\definecolor{currentstroke}{rgb}{0.121569,0.466667,0.705882}%
\pgfsetstrokecolor{currentstroke}%
\pgfsetstrokeopacity{0.527389}%
\pgfsetdash{}{0pt}%
\pgfpathmoveto{\pgfqpoint{1.072678in}{1.566074in}}%
\pgfpathcurveto{\pgfqpoint{1.080914in}{1.566074in}}{\pgfqpoint{1.088814in}{1.569346in}}{\pgfqpoint{1.094638in}{1.575170in}}%
\pgfpathcurveto{\pgfqpoint{1.100462in}{1.580994in}}{\pgfqpoint{1.103735in}{1.588894in}}{\pgfqpoint{1.103735in}{1.597130in}}%
\pgfpathcurveto{\pgfqpoint{1.103735in}{1.605367in}}{\pgfqpoint{1.100462in}{1.613267in}}{\pgfqpoint{1.094638in}{1.619091in}}%
\pgfpathcurveto{\pgfqpoint{1.088814in}{1.624915in}}{\pgfqpoint{1.080914in}{1.628187in}}{\pgfqpoint{1.072678in}{1.628187in}}%
\pgfpathcurveto{\pgfqpoint{1.064442in}{1.628187in}}{\pgfqpoint{1.056542in}{1.624915in}}{\pgfqpoint{1.050718in}{1.619091in}}%
\pgfpathcurveto{\pgfqpoint{1.044894in}{1.613267in}}{\pgfqpoint{1.041622in}{1.605367in}}{\pgfqpoint{1.041622in}{1.597130in}}%
\pgfpathcurveto{\pgfqpoint{1.041622in}{1.588894in}}{\pgfqpoint{1.044894in}{1.580994in}}{\pgfqpoint{1.050718in}{1.575170in}}%
\pgfpathcurveto{\pgfqpoint{1.056542in}{1.569346in}}{\pgfqpoint{1.064442in}{1.566074in}}{\pgfqpoint{1.072678in}{1.566074in}}%
\pgfpathclose%
\pgfusepath{stroke,fill}%
\end{pgfscope}%
\begin{pgfscope}%
\pgfpathrectangle{\pgfqpoint{0.100000in}{0.212622in}}{\pgfqpoint{3.696000in}{3.696000in}}%
\pgfusepath{clip}%
\pgfsetbuttcap%
\pgfsetroundjoin%
\definecolor{currentfill}{rgb}{0.121569,0.466667,0.705882}%
\pgfsetfillcolor{currentfill}%
\pgfsetfillopacity{0.528663}%
\pgfsetlinewidth{1.003750pt}%
\definecolor{currentstroke}{rgb}{0.121569,0.466667,0.705882}%
\pgfsetstrokecolor{currentstroke}%
\pgfsetstrokeopacity{0.528663}%
\pgfsetdash{}{0pt}%
\pgfpathmoveto{\pgfqpoint{2.893089in}{2.288616in}}%
\pgfpathcurveto{\pgfqpoint{2.901326in}{2.288616in}}{\pgfqpoint{2.909226in}{2.291889in}}{\pgfqpoint{2.915050in}{2.297713in}}%
\pgfpathcurveto{\pgfqpoint{2.920873in}{2.303536in}}{\pgfqpoint{2.924146in}{2.311437in}}{\pgfqpoint{2.924146in}{2.319673in}}%
\pgfpathcurveto{\pgfqpoint{2.924146in}{2.327909in}}{\pgfqpoint{2.920873in}{2.335809in}}{\pgfqpoint{2.915050in}{2.341633in}}%
\pgfpathcurveto{\pgfqpoint{2.909226in}{2.347457in}}{\pgfqpoint{2.901326in}{2.350729in}}{\pgfqpoint{2.893089in}{2.350729in}}%
\pgfpathcurveto{\pgfqpoint{2.884853in}{2.350729in}}{\pgfqpoint{2.876953in}{2.347457in}}{\pgfqpoint{2.871129in}{2.341633in}}%
\pgfpathcurveto{\pgfqpoint{2.865305in}{2.335809in}}{\pgfqpoint{2.862033in}{2.327909in}}{\pgfqpoint{2.862033in}{2.319673in}}%
\pgfpathcurveto{\pgfqpoint{2.862033in}{2.311437in}}{\pgfqpoint{2.865305in}{2.303536in}}{\pgfqpoint{2.871129in}{2.297713in}}%
\pgfpathcurveto{\pgfqpoint{2.876953in}{2.291889in}}{\pgfqpoint{2.884853in}{2.288616in}}{\pgfqpoint{2.893089in}{2.288616in}}%
\pgfpathclose%
\pgfusepath{stroke,fill}%
\end{pgfscope}%
\begin{pgfscope}%
\pgfpathrectangle{\pgfqpoint{0.100000in}{0.212622in}}{\pgfqpoint{3.696000in}{3.696000in}}%
\pgfusepath{clip}%
\pgfsetbuttcap%
\pgfsetroundjoin%
\definecolor{currentfill}{rgb}{0.121569,0.466667,0.705882}%
\pgfsetfillcolor{currentfill}%
\pgfsetfillopacity{0.530232}%
\pgfsetlinewidth{1.003750pt}%
\definecolor{currentstroke}{rgb}{0.121569,0.466667,0.705882}%
\pgfsetstrokecolor{currentstroke}%
\pgfsetstrokeopacity{0.530232}%
\pgfsetdash{}{0pt}%
\pgfpathmoveto{\pgfqpoint{2.901661in}{2.286675in}}%
\pgfpathcurveto{\pgfqpoint{2.909897in}{2.286675in}}{\pgfqpoint{2.917797in}{2.289948in}}{\pgfqpoint{2.923621in}{2.295772in}}%
\pgfpathcurveto{\pgfqpoint{2.929445in}{2.301595in}}{\pgfqpoint{2.932718in}{2.309496in}}{\pgfqpoint{2.932718in}{2.317732in}}%
\pgfpathcurveto{\pgfqpoint{2.932718in}{2.325968in}}{\pgfqpoint{2.929445in}{2.333868in}}{\pgfqpoint{2.923621in}{2.339692in}}%
\pgfpathcurveto{\pgfqpoint{2.917797in}{2.345516in}}{\pgfqpoint{2.909897in}{2.348788in}}{\pgfqpoint{2.901661in}{2.348788in}}%
\pgfpathcurveto{\pgfqpoint{2.893425in}{2.348788in}}{\pgfqpoint{2.885525in}{2.345516in}}{\pgfqpoint{2.879701in}{2.339692in}}%
\pgfpathcurveto{\pgfqpoint{2.873877in}{2.333868in}}{\pgfqpoint{2.870605in}{2.325968in}}{\pgfqpoint{2.870605in}{2.317732in}}%
\pgfpathcurveto{\pgfqpoint{2.870605in}{2.309496in}}{\pgfqpoint{2.873877in}{2.301595in}}{\pgfqpoint{2.879701in}{2.295772in}}%
\pgfpathcurveto{\pgfqpoint{2.885525in}{2.289948in}}{\pgfqpoint{2.893425in}{2.286675in}}{\pgfqpoint{2.901661in}{2.286675in}}%
\pgfpathclose%
\pgfusepath{stroke,fill}%
\end{pgfscope}%
\begin{pgfscope}%
\pgfpathrectangle{\pgfqpoint{0.100000in}{0.212622in}}{\pgfqpoint{3.696000in}{3.696000in}}%
\pgfusepath{clip}%
\pgfsetbuttcap%
\pgfsetroundjoin%
\definecolor{currentfill}{rgb}{0.121569,0.466667,0.705882}%
\pgfsetfillcolor{currentfill}%
\pgfsetfillopacity{0.532852}%
\pgfsetlinewidth{1.003750pt}%
\definecolor{currentstroke}{rgb}{0.121569,0.466667,0.705882}%
\pgfsetstrokecolor{currentstroke}%
\pgfsetstrokeopacity{0.532852}%
\pgfsetdash{}{0pt}%
\pgfpathmoveto{\pgfqpoint{2.915327in}{2.289548in}}%
\pgfpathcurveto{\pgfqpoint{2.923564in}{2.289548in}}{\pgfqpoint{2.931464in}{2.292820in}}{\pgfqpoint{2.937288in}{2.298644in}}%
\pgfpathcurveto{\pgfqpoint{2.943112in}{2.304468in}}{\pgfqpoint{2.946384in}{2.312368in}}{\pgfqpoint{2.946384in}{2.320604in}}%
\pgfpathcurveto{\pgfqpoint{2.946384in}{2.328841in}}{\pgfqpoint{2.943112in}{2.336741in}}{\pgfqpoint{2.937288in}{2.342565in}}%
\pgfpathcurveto{\pgfqpoint{2.931464in}{2.348388in}}{\pgfqpoint{2.923564in}{2.351661in}}{\pgfqpoint{2.915327in}{2.351661in}}%
\pgfpathcurveto{\pgfqpoint{2.907091in}{2.351661in}}{\pgfqpoint{2.899191in}{2.348388in}}{\pgfqpoint{2.893367in}{2.342565in}}%
\pgfpathcurveto{\pgfqpoint{2.887543in}{2.336741in}}{\pgfqpoint{2.884271in}{2.328841in}}{\pgfqpoint{2.884271in}{2.320604in}}%
\pgfpathcurveto{\pgfqpoint{2.884271in}{2.312368in}}{\pgfqpoint{2.887543in}{2.304468in}}{\pgfqpoint{2.893367in}{2.298644in}}%
\pgfpathcurveto{\pgfqpoint{2.899191in}{2.292820in}}{\pgfqpoint{2.907091in}{2.289548in}}{\pgfqpoint{2.915327in}{2.289548in}}%
\pgfpathclose%
\pgfusepath{stroke,fill}%
\end{pgfscope}%
\begin{pgfscope}%
\pgfpathrectangle{\pgfqpoint{0.100000in}{0.212622in}}{\pgfqpoint{3.696000in}{3.696000in}}%
\pgfusepath{clip}%
\pgfsetbuttcap%
\pgfsetroundjoin%
\definecolor{currentfill}{rgb}{0.121569,0.466667,0.705882}%
\pgfsetfillcolor{currentfill}%
\pgfsetfillopacity{0.535932}%
\pgfsetlinewidth{1.003750pt}%
\definecolor{currentstroke}{rgb}{0.121569,0.466667,0.705882}%
\pgfsetstrokecolor{currentstroke}%
\pgfsetstrokeopacity{0.535932}%
\pgfsetdash{}{0pt}%
\pgfpathmoveto{\pgfqpoint{2.930813in}{2.285448in}}%
\pgfpathcurveto{\pgfqpoint{2.939049in}{2.285448in}}{\pgfqpoint{2.946949in}{2.288720in}}{\pgfqpoint{2.952773in}{2.294544in}}%
\pgfpathcurveto{\pgfqpoint{2.958597in}{2.300368in}}{\pgfqpoint{2.961869in}{2.308268in}}{\pgfqpoint{2.961869in}{2.316505in}}%
\pgfpathcurveto{\pgfqpoint{2.961869in}{2.324741in}}{\pgfqpoint{2.958597in}{2.332641in}}{\pgfqpoint{2.952773in}{2.338465in}}%
\pgfpathcurveto{\pgfqpoint{2.946949in}{2.344289in}}{\pgfqpoint{2.939049in}{2.347561in}}{\pgfqpoint{2.930813in}{2.347561in}}%
\pgfpathcurveto{\pgfqpoint{2.922577in}{2.347561in}}{\pgfqpoint{2.914677in}{2.344289in}}{\pgfqpoint{2.908853in}{2.338465in}}%
\pgfpathcurveto{\pgfqpoint{2.903029in}{2.332641in}}{\pgfqpoint{2.899756in}{2.324741in}}{\pgfqpoint{2.899756in}{2.316505in}}%
\pgfpathcurveto{\pgfqpoint{2.899756in}{2.308268in}}{\pgfqpoint{2.903029in}{2.300368in}}{\pgfqpoint{2.908853in}{2.294544in}}%
\pgfpathcurveto{\pgfqpoint{2.914677in}{2.288720in}}{\pgfqpoint{2.922577in}{2.285448in}}{\pgfqpoint{2.930813in}{2.285448in}}%
\pgfpathclose%
\pgfusepath{stroke,fill}%
\end{pgfscope}%
\begin{pgfscope}%
\pgfpathrectangle{\pgfqpoint{0.100000in}{0.212622in}}{\pgfqpoint{3.696000in}{3.696000in}}%
\pgfusepath{clip}%
\pgfsetbuttcap%
\pgfsetroundjoin%
\definecolor{currentfill}{rgb}{0.121569,0.466667,0.705882}%
\pgfsetfillcolor{currentfill}%
\pgfsetfillopacity{0.537569}%
\pgfsetlinewidth{1.003750pt}%
\definecolor{currentstroke}{rgb}{0.121569,0.466667,0.705882}%
\pgfsetstrokecolor{currentstroke}%
\pgfsetstrokeopacity{0.537569}%
\pgfsetdash{}{0pt}%
\pgfpathmoveto{\pgfqpoint{1.050409in}{1.532486in}}%
\pgfpathcurveto{\pgfqpoint{1.058645in}{1.532486in}}{\pgfqpoint{1.066545in}{1.535758in}}{\pgfqpoint{1.072369in}{1.541582in}}%
\pgfpathcurveto{\pgfqpoint{1.078193in}{1.547406in}}{\pgfqpoint{1.081466in}{1.555306in}}{\pgfqpoint{1.081466in}{1.563542in}}%
\pgfpathcurveto{\pgfqpoint{1.081466in}{1.571779in}}{\pgfqpoint{1.078193in}{1.579679in}}{\pgfqpoint{1.072369in}{1.585503in}}%
\pgfpathcurveto{\pgfqpoint{1.066545in}{1.591327in}}{\pgfqpoint{1.058645in}{1.594599in}}{\pgfqpoint{1.050409in}{1.594599in}}%
\pgfpathcurveto{\pgfqpoint{1.042173in}{1.594599in}}{\pgfqpoint{1.034273in}{1.591327in}}{\pgfqpoint{1.028449in}{1.585503in}}%
\pgfpathcurveto{\pgfqpoint{1.022625in}{1.579679in}}{\pgfqpoint{1.019353in}{1.571779in}}{\pgfqpoint{1.019353in}{1.563542in}}%
\pgfpathcurveto{\pgfqpoint{1.019353in}{1.555306in}}{\pgfqpoint{1.022625in}{1.547406in}}{\pgfqpoint{1.028449in}{1.541582in}}%
\pgfpathcurveto{\pgfqpoint{1.034273in}{1.535758in}}{\pgfqpoint{1.042173in}{1.532486in}}{\pgfqpoint{1.050409in}{1.532486in}}%
\pgfpathclose%
\pgfusepath{stroke,fill}%
\end{pgfscope}%
\begin{pgfscope}%
\pgfpathrectangle{\pgfqpoint{0.100000in}{0.212622in}}{\pgfqpoint{3.696000in}{3.696000in}}%
\pgfusepath{clip}%
\pgfsetbuttcap%
\pgfsetroundjoin%
\definecolor{currentfill}{rgb}{0.121569,0.466667,0.705882}%
\pgfsetfillcolor{currentfill}%
\pgfsetfillopacity{0.537606}%
\pgfsetlinewidth{1.003750pt}%
\definecolor{currentstroke}{rgb}{0.121569,0.466667,0.705882}%
\pgfsetstrokecolor{currentstroke}%
\pgfsetstrokeopacity{0.537606}%
\pgfsetdash{}{0pt}%
\pgfpathmoveto{\pgfqpoint{2.939837in}{2.284402in}}%
\pgfpathcurveto{\pgfqpoint{2.948074in}{2.284402in}}{\pgfqpoint{2.955974in}{2.287674in}}{\pgfqpoint{2.961798in}{2.293498in}}%
\pgfpathcurveto{\pgfqpoint{2.967621in}{2.299322in}}{\pgfqpoint{2.970894in}{2.307222in}}{\pgfqpoint{2.970894in}{2.315458in}}%
\pgfpathcurveto{\pgfqpoint{2.970894in}{2.323694in}}{\pgfqpoint{2.967621in}{2.331594in}}{\pgfqpoint{2.961798in}{2.337418in}}%
\pgfpathcurveto{\pgfqpoint{2.955974in}{2.343242in}}{\pgfqpoint{2.948074in}{2.346515in}}{\pgfqpoint{2.939837in}{2.346515in}}%
\pgfpathcurveto{\pgfqpoint{2.931601in}{2.346515in}}{\pgfqpoint{2.923701in}{2.343242in}}{\pgfqpoint{2.917877in}{2.337418in}}%
\pgfpathcurveto{\pgfqpoint{2.912053in}{2.331594in}}{\pgfqpoint{2.908781in}{2.323694in}}{\pgfqpoint{2.908781in}{2.315458in}}%
\pgfpathcurveto{\pgfqpoint{2.908781in}{2.307222in}}{\pgfqpoint{2.912053in}{2.299322in}}{\pgfqpoint{2.917877in}{2.293498in}}%
\pgfpathcurveto{\pgfqpoint{2.923701in}{2.287674in}}{\pgfqpoint{2.931601in}{2.284402in}}{\pgfqpoint{2.939837in}{2.284402in}}%
\pgfpathclose%
\pgfusepath{stroke,fill}%
\end{pgfscope}%
\begin{pgfscope}%
\pgfpathrectangle{\pgfqpoint{0.100000in}{0.212622in}}{\pgfqpoint{3.696000in}{3.696000in}}%
\pgfusepath{clip}%
\pgfsetbuttcap%
\pgfsetroundjoin%
\definecolor{currentfill}{rgb}{0.121569,0.466667,0.705882}%
\pgfsetfillcolor{currentfill}%
\pgfsetfillopacity{0.540260}%
\pgfsetlinewidth{1.003750pt}%
\definecolor{currentstroke}{rgb}{0.121569,0.466667,0.705882}%
\pgfsetstrokecolor{currentstroke}%
\pgfsetstrokeopacity{0.540260}%
\pgfsetdash{}{0pt}%
\pgfpathmoveto{\pgfqpoint{2.953835in}{2.284877in}}%
\pgfpathcurveto{\pgfqpoint{2.962072in}{2.284877in}}{\pgfqpoint{2.969972in}{2.288150in}}{\pgfqpoint{2.975796in}{2.293974in}}%
\pgfpathcurveto{\pgfqpoint{2.981620in}{2.299798in}}{\pgfqpoint{2.984892in}{2.307698in}}{\pgfqpoint{2.984892in}{2.315934in}}%
\pgfpathcurveto{\pgfqpoint{2.984892in}{2.324170in}}{\pgfqpoint{2.981620in}{2.332070in}}{\pgfqpoint{2.975796in}{2.337894in}}%
\pgfpathcurveto{\pgfqpoint{2.969972in}{2.343718in}}{\pgfqpoint{2.962072in}{2.346990in}}{\pgfqpoint{2.953835in}{2.346990in}}%
\pgfpathcurveto{\pgfqpoint{2.945599in}{2.346990in}}{\pgfqpoint{2.937699in}{2.343718in}}{\pgfqpoint{2.931875in}{2.337894in}}%
\pgfpathcurveto{\pgfqpoint{2.926051in}{2.332070in}}{\pgfqpoint{2.922779in}{2.324170in}}{\pgfqpoint{2.922779in}{2.315934in}}%
\pgfpathcurveto{\pgfqpoint{2.922779in}{2.307698in}}{\pgfqpoint{2.926051in}{2.299798in}}{\pgfqpoint{2.931875in}{2.293974in}}%
\pgfpathcurveto{\pgfqpoint{2.937699in}{2.288150in}}{\pgfqpoint{2.945599in}{2.284877in}}{\pgfqpoint{2.953835in}{2.284877in}}%
\pgfpathclose%
\pgfusepath{stroke,fill}%
\end{pgfscope}%
\begin{pgfscope}%
\pgfpathrectangle{\pgfqpoint{0.100000in}{0.212622in}}{\pgfqpoint{3.696000in}{3.696000in}}%
\pgfusepath{clip}%
\pgfsetbuttcap%
\pgfsetroundjoin%
\definecolor{currentfill}{rgb}{0.121569,0.466667,0.705882}%
\pgfsetfillcolor{currentfill}%
\pgfsetfillopacity{0.541659}%
\pgfsetlinewidth{1.003750pt}%
\definecolor{currentstroke}{rgb}{0.121569,0.466667,0.705882}%
\pgfsetstrokecolor{currentstroke}%
\pgfsetstrokeopacity{0.541659}%
\pgfsetdash{}{0pt}%
\pgfpathmoveto{\pgfqpoint{2.961323in}{2.284102in}}%
\pgfpathcurveto{\pgfqpoint{2.969559in}{2.284102in}}{\pgfqpoint{2.977459in}{2.287374in}}{\pgfqpoint{2.983283in}{2.293198in}}%
\pgfpathcurveto{\pgfqpoint{2.989107in}{2.299022in}}{\pgfqpoint{2.992379in}{2.306922in}}{\pgfqpoint{2.992379in}{2.315159in}}%
\pgfpathcurveto{\pgfqpoint{2.992379in}{2.323395in}}{\pgfqpoint{2.989107in}{2.331295in}}{\pgfqpoint{2.983283in}{2.337119in}}%
\pgfpathcurveto{\pgfqpoint{2.977459in}{2.342943in}}{\pgfqpoint{2.969559in}{2.346215in}}{\pgfqpoint{2.961323in}{2.346215in}}%
\pgfpathcurveto{\pgfqpoint{2.953087in}{2.346215in}}{\pgfqpoint{2.945187in}{2.342943in}}{\pgfqpoint{2.939363in}{2.337119in}}%
\pgfpathcurveto{\pgfqpoint{2.933539in}{2.331295in}}{\pgfqpoint{2.930266in}{2.323395in}}{\pgfqpoint{2.930266in}{2.315159in}}%
\pgfpathcurveto{\pgfqpoint{2.930266in}{2.306922in}}{\pgfqpoint{2.933539in}{2.299022in}}{\pgfqpoint{2.939363in}{2.293198in}}%
\pgfpathcurveto{\pgfqpoint{2.945187in}{2.287374in}}{\pgfqpoint{2.953087in}{2.284102in}}{\pgfqpoint{2.961323in}{2.284102in}}%
\pgfpathclose%
\pgfusepath{stroke,fill}%
\end{pgfscope}%
\begin{pgfscope}%
\pgfpathrectangle{\pgfqpoint{0.100000in}{0.212622in}}{\pgfqpoint{3.696000in}{3.696000in}}%
\pgfusepath{clip}%
\pgfsetbuttcap%
\pgfsetroundjoin%
\definecolor{currentfill}{rgb}{0.121569,0.466667,0.705882}%
\pgfsetfillcolor{currentfill}%
\pgfsetfillopacity{0.543612}%
\pgfsetlinewidth{1.003750pt}%
\definecolor{currentstroke}{rgb}{0.121569,0.466667,0.705882}%
\pgfsetstrokecolor{currentstroke}%
\pgfsetstrokeopacity{0.543612}%
\pgfsetdash{}{0pt}%
\pgfpathmoveto{\pgfqpoint{2.972486in}{2.286167in}}%
\pgfpathcurveto{\pgfqpoint{2.980722in}{2.286167in}}{\pgfqpoint{2.988622in}{2.289439in}}{\pgfqpoint{2.994446in}{2.295263in}}%
\pgfpathcurveto{\pgfqpoint{3.000270in}{2.301087in}}{\pgfqpoint{3.003542in}{2.308987in}}{\pgfqpoint{3.003542in}{2.317223in}}%
\pgfpathcurveto{\pgfqpoint{3.003542in}{2.325459in}}{\pgfqpoint{3.000270in}{2.333359in}}{\pgfqpoint{2.994446in}{2.339183in}}%
\pgfpathcurveto{\pgfqpoint{2.988622in}{2.345007in}}{\pgfqpoint{2.980722in}{2.348280in}}{\pgfqpoint{2.972486in}{2.348280in}}%
\pgfpathcurveto{\pgfqpoint{2.964249in}{2.348280in}}{\pgfqpoint{2.956349in}{2.345007in}}{\pgfqpoint{2.950525in}{2.339183in}}%
\pgfpathcurveto{\pgfqpoint{2.944702in}{2.333359in}}{\pgfqpoint{2.941429in}{2.325459in}}{\pgfqpoint{2.941429in}{2.317223in}}%
\pgfpathcurveto{\pgfqpoint{2.941429in}{2.308987in}}{\pgfqpoint{2.944702in}{2.301087in}}{\pgfqpoint{2.950525in}{2.295263in}}%
\pgfpathcurveto{\pgfqpoint{2.956349in}{2.289439in}}{\pgfqpoint{2.964249in}{2.286167in}}{\pgfqpoint{2.972486in}{2.286167in}}%
\pgfpathclose%
\pgfusepath{stroke,fill}%
\end{pgfscope}%
\begin{pgfscope}%
\pgfpathrectangle{\pgfqpoint{0.100000in}{0.212622in}}{\pgfqpoint{3.696000in}{3.696000in}}%
\pgfusepath{clip}%
\pgfsetbuttcap%
\pgfsetroundjoin%
\definecolor{currentfill}{rgb}{0.121569,0.466667,0.705882}%
\pgfsetfillcolor{currentfill}%
\pgfsetfillopacity{0.546560}%
\pgfsetlinewidth{1.003750pt}%
\definecolor{currentstroke}{rgb}{0.121569,0.466667,0.705882}%
\pgfsetstrokecolor{currentstroke}%
\pgfsetstrokeopacity{0.546560}%
\pgfsetdash{}{0pt}%
\pgfpathmoveto{\pgfqpoint{2.989859in}{2.283209in}}%
\pgfpathcurveto{\pgfqpoint{2.998095in}{2.283209in}}{\pgfqpoint{3.005995in}{2.286481in}}{\pgfqpoint{3.011819in}{2.292305in}}%
\pgfpathcurveto{\pgfqpoint{3.017643in}{2.298129in}}{\pgfqpoint{3.020916in}{2.306029in}}{\pgfqpoint{3.020916in}{2.314265in}}%
\pgfpathcurveto{\pgfqpoint{3.020916in}{2.322502in}}{\pgfqpoint{3.017643in}{2.330402in}}{\pgfqpoint{3.011819in}{2.336226in}}%
\pgfpathcurveto{\pgfqpoint{3.005995in}{2.342050in}}{\pgfqpoint{2.998095in}{2.345322in}}{\pgfqpoint{2.989859in}{2.345322in}}%
\pgfpathcurveto{\pgfqpoint{2.981623in}{2.345322in}}{\pgfqpoint{2.973723in}{2.342050in}}{\pgfqpoint{2.967899in}{2.336226in}}%
\pgfpathcurveto{\pgfqpoint{2.962075in}{2.330402in}}{\pgfqpoint{2.958803in}{2.322502in}}{\pgfqpoint{2.958803in}{2.314265in}}%
\pgfpathcurveto{\pgfqpoint{2.958803in}{2.306029in}}{\pgfqpoint{2.962075in}{2.298129in}}{\pgfqpoint{2.967899in}{2.292305in}}%
\pgfpathcurveto{\pgfqpoint{2.973723in}{2.286481in}}{\pgfqpoint{2.981623in}{2.283209in}}{\pgfqpoint{2.989859in}{2.283209in}}%
\pgfpathclose%
\pgfusepath{stroke,fill}%
\end{pgfscope}%
\begin{pgfscope}%
\pgfpathrectangle{\pgfqpoint{0.100000in}{0.212622in}}{\pgfqpoint{3.696000in}{3.696000in}}%
\pgfusepath{clip}%
\pgfsetbuttcap%
\pgfsetroundjoin%
\definecolor{currentfill}{rgb}{0.121569,0.466667,0.705882}%
\pgfsetfillcolor{currentfill}%
\pgfsetfillopacity{0.546668}%
\pgfsetlinewidth{1.003750pt}%
\definecolor{currentstroke}{rgb}{0.121569,0.466667,0.705882}%
\pgfsetstrokecolor{currentstroke}%
\pgfsetstrokeopacity{0.546668}%
\pgfsetdash{}{0pt}%
\pgfpathmoveto{\pgfqpoint{1.031730in}{1.502448in}}%
\pgfpathcurveto{\pgfqpoint{1.039966in}{1.502448in}}{\pgfqpoint{1.047866in}{1.505720in}}{\pgfqpoint{1.053690in}{1.511544in}}%
\pgfpathcurveto{\pgfqpoint{1.059514in}{1.517368in}}{\pgfqpoint{1.062787in}{1.525268in}}{\pgfqpoint{1.062787in}{1.533504in}}%
\pgfpathcurveto{\pgfqpoint{1.062787in}{1.541741in}}{\pgfqpoint{1.059514in}{1.549641in}}{\pgfqpoint{1.053690in}{1.555465in}}%
\pgfpathcurveto{\pgfqpoint{1.047866in}{1.561289in}}{\pgfqpoint{1.039966in}{1.564561in}}{\pgfqpoint{1.031730in}{1.564561in}}%
\pgfpathcurveto{\pgfqpoint{1.023494in}{1.564561in}}{\pgfqpoint{1.015594in}{1.561289in}}{\pgfqpoint{1.009770in}{1.555465in}}%
\pgfpathcurveto{\pgfqpoint{1.003946in}{1.549641in}}{\pgfqpoint{1.000674in}{1.541741in}}{\pgfqpoint{1.000674in}{1.533504in}}%
\pgfpathcurveto{\pgfqpoint{1.000674in}{1.525268in}}{\pgfqpoint{1.003946in}{1.517368in}}{\pgfqpoint{1.009770in}{1.511544in}}%
\pgfpathcurveto{\pgfqpoint{1.015594in}{1.505720in}}{\pgfqpoint{1.023494in}{1.502448in}}{\pgfqpoint{1.031730in}{1.502448in}}%
\pgfpathclose%
\pgfusepath{stroke,fill}%
\end{pgfscope}%
\begin{pgfscope}%
\pgfpathrectangle{\pgfqpoint{0.100000in}{0.212622in}}{\pgfqpoint{3.696000in}{3.696000in}}%
\pgfusepath{clip}%
\pgfsetbuttcap%
\pgfsetroundjoin%
\definecolor{currentfill}{rgb}{0.121569,0.466667,0.705882}%
\pgfsetfillcolor{currentfill}%
\pgfsetfillopacity{0.550303}%
\pgfsetlinewidth{1.003750pt}%
\definecolor{currentstroke}{rgb}{0.121569,0.466667,0.705882}%
\pgfsetstrokecolor{currentstroke}%
\pgfsetstrokeopacity{0.550303}%
\pgfsetdash{}{0pt}%
\pgfpathmoveto{\pgfqpoint{3.009801in}{2.284191in}}%
\pgfpathcurveto{\pgfqpoint{3.018038in}{2.284191in}}{\pgfqpoint{3.025938in}{2.287464in}}{\pgfqpoint{3.031762in}{2.293288in}}%
\pgfpathcurveto{\pgfqpoint{3.037586in}{2.299112in}}{\pgfqpoint{3.040858in}{2.307012in}}{\pgfqpoint{3.040858in}{2.315248in}}%
\pgfpathcurveto{\pgfqpoint{3.040858in}{2.323484in}}{\pgfqpoint{3.037586in}{2.331384in}}{\pgfqpoint{3.031762in}{2.337208in}}%
\pgfpathcurveto{\pgfqpoint{3.025938in}{2.343032in}}{\pgfqpoint{3.018038in}{2.346304in}}{\pgfqpoint{3.009801in}{2.346304in}}%
\pgfpathcurveto{\pgfqpoint{3.001565in}{2.346304in}}{\pgfqpoint{2.993665in}{2.343032in}}{\pgfqpoint{2.987841in}{2.337208in}}%
\pgfpathcurveto{\pgfqpoint{2.982017in}{2.331384in}}{\pgfqpoint{2.978745in}{2.323484in}}{\pgfqpoint{2.978745in}{2.315248in}}%
\pgfpathcurveto{\pgfqpoint{2.978745in}{2.307012in}}{\pgfqpoint{2.982017in}{2.299112in}}{\pgfqpoint{2.987841in}{2.293288in}}%
\pgfpathcurveto{\pgfqpoint{2.993665in}{2.287464in}}{\pgfqpoint{3.001565in}{2.284191in}}{\pgfqpoint{3.009801in}{2.284191in}}%
\pgfpathclose%
\pgfusepath{stroke,fill}%
\end{pgfscope}%
\begin{pgfscope}%
\pgfpathrectangle{\pgfqpoint{0.100000in}{0.212622in}}{\pgfqpoint{3.696000in}{3.696000in}}%
\pgfusepath{clip}%
\pgfsetbuttcap%
\pgfsetroundjoin%
\definecolor{currentfill}{rgb}{0.121569,0.466667,0.705882}%
\pgfsetfillcolor{currentfill}%
\pgfsetfillopacity{0.554726}%
\pgfsetlinewidth{1.003750pt}%
\definecolor{currentstroke}{rgb}{0.121569,0.466667,0.705882}%
\pgfsetstrokecolor{currentstroke}%
\pgfsetstrokeopacity{0.554726}%
\pgfsetdash{}{0pt}%
\pgfpathmoveto{\pgfqpoint{3.032017in}{2.280932in}}%
\pgfpathcurveto{\pgfqpoint{3.040254in}{2.280932in}}{\pgfqpoint{3.048154in}{2.284205in}}{\pgfqpoint{3.053978in}{2.290029in}}%
\pgfpathcurveto{\pgfqpoint{3.059802in}{2.295853in}}{\pgfqpoint{3.063074in}{2.303753in}}{\pgfqpoint{3.063074in}{2.311989in}}%
\pgfpathcurveto{\pgfqpoint{3.063074in}{2.320225in}}{\pgfqpoint{3.059802in}{2.328125in}}{\pgfqpoint{3.053978in}{2.333949in}}%
\pgfpathcurveto{\pgfqpoint{3.048154in}{2.339773in}}{\pgfqpoint{3.040254in}{2.343045in}}{\pgfqpoint{3.032017in}{2.343045in}}%
\pgfpathcurveto{\pgfqpoint{3.023781in}{2.343045in}}{\pgfqpoint{3.015881in}{2.339773in}}{\pgfqpoint{3.010057in}{2.333949in}}%
\pgfpathcurveto{\pgfqpoint{3.004233in}{2.328125in}}{\pgfqpoint{3.000961in}{2.320225in}}{\pgfqpoint{3.000961in}{2.311989in}}%
\pgfpathcurveto{\pgfqpoint{3.000961in}{2.303753in}}{\pgfqpoint{3.004233in}{2.295853in}}{\pgfqpoint{3.010057in}{2.290029in}}%
\pgfpathcurveto{\pgfqpoint{3.015881in}{2.284205in}}{\pgfqpoint{3.023781in}{2.280932in}}{\pgfqpoint{3.032017in}{2.280932in}}%
\pgfpathclose%
\pgfusepath{stroke,fill}%
\end{pgfscope}%
\begin{pgfscope}%
\pgfpathrectangle{\pgfqpoint{0.100000in}{0.212622in}}{\pgfqpoint{3.696000in}{3.696000in}}%
\pgfusepath{clip}%
\pgfsetbuttcap%
\pgfsetroundjoin%
\definecolor{currentfill}{rgb}{0.121569,0.466667,0.705882}%
\pgfsetfillcolor{currentfill}%
\pgfsetfillopacity{0.555017}%
\pgfsetlinewidth{1.003750pt}%
\definecolor{currentstroke}{rgb}{0.121569,0.466667,0.705882}%
\pgfsetstrokecolor{currentstroke}%
\pgfsetstrokeopacity{0.555017}%
\pgfsetdash{}{0pt}%
\pgfpathmoveto{\pgfqpoint{1.006223in}{1.483509in}}%
\pgfpathcurveto{\pgfqpoint{1.014460in}{1.483509in}}{\pgfqpoint{1.022360in}{1.486781in}}{\pgfqpoint{1.028184in}{1.492605in}}%
\pgfpathcurveto{\pgfqpoint{1.034008in}{1.498429in}}{\pgfqpoint{1.037280in}{1.506329in}}{\pgfqpoint{1.037280in}{1.514565in}}%
\pgfpathcurveto{\pgfqpoint{1.037280in}{1.522801in}}{\pgfqpoint{1.034008in}{1.530702in}}{\pgfqpoint{1.028184in}{1.536525in}}%
\pgfpathcurveto{\pgfqpoint{1.022360in}{1.542349in}}{\pgfqpoint{1.014460in}{1.545622in}}{\pgfqpoint{1.006223in}{1.545622in}}%
\pgfpathcurveto{\pgfqpoint{0.997987in}{1.545622in}}{\pgfqpoint{0.990087in}{1.542349in}}{\pgfqpoint{0.984263in}{1.536525in}}%
\pgfpathcurveto{\pgfqpoint{0.978439in}{1.530702in}}{\pgfqpoint{0.975167in}{1.522801in}}{\pgfqpoint{0.975167in}{1.514565in}}%
\pgfpathcurveto{\pgfqpoint{0.975167in}{1.506329in}}{\pgfqpoint{0.978439in}{1.498429in}}{\pgfqpoint{0.984263in}{1.492605in}}%
\pgfpathcurveto{\pgfqpoint{0.990087in}{1.486781in}}{\pgfqpoint{0.997987in}{1.483509in}}{\pgfqpoint{1.006223in}{1.483509in}}%
\pgfpathclose%
\pgfusepath{stroke,fill}%
\end{pgfscope}%
\begin{pgfscope}%
\pgfpathrectangle{\pgfqpoint{0.100000in}{0.212622in}}{\pgfqpoint{3.696000in}{3.696000in}}%
\pgfusepath{clip}%
\pgfsetbuttcap%
\pgfsetroundjoin%
\definecolor{currentfill}{rgb}{0.121569,0.466667,0.705882}%
\pgfsetfillcolor{currentfill}%
\pgfsetfillopacity{0.560305}%
\pgfsetlinewidth{1.003750pt}%
\definecolor{currentstroke}{rgb}{0.121569,0.466667,0.705882}%
\pgfsetstrokecolor{currentstroke}%
\pgfsetstrokeopacity{0.560305}%
\pgfsetdash{}{0pt}%
\pgfpathmoveto{\pgfqpoint{3.060012in}{2.284862in}}%
\pgfpathcurveto{\pgfqpoint{3.068249in}{2.284862in}}{\pgfqpoint{3.076149in}{2.288135in}}{\pgfqpoint{3.081973in}{2.293959in}}%
\pgfpathcurveto{\pgfqpoint{3.087797in}{2.299782in}}{\pgfqpoint{3.091069in}{2.307683in}}{\pgfqpoint{3.091069in}{2.315919in}}%
\pgfpathcurveto{\pgfqpoint{3.091069in}{2.324155in}}{\pgfqpoint{3.087797in}{2.332055in}}{\pgfqpoint{3.081973in}{2.337879in}}%
\pgfpathcurveto{\pgfqpoint{3.076149in}{2.343703in}}{\pgfqpoint{3.068249in}{2.346975in}}{\pgfqpoint{3.060012in}{2.346975in}}%
\pgfpathcurveto{\pgfqpoint{3.051776in}{2.346975in}}{\pgfqpoint{3.043876in}{2.343703in}}{\pgfqpoint{3.038052in}{2.337879in}}%
\pgfpathcurveto{\pgfqpoint{3.032228in}{2.332055in}}{\pgfqpoint{3.028956in}{2.324155in}}{\pgfqpoint{3.028956in}{2.315919in}}%
\pgfpathcurveto{\pgfqpoint{3.028956in}{2.307683in}}{\pgfqpoint{3.032228in}{2.299782in}}{\pgfqpoint{3.038052in}{2.293959in}}%
\pgfpathcurveto{\pgfqpoint{3.043876in}{2.288135in}}{\pgfqpoint{3.051776in}{2.284862in}}{\pgfqpoint{3.060012in}{2.284862in}}%
\pgfpathclose%
\pgfusepath{stroke,fill}%
\end{pgfscope}%
\begin{pgfscope}%
\pgfpathrectangle{\pgfqpoint{0.100000in}{0.212622in}}{\pgfqpoint{3.696000in}{3.696000in}}%
\pgfusepath{clip}%
\pgfsetbuttcap%
\pgfsetroundjoin%
\definecolor{currentfill}{rgb}{0.121569,0.466667,0.705882}%
\pgfsetfillcolor{currentfill}%
\pgfsetfillopacity{0.562722}%
\pgfsetlinewidth{1.003750pt}%
\definecolor{currentstroke}{rgb}{0.121569,0.466667,0.705882}%
\pgfsetstrokecolor{currentstroke}%
\pgfsetstrokeopacity{0.562722}%
\pgfsetdash{}{0pt}%
\pgfpathmoveto{\pgfqpoint{0.994537in}{1.465596in}}%
\pgfpathcurveto{\pgfqpoint{1.002774in}{1.465596in}}{\pgfqpoint{1.010674in}{1.468868in}}{\pgfqpoint{1.016498in}{1.474692in}}%
\pgfpathcurveto{\pgfqpoint{1.022322in}{1.480516in}}{\pgfqpoint{1.025594in}{1.488416in}}{\pgfqpoint{1.025594in}{1.496652in}}%
\pgfpathcurveto{\pgfqpoint{1.025594in}{1.504889in}}{\pgfqpoint{1.022322in}{1.512789in}}{\pgfqpoint{1.016498in}{1.518613in}}%
\pgfpathcurveto{\pgfqpoint{1.010674in}{1.524437in}}{\pgfqpoint{1.002774in}{1.527709in}}{\pgfqpoint{0.994537in}{1.527709in}}%
\pgfpathcurveto{\pgfqpoint{0.986301in}{1.527709in}}{\pgfqpoint{0.978401in}{1.524437in}}{\pgfqpoint{0.972577in}{1.518613in}}%
\pgfpathcurveto{\pgfqpoint{0.966753in}{1.512789in}}{\pgfqpoint{0.963481in}{1.504889in}}{\pgfqpoint{0.963481in}{1.496652in}}%
\pgfpathcurveto{\pgfqpoint{0.963481in}{1.488416in}}{\pgfqpoint{0.966753in}{1.480516in}}{\pgfqpoint{0.972577in}{1.474692in}}%
\pgfpathcurveto{\pgfqpoint{0.978401in}{1.468868in}}{\pgfqpoint{0.986301in}{1.465596in}}{\pgfqpoint{0.994537in}{1.465596in}}%
\pgfpathclose%
\pgfusepath{stroke,fill}%
\end{pgfscope}%
\begin{pgfscope}%
\pgfpathrectangle{\pgfqpoint{0.100000in}{0.212622in}}{\pgfqpoint{3.696000in}{3.696000in}}%
\pgfusepath{clip}%
\pgfsetbuttcap%
\pgfsetroundjoin%
\definecolor{currentfill}{rgb}{0.121569,0.466667,0.705882}%
\pgfsetfillcolor{currentfill}%
\pgfsetfillopacity{0.566306}%
\pgfsetlinewidth{1.003750pt}%
\definecolor{currentstroke}{rgb}{0.121569,0.466667,0.705882}%
\pgfsetstrokecolor{currentstroke}%
\pgfsetstrokeopacity{0.566306}%
\pgfsetdash{}{0pt}%
\pgfpathmoveto{\pgfqpoint{3.088619in}{2.278692in}}%
\pgfpathcurveto{\pgfqpoint{3.096856in}{2.278692in}}{\pgfqpoint{3.104756in}{2.281964in}}{\pgfqpoint{3.110580in}{2.287788in}}%
\pgfpathcurveto{\pgfqpoint{3.116403in}{2.293612in}}{\pgfqpoint{3.119676in}{2.301512in}}{\pgfqpoint{3.119676in}{2.309748in}}%
\pgfpathcurveto{\pgfqpoint{3.119676in}{2.317984in}}{\pgfqpoint{3.116403in}{2.325884in}}{\pgfqpoint{3.110580in}{2.331708in}}%
\pgfpathcurveto{\pgfqpoint{3.104756in}{2.337532in}}{\pgfqpoint{3.096856in}{2.340805in}}{\pgfqpoint{3.088619in}{2.340805in}}%
\pgfpathcurveto{\pgfqpoint{3.080383in}{2.340805in}}{\pgfqpoint{3.072483in}{2.337532in}}{\pgfqpoint{3.066659in}{2.331708in}}%
\pgfpathcurveto{\pgfqpoint{3.060835in}{2.325884in}}{\pgfqpoint{3.057563in}{2.317984in}}{\pgfqpoint{3.057563in}{2.309748in}}%
\pgfpathcurveto{\pgfqpoint{3.057563in}{2.301512in}}{\pgfqpoint{3.060835in}{2.293612in}}{\pgfqpoint{3.066659in}{2.287788in}}%
\pgfpathcurveto{\pgfqpoint{3.072483in}{2.281964in}}{\pgfqpoint{3.080383in}{2.278692in}}{\pgfqpoint{3.088619in}{2.278692in}}%
\pgfpathclose%
\pgfusepath{stroke,fill}%
\end{pgfscope}%
\begin{pgfscope}%
\pgfpathrectangle{\pgfqpoint{0.100000in}{0.212622in}}{\pgfqpoint{3.696000in}{3.696000in}}%
\pgfusepath{clip}%
\pgfsetbuttcap%
\pgfsetroundjoin%
\definecolor{currentfill}{rgb}{0.121569,0.466667,0.705882}%
\pgfsetfillcolor{currentfill}%
\pgfsetfillopacity{0.567272}%
\pgfsetlinewidth{1.003750pt}%
\definecolor{currentstroke}{rgb}{0.121569,0.466667,0.705882}%
\pgfsetstrokecolor{currentstroke}%
\pgfsetstrokeopacity{0.567272}%
\pgfsetdash{}{0pt}%
\pgfpathmoveto{\pgfqpoint{0.979176in}{1.457106in}}%
\pgfpathcurveto{\pgfqpoint{0.987412in}{1.457106in}}{\pgfqpoint{0.995312in}{1.460378in}}{\pgfqpoint{1.001136in}{1.466202in}}%
\pgfpathcurveto{\pgfqpoint{1.006960in}{1.472026in}}{\pgfqpoint{1.010232in}{1.479926in}}{\pgfqpoint{1.010232in}{1.488163in}}%
\pgfpathcurveto{\pgfqpoint{1.010232in}{1.496399in}}{\pgfqpoint{1.006960in}{1.504299in}}{\pgfqpoint{1.001136in}{1.510123in}}%
\pgfpathcurveto{\pgfqpoint{0.995312in}{1.515947in}}{\pgfqpoint{0.987412in}{1.519219in}}{\pgfqpoint{0.979176in}{1.519219in}}%
\pgfpathcurveto{\pgfqpoint{0.970939in}{1.519219in}}{\pgfqpoint{0.963039in}{1.515947in}}{\pgfqpoint{0.957215in}{1.510123in}}%
\pgfpathcurveto{\pgfqpoint{0.951391in}{1.504299in}}{\pgfqpoint{0.948119in}{1.496399in}}{\pgfqpoint{0.948119in}{1.488163in}}%
\pgfpathcurveto{\pgfqpoint{0.948119in}{1.479926in}}{\pgfqpoint{0.951391in}{1.472026in}}{\pgfqpoint{0.957215in}{1.466202in}}%
\pgfpathcurveto{\pgfqpoint{0.963039in}{1.460378in}}{\pgfqpoint{0.970939in}{1.457106in}}{\pgfqpoint{0.979176in}{1.457106in}}%
\pgfpathclose%
\pgfusepath{stroke,fill}%
\end{pgfscope}%
\begin{pgfscope}%
\pgfpathrectangle{\pgfqpoint{0.100000in}{0.212622in}}{\pgfqpoint{3.696000in}{3.696000in}}%
\pgfusepath{clip}%
\pgfsetbuttcap%
\pgfsetroundjoin%
\definecolor{currentfill}{rgb}{0.121569,0.466667,0.705882}%
\pgfsetfillcolor{currentfill}%
\pgfsetfillopacity{0.569875}%
\pgfsetlinewidth{1.003750pt}%
\definecolor{currentstroke}{rgb}{0.121569,0.466667,0.705882}%
\pgfsetstrokecolor{currentstroke}%
\pgfsetstrokeopacity{0.569875}%
\pgfsetdash{}{0pt}%
\pgfpathmoveto{\pgfqpoint{0.974384in}{1.450218in}}%
\pgfpathcurveto{\pgfqpoint{0.982621in}{1.450218in}}{\pgfqpoint{0.990521in}{1.453491in}}{\pgfqpoint{0.996345in}{1.459315in}}%
\pgfpathcurveto{\pgfqpoint{1.002169in}{1.465138in}}{\pgfqpoint{1.005441in}{1.473039in}}{\pgfqpoint{1.005441in}{1.481275in}}%
\pgfpathcurveto{\pgfqpoint{1.005441in}{1.489511in}}{\pgfqpoint{1.002169in}{1.497411in}}{\pgfqpoint{0.996345in}{1.503235in}}%
\pgfpathcurveto{\pgfqpoint{0.990521in}{1.509059in}}{\pgfqpoint{0.982621in}{1.512331in}}{\pgfqpoint{0.974384in}{1.512331in}}%
\pgfpathcurveto{\pgfqpoint{0.966148in}{1.512331in}}{\pgfqpoint{0.958248in}{1.509059in}}{\pgfqpoint{0.952424in}{1.503235in}}%
\pgfpathcurveto{\pgfqpoint{0.946600in}{1.497411in}}{\pgfqpoint{0.943328in}{1.489511in}}{\pgfqpoint{0.943328in}{1.481275in}}%
\pgfpathcurveto{\pgfqpoint{0.943328in}{1.473039in}}{\pgfqpoint{0.946600in}{1.465138in}}{\pgfqpoint{0.952424in}{1.459315in}}%
\pgfpathcurveto{\pgfqpoint{0.958248in}{1.453491in}}{\pgfqpoint{0.966148in}{1.450218in}}{\pgfqpoint{0.974384in}{1.450218in}}%
\pgfpathclose%
\pgfusepath{stroke,fill}%
\end{pgfscope}%
\begin{pgfscope}%
\pgfpathrectangle{\pgfqpoint{0.100000in}{0.212622in}}{\pgfqpoint{3.696000in}{3.696000in}}%
\pgfusepath{clip}%
\pgfsetbuttcap%
\pgfsetroundjoin%
\definecolor{currentfill}{rgb}{0.121569,0.466667,0.705882}%
\pgfsetfillcolor{currentfill}%
\pgfsetfillopacity{0.570883}%
\pgfsetlinewidth{1.003750pt}%
\definecolor{currentstroke}{rgb}{0.121569,0.466667,0.705882}%
\pgfsetstrokecolor{currentstroke}%
\pgfsetstrokeopacity{0.570883}%
\pgfsetdash{}{0pt}%
\pgfpathmoveto{\pgfqpoint{0.971437in}{1.446720in}}%
\pgfpathcurveto{\pgfqpoint{0.979674in}{1.446720in}}{\pgfqpoint{0.987574in}{1.449993in}}{\pgfqpoint{0.993398in}{1.455817in}}%
\pgfpathcurveto{\pgfqpoint{0.999222in}{1.461640in}}{\pgfqpoint{1.002494in}{1.469540in}}{\pgfqpoint{1.002494in}{1.477777in}}%
\pgfpathcurveto{\pgfqpoint{1.002494in}{1.486013in}}{\pgfqpoint{0.999222in}{1.493913in}}{\pgfqpoint{0.993398in}{1.499737in}}%
\pgfpathcurveto{\pgfqpoint{0.987574in}{1.505561in}}{\pgfqpoint{0.979674in}{1.508833in}}{\pgfqpoint{0.971437in}{1.508833in}}%
\pgfpathcurveto{\pgfqpoint{0.963201in}{1.508833in}}{\pgfqpoint{0.955301in}{1.505561in}}{\pgfqpoint{0.949477in}{1.499737in}}%
\pgfpathcurveto{\pgfqpoint{0.943653in}{1.493913in}}{\pgfqpoint{0.940381in}{1.486013in}}{\pgfqpoint{0.940381in}{1.477777in}}%
\pgfpathcurveto{\pgfqpoint{0.940381in}{1.469540in}}{\pgfqpoint{0.943653in}{1.461640in}}{\pgfqpoint{0.949477in}{1.455817in}}%
\pgfpathcurveto{\pgfqpoint{0.955301in}{1.449993in}}{\pgfqpoint{0.963201in}{1.446720in}}{\pgfqpoint{0.971437in}{1.446720in}}%
\pgfpathclose%
\pgfusepath{stroke,fill}%
\end{pgfscope}%
\begin{pgfscope}%
\pgfpathrectangle{\pgfqpoint{0.100000in}{0.212622in}}{\pgfqpoint{3.696000in}{3.696000in}}%
\pgfusepath{clip}%
\pgfsetbuttcap%
\pgfsetroundjoin%
\definecolor{currentfill}{rgb}{0.121569,0.466667,0.705882}%
\pgfsetfillcolor{currentfill}%
\pgfsetfillopacity{0.572635}%
\pgfsetlinewidth{1.003750pt}%
\definecolor{currentstroke}{rgb}{0.121569,0.466667,0.705882}%
\pgfsetstrokecolor{currentstroke}%
\pgfsetstrokeopacity{0.572635}%
\pgfsetdash{}{0pt}%
\pgfpathmoveto{\pgfqpoint{0.965894in}{1.440008in}}%
\pgfpathcurveto{\pgfqpoint{0.974131in}{1.440008in}}{\pgfqpoint{0.982031in}{1.443280in}}{\pgfqpoint{0.987855in}{1.449104in}}%
\pgfpathcurveto{\pgfqpoint{0.993679in}{1.454928in}}{\pgfqpoint{0.996951in}{1.462828in}}{\pgfqpoint{0.996951in}{1.471065in}}%
\pgfpathcurveto{\pgfqpoint{0.996951in}{1.479301in}}{\pgfqpoint{0.993679in}{1.487201in}}{\pgfqpoint{0.987855in}{1.493025in}}%
\pgfpathcurveto{\pgfqpoint{0.982031in}{1.498849in}}{\pgfqpoint{0.974131in}{1.502121in}}{\pgfqpoint{0.965894in}{1.502121in}}%
\pgfpathcurveto{\pgfqpoint{0.957658in}{1.502121in}}{\pgfqpoint{0.949758in}{1.498849in}}{\pgfqpoint{0.943934in}{1.493025in}}%
\pgfpathcurveto{\pgfqpoint{0.938110in}{1.487201in}}{\pgfqpoint{0.934838in}{1.479301in}}{\pgfqpoint{0.934838in}{1.471065in}}%
\pgfpathcurveto{\pgfqpoint{0.934838in}{1.462828in}}{\pgfqpoint{0.938110in}{1.454928in}}{\pgfqpoint{0.943934in}{1.449104in}}%
\pgfpathcurveto{\pgfqpoint{0.949758in}{1.443280in}}{\pgfqpoint{0.957658in}{1.440008in}}{\pgfqpoint{0.965894in}{1.440008in}}%
\pgfpathclose%
\pgfusepath{stroke,fill}%
\end{pgfscope}%
\begin{pgfscope}%
\pgfpathrectangle{\pgfqpoint{0.100000in}{0.212622in}}{\pgfqpoint{3.696000in}{3.696000in}}%
\pgfusepath{clip}%
\pgfsetbuttcap%
\pgfsetroundjoin%
\definecolor{currentfill}{rgb}{0.121569,0.466667,0.705882}%
\pgfsetfillcolor{currentfill}%
\pgfsetfillopacity{0.573175}%
\pgfsetlinewidth{1.003750pt}%
\definecolor{currentstroke}{rgb}{0.121569,0.466667,0.705882}%
\pgfsetstrokecolor{currentstroke}%
\pgfsetstrokeopacity{0.573175}%
\pgfsetdash{}{0pt}%
\pgfpathmoveto{\pgfqpoint{3.126861in}{2.275510in}}%
\pgfpathcurveto{\pgfqpoint{3.135097in}{2.275510in}}{\pgfqpoint{3.142997in}{2.278782in}}{\pgfqpoint{3.148821in}{2.284606in}}%
\pgfpathcurveto{\pgfqpoint{3.154645in}{2.290430in}}{\pgfqpoint{3.157918in}{2.298330in}}{\pgfqpoint{3.157918in}{2.306567in}}%
\pgfpathcurveto{\pgfqpoint{3.157918in}{2.314803in}}{\pgfqpoint{3.154645in}{2.322703in}}{\pgfqpoint{3.148821in}{2.328527in}}%
\pgfpathcurveto{\pgfqpoint{3.142997in}{2.334351in}}{\pgfqpoint{3.135097in}{2.337623in}}{\pgfqpoint{3.126861in}{2.337623in}}%
\pgfpathcurveto{\pgfqpoint{3.118625in}{2.337623in}}{\pgfqpoint{3.110725in}{2.334351in}}{\pgfqpoint{3.104901in}{2.328527in}}%
\pgfpathcurveto{\pgfqpoint{3.099077in}{2.322703in}}{\pgfqpoint{3.095805in}{2.314803in}}{\pgfqpoint{3.095805in}{2.306567in}}%
\pgfpathcurveto{\pgfqpoint{3.095805in}{2.298330in}}{\pgfqpoint{3.099077in}{2.290430in}}{\pgfqpoint{3.104901in}{2.284606in}}%
\pgfpathcurveto{\pgfqpoint{3.110725in}{2.278782in}}{\pgfqpoint{3.118625in}{2.275510in}}{\pgfqpoint{3.126861in}{2.275510in}}%
\pgfpathclose%
\pgfusepath{stroke,fill}%
\end{pgfscope}%
\begin{pgfscope}%
\pgfpathrectangle{\pgfqpoint{0.100000in}{0.212622in}}{\pgfqpoint{3.696000in}{3.696000in}}%
\pgfusepath{clip}%
\pgfsetbuttcap%
\pgfsetroundjoin%
\definecolor{currentfill}{rgb}{0.121569,0.466667,0.705882}%
\pgfsetfillcolor{currentfill}%
\pgfsetfillopacity{0.575850}%
\pgfsetlinewidth{1.003750pt}%
\definecolor{currentstroke}{rgb}{0.121569,0.466667,0.705882}%
\pgfsetstrokecolor{currentstroke}%
\pgfsetstrokeopacity{0.575850}%
\pgfsetdash{}{0pt}%
\pgfpathmoveto{\pgfqpoint{0.957619in}{1.426135in}}%
\pgfpathcurveto{\pgfqpoint{0.965855in}{1.426135in}}{\pgfqpoint{0.973755in}{1.429408in}}{\pgfqpoint{0.979579in}{1.435231in}}%
\pgfpathcurveto{\pgfqpoint{0.985403in}{1.441055in}}{\pgfqpoint{0.988676in}{1.448955in}}{\pgfqpoint{0.988676in}{1.457192in}}%
\pgfpathcurveto{\pgfqpoint{0.988676in}{1.465428in}}{\pgfqpoint{0.985403in}{1.473328in}}{\pgfqpoint{0.979579in}{1.479152in}}%
\pgfpathcurveto{\pgfqpoint{0.973755in}{1.484976in}}{\pgfqpoint{0.965855in}{1.488248in}}{\pgfqpoint{0.957619in}{1.488248in}}%
\pgfpathcurveto{\pgfqpoint{0.949383in}{1.488248in}}{\pgfqpoint{0.941483in}{1.484976in}}{\pgfqpoint{0.935659in}{1.479152in}}%
\pgfpathcurveto{\pgfqpoint{0.929835in}{1.473328in}}{\pgfqpoint{0.926563in}{1.465428in}}{\pgfqpoint{0.926563in}{1.457192in}}%
\pgfpathcurveto{\pgfqpoint{0.926563in}{1.448955in}}{\pgfqpoint{0.929835in}{1.441055in}}{\pgfqpoint{0.935659in}{1.435231in}}%
\pgfpathcurveto{\pgfqpoint{0.941483in}{1.429408in}}{\pgfqpoint{0.949383in}{1.426135in}}{\pgfqpoint{0.957619in}{1.426135in}}%
\pgfpathclose%
\pgfusepath{stroke,fill}%
\end{pgfscope}%
\begin{pgfscope}%
\pgfpathrectangle{\pgfqpoint{0.100000in}{0.212622in}}{\pgfqpoint{3.696000in}{3.696000in}}%
\pgfusepath{clip}%
\pgfsetbuttcap%
\pgfsetroundjoin%
\definecolor{currentfill}{rgb}{0.121569,0.466667,0.705882}%
\pgfsetfillcolor{currentfill}%
\pgfsetfillopacity{0.582040}%
\pgfsetlinewidth{1.003750pt}%
\definecolor{currentstroke}{rgb}{0.121569,0.466667,0.705882}%
\pgfsetstrokecolor{currentstroke}%
\pgfsetstrokeopacity{0.582040}%
\pgfsetdash{}{0pt}%
\pgfpathmoveto{\pgfqpoint{3.166746in}{2.275767in}}%
\pgfpathcurveto{\pgfqpoint{3.174982in}{2.275767in}}{\pgfqpoint{3.182882in}{2.279040in}}{\pgfqpoint{3.188706in}{2.284863in}}%
\pgfpathcurveto{\pgfqpoint{3.194530in}{2.290687in}}{\pgfqpoint{3.197803in}{2.298587in}}{\pgfqpoint{3.197803in}{2.306824in}}%
\pgfpathcurveto{\pgfqpoint{3.197803in}{2.315060in}}{\pgfqpoint{3.194530in}{2.322960in}}{\pgfqpoint{3.188706in}{2.328784in}}%
\pgfpathcurveto{\pgfqpoint{3.182882in}{2.334608in}}{\pgfqpoint{3.174982in}{2.337880in}}{\pgfqpoint{3.166746in}{2.337880in}}%
\pgfpathcurveto{\pgfqpoint{3.158510in}{2.337880in}}{\pgfqpoint{3.150610in}{2.334608in}}{\pgfqpoint{3.144786in}{2.328784in}}%
\pgfpathcurveto{\pgfqpoint{3.138962in}{2.322960in}}{\pgfqpoint{3.135690in}{2.315060in}}{\pgfqpoint{3.135690in}{2.306824in}}%
\pgfpathcurveto{\pgfqpoint{3.135690in}{2.298587in}}{\pgfqpoint{3.138962in}{2.290687in}}{\pgfqpoint{3.144786in}{2.284863in}}%
\pgfpathcurveto{\pgfqpoint{3.150610in}{2.279040in}}{\pgfqpoint{3.158510in}{2.275767in}}{\pgfqpoint{3.166746in}{2.275767in}}%
\pgfpathclose%
\pgfusepath{stroke,fill}%
\end{pgfscope}%
\begin{pgfscope}%
\pgfpathrectangle{\pgfqpoint{0.100000in}{0.212622in}}{\pgfqpoint{3.696000in}{3.696000in}}%
\pgfusepath{clip}%
\pgfsetbuttcap%
\pgfsetroundjoin%
\definecolor{currentfill}{rgb}{0.121569,0.466667,0.705882}%
\pgfsetfillcolor{currentfill}%
\pgfsetfillopacity{0.582172}%
\pgfsetlinewidth{1.003750pt}%
\definecolor{currentstroke}{rgb}{0.121569,0.466667,0.705882}%
\pgfsetstrokecolor{currentstroke}%
\pgfsetstrokeopacity{0.582172}%
\pgfsetdash{}{0pt}%
\pgfpathmoveto{\pgfqpoint{0.937391in}{1.409356in}}%
\pgfpathcurveto{\pgfqpoint{0.945627in}{1.409356in}}{\pgfqpoint{0.953528in}{1.412629in}}{\pgfqpoint{0.959351in}{1.418453in}}%
\pgfpathcurveto{\pgfqpoint{0.965175in}{1.424277in}}{\pgfqpoint{0.968448in}{1.432177in}}{\pgfqpoint{0.968448in}{1.440413in}}%
\pgfpathcurveto{\pgfqpoint{0.968448in}{1.448649in}}{\pgfqpoint{0.965175in}{1.456549in}}{\pgfqpoint{0.959351in}{1.462373in}}%
\pgfpathcurveto{\pgfqpoint{0.953528in}{1.468197in}}{\pgfqpoint{0.945627in}{1.471469in}}{\pgfqpoint{0.937391in}{1.471469in}}%
\pgfpathcurveto{\pgfqpoint{0.929155in}{1.471469in}}{\pgfqpoint{0.921255in}{1.468197in}}{\pgfqpoint{0.915431in}{1.462373in}}%
\pgfpathcurveto{\pgfqpoint{0.909607in}{1.456549in}}{\pgfqpoint{0.906335in}{1.448649in}}{\pgfqpoint{0.906335in}{1.440413in}}%
\pgfpathcurveto{\pgfqpoint{0.906335in}{1.432177in}}{\pgfqpoint{0.909607in}{1.424277in}}{\pgfqpoint{0.915431in}{1.418453in}}%
\pgfpathcurveto{\pgfqpoint{0.921255in}{1.412629in}}{\pgfqpoint{0.929155in}{1.409356in}}{\pgfqpoint{0.937391in}{1.409356in}}%
\pgfpathclose%
\pgfusepath{stroke,fill}%
\end{pgfscope}%
\begin{pgfscope}%
\pgfpathrectangle{\pgfqpoint{0.100000in}{0.212622in}}{\pgfqpoint{3.696000in}{3.696000in}}%
\pgfusepath{clip}%
\pgfsetbuttcap%
\pgfsetroundjoin%
\definecolor{currentfill}{rgb}{0.121569,0.466667,0.705882}%
\pgfsetfillcolor{currentfill}%
\pgfsetfillopacity{0.587906}%
\pgfsetlinewidth{1.003750pt}%
\definecolor{currentstroke}{rgb}{0.121569,0.466667,0.705882}%
\pgfsetstrokecolor{currentstroke}%
\pgfsetstrokeopacity{0.587906}%
\pgfsetdash{}{0pt}%
\pgfpathmoveto{\pgfqpoint{0.929392in}{1.397361in}}%
\pgfpathcurveto{\pgfqpoint{0.937628in}{1.397361in}}{\pgfqpoint{0.945528in}{1.400634in}}{\pgfqpoint{0.951352in}{1.406458in}}%
\pgfpathcurveto{\pgfqpoint{0.957176in}{1.412282in}}{\pgfqpoint{0.960448in}{1.420182in}}{\pgfqpoint{0.960448in}{1.428418in}}%
\pgfpathcurveto{\pgfqpoint{0.960448in}{1.436654in}}{\pgfqpoint{0.957176in}{1.444554in}}{\pgfqpoint{0.951352in}{1.450378in}}%
\pgfpathcurveto{\pgfqpoint{0.945528in}{1.456202in}}{\pgfqpoint{0.937628in}{1.459474in}}{\pgfqpoint{0.929392in}{1.459474in}}%
\pgfpathcurveto{\pgfqpoint{0.921155in}{1.459474in}}{\pgfqpoint{0.913255in}{1.456202in}}{\pgfqpoint{0.907431in}{1.450378in}}%
\pgfpathcurveto{\pgfqpoint{0.901607in}{1.444554in}}{\pgfqpoint{0.898335in}{1.436654in}}{\pgfqpoint{0.898335in}{1.428418in}}%
\pgfpathcurveto{\pgfqpoint{0.898335in}{1.420182in}}{\pgfqpoint{0.901607in}{1.412282in}}{\pgfqpoint{0.907431in}{1.406458in}}%
\pgfpathcurveto{\pgfqpoint{0.913255in}{1.400634in}}{\pgfqpoint{0.921155in}{1.397361in}}{\pgfqpoint{0.929392in}{1.397361in}}%
\pgfpathclose%
\pgfusepath{stroke,fill}%
\end{pgfscope}%
\begin{pgfscope}%
\pgfpathrectangle{\pgfqpoint{0.100000in}{0.212622in}}{\pgfqpoint{3.696000in}{3.696000in}}%
\pgfusepath{clip}%
\pgfsetbuttcap%
\pgfsetroundjoin%
\definecolor{currentfill}{rgb}{0.121569,0.466667,0.705882}%
\pgfsetfillcolor{currentfill}%
\pgfsetfillopacity{0.590485}%
\pgfsetlinewidth{1.003750pt}%
\definecolor{currentstroke}{rgb}{0.121569,0.466667,0.705882}%
\pgfsetstrokecolor{currentstroke}%
\pgfsetstrokeopacity{0.590485}%
\pgfsetdash{}{0pt}%
\pgfpathmoveto{\pgfqpoint{3.210538in}{2.276275in}}%
\pgfpathcurveto{\pgfqpoint{3.218774in}{2.276275in}}{\pgfqpoint{3.226674in}{2.279548in}}{\pgfqpoint{3.232498in}{2.285372in}}%
\pgfpathcurveto{\pgfqpoint{3.238322in}{2.291195in}}{\pgfqpoint{3.241594in}{2.299096in}}{\pgfqpoint{3.241594in}{2.307332in}}%
\pgfpathcurveto{\pgfqpoint{3.241594in}{2.315568in}}{\pgfqpoint{3.238322in}{2.323468in}}{\pgfqpoint{3.232498in}{2.329292in}}%
\pgfpathcurveto{\pgfqpoint{3.226674in}{2.335116in}}{\pgfqpoint{3.218774in}{2.338388in}}{\pgfqpoint{3.210538in}{2.338388in}}%
\pgfpathcurveto{\pgfqpoint{3.202301in}{2.338388in}}{\pgfqpoint{3.194401in}{2.335116in}}{\pgfqpoint{3.188577in}{2.329292in}}%
\pgfpathcurveto{\pgfqpoint{3.182753in}{2.323468in}}{\pgfqpoint{3.179481in}{2.315568in}}{\pgfqpoint{3.179481in}{2.307332in}}%
\pgfpathcurveto{\pgfqpoint{3.179481in}{2.299096in}}{\pgfqpoint{3.182753in}{2.291195in}}{\pgfqpoint{3.188577in}{2.285372in}}%
\pgfpathcurveto{\pgfqpoint{3.194401in}{2.279548in}}{\pgfqpoint{3.202301in}{2.276275in}}{\pgfqpoint{3.210538in}{2.276275in}}%
\pgfpathclose%
\pgfusepath{stroke,fill}%
\end{pgfscope}%
\begin{pgfscope}%
\pgfpathrectangle{\pgfqpoint{0.100000in}{0.212622in}}{\pgfqpoint{3.696000in}{3.696000in}}%
\pgfusepath{clip}%
\pgfsetbuttcap%
\pgfsetroundjoin%
\definecolor{currentfill}{rgb}{0.121569,0.466667,0.705882}%
\pgfsetfillcolor{currentfill}%
\pgfsetfillopacity{0.591580}%
\pgfsetlinewidth{1.003750pt}%
\definecolor{currentstroke}{rgb}{0.121569,0.466667,0.705882}%
\pgfsetstrokecolor{currentstroke}%
\pgfsetstrokeopacity{0.591580}%
\pgfsetdash{}{0pt}%
\pgfpathmoveto{\pgfqpoint{0.914399in}{1.388800in}}%
\pgfpathcurveto{\pgfqpoint{0.922635in}{1.388800in}}{\pgfqpoint{0.930535in}{1.392073in}}{\pgfqpoint{0.936359in}{1.397897in}}%
\pgfpathcurveto{\pgfqpoint{0.942183in}{1.403721in}}{\pgfqpoint{0.945455in}{1.411621in}}{\pgfqpoint{0.945455in}{1.419857in}}%
\pgfpathcurveto{\pgfqpoint{0.945455in}{1.428093in}}{\pgfqpoint{0.942183in}{1.435993in}}{\pgfqpoint{0.936359in}{1.441817in}}%
\pgfpathcurveto{\pgfqpoint{0.930535in}{1.447641in}}{\pgfqpoint{0.922635in}{1.450913in}}{\pgfqpoint{0.914399in}{1.450913in}}%
\pgfpathcurveto{\pgfqpoint{0.906162in}{1.450913in}}{\pgfqpoint{0.898262in}{1.447641in}}{\pgfqpoint{0.892438in}{1.441817in}}%
\pgfpathcurveto{\pgfqpoint{0.886614in}{1.435993in}}{\pgfqpoint{0.883342in}{1.428093in}}{\pgfqpoint{0.883342in}{1.419857in}}%
\pgfpathcurveto{\pgfqpoint{0.883342in}{1.411621in}}{\pgfqpoint{0.886614in}{1.403721in}}{\pgfqpoint{0.892438in}{1.397897in}}%
\pgfpathcurveto{\pgfqpoint{0.898262in}{1.392073in}}{\pgfqpoint{0.906162in}{1.388800in}}{\pgfqpoint{0.914399in}{1.388800in}}%
\pgfpathclose%
\pgfusepath{stroke,fill}%
\end{pgfscope}%
\begin{pgfscope}%
\pgfpathrectangle{\pgfqpoint{0.100000in}{0.212622in}}{\pgfqpoint{3.696000in}{3.696000in}}%
\pgfusepath{clip}%
\pgfsetbuttcap%
\pgfsetroundjoin%
\definecolor{currentfill}{rgb}{0.121569,0.466667,0.705882}%
\pgfsetfillcolor{currentfill}%
\pgfsetfillopacity{0.592868}%
\pgfsetlinewidth{1.003750pt}%
\definecolor{currentstroke}{rgb}{0.121569,0.466667,0.705882}%
\pgfsetstrokecolor{currentstroke}%
\pgfsetstrokeopacity{0.592868}%
\pgfsetdash{}{0pt}%
\pgfpathmoveto{\pgfqpoint{0.910448in}{1.382569in}}%
\pgfpathcurveto{\pgfqpoint{0.918684in}{1.382569in}}{\pgfqpoint{0.926584in}{1.385841in}}{\pgfqpoint{0.932408in}{1.391665in}}%
\pgfpathcurveto{\pgfqpoint{0.938232in}{1.397489in}}{\pgfqpoint{0.941505in}{1.405389in}}{\pgfqpoint{0.941505in}{1.413626in}}%
\pgfpathcurveto{\pgfqpoint{0.941505in}{1.421862in}}{\pgfqpoint{0.938232in}{1.429762in}}{\pgfqpoint{0.932408in}{1.435586in}}%
\pgfpathcurveto{\pgfqpoint{0.926584in}{1.441410in}}{\pgfqpoint{0.918684in}{1.444682in}}{\pgfqpoint{0.910448in}{1.444682in}}%
\pgfpathcurveto{\pgfqpoint{0.902212in}{1.444682in}}{\pgfqpoint{0.894312in}{1.441410in}}{\pgfqpoint{0.888488in}{1.435586in}}%
\pgfpathcurveto{\pgfqpoint{0.882664in}{1.429762in}}{\pgfqpoint{0.879392in}{1.421862in}}{\pgfqpoint{0.879392in}{1.413626in}}%
\pgfpathcurveto{\pgfqpoint{0.879392in}{1.405389in}}{\pgfqpoint{0.882664in}{1.397489in}}{\pgfqpoint{0.888488in}{1.391665in}}%
\pgfpathcurveto{\pgfqpoint{0.894312in}{1.385841in}}{\pgfqpoint{0.902212in}{1.382569in}}{\pgfqpoint{0.910448in}{1.382569in}}%
\pgfpathclose%
\pgfusepath{stroke,fill}%
\end{pgfscope}%
\begin{pgfscope}%
\pgfpathrectangle{\pgfqpoint{0.100000in}{0.212622in}}{\pgfqpoint{3.696000in}{3.696000in}}%
\pgfusepath{clip}%
\pgfsetbuttcap%
\pgfsetroundjoin%
\definecolor{currentfill}{rgb}{0.121569,0.466667,0.705882}%
\pgfsetfillcolor{currentfill}%
\pgfsetfillopacity{0.595140}%
\pgfsetlinewidth{1.003750pt}%
\definecolor{currentstroke}{rgb}{0.121569,0.466667,0.705882}%
\pgfsetstrokecolor{currentstroke}%
\pgfsetstrokeopacity{0.595140}%
\pgfsetdash{}{0pt}%
\pgfpathmoveto{\pgfqpoint{3.233608in}{2.273470in}}%
\pgfpathcurveto{\pgfqpoint{3.241845in}{2.273470in}}{\pgfqpoint{3.249745in}{2.276742in}}{\pgfqpoint{3.255569in}{2.282566in}}%
\pgfpathcurveto{\pgfqpoint{3.261393in}{2.288390in}}{\pgfqpoint{3.264665in}{2.296290in}}{\pgfqpoint{3.264665in}{2.304526in}}%
\pgfpathcurveto{\pgfqpoint{3.264665in}{2.312762in}}{\pgfqpoint{3.261393in}{2.320662in}}{\pgfqpoint{3.255569in}{2.326486in}}%
\pgfpathcurveto{\pgfqpoint{3.249745in}{2.332310in}}{\pgfqpoint{3.241845in}{2.335583in}}{\pgfqpoint{3.233608in}{2.335583in}}%
\pgfpathcurveto{\pgfqpoint{3.225372in}{2.335583in}}{\pgfqpoint{3.217472in}{2.332310in}}{\pgfqpoint{3.211648in}{2.326486in}}%
\pgfpathcurveto{\pgfqpoint{3.205824in}{2.320662in}}{\pgfqpoint{3.202552in}{2.312762in}}{\pgfqpoint{3.202552in}{2.304526in}}%
\pgfpathcurveto{\pgfqpoint{3.202552in}{2.296290in}}{\pgfqpoint{3.205824in}{2.288390in}}{\pgfqpoint{3.211648in}{2.282566in}}%
\pgfpathcurveto{\pgfqpoint{3.217472in}{2.276742in}}{\pgfqpoint{3.225372in}{2.273470in}}{\pgfqpoint{3.233608in}{2.273470in}}%
\pgfpathclose%
\pgfusepath{stroke,fill}%
\end{pgfscope}%
\begin{pgfscope}%
\pgfpathrectangle{\pgfqpoint{0.100000in}{0.212622in}}{\pgfqpoint{3.696000in}{3.696000in}}%
\pgfusepath{clip}%
\pgfsetbuttcap%
\pgfsetroundjoin%
\definecolor{currentfill}{rgb}{0.121569,0.466667,0.705882}%
\pgfsetfillcolor{currentfill}%
\pgfsetfillopacity{0.596047}%
\pgfsetlinewidth{1.003750pt}%
\definecolor{currentstroke}{rgb}{0.121569,0.466667,0.705882}%
\pgfsetstrokecolor{currentstroke}%
\pgfsetstrokeopacity{0.596047}%
\pgfsetdash{}{0pt}%
\pgfpathmoveto{\pgfqpoint{0.904840in}{1.375271in}}%
\pgfpathcurveto{\pgfqpoint{0.913076in}{1.375271in}}{\pgfqpoint{0.920976in}{1.378544in}}{\pgfqpoint{0.926800in}{1.384368in}}%
\pgfpathcurveto{\pgfqpoint{0.932624in}{1.390192in}}{\pgfqpoint{0.935896in}{1.398092in}}{\pgfqpoint{0.935896in}{1.406328in}}%
\pgfpathcurveto{\pgfqpoint{0.935896in}{1.414564in}}{\pgfqpoint{0.932624in}{1.422464in}}{\pgfqpoint{0.926800in}{1.428288in}}%
\pgfpathcurveto{\pgfqpoint{0.920976in}{1.434112in}}{\pgfqpoint{0.913076in}{1.437384in}}{\pgfqpoint{0.904840in}{1.437384in}}%
\pgfpathcurveto{\pgfqpoint{0.896603in}{1.437384in}}{\pgfqpoint{0.888703in}{1.434112in}}{\pgfqpoint{0.882879in}{1.428288in}}%
\pgfpathcurveto{\pgfqpoint{0.877055in}{1.422464in}}{\pgfqpoint{0.873783in}{1.414564in}}{\pgfqpoint{0.873783in}{1.406328in}}%
\pgfpathcurveto{\pgfqpoint{0.873783in}{1.398092in}}{\pgfqpoint{0.877055in}{1.390192in}}{\pgfqpoint{0.882879in}{1.384368in}}%
\pgfpathcurveto{\pgfqpoint{0.888703in}{1.378544in}}{\pgfqpoint{0.896603in}{1.375271in}}{\pgfqpoint{0.904840in}{1.375271in}}%
\pgfpathclose%
\pgfusepath{stroke,fill}%
\end{pgfscope}%
\begin{pgfscope}%
\pgfpathrectangle{\pgfqpoint{0.100000in}{0.212622in}}{\pgfqpoint{3.696000in}{3.696000in}}%
\pgfusepath{clip}%
\pgfsetbuttcap%
\pgfsetroundjoin%
\definecolor{currentfill}{rgb}{0.121569,0.466667,0.705882}%
\pgfsetfillcolor{currentfill}%
\pgfsetfillopacity{0.600375}%
\pgfsetlinewidth{1.003750pt}%
\definecolor{currentstroke}{rgb}{0.121569,0.466667,0.705882}%
\pgfsetstrokecolor{currentstroke}%
\pgfsetstrokeopacity{0.600375}%
\pgfsetdash{}{0pt}%
\pgfpathmoveto{\pgfqpoint{0.886413in}{1.362354in}}%
\pgfpathcurveto{\pgfqpoint{0.894650in}{1.362354in}}{\pgfqpoint{0.902550in}{1.365627in}}{\pgfqpoint{0.908374in}{1.371451in}}%
\pgfpathcurveto{\pgfqpoint{0.914198in}{1.377275in}}{\pgfqpoint{0.917470in}{1.385175in}}{\pgfqpoint{0.917470in}{1.393411in}}%
\pgfpathcurveto{\pgfqpoint{0.917470in}{1.401647in}}{\pgfqpoint{0.914198in}{1.409547in}}{\pgfqpoint{0.908374in}{1.415371in}}%
\pgfpathcurveto{\pgfqpoint{0.902550in}{1.421195in}}{\pgfqpoint{0.894650in}{1.424467in}}{\pgfqpoint{0.886413in}{1.424467in}}%
\pgfpathcurveto{\pgfqpoint{0.878177in}{1.424467in}}{\pgfqpoint{0.870277in}{1.421195in}}{\pgfqpoint{0.864453in}{1.415371in}}%
\pgfpathcurveto{\pgfqpoint{0.858629in}{1.409547in}}{\pgfqpoint{0.855357in}{1.401647in}}{\pgfqpoint{0.855357in}{1.393411in}}%
\pgfpathcurveto{\pgfqpoint{0.855357in}{1.385175in}}{\pgfqpoint{0.858629in}{1.377275in}}{\pgfqpoint{0.864453in}{1.371451in}}%
\pgfpathcurveto{\pgfqpoint{0.870277in}{1.365627in}}{\pgfqpoint{0.878177in}{1.362354in}}{\pgfqpoint{0.886413in}{1.362354in}}%
\pgfpathclose%
\pgfusepath{stroke,fill}%
\end{pgfscope}%
\begin{pgfscope}%
\pgfpathrectangle{\pgfqpoint{0.100000in}{0.212622in}}{\pgfqpoint{3.696000in}{3.696000in}}%
\pgfusepath{clip}%
\pgfsetbuttcap%
\pgfsetroundjoin%
\definecolor{currentfill}{rgb}{0.121569,0.466667,0.705882}%
\pgfsetfillcolor{currentfill}%
\pgfsetfillopacity{0.600513}%
\pgfsetlinewidth{1.003750pt}%
\definecolor{currentstroke}{rgb}{0.121569,0.466667,0.705882}%
\pgfsetstrokecolor{currentstroke}%
\pgfsetstrokeopacity{0.600513}%
\pgfsetdash{}{0pt}%
\pgfpathmoveto{\pgfqpoint{3.258866in}{2.271485in}}%
\pgfpathcurveto{\pgfqpoint{3.267102in}{2.271485in}}{\pgfqpoint{3.275002in}{2.274758in}}{\pgfqpoint{3.280826in}{2.280582in}}%
\pgfpathcurveto{\pgfqpoint{3.286650in}{2.286406in}}{\pgfqpoint{3.289923in}{2.294306in}}{\pgfqpoint{3.289923in}{2.302542in}}%
\pgfpathcurveto{\pgfqpoint{3.289923in}{2.310778in}}{\pgfqpoint{3.286650in}{2.318678in}}{\pgfqpoint{3.280826in}{2.324502in}}%
\pgfpathcurveto{\pgfqpoint{3.275002in}{2.330326in}}{\pgfqpoint{3.267102in}{2.333598in}}{\pgfqpoint{3.258866in}{2.333598in}}%
\pgfpathcurveto{\pgfqpoint{3.250630in}{2.333598in}}{\pgfqpoint{3.242730in}{2.330326in}}{\pgfqpoint{3.236906in}{2.324502in}}%
\pgfpathcurveto{\pgfqpoint{3.231082in}{2.318678in}}{\pgfqpoint{3.227810in}{2.310778in}}{\pgfqpoint{3.227810in}{2.302542in}}%
\pgfpathcurveto{\pgfqpoint{3.227810in}{2.294306in}}{\pgfqpoint{3.231082in}{2.286406in}}{\pgfqpoint{3.236906in}{2.280582in}}%
\pgfpathcurveto{\pgfqpoint{3.242730in}{2.274758in}}{\pgfqpoint{3.250630in}{2.271485in}}{\pgfqpoint{3.258866in}{2.271485in}}%
\pgfpathclose%
\pgfusepath{stroke,fill}%
\end{pgfscope}%
\begin{pgfscope}%
\pgfpathrectangle{\pgfqpoint{0.100000in}{0.212622in}}{\pgfqpoint{3.696000in}{3.696000in}}%
\pgfusepath{clip}%
\pgfsetbuttcap%
\pgfsetroundjoin%
\definecolor{currentfill}{rgb}{0.121569,0.466667,0.705882}%
\pgfsetfillcolor{currentfill}%
\pgfsetfillopacity{0.603430}%
\pgfsetlinewidth{1.003750pt}%
\definecolor{currentstroke}{rgb}{0.121569,0.466667,0.705882}%
\pgfsetstrokecolor{currentstroke}%
\pgfsetstrokeopacity{0.603430}%
\pgfsetdash{}{0pt}%
\pgfpathmoveto{\pgfqpoint{3.273190in}{2.271252in}}%
\pgfpathcurveto{\pgfqpoint{3.281427in}{2.271252in}}{\pgfqpoint{3.289327in}{2.274524in}}{\pgfqpoint{3.295151in}{2.280348in}}%
\pgfpathcurveto{\pgfqpoint{3.300975in}{2.286172in}}{\pgfqpoint{3.304247in}{2.294072in}}{\pgfqpoint{3.304247in}{2.302308in}}%
\pgfpathcurveto{\pgfqpoint{3.304247in}{2.310545in}}{\pgfqpoint{3.300975in}{2.318445in}}{\pgfqpoint{3.295151in}{2.324269in}}%
\pgfpathcurveto{\pgfqpoint{3.289327in}{2.330093in}}{\pgfqpoint{3.281427in}{2.333365in}}{\pgfqpoint{3.273190in}{2.333365in}}%
\pgfpathcurveto{\pgfqpoint{3.264954in}{2.333365in}}{\pgfqpoint{3.257054in}{2.330093in}}{\pgfqpoint{3.251230in}{2.324269in}}%
\pgfpathcurveto{\pgfqpoint{3.245406in}{2.318445in}}{\pgfqpoint{3.242134in}{2.310545in}}{\pgfqpoint{3.242134in}{2.302308in}}%
\pgfpathcurveto{\pgfqpoint{3.242134in}{2.294072in}}{\pgfqpoint{3.245406in}{2.286172in}}{\pgfqpoint{3.251230in}{2.280348in}}%
\pgfpathcurveto{\pgfqpoint{3.257054in}{2.274524in}}{\pgfqpoint{3.264954in}{2.271252in}}{\pgfqpoint{3.273190in}{2.271252in}}%
\pgfpathclose%
\pgfusepath{stroke,fill}%
\end{pgfscope}%
\begin{pgfscope}%
\pgfpathrectangle{\pgfqpoint{0.100000in}{0.212622in}}{\pgfqpoint{3.696000in}{3.696000in}}%
\pgfusepath{clip}%
\pgfsetbuttcap%
\pgfsetroundjoin%
\definecolor{currentfill}{rgb}{0.121569,0.466667,0.705882}%
\pgfsetfillcolor{currentfill}%
\pgfsetfillopacity{0.607358}%
\pgfsetlinewidth{1.003750pt}%
\definecolor{currentstroke}{rgb}{0.121569,0.466667,0.705882}%
\pgfsetstrokecolor{currentstroke}%
\pgfsetstrokeopacity{0.607358}%
\pgfsetdash{}{0pt}%
\pgfpathmoveto{\pgfqpoint{3.291242in}{2.267530in}}%
\pgfpathcurveto{\pgfqpoint{3.299478in}{2.267530in}}{\pgfqpoint{3.307378in}{2.270802in}}{\pgfqpoint{3.313202in}{2.276626in}}%
\pgfpathcurveto{\pgfqpoint{3.319026in}{2.282450in}}{\pgfqpoint{3.322298in}{2.290350in}}{\pgfqpoint{3.322298in}{2.298587in}}%
\pgfpathcurveto{\pgfqpoint{3.322298in}{2.306823in}}{\pgfqpoint{3.319026in}{2.314723in}}{\pgfqpoint{3.313202in}{2.320547in}}%
\pgfpathcurveto{\pgfqpoint{3.307378in}{2.326371in}}{\pgfqpoint{3.299478in}{2.329643in}}{\pgfqpoint{3.291242in}{2.329643in}}%
\pgfpathcurveto{\pgfqpoint{3.283005in}{2.329643in}}{\pgfqpoint{3.275105in}{2.326371in}}{\pgfqpoint{3.269281in}{2.320547in}}%
\pgfpathcurveto{\pgfqpoint{3.263457in}{2.314723in}}{\pgfqpoint{3.260185in}{2.306823in}}{\pgfqpoint{3.260185in}{2.298587in}}%
\pgfpathcurveto{\pgfqpoint{3.260185in}{2.290350in}}{\pgfqpoint{3.263457in}{2.282450in}}{\pgfqpoint{3.269281in}{2.276626in}}%
\pgfpathcurveto{\pgfqpoint{3.275105in}{2.270802in}}{\pgfqpoint{3.283005in}{2.267530in}}{\pgfqpoint{3.291242in}{2.267530in}}%
\pgfpathclose%
\pgfusepath{stroke,fill}%
\end{pgfscope}%
\begin{pgfscope}%
\pgfpathrectangle{\pgfqpoint{0.100000in}{0.212622in}}{\pgfqpoint{3.696000in}{3.696000in}}%
\pgfusepath{clip}%
\pgfsetbuttcap%
\pgfsetroundjoin%
\definecolor{currentfill}{rgb}{0.121569,0.466667,0.705882}%
\pgfsetfillcolor{currentfill}%
\pgfsetfillopacity{0.609966}%
\pgfsetlinewidth{1.003750pt}%
\definecolor{currentstroke}{rgb}{0.121569,0.466667,0.705882}%
\pgfsetstrokecolor{currentstroke}%
\pgfsetstrokeopacity{0.609966}%
\pgfsetdash{}{0pt}%
\pgfpathmoveto{\pgfqpoint{3.299873in}{2.265495in}}%
\pgfpathcurveto{\pgfqpoint{3.308110in}{2.265495in}}{\pgfqpoint{3.316010in}{2.268767in}}{\pgfqpoint{3.321834in}{2.274591in}}%
\pgfpathcurveto{\pgfqpoint{3.327658in}{2.280415in}}{\pgfqpoint{3.330930in}{2.288315in}}{\pgfqpoint{3.330930in}{2.296551in}}%
\pgfpathcurveto{\pgfqpoint{3.330930in}{2.304787in}}{\pgfqpoint{3.327658in}{2.312687in}}{\pgfqpoint{3.321834in}{2.318511in}}%
\pgfpathcurveto{\pgfqpoint{3.316010in}{2.324335in}}{\pgfqpoint{3.308110in}{2.327608in}}{\pgfqpoint{3.299873in}{2.327608in}}%
\pgfpathcurveto{\pgfqpoint{3.291637in}{2.327608in}}{\pgfqpoint{3.283737in}{2.324335in}}{\pgfqpoint{3.277913in}{2.318511in}}%
\pgfpathcurveto{\pgfqpoint{3.272089in}{2.312687in}}{\pgfqpoint{3.268817in}{2.304787in}}{\pgfqpoint{3.268817in}{2.296551in}}%
\pgfpathcurveto{\pgfqpoint{3.268817in}{2.288315in}}{\pgfqpoint{3.272089in}{2.280415in}}{\pgfqpoint{3.277913in}{2.274591in}}%
\pgfpathcurveto{\pgfqpoint{3.283737in}{2.268767in}}{\pgfqpoint{3.291637in}{2.265495in}}{\pgfqpoint{3.299873in}{2.265495in}}%
\pgfpathclose%
\pgfusepath{stroke,fill}%
\end{pgfscope}%
\begin{pgfscope}%
\pgfpathrectangle{\pgfqpoint{0.100000in}{0.212622in}}{\pgfqpoint{3.696000in}{3.696000in}}%
\pgfusepath{clip}%
\pgfsetbuttcap%
\pgfsetroundjoin%
\definecolor{currentfill}{rgb}{0.121569,0.466667,0.705882}%
\pgfsetfillcolor{currentfill}%
\pgfsetfillopacity{0.610083}%
\pgfsetlinewidth{1.003750pt}%
\definecolor{currentstroke}{rgb}{0.121569,0.466667,0.705882}%
\pgfsetstrokecolor{currentstroke}%
\pgfsetstrokeopacity{0.610083}%
\pgfsetdash{}{0pt}%
\pgfpathmoveto{\pgfqpoint{0.865979in}{1.334072in}}%
\pgfpathcurveto{\pgfqpoint{0.874215in}{1.334072in}}{\pgfqpoint{0.882116in}{1.337344in}}{\pgfqpoint{0.887939in}{1.343168in}}%
\pgfpathcurveto{\pgfqpoint{0.893763in}{1.348992in}}{\pgfqpoint{0.897036in}{1.356892in}}{\pgfqpoint{0.897036in}{1.365128in}}%
\pgfpathcurveto{\pgfqpoint{0.897036in}{1.373365in}}{\pgfqpoint{0.893763in}{1.381265in}}{\pgfqpoint{0.887939in}{1.387089in}}%
\pgfpathcurveto{\pgfqpoint{0.882116in}{1.392913in}}{\pgfqpoint{0.874215in}{1.396185in}}{\pgfqpoint{0.865979in}{1.396185in}}%
\pgfpathcurveto{\pgfqpoint{0.857743in}{1.396185in}}{\pgfqpoint{0.849843in}{1.392913in}}{\pgfqpoint{0.844019in}{1.387089in}}%
\pgfpathcurveto{\pgfqpoint{0.838195in}{1.381265in}}{\pgfqpoint{0.834923in}{1.373365in}}{\pgfqpoint{0.834923in}{1.365128in}}%
\pgfpathcurveto{\pgfqpoint{0.834923in}{1.356892in}}{\pgfqpoint{0.838195in}{1.348992in}}{\pgfqpoint{0.844019in}{1.343168in}}%
\pgfpathcurveto{\pgfqpoint{0.849843in}{1.337344in}}{\pgfqpoint{0.857743in}{1.334072in}}{\pgfqpoint{0.865979in}{1.334072in}}%
\pgfpathclose%
\pgfusepath{stroke,fill}%
\end{pgfscope}%
\begin{pgfscope}%
\pgfpathrectangle{\pgfqpoint{0.100000in}{0.212622in}}{\pgfqpoint{3.696000in}{3.696000in}}%
\pgfusepath{clip}%
\pgfsetbuttcap%
\pgfsetroundjoin%
\definecolor{currentfill}{rgb}{0.121569,0.466667,0.705882}%
\pgfsetfillcolor{currentfill}%
\pgfsetfillopacity{0.611286}%
\pgfsetlinewidth{1.003750pt}%
\definecolor{currentstroke}{rgb}{0.121569,0.466667,0.705882}%
\pgfsetstrokecolor{currentstroke}%
\pgfsetstrokeopacity{0.611286}%
\pgfsetdash{}{0pt}%
\pgfpathmoveto{\pgfqpoint{3.302797in}{2.261206in}}%
\pgfpathcurveto{\pgfqpoint{3.311033in}{2.261206in}}{\pgfqpoint{3.318933in}{2.264479in}}{\pgfqpoint{3.324757in}{2.270303in}}%
\pgfpathcurveto{\pgfqpoint{3.330581in}{2.276127in}}{\pgfqpoint{3.333853in}{2.284027in}}{\pgfqpoint{3.333853in}{2.292263in}}%
\pgfpathcurveto{\pgfqpoint{3.333853in}{2.300499in}}{\pgfqpoint{3.330581in}{2.308399in}}{\pgfqpoint{3.324757in}{2.314223in}}%
\pgfpathcurveto{\pgfqpoint{3.318933in}{2.320047in}}{\pgfqpoint{3.311033in}{2.323319in}}{\pgfqpoint{3.302797in}{2.323319in}}%
\pgfpathcurveto{\pgfqpoint{3.294560in}{2.323319in}}{\pgfqpoint{3.286660in}{2.320047in}}{\pgfqpoint{3.280836in}{2.314223in}}%
\pgfpathcurveto{\pgfqpoint{3.275012in}{2.308399in}}{\pgfqpoint{3.271740in}{2.300499in}}{\pgfqpoint{3.271740in}{2.292263in}}%
\pgfpathcurveto{\pgfqpoint{3.271740in}{2.284027in}}{\pgfqpoint{3.275012in}{2.276127in}}{\pgfqpoint{3.280836in}{2.270303in}}%
\pgfpathcurveto{\pgfqpoint{3.286660in}{2.264479in}}{\pgfqpoint{3.294560in}{2.261206in}}{\pgfqpoint{3.302797in}{2.261206in}}%
\pgfpathclose%
\pgfusepath{stroke,fill}%
\end{pgfscope}%
\begin{pgfscope}%
\pgfpathrectangle{\pgfqpoint{0.100000in}{0.212622in}}{\pgfqpoint{3.696000in}{3.696000in}}%
\pgfusepath{clip}%
\pgfsetbuttcap%
\pgfsetroundjoin%
\definecolor{currentfill}{rgb}{0.121569,0.466667,0.705882}%
\pgfsetfillcolor{currentfill}%
\pgfsetfillopacity{0.613492}%
\pgfsetlinewidth{1.003750pt}%
\definecolor{currentstroke}{rgb}{0.121569,0.466667,0.705882}%
\pgfsetstrokecolor{currentstroke}%
\pgfsetstrokeopacity{0.613492}%
\pgfsetdash{}{0pt}%
\pgfpathmoveto{\pgfqpoint{3.306089in}{2.256770in}}%
\pgfpathcurveto{\pgfqpoint{3.314325in}{2.256770in}}{\pgfqpoint{3.322225in}{2.260042in}}{\pgfqpoint{3.328049in}{2.265866in}}%
\pgfpathcurveto{\pgfqpoint{3.333873in}{2.271690in}}{\pgfqpoint{3.337145in}{2.279590in}}{\pgfqpoint{3.337145in}{2.287826in}}%
\pgfpathcurveto{\pgfqpoint{3.337145in}{2.296063in}}{\pgfqpoint{3.333873in}{2.303963in}}{\pgfqpoint{3.328049in}{2.309787in}}%
\pgfpathcurveto{\pgfqpoint{3.322225in}{2.315611in}}{\pgfqpoint{3.314325in}{2.318883in}}{\pgfqpoint{3.306089in}{2.318883in}}%
\pgfpathcurveto{\pgfqpoint{3.297852in}{2.318883in}}{\pgfqpoint{3.289952in}{2.315611in}}{\pgfqpoint{3.284128in}{2.309787in}}%
\pgfpathcurveto{\pgfqpoint{3.278304in}{2.303963in}}{\pgfqpoint{3.275032in}{2.296063in}}{\pgfqpoint{3.275032in}{2.287826in}}%
\pgfpathcurveto{\pgfqpoint{3.275032in}{2.279590in}}{\pgfqpoint{3.278304in}{2.271690in}}{\pgfqpoint{3.284128in}{2.265866in}}%
\pgfpathcurveto{\pgfqpoint{3.289952in}{2.260042in}}{\pgfqpoint{3.297852in}{2.256770in}}{\pgfqpoint{3.306089in}{2.256770in}}%
\pgfpathclose%
\pgfusepath{stroke,fill}%
\end{pgfscope}%
\begin{pgfscope}%
\pgfpathrectangle{\pgfqpoint{0.100000in}{0.212622in}}{\pgfqpoint{3.696000in}{3.696000in}}%
\pgfusepath{clip}%
\pgfsetbuttcap%
\pgfsetroundjoin%
\definecolor{currentfill}{rgb}{0.121569,0.466667,0.705882}%
\pgfsetfillcolor{currentfill}%
\pgfsetfillopacity{0.616758}%
\pgfsetlinewidth{1.003750pt}%
\definecolor{currentstroke}{rgb}{0.121569,0.466667,0.705882}%
\pgfsetstrokecolor{currentstroke}%
\pgfsetstrokeopacity{0.616758}%
\pgfsetdash{}{0pt}%
\pgfpathmoveto{\pgfqpoint{3.306915in}{2.252559in}}%
\pgfpathcurveto{\pgfqpoint{3.315151in}{2.252559in}}{\pgfqpoint{3.323052in}{2.255831in}}{\pgfqpoint{3.328875in}{2.261655in}}%
\pgfpathcurveto{\pgfqpoint{3.334699in}{2.267479in}}{\pgfqpoint{3.337972in}{2.275379in}}{\pgfqpoint{3.337972in}{2.283615in}}%
\pgfpathcurveto{\pgfqpoint{3.337972in}{2.291852in}}{\pgfqpoint{3.334699in}{2.299752in}}{\pgfqpoint{3.328875in}{2.305576in}}%
\pgfpathcurveto{\pgfqpoint{3.323052in}{2.311400in}}{\pgfqpoint{3.315151in}{2.314672in}}{\pgfqpoint{3.306915in}{2.314672in}}%
\pgfpathcurveto{\pgfqpoint{3.298679in}{2.314672in}}{\pgfqpoint{3.290779in}{2.311400in}}{\pgfqpoint{3.284955in}{2.305576in}}%
\pgfpathcurveto{\pgfqpoint{3.279131in}{2.299752in}}{\pgfqpoint{3.275859in}{2.291852in}}{\pgfqpoint{3.275859in}{2.283615in}}%
\pgfpathcurveto{\pgfqpoint{3.275859in}{2.275379in}}{\pgfqpoint{3.279131in}{2.267479in}}{\pgfqpoint{3.284955in}{2.261655in}}%
\pgfpathcurveto{\pgfqpoint{3.290779in}{2.255831in}}{\pgfqpoint{3.298679in}{2.252559in}}{\pgfqpoint{3.306915in}{2.252559in}}%
\pgfpathclose%
\pgfusepath{stroke,fill}%
\end{pgfscope}%
\begin{pgfscope}%
\pgfpathrectangle{\pgfqpoint{0.100000in}{0.212622in}}{\pgfqpoint{3.696000in}{3.696000in}}%
\pgfusepath{clip}%
\pgfsetbuttcap%
\pgfsetroundjoin%
\definecolor{currentfill}{rgb}{0.121569,0.466667,0.705882}%
\pgfsetfillcolor{currentfill}%
\pgfsetfillopacity{0.618643}%
\pgfsetlinewidth{1.003750pt}%
\definecolor{currentstroke}{rgb}{0.121569,0.466667,0.705882}%
\pgfsetstrokecolor{currentstroke}%
\pgfsetstrokeopacity{0.618643}%
\pgfsetdash{}{0pt}%
\pgfpathmoveto{\pgfqpoint{3.305075in}{2.251358in}}%
\pgfpathcurveto{\pgfqpoint{3.313311in}{2.251358in}}{\pgfqpoint{3.321211in}{2.254630in}}{\pgfqpoint{3.327035in}{2.260454in}}%
\pgfpathcurveto{\pgfqpoint{3.332859in}{2.266278in}}{\pgfqpoint{3.336131in}{2.274178in}}{\pgfqpoint{3.336131in}{2.282414in}}%
\pgfpathcurveto{\pgfqpoint{3.336131in}{2.290651in}}{\pgfqpoint{3.332859in}{2.298551in}}{\pgfqpoint{3.327035in}{2.304375in}}%
\pgfpathcurveto{\pgfqpoint{3.321211in}{2.310198in}}{\pgfqpoint{3.313311in}{2.313471in}}{\pgfqpoint{3.305075in}{2.313471in}}%
\pgfpathcurveto{\pgfqpoint{3.296839in}{2.313471in}}{\pgfqpoint{3.288939in}{2.310198in}}{\pgfqpoint{3.283115in}{2.304375in}}%
\pgfpathcurveto{\pgfqpoint{3.277291in}{2.298551in}}{\pgfqpoint{3.274018in}{2.290651in}}{\pgfqpoint{3.274018in}{2.282414in}}%
\pgfpathcurveto{\pgfqpoint{3.274018in}{2.274178in}}{\pgfqpoint{3.277291in}{2.266278in}}{\pgfqpoint{3.283115in}{2.260454in}}%
\pgfpathcurveto{\pgfqpoint{3.288939in}{2.254630in}}{\pgfqpoint{3.296839in}{2.251358in}}{\pgfqpoint{3.305075in}{2.251358in}}%
\pgfpathclose%
\pgfusepath{stroke,fill}%
\end{pgfscope}%
\begin{pgfscope}%
\pgfpathrectangle{\pgfqpoint{0.100000in}{0.212622in}}{\pgfqpoint{3.696000in}{3.696000in}}%
\pgfusepath{clip}%
\pgfsetbuttcap%
\pgfsetroundjoin%
\definecolor{currentfill}{rgb}{0.121569,0.466667,0.705882}%
\pgfsetfillcolor{currentfill}%
\pgfsetfillopacity{0.618891}%
\pgfsetlinewidth{1.003750pt}%
\definecolor{currentstroke}{rgb}{0.121569,0.466667,0.705882}%
\pgfsetstrokecolor{currentstroke}%
\pgfsetstrokeopacity{0.618891}%
\pgfsetdash{}{0pt}%
\pgfpathmoveto{\pgfqpoint{0.837149in}{1.319025in}}%
\pgfpathcurveto{\pgfqpoint{0.845385in}{1.319025in}}{\pgfqpoint{0.853285in}{1.322297in}}{\pgfqpoint{0.859109in}{1.328121in}}%
\pgfpathcurveto{\pgfqpoint{0.864933in}{1.333945in}}{\pgfqpoint{0.868206in}{1.341845in}}{\pgfqpoint{0.868206in}{1.350082in}}%
\pgfpathcurveto{\pgfqpoint{0.868206in}{1.358318in}}{\pgfqpoint{0.864933in}{1.366218in}}{\pgfqpoint{0.859109in}{1.372042in}}%
\pgfpathcurveto{\pgfqpoint{0.853285in}{1.377866in}}{\pgfqpoint{0.845385in}{1.381138in}}{\pgfqpoint{0.837149in}{1.381138in}}%
\pgfpathcurveto{\pgfqpoint{0.828913in}{1.381138in}}{\pgfqpoint{0.821013in}{1.377866in}}{\pgfqpoint{0.815189in}{1.372042in}}%
\pgfpathcurveto{\pgfqpoint{0.809365in}{1.366218in}}{\pgfqpoint{0.806093in}{1.358318in}}{\pgfqpoint{0.806093in}{1.350082in}}%
\pgfpathcurveto{\pgfqpoint{0.806093in}{1.341845in}}{\pgfqpoint{0.809365in}{1.333945in}}{\pgfqpoint{0.815189in}{1.328121in}}%
\pgfpathcurveto{\pgfqpoint{0.821013in}{1.322297in}}{\pgfqpoint{0.828913in}{1.319025in}}{\pgfqpoint{0.837149in}{1.319025in}}%
\pgfpathclose%
\pgfusepath{stroke,fill}%
\end{pgfscope}%
\begin{pgfscope}%
\pgfpathrectangle{\pgfqpoint{0.100000in}{0.212622in}}{\pgfqpoint{3.696000in}{3.696000in}}%
\pgfusepath{clip}%
\pgfsetbuttcap%
\pgfsetroundjoin%
\definecolor{currentfill}{rgb}{0.121569,0.466667,0.705882}%
\pgfsetfillcolor{currentfill}%
\pgfsetfillopacity{0.620815}%
\pgfsetlinewidth{1.003750pt}%
\definecolor{currentstroke}{rgb}{0.121569,0.466667,0.705882}%
\pgfsetstrokecolor{currentstroke}%
\pgfsetstrokeopacity{0.620815}%
\pgfsetdash{}{0pt}%
\pgfpathmoveto{\pgfqpoint{0.916842in}{1.260655in}}%
\pgfpathcurveto{\pgfqpoint{0.925078in}{1.260655in}}{\pgfqpoint{0.932978in}{1.263927in}}{\pgfqpoint{0.938802in}{1.269751in}}%
\pgfpathcurveto{\pgfqpoint{0.944626in}{1.275575in}}{\pgfqpoint{0.947899in}{1.283475in}}{\pgfqpoint{0.947899in}{1.291711in}}%
\pgfpathcurveto{\pgfqpoint{0.947899in}{1.299948in}}{\pgfqpoint{0.944626in}{1.307848in}}{\pgfqpoint{0.938802in}{1.313672in}}%
\pgfpathcurveto{\pgfqpoint{0.932978in}{1.319496in}}{\pgfqpoint{0.925078in}{1.322768in}}{\pgfqpoint{0.916842in}{1.322768in}}%
\pgfpathcurveto{\pgfqpoint{0.908606in}{1.322768in}}{\pgfqpoint{0.900706in}{1.319496in}}{\pgfqpoint{0.894882in}{1.313672in}}%
\pgfpathcurveto{\pgfqpoint{0.889058in}{1.307848in}}{\pgfqpoint{0.885786in}{1.299948in}}{\pgfqpoint{0.885786in}{1.291711in}}%
\pgfpathcurveto{\pgfqpoint{0.885786in}{1.283475in}}{\pgfqpoint{0.889058in}{1.275575in}}{\pgfqpoint{0.894882in}{1.269751in}}%
\pgfpathcurveto{\pgfqpoint{0.900706in}{1.263927in}}{\pgfqpoint{0.908606in}{1.260655in}}{\pgfqpoint{0.916842in}{1.260655in}}%
\pgfpathclose%
\pgfusepath{stroke,fill}%
\end{pgfscope}%
\begin{pgfscope}%
\pgfpathrectangle{\pgfqpoint{0.100000in}{0.212622in}}{\pgfqpoint{3.696000in}{3.696000in}}%
\pgfusepath{clip}%
\pgfsetbuttcap%
\pgfsetroundjoin%
\definecolor{currentfill}{rgb}{0.121569,0.466667,0.705882}%
\pgfsetfillcolor{currentfill}%
\pgfsetfillopacity{0.620848}%
\pgfsetlinewidth{1.003750pt}%
\definecolor{currentstroke}{rgb}{0.121569,0.466667,0.705882}%
\pgfsetstrokecolor{currentstroke}%
\pgfsetstrokeopacity{0.620848}%
\pgfsetdash{}{0pt}%
\pgfpathmoveto{\pgfqpoint{3.300833in}{2.248757in}}%
\pgfpathcurveto{\pgfqpoint{3.309069in}{2.248757in}}{\pgfqpoint{3.316969in}{2.252029in}}{\pgfqpoint{3.322793in}{2.257853in}}%
\pgfpathcurveto{\pgfqpoint{3.328617in}{2.263677in}}{\pgfqpoint{3.331889in}{2.271577in}}{\pgfqpoint{3.331889in}{2.279814in}}%
\pgfpathcurveto{\pgfqpoint{3.331889in}{2.288050in}}{\pgfqpoint{3.328617in}{2.295950in}}{\pgfqpoint{3.322793in}{2.301774in}}%
\pgfpathcurveto{\pgfqpoint{3.316969in}{2.307598in}}{\pgfqpoint{3.309069in}{2.310870in}}{\pgfqpoint{3.300833in}{2.310870in}}%
\pgfpathcurveto{\pgfqpoint{3.292597in}{2.310870in}}{\pgfqpoint{3.284697in}{2.307598in}}{\pgfqpoint{3.278873in}{2.301774in}}%
\pgfpathcurveto{\pgfqpoint{3.273049in}{2.295950in}}{\pgfqpoint{3.269776in}{2.288050in}}{\pgfqpoint{3.269776in}{2.279814in}}%
\pgfpathcurveto{\pgfqpoint{3.269776in}{2.271577in}}{\pgfqpoint{3.273049in}{2.263677in}}{\pgfqpoint{3.278873in}{2.257853in}}%
\pgfpathcurveto{\pgfqpoint{3.284697in}{2.252029in}}{\pgfqpoint{3.292597in}{2.248757in}}{\pgfqpoint{3.300833in}{2.248757in}}%
\pgfpathclose%
\pgfusepath{stroke,fill}%
\end{pgfscope}%
\begin{pgfscope}%
\pgfpathrectangle{\pgfqpoint{0.100000in}{0.212622in}}{\pgfqpoint{3.696000in}{3.696000in}}%
\pgfusepath{clip}%
\pgfsetbuttcap%
\pgfsetroundjoin%
\definecolor{currentfill}{rgb}{0.121569,0.466667,0.705882}%
\pgfsetfillcolor{currentfill}%
\pgfsetfillopacity{0.621536}%
\pgfsetlinewidth{1.003750pt}%
\definecolor{currentstroke}{rgb}{0.121569,0.466667,0.705882}%
\pgfsetstrokecolor{currentstroke}%
\pgfsetstrokeopacity{0.621536}%
\pgfsetdash{}{0pt}%
\pgfpathmoveto{\pgfqpoint{0.913953in}{1.260210in}}%
\pgfpathcurveto{\pgfqpoint{0.922190in}{1.260210in}}{\pgfqpoint{0.930090in}{1.263483in}}{\pgfqpoint{0.935914in}{1.269307in}}%
\pgfpathcurveto{\pgfqpoint{0.941738in}{1.275131in}}{\pgfqpoint{0.945010in}{1.283031in}}{\pgfqpoint{0.945010in}{1.291267in}}%
\pgfpathcurveto{\pgfqpoint{0.945010in}{1.299503in}}{\pgfqpoint{0.941738in}{1.307403in}}{\pgfqpoint{0.935914in}{1.313227in}}%
\pgfpathcurveto{\pgfqpoint{0.930090in}{1.319051in}}{\pgfqpoint{0.922190in}{1.322323in}}{\pgfqpoint{0.913953in}{1.322323in}}%
\pgfpathcurveto{\pgfqpoint{0.905717in}{1.322323in}}{\pgfqpoint{0.897817in}{1.319051in}}{\pgfqpoint{0.891993in}{1.313227in}}%
\pgfpathcurveto{\pgfqpoint{0.886169in}{1.307403in}}{\pgfqpoint{0.882897in}{1.299503in}}{\pgfqpoint{0.882897in}{1.291267in}}%
\pgfpathcurveto{\pgfqpoint{0.882897in}{1.283031in}}{\pgfqpoint{0.886169in}{1.275131in}}{\pgfqpoint{0.891993in}{1.269307in}}%
\pgfpathcurveto{\pgfqpoint{0.897817in}{1.263483in}}{\pgfqpoint{0.905717in}{1.260210in}}{\pgfqpoint{0.913953in}{1.260210in}}%
\pgfpathclose%
\pgfusepath{stroke,fill}%
\end{pgfscope}%
\begin{pgfscope}%
\pgfpathrectangle{\pgfqpoint{0.100000in}{0.212622in}}{\pgfqpoint{3.696000in}{3.696000in}}%
\pgfusepath{clip}%
\pgfsetbuttcap%
\pgfsetroundjoin%
\definecolor{currentfill}{rgb}{0.121569,0.466667,0.705882}%
\pgfsetfillcolor{currentfill}%
\pgfsetfillopacity{0.623314}%
\pgfsetlinewidth{1.003750pt}%
\definecolor{currentstroke}{rgb}{0.121569,0.466667,0.705882}%
\pgfsetstrokecolor{currentstroke}%
\pgfsetstrokeopacity{0.623314}%
\pgfsetdash{}{0pt}%
\pgfpathmoveto{\pgfqpoint{0.906395in}{1.258032in}}%
\pgfpathcurveto{\pgfqpoint{0.914631in}{1.258032in}}{\pgfqpoint{0.922531in}{1.261304in}}{\pgfqpoint{0.928355in}{1.267128in}}%
\pgfpathcurveto{\pgfqpoint{0.934179in}{1.272952in}}{\pgfqpoint{0.937451in}{1.280852in}}{\pgfqpoint{0.937451in}{1.289089in}}%
\pgfpathcurveto{\pgfqpoint{0.937451in}{1.297325in}}{\pgfqpoint{0.934179in}{1.305225in}}{\pgfqpoint{0.928355in}{1.311049in}}%
\pgfpathcurveto{\pgfqpoint{0.922531in}{1.316873in}}{\pgfqpoint{0.914631in}{1.320145in}}{\pgfqpoint{0.906395in}{1.320145in}}%
\pgfpathcurveto{\pgfqpoint{0.898158in}{1.320145in}}{\pgfqpoint{0.890258in}{1.316873in}}{\pgfqpoint{0.884434in}{1.311049in}}%
\pgfpathcurveto{\pgfqpoint{0.878610in}{1.305225in}}{\pgfqpoint{0.875338in}{1.297325in}}{\pgfqpoint{0.875338in}{1.289089in}}%
\pgfpathcurveto{\pgfqpoint{0.875338in}{1.280852in}}{\pgfqpoint{0.878610in}{1.272952in}}{\pgfqpoint{0.884434in}{1.267128in}}%
\pgfpathcurveto{\pgfqpoint{0.890258in}{1.261304in}}{\pgfqpoint{0.898158in}{1.258032in}}{\pgfqpoint{0.906395in}{1.258032in}}%
\pgfpathclose%
\pgfusepath{stroke,fill}%
\end{pgfscope}%
\begin{pgfscope}%
\pgfpathrectangle{\pgfqpoint{0.100000in}{0.212622in}}{\pgfqpoint{3.696000in}{3.696000in}}%
\pgfusepath{clip}%
\pgfsetbuttcap%
\pgfsetroundjoin%
\definecolor{currentfill}{rgb}{0.121569,0.466667,0.705882}%
\pgfsetfillcolor{currentfill}%
\pgfsetfillopacity{0.624095}%
\pgfsetlinewidth{1.003750pt}%
\definecolor{currentstroke}{rgb}{0.121569,0.466667,0.705882}%
\pgfsetstrokecolor{currentstroke}%
\pgfsetstrokeopacity{0.624095}%
\pgfsetdash{}{0pt}%
\pgfpathmoveto{\pgfqpoint{3.295651in}{2.246681in}}%
\pgfpathcurveto{\pgfqpoint{3.303888in}{2.246681in}}{\pgfqpoint{3.311788in}{2.249954in}}{\pgfqpoint{3.317612in}{2.255778in}}%
\pgfpathcurveto{\pgfqpoint{3.323436in}{2.261602in}}{\pgfqpoint{3.326708in}{2.269502in}}{\pgfqpoint{3.326708in}{2.277738in}}%
\pgfpathcurveto{\pgfqpoint{3.326708in}{2.285974in}}{\pgfqpoint{3.323436in}{2.293874in}}{\pgfqpoint{3.317612in}{2.299698in}}%
\pgfpathcurveto{\pgfqpoint{3.311788in}{2.305522in}}{\pgfqpoint{3.303888in}{2.308794in}}{\pgfqpoint{3.295651in}{2.308794in}}%
\pgfpathcurveto{\pgfqpoint{3.287415in}{2.308794in}}{\pgfqpoint{3.279515in}{2.305522in}}{\pgfqpoint{3.273691in}{2.299698in}}%
\pgfpathcurveto{\pgfqpoint{3.267867in}{2.293874in}}{\pgfqpoint{3.264595in}{2.285974in}}{\pgfqpoint{3.264595in}{2.277738in}}%
\pgfpathcurveto{\pgfqpoint{3.264595in}{2.269502in}}{\pgfqpoint{3.267867in}{2.261602in}}{\pgfqpoint{3.273691in}{2.255778in}}%
\pgfpathcurveto{\pgfqpoint{3.279515in}{2.249954in}}{\pgfqpoint{3.287415in}{2.246681in}}{\pgfqpoint{3.295651in}{2.246681in}}%
\pgfpathclose%
\pgfusepath{stroke,fill}%
\end{pgfscope}%
\begin{pgfscope}%
\pgfpathrectangle{\pgfqpoint{0.100000in}{0.212622in}}{\pgfqpoint{3.696000in}{3.696000in}}%
\pgfusepath{clip}%
\pgfsetbuttcap%
\pgfsetroundjoin%
\definecolor{currentfill}{rgb}{0.121569,0.466667,0.705882}%
\pgfsetfillcolor{currentfill}%
\pgfsetfillopacity{0.626369}%
\pgfsetlinewidth{1.003750pt}%
\definecolor{currentstroke}{rgb}{0.121569,0.466667,0.705882}%
\pgfsetstrokecolor{currentstroke}%
\pgfsetstrokeopacity{0.626369}%
\pgfsetdash{}{0pt}%
\pgfpathmoveto{\pgfqpoint{0.892987in}{1.254184in}}%
\pgfpathcurveto{\pgfqpoint{0.901223in}{1.254184in}}{\pgfqpoint{0.909124in}{1.257456in}}{\pgfqpoint{0.914947in}{1.263280in}}%
\pgfpathcurveto{\pgfqpoint{0.920771in}{1.269104in}}{\pgfqpoint{0.924044in}{1.277004in}}{\pgfqpoint{0.924044in}{1.285240in}}%
\pgfpathcurveto{\pgfqpoint{0.924044in}{1.293477in}}{\pgfqpoint{0.920771in}{1.301377in}}{\pgfqpoint{0.914947in}{1.307201in}}%
\pgfpathcurveto{\pgfqpoint{0.909124in}{1.313024in}}{\pgfqpoint{0.901223in}{1.316297in}}{\pgfqpoint{0.892987in}{1.316297in}}%
\pgfpathcurveto{\pgfqpoint{0.884751in}{1.316297in}}{\pgfqpoint{0.876851in}{1.313024in}}{\pgfqpoint{0.871027in}{1.307201in}}%
\pgfpathcurveto{\pgfqpoint{0.865203in}{1.301377in}}{\pgfqpoint{0.861931in}{1.293477in}}{\pgfqpoint{0.861931in}{1.285240in}}%
\pgfpathcurveto{\pgfqpoint{0.861931in}{1.277004in}}{\pgfqpoint{0.865203in}{1.269104in}}{\pgfqpoint{0.871027in}{1.263280in}}%
\pgfpathcurveto{\pgfqpoint{0.876851in}{1.257456in}}{\pgfqpoint{0.884751in}{1.254184in}}{\pgfqpoint{0.892987in}{1.254184in}}%
\pgfpathclose%
\pgfusepath{stroke,fill}%
\end{pgfscope}%
\begin{pgfscope}%
\pgfpathrectangle{\pgfqpoint{0.100000in}{0.212622in}}{\pgfqpoint{3.696000in}{3.696000in}}%
\pgfusepath{clip}%
\pgfsetbuttcap%
\pgfsetroundjoin%
\definecolor{currentfill}{rgb}{0.121569,0.466667,0.705882}%
\pgfsetfillcolor{currentfill}%
\pgfsetfillopacity{0.626744}%
\pgfsetlinewidth{1.003750pt}%
\definecolor{currentstroke}{rgb}{0.121569,0.466667,0.705882}%
\pgfsetstrokecolor{currentstroke}%
\pgfsetstrokeopacity{0.626744}%
\pgfsetdash{}{0pt}%
\pgfpathmoveto{\pgfqpoint{0.824991in}{1.304263in}}%
\pgfpathcurveto{\pgfqpoint{0.833228in}{1.304263in}}{\pgfqpoint{0.841128in}{1.307535in}}{\pgfqpoint{0.846952in}{1.313359in}}%
\pgfpathcurveto{\pgfqpoint{0.852776in}{1.319183in}}{\pgfqpoint{0.856048in}{1.327083in}}{\pgfqpoint{0.856048in}{1.335319in}}%
\pgfpathcurveto{\pgfqpoint{0.856048in}{1.343556in}}{\pgfqpoint{0.852776in}{1.351456in}}{\pgfqpoint{0.846952in}{1.357280in}}%
\pgfpathcurveto{\pgfqpoint{0.841128in}{1.363103in}}{\pgfqpoint{0.833228in}{1.366376in}}{\pgfqpoint{0.824991in}{1.366376in}}%
\pgfpathcurveto{\pgfqpoint{0.816755in}{1.366376in}}{\pgfqpoint{0.808855in}{1.363103in}}{\pgfqpoint{0.803031in}{1.357280in}}%
\pgfpathcurveto{\pgfqpoint{0.797207in}{1.351456in}}{\pgfqpoint{0.793935in}{1.343556in}}{\pgfqpoint{0.793935in}{1.335319in}}%
\pgfpathcurveto{\pgfqpoint{0.793935in}{1.327083in}}{\pgfqpoint{0.797207in}{1.319183in}}{\pgfqpoint{0.803031in}{1.313359in}}%
\pgfpathcurveto{\pgfqpoint{0.808855in}{1.307535in}}{\pgfqpoint{0.816755in}{1.304263in}}{\pgfqpoint{0.824991in}{1.304263in}}%
\pgfpathclose%
\pgfusepath{stroke,fill}%
\end{pgfscope}%
\begin{pgfscope}%
\pgfpathrectangle{\pgfqpoint{0.100000in}{0.212622in}}{\pgfqpoint{3.696000in}{3.696000in}}%
\pgfusepath{clip}%
\pgfsetbuttcap%
\pgfsetroundjoin%
\definecolor{currentfill}{rgb}{0.121569,0.466667,0.705882}%
\pgfsetfillcolor{currentfill}%
\pgfsetfillopacity{0.627814}%
\pgfsetlinewidth{1.003750pt}%
\definecolor{currentstroke}{rgb}{0.121569,0.466667,0.705882}%
\pgfsetstrokecolor{currentstroke}%
\pgfsetstrokeopacity{0.627814}%
\pgfsetdash{}{0pt}%
\pgfpathmoveto{\pgfqpoint{3.287113in}{2.243309in}}%
\pgfpathcurveto{\pgfqpoint{3.295349in}{2.243309in}}{\pgfqpoint{3.303249in}{2.246581in}}{\pgfqpoint{3.309073in}{2.252405in}}%
\pgfpathcurveto{\pgfqpoint{3.314897in}{2.258229in}}{\pgfqpoint{3.318169in}{2.266129in}}{\pgfqpoint{3.318169in}{2.274365in}}%
\pgfpathcurveto{\pgfqpoint{3.318169in}{2.282602in}}{\pgfqpoint{3.314897in}{2.290502in}}{\pgfqpoint{3.309073in}{2.296326in}}%
\pgfpathcurveto{\pgfqpoint{3.303249in}{2.302150in}}{\pgfqpoint{3.295349in}{2.305422in}}{\pgfqpoint{3.287113in}{2.305422in}}%
\pgfpathcurveto{\pgfqpoint{3.278876in}{2.305422in}}{\pgfqpoint{3.270976in}{2.302150in}}{\pgfqpoint{3.265152in}{2.296326in}}%
\pgfpathcurveto{\pgfqpoint{3.259328in}{2.290502in}}{\pgfqpoint{3.256056in}{2.282602in}}{\pgfqpoint{3.256056in}{2.274365in}}%
\pgfpathcurveto{\pgfqpoint{3.256056in}{2.266129in}}{\pgfqpoint{3.259328in}{2.258229in}}{\pgfqpoint{3.265152in}{2.252405in}}%
\pgfpathcurveto{\pgfqpoint{3.270976in}{2.246581in}}{\pgfqpoint{3.278876in}{2.243309in}}{\pgfqpoint{3.287113in}{2.243309in}}%
\pgfpathclose%
\pgfusepath{stroke,fill}%
\end{pgfscope}%
\begin{pgfscope}%
\pgfpathrectangle{\pgfqpoint{0.100000in}{0.212622in}}{\pgfqpoint{3.696000in}{3.696000in}}%
\pgfusepath{clip}%
\pgfsetbuttcap%
\pgfsetroundjoin%
\definecolor{currentfill}{rgb}{0.121569,0.466667,0.705882}%
\pgfsetfillcolor{currentfill}%
\pgfsetfillopacity{0.630592}%
\pgfsetlinewidth{1.003750pt}%
\definecolor{currentstroke}{rgb}{0.121569,0.466667,0.705882}%
\pgfsetstrokecolor{currentstroke}%
\pgfsetstrokeopacity{0.630592}%
\pgfsetdash{}{0pt}%
\pgfpathmoveto{\pgfqpoint{0.875380in}{1.249488in}}%
\pgfpathcurveto{\pgfqpoint{0.883616in}{1.249488in}}{\pgfqpoint{0.891516in}{1.252761in}}{\pgfqpoint{0.897340in}{1.258585in}}%
\pgfpathcurveto{\pgfqpoint{0.903164in}{1.264409in}}{\pgfqpoint{0.906436in}{1.272309in}}{\pgfqpoint{0.906436in}{1.280545in}}%
\pgfpathcurveto{\pgfqpoint{0.906436in}{1.288781in}}{\pgfqpoint{0.903164in}{1.296681in}}{\pgfqpoint{0.897340in}{1.302505in}}%
\pgfpathcurveto{\pgfqpoint{0.891516in}{1.308329in}}{\pgfqpoint{0.883616in}{1.311601in}}{\pgfqpoint{0.875380in}{1.311601in}}%
\pgfpathcurveto{\pgfqpoint{0.867144in}{1.311601in}}{\pgfqpoint{0.859244in}{1.308329in}}{\pgfqpoint{0.853420in}{1.302505in}}%
\pgfpathcurveto{\pgfqpoint{0.847596in}{1.296681in}}{\pgfqpoint{0.844323in}{1.288781in}}{\pgfqpoint{0.844323in}{1.280545in}}%
\pgfpathcurveto{\pgfqpoint{0.844323in}{1.272309in}}{\pgfqpoint{0.847596in}{1.264409in}}{\pgfqpoint{0.853420in}{1.258585in}}%
\pgfpathcurveto{\pgfqpoint{0.859244in}{1.252761in}}{\pgfqpoint{0.867144in}{1.249488in}}{\pgfqpoint{0.875380in}{1.249488in}}%
\pgfpathclose%
\pgfusepath{stroke,fill}%
\end{pgfscope}%
\begin{pgfscope}%
\pgfpathrectangle{\pgfqpoint{0.100000in}{0.212622in}}{\pgfqpoint{3.696000in}{3.696000in}}%
\pgfusepath{clip}%
\pgfsetbuttcap%
\pgfsetroundjoin%
\definecolor{currentfill}{rgb}{0.121569,0.466667,0.705882}%
\pgfsetfillcolor{currentfill}%
\pgfsetfillopacity{0.631050}%
\pgfsetlinewidth{1.003750pt}%
\definecolor{currentstroke}{rgb}{0.121569,0.466667,0.705882}%
\pgfsetstrokecolor{currentstroke}%
\pgfsetstrokeopacity{0.631050}%
\pgfsetdash{}{0pt}%
\pgfpathmoveto{\pgfqpoint{0.807483in}{1.299048in}}%
\pgfpathcurveto{\pgfqpoint{0.815719in}{1.299048in}}{\pgfqpoint{0.823619in}{1.302321in}}{\pgfqpoint{0.829443in}{1.308145in}}%
\pgfpathcurveto{\pgfqpoint{0.835267in}{1.313969in}}{\pgfqpoint{0.838539in}{1.321869in}}{\pgfqpoint{0.838539in}{1.330105in}}%
\pgfpathcurveto{\pgfqpoint{0.838539in}{1.338341in}}{\pgfqpoint{0.835267in}{1.346241in}}{\pgfqpoint{0.829443in}{1.352065in}}%
\pgfpathcurveto{\pgfqpoint{0.823619in}{1.357889in}}{\pgfqpoint{0.815719in}{1.361161in}}{\pgfqpoint{0.807483in}{1.361161in}}%
\pgfpathcurveto{\pgfqpoint{0.799246in}{1.361161in}}{\pgfqpoint{0.791346in}{1.357889in}}{\pgfqpoint{0.785522in}{1.352065in}}%
\pgfpathcurveto{\pgfqpoint{0.779698in}{1.346241in}}{\pgfqpoint{0.776426in}{1.338341in}}{\pgfqpoint{0.776426in}{1.330105in}}%
\pgfpathcurveto{\pgfqpoint{0.776426in}{1.321869in}}{\pgfqpoint{0.779698in}{1.313969in}}{\pgfqpoint{0.785522in}{1.308145in}}%
\pgfpathcurveto{\pgfqpoint{0.791346in}{1.302321in}}{\pgfqpoint{0.799246in}{1.299048in}}{\pgfqpoint{0.807483in}{1.299048in}}%
\pgfpathclose%
\pgfusepath{stroke,fill}%
\end{pgfscope}%
\begin{pgfscope}%
\pgfpathrectangle{\pgfqpoint{0.100000in}{0.212622in}}{\pgfqpoint{3.696000in}{3.696000in}}%
\pgfusepath{clip}%
\pgfsetbuttcap%
\pgfsetroundjoin%
\definecolor{currentfill}{rgb}{0.121569,0.466667,0.705882}%
\pgfsetfillcolor{currentfill}%
\pgfsetfillopacity{0.632731}%
\pgfsetlinewidth{1.003750pt}%
\definecolor{currentstroke}{rgb}{0.121569,0.466667,0.705882}%
\pgfsetstrokecolor{currentstroke}%
\pgfsetstrokeopacity{0.632731}%
\pgfsetdash{}{0pt}%
\pgfpathmoveto{\pgfqpoint{3.277214in}{2.239485in}}%
\pgfpathcurveto{\pgfqpoint{3.285450in}{2.239485in}}{\pgfqpoint{3.293350in}{2.242757in}}{\pgfqpoint{3.299174in}{2.248581in}}%
\pgfpathcurveto{\pgfqpoint{3.304998in}{2.254405in}}{\pgfqpoint{3.308270in}{2.262305in}}{\pgfqpoint{3.308270in}{2.270541in}}%
\pgfpathcurveto{\pgfqpoint{3.308270in}{2.278778in}}{\pgfqpoint{3.304998in}{2.286678in}}{\pgfqpoint{3.299174in}{2.292502in}}%
\pgfpathcurveto{\pgfqpoint{3.293350in}{2.298326in}}{\pgfqpoint{3.285450in}{2.301598in}}{\pgfqpoint{3.277214in}{2.301598in}}%
\pgfpathcurveto{\pgfqpoint{3.268977in}{2.301598in}}{\pgfqpoint{3.261077in}{2.298326in}}{\pgfqpoint{3.255253in}{2.292502in}}%
\pgfpathcurveto{\pgfqpoint{3.249429in}{2.286678in}}{\pgfqpoint{3.246157in}{2.278778in}}{\pgfqpoint{3.246157in}{2.270541in}}%
\pgfpathcurveto{\pgfqpoint{3.246157in}{2.262305in}}{\pgfqpoint{3.249429in}{2.254405in}}{\pgfqpoint{3.255253in}{2.248581in}}%
\pgfpathcurveto{\pgfqpoint{3.261077in}{2.242757in}}{\pgfqpoint{3.268977in}{2.239485in}}{\pgfqpoint{3.277214in}{2.239485in}}%
\pgfpathclose%
\pgfusepath{stroke,fill}%
\end{pgfscope}%
\begin{pgfscope}%
\pgfpathrectangle{\pgfqpoint{0.100000in}{0.212622in}}{\pgfqpoint{3.696000in}{3.696000in}}%
\pgfusepath{clip}%
\pgfsetbuttcap%
\pgfsetroundjoin%
\definecolor{currentfill}{rgb}{0.121569,0.466667,0.705882}%
\pgfsetfillcolor{currentfill}%
\pgfsetfillopacity{0.633566}%
\pgfsetlinewidth{1.003750pt}%
\definecolor{currentstroke}{rgb}{0.121569,0.466667,0.705882}%
\pgfsetstrokecolor{currentstroke}%
\pgfsetstrokeopacity{0.633566}%
\pgfsetdash{}{0pt}%
\pgfpathmoveto{\pgfqpoint{0.802360in}{1.290259in}}%
\pgfpathcurveto{\pgfqpoint{0.810596in}{1.290259in}}{\pgfqpoint{0.818496in}{1.293531in}}{\pgfqpoint{0.824320in}{1.299355in}}%
\pgfpathcurveto{\pgfqpoint{0.830144in}{1.305179in}}{\pgfqpoint{0.833416in}{1.313079in}}{\pgfqpoint{0.833416in}{1.321315in}}%
\pgfpathcurveto{\pgfqpoint{0.833416in}{1.329551in}}{\pgfqpoint{0.830144in}{1.337451in}}{\pgfqpoint{0.824320in}{1.343275in}}%
\pgfpathcurveto{\pgfqpoint{0.818496in}{1.349099in}}{\pgfqpoint{0.810596in}{1.352372in}}{\pgfqpoint{0.802360in}{1.352372in}}%
\pgfpathcurveto{\pgfqpoint{0.794124in}{1.352372in}}{\pgfqpoint{0.786224in}{1.349099in}}{\pgfqpoint{0.780400in}{1.343275in}}%
\pgfpathcurveto{\pgfqpoint{0.774576in}{1.337451in}}{\pgfqpoint{0.771303in}{1.329551in}}{\pgfqpoint{0.771303in}{1.321315in}}%
\pgfpathcurveto{\pgfqpoint{0.771303in}{1.313079in}}{\pgfqpoint{0.774576in}{1.305179in}}{\pgfqpoint{0.780400in}{1.299355in}}%
\pgfpathcurveto{\pgfqpoint{0.786224in}{1.293531in}}{\pgfqpoint{0.794124in}{1.290259in}}{\pgfqpoint{0.802360in}{1.290259in}}%
\pgfpathclose%
\pgfusepath{stroke,fill}%
\end{pgfscope}%
\begin{pgfscope}%
\pgfpathrectangle{\pgfqpoint{0.100000in}{0.212622in}}{\pgfqpoint{3.696000in}{3.696000in}}%
\pgfusepath{clip}%
\pgfsetbuttcap%
\pgfsetroundjoin%
\definecolor{currentfill}{rgb}{0.121569,0.466667,0.705882}%
\pgfsetfillcolor{currentfill}%
\pgfsetfillopacity{0.634704}%
\pgfsetlinewidth{1.003750pt}%
\definecolor{currentstroke}{rgb}{0.121569,0.466667,0.705882}%
\pgfsetstrokecolor{currentstroke}%
\pgfsetstrokeopacity{0.634704}%
\pgfsetdash{}{0pt}%
\pgfpathmoveto{\pgfqpoint{0.799605in}{1.287167in}}%
\pgfpathcurveto{\pgfqpoint{0.807842in}{1.287167in}}{\pgfqpoint{0.815742in}{1.290439in}}{\pgfqpoint{0.821566in}{1.296263in}}%
\pgfpathcurveto{\pgfqpoint{0.827390in}{1.302087in}}{\pgfqpoint{0.830662in}{1.309987in}}{\pgfqpoint{0.830662in}{1.318223in}}%
\pgfpathcurveto{\pgfqpoint{0.830662in}{1.326460in}}{\pgfqpoint{0.827390in}{1.334360in}}{\pgfqpoint{0.821566in}{1.340184in}}%
\pgfpathcurveto{\pgfqpoint{0.815742in}{1.346008in}}{\pgfqpoint{0.807842in}{1.349280in}}{\pgfqpoint{0.799605in}{1.349280in}}%
\pgfpathcurveto{\pgfqpoint{0.791369in}{1.349280in}}{\pgfqpoint{0.783469in}{1.346008in}}{\pgfqpoint{0.777645in}{1.340184in}}%
\pgfpathcurveto{\pgfqpoint{0.771821in}{1.334360in}}{\pgfqpoint{0.768549in}{1.326460in}}{\pgfqpoint{0.768549in}{1.318223in}}%
\pgfpathcurveto{\pgfqpoint{0.768549in}{1.309987in}}{\pgfqpoint{0.771821in}{1.302087in}}{\pgfqpoint{0.777645in}{1.296263in}}%
\pgfpathcurveto{\pgfqpoint{0.783469in}{1.290439in}}{\pgfqpoint{0.791369in}{1.287167in}}{\pgfqpoint{0.799605in}{1.287167in}}%
\pgfpathclose%
\pgfusepath{stroke,fill}%
\end{pgfscope}%
\begin{pgfscope}%
\pgfpathrectangle{\pgfqpoint{0.100000in}{0.212622in}}{\pgfqpoint{3.696000in}{3.696000in}}%
\pgfusepath{clip}%
\pgfsetbuttcap%
\pgfsetroundjoin%
\definecolor{currentfill}{rgb}{0.121569,0.466667,0.705882}%
\pgfsetfillcolor{currentfill}%
\pgfsetfillopacity{0.635213}%
\pgfsetlinewidth{1.003750pt}%
\definecolor{currentstroke}{rgb}{0.121569,0.466667,0.705882}%
\pgfsetstrokecolor{currentstroke}%
\pgfsetstrokeopacity{0.635213}%
\pgfsetdash{}{0pt}%
\pgfpathmoveto{\pgfqpoint{3.271036in}{2.236792in}}%
\pgfpathcurveto{\pgfqpoint{3.279272in}{2.236792in}}{\pgfqpoint{3.287172in}{2.240064in}}{\pgfqpoint{3.292996in}{2.245888in}}%
\pgfpathcurveto{\pgfqpoint{3.298820in}{2.251712in}}{\pgfqpoint{3.302093in}{2.259612in}}{\pgfqpoint{3.302093in}{2.267848in}}%
\pgfpathcurveto{\pgfqpoint{3.302093in}{2.276084in}}{\pgfqpoint{3.298820in}{2.283984in}}{\pgfqpoint{3.292996in}{2.289808in}}%
\pgfpathcurveto{\pgfqpoint{3.287172in}{2.295632in}}{\pgfqpoint{3.279272in}{2.298905in}}{\pgfqpoint{3.271036in}{2.298905in}}%
\pgfpathcurveto{\pgfqpoint{3.262800in}{2.298905in}}{\pgfqpoint{3.254900in}{2.295632in}}{\pgfqpoint{3.249076in}{2.289808in}}%
\pgfpathcurveto{\pgfqpoint{3.243252in}{2.283984in}}{\pgfqpoint{3.239980in}{2.276084in}}{\pgfqpoint{3.239980in}{2.267848in}}%
\pgfpathcurveto{\pgfqpoint{3.239980in}{2.259612in}}{\pgfqpoint{3.243252in}{2.251712in}}{\pgfqpoint{3.249076in}{2.245888in}}%
\pgfpathcurveto{\pgfqpoint{3.254900in}{2.240064in}}{\pgfqpoint{3.262800in}{2.236792in}}{\pgfqpoint{3.271036in}{2.236792in}}%
\pgfpathclose%
\pgfusepath{stroke,fill}%
\end{pgfscope}%
\begin{pgfscope}%
\pgfpathrectangle{\pgfqpoint{0.100000in}{0.212622in}}{\pgfqpoint{3.696000in}{3.696000in}}%
\pgfusepath{clip}%
\pgfsetbuttcap%
\pgfsetroundjoin%
\definecolor{currentfill}{rgb}{0.121569,0.466667,0.705882}%
\pgfsetfillcolor{currentfill}%
\pgfsetfillopacity{0.635375}%
\pgfsetlinewidth{1.003750pt}%
\definecolor{currentstroke}{rgb}{0.121569,0.466667,0.705882}%
\pgfsetstrokecolor{currentstroke}%
\pgfsetstrokeopacity{0.635375}%
\pgfsetdash{}{0pt}%
\pgfpathmoveto{\pgfqpoint{0.851587in}{1.243957in}}%
\pgfpathcurveto{\pgfqpoint{0.859823in}{1.243957in}}{\pgfqpoint{0.867723in}{1.247230in}}{\pgfqpoint{0.873547in}{1.253054in}}%
\pgfpathcurveto{\pgfqpoint{0.879371in}{1.258878in}}{\pgfqpoint{0.882644in}{1.266778in}}{\pgfqpoint{0.882644in}{1.275014in}}%
\pgfpathcurveto{\pgfqpoint{0.882644in}{1.283250in}}{\pgfqpoint{0.879371in}{1.291150in}}{\pgfqpoint{0.873547in}{1.296974in}}%
\pgfpathcurveto{\pgfqpoint{0.867723in}{1.302798in}}{\pgfqpoint{0.859823in}{1.306070in}}{\pgfqpoint{0.851587in}{1.306070in}}%
\pgfpathcurveto{\pgfqpoint{0.843351in}{1.306070in}}{\pgfqpoint{0.835451in}{1.302798in}}{\pgfqpoint{0.829627in}{1.296974in}}%
\pgfpathcurveto{\pgfqpoint{0.823803in}{1.291150in}}{\pgfqpoint{0.820531in}{1.283250in}}{\pgfqpoint{0.820531in}{1.275014in}}%
\pgfpathcurveto{\pgfqpoint{0.820531in}{1.266778in}}{\pgfqpoint{0.823803in}{1.258878in}}{\pgfqpoint{0.829627in}{1.253054in}}%
\pgfpathcurveto{\pgfqpoint{0.835451in}{1.247230in}}{\pgfqpoint{0.843351in}{1.243957in}}{\pgfqpoint{0.851587in}{1.243957in}}%
\pgfpathclose%
\pgfusepath{stroke,fill}%
\end{pgfscope}%
\begin{pgfscope}%
\pgfpathrectangle{\pgfqpoint{0.100000in}{0.212622in}}{\pgfqpoint{3.696000in}{3.696000in}}%
\pgfusepath{clip}%
\pgfsetbuttcap%
\pgfsetroundjoin%
\definecolor{currentfill}{rgb}{0.121569,0.466667,0.705882}%
\pgfsetfillcolor{currentfill}%
\pgfsetfillopacity{0.636391}%
\pgfsetlinewidth{1.003750pt}%
\definecolor{currentstroke}{rgb}{0.121569,0.466667,0.705882}%
\pgfsetstrokecolor{currentstroke}%
\pgfsetstrokeopacity{0.636391}%
\pgfsetdash{}{0pt}%
\pgfpathmoveto{\pgfqpoint{0.793122in}{1.280919in}}%
\pgfpathcurveto{\pgfqpoint{0.801359in}{1.280919in}}{\pgfqpoint{0.809259in}{1.284191in}}{\pgfqpoint{0.815083in}{1.290015in}}%
\pgfpathcurveto{\pgfqpoint{0.820906in}{1.295839in}}{\pgfqpoint{0.824179in}{1.303739in}}{\pgfqpoint{0.824179in}{1.311976in}}%
\pgfpathcurveto{\pgfqpoint{0.824179in}{1.320212in}}{\pgfqpoint{0.820906in}{1.328112in}}{\pgfqpoint{0.815083in}{1.333936in}}%
\pgfpathcurveto{\pgfqpoint{0.809259in}{1.339760in}}{\pgfqpoint{0.801359in}{1.343032in}}{\pgfqpoint{0.793122in}{1.343032in}}%
\pgfpathcurveto{\pgfqpoint{0.784886in}{1.343032in}}{\pgfqpoint{0.776986in}{1.339760in}}{\pgfqpoint{0.771162in}{1.333936in}}%
\pgfpathcurveto{\pgfqpoint{0.765338in}{1.328112in}}{\pgfqpoint{0.762066in}{1.320212in}}{\pgfqpoint{0.762066in}{1.311976in}}%
\pgfpathcurveto{\pgfqpoint{0.762066in}{1.303739in}}{\pgfqpoint{0.765338in}{1.295839in}}{\pgfqpoint{0.771162in}{1.290015in}}%
\pgfpathcurveto{\pgfqpoint{0.776986in}{1.284191in}}{\pgfqpoint{0.784886in}{1.280919in}}{\pgfqpoint{0.793122in}{1.280919in}}%
\pgfpathclose%
\pgfusepath{stroke,fill}%
\end{pgfscope}%
\begin{pgfscope}%
\pgfpathrectangle{\pgfqpoint{0.100000in}{0.212622in}}{\pgfqpoint{3.696000in}{3.696000in}}%
\pgfusepath{clip}%
\pgfsetbuttcap%
\pgfsetroundjoin%
\definecolor{currentfill}{rgb}{0.121569,0.466667,0.705882}%
\pgfsetfillcolor{currentfill}%
\pgfsetfillopacity{0.636703}%
\pgfsetlinewidth{1.003750pt}%
\definecolor{currentstroke}{rgb}{0.121569,0.466667,0.705882}%
\pgfsetstrokecolor{currentstroke}%
\pgfsetstrokeopacity{0.636703}%
\pgfsetdash{}{0pt}%
\pgfpathmoveto{\pgfqpoint{3.267853in}{2.235862in}}%
\pgfpathcurveto{\pgfqpoint{3.276089in}{2.235862in}}{\pgfqpoint{3.283989in}{2.239134in}}{\pgfqpoint{3.289813in}{2.244958in}}%
\pgfpathcurveto{\pgfqpoint{3.295637in}{2.250782in}}{\pgfqpoint{3.298910in}{2.258682in}}{\pgfqpoint{3.298910in}{2.266918in}}%
\pgfpathcurveto{\pgfqpoint{3.298910in}{2.275155in}}{\pgfqpoint{3.295637in}{2.283055in}}{\pgfqpoint{3.289813in}{2.288879in}}%
\pgfpathcurveto{\pgfqpoint{3.283989in}{2.294703in}}{\pgfqpoint{3.276089in}{2.297975in}}{\pgfqpoint{3.267853in}{2.297975in}}%
\pgfpathcurveto{\pgfqpoint{3.259617in}{2.297975in}}{\pgfqpoint{3.251717in}{2.294703in}}{\pgfqpoint{3.245893in}{2.288879in}}%
\pgfpathcurveto{\pgfqpoint{3.240069in}{2.283055in}}{\pgfqpoint{3.236797in}{2.275155in}}{\pgfqpoint{3.236797in}{2.266918in}}%
\pgfpathcurveto{\pgfqpoint{3.236797in}{2.258682in}}{\pgfqpoint{3.240069in}{2.250782in}}{\pgfqpoint{3.245893in}{2.244958in}}%
\pgfpathcurveto{\pgfqpoint{3.251717in}{2.239134in}}{\pgfqpoint{3.259617in}{2.235862in}}{\pgfqpoint{3.267853in}{2.235862in}}%
\pgfpathclose%
\pgfusepath{stroke,fill}%
\end{pgfscope}%
\begin{pgfscope}%
\pgfpathrectangle{\pgfqpoint{0.100000in}{0.212622in}}{\pgfqpoint{3.696000in}{3.696000in}}%
\pgfusepath{clip}%
\pgfsetbuttcap%
\pgfsetroundjoin%
\definecolor{currentfill}{rgb}{0.121569,0.466667,0.705882}%
\pgfsetfillcolor{currentfill}%
\pgfsetfillopacity{0.638789}%
\pgfsetlinewidth{1.003750pt}%
\definecolor{currentstroke}{rgb}{0.121569,0.466667,0.705882}%
\pgfsetstrokecolor{currentstroke}%
\pgfsetstrokeopacity{0.638789}%
\pgfsetdash{}{0pt}%
\pgfpathmoveto{\pgfqpoint{3.263091in}{2.234798in}}%
\pgfpathcurveto{\pgfqpoint{3.271328in}{2.234798in}}{\pgfqpoint{3.279228in}{2.238070in}}{\pgfqpoint{3.285052in}{2.243894in}}%
\pgfpathcurveto{\pgfqpoint{3.290876in}{2.249718in}}{\pgfqpoint{3.294148in}{2.257618in}}{\pgfqpoint{3.294148in}{2.265854in}}%
\pgfpathcurveto{\pgfqpoint{3.294148in}{2.274090in}}{\pgfqpoint{3.290876in}{2.281991in}}{\pgfqpoint{3.285052in}{2.287814in}}%
\pgfpathcurveto{\pgfqpoint{3.279228in}{2.293638in}}{\pgfqpoint{3.271328in}{2.296911in}}{\pgfqpoint{3.263091in}{2.296911in}}%
\pgfpathcurveto{\pgfqpoint{3.254855in}{2.296911in}}{\pgfqpoint{3.246955in}{2.293638in}}{\pgfqpoint{3.241131in}{2.287814in}}%
\pgfpathcurveto{\pgfqpoint{3.235307in}{2.281991in}}{\pgfqpoint{3.232035in}{2.274090in}}{\pgfqpoint{3.232035in}{2.265854in}}%
\pgfpathcurveto{\pgfqpoint{3.232035in}{2.257618in}}{\pgfqpoint{3.235307in}{2.249718in}}{\pgfqpoint{3.241131in}{2.243894in}}%
\pgfpathcurveto{\pgfqpoint{3.246955in}{2.238070in}}{\pgfqpoint{3.254855in}{2.234798in}}{\pgfqpoint{3.263091in}{2.234798in}}%
\pgfpathclose%
\pgfusepath{stroke,fill}%
\end{pgfscope}%
\begin{pgfscope}%
\pgfpathrectangle{\pgfqpoint{0.100000in}{0.212622in}}{\pgfqpoint{3.696000in}{3.696000in}}%
\pgfusepath{clip}%
\pgfsetbuttcap%
\pgfsetroundjoin%
\definecolor{currentfill}{rgb}{0.121569,0.466667,0.705882}%
\pgfsetfillcolor{currentfill}%
\pgfsetfillopacity{0.639776}%
\pgfsetlinewidth{1.003750pt}%
\definecolor{currentstroke}{rgb}{0.121569,0.466667,0.705882}%
\pgfsetstrokecolor{currentstroke}%
\pgfsetstrokeopacity{0.639776}%
\pgfsetdash{}{0pt}%
\pgfpathmoveto{\pgfqpoint{3.260143in}{2.233619in}}%
\pgfpathcurveto{\pgfqpoint{3.268379in}{2.233619in}}{\pgfqpoint{3.276279in}{2.236891in}}{\pgfqpoint{3.282103in}{2.242715in}}%
\pgfpathcurveto{\pgfqpoint{3.287927in}{2.248539in}}{\pgfqpoint{3.291199in}{2.256439in}}{\pgfqpoint{3.291199in}{2.264675in}}%
\pgfpathcurveto{\pgfqpoint{3.291199in}{2.272911in}}{\pgfqpoint{3.287927in}{2.280811in}}{\pgfqpoint{3.282103in}{2.286635in}}%
\pgfpathcurveto{\pgfqpoint{3.276279in}{2.292459in}}{\pgfqpoint{3.268379in}{2.295732in}}{\pgfqpoint{3.260143in}{2.295732in}}%
\pgfpathcurveto{\pgfqpoint{3.251907in}{2.295732in}}{\pgfqpoint{3.244007in}{2.292459in}}{\pgfqpoint{3.238183in}{2.286635in}}%
\pgfpathcurveto{\pgfqpoint{3.232359in}{2.280811in}}{\pgfqpoint{3.229086in}{2.272911in}}{\pgfqpoint{3.229086in}{2.264675in}}%
\pgfpathcurveto{\pgfqpoint{3.229086in}{2.256439in}}{\pgfqpoint{3.232359in}{2.248539in}}{\pgfqpoint{3.238183in}{2.242715in}}%
\pgfpathcurveto{\pgfqpoint{3.244007in}{2.236891in}}{\pgfqpoint{3.251907in}{2.233619in}}{\pgfqpoint{3.260143in}{2.233619in}}%
\pgfpathclose%
\pgfusepath{stroke,fill}%
\end{pgfscope}%
\begin{pgfscope}%
\pgfpathrectangle{\pgfqpoint{0.100000in}{0.212622in}}{\pgfqpoint{3.696000in}{3.696000in}}%
\pgfusepath{clip}%
\pgfsetbuttcap%
\pgfsetroundjoin%
\definecolor{currentfill}{rgb}{0.121569,0.466667,0.705882}%
\pgfsetfillcolor{currentfill}%
\pgfsetfillopacity{0.640274}%
\pgfsetlinewidth{1.003750pt}%
\definecolor{currentstroke}{rgb}{0.121569,0.466667,0.705882}%
\pgfsetstrokecolor{currentstroke}%
\pgfsetstrokeopacity{0.640274}%
\pgfsetdash{}{0pt}%
\pgfpathmoveto{\pgfqpoint{0.786323in}{1.269630in}}%
\pgfpathcurveto{\pgfqpoint{0.794559in}{1.269630in}}{\pgfqpoint{0.802459in}{1.272902in}}{\pgfqpoint{0.808283in}{1.278726in}}%
\pgfpathcurveto{\pgfqpoint{0.814107in}{1.284550in}}{\pgfqpoint{0.817380in}{1.292450in}}{\pgfqpoint{0.817380in}{1.300686in}}%
\pgfpathcurveto{\pgfqpoint{0.817380in}{1.308923in}}{\pgfqpoint{0.814107in}{1.316823in}}{\pgfqpoint{0.808283in}{1.322647in}}%
\pgfpathcurveto{\pgfqpoint{0.802459in}{1.328471in}}{\pgfqpoint{0.794559in}{1.331743in}}{\pgfqpoint{0.786323in}{1.331743in}}%
\pgfpathcurveto{\pgfqpoint{0.778087in}{1.331743in}}{\pgfqpoint{0.770187in}{1.328471in}}{\pgfqpoint{0.764363in}{1.322647in}}%
\pgfpathcurveto{\pgfqpoint{0.758539in}{1.316823in}}{\pgfqpoint{0.755267in}{1.308923in}}{\pgfqpoint{0.755267in}{1.300686in}}%
\pgfpathcurveto{\pgfqpoint{0.755267in}{1.292450in}}{\pgfqpoint{0.758539in}{1.284550in}}{\pgfqpoint{0.764363in}{1.278726in}}%
\pgfpathcurveto{\pgfqpoint{0.770187in}{1.272902in}}{\pgfqpoint{0.778087in}{1.269630in}}{\pgfqpoint{0.786323in}{1.269630in}}%
\pgfpathclose%
\pgfusepath{stroke,fill}%
\end{pgfscope}%
\begin{pgfscope}%
\pgfpathrectangle{\pgfqpoint{0.100000in}{0.212622in}}{\pgfqpoint{3.696000in}{3.696000in}}%
\pgfusepath{clip}%
\pgfsetbuttcap%
\pgfsetroundjoin%
\definecolor{currentfill}{rgb}{0.121569,0.466667,0.705882}%
\pgfsetfillcolor{currentfill}%
\pgfsetfillopacity{0.641450}%
\pgfsetlinewidth{1.003750pt}%
\definecolor{currentstroke}{rgb}{0.121569,0.466667,0.705882}%
\pgfsetstrokecolor{currentstroke}%
\pgfsetstrokeopacity{0.641450}%
\pgfsetdash{}{0pt}%
\pgfpathmoveto{\pgfqpoint{0.824958in}{1.238196in}}%
\pgfpathcurveto{\pgfqpoint{0.833194in}{1.238196in}}{\pgfqpoint{0.841094in}{1.241468in}}{\pgfqpoint{0.846918in}{1.247292in}}%
\pgfpathcurveto{\pgfqpoint{0.852742in}{1.253116in}}{\pgfqpoint{0.856015in}{1.261016in}}{\pgfqpoint{0.856015in}{1.269252in}}%
\pgfpathcurveto{\pgfqpoint{0.856015in}{1.277488in}}{\pgfqpoint{0.852742in}{1.285388in}}{\pgfqpoint{0.846918in}{1.291212in}}%
\pgfpathcurveto{\pgfqpoint{0.841094in}{1.297036in}}{\pgfqpoint{0.833194in}{1.300309in}}{\pgfqpoint{0.824958in}{1.300309in}}%
\pgfpathcurveto{\pgfqpoint{0.816722in}{1.300309in}}{\pgfqpoint{0.808822in}{1.297036in}}{\pgfqpoint{0.802998in}{1.291212in}}%
\pgfpathcurveto{\pgfqpoint{0.797174in}{1.285388in}}{\pgfqpoint{0.793902in}{1.277488in}}{\pgfqpoint{0.793902in}{1.269252in}}%
\pgfpathcurveto{\pgfqpoint{0.793902in}{1.261016in}}{\pgfqpoint{0.797174in}{1.253116in}}{\pgfqpoint{0.802998in}{1.247292in}}%
\pgfpathcurveto{\pgfqpoint{0.808822in}{1.241468in}}{\pgfqpoint{0.816722in}{1.238196in}}{\pgfqpoint{0.824958in}{1.238196in}}%
\pgfpathclose%
\pgfusepath{stroke,fill}%
\end{pgfscope}%
\begin{pgfscope}%
\pgfpathrectangle{\pgfqpoint{0.100000in}{0.212622in}}{\pgfqpoint{3.696000in}{3.696000in}}%
\pgfusepath{clip}%
\pgfsetbuttcap%
\pgfsetroundjoin%
\definecolor{currentfill}{rgb}{0.121569,0.466667,0.705882}%
\pgfsetfillcolor{currentfill}%
\pgfsetfillopacity{0.641661}%
\pgfsetlinewidth{1.003750pt}%
\definecolor{currentstroke}{rgb}{0.121569,0.466667,0.705882}%
\pgfsetstrokecolor{currentstroke}%
\pgfsetstrokeopacity{0.641661}%
\pgfsetdash{}{0pt}%
\pgfpathmoveto{\pgfqpoint{3.255978in}{2.231338in}}%
\pgfpathcurveto{\pgfqpoint{3.264214in}{2.231338in}}{\pgfqpoint{3.272114in}{2.234610in}}{\pgfqpoint{3.277938in}{2.240434in}}%
\pgfpathcurveto{\pgfqpoint{3.283762in}{2.246258in}}{\pgfqpoint{3.287034in}{2.254158in}}{\pgfqpoint{3.287034in}{2.262394in}}%
\pgfpathcurveto{\pgfqpoint{3.287034in}{2.270631in}}{\pgfqpoint{3.283762in}{2.278531in}}{\pgfqpoint{3.277938in}{2.284355in}}%
\pgfpathcurveto{\pgfqpoint{3.272114in}{2.290179in}}{\pgfqpoint{3.264214in}{2.293451in}}{\pgfqpoint{3.255978in}{2.293451in}}%
\pgfpathcurveto{\pgfqpoint{3.247741in}{2.293451in}}{\pgfqpoint{3.239841in}{2.290179in}}{\pgfqpoint{3.234017in}{2.284355in}}%
\pgfpathcurveto{\pgfqpoint{3.228193in}{2.278531in}}{\pgfqpoint{3.224921in}{2.270631in}}{\pgfqpoint{3.224921in}{2.262394in}}%
\pgfpathcurveto{\pgfqpoint{3.224921in}{2.254158in}}{\pgfqpoint{3.228193in}{2.246258in}}{\pgfqpoint{3.234017in}{2.240434in}}%
\pgfpathcurveto{\pgfqpoint{3.239841in}{2.234610in}}{\pgfqpoint{3.247741in}{2.231338in}}{\pgfqpoint{3.255978in}{2.231338in}}%
\pgfpathclose%
\pgfusepath{stroke,fill}%
\end{pgfscope}%
\begin{pgfscope}%
\pgfpathrectangle{\pgfqpoint{0.100000in}{0.212622in}}{\pgfqpoint{3.696000in}{3.696000in}}%
\pgfusepath{clip}%
\pgfsetbuttcap%
\pgfsetroundjoin%
\definecolor{currentfill}{rgb}{0.121569,0.466667,0.705882}%
\pgfsetfillcolor{currentfill}%
\pgfsetfillopacity{0.643860}%
\pgfsetlinewidth{1.003750pt}%
\definecolor{currentstroke}{rgb}{0.121569,0.466667,0.705882}%
\pgfsetstrokecolor{currentstroke}%
\pgfsetstrokeopacity{0.643860}%
\pgfsetdash{}{0pt}%
\pgfpathmoveto{\pgfqpoint{3.247332in}{2.228499in}}%
\pgfpathcurveto{\pgfqpoint{3.255568in}{2.228499in}}{\pgfqpoint{3.263468in}{2.231771in}}{\pgfqpoint{3.269292in}{2.237595in}}%
\pgfpathcurveto{\pgfqpoint{3.275116in}{2.243419in}}{\pgfqpoint{3.278388in}{2.251319in}}{\pgfqpoint{3.278388in}{2.259555in}}%
\pgfpathcurveto{\pgfqpoint{3.278388in}{2.267792in}}{\pgfqpoint{3.275116in}{2.275692in}}{\pgfqpoint{3.269292in}{2.281516in}}%
\pgfpathcurveto{\pgfqpoint{3.263468in}{2.287340in}}{\pgfqpoint{3.255568in}{2.290612in}}{\pgfqpoint{3.247332in}{2.290612in}}%
\pgfpathcurveto{\pgfqpoint{3.239095in}{2.290612in}}{\pgfqpoint{3.231195in}{2.287340in}}{\pgfqpoint{3.225371in}{2.281516in}}%
\pgfpathcurveto{\pgfqpoint{3.219547in}{2.275692in}}{\pgfqpoint{3.216275in}{2.267792in}}{\pgfqpoint{3.216275in}{2.259555in}}%
\pgfpathcurveto{\pgfqpoint{3.216275in}{2.251319in}}{\pgfqpoint{3.219547in}{2.243419in}}{\pgfqpoint{3.225371in}{2.237595in}}%
\pgfpathcurveto{\pgfqpoint{3.231195in}{2.231771in}}{\pgfqpoint{3.239095in}{2.228499in}}{\pgfqpoint{3.247332in}{2.228499in}}%
\pgfpathclose%
\pgfusepath{stroke,fill}%
\end{pgfscope}%
\begin{pgfscope}%
\pgfpathrectangle{\pgfqpoint{0.100000in}{0.212622in}}{\pgfqpoint{3.696000in}{3.696000in}}%
\pgfusepath{clip}%
\pgfsetbuttcap%
\pgfsetroundjoin%
\definecolor{currentfill}{rgb}{0.121569,0.466667,0.705882}%
\pgfsetfillcolor{currentfill}%
\pgfsetfillopacity{0.645406}%
\pgfsetlinewidth{1.003750pt}%
\definecolor{currentstroke}{rgb}{0.121569,0.466667,0.705882}%
\pgfsetstrokecolor{currentstroke}%
\pgfsetstrokeopacity{0.645406}%
\pgfsetdash{}{0pt}%
\pgfpathmoveto{\pgfqpoint{3.243650in}{2.227316in}}%
\pgfpathcurveto{\pgfqpoint{3.251886in}{2.227316in}}{\pgfqpoint{3.259786in}{2.230588in}}{\pgfqpoint{3.265610in}{2.236412in}}%
\pgfpathcurveto{\pgfqpoint{3.271434in}{2.242236in}}{\pgfqpoint{3.274706in}{2.250136in}}{\pgfqpoint{3.274706in}{2.258372in}}%
\pgfpathcurveto{\pgfqpoint{3.274706in}{2.266609in}}{\pgfqpoint{3.271434in}{2.274509in}}{\pgfqpoint{3.265610in}{2.280333in}}%
\pgfpathcurveto{\pgfqpoint{3.259786in}{2.286156in}}{\pgfqpoint{3.251886in}{2.289429in}}{\pgfqpoint{3.243650in}{2.289429in}}%
\pgfpathcurveto{\pgfqpoint{3.235413in}{2.289429in}}{\pgfqpoint{3.227513in}{2.286156in}}{\pgfqpoint{3.221689in}{2.280333in}}%
\pgfpathcurveto{\pgfqpoint{3.215865in}{2.274509in}}{\pgfqpoint{3.212593in}{2.266609in}}{\pgfqpoint{3.212593in}{2.258372in}}%
\pgfpathcurveto{\pgfqpoint{3.212593in}{2.250136in}}{\pgfqpoint{3.215865in}{2.242236in}}{\pgfqpoint{3.221689in}{2.236412in}}%
\pgfpathcurveto{\pgfqpoint{3.227513in}{2.230588in}}{\pgfqpoint{3.235413in}{2.227316in}}{\pgfqpoint{3.243650in}{2.227316in}}%
\pgfpathclose%
\pgfusepath{stroke,fill}%
\end{pgfscope}%
\begin{pgfscope}%
\pgfpathrectangle{\pgfqpoint{0.100000in}{0.212622in}}{\pgfqpoint{3.696000in}{3.696000in}}%
\pgfusepath{clip}%
\pgfsetbuttcap%
\pgfsetroundjoin%
\definecolor{currentfill}{rgb}{0.121569,0.466667,0.705882}%
\pgfsetfillcolor{currentfill}%
\pgfsetfillopacity{0.646072}%
\pgfsetlinewidth{1.003750pt}%
\definecolor{currentstroke}{rgb}{0.121569,0.466667,0.705882}%
\pgfsetstrokecolor{currentstroke}%
\pgfsetstrokeopacity{0.646072}%
\pgfsetdash{}{0pt}%
\pgfpathmoveto{\pgfqpoint{3.241034in}{2.226449in}}%
\pgfpathcurveto{\pgfqpoint{3.249271in}{2.226449in}}{\pgfqpoint{3.257171in}{2.229722in}}{\pgfqpoint{3.262995in}{2.235546in}}%
\pgfpathcurveto{\pgfqpoint{3.268819in}{2.241370in}}{\pgfqpoint{3.272091in}{2.249270in}}{\pgfqpoint{3.272091in}{2.257506in}}%
\pgfpathcurveto{\pgfqpoint{3.272091in}{2.265742in}}{\pgfqpoint{3.268819in}{2.273642in}}{\pgfqpoint{3.262995in}{2.279466in}}%
\pgfpathcurveto{\pgfqpoint{3.257171in}{2.285290in}}{\pgfqpoint{3.249271in}{2.288562in}}{\pgfqpoint{3.241034in}{2.288562in}}%
\pgfpathcurveto{\pgfqpoint{3.232798in}{2.288562in}}{\pgfqpoint{3.224898in}{2.285290in}}{\pgfqpoint{3.219074in}{2.279466in}}%
\pgfpathcurveto{\pgfqpoint{3.213250in}{2.273642in}}{\pgfqpoint{3.209978in}{2.265742in}}{\pgfqpoint{3.209978in}{2.257506in}}%
\pgfpathcurveto{\pgfqpoint{3.209978in}{2.249270in}}{\pgfqpoint{3.213250in}{2.241370in}}{\pgfqpoint{3.219074in}{2.235546in}}%
\pgfpathcurveto{\pgfqpoint{3.224898in}{2.229722in}}{\pgfqpoint{3.232798in}{2.226449in}}{\pgfqpoint{3.241034in}{2.226449in}}%
\pgfpathclose%
\pgfusepath{stroke,fill}%
\end{pgfscope}%
\begin{pgfscope}%
\pgfpathrectangle{\pgfqpoint{0.100000in}{0.212622in}}{\pgfqpoint{3.696000in}{3.696000in}}%
\pgfusepath{clip}%
\pgfsetbuttcap%
\pgfsetroundjoin%
\definecolor{currentfill}{rgb}{0.121569,0.466667,0.705882}%
\pgfsetfillcolor{currentfill}%
\pgfsetfillopacity{0.646548}%
\pgfsetlinewidth{1.003750pt}%
\definecolor{currentstroke}{rgb}{0.121569,0.466667,0.705882}%
\pgfsetstrokecolor{currentstroke}%
\pgfsetstrokeopacity{0.646548}%
\pgfsetdash{}{0pt}%
\pgfpathmoveto{\pgfqpoint{3.239863in}{2.226237in}}%
\pgfpathcurveto{\pgfqpoint{3.248100in}{2.226237in}}{\pgfqpoint{3.256000in}{2.229509in}}{\pgfqpoint{3.261824in}{2.235333in}}%
\pgfpathcurveto{\pgfqpoint{3.267648in}{2.241157in}}{\pgfqpoint{3.270920in}{2.249057in}}{\pgfqpoint{3.270920in}{2.257294in}}%
\pgfpathcurveto{\pgfqpoint{3.270920in}{2.265530in}}{\pgfqpoint{3.267648in}{2.273430in}}{\pgfqpoint{3.261824in}{2.279254in}}%
\pgfpathcurveto{\pgfqpoint{3.256000in}{2.285078in}}{\pgfqpoint{3.248100in}{2.288350in}}{\pgfqpoint{3.239863in}{2.288350in}}%
\pgfpathcurveto{\pgfqpoint{3.231627in}{2.288350in}}{\pgfqpoint{3.223727in}{2.285078in}}{\pgfqpoint{3.217903in}{2.279254in}}%
\pgfpathcurveto{\pgfqpoint{3.212079in}{2.273430in}}{\pgfqpoint{3.208807in}{2.265530in}}{\pgfqpoint{3.208807in}{2.257294in}}%
\pgfpathcurveto{\pgfqpoint{3.208807in}{2.249057in}}{\pgfqpoint{3.212079in}{2.241157in}}{\pgfqpoint{3.217903in}{2.235333in}}%
\pgfpathcurveto{\pgfqpoint{3.223727in}{2.229509in}}{\pgfqpoint{3.231627in}{2.226237in}}{\pgfqpoint{3.239863in}{2.226237in}}%
\pgfpathclose%
\pgfusepath{stroke,fill}%
\end{pgfscope}%
\begin{pgfscope}%
\pgfpathrectangle{\pgfqpoint{0.100000in}{0.212622in}}{\pgfqpoint{3.696000in}{3.696000in}}%
\pgfusepath{clip}%
\pgfsetbuttcap%
\pgfsetroundjoin%
\definecolor{currentfill}{rgb}{0.121569,0.466667,0.705882}%
\pgfsetfillcolor{currentfill}%
\pgfsetfillopacity{0.646639}%
\pgfsetlinewidth{1.003750pt}%
\definecolor{currentstroke}{rgb}{0.121569,0.466667,0.705882}%
\pgfsetstrokecolor{currentstroke}%
\pgfsetstrokeopacity{0.646639}%
\pgfsetdash{}{0pt}%
\pgfpathmoveto{\pgfqpoint{0.762702in}{1.256866in}}%
\pgfpathcurveto{\pgfqpoint{0.770938in}{1.256866in}}{\pgfqpoint{0.778838in}{1.260139in}}{\pgfqpoint{0.784662in}{1.265963in}}%
\pgfpathcurveto{\pgfqpoint{0.790486in}{1.271787in}}{\pgfqpoint{0.793759in}{1.279687in}}{\pgfqpoint{0.793759in}{1.287923in}}%
\pgfpathcurveto{\pgfqpoint{0.793759in}{1.296159in}}{\pgfqpoint{0.790486in}{1.304059in}}{\pgfqpoint{0.784662in}{1.309883in}}%
\pgfpathcurveto{\pgfqpoint{0.778838in}{1.315707in}}{\pgfqpoint{0.770938in}{1.318979in}}{\pgfqpoint{0.762702in}{1.318979in}}%
\pgfpathcurveto{\pgfqpoint{0.754466in}{1.318979in}}{\pgfqpoint{0.746566in}{1.315707in}}{\pgfqpoint{0.740742in}{1.309883in}}%
\pgfpathcurveto{\pgfqpoint{0.734918in}{1.304059in}}{\pgfqpoint{0.731646in}{1.296159in}}{\pgfqpoint{0.731646in}{1.287923in}}%
\pgfpathcurveto{\pgfqpoint{0.731646in}{1.279687in}}{\pgfqpoint{0.734918in}{1.271787in}}{\pgfqpoint{0.740742in}{1.265963in}}%
\pgfpathcurveto{\pgfqpoint{0.746566in}{1.260139in}}{\pgfqpoint{0.754466in}{1.256866in}}{\pgfqpoint{0.762702in}{1.256866in}}%
\pgfpathclose%
\pgfusepath{stroke,fill}%
\end{pgfscope}%
\begin{pgfscope}%
\pgfpathrectangle{\pgfqpoint{0.100000in}{0.212622in}}{\pgfqpoint{3.696000in}{3.696000in}}%
\pgfusepath{clip}%
\pgfsetbuttcap%
\pgfsetroundjoin%
\definecolor{currentfill}{rgb}{0.121569,0.466667,0.705882}%
\pgfsetfillcolor{currentfill}%
\pgfsetfillopacity{0.646752}%
\pgfsetlinewidth{1.003750pt}%
\definecolor{currentstroke}{rgb}{0.121569,0.466667,0.705882}%
\pgfsetstrokecolor{currentstroke}%
\pgfsetstrokeopacity{0.646752}%
\pgfsetdash{}{0pt}%
\pgfpathmoveto{\pgfqpoint{3.239092in}{2.225945in}}%
\pgfpathcurveto{\pgfqpoint{3.247328in}{2.225945in}}{\pgfqpoint{3.255228in}{2.229217in}}{\pgfqpoint{3.261052in}{2.235041in}}%
\pgfpathcurveto{\pgfqpoint{3.266876in}{2.240865in}}{\pgfqpoint{3.270148in}{2.248765in}}{\pgfqpoint{3.270148in}{2.257001in}}%
\pgfpathcurveto{\pgfqpoint{3.270148in}{2.265238in}}{\pgfqpoint{3.266876in}{2.273138in}}{\pgfqpoint{3.261052in}{2.278962in}}%
\pgfpathcurveto{\pgfqpoint{3.255228in}{2.284786in}}{\pgfqpoint{3.247328in}{2.288058in}}{\pgfqpoint{3.239092in}{2.288058in}}%
\pgfpathcurveto{\pgfqpoint{3.230856in}{2.288058in}}{\pgfqpoint{3.222955in}{2.284786in}}{\pgfqpoint{3.217132in}{2.278962in}}%
\pgfpathcurveto{\pgfqpoint{3.211308in}{2.273138in}}{\pgfqpoint{3.208035in}{2.265238in}}{\pgfqpoint{3.208035in}{2.257001in}}%
\pgfpathcurveto{\pgfqpoint{3.208035in}{2.248765in}}{\pgfqpoint{3.211308in}{2.240865in}}{\pgfqpoint{3.217132in}{2.235041in}}%
\pgfpathcurveto{\pgfqpoint{3.222955in}{2.229217in}}{\pgfqpoint{3.230856in}{2.225945in}}{\pgfqpoint{3.239092in}{2.225945in}}%
\pgfpathclose%
\pgfusepath{stroke,fill}%
\end{pgfscope}%
\begin{pgfscope}%
\pgfpathrectangle{\pgfqpoint{0.100000in}{0.212622in}}{\pgfqpoint{3.696000in}{3.696000in}}%
\pgfusepath{clip}%
\pgfsetbuttcap%
\pgfsetroundjoin%
\definecolor{currentfill}{rgb}{0.121569,0.466667,0.705882}%
\pgfsetfillcolor{currentfill}%
\pgfsetfillopacity{0.646891}%
\pgfsetlinewidth{1.003750pt}%
\definecolor{currentstroke}{rgb}{0.121569,0.466667,0.705882}%
\pgfsetstrokecolor{currentstroke}%
\pgfsetstrokeopacity{0.646891}%
\pgfsetdash{}{0pt}%
\pgfpathmoveto{\pgfqpoint{3.238746in}{2.225837in}}%
\pgfpathcurveto{\pgfqpoint{3.246982in}{2.225837in}}{\pgfqpoint{3.254882in}{2.229109in}}{\pgfqpoint{3.260706in}{2.234933in}}%
\pgfpathcurveto{\pgfqpoint{3.266530in}{2.240757in}}{\pgfqpoint{3.269802in}{2.248657in}}{\pgfqpoint{3.269802in}{2.256893in}}%
\pgfpathcurveto{\pgfqpoint{3.269802in}{2.265130in}}{\pgfqpoint{3.266530in}{2.273030in}}{\pgfqpoint{3.260706in}{2.278854in}}%
\pgfpathcurveto{\pgfqpoint{3.254882in}{2.284678in}}{\pgfqpoint{3.246982in}{2.287950in}}{\pgfqpoint{3.238746in}{2.287950in}}%
\pgfpathcurveto{\pgfqpoint{3.230510in}{2.287950in}}{\pgfqpoint{3.222610in}{2.284678in}}{\pgfqpoint{3.216786in}{2.278854in}}%
\pgfpathcurveto{\pgfqpoint{3.210962in}{2.273030in}}{\pgfqpoint{3.207689in}{2.265130in}}{\pgfqpoint{3.207689in}{2.256893in}}%
\pgfpathcurveto{\pgfqpoint{3.207689in}{2.248657in}}{\pgfqpoint{3.210962in}{2.240757in}}{\pgfqpoint{3.216786in}{2.234933in}}%
\pgfpathcurveto{\pgfqpoint{3.222610in}{2.229109in}}{\pgfqpoint{3.230510in}{2.225837in}}{\pgfqpoint{3.238746in}{2.225837in}}%
\pgfpathclose%
\pgfusepath{stroke,fill}%
\end{pgfscope}%
\begin{pgfscope}%
\pgfpathrectangle{\pgfqpoint{0.100000in}{0.212622in}}{\pgfqpoint{3.696000in}{3.696000in}}%
\pgfusepath{clip}%
\pgfsetbuttcap%
\pgfsetroundjoin%
\definecolor{currentfill}{rgb}{0.121569,0.466667,0.705882}%
\pgfsetfillcolor{currentfill}%
\pgfsetfillopacity{0.646974}%
\pgfsetlinewidth{1.003750pt}%
\definecolor{currentstroke}{rgb}{0.121569,0.466667,0.705882}%
\pgfsetstrokecolor{currentstroke}%
\pgfsetstrokeopacity{0.646974}%
\pgfsetdash{}{0pt}%
\pgfpathmoveto{\pgfqpoint{3.238584in}{2.225776in}}%
\pgfpathcurveto{\pgfqpoint{3.246820in}{2.225776in}}{\pgfqpoint{3.254720in}{2.229048in}}{\pgfqpoint{3.260544in}{2.234872in}}%
\pgfpathcurveto{\pgfqpoint{3.266368in}{2.240696in}}{\pgfqpoint{3.269640in}{2.248596in}}{\pgfqpoint{3.269640in}{2.256832in}}%
\pgfpathcurveto{\pgfqpoint{3.269640in}{2.265069in}}{\pgfqpoint{3.266368in}{2.272969in}}{\pgfqpoint{3.260544in}{2.278793in}}%
\pgfpathcurveto{\pgfqpoint{3.254720in}{2.284617in}}{\pgfqpoint{3.246820in}{2.287889in}}{\pgfqpoint{3.238584in}{2.287889in}}%
\pgfpathcurveto{\pgfqpoint{3.230347in}{2.287889in}}{\pgfqpoint{3.222447in}{2.284617in}}{\pgfqpoint{3.216623in}{2.278793in}}%
\pgfpathcurveto{\pgfqpoint{3.210799in}{2.272969in}}{\pgfqpoint{3.207527in}{2.265069in}}{\pgfqpoint{3.207527in}{2.256832in}}%
\pgfpathcurveto{\pgfqpoint{3.207527in}{2.248596in}}{\pgfqpoint{3.210799in}{2.240696in}}{\pgfqpoint{3.216623in}{2.234872in}}%
\pgfpathcurveto{\pgfqpoint{3.222447in}{2.229048in}}{\pgfqpoint{3.230347in}{2.225776in}}{\pgfqpoint{3.238584in}{2.225776in}}%
\pgfpathclose%
\pgfusepath{stroke,fill}%
\end{pgfscope}%
\begin{pgfscope}%
\pgfpathrectangle{\pgfqpoint{0.100000in}{0.212622in}}{\pgfqpoint{3.696000in}{3.696000in}}%
\pgfusepath{clip}%
\pgfsetbuttcap%
\pgfsetroundjoin%
\definecolor{currentfill}{rgb}{0.121569,0.466667,0.705882}%
\pgfsetfillcolor{currentfill}%
\pgfsetfillopacity{0.647516}%
\pgfsetlinewidth{1.003750pt}%
\definecolor{currentstroke}{rgb}{0.121569,0.466667,0.705882}%
\pgfsetstrokecolor{currentstroke}%
\pgfsetstrokeopacity{0.647516}%
\pgfsetdash{}{0pt}%
\pgfpathmoveto{\pgfqpoint{3.236871in}{2.225631in}}%
\pgfpathcurveto{\pgfqpoint{3.245107in}{2.225631in}}{\pgfqpoint{3.253007in}{2.228904in}}{\pgfqpoint{3.258831in}{2.234728in}}%
\pgfpathcurveto{\pgfqpoint{3.264655in}{2.240552in}}{\pgfqpoint{3.267927in}{2.248452in}}{\pgfqpoint{3.267927in}{2.256688in}}%
\pgfpathcurveto{\pgfqpoint{3.267927in}{2.264924in}}{\pgfqpoint{3.264655in}{2.272824in}}{\pgfqpoint{3.258831in}{2.278648in}}%
\pgfpathcurveto{\pgfqpoint{3.253007in}{2.284472in}}{\pgfqpoint{3.245107in}{2.287744in}}{\pgfqpoint{3.236871in}{2.287744in}}%
\pgfpathcurveto{\pgfqpoint{3.228634in}{2.287744in}}{\pgfqpoint{3.220734in}{2.284472in}}{\pgfqpoint{3.214910in}{2.278648in}}%
\pgfpathcurveto{\pgfqpoint{3.209087in}{2.272824in}}{\pgfqpoint{3.205814in}{2.264924in}}{\pgfqpoint{3.205814in}{2.256688in}}%
\pgfpathcurveto{\pgfqpoint{3.205814in}{2.248452in}}{\pgfqpoint{3.209087in}{2.240552in}}{\pgfqpoint{3.214910in}{2.234728in}}%
\pgfpathcurveto{\pgfqpoint{3.220734in}{2.228904in}}{\pgfqpoint{3.228634in}{2.225631in}}{\pgfqpoint{3.236871in}{2.225631in}}%
\pgfpathclose%
\pgfusepath{stroke,fill}%
\end{pgfscope}%
\begin{pgfscope}%
\pgfpathrectangle{\pgfqpoint{0.100000in}{0.212622in}}{\pgfqpoint{3.696000in}{3.696000in}}%
\pgfusepath{clip}%
\pgfsetbuttcap%
\pgfsetroundjoin%
\definecolor{currentfill}{rgb}{0.121569,0.466667,0.705882}%
\pgfsetfillcolor{currentfill}%
\pgfsetfillopacity{0.648170}%
\pgfsetlinewidth{1.003750pt}%
\definecolor{currentstroke}{rgb}{0.121569,0.466667,0.705882}%
\pgfsetstrokecolor{currentstroke}%
\pgfsetstrokeopacity{0.648170}%
\pgfsetdash{}{0pt}%
\pgfpathmoveto{\pgfqpoint{0.796032in}{1.230456in}}%
\pgfpathcurveto{\pgfqpoint{0.804268in}{1.230456in}}{\pgfqpoint{0.812168in}{1.233728in}}{\pgfqpoint{0.817992in}{1.239552in}}%
\pgfpathcurveto{\pgfqpoint{0.823816in}{1.245376in}}{\pgfqpoint{0.827088in}{1.253276in}}{\pgfqpoint{0.827088in}{1.261512in}}%
\pgfpathcurveto{\pgfqpoint{0.827088in}{1.269748in}}{\pgfqpoint{0.823816in}{1.277648in}}{\pgfqpoint{0.817992in}{1.283472in}}%
\pgfpathcurveto{\pgfqpoint{0.812168in}{1.289296in}}{\pgfqpoint{0.804268in}{1.292569in}}{\pgfqpoint{0.796032in}{1.292569in}}%
\pgfpathcurveto{\pgfqpoint{0.787796in}{1.292569in}}{\pgfqpoint{0.779896in}{1.289296in}}{\pgfqpoint{0.774072in}{1.283472in}}%
\pgfpathcurveto{\pgfqpoint{0.768248in}{1.277648in}}{\pgfqpoint{0.764975in}{1.269748in}}{\pgfqpoint{0.764975in}{1.261512in}}%
\pgfpathcurveto{\pgfqpoint{0.764975in}{1.253276in}}{\pgfqpoint{0.768248in}{1.245376in}}{\pgfqpoint{0.774072in}{1.239552in}}%
\pgfpathcurveto{\pgfqpoint{0.779896in}{1.233728in}}{\pgfqpoint{0.787796in}{1.230456in}}{\pgfqpoint{0.796032in}{1.230456in}}%
\pgfpathclose%
\pgfusepath{stroke,fill}%
\end{pgfscope}%
\begin{pgfscope}%
\pgfpathrectangle{\pgfqpoint{0.100000in}{0.212622in}}{\pgfqpoint{3.696000in}{3.696000in}}%
\pgfusepath{clip}%
\pgfsetbuttcap%
\pgfsetroundjoin%
\definecolor{currentfill}{rgb}{0.121569,0.466667,0.705882}%
\pgfsetfillcolor{currentfill}%
\pgfsetfillopacity{0.649123}%
\pgfsetlinewidth{1.003750pt}%
\definecolor{currentstroke}{rgb}{0.121569,0.466667,0.705882}%
\pgfsetstrokecolor{currentstroke}%
\pgfsetstrokeopacity{0.649123}%
\pgfsetdash{}{0pt}%
\pgfpathmoveto{\pgfqpoint{3.233596in}{2.221780in}}%
\pgfpathcurveto{\pgfqpoint{3.241832in}{2.221780in}}{\pgfqpoint{3.249732in}{2.225052in}}{\pgfqpoint{3.255556in}{2.230876in}}%
\pgfpathcurveto{\pgfqpoint{3.261380in}{2.236700in}}{\pgfqpoint{3.264653in}{2.244600in}}{\pgfqpoint{3.264653in}{2.252836in}}%
\pgfpathcurveto{\pgfqpoint{3.264653in}{2.261073in}}{\pgfqpoint{3.261380in}{2.268973in}}{\pgfqpoint{3.255556in}{2.274797in}}%
\pgfpathcurveto{\pgfqpoint{3.249732in}{2.280621in}}{\pgfqpoint{3.241832in}{2.283893in}}{\pgfqpoint{3.233596in}{2.283893in}}%
\pgfpathcurveto{\pgfqpoint{3.225360in}{2.283893in}}{\pgfqpoint{3.217460in}{2.280621in}}{\pgfqpoint{3.211636in}{2.274797in}}%
\pgfpathcurveto{\pgfqpoint{3.205812in}{2.268973in}}{\pgfqpoint{3.202540in}{2.261073in}}{\pgfqpoint{3.202540in}{2.252836in}}%
\pgfpathcurveto{\pgfqpoint{3.202540in}{2.244600in}}{\pgfqpoint{3.205812in}{2.236700in}}{\pgfqpoint{3.211636in}{2.230876in}}%
\pgfpathcurveto{\pgfqpoint{3.217460in}{2.225052in}}{\pgfqpoint{3.225360in}{2.221780in}}{\pgfqpoint{3.233596in}{2.221780in}}%
\pgfpathclose%
\pgfusepath{stroke,fill}%
\end{pgfscope}%
\begin{pgfscope}%
\pgfpathrectangle{\pgfqpoint{0.100000in}{0.212622in}}{\pgfqpoint{3.696000in}{3.696000in}}%
\pgfusepath{clip}%
\pgfsetbuttcap%
\pgfsetroundjoin%
\definecolor{currentfill}{rgb}{0.121569,0.466667,0.705882}%
\pgfsetfillcolor{currentfill}%
\pgfsetfillopacity{0.652203}%
\pgfsetlinewidth{1.003750pt}%
\definecolor{currentstroke}{rgb}{0.121569,0.466667,0.705882}%
\pgfsetstrokecolor{currentstroke}%
\pgfsetstrokeopacity{0.652203}%
\pgfsetdash{}{0pt}%
\pgfpathmoveto{\pgfqpoint{3.225840in}{2.221464in}}%
\pgfpathcurveto{\pgfqpoint{3.234076in}{2.221464in}}{\pgfqpoint{3.241976in}{2.224736in}}{\pgfqpoint{3.247800in}{2.230560in}}%
\pgfpathcurveto{\pgfqpoint{3.253624in}{2.236384in}}{\pgfqpoint{3.256897in}{2.244284in}}{\pgfqpoint{3.256897in}{2.252520in}}%
\pgfpathcurveto{\pgfqpoint{3.256897in}{2.260757in}}{\pgfqpoint{3.253624in}{2.268657in}}{\pgfqpoint{3.247800in}{2.274481in}}%
\pgfpathcurveto{\pgfqpoint{3.241976in}{2.280305in}}{\pgfqpoint{3.234076in}{2.283577in}}{\pgfqpoint{3.225840in}{2.283577in}}%
\pgfpathcurveto{\pgfqpoint{3.217604in}{2.283577in}}{\pgfqpoint{3.209704in}{2.280305in}}{\pgfqpoint{3.203880in}{2.274481in}}%
\pgfpathcurveto{\pgfqpoint{3.198056in}{2.268657in}}{\pgfqpoint{3.194784in}{2.260757in}}{\pgfqpoint{3.194784in}{2.252520in}}%
\pgfpathcurveto{\pgfqpoint{3.194784in}{2.244284in}}{\pgfqpoint{3.198056in}{2.236384in}}{\pgfqpoint{3.203880in}{2.230560in}}%
\pgfpathcurveto{\pgfqpoint{3.209704in}{2.224736in}}{\pgfqpoint{3.217604in}{2.221464in}}{\pgfqpoint{3.225840in}{2.221464in}}%
\pgfpathclose%
\pgfusepath{stroke,fill}%
\end{pgfscope}%
\begin{pgfscope}%
\pgfpathrectangle{\pgfqpoint{0.100000in}{0.212622in}}{\pgfqpoint{3.696000in}{3.696000in}}%
\pgfusepath{clip}%
\pgfsetbuttcap%
\pgfsetroundjoin%
\definecolor{currentfill}{rgb}{0.121569,0.466667,0.705882}%
\pgfsetfillcolor{currentfill}%
\pgfsetfillopacity{0.652941}%
\pgfsetlinewidth{1.003750pt}%
\definecolor{currentstroke}{rgb}{0.121569,0.466667,0.705882}%
\pgfsetstrokecolor{currentstroke}%
\pgfsetstrokeopacity{0.652941}%
\pgfsetdash{}{0pt}%
\pgfpathmoveto{\pgfqpoint{0.751127in}{1.244467in}}%
\pgfpathcurveto{\pgfqpoint{0.759364in}{1.244467in}}{\pgfqpoint{0.767264in}{1.247740in}}{\pgfqpoint{0.773088in}{1.253563in}}%
\pgfpathcurveto{\pgfqpoint{0.778912in}{1.259387in}}{\pgfqpoint{0.782184in}{1.267287in}}{\pgfqpoint{0.782184in}{1.275524in}}%
\pgfpathcurveto{\pgfqpoint{0.782184in}{1.283760in}}{\pgfqpoint{0.778912in}{1.291660in}}{\pgfqpoint{0.773088in}{1.297484in}}%
\pgfpathcurveto{\pgfqpoint{0.767264in}{1.303308in}}{\pgfqpoint{0.759364in}{1.306580in}}{\pgfqpoint{0.751127in}{1.306580in}}%
\pgfpathcurveto{\pgfqpoint{0.742891in}{1.306580in}}{\pgfqpoint{0.734991in}{1.303308in}}{\pgfqpoint{0.729167in}{1.297484in}}%
\pgfpathcurveto{\pgfqpoint{0.723343in}{1.291660in}}{\pgfqpoint{0.720071in}{1.283760in}}{\pgfqpoint{0.720071in}{1.275524in}}%
\pgfpathcurveto{\pgfqpoint{0.720071in}{1.267287in}}{\pgfqpoint{0.723343in}{1.259387in}}{\pgfqpoint{0.729167in}{1.253563in}}%
\pgfpathcurveto{\pgfqpoint{0.734991in}{1.247740in}}{\pgfqpoint{0.742891in}{1.244467in}}{\pgfqpoint{0.751127in}{1.244467in}}%
\pgfpathclose%
\pgfusepath{stroke,fill}%
\end{pgfscope}%
\begin{pgfscope}%
\pgfpathrectangle{\pgfqpoint{0.100000in}{0.212622in}}{\pgfqpoint{3.696000in}{3.696000in}}%
\pgfusepath{clip}%
\pgfsetbuttcap%
\pgfsetroundjoin%
\definecolor{currentfill}{rgb}{0.121569,0.466667,0.705882}%
\pgfsetfillcolor{currentfill}%
\pgfsetfillopacity{0.655195}%
\pgfsetlinewidth{1.003750pt}%
\definecolor{currentstroke}{rgb}{0.121569,0.466667,0.705882}%
\pgfsetstrokecolor{currentstroke}%
\pgfsetstrokeopacity{0.655195}%
\pgfsetdash{}{0pt}%
\pgfpathmoveto{\pgfqpoint{0.764760in}{1.222747in}}%
\pgfpathcurveto{\pgfqpoint{0.772996in}{1.222747in}}{\pgfqpoint{0.780896in}{1.226020in}}{\pgfqpoint{0.786720in}{1.231844in}}%
\pgfpathcurveto{\pgfqpoint{0.792544in}{1.237668in}}{\pgfqpoint{0.795816in}{1.245568in}}{\pgfqpoint{0.795816in}{1.253804in}}%
\pgfpathcurveto{\pgfqpoint{0.795816in}{1.262040in}}{\pgfqpoint{0.792544in}{1.269940in}}{\pgfqpoint{0.786720in}{1.275764in}}%
\pgfpathcurveto{\pgfqpoint{0.780896in}{1.281588in}}{\pgfqpoint{0.772996in}{1.284860in}}{\pgfqpoint{0.764760in}{1.284860in}}%
\pgfpathcurveto{\pgfqpoint{0.756524in}{1.284860in}}{\pgfqpoint{0.748624in}{1.281588in}}{\pgfqpoint{0.742800in}{1.275764in}}%
\pgfpathcurveto{\pgfqpoint{0.736976in}{1.269940in}}{\pgfqpoint{0.733703in}{1.262040in}}{\pgfqpoint{0.733703in}{1.253804in}}%
\pgfpathcurveto{\pgfqpoint{0.733703in}{1.245568in}}{\pgfqpoint{0.736976in}{1.237668in}}{\pgfqpoint{0.742800in}{1.231844in}}%
\pgfpathcurveto{\pgfqpoint{0.748624in}{1.226020in}}{\pgfqpoint{0.756524in}{1.222747in}}{\pgfqpoint{0.764760in}{1.222747in}}%
\pgfpathclose%
\pgfusepath{stroke,fill}%
\end{pgfscope}%
\begin{pgfscope}%
\pgfpathrectangle{\pgfqpoint{0.100000in}{0.212622in}}{\pgfqpoint{3.696000in}{3.696000in}}%
\pgfusepath{clip}%
\pgfsetbuttcap%
\pgfsetroundjoin%
\definecolor{currentfill}{rgb}{0.121569,0.466667,0.705882}%
\pgfsetfillcolor{currentfill}%
\pgfsetfillopacity{0.656311}%
\pgfsetlinewidth{1.003750pt}%
\definecolor{currentstroke}{rgb}{0.121569,0.466667,0.705882}%
\pgfsetstrokecolor{currentstroke}%
\pgfsetstrokeopacity{0.656311}%
\pgfsetdash{}{0pt}%
\pgfpathmoveto{\pgfqpoint{3.216928in}{2.215728in}}%
\pgfpathcurveto{\pgfqpoint{3.225165in}{2.215728in}}{\pgfqpoint{3.233065in}{2.219000in}}{\pgfqpoint{3.238888in}{2.224824in}}%
\pgfpathcurveto{\pgfqpoint{3.244712in}{2.230648in}}{\pgfqpoint{3.247985in}{2.238548in}}{\pgfqpoint{3.247985in}{2.246784in}}%
\pgfpathcurveto{\pgfqpoint{3.247985in}{2.255020in}}{\pgfqpoint{3.244712in}{2.262920in}}{\pgfqpoint{3.238888in}{2.268744in}}%
\pgfpathcurveto{\pgfqpoint{3.233065in}{2.274568in}}{\pgfqpoint{3.225165in}{2.277841in}}{\pgfqpoint{3.216928in}{2.277841in}}%
\pgfpathcurveto{\pgfqpoint{3.208692in}{2.277841in}}{\pgfqpoint{3.200792in}{2.274568in}}{\pgfqpoint{3.194968in}{2.268744in}}%
\pgfpathcurveto{\pgfqpoint{3.189144in}{2.262920in}}{\pgfqpoint{3.185872in}{2.255020in}}{\pgfqpoint{3.185872in}{2.246784in}}%
\pgfpathcurveto{\pgfqpoint{3.185872in}{2.238548in}}{\pgfqpoint{3.189144in}{2.230648in}}{\pgfqpoint{3.194968in}{2.224824in}}%
\pgfpathcurveto{\pgfqpoint{3.200792in}{2.219000in}}{\pgfqpoint{3.208692in}{2.215728in}}{\pgfqpoint{3.216928in}{2.215728in}}%
\pgfpathclose%
\pgfusepath{stroke,fill}%
\end{pgfscope}%
\begin{pgfscope}%
\pgfpathrectangle{\pgfqpoint{0.100000in}{0.212622in}}{\pgfqpoint{3.696000in}{3.696000in}}%
\pgfusepath{clip}%
\pgfsetbuttcap%
\pgfsetroundjoin%
\definecolor{currentfill}{rgb}{0.121569,0.466667,0.705882}%
\pgfsetfillcolor{currentfill}%
\pgfsetfillopacity{0.656465}%
\pgfsetlinewidth{1.003750pt}%
\definecolor{currentstroke}{rgb}{0.121569,0.466667,0.705882}%
\pgfsetstrokecolor{currentstroke}%
\pgfsetstrokeopacity{0.656465}%
\pgfsetdash{}{0pt}%
\pgfpathmoveto{\pgfqpoint{0.737769in}{1.236475in}}%
\pgfpathcurveto{\pgfqpoint{0.746006in}{1.236475in}}{\pgfqpoint{0.753906in}{1.239747in}}{\pgfqpoint{0.759730in}{1.245571in}}%
\pgfpathcurveto{\pgfqpoint{0.765554in}{1.251395in}}{\pgfqpoint{0.768826in}{1.259295in}}{\pgfqpoint{0.768826in}{1.267531in}}%
\pgfpathcurveto{\pgfqpoint{0.768826in}{1.275767in}}{\pgfqpoint{0.765554in}{1.283667in}}{\pgfqpoint{0.759730in}{1.289491in}}%
\pgfpathcurveto{\pgfqpoint{0.753906in}{1.295315in}}{\pgfqpoint{0.746006in}{1.298588in}}{\pgfqpoint{0.737769in}{1.298588in}}%
\pgfpathcurveto{\pgfqpoint{0.729533in}{1.298588in}}{\pgfqpoint{0.721633in}{1.295315in}}{\pgfqpoint{0.715809in}{1.289491in}}%
\pgfpathcurveto{\pgfqpoint{0.709985in}{1.283667in}}{\pgfqpoint{0.706713in}{1.275767in}}{\pgfqpoint{0.706713in}{1.267531in}}%
\pgfpathcurveto{\pgfqpoint{0.706713in}{1.259295in}}{\pgfqpoint{0.709985in}{1.251395in}}{\pgfqpoint{0.715809in}{1.245571in}}%
\pgfpathcurveto{\pgfqpoint{0.721633in}{1.239747in}}{\pgfqpoint{0.729533in}{1.236475in}}{\pgfqpoint{0.737769in}{1.236475in}}%
\pgfpathclose%
\pgfusepath{stroke,fill}%
\end{pgfscope}%
\begin{pgfscope}%
\pgfpathrectangle{\pgfqpoint{0.100000in}{0.212622in}}{\pgfqpoint{3.696000in}{3.696000in}}%
\pgfusepath{clip}%
\pgfsetbuttcap%
\pgfsetroundjoin%
\definecolor{currentfill}{rgb}{0.121569,0.466667,0.705882}%
\pgfsetfillcolor{currentfill}%
\pgfsetfillopacity{0.657940}%
\pgfsetlinewidth{1.003750pt}%
\definecolor{currentstroke}{rgb}{0.121569,0.466667,0.705882}%
\pgfsetstrokecolor{currentstroke}%
\pgfsetstrokeopacity{0.657940}%
\pgfsetdash{}{0pt}%
\pgfpathmoveto{\pgfqpoint{0.733659in}{1.230853in}}%
\pgfpathcurveto{\pgfqpoint{0.741895in}{1.230853in}}{\pgfqpoint{0.749795in}{1.234125in}}{\pgfqpoint{0.755619in}{1.239949in}}%
\pgfpathcurveto{\pgfqpoint{0.761443in}{1.245773in}}{\pgfqpoint{0.764715in}{1.253673in}}{\pgfqpoint{0.764715in}{1.261909in}}%
\pgfpathcurveto{\pgfqpoint{0.764715in}{1.270146in}}{\pgfqpoint{0.761443in}{1.278046in}}{\pgfqpoint{0.755619in}{1.283870in}}%
\pgfpathcurveto{\pgfqpoint{0.749795in}{1.289694in}}{\pgfqpoint{0.741895in}{1.292966in}}{\pgfqpoint{0.733659in}{1.292966in}}%
\pgfpathcurveto{\pgfqpoint{0.725422in}{1.292966in}}{\pgfqpoint{0.717522in}{1.289694in}}{\pgfqpoint{0.711698in}{1.283870in}}%
\pgfpathcurveto{\pgfqpoint{0.705875in}{1.278046in}}{\pgfqpoint{0.702602in}{1.270146in}}{\pgfqpoint{0.702602in}{1.261909in}}%
\pgfpathcurveto{\pgfqpoint{0.702602in}{1.253673in}}{\pgfqpoint{0.705875in}{1.245773in}}{\pgfqpoint{0.711698in}{1.239949in}}%
\pgfpathcurveto{\pgfqpoint{0.717522in}{1.234125in}}{\pgfqpoint{0.725422in}{1.230853in}}{\pgfqpoint{0.733659in}{1.230853in}}%
\pgfpathclose%
\pgfusepath{stroke,fill}%
\end{pgfscope}%
\begin{pgfscope}%
\pgfpathrectangle{\pgfqpoint{0.100000in}{0.212622in}}{\pgfqpoint{3.696000in}{3.696000in}}%
\pgfusepath{clip}%
\pgfsetbuttcap%
\pgfsetroundjoin%
\definecolor{currentfill}{rgb}{0.121569,0.466667,0.705882}%
\pgfsetfillcolor{currentfill}%
\pgfsetfillopacity{0.658856}%
\pgfsetlinewidth{1.003750pt}%
\definecolor{currentstroke}{rgb}{0.121569,0.466667,0.705882}%
\pgfsetstrokecolor{currentstroke}%
\pgfsetstrokeopacity{0.658856}%
\pgfsetdash{}{0pt}%
\pgfpathmoveto{\pgfqpoint{0.731305in}{1.228405in}}%
\pgfpathcurveto{\pgfqpoint{0.739541in}{1.228405in}}{\pgfqpoint{0.747441in}{1.231677in}}{\pgfqpoint{0.753265in}{1.237501in}}%
\pgfpathcurveto{\pgfqpoint{0.759089in}{1.243325in}}{\pgfqpoint{0.762361in}{1.251225in}}{\pgfqpoint{0.762361in}{1.259461in}}%
\pgfpathcurveto{\pgfqpoint{0.762361in}{1.267697in}}{\pgfqpoint{0.759089in}{1.275598in}}{\pgfqpoint{0.753265in}{1.281421in}}%
\pgfpathcurveto{\pgfqpoint{0.747441in}{1.287245in}}{\pgfqpoint{0.739541in}{1.290518in}}{\pgfqpoint{0.731305in}{1.290518in}}%
\pgfpathcurveto{\pgfqpoint{0.723068in}{1.290518in}}{\pgfqpoint{0.715168in}{1.287245in}}{\pgfqpoint{0.709344in}{1.281421in}}%
\pgfpathcurveto{\pgfqpoint{0.703520in}{1.275598in}}{\pgfqpoint{0.700248in}{1.267697in}}{\pgfqpoint{0.700248in}{1.259461in}}%
\pgfpathcurveto{\pgfqpoint{0.700248in}{1.251225in}}{\pgfqpoint{0.703520in}{1.243325in}}{\pgfqpoint{0.709344in}{1.237501in}}%
\pgfpathcurveto{\pgfqpoint{0.715168in}{1.231677in}}{\pgfqpoint{0.723068in}{1.228405in}}{\pgfqpoint{0.731305in}{1.228405in}}%
\pgfpathclose%
\pgfusepath{stroke,fill}%
\end{pgfscope}%
\begin{pgfscope}%
\pgfpathrectangle{\pgfqpoint{0.100000in}{0.212622in}}{\pgfqpoint{3.696000in}{3.696000in}}%
\pgfusepath{clip}%
\pgfsetbuttcap%
\pgfsetroundjoin%
\definecolor{currentfill}{rgb}{0.121569,0.466667,0.705882}%
\pgfsetfillcolor{currentfill}%
\pgfsetfillopacity{0.659128}%
\pgfsetlinewidth{1.003750pt}%
\definecolor{currentstroke}{rgb}{0.121569,0.466667,0.705882}%
\pgfsetstrokecolor{currentstroke}%
\pgfsetstrokeopacity{0.659128}%
\pgfsetdash{}{0pt}%
\pgfpathmoveto{\pgfqpoint{0.747797in}{1.218196in}}%
\pgfpathcurveto{\pgfqpoint{0.756034in}{1.218196in}}{\pgfqpoint{0.763934in}{1.221468in}}{\pgfqpoint{0.769758in}{1.227292in}}%
\pgfpathcurveto{\pgfqpoint{0.775582in}{1.233116in}}{\pgfqpoint{0.778854in}{1.241016in}}{\pgfqpoint{0.778854in}{1.249252in}}%
\pgfpathcurveto{\pgfqpoint{0.778854in}{1.257488in}}{\pgfqpoint{0.775582in}{1.265389in}}{\pgfqpoint{0.769758in}{1.271212in}}%
\pgfpathcurveto{\pgfqpoint{0.763934in}{1.277036in}}{\pgfqpoint{0.756034in}{1.280309in}}{\pgfqpoint{0.747797in}{1.280309in}}%
\pgfpathcurveto{\pgfqpoint{0.739561in}{1.280309in}}{\pgfqpoint{0.731661in}{1.277036in}}{\pgfqpoint{0.725837in}{1.271212in}}%
\pgfpathcurveto{\pgfqpoint{0.720013in}{1.265389in}}{\pgfqpoint{0.716741in}{1.257488in}}{\pgfqpoint{0.716741in}{1.249252in}}%
\pgfpathcurveto{\pgfqpoint{0.716741in}{1.241016in}}{\pgfqpoint{0.720013in}{1.233116in}}{\pgfqpoint{0.725837in}{1.227292in}}%
\pgfpathcurveto{\pgfqpoint{0.731661in}{1.221468in}}{\pgfqpoint{0.739561in}{1.218196in}}{\pgfqpoint{0.747797in}{1.218196in}}%
\pgfpathclose%
\pgfusepath{stroke,fill}%
\end{pgfscope}%
\begin{pgfscope}%
\pgfpathrectangle{\pgfqpoint{0.100000in}{0.212622in}}{\pgfqpoint{3.696000in}{3.696000in}}%
\pgfusepath{clip}%
\pgfsetbuttcap%
\pgfsetroundjoin%
\definecolor{currentfill}{rgb}{0.121569,0.466667,0.705882}%
\pgfsetfillcolor{currentfill}%
\pgfsetfillopacity{0.660375}%
\pgfsetlinewidth{1.003750pt}%
\definecolor{currentstroke}{rgb}{0.121569,0.466667,0.705882}%
\pgfsetstrokecolor{currentstroke}%
\pgfsetstrokeopacity{0.660375}%
\pgfsetdash{}{0pt}%
\pgfpathmoveto{\pgfqpoint{0.725810in}{1.224566in}}%
\pgfpathcurveto{\pgfqpoint{0.734046in}{1.224566in}}{\pgfqpoint{0.741946in}{1.227839in}}{\pgfqpoint{0.747770in}{1.233663in}}%
\pgfpathcurveto{\pgfqpoint{0.753594in}{1.239486in}}{\pgfqpoint{0.756866in}{1.247387in}}{\pgfqpoint{0.756866in}{1.255623in}}%
\pgfpathcurveto{\pgfqpoint{0.756866in}{1.263859in}}{\pgfqpoint{0.753594in}{1.271759in}}{\pgfqpoint{0.747770in}{1.277583in}}%
\pgfpathcurveto{\pgfqpoint{0.741946in}{1.283407in}}{\pgfqpoint{0.734046in}{1.286679in}}{\pgfqpoint{0.725810in}{1.286679in}}%
\pgfpathcurveto{\pgfqpoint{0.717574in}{1.286679in}}{\pgfqpoint{0.709673in}{1.283407in}}{\pgfqpoint{0.703850in}{1.277583in}}%
\pgfpathcurveto{\pgfqpoint{0.698026in}{1.271759in}}{\pgfqpoint{0.694753in}{1.263859in}}{\pgfqpoint{0.694753in}{1.255623in}}%
\pgfpathcurveto{\pgfqpoint{0.694753in}{1.247387in}}{\pgfqpoint{0.698026in}{1.239486in}}{\pgfqpoint{0.703850in}{1.233663in}}%
\pgfpathcurveto{\pgfqpoint{0.709673in}{1.227839in}}{\pgfqpoint{0.717574in}{1.224566in}}{\pgfqpoint{0.725810in}{1.224566in}}%
\pgfpathclose%
\pgfusepath{stroke,fill}%
\end{pgfscope}%
\begin{pgfscope}%
\pgfpathrectangle{\pgfqpoint{0.100000in}{0.212622in}}{\pgfqpoint{3.696000in}{3.696000in}}%
\pgfusepath{clip}%
\pgfsetbuttcap%
\pgfsetroundjoin%
\definecolor{currentfill}{rgb}{0.121569,0.466667,0.705882}%
\pgfsetfillcolor{currentfill}%
\pgfsetfillopacity{0.660379}%
\pgfsetlinewidth{1.003750pt}%
\definecolor{currentstroke}{rgb}{0.121569,0.466667,0.705882}%
\pgfsetstrokecolor{currentstroke}%
\pgfsetstrokeopacity{0.660379}%
\pgfsetdash{}{0pt}%
\pgfpathmoveto{\pgfqpoint{0.725803in}{1.224556in}}%
\pgfpathcurveto{\pgfqpoint{0.734039in}{1.224556in}}{\pgfqpoint{0.741939in}{1.227829in}}{\pgfqpoint{0.747763in}{1.233653in}}%
\pgfpathcurveto{\pgfqpoint{0.753587in}{1.239477in}}{\pgfqpoint{0.756859in}{1.247377in}}{\pgfqpoint{0.756859in}{1.255613in}}%
\pgfpathcurveto{\pgfqpoint{0.756859in}{1.263849in}}{\pgfqpoint{0.753587in}{1.271749in}}{\pgfqpoint{0.747763in}{1.277573in}}%
\pgfpathcurveto{\pgfqpoint{0.741939in}{1.283397in}}{\pgfqpoint{0.734039in}{1.286669in}}{\pgfqpoint{0.725803in}{1.286669in}}%
\pgfpathcurveto{\pgfqpoint{0.717567in}{1.286669in}}{\pgfqpoint{0.709666in}{1.283397in}}{\pgfqpoint{0.703843in}{1.277573in}}%
\pgfpathcurveto{\pgfqpoint{0.698019in}{1.271749in}}{\pgfqpoint{0.694746in}{1.263849in}}{\pgfqpoint{0.694746in}{1.255613in}}%
\pgfpathcurveto{\pgfqpoint{0.694746in}{1.247377in}}{\pgfqpoint{0.698019in}{1.239477in}}{\pgfqpoint{0.703843in}{1.233653in}}%
\pgfpathcurveto{\pgfqpoint{0.709666in}{1.227829in}}{\pgfqpoint{0.717567in}{1.224556in}}{\pgfqpoint{0.725803in}{1.224556in}}%
\pgfpathclose%
\pgfusepath{stroke,fill}%
\end{pgfscope}%
\begin{pgfscope}%
\pgfpathrectangle{\pgfqpoint{0.100000in}{0.212622in}}{\pgfqpoint{3.696000in}{3.696000in}}%
\pgfusepath{clip}%
\pgfsetbuttcap%
\pgfsetroundjoin%
\definecolor{currentfill}{rgb}{0.121569,0.466667,0.705882}%
\pgfsetfillcolor{currentfill}%
\pgfsetfillopacity{0.660385}%
\pgfsetlinewidth{1.003750pt}%
\definecolor{currentstroke}{rgb}{0.121569,0.466667,0.705882}%
\pgfsetstrokecolor{currentstroke}%
\pgfsetstrokeopacity{0.660385}%
\pgfsetdash{}{0pt}%
\pgfpathmoveto{\pgfqpoint{0.725783in}{1.224541in}}%
\pgfpathcurveto{\pgfqpoint{0.734019in}{1.224541in}}{\pgfqpoint{0.741919in}{1.227813in}}{\pgfqpoint{0.747743in}{1.233637in}}%
\pgfpathcurveto{\pgfqpoint{0.753567in}{1.239461in}}{\pgfqpoint{0.756839in}{1.247361in}}{\pgfqpoint{0.756839in}{1.255597in}}%
\pgfpathcurveto{\pgfqpoint{0.756839in}{1.263833in}}{\pgfqpoint{0.753567in}{1.271733in}}{\pgfqpoint{0.747743in}{1.277557in}}%
\pgfpathcurveto{\pgfqpoint{0.741919in}{1.283381in}}{\pgfqpoint{0.734019in}{1.286654in}}{\pgfqpoint{0.725783in}{1.286654in}}%
\pgfpathcurveto{\pgfqpoint{0.717546in}{1.286654in}}{\pgfqpoint{0.709646in}{1.283381in}}{\pgfqpoint{0.703822in}{1.277557in}}%
\pgfpathcurveto{\pgfqpoint{0.697998in}{1.271733in}}{\pgfqpoint{0.694726in}{1.263833in}}{\pgfqpoint{0.694726in}{1.255597in}}%
\pgfpathcurveto{\pgfqpoint{0.694726in}{1.247361in}}{\pgfqpoint{0.697998in}{1.239461in}}{\pgfqpoint{0.703822in}{1.233637in}}%
\pgfpathcurveto{\pgfqpoint{0.709646in}{1.227813in}}{\pgfqpoint{0.717546in}{1.224541in}}{\pgfqpoint{0.725783in}{1.224541in}}%
\pgfpathclose%
\pgfusepath{stroke,fill}%
\end{pgfscope}%
\begin{pgfscope}%
\pgfpathrectangle{\pgfqpoint{0.100000in}{0.212622in}}{\pgfqpoint{3.696000in}{3.696000in}}%
\pgfusepath{clip}%
\pgfsetbuttcap%
\pgfsetroundjoin%
\definecolor{currentfill}{rgb}{0.121569,0.466667,0.705882}%
\pgfsetfillcolor{currentfill}%
\pgfsetfillopacity{0.660396}%
\pgfsetlinewidth{1.003750pt}%
\definecolor{currentstroke}{rgb}{0.121569,0.466667,0.705882}%
\pgfsetstrokecolor{currentstroke}%
\pgfsetstrokeopacity{0.660396}%
\pgfsetdash{}{0pt}%
\pgfpathmoveto{\pgfqpoint{0.725753in}{1.224502in}}%
\pgfpathcurveto{\pgfqpoint{0.733989in}{1.224502in}}{\pgfqpoint{0.741889in}{1.227774in}}{\pgfqpoint{0.747713in}{1.233598in}}%
\pgfpathcurveto{\pgfqpoint{0.753537in}{1.239422in}}{\pgfqpoint{0.756810in}{1.247322in}}{\pgfqpoint{0.756810in}{1.255558in}}%
\pgfpathcurveto{\pgfqpoint{0.756810in}{1.263795in}}{\pgfqpoint{0.753537in}{1.271695in}}{\pgfqpoint{0.747713in}{1.277519in}}%
\pgfpathcurveto{\pgfqpoint{0.741889in}{1.283342in}}{\pgfqpoint{0.733989in}{1.286615in}}{\pgfqpoint{0.725753in}{1.286615in}}%
\pgfpathcurveto{\pgfqpoint{0.717517in}{1.286615in}}{\pgfqpoint{0.709617in}{1.283342in}}{\pgfqpoint{0.703793in}{1.277519in}}%
\pgfpathcurveto{\pgfqpoint{0.697969in}{1.271695in}}{\pgfqpoint{0.694697in}{1.263795in}}{\pgfqpoint{0.694697in}{1.255558in}}%
\pgfpathcurveto{\pgfqpoint{0.694697in}{1.247322in}}{\pgfqpoint{0.697969in}{1.239422in}}{\pgfqpoint{0.703793in}{1.233598in}}%
\pgfpathcurveto{\pgfqpoint{0.709617in}{1.227774in}}{\pgfqpoint{0.717517in}{1.224502in}}{\pgfqpoint{0.725753in}{1.224502in}}%
\pgfpathclose%
\pgfusepath{stroke,fill}%
\end{pgfscope}%
\begin{pgfscope}%
\pgfpathrectangle{\pgfqpoint{0.100000in}{0.212622in}}{\pgfqpoint{3.696000in}{3.696000in}}%
\pgfusepath{clip}%
\pgfsetbuttcap%
\pgfsetroundjoin%
\definecolor{currentfill}{rgb}{0.121569,0.466667,0.705882}%
\pgfsetfillcolor{currentfill}%
\pgfsetfillopacity{0.660417}%
\pgfsetlinewidth{1.003750pt}%
\definecolor{currentstroke}{rgb}{0.121569,0.466667,0.705882}%
\pgfsetstrokecolor{currentstroke}%
\pgfsetstrokeopacity{0.660417}%
\pgfsetdash{}{0pt}%
\pgfpathmoveto{\pgfqpoint{0.725697in}{1.224440in}}%
\pgfpathcurveto{\pgfqpoint{0.733933in}{1.224440in}}{\pgfqpoint{0.741833in}{1.227713in}}{\pgfqpoint{0.747657in}{1.233536in}}%
\pgfpathcurveto{\pgfqpoint{0.753481in}{1.239360in}}{\pgfqpoint{0.756753in}{1.247260in}}{\pgfqpoint{0.756753in}{1.255497in}}%
\pgfpathcurveto{\pgfqpoint{0.756753in}{1.263733in}}{\pgfqpoint{0.753481in}{1.271633in}}{\pgfqpoint{0.747657in}{1.277457in}}%
\pgfpathcurveto{\pgfqpoint{0.741833in}{1.283281in}}{\pgfqpoint{0.733933in}{1.286553in}}{\pgfqpoint{0.725697in}{1.286553in}}%
\pgfpathcurveto{\pgfqpoint{0.717460in}{1.286553in}}{\pgfqpoint{0.709560in}{1.283281in}}{\pgfqpoint{0.703736in}{1.277457in}}%
\pgfpathcurveto{\pgfqpoint{0.697912in}{1.271633in}}{\pgfqpoint{0.694640in}{1.263733in}}{\pgfqpoint{0.694640in}{1.255497in}}%
\pgfpathcurveto{\pgfqpoint{0.694640in}{1.247260in}}{\pgfqpoint{0.697912in}{1.239360in}}{\pgfqpoint{0.703736in}{1.233536in}}%
\pgfpathcurveto{\pgfqpoint{0.709560in}{1.227713in}}{\pgfqpoint{0.717460in}{1.224440in}}{\pgfqpoint{0.725697in}{1.224440in}}%
\pgfpathclose%
\pgfusepath{stroke,fill}%
\end{pgfscope}%
\begin{pgfscope}%
\pgfpathrectangle{\pgfqpoint{0.100000in}{0.212622in}}{\pgfqpoint{3.696000in}{3.696000in}}%
\pgfusepath{clip}%
\pgfsetbuttcap%
\pgfsetroundjoin%
\definecolor{currentfill}{rgb}{0.121569,0.466667,0.705882}%
\pgfsetfillcolor{currentfill}%
\pgfsetfillopacity{0.660455}%
\pgfsetlinewidth{1.003750pt}%
\definecolor{currentstroke}{rgb}{0.121569,0.466667,0.705882}%
\pgfsetstrokecolor{currentstroke}%
\pgfsetstrokeopacity{0.660455}%
\pgfsetdash{}{0pt}%
\pgfpathmoveto{\pgfqpoint{0.725587in}{1.224333in}}%
\pgfpathcurveto{\pgfqpoint{0.733823in}{1.224333in}}{\pgfqpoint{0.741723in}{1.227606in}}{\pgfqpoint{0.747547in}{1.233430in}}%
\pgfpathcurveto{\pgfqpoint{0.753371in}{1.239254in}}{\pgfqpoint{0.756643in}{1.247154in}}{\pgfqpoint{0.756643in}{1.255390in}}%
\pgfpathcurveto{\pgfqpoint{0.756643in}{1.263626in}}{\pgfqpoint{0.753371in}{1.271526in}}{\pgfqpoint{0.747547in}{1.277350in}}%
\pgfpathcurveto{\pgfqpoint{0.741723in}{1.283174in}}{\pgfqpoint{0.733823in}{1.286446in}}{\pgfqpoint{0.725587in}{1.286446in}}%
\pgfpathcurveto{\pgfqpoint{0.717350in}{1.286446in}}{\pgfqpoint{0.709450in}{1.283174in}}{\pgfqpoint{0.703626in}{1.277350in}}%
\pgfpathcurveto{\pgfqpoint{0.697802in}{1.271526in}}{\pgfqpoint{0.694530in}{1.263626in}}{\pgfqpoint{0.694530in}{1.255390in}}%
\pgfpathcurveto{\pgfqpoint{0.694530in}{1.247154in}}{\pgfqpoint{0.697802in}{1.239254in}}{\pgfqpoint{0.703626in}{1.233430in}}%
\pgfpathcurveto{\pgfqpoint{0.709450in}{1.227606in}}{\pgfqpoint{0.717350in}{1.224333in}}{\pgfqpoint{0.725587in}{1.224333in}}%
\pgfpathclose%
\pgfusepath{stroke,fill}%
\end{pgfscope}%
\begin{pgfscope}%
\pgfpathrectangle{\pgfqpoint{0.100000in}{0.212622in}}{\pgfqpoint{3.696000in}{3.696000in}}%
\pgfusepath{clip}%
\pgfsetbuttcap%
\pgfsetroundjoin%
\definecolor{currentfill}{rgb}{0.121569,0.466667,0.705882}%
\pgfsetfillcolor{currentfill}%
\pgfsetfillopacity{0.660526}%
\pgfsetlinewidth{1.003750pt}%
\definecolor{currentstroke}{rgb}{0.121569,0.466667,0.705882}%
\pgfsetstrokecolor{currentstroke}%
\pgfsetstrokeopacity{0.660526}%
\pgfsetdash{}{0pt}%
\pgfpathmoveto{\pgfqpoint{0.725401in}{1.224136in}}%
\pgfpathcurveto{\pgfqpoint{0.733637in}{1.224136in}}{\pgfqpoint{0.741537in}{1.227408in}}{\pgfqpoint{0.747361in}{1.233232in}}%
\pgfpathcurveto{\pgfqpoint{0.753185in}{1.239056in}}{\pgfqpoint{0.756458in}{1.246956in}}{\pgfqpoint{0.756458in}{1.255192in}}%
\pgfpathcurveto{\pgfqpoint{0.756458in}{1.263429in}}{\pgfqpoint{0.753185in}{1.271329in}}{\pgfqpoint{0.747361in}{1.277153in}}%
\pgfpathcurveto{\pgfqpoint{0.741537in}{1.282976in}}{\pgfqpoint{0.733637in}{1.286249in}}{\pgfqpoint{0.725401in}{1.286249in}}%
\pgfpathcurveto{\pgfqpoint{0.717165in}{1.286249in}}{\pgfqpoint{0.709265in}{1.282976in}}{\pgfqpoint{0.703441in}{1.277153in}}%
\pgfpathcurveto{\pgfqpoint{0.697617in}{1.271329in}}{\pgfqpoint{0.694345in}{1.263429in}}{\pgfqpoint{0.694345in}{1.255192in}}%
\pgfpathcurveto{\pgfqpoint{0.694345in}{1.246956in}}{\pgfqpoint{0.697617in}{1.239056in}}{\pgfqpoint{0.703441in}{1.233232in}}%
\pgfpathcurveto{\pgfqpoint{0.709265in}{1.227408in}}{\pgfqpoint{0.717165in}{1.224136in}}{\pgfqpoint{0.725401in}{1.224136in}}%
\pgfpathclose%
\pgfusepath{stroke,fill}%
\end{pgfscope}%
\begin{pgfscope}%
\pgfpathrectangle{\pgfqpoint{0.100000in}{0.212622in}}{\pgfqpoint{3.696000in}{3.696000in}}%
\pgfusepath{clip}%
\pgfsetbuttcap%
\pgfsetroundjoin%
\definecolor{currentfill}{rgb}{0.121569,0.466667,0.705882}%
\pgfsetfillcolor{currentfill}%
\pgfsetfillopacity{0.660648}%
\pgfsetlinewidth{1.003750pt}%
\definecolor{currentstroke}{rgb}{0.121569,0.466667,0.705882}%
\pgfsetstrokecolor{currentstroke}%
\pgfsetstrokeopacity{0.660648}%
\pgfsetdash{}{0pt}%
\pgfpathmoveto{\pgfqpoint{0.725009in}{1.223799in}}%
\pgfpathcurveto{\pgfqpoint{0.733245in}{1.223799in}}{\pgfqpoint{0.741145in}{1.227071in}}{\pgfqpoint{0.746969in}{1.232895in}}%
\pgfpathcurveto{\pgfqpoint{0.752793in}{1.238719in}}{\pgfqpoint{0.756066in}{1.246619in}}{\pgfqpoint{0.756066in}{1.254856in}}%
\pgfpathcurveto{\pgfqpoint{0.756066in}{1.263092in}}{\pgfqpoint{0.752793in}{1.270992in}}{\pgfqpoint{0.746969in}{1.276816in}}%
\pgfpathcurveto{\pgfqpoint{0.741145in}{1.282640in}}{\pgfqpoint{0.733245in}{1.285912in}}{\pgfqpoint{0.725009in}{1.285912in}}%
\pgfpathcurveto{\pgfqpoint{0.716773in}{1.285912in}}{\pgfqpoint{0.708873in}{1.282640in}}{\pgfqpoint{0.703049in}{1.276816in}}%
\pgfpathcurveto{\pgfqpoint{0.697225in}{1.270992in}}{\pgfqpoint{0.693953in}{1.263092in}}{\pgfqpoint{0.693953in}{1.254856in}}%
\pgfpathcurveto{\pgfqpoint{0.693953in}{1.246619in}}{\pgfqpoint{0.697225in}{1.238719in}}{\pgfqpoint{0.703049in}{1.232895in}}%
\pgfpathcurveto{\pgfqpoint{0.708873in}{1.227071in}}{\pgfqpoint{0.716773in}{1.223799in}}{\pgfqpoint{0.725009in}{1.223799in}}%
\pgfpathclose%
\pgfusepath{stroke,fill}%
\end{pgfscope}%
\begin{pgfscope}%
\pgfpathrectangle{\pgfqpoint{0.100000in}{0.212622in}}{\pgfqpoint{3.696000in}{3.696000in}}%
\pgfusepath{clip}%
\pgfsetbuttcap%
\pgfsetroundjoin%
\definecolor{currentfill}{rgb}{0.121569,0.466667,0.705882}%
\pgfsetfillcolor{currentfill}%
\pgfsetfillopacity{0.660888}%
\pgfsetlinewidth{1.003750pt}%
\definecolor{currentstroke}{rgb}{0.121569,0.466667,0.705882}%
\pgfsetstrokecolor{currentstroke}%
\pgfsetstrokeopacity{0.660888}%
\pgfsetdash{}{0pt}%
\pgfpathmoveto{\pgfqpoint{0.724348in}{1.223236in}}%
\pgfpathcurveto{\pgfqpoint{0.732584in}{1.223236in}}{\pgfqpoint{0.740484in}{1.226508in}}{\pgfqpoint{0.746308in}{1.232332in}}%
\pgfpathcurveto{\pgfqpoint{0.752132in}{1.238156in}}{\pgfqpoint{0.755404in}{1.246056in}}{\pgfqpoint{0.755404in}{1.254292in}}%
\pgfpathcurveto{\pgfqpoint{0.755404in}{1.262529in}}{\pgfqpoint{0.752132in}{1.270429in}}{\pgfqpoint{0.746308in}{1.276253in}}%
\pgfpathcurveto{\pgfqpoint{0.740484in}{1.282077in}}{\pgfqpoint{0.732584in}{1.285349in}}{\pgfqpoint{0.724348in}{1.285349in}}%
\pgfpathcurveto{\pgfqpoint{0.716112in}{1.285349in}}{\pgfqpoint{0.708212in}{1.282077in}}{\pgfqpoint{0.702388in}{1.276253in}}%
\pgfpathcurveto{\pgfqpoint{0.696564in}{1.270429in}}{\pgfqpoint{0.693291in}{1.262529in}}{\pgfqpoint{0.693291in}{1.254292in}}%
\pgfpathcurveto{\pgfqpoint{0.693291in}{1.246056in}}{\pgfqpoint{0.696564in}{1.238156in}}{\pgfqpoint{0.702388in}{1.232332in}}%
\pgfpathcurveto{\pgfqpoint{0.708212in}{1.226508in}}{\pgfqpoint{0.716112in}{1.223236in}}{\pgfqpoint{0.724348in}{1.223236in}}%
\pgfpathclose%
\pgfusepath{stroke,fill}%
\end{pgfscope}%
\begin{pgfscope}%
\pgfpathrectangle{\pgfqpoint{0.100000in}{0.212622in}}{\pgfqpoint{3.696000in}{3.696000in}}%
\pgfusepath{clip}%
\pgfsetbuttcap%
\pgfsetroundjoin%
\definecolor{currentfill}{rgb}{0.121569,0.466667,0.705882}%
\pgfsetfillcolor{currentfill}%
\pgfsetfillopacity{0.660927}%
\pgfsetlinewidth{1.003750pt}%
\definecolor{currentstroke}{rgb}{0.121569,0.466667,0.705882}%
\pgfsetstrokecolor{currentstroke}%
\pgfsetstrokeopacity{0.660927}%
\pgfsetdash{}{0pt}%
\pgfpathmoveto{\pgfqpoint{3.200978in}{2.212097in}}%
\pgfpathcurveto{\pgfqpoint{3.209214in}{2.212097in}}{\pgfqpoint{3.217114in}{2.215369in}}{\pgfqpoint{3.222938in}{2.221193in}}%
\pgfpathcurveto{\pgfqpoint{3.228762in}{2.227017in}}{\pgfqpoint{3.232034in}{2.234917in}}{\pgfqpoint{3.232034in}{2.243154in}}%
\pgfpathcurveto{\pgfqpoint{3.232034in}{2.251390in}}{\pgfqpoint{3.228762in}{2.259290in}}{\pgfqpoint{3.222938in}{2.265114in}}%
\pgfpathcurveto{\pgfqpoint{3.217114in}{2.270938in}}{\pgfqpoint{3.209214in}{2.274210in}}{\pgfqpoint{3.200978in}{2.274210in}}%
\pgfpathcurveto{\pgfqpoint{3.192741in}{2.274210in}}{\pgfqpoint{3.184841in}{2.270938in}}{\pgfqpoint{3.179017in}{2.265114in}}%
\pgfpathcurveto{\pgfqpoint{3.173193in}{2.259290in}}{\pgfqpoint{3.169921in}{2.251390in}}{\pgfqpoint{3.169921in}{2.243154in}}%
\pgfpathcurveto{\pgfqpoint{3.169921in}{2.234917in}}{\pgfqpoint{3.173193in}{2.227017in}}{\pgfqpoint{3.179017in}{2.221193in}}%
\pgfpathcurveto{\pgfqpoint{3.184841in}{2.215369in}}{\pgfqpoint{3.192741in}{2.212097in}}{\pgfqpoint{3.200978in}{2.212097in}}%
\pgfpathclose%
\pgfusepath{stroke,fill}%
\end{pgfscope}%
\begin{pgfscope}%
\pgfpathrectangle{\pgfqpoint{0.100000in}{0.212622in}}{\pgfqpoint{3.696000in}{3.696000in}}%
\pgfusepath{clip}%
\pgfsetbuttcap%
\pgfsetroundjoin%
\definecolor{currentfill}{rgb}{0.121569,0.466667,0.705882}%
\pgfsetfillcolor{currentfill}%
\pgfsetfillopacity{0.661254}%
\pgfsetlinewidth{1.003750pt}%
\definecolor{currentstroke}{rgb}{0.121569,0.466667,0.705882}%
\pgfsetstrokecolor{currentstroke}%
\pgfsetstrokeopacity{0.661254}%
\pgfsetdash{}{0pt}%
\pgfpathmoveto{\pgfqpoint{0.738388in}{1.215628in}}%
\pgfpathcurveto{\pgfqpoint{0.746624in}{1.215628in}}{\pgfqpoint{0.754524in}{1.218901in}}{\pgfqpoint{0.760348in}{1.224725in}}%
\pgfpathcurveto{\pgfqpoint{0.766172in}{1.230548in}}{\pgfqpoint{0.769444in}{1.238448in}}{\pgfqpoint{0.769444in}{1.246685in}}%
\pgfpathcurveto{\pgfqpoint{0.769444in}{1.254921in}}{\pgfqpoint{0.766172in}{1.262821in}}{\pgfqpoint{0.760348in}{1.268645in}}%
\pgfpathcurveto{\pgfqpoint{0.754524in}{1.274469in}}{\pgfqpoint{0.746624in}{1.277741in}}{\pgfqpoint{0.738388in}{1.277741in}}%
\pgfpathcurveto{\pgfqpoint{0.730152in}{1.277741in}}{\pgfqpoint{0.722251in}{1.274469in}}{\pgfqpoint{0.716428in}{1.268645in}}%
\pgfpathcurveto{\pgfqpoint{0.710604in}{1.262821in}}{\pgfqpoint{0.707331in}{1.254921in}}{\pgfqpoint{0.707331in}{1.246685in}}%
\pgfpathcurveto{\pgfqpoint{0.707331in}{1.238448in}}{\pgfqpoint{0.710604in}{1.230548in}}{\pgfqpoint{0.716428in}{1.224725in}}%
\pgfpathcurveto{\pgfqpoint{0.722251in}{1.218901in}}{\pgfqpoint{0.730152in}{1.215628in}}{\pgfqpoint{0.738388in}{1.215628in}}%
\pgfpathclose%
\pgfusepath{stroke,fill}%
\end{pgfscope}%
\begin{pgfscope}%
\pgfpathrectangle{\pgfqpoint{0.100000in}{0.212622in}}{\pgfqpoint{3.696000in}{3.696000in}}%
\pgfusepath{clip}%
\pgfsetbuttcap%
\pgfsetroundjoin%
\definecolor{currentfill}{rgb}{0.121569,0.466667,0.705882}%
\pgfsetfillcolor{currentfill}%
\pgfsetfillopacity{0.661289}%
\pgfsetlinewidth{1.003750pt}%
\definecolor{currentstroke}{rgb}{0.121569,0.466667,0.705882}%
\pgfsetstrokecolor{currentstroke}%
\pgfsetstrokeopacity{0.661289}%
\pgfsetdash{}{0pt}%
\pgfpathmoveto{\pgfqpoint{0.723081in}{1.222071in}}%
\pgfpathcurveto{\pgfqpoint{0.731317in}{1.222071in}}{\pgfqpoint{0.739217in}{1.225344in}}{\pgfqpoint{0.745041in}{1.231168in}}%
\pgfpathcurveto{\pgfqpoint{0.750865in}{1.236991in}}{\pgfqpoint{0.754137in}{1.244892in}}{\pgfqpoint{0.754137in}{1.253128in}}%
\pgfpathcurveto{\pgfqpoint{0.754137in}{1.261364in}}{\pgfqpoint{0.750865in}{1.269264in}}{\pgfqpoint{0.745041in}{1.275088in}}%
\pgfpathcurveto{\pgfqpoint{0.739217in}{1.280912in}}{\pgfqpoint{0.731317in}{1.284184in}}{\pgfqpoint{0.723081in}{1.284184in}}%
\pgfpathcurveto{\pgfqpoint{0.714845in}{1.284184in}}{\pgfqpoint{0.706945in}{1.280912in}}{\pgfqpoint{0.701121in}{1.275088in}}%
\pgfpathcurveto{\pgfqpoint{0.695297in}{1.269264in}}{\pgfqpoint{0.692024in}{1.261364in}}{\pgfqpoint{0.692024in}{1.253128in}}%
\pgfpathcurveto{\pgfqpoint{0.692024in}{1.244892in}}{\pgfqpoint{0.695297in}{1.236991in}}{\pgfqpoint{0.701121in}{1.231168in}}%
\pgfpathcurveto{\pgfqpoint{0.706945in}{1.225344in}}{\pgfqpoint{0.714845in}{1.222071in}}{\pgfqpoint{0.723081in}{1.222071in}}%
\pgfpathclose%
\pgfusepath{stroke,fill}%
\end{pgfscope}%
\begin{pgfscope}%
\pgfpathrectangle{\pgfqpoint{0.100000in}{0.212622in}}{\pgfqpoint{3.696000in}{3.696000in}}%
\pgfusepath{clip}%
\pgfsetbuttcap%
\pgfsetroundjoin%
\definecolor{currentfill}{rgb}{0.121569,0.466667,0.705882}%
\pgfsetfillcolor{currentfill}%
\pgfsetfillopacity{0.661289}%
\pgfsetlinewidth{1.003750pt}%
\definecolor{currentstroke}{rgb}{0.121569,0.466667,0.705882}%
\pgfsetstrokecolor{currentstroke}%
\pgfsetstrokeopacity{0.661289}%
\pgfsetdash{}{0pt}%
\pgfpathmoveto{\pgfqpoint{0.723080in}{1.222071in}}%
\pgfpathcurveto{\pgfqpoint{0.731317in}{1.222071in}}{\pgfqpoint{0.739217in}{1.225343in}}{\pgfqpoint{0.745041in}{1.231167in}}%
\pgfpathcurveto{\pgfqpoint{0.750865in}{1.236991in}}{\pgfqpoint{0.754137in}{1.244891in}}{\pgfqpoint{0.754137in}{1.253127in}}%
\pgfpathcurveto{\pgfqpoint{0.754137in}{1.261364in}}{\pgfqpoint{0.750865in}{1.269264in}}{\pgfqpoint{0.745041in}{1.275088in}}%
\pgfpathcurveto{\pgfqpoint{0.739217in}{1.280912in}}{\pgfqpoint{0.731317in}{1.284184in}}{\pgfqpoint{0.723080in}{1.284184in}}%
\pgfpathcurveto{\pgfqpoint{0.714844in}{1.284184in}}{\pgfqpoint{0.706944in}{1.280912in}}{\pgfqpoint{0.701120in}{1.275088in}}%
\pgfpathcurveto{\pgfqpoint{0.695296in}{1.269264in}}{\pgfqpoint{0.692024in}{1.261364in}}{\pgfqpoint{0.692024in}{1.253127in}}%
\pgfpathcurveto{\pgfqpoint{0.692024in}{1.244891in}}{\pgfqpoint{0.695296in}{1.236991in}}{\pgfqpoint{0.701120in}{1.231167in}}%
\pgfpathcurveto{\pgfqpoint{0.706944in}{1.225343in}}{\pgfqpoint{0.714844in}{1.222071in}}{\pgfqpoint{0.723080in}{1.222071in}}%
\pgfpathclose%
\pgfusepath{stroke,fill}%
\end{pgfscope}%
\begin{pgfscope}%
\pgfpathrectangle{\pgfqpoint{0.100000in}{0.212622in}}{\pgfqpoint{3.696000in}{3.696000in}}%
\pgfusepath{clip}%
\pgfsetbuttcap%
\pgfsetroundjoin%
\definecolor{currentfill}{rgb}{0.121569,0.466667,0.705882}%
\pgfsetfillcolor{currentfill}%
\pgfsetfillopacity{0.661289}%
\pgfsetlinewidth{1.003750pt}%
\definecolor{currentstroke}{rgb}{0.121569,0.466667,0.705882}%
\pgfsetstrokecolor{currentstroke}%
\pgfsetstrokeopacity{0.661289}%
\pgfsetdash{}{0pt}%
\pgfpathmoveto{\pgfqpoint{0.723079in}{1.222070in}}%
\pgfpathcurveto{\pgfqpoint{0.731316in}{1.222070in}}{\pgfqpoint{0.739216in}{1.225342in}}{\pgfqpoint{0.745040in}{1.231166in}}%
\pgfpathcurveto{\pgfqpoint{0.750863in}{1.236990in}}{\pgfqpoint{0.754136in}{1.244890in}}{\pgfqpoint{0.754136in}{1.253126in}}%
\pgfpathcurveto{\pgfqpoint{0.754136in}{1.261363in}}{\pgfqpoint{0.750863in}{1.269263in}}{\pgfqpoint{0.745040in}{1.275087in}}%
\pgfpathcurveto{\pgfqpoint{0.739216in}{1.280911in}}{\pgfqpoint{0.731316in}{1.284183in}}{\pgfqpoint{0.723079in}{1.284183in}}%
\pgfpathcurveto{\pgfqpoint{0.714843in}{1.284183in}}{\pgfqpoint{0.706943in}{1.280911in}}{\pgfqpoint{0.701119in}{1.275087in}}%
\pgfpathcurveto{\pgfqpoint{0.695295in}{1.269263in}}{\pgfqpoint{0.692023in}{1.261363in}}{\pgfqpoint{0.692023in}{1.253126in}}%
\pgfpathcurveto{\pgfqpoint{0.692023in}{1.244890in}}{\pgfqpoint{0.695295in}{1.236990in}}{\pgfqpoint{0.701119in}{1.231166in}}%
\pgfpathcurveto{\pgfqpoint{0.706943in}{1.225342in}}{\pgfqpoint{0.714843in}{1.222070in}}{\pgfqpoint{0.723079in}{1.222070in}}%
\pgfpathclose%
\pgfusepath{stroke,fill}%
\end{pgfscope}%
\begin{pgfscope}%
\pgfpathrectangle{\pgfqpoint{0.100000in}{0.212622in}}{\pgfqpoint{3.696000in}{3.696000in}}%
\pgfusepath{clip}%
\pgfsetbuttcap%
\pgfsetroundjoin%
\definecolor{currentfill}{rgb}{0.121569,0.466667,0.705882}%
\pgfsetfillcolor{currentfill}%
\pgfsetfillopacity{0.661290}%
\pgfsetlinewidth{1.003750pt}%
\definecolor{currentstroke}{rgb}{0.121569,0.466667,0.705882}%
\pgfsetstrokecolor{currentstroke}%
\pgfsetstrokeopacity{0.661290}%
\pgfsetdash{}{0pt}%
\pgfpathmoveto{\pgfqpoint{0.723078in}{1.222068in}}%
\pgfpathcurveto{\pgfqpoint{0.731314in}{1.222068in}}{\pgfqpoint{0.739214in}{1.225340in}}{\pgfqpoint{0.745038in}{1.231164in}}%
\pgfpathcurveto{\pgfqpoint{0.750862in}{1.236988in}}{\pgfqpoint{0.754134in}{1.244888in}}{\pgfqpoint{0.754134in}{1.253125in}}%
\pgfpathcurveto{\pgfqpoint{0.754134in}{1.261361in}}{\pgfqpoint{0.750862in}{1.269261in}}{\pgfqpoint{0.745038in}{1.275085in}}%
\pgfpathcurveto{\pgfqpoint{0.739214in}{1.280909in}}{\pgfqpoint{0.731314in}{1.284181in}}{\pgfqpoint{0.723078in}{1.284181in}}%
\pgfpathcurveto{\pgfqpoint{0.714841in}{1.284181in}}{\pgfqpoint{0.706941in}{1.280909in}}{\pgfqpoint{0.701117in}{1.275085in}}%
\pgfpathcurveto{\pgfqpoint{0.695294in}{1.269261in}}{\pgfqpoint{0.692021in}{1.261361in}}{\pgfqpoint{0.692021in}{1.253125in}}%
\pgfpathcurveto{\pgfqpoint{0.692021in}{1.244888in}}{\pgfqpoint{0.695294in}{1.236988in}}{\pgfqpoint{0.701117in}{1.231164in}}%
\pgfpathcurveto{\pgfqpoint{0.706941in}{1.225340in}}{\pgfqpoint{0.714841in}{1.222068in}}{\pgfqpoint{0.723078in}{1.222068in}}%
\pgfpathclose%
\pgfusepath{stroke,fill}%
\end{pgfscope}%
\begin{pgfscope}%
\pgfpathrectangle{\pgfqpoint{0.100000in}{0.212622in}}{\pgfqpoint{3.696000in}{3.696000in}}%
\pgfusepath{clip}%
\pgfsetbuttcap%
\pgfsetroundjoin%
\definecolor{currentfill}{rgb}{0.121569,0.466667,0.705882}%
\pgfsetfillcolor{currentfill}%
\pgfsetfillopacity{0.661291}%
\pgfsetlinewidth{1.003750pt}%
\definecolor{currentstroke}{rgb}{0.121569,0.466667,0.705882}%
\pgfsetstrokecolor{currentstroke}%
\pgfsetstrokeopacity{0.661291}%
\pgfsetdash{}{0pt}%
\pgfpathmoveto{\pgfqpoint{0.723075in}{1.222065in}}%
\pgfpathcurveto{\pgfqpoint{0.731311in}{1.222065in}}{\pgfqpoint{0.739211in}{1.225337in}}{\pgfqpoint{0.745035in}{1.231161in}}%
\pgfpathcurveto{\pgfqpoint{0.750859in}{1.236985in}}{\pgfqpoint{0.754131in}{1.244885in}}{\pgfqpoint{0.754131in}{1.253122in}}%
\pgfpathcurveto{\pgfqpoint{0.754131in}{1.261358in}}{\pgfqpoint{0.750859in}{1.269258in}}{\pgfqpoint{0.745035in}{1.275082in}}%
\pgfpathcurveto{\pgfqpoint{0.739211in}{1.280906in}}{\pgfqpoint{0.731311in}{1.284178in}}{\pgfqpoint{0.723075in}{1.284178in}}%
\pgfpathcurveto{\pgfqpoint{0.714839in}{1.284178in}}{\pgfqpoint{0.706938in}{1.280906in}}{\pgfqpoint{0.701115in}{1.275082in}}%
\pgfpathcurveto{\pgfqpoint{0.695291in}{1.269258in}}{\pgfqpoint{0.692018in}{1.261358in}}{\pgfqpoint{0.692018in}{1.253122in}}%
\pgfpathcurveto{\pgfqpoint{0.692018in}{1.244885in}}{\pgfqpoint{0.695291in}{1.236985in}}{\pgfqpoint{0.701115in}{1.231161in}}%
\pgfpathcurveto{\pgfqpoint{0.706938in}{1.225337in}}{\pgfqpoint{0.714839in}{1.222065in}}{\pgfqpoint{0.723075in}{1.222065in}}%
\pgfpathclose%
\pgfusepath{stroke,fill}%
\end{pgfscope}%
\begin{pgfscope}%
\pgfpathrectangle{\pgfqpoint{0.100000in}{0.212622in}}{\pgfqpoint{3.696000in}{3.696000in}}%
\pgfusepath{clip}%
\pgfsetbuttcap%
\pgfsetroundjoin%
\definecolor{currentfill}{rgb}{0.121569,0.466667,0.705882}%
\pgfsetfillcolor{currentfill}%
\pgfsetfillopacity{0.661293}%
\pgfsetlinewidth{1.003750pt}%
\definecolor{currentstroke}{rgb}{0.121569,0.466667,0.705882}%
\pgfsetstrokecolor{currentstroke}%
\pgfsetstrokeopacity{0.661293}%
\pgfsetdash{}{0pt}%
\pgfpathmoveto{\pgfqpoint{0.723069in}{1.222058in}}%
\pgfpathcurveto{\pgfqpoint{0.731305in}{1.222058in}}{\pgfqpoint{0.739205in}{1.225331in}}{\pgfqpoint{0.745029in}{1.231155in}}%
\pgfpathcurveto{\pgfqpoint{0.750853in}{1.236979in}}{\pgfqpoint{0.754126in}{1.244879in}}{\pgfqpoint{0.754126in}{1.253115in}}%
\pgfpathcurveto{\pgfqpoint{0.754126in}{1.261351in}}{\pgfqpoint{0.750853in}{1.269251in}}{\pgfqpoint{0.745029in}{1.275075in}}%
\pgfpathcurveto{\pgfqpoint{0.739205in}{1.280899in}}{\pgfqpoint{0.731305in}{1.284171in}}{\pgfqpoint{0.723069in}{1.284171in}}%
\pgfpathcurveto{\pgfqpoint{0.714833in}{1.284171in}}{\pgfqpoint{0.706933in}{1.280899in}}{\pgfqpoint{0.701109in}{1.275075in}}%
\pgfpathcurveto{\pgfqpoint{0.695285in}{1.269251in}}{\pgfqpoint{0.692013in}{1.261351in}}{\pgfqpoint{0.692013in}{1.253115in}}%
\pgfpathcurveto{\pgfqpoint{0.692013in}{1.244879in}}{\pgfqpoint{0.695285in}{1.236979in}}{\pgfqpoint{0.701109in}{1.231155in}}%
\pgfpathcurveto{\pgfqpoint{0.706933in}{1.225331in}}{\pgfqpoint{0.714833in}{1.222058in}}{\pgfqpoint{0.723069in}{1.222058in}}%
\pgfpathclose%
\pgfusepath{stroke,fill}%
\end{pgfscope}%
\begin{pgfscope}%
\pgfpathrectangle{\pgfqpoint{0.100000in}{0.212622in}}{\pgfqpoint{3.696000in}{3.696000in}}%
\pgfusepath{clip}%
\pgfsetbuttcap%
\pgfsetroundjoin%
\definecolor{currentfill}{rgb}{0.121569,0.466667,0.705882}%
\pgfsetfillcolor{currentfill}%
\pgfsetfillopacity{0.661298}%
\pgfsetlinewidth{1.003750pt}%
\definecolor{currentstroke}{rgb}{0.121569,0.466667,0.705882}%
\pgfsetstrokecolor{currentstroke}%
\pgfsetstrokeopacity{0.661298}%
\pgfsetdash{}{0pt}%
\pgfpathmoveto{\pgfqpoint{0.723060in}{1.222050in}}%
\pgfpathcurveto{\pgfqpoint{0.731296in}{1.222050in}}{\pgfqpoint{0.739196in}{1.225322in}}{\pgfqpoint{0.745020in}{1.231146in}}%
\pgfpathcurveto{\pgfqpoint{0.750844in}{1.236970in}}{\pgfqpoint{0.754116in}{1.244870in}}{\pgfqpoint{0.754116in}{1.253106in}}%
\pgfpathcurveto{\pgfqpoint{0.754116in}{1.261343in}}{\pgfqpoint{0.750844in}{1.269243in}}{\pgfqpoint{0.745020in}{1.275067in}}%
\pgfpathcurveto{\pgfqpoint{0.739196in}{1.280890in}}{\pgfqpoint{0.731296in}{1.284163in}}{\pgfqpoint{0.723060in}{1.284163in}}%
\pgfpathcurveto{\pgfqpoint{0.714824in}{1.284163in}}{\pgfqpoint{0.706924in}{1.280890in}}{\pgfqpoint{0.701100in}{1.275067in}}%
\pgfpathcurveto{\pgfqpoint{0.695276in}{1.269243in}}{\pgfqpoint{0.692003in}{1.261343in}}{\pgfqpoint{0.692003in}{1.253106in}}%
\pgfpathcurveto{\pgfqpoint{0.692003in}{1.244870in}}{\pgfqpoint{0.695276in}{1.236970in}}{\pgfqpoint{0.701100in}{1.231146in}}%
\pgfpathcurveto{\pgfqpoint{0.706924in}{1.225322in}}{\pgfqpoint{0.714824in}{1.222050in}}{\pgfqpoint{0.723060in}{1.222050in}}%
\pgfpathclose%
\pgfusepath{stroke,fill}%
\end{pgfscope}%
\begin{pgfscope}%
\pgfpathrectangle{\pgfqpoint{0.100000in}{0.212622in}}{\pgfqpoint{3.696000in}{3.696000in}}%
\pgfusepath{clip}%
\pgfsetbuttcap%
\pgfsetroundjoin%
\definecolor{currentfill}{rgb}{0.121569,0.466667,0.705882}%
\pgfsetfillcolor{currentfill}%
\pgfsetfillopacity{0.661304}%
\pgfsetlinewidth{1.003750pt}%
\definecolor{currentstroke}{rgb}{0.121569,0.466667,0.705882}%
\pgfsetstrokecolor{currentstroke}%
\pgfsetstrokeopacity{0.661304}%
\pgfsetdash{}{0pt}%
\pgfpathmoveto{\pgfqpoint{0.723040in}{1.222026in}}%
\pgfpathcurveto{\pgfqpoint{0.731276in}{1.222026in}}{\pgfqpoint{0.739176in}{1.225298in}}{\pgfqpoint{0.745000in}{1.231122in}}%
\pgfpathcurveto{\pgfqpoint{0.750824in}{1.236946in}}{\pgfqpoint{0.754096in}{1.244846in}}{\pgfqpoint{0.754096in}{1.253082in}}%
\pgfpathcurveto{\pgfqpoint{0.754096in}{1.261318in}}{\pgfqpoint{0.750824in}{1.269218in}}{\pgfqpoint{0.745000in}{1.275042in}}%
\pgfpathcurveto{\pgfqpoint{0.739176in}{1.280866in}}{\pgfqpoint{0.731276in}{1.284139in}}{\pgfqpoint{0.723040in}{1.284139in}}%
\pgfpathcurveto{\pgfqpoint{0.714804in}{1.284139in}}{\pgfqpoint{0.706903in}{1.280866in}}{\pgfqpoint{0.701080in}{1.275042in}}%
\pgfpathcurveto{\pgfqpoint{0.695256in}{1.269218in}}{\pgfqpoint{0.691983in}{1.261318in}}{\pgfqpoint{0.691983in}{1.253082in}}%
\pgfpathcurveto{\pgfqpoint{0.691983in}{1.244846in}}{\pgfqpoint{0.695256in}{1.236946in}}{\pgfqpoint{0.701080in}{1.231122in}}%
\pgfpathcurveto{\pgfqpoint{0.706903in}{1.225298in}}{\pgfqpoint{0.714804in}{1.222026in}}{\pgfqpoint{0.723040in}{1.222026in}}%
\pgfpathclose%
\pgfusepath{stroke,fill}%
\end{pgfscope}%
\begin{pgfscope}%
\pgfpathrectangle{\pgfqpoint{0.100000in}{0.212622in}}{\pgfqpoint{3.696000in}{3.696000in}}%
\pgfusepath{clip}%
\pgfsetbuttcap%
\pgfsetroundjoin%
\definecolor{currentfill}{rgb}{0.121569,0.466667,0.705882}%
\pgfsetfillcolor{currentfill}%
\pgfsetfillopacity{0.661317}%
\pgfsetlinewidth{1.003750pt}%
\definecolor{currentstroke}{rgb}{0.121569,0.466667,0.705882}%
\pgfsetstrokecolor{currentstroke}%
\pgfsetstrokeopacity{0.661317}%
\pgfsetdash{}{0pt}%
\pgfpathmoveto{\pgfqpoint{0.723011in}{1.221986in}}%
\pgfpathcurveto{\pgfqpoint{0.731248in}{1.221986in}}{\pgfqpoint{0.739148in}{1.225258in}}{\pgfqpoint{0.744972in}{1.231082in}}%
\pgfpathcurveto{\pgfqpoint{0.750795in}{1.236906in}}{\pgfqpoint{0.754068in}{1.244806in}}{\pgfqpoint{0.754068in}{1.253042in}}%
\pgfpathcurveto{\pgfqpoint{0.754068in}{1.261279in}}{\pgfqpoint{0.750795in}{1.269179in}}{\pgfqpoint{0.744972in}{1.275003in}}%
\pgfpathcurveto{\pgfqpoint{0.739148in}{1.280827in}}{\pgfqpoint{0.731248in}{1.284099in}}{\pgfqpoint{0.723011in}{1.284099in}}%
\pgfpathcurveto{\pgfqpoint{0.714775in}{1.284099in}}{\pgfqpoint{0.706875in}{1.280827in}}{\pgfqpoint{0.701051in}{1.275003in}}%
\pgfpathcurveto{\pgfqpoint{0.695227in}{1.269179in}}{\pgfqpoint{0.691955in}{1.261279in}}{\pgfqpoint{0.691955in}{1.253042in}}%
\pgfpathcurveto{\pgfqpoint{0.691955in}{1.244806in}}{\pgfqpoint{0.695227in}{1.236906in}}{\pgfqpoint{0.701051in}{1.231082in}}%
\pgfpathcurveto{\pgfqpoint{0.706875in}{1.225258in}}{\pgfqpoint{0.714775in}{1.221986in}}{\pgfqpoint{0.723011in}{1.221986in}}%
\pgfpathclose%
\pgfusepath{stroke,fill}%
\end{pgfscope}%
\begin{pgfscope}%
\pgfpathrectangle{\pgfqpoint{0.100000in}{0.212622in}}{\pgfqpoint{3.696000in}{3.696000in}}%
\pgfusepath{clip}%
\pgfsetbuttcap%
\pgfsetroundjoin%
\definecolor{currentfill}{rgb}{0.121569,0.466667,0.705882}%
\pgfsetfillcolor{currentfill}%
\pgfsetfillopacity{0.661342}%
\pgfsetlinewidth{1.003750pt}%
\definecolor{currentstroke}{rgb}{0.121569,0.466667,0.705882}%
\pgfsetstrokecolor{currentstroke}%
\pgfsetstrokeopacity{0.661342}%
\pgfsetdash{}{0pt}%
\pgfpathmoveto{\pgfqpoint{0.722954in}{1.221924in}}%
\pgfpathcurveto{\pgfqpoint{0.731191in}{1.221924in}}{\pgfqpoint{0.739091in}{1.225197in}}{\pgfqpoint{0.744915in}{1.231021in}}%
\pgfpathcurveto{\pgfqpoint{0.750739in}{1.236844in}}{\pgfqpoint{0.754011in}{1.244744in}}{\pgfqpoint{0.754011in}{1.252981in}}%
\pgfpathcurveto{\pgfqpoint{0.754011in}{1.261217in}}{\pgfqpoint{0.750739in}{1.269117in}}{\pgfqpoint{0.744915in}{1.274941in}}%
\pgfpathcurveto{\pgfqpoint{0.739091in}{1.280765in}}{\pgfqpoint{0.731191in}{1.284037in}}{\pgfqpoint{0.722954in}{1.284037in}}%
\pgfpathcurveto{\pgfqpoint{0.714718in}{1.284037in}}{\pgfqpoint{0.706818in}{1.280765in}}{\pgfqpoint{0.700994in}{1.274941in}}%
\pgfpathcurveto{\pgfqpoint{0.695170in}{1.269117in}}{\pgfqpoint{0.691898in}{1.261217in}}{\pgfqpoint{0.691898in}{1.252981in}}%
\pgfpathcurveto{\pgfqpoint{0.691898in}{1.244744in}}{\pgfqpoint{0.695170in}{1.236844in}}{\pgfqpoint{0.700994in}{1.231021in}}%
\pgfpathcurveto{\pgfqpoint{0.706818in}{1.225197in}}{\pgfqpoint{0.714718in}{1.221924in}}{\pgfqpoint{0.722954in}{1.221924in}}%
\pgfpathclose%
\pgfusepath{stroke,fill}%
\end{pgfscope}%
\begin{pgfscope}%
\pgfpathrectangle{\pgfqpoint{0.100000in}{0.212622in}}{\pgfqpoint{3.696000in}{3.696000in}}%
\pgfusepath{clip}%
\pgfsetbuttcap%
\pgfsetroundjoin%
\definecolor{currentfill}{rgb}{0.121569,0.466667,0.705882}%
\pgfsetfillcolor{currentfill}%
\pgfsetfillopacity{0.661376}%
\pgfsetlinewidth{1.003750pt}%
\definecolor{currentstroke}{rgb}{0.121569,0.466667,0.705882}%
\pgfsetstrokecolor{currentstroke}%
\pgfsetstrokeopacity{0.661376}%
\pgfsetdash{}{0pt}%
\pgfpathmoveto{\pgfqpoint{0.722868in}{1.221735in}}%
\pgfpathcurveto{\pgfqpoint{0.731104in}{1.221735in}}{\pgfqpoint{0.739004in}{1.225007in}}{\pgfqpoint{0.744828in}{1.230831in}}%
\pgfpathcurveto{\pgfqpoint{0.750652in}{1.236655in}}{\pgfqpoint{0.753925in}{1.244555in}}{\pgfqpoint{0.753925in}{1.252791in}}%
\pgfpathcurveto{\pgfqpoint{0.753925in}{1.261027in}}{\pgfqpoint{0.750652in}{1.268927in}}{\pgfqpoint{0.744828in}{1.274751in}}%
\pgfpathcurveto{\pgfqpoint{0.739004in}{1.280575in}}{\pgfqpoint{0.731104in}{1.283848in}}{\pgfqpoint{0.722868in}{1.283848in}}%
\pgfpathcurveto{\pgfqpoint{0.714632in}{1.283848in}}{\pgfqpoint{0.706732in}{1.280575in}}{\pgfqpoint{0.700908in}{1.274751in}}%
\pgfpathcurveto{\pgfqpoint{0.695084in}{1.268927in}}{\pgfqpoint{0.691812in}{1.261027in}}{\pgfqpoint{0.691812in}{1.252791in}}%
\pgfpathcurveto{\pgfqpoint{0.691812in}{1.244555in}}{\pgfqpoint{0.695084in}{1.236655in}}{\pgfqpoint{0.700908in}{1.230831in}}%
\pgfpathcurveto{\pgfqpoint{0.706732in}{1.225007in}}{\pgfqpoint{0.714632in}{1.221735in}}{\pgfqpoint{0.722868in}{1.221735in}}%
\pgfpathclose%
\pgfusepath{stroke,fill}%
\end{pgfscope}%
\begin{pgfscope}%
\pgfpathrectangle{\pgfqpoint{0.100000in}{0.212622in}}{\pgfqpoint{3.696000in}{3.696000in}}%
\pgfusepath{clip}%
\pgfsetbuttcap%
\pgfsetroundjoin%
\definecolor{currentfill}{rgb}{0.121569,0.466667,0.705882}%
\pgfsetfillcolor{currentfill}%
\pgfsetfillopacity{0.661441}%
\pgfsetlinewidth{1.003750pt}%
\definecolor{currentstroke}{rgb}{0.121569,0.466667,0.705882}%
\pgfsetstrokecolor{currentstroke}%
\pgfsetstrokeopacity{0.661441}%
\pgfsetdash{}{0pt}%
\pgfpathmoveto{\pgfqpoint{0.722726in}{1.221384in}}%
\pgfpathcurveto{\pgfqpoint{0.730962in}{1.221384in}}{\pgfqpoint{0.738862in}{1.224657in}}{\pgfqpoint{0.744686in}{1.230480in}}%
\pgfpathcurveto{\pgfqpoint{0.750510in}{1.236304in}}{\pgfqpoint{0.753783in}{1.244204in}}{\pgfqpoint{0.753783in}{1.252441in}}%
\pgfpathcurveto{\pgfqpoint{0.753783in}{1.260677in}}{\pgfqpoint{0.750510in}{1.268577in}}{\pgfqpoint{0.744686in}{1.274401in}}%
\pgfpathcurveto{\pgfqpoint{0.738862in}{1.280225in}}{\pgfqpoint{0.730962in}{1.283497in}}{\pgfqpoint{0.722726in}{1.283497in}}%
\pgfpathcurveto{\pgfqpoint{0.714490in}{1.283497in}}{\pgfqpoint{0.706590in}{1.280225in}}{\pgfqpoint{0.700766in}{1.274401in}}%
\pgfpathcurveto{\pgfqpoint{0.694942in}{1.268577in}}{\pgfqpoint{0.691670in}{1.260677in}}{\pgfqpoint{0.691670in}{1.252441in}}%
\pgfpathcurveto{\pgfqpoint{0.691670in}{1.244204in}}{\pgfqpoint{0.694942in}{1.236304in}}{\pgfqpoint{0.700766in}{1.230480in}}%
\pgfpathcurveto{\pgfqpoint{0.706590in}{1.224657in}}{\pgfqpoint{0.714490in}{1.221384in}}{\pgfqpoint{0.722726in}{1.221384in}}%
\pgfpathclose%
\pgfusepath{stroke,fill}%
\end{pgfscope}%
\begin{pgfscope}%
\pgfpathrectangle{\pgfqpoint{0.100000in}{0.212622in}}{\pgfqpoint{3.696000in}{3.696000in}}%
\pgfusepath{clip}%
\pgfsetbuttcap%
\pgfsetroundjoin%
\definecolor{currentfill}{rgb}{0.121569,0.466667,0.705882}%
\pgfsetfillcolor{currentfill}%
\pgfsetfillopacity{0.661554}%
\pgfsetlinewidth{1.003750pt}%
\definecolor{currentstroke}{rgb}{0.121569,0.466667,0.705882}%
\pgfsetstrokecolor{currentstroke}%
\pgfsetstrokeopacity{0.661554}%
\pgfsetdash{}{0pt}%
\pgfpathmoveto{\pgfqpoint{0.722413in}{1.220772in}}%
\pgfpathcurveto{\pgfqpoint{0.730649in}{1.220772in}}{\pgfqpoint{0.738549in}{1.224045in}}{\pgfqpoint{0.744373in}{1.229869in}}%
\pgfpathcurveto{\pgfqpoint{0.750197in}{1.235693in}}{\pgfqpoint{0.753469in}{1.243593in}}{\pgfqpoint{0.753469in}{1.251829in}}%
\pgfpathcurveto{\pgfqpoint{0.753469in}{1.260065in}}{\pgfqpoint{0.750197in}{1.267965in}}{\pgfqpoint{0.744373in}{1.273789in}}%
\pgfpathcurveto{\pgfqpoint{0.738549in}{1.279613in}}{\pgfqpoint{0.730649in}{1.282885in}}{\pgfqpoint{0.722413in}{1.282885in}}%
\pgfpathcurveto{\pgfqpoint{0.714176in}{1.282885in}}{\pgfqpoint{0.706276in}{1.279613in}}{\pgfqpoint{0.700453in}{1.273789in}}%
\pgfpathcurveto{\pgfqpoint{0.694629in}{1.267965in}}{\pgfqpoint{0.691356in}{1.260065in}}{\pgfqpoint{0.691356in}{1.251829in}}%
\pgfpathcurveto{\pgfqpoint{0.691356in}{1.243593in}}{\pgfqpoint{0.694629in}{1.235693in}}{\pgfqpoint{0.700453in}{1.229869in}}%
\pgfpathcurveto{\pgfqpoint{0.706276in}{1.224045in}}{\pgfqpoint{0.714176in}{1.220772in}}{\pgfqpoint{0.722413in}{1.220772in}}%
\pgfpathclose%
\pgfusepath{stroke,fill}%
\end{pgfscope}%
\begin{pgfscope}%
\pgfpathrectangle{\pgfqpoint{0.100000in}{0.212622in}}{\pgfqpoint{3.696000in}{3.696000in}}%
\pgfusepath{clip}%
\pgfsetbuttcap%
\pgfsetroundjoin%
\definecolor{currentfill}{rgb}{0.121569,0.466667,0.705882}%
\pgfsetfillcolor{currentfill}%
\pgfsetfillopacity{0.661810}%
\pgfsetlinewidth{1.003750pt}%
\definecolor{currentstroke}{rgb}{0.121569,0.466667,0.705882}%
\pgfsetstrokecolor{currentstroke}%
\pgfsetstrokeopacity{0.661810}%
\pgfsetdash{}{0pt}%
\pgfpathmoveto{\pgfqpoint{0.722047in}{1.219839in}}%
\pgfpathcurveto{\pgfqpoint{0.730284in}{1.219839in}}{\pgfqpoint{0.738184in}{1.223111in}}{\pgfqpoint{0.744008in}{1.228935in}}%
\pgfpathcurveto{\pgfqpoint{0.749832in}{1.234759in}}{\pgfqpoint{0.753104in}{1.242659in}}{\pgfqpoint{0.753104in}{1.250896in}}%
\pgfpathcurveto{\pgfqpoint{0.753104in}{1.259132in}}{\pgfqpoint{0.749832in}{1.267032in}}{\pgfqpoint{0.744008in}{1.272856in}}%
\pgfpathcurveto{\pgfqpoint{0.738184in}{1.278680in}}{\pgfqpoint{0.730284in}{1.281952in}}{\pgfqpoint{0.722047in}{1.281952in}}%
\pgfpathcurveto{\pgfqpoint{0.713811in}{1.281952in}}{\pgfqpoint{0.705911in}{1.278680in}}{\pgfqpoint{0.700087in}{1.272856in}}%
\pgfpathcurveto{\pgfqpoint{0.694263in}{1.267032in}}{\pgfqpoint{0.690991in}{1.259132in}}{\pgfqpoint{0.690991in}{1.250896in}}%
\pgfpathcurveto{\pgfqpoint{0.690991in}{1.242659in}}{\pgfqpoint{0.694263in}{1.234759in}}{\pgfqpoint{0.700087in}{1.228935in}}%
\pgfpathcurveto{\pgfqpoint{0.705911in}{1.223111in}}{\pgfqpoint{0.713811in}{1.219839in}}{\pgfqpoint{0.722047in}{1.219839in}}%
\pgfpathclose%
\pgfusepath{stroke,fill}%
\end{pgfscope}%
\begin{pgfscope}%
\pgfpathrectangle{\pgfqpoint{0.100000in}{0.212622in}}{\pgfqpoint{3.696000in}{3.696000in}}%
\pgfusepath{clip}%
\pgfsetbuttcap%
\pgfsetroundjoin%
\definecolor{currentfill}{rgb}{0.121569,0.466667,0.705882}%
\pgfsetfillcolor{currentfill}%
\pgfsetfillopacity{0.662253}%
\pgfsetlinewidth{1.003750pt}%
\definecolor{currentstroke}{rgb}{0.121569,0.466667,0.705882}%
\pgfsetstrokecolor{currentstroke}%
\pgfsetstrokeopacity{0.662253}%
\pgfsetdash{}{0pt}%
\pgfpathmoveto{\pgfqpoint{0.720851in}{1.218361in}}%
\pgfpathcurveto{\pgfqpoint{0.729087in}{1.218361in}}{\pgfqpoint{0.736987in}{1.221633in}}{\pgfqpoint{0.742811in}{1.227457in}}%
\pgfpathcurveto{\pgfqpoint{0.748635in}{1.233281in}}{\pgfqpoint{0.751908in}{1.241181in}}{\pgfqpoint{0.751908in}{1.249417in}}%
\pgfpathcurveto{\pgfqpoint{0.751908in}{1.257654in}}{\pgfqpoint{0.748635in}{1.265554in}}{\pgfqpoint{0.742811in}{1.271378in}}%
\pgfpathcurveto{\pgfqpoint{0.736987in}{1.277201in}}{\pgfqpoint{0.729087in}{1.280474in}}{\pgfqpoint{0.720851in}{1.280474in}}%
\pgfpathcurveto{\pgfqpoint{0.712615in}{1.280474in}}{\pgfqpoint{0.704715in}{1.277201in}}{\pgfqpoint{0.698891in}{1.271378in}}%
\pgfpathcurveto{\pgfqpoint{0.693067in}{1.265554in}}{\pgfqpoint{0.689795in}{1.257654in}}{\pgfqpoint{0.689795in}{1.249417in}}%
\pgfpathcurveto{\pgfqpoint{0.689795in}{1.241181in}}{\pgfqpoint{0.693067in}{1.233281in}}{\pgfqpoint{0.698891in}{1.227457in}}%
\pgfpathcurveto{\pgfqpoint{0.704715in}{1.221633in}}{\pgfqpoint{0.712615in}{1.218361in}}{\pgfqpoint{0.720851in}{1.218361in}}%
\pgfpathclose%
\pgfusepath{stroke,fill}%
\end{pgfscope}%
\begin{pgfscope}%
\pgfpathrectangle{\pgfqpoint{0.100000in}{0.212622in}}{\pgfqpoint{3.696000in}{3.696000in}}%
\pgfusepath{clip}%
\pgfsetbuttcap%
\pgfsetroundjoin%
\definecolor{currentfill}{rgb}{0.121569,0.466667,0.705882}%
\pgfsetfillcolor{currentfill}%
\pgfsetfillopacity{0.662253}%
\pgfsetlinewidth{1.003750pt}%
\definecolor{currentstroke}{rgb}{0.121569,0.466667,0.705882}%
\pgfsetstrokecolor{currentstroke}%
\pgfsetstrokeopacity{0.662253}%
\pgfsetdash{}{0pt}%
\pgfpathmoveto{\pgfqpoint{0.720851in}{1.218360in}}%
\pgfpathcurveto{\pgfqpoint{0.729087in}{1.218360in}}{\pgfqpoint{0.736987in}{1.221632in}}{\pgfqpoint{0.742811in}{1.227456in}}%
\pgfpathcurveto{\pgfqpoint{0.748635in}{1.233280in}}{\pgfqpoint{0.751907in}{1.241180in}}{\pgfqpoint{0.751907in}{1.249417in}}%
\pgfpathcurveto{\pgfqpoint{0.751907in}{1.257653in}}{\pgfqpoint{0.748635in}{1.265553in}}{\pgfqpoint{0.742811in}{1.271377in}}%
\pgfpathcurveto{\pgfqpoint{0.736987in}{1.277201in}}{\pgfqpoint{0.729087in}{1.280473in}}{\pgfqpoint{0.720851in}{1.280473in}}%
\pgfpathcurveto{\pgfqpoint{0.712614in}{1.280473in}}{\pgfqpoint{0.704714in}{1.277201in}}{\pgfqpoint{0.698890in}{1.271377in}}%
\pgfpathcurveto{\pgfqpoint{0.693066in}{1.265553in}}{\pgfqpoint{0.689794in}{1.257653in}}{\pgfqpoint{0.689794in}{1.249417in}}%
\pgfpathcurveto{\pgfqpoint{0.689794in}{1.241180in}}{\pgfqpoint{0.693066in}{1.233280in}}{\pgfqpoint{0.698890in}{1.227456in}}%
\pgfpathcurveto{\pgfqpoint{0.704714in}{1.221632in}}{\pgfqpoint{0.712614in}{1.218360in}}{\pgfqpoint{0.720851in}{1.218360in}}%
\pgfpathclose%
\pgfusepath{stroke,fill}%
\end{pgfscope}%
\begin{pgfscope}%
\pgfpathrectangle{\pgfqpoint{0.100000in}{0.212622in}}{\pgfqpoint{3.696000in}{3.696000in}}%
\pgfusepath{clip}%
\pgfsetbuttcap%
\pgfsetroundjoin%
\definecolor{currentfill}{rgb}{0.121569,0.466667,0.705882}%
\pgfsetfillcolor{currentfill}%
\pgfsetfillopacity{0.662254}%
\pgfsetlinewidth{1.003750pt}%
\definecolor{currentstroke}{rgb}{0.121569,0.466667,0.705882}%
\pgfsetstrokecolor{currentstroke}%
\pgfsetstrokeopacity{0.662254}%
\pgfsetdash{}{0pt}%
\pgfpathmoveto{\pgfqpoint{0.720850in}{1.218359in}}%
\pgfpathcurveto{\pgfqpoint{0.729086in}{1.218359in}}{\pgfqpoint{0.736986in}{1.221631in}}{\pgfqpoint{0.742810in}{1.227455in}}%
\pgfpathcurveto{\pgfqpoint{0.748634in}{1.233279in}}{\pgfqpoint{0.751906in}{1.241179in}}{\pgfqpoint{0.751906in}{1.249415in}}%
\pgfpathcurveto{\pgfqpoint{0.751906in}{1.257651in}}{\pgfqpoint{0.748634in}{1.265552in}}{\pgfqpoint{0.742810in}{1.271375in}}%
\pgfpathcurveto{\pgfqpoint{0.736986in}{1.277199in}}{\pgfqpoint{0.729086in}{1.280472in}}{\pgfqpoint{0.720850in}{1.280472in}}%
\pgfpathcurveto{\pgfqpoint{0.712613in}{1.280472in}}{\pgfqpoint{0.704713in}{1.277199in}}{\pgfqpoint{0.698889in}{1.271375in}}%
\pgfpathcurveto{\pgfqpoint{0.693065in}{1.265552in}}{\pgfqpoint{0.689793in}{1.257651in}}{\pgfqpoint{0.689793in}{1.249415in}}%
\pgfpathcurveto{\pgfqpoint{0.689793in}{1.241179in}}{\pgfqpoint{0.693065in}{1.233279in}}{\pgfqpoint{0.698889in}{1.227455in}}%
\pgfpathcurveto{\pgfqpoint{0.704713in}{1.221631in}}{\pgfqpoint{0.712613in}{1.218359in}}{\pgfqpoint{0.720850in}{1.218359in}}%
\pgfpathclose%
\pgfusepath{stroke,fill}%
\end{pgfscope}%
\begin{pgfscope}%
\pgfpathrectangle{\pgfqpoint{0.100000in}{0.212622in}}{\pgfqpoint{3.696000in}{3.696000in}}%
\pgfusepath{clip}%
\pgfsetbuttcap%
\pgfsetroundjoin%
\definecolor{currentfill}{rgb}{0.121569,0.466667,0.705882}%
\pgfsetfillcolor{currentfill}%
\pgfsetfillopacity{0.662254}%
\pgfsetlinewidth{1.003750pt}%
\definecolor{currentstroke}{rgb}{0.121569,0.466667,0.705882}%
\pgfsetstrokecolor{currentstroke}%
\pgfsetstrokeopacity{0.662254}%
\pgfsetdash{}{0pt}%
\pgfpathmoveto{\pgfqpoint{0.720848in}{1.218356in}}%
\pgfpathcurveto{\pgfqpoint{0.729084in}{1.218356in}}{\pgfqpoint{0.736984in}{1.221629in}}{\pgfqpoint{0.742808in}{1.227453in}}%
\pgfpathcurveto{\pgfqpoint{0.748632in}{1.233276in}}{\pgfqpoint{0.751904in}{1.241177in}}{\pgfqpoint{0.751904in}{1.249413in}}%
\pgfpathcurveto{\pgfqpoint{0.751904in}{1.257649in}}{\pgfqpoint{0.748632in}{1.265549in}}{\pgfqpoint{0.742808in}{1.271373in}}%
\pgfpathcurveto{\pgfqpoint{0.736984in}{1.277197in}}{\pgfqpoint{0.729084in}{1.280469in}}{\pgfqpoint{0.720848in}{1.280469in}}%
\pgfpathcurveto{\pgfqpoint{0.712612in}{1.280469in}}{\pgfqpoint{0.704711in}{1.277197in}}{\pgfqpoint{0.698888in}{1.271373in}}%
\pgfpathcurveto{\pgfqpoint{0.693064in}{1.265549in}}{\pgfqpoint{0.689791in}{1.257649in}}{\pgfqpoint{0.689791in}{1.249413in}}%
\pgfpathcurveto{\pgfqpoint{0.689791in}{1.241177in}}{\pgfqpoint{0.693064in}{1.233276in}}{\pgfqpoint{0.698888in}{1.227453in}}%
\pgfpathcurveto{\pgfqpoint{0.704711in}{1.221629in}}{\pgfqpoint{0.712612in}{1.218356in}}{\pgfqpoint{0.720848in}{1.218356in}}%
\pgfpathclose%
\pgfusepath{stroke,fill}%
\end{pgfscope}%
\begin{pgfscope}%
\pgfpathrectangle{\pgfqpoint{0.100000in}{0.212622in}}{\pgfqpoint{3.696000in}{3.696000in}}%
\pgfusepath{clip}%
\pgfsetbuttcap%
\pgfsetroundjoin%
\definecolor{currentfill}{rgb}{0.121569,0.466667,0.705882}%
\pgfsetfillcolor{currentfill}%
\pgfsetfillopacity{0.662256}%
\pgfsetlinewidth{1.003750pt}%
\definecolor{currentstroke}{rgb}{0.121569,0.466667,0.705882}%
\pgfsetstrokecolor{currentstroke}%
\pgfsetstrokeopacity{0.662256}%
\pgfsetdash{}{0pt}%
\pgfpathmoveto{\pgfqpoint{0.720844in}{1.218351in}}%
\pgfpathcurveto{\pgfqpoint{0.729081in}{1.218351in}}{\pgfqpoint{0.736981in}{1.221624in}}{\pgfqpoint{0.742805in}{1.227448in}}%
\pgfpathcurveto{\pgfqpoint{0.748629in}{1.233272in}}{\pgfqpoint{0.751901in}{1.241172in}}{\pgfqpoint{0.751901in}{1.249408in}}%
\pgfpathcurveto{\pgfqpoint{0.751901in}{1.257644in}}{\pgfqpoint{0.748629in}{1.265544in}}{\pgfqpoint{0.742805in}{1.271368in}}%
\pgfpathcurveto{\pgfqpoint{0.736981in}{1.277192in}}{\pgfqpoint{0.729081in}{1.280464in}}{\pgfqpoint{0.720844in}{1.280464in}}%
\pgfpathcurveto{\pgfqpoint{0.712608in}{1.280464in}}{\pgfqpoint{0.704708in}{1.277192in}}{\pgfqpoint{0.698884in}{1.271368in}}%
\pgfpathcurveto{\pgfqpoint{0.693060in}{1.265544in}}{\pgfqpoint{0.689788in}{1.257644in}}{\pgfqpoint{0.689788in}{1.249408in}}%
\pgfpathcurveto{\pgfqpoint{0.689788in}{1.241172in}}{\pgfqpoint{0.693060in}{1.233272in}}{\pgfqpoint{0.698884in}{1.227448in}}%
\pgfpathcurveto{\pgfqpoint{0.704708in}{1.221624in}}{\pgfqpoint{0.712608in}{1.218351in}}{\pgfqpoint{0.720844in}{1.218351in}}%
\pgfpathclose%
\pgfusepath{stroke,fill}%
\end{pgfscope}%
\begin{pgfscope}%
\pgfpathrectangle{\pgfqpoint{0.100000in}{0.212622in}}{\pgfqpoint{3.696000in}{3.696000in}}%
\pgfusepath{clip}%
\pgfsetbuttcap%
\pgfsetroundjoin%
\definecolor{currentfill}{rgb}{0.121569,0.466667,0.705882}%
\pgfsetfillcolor{currentfill}%
\pgfsetfillopacity{0.662258}%
\pgfsetlinewidth{1.003750pt}%
\definecolor{currentstroke}{rgb}{0.121569,0.466667,0.705882}%
\pgfsetstrokecolor{currentstroke}%
\pgfsetstrokeopacity{0.662258}%
\pgfsetdash{}{0pt}%
\pgfpathmoveto{\pgfqpoint{0.720838in}{1.218343in}}%
\pgfpathcurveto{\pgfqpoint{0.729075in}{1.218343in}}{\pgfqpoint{0.736975in}{1.221616in}}{\pgfqpoint{0.742799in}{1.227440in}}%
\pgfpathcurveto{\pgfqpoint{0.748623in}{1.233263in}}{\pgfqpoint{0.751895in}{1.241164in}}{\pgfqpoint{0.751895in}{1.249400in}}%
\pgfpathcurveto{\pgfqpoint{0.751895in}{1.257636in}}{\pgfqpoint{0.748623in}{1.265536in}}{\pgfqpoint{0.742799in}{1.271360in}}%
\pgfpathcurveto{\pgfqpoint{0.736975in}{1.277184in}}{\pgfqpoint{0.729075in}{1.280456in}}{\pgfqpoint{0.720838in}{1.280456in}}%
\pgfpathcurveto{\pgfqpoint{0.712602in}{1.280456in}}{\pgfqpoint{0.704702in}{1.277184in}}{\pgfqpoint{0.698878in}{1.271360in}}%
\pgfpathcurveto{\pgfqpoint{0.693054in}{1.265536in}}{\pgfqpoint{0.689782in}{1.257636in}}{\pgfqpoint{0.689782in}{1.249400in}}%
\pgfpathcurveto{\pgfqpoint{0.689782in}{1.241164in}}{\pgfqpoint{0.693054in}{1.233263in}}{\pgfqpoint{0.698878in}{1.227440in}}%
\pgfpathcurveto{\pgfqpoint{0.704702in}{1.221616in}}{\pgfqpoint{0.712602in}{1.218343in}}{\pgfqpoint{0.720838in}{1.218343in}}%
\pgfpathclose%
\pgfusepath{stroke,fill}%
\end{pgfscope}%
\begin{pgfscope}%
\pgfpathrectangle{\pgfqpoint{0.100000in}{0.212622in}}{\pgfqpoint{3.696000in}{3.696000in}}%
\pgfusepath{clip}%
\pgfsetbuttcap%
\pgfsetroundjoin%
\definecolor{currentfill}{rgb}{0.121569,0.466667,0.705882}%
\pgfsetfillcolor{currentfill}%
\pgfsetfillopacity{0.662262}%
\pgfsetlinewidth{1.003750pt}%
\definecolor{currentstroke}{rgb}{0.121569,0.466667,0.705882}%
\pgfsetstrokecolor{currentstroke}%
\pgfsetstrokeopacity{0.662262}%
\pgfsetdash{}{0pt}%
\pgfpathmoveto{\pgfqpoint{0.720826in}{1.218327in}}%
\pgfpathcurveto{\pgfqpoint{0.729063in}{1.218327in}}{\pgfqpoint{0.736963in}{1.221600in}}{\pgfqpoint{0.742787in}{1.227424in}}%
\pgfpathcurveto{\pgfqpoint{0.748610in}{1.233248in}}{\pgfqpoint{0.751883in}{1.241148in}}{\pgfqpoint{0.751883in}{1.249384in}}%
\pgfpathcurveto{\pgfqpoint{0.751883in}{1.257620in}}{\pgfqpoint{0.748610in}{1.265520in}}{\pgfqpoint{0.742787in}{1.271344in}}%
\pgfpathcurveto{\pgfqpoint{0.736963in}{1.277168in}}{\pgfqpoint{0.729063in}{1.280440in}}{\pgfqpoint{0.720826in}{1.280440in}}%
\pgfpathcurveto{\pgfqpoint{0.712590in}{1.280440in}}{\pgfqpoint{0.704690in}{1.277168in}}{\pgfqpoint{0.698866in}{1.271344in}}%
\pgfpathcurveto{\pgfqpoint{0.693042in}{1.265520in}}{\pgfqpoint{0.689770in}{1.257620in}}{\pgfqpoint{0.689770in}{1.249384in}}%
\pgfpathcurveto{\pgfqpoint{0.689770in}{1.241148in}}{\pgfqpoint{0.693042in}{1.233248in}}{\pgfqpoint{0.698866in}{1.227424in}}%
\pgfpathcurveto{\pgfqpoint{0.704690in}{1.221600in}}{\pgfqpoint{0.712590in}{1.218327in}}{\pgfqpoint{0.720826in}{1.218327in}}%
\pgfpathclose%
\pgfusepath{stroke,fill}%
\end{pgfscope}%
\begin{pgfscope}%
\pgfpathrectangle{\pgfqpoint{0.100000in}{0.212622in}}{\pgfqpoint{3.696000in}{3.696000in}}%
\pgfusepath{clip}%
\pgfsetbuttcap%
\pgfsetroundjoin%
\definecolor{currentfill}{rgb}{0.121569,0.466667,0.705882}%
\pgfsetfillcolor{currentfill}%
\pgfsetfillopacity{0.662269}%
\pgfsetlinewidth{1.003750pt}%
\definecolor{currentstroke}{rgb}{0.121569,0.466667,0.705882}%
\pgfsetstrokecolor{currentstroke}%
\pgfsetstrokeopacity{0.662269}%
\pgfsetdash{}{0pt}%
\pgfpathmoveto{\pgfqpoint{0.720804in}{1.218295in}}%
\pgfpathcurveto{\pgfqpoint{0.729041in}{1.218295in}}{\pgfqpoint{0.736941in}{1.221567in}}{\pgfqpoint{0.742765in}{1.227391in}}%
\pgfpathcurveto{\pgfqpoint{0.748588in}{1.233215in}}{\pgfqpoint{0.751861in}{1.241115in}}{\pgfqpoint{0.751861in}{1.249351in}}%
\pgfpathcurveto{\pgfqpoint{0.751861in}{1.257588in}}{\pgfqpoint{0.748588in}{1.265488in}}{\pgfqpoint{0.742765in}{1.271312in}}%
\pgfpathcurveto{\pgfqpoint{0.736941in}{1.277136in}}{\pgfqpoint{0.729041in}{1.280408in}}{\pgfqpoint{0.720804in}{1.280408in}}%
\pgfpathcurveto{\pgfqpoint{0.712568in}{1.280408in}}{\pgfqpoint{0.704668in}{1.277136in}}{\pgfqpoint{0.698844in}{1.271312in}}%
\pgfpathcurveto{\pgfqpoint{0.693020in}{1.265488in}}{\pgfqpoint{0.689748in}{1.257588in}}{\pgfqpoint{0.689748in}{1.249351in}}%
\pgfpathcurveto{\pgfqpoint{0.689748in}{1.241115in}}{\pgfqpoint{0.693020in}{1.233215in}}{\pgfqpoint{0.698844in}{1.227391in}}%
\pgfpathcurveto{\pgfqpoint{0.704668in}{1.221567in}}{\pgfqpoint{0.712568in}{1.218295in}}{\pgfqpoint{0.720804in}{1.218295in}}%
\pgfpathclose%
\pgfusepath{stroke,fill}%
\end{pgfscope}%
\begin{pgfscope}%
\pgfpathrectangle{\pgfqpoint{0.100000in}{0.212622in}}{\pgfqpoint{3.696000in}{3.696000in}}%
\pgfusepath{clip}%
\pgfsetbuttcap%
\pgfsetroundjoin%
\definecolor{currentfill}{rgb}{0.121569,0.466667,0.705882}%
\pgfsetfillcolor{currentfill}%
\pgfsetfillopacity{0.662283}%
\pgfsetlinewidth{1.003750pt}%
\definecolor{currentstroke}{rgb}{0.121569,0.466667,0.705882}%
\pgfsetstrokecolor{currentstroke}%
\pgfsetstrokeopacity{0.662283}%
\pgfsetdash{}{0pt}%
\pgfpathmoveto{\pgfqpoint{0.720771in}{1.218245in}}%
\pgfpathcurveto{\pgfqpoint{0.729007in}{1.218245in}}{\pgfqpoint{0.736907in}{1.221517in}}{\pgfqpoint{0.742731in}{1.227341in}}%
\pgfpathcurveto{\pgfqpoint{0.748555in}{1.233165in}}{\pgfqpoint{0.751827in}{1.241065in}}{\pgfqpoint{0.751827in}{1.249301in}}%
\pgfpathcurveto{\pgfqpoint{0.751827in}{1.257537in}}{\pgfqpoint{0.748555in}{1.265437in}}{\pgfqpoint{0.742731in}{1.271261in}}%
\pgfpathcurveto{\pgfqpoint{0.736907in}{1.277085in}}{\pgfqpoint{0.729007in}{1.280358in}}{\pgfqpoint{0.720771in}{1.280358in}}%
\pgfpathcurveto{\pgfqpoint{0.712534in}{1.280358in}}{\pgfqpoint{0.704634in}{1.277085in}}{\pgfqpoint{0.698810in}{1.271261in}}%
\pgfpathcurveto{\pgfqpoint{0.692987in}{1.265437in}}{\pgfqpoint{0.689714in}{1.257537in}}{\pgfqpoint{0.689714in}{1.249301in}}%
\pgfpathcurveto{\pgfqpoint{0.689714in}{1.241065in}}{\pgfqpoint{0.692987in}{1.233165in}}{\pgfqpoint{0.698810in}{1.227341in}}%
\pgfpathcurveto{\pgfqpoint{0.704634in}{1.221517in}}{\pgfqpoint{0.712534in}{1.218245in}}{\pgfqpoint{0.720771in}{1.218245in}}%
\pgfpathclose%
\pgfusepath{stroke,fill}%
\end{pgfscope}%
\begin{pgfscope}%
\pgfpathrectangle{\pgfqpoint{0.100000in}{0.212622in}}{\pgfqpoint{3.696000in}{3.696000in}}%
\pgfusepath{clip}%
\pgfsetbuttcap%
\pgfsetroundjoin%
\definecolor{currentfill}{rgb}{0.121569,0.466667,0.705882}%
\pgfsetfillcolor{currentfill}%
\pgfsetfillopacity{0.662307}%
\pgfsetlinewidth{1.003750pt}%
\definecolor{currentstroke}{rgb}{0.121569,0.466667,0.705882}%
\pgfsetstrokecolor{currentstroke}%
\pgfsetstrokeopacity{0.662307}%
\pgfsetdash{}{0pt}%
\pgfpathmoveto{\pgfqpoint{0.720708in}{1.218144in}}%
\pgfpathcurveto{\pgfqpoint{0.728944in}{1.218144in}}{\pgfqpoint{0.736844in}{1.221416in}}{\pgfqpoint{0.742668in}{1.227240in}}%
\pgfpathcurveto{\pgfqpoint{0.748492in}{1.233064in}}{\pgfqpoint{0.751764in}{1.240964in}}{\pgfqpoint{0.751764in}{1.249201in}}%
\pgfpathcurveto{\pgfqpoint{0.751764in}{1.257437in}}{\pgfqpoint{0.748492in}{1.265337in}}{\pgfqpoint{0.742668in}{1.271161in}}%
\pgfpathcurveto{\pgfqpoint{0.736844in}{1.276985in}}{\pgfqpoint{0.728944in}{1.280257in}}{\pgfqpoint{0.720708in}{1.280257in}}%
\pgfpathcurveto{\pgfqpoint{0.712472in}{1.280257in}}{\pgfqpoint{0.704571in}{1.276985in}}{\pgfqpoint{0.698748in}{1.271161in}}%
\pgfpathcurveto{\pgfqpoint{0.692924in}{1.265337in}}{\pgfqpoint{0.689651in}{1.257437in}}{\pgfqpoint{0.689651in}{1.249201in}}%
\pgfpathcurveto{\pgfqpoint{0.689651in}{1.240964in}}{\pgfqpoint{0.692924in}{1.233064in}}{\pgfqpoint{0.698748in}{1.227240in}}%
\pgfpathcurveto{\pgfqpoint{0.704571in}{1.221416in}}{\pgfqpoint{0.712472in}{1.218144in}}{\pgfqpoint{0.720708in}{1.218144in}}%
\pgfpathclose%
\pgfusepath{stroke,fill}%
\end{pgfscope}%
\begin{pgfscope}%
\pgfpathrectangle{\pgfqpoint{0.100000in}{0.212622in}}{\pgfqpoint{3.696000in}{3.696000in}}%
\pgfusepath{clip}%
\pgfsetbuttcap%
\pgfsetroundjoin%
\definecolor{currentfill}{rgb}{0.121569,0.466667,0.705882}%
\pgfsetfillcolor{currentfill}%
\pgfsetfillopacity{0.662349}%
\pgfsetlinewidth{1.003750pt}%
\definecolor{currentstroke}{rgb}{0.121569,0.466667,0.705882}%
\pgfsetstrokecolor{currentstroke}%
\pgfsetstrokeopacity{0.662349}%
\pgfsetdash{}{0pt}%
\pgfpathmoveto{\pgfqpoint{0.720574in}{1.217962in}}%
\pgfpathcurveto{\pgfqpoint{0.728811in}{1.217962in}}{\pgfqpoint{0.736711in}{1.221234in}}{\pgfqpoint{0.742535in}{1.227058in}}%
\pgfpathcurveto{\pgfqpoint{0.748359in}{1.232882in}}{\pgfqpoint{0.751631in}{1.240782in}}{\pgfqpoint{0.751631in}{1.249018in}}%
\pgfpathcurveto{\pgfqpoint{0.751631in}{1.257254in}}{\pgfqpoint{0.748359in}{1.265154in}}{\pgfqpoint{0.742535in}{1.270978in}}%
\pgfpathcurveto{\pgfqpoint{0.736711in}{1.276802in}}{\pgfqpoint{0.728811in}{1.280075in}}{\pgfqpoint{0.720574in}{1.280075in}}%
\pgfpathcurveto{\pgfqpoint{0.712338in}{1.280075in}}{\pgfqpoint{0.704438in}{1.276802in}}{\pgfqpoint{0.698614in}{1.270978in}}%
\pgfpathcurveto{\pgfqpoint{0.692790in}{1.265154in}}{\pgfqpoint{0.689518in}{1.257254in}}{\pgfqpoint{0.689518in}{1.249018in}}%
\pgfpathcurveto{\pgfqpoint{0.689518in}{1.240782in}}{\pgfqpoint{0.692790in}{1.232882in}}{\pgfqpoint{0.698614in}{1.227058in}}%
\pgfpathcurveto{\pgfqpoint{0.704438in}{1.221234in}}{\pgfqpoint{0.712338in}{1.217962in}}{\pgfqpoint{0.720574in}{1.217962in}}%
\pgfpathclose%
\pgfusepath{stroke,fill}%
\end{pgfscope}%
\begin{pgfscope}%
\pgfpathrectangle{\pgfqpoint{0.100000in}{0.212622in}}{\pgfqpoint{3.696000in}{3.696000in}}%
\pgfusepath{clip}%
\pgfsetbuttcap%
\pgfsetroundjoin%
\definecolor{currentfill}{rgb}{0.121569,0.466667,0.705882}%
\pgfsetfillcolor{currentfill}%
\pgfsetfillopacity{0.662427}%
\pgfsetlinewidth{1.003750pt}%
\definecolor{currentstroke}{rgb}{0.121569,0.466667,0.705882}%
\pgfsetstrokecolor{currentstroke}%
\pgfsetstrokeopacity{0.662427}%
\pgfsetdash{}{0pt}%
\pgfpathmoveto{\pgfqpoint{0.720371in}{1.217612in}}%
\pgfpathcurveto{\pgfqpoint{0.728608in}{1.217612in}}{\pgfqpoint{0.736508in}{1.220884in}}{\pgfqpoint{0.742332in}{1.226708in}}%
\pgfpathcurveto{\pgfqpoint{0.748156in}{1.232532in}}{\pgfqpoint{0.751428in}{1.240432in}}{\pgfqpoint{0.751428in}{1.248668in}}%
\pgfpathcurveto{\pgfqpoint{0.751428in}{1.256904in}}{\pgfqpoint{0.748156in}{1.264804in}}{\pgfqpoint{0.742332in}{1.270628in}}%
\pgfpathcurveto{\pgfqpoint{0.736508in}{1.276452in}}{\pgfqpoint{0.728608in}{1.279725in}}{\pgfqpoint{0.720371in}{1.279725in}}%
\pgfpathcurveto{\pgfqpoint{0.712135in}{1.279725in}}{\pgfqpoint{0.704235in}{1.276452in}}{\pgfqpoint{0.698411in}{1.270628in}}%
\pgfpathcurveto{\pgfqpoint{0.692587in}{1.264804in}}{\pgfqpoint{0.689315in}{1.256904in}}{\pgfqpoint{0.689315in}{1.248668in}}%
\pgfpathcurveto{\pgfqpoint{0.689315in}{1.240432in}}{\pgfqpoint{0.692587in}{1.232532in}}{\pgfqpoint{0.698411in}{1.226708in}}%
\pgfpathcurveto{\pgfqpoint{0.704235in}{1.220884in}}{\pgfqpoint{0.712135in}{1.217612in}}{\pgfqpoint{0.720371in}{1.217612in}}%
\pgfpathclose%
\pgfusepath{stroke,fill}%
\end{pgfscope}%
\begin{pgfscope}%
\pgfpathrectangle{\pgfqpoint{0.100000in}{0.212622in}}{\pgfqpoint{3.696000in}{3.696000in}}%
\pgfusepath{clip}%
\pgfsetbuttcap%
\pgfsetroundjoin%
\definecolor{currentfill}{rgb}{0.121569,0.466667,0.705882}%
\pgfsetfillcolor{currentfill}%
\pgfsetfillopacity{0.662466}%
\pgfsetlinewidth{1.003750pt}%
\definecolor{currentstroke}{rgb}{0.121569,0.466667,0.705882}%
\pgfsetstrokecolor{currentstroke}%
\pgfsetstrokeopacity{0.662466}%
\pgfsetdash{}{0pt}%
\pgfpathmoveto{\pgfqpoint{0.733294in}{1.214258in}}%
\pgfpathcurveto{\pgfqpoint{0.741530in}{1.214258in}}{\pgfqpoint{0.749430in}{1.217530in}}{\pgfqpoint{0.755254in}{1.223354in}}%
\pgfpathcurveto{\pgfqpoint{0.761078in}{1.229178in}}{\pgfqpoint{0.764350in}{1.237078in}}{\pgfqpoint{0.764350in}{1.245314in}}%
\pgfpathcurveto{\pgfqpoint{0.764350in}{1.253551in}}{\pgfqpoint{0.761078in}{1.261451in}}{\pgfqpoint{0.755254in}{1.267275in}}%
\pgfpathcurveto{\pgfqpoint{0.749430in}{1.273099in}}{\pgfqpoint{0.741530in}{1.276371in}}{\pgfqpoint{0.733294in}{1.276371in}}%
\pgfpathcurveto{\pgfqpoint{0.725058in}{1.276371in}}{\pgfqpoint{0.717158in}{1.273099in}}{\pgfqpoint{0.711334in}{1.267275in}}%
\pgfpathcurveto{\pgfqpoint{0.705510in}{1.261451in}}{\pgfqpoint{0.702237in}{1.253551in}}{\pgfqpoint{0.702237in}{1.245314in}}%
\pgfpathcurveto{\pgfqpoint{0.702237in}{1.237078in}}{\pgfqpoint{0.705510in}{1.229178in}}{\pgfqpoint{0.711334in}{1.223354in}}%
\pgfpathcurveto{\pgfqpoint{0.717158in}{1.217530in}}{\pgfqpoint{0.725058in}{1.214258in}}{\pgfqpoint{0.733294in}{1.214258in}}%
\pgfpathclose%
\pgfusepath{stroke,fill}%
\end{pgfscope}%
\begin{pgfscope}%
\pgfpathrectangle{\pgfqpoint{0.100000in}{0.212622in}}{\pgfqpoint{3.696000in}{3.696000in}}%
\pgfusepath{clip}%
\pgfsetbuttcap%
\pgfsetroundjoin%
\definecolor{currentfill}{rgb}{0.121569,0.466667,0.705882}%
\pgfsetfillcolor{currentfill}%
\pgfsetfillopacity{0.662571}%
\pgfsetlinewidth{1.003750pt}%
\definecolor{currentstroke}{rgb}{0.121569,0.466667,0.705882}%
\pgfsetstrokecolor{currentstroke}%
\pgfsetstrokeopacity{0.662571}%
\pgfsetdash{}{0pt}%
\pgfpathmoveto{\pgfqpoint{0.719935in}{1.217054in}}%
\pgfpathcurveto{\pgfqpoint{0.728171in}{1.217054in}}{\pgfqpoint{0.736071in}{1.220326in}}{\pgfqpoint{0.741895in}{1.226150in}}%
\pgfpathcurveto{\pgfqpoint{0.747719in}{1.231974in}}{\pgfqpoint{0.750991in}{1.239874in}}{\pgfqpoint{0.750991in}{1.248111in}}%
\pgfpathcurveto{\pgfqpoint{0.750991in}{1.256347in}}{\pgfqpoint{0.747719in}{1.264247in}}{\pgfqpoint{0.741895in}{1.270071in}}%
\pgfpathcurveto{\pgfqpoint{0.736071in}{1.275895in}}{\pgfqpoint{0.728171in}{1.279167in}}{\pgfqpoint{0.719935in}{1.279167in}}%
\pgfpathcurveto{\pgfqpoint{0.711698in}{1.279167in}}{\pgfqpoint{0.703798in}{1.275895in}}{\pgfqpoint{0.697974in}{1.270071in}}%
\pgfpathcurveto{\pgfqpoint{0.692151in}{1.264247in}}{\pgfqpoint{0.688878in}{1.256347in}}{\pgfqpoint{0.688878in}{1.248111in}}%
\pgfpathcurveto{\pgfqpoint{0.688878in}{1.239874in}}{\pgfqpoint{0.692151in}{1.231974in}}{\pgfqpoint{0.697974in}{1.226150in}}%
\pgfpathcurveto{\pgfqpoint{0.703798in}{1.220326in}}{\pgfqpoint{0.711698in}{1.217054in}}{\pgfqpoint{0.719935in}{1.217054in}}%
\pgfpathclose%
\pgfusepath{stroke,fill}%
\end{pgfscope}%
\begin{pgfscope}%
\pgfpathrectangle{\pgfqpoint{0.100000in}{0.212622in}}{\pgfqpoint{3.696000in}{3.696000in}}%
\pgfusepath{clip}%
\pgfsetbuttcap%
\pgfsetroundjoin%
\definecolor{currentfill}{rgb}{0.121569,0.466667,0.705882}%
\pgfsetfillcolor{currentfill}%
\pgfsetfillopacity{0.662832}%
\pgfsetlinewidth{1.003750pt}%
\definecolor{currentstroke}{rgb}{0.121569,0.466667,0.705882}%
\pgfsetstrokecolor{currentstroke}%
\pgfsetstrokeopacity{0.662832}%
\pgfsetdash{}{0pt}%
\pgfpathmoveto{\pgfqpoint{0.719143in}{1.216014in}}%
\pgfpathcurveto{\pgfqpoint{0.727380in}{1.216014in}}{\pgfqpoint{0.735280in}{1.219286in}}{\pgfqpoint{0.741104in}{1.225110in}}%
\pgfpathcurveto{\pgfqpoint{0.746928in}{1.230934in}}{\pgfqpoint{0.750200in}{1.238834in}}{\pgfqpoint{0.750200in}{1.247070in}}%
\pgfpathcurveto{\pgfqpoint{0.750200in}{1.255306in}}{\pgfqpoint{0.746928in}{1.263206in}}{\pgfqpoint{0.741104in}{1.269030in}}%
\pgfpathcurveto{\pgfqpoint{0.735280in}{1.274854in}}{\pgfqpoint{0.727380in}{1.278127in}}{\pgfqpoint{0.719143in}{1.278127in}}%
\pgfpathcurveto{\pgfqpoint{0.710907in}{1.278127in}}{\pgfqpoint{0.703007in}{1.274854in}}{\pgfqpoint{0.697183in}{1.269030in}}%
\pgfpathcurveto{\pgfqpoint{0.691359in}{1.263206in}}{\pgfqpoint{0.688087in}{1.255306in}}{\pgfqpoint{0.688087in}{1.247070in}}%
\pgfpathcurveto{\pgfqpoint{0.688087in}{1.238834in}}{\pgfqpoint{0.691359in}{1.230934in}}{\pgfqpoint{0.697183in}{1.225110in}}%
\pgfpathcurveto{\pgfqpoint{0.703007in}{1.219286in}}{\pgfqpoint{0.710907in}{1.216014in}}{\pgfqpoint{0.719143in}{1.216014in}}%
\pgfpathclose%
\pgfusepath{stroke,fill}%
\end{pgfscope}%
\begin{pgfscope}%
\pgfpathrectangle{\pgfqpoint{0.100000in}{0.212622in}}{\pgfqpoint{3.696000in}{3.696000in}}%
\pgfusepath{clip}%
\pgfsetbuttcap%
\pgfsetroundjoin%
\definecolor{currentfill}{rgb}{0.121569,0.466667,0.705882}%
\pgfsetfillcolor{currentfill}%
\pgfsetfillopacity{0.663107}%
\pgfsetlinewidth{1.003750pt}%
\definecolor{currentstroke}{rgb}{0.121569,0.466667,0.705882}%
\pgfsetstrokecolor{currentstroke}%
\pgfsetstrokeopacity{0.663107}%
\pgfsetdash{}{0pt}%
\pgfpathmoveto{\pgfqpoint{0.730443in}{1.213473in}}%
\pgfpathcurveto{\pgfqpoint{0.738679in}{1.213473in}}{\pgfqpoint{0.746579in}{1.216745in}}{\pgfqpoint{0.752403in}{1.222569in}}%
\pgfpathcurveto{\pgfqpoint{0.758227in}{1.228393in}}{\pgfqpoint{0.761499in}{1.236293in}}{\pgfqpoint{0.761499in}{1.244530in}}%
\pgfpathcurveto{\pgfqpoint{0.761499in}{1.252766in}}{\pgfqpoint{0.758227in}{1.260666in}}{\pgfqpoint{0.752403in}{1.266490in}}%
\pgfpathcurveto{\pgfqpoint{0.746579in}{1.272314in}}{\pgfqpoint{0.738679in}{1.275586in}}{\pgfqpoint{0.730443in}{1.275586in}}%
\pgfpathcurveto{\pgfqpoint{0.722206in}{1.275586in}}{\pgfqpoint{0.714306in}{1.272314in}}{\pgfqpoint{0.708482in}{1.266490in}}%
\pgfpathcurveto{\pgfqpoint{0.702659in}{1.260666in}}{\pgfqpoint{0.699386in}{1.252766in}}{\pgfqpoint{0.699386in}{1.244530in}}%
\pgfpathcurveto{\pgfqpoint{0.699386in}{1.236293in}}{\pgfqpoint{0.702659in}{1.228393in}}{\pgfqpoint{0.708482in}{1.222569in}}%
\pgfpathcurveto{\pgfqpoint{0.714306in}{1.216745in}}{\pgfqpoint{0.722206in}{1.213473in}}{\pgfqpoint{0.730443in}{1.213473in}}%
\pgfpathclose%
\pgfusepath{stroke,fill}%
\end{pgfscope}%
\begin{pgfscope}%
\pgfpathrectangle{\pgfqpoint{0.100000in}{0.212622in}}{\pgfqpoint{3.696000in}{3.696000in}}%
\pgfusepath{clip}%
\pgfsetbuttcap%
\pgfsetroundjoin%
\definecolor{currentfill}{rgb}{0.121569,0.466667,0.705882}%
\pgfsetfillcolor{currentfill}%
\pgfsetfillopacity{0.663327}%
\pgfsetlinewidth{1.003750pt}%
\definecolor{currentstroke}{rgb}{0.121569,0.466667,0.705882}%
\pgfsetstrokecolor{currentstroke}%
\pgfsetstrokeopacity{0.663327}%
\pgfsetdash{}{0pt}%
\pgfpathmoveto{\pgfqpoint{0.717884in}{1.214075in}}%
\pgfpathcurveto{\pgfqpoint{0.726120in}{1.214075in}}{\pgfqpoint{0.734020in}{1.217348in}}{\pgfqpoint{0.739844in}{1.223172in}}%
\pgfpathcurveto{\pgfqpoint{0.745668in}{1.228996in}}{\pgfqpoint{0.748941in}{1.236896in}}{\pgfqpoint{0.748941in}{1.245132in}}%
\pgfpathcurveto{\pgfqpoint{0.748941in}{1.253368in}}{\pgfqpoint{0.745668in}{1.261268in}}{\pgfqpoint{0.739844in}{1.267092in}}%
\pgfpathcurveto{\pgfqpoint{0.734020in}{1.272916in}}{\pgfqpoint{0.726120in}{1.276188in}}{\pgfqpoint{0.717884in}{1.276188in}}%
\pgfpathcurveto{\pgfqpoint{0.709648in}{1.276188in}}{\pgfqpoint{0.701748in}{1.272916in}}{\pgfqpoint{0.695924in}{1.267092in}}%
\pgfpathcurveto{\pgfqpoint{0.690100in}{1.261268in}}{\pgfqpoint{0.686828in}{1.253368in}}{\pgfqpoint{0.686828in}{1.245132in}}%
\pgfpathcurveto{\pgfqpoint{0.686828in}{1.236896in}}{\pgfqpoint{0.690100in}{1.228996in}}{\pgfqpoint{0.695924in}{1.223172in}}%
\pgfpathcurveto{\pgfqpoint{0.701748in}{1.217348in}}{\pgfqpoint{0.709648in}{1.214075in}}{\pgfqpoint{0.717884in}{1.214075in}}%
\pgfpathclose%
\pgfusepath{stroke,fill}%
\end{pgfscope}%
\begin{pgfscope}%
\pgfpathrectangle{\pgfqpoint{0.100000in}{0.212622in}}{\pgfqpoint{3.696000in}{3.696000in}}%
\pgfusepath{clip}%
\pgfsetbuttcap%
\pgfsetroundjoin%
\definecolor{currentfill}{rgb}{0.121569,0.466667,0.705882}%
\pgfsetfillcolor{currentfill}%
\pgfsetfillopacity{0.663330}%
\pgfsetlinewidth{1.003750pt}%
\definecolor{currentstroke}{rgb}{0.121569,0.466667,0.705882}%
\pgfsetstrokecolor{currentstroke}%
\pgfsetstrokeopacity{0.663330}%
\pgfsetdash{}{0pt}%
\pgfpathmoveto{\pgfqpoint{0.717875in}{1.214060in}}%
\pgfpathcurveto{\pgfqpoint{0.726111in}{1.214060in}}{\pgfqpoint{0.734011in}{1.217332in}}{\pgfqpoint{0.739835in}{1.223156in}}%
\pgfpathcurveto{\pgfqpoint{0.745659in}{1.228980in}}{\pgfqpoint{0.748931in}{1.236880in}}{\pgfqpoint{0.748931in}{1.245116in}}%
\pgfpathcurveto{\pgfqpoint{0.748931in}{1.253352in}}{\pgfqpoint{0.745659in}{1.261252in}}{\pgfqpoint{0.739835in}{1.267076in}}%
\pgfpathcurveto{\pgfqpoint{0.734011in}{1.272900in}}{\pgfqpoint{0.726111in}{1.276173in}}{\pgfqpoint{0.717875in}{1.276173in}}%
\pgfpathcurveto{\pgfqpoint{0.709639in}{1.276173in}}{\pgfqpoint{0.701739in}{1.272900in}}{\pgfqpoint{0.695915in}{1.267076in}}%
\pgfpathcurveto{\pgfqpoint{0.690091in}{1.261252in}}{\pgfqpoint{0.686818in}{1.253352in}}{\pgfqpoint{0.686818in}{1.245116in}}%
\pgfpathcurveto{\pgfqpoint{0.686818in}{1.236880in}}{\pgfqpoint{0.690091in}{1.228980in}}{\pgfqpoint{0.695915in}{1.223156in}}%
\pgfpathcurveto{\pgfqpoint{0.701739in}{1.217332in}}{\pgfqpoint{0.709639in}{1.214060in}}{\pgfqpoint{0.717875in}{1.214060in}}%
\pgfpathclose%
\pgfusepath{stroke,fill}%
\end{pgfscope}%
\begin{pgfscope}%
\pgfpathrectangle{\pgfqpoint{0.100000in}{0.212622in}}{\pgfqpoint{3.696000in}{3.696000in}}%
\pgfusepath{clip}%
\pgfsetbuttcap%
\pgfsetroundjoin%
\definecolor{currentfill}{rgb}{0.121569,0.466667,0.705882}%
\pgfsetfillcolor{currentfill}%
\pgfsetfillopacity{0.663337}%
\pgfsetlinewidth{1.003750pt}%
\definecolor{currentstroke}{rgb}{0.121569,0.466667,0.705882}%
\pgfsetstrokecolor{currentstroke}%
\pgfsetstrokeopacity{0.663337}%
\pgfsetdash{}{0pt}%
\pgfpathmoveto{\pgfqpoint{0.717858in}{1.214035in}}%
\pgfpathcurveto{\pgfqpoint{0.726094in}{1.214035in}}{\pgfqpoint{0.733994in}{1.217307in}}{\pgfqpoint{0.739818in}{1.223131in}}%
\pgfpathcurveto{\pgfqpoint{0.745642in}{1.228955in}}{\pgfqpoint{0.748914in}{1.236855in}}{\pgfqpoint{0.748914in}{1.245091in}}%
\pgfpathcurveto{\pgfqpoint{0.748914in}{1.253328in}}{\pgfqpoint{0.745642in}{1.261228in}}{\pgfqpoint{0.739818in}{1.267052in}}%
\pgfpathcurveto{\pgfqpoint{0.733994in}{1.272875in}}{\pgfqpoint{0.726094in}{1.276148in}}{\pgfqpoint{0.717858in}{1.276148in}}%
\pgfpathcurveto{\pgfqpoint{0.709621in}{1.276148in}}{\pgfqpoint{0.701721in}{1.272875in}}{\pgfqpoint{0.695897in}{1.267052in}}%
\pgfpathcurveto{\pgfqpoint{0.690073in}{1.261228in}}{\pgfqpoint{0.686801in}{1.253328in}}{\pgfqpoint{0.686801in}{1.245091in}}%
\pgfpathcurveto{\pgfqpoint{0.686801in}{1.236855in}}{\pgfqpoint{0.690073in}{1.228955in}}{\pgfqpoint{0.695897in}{1.223131in}}%
\pgfpathcurveto{\pgfqpoint{0.701721in}{1.217307in}}{\pgfqpoint{0.709621in}{1.214035in}}{\pgfqpoint{0.717858in}{1.214035in}}%
\pgfpathclose%
\pgfusepath{stroke,fill}%
\end{pgfscope}%
\begin{pgfscope}%
\pgfpathrectangle{\pgfqpoint{0.100000in}{0.212622in}}{\pgfqpoint{3.696000in}{3.696000in}}%
\pgfusepath{clip}%
\pgfsetbuttcap%
\pgfsetroundjoin%
\definecolor{currentfill}{rgb}{0.121569,0.466667,0.705882}%
\pgfsetfillcolor{currentfill}%
\pgfsetfillopacity{0.663350}%
\pgfsetlinewidth{1.003750pt}%
\definecolor{currentstroke}{rgb}{0.121569,0.466667,0.705882}%
\pgfsetstrokecolor{currentstroke}%
\pgfsetstrokeopacity{0.663350}%
\pgfsetdash{}{0pt}%
\pgfpathmoveto{\pgfqpoint{0.717845in}{1.213984in}}%
\pgfpathcurveto{\pgfqpoint{0.726081in}{1.213984in}}{\pgfqpoint{0.733981in}{1.217256in}}{\pgfqpoint{0.739805in}{1.223080in}}%
\pgfpathcurveto{\pgfqpoint{0.745629in}{1.228904in}}{\pgfqpoint{0.748901in}{1.236804in}}{\pgfqpoint{0.748901in}{1.245040in}}%
\pgfpathcurveto{\pgfqpoint{0.748901in}{1.253277in}}{\pgfqpoint{0.745629in}{1.261177in}}{\pgfqpoint{0.739805in}{1.267001in}}%
\pgfpathcurveto{\pgfqpoint{0.733981in}{1.272824in}}{\pgfqpoint{0.726081in}{1.276097in}}{\pgfqpoint{0.717845in}{1.276097in}}%
\pgfpathcurveto{\pgfqpoint{0.709609in}{1.276097in}}{\pgfqpoint{0.701709in}{1.272824in}}{\pgfqpoint{0.695885in}{1.267001in}}%
\pgfpathcurveto{\pgfqpoint{0.690061in}{1.261177in}}{\pgfqpoint{0.686788in}{1.253277in}}{\pgfqpoint{0.686788in}{1.245040in}}%
\pgfpathcurveto{\pgfqpoint{0.686788in}{1.236804in}}{\pgfqpoint{0.690061in}{1.228904in}}{\pgfqpoint{0.695885in}{1.223080in}}%
\pgfpathcurveto{\pgfqpoint{0.701709in}{1.217256in}}{\pgfqpoint{0.709609in}{1.213984in}}{\pgfqpoint{0.717845in}{1.213984in}}%
\pgfpathclose%
\pgfusepath{stroke,fill}%
\end{pgfscope}%
\begin{pgfscope}%
\pgfpathrectangle{\pgfqpoint{0.100000in}{0.212622in}}{\pgfqpoint{3.696000in}{3.696000in}}%
\pgfusepath{clip}%
\pgfsetbuttcap%
\pgfsetroundjoin%
\definecolor{currentfill}{rgb}{0.121569,0.466667,0.705882}%
\pgfsetfillcolor{currentfill}%
\pgfsetfillopacity{0.663376}%
\pgfsetlinewidth{1.003750pt}%
\definecolor{currentstroke}{rgb}{0.121569,0.466667,0.705882}%
\pgfsetstrokecolor{currentstroke}%
\pgfsetstrokeopacity{0.663376}%
\pgfsetdash{}{0pt}%
\pgfpathmoveto{\pgfqpoint{0.717860in}{1.213914in}}%
\pgfpathcurveto{\pgfqpoint{0.726096in}{1.213914in}}{\pgfqpoint{0.733997in}{1.217186in}}{\pgfqpoint{0.739820in}{1.223010in}}%
\pgfpathcurveto{\pgfqpoint{0.745644in}{1.228834in}}{\pgfqpoint{0.748917in}{1.236734in}}{\pgfqpoint{0.748917in}{1.244971in}}%
\pgfpathcurveto{\pgfqpoint{0.748917in}{1.253207in}}{\pgfqpoint{0.745644in}{1.261107in}}{\pgfqpoint{0.739820in}{1.266931in}}%
\pgfpathcurveto{\pgfqpoint{0.733997in}{1.272755in}}{\pgfqpoint{0.726096in}{1.276027in}}{\pgfqpoint{0.717860in}{1.276027in}}%
\pgfpathcurveto{\pgfqpoint{0.709624in}{1.276027in}}{\pgfqpoint{0.701724in}{1.272755in}}{\pgfqpoint{0.695900in}{1.266931in}}%
\pgfpathcurveto{\pgfqpoint{0.690076in}{1.261107in}}{\pgfqpoint{0.686804in}{1.253207in}}{\pgfqpoint{0.686804in}{1.244971in}}%
\pgfpathcurveto{\pgfqpoint{0.686804in}{1.236734in}}{\pgfqpoint{0.690076in}{1.228834in}}{\pgfqpoint{0.695900in}{1.223010in}}%
\pgfpathcurveto{\pgfqpoint{0.701724in}{1.217186in}}{\pgfqpoint{0.709624in}{1.213914in}}{\pgfqpoint{0.717860in}{1.213914in}}%
\pgfpathclose%
\pgfusepath{stroke,fill}%
\end{pgfscope}%
\begin{pgfscope}%
\pgfpathrectangle{\pgfqpoint{0.100000in}{0.212622in}}{\pgfqpoint{3.696000in}{3.696000in}}%
\pgfusepath{clip}%
\pgfsetbuttcap%
\pgfsetroundjoin%
\definecolor{currentfill}{rgb}{0.121569,0.466667,0.705882}%
\pgfsetfillcolor{currentfill}%
\pgfsetfillopacity{0.663414}%
\pgfsetlinewidth{1.003750pt}%
\definecolor{currentstroke}{rgb}{0.121569,0.466667,0.705882}%
\pgfsetstrokecolor{currentstroke}%
\pgfsetstrokeopacity{0.663414}%
\pgfsetdash{}{0pt}%
\pgfpathmoveto{\pgfqpoint{0.717963in}{1.213793in}}%
\pgfpathcurveto{\pgfqpoint{0.726199in}{1.213793in}}{\pgfqpoint{0.734099in}{1.217066in}}{\pgfqpoint{0.739923in}{1.222890in}}%
\pgfpathcurveto{\pgfqpoint{0.745747in}{1.228713in}}{\pgfqpoint{0.749020in}{1.236614in}}{\pgfqpoint{0.749020in}{1.244850in}}%
\pgfpathcurveto{\pgfqpoint{0.749020in}{1.253086in}}{\pgfqpoint{0.745747in}{1.260986in}}{\pgfqpoint{0.739923in}{1.266810in}}%
\pgfpathcurveto{\pgfqpoint{0.734099in}{1.272634in}}{\pgfqpoint{0.726199in}{1.275906in}}{\pgfqpoint{0.717963in}{1.275906in}}%
\pgfpathcurveto{\pgfqpoint{0.709727in}{1.275906in}}{\pgfqpoint{0.701827in}{1.272634in}}{\pgfqpoint{0.696003in}{1.266810in}}%
\pgfpathcurveto{\pgfqpoint{0.690179in}{1.260986in}}{\pgfqpoint{0.686907in}{1.253086in}}{\pgfqpoint{0.686907in}{1.244850in}}%
\pgfpathcurveto{\pgfqpoint{0.686907in}{1.236614in}}{\pgfqpoint{0.690179in}{1.228713in}}{\pgfqpoint{0.696003in}{1.222890in}}%
\pgfpathcurveto{\pgfqpoint{0.701827in}{1.217066in}}{\pgfqpoint{0.709727in}{1.213793in}}{\pgfqpoint{0.717963in}{1.213793in}}%
\pgfpathclose%
\pgfusepath{stroke,fill}%
\end{pgfscope}%
\begin{pgfscope}%
\pgfpathrectangle{\pgfqpoint{0.100000in}{0.212622in}}{\pgfqpoint{3.696000in}{3.696000in}}%
\pgfusepath{clip}%
\pgfsetbuttcap%
\pgfsetroundjoin%
\definecolor{currentfill}{rgb}{0.121569,0.466667,0.705882}%
\pgfsetfillcolor{currentfill}%
\pgfsetfillopacity{0.663418}%
\pgfsetlinewidth{1.003750pt}%
\definecolor{currentstroke}{rgb}{0.121569,0.466667,0.705882}%
\pgfsetstrokecolor{currentstroke}%
\pgfsetstrokeopacity{0.663418}%
\pgfsetdash{}{0pt}%
\pgfpathmoveto{\pgfqpoint{0.720189in}{1.213014in}}%
\pgfpathcurveto{\pgfqpoint{0.728425in}{1.213014in}}{\pgfqpoint{0.736326in}{1.216287in}}{\pgfqpoint{0.742149in}{1.222110in}}%
\pgfpathcurveto{\pgfqpoint{0.747973in}{1.227934in}}{\pgfqpoint{0.751246in}{1.235834in}}{\pgfqpoint{0.751246in}{1.244071in}}%
\pgfpathcurveto{\pgfqpoint{0.751246in}{1.252307in}}{\pgfqpoint{0.747973in}{1.260207in}}{\pgfqpoint{0.742149in}{1.266031in}}%
\pgfpathcurveto{\pgfqpoint{0.736326in}{1.271855in}}{\pgfqpoint{0.728425in}{1.275127in}}{\pgfqpoint{0.720189in}{1.275127in}}%
\pgfpathcurveto{\pgfqpoint{0.711953in}{1.275127in}}{\pgfqpoint{0.704053in}{1.271855in}}{\pgfqpoint{0.698229in}{1.266031in}}%
\pgfpathcurveto{\pgfqpoint{0.692405in}{1.260207in}}{\pgfqpoint{0.689133in}{1.252307in}}{\pgfqpoint{0.689133in}{1.244071in}}%
\pgfpathcurveto{\pgfqpoint{0.689133in}{1.235834in}}{\pgfqpoint{0.692405in}{1.227934in}}{\pgfqpoint{0.698229in}{1.222110in}}%
\pgfpathcurveto{\pgfqpoint{0.704053in}{1.216287in}}{\pgfqpoint{0.711953in}{1.213014in}}{\pgfqpoint{0.720189in}{1.213014in}}%
\pgfpathclose%
\pgfusepath{stroke,fill}%
\end{pgfscope}%
\begin{pgfscope}%
\pgfpathrectangle{\pgfqpoint{0.100000in}{0.212622in}}{\pgfqpoint{3.696000in}{3.696000in}}%
\pgfusepath{clip}%
\pgfsetbuttcap%
\pgfsetroundjoin%
\definecolor{currentfill}{rgb}{0.121569,0.466667,0.705882}%
\pgfsetfillcolor{currentfill}%
\pgfsetfillopacity{0.663446}%
\pgfsetlinewidth{1.003750pt}%
\definecolor{currentstroke}{rgb}{0.121569,0.466667,0.705882}%
\pgfsetstrokecolor{currentstroke}%
\pgfsetstrokeopacity{0.663446}%
\pgfsetdash{}{0pt}%
\pgfpathmoveto{\pgfqpoint{0.728826in}{1.213083in}}%
\pgfpathcurveto{\pgfqpoint{0.737062in}{1.213083in}}{\pgfqpoint{0.744962in}{1.216355in}}{\pgfqpoint{0.750786in}{1.222179in}}%
\pgfpathcurveto{\pgfqpoint{0.756610in}{1.228003in}}{\pgfqpoint{0.759882in}{1.235903in}}{\pgfqpoint{0.759882in}{1.244140in}}%
\pgfpathcurveto{\pgfqpoint{0.759882in}{1.252376in}}{\pgfqpoint{0.756610in}{1.260276in}}{\pgfqpoint{0.750786in}{1.266100in}}%
\pgfpathcurveto{\pgfqpoint{0.744962in}{1.271924in}}{\pgfqpoint{0.737062in}{1.275196in}}{\pgfqpoint{0.728826in}{1.275196in}}%
\pgfpathcurveto{\pgfqpoint{0.720589in}{1.275196in}}{\pgfqpoint{0.712689in}{1.271924in}}{\pgfqpoint{0.706865in}{1.266100in}}%
\pgfpathcurveto{\pgfqpoint{0.701042in}{1.260276in}}{\pgfqpoint{0.697769in}{1.252376in}}{\pgfqpoint{0.697769in}{1.244140in}}%
\pgfpathcurveto{\pgfqpoint{0.697769in}{1.235903in}}{\pgfqpoint{0.701042in}{1.228003in}}{\pgfqpoint{0.706865in}{1.222179in}}%
\pgfpathcurveto{\pgfqpoint{0.712689in}{1.216355in}}{\pgfqpoint{0.720589in}{1.213083in}}{\pgfqpoint{0.728826in}{1.213083in}}%
\pgfpathclose%
\pgfusepath{stroke,fill}%
\end{pgfscope}%
\begin{pgfscope}%
\pgfpathrectangle{\pgfqpoint{0.100000in}{0.212622in}}{\pgfqpoint{3.696000in}{3.696000in}}%
\pgfusepath{clip}%
\pgfsetbuttcap%
\pgfsetroundjoin%
\definecolor{currentfill}{rgb}{0.121569,0.466667,0.705882}%
\pgfsetfillcolor{currentfill}%
\pgfsetfillopacity{0.663472}%
\pgfsetlinewidth{1.003750pt}%
\definecolor{currentstroke}{rgb}{0.121569,0.466667,0.705882}%
\pgfsetstrokecolor{currentstroke}%
\pgfsetstrokeopacity{0.663472}%
\pgfsetdash{}{0pt}%
\pgfpathmoveto{\pgfqpoint{0.718219in}{1.213611in}}%
\pgfpathcurveto{\pgfqpoint{0.726455in}{1.213611in}}{\pgfqpoint{0.734355in}{1.216884in}}{\pgfqpoint{0.740179in}{1.222707in}}%
\pgfpathcurveto{\pgfqpoint{0.746003in}{1.228531in}}{\pgfqpoint{0.749275in}{1.236431in}}{\pgfqpoint{0.749275in}{1.244668in}}%
\pgfpathcurveto{\pgfqpoint{0.749275in}{1.252904in}}{\pgfqpoint{0.746003in}{1.260804in}}{\pgfqpoint{0.740179in}{1.266628in}}%
\pgfpathcurveto{\pgfqpoint{0.734355in}{1.272452in}}{\pgfqpoint{0.726455in}{1.275724in}}{\pgfqpoint{0.718219in}{1.275724in}}%
\pgfpathcurveto{\pgfqpoint{0.709983in}{1.275724in}}{\pgfqpoint{0.702082in}{1.272452in}}{\pgfqpoint{0.696259in}{1.266628in}}%
\pgfpathcurveto{\pgfqpoint{0.690435in}{1.260804in}}{\pgfqpoint{0.687162in}{1.252904in}}{\pgfqpoint{0.687162in}{1.244668in}}%
\pgfpathcurveto{\pgfqpoint{0.687162in}{1.236431in}}{\pgfqpoint{0.690435in}{1.228531in}}{\pgfqpoint{0.696259in}{1.222707in}}%
\pgfpathcurveto{\pgfqpoint{0.702082in}{1.216884in}}{\pgfqpoint{0.709983in}{1.213611in}}{\pgfqpoint{0.718219in}{1.213611in}}%
\pgfpathclose%
\pgfusepath{stroke,fill}%
\end{pgfscope}%
\begin{pgfscope}%
\pgfpathrectangle{\pgfqpoint{0.100000in}{0.212622in}}{\pgfqpoint{3.696000in}{3.696000in}}%
\pgfusepath{clip}%
\pgfsetbuttcap%
\pgfsetroundjoin%
\definecolor{currentfill}{rgb}{0.121569,0.466667,0.705882}%
\pgfsetfillcolor{currentfill}%
\pgfsetfillopacity{0.663537}%
\pgfsetlinewidth{1.003750pt}%
\definecolor{currentstroke}{rgb}{0.121569,0.466667,0.705882}%
\pgfsetstrokecolor{currentstroke}%
\pgfsetstrokeopacity{0.663537}%
\pgfsetdash{}{0pt}%
\pgfpathmoveto{\pgfqpoint{0.718872in}{1.213618in}}%
\pgfpathcurveto{\pgfqpoint{0.727109in}{1.213618in}}{\pgfqpoint{0.735009in}{1.216890in}}{\pgfqpoint{0.740833in}{1.222714in}}%
\pgfpathcurveto{\pgfqpoint{0.746657in}{1.228538in}}{\pgfqpoint{0.749929in}{1.236438in}}{\pgfqpoint{0.749929in}{1.244675in}}%
\pgfpathcurveto{\pgfqpoint{0.749929in}{1.252911in}}{\pgfqpoint{0.746657in}{1.260811in}}{\pgfqpoint{0.740833in}{1.266635in}}%
\pgfpathcurveto{\pgfqpoint{0.735009in}{1.272459in}}{\pgfqpoint{0.727109in}{1.275731in}}{\pgfqpoint{0.718872in}{1.275731in}}%
\pgfpathcurveto{\pgfqpoint{0.710636in}{1.275731in}}{\pgfqpoint{0.702736in}{1.272459in}}{\pgfqpoint{0.696912in}{1.266635in}}%
\pgfpathcurveto{\pgfqpoint{0.691088in}{1.260811in}}{\pgfqpoint{0.687816in}{1.252911in}}{\pgfqpoint{0.687816in}{1.244675in}}%
\pgfpathcurveto{\pgfqpoint{0.687816in}{1.236438in}}{\pgfqpoint{0.691088in}{1.228538in}}{\pgfqpoint{0.696912in}{1.222714in}}%
\pgfpathcurveto{\pgfqpoint{0.702736in}{1.216890in}}{\pgfqpoint{0.710636in}{1.213618in}}{\pgfqpoint{0.718872in}{1.213618in}}%
\pgfpathclose%
\pgfusepath{stroke,fill}%
\end{pgfscope}%
\begin{pgfscope}%
\pgfpathrectangle{\pgfqpoint{0.100000in}{0.212622in}}{\pgfqpoint{3.696000in}{3.696000in}}%
\pgfusepath{clip}%
\pgfsetbuttcap%
\pgfsetroundjoin%
\definecolor{currentfill}{rgb}{0.121569,0.466667,0.705882}%
\pgfsetfillcolor{currentfill}%
\pgfsetfillopacity{0.663646}%
\pgfsetlinewidth{1.003750pt}%
\definecolor{currentstroke}{rgb}{0.121569,0.466667,0.705882}%
\pgfsetstrokecolor{currentstroke}%
\pgfsetstrokeopacity{0.663646}%
\pgfsetdash{}{0pt}%
\pgfpathmoveto{\pgfqpoint{0.727967in}{1.212867in}}%
\pgfpathcurveto{\pgfqpoint{0.736203in}{1.212867in}}{\pgfqpoint{0.744103in}{1.216139in}}{\pgfqpoint{0.749927in}{1.221963in}}%
\pgfpathcurveto{\pgfqpoint{0.755751in}{1.227787in}}{\pgfqpoint{0.759023in}{1.235687in}}{\pgfqpoint{0.759023in}{1.243923in}}%
\pgfpathcurveto{\pgfqpoint{0.759023in}{1.252160in}}{\pgfqpoint{0.755751in}{1.260060in}}{\pgfqpoint{0.749927in}{1.265884in}}%
\pgfpathcurveto{\pgfqpoint{0.744103in}{1.271708in}}{\pgfqpoint{0.736203in}{1.274980in}}{\pgfqpoint{0.727967in}{1.274980in}}%
\pgfpathcurveto{\pgfqpoint{0.719730in}{1.274980in}}{\pgfqpoint{0.711830in}{1.271708in}}{\pgfqpoint{0.706006in}{1.265884in}}%
\pgfpathcurveto{\pgfqpoint{0.700182in}{1.260060in}}{\pgfqpoint{0.696910in}{1.252160in}}{\pgfqpoint{0.696910in}{1.243923in}}%
\pgfpathcurveto{\pgfqpoint{0.696910in}{1.235687in}}{\pgfqpoint{0.700182in}{1.227787in}}{\pgfqpoint{0.706006in}{1.221963in}}%
\pgfpathcurveto{\pgfqpoint{0.711830in}{1.216139in}}{\pgfqpoint{0.719730in}{1.212867in}}{\pgfqpoint{0.727967in}{1.212867in}}%
\pgfpathclose%
\pgfusepath{stroke,fill}%
\end{pgfscope}%
\begin{pgfscope}%
\pgfpathrectangle{\pgfqpoint{0.100000in}{0.212622in}}{\pgfqpoint{3.696000in}{3.696000in}}%
\pgfusepath{clip}%
\pgfsetbuttcap%
\pgfsetroundjoin%
\definecolor{currentfill}{rgb}{0.121569,0.466667,0.705882}%
\pgfsetfillcolor{currentfill}%
\pgfsetfillopacity{0.663687}%
\pgfsetlinewidth{1.003750pt}%
\definecolor{currentstroke}{rgb}{0.121569,0.466667,0.705882}%
\pgfsetstrokecolor{currentstroke}%
\pgfsetstrokeopacity{0.663687}%
\pgfsetdash{}{0pt}%
\pgfpathmoveto{\pgfqpoint{0.724743in}{1.212308in}}%
\pgfpathcurveto{\pgfqpoint{0.732979in}{1.212308in}}{\pgfqpoint{0.740879in}{1.215580in}}{\pgfqpoint{0.746703in}{1.221404in}}%
\pgfpathcurveto{\pgfqpoint{0.752527in}{1.227228in}}{\pgfqpoint{0.755799in}{1.235128in}}{\pgfqpoint{0.755799in}{1.243364in}}%
\pgfpathcurveto{\pgfqpoint{0.755799in}{1.251601in}}{\pgfqpoint{0.752527in}{1.259501in}}{\pgfqpoint{0.746703in}{1.265325in}}%
\pgfpathcurveto{\pgfqpoint{0.740879in}{1.271149in}}{\pgfqpoint{0.732979in}{1.274421in}}{\pgfqpoint{0.724743in}{1.274421in}}%
\pgfpathcurveto{\pgfqpoint{0.716507in}{1.274421in}}{\pgfqpoint{0.708607in}{1.271149in}}{\pgfqpoint{0.702783in}{1.265325in}}%
\pgfpathcurveto{\pgfqpoint{0.696959in}{1.259501in}}{\pgfqpoint{0.693686in}{1.251601in}}{\pgfqpoint{0.693686in}{1.243364in}}%
\pgfpathcurveto{\pgfqpoint{0.693686in}{1.235128in}}{\pgfqpoint{0.696959in}{1.227228in}}{\pgfqpoint{0.702783in}{1.221404in}}%
\pgfpathcurveto{\pgfqpoint{0.708607in}{1.215580in}}{\pgfqpoint{0.716507in}{1.212308in}}{\pgfqpoint{0.724743in}{1.212308in}}%
\pgfpathclose%
\pgfusepath{stroke,fill}%
\end{pgfscope}%
\begin{pgfscope}%
\pgfpathrectangle{\pgfqpoint{0.100000in}{0.212622in}}{\pgfqpoint{3.696000in}{3.696000in}}%
\pgfusepath{clip}%
\pgfsetbuttcap%
\pgfsetroundjoin%
\definecolor{currentfill}{rgb}{0.121569,0.466667,0.705882}%
\pgfsetfillcolor{currentfill}%
\pgfsetfillopacity{0.663750}%
\pgfsetlinewidth{1.003750pt}%
\definecolor{currentstroke}{rgb}{0.121569,0.466667,0.705882}%
\pgfsetstrokecolor{currentstroke}%
\pgfsetstrokeopacity{0.663750}%
\pgfsetdash{}{0pt}%
\pgfpathmoveto{\pgfqpoint{0.727473in}{1.212767in}}%
\pgfpathcurveto{\pgfqpoint{0.735710in}{1.212767in}}{\pgfqpoint{0.743610in}{1.216039in}}{\pgfqpoint{0.749434in}{1.221863in}}%
\pgfpathcurveto{\pgfqpoint{0.755258in}{1.227687in}}{\pgfqpoint{0.758530in}{1.235587in}}{\pgfqpoint{0.758530in}{1.243823in}}%
\pgfpathcurveto{\pgfqpoint{0.758530in}{1.252060in}}{\pgfqpoint{0.755258in}{1.259960in}}{\pgfqpoint{0.749434in}{1.265784in}}%
\pgfpathcurveto{\pgfqpoint{0.743610in}{1.271608in}}{\pgfqpoint{0.735710in}{1.274880in}}{\pgfqpoint{0.727473in}{1.274880in}}%
\pgfpathcurveto{\pgfqpoint{0.719237in}{1.274880in}}{\pgfqpoint{0.711337in}{1.271608in}}{\pgfqpoint{0.705513in}{1.265784in}}%
\pgfpathcurveto{\pgfqpoint{0.699689in}{1.259960in}}{\pgfqpoint{0.696417in}{1.252060in}}{\pgfqpoint{0.696417in}{1.243823in}}%
\pgfpathcurveto{\pgfqpoint{0.696417in}{1.235587in}}{\pgfqpoint{0.699689in}{1.227687in}}{\pgfqpoint{0.705513in}{1.221863in}}%
\pgfpathcurveto{\pgfqpoint{0.711337in}{1.216039in}}{\pgfqpoint{0.719237in}{1.212767in}}{\pgfqpoint{0.727473in}{1.212767in}}%
\pgfpathclose%
\pgfusepath{stroke,fill}%
\end{pgfscope}%
\begin{pgfscope}%
\pgfpathrectangle{\pgfqpoint{0.100000in}{0.212622in}}{\pgfqpoint{3.696000in}{3.696000in}}%
\pgfusepath{clip}%
\pgfsetbuttcap%
\pgfsetroundjoin%
\definecolor{currentfill}{rgb}{0.121569,0.466667,0.705882}%
\pgfsetfillcolor{currentfill}%
\pgfsetfillopacity{0.663761}%
\pgfsetlinewidth{1.003750pt}%
\definecolor{currentstroke}{rgb}{0.121569,0.466667,0.705882}%
\pgfsetstrokecolor{currentstroke}%
\pgfsetstrokeopacity{0.663761}%
\pgfsetdash{}{0pt}%
\pgfpathmoveto{\pgfqpoint{0.722800in}{1.212793in}}%
\pgfpathcurveto{\pgfqpoint{0.731036in}{1.212793in}}{\pgfqpoint{0.738936in}{1.216066in}}{\pgfqpoint{0.744760in}{1.221890in}}%
\pgfpathcurveto{\pgfqpoint{0.750584in}{1.227714in}}{\pgfqpoint{0.753857in}{1.235614in}}{\pgfqpoint{0.753857in}{1.243850in}}%
\pgfpathcurveto{\pgfqpoint{0.753857in}{1.252086in}}{\pgfqpoint{0.750584in}{1.259986in}}{\pgfqpoint{0.744760in}{1.265810in}}%
\pgfpathcurveto{\pgfqpoint{0.738936in}{1.271634in}}{\pgfqpoint{0.731036in}{1.274906in}}{\pgfqpoint{0.722800in}{1.274906in}}%
\pgfpathcurveto{\pgfqpoint{0.714564in}{1.274906in}}{\pgfqpoint{0.706664in}{1.271634in}}{\pgfqpoint{0.700840in}{1.265810in}}%
\pgfpathcurveto{\pgfqpoint{0.695016in}{1.259986in}}{\pgfqpoint{0.691744in}{1.252086in}}{\pgfqpoint{0.691744in}{1.243850in}}%
\pgfpathcurveto{\pgfqpoint{0.691744in}{1.235614in}}{\pgfqpoint{0.695016in}{1.227714in}}{\pgfqpoint{0.700840in}{1.221890in}}%
\pgfpathcurveto{\pgfqpoint{0.706664in}{1.216066in}}{\pgfqpoint{0.714564in}{1.212793in}}{\pgfqpoint{0.722800in}{1.212793in}}%
\pgfpathclose%
\pgfusepath{stroke,fill}%
\end{pgfscope}%
\begin{pgfscope}%
\pgfpathrectangle{\pgfqpoint{0.100000in}{0.212622in}}{\pgfqpoint{3.696000in}{3.696000in}}%
\pgfusepath{clip}%
\pgfsetbuttcap%
\pgfsetroundjoin%
\definecolor{currentfill}{rgb}{0.121569,0.466667,0.705882}%
\pgfsetfillcolor{currentfill}%
\pgfsetfillopacity{0.663771}%
\pgfsetlinewidth{1.003750pt}%
\definecolor{currentstroke}{rgb}{0.121569,0.466667,0.705882}%
\pgfsetstrokecolor{currentstroke}%
\pgfsetstrokeopacity{0.663771}%
\pgfsetdash{}{0pt}%
\pgfpathmoveto{\pgfqpoint{0.722221in}{1.212974in}}%
\pgfpathcurveto{\pgfqpoint{0.730458in}{1.212974in}}{\pgfqpoint{0.738358in}{1.216247in}}{\pgfqpoint{0.744182in}{1.222071in}}%
\pgfpathcurveto{\pgfqpoint{0.750006in}{1.227895in}}{\pgfqpoint{0.753278in}{1.235795in}}{\pgfqpoint{0.753278in}{1.244031in}}%
\pgfpathcurveto{\pgfqpoint{0.753278in}{1.252267in}}{\pgfqpoint{0.750006in}{1.260167in}}{\pgfqpoint{0.744182in}{1.265991in}}%
\pgfpathcurveto{\pgfqpoint{0.738358in}{1.271815in}}{\pgfqpoint{0.730458in}{1.275087in}}{\pgfqpoint{0.722221in}{1.275087in}}%
\pgfpathcurveto{\pgfqpoint{0.713985in}{1.275087in}}{\pgfqpoint{0.706085in}{1.271815in}}{\pgfqpoint{0.700261in}{1.265991in}}%
\pgfpathcurveto{\pgfqpoint{0.694437in}{1.260167in}}{\pgfqpoint{0.691165in}{1.252267in}}{\pgfqpoint{0.691165in}{1.244031in}}%
\pgfpathcurveto{\pgfqpoint{0.691165in}{1.235795in}}{\pgfqpoint{0.694437in}{1.227895in}}{\pgfqpoint{0.700261in}{1.222071in}}%
\pgfpathcurveto{\pgfqpoint{0.706085in}{1.216247in}}{\pgfqpoint{0.713985in}{1.212974in}}{\pgfqpoint{0.722221in}{1.212974in}}%
\pgfpathclose%
\pgfusepath{stroke,fill}%
\end{pgfscope}%
\begin{pgfscope}%
\pgfpathrectangle{\pgfqpoint{0.100000in}{0.212622in}}{\pgfqpoint{3.696000in}{3.696000in}}%
\pgfusepath{clip}%
\pgfsetbuttcap%
\pgfsetroundjoin%
\definecolor{currentfill}{rgb}{0.121569,0.466667,0.705882}%
\pgfsetfillcolor{currentfill}%
\pgfsetfillopacity{0.663780}%
\pgfsetlinewidth{1.003750pt}%
\definecolor{currentstroke}{rgb}{0.121569,0.466667,0.705882}%
\pgfsetstrokecolor{currentstroke}%
\pgfsetstrokeopacity{0.663780}%
\pgfsetdash{}{0pt}%
\pgfpathmoveto{\pgfqpoint{0.722105in}{1.213045in}}%
\pgfpathcurveto{\pgfqpoint{0.730341in}{1.213045in}}{\pgfqpoint{0.738241in}{1.216318in}}{\pgfqpoint{0.744065in}{1.222142in}}%
\pgfpathcurveto{\pgfqpoint{0.749889in}{1.227966in}}{\pgfqpoint{0.753161in}{1.235866in}}{\pgfqpoint{0.753161in}{1.244102in}}%
\pgfpathcurveto{\pgfqpoint{0.753161in}{1.252338in}}{\pgfqpoint{0.749889in}{1.260238in}}{\pgfqpoint{0.744065in}{1.266062in}}%
\pgfpathcurveto{\pgfqpoint{0.738241in}{1.271886in}}{\pgfqpoint{0.730341in}{1.275158in}}{\pgfqpoint{0.722105in}{1.275158in}}%
\pgfpathcurveto{\pgfqpoint{0.713868in}{1.275158in}}{\pgfqpoint{0.705968in}{1.271886in}}{\pgfqpoint{0.700144in}{1.266062in}}%
\pgfpathcurveto{\pgfqpoint{0.694320in}{1.260238in}}{\pgfqpoint{0.691048in}{1.252338in}}{\pgfqpoint{0.691048in}{1.244102in}}%
\pgfpathcurveto{\pgfqpoint{0.691048in}{1.235866in}}{\pgfqpoint{0.694320in}{1.227966in}}{\pgfqpoint{0.700144in}{1.222142in}}%
\pgfpathcurveto{\pgfqpoint{0.705968in}{1.216318in}}{\pgfqpoint{0.713868in}{1.213045in}}{\pgfqpoint{0.722105in}{1.213045in}}%
\pgfpathclose%
\pgfusepath{stroke,fill}%
\end{pgfscope}%
\begin{pgfscope}%
\pgfpathrectangle{\pgfqpoint{0.100000in}{0.212622in}}{\pgfqpoint{3.696000in}{3.696000in}}%
\pgfusepath{clip}%
\pgfsetbuttcap%
\pgfsetroundjoin%
\definecolor{currentfill}{rgb}{0.121569,0.466667,0.705882}%
\pgfsetfillcolor{currentfill}%
\pgfsetfillopacity{0.663792}%
\pgfsetlinewidth{1.003750pt}%
\definecolor{currentstroke}{rgb}{0.121569,0.466667,0.705882}%
\pgfsetstrokecolor{currentstroke}%
\pgfsetstrokeopacity{0.663792}%
\pgfsetdash{}{0pt}%
\pgfpathmoveto{\pgfqpoint{0.722412in}{1.212975in}}%
\pgfpathcurveto{\pgfqpoint{0.730648in}{1.212975in}}{\pgfqpoint{0.738548in}{1.216248in}}{\pgfqpoint{0.744372in}{1.222072in}}%
\pgfpathcurveto{\pgfqpoint{0.750196in}{1.227896in}}{\pgfqpoint{0.753468in}{1.235796in}}{\pgfqpoint{0.753468in}{1.244032in}}%
\pgfpathcurveto{\pgfqpoint{0.753468in}{1.252268in}}{\pgfqpoint{0.750196in}{1.260168in}}{\pgfqpoint{0.744372in}{1.265992in}}%
\pgfpathcurveto{\pgfqpoint{0.738548in}{1.271816in}}{\pgfqpoint{0.730648in}{1.275088in}}{\pgfqpoint{0.722412in}{1.275088in}}%
\pgfpathcurveto{\pgfqpoint{0.714176in}{1.275088in}}{\pgfqpoint{0.706276in}{1.271816in}}{\pgfqpoint{0.700452in}{1.265992in}}%
\pgfpathcurveto{\pgfqpoint{0.694628in}{1.260168in}}{\pgfqpoint{0.691355in}{1.252268in}}{\pgfqpoint{0.691355in}{1.244032in}}%
\pgfpathcurveto{\pgfqpoint{0.691355in}{1.235796in}}{\pgfqpoint{0.694628in}{1.227896in}}{\pgfqpoint{0.700452in}{1.222072in}}%
\pgfpathcurveto{\pgfqpoint{0.706276in}{1.216248in}}{\pgfqpoint{0.714176in}{1.212975in}}{\pgfqpoint{0.722412in}{1.212975in}}%
\pgfpathclose%
\pgfusepath{stroke,fill}%
\end{pgfscope}%
\begin{pgfscope}%
\pgfpathrectangle{\pgfqpoint{0.100000in}{0.212622in}}{\pgfqpoint{3.696000in}{3.696000in}}%
\pgfusepath{clip}%
\pgfsetbuttcap%
\pgfsetroundjoin%
\definecolor{currentfill}{rgb}{0.121569,0.466667,0.705882}%
\pgfsetfillcolor{currentfill}%
\pgfsetfillopacity{0.663813}%
\pgfsetlinewidth{1.003750pt}%
\definecolor{currentstroke}{rgb}{0.121569,0.466667,0.705882}%
\pgfsetstrokecolor{currentstroke}%
\pgfsetstrokeopacity{0.663813}%
\pgfsetdash{}{0pt}%
\pgfpathmoveto{\pgfqpoint{0.723456in}{1.212817in}}%
\pgfpathcurveto{\pgfqpoint{0.731692in}{1.212817in}}{\pgfqpoint{0.739592in}{1.216089in}}{\pgfqpoint{0.745416in}{1.221913in}}%
\pgfpathcurveto{\pgfqpoint{0.751240in}{1.227737in}}{\pgfqpoint{0.754512in}{1.235637in}}{\pgfqpoint{0.754512in}{1.243873in}}%
\pgfpathcurveto{\pgfqpoint{0.754512in}{1.252109in}}{\pgfqpoint{0.751240in}{1.260009in}}{\pgfqpoint{0.745416in}{1.265833in}}%
\pgfpathcurveto{\pgfqpoint{0.739592in}{1.271657in}}{\pgfqpoint{0.731692in}{1.274930in}}{\pgfqpoint{0.723456in}{1.274930in}}%
\pgfpathcurveto{\pgfqpoint{0.715219in}{1.274930in}}{\pgfqpoint{0.707319in}{1.271657in}}{\pgfqpoint{0.701495in}{1.265833in}}%
\pgfpathcurveto{\pgfqpoint{0.695671in}{1.260009in}}{\pgfqpoint{0.692399in}{1.252109in}}{\pgfqpoint{0.692399in}{1.243873in}}%
\pgfpathcurveto{\pgfqpoint{0.692399in}{1.235637in}}{\pgfqpoint{0.695671in}{1.227737in}}{\pgfqpoint{0.701495in}{1.221913in}}%
\pgfpathcurveto{\pgfqpoint{0.707319in}{1.216089in}}{\pgfqpoint{0.715219in}{1.212817in}}{\pgfqpoint{0.723456in}{1.212817in}}%
\pgfpathclose%
\pgfusepath{stroke,fill}%
\end{pgfscope}%
\begin{pgfscope}%
\pgfpathrectangle{\pgfqpoint{0.100000in}{0.212622in}}{\pgfqpoint{3.696000in}{3.696000in}}%
\pgfusepath{clip}%
\pgfsetbuttcap%
\pgfsetroundjoin%
\definecolor{currentfill}{rgb}{0.121569,0.466667,0.705882}%
\pgfsetfillcolor{currentfill}%
\pgfsetfillopacity{0.663814}%
\pgfsetlinewidth{1.003750pt}%
\definecolor{currentstroke}{rgb}{0.121569,0.466667,0.705882}%
\pgfsetstrokecolor{currentstroke}%
\pgfsetstrokeopacity{0.663814}%
\pgfsetdash{}{0pt}%
\pgfpathmoveto{\pgfqpoint{0.727214in}{1.212723in}}%
\pgfpathcurveto{\pgfqpoint{0.735450in}{1.212723in}}{\pgfqpoint{0.743350in}{1.215996in}}{\pgfqpoint{0.749174in}{1.221820in}}%
\pgfpathcurveto{\pgfqpoint{0.754998in}{1.227644in}}{\pgfqpoint{0.758271in}{1.235544in}}{\pgfqpoint{0.758271in}{1.243780in}}%
\pgfpathcurveto{\pgfqpoint{0.758271in}{1.252016in}}{\pgfqpoint{0.754998in}{1.259916in}}{\pgfqpoint{0.749174in}{1.265740in}}%
\pgfpathcurveto{\pgfqpoint{0.743350in}{1.271564in}}{\pgfqpoint{0.735450in}{1.274836in}}{\pgfqpoint{0.727214in}{1.274836in}}%
\pgfpathcurveto{\pgfqpoint{0.718978in}{1.274836in}}{\pgfqpoint{0.711078in}{1.271564in}}{\pgfqpoint{0.705254in}{1.265740in}}%
\pgfpathcurveto{\pgfqpoint{0.699430in}{1.259916in}}{\pgfqpoint{0.696158in}{1.252016in}}{\pgfqpoint{0.696158in}{1.243780in}}%
\pgfpathcurveto{\pgfqpoint{0.696158in}{1.235544in}}{\pgfqpoint{0.699430in}{1.227644in}}{\pgfqpoint{0.705254in}{1.221820in}}%
\pgfpathcurveto{\pgfqpoint{0.711078in}{1.215996in}}{\pgfqpoint{0.718978in}{1.212723in}}{\pgfqpoint{0.727214in}{1.212723in}}%
\pgfpathclose%
\pgfusepath{stroke,fill}%
\end{pgfscope}%
\begin{pgfscope}%
\pgfpathrectangle{\pgfqpoint{0.100000in}{0.212622in}}{\pgfqpoint{3.696000in}{3.696000in}}%
\pgfusepath{clip}%
\pgfsetbuttcap%
\pgfsetroundjoin%
\definecolor{currentfill}{rgb}{0.121569,0.466667,0.705882}%
\pgfsetfillcolor{currentfill}%
\pgfsetfillopacity{0.663851}%
\pgfsetlinewidth{1.003750pt}%
\definecolor{currentstroke}{rgb}{0.121569,0.466667,0.705882}%
\pgfsetstrokecolor{currentstroke}%
\pgfsetstrokeopacity{0.663851}%
\pgfsetdash{}{0pt}%
\pgfpathmoveto{\pgfqpoint{0.727075in}{1.212703in}}%
\pgfpathcurveto{\pgfqpoint{0.735311in}{1.212703in}}{\pgfqpoint{0.743212in}{1.215976in}}{\pgfqpoint{0.749035in}{1.221799in}}%
\pgfpathcurveto{\pgfqpoint{0.754859in}{1.227623in}}{\pgfqpoint{0.758132in}{1.235523in}}{\pgfqpoint{0.758132in}{1.243760in}}%
\pgfpathcurveto{\pgfqpoint{0.758132in}{1.251996in}}{\pgfqpoint{0.754859in}{1.259896in}}{\pgfqpoint{0.749035in}{1.265720in}}%
\pgfpathcurveto{\pgfqpoint{0.743212in}{1.271544in}}{\pgfqpoint{0.735311in}{1.274816in}}{\pgfqpoint{0.727075in}{1.274816in}}%
\pgfpathcurveto{\pgfqpoint{0.718839in}{1.274816in}}{\pgfqpoint{0.710939in}{1.271544in}}{\pgfqpoint{0.705115in}{1.265720in}}%
\pgfpathcurveto{\pgfqpoint{0.699291in}{1.259896in}}{\pgfqpoint{0.696019in}{1.251996in}}{\pgfqpoint{0.696019in}{1.243760in}}%
\pgfpathcurveto{\pgfqpoint{0.696019in}{1.235523in}}{\pgfqpoint{0.699291in}{1.227623in}}{\pgfqpoint{0.705115in}{1.221799in}}%
\pgfpathcurveto{\pgfqpoint{0.710939in}{1.215976in}}{\pgfqpoint{0.718839in}{1.212703in}}{\pgfqpoint{0.727075in}{1.212703in}}%
\pgfpathclose%
\pgfusepath{stroke,fill}%
\end{pgfscope}%
\begin{pgfscope}%
\pgfpathrectangle{\pgfqpoint{0.100000in}{0.212622in}}{\pgfqpoint{3.696000in}{3.696000in}}%
\pgfusepath{clip}%
\pgfsetbuttcap%
\pgfsetroundjoin%
\definecolor{currentfill}{rgb}{0.121569,0.466667,0.705882}%
\pgfsetfillcolor{currentfill}%
\pgfsetfillopacity{0.663872}%
\pgfsetlinewidth{1.003750pt}%
\definecolor{currentstroke}{rgb}{0.121569,0.466667,0.705882}%
\pgfsetstrokecolor{currentstroke}%
\pgfsetstrokeopacity{0.663872}%
\pgfsetdash{}{0pt}%
\pgfpathmoveto{\pgfqpoint{0.726998in}{1.212693in}}%
\pgfpathcurveto{\pgfqpoint{0.735234in}{1.212693in}}{\pgfqpoint{0.743134in}{1.215965in}}{\pgfqpoint{0.748958in}{1.221789in}}%
\pgfpathcurveto{\pgfqpoint{0.754782in}{1.227613in}}{\pgfqpoint{0.758054in}{1.235513in}}{\pgfqpoint{0.758054in}{1.243749in}}%
\pgfpathcurveto{\pgfqpoint{0.758054in}{1.251985in}}{\pgfqpoint{0.754782in}{1.259885in}}{\pgfqpoint{0.748958in}{1.265709in}}%
\pgfpathcurveto{\pgfqpoint{0.743134in}{1.271533in}}{\pgfqpoint{0.735234in}{1.274806in}}{\pgfqpoint{0.726998in}{1.274806in}}%
\pgfpathcurveto{\pgfqpoint{0.718762in}{1.274806in}}{\pgfqpoint{0.710862in}{1.271533in}}{\pgfqpoint{0.705038in}{1.265709in}}%
\pgfpathcurveto{\pgfqpoint{0.699214in}{1.259885in}}{\pgfqpoint{0.695941in}{1.251985in}}{\pgfqpoint{0.695941in}{1.243749in}}%
\pgfpathcurveto{\pgfqpoint{0.695941in}{1.235513in}}{\pgfqpoint{0.699214in}{1.227613in}}{\pgfqpoint{0.705038in}{1.221789in}}%
\pgfpathcurveto{\pgfqpoint{0.710862in}{1.215965in}}{\pgfqpoint{0.718762in}{1.212693in}}{\pgfqpoint{0.726998in}{1.212693in}}%
\pgfpathclose%
\pgfusepath{stroke,fill}%
\end{pgfscope}%
\begin{pgfscope}%
\pgfpathrectangle{\pgfqpoint{0.100000in}{0.212622in}}{\pgfqpoint{3.696000in}{3.696000in}}%
\pgfusepath{clip}%
\pgfsetbuttcap%
\pgfsetroundjoin%
\definecolor{currentfill}{rgb}{0.121569,0.466667,0.705882}%
\pgfsetfillcolor{currentfill}%
\pgfsetfillopacity{0.663883}%
\pgfsetlinewidth{1.003750pt}%
\definecolor{currentstroke}{rgb}{0.121569,0.466667,0.705882}%
\pgfsetstrokecolor{currentstroke}%
\pgfsetstrokeopacity{0.663883}%
\pgfsetdash{}{0pt}%
\pgfpathmoveto{\pgfqpoint{0.726955in}{1.212687in}}%
\pgfpathcurveto{\pgfqpoint{0.735191in}{1.212687in}}{\pgfqpoint{0.743091in}{1.215959in}}{\pgfqpoint{0.748915in}{1.221783in}}%
\pgfpathcurveto{\pgfqpoint{0.754739in}{1.227607in}}{\pgfqpoint{0.758011in}{1.235507in}}{\pgfqpoint{0.758011in}{1.243743in}}%
\pgfpathcurveto{\pgfqpoint{0.758011in}{1.251980in}}{\pgfqpoint{0.754739in}{1.259880in}}{\pgfqpoint{0.748915in}{1.265704in}}%
\pgfpathcurveto{\pgfqpoint{0.743091in}{1.271527in}}{\pgfqpoint{0.735191in}{1.274800in}}{\pgfqpoint{0.726955in}{1.274800in}}%
\pgfpathcurveto{\pgfqpoint{0.718719in}{1.274800in}}{\pgfqpoint{0.710819in}{1.271527in}}{\pgfqpoint{0.704995in}{1.265704in}}%
\pgfpathcurveto{\pgfqpoint{0.699171in}{1.259880in}}{\pgfqpoint{0.695898in}{1.251980in}}{\pgfqpoint{0.695898in}{1.243743in}}%
\pgfpathcurveto{\pgfqpoint{0.695898in}{1.235507in}}{\pgfqpoint{0.699171in}{1.227607in}}{\pgfqpoint{0.704995in}{1.221783in}}%
\pgfpathcurveto{\pgfqpoint{0.710819in}{1.215959in}}{\pgfqpoint{0.718719in}{1.212687in}}{\pgfqpoint{0.726955in}{1.212687in}}%
\pgfpathclose%
\pgfusepath{stroke,fill}%
\end{pgfscope}%
\begin{pgfscope}%
\pgfpathrectangle{\pgfqpoint{0.100000in}{0.212622in}}{\pgfqpoint{3.696000in}{3.696000in}}%
\pgfusepath{clip}%
\pgfsetbuttcap%
\pgfsetroundjoin%
\definecolor{currentfill}{rgb}{0.121569,0.466667,0.705882}%
\pgfsetfillcolor{currentfill}%
\pgfsetfillopacity{0.663889}%
\pgfsetlinewidth{1.003750pt}%
\definecolor{currentstroke}{rgb}{0.121569,0.466667,0.705882}%
\pgfsetstrokecolor{currentstroke}%
\pgfsetstrokeopacity{0.663889}%
\pgfsetdash{}{0pt}%
\pgfpathmoveto{\pgfqpoint{0.726932in}{1.212683in}}%
\pgfpathcurveto{\pgfqpoint{0.735168in}{1.212683in}}{\pgfqpoint{0.743068in}{1.215955in}}{\pgfqpoint{0.748892in}{1.221779in}}%
\pgfpathcurveto{\pgfqpoint{0.754716in}{1.227603in}}{\pgfqpoint{0.757989in}{1.235503in}}{\pgfqpoint{0.757989in}{1.243739in}}%
\pgfpathcurveto{\pgfqpoint{0.757989in}{1.251976in}}{\pgfqpoint{0.754716in}{1.259876in}}{\pgfqpoint{0.748892in}{1.265700in}}%
\pgfpathcurveto{\pgfqpoint{0.743068in}{1.271524in}}{\pgfqpoint{0.735168in}{1.274796in}}{\pgfqpoint{0.726932in}{1.274796in}}%
\pgfpathcurveto{\pgfqpoint{0.718696in}{1.274796in}}{\pgfqpoint{0.710796in}{1.271524in}}{\pgfqpoint{0.704972in}{1.265700in}}%
\pgfpathcurveto{\pgfqpoint{0.699148in}{1.259876in}}{\pgfqpoint{0.695876in}{1.251976in}}{\pgfqpoint{0.695876in}{1.243739in}}%
\pgfpathcurveto{\pgfqpoint{0.695876in}{1.235503in}}{\pgfqpoint{0.699148in}{1.227603in}}{\pgfqpoint{0.704972in}{1.221779in}}%
\pgfpathcurveto{\pgfqpoint{0.710796in}{1.215955in}}{\pgfqpoint{0.718696in}{1.212683in}}{\pgfqpoint{0.726932in}{1.212683in}}%
\pgfpathclose%
\pgfusepath{stroke,fill}%
\end{pgfscope}%
\begin{pgfscope}%
\pgfpathrectangle{\pgfqpoint{0.100000in}{0.212622in}}{\pgfqpoint{3.696000in}{3.696000in}}%
\pgfusepath{clip}%
\pgfsetbuttcap%
\pgfsetroundjoin%
\definecolor{currentfill}{rgb}{0.121569,0.466667,0.705882}%
\pgfsetfillcolor{currentfill}%
\pgfsetfillopacity{0.663892}%
\pgfsetlinewidth{1.003750pt}%
\definecolor{currentstroke}{rgb}{0.121569,0.466667,0.705882}%
\pgfsetstrokecolor{currentstroke}%
\pgfsetstrokeopacity{0.663892}%
\pgfsetdash{}{0pt}%
\pgfpathmoveto{\pgfqpoint{0.726919in}{1.212681in}}%
\pgfpathcurveto{\pgfqpoint{0.735156in}{1.212681in}}{\pgfqpoint{0.743056in}{1.215953in}}{\pgfqpoint{0.748880in}{1.221777in}}%
\pgfpathcurveto{\pgfqpoint{0.754704in}{1.227601in}}{\pgfqpoint{0.757976in}{1.235501in}}{\pgfqpoint{0.757976in}{1.243737in}}%
\pgfpathcurveto{\pgfqpoint{0.757976in}{1.251974in}}{\pgfqpoint{0.754704in}{1.259874in}}{\pgfqpoint{0.748880in}{1.265698in}}%
\pgfpathcurveto{\pgfqpoint{0.743056in}{1.271522in}}{\pgfqpoint{0.735156in}{1.274794in}}{\pgfqpoint{0.726919in}{1.274794in}}%
\pgfpathcurveto{\pgfqpoint{0.718683in}{1.274794in}}{\pgfqpoint{0.710783in}{1.271522in}}{\pgfqpoint{0.704959in}{1.265698in}}%
\pgfpathcurveto{\pgfqpoint{0.699135in}{1.259874in}}{\pgfqpoint{0.695863in}{1.251974in}}{\pgfqpoint{0.695863in}{1.243737in}}%
\pgfpathcurveto{\pgfqpoint{0.695863in}{1.235501in}}{\pgfqpoint{0.699135in}{1.227601in}}{\pgfqpoint{0.704959in}{1.221777in}}%
\pgfpathcurveto{\pgfqpoint{0.710783in}{1.215953in}}{\pgfqpoint{0.718683in}{1.212681in}}{\pgfqpoint{0.726919in}{1.212681in}}%
\pgfpathclose%
\pgfusepath{stroke,fill}%
\end{pgfscope}%
\begin{pgfscope}%
\pgfpathrectangle{\pgfqpoint{0.100000in}{0.212622in}}{\pgfqpoint{3.696000in}{3.696000in}}%
\pgfusepath{clip}%
\pgfsetbuttcap%
\pgfsetroundjoin%
\definecolor{currentfill}{rgb}{0.121569,0.466667,0.705882}%
\pgfsetfillcolor{currentfill}%
\pgfsetfillopacity{0.663894}%
\pgfsetlinewidth{1.003750pt}%
\definecolor{currentstroke}{rgb}{0.121569,0.466667,0.705882}%
\pgfsetstrokecolor{currentstroke}%
\pgfsetstrokeopacity{0.663894}%
\pgfsetdash{}{0pt}%
\pgfpathmoveto{\pgfqpoint{0.726912in}{1.212680in}}%
\pgfpathcurveto{\pgfqpoint{0.735148in}{1.212680in}}{\pgfqpoint{0.743048in}{1.215952in}}{\pgfqpoint{0.748872in}{1.221776in}}%
\pgfpathcurveto{\pgfqpoint{0.754696in}{1.227600in}}{\pgfqpoint{0.757969in}{1.235500in}}{\pgfqpoint{0.757969in}{1.243737in}}%
\pgfpathcurveto{\pgfqpoint{0.757969in}{1.251973in}}{\pgfqpoint{0.754696in}{1.259873in}}{\pgfqpoint{0.748872in}{1.265697in}}%
\pgfpathcurveto{\pgfqpoint{0.743048in}{1.271521in}}{\pgfqpoint{0.735148in}{1.274793in}}{\pgfqpoint{0.726912in}{1.274793in}}%
\pgfpathcurveto{\pgfqpoint{0.718676in}{1.274793in}}{\pgfqpoint{0.710776in}{1.271521in}}{\pgfqpoint{0.704952in}{1.265697in}}%
\pgfpathcurveto{\pgfqpoint{0.699128in}{1.259873in}}{\pgfqpoint{0.695856in}{1.251973in}}{\pgfqpoint{0.695856in}{1.243737in}}%
\pgfpathcurveto{\pgfqpoint{0.695856in}{1.235500in}}{\pgfqpoint{0.699128in}{1.227600in}}{\pgfqpoint{0.704952in}{1.221776in}}%
\pgfpathcurveto{\pgfqpoint{0.710776in}{1.215952in}}{\pgfqpoint{0.718676in}{1.212680in}}{\pgfqpoint{0.726912in}{1.212680in}}%
\pgfpathclose%
\pgfusepath{stroke,fill}%
\end{pgfscope}%
\begin{pgfscope}%
\pgfpathrectangle{\pgfqpoint{0.100000in}{0.212622in}}{\pgfqpoint{3.696000in}{3.696000in}}%
\pgfusepath{clip}%
\pgfsetbuttcap%
\pgfsetroundjoin%
\definecolor{currentfill}{rgb}{0.121569,0.466667,0.705882}%
\pgfsetfillcolor{currentfill}%
\pgfsetfillopacity{0.663895}%
\pgfsetlinewidth{1.003750pt}%
\definecolor{currentstroke}{rgb}{0.121569,0.466667,0.705882}%
\pgfsetstrokecolor{currentstroke}%
\pgfsetstrokeopacity{0.663895}%
\pgfsetdash{}{0pt}%
\pgfpathmoveto{\pgfqpoint{0.726908in}{1.212680in}}%
\pgfpathcurveto{\pgfqpoint{0.735145in}{1.212680in}}{\pgfqpoint{0.743045in}{1.215952in}}{\pgfqpoint{0.748869in}{1.221776in}}%
\pgfpathcurveto{\pgfqpoint{0.754692in}{1.227600in}}{\pgfqpoint{0.757965in}{1.235500in}}{\pgfqpoint{0.757965in}{1.243736in}}%
\pgfpathcurveto{\pgfqpoint{0.757965in}{1.251972in}}{\pgfqpoint{0.754692in}{1.259872in}}{\pgfqpoint{0.748869in}{1.265696in}}%
\pgfpathcurveto{\pgfqpoint{0.743045in}{1.271520in}}{\pgfqpoint{0.735145in}{1.274793in}}{\pgfqpoint{0.726908in}{1.274793in}}%
\pgfpathcurveto{\pgfqpoint{0.718672in}{1.274793in}}{\pgfqpoint{0.710772in}{1.271520in}}{\pgfqpoint{0.704948in}{1.265696in}}%
\pgfpathcurveto{\pgfqpoint{0.699124in}{1.259872in}}{\pgfqpoint{0.695852in}{1.251972in}}{\pgfqpoint{0.695852in}{1.243736in}}%
\pgfpathcurveto{\pgfqpoint{0.695852in}{1.235500in}}{\pgfqpoint{0.699124in}{1.227600in}}{\pgfqpoint{0.704948in}{1.221776in}}%
\pgfpathcurveto{\pgfqpoint{0.710772in}{1.215952in}}{\pgfqpoint{0.718672in}{1.212680in}}{\pgfqpoint{0.726908in}{1.212680in}}%
\pgfpathclose%
\pgfusepath{stroke,fill}%
\end{pgfscope}%
\begin{pgfscope}%
\pgfpathrectangle{\pgfqpoint{0.100000in}{0.212622in}}{\pgfqpoint{3.696000in}{3.696000in}}%
\pgfusepath{clip}%
\pgfsetbuttcap%
\pgfsetroundjoin%
\definecolor{currentfill}{rgb}{0.121569,0.466667,0.705882}%
\pgfsetfillcolor{currentfill}%
\pgfsetfillopacity{0.663896}%
\pgfsetlinewidth{1.003750pt}%
\definecolor{currentstroke}{rgb}{0.121569,0.466667,0.705882}%
\pgfsetstrokecolor{currentstroke}%
\pgfsetstrokeopacity{0.663896}%
\pgfsetdash{}{0pt}%
\pgfpathmoveto{\pgfqpoint{0.726906in}{1.212679in}}%
\pgfpathcurveto{\pgfqpoint{0.735142in}{1.212679in}}{\pgfqpoint{0.743042in}{1.215952in}}{\pgfqpoint{0.748866in}{1.221776in}}%
\pgfpathcurveto{\pgfqpoint{0.754690in}{1.227600in}}{\pgfqpoint{0.757963in}{1.235500in}}{\pgfqpoint{0.757963in}{1.243736in}}%
\pgfpathcurveto{\pgfqpoint{0.757963in}{1.251972in}}{\pgfqpoint{0.754690in}{1.259872in}}{\pgfqpoint{0.748866in}{1.265696in}}%
\pgfpathcurveto{\pgfqpoint{0.743042in}{1.271520in}}{\pgfqpoint{0.735142in}{1.274792in}}{\pgfqpoint{0.726906in}{1.274792in}}%
\pgfpathcurveto{\pgfqpoint{0.718670in}{1.274792in}}{\pgfqpoint{0.710770in}{1.271520in}}{\pgfqpoint{0.704946in}{1.265696in}}%
\pgfpathcurveto{\pgfqpoint{0.699122in}{1.259872in}}{\pgfqpoint{0.695850in}{1.251972in}}{\pgfqpoint{0.695850in}{1.243736in}}%
\pgfpathcurveto{\pgfqpoint{0.695850in}{1.235500in}}{\pgfqpoint{0.699122in}{1.227600in}}{\pgfqpoint{0.704946in}{1.221776in}}%
\pgfpathcurveto{\pgfqpoint{0.710770in}{1.215952in}}{\pgfqpoint{0.718670in}{1.212679in}}{\pgfqpoint{0.726906in}{1.212679in}}%
\pgfpathclose%
\pgfusepath{stroke,fill}%
\end{pgfscope}%
\begin{pgfscope}%
\pgfpathrectangle{\pgfqpoint{0.100000in}{0.212622in}}{\pgfqpoint{3.696000in}{3.696000in}}%
\pgfusepath{clip}%
\pgfsetbuttcap%
\pgfsetroundjoin%
\definecolor{currentfill}{rgb}{0.121569,0.466667,0.705882}%
\pgfsetfillcolor{currentfill}%
\pgfsetfillopacity{0.663896}%
\pgfsetlinewidth{1.003750pt}%
\definecolor{currentstroke}{rgb}{0.121569,0.466667,0.705882}%
\pgfsetstrokecolor{currentstroke}%
\pgfsetstrokeopacity{0.663896}%
\pgfsetdash{}{0pt}%
\pgfpathmoveto{\pgfqpoint{0.726905in}{1.212679in}}%
\pgfpathcurveto{\pgfqpoint{0.735141in}{1.212679in}}{\pgfqpoint{0.743041in}{1.215952in}}{\pgfqpoint{0.748865in}{1.221776in}}%
\pgfpathcurveto{\pgfqpoint{0.754689in}{1.227600in}}{\pgfqpoint{0.757961in}{1.235500in}}{\pgfqpoint{0.757961in}{1.243736in}}%
\pgfpathcurveto{\pgfqpoint{0.757961in}{1.251972in}}{\pgfqpoint{0.754689in}{1.259872in}}{\pgfqpoint{0.748865in}{1.265696in}}%
\pgfpathcurveto{\pgfqpoint{0.743041in}{1.271520in}}{\pgfqpoint{0.735141in}{1.274792in}}{\pgfqpoint{0.726905in}{1.274792in}}%
\pgfpathcurveto{\pgfqpoint{0.718669in}{1.274792in}}{\pgfqpoint{0.710769in}{1.271520in}}{\pgfqpoint{0.704945in}{1.265696in}}%
\pgfpathcurveto{\pgfqpoint{0.699121in}{1.259872in}}{\pgfqpoint{0.695848in}{1.251972in}}{\pgfqpoint{0.695848in}{1.243736in}}%
\pgfpathcurveto{\pgfqpoint{0.695848in}{1.235500in}}{\pgfqpoint{0.699121in}{1.227600in}}{\pgfqpoint{0.704945in}{1.221776in}}%
\pgfpathcurveto{\pgfqpoint{0.710769in}{1.215952in}}{\pgfqpoint{0.718669in}{1.212679in}}{\pgfqpoint{0.726905in}{1.212679in}}%
\pgfpathclose%
\pgfusepath{stroke,fill}%
\end{pgfscope}%
\begin{pgfscope}%
\pgfpathrectangle{\pgfqpoint{0.100000in}{0.212622in}}{\pgfqpoint{3.696000in}{3.696000in}}%
\pgfusepath{clip}%
\pgfsetbuttcap%
\pgfsetroundjoin%
\definecolor{currentfill}{rgb}{0.121569,0.466667,0.705882}%
\pgfsetfillcolor{currentfill}%
\pgfsetfillopacity{0.663896}%
\pgfsetlinewidth{1.003750pt}%
\definecolor{currentstroke}{rgb}{0.121569,0.466667,0.705882}%
\pgfsetstrokecolor{currentstroke}%
\pgfsetstrokeopacity{0.663896}%
\pgfsetdash{}{0pt}%
\pgfpathmoveto{\pgfqpoint{0.726904in}{1.212679in}}%
\pgfpathcurveto{\pgfqpoint{0.735141in}{1.212679in}}{\pgfqpoint{0.743041in}{1.215952in}}{\pgfqpoint{0.748865in}{1.221776in}}%
\pgfpathcurveto{\pgfqpoint{0.754688in}{1.227599in}}{\pgfqpoint{0.757961in}{1.235500in}}{\pgfqpoint{0.757961in}{1.243736in}}%
\pgfpathcurveto{\pgfqpoint{0.757961in}{1.251972in}}{\pgfqpoint{0.754688in}{1.259872in}}{\pgfqpoint{0.748865in}{1.265696in}}%
\pgfpathcurveto{\pgfqpoint{0.743041in}{1.271520in}}{\pgfqpoint{0.735141in}{1.274792in}}{\pgfqpoint{0.726904in}{1.274792in}}%
\pgfpathcurveto{\pgfqpoint{0.718668in}{1.274792in}}{\pgfqpoint{0.710768in}{1.271520in}}{\pgfqpoint{0.704944in}{1.265696in}}%
\pgfpathcurveto{\pgfqpoint{0.699120in}{1.259872in}}{\pgfqpoint{0.695848in}{1.251972in}}{\pgfqpoint{0.695848in}{1.243736in}}%
\pgfpathcurveto{\pgfqpoint{0.695848in}{1.235500in}}{\pgfqpoint{0.699120in}{1.227599in}}{\pgfqpoint{0.704944in}{1.221776in}}%
\pgfpathcurveto{\pgfqpoint{0.710768in}{1.215952in}}{\pgfqpoint{0.718668in}{1.212679in}}{\pgfqpoint{0.726904in}{1.212679in}}%
\pgfpathclose%
\pgfusepath{stroke,fill}%
\end{pgfscope}%
\begin{pgfscope}%
\pgfpathrectangle{\pgfqpoint{0.100000in}{0.212622in}}{\pgfqpoint{3.696000in}{3.696000in}}%
\pgfusepath{clip}%
\pgfsetbuttcap%
\pgfsetroundjoin%
\definecolor{currentfill}{rgb}{0.121569,0.466667,0.705882}%
\pgfsetfillcolor{currentfill}%
\pgfsetfillopacity{0.663896}%
\pgfsetlinewidth{1.003750pt}%
\definecolor{currentstroke}{rgb}{0.121569,0.466667,0.705882}%
\pgfsetstrokecolor{currentstroke}%
\pgfsetstrokeopacity{0.663896}%
\pgfsetdash{}{0pt}%
\pgfpathmoveto{\pgfqpoint{0.726904in}{1.212679in}}%
\pgfpathcurveto{\pgfqpoint{0.735140in}{1.212679in}}{\pgfqpoint{0.743040in}{1.215952in}}{\pgfqpoint{0.748864in}{1.221776in}}%
\pgfpathcurveto{\pgfqpoint{0.754688in}{1.227599in}}{\pgfqpoint{0.757960in}{1.235500in}}{\pgfqpoint{0.757960in}{1.243736in}}%
\pgfpathcurveto{\pgfqpoint{0.757960in}{1.251972in}}{\pgfqpoint{0.754688in}{1.259872in}}{\pgfqpoint{0.748864in}{1.265696in}}%
\pgfpathcurveto{\pgfqpoint{0.743040in}{1.271520in}}{\pgfqpoint{0.735140in}{1.274792in}}{\pgfqpoint{0.726904in}{1.274792in}}%
\pgfpathcurveto{\pgfqpoint{0.718668in}{1.274792in}}{\pgfqpoint{0.710768in}{1.271520in}}{\pgfqpoint{0.704944in}{1.265696in}}%
\pgfpathcurveto{\pgfqpoint{0.699120in}{1.259872in}}{\pgfqpoint{0.695847in}{1.251972in}}{\pgfqpoint{0.695847in}{1.243736in}}%
\pgfpathcurveto{\pgfqpoint{0.695847in}{1.235500in}}{\pgfqpoint{0.699120in}{1.227599in}}{\pgfqpoint{0.704944in}{1.221776in}}%
\pgfpathcurveto{\pgfqpoint{0.710768in}{1.215952in}}{\pgfqpoint{0.718668in}{1.212679in}}{\pgfqpoint{0.726904in}{1.212679in}}%
\pgfpathclose%
\pgfusepath{stroke,fill}%
\end{pgfscope}%
\begin{pgfscope}%
\pgfpathrectangle{\pgfqpoint{0.100000in}{0.212622in}}{\pgfqpoint{3.696000in}{3.696000in}}%
\pgfusepath{clip}%
\pgfsetbuttcap%
\pgfsetroundjoin%
\definecolor{currentfill}{rgb}{0.121569,0.466667,0.705882}%
\pgfsetfillcolor{currentfill}%
\pgfsetfillopacity{0.663896}%
\pgfsetlinewidth{1.003750pt}%
\definecolor{currentstroke}{rgb}{0.121569,0.466667,0.705882}%
\pgfsetstrokecolor{currentstroke}%
\pgfsetstrokeopacity{0.663896}%
\pgfsetdash{}{0pt}%
\pgfpathmoveto{\pgfqpoint{0.726904in}{1.212679in}}%
\pgfpathcurveto{\pgfqpoint{0.735140in}{1.212679in}}{\pgfqpoint{0.743040in}{1.215952in}}{\pgfqpoint{0.748864in}{1.221776in}}%
\pgfpathcurveto{\pgfqpoint{0.754688in}{1.227599in}}{\pgfqpoint{0.757960in}{1.235500in}}{\pgfqpoint{0.757960in}{1.243736in}}%
\pgfpathcurveto{\pgfqpoint{0.757960in}{1.251972in}}{\pgfqpoint{0.754688in}{1.259872in}}{\pgfqpoint{0.748864in}{1.265696in}}%
\pgfpathcurveto{\pgfqpoint{0.743040in}{1.271520in}}{\pgfqpoint{0.735140in}{1.274792in}}{\pgfqpoint{0.726904in}{1.274792in}}%
\pgfpathcurveto{\pgfqpoint{0.718667in}{1.274792in}}{\pgfqpoint{0.710767in}{1.271520in}}{\pgfqpoint{0.704943in}{1.265696in}}%
\pgfpathcurveto{\pgfqpoint{0.699119in}{1.259872in}}{\pgfqpoint{0.695847in}{1.251972in}}{\pgfqpoint{0.695847in}{1.243736in}}%
\pgfpathcurveto{\pgfqpoint{0.695847in}{1.235500in}}{\pgfqpoint{0.699119in}{1.227599in}}{\pgfqpoint{0.704943in}{1.221776in}}%
\pgfpathcurveto{\pgfqpoint{0.710767in}{1.215952in}}{\pgfqpoint{0.718667in}{1.212679in}}{\pgfqpoint{0.726904in}{1.212679in}}%
\pgfpathclose%
\pgfusepath{stroke,fill}%
\end{pgfscope}%
\begin{pgfscope}%
\pgfpathrectangle{\pgfqpoint{0.100000in}{0.212622in}}{\pgfqpoint{3.696000in}{3.696000in}}%
\pgfusepath{clip}%
\pgfsetbuttcap%
\pgfsetroundjoin%
\definecolor{currentfill}{rgb}{0.121569,0.466667,0.705882}%
\pgfsetfillcolor{currentfill}%
\pgfsetfillopacity{0.663896}%
\pgfsetlinewidth{1.003750pt}%
\definecolor{currentstroke}{rgb}{0.121569,0.466667,0.705882}%
\pgfsetstrokecolor{currentstroke}%
\pgfsetstrokeopacity{0.663896}%
\pgfsetdash{}{0pt}%
\pgfpathmoveto{\pgfqpoint{0.726904in}{1.212679in}}%
\pgfpathcurveto{\pgfqpoint{0.735140in}{1.212679in}}{\pgfqpoint{0.743040in}{1.215952in}}{\pgfqpoint{0.748864in}{1.221776in}}%
\pgfpathcurveto{\pgfqpoint{0.754688in}{1.227599in}}{\pgfqpoint{0.757960in}{1.235500in}}{\pgfqpoint{0.757960in}{1.243736in}}%
\pgfpathcurveto{\pgfqpoint{0.757960in}{1.251972in}}{\pgfqpoint{0.754688in}{1.259872in}}{\pgfqpoint{0.748864in}{1.265696in}}%
\pgfpathcurveto{\pgfqpoint{0.743040in}{1.271520in}}{\pgfqpoint{0.735140in}{1.274792in}}{\pgfqpoint{0.726904in}{1.274792in}}%
\pgfpathcurveto{\pgfqpoint{0.718667in}{1.274792in}}{\pgfqpoint{0.710767in}{1.271520in}}{\pgfqpoint{0.704943in}{1.265696in}}%
\pgfpathcurveto{\pgfqpoint{0.699119in}{1.259872in}}{\pgfqpoint{0.695847in}{1.251972in}}{\pgfqpoint{0.695847in}{1.243736in}}%
\pgfpathcurveto{\pgfqpoint{0.695847in}{1.235500in}}{\pgfqpoint{0.699119in}{1.227599in}}{\pgfqpoint{0.704943in}{1.221776in}}%
\pgfpathcurveto{\pgfqpoint{0.710767in}{1.215952in}}{\pgfqpoint{0.718667in}{1.212679in}}{\pgfqpoint{0.726904in}{1.212679in}}%
\pgfpathclose%
\pgfusepath{stroke,fill}%
\end{pgfscope}%
\begin{pgfscope}%
\pgfpathrectangle{\pgfqpoint{0.100000in}{0.212622in}}{\pgfqpoint{3.696000in}{3.696000in}}%
\pgfusepath{clip}%
\pgfsetbuttcap%
\pgfsetroundjoin%
\definecolor{currentfill}{rgb}{0.121569,0.466667,0.705882}%
\pgfsetfillcolor{currentfill}%
\pgfsetfillopacity{0.663896}%
\pgfsetlinewidth{1.003750pt}%
\definecolor{currentstroke}{rgb}{0.121569,0.466667,0.705882}%
\pgfsetstrokecolor{currentstroke}%
\pgfsetstrokeopacity{0.663896}%
\pgfsetdash{}{0pt}%
\pgfpathmoveto{\pgfqpoint{0.726904in}{1.212679in}}%
\pgfpathcurveto{\pgfqpoint{0.735140in}{1.212679in}}{\pgfqpoint{0.743040in}{1.215952in}}{\pgfqpoint{0.748864in}{1.221776in}}%
\pgfpathcurveto{\pgfqpoint{0.754688in}{1.227599in}}{\pgfqpoint{0.757960in}{1.235500in}}{\pgfqpoint{0.757960in}{1.243736in}}%
\pgfpathcurveto{\pgfqpoint{0.757960in}{1.251972in}}{\pgfqpoint{0.754688in}{1.259872in}}{\pgfqpoint{0.748864in}{1.265696in}}%
\pgfpathcurveto{\pgfqpoint{0.743040in}{1.271520in}}{\pgfqpoint{0.735140in}{1.274792in}}{\pgfqpoint{0.726904in}{1.274792in}}%
\pgfpathcurveto{\pgfqpoint{0.718667in}{1.274792in}}{\pgfqpoint{0.710767in}{1.271520in}}{\pgfqpoint{0.704943in}{1.265696in}}%
\pgfpathcurveto{\pgfqpoint{0.699119in}{1.259872in}}{\pgfqpoint{0.695847in}{1.251972in}}{\pgfqpoint{0.695847in}{1.243736in}}%
\pgfpathcurveto{\pgfqpoint{0.695847in}{1.235500in}}{\pgfqpoint{0.699119in}{1.227599in}}{\pgfqpoint{0.704943in}{1.221776in}}%
\pgfpathcurveto{\pgfqpoint{0.710767in}{1.215952in}}{\pgfqpoint{0.718667in}{1.212679in}}{\pgfqpoint{0.726904in}{1.212679in}}%
\pgfpathclose%
\pgfusepath{stroke,fill}%
\end{pgfscope}%
\begin{pgfscope}%
\pgfpathrectangle{\pgfqpoint{0.100000in}{0.212622in}}{\pgfqpoint{3.696000in}{3.696000in}}%
\pgfusepath{clip}%
\pgfsetbuttcap%
\pgfsetroundjoin%
\definecolor{currentfill}{rgb}{0.121569,0.466667,0.705882}%
\pgfsetfillcolor{currentfill}%
\pgfsetfillopacity{0.663896}%
\pgfsetlinewidth{1.003750pt}%
\definecolor{currentstroke}{rgb}{0.121569,0.466667,0.705882}%
\pgfsetstrokecolor{currentstroke}%
\pgfsetstrokeopacity{0.663896}%
\pgfsetdash{}{0pt}%
\pgfpathmoveto{\pgfqpoint{0.726903in}{1.212679in}}%
\pgfpathcurveto{\pgfqpoint{0.735140in}{1.212679in}}{\pgfqpoint{0.743040in}{1.215952in}}{\pgfqpoint{0.748864in}{1.221776in}}%
\pgfpathcurveto{\pgfqpoint{0.754688in}{1.227599in}}{\pgfqpoint{0.757960in}{1.235500in}}{\pgfqpoint{0.757960in}{1.243736in}}%
\pgfpathcurveto{\pgfqpoint{0.757960in}{1.251972in}}{\pgfqpoint{0.754688in}{1.259872in}}{\pgfqpoint{0.748864in}{1.265696in}}%
\pgfpathcurveto{\pgfqpoint{0.743040in}{1.271520in}}{\pgfqpoint{0.735140in}{1.274792in}}{\pgfqpoint{0.726903in}{1.274792in}}%
\pgfpathcurveto{\pgfqpoint{0.718667in}{1.274792in}}{\pgfqpoint{0.710767in}{1.271520in}}{\pgfqpoint{0.704943in}{1.265696in}}%
\pgfpathcurveto{\pgfqpoint{0.699119in}{1.259872in}}{\pgfqpoint{0.695847in}{1.251972in}}{\pgfqpoint{0.695847in}{1.243736in}}%
\pgfpathcurveto{\pgfqpoint{0.695847in}{1.235500in}}{\pgfqpoint{0.699119in}{1.227599in}}{\pgfqpoint{0.704943in}{1.221776in}}%
\pgfpathcurveto{\pgfqpoint{0.710767in}{1.215952in}}{\pgfqpoint{0.718667in}{1.212679in}}{\pgfqpoint{0.726903in}{1.212679in}}%
\pgfpathclose%
\pgfusepath{stroke,fill}%
\end{pgfscope}%
\begin{pgfscope}%
\pgfpathrectangle{\pgfqpoint{0.100000in}{0.212622in}}{\pgfqpoint{3.696000in}{3.696000in}}%
\pgfusepath{clip}%
\pgfsetbuttcap%
\pgfsetroundjoin%
\definecolor{currentfill}{rgb}{0.121569,0.466667,0.705882}%
\pgfsetfillcolor{currentfill}%
\pgfsetfillopacity{0.663896}%
\pgfsetlinewidth{1.003750pt}%
\definecolor{currentstroke}{rgb}{0.121569,0.466667,0.705882}%
\pgfsetstrokecolor{currentstroke}%
\pgfsetstrokeopacity{0.663896}%
\pgfsetdash{}{0pt}%
\pgfpathmoveto{\pgfqpoint{0.726903in}{1.212679in}}%
\pgfpathcurveto{\pgfqpoint{0.735140in}{1.212679in}}{\pgfqpoint{0.743040in}{1.215952in}}{\pgfqpoint{0.748864in}{1.221776in}}%
\pgfpathcurveto{\pgfqpoint{0.754688in}{1.227599in}}{\pgfqpoint{0.757960in}{1.235500in}}{\pgfqpoint{0.757960in}{1.243736in}}%
\pgfpathcurveto{\pgfqpoint{0.757960in}{1.251972in}}{\pgfqpoint{0.754688in}{1.259872in}}{\pgfqpoint{0.748864in}{1.265696in}}%
\pgfpathcurveto{\pgfqpoint{0.743040in}{1.271520in}}{\pgfqpoint{0.735140in}{1.274792in}}{\pgfqpoint{0.726903in}{1.274792in}}%
\pgfpathcurveto{\pgfqpoint{0.718667in}{1.274792in}}{\pgfqpoint{0.710767in}{1.271520in}}{\pgfqpoint{0.704943in}{1.265696in}}%
\pgfpathcurveto{\pgfqpoint{0.699119in}{1.259872in}}{\pgfqpoint{0.695847in}{1.251972in}}{\pgfqpoint{0.695847in}{1.243736in}}%
\pgfpathcurveto{\pgfqpoint{0.695847in}{1.235500in}}{\pgfqpoint{0.699119in}{1.227599in}}{\pgfqpoint{0.704943in}{1.221776in}}%
\pgfpathcurveto{\pgfqpoint{0.710767in}{1.215952in}}{\pgfqpoint{0.718667in}{1.212679in}}{\pgfqpoint{0.726903in}{1.212679in}}%
\pgfpathclose%
\pgfusepath{stroke,fill}%
\end{pgfscope}%
\begin{pgfscope}%
\pgfpathrectangle{\pgfqpoint{0.100000in}{0.212622in}}{\pgfqpoint{3.696000in}{3.696000in}}%
\pgfusepath{clip}%
\pgfsetbuttcap%
\pgfsetroundjoin%
\definecolor{currentfill}{rgb}{0.121569,0.466667,0.705882}%
\pgfsetfillcolor{currentfill}%
\pgfsetfillopacity{0.663896}%
\pgfsetlinewidth{1.003750pt}%
\definecolor{currentstroke}{rgb}{0.121569,0.466667,0.705882}%
\pgfsetstrokecolor{currentstroke}%
\pgfsetstrokeopacity{0.663896}%
\pgfsetdash{}{0pt}%
\pgfpathmoveto{\pgfqpoint{0.726903in}{1.212679in}}%
\pgfpathcurveto{\pgfqpoint{0.735140in}{1.212679in}}{\pgfqpoint{0.743040in}{1.215952in}}{\pgfqpoint{0.748864in}{1.221776in}}%
\pgfpathcurveto{\pgfqpoint{0.754688in}{1.227599in}}{\pgfqpoint{0.757960in}{1.235500in}}{\pgfqpoint{0.757960in}{1.243736in}}%
\pgfpathcurveto{\pgfqpoint{0.757960in}{1.251972in}}{\pgfqpoint{0.754688in}{1.259872in}}{\pgfqpoint{0.748864in}{1.265696in}}%
\pgfpathcurveto{\pgfqpoint{0.743040in}{1.271520in}}{\pgfqpoint{0.735140in}{1.274792in}}{\pgfqpoint{0.726903in}{1.274792in}}%
\pgfpathcurveto{\pgfqpoint{0.718667in}{1.274792in}}{\pgfqpoint{0.710767in}{1.271520in}}{\pgfqpoint{0.704943in}{1.265696in}}%
\pgfpathcurveto{\pgfqpoint{0.699119in}{1.259872in}}{\pgfqpoint{0.695847in}{1.251972in}}{\pgfqpoint{0.695847in}{1.243736in}}%
\pgfpathcurveto{\pgfqpoint{0.695847in}{1.235500in}}{\pgfqpoint{0.699119in}{1.227599in}}{\pgfqpoint{0.704943in}{1.221776in}}%
\pgfpathcurveto{\pgfqpoint{0.710767in}{1.215952in}}{\pgfqpoint{0.718667in}{1.212679in}}{\pgfqpoint{0.726903in}{1.212679in}}%
\pgfpathclose%
\pgfusepath{stroke,fill}%
\end{pgfscope}%
\begin{pgfscope}%
\pgfpathrectangle{\pgfqpoint{0.100000in}{0.212622in}}{\pgfqpoint{3.696000in}{3.696000in}}%
\pgfusepath{clip}%
\pgfsetbuttcap%
\pgfsetroundjoin%
\definecolor{currentfill}{rgb}{0.121569,0.466667,0.705882}%
\pgfsetfillcolor{currentfill}%
\pgfsetfillopacity{0.663896}%
\pgfsetlinewidth{1.003750pt}%
\definecolor{currentstroke}{rgb}{0.121569,0.466667,0.705882}%
\pgfsetstrokecolor{currentstroke}%
\pgfsetstrokeopacity{0.663896}%
\pgfsetdash{}{0pt}%
\pgfpathmoveto{\pgfqpoint{0.726903in}{1.212679in}}%
\pgfpathcurveto{\pgfqpoint{0.735140in}{1.212679in}}{\pgfqpoint{0.743040in}{1.215952in}}{\pgfqpoint{0.748864in}{1.221776in}}%
\pgfpathcurveto{\pgfqpoint{0.754688in}{1.227599in}}{\pgfqpoint{0.757960in}{1.235500in}}{\pgfqpoint{0.757960in}{1.243736in}}%
\pgfpathcurveto{\pgfqpoint{0.757960in}{1.251972in}}{\pgfqpoint{0.754688in}{1.259872in}}{\pgfqpoint{0.748864in}{1.265696in}}%
\pgfpathcurveto{\pgfqpoint{0.743040in}{1.271520in}}{\pgfqpoint{0.735140in}{1.274792in}}{\pgfqpoint{0.726903in}{1.274792in}}%
\pgfpathcurveto{\pgfqpoint{0.718667in}{1.274792in}}{\pgfqpoint{0.710767in}{1.271520in}}{\pgfqpoint{0.704943in}{1.265696in}}%
\pgfpathcurveto{\pgfqpoint{0.699119in}{1.259872in}}{\pgfqpoint{0.695847in}{1.251972in}}{\pgfqpoint{0.695847in}{1.243736in}}%
\pgfpathcurveto{\pgfqpoint{0.695847in}{1.235500in}}{\pgfqpoint{0.699119in}{1.227599in}}{\pgfqpoint{0.704943in}{1.221776in}}%
\pgfpathcurveto{\pgfqpoint{0.710767in}{1.215952in}}{\pgfqpoint{0.718667in}{1.212679in}}{\pgfqpoint{0.726903in}{1.212679in}}%
\pgfpathclose%
\pgfusepath{stroke,fill}%
\end{pgfscope}%
\begin{pgfscope}%
\pgfpathrectangle{\pgfqpoint{0.100000in}{0.212622in}}{\pgfqpoint{3.696000in}{3.696000in}}%
\pgfusepath{clip}%
\pgfsetbuttcap%
\pgfsetroundjoin%
\definecolor{currentfill}{rgb}{0.121569,0.466667,0.705882}%
\pgfsetfillcolor{currentfill}%
\pgfsetfillopacity{0.663896}%
\pgfsetlinewidth{1.003750pt}%
\definecolor{currentstroke}{rgb}{0.121569,0.466667,0.705882}%
\pgfsetstrokecolor{currentstroke}%
\pgfsetstrokeopacity{0.663896}%
\pgfsetdash{}{0pt}%
\pgfpathmoveto{\pgfqpoint{0.726903in}{1.212679in}}%
\pgfpathcurveto{\pgfqpoint{0.735140in}{1.212679in}}{\pgfqpoint{0.743040in}{1.215952in}}{\pgfqpoint{0.748864in}{1.221776in}}%
\pgfpathcurveto{\pgfqpoint{0.754688in}{1.227599in}}{\pgfqpoint{0.757960in}{1.235500in}}{\pgfqpoint{0.757960in}{1.243736in}}%
\pgfpathcurveto{\pgfqpoint{0.757960in}{1.251972in}}{\pgfqpoint{0.754688in}{1.259872in}}{\pgfqpoint{0.748864in}{1.265696in}}%
\pgfpathcurveto{\pgfqpoint{0.743040in}{1.271520in}}{\pgfqpoint{0.735140in}{1.274792in}}{\pgfqpoint{0.726903in}{1.274792in}}%
\pgfpathcurveto{\pgfqpoint{0.718667in}{1.274792in}}{\pgfqpoint{0.710767in}{1.271520in}}{\pgfqpoint{0.704943in}{1.265696in}}%
\pgfpathcurveto{\pgfqpoint{0.699119in}{1.259872in}}{\pgfqpoint{0.695847in}{1.251972in}}{\pgfqpoint{0.695847in}{1.243736in}}%
\pgfpathcurveto{\pgfqpoint{0.695847in}{1.235500in}}{\pgfqpoint{0.699119in}{1.227599in}}{\pgfqpoint{0.704943in}{1.221776in}}%
\pgfpathcurveto{\pgfqpoint{0.710767in}{1.215952in}}{\pgfqpoint{0.718667in}{1.212679in}}{\pgfqpoint{0.726903in}{1.212679in}}%
\pgfpathclose%
\pgfusepath{stroke,fill}%
\end{pgfscope}%
\begin{pgfscope}%
\pgfpathrectangle{\pgfqpoint{0.100000in}{0.212622in}}{\pgfqpoint{3.696000in}{3.696000in}}%
\pgfusepath{clip}%
\pgfsetbuttcap%
\pgfsetroundjoin%
\definecolor{currentfill}{rgb}{0.121569,0.466667,0.705882}%
\pgfsetfillcolor{currentfill}%
\pgfsetfillopacity{0.663896}%
\pgfsetlinewidth{1.003750pt}%
\definecolor{currentstroke}{rgb}{0.121569,0.466667,0.705882}%
\pgfsetstrokecolor{currentstroke}%
\pgfsetstrokeopacity{0.663896}%
\pgfsetdash{}{0pt}%
\pgfpathmoveto{\pgfqpoint{0.726903in}{1.212679in}}%
\pgfpathcurveto{\pgfqpoint{0.735140in}{1.212679in}}{\pgfqpoint{0.743040in}{1.215952in}}{\pgfqpoint{0.748864in}{1.221776in}}%
\pgfpathcurveto{\pgfqpoint{0.754688in}{1.227599in}}{\pgfqpoint{0.757960in}{1.235500in}}{\pgfqpoint{0.757960in}{1.243736in}}%
\pgfpathcurveto{\pgfqpoint{0.757960in}{1.251972in}}{\pgfqpoint{0.754688in}{1.259872in}}{\pgfqpoint{0.748864in}{1.265696in}}%
\pgfpathcurveto{\pgfqpoint{0.743040in}{1.271520in}}{\pgfqpoint{0.735140in}{1.274792in}}{\pgfqpoint{0.726903in}{1.274792in}}%
\pgfpathcurveto{\pgfqpoint{0.718667in}{1.274792in}}{\pgfqpoint{0.710767in}{1.271520in}}{\pgfqpoint{0.704943in}{1.265696in}}%
\pgfpathcurveto{\pgfqpoint{0.699119in}{1.259872in}}{\pgfqpoint{0.695847in}{1.251972in}}{\pgfqpoint{0.695847in}{1.243736in}}%
\pgfpathcurveto{\pgfqpoint{0.695847in}{1.235500in}}{\pgfqpoint{0.699119in}{1.227599in}}{\pgfqpoint{0.704943in}{1.221776in}}%
\pgfpathcurveto{\pgfqpoint{0.710767in}{1.215952in}}{\pgfqpoint{0.718667in}{1.212679in}}{\pgfqpoint{0.726903in}{1.212679in}}%
\pgfpathclose%
\pgfusepath{stroke,fill}%
\end{pgfscope}%
\begin{pgfscope}%
\pgfpathrectangle{\pgfqpoint{0.100000in}{0.212622in}}{\pgfqpoint{3.696000in}{3.696000in}}%
\pgfusepath{clip}%
\pgfsetbuttcap%
\pgfsetroundjoin%
\definecolor{currentfill}{rgb}{0.121569,0.466667,0.705882}%
\pgfsetfillcolor{currentfill}%
\pgfsetfillopacity{0.663896}%
\pgfsetlinewidth{1.003750pt}%
\definecolor{currentstroke}{rgb}{0.121569,0.466667,0.705882}%
\pgfsetstrokecolor{currentstroke}%
\pgfsetstrokeopacity{0.663896}%
\pgfsetdash{}{0pt}%
\pgfpathmoveto{\pgfqpoint{0.726903in}{1.212679in}}%
\pgfpathcurveto{\pgfqpoint{0.735140in}{1.212679in}}{\pgfqpoint{0.743040in}{1.215952in}}{\pgfqpoint{0.748864in}{1.221776in}}%
\pgfpathcurveto{\pgfqpoint{0.754688in}{1.227599in}}{\pgfqpoint{0.757960in}{1.235500in}}{\pgfqpoint{0.757960in}{1.243736in}}%
\pgfpathcurveto{\pgfqpoint{0.757960in}{1.251972in}}{\pgfqpoint{0.754688in}{1.259872in}}{\pgfqpoint{0.748864in}{1.265696in}}%
\pgfpathcurveto{\pgfqpoint{0.743040in}{1.271520in}}{\pgfqpoint{0.735140in}{1.274792in}}{\pgfqpoint{0.726903in}{1.274792in}}%
\pgfpathcurveto{\pgfqpoint{0.718667in}{1.274792in}}{\pgfqpoint{0.710767in}{1.271520in}}{\pgfqpoint{0.704943in}{1.265696in}}%
\pgfpathcurveto{\pgfqpoint{0.699119in}{1.259872in}}{\pgfqpoint{0.695847in}{1.251972in}}{\pgfqpoint{0.695847in}{1.243736in}}%
\pgfpathcurveto{\pgfqpoint{0.695847in}{1.235500in}}{\pgfqpoint{0.699119in}{1.227599in}}{\pgfqpoint{0.704943in}{1.221776in}}%
\pgfpathcurveto{\pgfqpoint{0.710767in}{1.215952in}}{\pgfqpoint{0.718667in}{1.212679in}}{\pgfqpoint{0.726903in}{1.212679in}}%
\pgfpathclose%
\pgfusepath{stroke,fill}%
\end{pgfscope}%
\begin{pgfscope}%
\pgfpathrectangle{\pgfqpoint{0.100000in}{0.212622in}}{\pgfqpoint{3.696000in}{3.696000in}}%
\pgfusepath{clip}%
\pgfsetbuttcap%
\pgfsetroundjoin%
\definecolor{currentfill}{rgb}{0.121569,0.466667,0.705882}%
\pgfsetfillcolor{currentfill}%
\pgfsetfillopacity{0.663896}%
\pgfsetlinewidth{1.003750pt}%
\definecolor{currentstroke}{rgb}{0.121569,0.466667,0.705882}%
\pgfsetstrokecolor{currentstroke}%
\pgfsetstrokeopacity{0.663896}%
\pgfsetdash{}{0pt}%
\pgfpathmoveto{\pgfqpoint{0.726903in}{1.212679in}}%
\pgfpathcurveto{\pgfqpoint{0.735140in}{1.212679in}}{\pgfqpoint{0.743040in}{1.215952in}}{\pgfqpoint{0.748864in}{1.221776in}}%
\pgfpathcurveto{\pgfqpoint{0.754688in}{1.227599in}}{\pgfqpoint{0.757960in}{1.235500in}}{\pgfqpoint{0.757960in}{1.243736in}}%
\pgfpathcurveto{\pgfqpoint{0.757960in}{1.251972in}}{\pgfqpoint{0.754688in}{1.259872in}}{\pgfqpoint{0.748864in}{1.265696in}}%
\pgfpathcurveto{\pgfqpoint{0.743040in}{1.271520in}}{\pgfqpoint{0.735140in}{1.274792in}}{\pgfqpoint{0.726903in}{1.274792in}}%
\pgfpathcurveto{\pgfqpoint{0.718667in}{1.274792in}}{\pgfqpoint{0.710767in}{1.271520in}}{\pgfqpoint{0.704943in}{1.265696in}}%
\pgfpathcurveto{\pgfqpoint{0.699119in}{1.259872in}}{\pgfqpoint{0.695847in}{1.251972in}}{\pgfqpoint{0.695847in}{1.243736in}}%
\pgfpathcurveto{\pgfqpoint{0.695847in}{1.235500in}}{\pgfqpoint{0.699119in}{1.227599in}}{\pgfqpoint{0.704943in}{1.221776in}}%
\pgfpathcurveto{\pgfqpoint{0.710767in}{1.215952in}}{\pgfqpoint{0.718667in}{1.212679in}}{\pgfqpoint{0.726903in}{1.212679in}}%
\pgfpathclose%
\pgfusepath{stroke,fill}%
\end{pgfscope}%
\begin{pgfscope}%
\pgfpathrectangle{\pgfqpoint{0.100000in}{0.212622in}}{\pgfqpoint{3.696000in}{3.696000in}}%
\pgfusepath{clip}%
\pgfsetbuttcap%
\pgfsetroundjoin%
\definecolor{currentfill}{rgb}{0.121569,0.466667,0.705882}%
\pgfsetfillcolor{currentfill}%
\pgfsetfillopacity{0.663896}%
\pgfsetlinewidth{1.003750pt}%
\definecolor{currentstroke}{rgb}{0.121569,0.466667,0.705882}%
\pgfsetstrokecolor{currentstroke}%
\pgfsetstrokeopacity{0.663896}%
\pgfsetdash{}{0pt}%
\pgfpathmoveto{\pgfqpoint{0.726903in}{1.212679in}}%
\pgfpathcurveto{\pgfqpoint{0.735140in}{1.212679in}}{\pgfqpoint{0.743040in}{1.215952in}}{\pgfqpoint{0.748864in}{1.221776in}}%
\pgfpathcurveto{\pgfqpoint{0.754688in}{1.227599in}}{\pgfqpoint{0.757960in}{1.235500in}}{\pgfqpoint{0.757960in}{1.243736in}}%
\pgfpathcurveto{\pgfqpoint{0.757960in}{1.251972in}}{\pgfqpoint{0.754688in}{1.259872in}}{\pgfqpoint{0.748864in}{1.265696in}}%
\pgfpathcurveto{\pgfqpoint{0.743040in}{1.271520in}}{\pgfqpoint{0.735140in}{1.274792in}}{\pgfqpoint{0.726903in}{1.274792in}}%
\pgfpathcurveto{\pgfqpoint{0.718667in}{1.274792in}}{\pgfqpoint{0.710767in}{1.271520in}}{\pgfqpoint{0.704943in}{1.265696in}}%
\pgfpathcurveto{\pgfqpoint{0.699119in}{1.259872in}}{\pgfqpoint{0.695847in}{1.251972in}}{\pgfqpoint{0.695847in}{1.243736in}}%
\pgfpathcurveto{\pgfqpoint{0.695847in}{1.235500in}}{\pgfqpoint{0.699119in}{1.227599in}}{\pgfqpoint{0.704943in}{1.221776in}}%
\pgfpathcurveto{\pgfqpoint{0.710767in}{1.215952in}}{\pgfqpoint{0.718667in}{1.212679in}}{\pgfqpoint{0.726903in}{1.212679in}}%
\pgfpathclose%
\pgfusepath{stroke,fill}%
\end{pgfscope}%
\begin{pgfscope}%
\pgfpathrectangle{\pgfqpoint{0.100000in}{0.212622in}}{\pgfqpoint{3.696000in}{3.696000in}}%
\pgfusepath{clip}%
\pgfsetbuttcap%
\pgfsetroundjoin%
\definecolor{currentfill}{rgb}{0.121569,0.466667,0.705882}%
\pgfsetfillcolor{currentfill}%
\pgfsetfillopacity{0.663896}%
\pgfsetlinewidth{1.003750pt}%
\definecolor{currentstroke}{rgb}{0.121569,0.466667,0.705882}%
\pgfsetstrokecolor{currentstroke}%
\pgfsetstrokeopacity{0.663896}%
\pgfsetdash{}{0pt}%
\pgfpathmoveto{\pgfqpoint{0.726903in}{1.212679in}}%
\pgfpathcurveto{\pgfqpoint{0.735140in}{1.212679in}}{\pgfqpoint{0.743040in}{1.215952in}}{\pgfqpoint{0.748864in}{1.221776in}}%
\pgfpathcurveto{\pgfqpoint{0.754688in}{1.227599in}}{\pgfqpoint{0.757960in}{1.235500in}}{\pgfqpoint{0.757960in}{1.243736in}}%
\pgfpathcurveto{\pgfqpoint{0.757960in}{1.251972in}}{\pgfqpoint{0.754688in}{1.259872in}}{\pgfqpoint{0.748864in}{1.265696in}}%
\pgfpathcurveto{\pgfqpoint{0.743040in}{1.271520in}}{\pgfqpoint{0.735140in}{1.274792in}}{\pgfqpoint{0.726903in}{1.274792in}}%
\pgfpathcurveto{\pgfqpoint{0.718667in}{1.274792in}}{\pgfqpoint{0.710767in}{1.271520in}}{\pgfqpoint{0.704943in}{1.265696in}}%
\pgfpathcurveto{\pgfqpoint{0.699119in}{1.259872in}}{\pgfqpoint{0.695847in}{1.251972in}}{\pgfqpoint{0.695847in}{1.243736in}}%
\pgfpathcurveto{\pgfqpoint{0.695847in}{1.235500in}}{\pgfqpoint{0.699119in}{1.227599in}}{\pgfqpoint{0.704943in}{1.221776in}}%
\pgfpathcurveto{\pgfqpoint{0.710767in}{1.215952in}}{\pgfqpoint{0.718667in}{1.212679in}}{\pgfqpoint{0.726903in}{1.212679in}}%
\pgfpathclose%
\pgfusepath{stroke,fill}%
\end{pgfscope}%
\begin{pgfscope}%
\pgfpathrectangle{\pgfqpoint{0.100000in}{0.212622in}}{\pgfqpoint{3.696000in}{3.696000in}}%
\pgfusepath{clip}%
\pgfsetbuttcap%
\pgfsetroundjoin%
\definecolor{currentfill}{rgb}{0.121569,0.466667,0.705882}%
\pgfsetfillcolor{currentfill}%
\pgfsetfillopacity{0.663896}%
\pgfsetlinewidth{1.003750pt}%
\definecolor{currentstroke}{rgb}{0.121569,0.466667,0.705882}%
\pgfsetstrokecolor{currentstroke}%
\pgfsetstrokeopacity{0.663896}%
\pgfsetdash{}{0pt}%
\pgfpathmoveto{\pgfqpoint{0.726903in}{1.212679in}}%
\pgfpathcurveto{\pgfqpoint{0.735140in}{1.212679in}}{\pgfqpoint{0.743040in}{1.215952in}}{\pgfqpoint{0.748864in}{1.221776in}}%
\pgfpathcurveto{\pgfqpoint{0.754688in}{1.227599in}}{\pgfqpoint{0.757960in}{1.235500in}}{\pgfqpoint{0.757960in}{1.243736in}}%
\pgfpathcurveto{\pgfqpoint{0.757960in}{1.251972in}}{\pgfqpoint{0.754688in}{1.259872in}}{\pgfqpoint{0.748864in}{1.265696in}}%
\pgfpathcurveto{\pgfqpoint{0.743040in}{1.271520in}}{\pgfqpoint{0.735140in}{1.274792in}}{\pgfqpoint{0.726903in}{1.274792in}}%
\pgfpathcurveto{\pgfqpoint{0.718667in}{1.274792in}}{\pgfqpoint{0.710767in}{1.271520in}}{\pgfqpoint{0.704943in}{1.265696in}}%
\pgfpathcurveto{\pgfqpoint{0.699119in}{1.259872in}}{\pgfqpoint{0.695847in}{1.251972in}}{\pgfqpoint{0.695847in}{1.243736in}}%
\pgfpathcurveto{\pgfqpoint{0.695847in}{1.235500in}}{\pgfqpoint{0.699119in}{1.227599in}}{\pgfqpoint{0.704943in}{1.221776in}}%
\pgfpathcurveto{\pgfqpoint{0.710767in}{1.215952in}}{\pgfqpoint{0.718667in}{1.212679in}}{\pgfqpoint{0.726903in}{1.212679in}}%
\pgfpathclose%
\pgfusepath{stroke,fill}%
\end{pgfscope}%
\begin{pgfscope}%
\pgfpathrectangle{\pgfqpoint{0.100000in}{0.212622in}}{\pgfqpoint{3.696000in}{3.696000in}}%
\pgfusepath{clip}%
\pgfsetbuttcap%
\pgfsetroundjoin%
\definecolor{currentfill}{rgb}{0.121569,0.466667,0.705882}%
\pgfsetfillcolor{currentfill}%
\pgfsetfillopacity{0.663896}%
\pgfsetlinewidth{1.003750pt}%
\definecolor{currentstroke}{rgb}{0.121569,0.466667,0.705882}%
\pgfsetstrokecolor{currentstroke}%
\pgfsetstrokeopacity{0.663896}%
\pgfsetdash{}{0pt}%
\pgfpathmoveto{\pgfqpoint{0.726903in}{1.212679in}}%
\pgfpathcurveto{\pgfqpoint{0.735140in}{1.212679in}}{\pgfqpoint{0.743040in}{1.215952in}}{\pgfqpoint{0.748864in}{1.221776in}}%
\pgfpathcurveto{\pgfqpoint{0.754688in}{1.227599in}}{\pgfqpoint{0.757960in}{1.235500in}}{\pgfqpoint{0.757960in}{1.243736in}}%
\pgfpathcurveto{\pgfqpoint{0.757960in}{1.251972in}}{\pgfqpoint{0.754688in}{1.259872in}}{\pgfqpoint{0.748864in}{1.265696in}}%
\pgfpathcurveto{\pgfqpoint{0.743040in}{1.271520in}}{\pgfqpoint{0.735140in}{1.274792in}}{\pgfqpoint{0.726903in}{1.274792in}}%
\pgfpathcurveto{\pgfqpoint{0.718667in}{1.274792in}}{\pgfqpoint{0.710767in}{1.271520in}}{\pgfqpoint{0.704943in}{1.265696in}}%
\pgfpathcurveto{\pgfqpoint{0.699119in}{1.259872in}}{\pgfqpoint{0.695847in}{1.251972in}}{\pgfqpoint{0.695847in}{1.243736in}}%
\pgfpathcurveto{\pgfqpoint{0.695847in}{1.235500in}}{\pgfqpoint{0.699119in}{1.227599in}}{\pgfqpoint{0.704943in}{1.221776in}}%
\pgfpathcurveto{\pgfqpoint{0.710767in}{1.215952in}}{\pgfqpoint{0.718667in}{1.212679in}}{\pgfqpoint{0.726903in}{1.212679in}}%
\pgfpathclose%
\pgfusepath{stroke,fill}%
\end{pgfscope}%
\begin{pgfscope}%
\pgfpathrectangle{\pgfqpoint{0.100000in}{0.212622in}}{\pgfqpoint{3.696000in}{3.696000in}}%
\pgfusepath{clip}%
\pgfsetbuttcap%
\pgfsetroundjoin%
\definecolor{currentfill}{rgb}{0.121569,0.466667,0.705882}%
\pgfsetfillcolor{currentfill}%
\pgfsetfillopacity{0.663896}%
\pgfsetlinewidth{1.003750pt}%
\definecolor{currentstroke}{rgb}{0.121569,0.466667,0.705882}%
\pgfsetstrokecolor{currentstroke}%
\pgfsetstrokeopacity{0.663896}%
\pgfsetdash{}{0pt}%
\pgfpathmoveto{\pgfqpoint{0.726903in}{1.212679in}}%
\pgfpathcurveto{\pgfqpoint{0.735140in}{1.212679in}}{\pgfqpoint{0.743040in}{1.215952in}}{\pgfqpoint{0.748864in}{1.221776in}}%
\pgfpathcurveto{\pgfqpoint{0.754688in}{1.227599in}}{\pgfqpoint{0.757960in}{1.235500in}}{\pgfqpoint{0.757960in}{1.243736in}}%
\pgfpathcurveto{\pgfqpoint{0.757960in}{1.251972in}}{\pgfqpoint{0.754688in}{1.259872in}}{\pgfqpoint{0.748864in}{1.265696in}}%
\pgfpathcurveto{\pgfqpoint{0.743040in}{1.271520in}}{\pgfqpoint{0.735140in}{1.274792in}}{\pgfqpoint{0.726903in}{1.274792in}}%
\pgfpathcurveto{\pgfqpoint{0.718667in}{1.274792in}}{\pgfqpoint{0.710767in}{1.271520in}}{\pgfqpoint{0.704943in}{1.265696in}}%
\pgfpathcurveto{\pgfqpoint{0.699119in}{1.259872in}}{\pgfqpoint{0.695847in}{1.251972in}}{\pgfqpoint{0.695847in}{1.243736in}}%
\pgfpathcurveto{\pgfqpoint{0.695847in}{1.235500in}}{\pgfqpoint{0.699119in}{1.227599in}}{\pgfqpoint{0.704943in}{1.221776in}}%
\pgfpathcurveto{\pgfqpoint{0.710767in}{1.215952in}}{\pgfqpoint{0.718667in}{1.212679in}}{\pgfqpoint{0.726903in}{1.212679in}}%
\pgfpathclose%
\pgfusepath{stroke,fill}%
\end{pgfscope}%
\begin{pgfscope}%
\pgfpathrectangle{\pgfqpoint{0.100000in}{0.212622in}}{\pgfqpoint{3.696000in}{3.696000in}}%
\pgfusepath{clip}%
\pgfsetbuttcap%
\pgfsetroundjoin%
\definecolor{currentfill}{rgb}{0.121569,0.466667,0.705882}%
\pgfsetfillcolor{currentfill}%
\pgfsetfillopacity{0.663896}%
\pgfsetlinewidth{1.003750pt}%
\definecolor{currentstroke}{rgb}{0.121569,0.466667,0.705882}%
\pgfsetstrokecolor{currentstroke}%
\pgfsetstrokeopacity{0.663896}%
\pgfsetdash{}{0pt}%
\pgfpathmoveto{\pgfqpoint{0.726903in}{1.212679in}}%
\pgfpathcurveto{\pgfqpoint{0.735140in}{1.212679in}}{\pgfqpoint{0.743040in}{1.215952in}}{\pgfqpoint{0.748864in}{1.221776in}}%
\pgfpathcurveto{\pgfqpoint{0.754688in}{1.227599in}}{\pgfqpoint{0.757960in}{1.235500in}}{\pgfqpoint{0.757960in}{1.243736in}}%
\pgfpathcurveto{\pgfqpoint{0.757960in}{1.251972in}}{\pgfqpoint{0.754688in}{1.259872in}}{\pgfqpoint{0.748864in}{1.265696in}}%
\pgfpathcurveto{\pgfqpoint{0.743040in}{1.271520in}}{\pgfqpoint{0.735140in}{1.274792in}}{\pgfqpoint{0.726903in}{1.274792in}}%
\pgfpathcurveto{\pgfqpoint{0.718667in}{1.274792in}}{\pgfqpoint{0.710767in}{1.271520in}}{\pgfqpoint{0.704943in}{1.265696in}}%
\pgfpathcurveto{\pgfqpoint{0.699119in}{1.259872in}}{\pgfqpoint{0.695847in}{1.251972in}}{\pgfqpoint{0.695847in}{1.243736in}}%
\pgfpathcurveto{\pgfqpoint{0.695847in}{1.235500in}}{\pgfqpoint{0.699119in}{1.227599in}}{\pgfqpoint{0.704943in}{1.221776in}}%
\pgfpathcurveto{\pgfqpoint{0.710767in}{1.215952in}}{\pgfqpoint{0.718667in}{1.212679in}}{\pgfqpoint{0.726903in}{1.212679in}}%
\pgfpathclose%
\pgfusepath{stroke,fill}%
\end{pgfscope}%
\begin{pgfscope}%
\pgfpathrectangle{\pgfqpoint{0.100000in}{0.212622in}}{\pgfqpoint{3.696000in}{3.696000in}}%
\pgfusepath{clip}%
\pgfsetbuttcap%
\pgfsetroundjoin%
\definecolor{currentfill}{rgb}{0.121569,0.466667,0.705882}%
\pgfsetfillcolor{currentfill}%
\pgfsetfillopacity{0.663896}%
\pgfsetlinewidth{1.003750pt}%
\definecolor{currentstroke}{rgb}{0.121569,0.466667,0.705882}%
\pgfsetstrokecolor{currentstroke}%
\pgfsetstrokeopacity{0.663896}%
\pgfsetdash{}{0pt}%
\pgfpathmoveto{\pgfqpoint{0.726903in}{1.212679in}}%
\pgfpathcurveto{\pgfqpoint{0.735140in}{1.212679in}}{\pgfqpoint{0.743040in}{1.215952in}}{\pgfqpoint{0.748864in}{1.221776in}}%
\pgfpathcurveto{\pgfqpoint{0.754688in}{1.227599in}}{\pgfqpoint{0.757960in}{1.235500in}}{\pgfqpoint{0.757960in}{1.243736in}}%
\pgfpathcurveto{\pgfqpoint{0.757960in}{1.251972in}}{\pgfqpoint{0.754688in}{1.259872in}}{\pgfqpoint{0.748864in}{1.265696in}}%
\pgfpathcurveto{\pgfqpoint{0.743040in}{1.271520in}}{\pgfqpoint{0.735140in}{1.274792in}}{\pgfqpoint{0.726903in}{1.274792in}}%
\pgfpathcurveto{\pgfqpoint{0.718667in}{1.274792in}}{\pgfqpoint{0.710767in}{1.271520in}}{\pgfqpoint{0.704943in}{1.265696in}}%
\pgfpathcurveto{\pgfqpoint{0.699119in}{1.259872in}}{\pgfqpoint{0.695847in}{1.251972in}}{\pgfqpoint{0.695847in}{1.243736in}}%
\pgfpathcurveto{\pgfqpoint{0.695847in}{1.235500in}}{\pgfqpoint{0.699119in}{1.227599in}}{\pgfqpoint{0.704943in}{1.221776in}}%
\pgfpathcurveto{\pgfqpoint{0.710767in}{1.215952in}}{\pgfqpoint{0.718667in}{1.212679in}}{\pgfqpoint{0.726903in}{1.212679in}}%
\pgfpathclose%
\pgfusepath{stroke,fill}%
\end{pgfscope}%
\begin{pgfscope}%
\pgfpathrectangle{\pgfqpoint{0.100000in}{0.212622in}}{\pgfqpoint{3.696000in}{3.696000in}}%
\pgfusepath{clip}%
\pgfsetbuttcap%
\pgfsetroundjoin%
\definecolor{currentfill}{rgb}{0.121569,0.466667,0.705882}%
\pgfsetfillcolor{currentfill}%
\pgfsetfillopacity{0.663896}%
\pgfsetlinewidth{1.003750pt}%
\definecolor{currentstroke}{rgb}{0.121569,0.466667,0.705882}%
\pgfsetstrokecolor{currentstroke}%
\pgfsetstrokeopacity{0.663896}%
\pgfsetdash{}{0pt}%
\pgfpathmoveto{\pgfqpoint{0.726903in}{1.212679in}}%
\pgfpathcurveto{\pgfqpoint{0.735140in}{1.212679in}}{\pgfqpoint{0.743040in}{1.215952in}}{\pgfqpoint{0.748864in}{1.221776in}}%
\pgfpathcurveto{\pgfqpoint{0.754688in}{1.227599in}}{\pgfqpoint{0.757960in}{1.235500in}}{\pgfqpoint{0.757960in}{1.243736in}}%
\pgfpathcurveto{\pgfqpoint{0.757960in}{1.251972in}}{\pgfqpoint{0.754688in}{1.259872in}}{\pgfqpoint{0.748864in}{1.265696in}}%
\pgfpathcurveto{\pgfqpoint{0.743040in}{1.271520in}}{\pgfqpoint{0.735140in}{1.274792in}}{\pgfqpoint{0.726903in}{1.274792in}}%
\pgfpathcurveto{\pgfqpoint{0.718667in}{1.274792in}}{\pgfqpoint{0.710767in}{1.271520in}}{\pgfqpoint{0.704943in}{1.265696in}}%
\pgfpathcurveto{\pgfqpoint{0.699119in}{1.259872in}}{\pgfqpoint{0.695847in}{1.251972in}}{\pgfqpoint{0.695847in}{1.243736in}}%
\pgfpathcurveto{\pgfqpoint{0.695847in}{1.235500in}}{\pgfqpoint{0.699119in}{1.227599in}}{\pgfqpoint{0.704943in}{1.221776in}}%
\pgfpathcurveto{\pgfqpoint{0.710767in}{1.215952in}}{\pgfqpoint{0.718667in}{1.212679in}}{\pgfqpoint{0.726903in}{1.212679in}}%
\pgfpathclose%
\pgfusepath{stroke,fill}%
\end{pgfscope}%
\begin{pgfscope}%
\pgfpathrectangle{\pgfqpoint{0.100000in}{0.212622in}}{\pgfqpoint{3.696000in}{3.696000in}}%
\pgfusepath{clip}%
\pgfsetbuttcap%
\pgfsetroundjoin%
\definecolor{currentfill}{rgb}{0.121569,0.466667,0.705882}%
\pgfsetfillcolor{currentfill}%
\pgfsetfillopacity{0.663896}%
\pgfsetlinewidth{1.003750pt}%
\definecolor{currentstroke}{rgb}{0.121569,0.466667,0.705882}%
\pgfsetstrokecolor{currentstroke}%
\pgfsetstrokeopacity{0.663896}%
\pgfsetdash{}{0pt}%
\pgfpathmoveto{\pgfqpoint{0.726903in}{1.212679in}}%
\pgfpathcurveto{\pgfqpoint{0.735140in}{1.212679in}}{\pgfqpoint{0.743040in}{1.215952in}}{\pgfqpoint{0.748864in}{1.221776in}}%
\pgfpathcurveto{\pgfqpoint{0.754688in}{1.227599in}}{\pgfqpoint{0.757960in}{1.235500in}}{\pgfqpoint{0.757960in}{1.243736in}}%
\pgfpathcurveto{\pgfqpoint{0.757960in}{1.251972in}}{\pgfqpoint{0.754688in}{1.259872in}}{\pgfqpoint{0.748864in}{1.265696in}}%
\pgfpathcurveto{\pgfqpoint{0.743040in}{1.271520in}}{\pgfqpoint{0.735140in}{1.274792in}}{\pgfqpoint{0.726903in}{1.274792in}}%
\pgfpathcurveto{\pgfqpoint{0.718667in}{1.274792in}}{\pgfqpoint{0.710767in}{1.271520in}}{\pgfqpoint{0.704943in}{1.265696in}}%
\pgfpathcurveto{\pgfqpoint{0.699119in}{1.259872in}}{\pgfqpoint{0.695847in}{1.251972in}}{\pgfqpoint{0.695847in}{1.243736in}}%
\pgfpathcurveto{\pgfqpoint{0.695847in}{1.235500in}}{\pgfqpoint{0.699119in}{1.227599in}}{\pgfqpoint{0.704943in}{1.221776in}}%
\pgfpathcurveto{\pgfqpoint{0.710767in}{1.215952in}}{\pgfqpoint{0.718667in}{1.212679in}}{\pgfqpoint{0.726903in}{1.212679in}}%
\pgfpathclose%
\pgfusepath{stroke,fill}%
\end{pgfscope}%
\begin{pgfscope}%
\pgfpathrectangle{\pgfqpoint{0.100000in}{0.212622in}}{\pgfqpoint{3.696000in}{3.696000in}}%
\pgfusepath{clip}%
\pgfsetbuttcap%
\pgfsetroundjoin%
\definecolor{currentfill}{rgb}{0.121569,0.466667,0.705882}%
\pgfsetfillcolor{currentfill}%
\pgfsetfillopacity{0.663896}%
\pgfsetlinewidth{1.003750pt}%
\definecolor{currentstroke}{rgb}{0.121569,0.466667,0.705882}%
\pgfsetstrokecolor{currentstroke}%
\pgfsetstrokeopacity{0.663896}%
\pgfsetdash{}{0pt}%
\pgfpathmoveto{\pgfqpoint{0.726903in}{1.212679in}}%
\pgfpathcurveto{\pgfqpoint{0.735140in}{1.212679in}}{\pgfqpoint{0.743040in}{1.215952in}}{\pgfqpoint{0.748864in}{1.221776in}}%
\pgfpathcurveto{\pgfqpoint{0.754688in}{1.227599in}}{\pgfqpoint{0.757960in}{1.235500in}}{\pgfqpoint{0.757960in}{1.243736in}}%
\pgfpathcurveto{\pgfqpoint{0.757960in}{1.251972in}}{\pgfqpoint{0.754688in}{1.259872in}}{\pgfqpoint{0.748864in}{1.265696in}}%
\pgfpathcurveto{\pgfqpoint{0.743040in}{1.271520in}}{\pgfqpoint{0.735140in}{1.274792in}}{\pgfqpoint{0.726903in}{1.274792in}}%
\pgfpathcurveto{\pgfqpoint{0.718667in}{1.274792in}}{\pgfqpoint{0.710767in}{1.271520in}}{\pgfqpoint{0.704943in}{1.265696in}}%
\pgfpathcurveto{\pgfqpoint{0.699119in}{1.259872in}}{\pgfqpoint{0.695847in}{1.251972in}}{\pgfqpoint{0.695847in}{1.243736in}}%
\pgfpathcurveto{\pgfqpoint{0.695847in}{1.235500in}}{\pgfqpoint{0.699119in}{1.227599in}}{\pgfqpoint{0.704943in}{1.221776in}}%
\pgfpathcurveto{\pgfqpoint{0.710767in}{1.215952in}}{\pgfqpoint{0.718667in}{1.212679in}}{\pgfqpoint{0.726903in}{1.212679in}}%
\pgfpathclose%
\pgfusepath{stroke,fill}%
\end{pgfscope}%
\begin{pgfscope}%
\pgfpathrectangle{\pgfqpoint{0.100000in}{0.212622in}}{\pgfqpoint{3.696000in}{3.696000in}}%
\pgfusepath{clip}%
\pgfsetbuttcap%
\pgfsetroundjoin%
\definecolor{currentfill}{rgb}{0.121569,0.466667,0.705882}%
\pgfsetfillcolor{currentfill}%
\pgfsetfillopacity{0.663896}%
\pgfsetlinewidth{1.003750pt}%
\definecolor{currentstroke}{rgb}{0.121569,0.466667,0.705882}%
\pgfsetstrokecolor{currentstroke}%
\pgfsetstrokeopacity{0.663896}%
\pgfsetdash{}{0pt}%
\pgfpathmoveto{\pgfqpoint{0.726903in}{1.212679in}}%
\pgfpathcurveto{\pgfqpoint{0.735140in}{1.212679in}}{\pgfqpoint{0.743040in}{1.215952in}}{\pgfqpoint{0.748864in}{1.221776in}}%
\pgfpathcurveto{\pgfqpoint{0.754688in}{1.227599in}}{\pgfqpoint{0.757960in}{1.235500in}}{\pgfqpoint{0.757960in}{1.243736in}}%
\pgfpathcurveto{\pgfqpoint{0.757960in}{1.251972in}}{\pgfqpoint{0.754688in}{1.259872in}}{\pgfqpoint{0.748864in}{1.265696in}}%
\pgfpathcurveto{\pgfqpoint{0.743040in}{1.271520in}}{\pgfqpoint{0.735140in}{1.274792in}}{\pgfqpoint{0.726903in}{1.274792in}}%
\pgfpathcurveto{\pgfqpoint{0.718667in}{1.274792in}}{\pgfqpoint{0.710767in}{1.271520in}}{\pgfqpoint{0.704943in}{1.265696in}}%
\pgfpathcurveto{\pgfqpoint{0.699119in}{1.259872in}}{\pgfqpoint{0.695847in}{1.251972in}}{\pgfqpoint{0.695847in}{1.243736in}}%
\pgfpathcurveto{\pgfqpoint{0.695847in}{1.235500in}}{\pgfqpoint{0.699119in}{1.227599in}}{\pgfqpoint{0.704943in}{1.221776in}}%
\pgfpathcurveto{\pgfqpoint{0.710767in}{1.215952in}}{\pgfqpoint{0.718667in}{1.212679in}}{\pgfqpoint{0.726903in}{1.212679in}}%
\pgfpathclose%
\pgfusepath{stroke,fill}%
\end{pgfscope}%
\begin{pgfscope}%
\pgfpathrectangle{\pgfqpoint{0.100000in}{0.212622in}}{\pgfqpoint{3.696000in}{3.696000in}}%
\pgfusepath{clip}%
\pgfsetbuttcap%
\pgfsetroundjoin%
\definecolor{currentfill}{rgb}{0.121569,0.466667,0.705882}%
\pgfsetfillcolor{currentfill}%
\pgfsetfillopacity{0.663896}%
\pgfsetlinewidth{1.003750pt}%
\definecolor{currentstroke}{rgb}{0.121569,0.466667,0.705882}%
\pgfsetstrokecolor{currentstroke}%
\pgfsetstrokeopacity{0.663896}%
\pgfsetdash{}{0pt}%
\pgfpathmoveto{\pgfqpoint{0.726903in}{1.212679in}}%
\pgfpathcurveto{\pgfqpoint{0.735140in}{1.212679in}}{\pgfqpoint{0.743040in}{1.215952in}}{\pgfqpoint{0.748864in}{1.221776in}}%
\pgfpathcurveto{\pgfqpoint{0.754688in}{1.227599in}}{\pgfqpoint{0.757960in}{1.235500in}}{\pgfqpoint{0.757960in}{1.243736in}}%
\pgfpathcurveto{\pgfqpoint{0.757960in}{1.251972in}}{\pgfqpoint{0.754688in}{1.259872in}}{\pgfqpoint{0.748864in}{1.265696in}}%
\pgfpathcurveto{\pgfqpoint{0.743040in}{1.271520in}}{\pgfqpoint{0.735140in}{1.274792in}}{\pgfqpoint{0.726903in}{1.274792in}}%
\pgfpathcurveto{\pgfqpoint{0.718667in}{1.274792in}}{\pgfqpoint{0.710767in}{1.271520in}}{\pgfqpoint{0.704943in}{1.265696in}}%
\pgfpathcurveto{\pgfqpoint{0.699119in}{1.259872in}}{\pgfqpoint{0.695847in}{1.251972in}}{\pgfqpoint{0.695847in}{1.243736in}}%
\pgfpathcurveto{\pgfqpoint{0.695847in}{1.235500in}}{\pgfqpoint{0.699119in}{1.227599in}}{\pgfqpoint{0.704943in}{1.221776in}}%
\pgfpathcurveto{\pgfqpoint{0.710767in}{1.215952in}}{\pgfqpoint{0.718667in}{1.212679in}}{\pgfqpoint{0.726903in}{1.212679in}}%
\pgfpathclose%
\pgfusepath{stroke,fill}%
\end{pgfscope}%
\begin{pgfscope}%
\pgfpathrectangle{\pgfqpoint{0.100000in}{0.212622in}}{\pgfqpoint{3.696000in}{3.696000in}}%
\pgfusepath{clip}%
\pgfsetbuttcap%
\pgfsetroundjoin%
\definecolor{currentfill}{rgb}{0.121569,0.466667,0.705882}%
\pgfsetfillcolor{currentfill}%
\pgfsetfillopacity{0.663896}%
\pgfsetlinewidth{1.003750pt}%
\definecolor{currentstroke}{rgb}{0.121569,0.466667,0.705882}%
\pgfsetstrokecolor{currentstroke}%
\pgfsetstrokeopacity{0.663896}%
\pgfsetdash{}{0pt}%
\pgfpathmoveto{\pgfqpoint{0.726903in}{1.212679in}}%
\pgfpathcurveto{\pgfqpoint{0.735140in}{1.212679in}}{\pgfqpoint{0.743040in}{1.215952in}}{\pgfqpoint{0.748864in}{1.221776in}}%
\pgfpathcurveto{\pgfqpoint{0.754688in}{1.227599in}}{\pgfqpoint{0.757960in}{1.235500in}}{\pgfqpoint{0.757960in}{1.243736in}}%
\pgfpathcurveto{\pgfqpoint{0.757960in}{1.251972in}}{\pgfqpoint{0.754688in}{1.259872in}}{\pgfqpoint{0.748864in}{1.265696in}}%
\pgfpathcurveto{\pgfqpoint{0.743040in}{1.271520in}}{\pgfqpoint{0.735140in}{1.274792in}}{\pgfqpoint{0.726903in}{1.274792in}}%
\pgfpathcurveto{\pgfqpoint{0.718667in}{1.274792in}}{\pgfqpoint{0.710767in}{1.271520in}}{\pgfqpoint{0.704943in}{1.265696in}}%
\pgfpathcurveto{\pgfqpoint{0.699119in}{1.259872in}}{\pgfqpoint{0.695847in}{1.251972in}}{\pgfqpoint{0.695847in}{1.243736in}}%
\pgfpathcurveto{\pgfqpoint{0.695847in}{1.235500in}}{\pgfqpoint{0.699119in}{1.227599in}}{\pgfqpoint{0.704943in}{1.221776in}}%
\pgfpathcurveto{\pgfqpoint{0.710767in}{1.215952in}}{\pgfqpoint{0.718667in}{1.212679in}}{\pgfqpoint{0.726903in}{1.212679in}}%
\pgfpathclose%
\pgfusepath{stroke,fill}%
\end{pgfscope}%
\begin{pgfscope}%
\pgfpathrectangle{\pgfqpoint{0.100000in}{0.212622in}}{\pgfqpoint{3.696000in}{3.696000in}}%
\pgfusepath{clip}%
\pgfsetbuttcap%
\pgfsetroundjoin%
\definecolor{currentfill}{rgb}{0.121569,0.466667,0.705882}%
\pgfsetfillcolor{currentfill}%
\pgfsetfillopacity{0.663896}%
\pgfsetlinewidth{1.003750pt}%
\definecolor{currentstroke}{rgb}{0.121569,0.466667,0.705882}%
\pgfsetstrokecolor{currentstroke}%
\pgfsetstrokeopacity{0.663896}%
\pgfsetdash{}{0pt}%
\pgfpathmoveto{\pgfqpoint{0.726903in}{1.212679in}}%
\pgfpathcurveto{\pgfqpoint{0.735140in}{1.212679in}}{\pgfqpoint{0.743040in}{1.215952in}}{\pgfqpoint{0.748864in}{1.221776in}}%
\pgfpathcurveto{\pgfqpoint{0.754688in}{1.227599in}}{\pgfqpoint{0.757960in}{1.235500in}}{\pgfqpoint{0.757960in}{1.243736in}}%
\pgfpathcurveto{\pgfqpoint{0.757960in}{1.251972in}}{\pgfqpoint{0.754688in}{1.259872in}}{\pgfqpoint{0.748864in}{1.265696in}}%
\pgfpathcurveto{\pgfqpoint{0.743040in}{1.271520in}}{\pgfqpoint{0.735140in}{1.274792in}}{\pgfqpoint{0.726903in}{1.274792in}}%
\pgfpathcurveto{\pgfqpoint{0.718667in}{1.274792in}}{\pgfqpoint{0.710767in}{1.271520in}}{\pgfqpoint{0.704943in}{1.265696in}}%
\pgfpathcurveto{\pgfqpoint{0.699119in}{1.259872in}}{\pgfqpoint{0.695847in}{1.251972in}}{\pgfqpoint{0.695847in}{1.243736in}}%
\pgfpathcurveto{\pgfqpoint{0.695847in}{1.235500in}}{\pgfqpoint{0.699119in}{1.227599in}}{\pgfqpoint{0.704943in}{1.221776in}}%
\pgfpathcurveto{\pgfqpoint{0.710767in}{1.215952in}}{\pgfqpoint{0.718667in}{1.212679in}}{\pgfqpoint{0.726903in}{1.212679in}}%
\pgfpathclose%
\pgfusepath{stroke,fill}%
\end{pgfscope}%
\begin{pgfscope}%
\pgfpathrectangle{\pgfqpoint{0.100000in}{0.212622in}}{\pgfqpoint{3.696000in}{3.696000in}}%
\pgfusepath{clip}%
\pgfsetbuttcap%
\pgfsetroundjoin%
\definecolor{currentfill}{rgb}{0.121569,0.466667,0.705882}%
\pgfsetfillcolor{currentfill}%
\pgfsetfillopacity{0.663896}%
\pgfsetlinewidth{1.003750pt}%
\definecolor{currentstroke}{rgb}{0.121569,0.466667,0.705882}%
\pgfsetstrokecolor{currentstroke}%
\pgfsetstrokeopacity{0.663896}%
\pgfsetdash{}{0pt}%
\pgfpathmoveto{\pgfqpoint{0.726903in}{1.212679in}}%
\pgfpathcurveto{\pgfqpoint{0.735140in}{1.212679in}}{\pgfqpoint{0.743040in}{1.215952in}}{\pgfqpoint{0.748864in}{1.221776in}}%
\pgfpathcurveto{\pgfqpoint{0.754688in}{1.227599in}}{\pgfqpoint{0.757960in}{1.235500in}}{\pgfqpoint{0.757960in}{1.243736in}}%
\pgfpathcurveto{\pgfqpoint{0.757960in}{1.251972in}}{\pgfqpoint{0.754688in}{1.259872in}}{\pgfqpoint{0.748864in}{1.265696in}}%
\pgfpathcurveto{\pgfqpoint{0.743040in}{1.271520in}}{\pgfqpoint{0.735140in}{1.274792in}}{\pgfqpoint{0.726903in}{1.274792in}}%
\pgfpathcurveto{\pgfqpoint{0.718667in}{1.274792in}}{\pgfqpoint{0.710767in}{1.271520in}}{\pgfqpoint{0.704943in}{1.265696in}}%
\pgfpathcurveto{\pgfqpoint{0.699119in}{1.259872in}}{\pgfqpoint{0.695847in}{1.251972in}}{\pgfqpoint{0.695847in}{1.243736in}}%
\pgfpathcurveto{\pgfqpoint{0.695847in}{1.235500in}}{\pgfqpoint{0.699119in}{1.227599in}}{\pgfqpoint{0.704943in}{1.221776in}}%
\pgfpathcurveto{\pgfqpoint{0.710767in}{1.215952in}}{\pgfqpoint{0.718667in}{1.212679in}}{\pgfqpoint{0.726903in}{1.212679in}}%
\pgfpathclose%
\pgfusepath{stroke,fill}%
\end{pgfscope}%
\begin{pgfscope}%
\pgfpathrectangle{\pgfqpoint{0.100000in}{0.212622in}}{\pgfqpoint{3.696000in}{3.696000in}}%
\pgfusepath{clip}%
\pgfsetbuttcap%
\pgfsetroundjoin%
\definecolor{currentfill}{rgb}{0.121569,0.466667,0.705882}%
\pgfsetfillcolor{currentfill}%
\pgfsetfillopacity{0.663896}%
\pgfsetlinewidth{1.003750pt}%
\definecolor{currentstroke}{rgb}{0.121569,0.466667,0.705882}%
\pgfsetstrokecolor{currentstroke}%
\pgfsetstrokeopacity{0.663896}%
\pgfsetdash{}{0pt}%
\pgfpathmoveto{\pgfqpoint{0.726903in}{1.212679in}}%
\pgfpathcurveto{\pgfqpoint{0.735140in}{1.212679in}}{\pgfqpoint{0.743040in}{1.215952in}}{\pgfqpoint{0.748864in}{1.221776in}}%
\pgfpathcurveto{\pgfqpoint{0.754688in}{1.227599in}}{\pgfqpoint{0.757960in}{1.235500in}}{\pgfqpoint{0.757960in}{1.243736in}}%
\pgfpathcurveto{\pgfqpoint{0.757960in}{1.251972in}}{\pgfqpoint{0.754688in}{1.259872in}}{\pgfqpoint{0.748864in}{1.265696in}}%
\pgfpathcurveto{\pgfqpoint{0.743040in}{1.271520in}}{\pgfqpoint{0.735140in}{1.274792in}}{\pgfqpoint{0.726903in}{1.274792in}}%
\pgfpathcurveto{\pgfqpoint{0.718667in}{1.274792in}}{\pgfqpoint{0.710767in}{1.271520in}}{\pgfqpoint{0.704943in}{1.265696in}}%
\pgfpathcurveto{\pgfqpoint{0.699119in}{1.259872in}}{\pgfqpoint{0.695847in}{1.251972in}}{\pgfqpoint{0.695847in}{1.243736in}}%
\pgfpathcurveto{\pgfqpoint{0.695847in}{1.235500in}}{\pgfqpoint{0.699119in}{1.227599in}}{\pgfqpoint{0.704943in}{1.221776in}}%
\pgfpathcurveto{\pgfqpoint{0.710767in}{1.215952in}}{\pgfqpoint{0.718667in}{1.212679in}}{\pgfqpoint{0.726903in}{1.212679in}}%
\pgfpathclose%
\pgfusepath{stroke,fill}%
\end{pgfscope}%
\begin{pgfscope}%
\pgfpathrectangle{\pgfqpoint{0.100000in}{0.212622in}}{\pgfqpoint{3.696000in}{3.696000in}}%
\pgfusepath{clip}%
\pgfsetbuttcap%
\pgfsetroundjoin%
\definecolor{currentfill}{rgb}{0.121569,0.466667,0.705882}%
\pgfsetfillcolor{currentfill}%
\pgfsetfillopacity{0.663896}%
\pgfsetlinewidth{1.003750pt}%
\definecolor{currentstroke}{rgb}{0.121569,0.466667,0.705882}%
\pgfsetstrokecolor{currentstroke}%
\pgfsetstrokeopacity{0.663896}%
\pgfsetdash{}{0pt}%
\pgfpathmoveto{\pgfqpoint{0.726903in}{1.212679in}}%
\pgfpathcurveto{\pgfqpoint{0.735140in}{1.212679in}}{\pgfqpoint{0.743040in}{1.215952in}}{\pgfqpoint{0.748864in}{1.221776in}}%
\pgfpathcurveto{\pgfqpoint{0.754688in}{1.227599in}}{\pgfqpoint{0.757960in}{1.235500in}}{\pgfqpoint{0.757960in}{1.243736in}}%
\pgfpathcurveto{\pgfqpoint{0.757960in}{1.251972in}}{\pgfqpoint{0.754688in}{1.259872in}}{\pgfqpoint{0.748864in}{1.265696in}}%
\pgfpathcurveto{\pgfqpoint{0.743040in}{1.271520in}}{\pgfqpoint{0.735140in}{1.274792in}}{\pgfqpoint{0.726903in}{1.274792in}}%
\pgfpathcurveto{\pgfqpoint{0.718667in}{1.274792in}}{\pgfqpoint{0.710767in}{1.271520in}}{\pgfqpoint{0.704943in}{1.265696in}}%
\pgfpathcurveto{\pgfqpoint{0.699119in}{1.259872in}}{\pgfqpoint{0.695847in}{1.251972in}}{\pgfqpoint{0.695847in}{1.243736in}}%
\pgfpathcurveto{\pgfqpoint{0.695847in}{1.235500in}}{\pgfqpoint{0.699119in}{1.227599in}}{\pgfqpoint{0.704943in}{1.221776in}}%
\pgfpathcurveto{\pgfqpoint{0.710767in}{1.215952in}}{\pgfqpoint{0.718667in}{1.212679in}}{\pgfqpoint{0.726903in}{1.212679in}}%
\pgfpathclose%
\pgfusepath{stroke,fill}%
\end{pgfscope}%
\begin{pgfscope}%
\pgfpathrectangle{\pgfqpoint{0.100000in}{0.212622in}}{\pgfqpoint{3.696000in}{3.696000in}}%
\pgfusepath{clip}%
\pgfsetbuttcap%
\pgfsetroundjoin%
\definecolor{currentfill}{rgb}{0.121569,0.466667,0.705882}%
\pgfsetfillcolor{currentfill}%
\pgfsetfillopacity{0.663896}%
\pgfsetlinewidth{1.003750pt}%
\definecolor{currentstroke}{rgb}{0.121569,0.466667,0.705882}%
\pgfsetstrokecolor{currentstroke}%
\pgfsetstrokeopacity{0.663896}%
\pgfsetdash{}{0pt}%
\pgfpathmoveto{\pgfqpoint{0.726903in}{1.212679in}}%
\pgfpathcurveto{\pgfqpoint{0.735140in}{1.212679in}}{\pgfqpoint{0.743040in}{1.215952in}}{\pgfqpoint{0.748864in}{1.221776in}}%
\pgfpathcurveto{\pgfqpoint{0.754688in}{1.227599in}}{\pgfqpoint{0.757960in}{1.235500in}}{\pgfqpoint{0.757960in}{1.243736in}}%
\pgfpathcurveto{\pgfqpoint{0.757960in}{1.251972in}}{\pgfqpoint{0.754688in}{1.259872in}}{\pgfqpoint{0.748864in}{1.265696in}}%
\pgfpathcurveto{\pgfqpoint{0.743040in}{1.271520in}}{\pgfqpoint{0.735140in}{1.274792in}}{\pgfqpoint{0.726903in}{1.274792in}}%
\pgfpathcurveto{\pgfqpoint{0.718667in}{1.274792in}}{\pgfqpoint{0.710767in}{1.271520in}}{\pgfqpoint{0.704943in}{1.265696in}}%
\pgfpathcurveto{\pgfqpoint{0.699119in}{1.259872in}}{\pgfqpoint{0.695847in}{1.251972in}}{\pgfqpoint{0.695847in}{1.243736in}}%
\pgfpathcurveto{\pgfqpoint{0.695847in}{1.235500in}}{\pgfqpoint{0.699119in}{1.227599in}}{\pgfqpoint{0.704943in}{1.221776in}}%
\pgfpathcurveto{\pgfqpoint{0.710767in}{1.215952in}}{\pgfqpoint{0.718667in}{1.212679in}}{\pgfqpoint{0.726903in}{1.212679in}}%
\pgfpathclose%
\pgfusepath{stroke,fill}%
\end{pgfscope}%
\begin{pgfscope}%
\pgfpathrectangle{\pgfqpoint{0.100000in}{0.212622in}}{\pgfqpoint{3.696000in}{3.696000in}}%
\pgfusepath{clip}%
\pgfsetbuttcap%
\pgfsetroundjoin%
\definecolor{currentfill}{rgb}{0.121569,0.466667,0.705882}%
\pgfsetfillcolor{currentfill}%
\pgfsetfillopacity{0.663896}%
\pgfsetlinewidth{1.003750pt}%
\definecolor{currentstroke}{rgb}{0.121569,0.466667,0.705882}%
\pgfsetstrokecolor{currentstroke}%
\pgfsetstrokeopacity{0.663896}%
\pgfsetdash{}{0pt}%
\pgfpathmoveto{\pgfqpoint{0.726903in}{1.212679in}}%
\pgfpathcurveto{\pgfqpoint{0.735140in}{1.212679in}}{\pgfqpoint{0.743040in}{1.215952in}}{\pgfqpoint{0.748864in}{1.221776in}}%
\pgfpathcurveto{\pgfqpoint{0.754688in}{1.227599in}}{\pgfqpoint{0.757960in}{1.235500in}}{\pgfqpoint{0.757960in}{1.243736in}}%
\pgfpathcurveto{\pgfqpoint{0.757960in}{1.251972in}}{\pgfqpoint{0.754688in}{1.259872in}}{\pgfqpoint{0.748864in}{1.265696in}}%
\pgfpathcurveto{\pgfqpoint{0.743040in}{1.271520in}}{\pgfqpoint{0.735140in}{1.274792in}}{\pgfqpoint{0.726903in}{1.274792in}}%
\pgfpathcurveto{\pgfqpoint{0.718667in}{1.274792in}}{\pgfqpoint{0.710767in}{1.271520in}}{\pgfqpoint{0.704943in}{1.265696in}}%
\pgfpathcurveto{\pgfqpoint{0.699119in}{1.259872in}}{\pgfqpoint{0.695847in}{1.251972in}}{\pgfqpoint{0.695847in}{1.243736in}}%
\pgfpathcurveto{\pgfqpoint{0.695847in}{1.235500in}}{\pgfqpoint{0.699119in}{1.227599in}}{\pgfqpoint{0.704943in}{1.221776in}}%
\pgfpathcurveto{\pgfqpoint{0.710767in}{1.215952in}}{\pgfqpoint{0.718667in}{1.212679in}}{\pgfqpoint{0.726903in}{1.212679in}}%
\pgfpathclose%
\pgfusepath{stroke,fill}%
\end{pgfscope}%
\begin{pgfscope}%
\pgfpathrectangle{\pgfqpoint{0.100000in}{0.212622in}}{\pgfqpoint{3.696000in}{3.696000in}}%
\pgfusepath{clip}%
\pgfsetbuttcap%
\pgfsetroundjoin%
\definecolor{currentfill}{rgb}{0.121569,0.466667,0.705882}%
\pgfsetfillcolor{currentfill}%
\pgfsetfillopacity{0.663896}%
\pgfsetlinewidth{1.003750pt}%
\definecolor{currentstroke}{rgb}{0.121569,0.466667,0.705882}%
\pgfsetstrokecolor{currentstroke}%
\pgfsetstrokeopacity{0.663896}%
\pgfsetdash{}{0pt}%
\pgfpathmoveto{\pgfqpoint{0.726903in}{1.212679in}}%
\pgfpathcurveto{\pgfqpoint{0.735140in}{1.212679in}}{\pgfqpoint{0.743040in}{1.215952in}}{\pgfqpoint{0.748864in}{1.221776in}}%
\pgfpathcurveto{\pgfqpoint{0.754688in}{1.227599in}}{\pgfqpoint{0.757960in}{1.235500in}}{\pgfqpoint{0.757960in}{1.243736in}}%
\pgfpathcurveto{\pgfqpoint{0.757960in}{1.251972in}}{\pgfqpoint{0.754688in}{1.259872in}}{\pgfqpoint{0.748864in}{1.265696in}}%
\pgfpathcurveto{\pgfqpoint{0.743040in}{1.271520in}}{\pgfqpoint{0.735140in}{1.274792in}}{\pgfqpoint{0.726903in}{1.274792in}}%
\pgfpathcurveto{\pgfqpoint{0.718667in}{1.274792in}}{\pgfqpoint{0.710767in}{1.271520in}}{\pgfqpoint{0.704943in}{1.265696in}}%
\pgfpathcurveto{\pgfqpoint{0.699119in}{1.259872in}}{\pgfqpoint{0.695847in}{1.251972in}}{\pgfqpoint{0.695847in}{1.243736in}}%
\pgfpathcurveto{\pgfqpoint{0.695847in}{1.235500in}}{\pgfqpoint{0.699119in}{1.227599in}}{\pgfqpoint{0.704943in}{1.221776in}}%
\pgfpathcurveto{\pgfqpoint{0.710767in}{1.215952in}}{\pgfqpoint{0.718667in}{1.212679in}}{\pgfqpoint{0.726903in}{1.212679in}}%
\pgfpathclose%
\pgfusepath{stroke,fill}%
\end{pgfscope}%
\begin{pgfscope}%
\pgfpathrectangle{\pgfqpoint{0.100000in}{0.212622in}}{\pgfqpoint{3.696000in}{3.696000in}}%
\pgfusepath{clip}%
\pgfsetbuttcap%
\pgfsetroundjoin%
\definecolor{currentfill}{rgb}{0.121569,0.466667,0.705882}%
\pgfsetfillcolor{currentfill}%
\pgfsetfillopacity{0.663896}%
\pgfsetlinewidth{1.003750pt}%
\definecolor{currentstroke}{rgb}{0.121569,0.466667,0.705882}%
\pgfsetstrokecolor{currentstroke}%
\pgfsetstrokeopacity{0.663896}%
\pgfsetdash{}{0pt}%
\pgfpathmoveto{\pgfqpoint{0.726903in}{1.212679in}}%
\pgfpathcurveto{\pgfqpoint{0.735140in}{1.212679in}}{\pgfqpoint{0.743040in}{1.215952in}}{\pgfqpoint{0.748864in}{1.221776in}}%
\pgfpathcurveto{\pgfqpoint{0.754688in}{1.227599in}}{\pgfqpoint{0.757960in}{1.235500in}}{\pgfqpoint{0.757960in}{1.243736in}}%
\pgfpathcurveto{\pgfqpoint{0.757960in}{1.251972in}}{\pgfqpoint{0.754688in}{1.259872in}}{\pgfqpoint{0.748864in}{1.265696in}}%
\pgfpathcurveto{\pgfqpoint{0.743040in}{1.271520in}}{\pgfqpoint{0.735140in}{1.274792in}}{\pgfqpoint{0.726903in}{1.274792in}}%
\pgfpathcurveto{\pgfqpoint{0.718667in}{1.274792in}}{\pgfqpoint{0.710767in}{1.271520in}}{\pgfqpoint{0.704943in}{1.265696in}}%
\pgfpathcurveto{\pgfqpoint{0.699119in}{1.259872in}}{\pgfqpoint{0.695847in}{1.251972in}}{\pgfqpoint{0.695847in}{1.243736in}}%
\pgfpathcurveto{\pgfqpoint{0.695847in}{1.235500in}}{\pgfqpoint{0.699119in}{1.227599in}}{\pgfqpoint{0.704943in}{1.221776in}}%
\pgfpathcurveto{\pgfqpoint{0.710767in}{1.215952in}}{\pgfqpoint{0.718667in}{1.212679in}}{\pgfqpoint{0.726903in}{1.212679in}}%
\pgfpathclose%
\pgfusepath{stroke,fill}%
\end{pgfscope}%
\begin{pgfscope}%
\pgfpathrectangle{\pgfqpoint{0.100000in}{0.212622in}}{\pgfqpoint{3.696000in}{3.696000in}}%
\pgfusepath{clip}%
\pgfsetbuttcap%
\pgfsetroundjoin%
\definecolor{currentfill}{rgb}{0.121569,0.466667,0.705882}%
\pgfsetfillcolor{currentfill}%
\pgfsetfillopacity{0.663896}%
\pgfsetlinewidth{1.003750pt}%
\definecolor{currentstroke}{rgb}{0.121569,0.466667,0.705882}%
\pgfsetstrokecolor{currentstroke}%
\pgfsetstrokeopacity{0.663896}%
\pgfsetdash{}{0pt}%
\pgfpathmoveto{\pgfqpoint{0.726903in}{1.212679in}}%
\pgfpathcurveto{\pgfqpoint{0.735140in}{1.212679in}}{\pgfqpoint{0.743040in}{1.215952in}}{\pgfqpoint{0.748864in}{1.221776in}}%
\pgfpathcurveto{\pgfqpoint{0.754688in}{1.227599in}}{\pgfqpoint{0.757960in}{1.235500in}}{\pgfqpoint{0.757960in}{1.243736in}}%
\pgfpathcurveto{\pgfqpoint{0.757960in}{1.251972in}}{\pgfqpoint{0.754688in}{1.259872in}}{\pgfqpoint{0.748864in}{1.265696in}}%
\pgfpathcurveto{\pgfqpoint{0.743040in}{1.271520in}}{\pgfqpoint{0.735140in}{1.274792in}}{\pgfqpoint{0.726903in}{1.274792in}}%
\pgfpathcurveto{\pgfqpoint{0.718667in}{1.274792in}}{\pgfqpoint{0.710767in}{1.271520in}}{\pgfqpoint{0.704943in}{1.265696in}}%
\pgfpathcurveto{\pgfqpoint{0.699119in}{1.259872in}}{\pgfqpoint{0.695847in}{1.251972in}}{\pgfqpoint{0.695847in}{1.243736in}}%
\pgfpathcurveto{\pgfqpoint{0.695847in}{1.235500in}}{\pgfqpoint{0.699119in}{1.227599in}}{\pgfqpoint{0.704943in}{1.221776in}}%
\pgfpathcurveto{\pgfqpoint{0.710767in}{1.215952in}}{\pgfqpoint{0.718667in}{1.212679in}}{\pgfqpoint{0.726903in}{1.212679in}}%
\pgfpathclose%
\pgfusepath{stroke,fill}%
\end{pgfscope}%
\begin{pgfscope}%
\pgfpathrectangle{\pgfqpoint{0.100000in}{0.212622in}}{\pgfqpoint{3.696000in}{3.696000in}}%
\pgfusepath{clip}%
\pgfsetbuttcap%
\pgfsetroundjoin%
\definecolor{currentfill}{rgb}{0.121569,0.466667,0.705882}%
\pgfsetfillcolor{currentfill}%
\pgfsetfillopacity{0.663896}%
\pgfsetlinewidth{1.003750pt}%
\definecolor{currentstroke}{rgb}{0.121569,0.466667,0.705882}%
\pgfsetstrokecolor{currentstroke}%
\pgfsetstrokeopacity{0.663896}%
\pgfsetdash{}{0pt}%
\pgfpathmoveto{\pgfqpoint{0.726903in}{1.212679in}}%
\pgfpathcurveto{\pgfqpoint{0.735140in}{1.212679in}}{\pgfqpoint{0.743040in}{1.215952in}}{\pgfqpoint{0.748864in}{1.221776in}}%
\pgfpathcurveto{\pgfqpoint{0.754688in}{1.227599in}}{\pgfqpoint{0.757960in}{1.235500in}}{\pgfqpoint{0.757960in}{1.243736in}}%
\pgfpathcurveto{\pgfqpoint{0.757960in}{1.251972in}}{\pgfqpoint{0.754688in}{1.259872in}}{\pgfqpoint{0.748864in}{1.265696in}}%
\pgfpathcurveto{\pgfqpoint{0.743040in}{1.271520in}}{\pgfqpoint{0.735140in}{1.274792in}}{\pgfqpoint{0.726903in}{1.274792in}}%
\pgfpathcurveto{\pgfqpoint{0.718667in}{1.274792in}}{\pgfqpoint{0.710767in}{1.271520in}}{\pgfqpoint{0.704943in}{1.265696in}}%
\pgfpathcurveto{\pgfqpoint{0.699119in}{1.259872in}}{\pgfqpoint{0.695847in}{1.251972in}}{\pgfqpoint{0.695847in}{1.243736in}}%
\pgfpathcurveto{\pgfqpoint{0.695847in}{1.235500in}}{\pgfqpoint{0.699119in}{1.227599in}}{\pgfqpoint{0.704943in}{1.221776in}}%
\pgfpathcurveto{\pgfqpoint{0.710767in}{1.215952in}}{\pgfqpoint{0.718667in}{1.212679in}}{\pgfqpoint{0.726903in}{1.212679in}}%
\pgfpathclose%
\pgfusepath{stroke,fill}%
\end{pgfscope}%
\begin{pgfscope}%
\pgfpathrectangle{\pgfqpoint{0.100000in}{0.212622in}}{\pgfqpoint{3.696000in}{3.696000in}}%
\pgfusepath{clip}%
\pgfsetbuttcap%
\pgfsetroundjoin%
\definecolor{currentfill}{rgb}{0.121569,0.466667,0.705882}%
\pgfsetfillcolor{currentfill}%
\pgfsetfillopacity{0.663896}%
\pgfsetlinewidth{1.003750pt}%
\definecolor{currentstroke}{rgb}{0.121569,0.466667,0.705882}%
\pgfsetstrokecolor{currentstroke}%
\pgfsetstrokeopacity{0.663896}%
\pgfsetdash{}{0pt}%
\pgfpathmoveto{\pgfqpoint{0.726903in}{1.212679in}}%
\pgfpathcurveto{\pgfqpoint{0.735140in}{1.212679in}}{\pgfqpoint{0.743040in}{1.215952in}}{\pgfqpoint{0.748864in}{1.221776in}}%
\pgfpathcurveto{\pgfqpoint{0.754688in}{1.227599in}}{\pgfqpoint{0.757960in}{1.235500in}}{\pgfqpoint{0.757960in}{1.243736in}}%
\pgfpathcurveto{\pgfqpoint{0.757960in}{1.251972in}}{\pgfqpoint{0.754688in}{1.259872in}}{\pgfqpoint{0.748864in}{1.265696in}}%
\pgfpathcurveto{\pgfqpoint{0.743040in}{1.271520in}}{\pgfqpoint{0.735140in}{1.274792in}}{\pgfqpoint{0.726903in}{1.274792in}}%
\pgfpathcurveto{\pgfqpoint{0.718667in}{1.274792in}}{\pgfqpoint{0.710767in}{1.271520in}}{\pgfqpoint{0.704943in}{1.265696in}}%
\pgfpathcurveto{\pgfqpoint{0.699119in}{1.259872in}}{\pgfqpoint{0.695847in}{1.251972in}}{\pgfqpoint{0.695847in}{1.243736in}}%
\pgfpathcurveto{\pgfqpoint{0.695847in}{1.235500in}}{\pgfqpoint{0.699119in}{1.227599in}}{\pgfqpoint{0.704943in}{1.221776in}}%
\pgfpathcurveto{\pgfqpoint{0.710767in}{1.215952in}}{\pgfqpoint{0.718667in}{1.212679in}}{\pgfqpoint{0.726903in}{1.212679in}}%
\pgfpathclose%
\pgfusepath{stroke,fill}%
\end{pgfscope}%
\begin{pgfscope}%
\pgfpathrectangle{\pgfqpoint{0.100000in}{0.212622in}}{\pgfqpoint{3.696000in}{3.696000in}}%
\pgfusepath{clip}%
\pgfsetbuttcap%
\pgfsetroundjoin%
\definecolor{currentfill}{rgb}{0.121569,0.466667,0.705882}%
\pgfsetfillcolor{currentfill}%
\pgfsetfillopacity{0.663896}%
\pgfsetlinewidth{1.003750pt}%
\definecolor{currentstroke}{rgb}{0.121569,0.466667,0.705882}%
\pgfsetstrokecolor{currentstroke}%
\pgfsetstrokeopacity{0.663896}%
\pgfsetdash{}{0pt}%
\pgfpathmoveto{\pgfqpoint{0.726903in}{1.212679in}}%
\pgfpathcurveto{\pgfqpoint{0.735140in}{1.212679in}}{\pgfqpoint{0.743040in}{1.215952in}}{\pgfqpoint{0.748864in}{1.221776in}}%
\pgfpathcurveto{\pgfqpoint{0.754688in}{1.227599in}}{\pgfqpoint{0.757960in}{1.235500in}}{\pgfqpoint{0.757960in}{1.243736in}}%
\pgfpathcurveto{\pgfqpoint{0.757960in}{1.251972in}}{\pgfqpoint{0.754688in}{1.259872in}}{\pgfqpoint{0.748864in}{1.265696in}}%
\pgfpathcurveto{\pgfqpoint{0.743040in}{1.271520in}}{\pgfqpoint{0.735140in}{1.274792in}}{\pgfqpoint{0.726903in}{1.274792in}}%
\pgfpathcurveto{\pgfqpoint{0.718667in}{1.274792in}}{\pgfqpoint{0.710767in}{1.271520in}}{\pgfqpoint{0.704943in}{1.265696in}}%
\pgfpathcurveto{\pgfqpoint{0.699119in}{1.259872in}}{\pgfqpoint{0.695847in}{1.251972in}}{\pgfqpoint{0.695847in}{1.243736in}}%
\pgfpathcurveto{\pgfqpoint{0.695847in}{1.235500in}}{\pgfqpoint{0.699119in}{1.227599in}}{\pgfqpoint{0.704943in}{1.221776in}}%
\pgfpathcurveto{\pgfqpoint{0.710767in}{1.215952in}}{\pgfqpoint{0.718667in}{1.212679in}}{\pgfqpoint{0.726903in}{1.212679in}}%
\pgfpathclose%
\pgfusepath{stroke,fill}%
\end{pgfscope}%
\begin{pgfscope}%
\pgfpathrectangle{\pgfqpoint{0.100000in}{0.212622in}}{\pgfqpoint{3.696000in}{3.696000in}}%
\pgfusepath{clip}%
\pgfsetbuttcap%
\pgfsetroundjoin%
\definecolor{currentfill}{rgb}{0.121569,0.466667,0.705882}%
\pgfsetfillcolor{currentfill}%
\pgfsetfillopacity{0.663896}%
\pgfsetlinewidth{1.003750pt}%
\definecolor{currentstroke}{rgb}{0.121569,0.466667,0.705882}%
\pgfsetstrokecolor{currentstroke}%
\pgfsetstrokeopacity{0.663896}%
\pgfsetdash{}{0pt}%
\pgfpathmoveto{\pgfqpoint{0.726903in}{1.212679in}}%
\pgfpathcurveto{\pgfqpoint{0.735140in}{1.212679in}}{\pgfqpoint{0.743040in}{1.215952in}}{\pgfqpoint{0.748864in}{1.221776in}}%
\pgfpathcurveto{\pgfqpoint{0.754688in}{1.227599in}}{\pgfqpoint{0.757960in}{1.235500in}}{\pgfqpoint{0.757960in}{1.243736in}}%
\pgfpathcurveto{\pgfqpoint{0.757960in}{1.251972in}}{\pgfqpoint{0.754688in}{1.259872in}}{\pgfqpoint{0.748864in}{1.265696in}}%
\pgfpathcurveto{\pgfqpoint{0.743040in}{1.271520in}}{\pgfqpoint{0.735140in}{1.274792in}}{\pgfqpoint{0.726903in}{1.274792in}}%
\pgfpathcurveto{\pgfqpoint{0.718667in}{1.274792in}}{\pgfqpoint{0.710767in}{1.271520in}}{\pgfqpoint{0.704943in}{1.265696in}}%
\pgfpathcurveto{\pgfqpoint{0.699119in}{1.259872in}}{\pgfqpoint{0.695847in}{1.251972in}}{\pgfqpoint{0.695847in}{1.243736in}}%
\pgfpathcurveto{\pgfqpoint{0.695847in}{1.235500in}}{\pgfqpoint{0.699119in}{1.227599in}}{\pgfqpoint{0.704943in}{1.221776in}}%
\pgfpathcurveto{\pgfqpoint{0.710767in}{1.215952in}}{\pgfqpoint{0.718667in}{1.212679in}}{\pgfqpoint{0.726903in}{1.212679in}}%
\pgfpathclose%
\pgfusepath{stroke,fill}%
\end{pgfscope}%
\begin{pgfscope}%
\pgfpathrectangle{\pgfqpoint{0.100000in}{0.212622in}}{\pgfqpoint{3.696000in}{3.696000in}}%
\pgfusepath{clip}%
\pgfsetbuttcap%
\pgfsetroundjoin%
\definecolor{currentfill}{rgb}{0.121569,0.466667,0.705882}%
\pgfsetfillcolor{currentfill}%
\pgfsetfillopacity{0.663896}%
\pgfsetlinewidth{1.003750pt}%
\definecolor{currentstroke}{rgb}{0.121569,0.466667,0.705882}%
\pgfsetstrokecolor{currentstroke}%
\pgfsetstrokeopacity{0.663896}%
\pgfsetdash{}{0pt}%
\pgfpathmoveto{\pgfqpoint{0.726903in}{1.212679in}}%
\pgfpathcurveto{\pgfqpoint{0.735140in}{1.212679in}}{\pgfqpoint{0.743040in}{1.215952in}}{\pgfqpoint{0.748864in}{1.221776in}}%
\pgfpathcurveto{\pgfqpoint{0.754688in}{1.227599in}}{\pgfqpoint{0.757960in}{1.235500in}}{\pgfqpoint{0.757960in}{1.243736in}}%
\pgfpathcurveto{\pgfqpoint{0.757960in}{1.251972in}}{\pgfqpoint{0.754688in}{1.259872in}}{\pgfqpoint{0.748864in}{1.265696in}}%
\pgfpathcurveto{\pgfqpoint{0.743040in}{1.271520in}}{\pgfqpoint{0.735140in}{1.274792in}}{\pgfqpoint{0.726903in}{1.274792in}}%
\pgfpathcurveto{\pgfqpoint{0.718667in}{1.274792in}}{\pgfqpoint{0.710767in}{1.271520in}}{\pgfqpoint{0.704943in}{1.265696in}}%
\pgfpathcurveto{\pgfqpoint{0.699119in}{1.259872in}}{\pgfqpoint{0.695847in}{1.251972in}}{\pgfqpoint{0.695847in}{1.243736in}}%
\pgfpathcurveto{\pgfqpoint{0.695847in}{1.235500in}}{\pgfqpoint{0.699119in}{1.227599in}}{\pgfqpoint{0.704943in}{1.221776in}}%
\pgfpathcurveto{\pgfqpoint{0.710767in}{1.215952in}}{\pgfqpoint{0.718667in}{1.212679in}}{\pgfqpoint{0.726903in}{1.212679in}}%
\pgfpathclose%
\pgfusepath{stroke,fill}%
\end{pgfscope}%
\begin{pgfscope}%
\pgfpathrectangle{\pgfqpoint{0.100000in}{0.212622in}}{\pgfqpoint{3.696000in}{3.696000in}}%
\pgfusepath{clip}%
\pgfsetbuttcap%
\pgfsetroundjoin%
\definecolor{currentfill}{rgb}{0.121569,0.466667,0.705882}%
\pgfsetfillcolor{currentfill}%
\pgfsetfillopacity{0.663896}%
\pgfsetlinewidth{1.003750pt}%
\definecolor{currentstroke}{rgb}{0.121569,0.466667,0.705882}%
\pgfsetstrokecolor{currentstroke}%
\pgfsetstrokeopacity{0.663896}%
\pgfsetdash{}{0pt}%
\pgfpathmoveto{\pgfqpoint{0.726903in}{1.212679in}}%
\pgfpathcurveto{\pgfqpoint{0.735140in}{1.212679in}}{\pgfqpoint{0.743040in}{1.215952in}}{\pgfqpoint{0.748864in}{1.221776in}}%
\pgfpathcurveto{\pgfqpoint{0.754688in}{1.227599in}}{\pgfqpoint{0.757960in}{1.235500in}}{\pgfqpoint{0.757960in}{1.243736in}}%
\pgfpathcurveto{\pgfqpoint{0.757960in}{1.251972in}}{\pgfqpoint{0.754688in}{1.259872in}}{\pgfqpoint{0.748864in}{1.265696in}}%
\pgfpathcurveto{\pgfqpoint{0.743040in}{1.271520in}}{\pgfqpoint{0.735140in}{1.274792in}}{\pgfqpoint{0.726903in}{1.274792in}}%
\pgfpathcurveto{\pgfqpoint{0.718667in}{1.274792in}}{\pgfqpoint{0.710767in}{1.271520in}}{\pgfqpoint{0.704943in}{1.265696in}}%
\pgfpathcurveto{\pgfqpoint{0.699119in}{1.259872in}}{\pgfqpoint{0.695847in}{1.251972in}}{\pgfqpoint{0.695847in}{1.243736in}}%
\pgfpathcurveto{\pgfqpoint{0.695847in}{1.235500in}}{\pgfqpoint{0.699119in}{1.227599in}}{\pgfqpoint{0.704943in}{1.221776in}}%
\pgfpathcurveto{\pgfqpoint{0.710767in}{1.215952in}}{\pgfqpoint{0.718667in}{1.212679in}}{\pgfqpoint{0.726903in}{1.212679in}}%
\pgfpathclose%
\pgfusepath{stroke,fill}%
\end{pgfscope}%
\begin{pgfscope}%
\pgfpathrectangle{\pgfqpoint{0.100000in}{0.212622in}}{\pgfqpoint{3.696000in}{3.696000in}}%
\pgfusepath{clip}%
\pgfsetbuttcap%
\pgfsetroundjoin%
\definecolor{currentfill}{rgb}{0.121569,0.466667,0.705882}%
\pgfsetfillcolor{currentfill}%
\pgfsetfillopacity{0.663896}%
\pgfsetlinewidth{1.003750pt}%
\definecolor{currentstroke}{rgb}{0.121569,0.466667,0.705882}%
\pgfsetstrokecolor{currentstroke}%
\pgfsetstrokeopacity{0.663896}%
\pgfsetdash{}{0pt}%
\pgfpathmoveto{\pgfqpoint{0.726903in}{1.212679in}}%
\pgfpathcurveto{\pgfqpoint{0.735140in}{1.212679in}}{\pgfqpoint{0.743040in}{1.215952in}}{\pgfqpoint{0.748864in}{1.221776in}}%
\pgfpathcurveto{\pgfqpoint{0.754688in}{1.227599in}}{\pgfqpoint{0.757960in}{1.235500in}}{\pgfqpoint{0.757960in}{1.243736in}}%
\pgfpathcurveto{\pgfqpoint{0.757960in}{1.251972in}}{\pgfqpoint{0.754688in}{1.259872in}}{\pgfqpoint{0.748864in}{1.265696in}}%
\pgfpathcurveto{\pgfqpoint{0.743040in}{1.271520in}}{\pgfqpoint{0.735140in}{1.274792in}}{\pgfqpoint{0.726903in}{1.274792in}}%
\pgfpathcurveto{\pgfqpoint{0.718667in}{1.274792in}}{\pgfqpoint{0.710767in}{1.271520in}}{\pgfqpoint{0.704943in}{1.265696in}}%
\pgfpathcurveto{\pgfqpoint{0.699119in}{1.259872in}}{\pgfqpoint{0.695847in}{1.251972in}}{\pgfqpoint{0.695847in}{1.243736in}}%
\pgfpathcurveto{\pgfqpoint{0.695847in}{1.235500in}}{\pgfqpoint{0.699119in}{1.227599in}}{\pgfqpoint{0.704943in}{1.221776in}}%
\pgfpathcurveto{\pgfqpoint{0.710767in}{1.215952in}}{\pgfqpoint{0.718667in}{1.212679in}}{\pgfqpoint{0.726903in}{1.212679in}}%
\pgfpathclose%
\pgfusepath{stroke,fill}%
\end{pgfscope}%
\begin{pgfscope}%
\pgfpathrectangle{\pgfqpoint{0.100000in}{0.212622in}}{\pgfqpoint{3.696000in}{3.696000in}}%
\pgfusepath{clip}%
\pgfsetbuttcap%
\pgfsetroundjoin%
\definecolor{currentfill}{rgb}{0.121569,0.466667,0.705882}%
\pgfsetfillcolor{currentfill}%
\pgfsetfillopacity{0.663896}%
\pgfsetlinewidth{1.003750pt}%
\definecolor{currentstroke}{rgb}{0.121569,0.466667,0.705882}%
\pgfsetstrokecolor{currentstroke}%
\pgfsetstrokeopacity{0.663896}%
\pgfsetdash{}{0pt}%
\pgfpathmoveto{\pgfqpoint{0.726903in}{1.212679in}}%
\pgfpathcurveto{\pgfqpoint{0.735140in}{1.212679in}}{\pgfqpoint{0.743040in}{1.215952in}}{\pgfqpoint{0.748864in}{1.221776in}}%
\pgfpathcurveto{\pgfqpoint{0.754688in}{1.227599in}}{\pgfqpoint{0.757960in}{1.235500in}}{\pgfqpoint{0.757960in}{1.243736in}}%
\pgfpathcurveto{\pgfqpoint{0.757960in}{1.251972in}}{\pgfqpoint{0.754688in}{1.259872in}}{\pgfqpoint{0.748864in}{1.265696in}}%
\pgfpathcurveto{\pgfqpoint{0.743040in}{1.271520in}}{\pgfqpoint{0.735140in}{1.274792in}}{\pgfqpoint{0.726903in}{1.274792in}}%
\pgfpathcurveto{\pgfqpoint{0.718667in}{1.274792in}}{\pgfqpoint{0.710767in}{1.271520in}}{\pgfqpoint{0.704943in}{1.265696in}}%
\pgfpathcurveto{\pgfqpoint{0.699119in}{1.259872in}}{\pgfqpoint{0.695847in}{1.251972in}}{\pgfqpoint{0.695847in}{1.243736in}}%
\pgfpathcurveto{\pgfqpoint{0.695847in}{1.235500in}}{\pgfqpoint{0.699119in}{1.227599in}}{\pgfqpoint{0.704943in}{1.221776in}}%
\pgfpathcurveto{\pgfqpoint{0.710767in}{1.215952in}}{\pgfqpoint{0.718667in}{1.212679in}}{\pgfqpoint{0.726903in}{1.212679in}}%
\pgfpathclose%
\pgfusepath{stroke,fill}%
\end{pgfscope}%
\begin{pgfscope}%
\pgfpathrectangle{\pgfqpoint{0.100000in}{0.212622in}}{\pgfqpoint{3.696000in}{3.696000in}}%
\pgfusepath{clip}%
\pgfsetbuttcap%
\pgfsetroundjoin%
\definecolor{currentfill}{rgb}{0.121569,0.466667,0.705882}%
\pgfsetfillcolor{currentfill}%
\pgfsetfillopacity{0.663896}%
\pgfsetlinewidth{1.003750pt}%
\definecolor{currentstroke}{rgb}{0.121569,0.466667,0.705882}%
\pgfsetstrokecolor{currentstroke}%
\pgfsetstrokeopacity{0.663896}%
\pgfsetdash{}{0pt}%
\pgfpathmoveto{\pgfqpoint{0.726903in}{1.212679in}}%
\pgfpathcurveto{\pgfqpoint{0.735140in}{1.212679in}}{\pgfqpoint{0.743040in}{1.215952in}}{\pgfqpoint{0.748864in}{1.221776in}}%
\pgfpathcurveto{\pgfqpoint{0.754688in}{1.227599in}}{\pgfqpoint{0.757960in}{1.235500in}}{\pgfqpoint{0.757960in}{1.243736in}}%
\pgfpathcurveto{\pgfqpoint{0.757960in}{1.251972in}}{\pgfqpoint{0.754688in}{1.259872in}}{\pgfqpoint{0.748864in}{1.265696in}}%
\pgfpathcurveto{\pgfqpoint{0.743040in}{1.271520in}}{\pgfqpoint{0.735140in}{1.274792in}}{\pgfqpoint{0.726903in}{1.274792in}}%
\pgfpathcurveto{\pgfqpoint{0.718667in}{1.274792in}}{\pgfqpoint{0.710767in}{1.271520in}}{\pgfqpoint{0.704943in}{1.265696in}}%
\pgfpathcurveto{\pgfqpoint{0.699119in}{1.259872in}}{\pgfqpoint{0.695847in}{1.251972in}}{\pgfqpoint{0.695847in}{1.243736in}}%
\pgfpathcurveto{\pgfqpoint{0.695847in}{1.235500in}}{\pgfqpoint{0.699119in}{1.227599in}}{\pgfqpoint{0.704943in}{1.221776in}}%
\pgfpathcurveto{\pgfqpoint{0.710767in}{1.215952in}}{\pgfqpoint{0.718667in}{1.212679in}}{\pgfqpoint{0.726903in}{1.212679in}}%
\pgfpathclose%
\pgfusepath{stroke,fill}%
\end{pgfscope}%
\begin{pgfscope}%
\pgfpathrectangle{\pgfqpoint{0.100000in}{0.212622in}}{\pgfqpoint{3.696000in}{3.696000in}}%
\pgfusepath{clip}%
\pgfsetbuttcap%
\pgfsetroundjoin%
\definecolor{currentfill}{rgb}{0.121569,0.466667,0.705882}%
\pgfsetfillcolor{currentfill}%
\pgfsetfillopacity{0.663896}%
\pgfsetlinewidth{1.003750pt}%
\definecolor{currentstroke}{rgb}{0.121569,0.466667,0.705882}%
\pgfsetstrokecolor{currentstroke}%
\pgfsetstrokeopacity{0.663896}%
\pgfsetdash{}{0pt}%
\pgfpathmoveto{\pgfqpoint{0.726903in}{1.212679in}}%
\pgfpathcurveto{\pgfqpoint{0.735140in}{1.212679in}}{\pgfqpoint{0.743040in}{1.215952in}}{\pgfqpoint{0.748864in}{1.221776in}}%
\pgfpathcurveto{\pgfqpoint{0.754688in}{1.227599in}}{\pgfqpoint{0.757960in}{1.235500in}}{\pgfqpoint{0.757960in}{1.243736in}}%
\pgfpathcurveto{\pgfqpoint{0.757960in}{1.251972in}}{\pgfqpoint{0.754688in}{1.259872in}}{\pgfqpoint{0.748864in}{1.265696in}}%
\pgfpathcurveto{\pgfqpoint{0.743040in}{1.271520in}}{\pgfqpoint{0.735140in}{1.274792in}}{\pgfqpoint{0.726903in}{1.274792in}}%
\pgfpathcurveto{\pgfqpoint{0.718667in}{1.274792in}}{\pgfqpoint{0.710767in}{1.271520in}}{\pgfqpoint{0.704943in}{1.265696in}}%
\pgfpathcurveto{\pgfqpoint{0.699119in}{1.259872in}}{\pgfqpoint{0.695847in}{1.251972in}}{\pgfqpoint{0.695847in}{1.243736in}}%
\pgfpathcurveto{\pgfqpoint{0.695847in}{1.235500in}}{\pgfqpoint{0.699119in}{1.227599in}}{\pgfqpoint{0.704943in}{1.221776in}}%
\pgfpathcurveto{\pgfqpoint{0.710767in}{1.215952in}}{\pgfqpoint{0.718667in}{1.212679in}}{\pgfqpoint{0.726903in}{1.212679in}}%
\pgfpathclose%
\pgfusepath{stroke,fill}%
\end{pgfscope}%
\begin{pgfscope}%
\pgfpathrectangle{\pgfqpoint{0.100000in}{0.212622in}}{\pgfqpoint{3.696000in}{3.696000in}}%
\pgfusepath{clip}%
\pgfsetbuttcap%
\pgfsetroundjoin%
\definecolor{currentfill}{rgb}{0.121569,0.466667,0.705882}%
\pgfsetfillcolor{currentfill}%
\pgfsetfillopacity{0.663896}%
\pgfsetlinewidth{1.003750pt}%
\definecolor{currentstroke}{rgb}{0.121569,0.466667,0.705882}%
\pgfsetstrokecolor{currentstroke}%
\pgfsetstrokeopacity{0.663896}%
\pgfsetdash{}{0pt}%
\pgfpathmoveto{\pgfqpoint{0.726903in}{1.212679in}}%
\pgfpathcurveto{\pgfqpoint{0.735140in}{1.212679in}}{\pgfqpoint{0.743040in}{1.215952in}}{\pgfqpoint{0.748864in}{1.221776in}}%
\pgfpathcurveto{\pgfqpoint{0.754688in}{1.227599in}}{\pgfqpoint{0.757960in}{1.235500in}}{\pgfqpoint{0.757960in}{1.243736in}}%
\pgfpathcurveto{\pgfqpoint{0.757960in}{1.251972in}}{\pgfqpoint{0.754688in}{1.259872in}}{\pgfqpoint{0.748864in}{1.265696in}}%
\pgfpathcurveto{\pgfqpoint{0.743040in}{1.271520in}}{\pgfqpoint{0.735140in}{1.274792in}}{\pgfqpoint{0.726903in}{1.274792in}}%
\pgfpathcurveto{\pgfqpoint{0.718667in}{1.274792in}}{\pgfqpoint{0.710767in}{1.271520in}}{\pgfqpoint{0.704943in}{1.265696in}}%
\pgfpathcurveto{\pgfqpoint{0.699119in}{1.259872in}}{\pgfqpoint{0.695847in}{1.251972in}}{\pgfqpoint{0.695847in}{1.243736in}}%
\pgfpathcurveto{\pgfqpoint{0.695847in}{1.235500in}}{\pgfqpoint{0.699119in}{1.227599in}}{\pgfqpoint{0.704943in}{1.221776in}}%
\pgfpathcurveto{\pgfqpoint{0.710767in}{1.215952in}}{\pgfqpoint{0.718667in}{1.212679in}}{\pgfqpoint{0.726903in}{1.212679in}}%
\pgfpathclose%
\pgfusepath{stroke,fill}%
\end{pgfscope}%
\begin{pgfscope}%
\pgfpathrectangle{\pgfqpoint{0.100000in}{0.212622in}}{\pgfqpoint{3.696000in}{3.696000in}}%
\pgfusepath{clip}%
\pgfsetbuttcap%
\pgfsetroundjoin%
\definecolor{currentfill}{rgb}{0.121569,0.466667,0.705882}%
\pgfsetfillcolor{currentfill}%
\pgfsetfillopacity{0.663896}%
\pgfsetlinewidth{1.003750pt}%
\definecolor{currentstroke}{rgb}{0.121569,0.466667,0.705882}%
\pgfsetstrokecolor{currentstroke}%
\pgfsetstrokeopacity{0.663896}%
\pgfsetdash{}{0pt}%
\pgfpathmoveto{\pgfqpoint{0.726903in}{1.212679in}}%
\pgfpathcurveto{\pgfqpoint{0.735140in}{1.212679in}}{\pgfqpoint{0.743040in}{1.215952in}}{\pgfqpoint{0.748864in}{1.221776in}}%
\pgfpathcurveto{\pgfqpoint{0.754688in}{1.227599in}}{\pgfqpoint{0.757960in}{1.235500in}}{\pgfqpoint{0.757960in}{1.243736in}}%
\pgfpathcurveto{\pgfqpoint{0.757960in}{1.251972in}}{\pgfqpoint{0.754688in}{1.259872in}}{\pgfqpoint{0.748864in}{1.265696in}}%
\pgfpathcurveto{\pgfqpoint{0.743040in}{1.271520in}}{\pgfqpoint{0.735140in}{1.274792in}}{\pgfqpoint{0.726903in}{1.274792in}}%
\pgfpathcurveto{\pgfqpoint{0.718667in}{1.274792in}}{\pgfqpoint{0.710767in}{1.271520in}}{\pgfqpoint{0.704943in}{1.265696in}}%
\pgfpathcurveto{\pgfqpoint{0.699119in}{1.259872in}}{\pgfqpoint{0.695847in}{1.251972in}}{\pgfqpoint{0.695847in}{1.243736in}}%
\pgfpathcurveto{\pgfqpoint{0.695847in}{1.235500in}}{\pgfqpoint{0.699119in}{1.227599in}}{\pgfqpoint{0.704943in}{1.221776in}}%
\pgfpathcurveto{\pgfqpoint{0.710767in}{1.215952in}}{\pgfqpoint{0.718667in}{1.212679in}}{\pgfqpoint{0.726903in}{1.212679in}}%
\pgfpathclose%
\pgfusepath{stroke,fill}%
\end{pgfscope}%
\begin{pgfscope}%
\pgfpathrectangle{\pgfqpoint{0.100000in}{0.212622in}}{\pgfqpoint{3.696000in}{3.696000in}}%
\pgfusepath{clip}%
\pgfsetbuttcap%
\pgfsetroundjoin%
\definecolor{currentfill}{rgb}{0.121569,0.466667,0.705882}%
\pgfsetfillcolor{currentfill}%
\pgfsetfillopacity{0.663896}%
\pgfsetlinewidth{1.003750pt}%
\definecolor{currentstroke}{rgb}{0.121569,0.466667,0.705882}%
\pgfsetstrokecolor{currentstroke}%
\pgfsetstrokeopacity{0.663896}%
\pgfsetdash{}{0pt}%
\pgfpathmoveto{\pgfqpoint{0.726903in}{1.212679in}}%
\pgfpathcurveto{\pgfqpoint{0.735140in}{1.212679in}}{\pgfqpoint{0.743040in}{1.215952in}}{\pgfqpoint{0.748864in}{1.221776in}}%
\pgfpathcurveto{\pgfqpoint{0.754688in}{1.227599in}}{\pgfqpoint{0.757960in}{1.235500in}}{\pgfqpoint{0.757960in}{1.243736in}}%
\pgfpathcurveto{\pgfqpoint{0.757960in}{1.251972in}}{\pgfqpoint{0.754688in}{1.259872in}}{\pgfqpoint{0.748864in}{1.265696in}}%
\pgfpathcurveto{\pgfqpoint{0.743040in}{1.271520in}}{\pgfqpoint{0.735140in}{1.274792in}}{\pgfqpoint{0.726903in}{1.274792in}}%
\pgfpathcurveto{\pgfqpoint{0.718667in}{1.274792in}}{\pgfqpoint{0.710767in}{1.271520in}}{\pgfqpoint{0.704943in}{1.265696in}}%
\pgfpathcurveto{\pgfqpoint{0.699119in}{1.259872in}}{\pgfqpoint{0.695847in}{1.251972in}}{\pgfqpoint{0.695847in}{1.243736in}}%
\pgfpathcurveto{\pgfqpoint{0.695847in}{1.235500in}}{\pgfqpoint{0.699119in}{1.227599in}}{\pgfqpoint{0.704943in}{1.221776in}}%
\pgfpathcurveto{\pgfqpoint{0.710767in}{1.215952in}}{\pgfqpoint{0.718667in}{1.212679in}}{\pgfqpoint{0.726903in}{1.212679in}}%
\pgfpathclose%
\pgfusepath{stroke,fill}%
\end{pgfscope}%
\begin{pgfscope}%
\pgfpathrectangle{\pgfqpoint{0.100000in}{0.212622in}}{\pgfqpoint{3.696000in}{3.696000in}}%
\pgfusepath{clip}%
\pgfsetbuttcap%
\pgfsetroundjoin%
\definecolor{currentfill}{rgb}{0.121569,0.466667,0.705882}%
\pgfsetfillcolor{currentfill}%
\pgfsetfillopacity{0.663896}%
\pgfsetlinewidth{1.003750pt}%
\definecolor{currentstroke}{rgb}{0.121569,0.466667,0.705882}%
\pgfsetstrokecolor{currentstroke}%
\pgfsetstrokeopacity{0.663896}%
\pgfsetdash{}{0pt}%
\pgfpathmoveto{\pgfqpoint{0.726903in}{1.212679in}}%
\pgfpathcurveto{\pgfqpoint{0.735140in}{1.212679in}}{\pgfqpoint{0.743040in}{1.215952in}}{\pgfqpoint{0.748864in}{1.221776in}}%
\pgfpathcurveto{\pgfqpoint{0.754688in}{1.227599in}}{\pgfqpoint{0.757960in}{1.235500in}}{\pgfqpoint{0.757960in}{1.243736in}}%
\pgfpathcurveto{\pgfqpoint{0.757960in}{1.251972in}}{\pgfqpoint{0.754688in}{1.259872in}}{\pgfqpoint{0.748864in}{1.265696in}}%
\pgfpathcurveto{\pgfqpoint{0.743040in}{1.271520in}}{\pgfqpoint{0.735140in}{1.274792in}}{\pgfqpoint{0.726903in}{1.274792in}}%
\pgfpathcurveto{\pgfqpoint{0.718667in}{1.274792in}}{\pgfqpoint{0.710767in}{1.271520in}}{\pgfqpoint{0.704943in}{1.265696in}}%
\pgfpathcurveto{\pgfqpoint{0.699119in}{1.259872in}}{\pgfqpoint{0.695847in}{1.251972in}}{\pgfqpoint{0.695847in}{1.243736in}}%
\pgfpathcurveto{\pgfqpoint{0.695847in}{1.235500in}}{\pgfqpoint{0.699119in}{1.227599in}}{\pgfqpoint{0.704943in}{1.221776in}}%
\pgfpathcurveto{\pgfqpoint{0.710767in}{1.215952in}}{\pgfqpoint{0.718667in}{1.212679in}}{\pgfqpoint{0.726903in}{1.212679in}}%
\pgfpathclose%
\pgfusepath{stroke,fill}%
\end{pgfscope}%
\begin{pgfscope}%
\pgfpathrectangle{\pgfqpoint{0.100000in}{0.212622in}}{\pgfqpoint{3.696000in}{3.696000in}}%
\pgfusepath{clip}%
\pgfsetbuttcap%
\pgfsetroundjoin%
\definecolor{currentfill}{rgb}{0.121569,0.466667,0.705882}%
\pgfsetfillcolor{currentfill}%
\pgfsetfillopacity{0.663896}%
\pgfsetlinewidth{1.003750pt}%
\definecolor{currentstroke}{rgb}{0.121569,0.466667,0.705882}%
\pgfsetstrokecolor{currentstroke}%
\pgfsetstrokeopacity{0.663896}%
\pgfsetdash{}{0pt}%
\pgfpathmoveto{\pgfqpoint{0.726903in}{1.212679in}}%
\pgfpathcurveto{\pgfqpoint{0.735140in}{1.212679in}}{\pgfqpoint{0.743040in}{1.215952in}}{\pgfqpoint{0.748864in}{1.221776in}}%
\pgfpathcurveto{\pgfqpoint{0.754688in}{1.227599in}}{\pgfqpoint{0.757960in}{1.235500in}}{\pgfqpoint{0.757960in}{1.243736in}}%
\pgfpathcurveto{\pgfqpoint{0.757960in}{1.251972in}}{\pgfqpoint{0.754688in}{1.259872in}}{\pgfqpoint{0.748864in}{1.265696in}}%
\pgfpathcurveto{\pgfqpoint{0.743040in}{1.271520in}}{\pgfqpoint{0.735140in}{1.274792in}}{\pgfqpoint{0.726903in}{1.274792in}}%
\pgfpathcurveto{\pgfqpoint{0.718667in}{1.274792in}}{\pgfqpoint{0.710767in}{1.271520in}}{\pgfqpoint{0.704943in}{1.265696in}}%
\pgfpathcurveto{\pgfqpoint{0.699119in}{1.259872in}}{\pgfqpoint{0.695847in}{1.251972in}}{\pgfqpoint{0.695847in}{1.243736in}}%
\pgfpathcurveto{\pgfqpoint{0.695847in}{1.235500in}}{\pgfqpoint{0.699119in}{1.227599in}}{\pgfqpoint{0.704943in}{1.221776in}}%
\pgfpathcurveto{\pgfqpoint{0.710767in}{1.215952in}}{\pgfqpoint{0.718667in}{1.212679in}}{\pgfqpoint{0.726903in}{1.212679in}}%
\pgfpathclose%
\pgfusepath{stroke,fill}%
\end{pgfscope}%
\begin{pgfscope}%
\pgfpathrectangle{\pgfqpoint{0.100000in}{0.212622in}}{\pgfqpoint{3.696000in}{3.696000in}}%
\pgfusepath{clip}%
\pgfsetbuttcap%
\pgfsetroundjoin%
\definecolor{currentfill}{rgb}{0.121569,0.466667,0.705882}%
\pgfsetfillcolor{currentfill}%
\pgfsetfillopacity{0.667329}%
\pgfsetlinewidth{1.003750pt}%
\definecolor{currentstroke}{rgb}{0.121569,0.466667,0.705882}%
\pgfsetstrokecolor{currentstroke}%
\pgfsetstrokeopacity{0.667329}%
\pgfsetdash{}{0pt}%
\pgfpathmoveto{\pgfqpoint{3.189546in}{2.203403in}}%
\pgfpathcurveto{\pgfqpoint{3.197782in}{2.203403in}}{\pgfqpoint{3.205682in}{2.206675in}}{\pgfqpoint{3.211506in}{2.212499in}}%
\pgfpathcurveto{\pgfqpoint{3.217330in}{2.218323in}}{\pgfqpoint{3.220603in}{2.226223in}}{\pgfqpoint{3.220603in}{2.234460in}}%
\pgfpathcurveto{\pgfqpoint{3.220603in}{2.242696in}}{\pgfqpoint{3.217330in}{2.250596in}}{\pgfqpoint{3.211506in}{2.256420in}}%
\pgfpathcurveto{\pgfqpoint{3.205682in}{2.262244in}}{\pgfqpoint{3.197782in}{2.265516in}}{\pgfqpoint{3.189546in}{2.265516in}}%
\pgfpathcurveto{\pgfqpoint{3.181310in}{2.265516in}}{\pgfqpoint{3.173410in}{2.262244in}}{\pgfqpoint{3.167586in}{2.256420in}}%
\pgfpathcurveto{\pgfqpoint{3.161762in}{2.250596in}}{\pgfqpoint{3.158490in}{2.242696in}}{\pgfqpoint{3.158490in}{2.234460in}}%
\pgfpathcurveto{\pgfqpoint{3.158490in}{2.226223in}}{\pgfqpoint{3.161762in}{2.218323in}}{\pgfqpoint{3.167586in}{2.212499in}}%
\pgfpathcurveto{\pgfqpoint{3.173410in}{2.206675in}}{\pgfqpoint{3.181310in}{2.203403in}}{\pgfqpoint{3.189546in}{2.203403in}}%
\pgfpathclose%
\pgfusepath{stroke,fill}%
\end{pgfscope}%
\begin{pgfscope}%
\pgfpathrectangle{\pgfqpoint{0.100000in}{0.212622in}}{\pgfqpoint{3.696000in}{3.696000in}}%
\pgfusepath{clip}%
\pgfsetbuttcap%
\pgfsetroundjoin%
\definecolor{currentfill}{rgb}{0.121569,0.466667,0.705882}%
\pgfsetfillcolor{currentfill}%
\pgfsetfillopacity{0.673712}%
\pgfsetlinewidth{1.003750pt}%
\definecolor{currentstroke}{rgb}{0.121569,0.466667,0.705882}%
\pgfsetstrokecolor{currentstroke}%
\pgfsetstrokeopacity{0.673712}%
\pgfsetdash{}{0pt}%
\pgfpathmoveto{\pgfqpoint{3.167423in}{2.196227in}}%
\pgfpathcurveto{\pgfqpoint{3.175659in}{2.196227in}}{\pgfqpoint{3.183559in}{2.199499in}}{\pgfqpoint{3.189383in}{2.205323in}}%
\pgfpathcurveto{\pgfqpoint{3.195207in}{2.211147in}}{\pgfqpoint{3.198479in}{2.219047in}}{\pgfqpoint{3.198479in}{2.227283in}}%
\pgfpathcurveto{\pgfqpoint{3.198479in}{2.235520in}}{\pgfqpoint{3.195207in}{2.243420in}}{\pgfqpoint{3.189383in}{2.249244in}}%
\pgfpathcurveto{\pgfqpoint{3.183559in}{2.255068in}}{\pgfqpoint{3.175659in}{2.258340in}}{\pgfqpoint{3.167423in}{2.258340in}}%
\pgfpathcurveto{\pgfqpoint{3.159187in}{2.258340in}}{\pgfqpoint{3.151287in}{2.255068in}}{\pgfqpoint{3.145463in}{2.249244in}}%
\pgfpathcurveto{\pgfqpoint{3.139639in}{2.243420in}}{\pgfqpoint{3.136366in}{2.235520in}}{\pgfqpoint{3.136366in}{2.227283in}}%
\pgfpathcurveto{\pgfqpoint{3.136366in}{2.219047in}}{\pgfqpoint{3.139639in}{2.211147in}}{\pgfqpoint{3.145463in}{2.205323in}}%
\pgfpathcurveto{\pgfqpoint{3.151287in}{2.199499in}}{\pgfqpoint{3.159187in}{2.196227in}}{\pgfqpoint{3.167423in}{2.196227in}}%
\pgfpathclose%
\pgfusepath{stroke,fill}%
\end{pgfscope}%
\begin{pgfscope}%
\pgfpathrectangle{\pgfqpoint{0.100000in}{0.212622in}}{\pgfqpoint{3.696000in}{3.696000in}}%
\pgfusepath{clip}%
\pgfsetbuttcap%
\pgfsetroundjoin%
\definecolor{currentfill}{rgb}{0.121569,0.466667,0.705882}%
\pgfsetfillcolor{currentfill}%
\pgfsetfillopacity{0.681951}%
\pgfsetlinewidth{1.003750pt}%
\definecolor{currentstroke}{rgb}{0.121569,0.466667,0.705882}%
\pgfsetstrokecolor{currentstroke}%
\pgfsetstrokeopacity{0.681951}%
\pgfsetdash{}{0pt}%
\pgfpathmoveto{\pgfqpoint{3.149094in}{2.183491in}}%
\pgfpathcurveto{\pgfqpoint{3.157330in}{2.183491in}}{\pgfqpoint{3.165230in}{2.186763in}}{\pgfqpoint{3.171054in}{2.192587in}}%
\pgfpathcurveto{\pgfqpoint{3.176878in}{2.198411in}}{\pgfqpoint{3.180151in}{2.206311in}}{\pgfqpoint{3.180151in}{2.214548in}}%
\pgfpathcurveto{\pgfqpoint{3.180151in}{2.222784in}}{\pgfqpoint{3.176878in}{2.230684in}}{\pgfqpoint{3.171054in}{2.236508in}}%
\pgfpathcurveto{\pgfqpoint{3.165230in}{2.242332in}}{\pgfqpoint{3.157330in}{2.245604in}}{\pgfqpoint{3.149094in}{2.245604in}}%
\pgfpathcurveto{\pgfqpoint{3.140858in}{2.245604in}}{\pgfqpoint{3.132958in}{2.242332in}}{\pgfqpoint{3.127134in}{2.236508in}}%
\pgfpathcurveto{\pgfqpoint{3.121310in}{2.230684in}}{\pgfqpoint{3.118038in}{2.222784in}}{\pgfqpoint{3.118038in}{2.214548in}}%
\pgfpathcurveto{\pgfqpoint{3.118038in}{2.206311in}}{\pgfqpoint{3.121310in}{2.198411in}}{\pgfqpoint{3.127134in}{2.192587in}}%
\pgfpathcurveto{\pgfqpoint{3.132958in}{2.186763in}}{\pgfqpoint{3.140858in}{2.183491in}}{\pgfqpoint{3.149094in}{2.183491in}}%
\pgfpathclose%
\pgfusepath{stroke,fill}%
\end{pgfscope}%
\begin{pgfscope}%
\pgfpathrectangle{\pgfqpoint{0.100000in}{0.212622in}}{\pgfqpoint{3.696000in}{3.696000in}}%
\pgfusepath{clip}%
\pgfsetbuttcap%
\pgfsetroundjoin%
\definecolor{currentfill}{rgb}{0.121569,0.466667,0.705882}%
\pgfsetfillcolor{currentfill}%
\pgfsetfillopacity{0.690380}%
\pgfsetlinewidth{1.003750pt}%
\definecolor{currentstroke}{rgb}{0.121569,0.466667,0.705882}%
\pgfsetstrokecolor{currentstroke}%
\pgfsetstrokeopacity{0.690380}%
\pgfsetdash{}{0pt}%
\pgfpathmoveto{\pgfqpoint{3.121159in}{2.176459in}}%
\pgfpathcurveto{\pgfqpoint{3.129396in}{2.176459in}}{\pgfqpoint{3.137296in}{2.179731in}}{\pgfqpoint{3.143120in}{2.185555in}}%
\pgfpathcurveto{\pgfqpoint{3.148943in}{2.191379in}}{\pgfqpoint{3.152216in}{2.199279in}}{\pgfqpoint{3.152216in}{2.207515in}}%
\pgfpathcurveto{\pgfqpoint{3.152216in}{2.215752in}}{\pgfqpoint{3.148943in}{2.223652in}}{\pgfqpoint{3.143120in}{2.229476in}}%
\pgfpathcurveto{\pgfqpoint{3.137296in}{2.235300in}}{\pgfqpoint{3.129396in}{2.238572in}}{\pgfqpoint{3.121159in}{2.238572in}}%
\pgfpathcurveto{\pgfqpoint{3.112923in}{2.238572in}}{\pgfqpoint{3.105023in}{2.235300in}}{\pgfqpoint{3.099199in}{2.229476in}}%
\pgfpathcurveto{\pgfqpoint{3.093375in}{2.223652in}}{\pgfqpoint{3.090103in}{2.215752in}}{\pgfqpoint{3.090103in}{2.207515in}}%
\pgfpathcurveto{\pgfqpoint{3.090103in}{2.199279in}}{\pgfqpoint{3.093375in}{2.191379in}}{\pgfqpoint{3.099199in}{2.185555in}}%
\pgfpathcurveto{\pgfqpoint{3.105023in}{2.179731in}}{\pgfqpoint{3.112923in}{2.176459in}}{\pgfqpoint{3.121159in}{2.176459in}}%
\pgfpathclose%
\pgfusepath{stroke,fill}%
\end{pgfscope}%
\begin{pgfscope}%
\pgfpathrectangle{\pgfqpoint{0.100000in}{0.212622in}}{\pgfqpoint{3.696000in}{3.696000in}}%
\pgfusepath{clip}%
\pgfsetbuttcap%
\pgfsetroundjoin%
\definecolor{currentfill}{rgb}{0.121569,0.466667,0.705882}%
\pgfsetfillcolor{currentfill}%
\pgfsetfillopacity{0.696249}%
\pgfsetlinewidth{1.003750pt}%
\definecolor{currentstroke}{rgb}{0.121569,0.466667,0.705882}%
\pgfsetstrokecolor{currentstroke}%
\pgfsetstrokeopacity{0.696249}%
\pgfsetdash{}{0pt}%
\pgfpathmoveto{\pgfqpoint{3.109794in}{2.174407in}}%
\pgfpathcurveto{\pgfqpoint{3.118031in}{2.174407in}}{\pgfqpoint{3.125931in}{2.177679in}}{\pgfqpoint{3.131755in}{2.183503in}}%
\pgfpathcurveto{\pgfqpoint{3.137579in}{2.189327in}}{\pgfqpoint{3.140851in}{2.197227in}}{\pgfqpoint{3.140851in}{2.205463in}}%
\pgfpathcurveto{\pgfqpoint{3.140851in}{2.213699in}}{\pgfqpoint{3.137579in}{2.221599in}}{\pgfqpoint{3.131755in}{2.227423in}}%
\pgfpathcurveto{\pgfqpoint{3.125931in}{2.233247in}}{\pgfqpoint{3.118031in}{2.236520in}}{\pgfqpoint{3.109794in}{2.236520in}}%
\pgfpathcurveto{\pgfqpoint{3.101558in}{2.236520in}}{\pgfqpoint{3.093658in}{2.233247in}}{\pgfqpoint{3.087834in}{2.227423in}}%
\pgfpathcurveto{\pgfqpoint{3.082010in}{2.221599in}}{\pgfqpoint{3.078738in}{2.213699in}}{\pgfqpoint{3.078738in}{2.205463in}}%
\pgfpathcurveto{\pgfqpoint{3.078738in}{2.197227in}}{\pgfqpoint{3.082010in}{2.189327in}}{\pgfqpoint{3.087834in}{2.183503in}}%
\pgfpathcurveto{\pgfqpoint{3.093658in}{2.177679in}}{\pgfqpoint{3.101558in}{2.174407in}}{\pgfqpoint{3.109794in}{2.174407in}}%
\pgfpathclose%
\pgfusepath{stroke,fill}%
\end{pgfscope}%
\begin{pgfscope}%
\pgfpathrectangle{\pgfqpoint{0.100000in}{0.212622in}}{\pgfqpoint{3.696000in}{3.696000in}}%
\pgfusepath{clip}%
\pgfsetbuttcap%
\pgfsetroundjoin%
\definecolor{currentfill}{rgb}{0.121569,0.466667,0.705882}%
\pgfsetfillcolor{currentfill}%
\pgfsetfillopacity{0.698614}%
\pgfsetlinewidth{1.003750pt}%
\definecolor{currentstroke}{rgb}{0.121569,0.466667,0.705882}%
\pgfsetstrokecolor{currentstroke}%
\pgfsetstrokeopacity{0.698614}%
\pgfsetdash{}{0pt}%
\pgfpathmoveto{\pgfqpoint{3.101195in}{2.171147in}}%
\pgfpathcurveto{\pgfqpoint{3.109431in}{2.171147in}}{\pgfqpoint{3.117331in}{2.174420in}}{\pgfqpoint{3.123155in}{2.180244in}}%
\pgfpathcurveto{\pgfqpoint{3.128979in}{2.186068in}}{\pgfqpoint{3.132252in}{2.193968in}}{\pgfqpoint{3.132252in}{2.202204in}}%
\pgfpathcurveto{\pgfqpoint{3.132252in}{2.210440in}}{\pgfqpoint{3.128979in}{2.218340in}}{\pgfqpoint{3.123155in}{2.224164in}}%
\pgfpathcurveto{\pgfqpoint{3.117331in}{2.229988in}}{\pgfqpoint{3.109431in}{2.233260in}}{\pgfqpoint{3.101195in}{2.233260in}}%
\pgfpathcurveto{\pgfqpoint{3.092959in}{2.233260in}}{\pgfqpoint{3.085059in}{2.229988in}}{\pgfqpoint{3.079235in}{2.224164in}}%
\pgfpathcurveto{\pgfqpoint{3.073411in}{2.218340in}}{\pgfqpoint{3.070139in}{2.210440in}}{\pgfqpoint{3.070139in}{2.202204in}}%
\pgfpathcurveto{\pgfqpoint{3.070139in}{2.193968in}}{\pgfqpoint{3.073411in}{2.186068in}}{\pgfqpoint{3.079235in}{2.180244in}}%
\pgfpathcurveto{\pgfqpoint{3.085059in}{2.174420in}}{\pgfqpoint{3.092959in}{2.171147in}}{\pgfqpoint{3.101195in}{2.171147in}}%
\pgfpathclose%
\pgfusepath{stroke,fill}%
\end{pgfscope}%
\begin{pgfscope}%
\pgfpathrectangle{\pgfqpoint{0.100000in}{0.212622in}}{\pgfqpoint{3.696000in}{3.696000in}}%
\pgfusepath{clip}%
\pgfsetbuttcap%
\pgfsetroundjoin%
\definecolor{currentfill}{rgb}{0.121569,0.466667,0.705882}%
\pgfsetfillcolor{currentfill}%
\pgfsetfillopacity{0.700434}%
\pgfsetlinewidth{1.003750pt}%
\definecolor{currentstroke}{rgb}{0.121569,0.466667,0.705882}%
\pgfsetstrokecolor{currentstroke}%
\pgfsetstrokeopacity{0.700434}%
\pgfsetdash{}{0pt}%
\pgfpathmoveto{\pgfqpoint{3.097933in}{2.170568in}}%
\pgfpathcurveto{\pgfqpoint{3.106170in}{2.170568in}}{\pgfqpoint{3.114070in}{2.173840in}}{\pgfqpoint{3.119894in}{2.179664in}}%
\pgfpathcurveto{\pgfqpoint{3.125718in}{2.185488in}}{\pgfqpoint{3.128990in}{2.193388in}}{\pgfqpoint{3.128990in}{2.201624in}}%
\pgfpathcurveto{\pgfqpoint{3.128990in}{2.209861in}}{\pgfqpoint{3.125718in}{2.217761in}}{\pgfqpoint{3.119894in}{2.223585in}}%
\pgfpathcurveto{\pgfqpoint{3.114070in}{2.229409in}}{\pgfqpoint{3.106170in}{2.232681in}}{\pgfqpoint{3.097933in}{2.232681in}}%
\pgfpathcurveto{\pgfqpoint{3.089697in}{2.232681in}}{\pgfqpoint{3.081797in}{2.229409in}}{\pgfqpoint{3.075973in}{2.223585in}}%
\pgfpathcurveto{\pgfqpoint{3.070149in}{2.217761in}}{\pgfqpoint{3.066877in}{2.209861in}}{\pgfqpoint{3.066877in}{2.201624in}}%
\pgfpathcurveto{\pgfqpoint{3.066877in}{2.193388in}}{\pgfqpoint{3.070149in}{2.185488in}}{\pgfqpoint{3.075973in}{2.179664in}}%
\pgfpathcurveto{\pgfqpoint{3.081797in}{2.173840in}}{\pgfqpoint{3.089697in}{2.170568in}}{\pgfqpoint{3.097933in}{2.170568in}}%
\pgfpathclose%
\pgfusepath{stroke,fill}%
\end{pgfscope}%
\begin{pgfscope}%
\pgfpathrectangle{\pgfqpoint{0.100000in}{0.212622in}}{\pgfqpoint{3.696000in}{3.696000in}}%
\pgfusepath{clip}%
\pgfsetbuttcap%
\pgfsetroundjoin%
\definecolor{currentfill}{rgb}{0.121569,0.466667,0.705882}%
\pgfsetfillcolor{currentfill}%
\pgfsetfillopacity{0.701311}%
\pgfsetlinewidth{1.003750pt}%
\definecolor{currentstroke}{rgb}{0.121569,0.466667,0.705882}%
\pgfsetstrokecolor{currentstroke}%
\pgfsetstrokeopacity{0.701311}%
\pgfsetdash{}{0pt}%
\pgfpathmoveto{\pgfqpoint{3.095579in}{2.170204in}}%
\pgfpathcurveto{\pgfqpoint{3.103815in}{2.170204in}}{\pgfqpoint{3.111715in}{2.173477in}}{\pgfqpoint{3.117539in}{2.179301in}}%
\pgfpathcurveto{\pgfqpoint{3.123363in}{2.185125in}}{\pgfqpoint{3.126636in}{2.193025in}}{\pgfqpoint{3.126636in}{2.201261in}}%
\pgfpathcurveto{\pgfqpoint{3.126636in}{2.209497in}}{\pgfqpoint{3.123363in}{2.217397in}}{\pgfqpoint{3.117539in}{2.223221in}}%
\pgfpathcurveto{\pgfqpoint{3.111715in}{2.229045in}}{\pgfqpoint{3.103815in}{2.232317in}}{\pgfqpoint{3.095579in}{2.232317in}}%
\pgfpathcurveto{\pgfqpoint{3.087343in}{2.232317in}}{\pgfqpoint{3.079443in}{2.229045in}}{\pgfqpoint{3.073619in}{2.223221in}}%
\pgfpathcurveto{\pgfqpoint{3.067795in}{2.217397in}}{\pgfqpoint{3.064523in}{2.209497in}}{\pgfqpoint{3.064523in}{2.201261in}}%
\pgfpathcurveto{\pgfqpoint{3.064523in}{2.193025in}}{\pgfqpoint{3.067795in}{2.185125in}}{\pgfqpoint{3.073619in}{2.179301in}}%
\pgfpathcurveto{\pgfqpoint{3.079443in}{2.173477in}}{\pgfqpoint{3.087343in}{2.170204in}}{\pgfqpoint{3.095579in}{2.170204in}}%
\pgfpathclose%
\pgfusepath{stroke,fill}%
\end{pgfscope}%
\begin{pgfscope}%
\pgfpathrectangle{\pgfqpoint{0.100000in}{0.212622in}}{\pgfqpoint{3.696000in}{3.696000in}}%
\pgfusepath{clip}%
\pgfsetbuttcap%
\pgfsetroundjoin%
\definecolor{currentfill}{rgb}{0.121569,0.466667,0.705882}%
\pgfsetfillcolor{currentfill}%
\pgfsetfillopacity{0.701795}%
\pgfsetlinewidth{1.003750pt}%
\definecolor{currentstroke}{rgb}{0.121569,0.466667,0.705882}%
\pgfsetstrokecolor{currentstroke}%
\pgfsetstrokeopacity{0.701795}%
\pgfsetdash{}{0pt}%
\pgfpathmoveto{\pgfqpoint{3.094279in}{2.170027in}}%
\pgfpathcurveto{\pgfqpoint{3.102515in}{2.170027in}}{\pgfqpoint{3.110415in}{2.173299in}}{\pgfqpoint{3.116239in}{2.179123in}}%
\pgfpathcurveto{\pgfqpoint{3.122063in}{2.184947in}}{\pgfqpoint{3.125336in}{2.192847in}}{\pgfqpoint{3.125336in}{2.201083in}}%
\pgfpathcurveto{\pgfqpoint{3.125336in}{2.209320in}}{\pgfqpoint{3.122063in}{2.217220in}}{\pgfqpoint{3.116239in}{2.223043in}}%
\pgfpathcurveto{\pgfqpoint{3.110415in}{2.228867in}}{\pgfqpoint{3.102515in}{2.232140in}}{\pgfqpoint{3.094279in}{2.232140in}}%
\pgfpathcurveto{\pgfqpoint{3.086043in}{2.232140in}}{\pgfqpoint{3.078143in}{2.228867in}}{\pgfqpoint{3.072319in}{2.223043in}}%
\pgfpathcurveto{\pgfqpoint{3.066495in}{2.217220in}}{\pgfqpoint{3.063223in}{2.209320in}}{\pgfqpoint{3.063223in}{2.201083in}}%
\pgfpathcurveto{\pgfqpoint{3.063223in}{2.192847in}}{\pgfqpoint{3.066495in}{2.184947in}}{\pgfqpoint{3.072319in}{2.179123in}}%
\pgfpathcurveto{\pgfqpoint{3.078143in}{2.173299in}}{\pgfqpoint{3.086043in}{2.170027in}}{\pgfqpoint{3.094279in}{2.170027in}}%
\pgfpathclose%
\pgfusepath{stroke,fill}%
\end{pgfscope}%
\begin{pgfscope}%
\pgfpathrectangle{\pgfqpoint{0.100000in}{0.212622in}}{\pgfqpoint{3.696000in}{3.696000in}}%
\pgfusepath{clip}%
\pgfsetbuttcap%
\pgfsetroundjoin%
\definecolor{currentfill}{rgb}{0.121569,0.466667,0.705882}%
\pgfsetfillcolor{currentfill}%
\pgfsetfillopacity{0.702066}%
\pgfsetlinewidth{1.003750pt}%
\definecolor{currentstroke}{rgb}{0.121569,0.466667,0.705882}%
\pgfsetstrokecolor{currentstroke}%
\pgfsetstrokeopacity{0.702066}%
\pgfsetdash{}{0pt}%
\pgfpathmoveto{\pgfqpoint{3.093702in}{2.169755in}}%
\pgfpathcurveto{\pgfqpoint{3.101938in}{2.169755in}}{\pgfqpoint{3.109838in}{2.173027in}}{\pgfqpoint{3.115662in}{2.178851in}}%
\pgfpathcurveto{\pgfqpoint{3.121486in}{2.184675in}}{\pgfqpoint{3.124758in}{2.192575in}}{\pgfqpoint{3.124758in}{2.200811in}}%
\pgfpathcurveto{\pgfqpoint{3.124758in}{2.209048in}}{\pgfqpoint{3.121486in}{2.216948in}}{\pgfqpoint{3.115662in}{2.222772in}}%
\pgfpathcurveto{\pgfqpoint{3.109838in}{2.228596in}}{\pgfqpoint{3.101938in}{2.231868in}}{\pgfqpoint{3.093702in}{2.231868in}}%
\pgfpathcurveto{\pgfqpoint{3.085466in}{2.231868in}}{\pgfqpoint{3.077566in}{2.228596in}}{\pgfqpoint{3.071742in}{2.222772in}}%
\pgfpathcurveto{\pgfqpoint{3.065918in}{2.216948in}}{\pgfqpoint{3.062645in}{2.209048in}}{\pgfqpoint{3.062645in}{2.200811in}}%
\pgfpathcurveto{\pgfqpoint{3.062645in}{2.192575in}}{\pgfqpoint{3.065918in}{2.184675in}}{\pgfqpoint{3.071742in}{2.178851in}}%
\pgfpathcurveto{\pgfqpoint{3.077566in}{2.173027in}}{\pgfqpoint{3.085466in}{2.169755in}}{\pgfqpoint{3.093702in}{2.169755in}}%
\pgfpathclose%
\pgfusepath{stroke,fill}%
\end{pgfscope}%
\begin{pgfscope}%
\pgfpathrectangle{\pgfqpoint{0.100000in}{0.212622in}}{\pgfqpoint{3.696000in}{3.696000in}}%
\pgfusepath{clip}%
\pgfsetbuttcap%
\pgfsetroundjoin%
\definecolor{currentfill}{rgb}{0.121569,0.466667,0.705882}%
\pgfsetfillcolor{currentfill}%
\pgfsetfillopacity{0.702216}%
\pgfsetlinewidth{1.003750pt}%
\definecolor{currentstroke}{rgb}{0.121569,0.466667,0.705882}%
\pgfsetstrokecolor{currentstroke}%
\pgfsetstrokeopacity{0.702216}%
\pgfsetdash{}{0pt}%
\pgfpathmoveto{\pgfqpoint{3.093276in}{2.169784in}}%
\pgfpathcurveto{\pgfqpoint{3.101512in}{2.169784in}}{\pgfqpoint{3.109412in}{2.173056in}}{\pgfqpoint{3.115236in}{2.178880in}}%
\pgfpathcurveto{\pgfqpoint{3.121060in}{2.184704in}}{\pgfqpoint{3.124332in}{2.192604in}}{\pgfqpoint{3.124332in}{2.200840in}}%
\pgfpathcurveto{\pgfqpoint{3.124332in}{2.209077in}}{\pgfqpoint{3.121060in}{2.216977in}}{\pgfqpoint{3.115236in}{2.222801in}}%
\pgfpathcurveto{\pgfqpoint{3.109412in}{2.228624in}}{\pgfqpoint{3.101512in}{2.231897in}}{\pgfqpoint{3.093276in}{2.231897in}}%
\pgfpathcurveto{\pgfqpoint{3.085040in}{2.231897in}}{\pgfqpoint{3.077140in}{2.228624in}}{\pgfqpoint{3.071316in}{2.222801in}}%
\pgfpathcurveto{\pgfqpoint{3.065492in}{2.216977in}}{\pgfqpoint{3.062219in}{2.209077in}}{\pgfqpoint{3.062219in}{2.200840in}}%
\pgfpathcurveto{\pgfqpoint{3.062219in}{2.192604in}}{\pgfqpoint{3.065492in}{2.184704in}}{\pgfqpoint{3.071316in}{2.178880in}}%
\pgfpathcurveto{\pgfqpoint{3.077140in}{2.173056in}}{\pgfqpoint{3.085040in}{2.169784in}}{\pgfqpoint{3.093276in}{2.169784in}}%
\pgfpathclose%
\pgfusepath{stroke,fill}%
\end{pgfscope}%
\begin{pgfscope}%
\pgfpathrectangle{\pgfqpoint{0.100000in}{0.212622in}}{\pgfqpoint{3.696000in}{3.696000in}}%
\pgfusepath{clip}%
\pgfsetbuttcap%
\pgfsetroundjoin%
\definecolor{currentfill}{rgb}{0.121569,0.466667,0.705882}%
\pgfsetfillcolor{currentfill}%
\pgfsetfillopacity{0.702300}%
\pgfsetlinewidth{1.003750pt}%
\definecolor{currentstroke}{rgb}{0.121569,0.466667,0.705882}%
\pgfsetstrokecolor{currentstroke}%
\pgfsetstrokeopacity{0.702300}%
\pgfsetdash{}{0pt}%
\pgfpathmoveto{\pgfqpoint{3.093123in}{2.169691in}}%
\pgfpathcurveto{\pgfqpoint{3.101359in}{2.169691in}}{\pgfqpoint{3.109259in}{2.172963in}}{\pgfqpoint{3.115083in}{2.178787in}}%
\pgfpathcurveto{\pgfqpoint{3.120907in}{2.184611in}}{\pgfqpoint{3.124179in}{2.192511in}}{\pgfqpoint{3.124179in}{2.200747in}}%
\pgfpathcurveto{\pgfqpoint{3.124179in}{2.208984in}}{\pgfqpoint{3.120907in}{2.216884in}}{\pgfqpoint{3.115083in}{2.222708in}}%
\pgfpathcurveto{\pgfqpoint{3.109259in}{2.228532in}}{\pgfqpoint{3.101359in}{2.231804in}}{\pgfqpoint{3.093123in}{2.231804in}}%
\pgfpathcurveto{\pgfqpoint{3.084887in}{2.231804in}}{\pgfqpoint{3.076986in}{2.228532in}}{\pgfqpoint{3.071163in}{2.222708in}}%
\pgfpathcurveto{\pgfqpoint{3.065339in}{2.216884in}}{\pgfqpoint{3.062066in}{2.208984in}}{\pgfqpoint{3.062066in}{2.200747in}}%
\pgfpathcurveto{\pgfqpoint{3.062066in}{2.192511in}}{\pgfqpoint{3.065339in}{2.184611in}}{\pgfqpoint{3.071163in}{2.178787in}}%
\pgfpathcurveto{\pgfqpoint{3.076986in}{2.172963in}}{\pgfqpoint{3.084887in}{2.169691in}}{\pgfqpoint{3.093123in}{2.169691in}}%
\pgfpathclose%
\pgfusepath{stroke,fill}%
\end{pgfscope}%
\begin{pgfscope}%
\pgfpathrectangle{\pgfqpoint{0.100000in}{0.212622in}}{\pgfqpoint{3.696000in}{3.696000in}}%
\pgfusepath{clip}%
\pgfsetbuttcap%
\pgfsetroundjoin%
\definecolor{currentfill}{rgb}{0.121569,0.466667,0.705882}%
\pgfsetfillcolor{currentfill}%
\pgfsetfillopacity{0.703016}%
\pgfsetlinewidth{1.003750pt}%
\definecolor{currentstroke}{rgb}{0.121569,0.466667,0.705882}%
\pgfsetstrokecolor{currentstroke}%
\pgfsetstrokeopacity{0.703016}%
\pgfsetdash{}{0pt}%
\pgfpathmoveto{\pgfqpoint{3.090362in}{2.168475in}}%
\pgfpathcurveto{\pgfqpoint{3.098598in}{2.168475in}}{\pgfqpoint{3.106498in}{2.171747in}}{\pgfqpoint{3.112322in}{2.177571in}}%
\pgfpathcurveto{\pgfqpoint{3.118146in}{2.183395in}}{\pgfqpoint{3.121418in}{2.191295in}}{\pgfqpoint{3.121418in}{2.199531in}}%
\pgfpathcurveto{\pgfqpoint{3.121418in}{2.207768in}}{\pgfqpoint{3.118146in}{2.215668in}}{\pgfqpoint{3.112322in}{2.221492in}}%
\pgfpathcurveto{\pgfqpoint{3.106498in}{2.227316in}}{\pgfqpoint{3.098598in}{2.230588in}}{\pgfqpoint{3.090362in}{2.230588in}}%
\pgfpathcurveto{\pgfqpoint{3.082125in}{2.230588in}}{\pgfqpoint{3.074225in}{2.227316in}}{\pgfqpoint{3.068401in}{2.221492in}}%
\pgfpathcurveto{\pgfqpoint{3.062577in}{2.215668in}}{\pgfqpoint{3.059305in}{2.207768in}}{\pgfqpoint{3.059305in}{2.199531in}}%
\pgfpathcurveto{\pgfqpoint{3.059305in}{2.191295in}}{\pgfqpoint{3.062577in}{2.183395in}}{\pgfqpoint{3.068401in}{2.177571in}}%
\pgfpathcurveto{\pgfqpoint{3.074225in}{2.171747in}}{\pgfqpoint{3.082125in}{2.168475in}}{\pgfqpoint{3.090362in}{2.168475in}}%
\pgfpathclose%
\pgfusepath{stroke,fill}%
\end{pgfscope}%
\begin{pgfscope}%
\pgfpathrectangle{\pgfqpoint{0.100000in}{0.212622in}}{\pgfqpoint{3.696000in}{3.696000in}}%
\pgfusepath{clip}%
\pgfsetbuttcap%
\pgfsetroundjoin%
\definecolor{currentfill}{rgb}{0.121569,0.466667,0.705882}%
\pgfsetfillcolor{currentfill}%
\pgfsetfillopacity{0.705349}%
\pgfsetlinewidth{1.003750pt}%
\definecolor{currentstroke}{rgb}{0.121569,0.466667,0.705882}%
\pgfsetstrokecolor{currentstroke}%
\pgfsetstrokeopacity{0.705349}%
\pgfsetdash{}{0pt}%
\pgfpathmoveto{\pgfqpoint{3.085849in}{2.164764in}}%
\pgfpathcurveto{\pgfqpoint{3.094085in}{2.164764in}}{\pgfqpoint{3.101985in}{2.168037in}}{\pgfqpoint{3.107809in}{2.173861in}}%
\pgfpathcurveto{\pgfqpoint{3.113633in}{2.179684in}}{\pgfqpoint{3.116905in}{2.187584in}}{\pgfqpoint{3.116905in}{2.195821in}}%
\pgfpathcurveto{\pgfqpoint{3.116905in}{2.204057in}}{\pgfqpoint{3.113633in}{2.211957in}}{\pgfqpoint{3.107809in}{2.217781in}}%
\pgfpathcurveto{\pgfqpoint{3.101985in}{2.223605in}}{\pgfqpoint{3.094085in}{2.226877in}}{\pgfqpoint{3.085849in}{2.226877in}}%
\pgfpathcurveto{\pgfqpoint{3.077613in}{2.226877in}}{\pgfqpoint{3.069712in}{2.223605in}}{\pgfqpoint{3.063889in}{2.217781in}}%
\pgfpathcurveto{\pgfqpoint{3.058065in}{2.211957in}}{\pgfqpoint{3.054792in}{2.204057in}}{\pgfqpoint{3.054792in}{2.195821in}}%
\pgfpathcurveto{\pgfqpoint{3.054792in}{2.187584in}}{\pgfqpoint{3.058065in}{2.179684in}}{\pgfqpoint{3.063889in}{2.173861in}}%
\pgfpathcurveto{\pgfqpoint{3.069712in}{2.168037in}}{\pgfqpoint{3.077613in}{2.164764in}}{\pgfqpoint{3.085849in}{2.164764in}}%
\pgfpathclose%
\pgfusepath{stroke,fill}%
\end{pgfscope}%
\begin{pgfscope}%
\pgfpathrectangle{\pgfqpoint{0.100000in}{0.212622in}}{\pgfqpoint{3.696000in}{3.696000in}}%
\pgfusepath{clip}%
\pgfsetbuttcap%
\pgfsetroundjoin%
\definecolor{currentfill}{rgb}{0.121569,0.466667,0.705882}%
\pgfsetfillcolor{currentfill}%
\pgfsetfillopacity{0.706548}%
\pgfsetlinewidth{1.003750pt}%
\definecolor{currentstroke}{rgb}{0.121569,0.466667,0.705882}%
\pgfsetstrokecolor{currentstroke}%
\pgfsetstrokeopacity{0.706548}%
\pgfsetdash{}{0pt}%
\pgfpathmoveto{\pgfqpoint{3.081963in}{2.164231in}}%
\pgfpathcurveto{\pgfqpoint{3.090199in}{2.164231in}}{\pgfqpoint{3.098099in}{2.167503in}}{\pgfqpoint{3.103923in}{2.173327in}}%
\pgfpathcurveto{\pgfqpoint{3.109747in}{2.179151in}}{\pgfqpoint{3.113019in}{2.187051in}}{\pgfqpoint{3.113019in}{2.195287in}}%
\pgfpathcurveto{\pgfqpoint{3.113019in}{2.203524in}}{\pgfqpoint{3.109747in}{2.211424in}}{\pgfqpoint{3.103923in}{2.217248in}}%
\pgfpathcurveto{\pgfqpoint{3.098099in}{2.223072in}}{\pgfqpoint{3.090199in}{2.226344in}}{\pgfqpoint{3.081963in}{2.226344in}}%
\pgfpathcurveto{\pgfqpoint{3.073727in}{2.226344in}}{\pgfqpoint{3.065827in}{2.223072in}}{\pgfqpoint{3.060003in}{2.217248in}}%
\pgfpathcurveto{\pgfqpoint{3.054179in}{2.211424in}}{\pgfqpoint{3.050906in}{2.203524in}}{\pgfqpoint{3.050906in}{2.195287in}}%
\pgfpathcurveto{\pgfqpoint{3.050906in}{2.187051in}}{\pgfqpoint{3.054179in}{2.179151in}}{\pgfqpoint{3.060003in}{2.173327in}}%
\pgfpathcurveto{\pgfqpoint{3.065827in}{2.167503in}}{\pgfqpoint{3.073727in}{2.164231in}}{\pgfqpoint{3.081963in}{2.164231in}}%
\pgfpathclose%
\pgfusepath{stroke,fill}%
\end{pgfscope}%
\begin{pgfscope}%
\pgfpathrectangle{\pgfqpoint{0.100000in}{0.212622in}}{\pgfqpoint{3.696000in}{3.696000in}}%
\pgfusepath{clip}%
\pgfsetbuttcap%
\pgfsetroundjoin%
\definecolor{currentfill}{rgb}{0.121569,0.466667,0.705882}%
\pgfsetfillcolor{currentfill}%
\pgfsetfillopacity{0.708797}%
\pgfsetlinewidth{1.003750pt}%
\definecolor{currentstroke}{rgb}{0.121569,0.466667,0.705882}%
\pgfsetstrokecolor{currentstroke}%
\pgfsetstrokeopacity{0.708797}%
\pgfsetdash{}{0pt}%
\pgfpathmoveto{\pgfqpoint{3.077137in}{2.159858in}}%
\pgfpathcurveto{\pgfqpoint{3.085374in}{2.159858in}}{\pgfqpoint{3.093274in}{2.163131in}}{\pgfqpoint{3.099098in}{2.168955in}}%
\pgfpathcurveto{\pgfqpoint{3.104922in}{2.174779in}}{\pgfqpoint{3.108194in}{2.182679in}}{\pgfqpoint{3.108194in}{2.190915in}}%
\pgfpathcurveto{\pgfqpoint{3.108194in}{2.199151in}}{\pgfqpoint{3.104922in}{2.207051in}}{\pgfqpoint{3.099098in}{2.212875in}}%
\pgfpathcurveto{\pgfqpoint{3.093274in}{2.218699in}}{\pgfqpoint{3.085374in}{2.221971in}}{\pgfqpoint{3.077137in}{2.221971in}}%
\pgfpathcurveto{\pgfqpoint{3.068901in}{2.221971in}}{\pgfqpoint{3.061001in}{2.218699in}}{\pgfqpoint{3.055177in}{2.212875in}}%
\pgfpathcurveto{\pgfqpoint{3.049353in}{2.207051in}}{\pgfqpoint{3.046081in}{2.199151in}}{\pgfqpoint{3.046081in}{2.190915in}}%
\pgfpathcurveto{\pgfqpoint{3.046081in}{2.182679in}}{\pgfqpoint{3.049353in}{2.174779in}}{\pgfqpoint{3.055177in}{2.168955in}}%
\pgfpathcurveto{\pgfqpoint{3.061001in}{2.163131in}}{\pgfqpoint{3.068901in}{2.159858in}}{\pgfqpoint{3.077137in}{2.159858in}}%
\pgfpathclose%
\pgfusepath{stroke,fill}%
\end{pgfscope}%
\begin{pgfscope}%
\pgfpathrectangle{\pgfqpoint{0.100000in}{0.212622in}}{\pgfqpoint{3.696000in}{3.696000in}}%
\pgfusepath{clip}%
\pgfsetbuttcap%
\pgfsetroundjoin%
\definecolor{currentfill}{rgb}{0.121569,0.466667,0.705882}%
\pgfsetfillcolor{currentfill}%
\pgfsetfillopacity{0.712157}%
\pgfsetlinewidth{1.003750pt}%
\definecolor{currentstroke}{rgb}{0.121569,0.466667,0.705882}%
\pgfsetstrokecolor{currentstroke}%
\pgfsetstrokeopacity{0.712157}%
\pgfsetdash{}{0pt}%
\pgfpathmoveto{\pgfqpoint{3.069423in}{2.158217in}}%
\pgfpathcurveto{\pgfqpoint{3.077659in}{2.158217in}}{\pgfqpoint{3.085559in}{2.161489in}}{\pgfqpoint{3.091383in}{2.167313in}}%
\pgfpathcurveto{\pgfqpoint{3.097207in}{2.173137in}}{\pgfqpoint{3.100480in}{2.181037in}}{\pgfqpoint{3.100480in}{2.189273in}}%
\pgfpathcurveto{\pgfqpoint{3.100480in}{2.197510in}}{\pgfqpoint{3.097207in}{2.205410in}}{\pgfqpoint{3.091383in}{2.211234in}}%
\pgfpathcurveto{\pgfqpoint{3.085559in}{2.217058in}}{\pgfqpoint{3.077659in}{2.220330in}}{\pgfqpoint{3.069423in}{2.220330in}}%
\pgfpathcurveto{\pgfqpoint{3.061187in}{2.220330in}}{\pgfqpoint{3.053287in}{2.217058in}}{\pgfqpoint{3.047463in}{2.211234in}}%
\pgfpathcurveto{\pgfqpoint{3.041639in}{2.205410in}}{\pgfqpoint{3.038367in}{2.197510in}}{\pgfqpoint{3.038367in}{2.189273in}}%
\pgfpathcurveto{\pgfqpoint{3.038367in}{2.181037in}}{\pgfqpoint{3.041639in}{2.173137in}}{\pgfqpoint{3.047463in}{2.167313in}}%
\pgfpathcurveto{\pgfqpoint{3.053287in}{2.161489in}}{\pgfqpoint{3.061187in}{2.158217in}}{\pgfqpoint{3.069423in}{2.158217in}}%
\pgfpathclose%
\pgfusepath{stroke,fill}%
\end{pgfscope}%
\begin{pgfscope}%
\pgfpathrectangle{\pgfqpoint{0.100000in}{0.212622in}}{\pgfqpoint{3.696000in}{3.696000in}}%
\pgfusepath{clip}%
\pgfsetbuttcap%
\pgfsetroundjoin%
\definecolor{currentfill}{rgb}{0.121569,0.466667,0.705882}%
\pgfsetfillcolor{currentfill}%
\pgfsetfillopacity{0.716534}%
\pgfsetlinewidth{1.003750pt}%
\definecolor{currentstroke}{rgb}{0.121569,0.466667,0.705882}%
\pgfsetstrokecolor{currentstroke}%
\pgfsetstrokeopacity{0.716534}%
\pgfsetdash{}{0pt}%
\pgfpathmoveto{\pgfqpoint{3.059546in}{2.153338in}}%
\pgfpathcurveto{\pgfqpoint{3.067782in}{2.153338in}}{\pgfqpoint{3.075682in}{2.156610in}}{\pgfqpoint{3.081506in}{2.162434in}}%
\pgfpathcurveto{\pgfqpoint{3.087330in}{2.168258in}}{\pgfqpoint{3.090602in}{2.176158in}}{\pgfqpoint{3.090602in}{2.184394in}}%
\pgfpathcurveto{\pgfqpoint{3.090602in}{2.192631in}}{\pgfqpoint{3.087330in}{2.200531in}}{\pgfqpoint{3.081506in}{2.206355in}}%
\pgfpathcurveto{\pgfqpoint{3.075682in}{2.212179in}}{\pgfqpoint{3.067782in}{2.215451in}}{\pgfqpoint{3.059546in}{2.215451in}}%
\pgfpathcurveto{\pgfqpoint{3.051309in}{2.215451in}}{\pgfqpoint{3.043409in}{2.212179in}}{\pgfqpoint{3.037585in}{2.206355in}}%
\pgfpathcurveto{\pgfqpoint{3.031761in}{2.200531in}}{\pgfqpoint{3.028489in}{2.192631in}}{\pgfqpoint{3.028489in}{2.184394in}}%
\pgfpathcurveto{\pgfqpoint{3.028489in}{2.176158in}}{\pgfqpoint{3.031761in}{2.168258in}}{\pgfqpoint{3.037585in}{2.162434in}}%
\pgfpathcurveto{\pgfqpoint{3.043409in}{2.156610in}}{\pgfqpoint{3.051309in}{2.153338in}}{\pgfqpoint{3.059546in}{2.153338in}}%
\pgfpathclose%
\pgfusepath{stroke,fill}%
\end{pgfscope}%
\begin{pgfscope}%
\pgfpathrectangle{\pgfqpoint{0.100000in}{0.212622in}}{\pgfqpoint{3.696000in}{3.696000in}}%
\pgfusepath{clip}%
\pgfsetbuttcap%
\pgfsetroundjoin%
\definecolor{currentfill}{rgb}{0.121569,0.466667,0.705882}%
\pgfsetfillcolor{currentfill}%
\pgfsetfillopacity{0.721067}%
\pgfsetlinewidth{1.003750pt}%
\definecolor{currentstroke}{rgb}{0.121569,0.466667,0.705882}%
\pgfsetstrokecolor{currentstroke}%
\pgfsetstrokeopacity{0.721067}%
\pgfsetdash{}{0pt}%
\pgfpathmoveto{\pgfqpoint{3.045755in}{2.148839in}}%
\pgfpathcurveto{\pgfqpoint{3.053992in}{2.148839in}}{\pgfqpoint{3.061892in}{2.152111in}}{\pgfqpoint{3.067716in}{2.157935in}}%
\pgfpathcurveto{\pgfqpoint{3.073540in}{2.163759in}}{\pgfqpoint{3.076812in}{2.171659in}}{\pgfqpoint{3.076812in}{2.179895in}}%
\pgfpathcurveto{\pgfqpoint{3.076812in}{2.188132in}}{\pgfqpoint{3.073540in}{2.196032in}}{\pgfqpoint{3.067716in}{2.201856in}}%
\pgfpathcurveto{\pgfqpoint{3.061892in}{2.207680in}}{\pgfqpoint{3.053992in}{2.210952in}}{\pgfqpoint{3.045755in}{2.210952in}}%
\pgfpathcurveto{\pgfqpoint{3.037519in}{2.210952in}}{\pgfqpoint{3.029619in}{2.207680in}}{\pgfqpoint{3.023795in}{2.201856in}}%
\pgfpathcurveto{\pgfqpoint{3.017971in}{2.196032in}}{\pgfqpoint{3.014699in}{2.188132in}}{\pgfqpoint{3.014699in}{2.179895in}}%
\pgfpathcurveto{\pgfqpoint{3.014699in}{2.171659in}}{\pgfqpoint{3.017971in}{2.163759in}}{\pgfqpoint{3.023795in}{2.157935in}}%
\pgfpathcurveto{\pgfqpoint{3.029619in}{2.152111in}}{\pgfqpoint{3.037519in}{2.148839in}}{\pgfqpoint{3.045755in}{2.148839in}}%
\pgfpathclose%
\pgfusepath{stroke,fill}%
\end{pgfscope}%
\begin{pgfscope}%
\pgfpathrectangle{\pgfqpoint{0.100000in}{0.212622in}}{\pgfqpoint{3.696000in}{3.696000in}}%
\pgfusepath{clip}%
\pgfsetbuttcap%
\pgfsetroundjoin%
\definecolor{currentfill}{rgb}{0.121569,0.466667,0.705882}%
\pgfsetfillcolor{currentfill}%
\pgfsetfillopacity{0.723909}%
\pgfsetlinewidth{1.003750pt}%
\definecolor{currentstroke}{rgb}{0.121569,0.466667,0.705882}%
\pgfsetstrokecolor{currentstroke}%
\pgfsetstrokeopacity{0.723909}%
\pgfsetdash{}{0pt}%
\pgfpathmoveto{\pgfqpoint{3.039477in}{2.146687in}}%
\pgfpathcurveto{\pgfqpoint{3.047713in}{2.146687in}}{\pgfqpoint{3.055613in}{2.149960in}}{\pgfqpoint{3.061437in}{2.155783in}}%
\pgfpathcurveto{\pgfqpoint{3.067261in}{2.161607in}}{\pgfqpoint{3.070533in}{2.169507in}}{\pgfqpoint{3.070533in}{2.177744in}}%
\pgfpathcurveto{\pgfqpoint{3.070533in}{2.185980in}}{\pgfqpoint{3.067261in}{2.193880in}}{\pgfqpoint{3.061437in}{2.199704in}}%
\pgfpathcurveto{\pgfqpoint{3.055613in}{2.205528in}}{\pgfqpoint{3.047713in}{2.208800in}}{\pgfqpoint{3.039477in}{2.208800in}}%
\pgfpathcurveto{\pgfqpoint{3.031241in}{2.208800in}}{\pgfqpoint{3.023341in}{2.205528in}}{\pgfqpoint{3.017517in}{2.199704in}}%
\pgfpathcurveto{\pgfqpoint{3.011693in}{2.193880in}}{\pgfqpoint{3.008420in}{2.185980in}}{\pgfqpoint{3.008420in}{2.177744in}}%
\pgfpathcurveto{\pgfqpoint{3.008420in}{2.169507in}}{\pgfqpoint{3.011693in}{2.161607in}}{\pgfqpoint{3.017517in}{2.155783in}}%
\pgfpathcurveto{\pgfqpoint{3.023341in}{2.149960in}}{\pgfqpoint{3.031241in}{2.146687in}}{\pgfqpoint{3.039477in}{2.146687in}}%
\pgfpathclose%
\pgfusepath{stroke,fill}%
\end{pgfscope}%
\begin{pgfscope}%
\pgfpathrectangle{\pgfqpoint{0.100000in}{0.212622in}}{\pgfqpoint{3.696000in}{3.696000in}}%
\pgfusepath{clip}%
\pgfsetbuttcap%
\pgfsetroundjoin%
\definecolor{currentfill}{rgb}{0.121569,0.466667,0.705882}%
\pgfsetfillcolor{currentfill}%
\pgfsetfillopacity{0.725382}%
\pgfsetlinewidth{1.003750pt}%
\definecolor{currentstroke}{rgb}{0.121569,0.466667,0.705882}%
\pgfsetstrokecolor{currentstroke}%
\pgfsetstrokeopacity{0.725382}%
\pgfsetdash{}{0pt}%
\pgfpathmoveto{\pgfqpoint{3.035990in}{2.144923in}}%
\pgfpathcurveto{\pgfqpoint{3.044227in}{2.144923in}}{\pgfqpoint{3.052127in}{2.148195in}}{\pgfqpoint{3.057951in}{2.154019in}}%
\pgfpathcurveto{\pgfqpoint{3.063775in}{2.159843in}}{\pgfqpoint{3.067047in}{2.167743in}}{\pgfqpoint{3.067047in}{2.175980in}}%
\pgfpathcurveto{\pgfqpoint{3.067047in}{2.184216in}}{\pgfqpoint{3.063775in}{2.192116in}}{\pgfqpoint{3.057951in}{2.197940in}}%
\pgfpathcurveto{\pgfqpoint{3.052127in}{2.203764in}}{\pgfqpoint{3.044227in}{2.207036in}}{\pgfqpoint{3.035990in}{2.207036in}}%
\pgfpathcurveto{\pgfqpoint{3.027754in}{2.207036in}}{\pgfqpoint{3.019854in}{2.203764in}}{\pgfqpoint{3.014030in}{2.197940in}}%
\pgfpathcurveto{\pgfqpoint{3.008206in}{2.192116in}}{\pgfqpoint{3.004934in}{2.184216in}}{\pgfqpoint{3.004934in}{2.175980in}}%
\pgfpathcurveto{\pgfqpoint{3.004934in}{2.167743in}}{\pgfqpoint{3.008206in}{2.159843in}}{\pgfqpoint{3.014030in}{2.154019in}}%
\pgfpathcurveto{\pgfqpoint{3.019854in}{2.148195in}}{\pgfqpoint{3.027754in}{2.144923in}}{\pgfqpoint{3.035990in}{2.144923in}}%
\pgfpathclose%
\pgfusepath{stroke,fill}%
\end{pgfscope}%
\begin{pgfscope}%
\pgfpathrectangle{\pgfqpoint{0.100000in}{0.212622in}}{\pgfqpoint{3.696000in}{3.696000in}}%
\pgfusepath{clip}%
\pgfsetbuttcap%
\pgfsetroundjoin%
\definecolor{currentfill}{rgb}{0.121569,0.466667,0.705882}%
\pgfsetfillcolor{currentfill}%
\pgfsetfillopacity{0.726344}%
\pgfsetlinewidth{1.003750pt}%
\definecolor{currentstroke}{rgb}{0.121569,0.466667,0.705882}%
\pgfsetstrokecolor{currentstroke}%
\pgfsetstrokeopacity{0.726344}%
\pgfsetdash{}{0pt}%
\pgfpathmoveto{\pgfqpoint{3.034087in}{2.144950in}}%
\pgfpathcurveto{\pgfqpoint{3.042323in}{2.144950in}}{\pgfqpoint{3.050223in}{2.148222in}}{\pgfqpoint{3.056047in}{2.154046in}}%
\pgfpathcurveto{\pgfqpoint{3.061871in}{2.159870in}}{\pgfqpoint{3.065143in}{2.167770in}}{\pgfqpoint{3.065143in}{2.176006in}}%
\pgfpathcurveto{\pgfqpoint{3.065143in}{2.184243in}}{\pgfqpoint{3.061871in}{2.192143in}}{\pgfqpoint{3.056047in}{2.197967in}}%
\pgfpathcurveto{\pgfqpoint{3.050223in}{2.203791in}}{\pgfqpoint{3.042323in}{2.207063in}}{\pgfqpoint{3.034087in}{2.207063in}}%
\pgfpathcurveto{\pgfqpoint{3.025851in}{2.207063in}}{\pgfqpoint{3.017951in}{2.203791in}}{\pgfqpoint{3.012127in}{2.197967in}}%
\pgfpathcurveto{\pgfqpoint{3.006303in}{2.192143in}}{\pgfqpoint{3.003030in}{2.184243in}}{\pgfqpoint{3.003030in}{2.176006in}}%
\pgfpathcurveto{\pgfqpoint{3.003030in}{2.167770in}}{\pgfqpoint{3.006303in}{2.159870in}}{\pgfqpoint{3.012127in}{2.154046in}}%
\pgfpathcurveto{\pgfqpoint{3.017951in}{2.148222in}}{\pgfqpoint{3.025851in}{2.144950in}}{\pgfqpoint{3.034087in}{2.144950in}}%
\pgfpathclose%
\pgfusepath{stroke,fill}%
\end{pgfscope}%
\begin{pgfscope}%
\pgfpathrectangle{\pgfqpoint{0.100000in}{0.212622in}}{\pgfqpoint{3.696000in}{3.696000in}}%
\pgfusepath{clip}%
\pgfsetbuttcap%
\pgfsetroundjoin%
\definecolor{currentfill}{rgb}{0.121569,0.466667,0.705882}%
\pgfsetfillcolor{currentfill}%
\pgfsetfillopacity{0.726822}%
\pgfsetlinewidth{1.003750pt}%
\definecolor{currentstroke}{rgb}{0.121569,0.466667,0.705882}%
\pgfsetstrokecolor{currentstroke}%
\pgfsetstrokeopacity{0.726822}%
\pgfsetdash{}{0pt}%
\pgfpathmoveto{\pgfqpoint{3.033106in}{2.144528in}}%
\pgfpathcurveto{\pgfqpoint{3.041343in}{2.144528in}}{\pgfqpoint{3.049243in}{2.147800in}}{\pgfqpoint{3.055067in}{2.153624in}}%
\pgfpathcurveto{\pgfqpoint{3.060891in}{2.159448in}}{\pgfqpoint{3.064163in}{2.167348in}}{\pgfqpoint{3.064163in}{2.175585in}}%
\pgfpathcurveto{\pgfqpoint{3.064163in}{2.183821in}}{\pgfqpoint{3.060891in}{2.191721in}}{\pgfqpoint{3.055067in}{2.197545in}}%
\pgfpathcurveto{\pgfqpoint{3.049243in}{2.203369in}}{\pgfqpoint{3.041343in}{2.206641in}}{\pgfqpoint{3.033106in}{2.206641in}}%
\pgfpathcurveto{\pgfqpoint{3.024870in}{2.206641in}}{\pgfqpoint{3.016970in}{2.203369in}}{\pgfqpoint{3.011146in}{2.197545in}}%
\pgfpathcurveto{\pgfqpoint{3.005322in}{2.191721in}}{\pgfqpoint{3.002050in}{2.183821in}}{\pgfqpoint{3.002050in}{2.175585in}}%
\pgfpathcurveto{\pgfqpoint{3.002050in}{2.167348in}}{\pgfqpoint{3.005322in}{2.159448in}}{\pgfqpoint{3.011146in}{2.153624in}}%
\pgfpathcurveto{\pgfqpoint{3.016970in}{2.147800in}}{\pgfqpoint{3.024870in}{2.144528in}}{\pgfqpoint{3.033106in}{2.144528in}}%
\pgfpathclose%
\pgfusepath{stroke,fill}%
\end{pgfscope}%
\begin{pgfscope}%
\pgfpathrectangle{\pgfqpoint{0.100000in}{0.212622in}}{\pgfqpoint{3.696000in}{3.696000in}}%
\pgfusepath{clip}%
\pgfsetbuttcap%
\pgfsetroundjoin%
\definecolor{currentfill}{rgb}{0.121569,0.466667,0.705882}%
\pgfsetfillcolor{currentfill}%
\pgfsetfillopacity{0.727082}%
\pgfsetlinewidth{1.003750pt}%
\definecolor{currentstroke}{rgb}{0.121569,0.466667,0.705882}%
\pgfsetstrokecolor{currentstroke}%
\pgfsetstrokeopacity{0.727082}%
\pgfsetdash{}{0pt}%
\pgfpathmoveto{\pgfqpoint{3.032435in}{2.144480in}}%
\pgfpathcurveto{\pgfqpoint{3.040671in}{2.144480in}}{\pgfqpoint{3.048571in}{2.147752in}}{\pgfqpoint{3.054395in}{2.153576in}}%
\pgfpathcurveto{\pgfqpoint{3.060219in}{2.159400in}}{\pgfqpoint{3.063492in}{2.167300in}}{\pgfqpoint{3.063492in}{2.175536in}}%
\pgfpathcurveto{\pgfqpoint{3.063492in}{2.183773in}}{\pgfqpoint{3.060219in}{2.191673in}}{\pgfqpoint{3.054395in}{2.197497in}}%
\pgfpathcurveto{\pgfqpoint{3.048571in}{2.203320in}}{\pgfqpoint{3.040671in}{2.206593in}}{\pgfqpoint{3.032435in}{2.206593in}}%
\pgfpathcurveto{\pgfqpoint{3.024199in}{2.206593in}}{\pgfqpoint{3.016299in}{2.203320in}}{\pgfqpoint{3.010475in}{2.197497in}}%
\pgfpathcurveto{\pgfqpoint{3.004651in}{2.191673in}}{\pgfqpoint{3.001379in}{2.183773in}}{\pgfqpoint{3.001379in}{2.175536in}}%
\pgfpathcurveto{\pgfqpoint{3.001379in}{2.167300in}}{\pgfqpoint{3.004651in}{2.159400in}}{\pgfqpoint{3.010475in}{2.153576in}}%
\pgfpathcurveto{\pgfqpoint{3.016299in}{2.147752in}}{\pgfqpoint{3.024199in}{2.144480in}}{\pgfqpoint{3.032435in}{2.144480in}}%
\pgfpathclose%
\pgfusepath{stroke,fill}%
\end{pgfscope}%
\begin{pgfscope}%
\pgfpathrectangle{\pgfqpoint{0.100000in}{0.212622in}}{\pgfqpoint{3.696000in}{3.696000in}}%
\pgfusepath{clip}%
\pgfsetbuttcap%
\pgfsetroundjoin%
\definecolor{currentfill}{rgb}{0.121569,0.466667,0.705882}%
\pgfsetfillcolor{currentfill}%
\pgfsetfillopacity{0.728112}%
\pgfsetlinewidth{1.003750pt}%
\definecolor{currentstroke}{rgb}{0.121569,0.466667,0.705882}%
\pgfsetstrokecolor{currentstroke}%
\pgfsetstrokeopacity{0.728112}%
\pgfsetdash{}{0pt}%
\pgfpathmoveto{\pgfqpoint{3.030421in}{2.142307in}}%
\pgfpathcurveto{\pgfqpoint{3.038657in}{2.142307in}}{\pgfqpoint{3.046557in}{2.145580in}}{\pgfqpoint{3.052381in}{2.151404in}}%
\pgfpathcurveto{\pgfqpoint{3.058205in}{2.157227in}}{\pgfqpoint{3.061478in}{2.165128in}}{\pgfqpoint{3.061478in}{2.173364in}}%
\pgfpathcurveto{\pgfqpoint{3.061478in}{2.181600in}}{\pgfqpoint{3.058205in}{2.189500in}}{\pgfqpoint{3.052381in}{2.195324in}}%
\pgfpathcurveto{\pgfqpoint{3.046557in}{2.201148in}}{\pgfqpoint{3.038657in}{2.204420in}}{\pgfqpoint{3.030421in}{2.204420in}}%
\pgfpathcurveto{\pgfqpoint{3.022185in}{2.204420in}}{\pgfqpoint{3.014285in}{2.201148in}}{\pgfqpoint{3.008461in}{2.195324in}}%
\pgfpathcurveto{\pgfqpoint{3.002637in}{2.189500in}}{\pgfqpoint{2.999365in}{2.181600in}}{\pgfqpoint{2.999365in}{2.173364in}}%
\pgfpathcurveto{\pgfqpoint{2.999365in}{2.165128in}}{\pgfqpoint{3.002637in}{2.157227in}}{\pgfqpoint{3.008461in}{2.151404in}}%
\pgfpathcurveto{\pgfqpoint{3.014285in}{2.145580in}}{\pgfqpoint{3.022185in}{2.142307in}}{\pgfqpoint{3.030421in}{2.142307in}}%
\pgfpathclose%
\pgfusepath{stroke,fill}%
\end{pgfscope}%
\begin{pgfscope}%
\pgfpathrectangle{\pgfqpoint{0.100000in}{0.212622in}}{\pgfqpoint{3.696000in}{3.696000in}}%
\pgfusepath{clip}%
\pgfsetbuttcap%
\pgfsetroundjoin%
\definecolor{currentfill}{rgb}{0.121569,0.466667,0.705882}%
\pgfsetfillcolor{currentfill}%
\pgfsetfillopacity{0.728729}%
\pgfsetlinewidth{1.003750pt}%
\definecolor{currentstroke}{rgb}{0.121569,0.466667,0.705882}%
\pgfsetstrokecolor{currentstroke}%
\pgfsetstrokeopacity{0.728729}%
\pgfsetdash{}{0pt}%
\pgfpathmoveto{\pgfqpoint{3.028801in}{2.142114in}}%
\pgfpathcurveto{\pgfqpoint{3.037037in}{2.142114in}}{\pgfqpoint{3.044937in}{2.145387in}}{\pgfqpoint{3.050761in}{2.151211in}}%
\pgfpathcurveto{\pgfqpoint{3.056585in}{2.157035in}}{\pgfqpoint{3.059857in}{2.164935in}}{\pgfqpoint{3.059857in}{2.173171in}}%
\pgfpathcurveto{\pgfqpoint{3.059857in}{2.181407in}}{\pgfqpoint{3.056585in}{2.189307in}}{\pgfqpoint{3.050761in}{2.195131in}}%
\pgfpathcurveto{\pgfqpoint{3.044937in}{2.200955in}}{\pgfqpoint{3.037037in}{2.204227in}}{\pgfqpoint{3.028801in}{2.204227in}}%
\pgfpathcurveto{\pgfqpoint{3.020565in}{2.204227in}}{\pgfqpoint{3.012665in}{2.200955in}}{\pgfqpoint{3.006841in}{2.195131in}}%
\pgfpathcurveto{\pgfqpoint{3.001017in}{2.189307in}}{\pgfqpoint{2.997744in}{2.181407in}}{\pgfqpoint{2.997744in}{2.173171in}}%
\pgfpathcurveto{\pgfqpoint{2.997744in}{2.164935in}}{\pgfqpoint{3.001017in}{2.157035in}}{\pgfqpoint{3.006841in}{2.151211in}}%
\pgfpathcurveto{\pgfqpoint{3.012665in}{2.145387in}}{\pgfqpoint{3.020565in}{2.142114in}}{\pgfqpoint{3.028801in}{2.142114in}}%
\pgfpathclose%
\pgfusepath{stroke,fill}%
\end{pgfscope}%
\begin{pgfscope}%
\pgfpathrectangle{\pgfqpoint{0.100000in}{0.212622in}}{\pgfqpoint{3.696000in}{3.696000in}}%
\pgfusepath{clip}%
\pgfsetbuttcap%
\pgfsetroundjoin%
\definecolor{currentfill}{rgb}{0.121569,0.466667,0.705882}%
\pgfsetfillcolor{currentfill}%
\pgfsetfillopacity{0.730366}%
\pgfsetlinewidth{1.003750pt}%
\definecolor{currentstroke}{rgb}{0.121569,0.466667,0.705882}%
\pgfsetstrokecolor{currentstroke}%
\pgfsetstrokeopacity{0.730366}%
\pgfsetdash{}{0pt}%
\pgfpathmoveto{\pgfqpoint{3.025859in}{2.138219in}}%
\pgfpathcurveto{\pgfqpoint{3.034095in}{2.138219in}}{\pgfqpoint{3.041995in}{2.141492in}}{\pgfqpoint{3.047819in}{2.147316in}}%
\pgfpathcurveto{\pgfqpoint{3.053643in}{2.153139in}}{\pgfqpoint{3.056915in}{2.161039in}}{\pgfqpoint{3.056915in}{2.169276in}}%
\pgfpathcurveto{\pgfqpoint{3.056915in}{2.177512in}}{\pgfqpoint{3.053643in}{2.185412in}}{\pgfqpoint{3.047819in}{2.191236in}}%
\pgfpathcurveto{\pgfqpoint{3.041995in}{2.197060in}}{\pgfqpoint{3.034095in}{2.200332in}}{\pgfqpoint{3.025859in}{2.200332in}}%
\pgfpathcurveto{\pgfqpoint{3.017623in}{2.200332in}}{\pgfqpoint{3.009722in}{2.197060in}}{\pgfqpoint{3.003899in}{2.191236in}}%
\pgfpathcurveto{\pgfqpoint{2.998075in}{2.185412in}}{\pgfqpoint{2.994802in}{2.177512in}}{\pgfqpoint{2.994802in}{2.169276in}}%
\pgfpathcurveto{\pgfqpoint{2.994802in}{2.161039in}}{\pgfqpoint{2.998075in}{2.153139in}}{\pgfqpoint{3.003899in}{2.147316in}}%
\pgfpathcurveto{\pgfqpoint{3.009722in}{2.141492in}}{\pgfqpoint{3.017623in}{2.138219in}}{\pgfqpoint{3.025859in}{2.138219in}}%
\pgfpathclose%
\pgfusepath{stroke,fill}%
\end{pgfscope}%
\begin{pgfscope}%
\pgfpathrectangle{\pgfqpoint{0.100000in}{0.212622in}}{\pgfqpoint{3.696000in}{3.696000in}}%
\pgfusepath{clip}%
\pgfsetbuttcap%
\pgfsetroundjoin%
\definecolor{currentfill}{rgb}{0.121569,0.466667,0.705882}%
\pgfsetfillcolor{currentfill}%
\pgfsetfillopacity{0.732444}%
\pgfsetlinewidth{1.003750pt}%
\definecolor{currentstroke}{rgb}{0.121569,0.466667,0.705882}%
\pgfsetstrokecolor{currentstroke}%
\pgfsetstrokeopacity{0.732444}%
\pgfsetdash{}{0pt}%
\pgfpathmoveto{\pgfqpoint{3.018903in}{2.136116in}}%
\pgfpathcurveto{\pgfqpoint{3.027139in}{2.136116in}}{\pgfqpoint{3.035040in}{2.139389in}}{\pgfqpoint{3.040863in}{2.145213in}}%
\pgfpathcurveto{\pgfqpoint{3.046687in}{2.151036in}}{\pgfqpoint{3.049960in}{2.158937in}}{\pgfqpoint{3.049960in}{2.167173in}}%
\pgfpathcurveto{\pgfqpoint{3.049960in}{2.175409in}}{\pgfqpoint{3.046687in}{2.183309in}}{\pgfqpoint{3.040863in}{2.189133in}}%
\pgfpathcurveto{\pgfqpoint{3.035040in}{2.194957in}}{\pgfqpoint{3.027139in}{2.198229in}}{\pgfqpoint{3.018903in}{2.198229in}}%
\pgfpathcurveto{\pgfqpoint{3.010667in}{2.198229in}}{\pgfqpoint{3.002767in}{2.194957in}}{\pgfqpoint{2.996943in}{2.189133in}}%
\pgfpathcurveto{\pgfqpoint{2.991119in}{2.183309in}}{\pgfqpoint{2.987847in}{2.175409in}}{\pgfqpoint{2.987847in}{2.167173in}}%
\pgfpathcurveto{\pgfqpoint{2.987847in}{2.158937in}}{\pgfqpoint{2.991119in}{2.151036in}}{\pgfqpoint{2.996943in}{2.145213in}}%
\pgfpathcurveto{\pgfqpoint{3.002767in}{2.139389in}}{\pgfqpoint{3.010667in}{2.136116in}}{\pgfqpoint{3.018903in}{2.136116in}}%
\pgfpathclose%
\pgfusepath{stroke,fill}%
\end{pgfscope}%
\begin{pgfscope}%
\pgfpathrectangle{\pgfqpoint{0.100000in}{0.212622in}}{\pgfqpoint{3.696000in}{3.696000in}}%
\pgfusepath{clip}%
\pgfsetbuttcap%
\pgfsetroundjoin%
\definecolor{currentfill}{rgb}{0.121569,0.466667,0.705882}%
\pgfsetfillcolor{currentfill}%
\pgfsetfillopacity{0.735669}%
\pgfsetlinewidth{1.003750pt}%
\definecolor{currentstroke}{rgb}{0.121569,0.466667,0.705882}%
\pgfsetstrokecolor{currentstroke}%
\pgfsetstrokeopacity{0.735669}%
\pgfsetdash{}{0pt}%
\pgfpathmoveto{\pgfqpoint{3.012668in}{2.129516in}}%
\pgfpathcurveto{\pgfqpoint{3.020904in}{2.129516in}}{\pgfqpoint{3.028804in}{2.132788in}}{\pgfqpoint{3.034628in}{2.138612in}}%
\pgfpathcurveto{\pgfqpoint{3.040452in}{2.144436in}}{\pgfqpoint{3.043724in}{2.152336in}}{\pgfqpoint{3.043724in}{2.160572in}}%
\pgfpathcurveto{\pgfqpoint{3.043724in}{2.168809in}}{\pgfqpoint{3.040452in}{2.176709in}}{\pgfqpoint{3.034628in}{2.182533in}}%
\pgfpathcurveto{\pgfqpoint{3.028804in}{2.188357in}}{\pgfqpoint{3.020904in}{2.191629in}}{\pgfqpoint{3.012668in}{2.191629in}}%
\pgfpathcurveto{\pgfqpoint{3.004431in}{2.191629in}}{\pgfqpoint{2.996531in}{2.188357in}}{\pgfqpoint{2.990707in}{2.182533in}}%
\pgfpathcurveto{\pgfqpoint{2.984883in}{2.176709in}}{\pgfqpoint{2.981611in}{2.168809in}}{\pgfqpoint{2.981611in}{2.160572in}}%
\pgfpathcurveto{\pgfqpoint{2.981611in}{2.152336in}}{\pgfqpoint{2.984883in}{2.144436in}}{\pgfqpoint{2.990707in}{2.138612in}}%
\pgfpathcurveto{\pgfqpoint{2.996531in}{2.132788in}}{\pgfqpoint{3.004431in}{2.129516in}}{\pgfqpoint{3.012668in}{2.129516in}}%
\pgfpathclose%
\pgfusepath{stroke,fill}%
\end{pgfscope}%
\begin{pgfscope}%
\pgfpathrectangle{\pgfqpoint{0.100000in}{0.212622in}}{\pgfqpoint{3.696000in}{3.696000in}}%
\pgfusepath{clip}%
\pgfsetbuttcap%
\pgfsetroundjoin%
\definecolor{currentfill}{rgb}{0.121569,0.466667,0.705882}%
\pgfsetfillcolor{currentfill}%
\pgfsetfillopacity{0.739656}%
\pgfsetlinewidth{1.003750pt}%
\definecolor{currentstroke}{rgb}{0.121569,0.466667,0.705882}%
\pgfsetstrokecolor{currentstroke}%
\pgfsetstrokeopacity{0.739656}%
\pgfsetdash{}{0pt}%
\pgfpathmoveto{\pgfqpoint{2.999911in}{2.125222in}}%
\pgfpathcurveto{\pgfqpoint{3.008147in}{2.125222in}}{\pgfqpoint{3.016047in}{2.128494in}}{\pgfqpoint{3.021871in}{2.134318in}}%
\pgfpathcurveto{\pgfqpoint{3.027695in}{2.140142in}}{\pgfqpoint{3.030967in}{2.148042in}}{\pgfqpoint{3.030967in}{2.156279in}}%
\pgfpathcurveto{\pgfqpoint{3.030967in}{2.164515in}}{\pgfqpoint{3.027695in}{2.172415in}}{\pgfqpoint{3.021871in}{2.178239in}}%
\pgfpathcurveto{\pgfqpoint{3.016047in}{2.184063in}}{\pgfqpoint{3.008147in}{2.187335in}}{\pgfqpoint{2.999911in}{2.187335in}}%
\pgfpathcurveto{\pgfqpoint{2.991675in}{2.187335in}}{\pgfqpoint{2.983775in}{2.184063in}}{\pgfqpoint{2.977951in}{2.178239in}}%
\pgfpathcurveto{\pgfqpoint{2.972127in}{2.172415in}}{\pgfqpoint{2.968854in}{2.164515in}}{\pgfqpoint{2.968854in}{2.156279in}}%
\pgfpathcurveto{\pgfqpoint{2.968854in}{2.148042in}}{\pgfqpoint{2.972127in}{2.140142in}}{\pgfqpoint{2.977951in}{2.134318in}}%
\pgfpathcurveto{\pgfqpoint{2.983775in}{2.128494in}}{\pgfqpoint{2.991675in}{2.125222in}}{\pgfqpoint{2.999911in}{2.125222in}}%
\pgfpathclose%
\pgfusepath{stroke,fill}%
\end{pgfscope}%
\begin{pgfscope}%
\pgfpathrectangle{\pgfqpoint{0.100000in}{0.212622in}}{\pgfqpoint{3.696000in}{3.696000in}}%
\pgfusepath{clip}%
\pgfsetbuttcap%
\pgfsetroundjoin%
\definecolor{currentfill}{rgb}{0.121569,0.466667,0.705882}%
\pgfsetfillcolor{currentfill}%
\pgfsetfillopacity{0.745387}%
\pgfsetlinewidth{1.003750pt}%
\definecolor{currentstroke}{rgb}{0.121569,0.466667,0.705882}%
\pgfsetstrokecolor{currentstroke}%
\pgfsetstrokeopacity{0.745387}%
\pgfsetdash{}{0pt}%
\pgfpathmoveto{\pgfqpoint{2.990124in}{2.118903in}}%
\pgfpathcurveto{\pgfqpoint{2.998360in}{2.118903in}}{\pgfqpoint{3.006260in}{2.122176in}}{\pgfqpoint{3.012084in}{2.128000in}}%
\pgfpathcurveto{\pgfqpoint{3.017908in}{2.133823in}}{\pgfqpoint{3.021180in}{2.141724in}}{\pgfqpoint{3.021180in}{2.149960in}}%
\pgfpathcurveto{\pgfqpoint{3.021180in}{2.158196in}}{\pgfqpoint{3.017908in}{2.166096in}}{\pgfqpoint{3.012084in}{2.171920in}}%
\pgfpathcurveto{\pgfqpoint{3.006260in}{2.177744in}}{\pgfqpoint{2.998360in}{2.181016in}}{\pgfqpoint{2.990124in}{2.181016in}}%
\pgfpathcurveto{\pgfqpoint{2.981888in}{2.181016in}}{\pgfqpoint{2.973988in}{2.177744in}}{\pgfqpoint{2.968164in}{2.171920in}}%
\pgfpathcurveto{\pgfqpoint{2.962340in}{2.166096in}}{\pgfqpoint{2.959067in}{2.158196in}}{\pgfqpoint{2.959067in}{2.149960in}}%
\pgfpathcurveto{\pgfqpoint{2.959067in}{2.141724in}}{\pgfqpoint{2.962340in}{2.133823in}}{\pgfqpoint{2.968164in}{2.128000in}}%
\pgfpathcurveto{\pgfqpoint{2.973988in}{2.122176in}}{\pgfqpoint{2.981888in}{2.118903in}}{\pgfqpoint{2.990124in}{2.118903in}}%
\pgfpathclose%
\pgfusepath{stroke,fill}%
\end{pgfscope}%
\begin{pgfscope}%
\pgfpathrectangle{\pgfqpoint{0.100000in}{0.212622in}}{\pgfqpoint{3.696000in}{3.696000in}}%
\pgfusepath{clip}%
\pgfsetbuttcap%
\pgfsetroundjoin%
\definecolor{currentfill}{rgb}{0.121569,0.466667,0.705882}%
\pgfsetfillcolor{currentfill}%
\pgfsetfillopacity{0.748130}%
\pgfsetlinewidth{1.003750pt}%
\definecolor{currentstroke}{rgb}{0.121569,0.466667,0.705882}%
\pgfsetstrokecolor{currentstroke}%
\pgfsetstrokeopacity{0.748130}%
\pgfsetdash{}{0pt}%
\pgfpathmoveto{\pgfqpoint{2.982595in}{2.115322in}}%
\pgfpathcurveto{\pgfqpoint{2.990831in}{2.115322in}}{\pgfqpoint{2.998732in}{2.118595in}}{\pgfqpoint{3.004555in}{2.124419in}}%
\pgfpathcurveto{\pgfqpoint{3.010379in}{2.130243in}}{\pgfqpoint{3.013652in}{2.138143in}}{\pgfqpoint{3.013652in}{2.146379in}}%
\pgfpathcurveto{\pgfqpoint{3.013652in}{2.154615in}}{\pgfqpoint{3.010379in}{2.162515in}}{\pgfqpoint{3.004555in}{2.168339in}}%
\pgfpathcurveto{\pgfqpoint{2.998732in}{2.174163in}}{\pgfqpoint{2.990831in}{2.177435in}}{\pgfqpoint{2.982595in}{2.177435in}}%
\pgfpathcurveto{\pgfqpoint{2.974359in}{2.177435in}}{\pgfqpoint{2.966459in}{2.174163in}}{\pgfqpoint{2.960635in}{2.168339in}}%
\pgfpathcurveto{\pgfqpoint{2.954811in}{2.162515in}}{\pgfqpoint{2.951539in}{2.154615in}}{\pgfqpoint{2.951539in}{2.146379in}}%
\pgfpathcurveto{\pgfqpoint{2.951539in}{2.138143in}}{\pgfqpoint{2.954811in}{2.130243in}}{\pgfqpoint{2.960635in}{2.124419in}}%
\pgfpathcurveto{\pgfqpoint{2.966459in}{2.118595in}}{\pgfqpoint{2.974359in}{2.115322in}}{\pgfqpoint{2.982595in}{2.115322in}}%
\pgfpathclose%
\pgfusepath{stroke,fill}%
\end{pgfscope}%
\begin{pgfscope}%
\pgfpathrectangle{\pgfqpoint{0.100000in}{0.212622in}}{\pgfqpoint{3.696000in}{3.696000in}}%
\pgfusepath{clip}%
\pgfsetbuttcap%
\pgfsetroundjoin%
\definecolor{currentfill}{rgb}{0.121569,0.466667,0.705882}%
\pgfsetfillcolor{currentfill}%
\pgfsetfillopacity{0.749897}%
\pgfsetlinewidth{1.003750pt}%
\definecolor{currentstroke}{rgb}{0.121569,0.466667,0.705882}%
\pgfsetstrokecolor{currentstroke}%
\pgfsetstrokeopacity{0.749897}%
\pgfsetdash{}{0pt}%
\pgfpathmoveto{\pgfqpoint{2.979083in}{2.114250in}}%
\pgfpathcurveto{\pgfqpoint{2.987320in}{2.114250in}}{\pgfqpoint{2.995220in}{2.117522in}}{\pgfqpoint{3.001044in}{2.123346in}}%
\pgfpathcurveto{\pgfqpoint{3.006868in}{2.129170in}}{\pgfqpoint{3.010140in}{2.137070in}}{\pgfqpoint{3.010140in}{2.145306in}}%
\pgfpathcurveto{\pgfqpoint{3.010140in}{2.153542in}}{\pgfqpoint{3.006868in}{2.161442in}}{\pgfqpoint{3.001044in}{2.167266in}}%
\pgfpathcurveto{\pgfqpoint{2.995220in}{2.173090in}}{\pgfqpoint{2.987320in}{2.176363in}}{\pgfqpoint{2.979083in}{2.176363in}}%
\pgfpathcurveto{\pgfqpoint{2.970847in}{2.176363in}}{\pgfqpoint{2.962947in}{2.173090in}}{\pgfqpoint{2.957123in}{2.167266in}}%
\pgfpathcurveto{\pgfqpoint{2.951299in}{2.161442in}}{\pgfqpoint{2.948027in}{2.153542in}}{\pgfqpoint{2.948027in}{2.145306in}}%
\pgfpathcurveto{\pgfqpoint{2.948027in}{2.137070in}}{\pgfqpoint{2.951299in}{2.129170in}}{\pgfqpoint{2.957123in}{2.123346in}}%
\pgfpathcurveto{\pgfqpoint{2.962947in}{2.117522in}}{\pgfqpoint{2.970847in}{2.114250in}}{\pgfqpoint{2.979083in}{2.114250in}}%
\pgfpathclose%
\pgfusepath{stroke,fill}%
\end{pgfscope}%
\begin{pgfscope}%
\pgfpathrectangle{\pgfqpoint{0.100000in}{0.212622in}}{\pgfqpoint{3.696000in}{3.696000in}}%
\pgfusepath{clip}%
\pgfsetbuttcap%
\pgfsetroundjoin%
\definecolor{currentfill}{rgb}{0.121569,0.466667,0.705882}%
\pgfsetfillcolor{currentfill}%
\pgfsetfillopacity{0.750874}%
\pgfsetlinewidth{1.003750pt}%
\definecolor{currentstroke}{rgb}{0.121569,0.466667,0.705882}%
\pgfsetstrokecolor{currentstroke}%
\pgfsetstrokeopacity{0.750874}%
\pgfsetdash{}{0pt}%
\pgfpathmoveto{\pgfqpoint{2.977162in}{2.113645in}}%
\pgfpathcurveto{\pgfqpoint{2.985399in}{2.113645in}}{\pgfqpoint{2.993299in}{2.116918in}}{\pgfqpoint{2.999123in}{2.122741in}}%
\pgfpathcurveto{\pgfqpoint{3.004947in}{2.128565in}}{\pgfqpoint{3.008219in}{2.136465in}}{\pgfqpoint{3.008219in}{2.144702in}}%
\pgfpathcurveto{\pgfqpoint{3.008219in}{2.152938in}}{\pgfqpoint{3.004947in}{2.160838in}}{\pgfqpoint{2.999123in}{2.166662in}}%
\pgfpathcurveto{\pgfqpoint{2.993299in}{2.172486in}}{\pgfqpoint{2.985399in}{2.175758in}}{\pgfqpoint{2.977162in}{2.175758in}}%
\pgfpathcurveto{\pgfqpoint{2.968926in}{2.175758in}}{\pgfqpoint{2.961026in}{2.172486in}}{\pgfqpoint{2.955202in}{2.166662in}}%
\pgfpathcurveto{\pgfqpoint{2.949378in}{2.160838in}}{\pgfqpoint{2.946106in}{2.152938in}}{\pgfqpoint{2.946106in}{2.144702in}}%
\pgfpathcurveto{\pgfqpoint{2.946106in}{2.136465in}}{\pgfqpoint{2.949378in}{2.128565in}}{\pgfqpoint{2.955202in}{2.122741in}}%
\pgfpathcurveto{\pgfqpoint{2.961026in}{2.116918in}}{\pgfqpoint{2.968926in}{2.113645in}}{\pgfqpoint{2.977162in}{2.113645in}}%
\pgfpathclose%
\pgfusepath{stroke,fill}%
\end{pgfscope}%
\begin{pgfscope}%
\pgfpathrectangle{\pgfqpoint{0.100000in}{0.212622in}}{\pgfqpoint{3.696000in}{3.696000in}}%
\pgfusepath{clip}%
\pgfsetbuttcap%
\pgfsetroundjoin%
\definecolor{currentfill}{rgb}{0.121569,0.466667,0.705882}%
\pgfsetfillcolor{currentfill}%
\pgfsetfillopacity{0.751399}%
\pgfsetlinewidth{1.003750pt}%
\definecolor{currentstroke}{rgb}{0.121569,0.466667,0.705882}%
\pgfsetstrokecolor{currentstroke}%
\pgfsetstrokeopacity{0.751399}%
\pgfsetdash{}{0pt}%
\pgfpathmoveto{\pgfqpoint{2.975953in}{2.113436in}}%
\pgfpathcurveto{\pgfqpoint{2.984189in}{2.113436in}}{\pgfqpoint{2.992089in}{2.116708in}}{\pgfqpoint{2.997913in}{2.122532in}}%
\pgfpathcurveto{\pgfqpoint{3.003737in}{2.128356in}}{\pgfqpoint{3.007009in}{2.136256in}}{\pgfqpoint{3.007009in}{2.144492in}}%
\pgfpathcurveto{\pgfqpoint{3.007009in}{2.152729in}}{\pgfqpoint{3.003737in}{2.160629in}}{\pgfqpoint{2.997913in}{2.166453in}}%
\pgfpathcurveto{\pgfqpoint{2.992089in}{2.172276in}}{\pgfqpoint{2.984189in}{2.175549in}}{\pgfqpoint{2.975953in}{2.175549in}}%
\pgfpathcurveto{\pgfqpoint{2.967717in}{2.175549in}}{\pgfqpoint{2.959817in}{2.172276in}}{\pgfqpoint{2.953993in}{2.166453in}}%
\pgfpathcurveto{\pgfqpoint{2.948169in}{2.160629in}}{\pgfqpoint{2.944896in}{2.152729in}}{\pgfqpoint{2.944896in}{2.144492in}}%
\pgfpathcurveto{\pgfqpoint{2.944896in}{2.136256in}}{\pgfqpoint{2.948169in}{2.128356in}}{\pgfqpoint{2.953993in}{2.122532in}}%
\pgfpathcurveto{\pgfqpoint{2.959817in}{2.116708in}}{\pgfqpoint{2.967717in}{2.113436in}}{\pgfqpoint{2.975953in}{2.113436in}}%
\pgfpathclose%
\pgfusepath{stroke,fill}%
\end{pgfscope}%
\begin{pgfscope}%
\pgfpathrectangle{\pgfqpoint{0.100000in}{0.212622in}}{\pgfqpoint{3.696000in}{3.696000in}}%
\pgfusepath{clip}%
\pgfsetbuttcap%
\pgfsetroundjoin%
\definecolor{currentfill}{rgb}{0.121569,0.466667,0.705882}%
\pgfsetfillcolor{currentfill}%
\pgfsetfillopacity{0.752548}%
\pgfsetlinewidth{1.003750pt}%
\definecolor{currentstroke}{rgb}{0.121569,0.466667,0.705882}%
\pgfsetstrokecolor{currentstroke}%
\pgfsetstrokeopacity{0.752548}%
\pgfsetdash{}{0pt}%
\pgfpathmoveto{\pgfqpoint{2.973921in}{2.111480in}}%
\pgfpathcurveto{\pgfqpoint{2.982157in}{2.111480in}}{\pgfqpoint{2.990057in}{2.114753in}}{\pgfqpoint{2.995881in}{2.120577in}}%
\pgfpathcurveto{\pgfqpoint{3.001705in}{2.126401in}}{\pgfqpoint{3.004977in}{2.134301in}}{\pgfqpoint{3.004977in}{2.142537in}}%
\pgfpathcurveto{\pgfqpoint{3.004977in}{2.150773in}}{\pgfqpoint{3.001705in}{2.158673in}}{\pgfqpoint{2.995881in}{2.164497in}}%
\pgfpathcurveto{\pgfqpoint{2.990057in}{2.170321in}}{\pgfqpoint{2.982157in}{2.173593in}}{\pgfqpoint{2.973921in}{2.173593in}}%
\pgfpathcurveto{\pgfqpoint{2.965685in}{2.173593in}}{\pgfqpoint{2.957785in}{2.170321in}}{\pgfqpoint{2.951961in}{2.164497in}}%
\pgfpathcurveto{\pgfqpoint{2.946137in}{2.158673in}}{\pgfqpoint{2.942864in}{2.150773in}}{\pgfqpoint{2.942864in}{2.142537in}}%
\pgfpathcurveto{\pgfqpoint{2.942864in}{2.134301in}}{\pgfqpoint{2.946137in}{2.126401in}}{\pgfqpoint{2.951961in}{2.120577in}}%
\pgfpathcurveto{\pgfqpoint{2.957785in}{2.114753in}}{\pgfqpoint{2.965685in}{2.111480in}}{\pgfqpoint{2.973921in}{2.111480in}}%
\pgfpathclose%
\pgfusepath{stroke,fill}%
\end{pgfscope}%
\begin{pgfscope}%
\pgfpathrectangle{\pgfqpoint{0.100000in}{0.212622in}}{\pgfqpoint{3.696000in}{3.696000in}}%
\pgfusepath{clip}%
\pgfsetbuttcap%
\pgfsetroundjoin%
\definecolor{currentfill}{rgb}{0.121569,0.466667,0.705882}%
\pgfsetfillcolor{currentfill}%
\pgfsetfillopacity{0.754236}%
\pgfsetlinewidth{1.003750pt}%
\definecolor{currentstroke}{rgb}{0.121569,0.466667,0.705882}%
\pgfsetstrokecolor{currentstroke}%
\pgfsetstrokeopacity{0.754236}%
\pgfsetdash{}{0pt}%
\pgfpathmoveto{\pgfqpoint{2.969857in}{2.110390in}}%
\pgfpathcurveto{\pgfqpoint{2.978093in}{2.110390in}}{\pgfqpoint{2.985993in}{2.113663in}}{\pgfqpoint{2.991817in}{2.119487in}}%
\pgfpathcurveto{\pgfqpoint{2.997641in}{2.125310in}}{\pgfqpoint{3.000913in}{2.133211in}}{\pgfqpoint{3.000913in}{2.141447in}}%
\pgfpathcurveto{\pgfqpoint{3.000913in}{2.149683in}}{\pgfqpoint{2.997641in}{2.157583in}}{\pgfqpoint{2.991817in}{2.163407in}}%
\pgfpathcurveto{\pgfqpoint{2.985993in}{2.169231in}}{\pgfqpoint{2.978093in}{2.172503in}}{\pgfqpoint{2.969857in}{2.172503in}}%
\pgfpathcurveto{\pgfqpoint{2.961620in}{2.172503in}}{\pgfqpoint{2.953720in}{2.169231in}}{\pgfqpoint{2.947896in}{2.163407in}}%
\pgfpathcurveto{\pgfqpoint{2.942072in}{2.157583in}}{\pgfqpoint{2.938800in}{2.149683in}}{\pgfqpoint{2.938800in}{2.141447in}}%
\pgfpathcurveto{\pgfqpoint{2.938800in}{2.133211in}}{\pgfqpoint{2.942072in}{2.125310in}}{\pgfqpoint{2.947896in}{2.119487in}}%
\pgfpathcurveto{\pgfqpoint{2.953720in}{2.113663in}}{\pgfqpoint{2.961620in}{2.110390in}}{\pgfqpoint{2.969857in}{2.110390in}}%
\pgfpathclose%
\pgfusepath{stroke,fill}%
\end{pgfscope}%
\begin{pgfscope}%
\pgfpathrectangle{\pgfqpoint{0.100000in}{0.212622in}}{\pgfqpoint{3.696000in}{3.696000in}}%
\pgfusepath{clip}%
\pgfsetbuttcap%
\pgfsetroundjoin%
\definecolor{currentfill}{rgb}{0.121569,0.466667,0.705882}%
\pgfsetfillcolor{currentfill}%
\pgfsetfillopacity{0.756912}%
\pgfsetlinewidth{1.003750pt}%
\definecolor{currentstroke}{rgb}{0.121569,0.466667,0.705882}%
\pgfsetstrokecolor{currentstroke}%
\pgfsetstrokeopacity{0.756912}%
\pgfsetdash{}{0pt}%
\pgfpathmoveto{\pgfqpoint{2.964766in}{2.104999in}}%
\pgfpathcurveto{\pgfqpoint{2.973002in}{2.104999in}}{\pgfqpoint{2.980902in}{2.108271in}}{\pgfqpoint{2.986726in}{2.114095in}}%
\pgfpathcurveto{\pgfqpoint{2.992550in}{2.119919in}}{\pgfqpoint{2.995823in}{2.127819in}}{\pgfqpoint{2.995823in}{2.136056in}}%
\pgfpathcurveto{\pgfqpoint{2.995823in}{2.144292in}}{\pgfqpoint{2.992550in}{2.152192in}}{\pgfqpoint{2.986726in}{2.158016in}}%
\pgfpathcurveto{\pgfqpoint{2.980902in}{2.163840in}}{\pgfqpoint{2.973002in}{2.167112in}}{\pgfqpoint{2.964766in}{2.167112in}}%
\pgfpathcurveto{\pgfqpoint{2.956530in}{2.167112in}}{\pgfqpoint{2.948630in}{2.163840in}}{\pgfqpoint{2.942806in}{2.158016in}}%
\pgfpathcurveto{\pgfqpoint{2.936982in}{2.152192in}}{\pgfqpoint{2.933710in}{2.144292in}}{\pgfqpoint{2.933710in}{2.136056in}}%
\pgfpathcurveto{\pgfqpoint{2.933710in}{2.127819in}}{\pgfqpoint{2.936982in}{2.119919in}}{\pgfqpoint{2.942806in}{2.114095in}}%
\pgfpathcurveto{\pgfqpoint{2.948630in}{2.108271in}}{\pgfqpoint{2.956530in}{2.104999in}}{\pgfqpoint{2.964766in}{2.104999in}}%
\pgfpathclose%
\pgfusepath{stroke,fill}%
\end{pgfscope}%
\begin{pgfscope}%
\pgfpathrectangle{\pgfqpoint{0.100000in}{0.212622in}}{\pgfqpoint{3.696000in}{3.696000in}}%
\pgfusepath{clip}%
\pgfsetbuttcap%
\pgfsetroundjoin%
\definecolor{currentfill}{rgb}{0.121569,0.466667,0.705882}%
\pgfsetfillcolor{currentfill}%
\pgfsetfillopacity{0.758313}%
\pgfsetlinewidth{1.003750pt}%
\definecolor{currentstroke}{rgb}{0.121569,0.466667,0.705882}%
\pgfsetstrokecolor{currentstroke}%
\pgfsetstrokeopacity{0.758313}%
\pgfsetdash{}{0pt}%
\pgfpathmoveto{\pgfqpoint{2.960517in}{2.103516in}}%
\pgfpathcurveto{\pgfqpoint{2.968753in}{2.103516in}}{\pgfqpoint{2.976653in}{2.106788in}}{\pgfqpoint{2.982477in}{2.112612in}}%
\pgfpathcurveto{\pgfqpoint{2.988301in}{2.118436in}}{\pgfqpoint{2.991573in}{2.126336in}}{\pgfqpoint{2.991573in}{2.134572in}}%
\pgfpathcurveto{\pgfqpoint{2.991573in}{2.142808in}}{\pgfqpoint{2.988301in}{2.150708in}}{\pgfqpoint{2.982477in}{2.156532in}}%
\pgfpathcurveto{\pgfqpoint{2.976653in}{2.162356in}}{\pgfqpoint{2.968753in}{2.165629in}}{\pgfqpoint{2.960517in}{2.165629in}}%
\pgfpathcurveto{\pgfqpoint{2.952280in}{2.165629in}}{\pgfqpoint{2.944380in}{2.162356in}}{\pgfqpoint{2.938557in}{2.156532in}}%
\pgfpathcurveto{\pgfqpoint{2.932733in}{2.150708in}}{\pgfqpoint{2.929460in}{2.142808in}}{\pgfqpoint{2.929460in}{2.134572in}}%
\pgfpathcurveto{\pgfqpoint{2.929460in}{2.126336in}}{\pgfqpoint{2.932733in}{2.118436in}}{\pgfqpoint{2.938557in}{2.112612in}}%
\pgfpathcurveto{\pgfqpoint{2.944380in}{2.106788in}}{\pgfqpoint{2.952280in}{2.103516in}}{\pgfqpoint{2.960517in}{2.103516in}}%
\pgfpathclose%
\pgfusepath{stroke,fill}%
\end{pgfscope}%
\begin{pgfscope}%
\pgfpathrectangle{\pgfqpoint{0.100000in}{0.212622in}}{\pgfqpoint{3.696000in}{3.696000in}}%
\pgfusepath{clip}%
\pgfsetbuttcap%
\pgfsetroundjoin%
\definecolor{currentfill}{rgb}{0.121569,0.466667,0.705882}%
\pgfsetfillcolor{currentfill}%
\pgfsetfillopacity{0.760894}%
\pgfsetlinewidth{1.003750pt}%
\definecolor{currentstroke}{rgb}{0.121569,0.466667,0.705882}%
\pgfsetstrokecolor{currentstroke}%
\pgfsetstrokeopacity{0.760894}%
\pgfsetdash{}{0pt}%
\pgfpathmoveto{\pgfqpoint{2.955825in}{2.101240in}}%
\pgfpathcurveto{\pgfqpoint{2.964061in}{2.101240in}}{\pgfqpoint{2.971961in}{2.104513in}}{\pgfqpoint{2.977785in}{2.110337in}}%
\pgfpathcurveto{\pgfqpoint{2.983609in}{2.116160in}}{\pgfqpoint{2.986882in}{2.124060in}}{\pgfqpoint{2.986882in}{2.132297in}}%
\pgfpathcurveto{\pgfqpoint{2.986882in}{2.140533in}}{\pgfqpoint{2.983609in}{2.148433in}}{\pgfqpoint{2.977785in}{2.154257in}}%
\pgfpathcurveto{\pgfqpoint{2.971961in}{2.160081in}}{\pgfqpoint{2.964061in}{2.163353in}}{\pgfqpoint{2.955825in}{2.163353in}}%
\pgfpathcurveto{\pgfqpoint{2.947589in}{2.163353in}}{\pgfqpoint{2.939689in}{2.160081in}}{\pgfqpoint{2.933865in}{2.154257in}}%
\pgfpathcurveto{\pgfqpoint{2.928041in}{2.148433in}}{\pgfqpoint{2.924769in}{2.140533in}}{\pgfqpoint{2.924769in}{2.132297in}}%
\pgfpathcurveto{\pgfqpoint{2.924769in}{2.124060in}}{\pgfqpoint{2.928041in}{2.116160in}}{\pgfqpoint{2.933865in}{2.110337in}}%
\pgfpathcurveto{\pgfqpoint{2.939689in}{2.104513in}}{\pgfqpoint{2.947589in}{2.101240in}}{\pgfqpoint{2.955825in}{2.101240in}}%
\pgfpathclose%
\pgfusepath{stroke,fill}%
\end{pgfscope}%
\begin{pgfscope}%
\pgfpathrectangle{\pgfqpoint{0.100000in}{0.212622in}}{\pgfqpoint{3.696000in}{3.696000in}}%
\pgfusepath{clip}%
\pgfsetbuttcap%
\pgfsetroundjoin%
\definecolor{currentfill}{rgb}{0.121569,0.466667,0.705882}%
\pgfsetfillcolor{currentfill}%
\pgfsetfillopacity{0.762193}%
\pgfsetlinewidth{1.003750pt}%
\definecolor{currentstroke}{rgb}{0.121569,0.466667,0.705882}%
\pgfsetstrokecolor{currentstroke}%
\pgfsetstrokeopacity{0.762193}%
\pgfsetdash{}{0pt}%
\pgfpathmoveto{\pgfqpoint{2.952598in}{2.100012in}}%
\pgfpathcurveto{\pgfqpoint{2.960834in}{2.100012in}}{\pgfqpoint{2.968734in}{2.103284in}}{\pgfqpoint{2.974558in}{2.109108in}}%
\pgfpathcurveto{\pgfqpoint{2.980382in}{2.114932in}}{\pgfqpoint{2.983654in}{2.122832in}}{\pgfqpoint{2.983654in}{2.131068in}}%
\pgfpathcurveto{\pgfqpoint{2.983654in}{2.139305in}}{\pgfqpoint{2.980382in}{2.147205in}}{\pgfqpoint{2.974558in}{2.153029in}}%
\pgfpathcurveto{\pgfqpoint{2.968734in}{2.158853in}}{\pgfqpoint{2.960834in}{2.162125in}}{\pgfqpoint{2.952598in}{2.162125in}}%
\pgfpathcurveto{\pgfqpoint{2.944362in}{2.162125in}}{\pgfqpoint{2.936462in}{2.158853in}}{\pgfqpoint{2.930638in}{2.153029in}}%
\pgfpathcurveto{\pgfqpoint{2.924814in}{2.147205in}}{\pgfqpoint{2.921541in}{2.139305in}}{\pgfqpoint{2.921541in}{2.131068in}}%
\pgfpathcurveto{\pgfqpoint{2.921541in}{2.122832in}}{\pgfqpoint{2.924814in}{2.114932in}}{\pgfqpoint{2.930638in}{2.109108in}}%
\pgfpathcurveto{\pgfqpoint{2.936462in}{2.103284in}}{\pgfqpoint{2.944362in}{2.100012in}}{\pgfqpoint{2.952598in}{2.100012in}}%
\pgfpathclose%
\pgfusepath{stroke,fill}%
\end{pgfscope}%
\begin{pgfscope}%
\pgfpathrectangle{\pgfqpoint{0.100000in}{0.212622in}}{\pgfqpoint{3.696000in}{3.696000in}}%
\pgfusepath{clip}%
\pgfsetbuttcap%
\pgfsetroundjoin%
\definecolor{currentfill}{rgb}{0.121569,0.466667,0.705882}%
\pgfsetfillcolor{currentfill}%
\pgfsetfillopacity{0.762904}%
\pgfsetlinewidth{1.003750pt}%
\definecolor{currentstroke}{rgb}{0.121569,0.466667,0.705882}%
\pgfsetstrokecolor{currentstroke}%
\pgfsetstrokeopacity{0.762904}%
\pgfsetdash{}{0pt}%
\pgfpathmoveto{\pgfqpoint{2.951073in}{2.098969in}}%
\pgfpathcurveto{\pgfqpoint{2.959309in}{2.098969in}}{\pgfqpoint{2.967209in}{2.102241in}}{\pgfqpoint{2.973033in}{2.108065in}}%
\pgfpathcurveto{\pgfqpoint{2.978857in}{2.113889in}}{\pgfqpoint{2.982129in}{2.121789in}}{\pgfqpoint{2.982129in}{2.130025in}}%
\pgfpathcurveto{\pgfqpoint{2.982129in}{2.138262in}}{\pgfqpoint{2.978857in}{2.146162in}}{\pgfqpoint{2.973033in}{2.151986in}}%
\pgfpathcurveto{\pgfqpoint{2.967209in}{2.157810in}}{\pgfqpoint{2.959309in}{2.161082in}}{\pgfqpoint{2.951073in}{2.161082in}}%
\pgfpathcurveto{\pgfqpoint{2.942836in}{2.161082in}}{\pgfqpoint{2.934936in}{2.157810in}}{\pgfqpoint{2.929112in}{2.151986in}}%
\pgfpathcurveto{\pgfqpoint{2.923288in}{2.146162in}}{\pgfqpoint{2.920016in}{2.138262in}}{\pgfqpoint{2.920016in}{2.130025in}}%
\pgfpathcurveto{\pgfqpoint{2.920016in}{2.121789in}}{\pgfqpoint{2.923288in}{2.113889in}}{\pgfqpoint{2.929112in}{2.108065in}}%
\pgfpathcurveto{\pgfqpoint{2.934936in}{2.102241in}}{\pgfqpoint{2.942836in}{2.098969in}}{\pgfqpoint{2.951073in}{2.098969in}}%
\pgfpathclose%
\pgfusepath{stroke,fill}%
\end{pgfscope}%
\begin{pgfscope}%
\pgfpathrectangle{\pgfqpoint{0.100000in}{0.212622in}}{\pgfqpoint{3.696000in}{3.696000in}}%
\pgfusepath{clip}%
\pgfsetbuttcap%
\pgfsetroundjoin%
\definecolor{currentfill}{rgb}{0.121569,0.466667,0.705882}%
\pgfsetfillcolor{currentfill}%
\pgfsetfillopacity{0.763325}%
\pgfsetlinewidth{1.003750pt}%
\definecolor{currentstroke}{rgb}{0.121569,0.466667,0.705882}%
\pgfsetstrokecolor{currentstroke}%
\pgfsetstrokeopacity{0.763325}%
\pgfsetdash{}{0pt}%
\pgfpathmoveto{\pgfqpoint{2.950075in}{2.098820in}}%
\pgfpathcurveto{\pgfqpoint{2.958312in}{2.098820in}}{\pgfqpoint{2.966212in}{2.102092in}}{\pgfqpoint{2.972036in}{2.107916in}}%
\pgfpathcurveto{\pgfqpoint{2.977860in}{2.113740in}}{\pgfqpoint{2.981132in}{2.121640in}}{\pgfqpoint{2.981132in}{2.129876in}}%
\pgfpathcurveto{\pgfqpoint{2.981132in}{2.138112in}}{\pgfqpoint{2.977860in}{2.146012in}}{\pgfqpoint{2.972036in}{2.151836in}}%
\pgfpathcurveto{\pgfqpoint{2.966212in}{2.157660in}}{\pgfqpoint{2.958312in}{2.160933in}}{\pgfqpoint{2.950075in}{2.160933in}}%
\pgfpathcurveto{\pgfqpoint{2.941839in}{2.160933in}}{\pgfqpoint{2.933939in}{2.157660in}}{\pgfqpoint{2.928115in}{2.151836in}}%
\pgfpathcurveto{\pgfqpoint{2.922291in}{2.146012in}}{\pgfqpoint{2.919019in}{2.138112in}}{\pgfqpoint{2.919019in}{2.129876in}}%
\pgfpathcurveto{\pgfqpoint{2.919019in}{2.121640in}}{\pgfqpoint{2.922291in}{2.113740in}}{\pgfqpoint{2.928115in}{2.107916in}}%
\pgfpathcurveto{\pgfqpoint{2.933939in}{2.102092in}}{\pgfqpoint{2.941839in}{2.098820in}}{\pgfqpoint{2.950075in}{2.098820in}}%
\pgfpathclose%
\pgfusepath{stroke,fill}%
\end{pgfscope}%
\begin{pgfscope}%
\pgfpathrectangle{\pgfqpoint{0.100000in}{0.212622in}}{\pgfqpoint{3.696000in}{3.696000in}}%
\pgfusepath{clip}%
\pgfsetbuttcap%
\pgfsetroundjoin%
\definecolor{currentfill}{rgb}{0.121569,0.466667,0.705882}%
\pgfsetfillcolor{currentfill}%
\pgfsetfillopacity{0.763536}%
\pgfsetlinewidth{1.003750pt}%
\definecolor{currentstroke}{rgb}{0.121569,0.466667,0.705882}%
\pgfsetstrokecolor{currentstroke}%
\pgfsetstrokeopacity{0.763536}%
\pgfsetdash{}{0pt}%
\pgfpathmoveto{\pgfqpoint{2.949520in}{2.098609in}}%
\pgfpathcurveto{\pgfqpoint{2.957756in}{2.098609in}}{\pgfqpoint{2.965656in}{2.101881in}}{\pgfqpoint{2.971480in}{2.107705in}}%
\pgfpathcurveto{\pgfqpoint{2.977304in}{2.113529in}}{\pgfqpoint{2.980576in}{2.121429in}}{\pgfqpoint{2.980576in}{2.129666in}}%
\pgfpathcurveto{\pgfqpoint{2.980576in}{2.137902in}}{\pgfqpoint{2.977304in}{2.145802in}}{\pgfqpoint{2.971480in}{2.151626in}}%
\pgfpathcurveto{\pgfqpoint{2.965656in}{2.157450in}}{\pgfqpoint{2.957756in}{2.160722in}}{\pgfqpoint{2.949520in}{2.160722in}}%
\pgfpathcurveto{\pgfqpoint{2.941283in}{2.160722in}}{\pgfqpoint{2.933383in}{2.157450in}}{\pgfqpoint{2.927559in}{2.151626in}}%
\pgfpathcurveto{\pgfqpoint{2.921735in}{2.145802in}}{\pgfqpoint{2.918463in}{2.137902in}}{\pgfqpoint{2.918463in}{2.129666in}}%
\pgfpathcurveto{\pgfqpoint{2.918463in}{2.121429in}}{\pgfqpoint{2.921735in}{2.113529in}}{\pgfqpoint{2.927559in}{2.107705in}}%
\pgfpathcurveto{\pgfqpoint{2.933383in}{2.101881in}}{\pgfqpoint{2.941283in}{2.098609in}}{\pgfqpoint{2.949520in}{2.098609in}}%
\pgfpathclose%
\pgfusepath{stroke,fill}%
\end{pgfscope}%
\begin{pgfscope}%
\pgfpathrectangle{\pgfqpoint{0.100000in}{0.212622in}}{\pgfqpoint{3.696000in}{3.696000in}}%
\pgfusepath{clip}%
\pgfsetbuttcap%
\pgfsetroundjoin%
\definecolor{currentfill}{rgb}{0.121569,0.466667,0.705882}%
\pgfsetfillcolor{currentfill}%
\pgfsetfillopacity{0.763675}%
\pgfsetlinewidth{1.003750pt}%
\definecolor{currentstroke}{rgb}{0.121569,0.466667,0.705882}%
\pgfsetstrokecolor{currentstroke}%
\pgfsetstrokeopacity{0.763675}%
\pgfsetdash{}{0pt}%
\pgfpathmoveto{\pgfqpoint{2.949287in}{2.098546in}}%
\pgfpathcurveto{\pgfqpoint{2.957523in}{2.098546in}}{\pgfqpoint{2.965423in}{2.101818in}}{\pgfqpoint{2.971247in}{2.107642in}}%
\pgfpathcurveto{\pgfqpoint{2.977071in}{2.113466in}}{\pgfqpoint{2.980343in}{2.121366in}}{\pgfqpoint{2.980343in}{2.129603in}}%
\pgfpathcurveto{\pgfqpoint{2.980343in}{2.137839in}}{\pgfqpoint{2.977071in}{2.145739in}}{\pgfqpoint{2.971247in}{2.151563in}}%
\pgfpathcurveto{\pgfqpoint{2.965423in}{2.157387in}}{\pgfqpoint{2.957523in}{2.160659in}}{\pgfqpoint{2.949287in}{2.160659in}}%
\pgfpathcurveto{\pgfqpoint{2.941050in}{2.160659in}}{\pgfqpoint{2.933150in}{2.157387in}}{\pgfqpoint{2.927326in}{2.151563in}}%
\pgfpathcurveto{\pgfqpoint{2.921503in}{2.145739in}}{\pgfqpoint{2.918230in}{2.137839in}}{\pgfqpoint{2.918230in}{2.129603in}}%
\pgfpathcurveto{\pgfqpoint{2.918230in}{2.121366in}}{\pgfqpoint{2.921503in}{2.113466in}}{\pgfqpoint{2.927326in}{2.107642in}}%
\pgfpathcurveto{\pgfqpoint{2.933150in}{2.101818in}}{\pgfqpoint{2.941050in}{2.098546in}}{\pgfqpoint{2.949287in}{2.098546in}}%
\pgfpathclose%
\pgfusepath{stroke,fill}%
\end{pgfscope}%
\begin{pgfscope}%
\pgfpathrectangle{\pgfqpoint{0.100000in}{0.212622in}}{\pgfqpoint{3.696000in}{3.696000in}}%
\pgfusepath{clip}%
\pgfsetbuttcap%
\pgfsetroundjoin%
\definecolor{currentfill}{rgb}{0.121569,0.466667,0.705882}%
\pgfsetfillcolor{currentfill}%
\pgfsetfillopacity{0.764302}%
\pgfsetlinewidth{1.003750pt}%
\definecolor{currentstroke}{rgb}{0.121569,0.466667,0.705882}%
\pgfsetstrokecolor{currentstroke}%
\pgfsetstrokeopacity{0.764302}%
\pgfsetdash{}{0pt}%
\pgfpathmoveto{\pgfqpoint{2.947326in}{2.097992in}}%
\pgfpathcurveto{\pgfqpoint{2.955562in}{2.097992in}}{\pgfqpoint{2.963462in}{2.101264in}}{\pgfqpoint{2.969286in}{2.107088in}}%
\pgfpathcurveto{\pgfqpoint{2.975110in}{2.112912in}}{\pgfqpoint{2.978382in}{2.120812in}}{\pgfqpoint{2.978382in}{2.129049in}}%
\pgfpathcurveto{\pgfqpoint{2.978382in}{2.137285in}}{\pgfqpoint{2.975110in}{2.145185in}}{\pgfqpoint{2.969286in}{2.151009in}}%
\pgfpathcurveto{\pgfqpoint{2.963462in}{2.156833in}}{\pgfqpoint{2.955562in}{2.160105in}}{\pgfqpoint{2.947326in}{2.160105in}}%
\pgfpathcurveto{\pgfqpoint{2.939090in}{2.160105in}}{\pgfqpoint{2.931190in}{2.156833in}}{\pgfqpoint{2.925366in}{2.151009in}}%
\pgfpathcurveto{\pgfqpoint{2.919542in}{2.145185in}}{\pgfqpoint{2.916269in}{2.137285in}}{\pgfqpoint{2.916269in}{2.129049in}}%
\pgfpathcurveto{\pgfqpoint{2.916269in}{2.120812in}}{\pgfqpoint{2.919542in}{2.112912in}}{\pgfqpoint{2.925366in}{2.107088in}}%
\pgfpathcurveto{\pgfqpoint{2.931190in}{2.101264in}}{\pgfqpoint{2.939090in}{2.097992in}}{\pgfqpoint{2.947326in}{2.097992in}}%
\pgfpathclose%
\pgfusepath{stroke,fill}%
\end{pgfscope}%
\begin{pgfscope}%
\pgfpathrectangle{\pgfqpoint{0.100000in}{0.212622in}}{\pgfqpoint{3.696000in}{3.696000in}}%
\pgfusepath{clip}%
\pgfsetbuttcap%
\pgfsetroundjoin%
\definecolor{currentfill}{rgb}{0.121569,0.466667,0.705882}%
\pgfsetfillcolor{currentfill}%
\pgfsetfillopacity{0.765834}%
\pgfsetlinewidth{1.003750pt}%
\definecolor{currentstroke}{rgb}{0.121569,0.466667,0.705882}%
\pgfsetstrokecolor{currentstroke}%
\pgfsetstrokeopacity{0.765834}%
\pgfsetdash{}{0pt}%
\pgfpathmoveto{\pgfqpoint{2.944452in}{2.095069in}}%
\pgfpathcurveto{\pgfqpoint{2.952689in}{2.095069in}}{\pgfqpoint{2.960589in}{2.098341in}}{\pgfqpoint{2.966413in}{2.104165in}}%
\pgfpathcurveto{\pgfqpoint{2.972237in}{2.109989in}}{\pgfqpoint{2.975509in}{2.117889in}}{\pgfqpoint{2.975509in}{2.126125in}}%
\pgfpathcurveto{\pgfqpoint{2.975509in}{2.134362in}}{\pgfqpoint{2.972237in}{2.142262in}}{\pgfqpoint{2.966413in}{2.148086in}}%
\pgfpathcurveto{\pgfqpoint{2.960589in}{2.153910in}}{\pgfqpoint{2.952689in}{2.157182in}}{\pgfqpoint{2.944452in}{2.157182in}}%
\pgfpathcurveto{\pgfqpoint{2.936216in}{2.157182in}}{\pgfqpoint{2.928316in}{2.153910in}}{\pgfqpoint{2.922492in}{2.148086in}}%
\pgfpathcurveto{\pgfqpoint{2.916668in}{2.142262in}}{\pgfqpoint{2.913396in}{2.134362in}}{\pgfqpoint{2.913396in}{2.126125in}}%
\pgfpathcurveto{\pgfqpoint{2.913396in}{2.117889in}}{\pgfqpoint{2.916668in}{2.109989in}}{\pgfqpoint{2.922492in}{2.104165in}}%
\pgfpathcurveto{\pgfqpoint{2.928316in}{2.098341in}}{\pgfqpoint{2.936216in}{2.095069in}}{\pgfqpoint{2.944452in}{2.095069in}}%
\pgfpathclose%
\pgfusepath{stroke,fill}%
\end{pgfscope}%
\begin{pgfscope}%
\pgfpathrectangle{\pgfqpoint{0.100000in}{0.212622in}}{\pgfqpoint{3.696000in}{3.696000in}}%
\pgfusepath{clip}%
\pgfsetbuttcap%
\pgfsetroundjoin%
\definecolor{currentfill}{rgb}{0.121569,0.466667,0.705882}%
\pgfsetfillcolor{currentfill}%
\pgfsetfillopacity{0.768721}%
\pgfsetlinewidth{1.003750pt}%
\definecolor{currentstroke}{rgb}{0.121569,0.466667,0.705882}%
\pgfsetstrokecolor{currentstroke}%
\pgfsetstrokeopacity{0.768721}%
\pgfsetdash{}{0pt}%
\pgfpathmoveto{\pgfqpoint{2.936809in}{2.092610in}}%
\pgfpathcurveto{\pgfqpoint{2.945045in}{2.092610in}}{\pgfqpoint{2.952945in}{2.095882in}}{\pgfqpoint{2.958769in}{2.101706in}}%
\pgfpathcurveto{\pgfqpoint{2.964593in}{2.107530in}}{\pgfqpoint{2.967865in}{2.115430in}}{\pgfqpoint{2.967865in}{2.123666in}}%
\pgfpathcurveto{\pgfqpoint{2.967865in}{2.131902in}}{\pgfqpoint{2.964593in}{2.139802in}}{\pgfqpoint{2.958769in}{2.145626in}}%
\pgfpathcurveto{\pgfqpoint{2.952945in}{2.151450in}}{\pgfqpoint{2.945045in}{2.154723in}}{\pgfqpoint{2.936809in}{2.154723in}}%
\pgfpathcurveto{\pgfqpoint{2.928573in}{2.154723in}}{\pgfqpoint{2.920673in}{2.151450in}}{\pgfqpoint{2.914849in}{2.145626in}}%
\pgfpathcurveto{\pgfqpoint{2.909025in}{2.139802in}}{\pgfqpoint{2.905752in}{2.131902in}}{\pgfqpoint{2.905752in}{2.123666in}}%
\pgfpathcurveto{\pgfqpoint{2.905752in}{2.115430in}}{\pgfqpoint{2.909025in}{2.107530in}}{\pgfqpoint{2.914849in}{2.101706in}}%
\pgfpathcurveto{\pgfqpoint{2.920673in}{2.095882in}}{\pgfqpoint{2.928573in}{2.092610in}}{\pgfqpoint{2.936809in}{2.092610in}}%
\pgfpathclose%
\pgfusepath{stroke,fill}%
\end{pgfscope}%
\begin{pgfscope}%
\pgfpathrectangle{\pgfqpoint{0.100000in}{0.212622in}}{\pgfqpoint{3.696000in}{3.696000in}}%
\pgfusepath{clip}%
\pgfsetbuttcap%
\pgfsetroundjoin%
\definecolor{currentfill}{rgb}{0.121569,0.466667,0.705882}%
\pgfsetfillcolor{currentfill}%
\pgfsetfillopacity{0.773426}%
\pgfsetlinewidth{1.003750pt}%
\definecolor{currentstroke}{rgb}{0.121569,0.466667,0.705882}%
\pgfsetstrokecolor{currentstroke}%
\pgfsetstrokeopacity{0.773426}%
\pgfsetdash{}{0pt}%
\pgfpathmoveto{\pgfqpoint{2.928235in}{2.087193in}}%
\pgfpathcurveto{\pgfqpoint{2.936471in}{2.087193in}}{\pgfqpoint{2.944371in}{2.090465in}}{\pgfqpoint{2.950195in}{2.096289in}}%
\pgfpathcurveto{\pgfqpoint{2.956019in}{2.102113in}}{\pgfqpoint{2.959292in}{2.110013in}}{\pgfqpoint{2.959292in}{2.118249in}}%
\pgfpathcurveto{\pgfqpoint{2.959292in}{2.126485in}}{\pgfqpoint{2.956019in}{2.134385in}}{\pgfqpoint{2.950195in}{2.140209in}}%
\pgfpathcurveto{\pgfqpoint{2.944371in}{2.146033in}}{\pgfqpoint{2.936471in}{2.149306in}}{\pgfqpoint{2.928235in}{2.149306in}}%
\pgfpathcurveto{\pgfqpoint{2.919999in}{2.149306in}}{\pgfqpoint{2.912099in}{2.146033in}}{\pgfqpoint{2.906275in}{2.140209in}}%
\pgfpathcurveto{\pgfqpoint{2.900451in}{2.134385in}}{\pgfqpoint{2.897179in}{2.126485in}}{\pgfqpoint{2.897179in}{2.118249in}}%
\pgfpathcurveto{\pgfqpoint{2.897179in}{2.110013in}}{\pgfqpoint{2.900451in}{2.102113in}}{\pgfqpoint{2.906275in}{2.096289in}}%
\pgfpathcurveto{\pgfqpoint{2.912099in}{2.090465in}}{\pgfqpoint{2.919999in}{2.087193in}}{\pgfqpoint{2.928235in}{2.087193in}}%
\pgfpathclose%
\pgfusepath{stroke,fill}%
\end{pgfscope}%
\begin{pgfscope}%
\pgfpathrectangle{\pgfqpoint{0.100000in}{0.212622in}}{\pgfqpoint{3.696000in}{3.696000in}}%
\pgfusepath{clip}%
\pgfsetbuttcap%
\pgfsetroundjoin%
\definecolor{currentfill}{rgb}{0.121569,0.466667,0.705882}%
\pgfsetfillcolor{currentfill}%
\pgfsetfillopacity{0.778524}%
\pgfsetlinewidth{1.003750pt}%
\definecolor{currentstroke}{rgb}{0.121569,0.466667,0.705882}%
\pgfsetstrokecolor{currentstroke}%
\pgfsetstrokeopacity{0.778524}%
\pgfsetdash{}{0pt}%
\pgfpathmoveto{\pgfqpoint{2.913346in}{2.082209in}}%
\pgfpathcurveto{\pgfqpoint{2.921582in}{2.082209in}}{\pgfqpoint{2.929482in}{2.085481in}}{\pgfqpoint{2.935306in}{2.091305in}}%
\pgfpathcurveto{\pgfqpoint{2.941130in}{2.097129in}}{\pgfqpoint{2.944402in}{2.105029in}}{\pgfqpoint{2.944402in}{2.113265in}}%
\pgfpathcurveto{\pgfqpoint{2.944402in}{2.121502in}}{\pgfqpoint{2.941130in}{2.129402in}}{\pgfqpoint{2.935306in}{2.135226in}}%
\pgfpathcurveto{\pgfqpoint{2.929482in}{2.141049in}}{\pgfqpoint{2.921582in}{2.144322in}}{\pgfqpoint{2.913346in}{2.144322in}}%
\pgfpathcurveto{\pgfqpoint{2.905110in}{2.144322in}}{\pgfqpoint{2.897210in}{2.141049in}}{\pgfqpoint{2.891386in}{2.135226in}}%
\pgfpathcurveto{\pgfqpoint{2.885562in}{2.129402in}}{\pgfqpoint{2.882289in}{2.121502in}}{\pgfqpoint{2.882289in}{2.113265in}}%
\pgfpathcurveto{\pgfqpoint{2.882289in}{2.105029in}}{\pgfqpoint{2.885562in}{2.097129in}}{\pgfqpoint{2.891386in}{2.091305in}}%
\pgfpathcurveto{\pgfqpoint{2.897210in}{2.085481in}}{\pgfqpoint{2.905110in}{2.082209in}}{\pgfqpoint{2.913346in}{2.082209in}}%
\pgfpathclose%
\pgfusepath{stroke,fill}%
\end{pgfscope}%
\begin{pgfscope}%
\pgfpathrectangle{\pgfqpoint{0.100000in}{0.212622in}}{\pgfqpoint{3.696000in}{3.696000in}}%
\pgfusepath{clip}%
\pgfsetbuttcap%
\pgfsetroundjoin%
\definecolor{currentfill}{rgb}{0.121569,0.466667,0.705882}%
\pgfsetfillcolor{currentfill}%
\pgfsetfillopacity{0.784911}%
\pgfsetlinewidth{1.003750pt}%
\definecolor{currentstroke}{rgb}{0.121569,0.466667,0.705882}%
\pgfsetstrokecolor{currentstroke}%
\pgfsetstrokeopacity{0.784911}%
\pgfsetdash{}{0pt}%
\pgfpathmoveto{\pgfqpoint{2.900326in}{2.074034in}}%
\pgfpathcurveto{\pgfqpoint{2.908562in}{2.074034in}}{\pgfqpoint{2.916462in}{2.077306in}}{\pgfqpoint{2.922286in}{2.083130in}}%
\pgfpathcurveto{\pgfqpoint{2.928110in}{2.088954in}}{\pgfqpoint{2.931383in}{2.096854in}}{\pgfqpoint{2.931383in}{2.105090in}}%
\pgfpathcurveto{\pgfqpoint{2.931383in}{2.113327in}}{\pgfqpoint{2.928110in}{2.121227in}}{\pgfqpoint{2.922286in}{2.127051in}}%
\pgfpathcurveto{\pgfqpoint{2.916462in}{2.132874in}}{\pgfqpoint{2.908562in}{2.136147in}}{\pgfqpoint{2.900326in}{2.136147in}}%
\pgfpathcurveto{\pgfqpoint{2.892090in}{2.136147in}}{\pgfqpoint{2.884190in}{2.132874in}}{\pgfqpoint{2.878366in}{2.127051in}}%
\pgfpathcurveto{\pgfqpoint{2.872542in}{2.121227in}}{\pgfqpoint{2.869270in}{2.113327in}}{\pgfqpoint{2.869270in}{2.105090in}}%
\pgfpathcurveto{\pgfqpoint{2.869270in}{2.096854in}}{\pgfqpoint{2.872542in}{2.088954in}}{\pgfqpoint{2.878366in}{2.083130in}}%
\pgfpathcurveto{\pgfqpoint{2.884190in}{2.077306in}}{\pgfqpoint{2.892090in}{2.074034in}}{\pgfqpoint{2.900326in}{2.074034in}}%
\pgfpathclose%
\pgfusepath{stroke,fill}%
\end{pgfscope}%
\begin{pgfscope}%
\pgfpathrectangle{\pgfqpoint{0.100000in}{0.212622in}}{\pgfqpoint{3.696000in}{3.696000in}}%
\pgfusepath{clip}%
\pgfsetbuttcap%
\pgfsetroundjoin%
\definecolor{currentfill}{rgb}{0.121569,0.466667,0.705882}%
\pgfsetfillcolor{currentfill}%
\pgfsetfillopacity{0.790925}%
\pgfsetlinewidth{1.003750pt}%
\definecolor{currentstroke}{rgb}{0.121569,0.466667,0.705882}%
\pgfsetstrokecolor{currentstroke}%
\pgfsetstrokeopacity{0.790925}%
\pgfsetdash{}{0pt}%
\pgfpathmoveto{\pgfqpoint{2.880711in}{2.067015in}}%
\pgfpathcurveto{\pgfqpoint{2.888947in}{2.067015in}}{\pgfqpoint{2.896847in}{2.070287in}}{\pgfqpoint{2.902671in}{2.076111in}}%
\pgfpathcurveto{\pgfqpoint{2.908495in}{2.081935in}}{\pgfqpoint{2.911767in}{2.089835in}}{\pgfqpoint{2.911767in}{2.098071in}}%
\pgfpathcurveto{\pgfqpoint{2.911767in}{2.106307in}}{\pgfqpoint{2.908495in}{2.114207in}}{\pgfqpoint{2.902671in}{2.120031in}}%
\pgfpathcurveto{\pgfqpoint{2.896847in}{2.125855in}}{\pgfqpoint{2.888947in}{2.129128in}}{\pgfqpoint{2.880711in}{2.129128in}}%
\pgfpathcurveto{\pgfqpoint{2.872474in}{2.129128in}}{\pgfqpoint{2.864574in}{2.125855in}}{\pgfqpoint{2.858750in}{2.120031in}}%
\pgfpathcurveto{\pgfqpoint{2.852927in}{2.114207in}}{\pgfqpoint{2.849654in}{2.106307in}}{\pgfqpoint{2.849654in}{2.098071in}}%
\pgfpathcurveto{\pgfqpoint{2.849654in}{2.089835in}}{\pgfqpoint{2.852927in}{2.081935in}}{\pgfqpoint{2.858750in}{2.076111in}}%
\pgfpathcurveto{\pgfqpoint{2.864574in}{2.070287in}}{\pgfqpoint{2.872474in}{2.067015in}}{\pgfqpoint{2.880711in}{2.067015in}}%
\pgfpathclose%
\pgfusepath{stroke,fill}%
\end{pgfscope}%
\begin{pgfscope}%
\pgfpathrectangle{\pgfqpoint{0.100000in}{0.212622in}}{\pgfqpoint{3.696000in}{3.696000in}}%
\pgfusepath{clip}%
\pgfsetbuttcap%
\pgfsetroundjoin%
\definecolor{currentfill}{rgb}{0.121569,0.466667,0.705882}%
\pgfsetfillcolor{currentfill}%
\pgfsetfillopacity{0.794780}%
\pgfsetlinewidth{1.003750pt}%
\definecolor{currentstroke}{rgb}{0.121569,0.466667,0.705882}%
\pgfsetstrokecolor{currentstroke}%
\pgfsetstrokeopacity{0.794780}%
\pgfsetdash{}{0pt}%
\pgfpathmoveto{\pgfqpoint{2.872468in}{2.062750in}}%
\pgfpathcurveto{\pgfqpoint{2.880704in}{2.062750in}}{\pgfqpoint{2.888604in}{2.066023in}}{\pgfqpoint{2.894428in}{2.071846in}}%
\pgfpathcurveto{\pgfqpoint{2.900252in}{2.077670in}}{\pgfqpoint{2.903524in}{2.085570in}}{\pgfqpoint{2.903524in}{2.093807in}}%
\pgfpathcurveto{\pgfqpoint{2.903524in}{2.102043in}}{\pgfqpoint{2.900252in}{2.109943in}}{\pgfqpoint{2.894428in}{2.115767in}}%
\pgfpathcurveto{\pgfqpoint{2.888604in}{2.121591in}}{\pgfqpoint{2.880704in}{2.124863in}}{\pgfqpoint{2.872468in}{2.124863in}}%
\pgfpathcurveto{\pgfqpoint{2.864231in}{2.124863in}}{\pgfqpoint{2.856331in}{2.121591in}}{\pgfqpoint{2.850507in}{2.115767in}}%
\pgfpathcurveto{\pgfqpoint{2.844683in}{2.109943in}}{\pgfqpoint{2.841411in}{2.102043in}}{\pgfqpoint{2.841411in}{2.093807in}}%
\pgfpathcurveto{\pgfqpoint{2.841411in}{2.085570in}}{\pgfqpoint{2.844683in}{2.077670in}}{\pgfqpoint{2.850507in}{2.071846in}}%
\pgfpathcurveto{\pgfqpoint{2.856331in}{2.066023in}}{\pgfqpoint{2.864231in}{2.062750in}}{\pgfqpoint{2.872468in}{2.062750in}}%
\pgfpathclose%
\pgfusepath{stroke,fill}%
\end{pgfscope}%
\begin{pgfscope}%
\pgfpathrectangle{\pgfqpoint{0.100000in}{0.212622in}}{\pgfqpoint{3.696000in}{3.696000in}}%
\pgfusepath{clip}%
\pgfsetbuttcap%
\pgfsetroundjoin%
\definecolor{currentfill}{rgb}{0.121569,0.466667,0.705882}%
\pgfsetfillcolor{currentfill}%
\pgfsetfillopacity{0.796740}%
\pgfsetlinewidth{1.003750pt}%
\definecolor{currentstroke}{rgb}{0.121569,0.466667,0.705882}%
\pgfsetstrokecolor{currentstroke}%
\pgfsetstrokeopacity{0.796740}%
\pgfsetdash{}{0pt}%
\pgfpathmoveto{\pgfqpoint{0.668713in}{2.668621in}}%
\pgfpathcurveto{\pgfqpoint{0.676949in}{2.668621in}}{\pgfqpoint{0.684849in}{2.671893in}}{\pgfqpoint{0.690673in}{2.677717in}}%
\pgfpathcurveto{\pgfqpoint{0.696497in}{2.683541in}}{\pgfqpoint{0.699770in}{2.691441in}}{\pgfqpoint{0.699770in}{2.699677in}}%
\pgfpathcurveto{\pgfqpoint{0.699770in}{2.707914in}}{\pgfqpoint{0.696497in}{2.715814in}}{\pgfqpoint{0.690673in}{2.721638in}}%
\pgfpathcurveto{\pgfqpoint{0.684849in}{2.727462in}}{\pgfqpoint{0.676949in}{2.730734in}}{\pgfqpoint{0.668713in}{2.730734in}}%
\pgfpathcurveto{\pgfqpoint{0.660477in}{2.730734in}}{\pgfqpoint{0.652577in}{2.727462in}}{\pgfqpoint{0.646753in}{2.721638in}}%
\pgfpathcurveto{\pgfqpoint{0.640929in}{2.715814in}}{\pgfqpoint{0.637657in}{2.707914in}}{\pgfqpoint{0.637657in}{2.699677in}}%
\pgfpathcurveto{\pgfqpoint{0.637657in}{2.691441in}}{\pgfqpoint{0.640929in}{2.683541in}}{\pgfqpoint{0.646753in}{2.677717in}}%
\pgfpathcurveto{\pgfqpoint{0.652577in}{2.671893in}}{\pgfqpoint{0.660477in}{2.668621in}}{\pgfqpoint{0.668713in}{2.668621in}}%
\pgfpathclose%
\pgfusepath{stroke,fill}%
\end{pgfscope}%
\begin{pgfscope}%
\pgfpathrectangle{\pgfqpoint{0.100000in}{0.212622in}}{\pgfqpoint{3.696000in}{3.696000in}}%
\pgfusepath{clip}%
\pgfsetbuttcap%
\pgfsetroundjoin%
\definecolor{currentfill}{rgb}{0.121569,0.466667,0.705882}%
\pgfsetfillcolor{currentfill}%
\pgfsetfillopacity{0.797022}%
\pgfsetlinewidth{1.003750pt}%
\definecolor{currentstroke}{rgb}{0.121569,0.466667,0.705882}%
\pgfsetstrokecolor{currentstroke}%
\pgfsetstrokeopacity{0.797022}%
\pgfsetdash{}{0pt}%
\pgfpathmoveto{\pgfqpoint{2.867300in}{2.062106in}}%
\pgfpathcurveto{\pgfqpoint{2.875537in}{2.062106in}}{\pgfqpoint{2.883437in}{2.065379in}}{\pgfqpoint{2.889261in}{2.071203in}}%
\pgfpathcurveto{\pgfqpoint{2.895085in}{2.077027in}}{\pgfqpoint{2.898357in}{2.084927in}}{\pgfqpoint{2.898357in}{2.093163in}}%
\pgfpathcurveto{\pgfqpoint{2.898357in}{2.101399in}}{\pgfqpoint{2.895085in}{2.109299in}}{\pgfqpoint{2.889261in}{2.115123in}}%
\pgfpathcurveto{\pgfqpoint{2.883437in}{2.120947in}}{\pgfqpoint{2.875537in}{2.124219in}}{\pgfqpoint{2.867300in}{2.124219in}}%
\pgfpathcurveto{\pgfqpoint{2.859064in}{2.124219in}}{\pgfqpoint{2.851164in}{2.120947in}}{\pgfqpoint{2.845340in}{2.115123in}}%
\pgfpathcurveto{\pgfqpoint{2.839516in}{2.109299in}}{\pgfqpoint{2.836244in}{2.101399in}}{\pgfqpoint{2.836244in}{2.093163in}}%
\pgfpathcurveto{\pgfqpoint{2.836244in}{2.084927in}}{\pgfqpoint{2.839516in}{2.077027in}}{\pgfqpoint{2.845340in}{2.071203in}}%
\pgfpathcurveto{\pgfqpoint{2.851164in}{2.065379in}}{\pgfqpoint{2.859064in}{2.062106in}}{\pgfqpoint{2.867300in}{2.062106in}}%
\pgfpathclose%
\pgfusepath{stroke,fill}%
\end{pgfscope}%
\begin{pgfscope}%
\pgfpathrectangle{\pgfqpoint{0.100000in}{0.212622in}}{\pgfqpoint{3.696000in}{3.696000in}}%
\pgfusepath{clip}%
\pgfsetbuttcap%
\pgfsetroundjoin%
\definecolor{currentfill}{rgb}{0.121569,0.466667,0.705882}%
\pgfsetfillcolor{currentfill}%
\pgfsetfillopacity{0.798137}%
\pgfsetlinewidth{1.003750pt}%
\definecolor{currentstroke}{rgb}{0.121569,0.466667,0.705882}%
\pgfsetstrokecolor{currentstroke}%
\pgfsetstrokeopacity{0.798137}%
\pgfsetdash{}{0pt}%
\pgfpathmoveto{\pgfqpoint{2.864104in}{2.061485in}}%
\pgfpathcurveto{\pgfqpoint{2.872340in}{2.061485in}}{\pgfqpoint{2.880240in}{2.064758in}}{\pgfqpoint{2.886064in}{2.070582in}}%
\pgfpathcurveto{\pgfqpoint{2.891888in}{2.076406in}}{\pgfqpoint{2.895161in}{2.084306in}}{\pgfqpoint{2.895161in}{2.092542in}}%
\pgfpathcurveto{\pgfqpoint{2.895161in}{2.100778in}}{\pgfqpoint{2.891888in}{2.108678in}}{\pgfqpoint{2.886064in}{2.114502in}}%
\pgfpathcurveto{\pgfqpoint{2.880240in}{2.120326in}}{\pgfqpoint{2.872340in}{2.123598in}}{\pgfqpoint{2.864104in}{2.123598in}}%
\pgfpathcurveto{\pgfqpoint{2.855868in}{2.123598in}}{\pgfqpoint{2.847968in}{2.120326in}}{\pgfqpoint{2.842144in}{2.114502in}}%
\pgfpathcurveto{\pgfqpoint{2.836320in}{2.108678in}}{\pgfqpoint{2.833048in}{2.100778in}}{\pgfqpoint{2.833048in}{2.092542in}}%
\pgfpathcurveto{\pgfqpoint{2.833048in}{2.084306in}}{\pgfqpoint{2.836320in}{2.076406in}}{\pgfqpoint{2.842144in}{2.070582in}}%
\pgfpathcurveto{\pgfqpoint{2.847968in}{2.064758in}}{\pgfqpoint{2.855868in}{2.061485in}}{\pgfqpoint{2.864104in}{2.061485in}}%
\pgfpathclose%
\pgfusepath{stroke,fill}%
\end{pgfscope}%
\begin{pgfscope}%
\pgfpathrectangle{\pgfqpoint{0.100000in}{0.212622in}}{\pgfqpoint{3.696000in}{3.696000in}}%
\pgfusepath{clip}%
\pgfsetbuttcap%
\pgfsetroundjoin%
\definecolor{currentfill}{rgb}{0.121569,0.466667,0.705882}%
\pgfsetfillcolor{currentfill}%
\pgfsetfillopacity{0.798758}%
\pgfsetlinewidth{1.003750pt}%
\definecolor{currentstroke}{rgb}{0.121569,0.466667,0.705882}%
\pgfsetstrokecolor{currentstroke}%
\pgfsetstrokeopacity{0.798758}%
\pgfsetdash{}{0pt}%
\pgfpathmoveto{\pgfqpoint{2.862894in}{2.060443in}}%
\pgfpathcurveto{\pgfqpoint{2.871130in}{2.060443in}}{\pgfqpoint{2.879030in}{2.063715in}}{\pgfqpoint{2.884854in}{2.069539in}}%
\pgfpathcurveto{\pgfqpoint{2.890678in}{2.075363in}}{\pgfqpoint{2.893950in}{2.083263in}}{\pgfqpoint{2.893950in}{2.091500in}}%
\pgfpathcurveto{\pgfqpoint{2.893950in}{2.099736in}}{\pgfqpoint{2.890678in}{2.107636in}}{\pgfqpoint{2.884854in}{2.113460in}}%
\pgfpathcurveto{\pgfqpoint{2.879030in}{2.119284in}}{\pgfqpoint{2.871130in}{2.122556in}}{\pgfqpoint{2.862894in}{2.122556in}}%
\pgfpathcurveto{\pgfqpoint{2.854657in}{2.122556in}}{\pgfqpoint{2.846757in}{2.119284in}}{\pgfqpoint{2.840933in}{2.113460in}}%
\pgfpathcurveto{\pgfqpoint{2.835110in}{2.107636in}}{\pgfqpoint{2.831837in}{2.099736in}}{\pgfqpoint{2.831837in}{2.091500in}}%
\pgfpathcurveto{\pgfqpoint{2.831837in}{2.083263in}}{\pgfqpoint{2.835110in}{2.075363in}}{\pgfqpoint{2.840933in}{2.069539in}}%
\pgfpathcurveto{\pgfqpoint{2.846757in}{2.063715in}}{\pgfqpoint{2.854657in}{2.060443in}}{\pgfqpoint{2.862894in}{2.060443in}}%
\pgfpathclose%
\pgfusepath{stroke,fill}%
\end{pgfscope}%
\begin{pgfscope}%
\pgfpathrectangle{\pgfqpoint{0.100000in}{0.212622in}}{\pgfqpoint{3.696000in}{3.696000in}}%
\pgfusepath{clip}%
\pgfsetbuttcap%
\pgfsetroundjoin%
\definecolor{currentfill}{rgb}{0.121569,0.466667,0.705882}%
\pgfsetfillcolor{currentfill}%
\pgfsetfillopacity{0.799101}%
\pgfsetlinewidth{1.003750pt}%
\definecolor{currentstroke}{rgb}{0.121569,0.466667,0.705882}%
\pgfsetstrokecolor{currentstroke}%
\pgfsetstrokeopacity{0.799101}%
\pgfsetdash{}{0pt}%
\pgfpathmoveto{\pgfqpoint{2.862064in}{2.060081in}}%
\pgfpathcurveto{\pgfqpoint{2.870300in}{2.060081in}}{\pgfqpoint{2.878200in}{2.063353in}}{\pgfqpoint{2.884024in}{2.069177in}}%
\pgfpathcurveto{\pgfqpoint{2.889848in}{2.075001in}}{\pgfqpoint{2.893120in}{2.082901in}}{\pgfqpoint{2.893120in}{2.091137in}}%
\pgfpathcurveto{\pgfqpoint{2.893120in}{2.099374in}}{\pgfqpoint{2.889848in}{2.107274in}}{\pgfqpoint{2.884024in}{2.113098in}}%
\pgfpathcurveto{\pgfqpoint{2.878200in}{2.118922in}}{\pgfqpoint{2.870300in}{2.122194in}}{\pgfqpoint{2.862064in}{2.122194in}}%
\pgfpathcurveto{\pgfqpoint{2.853828in}{2.122194in}}{\pgfqpoint{2.845928in}{2.118922in}}{\pgfqpoint{2.840104in}{2.113098in}}%
\pgfpathcurveto{\pgfqpoint{2.834280in}{2.107274in}}{\pgfqpoint{2.831007in}{2.099374in}}{\pgfqpoint{2.831007in}{2.091137in}}%
\pgfpathcurveto{\pgfqpoint{2.831007in}{2.082901in}}{\pgfqpoint{2.834280in}{2.075001in}}{\pgfqpoint{2.840104in}{2.069177in}}%
\pgfpathcurveto{\pgfqpoint{2.845928in}{2.063353in}}{\pgfqpoint{2.853828in}{2.060081in}}{\pgfqpoint{2.862064in}{2.060081in}}%
\pgfpathclose%
\pgfusepath{stroke,fill}%
\end{pgfscope}%
\begin{pgfscope}%
\pgfpathrectangle{\pgfqpoint{0.100000in}{0.212622in}}{\pgfqpoint{3.696000in}{3.696000in}}%
\pgfusepath{clip}%
\pgfsetbuttcap%
\pgfsetroundjoin%
\definecolor{currentfill}{rgb}{0.121569,0.466667,0.705882}%
\pgfsetfillcolor{currentfill}%
\pgfsetfillopacity{0.800324}%
\pgfsetlinewidth{1.003750pt}%
\definecolor{currentstroke}{rgb}{0.121569,0.466667,0.705882}%
\pgfsetstrokecolor{currentstroke}%
\pgfsetstrokeopacity{0.800324}%
\pgfsetdash{}{0pt}%
\pgfpathmoveto{\pgfqpoint{2.859849in}{2.057680in}}%
\pgfpathcurveto{\pgfqpoint{2.868086in}{2.057680in}}{\pgfqpoint{2.875986in}{2.060952in}}{\pgfqpoint{2.881810in}{2.066776in}}%
\pgfpathcurveto{\pgfqpoint{2.887634in}{2.072600in}}{\pgfqpoint{2.890906in}{2.080500in}}{\pgfqpoint{2.890906in}{2.088737in}}%
\pgfpathcurveto{\pgfqpoint{2.890906in}{2.096973in}}{\pgfqpoint{2.887634in}{2.104873in}}{\pgfqpoint{2.881810in}{2.110697in}}%
\pgfpathcurveto{\pgfqpoint{2.875986in}{2.116521in}}{\pgfqpoint{2.868086in}{2.119793in}}{\pgfqpoint{2.859849in}{2.119793in}}%
\pgfpathcurveto{\pgfqpoint{2.851613in}{2.119793in}}{\pgfqpoint{2.843713in}{2.116521in}}{\pgfqpoint{2.837889in}{2.110697in}}%
\pgfpathcurveto{\pgfqpoint{2.832065in}{2.104873in}}{\pgfqpoint{2.828793in}{2.096973in}}{\pgfqpoint{2.828793in}{2.088737in}}%
\pgfpathcurveto{\pgfqpoint{2.828793in}{2.080500in}}{\pgfqpoint{2.832065in}{2.072600in}}{\pgfqpoint{2.837889in}{2.066776in}}%
\pgfpathcurveto{\pgfqpoint{2.843713in}{2.060952in}}{\pgfqpoint{2.851613in}{2.057680in}}{\pgfqpoint{2.859849in}{2.057680in}}%
\pgfpathclose%
\pgfusepath{stroke,fill}%
\end{pgfscope}%
\begin{pgfscope}%
\pgfpathrectangle{\pgfqpoint{0.100000in}{0.212622in}}{\pgfqpoint{3.696000in}{3.696000in}}%
\pgfusepath{clip}%
\pgfsetbuttcap%
\pgfsetroundjoin%
\definecolor{currentfill}{rgb}{0.121569,0.466667,0.705882}%
\pgfsetfillcolor{currentfill}%
\pgfsetfillopacity{0.801364}%
\pgfsetlinewidth{1.003750pt}%
\definecolor{currentstroke}{rgb}{0.121569,0.466667,0.705882}%
\pgfsetstrokecolor{currentstroke}%
\pgfsetstrokeopacity{0.801364}%
\pgfsetdash{}{0pt}%
\pgfpathmoveto{\pgfqpoint{0.688958in}{2.672172in}}%
\pgfpathcurveto{\pgfqpoint{0.697195in}{2.672172in}}{\pgfqpoint{0.705095in}{2.675444in}}{\pgfqpoint{0.710919in}{2.681268in}}%
\pgfpathcurveto{\pgfqpoint{0.716743in}{2.687092in}}{\pgfqpoint{0.720015in}{2.694992in}}{\pgfqpoint{0.720015in}{2.703228in}}%
\pgfpathcurveto{\pgfqpoint{0.720015in}{2.711465in}}{\pgfqpoint{0.716743in}{2.719365in}}{\pgfqpoint{0.710919in}{2.725189in}}%
\pgfpathcurveto{\pgfqpoint{0.705095in}{2.731013in}}{\pgfqpoint{0.697195in}{2.734285in}}{\pgfqpoint{0.688958in}{2.734285in}}%
\pgfpathcurveto{\pgfqpoint{0.680722in}{2.734285in}}{\pgfqpoint{0.672822in}{2.731013in}}{\pgfqpoint{0.666998in}{2.725189in}}%
\pgfpathcurveto{\pgfqpoint{0.661174in}{2.719365in}}{\pgfqpoint{0.657902in}{2.711465in}}{\pgfqpoint{0.657902in}{2.703228in}}%
\pgfpathcurveto{\pgfqpoint{0.657902in}{2.694992in}}{\pgfqpoint{0.661174in}{2.687092in}}{\pgfqpoint{0.666998in}{2.681268in}}%
\pgfpathcurveto{\pgfqpoint{0.672822in}{2.675444in}}{\pgfqpoint{0.680722in}{2.672172in}}{\pgfqpoint{0.688958in}{2.672172in}}%
\pgfpathclose%
\pgfusepath{stroke,fill}%
\end{pgfscope}%
\begin{pgfscope}%
\pgfpathrectangle{\pgfqpoint{0.100000in}{0.212622in}}{\pgfqpoint{3.696000in}{3.696000in}}%
\pgfusepath{clip}%
\pgfsetbuttcap%
\pgfsetroundjoin%
\definecolor{currentfill}{rgb}{0.121569,0.466667,0.705882}%
\pgfsetfillcolor{currentfill}%
\pgfsetfillopacity{0.802205}%
\pgfsetlinewidth{1.003750pt}%
\definecolor{currentstroke}{rgb}{0.121569,0.466667,0.705882}%
\pgfsetstrokecolor{currentstroke}%
\pgfsetstrokeopacity{0.802205}%
\pgfsetdash{}{0pt}%
\pgfpathmoveto{\pgfqpoint{2.855005in}{2.055684in}}%
\pgfpathcurveto{\pgfqpoint{2.863242in}{2.055684in}}{\pgfqpoint{2.871142in}{2.058957in}}{\pgfqpoint{2.876966in}{2.064780in}}%
\pgfpathcurveto{\pgfqpoint{2.882790in}{2.070604in}}{\pgfqpoint{2.886062in}{2.078504in}}{\pgfqpoint{2.886062in}{2.086741in}}%
\pgfpathcurveto{\pgfqpoint{2.886062in}{2.094977in}}{\pgfqpoint{2.882790in}{2.102877in}}{\pgfqpoint{2.876966in}{2.108701in}}%
\pgfpathcurveto{\pgfqpoint{2.871142in}{2.114525in}}{\pgfqpoint{2.863242in}{2.117797in}}{\pgfqpoint{2.855005in}{2.117797in}}%
\pgfpathcurveto{\pgfqpoint{2.846769in}{2.117797in}}{\pgfqpoint{2.838869in}{2.114525in}}{\pgfqpoint{2.833045in}{2.108701in}}%
\pgfpathcurveto{\pgfqpoint{2.827221in}{2.102877in}}{\pgfqpoint{2.823949in}{2.094977in}}{\pgfqpoint{2.823949in}{2.086741in}}%
\pgfpathcurveto{\pgfqpoint{2.823949in}{2.078504in}}{\pgfqpoint{2.827221in}{2.070604in}}{\pgfqpoint{2.833045in}{2.064780in}}%
\pgfpathcurveto{\pgfqpoint{2.838869in}{2.058957in}}{\pgfqpoint{2.846769in}{2.055684in}}{\pgfqpoint{2.855005in}{2.055684in}}%
\pgfpathclose%
\pgfusepath{stroke,fill}%
\end{pgfscope}%
\begin{pgfscope}%
\pgfpathrectangle{\pgfqpoint{0.100000in}{0.212622in}}{\pgfqpoint{3.696000in}{3.696000in}}%
\pgfusepath{clip}%
\pgfsetbuttcap%
\pgfsetroundjoin%
\definecolor{currentfill}{rgb}{0.121569,0.466667,0.705882}%
\pgfsetfillcolor{currentfill}%
\pgfsetfillopacity{0.802994}%
\pgfsetlinewidth{1.003750pt}%
\definecolor{currentstroke}{rgb}{0.121569,0.466667,0.705882}%
\pgfsetstrokecolor{currentstroke}%
\pgfsetstrokeopacity{0.802994}%
\pgfsetdash{}{0pt}%
\pgfpathmoveto{\pgfqpoint{0.724919in}{2.660142in}}%
\pgfpathcurveto{\pgfqpoint{0.733156in}{2.660142in}}{\pgfqpoint{0.741056in}{2.663415in}}{\pgfqpoint{0.746880in}{2.669239in}}%
\pgfpathcurveto{\pgfqpoint{0.752703in}{2.675063in}}{\pgfqpoint{0.755976in}{2.682963in}}{\pgfqpoint{0.755976in}{2.691199in}}%
\pgfpathcurveto{\pgfqpoint{0.755976in}{2.699435in}}{\pgfqpoint{0.752703in}{2.707335in}}{\pgfqpoint{0.746880in}{2.713159in}}%
\pgfpathcurveto{\pgfqpoint{0.741056in}{2.718983in}}{\pgfqpoint{0.733156in}{2.722255in}}{\pgfqpoint{0.724919in}{2.722255in}}%
\pgfpathcurveto{\pgfqpoint{0.716683in}{2.722255in}}{\pgfqpoint{0.708783in}{2.718983in}}{\pgfqpoint{0.702959in}{2.713159in}}%
\pgfpathcurveto{\pgfqpoint{0.697135in}{2.707335in}}{\pgfqpoint{0.693863in}{2.699435in}}{\pgfqpoint{0.693863in}{2.691199in}}%
\pgfpathcurveto{\pgfqpoint{0.693863in}{2.682963in}}{\pgfqpoint{0.697135in}{2.675063in}}{\pgfqpoint{0.702959in}{2.669239in}}%
\pgfpathcurveto{\pgfqpoint{0.708783in}{2.663415in}}{\pgfqpoint{0.716683in}{2.660142in}}{\pgfqpoint{0.724919in}{2.660142in}}%
\pgfpathclose%
\pgfusepath{stroke,fill}%
\end{pgfscope}%
\begin{pgfscope}%
\pgfpathrectangle{\pgfqpoint{0.100000in}{0.212622in}}{\pgfqpoint{3.696000in}{3.696000in}}%
\pgfusepath{clip}%
\pgfsetbuttcap%
\pgfsetroundjoin%
\definecolor{currentfill}{rgb}{0.121569,0.466667,0.705882}%
\pgfsetfillcolor{currentfill}%
\pgfsetfillopacity{0.805230}%
\pgfsetlinewidth{1.003750pt}%
\definecolor{currentstroke}{rgb}{0.121569,0.466667,0.705882}%
\pgfsetstrokecolor{currentstroke}%
\pgfsetstrokeopacity{0.805230}%
\pgfsetdash{}{0pt}%
\pgfpathmoveto{\pgfqpoint{0.705888in}{2.675141in}}%
\pgfpathcurveto{\pgfqpoint{0.714124in}{2.675141in}}{\pgfqpoint{0.722024in}{2.678414in}}{\pgfqpoint{0.727848in}{2.684238in}}%
\pgfpathcurveto{\pgfqpoint{0.733672in}{2.690062in}}{\pgfqpoint{0.736945in}{2.697962in}}{\pgfqpoint{0.736945in}{2.706198in}}%
\pgfpathcurveto{\pgfqpoint{0.736945in}{2.714434in}}{\pgfqpoint{0.733672in}{2.722334in}}{\pgfqpoint{0.727848in}{2.728158in}}%
\pgfpathcurveto{\pgfqpoint{0.722024in}{2.733982in}}{\pgfqpoint{0.714124in}{2.737254in}}{\pgfqpoint{0.705888in}{2.737254in}}%
\pgfpathcurveto{\pgfqpoint{0.697652in}{2.737254in}}{\pgfqpoint{0.689752in}{2.733982in}}{\pgfqpoint{0.683928in}{2.728158in}}%
\pgfpathcurveto{\pgfqpoint{0.678104in}{2.722334in}}{\pgfqpoint{0.674832in}{2.714434in}}{\pgfqpoint{0.674832in}{2.706198in}}%
\pgfpathcurveto{\pgfqpoint{0.674832in}{2.697962in}}{\pgfqpoint{0.678104in}{2.690062in}}{\pgfqpoint{0.683928in}{2.684238in}}%
\pgfpathcurveto{\pgfqpoint{0.689752in}{2.678414in}}{\pgfqpoint{0.697652in}{2.675141in}}{\pgfqpoint{0.705888in}{2.675141in}}%
\pgfpathclose%
\pgfusepath{stroke,fill}%
\end{pgfscope}%
\begin{pgfscope}%
\pgfpathrectangle{\pgfqpoint{0.100000in}{0.212622in}}{\pgfqpoint{3.696000in}{3.696000in}}%
\pgfusepath{clip}%
\pgfsetbuttcap%
\pgfsetroundjoin%
\definecolor{currentfill}{rgb}{0.121569,0.466667,0.705882}%
\pgfsetfillcolor{currentfill}%
\pgfsetfillopacity{0.805267}%
\pgfsetlinewidth{1.003750pt}%
\definecolor{currentstroke}{rgb}{0.121569,0.466667,0.705882}%
\pgfsetstrokecolor{currentstroke}%
\pgfsetstrokeopacity{0.805267}%
\pgfsetdash{}{0pt}%
\pgfpathmoveto{\pgfqpoint{0.751516in}{2.653391in}}%
\pgfpathcurveto{\pgfqpoint{0.759753in}{2.653391in}}{\pgfqpoint{0.767653in}{2.656663in}}{\pgfqpoint{0.773477in}{2.662487in}}%
\pgfpathcurveto{\pgfqpoint{0.779301in}{2.668311in}}{\pgfqpoint{0.782573in}{2.676211in}}{\pgfqpoint{0.782573in}{2.684448in}}%
\pgfpathcurveto{\pgfqpoint{0.782573in}{2.692684in}}{\pgfqpoint{0.779301in}{2.700584in}}{\pgfqpoint{0.773477in}{2.706408in}}%
\pgfpathcurveto{\pgfqpoint{0.767653in}{2.712232in}}{\pgfqpoint{0.759753in}{2.715504in}}{\pgfqpoint{0.751516in}{2.715504in}}%
\pgfpathcurveto{\pgfqpoint{0.743280in}{2.715504in}}{\pgfqpoint{0.735380in}{2.712232in}}{\pgfqpoint{0.729556in}{2.706408in}}%
\pgfpathcurveto{\pgfqpoint{0.723732in}{2.700584in}}{\pgfqpoint{0.720460in}{2.692684in}}{\pgfqpoint{0.720460in}{2.684448in}}%
\pgfpathcurveto{\pgfqpoint{0.720460in}{2.676211in}}{\pgfqpoint{0.723732in}{2.668311in}}{\pgfqpoint{0.729556in}{2.662487in}}%
\pgfpathcurveto{\pgfqpoint{0.735380in}{2.656663in}}{\pgfqpoint{0.743280in}{2.653391in}}{\pgfqpoint{0.751516in}{2.653391in}}%
\pgfpathclose%
\pgfusepath{stroke,fill}%
\end{pgfscope}%
\begin{pgfscope}%
\pgfpathrectangle{\pgfqpoint{0.100000in}{0.212622in}}{\pgfqpoint{3.696000in}{3.696000in}}%
\pgfusepath{clip}%
\pgfsetbuttcap%
\pgfsetroundjoin%
\definecolor{currentfill}{rgb}{0.121569,0.466667,0.705882}%
\pgfsetfillcolor{currentfill}%
\pgfsetfillopacity{0.805489}%
\pgfsetlinewidth{1.003750pt}%
\definecolor{currentstroke}{rgb}{0.121569,0.466667,0.705882}%
\pgfsetstrokecolor{currentstroke}%
\pgfsetstrokeopacity{0.805489}%
\pgfsetdash{}{0pt}%
\pgfpathmoveto{\pgfqpoint{2.849418in}{2.051764in}}%
\pgfpathcurveto{\pgfqpoint{2.857654in}{2.051764in}}{\pgfqpoint{2.865554in}{2.055036in}}{\pgfqpoint{2.871378in}{2.060860in}}%
\pgfpathcurveto{\pgfqpoint{2.877202in}{2.066684in}}{\pgfqpoint{2.880475in}{2.074584in}}{\pgfqpoint{2.880475in}{2.082820in}}%
\pgfpathcurveto{\pgfqpoint{2.880475in}{2.091057in}}{\pgfqpoint{2.877202in}{2.098957in}}{\pgfqpoint{2.871378in}{2.104781in}}%
\pgfpathcurveto{\pgfqpoint{2.865554in}{2.110605in}}{\pgfqpoint{2.857654in}{2.113877in}}{\pgfqpoint{2.849418in}{2.113877in}}%
\pgfpathcurveto{\pgfqpoint{2.841182in}{2.113877in}}{\pgfqpoint{2.833282in}{2.110605in}}{\pgfqpoint{2.827458in}{2.104781in}}%
\pgfpathcurveto{\pgfqpoint{2.821634in}{2.098957in}}{\pgfqpoint{2.818362in}{2.091057in}}{\pgfqpoint{2.818362in}{2.082820in}}%
\pgfpathcurveto{\pgfqpoint{2.818362in}{2.074584in}}{\pgfqpoint{2.821634in}{2.066684in}}{\pgfqpoint{2.827458in}{2.060860in}}%
\pgfpathcurveto{\pgfqpoint{2.833282in}{2.055036in}}{\pgfqpoint{2.841182in}{2.051764in}}{\pgfqpoint{2.849418in}{2.051764in}}%
\pgfpathclose%
\pgfusepath{stroke,fill}%
\end{pgfscope}%
\begin{pgfscope}%
\pgfpathrectangle{\pgfqpoint{0.100000in}{0.212622in}}{\pgfqpoint{3.696000in}{3.696000in}}%
\pgfusepath{clip}%
\pgfsetbuttcap%
\pgfsetroundjoin%
\definecolor{currentfill}{rgb}{0.121569,0.466667,0.705882}%
\pgfsetfillcolor{currentfill}%
\pgfsetfillopacity{0.805758}%
\pgfsetlinewidth{1.003750pt}%
\definecolor{currentstroke}{rgb}{0.121569,0.466667,0.705882}%
\pgfsetstrokecolor{currentstroke}%
\pgfsetstrokeopacity{0.805758}%
\pgfsetdash{}{0pt}%
\pgfpathmoveto{\pgfqpoint{0.737995in}{2.661962in}}%
\pgfpathcurveto{\pgfqpoint{0.746232in}{2.661962in}}{\pgfqpoint{0.754132in}{2.665235in}}{\pgfqpoint{0.759956in}{2.671059in}}%
\pgfpathcurveto{\pgfqpoint{0.765780in}{2.676883in}}{\pgfqpoint{0.769052in}{2.684783in}}{\pgfqpoint{0.769052in}{2.693019in}}%
\pgfpathcurveto{\pgfqpoint{0.769052in}{2.701255in}}{\pgfqpoint{0.765780in}{2.709155in}}{\pgfqpoint{0.759956in}{2.714979in}}%
\pgfpathcurveto{\pgfqpoint{0.754132in}{2.720803in}}{\pgfqpoint{0.746232in}{2.724075in}}{\pgfqpoint{0.737995in}{2.724075in}}%
\pgfpathcurveto{\pgfqpoint{0.729759in}{2.724075in}}{\pgfqpoint{0.721859in}{2.720803in}}{\pgfqpoint{0.716035in}{2.714979in}}%
\pgfpathcurveto{\pgfqpoint{0.710211in}{2.709155in}}{\pgfqpoint{0.706939in}{2.701255in}}{\pgfqpoint{0.706939in}{2.693019in}}%
\pgfpathcurveto{\pgfqpoint{0.706939in}{2.684783in}}{\pgfqpoint{0.710211in}{2.676883in}}{\pgfqpoint{0.716035in}{2.671059in}}%
\pgfpathcurveto{\pgfqpoint{0.721859in}{2.665235in}}{\pgfqpoint{0.729759in}{2.661962in}}{\pgfqpoint{0.737995in}{2.661962in}}%
\pgfpathclose%
\pgfusepath{stroke,fill}%
\end{pgfscope}%
\begin{pgfscope}%
\pgfpathrectangle{\pgfqpoint{0.100000in}{0.212622in}}{\pgfqpoint{3.696000in}{3.696000in}}%
\pgfusepath{clip}%
\pgfsetbuttcap%
\pgfsetroundjoin%
\definecolor{currentfill}{rgb}{0.121569,0.466667,0.705882}%
\pgfsetfillcolor{currentfill}%
\pgfsetfillopacity{0.807139}%
\pgfsetlinewidth{1.003750pt}%
\definecolor{currentstroke}{rgb}{0.121569,0.466667,0.705882}%
\pgfsetstrokecolor{currentstroke}%
\pgfsetstrokeopacity{0.807139}%
\pgfsetdash{}{0pt}%
\pgfpathmoveto{\pgfqpoint{0.760543in}{2.656476in}}%
\pgfpathcurveto{\pgfqpoint{0.768780in}{2.656476in}}{\pgfqpoint{0.776680in}{2.659748in}}{\pgfqpoint{0.782504in}{2.665572in}}%
\pgfpathcurveto{\pgfqpoint{0.788328in}{2.671396in}}{\pgfqpoint{0.791600in}{2.679296in}}{\pgfqpoint{0.791600in}{2.687532in}}%
\pgfpathcurveto{\pgfqpoint{0.791600in}{2.695769in}}{\pgfqpoint{0.788328in}{2.703669in}}{\pgfqpoint{0.782504in}{2.709493in}}%
\pgfpathcurveto{\pgfqpoint{0.776680in}{2.715317in}}{\pgfqpoint{0.768780in}{2.718589in}}{\pgfqpoint{0.760543in}{2.718589in}}%
\pgfpathcurveto{\pgfqpoint{0.752307in}{2.718589in}}{\pgfqpoint{0.744407in}{2.715317in}}{\pgfqpoint{0.738583in}{2.709493in}}%
\pgfpathcurveto{\pgfqpoint{0.732759in}{2.703669in}}{\pgfqpoint{0.729487in}{2.695769in}}{\pgfqpoint{0.729487in}{2.687532in}}%
\pgfpathcurveto{\pgfqpoint{0.729487in}{2.679296in}}{\pgfqpoint{0.732759in}{2.671396in}}{\pgfqpoint{0.738583in}{2.665572in}}%
\pgfpathcurveto{\pgfqpoint{0.744407in}{2.659748in}}{\pgfqpoint{0.752307in}{2.656476in}}{\pgfqpoint{0.760543in}{2.656476in}}%
\pgfpathclose%
\pgfusepath{stroke,fill}%
\end{pgfscope}%
\begin{pgfscope}%
\pgfpathrectangle{\pgfqpoint{0.100000in}{0.212622in}}{\pgfqpoint{3.696000in}{3.696000in}}%
\pgfusepath{clip}%
\pgfsetbuttcap%
\pgfsetroundjoin%
\definecolor{currentfill}{rgb}{0.121569,0.466667,0.705882}%
\pgfsetfillcolor{currentfill}%
\pgfsetfillopacity{0.808597}%
\pgfsetlinewidth{1.003750pt}%
\definecolor{currentstroke}{rgb}{0.121569,0.466667,0.705882}%
\pgfsetstrokecolor{currentstroke}%
\pgfsetstrokeopacity{0.808597}%
\pgfsetdash{}{0pt}%
\pgfpathmoveto{\pgfqpoint{0.776362in}{2.645692in}}%
\pgfpathcurveto{\pgfqpoint{0.784599in}{2.645692in}}{\pgfqpoint{0.792499in}{2.648964in}}{\pgfqpoint{0.798323in}{2.654788in}}%
\pgfpathcurveto{\pgfqpoint{0.804147in}{2.660612in}}{\pgfqpoint{0.807419in}{2.668512in}}{\pgfqpoint{0.807419in}{2.676748in}}%
\pgfpathcurveto{\pgfqpoint{0.807419in}{2.684984in}}{\pgfqpoint{0.804147in}{2.692884in}}{\pgfqpoint{0.798323in}{2.698708in}}%
\pgfpathcurveto{\pgfqpoint{0.792499in}{2.704532in}}{\pgfqpoint{0.784599in}{2.707805in}}{\pgfqpoint{0.776362in}{2.707805in}}%
\pgfpathcurveto{\pgfqpoint{0.768126in}{2.707805in}}{\pgfqpoint{0.760226in}{2.704532in}}{\pgfqpoint{0.754402in}{2.698708in}}%
\pgfpathcurveto{\pgfqpoint{0.748578in}{2.692884in}}{\pgfqpoint{0.745306in}{2.684984in}}{\pgfqpoint{0.745306in}{2.676748in}}%
\pgfpathcurveto{\pgfqpoint{0.745306in}{2.668512in}}{\pgfqpoint{0.748578in}{2.660612in}}{\pgfqpoint{0.754402in}{2.654788in}}%
\pgfpathcurveto{\pgfqpoint{0.760226in}{2.648964in}}{\pgfqpoint{0.768126in}{2.645692in}}{\pgfqpoint{0.776362in}{2.645692in}}%
\pgfpathclose%
\pgfusepath{stroke,fill}%
\end{pgfscope}%
\begin{pgfscope}%
\pgfpathrectangle{\pgfqpoint{0.100000in}{0.212622in}}{\pgfqpoint{3.696000in}{3.696000in}}%
\pgfusepath{clip}%
\pgfsetbuttcap%
\pgfsetroundjoin%
\definecolor{currentfill}{rgb}{0.121569,0.466667,0.705882}%
\pgfsetfillcolor{currentfill}%
\pgfsetfillopacity{0.809006}%
\pgfsetlinewidth{1.003750pt}%
\definecolor{currentstroke}{rgb}{0.121569,0.466667,0.705882}%
\pgfsetstrokecolor{currentstroke}%
\pgfsetstrokeopacity{0.809006}%
\pgfsetdash{}{0pt}%
\pgfpathmoveto{\pgfqpoint{2.839405in}{2.047949in}}%
\pgfpathcurveto{\pgfqpoint{2.847641in}{2.047949in}}{\pgfqpoint{2.855541in}{2.051222in}}{\pgfqpoint{2.861365in}{2.057046in}}%
\pgfpathcurveto{\pgfqpoint{2.867189in}{2.062870in}}{\pgfqpoint{2.870461in}{2.070770in}}{\pgfqpoint{2.870461in}{2.079006in}}%
\pgfpathcurveto{\pgfqpoint{2.870461in}{2.087242in}}{\pgfqpoint{2.867189in}{2.095142in}}{\pgfqpoint{2.861365in}{2.100966in}}%
\pgfpathcurveto{\pgfqpoint{2.855541in}{2.106790in}}{\pgfqpoint{2.847641in}{2.110062in}}{\pgfqpoint{2.839405in}{2.110062in}}%
\pgfpathcurveto{\pgfqpoint{2.831169in}{2.110062in}}{\pgfqpoint{2.823269in}{2.106790in}}{\pgfqpoint{2.817445in}{2.100966in}}%
\pgfpathcurveto{\pgfqpoint{2.811621in}{2.095142in}}{\pgfqpoint{2.808348in}{2.087242in}}{\pgfqpoint{2.808348in}{2.079006in}}%
\pgfpathcurveto{\pgfqpoint{2.808348in}{2.070770in}}{\pgfqpoint{2.811621in}{2.062870in}}{\pgfqpoint{2.817445in}{2.057046in}}%
\pgfpathcurveto{\pgfqpoint{2.823269in}{2.051222in}}{\pgfqpoint{2.831169in}{2.047949in}}{\pgfqpoint{2.839405in}{2.047949in}}%
\pgfpathclose%
\pgfusepath{stroke,fill}%
\end{pgfscope}%
\begin{pgfscope}%
\pgfpathrectangle{\pgfqpoint{0.100000in}{0.212622in}}{\pgfqpoint{3.696000in}{3.696000in}}%
\pgfusepath{clip}%
\pgfsetbuttcap%
\pgfsetroundjoin%
\definecolor{currentfill}{rgb}{0.121569,0.466667,0.705882}%
\pgfsetfillcolor{currentfill}%
\pgfsetfillopacity{0.809310}%
\pgfsetlinewidth{1.003750pt}%
\definecolor{currentstroke}{rgb}{0.121569,0.466667,0.705882}%
\pgfsetstrokecolor{currentstroke}%
\pgfsetstrokeopacity{0.809310}%
\pgfsetdash{}{0pt}%
\pgfpathmoveto{\pgfqpoint{0.786008in}{2.639378in}}%
\pgfpathcurveto{\pgfqpoint{0.794244in}{2.639378in}}{\pgfqpoint{0.802144in}{2.642651in}}{\pgfqpoint{0.807968in}{2.648475in}}%
\pgfpathcurveto{\pgfqpoint{0.813792in}{2.654299in}}{\pgfqpoint{0.817064in}{2.662199in}}{\pgfqpoint{0.817064in}{2.670435in}}%
\pgfpathcurveto{\pgfqpoint{0.817064in}{2.678671in}}{\pgfqpoint{0.813792in}{2.686571in}}{\pgfqpoint{0.807968in}{2.692395in}}%
\pgfpathcurveto{\pgfqpoint{0.802144in}{2.698219in}}{\pgfqpoint{0.794244in}{2.701491in}}{\pgfqpoint{0.786008in}{2.701491in}}%
\pgfpathcurveto{\pgfqpoint{0.777772in}{2.701491in}}{\pgfqpoint{0.769872in}{2.698219in}}{\pgfqpoint{0.764048in}{2.692395in}}%
\pgfpathcurveto{\pgfqpoint{0.758224in}{2.686571in}}{\pgfqpoint{0.754951in}{2.678671in}}{\pgfqpoint{0.754951in}{2.670435in}}%
\pgfpathcurveto{\pgfqpoint{0.754951in}{2.662199in}}{\pgfqpoint{0.758224in}{2.654299in}}{\pgfqpoint{0.764048in}{2.648475in}}%
\pgfpathcurveto{\pgfqpoint{0.769872in}{2.642651in}}{\pgfqpoint{0.777772in}{2.639378in}}{\pgfqpoint{0.786008in}{2.639378in}}%
\pgfpathclose%
\pgfusepath{stroke,fill}%
\end{pgfscope}%
\begin{pgfscope}%
\pgfpathrectangle{\pgfqpoint{0.100000in}{0.212622in}}{\pgfqpoint{3.696000in}{3.696000in}}%
\pgfusepath{clip}%
\pgfsetbuttcap%
\pgfsetroundjoin%
\definecolor{currentfill}{rgb}{0.121569,0.466667,0.705882}%
\pgfsetfillcolor{currentfill}%
\pgfsetfillopacity{0.809760}%
\pgfsetlinewidth{1.003750pt}%
\definecolor{currentstroke}{rgb}{0.121569,0.466667,0.705882}%
\pgfsetstrokecolor{currentstroke}%
\pgfsetstrokeopacity{0.809760}%
\pgfsetdash{}{0pt}%
\pgfpathmoveto{\pgfqpoint{0.790396in}{2.636968in}}%
\pgfpathcurveto{\pgfqpoint{0.798633in}{2.636968in}}{\pgfqpoint{0.806533in}{2.640240in}}{\pgfqpoint{0.812357in}{2.646064in}}%
\pgfpathcurveto{\pgfqpoint{0.818180in}{2.651888in}}{\pgfqpoint{0.821453in}{2.659788in}}{\pgfqpoint{0.821453in}{2.668025in}}%
\pgfpathcurveto{\pgfqpoint{0.821453in}{2.676261in}}{\pgfqpoint{0.818180in}{2.684161in}}{\pgfqpoint{0.812357in}{2.689985in}}%
\pgfpathcurveto{\pgfqpoint{0.806533in}{2.695809in}}{\pgfqpoint{0.798633in}{2.699081in}}{\pgfqpoint{0.790396in}{2.699081in}}%
\pgfpathcurveto{\pgfqpoint{0.782160in}{2.699081in}}{\pgfqpoint{0.774260in}{2.695809in}}{\pgfqpoint{0.768436in}{2.689985in}}%
\pgfpathcurveto{\pgfqpoint{0.762612in}{2.684161in}}{\pgfqpoint{0.759340in}{2.676261in}}{\pgfqpoint{0.759340in}{2.668025in}}%
\pgfpathcurveto{\pgfqpoint{0.759340in}{2.659788in}}{\pgfqpoint{0.762612in}{2.651888in}}{\pgfqpoint{0.768436in}{2.646064in}}%
\pgfpathcurveto{\pgfqpoint{0.774260in}{2.640240in}}{\pgfqpoint{0.782160in}{2.636968in}}{\pgfqpoint{0.790396in}{2.636968in}}%
\pgfpathclose%
\pgfusepath{stroke,fill}%
\end{pgfscope}%
\begin{pgfscope}%
\pgfpathrectangle{\pgfqpoint{0.100000in}{0.212622in}}{\pgfqpoint{3.696000in}{3.696000in}}%
\pgfusepath{clip}%
\pgfsetbuttcap%
\pgfsetroundjoin%
\definecolor{currentfill}{rgb}{0.121569,0.466667,0.705882}%
\pgfsetfillcolor{currentfill}%
\pgfsetfillopacity{0.810935}%
\pgfsetlinewidth{1.003750pt}%
\definecolor{currentstroke}{rgb}{0.121569,0.466667,0.705882}%
\pgfsetstrokecolor{currentstroke}%
\pgfsetstrokeopacity{0.810935}%
\pgfsetdash{}{0pt}%
\pgfpathmoveto{\pgfqpoint{0.796931in}{2.631222in}}%
\pgfpathcurveto{\pgfqpoint{0.805168in}{2.631222in}}{\pgfqpoint{0.813068in}{2.634494in}}{\pgfqpoint{0.818892in}{2.640318in}}%
\pgfpathcurveto{\pgfqpoint{0.824716in}{2.646142in}}{\pgfqpoint{0.827988in}{2.654042in}}{\pgfqpoint{0.827988in}{2.662279in}}%
\pgfpathcurveto{\pgfqpoint{0.827988in}{2.670515in}}{\pgfqpoint{0.824716in}{2.678415in}}{\pgfqpoint{0.818892in}{2.684239in}}%
\pgfpathcurveto{\pgfqpoint{0.813068in}{2.690063in}}{\pgfqpoint{0.805168in}{2.693335in}}{\pgfqpoint{0.796931in}{2.693335in}}%
\pgfpathcurveto{\pgfqpoint{0.788695in}{2.693335in}}{\pgfqpoint{0.780795in}{2.690063in}}{\pgfqpoint{0.774971in}{2.684239in}}%
\pgfpathcurveto{\pgfqpoint{0.769147in}{2.678415in}}{\pgfqpoint{0.765875in}{2.670515in}}{\pgfqpoint{0.765875in}{2.662279in}}%
\pgfpathcurveto{\pgfqpoint{0.765875in}{2.654042in}}{\pgfqpoint{0.769147in}{2.646142in}}{\pgfqpoint{0.774971in}{2.640318in}}%
\pgfpathcurveto{\pgfqpoint{0.780795in}{2.634494in}}{\pgfqpoint{0.788695in}{2.631222in}}{\pgfqpoint{0.796931in}{2.631222in}}%
\pgfpathclose%
\pgfusepath{stroke,fill}%
\end{pgfscope}%
\begin{pgfscope}%
\pgfpathrectangle{\pgfqpoint{0.100000in}{0.212622in}}{\pgfqpoint{3.696000in}{3.696000in}}%
\pgfusepath{clip}%
\pgfsetbuttcap%
\pgfsetroundjoin%
\definecolor{currentfill}{rgb}{0.121569,0.466667,0.705882}%
\pgfsetfillcolor{currentfill}%
\pgfsetfillopacity{0.811853}%
\pgfsetlinewidth{1.003750pt}%
\definecolor{currentstroke}{rgb}{0.121569,0.466667,0.705882}%
\pgfsetstrokecolor{currentstroke}%
\pgfsetstrokeopacity{0.811853}%
\pgfsetdash{}{0pt}%
\pgfpathmoveto{\pgfqpoint{0.802304in}{2.627677in}}%
\pgfpathcurveto{\pgfqpoint{0.810540in}{2.627677in}}{\pgfqpoint{0.818440in}{2.630950in}}{\pgfqpoint{0.824264in}{2.636774in}}%
\pgfpathcurveto{\pgfqpoint{0.830088in}{2.642597in}}{\pgfqpoint{0.833360in}{2.650498in}}{\pgfqpoint{0.833360in}{2.658734in}}%
\pgfpathcurveto{\pgfqpoint{0.833360in}{2.666970in}}{\pgfqpoint{0.830088in}{2.674870in}}{\pgfqpoint{0.824264in}{2.680694in}}%
\pgfpathcurveto{\pgfqpoint{0.818440in}{2.686518in}}{\pgfqpoint{0.810540in}{2.689790in}}{\pgfqpoint{0.802304in}{2.689790in}}%
\pgfpathcurveto{\pgfqpoint{0.794067in}{2.689790in}}{\pgfqpoint{0.786167in}{2.686518in}}{\pgfqpoint{0.780343in}{2.680694in}}%
\pgfpathcurveto{\pgfqpoint{0.774520in}{2.674870in}}{\pgfqpoint{0.771247in}{2.666970in}}{\pgfqpoint{0.771247in}{2.658734in}}%
\pgfpathcurveto{\pgfqpoint{0.771247in}{2.650498in}}{\pgfqpoint{0.774520in}{2.642597in}}{\pgfqpoint{0.780343in}{2.636774in}}%
\pgfpathcurveto{\pgfqpoint{0.786167in}{2.630950in}}{\pgfqpoint{0.794067in}{2.627677in}}{\pgfqpoint{0.802304in}{2.627677in}}%
\pgfpathclose%
\pgfusepath{stroke,fill}%
\end{pgfscope}%
\begin{pgfscope}%
\pgfpathrectangle{\pgfqpoint{0.100000in}{0.212622in}}{\pgfqpoint{3.696000in}{3.696000in}}%
\pgfusepath{clip}%
\pgfsetbuttcap%
\pgfsetroundjoin%
\definecolor{currentfill}{rgb}{0.121569,0.466667,0.705882}%
\pgfsetfillcolor{currentfill}%
\pgfsetfillopacity{0.811984}%
\pgfsetlinewidth{1.003750pt}%
\definecolor{currentstroke}{rgb}{0.121569,0.466667,0.705882}%
\pgfsetstrokecolor{currentstroke}%
\pgfsetstrokeopacity{0.811984}%
\pgfsetdash{}{0pt}%
\pgfpathmoveto{\pgfqpoint{0.803139in}{2.626931in}}%
\pgfpathcurveto{\pgfqpoint{0.811375in}{2.626931in}}{\pgfqpoint{0.819275in}{2.630204in}}{\pgfqpoint{0.825099in}{2.636028in}}%
\pgfpathcurveto{\pgfqpoint{0.830923in}{2.641852in}}{\pgfqpoint{0.834195in}{2.649752in}}{\pgfqpoint{0.834195in}{2.657988in}}%
\pgfpathcurveto{\pgfqpoint{0.834195in}{2.666224in}}{\pgfqpoint{0.830923in}{2.674124in}}{\pgfqpoint{0.825099in}{2.679948in}}%
\pgfpathcurveto{\pgfqpoint{0.819275in}{2.685772in}}{\pgfqpoint{0.811375in}{2.689044in}}{\pgfqpoint{0.803139in}{2.689044in}}%
\pgfpathcurveto{\pgfqpoint{0.794903in}{2.689044in}}{\pgfqpoint{0.787002in}{2.685772in}}{\pgfqpoint{0.781179in}{2.679948in}}%
\pgfpathcurveto{\pgfqpoint{0.775355in}{2.674124in}}{\pgfqpoint{0.772082in}{2.666224in}}{\pgfqpoint{0.772082in}{2.657988in}}%
\pgfpathcurveto{\pgfqpoint{0.772082in}{2.649752in}}{\pgfqpoint{0.775355in}{2.641852in}}{\pgfqpoint{0.781179in}{2.636028in}}%
\pgfpathcurveto{\pgfqpoint{0.787002in}{2.630204in}}{\pgfqpoint{0.794903in}{2.626931in}}{\pgfqpoint{0.803139in}{2.626931in}}%
\pgfpathclose%
\pgfusepath{stroke,fill}%
\end{pgfscope}%
\begin{pgfscope}%
\pgfpathrectangle{\pgfqpoint{0.100000in}{0.212622in}}{\pgfqpoint{3.696000in}{3.696000in}}%
\pgfusepath{clip}%
\pgfsetbuttcap%
\pgfsetroundjoin%
\definecolor{currentfill}{rgb}{0.121569,0.466667,0.705882}%
\pgfsetfillcolor{currentfill}%
\pgfsetfillopacity{0.812114}%
\pgfsetlinewidth{1.003750pt}%
\definecolor{currentstroke}{rgb}{0.121569,0.466667,0.705882}%
\pgfsetstrokecolor{currentstroke}%
\pgfsetstrokeopacity{0.812114}%
\pgfsetdash{}{0pt}%
\pgfpathmoveto{\pgfqpoint{0.804874in}{2.625378in}}%
\pgfpathcurveto{\pgfqpoint{0.813110in}{2.625378in}}{\pgfqpoint{0.821010in}{2.628651in}}{\pgfqpoint{0.826834in}{2.634475in}}%
\pgfpathcurveto{\pgfqpoint{0.832658in}{2.640299in}}{\pgfqpoint{0.835930in}{2.648199in}}{\pgfqpoint{0.835930in}{2.656435in}}%
\pgfpathcurveto{\pgfqpoint{0.835930in}{2.664671in}}{\pgfqpoint{0.832658in}{2.672571in}}{\pgfqpoint{0.826834in}{2.678395in}}%
\pgfpathcurveto{\pgfqpoint{0.821010in}{2.684219in}}{\pgfqpoint{0.813110in}{2.687491in}}{\pgfqpoint{0.804874in}{2.687491in}}%
\pgfpathcurveto{\pgfqpoint{0.796637in}{2.687491in}}{\pgfqpoint{0.788737in}{2.684219in}}{\pgfqpoint{0.782913in}{2.678395in}}%
\pgfpathcurveto{\pgfqpoint{0.777089in}{2.672571in}}{\pgfqpoint{0.773817in}{2.664671in}}{\pgfqpoint{0.773817in}{2.656435in}}%
\pgfpathcurveto{\pgfqpoint{0.773817in}{2.648199in}}{\pgfqpoint{0.777089in}{2.640299in}}{\pgfqpoint{0.782913in}{2.634475in}}%
\pgfpathcurveto{\pgfqpoint{0.788737in}{2.628651in}}{\pgfqpoint{0.796637in}{2.625378in}}{\pgfqpoint{0.804874in}{2.625378in}}%
\pgfpathclose%
\pgfusepath{stroke,fill}%
\end{pgfscope}%
\begin{pgfscope}%
\pgfpathrectangle{\pgfqpoint{0.100000in}{0.212622in}}{\pgfqpoint{3.696000in}{3.696000in}}%
\pgfusepath{clip}%
\pgfsetbuttcap%
\pgfsetroundjoin%
\definecolor{currentfill}{rgb}{0.121569,0.466667,0.705882}%
\pgfsetfillcolor{currentfill}%
\pgfsetfillopacity{0.812683}%
\pgfsetlinewidth{1.003750pt}%
\definecolor{currentstroke}{rgb}{0.121569,0.466667,0.705882}%
\pgfsetstrokecolor{currentstroke}%
\pgfsetstrokeopacity{0.812683}%
\pgfsetdash{}{0pt}%
\pgfpathmoveto{\pgfqpoint{0.807586in}{2.623738in}}%
\pgfpathcurveto{\pgfqpoint{0.815823in}{2.623738in}}{\pgfqpoint{0.823723in}{2.627010in}}{\pgfqpoint{0.829547in}{2.632834in}}%
\pgfpathcurveto{\pgfqpoint{0.835370in}{2.638658in}}{\pgfqpoint{0.838643in}{2.646558in}}{\pgfqpoint{0.838643in}{2.654794in}}%
\pgfpathcurveto{\pgfqpoint{0.838643in}{2.663031in}}{\pgfqpoint{0.835370in}{2.670931in}}{\pgfqpoint{0.829547in}{2.676755in}}%
\pgfpathcurveto{\pgfqpoint{0.823723in}{2.682579in}}{\pgfqpoint{0.815823in}{2.685851in}}{\pgfqpoint{0.807586in}{2.685851in}}%
\pgfpathcurveto{\pgfqpoint{0.799350in}{2.685851in}}{\pgfqpoint{0.791450in}{2.682579in}}{\pgfqpoint{0.785626in}{2.676755in}}%
\pgfpathcurveto{\pgfqpoint{0.779802in}{2.670931in}}{\pgfqpoint{0.776530in}{2.663031in}}{\pgfqpoint{0.776530in}{2.654794in}}%
\pgfpathcurveto{\pgfqpoint{0.776530in}{2.646558in}}{\pgfqpoint{0.779802in}{2.638658in}}{\pgfqpoint{0.785626in}{2.632834in}}%
\pgfpathcurveto{\pgfqpoint{0.791450in}{2.627010in}}{\pgfqpoint{0.799350in}{2.623738in}}{\pgfqpoint{0.807586in}{2.623738in}}%
\pgfpathclose%
\pgfusepath{stroke,fill}%
\end{pgfscope}%
\begin{pgfscope}%
\pgfpathrectangle{\pgfqpoint{0.100000in}{0.212622in}}{\pgfqpoint{3.696000in}{3.696000in}}%
\pgfusepath{clip}%
\pgfsetbuttcap%
\pgfsetroundjoin%
\definecolor{currentfill}{rgb}{0.121569,0.466667,0.705882}%
\pgfsetfillcolor{currentfill}%
\pgfsetfillopacity{0.813318}%
\pgfsetlinewidth{1.003750pt}%
\definecolor{currentstroke}{rgb}{0.121569,0.466667,0.705882}%
\pgfsetstrokecolor{currentstroke}%
\pgfsetstrokeopacity{0.813318}%
\pgfsetdash{}{0pt}%
\pgfpathmoveto{\pgfqpoint{0.813048in}{2.619251in}}%
\pgfpathcurveto{\pgfqpoint{0.821284in}{2.619251in}}{\pgfqpoint{0.829184in}{2.622523in}}{\pgfqpoint{0.835008in}{2.628347in}}%
\pgfpathcurveto{\pgfqpoint{0.840832in}{2.634171in}}{\pgfqpoint{0.844104in}{2.642071in}}{\pgfqpoint{0.844104in}{2.650307in}}%
\pgfpathcurveto{\pgfqpoint{0.844104in}{2.658544in}}{\pgfqpoint{0.840832in}{2.666444in}}{\pgfqpoint{0.835008in}{2.672268in}}%
\pgfpathcurveto{\pgfqpoint{0.829184in}{2.678091in}}{\pgfqpoint{0.821284in}{2.681364in}}{\pgfqpoint{0.813048in}{2.681364in}}%
\pgfpathcurveto{\pgfqpoint{0.804812in}{2.681364in}}{\pgfqpoint{0.796912in}{2.678091in}}{\pgfqpoint{0.791088in}{2.672268in}}%
\pgfpathcurveto{\pgfqpoint{0.785264in}{2.666444in}}{\pgfqpoint{0.781991in}{2.658544in}}{\pgfqpoint{0.781991in}{2.650307in}}%
\pgfpathcurveto{\pgfqpoint{0.781991in}{2.642071in}}{\pgfqpoint{0.785264in}{2.634171in}}{\pgfqpoint{0.791088in}{2.628347in}}%
\pgfpathcurveto{\pgfqpoint{0.796912in}{2.622523in}}{\pgfqpoint{0.804812in}{2.619251in}}{\pgfqpoint{0.813048in}{2.619251in}}%
\pgfpathclose%
\pgfusepath{stroke,fill}%
\end{pgfscope}%
\begin{pgfscope}%
\pgfpathrectangle{\pgfqpoint{0.100000in}{0.212622in}}{\pgfqpoint{3.696000in}{3.696000in}}%
\pgfusepath{clip}%
\pgfsetbuttcap%
\pgfsetroundjoin%
\definecolor{currentfill}{rgb}{0.121569,0.466667,0.705882}%
\pgfsetfillcolor{currentfill}%
\pgfsetfillopacity{0.813768}%
\pgfsetlinewidth{1.003750pt}%
\definecolor{currentstroke}{rgb}{0.121569,0.466667,0.705882}%
\pgfsetstrokecolor{currentstroke}%
\pgfsetstrokeopacity{0.813768}%
\pgfsetdash{}{0pt}%
\pgfpathmoveto{\pgfqpoint{2.829916in}{2.043560in}}%
\pgfpathcurveto{\pgfqpoint{2.838153in}{2.043560in}}{\pgfqpoint{2.846053in}{2.046832in}}{\pgfqpoint{2.851877in}{2.052656in}}%
\pgfpathcurveto{\pgfqpoint{2.857701in}{2.058480in}}{\pgfqpoint{2.860973in}{2.066380in}}{\pgfqpoint{2.860973in}{2.074616in}}%
\pgfpathcurveto{\pgfqpoint{2.860973in}{2.082853in}}{\pgfqpoint{2.857701in}{2.090753in}}{\pgfqpoint{2.851877in}{2.096576in}}%
\pgfpathcurveto{\pgfqpoint{2.846053in}{2.102400in}}{\pgfqpoint{2.838153in}{2.105673in}}{\pgfqpoint{2.829916in}{2.105673in}}%
\pgfpathcurveto{\pgfqpoint{2.821680in}{2.105673in}}{\pgfqpoint{2.813780in}{2.102400in}}{\pgfqpoint{2.807956in}{2.096576in}}%
\pgfpathcurveto{\pgfqpoint{2.802132in}{2.090753in}}{\pgfqpoint{2.798860in}{2.082853in}}{\pgfqpoint{2.798860in}{2.074616in}}%
\pgfpathcurveto{\pgfqpoint{2.798860in}{2.066380in}}{\pgfqpoint{2.802132in}{2.058480in}}{\pgfqpoint{2.807956in}{2.052656in}}%
\pgfpathcurveto{\pgfqpoint{2.813780in}{2.046832in}}{\pgfqpoint{2.821680in}{2.043560in}}{\pgfqpoint{2.829916in}{2.043560in}}%
\pgfpathclose%
\pgfusepath{stroke,fill}%
\end{pgfscope}%
\begin{pgfscope}%
\pgfpathrectangle{\pgfqpoint{0.100000in}{0.212622in}}{\pgfqpoint{3.696000in}{3.696000in}}%
\pgfusepath{clip}%
\pgfsetbuttcap%
\pgfsetroundjoin%
\definecolor{currentfill}{rgb}{0.121569,0.466667,0.705882}%
\pgfsetfillcolor{currentfill}%
\pgfsetfillopacity{0.814780}%
\pgfsetlinewidth{1.003750pt}%
\definecolor{currentstroke}{rgb}{0.121569,0.466667,0.705882}%
\pgfsetstrokecolor{currentstroke}%
\pgfsetstrokeopacity{0.814780}%
\pgfsetdash{}{0pt}%
\pgfpathmoveto{\pgfqpoint{0.822793in}{2.612799in}}%
\pgfpathcurveto{\pgfqpoint{0.831029in}{2.612799in}}{\pgfqpoint{0.838930in}{2.616072in}}{\pgfqpoint{0.844753in}{2.621896in}}%
\pgfpathcurveto{\pgfqpoint{0.850577in}{2.627719in}}{\pgfqpoint{0.853850in}{2.635620in}}{\pgfqpoint{0.853850in}{2.643856in}}%
\pgfpathcurveto{\pgfqpoint{0.853850in}{2.652092in}}{\pgfqpoint{0.850577in}{2.659992in}}{\pgfqpoint{0.844753in}{2.665816in}}%
\pgfpathcurveto{\pgfqpoint{0.838930in}{2.671640in}}{\pgfqpoint{0.831029in}{2.674912in}}{\pgfqpoint{0.822793in}{2.674912in}}%
\pgfpathcurveto{\pgfqpoint{0.814557in}{2.674912in}}{\pgfqpoint{0.806657in}{2.671640in}}{\pgfqpoint{0.800833in}{2.665816in}}%
\pgfpathcurveto{\pgfqpoint{0.795009in}{2.659992in}}{\pgfqpoint{0.791737in}{2.652092in}}{\pgfqpoint{0.791737in}{2.643856in}}%
\pgfpathcurveto{\pgfqpoint{0.791737in}{2.635620in}}{\pgfqpoint{0.795009in}{2.627719in}}{\pgfqpoint{0.800833in}{2.621896in}}%
\pgfpathcurveto{\pgfqpoint{0.806657in}{2.616072in}}{\pgfqpoint{0.814557in}{2.612799in}}{\pgfqpoint{0.822793in}{2.612799in}}%
\pgfpathclose%
\pgfusepath{stroke,fill}%
\end{pgfscope}%
\begin{pgfscope}%
\pgfpathrectangle{\pgfqpoint{0.100000in}{0.212622in}}{\pgfqpoint{3.696000in}{3.696000in}}%
\pgfusepath{clip}%
\pgfsetbuttcap%
\pgfsetroundjoin%
\definecolor{currentfill}{rgb}{0.121569,0.466667,0.705882}%
\pgfsetfillcolor{currentfill}%
\pgfsetfillopacity{0.816829}%
\pgfsetlinewidth{1.003750pt}%
\definecolor{currentstroke}{rgb}{0.121569,0.466667,0.705882}%
\pgfsetstrokecolor{currentstroke}%
\pgfsetstrokeopacity{0.816829}%
\pgfsetdash{}{0pt}%
\pgfpathmoveto{\pgfqpoint{0.842016in}{2.600977in}}%
\pgfpathcurveto{\pgfqpoint{0.850252in}{2.600977in}}{\pgfqpoint{0.858152in}{2.604249in}}{\pgfqpoint{0.863976in}{2.610073in}}%
\pgfpathcurveto{\pgfqpoint{0.869800in}{2.615897in}}{\pgfqpoint{0.873072in}{2.623797in}}{\pgfqpoint{0.873072in}{2.632033in}}%
\pgfpathcurveto{\pgfqpoint{0.873072in}{2.640270in}}{\pgfqpoint{0.869800in}{2.648170in}}{\pgfqpoint{0.863976in}{2.653994in}}%
\pgfpathcurveto{\pgfqpoint{0.858152in}{2.659818in}}{\pgfqpoint{0.850252in}{2.663090in}}{\pgfqpoint{0.842016in}{2.663090in}}%
\pgfpathcurveto{\pgfqpoint{0.833779in}{2.663090in}}{\pgfqpoint{0.825879in}{2.659818in}}{\pgfqpoint{0.820055in}{2.653994in}}%
\pgfpathcurveto{\pgfqpoint{0.814232in}{2.648170in}}{\pgfqpoint{0.810959in}{2.640270in}}{\pgfqpoint{0.810959in}{2.632033in}}%
\pgfpathcurveto{\pgfqpoint{0.810959in}{2.623797in}}{\pgfqpoint{0.814232in}{2.615897in}}{\pgfqpoint{0.820055in}{2.610073in}}%
\pgfpathcurveto{\pgfqpoint{0.825879in}{2.604249in}}{\pgfqpoint{0.833779in}{2.600977in}}{\pgfqpoint{0.842016in}{2.600977in}}%
\pgfpathclose%
\pgfusepath{stroke,fill}%
\end{pgfscope}%
\begin{pgfscope}%
\pgfpathrectangle{\pgfqpoint{0.100000in}{0.212622in}}{\pgfqpoint{3.696000in}{3.696000in}}%
\pgfusepath{clip}%
\pgfsetbuttcap%
\pgfsetroundjoin%
\definecolor{currentfill}{rgb}{0.121569,0.466667,0.705882}%
\pgfsetfillcolor{currentfill}%
\pgfsetfillopacity{0.818635}%
\pgfsetlinewidth{1.003750pt}%
\definecolor{currentstroke}{rgb}{0.121569,0.466667,0.705882}%
\pgfsetstrokecolor{currentstroke}%
\pgfsetstrokeopacity{0.818635}%
\pgfsetdash{}{0pt}%
\pgfpathmoveto{\pgfqpoint{2.817734in}{2.037750in}}%
\pgfpathcurveto{\pgfqpoint{2.825970in}{2.037750in}}{\pgfqpoint{2.833870in}{2.041022in}}{\pgfqpoint{2.839694in}{2.046846in}}%
\pgfpathcurveto{\pgfqpoint{2.845518in}{2.052670in}}{\pgfqpoint{2.848791in}{2.060570in}}{\pgfqpoint{2.848791in}{2.068806in}}%
\pgfpathcurveto{\pgfqpoint{2.848791in}{2.077042in}}{\pgfqpoint{2.845518in}{2.084942in}}{\pgfqpoint{2.839694in}{2.090766in}}%
\pgfpathcurveto{\pgfqpoint{2.833870in}{2.096590in}}{\pgfqpoint{2.825970in}{2.099863in}}{\pgfqpoint{2.817734in}{2.099863in}}%
\pgfpathcurveto{\pgfqpoint{2.809498in}{2.099863in}}{\pgfqpoint{2.801598in}{2.096590in}}{\pgfqpoint{2.795774in}{2.090766in}}%
\pgfpathcurveto{\pgfqpoint{2.789950in}{2.084942in}}{\pgfqpoint{2.786678in}{2.077042in}}{\pgfqpoint{2.786678in}{2.068806in}}%
\pgfpathcurveto{\pgfqpoint{2.786678in}{2.060570in}}{\pgfqpoint{2.789950in}{2.052670in}}{\pgfqpoint{2.795774in}{2.046846in}}%
\pgfpathcurveto{\pgfqpoint{2.801598in}{2.041022in}}{\pgfqpoint{2.809498in}{2.037750in}}{\pgfqpoint{2.817734in}{2.037750in}}%
\pgfpathclose%
\pgfusepath{stroke,fill}%
\end{pgfscope}%
\begin{pgfscope}%
\pgfpathrectangle{\pgfqpoint{0.100000in}{0.212622in}}{\pgfqpoint{3.696000in}{3.696000in}}%
\pgfusepath{clip}%
\pgfsetbuttcap%
\pgfsetroundjoin%
\definecolor{currentfill}{rgb}{0.121569,0.466667,0.705882}%
\pgfsetfillcolor{currentfill}%
\pgfsetfillopacity{0.819513}%
\pgfsetlinewidth{1.003750pt}%
\definecolor{currentstroke}{rgb}{0.121569,0.466667,0.705882}%
\pgfsetstrokecolor{currentstroke}%
\pgfsetstrokeopacity{0.819513}%
\pgfsetdash{}{0pt}%
\pgfpathmoveto{\pgfqpoint{0.858322in}{2.592626in}}%
\pgfpathcurveto{\pgfqpoint{0.866558in}{2.592626in}}{\pgfqpoint{0.874458in}{2.595899in}}{\pgfqpoint{0.880282in}{2.601723in}}%
\pgfpathcurveto{\pgfqpoint{0.886106in}{2.607547in}}{\pgfqpoint{0.889378in}{2.615447in}}{\pgfqpoint{0.889378in}{2.623683in}}%
\pgfpathcurveto{\pgfqpoint{0.889378in}{2.631919in}}{\pgfqpoint{0.886106in}{2.639819in}}{\pgfqpoint{0.880282in}{2.645643in}}%
\pgfpathcurveto{\pgfqpoint{0.874458in}{2.651467in}}{\pgfqpoint{0.866558in}{2.654739in}}{\pgfqpoint{0.858322in}{2.654739in}}%
\pgfpathcurveto{\pgfqpoint{0.850085in}{2.654739in}}{\pgfqpoint{0.842185in}{2.651467in}}{\pgfqpoint{0.836361in}{2.645643in}}%
\pgfpathcurveto{\pgfqpoint{0.830537in}{2.639819in}}{\pgfqpoint{0.827265in}{2.631919in}}{\pgfqpoint{0.827265in}{2.623683in}}%
\pgfpathcurveto{\pgfqpoint{0.827265in}{2.615447in}}{\pgfqpoint{0.830537in}{2.607547in}}{\pgfqpoint{0.836361in}{2.601723in}}%
\pgfpathcurveto{\pgfqpoint{0.842185in}{2.595899in}}{\pgfqpoint{0.850085in}{2.592626in}}{\pgfqpoint{0.858322in}{2.592626in}}%
\pgfpathclose%
\pgfusepath{stroke,fill}%
\end{pgfscope}%
\begin{pgfscope}%
\pgfpathrectangle{\pgfqpoint{0.100000in}{0.212622in}}{\pgfqpoint{3.696000in}{3.696000in}}%
\pgfusepath{clip}%
\pgfsetbuttcap%
\pgfsetroundjoin%
\definecolor{currentfill}{rgb}{0.121569,0.466667,0.705882}%
\pgfsetfillcolor{currentfill}%
\pgfsetfillopacity{0.821006}%
\pgfsetlinewidth{1.003750pt}%
\definecolor{currentstroke}{rgb}{0.121569,0.466667,0.705882}%
\pgfsetstrokecolor{currentstroke}%
\pgfsetstrokeopacity{0.821006}%
\pgfsetdash{}{0pt}%
\pgfpathmoveto{\pgfqpoint{0.873621in}{2.582728in}}%
\pgfpathcurveto{\pgfqpoint{0.881857in}{2.582728in}}{\pgfqpoint{0.889757in}{2.586000in}}{\pgfqpoint{0.895581in}{2.591824in}}%
\pgfpathcurveto{\pgfqpoint{0.901405in}{2.597648in}}{\pgfqpoint{0.904678in}{2.605548in}}{\pgfqpoint{0.904678in}{2.613784in}}%
\pgfpathcurveto{\pgfqpoint{0.904678in}{2.622021in}}{\pgfqpoint{0.901405in}{2.629921in}}{\pgfqpoint{0.895581in}{2.635745in}}%
\pgfpathcurveto{\pgfqpoint{0.889757in}{2.641569in}}{\pgfqpoint{0.881857in}{2.644841in}}{\pgfqpoint{0.873621in}{2.644841in}}%
\pgfpathcurveto{\pgfqpoint{0.865385in}{2.644841in}}{\pgfqpoint{0.857485in}{2.641569in}}{\pgfqpoint{0.851661in}{2.635745in}}%
\pgfpathcurveto{\pgfqpoint{0.845837in}{2.629921in}}{\pgfqpoint{0.842565in}{2.622021in}}{\pgfqpoint{0.842565in}{2.613784in}}%
\pgfpathcurveto{\pgfqpoint{0.842565in}{2.605548in}}{\pgfqpoint{0.845837in}{2.597648in}}{\pgfqpoint{0.851661in}{2.591824in}}%
\pgfpathcurveto{\pgfqpoint{0.857485in}{2.586000in}}{\pgfqpoint{0.865385in}{2.582728in}}{\pgfqpoint{0.873621in}{2.582728in}}%
\pgfpathclose%
\pgfusepath{stroke,fill}%
\end{pgfscope}%
\begin{pgfscope}%
\pgfpathrectangle{\pgfqpoint{0.100000in}{0.212622in}}{\pgfqpoint{3.696000in}{3.696000in}}%
\pgfusepath{clip}%
\pgfsetbuttcap%
\pgfsetroundjoin%
\definecolor{currentfill}{rgb}{0.121569,0.466667,0.705882}%
\pgfsetfillcolor{currentfill}%
\pgfsetfillopacity{0.821678}%
\pgfsetlinewidth{1.003750pt}%
\definecolor{currentstroke}{rgb}{0.121569,0.466667,0.705882}%
\pgfsetstrokecolor{currentstroke}%
\pgfsetstrokeopacity{0.821678}%
\pgfsetdash{}{0pt}%
\pgfpathmoveto{\pgfqpoint{2.812014in}{2.035645in}}%
\pgfpathcurveto{\pgfqpoint{2.820250in}{2.035645in}}{\pgfqpoint{2.828150in}{2.038917in}}{\pgfqpoint{2.833974in}{2.044741in}}%
\pgfpathcurveto{\pgfqpoint{2.839798in}{2.050565in}}{\pgfqpoint{2.843070in}{2.058465in}}{\pgfqpoint{2.843070in}{2.066701in}}%
\pgfpathcurveto{\pgfqpoint{2.843070in}{2.074937in}}{\pgfqpoint{2.839798in}{2.082838in}}{\pgfqpoint{2.833974in}{2.088661in}}%
\pgfpathcurveto{\pgfqpoint{2.828150in}{2.094485in}}{\pgfqpoint{2.820250in}{2.097758in}}{\pgfqpoint{2.812014in}{2.097758in}}%
\pgfpathcurveto{\pgfqpoint{2.803777in}{2.097758in}}{\pgfqpoint{2.795877in}{2.094485in}}{\pgfqpoint{2.790053in}{2.088661in}}%
\pgfpathcurveto{\pgfqpoint{2.784229in}{2.082838in}}{\pgfqpoint{2.780957in}{2.074937in}}{\pgfqpoint{2.780957in}{2.066701in}}%
\pgfpathcurveto{\pgfqpoint{2.780957in}{2.058465in}}{\pgfqpoint{2.784229in}{2.050565in}}{\pgfqpoint{2.790053in}{2.044741in}}%
\pgfpathcurveto{\pgfqpoint{2.795877in}{2.038917in}}{\pgfqpoint{2.803777in}{2.035645in}}{\pgfqpoint{2.812014in}{2.035645in}}%
\pgfpathclose%
\pgfusepath{stroke,fill}%
\end{pgfscope}%
\begin{pgfscope}%
\pgfpathrectangle{\pgfqpoint{0.100000in}{0.212622in}}{\pgfqpoint{3.696000in}{3.696000in}}%
\pgfusepath{clip}%
\pgfsetbuttcap%
\pgfsetroundjoin%
\definecolor{currentfill}{rgb}{0.121569,0.466667,0.705882}%
\pgfsetfillcolor{currentfill}%
\pgfsetfillopacity{0.822816}%
\pgfsetlinewidth{1.003750pt}%
\definecolor{currentstroke}{rgb}{0.121569,0.466667,0.705882}%
\pgfsetstrokecolor{currentstroke}%
\pgfsetstrokeopacity{0.822816}%
\pgfsetdash{}{0pt}%
\pgfpathmoveto{\pgfqpoint{0.884801in}{2.578097in}}%
\pgfpathcurveto{\pgfqpoint{0.893037in}{2.578097in}}{\pgfqpoint{0.900937in}{2.581369in}}{\pgfqpoint{0.906761in}{2.587193in}}%
\pgfpathcurveto{\pgfqpoint{0.912585in}{2.593017in}}{\pgfqpoint{0.915857in}{2.600917in}}{\pgfqpoint{0.915857in}{2.609154in}}%
\pgfpathcurveto{\pgfqpoint{0.915857in}{2.617390in}}{\pgfqpoint{0.912585in}{2.625290in}}{\pgfqpoint{0.906761in}{2.631114in}}%
\pgfpathcurveto{\pgfqpoint{0.900937in}{2.636938in}}{\pgfqpoint{0.893037in}{2.640210in}}{\pgfqpoint{0.884801in}{2.640210in}}%
\pgfpathcurveto{\pgfqpoint{0.876564in}{2.640210in}}{\pgfqpoint{0.868664in}{2.636938in}}{\pgfqpoint{0.862840in}{2.631114in}}%
\pgfpathcurveto{\pgfqpoint{0.857017in}{2.625290in}}{\pgfqpoint{0.853744in}{2.617390in}}{\pgfqpoint{0.853744in}{2.609154in}}%
\pgfpathcurveto{\pgfqpoint{0.853744in}{2.600917in}}{\pgfqpoint{0.857017in}{2.593017in}}{\pgfqpoint{0.862840in}{2.587193in}}%
\pgfpathcurveto{\pgfqpoint{0.868664in}{2.581369in}}{\pgfqpoint{0.876564in}{2.578097in}}{\pgfqpoint{0.884801in}{2.578097in}}%
\pgfpathclose%
\pgfusepath{stroke,fill}%
\end{pgfscope}%
\begin{pgfscope}%
\pgfpathrectangle{\pgfqpoint{0.100000in}{0.212622in}}{\pgfqpoint{3.696000in}{3.696000in}}%
\pgfusepath{clip}%
\pgfsetbuttcap%
\pgfsetroundjoin%
\definecolor{currentfill}{rgb}{0.121569,0.466667,0.705882}%
\pgfsetfillcolor{currentfill}%
\pgfsetfillopacity{0.823275}%
\pgfsetlinewidth{1.003750pt}%
\definecolor{currentstroke}{rgb}{0.121569,0.466667,0.705882}%
\pgfsetstrokecolor{currentstroke}%
\pgfsetstrokeopacity{0.823275}%
\pgfsetdash{}{0pt}%
\pgfpathmoveto{\pgfqpoint{2.808488in}{2.034445in}}%
\pgfpathcurveto{\pgfqpoint{2.816725in}{2.034445in}}{\pgfqpoint{2.824625in}{2.037718in}}{\pgfqpoint{2.830449in}{2.043542in}}%
\pgfpathcurveto{\pgfqpoint{2.836273in}{2.049366in}}{\pgfqpoint{2.839545in}{2.057266in}}{\pgfqpoint{2.839545in}{2.065502in}}%
\pgfpathcurveto{\pgfqpoint{2.839545in}{2.073738in}}{\pgfqpoint{2.836273in}{2.081638in}}{\pgfqpoint{2.830449in}{2.087462in}}%
\pgfpathcurveto{\pgfqpoint{2.824625in}{2.093286in}}{\pgfqpoint{2.816725in}{2.096558in}}{\pgfqpoint{2.808488in}{2.096558in}}%
\pgfpathcurveto{\pgfqpoint{2.800252in}{2.096558in}}{\pgfqpoint{2.792352in}{2.093286in}}{\pgfqpoint{2.786528in}{2.087462in}}%
\pgfpathcurveto{\pgfqpoint{2.780704in}{2.081638in}}{\pgfqpoint{2.777432in}{2.073738in}}{\pgfqpoint{2.777432in}{2.065502in}}%
\pgfpathcurveto{\pgfqpoint{2.777432in}{2.057266in}}{\pgfqpoint{2.780704in}{2.049366in}}{\pgfqpoint{2.786528in}{2.043542in}}%
\pgfpathcurveto{\pgfqpoint{2.792352in}{2.037718in}}{\pgfqpoint{2.800252in}{2.034445in}}{\pgfqpoint{2.808488in}{2.034445in}}%
\pgfpathclose%
\pgfusepath{stroke,fill}%
\end{pgfscope}%
\begin{pgfscope}%
\pgfpathrectangle{\pgfqpoint{0.100000in}{0.212622in}}{\pgfqpoint{3.696000in}{3.696000in}}%
\pgfusepath{clip}%
\pgfsetbuttcap%
\pgfsetroundjoin%
\definecolor{currentfill}{rgb}{0.121569,0.466667,0.705882}%
\pgfsetfillcolor{currentfill}%
\pgfsetfillopacity{0.823822}%
\pgfsetlinewidth{1.003750pt}%
\definecolor{currentstroke}{rgb}{0.121569,0.466667,0.705882}%
\pgfsetstrokecolor{currentstroke}%
\pgfsetstrokeopacity{0.823822}%
\pgfsetdash{}{0pt}%
\pgfpathmoveto{\pgfqpoint{0.893712in}{2.572189in}}%
\pgfpathcurveto{\pgfqpoint{0.901948in}{2.572189in}}{\pgfqpoint{0.909848in}{2.575462in}}{\pgfqpoint{0.915672in}{2.581285in}}%
\pgfpathcurveto{\pgfqpoint{0.921496in}{2.587109in}}{\pgfqpoint{0.924768in}{2.595009in}}{\pgfqpoint{0.924768in}{2.603246in}}%
\pgfpathcurveto{\pgfqpoint{0.924768in}{2.611482in}}{\pgfqpoint{0.921496in}{2.619382in}}{\pgfqpoint{0.915672in}{2.625206in}}%
\pgfpathcurveto{\pgfqpoint{0.909848in}{2.631030in}}{\pgfqpoint{0.901948in}{2.634302in}}{\pgfqpoint{0.893712in}{2.634302in}}%
\pgfpathcurveto{\pgfqpoint{0.885476in}{2.634302in}}{\pgfqpoint{0.877576in}{2.631030in}}{\pgfqpoint{0.871752in}{2.625206in}}%
\pgfpathcurveto{\pgfqpoint{0.865928in}{2.619382in}}{\pgfqpoint{0.862655in}{2.611482in}}{\pgfqpoint{0.862655in}{2.603246in}}%
\pgfpathcurveto{\pgfqpoint{0.862655in}{2.595009in}}{\pgfqpoint{0.865928in}{2.587109in}}{\pgfqpoint{0.871752in}{2.581285in}}%
\pgfpathcurveto{\pgfqpoint{0.877576in}{2.575462in}}{\pgfqpoint{0.885476in}{2.572189in}}{\pgfqpoint{0.893712in}{2.572189in}}%
\pgfpathclose%
\pgfusepath{stroke,fill}%
\end{pgfscope}%
\begin{pgfscope}%
\pgfpathrectangle{\pgfqpoint{0.100000in}{0.212622in}}{\pgfqpoint{3.696000in}{3.696000in}}%
\pgfusepath{clip}%
\pgfsetbuttcap%
\pgfsetroundjoin%
\definecolor{currentfill}{rgb}{0.121569,0.466667,0.705882}%
\pgfsetfillcolor{currentfill}%
\pgfsetfillopacity{0.824105}%
\pgfsetlinewidth{1.003750pt}%
\definecolor{currentstroke}{rgb}{0.121569,0.466667,0.705882}%
\pgfsetstrokecolor{currentstroke}%
\pgfsetstrokeopacity{0.824105}%
\pgfsetdash{}{0pt}%
\pgfpathmoveto{\pgfqpoint{2.806427in}{2.033634in}}%
\pgfpathcurveto{\pgfqpoint{2.814663in}{2.033634in}}{\pgfqpoint{2.822563in}{2.036907in}}{\pgfqpoint{2.828387in}{2.042731in}}%
\pgfpathcurveto{\pgfqpoint{2.834211in}{2.048555in}}{\pgfqpoint{2.837483in}{2.056455in}}{\pgfqpoint{2.837483in}{2.064691in}}%
\pgfpathcurveto{\pgfqpoint{2.837483in}{2.072927in}}{\pgfqpoint{2.834211in}{2.080827in}}{\pgfqpoint{2.828387in}{2.086651in}}%
\pgfpathcurveto{\pgfqpoint{2.822563in}{2.092475in}}{\pgfqpoint{2.814663in}{2.095747in}}{\pgfqpoint{2.806427in}{2.095747in}}%
\pgfpathcurveto{\pgfqpoint{2.798190in}{2.095747in}}{\pgfqpoint{2.790290in}{2.092475in}}{\pgfqpoint{2.784466in}{2.086651in}}%
\pgfpathcurveto{\pgfqpoint{2.778642in}{2.080827in}}{\pgfqpoint{2.775370in}{2.072927in}}{\pgfqpoint{2.775370in}{2.064691in}}%
\pgfpathcurveto{\pgfqpoint{2.775370in}{2.056455in}}{\pgfqpoint{2.778642in}{2.048555in}}{\pgfqpoint{2.784466in}{2.042731in}}%
\pgfpathcurveto{\pgfqpoint{2.790290in}{2.036907in}}{\pgfqpoint{2.798190in}{2.033634in}}{\pgfqpoint{2.806427in}{2.033634in}}%
\pgfpathclose%
\pgfusepath{stroke,fill}%
\end{pgfscope}%
\begin{pgfscope}%
\pgfpathrectangle{\pgfqpoint{0.100000in}{0.212622in}}{\pgfqpoint{3.696000in}{3.696000in}}%
\pgfusepath{clip}%
\pgfsetbuttcap%
\pgfsetroundjoin%
\definecolor{currentfill}{rgb}{0.121569,0.466667,0.705882}%
\pgfsetfillcolor{currentfill}%
\pgfsetfillopacity{0.824558}%
\pgfsetlinewidth{1.003750pt}%
\definecolor{currentstroke}{rgb}{0.121569,0.466667,0.705882}%
\pgfsetstrokecolor{currentstroke}%
\pgfsetstrokeopacity{0.824558}%
\pgfsetdash{}{0pt}%
\pgfpathmoveto{\pgfqpoint{0.898638in}{2.570002in}}%
\pgfpathcurveto{\pgfqpoint{0.906874in}{2.570002in}}{\pgfqpoint{0.914774in}{2.573275in}}{\pgfqpoint{0.920598in}{2.579098in}}%
\pgfpathcurveto{\pgfqpoint{0.926422in}{2.584922in}}{\pgfqpoint{0.929694in}{2.592822in}}{\pgfqpoint{0.929694in}{2.601059in}}%
\pgfpathcurveto{\pgfqpoint{0.929694in}{2.609295in}}{\pgfqpoint{0.926422in}{2.617195in}}{\pgfqpoint{0.920598in}{2.623019in}}%
\pgfpathcurveto{\pgfqpoint{0.914774in}{2.628843in}}{\pgfqpoint{0.906874in}{2.632115in}}{\pgfqpoint{0.898638in}{2.632115in}}%
\pgfpathcurveto{\pgfqpoint{0.890402in}{2.632115in}}{\pgfqpoint{0.882502in}{2.628843in}}{\pgfqpoint{0.876678in}{2.623019in}}%
\pgfpathcurveto{\pgfqpoint{0.870854in}{2.617195in}}{\pgfqpoint{0.867581in}{2.609295in}}{\pgfqpoint{0.867581in}{2.601059in}}%
\pgfpathcurveto{\pgfqpoint{0.867581in}{2.592822in}}{\pgfqpoint{0.870854in}{2.584922in}}{\pgfqpoint{0.876678in}{2.579098in}}%
\pgfpathcurveto{\pgfqpoint{0.882502in}{2.573275in}}{\pgfqpoint{0.890402in}{2.570002in}}{\pgfqpoint{0.898638in}{2.570002in}}%
\pgfpathclose%
\pgfusepath{stroke,fill}%
\end{pgfscope}%
\begin{pgfscope}%
\pgfpathrectangle{\pgfqpoint{0.100000in}{0.212622in}}{\pgfqpoint{3.696000in}{3.696000in}}%
\pgfusepath{clip}%
\pgfsetbuttcap%
\pgfsetroundjoin%
\definecolor{currentfill}{rgb}{0.121569,0.466667,0.705882}%
\pgfsetfillcolor{currentfill}%
\pgfsetfillopacity{0.824581}%
\pgfsetlinewidth{1.003750pt}%
\definecolor{currentstroke}{rgb}{0.121569,0.466667,0.705882}%
\pgfsetstrokecolor{currentstroke}%
\pgfsetstrokeopacity{0.824581}%
\pgfsetdash{}{0pt}%
\pgfpathmoveto{\pgfqpoint{2.805715in}{2.032849in}}%
\pgfpathcurveto{\pgfqpoint{2.813952in}{2.032849in}}{\pgfqpoint{2.821852in}{2.036121in}}{\pgfqpoint{2.827676in}{2.041945in}}%
\pgfpathcurveto{\pgfqpoint{2.833499in}{2.047769in}}{\pgfqpoint{2.836772in}{2.055669in}}{\pgfqpoint{2.836772in}{2.063906in}}%
\pgfpathcurveto{\pgfqpoint{2.836772in}{2.072142in}}{\pgfqpoint{2.833499in}{2.080042in}}{\pgfqpoint{2.827676in}{2.085866in}}%
\pgfpathcurveto{\pgfqpoint{2.821852in}{2.091690in}}{\pgfqpoint{2.813952in}{2.094962in}}{\pgfqpoint{2.805715in}{2.094962in}}%
\pgfpathcurveto{\pgfqpoint{2.797479in}{2.094962in}}{\pgfqpoint{2.789579in}{2.091690in}}{\pgfqpoint{2.783755in}{2.085866in}}%
\pgfpathcurveto{\pgfqpoint{2.777931in}{2.080042in}}{\pgfqpoint{2.774659in}{2.072142in}}{\pgfqpoint{2.774659in}{2.063906in}}%
\pgfpathcurveto{\pgfqpoint{2.774659in}{2.055669in}}{\pgfqpoint{2.777931in}{2.047769in}}{\pgfqpoint{2.783755in}{2.041945in}}%
\pgfpathcurveto{\pgfqpoint{2.789579in}{2.036121in}}{\pgfqpoint{2.797479in}{2.032849in}}{\pgfqpoint{2.805715in}{2.032849in}}%
\pgfpathclose%
\pgfusepath{stroke,fill}%
\end{pgfscope}%
\begin{pgfscope}%
\pgfpathrectangle{\pgfqpoint{0.100000in}{0.212622in}}{\pgfqpoint{3.696000in}{3.696000in}}%
\pgfusepath{clip}%
\pgfsetbuttcap%
\pgfsetroundjoin%
\definecolor{currentfill}{rgb}{0.121569,0.466667,0.705882}%
\pgfsetfillcolor{currentfill}%
\pgfsetfillopacity{0.825529}%
\pgfsetlinewidth{1.003750pt}%
\definecolor{currentstroke}{rgb}{0.121569,0.466667,0.705882}%
\pgfsetstrokecolor{currentstroke}%
\pgfsetstrokeopacity{0.825529}%
\pgfsetdash{}{0pt}%
\pgfpathmoveto{\pgfqpoint{0.907671in}{2.563580in}}%
\pgfpathcurveto{\pgfqpoint{0.915907in}{2.563580in}}{\pgfqpoint{0.923807in}{2.566852in}}{\pgfqpoint{0.929631in}{2.572676in}}%
\pgfpathcurveto{\pgfqpoint{0.935455in}{2.578500in}}{\pgfqpoint{0.938727in}{2.586400in}}{\pgfqpoint{0.938727in}{2.594636in}}%
\pgfpathcurveto{\pgfqpoint{0.938727in}{2.602872in}}{\pgfqpoint{0.935455in}{2.610772in}}{\pgfqpoint{0.929631in}{2.616596in}}%
\pgfpathcurveto{\pgfqpoint{0.923807in}{2.622420in}}{\pgfqpoint{0.915907in}{2.625693in}}{\pgfqpoint{0.907671in}{2.625693in}}%
\pgfpathcurveto{\pgfqpoint{0.899434in}{2.625693in}}{\pgfqpoint{0.891534in}{2.622420in}}{\pgfqpoint{0.885710in}{2.616596in}}%
\pgfpathcurveto{\pgfqpoint{0.879886in}{2.610772in}}{\pgfqpoint{0.876614in}{2.602872in}}{\pgfqpoint{0.876614in}{2.594636in}}%
\pgfpathcurveto{\pgfqpoint{0.876614in}{2.586400in}}{\pgfqpoint{0.879886in}{2.578500in}}{\pgfqpoint{0.885710in}{2.572676in}}%
\pgfpathcurveto{\pgfqpoint{0.891534in}{2.566852in}}{\pgfqpoint{0.899434in}{2.563580in}}{\pgfqpoint{0.907671in}{2.563580in}}%
\pgfpathclose%
\pgfusepath{stroke,fill}%
\end{pgfscope}%
\begin{pgfscope}%
\pgfpathrectangle{\pgfqpoint{0.100000in}{0.212622in}}{\pgfqpoint{3.696000in}{3.696000in}}%
\pgfusepath{clip}%
\pgfsetbuttcap%
\pgfsetroundjoin%
\definecolor{currentfill}{rgb}{0.121569,0.466667,0.705882}%
\pgfsetfillcolor{currentfill}%
\pgfsetfillopacity{0.826338}%
\pgfsetlinewidth{1.003750pt}%
\definecolor{currentstroke}{rgb}{0.121569,0.466667,0.705882}%
\pgfsetstrokecolor{currentstroke}%
\pgfsetstrokeopacity{0.826338}%
\pgfsetdash{}{0pt}%
\pgfpathmoveto{\pgfqpoint{2.800412in}{2.030993in}}%
\pgfpathcurveto{\pgfqpoint{2.808648in}{2.030993in}}{\pgfqpoint{2.816548in}{2.034265in}}{\pgfqpoint{2.822372in}{2.040089in}}%
\pgfpathcurveto{\pgfqpoint{2.828196in}{2.045913in}}{\pgfqpoint{2.831468in}{2.053813in}}{\pgfqpoint{2.831468in}{2.062049in}}%
\pgfpathcurveto{\pgfqpoint{2.831468in}{2.070286in}}{\pgfqpoint{2.828196in}{2.078186in}}{\pgfqpoint{2.822372in}{2.084010in}}%
\pgfpathcurveto{\pgfqpoint{2.816548in}{2.089834in}}{\pgfqpoint{2.808648in}{2.093106in}}{\pgfqpoint{2.800412in}{2.093106in}}%
\pgfpathcurveto{\pgfqpoint{2.792175in}{2.093106in}}{\pgfqpoint{2.784275in}{2.089834in}}{\pgfqpoint{2.778451in}{2.084010in}}%
\pgfpathcurveto{\pgfqpoint{2.772628in}{2.078186in}}{\pgfqpoint{2.769355in}{2.070286in}}{\pgfqpoint{2.769355in}{2.062049in}}%
\pgfpathcurveto{\pgfqpoint{2.769355in}{2.053813in}}{\pgfqpoint{2.772628in}{2.045913in}}{\pgfqpoint{2.778451in}{2.040089in}}%
\pgfpathcurveto{\pgfqpoint{2.784275in}{2.034265in}}{\pgfqpoint{2.792175in}{2.030993in}}{\pgfqpoint{2.800412in}{2.030993in}}%
\pgfpathclose%
\pgfusepath{stroke,fill}%
\end{pgfscope}%
\begin{pgfscope}%
\pgfpathrectangle{\pgfqpoint{0.100000in}{0.212622in}}{\pgfqpoint{3.696000in}{3.696000in}}%
\pgfusepath{clip}%
\pgfsetbuttcap%
\pgfsetroundjoin%
\definecolor{currentfill}{rgb}{0.121569,0.466667,0.705882}%
\pgfsetfillcolor{currentfill}%
\pgfsetfillopacity{0.827979}%
\pgfsetlinewidth{1.003750pt}%
\definecolor{currentstroke}{rgb}{0.121569,0.466667,0.705882}%
\pgfsetstrokecolor{currentstroke}%
\pgfsetstrokeopacity{0.827979}%
\pgfsetdash{}{0pt}%
\pgfpathmoveto{\pgfqpoint{0.923220in}{2.554281in}}%
\pgfpathcurveto{\pgfqpoint{0.931456in}{2.554281in}}{\pgfqpoint{0.939356in}{2.557553in}}{\pgfqpoint{0.945180in}{2.563377in}}%
\pgfpathcurveto{\pgfqpoint{0.951004in}{2.569201in}}{\pgfqpoint{0.954276in}{2.577101in}}{\pgfqpoint{0.954276in}{2.585337in}}%
\pgfpathcurveto{\pgfqpoint{0.954276in}{2.593574in}}{\pgfqpoint{0.951004in}{2.601474in}}{\pgfqpoint{0.945180in}{2.607297in}}%
\pgfpathcurveto{\pgfqpoint{0.939356in}{2.613121in}}{\pgfqpoint{0.931456in}{2.616394in}}{\pgfqpoint{0.923220in}{2.616394in}}%
\pgfpathcurveto{\pgfqpoint{0.914983in}{2.616394in}}{\pgfqpoint{0.907083in}{2.613121in}}{\pgfqpoint{0.901259in}{2.607297in}}%
\pgfpathcurveto{\pgfqpoint{0.895435in}{2.601474in}}{\pgfqpoint{0.892163in}{2.593574in}}{\pgfqpoint{0.892163in}{2.585337in}}%
\pgfpathcurveto{\pgfqpoint{0.892163in}{2.577101in}}{\pgfqpoint{0.895435in}{2.569201in}}{\pgfqpoint{0.901259in}{2.563377in}}%
\pgfpathcurveto{\pgfqpoint{0.907083in}{2.557553in}}{\pgfqpoint{0.914983in}{2.554281in}}{\pgfqpoint{0.923220in}{2.554281in}}%
\pgfpathclose%
\pgfusepath{stroke,fill}%
\end{pgfscope}%
\begin{pgfscope}%
\pgfpathrectangle{\pgfqpoint{0.100000in}{0.212622in}}{\pgfqpoint{3.696000in}{3.696000in}}%
\pgfusepath{clip}%
\pgfsetbuttcap%
\pgfsetroundjoin%
\definecolor{currentfill}{rgb}{0.121569,0.466667,0.705882}%
\pgfsetfillcolor{currentfill}%
\pgfsetfillopacity{0.829296}%
\pgfsetlinewidth{1.003750pt}%
\definecolor{currentstroke}{rgb}{0.121569,0.466667,0.705882}%
\pgfsetstrokecolor{currentstroke}%
\pgfsetstrokeopacity{0.829296}%
\pgfsetdash{}{0pt}%
\pgfpathmoveto{\pgfqpoint{2.794047in}{2.024354in}}%
\pgfpathcurveto{\pgfqpoint{2.802283in}{2.024354in}}{\pgfqpoint{2.810183in}{2.027626in}}{\pgfqpoint{2.816007in}{2.033450in}}%
\pgfpathcurveto{\pgfqpoint{2.821831in}{2.039274in}}{\pgfqpoint{2.825103in}{2.047174in}}{\pgfqpoint{2.825103in}{2.055411in}}%
\pgfpathcurveto{\pgfqpoint{2.825103in}{2.063647in}}{\pgfqpoint{2.821831in}{2.071547in}}{\pgfqpoint{2.816007in}{2.077371in}}%
\pgfpathcurveto{\pgfqpoint{2.810183in}{2.083195in}}{\pgfqpoint{2.802283in}{2.086467in}}{\pgfqpoint{2.794047in}{2.086467in}}%
\pgfpathcurveto{\pgfqpoint{2.785810in}{2.086467in}}{\pgfqpoint{2.777910in}{2.083195in}}{\pgfqpoint{2.772086in}{2.077371in}}%
\pgfpathcurveto{\pgfqpoint{2.766263in}{2.071547in}}{\pgfqpoint{2.762990in}{2.063647in}}{\pgfqpoint{2.762990in}{2.055411in}}%
\pgfpathcurveto{\pgfqpoint{2.762990in}{2.047174in}}{\pgfqpoint{2.766263in}{2.039274in}}{\pgfqpoint{2.772086in}{2.033450in}}%
\pgfpathcurveto{\pgfqpoint{2.777910in}{2.027626in}}{\pgfqpoint{2.785810in}{2.024354in}}{\pgfqpoint{2.794047in}{2.024354in}}%
\pgfpathclose%
\pgfusepath{stroke,fill}%
\end{pgfscope}%
\begin{pgfscope}%
\pgfpathrectangle{\pgfqpoint{0.100000in}{0.212622in}}{\pgfqpoint{3.696000in}{3.696000in}}%
\pgfusepath{clip}%
\pgfsetbuttcap%
\pgfsetroundjoin%
\definecolor{currentfill}{rgb}{0.121569,0.466667,0.705882}%
\pgfsetfillcolor{currentfill}%
\pgfsetfillopacity{0.830004}%
\pgfsetlinewidth{1.003750pt}%
\definecolor{currentstroke}{rgb}{0.121569,0.466667,0.705882}%
\pgfsetstrokecolor{currentstroke}%
\pgfsetstrokeopacity{0.830004}%
\pgfsetdash{}{0pt}%
\pgfpathmoveto{\pgfqpoint{0.936273in}{2.544233in}}%
\pgfpathcurveto{\pgfqpoint{0.944510in}{2.544233in}}{\pgfqpoint{0.952410in}{2.547505in}}{\pgfqpoint{0.958234in}{2.553329in}}%
\pgfpathcurveto{\pgfqpoint{0.964058in}{2.559153in}}{\pgfqpoint{0.967330in}{2.567053in}}{\pgfqpoint{0.967330in}{2.575289in}}%
\pgfpathcurveto{\pgfqpoint{0.967330in}{2.583526in}}{\pgfqpoint{0.964058in}{2.591426in}}{\pgfqpoint{0.958234in}{2.597250in}}%
\pgfpathcurveto{\pgfqpoint{0.952410in}{2.603073in}}{\pgfqpoint{0.944510in}{2.606346in}}{\pgfqpoint{0.936273in}{2.606346in}}%
\pgfpathcurveto{\pgfqpoint{0.928037in}{2.606346in}}{\pgfqpoint{0.920137in}{2.603073in}}{\pgfqpoint{0.914313in}{2.597250in}}%
\pgfpathcurveto{\pgfqpoint{0.908489in}{2.591426in}}{\pgfqpoint{0.905217in}{2.583526in}}{\pgfqpoint{0.905217in}{2.575289in}}%
\pgfpathcurveto{\pgfqpoint{0.905217in}{2.567053in}}{\pgfqpoint{0.908489in}{2.559153in}}{\pgfqpoint{0.914313in}{2.553329in}}%
\pgfpathcurveto{\pgfqpoint{0.920137in}{2.547505in}}{\pgfqpoint{0.928037in}{2.544233in}}{\pgfqpoint{0.936273in}{2.544233in}}%
\pgfpathclose%
\pgfusepath{stroke,fill}%
\end{pgfscope}%
\begin{pgfscope}%
\pgfpathrectangle{\pgfqpoint{0.100000in}{0.212622in}}{\pgfqpoint{3.696000in}{3.696000in}}%
\pgfusepath{clip}%
\pgfsetbuttcap%
\pgfsetroundjoin%
\definecolor{currentfill}{rgb}{0.121569,0.466667,0.705882}%
\pgfsetfillcolor{currentfill}%
\pgfsetfillopacity{0.831483}%
\pgfsetlinewidth{1.003750pt}%
\definecolor{currentstroke}{rgb}{0.121569,0.466667,0.705882}%
\pgfsetstrokecolor{currentstroke}%
\pgfsetstrokeopacity{0.831483}%
\pgfsetdash{}{0pt}%
\pgfpathmoveto{\pgfqpoint{0.946902in}{2.537608in}}%
\pgfpathcurveto{\pgfqpoint{0.955139in}{2.537608in}}{\pgfqpoint{0.963039in}{2.540880in}}{\pgfqpoint{0.968863in}{2.546704in}}%
\pgfpathcurveto{\pgfqpoint{0.974687in}{2.552528in}}{\pgfqpoint{0.977959in}{2.560428in}}{\pgfqpoint{0.977959in}{2.568664in}}%
\pgfpathcurveto{\pgfqpoint{0.977959in}{2.576900in}}{\pgfqpoint{0.974687in}{2.584801in}}{\pgfqpoint{0.968863in}{2.590624in}}%
\pgfpathcurveto{\pgfqpoint{0.963039in}{2.596448in}}{\pgfqpoint{0.955139in}{2.599721in}}{\pgfqpoint{0.946902in}{2.599721in}}%
\pgfpathcurveto{\pgfqpoint{0.938666in}{2.599721in}}{\pgfqpoint{0.930766in}{2.596448in}}{\pgfqpoint{0.924942in}{2.590624in}}%
\pgfpathcurveto{\pgfqpoint{0.919118in}{2.584801in}}{\pgfqpoint{0.915846in}{2.576900in}}{\pgfqpoint{0.915846in}{2.568664in}}%
\pgfpathcurveto{\pgfqpoint{0.915846in}{2.560428in}}{\pgfqpoint{0.919118in}{2.552528in}}{\pgfqpoint{0.924942in}{2.546704in}}%
\pgfpathcurveto{\pgfqpoint{0.930766in}{2.540880in}}{\pgfqpoint{0.938666in}{2.537608in}}{\pgfqpoint{0.946902in}{2.537608in}}%
\pgfpathclose%
\pgfusepath{stroke,fill}%
\end{pgfscope}%
\begin{pgfscope}%
\pgfpathrectangle{\pgfqpoint{0.100000in}{0.212622in}}{\pgfqpoint{3.696000in}{3.696000in}}%
\pgfusepath{clip}%
\pgfsetbuttcap%
\pgfsetroundjoin%
\definecolor{currentfill}{rgb}{0.121569,0.466667,0.705882}%
\pgfsetfillcolor{currentfill}%
\pgfsetfillopacity{0.832624}%
\pgfsetlinewidth{1.003750pt}%
\definecolor{currentstroke}{rgb}{0.121569,0.466667,0.705882}%
\pgfsetstrokecolor{currentstroke}%
\pgfsetstrokeopacity{0.832624}%
\pgfsetdash{}{0pt}%
\pgfpathmoveto{\pgfqpoint{0.954018in}{2.534437in}}%
\pgfpathcurveto{\pgfqpoint{0.962254in}{2.534437in}}{\pgfqpoint{0.970154in}{2.537709in}}{\pgfqpoint{0.975978in}{2.543533in}}%
\pgfpathcurveto{\pgfqpoint{0.981802in}{2.549357in}}{\pgfqpoint{0.985074in}{2.557257in}}{\pgfqpoint{0.985074in}{2.565493in}}%
\pgfpathcurveto{\pgfqpoint{0.985074in}{2.573730in}}{\pgfqpoint{0.981802in}{2.581630in}}{\pgfqpoint{0.975978in}{2.587454in}}%
\pgfpathcurveto{\pgfqpoint{0.970154in}{2.593278in}}{\pgfqpoint{0.962254in}{2.596550in}}{\pgfqpoint{0.954018in}{2.596550in}}%
\pgfpathcurveto{\pgfqpoint{0.945782in}{2.596550in}}{\pgfqpoint{0.937882in}{2.593278in}}{\pgfqpoint{0.932058in}{2.587454in}}%
\pgfpathcurveto{\pgfqpoint{0.926234in}{2.581630in}}{\pgfqpoint{0.922961in}{2.573730in}}{\pgfqpoint{0.922961in}{2.565493in}}%
\pgfpathcurveto{\pgfqpoint{0.922961in}{2.557257in}}{\pgfqpoint{0.926234in}{2.549357in}}{\pgfqpoint{0.932058in}{2.543533in}}%
\pgfpathcurveto{\pgfqpoint{0.937882in}{2.537709in}}{\pgfqpoint{0.945782in}{2.534437in}}{\pgfqpoint{0.954018in}{2.534437in}}%
\pgfpathclose%
\pgfusepath{stroke,fill}%
\end{pgfscope}%
\begin{pgfscope}%
\pgfpathrectangle{\pgfqpoint{0.100000in}{0.212622in}}{\pgfqpoint{3.696000in}{3.696000in}}%
\pgfusepath{clip}%
\pgfsetbuttcap%
\pgfsetroundjoin%
\definecolor{currentfill}{rgb}{0.121569,0.466667,0.705882}%
\pgfsetfillcolor{currentfill}%
\pgfsetfillopacity{0.833156}%
\pgfsetlinewidth{1.003750pt}%
\definecolor{currentstroke}{rgb}{0.121569,0.466667,0.705882}%
\pgfsetstrokecolor{currentstroke}%
\pgfsetstrokeopacity{0.833156}%
\pgfsetdash{}{0pt}%
\pgfpathmoveto{\pgfqpoint{0.957570in}{2.531453in}}%
\pgfpathcurveto{\pgfqpoint{0.965806in}{2.531453in}}{\pgfqpoint{0.973706in}{2.534726in}}{\pgfqpoint{0.979530in}{2.540549in}}%
\pgfpathcurveto{\pgfqpoint{0.985354in}{2.546373in}}{\pgfqpoint{0.988626in}{2.554273in}}{\pgfqpoint{0.988626in}{2.562510in}}%
\pgfpathcurveto{\pgfqpoint{0.988626in}{2.570746in}}{\pgfqpoint{0.985354in}{2.578646in}}{\pgfqpoint{0.979530in}{2.584470in}}%
\pgfpathcurveto{\pgfqpoint{0.973706in}{2.590294in}}{\pgfqpoint{0.965806in}{2.593566in}}{\pgfqpoint{0.957570in}{2.593566in}}%
\pgfpathcurveto{\pgfqpoint{0.949333in}{2.593566in}}{\pgfqpoint{0.941433in}{2.590294in}}{\pgfqpoint{0.935609in}{2.584470in}}%
\pgfpathcurveto{\pgfqpoint{0.929786in}{2.578646in}}{\pgfqpoint{0.926513in}{2.570746in}}{\pgfqpoint{0.926513in}{2.562510in}}%
\pgfpathcurveto{\pgfqpoint{0.926513in}{2.554273in}}{\pgfqpoint{0.929786in}{2.546373in}}{\pgfqpoint{0.935609in}{2.540549in}}%
\pgfpathcurveto{\pgfqpoint{0.941433in}{2.534726in}}{\pgfqpoint{0.949333in}{2.531453in}}{\pgfqpoint{0.957570in}{2.531453in}}%
\pgfpathclose%
\pgfusepath{stroke,fill}%
\end{pgfscope}%
\begin{pgfscope}%
\pgfpathrectangle{\pgfqpoint{0.100000in}{0.212622in}}{\pgfqpoint{3.696000in}{3.696000in}}%
\pgfusepath{clip}%
\pgfsetbuttcap%
\pgfsetroundjoin%
\definecolor{currentfill}{rgb}{0.121569,0.466667,0.705882}%
\pgfsetfillcolor{currentfill}%
\pgfsetfillopacity{0.833443}%
\pgfsetlinewidth{1.003750pt}%
\definecolor{currentstroke}{rgb}{0.121569,0.466667,0.705882}%
\pgfsetstrokecolor{currentstroke}%
\pgfsetstrokeopacity{0.833443}%
\pgfsetdash{}{0pt}%
\pgfpathmoveto{\pgfqpoint{2.782218in}{2.020667in}}%
\pgfpathcurveto{\pgfqpoint{2.790454in}{2.020667in}}{\pgfqpoint{2.798354in}{2.023940in}}{\pgfqpoint{2.804178in}{2.029763in}}%
\pgfpathcurveto{\pgfqpoint{2.810002in}{2.035587in}}{\pgfqpoint{2.813274in}{2.043487in}}{\pgfqpoint{2.813274in}{2.051724in}}%
\pgfpathcurveto{\pgfqpoint{2.813274in}{2.059960in}}{\pgfqpoint{2.810002in}{2.067860in}}{\pgfqpoint{2.804178in}{2.073684in}}%
\pgfpathcurveto{\pgfqpoint{2.798354in}{2.079508in}}{\pgfqpoint{2.790454in}{2.082780in}}{\pgfqpoint{2.782218in}{2.082780in}}%
\pgfpathcurveto{\pgfqpoint{2.773982in}{2.082780in}}{\pgfqpoint{2.766082in}{2.079508in}}{\pgfqpoint{2.760258in}{2.073684in}}%
\pgfpathcurveto{\pgfqpoint{2.754434in}{2.067860in}}{\pgfqpoint{2.751161in}{2.059960in}}{\pgfqpoint{2.751161in}{2.051724in}}%
\pgfpathcurveto{\pgfqpoint{2.751161in}{2.043487in}}{\pgfqpoint{2.754434in}{2.035587in}}{\pgfqpoint{2.760258in}{2.029763in}}%
\pgfpathcurveto{\pgfqpoint{2.766082in}{2.023940in}}{\pgfqpoint{2.773982in}{2.020667in}}{\pgfqpoint{2.782218in}{2.020667in}}%
\pgfpathclose%
\pgfusepath{stroke,fill}%
\end{pgfscope}%
\begin{pgfscope}%
\pgfpathrectangle{\pgfqpoint{0.100000in}{0.212622in}}{\pgfqpoint{3.696000in}{3.696000in}}%
\pgfusepath{clip}%
\pgfsetbuttcap%
\pgfsetroundjoin%
\definecolor{currentfill}{rgb}{0.121569,0.466667,0.705882}%
\pgfsetfillcolor{currentfill}%
\pgfsetfillopacity{0.833521}%
\pgfsetlinewidth{1.003750pt}%
\definecolor{currentstroke}{rgb}{0.121569,0.466667,0.705882}%
\pgfsetstrokecolor{currentstroke}%
\pgfsetstrokeopacity{0.833521}%
\pgfsetdash{}{0pt}%
\pgfpathmoveto{\pgfqpoint{0.959419in}{2.530137in}}%
\pgfpathcurveto{\pgfqpoint{0.967655in}{2.530137in}}{\pgfqpoint{0.975555in}{2.533409in}}{\pgfqpoint{0.981379in}{2.539233in}}%
\pgfpathcurveto{\pgfqpoint{0.987203in}{2.545057in}}{\pgfqpoint{0.990475in}{2.552957in}}{\pgfqpoint{0.990475in}{2.561194in}}%
\pgfpathcurveto{\pgfqpoint{0.990475in}{2.569430in}}{\pgfqpoint{0.987203in}{2.577330in}}{\pgfqpoint{0.981379in}{2.583154in}}%
\pgfpathcurveto{\pgfqpoint{0.975555in}{2.588978in}}{\pgfqpoint{0.967655in}{2.592250in}}{\pgfqpoint{0.959419in}{2.592250in}}%
\pgfpathcurveto{\pgfqpoint{0.951182in}{2.592250in}}{\pgfqpoint{0.943282in}{2.588978in}}{\pgfqpoint{0.937458in}{2.583154in}}%
\pgfpathcurveto{\pgfqpoint{0.931634in}{2.577330in}}{\pgfqpoint{0.928362in}{2.569430in}}{\pgfqpoint{0.928362in}{2.561194in}}%
\pgfpathcurveto{\pgfqpoint{0.928362in}{2.552957in}}{\pgfqpoint{0.931634in}{2.545057in}}{\pgfqpoint{0.937458in}{2.539233in}}%
\pgfpathcurveto{\pgfqpoint{0.943282in}{2.533409in}}{\pgfqpoint{0.951182in}{2.530137in}}{\pgfqpoint{0.959419in}{2.530137in}}%
\pgfpathclose%
\pgfusepath{stroke,fill}%
\end{pgfscope}%
\begin{pgfscope}%
\pgfpathrectangle{\pgfqpoint{0.100000in}{0.212622in}}{\pgfqpoint{3.696000in}{3.696000in}}%
\pgfusepath{clip}%
\pgfsetbuttcap%
\pgfsetroundjoin%
\definecolor{currentfill}{rgb}{0.121569,0.466667,0.705882}%
\pgfsetfillcolor{currentfill}%
\pgfsetfillopacity{0.833894}%
\pgfsetlinewidth{1.003750pt}%
\definecolor{currentstroke}{rgb}{0.121569,0.466667,0.705882}%
\pgfsetstrokecolor{currentstroke}%
\pgfsetstrokeopacity{0.833894}%
\pgfsetdash{}{0pt}%
\pgfpathmoveto{\pgfqpoint{0.963187in}{2.526746in}}%
\pgfpathcurveto{\pgfqpoint{0.971423in}{2.526746in}}{\pgfqpoint{0.979323in}{2.530018in}}{\pgfqpoint{0.985147in}{2.535842in}}%
\pgfpathcurveto{\pgfqpoint{0.990971in}{2.541666in}}{\pgfqpoint{0.994243in}{2.549566in}}{\pgfqpoint{0.994243in}{2.557802in}}%
\pgfpathcurveto{\pgfqpoint{0.994243in}{2.566039in}}{\pgfqpoint{0.990971in}{2.573939in}}{\pgfqpoint{0.985147in}{2.579763in}}%
\pgfpathcurveto{\pgfqpoint{0.979323in}{2.585586in}}{\pgfqpoint{0.971423in}{2.588859in}}{\pgfqpoint{0.963187in}{2.588859in}}%
\pgfpathcurveto{\pgfqpoint{0.954950in}{2.588859in}}{\pgfqpoint{0.947050in}{2.585586in}}{\pgfqpoint{0.941226in}{2.579763in}}%
\pgfpathcurveto{\pgfqpoint{0.935403in}{2.573939in}}{\pgfqpoint{0.932130in}{2.566039in}}{\pgfqpoint{0.932130in}{2.557802in}}%
\pgfpathcurveto{\pgfqpoint{0.932130in}{2.549566in}}{\pgfqpoint{0.935403in}{2.541666in}}{\pgfqpoint{0.941226in}{2.535842in}}%
\pgfpathcurveto{\pgfqpoint{0.947050in}{2.530018in}}{\pgfqpoint{0.954950in}{2.526746in}}{\pgfqpoint{0.963187in}{2.526746in}}%
\pgfpathclose%
\pgfusepath{stroke,fill}%
\end{pgfscope}%
\begin{pgfscope}%
\pgfpathrectangle{\pgfqpoint{0.100000in}{0.212622in}}{\pgfqpoint{3.696000in}{3.696000in}}%
\pgfusepath{clip}%
\pgfsetbuttcap%
\pgfsetroundjoin%
\definecolor{currentfill}{rgb}{0.121569,0.466667,0.705882}%
\pgfsetfillcolor{currentfill}%
\pgfsetfillopacity{0.835190}%
\pgfsetlinewidth{1.003750pt}%
\definecolor{currentstroke}{rgb}{0.121569,0.466667,0.705882}%
\pgfsetstrokecolor{currentstroke}%
\pgfsetstrokeopacity{0.835190}%
\pgfsetdash{}{0pt}%
\pgfpathmoveto{\pgfqpoint{0.969254in}{2.522927in}}%
\pgfpathcurveto{\pgfqpoint{0.977490in}{2.522927in}}{\pgfqpoint{0.985390in}{2.526199in}}{\pgfqpoint{0.991214in}{2.532023in}}%
\pgfpathcurveto{\pgfqpoint{0.997038in}{2.537847in}}{\pgfqpoint{1.000310in}{2.545747in}}{\pgfqpoint{1.000310in}{2.553983in}}%
\pgfpathcurveto{\pgfqpoint{1.000310in}{2.562220in}}{\pgfqpoint{0.997038in}{2.570120in}}{\pgfqpoint{0.991214in}{2.575944in}}%
\pgfpathcurveto{\pgfqpoint{0.985390in}{2.581768in}}{\pgfqpoint{0.977490in}{2.585040in}}{\pgfqpoint{0.969254in}{2.585040in}}%
\pgfpathcurveto{\pgfqpoint{0.961018in}{2.585040in}}{\pgfqpoint{0.953118in}{2.581768in}}{\pgfqpoint{0.947294in}{2.575944in}}%
\pgfpathcurveto{\pgfqpoint{0.941470in}{2.570120in}}{\pgfqpoint{0.938197in}{2.562220in}}{\pgfqpoint{0.938197in}{2.553983in}}%
\pgfpathcurveto{\pgfqpoint{0.938197in}{2.545747in}}{\pgfqpoint{0.941470in}{2.537847in}}{\pgfqpoint{0.947294in}{2.532023in}}%
\pgfpathcurveto{\pgfqpoint{0.953118in}{2.526199in}}{\pgfqpoint{0.961018in}{2.522927in}}{\pgfqpoint{0.969254in}{2.522927in}}%
\pgfpathclose%
\pgfusepath{stroke,fill}%
\end{pgfscope}%
\begin{pgfscope}%
\pgfpathrectangle{\pgfqpoint{0.100000in}{0.212622in}}{\pgfqpoint{3.696000in}{3.696000in}}%
\pgfusepath{clip}%
\pgfsetbuttcap%
\pgfsetroundjoin%
\definecolor{currentfill}{rgb}{0.121569,0.466667,0.705882}%
\pgfsetfillcolor{currentfill}%
\pgfsetfillopacity{0.836035}%
\pgfsetlinewidth{1.003750pt}%
\definecolor{currentstroke}{rgb}{0.121569,0.466667,0.705882}%
\pgfsetstrokecolor{currentstroke}%
\pgfsetstrokeopacity{0.836035}%
\pgfsetdash{}{0pt}%
\pgfpathmoveto{\pgfqpoint{0.982843in}{2.512408in}}%
\pgfpathcurveto{\pgfqpoint{0.991079in}{2.512408in}}{\pgfqpoint{0.998979in}{2.515680in}}{\pgfqpoint{1.004803in}{2.521504in}}%
\pgfpathcurveto{\pgfqpoint{1.010627in}{2.527328in}}{\pgfqpoint{1.013899in}{2.535228in}}{\pgfqpoint{1.013899in}{2.543464in}}%
\pgfpathcurveto{\pgfqpoint{1.013899in}{2.551700in}}{\pgfqpoint{1.010627in}{2.559601in}}{\pgfqpoint{1.004803in}{2.565424in}}%
\pgfpathcurveto{\pgfqpoint{0.998979in}{2.571248in}}{\pgfqpoint{0.991079in}{2.574521in}}{\pgfqpoint{0.982843in}{2.574521in}}%
\pgfpathcurveto{\pgfqpoint{0.974607in}{2.574521in}}{\pgfqpoint{0.966707in}{2.571248in}}{\pgfqpoint{0.960883in}{2.565424in}}%
\pgfpathcurveto{\pgfqpoint{0.955059in}{2.559601in}}{\pgfqpoint{0.951786in}{2.551700in}}{\pgfqpoint{0.951786in}{2.543464in}}%
\pgfpathcurveto{\pgfqpoint{0.951786in}{2.535228in}}{\pgfqpoint{0.955059in}{2.527328in}}{\pgfqpoint{0.960883in}{2.521504in}}%
\pgfpathcurveto{\pgfqpoint{0.966707in}{2.515680in}}{\pgfqpoint{0.974607in}{2.512408in}}{\pgfqpoint{0.982843in}{2.512408in}}%
\pgfpathclose%
\pgfusepath{stroke,fill}%
\end{pgfscope}%
\begin{pgfscope}%
\pgfpathrectangle{\pgfqpoint{0.100000in}{0.212622in}}{\pgfqpoint{3.696000in}{3.696000in}}%
\pgfusepath{clip}%
\pgfsetbuttcap%
\pgfsetroundjoin%
\definecolor{currentfill}{rgb}{0.121569,0.466667,0.705882}%
\pgfsetfillcolor{currentfill}%
\pgfsetfillopacity{0.838861}%
\pgfsetlinewidth{1.003750pt}%
\definecolor{currentstroke}{rgb}{0.121569,0.466667,0.705882}%
\pgfsetstrokecolor{currentstroke}%
\pgfsetstrokeopacity{0.838861}%
\pgfsetdash{}{0pt}%
\pgfpathmoveto{\pgfqpoint{2.771357in}{2.012704in}}%
\pgfpathcurveto{\pgfqpoint{2.779594in}{2.012704in}}{\pgfqpoint{2.787494in}{2.015976in}}{\pgfqpoint{2.793318in}{2.021800in}}%
\pgfpathcurveto{\pgfqpoint{2.799141in}{2.027624in}}{\pgfqpoint{2.802414in}{2.035524in}}{\pgfqpoint{2.802414in}{2.043761in}}%
\pgfpathcurveto{\pgfqpoint{2.802414in}{2.051997in}}{\pgfqpoint{2.799141in}{2.059897in}}{\pgfqpoint{2.793318in}{2.065721in}}%
\pgfpathcurveto{\pgfqpoint{2.787494in}{2.071545in}}{\pgfqpoint{2.779594in}{2.074817in}}{\pgfqpoint{2.771357in}{2.074817in}}%
\pgfpathcurveto{\pgfqpoint{2.763121in}{2.074817in}}{\pgfqpoint{2.755221in}{2.071545in}}{\pgfqpoint{2.749397in}{2.065721in}}%
\pgfpathcurveto{\pgfqpoint{2.743573in}{2.059897in}}{\pgfqpoint{2.740301in}{2.051997in}}{\pgfqpoint{2.740301in}{2.043761in}}%
\pgfpathcurveto{\pgfqpoint{2.740301in}{2.035524in}}{\pgfqpoint{2.743573in}{2.027624in}}{\pgfqpoint{2.749397in}{2.021800in}}%
\pgfpathcurveto{\pgfqpoint{2.755221in}{2.015976in}}{\pgfqpoint{2.763121in}{2.012704in}}{\pgfqpoint{2.771357in}{2.012704in}}%
\pgfpathclose%
\pgfusepath{stroke,fill}%
\end{pgfscope}%
\begin{pgfscope}%
\pgfpathrectangle{\pgfqpoint{0.100000in}{0.212622in}}{\pgfqpoint{3.696000in}{3.696000in}}%
\pgfusepath{clip}%
\pgfsetbuttcap%
\pgfsetroundjoin%
\definecolor{currentfill}{rgb}{0.121569,0.466667,0.705882}%
\pgfsetfillcolor{currentfill}%
\pgfsetfillopacity{0.839213}%
\pgfsetlinewidth{1.003750pt}%
\definecolor{currentstroke}{rgb}{0.121569,0.466667,0.705882}%
\pgfsetstrokecolor{currentstroke}%
\pgfsetstrokeopacity{0.839213}%
\pgfsetdash{}{0pt}%
\pgfpathmoveto{\pgfqpoint{1.006031in}{2.499924in}}%
\pgfpathcurveto{\pgfqpoint{1.014267in}{2.499924in}}{\pgfqpoint{1.022167in}{2.503196in}}{\pgfqpoint{1.027991in}{2.509020in}}%
\pgfpathcurveto{\pgfqpoint{1.033815in}{2.514844in}}{\pgfqpoint{1.037088in}{2.522744in}}{\pgfqpoint{1.037088in}{2.530981in}}%
\pgfpathcurveto{\pgfqpoint{1.037088in}{2.539217in}}{\pgfqpoint{1.033815in}{2.547117in}}{\pgfqpoint{1.027991in}{2.552941in}}%
\pgfpathcurveto{\pgfqpoint{1.022167in}{2.558765in}}{\pgfqpoint{1.014267in}{2.562037in}}{\pgfqpoint{1.006031in}{2.562037in}}%
\pgfpathcurveto{\pgfqpoint{0.997795in}{2.562037in}}{\pgfqpoint{0.989895in}{2.558765in}}{\pgfqpoint{0.984071in}{2.552941in}}%
\pgfpathcurveto{\pgfqpoint{0.978247in}{2.547117in}}{\pgfqpoint{0.974975in}{2.539217in}}{\pgfqpoint{0.974975in}{2.530981in}}%
\pgfpathcurveto{\pgfqpoint{0.974975in}{2.522744in}}{\pgfqpoint{0.978247in}{2.514844in}}{\pgfqpoint{0.984071in}{2.509020in}}%
\pgfpathcurveto{\pgfqpoint{0.989895in}{2.503196in}}{\pgfqpoint{0.997795in}{2.499924in}}{\pgfqpoint{1.006031in}{2.499924in}}%
\pgfpathclose%
\pgfusepath{stroke,fill}%
\end{pgfscope}%
\begin{pgfscope}%
\pgfpathrectangle{\pgfqpoint{0.100000in}{0.212622in}}{\pgfqpoint{3.696000in}{3.696000in}}%
\pgfusepath{clip}%
\pgfsetbuttcap%
\pgfsetroundjoin%
\definecolor{currentfill}{rgb}{0.121569,0.466667,0.705882}%
\pgfsetfillcolor{currentfill}%
\pgfsetfillopacity{0.840593}%
\pgfsetlinewidth{1.003750pt}%
\definecolor{currentstroke}{rgb}{0.121569,0.466667,0.705882}%
\pgfsetstrokecolor{currentstroke}%
\pgfsetstrokeopacity{0.840593}%
\pgfsetdash{}{0pt}%
\pgfpathmoveto{\pgfqpoint{1.025719in}{2.488689in}}%
\pgfpathcurveto{\pgfqpoint{1.033955in}{2.488689in}}{\pgfqpoint{1.041855in}{2.491962in}}{\pgfqpoint{1.047679in}{2.497786in}}%
\pgfpathcurveto{\pgfqpoint{1.053503in}{2.503609in}}{\pgfqpoint{1.056776in}{2.511509in}}{\pgfqpoint{1.056776in}{2.519746in}}%
\pgfpathcurveto{\pgfqpoint{1.056776in}{2.527982in}}{\pgfqpoint{1.053503in}{2.535882in}}{\pgfqpoint{1.047679in}{2.541706in}}%
\pgfpathcurveto{\pgfqpoint{1.041855in}{2.547530in}}{\pgfqpoint{1.033955in}{2.550802in}}{\pgfqpoint{1.025719in}{2.550802in}}%
\pgfpathcurveto{\pgfqpoint{1.017483in}{2.550802in}}{\pgfqpoint{1.009583in}{2.547530in}}{\pgfqpoint{1.003759in}{2.541706in}}%
\pgfpathcurveto{\pgfqpoint{0.997935in}{2.535882in}}{\pgfqpoint{0.994663in}{2.527982in}}{\pgfqpoint{0.994663in}{2.519746in}}%
\pgfpathcurveto{\pgfqpoint{0.994663in}{2.511509in}}{\pgfqpoint{0.997935in}{2.503609in}}{\pgfqpoint{1.003759in}{2.497786in}}%
\pgfpathcurveto{\pgfqpoint{1.009583in}{2.491962in}}{\pgfqpoint{1.017483in}{2.488689in}}{\pgfqpoint{1.025719in}{2.488689in}}%
\pgfpathclose%
\pgfusepath{stroke,fill}%
\end{pgfscope}%
\begin{pgfscope}%
\pgfpathrectangle{\pgfqpoint{0.100000in}{0.212622in}}{\pgfqpoint{3.696000in}{3.696000in}}%
\pgfusepath{clip}%
\pgfsetbuttcap%
\pgfsetroundjoin%
\definecolor{currentfill}{rgb}{0.121569,0.466667,0.705882}%
\pgfsetfillcolor{currentfill}%
\pgfsetfillopacity{0.842586}%
\pgfsetlinewidth{1.003750pt}%
\definecolor{currentstroke}{rgb}{0.121569,0.466667,0.705882}%
\pgfsetstrokecolor{currentstroke}%
\pgfsetstrokeopacity{0.842586}%
\pgfsetdash{}{0pt}%
\pgfpathmoveto{\pgfqpoint{1.038550in}{2.485036in}}%
\pgfpathcurveto{\pgfqpoint{1.046786in}{2.485036in}}{\pgfqpoint{1.054686in}{2.488308in}}{\pgfqpoint{1.060510in}{2.494132in}}%
\pgfpathcurveto{\pgfqpoint{1.066334in}{2.499956in}}{\pgfqpoint{1.069606in}{2.507856in}}{\pgfqpoint{1.069606in}{2.516092in}}%
\pgfpathcurveto{\pgfqpoint{1.069606in}{2.524329in}}{\pgfqpoint{1.066334in}{2.532229in}}{\pgfqpoint{1.060510in}{2.538053in}}%
\pgfpathcurveto{\pgfqpoint{1.054686in}{2.543876in}}{\pgfqpoint{1.046786in}{2.547149in}}{\pgfqpoint{1.038550in}{2.547149in}}%
\pgfpathcurveto{\pgfqpoint{1.030313in}{2.547149in}}{\pgfqpoint{1.022413in}{2.543876in}}{\pgfqpoint{1.016589in}{2.538053in}}%
\pgfpathcurveto{\pgfqpoint{1.010765in}{2.532229in}}{\pgfqpoint{1.007493in}{2.524329in}}{\pgfqpoint{1.007493in}{2.516092in}}%
\pgfpathcurveto{\pgfqpoint{1.007493in}{2.507856in}}{\pgfqpoint{1.010765in}{2.499956in}}{\pgfqpoint{1.016589in}{2.494132in}}%
\pgfpathcurveto{\pgfqpoint{1.022413in}{2.488308in}}{\pgfqpoint{1.030313in}{2.485036in}}{\pgfqpoint{1.038550in}{2.485036in}}%
\pgfpathclose%
\pgfusepath{stroke,fill}%
\end{pgfscope}%
\begin{pgfscope}%
\pgfpathrectangle{\pgfqpoint{0.100000in}{0.212622in}}{\pgfqpoint{3.696000in}{3.696000in}}%
\pgfusepath{clip}%
\pgfsetbuttcap%
\pgfsetroundjoin%
\definecolor{currentfill}{rgb}{0.121569,0.466667,0.705882}%
\pgfsetfillcolor{currentfill}%
\pgfsetfillopacity{0.843141}%
\pgfsetlinewidth{1.003750pt}%
\definecolor{currentstroke}{rgb}{0.121569,0.466667,0.705882}%
\pgfsetstrokecolor{currentstroke}%
\pgfsetstrokeopacity{0.843141}%
\pgfsetdash{}{0pt}%
\pgfpathmoveto{\pgfqpoint{1.046932in}{2.480212in}}%
\pgfpathcurveto{\pgfqpoint{1.055168in}{2.480212in}}{\pgfqpoint{1.063068in}{2.483484in}}{\pgfqpoint{1.068892in}{2.489308in}}%
\pgfpathcurveto{\pgfqpoint{1.074716in}{2.495132in}}{\pgfqpoint{1.077988in}{2.503032in}}{\pgfqpoint{1.077988in}{2.511268in}}%
\pgfpathcurveto{\pgfqpoint{1.077988in}{2.519504in}}{\pgfqpoint{1.074716in}{2.527405in}}{\pgfqpoint{1.068892in}{2.533228in}}%
\pgfpathcurveto{\pgfqpoint{1.063068in}{2.539052in}}{\pgfqpoint{1.055168in}{2.542325in}}{\pgfqpoint{1.046932in}{2.542325in}}%
\pgfpathcurveto{\pgfqpoint{1.038696in}{2.542325in}}{\pgfqpoint{1.030796in}{2.539052in}}{\pgfqpoint{1.024972in}{2.533228in}}%
\pgfpathcurveto{\pgfqpoint{1.019148in}{2.527405in}}{\pgfqpoint{1.015875in}{2.519504in}}{\pgfqpoint{1.015875in}{2.511268in}}%
\pgfpathcurveto{\pgfqpoint{1.015875in}{2.503032in}}{\pgfqpoint{1.019148in}{2.495132in}}{\pgfqpoint{1.024972in}{2.489308in}}%
\pgfpathcurveto{\pgfqpoint{1.030796in}{2.483484in}}{\pgfqpoint{1.038696in}{2.480212in}}{\pgfqpoint{1.046932in}{2.480212in}}%
\pgfpathclose%
\pgfusepath{stroke,fill}%
\end{pgfscope}%
\begin{pgfscope}%
\pgfpathrectangle{\pgfqpoint{0.100000in}{0.212622in}}{\pgfqpoint{3.696000in}{3.696000in}}%
\pgfusepath{clip}%
\pgfsetbuttcap%
\pgfsetroundjoin%
\definecolor{currentfill}{rgb}{0.121569,0.466667,0.705882}%
\pgfsetfillcolor{currentfill}%
\pgfsetfillopacity{0.843704}%
\pgfsetlinewidth{1.003750pt}%
\definecolor{currentstroke}{rgb}{0.121569,0.466667,0.705882}%
\pgfsetstrokecolor{currentstroke}%
\pgfsetstrokeopacity{0.843704}%
\pgfsetdash{}{0pt}%
\pgfpathmoveto{\pgfqpoint{1.050512in}{2.478716in}}%
\pgfpathcurveto{\pgfqpoint{1.058748in}{2.478716in}}{\pgfqpoint{1.066649in}{2.481988in}}{\pgfqpoint{1.072472in}{2.487812in}}%
\pgfpathcurveto{\pgfqpoint{1.078296in}{2.493636in}}{\pgfqpoint{1.081569in}{2.501536in}}{\pgfqpoint{1.081569in}{2.509772in}}%
\pgfpathcurveto{\pgfqpoint{1.081569in}{2.518008in}}{\pgfqpoint{1.078296in}{2.525908in}}{\pgfqpoint{1.072472in}{2.531732in}}%
\pgfpathcurveto{\pgfqpoint{1.066649in}{2.537556in}}{\pgfqpoint{1.058748in}{2.540829in}}{\pgfqpoint{1.050512in}{2.540829in}}%
\pgfpathcurveto{\pgfqpoint{1.042276in}{2.540829in}}{\pgfqpoint{1.034376in}{2.537556in}}{\pgfqpoint{1.028552in}{2.531732in}}%
\pgfpathcurveto{\pgfqpoint{1.022728in}{2.525908in}}{\pgfqpoint{1.019456in}{2.518008in}}{\pgfqpoint{1.019456in}{2.509772in}}%
\pgfpathcurveto{\pgfqpoint{1.019456in}{2.501536in}}{\pgfqpoint{1.022728in}{2.493636in}}{\pgfqpoint{1.028552in}{2.487812in}}%
\pgfpathcurveto{\pgfqpoint{1.034376in}{2.481988in}}{\pgfqpoint{1.042276in}{2.478716in}}{\pgfqpoint{1.050512in}{2.478716in}}%
\pgfpathclose%
\pgfusepath{stroke,fill}%
\end{pgfscope}%
\begin{pgfscope}%
\pgfpathrectangle{\pgfqpoint{0.100000in}{0.212622in}}{\pgfqpoint{3.696000in}{3.696000in}}%
\pgfusepath{clip}%
\pgfsetbuttcap%
\pgfsetroundjoin%
\definecolor{currentfill}{rgb}{0.121569,0.466667,0.705882}%
\pgfsetfillcolor{currentfill}%
\pgfsetfillopacity{0.844286}%
\pgfsetlinewidth{1.003750pt}%
\definecolor{currentstroke}{rgb}{0.121569,0.466667,0.705882}%
\pgfsetstrokecolor{currentstroke}%
\pgfsetstrokeopacity{0.844286}%
\pgfsetdash{}{0pt}%
\pgfpathmoveto{\pgfqpoint{1.057717in}{2.475402in}}%
\pgfpathcurveto{\pgfqpoint{1.065953in}{2.475402in}}{\pgfqpoint{1.073853in}{2.478674in}}{\pgfqpoint{1.079677in}{2.484498in}}%
\pgfpathcurveto{\pgfqpoint{1.085501in}{2.490322in}}{\pgfqpoint{1.088773in}{2.498222in}}{\pgfqpoint{1.088773in}{2.506458in}}%
\pgfpathcurveto{\pgfqpoint{1.088773in}{2.514695in}}{\pgfqpoint{1.085501in}{2.522595in}}{\pgfqpoint{1.079677in}{2.528418in}}%
\pgfpathcurveto{\pgfqpoint{1.073853in}{2.534242in}}{\pgfqpoint{1.065953in}{2.537515in}}{\pgfqpoint{1.057717in}{2.537515in}}%
\pgfpathcurveto{\pgfqpoint{1.049481in}{2.537515in}}{\pgfqpoint{1.041581in}{2.534242in}}{\pgfqpoint{1.035757in}{2.528418in}}%
\pgfpathcurveto{\pgfqpoint{1.029933in}{2.522595in}}{\pgfqpoint{1.026660in}{2.514695in}}{\pgfqpoint{1.026660in}{2.506458in}}%
\pgfpathcurveto{\pgfqpoint{1.026660in}{2.498222in}}{\pgfqpoint{1.029933in}{2.490322in}}{\pgfqpoint{1.035757in}{2.484498in}}%
\pgfpathcurveto{\pgfqpoint{1.041581in}{2.478674in}}{\pgfqpoint{1.049481in}{2.475402in}}{\pgfqpoint{1.057717in}{2.475402in}}%
\pgfpathclose%
\pgfusepath{stroke,fill}%
\end{pgfscope}%
\begin{pgfscope}%
\pgfpathrectangle{\pgfqpoint{0.100000in}{0.212622in}}{\pgfqpoint{3.696000in}{3.696000in}}%
\pgfusepath{clip}%
\pgfsetbuttcap%
\pgfsetroundjoin%
\definecolor{currentfill}{rgb}{0.121569,0.466667,0.705882}%
\pgfsetfillcolor{currentfill}%
\pgfsetfillopacity{0.844754}%
\pgfsetlinewidth{1.003750pt}%
\definecolor{currentstroke}{rgb}{0.121569,0.466667,0.705882}%
\pgfsetstrokecolor{currentstroke}%
\pgfsetstrokeopacity{0.844754}%
\pgfsetdash{}{0pt}%
\pgfpathmoveto{\pgfqpoint{2.752887in}{2.007090in}}%
\pgfpathcurveto{\pgfqpoint{2.761123in}{2.007090in}}{\pgfqpoint{2.769023in}{2.010362in}}{\pgfqpoint{2.774847in}{2.016186in}}%
\pgfpathcurveto{\pgfqpoint{2.780671in}{2.022010in}}{\pgfqpoint{2.783943in}{2.029910in}}{\pgfqpoint{2.783943in}{2.038146in}}%
\pgfpathcurveto{\pgfqpoint{2.783943in}{2.046383in}}{\pgfqpoint{2.780671in}{2.054283in}}{\pgfqpoint{2.774847in}{2.060107in}}%
\pgfpathcurveto{\pgfqpoint{2.769023in}{2.065931in}}{\pgfqpoint{2.761123in}{2.069203in}}{\pgfqpoint{2.752887in}{2.069203in}}%
\pgfpathcurveto{\pgfqpoint{2.744650in}{2.069203in}}{\pgfqpoint{2.736750in}{2.065931in}}{\pgfqpoint{2.730926in}{2.060107in}}%
\pgfpathcurveto{\pgfqpoint{2.725102in}{2.054283in}}{\pgfqpoint{2.721830in}{2.046383in}}{\pgfqpoint{2.721830in}{2.038146in}}%
\pgfpathcurveto{\pgfqpoint{2.721830in}{2.029910in}}{\pgfqpoint{2.725102in}{2.022010in}}{\pgfqpoint{2.730926in}{2.016186in}}%
\pgfpathcurveto{\pgfqpoint{2.736750in}{2.010362in}}{\pgfqpoint{2.744650in}{2.007090in}}{\pgfqpoint{2.752887in}{2.007090in}}%
\pgfpathclose%
\pgfusepath{stroke,fill}%
\end{pgfscope}%
\begin{pgfscope}%
\pgfpathrectangle{\pgfqpoint{0.100000in}{0.212622in}}{\pgfqpoint{3.696000in}{3.696000in}}%
\pgfusepath{clip}%
\pgfsetbuttcap%
\pgfsetroundjoin%
\definecolor{currentfill}{rgb}{0.121569,0.466667,0.705882}%
\pgfsetfillcolor{currentfill}%
\pgfsetfillopacity{0.846421}%
\pgfsetlinewidth{1.003750pt}%
\definecolor{currentstroke}{rgb}{0.121569,0.466667,0.705882}%
\pgfsetstrokecolor{currentstroke}%
\pgfsetstrokeopacity{0.846421}%
\pgfsetdash{}{0pt}%
\pgfpathmoveto{\pgfqpoint{1.069656in}{2.472821in}}%
\pgfpathcurveto{\pgfqpoint{1.077892in}{2.472821in}}{\pgfqpoint{1.085792in}{2.476093in}}{\pgfqpoint{1.091616in}{2.481917in}}%
\pgfpathcurveto{\pgfqpoint{1.097440in}{2.487741in}}{\pgfqpoint{1.100712in}{2.495641in}}{\pgfqpoint{1.100712in}{2.503877in}}%
\pgfpathcurveto{\pgfqpoint{1.100712in}{2.512114in}}{\pgfqpoint{1.097440in}{2.520014in}}{\pgfqpoint{1.091616in}{2.525838in}}%
\pgfpathcurveto{\pgfqpoint{1.085792in}{2.531662in}}{\pgfqpoint{1.077892in}{2.534934in}}{\pgfqpoint{1.069656in}{2.534934in}}%
\pgfpathcurveto{\pgfqpoint{1.061419in}{2.534934in}}{\pgfqpoint{1.053519in}{2.531662in}}{\pgfqpoint{1.047695in}{2.525838in}}%
\pgfpathcurveto{\pgfqpoint{1.041871in}{2.520014in}}{\pgfqpoint{1.038599in}{2.512114in}}{\pgfqpoint{1.038599in}{2.503877in}}%
\pgfpathcurveto{\pgfqpoint{1.038599in}{2.495641in}}{\pgfqpoint{1.041871in}{2.487741in}}{\pgfqpoint{1.047695in}{2.481917in}}%
\pgfpathcurveto{\pgfqpoint{1.053519in}{2.476093in}}{\pgfqpoint{1.061419in}{2.472821in}}{\pgfqpoint{1.069656in}{2.472821in}}%
\pgfpathclose%
\pgfusepath{stroke,fill}%
\end{pgfscope}%
\begin{pgfscope}%
\pgfpathrectangle{\pgfqpoint{0.100000in}{0.212622in}}{\pgfqpoint{3.696000in}{3.696000in}}%
\pgfusepath{clip}%
\pgfsetbuttcap%
\pgfsetroundjoin%
\definecolor{currentfill}{rgb}{0.121569,0.466667,0.705882}%
\pgfsetfillcolor{currentfill}%
\pgfsetfillopacity{0.848627}%
\pgfsetlinewidth{1.003750pt}%
\definecolor{currentstroke}{rgb}{0.121569,0.466667,0.705882}%
\pgfsetstrokecolor{currentstroke}%
\pgfsetstrokeopacity{0.848627}%
\pgfsetdash{}{0pt}%
\pgfpathmoveto{\pgfqpoint{1.092741in}{2.460891in}}%
\pgfpathcurveto{\pgfqpoint{1.100977in}{2.460891in}}{\pgfqpoint{1.108877in}{2.464164in}}{\pgfqpoint{1.114701in}{2.469988in}}%
\pgfpathcurveto{\pgfqpoint{1.120525in}{2.475812in}}{\pgfqpoint{1.123797in}{2.483712in}}{\pgfqpoint{1.123797in}{2.491948in}}%
\pgfpathcurveto{\pgfqpoint{1.123797in}{2.500184in}}{\pgfqpoint{1.120525in}{2.508084in}}{\pgfqpoint{1.114701in}{2.513908in}}%
\pgfpathcurveto{\pgfqpoint{1.108877in}{2.519732in}}{\pgfqpoint{1.100977in}{2.523004in}}{\pgfqpoint{1.092741in}{2.523004in}}%
\pgfpathcurveto{\pgfqpoint{1.084505in}{2.523004in}}{\pgfqpoint{1.076605in}{2.519732in}}{\pgfqpoint{1.070781in}{2.513908in}}%
\pgfpathcurveto{\pgfqpoint{1.064957in}{2.508084in}}{\pgfqpoint{1.061684in}{2.500184in}}{\pgfqpoint{1.061684in}{2.491948in}}%
\pgfpathcurveto{\pgfqpoint{1.061684in}{2.483712in}}{\pgfqpoint{1.064957in}{2.475812in}}{\pgfqpoint{1.070781in}{2.469988in}}%
\pgfpathcurveto{\pgfqpoint{1.076605in}{2.464164in}}{\pgfqpoint{1.084505in}{2.460891in}}{\pgfqpoint{1.092741in}{2.460891in}}%
\pgfpathclose%
\pgfusepath{stroke,fill}%
\end{pgfscope}%
\begin{pgfscope}%
\pgfpathrectangle{\pgfqpoint{0.100000in}{0.212622in}}{\pgfqpoint{3.696000in}{3.696000in}}%
\pgfusepath{clip}%
\pgfsetbuttcap%
\pgfsetroundjoin%
\definecolor{currentfill}{rgb}{0.121569,0.466667,0.705882}%
\pgfsetfillcolor{currentfill}%
\pgfsetfillopacity{0.851290}%
\pgfsetlinewidth{1.003750pt}%
\definecolor{currentstroke}{rgb}{0.121569,0.466667,0.705882}%
\pgfsetstrokecolor{currentstroke}%
\pgfsetstrokeopacity{0.851290}%
\pgfsetdash{}{0pt}%
\pgfpathmoveto{\pgfqpoint{1.112046in}{2.450850in}}%
\pgfpathcurveto{\pgfqpoint{1.120282in}{2.450850in}}{\pgfqpoint{1.128182in}{2.454123in}}{\pgfqpoint{1.134006in}{2.459947in}}%
\pgfpathcurveto{\pgfqpoint{1.139830in}{2.465771in}}{\pgfqpoint{1.143102in}{2.473671in}}{\pgfqpoint{1.143102in}{2.481907in}}%
\pgfpathcurveto{\pgfqpoint{1.143102in}{2.490143in}}{\pgfqpoint{1.139830in}{2.498043in}}{\pgfqpoint{1.134006in}{2.503867in}}%
\pgfpathcurveto{\pgfqpoint{1.128182in}{2.509691in}}{\pgfqpoint{1.120282in}{2.512963in}}{\pgfqpoint{1.112046in}{2.512963in}}%
\pgfpathcurveto{\pgfqpoint{1.103810in}{2.512963in}}{\pgfqpoint{1.095909in}{2.509691in}}{\pgfqpoint{1.090086in}{2.503867in}}%
\pgfpathcurveto{\pgfqpoint{1.084262in}{2.498043in}}{\pgfqpoint{1.080989in}{2.490143in}}{\pgfqpoint{1.080989in}{2.481907in}}%
\pgfpathcurveto{\pgfqpoint{1.080989in}{2.473671in}}{\pgfqpoint{1.084262in}{2.465771in}}{\pgfqpoint{1.090086in}{2.459947in}}%
\pgfpathcurveto{\pgfqpoint{1.095909in}{2.454123in}}{\pgfqpoint{1.103810in}{2.450850in}}{\pgfqpoint{1.112046in}{2.450850in}}%
\pgfpathclose%
\pgfusepath{stroke,fill}%
\end{pgfscope}%
\begin{pgfscope}%
\pgfpathrectangle{\pgfqpoint{0.100000in}{0.212622in}}{\pgfqpoint{3.696000in}{3.696000in}}%
\pgfusepath{clip}%
\pgfsetbuttcap%
\pgfsetroundjoin%
\definecolor{currentfill}{rgb}{0.121569,0.466667,0.705882}%
\pgfsetfillcolor{currentfill}%
\pgfsetfillopacity{0.851844}%
\pgfsetlinewidth{1.003750pt}%
\definecolor{currentstroke}{rgb}{0.121569,0.466667,0.705882}%
\pgfsetstrokecolor{currentstroke}%
\pgfsetstrokeopacity{0.851844}%
\pgfsetdash{}{0pt}%
\pgfpathmoveto{\pgfqpoint{2.738655in}{1.996885in}}%
\pgfpathcurveto{\pgfqpoint{2.746891in}{1.996885in}}{\pgfqpoint{2.754792in}{2.000157in}}{\pgfqpoint{2.760615in}{2.005981in}}%
\pgfpathcurveto{\pgfqpoint{2.766439in}{2.011805in}}{\pgfqpoint{2.769712in}{2.019705in}}{\pgfqpoint{2.769712in}{2.027941in}}%
\pgfpathcurveto{\pgfqpoint{2.769712in}{2.036178in}}{\pgfqpoint{2.766439in}{2.044078in}}{\pgfqpoint{2.760615in}{2.049902in}}%
\pgfpathcurveto{\pgfqpoint{2.754792in}{2.055726in}}{\pgfqpoint{2.746891in}{2.058998in}}{\pgfqpoint{2.738655in}{2.058998in}}%
\pgfpathcurveto{\pgfqpoint{2.730419in}{2.058998in}}{\pgfqpoint{2.722519in}{2.055726in}}{\pgfqpoint{2.716695in}{2.049902in}}%
\pgfpathcurveto{\pgfqpoint{2.710871in}{2.044078in}}{\pgfqpoint{2.707599in}{2.036178in}}{\pgfqpoint{2.707599in}{2.027941in}}%
\pgfpathcurveto{\pgfqpoint{2.707599in}{2.019705in}}{\pgfqpoint{2.710871in}{2.011805in}}{\pgfqpoint{2.716695in}{2.005981in}}%
\pgfpathcurveto{\pgfqpoint{2.722519in}{2.000157in}}{\pgfqpoint{2.730419in}{1.996885in}}{\pgfqpoint{2.738655in}{1.996885in}}%
\pgfpathclose%
\pgfusepath{stroke,fill}%
\end{pgfscope}%
\begin{pgfscope}%
\pgfpathrectangle{\pgfqpoint{0.100000in}{0.212622in}}{\pgfqpoint{3.696000in}{3.696000in}}%
\pgfusepath{clip}%
\pgfsetbuttcap%
\pgfsetroundjoin%
\definecolor{currentfill}{rgb}{0.121569,0.466667,0.705882}%
\pgfsetfillcolor{currentfill}%
\pgfsetfillopacity{0.852707}%
\pgfsetlinewidth{1.003750pt}%
\definecolor{currentstroke}{rgb}{0.121569,0.466667,0.705882}%
\pgfsetstrokecolor{currentstroke}%
\pgfsetstrokeopacity{0.852707}%
\pgfsetdash{}{0pt}%
\pgfpathmoveto{\pgfqpoint{1.129259in}{2.441205in}}%
\pgfpathcurveto{\pgfqpoint{1.137496in}{2.441205in}}{\pgfqpoint{1.145396in}{2.444478in}}{\pgfqpoint{1.151220in}{2.450302in}}%
\pgfpathcurveto{\pgfqpoint{1.157043in}{2.456126in}}{\pgfqpoint{1.160316in}{2.464026in}}{\pgfqpoint{1.160316in}{2.472262in}}%
\pgfpathcurveto{\pgfqpoint{1.160316in}{2.480498in}}{\pgfqpoint{1.157043in}{2.488398in}}{\pgfqpoint{1.151220in}{2.494222in}}%
\pgfpathcurveto{\pgfqpoint{1.145396in}{2.500046in}}{\pgfqpoint{1.137496in}{2.503318in}}{\pgfqpoint{1.129259in}{2.503318in}}%
\pgfpathcurveto{\pgfqpoint{1.121023in}{2.503318in}}{\pgfqpoint{1.113123in}{2.500046in}}{\pgfqpoint{1.107299in}{2.494222in}}%
\pgfpathcurveto{\pgfqpoint{1.101475in}{2.488398in}}{\pgfqpoint{1.098203in}{2.480498in}}{\pgfqpoint{1.098203in}{2.472262in}}%
\pgfpathcurveto{\pgfqpoint{1.098203in}{2.464026in}}{\pgfqpoint{1.101475in}{2.456126in}}{\pgfqpoint{1.107299in}{2.450302in}}%
\pgfpathcurveto{\pgfqpoint{1.113123in}{2.444478in}}{\pgfqpoint{1.121023in}{2.441205in}}{\pgfqpoint{1.129259in}{2.441205in}}%
\pgfpathclose%
\pgfusepath{stroke,fill}%
\end{pgfscope}%
\begin{pgfscope}%
\pgfpathrectangle{\pgfqpoint{0.100000in}{0.212622in}}{\pgfqpoint{3.696000in}{3.696000in}}%
\pgfusepath{clip}%
\pgfsetbuttcap%
\pgfsetroundjoin%
\definecolor{currentfill}{rgb}{0.121569,0.466667,0.705882}%
\pgfsetfillcolor{currentfill}%
\pgfsetfillopacity{0.854447}%
\pgfsetlinewidth{1.003750pt}%
\definecolor{currentstroke}{rgb}{0.121569,0.466667,0.705882}%
\pgfsetstrokecolor{currentstroke}%
\pgfsetstrokeopacity{0.854447}%
\pgfsetdash{}{0pt}%
\pgfpathmoveto{\pgfqpoint{1.139823in}{2.437997in}}%
\pgfpathcurveto{\pgfqpoint{1.148059in}{2.437997in}}{\pgfqpoint{1.155959in}{2.441270in}}{\pgfqpoint{1.161783in}{2.447093in}}%
\pgfpathcurveto{\pgfqpoint{1.167607in}{2.452917in}}{\pgfqpoint{1.170879in}{2.460817in}}{\pgfqpoint{1.170879in}{2.469054in}}%
\pgfpathcurveto{\pgfqpoint{1.170879in}{2.477290in}}{\pgfqpoint{1.167607in}{2.485190in}}{\pgfqpoint{1.161783in}{2.491014in}}%
\pgfpathcurveto{\pgfqpoint{1.155959in}{2.496838in}}{\pgfqpoint{1.148059in}{2.500110in}}{\pgfqpoint{1.139823in}{2.500110in}}%
\pgfpathcurveto{\pgfqpoint{1.131586in}{2.500110in}}{\pgfqpoint{1.123686in}{2.496838in}}{\pgfqpoint{1.117862in}{2.491014in}}%
\pgfpathcurveto{\pgfqpoint{1.112039in}{2.485190in}}{\pgfqpoint{1.108766in}{2.477290in}}{\pgfqpoint{1.108766in}{2.469054in}}%
\pgfpathcurveto{\pgfqpoint{1.108766in}{2.460817in}}{\pgfqpoint{1.112039in}{2.452917in}}{\pgfqpoint{1.117862in}{2.447093in}}%
\pgfpathcurveto{\pgfqpoint{1.123686in}{2.441270in}}{\pgfqpoint{1.131586in}{2.437997in}}{\pgfqpoint{1.139823in}{2.437997in}}%
\pgfpathclose%
\pgfusepath{stroke,fill}%
\end{pgfscope}%
\begin{pgfscope}%
\pgfpathrectangle{\pgfqpoint{0.100000in}{0.212622in}}{\pgfqpoint{3.696000in}{3.696000in}}%
\pgfusepath{clip}%
\pgfsetbuttcap%
\pgfsetroundjoin%
\definecolor{currentfill}{rgb}{0.121569,0.466667,0.705882}%
\pgfsetfillcolor{currentfill}%
\pgfsetfillopacity{0.855200}%
\pgfsetlinewidth{1.003750pt}%
\definecolor{currentstroke}{rgb}{0.121569,0.466667,0.705882}%
\pgfsetstrokecolor{currentstroke}%
\pgfsetstrokeopacity{0.855200}%
\pgfsetdash{}{0pt}%
\pgfpathmoveto{\pgfqpoint{1.147167in}{2.434821in}}%
\pgfpathcurveto{\pgfqpoint{1.155403in}{2.434821in}}{\pgfqpoint{1.163303in}{2.438094in}}{\pgfqpoint{1.169127in}{2.443918in}}%
\pgfpathcurveto{\pgfqpoint{1.174951in}{2.449741in}}{\pgfqpoint{1.178223in}{2.457641in}}{\pgfqpoint{1.178223in}{2.465878in}}%
\pgfpathcurveto{\pgfqpoint{1.178223in}{2.474114in}}{\pgfqpoint{1.174951in}{2.482014in}}{\pgfqpoint{1.169127in}{2.487838in}}%
\pgfpathcurveto{\pgfqpoint{1.163303in}{2.493662in}}{\pgfqpoint{1.155403in}{2.496934in}}{\pgfqpoint{1.147167in}{2.496934in}}%
\pgfpathcurveto{\pgfqpoint{1.138930in}{2.496934in}}{\pgfqpoint{1.131030in}{2.493662in}}{\pgfqpoint{1.125206in}{2.487838in}}%
\pgfpathcurveto{\pgfqpoint{1.119382in}{2.482014in}}{\pgfqpoint{1.116110in}{2.474114in}}{\pgfqpoint{1.116110in}{2.465878in}}%
\pgfpathcurveto{\pgfqpoint{1.116110in}{2.457641in}}{\pgfqpoint{1.119382in}{2.449741in}}{\pgfqpoint{1.125206in}{2.443918in}}%
\pgfpathcurveto{\pgfqpoint{1.131030in}{2.438094in}}{\pgfqpoint{1.138930in}{2.434821in}}{\pgfqpoint{1.147167in}{2.434821in}}%
\pgfpathclose%
\pgfusepath{stroke,fill}%
\end{pgfscope}%
\begin{pgfscope}%
\pgfpathrectangle{\pgfqpoint{0.100000in}{0.212622in}}{\pgfqpoint{3.696000in}{3.696000in}}%
\pgfusepath{clip}%
\pgfsetbuttcap%
\pgfsetroundjoin%
\definecolor{currentfill}{rgb}{0.121569,0.466667,0.705882}%
\pgfsetfillcolor{currentfill}%
\pgfsetfillopacity{0.855509}%
\pgfsetlinewidth{1.003750pt}%
\definecolor{currentstroke}{rgb}{0.121569,0.466667,0.705882}%
\pgfsetstrokecolor{currentstroke}%
\pgfsetstrokeopacity{0.855509}%
\pgfsetdash{}{0pt}%
\pgfpathmoveto{\pgfqpoint{2.728781in}{1.992398in}}%
\pgfpathcurveto{\pgfqpoint{2.737018in}{1.992398in}}{\pgfqpoint{2.744918in}{1.995670in}}{\pgfqpoint{2.750742in}{2.001494in}}%
\pgfpathcurveto{\pgfqpoint{2.756565in}{2.007318in}}{\pgfqpoint{2.759838in}{2.015218in}}{\pgfqpoint{2.759838in}{2.023455in}}%
\pgfpathcurveto{\pgfqpoint{2.759838in}{2.031691in}}{\pgfqpoint{2.756565in}{2.039591in}}{\pgfqpoint{2.750742in}{2.045415in}}%
\pgfpathcurveto{\pgfqpoint{2.744918in}{2.051239in}}{\pgfqpoint{2.737018in}{2.054511in}}{\pgfqpoint{2.728781in}{2.054511in}}%
\pgfpathcurveto{\pgfqpoint{2.720545in}{2.054511in}}{\pgfqpoint{2.712645in}{2.051239in}}{\pgfqpoint{2.706821in}{2.045415in}}%
\pgfpathcurveto{\pgfqpoint{2.700997in}{2.039591in}}{\pgfqpoint{2.697725in}{2.031691in}}{\pgfqpoint{2.697725in}{2.023455in}}%
\pgfpathcurveto{\pgfqpoint{2.697725in}{2.015218in}}{\pgfqpoint{2.700997in}{2.007318in}}{\pgfqpoint{2.706821in}{2.001494in}}%
\pgfpathcurveto{\pgfqpoint{2.712645in}{1.995670in}}{\pgfqpoint{2.720545in}{1.992398in}}{\pgfqpoint{2.728781in}{1.992398in}}%
\pgfpathclose%
\pgfusepath{stroke,fill}%
\end{pgfscope}%
\begin{pgfscope}%
\pgfpathrectangle{\pgfqpoint{0.100000in}{0.212622in}}{\pgfqpoint{3.696000in}{3.696000in}}%
\pgfusepath{clip}%
\pgfsetbuttcap%
\pgfsetroundjoin%
\definecolor{currentfill}{rgb}{0.121569,0.466667,0.705882}%
\pgfsetfillcolor{currentfill}%
\pgfsetfillopacity{0.856779}%
\pgfsetlinewidth{1.003750pt}%
\definecolor{currentstroke}{rgb}{0.121569,0.466667,0.705882}%
\pgfsetstrokecolor{currentstroke}%
\pgfsetstrokeopacity{0.856779}%
\pgfsetdash{}{0pt}%
\pgfpathmoveto{\pgfqpoint{1.159633in}{2.427633in}}%
\pgfpathcurveto{\pgfqpoint{1.167870in}{2.427633in}}{\pgfqpoint{1.175770in}{2.430906in}}{\pgfqpoint{1.181593in}{2.436730in}}%
\pgfpathcurveto{\pgfqpoint{1.187417in}{2.442553in}}{\pgfqpoint{1.190690in}{2.450454in}}{\pgfqpoint{1.190690in}{2.458690in}}%
\pgfpathcurveto{\pgfqpoint{1.190690in}{2.466926in}}{\pgfqpoint{1.187417in}{2.474826in}}{\pgfqpoint{1.181593in}{2.480650in}}%
\pgfpathcurveto{\pgfqpoint{1.175770in}{2.486474in}}{\pgfqpoint{1.167870in}{2.489746in}}{\pgfqpoint{1.159633in}{2.489746in}}%
\pgfpathcurveto{\pgfqpoint{1.151397in}{2.489746in}}{\pgfqpoint{1.143497in}{2.486474in}}{\pgfqpoint{1.137673in}{2.480650in}}%
\pgfpathcurveto{\pgfqpoint{1.131849in}{2.474826in}}{\pgfqpoint{1.128577in}{2.466926in}}{\pgfqpoint{1.128577in}{2.458690in}}%
\pgfpathcurveto{\pgfqpoint{1.128577in}{2.450454in}}{\pgfqpoint{1.131849in}{2.442553in}}{\pgfqpoint{1.137673in}{2.436730in}}%
\pgfpathcurveto{\pgfqpoint{1.143497in}{2.430906in}}{\pgfqpoint{1.151397in}{2.427633in}}{\pgfqpoint{1.159633in}{2.427633in}}%
\pgfpathclose%
\pgfusepath{stroke,fill}%
\end{pgfscope}%
\begin{pgfscope}%
\pgfpathrectangle{\pgfqpoint{0.100000in}{0.212622in}}{\pgfqpoint{3.696000in}{3.696000in}}%
\pgfusepath{clip}%
\pgfsetbuttcap%
\pgfsetroundjoin%
\definecolor{currentfill}{rgb}{0.121569,0.466667,0.705882}%
\pgfsetfillcolor{currentfill}%
\pgfsetfillopacity{0.857641}%
\pgfsetlinewidth{1.003750pt}%
\definecolor{currentstroke}{rgb}{0.121569,0.466667,0.705882}%
\pgfsetstrokecolor{currentstroke}%
\pgfsetstrokeopacity{0.857641}%
\pgfsetdash{}{0pt}%
\pgfpathmoveto{\pgfqpoint{2.723485in}{1.990460in}}%
\pgfpathcurveto{\pgfqpoint{2.731721in}{1.990460in}}{\pgfqpoint{2.739621in}{1.993732in}}{\pgfqpoint{2.745445in}{1.999556in}}%
\pgfpathcurveto{\pgfqpoint{2.751269in}{2.005380in}}{\pgfqpoint{2.754542in}{2.013280in}}{\pgfqpoint{2.754542in}{2.021516in}}%
\pgfpathcurveto{\pgfqpoint{2.754542in}{2.029752in}}{\pgfqpoint{2.751269in}{2.037652in}}{\pgfqpoint{2.745445in}{2.043476in}}%
\pgfpathcurveto{\pgfqpoint{2.739621in}{2.049300in}}{\pgfqpoint{2.731721in}{2.052573in}}{\pgfqpoint{2.723485in}{2.052573in}}%
\pgfpathcurveto{\pgfqpoint{2.715249in}{2.052573in}}{\pgfqpoint{2.707349in}{2.049300in}}{\pgfqpoint{2.701525in}{2.043476in}}%
\pgfpathcurveto{\pgfqpoint{2.695701in}{2.037652in}}{\pgfqpoint{2.692429in}{2.029752in}}{\pgfqpoint{2.692429in}{2.021516in}}%
\pgfpathcurveto{\pgfqpoint{2.692429in}{2.013280in}}{\pgfqpoint{2.695701in}{2.005380in}}{\pgfqpoint{2.701525in}{1.999556in}}%
\pgfpathcurveto{\pgfqpoint{2.707349in}{1.993732in}}{\pgfqpoint{2.715249in}{1.990460in}}{\pgfqpoint{2.723485in}{1.990460in}}%
\pgfpathclose%
\pgfusepath{stroke,fill}%
\end{pgfscope}%
\begin{pgfscope}%
\pgfpathrectangle{\pgfqpoint{0.100000in}{0.212622in}}{\pgfqpoint{3.696000in}{3.696000in}}%
\pgfusepath{clip}%
\pgfsetbuttcap%
\pgfsetroundjoin%
\definecolor{currentfill}{rgb}{0.121569,0.466667,0.705882}%
\pgfsetfillcolor{currentfill}%
\pgfsetfillopacity{0.858819}%
\pgfsetlinewidth{1.003750pt}%
\definecolor{currentstroke}{rgb}{0.121569,0.466667,0.705882}%
\pgfsetstrokecolor{currentstroke}%
\pgfsetstrokeopacity{0.858819}%
\pgfsetdash{}{0pt}%
\pgfpathmoveto{\pgfqpoint{2.720689in}{1.989203in}}%
\pgfpathcurveto{\pgfqpoint{2.728925in}{1.989203in}}{\pgfqpoint{2.736825in}{1.992475in}}{\pgfqpoint{2.742649in}{1.998299in}}%
\pgfpathcurveto{\pgfqpoint{2.748473in}{2.004123in}}{\pgfqpoint{2.751745in}{2.012023in}}{\pgfqpoint{2.751745in}{2.020259in}}%
\pgfpathcurveto{\pgfqpoint{2.751745in}{2.028496in}}{\pgfqpoint{2.748473in}{2.036396in}}{\pgfqpoint{2.742649in}{2.042220in}}%
\pgfpathcurveto{\pgfqpoint{2.736825in}{2.048044in}}{\pgfqpoint{2.728925in}{2.051316in}}{\pgfqpoint{2.720689in}{2.051316in}}%
\pgfpathcurveto{\pgfqpoint{2.712452in}{2.051316in}}{\pgfqpoint{2.704552in}{2.048044in}}{\pgfqpoint{2.698729in}{2.042220in}}%
\pgfpathcurveto{\pgfqpoint{2.692905in}{2.036396in}}{\pgfqpoint{2.689632in}{2.028496in}}{\pgfqpoint{2.689632in}{2.020259in}}%
\pgfpathcurveto{\pgfqpoint{2.689632in}{2.012023in}}{\pgfqpoint{2.692905in}{2.004123in}}{\pgfqpoint{2.698729in}{1.998299in}}%
\pgfpathcurveto{\pgfqpoint{2.704552in}{1.992475in}}{\pgfqpoint{2.712452in}{1.989203in}}{\pgfqpoint{2.720689in}{1.989203in}}%
\pgfpathclose%
\pgfusepath{stroke,fill}%
\end{pgfscope}%
\begin{pgfscope}%
\pgfpathrectangle{\pgfqpoint{0.100000in}{0.212622in}}{\pgfqpoint{3.696000in}{3.696000in}}%
\pgfusepath{clip}%
\pgfsetbuttcap%
\pgfsetroundjoin%
\definecolor{currentfill}{rgb}{0.121569,0.466667,0.705882}%
\pgfsetfillcolor{currentfill}%
\pgfsetfillopacity{0.859122}%
\pgfsetlinewidth{1.003750pt}%
\definecolor{currentstroke}{rgb}{0.121569,0.466667,0.705882}%
\pgfsetstrokecolor{currentstroke}%
\pgfsetstrokeopacity{0.859122}%
\pgfsetdash{}{0pt}%
\pgfpathmoveto{\pgfqpoint{1.183636in}{2.415204in}}%
\pgfpathcurveto{\pgfqpoint{1.191873in}{2.415204in}}{\pgfqpoint{1.199773in}{2.418477in}}{\pgfqpoint{1.205597in}{2.424301in}}%
\pgfpathcurveto{\pgfqpoint{1.211421in}{2.430125in}}{\pgfqpoint{1.214693in}{2.438025in}}{\pgfqpoint{1.214693in}{2.446261in}}%
\pgfpathcurveto{\pgfqpoint{1.214693in}{2.454497in}}{\pgfqpoint{1.211421in}{2.462397in}}{\pgfqpoint{1.205597in}{2.468221in}}%
\pgfpathcurveto{\pgfqpoint{1.199773in}{2.474045in}}{\pgfqpoint{1.191873in}{2.477317in}}{\pgfqpoint{1.183636in}{2.477317in}}%
\pgfpathcurveto{\pgfqpoint{1.175400in}{2.477317in}}{\pgfqpoint{1.167500in}{2.474045in}}{\pgfqpoint{1.161676in}{2.468221in}}%
\pgfpathcurveto{\pgfqpoint{1.155852in}{2.462397in}}{\pgfqpoint{1.152580in}{2.454497in}}{\pgfqpoint{1.152580in}{2.446261in}}%
\pgfpathcurveto{\pgfqpoint{1.152580in}{2.438025in}}{\pgfqpoint{1.155852in}{2.430125in}}{\pgfqpoint{1.161676in}{2.424301in}}%
\pgfpathcurveto{\pgfqpoint{1.167500in}{2.418477in}}{\pgfqpoint{1.175400in}{2.415204in}}{\pgfqpoint{1.183636in}{2.415204in}}%
\pgfpathclose%
\pgfusepath{stroke,fill}%
\end{pgfscope}%
\begin{pgfscope}%
\pgfpathrectangle{\pgfqpoint{0.100000in}{0.212622in}}{\pgfqpoint{3.696000in}{3.696000in}}%
\pgfusepath{clip}%
\pgfsetbuttcap%
\pgfsetroundjoin%
\definecolor{currentfill}{rgb}{0.121569,0.466667,0.705882}%
\pgfsetfillcolor{currentfill}%
\pgfsetfillopacity{0.859557}%
\pgfsetlinewidth{1.003750pt}%
\definecolor{currentstroke}{rgb}{0.121569,0.466667,0.705882}%
\pgfsetstrokecolor{currentstroke}%
\pgfsetstrokeopacity{0.859557}%
\pgfsetdash{}{0pt}%
\pgfpathmoveto{\pgfqpoint{2.719155in}{1.989104in}}%
\pgfpathcurveto{\pgfqpoint{2.727391in}{1.989104in}}{\pgfqpoint{2.735291in}{1.992376in}}{\pgfqpoint{2.741115in}{1.998200in}}%
\pgfpathcurveto{\pgfqpoint{2.746939in}{2.004024in}}{\pgfqpoint{2.750211in}{2.011924in}}{\pgfqpoint{2.750211in}{2.020161in}}%
\pgfpathcurveto{\pgfqpoint{2.750211in}{2.028397in}}{\pgfqpoint{2.746939in}{2.036297in}}{\pgfqpoint{2.741115in}{2.042121in}}%
\pgfpathcurveto{\pgfqpoint{2.735291in}{2.047945in}}{\pgfqpoint{2.727391in}{2.051217in}}{\pgfqpoint{2.719155in}{2.051217in}}%
\pgfpathcurveto{\pgfqpoint{2.710918in}{2.051217in}}{\pgfqpoint{2.703018in}{2.047945in}}{\pgfqpoint{2.697194in}{2.042121in}}%
\pgfpathcurveto{\pgfqpoint{2.691370in}{2.036297in}}{\pgfqpoint{2.688098in}{2.028397in}}{\pgfqpoint{2.688098in}{2.020161in}}%
\pgfpathcurveto{\pgfqpoint{2.688098in}{2.011924in}}{\pgfqpoint{2.691370in}{2.004024in}}{\pgfqpoint{2.697194in}{1.998200in}}%
\pgfpathcurveto{\pgfqpoint{2.703018in}{1.992376in}}{\pgfqpoint{2.710918in}{1.989104in}}{\pgfqpoint{2.719155in}{1.989104in}}%
\pgfpathclose%
\pgfusepath{stroke,fill}%
\end{pgfscope}%
\begin{pgfscope}%
\pgfpathrectangle{\pgfqpoint{0.100000in}{0.212622in}}{\pgfqpoint{3.696000in}{3.696000in}}%
\pgfusepath{clip}%
\pgfsetbuttcap%
\pgfsetroundjoin%
\definecolor{currentfill}{rgb}{0.121569,0.466667,0.705882}%
\pgfsetfillcolor{currentfill}%
\pgfsetfillopacity{0.859899}%
\pgfsetlinewidth{1.003750pt}%
\definecolor{currentstroke}{rgb}{0.121569,0.466667,0.705882}%
\pgfsetstrokecolor{currentstroke}%
\pgfsetstrokeopacity{0.859899}%
\pgfsetdash{}{0pt}%
\pgfpathmoveto{\pgfqpoint{2.718516in}{1.988405in}}%
\pgfpathcurveto{\pgfqpoint{2.726752in}{1.988405in}}{\pgfqpoint{2.734652in}{1.991677in}}{\pgfqpoint{2.740476in}{1.997501in}}%
\pgfpathcurveto{\pgfqpoint{2.746300in}{2.003325in}}{\pgfqpoint{2.749572in}{2.011225in}}{\pgfqpoint{2.749572in}{2.019461in}}%
\pgfpathcurveto{\pgfqpoint{2.749572in}{2.027698in}}{\pgfqpoint{2.746300in}{2.035598in}}{\pgfqpoint{2.740476in}{2.041422in}}%
\pgfpathcurveto{\pgfqpoint{2.734652in}{2.047246in}}{\pgfqpoint{2.726752in}{2.050518in}}{\pgfqpoint{2.718516in}{2.050518in}}%
\pgfpathcurveto{\pgfqpoint{2.710280in}{2.050518in}}{\pgfqpoint{2.702380in}{2.047246in}}{\pgfqpoint{2.696556in}{2.041422in}}%
\pgfpathcurveto{\pgfqpoint{2.690732in}{2.035598in}}{\pgfqpoint{2.687459in}{2.027698in}}{\pgfqpoint{2.687459in}{2.019461in}}%
\pgfpathcurveto{\pgfqpoint{2.687459in}{2.011225in}}{\pgfqpoint{2.690732in}{2.003325in}}{\pgfqpoint{2.696556in}{1.997501in}}%
\pgfpathcurveto{\pgfqpoint{2.702380in}{1.991677in}}{\pgfqpoint{2.710280in}{1.988405in}}{\pgfqpoint{2.718516in}{1.988405in}}%
\pgfpathclose%
\pgfusepath{stroke,fill}%
\end{pgfscope}%
\begin{pgfscope}%
\pgfpathrectangle{\pgfqpoint{0.100000in}{0.212622in}}{\pgfqpoint{3.696000in}{3.696000in}}%
\pgfusepath{clip}%
\pgfsetbuttcap%
\pgfsetroundjoin%
\definecolor{currentfill}{rgb}{0.121569,0.466667,0.705882}%
\pgfsetfillcolor{currentfill}%
\pgfsetfillopacity{0.860107}%
\pgfsetlinewidth{1.003750pt}%
\definecolor{currentstroke}{rgb}{0.121569,0.466667,0.705882}%
\pgfsetstrokecolor{currentstroke}%
\pgfsetstrokeopacity{0.860107}%
\pgfsetdash{}{0pt}%
\pgfpathmoveto{\pgfqpoint{2.718074in}{1.988251in}}%
\pgfpathcurveto{\pgfqpoint{2.726310in}{1.988251in}}{\pgfqpoint{2.734210in}{1.991524in}}{\pgfqpoint{2.740034in}{1.997347in}}%
\pgfpathcurveto{\pgfqpoint{2.745858in}{2.003171in}}{\pgfqpoint{2.749130in}{2.011071in}}{\pgfqpoint{2.749130in}{2.019308in}}%
\pgfpathcurveto{\pgfqpoint{2.749130in}{2.027544in}}{\pgfqpoint{2.745858in}{2.035444in}}{\pgfqpoint{2.740034in}{2.041268in}}%
\pgfpathcurveto{\pgfqpoint{2.734210in}{2.047092in}}{\pgfqpoint{2.726310in}{2.050364in}}{\pgfqpoint{2.718074in}{2.050364in}}%
\pgfpathcurveto{\pgfqpoint{2.709837in}{2.050364in}}{\pgfqpoint{2.701937in}{2.047092in}}{\pgfqpoint{2.696113in}{2.041268in}}%
\pgfpathcurveto{\pgfqpoint{2.690289in}{2.035444in}}{\pgfqpoint{2.687017in}{2.027544in}}{\pgfqpoint{2.687017in}{2.019308in}}%
\pgfpathcurveto{\pgfqpoint{2.687017in}{2.011071in}}{\pgfqpoint{2.690289in}{2.003171in}}{\pgfqpoint{2.696113in}{1.997347in}}%
\pgfpathcurveto{\pgfqpoint{2.701937in}{1.991524in}}{\pgfqpoint{2.709837in}{1.988251in}}{\pgfqpoint{2.718074in}{1.988251in}}%
\pgfpathclose%
\pgfusepath{stroke,fill}%
\end{pgfscope}%
\begin{pgfscope}%
\pgfpathrectangle{\pgfqpoint{0.100000in}{0.212622in}}{\pgfqpoint{3.696000in}{3.696000in}}%
\pgfusepath{clip}%
\pgfsetbuttcap%
\pgfsetroundjoin%
\definecolor{currentfill}{rgb}{0.121569,0.466667,0.705882}%
\pgfsetfillcolor{currentfill}%
\pgfsetfillopacity{0.861490}%
\pgfsetlinewidth{1.003750pt}%
\definecolor{currentstroke}{rgb}{0.121569,0.466667,0.705882}%
\pgfsetstrokecolor{currentstroke}%
\pgfsetstrokeopacity{0.861490}%
\pgfsetdash{}{0pt}%
\pgfpathmoveto{\pgfqpoint{2.715126in}{1.984593in}}%
\pgfpathcurveto{\pgfqpoint{2.723362in}{1.984593in}}{\pgfqpoint{2.731262in}{1.987866in}}{\pgfqpoint{2.737086in}{1.993690in}}%
\pgfpathcurveto{\pgfqpoint{2.742910in}{1.999514in}}{\pgfqpoint{2.746182in}{2.007414in}}{\pgfqpoint{2.746182in}{2.015650in}}%
\pgfpathcurveto{\pgfqpoint{2.746182in}{2.023886in}}{\pgfqpoint{2.742910in}{2.031786in}}{\pgfqpoint{2.737086in}{2.037610in}}%
\pgfpathcurveto{\pgfqpoint{2.731262in}{2.043434in}}{\pgfqpoint{2.723362in}{2.046706in}}{\pgfqpoint{2.715126in}{2.046706in}}%
\pgfpathcurveto{\pgfqpoint{2.706890in}{2.046706in}}{\pgfqpoint{2.698990in}{2.043434in}}{\pgfqpoint{2.693166in}{2.037610in}}%
\pgfpathcurveto{\pgfqpoint{2.687342in}{2.031786in}}{\pgfqpoint{2.684069in}{2.023886in}}{\pgfqpoint{2.684069in}{2.015650in}}%
\pgfpathcurveto{\pgfqpoint{2.684069in}{2.007414in}}{\pgfqpoint{2.687342in}{1.999514in}}{\pgfqpoint{2.693166in}{1.993690in}}%
\pgfpathcurveto{\pgfqpoint{2.698990in}{1.987866in}}{\pgfqpoint{2.706890in}{1.984593in}}{\pgfqpoint{2.715126in}{1.984593in}}%
\pgfpathclose%
\pgfusepath{stroke,fill}%
\end{pgfscope}%
\begin{pgfscope}%
\pgfpathrectangle{\pgfqpoint{0.100000in}{0.212622in}}{\pgfqpoint{3.696000in}{3.696000in}}%
\pgfusepath{clip}%
\pgfsetbuttcap%
\pgfsetroundjoin%
\definecolor{currentfill}{rgb}{0.121569,0.466667,0.705882}%
\pgfsetfillcolor{currentfill}%
\pgfsetfillopacity{0.861914}%
\pgfsetlinewidth{1.003750pt}%
\definecolor{currentstroke}{rgb}{0.121569,0.466667,0.705882}%
\pgfsetstrokecolor{currentstroke}%
\pgfsetstrokeopacity{0.861914}%
\pgfsetdash{}{0pt}%
\pgfpathmoveto{\pgfqpoint{1.204895in}{2.406125in}}%
\pgfpathcurveto{\pgfqpoint{1.213132in}{2.406125in}}{\pgfqpoint{1.221032in}{2.409397in}}{\pgfqpoint{1.226856in}{2.415221in}}%
\pgfpathcurveto{\pgfqpoint{1.232679in}{2.421045in}}{\pgfqpoint{1.235952in}{2.428945in}}{\pgfqpoint{1.235952in}{2.437182in}}%
\pgfpathcurveto{\pgfqpoint{1.235952in}{2.445418in}}{\pgfqpoint{1.232679in}{2.453318in}}{\pgfqpoint{1.226856in}{2.459142in}}%
\pgfpathcurveto{\pgfqpoint{1.221032in}{2.464966in}}{\pgfqpoint{1.213132in}{2.468238in}}{\pgfqpoint{1.204895in}{2.468238in}}%
\pgfpathcurveto{\pgfqpoint{1.196659in}{2.468238in}}{\pgfqpoint{1.188759in}{2.464966in}}{\pgfqpoint{1.182935in}{2.459142in}}%
\pgfpathcurveto{\pgfqpoint{1.177111in}{2.453318in}}{\pgfqpoint{1.173839in}{2.445418in}}{\pgfqpoint{1.173839in}{2.437182in}}%
\pgfpathcurveto{\pgfqpoint{1.173839in}{2.428945in}}{\pgfqpoint{1.177111in}{2.421045in}}{\pgfqpoint{1.182935in}{2.415221in}}%
\pgfpathcurveto{\pgfqpoint{1.188759in}{2.409397in}}{\pgfqpoint{1.196659in}{2.406125in}}{\pgfqpoint{1.204895in}{2.406125in}}%
\pgfpathclose%
\pgfusepath{stroke,fill}%
\end{pgfscope}%
\begin{pgfscope}%
\pgfpathrectangle{\pgfqpoint{0.100000in}{0.212622in}}{\pgfqpoint{3.696000in}{3.696000in}}%
\pgfusepath{clip}%
\pgfsetbuttcap%
\pgfsetroundjoin%
\definecolor{currentfill}{rgb}{0.121569,0.466667,0.705882}%
\pgfsetfillcolor{currentfill}%
\pgfsetfillopacity{0.862734}%
\pgfsetlinewidth{1.003750pt}%
\definecolor{currentstroke}{rgb}{0.121569,0.466667,0.705882}%
\pgfsetstrokecolor{currentstroke}%
\pgfsetstrokeopacity{0.862734}%
\pgfsetdash{}{0pt}%
\pgfpathmoveto{\pgfqpoint{1.225639in}{2.395037in}}%
\pgfpathcurveto{\pgfqpoint{1.233875in}{2.395037in}}{\pgfqpoint{1.241775in}{2.398310in}}{\pgfqpoint{1.247599in}{2.404134in}}%
\pgfpathcurveto{\pgfqpoint{1.253423in}{2.409957in}}{\pgfqpoint{1.256695in}{2.417857in}}{\pgfqpoint{1.256695in}{2.426094in}}%
\pgfpathcurveto{\pgfqpoint{1.256695in}{2.434330in}}{\pgfqpoint{1.253423in}{2.442230in}}{\pgfqpoint{1.247599in}{2.448054in}}%
\pgfpathcurveto{\pgfqpoint{1.241775in}{2.453878in}}{\pgfqpoint{1.233875in}{2.457150in}}{\pgfqpoint{1.225639in}{2.457150in}}%
\pgfpathcurveto{\pgfqpoint{1.217402in}{2.457150in}}{\pgfqpoint{1.209502in}{2.453878in}}{\pgfqpoint{1.203678in}{2.448054in}}%
\pgfpathcurveto{\pgfqpoint{1.197854in}{2.442230in}}{\pgfqpoint{1.194582in}{2.434330in}}{\pgfqpoint{1.194582in}{2.426094in}}%
\pgfpathcurveto{\pgfqpoint{1.194582in}{2.417857in}}{\pgfqpoint{1.197854in}{2.409957in}}{\pgfqpoint{1.203678in}{2.404134in}}%
\pgfpathcurveto{\pgfqpoint{1.209502in}{2.398310in}}{\pgfqpoint{1.217402in}{2.395037in}}{\pgfqpoint{1.225639in}{2.395037in}}%
\pgfpathclose%
\pgfusepath{stroke,fill}%
\end{pgfscope}%
\begin{pgfscope}%
\pgfpathrectangle{\pgfqpoint{0.100000in}{0.212622in}}{\pgfqpoint{3.696000in}{3.696000in}}%
\pgfusepath{clip}%
\pgfsetbuttcap%
\pgfsetroundjoin%
\definecolor{currentfill}{rgb}{0.121569,0.466667,0.705882}%
\pgfsetfillcolor{currentfill}%
\pgfsetfillopacity{0.864823}%
\pgfsetlinewidth{1.003750pt}%
\definecolor{currentstroke}{rgb}{0.121569,0.466667,0.705882}%
\pgfsetstrokecolor{currentstroke}%
\pgfsetstrokeopacity{0.864823}%
\pgfsetdash{}{0pt}%
\pgfpathmoveto{\pgfqpoint{2.710125in}{1.984868in}}%
\pgfpathcurveto{\pgfqpoint{2.718362in}{1.984868in}}{\pgfqpoint{2.726262in}{1.988140in}}{\pgfqpoint{2.732086in}{1.993964in}}%
\pgfpathcurveto{\pgfqpoint{2.737910in}{1.999788in}}{\pgfqpoint{2.741182in}{2.007688in}}{\pgfqpoint{2.741182in}{2.015924in}}%
\pgfpathcurveto{\pgfqpoint{2.741182in}{2.024161in}}{\pgfqpoint{2.737910in}{2.032061in}}{\pgfqpoint{2.732086in}{2.037885in}}%
\pgfpathcurveto{\pgfqpoint{2.726262in}{2.043708in}}{\pgfqpoint{2.718362in}{2.046981in}}{\pgfqpoint{2.710125in}{2.046981in}}%
\pgfpathcurveto{\pgfqpoint{2.701889in}{2.046981in}}{\pgfqpoint{2.693989in}{2.043708in}}{\pgfqpoint{2.688165in}{2.037885in}}%
\pgfpathcurveto{\pgfqpoint{2.682341in}{2.032061in}}{\pgfqpoint{2.679069in}{2.024161in}}{\pgfqpoint{2.679069in}{2.015924in}}%
\pgfpathcurveto{\pgfqpoint{2.679069in}{2.007688in}}{\pgfqpoint{2.682341in}{1.999788in}}{\pgfqpoint{2.688165in}{1.993964in}}%
\pgfpathcurveto{\pgfqpoint{2.693989in}{1.988140in}}{\pgfqpoint{2.701889in}{1.984868in}}{\pgfqpoint{2.710125in}{1.984868in}}%
\pgfpathclose%
\pgfusepath{stroke,fill}%
\end{pgfscope}%
\begin{pgfscope}%
\pgfpathrectangle{\pgfqpoint{0.100000in}{0.212622in}}{\pgfqpoint{3.696000in}{3.696000in}}%
\pgfusepath{clip}%
\pgfsetbuttcap%
\pgfsetroundjoin%
\definecolor{currentfill}{rgb}{0.121569,0.466667,0.705882}%
\pgfsetfillcolor{currentfill}%
\pgfsetfillopacity{0.865345}%
\pgfsetlinewidth{1.003750pt}%
\definecolor{currentstroke}{rgb}{0.121569,0.466667,0.705882}%
\pgfsetstrokecolor{currentstroke}%
\pgfsetstrokeopacity{0.865345}%
\pgfsetdash{}{0pt}%
\pgfpathmoveto{\pgfqpoint{1.241896in}{2.389533in}}%
\pgfpathcurveto{\pgfqpoint{1.250132in}{2.389533in}}{\pgfqpoint{1.258032in}{2.392806in}}{\pgfqpoint{1.263856in}{2.398630in}}%
\pgfpathcurveto{\pgfqpoint{1.269680in}{2.404453in}}{\pgfqpoint{1.272952in}{2.412354in}}{\pgfqpoint{1.272952in}{2.420590in}}%
\pgfpathcurveto{\pgfqpoint{1.272952in}{2.428826in}}{\pgfqpoint{1.269680in}{2.436726in}}{\pgfqpoint{1.263856in}{2.442550in}}%
\pgfpathcurveto{\pgfqpoint{1.258032in}{2.448374in}}{\pgfqpoint{1.250132in}{2.451646in}}{\pgfqpoint{1.241896in}{2.451646in}}%
\pgfpathcurveto{\pgfqpoint{1.233659in}{2.451646in}}{\pgfqpoint{1.225759in}{2.448374in}}{\pgfqpoint{1.219935in}{2.442550in}}%
\pgfpathcurveto{\pgfqpoint{1.214111in}{2.436726in}}{\pgfqpoint{1.210839in}{2.428826in}}{\pgfqpoint{1.210839in}{2.420590in}}%
\pgfpathcurveto{\pgfqpoint{1.210839in}{2.412354in}}{\pgfqpoint{1.214111in}{2.404453in}}{\pgfqpoint{1.219935in}{2.398630in}}%
\pgfpathcurveto{\pgfqpoint{1.225759in}{2.392806in}}{\pgfqpoint{1.233659in}{2.389533in}}{\pgfqpoint{1.241896in}{2.389533in}}%
\pgfpathclose%
\pgfusepath{stroke,fill}%
\end{pgfscope}%
\begin{pgfscope}%
\pgfpathrectangle{\pgfqpoint{0.100000in}{0.212622in}}{\pgfqpoint{3.696000in}{3.696000in}}%
\pgfusepath{clip}%
\pgfsetbuttcap%
\pgfsetroundjoin%
\definecolor{currentfill}{rgb}{0.121569,0.466667,0.705882}%
\pgfsetfillcolor{currentfill}%
\pgfsetfillopacity{0.866167}%
\pgfsetlinewidth{1.003750pt}%
\definecolor{currentstroke}{rgb}{0.121569,0.466667,0.705882}%
\pgfsetstrokecolor{currentstroke}%
\pgfsetstrokeopacity{0.866167}%
\pgfsetdash{}{0pt}%
\pgfpathmoveto{\pgfqpoint{1.253137in}{2.383866in}}%
\pgfpathcurveto{\pgfqpoint{1.261374in}{2.383866in}}{\pgfqpoint{1.269274in}{2.387138in}}{\pgfqpoint{1.275098in}{2.392962in}}%
\pgfpathcurveto{\pgfqpoint{1.280922in}{2.398786in}}{\pgfqpoint{1.284194in}{2.406686in}}{\pgfqpoint{1.284194in}{2.414922in}}%
\pgfpathcurveto{\pgfqpoint{1.284194in}{2.423158in}}{\pgfqpoint{1.280922in}{2.431058in}}{\pgfqpoint{1.275098in}{2.436882in}}%
\pgfpathcurveto{\pgfqpoint{1.269274in}{2.442706in}}{\pgfqpoint{1.261374in}{2.445979in}}{\pgfqpoint{1.253137in}{2.445979in}}%
\pgfpathcurveto{\pgfqpoint{1.244901in}{2.445979in}}{\pgfqpoint{1.237001in}{2.442706in}}{\pgfqpoint{1.231177in}{2.436882in}}%
\pgfpathcurveto{\pgfqpoint{1.225353in}{2.431058in}}{\pgfqpoint{1.222081in}{2.423158in}}{\pgfqpoint{1.222081in}{2.414922in}}%
\pgfpathcurveto{\pgfqpoint{1.222081in}{2.406686in}}{\pgfqpoint{1.225353in}{2.398786in}}{\pgfqpoint{1.231177in}{2.392962in}}%
\pgfpathcurveto{\pgfqpoint{1.237001in}{2.387138in}}{\pgfqpoint{1.244901in}{2.383866in}}{\pgfqpoint{1.253137in}{2.383866in}}%
\pgfpathclose%
\pgfusepath{stroke,fill}%
\end{pgfscope}%
\begin{pgfscope}%
\pgfpathrectangle{\pgfqpoint{0.100000in}{0.212622in}}{\pgfqpoint{3.696000in}{3.696000in}}%
\pgfusepath{clip}%
\pgfsetbuttcap%
\pgfsetroundjoin%
\definecolor{currentfill}{rgb}{0.121569,0.466667,0.705882}%
\pgfsetfillcolor{currentfill}%
\pgfsetfillopacity{0.866796}%
\pgfsetlinewidth{1.003750pt}%
\definecolor{currentstroke}{rgb}{0.121569,0.466667,0.705882}%
\pgfsetstrokecolor{currentstroke}%
\pgfsetstrokeopacity{0.866796}%
\pgfsetdash{}{0pt}%
\pgfpathmoveto{\pgfqpoint{1.258229in}{2.382412in}}%
\pgfpathcurveto{\pgfqpoint{1.266466in}{2.382412in}}{\pgfqpoint{1.274366in}{2.385684in}}{\pgfqpoint{1.280190in}{2.391508in}}%
\pgfpathcurveto{\pgfqpoint{1.286014in}{2.397332in}}{\pgfqpoint{1.289286in}{2.405232in}}{\pgfqpoint{1.289286in}{2.413468in}}%
\pgfpathcurveto{\pgfqpoint{1.289286in}{2.421705in}}{\pgfqpoint{1.286014in}{2.429605in}}{\pgfqpoint{1.280190in}{2.435429in}}%
\pgfpathcurveto{\pgfqpoint{1.274366in}{2.441253in}}{\pgfqpoint{1.266466in}{2.444525in}}{\pgfqpoint{1.258229in}{2.444525in}}%
\pgfpathcurveto{\pgfqpoint{1.249993in}{2.444525in}}{\pgfqpoint{1.242093in}{2.441253in}}{\pgfqpoint{1.236269in}{2.435429in}}%
\pgfpathcurveto{\pgfqpoint{1.230445in}{2.429605in}}{\pgfqpoint{1.227173in}{2.421705in}}{\pgfqpoint{1.227173in}{2.413468in}}%
\pgfpathcurveto{\pgfqpoint{1.227173in}{2.405232in}}{\pgfqpoint{1.230445in}{2.397332in}}{\pgfqpoint{1.236269in}{2.391508in}}%
\pgfpathcurveto{\pgfqpoint{1.242093in}{2.385684in}}{\pgfqpoint{1.249993in}{2.382412in}}{\pgfqpoint{1.258229in}{2.382412in}}%
\pgfpathclose%
\pgfusepath{stroke,fill}%
\end{pgfscope}%
\begin{pgfscope}%
\pgfpathrectangle{\pgfqpoint{0.100000in}{0.212622in}}{\pgfqpoint{3.696000in}{3.696000in}}%
\pgfusepath{clip}%
\pgfsetbuttcap%
\pgfsetroundjoin%
\definecolor{currentfill}{rgb}{0.121569,0.466667,0.705882}%
\pgfsetfillcolor{currentfill}%
\pgfsetfillopacity{0.866850}%
\pgfsetlinewidth{1.003750pt}%
\definecolor{currentstroke}{rgb}{0.121569,0.466667,0.705882}%
\pgfsetstrokecolor{currentstroke}%
\pgfsetstrokeopacity{0.866850}%
\pgfsetdash{}{0pt}%
\pgfpathmoveto{\pgfqpoint{1.259359in}{2.381832in}}%
\pgfpathcurveto{\pgfqpoint{1.267595in}{2.381832in}}{\pgfqpoint{1.275495in}{2.385105in}}{\pgfqpoint{1.281319in}{2.390929in}}%
\pgfpathcurveto{\pgfqpoint{1.287143in}{2.396753in}}{\pgfqpoint{1.290416in}{2.404653in}}{\pgfqpoint{1.290416in}{2.412889in}}%
\pgfpathcurveto{\pgfqpoint{1.290416in}{2.421125in}}{\pgfqpoint{1.287143in}{2.429025in}}{\pgfqpoint{1.281319in}{2.434849in}}%
\pgfpathcurveto{\pgfqpoint{1.275495in}{2.440673in}}{\pgfqpoint{1.267595in}{2.443945in}}{\pgfqpoint{1.259359in}{2.443945in}}%
\pgfpathcurveto{\pgfqpoint{1.251123in}{2.443945in}}{\pgfqpoint{1.243223in}{2.440673in}}{\pgfqpoint{1.237399in}{2.434849in}}%
\pgfpathcurveto{\pgfqpoint{1.231575in}{2.429025in}}{\pgfqpoint{1.228303in}{2.421125in}}{\pgfqpoint{1.228303in}{2.412889in}}%
\pgfpathcurveto{\pgfqpoint{1.228303in}{2.404653in}}{\pgfqpoint{1.231575in}{2.396753in}}{\pgfqpoint{1.237399in}{2.390929in}}%
\pgfpathcurveto{\pgfqpoint{1.243223in}{2.385105in}}{\pgfqpoint{1.251123in}{2.381832in}}{\pgfqpoint{1.259359in}{2.381832in}}%
\pgfpathclose%
\pgfusepath{stroke,fill}%
\end{pgfscope}%
\begin{pgfscope}%
\pgfpathrectangle{\pgfqpoint{0.100000in}{0.212622in}}{\pgfqpoint{3.696000in}{3.696000in}}%
\pgfusepath{clip}%
\pgfsetbuttcap%
\pgfsetroundjoin%
\definecolor{currentfill}{rgb}{0.121569,0.466667,0.705882}%
\pgfsetfillcolor{currentfill}%
\pgfsetfillopacity{0.867177}%
\pgfsetlinewidth{1.003750pt}%
\definecolor{currentstroke}{rgb}{0.121569,0.466667,0.705882}%
\pgfsetstrokecolor{currentstroke}%
\pgfsetstrokeopacity{0.867177}%
\pgfsetdash{}{0pt}%
\pgfpathmoveto{\pgfqpoint{1.261205in}{2.381562in}}%
\pgfpathcurveto{\pgfqpoint{1.269442in}{2.381562in}}{\pgfqpoint{1.277342in}{2.384834in}}{\pgfqpoint{1.283166in}{2.390658in}}%
\pgfpathcurveto{\pgfqpoint{1.288989in}{2.396482in}}{\pgfqpoint{1.292262in}{2.404382in}}{\pgfqpoint{1.292262in}{2.412618in}}%
\pgfpathcurveto{\pgfqpoint{1.292262in}{2.420854in}}{\pgfqpoint{1.288989in}{2.428754in}}{\pgfqpoint{1.283166in}{2.434578in}}%
\pgfpathcurveto{\pgfqpoint{1.277342in}{2.440402in}}{\pgfqpoint{1.269442in}{2.443675in}}{\pgfqpoint{1.261205in}{2.443675in}}%
\pgfpathcurveto{\pgfqpoint{1.252969in}{2.443675in}}{\pgfqpoint{1.245069in}{2.440402in}}{\pgfqpoint{1.239245in}{2.434578in}}%
\pgfpathcurveto{\pgfqpoint{1.233421in}{2.428754in}}{\pgfqpoint{1.230149in}{2.420854in}}{\pgfqpoint{1.230149in}{2.412618in}}%
\pgfpathcurveto{\pgfqpoint{1.230149in}{2.404382in}}{\pgfqpoint{1.233421in}{2.396482in}}{\pgfqpoint{1.239245in}{2.390658in}}%
\pgfpathcurveto{\pgfqpoint{1.245069in}{2.384834in}}{\pgfqpoint{1.252969in}{2.381562in}}{\pgfqpoint{1.261205in}{2.381562in}}%
\pgfpathclose%
\pgfusepath{stroke,fill}%
\end{pgfscope}%
\begin{pgfscope}%
\pgfpathrectangle{\pgfqpoint{0.100000in}{0.212622in}}{\pgfqpoint{3.696000in}{3.696000in}}%
\pgfusepath{clip}%
\pgfsetbuttcap%
\pgfsetroundjoin%
\definecolor{currentfill}{rgb}{0.121569,0.466667,0.705882}%
\pgfsetfillcolor{currentfill}%
\pgfsetfillopacity{0.867250}%
\pgfsetlinewidth{1.003750pt}%
\definecolor{currentstroke}{rgb}{0.121569,0.466667,0.705882}%
\pgfsetstrokecolor{currentstroke}%
\pgfsetstrokeopacity{0.867250}%
\pgfsetdash{}{0pt}%
\pgfpathmoveto{\pgfqpoint{1.264901in}{2.378765in}}%
\pgfpathcurveto{\pgfqpoint{1.273137in}{2.378765in}}{\pgfqpoint{1.281037in}{2.382037in}}{\pgfqpoint{1.286861in}{2.387861in}}%
\pgfpathcurveto{\pgfqpoint{1.292685in}{2.393685in}}{\pgfqpoint{1.295958in}{2.401585in}}{\pgfqpoint{1.295958in}{2.409821in}}%
\pgfpathcurveto{\pgfqpoint{1.295958in}{2.418057in}}{\pgfqpoint{1.292685in}{2.425957in}}{\pgfqpoint{1.286861in}{2.431781in}}%
\pgfpathcurveto{\pgfqpoint{1.281037in}{2.437605in}}{\pgfqpoint{1.273137in}{2.440878in}}{\pgfqpoint{1.264901in}{2.440878in}}%
\pgfpathcurveto{\pgfqpoint{1.256665in}{2.440878in}}{\pgfqpoint{1.248765in}{2.437605in}}{\pgfqpoint{1.242941in}{2.431781in}}%
\pgfpathcurveto{\pgfqpoint{1.237117in}{2.425957in}}{\pgfqpoint{1.233845in}{2.418057in}}{\pgfqpoint{1.233845in}{2.409821in}}%
\pgfpathcurveto{\pgfqpoint{1.233845in}{2.401585in}}{\pgfqpoint{1.237117in}{2.393685in}}{\pgfqpoint{1.242941in}{2.387861in}}%
\pgfpathcurveto{\pgfqpoint{1.248765in}{2.382037in}}{\pgfqpoint{1.256665in}{2.378765in}}{\pgfqpoint{1.264901in}{2.378765in}}%
\pgfpathclose%
\pgfusepath{stroke,fill}%
\end{pgfscope}%
\begin{pgfscope}%
\pgfpathrectangle{\pgfqpoint{0.100000in}{0.212622in}}{\pgfqpoint{3.696000in}{3.696000in}}%
\pgfusepath{clip}%
\pgfsetbuttcap%
\pgfsetroundjoin%
\definecolor{currentfill}{rgb}{0.121569,0.466667,0.705882}%
\pgfsetfillcolor{currentfill}%
\pgfsetfillopacity{0.868244}%
\pgfsetlinewidth{1.003750pt}%
\definecolor{currentstroke}{rgb}{0.121569,0.466667,0.705882}%
\pgfsetstrokecolor{currentstroke}%
\pgfsetstrokeopacity{0.868244}%
\pgfsetdash{}{0pt}%
\pgfpathmoveto{\pgfqpoint{1.271026in}{2.377265in}}%
\pgfpathcurveto{\pgfqpoint{1.279262in}{2.377265in}}{\pgfqpoint{1.287162in}{2.380538in}}{\pgfqpoint{1.292986in}{2.386362in}}%
\pgfpathcurveto{\pgfqpoint{1.298810in}{2.392186in}}{\pgfqpoint{1.302083in}{2.400086in}}{\pgfqpoint{1.302083in}{2.408322in}}%
\pgfpathcurveto{\pgfqpoint{1.302083in}{2.416558in}}{\pgfqpoint{1.298810in}{2.424458in}}{\pgfqpoint{1.292986in}{2.430282in}}%
\pgfpathcurveto{\pgfqpoint{1.287162in}{2.436106in}}{\pgfqpoint{1.279262in}{2.439378in}}{\pgfqpoint{1.271026in}{2.439378in}}%
\pgfpathcurveto{\pgfqpoint{1.262790in}{2.439378in}}{\pgfqpoint{1.254890in}{2.436106in}}{\pgfqpoint{1.249066in}{2.430282in}}%
\pgfpathcurveto{\pgfqpoint{1.243242in}{2.424458in}}{\pgfqpoint{1.239970in}{2.416558in}}{\pgfqpoint{1.239970in}{2.408322in}}%
\pgfpathcurveto{\pgfqpoint{1.239970in}{2.400086in}}{\pgfqpoint{1.243242in}{2.392186in}}{\pgfqpoint{1.249066in}{2.386362in}}%
\pgfpathcurveto{\pgfqpoint{1.254890in}{2.380538in}}{\pgfqpoint{1.262790in}{2.377265in}}{\pgfqpoint{1.271026in}{2.377265in}}%
\pgfpathclose%
\pgfusepath{stroke,fill}%
\end{pgfscope}%
\begin{pgfscope}%
\pgfpathrectangle{\pgfqpoint{0.100000in}{0.212622in}}{\pgfqpoint{3.696000in}{3.696000in}}%
\pgfusepath{clip}%
\pgfsetbuttcap%
\pgfsetroundjoin%
\definecolor{currentfill}{rgb}{0.121569,0.466667,0.705882}%
\pgfsetfillcolor{currentfill}%
\pgfsetfillopacity{0.868801}%
\pgfsetlinewidth{1.003750pt}%
\definecolor{currentstroke}{rgb}{0.121569,0.466667,0.705882}%
\pgfsetstrokecolor{currentstroke}%
\pgfsetstrokeopacity{0.868801}%
\pgfsetdash{}{0pt}%
\pgfpathmoveto{\pgfqpoint{2.701946in}{1.973787in}}%
\pgfpathcurveto{\pgfqpoint{2.710182in}{1.973787in}}{\pgfqpoint{2.718082in}{1.977059in}}{\pgfqpoint{2.723906in}{1.982883in}}%
\pgfpathcurveto{\pgfqpoint{2.729730in}{1.988707in}}{\pgfqpoint{2.733002in}{1.996607in}}{\pgfqpoint{2.733002in}{2.004843in}}%
\pgfpathcurveto{\pgfqpoint{2.733002in}{2.013079in}}{\pgfqpoint{2.729730in}{2.020979in}}{\pgfqpoint{2.723906in}{2.026803in}}%
\pgfpathcurveto{\pgfqpoint{2.718082in}{2.032627in}}{\pgfqpoint{2.710182in}{2.035900in}}{\pgfqpoint{2.701946in}{2.035900in}}%
\pgfpathcurveto{\pgfqpoint{2.693710in}{2.035900in}}{\pgfqpoint{2.685810in}{2.032627in}}{\pgfqpoint{2.679986in}{2.026803in}}%
\pgfpathcurveto{\pgfqpoint{2.674162in}{2.020979in}}{\pgfqpoint{2.670889in}{2.013079in}}{\pgfqpoint{2.670889in}{2.004843in}}%
\pgfpathcurveto{\pgfqpoint{2.670889in}{1.996607in}}{\pgfqpoint{2.674162in}{1.988707in}}{\pgfqpoint{2.679986in}{1.982883in}}%
\pgfpathcurveto{\pgfqpoint{2.685810in}{1.977059in}}{\pgfqpoint{2.693710in}{1.973787in}}{\pgfqpoint{2.701946in}{1.973787in}}%
\pgfpathclose%
\pgfusepath{stroke,fill}%
\end{pgfscope}%
\begin{pgfscope}%
\pgfpathrectangle{\pgfqpoint{0.100000in}{0.212622in}}{\pgfqpoint{3.696000in}{3.696000in}}%
\pgfusepath{clip}%
\pgfsetbuttcap%
\pgfsetroundjoin%
\definecolor{currentfill}{rgb}{0.121569,0.466667,0.705882}%
\pgfsetfillcolor{currentfill}%
\pgfsetfillopacity{0.869377}%
\pgfsetlinewidth{1.003750pt}%
\definecolor{currentstroke}{rgb}{0.121569,0.466667,0.705882}%
\pgfsetstrokecolor{currentstroke}%
\pgfsetstrokeopacity{0.869377}%
\pgfsetdash{}{0pt}%
\pgfpathmoveto{\pgfqpoint{1.282281in}{2.370543in}}%
\pgfpathcurveto{\pgfqpoint{1.290517in}{2.370543in}}{\pgfqpoint{1.298417in}{2.373815in}}{\pgfqpoint{1.304241in}{2.379639in}}%
\pgfpathcurveto{\pgfqpoint{1.310065in}{2.385463in}}{\pgfqpoint{1.313338in}{2.393363in}}{\pgfqpoint{1.313338in}{2.401600in}}%
\pgfpathcurveto{\pgfqpoint{1.313338in}{2.409836in}}{\pgfqpoint{1.310065in}{2.417736in}}{\pgfqpoint{1.304241in}{2.423560in}}%
\pgfpathcurveto{\pgfqpoint{1.298417in}{2.429384in}}{\pgfqpoint{1.290517in}{2.432656in}}{\pgfqpoint{1.282281in}{2.432656in}}%
\pgfpathcurveto{\pgfqpoint{1.274045in}{2.432656in}}{\pgfqpoint{1.266145in}{2.429384in}}{\pgfqpoint{1.260321in}{2.423560in}}%
\pgfpathcurveto{\pgfqpoint{1.254497in}{2.417736in}}{\pgfqpoint{1.251225in}{2.409836in}}{\pgfqpoint{1.251225in}{2.401600in}}%
\pgfpathcurveto{\pgfqpoint{1.251225in}{2.393363in}}{\pgfqpoint{1.254497in}{2.385463in}}{\pgfqpoint{1.260321in}{2.379639in}}%
\pgfpathcurveto{\pgfqpoint{1.266145in}{2.373815in}}{\pgfqpoint{1.274045in}{2.370543in}}{\pgfqpoint{1.282281in}{2.370543in}}%
\pgfpathclose%
\pgfusepath{stroke,fill}%
\end{pgfscope}%
\begin{pgfscope}%
\pgfpathrectangle{\pgfqpoint{0.100000in}{0.212622in}}{\pgfqpoint{3.696000in}{3.696000in}}%
\pgfusepath{clip}%
\pgfsetbuttcap%
\pgfsetroundjoin%
\definecolor{currentfill}{rgb}{0.121569,0.466667,0.705882}%
\pgfsetfillcolor{currentfill}%
\pgfsetfillopacity{0.871413}%
\pgfsetlinewidth{1.003750pt}%
\definecolor{currentstroke}{rgb}{0.121569,0.466667,0.705882}%
\pgfsetstrokecolor{currentstroke}%
\pgfsetstrokeopacity{0.871413}%
\pgfsetdash{}{0pt}%
\pgfpathmoveto{\pgfqpoint{1.303222in}{2.359518in}}%
\pgfpathcurveto{\pgfqpoint{1.311459in}{2.359518in}}{\pgfqpoint{1.319359in}{2.362790in}}{\pgfqpoint{1.325182in}{2.368614in}}%
\pgfpathcurveto{\pgfqpoint{1.331006in}{2.374438in}}{\pgfqpoint{1.334279in}{2.382338in}}{\pgfqpoint{1.334279in}{2.390575in}}%
\pgfpathcurveto{\pgfqpoint{1.334279in}{2.398811in}}{\pgfqpoint{1.331006in}{2.406711in}}{\pgfqpoint{1.325182in}{2.412535in}}%
\pgfpathcurveto{\pgfqpoint{1.319359in}{2.418359in}}{\pgfqpoint{1.311459in}{2.421631in}}{\pgfqpoint{1.303222in}{2.421631in}}%
\pgfpathcurveto{\pgfqpoint{1.294986in}{2.421631in}}{\pgfqpoint{1.287086in}{2.418359in}}{\pgfqpoint{1.281262in}{2.412535in}}%
\pgfpathcurveto{\pgfqpoint{1.275438in}{2.406711in}}{\pgfqpoint{1.272166in}{2.398811in}}{\pgfqpoint{1.272166in}{2.390575in}}%
\pgfpathcurveto{\pgfqpoint{1.272166in}{2.382338in}}{\pgfqpoint{1.275438in}{2.374438in}}{\pgfqpoint{1.281262in}{2.368614in}}%
\pgfpathcurveto{\pgfqpoint{1.287086in}{2.362790in}}{\pgfqpoint{1.294986in}{2.359518in}}{\pgfqpoint{1.303222in}{2.359518in}}%
\pgfpathclose%
\pgfusepath{stroke,fill}%
\end{pgfscope}%
\begin{pgfscope}%
\pgfpathrectangle{\pgfqpoint{0.100000in}{0.212622in}}{\pgfqpoint{3.696000in}{3.696000in}}%
\pgfusepath{clip}%
\pgfsetbuttcap%
\pgfsetroundjoin%
\definecolor{currentfill}{rgb}{0.121569,0.466667,0.705882}%
\pgfsetfillcolor{currentfill}%
\pgfsetfillopacity{0.874123}%
\pgfsetlinewidth{1.003750pt}%
\definecolor{currentstroke}{rgb}{0.121569,0.466667,0.705882}%
\pgfsetstrokecolor{currentstroke}%
\pgfsetstrokeopacity{0.874123}%
\pgfsetdash{}{0pt}%
\pgfpathmoveto{\pgfqpoint{2.689869in}{1.970142in}}%
\pgfpathcurveto{\pgfqpoint{2.698106in}{1.970142in}}{\pgfqpoint{2.706006in}{1.973414in}}{\pgfqpoint{2.711830in}{1.979238in}}%
\pgfpathcurveto{\pgfqpoint{2.717654in}{1.985062in}}{\pgfqpoint{2.720926in}{1.992962in}}{\pgfqpoint{2.720926in}{2.001199in}}%
\pgfpathcurveto{\pgfqpoint{2.720926in}{2.009435in}}{\pgfqpoint{2.717654in}{2.017335in}}{\pgfqpoint{2.711830in}{2.023159in}}%
\pgfpathcurveto{\pgfqpoint{2.706006in}{2.028983in}}{\pgfqpoint{2.698106in}{2.032255in}}{\pgfqpoint{2.689869in}{2.032255in}}%
\pgfpathcurveto{\pgfqpoint{2.681633in}{2.032255in}}{\pgfqpoint{2.673733in}{2.028983in}}{\pgfqpoint{2.667909in}{2.023159in}}%
\pgfpathcurveto{\pgfqpoint{2.662085in}{2.017335in}}{\pgfqpoint{2.658813in}{2.009435in}}{\pgfqpoint{2.658813in}{2.001199in}}%
\pgfpathcurveto{\pgfqpoint{2.658813in}{1.992962in}}{\pgfqpoint{2.662085in}{1.985062in}}{\pgfqpoint{2.667909in}{1.979238in}}%
\pgfpathcurveto{\pgfqpoint{2.673733in}{1.973414in}}{\pgfqpoint{2.681633in}{1.970142in}}{\pgfqpoint{2.689869in}{1.970142in}}%
\pgfpathclose%
\pgfusepath{stroke,fill}%
\end{pgfscope}%
\begin{pgfscope}%
\pgfpathrectangle{\pgfqpoint{0.100000in}{0.212622in}}{\pgfqpoint{3.696000in}{3.696000in}}%
\pgfusepath{clip}%
\pgfsetbuttcap%
\pgfsetroundjoin%
\definecolor{currentfill}{rgb}{0.121569,0.466667,0.705882}%
\pgfsetfillcolor{currentfill}%
\pgfsetfillopacity{0.875301}%
\pgfsetlinewidth{1.003750pt}%
\definecolor{currentstroke}{rgb}{0.121569,0.466667,0.705882}%
\pgfsetstrokecolor{currentstroke}%
\pgfsetstrokeopacity{0.875301}%
\pgfsetdash{}{0pt}%
\pgfpathmoveto{\pgfqpoint{1.340107in}{2.337151in}}%
\pgfpathcurveto{\pgfqpoint{1.348343in}{2.337151in}}{\pgfqpoint{1.356243in}{2.340424in}}{\pgfqpoint{1.362067in}{2.346247in}}%
\pgfpathcurveto{\pgfqpoint{1.367891in}{2.352071in}}{\pgfqpoint{1.371163in}{2.359971in}}{\pgfqpoint{1.371163in}{2.368208in}}%
\pgfpathcurveto{\pgfqpoint{1.371163in}{2.376444in}}{\pgfqpoint{1.367891in}{2.384344in}}{\pgfqpoint{1.362067in}{2.390168in}}%
\pgfpathcurveto{\pgfqpoint{1.356243in}{2.395992in}}{\pgfqpoint{1.348343in}{2.399264in}}{\pgfqpoint{1.340107in}{2.399264in}}%
\pgfpathcurveto{\pgfqpoint{1.331870in}{2.399264in}}{\pgfqpoint{1.323970in}{2.395992in}}{\pgfqpoint{1.318147in}{2.390168in}}%
\pgfpathcurveto{\pgfqpoint{1.312323in}{2.384344in}}{\pgfqpoint{1.309050in}{2.376444in}}{\pgfqpoint{1.309050in}{2.368208in}}%
\pgfpathcurveto{\pgfqpoint{1.309050in}{2.359971in}}{\pgfqpoint{1.312323in}{2.352071in}}{\pgfqpoint{1.318147in}{2.346247in}}%
\pgfpathcurveto{\pgfqpoint{1.323970in}{2.340424in}}{\pgfqpoint{1.331870in}{2.337151in}}{\pgfqpoint{1.340107in}{2.337151in}}%
\pgfpathclose%
\pgfusepath{stroke,fill}%
\end{pgfscope}%
\begin{pgfscope}%
\pgfpathrectangle{\pgfqpoint{0.100000in}{0.212622in}}{\pgfqpoint{3.696000in}{3.696000in}}%
\pgfusepath{clip}%
\pgfsetbuttcap%
\pgfsetroundjoin%
\definecolor{currentfill}{rgb}{0.121569,0.466667,0.705882}%
\pgfsetfillcolor{currentfill}%
\pgfsetfillopacity{0.878602}%
\pgfsetlinewidth{1.003750pt}%
\definecolor{currentstroke}{rgb}{0.121569,0.466667,0.705882}%
\pgfsetstrokecolor{currentstroke}%
\pgfsetstrokeopacity{0.878602}%
\pgfsetdash{}{0pt}%
\pgfpathmoveto{\pgfqpoint{1.375672in}{2.314769in}}%
\pgfpathcurveto{\pgfqpoint{1.383908in}{2.314769in}}{\pgfqpoint{1.391808in}{2.318041in}}{\pgfqpoint{1.397632in}{2.323865in}}%
\pgfpathcurveto{\pgfqpoint{1.403456in}{2.329689in}}{\pgfqpoint{1.406728in}{2.337589in}}{\pgfqpoint{1.406728in}{2.345826in}}%
\pgfpathcurveto{\pgfqpoint{1.406728in}{2.354062in}}{\pgfqpoint{1.403456in}{2.361962in}}{\pgfqpoint{1.397632in}{2.367786in}}%
\pgfpathcurveto{\pgfqpoint{1.391808in}{2.373610in}}{\pgfqpoint{1.383908in}{2.376882in}}{\pgfqpoint{1.375672in}{2.376882in}}%
\pgfpathcurveto{\pgfqpoint{1.367436in}{2.376882in}}{\pgfqpoint{1.359536in}{2.373610in}}{\pgfqpoint{1.353712in}{2.367786in}}%
\pgfpathcurveto{\pgfqpoint{1.347888in}{2.361962in}}{\pgfqpoint{1.344615in}{2.354062in}}{\pgfqpoint{1.344615in}{2.345826in}}%
\pgfpathcurveto{\pgfqpoint{1.344615in}{2.337589in}}{\pgfqpoint{1.347888in}{2.329689in}}{\pgfqpoint{1.353712in}{2.323865in}}%
\pgfpathcurveto{\pgfqpoint{1.359536in}{2.318041in}}{\pgfqpoint{1.367436in}{2.314769in}}{\pgfqpoint{1.375672in}{2.314769in}}%
\pgfpathclose%
\pgfusepath{stroke,fill}%
\end{pgfscope}%
\begin{pgfscope}%
\pgfpathrectangle{\pgfqpoint{0.100000in}{0.212622in}}{\pgfqpoint{3.696000in}{3.696000in}}%
\pgfusepath{clip}%
\pgfsetbuttcap%
\pgfsetroundjoin%
\definecolor{currentfill}{rgb}{0.121569,0.466667,0.705882}%
\pgfsetfillcolor{currentfill}%
\pgfsetfillopacity{0.880230}%
\pgfsetlinewidth{1.003750pt}%
\definecolor{currentstroke}{rgb}{0.121569,0.466667,0.705882}%
\pgfsetstrokecolor{currentstroke}%
\pgfsetstrokeopacity{0.880230}%
\pgfsetdash{}{0pt}%
\pgfpathmoveto{\pgfqpoint{2.678161in}{1.958876in}}%
\pgfpathcurveto{\pgfqpoint{2.686397in}{1.958876in}}{\pgfqpoint{2.694297in}{1.962149in}}{\pgfqpoint{2.700121in}{1.967972in}}%
\pgfpathcurveto{\pgfqpoint{2.705945in}{1.973796in}}{\pgfqpoint{2.709218in}{1.981696in}}{\pgfqpoint{2.709218in}{1.989933in}}%
\pgfpathcurveto{\pgfqpoint{2.709218in}{1.998169in}}{\pgfqpoint{2.705945in}{2.006069in}}{\pgfqpoint{2.700121in}{2.011893in}}%
\pgfpathcurveto{\pgfqpoint{2.694297in}{2.017717in}}{\pgfqpoint{2.686397in}{2.020989in}}{\pgfqpoint{2.678161in}{2.020989in}}%
\pgfpathcurveto{\pgfqpoint{2.669925in}{2.020989in}}{\pgfqpoint{2.662025in}{2.017717in}}{\pgfqpoint{2.656201in}{2.011893in}}%
\pgfpathcurveto{\pgfqpoint{2.650377in}{2.006069in}}{\pgfqpoint{2.647105in}{1.998169in}}{\pgfqpoint{2.647105in}{1.989933in}}%
\pgfpathcurveto{\pgfqpoint{2.647105in}{1.981696in}}{\pgfqpoint{2.650377in}{1.973796in}}{\pgfqpoint{2.656201in}{1.967972in}}%
\pgfpathcurveto{\pgfqpoint{2.662025in}{1.962149in}}{\pgfqpoint{2.669925in}{1.958876in}}{\pgfqpoint{2.678161in}{1.958876in}}%
\pgfpathclose%
\pgfusepath{stroke,fill}%
\end{pgfscope}%
\begin{pgfscope}%
\pgfpathrectangle{\pgfqpoint{0.100000in}{0.212622in}}{\pgfqpoint{3.696000in}{3.696000in}}%
\pgfusepath{clip}%
\pgfsetbuttcap%
\pgfsetroundjoin%
\definecolor{currentfill}{rgb}{0.121569,0.466667,0.705882}%
\pgfsetfillcolor{currentfill}%
\pgfsetfillopacity{0.882776}%
\pgfsetlinewidth{1.003750pt}%
\definecolor{currentstroke}{rgb}{0.121569,0.466667,0.705882}%
\pgfsetstrokecolor{currentstroke}%
\pgfsetstrokeopacity{0.882776}%
\pgfsetdash{}{0pt}%
\pgfpathmoveto{\pgfqpoint{1.405994in}{2.299397in}}%
\pgfpathcurveto{\pgfqpoint{1.414230in}{2.299397in}}{\pgfqpoint{1.422130in}{2.302670in}}{\pgfqpoint{1.427954in}{2.308494in}}%
\pgfpathcurveto{\pgfqpoint{1.433778in}{2.314318in}}{\pgfqpoint{1.437050in}{2.322218in}}{\pgfqpoint{1.437050in}{2.330454in}}%
\pgfpathcurveto{\pgfqpoint{1.437050in}{2.338690in}}{\pgfqpoint{1.433778in}{2.346590in}}{\pgfqpoint{1.427954in}{2.352414in}}%
\pgfpathcurveto{\pgfqpoint{1.422130in}{2.358238in}}{\pgfqpoint{1.414230in}{2.361510in}}{\pgfqpoint{1.405994in}{2.361510in}}%
\pgfpathcurveto{\pgfqpoint{1.397757in}{2.361510in}}{\pgfqpoint{1.389857in}{2.358238in}}{\pgfqpoint{1.384033in}{2.352414in}}%
\pgfpathcurveto{\pgfqpoint{1.378209in}{2.346590in}}{\pgfqpoint{1.374937in}{2.338690in}}{\pgfqpoint{1.374937in}{2.330454in}}%
\pgfpathcurveto{\pgfqpoint{1.374937in}{2.322218in}}{\pgfqpoint{1.378209in}{2.314318in}}{\pgfqpoint{1.384033in}{2.308494in}}%
\pgfpathcurveto{\pgfqpoint{1.389857in}{2.302670in}}{\pgfqpoint{1.397757in}{2.299397in}}{\pgfqpoint{1.405994in}{2.299397in}}%
\pgfpathclose%
\pgfusepath{stroke,fill}%
\end{pgfscope}%
\begin{pgfscope}%
\pgfpathrectangle{\pgfqpoint{0.100000in}{0.212622in}}{\pgfqpoint{3.696000in}{3.696000in}}%
\pgfusepath{clip}%
\pgfsetbuttcap%
\pgfsetroundjoin%
\definecolor{currentfill}{rgb}{0.121569,0.466667,0.705882}%
\pgfsetfillcolor{currentfill}%
\pgfsetfillopacity{0.884396}%
\pgfsetlinewidth{1.003750pt}%
\definecolor{currentstroke}{rgb}{0.121569,0.466667,0.705882}%
\pgfsetstrokecolor{currentstroke}%
\pgfsetstrokeopacity{0.884396}%
\pgfsetdash{}{0pt}%
\pgfpathmoveto{\pgfqpoint{1.428873in}{2.283523in}}%
\pgfpathcurveto{\pgfqpoint{1.437109in}{2.283523in}}{\pgfqpoint{1.445009in}{2.286795in}}{\pgfqpoint{1.450833in}{2.292619in}}%
\pgfpathcurveto{\pgfqpoint{1.456657in}{2.298443in}}{\pgfqpoint{1.459930in}{2.306343in}}{\pgfqpoint{1.459930in}{2.314579in}}%
\pgfpathcurveto{\pgfqpoint{1.459930in}{2.322816in}}{\pgfqpoint{1.456657in}{2.330716in}}{\pgfqpoint{1.450833in}{2.336540in}}%
\pgfpathcurveto{\pgfqpoint{1.445009in}{2.342364in}}{\pgfqpoint{1.437109in}{2.345636in}}{\pgfqpoint{1.428873in}{2.345636in}}%
\pgfpathcurveto{\pgfqpoint{1.420637in}{2.345636in}}{\pgfqpoint{1.412737in}{2.342364in}}{\pgfqpoint{1.406913in}{2.336540in}}%
\pgfpathcurveto{\pgfqpoint{1.401089in}{2.330716in}}{\pgfqpoint{1.397817in}{2.322816in}}{\pgfqpoint{1.397817in}{2.314579in}}%
\pgfpathcurveto{\pgfqpoint{1.397817in}{2.306343in}}{\pgfqpoint{1.401089in}{2.298443in}}{\pgfqpoint{1.406913in}{2.292619in}}%
\pgfpathcurveto{\pgfqpoint{1.412737in}{2.286795in}}{\pgfqpoint{1.420637in}{2.283523in}}{\pgfqpoint{1.428873in}{2.283523in}}%
\pgfpathclose%
\pgfusepath{stroke,fill}%
\end{pgfscope}%
\begin{pgfscope}%
\pgfpathrectangle{\pgfqpoint{0.100000in}{0.212622in}}{\pgfqpoint{3.696000in}{3.696000in}}%
\pgfusepath{clip}%
\pgfsetbuttcap%
\pgfsetroundjoin%
\definecolor{currentfill}{rgb}{0.121569,0.466667,0.705882}%
\pgfsetfillcolor{currentfill}%
\pgfsetfillopacity{0.886401}%
\pgfsetlinewidth{1.003750pt}%
\definecolor{currentstroke}{rgb}{0.121569,0.466667,0.705882}%
\pgfsetstrokecolor{currentstroke}%
\pgfsetstrokeopacity{0.886401}%
\pgfsetdash{}{0pt}%
\pgfpathmoveto{\pgfqpoint{1.448551in}{2.274740in}}%
\pgfpathcurveto{\pgfqpoint{1.456787in}{2.274740in}}{\pgfqpoint{1.464687in}{2.278012in}}{\pgfqpoint{1.470511in}{2.283836in}}%
\pgfpathcurveto{\pgfqpoint{1.476335in}{2.289660in}}{\pgfqpoint{1.479607in}{2.297560in}}{\pgfqpoint{1.479607in}{2.305796in}}%
\pgfpathcurveto{\pgfqpoint{1.479607in}{2.314032in}}{\pgfqpoint{1.476335in}{2.321932in}}{\pgfqpoint{1.470511in}{2.327756in}}%
\pgfpathcurveto{\pgfqpoint{1.464687in}{2.333580in}}{\pgfqpoint{1.456787in}{2.336853in}}{\pgfqpoint{1.448551in}{2.336853in}}%
\pgfpathcurveto{\pgfqpoint{1.440314in}{2.336853in}}{\pgfqpoint{1.432414in}{2.333580in}}{\pgfqpoint{1.426590in}{2.327756in}}%
\pgfpathcurveto{\pgfqpoint{1.420766in}{2.321932in}}{\pgfqpoint{1.417494in}{2.314032in}}{\pgfqpoint{1.417494in}{2.305796in}}%
\pgfpathcurveto{\pgfqpoint{1.417494in}{2.297560in}}{\pgfqpoint{1.420766in}{2.289660in}}{\pgfqpoint{1.426590in}{2.283836in}}%
\pgfpathcurveto{\pgfqpoint{1.432414in}{2.278012in}}{\pgfqpoint{1.440314in}{2.274740in}}{\pgfqpoint{1.448551in}{2.274740in}}%
\pgfpathclose%
\pgfusepath{stroke,fill}%
\end{pgfscope}%
\begin{pgfscope}%
\pgfpathrectangle{\pgfqpoint{0.100000in}{0.212622in}}{\pgfqpoint{3.696000in}{3.696000in}}%
\pgfusepath{clip}%
\pgfsetbuttcap%
\pgfsetroundjoin%
\definecolor{currentfill}{rgb}{0.121569,0.466667,0.705882}%
\pgfsetfillcolor{currentfill}%
\pgfsetfillopacity{0.887092}%
\pgfsetlinewidth{1.003750pt}%
\definecolor{currentstroke}{rgb}{0.121569,0.466667,0.705882}%
\pgfsetstrokecolor{currentstroke}%
\pgfsetstrokeopacity{0.887092}%
\pgfsetdash{}{0pt}%
\pgfpathmoveto{\pgfqpoint{2.662128in}{1.950146in}}%
\pgfpathcurveto{\pgfqpoint{2.670364in}{1.950146in}}{\pgfqpoint{2.678264in}{1.953418in}}{\pgfqpoint{2.684088in}{1.959242in}}%
\pgfpathcurveto{\pgfqpoint{2.689912in}{1.965066in}}{\pgfqpoint{2.693185in}{1.972966in}}{\pgfqpoint{2.693185in}{1.981202in}}%
\pgfpathcurveto{\pgfqpoint{2.693185in}{1.989438in}}{\pgfqpoint{2.689912in}{1.997338in}}{\pgfqpoint{2.684088in}{2.003162in}}%
\pgfpathcurveto{\pgfqpoint{2.678264in}{2.008986in}}{\pgfqpoint{2.670364in}{2.012259in}}{\pgfqpoint{2.662128in}{2.012259in}}%
\pgfpathcurveto{\pgfqpoint{2.653892in}{2.012259in}}{\pgfqpoint{2.645992in}{2.008986in}}{\pgfqpoint{2.640168in}{2.003162in}}%
\pgfpathcurveto{\pgfqpoint{2.634344in}{1.997338in}}{\pgfqpoint{2.631072in}{1.989438in}}{\pgfqpoint{2.631072in}{1.981202in}}%
\pgfpathcurveto{\pgfqpoint{2.631072in}{1.972966in}}{\pgfqpoint{2.634344in}{1.965066in}}{\pgfqpoint{2.640168in}{1.959242in}}%
\pgfpathcurveto{\pgfqpoint{2.645992in}{1.953418in}}{\pgfqpoint{2.653892in}{1.950146in}}{\pgfqpoint{2.662128in}{1.950146in}}%
\pgfpathclose%
\pgfusepath{stroke,fill}%
\end{pgfscope}%
\begin{pgfscope}%
\pgfpathrectangle{\pgfqpoint{0.100000in}{0.212622in}}{\pgfqpoint{3.696000in}{3.696000in}}%
\pgfusepath{clip}%
\pgfsetbuttcap%
\pgfsetroundjoin%
\definecolor{currentfill}{rgb}{0.121569,0.466667,0.705882}%
\pgfsetfillcolor{currentfill}%
\pgfsetfillopacity{0.887858}%
\pgfsetlinewidth{1.003750pt}%
\definecolor{currentstroke}{rgb}{0.121569,0.466667,0.705882}%
\pgfsetstrokecolor{currentstroke}%
\pgfsetstrokeopacity{0.887858}%
\pgfsetdash{}{0pt}%
\pgfpathmoveto{\pgfqpoint{1.465250in}{2.267173in}}%
\pgfpathcurveto{\pgfqpoint{1.473486in}{2.267173in}}{\pgfqpoint{1.481386in}{2.270445in}}{\pgfqpoint{1.487210in}{2.276269in}}%
\pgfpathcurveto{\pgfqpoint{1.493034in}{2.282093in}}{\pgfqpoint{1.496306in}{2.289993in}}{\pgfqpoint{1.496306in}{2.298229in}}%
\pgfpathcurveto{\pgfqpoint{1.496306in}{2.306466in}}{\pgfqpoint{1.493034in}{2.314366in}}{\pgfqpoint{1.487210in}{2.320190in}}%
\pgfpathcurveto{\pgfqpoint{1.481386in}{2.326014in}}{\pgfqpoint{1.473486in}{2.329286in}}{\pgfqpoint{1.465250in}{2.329286in}}%
\pgfpathcurveto{\pgfqpoint{1.457013in}{2.329286in}}{\pgfqpoint{1.449113in}{2.326014in}}{\pgfqpoint{1.443289in}{2.320190in}}%
\pgfpathcurveto{\pgfqpoint{1.437465in}{2.314366in}}{\pgfqpoint{1.434193in}{2.306466in}}{\pgfqpoint{1.434193in}{2.298229in}}%
\pgfpathcurveto{\pgfqpoint{1.434193in}{2.289993in}}{\pgfqpoint{1.437465in}{2.282093in}}{\pgfqpoint{1.443289in}{2.276269in}}%
\pgfpathcurveto{\pgfqpoint{1.449113in}{2.270445in}}{\pgfqpoint{1.457013in}{2.267173in}}{\pgfqpoint{1.465250in}{2.267173in}}%
\pgfpathclose%
\pgfusepath{stroke,fill}%
\end{pgfscope}%
\begin{pgfscope}%
\pgfpathrectangle{\pgfqpoint{0.100000in}{0.212622in}}{\pgfqpoint{3.696000in}{3.696000in}}%
\pgfusepath{clip}%
\pgfsetbuttcap%
\pgfsetroundjoin%
\definecolor{currentfill}{rgb}{0.121569,0.466667,0.705882}%
\pgfsetfillcolor{currentfill}%
\pgfsetfillopacity{0.889691}%
\pgfsetlinewidth{1.003750pt}%
\definecolor{currentstroke}{rgb}{0.121569,0.466667,0.705882}%
\pgfsetstrokecolor{currentstroke}%
\pgfsetstrokeopacity{0.889691}%
\pgfsetdash{}{0pt}%
\pgfpathmoveto{\pgfqpoint{1.479158in}{2.262488in}}%
\pgfpathcurveto{\pgfqpoint{1.487394in}{2.262488in}}{\pgfqpoint{1.495295in}{2.265761in}}{\pgfqpoint{1.501118in}{2.271584in}}%
\pgfpathcurveto{\pgfqpoint{1.506942in}{2.277408in}}{\pgfqpoint{1.510215in}{2.285308in}}{\pgfqpoint{1.510215in}{2.293545in}}%
\pgfpathcurveto{\pgfqpoint{1.510215in}{2.301781in}}{\pgfqpoint{1.506942in}{2.309681in}}{\pgfqpoint{1.501118in}{2.315505in}}%
\pgfpathcurveto{\pgfqpoint{1.495295in}{2.321329in}}{\pgfqpoint{1.487394in}{2.324601in}}{\pgfqpoint{1.479158in}{2.324601in}}%
\pgfpathcurveto{\pgfqpoint{1.470922in}{2.324601in}}{\pgfqpoint{1.463022in}{2.321329in}}{\pgfqpoint{1.457198in}{2.315505in}}%
\pgfpathcurveto{\pgfqpoint{1.451374in}{2.309681in}}{\pgfqpoint{1.448102in}{2.301781in}}{\pgfqpoint{1.448102in}{2.293545in}}%
\pgfpathcurveto{\pgfqpoint{1.448102in}{2.285308in}}{\pgfqpoint{1.451374in}{2.277408in}}{\pgfqpoint{1.457198in}{2.271584in}}%
\pgfpathcurveto{\pgfqpoint{1.463022in}{2.265761in}}{\pgfqpoint{1.470922in}{2.262488in}}{\pgfqpoint{1.479158in}{2.262488in}}%
\pgfpathclose%
\pgfusepath{stroke,fill}%
\end{pgfscope}%
\begin{pgfscope}%
\pgfpathrectangle{\pgfqpoint{0.100000in}{0.212622in}}{\pgfqpoint{3.696000in}{3.696000in}}%
\pgfusepath{clip}%
\pgfsetbuttcap%
\pgfsetroundjoin%
\definecolor{currentfill}{rgb}{0.121569,0.466667,0.705882}%
\pgfsetfillcolor{currentfill}%
\pgfsetfillopacity{0.890953}%
\pgfsetlinewidth{1.003750pt}%
\definecolor{currentstroke}{rgb}{0.121569,0.466667,0.705882}%
\pgfsetstrokecolor{currentstroke}%
\pgfsetstrokeopacity{0.890953}%
\pgfsetdash{}{0pt}%
\pgfpathmoveto{\pgfqpoint{2.653374in}{1.945675in}}%
\pgfpathcurveto{\pgfqpoint{2.661610in}{1.945675in}}{\pgfqpoint{2.669510in}{1.948947in}}{\pgfqpoint{2.675334in}{1.954771in}}%
\pgfpathcurveto{\pgfqpoint{2.681158in}{1.960595in}}{\pgfqpoint{2.684430in}{1.968495in}}{\pgfqpoint{2.684430in}{1.976731in}}%
\pgfpathcurveto{\pgfqpoint{2.684430in}{1.984967in}}{\pgfqpoint{2.681158in}{1.992867in}}{\pgfqpoint{2.675334in}{1.998691in}}%
\pgfpathcurveto{\pgfqpoint{2.669510in}{2.004515in}}{\pgfqpoint{2.661610in}{2.007788in}}{\pgfqpoint{2.653374in}{2.007788in}}%
\pgfpathcurveto{\pgfqpoint{2.645137in}{2.007788in}}{\pgfqpoint{2.637237in}{2.004515in}}{\pgfqpoint{2.631414in}{1.998691in}}%
\pgfpathcurveto{\pgfqpoint{2.625590in}{1.992867in}}{\pgfqpoint{2.622317in}{1.984967in}}{\pgfqpoint{2.622317in}{1.976731in}}%
\pgfpathcurveto{\pgfqpoint{2.622317in}{1.968495in}}{\pgfqpoint{2.625590in}{1.960595in}}{\pgfqpoint{2.631414in}{1.954771in}}%
\pgfpathcurveto{\pgfqpoint{2.637237in}{1.948947in}}{\pgfqpoint{2.645137in}{1.945675in}}{\pgfqpoint{2.653374in}{1.945675in}}%
\pgfpathclose%
\pgfusepath{stroke,fill}%
\end{pgfscope}%
\begin{pgfscope}%
\pgfpathrectangle{\pgfqpoint{0.100000in}{0.212622in}}{\pgfqpoint{3.696000in}{3.696000in}}%
\pgfusepath{clip}%
\pgfsetbuttcap%
\pgfsetroundjoin%
\definecolor{currentfill}{rgb}{0.121569,0.466667,0.705882}%
\pgfsetfillcolor{currentfill}%
\pgfsetfillopacity{0.893059}%
\pgfsetlinewidth{1.003750pt}%
\definecolor{currentstroke}{rgb}{0.121569,0.466667,0.705882}%
\pgfsetstrokecolor{currentstroke}%
\pgfsetstrokeopacity{0.893059}%
\pgfsetdash{}{0pt}%
\pgfpathmoveto{\pgfqpoint{2.648447in}{1.943202in}}%
\pgfpathcurveto{\pgfqpoint{2.656683in}{1.943202in}}{\pgfqpoint{2.664583in}{1.946474in}}{\pgfqpoint{2.670407in}{1.952298in}}%
\pgfpathcurveto{\pgfqpoint{2.676231in}{1.958122in}}{\pgfqpoint{2.679503in}{1.966022in}}{\pgfqpoint{2.679503in}{1.974258in}}%
\pgfpathcurveto{\pgfqpoint{2.679503in}{1.982495in}}{\pgfqpoint{2.676231in}{1.990395in}}{\pgfqpoint{2.670407in}{1.996219in}}%
\pgfpathcurveto{\pgfqpoint{2.664583in}{2.002043in}}{\pgfqpoint{2.656683in}{2.005315in}}{\pgfqpoint{2.648447in}{2.005315in}}%
\pgfpathcurveto{\pgfqpoint{2.640210in}{2.005315in}}{\pgfqpoint{2.632310in}{2.002043in}}{\pgfqpoint{2.626486in}{1.996219in}}%
\pgfpathcurveto{\pgfqpoint{2.620662in}{1.990395in}}{\pgfqpoint{2.617390in}{1.982495in}}{\pgfqpoint{2.617390in}{1.974258in}}%
\pgfpathcurveto{\pgfqpoint{2.617390in}{1.966022in}}{\pgfqpoint{2.620662in}{1.958122in}}{\pgfqpoint{2.626486in}{1.952298in}}%
\pgfpathcurveto{\pgfqpoint{2.632310in}{1.946474in}}{\pgfqpoint{2.640210in}{1.943202in}}{\pgfqpoint{2.648447in}{1.943202in}}%
\pgfpathclose%
\pgfusepath{stroke,fill}%
\end{pgfscope}%
\begin{pgfscope}%
\pgfpathrectangle{\pgfqpoint{0.100000in}{0.212622in}}{\pgfqpoint{3.696000in}{3.696000in}}%
\pgfusepath{clip}%
\pgfsetbuttcap%
\pgfsetroundjoin%
\definecolor{currentfill}{rgb}{0.121569,0.466667,0.705882}%
\pgfsetfillcolor{currentfill}%
\pgfsetfillopacity{0.893323}%
\pgfsetlinewidth{1.003750pt}%
\definecolor{currentstroke}{rgb}{0.121569,0.466667,0.705882}%
\pgfsetstrokecolor{currentstroke}%
\pgfsetstrokeopacity{0.893323}%
\pgfsetdash{}{0pt}%
\pgfpathmoveto{\pgfqpoint{1.502832in}{2.250711in}}%
\pgfpathcurveto{\pgfqpoint{1.511068in}{2.250711in}}{\pgfqpoint{1.518968in}{2.253983in}}{\pgfqpoint{1.524792in}{2.259807in}}%
\pgfpathcurveto{\pgfqpoint{1.530616in}{2.265631in}}{\pgfqpoint{1.533888in}{2.273531in}}{\pgfqpoint{1.533888in}{2.281767in}}%
\pgfpathcurveto{\pgfqpoint{1.533888in}{2.290004in}}{\pgfqpoint{1.530616in}{2.297904in}}{\pgfqpoint{1.524792in}{2.303727in}}%
\pgfpathcurveto{\pgfqpoint{1.518968in}{2.309551in}}{\pgfqpoint{1.511068in}{2.312824in}}{\pgfqpoint{1.502832in}{2.312824in}}%
\pgfpathcurveto{\pgfqpoint{1.494596in}{2.312824in}}{\pgfqpoint{1.486696in}{2.309551in}}{\pgfqpoint{1.480872in}{2.303727in}}%
\pgfpathcurveto{\pgfqpoint{1.475048in}{2.297904in}}{\pgfqpoint{1.471775in}{2.290004in}}{\pgfqpoint{1.471775in}{2.281767in}}%
\pgfpathcurveto{\pgfqpoint{1.471775in}{2.273531in}}{\pgfqpoint{1.475048in}{2.265631in}}{\pgfqpoint{1.480872in}{2.259807in}}%
\pgfpathcurveto{\pgfqpoint{1.486696in}{2.253983in}}{\pgfqpoint{1.494596in}{2.250711in}}{\pgfqpoint{1.502832in}{2.250711in}}%
\pgfpathclose%
\pgfusepath{stroke,fill}%
\end{pgfscope}%
\begin{pgfscope}%
\pgfpathrectangle{\pgfqpoint{0.100000in}{0.212622in}}{\pgfqpoint{3.696000in}{3.696000in}}%
\pgfusepath{clip}%
\pgfsetbuttcap%
\pgfsetroundjoin%
\definecolor{currentfill}{rgb}{0.121569,0.466667,0.705882}%
\pgfsetfillcolor{currentfill}%
\pgfsetfillopacity{0.894099}%
\pgfsetlinewidth{1.003750pt}%
\definecolor{currentstroke}{rgb}{0.121569,0.466667,0.705882}%
\pgfsetstrokecolor{currentstroke}%
\pgfsetstrokeopacity{0.894099}%
\pgfsetdash{}{0pt}%
\pgfpathmoveto{\pgfqpoint{2.645333in}{1.941717in}}%
\pgfpathcurveto{\pgfqpoint{2.653569in}{1.941717in}}{\pgfqpoint{2.661469in}{1.944989in}}{\pgfqpoint{2.667293in}{1.950813in}}%
\pgfpathcurveto{\pgfqpoint{2.673117in}{1.956637in}}{\pgfqpoint{2.676389in}{1.964537in}}{\pgfqpoint{2.676389in}{1.972773in}}%
\pgfpathcurveto{\pgfqpoint{2.676389in}{1.981009in}}{\pgfqpoint{2.673117in}{1.988909in}}{\pgfqpoint{2.667293in}{1.994733in}}%
\pgfpathcurveto{\pgfqpoint{2.661469in}{2.000557in}}{\pgfqpoint{2.653569in}{2.003830in}}{\pgfqpoint{2.645333in}{2.003830in}}%
\pgfpathcurveto{\pgfqpoint{2.637096in}{2.003830in}}{\pgfqpoint{2.629196in}{2.000557in}}{\pgfqpoint{2.623372in}{1.994733in}}%
\pgfpathcurveto{\pgfqpoint{2.617549in}{1.988909in}}{\pgfqpoint{2.614276in}{1.981009in}}{\pgfqpoint{2.614276in}{1.972773in}}%
\pgfpathcurveto{\pgfqpoint{2.614276in}{1.964537in}}{\pgfqpoint{2.617549in}{1.956637in}}{\pgfqpoint{2.623372in}{1.950813in}}%
\pgfpathcurveto{\pgfqpoint{2.629196in}{1.944989in}}{\pgfqpoint{2.637096in}{1.941717in}}{\pgfqpoint{2.645333in}{1.941717in}}%
\pgfpathclose%
\pgfusepath{stroke,fill}%
\end{pgfscope}%
\begin{pgfscope}%
\pgfpathrectangle{\pgfqpoint{0.100000in}{0.212622in}}{\pgfqpoint{3.696000in}{3.696000in}}%
\pgfusepath{clip}%
\pgfsetbuttcap%
\pgfsetroundjoin%
\definecolor{currentfill}{rgb}{0.121569,0.466667,0.705882}%
\pgfsetfillcolor{currentfill}%
\pgfsetfillopacity{0.894693}%
\pgfsetlinewidth{1.003750pt}%
\definecolor{currentstroke}{rgb}{0.121569,0.466667,0.705882}%
\pgfsetstrokecolor{currentstroke}%
\pgfsetstrokeopacity{0.894693}%
\pgfsetdash{}{0pt}%
\pgfpathmoveto{\pgfqpoint{2.643691in}{1.940919in}}%
\pgfpathcurveto{\pgfqpoint{2.651927in}{1.940919in}}{\pgfqpoint{2.659827in}{1.944192in}}{\pgfqpoint{2.665651in}{1.950015in}}%
\pgfpathcurveto{\pgfqpoint{2.671475in}{1.955839in}}{\pgfqpoint{2.674747in}{1.963739in}}{\pgfqpoint{2.674747in}{1.971976in}}%
\pgfpathcurveto{\pgfqpoint{2.674747in}{1.980212in}}{\pgfqpoint{2.671475in}{1.988112in}}{\pgfqpoint{2.665651in}{1.993936in}}%
\pgfpathcurveto{\pgfqpoint{2.659827in}{1.999760in}}{\pgfqpoint{2.651927in}{2.003032in}}{\pgfqpoint{2.643691in}{2.003032in}}%
\pgfpathcurveto{\pgfqpoint{2.635454in}{2.003032in}}{\pgfqpoint{2.627554in}{1.999760in}}{\pgfqpoint{2.621730in}{1.993936in}}%
\pgfpathcurveto{\pgfqpoint{2.615906in}{1.988112in}}{\pgfqpoint{2.612634in}{1.980212in}}{\pgfqpoint{2.612634in}{1.971976in}}%
\pgfpathcurveto{\pgfqpoint{2.612634in}{1.963739in}}{\pgfqpoint{2.615906in}{1.955839in}}{\pgfqpoint{2.621730in}{1.950015in}}%
\pgfpathcurveto{\pgfqpoint{2.627554in}{1.944192in}}{\pgfqpoint{2.635454in}{1.940919in}}{\pgfqpoint{2.643691in}{1.940919in}}%
\pgfpathclose%
\pgfusepath{stroke,fill}%
\end{pgfscope}%
\begin{pgfscope}%
\pgfpathrectangle{\pgfqpoint{0.100000in}{0.212622in}}{\pgfqpoint{3.696000in}{3.696000in}}%
\pgfusepath{clip}%
\pgfsetbuttcap%
\pgfsetroundjoin%
\definecolor{currentfill}{rgb}{0.121569,0.466667,0.705882}%
\pgfsetfillcolor{currentfill}%
\pgfsetfillopacity{0.895092}%
\pgfsetlinewidth{1.003750pt}%
\definecolor{currentstroke}{rgb}{0.121569,0.466667,0.705882}%
\pgfsetstrokecolor{currentstroke}%
\pgfsetstrokeopacity{0.895092}%
\pgfsetdash{}{0pt}%
\pgfpathmoveto{\pgfqpoint{2.642883in}{1.940801in}}%
\pgfpathcurveto{\pgfqpoint{2.651119in}{1.940801in}}{\pgfqpoint{2.659019in}{1.944073in}}{\pgfqpoint{2.664843in}{1.949897in}}%
\pgfpathcurveto{\pgfqpoint{2.670667in}{1.955721in}}{\pgfqpoint{2.673939in}{1.963621in}}{\pgfqpoint{2.673939in}{1.971858in}}%
\pgfpathcurveto{\pgfqpoint{2.673939in}{1.980094in}}{\pgfqpoint{2.670667in}{1.987994in}}{\pgfqpoint{2.664843in}{1.993818in}}%
\pgfpathcurveto{\pgfqpoint{2.659019in}{1.999642in}}{\pgfqpoint{2.651119in}{2.002914in}}{\pgfqpoint{2.642883in}{2.002914in}}%
\pgfpathcurveto{\pgfqpoint{2.634647in}{2.002914in}}{\pgfqpoint{2.626746in}{1.999642in}}{\pgfqpoint{2.620923in}{1.993818in}}%
\pgfpathcurveto{\pgfqpoint{2.615099in}{1.987994in}}{\pgfqpoint{2.611826in}{1.980094in}}{\pgfqpoint{2.611826in}{1.971858in}}%
\pgfpathcurveto{\pgfqpoint{2.611826in}{1.963621in}}{\pgfqpoint{2.615099in}{1.955721in}}{\pgfqpoint{2.620923in}{1.949897in}}%
\pgfpathcurveto{\pgfqpoint{2.626746in}{1.944073in}}{\pgfqpoint{2.634647in}{1.940801in}}{\pgfqpoint{2.642883in}{1.940801in}}%
\pgfpathclose%
\pgfusepath{stroke,fill}%
\end{pgfscope}%
\begin{pgfscope}%
\pgfpathrectangle{\pgfqpoint{0.100000in}{0.212622in}}{\pgfqpoint{3.696000in}{3.696000in}}%
\pgfusepath{clip}%
\pgfsetbuttcap%
\pgfsetroundjoin%
\definecolor{currentfill}{rgb}{0.121569,0.466667,0.705882}%
\pgfsetfillcolor{currentfill}%
\pgfsetfillopacity{0.895277}%
\pgfsetlinewidth{1.003750pt}%
\definecolor{currentstroke}{rgb}{0.121569,0.466667,0.705882}%
\pgfsetstrokecolor{currentstroke}%
\pgfsetstrokeopacity{0.895277}%
\pgfsetdash{}{0pt}%
\pgfpathmoveto{\pgfqpoint{1.551380in}{2.217775in}}%
\pgfpathcurveto{\pgfqpoint{1.559616in}{2.217775in}}{\pgfqpoint{1.567516in}{2.221047in}}{\pgfqpoint{1.573340in}{2.226871in}}%
\pgfpathcurveto{\pgfqpoint{1.579164in}{2.232695in}}{\pgfqpoint{1.582436in}{2.240595in}}{\pgfqpoint{1.582436in}{2.248832in}}%
\pgfpathcurveto{\pgfqpoint{1.582436in}{2.257068in}}{\pgfqpoint{1.579164in}{2.264968in}}{\pgfqpoint{1.573340in}{2.270792in}}%
\pgfpathcurveto{\pgfqpoint{1.567516in}{2.276616in}}{\pgfqpoint{1.559616in}{2.279888in}}{\pgfqpoint{1.551380in}{2.279888in}}%
\pgfpathcurveto{\pgfqpoint{1.543143in}{2.279888in}}{\pgfqpoint{1.535243in}{2.276616in}}{\pgfqpoint{1.529419in}{2.270792in}}%
\pgfpathcurveto{\pgfqpoint{1.523595in}{2.264968in}}{\pgfqpoint{1.520323in}{2.257068in}}{\pgfqpoint{1.520323in}{2.248832in}}%
\pgfpathcurveto{\pgfqpoint{1.520323in}{2.240595in}}{\pgfqpoint{1.523595in}{2.232695in}}{\pgfqpoint{1.529419in}{2.226871in}}%
\pgfpathcurveto{\pgfqpoint{1.535243in}{2.221047in}}{\pgfqpoint{1.543143in}{2.217775in}}{\pgfqpoint{1.551380in}{2.217775in}}%
\pgfpathclose%
\pgfusepath{stroke,fill}%
\end{pgfscope}%
\begin{pgfscope}%
\pgfpathrectangle{\pgfqpoint{0.100000in}{0.212622in}}{\pgfqpoint{3.696000in}{3.696000in}}%
\pgfusepath{clip}%
\pgfsetbuttcap%
\pgfsetroundjoin%
\definecolor{currentfill}{rgb}{0.121569,0.466667,0.705882}%
\pgfsetfillcolor{currentfill}%
\pgfsetfillopacity{0.896032}%
\pgfsetlinewidth{1.003750pt}%
\definecolor{currentstroke}{rgb}{0.121569,0.466667,0.705882}%
\pgfsetstrokecolor{currentstroke}%
\pgfsetstrokeopacity{0.896032}%
\pgfsetdash{}{0pt}%
\pgfpathmoveto{\pgfqpoint{2.640882in}{1.938194in}}%
\pgfpathcurveto{\pgfqpoint{2.649119in}{1.938194in}}{\pgfqpoint{2.657019in}{1.941466in}}{\pgfqpoint{2.662842in}{1.947290in}}%
\pgfpathcurveto{\pgfqpoint{2.668666in}{1.953114in}}{\pgfqpoint{2.671939in}{1.961014in}}{\pgfqpoint{2.671939in}{1.969250in}}%
\pgfpathcurveto{\pgfqpoint{2.671939in}{1.977487in}}{\pgfqpoint{2.668666in}{1.985387in}}{\pgfqpoint{2.662842in}{1.991211in}}%
\pgfpathcurveto{\pgfqpoint{2.657019in}{1.997035in}}{\pgfqpoint{2.649119in}{2.000307in}}{\pgfqpoint{2.640882in}{2.000307in}}%
\pgfpathcurveto{\pgfqpoint{2.632646in}{2.000307in}}{\pgfqpoint{2.624746in}{1.997035in}}{\pgfqpoint{2.618922in}{1.991211in}}%
\pgfpathcurveto{\pgfqpoint{2.613098in}{1.985387in}}{\pgfqpoint{2.609826in}{1.977487in}}{\pgfqpoint{2.609826in}{1.969250in}}%
\pgfpathcurveto{\pgfqpoint{2.609826in}{1.961014in}}{\pgfqpoint{2.613098in}{1.953114in}}{\pgfqpoint{2.618922in}{1.947290in}}%
\pgfpathcurveto{\pgfqpoint{2.624746in}{1.941466in}}{\pgfqpoint{2.632646in}{1.938194in}}{\pgfqpoint{2.640882in}{1.938194in}}%
\pgfpathclose%
\pgfusepath{stroke,fill}%
\end{pgfscope}%
\begin{pgfscope}%
\pgfpathrectangle{\pgfqpoint{0.100000in}{0.212622in}}{\pgfqpoint{3.696000in}{3.696000in}}%
\pgfusepath{clip}%
\pgfsetbuttcap%
\pgfsetroundjoin%
\definecolor{currentfill}{rgb}{0.121569,0.466667,0.705882}%
\pgfsetfillcolor{currentfill}%
\pgfsetfillopacity{0.898276}%
\pgfsetlinewidth{1.003750pt}%
\definecolor{currentstroke}{rgb}{0.121569,0.466667,0.705882}%
\pgfsetstrokecolor{currentstroke}%
\pgfsetstrokeopacity{0.898276}%
\pgfsetdash{}{0pt}%
\pgfpathmoveto{\pgfqpoint{2.636345in}{1.937022in}}%
\pgfpathcurveto{\pgfqpoint{2.644581in}{1.937022in}}{\pgfqpoint{2.652482in}{1.940294in}}{\pgfqpoint{2.658305in}{1.946118in}}%
\pgfpathcurveto{\pgfqpoint{2.664129in}{1.951942in}}{\pgfqpoint{2.667402in}{1.959842in}}{\pgfqpoint{2.667402in}{1.968078in}}%
\pgfpathcurveto{\pgfqpoint{2.667402in}{1.976315in}}{\pgfqpoint{2.664129in}{1.984215in}}{\pgfqpoint{2.658305in}{1.990038in}}%
\pgfpathcurveto{\pgfqpoint{2.652482in}{1.995862in}}{\pgfqpoint{2.644581in}{1.999135in}}{\pgfqpoint{2.636345in}{1.999135in}}%
\pgfpathcurveto{\pgfqpoint{2.628109in}{1.999135in}}{\pgfqpoint{2.620209in}{1.995862in}}{\pgfqpoint{2.614385in}{1.990038in}}%
\pgfpathcurveto{\pgfqpoint{2.608561in}{1.984215in}}{\pgfqpoint{2.605289in}{1.976315in}}{\pgfqpoint{2.605289in}{1.968078in}}%
\pgfpathcurveto{\pgfqpoint{2.605289in}{1.959842in}}{\pgfqpoint{2.608561in}{1.951942in}}{\pgfqpoint{2.614385in}{1.946118in}}%
\pgfpathcurveto{\pgfqpoint{2.620209in}{1.940294in}}{\pgfqpoint{2.628109in}{1.937022in}}{\pgfqpoint{2.636345in}{1.937022in}}%
\pgfpathclose%
\pgfusepath{stroke,fill}%
\end{pgfscope}%
\begin{pgfscope}%
\pgfpathrectangle{\pgfqpoint{0.100000in}{0.212622in}}{\pgfqpoint{3.696000in}{3.696000in}}%
\pgfusepath{clip}%
\pgfsetbuttcap%
\pgfsetroundjoin%
\definecolor{currentfill}{rgb}{0.121569,0.466667,0.705882}%
\pgfsetfillcolor{currentfill}%
\pgfsetfillopacity{0.901110}%
\pgfsetlinewidth{1.003750pt}%
\definecolor{currentstroke}{rgb}{0.121569,0.466667,0.705882}%
\pgfsetstrokecolor{currentstroke}%
\pgfsetstrokeopacity{0.901110}%
\pgfsetdash{}{0pt}%
\pgfpathmoveto{\pgfqpoint{2.630133in}{1.933517in}}%
\pgfpathcurveto{\pgfqpoint{2.638369in}{1.933517in}}{\pgfqpoint{2.646269in}{1.936789in}}{\pgfqpoint{2.652093in}{1.942613in}}%
\pgfpathcurveto{\pgfqpoint{2.657917in}{1.948437in}}{\pgfqpoint{2.661189in}{1.956337in}}{\pgfqpoint{2.661189in}{1.964574in}}%
\pgfpathcurveto{\pgfqpoint{2.661189in}{1.972810in}}{\pgfqpoint{2.657917in}{1.980710in}}{\pgfqpoint{2.652093in}{1.986534in}}%
\pgfpathcurveto{\pgfqpoint{2.646269in}{1.992358in}}{\pgfqpoint{2.638369in}{1.995630in}}{\pgfqpoint{2.630133in}{1.995630in}}%
\pgfpathcurveto{\pgfqpoint{2.621896in}{1.995630in}}{\pgfqpoint{2.613996in}{1.992358in}}{\pgfqpoint{2.608173in}{1.986534in}}%
\pgfpathcurveto{\pgfqpoint{2.602349in}{1.980710in}}{\pgfqpoint{2.599076in}{1.972810in}}{\pgfqpoint{2.599076in}{1.964574in}}%
\pgfpathcurveto{\pgfqpoint{2.599076in}{1.956337in}}{\pgfqpoint{2.602349in}{1.948437in}}{\pgfqpoint{2.608173in}{1.942613in}}%
\pgfpathcurveto{\pgfqpoint{2.613996in}{1.936789in}}{\pgfqpoint{2.621896in}{1.933517in}}{\pgfqpoint{2.630133in}{1.933517in}}%
\pgfpathclose%
\pgfusepath{stroke,fill}%
\end{pgfscope}%
\begin{pgfscope}%
\pgfpathrectangle{\pgfqpoint{0.100000in}{0.212622in}}{\pgfqpoint{3.696000in}{3.696000in}}%
\pgfusepath{clip}%
\pgfsetbuttcap%
\pgfsetroundjoin%
\definecolor{currentfill}{rgb}{0.121569,0.466667,0.705882}%
\pgfsetfillcolor{currentfill}%
\pgfsetfillopacity{0.903068}%
\pgfsetlinewidth{1.003750pt}%
\definecolor{currentstroke}{rgb}{0.121569,0.466667,0.705882}%
\pgfsetstrokecolor{currentstroke}%
\pgfsetstrokeopacity{0.903068}%
\pgfsetdash{}{0pt}%
\pgfpathmoveto{\pgfqpoint{1.592452in}{2.205786in}}%
\pgfpathcurveto{\pgfqpoint{1.600689in}{2.205786in}}{\pgfqpoint{1.608589in}{2.209058in}}{\pgfqpoint{1.614413in}{2.214882in}}%
\pgfpathcurveto{\pgfqpoint{1.620237in}{2.220706in}}{\pgfqpoint{1.623509in}{2.228606in}}{\pgfqpoint{1.623509in}{2.236842in}}%
\pgfpathcurveto{\pgfqpoint{1.623509in}{2.245079in}}{\pgfqpoint{1.620237in}{2.252979in}}{\pgfqpoint{1.614413in}{2.258803in}}%
\pgfpathcurveto{\pgfqpoint{1.608589in}{2.264626in}}{\pgfqpoint{1.600689in}{2.267899in}}{\pgfqpoint{1.592452in}{2.267899in}}%
\pgfpathcurveto{\pgfqpoint{1.584216in}{2.267899in}}{\pgfqpoint{1.576316in}{2.264626in}}{\pgfqpoint{1.570492in}{2.258803in}}%
\pgfpathcurveto{\pgfqpoint{1.564668in}{2.252979in}}{\pgfqpoint{1.561396in}{2.245079in}}{\pgfqpoint{1.561396in}{2.236842in}}%
\pgfpathcurveto{\pgfqpoint{1.561396in}{2.228606in}}{\pgfqpoint{1.564668in}{2.220706in}}{\pgfqpoint{1.570492in}{2.214882in}}%
\pgfpathcurveto{\pgfqpoint{1.576316in}{2.209058in}}{\pgfqpoint{1.584216in}{2.205786in}}{\pgfqpoint{1.592452in}{2.205786in}}%
\pgfpathclose%
\pgfusepath{stroke,fill}%
\end{pgfscope}%
\begin{pgfscope}%
\pgfpathrectangle{\pgfqpoint{0.100000in}{0.212622in}}{\pgfqpoint{3.696000in}{3.696000in}}%
\pgfusepath{clip}%
\pgfsetbuttcap%
\pgfsetroundjoin%
\definecolor{currentfill}{rgb}{0.121569,0.466667,0.705882}%
\pgfsetfillcolor{currentfill}%
\pgfsetfillopacity{0.904942}%
\pgfsetlinewidth{1.003750pt}%
\definecolor{currentstroke}{rgb}{0.121569,0.466667,0.705882}%
\pgfsetstrokecolor{currentstroke}%
\pgfsetstrokeopacity{0.904942}%
\pgfsetdash{}{0pt}%
\pgfpathmoveto{\pgfqpoint{2.620329in}{1.930620in}}%
\pgfpathcurveto{\pgfqpoint{2.628565in}{1.930620in}}{\pgfqpoint{2.636465in}{1.933892in}}{\pgfqpoint{2.642289in}{1.939716in}}%
\pgfpathcurveto{\pgfqpoint{2.648113in}{1.945540in}}{\pgfqpoint{2.651385in}{1.953440in}}{\pgfqpoint{2.651385in}{1.961676in}}%
\pgfpathcurveto{\pgfqpoint{2.651385in}{1.969913in}}{\pgfqpoint{2.648113in}{1.977813in}}{\pgfqpoint{2.642289in}{1.983637in}}%
\pgfpathcurveto{\pgfqpoint{2.636465in}{1.989461in}}{\pgfqpoint{2.628565in}{1.992733in}}{\pgfqpoint{2.620329in}{1.992733in}}%
\pgfpathcurveto{\pgfqpoint{2.612093in}{1.992733in}}{\pgfqpoint{2.604193in}{1.989461in}}{\pgfqpoint{2.598369in}{1.983637in}}%
\pgfpathcurveto{\pgfqpoint{2.592545in}{1.977813in}}{\pgfqpoint{2.589272in}{1.969913in}}{\pgfqpoint{2.589272in}{1.961676in}}%
\pgfpathcurveto{\pgfqpoint{2.589272in}{1.953440in}}{\pgfqpoint{2.592545in}{1.945540in}}{\pgfqpoint{2.598369in}{1.939716in}}%
\pgfpathcurveto{\pgfqpoint{2.604193in}{1.933892in}}{\pgfqpoint{2.612093in}{1.930620in}}{\pgfqpoint{2.620329in}{1.930620in}}%
\pgfpathclose%
\pgfusepath{stroke,fill}%
\end{pgfscope}%
\begin{pgfscope}%
\pgfpathrectangle{\pgfqpoint{0.100000in}{0.212622in}}{\pgfqpoint{3.696000in}{3.696000in}}%
\pgfusepath{clip}%
\pgfsetbuttcap%
\pgfsetroundjoin%
\definecolor{currentfill}{rgb}{0.121569,0.466667,0.705882}%
\pgfsetfillcolor{currentfill}%
\pgfsetfillopacity{0.906069}%
\pgfsetlinewidth{1.003750pt}%
\definecolor{currentstroke}{rgb}{0.121569,0.466667,0.705882}%
\pgfsetstrokecolor{currentstroke}%
\pgfsetstrokeopacity{0.906069}%
\pgfsetdash{}{0pt}%
\pgfpathmoveto{\pgfqpoint{1.633678in}{2.182266in}}%
\pgfpathcurveto{\pgfqpoint{1.641915in}{2.182266in}}{\pgfqpoint{1.649815in}{2.185539in}}{\pgfqpoint{1.655639in}{2.191363in}}%
\pgfpathcurveto{\pgfqpoint{1.661462in}{2.197186in}}{\pgfqpoint{1.664735in}{2.205086in}}{\pgfqpoint{1.664735in}{2.213323in}}%
\pgfpathcurveto{\pgfqpoint{1.664735in}{2.221559in}}{\pgfqpoint{1.661462in}{2.229459in}}{\pgfqpoint{1.655639in}{2.235283in}}%
\pgfpathcurveto{\pgfqpoint{1.649815in}{2.241107in}}{\pgfqpoint{1.641915in}{2.244379in}}{\pgfqpoint{1.633678in}{2.244379in}}%
\pgfpathcurveto{\pgfqpoint{1.625442in}{2.244379in}}{\pgfqpoint{1.617542in}{2.241107in}}{\pgfqpoint{1.611718in}{2.235283in}}%
\pgfpathcurveto{\pgfqpoint{1.605894in}{2.229459in}}{\pgfqpoint{1.602622in}{2.221559in}}{\pgfqpoint{1.602622in}{2.213323in}}%
\pgfpathcurveto{\pgfqpoint{1.602622in}{2.205086in}}{\pgfqpoint{1.605894in}{2.197186in}}{\pgfqpoint{1.611718in}{2.191363in}}%
\pgfpathcurveto{\pgfqpoint{1.617542in}{2.185539in}}{\pgfqpoint{1.625442in}{2.182266in}}{\pgfqpoint{1.633678in}{2.182266in}}%
\pgfpathclose%
\pgfusepath{stroke,fill}%
\end{pgfscope}%
\begin{pgfscope}%
\pgfpathrectangle{\pgfqpoint{0.100000in}{0.212622in}}{\pgfqpoint{3.696000in}{3.696000in}}%
\pgfusepath{clip}%
\pgfsetbuttcap%
\pgfsetroundjoin%
\definecolor{currentfill}{rgb}{0.121569,0.466667,0.705882}%
\pgfsetfillcolor{currentfill}%
\pgfsetfillopacity{0.909394}%
\pgfsetlinewidth{1.003750pt}%
\definecolor{currentstroke}{rgb}{0.121569,0.466667,0.705882}%
\pgfsetstrokecolor{currentstroke}%
\pgfsetstrokeopacity{0.909394}%
\pgfsetdash{}{0pt}%
\pgfpathmoveto{\pgfqpoint{2.609915in}{1.925060in}}%
\pgfpathcurveto{\pgfqpoint{2.618151in}{1.925060in}}{\pgfqpoint{2.626051in}{1.928332in}}{\pgfqpoint{2.631875in}{1.934156in}}%
\pgfpathcurveto{\pgfqpoint{2.637699in}{1.939980in}}{\pgfqpoint{2.640971in}{1.947880in}}{\pgfqpoint{2.640971in}{1.956116in}}%
\pgfpathcurveto{\pgfqpoint{2.640971in}{1.964353in}}{\pgfqpoint{2.637699in}{1.972253in}}{\pgfqpoint{2.631875in}{1.978077in}}%
\pgfpathcurveto{\pgfqpoint{2.626051in}{1.983900in}}{\pgfqpoint{2.618151in}{1.987173in}}{\pgfqpoint{2.609915in}{1.987173in}}%
\pgfpathcurveto{\pgfqpoint{2.601678in}{1.987173in}}{\pgfqpoint{2.593778in}{1.983900in}}{\pgfqpoint{2.587954in}{1.978077in}}%
\pgfpathcurveto{\pgfqpoint{2.582130in}{1.972253in}}{\pgfqpoint{2.578858in}{1.964353in}}{\pgfqpoint{2.578858in}{1.956116in}}%
\pgfpathcurveto{\pgfqpoint{2.578858in}{1.947880in}}{\pgfqpoint{2.582130in}{1.939980in}}{\pgfqpoint{2.587954in}{1.934156in}}%
\pgfpathcurveto{\pgfqpoint{2.593778in}{1.928332in}}{\pgfqpoint{2.601678in}{1.925060in}}{\pgfqpoint{2.609915in}{1.925060in}}%
\pgfpathclose%
\pgfusepath{stroke,fill}%
\end{pgfscope}%
\begin{pgfscope}%
\pgfpathrectangle{\pgfqpoint{0.100000in}{0.212622in}}{\pgfqpoint{3.696000in}{3.696000in}}%
\pgfusepath{clip}%
\pgfsetbuttcap%
\pgfsetroundjoin%
\definecolor{currentfill}{rgb}{0.121569,0.466667,0.705882}%
\pgfsetfillcolor{currentfill}%
\pgfsetfillopacity{0.911548}%
\pgfsetlinewidth{1.003750pt}%
\definecolor{currentstroke}{rgb}{0.121569,0.466667,0.705882}%
\pgfsetstrokecolor{currentstroke}%
\pgfsetstrokeopacity{0.911548}%
\pgfsetdash{}{0pt}%
\pgfpathmoveto{\pgfqpoint{1.666674in}{2.167893in}}%
\pgfpathcurveto{\pgfqpoint{1.674911in}{2.167893in}}{\pgfqpoint{1.682811in}{2.171166in}}{\pgfqpoint{1.688635in}{2.176990in}}%
\pgfpathcurveto{\pgfqpoint{1.694459in}{2.182814in}}{\pgfqpoint{1.697731in}{2.190714in}}{\pgfqpoint{1.697731in}{2.198950in}}%
\pgfpathcurveto{\pgfqpoint{1.697731in}{2.207186in}}{\pgfqpoint{1.694459in}{2.215086in}}{\pgfqpoint{1.688635in}{2.220910in}}%
\pgfpathcurveto{\pgfqpoint{1.682811in}{2.226734in}}{\pgfqpoint{1.674911in}{2.230006in}}{\pgfqpoint{1.666674in}{2.230006in}}%
\pgfpathcurveto{\pgfqpoint{1.658438in}{2.230006in}}{\pgfqpoint{1.650538in}{2.226734in}}{\pgfqpoint{1.644714in}{2.220910in}}%
\pgfpathcurveto{\pgfqpoint{1.638890in}{2.215086in}}{\pgfqpoint{1.635618in}{2.207186in}}{\pgfqpoint{1.635618in}{2.198950in}}%
\pgfpathcurveto{\pgfqpoint{1.635618in}{2.190714in}}{\pgfqpoint{1.638890in}{2.182814in}}{\pgfqpoint{1.644714in}{2.176990in}}%
\pgfpathcurveto{\pgfqpoint{1.650538in}{2.171166in}}{\pgfqpoint{1.658438in}{2.167893in}}{\pgfqpoint{1.666674in}{2.167893in}}%
\pgfpathclose%
\pgfusepath{stroke,fill}%
\end{pgfscope}%
\begin{pgfscope}%
\pgfpathrectangle{\pgfqpoint{0.100000in}{0.212622in}}{\pgfqpoint{3.696000in}{3.696000in}}%
\pgfusepath{clip}%
\pgfsetbuttcap%
\pgfsetroundjoin%
\definecolor{currentfill}{rgb}{0.121569,0.466667,0.705882}%
\pgfsetfillcolor{currentfill}%
\pgfsetfillopacity{0.914046}%
\pgfsetlinewidth{1.003750pt}%
\definecolor{currentstroke}{rgb}{0.121569,0.466667,0.705882}%
\pgfsetstrokecolor{currentstroke}%
\pgfsetstrokeopacity{0.914046}%
\pgfsetdash{}{0pt}%
\pgfpathmoveto{\pgfqpoint{1.696836in}{2.147790in}}%
\pgfpathcurveto{\pgfqpoint{1.705072in}{2.147790in}}{\pgfqpoint{1.712972in}{2.151062in}}{\pgfqpoint{1.718796in}{2.156886in}}%
\pgfpathcurveto{\pgfqpoint{1.724620in}{2.162710in}}{\pgfqpoint{1.727892in}{2.170610in}}{\pgfqpoint{1.727892in}{2.178847in}}%
\pgfpathcurveto{\pgfqpoint{1.727892in}{2.187083in}}{\pgfqpoint{1.724620in}{2.194983in}}{\pgfqpoint{1.718796in}{2.200807in}}%
\pgfpathcurveto{\pgfqpoint{1.712972in}{2.206631in}}{\pgfqpoint{1.705072in}{2.209903in}}{\pgfqpoint{1.696836in}{2.209903in}}%
\pgfpathcurveto{\pgfqpoint{1.688599in}{2.209903in}}{\pgfqpoint{1.680699in}{2.206631in}}{\pgfqpoint{1.674875in}{2.200807in}}%
\pgfpathcurveto{\pgfqpoint{1.669052in}{2.194983in}}{\pgfqpoint{1.665779in}{2.187083in}}{\pgfqpoint{1.665779in}{2.178847in}}%
\pgfpathcurveto{\pgfqpoint{1.665779in}{2.170610in}}{\pgfqpoint{1.669052in}{2.162710in}}{\pgfqpoint{1.674875in}{2.156886in}}%
\pgfpathcurveto{\pgfqpoint{1.680699in}{2.151062in}}{\pgfqpoint{1.688599in}{2.147790in}}{\pgfqpoint{1.696836in}{2.147790in}}%
\pgfpathclose%
\pgfusepath{stroke,fill}%
\end{pgfscope}%
\begin{pgfscope}%
\pgfpathrectangle{\pgfqpoint{0.100000in}{0.212622in}}{\pgfqpoint{3.696000in}{3.696000in}}%
\pgfusepath{clip}%
\pgfsetbuttcap%
\pgfsetroundjoin%
\definecolor{currentfill}{rgb}{0.121569,0.466667,0.705882}%
\pgfsetfillcolor{currentfill}%
\pgfsetfillopacity{0.914136}%
\pgfsetlinewidth{1.003750pt}%
\definecolor{currentstroke}{rgb}{0.121569,0.466667,0.705882}%
\pgfsetstrokecolor{currentstroke}%
\pgfsetstrokeopacity{0.914136}%
\pgfsetdash{}{0pt}%
\pgfpathmoveto{\pgfqpoint{2.596745in}{1.919081in}}%
\pgfpathcurveto{\pgfqpoint{2.604981in}{1.919081in}}{\pgfqpoint{2.612882in}{1.922354in}}{\pgfqpoint{2.618705in}{1.928178in}}%
\pgfpathcurveto{\pgfqpoint{2.624529in}{1.934001in}}{\pgfqpoint{2.627802in}{1.941901in}}{\pgfqpoint{2.627802in}{1.950138in}}%
\pgfpathcurveto{\pgfqpoint{2.627802in}{1.958374in}}{\pgfqpoint{2.624529in}{1.966274in}}{\pgfqpoint{2.618705in}{1.972098in}}%
\pgfpathcurveto{\pgfqpoint{2.612882in}{1.977922in}}{\pgfqpoint{2.604981in}{1.981194in}}{\pgfqpoint{2.596745in}{1.981194in}}%
\pgfpathcurveto{\pgfqpoint{2.588509in}{1.981194in}}{\pgfqpoint{2.580609in}{1.977922in}}{\pgfqpoint{2.574785in}{1.972098in}}%
\pgfpathcurveto{\pgfqpoint{2.568961in}{1.966274in}}{\pgfqpoint{2.565689in}{1.958374in}}{\pgfqpoint{2.565689in}{1.950138in}}%
\pgfpathcurveto{\pgfqpoint{2.565689in}{1.941901in}}{\pgfqpoint{2.568961in}{1.934001in}}{\pgfqpoint{2.574785in}{1.928178in}}%
\pgfpathcurveto{\pgfqpoint{2.580609in}{1.922354in}}{\pgfqpoint{2.588509in}{1.919081in}}{\pgfqpoint{2.596745in}{1.919081in}}%
\pgfpathclose%
\pgfusepath{stroke,fill}%
\end{pgfscope}%
\begin{pgfscope}%
\pgfpathrectangle{\pgfqpoint{0.100000in}{0.212622in}}{\pgfqpoint{3.696000in}{3.696000in}}%
\pgfusepath{clip}%
\pgfsetbuttcap%
\pgfsetroundjoin%
\definecolor{currentfill}{rgb}{0.121569,0.466667,0.705882}%
\pgfsetfillcolor{currentfill}%
\pgfsetfillopacity{0.916729}%
\pgfsetlinewidth{1.003750pt}%
\definecolor{currentstroke}{rgb}{0.121569,0.466667,0.705882}%
\pgfsetstrokecolor{currentstroke}%
\pgfsetstrokeopacity{0.916729}%
\pgfsetdash{}{0pt}%
\pgfpathmoveto{\pgfqpoint{1.720057in}{2.136545in}}%
\pgfpathcurveto{\pgfqpoint{1.728293in}{2.136545in}}{\pgfqpoint{1.736193in}{2.139817in}}{\pgfqpoint{1.742017in}{2.145641in}}%
\pgfpathcurveto{\pgfqpoint{1.747841in}{2.151465in}}{\pgfqpoint{1.751113in}{2.159365in}}{\pgfqpoint{1.751113in}{2.167601in}}%
\pgfpathcurveto{\pgfqpoint{1.751113in}{2.175838in}}{\pgfqpoint{1.747841in}{2.183738in}}{\pgfqpoint{1.742017in}{2.189562in}}%
\pgfpathcurveto{\pgfqpoint{1.736193in}{2.195386in}}{\pgfqpoint{1.728293in}{2.198658in}}{\pgfqpoint{1.720057in}{2.198658in}}%
\pgfpathcurveto{\pgfqpoint{1.711820in}{2.198658in}}{\pgfqpoint{1.703920in}{2.195386in}}{\pgfqpoint{1.698096in}{2.189562in}}%
\pgfpathcurveto{\pgfqpoint{1.692272in}{2.183738in}}{\pgfqpoint{1.689000in}{2.175838in}}{\pgfqpoint{1.689000in}{2.167601in}}%
\pgfpathcurveto{\pgfqpoint{1.689000in}{2.159365in}}{\pgfqpoint{1.692272in}{2.151465in}}{\pgfqpoint{1.698096in}{2.145641in}}%
\pgfpathcurveto{\pgfqpoint{1.703920in}{2.139817in}}{\pgfqpoint{1.711820in}{2.136545in}}{\pgfqpoint{1.720057in}{2.136545in}}%
\pgfpathclose%
\pgfusepath{stroke,fill}%
\end{pgfscope}%
\begin{pgfscope}%
\pgfpathrectangle{\pgfqpoint{0.100000in}{0.212622in}}{\pgfqpoint{3.696000in}{3.696000in}}%
\pgfusepath{clip}%
\pgfsetbuttcap%
\pgfsetroundjoin%
\definecolor{currentfill}{rgb}{0.121569,0.466667,0.705882}%
\pgfsetfillcolor{currentfill}%
\pgfsetfillopacity{0.917251}%
\pgfsetlinewidth{1.003750pt}%
\definecolor{currentstroke}{rgb}{0.121569,0.466667,0.705882}%
\pgfsetstrokecolor{currentstroke}%
\pgfsetstrokeopacity{0.917251}%
\pgfsetdash{}{0pt}%
\pgfpathmoveto{\pgfqpoint{2.590690in}{1.917327in}}%
\pgfpathcurveto{\pgfqpoint{2.598926in}{1.917327in}}{\pgfqpoint{2.606827in}{1.920599in}}{\pgfqpoint{2.612650in}{1.926423in}}%
\pgfpathcurveto{\pgfqpoint{2.618474in}{1.932247in}}{\pgfqpoint{2.621747in}{1.940147in}}{\pgfqpoint{2.621747in}{1.948383in}}%
\pgfpathcurveto{\pgfqpoint{2.621747in}{1.956620in}}{\pgfqpoint{2.618474in}{1.964520in}}{\pgfqpoint{2.612650in}{1.970344in}}%
\pgfpathcurveto{\pgfqpoint{2.606827in}{1.976168in}}{\pgfqpoint{2.598926in}{1.979440in}}{\pgfqpoint{2.590690in}{1.979440in}}%
\pgfpathcurveto{\pgfqpoint{2.582454in}{1.979440in}}{\pgfqpoint{2.574554in}{1.976168in}}{\pgfqpoint{2.568730in}{1.970344in}}%
\pgfpathcurveto{\pgfqpoint{2.562906in}{1.964520in}}{\pgfqpoint{2.559634in}{1.956620in}}{\pgfqpoint{2.559634in}{1.948383in}}%
\pgfpathcurveto{\pgfqpoint{2.559634in}{1.940147in}}{\pgfqpoint{2.562906in}{1.932247in}}{\pgfqpoint{2.568730in}{1.926423in}}%
\pgfpathcurveto{\pgfqpoint{2.574554in}{1.920599in}}{\pgfqpoint{2.582454in}{1.917327in}}{\pgfqpoint{2.590690in}{1.917327in}}%
\pgfpathclose%
\pgfusepath{stroke,fill}%
\end{pgfscope}%
\begin{pgfscope}%
\pgfpathrectangle{\pgfqpoint{0.100000in}{0.212622in}}{\pgfqpoint{3.696000in}{3.696000in}}%
\pgfusepath{clip}%
\pgfsetbuttcap%
\pgfsetroundjoin%
\definecolor{currentfill}{rgb}{0.121569,0.466667,0.705882}%
\pgfsetfillcolor{currentfill}%
\pgfsetfillopacity{0.918635}%
\pgfsetlinewidth{1.003750pt}%
\definecolor{currentstroke}{rgb}{0.121569,0.466667,0.705882}%
\pgfsetstrokecolor{currentstroke}%
\pgfsetstrokeopacity{0.918635}%
\pgfsetdash{}{0pt}%
\pgfpathmoveto{\pgfqpoint{2.586525in}{1.915404in}}%
\pgfpathcurveto{\pgfqpoint{2.594761in}{1.915404in}}{\pgfqpoint{2.602661in}{1.918676in}}{\pgfqpoint{2.608485in}{1.924500in}}%
\pgfpathcurveto{\pgfqpoint{2.614309in}{1.930324in}}{\pgfqpoint{2.617581in}{1.938224in}}{\pgfqpoint{2.617581in}{1.946460in}}%
\pgfpathcurveto{\pgfqpoint{2.617581in}{1.954697in}}{\pgfqpoint{2.614309in}{1.962597in}}{\pgfqpoint{2.608485in}{1.968421in}}%
\pgfpathcurveto{\pgfqpoint{2.602661in}{1.974245in}}{\pgfqpoint{2.594761in}{1.977517in}}{\pgfqpoint{2.586525in}{1.977517in}}%
\pgfpathcurveto{\pgfqpoint{2.578289in}{1.977517in}}{\pgfqpoint{2.570388in}{1.974245in}}{\pgfqpoint{2.564565in}{1.968421in}}%
\pgfpathcurveto{\pgfqpoint{2.558741in}{1.962597in}}{\pgfqpoint{2.555468in}{1.954697in}}{\pgfqpoint{2.555468in}{1.946460in}}%
\pgfpathcurveto{\pgfqpoint{2.555468in}{1.938224in}}{\pgfqpoint{2.558741in}{1.930324in}}{\pgfqpoint{2.564565in}{1.924500in}}%
\pgfpathcurveto{\pgfqpoint{2.570388in}{1.918676in}}{\pgfqpoint{2.578289in}{1.915404in}}{\pgfqpoint{2.586525in}{1.915404in}}%
\pgfpathclose%
\pgfusepath{stroke,fill}%
\end{pgfscope}%
\begin{pgfscope}%
\pgfpathrectangle{\pgfqpoint{0.100000in}{0.212622in}}{\pgfqpoint{3.696000in}{3.696000in}}%
\pgfusepath{clip}%
\pgfsetbuttcap%
\pgfsetroundjoin%
\definecolor{currentfill}{rgb}{0.121569,0.466667,0.705882}%
\pgfsetfillcolor{currentfill}%
\pgfsetfillopacity{0.918855}%
\pgfsetlinewidth{1.003750pt}%
\definecolor{currentstroke}{rgb}{0.121569,0.466667,0.705882}%
\pgfsetstrokecolor{currentstroke}%
\pgfsetstrokeopacity{0.918855}%
\pgfsetdash{}{0pt}%
\pgfpathmoveto{\pgfqpoint{1.740093in}{2.126322in}}%
\pgfpathcurveto{\pgfqpoint{1.748329in}{2.126322in}}{\pgfqpoint{1.756229in}{2.129594in}}{\pgfqpoint{1.762053in}{2.135418in}}%
\pgfpathcurveto{\pgfqpoint{1.767877in}{2.141242in}}{\pgfqpoint{1.771150in}{2.149142in}}{\pgfqpoint{1.771150in}{2.157378in}}%
\pgfpathcurveto{\pgfqpoint{1.771150in}{2.165614in}}{\pgfqpoint{1.767877in}{2.173514in}}{\pgfqpoint{1.762053in}{2.179338in}}%
\pgfpathcurveto{\pgfqpoint{1.756229in}{2.185162in}}{\pgfqpoint{1.748329in}{2.188435in}}{\pgfqpoint{1.740093in}{2.188435in}}%
\pgfpathcurveto{\pgfqpoint{1.731857in}{2.188435in}}{\pgfqpoint{1.723957in}{2.185162in}}{\pgfqpoint{1.718133in}{2.179338in}}%
\pgfpathcurveto{\pgfqpoint{1.712309in}{2.173514in}}{\pgfqpoint{1.709037in}{2.165614in}}{\pgfqpoint{1.709037in}{2.157378in}}%
\pgfpathcurveto{\pgfqpoint{1.709037in}{2.149142in}}{\pgfqpoint{1.712309in}{2.141242in}}{\pgfqpoint{1.718133in}{2.135418in}}%
\pgfpathcurveto{\pgfqpoint{1.723957in}{2.129594in}}{\pgfqpoint{1.731857in}{2.126322in}}{\pgfqpoint{1.740093in}{2.126322in}}%
\pgfpathclose%
\pgfusepath{stroke,fill}%
\end{pgfscope}%
\begin{pgfscope}%
\pgfpathrectangle{\pgfqpoint{0.100000in}{0.212622in}}{\pgfqpoint{3.696000in}{3.696000in}}%
\pgfusepath{clip}%
\pgfsetbuttcap%
\pgfsetroundjoin%
\definecolor{currentfill}{rgb}{0.121569,0.466667,0.705882}%
\pgfsetfillcolor{currentfill}%
\pgfsetfillopacity{0.919598}%
\pgfsetlinewidth{1.003750pt}%
\definecolor{currentstroke}{rgb}{0.121569,0.466667,0.705882}%
\pgfsetstrokecolor{currentstroke}%
\pgfsetstrokeopacity{0.919598}%
\pgfsetdash{}{0pt}%
\pgfpathmoveto{\pgfqpoint{2.584653in}{1.915038in}}%
\pgfpathcurveto{\pgfqpoint{2.592889in}{1.915038in}}{\pgfqpoint{2.600789in}{1.918310in}}{\pgfqpoint{2.606613in}{1.924134in}}%
\pgfpathcurveto{\pgfqpoint{2.612437in}{1.929958in}}{\pgfqpoint{2.615709in}{1.937858in}}{\pgfqpoint{2.615709in}{1.946094in}}%
\pgfpathcurveto{\pgfqpoint{2.615709in}{1.954331in}}{\pgfqpoint{2.612437in}{1.962231in}}{\pgfqpoint{2.606613in}{1.968055in}}%
\pgfpathcurveto{\pgfqpoint{2.600789in}{1.973879in}}{\pgfqpoint{2.592889in}{1.977151in}}{\pgfqpoint{2.584653in}{1.977151in}}%
\pgfpathcurveto{\pgfqpoint{2.576416in}{1.977151in}}{\pgfqpoint{2.568516in}{1.973879in}}{\pgfqpoint{2.562692in}{1.968055in}}%
\pgfpathcurveto{\pgfqpoint{2.556868in}{1.962231in}}{\pgfqpoint{2.553596in}{1.954331in}}{\pgfqpoint{2.553596in}{1.946094in}}%
\pgfpathcurveto{\pgfqpoint{2.553596in}{1.937858in}}{\pgfqpoint{2.556868in}{1.929958in}}{\pgfqpoint{2.562692in}{1.924134in}}%
\pgfpathcurveto{\pgfqpoint{2.568516in}{1.918310in}}{\pgfqpoint{2.576416in}{1.915038in}}{\pgfqpoint{2.584653in}{1.915038in}}%
\pgfpathclose%
\pgfusepath{stroke,fill}%
\end{pgfscope}%
\begin{pgfscope}%
\pgfpathrectangle{\pgfqpoint{0.100000in}{0.212622in}}{\pgfqpoint{3.696000in}{3.696000in}}%
\pgfusepath{clip}%
\pgfsetbuttcap%
\pgfsetroundjoin%
\definecolor{currentfill}{rgb}{0.121569,0.466667,0.705882}%
\pgfsetfillcolor{currentfill}%
\pgfsetfillopacity{0.920057}%
\pgfsetlinewidth{1.003750pt}%
\definecolor{currentstroke}{rgb}{0.121569,0.466667,0.705882}%
\pgfsetstrokecolor{currentstroke}%
\pgfsetstrokeopacity{0.920057}%
\pgfsetdash{}{0pt}%
\pgfpathmoveto{\pgfqpoint{2.583624in}{1.914376in}}%
\pgfpathcurveto{\pgfqpoint{2.591861in}{1.914376in}}{\pgfqpoint{2.599761in}{1.917648in}}{\pgfqpoint{2.605585in}{1.923472in}}%
\pgfpathcurveto{\pgfqpoint{2.611409in}{1.929296in}}{\pgfqpoint{2.614681in}{1.937196in}}{\pgfqpoint{2.614681in}{1.945433in}}%
\pgfpathcurveto{\pgfqpoint{2.614681in}{1.953669in}}{\pgfqpoint{2.611409in}{1.961569in}}{\pgfqpoint{2.605585in}{1.967393in}}%
\pgfpathcurveto{\pgfqpoint{2.599761in}{1.973217in}}{\pgfqpoint{2.591861in}{1.976489in}}{\pgfqpoint{2.583624in}{1.976489in}}%
\pgfpathcurveto{\pgfqpoint{2.575388in}{1.976489in}}{\pgfqpoint{2.567488in}{1.973217in}}{\pgfqpoint{2.561664in}{1.967393in}}%
\pgfpathcurveto{\pgfqpoint{2.555840in}{1.961569in}}{\pgfqpoint{2.552568in}{1.953669in}}{\pgfqpoint{2.552568in}{1.945433in}}%
\pgfpathcurveto{\pgfqpoint{2.552568in}{1.937196in}}{\pgfqpoint{2.555840in}{1.929296in}}{\pgfqpoint{2.561664in}{1.923472in}}%
\pgfpathcurveto{\pgfqpoint{2.567488in}{1.917648in}}{\pgfqpoint{2.575388in}{1.914376in}}{\pgfqpoint{2.583624in}{1.914376in}}%
\pgfpathclose%
\pgfusepath{stroke,fill}%
\end{pgfscope}%
\begin{pgfscope}%
\pgfpathrectangle{\pgfqpoint{0.100000in}{0.212622in}}{\pgfqpoint{3.696000in}{3.696000in}}%
\pgfusepath{clip}%
\pgfsetbuttcap%
\pgfsetroundjoin%
\definecolor{currentfill}{rgb}{0.121569,0.466667,0.705882}%
\pgfsetfillcolor{currentfill}%
\pgfsetfillopacity{0.920332}%
\pgfsetlinewidth{1.003750pt}%
\definecolor{currentstroke}{rgb}{0.121569,0.466667,0.705882}%
\pgfsetstrokecolor{currentstroke}%
\pgfsetstrokeopacity{0.920332}%
\pgfsetdash{}{0pt}%
\pgfpathmoveto{\pgfqpoint{2.583013in}{1.914221in}}%
\pgfpathcurveto{\pgfqpoint{2.591250in}{1.914221in}}{\pgfqpoint{2.599150in}{1.917493in}}{\pgfqpoint{2.604974in}{1.923317in}}%
\pgfpathcurveto{\pgfqpoint{2.610798in}{1.929141in}}{\pgfqpoint{2.614070in}{1.937041in}}{\pgfqpoint{2.614070in}{1.945277in}}%
\pgfpathcurveto{\pgfqpoint{2.614070in}{1.953513in}}{\pgfqpoint{2.610798in}{1.961413in}}{\pgfqpoint{2.604974in}{1.967237in}}%
\pgfpathcurveto{\pgfqpoint{2.599150in}{1.973061in}}{\pgfqpoint{2.591250in}{1.976334in}}{\pgfqpoint{2.583013in}{1.976334in}}%
\pgfpathcurveto{\pgfqpoint{2.574777in}{1.976334in}}{\pgfqpoint{2.566877in}{1.973061in}}{\pgfqpoint{2.561053in}{1.967237in}}%
\pgfpathcurveto{\pgfqpoint{2.555229in}{1.961413in}}{\pgfqpoint{2.551957in}{1.953513in}}{\pgfqpoint{2.551957in}{1.945277in}}%
\pgfpathcurveto{\pgfqpoint{2.551957in}{1.937041in}}{\pgfqpoint{2.555229in}{1.929141in}}{\pgfqpoint{2.561053in}{1.923317in}}%
\pgfpathcurveto{\pgfqpoint{2.566877in}{1.917493in}}{\pgfqpoint{2.574777in}{1.914221in}}{\pgfqpoint{2.583013in}{1.914221in}}%
\pgfpathclose%
\pgfusepath{stroke,fill}%
\end{pgfscope}%
\begin{pgfscope}%
\pgfpathrectangle{\pgfqpoint{0.100000in}{0.212622in}}{\pgfqpoint{3.696000in}{3.696000in}}%
\pgfusepath{clip}%
\pgfsetbuttcap%
\pgfsetroundjoin%
\definecolor{currentfill}{rgb}{0.121569,0.466667,0.705882}%
\pgfsetfillcolor{currentfill}%
\pgfsetfillopacity{0.921011}%
\pgfsetlinewidth{1.003750pt}%
\definecolor{currentstroke}{rgb}{0.121569,0.466667,0.705882}%
\pgfsetstrokecolor{currentstroke}%
\pgfsetstrokeopacity{0.921011}%
\pgfsetdash{}{0pt}%
\pgfpathmoveto{\pgfqpoint{1.755069in}{2.118341in}}%
\pgfpathcurveto{\pgfqpoint{1.763306in}{2.118341in}}{\pgfqpoint{1.771206in}{2.121613in}}{\pgfqpoint{1.777030in}{2.127437in}}%
\pgfpathcurveto{\pgfqpoint{1.782854in}{2.133261in}}{\pgfqpoint{1.786126in}{2.141161in}}{\pgfqpoint{1.786126in}{2.149398in}}%
\pgfpathcurveto{\pgfqpoint{1.786126in}{2.157634in}}{\pgfqpoint{1.782854in}{2.165534in}}{\pgfqpoint{1.777030in}{2.171358in}}%
\pgfpathcurveto{\pgfqpoint{1.771206in}{2.177182in}}{\pgfqpoint{1.763306in}{2.180454in}}{\pgfqpoint{1.755069in}{2.180454in}}%
\pgfpathcurveto{\pgfqpoint{1.746833in}{2.180454in}}{\pgfqpoint{1.738933in}{2.177182in}}{\pgfqpoint{1.733109in}{2.171358in}}%
\pgfpathcurveto{\pgfqpoint{1.727285in}{2.165534in}}{\pgfqpoint{1.724013in}{2.157634in}}{\pgfqpoint{1.724013in}{2.149398in}}%
\pgfpathcurveto{\pgfqpoint{1.724013in}{2.141161in}}{\pgfqpoint{1.727285in}{2.133261in}}{\pgfqpoint{1.733109in}{2.127437in}}%
\pgfpathcurveto{\pgfqpoint{1.738933in}{2.121613in}}{\pgfqpoint{1.746833in}{2.118341in}}{\pgfqpoint{1.755069in}{2.118341in}}%
\pgfpathclose%
\pgfusepath{stroke,fill}%
\end{pgfscope}%
\begin{pgfscope}%
\pgfpathrectangle{\pgfqpoint{0.100000in}{0.212622in}}{\pgfqpoint{3.696000in}{3.696000in}}%
\pgfusepath{clip}%
\pgfsetbuttcap%
\pgfsetroundjoin%
\definecolor{currentfill}{rgb}{0.121569,0.466667,0.705882}%
\pgfsetfillcolor{currentfill}%
\pgfsetfillopacity{0.921901}%
\pgfsetlinewidth{1.003750pt}%
\definecolor{currentstroke}{rgb}{0.121569,0.466667,0.705882}%
\pgfsetstrokecolor{currentstroke}%
\pgfsetstrokeopacity{0.921901}%
\pgfsetdash{}{0pt}%
\pgfpathmoveto{\pgfqpoint{2.579480in}{1.910696in}}%
\pgfpathcurveto{\pgfqpoint{2.587716in}{1.910696in}}{\pgfqpoint{2.595616in}{1.913968in}}{\pgfqpoint{2.601440in}{1.919792in}}%
\pgfpathcurveto{\pgfqpoint{2.607264in}{1.925616in}}{\pgfqpoint{2.610537in}{1.933516in}}{\pgfqpoint{2.610537in}{1.941753in}}%
\pgfpathcurveto{\pgfqpoint{2.610537in}{1.949989in}}{\pgfqpoint{2.607264in}{1.957889in}}{\pgfqpoint{2.601440in}{1.963713in}}%
\pgfpathcurveto{\pgfqpoint{2.595616in}{1.969537in}}{\pgfqpoint{2.587716in}{1.972809in}}{\pgfqpoint{2.579480in}{1.972809in}}%
\pgfpathcurveto{\pgfqpoint{2.571244in}{1.972809in}}{\pgfqpoint{2.563344in}{1.969537in}}{\pgfqpoint{2.557520in}{1.963713in}}%
\pgfpathcurveto{\pgfqpoint{2.551696in}{1.957889in}}{\pgfqpoint{2.548424in}{1.949989in}}{\pgfqpoint{2.548424in}{1.941753in}}%
\pgfpathcurveto{\pgfqpoint{2.548424in}{1.933516in}}{\pgfqpoint{2.551696in}{1.925616in}}{\pgfqpoint{2.557520in}{1.919792in}}%
\pgfpathcurveto{\pgfqpoint{2.563344in}{1.913968in}}{\pgfqpoint{2.571244in}{1.910696in}}{\pgfqpoint{2.579480in}{1.910696in}}%
\pgfpathclose%
\pgfusepath{stroke,fill}%
\end{pgfscope}%
\begin{pgfscope}%
\pgfpathrectangle{\pgfqpoint{0.100000in}{0.212622in}}{\pgfqpoint{3.696000in}{3.696000in}}%
\pgfusepath{clip}%
\pgfsetbuttcap%
\pgfsetroundjoin%
\definecolor{currentfill}{rgb}{0.121569,0.466667,0.705882}%
\pgfsetfillcolor{currentfill}%
\pgfsetfillopacity{0.922758}%
\pgfsetlinewidth{1.003750pt}%
\definecolor{currentstroke}{rgb}{0.121569,0.466667,0.705882}%
\pgfsetstrokecolor{currentstroke}%
\pgfsetstrokeopacity{0.922758}%
\pgfsetdash{}{0pt}%
\pgfpathmoveto{\pgfqpoint{1.767935in}{2.110341in}}%
\pgfpathcurveto{\pgfqpoint{1.776171in}{2.110341in}}{\pgfqpoint{1.784071in}{2.113613in}}{\pgfqpoint{1.789895in}{2.119437in}}%
\pgfpathcurveto{\pgfqpoint{1.795719in}{2.125261in}}{\pgfqpoint{1.798992in}{2.133161in}}{\pgfqpoint{1.798992in}{2.141397in}}%
\pgfpathcurveto{\pgfqpoint{1.798992in}{2.149633in}}{\pgfqpoint{1.795719in}{2.157534in}}{\pgfqpoint{1.789895in}{2.163357in}}%
\pgfpathcurveto{\pgfqpoint{1.784071in}{2.169181in}}{\pgfqpoint{1.776171in}{2.172454in}}{\pgfqpoint{1.767935in}{2.172454in}}%
\pgfpathcurveto{\pgfqpoint{1.759699in}{2.172454in}}{\pgfqpoint{1.751799in}{2.169181in}}{\pgfqpoint{1.745975in}{2.163357in}}%
\pgfpathcurveto{\pgfqpoint{1.740151in}{2.157534in}}{\pgfqpoint{1.736879in}{2.149633in}}{\pgfqpoint{1.736879in}{2.141397in}}%
\pgfpathcurveto{\pgfqpoint{1.736879in}{2.133161in}}{\pgfqpoint{1.740151in}{2.125261in}}{\pgfqpoint{1.745975in}{2.119437in}}%
\pgfpathcurveto{\pgfqpoint{1.751799in}{2.113613in}}{\pgfqpoint{1.759699in}{2.110341in}}{\pgfqpoint{1.767935in}{2.110341in}}%
\pgfpathclose%
\pgfusepath{stroke,fill}%
\end{pgfscope}%
\begin{pgfscope}%
\pgfpathrectangle{\pgfqpoint{0.100000in}{0.212622in}}{\pgfqpoint{3.696000in}{3.696000in}}%
\pgfusepath{clip}%
\pgfsetbuttcap%
\pgfsetroundjoin%
\definecolor{currentfill}{rgb}{0.121569,0.466667,0.705882}%
\pgfsetfillcolor{currentfill}%
\pgfsetfillopacity{0.923599}%
\pgfsetlinewidth{1.003750pt}%
\definecolor{currentstroke}{rgb}{0.121569,0.466667,0.705882}%
\pgfsetstrokecolor{currentstroke}%
\pgfsetstrokeopacity{0.923599}%
\pgfsetdash{}{0pt}%
\pgfpathmoveto{\pgfqpoint{1.780026in}{2.102811in}}%
\pgfpathcurveto{\pgfqpoint{1.788263in}{2.102811in}}{\pgfqpoint{1.796163in}{2.106083in}}{\pgfqpoint{1.801987in}{2.111907in}}%
\pgfpathcurveto{\pgfqpoint{1.807811in}{2.117731in}}{\pgfqpoint{1.811083in}{2.125631in}}{\pgfqpoint{1.811083in}{2.133867in}}%
\pgfpathcurveto{\pgfqpoint{1.811083in}{2.142104in}}{\pgfqpoint{1.807811in}{2.150004in}}{\pgfqpoint{1.801987in}{2.155828in}}%
\pgfpathcurveto{\pgfqpoint{1.796163in}{2.161652in}}{\pgfqpoint{1.788263in}{2.164924in}}{\pgfqpoint{1.780026in}{2.164924in}}%
\pgfpathcurveto{\pgfqpoint{1.771790in}{2.164924in}}{\pgfqpoint{1.763890in}{2.161652in}}{\pgfqpoint{1.758066in}{2.155828in}}%
\pgfpathcurveto{\pgfqpoint{1.752242in}{2.150004in}}{\pgfqpoint{1.748970in}{2.142104in}}{\pgfqpoint{1.748970in}{2.133867in}}%
\pgfpathcurveto{\pgfqpoint{1.748970in}{2.125631in}}{\pgfqpoint{1.752242in}{2.117731in}}{\pgfqpoint{1.758066in}{2.111907in}}%
\pgfpathcurveto{\pgfqpoint{1.763890in}{2.106083in}}{\pgfqpoint{1.771790in}{2.102811in}}{\pgfqpoint{1.780026in}{2.102811in}}%
\pgfpathclose%
\pgfusepath{stroke,fill}%
\end{pgfscope}%
\begin{pgfscope}%
\pgfpathrectangle{\pgfqpoint{0.100000in}{0.212622in}}{\pgfqpoint{3.696000in}{3.696000in}}%
\pgfusepath{clip}%
\pgfsetbuttcap%
\pgfsetroundjoin%
\definecolor{currentfill}{rgb}{0.121569,0.466667,0.705882}%
\pgfsetfillcolor{currentfill}%
\pgfsetfillopacity{0.924508}%
\pgfsetlinewidth{1.003750pt}%
\definecolor{currentstroke}{rgb}{0.121569,0.466667,0.705882}%
\pgfsetstrokecolor{currentstroke}%
\pgfsetstrokeopacity{0.924508}%
\pgfsetdash{}{0pt}%
\pgfpathmoveto{\pgfqpoint{2.573391in}{1.908973in}}%
\pgfpathcurveto{\pgfqpoint{2.581627in}{1.908973in}}{\pgfqpoint{2.589527in}{1.912245in}}{\pgfqpoint{2.595351in}{1.918069in}}%
\pgfpathcurveto{\pgfqpoint{2.601175in}{1.923893in}}{\pgfqpoint{2.604447in}{1.931793in}}{\pgfqpoint{2.604447in}{1.940029in}}%
\pgfpathcurveto{\pgfqpoint{2.604447in}{1.948265in}}{\pgfqpoint{2.601175in}{1.956165in}}{\pgfqpoint{2.595351in}{1.961989in}}%
\pgfpathcurveto{\pgfqpoint{2.589527in}{1.967813in}}{\pgfqpoint{2.581627in}{1.971086in}}{\pgfqpoint{2.573391in}{1.971086in}}%
\pgfpathcurveto{\pgfqpoint{2.565155in}{1.971086in}}{\pgfqpoint{2.557254in}{1.967813in}}{\pgfqpoint{2.551431in}{1.961989in}}%
\pgfpathcurveto{\pgfqpoint{2.545607in}{1.956165in}}{\pgfqpoint{2.542334in}{1.948265in}}{\pgfqpoint{2.542334in}{1.940029in}}%
\pgfpathcurveto{\pgfqpoint{2.542334in}{1.931793in}}{\pgfqpoint{2.545607in}{1.923893in}}{\pgfqpoint{2.551431in}{1.918069in}}%
\pgfpathcurveto{\pgfqpoint{2.557254in}{1.912245in}}{\pgfqpoint{2.565155in}{1.908973in}}{\pgfqpoint{2.573391in}{1.908973in}}%
\pgfpathclose%
\pgfusepath{stroke,fill}%
\end{pgfscope}%
\begin{pgfscope}%
\pgfpathrectangle{\pgfqpoint{0.100000in}{0.212622in}}{\pgfqpoint{3.696000in}{3.696000in}}%
\pgfusepath{clip}%
\pgfsetbuttcap%
\pgfsetroundjoin%
\definecolor{currentfill}{rgb}{0.121569,0.466667,0.705882}%
\pgfsetfillcolor{currentfill}%
\pgfsetfillopacity{0.925524}%
\pgfsetlinewidth{1.003750pt}%
\definecolor{currentstroke}{rgb}{0.121569,0.466667,0.705882}%
\pgfsetstrokecolor{currentstroke}%
\pgfsetstrokeopacity{0.925524}%
\pgfsetdash{}{0pt}%
\pgfpathmoveto{\pgfqpoint{1.788430in}{2.100987in}}%
\pgfpathcurveto{\pgfqpoint{1.796667in}{2.100987in}}{\pgfqpoint{1.804567in}{2.104259in}}{\pgfqpoint{1.810391in}{2.110083in}}%
\pgfpathcurveto{\pgfqpoint{1.816214in}{2.115907in}}{\pgfqpoint{1.819487in}{2.123807in}}{\pgfqpoint{1.819487in}{2.132043in}}%
\pgfpathcurveto{\pgfqpoint{1.819487in}{2.140279in}}{\pgfqpoint{1.816214in}{2.148179in}}{\pgfqpoint{1.810391in}{2.154003in}}%
\pgfpathcurveto{\pgfqpoint{1.804567in}{2.159827in}}{\pgfqpoint{1.796667in}{2.163100in}}{\pgfqpoint{1.788430in}{2.163100in}}%
\pgfpathcurveto{\pgfqpoint{1.780194in}{2.163100in}}{\pgfqpoint{1.772294in}{2.159827in}}{\pgfqpoint{1.766470in}{2.154003in}}%
\pgfpathcurveto{\pgfqpoint{1.760646in}{2.148179in}}{\pgfqpoint{1.757374in}{2.140279in}}{\pgfqpoint{1.757374in}{2.132043in}}%
\pgfpathcurveto{\pgfqpoint{1.757374in}{2.123807in}}{\pgfqpoint{1.760646in}{2.115907in}}{\pgfqpoint{1.766470in}{2.110083in}}%
\pgfpathcurveto{\pgfqpoint{1.772294in}{2.104259in}}{\pgfqpoint{1.780194in}{2.100987in}}{\pgfqpoint{1.788430in}{2.100987in}}%
\pgfpathclose%
\pgfusepath{stroke,fill}%
\end{pgfscope}%
\begin{pgfscope}%
\pgfpathrectangle{\pgfqpoint{0.100000in}{0.212622in}}{\pgfqpoint{3.696000in}{3.696000in}}%
\pgfusepath{clip}%
\pgfsetbuttcap%
\pgfsetroundjoin%
\definecolor{currentfill}{rgb}{0.121569,0.466667,0.705882}%
\pgfsetfillcolor{currentfill}%
\pgfsetfillopacity{0.925898}%
\pgfsetlinewidth{1.003750pt}%
\definecolor{currentstroke}{rgb}{0.121569,0.466667,0.705882}%
\pgfsetstrokecolor{currentstroke}%
\pgfsetstrokeopacity{0.925898}%
\pgfsetdash{}{0pt}%
\pgfpathmoveto{\pgfqpoint{1.806982in}{2.089156in}}%
\pgfpathcurveto{\pgfqpoint{1.815218in}{2.089156in}}{\pgfqpoint{1.823118in}{2.092428in}}{\pgfqpoint{1.828942in}{2.098252in}}%
\pgfpathcurveto{\pgfqpoint{1.834766in}{2.104076in}}{\pgfqpoint{1.838038in}{2.111976in}}{\pgfqpoint{1.838038in}{2.120212in}}%
\pgfpathcurveto{\pgfqpoint{1.838038in}{2.128448in}}{\pgfqpoint{1.834766in}{2.136348in}}{\pgfqpoint{1.828942in}{2.142172in}}%
\pgfpathcurveto{\pgfqpoint{1.823118in}{2.147996in}}{\pgfqpoint{1.815218in}{2.151269in}}{\pgfqpoint{1.806982in}{2.151269in}}%
\pgfpathcurveto{\pgfqpoint{1.798746in}{2.151269in}}{\pgfqpoint{1.790846in}{2.147996in}}{\pgfqpoint{1.785022in}{2.142172in}}%
\pgfpathcurveto{\pgfqpoint{1.779198in}{2.136348in}}{\pgfqpoint{1.775925in}{2.128448in}}{\pgfqpoint{1.775925in}{2.120212in}}%
\pgfpathcurveto{\pgfqpoint{1.775925in}{2.111976in}}{\pgfqpoint{1.779198in}{2.104076in}}{\pgfqpoint{1.785022in}{2.098252in}}%
\pgfpathcurveto{\pgfqpoint{1.790846in}{2.092428in}}{\pgfqpoint{1.798746in}{2.089156in}}{\pgfqpoint{1.806982in}{2.089156in}}%
\pgfpathclose%
\pgfusepath{stroke,fill}%
\end{pgfscope}%
\begin{pgfscope}%
\pgfpathrectangle{\pgfqpoint{0.100000in}{0.212622in}}{\pgfqpoint{3.696000in}{3.696000in}}%
\pgfusepath{clip}%
\pgfsetbuttcap%
\pgfsetroundjoin%
\definecolor{currentfill}{rgb}{0.121569,0.466667,0.705882}%
\pgfsetfillcolor{currentfill}%
\pgfsetfillopacity{0.927838}%
\pgfsetlinewidth{1.003750pt}%
\definecolor{currentstroke}{rgb}{0.121569,0.466667,0.705882}%
\pgfsetstrokecolor{currentstroke}%
\pgfsetstrokeopacity{0.927838}%
\pgfsetdash{}{0pt}%
\pgfpathmoveto{\pgfqpoint{2.566111in}{1.902356in}}%
\pgfpathcurveto{\pgfqpoint{2.574347in}{1.902356in}}{\pgfqpoint{2.582247in}{1.905628in}}{\pgfqpoint{2.588071in}{1.911452in}}%
\pgfpathcurveto{\pgfqpoint{2.593895in}{1.917276in}}{\pgfqpoint{2.597167in}{1.925176in}}{\pgfqpoint{2.597167in}{1.933413in}}%
\pgfpathcurveto{\pgfqpoint{2.597167in}{1.941649in}}{\pgfqpoint{2.593895in}{1.949549in}}{\pgfqpoint{2.588071in}{1.955373in}}%
\pgfpathcurveto{\pgfqpoint{2.582247in}{1.961197in}}{\pgfqpoint{2.574347in}{1.964469in}}{\pgfqpoint{2.566111in}{1.964469in}}%
\pgfpathcurveto{\pgfqpoint{2.557875in}{1.964469in}}{\pgfqpoint{2.549975in}{1.961197in}}{\pgfqpoint{2.544151in}{1.955373in}}%
\pgfpathcurveto{\pgfqpoint{2.538327in}{1.949549in}}{\pgfqpoint{2.535054in}{1.941649in}}{\pgfqpoint{2.535054in}{1.933413in}}%
\pgfpathcurveto{\pgfqpoint{2.535054in}{1.925176in}}{\pgfqpoint{2.538327in}{1.917276in}}{\pgfqpoint{2.544151in}{1.911452in}}%
\pgfpathcurveto{\pgfqpoint{2.549975in}{1.905628in}}{\pgfqpoint{2.557875in}{1.902356in}}{\pgfqpoint{2.566111in}{1.902356in}}%
\pgfpathclose%
\pgfusepath{stroke,fill}%
\end{pgfscope}%
\begin{pgfscope}%
\pgfpathrectangle{\pgfqpoint{0.100000in}{0.212622in}}{\pgfqpoint{3.696000in}{3.696000in}}%
\pgfusepath{clip}%
\pgfsetbuttcap%
\pgfsetroundjoin%
\definecolor{currentfill}{rgb}{0.121569,0.466667,0.705882}%
\pgfsetfillcolor{currentfill}%
\pgfsetfillopacity{0.929750}%
\pgfsetlinewidth{1.003750pt}%
\definecolor{currentstroke}{rgb}{0.121569,0.466667,0.705882}%
\pgfsetstrokecolor{currentstroke}%
\pgfsetstrokeopacity{0.929750}%
\pgfsetdash{}{0pt}%
\pgfpathmoveto{\pgfqpoint{2.561468in}{1.899956in}}%
\pgfpathcurveto{\pgfqpoint{2.569704in}{1.899956in}}{\pgfqpoint{2.577604in}{1.903228in}}{\pgfqpoint{2.583428in}{1.909052in}}%
\pgfpathcurveto{\pgfqpoint{2.589252in}{1.914876in}}{\pgfqpoint{2.592524in}{1.922776in}}{\pgfqpoint{2.592524in}{1.931013in}}%
\pgfpathcurveto{\pgfqpoint{2.592524in}{1.939249in}}{\pgfqpoint{2.589252in}{1.947149in}}{\pgfqpoint{2.583428in}{1.952973in}}%
\pgfpathcurveto{\pgfqpoint{2.577604in}{1.958797in}}{\pgfqpoint{2.569704in}{1.962069in}}{\pgfqpoint{2.561468in}{1.962069in}}%
\pgfpathcurveto{\pgfqpoint{2.553231in}{1.962069in}}{\pgfqpoint{2.545331in}{1.958797in}}{\pgfqpoint{2.539507in}{1.952973in}}%
\pgfpathcurveto{\pgfqpoint{2.533684in}{1.947149in}}{\pgfqpoint{2.530411in}{1.939249in}}{\pgfqpoint{2.530411in}{1.931013in}}%
\pgfpathcurveto{\pgfqpoint{2.530411in}{1.922776in}}{\pgfqpoint{2.533684in}{1.914876in}}{\pgfqpoint{2.539507in}{1.909052in}}%
\pgfpathcurveto{\pgfqpoint{2.545331in}{1.903228in}}{\pgfqpoint{2.553231in}{1.899956in}}{\pgfqpoint{2.561468in}{1.899956in}}%
\pgfpathclose%
\pgfusepath{stroke,fill}%
\end{pgfscope}%
\begin{pgfscope}%
\pgfpathrectangle{\pgfqpoint{0.100000in}{0.212622in}}{\pgfqpoint{3.696000in}{3.696000in}}%
\pgfusepath{clip}%
\pgfsetbuttcap%
\pgfsetroundjoin%
\definecolor{currentfill}{rgb}{0.121569,0.466667,0.705882}%
\pgfsetfillcolor{currentfill}%
\pgfsetfillopacity{0.930833}%
\pgfsetlinewidth{1.003750pt}%
\definecolor{currentstroke}{rgb}{0.121569,0.466667,0.705882}%
\pgfsetstrokecolor{currentstroke}%
\pgfsetstrokeopacity{0.930833}%
\pgfsetdash{}{0pt}%
\pgfpathmoveto{\pgfqpoint{2.558921in}{1.898864in}}%
\pgfpathcurveto{\pgfqpoint{2.567157in}{1.898864in}}{\pgfqpoint{2.575057in}{1.902136in}}{\pgfqpoint{2.580881in}{1.907960in}}%
\pgfpathcurveto{\pgfqpoint{2.586705in}{1.913784in}}{\pgfqpoint{2.589977in}{1.921684in}}{\pgfqpoint{2.589977in}{1.929920in}}%
\pgfpathcurveto{\pgfqpoint{2.589977in}{1.938156in}}{\pgfqpoint{2.586705in}{1.946056in}}{\pgfqpoint{2.580881in}{1.951880in}}%
\pgfpathcurveto{\pgfqpoint{2.575057in}{1.957704in}}{\pgfqpoint{2.567157in}{1.960977in}}{\pgfqpoint{2.558921in}{1.960977in}}%
\pgfpathcurveto{\pgfqpoint{2.550685in}{1.960977in}}{\pgfqpoint{2.542785in}{1.957704in}}{\pgfqpoint{2.536961in}{1.951880in}}%
\pgfpathcurveto{\pgfqpoint{2.531137in}{1.946056in}}{\pgfqpoint{2.527864in}{1.938156in}}{\pgfqpoint{2.527864in}{1.929920in}}%
\pgfpathcurveto{\pgfqpoint{2.527864in}{1.921684in}}{\pgfqpoint{2.531137in}{1.913784in}}{\pgfqpoint{2.536961in}{1.907960in}}%
\pgfpathcurveto{\pgfqpoint{2.542785in}{1.902136in}}{\pgfqpoint{2.550685in}{1.898864in}}{\pgfqpoint{2.558921in}{1.898864in}}%
\pgfpathclose%
\pgfusepath{stroke,fill}%
\end{pgfscope}%
\begin{pgfscope}%
\pgfpathrectangle{\pgfqpoint{0.100000in}{0.212622in}}{\pgfqpoint{3.696000in}{3.696000in}}%
\pgfusepath{clip}%
\pgfsetbuttcap%
\pgfsetroundjoin%
\definecolor{currentfill}{rgb}{0.121569,0.466667,0.705882}%
\pgfsetfillcolor{currentfill}%
\pgfsetfillopacity{0.931440}%
\pgfsetlinewidth{1.003750pt}%
\definecolor{currentstroke}{rgb}{0.121569,0.466667,0.705882}%
\pgfsetstrokecolor{currentstroke}%
\pgfsetstrokeopacity{0.931440}%
\pgfsetdash{}{0pt}%
\pgfpathmoveto{\pgfqpoint{2.557552in}{1.898270in}}%
\pgfpathcurveto{\pgfqpoint{2.565788in}{1.898270in}}{\pgfqpoint{2.573688in}{1.901542in}}{\pgfqpoint{2.579512in}{1.907366in}}%
\pgfpathcurveto{\pgfqpoint{2.585336in}{1.913190in}}{\pgfqpoint{2.588608in}{1.921090in}}{\pgfqpoint{2.588608in}{1.929327in}}%
\pgfpathcurveto{\pgfqpoint{2.588608in}{1.937563in}}{\pgfqpoint{2.585336in}{1.945463in}}{\pgfqpoint{2.579512in}{1.951287in}}%
\pgfpathcurveto{\pgfqpoint{2.573688in}{1.957111in}}{\pgfqpoint{2.565788in}{1.960383in}}{\pgfqpoint{2.557552in}{1.960383in}}%
\pgfpathcurveto{\pgfqpoint{2.549316in}{1.960383in}}{\pgfqpoint{2.541416in}{1.957111in}}{\pgfqpoint{2.535592in}{1.951287in}}%
\pgfpathcurveto{\pgfqpoint{2.529768in}{1.945463in}}{\pgfqpoint{2.526495in}{1.937563in}}{\pgfqpoint{2.526495in}{1.929327in}}%
\pgfpathcurveto{\pgfqpoint{2.526495in}{1.921090in}}{\pgfqpoint{2.529768in}{1.913190in}}{\pgfqpoint{2.535592in}{1.907366in}}%
\pgfpathcurveto{\pgfqpoint{2.541416in}{1.901542in}}{\pgfqpoint{2.549316in}{1.898270in}}{\pgfqpoint{2.557552in}{1.898270in}}%
\pgfpathclose%
\pgfusepath{stroke,fill}%
\end{pgfscope}%
\begin{pgfscope}%
\pgfpathrectangle{\pgfqpoint{0.100000in}{0.212622in}}{\pgfqpoint{3.696000in}{3.696000in}}%
\pgfusepath{clip}%
\pgfsetbuttcap%
\pgfsetroundjoin%
\definecolor{currentfill}{rgb}{0.121569,0.466667,0.705882}%
\pgfsetfillcolor{currentfill}%
\pgfsetfillopacity{0.931767}%
\pgfsetlinewidth{1.003750pt}%
\definecolor{currentstroke}{rgb}{0.121569,0.466667,0.705882}%
\pgfsetstrokecolor{currentstroke}%
\pgfsetstrokeopacity{0.931767}%
\pgfsetdash{}{0pt}%
\pgfpathmoveto{\pgfqpoint{2.556648in}{1.898145in}}%
\pgfpathcurveto{\pgfqpoint{2.564884in}{1.898145in}}{\pgfqpoint{2.572784in}{1.901418in}}{\pgfqpoint{2.578608in}{1.907242in}}%
\pgfpathcurveto{\pgfqpoint{2.584432in}{1.913066in}}{\pgfqpoint{2.587704in}{1.920966in}}{\pgfqpoint{2.587704in}{1.929202in}}%
\pgfpathcurveto{\pgfqpoint{2.587704in}{1.937438in}}{\pgfqpoint{2.584432in}{1.945338in}}{\pgfqpoint{2.578608in}{1.951162in}}%
\pgfpathcurveto{\pgfqpoint{2.572784in}{1.956986in}}{\pgfqpoint{2.564884in}{1.960258in}}{\pgfqpoint{2.556648in}{1.960258in}}%
\pgfpathcurveto{\pgfqpoint{2.548412in}{1.960258in}}{\pgfqpoint{2.540512in}{1.956986in}}{\pgfqpoint{2.534688in}{1.951162in}}%
\pgfpathcurveto{\pgfqpoint{2.528864in}{1.945338in}}{\pgfqpoint{2.525591in}{1.937438in}}{\pgfqpoint{2.525591in}{1.929202in}}%
\pgfpathcurveto{\pgfqpoint{2.525591in}{1.920966in}}{\pgfqpoint{2.528864in}{1.913066in}}{\pgfqpoint{2.534688in}{1.907242in}}%
\pgfpathcurveto{\pgfqpoint{2.540512in}{1.901418in}}{\pgfqpoint{2.548412in}{1.898145in}}{\pgfqpoint{2.556648in}{1.898145in}}%
\pgfpathclose%
\pgfusepath{stroke,fill}%
\end{pgfscope}%
\begin{pgfscope}%
\pgfpathrectangle{\pgfqpoint{0.100000in}{0.212622in}}{\pgfqpoint{3.696000in}{3.696000in}}%
\pgfusepath{clip}%
\pgfsetbuttcap%
\pgfsetroundjoin%
\definecolor{currentfill}{rgb}{0.121569,0.466667,0.705882}%
\pgfsetfillcolor{currentfill}%
\pgfsetfillopacity{0.931925}%
\pgfsetlinewidth{1.003750pt}%
\definecolor{currentstroke}{rgb}{0.121569,0.466667,0.705882}%
\pgfsetstrokecolor{currentstroke}%
\pgfsetstrokeopacity{0.931925}%
\pgfsetdash{}{0pt}%
\pgfpathmoveto{\pgfqpoint{2.556274in}{1.897753in}}%
\pgfpathcurveto{\pgfqpoint{2.564510in}{1.897753in}}{\pgfqpoint{2.572410in}{1.901025in}}{\pgfqpoint{2.578234in}{1.906849in}}%
\pgfpathcurveto{\pgfqpoint{2.584058in}{1.912673in}}{\pgfqpoint{2.587330in}{1.920573in}}{\pgfqpoint{2.587330in}{1.928810in}}%
\pgfpathcurveto{\pgfqpoint{2.587330in}{1.937046in}}{\pgfqpoint{2.584058in}{1.944946in}}{\pgfqpoint{2.578234in}{1.950770in}}%
\pgfpathcurveto{\pgfqpoint{2.572410in}{1.956594in}}{\pgfqpoint{2.564510in}{1.959866in}}{\pgfqpoint{2.556274in}{1.959866in}}%
\pgfpathcurveto{\pgfqpoint{2.548037in}{1.959866in}}{\pgfqpoint{2.540137in}{1.956594in}}{\pgfqpoint{2.534313in}{1.950770in}}%
\pgfpathcurveto{\pgfqpoint{2.528490in}{1.944946in}}{\pgfqpoint{2.525217in}{1.937046in}}{\pgfqpoint{2.525217in}{1.928810in}}%
\pgfpathcurveto{\pgfqpoint{2.525217in}{1.920573in}}{\pgfqpoint{2.528490in}{1.912673in}}{\pgfqpoint{2.534313in}{1.906849in}}%
\pgfpathcurveto{\pgfqpoint{2.540137in}{1.901025in}}{\pgfqpoint{2.548037in}{1.897753in}}{\pgfqpoint{2.556274in}{1.897753in}}%
\pgfpathclose%
\pgfusepath{stroke,fill}%
\end{pgfscope}%
\begin{pgfscope}%
\pgfpathrectangle{\pgfqpoint{0.100000in}{0.212622in}}{\pgfqpoint{3.696000in}{3.696000in}}%
\pgfusepath{clip}%
\pgfsetbuttcap%
\pgfsetroundjoin%
\definecolor{currentfill}{rgb}{0.121569,0.466667,0.705882}%
\pgfsetfillcolor{currentfill}%
\pgfsetfillopacity{0.932050}%
\pgfsetlinewidth{1.003750pt}%
\definecolor{currentstroke}{rgb}{0.121569,0.466667,0.705882}%
\pgfsetstrokecolor{currentstroke}%
\pgfsetstrokeopacity{0.932050}%
\pgfsetdash{}{0pt}%
\pgfpathmoveto{\pgfqpoint{2.556089in}{1.897756in}}%
\pgfpathcurveto{\pgfqpoint{2.564325in}{1.897756in}}{\pgfqpoint{2.572225in}{1.901028in}}{\pgfqpoint{2.578049in}{1.906852in}}%
\pgfpathcurveto{\pgfqpoint{2.583873in}{1.912676in}}{\pgfqpoint{2.587145in}{1.920576in}}{\pgfqpoint{2.587145in}{1.928813in}}%
\pgfpathcurveto{\pgfqpoint{2.587145in}{1.937049in}}{\pgfqpoint{2.583873in}{1.944949in}}{\pgfqpoint{2.578049in}{1.950773in}}%
\pgfpathcurveto{\pgfqpoint{2.572225in}{1.956597in}}{\pgfqpoint{2.564325in}{1.959869in}}{\pgfqpoint{2.556089in}{1.959869in}}%
\pgfpathcurveto{\pgfqpoint{2.547852in}{1.959869in}}{\pgfqpoint{2.539952in}{1.956597in}}{\pgfqpoint{2.534129in}{1.950773in}}%
\pgfpathcurveto{\pgfqpoint{2.528305in}{1.944949in}}{\pgfqpoint{2.525032in}{1.937049in}}{\pgfqpoint{2.525032in}{1.928813in}}%
\pgfpathcurveto{\pgfqpoint{2.525032in}{1.920576in}}{\pgfqpoint{2.528305in}{1.912676in}}{\pgfqpoint{2.534129in}{1.906852in}}%
\pgfpathcurveto{\pgfqpoint{2.539952in}{1.901028in}}{\pgfqpoint{2.547852in}{1.897756in}}{\pgfqpoint{2.556089in}{1.897756in}}%
\pgfpathclose%
\pgfusepath{stroke,fill}%
\end{pgfscope}%
\begin{pgfscope}%
\pgfpathrectangle{\pgfqpoint{0.100000in}{0.212622in}}{\pgfqpoint{3.696000in}{3.696000in}}%
\pgfusepath{clip}%
\pgfsetbuttcap%
\pgfsetroundjoin%
\definecolor{currentfill}{rgb}{0.121569,0.466667,0.705882}%
\pgfsetfillcolor{currentfill}%
\pgfsetfillopacity{0.932951}%
\pgfsetlinewidth{1.003750pt}%
\definecolor{currentstroke}{rgb}{0.121569,0.466667,0.705882}%
\pgfsetstrokecolor{currentstroke}%
\pgfsetstrokeopacity{0.932951}%
\pgfsetdash{}{0pt}%
\pgfpathmoveto{\pgfqpoint{2.554277in}{1.895562in}}%
\pgfpathcurveto{\pgfqpoint{2.562513in}{1.895562in}}{\pgfqpoint{2.570413in}{1.898835in}}{\pgfqpoint{2.576237in}{1.904658in}}%
\pgfpathcurveto{\pgfqpoint{2.582061in}{1.910482in}}{\pgfqpoint{2.585334in}{1.918382in}}{\pgfqpoint{2.585334in}{1.926619in}}%
\pgfpathcurveto{\pgfqpoint{2.585334in}{1.934855in}}{\pgfqpoint{2.582061in}{1.942755in}}{\pgfqpoint{2.576237in}{1.948579in}}%
\pgfpathcurveto{\pgfqpoint{2.570413in}{1.954403in}}{\pgfqpoint{2.562513in}{1.957675in}}{\pgfqpoint{2.554277in}{1.957675in}}%
\pgfpathcurveto{\pgfqpoint{2.546041in}{1.957675in}}{\pgfqpoint{2.538141in}{1.954403in}}{\pgfqpoint{2.532317in}{1.948579in}}%
\pgfpathcurveto{\pgfqpoint{2.526493in}{1.942755in}}{\pgfqpoint{2.523221in}{1.934855in}}{\pgfqpoint{2.523221in}{1.926619in}}%
\pgfpathcurveto{\pgfqpoint{2.523221in}{1.918382in}}{\pgfqpoint{2.526493in}{1.910482in}}{\pgfqpoint{2.532317in}{1.904658in}}%
\pgfpathcurveto{\pgfqpoint{2.538141in}{1.898835in}}{\pgfqpoint{2.546041in}{1.895562in}}{\pgfqpoint{2.554277in}{1.895562in}}%
\pgfpathclose%
\pgfusepath{stroke,fill}%
\end{pgfscope}%
\begin{pgfscope}%
\pgfpathrectangle{\pgfqpoint{0.100000in}{0.212622in}}{\pgfqpoint{3.696000in}{3.696000in}}%
\pgfusepath{clip}%
\pgfsetbuttcap%
\pgfsetroundjoin%
\definecolor{currentfill}{rgb}{0.121569,0.466667,0.705882}%
\pgfsetfillcolor{currentfill}%
\pgfsetfillopacity{0.932955}%
\pgfsetlinewidth{1.003750pt}%
\definecolor{currentstroke}{rgb}{0.121569,0.466667,0.705882}%
\pgfsetstrokecolor{currentstroke}%
\pgfsetstrokeopacity{0.932955}%
\pgfsetdash{}{0pt}%
\pgfpathmoveto{\pgfqpoint{1.833729in}{2.084853in}}%
\pgfpathcurveto{\pgfqpoint{1.841965in}{2.084853in}}{\pgfqpoint{1.849865in}{2.088125in}}{\pgfqpoint{1.855689in}{2.093949in}}%
\pgfpathcurveto{\pgfqpoint{1.861513in}{2.099773in}}{\pgfqpoint{1.864785in}{2.107673in}}{\pgfqpoint{1.864785in}{2.115909in}}%
\pgfpathcurveto{\pgfqpoint{1.864785in}{2.124145in}}{\pgfqpoint{1.861513in}{2.132046in}}{\pgfqpoint{1.855689in}{2.137869in}}%
\pgfpathcurveto{\pgfqpoint{1.849865in}{2.143693in}}{\pgfqpoint{1.841965in}{2.146966in}}{\pgfqpoint{1.833729in}{2.146966in}}%
\pgfpathcurveto{\pgfqpoint{1.825493in}{2.146966in}}{\pgfqpoint{1.817593in}{2.143693in}}{\pgfqpoint{1.811769in}{2.137869in}}%
\pgfpathcurveto{\pgfqpoint{1.805945in}{2.132046in}}{\pgfqpoint{1.802672in}{2.124145in}}{\pgfqpoint{1.802672in}{2.115909in}}%
\pgfpathcurveto{\pgfqpoint{1.802672in}{2.107673in}}{\pgfqpoint{1.805945in}{2.099773in}}{\pgfqpoint{1.811769in}{2.093949in}}%
\pgfpathcurveto{\pgfqpoint{1.817593in}{2.088125in}}{\pgfqpoint{1.825493in}{2.084853in}}{\pgfqpoint{1.833729in}{2.084853in}}%
\pgfpathclose%
\pgfusepath{stroke,fill}%
\end{pgfscope}%
\begin{pgfscope}%
\pgfpathrectangle{\pgfqpoint{0.100000in}{0.212622in}}{\pgfqpoint{3.696000in}{3.696000in}}%
\pgfusepath{clip}%
\pgfsetbuttcap%
\pgfsetroundjoin%
\definecolor{currentfill}{rgb}{0.121569,0.466667,0.705882}%
\pgfsetfillcolor{currentfill}%
\pgfsetfillopacity{0.934186}%
\pgfsetlinewidth{1.003750pt}%
\definecolor{currentstroke}{rgb}{0.121569,0.466667,0.705882}%
\pgfsetstrokecolor{currentstroke}%
\pgfsetstrokeopacity{0.934186}%
\pgfsetdash{}{0pt}%
\pgfpathmoveto{\pgfqpoint{1.861053in}{2.064345in}}%
\pgfpathcurveto{\pgfqpoint{1.869290in}{2.064345in}}{\pgfqpoint{1.877190in}{2.067617in}}{\pgfqpoint{1.883014in}{2.073441in}}%
\pgfpathcurveto{\pgfqpoint{1.888838in}{2.079265in}}{\pgfqpoint{1.892110in}{2.087165in}}{\pgfqpoint{1.892110in}{2.095401in}}%
\pgfpathcurveto{\pgfqpoint{1.892110in}{2.103638in}}{\pgfqpoint{1.888838in}{2.111538in}}{\pgfqpoint{1.883014in}{2.117362in}}%
\pgfpathcurveto{\pgfqpoint{1.877190in}{2.123186in}}{\pgfqpoint{1.869290in}{2.126458in}}{\pgfqpoint{1.861053in}{2.126458in}}%
\pgfpathcurveto{\pgfqpoint{1.852817in}{2.126458in}}{\pgfqpoint{1.844917in}{2.123186in}}{\pgfqpoint{1.839093in}{2.117362in}}%
\pgfpathcurveto{\pgfqpoint{1.833269in}{2.111538in}}{\pgfqpoint{1.829997in}{2.103638in}}{\pgfqpoint{1.829997in}{2.095401in}}%
\pgfpathcurveto{\pgfqpoint{1.829997in}{2.087165in}}{\pgfqpoint{1.833269in}{2.079265in}}{\pgfqpoint{1.839093in}{2.073441in}}%
\pgfpathcurveto{\pgfqpoint{1.844917in}{2.067617in}}{\pgfqpoint{1.852817in}{2.064345in}}{\pgfqpoint{1.861053in}{2.064345in}}%
\pgfpathclose%
\pgfusepath{stroke,fill}%
\end{pgfscope}%
\begin{pgfscope}%
\pgfpathrectangle{\pgfqpoint{0.100000in}{0.212622in}}{\pgfqpoint{3.696000in}{3.696000in}}%
\pgfusepath{clip}%
\pgfsetbuttcap%
\pgfsetroundjoin%
\definecolor{currentfill}{rgb}{0.121569,0.466667,0.705882}%
\pgfsetfillcolor{currentfill}%
\pgfsetfillopacity{0.934976}%
\pgfsetlinewidth{1.003750pt}%
\definecolor{currentstroke}{rgb}{0.121569,0.466667,0.705882}%
\pgfsetstrokecolor{currentstroke}%
\pgfsetstrokeopacity{0.934976}%
\pgfsetdash{}{0pt}%
\pgfpathmoveto{\pgfqpoint{2.549799in}{1.894055in}}%
\pgfpathcurveto{\pgfqpoint{2.558036in}{1.894055in}}{\pgfqpoint{2.565936in}{1.897328in}}{\pgfqpoint{2.571760in}{1.903151in}}%
\pgfpathcurveto{\pgfqpoint{2.577584in}{1.908975in}}{\pgfqpoint{2.580856in}{1.916875in}}{\pgfqpoint{2.580856in}{1.925112in}}%
\pgfpathcurveto{\pgfqpoint{2.580856in}{1.933348in}}{\pgfqpoint{2.577584in}{1.941248in}}{\pgfqpoint{2.571760in}{1.947072in}}%
\pgfpathcurveto{\pgfqpoint{2.565936in}{1.952896in}}{\pgfqpoint{2.558036in}{1.956168in}}{\pgfqpoint{2.549799in}{1.956168in}}%
\pgfpathcurveto{\pgfqpoint{2.541563in}{1.956168in}}{\pgfqpoint{2.533663in}{1.952896in}}{\pgfqpoint{2.527839in}{1.947072in}}%
\pgfpathcurveto{\pgfqpoint{2.522015in}{1.941248in}}{\pgfqpoint{2.518743in}{1.933348in}}{\pgfqpoint{2.518743in}{1.925112in}}%
\pgfpathcurveto{\pgfqpoint{2.518743in}{1.916875in}}{\pgfqpoint{2.522015in}{1.908975in}}{\pgfqpoint{2.527839in}{1.903151in}}%
\pgfpathcurveto{\pgfqpoint{2.533663in}{1.897328in}}{\pgfqpoint{2.541563in}{1.894055in}}{\pgfqpoint{2.549799in}{1.894055in}}%
\pgfpathclose%
\pgfusepath{stroke,fill}%
\end{pgfscope}%
\begin{pgfscope}%
\pgfpathrectangle{\pgfqpoint{0.100000in}{0.212622in}}{\pgfqpoint{3.696000in}{3.696000in}}%
\pgfusepath{clip}%
\pgfsetbuttcap%
\pgfsetroundjoin%
\definecolor{currentfill}{rgb}{0.121569,0.466667,0.705882}%
\pgfsetfillcolor{currentfill}%
\pgfsetfillopacity{0.937669}%
\pgfsetlinewidth{1.003750pt}%
\definecolor{currentstroke}{rgb}{0.121569,0.466667,0.705882}%
\pgfsetstrokecolor{currentstroke}%
\pgfsetstrokeopacity{0.937669}%
\pgfsetdash{}{0pt}%
\pgfpathmoveto{\pgfqpoint{1.879480in}{2.059647in}}%
\pgfpathcurveto{\pgfqpoint{1.887716in}{2.059647in}}{\pgfqpoint{1.895617in}{2.062919in}}{\pgfqpoint{1.901440in}{2.068743in}}%
\pgfpathcurveto{\pgfqpoint{1.907264in}{2.074567in}}{\pgfqpoint{1.910537in}{2.082467in}}{\pgfqpoint{1.910537in}{2.090704in}}%
\pgfpathcurveto{\pgfqpoint{1.910537in}{2.098940in}}{\pgfqpoint{1.907264in}{2.106840in}}{\pgfqpoint{1.901440in}{2.112664in}}%
\pgfpathcurveto{\pgfqpoint{1.895617in}{2.118488in}}{\pgfqpoint{1.887716in}{2.121760in}}{\pgfqpoint{1.879480in}{2.121760in}}%
\pgfpathcurveto{\pgfqpoint{1.871244in}{2.121760in}}{\pgfqpoint{1.863344in}{2.118488in}}{\pgfqpoint{1.857520in}{2.112664in}}%
\pgfpathcurveto{\pgfqpoint{1.851696in}{2.106840in}}{\pgfqpoint{1.848424in}{2.098940in}}{\pgfqpoint{1.848424in}{2.090704in}}%
\pgfpathcurveto{\pgfqpoint{1.848424in}{2.082467in}}{\pgfqpoint{1.851696in}{2.074567in}}{\pgfqpoint{1.857520in}{2.068743in}}%
\pgfpathcurveto{\pgfqpoint{1.863344in}{2.062919in}}{\pgfqpoint{1.871244in}{2.059647in}}{\pgfqpoint{1.879480in}{2.059647in}}%
\pgfpathclose%
\pgfusepath{stroke,fill}%
\end{pgfscope}%
\begin{pgfscope}%
\pgfpathrectangle{\pgfqpoint{0.100000in}{0.212622in}}{\pgfqpoint{3.696000in}{3.696000in}}%
\pgfusepath{clip}%
\pgfsetbuttcap%
\pgfsetroundjoin%
\definecolor{currentfill}{rgb}{0.121569,0.466667,0.705882}%
\pgfsetfillcolor{currentfill}%
\pgfsetfillopacity{0.937966}%
\pgfsetlinewidth{1.003750pt}%
\definecolor{currentstroke}{rgb}{0.121569,0.466667,0.705882}%
\pgfsetstrokecolor{currentstroke}%
\pgfsetstrokeopacity{0.937966}%
\pgfsetdash{}{0pt}%
\pgfpathmoveto{\pgfqpoint{2.542731in}{1.887782in}}%
\pgfpathcurveto{\pgfqpoint{2.550967in}{1.887782in}}{\pgfqpoint{2.558867in}{1.891055in}}{\pgfqpoint{2.564691in}{1.896879in}}%
\pgfpathcurveto{\pgfqpoint{2.570515in}{1.902702in}}{\pgfqpoint{2.573788in}{1.910603in}}{\pgfqpoint{2.573788in}{1.918839in}}%
\pgfpathcurveto{\pgfqpoint{2.573788in}{1.927075in}}{\pgfqpoint{2.570515in}{1.934975in}}{\pgfqpoint{2.564691in}{1.940799in}}%
\pgfpathcurveto{\pgfqpoint{2.558867in}{1.946623in}}{\pgfqpoint{2.550967in}{1.949895in}}{\pgfqpoint{2.542731in}{1.949895in}}%
\pgfpathcurveto{\pgfqpoint{2.534495in}{1.949895in}}{\pgfqpoint{2.526595in}{1.946623in}}{\pgfqpoint{2.520771in}{1.940799in}}%
\pgfpathcurveto{\pgfqpoint{2.514947in}{1.934975in}}{\pgfqpoint{2.511675in}{1.927075in}}{\pgfqpoint{2.511675in}{1.918839in}}%
\pgfpathcurveto{\pgfqpoint{2.511675in}{1.910603in}}{\pgfqpoint{2.514947in}{1.902702in}}{\pgfqpoint{2.520771in}{1.896879in}}%
\pgfpathcurveto{\pgfqpoint{2.526595in}{1.891055in}}{\pgfqpoint{2.534495in}{1.887782in}}{\pgfqpoint{2.542731in}{1.887782in}}%
\pgfpathclose%
\pgfusepath{stroke,fill}%
\end{pgfscope}%
\begin{pgfscope}%
\pgfpathrectangle{\pgfqpoint{0.100000in}{0.212622in}}{\pgfqpoint{3.696000in}{3.696000in}}%
\pgfusepath{clip}%
\pgfsetbuttcap%
\pgfsetroundjoin%
\definecolor{currentfill}{rgb}{0.121569,0.466667,0.705882}%
\pgfsetfillcolor{currentfill}%
\pgfsetfillopacity{0.938302}%
\pgfsetlinewidth{1.003750pt}%
\definecolor{currentstroke}{rgb}{0.121569,0.466667,0.705882}%
\pgfsetstrokecolor{currentstroke}%
\pgfsetstrokeopacity{0.938302}%
\pgfsetdash{}{0pt}%
\pgfpathmoveto{\pgfqpoint{1.895117in}{2.048968in}}%
\pgfpathcurveto{\pgfqpoint{1.903353in}{2.048968in}}{\pgfqpoint{1.911253in}{2.052240in}}{\pgfqpoint{1.917077in}{2.058064in}}%
\pgfpathcurveto{\pgfqpoint{1.922901in}{2.063888in}}{\pgfqpoint{1.926174in}{2.071788in}}{\pgfqpoint{1.926174in}{2.080024in}}%
\pgfpathcurveto{\pgfqpoint{1.926174in}{2.088260in}}{\pgfqpoint{1.922901in}{2.096160in}}{\pgfqpoint{1.917077in}{2.101984in}}%
\pgfpathcurveto{\pgfqpoint{1.911253in}{2.107808in}}{\pgfqpoint{1.903353in}{2.111081in}}{\pgfqpoint{1.895117in}{2.111081in}}%
\pgfpathcurveto{\pgfqpoint{1.886881in}{2.111081in}}{\pgfqpoint{1.878981in}{2.107808in}}{\pgfqpoint{1.873157in}{2.101984in}}%
\pgfpathcurveto{\pgfqpoint{1.867333in}{2.096160in}}{\pgfqpoint{1.864061in}{2.088260in}}{\pgfqpoint{1.864061in}{2.080024in}}%
\pgfpathcurveto{\pgfqpoint{1.864061in}{2.071788in}}{\pgfqpoint{1.867333in}{2.063888in}}{\pgfqpoint{1.873157in}{2.058064in}}%
\pgfpathcurveto{\pgfqpoint{1.878981in}{2.052240in}}{\pgfqpoint{1.886881in}{2.048968in}}{\pgfqpoint{1.895117in}{2.048968in}}%
\pgfpathclose%
\pgfusepath{stroke,fill}%
\end{pgfscope}%
\begin{pgfscope}%
\pgfpathrectangle{\pgfqpoint{0.100000in}{0.212622in}}{\pgfqpoint{3.696000in}{3.696000in}}%
\pgfusepath{clip}%
\pgfsetbuttcap%
\pgfsetroundjoin%
\definecolor{currentfill}{rgb}{0.121569,0.466667,0.705882}%
\pgfsetfillcolor{currentfill}%
\pgfsetfillopacity{0.940141}%
\pgfsetlinewidth{1.003750pt}%
\definecolor{currentstroke}{rgb}{0.121569,0.466667,0.705882}%
\pgfsetstrokecolor{currentstroke}%
\pgfsetstrokeopacity{0.940141}%
\pgfsetdash{}{0pt}%
\pgfpathmoveto{\pgfqpoint{1.906427in}{2.044691in}}%
\pgfpathcurveto{\pgfqpoint{1.914664in}{2.044691in}}{\pgfqpoint{1.922564in}{2.047963in}}{\pgfqpoint{1.928387in}{2.053787in}}%
\pgfpathcurveto{\pgfqpoint{1.934211in}{2.059611in}}{\pgfqpoint{1.937484in}{2.067511in}}{\pgfqpoint{1.937484in}{2.075747in}}%
\pgfpathcurveto{\pgfqpoint{1.937484in}{2.083983in}}{\pgfqpoint{1.934211in}{2.091883in}}{\pgfqpoint{1.928387in}{2.097707in}}%
\pgfpathcurveto{\pgfqpoint{1.922564in}{2.103531in}}{\pgfqpoint{1.914664in}{2.106804in}}{\pgfqpoint{1.906427in}{2.106804in}}%
\pgfpathcurveto{\pgfqpoint{1.898191in}{2.106804in}}{\pgfqpoint{1.890291in}{2.103531in}}{\pgfqpoint{1.884467in}{2.097707in}}%
\pgfpathcurveto{\pgfqpoint{1.878643in}{2.091883in}}{\pgfqpoint{1.875371in}{2.083983in}}{\pgfqpoint{1.875371in}{2.075747in}}%
\pgfpathcurveto{\pgfqpoint{1.875371in}{2.067511in}}{\pgfqpoint{1.878643in}{2.059611in}}{\pgfqpoint{1.884467in}{2.053787in}}%
\pgfpathcurveto{\pgfqpoint{1.890291in}{2.047963in}}{\pgfqpoint{1.898191in}{2.044691in}}{\pgfqpoint{1.906427in}{2.044691in}}%
\pgfpathclose%
\pgfusepath{stroke,fill}%
\end{pgfscope}%
\begin{pgfscope}%
\pgfpathrectangle{\pgfqpoint{0.100000in}{0.212622in}}{\pgfqpoint{3.696000in}{3.696000in}}%
\pgfusepath{clip}%
\pgfsetbuttcap%
\pgfsetroundjoin%
\definecolor{currentfill}{rgb}{0.121569,0.466667,0.705882}%
\pgfsetfillcolor{currentfill}%
\pgfsetfillopacity{0.940738}%
\pgfsetlinewidth{1.003750pt}%
\definecolor{currentstroke}{rgb}{0.121569,0.466667,0.705882}%
\pgfsetstrokecolor{currentstroke}%
\pgfsetstrokeopacity{0.940738}%
\pgfsetdash{}{0pt}%
\pgfpathmoveto{\pgfqpoint{1.913105in}{2.040803in}}%
\pgfpathcurveto{\pgfqpoint{1.921341in}{2.040803in}}{\pgfqpoint{1.929241in}{2.044075in}}{\pgfqpoint{1.935065in}{2.049899in}}%
\pgfpathcurveto{\pgfqpoint{1.940889in}{2.055723in}}{\pgfqpoint{1.944161in}{2.063623in}}{\pgfqpoint{1.944161in}{2.071860in}}%
\pgfpathcurveto{\pgfqpoint{1.944161in}{2.080096in}}{\pgfqpoint{1.940889in}{2.087996in}}{\pgfqpoint{1.935065in}{2.093820in}}%
\pgfpathcurveto{\pgfqpoint{1.929241in}{2.099644in}}{\pgfqpoint{1.921341in}{2.102916in}}{\pgfqpoint{1.913105in}{2.102916in}}%
\pgfpathcurveto{\pgfqpoint{1.904869in}{2.102916in}}{\pgfqpoint{1.896969in}{2.099644in}}{\pgfqpoint{1.891145in}{2.093820in}}%
\pgfpathcurveto{\pgfqpoint{1.885321in}{2.087996in}}{\pgfqpoint{1.882048in}{2.080096in}}{\pgfqpoint{1.882048in}{2.071860in}}%
\pgfpathcurveto{\pgfqpoint{1.882048in}{2.063623in}}{\pgfqpoint{1.885321in}{2.055723in}}{\pgfqpoint{1.891145in}{2.049899in}}%
\pgfpathcurveto{\pgfqpoint{1.896969in}{2.044075in}}{\pgfqpoint{1.904869in}{2.040803in}}{\pgfqpoint{1.913105in}{2.040803in}}%
\pgfpathclose%
\pgfusepath{stroke,fill}%
\end{pgfscope}%
\begin{pgfscope}%
\pgfpathrectangle{\pgfqpoint{0.100000in}{0.212622in}}{\pgfqpoint{3.696000in}{3.696000in}}%
\pgfusepath{clip}%
\pgfsetbuttcap%
\pgfsetroundjoin%
\definecolor{currentfill}{rgb}{0.121569,0.466667,0.705882}%
\pgfsetfillcolor{currentfill}%
\pgfsetfillopacity{0.942431}%
\pgfsetlinewidth{1.003750pt}%
\definecolor{currentstroke}{rgb}{0.121569,0.466667,0.705882}%
\pgfsetstrokecolor{currentstroke}%
\pgfsetstrokeopacity{0.942431}%
\pgfsetdash{}{0pt}%
\pgfpathmoveto{\pgfqpoint{2.529836in}{1.882708in}}%
\pgfpathcurveto{\pgfqpoint{2.538072in}{1.882708in}}{\pgfqpoint{2.545972in}{1.885980in}}{\pgfqpoint{2.551796in}{1.891804in}}%
\pgfpathcurveto{\pgfqpoint{2.557620in}{1.897628in}}{\pgfqpoint{2.560892in}{1.905528in}}{\pgfqpoint{2.560892in}{1.913764in}}%
\pgfpathcurveto{\pgfqpoint{2.560892in}{1.922000in}}{\pgfqpoint{2.557620in}{1.929900in}}{\pgfqpoint{2.551796in}{1.935724in}}%
\pgfpathcurveto{\pgfqpoint{2.545972in}{1.941548in}}{\pgfqpoint{2.538072in}{1.944821in}}{\pgfqpoint{2.529836in}{1.944821in}}%
\pgfpathcurveto{\pgfqpoint{2.521599in}{1.944821in}}{\pgfqpoint{2.513699in}{1.941548in}}{\pgfqpoint{2.507875in}{1.935724in}}%
\pgfpathcurveto{\pgfqpoint{2.502052in}{1.929900in}}{\pgfqpoint{2.498779in}{1.922000in}}{\pgfqpoint{2.498779in}{1.913764in}}%
\pgfpathcurveto{\pgfqpoint{2.498779in}{1.905528in}}{\pgfqpoint{2.502052in}{1.897628in}}{\pgfqpoint{2.507875in}{1.891804in}}%
\pgfpathcurveto{\pgfqpoint{2.513699in}{1.885980in}}{\pgfqpoint{2.521599in}{1.882708in}}{\pgfqpoint{2.529836in}{1.882708in}}%
\pgfpathclose%
\pgfusepath{stroke,fill}%
\end{pgfscope}%
\begin{pgfscope}%
\pgfpathrectangle{\pgfqpoint{0.100000in}{0.212622in}}{\pgfqpoint{3.696000in}{3.696000in}}%
\pgfusepath{clip}%
\pgfsetbuttcap%
\pgfsetroundjoin%
\definecolor{currentfill}{rgb}{0.121569,0.466667,0.705882}%
\pgfsetfillcolor{currentfill}%
\pgfsetfillopacity{0.942543}%
\pgfsetlinewidth{1.003750pt}%
\definecolor{currentstroke}{rgb}{0.121569,0.466667,0.705882}%
\pgfsetstrokecolor{currentstroke}%
\pgfsetstrokeopacity{0.942543}%
\pgfsetdash{}{0pt}%
\pgfpathmoveto{\pgfqpoint{1.924767in}{2.036554in}}%
\pgfpathcurveto{\pgfqpoint{1.933003in}{2.036554in}}{\pgfqpoint{1.940903in}{2.039826in}}{\pgfqpoint{1.946727in}{2.045650in}}%
\pgfpathcurveto{\pgfqpoint{1.952551in}{2.051474in}}{\pgfqpoint{1.955824in}{2.059374in}}{\pgfqpoint{1.955824in}{2.067610in}}%
\pgfpathcurveto{\pgfqpoint{1.955824in}{2.075847in}}{\pgfqpoint{1.952551in}{2.083747in}}{\pgfqpoint{1.946727in}{2.089571in}}%
\pgfpathcurveto{\pgfqpoint{1.940903in}{2.095395in}}{\pgfqpoint{1.933003in}{2.098667in}}{\pgfqpoint{1.924767in}{2.098667in}}%
\pgfpathcurveto{\pgfqpoint{1.916531in}{2.098667in}}{\pgfqpoint{1.908631in}{2.095395in}}{\pgfqpoint{1.902807in}{2.089571in}}%
\pgfpathcurveto{\pgfqpoint{1.896983in}{2.083747in}}{\pgfqpoint{1.893711in}{2.075847in}}{\pgfqpoint{1.893711in}{2.067610in}}%
\pgfpathcurveto{\pgfqpoint{1.893711in}{2.059374in}}{\pgfqpoint{1.896983in}{2.051474in}}{\pgfqpoint{1.902807in}{2.045650in}}%
\pgfpathcurveto{\pgfqpoint{1.908631in}{2.039826in}}{\pgfqpoint{1.916531in}{2.036554in}}{\pgfqpoint{1.924767in}{2.036554in}}%
\pgfpathclose%
\pgfusepath{stroke,fill}%
\end{pgfscope}%
\begin{pgfscope}%
\pgfpathrectangle{\pgfqpoint{0.100000in}{0.212622in}}{\pgfqpoint{3.696000in}{3.696000in}}%
\pgfusepath{clip}%
\pgfsetbuttcap%
\pgfsetroundjoin%
\definecolor{currentfill}{rgb}{0.121569,0.466667,0.705882}%
\pgfsetfillcolor{currentfill}%
\pgfsetfillopacity{0.944070}%
\pgfsetlinewidth{1.003750pt}%
\definecolor{currentstroke}{rgb}{0.121569,0.466667,0.705882}%
\pgfsetstrokecolor{currentstroke}%
\pgfsetstrokeopacity{0.944070}%
\pgfsetdash{}{0pt}%
\pgfpathmoveto{\pgfqpoint{1.947006in}{2.021177in}}%
\pgfpathcurveto{\pgfqpoint{1.955242in}{2.021177in}}{\pgfqpoint{1.963143in}{2.024449in}}{\pgfqpoint{1.968966in}{2.030273in}}%
\pgfpathcurveto{\pgfqpoint{1.974790in}{2.036097in}}{\pgfqpoint{1.978063in}{2.043997in}}{\pgfqpoint{1.978063in}{2.052233in}}%
\pgfpathcurveto{\pgfqpoint{1.978063in}{2.060469in}}{\pgfqpoint{1.974790in}{2.068370in}}{\pgfqpoint{1.968966in}{2.074193in}}%
\pgfpathcurveto{\pgfqpoint{1.963143in}{2.080017in}}{\pgfqpoint{1.955242in}{2.083290in}}{\pgfqpoint{1.947006in}{2.083290in}}%
\pgfpathcurveto{\pgfqpoint{1.938770in}{2.083290in}}{\pgfqpoint{1.930870in}{2.080017in}}{\pgfqpoint{1.925046in}{2.074193in}}%
\pgfpathcurveto{\pgfqpoint{1.919222in}{2.068370in}}{\pgfqpoint{1.915950in}{2.060469in}}{\pgfqpoint{1.915950in}{2.052233in}}%
\pgfpathcurveto{\pgfqpoint{1.915950in}{2.043997in}}{\pgfqpoint{1.919222in}{2.036097in}}{\pgfqpoint{1.925046in}{2.030273in}}%
\pgfpathcurveto{\pgfqpoint{1.930870in}{2.024449in}}{\pgfqpoint{1.938770in}{2.021177in}}{\pgfqpoint{1.947006in}{2.021177in}}%
\pgfpathclose%
\pgfusepath{stroke,fill}%
\end{pgfscope}%
\begin{pgfscope}%
\pgfpathrectangle{\pgfqpoint{0.100000in}{0.212622in}}{\pgfqpoint{3.696000in}{3.696000in}}%
\pgfusepath{clip}%
\pgfsetbuttcap%
\pgfsetroundjoin%
\definecolor{currentfill}{rgb}{0.121569,0.466667,0.705882}%
\pgfsetfillcolor{currentfill}%
\pgfsetfillopacity{0.947833}%
\pgfsetlinewidth{1.003750pt}%
\definecolor{currentstroke}{rgb}{0.121569,0.466667,0.705882}%
\pgfsetstrokecolor{currentstroke}%
\pgfsetstrokeopacity{0.947833}%
\pgfsetdash{}{0pt}%
\pgfpathmoveto{\pgfqpoint{2.519174in}{1.872472in}}%
\pgfpathcurveto{\pgfqpoint{2.527410in}{1.872472in}}{\pgfqpoint{2.535310in}{1.875745in}}{\pgfqpoint{2.541134in}{1.881569in}}%
\pgfpathcurveto{\pgfqpoint{2.546958in}{1.887393in}}{\pgfqpoint{2.550230in}{1.895293in}}{\pgfqpoint{2.550230in}{1.903529in}}%
\pgfpathcurveto{\pgfqpoint{2.550230in}{1.911765in}}{\pgfqpoint{2.546958in}{1.919665in}}{\pgfqpoint{2.541134in}{1.925489in}}%
\pgfpathcurveto{\pgfqpoint{2.535310in}{1.931313in}}{\pgfqpoint{2.527410in}{1.934585in}}{\pgfqpoint{2.519174in}{1.934585in}}%
\pgfpathcurveto{\pgfqpoint{2.510937in}{1.934585in}}{\pgfqpoint{2.503037in}{1.931313in}}{\pgfqpoint{2.497213in}{1.925489in}}%
\pgfpathcurveto{\pgfqpoint{2.491389in}{1.919665in}}{\pgfqpoint{2.488117in}{1.911765in}}{\pgfqpoint{2.488117in}{1.903529in}}%
\pgfpathcurveto{\pgfqpoint{2.488117in}{1.895293in}}{\pgfqpoint{2.491389in}{1.887393in}}{\pgfqpoint{2.497213in}{1.881569in}}%
\pgfpathcurveto{\pgfqpoint{2.503037in}{1.875745in}}{\pgfqpoint{2.510937in}{1.872472in}}{\pgfqpoint{2.519174in}{1.872472in}}%
\pgfpathclose%
\pgfusepath{stroke,fill}%
\end{pgfscope}%
\begin{pgfscope}%
\pgfpathrectangle{\pgfqpoint{0.100000in}{0.212622in}}{\pgfqpoint{3.696000in}{3.696000in}}%
\pgfusepath{clip}%
\pgfsetbuttcap%
\pgfsetroundjoin%
\definecolor{currentfill}{rgb}{0.121569,0.466667,0.705882}%
\pgfsetfillcolor{currentfill}%
\pgfsetfillopacity{0.950961}%
\pgfsetlinewidth{1.003750pt}%
\definecolor{currentstroke}{rgb}{0.121569,0.466667,0.705882}%
\pgfsetstrokecolor{currentstroke}%
\pgfsetstrokeopacity{0.950961}%
\pgfsetdash{}{0pt}%
\pgfpathmoveto{\pgfqpoint{1.982786in}{2.004462in}}%
\pgfpathcurveto{\pgfqpoint{1.991022in}{2.004462in}}{\pgfqpoint{1.998922in}{2.007735in}}{\pgfqpoint{2.004746in}{2.013559in}}%
\pgfpathcurveto{\pgfqpoint{2.010570in}{2.019382in}}{\pgfqpoint{2.013842in}{2.027283in}}{\pgfqpoint{2.013842in}{2.035519in}}%
\pgfpathcurveto{\pgfqpoint{2.013842in}{2.043755in}}{\pgfqpoint{2.010570in}{2.051655in}}{\pgfqpoint{2.004746in}{2.057479in}}%
\pgfpathcurveto{\pgfqpoint{1.998922in}{2.063303in}}{\pgfqpoint{1.991022in}{2.066575in}}{\pgfqpoint{1.982786in}{2.066575in}}%
\pgfpathcurveto{\pgfqpoint{1.974549in}{2.066575in}}{\pgfqpoint{1.966649in}{2.063303in}}{\pgfqpoint{1.960825in}{2.057479in}}%
\pgfpathcurveto{\pgfqpoint{1.955001in}{2.051655in}}{\pgfqpoint{1.951729in}{2.043755in}}{\pgfqpoint{1.951729in}{2.035519in}}%
\pgfpathcurveto{\pgfqpoint{1.951729in}{2.027283in}}{\pgfqpoint{1.955001in}{2.019382in}}{\pgfqpoint{1.960825in}{2.013559in}}%
\pgfpathcurveto{\pgfqpoint{1.966649in}{2.007735in}}{\pgfqpoint{1.974549in}{2.004462in}}{\pgfqpoint{1.982786in}{2.004462in}}%
\pgfpathclose%
\pgfusepath{stroke,fill}%
\end{pgfscope}%
\begin{pgfscope}%
\pgfpathrectangle{\pgfqpoint{0.100000in}{0.212622in}}{\pgfqpoint{3.696000in}{3.696000in}}%
\pgfusepath{clip}%
\pgfsetbuttcap%
\pgfsetroundjoin%
\definecolor{currentfill}{rgb}{0.121569,0.466667,0.705882}%
\pgfsetfillcolor{currentfill}%
\pgfsetfillopacity{0.953279}%
\pgfsetlinewidth{1.003750pt}%
\definecolor{currentstroke}{rgb}{0.121569,0.466667,0.705882}%
\pgfsetstrokecolor{currentstroke}%
\pgfsetstrokeopacity{0.953279}%
\pgfsetdash{}{0pt}%
\pgfpathmoveto{\pgfqpoint{2.501084in}{1.863729in}}%
\pgfpathcurveto{\pgfqpoint{2.509320in}{1.863729in}}{\pgfqpoint{2.517220in}{1.867002in}}{\pgfqpoint{2.523044in}{1.872826in}}%
\pgfpathcurveto{\pgfqpoint{2.528868in}{1.878650in}}{\pgfqpoint{2.532140in}{1.886550in}}{\pgfqpoint{2.532140in}{1.894786in}}%
\pgfpathcurveto{\pgfqpoint{2.532140in}{1.903022in}}{\pgfqpoint{2.528868in}{1.910922in}}{\pgfqpoint{2.523044in}{1.916746in}}%
\pgfpathcurveto{\pgfqpoint{2.517220in}{1.922570in}}{\pgfqpoint{2.509320in}{1.925842in}}{\pgfqpoint{2.501084in}{1.925842in}}%
\pgfpathcurveto{\pgfqpoint{2.492847in}{1.925842in}}{\pgfqpoint{2.484947in}{1.922570in}}{\pgfqpoint{2.479123in}{1.916746in}}%
\pgfpathcurveto{\pgfqpoint{2.473300in}{1.910922in}}{\pgfqpoint{2.470027in}{1.903022in}}{\pgfqpoint{2.470027in}{1.894786in}}%
\pgfpathcurveto{\pgfqpoint{2.470027in}{1.886550in}}{\pgfqpoint{2.473300in}{1.878650in}}{\pgfqpoint{2.479123in}{1.872826in}}%
\pgfpathcurveto{\pgfqpoint{2.484947in}{1.867002in}}{\pgfqpoint{2.492847in}{1.863729in}}{\pgfqpoint{2.501084in}{1.863729in}}%
\pgfpathclose%
\pgfusepath{stroke,fill}%
\end{pgfscope}%
\begin{pgfscope}%
\pgfpathrectangle{\pgfqpoint{0.100000in}{0.212622in}}{\pgfqpoint{3.696000in}{3.696000in}}%
\pgfusepath{clip}%
\pgfsetbuttcap%
\pgfsetroundjoin%
\definecolor{currentfill}{rgb}{0.121569,0.466667,0.705882}%
\pgfsetfillcolor{currentfill}%
\pgfsetfillopacity{0.953721}%
\pgfsetlinewidth{1.003750pt}%
\definecolor{currentstroke}{rgb}{0.121569,0.466667,0.705882}%
\pgfsetstrokecolor{currentstroke}%
\pgfsetstrokeopacity{0.953721}%
\pgfsetdash{}{0pt}%
\pgfpathmoveto{\pgfqpoint{2.020791in}{1.979270in}}%
\pgfpathcurveto{\pgfqpoint{2.029027in}{1.979270in}}{\pgfqpoint{2.036927in}{1.982542in}}{\pgfqpoint{2.042751in}{1.988366in}}%
\pgfpathcurveto{\pgfqpoint{2.048575in}{1.994190in}}{\pgfqpoint{2.051848in}{2.002090in}}{\pgfqpoint{2.051848in}{2.010327in}}%
\pgfpathcurveto{\pgfqpoint{2.051848in}{2.018563in}}{\pgfqpoint{2.048575in}{2.026463in}}{\pgfqpoint{2.042751in}{2.032287in}}%
\pgfpathcurveto{\pgfqpoint{2.036927in}{2.038111in}}{\pgfqpoint{2.029027in}{2.041383in}}{\pgfqpoint{2.020791in}{2.041383in}}%
\pgfpathcurveto{\pgfqpoint{2.012555in}{2.041383in}}{\pgfqpoint{2.004655in}{2.038111in}}{\pgfqpoint{1.998831in}{2.032287in}}%
\pgfpathcurveto{\pgfqpoint{1.993007in}{2.026463in}}{\pgfqpoint{1.989735in}{2.018563in}}{\pgfqpoint{1.989735in}{2.010327in}}%
\pgfpathcurveto{\pgfqpoint{1.989735in}{2.002090in}}{\pgfqpoint{1.993007in}{1.994190in}}{\pgfqpoint{1.998831in}{1.988366in}}%
\pgfpathcurveto{\pgfqpoint{2.004655in}{1.982542in}}{\pgfqpoint{2.012555in}{1.979270in}}{\pgfqpoint{2.020791in}{1.979270in}}%
\pgfpathclose%
\pgfusepath{stroke,fill}%
\end{pgfscope}%
\begin{pgfscope}%
\pgfpathrectangle{\pgfqpoint{0.100000in}{0.212622in}}{\pgfqpoint{3.696000in}{3.696000in}}%
\pgfusepath{clip}%
\pgfsetbuttcap%
\pgfsetroundjoin%
\definecolor{currentfill}{rgb}{0.121569,0.466667,0.705882}%
\pgfsetfillcolor{currentfill}%
\pgfsetfillopacity{0.956945}%
\pgfsetlinewidth{1.003750pt}%
\definecolor{currentstroke}{rgb}{0.121569,0.466667,0.705882}%
\pgfsetstrokecolor{currentstroke}%
\pgfsetstrokeopacity{0.956945}%
\pgfsetdash{}{0pt}%
\pgfpathmoveto{\pgfqpoint{2.493448in}{1.859577in}}%
\pgfpathcurveto{\pgfqpoint{2.501684in}{1.859577in}}{\pgfqpoint{2.509584in}{1.862850in}}{\pgfqpoint{2.515408in}{1.868674in}}%
\pgfpathcurveto{\pgfqpoint{2.521232in}{1.874498in}}{\pgfqpoint{2.524504in}{1.882398in}}{\pgfqpoint{2.524504in}{1.890634in}}%
\pgfpathcurveto{\pgfqpoint{2.524504in}{1.898870in}}{\pgfqpoint{2.521232in}{1.906770in}}{\pgfqpoint{2.515408in}{1.912594in}}%
\pgfpathcurveto{\pgfqpoint{2.509584in}{1.918418in}}{\pgfqpoint{2.501684in}{1.921690in}}{\pgfqpoint{2.493448in}{1.921690in}}%
\pgfpathcurveto{\pgfqpoint{2.485212in}{1.921690in}}{\pgfqpoint{2.477311in}{1.918418in}}{\pgfqpoint{2.471488in}{1.912594in}}%
\pgfpathcurveto{\pgfqpoint{2.465664in}{1.906770in}}{\pgfqpoint{2.462391in}{1.898870in}}{\pgfqpoint{2.462391in}{1.890634in}}%
\pgfpathcurveto{\pgfqpoint{2.462391in}{1.882398in}}{\pgfqpoint{2.465664in}{1.874498in}}{\pgfqpoint{2.471488in}{1.868674in}}%
\pgfpathcurveto{\pgfqpoint{2.477311in}{1.862850in}}{\pgfqpoint{2.485212in}{1.859577in}}{\pgfqpoint{2.493448in}{1.859577in}}%
\pgfpathclose%
\pgfusepath{stroke,fill}%
\end{pgfscope}%
\begin{pgfscope}%
\pgfpathrectangle{\pgfqpoint{0.100000in}{0.212622in}}{\pgfqpoint{3.696000in}{3.696000in}}%
\pgfusepath{clip}%
\pgfsetbuttcap%
\pgfsetroundjoin%
\definecolor{currentfill}{rgb}{0.121569,0.466667,0.705882}%
\pgfsetfillcolor{currentfill}%
\pgfsetfillopacity{0.957762}%
\pgfsetlinewidth{1.003750pt}%
\definecolor{currentstroke}{rgb}{0.121569,0.466667,0.705882}%
\pgfsetstrokecolor{currentstroke}%
\pgfsetstrokeopacity{0.957762}%
\pgfsetdash{}{0pt}%
\pgfpathmoveto{\pgfqpoint{2.054452in}{1.965912in}}%
\pgfpathcurveto{\pgfqpoint{2.062688in}{1.965912in}}{\pgfqpoint{2.070588in}{1.969184in}}{\pgfqpoint{2.076412in}{1.975008in}}%
\pgfpathcurveto{\pgfqpoint{2.082236in}{1.980832in}}{\pgfqpoint{2.085508in}{1.988732in}}{\pgfqpoint{2.085508in}{1.996968in}}%
\pgfpathcurveto{\pgfqpoint{2.085508in}{2.005204in}}{\pgfqpoint{2.082236in}{2.013104in}}{\pgfqpoint{2.076412in}{2.018928in}}%
\pgfpathcurveto{\pgfqpoint{2.070588in}{2.024752in}}{\pgfqpoint{2.062688in}{2.028025in}}{\pgfqpoint{2.054452in}{2.028025in}}%
\pgfpathcurveto{\pgfqpoint{2.046216in}{2.028025in}}{\pgfqpoint{2.038315in}{2.024752in}}{\pgfqpoint{2.032492in}{2.018928in}}%
\pgfpathcurveto{\pgfqpoint{2.026668in}{2.013104in}}{\pgfqpoint{2.023395in}{2.005204in}}{\pgfqpoint{2.023395in}{1.996968in}}%
\pgfpathcurveto{\pgfqpoint{2.023395in}{1.988732in}}{\pgfqpoint{2.026668in}{1.980832in}}{\pgfqpoint{2.032492in}{1.975008in}}%
\pgfpathcurveto{\pgfqpoint{2.038315in}{1.969184in}}{\pgfqpoint{2.046216in}{1.965912in}}{\pgfqpoint{2.054452in}{1.965912in}}%
\pgfpathclose%
\pgfusepath{stroke,fill}%
\end{pgfscope}%
\begin{pgfscope}%
\pgfpathrectangle{\pgfqpoint{0.100000in}{0.212622in}}{\pgfqpoint{3.696000in}{3.696000in}}%
\pgfusepath{clip}%
\pgfsetbuttcap%
\pgfsetroundjoin%
\definecolor{currentfill}{rgb}{0.121569,0.466667,0.705882}%
\pgfsetfillcolor{currentfill}%
\pgfsetfillopacity{0.958873}%
\pgfsetlinewidth{1.003750pt}%
\definecolor{currentstroke}{rgb}{0.121569,0.466667,0.705882}%
\pgfsetstrokecolor{currentstroke}%
\pgfsetstrokeopacity{0.958873}%
\pgfsetdash{}{0pt}%
\pgfpathmoveto{\pgfqpoint{2.488975in}{1.857048in}}%
\pgfpathcurveto{\pgfqpoint{2.497211in}{1.857048in}}{\pgfqpoint{2.505111in}{1.860320in}}{\pgfqpoint{2.510935in}{1.866144in}}%
\pgfpathcurveto{\pgfqpoint{2.516759in}{1.871968in}}{\pgfqpoint{2.520031in}{1.879868in}}{\pgfqpoint{2.520031in}{1.888104in}}%
\pgfpathcurveto{\pgfqpoint{2.520031in}{1.896340in}}{\pgfqpoint{2.516759in}{1.904240in}}{\pgfqpoint{2.510935in}{1.910064in}}%
\pgfpathcurveto{\pgfqpoint{2.505111in}{1.915888in}}{\pgfqpoint{2.497211in}{1.919161in}}{\pgfqpoint{2.488975in}{1.919161in}}%
\pgfpathcurveto{\pgfqpoint{2.480739in}{1.919161in}}{\pgfqpoint{2.472839in}{1.915888in}}{\pgfqpoint{2.467015in}{1.910064in}}%
\pgfpathcurveto{\pgfqpoint{2.461191in}{1.904240in}}{\pgfqpoint{2.457918in}{1.896340in}}{\pgfqpoint{2.457918in}{1.888104in}}%
\pgfpathcurveto{\pgfqpoint{2.457918in}{1.879868in}}{\pgfqpoint{2.461191in}{1.871968in}}{\pgfqpoint{2.467015in}{1.866144in}}%
\pgfpathcurveto{\pgfqpoint{2.472839in}{1.860320in}}{\pgfqpoint{2.480739in}{1.857048in}}{\pgfqpoint{2.488975in}{1.857048in}}%
\pgfpathclose%
\pgfusepath{stroke,fill}%
\end{pgfscope}%
\begin{pgfscope}%
\pgfpathrectangle{\pgfqpoint{0.100000in}{0.212622in}}{\pgfqpoint{3.696000in}{3.696000in}}%
\pgfusepath{clip}%
\pgfsetbuttcap%
\pgfsetroundjoin%
\definecolor{currentfill}{rgb}{0.121569,0.466667,0.705882}%
\pgfsetfillcolor{currentfill}%
\pgfsetfillopacity{0.960064}%
\pgfsetlinewidth{1.003750pt}%
\definecolor{currentstroke}{rgb}{0.121569,0.466667,0.705882}%
\pgfsetstrokecolor{currentstroke}%
\pgfsetstrokeopacity{0.960064}%
\pgfsetdash{}{0pt}%
\pgfpathmoveto{\pgfqpoint{2.486606in}{1.856389in}}%
\pgfpathcurveto{\pgfqpoint{2.494842in}{1.856389in}}{\pgfqpoint{2.502742in}{1.859661in}}{\pgfqpoint{2.508566in}{1.865485in}}%
\pgfpathcurveto{\pgfqpoint{2.514390in}{1.871309in}}{\pgfqpoint{2.517662in}{1.879209in}}{\pgfqpoint{2.517662in}{1.887445in}}%
\pgfpathcurveto{\pgfqpoint{2.517662in}{1.895682in}}{\pgfqpoint{2.514390in}{1.903582in}}{\pgfqpoint{2.508566in}{1.909406in}}%
\pgfpathcurveto{\pgfqpoint{2.502742in}{1.915230in}}{\pgfqpoint{2.494842in}{1.918502in}}{\pgfqpoint{2.486606in}{1.918502in}}%
\pgfpathcurveto{\pgfqpoint{2.478369in}{1.918502in}}{\pgfqpoint{2.470469in}{1.915230in}}{\pgfqpoint{2.464645in}{1.909406in}}%
\pgfpathcurveto{\pgfqpoint{2.458821in}{1.903582in}}{\pgfqpoint{2.455549in}{1.895682in}}{\pgfqpoint{2.455549in}{1.887445in}}%
\pgfpathcurveto{\pgfqpoint{2.455549in}{1.879209in}}{\pgfqpoint{2.458821in}{1.871309in}}{\pgfqpoint{2.464645in}{1.865485in}}%
\pgfpathcurveto{\pgfqpoint{2.470469in}{1.859661in}}{\pgfqpoint{2.478369in}{1.856389in}}{\pgfqpoint{2.486606in}{1.856389in}}%
\pgfpathclose%
\pgfusepath{stroke,fill}%
\end{pgfscope}%
\begin{pgfscope}%
\pgfpathrectangle{\pgfqpoint{0.100000in}{0.212622in}}{\pgfqpoint{3.696000in}{3.696000in}}%
\pgfusepath{clip}%
\pgfsetbuttcap%
\pgfsetroundjoin%
\definecolor{currentfill}{rgb}{0.121569,0.466667,0.705882}%
\pgfsetfillcolor{currentfill}%
\pgfsetfillopacity{0.960611}%
\pgfsetlinewidth{1.003750pt}%
\definecolor{currentstroke}{rgb}{0.121569,0.466667,0.705882}%
\pgfsetstrokecolor{currentstroke}%
\pgfsetstrokeopacity{0.960611}%
\pgfsetdash{}{0pt}%
\pgfpathmoveto{\pgfqpoint{2.485305in}{1.855332in}}%
\pgfpathcurveto{\pgfqpoint{2.493541in}{1.855332in}}{\pgfqpoint{2.501441in}{1.858604in}}{\pgfqpoint{2.507265in}{1.864428in}}%
\pgfpathcurveto{\pgfqpoint{2.513089in}{1.870252in}}{\pgfqpoint{2.516361in}{1.878152in}}{\pgfqpoint{2.516361in}{1.886389in}}%
\pgfpathcurveto{\pgfqpoint{2.516361in}{1.894625in}}{\pgfqpoint{2.513089in}{1.902525in}}{\pgfqpoint{2.507265in}{1.908349in}}%
\pgfpathcurveto{\pgfqpoint{2.501441in}{1.914173in}}{\pgfqpoint{2.493541in}{1.917445in}}{\pgfqpoint{2.485305in}{1.917445in}}%
\pgfpathcurveto{\pgfqpoint{2.477068in}{1.917445in}}{\pgfqpoint{2.469168in}{1.914173in}}{\pgfqpoint{2.463344in}{1.908349in}}%
\pgfpathcurveto{\pgfqpoint{2.457520in}{1.902525in}}{\pgfqpoint{2.454248in}{1.894625in}}{\pgfqpoint{2.454248in}{1.886389in}}%
\pgfpathcurveto{\pgfqpoint{2.454248in}{1.878152in}}{\pgfqpoint{2.457520in}{1.870252in}}{\pgfqpoint{2.463344in}{1.864428in}}%
\pgfpathcurveto{\pgfqpoint{2.469168in}{1.858604in}}{\pgfqpoint{2.477068in}{1.855332in}}{\pgfqpoint{2.485305in}{1.855332in}}%
\pgfpathclose%
\pgfusepath{stroke,fill}%
\end{pgfscope}%
\begin{pgfscope}%
\pgfpathrectangle{\pgfqpoint{0.100000in}{0.212622in}}{\pgfqpoint{3.696000in}{3.696000in}}%
\pgfusepath{clip}%
\pgfsetbuttcap%
\pgfsetroundjoin%
\definecolor{currentfill}{rgb}{0.121569,0.466667,0.705882}%
\pgfsetfillcolor{currentfill}%
\pgfsetfillopacity{0.960711}%
\pgfsetlinewidth{1.003750pt}%
\definecolor{currentstroke}{rgb}{0.121569,0.466667,0.705882}%
\pgfsetstrokecolor{currentstroke}%
\pgfsetstrokeopacity{0.960711}%
\pgfsetdash{}{0pt}%
\pgfpathmoveto{\pgfqpoint{2.079999in}{1.954508in}}%
\pgfpathcurveto{\pgfqpoint{2.088235in}{1.954508in}}{\pgfqpoint{2.096135in}{1.957780in}}{\pgfqpoint{2.101959in}{1.963604in}}%
\pgfpathcurveto{\pgfqpoint{2.107783in}{1.969428in}}{\pgfqpoint{2.111055in}{1.977328in}}{\pgfqpoint{2.111055in}{1.985564in}}%
\pgfpathcurveto{\pgfqpoint{2.111055in}{1.993800in}}{\pgfqpoint{2.107783in}{2.001701in}}{\pgfqpoint{2.101959in}{2.007524in}}%
\pgfpathcurveto{\pgfqpoint{2.096135in}{2.013348in}}{\pgfqpoint{2.088235in}{2.016621in}}{\pgfqpoint{2.079999in}{2.016621in}}%
\pgfpathcurveto{\pgfqpoint{2.071763in}{2.016621in}}{\pgfqpoint{2.063863in}{2.013348in}}{\pgfqpoint{2.058039in}{2.007524in}}%
\pgfpathcurveto{\pgfqpoint{2.052215in}{2.001701in}}{\pgfqpoint{2.048942in}{1.993800in}}{\pgfqpoint{2.048942in}{1.985564in}}%
\pgfpathcurveto{\pgfqpoint{2.048942in}{1.977328in}}{\pgfqpoint{2.052215in}{1.969428in}}{\pgfqpoint{2.058039in}{1.963604in}}%
\pgfpathcurveto{\pgfqpoint{2.063863in}{1.957780in}}{\pgfqpoint{2.071763in}{1.954508in}}{\pgfqpoint{2.079999in}{1.954508in}}%
\pgfpathclose%
\pgfusepath{stroke,fill}%
\end{pgfscope}%
\begin{pgfscope}%
\pgfpathrectangle{\pgfqpoint{0.100000in}{0.212622in}}{\pgfqpoint{3.696000in}{3.696000in}}%
\pgfusepath{clip}%
\pgfsetbuttcap%
\pgfsetroundjoin%
\definecolor{currentfill}{rgb}{0.121569,0.466667,0.705882}%
\pgfsetfillcolor{currentfill}%
\pgfsetfillopacity{0.960951}%
\pgfsetlinewidth{1.003750pt}%
\definecolor{currentstroke}{rgb}{0.121569,0.466667,0.705882}%
\pgfsetstrokecolor{currentstroke}%
\pgfsetstrokeopacity{0.960951}%
\pgfsetdash{}{0pt}%
\pgfpathmoveto{\pgfqpoint{2.484536in}{1.855068in}}%
\pgfpathcurveto{\pgfqpoint{2.492772in}{1.855068in}}{\pgfqpoint{2.500672in}{1.858340in}}{\pgfqpoint{2.506496in}{1.864164in}}%
\pgfpathcurveto{\pgfqpoint{2.512320in}{1.869988in}}{\pgfqpoint{2.515592in}{1.877888in}}{\pgfqpoint{2.515592in}{1.886124in}}%
\pgfpathcurveto{\pgfqpoint{2.515592in}{1.894360in}}{\pgfqpoint{2.512320in}{1.902260in}}{\pgfqpoint{2.506496in}{1.908084in}}%
\pgfpathcurveto{\pgfqpoint{2.500672in}{1.913908in}}{\pgfqpoint{2.492772in}{1.917181in}}{\pgfqpoint{2.484536in}{1.917181in}}%
\pgfpathcurveto{\pgfqpoint{2.476299in}{1.917181in}}{\pgfqpoint{2.468399in}{1.913908in}}{\pgfqpoint{2.462575in}{1.908084in}}%
\pgfpathcurveto{\pgfqpoint{2.456751in}{1.902260in}}{\pgfqpoint{2.453479in}{1.894360in}}{\pgfqpoint{2.453479in}{1.886124in}}%
\pgfpathcurveto{\pgfqpoint{2.453479in}{1.877888in}}{\pgfqpoint{2.456751in}{1.869988in}}{\pgfqpoint{2.462575in}{1.864164in}}%
\pgfpathcurveto{\pgfqpoint{2.468399in}{1.858340in}}{\pgfqpoint{2.476299in}{1.855068in}}{\pgfqpoint{2.484536in}{1.855068in}}%
\pgfpathclose%
\pgfusepath{stroke,fill}%
\end{pgfscope}%
\begin{pgfscope}%
\pgfpathrectangle{\pgfqpoint{0.100000in}{0.212622in}}{\pgfqpoint{3.696000in}{3.696000in}}%
\pgfusepath{clip}%
\pgfsetbuttcap%
\pgfsetroundjoin%
\definecolor{currentfill}{rgb}{0.121569,0.466667,0.705882}%
\pgfsetfillcolor{currentfill}%
\pgfsetfillopacity{0.962337}%
\pgfsetlinewidth{1.003750pt}%
\definecolor{currentstroke}{rgb}{0.121569,0.466667,0.705882}%
\pgfsetstrokecolor{currentstroke}%
\pgfsetstrokeopacity{0.962337}%
\pgfsetdash{}{0pt}%
\pgfpathmoveto{\pgfqpoint{2.481419in}{1.852212in}}%
\pgfpathcurveto{\pgfqpoint{2.489655in}{1.852212in}}{\pgfqpoint{2.497555in}{1.855485in}}{\pgfqpoint{2.503379in}{1.861309in}}%
\pgfpathcurveto{\pgfqpoint{2.509203in}{1.867133in}}{\pgfqpoint{2.512475in}{1.875033in}}{\pgfqpoint{2.512475in}{1.883269in}}%
\pgfpathcurveto{\pgfqpoint{2.512475in}{1.891505in}}{\pgfqpoint{2.509203in}{1.899405in}}{\pgfqpoint{2.503379in}{1.905229in}}%
\pgfpathcurveto{\pgfqpoint{2.497555in}{1.911053in}}{\pgfqpoint{2.489655in}{1.914325in}}{\pgfqpoint{2.481419in}{1.914325in}}%
\pgfpathcurveto{\pgfqpoint{2.473183in}{1.914325in}}{\pgfqpoint{2.465283in}{1.911053in}}{\pgfqpoint{2.459459in}{1.905229in}}%
\pgfpathcurveto{\pgfqpoint{2.453635in}{1.899405in}}{\pgfqpoint{2.450362in}{1.891505in}}{\pgfqpoint{2.450362in}{1.883269in}}%
\pgfpathcurveto{\pgfqpoint{2.450362in}{1.875033in}}{\pgfqpoint{2.453635in}{1.867133in}}{\pgfqpoint{2.459459in}{1.861309in}}%
\pgfpathcurveto{\pgfqpoint{2.465283in}{1.855485in}}{\pgfqpoint{2.473183in}{1.852212in}}{\pgfqpoint{2.481419in}{1.852212in}}%
\pgfpathclose%
\pgfusepath{stroke,fill}%
\end{pgfscope}%
\begin{pgfscope}%
\pgfpathrectangle{\pgfqpoint{0.100000in}{0.212622in}}{\pgfqpoint{3.696000in}{3.696000in}}%
\pgfusepath{clip}%
\pgfsetbuttcap%
\pgfsetroundjoin%
\definecolor{currentfill}{rgb}{0.121569,0.466667,0.705882}%
\pgfsetfillcolor{currentfill}%
\pgfsetfillopacity{0.963055}%
\pgfsetlinewidth{1.003750pt}%
\definecolor{currentstroke}{rgb}{0.121569,0.466667,0.705882}%
\pgfsetstrokecolor{currentstroke}%
\pgfsetstrokeopacity{0.963055}%
\pgfsetdash{}{0pt}%
\pgfpathmoveto{\pgfqpoint{2.097616in}{1.947364in}}%
\pgfpathcurveto{\pgfqpoint{2.105852in}{1.947364in}}{\pgfqpoint{2.113752in}{1.950636in}}{\pgfqpoint{2.119576in}{1.956460in}}%
\pgfpathcurveto{\pgfqpoint{2.125400in}{1.962284in}}{\pgfqpoint{2.128672in}{1.970184in}}{\pgfqpoint{2.128672in}{1.978420in}}%
\pgfpathcurveto{\pgfqpoint{2.128672in}{1.986656in}}{\pgfqpoint{2.125400in}{1.994556in}}{\pgfqpoint{2.119576in}{2.000380in}}%
\pgfpathcurveto{\pgfqpoint{2.113752in}{2.006204in}}{\pgfqpoint{2.105852in}{2.009477in}}{\pgfqpoint{2.097616in}{2.009477in}}%
\pgfpathcurveto{\pgfqpoint{2.089379in}{2.009477in}}{\pgfqpoint{2.081479in}{2.006204in}}{\pgfqpoint{2.075655in}{2.000380in}}%
\pgfpathcurveto{\pgfqpoint{2.069832in}{1.994556in}}{\pgfqpoint{2.066559in}{1.986656in}}{\pgfqpoint{2.066559in}{1.978420in}}%
\pgfpathcurveto{\pgfqpoint{2.066559in}{1.970184in}}{\pgfqpoint{2.069832in}{1.962284in}}{\pgfqpoint{2.075655in}{1.956460in}}%
\pgfpathcurveto{\pgfqpoint{2.081479in}{1.950636in}}{\pgfqpoint{2.089379in}{1.947364in}}{\pgfqpoint{2.097616in}{1.947364in}}%
\pgfpathclose%
\pgfusepath{stroke,fill}%
\end{pgfscope}%
\begin{pgfscope}%
\pgfpathrectangle{\pgfqpoint{0.100000in}{0.212622in}}{\pgfqpoint{3.696000in}{3.696000in}}%
\pgfusepath{clip}%
\pgfsetbuttcap%
\pgfsetroundjoin%
\definecolor{currentfill}{rgb}{0.121569,0.466667,0.705882}%
\pgfsetfillcolor{currentfill}%
\pgfsetfillopacity{0.964324}%
\pgfsetlinewidth{1.003750pt}%
\definecolor{currentstroke}{rgb}{0.121569,0.466667,0.705882}%
\pgfsetstrokecolor{currentstroke}%
\pgfsetstrokeopacity{0.964324}%
\pgfsetdash{}{0pt}%
\pgfpathmoveto{\pgfqpoint{2.476110in}{1.850088in}}%
\pgfpathcurveto{\pgfqpoint{2.484346in}{1.850088in}}{\pgfqpoint{2.492246in}{1.853360in}}{\pgfqpoint{2.498070in}{1.859184in}}%
\pgfpathcurveto{\pgfqpoint{2.503894in}{1.865008in}}{\pgfqpoint{2.507166in}{1.872908in}}{\pgfqpoint{2.507166in}{1.881145in}}%
\pgfpathcurveto{\pgfqpoint{2.507166in}{1.889381in}}{\pgfqpoint{2.503894in}{1.897281in}}{\pgfqpoint{2.498070in}{1.903105in}}%
\pgfpathcurveto{\pgfqpoint{2.492246in}{1.908929in}}{\pgfqpoint{2.484346in}{1.912201in}}{\pgfqpoint{2.476110in}{1.912201in}}%
\pgfpathcurveto{\pgfqpoint{2.467873in}{1.912201in}}{\pgfqpoint{2.459973in}{1.908929in}}{\pgfqpoint{2.454149in}{1.903105in}}%
\pgfpathcurveto{\pgfqpoint{2.448325in}{1.897281in}}{\pgfqpoint{2.445053in}{1.889381in}}{\pgfqpoint{2.445053in}{1.881145in}}%
\pgfpathcurveto{\pgfqpoint{2.445053in}{1.872908in}}{\pgfqpoint{2.448325in}{1.865008in}}{\pgfqpoint{2.454149in}{1.859184in}}%
\pgfpathcurveto{\pgfqpoint{2.459973in}{1.853360in}}{\pgfqpoint{2.467873in}{1.850088in}}{\pgfqpoint{2.476110in}{1.850088in}}%
\pgfpathclose%
\pgfusepath{stroke,fill}%
\end{pgfscope}%
\begin{pgfscope}%
\pgfpathrectangle{\pgfqpoint{0.100000in}{0.212622in}}{\pgfqpoint{3.696000in}{3.696000in}}%
\pgfusepath{clip}%
\pgfsetbuttcap%
\pgfsetroundjoin%
\definecolor{currentfill}{rgb}{0.121569,0.466667,0.705882}%
\pgfsetfillcolor{currentfill}%
\pgfsetfillopacity{0.964915}%
\pgfsetlinewidth{1.003750pt}%
\definecolor{currentstroke}{rgb}{0.121569,0.466667,0.705882}%
\pgfsetstrokecolor{currentstroke}%
\pgfsetstrokeopacity{0.964915}%
\pgfsetdash{}{0pt}%
\pgfpathmoveto{\pgfqpoint{2.111102in}{1.943138in}}%
\pgfpathcurveto{\pgfqpoint{2.119339in}{1.943138in}}{\pgfqpoint{2.127239in}{1.946410in}}{\pgfqpoint{2.133063in}{1.952234in}}%
\pgfpathcurveto{\pgfqpoint{2.138887in}{1.958058in}}{\pgfqpoint{2.142159in}{1.965958in}}{\pgfqpoint{2.142159in}{1.974194in}}%
\pgfpathcurveto{\pgfqpoint{2.142159in}{1.982430in}}{\pgfqpoint{2.138887in}{1.990330in}}{\pgfqpoint{2.133063in}{1.996154in}}%
\pgfpathcurveto{\pgfqpoint{2.127239in}{2.001978in}}{\pgfqpoint{2.119339in}{2.005251in}}{\pgfqpoint{2.111102in}{2.005251in}}%
\pgfpathcurveto{\pgfqpoint{2.102866in}{2.005251in}}{\pgfqpoint{2.094966in}{2.001978in}}{\pgfqpoint{2.089142in}{1.996154in}}%
\pgfpathcurveto{\pgfqpoint{2.083318in}{1.990330in}}{\pgfqpoint{2.080046in}{1.982430in}}{\pgfqpoint{2.080046in}{1.974194in}}%
\pgfpathcurveto{\pgfqpoint{2.080046in}{1.965958in}}{\pgfqpoint{2.083318in}{1.958058in}}{\pgfqpoint{2.089142in}{1.952234in}}%
\pgfpathcurveto{\pgfqpoint{2.094966in}{1.946410in}}{\pgfqpoint{2.102866in}{1.943138in}}{\pgfqpoint{2.111102in}{1.943138in}}%
\pgfpathclose%
\pgfusepath{stroke,fill}%
\end{pgfscope}%
\begin{pgfscope}%
\pgfpathrectangle{\pgfqpoint{0.100000in}{0.212622in}}{\pgfqpoint{3.696000in}{3.696000in}}%
\pgfusepath{clip}%
\pgfsetbuttcap%
\pgfsetroundjoin%
\definecolor{currentfill}{rgb}{0.121569,0.466667,0.705882}%
\pgfsetfillcolor{currentfill}%
\pgfsetfillopacity{0.965812}%
\pgfsetlinewidth{1.003750pt}%
\definecolor{currentstroke}{rgb}{0.121569,0.466667,0.705882}%
\pgfsetstrokecolor{currentstroke}%
\pgfsetstrokeopacity{0.965812}%
\pgfsetdash{}{0pt}%
\pgfpathmoveto{\pgfqpoint{2.121227in}{1.938993in}}%
\pgfpathcurveto{\pgfqpoint{2.129463in}{1.938993in}}{\pgfqpoint{2.137363in}{1.942266in}}{\pgfqpoint{2.143187in}{1.948090in}}%
\pgfpathcurveto{\pgfqpoint{2.149011in}{1.953914in}}{\pgfqpoint{2.152284in}{1.961814in}}{\pgfqpoint{2.152284in}{1.970050in}}%
\pgfpathcurveto{\pgfqpoint{2.152284in}{1.978286in}}{\pgfqpoint{2.149011in}{1.986186in}}{\pgfqpoint{2.143187in}{1.992010in}}%
\pgfpathcurveto{\pgfqpoint{2.137363in}{1.997834in}}{\pgfqpoint{2.129463in}{2.001106in}}{\pgfqpoint{2.121227in}{2.001106in}}%
\pgfpathcurveto{\pgfqpoint{2.112991in}{2.001106in}}{\pgfqpoint{2.105091in}{1.997834in}}{\pgfqpoint{2.099267in}{1.992010in}}%
\pgfpathcurveto{\pgfqpoint{2.093443in}{1.986186in}}{\pgfqpoint{2.090171in}{1.978286in}}{\pgfqpoint{2.090171in}{1.970050in}}%
\pgfpathcurveto{\pgfqpoint{2.090171in}{1.961814in}}{\pgfqpoint{2.093443in}{1.953914in}}{\pgfqpoint{2.099267in}{1.948090in}}%
\pgfpathcurveto{\pgfqpoint{2.105091in}{1.942266in}}{\pgfqpoint{2.112991in}{1.938993in}}{\pgfqpoint{2.121227in}{1.938993in}}%
\pgfpathclose%
\pgfusepath{stroke,fill}%
\end{pgfscope}%
\begin{pgfscope}%
\pgfpathrectangle{\pgfqpoint{0.100000in}{0.212622in}}{\pgfqpoint{3.696000in}{3.696000in}}%
\pgfusepath{clip}%
\pgfsetbuttcap%
\pgfsetroundjoin%
\definecolor{currentfill}{rgb}{0.121569,0.466667,0.705882}%
\pgfsetfillcolor{currentfill}%
\pgfsetfillopacity{0.967128}%
\pgfsetlinewidth{1.003750pt}%
\definecolor{currentstroke}{rgb}{0.121569,0.466667,0.705882}%
\pgfsetstrokecolor{currentstroke}%
\pgfsetstrokeopacity{0.967128}%
\pgfsetdash{}{0pt}%
\pgfpathmoveto{\pgfqpoint{2.128507in}{1.937625in}}%
\pgfpathcurveto{\pgfqpoint{2.136743in}{1.937625in}}{\pgfqpoint{2.144643in}{1.940897in}}{\pgfqpoint{2.150467in}{1.946721in}}%
\pgfpathcurveto{\pgfqpoint{2.156291in}{1.952545in}}{\pgfqpoint{2.159563in}{1.960445in}}{\pgfqpoint{2.159563in}{1.968681in}}%
\pgfpathcurveto{\pgfqpoint{2.159563in}{1.976918in}}{\pgfqpoint{2.156291in}{1.984818in}}{\pgfqpoint{2.150467in}{1.990642in}}%
\pgfpathcurveto{\pgfqpoint{2.144643in}{1.996466in}}{\pgfqpoint{2.136743in}{1.999738in}}{\pgfqpoint{2.128507in}{1.999738in}}%
\pgfpathcurveto{\pgfqpoint{2.120271in}{1.999738in}}{\pgfqpoint{2.112371in}{1.996466in}}{\pgfqpoint{2.106547in}{1.990642in}}%
\pgfpathcurveto{\pgfqpoint{2.100723in}{1.984818in}}{\pgfqpoint{2.097450in}{1.976918in}}{\pgfqpoint{2.097450in}{1.968681in}}%
\pgfpathcurveto{\pgfqpoint{2.097450in}{1.960445in}}{\pgfqpoint{2.100723in}{1.952545in}}{\pgfqpoint{2.106547in}{1.946721in}}%
\pgfpathcurveto{\pgfqpoint{2.112371in}{1.940897in}}{\pgfqpoint{2.120271in}{1.937625in}}{\pgfqpoint{2.128507in}{1.937625in}}%
\pgfpathclose%
\pgfusepath{stroke,fill}%
\end{pgfscope}%
\begin{pgfscope}%
\pgfpathrectangle{\pgfqpoint{0.100000in}{0.212622in}}{\pgfqpoint{3.696000in}{3.696000in}}%
\pgfusepath{clip}%
\pgfsetbuttcap%
\pgfsetroundjoin%
\definecolor{currentfill}{rgb}{0.121569,0.466667,0.705882}%
\pgfsetfillcolor{currentfill}%
\pgfsetfillopacity{0.967299}%
\pgfsetlinewidth{1.003750pt}%
\definecolor{currentstroke}{rgb}{0.121569,0.466667,0.705882}%
\pgfsetstrokecolor{currentstroke}%
\pgfsetstrokeopacity{0.967299}%
\pgfsetdash{}{0pt}%
\pgfpathmoveto{\pgfqpoint{2.469091in}{1.845156in}}%
\pgfpathcurveto{\pgfqpoint{2.477328in}{1.845156in}}{\pgfqpoint{2.485228in}{1.848428in}}{\pgfqpoint{2.491051in}{1.854252in}}%
\pgfpathcurveto{\pgfqpoint{2.496875in}{1.860076in}}{\pgfqpoint{2.500148in}{1.867976in}}{\pgfqpoint{2.500148in}{1.876212in}}%
\pgfpathcurveto{\pgfqpoint{2.500148in}{1.884448in}}{\pgfqpoint{2.496875in}{1.892348in}}{\pgfqpoint{2.491051in}{1.898172in}}%
\pgfpathcurveto{\pgfqpoint{2.485228in}{1.903996in}}{\pgfqpoint{2.477328in}{1.907269in}}{\pgfqpoint{2.469091in}{1.907269in}}%
\pgfpathcurveto{\pgfqpoint{2.460855in}{1.907269in}}{\pgfqpoint{2.452955in}{1.903996in}}{\pgfqpoint{2.447131in}{1.898172in}}%
\pgfpathcurveto{\pgfqpoint{2.441307in}{1.892348in}}{\pgfqpoint{2.438035in}{1.884448in}}{\pgfqpoint{2.438035in}{1.876212in}}%
\pgfpathcurveto{\pgfqpoint{2.438035in}{1.867976in}}{\pgfqpoint{2.441307in}{1.860076in}}{\pgfqpoint{2.447131in}{1.854252in}}%
\pgfpathcurveto{\pgfqpoint{2.452955in}{1.848428in}}{\pgfqpoint{2.460855in}{1.845156in}}{\pgfqpoint{2.469091in}{1.845156in}}%
\pgfpathclose%
\pgfusepath{stroke,fill}%
\end{pgfscope}%
\begin{pgfscope}%
\pgfpathrectangle{\pgfqpoint{0.100000in}{0.212622in}}{\pgfqpoint{3.696000in}{3.696000in}}%
\pgfusepath{clip}%
\pgfsetbuttcap%
\pgfsetroundjoin%
\definecolor{currentfill}{rgb}{0.121569,0.466667,0.705882}%
\pgfsetfillcolor{currentfill}%
\pgfsetfillopacity{0.967629}%
\pgfsetlinewidth{1.003750pt}%
\definecolor{currentstroke}{rgb}{0.121569,0.466667,0.705882}%
\pgfsetstrokecolor{currentstroke}%
\pgfsetstrokeopacity{0.967629}%
\pgfsetdash{}{0pt}%
\pgfpathmoveto{\pgfqpoint{2.142868in}{1.927679in}}%
\pgfpathcurveto{\pgfqpoint{2.151104in}{1.927679in}}{\pgfqpoint{2.159004in}{1.930951in}}{\pgfqpoint{2.164828in}{1.936775in}}%
\pgfpathcurveto{\pgfqpoint{2.170652in}{1.942599in}}{\pgfqpoint{2.173924in}{1.950499in}}{\pgfqpoint{2.173924in}{1.958735in}}%
\pgfpathcurveto{\pgfqpoint{2.173924in}{1.966971in}}{\pgfqpoint{2.170652in}{1.974872in}}{\pgfqpoint{2.164828in}{1.980695in}}%
\pgfpathcurveto{\pgfqpoint{2.159004in}{1.986519in}}{\pgfqpoint{2.151104in}{1.989792in}}{\pgfqpoint{2.142868in}{1.989792in}}%
\pgfpathcurveto{\pgfqpoint{2.134632in}{1.989792in}}{\pgfqpoint{2.126731in}{1.986519in}}{\pgfqpoint{2.120908in}{1.980695in}}%
\pgfpathcurveto{\pgfqpoint{2.115084in}{1.974872in}}{\pgfqpoint{2.111811in}{1.966971in}}{\pgfqpoint{2.111811in}{1.958735in}}%
\pgfpathcurveto{\pgfqpoint{2.111811in}{1.950499in}}{\pgfqpoint{2.115084in}{1.942599in}}{\pgfqpoint{2.120908in}{1.936775in}}%
\pgfpathcurveto{\pgfqpoint{2.126731in}{1.930951in}}{\pgfqpoint{2.134632in}{1.927679in}}{\pgfqpoint{2.142868in}{1.927679in}}%
\pgfpathclose%
\pgfusepath{stroke,fill}%
\end{pgfscope}%
\begin{pgfscope}%
\pgfpathrectangle{\pgfqpoint{0.100000in}{0.212622in}}{\pgfqpoint{3.696000in}{3.696000in}}%
\pgfusepath{clip}%
\pgfsetbuttcap%
\pgfsetroundjoin%
\definecolor{currentfill}{rgb}{0.121569,0.466667,0.705882}%
\pgfsetfillcolor{currentfill}%
\pgfsetfillopacity{0.968928}%
\pgfsetlinewidth{1.003750pt}%
\definecolor{currentstroke}{rgb}{0.121569,0.466667,0.705882}%
\pgfsetstrokecolor{currentstroke}%
\pgfsetstrokeopacity{0.968928}%
\pgfsetdash{}{0pt}%
\pgfpathmoveto{\pgfqpoint{2.464762in}{1.843070in}}%
\pgfpathcurveto{\pgfqpoint{2.472998in}{1.843070in}}{\pgfqpoint{2.480898in}{1.846342in}}{\pgfqpoint{2.486722in}{1.852166in}}%
\pgfpathcurveto{\pgfqpoint{2.492546in}{1.857990in}}{\pgfqpoint{2.495819in}{1.865890in}}{\pgfqpoint{2.495819in}{1.874126in}}%
\pgfpathcurveto{\pgfqpoint{2.495819in}{1.882362in}}{\pgfqpoint{2.492546in}{1.890262in}}{\pgfqpoint{2.486722in}{1.896086in}}%
\pgfpathcurveto{\pgfqpoint{2.480898in}{1.901910in}}{\pgfqpoint{2.472998in}{1.905183in}}{\pgfqpoint{2.464762in}{1.905183in}}%
\pgfpathcurveto{\pgfqpoint{2.456526in}{1.905183in}}{\pgfqpoint{2.448626in}{1.901910in}}{\pgfqpoint{2.442802in}{1.896086in}}%
\pgfpathcurveto{\pgfqpoint{2.436978in}{1.890262in}}{\pgfqpoint{2.433706in}{1.882362in}}{\pgfqpoint{2.433706in}{1.874126in}}%
\pgfpathcurveto{\pgfqpoint{2.433706in}{1.865890in}}{\pgfqpoint{2.436978in}{1.857990in}}{\pgfqpoint{2.442802in}{1.852166in}}%
\pgfpathcurveto{\pgfqpoint{2.448626in}{1.846342in}}{\pgfqpoint{2.456526in}{1.843070in}}{\pgfqpoint{2.464762in}{1.843070in}}%
\pgfpathclose%
\pgfusepath{stroke,fill}%
\end{pgfscope}%
\begin{pgfscope}%
\pgfpathrectangle{\pgfqpoint{0.100000in}{0.212622in}}{\pgfqpoint{3.696000in}{3.696000in}}%
\pgfusepath{clip}%
\pgfsetbuttcap%
\pgfsetroundjoin%
\definecolor{currentfill}{rgb}{0.121569,0.466667,0.705882}%
\pgfsetfillcolor{currentfill}%
\pgfsetfillopacity{0.969903}%
\pgfsetlinewidth{1.003750pt}%
\definecolor{currentstroke}{rgb}{0.121569,0.466667,0.705882}%
\pgfsetstrokecolor{currentstroke}%
\pgfsetstrokeopacity{0.969903}%
\pgfsetdash{}{0pt}%
\pgfpathmoveto{\pgfqpoint{2.462289in}{1.842530in}}%
\pgfpathcurveto{\pgfqpoint{2.470525in}{1.842530in}}{\pgfqpoint{2.478425in}{1.845802in}}{\pgfqpoint{2.484249in}{1.851626in}}%
\pgfpathcurveto{\pgfqpoint{2.490073in}{1.857450in}}{\pgfqpoint{2.493345in}{1.865350in}}{\pgfqpoint{2.493345in}{1.873586in}}%
\pgfpathcurveto{\pgfqpoint{2.493345in}{1.881823in}}{\pgfqpoint{2.490073in}{1.889723in}}{\pgfqpoint{2.484249in}{1.895547in}}%
\pgfpathcurveto{\pgfqpoint{2.478425in}{1.901370in}}{\pgfqpoint{2.470525in}{1.904643in}}{\pgfqpoint{2.462289in}{1.904643in}}%
\pgfpathcurveto{\pgfqpoint{2.454053in}{1.904643in}}{\pgfqpoint{2.446153in}{1.901370in}}{\pgfqpoint{2.440329in}{1.895547in}}%
\pgfpathcurveto{\pgfqpoint{2.434505in}{1.889723in}}{\pgfqpoint{2.431232in}{1.881823in}}{\pgfqpoint{2.431232in}{1.873586in}}%
\pgfpathcurveto{\pgfqpoint{2.431232in}{1.865350in}}{\pgfqpoint{2.434505in}{1.857450in}}{\pgfqpoint{2.440329in}{1.851626in}}%
\pgfpathcurveto{\pgfqpoint{2.446153in}{1.845802in}}{\pgfqpoint{2.454053in}{1.842530in}}{\pgfqpoint{2.462289in}{1.842530in}}%
\pgfpathclose%
\pgfusepath{stroke,fill}%
\end{pgfscope}%
\begin{pgfscope}%
\pgfpathrectangle{\pgfqpoint{0.100000in}{0.212622in}}{\pgfqpoint{3.696000in}{3.696000in}}%
\pgfusepath{clip}%
\pgfsetbuttcap%
\pgfsetroundjoin%
\definecolor{currentfill}{rgb}{0.121569,0.466667,0.705882}%
\pgfsetfillcolor{currentfill}%
\pgfsetfillopacity{0.970397}%
\pgfsetlinewidth{1.003750pt}%
\definecolor{currentstroke}{rgb}{0.121569,0.466667,0.705882}%
\pgfsetstrokecolor{currentstroke}%
\pgfsetstrokeopacity{0.970397}%
\pgfsetdash{}{0pt}%
\pgfpathmoveto{\pgfqpoint{2.461125in}{1.841697in}}%
\pgfpathcurveto{\pgfqpoint{2.469361in}{1.841697in}}{\pgfqpoint{2.477261in}{1.844969in}}{\pgfqpoint{2.483085in}{1.850793in}}%
\pgfpathcurveto{\pgfqpoint{2.488909in}{1.856617in}}{\pgfqpoint{2.492181in}{1.864517in}}{\pgfqpoint{2.492181in}{1.872753in}}%
\pgfpathcurveto{\pgfqpoint{2.492181in}{1.880990in}}{\pgfqpoint{2.488909in}{1.888890in}}{\pgfqpoint{2.483085in}{1.894714in}}%
\pgfpathcurveto{\pgfqpoint{2.477261in}{1.900537in}}{\pgfqpoint{2.469361in}{1.903810in}}{\pgfqpoint{2.461125in}{1.903810in}}%
\pgfpathcurveto{\pgfqpoint{2.452888in}{1.903810in}}{\pgfqpoint{2.444988in}{1.900537in}}{\pgfqpoint{2.439164in}{1.894714in}}%
\pgfpathcurveto{\pgfqpoint{2.433340in}{1.888890in}}{\pgfqpoint{2.430068in}{1.880990in}}{\pgfqpoint{2.430068in}{1.872753in}}%
\pgfpathcurveto{\pgfqpoint{2.430068in}{1.864517in}}{\pgfqpoint{2.433340in}{1.856617in}}{\pgfqpoint{2.439164in}{1.850793in}}%
\pgfpathcurveto{\pgfqpoint{2.444988in}{1.844969in}}{\pgfqpoint{2.452888in}{1.841697in}}{\pgfqpoint{2.461125in}{1.841697in}}%
\pgfpathclose%
\pgfusepath{stroke,fill}%
\end{pgfscope}%
\begin{pgfscope}%
\pgfpathrectangle{\pgfqpoint{0.100000in}{0.212622in}}{\pgfqpoint{3.696000in}{3.696000in}}%
\pgfusepath{clip}%
\pgfsetbuttcap%
\pgfsetroundjoin%
\definecolor{currentfill}{rgb}{0.121569,0.466667,0.705882}%
\pgfsetfillcolor{currentfill}%
\pgfsetfillopacity{0.970699}%
\pgfsetlinewidth{1.003750pt}%
\definecolor{currentstroke}{rgb}{0.121569,0.466667,0.705882}%
\pgfsetstrokecolor{currentstroke}%
\pgfsetstrokeopacity{0.970699}%
\pgfsetdash{}{0pt}%
\pgfpathmoveto{\pgfqpoint{2.460421in}{1.841507in}}%
\pgfpathcurveto{\pgfqpoint{2.468657in}{1.841507in}}{\pgfqpoint{2.476557in}{1.844779in}}{\pgfqpoint{2.482381in}{1.850603in}}%
\pgfpathcurveto{\pgfqpoint{2.488205in}{1.856427in}}{\pgfqpoint{2.491478in}{1.864327in}}{\pgfqpoint{2.491478in}{1.872563in}}%
\pgfpathcurveto{\pgfqpoint{2.491478in}{1.880800in}}{\pgfqpoint{2.488205in}{1.888700in}}{\pgfqpoint{2.482381in}{1.894524in}}%
\pgfpathcurveto{\pgfqpoint{2.476557in}{1.900348in}}{\pgfqpoint{2.468657in}{1.903620in}}{\pgfqpoint{2.460421in}{1.903620in}}%
\pgfpathcurveto{\pgfqpoint{2.452185in}{1.903620in}}{\pgfqpoint{2.444285in}{1.900348in}}{\pgfqpoint{2.438461in}{1.894524in}}%
\pgfpathcurveto{\pgfqpoint{2.432637in}{1.888700in}}{\pgfqpoint{2.429365in}{1.880800in}}{\pgfqpoint{2.429365in}{1.872563in}}%
\pgfpathcurveto{\pgfqpoint{2.429365in}{1.864327in}}{\pgfqpoint{2.432637in}{1.856427in}}{\pgfqpoint{2.438461in}{1.850603in}}%
\pgfpathcurveto{\pgfqpoint{2.444285in}{1.844779in}}{\pgfqpoint{2.452185in}{1.841507in}}{\pgfqpoint{2.460421in}{1.841507in}}%
\pgfpathclose%
\pgfusepath{stroke,fill}%
\end{pgfscope}%
\begin{pgfscope}%
\pgfpathrectangle{\pgfqpoint{0.100000in}{0.212622in}}{\pgfqpoint{3.696000in}{3.696000in}}%
\pgfusepath{clip}%
\pgfsetbuttcap%
\pgfsetroundjoin%
\definecolor{currentfill}{rgb}{0.121569,0.466667,0.705882}%
\pgfsetfillcolor{currentfill}%
\pgfsetfillopacity{0.970856}%
\pgfsetlinewidth{1.003750pt}%
\definecolor{currentstroke}{rgb}{0.121569,0.466667,0.705882}%
\pgfsetstrokecolor{currentstroke}%
\pgfsetstrokeopacity{0.970856}%
\pgfsetdash{}{0pt}%
\pgfpathmoveto{\pgfqpoint{2.460096in}{1.841259in}}%
\pgfpathcurveto{\pgfqpoint{2.468332in}{1.841259in}}{\pgfqpoint{2.476232in}{1.844531in}}{\pgfqpoint{2.482056in}{1.850355in}}%
\pgfpathcurveto{\pgfqpoint{2.487880in}{1.856179in}}{\pgfqpoint{2.491153in}{1.864079in}}{\pgfqpoint{2.491153in}{1.872316in}}%
\pgfpathcurveto{\pgfqpoint{2.491153in}{1.880552in}}{\pgfqpoint{2.487880in}{1.888452in}}{\pgfqpoint{2.482056in}{1.894276in}}%
\pgfpathcurveto{\pgfqpoint{2.476232in}{1.900100in}}{\pgfqpoint{2.468332in}{1.903372in}}{\pgfqpoint{2.460096in}{1.903372in}}%
\pgfpathcurveto{\pgfqpoint{2.451860in}{1.903372in}}{\pgfqpoint{2.443960in}{1.900100in}}{\pgfqpoint{2.438136in}{1.894276in}}%
\pgfpathcurveto{\pgfqpoint{2.432312in}{1.888452in}}{\pgfqpoint{2.429040in}{1.880552in}}{\pgfqpoint{2.429040in}{1.872316in}}%
\pgfpathcurveto{\pgfqpoint{2.429040in}{1.864079in}}{\pgfqpoint{2.432312in}{1.856179in}}{\pgfqpoint{2.438136in}{1.850355in}}%
\pgfpathcurveto{\pgfqpoint{2.443960in}{1.844531in}}{\pgfqpoint{2.451860in}{1.841259in}}{\pgfqpoint{2.460096in}{1.841259in}}%
\pgfpathclose%
\pgfusepath{stroke,fill}%
\end{pgfscope}%
\begin{pgfscope}%
\pgfpathrectangle{\pgfqpoint{0.100000in}{0.212622in}}{\pgfqpoint{3.696000in}{3.696000in}}%
\pgfusepath{clip}%
\pgfsetbuttcap%
\pgfsetroundjoin%
\definecolor{currentfill}{rgb}{0.121569,0.466667,0.705882}%
\pgfsetfillcolor{currentfill}%
\pgfsetfillopacity{0.970948}%
\pgfsetlinewidth{1.003750pt}%
\definecolor{currentstroke}{rgb}{0.121569,0.466667,0.705882}%
\pgfsetstrokecolor{currentstroke}%
\pgfsetstrokeopacity{0.970948}%
\pgfsetdash{}{0pt}%
\pgfpathmoveto{\pgfqpoint{2.459903in}{1.841180in}}%
\pgfpathcurveto{\pgfqpoint{2.468139in}{1.841180in}}{\pgfqpoint{2.476039in}{1.844452in}}{\pgfqpoint{2.481863in}{1.850276in}}%
\pgfpathcurveto{\pgfqpoint{2.487687in}{1.856100in}}{\pgfqpoint{2.490959in}{1.864000in}}{\pgfqpoint{2.490959in}{1.872236in}}%
\pgfpathcurveto{\pgfqpoint{2.490959in}{1.880473in}}{\pgfqpoint{2.487687in}{1.888373in}}{\pgfqpoint{2.481863in}{1.894197in}}%
\pgfpathcurveto{\pgfqpoint{2.476039in}{1.900021in}}{\pgfqpoint{2.468139in}{1.903293in}}{\pgfqpoint{2.459903in}{1.903293in}}%
\pgfpathcurveto{\pgfqpoint{2.451666in}{1.903293in}}{\pgfqpoint{2.443766in}{1.900021in}}{\pgfqpoint{2.437942in}{1.894197in}}%
\pgfpathcurveto{\pgfqpoint{2.432119in}{1.888373in}}{\pgfqpoint{2.428846in}{1.880473in}}{\pgfqpoint{2.428846in}{1.872236in}}%
\pgfpathcurveto{\pgfqpoint{2.428846in}{1.864000in}}{\pgfqpoint{2.432119in}{1.856100in}}{\pgfqpoint{2.437942in}{1.850276in}}%
\pgfpathcurveto{\pgfqpoint{2.443766in}{1.844452in}}{\pgfqpoint{2.451666in}{1.841180in}}{\pgfqpoint{2.459903in}{1.841180in}}%
\pgfpathclose%
\pgfusepath{stroke,fill}%
\end{pgfscope}%
\begin{pgfscope}%
\pgfpathrectangle{\pgfqpoint{0.100000in}{0.212622in}}{\pgfqpoint{3.696000in}{3.696000in}}%
\pgfusepath{clip}%
\pgfsetbuttcap%
\pgfsetroundjoin%
\definecolor{currentfill}{rgb}{0.121569,0.466667,0.705882}%
\pgfsetfillcolor{currentfill}%
\pgfsetfillopacity{0.971961}%
\pgfsetlinewidth{1.003750pt}%
\definecolor{currentstroke}{rgb}{0.121569,0.466667,0.705882}%
\pgfsetstrokecolor{currentstroke}%
\pgfsetstrokeopacity{0.971961}%
\pgfsetdash{}{0pt}%
\pgfpathmoveto{\pgfqpoint{2.457980in}{1.839338in}}%
\pgfpathcurveto{\pgfqpoint{2.466217in}{1.839338in}}{\pgfqpoint{2.474117in}{1.842611in}}{\pgfqpoint{2.479941in}{1.848435in}}%
\pgfpathcurveto{\pgfqpoint{2.485765in}{1.854259in}}{\pgfqpoint{2.489037in}{1.862159in}}{\pgfqpoint{2.489037in}{1.870395in}}%
\pgfpathcurveto{\pgfqpoint{2.489037in}{1.878631in}}{\pgfqpoint{2.485765in}{1.886531in}}{\pgfqpoint{2.479941in}{1.892355in}}%
\pgfpathcurveto{\pgfqpoint{2.474117in}{1.898179in}}{\pgfqpoint{2.466217in}{1.901451in}}{\pgfqpoint{2.457980in}{1.901451in}}%
\pgfpathcurveto{\pgfqpoint{2.449744in}{1.901451in}}{\pgfqpoint{2.441844in}{1.898179in}}{\pgfqpoint{2.436020in}{1.892355in}}%
\pgfpathcurveto{\pgfqpoint{2.430196in}{1.886531in}}{\pgfqpoint{2.426924in}{1.878631in}}{\pgfqpoint{2.426924in}{1.870395in}}%
\pgfpathcurveto{\pgfqpoint{2.426924in}{1.862159in}}{\pgfqpoint{2.430196in}{1.854259in}}{\pgfqpoint{2.436020in}{1.848435in}}%
\pgfpathcurveto{\pgfqpoint{2.441844in}{1.842611in}}{\pgfqpoint{2.449744in}{1.839338in}}{\pgfqpoint{2.457980in}{1.839338in}}%
\pgfpathclose%
\pgfusepath{stroke,fill}%
\end{pgfscope}%
\begin{pgfscope}%
\pgfpathrectangle{\pgfqpoint{0.100000in}{0.212622in}}{\pgfqpoint{3.696000in}{3.696000in}}%
\pgfusepath{clip}%
\pgfsetbuttcap%
\pgfsetroundjoin%
\definecolor{currentfill}{rgb}{0.121569,0.466667,0.705882}%
\pgfsetfillcolor{currentfill}%
\pgfsetfillopacity{0.972508}%
\pgfsetlinewidth{1.003750pt}%
\definecolor{currentstroke}{rgb}{0.121569,0.466667,0.705882}%
\pgfsetstrokecolor{currentstroke}%
\pgfsetstrokeopacity{0.972508}%
\pgfsetdash{}{0pt}%
\pgfpathmoveto{\pgfqpoint{2.166009in}{1.923538in}}%
\pgfpathcurveto{\pgfqpoint{2.174246in}{1.923538in}}{\pgfqpoint{2.182146in}{1.926811in}}{\pgfqpoint{2.187970in}{1.932635in}}%
\pgfpathcurveto{\pgfqpoint{2.193794in}{1.938459in}}{\pgfqpoint{2.197066in}{1.946359in}}{\pgfqpoint{2.197066in}{1.954595in}}%
\pgfpathcurveto{\pgfqpoint{2.197066in}{1.962831in}}{\pgfqpoint{2.193794in}{1.970731in}}{\pgfqpoint{2.187970in}{1.976555in}}%
\pgfpathcurveto{\pgfqpoint{2.182146in}{1.982379in}}{\pgfqpoint{2.174246in}{1.985651in}}{\pgfqpoint{2.166009in}{1.985651in}}%
\pgfpathcurveto{\pgfqpoint{2.157773in}{1.985651in}}{\pgfqpoint{2.149873in}{1.982379in}}{\pgfqpoint{2.144049in}{1.976555in}}%
\pgfpathcurveto{\pgfqpoint{2.138225in}{1.970731in}}{\pgfqpoint{2.134953in}{1.962831in}}{\pgfqpoint{2.134953in}{1.954595in}}%
\pgfpathcurveto{\pgfqpoint{2.134953in}{1.946359in}}{\pgfqpoint{2.138225in}{1.938459in}}{\pgfqpoint{2.144049in}{1.932635in}}%
\pgfpathcurveto{\pgfqpoint{2.149873in}{1.926811in}}{\pgfqpoint{2.157773in}{1.923538in}}{\pgfqpoint{2.166009in}{1.923538in}}%
\pgfpathclose%
\pgfusepath{stroke,fill}%
\end{pgfscope}%
\begin{pgfscope}%
\pgfpathrectangle{\pgfqpoint{0.100000in}{0.212622in}}{\pgfqpoint{3.696000in}{3.696000in}}%
\pgfusepath{clip}%
\pgfsetbuttcap%
\pgfsetroundjoin%
\definecolor{currentfill}{rgb}{0.121569,0.466667,0.705882}%
\pgfsetfillcolor{currentfill}%
\pgfsetfillopacity{0.973702}%
\pgfsetlinewidth{1.003750pt}%
\definecolor{currentstroke}{rgb}{0.121569,0.466667,0.705882}%
\pgfsetstrokecolor{currentstroke}%
\pgfsetstrokeopacity{0.973702}%
\pgfsetdash{}{0pt}%
\pgfpathmoveto{\pgfqpoint{2.454047in}{1.837479in}}%
\pgfpathcurveto{\pgfqpoint{2.462283in}{1.837479in}}{\pgfqpoint{2.470183in}{1.840752in}}{\pgfqpoint{2.476007in}{1.846576in}}%
\pgfpathcurveto{\pgfqpoint{2.481831in}{1.852400in}}{\pgfqpoint{2.485103in}{1.860300in}}{\pgfqpoint{2.485103in}{1.868536in}}%
\pgfpathcurveto{\pgfqpoint{2.485103in}{1.876772in}}{\pgfqpoint{2.481831in}{1.884672in}}{\pgfqpoint{2.476007in}{1.890496in}}%
\pgfpathcurveto{\pgfqpoint{2.470183in}{1.896320in}}{\pgfqpoint{2.462283in}{1.899592in}}{\pgfqpoint{2.454047in}{1.899592in}}%
\pgfpathcurveto{\pgfqpoint{2.445810in}{1.899592in}}{\pgfqpoint{2.437910in}{1.896320in}}{\pgfqpoint{2.432086in}{1.890496in}}%
\pgfpathcurveto{\pgfqpoint{2.426262in}{1.884672in}}{\pgfqpoint{2.422990in}{1.876772in}}{\pgfqpoint{2.422990in}{1.868536in}}%
\pgfpathcurveto{\pgfqpoint{2.422990in}{1.860300in}}{\pgfqpoint{2.426262in}{1.852400in}}{\pgfqpoint{2.432086in}{1.846576in}}%
\pgfpathcurveto{\pgfqpoint{2.437910in}{1.840752in}}{\pgfqpoint{2.445810in}{1.837479in}}{\pgfqpoint{2.454047in}{1.837479in}}%
\pgfpathclose%
\pgfusepath{stroke,fill}%
\end{pgfscope}%
\begin{pgfscope}%
\pgfpathrectangle{\pgfqpoint{0.100000in}{0.212622in}}{\pgfqpoint{3.696000in}{3.696000in}}%
\pgfusepath{clip}%
\pgfsetbuttcap%
\pgfsetroundjoin%
\definecolor{currentfill}{rgb}{0.121569,0.466667,0.705882}%
\pgfsetfillcolor{currentfill}%
\pgfsetfillopacity{0.973917}%
\pgfsetlinewidth{1.003750pt}%
\definecolor{currentstroke}{rgb}{0.121569,0.466667,0.705882}%
\pgfsetstrokecolor{currentstroke}%
\pgfsetstrokeopacity{0.973917}%
\pgfsetdash{}{0pt}%
\pgfpathmoveto{\pgfqpoint{2.188848in}{1.908392in}}%
\pgfpathcurveto{\pgfqpoint{2.197084in}{1.908392in}}{\pgfqpoint{2.204984in}{1.911665in}}{\pgfqpoint{2.210808in}{1.917489in}}%
\pgfpathcurveto{\pgfqpoint{2.216632in}{1.923313in}}{\pgfqpoint{2.219905in}{1.931213in}}{\pgfqpoint{2.219905in}{1.939449in}}%
\pgfpathcurveto{\pgfqpoint{2.219905in}{1.947685in}}{\pgfqpoint{2.216632in}{1.955585in}}{\pgfqpoint{2.210808in}{1.961409in}}%
\pgfpathcurveto{\pgfqpoint{2.204984in}{1.967233in}}{\pgfqpoint{2.197084in}{1.970505in}}{\pgfqpoint{2.188848in}{1.970505in}}%
\pgfpathcurveto{\pgfqpoint{2.180612in}{1.970505in}}{\pgfqpoint{2.172712in}{1.967233in}}{\pgfqpoint{2.166888in}{1.961409in}}%
\pgfpathcurveto{\pgfqpoint{2.161064in}{1.955585in}}{\pgfqpoint{2.157792in}{1.947685in}}{\pgfqpoint{2.157792in}{1.939449in}}%
\pgfpathcurveto{\pgfqpoint{2.157792in}{1.931213in}}{\pgfqpoint{2.161064in}{1.923313in}}{\pgfqpoint{2.166888in}{1.917489in}}%
\pgfpathcurveto{\pgfqpoint{2.172712in}{1.911665in}}{\pgfqpoint{2.180612in}{1.908392in}}{\pgfqpoint{2.188848in}{1.908392in}}%
\pgfpathclose%
\pgfusepath{stroke,fill}%
\end{pgfscope}%
\begin{pgfscope}%
\pgfpathrectangle{\pgfqpoint{0.100000in}{0.212622in}}{\pgfqpoint{3.696000in}{3.696000in}}%
\pgfusepath{clip}%
\pgfsetbuttcap%
\pgfsetroundjoin%
\definecolor{currentfill}{rgb}{0.121569,0.466667,0.705882}%
\pgfsetfillcolor{currentfill}%
\pgfsetfillopacity{0.976188}%
\pgfsetlinewidth{1.003750pt}%
\definecolor{currentstroke}{rgb}{0.121569,0.466667,0.705882}%
\pgfsetstrokecolor{currentstroke}%
\pgfsetstrokeopacity{0.976188}%
\pgfsetdash{}{0pt}%
\pgfpathmoveto{\pgfqpoint{2.207074in}{1.902040in}}%
\pgfpathcurveto{\pgfqpoint{2.215310in}{1.902040in}}{\pgfqpoint{2.223210in}{1.905312in}}{\pgfqpoint{2.229034in}{1.911136in}}%
\pgfpathcurveto{\pgfqpoint{2.234858in}{1.916960in}}{\pgfqpoint{2.238130in}{1.924860in}}{\pgfqpoint{2.238130in}{1.933096in}}%
\pgfpathcurveto{\pgfqpoint{2.238130in}{1.941332in}}{\pgfqpoint{2.234858in}{1.949233in}}{\pgfqpoint{2.229034in}{1.955056in}}%
\pgfpathcurveto{\pgfqpoint{2.223210in}{1.960880in}}{\pgfqpoint{2.215310in}{1.964153in}}{\pgfqpoint{2.207074in}{1.964153in}}%
\pgfpathcurveto{\pgfqpoint{2.198838in}{1.964153in}}{\pgfqpoint{2.190938in}{1.960880in}}{\pgfqpoint{2.185114in}{1.955056in}}%
\pgfpathcurveto{\pgfqpoint{2.179290in}{1.949233in}}{\pgfqpoint{2.176017in}{1.941332in}}{\pgfqpoint{2.176017in}{1.933096in}}%
\pgfpathcurveto{\pgfqpoint{2.176017in}{1.924860in}}{\pgfqpoint{2.179290in}{1.916960in}}{\pgfqpoint{2.185114in}{1.911136in}}%
\pgfpathcurveto{\pgfqpoint{2.190938in}{1.905312in}}{\pgfqpoint{2.198838in}{1.902040in}}{\pgfqpoint{2.207074in}{1.902040in}}%
\pgfpathclose%
\pgfusepath{stroke,fill}%
\end{pgfscope}%
\begin{pgfscope}%
\pgfpathrectangle{\pgfqpoint{0.100000in}{0.212622in}}{\pgfqpoint{3.696000in}{3.696000in}}%
\pgfusepath{clip}%
\pgfsetbuttcap%
\pgfsetroundjoin%
\definecolor{currentfill}{rgb}{0.121569,0.466667,0.705882}%
\pgfsetfillcolor{currentfill}%
\pgfsetfillopacity{0.976517}%
\pgfsetlinewidth{1.003750pt}%
\definecolor{currentstroke}{rgb}{0.121569,0.466667,0.705882}%
\pgfsetstrokecolor{currentstroke}%
\pgfsetstrokeopacity{0.976517}%
\pgfsetdash{}{0pt}%
\pgfpathmoveto{\pgfqpoint{2.448119in}{1.834659in}}%
\pgfpathcurveto{\pgfqpoint{2.456355in}{1.834659in}}{\pgfqpoint{2.464255in}{1.837931in}}{\pgfqpoint{2.470079in}{1.843755in}}%
\pgfpathcurveto{\pgfqpoint{2.475903in}{1.849579in}}{\pgfqpoint{2.479175in}{1.857479in}}{\pgfqpoint{2.479175in}{1.865715in}}%
\pgfpathcurveto{\pgfqpoint{2.479175in}{1.873951in}}{\pgfqpoint{2.475903in}{1.881851in}}{\pgfqpoint{2.470079in}{1.887675in}}%
\pgfpathcurveto{\pgfqpoint{2.464255in}{1.893499in}}{\pgfqpoint{2.456355in}{1.896772in}}{\pgfqpoint{2.448119in}{1.896772in}}%
\pgfpathcurveto{\pgfqpoint{2.439882in}{1.896772in}}{\pgfqpoint{2.431982in}{1.893499in}}{\pgfqpoint{2.426158in}{1.887675in}}%
\pgfpathcurveto{\pgfqpoint{2.420334in}{1.881851in}}{\pgfqpoint{2.417062in}{1.873951in}}{\pgfqpoint{2.417062in}{1.865715in}}%
\pgfpathcurveto{\pgfqpoint{2.417062in}{1.857479in}}{\pgfqpoint{2.420334in}{1.849579in}}{\pgfqpoint{2.426158in}{1.843755in}}%
\pgfpathcurveto{\pgfqpoint{2.431982in}{1.837931in}}{\pgfqpoint{2.439882in}{1.834659in}}{\pgfqpoint{2.448119in}{1.834659in}}%
\pgfpathclose%
\pgfusepath{stroke,fill}%
\end{pgfscope}%
\begin{pgfscope}%
\pgfpathrectangle{\pgfqpoint{0.100000in}{0.212622in}}{\pgfqpoint{3.696000in}{3.696000in}}%
\pgfusepath{clip}%
\pgfsetbuttcap%
\pgfsetroundjoin%
\definecolor{currentfill}{rgb}{0.121569,0.466667,0.705882}%
\pgfsetfillcolor{currentfill}%
\pgfsetfillopacity{0.979569}%
\pgfsetlinewidth{1.003750pt}%
\definecolor{currentstroke}{rgb}{0.121569,0.466667,0.705882}%
\pgfsetstrokecolor{currentstroke}%
\pgfsetstrokeopacity{0.979569}%
\pgfsetdash{}{0pt}%
\pgfpathmoveto{\pgfqpoint{2.439689in}{1.830584in}}%
\pgfpathcurveto{\pgfqpoint{2.447926in}{1.830584in}}{\pgfqpoint{2.455826in}{1.833856in}}{\pgfqpoint{2.461650in}{1.839680in}}%
\pgfpathcurveto{\pgfqpoint{2.467473in}{1.845504in}}{\pgfqpoint{2.470746in}{1.853404in}}{\pgfqpoint{2.470746in}{1.861640in}}%
\pgfpathcurveto{\pgfqpoint{2.470746in}{1.869877in}}{\pgfqpoint{2.467473in}{1.877777in}}{\pgfqpoint{2.461650in}{1.883601in}}%
\pgfpathcurveto{\pgfqpoint{2.455826in}{1.889425in}}{\pgfqpoint{2.447926in}{1.892697in}}{\pgfqpoint{2.439689in}{1.892697in}}%
\pgfpathcurveto{\pgfqpoint{2.431453in}{1.892697in}}{\pgfqpoint{2.423553in}{1.889425in}}{\pgfqpoint{2.417729in}{1.883601in}}%
\pgfpathcurveto{\pgfqpoint{2.411905in}{1.877777in}}{\pgfqpoint{2.408633in}{1.869877in}}{\pgfqpoint{2.408633in}{1.861640in}}%
\pgfpathcurveto{\pgfqpoint{2.408633in}{1.853404in}}{\pgfqpoint{2.411905in}{1.845504in}}{\pgfqpoint{2.417729in}{1.839680in}}%
\pgfpathcurveto{\pgfqpoint{2.423553in}{1.833856in}}{\pgfqpoint{2.431453in}{1.830584in}}{\pgfqpoint{2.439689in}{1.830584in}}%
\pgfpathclose%
\pgfusepath{stroke,fill}%
\end{pgfscope}%
\begin{pgfscope}%
\pgfpathrectangle{\pgfqpoint{0.100000in}{0.212622in}}{\pgfqpoint{3.696000in}{3.696000in}}%
\pgfusepath{clip}%
\pgfsetbuttcap%
\pgfsetroundjoin%
\definecolor{currentfill}{rgb}{0.121569,0.466667,0.705882}%
\pgfsetfillcolor{currentfill}%
\pgfsetfillopacity{0.981423}%
\pgfsetlinewidth{1.003750pt}%
\definecolor{currentstroke}{rgb}{0.121569,0.466667,0.705882}%
\pgfsetstrokecolor{currentstroke}%
\pgfsetstrokeopacity{0.981423}%
\pgfsetdash{}{0pt}%
\pgfpathmoveto{\pgfqpoint{2.435957in}{1.828264in}}%
\pgfpathcurveto{\pgfqpoint{2.444193in}{1.828264in}}{\pgfqpoint{2.452093in}{1.831537in}}{\pgfqpoint{2.457917in}{1.837361in}}%
\pgfpathcurveto{\pgfqpoint{2.463741in}{1.843185in}}{\pgfqpoint{2.467014in}{1.851085in}}{\pgfqpoint{2.467014in}{1.859321in}}%
\pgfpathcurveto{\pgfqpoint{2.467014in}{1.867557in}}{\pgfqpoint{2.463741in}{1.875457in}}{\pgfqpoint{2.457917in}{1.881281in}}%
\pgfpathcurveto{\pgfqpoint{2.452093in}{1.887105in}}{\pgfqpoint{2.444193in}{1.890377in}}{\pgfqpoint{2.435957in}{1.890377in}}%
\pgfpathcurveto{\pgfqpoint{2.427721in}{1.890377in}}{\pgfqpoint{2.419821in}{1.887105in}}{\pgfqpoint{2.413997in}{1.881281in}}%
\pgfpathcurveto{\pgfqpoint{2.408173in}{1.875457in}}{\pgfqpoint{2.404901in}{1.867557in}}{\pgfqpoint{2.404901in}{1.859321in}}%
\pgfpathcurveto{\pgfqpoint{2.404901in}{1.851085in}}{\pgfqpoint{2.408173in}{1.843185in}}{\pgfqpoint{2.413997in}{1.837361in}}%
\pgfpathcurveto{\pgfqpoint{2.419821in}{1.831537in}}{\pgfqpoint{2.427721in}{1.828264in}}{\pgfqpoint{2.435957in}{1.828264in}}%
\pgfpathclose%
\pgfusepath{stroke,fill}%
\end{pgfscope}%
\begin{pgfscope}%
\pgfpathrectangle{\pgfqpoint{0.100000in}{0.212622in}}{\pgfqpoint{3.696000in}{3.696000in}}%
\pgfusepath{clip}%
\pgfsetbuttcap%
\pgfsetroundjoin%
\definecolor{currentfill}{rgb}{0.121569,0.466667,0.705882}%
\pgfsetfillcolor{currentfill}%
\pgfsetfillopacity{0.981828}%
\pgfsetlinewidth{1.003750pt}%
\definecolor{currentstroke}{rgb}{0.121569,0.466667,0.705882}%
\pgfsetstrokecolor{currentstroke}%
\pgfsetstrokeopacity{0.981828}%
\pgfsetdash{}{0pt}%
\pgfpathmoveto{\pgfqpoint{2.236776in}{1.889152in}}%
\pgfpathcurveto{\pgfqpoint{2.245012in}{1.889152in}}{\pgfqpoint{2.252912in}{1.892424in}}{\pgfqpoint{2.258736in}{1.898248in}}%
\pgfpathcurveto{\pgfqpoint{2.264560in}{1.904072in}}{\pgfqpoint{2.267833in}{1.911972in}}{\pgfqpoint{2.267833in}{1.920208in}}%
\pgfpathcurveto{\pgfqpoint{2.267833in}{1.928445in}}{\pgfqpoint{2.264560in}{1.936345in}}{\pgfqpoint{2.258736in}{1.942169in}}%
\pgfpathcurveto{\pgfqpoint{2.252912in}{1.947993in}}{\pgfqpoint{2.245012in}{1.951265in}}{\pgfqpoint{2.236776in}{1.951265in}}%
\pgfpathcurveto{\pgfqpoint{2.228540in}{1.951265in}}{\pgfqpoint{2.220640in}{1.947993in}}{\pgfqpoint{2.214816in}{1.942169in}}%
\pgfpathcurveto{\pgfqpoint{2.208992in}{1.936345in}}{\pgfqpoint{2.205720in}{1.928445in}}{\pgfqpoint{2.205720in}{1.920208in}}%
\pgfpathcurveto{\pgfqpoint{2.205720in}{1.911972in}}{\pgfqpoint{2.208992in}{1.904072in}}{\pgfqpoint{2.214816in}{1.898248in}}%
\pgfpathcurveto{\pgfqpoint{2.220640in}{1.892424in}}{\pgfqpoint{2.228540in}{1.889152in}}{\pgfqpoint{2.236776in}{1.889152in}}%
\pgfpathclose%
\pgfusepath{stroke,fill}%
\end{pgfscope}%
\begin{pgfscope}%
\pgfpathrectangle{\pgfqpoint{0.100000in}{0.212622in}}{\pgfqpoint{3.696000in}{3.696000in}}%
\pgfusepath{clip}%
\pgfsetbuttcap%
\pgfsetroundjoin%
\definecolor{currentfill}{rgb}{0.121569,0.466667,0.705882}%
\pgfsetfillcolor{currentfill}%
\pgfsetfillopacity{0.982305}%
\pgfsetlinewidth{1.003750pt}%
\definecolor{currentstroke}{rgb}{0.121569,0.466667,0.705882}%
\pgfsetstrokecolor{currentstroke}%
\pgfsetstrokeopacity{0.982305}%
\pgfsetdash{}{0pt}%
\pgfpathmoveto{\pgfqpoint{2.433216in}{1.827093in}}%
\pgfpathcurveto{\pgfqpoint{2.441453in}{1.827093in}}{\pgfqpoint{2.449353in}{1.830365in}}{\pgfqpoint{2.455177in}{1.836189in}}%
\pgfpathcurveto{\pgfqpoint{2.461001in}{1.842013in}}{\pgfqpoint{2.464273in}{1.849913in}}{\pgfqpoint{2.464273in}{1.858149in}}%
\pgfpathcurveto{\pgfqpoint{2.464273in}{1.866385in}}{\pgfqpoint{2.461001in}{1.874285in}}{\pgfqpoint{2.455177in}{1.880109in}}%
\pgfpathcurveto{\pgfqpoint{2.449353in}{1.885933in}}{\pgfqpoint{2.441453in}{1.889206in}}{\pgfqpoint{2.433216in}{1.889206in}}%
\pgfpathcurveto{\pgfqpoint{2.424980in}{1.889206in}}{\pgfqpoint{2.417080in}{1.885933in}}{\pgfqpoint{2.411256in}{1.880109in}}%
\pgfpathcurveto{\pgfqpoint{2.405432in}{1.874285in}}{\pgfqpoint{2.402160in}{1.866385in}}{\pgfqpoint{2.402160in}{1.858149in}}%
\pgfpathcurveto{\pgfqpoint{2.402160in}{1.849913in}}{\pgfqpoint{2.405432in}{1.842013in}}{\pgfqpoint{2.411256in}{1.836189in}}%
\pgfpathcurveto{\pgfqpoint{2.417080in}{1.830365in}}{\pgfqpoint{2.424980in}{1.827093in}}{\pgfqpoint{2.433216in}{1.827093in}}%
\pgfpathclose%
\pgfusepath{stroke,fill}%
\end{pgfscope}%
\begin{pgfscope}%
\pgfpathrectangle{\pgfqpoint{0.100000in}{0.212622in}}{\pgfqpoint{3.696000in}{3.696000in}}%
\pgfusepath{clip}%
\pgfsetbuttcap%
\pgfsetroundjoin%
\definecolor{currentfill}{rgb}{0.121569,0.466667,0.705882}%
\pgfsetfillcolor{currentfill}%
\pgfsetfillopacity{0.982941}%
\pgfsetlinewidth{1.003750pt}%
\definecolor{currentstroke}{rgb}{0.121569,0.466667,0.705882}%
\pgfsetstrokecolor{currentstroke}%
\pgfsetstrokeopacity{0.982941}%
\pgfsetdash{}{0pt}%
\pgfpathmoveto{\pgfqpoint{2.432073in}{1.826866in}}%
\pgfpathcurveto{\pgfqpoint{2.440309in}{1.826866in}}{\pgfqpoint{2.448209in}{1.830138in}}{\pgfqpoint{2.454033in}{1.835962in}}%
\pgfpathcurveto{\pgfqpoint{2.459857in}{1.841786in}}{\pgfqpoint{2.463130in}{1.849686in}}{\pgfqpoint{2.463130in}{1.857922in}}%
\pgfpathcurveto{\pgfqpoint{2.463130in}{1.866159in}}{\pgfqpoint{2.459857in}{1.874059in}}{\pgfqpoint{2.454033in}{1.879883in}}%
\pgfpathcurveto{\pgfqpoint{2.448209in}{1.885707in}}{\pgfqpoint{2.440309in}{1.888979in}}{\pgfqpoint{2.432073in}{1.888979in}}%
\pgfpathcurveto{\pgfqpoint{2.423837in}{1.888979in}}{\pgfqpoint{2.415937in}{1.885707in}}{\pgfqpoint{2.410113in}{1.879883in}}%
\pgfpathcurveto{\pgfqpoint{2.404289in}{1.874059in}}{\pgfqpoint{2.401017in}{1.866159in}}{\pgfqpoint{2.401017in}{1.857922in}}%
\pgfpathcurveto{\pgfqpoint{2.401017in}{1.849686in}}{\pgfqpoint{2.404289in}{1.841786in}}{\pgfqpoint{2.410113in}{1.835962in}}%
\pgfpathcurveto{\pgfqpoint{2.415937in}{1.830138in}}{\pgfqpoint{2.423837in}{1.826866in}}{\pgfqpoint{2.432073in}{1.826866in}}%
\pgfpathclose%
\pgfusepath{stroke,fill}%
\end{pgfscope}%
\begin{pgfscope}%
\pgfpathrectangle{\pgfqpoint{0.100000in}{0.212622in}}{\pgfqpoint{3.696000in}{3.696000in}}%
\pgfusepath{clip}%
\pgfsetbuttcap%
\pgfsetroundjoin%
\definecolor{currentfill}{rgb}{0.121569,0.466667,0.705882}%
\pgfsetfillcolor{currentfill}%
\pgfsetfillopacity{0.983228}%
\pgfsetlinewidth{1.003750pt}%
\definecolor{currentstroke}{rgb}{0.121569,0.466667,0.705882}%
\pgfsetstrokecolor{currentstroke}%
\pgfsetstrokeopacity{0.983228}%
\pgfsetdash{}{0pt}%
\pgfpathmoveto{\pgfqpoint{2.431457in}{1.826336in}}%
\pgfpathcurveto{\pgfqpoint{2.439694in}{1.826336in}}{\pgfqpoint{2.447594in}{1.829608in}}{\pgfqpoint{2.453418in}{1.835432in}}%
\pgfpathcurveto{\pgfqpoint{2.459241in}{1.841256in}}{\pgfqpoint{2.462514in}{1.849156in}}{\pgfqpoint{2.462514in}{1.857393in}}%
\pgfpathcurveto{\pgfqpoint{2.462514in}{1.865629in}}{\pgfqpoint{2.459241in}{1.873529in}}{\pgfqpoint{2.453418in}{1.879353in}}%
\pgfpathcurveto{\pgfqpoint{2.447594in}{1.885177in}}{\pgfqpoint{2.439694in}{1.888449in}}{\pgfqpoint{2.431457in}{1.888449in}}%
\pgfpathcurveto{\pgfqpoint{2.423221in}{1.888449in}}{\pgfqpoint{2.415321in}{1.885177in}}{\pgfqpoint{2.409497in}{1.879353in}}%
\pgfpathcurveto{\pgfqpoint{2.403673in}{1.873529in}}{\pgfqpoint{2.400401in}{1.865629in}}{\pgfqpoint{2.400401in}{1.857393in}}%
\pgfpathcurveto{\pgfqpoint{2.400401in}{1.849156in}}{\pgfqpoint{2.403673in}{1.841256in}}{\pgfqpoint{2.409497in}{1.835432in}}%
\pgfpathcurveto{\pgfqpoint{2.415321in}{1.829608in}}{\pgfqpoint{2.423221in}{1.826336in}}{\pgfqpoint{2.431457in}{1.826336in}}%
\pgfpathclose%
\pgfusepath{stroke,fill}%
\end{pgfscope}%
\begin{pgfscope}%
\pgfpathrectangle{\pgfqpoint{0.100000in}{0.212622in}}{\pgfqpoint{3.696000in}{3.696000in}}%
\pgfusepath{clip}%
\pgfsetbuttcap%
\pgfsetroundjoin%
\definecolor{currentfill}{rgb}{0.121569,0.466667,0.705882}%
\pgfsetfillcolor{currentfill}%
\pgfsetfillopacity{0.983408}%
\pgfsetlinewidth{1.003750pt}%
\definecolor{currentstroke}{rgb}{0.121569,0.466667,0.705882}%
\pgfsetstrokecolor{currentstroke}%
\pgfsetstrokeopacity{0.983408}%
\pgfsetdash{}{0pt}%
\pgfpathmoveto{\pgfqpoint{2.431081in}{1.826221in}}%
\pgfpathcurveto{\pgfqpoint{2.439317in}{1.826221in}}{\pgfqpoint{2.447217in}{1.829494in}}{\pgfqpoint{2.453041in}{1.835317in}}%
\pgfpathcurveto{\pgfqpoint{2.458865in}{1.841141in}}{\pgfqpoint{2.462138in}{1.849041in}}{\pgfqpoint{2.462138in}{1.857278in}}%
\pgfpathcurveto{\pgfqpoint{2.462138in}{1.865514in}}{\pgfqpoint{2.458865in}{1.873414in}}{\pgfqpoint{2.453041in}{1.879238in}}%
\pgfpathcurveto{\pgfqpoint{2.447217in}{1.885062in}}{\pgfqpoint{2.439317in}{1.888334in}}{\pgfqpoint{2.431081in}{1.888334in}}%
\pgfpathcurveto{\pgfqpoint{2.422845in}{1.888334in}}{\pgfqpoint{2.414945in}{1.885062in}}{\pgfqpoint{2.409121in}{1.879238in}}%
\pgfpathcurveto{\pgfqpoint{2.403297in}{1.873414in}}{\pgfqpoint{2.400025in}{1.865514in}}{\pgfqpoint{2.400025in}{1.857278in}}%
\pgfpathcurveto{\pgfqpoint{2.400025in}{1.849041in}}{\pgfqpoint{2.403297in}{1.841141in}}{\pgfqpoint{2.409121in}{1.835317in}}%
\pgfpathcurveto{\pgfqpoint{2.414945in}{1.829494in}}{\pgfqpoint{2.422845in}{1.826221in}}{\pgfqpoint{2.431081in}{1.826221in}}%
\pgfpathclose%
\pgfusepath{stroke,fill}%
\end{pgfscope}%
\begin{pgfscope}%
\pgfpathrectangle{\pgfqpoint{0.100000in}{0.212622in}}{\pgfqpoint{3.696000in}{3.696000in}}%
\pgfusepath{clip}%
\pgfsetbuttcap%
\pgfsetroundjoin%
\definecolor{currentfill}{rgb}{0.121569,0.466667,0.705882}%
\pgfsetfillcolor{currentfill}%
\pgfsetfillopacity{0.984741}%
\pgfsetlinewidth{1.003750pt}%
\definecolor{currentstroke}{rgb}{0.121569,0.466667,0.705882}%
\pgfsetstrokecolor{currentstroke}%
\pgfsetstrokeopacity{0.984741}%
\pgfsetdash{}{0pt}%
\pgfpathmoveto{\pgfqpoint{2.428622in}{1.824010in}}%
\pgfpathcurveto{\pgfqpoint{2.436858in}{1.824010in}}{\pgfqpoint{2.444759in}{1.827282in}}{\pgfqpoint{2.450582in}{1.833106in}}%
\pgfpathcurveto{\pgfqpoint{2.456406in}{1.838930in}}{\pgfqpoint{2.459679in}{1.846830in}}{\pgfqpoint{2.459679in}{1.855066in}}%
\pgfpathcurveto{\pgfqpoint{2.459679in}{1.863303in}}{\pgfqpoint{2.456406in}{1.871203in}}{\pgfqpoint{2.450582in}{1.877027in}}%
\pgfpathcurveto{\pgfqpoint{2.444759in}{1.882851in}}{\pgfqpoint{2.436858in}{1.886123in}}{\pgfqpoint{2.428622in}{1.886123in}}%
\pgfpathcurveto{\pgfqpoint{2.420386in}{1.886123in}}{\pgfqpoint{2.412486in}{1.882851in}}{\pgfqpoint{2.406662in}{1.877027in}}%
\pgfpathcurveto{\pgfqpoint{2.400838in}{1.871203in}}{\pgfqpoint{2.397566in}{1.863303in}}{\pgfqpoint{2.397566in}{1.855066in}}%
\pgfpathcurveto{\pgfqpoint{2.397566in}{1.846830in}}{\pgfqpoint{2.400838in}{1.838930in}}{\pgfqpoint{2.406662in}{1.833106in}}%
\pgfpathcurveto{\pgfqpoint{2.412486in}{1.827282in}}{\pgfqpoint{2.420386in}{1.824010in}}{\pgfqpoint{2.428622in}{1.824010in}}%
\pgfpathclose%
\pgfusepath{stroke,fill}%
\end{pgfscope}%
\begin{pgfscope}%
\pgfpathrectangle{\pgfqpoint{0.100000in}{0.212622in}}{\pgfqpoint{3.696000in}{3.696000in}}%
\pgfusepath{clip}%
\pgfsetbuttcap%
\pgfsetroundjoin%
\definecolor{currentfill}{rgb}{0.121569,0.466667,0.705882}%
\pgfsetfillcolor{currentfill}%
\pgfsetfillopacity{0.984915}%
\pgfsetlinewidth{1.003750pt}%
\definecolor{currentstroke}{rgb}{0.121569,0.466667,0.705882}%
\pgfsetstrokecolor{currentstroke}%
\pgfsetstrokeopacity{0.984915}%
\pgfsetdash{}{0pt}%
\pgfpathmoveto{\pgfqpoint{2.262824in}{1.874191in}}%
\pgfpathcurveto{\pgfqpoint{2.271061in}{1.874191in}}{\pgfqpoint{2.278961in}{1.877463in}}{\pgfqpoint{2.284785in}{1.883287in}}%
\pgfpathcurveto{\pgfqpoint{2.290609in}{1.889111in}}{\pgfqpoint{2.293881in}{1.897011in}}{\pgfqpoint{2.293881in}{1.905248in}}%
\pgfpathcurveto{\pgfqpoint{2.293881in}{1.913484in}}{\pgfqpoint{2.290609in}{1.921384in}}{\pgfqpoint{2.284785in}{1.927208in}}%
\pgfpathcurveto{\pgfqpoint{2.278961in}{1.933032in}}{\pgfqpoint{2.271061in}{1.936304in}}{\pgfqpoint{2.262824in}{1.936304in}}%
\pgfpathcurveto{\pgfqpoint{2.254588in}{1.936304in}}{\pgfqpoint{2.246688in}{1.933032in}}{\pgfqpoint{2.240864in}{1.927208in}}%
\pgfpathcurveto{\pgfqpoint{2.235040in}{1.921384in}}{\pgfqpoint{2.231768in}{1.913484in}}{\pgfqpoint{2.231768in}{1.905248in}}%
\pgfpathcurveto{\pgfqpoint{2.231768in}{1.897011in}}{\pgfqpoint{2.235040in}{1.889111in}}{\pgfqpoint{2.240864in}{1.883287in}}%
\pgfpathcurveto{\pgfqpoint{2.246688in}{1.877463in}}{\pgfqpoint{2.254588in}{1.874191in}}{\pgfqpoint{2.262824in}{1.874191in}}%
\pgfpathclose%
\pgfusepath{stroke,fill}%
\end{pgfscope}%
\begin{pgfscope}%
\pgfpathrectangle{\pgfqpoint{0.100000in}{0.212622in}}{\pgfqpoint{3.696000in}{3.696000in}}%
\pgfusepath{clip}%
\pgfsetbuttcap%
\pgfsetroundjoin%
\definecolor{currentfill}{rgb}{0.121569,0.466667,0.705882}%
\pgfsetfillcolor{currentfill}%
\pgfsetfillopacity{0.985467}%
\pgfsetlinewidth{1.003750pt}%
\definecolor{currentstroke}{rgb}{0.121569,0.466667,0.705882}%
\pgfsetstrokecolor{currentstroke}%
\pgfsetstrokeopacity{0.985467}%
\pgfsetdash{}{0pt}%
\pgfpathmoveto{\pgfqpoint{2.426848in}{1.823222in}}%
\pgfpathcurveto{\pgfqpoint{2.435085in}{1.823222in}}{\pgfqpoint{2.442985in}{1.826494in}}{\pgfqpoint{2.448809in}{1.832318in}}%
\pgfpathcurveto{\pgfqpoint{2.454633in}{1.838142in}}{\pgfqpoint{2.457905in}{1.846042in}}{\pgfqpoint{2.457905in}{1.854279in}}%
\pgfpathcurveto{\pgfqpoint{2.457905in}{1.862515in}}{\pgfqpoint{2.454633in}{1.870415in}}{\pgfqpoint{2.448809in}{1.876239in}}%
\pgfpathcurveto{\pgfqpoint{2.442985in}{1.882063in}}{\pgfqpoint{2.435085in}{1.885335in}}{\pgfqpoint{2.426848in}{1.885335in}}%
\pgfpathcurveto{\pgfqpoint{2.418612in}{1.885335in}}{\pgfqpoint{2.410712in}{1.882063in}}{\pgfqpoint{2.404888in}{1.876239in}}%
\pgfpathcurveto{\pgfqpoint{2.399064in}{1.870415in}}{\pgfqpoint{2.395792in}{1.862515in}}{\pgfqpoint{2.395792in}{1.854279in}}%
\pgfpathcurveto{\pgfqpoint{2.395792in}{1.846042in}}{\pgfqpoint{2.399064in}{1.838142in}}{\pgfqpoint{2.404888in}{1.832318in}}%
\pgfpathcurveto{\pgfqpoint{2.410712in}{1.826494in}}{\pgfqpoint{2.418612in}{1.823222in}}{\pgfqpoint{2.426848in}{1.823222in}}%
\pgfpathclose%
\pgfusepath{stroke,fill}%
\end{pgfscope}%
\begin{pgfscope}%
\pgfpathrectangle{\pgfqpoint{0.100000in}{0.212622in}}{\pgfqpoint{3.696000in}{3.696000in}}%
\pgfusepath{clip}%
\pgfsetbuttcap%
\pgfsetroundjoin%
\definecolor{currentfill}{rgb}{0.121569,0.466667,0.705882}%
\pgfsetfillcolor{currentfill}%
\pgfsetfillopacity{0.987051}%
\pgfsetlinewidth{1.003750pt}%
\definecolor{currentstroke}{rgb}{0.121569,0.466667,0.705882}%
\pgfsetstrokecolor{currentstroke}%
\pgfsetstrokeopacity{0.987051}%
\pgfsetdash{}{0pt}%
\pgfpathmoveto{\pgfqpoint{2.423420in}{1.821087in}}%
\pgfpathcurveto{\pgfqpoint{2.431657in}{1.821087in}}{\pgfqpoint{2.439557in}{1.824360in}}{\pgfqpoint{2.445380in}{1.830184in}}%
\pgfpathcurveto{\pgfqpoint{2.451204in}{1.836008in}}{\pgfqpoint{2.454477in}{1.843908in}}{\pgfqpoint{2.454477in}{1.852144in}}%
\pgfpathcurveto{\pgfqpoint{2.454477in}{1.860380in}}{\pgfqpoint{2.451204in}{1.868280in}}{\pgfqpoint{2.445380in}{1.874104in}}%
\pgfpathcurveto{\pgfqpoint{2.439557in}{1.879928in}}{\pgfqpoint{2.431657in}{1.883200in}}{\pgfqpoint{2.423420in}{1.883200in}}%
\pgfpathcurveto{\pgfqpoint{2.415184in}{1.883200in}}{\pgfqpoint{2.407284in}{1.879928in}}{\pgfqpoint{2.401460in}{1.874104in}}%
\pgfpathcurveto{\pgfqpoint{2.395636in}{1.868280in}}{\pgfqpoint{2.392364in}{1.860380in}}{\pgfqpoint{2.392364in}{1.852144in}}%
\pgfpathcurveto{\pgfqpoint{2.392364in}{1.843908in}}{\pgfqpoint{2.395636in}{1.836008in}}{\pgfqpoint{2.401460in}{1.830184in}}%
\pgfpathcurveto{\pgfqpoint{2.407284in}{1.824360in}}{\pgfqpoint{2.415184in}{1.821087in}}{\pgfqpoint{2.423420in}{1.821087in}}%
\pgfpathclose%
\pgfusepath{stroke,fill}%
\end{pgfscope}%
\begin{pgfscope}%
\pgfpathrectangle{\pgfqpoint{0.100000in}{0.212622in}}{\pgfqpoint{3.696000in}{3.696000in}}%
\pgfusepath{clip}%
\pgfsetbuttcap%
\pgfsetroundjoin%
\definecolor{currentfill}{rgb}{0.121569,0.466667,0.705882}%
\pgfsetfillcolor{currentfill}%
\pgfsetfillopacity{0.987795}%
\pgfsetlinewidth{1.003750pt}%
\definecolor{currentstroke}{rgb}{0.121569,0.466667,0.705882}%
\pgfsetstrokecolor{currentstroke}%
\pgfsetstrokeopacity{0.987795}%
\pgfsetdash{}{0pt}%
\pgfpathmoveto{\pgfqpoint{2.420954in}{1.819965in}}%
\pgfpathcurveto{\pgfqpoint{2.429191in}{1.819965in}}{\pgfqpoint{2.437091in}{1.823237in}}{\pgfqpoint{2.442915in}{1.829061in}}%
\pgfpathcurveto{\pgfqpoint{2.448739in}{1.834885in}}{\pgfqpoint{2.452011in}{1.842785in}}{\pgfqpoint{2.452011in}{1.851021in}}%
\pgfpathcurveto{\pgfqpoint{2.452011in}{1.859258in}}{\pgfqpoint{2.448739in}{1.867158in}}{\pgfqpoint{2.442915in}{1.872982in}}%
\pgfpathcurveto{\pgfqpoint{2.437091in}{1.878806in}}{\pgfqpoint{2.429191in}{1.882078in}}{\pgfqpoint{2.420954in}{1.882078in}}%
\pgfpathcurveto{\pgfqpoint{2.412718in}{1.882078in}}{\pgfqpoint{2.404818in}{1.878806in}}{\pgfqpoint{2.398994in}{1.872982in}}%
\pgfpathcurveto{\pgfqpoint{2.393170in}{1.867158in}}{\pgfqpoint{2.389898in}{1.859258in}}{\pgfqpoint{2.389898in}{1.851021in}}%
\pgfpathcurveto{\pgfqpoint{2.389898in}{1.842785in}}{\pgfqpoint{2.393170in}{1.834885in}}{\pgfqpoint{2.398994in}{1.829061in}}%
\pgfpathcurveto{\pgfqpoint{2.404818in}{1.823237in}}{\pgfqpoint{2.412718in}{1.819965in}}{\pgfqpoint{2.420954in}{1.819965in}}%
\pgfpathclose%
\pgfusepath{stroke,fill}%
\end{pgfscope}%
\begin{pgfscope}%
\pgfpathrectangle{\pgfqpoint{0.100000in}{0.212622in}}{\pgfqpoint{3.696000in}{3.696000in}}%
\pgfusepath{clip}%
\pgfsetbuttcap%
\pgfsetroundjoin%
\definecolor{currentfill}{rgb}{0.121569,0.466667,0.705882}%
\pgfsetfillcolor{currentfill}%
\pgfsetfillopacity{0.988292}%
\pgfsetlinewidth{1.003750pt}%
\definecolor{currentstroke}{rgb}{0.121569,0.466667,0.705882}%
\pgfsetstrokecolor{currentstroke}%
\pgfsetstrokeopacity{0.988292}%
\pgfsetdash{}{0pt}%
\pgfpathmoveto{\pgfqpoint{2.283786in}{1.864520in}}%
\pgfpathcurveto{\pgfqpoint{2.292022in}{1.864520in}}{\pgfqpoint{2.299922in}{1.867792in}}{\pgfqpoint{2.305746in}{1.873616in}}%
\pgfpathcurveto{\pgfqpoint{2.311570in}{1.879440in}}{\pgfqpoint{2.314842in}{1.887340in}}{\pgfqpoint{2.314842in}{1.895577in}}%
\pgfpathcurveto{\pgfqpoint{2.314842in}{1.903813in}}{\pgfqpoint{2.311570in}{1.911713in}}{\pgfqpoint{2.305746in}{1.917537in}}%
\pgfpathcurveto{\pgfqpoint{2.299922in}{1.923361in}}{\pgfqpoint{2.292022in}{1.926633in}}{\pgfqpoint{2.283786in}{1.926633in}}%
\pgfpathcurveto{\pgfqpoint{2.275550in}{1.926633in}}{\pgfqpoint{2.267650in}{1.923361in}}{\pgfqpoint{2.261826in}{1.917537in}}%
\pgfpathcurveto{\pgfqpoint{2.256002in}{1.911713in}}{\pgfqpoint{2.252729in}{1.903813in}}{\pgfqpoint{2.252729in}{1.895577in}}%
\pgfpathcurveto{\pgfqpoint{2.252729in}{1.887340in}}{\pgfqpoint{2.256002in}{1.879440in}}{\pgfqpoint{2.261826in}{1.873616in}}%
\pgfpathcurveto{\pgfqpoint{2.267650in}{1.867792in}}{\pgfqpoint{2.275550in}{1.864520in}}{\pgfqpoint{2.283786in}{1.864520in}}%
\pgfpathclose%
\pgfusepath{stroke,fill}%
\end{pgfscope}%
\begin{pgfscope}%
\pgfpathrectangle{\pgfqpoint{0.100000in}{0.212622in}}{\pgfqpoint{3.696000in}{3.696000in}}%
\pgfusepath{clip}%
\pgfsetbuttcap%
\pgfsetroundjoin%
\definecolor{currentfill}{rgb}{0.121569,0.466667,0.705882}%
\pgfsetfillcolor{currentfill}%
\pgfsetfillopacity{0.988352}%
\pgfsetlinewidth{1.003750pt}%
\definecolor{currentstroke}{rgb}{0.121569,0.466667,0.705882}%
\pgfsetstrokecolor{currentstroke}%
\pgfsetstrokeopacity{0.988352}%
\pgfsetdash{}{0pt}%
\pgfpathmoveto{\pgfqpoint{2.419889in}{1.819823in}}%
\pgfpathcurveto{\pgfqpoint{2.428126in}{1.819823in}}{\pgfqpoint{2.436026in}{1.823095in}}{\pgfqpoint{2.441850in}{1.828919in}}%
\pgfpathcurveto{\pgfqpoint{2.447674in}{1.834743in}}{\pgfqpoint{2.450946in}{1.842643in}}{\pgfqpoint{2.450946in}{1.850879in}}%
\pgfpathcurveto{\pgfqpoint{2.450946in}{1.859115in}}{\pgfqpoint{2.447674in}{1.867015in}}{\pgfqpoint{2.441850in}{1.872839in}}%
\pgfpathcurveto{\pgfqpoint{2.436026in}{1.878663in}}{\pgfqpoint{2.428126in}{1.881936in}}{\pgfqpoint{2.419889in}{1.881936in}}%
\pgfpathcurveto{\pgfqpoint{2.411653in}{1.881936in}}{\pgfqpoint{2.403753in}{1.878663in}}{\pgfqpoint{2.397929in}{1.872839in}}%
\pgfpathcurveto{\pgfqpoint{2.392105in}{1.867015in}}{\pgfqpoint{2.388833in}{1.859115in}}{\pgfqpoint{2.388833in}{1.850879in}}%
\pgfpathcurveto{\pgfqpoint{2.388833in}{1.842643in}}{\pgfqpoint{2.392105in}{1.834743in}}{\pgfqpoint{2.397929in}{1.828919in}}%
\pgfpathcurveto{\pgfqpoint{2.403753in}{1.823095in}}{\pgfqpoint{2.411653in}{1.819823in}}{\pgfqpoint{2.419889in}{1.819823in}}%
\pgfpathclose%
\pgfusepath{stroke,fill}%
\end{pgfscope}%
\begin{pgfscope}%
\pgfpathrectangle{\pgfqpoint{0.100000in}{0.212622in}}{\pgfqpoint{3.696000in}{3.696000in}}%
\pgfusepath{clip}%
\pgfsetbuttcap%
\pgfsetroundjoin%
\definecolor{currentfill}{rgb}{0.121569,0.466667,0.705882}%
\pgfsetfillcolor{currentfill}%
\pgfsetfillopacity{0.988609}%
\pgfsetlinewidth{1.003750pt}%
\definecolor{currentstroke}{rgb}{0.121569,0.466667,0.705882}%
\pgfsetstrokecolor{currentstroke}%
\pgfsetstrokeopacity{0.988609}%
\pgfsetdash{}{0pt}%
\pgfpathmoveto{\pgfqpoint{2.419255in}{1.819490in}}%
\pgfpathcurveto{\pgfqpoint{2.427491in}{1.819490in}}{\pgfqpoint{2.435391in}{1.822763in}}{\pgfqpoint{2.441215in}{1.828586in}}%
\pgfpathcurveto{\pgfqpoint{2.447039in}{1.834410in}}{\pgfqpoint{2.450311in}{1.842310in}}{\pgfqpoint{2.450311in}{1.850547in}}%
\pgfpathcurveto{\pgfqpoint{2.450311in}{1.858783in}}{\pgfqpoint{2.447039in}{1.866683in}}{\pgfqpoint{2.441215in}{1.872507in}}%
\pgfpathcurveto{\pgfqpoint{2.435391in}{1.878331in}}{\pgfqpoint{2.427491in}{1.881603in}}{\pgfqpoint{2.419255in}{1.881603in}}%
\pgfpathcurveto{\pgfqpoint{2.411018in}{1.881603in}}{\pgfqpoint{2.403118in}{1.878331in}}{\pgfqpoint{2.397295in}{1.872507in}}%
\pgfpathcurveto{\pgfqpoint{2.391471in}{1.866683in}}{\pgfqpoint{2.388198in}{1.858783in}}{\pgfqpoint{2.388198in}{1.850547in}}%
\pgfpathcurveto{\pgfqpoint{2.388198in}{1.842310in}}{\pgfqpoint{2.391471in}{1.834410in}}{\pgfqpoint{2.397295in}{1.828586in}}%
\pgfpathcurveto{\pgfqpoint{2.403118in}{1.822763in}}{\pgfqpoint{2.411018in}{1.819490in}}{\pgfqpoint{2.419255in}{1.819490in}}%
\pgfpathclose%
\pgfusepath{stroke,fill}%
\end{pgfscope}%
\begin{pgfscope}%
\pgfpathrectangle{\pgfqpoint{0.100000in}{0.212622in}}{\pgfqpoint{3.696000in}{3.696000in}}%
\pgfusepath{clip}%
\pgfsetbuttcap%
\pgfsetroundjoin%
\definecolor{currentfill}{rgb}{0.121569,0.466667,0.705882}%
\pgfsetfillcolor{currentfill}%
\pgfsetfillopacity{0.988769}%
\pgfsetlinewidth{1.003750pt}%
\definecolor{currentstroke}{rgb}{0.121569,0.466667,0.705882}%
\pgfsetstrokecolor{currentstroke}%
\pgfsetstrokeopacity{0.988769}%
\pgfsetdash{}{0pt}%
\pgfpathmoveto{\pgfqpoint{2.418912in}{1.819420in}}%
\pgfpathcurveto{\pgfqpoint{2.427148in}{1.819420in}}{\pgfqpoint{2.435048in}{1.822692in}}{\pgfqpoint{2.440872in}{1.828516in}}%
\pgfpathcurveto{\pgfqpoint{2.446696in}{1.834340in}}{\pgfqpoint{2.449969in}{1.842240in}}{\pgfqpoint{2.449969in}{1.850476in}}%
\pgfpathcurveto{\pgfqpoint{2.449969in}{1.858712in}}{\pgfqpoint{2.446696in}{1.866613in}}{\pgfqpoint{2.440872in}{1.872436in}}%
\pgfpathcurveto{\pgfqpoint{2.435048in}{1.878260in}}{\pgfqpoint{2.427148in}{1.881533in}}{\pgfqpoint{2.418912in}{1.881533in}}%
\pgfpathcurveto{\pgfqpoint{2.410676in}{1.881533in}}{\pgfqpoint{2.402776in}{1.878260in}}{\pgfqpoint{2.396952in}{1.872436in}}%
\pgfpathcurveto{\pgfqpoint{2.391128in}{1.866613in}}{\pgfqpoint{2.387856in}{1.858712in}}{\pgfqpoint{2.387856in}{1.850476in}}%
\pgfpathcurveto{\pgfqpoint{2.387856in}{1.842240in}}{\pgfqpoint{2.391128in}{1.834340in}}{\pgfqpoint{2.396952in}{1.828516in}}%
\pgfpathcurveto{\pgfqpoint{2.402776in}{1.822692in}}{\pgfqpoint{2.410676in}{1.819420in}}{\pgfqpoint{2.418912in}{1.819420in}}%
\pgfpathclose%
\pgfusepath{stroke,fill}%
\end{pgfscope}%
\begin{pgfscope}%
\pgfpathrectangle{\pgfqpoint{0.100000in}{0.212622in}}{\pgfqpoint{3.696000in}{3.696000in}}%
\pgfusepath{clip}%
\pgfsetbuttcap%
\pgfsetroundjoin%
\definecolor{currentfill}{rgb}{0.121569,0.466667,0.705882}%
\pgfsetfillcolor{currentfill}%
\pgfsetfillopacity{0.988842}%
\pgfsetlinewidth{1.003750pt}%
\definecolor{currentstroke}{rgb}{0.121569,0.466667,0.705882}%
\pgfsetstrokecolor{currentstroke}%
\pgfsetstrokeopacity{0.988842}%
\pgfsetdash{}{0pt}%
\pgfpathmoveto{\pgfqpoint{2.418749in}{1.819256in}}%
\pgfpathcurveto{\pgfqpoint{2.426985in}{1.819256in}}{\pgfqpoint{2.434885in}{1.822529in}}{\pgfqpoint{2.440709in}{1.828353in}}%
\pgfpathcurveto{\pgfqpoint{2.446533in}{1.834176in}}{\pgfqpoint{2.449805in}{1.842077in}}{\pgfqpoint{2.449805in}{1.850313in}}%
\pgfpathcurveto{\pgfqpoint{2.449805in}{1.858549in}}{\pgfqpoint{2.446533in}{1.866449in}}{\pgfqpoint{2.440709in}{1.872273in}}%
\pgfpathcurveto{\pgfqpoint{2.434885in}{1.878097in}}{\pgfqpoint{2.426985in}{1.881369in}}{\pgfqpoint{2.418749in}{1.881369in}}%
\pgfpathcurveto{\pgfqpoint{2.410512in}{1.881369in}}{\pgfqpoint{2.402612in}{1.878097in}}{\pgfqpoint{2.396788in}{1.872273in}}%
\pgfpathcurveto{\pgfqpoint{2.390964in}{1.866449in}}{\pgfqpoint{2.387692in}{1.858549in}}{\pgfqpoint{2.387692in}{1.850313in}}%
\pgfpathcurveto{\pgfqpoint{2.387692in}{1.842077in}}{\pgfqpoint{2.390964in}{1.834176in}}{\pgfqpoint{2.396788in}{1.828353in}}%
\pgfpathcurveto{\pgfqpoint{2.402612in}{1.822529in}}{\pgfqpoint{2.410512in}{1.819256in}}{\pgfqpoint{2.418749in}{1.819256in}}%
\pgfpathclose%
\pgfusepath{stroke,fill}%
\end{pgfscope}%
\begin{pgfscope}%
\pgfpathrectangle{\pgfqpoint{0.100000in}{0.212622in}}{\pgfqpoint{3.696000in}{3.696000in}}%
\pgfusepath{clip}%
\pgfsetbuttcap%
\pgfsetroundjoin%
\definecolor{currentfill}{rgb}{0.121569,0.466667,0.705882}%
\pgfsetfillcolor{currentfill}%
\pgfsetfillopacity{0.988885}%
\pgfsetlinewidth{1.003750pt}%
\definecolor{currentstroke}{rgb}{0.121569,0.466667,0.705882}%
\pgfsetstrokecolor{currentstroke}%
\pgfsetstrokeopacity{0.988885}%
\pgfsetdash{}{0pt}%
\pgfpathmoveto{\pgfqpoint{2.418641in}{1.819210in}}%
\pgfpathcurveto{\pgfqpoint{2.426877in}{1.819210in}}{\pgfqpoint{2.434777in}{1.822482in}}{\pgfqpoint{2.440601in}{1.828306in}}%
\pgfpathcurveto{\pgfqpoint{2.446425in}{1.834130in}}{\pgfqpoint{2.449697in}{1.842030in}}{\pgfqpoint{2.449697in}{1.850267in}}%
\pgfpathcurveto{\pgfqpoint{2.449697in}{1.858503in}}{\pgfqpoint{2.446425in}{1.866403in}}{\pgfqpoint{2.440601in}{1.872227in}}%
\pgfpathcurveto{\pgfqpoint{2.434777in}{1.878051in}}{\pgfqpoint{2.426877in}{1.881323in}}{\pgfqpoint{2.418641in}{1.881323in}}%
\pgfpathcurveto{\pgfqpoint{2.410405in}{1.881323in}}{\pgfqpoint{2.402505in}{1.878051in}}{\pgfqpoint{2.396681in}{1.872227in}}%
\pgfpathcurveto{\pgfqpoint{2.390857in}{1.866403in}}{\pgfqpoint{2.387584in}{1.858503in}}{\pgfqpoint{2.387584in}{1.850267in}}%
\pgfpathcurveto{\pgfqpoint{2.387584in}{1.842030in}}{\pgfqpoint{2.390857in}{1.834130in}}{\pgfqpoint{2.396681in}{1.828306in}}%
\pgfpathcurveto{\pgfqpoint{2.402505in}{1.822482in}}{\pgfqpoint{2.410405in}{1.819210in}}{\pgfqpoint{2.418641in}{1.819210in}}%
\pgfpathclose%
\pgfusepath{stroke,fill}%
\end{pgfscope}%
\begin{pgfscope}%
\pgfpathrectangle{\pgfqpoint{0.100000in}{0.212622in}}{\pgfqpoint{3.696000in}{3.696000in}}%
\pgfusepath{clip}%
\pgfsetbuttcap%
\pgfsetroundjoin%
\definecolor{currentfill}{rgb}{0.121569,0.466667,0.705882}%
\pgfsetfillcolor{currentfill}%
\pgfsetfillopacity{0.989510}%
\pgfsetlinewidth{1.003750pt}%
\definecolor{currentstroke}{rgb}{0.121569,0.466667,0.705882}%
\pgfsetstrokecolor{currentstroke}%
\pgfsetstrokeopacity{0.989510}%
\pgfsetdash{}{0pt}%
\pgfpathmoveto{\pgfqpoint{2.417329in}{1.818285in}}%
\pgfpathcurveto{\pgfqpoint{2.425565in}{1.818285in}}{\pgfqpoint{2.433465in}{1.821557in}}{\pgfqpoint{2.439289in}{1.827381in}}%
\pgfpathcurveto{\pgfqpoint{2.445113in}{1.833205in}}{\pgfqpoint{2.448386in}{1.841105in}}{\pgfqpoint{2.448386in}{1.849342in}}%
\pgfpathcurveto{\pgfqpoint{2.448386in}{1.857578in}}{\pgfqpoint{2.445113in}{1.865478in}}{\pgfqpoint{2.439289in}{1.871302in}}%
\pgfpathcurveto{\pgfqpoint{2.433465in}{1.877126in}}{\pgfqpoint{2.425565in}{1.880398in}}{\pgfqpoint{2.417329in}{1.880398in}}%
\pgfpathcurveto{\pgfqpoint{2.409093in}{1.880398in}}{\pgfqpoint{2.401193in}{1.877126in}}{\pgfqpoint{2.395369in}{1.871302in}}%
\pgfpathcurveto{\pgfqpoint{2.389545in}{1.865478in}}{\pgfqpoint{2.386273in}{1.857578in}}{\pgfqpoint{2.386273in}{1.849342in}}%
\pgfpathcurveto{\pgfqpoint{2.386273in}{1.841105in}}{\pgfqpoint{2.389545in}{1.833205in}}{\pgfqpoint{2.395369in}{1.827381in}}%
\pgfpathcurveto{\pgfqpoint{2.401193in}{1.821557in}}{\pgfqpoint{2.409093in}{1.818285in}}{\pgfqpoint{2.417329in}{1.818285in}}%
\pgfpathclose%
\pgfusepath{stroke,fill}%
\end{pgfscope}%
\begin{pgfscope}%
\pgfpathrectangle{\pgfqpoint{0.100000in}{0.212622in}}{\pgfqpoint{3.696000in}{3.696000in}}%
\pgfusepath{clip}%
\pgfsetbuttcap%
\pgfsetroundjoin%
\definecolor{currentfill}{rgb}{0.121569,0.466667,0.705882}%
\pgfsetfillcolor{currentfill}%
\pgfsetfillopacity{0.989809}%
\pgfsetlinewidth{1.003750pt}%
\definecolor{currentstroke}{rgb}{0.121569,0.466667,0.705882}%
\pgfsetstrokecolor{currentstroke}%
\pgfsetstrokeopacity{0.989809}%
\pgfsetdash{}{0pt}%
\pgfpathmoveto{\pgfqpoint{2.416376in}{1.817841in}}%
\pgfpathcurveto{\pgfqpoint{2.424613in}{1.817841in}}{\pgfqpoint{2.432513in}{1.821113in}}{\pgfqpoint{2.438337in}{1.826937in}}%
\pgfpathcurveto{\pgfqpoint{2.444161in}{1.832761in}}{\pgfqpoint{2.447433in}{1.840661in}}{\pgfqpoint{2.447433in}{1.848897in}}%
\pgfpathcurveto{\pgfqpoint{2.447433in}{1.857134in}}{\pgfqpoint{2.444161in}{1.865034in}}{\pgfqpoint{2.438337in}{1.870858in}}%
\pgfpathcurveto{\pgfqpoint{2.432513in}{1.876682in}}{\pgfqpoint{2.424613in}{1.879954in}}{\pgfqpoint{2.416376in}{1.879954in}}%
\pgfpathcurveto{\pgfqpoint{2.408140in}{1.879954in}}{\pgfqpoint{2.400240in}{1.876682in}}{\pgfqpoint{2.394416in}{1.870858in}}%
\pgfpathcurveto{\pgfqpoint{2.388592in}{1.865034in}}{\pgfqpoint{2.385320in}{1.857134in}}{\pgfqpoint{2.385320in}{1.848897in}}%
\pgfpathcurveto{\pgfqpoint{2.385320in}{1.840661in}}{\pgfqpoint{2.388592in}{1.832761in}}{\pgfqpoint{2.394416in}{1.826937in}}%
\pgfpathcurveto{\pgfqpoint{2.400240in}{1.821113in}}{\pgfqpoint{2.408140in}{1.817841in}}{\pgfqpoint{2.416376in}{1.817841in}}%
\pgfpathclose%
\pgfusepath{stroke,fill}%
\end{pgfscope}%
\begin{pgfscope}%
\pgfpathrectangle{\pgfqpoint{0.100000in}{0.212622in}}{\pgfqpoint{3.696000in}{3.696000in}}%
\pgfusepath{clip}%
\pgfsetbuttcap%
\pgfsetroundjoin%
\definecolor{currentfill}{rgb}{0.121569,0.466667,0.705882}%
\pgfsetfillcolor{currentfill}%
\pgfsetfillopacity{0.989921}%
\pgfsetlinewidth{1.003750pt}%
\definecolor{currentstroke}{rgb}{0.121569,0.466667,0.705882}%
\pgfsetstrokecolor{currentstroke}%
\pgfsetstrokeopacity{0.989921}%
\pgfsetdash{}{0pt}%
\pgfpathmoveto{\pgfqpoint{2.296007in}{1.858391in}}%
\pgfpathcurveto{\pgfqpoint{2.304243in}{1.858391in}}{\pgfqpoint{2.312143in}{1.861664in}}{\pgfqpoint{2.317967in}{1.867488in}}%
\pgfpathcurveto{\pgfqpoint{2.323791in}{1.873312in}}{\pgfqpoint{2.327063in}{1.881212in}}{\pgfqpoint{2.327063in}{1.889448in}}%
\pgfpathcurveto{\pgfqpoint{2.327063in}{1.897684in}}{\pgfqpoint{2.323791in}{1.905584in}}{\pgfqpoint{2.317967in}{1.911408in}}%
\pgfpathcurveto{\pgfqpoint{2.312143in}{1.917232in}}{\pgfqpoint{2.304243in}{1.920504in}}{\pgfqpoint{2.296007in}{1.920504in}}%
\pgfpathcurveto{\pgfqpoint{2.287771in}{1.920504in}}{\pgfqpoint{2.279871in}{1.917232in}}{\pgfqpoint{2.274047in}{1.911408in}}%
\pgfpathcurveto{\pgfqpoint{2.268223in}{1.905584in}}{\pgfqpoint{2.264950in}{1.897684in}}{\pgfqpoint{2.264950in}{1.889448in}}%
\pgfpathcurveto{\pgfqpoint{2.264950in}{1.881212in}}{\pgfqpoint{2.268223in}{1.873312in}}{\pgfqpoint{2.274047in}{1.867488in}}%
\pgfpathcurveto{\pgfqpoint{2.279871in}{1.861664in}}{\pgfqpoint{2.287771in}{1.858391in}}{\pgfqpoint{2.296007in}{1.858391in}}%
\pgfpathclose%
\pgfusepath{stroke,fill}%
\end{pgfscope}%
\begin{pgfscope}%
\pgfpathrectangle{\pgfqpoint{0.100000in}{0.212622in}}{\pgfqpoint{3.696000in}{3.696000in}}%
\pgfusepath{clip}%
\pgfsetbuttcap%
\pgfsetroundjoin%
\definecolor{currentfill}{rgb}{0.121569,0.466667,0.705882}%
\pgfsetfillcolor{currentfill}%
\pgfsetfillopacity{0.990015}%
\pgfsetlinewidth{1.003750pt}%
\definecolor{currentstroke}{rgb}{0.121569,0.466667,0.705882}%
\pgfsetstrokecolor{currentstroke}%
\pgfsetstrokeopacity{0.990015}%
\pgfsetdash{}{0pt}%
\pgfpathmoveto{\pgfqpoint{2.416011in}{1.817636in}}%
\pgfpathcurveto{\pgfqpoint{2.424248in}{1.817636in}}{\pgfqpoint{2.432148in}{1.820908in}}{\pgfqpoint{2.437972in}{1.826732in}}%
\pgfpathcurveto{\pgfqpoint{2.443796in}{1.832556in}}{\pgfqpoint{2.447068in}{1.840456in}}{\pgfqpoint{2.447068in}{1.848692in}}%
\pgfpathcurveto{\pgfqpoint{2.447068in}{1.856929in}}{\pgfqpoint{2.443796in}{1.864829in}}{\pgfqpoint{2.437972in}{1.870653in}}%
\pgfpathcurveto{\pgfqpoint{2.432148in}{1.876477in}}{\pgfqpoint{2.424248in}{1.879749in}}{\pgfqpoint{2.416011in}{1.879749in}}%
\pgfpathcurveto{\pgfqpoint{2.407775in}{1.879749in}}{\pgfqpoint{2.399875in}{1.876477in}}{\pgfqpoint{2.394051in}{1.870653in}}%
\pgfpathcurveto{\pgfqpoint{2.388227in}{1.864829in}}{\pgfqpoint{2.384955in}{1.856929in}}{\pgfqpoint{2.384955in}{1.848692in}}%
\pgfpathcurveto{\pgfqpoint{2.384955in}{1.840456in}}{\pgfqpoint{2.388227in}{1.832556in}}{\pgfqpoint{2.394051in}{1.826732in}}%
\pgfpathcurveto{\pgfqpoint{2.399875in}{1.820908in}}{\pgfqpoint{2.407775in}{1.817636in}}{\pgfqpoint{2.416011in}{1.817636in}}%
\pgfpathclose%
\pgfusepath{stroke,fill}%
\end{pgfscope}%
\begin{pgfscope}%
\pgfpathrectangle{\pgfqpoint{0.100000in}{0.212622in}}{\pgfqpoint{3.696000in}{3.696000in}}%
\pgfusepath{clip}%
\pgfsetbuttcap%
\pgfsetroundjoin%
\definecolor{currentfill}{rgb}{0.121569,0.466667,0.705882}%
\pgfsetfillcolor{currentfill}%
\pgfsetfillopacity{0.990108}%
\pgfsetlinewidth{1.003750pt}%
\definecolor{currentstroke}{rgb}{0.121569,0.466667,0.705882}%
\pgfsetstrokecolor{currentstroke}%
\pgfsetstrokeopacity{0.990108}%
\pgfsetdash{}{0pt}%
\pgfpathmoveto{\pgfqpoint{2.415729in}{1.817510in}}%
\pgfpathcurveto{\pgfqpoint{2.423965in}{1.817510in}}{\pgfqpoint{2.431865in}{1.820782in}}{\pgfqpoint{2.437689in}{1.826606in}}%
\pgfpathcurveto{\pgfqpoint{2.443513in}{1.832430in}}{\pgfqpoint{2.446786in}{1.840330in}}{\pgfqpoint{2.446786in}{1.848566in}}%
\pgfpathcurveto{\pgfqpoint{2.446786in}{1.856803in}}{\pgfqpoint{2.443513in}{1.864703in}}{\pgfqpoint{2.437689in}{1.870526in}}%
\pgfpathcurveto{\pgfqpoint{2.431865in}{1.876350in}}{\pgfqpoint{2.423965in}{1.879623in}}{\pgfqpoint{2.415729in}{1.879623in}}%
\pgfpathcurveto{\pgfqpoint{2.407493in}{1.879623in}}{\pgfqpoint{2.399593in}{1.876350in}}{\pgfqpoint{2.393769in}{1.870526in}}%
\pgfpathcurveto{\pgfqpoint{2.387945in}{1.864703in}}{\pgfqpoint{2.384673in}{1.856803in}}{\pgfqpoint{2.384673in}{1.848566in}}%
\pgfpathcurveto{\pgfqpoint{2.384673in}{1.840330in}}{\pgfqpoint{2.387945in}{1.832430in}}{\pgfqpoint{2.393769in}{1.826606in}}%
\pgfpathcurveto{\pgfqpoint{2.399593in}{1.820782in}}{\pgfqpoint{2.407493in}{1.817510in}}{\pgfqpoint{2.415729in}{1.817510in}}%
\pgfpathclose%
\pgfusepath{stroke,fill}%
\end{pgfscope}%
\begin{pgfscope}%
\pgfpathrectangle{\pgfqpoint{0.100000in}{0.212622in}}{\pgfqpoint{3.696000in}{3.696000in}}%
\pgfusepath{clip}%
\pgfsetbuttcap%
\pgfsetroundjoin%
\definecolor{currentfill}{rgb}{0.121569,0.466667,0.705882}%
\pgfsetfillcolor{currentfill}%
\pgfsetfillopacity{0.990175}%
\pgfsetlinewidth{1.003750pt}%
\definecolor{currentstroke}{rgb}{0.121569,0.466667,0.705882}%
\pgfsetstrokecolor{currentstroke}%
\pgfsetstrokeopacity{0.990175}%
\pgfsetdash{}{0pt}%
\pgfpathmoveto{\pgfqpoint{2.415609in}{1.817486in}}%
\pgfpathcurveto{\pgfqpoint{2.423845in}{1.817486in}}{\pgfqpoint{2.431745in}{1.820758in}}{\pgfqpoint{2.437569in}{1.826582in}}%
\pgfpathcurveto{\pgfqpoint{2.443393in}{1.832406in}}{\pgfqpoint{2.446665in}{1.840306in}}{\pgfqpoint{2.446665in}{1.848542in}}%
\pgfpathcurveto{\pgfqpoint{2.446665in}{1.856779in}}{\pgfqpoint{2.443393in}{1.864679in}}{\pgfqpoint{2.437569in}{1.870503in}}%
\pgfpathcurveto{\pgfqpoint{2.431745in}{1.876326in}}{\pgfqpoint{2.423845in}{1.879599in}}{\pgfqpoint{2.415609in}{1.879599in}}%
\pgfpathcurveto{\pgfqpoint{2.407372in}{1.879599in}}{\pgfqpoint{2.399472in}{1.876326in}}{\pgfqpoint{2.393648in}{1.870503in}}%
\pgfpathcurveto{\pgfqpoint{2.387825in}{1.864679in}}{\pgfqpoint{2.384552in}{1.856779in}}{\pgfqpoint{2.384552in}{1.848542in}}%
\pgfpathcurveto{\pgfqpoint{2.384552in}{1.840306in}}{\pgfqpoint{2.387825in}{1.832406in}}{\pgfqpoint{2.393648in}{1.826582in}}%
\pgfpathcurveto{\pgfqpoint{2.399472in}{1.820758in}}{\pgfqpoint{2.407372in}{1.817486in}}{\pgfqpoint{2.415609in}{1.817486in}}%
\pgfpathclose%
\pgfusepath{stroke,fill}%
\end{pgfscope}%
\begin{pgfscope}%
\pgfpathrectangle{\pgfqpoint{0.100000in}{0.212622in}}{\pgfqpoint{3.696000in}{3.696000in}}%
\pgfusepath{clip}%
\pgfsetbuttcap%
\pgfsetroundjoin%
\definecolor{currentfill}{rgb}{0.121569,0.466667,0.705882}%
\pgfsetfillcolor{currentfill}%
\pgfsetfillopacity{0.990205}%
\pgfsetlinewidth{1.003750pt}%
\definecolor{currentstroke}{rgb}{0.121569,0.466667,0.705882}%
\pgfsetstrokecolor{currentstroke}%
\pgfsetstrokeopacity{0.990205}%
\pgfsetdash{}{0pt}%
\pgfpathmoveto{\pgfqpoint{2.415542in}{1.817427in}}%
\pgfpathcurveto{\pgfqpoint{2.423778in}{1.817427in}}{\pgfqpoint{2.431678in}{1.820700in}}{\pgfqpoint{2.437502in}{1.826524in}}%
\pgfpathcurveto{\pgfqpoint{2.443326in}{1.832348in}}{\pgfqpoint{2.446598in}{1.840248in}}{\pgfqpoint{2.446598in}{1.848484in}}%
\pgfpathcurveto{\pgfqpoint{2.446598in}{1.856720in}}{\pgfqpoint{2.443326in}{1.864620in}}{\pgfqpoint{2.437502in}{1.870444in}}%
\pgfpathcurveto{\pgfqpoint{2.431678in}{1.876268in}}{\pgfqpoint{2.423778in}{1.879540in}}{\pgfqpoint{2.415542in}{1.879540in}}%
\pgfpathcurveto{\pgfqpoint{2.407306in}{1.879540in}}{\pgfqpoint{2.399406in}{1.876268in}}{\pgfqpoint{2.393582in}{1.870444in}}%
\pgfpathcurveto{\pgfqpoint{2.387758in}{1.864620in}}{\pgfqpoint{2.384485in}{1.856720in}}{\pgfqpoint{2.384485in}{1.848484in}}%
\pgfpathcurveto{\pgfqpoint{2.384485in}{1.840248in}}{\pgfqpoint{2.387758in}{1.832348in}}{\pgfqpoint{2.393582in}{1.826524in}}%
\pgfpathcurveto{\pgfqpoint{2.399406in}{1.820700in}}{\pgfqpoint{2.407306in}{1.817427in}}{\pgfqpoint{2.415542in}{1.817427in}}%
\pgfpathclose%
\pgfusepath{stroke,fill}%
\end{pgfscope}%
\begin{pgfscope}%
\pgfpathrectangle{\pgfqpoint{0.100000in}{0.212622in}}{\pgfqpoint{3.696000in}{3.696000in}}%
\pgfusepath{clip}%
\pgfsetbuttcap%
\pgfsetroundjoin%
\definecolor{currentfill}{rgb}{0.121569,0.466667,0.705882}%
\pgfsetfillcolor{currentfill}%
\pgfsetfillopacity{0.990952}%
\pgfsetlinewidth{1.003750pt}%
\definecolor{currentstroke}{rgb}{0.121569,0.466667,0.705882}%
\pgfsetstrokecolor{currentstroke}%
\pgfsetstrokeopacity{0.990952}%
\pgfsetdash{}{0pt}%
\pgfpathmoveto{\pgfqpoint{2.413905in}{1.816737in}}%
\pgfpathcurveto{\pgfqpoint{2.422141in}{1.816737in}}{\pgfqpoint{2.430041in}{1.820009in}}{\pgfqpoint{2.435865in}{1.825833in}}%
\pgfpathcurveto{\pgfqpoint{2.441689in}{1.831657in}}{\pgfqpoint{2.444961in}{1.839557in}}{\pgfqpoint{2.444961in}{1.847793in}}%
\pgfpathcurveto{\pgfqpoint{2.444961in}{1.856029in}}{\pgfqpoint{2.441689in}{1.863929in}}{\pgfqpoint{2.435865in}{1.869753in}}%
\pgfpathcurveto{\pgfqpoint{2.430041in}{1.875577in}}{\pgfqpoint{2.422141in}{1.878850in}}{\pgfqpoint{2.413905in}{1.878850in}}%
\pgfpathcurveto{\pgfqpoint{2.405668in}{1.878850in}}{\pgfqpoint{2.397768in}{1.875577in}}{\pgfqpoint{2.391944in}{1.869753in}}%
\pgfpathcurveto{\pgfqpoint{2.386120in}{1.863929in}}{\pgfqpoint{2.382848in}{1.856029in}}{\pgfqpoint{2.382848in}{1.847793in}}%
\pgfpathcurveto{\pgfqpoint{2.382848in}{1.839557in}}{\pgfqpoint{2.386120in}{1.831657in}}{\pgfqpoint{2.391944in}{1.825833in}}%
\pgfpathcurveto{\pgfqpoint{2.397768in}{1.820009in}}{\pgfqpoint{2.405668in}{1.816737in}}{\pgfqpoint{2.413905in}{1.816737in}}%
\pgfpathclose%
\pgfusepath{stroke,fill}%
\end{pgfscope}%
\begin{pgfscope}%
\pgfpathrectangle{\pgfqpoint{0.100000in}{0.212622in}}{\pgfqpoint{3.696000in}{3.696000in}}%
\pgfusepath{clip}%
\pgfsetbuttcap%
\pgfsetroundjoin%
\definecolor{currentfill}{rgb}{0.121569,0.466667,0.705882}%
\pgfsetfillcolor{currentfill}%
\pgfsetfillopacity{0.991349}%
\pgfsetlinewidth{1.003750pt}%
\definecolor{currentstroke}{rgb}{0.121569,0.466667,0.705882}%
\pgfsetstrokecolor{currentstroke}%
\pgfsetstrokeopacity{0.991349}%
\pgfsetdash{}{0pt}%
\pgfpathmoveto{\pgfqpoint{2.304804in}{1.855306in}}%
\pgfpathcurveto{\pgfqpoint{2.313040in}{1.855306in}}{\pgfqpoint{2.320941in}{1.858578in}}{\pgfqpoint{2.326764in}{1.864402in}}%
\pgfpathcurveto{\pgfqpoint{2.332588in}{1.870226in}}{\pgfqpoint{2.335861in}{1.878126in}}{\pgfqpoint{2.335861in}{1.886362in}}%
\pgfpathcurveto{\pgfqpoint{2.335861in}{1.894598in}}{\pgfqpoint{2.332588in}{1.902498in}}{\pgfqpoint{2.326764in}{1.908322in}}%
\pgfpathcurveto{\pgfqpoint{2.320941in}{1.914146in}}{\pgfqpoint{2.313040in}{1.917419in}}{\pgfqpoint{2.304804in}{1.917419in}}%
\pgfpathcurveto{\pgfqpoint{2.296568in}{1.917419in}}{\pgfqpoint{2.288668in}{1.914146in}}{\pgfqpoint{2.282844in}{1.908322in}}%
\pgfpathcurveto{\pgfqpoint{2.277020in}{1.902498in}}{\pgfqpoint{2.273748in}{1.894598in}}{\pgfqpoint{2.273748in}{1.886362in}}%
\pgfpathcurveto{\pgfqpoint{2.273748in}{1.878126in}}{\pgfqpoint{2.277020in}{1.870226in}}{\pgfqpoint{2.282844in}{1.864402in}}%
\pgfpathcurveto{\pgfqpoint{2.288668in}{1.858578in}}{\pgfqpoint{2.296568in}{1.855306in}}{\pgfqpoint{2.304804in}{1.855306in}}%
\pgfpathclose%
\pgfusepath{stroke,fill}%
\end{pgfscope}%
\begin{pgfscope}%
\pgfpathrectangle{\pgfqpoint{0.100000in}{0.212622in}}{\pgfqpoint{3.696000in}{3.696000in}}%
\pgfusepath{clip}%
\pgfsetbuttcap%
\pgfsetroundjoin%
\definecolor{currentfill}{rgb}{0.121569,0.466667,0.705882}%
\pgfsetfillcolor{currentfill}%
\pgfsetfillopacity{0.991683}%
\pgfsetlinewidth{1.003750pt}%
\definecolor{currentstroke}{rgb}{0.121569,0.466667,0.705882}%
\pgfsetstrokecolor{currentstroke}%
\pgfsetstrokeopacity{0.991683}%
\pgfsetdash{}{0pt}%
\pgfpathmoveto{\pgfqpoint{2.306833in}{1.853905in}}%
\pgfpathcurveto{\pgfqpoint{2.315069in}{1.853905in}}{\pgfqpoint{2.322970in}{1.857177in}}{\pgfqpoint{2.328793in}{1.863001in}}%
\pgfpathcurveto{\pgfqpoint{2.334617in}{1.868825in}}{\pgfqpoint{2.337890in}{1.876725in}}{\pgfqpoint{2.337890in}{1.884961in}}%
\pgfpathcurveto{\pgfqpoint{2.337890in}{1.893198in}}{\pgfqpoint{2.334617in}{1.901098in}}{\pgfqpoint{2.328793in}{1.906922in}}%
\pgfpathcurveto{\pgfqpoint{2.322970in}{1.912746in}}{\pgfqpoint{2.315069in}{1.916018in}}{\pgfqpoint{2.306833in}{1.916018in}}%
\pgfpathcurveto{\pgfqpoint{2.298597in}{1.916018in}}{\pgfqpoint{2.290697in}{1.912746in}}{\pgfqpoint{2.284873in}{1.906922in}}%
\pgfpathcurveto{\pgfqpoint{2.279049in}{1.901098in}}{\pgfqpoint{2.275777in}{1.893198in}}{\pgfqpoint{2.275777in}{1.884961in}}%
\pgfpathcurveto{\pgfqpoint{2.275777in}{1.876725in}}{\pgfqpoint{2.279049in}{1.868825in}}{\pgfqpoint{2.284873in}{1.863001in}}%
\pgfpathcurveto{\pgfqpoint{2.290697in}{1.857177in}}{\pgfqpoint{2.298597in}{1.853905in}}{\pgfqpoint{2.306833in}{1.853905in}}%
\pgfpathclose%
\pgfusepath{stroke,fill}%
\end{pgfscope}%
\begin{pgfscope}%
\pgfpathrectangle{\pgfqpoint{0.100000in}{0.212622in}}{\pgfqpoint{3.696000in}{3.696000in}}%
\pgfusepath{clip}%
\pgfsetbuttcap%
\pgfsetroundjoin%
\definecolor{currentfill}{rgb}{0.121569,0.466667,0.705882}%
\pgfsetfillcolor{currentfill}%
\pgfsetfillopacity{0.992346}%
\pgfsetlinewidth{1.003750pt}%
\definecolor{currentstroke}{rgb}{0.121569,0.466667,0.705882}%
\pgfsetstrokecolor{currentstroke}%
\pgfsetstrokeopacity{0.992346}%
\pgfsetdash{}{0pt}%
\pgfpathmoveto{\pgfqpoint{2.310735in}{1.852216in}}%
\pgfpathcurveto{\pgfqpoint{2.318971in}{1.852216in}}{\pgfqpoint{2.326872in}{1.855488in}}{\pgfqpoint{2.332695in}{1.861312in}}%
\pgfpathcurveto{\pgfqpoint{2.338519in}{1.867136in}}{\pgfqpoint{2.341792in}{1.875036in}}{\pgfqpoint{2.341792in}{1.883273in}}%
\pgfpathcurveto{\pgfqpoint{2.341792in}{1.891509in}}{\pgfqpoint{2.338519in}{1.899409in}}{\pgfqpoint{2.332695in}{1.905233in}}%
\pgfpathcurveto{\pgfqpoint{2.326872in}{1.911057in}}{\pgfqpoint{2.318971in}{1.914329in}}{\pgfqpoint{2.310735in}{1.914329in}}%
\pgfpathcurveto{\pgfqpoint{2.302499in}{1.914329in}}{\pgfqpoint{2.294599in}{1.911057in}}{\pgfqpoint{2.288775in}{1.905233in}}%
\pgfpathcurveto{\pgfqpoint{2.282951in}{1.899409in}}{\pgfqpoint{2.279679in}{1.891509in}}{\pgfqpoint{2.279679in}{1.883273in}}%
\pgfpathcurveto{\pgfqpoint{2.279679in}{1.875036in}}{\pgfqpoint{2.282951in}{1.867136in}}{\pgfqpoint{2.288775in}{1.861312in}}%
\pgfpathcurveto{\pgfqpoint{2.294599in}{1.855488in}}{\pgfqpoint{2.302499in}{1.852216in}}{\pgfqpoint{2.310735in}{1.852216in}}%
\pgfpathclose%
\pgfusepath{stroke,fill}%
\end{pgfscope}%
\begin{pgfscope}%
\pgfpathrectangle{\pgfqpoint{0.100000in}{0.212622in}}{\pgfqpoint{3.696000in}{3.696000in}}%
\pgfusepath{clip}%
\pgfsetbuttcap%
\pgfsetroundjoin%
\definecolor{currentfill}{rgb}{0.121569,0.466667,0.705882}%
\pgfsetfillcolor{currentfill}%
\pgfsetfillopacity{0.992580}%
\pgfsetlinewidth{1.003750pt}%
\definecolor{currentstroke}{rgb}{0.121569,0.466667,0.705882}%
\pgfsetstrokecolor{currentstroke}%
\pgfsetstrokeopacity{0.992580}%
\pgfsetdash{}{0pt}%
\pgfpathmoveto{\pgfqpoint{2.410481in}{1.814409in}}%
\pgfpathcurveto{\pgfqpoint{2.418717in}{1.814409in}}{\pgfqpoint{2.426617in}{1.817681in}}{\pgfqpoint{2.432441in}{1.823505in}}%
\pgfpathcurveto{\pgfqpoint{2.438265in}{1.829329in}}{\pgfqpoint{2.441538in}{1.837229in}}{\pgfqpoint{2.441538in}{1.845465in}}%
\pgfpathcurveto{\pgfqpoint{2.441538in}{1.853701in}}{\pgfqpoint{2.438265in}{1.861601in}}{\pgfqpoint{2.432441in}{1.867425in}}%
\pgfpathcurveto{\pgfqpoint{2.426617in}{1.873249in}}{\pgfqpoint{2.418717in}{1.876522in}}{\pgfqpoint{2.410481in}{1.876522in}}%
\pgfpathcurveto{\pgfqpoint{2.402245in}{1.876522in}}{\pgfqpoint{2.394345in}{1.873249in}}{\pgfqpoint{2.388521in}{1.867425in}}%
\pgfpathcurveto{\pgfqpoint{2.382697in}{1.861601in}}{\pgfqpoint{2.379425in}{1.853701in}}{\pgfqpoint{2.379425in}{1.845465in}}%
\pgfpathcurveto{\pgfqpoint{2.379425in}{1.837229in}}{\pgfqpoint{2.382697in}{1.829329in}}{\pgfqpoint{2.388521in}{1.823505in}}%
\pgfpathcurveto{\pgfqpoint{2.394345in}{1.817681in}}{\pgfqpoint{2.402245in}{1.814409in}}{\pgfqpoint{2.410481in}{1.814409in}}%
\pgfpathclose%
\pgfusepath{stroke,fill}%
\end{pgfscope}%
\begin{pgfscope}%
\pgfpathrectangle{\pgfqpoint{0.100000in}{0.212622in}}{\pgfqpoint{3.696000in}{3.696000in}}%
\pgfusepath{clip}%
\pgfsetbuttcap%
\pgfsetroundjoin%
\definecolor{currentfill}{rgb}{0.121569,0.466667,0.705882}%
\pgfsetfillcolor{currentfill}%
\pgfsetfillopacity{0.993550}%
\pgfsetlinewidth{1.003750pt}%
\definecolor{currentstroke}{rgb}{0.121569,0.466667,0.705882}%
\pgfsetstrokecolor{currentstroke}%
\pgfsetstrokeopacity{0.993550}%
\pgfsetdash{}{0pt}%
\pgfpathmoveto{\pgfqpoint{2.316733in}{1.846670in}}%
\pgfpathcurveto{\pgfqpoint{2.324969in}{1.846670in}}{\pgfqpoint{2.332869in}{1.849942in}}{\pgfqpoint{2.338693in}{1.855766in}}%
\pgfpathcurveto{\pgfqpoint{2.344517in}{1.861590in}}{\pgfqpoint{2.347789in}{1.869490in}}{\pgfqpoint{2.347789in}{1.877727in}}%
\pgfpathcurveto{\pgfqpoint{2.347789in}{1.885963in}}{\pgfqpoint{2.344517in}{1.893863in}}{\pgfqpoint{2.338693in}{1.899687in}}%
\pgfpathcurveto{\pgfqpoint{2.332869in}{1.905511in}}{\pgfqpoint{2.324969in}{1.908783in}}{\pgfqpoint{2.316733in}{1.908783in}}%
\pgfpathcurveto{\pgfqpoint{2.308496in}{1.908783in}}{\pgfqpoint{2.300596in}{1.905511in}}{\pgfqpoint{2.294772in}{1.899687in}}%
\pgfpathcurveto{\pgfqpoint{2.288949in}{1.893863in}}{\pgfqpoint{2.285676in}{1.885963in}}{\pgfqpoint{2.285676in}{1.877727in}}%
\pgfpathcurveto{\pgfqpoint{2.285676in}{1.869490in}}{\pgfqpoint{2.288949in}{1.861590in}}{\pgfqpoint{2.294772in}{1.855766in}}%
\pgfpathcurveto{\pgfqpoint{2.300596in}{1.849942in}}{\pgfqpoint{2.308496in}{1.846670in}}{\pgfqpoint{2.316733in}{1.846670in}}%
\pgfpathclose%
\pgfusepath{stroke,fill}%
\end{pgfscope}%
\begin{pgfscope}%
\pgfpathrectangle{\pgfqpoint{0.100000in}{0.212622in}}{\pgfqpoint{3.696000in}{3.696000in}}%
\pgfusepath{clip}%
\pgfsetbuttcap%
\pgfsetroundjoin%
\definecolor{currentfill}{rgb}{0.121569,0.466667,0.705882}%
\pgfsetfillcolor{currentfill}%
\pgfsetfillopacity{0.994043}%
\pgfsetlinewidth{1.003750pt}%
\definecolor{currentstroke}{rgb}{0.121569,0.466667,0.705882}%
\pgfsetstrokecolor{currentstroke}%
\pgfsetstrokeopacity{0.994043}%
\pgfsetdash{}{0pt}%
\pgfpathmoveto{\pgfqpoint{2.319176in}{1.843776in}}%
\pgfpathcurveto{\pgfqpoint{2.327412in}{1.843776in}}{\pgfqpoint{2.335312in}{1.847048in}}{\pgfqpoint{2.341136in}{1.852872in}}%
\pgfpathcurveto{\pgfqpoint{2.346960in}{1.858696in}}{\pgfqpoint{2.350232in}{1.866596in}}{\pgfqpoint{2.350232in}{1.874832in}}%
\pgfpathcurveto{\pgfqpoint{2.350232in}{1.883069in}}{\pgfqpoint{2.346960in}{1.890969in}}{\pgfqpoint{2.341136in}{1.896793in}}%
\pgfpathcurveto{\pgfqpoint{2.335312in}{1.902617in}}{\pgfqpoint{2.327412in}{1.905889in}}{\pgfqpoint{2.319176in}{1.905889in}}%
\pgfpathcurveto{\pgfqpoint{2.310939in}{1.905889in}}{\pgfqpoint{2.303039in}{1.902617in}}{\pgfqpoint{2.297215in}{1.896793in}}%
\pgfpathcurveto{\pgfqpoint{2.291391in}{1.890969in}}{\pgfqpoint{2.288119in}{1.883069in}}{\pgfqpoint{2.288119in}{1.874832in}}%
\pgfpathcurveto{\pgfqpoint{2.288119in}{1.866596in}}{\pgfqpoint{2.291391in}{1.858696in}}{\pgfqpoint{2.297215in}{1.852872in}}%
\pgfpathcurveto{\pgfqpoint{2.303039in}{1.847048in}}{\pgfqpoint{2.310939in}{1.843776in}}{\pgfqpoint{2.319176in}{1.843776in}}%
\pgfpathclose%
\pgfusepath{stroke,fill}%
\end{pgfscope}%
\begin{pgfscope}%
\pgfpathrectangle{\pgfqpoint{0.100000in}{0.212622in}}{\pgfqpoint{3.696000in}{3.696000in}}%
\pgfusepath{clip}%
\pgfsetbuttcap%
\pgfsetroundjoin%
\definecolor{currentfill}{rgb}{0.121569,0.466667,0.705882}%
\pgfsetfillcolor{currentfill}%
\pgfsetfillopacity{0.994887}%
\pgfsetlinewidth{1.003750pt}%
\definecolor{currentstroke}{rgb}{0.121569,0.466667,0.705882}%
\pgfsetstrokecolor{currentstroke}%
\pgfsetstrokeopacity{0.994887}%
\pgfsetdash{}{0pt}%
\pgfpathmoveto{\pgfqpoint{2.403343in}{1.811176in}}%
\pgfpathcurveto{\pgfqpoint{2.411579in}{1.811176in}}{\pgfqpoint{2.419479in}{1.814448in}}{\pgfqpoint{2.425303in}{1.820272in}}%
\pgfpathcurveto{\pgfqpoint{2.431127in}{1.826096in}}{\pgfqpoint{2.434399in}{1.833996in}}{\pgfqpoint{2.434399in}{1.842232in}}%
\pgfpathcurveto{\pgfqpoint{2.434399in}{1.850469in}}{\pgfqpoint{2.431127in}{1.858369in}}{\pgfqpoint{2.425303in}{1.864193in}}%
\pgfpathcurveto{\pgfqpoint{2.419479in}{1.870017in}}{\pgfqpoint{2.411579in}{1.873289in}}{\pgfqpoint{2.403343in}{1.873289in}}%
\pgfpathcurveto{\pgfqpoint{2.395107in}{1.873289in}}{\pgfqpoint{2.387207in}{1.870017in}}{\pgfqpoint{2.381383in}{1.864193in}}%
\pgfpathcurveto{\pgfqpoint{2.375559in}{1.858369in}}{\pgfqpoint{2.372286in}{1.850469in}}{\pgfqpoint{2.372286in}{1.842232in}}%
\pgfpathcurveto{\pgfqpoint{2.372286in}{1.833996in}}{\pgfqpoint{2.375559in}{1.826096in}}{\pgfqpoint{2.381383in}{1.820272in}}%
\pgfpathcurveto{\pgfqpoint{2.387207in}{1.814448in}}{\pgfqpoint{2.395107in}{1.811176in}}{\pgfqpoint{2.403343in}{1.811176in}}%
\pgfpathclose%
\pgfusepath{stroke,fill}%
\end{pgfscope}%
\begin{pgfscope}%
\pgfpathrectangle{\pgfqpoint{0.100000in}{0.212622in}}{\pgfqpoint{3.696000in}{3.696000in}}%
\pgfusepath{clip}%
\pgfsetbuttcap%
\pgfsetroundjoin%
\definecolor{currentfill}{rgb}{0.121569,0.466667,0.705882}%
\pgfsetfillcolor{currentfill}%
\pgfsetfillopacity{0.994943}%
\pgfsetlinewidth{1.003750pt}%
\definecolor{currentstroke}{rgb}{0.121569,0.466667,0.705882}%
\pgfsetstrokecolor{currentstroke}%
\pgfsetstrokeopacity{0.994943}%
\pgfsetdash{}{0pt}%
\pgfpathmoveto{\pgfqpoint{2.323888in}{1.839003in}}%
\pgfpathcurveto{\pgfqpoint{2.332124in}{1.839003in}}{\pgfqpoint{2.340024in}{1.842275in}}{\pgfqpoint{2.345848in}{1.848099in}}%
\pgfpathcurveto{\pgfqpoint{2.351672in}{1.853923in}}{\pgfqpoint{2.354945in}{1.861823in}}{\pgfqpoint{2.354945in}{1.870060in}}%
\pgfpathcurveto{\pgfqpoint{2.354945in}{1.878296in}}{\pgfqpoint{2.351672in}{1.886196in}}{\pgfqpoint{2.345848in}{1.892020in}}%
\pgfpathcurveto{\pgfqpoint{2.340024in}{1.897844in}}{\pgfqpoint{2.332124in}{1.901116in}}{\pgfqpoint{2.323888in}{1.901116in}}%
\pgfpathcurveto{\pgfqpoint{2.315652in}{1.901116in}}{\pgfqpoint{2.307752in}{1.897844in}}{\pgfqpoint{2.301928in}{1.892020in}}%
\pgfpathcurveto{\pgfqpoint{2.296104in}{1.886196in}}{\pgfqpoint{2.292832in}{1.878296in}}{\pgfqpoint{2.292832in}{1.870060in}}%
\pgfpathcurveto{\pgfqpoint{2.292832in}{1.861823in}}{\pgfqpoint{2.296104in}{1.853923in}}{\pgfqpoint{2.301928in}{1.848099in}}%
\pgfpathcurveto{\pgfqpoint{2.307752in}{1.842275in}}{\pgfqpoint{2.315652in}{1.839003in}}{\pgfqpoint{2.323888in}{1.839003in}}%
\pgfpathclose%
\pgfusepath{stroke,fill}%
\end{pgfscope}%
\begin{pgfscope}%
\pgfpathrectangle{\pgfqpoint{0.100000in}{0.212622in}}{\pgfqpoint{3.696000in}{3.696000in}}%
\pgfusepath{clip}%
\pgfsetbuttcap%
\pgfsetroundjoin%
\definecolor{currentfill}{rgb}{0.121569,0.466667,0.705882}%
\pgfsetfillcolor{currentfill}%
\pgfsetfillopacity{0.996403}%
\pgfsetlinewidth{1.003750pt}%
\definecolor{currentstroke}{rgb}{0.121569,0.466667,0.705882}%
\pgfsetstrokecolor{currentstroke}%
\pgfsetstrokeopacity{0.996403}%
\pgfsetdash{}{0pt}%
\pgfpathmoveto{\pgfqpoint{2.400035in}{1.809941in}}%
\pgfpathcurveto{\pgfqpoint{2.408271in}{1.809941in}}{\pgfqpoint{2.416171in}{1.813213in}}{\pgfqpoint{2.421995in}{1.819037in}}%
\pgfpathcurveto{\pgfqpoint{2.427819in}{1.824861in}}{\pgfqpoint{2.431091in}{1.832761in}}{\pgfqpoint{2.431091in}{1.840997in}}%
\pgfpathcurveto{\pgfqpoint{2.431091in}{1.849234in}}{\pgfqpoint{2.427819in}{1.857134in}}{\pgfqpoint{2.421995in}{1.862958in}}%
\pgfpathcurveto{\pgfqpoint{2.416171in}{1.868782in}}{\pgfqpoint{2.408271in}{1.872054in}}{\pgfqpoint{2.400035in}{1.872054in}}%
\pgfpathcurveto{\pgfqpoint{2.391798in}{1.872054in}}{\pgfqpoint{2.383898in}{1.868782in}}{\pgfqpoint{2.378074in}{1.862958in}}%
\pgfpathcurveto{\pgfqpoint{2.372251in}{1.857134in}}{\pgfqpoint{2.368978in}{1.849234in}}{\pgfqpoint{2.368978in}{1.840997in}}%
\pgfpathcurveto{\pgfqpoint{2.368978in}{1.832761in}}{\pgfqpoint{2.372251in}{1.824861in}}{\pgfqpoint{2.378074in}{1.819037in}}%
\pgfpathcurveto{\pgfqpoint{2.383898in}{1.813213in}}{\pgfqpoint{2.391798in}{1.809941in}}{\pgfqpoint{2.400035in}{1.809941in}}%
\pgfpathclose%
\pgfusepath{stroke,fill}%
\end{pgfscope}%
\begin{pgfscope}%
\pgfpathrectangle{\pgfqpoint{0.100000in}{0.212622in}}{\pgfqpoint{3.696000in}{3.696000in}}%
\pgfusepath{clip}%
\pgfsetbuttcap%
\pgfsetroundjoin%
\definecolor{currentfill}{rgb}{0.121569,0.466667,0.705882}%
\pgfsetfillcolor{currentfill}%
\pgfsetfillopacity{0.996503}%
\pgfsetlinewidth{1.003750pt}%
\definecolor{currentstroke}{rgb}{0.121569,0.466667,0.705882}%
\pgfsetstrokecolor{currentstroke}%
\pgfsetstrokeopacity{0.996503}%
\pgfsetdash{}{0pt}%
\pgfpathmoveto{\pgfqpoint{2.332392in}{1.829733in}}%
\pgfpathcurveto{\pgfqpoint{2.340628in}{1.829733in}}{\pgfqpoint{2.348528in}{1.833005in}}{\pgfqpoint{2.354352in}{1.838829in}}%
\pgfpathcurveto{\pgfqpoint{2.360176in}{1.844653in}}{\pgfqpoint{2.363448in}{1.852553in}}{\pgfqpoint{2.363448in}{1.860789in}}%
\pgfpathcurveto{\pgfqpoint{2.363448in}{1.869026in}}{\pgfqpoint{2.360176in}{1.876926in}}{\pgfqpoint{2.354352in}{1.882750in}}%
\pgfpathcurveto{\pgfqpoint{2.348528in}{1.888574in}}{\pgfqpoint{2.340628in}{1.891846in}}{\pgfqpoint{2.332392in}{1.891846in}}%
\pgfpathcurveto{\pgfqpoint{2.324155in}{1.891846in}}{\pgfqpoint{2.316255in}{1.888574in}}{\pgfqpoint{2.310432in}{1.882750in}}%
\pgfpathcurveto{\pgfqpoint{2.304608in}{1.876926in}}{\pgfqpoint{2.301335in}{1.869026in}}{\pgfqpoint{2.301335in}{1.860789in}}%
\pgfpathcurveto{\pgfqpoint{2.301335in}{1.852553in}}{\pgfqpoint{2.304608in}{1.844653in}}{\pgfqpoint{2.310432in}{1.838829in}}%
\pgfpathcurveto{\pgfqpoint{2.316255in}{1.833005in}}{\pgfqpoint{2.324155in}{1.829733in}}{\pgfqpoint{2.332392in}{1.829733in}}%
\pgfpathclose%
\pgfusepath{stroke,fill}%
\end{pgfscope}%
\begin{pgfscope}%
\pgfpathrectangle{\pgfqpoint{0.100000in}{0.212622in}}{\pgfqpoint{3.696000in}{3.696000in}}%
\pgfusepath{clip}%
\pgfsetbuttcap%
\pgfsetroundjoin%
\definecolor{currentfill}{rgb}{0.121569,0.466667,0.705882}%
\pgfsetfillcolor{currentfill}%
\pgfsetfillopacity{0.997048}%
\pgfsetlinewidth{1.003750pt}%
\definecolor{currentstroke}{rgb}{0.121569,0.466667,0.705882}%
\pgfsetstrokecolor{currentstroke}%
\pgfsetstrokeopacity{0.997048}%
\pgfsetdash{}{0pt}%
\pgfpathmoveto{\pgfqpoint{2.397947in}{1.808494in}}%
\pgfpathcurveto{\pgfqpoint{2.406184in}{1.808494in}}{\pgfqpoint{2.414084in}{1.811766in}}{\pgfqpoint{2.419908in}{1.817590in}}%
\pgfpathcurveto{\pgfqpoint{2.425731in}{1.823414in}}{\pgfqpoint{2.429004in}{1.831314in}}{\pgfqpoint{2.429004in}{1.839550in}}%
\pgfpathcurveto{\pgfqpoint{2.429004in}{1.847787in}}{\pgfqpoint{2.425731in}{1.855687in}}{\pgfqpoint{2.419908in}{1.861511in}}%
\pgfpathcurveto{\pgfqpoint{2.414084in}{1.867335in}}{\pgfqpoint{2.406184in}{1.870607in}}{\pgfqpoint{2.397947in}{1.870607in}}%
\pgfpathcurveto{\pgfqpoint{2.389711in}{1.870607in}}{\pgfqpoint{2.381811in}{1.867335in}}{\pgfqpoint{2.375987in}{1.861511in}}%
\pgfpathcurveto{\pgfqpoint{2.370163in}{1.855687in}}{\pgfqpoint{2.366891in}{1.847787in}}{\pgfqpoint{2.366891in}{1.839550in}}%
\pgfpathcurveto{\pgfqpoint{2.366891in}{1.831314in}}{\pgfqpoint{2.370163in}{1.823414in}}{\pgfqpoint{2.375987in}{1.817590in}}%
\pgfpathcurveto{\pgfqpoint{2.381811in}{1.811766in}}{\pgfqpoint{2.389711in}{1.808494in}}{\pgfqpoint{2.397947in}{1.808494in}}%
\pgfpathclose%
\pgfusepath{stroke,fill}%
\end{pgfscope}%
\begin{pgfscope}%
\pgfpathrectangle{\pgfqpoint{0.100000in}{0.212622in}}{\pgfqpoint{3.696000in}{3.696000in}}%
\pgfusepath{clip}%
\pgfsetbuttcap%
\pgfsetroundjoin%
\definecolor{currentfill}{rgb}{0.121569,0.466667,0.705882}%
\pgfsetfillcolor{currentfill}%
\pgfsetfillopacity{0.997552}%
\pgfsetlinewidth{1.003750pt}%
\definecolor{currentstroke}{rgb}{0.121569,0.466667,0.705882}%
\pgfsetstrokecolor{currentstroke}%
\pgfsetstrokeopacity{0.997552}%
\pgfsetdash{}{0pt}%
\pgfpathmoveto{\pgfqpoint{2.397132in}{1.808189in}}%
\pgfpathcurveto{\pgfqpoint{2.405368in}{1.808189in}}{\pgfqpoint{2.413269in}{1.811462in}}{\pgfqpoint{2.419092in}{1.817286in}}%
\pgfpathcurveto{\pgfqpoint{2.424916in}{1.823110in}}{\pgfqpoint{2.428189in}{1.831010in}}{\pgfqpoint{2.428189in}{1.839246in}}%
\pgfpathcurveto{\pgfqpoint{2.428189in}{1.847482in}}{\pgfqpoint{2.424916in}{1.855382in}}{\pgfqpoint{2.419092in}{1.861206in}}%
\pgfpathcurveto{\pgfqpoint{2.413269in}{1.867030in}}{\pgfqpoint{2.405368in}{1.870302in}}{\pgfqpoint{2.397132in}{1.870302in}}%
\pgfpathcurveto{\pgfqpoint{2.388896in}{1.870302in}}{\pgfqpoint{2.380996in}{1.867030in}}{\pgfqpoint{2.375172in}{1.861206in}}%
\pgfpathcurveto{\pgfqpoint{2.369348in}{1.855382in}}{\pgfqpoint{2.366076in}{1.847482in}}{\pgfqpoint{2.366076in}{1.839246in}}%
\pgfpathcurveto{\pgfqpoint{2.366076in}{1.831010in}}{\pgfqpoint{2.369348in}{1.823110in}}{\pgfqpoint{2.375172in}{1.817286in}}%
\pgfpathcurveto{\pgfqpoint{2.380996in}{1.811462in}}{\pgfqpoint{2.388896in}{1.808189in}}{\pgfqpoint{2.397132in}{1.808189in}}%
\pgfpathclose%
\pgfusepath{stroke,fill}%
\end{pgfscope}%
\begin{pgfscope}%
\pgfpathrectangle{\pgfqpoint{0.100000in}{0.212622in}}{\pgfqpoint{3.696000in}{3.696000in}}%
\pgfusepath{clip}%
\pgfsetbuttcap%
\pgfsetroundjoin%
\definecolor{currentfill}{rgb}{0.121569,0.466667,0.705882}%
\pgfsetfillcolor{currentfill}%
\pgfsetfillopacity{0.997777}%
\pgfsetlinewidth{1.003750pt}%
\definecolor{currentstroke}{rgb}{0.121569,0.466667,0.705882}%
\pgfsetstrokecolor{currentstroke}%
\pgfsetstrokeopacity{0.997777}%
\pgfsetdash{}{0pt}%
\pgfpathmoveto{\pgfqpoint{2.396539in}{1.807871in}}%
\pgfpathcurveto{\pgfqpoint{2.404776in}{1.807871in}}{\pgfqpoint{2.412676in}{1.811143in}}{\pgfqpoint{2.418500in}{1.816967in}}%
\pgfpathcurveto{\pgfqpoint{2.424324in}{1.822791in}}{\pgfqpoint{2.427596in}{1.830691in}}{\pgfqpoint{2.427596in}{1.838927in}}%
\pgfpathcurveto{\pgfqpoint{2.427596in}{1.847164in}}{\pgfqpoint{2.424324in}{1.855064in}}{\pgfqpoint{2.418500in}{1.860888in}}%
\pgfpathcurveto{\pgfqpoint{2.412676in}{1.866712in}}{\pgfqpoint{2.404776in}{1.869984in}}{\pgfqpoint{2.396539in}{1.869984in}}%
\pgfpathcurveto{\pgfqpoint{2.388303in}{1.869984in}}{\pgfqpoint{2.380403in}{1.866712in}}{\pgfqpoint{2.374579in}{1.860888in}}%
\pgfpathcurveto{\pgfqpoint{2.368755in}{1.855064in}}{\pgfqpoint{2.365483in}{1.847164in}}{\pgfqpoint{2.365483in}{1.838927in}}%
\pgfpathcurveto{\pgfqpoint{2.365483in}{1.830691in}}{\pgfqpoint{2.368755in}{1.822791in}}{\pgfqpoint{2.374579in}{1.816967in}}%
\pgfpathcurveto{\pgfqpoint{2.380403in}{1.811143in}}{\pgfqpoint{2.388303in}{1.807871in}}{\pgfqpoint{2.396539in}{1.807871in}}%
\pgfpathclose%
\pgfusepath{stroke,fill}%
\end{pgfscope}%
\begin{pgfscope}%
\pgfpathrectangle{\pgfqpoint{0.100000in}{0.212622in}}{\pgfqpoint{3.696000in}{3.696000in}}%
\pgfusepath{clip}%
\pgfsetbuttcap%
\pgfsetroundjoin%
\definecolor{currentfill}{rgb}{0.121569,0.466667,0.705882}%
\pgfsetfillcolor{currentfill}%
\pgfsetfillopacity{0.997908}%
\pgfsetlinewidth{1.003750pt}%
\definecolor{currentstroke}{rgb}{0.121569,0.466667,0.705882}%
\pgfsetstrokecolor{currentstroke}%
\pgfsetstrokeopacity{0.997908}%
\pgfsetdash{}{0pt}%
\pgfpathmoveto{\pgfqpoint{2.396201in}{1.807756in}}%
\pgfpathcurveto{\pgfqpoint{2.404437in}{1.807756in}}{\pgfqpoint{2.412337in}{1.811029in}}{\pgfqpoint{2.418161in}{1.816852in}}%
\pgfpathcurveto{\pgfqpoint{2.423985in}{1.822676in}}{\pgfqpoint{2.427257in}{1.830576in}}{\pgfqpoint{2.427257in}{1.838813in}}%
\pgfpathcurveto{\pgfqpoint{2.427257in}{1.847049in}}{\pgfqpoint{2.423985in}{1.854949in}}{\pgfqpoint{2.418161in}{1.860773in}}%
\pgfpathcurveto{\pgfqpoint{2.412337in}{1.866597in}}{\pgfqpoint{2.404437in}{1.869869in}}{\pgfqpoint{2.396201in}{1.869869in}}%
\pgfpathcurveto{\pgfqpoint{2.387964in}{1.869869in}}{\pgfqpoint{2.380064in}{1.866597in}}{\pgfqpoint{2.374241in}{1.860773in}}%
\pgfpathcurveto{\pgfqpoint{2.368417in}{1.854949in}}{\pgfqpoint{2.365144in}{1.847049in}}{\pgfqpoint{2.365144in}{1.838813in}}%
\pgfpathcurveto{\pgfqpoint{2.365144in}{1.830576in}}{\pgfqpoint{2.368417in}{1.822676in}}{\pgfqpoint{2.374241in}{1.816852in}}%
\pgfpathcurveto{\pgfqpoint{2.380064in}{1.811029in}}{\pgfqpoint{2.387964in}{1.807756in}}{\pgfqpoint{2.396201in}{1.807756in}}%
\pgfpathclose%
\pgfusepath{stroke,fill}%
\end{pgfscope}%
\begin{pgfscope}%
\pgfpathrectangle{\pgfqpoint{0.100000in}{0.212622in}}{\pgfqpoint{3.696000in}{3.696000in}}%
\pgfusepath{clip}%
\pgfsetbuttcap%
\pgfsetroundjoin%
\definecolor{currentfill}{rgb}{0.121569,0.466667,0.705882}%
\pgfsetfillcolor{currentfill}%
\pgfsetfillopacity{0.997979}%
\pgfsetlinewidth{1.003750pt}%
\definecolor{currentstroke}{rgb}{0.121569,0.466667,0.705882}%
\pgfsetstrokecolor{currentstroke}%
\pgfsetstrokeopacity{0.997979}%
\pgfsetdash{}{0pt}%
\pgfpathmoveto{\pgfqpoint{2.396049in}{1.807641in}}%
\pgfpathcurveto{\pgfqpoint{2.404285in}{1.807641in}}{\pgfqpoint{2.412185in}{1.810913in}}{\pgfqpoint{2.418009in}{1.816737in}}%
\pgfpathcurveto{\pgfqpoint{2.423833in}{1.822561in}}{\pgfqpoint{2.427105in}{1.830461in}}{\pgfqpoint{2.427105in}{1.838697in}}%
\pgfpathcurveto{\pgfqpoint{2.427105in}{1.846934in}}{\pgfqpoint{2.423833in}{1.854834in}}{\pgfqpoint{2.418009in}{1.860658in}}%
\pgfpathcurveto{\pgfqpoint{2.412185in}{1.866482in}}{\pgfqpoint{2.404285in}{1.869754in}}{\pgfqpoint{2.396049in}{1.869754in}}%
\pgfpathcurveto{\pgfqpoint{2.387812in}{1.869754in}}{\pgfqpoint{2.379912in}{1.866482in}}{\pgfqpoint{2.374088in}{1.860658in}}%
\pgfpathcurveto{\pgfqpoint{2.368264in}{1.854834in}}{\pgfqpoint{2.364992in}{1.846934in}}{\pgfqpoint{2.364992in}{1.838697in}}%
\pgfpathcurveto{\pgfqpoint{2.364992in}{1.830461in}}{\pgfqpoint{2.368264in}{1.822561in}}{\pgfqpoint{2.374088in}{1.816737in}}%
\pgfpathcurveto{\pgfqpoint{2.379912in}{1.810913in}}{\pgfqpoint{2.387812in}{1.807641in}}{\pgfqpoint{2.396049in}{1.807641in}}%
\pgfpathclose%
\pgfusepath{stroke,fill}%
\end{pgfscope}%
\begin{pgfscope}%
\pgfpathrectangle{\pgfqpoint{0.100000in}{0.212622in}}{\pgfqpoint{3.696000in}{3.696000in}}%
\pgfusepath{clip}%
\pgfsetbuttcap%
\pgfsetroundjoin%
\definecolor{currentfill}{rgb}{0.121569,0.466667,0.705882}%
\pgfsetfillcolor{currentfill}%
\pgfsetfillopacity{0.998019}%
\pgfsetlinewidth{1.003750pt}%
\definecolor{currentstroke}{rgb}{0.121569,0.466667,0.705882}%
\pgfsetstrokecolor{currentstroke}%
\pgfsetstrokeopacity{0.998019}%
\pgfsetdash{}{0pt}%
\pgfpathmoveto{\pgfqpoint{2.395944in}{1.807611in}}%
\pgfpathcurveto{\pgfqpoint{2.404181in}{1.807611in}}{\pgfqpoint{2.412081in}{1.810883in}}{\pgfqpoint{2.417905in}{1.816707in}}%
\pgfpathcurveto{\pgfqpoint{2.423729in}{1.822531in}}{\pgfqpoint{2.427001in}{1.830431in}}{\pgfqpoint{2.427001in}{1.838668in}}%
\pgfpathcurveto{\pgfqpoint{2.427001in}{1.846904in}}{\pgfqpoint{2.423729in}{1.854804in}}{\pgfqpoint{2.417905in}{1.860628in}}%
\pgfpathcurveto{\pgfqpoint{2.412081in}{1.866452in}}{\pgfqpoint{2.404181in}{1.869724in}}{\pgfqpoint{2.395944in}{1.869724in}}%
\pgfpathcurveto{\pgfqpoint{2.387708in}{1.869724in}}{\pgfqpoint{2.379808in}{1.866452in}}{\pgfqpoint{2.373984in}{1.860628in}}%
\pgfpathcurveto{\pgfqpoint{2.368160in}{1.854804in}}{\pgfqpoint{2.364888in}{1.846904in}}{\pgfqpoint{2.364888in}{1.838668in}}%
\pgfpathcurveto{\pgfqpoint{2.364888in}{1.830431in}}{\pgfqpoint{2.368160in}{1.822531in}}{\pgfqpoint{2.373984in}{1.816707in}}%
\pgfpathcurveto{\pgfqpoint{2.379808in}{1.810883in}}{\pgfqpoint{2.387708in}{1.807611in}}{\pgfqpoint{2.395944in}{1.807611in}}%
\pgfpathclose%
\pgfusepath{stroke,fill}%
\end{pgfscope}%
\begin{pgfscope}%
\pgfpathrectangle{\pgfqpoint{0.100000in}{0.212622in}}{\pgfqpoint{3.696000in}{3.696000in}}%
\pgfusepath{clip}%
\pgfsetbuttcap%
\pgfsetroundjoin%
\definecolor{currentfill}{rgb}{0.121569,0.466667,0.705882}%
\pgfsetfillcolor{currentfill}%
\pgfsetfillopacity{0.998041}%
\pgfsetlinewidth{1.003750pt}%
\definecolor{currentstroke}{rgb}{0.121569,0.466667,0.705882}%
\pgfsetstrokecolor{currentstroke}%
\pgfsetstrokeopacity{0.998041}%
\pgfsetdash{}{0pt}%
\pgfpathmoveto{\pgfqpoint{2.395893in}{1.807586in}}%
\pgfpathcurveto{\pgfqpoint{2.404130in}{1.807586in}}{\pgfqpoint{2.412030in}{1.810859in}}{\pgfqpoint{2.417854in}{1.816683in}}%
\pgfpathcurveto{\pgfqpoint{2.423677in}{1.822507in}}{\pgfqpoint{2.426950in}{1.830407in}}{\pgfqpoint{2.426950in}{1.838643in}}%
\pgfpathcurveto{\pgfqpoint{2.426950in}{1.846879in}}{\pgfqpoint{2.423677in}{1.854779in}}{\pgfqpoint{2.417854in}{1.860603in}}%
\pgfpathcurveto{\pgfqpoint{2.412030in}{1.866427in}}{\pgfqpoint{2.404130in}{1.869699in}}{\pgfqpoint{2.395893in}{1.869699in}}%
\pgfpathcurveto{\pgfqpoint{2.387657in}{1.869699in}}{\pgfqpoint{2.379757in}{1.866427in}}{\pgfqpoint{2.373933in}{1.860603in}}%
\pgfpathcurveto{\pgfqpoint{2.368109in}{1.854779in}}{\pgfqpoint{2.364837in}{1.846879in}}{\pgfqpoint{2.364837in}{1.838643in}}%
\pgfpathcurveto{\pgfqpoint{2.364837in}{1.830407in}}{\pgfqpoint{2.368109in}{1.822507in}}{\pgfqpoint{2.373933in}{1.816683in}}%
\pgfpathcurveto{\pgfqpoint{2.379757in}{1.810859in}}{\pgfqpoint{2.387657in}{1.807586in}}{\pgfqpoint{2.395893in}{1.807586in}}%
\pgfpathclose%
\pgfusepath{stroke,fill}%
\end{pgfscope}%
\begin{pgfscope}%
\pgfpathrectangle{\pgfqpoint{0.100000in}{0.212622in}}{\pgfqpoint{3.696000in}{3.696000in}}%
\pgfusepath{clip}%
\pgfsetbuttcap%
\pgfsetroundjoin%
\definecolor{currentfill}{rgb}{0.121569,0.466667,0.705882}%
\pgfsetfillcolor{currentfill}%
\pgfsetfillopacity{0.998053}%
\pgfsetlinewidth{1.003750pt}%
\definecolor{currentstroke}{rgb}{0.121569,0.466667,0.705882}%
\pgfsetstrokecolor{currentstroke}%
\pgfsetstrokeopacity{0.998053}%
\pgfsetdash{}{0pt}%
\pgfpathmoveto{\pgfqpoint{2.395864in}{1.807574in}}%
\pgfpathcurveto{\pgfqpoint{2.404100in}{1.807574in}}{\pgfqpoint{2.412000in}{1.810846in}}{\pgfqpoint{2.417824in}{1.816670in}}%
\pgfpathcurveto{\pgfqpoint{2.423648in}{1.822494in}}{\pgfqpoint{2.426920in}{1.830394in}}{\pgfqpoint{2.426920in}{1.838631in}}%
\pgfpathcurveto{\pgfqpoint{2.426920in}{1.846867in}}{\pgfqpoint{2.423648in}{1.854767in}}{\pgfqpoint{2.417824in}{1.860591in}}%
\pgfpathcurveto{\pgfqpoint{2.412000in}{1.866415in}}{\pgfqpoint{2.404100in}{1.869687in}}{\pgfqpoint{2.395864in}{1.869687in}}%
\pgfpathcurveto{\pgfqpoint{2.387628in}{1.869687in}}{\pgfqpoint{2.379728in}{1.866415in}}{\pgfqpoint{2.373904in}{1.860591in}}%
\pgfpathcurveto{\pgfqpoint{2.368080in}{1.854767in}}{\pgfqpoint{2.364807in}{1.846867in}}{\pgfqpoint{2.364807in}{1.838631in}}%
\pgfpathcurveto{\pgfqpoint{2.364807in}{1.830394in}}{\pgfqpoint{2.368080in}{1.822494in}}{\pgfqpoint{2.373904in}{1.816670in}}%
\pgfpathcurveto{\pgfqpoint{2.379728in}{1.810846in}}{\pgfqpoint{2.387628in}{1.807574in}}{\pgfqpoint{2.395864in}{1.807574in}}%
\pgfpathclose%
\pgfusepath{stroke,fill}%
\end{pgfscope}%
\begin{pgfscope}%
\pgfpathrectangle{\pgfqpoint{0.100000in}{0.212622in}}{\pgfqpoint{3.696000in}{3.696000in}}%
\pgfusepath{clip}%
\pgfsetbuttcap%
\pgfsetroundjoin%
\definecolor{currentfill}{rgb}{0.121569,0.466667,0.705882}%
\pgfsetfillcolor{currentfill}%
\pgfsetfillopacity{0.998059}%
\pgfsetlinewidth{1.003750pt}%
\definecolor{currentstroke}{rgb}{0.121569,0.466667,0.705882}%
\pgfsetstrokecolor{currentstroke}%
\pgfsetstrokeopacity{0.998059}%
\pgfsetdash{}{0pt}%
\pgfpathmoveto{\pgfqpoint{2.395847in}{1.807568in}}%
\pgfpathcurveto{\pgfqpoint{2.404083in}{1.807568in}}{\pgfqpoint{2.411983in}{1.810841in}}{\pgfqpoint{2.417807in}{1.816664in}}%
\pgfpathcurveto{\pgfqpoint{2.423631in}{1.822488in}}{\pgfqpoint{2.426903in}{1.830388in}}{\pgfqpoint{2.426903in}{1.838625in}}%
\pgfpathcurveto{\pgfqpoint{2.426903in}{1.846861in}}{\pgfqpoint{2.423631in}{1.854761in}}{\pgfqpoint{2.417807in}{1.860585in}}%
\pgfpathcurveto{\pgfqpoint{2.411983in}{1.866409in}}{\pgfqpoint{2.404083in}{1.869681in}}{\pgfqpoint{2.395847in}{1.869681in}}%
\pgfpathcurveto{\pgfqpoint{2.387610in}{1.869681in}}{\pgfqpoint{2.379710in}{1.866409in}}{\pgfqpoint{2.373887in}{1.860585in}}%
\pgfpathcurveto{\pgfqpoint{2.368063in}{1.854761in}}{\pgfqpoint{2.364790in}{1.846861in}}{\pgfqpoint{2.364790in}{1.838625in}}%
\pgfpathcurveto{\pgfqpoint{2.364790in}{1.830388in}}{\pgfqpoint{2.368063in}{1.822488in}}{\pgfqpoint{2.373887in}{1.816664in}}%
\pgfpathcurveto{\pgfqpoint{2.379710in}{1.810841in}}{\pgfqpoint{2.387610in}{1.807568in}}{\pgfqpoint{2.395847in}{1.807568in}}%
\pgfpathclose%
\pgfusepath{stroke,fill}%
\end{pgfscope}%
\begin{pgfscope}%
\pgfpathrectangle{\pgfqpoint{0.100000in}{0.212622in}}{\pgfqpoint{3.696000in}{3.696000in}}%
\pgfusepath{clip}%
\pgfsetbuttcap%
\pgfsetroundjoin%
\definecolor{currentfill}{rgb}{0.121569,0.466667,0.705882}%
\pgfsetfillcolor{currentfill}%
\pgfsetfillopacity{0.998063}%
\pgfsetlinewidth{1.003750pt}%
\definecolor{currentstroke}{rgb}{0.121569,0.466667,0.705882}%
\pgfsetstrokecolor{currentstroke}%
\pgfsetstrokeopacity{0.998063}%
\pgfsetdash{}{0pt}%
\pgfpathmoveto{\pgfqpoint{2.395837in}{1.807566in}}%
\pgfpathcurveto{\pgfqpoint{2.404073in}{1.807566in}}{\pgfqpoint{2.411974in}{1.810838in}}{\pgfqpoint{2.417797in}{1.816662in}}%
\pgfpathcurveto{\pgfqpoint{2.423621in}{1.822486in}}{\pgfqpoint{2.426894in}{1.830386in}}{\pgfqpoint{2.426894in}{1.838622in}}%
\pgfpathcurveto{\pgfqpoint{2.426894in}{1.846859in}}{\pgfqpoint{2.423621in}{1.854759in}}{\pgfqpoint{2.417797in}{1.860583in}}%
\pgfpathcurveto{\pgfqpoint{2.411974in}{1.866407in}}{\pgfqpoint{2.404073in}{1.869679in}}{\pgfqpoint{2.395837in}{1.869679in}}%
\pgfpathcurveto{\pgfqpoint{2.387601in}{1.869679in}}{\pgfqpoint{2.379701in}{1.866407in}}{\pgfqpoint{2.373877in}{1.860583in}}%
\pgfpathcurveto{\pgfqpoint{2.368053in}{1.854759in}}{\pgfqpoint{2.364781in}{1.846859in}}{\pgfqpoint{2.364781in}{1.838622in}}%
\pgfpathcurveto{\pgfqpoint{2.364781in}{1.830386in}}{\pgfqpoint{2.368053in}{1.822486in}}{\pgfqpoint{2.373877in}{1.816662in}}%
\pgfpathcurveto{\pgfqpoint{2.379701in}{1.810838in}}{\pgfqpoint{2.387601in}{1.807566in}}{\pgfqpoint{2.395837in}{1.807566in}}%
\pgfpathclose%
\pgfusepath{stroke,fill}%
\end{pgfscope}%
\begin{pgfscope}%
\pgfpathrectangle{\pgfqpoint{0.100000in}{0.212622in}}{\pgfqpoint{3.696000in}{3.696000in}}%
\pgfusepath{clip}%
\pgfsetbuttcap%
\pgfsetroundjoin%
\definecolor{currentfill}{rgb}{0.121569,0.466667,0.705882}%
\pgfsetfillcolor{currentfill}%
\pgfsetfillopacity{0.998065}%
\pgfsetlinewidth{1.003750pt}%
\definecolor{currentstroke}{rgb}{0.121569,0.466667,0.705882}%
\pgfsetstrokecolor{currentstroke}%
\pgfsetstrokeopacity{0.998065}%
\pgfsetdash{}{0pt}%
\pgfpathmoveto{\pgfqpoint{2.395831in}{1.807566in}}%
\pgfpathcurveto{\pgfqpoint{2.404068in}{1.807566in}}{\pgfqpoint{2.411968in}{1.810838in}}{\pgfqpoint{2.417792in}{1.816662in}}%
\pgfpathcurveto{\pgfqpoint{2.423615in}{1.822486in}}{\pgfqpoint{2.426888in}{1.830386in}}{\pgfqpoint{2.426888in}{1.838622in}}%
\pgfpathcurveto{\pgfqpoint{2.426888in}{1.846859in}}{\pgfqpoint{2.423615in}{1.854759in}}{\pgfqpoint{2.417792in}{1.860583in}}%
\pgfpathcurveto{\pgfqpoint{2.411968in}{1.866407in}}{\pgfqpoint{2.404068in}{1.869679in}}{\pgfqpoint{2.395831in}{1.869679in}}%
\pgfpathcurveto{\pgfqpoint{2.387595in}{1.869679in}}{\pgfqpoint{2.379695in}{1.866407in}}{\pgfqpoint{2.373871in}{1.860583in}}%
\pgfpathcurveto{\pgfqpoint{2.368047in}{1.854759in}}{\pgfqpoint{2.364775in}{1.846859in}}{\pgfqpoint{2.364775in}{1.838622in}}%
\pgfpathcurveto{\pgfqpoint{2.364775in}{1.830386in}}{\pgfqpoint{2.368047in}{1.822486in}}{\pgfqpoint{2.373871in}{1.816662in}}%
\pgfpathcurveto{\pgfqpoint{2.379695in}{1.810838in}}{\pgfqpoint{2.387595in}{1.807566in}}{\pgfqpoint{2.395831in}{1.807566in}}%
\pgfpathclose%
\pgfusepath{stroke,fill}%
\end{pgfscope}%
\begin{pgfscope}%
\pgfpathrectangle{\pgfqpoint{0.100000in}{0.212622in}}{\pgfqpoint{3.696000in}{3.696000in}}%
\pgfusepath{clip}%
\pgfsetbuttcap%
\pgfsetroundjoin%
\definecolor{currentfill}{rgb}{0.121569,0.466667,0.705882}%
\pgfsetfillcolor{currentfill}%
\pgfsetfillopacity{0.998423}%
\pgfsetlinewidth{1.003750pt}%
\definecolor{currentstroke}{rgb}{0.121569,0.466667,0.705882}%
\pgfsetstrokecolor{currentstroke}%
\pgfsetstrokeopacity{0.998423}%
\pgfsetdash{}{0pt}%
\pgfpathmoveto{\pgfqpoint{2.350975in}{1.814711in}}%
\pgfpathcurveto{\pgfqpoint{2.359212in}{1.814711in}}{\pgfqpoint{2.367112in}{1.817984in}}{\pgfqpoint{2.372935in}{1.823808in}}%
\pgfpathcurveto{\pgfqpoint{2.378759in}{1.829632in}}{\pgfqpoint{2.382032in}{1.837532in}}{\pgfqpoint{2.382032in}{1.845768in}}%
\pgfpathcurveto{\pgfqpoint{2.382032in}{1.854004in}}{\pgfqpoint{2.378759in}{1.861904in}}{\pgfqpoint{2.372935in}{1.867728in}}%
\pgfpathcurveto{\pgfqpoint{2.367112in}{1.873552in}}{\pgfqpoint{2.359212in}{1.876824in}}{\pgfqpoint{2.350975in}{1.876824in}}%
\pgfpathcurveto{\pgfqpoint{2.342739in}{1.876824in}}{\pgfqpoint{2.334839in}{1.873552in}}{\pgfqpoint{2.329015in}{1.867728in}}%
\pgfpathcurveto{\pgfqpoint{2.323191in}{1.861904in}}{\pgfqpoint{2.319919in}{1.854004in}}{\pgfqpoint{2.319919in}{1.845768in}}%
\pgfpathcurveto{\pgfqpoint{2.319919in}{1.837532in}}{\pgfqpoint{2.323191in}{1.829632in}}{\pgfqpoint{2.329015in}{1.823808in}}%
\pgfpathcurveto{\pgfqpoint{2.334839in}{1.817984in}}{\pgfqpoint{2.342739in}{1.814711in}}{\pgfqpoint{2.350975in}{1.814711in}}%
\pgfpathclose%
\pgfusepath{stroke,fill}%
\end{pgfscope}%
\begin{pgfscope}%
\pgfpathrectangle{\pgfqpoint{0.100000in}{0.212622in}}{\pgfqpoint{3.696000in}{3.696000in}}%
\pgfusepath{clip}%
\pgfsetbuttcap%
\pgfsetroundjoin%
\definecolor{currentfill}{rgb}{0.121569,0.466667,0.705882}%
\pgfsetfillcolor{currentfill}%
\pgfsetfillopacity{0.998561}%
\pgfsetlinewidth{1.003750pt}%
\definecolor{currentstroke}{rgb}{0.121569,0.466667,0.705882}%
\pgfsetstrokecolor{currentstroke}%
\pgfsetstrokeopacity{0.998561}%
\pgfsetdash{}{0pt}%
\pgfpathmoveto{\pgfqpoint{2.393989in}{1.807084in}}%
\pgfpathcurveto{\pgfqpoint{2.402226in}{1.807084in}}{\pgfqpoint{2.410126in}{1.810356in}}{\pgfqpoint{2.415949in}{1.816180in}}%
\pgfpathcurveto{\pgfqpoint{2.421773in}{1.822004in}}{\pgfqpoint{2.425046in}{1.829904in}}{\pgfqpoint{2.425046in}{1.838141in}}%
\pgfpathcurveto{\pgfqpoint{2.425046in}{1.846377in}}{\pgfqpoint{2.421773in}{1.854277in}}{\pgfqpoint{2.415949in}{1.860101in}}%
\pgfpathcurveto{\pgfqpoint{2.410126in}{1.865925in}}{\pgfqpoint{2.402226in}{1.869197in}}{\pgfqpoint{2.393989in}{1.869197in}}%
\pgfpathcurveto{\pgfqpoint{2.385753in}{1.869197in}}{\pgfqpoint{2.377853in}{1.865925in}}{\pgfqpoint{2.372029in}{1.860101in}}%
\pgfpathcurveto{\pgfqpoint{2.366205in}{1.854277in}}{\pgfqpoint{2.362933in}{1.846377in}}{\pgfqpoint{2.362933in}{1.838141in}}%
\pgfpathcurveto{\pgfqpoint{2.362933in}{1.829904in}}{\pgfqpoint{2.366205in}{1.822004in}}{\pgfqpoint{2.372029in}{1.816180in}}%
\pgfpathcurveto{\pgfqpoint{2.377853in}{1.810356in}}{\pgfqpoint{2.385753in}{1.807084in}}{\pgfqpoint{2.393989in}{1.807084in}}%
\pgfpathclose%
\pgfusepath{stroke,fill}%
\end{pgfscope}%
\begin{pgfscope}%
\pgfpathrectangle{\pgfqpoint{0.100000in}{0.212622in}}{\pgfqpoint{3.696000in}{3.696000in}}%
\pgfusepath{clip}%
\pgfsetbuttcap%
\pgfsetroundjoin%
\definecolor{currentfill}{rgb}{0.121569,0.466667,0.705882}%
\pgfsetfillcolor{currentfill}%
\pgfsetfillopacity{0.999323}%
\pgfsetlinewidth{1.003750pt}%
\definecolor{currentstroke}{rgb}{0.121569,0.466667,0.705882}%
\pgfsetstrokecolor{currentstroke}%
\pgfsetstrokeopacity{0.999323}%
\pgfsetdash{}{0pt}%
\pgfpathmoveto{\pgfqpoint{2.389510in}{1.806872in}}%
\pgfpathcurveto{\pgfqpoint{2.397746in}{1.806872in}}{\pgfqpoint{2.405646in}{1.810144in}}{\pgfqpoint{2.411470in}{1.815968in}}%
\pgfpathcurveto{\pgfqpoint{2.417294in}{1.821792in}}{\pgfqpoint{2.420567in}{1.829692in}}{\pgfqpoint{2.420567in}{1.837929in}}%
\pgfpathcurveto{\pgfqpoint{2.420567in}{1.846165in}}{\pgfqpoint{2.417294in}{1.854065in}}{\pgfqpoint{2.411470in}{1.859889in}}%
\pgfpathcurveto{\pgfqpoint{2.405646in}{1.865713in}}{\pgfqpoint{2.397746in}{1.868985in}}{\pgfqpoint{2.389510in}{1.868985in}}%
\pgfpathcurveto{\pgfqpoint{2.381274in}{1.868985in}}{\pgfqpoint{2.373374in}{1.865713in}}{\pgfqpoint{2.367550in}{1.859889in}}%
\pgfpathcurveto{\pgfqpoint{2.361726in}{1.854065in}}{\pgfqpoint{2.358454in}{1.846165in}}{\pgfqpoint{2.358454in}{1.837929in}}%
\pgfpathcurveto{\pgfqpoint{2.358454in}{1.829692in}}{\pgfqpoint{2.361726in}{1.821792in}}{\pgfqpoint{2.367550in}{1.815968in}}%
\pgfpathcurveto{\pgfqpoint{2.373374in}{1.810144in}}{\pgfqpoint{2.381274in}{1.806872in}}{\pgfqpoint{2.389510in}{1.806872in}}%
\pgfpathclose%
\pgfusepath{stroke,fill}%
\end{pgfscope}%
\begin{pgfscope}%
\pgfpathrectangle{\pgfqpoint{0.100000in}{0.212622in}}{\pgfqpoint{3.696000in}{3.696000in}}%
\pgfusepath{clip}%
\pgfsetbuttcap%
\pgfsetroundjoin%
\definecolor{currentfill}{rgb}{0.121569,0.466667,0.705882}%
\pgfsetfillcolor{currentfill}%
\pgfsetfillopacity{0.999777}%
\pgfsetlinewidth{1.003750pt}%
\definecolor{currentstroke}{rgb}{0.121569,0.466667,0.705882}%
\pgfsetstrokecolor{currentstroke}%
\pgfsetstrokeopacity{0.999777}%
\pgfsetdash{}{0pt}%
\pgfpathmoveto{\pgfqpoint{2.381163in}{1.805893in}}%
\pgfpathcurveto{\pgfqpoint{2.389399in}{1.805893in}}{\pgfqpoint{2.397299in}{1.809166in}}{\pgfqpoint{2.403123in}{1.814990in}}%
\pgfpathcurveto{\pgfqpoint{2.408947in}{1.820814in}}{\pgfqpoint{2.412219in}{1.828714in}}{\pgfqpoint{2.412219in}{1.836950in}}%
\pgfpathcurveto{\pgfqpoint{2.412219in}{1.845186in}}{\pgfqpoint{2.408947in}{1.853086in}}{\pgfqpoint{2.403123in}{1.858910in}}%
\pgfpathcurveto{\pgfqpoint{2.397299in}{1.864734in}}{\pgfqpoint{2.389399in}{1.868006in}}{\pgfqpoint{2.381163in}{1.868006in}}%
\pgfpathcurveto{\pgfqpoint{2.372926in}{1.868006in}}{\pgfqpoint{2.365026in}{1.864734in}}{\pgfqpoint{2.359203in}{1.858910in}}%
\pgfpathcurveto{\pgfqpoint{2.353379in}{1.853086in}}{\pgfqpoint{2.350106in}{1.845186in}}{\pgfqpoint{2.350106in}{1.836950in}}%
\pgfpathcurveto{\pgfqpoint{2.350106in}{1.828714in}}{\pgfqpoint{2.353379in}{1.820814in}}{\pgfqpoint{2.359203in}{1.814990in}}%
\pgfpathcurveto{\pgfqpoint{2.365026in}{1.809166in}}{\pgfqpoint{2.372926in}{1.805893in}}{\pgfqpoint{2.381163in}{1.805893in}}%
\pgfpathclose%
\pgfusepath{stroke,fill}%
\end{pgfscope}%
\begin{pgfscope}%
\pgfpathrectangle{\pgfqpoint{0.100000in}{0.212622in}}{\pgfqpoint{3.696000in}{3.696000in}}%
\pgfusepath{clip}%
\pgfsetbuttcap%
\pgfsetroundjoin%
\definecolor{currentfill}{rgb}{0.121569,0.466667,0.705882}%
\pgfsetfillcolor{currentfill}%
\pgfsetfillopacity{1.000000}%
\pgfsetlinewidth{1.003750pt}%
\definecolor{currentstroke}{rgb}{0.121569,0.466667,0.705882}%
\pgfsetstrokecolor{currentstroke}%
\pgfsetstrokeopacity{1.000000}%
\pgfsetdash{}{0pt}%
\pgfpathmoveto{\pgfqpoint{2.366282in}{1.808468in}}%
\pgfpathcurveto{\pgfqpoint{2.374518in}{1.808468in}}{\pgfqpoint{2.382418in}{1.811741in}}{\pgfqpoint{2.388242in}{1.817565in}}%
\pgfpathcurveto{\pgfqpoint{2.394066in}{1.823388in}}{\pgfqpoint{2.397339in}{1.831289in}}{\pgfqpoint{2.397339in}{1.839525in}}%
\pgfpathcurveto{\pgfqpoint{2.397339in}{1.847761in}}{\pgfqpoint{2.394066in}{1.855661in}}{\pgfqpoint{2.388242in}{1.861485in}}%
\pgfpathcurveto{\pgfqpoint{2.382418in}{1.867309in}}{\pgfqpoint{2.374518in}{1.870581in}}{\pgfqpoint{2.366282in}{1.870581in}}%
\pgfpathcurveto{\pgfqpoint{2.358046in}{1.870581in}}{\pgfqpoint{2.350146in}{1.867309in}}{\pgfqpoint{2.344322in}{1.861485in}}%
\pgfpathcurveto{\pgfqpoint{2.338498in}{1.855661in}}{\pgfqpoint{2.335226in}{1.847761in}}{\pgfqpoint{2.335226in}{1.839525in}}%
\pgfpathcurveto{\pgfqpoint{2.335226in}{1.831289in}}{\pgfqpoint{2.338498in}{1.823388in}}{\pgfqpoint{2.344322in}{1.817565in}}%
\pgfpathcurveto{\pgfqpoint{2.350146in}{1.811741in}}{\pgfqpoint{2.358046in}{1.808468in}}{\pgfqpoint{2.366282in}{1.808468in}}%
\pgfpathclose%
\pgfusepath{stroke,fill}%
\end{pgfscope}%
\begin{pgfscope}%
\pgfpathrectangle{\pgfqpoint{0.100000in}{0.212622in}}{\pgfqpoint{3.696000in}{3.696000in}}%
\pgfusepath{clip}%
\pgfsetbuttcap%
\pgfsetroundjoin%
\definecolor{currentfill}{rgb}{0.121569,0.466667,0.705882}%
\pgfsetfillcolor{currentfill}%
\pgfsetlinewidth{1.003750pt}%
\definecolor{currentstroke}{rgb}{0.121569,0.466667,0.705882}%
\pgfsetstrokecolor{currentstroke}%
\pgfsetdash{}{0pt}%
\pgfpathmoveto{\pgfqpoint{2.376496in}{1.806361in}}%
\pgfpathcurveto{\pgfqpoint{2.384733in}{1.806361in}}{\pgfqpoint{2.392633in}{1.809633in}}{\pgfqpoint{2.398457in}{1.815457in}}%
\pgfpathcurveto{\pgfqpoint{2.404281in}{1.821281in}}{\pgfqpoint{2.407553in}{1.829181in}}{\pgfqpoint{2.407553in}{1.837417in}}%
\pgfpathcurveto{\pgfqpoint{2.407553in}{1.845654in}}{\pgfqpoint{2.404281in}{1.853554in}}{\pgfqpoint{2.398457in}{1.859378in}}%
\pgfpathcurveto{\pgfqpoint{2.392633in}{1.865202in}}{\pgfqpoint{2.384733in}{1.868474in}}{\pgfqpoint{2.376496in}{1.868474in}}%
\pgfpathcurveto{\pgfqpoint{2.368260in}{1.868474in}}{\pgfqpoint{2.360360in}{1.865202in}}{\pgfqpoint{2.354536in}{1.859378in}}%
\pgfpathcurveto{\pgfqpoint{2.348712in}{1.853554in}}{\pgfqpoint{2.345440in}{1.845654in}}{\pgfqpoint{2.345440in}{1.837417in}}%
\pgfpathcurveto{\pgfqpoint{2.345440in}{1.829181in}}{\pgfqpoint{2.348712in}{1.821281in}}{\pgfqpoint{2.354536in}{1.815457in}}%
\pgfpathcurveto{\pgfqpoint{2.360360in}{1.809633in}}{\pgfqpoint{2.368260in}{1.806361in}}{\pgfqpoint{2.376496in}{1.806361in}}%
\pgfpathclose%
\pgfusepath{stroke,fill}%
\end{pgfscope}%
\begin{pgfscope}%
\definecolor{textcolor}{rgb}{0.000000,0.000000,0.000000}%
\pgfsetstrokecolor{textcolor}%
\pgfsetfillcolor{textcolor}%
\pgftext[x=1.948000in,y=3.991956in,,base]{\color{textcolor}\rmfamily\fontsize{12.000000}{14.400000}\selectfont Mahony}%
\end{pgfscope}%
\begin{pgfscope}%
\pgfsetbuttcap%
\pgfsetmiterjoin%
\definecolor{currentfill}{rgb}{1.000000,1.000000,1.000000}%
\pgfsetfillcolor{currentfill}%
\pgfsetfillopacity{0.800000}%
\pgfsetlinewidth{1.003750pt}%
\definecolor{currentstroke}{rgb}{0.800000,0.800000,0.800000}%
\pgfsetstrokecolor{currentstroke}%
\pgfsetstrokeopacity{0.800000}%
\pgfsetdash{}{0pt}%
\pgfpathmoveto{\pgfqpoint{2.104889in}{3.410289in}}%
\pgfpathlineto{\pgfqpoint{3.698778in}{3.410289in}}%
\pgfpathquadraticcurveto{\pgfqpoint{3.726556in}{3.410289in}}{\pgfqpoint{3.726556in}{3.438067in}}%
\pgfpathlineto{\pgfqpoint{3.726556in}{3.811400in}}%
\pgfpathquadraticcurveto{\pgfqpoint{3.726556in}{3.839178in}}{\pgfqpoint{3.698778in}{3.839178in}}%
\pgfpathlineto{\pgfqpoint{2.104889in}{3.839178in}}%
\pgfpathquadraticcurveto{\pgfqpoint{2.077111in}{3.839178in}}{\pgfqpoint{2.077111in}{3.811400in}}%
\pgfpathlineto{\pgfqpoint{2.077111in}{3.438067in}}%
\pgfpathquadraticcurveto{\pgfqpoint{2.077111in}{3.410289in}}{\pgfqpoint{2.104889in}{3.410289in}}%
\pgfpathclose%
\pgfusepath{stroke,fill}%
\end{pgfscope}%
\begin{pgfscope}%
\pgfsetrectcap%
\pgfsetroundjoin%
\pgfsetlinewidth{1.505625pt}%
\definecolor{currentstroke}{rgb}{0.121569,0.466667,0.705882}%
\pgfsetstrokecolor{currentstroke}%
\pgfsetdash{}{0pt}%
\pgfpathmoveto{\pgfqpoint{2.132667in}{3.735011in}}%
\pgfpathlineto{\pgfqpoint{2.410444in}{3.735011in}}%
\pgfusepath{stroke}%
\end{pgfscope}%
\begin{pgfscope}%
\definecolor{textcolor}{rgb}{0.000000,0.000000,0.000000}%
\pgfsetstrokecolor{textcolor}%
\pgfsetfillcolor{textcolor}%
\pgftext[x=2.521555in,y=3.686400in,left,base]{\color{textcolor}\rmfamily\fontsize{10.000000}{12.000000}\selectfont Ground truth}%
\end{pgfscope}%
\begin{pgfscope}%
\pgfsetbuttcap%
\pgfsetroundjoin%
\definecolor{currentfill}{rgb}{0.121569,0.466667,0.705882}%
\pgfsetfillcolor{currentfill}%
\pgfsetlinewidth{1.003750pt}%
\definecolor{currentstroke}{rgb}{0.121569,0.466667,0.705882}%
\pgfsetstrokecolor{currentstroke}%
\pgfsetdash{}{0pt}%
\pgfsys@defobject{currentmarker}{\pgfqpoint{-0.031056in}{-0.031056in}}{\pgfqpoint{0.031056in}{0.031056in}}{%
\pgfpathmoveto{\pgfqpoint{0.000000in}{-0.031056in}}%
\pgfpathcurveto{\pgfqpoint{0.008236in}{-0.031056in}}{\pgfqpoint{0.016136in}{-0.027784in}}{\pgfqpoint{0.021960in}{-0.021960in}}%
\pgfpathcurveto{\pgfqpoint{0.027784in}{-0.016136in}}{\pgfqpoint{0.031056in}{-0.008236in}}{\pgfqpoint{0.031056in}{0.000000in}}%
\pgfpathcurveto{\pgfqpoint{0.031056in}{0.008236in}}{\pgfqpoint{0.027784in}{0.016136in}}{\pgfqpoint{0.021960in}{0.021960in}}%
\pgfpathcurveto{\pgfqpoint{0.016136in}{0.027784in}}{\pgfqpoint{0.008236in}{0.031056in}}{\pgfqpoint{0.000000in}{0.031056in}}%
\pgfpathcurveto{\pgfqpoint{-0.008236in}{0.031056in}}{\pgfqpoint{-0.016136in}{0.027784in}}{\pgfqpoint{-0.021960in}{0.021960in}}%
\pgfpathcurveto{\pgfqpoint{-0.027784in}{0.016136in}}{\pgfqpoint{-0.031056in}{0.008236in}}{\pgfqpoint{-0.031056in}{0.000000in}}%
\pgfpathcurveto{\pgfqpoint{-0.031056in}{-0.008236in}}{\pgfqpoint{-0.027784in}{-0.016136in}}{\pgfqpoint{-0.021960in}{-0.021960in}}%
\pgfpathcurveto{\pgfqpoint{-0.016136in}{-0.027784in}}{\pgfqpoint{-0.008236in}{-0.031056in}}{\pgfqpoint{0.000000in}{-0.031056in}}%
\pgfpathclose%
\pgfusepath{stroke,fill}%
}%
\begin{pgfscope}%
\pgfsys@transformshift{2.271555in}{3.529248in}%
\pgfsys@useobject{currentmarker}{}%
\end{pgfscope}%
\end{pgfscope}%
\begin{pgfscope}%
\definecolor{textcolor}{rgb}{0.000000,0.000000,0.000000}%
\pgfsetstrokecolor{textcolor}%
\pgfsetfillcolor{textcolor}%
\pgftext[x=2.521555in,y=3.492789in,left,base]{\color{textcolor}\rmfamily\fontsize{10.000000}{12.000000}\selectfont Estimated position}%
\end{pgfscope}%
\end{pgfpicture}%
\makeatother%
\endgroup%
}
%         \caption{INS Hardware}
%         \label{fig:square42D}
%     \end{subfigure}
%     \begin{subfigure}{0.49\textwidth}
%         \centering
%         \resizebox{1\linewidth}{!}{%% Creator: Matplotlib, PGF backend
%%
%% To include the figure in your LaTeX document, write
%%   \input{<filename>.pgf}
%%
%% Make sure the required packages are loaded in your preamble
%%   \usepackage{pgf}
%%
%% and, on pdftex
%%   \usepackage[utf8]{inputenc}\DeclareUnicodeCharacter{2212}{-}
%%
%% or, on luatex and xetex
%%   \usepackage{unicode-math}
%%
%% Figures using additional raster images can only be included by \input if
%% they are in the same directory as the main LaTeX file. For loading figures
%% from other directories you can use the `import` package
%%   \usepackage{import}
%%
%% and then include the figures with
%%   \import{<path to file>}{<filename>.pgf}
%%
%% Matplotlib used the following preamble
%%   \usepackage{fontspec}
%%   \setmainfont{DejaVuSerif.ttf}[Path=C:/Users/Claudio/AppData/Local/Programs/Python/Python39/Lib/site-packages/matplotlib/mpl-data/fonts/ttf/]
%%   \setsansfont{DejaVuSans.ttf}[Path=C:/Users/Claudio/AppData/Local/Programs/Python/Python39/Lib/site-packages/matplotlib/mpl-data/fonts/ttf/]
%%   \setmonofont{DejaVuSansMono.ttf}[Path=C:/Users/Claudio/AppData/Local/Programs/Python/Python39/Lib/site-packages/matplotlib/mpl-data/fonts/ttf/]
%%
\begingroup%
\makeatletter%
\begin{pgfpicture}%
\pgfpathrectangle{\pgfpointorigin}{\pgfqpoint{4.342069in}{4.226689in}}%
\pgfusepath{use as bounding box, clip}%
\begin{pgfscope}%
\pgfsetbuttcap%
\pgfsetmiterjoin%
\definecolor{currentfill}{rgb}{1.000000,1.000000,1.000000}%
\pgfsetfillcolor{currentfill}%
\pgfsetlinewidth{0.000000pt}%
\definecolor{currentstroke}{rgb}{1.000000,1.000000,1.000000}%
\pgfsetstrokecolor{currentstroke}%
\pgfsetdash{}{0pt}%
\pgfpathmoveto{\pgfqpoint{0.000000in}{0.000000in}}%
\pgfpathlineto{\pgfqpoint{4.342069in}{0.000000in}}%
\pgfpathlineto{\pgfqpoint{4.342069in}{4.226689in}}%
\pgfpathlineto{\pgfqpoint{0.000000in}{4.226689in}}%
\pgfpathclose%
\pgfusepath{fill}%
\end{pgfscope}%
\begin{pgfscope}%
\pgfsetbuttcap%
\pgfsetmiterjoin%
\definecolor{currentfill}{rgb}{1.000000,1.000000,1.000000}%
\pgfsetfillcolor{currentfill}%
\pgfsetlinewidth{0.000000pt}%
\definecolor{currentstroke}{rgb}{0.000000,0.000000,0.000000}%
\pgfsetstrokecolor{currentstroke}%
\pgfsetstrokeopacity{0.000000}%
\pgfsetdash{}{0pt}%
\pgfpathmoveto{\pgfqpoint{0.100000in}{0.220728in}}%
\pgfpathlineto{\pgfqpoint{3.796000in}{0.220728in}}%
\pgfpathlineto{\pgfqpoint{3.796000in}{3.916728in}}%
\pgfpathlineto{\pgfqpoint{0.100000in}{3.916728in}}%
\pgfpathclose%
\pgfusepath{fill}%
\end{pgfscope}%
\begin{pgfscope}%
\pgfsetbuttcap%
\pgfsetmiterjoin%
\definecolor{currentfill}{rgb}{0.950000,0.950000,0.950000}%
\pgfsetfillcolor{currentfill}%
\pgfsetfillopacity{0.500000}%
\pgfsetlinewidth{1.003750pt}%
\definecolor{currentstroke}{rgb}{0.950000,0.950000,0.950000}%
\pgfsetstrokecolor{currentstroke}%
\pgfsetstrokeopacity{0.500000}%
\pgfsetdash{}{0pt}%
\pgfpathmoveto{\pgfqpoint{0.379073in}{1.132043in}}%
\pgfpathlineto{\pgfqpoint{1.599613in}{2.155124in}}%
\pgfpathlineto{\pgfqpoint{1.582647in}{3.630589in}}%
\pgfpathlineto{\pgfqpoint{0.303698in}{2.697271in}}%
\pgfusepath{stroke,fill}%
\end{pgfscope}%
\begin{pgfscope}%
\pgfsetbuttcap%
\pgfsetmiterjoin%
\definecolor{currentfill}{rgb}{0.900000,0.900000,0.900000}%
\pgfsetfillcolor{currentfill}%
\pgfsetfillopacity{0.500000}%
\pgfsetlinewidth{1.003750pt}%
\definecolor{currentstroke}{rgb}{0.900000,0.900000,0.900000}%
\pgfsetstrokecolor{currentstroke}%
\pgfsetstrokeopacity{0.500000}%
\pgfsetdash{}{0pt}%
\pgfpathmoveto{\pgfqpoint{1.599613in}{2.155124in}}%
\pgfpathlineto{\pgfqpoint{3.558144in}{1.585856in}}%
\pgfpathlineto{\pgfqpoint{3.628038in}{3.112142in}}%
\pgfpathlineto{\pgfqpoint{1.582647in}{3.630589in}}%
\pgfusepath{stroke,fill}%
\end{pgfscope}%
\begin{pgfscope}%
\pgfsetbuttcap%
\pgfsetmiterjoin%
\definecolor{currentfill}{rgb}{0.925000,0.925000,0.925000}%
\pgfsetfillcolor{currentfill}%
\pgfsetfillopacity{0.500000}%
\pgfsetlinewidth{1.003750pt}%
\definecolor{currentstroke}{rgb}{0.925000,0.925000,0.925000}%
\pgfsetstrokecolor{currentstroke}%
\pgfsetstrokeopacity{0.500000}%
\pgfsetdash{}{0pt}%
\pgfpathmoveto{\pgfqpoint{0.379073in}{1.132043in}}%
\pgfpathlineto{\pgfqpoint{2.455212in}{0.453976in}}%
\pgfpathlineto{\pgfqpoint{3.558144in}{1.585856in}}%
\pgfpathlineto{\pgfqpoint{1.599613in}{2.155124in}}%
\pgfusepath{stroke,fill}%
\end{pgfscope}%
\begin{pgfscope}%
\pgfsetrectcap%
\pgfsetroundjoin%
\pgfsetlinewidth{0.803000pt}%
\definecolor{currentstroke}{rgb}{0.000000,0.000000,0.000000}%
\pgfsetstrokecolor{currentstroke}%
\pgfsetdash{}{0pt}%
\pgfpathmoveto{\pgfqpoint{0.379073in}{1.132043in}}%
\pgfpathlineto{\pgfqpoint{2.455212in}{0.453976in}}%
\pgfusepath{stroke}%
\end{pgfscope}%
\begin{pgfscope}%
\definecolor{textcolor}{rgb}{0.000000,0.000000,0.000000}%
\pgfsetstrokecolor{textcolor}%
\pgfsetfillcolor{textcolor}%
\pgftext[x=0.697927in, y=0.423808in, left, base,rotate=341.912962]{\color{textcolor}\sffamily\fontsize{10.000000}{12.000000}\selectfont Position X [\(\displaystyle m\)]}%
\end{pgfscope}%
\begin{pgfscope}%
\pgfsetbuttcap%
\pgfsetroundjoin%
\pgfsetlinewidth{0.803000pt}%
\definecolor{currentstroke}{rgb}{0.690196,0.690196,0.690196}%
\pgfsetstrokecolor{currentstroke}%
\pgfsetdash{}{0pt}%
\pgfpathmoveto{\pgfqpoint{0.648810in}{1.043947in}}%
\pgfpathlineto{\pgfqpoint{1.855049in}{2.080879in}}%
\pgfpathlineto{\pgfqpoint{1.848922in}{3.563096in}}%
\pgfusepath{stroke}%
\end{pgfscope}%
\begin{pgfscope}%
\pgfsetbuttcap%
\pgfsetroundjoin%
\pgfsetlinewidth{0.803000pt}%
\definecolor{currentstroke}{rgb}{0.690196,0.690196,0.690196}%
\pgfsetstrokecolor{currentstroke}%
\pgfsetdash{}{0pt}%
\pgfpathmoveto{\pgfqpoint{0.995809in}{0.930617in}}%
\pgfpathlineto{\pgfqpoint{2.183220in}{1.985493in}}%
\pgfpathlineto{\pgfqpoint{2.191233in}{3.476330in}}%
\pgfusepath{stroke}%
\end{pgfscope}%
\begin{pgfscope}%
\pgfsetbuttcap%
\pgfsetroundjoin%
\pgfsetlinewidth{0.803000pt}%
\definecolor{currentstroke}{rgb}{0.690196,0.690196,0.690196}%
\pgfsetstrokecolor{currentstroke}%
\pgfsetdash{}{0pt}%
\pgfpathmoveto{\pgfqpoint{1.349102in}{0.815232in}}%
\pgfpathlineto{\pgfqpoint{2.516845in}{1.888521in}}%
\pgfpathlineto{\pgfqpoint{2.539482in}{3.388059in}}%
\pgfusepath{stroke}%
\end{pgfscope}%
\begin{pgfscope}%
\pgfsetbuttcap%
\pgfsetroundjoin%
\pgfsetlinewidth{0.803000pt}%
\definecolor{currentstroke}{rgb}{0.690196,0.690196,0.690196}%
\pgfsetstrokecolor{currentstroke}%
\pgfsetdash{}{0pt}%
\pgfpathmoveto{\pgfqpoint{1.708862in}{0.697734in}}%
\pgfpathlineto{\pgfqpoint{2.856063in}{1.789924in}}%
\pgfpathlineto{\pgfqpoint{2.893826in}{3.298244in}}%
\pgfusepath{stroke}%
\end{pgfscope}%
\begin{pgfscope}%
\pgfsetbuttcap%
\pgfsetroundjoin%
\pgfsetlinewidth{0.803000pt}%
\definecolor{currentstroke}{rgb}{0.690196,0.690196,0.690196}%
\pgfsetstrokecolor{currentstroke}%
\pgfsetdash{}{0pt}%
\pgfpathmoveto{\pgfqpoint{2.075268in}{0.578066in}}%
\pgfpathlineto{\pgfqpoint{3.201014in}{1.689660in}}%
\pgfpathlineto{\pgfqpoint{3.254425in}{3.206842in}}%
\pgfusepath{stroke}%
\end{pgfscope}%
\begin{pgfscope}%
\pgfsetrectcap%
\pgfsetroundjoin%
\pgfsetlinewidth{0.803000pt}%
\definecolor{currentstroke}{rgb}{0.000000,0.000000,0.000000}%
\pgfsetstrokecolor{currentstroke}%
\pgfsetdash{}{0pt}%
\pgfpathmoveto{\pgfqpoint{0.659317in}{1.052979in}}%
\pgfpathlineto{\pgfqpoint{0.627751in}{1.025844in}}%
\pgfusepath{stroke}%
\end{pgfscope}%
\begin{pgfscope}%
\definecolor{textcolor}{rgb}{0.000000,0.000000,0.000000}%
\pgfsetstrokecolor{textcolor}%
\pgfsetfillcolor{textcolor}%
\pgftext[x=0.544384in,y=0.824747in,,top]{\color{textcolor}\sffamily\fontsize{10.000000}{12.000000}\selectfont 0}%
\end{pgfscope}%
\begin{pgfscope}%
\pgfsetrectcap%
\pgfsetroundjoin%
\pgfsetlinewidth{0.803000pt}%
\definecolor{currentstroke}{rgb}{0.000000,0.000000,0.000000}%
\pgfsetstrokecolor{currentstroke}%
\pgfsetdash{}{0pt}%
\pgfpathmoveto{\pgfqpoint{1.006160in}{0.939813in}}%
\pgfpathlineto{\pgfqpoint{0.975063in}{0.912187in}}%
\pgfusepath{stroke}%
\end{pgfscope}%
\begin{pgfscope}%
\definecolor{textcolor}{rgb}{0.000000,0.000000,0.000000}%
\pgfsetstrokecolor{textcolor}%
\pgfsetfillcolor{textcolor}%
\pgftext[x=0.891748in,y=0.709013in,,top]{\color{textcolor}\sffamily\fontsize{10.000000}{12.000000}\selectfont 1}%
\end{pgfscope}%
\begin{pgfscope}%
\pgfsetrectcap%
\pgfsetroundjoin%
\pgfsetlinewidth{0.803000pt}%
\definecolor{currentstroke}{rgb}{0.000000,0.000000,0.000000}%
\pgfsetstrokecolor{currentstroke}%
\pgfsetdash{}{0pt}%
\pgfpathmoveto{\pgfqpoint{1.359289in}{0.824595in}}%
\pgfpathlineto{\pgfqpoint{1.328684in}{0.796465in}}%
\pgfusepath{stroke}%
\end{pgfscope}%
\begin{pgfscope}%
\definecolor{textcolor}{rgb}{0.000000,0.000000,0.000000}%
\pgfsetstrokecolor{textcolor}%
\pgfsetfillcolor{textcolor}%
\pgftext[x=1.245436in,y=0.591172in,,top]{\color{textcolor}\sffamily\fontsize{10.000000}{12.000000}\selectfont 2}%
\end{pgfscope}%
\begin{pgfscope}%
\pgfsetrectcap%
\pgfsetroundjoin%
\pgfsetlinewidth{0.803000pt}%
\definecolor{currentstroke}{rgb}{0.000000,0.000000,0.000000}%
\pgfsetstrokecolor{currentstroke}%
\pgfsetdash{}{0pt}%
\pgfpathmoveto{\pgfqpoint{1.718877in}{0.707269in}}%
\pgfpathlineto{\pgfqpoint{1.688788in}{0.678622in}}%
\pgfusepath{stroke}%
\end{pgfscope}%
\begin{pgfscope}%
\definecolor{textcolor}{rgb}{0.000000,0.000000,0.000000}%
\pgfsetstrokecolor{textcolor}%
\pgfsetfillcolor{textcolor}%
\pgftext[x=1.605622in,y=0.471165in,,top]{\color{textcolor}\sffamily\fontsize{10.000000}{12.000000}\selectfont 3}%
\end{pgfscope}%
\begin{pgfscope}%
\pgfsetrectcap%
\pgfsetroundjoin%
\pgfsetlinewidth{0.803000pt}%
\definecolor{currentstroke}{rgb}{0.000000,0.000000,0.000000}%
\pgfsetstrokecolor{currentstroke}%
\pgfsetdash{}{0pt}%
\pgfpathmoveto{\pgfqpoint{2.085104in}{0.587778in}}%
\pgfpathlineto{\pgfqpoint{2.055553in}{0.558599in}}%
\pgfusepath{stroke}%
\end{pgfscope}%
\begin{pgfscope}%
\definecolor{textcolor}{rgb}{0.000000,0.000000,0.000000}%
\pgfsetstrokecolor{textcolor}%
\pgfsetfillcolor{textcolor}%
\pgftext[x=1.972489in,y=0.348933in,,top]{\color{textcolor}\sffamily\fontsize{10.000000}{12.000000}\selectfont 4}%
\end{pgfscope}%
\begin{pgfscope}%
\pgfsetrectcap%
\pgfsetroundjoin%
\pgfsetlinewidth{0.803000pt}%
\definecolor{currentstroke}{rgb}{0.000000,0.000000,0.000000}%
\pgfsetstrokecolor{currentstroke}%
\pgfsetdash{}{0pt}%
\pgfpathmoveto{\pgfqpoint{3.558144in}{1.585856in}}%
\pgfpathlineto{\pgfqpoint{2.455212in}{0.453976in}}%
\pgfusepath{stroke}%
\end{pgfscope}%
\begin{pgfscope}%
\definecolor{textcolor}{rgb}{0.000000,0.000000,0.000000}%
\pgfsetstrokecolor{textcolor}%
\pgfsetfillcolor{textcolor}%
\pgftext[x=3.103916in, y=0.291339in, left, base,rotate=45.742112]{\color{textcolor}\sffamily\fontsize{10.000000}{12.000000}\selectfont Position Y [\(\displaystyle m\)]}%
\end{pgfscope}%
\begin{pgfscope}%
\pgfsetbuttcap%
\pgfsetroundjoin%
\pgfsetlinewidth{0.803000pt}%
\definecolor{currentstroke}{rgb}{0.690196,0.690196,0.690196}%
\pgfsetstrokecolor{currentstroke}%
\pgfsetdash{}{0pt}%
\pgfpathmoveto{\pgfqpoint{0.529420in}{2.861993in}}%
\pgfpathlineto{\pgfqpoint{0.593815in}{1.312044in}}%
\pgfpathlineto{\pgfqpoint{2.649969in}{0.653844in}}%
\pgfusepath{stroke}%
\end{pgfscope}%
\begin{pgfscope}%
\pgfsetbuttcap%
\pgfsetroundjoin%
\pgfsetlinewidth{0.803000pt}%
\definecolor{currentstroke}{rgb}{0.690196,0.690196,0.690196}%
\pgfsetstrokecolor{currentstroke}%
\pgfsetdash{}{0pt}%
\pgfpathmoveto{\pgfqpoint{0.772526in}{3.039400in}}%
\pgfpathlineto{\pgfqpoint{0.825415in}{1.506176in}}%
\pgfpathlineto{\pgfqpoint{2.859676in}{0.869056in}}%
\pgfusepath{stroke}%
\end{pgfscope}%
\begin{pgfscope}%
\pgfsetbuttcap%
\pgfsetroundjoin%
\pgfsetlinewidth{0.803000pt}%
\definecolor{currentstroke}{rgb}{0.690196,0.690196,0.690196}%
\pgfsetstrokecolor{currentstroke}%
\pgfsetdash{}{0pt}%
\pgfpathmoveto{\pgfqpoint{1.007768in}{3.211069in}}%
\pgfpathlineto{\pgfqpoint{1.049842in}{1.694295in}}%
\pgfpathlineto{\pgfqpoint{3.062553in}{1.077257in}}%
\pgfusepath{stroke}%
\end{pgfscope}%
\begin{pgfscope}%
\pgfsetbuttcap%
\pgfsetroundjoin%
\pgfsetlinewidth{0.803000pt}%
\definecolor{currentstroke}{rgb}{0.690196,0.690196,0.690196}%
\pgfsetstrokecolor{currentstroke}%
\pgfsetdash{}{0pt}%
\pgfpathmoveto{\pgfqpoint{1.235521in}{3.377273in}}%
\pgfpathlineto{\pgfqpoint{1.267424in}{1.876676in}}%
\pgfpathlineto{\pgfqpoint{3.258927in}{1.278786in}}%
\pgfusepath{stroke}%
\end{pgfscope}%
\begin{pgfscope}%
\pgfsetbuttcap%
\pgfsetroundjoin%
\pgfsetlinewidth{0.803000pt}%
\definecolor{currentstroke}{rgb}{0.690196,0.690196,0.690196}%
\pgfsetstrokecolor{currentstroke}%
\pgfsetdash{}{0pt}%
\pgfpathmoveto{\pgfqpoint{1.456138in}{3.538269in}}%
\pgfpathlineto{\pgfqpoint{1.478469in}{2.053578in}}%
\pgfpathlineto{\pgfqpoint{3.449107in}{1.473957in}}%
\pgfusepath{stroke}%
\end{pgfscope}%
\begin{pgfscope}%
\pgfsetrectcap%
\pgfsetroundjoin%
\pgfsetlinewidth{0.803000pt}%
\definecolor{currentstroke}{rgb}{0.000000,0.000000,0.000000}%
\pgfsetstrokecolor{currentstroke}%
\pgfsetdash{}{0pt}%
\pgfpathmoveto{\pgfqpoint{2.632649in}{0.659388in}}%
\pgfpathlineto{\pgfqpoint{2.684651in}{0.642742in}}%
\pgfusepath{stroke}%
\end{pgfscope}%
\begin{pgfscope}%
\definecolor{textcolor}{rgb}{0.000000,0.000000,0.000000}%
\pgfsetstrokecolor{textcolor}%
\pgfsetfillcolor{textcolor}%
\pgftext[x=2.827278in,y=0.469210in,,top]{\color{textcolor}\sffamily\fontsize{10.000000}{12.000000}\selectfont 0}%
\end{pgfscope}%
\begin{pgfscope}%
\pgfsetrectcap%
\pgfsetroundjoin%
\pgfsetlinewidth{0.803000pt}%
\definecolor{currentstroke}{rgb}{0.000000,0.000000,0.000000}%
\pgfsetstrokecolor{currentstroke}%
\pgfsetdash{}{0pt}%
\pgfpathmoveto{\pgfqpoint{2.842555in}{0.874418in}}%
\pgfpathlineto{\pgfqpoint{2.893960in}{0.858318in}}%
\pgfusepath{stroke}%
\end{pgfscope}%
\begin{pgfscope}%
\definecolor{textcolor}{rgb}{0.000000,0.000000,0.000000}%
\pgfsetstrokecolor{textcolor}%
\pgfsetfillcolor{textcolor}%
\pgftext[x=3.034171in,y=0.687605in,,top]{\color{textcolor}\sffamily\fontsize{10.000000}{12.000000}\selectfont 1}%
\end{pgfscope}%
\begin{pgfscope}%
\pgfsetrectcap%
\pgfsetroundjoin%
\pgfsetlinewidth{0.803000pt}%
\definecolor{currentstroke}{rgb}{0.000000,0.000000,0.000000}%
\pgfsetstrokecolor{currentstroke}%
\pgfsetdash{}{0pt}%
\pgfpathmoveto{\pgfqpoint{3.045627in}{1.082446in}}%
\pgfpathlineto{\pgfqpoint{3.096446in}{1.066867in}}%
\pgfusepath{stroke}%
\end{pgfscope}%
\begin{pgfscope}%
\definecolor{textcolor}{rgb}{0.000000,0.000000,0.000000}%
\pgfsetstrokecolor{textcolor}%
\pgfsetfillcolor{textcolor}%
\pgftext[x=3.234321in,y=0.898883in,,top]{\color{textcolor}\sffamily\fontsize{10.000000}{12.000000}\selectfont 2}%
\end{pgfscope}%
\begin{pgfscope}%
\pgfsetrectcap%
\pgfsetroundjoin%
\pgfsetlinewidth{0.803000pt}%
\definecolor{currentstroke}{rgb}{0.000000,0.000000,0.000000}%
\pgfsetstrokecolor{currentstroke}%
\pgfsetdash{}{0pt}%
\pgfpathmoveto{\pgfqpoint{3.242193in}{1.283810in}}%
\pgfpathlineto{\pgfqpoint{3.292436in}{1.268726in}}%
\pgfusepath{stroke}%
\end{pgfscope}%
\begin{pgfscope}%
\definecolor{textcolor}{rgb}{0.000000,0.000000,0.000000}%
\pgfsetstrokecolor{textcolor}%
\pgfsetfillcolor{textcolor}%
\pgftext[x=3.428052in,y=1.103386in,,top]{\color{textcolor}\sffamily\fontsize{10.000000}{12.000000}\selectfont 3}%
\end{pgfscope}%
\begin{pgfscope}%
\pgfsetrectcap%
\pgfsetroundjoin%
\pgfsetlinewidth{0.803000pt}%
\definecolor{currentstroke}{rgb}{0.000000,0.000000,0.000000}%
\pgfsetstrokecolor{currentstroke}%
\pgfsetdash{}{0pt}%
\pgfpathmoveto{\pgfqpoint{3.432561in}{1.478824in}}%
\pgfpathlineto{\pgfqpoint{3.482240in}{1.464212in}}%
\pgfusepath{stroke}%
\end{pgfscope}%
\begin{pgfscope}%
\definecolor{textcolor}{rgb}{0.000000,0.000000,0.000000}%
\pgfsetstrokecolor{textcolor}%
\pgfsetfillcolor{textcolor}%
\pgftext[x=3.615669in,y=1.301434in,,top]{\color{textcolor}\sffamily\fontsize{10.000000}{12.000000}\selectfont 4}%
\end{pgfscope}%
\begin{pgfscope}%
\pgfsetrectcap%
\pgfsetroundjoin%
\pgfsetlinewidth{0.803000pt}%
\definecolor{currentstroke}{rgb}{0.000000,0.000000,0.000000}%
\pgfsetstrokecolor{currentstroke}%
\pgfsetdash{}{0pt}%
\pgfpathmoveto{\pgfqpoint{3.558144in}{1.585856in}}%
\pgfpathlineto{\pgfqpoint{3.628038in}{3.112142in}}%
\pgfusepath{stroke}%
\end{pgfscope}%
\begin{pgfscope}%
\definecolor{textcolor}{rgb}{0.000000,0.000000,0.000000}%
\pgfsetstrokecolor{textcolor}%
\pgfsetfillcolor{textcolor}%
\pgftext[x=4.169544in, y=1.928890in, left, base,rotate=87.378092]{\color{textcolor}\sffamily\fontsize{10.000000}{12.000000}\selectfont Position Z [\(\displaystyle m\)]}%
\end{pgfscope}%
\begin{pgfscope}%
\pgfsetbuttcap%
\pgfsetroundjoin%
\pgfsetlinewidth{0.803000pt}%
\definecolor{currentstroke}{rgb}{0.690196,0.690196,0.690196}%
\pgfsetstrokecolor{currentstroke}%
\pgfsetdash{}{0pt}%
\pgfpathmoveto{\pgfqpoint{3.566860in}{1.776185in}}%
\pgfpathlineto{\pgfqpoint{1.597494in}{2.339455in}}%
\pgfpathlineto{\pgfqpoint{0.369688in}{1.326940in}}%
\pgfusepath{stroke}%
\end{pgfscope}%
\begin{pgfscope}%
\pgfsetbuttcap%
\pgfsetroundjoin%
\pgfsetlinewidth{0.803000pt}%
\definecolor{currentstroke}{rgb}{0.690196,0.690196,0.690196}%
\pgfsetstrokecolor{currentstroke}%
\pgfsetdash{}{0pt}%
\pgfpathmoveto{\pgfqpoint{3.582846in}{2.125284in}}%
\pgfpathlineto{\pgfqpoint{1.593609in}{2.677303in}}%
\pgfpathlineto{\pgfqpoint{0.352463in}{1.684631in}}%
\pgfusepath{stroke}%
\end{pgfscope}%
\begin{pgfscope}%
\pgfsetbuttcap%
\pgfsetroundjoin%
\pgfsetlinewidth{0.803000pt}%
\definecolor{currentstroke}{rgb}{0.690196,0.690196,0.690196}%
\pgfsetstrokecolor{currentstroke}%
\pgfsetdash{}{0pt}%
\pgfpathmoveto{\pgfqpoint{3.599162in}{2.481571in}}%
\pgfpathlineto{\pgfqpoint{1.589648in}{3.021769in}}%
\pgfpathlineto{\pgfqpoint{0.334869in}{2.049971in}}%
\pgfusepath{stroke}%
\end{pgfscope}%
\begin{pgfscope}%
\pgfsetbuttcap%
\pgfsetroundjoin%
\pgfsetlinewidth{0.803000pt}%
\definecolor{currentstroke}{rgb}{0.690196,0.690196,0.690196}%
\pgfsetstrokecolor{currentstroke}%
\pgfsetdash{}{0pt}%
\pgfpathmoveto{\pgfqpoint{3.615817in}{2.845270in}}%
\pgfpathlineto{\pgfqpoint{1.585608in}{3.373052in}}%
\pgfpathlineto{\pgfqpoint{0.316896in}{2.423208in}}%
\pgfusepath{stroke}%
\end{pgfscope}%
\begin{pgfscope}%
\pgfsetrectcap%
\pgfsetroundjoin%
\pgfsetlinewidth{0.803000pt}%
\definecolor{currentstroke}{rgb}{0.000000,0.000000,0.000000}%
\pgfsetstrokecolor{currentstroke}%
\pgfsetdash{}{0pt}%
\pgfpathmoveto{\pgfqpoint{3.550327in}{1.780913in}}%
\pgfpathlineto{\pgfqpoint{3.599965in}{1.766716in}}%
\pgfusepath{stroke}%
\end{pgfscope}%
\begin{pgfscope}%
\definecolor{textcolor}{rgb}{0.000000,0.000000,0.000000}%
\pgfsetstrokecolor{textcolor}%
\pgfsetfillcolor{textcolor}%
\pgftext[x=3.821699in,y=1.811993in,,top]{\color{textcolor}\sffamily\fontsize{10.000000}{12.000000}\selectfont 0.0}%
\end{pgfscope}%
\begin{pgfscope}%
\pgfsetrectcap%
\pgfsetroundjoin%
\pgfsetlinewidth{0.803000pt}%
\definecolor{currentstroke}{rgb}{0.000000,0.000000,0.000000}%
\pgfsetstrokecolor{currentstroke}%
\pgfsetdash{}{0pt}%
\pgfpathmoveto{\pgfqpoint{3.566139in}{2.129921in}}%
\pgfpathlineto{\pgfqpoint{3.616302in}{2.116000in}}%
\pgfusepath{stroke}%
\end{pgfscope}%
\begin{pgfscope}%
\definecolor{textcolor}{rgb}{0.000000,0.000000,0.000000}%
\pgfsetstrokecolor{textcolor}%
\pgfsetfillcolor{textcolor}%
\pgftext[x=3.840225in,y=2.160393in,,top]{\color{textcolor}\sffamily\fontsize{10.000000}{12.000000}\selectfont 0.1}%
\end{pgfscope}%
\begin{pgfscope}%
\pgfsetrectcap%
\pgfsetroundjoin%
\pgfsetlinewidth{0.803000pt}%
\definecolor{currentstroke}{rgb}{0.000000,0.000000,0.000000}%
\pgfsetstrokecolor{currentstroke}%
\pgfsetdash{}{0pt}%
\pgfpathmoveto{\pgfqpoint{3.582276in}{2.486110in}}%
\pgfpathlineto{\pgfqpoint{3.632975in}{2.472481in}}%
\pgfusepath{stroke}%
\end{pgfscope}%
\begin{pgfscope}%
\definecolor{textcolor}{rgb}{0.000000,0.000000,0.000000}%
\pgfsetstrokecolor{textcolor}%
\pgfsetfillcolor{textcolor}%
\pgftext[x=3.859132in,y=2.515944in,,top]{\color{textcolor}\sffamily\fontsize{10.000000}{12.000000}\selectfont 0.2}%
\end{pgfscope}%
\begin{pgfscope}%
\pgfsetrectcap%
\pgfsetroundjoin%
\pgfsetlinewidth{0.803000pt}%
\definecolor{currentstroke}{rgb}{0.000000,0.000000,0.000000}%
\pgfsetstrokecolor{currentstroke}%
\pgfsetdash{}{0pt}%
\pgfpathmoveto{\pgfqpoint{3.598748in}{2.849707in}}%
\pgfpathlineto{\pgfqpoint{3.649996in}{2.836385in}}%
\pgfusepath{stroke}%
\end{pgfscope}%
\begin{pgfscope}%
\definecolor{textcolor}{rgb}{0.000000,0.000000,0.000000}%
\pgfsetstrokecolor{textcolor}%
\pgfsetfillcolor{textcolor}%
\pgftext[x=3.878430in,y=2.878868in,,top]{\color{textcolor}\sffamily\fontsize{10.000000}{12.000000}\selectfont 0.3}%
\end{pgfscope}%
\begin{pgfscope}%
\pgfpathrectangle{\pgfqpoint{0.100000in}{0.220728in}}{\pgfqpoint{3.696000in}{3.696000in}}%
\pgfusepath{clip}%
\pgfsetrectcap%
\pgfsetroundjoin%
\pgfsetlinewidth{1.505625pt}%
\definecolor{currentstroke}{rgb}{1.000000,0.000000,0.000000}%
\pgfsetstrokecolor{currentstroke}%
\pgfsetdash{}{0pt}%
\pgfpathmoveto{\pgfqpoint{0.854595in}{1.420432in}}%
\pgfpathlineto{\pgfqpoint{0.854595in}{1.420432in}}%
\pgfusepath{stroke}%
\end{pgfscope}%
\begin{pgfscope}%
\pgfpathrectangle{\pgfqpoint{0.100000in}{0.220728in}}{\pgfqpoint{3.696000in}{3.696000in}}%
\pgfusepath{clip}%
\pgfsetrectcap%
\pgfsetroundjoin%
\pgfsetlinewidth{1.505625pt}%
\definecolor{currentstroke}{rgb}{1.000000,0.000000,0.000000}%
\pgfsetstrokecolor{currentstroke}%
\pgfsetdash{}{0pt}%
\pgfpathmoveto{\pgfqpoint{1.654126in}{3.157081in}}%
\pgfpathlineto{\pgfqpoint{1.733967in}{2.164240in}}%
\pgfusepath{stroke}%
\end{pgfscope}%
\begin{pgfscope}%
\pgfpathrectangle{\pgfqpoint{0.100000in}{0.220728in}}{\pgfqpoint{3.696000in}{3.696000in}}%
\pgfusepath{clip}%
\pgfsetrectcap%
\pgfsetroundjoin%
\pgfsetlinewidth{1.505625pt}%
\definecolor{currentstroke}{rgb}{1.000000,0.000000,0.000000}%
\pgfsetstrokecolor{currentstroke}%
\pgfsetdash{}{0pt}%
\pgfpathmoveto{\pgfqpoint{3.311522in}{2.907170in}}%
\pgfpathlineto{\pgfqpoint{3.095743in}{1.770078in}}%
\pgfusepath{stroke}%
\end{pgfscope}%
\begin{pgfscope}%
\pgfpathrectangle{\pgfqpoint{0.100000in}{0.220728in}}{\pgfqpoint{3.696000in}{3.696000in}}%
\pgfusepath{clip}%
\pgfsetrectcap%
\pgfsetroundjoin%
\pgfsetlinewidth{1.505625pt}%
\definecolor{currentstroke}{rgb}{1.000000,0.000000,0.000000}%
\pgfsetstrokecolor{currentstroke}%
\pgfsetdash{}{0pt}%
\pgfpathmoveto{\pgfqpoint{2.349891in}{1.416973in}}%
\pgfpathlineto{\pgfqpoint{2.275567in}{0.972587in}}%
\pgfusepath{stroke}%
\end{pgfscope}%
\begin{pgfscope}%
\pgfpathrectangle{\pgfqpoint{0.100000in}{0.220728in}}{\pgfqpoint{3.696000in}{3.696000in}}%
\pgfusepath{clip}%
\pgfsetrectcap%
\pgfsetroundjoin%
\pgfsetlinewidth{1.505625pt}%
\definecolor{currentstroke}{rgb}{1.000000,0.000000,0.000000}%
\pgfsetstrokecolor{currentstroke}%
\pgfsetdash{}{0pt}%
\pgfpathmoveto{\pgfqpoint{0.754893in}{2.350324in}}%
\pgfpathlineto{\pgfqpoint{0.854595in}{1.420432in}}%
\pgfusepath{stroke}%
\end{pgfscope}%
\begin{pgfscope}%
\pgfpathrectangle{\pgfqpoint{0.100000in}{0.220728in}}{\pgfqpoint{3.696000in}{3.696000in}}%
\pgfusepath{clip}%
\pgfsetrectcap%
\pgfsetroundjoin%
\pgfsetlinewidth{1.505625pt}%
\definecolor{currentstroke}{rgb}{0.121569,0.466667,0.705882}%
\pgfsetstrokecolor{currentstroke}%
\pgfsetdash{}{0pt}%
\pgfpathmoveto{\pgfqpoint{0.854595in}{1.420432in}}%
\pgfpathlineto{\pgfqpoint{1.733967in}{2.164240in}}%
\pgfpathlineto{\pgfqpoint{3.095743in}{1.770078in}}%
\pgfpathlineto{\pgfqpoint{2.275567in}{0.972587in}}%
\pgfpathlineto{\pgfqpoint{0.854595in}{1.420432in}}%
\pgfusepath{stroke}%
\end{pgfscope}%
\begin{pgfscope}%
\pgfpathrectangle{\pgfqpoint{0.100000in}{0.220728in}}{\pgfqpoint{3.696000in}{3.696000in}}%
\pgfusepath{clip}%
\pgfsetbuttcap%
\pgfsetroundjoin%
\definecolor{currentfill}{rgb}{0.121569,0.466667,0.705882}%
\pgfsetfillcolor{currentfill}%
\pgfsetfillopacity{0.300000}%
\pgfsetlinewidth{1.003750pt}%
\definecolor{currentstroke}{rgb}{0.121569,0.466667,0.705882}%
\pgfsetstrokecolor{currentstroke}%
\pgfsetstrokeopacity{0.300000}%
\pgfsetdash{}{0pt}%
\pgfpathmoveto{\pgfqpoint{1.643450in}{3.120439in}}%
\pgfpathcurveto{\pgfqpoint{1.651687in}{3.120439in}}{\pgfqpoint{1.659587in}{3.123711in}}{\pgfqpoint{1.665411in}{3.129535in}}%
\pgfpathcurveto{\pgfqpoint{1.671235in}{3.135359in}}{\pgfqpoint{1.674507in}{3.143259in}}{\pgfqpoint{1.674507in}{3.151495in}}%
\pgfpathcurveto{\pgfqpoint{1.674507in}{3.159732in}}{\pgfqpoint{1.671235in}{3.167632in}}{\pgfqpoint{1.665411in}{3.173456in}}%
\pgfpathcurveto{\pgfqpoint{1.659587in}{3.179280in}}{\pgfqpoint{1.651687in}{3.182552in}}{\pgfqpoint{1.643450in}{3.182552in}}%
\pgfpathcurveto{\pgfqpoint{1.635214in}{3.182552in}}{\pgfqpoint{1.627314in}{3.179280in}}{\pgfqpoint{1.621490in}{3.173456in}}%
\pgfpathcurveto{\pgfqpoint{1.615666in}{3.167632in}}{\pgfqpoint{1.612394in}{3.159732in}}{\pgfqpoint{1.612394in}{3.151495in}}%
\pgfpathcurveto{\pgfqpoint{1.612394in}{3.143259in}}{\pgfqpoint{1.615666in}{3.135359in}}{\pgfqpoint{1.621490in}{3.129535in}}%
\pgfpathcurveto{\pgfqpoint{1.627314in}{3.123711in}}{\pgfqpoint{1.635214in}{3.120439in}}{\pgfqpoint{1.643450in}{3.120439in}}%
\pgfpathclose%
\pgfusepath{stroke,fill}%
\end{pgfscope}%
\begin{pgfscope}%
\pgfpathrectangle{\pgfqpoint{0.100000in}{0.220728in}}{\pgfqpoint{3.696000in}{3.696000in}}%
\pgfusepath{clip}%
\pgfsetbuttcap%
\pgfsetroundjoin%
\definecolor{currentfill}{rgb}{0.121569,0.466667,0.705882}%
\pgfsetfillcolor{currentfill}%
\pgfsetfillopacity{0.300180}%
\pgfsetlinewidth{1.003750pt}%
\definecolor{currentstroke}{rgb}{0.121569,0.466667,0.705882}%
\pgfsetstrokecolor{currentstroke}%
\pgfsetstrokeopacity{0.300180}%
\pgfsetdash{}{0pt}%
\pgfpathmoveto{\pgfqpoint{1.654126in}{3.126024in}}%
\pgfpathcurveto{\pgfqpoint{1.662362in}{3.126024in}}{\pgfqpoint{1.670263in}{3.129297in}}{\pgfqpoint{1.676086in}{3.135121in}}%
\pgfpathcurveto{\pgfqpoint{1.681910in}{3.140945in}}{\pgfqpoint{1.685183in}{3.148845in}}{\pgfqpoint{1.685183in}{3.157081in}}%
\pgfpathcurveto{\pgfqpoint{1.685183in}{3.165317in}}{\pgfqpoint{1.681910in}{3.173217in}}{\pgfqpoint{1.676086in}{3.179041in}}%
\pgfpathcurveto{\pgfqpoint{1.670263in}{3.184865in}}{\pgfqpoint{1.662362in}{3.188137in}}{\pgfqpoint{1.654126in}{3.188137in}}%
\pgfpathcurveto{\pgfqpoint{1.645890in}{3.188137in}}{\pgfqpoint{1.637990in}{3.184865in}}{\pgfqpoint{1.632166in}{3.179041in}}%
\pgfpathcurveto{\pgfqpoint{1.626342in}{3.173217in}}{\pgfqpoint{1.623070in}{3.165317in}}{\pgfqpoint{1.623070in}{3.157081in}}%
\pgfpathcurveto{\pgfqpoint{1.623070in}{3.148845in}}{\pgfqpoint{1.626342in}{3.140945in}}{\pgfqpoint{1.632166in}{3.135121in}}%
\pgfpathcurveto{\pgfqpoint{1.637990in}{3.129297in}}{\pgfqpoint{1.645890in}{3.126024in}}{\pgfqpoint{1.654126in}{3.126024in}}%
\pgfpathclose%
\pgfusepath{stroke,fill}%
\end{pgfscope}%
\begin{pgfscope}%
\pgfpathrectangle{\pgfqpoint{0.100000in}{0.220728in}}{\pgfqpoint{3.696000in}{3.696000in}}%
\pgfusepath{clip}%
\pgfsetbuttcap%
\pgfsetroundjoin%
\definecolor{currentfill}{rgb}{0.121569,0.466667,0.705882}%
\pgfsetfillcolor{currentfill}%
\pgfsetfillopacity{0.300241}%
\pgfsetlinewidth{1.003750pt}%
\definecolor{currentstroke}{rgb}{0.121569,0.466667,0.705882}%
\pgfsetstrokecolor{currentstroke}%
\pgfsetstrokeopacity{0.300241}%
\pgfsetdash{}{0pt}%
\pgfpathmoveto{\pgfqpoint{1.636383in}{3.115577in}}%
\pgfpathcurveto{\pgfqpoint{1.644619in}{3.115577in}}{\pgfqpoint{1.652519in}{3.118849in}}{\pgfqpoint{1.658343in}{3.124673in}}%
\pgfpathcurveto{\pgfqpoint{1.664167in}{3.130497in}}{\pgfqpoint{1.667439in}{3.138397in}}{\pgfqpoint{1.667439in}{3.146634in}}%
\pgfpathcurveto{\pgfqpoint{1.667439in}{3.154870in}}{\pgfqpoint{1.664167in}{3.162770in}}{\pgfqpoint{1.658343in}{3.168594in}}%
\pgfpathcurveto{\pgfqpoint{1.652519in}{3.174418in}}{\pgfqpoint{1.644619in}{3.177690in}}{\pgfqpoint{1.636383in}{3.177690in}}%
\pgfpathcurveto{\pgfqpoint{1.628146in}{3.177690in}}{\pgfqpoint{1.620246in}{3.174418in}}{\pgfqpoint{1.614422in}{3.168594in}}%
\pgfpathcurveto{\pgfqpoint{1.608598in}{3.162770in}}{\pgfqpoint{1.605326in}{3.154870in}}{\pgfqpoint{1.605326in}{3.146634in}}%
\pgfpathcurveto{\pgfqpoint{1.605326in}{3.138397in}}{\pgfqpoint{1.608598in}{3.130497in}}{\pgfqpoint{1.614422in}{3.124673in}}%
\pgfpathcurveto{\pgfqpoint{1.620246in}{3.118849in}}{\pgfqpoint{1.628146in}{3.115577in}}{\pgfqpoint{1.636383in}{3.115577in}}%
\pgfpathclose%
\pgfusepath{stroke,fill}%
\end{pgfscope}%
\begin{pgfscope}%
\pgfpathrectangle{\pgfqpoint{0.100000in}{0.220728in}}{\pgfqpoint{3.696000in}{3.696000in}}%
\pgfusepath{clip}%
\pgfsetbuttcap%
\pgfsetroundjoin%
\definecolor{currentfill}{rgb}{0.121569,0.466667,0.705882}%
\pgfsetfillcolor{currentfill}%
\pgfsetfillopacity{0.300584}%
\pgfsetlinewidth{1.003750pt}%
\definecolor{currentstroke}{rgb}{0.121569,0.466667,0.705882}%
\pgfsetstrokecolor{currentstroke}%
\pgfsetstrokeopacity{0.300584}%
\pgfsetdash{}{0pt}%
\pgfpathmoveto{\pgfqpoint{1.632008in}{3.110993in}}%
\pgfpathcurveto{\pgfqpoint{1.640244in}{3.110993in}}{\pgfqpoint{1.648144in}{3.114265in}}{\pgfqpoint{1.653968in}{3.120089in}}%
\pgfpathcurveto{\pgfqpoint{1.659792in}{3.125913in}}{\pgfqpoint{1.663065in}{3.133813in}}{\pgfqpoint{1.663065in}{3.142049in}}%
\pgfpathcurveto{\pgfqpoint{1.663065in}{3.150285in}}{\pgfqpoint{1.659792in}{3.158185in}}{\pgfqpoint{1.653968in}{3.164009in}}%
\pgfpathcurveto{\pgfqpoint{1.648144in}{3.169833in}}{\pgfqpoint{1.640244in}{3.173106in}}{\pgfqpoint{1.632008in}{3.173106in}}%
\pgfpathcurveto{\pgfqpoint{1.623772in}{3.173106in}}{\pgfqpoint{1.615872in}{3.169833in}}{\pgfqpoint{1.610048in}{3.164009in}}%
\pgfpathcurveto{\pgfqpoint{1.604224in}{3.158185in}}{\pgfqpoint{1.600952in}{3.150285in}}{\pgfqpoint{1.600952in}{3.142049in}}%
\pgfpathcurveto{\pgfqpoint{1.600952in}{3.133813in}}{\pgfqpoint{1.604224in}{3.125913in}}{\pgfqpoint{1.610048in}{3.120089in}}%
\pgfpathcurveto{\pgfqpoint{1.615872in}{3.114265in}}{\pgfqpoint{1.623772in}{3.110993in}}{\pgfqpoint{1.632008in}{3.110993in}}%
\pgfpathclose%
\pgfusepath{stroke,fill}%
\end{pgfscope}%
\begin{pgfscope}%
\pgfpathrectangle{\pgfqpoint{0.100000in}{0.220728in}}{\pgfqpoint{3.696000in}{3.696000in}}%
\pgfusepath{clip}%
\pgfsetbuttcap%
\pgfsetroundjoin%
\definecolor{currentfill}{rgb}{0.121569,0.466667,0.705882}%
\pgfsetfillcolor{currentfill}%
\pgfsetfillopacity{0.300641}%
\pgfsetlinewidth{1.003750pt}%
\definecolor{currentstroke}{rgb}{0.121569,0.466667,0.705882}%
\pgfsetstrokecolor{currentstroke}%
\pgfsetstrokeopacity{0.300641}%
\pgfsetdash{}{0pt}%
\pgfpathmoveto{\pgfqpoint{1.667209in}{3.131193in}}%
\pgfpathcurveto{\pgfqpoint{1.675446in}{3.131193in}}{\pgfqpoint{1.683346in}{3.134466in}}{\pgfqpoint{1.689170in}{3.140289in}}%
\pgfpathcurveto{\pgfqpoint{1.694994in}{3.146113in}}{\pgfqpoint{1.698266in}{3.154013in}}{\pgfqpoint{1.698266in}{3.162250in}}%
\pgfpathcurveto{\pgfqpoint{1.698266in}{3.170486in}}{\pgfqpoint{1.694994in}{3.178386in}}{\pgfqpoint{1.689170in}{3.184210in}}%
\pgfpathcurveto{\pgfqpoint{1.683346in}{3.190034in}}{\pgfqpoint{1.675446in}{3.193306in}}{\pgfqpoint{1.667209in}{3.193306in}}%
\pgfpathcurveto{\pgfqpoint{1.658973in}{3.193306in}}{\pgfqpoint{1.651073in}{3.190034in}}{\pgfqpoint{1.645249in}{3.184210in}}%
\pgfpathcurveto{\pgfqpoint{1.639425in}{3.178386in}}{\pgfqpoint{1.636153in}{3.170486in}}{\pgfqpoint{1.636153in}{3.162250in}}%
\pgfpathcurveto{\pgfqpoint{1.636153in}{3.154013in}}{\pgfqpoint{1.639425in}{3.146113in}}{\pgfqpoint{1.645249in}{3.140289in}}%
\pgfpathcurveto{\pgfqpoint{1.651073in}{3.134466in}}{\pgfqpoint{1.658973in}{3.131193in}}{\pgfqpoint{1.667209in}{3.131193in}}%
\pgfpathclose%
\pgfusepath{stroke,fill}%
\end{pgfscope}%
\begin{pgfscope}%
\pgfpathrectangle{\pgfqpoint{0.100000in}{0.220728in}}{\pgfqpoint{3.696000in}{3.696000in}}%
\pgfusepath{clip}%
\pgfsetbuttcap%
\pgfsetroundjoin%
\definecolor{currentfill}{rgb}{0.121569,0.466667,0.705882}%
\pgfsetfillcolor{currentfill}%
\pgfsetfillopacity{0.300751}%
\pgfsetlinewidth{1.003750pt}%
\definecolor{currentstroke}{rgb}{0.121569,0.466667,0.705882}%
\pgfsetstrokecolor{currentstroke}%
\pgfsetstrokeopacity{0.300751}%
\pgfsetdash{}{0pt}%
\pgfpathmoveto{\pgfqpoint{1.630711in}{3.109411in}}%
\pgfpathcurveto{\pgfqpoint{1.638947in}{3.109411in}}{\pgfqpoint{1.646847in}{3.112683in}}{\pgfqpoint{1.652671in}{3.118507in}}%
\pgfpathcurveto{\pgfqpoint{1.658495in}{3.124331in}}{\pgfqpoint{1.661768in}{3.132231in}}{\pgfqpoint{1.661768in}{3.140467in}}%
\pgfpathcurveto{\pgfqpoint{1.661768in}{3.148703in}}{\pgfqpoint{1.658495in}{3.156603in}}{\pgfqpoint{1.652671in}{3.162427in}}%
\pgfpathcurveto{\pgfqpoint{1.646847in}{3.168251in}}{\pgfqpoint{1.638947in}{3.171524in}}{\pgfqpoint{1.630711in}{3.171524in}}%
\pgfpathcurveto{\pgfqpoint{1.622475in}{3.171524in}}{\pgfqpoint{1.614575in}{3.168251in}}{\pgfqpoint{1.608751in}{3.162427in}}%
\pgfpathcurveto{\pgfqpoint{1.602927in}{3.156603in}}{\pgfqpoint{1.599655in}{3.148703in}}{\pgfqpoint{1.599655in}{3.140467in}}%
\pgfpathcurveto{\pgfqpoint{1.599655in}{3.132231in}}{\pgfqpoint{1.602927in}{3.124331in}}{\pgfqpoint{1.608751in}{3.118507in}}%
\pgfpathcurveto{\pgfqpoint{1.614575in}{3.112683in}}{\pgfqpoint{1.622475in}{3.109411in}}{\pgfqpoint{1.630711in}{3.109411in}}%
\pgfpathclose%
\pgfusepath{stroke,fill}%
\end{pgfscope}%
\begin{pgfscope}%
\pgfpathrectangle{\pgfqpoint{0.100000in}{0.220728in}}{\pgfqpoint{3.696000in}{3.696000in}}%
\pgfusepath{clip}%
\pgfsetbuttcap%
\pgfsetroundjoin%
\definecolor{currentfill}{rgb}{0.121569,0.466667,0.705882}%
\pgfsetfillcolor{currentfill}%
\pgfsetfillopacity{0.301102}%
\pgfsetlinewidth{1.003750pt}%
\definecolor{currentstroke}{rgb}{0.121569,0.466667,0.705882}%
\pgfsetstrokecolor{currentstroke}%
\pgfsetstrokeopacity{0.301102}%
\pgfsetdash{}{0pt}%
\pgfpathmoveto{\pgfqpoint{1.628590in}{3.106317in}}%
\pgfpathcurveto{\pgfqpoint{1.636826in}{3.106317in}}{\pgfqpoint{1.644726in}{3.109590in}}{\pgfqpoint{1.650550in}{3.115414in}}%
\pgfpathcurveto{\pgfqpoint{1.656374in}{3.121237in}}{\pgfqpoint{1.659646in}{3.129137in}}{\pgfqpoint{1.659646in}{3.137374in}}%
\pgfpathcurveto{\pgfqpoint{1.659646in}{3.145610in}}{\pgfqpoint{1.656374in}{3.153510in}}{\pgfqpoint{1.650550in}{3.159334in}}%
\pgfpathcurveto{\pgfqpoint{1.644726in}{3.165158in}}{\pgfqpoint{1.636826in}{3.168430in}}{\pgfqpoint{1.628590in}{3.168430in}}%
\pgfpathcurveto{\pgfqpoint{1.620353in}{3.168430in}}{\pgfqpoint{1.612453in}{3.165158in}}{\pgfqpoint{1.606629in}{3.159334in}}%
\pgfpathcurveto{\pgfqpoint{1.600805in}{3.153510in}}{\pgfqpoint{1.597533in}{3.145610in}}{\pgfqpoint{1.597533in}{3.137374in}}%
\pgfpathcurveto{\pgfqpoint{1.597533in}{3.129137in}}{\pgfqpoint{1.600805in}{3.121237in}}{\pgfqpoint{1.606629in}{3.115414in}}%
\pgfpathcurveto{\pgfqpoint{1.612453in}{3.109590in}}{\pgfqpoint{1.620353in}{3.106317in}}{\pgfqpoint{1.628590in}{3.106317in}}%
\pgfpathclose%
\pgfusepath{stroke,fill}%
\end{pgfscope}%
\begin{pgfscope}%
\pgfpathrectangle{\pgfqpoint{0.100000in}{0.220728in}}{\pgfqpoint{3.696000in}{3.696000in}}%
\pgfusepath{clip}%
\pgfsetbuttcap%
\pgfsetroundjoin%
\definecolor{currentfill}{rgb}{0.121569,0.466667,0.705882}%
\pgfsetfillcolor{currentfill}%
\pgfsetfillopacity{0.301102}%
\pgfsetlinewidth{1.003750pt}%
\definecolor{currentstroke}{rgb}{0.121569,0.466667,0.705882}%
\pgfsetstrokecolor{currentstroke}%
\pgfsetstrokeopacity{0.301102}%
\pgfsetdash{}{0pt}%
\pgfpathmoveto{\pgfqpoint{1.628588in}{3.106315in}}%
\pgfpathcurveto{\pgfqpoint{1.636824in}{3.106315in}}{\pgfqpoint{1.644724in}{3.109587in}}{\pgfqpoint{1.650548in}{3.115411in}}%
\pgfpathcurveto{\pgfqpoint{1.656372in}{3.121235in}}{\pgfqpoint{1.659645in}{3.129135in}}{\pgfqpoint{1.659645in}{3.137371in}}%
\pgfpathcurveto{\pgfqpoint{1.659645in}{3.145607in}}{\pgfqpoint{1.656372in}{3.153508in}}{\pgfqpoint{1.650548in}{3.159331in}}%
\pgfpathcurveto{\pgfqpoint{1.644724in}{3.165155in}}{\pgfqpoint{1.636824in}{3.168428in}}{\pgfqpoint{1.628588in}{3.168428in}}%
\pgfpathcurveto{\pgfqpoint{1.620352in}{3.168428in}}{\pgfqpoint{1.612452in}{3.165155in}}{\pgfqpoint{1.606628in}{3.159331in}}%
\pgfpathcurveto{\pgfqpoint{1.600804in}{3.153508in}}{\pgfqpoint{1.597532in}{3.145607in}}{\pgfqpoint{1.597532in}{3.137371in}}%
\pgfpathcurveto{\pgfqpoint{1.597532in}{3.129135in}}{\pgfqpoint{1.600804in}{3.121235in}}{\pgfqpoint{1.606628in}{3.115411in}}%
\pgfpathcurveto{\pgfqpoint{1.612452in}{3.109587in}}{\pgfqpoint{1.620352in}{3.106315in}}{\pgfqpoint{1.628588in}{3.106315in}}%
\pgfpathclose%
\pgfusepath{stroke,fill}%
\end{pgfscope}%
\begin{pgfscope}%
\pgfpathrectangle{\pgfqpoint{0.100000in}{0.220728in}}{\pgfqpoint{3.696000in}{3.696000in}}%
\pgfusepath{clip}%
\pgfsetbuttcap%
\pgfsetroundjoin%
\definecolor{currentfill}{rgb}{0.121569,0.466667,0.705882}%
\pgfsetfillcolor{currentfill}%
\pgfsetfillopacity{0.301103}%
\pgfsetlinewidth{1.003750pt}%
\definecolor{currentstroke}{rgb}{0.121569,0.466667,0.705882}%
\pgfsetstrokecolor{currentstroke}%
\pgfsetstrokeopacity{0.301103}%
\pgfsetdash{}{0pt}%
\pgfpathmoveto{\pgfqpoint{1.628586in}{3.106310in}}%
\pgfpathcurveto{\pgfqpoint{1.636822in}{3.106310in}}{\pgfqpoint{1.644722in}{3.109582in}}{\pgfqpoint{1.650546in}{3.115406in}}%
\pgfpathcurveto{\pgfqpoint{1.656370in}{3.121230in}}{\pgfqpoint{1.659642in}{3.129130in}}{\pgfqpoint{1.659642in}{3.137366in}}%
\pgfpathcurveto{\pgfqpoint{1.659642in}{3.145603in}}{\pgfqpoint{1.656370in}{3.153503in}}{\pgfqpoint{1.650546in}{3.159327in}}%
\pgfpathcurveto{\pgfqpoint{1.644722in}{3.165151in}}{\pgfqpoint{1.636822in}{3.168423in}}{\pgfqpoint{1.628586in}{3.168423in}}%
\pgfpathcurveto{\pgfqpoint{1.620349in}{3.168423in}}{\pgfqpoint{1.612449in}{3.165151in}}{\pgfqpoint{1.606625in}{3.159327in}}%
\pgfpathcurveto{\pgfqpoint{1.600801in}{3.153503in}}{\pgfqpoint{1.597529in}{3.145603in}}{\pgfqpoint{1.597529in}{3.137366in}}%
\pgfpathcurveto{\pgfqpoint{1.597529in}{3.129130in}}{\pgfqpoint{1.600801in}{3.121230in}}{\pgfqpoint{1.606625in}{3.115406in}}%
\pgfpathcurveto{\pgfqpoint{1.612449in}{3.109582in}}{\pgfqpoint{1.620349in}{3.106310in}}{\pgfqpoint{1.628586in}{3.106310in}}%
\pgfpathclose%
\pgfusepath{stroke,fill}%
\end{pgfscope}%
\begin{pgfscope}%
\pgfpathrectangle{\pgfqpoint{0.100000in}{0.220728in}}{\pgfqpoint{3.696000in}{3.696000in}}%
\pgfusepath{clip}%
\pgfsetbuttcap%
\pgfsetroundjoin%
\definecolor{currentfill}{rgb}{0.121569,0.466667,0.705882}%
\pgfsetfillcolor{currentfill}%
\pgfsetfillopacity{0.301104}%
\pgfsetlinewidth{1.003750pt}%
\definecolor{currentstroke}{rgb}{0.121569,0.466667,0.705882}%
\pgfsetstrokecolor{currentstroke}%
\pgfsetstrokeopacity{0.301104}%
\pgfsetdash{}{0pt}%
\pgfpathmoveto{\pgfqpoint{1.628581in}{3.106302in}}%
\pgfpathcurveto{\pgfqpoint{1.636817in}{3.106302in}}{\pgfqpoint{1.644717in}{3.109574in}}{\pgfqpoint{1.650541in}{3.115398in}}%
\pgfpathcurveto{\pgfqpoint{1.656365in}{3.121222in}}{\pgfqpoint{1.659637in}{3.129122in}}{\pgfqpoint{1.659637in}{3.137358in}}%
\pgfpathcurveto{\pgfqpoint{1.659637in}{3.145594in}}{\pgfqpoint{1.656365in}{3.153495in}}{\pgfqpoint{1.650541in}{3.159318in}}%
\pgfpathcurveto{\pgfqpoint{1.644717in}{3.165142in}}{\pgfqpoint{1.636817in}{3.168415in}}{\pgfqpoint{1.628581in}{3.168415in}}%
\pgfpathcurveto{\pgfqpoint{1.620345in}{3.168415in}}{\pgfqpoint{1.612445in}{3.165142in}}{\pgfqpoint{1.606621in}{3.159318in}}%
\pgfpathcurveto{\pgfqpoint{1.600797in}{3.153495in}}{\pgfqpoint{1.597524in}{3.145594in}}{\pgfqpoint{1.597524in}{3.137358in}}%
\pgfpathcurveto{\pgfqpoint{1.597524in}{3.129122in}}{\pgfqpoint{1.600797in}{3.121222in}}{\pgfqpoint{1.606621in}{3.115398in}}%
\pgfpathcurveto{\pgfqpoint{1.612445in}{3.109574in}}{\pgfqpoint{1.620345in}{3.106302in}}{\pgfqpoint{1.628581in}{3.106302in}}%
\pgfpathclose%
\pgfusepath{stroke,fill}%
\end{pgfscope}%
\begin{pgfscope}%
\pgfpathrectangle{\pgfqpoint{0.100000in}{0.220728in}}{\pgfqpoint{3.696000in}{3.696000in}}%
\pgfusepath{clip}%
\pgfsetbuttcap%
\pgfsetroundjoin%
\definecolor{currentfill}{rgb}{0.121569,0.466667,0.705882}%
\pgfsetfillcolor{currentfill}%
\pgfsetfillopacity{0.301106}%
\pgfsetlinewidth{1.003750pt}%
\definecolor{currentstroke}{rgb}{0.121569,0.466667,0.705882}%
\pgfsetstrokecolor{currentstroke}%
\pgfsetstrokeopacity{0.301106}%
\pgfsetdash{}{0pt}%
\pgfpathmoveto{\pgfqpoint{1.628574in}{3.106286in}}%
\pgfpathcurveto{\pgfqpoint{1.636810in}{3.106286in}}{\pgfqpoint{1.644710in}{3.109559in}}{\pgfqpoint{1.650534in}{3.115382in}}%
\pgfpathcurveto{\pgfqpoint{1.656358in}{3.121206in}}{\pgfqpoint{1.659630in}{3.129106in}}{\pgfqpoint{1.659630in}{3.137343in}}%
\pgfpathcurveto{\pgfqpoint{1.659630in}{3.145579in}}{\pgfqpoint{1.656358in}{3.153479in}}{\pgfqpoint{1.650534in}{3.159303in}}%
\pgfpathcurveto{\pgfqpoint{1.644710in}{3.165127in}}{\pgfqpoint{1.636810in}{3.168399in}}{\pgfqpoint{1.628574in}{3.168399in}}%
\pgfpathcurveto{\pgfqpoint{1.620338in}{3.168399in}}{\pgfqpoint{1.612438in}{3.165127in}}{\pgfqpoint{1.606614in}{3.159303in}}%
\pgfpathcurveto{\pgfqpoint{1.600790in}{3.153479in}}{\pgfqpoint{1.597517in}{3.145579in}}{\pgfqpoint{1.597517in}{3.137343in}}%
\pgfpathcurveto{\pgfqpoint{1.597517in}{3.129106in}}{\pgfqpoint{1.600790in}{3.121206in}}{\pgfqpoint{1.606614in}{3.115382in}}%
\pgfpathcurveto{\pgfqpoint{1.612438in}{3.109559in}}{\pgfqpoint{1.620338in}{3.106286in}}{\pgfqpoint{1.628574in}{3.106286in}}%
\pgfpathclose%
\pgfusepath{stroke,fill}%
\end{pgfscope}%
\begin{pgfscope}%
\pgfpathrectangle{\pgfqpoint{0.100000in}{0.220728in}}{\pgfqpoint{3.696000in}{3.696000in}}%
\pgfusepath{clip}%
\pgfsetbuttcap%
\pgfsetroundjoin%
\definecolor{currentfill}{rgb}{0.121569,0.466667,0.705882}%
\pgfsetfillcolor{currentfill}%
\pgfsetfillopacity{0.301110}%
\pgfsetlinewidth{1.003750pt}%
\definecolor{currentstroke}{rgb}{0.121569,0.466667,0.705882}%
\pgfsetstrokecolor{currentstroke}%
\pgfsetstrokeopacity{0.301110}%
\pgfsetdash{}{0pt}%
\pgfpathmoveto{\pgfqpoint{1.628560in}{3.106258in}}%
\pgfpathcurveto{\pgfqpoint{1.636797in}{3.106258in}}{\pgfqpoint{1.644697in}{3.109530in}}{\pgfqpoint{1.650521in}{3.115354in}}%
\pgfpathcurveto{\pgfqpoint{1.656345in}{3.121178in}}{\pgfqpoint{1.659617in}{3.129078in}}{\pgfqpoint{1.659617in}{3.137314in}}%
\pgfpathcurveto{\pgfqpoint{1.659617in}{3.145551in}}{\pgfqpoint{1.656345in}{3.153451in}}{\pgfqpoint{1.650521in}{3.159275in}}%
\pgfpathcurveto{\pgfqpoint{1.644697in}{3.165099in}}{\pgfqpoint{1.636797in}{3.168371in}}{\pgfqpoint{1.628560in}{3.168371in}}%
\pgfpathcurveto{\pgfqpoint{1.620324in}{3.168371in}}{\pgfqpoint{1.612424in}{3.165099in}}{\pgfqpoint{1.606600in}{3.159275in}}%
\pgfpathcurveto{\pgfqpoint{1.600776in}{3.153451in}}{\pgfqpoint{1.597504in}{3.145551in}}{\pgfqpoint{1.597504in}{3.137314in}}%
\pgfpathcurveto{\pgfqpoint{1.597504in}{3.129078in}}{\pgfqpoint{1.600776in}{3.121178in}}{\pgfqpoint{1.606600in}{3.115354in}}%
\pgfpathcurveto{\pgfqpoint{1.612424in}{3.109530in}}{\pgfqpoint{1.620324in}{3.106258in}}{\pgfqpoint{1.628560in}{3.106258in}}%
\pgfpathclose%
\pgfusepath{stroke,fill}%
\end{pgfscope}%
\begin{pgfscope}%
\pgfpathrectangle{\pgfqpoint{0.100000in}{0.220728in}}{\pgfqpoint{3.696000in}{3.696000in}}%
\pgfusepath{clip}%
\pgfsetbuttcap%
\pgfsetroundjoin%
\definecolor{currentfill}{rgb}{0.121569,0.466667,0.705882}%
\pgfsetfillcolor{currentfill}%
\pgfsetfillopacity{0.301118}%
\pgfsetlinewidth{1.003750pt}%
\definecolor{currentstroke}{rgb}{0.121569,0.466667,0.705882}%
\pgfsetstrokecolor{currentstroke}%
\pgfsetstrokeopacity{0.301118}%
\pgfsetdash{}{0pt}%
\pgfpathmoveto{\pgfqpoint{1.628537in}{3.106207in}}%
\pgfpathcurveto{\pgfqpoint{1.636773in}{3.106207in}}{\pgfqpoint{1.644673in}{3.109480in}}{\pgfqpoint{1.650497in}{3.115303in}}%
\pgfpathcurveto{\pgfqpoint{1.656321in}{3.121127in}}{\pgfqpoint{1.659593in}{3.129027in}}{\pgfqpoint{1.659593in}{3.137264in}}%
\pgfpathcurveto{\pgfqpoint{1.659593in}{3.145500in}}{\pgfqpoint{1.656321in}{3.153400in}}{\pgfqpoint{1.650497in}{3.159224in}}%
\pgfpathcurveto{\pgfqpoint{1.644673in}{3.165048in}}{\pgfqpoint{1.636773in}{3.168320in}}{\pgfqpoint{1.628537in}{3.168320in}}%
\pgfpathcurveto{\pgfqpoint{1.620300in}{3.168320in}}{\pgfqpoint{1.612400in}{3.165048in}}{\pgfqpoint{1.606577in}{3.159224in}}%
\pgfpathcurveto{\pgfqpoint{1.600753in}{3.153400in}}{\pgfqpoint{1.597480in}{3.145500in}}{\pgfqpoint{1.597480in}{3.137264in}}%
\pgfpathcurveto{\pgfqpoint{1.597480in}{3.129027in}}{\pgfqpoint{1.600753in}{3.121127in}}{\pgfqpoint{1.606577in}{3.115303in}}%
\pgfpathcurveto{\pgfqpoint{1.612400in}{3.109480in}}{\pgfqpoint{1.620300in}{3.106207in}}{\pgfqpoint{1.628537in}{3.106207in}}%
\pgfpathclose%
\pgfusepath{stroke,fill}%
\end{pgfscope}%
\begin{pgfscope}%
\pgfpathrectangle{\pgfqpoint{0.100000in}{0.220728in}}{\pgfqpoint{3.696000in}{3.696000in}}%
\pgfusepath{clip}%
\pgfsetbuttcap%
\pgfsetroundjoin%
\definecolor{currentfill}{rgb}{0.121569,0.466667,0.705882}%
\pgfsetfillcolor{currentfill}%
\pgfsetfillopacity{0.301131}%
\pgfsetlinewidth{1.003750pt}%
\definecolor{currentstroke}{rgb}{0.121569,0.466667,0.705882}%
\pgfsetstrokecolor{currentstroke}%
\pgfsetstrokeopacity{0.301131}%
\pgfsetdash{}{0pt}%
\pgfpathmoveto{\pgfqpoint{1.628492in}{3.106112in}}%
\pgfpathcurveto{\pgfqpoint{1.636729in}{3.106112in}}{\pgfqpoint{1.644629in}{3.109384in}}{\pgfqpoint{1.650453in}{3.115208in}}%
\pgfpathcurveto{\pgfqpoint{1.656276in}{3.121032in}}{\pgfqpoint{1.659549in}{3.128932in}}{\pgfqpoint{1.659549in}{3.137168in}}%
\pgfpathcurveto{\pgfqpoint{1.659549in}{3.145405in}}{\pgfqpoint{1.656276in}{3.153305in}}{\pgfqpoint{1.650453in}{3.159129in}}%
\pgfpathcurveto{\pgfqpoint{1.644629in}{3.164953in}}{\pgfqpoint{1.636729in}{3.168225in}}{\pgfqpoint{1.628492in}{3.168225in}}%
\pgfpathcurveto{\pgfqpoint{1.620256in}{3.168225in}}{\pgfqpoint{1.612356in}{3.164953in}}{\pgfqpoint{1.606532in}{3.159129in}}%
\pgfpathcurveto{\pgfqpoint{1.600708in}{3.153305in}}{\pgfqpoint{1.597436in}{3.145405in}}{\pgfqpoint{1.597436in}{3.137168in}}%
\pgfpathcurveto{\pgfqpoint{1.597436in}{3.128932in}}{\pgfqpoint{1.600708in}{3.121032in}}{\pgfqpoint{1.606532in}{3.115208in}}%
\pgfpathcurveto{\pgfqpoint{1.612356in}{3.109384in}}{\pgfqpoint{1.620256in}{3.106112in}}{\pgfqpoint{1.628492in}{3.106112in}}%
\pgfpathclose%
\pgfusepath{stroke,fill}%
\end{pgfscope}%
\begin{pgfscope}%
\pgfpathrectangle{\pgfqpoint{0.100000in}{0.220728in}}{\pgfqpoint{3.696000in}{3.696000in}}%
\pgfusepath{clip}%
\pgfsetbuttcap%
\pgfsetroundjoin%
\definecolor{currentfill}{rgb}{0.121569,0.466667,0.705882}%
\pgfsetfillcolor{currentfill}%
\pgfsetfillopacity{0.301157}%
\pgfsetlinewidth{1.003750pt}%
\definecolor{currentstroke}{rgb}{0.121569,0.466667,0.705882}%
\pgfsetstrokecolor{currentstroke}%
\pgfsetstrokeopacity{0.301157}%
\pgfsetdash{}{0pt}%
\pgfpathmoveto{\pgfqpoint{1.628412in}{3.105945in}}%
\pgfpathcurveto{\pgfqpoint{1.636649in}{3.105945in}}{\pgfqpoint{1.644549in}{3.109218in}}{\pgfqpoint{1.650373in}{3.115041in}}%
\pgfpathcurveto{\pgfqpoint{1.656197in}{3.120865in}}{\pgfqpoint{1.659469in}{3.128765in}}{\pgfqpoint{1.659469in}{3.137002in}}%
\pgfpathcurveto{\pgfqpoint{1.659469in}{3.145238in}}{\pgfqpoint{1.656197in}{3.153138in}}{\pgfqpoint{1.650373in}{3.158962in}}%
\pgfpathcurveto{\pgfqpoint{1.644549in}{3.164786in}}{\pgfqpoint{1.636649in}{3.168058in}}{\pgfqpoint{1.628412in}{3.168058in}}%
\pgfpathcurveto{\pgfqpoint{1.620176in}{3.168058in}}{\pgfqpoint{1.612276in}{3.164786in}}{\pgfqpoint{1.606452in}{3.158962in}}%
\pgfpathcurveto{\pgfqpoint{1.600628in}{3.153138in}}{\pgfqpoint{1.597356in}{3.145238in}}{\pgfqpoint{1.597356in}{3.137002in}}%
\pgfpathcurveto{\pgfqpoint{1.597356in}{3.128765in}}{\pgfqpoint{1.600628in}{3.120865in}}{\pgfqpoint{1.606452in}{3.115041in}}%
\pgfpathcurveto{\pgfqpoint{1.612276in}{3.109218in}}{\pgfqpoint{1.620176in}{3.105945in}}{\pgfqpoint{1.628412in}{3.105945in}}%
\pgfpathclose%
\pgfusepath{stroke,fill}%
\end{pgfscope}%
\begin{pgfscope}%
\pgfpathrectangle{\pgfqpoint{0.100000in}{0.220728in}}{\pgfqpoint{3.696000in}{3.696000in}}%
\pgfusepath{clip}%
\pgfsetbuttcap%
\pgfsetroundjoin%
\definecolor{currentfill}{rgb}{0.121569,0.466667,0.705882}%
\pgfsetfillcolor{currentfill}%
\pgfsetfillopacity{0.301204}%
\pgfsetlinewidth{1.003750pt}%
\definecolor{currentstroke}{rgb}{0.121569,0.466667,0.705882}%
\pgfsetstrokecolor{currentstroke}%
\pgfsetstrokeopacity{0.301204}%
\pgfsetdash{}{0pt}%
\pgfpathmoveto{\pgfqpoint{1.628272in}{3.105638in}}%
\pgfpathcurveto{\pgfqpoint{1.636509in}{3.105638in}}{\pgfqpoint{1.644409in}{3.108911in}}{\pgfqpoint{1.650233in}{3.114735in}}%
\pgfpathcurveto{\pgfqpoint{1.656056in}{3.120558in}}{\pgfqpoint{1.659329in}{3.128458in}}{\pgfqpoint{1.659329in}{3.136695in}}%
\pgfpathcurveto{\pgfqpoint{1.659329in}{3.144931in}}{\pgfqpoint{1.656056in}{3.152831in}}{\pgfqpoint{1.650233in}{3.158655in}}%
\pgfpathcurveto{\pgfqpoint{1.644409in}{3.164479in}}{\pgfqpoint{1.636509in}{3.167751in}}{\pgfqpoint{1.628272in}{3.167751in}}%
\pgfpathcurveto{\pgfqpoint{1.620036in}{3.167751in}}{\pgfqpoint{1.612136in}{3.164479in}}{\pgfqpoint{1.606312in}{3.158655in}}%
\pgfpathcurveto{\pgfqpoint{1.600488in}{3.152831in}}{\pgfqpoint{1.597216in}{3.144931in}}{\pgfqpoint{1.597216in}{3.136695in}}%
\pgfpathcurveto{\pgfqpoint{1.597216in}{3.128458in}}{\pgfqpoint{1.600488in}{3.120558in}}{\pgfqpoint{1.606312in}{3.114735in}}%
\pgfpathcurveto{\pgfqpoint{1.612136in}{3.108911in}}{\pgfqpoint{1.620036in}{3.105638in}}{\pgfqpoint{1.628272in}{3.105638in}}%
\pgfpathclose%
\pgfusepath{stroke,fill}%
\end{pgfscope}%
\begin{pgfscope}%
\pgfpathrectangle{\pgfqpoint{0.100000in}{0.220728in}}{\pgfqpoint{3.696000in}{3.696000in}}%
\pgfusepath{clip}%
\pgfsetbuttcap%
\pgfsetroundjoin%
\definecolor{currentfill}{rgb}{0.121569,0.466667,0.705882}%
\pgfsetfillcolor{currentfill}%
\pgfsetfillopacity{0.301293}%
\pgfsetlinewidth{1.003750pt}%
\definecolor{currentstroke}{rgb}{0.121569,0.466667,0.705882}%
\pgfsetstrokecolor{currentstroke}%
\pgfsetstrokeopacity{0.301293}%
\pgfsetdash{}{0pt}%
\pgfpathmoveto{\pgfqpoint{1.628022in}{3.105096in}}%
\pgfpathcurveto{\pgfqpoint{1.636259in}{3.105096in}}{\pgfqpoint{1.644159in}{3.108368in}}{\pgfqpoint{1.649983in}{3.114192in}}%
\pgfpathcurveto{\pgfqpoint{1.655807in}{3.120016in}}{\pgfqpoint{1.659079in}{3.127916in}}{\pgfqpoint{1.659079in}{3.136152in}}%
\pgfpathcurveto{\pgfqpoint{1.659079in}{3.144388in}}{\pgfqpoint{1.655807in}{3.152288in}}{\pgfqpoint{1.649983in}{3.158112in}}%
\pgfpathcurveto{\pgfqpoint{1.644159in}{3.163936in}}{\pgfqpoint{1.636259in}{3.167209in}}{\pgfqpoint{1.628022in}{3.167209in}}%
\pgfpathcurveto{\pgfqpoint{1.619786in}{3.167209in}}{\pgfqpoint{1.611886in}{3.163936in}}{\pgfqpoint{1.606062in}{3.158112in}}%
\pgfpathcurveto{\pgfqpoint{1.600238in}{3.152288in}}{\pgfqpoint{1.596966in}{3.144388in}}{\pgfqpoint{1.596966in}{3.136152in}}%
\pgfpathcurveto{\pgfqpoint{1.596966in}{3.127916in}}{\pgfqpoint{1.600238in}{3.120016in}}{\pgfqpoint{1.606062in}{3.114192in}}%
\pgfpathcurveto{\pgfqpoint{1.611886in}{3.108368in}}{\pgfqpoint{1.619786in}{3.105096in}}{\pgfqpoint{1.628022in}{3.105096in}}%
\pgfpathclose%
\pgfusepath{stroke,fill}%
\end{pgfscope}%
\begin{pgfscope}%
\pgfpathrectangle{\pgfqpoint{0.100000in}{0.220728in}}{\pgfqpoint{3.696000in}{3.696000in}}%
\pgfusepath{clip}%
\pgfsetbuttcap%
\pgfsetroundjoin%
\definecolor{currentfill}{rgb}{0.121569,0.466667,0.705882}%
\pgfsetfillcolor{currentfill}%
\pgfsetfillopacity{0.301455}%
\pgfsetlinewidth{1.003750pt}%
\definecolor{currentstroke}{rgb}{0.121569,0.466667,0.705882}%
\pgfsetstrokecolor{currentstroke}%
\pgfsetstrokeopacity{0.301455}%
\pgfsetdash{}{0pt}%
\pgfpathmoveto{\pgfqpoint{1.627582in}{3.104092in}}%
\pgfpathcurveto{\pgfqpoint{1.635818in}{3.104092in}}{\pgfqpoint{1.643718in}{3.107365in}}{\pgfqpoint{1.649542in}{3.113189in}}%
\pgfpathcurveto{\pgfqpoint{1.655366in}{3.119013in}}{\pgfqpoint{1.658638in}{3.126913in}}{\pgfqpoint{1.658638in}{3.135149in}}%
\pgfpathcurveto{\pgfqpoint{1.658638in}{3.143385in}}{\pgfqpoint{1.655366in}{3.151285in}}{\pgfqpoint{1.649542in}{3.157109in}}%
\pgfpathcurveto{\pgfqpoint{1.643718in}{3.162933in}}{\pgfqpoint{1.635818in}{3.166205in}}{\pgfqpoint{1.627582in}{3.166205in}}%
\pgfpathcurveto{\pgfqpoint{1.619345in}{3.166205in}}{\pgfqpoint{1.611445in}{3.162933in}}{\pgfqpoint{1.605621in}{3.157109in}}%
\pgfpathcurveto{\pgfqpoint{1.599797in}{3.151285in}}{\pgfqpoint{1.596525in}{3.143385in}}{\pgfqpoint{1.596525in}{3.135149in}}%
\pgfpathcurveto{\pgfqpoint{1.596525in}{3.126913in}}{\pgfqpoint{1.599797in}{3.119013in}}{\pgfqpoint{1.605621in}{3.113189in}}%
\pgfpathcurveto{\pgfqpoint{1.611445in}{3.107365in}}{\pgfqpoint{1.619345in}{3.104092in}}{\pgfqpoint{1.627582in}{3.104092in}}%
\pgfpathclose%
\pgfusepath{stroke,fill}%
\end{pgfscope}%
\begin{pgfscope}%
\pgfpathrectangle{\pgfqpoint{0.100000in}{0.220728in}}{\pgfqpoint{3.696000in}{3.696000in}}%
\pgfusepath{clip}%
\pgfsetbuttcap%
\pgfsetroundjoin%
\definecolor{currentfill}{rgb}{0.121569,0.466667,0.705882}%
\pgfsetfillcolor{currentfill}%
\pgfsetfillopacity{0.301756}%
\pgfsetlinewidth{1.003750pt}%
\definecolor{currentstroke}{rgb}{0.121569,0.466667,0.705882}%
\pgfsetstrokecolor{currentstroke}%
\pgfsetstrokeopacity{0.301756}%
\pgfsetdash{}{0pt}%
\pgfpathmoveto{\pgfqpoint{1.626774in}{3.102312in}}%
\pgfpathcurveto{\pgfqpoint{1.635010in}{3.102312in}}{\pgfqpoint{1.642910in}{3.105585in}}{\pgfqpoint{1.648734in}{3.111409in}}%
\pgfpathcurveto{\pgfqpoint{1.654558in}{3.117233in}}{\pgfqpoint{1.657830in}{3.125133in}}{\pgfqpoint{1.657830in}{3.133369in}}%
\pgfpathcurveto{\pgfqpoint{1.657830in}{3.141605in}}{\pgfqpoint{1.654558in}{3.149505in}}{\pgfqpoint{1.648734in}{3.155329in}}%
\pgfpathcurveto{\pgfqpoint{1.642910in}{3.161153in}}{\pgfqpoint{1.635010in}{3.164425in}}{\pgfqpoint{1.626774in}{3.164425in}}%
\pgfpathcurveto{\pgfqpoint{1.618537in}{3.164425in}}{\pgfqpoint{1.610637in}{3.161153in}}{\pgfqpoint{1.604813in}{3.155329in}}%
\pgfpathcurveto{\pgfqpoint{1.598990in}{3.149505in}}{\pgfqpoint{1.595717in}{3.141605in}}{\pgfqpoint{1.595717in}{3.133369in}}%
\pgfpathcurveto{\pgfqpoint{1.595717in}{3.125133in}}{\pgfqpoint{1.598990in}{3.117233in}}{\pgfqpoint{1.604813in}{3.111409in}}%
\pgfpathcurveto{\pgfqpoint{1.610637in}{3.105585in}}{\pgfqpoint{1.618537in}{3.102312in}}{\pgfqpoint{1.626774in}{3.102312in}}%
\pgfpathclose%
\pgfusepath{stroke,fill}%
\end{pgfscope}%
\begin{pgfscope}%
\pgfpathrectangle{\pgfqpoint{0.100000in}{0.220728in}}{\pgfqpoint{3.696000in}{3.696000in}}%
\pgfusepath{clip}%
\pgfsetbuttcap%
\pgfsetroundjoin%
\definecolor{currentfill}{rgb}{0.121569,0.466667,0.705882}%
\pgfsetfillcolor{currentfill}%
\pgfsetfillopacity{0.302322}%
\pgfsetlinewidth{1.003750pt}%
\definecolor{currentstroke}{rgb}{0.121569,0.466667,0.705882}%
\pgfsetstrokecolor{currentstroke}%
\pgfsetstrokeopacity{0.302322}%
\pgfsetdash{}{0pt}%
\pgfpathmoveto{\pgfqpoint{1.625340in}{3.099139in}}%
\pgfpathcurveto{\pgfqpoint{1.633576in}{3.099139in}}{\pgfqpoint{1.641476in}{3.102411in}}{\pgfqpoint{1.647300in}{3.108235in}}%
\pgfpathcurveto{\pgfqpoint{1.653124in}{3.114059in}}{\pgfqpoint{1.656396in}{3.121959in}}{\pgfqpoint{1.656396in}{3.130195in}}%
\pgfpathcurveto{\pgfqpoint{1.656396in}{3.138431in}}{\pgfqpoint{1.653124in}{3.146332in}}{\pgfqpoint{1.647300in}{3.152155in}}%
\pgfpathcurveto{\pgfqpoint{1.641476in}{3.157979in}}{\pgfqpoint{1.633576in}{3.161252in}}{\pgfqpoint{1.625340in}{3.161252in}}%
\pgfpathcurveto{\pgfqpoint{1.617104in}{3.161252in}}{\pgfqpoint{1.609204in}{3.157979in}}{\pgfqpoint{1.603380in}{3.152155in}}%
\pgfpathcurveto{\pgfqpoint{1.597556in}{3.146332in}}{\pgfqpoint{1.594283in}{3.138431in}}{\pgfqpoint{1.594283in}{3.130195in}}%
\pgfpathcurveto{\pgfqpoint{1.594283in}{3.121959in}}{\pgfqpoint{1.597556in}{3.114059in}}{\pgfqpoint{1.603380in}{3.108235in}}%
\pgfpathcurveto{\pgfqpoint{1.609204in}{3.102411in}}{\pgfqpoint{1.617104in}{3.099139in}}{\pgfqpoint{1.625340in}{3.099139in}}%
\pgfpathclose%
\pgfusepath{stroke,fill}%
\end{pgfscope}%
\begin{pgfscope}%
\pgfpathrectangle{\pgfqpoint{0.100000in}{0.220728in}}{\pgfqpoint{3.696000in}{3.696000in}}%
\pgfusepath{clip}%
\pgfsetbuttcap%
\pgfsetroundjoin%
\definecolor{currentfill}{rgb}{0.121569,0.466667,0.705882}%
\pgfsetfillcolor{currentfill}%
\pgfsetfillopacity{0.302523}%
\pgfsetlinewidth{1.003750pt}%
\definecolor{currentstroke}{rgb}{0.121569,0.466667,0.705882}%
\pgfsetstrokecolor{currentstroke}%
\pgfsetstrokeopacity{0.302523}%
\pgfsetdash{}{0pt}%
\pgfpathmoveto{\pgfqpoint{1.681683in}{3.133624in}}%
\pgfpathcurveto{\pgfqpoint{1.689920in}{3.133624in}}{\pgfqpoint{1.697820in}{3.136897in}}{\pgfqpoint{1.703644in}{3.142720in}}%
\pgfpathcurveto{\pgfqpoint{1.709468in}{3.148544in}}{\pgfqpoint{1.712740in}{3.156444in}}{\pgfqpoint{1.712740in}{3.164681in}}%
\pgfpathcurveto{\pgfqpoint{1.712740in}{3.172917in}}{\pgfqpoint{1.709468in}{3.180817in}}{\pgfqpoint{1.703644in}{3.186641in}}%
\pgfpathcurveto{\pgfqpoint{1.697820in}{3.192465in}}{\pgfqpoint{1.689920in}{3.195737in}}{\pgfqpoint{1.681683in}{3.195737in}}%
\pgfpathcurveto{\pgfqpoint{1.673447in}{3.195737in}}{\pgfqpoint{1.665547in}{3.192465in}}{\pgfqpoint{1.659723in}{3.186641in}}%
\pgfpathcurveto{\pgfqpoint{1.653899in}{3.180817in}}{\pgfqpoint{1.650627in}{3.172917in}}{\pgfqpoint{1.650627in}{3.164681in}}%
\pgfpathcurveto{\pgfqpoint{1.650627in}{3.156444in}}{\pgfqpoint{1.653899in}{3.148544in}}{\pgfqpoint{1.659723in}{3.142720in}}%
\pgfpathcurveto{\pgfqpoint{1.665547in}{3.136897in}}{\pgfqpoint{1.673447in}{3.133624in}}{\pgfqpoint{1.681683in}{3.133624in}}%
\pgfpathclose%
\pgfusepath{stroke,fill}%
\end{pgfscope}%
\begin{pgfscope}%
\pgfpathrectangle{\pgfqpoint{0.100000in}{0.220728in}}{\pgfqpoint{3.696000in}{3.696000in}}%
\pgfusepath{clip}%
\pgfsetbuttcap%
\pgfsetroundjoin%
\definecolor{currentfill}{rgb}{0.121569,0.466667,0.705882}%
\pgfsetfillcolor{currentfill}%
\pgfsetfillopacity{0.303210}%
\pgfsetlinewidth{1.003750pt}%
\definecolor{currentstroke}{rgb}{0.121569,0.466667,0.705882}%
\pgfsetstrokecolor{currentstroke}%
\pgfsetstrokeopacity{0.303210}%
\pgfsetdash{}{0pt}%
\pgfpathmoveto{\pgfqpoint{1.622037in}{3.093372in}}%
\pgfpathcurveto{\pgfqpoint{1.630273in}{3.093372in}}{\pgfqpoint{1.638173in}{3.096644in}}{\pgfqpoint{1.643997in}{3.102468in}}%
\pgfpathcurveto{\pgfqpoint{1.649821in}{3.108292in}}{\pgfqpoint{1.653093in}{3.116192in}}{\pgfqpoint{1.653093in}{3.124428in}}%
\pgfpathcurveto{\pgfqpoint{1.653093in}{3.132664in}}{\pgfqpoint{1.649821in}{3.140565in}}{\pgfqpoint{1.643997in}{3.146388in}}%
\pgfpathcurveto{\pgfqpoint{1.638173in}{3.152212in}}{\pgfqpoint{1.630273in}{3.155485in}}{\pgfqpoint{1.622037in}{3.155485in}}%
\pgfpathcurveto{\pgfqpoint{1.613801in}{3.155485in}}{\pgfqpoint{1.605901in}{3.152212in}}{\pgfqpoint{1.600077in}{3.146388in}}%
\pgfpathcurveto{\pgfqpoint{1.594253in}{3.140565in}}{\pgfqpoint{1.590980in}{3.132664in}}{\pgfqpoint{1.590980in}{3.124428in}}%
\pgfpathcurveto{\pgfqpoint{1.590980in}{3.116192in}}{\pgfqpoint{1.594253in}{3.108292in}}{\pgfqpoint{1.600077in}{3.102468in}}%
\pgfpathcurveto{\pgfqpoint{1.605901in}{3.096644in}}{\pgfqpoint{1.613801in}{3.093372in}}{\pgfqpoint{1.622037in}{3.093372in}}%
\pgfpathclose%
\pgfusepath{stroke,fill}%
\end{pgfscope}%
\begin{pgfscope}%
\pgfpathrectangle{\pgfqpoint{0.100000in}{0.220728in}}{\pgfqpoint{3.696000in}{3.696000in}}%
\pgfusepath{clip}%
\pgfsetbuttcap%
\pgfsetroundjoin%
\definecolor{currentfill}{rgb}{0.121569,0.466667,0.705882}%
\pgfsetfillcolor{currentfill}%
\pgfsetfillopacity{0.304986}%
\pgfsetlinewidth{1.003750pt}%
\definecolor{currentstroke}{rgb}{0.121569,0.466667,0.705882}%
\pgfsetstrokecolor{currentstroke}%
\pgfsetstrokeopacity{0.304986}%
\pgfsetdash{}{0pt}%
\pgfpathmoveto{\pgfqpoint{1.616681in}{3.082966in}}%
\pgfpathcurveto{\pgfqpoint{1.624917in}{3.082966in}}{\pgfqpoint{1.632817in}{3.086238in}}{\pgfqpoint{1.638641in}{3.092062in}}%
\pgfpathcurveto{\pgfqpoint{1.644465in}{3.097886in}}{\pgfqpoint{1.647738in}{3.105786in}}{\pgfqpoint{1.647738in}{3.114023in}}%
\pgfpathcurveto{\pgfqpoint{1.647738in}{3.122259in}}{\pgfqpoint{1.644465in}{3.130159in}}{\pgfqpoint{1.638641in}{3.135983in}}%
\pgfpathcurveto{\pgfqpoint{1.632817in}{3.141807in}}{\pgfqpoint{1.624917in}{3.145079in}}{\pgfqpoint{1.616681in}{3.145079in}}%
\pgfpathcurveto{\pgfqpoint{1.608445in}{3.145079in}}{\pgfqpoint{1.600545in}{3.141807in}}{\pgfqpoint{1.594721in}{3.135983in}}%
\pgfpathcurveto{\pgfqpoint{1.588897in}{3.130159in}}{\pgfqpoint{1.585625in}{3.122259in}}{\pgfqpoint{1.585625in}{3.114023in}}%
\pgfpathcurveto{\pgfqpoint{1.585625in}{3.105786in}}{\pgfqpoint{1.588897in}{3.097886in}}{\pgfqpoint{1.594721in}{3.092062in}}%
\pgfpathcurveto{\pgfqpoint{1.600545in}{3.086238in}}{\pgfqpoint{1.608445in}{3.082966in}}{\pgfqpoint{1.616681in}{3.082966in}}%
\pgfpathclose%
\pgfusepath{stroke,fill}%
\end{pgfscope}%
\begin{pgfscope}%
\pgfpathrectangle{\pgfqpoint{0.100000in}{0.220728in}}{\pgfqpoint{3.696000in}{3.696000in}}%
\pgfusepath{clip}%
\pgfsetbuttcap%
\pgfsetroundjoin%
\definecolor{currentfill}{rgb}{0.121569,0.466667,0.705882}%
\pgfsetfillcolor{currentfill}%
\pgfsetfillopacity{0.306192}%
\pgfsetlinewidth{1.003750pt}%
\definecolor{currentstroke}{rgb}{0.121569,0.466667,0.705882}%
\pgfsetstrokecolor{currentstroke}%
\pgfsetstrokeopacity{0.306192}%
\pgfsetdash{}{0pt}%
\pgfpathmoveto{\pgfqpoint{1.698646in}{3.134050in}}%
\pgfpathcurveto{\pgfqpoint{1.706883in}{3.134050in}}{\pgfqpoint{1.714783in}{3.137323in}}{\pgfqpoint{1.720607in}{3.143147in}}%
\pgfpathcurveto{\pgfqpoint{1.726430in}{3.148970in}}{\pgfqpoint{1.729703in}{3.156871in}}{\pgfqpoint{1.729703in}{3.165107in}}%
\pgfpathcurveto{\pgfqpoint{1.729703in}{3.173343in}}{\pgfqpoint{1.726430in}{3.181243in}}{\pgfqpoint{1.720607in}{3.187067in}}%
\pgfpathcurveto{\pgfqpoint{1.714783in}{3.192891in}}{\pgfqpoint{1.706883in}{3.196163in}}{\pgfqpoint{1.698646in}{3.196163in}}%
\pgfpathcurveto{\pgfqpoint{1.690410in}{3.196163in}}{\pgfqpoint{1.682510in}{3.192891in}}{\pgfqpoint{1.676686in}{3.187067in}}%
\pgfpathcurveto{\pgfqpoint{1.670862in}{3.181243in}}{\pgfqpoint{1.667590in}{3.173343in}}{\pgfqpoint{1.667590in}{3.165107in}}%
\pgfpathcurveto{\pgfqpoint{1.667590in}{3.156871in}}{\pgfqpoint{1.670862in}{3.148970in}}{\pgfqpoint{1.676686in}{3.143147in}}%
\pgfpathcurveto{\pgfqpoint{1.682510in}{3.137323in}}{\pgfqpoint{1.690410in}{3.134050in}}{\pgfqpoint{1.698646in}{3.134050in}}%
\pgfpathclose%
\pgfusepath{stroke,fill}%
\end{pgfscope}%
\begin{pgfscope}%
\pgfpathrectangle{\pgfqpoint{0.100000in}{0.220728in}}{\pgfqpoint{3.696000in}{3.696000in}}%
\pgfusepath{clip}%
\pgfsetbuttcap%
\pgfsetroundjoin%
\definecolor{currentfill}{rgb}{0.121569,0.466667,0.705882}%
\pgfsetfillcolor{currentfill}%
\pgfsetfillopacity{0.308304}%
\pgfsetlinewidth{1.003750pt}%
\definecolor{currentstroke}{rgb}{0.121569,0.466667,0.705882}%
\pgfsetstrokecolor{currentstroke}%
\pgfsetstrokeopacity{0.308304}%
\pgfsetdash{}{0pt}%
\pgfpathmoveto{\pgfqpoint{1.607209in}{3.064171in}}%
\pgfpathcurveto{\pgfqpoint{1.615446in}{3.064171in}}{\pgfqpoint{1.623346in}{3.067443in}}{\pgfqpoint{1.629169in}{3.073267in}}%
\pgfpathcurveto{\pgfqpoint{1.634993in}{3.079091in}}{\pgfqpoint{1.638266in}{3.086991in}}{\pgfqpoint{1.638266in}{3.095228in}}%
\pgfpathcurveto{\pgfqpoint{1.638266in}{3.103464in}}{\pgfqpoint{1.634993in}{3.111364in}}{\pgfqpoint{1.629169in}{3.117188in}}%
\pgfpathcurveto{\pgfqpoint{1.623346in}{3.123012in}}{\pgfqpoint{1.615446in}{3.126284in}}{\pgfqpoint{1.607209in}{3.126284in}}%
\pgfpathcurveto{\pgfqpoint{1.598973in}{3.126284in}}{\pgfqpoint{1.591073in}{3.123012in}}{\pgfqpoint{1.585249in}{3.117188in}}%
\pgfpathcurveto{\pgfqpoint{1.579425in}{3.111364in}}{\pgfqpoint{1.576153in}{3.103464in}}{\pgfqpoint{1.576153in}{3.095228in}}%
\pgfpathcurveto{\pgfqpoint{1.576153in}{3.086991in}}{\pgfqpoint{1.579425in}{3.079091in}}{\pgfqpoint{1.585249in}{3.073267in}}%
\pgfpathcurveto{\pgfqpoint{1.591073in}{3.067443in}}{\pgfqpoint{1.598973in}{3.064171in}}{\pgfqpoint{1.607209in}{3.064171in}}%
\pgfpathclose%
\pgfusepath{stroke,fill}%
\end{pgfscope}%
\begin{pgfscope}%
\pgfpathrectangle{\pgfqpoint{0.100000in}{0.220728in}}{\pgfqpoint{3.696000in}{3.696000in}}%
\pgfusepath{clip}%
\pgfsetbuttcap%
\pgfsetroundjoin%
\definecolor{currentfill}{rgb}{0.121569,0.466667,0.705882}%
\pgfsetfillcolor{currentfill}%
\pgfsetfillopacity{0.309620}%
\pgfsetlinewidth{1.003750pt}%
\definecolor{currentstroke}{rgb}{0.121569,0.466667,0.705882}%
\pgfsetstrokecolor{currentstroke}%
\pgfsetstrokeopacity{0.309620}%
\pgfsetdash{}{0pt}%
\pgfpathmoveto{\pgfqpoint{1.718933in}{3.133042in}}%
\pgfpathcurveto{\pgfqpoint{1.727169in}{3.133042in}}{\pgfqpoint{1.735069in}{3.136315in}}{\pgfqpoint{1.740893in}{3.142138in}}%
\pgfpathcurveto{\pgfqpoint{1.746717in}{3.147962in}}{\pgfqpoint{1.749989in}{3.155862in}}{\pgfqpoint{1.749989in}{3.164099in}}%
\pgfpathcurveto{\pgfqpoint{1.749989in}{3.172335in}}{\pgfqpoint{1.746717in}{3.180235in}}{\pgfqpoint{1.740893in}{3.186059in}}%
\pgfpathcurveto{\pgfqpoint{1.735069in}{3.191883in}}{\pgfqpoint{1.727169in}{3.195155in}}{\pgfqpoint{1.718933in}{3.195155in}}%
\pgfpathcurveto{\pgfqpoint{1.710697in}{3.195155in}}{\pgfqpoint{1.702797in}{3.191883in}}{\pgfqpoint{1.696973in}{3.186059in}}%
\pgfpathcurveto{\pgfqpoint{1.691149in}{3.180235in}}{\pgfqpoint{1.687876in}{3.172335in}}{\pgfqpoint{1.687876in}{3.164099in}}%
\pgfpathcurveto{\pgfqpoint{1.687876in}{3.155862in}}{\pgfqpoint{1.691149in}{3.147962in}}{\pgfqpoint{1.696973in}{3.142138in}}%
\pgfpathcurveto{\pgfqpoint{1.702797in}{3.136315in}}{\pgfqpoint{1.710697in}{3.133042in}}{\pgfqpoint{1.718933in}{3.133042in}}%
\pgfpathclose%
\pgfusepath{stroke,fill}%
\end{pgfscope}%
\begin{pgfscope}%
\pgfpathrectangle{\pgfqpoint{0.100000in}{0.220728in}}{\pgfqpoint{3.696000in}{3.696000in}}%
\pgfusepath{clip}%
\pgfsetbuttcap%
\pgfsetroundjoin%
\definecolor{currentfill}{rgb}{0.121569,0.466667,0.705882}%
\pgfsetfillcolor{currentfill}%
\pgfsetfillopacity{0.312952}%
\pgfsetlinewidth{1.003750pt}%
\definecolor{currentstroke}{rgb}{0.121569,0.466667,0.705882}%
\pgfsetstrokecolor{currentstroke}%
\pgfsetstrokeopacity{0.312952}%
\pgfsetdash{}{0pt}%
\pgfpathmoveto{\pgfqpoint{1.743901in}{3.132974in}}%
\pgfpathcurveto{\pgfqpoint{1.752137in}{3.132974in}}{\pgfqpoint{1.760037in}{3.136246in}}{\pgfqpoint{1.765861in}{3.142070in}}%
\pgfpathcurveto{\pgfqpoint{1.771685in}{3.147894in}}{\pgfqpoint{1.774957in}{3.155794in}}{\pgfqpoint{1.774957in}{3.164030in}}%
\pgfpathcurveto{\pgfqpoint{1.774957in}{3.172266in}}{\pgfqpoint{1.771685in}{3.180167in}}{\pgfqpoint{1.765861in}{3.185990in}}%
\pgfpathcurveto{\pgfqpoint{1.760037in}{3.191814in}}{\pgfqpoint{1.752137in}{3.195087in}}{\pgfqpoint{1.743901in}{3.195087in}}%
\pgfpathcurveto{\pgfqpoint{1.735664in}{3.195087in}}{\pgfqpoint{1.727764in}{3.191814in}}{\pgfqpoint{1.721940in}{3.185990in}}%
\pgfpathcurveto{\pgfqpoint{1.716116in}{3.180167in}}{\pgfqpoint{1.712844in}{3.172266in}}{\pgfqpoint{1.712844in}{3.164030in}}%
\pgfpathcurveto{\pgfqpoint{1.712844in}{3.155794in}}{\pgfqpoint{1.716116in}{3.147894in}}{\pgfqpoint{1.721940in}{3.142070in}}%
\pgfpathcurveto{\pgfqpoint{1.727764in}{3.136246in}}{\pgfqpoint{1.735664in}{3.132974in}}{\pgfqpoint{1.743901in}{3.132974in}}%
\pgfpathclose%
\pgfusepath{stroke,fill}%
\end{pgfscope}%
\begin{pgfscope}%
\pgfpathrectangle{\pgfqpoint{0.100000in}{0.220728in}}{\pgfqpoint{3.696000in}{3.696000in}}%
\pgfusepath{clip}%
\pgfsetbuttcap%
\pgfsetroundjoin%
\definecolor{currentfill}{rgb}{0.121569,0.466667,0.705882}%
\pgfsetfillcolor{currentfill}%
\pgfsetfillopacity{0.313349}%
\pgfsetlinewidth{1.003750pt}%
\definecolor{currentstroke}{rgb}{0.121569,0.466667,0.705882}%
\pgfsetstrokecolor{currentstroke}%
\pgfsetstrokeopacity{0.313349}%
\pgfsetdash{}{0pt}%
\pgfpathmoveto{\pgfqpoint{1.586499in}{3.028855in}}%
\pgfpathcurveto{\pgfqpoint{1.594735in}{3.028855in}}{\pgfqpoint{1.602635in}{3.032127in}}{\pgfqpoint{1.608459in}{3.037951in}}%
\pgfpathcurveto{\pgfqpoint{1.614283in}{3.043775in}}{\pgfqpoint{1.617555in}{3.051675in}}{\pgfqpoint{1.617555in}{3.059912in}}%
\pgfpathcurveto{\pgfqpoint{1.617555in}{3.068148in}}{\pgfqpoint{1.614283in}{3.076048in}}{\pgfqpoint{1.608459in}{3.081872in}}%
\pgfpathcurveto{\pgfqpoint{1.602635in}{3.087696in}}{\pgfqpoint{1.594735in}{3.090968in}}{\pgfqpoint{1.586499in}{3.090968in}}%
\pgfpathcurveto{\pgfqpoint{1.578262in}{3.090968in}}{\pgfqpoint{1.570362in}{3.087696in}}{\pgfqpoint{1.564538in}{3.081872in}}%
\pgfpathcurveto{\pgfqpoint{1.558714in}{3.076048in}}{\pgfqpoint{1.555442in}{3.068148in}}{\pgfqpoint{1.555442in}{3.059912in}}%
\pgfpathcurveto{\pgfqpoint{1.555442in}{3.051675in}}{\pgfqpoint{1.558714in}{3.043775in}}{\pgfqpoint{1.564538in}{3.037951in}}%
\pgfpathcurveto{\pgfqpoint{1.570362in}{3.032127in}}{\pgfqpoint{1.578262in}{3.028855in}}{\pgfqpoint{1.586499in}{3.028855in}}%
\pgfpathclose%
\pgfusepath{stroke,fill}%
\end{pgfscope}%
\begin{pgfscope}%
\pgfpathrectangle{\pgfqpoint{0.100000in}{0.220728in}}{\pgfqpoint{3.696000in}{3.696000in}}%
\pgfusepath{clip}%
\pgfsetbuttcap%
\pgfsetroundjoin%
\definecolor{currentfill}{rgb}{0.121569,0.466667,0.705882}%
\pgfsetfillcolor{currentfill}%
\pgfsetfillopacity{0.314573}%
\pgfsetlinewidth{1.003750pt}%
\definecolor{currentstroke}{rgb}{0.121569,0.466667,0.705882}%
\pgfsetstrokecolor{currentstroke}%
\pgfsetstrokeopacity{0.314573}%
\pgfsetdash{}{0pt}%
\pgfpathmoveto{\pgfqpoint{1.757613in}{3.131612in}}%
\pgfpathcurveto{\pgfqpoint{1.765849in}{3.131612in}}{\pgfqpoint{1.773749in}{3.134885in}}{\pgfqpoint{1.779573in}{3.140709in}}%
\pgfpathcurveto{\pgfqpoint{1.785397in}{3.146533in}}{\pgfqpoint{1.788669in}{3.154433in}}{\pgfqpoint{1.788669in}{3.162669in}}%
\pgfpathcurveto{\pgfqpoint{1.788669in}{3.170905in}}{\pgfqpoint{1.785397in}{3.178805in}}{\pgfqpoint{1.779573in}{3.184629in}}%
\pgfpathcurveto{\pgfqpoint{1.773749in}{3.190453in}}{\pgfqpoint{1.765849in}{3.193725in}}{\pgfqpoint{1.757613in}{3.193725in}}%
\pgfpathcurveto{\pgfqpoint{1.749377in}{3.193725in}}{\pgfqpoint{1.741477in}{3.190453in}}{\pgfqpoint{1.735653in}{3.184629in}}%
\pgfpathcurveto{\pgfqpoint{1.729829in}{3.178805in}}{\pgfqpoint{1.726556in}{3.170905in}}{\pgfqpoint{1.726556in}{3.162669in}}%
\pgfpathcurveto{\pgfqpoint{1.726556in}{3.154433in}}{\pgfqpoint{1.729829in}{3.146533in}}{\pgfqpoint{1.735653in}{3.140709in}}%
\pgfpathcurveto{\pgfqpoint{1.741477in}{3.134885in}}{\pgfqpoint{1.749377in}{3.131612in}}{\pgfqpoint{1.757613in}{3.131612in}}%
\pgfpathclose%
\pgfusepath{stroke,fill}%
\end{pgfscope}%
\begin{pgfscope}%
\pgfpathrectangle{\pgfqpoint{0.100000in}{0.220728in}}{\pgfqpoint{3.696000in}{3.696000in}}%
\pgfusepath{clip}%
\pgfsetbuttcap%
\pgfsetroundjoin%
\definecolor{currentfill}{rgb}{0.121569,0.466667,0.705882}%
\pgfsetfillcolor{currentfill}%
\pgfsetfillopacity{0.317185}%
\pgfsetlinewidth{1.003750pt}%
\definecolor{currentstroke}{rgb}{0.121569,0.466667,0.705882}%
\pgfsetstrokecolor{currentstroke}%
\pgfsetstrokeopacity{0.317185}%
\pgfsetdash{}{0pt}%
\pgfpathmoveto{\pgfqpoint{1.773563in}{3.128562in}}%
\pgfpathcurveto{\pgfqpoint{1.781799in}{3.128562in}}{\pgfqpoint{1.789699in}{3.131834in}}{\pgfqpoint{1.795523in}{3.137658in}}%
\pgfpathcurveto{\pgfqpoint{1.801347in}{3.143482in}}{\pgfqpoint{1.804620in}{3.151382in}}{\pgfqpoint{1.804620in}{3.159618in}}%
\pgfpathcurveto{\pgfqpoint{1.804620in}{3.167855in}}{\pgfqpoint{1.801347in}{3.175755in}}{\pgfqpoint{1.795523in}{3.181579in}}%
\pgfpathcurveto{\pgfqpoint{1.789699in}{3.187403in}}{\pgfqpoint{1.781799in}{3.190675in}}{\pgfqpoint{1.773563in}{3.190675in}}%
\pgfpathcurveto{\pgfqpoint{1.765327in}{3.190675in}}{\pgfqpoint{1.757427in}{3.187403in}}{\pgfqpoint{1.751603in}{3.181579in}}%
\pgfpathcurveto{\pgfqpoint{1.745779in}{3.175755in}}{\pgfqpoint{1.742507in}{3.167855in}}{\pgfqpoint{1.742507in}{3.159618in}}%
\pgfpathcurveto{\pgfqpoint{1.742507in}{3.151382in}}{\pgfqpoint{1.745779in}{3.143482in}}{\pgfqpoint{1.751603in}{3.137658in}}%
\pgfpathcurveto{\pgfqpoint{1.757427in}{3.131834in}}{\pgfqpoint{1.765327in}{3.128562in}}{\pgfqpoint{1.773563in}{3.128562in}}%
\pgfpathclose%
\pgfusepath{stroke,fill}%
\end{pgfscope}%
\begin{pgfscope}%
\pgfpathrectangle{\pgfqpoint{0.100000in}{0.220728in}}{\pgfqpoint{3.696000in}{3.696000in}}%
\pgfusepath{clip}%
\pgfsetbuttcap%
\pgfsetroundjoin%
\definecolor{currentfill}{rgb}{0.121569,0.466667,0.705882}%
\pgfsetfillcolor{currentfill}%
\pgfsetfillopacity{0.319077}%
\pgfsetlinewidth{1.003750pt}%
\definecolor{currentstroke}{rgb}{0.121569,0.466667,0.705882}%
\pgfsetstrokecolor{currentstroke}%
\pgfsetstrokeopacity{0.319077}%
\pgfsetdash{}{0pt}%
\pgfpathmoveto{\pgfqpoint{1.574698in}{2.999741in}}%
\pgfpathcurveto{\pgfqpoint{1.582934in}{2.999741in}}{\pgfqpoint{1.590834in}{3.003013in}}{\pgfqpoint{1.596658in}{3.008837in}}%
\pgfpathcurveto{\pgfqpoint{1.602482in}{3.014661in}}{\pgfqpoint{1.605754in}{3.022561in}}{\pgfqpoint{1.605754in}{3.030798in}}%
\pgfpathcurveto{\pgfqpoint{1.605754in}{3.039034in}}{\pgfqpoint{1.602482in}{3.046934in}}{\pgfqpoint{1.596658in}{3.052758in}}%
\pgfpathcurveto{\pgfqpoint{1.590834in}{3.058582in}}{\pgfqpoint{1.582934in}{3.061854in}}{\pgfqpoint{1.574698in}{3.061854in}}%
\pgfpathcurveto{\pgfqpoint{1.566461in}{3.061854in}}{\pgfqpoint{1.558561in}{3.058582in}}{\pgfqpoint{1.552737in}{3.052758in}}%
\pgfpathcurveto{\pgfqpoint{1.546913in}{3.046934in}}{\pgfqpoint{1.543641in}{3.039034in}}{\pgfqpoint{1.543641in}{3.030798in}}%
\pgfpathcurveto{\pgfqpoint{1.543641in}{3.022561in}}{\pgfqpoint{1.546913in}{3.014661in}}{\pgfqpoint{1.552737in}{3.008837in}}%
\pgfpathcurveto{\pgfqpoint{1.558561in}{3.003013in}}{\pgfqpoint{1.566461in}{2.999741in}}{\pgfqpoint{1.574698in}{2.999741in}}%
\pgfpathclose%
\pgfusepath{stroke,fill}%
\end{pgfscope}%
\begin{pgfscope}%
\pgfpathrectangle{\pgfqpoint{0.100000in}{0.220728in}}{\pgfqpoint{3.696000in}{3.696000in}}%
\pgfusepath{clip}%
\pgfsetbuttcap%
\pgfsetroundjoin%
\definecolor{currentfill}{rgb}{0.121569,0.466667,0.705882}%
\pgfsetfillcolor{currentfill}%
\pgfsetfillopacity{0.319104}%
\pgfsetlinewidth{1.003750pt}%
\definecolor{currentstroke}{rgb}{0.121569,0.466667,0.705882}%
\pgfsetstrokecolor{currentstroke}%
\pgfsetstrokeopacity{0.319104}%
\pgfsetdash{}{0pt}%
\pgfpathmoveto{\pgfqpoint{1.781565in}{3.126964in}}%
\pgfpathcurveto{\pgfqpoint{1.789801in}{3.126964in}}{\pgfqpoint{1.797701in}{3.130237in}}{\pgfqpoint{1.803525in}{3.136060in}}%
\pgfpathcurveto{\pgfqpoint{1.809349in}{3.141884in}}{\pgfqpoint{1.812621in}{3.149784in}}{\pgfqpoint{1.812621in}{3.158021in}}%
\pgfpathcurveto{\pgfqpoint{1.812621in}{3.166257in}}{\pgfqpoint{1.809349in}{3.174157in}}{\pgfqpoint{1.803525in}{3.179981in}}%
\pgfpathcurveto{\pgfqpoint{1.797701in}{3.185805in}}{\pgfqpoint{1.789801in}{3.189077in}}{\pgfqpoint{1.781565in}{3.189077in}}%
\pgfpathcurveto{\pgfqpoint{1.773329in}{3.189077in}}{\pgfqpoint{1.765429in}{3.185805in}}{\pgfqpoint{1.759605in}{3.179981in}}%
\pgfpathcurveto{\pgfqpoint{1.753781in}{3.174157in}}{\pgfqpoint{1.750508in}{3.166257in}}{\pgfqpoint{1.750508in}{3.158021in}}%
\pgfpathcurveto{\pgfqpoint{1.750508in}{3.149784in}}{\pgfqpoint{1.753781in}{3.141884in}}{\pgfqpoint{1.759605in}{3.136060in}}%
\pgfpathcurveto{\pgfqpoint{1.765429in}{3.130237in}}{\pgfqpoint{1.773329in}{3.126964in}}{\pgfqpoint{1.781565in}{3.126964in}}%
\pgfpathclose%
\pgfusepath{stroke,fill}%
\end{pgfscope}%
\begin{pgfscope}%
\pgfpathrectangle{\pgfqpoint{0.100000in}{0.220728in}}{\pgfqpoint{3.696000in}{3.696000in}}%
\pgfusepath{clip}%
\pgfsetbuttcap%
\pgfsetroundjoin%
\definecolor{currentfill}{rgb}{0.121569,0.466667,0.705882}%
\pgfsetfillcolor{currentfill}%
\pgfsetfillopacity{0.321636}%
\pgfsetlinewidth{1.003750pt}%
\definecolor{currentstroke}{rgb}{0.121569,0.466667,0.705882}%
\pgfsetstrokecolor{currentstroke}%
\pgfsetstrokeopacity{0.321636}%
\pgfsetdash{}{0pt}%
\pgfpathmoveto{\pgfqpoint{1.791471in}{3.125739in}}%
\pgfpathcurveto{\pgfqpoint{1.799707in}{3.125739in}}{\pgfqpoint{1.807607in}{3.129011in}}{\pgfqpoint{1.813431in}{3.134835in}}%
\pgfpathcurveto{\pgfqpoint{1.819255in}{3.140659in}}{\pgfqpoint{1.822527in}{3.148559in}}{\pgfqpoint{1.822527in}{3.156795in}}%
\pgfpathcurveto{\pgfqpoint{1.822527in}{3.165031in}}{\pgfqpoint{1.819255in}{3.172932in}}{\pgfqpoint{1.813431in}{3.178755in}}%
\pgfpathcurveto{\pgfqpoint{1.807607in}{3.184579in}}{\pgfqpoint{1.799707in}{3.187852in}}{\pgfqpoint{1.791471in}{3.187852in}}%
\pgfpathcurveto{\pgfqpoint{1.783235in}{3.187852in}}{\pgfqpoint{1.775335in}{3.184579in}}{\pgfqpoint{1.769511in}{3.178755in}}%
\pgfpathcurveto{\pgfqpoint{1.763687in}{3.172932in}}{\pgfqpoint{1.760414in}{3.165031in}}{\pgfqpoint{1.760414in}{3.156795in}}%
\pgfpathcurveto{\pgfqpoint{1.760414in}{3.148559in}}{\pgfqpoint{1.763687in}{3.140659in}}{\pgfqpoint{1.769511in}{3.134835in}}%
\pgfpathcurveto{\pgfqpoint{1.775335in}{3.129011in}}{\pgfqpoint{1.783235in}{3.125739in}}{\pgfqpoint{1.791471in}{3.125739in}}%
\pgfpathclose%
\pgfusepath{stroke,fill}%
\end{pgfscope}%
\begin{pgfscope}%
\pgfpathrectangle{\pgfqpoint{0.100000in}{0.220728in}}{\pgfqpoint{3.696000in}{3.696000in}}%
\pgfusepath{clip}%
\pgfsetbuttcap%
\pgfsetroundjoin%
\definecolor{currentfill}{rgb}{0.121569,0.466667,0.705882}%
\pgfsetfillcolor{currentfill}%
\pgfsetfillopacity{0.322565}%
\pgfsetlinewidth{1.003750pt}%
\definecolor{currentstroke}{rgb}{0.121569,0.466667,0.705882}%
\pgfsetstrokecolor{currentstroke}%
\pgfsetstrokeopacity{0.322565}%
\pgfsetdash{}{0pt}%
\pgfpathmoveto{\pgfqpoint{1.559236in}{2.974258in}}%
\pgfpathcurveto{\pgfqpoint{1.567472in}{2.974258in}}{\pgfqpoint{1.575372in}{2.977530in}}{\pgfqpoint{1.581196in}{2.983354in}}%
\pgfpathcurveto{\pgfqpoint{1.587020in}{2.989178in}}{\pgfqpoint{1.590292in}{2.997078in}}{\pgfqpoint{1.590292in}{3.005314in}}%
\pgfpathcurveto{\pgfqpoint{1.590292in}{3.013550in}}{\pgfqpoint{1.587020in}{3.021450in}}{\pgfqpoint{1.581196in}{3.027274in}}%
\pgfpathcurveto{\pgfqpoint{1.575372in}{3.033098in}}{\pgfqpoint{1.567472in}{3.036371in}}{\pgfqpoint{1.559236in}{3.036371in}}%
\pgfpathcurveto{\pgfqpoint{1.550999in}{3.036371in}}{\pgfqpoint{1.543099in}{3.033098in}}{\pgfqpoint{1.537275in}{3.027274in}}%
\pgfpathcurveto{\pgfqpoint{1.531451in}{3.021450in}}{\pgfqpoint{1.528179in}{3.013550in}}{\pgfqpoint{1.528179in}{3.005314in}}%
\pgfpathcurveto{\pgfqpoint{1.528179in}{2.997078in}}{\pgfqpoint{1.531451in}{2.989178in}}{\pgfqpoint{1.537275in}{2.983354in}}%
\pgfpathcurveto{\pgfqpoint{1.543099in}{2.977530in}}{\pgfqpoint{1.550999in}{2.974258in}}{\pgfqpoint{1.559236in}{2.974258in}}%
\pgfpathclose%
\pgfusepath{stroke,fill}%
\end{pgfscope}%
\begin{pgfscope}%
\pgfpathrectangle{\pgfqpoint{0.100000in}{0.220728in}}{\pgfqpoint{3.696000in}{3.696000in}}%
\pgfusepath{clip}%
\pgfsetbuttcap%
\pgfsetroundjoin%
\definecolor{currentfill}{rgb}{0.121569,0.466667,0.705882}%
\pgfsetfillcolor{currentfill}%
\pgfsetfillopacity{0.325700}%
\pgfsetlinewidth{1.003750pt}%
\definecolor{currentstroke}{rgb}{0.121569,0.466667,0.705882}%
\pgfsetstrokecolor{currentstroke}%
\pgfsetstrokeopacity{0.325700}%
\pgfsetdash{}{0pt}%
\pgfpathmoveto{\pgfqpoint{1.553572in}{2.957381in}}%
\pgfpathcurveto{\pgfqpoint{1.561809in}{2.957381in}}{\pgfqpoint{1.569709in}{2.960653in}}{\pgfqpoint{1.575533in}{2.966477in}}%
\pgfpathcurveto{\pgfqpoint{1.581357in}{2.972301in}}{\pgfqpoint{1.584629in}{2.980201in}}{\pgfqpoint{1.584629in}{2.988437in}}%
\pgfpathcurveto{\pgfqpoint{1.584629in}{2.996673in}}{\pgfqpoint{1.581357in}{3.004573in}}{\pgfqpoint{1.575533in}{3.010397in}}%
\pgfpathcurveto{\pgfqpoint{1.569709in}{3.016221in}}{\pgfqpoint{1.561809in}{3.019494in}}{\pgfqpoint{1.553572in}{3.019494in}}%
\pgfpathcurveto{\pgfqpoint{1.545336in}{3.019494in}}{\pgfqpoint{1.537436in}{3.016221in}}{\pgfqpoint{1.531612in}{3.010397in}}%
\pgfpathcurveto{\pgfqpoint{1.525788in}{3.004573in}}{\pgfqpoint{1.522516in}{2.996673in}}{\pgfqpoint{1.522516in}{2.988437in}}%
\pgfpathcurveto{\pgfqpoint{1.522516in}{2.980201in}}{\pgfqpoint{1.525788in}{2.972301in}}{\pgfqpoint{1.531612in}{2.966477in}}%
\pgfpathcurveto{\pgfqpoint{1.537436in}{2.960653in}}{\pgfqpoint{1.545336in}{2.957381in}}{\pgfqpoint{1.553572in}{2.957381in}}%
\pgfpathclose%
\pgfusepath{stroke,fill}%
\end{pgfscope}%
\begin{pgfscope}%
\pgfpathrectangle{\pgfqpoint{0.100000in}{0.220728in}}{\pgfqpoint{3.696000in}{3.696000in}}%
\pgfusepath{clip}%
\pgfsetbuttcap%
\pgfsetroundjoin%
\definecolor{currentfill}{rgb}{0.121569,0.466667,0.705882}%
\pgfsetfillcolor{currentfill}%
\pgfsetfillopacity{0.325727}%
\pgfsetlinewidth{1.003750pt}%
\definecolor{currentstroke}{rgb}{0.121569,0.466667,0.705882}%
\pgfsetstrokecolor{currentstroke}%
\pgfsetstrokeopacity{0.325727}%
\pgfsetdash{}{0pt}%
\pgfpathmoveto{\pgfqpoint{1.811274in}{3.122590in}}%
\pgfpathcurveto{\pgfqpoint{1.819511in}{3.122590in}}{\pgfqpoint{1.827411in}{3.125862in}}{\pgfqpoint{1.833234in}{3.131686in}}%
\pgfpathcurveto{\pgfqpoint{1.839058in}{3.137510in}}{\pgfqpoint{1.842331in}{3.145410in}}{\pgfqpoint{1.842331in}{3.153646in}}%
\pgfpathcurveto{\pgfqpoint{1.842331in}{3.161882in}}{\pgfqpoint{1.839058in}{3.169782in}}{\pgfqpoint{1.833234in}{3.175606in}}%
\pgfpathcurveto{\pgfqpoint{1.827411in}{3.181430in}}{\pgfqpoint{1.819511in}{3.184703in}}{\pgfqpoint{1.811274in}{3.184703in}}%
\pgfpathcurveto{\pgfqpoint{1.803038in}{3.184703in}}{\pgfqpoint{1.795138in}{3.181430in}}{\pgfqpoint{1.789314in}{3.175606in}}%
\pgfpathcurveto{\pgfqpoint{1.783490in}{3.169782in}}{\pgfqpoint{1.780218in}{3.161882in}}{\pgfqpoint{1.780218in}{3.153646in}}%
\pgfpathcurveto{\pgfqpoint{1.780218in}{3.145410in}}{\pgfqpoint{1.783490in}{3.137510in}}{\pgfqpoint{1.789314in}{3.131686in}}%
\pgfpathcurveto{\pgfqpoint{1.795138in}{3.125862in}}{\pgfqpoint{1.803038in}{3.122590in}}{\pgfqpoint{1.811274in}{3.122590in}}%
\pgfpathclose%
\pgfusepath{stroke,fill}%
\end{pgfscope}%
\begin{pgfscope}%
\pgfpathrectangle{\pgfqpoint{0.100000in}{0.220728in}}{\pgfqpoint{3.696000in}{3.696000in}}%
\pgfusepath{clip}%
\pgfsetbuttcap%
\pgfsetroundjoin%
\definecolor{currentfill}{rgb}{0.121569,0.466667,0.705882}%
\pgfsetfillcolor{currentfill}%
\pgfsetfillopacity{0.327067}%
\pgfsetlinewidth{1.003750pt}%
\definecolor{currentstroke}{rgb}{0.121569,0.466667,0.705882}%
\pgfsetstrokecolor{currentstroke}%
\pgfsetstrokeopacity{0.327067}%
\pgfsetdash{}{0pt}%
\pgfpathmoveto{\pgfqpoint{1.548607in}{2.948143in}}%
\pgfpathcurveto{\pgfqpoint{1.556844in}{2.948143in}}{\pgfqpoint{1.564744in}{2.951415in}}{\pgfqpoint{1.570568in}{2.957239in}}%
\pgfpathcurveto{\pgfqpoint{1.576392in}{2.963063in}}{\pgfqpoint{1.579664in}{2.970963in}}{\pgfqpoint{1.579664in}{2.979199in}}%
\pgfpathcurveto{\pgfqpoint{1.579664in}{2.987435in}}{\pgfqpoint{1.576392in}{2.995335in}}{\pgfqpoint{1.570568in}{3.001159in}}%
\pgfpathcurveto{\pgfqpoint{1.564744in}{3.006983in}}{\pgfqpoint{1.556844in}{3.010256in}}{\pgfqpoint{1.548607in}{3.010256in}}%
\pgfpathcurveto{\pgfqpoint{1.540371in}{3.010256in}}{\pgfqpoint{1.532471in}{3.006983in}}{\pgfqpoint{1.526647in}{3.001159in}}%
\pgfpathcurveto{\pgfqpoint{1.520823in}{2.995335in}}{\pgfqpoint{1.517551in}{2.987435in}}{\pgfqpoint{1.517551in}{2.979199in}}%
\pgfpathcurveto{\pgfqpoint{1.517551in}{2.970963in}}{\pgfqpoint{1.520823in}{2.963063in}}{\pgfqpoint{1.526647in}{2.957239in}}%
\pgfpathcurveto{\pgfqpoint{1.532471in}{2.951415in}}{\pgfqpoint{1.540371in}{2.948143in}}{\pgfqpoint{1.548607in}{2.948143in}}%
\pgfpathclose%
\pgfusepath{stroke,fill}%
\end{pgfscope}%
\begin{pgfscope}%
\pgfpathrectangle{\pgfqpoint{0.100000in}{0.220728in}}{\pgfqpoint{3.696000in}{3.696000in}}%
\pgfusepath{clip}%
\pgfsetbuttcap%
\pgfsetroundjoin%
\definecolor{currentfill}{rgb}{0.121569,0.466667,0.705882}%
\pgfsetfillcolor{currentfill}%
\pgfsetfillopacity{0.327348}%
\pgfsetlinewidth{1.003750pt}%
\definecolor{currentstroke}{rgb}{0.121569,0.466667,0.705882}%
\pgfsetstrokecolor{currentstroke}%
\pgfsetstrokeopacity{0.327348}%
\pgfsetdash{}{0pt}%
\pgfpathmoveto{\pgfqpoint{1.547983in}{2.946667in}}%
\pgfpathcurveto{\pgfqpoint{1.556220in}{2.946667in}}{\pgfqpoint{1.564120in}{2.949940in}}{\pgfqpoint{1.569944in}{2.955763in}}%
\pgfpathcurveto{\pgfqpoint{1.575768in}{2.961587in}}{\pgfqpoint{1.579040in}{2.969487in}}{\pgfqpoint{1.579040in}{2.977724in}}%
\pgfpathcurveto{\pgfqpoint{1.579040in}{2.985960in}}{\pgfqpoint{1.575768in}{2.993860in}}{\pgfqpoint{1.569944in}{2.999684in}}%
\pgfpathcurveto{\pgfqpoint{1.564120in}{3.005508in}}{\pgfqpoint{1.556220in}{3.008780in}}{\pgfqpoint{1.547983in}{3.008780in}}%
\pgfpathcurveto{\pgfqpoint{1.539747in}{3.008780in}}{\pgfqpoint{1.531847in}{3.005508in}}{\pgfqpoint{1.526023in}{2.999684in}}%
\pgfpathcurveto{\pgfqpoint{1.520199in}{2.993860in}}{\pgfqpoint{1.516927in}{2.985960in}}{\pgfqpoint{1.516927in}{2.977724in}}%
\pgfpathcurveto{\pgfqpoint{1.516927in}{2.969487in}}{\pgfqpoint{1.520199in}{2.961587in}}{\pgfqpoint{1.526023in}{2.955763in}}%
\pgfpathcurveto{\pgfqpoint{1.531847in}{2.949940in}}{\pgfqpoint{1.539747in}{2.946667in}}{\pgfqpoint{1.547983in}{2.946667in}}%
\pgfpathclose%
\pgfusepath{stroke,fill}%
\end{pgfscope}%
\begin{pgfscope}%
\pgfpathrectangle{\pgfqpoint{0.100000in}{0.220728in}}{\pgfqpoint{3.696000in}{3.696000in}}%
\pgfusepath{clip}%
\pgfsetbuttcap%
\pgfsetroundjoin%
\definecolor{currentfill}{rgb}{0.121569,0.466667,0.705882}%
\pgfsetfillcolor{currentfill}%
\pgfsetfillopacity{0.327807}%
\pgfsetlinewidth{1.003750pt}%
\definecolor{currentstroke}{rgb}{0.121569,0.466667,0.705882}%
\pgfsetstrokecolor{currentstroke}%
\pgfsetstrokeopacity{0.327807}%
\pgfsetdash{}{0pt}%
\pgfpathmoveto{\pgfqpoint{1.546542in}{2.944005in}}%
\pgfpathcurveto{\pgfqpoint{1.554778in}{2.944005in}}{\pgfqpoint{1.562678in}{2.947277in}}{\pgfqpoint{1.568502in}{2.953101in}}%
\pgfpathcurveto{\pgfqpoint{1.574326in}{2.958925in}}{\pgfqpoint{1.577598in}{2.966825in}}{\pgfqpoint{1.577598in}{2.975061in}}%
\pgfpathcurveto{\pgfqpoint{1.577598in}{2.983298in}}{\pgfqpoint{1.574326in}{2.991198in}}{\pgfqpoint{1.568502in}{2.997022in}}%
\pgfpathcurveto{\pgfqpoint{1.562678in}{3.002846in}}{\pgfqpoint{1.554778in}{3.006118in}}{\pgfqpoint{1.546542in}{3.006118in}}%
\pgfpathcurveto{\pgfqpoint{1.538306in}{3.006118in}}{\pgfqpoint{1.530405in}{3.002846in}}{\pgfqpoint{1.524582in}{2.997022in}}%
\pgfpathcurveto{\pgfqpoint{1.518758in}{2.991198in}}{\pgfqpoint{1.515485in}{2.983298in}}{\pgfqpoint{1.515485in}{2.975061in}}%
\pgfpathcurveto{\pgfqpoint{1.515485in}{2.966825in}}{\pgfqpoint{1.518758in}{2.958925in}}{\pgfqpoint{1.524582in}{2.953101in}}%
\pgfpathcurveto{\pgfqpoint{1.530405in}{2.947277in}}{\pgfqpoint{1.538306in}{2.944005in}}{\pgfqpoint{1.546542in}{2.944005in}}%
\pgfpathclose%
\pgfusepath{stroke,fill}%
\end{pgfscope}%
\begin{pgfscope}%
\pgfpathrectangle{\pgfqpoint{0.100000in}{0.220728in}}{\pgfqpoint{3.696000in}{3.696000in}}%
\pgfusepath{clip}%
\pgfsetbuttcap%
\pgfsetroundjoin%
\definecolor{currentfill}{rgb}{0.121569,0.466667,0.705882}%
\pgfsetfillcolor{currentfill}%
\pgfsetfillopacity{0.328651}%
\pgfsetlinewidth{1.003750pt}%
\definecolor{currentstroke}{rgb}{0.121569,0.466667,0.705882}%
\pgfsetstrokecolor{currentstroke}%
\pgfsetstrokeopacity{0.328651}%
\pgfsetdash{}{0pt}%
\pgfpathmoveto{\pgfqpoint{1.544158in}{2.938975in}}%
\pgfpathcurveto{\pgfqpoint{1.552395in}{2.938975in}}{\pgfqpoint{1.560295in}{2.942247in}}{\pgfqpoint{1.566118in}{2.948071in}}%
\pgfpathcurveto{\pgfqpoint{1.571942in}{2.953895in}}{\pgfqpoint{1.575215in}{2.961795in}}{\pgfqpoint{1.575215in}{2.970031in}}%
\pgfpathcurveto{\pgfqpoint{1.575215in}{2.978268in}}{\pgfqpoint{1.571942in}{2.986168in}}{\pgfqpoint{1.566118in}{2.991992in}}%
\pgfpathcurveto{\pgfqpoint{1.560295in}{2.997816in}}{\pgfqpoint{1.552395in}{3.001088in}}{\pgfqpoint{1.544158in}{3.001088in}}%
\pgfpathcurveto{\pgfqpoint{1.535922in}{3.001088in}}{\pgfqpoint{1.528022in}{2.997816in}}{\pgfqpoint{1.522198in}{2.991992in}}%
\pgfpathcurveto{\pgfqpoint{1.516374in}{2.986168in}}{\pgfqpoint{1.513102in}{2.978268in}}{\pgfqpoint{1.513102in}{2.970031in}}%
\pgfpathcurveto{\pgfqpoint{1.513102in}{2.961795in}}{\pgfqpoint{1.516374in}{2.953895in}}{\pgfqpoint{1.522198in}{2.948071in}}%
\pgfpathcurveto{\pgfqpoint{1.528022in}{2.942247in}}{\pgfqpoint{1.535922in}{2.938975in}}{\pgfqpoint{1.544158in}{2.938975in}}%
\pgfpathclose%
\pgfusepath{stroke,fill}%
\end{pgfscope}%
\begin{pgfscope}%
\pgfpathrectangle{\pgfqpoint{0.100000in}{0.220728in}}{\pgfqpoint{3.696000in}{3.696000in}}%
\pgfusepath{clip}%
\pgfsetbuttcap%
\pgfsetroundjoin%
\definecolor{currentfill}{rgb}{0.121569,0.466667,0.705882}%
\pgfsetfillcolor{currentfill}%
\pgfsetfillopacity{0.330209}%
\pgfsetlinewidth{1.003750pt}%
\definecolor{currentstroke}{rgb}{0.121569,0.466667,0.705882}%
\pgfsetstrokecolor{currentstroke}%
\pgfsetstrokeopacity{0.330209}%
\pgfsetdash{}{0pt}%
\pgfpathmoveto{\pgfqpoint{1.539770in}{2.929988in}}%
\pgfpathcurveto{\pgfqpoint{1.548007in}{2.929988in}}{\pgfqpoint{1.555907in}{2.933261in}}{\pgfqpoint{1.561731in}{2.939085in}}%
\pgfpathcurveto{\pgfqpoint{1.567555in}{2.944909in}}{\pgfqpoint{1.570827in}{2.952809in}}{\pgfqpoint{1.570827in}{2.961045in}}%
\pgfpathcurveto{\pgfqpoint{1.570827in}{2.969281in}}{\pgfqpoint{1.567555in}{2.977181in}}{\pgfqpoint{1.561731in}{2.983005in}}%
\pgfpathcurveto{\pgfqpoint{1.555907in}{2.988829in}}{\pgfqpoint{1.548007in}{2.992101in}}{\pgfqpoint{1.539770in}{2.992101in}}%
\pgfpathcurveto{\pgfqpoint{1.531534in}{2.992101in}}{\pgfqpoint{1.523634in}{2.988829in}}{\pgfqpoint{1.517810in}{2.983005in}}%
\pgfpathcurveto{\pgfqpoint{1.511986in}{2.977181in}}{\pgfqpoint{1.508714in}{2.969281in}}{\pgfqpoint{1.508714in}{2.961045in}}%
\pgfpathcurveto{\pgfqpoint{1.508714in}{2.952809in}}{\pgfqpoint{1.511986in}{2.944909in}}{\pgfqpoint{1.517810in}{2.939085in}}%
\pgfpathcurveto{\pgfqpoint{1.523634in}{2.933261in}}{\pgfqpoint{1.531534in}{2.929988in}}{\pgfqpoint{1.539770in}{2.929988in}}%
\pgfpathclose%
\pgfusepath{stroke,fill}%
\end{pgfscope}%
\begin{pgfscope}%
\pgfpathrectangle{\pgfqpoint{0.100000in}{0.220728in}}{\pgfqpoint{3.696000in}{3.696000in}}%
\pgfusepath{clip}%
\pgfsetbuttcap%
\pgfsetroundjoin%
\definecolor{currentfill}{rgb}{0.121569,0.466667,0.705882}%
\pgfsetfillcolor{currentfill}%
\pgfsetfillopacity{0.331598}%
\pgfsetlinewidth{1.003750pt}%
\definecolor{currentstroke}{rgb}{0.121569,0.466667,0.705882}%
\pgfsetstrokecolor{currentstroke}%
\pgfsetstrokeopacity{0.331598}%
\pgfsetdash{}{0pt}%
\pgfpathmoveto{\pgfqpoint{1.835466in}{3.117630in}}%
\pgfpathcurveto{\pgfqpoint{1.843703in}{3.117630in}}{\pgfqpoint{1.851603in}{3.120902in}}{\pgfqpoint{1.857427in}{3.126726in}}%
\pgfpathcurveto{\pgfqpoint{1.863250in}{3.132550in}}{\pgfqpoint{1.866523in}{3.140450in}}{\pgfqpoint{1.866523in}{3.148687in}}%
\pgfpathcurveto{\pgfqpoint{1.866523in}{3.156923in}}{\pgfqpoint{1.863250in}{3.164823in}}{\pgfqpoint{1.857427in}{3.170647in}}%
\pgfpathcurveto{\pgfqpoint{1.851603in}{3.176471in}}{\pgfqpoint{1.843703in}{3.179743in}}{\pgfqpoint{1.835466in}{3.179743in}}%
\pgfpathcurveto{\pgfqpoint{1.827230in}{3.179743in}}{\pgfqpoint{1.819330in}{3.176471in}}{\pgfqpoint{1.813506in}{3.170647in}}%
\pgfpathcurveto{\pgfqpoint{1.807682in}{3.164823in}}{\pgfqpoint{1.804410in}{3.156923in}}{\pgfqpoint{1.804410in}{3.148687in}}%
\pgfpathcurveto{\pgfqpoint{1.804410in}{3.140450in}}{\pgfqpoint{1.807682in}{3.132550in}}{\pgfqpoint{1.813506in}{3.126726in}}%
\pgfpathcurveto{\pgfqpoint{1.819330in}{3.120902in}}{\pgfqpoint{1.827230in}{3.117630in}}{\pgfqpoint{1.835466in}{3.117630in}}%
\pgfpathclose%
\pgfusepath{stroke,fill}%
\end{pgfscope}%
\begin{pgfscope}%
\pgfpathrectangle{\pgfqpoint{0.100000in}{0.220728in}}{\pgfqpoint{3.696000in}{3.696000in}}%
\pgfusepath{clip}%
\pgfsetbuttcap%
\pgfsetroundjoin%
\definecolor{currentfill}{rgb}{0.121569,0.466667,0.705882}%
\pgfsetfillcolor{currentfill}%
\pgfsetfillopacity{0.333123}%
\pgfsetlinewidth{1.003750pt}%
\definecolor{currentstroke}{rgb}{0.121569,0.466667,0.705882}%
\pgfsetstrokecolor{currentstroke}%
\pgfsetstrokeopacity{0.333123}%
\pgfsetdash{}{0pt}%
\pgfpathmoveto{\pgfqpoint{1.532305in}{2.913574in}}%
\pgfpathcurveto{\pgfqpoint{1.540541in}{2.913574in}}{\pgfqpoint{1.548441in}{2.916846in}}{\pgfqpoint{1.554265in}{2.922670in}}%
\pgfpathcurveto{\pgfqpoint{1.560089in}{2.928494in}}{\pgfqpoint{1.563362in}{2.936394in}}{\pgfqpoint{1.563362in}{2.944630in}}%
\pgfpathcurveto{\pgfqpoint{1.563362in}{2.952866in}}{\pgfqpoint{1.560089in}{2.960766in}}{\pgfqpoint{1.554265in}{2.966590in}}%
\pgfpathcurveto{\pgfqpoint{1.548441in}{2.972414in}}{\pgfqpoint{1.540541in}{2.975687in}}{\pgfqpoint{1.532305in}{2.975687in}}%
\pgfpathcurveto{\pgfqpoint{1.524069in}{2.975687in}}{\pgfqpoint{1.516169in}{2.972414in}}{\pgfqpoint{1.510345in}{2.966590in}}%
\pgfpathcurveto{\pgfqpoint{1.504521in}{2.960766in}}{\pgfqpoint{1.501249in}{2.952866in}}{\pgfqpoint{1.501249in}{2.944630in}}%
\pgfpathcurveto{\pgfqpoint{1.501249in}{2.936394in}}{\pgfqpoint{1.504521in}{2.928494in}}{\pgfqpoint{1.510345in}{2.922670in}}%
\pgfpathcurveto{\pgfqpoint{1.516169in}{2.916846in}}{\pgfqpoint{1.524069in}{2.913574in}}{\pgfqpoint{1.532305in}{2.913574in}}%
\pgfpathclose%
\pgfusepath{stroke,fill}%
\end{pgfscope}%
\begin{pgfscope}%
\pgfpathrectangle{\pgfqpoint{0.100000in}{0.220728in}}{\pgfqpoint{3.696000in}{3.696000in}}%
\pgfusepath{clip}%
\pgfsetbuttcap%
\pgfsetroundjoin%
\definecolor{currentfill}{rgb}{0.121569,0.466667,0.705882}%
\pgfsetfillcolor{currentfill}%
\pgfsetfillopacity{0.337406}%
\pgfsetlinewidth{1.003750pt}%
\definecolor{currentstroke}{rgb}{0.121569,0.466667,0.705882}%
\pgfsetstrokecolor{currentstroke}%
\pgfsetstrokeopacity{0.337406}%
\pgfsetdash{}{0pt}%
\pgfpathmoveto{\pgfqpoint{1.867871in}{3.109071in}}%
\pgfpathcurveto{\pgfqpoint{1.876108in}{3.109071in}}{\pgfqpoint{1.884008in}{3.112343in}}{\pgfqpoint{1.889832in}{3.118167in}}%
\pgfpathcurveto{\pgfqpoint{1.895656in}{3.123991in}}{\pgfqpoint{1.898928in}{3.131891in}}{\pgfqpoint{1.898928in}{3.140127in}}%
\pgfpathcurveto{\pgfqpoint{1.898928in}{3.148364in}}{\pgfqpoint{1.895656in}{3.156264in}}{\pgfqpoint{1.889832in}{3.162088in}}%
\pgfpathcurveto{\pgfqpoint{1.884008in}{3.167912in}}{\pgfqpoint{1.876108in}{3.171184in}}{\pgfqpoint{1.867871in}{3.171184in}}%
\pgfpathcurveto{\pgfqpoint{1.859635in}{3.171184in}}{\pgfqpoint{1.851735in}{3.167912in}}{\pgfqpoint{1.845911in}{3.162088in}}%
\pgfpathcurveto{\pgfqpoint{1.840087in}{3.156264in}}{\pgfqpoint{1.836815in}{3.148364in}}{\pgfqpoint{1.836815in}{3.140127in}}%
\pgfpathcurveto{\pgfqpoint{1.836815in}{3.131891in}}{\pgfqpoint{1.840087in}{3.123991in}}{\pgfqpoint{1.845911in}{3.118167in}}%
\pgfpathcurveto{\pgfqpoint{1.851735in}{3.112343in}}{\pgfqpoint{1.859635in}{3.109071in}}{\pgfqpoint{1.867871in}{3.109071in}}%
\pgfpathclose%
\pgfusepath{stroke,fill}%
\end{pgfscope}%
\begin{pgfscope}%
\pgfpathrectangle{\pgfqpoint{0.100000in}{0.220728in}}{\pgfqpoint{3.696000in}{3.696000in}}%
\pgfusepath{clip}%
\pgfsetbuttcap%
\pgfsetroundjoin%
\definecolor{currentfill}{rgb}{0.121569,0.466667,0.705882}%
\pgfsetfillcolor{currentfill}%
\pgfsetfillopacity{0.338065}%
\pgfsetlinewidth{1.003750pt}%
\definecolor{currentstroke}{rgb}{0.121569,0.466667,0.705882}%
\pgfsetstrokecolor{currentstroke}%
\pgfsetstrokeopacity{0.338065}%
\pgfsetdash{}{0pt}%
\pgfpathmoveto{\pgfqpoint{1.517530in}{2.882897in}}%
\pgfpathcurveto{\pgfqpoint{1.525767in}{2.882897in}}{\pgfqpoint{1.533667in}{2.886169in}}{\pgfqpoint{1.539491in}{2.891993in}}%
\pgfpathcurveto{\pgfqpoint{1.545315in}{2.897817in}}{\pgfqpoint{1.548587in}{2.905717in}}{\pgfqpoint{1.548587in}{2.913953in}}%
\pgfpathcurveto{\pgfqpoint{1.548587in}{2.922190in}}{\pgfqpoint{1.545315in}{2.930090in}}{\pgfqpoint{1.539491in}{2.935914in}}%
\pgfpathcurveto{\pgfqpoint{1.533667in}{2.941738in}}{\pgfqpoint{1.525767in}{2.945010in}}{\pgfqpoint{1.517530in}{2.945010in}}%
\pgfpathcurveto{\pgfqpoint{1.509294in}{2.945010in}}{\pgfqpoint{1.501394in}{2.941738in}}{\pgfqpoint{1.495570in}{2.935914in}}%
\pgfpathcurveto{\pgfqpoint{1.489746in}{2.930090in}}{\pgfqpoint{1.486474in}{2.922190in}}{\pgfqpoint{1.486474in}{2.913953in}}%
\pgfpathcurveto{\pgfqpoint{1.486474in}{2.905717in}}{\pgfqpoint{1.489746in}{2.897817in}}{\pgfqpoint{1.495570in}{2.891993in}}%
\pgfpathcurveto{\pgfqpoint{1.501394in}{2.886169in}}{\pgfqpoint{1.509294in}{2.882897in}}{\pgfqpoint{1.517530in}{2.882897in}}%
\pgfpathclose%
\pgfusepath{stroke,fill}%
\end{pgfscope}%
\begin{pgfscope}%
\pgfpathrectangle{\pgfqpoint{0.100000in}{0.220728in}}{\pgfqpoint{3.696000in}{3.696000in}}%
\pgfusepath{clip}%
\pgfsetbuttcap%
\pgfsetroundjoin%
\definecolor{currentfill}{rgb}{0.121569,0.466667,0.705882}%
\pgfsetfillcolor{currentfill}%
\pgfsetfillopacity{0.344914}%
\pgfsetlinewidth{1.003750pt}%
\definecolor{currentstroke}{rgb}{0.121569,0.466667,0.705882}%
\pgfsetstrokecolor{currentstroke}%
\pgfsetstrokeopacity{0.344914}%
\pgfsetdash{}{0pt}%
\pgfpathmoveto{\pgfqpoint{1.900911in}{3.102087in}}%
\pgfpathcurveto{\pgfqpoint{1.909148in}{3.102087in}}{\pgfqpoint{1.917048in}{3.105359in}}{\pgfqpoint{1.922872in}{3.111183in}}%
\pgfpathcurveto{\pgfqpoint{1.928696in}{3.117007in}}{\pgfqpoint{1.931968in}{3.124907in}}{\pgfqpoint{1.931968in}{3.133143in}}%
\pgfpathcurveto{\pgfqpoint{1.931968in}{3.141379in}}{\pgfqpoint{1.928696in}{3.149279in}}{\pgfqpoint{1.922872in}{3.155103in}}%
\pgfpathcurveto{\pgfqpoint{1.917048in}{3.160927in}}{\pgfqpoint{1.909148in}{3.164200in}}{\pgfqpoint{1.900911in}{3.164200in}}%
\pgfpathcurveto{\pgfqpoint{1.892675in}{3.164200in}}{\pgfqpoint{1.884775in}{3.160927in}}{\pgfqpoint{1.878951in}{3.155103in}}%
\pgfpathcurveto{\pgfqpoint{1.873127in}{3.149279in}}{\pgfqpoint{1.869855in}{3.141379in}}{\pgfqpoint{1.869855in}{3.133143in}}%
\pgfpathcurveto{\pgfqpoint{1.869855in}{3.124907in}}{\pgfqpoint{1.873127in}{3.117007in}}{\pgfqpoint{1.878951in}{3.111183in}}%
\pgfpathcurveto{\pgfqpoint{1.884775in}{3.105359in}}{\pgfqpoint{1.892675in}{3.102087in}}{\pgfqpoint{1.900911in}{3.102087in}}%
\pgfpathclose%
\pgfusepath{stroke,fill}%
\end{pgfscope}%
\begin{pgfscope}%
\pgfpathrectangle{\pgfqpoint{0.100000in}{0.220728in}}{\pgfqpoint{3.696000in}{3.696000in}}%
\pgfusepath{clip}%
\pgfsetbuttcap%
\pgfsetroundjoin%
\definecolor{currentfill}{rgb}{0.121569,0.466667,0.705882}%
\pgfsetfillcolor{currentfill}%
\pgfsetfillopacity{0.347669}%
\pgfsetlinewidth{1.003750pt}%
\definecolor{currentstroke}{rgb}{0.121569,0.466667,0.705882}%
\pgfsetstrokecolor{currentstroke}%
\pgfsetstrokeopacity{0.347669}%
\pgfsetdash{}{0pt}%
\pgfpathmoveto{\pgfqpoint{1.498457in}{2.823694in}}%
\pgfpathcurveto{\pgfqpoint{1.506693in}{2.823694in}}{\pgfqpoint{1.514593in}{2.826966in}}{\pgfqpoint{1.520417in}{2.832790in}}%
\pgfpathcurveto{\pgfqpoint{1.526241in}{2.838614in}}{\pgfqpoint{1.529513in}{2.846514in}}{\pgfqpoint{1.529513in}{2.854750in}}%
\pgfpathcurveto{\pgfqpoint{1.529513in}{2.862986in}}{\pgfqpoint{1.526241in}{2.870886in}}{\pgfqpoint{1.520417in}{2.876710in}}%
\pgfpathcurveto{\pgfqpoint{1.514593in}{2.882534in}}{\pgfqpoint{1.506693in}{2.885807in}}{\pgfqpoint{1.498457in}{2.885807in}}%
\pgfpathcurveto{\pgfqpoint{1.490220in}{2.885807in}}{\pgfqpoint{1.482320in}{2.882534in}}{\pgfqpoint{1.476496in}{2.876710in}}%
\pgfpathcurveto{\pgfqpoint{1.470672in}{2.870886in}}{\pgfqpoint{1.467400in}{2.862986in}}{\pgfqpoint{1.467400in}{2.854750in}}%
\pgfpathcurveto{\pgfqpoint{1.467400in}{2.846514in}}{\pgfqpoint{1.470672in}{2.838614in}}{\pgfqpoint{1.476496in}{2.832790in}}%
\pgfpathcurveto{\pgfqpoint{1.482320in}{2.826966in}}{\pgfqpoint{1.490220in}{2.823694in}}{\pgfqpoint{1.498457in}{2.823694in}}%
\pgfpathclose%
\pgfusepath{stroke,fill}%
\end{pgfscope}%
\begin{pgfscope}%
\pgfpathrectangle{\pgfqpoint{0.100000in}{0.220728in}}{\pgfqpoint{3.696000in}{3.696000in}}%
\pgfusepath{clip}%
\pgfsetbuttcap%
\pgfsetroundjoin%
\definecolor{currentfill}{rgb}{0.121569,0.466667,0.705882}%
\pgfsetfillcolor{currentfill}%
\pgfsetfillopacity{0.354960}%
\pgfsetlinewidth{1.003750pt}%
\definecolor{currentstroke}{rgb}{0.121569,0.466667,0.705882}%
\pgfsetstrokecolor{currentstroke}%
\pgfsetstrokeopacity{0.354960}%
\pgfsetdash{}{0pt}%
\pgfpathmoveto{\pgfqpoint{1.464572in}{2.775409in}}%
\pgfpathcurveto{\pgfqpoint{1.472808in}{2.775409in}}{\pgfqpoint{1.480709in}{2.778681in}}{\pgfqpoint{1.486532in}{2.784505in}}%
\pgfpathcurveto{\pgfqpoint{1.492356in}{2.790329in}}{\pgfqpoint{1.495629in}{2.798229in}}{\pgfqpoint{1.495629in}{2.806465in}}%
\pgfpathcurveto{\pgfqpoint{1.495629in}{2.814701in}}{\pgfqpoint{1.492356in}{2.822601in}}{\pgfqpoint{1.486532in}{2.828425in}}%
\pgfpathcurveto{\pgfqpoint{1.480709in}{2.834249in}}{\pgfqpoint{1.472808in}{2.837522in}}{\pgfqpoint{1.464572in}{2.837522in}}%
\pgfpathcurveto{\pgfqpoint{1.456336in}{2.837522in}}{\pgfqpoint{1.448436in}{2.834249in}}{\pgfqpoint{1.442612in}{2.828425in}}%
\pgfpathcurveto{\pgfqpoint{1.436788in}{2.822601in}}{\pgfqpoint{1.433516in}{2.814701in}}{\pgfqpoint{1.433516in}{2.806465in}}%
\pgfpathcurveto{\pgfqpoint{1.433516in}{2.798229in}}{\pgfqpoint{1.436788in}{2.790329in}}{\pgfqpoint{1.442612in}{2.784505in}}%
\pgfpathcurveto{\pgfqpoint{1.448436in}{2.778681in}}{\pgfqpoint{1.456336in}{2.775409in}}{\pgfqpoint{1.464572in}{2.775409in}}%
\pgfpathclose%
\pgfusepath{stroke,fill}%
\end{pgfscope}%
\begin{pgfscope}%
\pgfpathrectangle{\pgfqpoint{0.100000in}{0.220728in}}{\pgfqpoint{3.696000in}{3.696000in}}%
\pgfusepath{clip}%
\pgfsetbuttcap%
\pgfsetroundjoin%
\definecolor{currentfill}{rgb}{0.121569,0.466667,0.705882}%
\pgfsetfillcolor{currentfill}%
\pgfsetfillopacity{0.356689}%
\pgfsetlinewidth{1.003750pt}%
\definecolor{currentstroke}{rgb}{0.121569,0.466667,0.705882}%
\pgfsetstrokecolor{currentstroke}%
\pgfsetstrokeopacity{0.356689}%
\pgfsetdash{}{0pt}%
\pgfpathmoveto{\pgfqpoint{1.939397in}{3.096577in}}%
\pgfpathcurveto{\pgfqpoint{1.947633in}{3.096577in}}{\pgfqpoint{1.955533in}{3.099850in}}{\pgfqpoint{1.961357in}{3.105673in}}%
\pgfpathcurveto{\pgfqpoint{1.967181in}{3.111497in}}{\pgfqpoint{1.970453in}{3.119397in}}{\pgfqpoint{1.970453in}{3.127634in}}%
\pgfpathcurveto{\pgfqpoint{1.970453in}{3.135870in}}{\pgfqpoint{1.967181in}{3.143770in}}{\pgfqpoint{1.961357in}{3.149594in}}%
\pgfpathcurveto{\pgfqpoint{1.955533in}{3.155418in}}{\pgfqpoint{1.947633in}{3.158690in}}{\pgfqpoint{1.939397in}{3.158690in}}%
\pgfpathcurveto{\pgfqpoint{1.931160in}{3.158690in}}{\pgfqpoint{1.923260in}{3.155418in}}{\pgfqpoint{1.917436in}{3.149594in}}%
\pgfpathcurveto{\pgfqpoint{1.911612in}{3.143770in}}{\pgfqpoint{1.908340in}{3.135870in}}{\pgfqpoint{1.908340in}{3.127634in}}%
\pgfpathcurveto{\pgfqpoint{1.908340in}{3.119397in}}{\pgfqpoint{1.911612in}{3.111497in}}{\pgfqpoint{1.917436in}{3.105673in}}%
\pgfpathcurveto{\pgfqpoint{1.923260in}{3.099850in}}{\pgfqpoint{1.931160in}{3.096577in}}{\pgfqpoint{1.939397in}{3.096577in}}%
\pgfpathclose%
\pgfusepath{stroke,fill}%
\end{pgfscope}%
\begin{pgfscope}%
\pgfpathrectangle{\pgfqpoint{0.100000in}{0.220728in}}{\pgfqpoint{3.696000in}{3.696000in}}%
\pgfusepath{clip}%
\pgfsetbuttcap%
\pgfsetroundjoin%
\definecolor{currentfill}{rgb}{0.121569,0.466667,0.705882}%
\pgfsetfillcolor{currentfill}%
\pgfsetfillopacity{0.362054}%
\pgfsetlinewidth{1.003750pt}%
\definecolor{currentstroke}{rgb}{0.121569,0.466667,0.705882}%
\pgfsetstrokecolor{currentstroke}%
\pgfsetstrokeopacity{0.362054}%
\pgfsetdash{}{0pt}%
\pgfpathmoveto{\pgfqpoint{1.963250in}{3.094880in}}%
\pgfpathcurveto{\pgfqpoint{1.971486in}{3.094880in}}{\pgfqpoint{1.979386in}{3.098152in}}{\pgfqpoint{1.985210in}{3.103976in}}%
\pgfpathcurveto{\pgfqpoint{1.991034in}{3.109800in}}{\pgfqpoint{1.994306in}{3.117700in}}{\pgfqpoint{1.994306in}{3.125936in}}%
\pgfpathcurveto{\pgfqpoint{1.994306in}{3.134172in}}{\pgfqpoint{1.991034in}{3.142072in}}{\pgfqpoint{1.985210in}{3.147896in}}%
\pgfpathcurveto{\pgfqpoint{1.979386in}{3.153720in}}{\pgfqpoint{1.971486in}{3.156993in}}{\pgfqpoint{1.963250in}{3.156993in}}%
\pgfpathcurveto{\pgfqpoint{1.955014in}{3.156993in}}{\pgfqpoint{1.947113in}{3.153720in}}{\pgfqpoint{1.941290in}{3.147896in}}%
\pgfpathcurveto{\pgfqpoint{1.935466in}{3.142072in}}{\pgfqpoint{1.932193in}{3.134172in}}{\pgfqpoint{1.932193in}{3.125936in}}%
\pgfpathcurveto{\pgfqpoint{1.932193in}{3.117700in}}{\pgfqpoint{1.935466in}{3.109800in}}{\pgfqpoint{1.941290in}{3.103976in}}%
\pgfpathcurveto{\pgfqpoint{1.947113in}{3.098152in}}{\pgfqpoint{1.955014in}{3.094880in}}{\pgfqpoint{1.963250in}{3.094880in}}%
\pgfpathclose%
\pgfusepath{stroke,fill}%
\end{pgfscope}%
\begin{pgfscope}%
\pgfpathrectangle{\pgfqpoint{0.100000in}{0.220728in}}{\pgfqpoint{3.696000in}{3.696000in}}%
\pgfusepath{clip}%
\pgfsetbuttcap%
\pgfsetroundjoin%
\definecolor{currentfill}{rgb}{0.121569,0.466667,0.705882}%
\pgfsetfillcolor{currentfill}%
\pgfsetfillopacity{0.363628}%
\pgfsetlinewidth{1.003750pt}%
\definecolor{currentstroke}{rgb}{0.121569,0.466667,0.705882}%
\pgfsetstrokecolor{currentstroke}%
\pgfsetstrokeopacity{0.363628}%
\pgfsetdash{}{0pt}%
\pgfpathmoveto{\pgfqpoint{1.449718in}{2.727381in}}%
\pgfpathcurveto{\pgfqpoint{1.457955in}{2.727381in}}{\pgfqpoint{1.465855in}{2.730654in}}{\pgfqpoint{1.471679in}{2.736478in}}%
\pgfpathcurveto{\pgfqpoint{1.477502in}{2.742302in}}{\pgfqpoint{1.480775in}{2.750202in}}{\pgfqpoint{1.480775in}{2.758438in}}%
\pgfpathcurveto{\pgfqpoint{1.480775in}{2.766674in}}{\pgfqpoint{1.477502in}{2.774574in}}{\pgfqpoint{1.471679in}{2.780398in}}%
\pgfpathcurveto{\pgfqpoint{1.465855in}{2.786222in}}{\pgfqpoint{1.457955in}{2.789494in}}{\pgfqpoint{1.449718in}{2.789494in}}%
\pgfpathcurveto{\pgfqpoint{1.441482in}{2.789494in}}{\pgfqpoint{1.433582in}{2.786222in}}{\pgfqpoint{1.427758in}{2.780398in}}%
\pgfpathcurveto{\pgfqpoint{1.421934in}{2.774574in}}{\pgfqpoint{1.418662in}{2.766674in}}{\pgfqpoint{1.418662in}{2.758438in}}%
\pgfpathcurveto{\pgfqpoint{1.418662in}{2.750202in}}{\pgfqpoint{1.421934in}{2.742302in}}{\pgfqpoint{1.427758in}{2.736478in}}%
\pgfpathcurveto{\pgfqpoint{1.433582in}{2.730654in}}{\pgfqpoint{1.441482in}{2.727381in}}{\pgfqpoint{1.449718in}{2.727381in}}%
\pgfpathclose%
\pgfusepath{stroke,fill}%
\end{pgfscope}%
\begin{pgfscope}%
\pgfpathrectangle{\pgfqpoint{0.100000in}{0.220728in}}{\pgfqpoint{3.696000in}{3.696000in}}%
\pgfusepath{clip}%
\pgfsetbuttcap%
\pgfsetroundjoin%
\definecolor{currentfill}{rgb}{0.121569,0.466667,0.705882}%
\pgfsetfillcolor{currentfill}%
\pgfsetfillopacity{0.369194}%
\pgfsetlinewidth{1.003750pt}%
\definecolor{currentstroke}{rgb}{0.121569,0.466667,0.705882}%
\pgfsetstrokecolor{currentstroke}%
\pgfsetstrokeopacity{0.369194}%
\pgfsetdash{}{0pt}%
\pgfpathmoveto{\pgfqpoint{1.994890in}{3.090339in}}%
\pgfpathcurveto{\pgfqpoint{2.003127in}{3.090339in}}{\pgfqpoint{2.011027in}{3.093612in}}{\pgfqpoint{2.016851in}{3.099436in}}%
\pgfpathcurveto{\pgfqpoint{2.022675in}{3.105260in}}{\pgfqpoint{2.025947in}{3.113160in}}{\pgfqpoint{2.025947in}{3.121396in}}%
\pgfpathcurveto{\pgfqpoint{2.025947in}{3.129632in}}{\pgfqpoint{2.022675in}{3.137532in}}{\pgfqpoint{2.016851in}{3.143356in}}%
\pgfpathcurveto{\pgfqpoint{2.011027in}{3.149180in}}{\pgfqpoint{2.003127in}{3.152452in}}{\pgfqpoint{1.994890in}{3.152452in}}%
\pgfpathcurveto{\pgfqpoint{1.986654in}{3.152452in}}{\pgfqpoint{1.978754in}{3.149180in}}{\pgfqpoint{1.972930in}{3.143356in}}%
\pgfpathcurveto{\pgfqpoint{1.967106in}{3.137532in}}{\pgfqpoint{1.963834in}{3.129632in}}{\pgfqpoint{1.963834in}{3.121396in}}%
\pgfpathcurveto{\pgfqpoint{1.963834in}{3.113160in}}{\pgfqpoint{1.967106in}{3.105260in}}{\pgfqpoint{1.972930in}{3.099436in}}%
\pgfpathcurveto{\pgfqpoint{1.978754in}{3.093612in}}{\pgfqpoint{1.986654in}{3.090339in}}{\pgfqpoint{1.994890in}{3.090339in}}%
\pgfpathclose%
\pgfusepath{stroke,fill}%
\end{pgfscope}%
\begin{pgfscope}%
\pgfpathrectangle{\pgfqpoint{0.100000in}{0.220728in}}{\pgfqpoint{3.696000in}{3.696000in}}%
\pgfusepath{clip}%
\pgfsetbuttcap%
\pgfsetroundjoin%
\definecolor{currentfill}{rgb}{0.121569,0.466667,0.705882}%
\pgfsetfillcolor{currentfill}%
\pgfsetfillopacity{0.369420}%
\pgfsetlinewidth{1.003750pt}%
\definecolor{currentstroke}{rgb}{0.121569,0.466667,0.705882}%
\pgfsetstrokecolor{currentstroke}%
\pgfsetstrokeopacity{0.369420}%
\pgfsetdash{}{0pt}%
\pgfpathmoveto{\pgfqpoint{1.423733in}{2.689296in}}%
\pgfpathcurveto{\pgfqpoint{1.431969in}{2.689296in}}{\pgfqpoint{1.439870in}{2.692569in}}{\pgfqpoint{1.445693in}{2.698393in}}%
\pgfpathcurveto{\pgfqpoint{1.451517in}{2.704216in}}{\pgfqpoint{1.454790in}{2.712117in}}{\pgfqpoint{1.454790in}{2.720353in}}%
\pgfpathcurveto{\pgfqpoint{1.454790in}{2.728589in}}{\pgfqpoint{1.451517in}{2.736489in}}{\pgfqpoint{1.445693in}{2.742313in}}%
\pgfpathcurveto{\pgfqpoint{1.439870in}{2.748137in}}{\pgfqpoint{1.431969in}{2.751409in}}{\pgfqpoint{1.423733in}{2.751409in}}%
\pgfpathcurveto{\pgfqpoint{1.415497in}{2.751409in}}{\pgfqpoint{1.407597in}{2.748137in}}{\pgfqpoint{1.401773in}{2.742313in}}%
\pgfpathcurveto{\pgfqpoint{1.395949in}{2.736489in}}{\pgfqpoint{1.392677in}{2.728589in}}{\pgfqpoint{1.392677in}{2.720353in}}%
\pgfpathcurveto{\pgfqpoint{1.392677in}{2.712117in}}{\pgfqpoint{1.395949in}{2.704216in}}{\pgfqpoint{1.401773in}{2.698393in}}%
\pgfpathcurveto{\pgfqpoint{1.407597in}{2.692569in}}{\pgfqpoint{1.415497in}{2.689296in}}{\pgfqpoint{1.423733in}{2.689296in}}%
\pgfpathclose%
\pgfusepath{stroke,fill}%
\end{pgfscope}%
\begin{pgfscope}%
\pgfpathrectangle{\pgfqpoint{0.100000in}{0.220728in}}{\pgfqpoint{3.696000in}{3.696000in}}%
\pgfusepath{clip}%
\pgfsetbuttcap%
\pgfsetroundjoin%
\definecolor{currentfill}{rgb}{0.121569,0.466667,0.705882}%
\pgfsetfillcolor{currentfill}%
\pgfsetfillopacity{0.372466}%
\pgfsetlinewidth{1.003750pt}%
\definecolor{currentstroke}{rgb}{0.121569,0.466667,0.705882}%
\pgfsetstrokecolor{currentstroke}%
\pgfsetstrokeopacity{0.372466}%
\pgfsetdash{}{0pt}%
\pgfpathmoveto{\pgfqpoint{2.013517in}{3.088480in}}%
\pgfpathcurveto{\pgfqpoint{2.021753in}{3.088480in}}{\pgfqpoint{2.029653in}{3.091752in}}{\pgfqpoint{2.035477in}{3.097576in}}%
\pgfpathcurveto{\pgfqpoint{2.041301in}{3.103400in}}{\pgfqpoint{2.044573in}{3.111300in}}{\pgfqpoint{2.044573in}{3.119536in}}%
\pgfpathcurveto{\pgfqpoint{2.044573in}{3.127773in}}{\pgfqpoint{2.041301in}{3.135673in}}{\pgfqpoint{2.035477in}{3.141497in}}%
\pgfpathcurveto{\pgfqpoint{2.029653in}{3.147321in}}{\pgfqpoint{2.021753in}{3.150593in}}{\pgfqpoint{2.013517in}{3.150593in}}%
\pgfpathcurveto{\pgfqpoint{2.005281in}{3.150593in}}{\pgfqpoint{1.997381in}{3.147321in}}{\pgfqpoint{1.991557in}{3.141497in}}%
\pgfpathcurveto{\pgfqpoint{1.985733in}{3.135673in}}{\pgfqpoint{1.982460in}{3.127773in}}{\pgfqpoint{1.982460in}{3.119536in}}%
\pgfpathcurveto{\pgfqpoint{1.982460in}{3.111300in}}{\pgfqpoint{1.985733in}{3.103400in}}{\pgfqpoint{1.991557in}{3.097576in}}%
\pgfpathcurveto{\pgfqpoint{1.997381in}{3.091752in}}{\pgfqpoint{2.005281in}{3.088480in}}{\pgfqpoint{2.013517in}{3.088480in}}%
\pgfpathclose%
\pgfusepath{stroke,fill}%
\end{pgfscope}%
\begin{pgfscope}%
\pgfpathrectangle{\pgfqpoint{0.100000in}{0.220728in}}{\pgfqpoint{3.696000in}{3.696000in}}%
\pgfusepath{clip}%
\pgfsetbuttcap%
\pgfsetroundjoin%
\definecolor{currentfill}{rgb}{0.121569,0.466667,0.705882}%
\pgfsetfillcolor{currentfill}%
\pgfsetfillopacity{0.375684}%
\pgfsetlinewidth{1.003750pt}%
\definecolor{currentstroke}{rgb}{0.121569,0.466667,0.705882}%
\pgfsetstrokecolor{currentstroke}%
\pgfsetstrokeopacity{0.375684}%
\pgfsetdash{}{0pt}%
\pgfpathmoveto{\pgfqpoint{1.412509in}{2.651576in}}%
\pgfpathcurveto{\pgfqpoint{1.420746in}{2.651576in}}{\pgfqpoint{1.428646in}{2.654849in}}{\pgfqpoint{1.434470in}{2.660673in}}%
\pgfpathcurveto{\pgfqpoint{1.440294in}{2.666497in}}{\pgfqpoint{1.443566in}{2.674397in}}{\pgfqpoint{1.443566in}{2.682633in}}%
\pgfpathcurveto{\pgfqpoint{1.443566in}{2.690869in}}{\pgfqpoint{1.440294in}{2.698769in}}{\pgfqpoint{1.434470in}{2.704593in}}%
\pgfpathcurveto{\pgfqpoint{1.428646in}{2.710417in}}{\pgfqpoint{1.420746in}{2.713689in}}{\pgfqpoint{1.412509in}{2.713689in}}%
\pgfpathcurveto{\pgfqpoint{1.404273in}{2.713689in}}{\pgfqpoint{1.396373in}{2.710417in}}{\pgfqpoint{1.390549in}{2.704593in}}%
\pgfpathcurveto{\pgfqpoint{1.384725in}{2.698769in}}{\pgfqpoint{1.381453in}{2.690869in}}{\pgfqpoint{1.381453in}{2.682633in}}%
\pgfpathcurveto{\pgfqpoint{1.381453in}{2.674397in}}{\pgfqpoint{1.384725in}{2.666497in}}{\pgfqpoint{1.390549in}{2.660673in}}%
\pgfpathcurveto{\pgfqpoint{1.396373in}{2.654849in}}{\pgfqpoint{1.404273in}{2.651576in}}{\pgfqpoint{1.412509in}{2.651576in}}%
\pgfpathclose%
\pgfusepath{stroke,fill}%
\end{pgfscope}%
\begin{pgfscope}%
\pgfpathrectangle{\pgfqpoint{0.100000in}{0.220728in}}{\pgfqpoint{3.696000in}{3.696000in}}%
\pgfusepath{clip}%
\pgfsetbuttcap%
\pgfsetroundjoin%
\definecolor{currentfill}{rgb}{0.121569,0.466667,0.705882}%
\pgfsetfillcolor{currentfill}%
\pgfsetfillopacity{0.379380}%
\pgfsetlinewidth{1.003750pt}%
\definecolor{currentstroke}{rgb}{0.121569,0.466667,0.705882}%
\pgfsetstrokecolor{currentstroke}%
\pgfsetstrokeopacity{0.379380}%
\pgfsetdash{}{0pt}%
\pgfpathmoveto{\pgfqpoint{2.034812in}{3.083201in}}%
\pgfpathcurveto{\pgfqpoint{2.043049in}{3.083201in}}{\pgfqpoint{2.050949in}{3.086474in}}{\pgfqpoint{2.056773in}{3.092298in}}%
\pgfpathcurveto{\pgfqpoint{2.062596in}{3.098121in}}{\pgfqpoint{2.065869in}{3.106022in}}{\pgfqpoint{2.065869in}{3.114258in}}%
\pgfpathcurveto{\pgfqpoint{2.065869in}{3.122494in}}{\pgfqpoint{2.062596in}{3.130394in}}{\pgfqpoint{2.056773in}{3.136218in}}%
\pgfpathcurveto{\pgfqpoint{2.050949in}{3.142042in}}{\pgfqpoint{2.043049in}{3.145314in}}{\pgfqpoint{2.034812in}{3.145314in}}%
\pgfpathcurveto{\pgfqpoint{2.026576in}{3.145314in}}{\pgfqpoint{2.018676in}{3.142042in}}{\pgfqpoint{2.012852in}{3.136218in}}%
\pgfpathcurveto{\pgfqpoint{2.007028in}{3.130394in}}{\pgfqpoint{2.003756in}{3.122494in}}{\pgfqpoint{2.003756in}{3.114258in}}%
\pgfpathcurveto{\pgfqpoint{2.003756in}{3.106022in}}{\pgfqpoint{2.007028in}{3.098121in}}{\pgfqpoint{2.012852in}{3.092298in}}%
\pgfpathcurveto{\pgfqpoint{2.018676in}{3.086474in}}{\pgfqpoint{2.026576in}{3.083201in}}{\pgfqpoint{2.034812in}{3.083201in}}%
\pgfpathclose%
\pgfusepath{stroke,fill}%
\end{pgfscope}%
\begin{pgfscope}%
\pgfpathrectangle{\pgfqpoint{0.100000in}{0.220728in}}{\pgfqpoint{3.696000in}{3.696000in}}%
\pgfusepath{clip}%
\pgfsetbuttcap%
\pgfsetroundjoin%
\definecolor{currentfill}{rgb}{0.121569,0.466667,0.705882}%
\pgfsetfillcolor{currentfill}%
\pgfsetfillopacity{0.379498}%
\pgfsetlinewidth{1.003750pt}%
\definecolor{currentstroke}{rgb}{0.121569,0.466667,0.705882}%
\pgfsetstrokecolor{currentstroke}%
\pgfsetstrokeopacity{0.379498}%
\pgfsetdash{}{0pt}%
\pgfpathmoveto{\pgfqpoint{1.398055in}{2.627608in}}%
\pgfpathcurveto{\pgfqpoint{1.406291in}{2.627608in}}{\pgfqpoint{1.414191in}{2.630880in}}{\pgfqpoint{1.420015in}{2.636704in}}%
\pgfpathcurveto{\pgfqpoint{1.425839in}{2.642528in}}{\pgfqpoint{1.429111in}{2.650428in}}{\pgfqpoint{1.429111in}{2.658665in}}%
\pgfpathcurveto{\pgfqpoint{1.429111in}{2.666901in}}{\pgfqpoint{1.425839in}{2.674801in}}{\pgfqpoint{1.420015in}{2.680625in}}%
\pgfpathcurveto{\pgfqpoint{1.414191in}{2.686449in}}{\pgfqpoint{1.406291in}{2.689721in}}{\pgfqpoint{1.398055in}{2.689721in}}%
\pgfpathcurveto{\pgfqpoint{1.389819in}{2.689721in}}{\pgfqpoint{1.381919in}{2.686449in}}{\pgfqpoint{1.376095in}{2.680625in}}%
\pgfpathcurveto{\pgfqpoint{1.370271in}{2.674801in}}{\pgfqpoint{1.366998in}{2.666901in}}{\pgfqpoint{1.366998in}{2.658665in}}%
\pgfpathcurveto{\pgfqpoint{1.366998in}{2.650428in}}{\pgfqpoint{1.370271in}{2.642528in}}{\pgfqpoint{1.376095in}{2.636704in}}%
\pgfpathcurveto{\pgfqpoint{1.381919in}{2.630880in}}{\pgfqpoint{1.389819in}{2.627608in}}{\pgfqpoint{1.398055in}{2.627608in}}%
\pgfpathclose%
\pgfusepath{stroke,fill}%
\end{pgfscope}%
\begin{pgfscope}%
\pgfpathrectangle{\pgfqpoint{0.100000in}{0.220728in}}{\pgfqpoint{3.696000in}{3.696000in}}%
\pgfusepath{clip}%
\pgfsetbuttcap%
\pgfsetroundjoin%
\definecolor{currentfill}{rgb}{0.121569,0.466667,0.705882}%
\pgfsetfillcolor{currentfill}%
\pgfsetfillopacity{0.381660}%
\pgfsetlinewidth{1.003750pt}%
\definecolor{currentstroke}{rgb}{0.121569,0.466667,0.705882}%
\pgfsetstrokecolor{currentstroke}%
\pgfsetstrokeopacity{0.381660}%
\pgfsetdash{}{0pt}%
\pgfpathmoveto{\pgfqpoint{2.049376in}{3.080434in}}%
\pgfpathcurveto{\pgfqpoint{2.057612in}{3.080434in}}{\pgfqpoint{2.065512in}{3.083707in}}{\pgfqpoint{2.071336in}{3.089531in}}%
\pgfpathcurveto{\pgfqpoint{2.077160in}{3.095354in}}{\pgfqpoint{2.080432in}{3.103254in}}{\pgfqpoint{2.080432in}{3.111491in}}%
\pgfpathcurveto{\pgfqpoint{2.080432in}{3.119727in}}{\pgfqpoint{2.077160in}{3.127627in}}{\pgfqpoint{2.071336in}{3.133451in}}%
\pgfpathcurveto{\pgfqpoint{2.065512in}{3.139275in}}{\pgfqpoint{2.057612in}{3.142547in}}{\pgfqpoint{2.049376in}{3.142547in}}%
\pgfpathcurveto{\pgfqpoint{2.041140in}{3.142547in}}{\pgfqpoint{2.033240in}{3.139275in}}{\pgfqpoint{2.027416in}{3.133451in}}%
\pgfpathcurveto{\pgfqpoint{2.021592in}{3.127627in}}{\pgfqpoint{2.018319in}{3.119727in}}{\pgfqpoint{2.018319in}{3.111491in}}%
\pgfpathcurveto{\pgfqpoint{2.018319in}{3.103254in}}{\pgfqpoint{2.021592in}{3.095354in}}{\pgfqpoint{2.027416in}{3.089531in}}%
\pgfpathcurveto{\pgfqpoint{2.033240in}{3.083707in}}{\pgfqpoint{2.041140in}{3.080434in}}{\pgfqpoint{2.049376in}{3.080434in}}%
\pgfpathclose%
\pgfusepath{stroke,fill}%
\end{pgfscope}%
\begin{pgfscope}%
\pgfpathrectangle{\pgfqpoint{0.100000in}{0.220728in}}{\pgfqpoint{3.696000in}{3.696000in}}%
\pgfusepath{clip}%
\pgfsetbuttcap%
\pgfsetroundjoin%
\definecolor{currentfill}{rgb}{0.121569,0.466667,0.705882}%
\pgfsetfillcolor{currentfill}%
\pgfsetfillopacity{0.382900}%
\pgfsetlinewidth{1.003750pt}%
\definecolor{currentstroke}{rgb}{0.121569,0.466667,0.705882}%
\pgfsetstrokecolor{currentstroke}%
\pgfsetstrokeopacity{0.382900}%
\pgfsetdash{}{0pt}%
\pgfpathmoveto{\pgfqpoint{1.391517in}{2.607960in}}%
\pgfpathcurveto{\pgfqpoint{1.399753in}{2.607960in}}{\pgfqpoint{1.407654in}{2.611233in}}{\pgfqpoint{1.413477in}{2.617057in}}%
\pgfpathcurveto{\pgfqpoint{1.419301in}{2.622881in}}{\pgfqpoint{1.422574in}{2.630781in}}{\pgfqpoint{1.422574in}{2.639017in}}%
\pgfpathcurveto{\pgfqpoint{1.422574in}{2.647253in}}{\pgfqpoint{1.419301in}{2.655153in}}{\pgfqpoint{1.413477in}{2.660977in}}%
\pgfpathcurveto{\pgfqpoint{1.407654in}{2.666801in}}{\pgfqpoint{1.399753in}{2.670073in}}{\pgfqpoint{1.391517in}{2.670073in}}%
\pgfpathcurveto{\pgfqpoint{1.383281in}{2.670073in}}{\pgfqpoint{1.375381in}{2.666801in}}{\pgfqpoint{1.369557in}{2.660977in}}%
\pgfpathcurveto{\pgfqpoint{1.363733in}{2.655153in}}{\pgfqpoint{1.360461in}{2.647253in}}{\pgfqpoint{1.360461in}{2.639017in}}%
\pgfpathcurveto{\pgfqpoint{1.360461in}{2.630781in}}{\pgfqpoint{1.363733in}{2.622881in}}{\pgfqpoint{1.369557in}{2.617057in}}%
\pgfpathcurveto{\pgfqpoint{1.375381in}{2.611233in}}{\pgfqpoint{1.383281in}{2.607960in}}{\pgfqpoint{1.391517in}{2.607960in}}%
\pgfpathclose%
\pgfusepath{stroke,fill}%
\end{pgfscope}%
\begin{pgfscope}%
\pgfpathrectangle{\pgfqpoint{0.100000in}{0.220728in}}{\pgfqpoint{3.696000in}{3.696000in}}%
\pgfusepath{clip}%
\pgfsetbuttcap%
\pgfsetroundjoin%
\definecolor{currentfill}{rgb}{0.121569,0.466667,0.705882}%
\pgfsetfillcolor{currentfill}%
\pgfsetfillopacity{0.385099}%
\pgfsetlinewidth{1.003750pt}%
\definecolor{currentstroke}{rgb}{0.121569,0.466667,0.705882}%
\pgfsetstrokecolor{currentstroke}%
\pgfsetstrokeopacity{0.385099}%
\pgfsetdash{}{0pt}%
\pgfpathmoveto{\pgfqpoint{1.382540in}{2.591281in}}%
\pgfpathcurveto{\pgfqpoint{1.390776in}{2.591281in}}{\pgfqpoint{1.398676in}{2.594553in}}{\pgfqpoint{1.404500in}{2.600377in}}%
\pgfpathcurveto{\pgfqpoint{1.410324in}{2.606201in}}{\pgfqpoint{1.413596in}{2.614101in}}{\pgfqpoint{1.413596in}{2.622337in}}%
\pgfpathcurveto{\pgfqpoint{1.413596in}{2.630574in}}{\pgfqpoint{1.410324in}{2.638474in}}{\pgfqpoint{1.404500in}{2.644298in}}%
\pgfpathcurveto{\pgfqpoint{1.398676in}{2.650122in}}{\pgfqpoint{1.390776in}{2.653394in}}{\pgfqpoint{1.382540in}{2.653394in}}%
\pgfpathcurveto{\pgfqpoint{1.374304in}{2.653394in}}{\pgfqpoint{1.366404in}{2.650122in}}{\pgfqpoint{1.360580in}{2.644298in}}%
\pgfpathcurveto{\pgfqpoint{1.354756in}{2.638474in}}{\pgfqpoint{1.351483in}{2.630574in}}{\pgfqpoint{1.351483in}{2.622337in}}%
\pgfpathcurveto{\pgfqpoint{1.351483in}{2.614101in}}{\pgfqpoint{1.354756in}{2.606201in}}{\pgfqpoint{1.360580in}{2.600377in}}%
\pgfpathcurveto{\pgfqpoint{1.366404in}{2.594553in}}{\pgfqpoint{1.374304in}{2.591281in}}{\pgfqpoint{1.382540in}{2.591281in}}%
\pgfpathclose%
\pgfusepath{stroke,fill}%
\end{pgfscope}%
\begin{pgfscope}%
\pgfpathrectangle{\pgfqpoint{0.100000in}{0.220728in}}{\pgfqpoint{3.696000in}{3.696000in}}%
\pgfusepath{clip}%
\pgfsetbuttcap%
\pgfsetroundjoin%
\definecolor{currentfill}{rgb}{0.121569,0.466667,0.705882}%
\pgfsetfillcolor{currentfill}%
\pgfsetfillopacity{0.385924}%
\pgfsetlinewidth{1.003750pt}%
\definecolor{currentstroke}{rgb}{0.121569,0.466667,0.705882}%
\pgfsetstrokecolor{currentstroke}%
\pgfsetstrokeopacity{0.385924}%
\pgfsetdash{}{0pt}%
\pgfpathmoveto{\pgfqpoint{2.066324in}{3.077417in}}%
\pgfpathcurveto{\pgfqpoint{2.074560in}{3.077417in}}{\pgfqpoint{2.082460in}{3.080690in}}{\pgfqpoint{2.088284in}{3.086514in}}%
\pgfpathcurveto{\pgfqpoint{2.094108in}{3.092337in}}{\pgfqpoint{2.097381in}{3.100238in}}{\pgfqpoint{2.097381in}{3.108474in}}%
\pgfpathcurveto{\pgfqpoint{2.097381in}{3.116710in}}{\pgfqpoint{2.094108in}{3.124610in}}{\pgfqpoint{2.088284in}{3.130434in}}%
\pgfpathcurveto{\pgfqpoint{2.082460in}{3.136258in}}{\pgfqpoint{2.074560in}{3.139530in}}{\pgfqpoint{2.066324in}{3.139530in}}%
\pgfpathcurveto{\pgfqpoint{2.058088in}{3.139530in}}{\pgfqpoint{2.050188in}{3.136258in}}{\pgfqpoint{2.044364in}{3.130434in}}%
\pgfpathcurveto{\pgfqpoint{2.038540in}{3.124610in}}{\pgfqpoint{2.035268in}{3.116710in}}{\pgfqpoint{2.035268in}{3.108474in}}%
\pgfpathcurveto{\pgfqpoint{2.035268in}{3.100238in}}{\pgfqpoint{2.038540in}{3.092337in}}{\pgfqpoint{2.044364in}{3.086514in}}%
\pgfpathcurveto{\pgfqpoint{2.050188in}{3.080690in}}{\pgfqpoint{2.058088in}{3.077417in}}{\pgfqpoint{2.066324in}{3.077417in}}%
\pgfpathclose%
\pgfusepath{stroke,fill}%
\end{pgfscope}%
\begin{pgfscope}%
\pgfpathrectangle{\pgfqpoint{0.100000in}{0.220728in}}{\pgfqpoint{3.696000in}{3.696000in}}%
\pgfusepath{clip}%
\pgfsetbuttcap%
\pgfsetroundjoin%
\definecolor{currentfill}{rgb}{0.121569,0.466667,0.705882}%
\pgfsetfillcolor{currentfill}%
\pgfsetfillopacity{0.386383}%
\pgfsetlinewidth{1.003750pt}%
\definecolor{currentstroke}{rgb}{0.121569,0.466667,0.705882}%
\pgfsetstrokecolor{currentstroke}%
\pgfsetstrokeopacity{0.386383}%
\pgfsetdash{}{0pt}%
\pgfpathmoveto{\pgfqpoint{1.379686in}{2.583344in}}%
\pgfpathcurveto{\pgfqpoint{1.387922in}{2.583344in}}{\pgfqpoint{1.395823in}{2.586616in}}{\pgfqpoint{1.401646in}{2.592440in}}%
\pgfpathcurveto{\pgfqpoint{1.407470in}{2.598264in}}{\pgfqpoint{1.410743in}{2.606164in}}{\pgfqpoint{1.410743in}{2.614400in}}%
\pgfpathcurveto{\pgfqpoint{1.410743in}{2.622636in}}{\pgfqpoint{1.407470in}{2.630537in}}{\pgfqpoint{1.401646in}{2.636360in}}%
\pgfpathcurveto{\pgfqpoint{1.395823in}{2.642184in}}{\pgfqpoint{1.387922in}{2.645457in}}{\pgfqpoint{1.379686in}{2.645457in}}%
\pgfpathcurveto{\pgfqpoint{1.371450in}{2.645457in}}{\pgfqpoint{1.363550in}{2.642184in}}{\pgfqpoint{1.357726in}{2.636360in}}%
\pgfpathcurveto{\pgfqpoint{1.351902in}{2.630537in}}{\pgfqpoint{1.348630in}{2.622636in}}{\pgfqpoint{1.348630in}{2.614400in}}%
\pgfpathcurveto{\pgfqpoint{1.348630in}{2.606164in}}{\pgfqpoint{1.351902in}{2.598264in}}{\pgfqpoint{1.357726in}{2.592440in}}%
\pgfpathcurveto{\pgfqpoint{1.363550in}{2.586616in}}{\pgfqpoint{1.371450in}{2.583344in}}{\pgfqpoint{1.379686in}{2.583344in}}%
\pgfpathclose%
\pgfusepath{stroke,fill}%
\end{pgfscope}%
\begin{pgfscope}%
\pgfpathrectangle{\pgfqpoint{0.100000in}{0.220728in}}{\pgfqpoint{3.696000in}{3.696000in}}%
\pgfusepath{clip}%
\pgfsetbuttcap%
\pgfsetroundjoin%
\definecolor{currentfill}{rgb}{0.121569,0.466667,0.705882}%
\pgfsetfillcolor{currentfill}%
\pgfsetfillopacity{0.386974}%
\pgfsetlinewidth{1.003750pt}%
\definecolor{currentstroke}{rgb}{0.121569,0.466667,0.705882}%
\pgfsetstrokecolor{currentstroke}%
\pgfsetstrokeopacity{0.386974}%
\pgfsetdash{}{0pt}%
\pgfpathmoveto{\pgfqpoint{1.378249in}{2.579887in}}%
\pgfpathcurveto{\pgfqpoint{1.386485in}{2.579887in}}{\pgfqpoint{1.394385in}{2.583160in}}{\pgfqpoint{1.400209in}{2.588984in}}%
\pgfpathcurveto{\pgfqpoint{1.406033in}{2.594808in}}{\pgfqpoint{1.409306in}{2.602708in}}{\pgfqpoint{1.409306in}{2.610944in}}%
\pgfpathcurveto{\pgfqpoint{1.409306in}{2.619180in}}{\pgfqpoint{1.406033in}{2.627080in}}{\pgfqpoint{1.400209in}{2.632904in}}%
\pgfpathcurveto{\pgfqpoint{1.394385in}{2.638728in}}{\pgfqpoint{1.386485in}{2.642000in}}{\pgfqpoint{1.378249in}{2.642000in}}%
\pgfpathcurveto{\pgfqpoint{1.370013in}{2.642000in}}{\pgfqpoint{1.362113in}{2.638728in}}{\pgfqpoint{1.356289in}{2.632904in}}%
\pgfpathcurveto{\pgfqpoint{1.350465in}{2.627080in}}{\pgfqpoint{1.347193in}{2.619180in}}{\pgfqpoint{1.347193in}{2.610944in}}%
\pgfpathcurveto{\pgfqpoint{1.347193in}{2.602708in}}{\pgfqpoint{1.350465in}{2.594808in}}{\pgfqpoint{1.356289in}{2.588984in}}%
\pgfpathcurveto{\pgfqpoint{1.362113in}{2.583160in}}{\pgfqpoint{1.370013in}{2.579887in}}{\pgfqpoint{1.378249in}{2.579887in}}%
\pgfpathclose%
\pgfusepath{stroke,fill}%
\end{pgfscope}%
\begin{pgfscope}%
\pgfpathrectangle{\pgfqpoint{0.100000in}{0.220728in}}{\pgfqpoint{3.696000in}{3.696000in}}%
\pgfusepath{clip}%
\pgfsetbuttcap%
\pgfsetroundjoin%
\definecolor{currentfill}{rgb}{0.121569,0.466667,0.705882}%
\pgfsetfillcolor{currentfill}%
\pgfsetfillopacity{0.388059}%
\pgfsetlinewidth{1.003750pt}%
\definecolor{currentstroke}{rgb}{0.121569,0.466667,0.705882}%
\pgfsetstrokecolor{currentstroke}%
\pgfsetstrokeopacity{0.388059}%
\pgfsetdash{}{0pt}%
\pgfpathmoveto{\pgfqpoint{1.375191in}{2.574112in}}%
\pgfpathcurveto{\pgfqpoint{1.383427in}{2.574112in}}{\pgfqpoint{1.391327in}{2.577384in}}{\pgfqpoint{1.397151in}{2.583208in}}%
\pgfpathcurveto{\pgfqpoint{1.402975in}{2.589032in}}{\pgfqpoint{1.406247in}{2.596932in}}{\pgfqpoint{1.406247in}{2.605168in}}%
\pgfpathcurveto{\pgfqpoint{1.406247in}{2.613404in}}{\pgfqpoint{1.402975in}{2.621304in}}{\pgfqpoint{1.397151in}{2.627128in}}%
\pgfpathcurveto{\pgfqpoint{1.391327in}{2.632952in}}{\pgfqpoint{1.383427in}{2.636225in}}{\pgfqpoint{1.375191in}{2.636225in}}%
\pgfpathcurveto{\pgfqpoint{1.366955in}{2.636225in}}{\pgfqpoint{1.359055in}{2.632952in}}{\pgfqpoint{1.353231in}{2.627128in}}%
\pgfpathcurveto{\pgfqpoint{1.347407in}{2.621304in}}{\pgfqpoint{1.344134in}{2.613404in}}{\pgfqpoint{1.344134in}{2.605168in}}%
\pgfpathcurveto{\pgfqpoint{1.344134in}{2.596932in}}{\pgfqpoint{1.347407in}{2.589032in}}{\pgfqpoint{1.353231in}{2.583208in}}%
\pgfpathcurveto{\pgfqpoint{1.359055in}{2.577384in}}{\pgfqpoint{1.366955in}{2.574112in}}{\pgfqpoint{1.375191in}{2.574112in}}%
\pgfpathclose%
\pgfusepath{stroke,fill}%
\end{pgfscope}%
\begin{pgfscope}%
\pgfpathrectangle{\pgfqpoint{0.100000in}{0.220728in}}{\pgfqpoint{3.696000in}{3.696000in}}%
\pgfusepath{clip}%
\pgfsetbuttcap%
\pgfsetroundjoin%
\definecolor{currentfill}{rgb}{0.121569,0.466667,0.705882}%
\pgfsetfillcolor{currentfill}%
\pgfsetfillopacity{0.389480}%
\pgfsetlinewidth{1.003750pt}%
\definecolor{currentstroke}{rgb}{0.121569,0.466667,0.705882}%
\pgfsetstrokecolor{currentstroke}%
\pgfsetstrokeopacity{0.389480}%
\pgfsetdash{}{0pt}%
\pgfpathmoveto{\pgfqpoint{2.091781in}{3.074478in}}%
\pgfpathcurveto{\pgfqpoint{2.100018in}{3.074478in}}{\pgfqpoint{2.107918in}{3.077750in}}{\pgfqpoint{2.113742in}{3.083574in}}%
\pgfpathcurveto{\pgfqpoint{2.119566in}{3.089398in}}{\pgfqpoint{2.122838in}{3.097298in}}{\pgfqpoint{2.122838in}{3.105534in}}%
\pgfpathcurveto{\pgfqpoint{2.122838in}{3.113771in}}{\pgfqpoint{2.119566in}{3.121671in}}{\pgfqpoint{2.113742in}{3.127495in}}%
\pgfpathcurveto{\pgfqpoint{2.107918in}{3.133319in}}{\pgfqpoint{2.100018in}{3.136591in}}{\pgfqpoint{2.091781in}{3.136591in}}%
\pgfpathcurveto{\pgfqpoint{2.083545in}{3.136591in}}{\pgfqpoint{2.075645in}{3.133319in}}{\pgfqpoint{2.069821in}{3.127495in}}%
\pgfpathcurveto{\pgfqpoint{2.063997in}{3.121671in}}{\pgfqpoint{2.060725in}{3.113771in}}{\pgfqpoint{2.060725in}{3.105534in}}%
\pgfpathcurveto{\pgfqpoint{2.060725in}{3.097298in}}{\pgfqpoint{2.063997in}{3.089398in}}{\pgfqpoint{2.069821in}{3.083574in}}%
\pgfpathcurveto{\pgfqpoint{2.075645in}{3.077750in}}{\pgfqpoint{2.083545in}{3.074478in}}{\pgfqpoint{2.091781in}{3.074478in}}%
\pgfpathclose%
\pgfusepath{stroke,fill}%
\end{pgfscope}%
\begin{pgfscope}%
\pgfpathrectangle{\pgfqpoint{0.100000in}{0.220728in}}{\pgfqpoint{3.696000in}{3.696000in}}%
\pgfusepath{clip}%
\pgfsetbuttcap%
\pgfsetroundjoin%
\definecolor{currentfill}{rgb}{0.121569,0.466667,0.705882}%
\pgfsetfillcolor{currentfill}%
\pgfsetfillopacity{0.390076}%
\pgfsetlinewidth{1.003750pt}%
\definecolor{currentstroke}{rgb}{0.121569,0.466667,0.705882}%
\pgfsetstrokecolor{currentstroke}%
\pgfsetstrokeopacity{0.390076}%
\pgfsetdash{}{0pt}%
\pgfpathmoveto{\pgfqpoint{1.370786in}{2.562735in}}%
\pgfpathcurveto{\pgfqpoint{1.379022in}{2.562735in}}{\pgfqpoint{1.386923in}{2.566007in}}{\pgfqpoint{1.392746in}{2.571831in}}%
\pgfpathcurveto{\pgfqpoint{1.398570in}{2.577655in}}{\pgfqpoint{1.401843in}{2.585555in}}{\pgfqpoint{1.401843in}{2.593791in}}%
\pgfpathcurveto{\pgfqpoint{1.401843in}{2.602028in}}{\pgfqpoint{1.398570in}{2.609928in}}{\pgfqpoint{1.392746in}{2.615752in}}%
\pgfpathcurveto{\pgfqpoint{1.386923in}{2.621576in}}{\pgfqpoint{1.379022in}{2.624848in}}{\pgfqpoint{1.370786in}{2.624848in}}%
\pgfpathcurveto{\pgfqpoint{1.362550in}{2.624848in}}{\pgfqpoint{1.354650in}{2.621576in}}{\pgfqpoint{1.348826in}{2.615752in}}%
\pgfpathcurveto{\pgfqpoint{1.343002in}{2.609928in}}{\pgfqpoint{1.339730in}{2.602028in}}{\pgfqpoint{1.339730in}{2.593791in}}%
\pgfpathcurveto{\pgfqpoint{1.339730in}{2.585555in}}{\pgfqpoint{1.343002in}{2.577655in}}{\pgfqpoint{1.348826in}{2.571831in}}%
\pgfpathcurveto{\pgfqpoint{1.354650in}{2.566007in}}{\pgfqpoint{1.362550in}{2.562735in}}{\pgfqpoint{1.370786in}{2.562735in}}%
\pgfpathclose%
\pgfusepath{stroke,fill}%
\end{pgfscope}%
\begin{pgfscope}%
\pgfpathrectangle{\pgfqpoint{0.100000in}{0.220728in}}{\pgfqpoint{3.696000in}{3.696000in}}%
\pgfusepath{clip}%
\pgfsetbuttcap%
\pgfsetroundjoin%
\definecolor{currentfill}{rgb}{0.121569,0.466667,0.705882}%
\pgfsetfillcolor{currentfill}%
\pgfsetfillopacity{0.393155}%
\pgfsetlinewidth{1.003750pt}%
\definecolor{currentstroke}{rgb}{0.121569,0.466667,0.705882}%
\pgfsetstrokecolor{currentstroke}%
\pgfsetstrokeopacity{0.393155}%
\pgfsetdash{}{0pt}%
\pgfpathmoveto{\pgfqpoint{1.358081in}{2.544128in}}%
\pgfpathcurveto{\pgfqpoint{1.366317in}{2.544128in}}{\pgfqpoint{1.374217in}{2.547400in}}{\pgfqpoint{1.380041in}{2.553224in}}%
\pgfpathcurveto{\pgfqpoint{1.385865in}{2.559048in}}{\pgfqpoint{1.389137in}{2.566948in}}{\pgfqpoint{1.389137in}{2.575184in}}%
\pgfpathcurveto{\pgfqpoint{1.389137in}{2.583421in}}{\pgfqpoint{1.385865in}{2.591321in}}{\pgfqpoint{1.380041in}{2.597145in}}%
\pgfpathcurveto{\pgfqpoint{1.374217in}{2.602969in}}{\pgfqpoint{1.366317in}{2.606241in}}{\pgfqpoint{1.358081in}{2.606241in}}%
\pgfpathcurveto{\pgfqpoint{1.349845in}{2.606241in}}{\pgfqpoint{1.341945in}{2.602969in}}{\pgfqpoint{1.336121in}{2.597145in}}%
\pgfpathcurveto{\pgfqpoint{1.330297in}{2.591321in}}{\pgfqpoint{1.327024in}{2.583421in}}{\pgfqpoint{1.327024in}{2.575184in}}%
\pgfpathcurveto{\pgfqpoint{1.327024in}{2.566948in}}{\pgfqpoint{1.330297in}{2.559048in}}{\pgfqpoint{1.336121in}{2.553224in}}%
\pgfpathcurveto{\pgfqpoint{1.341945in}{2.547400in}}{\pgfqpoint{1.349845in}{2.544128in}}{\pgfqpoint{1.358081in}{2.544128in}}%
\pgfpathclose%
\pgfusepath{stroke,fill}%
\end{pgfscope}%
\begin{pgfscope}%
\pgfpathrectangle{\pgfqpoint{0.100000in}{0.220728in}}{\pgfqpoint{3.696000in}{3.696000in}}%
\pgfusepath{clip}%
\pgfsetbuttcap%
\pgfsetroundjoin%
\definecolor{currentfill}{rgb}{0.121569,0.466667,0.705882}%
\pgfsetfillcolor{currentfill}%
\pgfsetfillopacity{0.396736}%
\pgfsetlinewidth{1.003750pt}%
\definecolor{currentstroke}{rgb}{0.121569,0.466667,0.705882}%
\pgfsetstrokecolor{currentstroke}%
\pgfsetstrokeopacity{0.396736}%
\pgfsetdash{}{0pt}%
\pgfpathmoveto{\pgfqpoint{2.119596in}{3.068322in}}%
\pgfpathcurveto{\pgfqpoint{2.127832in}{3.068322in}}{\pgfqpoint{2.135732in}{3.071594in}}{\pgfqpoint{2.141556in}{3.077418in}}%
\pgfpathcurveto{\pgfqpoint{2.147380in}{3.083242in}}{\pgfqpoint{2.150652in}{3.091142in}}{\pgfqpoint{2.150652in}{3.099378in}}%
\pgfpathcurveto{\pgfqpoint{2.150652in}{3.107615in}}{\pgfqpoint{2.147380in}{3.115515in}}{\pgfqpoint{2.141556in}{3.121339in}}%
\pgfpathcurveto{\pgfqpoint{2.135732in}{3.127163in}}{\pgfqpoint{2.127832in}{3.130435in}}{\pgfqpoint{2.119596in}{3.130435in}}%
\pgfpathcurveto{\pgfqpoint{2.111360in}{3.130435in}}{\pgfqpoint{2.103459in}{3.127163in}}{\pgfqpoint{2.097636in}{3.121339in}}%
\pgfpathcurveto{\pgfqpoint{2.091812in}{3.115515in}}{\pgfqpoint{2.088539in}{3.107615in}}{\pgfqpoint{2.088539in}{3.099378in}}%
\pgfpathcurveto{\pgfqpoint{2.088539in}{3.091142in}}{\pgfqpoint{2.091812in}{3.083242in}}{\pgfqpoint{2.097636in}{3.077418in}}%
\pgfpathcurveto{\pgfqpoint{2.103459in}{3.071594in}}{\pgfqpoint{2.111360in}{3.068322in}}{\pgfqpoint{2.119596in}{3.068322in}}%
\pgfpathclose%
\pgfusepath{stroke,fill}%
\end{pgfscope}%
\begin{pgfscope}%
\pgfpathrectangle{\pgfqpoint{0.100000in}{0.220728in}}{\pgfqpoint{3.696000in}{3.696000in}}%
\pgfusepath{clip}%
\pgfsetbuttcap%
\pgfsetroundjoin%
\definecolor{currentfill}{rgb}{0.121569,0.466667,0.705882}%
\pgfsetfillcolor{currentfill}%
\pgfsetfillopacity{0.399599}%
\pgfsetlinewidth{1.003750pt}%
\definecolor{currentstroke}{rgb}{0.121569,0.466667,0.705882}%
\pgfsetstrokecolor{currentstroke}%
\pgfsetstrokeopacity{0.399599}%
\pgfsetdash{}{0pt}%
\pgfpathmoveto{\pgfqpoint{1.341270in}{2.507049in}}%
\pgfpathcurveto{\pgfqpoint{1.349506in}{2.507049in}}{\pgfqpoint{1.357406in}{2.510321in}}{\pgfqpoint{1.363230in}{2.516145in}}%
\pgfpathcurveto{\pgfqpoint{1.369054in}{2.521969in}}{\pgfqpoint{1.372326in}{2.529869in}}{\pgfqpoint{1.372326in}{2.538106in}}%
\pgfpathcurveto{\pgfqpoint{1.372326in}{2.546342in}}{\pgfqpoint{1.369054in}{2.554242in}}{\pgfqpoint{1.363230in}{2.560066in}}%
\pgfpathcurveto{\pgfqpoint{1.357406in}{2.565890in}}{\pgfqpoint{1.349506in}{2.569162in}}{\pgfqpoint{1.341270in}{2.569162in}}%
\pgfpathcurveto{\pgfqpoint{1.333033in}{2.569162in}}{\pgfqpoint{1.325133in}{2.565890in}}{\pgfqpoint{1.319309in}{2.560066in}}%
\pgfpathcurveto{\pgfqpoint{1.313485in}{2.554242in}}{\pgfqpoint{1.310213in}{2.546342in}}{\pgfqpoint{1.310213in}{2.538106in}}%
\pgfpathcurveto{\pgfqpoint{1.310213in}{2.529869in}}{\pgfqpoint{1.313485in}{2.521969in}}{\pgfqpoint{1.319309in}{2.516145in}}%
\pgfpathcurveto{\pgfqpoint{1.325133in}{2.510321in}}{\pgfqpoint{1.333033in}{2.507049in}}{\pgfqpoint{1.341270in}{2.507049in}}%
\pgfpathclose%
\pgfusepath{stroke,fill}%
\end{pgfscope}%
\begin{pgfscope}%
\pgfpathrectangle{\pgfqpoint{0.100000in}{0.220728in}}{\pgfqpoint{3.696000in}{3.696000in}}%
\pgfusepath{clip}%
\pgfsetbuttcap%
\pgfsetroundjoin%
\definecolor{currentfill}{rgb}{0.121569,0.466667,0.705882}%
\pgfsetfillcolor{currentfill}%
\pgfsetfillopacity{0.406314}%
\pgfsetlinewidth{1.003750pt}%
\definecolor{currentstroke}{rgb}{0.121569,0.466667,0.705882}%
\pgfsetstrokecolor{currentstroke}%
\pgfsetstrokeopacity{0.406314}%
\pgfsetdash{}{0pt}%
\pgfpathmoveto{\pgfqpoint{2.150808in}{3.067069in}}%
\pgfpathcurveto{\pgfqpoint{2.159044in}{3.067069in}}{\pgfqpoint{2.166944in}{3.070341in}}{\pgfqpoint{2.172768in}{3.076165in}}%
\pgfpathcurveto{\pgfqpoint{2.178592in}{3.081989in}}{\pgfqpoint{2.181864in}{3.089889in}}{\pgfqpoint{2.181864in}{3.098126in}}%
\pgfpathcurveto{\pgfqpoint{2.181864in}{3.106362in}}{\pgfqpoint{2.178592in}{3.114262in}}{\pgfqpoint{2.172768in}{3.120086in}}%
\pgfpathcurveto{\pgfqpoint{2.166944in}{3.125910in}}{\pgfqpoint{2.159044in}{3.129182in}}{\pgfqpoint{2.150808in}{3.129182in}}%
\pgfpathcurveto{\pgfqpoint{2.142571in}{3.129182in}}{\pgfqpoint{2.134671in}{3.125910in}}{\pgfqpoint{2.128847in}{3.120086in}}%
\pgfpathcurveto{\pgfqpoint{2.123023in}{3.114262in}}{\pgfqpoint{2.119751in}{3.106362in}}{\pgfqpoint{2.119751in}{3.098126in}}%
\pgfpathcurveto{\pgfqpoint{2.119751in}{3.089889in}}{\pgfqpoint{2.123023in}{3.081989in}}{\pgfqpoint{2.128847in}{3.076165in}}%
\pgfpathcurveto{\pgfqpoint{2.134671in}{3.070341in}}{\pgfqpoint{2.142571in}{3.067069in}}{\pgfqpoint{2.150808in}{3.067069in}}%
\pgfpathclose%
\pgfusepath{stroke,fill}%
\end{pgfscope}%
\begin{pgfscope}%
\pgfpathrectangle{\pgfqpoint{0.100000in}{0.220728in}}{\pgfqpoint{3.696000in}{3.696000in}}%
\pgfusepath{clip}%
\pgfsetbuttcap%
\pgfsetroundjoin%
\definecolor{currentfill}{rgb}{0.121569,0.466667,0.705882}%
\pgfsetfillcolor{currentfill}%
\pgfsetfillopacity{0.409134}%
\pgfsetlinewidth{1.003750pt}%
\definecolor{currentstroke}{rgb}{0.121569,0.466667,0.705882}%
\pgfsetstrokecolor{currentstroke}%
\pgfsetstrokeopacity{0.409134}%
\pgfsetdash{}{0pt}%
\pgfpathmoveto{\pgfqpoint{1.301557in}{2.438075in}}%
\pgfpathcurveto{\pgfqpoint{1.309794in}{2.438075in}}{\pgfqpoint{1.317694in}{2.441347in}}{\pgfqpoint{1.323518in}{2.447171in}}%
\pgfpathcurveto{\pgfqpoint{1.329342in}{2.452995in}}{\pgfqpoint{1.332614in}{2.460895in}}{\pgfqpoint{1.332614in}{2.469131in}}%
\pgfpathcurveto{\pgfqpoint{1.332614in}{2.477368in}}{\pgfqpoint{1.329342in}{2.485268in}}{\pgfqpoint{1.323518in}{2.491092in}}%
\pgfpathcurveto{\pgfqpoint{1.317694in}{2.496915in}}{\pgfqpoint{1.309794in}{2.500188in}}{\pgfqpoint{1.301557in}{2.500188in}}%
\pgfpathcurveto{\pgfqpoint{1.293321in}{2.500188in}}{\pgfqpoint{1.285421in}{2.496915in}}{\pgfqpoint{1.279597in}{2.491092in}}%
\pgfpathcurveto{\pgfqpoint{1.273773in}{2.485268in}}{\pgfqpoint{1.270501in}{2.477368in}}{\pgfqpoint{1.270501in}{2.469131in}}%
\pgfpathcurveto{\pgfqpoint{1.270501in}{2.460895in}}{\pgfqpoint{1.273773in}{2.452995in}}{\pgfqpoint{1.279597in}{2.447171in}}%
\pgfpathcurveto{\pgfqpoint{1.285421in}{2.441347in}}{\pgfqpoint{1.293321in}{2.438075in}}{\pgfqpoint{1.301557in}{2.438075in}}%
\pgfpathclose%
\pgfusepath{stroke,fill}%
\end{pgfscope}%
\begin{pgfscope}%
\pgfpathrectangle{\pgfqpoint{0.100000in}{0.220728in}}{\pgfqpoint{3.696000in}{3.696000in}}%
\pgfusepath{clip}%
\pgfsetbuttcap%
\pgfsetroundjoin%
\definecolor{currentfill}{rgb}{0.121569,0.466667,0.705882}%
\pgfsetfillcolor{currentfill}%
\pgfsetfillopacity{0.416609}%
\pgfsetlinewidth{1.003750pt}%
\definecolor{currentstroke}{rgb}{0.121569,0.466667,0.705882}%
\pgfsetstrokecolor{currentstroke}%
\pgfsetstrokeopacity{0.416609}%
\pgfsetdash{}{0pt}%
\pgfpathmoveto{\pgfqpoint{2.185957in}{3.059949in}}%
\pgfpathcurveto{\pgfqpoint{2.194193in}{3.059949in}}{\pgfqpoint{2.202094in}{3.063221in}}{\pgfqpoint{2.207917in}{3.069045in}}%
\pgfpathcurveto{\pgfqpoint{2.213741in}{3.074869in}}{\pgfqpoint{2.217014in}{3.082769in}}{\pgfqpoint{2.217014in}{3.091005in}}%
\pgfpathcurveto{\pgfqpoint{2.217014in}{3.099242in}}{\pgfqpoint{2.213741in}{3.107142in}}{\pgfqpoint{2.207917in}{3.112965in}}%
\pgfpathcurveto{\pgfqpoint{2.202094in}{3.118789in}}{\pgfqpoint{2.194193in}{3.122062in}}{\pgfqpoint{2.185957in}{3.122062in}}%
\pgfpathcurveto{\pgfqpoint{2.177721in}{3.122062in}}{\pgfqpoint{2.169821in}{3.118789in}}{\pgfqpoint{2.163997in}{3.112965in}}%
\pgfpathcurveto{\pgfqpoint{2.158173in}{3.107142in}}{\pgfqpoint{2.154901in}{3.099242in}}{\pgfqpoint{2.154901in}{3.091005in}}%
\pgfpathcurveto{\pgfqpoint{2.154901in}{3.082769in}}{\pgfqpoint{2.158173in}{3.074869in}}{\pgfqpoint{2.163997in}{3.069045in}}%
\pgfpathcurveto{\pgfqpoint{2.169821in}{3.063221in}}{\pgfqpoint{2.177721in}{3.059949in}}{\pgfqpoint{2.185957in}{3.059949in}}%
\pgfpathclose%
\pgfusepath{stroke,fill}%
\end{pgfscope}%
\begin{pgfscope}%
\pgfpathrectangle{\pgfqpoint{0.100000in}{0.220728in}}{\pgfqpoint{3.696000in}{3.696000in}}%
\pgfusepath{clip}%
\pgfsetbuttcap%
\pgfsetroundjoin%
\definecolor{currentfill}{rgb}{0.121569,0.466667,0.705882}%
\pgfsetfillcolor{currentfill}%
\pgfsetfillopacity{0.420355}%
\pgfsetlinewidth{1.003750pt}%
\definecolor{currentstroke}{rgb}{0.121569,0.466667,0.705882}%
\pgfsetstrokecolor{currentstroke}%
\pgfsetstrokeopacity{0.420355}%
\pgfsetdash{}{0pt}%
\pgfpathmoveto{\pgfqpoint{1.280202in}{2.366529in}}%
\pgfpathcurveto{\pgfqpoint{1.288438in}{2.366529in}}{\pgfqpoint{1.296338in}{2.369801in}}{\pgfqpoint{1.302162in}{2.375625in}}%
\pgfpathcurveto{\pgfqpoint{1.307986in}{2.381449in}}{\pgfqpoint{1.311259in}{2.389349in}}{\pgfqpoint{1.311259in}{2.397585in}}%
\pgfpathcurveto{\pgfqpoint{1.311259in}{2.405822in}}{\pgfqpoint{1.307986in}{2.413722in}}{\pgfqpoint{1.302162in}{2.419546in}}%
\pgfpathcurveto{\pgfqpoint{1.296338in}{2.425370in}}{\pgfqpoint{1.288438in}{2.428642in}}{\pgfqpoint{1.280202in}{2.428642in}}%
\pgfpathcurveto{\pgfqpoint{1.271966in}{2.428642in}}{\pgfqpoint{1.264066in}{2.425370in}}{\pgfqpoint{1.258242in}{2.419546in}}%
\pgfpathcurveto{\pgfqpoint{1.252418in}{2.413722in}}{\pgfqpoint{1.249146in}{2.405822in}}{\pgfqpoint{1.249146in}{2.397585in}}%
\pgfpathcurveto{\pgfqpoint{1.249146in}{2.389349in}}{\pgfqpoint{1.252418in}{2.381449in}}{\pgfqpoint{1.258242in}{2.375625in}}%
\pgfpathcurveto{\pgfqpoint{1.264066in}{2.369801in}}{\pgfqpoint{1.271966in}{2.366529in}}{\pgfqpoint{1.280202in}{2.366529in}}%
\pgfpathclose%
\pgfusepath{stroke,fill}%
\end{pgfscope}%
\begin{pgfscope}%
\pgfpathrectangle{\pgfqpoint{0.100000in}{0.220728in}}{\pgfqpoint{3.696000in}{3.696000in}}%
\pgfusepath{clip}%
\pgfsetbuttcap%
\pgfsetroundjoin%
\definecolor{currentfill}{rgb}{0.121569,0.466667,0.705882}%
\pgfsetfillcolor{currentfill}%
\pgfsetfillopacity{0.426364}%
\pgfsetlinewidth{1.003750pt}%
\definecolor{currentstroke}{rgb}{0.121569,0.466667,0.705882}%
\pgfsetstrokecolor{currentstroke}%
\pgfsetstrokeopacity{0.426364}%
\pgfsetdash{}{0pt}%
\pgfpathmoveto{\pgfqpoint{2.230405in}{3.053558in}}%
\pgfpathcurveto{\pgfqpoint{2.238641in}{3.053558in}}{\pgfqpoint{2.246542in}{3.056830in}}{\pgfqpoint{2.252365in}{3.062654in}}%
\pgfpathcurveto{\pgfqpoint{2.258189in}{3.068478in}}{\pgfqpoint{2.261462in}{3.076378in}}{\pgfqpoint{2.261462in}{3.084614in}}%
\pgfpathcurveto{\pgfqpoint{2.261462in}{3.092850in}}{\pgfqpoint{2.258189in}{3.100750in}}{\pgfqpoint{2.252365in}{3.106574in}}%
\pgfpathcurveto{\pgfqpoint{2.246542in}{3.112398in}}{\pgfqpoint{2.238641in}{3.115671in}}{\pgfqpoint{2.230405in}{3.115671in}}%
\pgfpathcurveto{\pgfqpoint{2.222169in}{3.115671in}}{\pgfqpoint{2.214269in}{3.112398in}}{\pgfqpoint{2.208445in}{3.106574in}}%
\pgfpathcurveto{\pgfqpoint{2.202621in}{3.100750in}}{\pgfqpoint{2.199349in}{3.092850in}}{\pgfqpoint{2.199349in}{3.084614in}}%
\pgfpathcurveto{\pgfqpoint{2.199349in}{3.076378in}}{\pgfqpoint{2.202621in}{3.068478in}}{\pgfqpoint{2.208445in}{3.062654in}}%
\pgfpathcurveto{\pgfqpoint{2.214269in}{3.056830in}}{\pgfqpoint{2.222169in}{3.053558in}}{\pgfqpoint{2.230405in}{3.053558in}}%
\pgfpathclose%
\pgfusepath{stroke,fill}%
\end{pgfscope}%
\begin{pgfscope}%
\pgfpathrectangle{\pgfqpoint{0.100000in}{0.220728in}}{\pgfqpoint{3.696000in}{3.696000in}}%
\pgfusepath{clip}%
\pgfsetbuttcap%
\pgfsetroundjoin%
\definecolor{currentfill}{rgb}{0.121569,0.466667,0.705882}%
\pgfsetfillcolor{currentfill}%
\pgfsetfillopacity{0.427712}%
\pgfsetlinewidth{1.003750pt}%
\definecolor{currentstroke}{rgb}{0.121569,0.466667,0.705882}%
\pgfsetstrokecolor{currentstroke}%
\pgfsetstrokeopacity{0.427712}%
\pgfsetdash{}{0pt}%
\pgfpathmoveto{\pgfqpoint{1.239820in}{2.304673in}}%
\pgfpathcurveto{\pgfqpoint{1.248057in}{2.304673in}}{\pgfqpoint{1.255957in}{2.307945in}}{\pgfqpoint{1.261781in}{2.313769in}}%
\pgfpathcurveto{\pgfqpoint{1.267605in}{2.319593in}}{\pgfqpoint{1.270877in}{2.327493in}}{\pgfqpoint{1.270877in}{2.335729in}}%
\pgfpathcurveto{\pgfqpoint{1.270877in}{2.343965in}}{\pgfqpoint{1.267605in}{2.351866in}}{\pgfqpoint{1.261781in}{2.357689in}}%
\pgfpathcurveto{\pgfqpoint{1.255957in}{2.363513in}}{\pgfqpoint{1.248057in}{2.366786in}}{\pgfqpoint{1.239820in}{2.366786in}}%
\pgfpathcurveto{\pgfqpoint{1.231584in}{2.366786in}}{\pgfqpoint{1.223684in}{2.363513in}}{\pgfqpoint{1.217860in}{2.357689in}}%
\pgfpathcurveto{\pgfqpoint{1.212036in}{2.351866in}}{\pgfqpoint{1.208764in}{2.343965in}}{\pgfqpoint{1.208764in}{2.335729in}}%
\pgfpathcurveto{\pgfqpoint{1.208764in}{2.327493in}}{\pgfqpoint{1.212036in}{2.319593in}}{\pgfqpoint{1.217860in}{2.313769in}}%
\pgfpathcurveto{\pgfqpoint{1.223684in}{2.307945in}}{\pgfqpoint{1.231584in}{2.304673in}}{\pgfqpoint{1.239820in}{2.304673in}}%
\pgfpathclose%
\pgfusepath{stroke,fill}%
\end{pgfscope}%
\begin{pgfscope}%
\pgfpathrectangle{\pgfqpoint{0.100000in}{0.220728in}}{\pgfqpoint{3.696000in}{3.696000in}}%
\pgfusepath{clip}%
\pgfsetbuttcap%
\pgfsetroundjoin%
\definecolor{currentfill}{rgb}{0.121569,0.466667,0.705882}%
\pgfsetfillcolor{currentfill}%
\pgfsetfillopacity{0.430976}%
\pgfsetlinewidth{1.003750pt}%
\definecolor{currentstroke}{rgb}{0.121569,0.466667,0.705882}%
\pgfsetstrokecolor{currentstroke}%
\pgfsetstrokeopacity{0.430976}%
\pgfsetdash{}{0pt}%
\pgfpathmoveto{\pgfqpoint{2.255535in}{3.048215in}}%
\pgfpathcurveto{\pgfqpoint{2.263771in}{3.048215in}}{\pgfqpoint{2.271671in}{3.051487in}}{\pgfqpoint{2.277495in}{3.057311in}}%
\pgfpathcurveto{\pgfqpoint{2.283319in}{3.063135in}}{\pgfqpoint{2.286591in}{3.071035in}}{\pgfqpoint{2.286591in}{3.079271in}}%
\pgfpathcurveto{\pgfqpoint{2.286591in}{3.087508in}}{\pgfqpoint{2.283319in}{3.095408in}}{\pgfqpoint{2.277495in}{3.101231in}}%
\pgfpathcurveto{\pgfqpoint{2.271671in}{3.107055in}}{\pgfqpoint{2.263771in}{3.110328in}}{\pgfqpoint{2.255535in}{3.110328in}}%
\pgfpathcurveto{\pgfqpoint{2.247298in}{3.110328in}}{\pgfqpoint{2.239398in}{3.107055in}}{\pgfqpoint{2.233574in}{3.101231in}}%
\pgfpathcurveto{\pgfqpoint{2.227750in}{3.095408in}}{\pgfqpoint{2.224478in}{3.087508in}}{\pgfqpoint{2.224478in}{3.079271in}}%
\pgfpathcurveto{\pgfqpoint{2.224478in}{3.071035in}}{\pgfqpoint{2.227750in}{3.063135in}}{\pgfqpoint{2.233574in}{3.057311in}}%
\pgfpathcurveto{\pgfqpoint{2.239398in}{3.051487in}}{\pgfqpoint{2.247298in}{3.048215in}}{\pgfqpoint{2.255535in}{3.048215in}}%
\pgfpathclose%
\pgfusepath{stroke,fill}%
\end{pgfscope}%
\begin{pgfscope}%
\pgfpathrectangle{\pgfqpoint{0.100000in}{0.220728in}}{\pgfqpoint{3.696000in}{3.696000in}}%
\pgfusepath{clip}%
\pgfsetbuttcap%
\pgfsetroundjoin%
\definecolor{currentfill}{rgb}{0.121569,0.466667,0.705882}%
\pgfsetfillcolor{currentfill}%
\pgfsetfillopacity{0.436521}%
\pgfsetlinewidth{1.003750pt}%
\definecolor{currentstroke}{rgb}{0.121569,0.466667,0.705882}%
\pgfsetstrokecolor{currentstroke}%
\pgfsetstrokeopacity{0.436521}%
\pgfsetdash{}{0pt}%
\pgfpathmoveto{\pgfqpoint{1.225028in}{2.250676in}}%
\pgfpathcurveto{\pgfqpoint{1.233265in}{2.250676in}}{\pgfqpoint{1.241165in}{2.253948in}}{\pgfqpoint{1.246989in}{2.259772in}}%
\pgfpathcurveto{\pgfqpoint{1.252813in}{2.265596in}}{\pgfqpoint{1.256085in}{2.273496in}}{\pgfqpoint{1.256085in}{2.281733in}}%
\pgfpathcurveto{\pgfqpoint{1.256085in}{2.289969in}}{\pgfqpoint{1.252813in}{2.297869in}}{\pgfqpoint{1.246989in}{2.303693in}}%
\pgfpathcurveto{\pgfqpoint{1.241165in}{2.309517in}}{\pgfqpoint{1.233265in}{2.312789in}}{\pgfqpoint{1.225028in}{2.312789in}}%
\pgfpathcurveto{\pgfqpoint{1.216792in}{2.312789in}}{\pgfqpoint{1.208892in}{2.309517in}}{\pgfqpoint{1.203068in}{2.303693in}}%
\pgfpathcurveto{\pgfqpoint{1.197244in}{2.297869in}}{\pgfqpoint{1.193972in}{2.289969in}}{\pgfqpoint{1.193972in}{2.281733in}}%
\pgfpathcurveto{\pgfqpoint{1.193972in}{2.273496in}}{\pgfqpoint{1.197244in}{2.265596in}}{\pgfqpoint{1.203068in}{2.259772in}}%
\pgfpathcurveto{\pgfqpoint{1.208892in}{2.253948in}}{\pgfqpoint{1.216792in}{2.250676in}}{\pgfqpoint{1.225028in}{2.250676in}}%
\pgfpathclose%
\pgfusepath{stroke,fill}%
\end{pgfscope}%
\begin{pgfscope}%
\pgfpathrectangle{\pgfqpoint{0.100000in}{0.220728in}}{\pgfqpoint{3.696000in}{3.696000in}}%
\pgfusepath{clip}%
\pgfsetbuttcap%
\pgfsetroundjoin%
\definecolor{currentfill}{rgb}{0.121569,0.466667,0.705882}%
\pgfsetfillcolor{currentfill}%
\pgfsetfillopacity{0.439621}%
\pgfsetlinewidth{1.003750pt}%
\definecolor{currentstroke}{rgb}{0.121569,0.466667,0.705882}%
\pgfsetstrokecolor{currentstroke}%
\pgfsetstrokeopacity{0.439621}%
\pgfsetdash{}{0pt}%
\pgfpathmoveto{\pgfqpoint{2.285518in}{3.043118in}}%
\pgfpathcurveto{\pgfqpoint{2.293754in}{3.043118in}}{\pgfqpoint{2.301654in}{3.046390in}}{\pgfqpoint{2.307478in}{3.052214in}}%
\pgfpathcurveto{\pgfqpoint{2.313302in}{3.058038in}}{\pgfqpoint{2.316574in}{3.065938in}}{\pgfqpoint{2.316574in}{3.074174in}}%
\pgfpathcurveto{\pgfqpoint{2.316574in}{3.082411in}}{\pgfqpoint{2.313302in}{3.090311in}}{\pgfqpoint{2.307478in}{3.096135in}}%
\pgfpathcurveto{\pgfqpoint{2.301654in}{3.101959in}}{\pgfqpoint{2.293754in}{3.105231in}}{\pgfqpoint{2.285518in}{3.105231in}}%
\pgfpathcurveto{\pgfqpoint{2.277282in}{3.105231in}}{\pgfqpoint{2.269381in}{3.101959in}}{\pgfqpoint{2.263558in}{3.096135in}}%
\pgfpathcurveto{\pgfqpoint{2.257734in}{3.090311in}}{\pgfqpoint{2.254461in}{3.082411in}}{\pgfqpoint{2.254461in}{3.074174in}}%
\pgfpathcurveto{\pgfqpoint{2.254461in}{3.065938in}}{\pgfqpoint{2.257734in}{3.058038in}}{\pgfqpoint{2.263558in}{3.052214in}}%
\pgfpathcurveto{\pgfqpoint{2.269381in}{3.046390in}}{\pgfqpoint{2.277282in}{3.043118in}}{\pgfqpoint{2.285518in}{3.043118in}}%
\pgfpathclose%
\pgfusepath{stroke,fill}%
\end{pgfscope}%
\begin{pgfscope}%
\pgfpathrectangle{\pgfqpoint{0.100000in}{0.220728in}}{\pgfqpoint{3.696000in}{3.696000in}}%
\pgfusepath{clip}%
\pgfsetbuttcap%
\pgfsetroundjoin%
\definecolor{currentfill}{rgb}{0.121569,0.466667,0.705882}%
\pgfsetfillcolor{currentfill}%
\pgfsetfillopacity{0.441792}%
\pgfsetlinewidth{1.003750pt}%
\definecolor{currentstroke}{rgb}{0.121569,0.466667,0.705882}%
\pgfsetstrokecolor{currentstroke}%
\pgfsetstrokeopacity{0.441792}%
\pgfsetdash{}{0pt}%
\pgfpathmoveto{\pgfqpoint{1.200484in}{2.215915in}}%
\pgfpathcurveto{\pgfqpoint{1.208720in}{2.215915in}}{\pgfqpoint{1.216620in}{2.219187in}}{\pgfqpoint{1.222444in}{2.225011in}}%
\pgfpathcurveto{\pgfqpoint{1.228268in}{2.230835in}}{\pgfqpoint{1.231541in}{2.238735in}}{\pgfqpoint{1.231541in}{2.246971in}}%
\pgfpathcurveto{\pgfqpoint{1.231541in}{2.255208in}}{\pgfqpoint{1.228268in}{2.263108in}}{\pgfqpoint{1.222444in}{2.268932in}}%
\pgfpathcurveto{\pgfqpoint{1.216620in}{2.274756in}}{\pgfqpoint{1.208720in}{2.278028in}}{\pgfqpoint{1.200484in}{2.278028in}}%
\pgfpathcurveto{\pgfqpoint{1.192248in}{2.278028in}}{\pgfqpoint{1.184348in}{2.274756in}}{\pgfqpoint{1.178524in}{2.268932in}}%
\pgfpathcurveto{\pgfqpoint{1.172700in}{2.263108in}}{\pgfqpoint{1.169428in}{2.255208in}}{\pgfqpoint{1.169428in}{2.246971in}}%
\pgfpathcurveto{\pgfqpoint{1.169428in}{2.238735in}}{\pgfqpoint{1.172700in}{2.230835in}}{\pgfqpoint{1.178524in}{2.225011in}}%
\pgfpathcurveto{\pgfqpoint{1.184348in}{2.219187in}}{\pgfqpoint{1.192248in}{2.215915in}}{\pgfqpoint{1.200484in}{2.215915in}}%
\pgfpathclose%
\pgfusepath{stroke,fill}%
\end{pgfscope}%
\begin{pgfscope}%
\pgfpathrectangle{\pgfqpoint{0.100000in}{0.220728in}}{\pgfqpoint{3.696000in}{3.696000in}}%
\pgfusepath{clip}%
\pgfsetbuttcap%
\pgfsetroundjoin%
\definecolor{currentfill}{rgb}{0.121569,0.466667,0.705882}%
\pgfsetfillcolor{currentfill}%
\pgfsetfillopacity{0.445618}%
\pgfsetlinewidth{1.003750pt}%
\definecolor{currentstroke}{rgb}{0.121569,0.466667,0.705882}%
\pgfsetstrokecolor{currentstroke}%
\pgfsetstrokeopacity{0.445618}%
\pgfsetdash{}{0pt}%
\pgfpathmoveto{\pgfqpoint{2.322475in}{3.037529in}}%
\pgfpathcurveto{\pgfqpoint{2.330712in}{3.037529in}}{\pgfqpoint{2.338612in}{3.040802in}}{\pgfqpoint{2.344436in}{3.046625in}}%
\pgfpathcurveto{\pgfqpoint{2.350260in}{3.052449in}}{\pgfqpoint{2.353532in}{3.060349in}}{\pgfqpoint{2.353532in}{3.068586in}}%
\pgfpathcurveto{\pgfqpoint{2.353532in}{3.076822in}}{\pgfqpoint{2.350260in}{3.084722in}}{\pgfqpoint{2.344436in}{3.090546in}}%
\pgfpathcurveto{\pgfqpoint{2.338612in}{3.096370in}}{\pgfqpoint{2.330712in}{3.099642in}}{\pgfqpoint{2.322475in}{3.099642in}}%
\pgfpathcurveto{\pgfqpoint{2.314239in}{3.099642in}}{\pgfqpoint{2.306339in}{3.096370in}}{\pgfqpoint{2.300515in}{3.090546in}}%
\pgfpathcurveto{\pgfqpoint{2.294691in}{3.084722in}}{\pgfqpoint{2.291419in}{3.076822in}}{\pgfqpoint{2.291419in}{3.068586in}}%
\pgfpathcurveto{\pgfqpoint{2.291419in}{3.060349in}}{\pgfqpoint{2.294691in}{3.052449in}}{\pgfqpoint{2.300515in}{3.046625in}}%
\pgfpathcurveto{\pgfqpoint{2.306339in}{3.040802in}}{\pgfqpoint{2.314239in}{3.037529in}}{\pgfqpoint{2.322475in}{3.037529in}}%
\pgfpathclose%
\pgfusepath{stroke,fill}%
\end{pgfscope}%
\begin{pgfscope}%
\pgfpathrectangle{\pgfqpoint{0.100000in}{0.220728in}}{\pgfqpoint{3.696000in}{3.696000in}}%
\pgfusepath{clip}%
\pgfsetbuttcap%
\pgfsetroundjoin%
\definecolor{currentfill}{rgb}{0.121569,0.466667,0.705882}%
\pgfsetfillcolor{currentfill}%
\pgfsetfillopacity{0.446898}%
\pgfsetlinewidth{1.003750pt}%
\definecolor{currentstroke}{rgb}{0.121569,0.466667,0.705882}%
\pgfsetstrokecolor{currentstroke}%
\pgfsetstrokeopacity{0.446898}%
\pgfsetdash{}{0pt}%
\pgfpathmoveto{\pgfqpoint{1.190520in}{2.187380in}}%
\pgfpathcurveto{\pgfqpoint{1.198757in}{2.187380in}}{\pgfqpoint{1.206657in}{2.190652in}}{\pgfqpoint{1.212481in}{2.196476in}}%
\pgfpathcurveto{\pgfqpoint{1.218305in}{2.202300in}}{\pgfqpoint{1.221577in}{2.210200in}}{\pgfqpoint{1.221577in}{2.218437in}}%
\pgfpathcurveto{\pgfqpoint{1.221577in}{2.226673in}}{\pgfqpoint{1.218305in}{2.234573in}}{\pgfqpoint{1.212481in}{2.240397in}}%
\pgfpathcurveto{\pgfqpoint{1.206657in}{2.246221in}}{\pgfqpoint{1.198757in}{2.249493in}}{\pgfqpoint{1.190520in}{2.249493in}}%
\pgfpathcurveto{\pgfqpoint{1.182284in}{2.249493in}}{\pgfqpoint{1.174384in}{2.246221in}}{\pgfqpoint{1.168560in}{2.240397in}}%
\pgfpathcurveto{\pgfqpoint{1.162736in}{2.234573in}}{\pgfqpoint{1.159464in}{2.226673in}}{\pgfqpoint{1.159464in}{2.218437in}}%
\pgfpathcurveto{\pgfqpoint{1.159464in}{2.210200in}}{\pgfqpoint{1.162736in}{2.202300in}}{\pgfqpoint{1.168560in}{2.196476in}}%
\pgfpathcurveto{\pgfqpoint{1.174384in}{2.190652in}}{\pgfqpoint{1.182284in}{2.187380in}}{\pgfqpoint{1.190520in}{2.187380in}}%
\pgfpathclose%
\pgfusepath{stroke,fill}%
\end{pgfscope}%
\begin{pgfscope}%
\pgfpathrectangle{\pgfqpoint{0.100000in}{0.220728in}}{\pgfqpoint{3.696000in}{3.696000in}}%
\pgfusepath{clip}%
\pgfsetbuttcap%
\pgfsetroundjoin%
\definecolor{currentfill}{rgb}{0.121569,0.466667,0.705882}%
\pgfsetfillcolor{currentfill}%
\pgfsetfillopacity{0.449162}%
\pgfsetlinewidth{1.003750pt}%
\definecolor{currentstroke}{rgb}{0.121569,0.466667,0.705882}%
\pgfsetstrokecolor{currentstroke}%
\pgfsetstrokeopacity{0.449162}%
\pgfsetdash{}{0pt}%
\pgfpathmoveto{\pgfqpoint{1.180608in}{2.169118in}}%
\pgfpathcurveto{\pgfqpoint{1.188844in}{2.169118in}}{\pgfqpoint{1.196744in}{2.172390in}}{\pgfqpoint{1.202568in}{2.178214in}}%
\pgfpathcurveto{\pgfqpoint{1.208392in}{2.184038in}}{\pgfqpoint{1.211664in}{2.191938in}}{\pgfqpoint{1.211664in}{2.200174in}}%
\pgfpathcurveto{\pgfqpoint{1.211664in}{2.208411in}}{\pgfqpoint{1.208392in}{2.216311in}}{\pgfqpoint{1.202568in}{2.222135in}}%
\pgfpathcurveto{\pgfqpoint{1.196744in}{2.227959in}}{\pgfqpoint{1.188844in}{2.231231in}}{\pgfqpoint{1.180608in}{2.231231in}}%
\pgfpathcurveto{\pgfqpoint{1.172372in}{2.231231in}}{\pgfqpoint{1.164472in}{2.227959in}}{\pgfqpoint{1.158648in}{2.222135in}}%
\pgfpathcurveto{\pgfqpoint{1.152824in}{2.216311in}}{\pgfqpoint{1.149551in}{2.208411in}}{\pgfqpoint{1.149551in}{2.200174in}}%
\pgfpathcurveto{\pgfqpoint{1.149551in}{2.191938in}}{\pgfqpoint{1.152824in}{2.184038in}}{\pgfqpoint{1.158648in}{2.178214in}}%
\pgfpathcurveto{\pgfqpoint{1.164472in}{2.172390in}}{\pgfqpoint{1.172372in}{2.169118in}}{\pgfqpoint{1.180608in}{2.169118in}}%
\pgfpathclose%
\pgfusepath{stroke,fill}%
\end{pgfscope}%
\begin{pgfscope}%
\pgfpathrectangle{\pgfqpoint{0.100000in}{0.220728in}}{\pgfqpoint{3.696000in}{3.696000in}}%
\pgfusepath{clip}%
\pgfsetbuttcap%
\pgfsetroundjoin%
\definecolor{currentfill}{rgb}{0.121569,0.466667,0.705882}%
\pgfsetfillcolor{currentfill}%
\pgfsetfillopacity{0.450225}%
\pgfsetlinewidth{1.003750pt}%
\definecolor{currentstroke}{rgb}{0.121569,0.466667,0.705882}%
\pgfsetstrokecolor{currentstroke}%
\pgfsetstrokeopacity{0.450225}%
\pgfsetdash{}{0pt}%
\pgfpathmoveto{\pgfqpoint{2.341815in}{3.037364in}}%
\pgfpathcurveto{\pgfqpoint{2.350051in}{3.037364in}}{\pgfqpoint{2.357951in}{3.040636in}}{\pgfqpoint{2.363775in}{3.046460in}}%
\pgfpathcurveto{\pgfqpoint{2.369599in}{3.052284in}}{\pgfqpoint{2.372872in}{3.060184in}}{\pgfqpoint{2.372872in}{3.068420in}}%
\pgfpathcurveto{\pgfqpoint{2.372872in}{3.076657in}}{\pgfqpoint{2.369599in}{3.084557in}}{\pgfqpoint{2.363775in}{3.090380in}}%
\pgfpathcurveto{\pgfqpoint{2.357951in}{3.096204in}}{\pgfqpoint{2.350051in}{3.099477in}}{\pgfqpoint{2.341815in}{3.099477in}}%
\pgfpathcurveto{\pgfqpoint{2.333579in}{3.099477in}}{\pgfqpoint{2.325679in}{3.096204in}}{\pgfqpoint{2.319855in}{3.090380in}}%
\pgfpathcurveto{\pgfqpoint{2.314031in}{3.084557in}}{\pgfqpoint{2.310759in}{3.076657in}}{\pgfqpoint{2.310759in}{3.068420in}}%
\pgfpathcurveto{\pgfqpoint{2.310759in}{3.060184in}}{\pgfqpoint{2.314031in}{3.052284in}}{\pgfqpoint{2.319855in}{3.046460in}}%
\pgfpathcurveto{\pgfqpoint{2.325679in}{3.040636in}}{\pgfqpoint{2.333579in}{3.037364in}}{\pgfqpoint{2.341815in}{3.037364in}}%
\pgfpathclose%
\pgfusepath{stroke,fill}%
\end{pgfscope}%
\begin{pgfscope}%
\pgfpathrectangle{\pgfqpoint{0.100000in}{0.220728in}}{\pgfqpoint{3.696000in}{3.696000in}}%
\pgfusepath{clip}%
\pgfsetbuttcap%
\pgfsetroundjoin%
\definecolor{currentfill}{rgb}{0.121569,0.466667,0.705882}%
\pgfsetfillcolor{currentfill}%
\pgfsetfillopacity{0.450574}%
\pgfsetlinewidth{1.003750pt}%
\definecolor{currentstroke}{rgb}{0.121569,0.466667,0.705882}%
\pgfsetstrokecolor{currentstroke}%
\pgfsetstrokeopacity{0.450574}%
\pgfsetdash{}{0pt}%
\pgfpathmoveto{\pgfqpoint{1.176780in}{2.160915in}}%
\pgfpathcurveto{\pgfqpoint{1.185017in}{2.160915in}}{\pgfqpoint{1.192917in}{2.164187in}}{\pgfqpoint{1.198741in}{2.170011in}}%
\pgfpathcurveto{\pgfqpoint{1.204565in}{2.175835in}}{\pgfqpoint{1.207837in}{2.183735in}}{\pgfqpoint{1.207837in}{2.191971in}}%
\pgfpathcurveto{\pgfqpoint{1.207837in}{2.200207in}}{\pgfqpoint{1.204565in}{2.208107in}}{\pgfqpoint{1.198741in}{2.213931in}}%
\pgfpathcurveto{\pgfqpoint{1.192917in}{2.219755in}}{\pgfqpoint{1.185017in}{2.223028in}}{\pgfqpoint{1.176780in}{2.223028in}}%
\pgfpathcurveto{\pgfqpoint{1.168544in}{2.223028in}}{\pgfqpoint{1.160644in}{2.219755in}}{\pgfqpoint{1.154820in}{2.213931in}}%
\pgfpathcurveto{\pgfqpoint{1.148996in}{2.208107in}}{\pgfqpoint{1.145724in}{2.200207in}}{\pgfqpoint{1.145724in}{2.191971in}}%
\pgfpathcurveto{\pgfqpoint{1.145724in}{2.183735in}}{\pgfqpoint{1.148996in}{2.175835in}}{\pgfqpoint{1.154820in}{2.170011in}}%
\pgfpathcurveto{\pgfqpoint{1.160644in}{2.164187in}}{\pgfqpoint{1.168544in}{2.160915in}}{\pgfqpoint{1.176780in}{2.160915in}}%
\pgfpathclose%
\pgfusepath{stroke,fill}%
\end{pgfscope}%
\begin{pgfscope}%
\pgfpathrectangle{\pgfqpoint{0.100000in}{0.220728in}}{\pgfqpoint{3.696000in}{3.696000in}}%
\pgfusepath{clip}%
\pgfsetbuttcap%
\pgfsetroundjoin%
\definecolor{currentfill}{rgb}{0.121569,0.466667,0.705882}%
\pgfsetfillcolor{currentfill}%
\pgfsetfillopacity{0.450928}%
\pgfsetlinewidth{1.003750pt}%
\definecolor{currentstroke}{rgb}{0.121569,0.466667,0.705882}%
\pgfsetstrokecolor{currentstroke}%
\pgfsetstrokeopacity{0.450928}%
\pgfsetdash{}{0pt}%
\pgfpathmoveto{\pgfqpoint{1.175713in}{2.158615in}}%
\pgfpathcurveto{\pgfqpoint{1.183949in}{2.158615in}}{\pgfqpoint{1.191849in}{2.161888in}}{\pgfqpoint{1.197673in}{2.167712in}}%
\pgfpathcurveto{\pgfqpoint{1.203497in}{2.173536in}}{\pgfqpoint{1.206769in}{2.181436in}}{\pgfqpoint{1.206769in}{2.189672in}}%
\pgfpathcurveto{\pgfqpoint{1.206769in}{2.197908in}}{\pgfqpoint{1.203497in}{2.205808in}}{\pgfqpoint{1.197673in}{2.211632in}}%
\pgfpathcurveto{\pgfqpoint{1.191849in}{2.217456in}}{\pgfqpoint{1.183949in}{2.220728in}}{\pgfqpoint{1.175713in}{2.220728in}}%
\pgfpathcurveto{\pgfqpoint{1.167476in}{2.220728in}}{\pgfqpoint{1.159576in}{2.217456in}}{\pgfqpoint{1.153752in}{2.211632in}}%
\pgfpathcurveto{\pgfqpoint{1.147929in}{2.205808in}}{\pgfqpoint{1.144656in}{2.197908in}}{\pgfqpoint{1.144656in}{2.189672in}}%
\pgfpathcurveto{\pgfqpoint{1.144656in}{2.181436in}}{\pgfqpoint{1.147929in}{2.173536in}}{\pgfqpoint{1.153752in}{2.167712in}}%
\pgfpathcurveto{\pgfqpoint{1.159576in}{2.161888in}}{\pgfqpoint{1.167476in}{2.158615in}}{\pgfqpoint{1.175713in}{2.158615in}}%
\pgfpathclose%
\pgfusepath{stroke,fill}%
\end{pgfscope}%
\begin{pgfscope}%
\pgfpathrectangle{\pgfqpoint{0.100000in}{0.220728in}}{\pgfqpoint{3.696000in}{3.696000in}}%
\pgfusepath{clip}%
\pgfsetbuttcap%
\pgfsetroundjoin%
\definecolor{currentfill}{rgb}{0.121569,0.466667,0.705882}%
\pgfsetfillcolor{currentfill}%
\pgfsetfillopacity{0.451618}%
\pgfsetlinewidth{1.003750pt}%
\definecolor{currentstroke}{rgb}{0.121569,0.466667,0.705882}%
\pgfsetstrokecolor{currentstroke}%
\pgfsetstrokeopacity{0.451618}%
\pgfsetdash{}{0pt}%
\pgfpathmoveto{\pgfqpoint{1.173722in}{2.154700in}}%
\pgfpathcurveto{\pgfqpoint{1.181958in}{2.154700in}}{\pgfqpoint{1.189858in}{2.157973in}}{\pgfqpoint{1.195682in}{2.163797in}}%
\pgfpathcurveto{\pgfqpoint{1.201506in}{2.169620in}}{\pgfqpoint{1.204778in}{2.177521in}}{\pgfqpoint{1.204778in}{2.185757in}}%
\pgfpathcurveto{\pgfqpoint{1.204778in}{2.193993in}}{\pgfqpoint{1.201506in}{2.201893in}}{\pgfqpoint{1.195682in}{2.207717in}}%
\pgfpathcurveto{\pgfqpoint{1.189858in}{2.213541in}}{\pgfqpoint{1.181958in}{2.216813in}}{\pgfqpoint{1.173722in}{2.216813in}}%
\pgfpathcurveto{\pgfqpoint{1.165485in}{2.216813in}}{\pgfqpoint{1.157585in}{2.213541in}}{\pgfqpoint{1.151761in}{2.207717in}}%
\pgfpathcurveto{\pgfqpoint{1.145937in}{2.201893in}}{\pgfqpoint{1.142665in}{2.193993in}}{\pgfqpoint{1.142665in}{2.185757in}}%
\pgfpathcurveto{\pgfqpoint{1.142665in}{2.177521in}}{\pgfqpoint{1.145937in}{2.169620in}}{\pgfqpoint{1.151761in}{2.163797in}}%
\pgfpathcurveto{\pgfqpoint{1.157585in}{2.157973in}}{\pgfqpoint{1.165485in}{2.154700in}}{\pgfqpoint{1.173722in}{2.154700in}}%
\pgfpathclose%
\pgfusepath{stroke,fill}%
\end{pgfscope}%
\begin{pgfscope}%
\pgfpathrectangle{\pgfqpoint{0.100000in}{0.220728in}}{\pgfqpoint{3.696000in}{3.696000in}}%
\pgfusepath{clip}%
\pgfsetbuttcap%
\pgfsetroundjoin%
\definecolor{currentfill}{rgb}{0.121569,0.466667,0.705882}%
\pgfsetfillcolor{currentfill}%
\pgfsetfillopacity{0.452112}%
\pgfsetlinewidth{1.003750pt}%
\definecolor{currentstroke}{rgb}{0.121569,0.466667,0.705882}%
\pgfsetstrokecolor{currentstroke}%
\pgfsetstrokeopacity{0.452112}%
\pgfsetdash{}{0pt}%
\pgfpathmoveto{\pgfqpoint{2.352801in}{3.035179in}}%
\pgfpathcurveto{\pgfqpoint{2.361037in}{3.035179in}}{\pgfqpoint{2.368937in}{3.038452in}}{\pgfqpoint{2.374761in}{3.044276in}}%
\pgfpathcurveto{\pgfqpoint{2.380585in}{3.050100in}}{\pgfqpoint{2.383857in}{3.058000in}}{\pgfqpoint{2.383857in}{3.066236in}}%
\pgfpathcurveto{\pgfqpoint{2.383857in}{3.074472in}}{\pgfqpoint{2.380585in}{3.082372in}}{\pgfqpoint{2.374761in}{3.088196in}}%
\pgfpathcurveto{\pgfqpoint{2.368937in}{3.094020in}}{\pgfqpoint{2.361037in}{3.097292in}}{\pgfqpoint{2.352801in}{3.097292in}}%
\pgfpathcurveto{\pgfqpoint{2.344565in}{3.097292in}}{\pgfqpoint{2.336664in}{3.094020in}}{\pgfqpoint{2.330841in}{3.088196in}}%
\pgfpathcurveto{\pgfqpoint{2.325017in}{3.082372in}}{\pgfqpoint{2.321744in}{3.074472in}}{\pgfqpoint{2.321744in}{3.066236in}}%
\pgfpathcurveto{\pgfqpoint{2.321744in}{3.058000in}}{\pgfqpoint{2.325017in}{3.050100in}}{\pgfqpoint{2.330841in}{3.044276in}}%
\pgfpathcurveto{\pgfqpoint{2.336664in}{3.038452in}}{\pgfqpoint{2.344565in}{3.035179in}}{\pgfqpoint{2.352801in}{3.035179in}}%
\pgfpathclose%
\pgfusepath{stroke,fill}%
\end{pgfscope}%
\begin{pgfscope}%
\pgfpathrectangle{\pgfqpoint{0.100000in}{0.220728in}}{\pgfqpoint{3.696000in}{3.696000in}}%
\pgfusepath{clip}%
\pgfsetbuttcap%
\pgfsetroundjoin%
\definecolor{currentfill}{rgb}{0.121569,0.466667,0.705882}%
\pgfsetfillcolor{currentfill}%
\pgfsetfillopacity{0.452904}%
\pgfsetlinewidth{1.003750pt}%
\definecolor{currentstroke}{rgb}{0.121569,0.466667,0.705882}%
\pgfsetstrokecolor{currentstroke}%
\pgfsetstrokeopacity{0.452904}%
\pgfsetdash{}{0pt}%
\pgfpathmoveto{\pgfqpoint{1.170631in}{2.147250in}}%
\pgfpathcurveto{\pgfqpoint{1.178868in}{2.147250in}}{\pgfqpoint{1.186768in}{2.150522in}}{\pgfqpoint{1.192592in}{2.156346in}}%
\pgfpathcurveto{\pgfqpoint{1.198416in}{2.162170in}}{\pgfqpoint{1.201688in}{2.170070in}}{\pgfqpoint{1.201688in}{2.178306in}}%
\pgfpathcurveto{\pgfqpoint{1.201688in}{2.186542in}}{\pgfqpoint{1.198416in}{2.194442in}}{\pgfqpoint{1.192592in}{2.200266in}}%
\pgfpathcurveto{\pgfqpoint{1.186768in}{2.206090in}}{\pgfqpoint{1.178868in}{2.209363in}}{\pgfqpoint{1.170631in}{2.209363in}}%
\pgfpathcurveto{\pgfqpoint{1.162395in}{2.209363in}}{\pgfqpoint{1.154495in}{2.206090in}}{\pgfqpoint{1.148671in}{2.200266in}}%
\pgfpathcurveto{\pgfqpoint{1.142847in}{2.194442in}}{\pgfqpoint{1.139575in}{2.186542in}}{\pgfqpoint{1.139575in}{2.178306in}}%
\pgfpathcurveto{\pgfqpoint{1.139575in}{2.170070in}}{\pgfqpoint{1.142847in}{2.162170in}}{\pgfqpoint{1.148671in}{2.156346in}}%
\pgfpathcurveto{\pgfqpoint{1.154495in}{2.150522in}}{\pgfqpoint{1.162395in}{2.147250in}}{\pgfqpoint{1.170631in}{2.147250in}}%
\pgfpathclose%
\pgfusepath{stroke,fill}%
\end{pgfscope}%
\begin{pgfscope}%
\pgfpathrectangle{\pgfqpoint{0.100000in}{0.220728in}}{\pgfqpoint{3.696000in}{3.696000in}}%
\pgfusepath{clip}%
\pgfsetbuttcap%
\pgfsetroundjoin%
\definecolor{currentfill}{rgb}{0.121569,0.466667,0.705882}%
\pgfsetfillcolor{currentfill}%
\pgfsetfillopacity{0.453468}%
\pgfsetlinewidth{1.003750pt}%
\definecolor{currentstroke}{rgb}{0.121569,0.466667,0.705882}%
\pgfsetstrokecolor{currentstroke}%
\pgfsetstrokeopacity{0.453468}%
\pgfsetdash{}{0pt}%
\pgfpathmoveto{\pgfqpoint{2.358490in}{3.034408in}}%
\pgfpathcurveto{\pgfqpoint{2.366727in}{3.034408in}}{\pgfqpoint{2.374627in}{3.037680in}}{\pgfqpoint{2.380451in}{3.043504in}}%
\pgfpathcurveto{\pgfqpoint{2.386275in}{3.049328in}}{\pgfqpoint{2.389547in}{3.057228in}}{\pgfqpoint{2.389547in}{3.065464in}}%
\pgfpathcurveto{\pgfqpoint{2.389547in}{3.073701in}}{\pgfqpoint{2.386275in}{3.081601in}}{\pgfqpoint{2.380451in}{3.087425in}}%
\pgfpathcurveto{\pgfqpoint{2.374627in}{3.093249in}}{\pgfqpoint{2.366727in}{3.096521in}}{\pgfqpoint{2.358490in}{3.096521in}}%
\pgfpathcurveto{\pgfqpoint{2.350254in}{3.096521in}}{\pgfqpoint{2.342354in}{3.093249in}}{\pgfqpoint{2.336530in}{3.087425in}}%
\pgfpathcurveto{\pgfqpoint{2.330706in}{3.081601in}}{\pgfqpoint{2.327434in}{3.073701in}}{\pgfqpoint{2.327434in}{3.065464in}}%
\pgfpathcurveto{\pgfqpoint{2.327434in}{3.057228in}}{\pgfqpoint{2.330706in}{3.049328in}}{\pgfqpoint{2.336530in}{3.043504in}}%
\pgfpathcurveto{\pgfqpoint{2.342354in}{3.037680in}}{\pgfqpoint{2.350254in}{3.034408in}}{\pgfqpoint{2.358490in}{3.034408in}}%
\pgfpathclose%
\pgfusepath{stroke,fill}%
\end{pgfscope}%
\begin{pgfscope}%
\pgfpathrectangle{\pgfqpoint{0.100000in}{0.220728in}}{\pgfqpoint{3.696000in}{3.696000in}}%
\pgfusepath{clip}%
\pgfsetbuttcap%
\pgfsetroundjoin%
\definecolor{currentfill}{rgb}{0.121569,0.466667,0.705882}%
\pgfsetfillcolor{currentfill}%
\pgfsetfillopacity{0.454160}%
\pgfsetlinewidth{1.003750pt}%
\definecolor{currentstroke}{rgb}{0.121569,0.466667,0.705882}%
\pgfsetstrokecolor{currentstroke}%
\pgfsetstrokeopacity{0.454160}%
\pgfsetdash{}{0pt}%
\pgfpathmoveto{\pgfqpoint{2.361652in}{3.033809in}}%
\pgfpathcurveto{\pgfqpoint{2.369888in}{3.033809in}}{\pgfqpoint{2.377788in}{3.037081in}}{\pgfqpoint{2.383612in}{3.042905in}}%
\pgfpathcurveto{\pgfqpoint{2.389436in}{3.048729in}}{\pgfqpoint{2.392708in}{3.056629in}}{\pgfqpoint{2.392708in}{3.064865in}}%
\pgfpathcurveto{\pgfqpoint{2.392708in}{3.073101in}}{\pgfqpoint{2.389436in}{3.081001in}}{\pgfqpoint{2.383612in}{3.086825in}}%
\pgfpathcurveto{\pgfqpoint{2.377788in}{3.092649in}}{\pgfqpoint{2.369888in}{3.095922in}}{\pgfqpoint{2.361652in}{3.095922in}}%
\pgfpathcurveto{\pgfqpoint{2.353415in}{3.095922in}}{\pgfqpoint{2.345515in}{3.092649in}}{\pgfqpoint{2.339691in}{3.086825in}}%
\pgfpathcurveto{\pgfqpoint{2.333868in}{3.081001in}}{\pgfqpoint{2.330595in}{3.073101in}}{\pgfqpoint{2.330595in}{3.064865in}}%
\pgfpathcurveto{\pgfqpoint{2.330595in}{3.056629in}}{\pgfqpoint{2.333868in}{3.048729in}}{\pgfqpoint{2.339691in}{3.042905in}}%
\pgfpathcurveto{\pgfqpoint{2.345515in}{3.037081in}}{\pgfqpoint{2.353415in}{3.033809in}}{\pgfqpoint{2.361652in}{3.033809in}}%
\pgfpathclose%
\pgfusepath{stroke,fill}%
\end{pgfscope}%
\begin{pgfscope}%
\pgfpathrectangle{\pgfqpoint{0.100000in}{0.220728in}}{\pgfqpoint{3.696000in}{3.696000in}}%
\pgfusepath{clip}%
\pgfsetbuttcap%
\pgfsetroundjoin%
\definecolor{currentfill}{rgb}{0.121569,0.466667,0.705882}%
\pgfsetfillcolor{currentfill}%
\pgfsetfillopacity{0.454796}%
\pgfsetlinewidth{1.003750pt}%
\definecolor{currentstroke}{rgb}{0.121569,0.466667,0.705882}%
\pgfsetstrokecolor{currentstroke}%
\pgfsetstrokeopacity{0.454796}%
\pgfsetdash{}{0pt}%
\pgfpathmoveto{\pgfqpoint{1.163123in}{2.133466in}}%
\pgfpathcurveto{\pgfqpoint{1.171359in}{2.133466in}}{\pgfqpoint{1.179259in}{2.136738in}}{\pgfqpoint{1.185083in}{2.142562in}}%
\pgfpathcurveto{\pgfqpoint{1.190907in}{2.148386in}}{\pgfqpoint{1.194179in}{2.156286in}}{\pgfqpoint{1.194179in}{2.164522in}}%
\pgfpathcurveto{\pgfqpoint{1.194179in}{2.172759in}}{\pgfqpoint{1.190907in}{2.180659in}}{\pgfqpoint{1.185083in}{2.186483in}}%
\pgfpathcurveto{\pgfqpoint{1.179259in}{2.192307in}}{\pgfqpoint{1.171359in}{2.195579in}}{\pgfqpoint{1.163123in}{2.195579in}}%
\pgfpathcurveto{\pgfqpoint{1.154886in}{2.195579in}}{\pgfqpoint{1.146986in}{2.192307in}}{\pgfqpoint{1.141162in}{2.186483in}}%
\pgfpathcurveto{\pgfqpoint{1.135338in}{2.180659in}}{\pgfqpoint{1.132066in}{2.172759in}}{\pgfqpoint{1.132066in}{2.164522in}}%
\pgfpathcurveto{\pgfqpoint{1.132066in}{2.156286in}}{\pgfqpoint{1.135338in}{2.148386in}}{\pgfqpoint{1.141162in}{2.142562in}}%
\pgfpathcurveto{\pgfqpoint{1.146986in}{2.136738in}}{\pgfqpoint{1.154886in}{2.133466in}}{\pgfqpoint{1.163123in}{2.133466in}}%
\pgfpathclose%
\pgfusepath{stroke,fill}%
\end{pgfscope}%
\begin{pgfscope}%
\pgfpathrectangle{\pgfqpoint{0.100000in}{0.220728in}}{\pgfqpoint{3.696000in}{3.696000in}}%
\pgfusepath{clip}%
\pgfsetbuttcap%
\pgfsetroundjoin%
\definecolor{currentfill}{rgb}{0.121569,0.466667,0.705882}%
\pgfsetfillcolor{currentfill}%
\pgfsetfillopacity{0.455487}%
\pgfsetlinewidth{1.003750pt}%
\definecolor{currentstroke}{rgb}{0.121569,0.466667,0.705882}%
\pgfsetstrokecolor{currentstroke}%
\pgfsetstrokeopacity{0.455487}%
\pgfsetdash{}{0pt}%
\pgfpathmoveto{\pgfqpoint{2.367293in}{3.033466in}}%
\pgfpathcurveto{\pgfqpoint{2.375529in}{3.033466in}}{\pgfqpoint{2.383429in}{3.036739in}}{\pgfqpoint{2.389253in}{3.042563in}}%
\pgfpathcurveto{\pgfqpoint{2.395077in}{3.048386in}}{\pgfqpoint{2.398349in}{3.056287in}}{\pgfqpoint{2.398349in}{3.064523in}}%
\pgfpathcurveto{\pgfqpoint{2.398349in}{3.072759in}}{\pgfqpoint{2.395077in}{3.080659in}}{\pgfqpoint{2.389253in}{3.086483in}}%
\pgfpathcurveto{\pgfqpoint{2.383429in}{3.092307in}}{\pgfqpoint{2.375529in}{3.095579in}}{\pgfqpoint{2.367293in}{3.095579in}}%
\pgfpathcurveto{\pgfqpoint{2.359056in}{3.095579in}}{\pgfqpoint{2.351156in}{3.092307in}}{\pgfqpoint{2.345332in}{3.086483in}}%
\pgfpathcurveto{\pgfqpoint{2.339509in}{3.080659in}}{\pgfqpoint{2.336236in}{3.072759in}}{\pgfqpoint{2.336236in}{3.064523in}}%
\pgfpathcurveto{\pgfqpoint{2.336236in}{3.056287in}}{\pgfqpoint{2.339509in}{3.048386in}}{\pgfqpoint{2.345332in}{3.042563in}}%
\pgfpathcurveto{\pgfqpoint{2.351156in}{3.036739in}}{\pgfqpoint{2.359056in}{3.033466in}}{\pgfqpoint{2.367293in}{3.033466in}}%
\pgfpathclose%
\pgfusepath{stroke,fill}%
\end{pgfscope}%
\begin{pgfscope}%
\pgfpathrectangle{\pgfqpoint{0.100000in}{0.220728in}}{\pgfqpoint{3.696000in}{3.696000in}}%
\pgfusepath{clip}%
\pgfsetbuttcap%
\pgfsetroundjoin%
\definecolor{currentfill}{rgb}{0.121569,0.466667,0.705882}%
\pgfsetfillcolor{currentfill}%
\pgfsetfillopacity{0.457867}%
\pgfsetlinewidth{1.003750pt}%
\definecolor{currentstroke}{rgb}{0.121569,0.466667,0.705882}%
\pgfsetstrokecolor{currentstroke}%
\pgfsetstrokeopacity{0.457867}%
\pgfsetdash{}{0pt}%
\pgfpathmoveto{\pgfqpoint{2.376296in}{3.030831in}}%
\pgfpathcurveto{\pgfqpoint{2.384532in}{3.030831in}}{\pgfqpoint{2.392432in}{3.034103in}}{\pgfqpoint{2.398256in}{3.039927in}}%
\pgfpathcurveto{\pgfqpoint{2.404080in}{3.045751in}}{\pgfqpoint{2.407352in}{3.053651in}}{\pgfqpoint{2.407352in}{3.061887in}}%
\pgfpathcurveto{\pgfqpoint{2.407352in}{3.070124in}}{\pgfqpoint{2.404080in}{3.078024in}}{\pgfqpoint{2.398256in}{3.083848in}}%
\pgfpathcurveto{\pgfqpoint{2.392432in}{3.089672in}}{\pgfqpoint{2.384532in}{3.092944in}}{\pgfqpoint{2.376296in}{3.092944in}}%
\pgfpathcurveto{\pgfqpoint{2.368060in}{3.092944in}}{\pgfqpoint{2.360160in}{3.089672in}}{\pgfqpoint{2.354336in}{3.083848in}}%
\pgfpathcurveto{\pgfqpoint{2.348512in}{3.078024in}}{\pgfqpoint{2.345239in}{3.070124in}}{\pgfqpoint{2.345239in}{3.061887in}}%
\pgfpathcurveto{\pgfqpoint{2.345239in}{3.053651in}}{\pgfqpoint{2.348512in}{3.045751in}}{\pgfqpoint{2.354336in}{3.039927in}}%
\pgfpathcurveto{\pgfqpoint{2.360160in}{3.034103in}}{\pgfqpoint{2.368060in}{3.030831in}}{\pgfqpoint{2.376296in}{3.030831in}}%
\pgfpathclose%
\pgfusepath{stroke,fill}%
\end{pgfscope}%
\begin{pgfscope}%
\pgfpathrectangle{\pgfqpoint{0.100000in}{0.220728in}}{\pgfqpoint{3.696000in}{3.696000in}}%
\pgfusepath{clip}%
\pgfsetbuttcap%
\pgfsetroundjoin%
\definecolor{currentfill}{rgb}{0.121569,0.466667,0.705882}%
\pgfsetfillcolor{currentfill}%
\pgfsetfillopacity{0.459135}%
\pgfsetlinewidth{1.003750pt}%
\definecolor{currentstroke}{rgb}{0.121569,0.466667,0.705882}%
\pgfsetstrokecolor{currentstroke}%
\pgfsetstrokeopacity{0.459135}%
\pgfsetdash{}{0pt}%
\pgfpathmoveto{\pgfqpoint{1.153616in}{2.108705in}}%
\pgfpathcurveto{\pgfqpoint{1.161853in}{2.108705in}}{\pgfqpoint{1.169753in}{2.111978in}}{\pgfqpoint{1.175577in}{2.117802in}}%
\pgfpathcurveto{\pgfqpoint{1.181400in}{2.123626in}}{\pgfqpoint{1.184673in}{2.131526in}}{\pgfqpoint{1.184673in}{2.139762in}}%
\pgfpathcurveto{\pgfqpoint{1.184673in}{2.147998in}}{\pgfqpoint{1.181400in}{2.155898in}}{\pgfqpoint{1.175577in}{2.161722in}}%
\pgfpathcurveto{\pgfqpoint{1.169753in}{2.167546in}}{\pgfqpoint{1.161853in}{2.170818in}}{\pgfqpoint{1.153616in}{2.170818in}}%
\pgfpathcurveto{\pgfqpoint{1.145380in}{2.170818in}}{\pgfqpoint{1.137480in}{2.167546in}}{\pgfqpoint{1.131656in}{2.161722in}}%
\pgfpathcurveto{\pgfqpoint{1.125832in}{2.155898in}}{\pgfqpoint{1.122560in}{2.147998in}}{\pgfqpoint{1.122560in}{2.139762in}}%
\pgfpathcurveto{\pgfqpoint{1.122560in}{2.131526in}}{\pgfqpoint{1.125832in}{2.123626in}}{\pgfqpoint{1.131656in}{2.117802in}}%
\pgfpathcurveto{\pgfqpoint{1.137480in}{2.111978in}}{\pgfqpoint{1.145380in}{2.108705in}}{\pgfqpoint{1.153616in}{2.108705in}}%
\pgfpathclose%
\pgfusepath{stroke,fill}%
\end{pgfscope}%
\begin{pgfscope}%
\pgfpathrectangle{\pgfqpoint{0.100000in}{0.220728in}}{\pgfqpoint{3.696000in}{3.696000in}}%
\pgfusepath{clip}%
\pgfsetbuttcap%
\pgfsetroundjoin%
\definecolor{currentfill}{rgb}{0.121569,0.466667,0.705882}%
\pgfsetfillcolor{currentfill}%
\pgfsetfillopacity{0.461378}%
\pgfsetlinewidth{1.003750pt}%
\definecolor{currentstroke}{rgb}{0.121569,0.466667,0.705882}%
\pgfsetstrokecolor{currentstroke}%
\pgfsetstrokeopacity{0.461378}%
\pgfsetdash{}{0pt}%
\pgfpathmoveto{\pgfqpoint{2.386940in}{3.029327in}}%
\pgfpathcurveto{\pgfqpoint{2.395177in}{3.029327in}}{\pgfqpoint{2.403077in}{3.032600in}}{\pgfqpoint{2.408901in}{3.038423in}}%
\pgfpathcurveto{\pgfqpoint{2.414724in}{3.044247in}}{\pgfqpoint{2.417997in}{3.052147in}}{\pgfqpoint{2.417997in}{3.060384in}}%
\pgfpathcurveto{\pgfqpoint{2.417997in}{3.068620in}}{\pgfqpoint{2.414724in}{3.076520in}}{\pgfqpoint{2.408901in}{3.082344in}}%
\pgfpathcurveto{\pgfqpoint{2.403077in}{3.088168in}}{\pgfqpoint{2.395177in}{3.091440in}}{\pgfqpoint{2.386940in}{3.091440in}}%
\pgfpathcurveto{\pgfqpoint{2.378704in}{3.091440in}}{\pgfqpoint{2.370804in}{3.088168in}}{\pgfqpoint{2.364980in}{3.082344in}}%
\pgfpathcurveto{\pgfqpoint{2.359156in}{3.076520in}}{\pgfqpoint{2.355884in}{3.068620in}}{\pgfqpoint{2.355884in}{3.060384in}}%
\pgfpathcurveto{\pgfqpoint{2.355884in}{3.052147in}}{\pgfqpoint{2.359156in}{3.044247in}}{\pgfqpoint{2.364980in}{3.038423in}}%
\pgfpathcurveto{\pgfqpoint{2.370804in}{3.032600in}}{\pgfqpoint{2.378704in}{3.029327in}}{\pgfqpoint{2.386940in}{3.029327in}}%
\pgfpathclose%
\pgfusepath{stroke,fill}%
\end{pgfscope}%
\begin{pgfscope}%
\pgfpathrectangle{\pgfqpoint{0.100000in}{0.220728in}}{\pgfqpoint{3.696000in}{3.696000in}}%
\pgfusepath{clip}%
\pgfsetbuttcap%
\pgfsetroundjoin%
\definecolor{currentfill}{rgb}{0.121569,0.466667,0.705882}%
\pgfsetfillcolor{currentfill}%
\pgfsetfillopacity{0.465294}%
\pgfsetlinewidth{1.003750pt}%
\definecolor{currentstroke}{rgb}{0.121569,0.466667,0.705882}%
\pgfsetstrokecolor{currentstroke}%
\pgfsetstrokeopacity{0.465294}%
\pgfsetdash{}{0pt}%
\pgfpathmoveto{\pgfqpoint{2.402489in}{3.026637in}}%
\pgfpathcurveto{\pgfqpoint{2.410725in}{3.026637in}}{\pgfqpoint{2.418625in}{3.029909in}}{\pgfqpoint{2.424449in}{3.035733in}}%
\pgfpathcurveto{\pgfqpoint{2.430273in}{3.041557in}}{\pgfqpoint{2.433546in}{3.049457in}}{\pgfqpoint{2.433546in}{3.057693in}}%
\pgfpathcurveto{\pgfqpoint{2.433546in}{3.065930in}}{\pgfqpoint{2.430273in}{3.073830in}}{\pgfqpoint{2.424449in}{3.079654in}}%
\pgfpathcurveto{\pgfqpoint{2.418625in}{3.085477in}}{\pgfqpoint{2.410725in}{3.088750in}}{\pgfqpoint{2.402489in}{3.088750in}}%
\pgfpathcurveto{\pgfqpoint{2.394253in}{3.088750in}}{\pgfqpoint{2.386353in}{3.085477in}}{\pgfqpoint{2.380529in}{3.079654in}}%
\pgfpathcurveto{\pgfqpoint{2.374705in}{3.073830in}}{\pgfqpoint{2.371433in}{3.065930in}}{\pgfqpoint{2.371433in}{3.057693in}}%
\pgfpathcurveto{\pgfqpoint{2.371433in}{3.049457in}}{\pgfqpoint{2.374705in}{3.041557in}}{\pgfqpoint{2.380529in}{3.035733in}}%
\pgfpathcurveto{\pgfqpoint{2.386353in}{3.029909in}}{\pgfqpoint{2.394253in}{3.026637in}}{\pgfqpoint{2.402489in}{3.026637in}}%
\pgfpathclose%
\pgfusepath{stroke,fill}%
\end{pgfscope}%
\begin{pgfscope}%
\pgfpathrectangle{\pgfqpoint{0.100000in}{0.220728in}}{\pgfqpoint{3.696000in}{3.696000in}}%
\pgfusepath{clip}%
\pgfsetbuttcap%
\pgfsetroundjoin%
\definecolor{currentfill}{rgb}{0.121569,0.466667,0.705882}%
\pgfsetfillcolor{currentfill}%
\pgfsetfillopacity{0.465689}%
\pgfsetlinewidth{1.003750pt}%
\definecolor{currentstroke}{rgb}{0.121569,0.466667,0.705882}%
\pgfsetstrokecolor{currentstroke}%
\pgfsetstrokeopacity{0.465689}%
\pgfsetdash{}{0pt}%
\pgfpathmoveto{\pgfqpoint{1.127028in}{2.066116in}}%
\pgfpathcurveto{\pgfqpoint{1.135264in}{2.066116in}}{\pgfqpoint{1.143164in}{2.069389in}}{\pgfqpoint{1.148988in}{2.075212in}}%
\pgfpathcurveto{\pgfqpoint{1.154812in}{2.081036in}}{\pgfqpoint{1.158084in}{2.088936in}}{\pgfqpoint{1.158084in}{2.097173in}}%
\pgfpathcurveto{\pgfqpoint{1.158084in}{2.105409in}}{\pgfqpoint{1.154812in}{2.113309in}}{\pgfqpoint{1.148988in}{2.119133in}}%
\pgfpathcurveto{\pgfqpoint{1.143164in}{2.124957in}}{\pgfqpoint{1.135264in}{2.128229in}}{\pgfqpoint{1.127028in}{2.128229in}}%
\pgfpathcurveto{\pgfqpoint{1.118791in}{2.128229in}}{\pgfqpoint{1.110891in}{2.124957in}}{\pgfqpoint{1.105067in}{2.119133in}}%
\pgfpathcurveto{\pgfqpoint{1.099243in}{2.113309in}}{\pgfqpoint{1.095971in}{2.105409in}}{\pgfqpoint{1.095971in}{2.097173in}}%
\pgfpathcurveto{\pgfqpoint{1.095971in}{2.088936in}}{\pgfqpoint{1.099243in}{2.081036in}}{\pgfqpoint{1.105067in}{2.075212in}}%
\pgfpathcurveto{\pgfqpoint{1.110891in}{2.069389in}}{\pgfqpoint{1.118791in}{2.066116in}}{\pgfqpoint{1.127028in}{2.066116in}}%
\pgfpathclose%
\pgfusepath{stroke,fill}%
\end{pgfscope}%
\begin{pgfscope}%
\pgfpathrectangle{\pgfqpoint{0.100000in}{0.220728in}}{\pgfqpoint{3.696000in}{3.696000in}}%
\pgfusepath{clip}%
\pgfsetbuttcap%
\pgfsetroundjoin%
\definecolor{currentfill}{rgb}{0.121569,0.466667,0.705882}%
\pgfsetfillcolor{currentfill}%
\pgfsetfillopacity{0.467730}%
\pgfsetlinewidth{1.003750pt}%
\definecolor{currentstroke}{rgb}{0.121569,0.466667,0.705882}%
\pgfsetstrokecolor{currentstroke}%
\pgfsetstrokeopacity{0.467730}%
\pgfsetdash{}{0pt}%
\pgfpathmoveto{\pgfqpoint{2.410538in}{3.025258in}}%
\pgfpathcurveto{\pgfqpoint{2.418774in}{3.025258in}}{\pgfqpoint{2.426674in}{3.028530in}}{\pgfqpoint{2.432498in}{3.034354in}}%
\pgfpathcurveto{\pgfqpoint{2.438322in}{3.040178in}}{\pgfqpoint{2.441594in}{3.048078in}}{\pgfqpoint{2.441594in}{3.056315in}}%
\pgfpathcurveto{\pgfqpoint{2.441594in}{3.064551in}}{\pgfqpoint{2.438322in}{3.072451in}}{\pgfqpoint{2.432498in}{3.078275in}}%
\pgfpathcurveto{\pgfqpoint{2.426674in}{3.084099in}}{\pgfqpoint{2.418774in}{3.087371in}}{\pgfqpoint{2.410538in}{3.087371in}}%
\pgfpathcurveto{\pgfqpoint{2.402301in}{3.087371in}}{\pgfqpoint{2.394401in}{3.084099in}}{\pgfqpoint{2.388577in}{3.078275in}}%
\pgfpathcurveto{\pgfqpoint{2.382753in}{3.072451in}}{\pgfqpoint{2.379481in}{3.064551in}}{\pgfqpoint{2.379481in}{3.056315in}}%
\pgfpathcurveto{\pgfqpoint{2.379481in}{3.048078in}}{\pgfqpoint{2.382753in}{3.040178in}}{\pgfqpoint{2.388577in}{3.034354in}}%
\pgfpathcurveto{\pgfqpoint{2.394401in}{3.028530in}}{\pgfqpoint{2.402301in}{3.025258in}}{\pgfqpoint{2.410538in}{3.025258in}}%
\pgfpathclose%
\pgfusepath{stroke,fill}%
\end{pgfscope}%
\begin{pgfscope}%
\pgfpathrectangle{\pgfqpoint{0.100000in}{0.220728in}}{\pgfqpoint{3.696000in}{3.696000in}}%
\pgfusepath{clip}%
\pgfsetbuttcap%
\pgfsetroundjoin%
\definecolor{currentfill}{rgb}{0.121569,0.466667,0.705882}%
\pgfsetfillcolor{currentfill}%
\pgfsetfillopacity{0.470311}%
\pgfsetlinewidth{1.003750pt}%
\definecolor{currentstroke}{rgb}{0.121569,0.466667,0.705882}%
\pgfsetstrokecolor{currentstroke}%
\pgfsetstrokeopacity{0.470311}%
\pgfsetdash{}{0pt}%
\pgfpathmoveto{\pgfqpoint{2.426416in}{3.023336in}}%
\pgfpathcurveto{\pgfqpoint{2.434652in}{3.023336in}}{\pgfqpoint{2.442552in}{3.026609in}}{\pgfqpoint{2.448376in}{3.032433in}}%
\pgfpathcurveto{\pgfqpoint{2.454200in}{3.038256in}}{\pgfqpoint{2.457472in}{3.046156in}}{\pgfqpoint{2.457472in}{3.054393in}}%
\pgfpathcurveto{\pgfqpoint{2.457472in}{3.062629in}}{\pgfqpoint{2.454200in}{3.070529in}}{\pgfqpoint{2.448376in}{3.076353in}}%
\pgfpathcurveto{\pgfqpoint{2.442552in}{3.082177in}}{\pgfqpoint{2.434652in}{3.085449in}}{\pgfqpoint{2.426416in}{3.085449in}}%
\pgfpathcurveto{\pgfqpoint{2.418179in}{3.085449in}}{\pgfqpoint{2.410279in}{3.082177in}}{\pgfqpoint{2.404455in}{3.076353in}}%
\pgfpathcurveto{\pgfqpoint{2.398631in}{3.070529in}}{\pgfqpoint{2.395359in}{3.062629in}}{\pgfqpoint{2.395359in}{3.054393in}}%
\pgfpathcurveto{\pgfqpoint{2.395359in}{3.046156in}}{\pgfqpoint{2.398631in}{3.038256in}}{\pgfqpoint{2.404455in}{3.032433in}}%
\pgfpathcurveto{\pgfqpoint{2.410279in}{3.026609in}}{\pgfqpoint{2.418179in}{3.023336in}}{\pgfqpoint{2.426416in}{3.023336in}}%
\pgfpathclose%
\pgfusepath{stroke,fill}%
\end{pgfscope}%
\begin{pgfscope}%
\pgfpathrectangle{\pgfqpoint{0.100000in}{0.220728in}}{\pgfqpoint{3.696000in}{3.696000in}}%
\pgfusepath{clip}%
\pgfsetbuttcap%
\pgfsetroundjoin%
\definecolor{currentfill}{rgb}{0.121569,0.466667,0.705882}%
\pgfsetfillcolor{currentfill}%
\pgfsetfillopacity{0.472008}%
\pgfsetlinewidth{1.003750pt}%
\definecolor{currentstroke}{rgb}{0.121569,0.466667,0.705882}%
\pgfsetstrokecolor{currentstroke}%
\pgfsetstrokeopacity{0.472008}%
\pgfsetdash{}{0pt}%
\pgfpathmoveto{\pgfqpoint{1.118449in}{2.025442in}}%
\pgfpathcurveto{\pgfqpoint{1.126685in}{2.025442in}}{\pgfqpoint{1.134585in}{2.028714in}}{\pgfqpoint{1.140409in}{2.034538in}}%
\pgfpathcurveto{\pgfqpoint{1.146233in}{2.040362in}}{\pgfqpoint{1.149506in}{2.048262in}}{\pgfqpoint{1.149506in}{2.056499in}}%
\pgfpathcurveto{\pgfqpoint{1.149506in}{2.064735in}}{\pgfqpoint{1.146233in}{2.072635in}}{\pgfqpoint{1.140409in}{2.078459in}}%
\pgfpathcurveto{\pgfqpoint{1.134585in}{2.084283in}}{\pgfqpoint{1.126685in}{2.087555in}}{\pgfqpoint{1.118449in}{2.087555in}}%
\pgfpathcurveto{\pgfqpoint{1.110213in}{2.087555in}}{\pgfqpoint{1.102313in}{2.084283in}}{\pgfqpoint{1.096489in}{2.078459in}}%
\pgfpathcurveto{\pgfqpoint{1.090665in}{2.072635in}}{\pgfqpoint{1.087393in}{2.064735in}}{\pgfqpoint{1.087393in}{2.056499in}}%
\pgfpathcurveto{\pgfqpoint{1.087393in}{2.048262in}}{\pgfqpoint{1.090665in}{2.040362in}}{\pgfqpoint{1.096489in}{2.034538in}}%
\pgfpathcurveto{\pgfqpoint{1.102313in}{2.028714in}}{\pgfqpoint{1.110213in}{2.025442in}}{\pgfqpoint{1.118449in}{2.025442in}}%
\pgfpathclose%
\pgfusepath{stroke,fill}%
\end{pgfscope}%
\begin{pgfscope}%
\pgfpathrectangle{\pgfqpoint{0.100000in}{0.220728in}}{\pgfqpoint{3.696000in}{3.696000in}}%
\pgfusepath{clip}%
\pgfsetbuttcap%
\pgfsetroundjoin%
\definecolor{currentfill}{rgb}{0.121569,0.466667,0.705882}%
\pgfsetfillcolor{currentfill}%
\pgfsetfillopacity{0.475521}%
\pgfsetlinewidth{1.003750pt}%
\definecolor{currentstroke}{rgb}{0.121569,0.466667,0.705882}%
\pgfsetstrokecolor{currentstroke}%
\pgfsetstrokeopacity{0.475521}%
\pgfsetdash{}{0pt}%
\pgfpathmoveto{\pgfqpoint{1.102289in}{1.998887in}}%
\pgfpathcurveto{\pgfqpoint{1.110526in}{1.998887in}}{\pgfqpoint{1.118426in}{2.002159in}}{\pgfqpoint{1.124250in}{2.007983in}}%
\pgfpathcurveto{\pgfqpoint{1.130073in}{2.013807in}}{\pgfqpoint{1.133346in}{2.021707in}}{\pgfqpoint{1.133346in}{2.029943in}}%
\pgfpathcurveto{\pgfqpoint{1.133346in}{2.038180in}}{\pgfqpoint{1.130073in}{2.046080in}}{\pgfqpoint{1.124250in}{2.051904in}}%
\pgfpathcurveto{\pgfqpoint{1.118426in}{2.057727in}}{\pgfqpoint{1.110526in}{2.061000in}}{\pgfqpoint{1.102289in}{2.061000in}}%
\pgfpathcurveto{\pgfqpoint{1.094053in}{2.061000in}}{\pgfqpoint{1.086153in}{2.057727in}}{\pgfqpoint{1.080329in}{2.051904in}}%
\pgfpathcurveto{\pgfqpoint{1.074505in}{2.046080in}}{\pgfqpoint{1.071233in}{2.038180in}}{\pgfqpoint{1.071233in}{2.029943in}}%
\pgfpathcurveto{\pgfqpoint{1.071233in}{2.021707in}}{\pgfqpoint{1.074505in}{2.013807in}}{\pgfqpoint{1.080329in}{2.007983in}}%
\pgfpathcurveto{\pgfqpoint{1.086153in}{2.002159in}}{\pgfqpoint{1.094053in}{1.998887in}}{\pgfqpoint{1.102289in}{1.998887in}}%
\pgfpathclose%
\pgfusepath{stroke,fill}%
\end{pgfscope}%
\begin{pgfscope}%
\pgfpathrectangle{\pgfqpoint{0.100000in}{0.220728in}}{\pgfqpoint{3.696000in}{3.696000in}}%
\pgfusepath{clip}%
\pgfsetbuttcap%
\pgfsetroundjoin%
\definecolor{currentfill}{rgb}{0.121569,0.466667,0.705882}%
\pgfsetfillcolor{currentfill}%
\pgfsetfillopacity{0.475659}%
\pgfsetlinewidth{1.003750pt}%
\definecolor{currentstroke}{rgb}{0.121569,0.466667,0.705882}%
\pgfsetstrokecolor{currentstroke}%
\pgfsetstrokeopacity{0.475659}%
\pgfsetdash{}{0pt}%
\pgfpathmoveto{\pgfqpoint{2.441575in}{3.020488in}}%
\pgfpathcurveto{\pgfqpoint{2.449811in}{3.020488in}}{\pgfqpoint{2.457711in}{3.023760in}}{\pgfqpoint{2.463535in}{3.029584in}}%
\pgfpathcurveto{\pgfqpoint{2.469359in}{3.035408in}}{\pgfqpoint{2.472631in}{3.043308in}}{\pgfqpoint{2.472631in}{3.051544in}}%
\pgfpathcurveto{\pgfqpoint{2.472631in}{3.059780in}}{\pgfqpoint{2.469359in}{3.067680in}}{\pgfqpoint{2.463535in}{3.073504in}}%
\pgfpathcurveto{\pgfqpoint{2.457711in}{3.079328in}}{\pgfqpoint{2.449811in}{3.082601in}}{\pgfqpoint{2.441575in}{3.082601in}}%
\pgfpathcurveto{\pgfqpoint{2.433338in}{3.082601in}}{\pgfqpoint{2.425438in}{3.079328in}}{\pgfqpoint{2.419615in}{3.073504in}}%
\pgfpathcurveto{\pgfqpoint{2.413791in}{3.067680in}}{\pgfqpoint{2.410518in}{3.059780in}}{\pgfqpoint{2.410518in}{3.051544in}}%
\pgfpathcurveto{\pgfqpoint{2.410518in}{3.043308in}}{\pgfqpoint{2.413791in}{3.035408in}}{\pgfqpoint{2.419615in}{3.029584in}}%
\pgfpathcurveto{\pgfqpoint{2.425438in}{3.023760in}}{\pgfqpoint{2.433338in}{3.020488in}}{\pgfqpoint{2.441575in}{3.020488in}}%
\pgfpathclose%
\pgfusepath{stroke,fill}%
\end{pgfscope}%
\begin{pgfscope}%
\pgfpathrectangle{\pgfqpoint{0.100000in}{0.220728in}}{\pgfqpoint{3.696000in}{3.696000in}}%
\pgfusepath{clip}%
\pgfsetbuttcap%
\pgfsetroundjoin%
\definecolor{currentfill}{rgb}{0.121569,0.466667,0.705882}%
\pgfsetfillcolor{currentfill}%
\pgfsetfillopacity{0.478453}%
\pgfsetlinewidth{1.003750pt}%
\definecolor{currentstroke}{rgb}{0.121569,0.466667,0.705882}%
\pgfsetstrokecolor{currentstroke}%
\pgfsetstrokeopacity{0.478453}%
\pgfsetdash{}{0pt}%
\pgfpathmoveto{\pgfqpoint{1.095742in}{1.982057in}}%
\pgfpathcurveto{\pgfqpoint{1.103978in}{1.982057in}}{\pgfqpoint{1.111878in}{1.985329in}}{\pgfqpoint{1.117702in}{1.991153in}}%
\pgfpathcurveto{\pgfqpoint{1.123526in}{1.996977in}}{\pgfqpoint{1.126798in}{2.004877in}}{\pgfqpoint{1.126798in}{2.013114in}}%
\pgfpathcurveto{\pgfqpoint{1.126798in}{2.021350in}}{\pgfqpoint{1.123526in}{2.029250in}}{\pgfqpoint{1.117702in}{2.035074in}}%
\pgfpathcurveto{\pgfqpoint{1.111878in}{2.040898in}}{\pgfqpoint{1.103978in}{2.044170in}}{\pgfqpoint{1.095742in}{2.044170in}}%
\pgfpathcurveto{\pgfqpoint{1.087505in}{2.044170in}}{\pgfqpoint{1.079605in}{2.040898in}}{\pgfqpoint{1.073782in}{2.035074in}}%
\pgfpathcurveto{\pgfqpoint{1.067958in}{2.029250in}}{\pgfqpoint{1.064685in}{2.021350in}}{\pgfqpoint{1.064685in}{2.013114in}}%
\pgfpathcurveto{\pgfqpoint{1.064685in}{2.004877in}}{\pgfqpoint{1.067958in}{1.996977in}}{\pgfqpoint{1.073782in}{1.991153in}}%
\pgfpathcurveto{\pgfqpoint{1.079605in}{1.985329in}}{\pgfqpoint{1.087505in}{1.982057in}}{\pgfqpoint{1.095742in}{1.982057in}}%
\pgfpathclose%
\pgfusepath{stroke,fill}%
\end{pgfscope}%
\begin{pgfscope}%
\pgfpathrectangle{\pgfqpoint{0.100000in}{0.220728in}}{\pgfqpoint{3.696000in}{3.696000in}}%
\pgfusepath{clip}%
\pgfsetbuttcap%
\pgfsetroundjoin%
\definecolor{currentfill}{rgb}{0.121569,0.466667,0.705882}%
\pgfsetfillcolor{currentfill}%
\pgfsetfillopacity{0.479584}%
\pgfsetlinewidth{1.003750pt}%
\definecolor{currentstroke}{rgb}{0.121569,0.466667,0.705882}%
\pgfsetstrokecolor{currentstroke}%
\pgfsetstrokeopacity{0.479584}%
\pgfsetdash{}{0pt}%
\pgfpathmoveto{\pgfqpoint{1.091490in}{1.974121in}}%
\pgfpathcurveto{\pgfqpoint{1.099726in}{1.974121in}}{\pgfqpoint{1.107627in}{1.977394in}}{\pgfqpoint{1.113450in}{1.983217in}}%
\pgfpathcurveto{\pgfqpoint{1.119274in}{1.989041in}}{\pgfqpoint{1.122547in}{1.996941in}}{\pgfqpoint{1.122547in}{2.005178in}}%
\pgfpathcurveto{\pgfqpoint{1.122547in}{2.013414in}}{\pgfqpoint{1.119274in}{2.021314in}}{\pgfqpoint{1.113450in}{2.027138in}}%
\pgfpathcurveto{\pgfqpoint{1.107627in}{2.032962in}}{\pgfqpoint{1.099726in}{2.036234in}}{\pgfqpoint{1.091490in}{2.036234in}}%
\pgfpathcurveto{\pgfqpoint{1.083254in}{2.036234in}}{\pgfqpoint{1.075354in}{2.032962in}}{\pgfqpoint{1.069530in}{2.027138in}}%
\pgfpathcurveto{\pgfqpoint{1.063706in}{2.021314in}}{\pgfqpoint{1.060434in}{2.013414in}}{\pgfqpoint{1.060434in}{2.005178in}}%
\pgfpathcurveto{\pgfqpoint{1.060434in}{1.996941in}}{\pgfqpoint{1.063706in}{1.989041in}}{\pgfqpoint{1.069530in}{1.983217in}}%
\pgfpathcurveto{\pgfqpoint{1.075354in}{1.977394in}}{\pgfqpoint{1.083254in}{1.974121in}}{\pgfqpoint{1.091490in}{1.974121in}}%
\pgfpathclose%
\pgfusepath{stroke,fill}%
\end{pgfscope}%
\begin{pgfscope}%
\pgfpathrectangle{\pgfqpoint{0.100000in}{0.220728in}}{\pgfqpoint{3.696000in}{3.696000in}}%
\pgfusepath{clip}%
\pgfsetbuttcap%
\pgfsetroundjoin%
\definecolor{currentfill}{rgb}{0.121569,0.466667,0.705882}%
\pgfsetfillcolor{currentfill}%
\pgfsetfillopacity{0.480076}%
\pgfsetlinewidth{1.003750pt}%
\definecolor{currentstroke}{rgb}{0.121569,0.466667,0.705882}%
\pgfsetstrokecolor{currentstroke}%
\pgfsetstrokeopacity{0.480076}%
\pgfsetdash{}{0pt}%
\pgfpathmoveto{\pgfqpoint{1.089970in}{1.970954in}}%
\pgfpathcurveto{\pgfqpoint{1.098206in}{1.970954in}}{\pgfqpoint{1.106106in}{1.974226in}}{\pgfqpoint{1.111930in}{1.980050in}}%
\pgfpathcurveto{\pgfqpoint{1.117754in}{1.985874in}}{\pgfqpoint{1.121027in}{1.993774in}}{\pgfqpoint{1.121027in}{2.002011in}}%
\pgfpathcurveto{\pgfqpoint{1.121027in}{2.010247in}}{\pgfqpoint{1.117754in}{2.018147in}}{\pgfqpoint{1.111930in}{2.023971in}}%
\pgfpathcurveto{\pgfqpoint{1.106106in}{2.029795in}}{\pgfqpoint{1.098206in}{2.033067in}}{\pgfqpoint{1.089970in}{2.033067in}}%
\pgfpathcurveto{\pgfqpoint{1.081734in}{2.033067in}}{\pgfqpoint{1.073834in}{2.029795in}}{\pgfqpoint{1.068010in}{2.023971in}}%
\pgfpathcurveto{\pgfqpoint{1.062186in}{2.018147in}}{\pgfqpoint{1.058914in}{2.010247in}}{\pgfqpoint{1.058914in}{2.002011in}}%
\pgfpathcurveto{\pgfqpoint{1.058914in}{1.993774in}}{\pgfqpoint{1.062186in}{1.985874in}}{\pgfqpoint{1.068010in}{1.980050in}}%
\pgfpathcurveto{\pgfqpoint{1.073834in}{1.974226in}}{\pgfqpoint{1.081734in}{1.970954in}}{\pgfqpoint{1.089970in}{1.970954in}}%
\pgfpathclose%
\pgfusepath{stroke,fill}%
\end{pgfscope}%
\begin{pgfscope}%
\pgfpathrectangle{\pgfqpoint{0.100000in}{0.220728in}}{\pgfqpoint{3.696000in}{3.696000in}}%
\pgfusepath{clip}%
\pgfsetbuttcap%
\pgfsetroundjoin%
\definecolor{currentfill}{rgb}{0.121569,0.466667,0.705882}%
\pgfsetfillcolor{currentfill}%
\pgfsetfillopacity{0.481014}%
\pgfsetlinewidth{1.003750pt}%
\definecolor{currentstroke}{rgb}{0.121569,0.466667,0.705882}%
\pgfsetstrokecolor{currentstroke}%
\pgfsetstrokeopacity{0.481014}%
\pgfsetdash{}{0pt}%
\pgfpathmoveto{\pgfqpoint{1.087367in}{1.965213in}}%
\pgfpathcurveto{\pgfqpoint{1.095603in}{1.965213in}}{\pgfqpoint{1.103503in}{1.968486in}}{\pgfqpoint{1.109327in}{1.974310in}}%
\pgfpathcurveto{\pgfqpoint{1.115151in}{1.980133in}}{\pgfqpoint{1.118423in}{1.988034in}}{\pgfqpoint{1.118423in}{1.996270in}}%
\pgfpathcurveto{\pgfqpoint{1.118423in}{2.004506in}}{\pgfqpoint{1.115151in}{2.012406in}}{\pgfqpoint{1.109327in}{2.018230in}}%
\pgfpathcurveto{\pgfqpoint{1.103503in}{2.024054in}}{\pgfqpoint{1.095603in}{2.027326in}}{\pgfqpoint{1.087367in}{2.027326in}}%
\pgfpathcurveto{\pgfqpoint{1.079131in}{2.027326in}}{\pgfqpoint{1.071231in}{2.024054in}}{\pgfqpoint{1.065407in}{2.018230in}}%
\pgfpathcurveto{\pgfqpoint{1.059583in}{2.012406in}}{\pgfqpoint{1.056310in}{2.004506in}}{\pgfqpoint{1.056310in}{1.996270in}}%
\pgfpathcurveto{\pgfqpoint{1.056310in}{1.988034in}}{\pgfqpoint{1.059583in}{1.980133in}}{\pgfqpoint{1.065407in}{1.974310in}}%
\pgfpathcurveto{\pgfqpoint{1.071231in}{1.968486in}}{\pgfqpoint{1.079131in}{1.965213in}}{\pgfqpoint{1.087367in}{1.965213in}}%
\pgfpathclose%
\pgfusepath{stroke,fill}%
\end{pgfscope}%
\begin{pgfscope}%
\pgfpathrectangle{\pgfqpoint{0.100000in}{0.220728in}}{\pgfqpoint{3.696000in}{3.696000in}}%
\pgfusepath{clip}%
\pgfsetbuttcap%
\pgfsetroundjoin%
\definecolor{currentfill}{rgb}{0.121569,0.466667,0.705882}%
\pgfsetfillcolor{currentfill}%
\pgfsetfillopacity{0.482064}%
\pgfsetlinewidth{1.003750pt}%
\definecolor{currentstroke}{rgb}{0.121569,0.466667,0.705882}%
\pgfsetstrokecolor{currentstroke}%
\pgfsetstrokeopacity{0.482064}%
\pgfsetdash{}{0pt}%
\pgfpathmoveto{\pgfqpoint{2.463963in}{3.014583in}}%
\pgfpathcurveto{\pgfqpoint{2.472199in}{3.014583in}}{\pgfqpoint{2.480099in}{3.017856in}}{\pgfqpoint{2.485923in}{3.023679in}}%
\pgfpathcurveto{\pgfqpoint{2.491747in}{3.029503in}}{\pgfqpoint{2.495020in}{3.037403in}}{\pgfqpoint{2.495020in}{3.045640in}}%
\pgfpathcurveto{\pgfqpoint{2.495020in}{3.053876in}}{\pgfqpoint{2.491747in}{3.061776in}}{\pgfqpoint{2.485923in}{3.067600in}}%
\pgfpathcurveto{\pgfqpoint{2.480099in}{3.073424in}}{\pgfqpoint{2.472199in}{3.076696in}}{\pgfqpoint{2.463963in}{3.076696in}}%
\pgfpathcurveto{\pgfqpoint{2.455727in}{3.076696in}}{\pgfqpoint{2.447827in}{3.073424in}}{\pgfqpoint{2.442003in}{3.067600in}}%
\pgfpathcurveto{\pgfqpoint{2.436179in}{3.061776in}}{\pgfqpoint{2.432907in}{3.053876in}}{\pgfqpoint{2.432907in}{3.045640in}}%
\pgfpathcurveto{\pgfqpoint{2.432907in}{3.037403in}}{\pgfqpoint{2.436179in}{3.029503in}}{\pgfqpoint{2.442003in}{3.023679in}}%
\pgfpathcurveto{\pgfqpoint{2.447827in}{3.017856in}}{\pgfqpoint{2.455727in}{3.014583in}}{\pgfqpoint{2.463963in}{3.014583in}}%
\pgfpathclose%
\pgfusepath{stroke,fill}%
\end{pgfscope}%
\begin{pgfscope}%
\pgfpathrectangle{\pgfqpoint{0.100000in}{0.220728in}}{\pgfqpoint{3.696000in}{3.696000in}}%
\pgfusepath{clip}%
\pgfsetbuttcap%
\pgfsetroundjoin%
\definecolor{currentfill}{rgb}{0.121569,0.466667,0.705882}%
\pgfsetfillcolor{currentfill}%
\pgfsetfillopacity{0.482638}%
\pgfsetlinewidth{1.003750pt}%
\definecolor{currentstroke}{rgb}{0.121569,0.466667,0.705882}%
\pgfsetstrokecolor{currentstroke}%
\pgfsetstrokeopacity{0.482638}%
\pgfsetdash{}{0pt}%
\pgfpathmoveto{\pgfqpoint{1.082495in}{1.954528in}}%
\pgfpathcurveto{\pgfqpoint{1.090731in}{1.954528in}}{\pgfqpoint{1.098631in}{1.957800in}}{\pgfqpoint{1.104455in}{1.963624in}}%
\pgfpathcurveto{\pgfqpoint{1.110279in}{1.969448in}}{\pgfqpoint{1.113551in}{1.977348in}}{\pgfqpoint{1.113551in}{1.985584in}}%
\pgfpathcurveto{\pgfqpoint{1.113551in}{1.993820in}}{\pgfqpoint{1.110279in}{2.001720in}}{\pgfqpoint{1.104455in}{2.007544in}}%
\pgfpathcurveto{\pgfqpoint{1.098631in}{2.013368in}}{\pgfqpoint{1.090731in}{2.016641in}}{\pgfqpoint{1.082495in}{2.016641in}}%
\pgfpathcurveto{\pgfqpoint{1.074259in}{2.016641in}}{\pgfqpoint{1.066359in}{2.013368in}}{\pgfqpoint{1.060535in}{2.007544in}}%
\pgfpathcurveto{\pgfqpoint{1.054711in}{2.001720in}}{\pgfqpoint{1.051438in}{1.993820in}}{\pgfqpoint{1.051438in}{1.985584in}}%
\pgfpathcurveto{\pgfqpoint{1.051438in}{1.977348in}}{\pgfqpoint{1.054711in}{1.969448in}}{\pgfqpoint{1.060535in}{1.963624in}}%
\pgfpathcurveto{\pgfqpoint{1.066359in}{1.957800in}}{\pgfqpoint{1.074259in}{1.954528in}}{\pgfqpoint{1.082495in}{1.954528in}}%
\pgfpathclose%
\pgfusepath{stroke,fill}%
\end{pgfscope}%
\begin{pgfscope}%
\pgfpathrectangle{\pgfqpoint{0.100000in}{0.220728in}}{\pgfqpoint{3.696000in}{3.696000in}}%
\pgfusepath{clip}%
\pgfsetbuttcap%
\pgfsetroundjoin%
\definecolor{currentfill}{rgb}{0.121569,0.466667,0.705882}%
\pgfsetfillcolor{currentfill}%
\pgfsetfillopacity{0.485469}%
\pgfsetlinewidth{1.003750pt}%
\definecolor{currentstroke}{rgb}{0.121569,0.466667,0.705882}%
\pgfsetstrokecolor{currentstroke}%
\pgfsetstrokeopacity{0.485469}%
\pgfsetdash{}{0pt}%
\pgfpathmoveto{\pgfqpoint{2.476841in}{3.012081in}}%
\pgfpathcurveto{\pgfqpoint{2.485077in}{3.012081in}}{\pgfqpoint{2.492977in}{3.015353in}}{\pgfqpoint{2.498801in}{3.021177in}}%
\pgfpathcurveto{\pgfqpoint{2.504625in}{3.027001in}}{\pgfqpoint{2.507897in}{3.034901in}}{\pgfqpoint{2.507897in}{3.043137in}}%
\pgfpathcurveto{\pgfqpoint{2.507897in}{3.051373in}}{\pgfqpoint{2.504625in}{3.059273in}}{\pgfqpoint{2.498801in}{3.065097in}}%
\pgfpathcurveto{\pgfqpoint{2.492977in}{3.070921in}}{\pgfqpoint{2.485077in}{3.074194in}}{\pgfqpoint{2.476841in}{3.074194in}}%
\pgfpathcurveto{\pgfqpoint{2.468604in}{3.074194in}}{\pgfqpoint{2.460704in}{3.070921in}}{\pgfqpoint{2.454880in}{3.065097in}}%
\pgfpathcurveto{\pgfqpoint{2.449056in}{3.059273in}}{\pgfqpoint{2.445784in}{3.051373in}}{\pgfqpoint{2.445784in}{3.043137in}}%
\pgfpathcurveto{\pgfqpoint{2.445784in}{3.034901in}}{\pgfqpoint{2.449056in}{3.027001in}}{\pgfqpoint{2.454880in}{3.021177in}}%
\pgfpathcurveto{\pgfqpoint{2.460704in}{3.015353in}}{\pgfqpoint{2.468604in}{3.012081in}}{\pgfqpoint{2.476841in}{3.012081in}}%
\pgfpathclose%
\pgfusepath{stroke,fill}%
\end{pgfscope}%
\begin{pgfscope}%
\pgfpathrectangle{\pgfqpoint{0.100000in}{0.220728in}}{\pgfqpoint{3.696000in}{3.696000in}}%
\pgfusepath{clip}%
\pgfsetbuttcap%
\pgfsetroundjoin%
\definecolor{currentfill}{rgb}{0.121569,0.466667,0.705882}%
\pgfsetfillcolor{currentfill}%
\pgfsetfillopacity{0.485800}%
\pgfsetlinewidth{1.003750pt}%
\definecolor{currentstroke}{rgb}{0.121569,0.466667,0.705882}%
\pgfsetstrokecolor{currentstroke}%
\pgfsetstrokeopacity{0.485800}%
\pgfsetdash{}{0pt}%
\pgfpathmoveto{\pgfqpoint{1.073756in}{1.935989in}}%
\pgfpathcurveto{\pgfqpoint{1.081992in}{1.935989in}}{\pgfqpoint{1.089892in}{1.939261in}}{\pgfqpoint{1.095716in}{1.945085in}}%
\pgfpathcurveto{\pgfqpoint{1.101540in}{1.950909in}}{\pgfqpoint{1.104812in}{1.958809in}}{\pgfqpoint{1.104812in}{1.967045in}}%
\pgfpathcurveto{\pgfqpoint{1.104812in}{1.975282in}}{\pgfqpoint{1.101540in}{1.983182in}}{\pgfqpoint{1.095716in}{1.989006in}}%
\pgfpathcurveto{\pgfqpoint{1.089892in}{1.994830in}}{\pgfqpoint{1.081992in}{1.998102in}}{\pgfqpoint{1.073756in}{1.998102in}}%
\pgfpathcurveto{\pgfqpoint{1.065520in}{1.998102in}}{\pgfqpoint{1.057620in}{1.994830in}}{\pgfqpoint{1.051796in}{1.989006in}}%
\pgfpathcurveto{\pgfqpoint{1.045972in}{1.983182in}}{\pgfqpoint{1.042699in}{1.975282in}}{\pgfqpoint{1.042699in}{1.967045in}}%
\pgfpathcurveto{\pgfqpoint{1.042699in}{1.958809in}}{\pgfqpoint{1.045972in}{1.950909in}}{\pgfqpoint{1.051796in}{1.945085in}}%
\pgfpathcurveto{\pgfqpoint{1.057620in}{1.939261in}}{\pgfqpoint{1.065520in}{1.935989in}}{\pgfqpoint{1.073756in}{1.935989in}}%
\pgfpathclose%
\pgfusepath{stroke,fill}%
\end{pgfscope}%
\begin{pgfscope}%
\pgfpathrectangle{\pgfqpoint{0.100000in}{0.220728in}}{\pgfqpoint{3.696000in}{3.696000in}}%
\pgfusepath{clip}%
\pgfsetbuttcap%
\pgfsetroundjoin%
\definecolor{currentfill}{rgb}{0.121569,0.466667,0.705882}%
\pgfsetfillcolor{currentfill}%
\pgfsetfillopacity{0.489576}%
\pgfsetlinewidth{1.003750pt}%
\definecolor{currentstroke}{rgb}{0.121569,0.466667,0.705882}%
\pgfsetstrokecolor{currentstroke}%
\pgfsetstrokeopacity{0.489576}%
\pgfsetdash{}{0pt}%
\pgfpathmoveto{\pgfqpoint{2.494091in}{3.008970in}}%
\pgfpathcurveto{\pgfqpoint{2.502327in}{3.008970in}}{\pgfqpoint{2.510227in}{3.012243in}}{\pgfqpoint{2.516051in}{3.018067in}}%
\pgfpathcurveto{\pgfqpoint{2.521875in}{3.023890in}}{\pgfqpoint{2.525147in}{3.031791in}}{\pgfqpoint{2.525147in}{3.040027in}}%
\pgfpathcurveto{\pgfqpoint{2.525147in}{3.048263in}}{\pgfqpoint{2.521875in}{3.056163in}}{\pgfqpoint{2.516051in}{3.061987in}}%
\pgfpathcurveto{\pgfqpoint{2.510227in}{3.067811in}}{\pgfqpoint{2.502327in}{3.071083in}}{\pgfqpoint{2.494091in}{3.071083in}}%
\pgfpathcurveto{\pgfqpoint{2.485854in}{3.071083in}}{\pgfqpoint{2.477954in}{3.067811in}}{\pgfqpoint{2.472130in}{3.061987in}}%
\pgfpathcurveto{\pgfqpoint{2.466306in}{3.056163in}}{\pgfqpoint{2.463034in}{3.048263in}}{\pgfqpoint{2.463034in}{3.040027in}}%
\pgfpathcurveto{\pgfqpoint{2.463034in}{3.031791in}}{\pgfqpoint{2.466306in}{3.023890in}}{\pgfqpoint{2.472130in}{3.018067in}}%
\pgfpathcurveto{\pgfqpoint{2.477954in}{3.012243in}}{\pgfqpoint{2.485854in}{3.008970in}}{\pgfqpoint{2.494091in}{3.008970in}}%
\pgfpathclose%
\pgfusepath{stroke,fill}%
\end{pgfscope}%
\begin{pgfscope}%
\pgfpathrectangle{\pgfqpoint{0.100000in}{0.220728in}}{\pgfqpoint{3.696000in}{3.696000in}}%
\pgfusepath{clip}%
\pgfsetbuttcap%
\pgfsetroundjoin%
\definecolor{currentfill}{rgb}{0.121569,0.466667,0.705882}%
\pgfsetfillcolor{currentfill}%
\pgfsetfillopacity{0.490777}%
\pgfsetlinewidth{1.003750pt}%
\definecolor{currentstroke}{rgb}{0.121569,0.466667,0.705882}%
\pgfsetstrokecolor{currentstroke}%
\pgfsetstrokeopacity{0.490777}%
\pgfsetdash{}{0pt}%
\pgfpathmoveto{\pgfqpoint{1.055254in}{1.901284in}}%
\pgfpathcurveto{\pgfqpoint{1.063490in}{1.901284in}}{\pgfqpoint{1.071390in}{1.904557in}}{\pgfqpoint{1.077214in}{1.910380in}}%
\pgfpathcurveto{\pgfqpoint{1.083038in}{1.916204in}}{\pgfqpoint{1.086310in}{1.924104in}}{\pgfqpoint{1.086310in}{1.932341in}}%
\pgfpathcurveto{\pgfqpoint{1.086310in}{1.940577in}}{\pgfqpoint{1.083038in}{1.948477in}}{\pgfqpoint{1.077214in}{1.954301in}}%
\pgfpathcurveto{\pgfqpoint{1.071390in}{1.960125in}}{\pgfqpoint{1.063490in}{1.963397in}}{\pgfqpoint{1.055254in}{1.963397in}}%
\pgfpathcurveto{\pgfqpoint{1.047017in}{1.963397in}}{\pgfqpoint{1.039117in}{1.960125in}}{\pgfqpoint{1.033293in}{1.954301in}}%
\pgfpathcurveto{\pgfqpoint{1.027469in}{1.948477in}}{\pgfqpoint{1.024197in}{1.940577in}}{\pgfqpoint{1.024197in}{1.932341in}}%
\pgfpathcurveto{\pgfqpoint{1.024197in}{1.924104in}}{\pgfqpoint{1.027469in}{1.916204in}}{\pgfqpoint{1.033293in}{1.910380in}}%
\pgfpathcurveto{\pgfqpoint{1.039117in}{1.904557in}}{\pgfqpoint{1.047017in}{1.901284in}}{\pgfqpoint{1.055254in}{1.901284in}}%
\pgfpathclose%
\pgfusepath{stroke,fill}%
\end{pgfscope}%
\begin{pgfscope}%
\pgfpathrectangle{\pgfqpoint{0.100000in}{0.220728in}}{\pgfqpoint{3.696000in}{3.696000in}}%
\pgfusepath{clip}%
\pgfsetbuttcap%
\pgfsetroundjoin%
\definecolor{currentfill}{rgb}{0.121569,0.466667,0.705882}%
\pgfsetfillcolor{currentfill}%
\pgfsetfillopacity{0.494844}%
\pgfsetlinewidth{1.003750pt}%
\definecolor{currentstroke}{rgb}{0.121569,0.466667,0.705882}%
\pgfsetstrokecolor{currentstroke}%
\pgfsetstrokeopacity{0.494844}%
\pgfsetdash{}{0pt}%
\pgfpathmoveto{\pgfqpoint{2.512681in}{3.006225in}}%
\pgfpathcurveto{\pgfqpoint{2.520917in}{3.006225in}}{\pgfqpoint{2.528817in}{3.009497in}}{\pgfqpoint{2.534641in}{3.015321in}}%
\pgfpathcurveto{\pgfqpoint{2.540465in}{3.021145in}}{\pgfqpoint{2.543738in}{3.029045in}}{\pgfqpoint{2.543738in}{3.037281in}}%
\pgfpathcurveto{\pgfqpoint{2.543738in}{3.045518in}}{\pgfqpoint{2.540465in}{3.053418in}}{\pgfqpoint{2.534641in}{3.059242in}}%
\pgfpathcurveto{\pgfqpoint{2.528817in}{3.065066in}}{\pgfqpoint{2.520917in}{3.068338in}}{\pgfqpoint{2.512681in}{3.068338in}}%
\pgfpathcurveto{\pgfqpoint{2.504445in}{3.068338in}}{\pgfqpoint{2.496545in}{3.065066in}}{\pgfqpoint{2.490721in}{3.059242in}}%
\pgfpathcurveto{\pgfqpoint{2.484897in}{3.053418in}}{\pgfqpoint{2.481625in}{3.045518in}}{\pgfqpoint{2.481625in}{3.037281in}}%
\pgfpathcurveto{\pgfqpoint{2.481625in}{3.029045in}}{\pgfqpoint{2.484897in}{3.021145in}}{\pgfqpoint{2.490721in}{3.015321in}}%
\pgfpathcurveto{\pgfqpoint{2.496545in}{3.009497in}}{\pgfqpoint{2.504445in}{3.006225in}}{\pgfqpoint{2.512681in}{3.006225in}}%
\pgfpathclose%
\pgfusepath{stroke,fill}%
\end{pgfscope}%
\begin{pgfscope}%
\pgfpathrectangle{\pgfqpoint{0.100000in}{0.220728in}}{\pgfqpoint{3.696000in}{3.696000in}}%
\pgfusepath{clip}%
\pgfsetbuttcap%
\pgfsetroundjoin%
\definecolor{currentfill}{rgb}{0.121569,0.466667,0.705882}%
\pgfsetfillcolor{currentfill}%
\pgfsetfillopacity{0.500489}%
\pgfsetlinewidth{1.003750pt}%
\definecolor{currentstroke}{rgb}{0.121569,0.466667,0.705882}%
\pgfsetstrokecolor{currentstroke}%
\pgfsetstrokeopacity{0.500489}%
\pgfsetdash{}{0pt}%
\pgfpathmoveto{\pgfqpoint{1.030248in}{1.832981in}}%
\pgfpathcurveto{\pgfqpoint{1.038484in}{1.832981in}}{\pgfqpoint{1.046385in}{1.836253in}}{\pgfqpoint{1.052208in}{1.842077in}}%
\pgfpathcurveto{\pgfqpoint{1.058032in}{1.847901in}}{\pgfqpoint{1.061305in}{1.855801in}}{\pgfqpoint{1.061305in}{1.864037in}}%
\pgfpathcurveto{\pgfqpoint{1.061305in}{1.872273in}}{\pgfqpoint{1.058032in}{1.880173in}}{\pgfqpoint{1.052208in}{1.885997in}}%
\pgfpathcurveto{\pgfqpoint{1.046385in}{1.891821in}}{\pgfqpoint{1.038484in}{1.895094in}}{\pgfqpoint{1.030248in}{1.895094in}}%
\pgfpathcurveto{\pgfqpoint{1.022012in}{1.895094in}}{\pgfqpoint{1.014112in}{1.891821in}}{\pgfqpoint{1.008288in}{1.885997in}}%
\pgfpathcurveto{\pgfqpoint{1.002464in}{1.880173in}}{\pgfqpoint{0.999192in}{1.872273in}}{\pgfqpoint{0.999192in}{1.864037in}}%
\pgfpathcurveto{\pgfqpoint{0.999192in}{1.855801in}}{\pgfqpoint{1.002464in}{1.847901in}}{\pgfqpoint{1.008288in}{1.842077in}}%
\pgfpathcurveto{\pgfqpoint{1.014112in}{1.836253in}}{\pgfqpoint{1.022012in}{1.832981in}}{\pgfqpoint{1.030248in}{1.832981in}}%
\pgfpathclose%
\pgfusepath{stroke,fill}%
\end{pgfscope}%
\begin{pgfscope}%
\pgfpathrectangle{\pgfqpoint{0.100000in}{0.220728in}}{\pgfqpoint{3.696000in}{3.696000in}}%
\pgfusepath{clip}%
\pgfsetbuttcap%
\pgfsetroundjoin%
\definecolor{currentfill}{rgb}{0.121569,0.466667,0.705882}%
\pgfsetfillcolor{currentfill}%
\pgfsetfillopacity{0.500829}%
\pgfsetlinewidth{1.003750pt}%
\definecolor{currentstroke}{rgb}{0.121569,0.466667,0.705882}%
\pgfsetstrokecolor{currentstroke}%
\pgfsetstrokeopacity{0.500829}%
\pgfsetdash{}{0pt}%
\pgfpathmoveto{\pgfqpoint{2.533461in}{2.998439in}}%
\pgfpathcurveto{\pgfqpoint{2.541697in}{2.998439in}}{\pgfqpoint{2.549597in}{3.001711in}}{\pgfqpoint{2.555421in}{3.007535in}}%
\pgfpathcurveto{\pgfqpoint{2.561245in}{3.013359in}}{\pgfqpoint{2.564518in}{3.021259in}}{\pgfqpoint{2.564518in}{3.029496in}}%
\pgfpathcurveto{\pgfqpoint{2.564518in}{3.037732in}}{\pgfqpoint{2.561245in}{3.045632in}}{\pgfqpoint{2.555421in}{3.051456in}}%
\pgfpathcurveto{\pgfqpoint{2.549597in}{3.057280in}}{\pgfqpoint{2.541697in}{3.060552in}}{\pgfqpoint{2.533461in}{3.060552in}}%
\pgfpathcurveto{\pgfqpoint{2.525225in}{3.060552in}}{\pgfqpoint{2.517325in}{3.057280in}}{\pgfqpoint{2.511501in}{3.051456in}}%
\pgfpathcurveto{\pgfqpoint{2.505677in}{3.045632in}}{\pgfqpoint{2.502405in}{3.037732in}}{\pgfqpoint{2.502405in}{3.029496in}}%
\pgfpathcurveto{\pgfqpoint{2.502405in}{3.021259in}}{\pgfqpoint{2.505677in}{3.013359in}}{\pgfqpoint{2.511501in}{3.007535in}}%
\pgfpathcurveto{\pgfqpoint{2.517325in}{3.001711in}}{\pgfqpoint{2.525225in}{2.998439in}}{\pgfqpoint{2.533461in}{2.998439in}}%
\pgfpathclose%
\pgfusepath{stroke,fill}%
\end{pgfscope}%
\begin{pgfscope}%
\pgfpathrectangle{\pgfqpoint{0.100000in}{0.220728in}}{\pgfqpoint{3.696000in}{3.696000in}}%
\pgfusepath{clip}%
\pgfsetbuttcap%
\pgfsetroundjoin%
\definecolor{currentfill}{rgb}{0.121569,0.466667,0.705882}%
\pgfsetfillcolor{currentfill}%
\pgfsetfillopacity{0.507660}%
\pgfsetlinewidth{1.003750pt}%
\definecolor{currentstroke}{rgb}{0.121569,0.466667,0.705882}%
\pgfsetstrokecolor{currentstroke}%
\pgfsetstrokeopacity{0.507660}%
\pgfsetdash{}{0pt}%
\pgfpathmoveto{\pgfqpoint{2.556606in}{2.992003in}}%
\pgfpathcurveto{\pgfqpoint{2.564843in}{2.992003in}}{\pgfqpoint{2.572743in}{2.995275in}}{\pgfqpoint{2.578567in}{3.001099in}}%
\pgfpathcurveto{\pgfqpoint{2.584391in}{3.006923in}}{\pgfqpoint{2.587663in}{3.014823in}}{\pgfqpoint{2.587663in}{3.023059in}}%
\pgfpathcurveto{\pgfqpoint{2.587663in}{3.031296in}}{\pgfqpoint{2.584391in}{3.039196in}}{\pgfqpoint{2.578567in}{3.045020in}}%
\pgfpathcurveto{\pgfqpoint{2.572743in}{3.050843in}}{\pgfqpoint{2.564843in}{3.054116in}}{\pgfqpoint{2.556606in}{3.054116in}}%
\pgfpathcurveto{\pgfqpoint{2.548370in}{3.054116in}}{\pgfqpoint{2.540470in}{3.050843in}}{\pgfqpoint{2.534646in}{3.045020in}}%
\pgfpathcurveto{\pgfqpoint{2.528822in}{3.039196in}}{\pgfqpoint{2.525550in}{3.031296in}}{\pgfqpoint{2.525550in}{3.023059in}}%
\pgfpathcurveto{\pgfqpoint{2.525550in}{3.014823in}}{\pgfqpoint{2.528822in}{3.006923in}}{\pgfqpoint{2.534646in}{3.001099in}}%
\pgfpathcurveto{\pgfqpoint{2.540470in}{2.995275in}}{\pgfqpoint{2.548370in}{2.992003in}}{\pgfqpoint{2.556606in}{2.992003in}}%
\pgfpathclose%
\pgfusepath{stroke,fill}%
\end{pgfscope}%
\begin{pgfscope}%
\pgfpathrectangle{\pgfqpoint{0.100000in}{0.220728in}}{\pgfqpoint{3.696000in}{3.696000in}}%
\pgfusepath{clip}%
\pgfsetbuttcap%
\pgfsetroundjoin%
\definecolor{currentfill}{rgb}{0.121569,0.466667,0.705882}%
\pgfsetfillcolor{currentfill}%
\pgfsetfillopacity{0.507772}%
\pgfsetlinewidth{1.003750pt}%
\definecolor{currentstroke}{rgb}{0.121569,0.466667,0.705882}%
\pgfsetstrokecolor{currentstroke}%
\pgfsetstrokeopacity{0.507772}%
\pgfsetdash{}{0pt}%
\pgfpathmoveto{\pgfqpoint{0.997693in}{1.780132in}}%
\pgfpathcurveto{\pgfqpoint{1.005930in}{1.780132in}}{\pgfqpoint{1.013830in}{1.783404in}}{\pgfqpoint{1.019654in}{1.789228in}}%
\pgfpathcurveto{\pgfqpoint{1.025478in}{1.795052in}}{\pgfqpoint{1.028750in}{1.802952in}}{\pgfqpoint{1.028750in}{1.811188in}}%
\pgfpathcurveto{\pgfqpoint{1.028750in}{1.819425in}}{\pgfqpoint{1.025478in}{1.827325in}}{\pgfqpoint{1.019654in}{1.833149in}}%
\pgfpathcurveto{\pgfqpoint{1.013830in}{1.838972in}}{\pgfqpoint{1.005930in}{1.842245in}}{\pgfqpoint{0.997693in}{1.842245in}}%
\pgfpathcurveto{\pgfqpoint{0.989457in}{1.842245in}}{\pgfqpoint{0.981557in}{1.838972in}}{\pgfqpoint{0.975733in}{1.833149in}}%
\pgfpathcurveto{\pgfqpoint{0.969909in}{1.827325in}}{\pgfqpoint{0.966637in}{1.819425in}}{\pgfqpoint{0.966637in}{1.811188in}}%
\pgfpathcurveto{\pgfqpoint{0.966637in}{1.802952in}}{\pgfqpoint{0.969909in}{1.795052in}}{\pgfqpoint{0.975733in}{1.789228in}}%
\pgfpathcurveto{\pgfqpoint{0.981557in}{1.783404in}}{\pgfqpoint{0.989457in}{1.780132in}}{\pgfqpoint{0.997693in}{1.780132in}}%
\pgfpathclose%
\pgfusepath{stroke,fill}%
\end{pgfscope}%
\begin{pgfscope}%
\pgfpathrectangle{\pgfqpoint{0.100000in}{0.220728in}}{\pgfqpoint{3.696000in}{3.696000in}}%
\pgfusepath{clip}%
\pgfsetbuttcap%
\pgfsetroundjoin%
\definecolor{currentfill}{rgb}{0.121569,0.466667,0.705882}%
\pgfsetfillcolor{currentfill}%
\pgfsetfillopacity{0.511661}%
\pgfsetlinewidth{1.003750pt}%
\definecolor{currentstroke}{rgb}{0.121569,0.466667,0.705882}%
\pgfsetstrokecolor{currentstroke}%
\pgfsetstrokeopacity{0.511661}%
\pgfsetdash{}{0pt}%
\pgfpathmoveto{\pgfqpoint{2.568819in}{2.988459in}}%
\pgfpathcurveto{\pgfqpoint{2.577055in}{2.988459in}}{\pgfqpoint{2.584955in}{2.991732in}}{\pgfqpoint{2.590779in}{2.997556in}}%
\pgfpathcurveto{\pgfqpoint{2.596603in}{3.003380in}}{\pgfqpoint{2.599876in}{3.011280in}}{\pgfqpoint{2.599876in}{3.019516in}}%
\pgfpathcurveto{\pgfqpoint{2.599876in}{3.027752in}}{\pgfqpoint{2.596603in}{3.035652in}}{\pgfqpoint{2.590779in}{3.041476in}}%
\pgfpathcurveto{\pgfqpoint{2.584955in}{3.047300in}}{\pgfqpoint{2.577055in}{3.050572in}}{\pgfqpoint{2.568819in}{3.050572in}}%
\pgfpathcurveto{\pgfqpoint{2.560583in}{3.050572in}}{\pgfqpoint{2.552683in}{3.047300in}}{\pgfqpoint{2.546859in}{3.041476in}}%
\pgfpathcurveto{\pgfqpoint{2.541035in}{3.035652in}}{\pgfqpoint{2.537763in}{3.027752in}}{\pgfqpoint{2.537763in}{3.019516in}}%
\pgfpathcurveto{\pgfqpoint{2.537763in}{3.011280in}}{\pgfqpoint{2.541035in}{3.003380in}}{\pgfqpoint{2.546859in}{2.997556in}}%
\pgfpathcurveto{\pgfqpoint{2.552683in}{2.991732in}}{\pgfqpoint{2.560583in}{2.988459in}}{\pgfqpoint{2.568819in}{2.988459in}}%
\pgfpathclose%
\pgfusepath{stroke,fill}%
\end{pgfscope}%
\begin{pgfscope}%
\pgfpathrectangle{\pgfqpoint{0.100000in}{0.220728in}}{\pgfqpoint{3.696000in}{3.696000in}}%
\pgfusepath{clip}%
\pgfsetbuttcap%
\pgfsetroundjoin%
\definecolor{currentfill}{rgb}{0.121569,0.466667,0.705882}%
\pgfsetfillcolor{currentfill}%
\pgfsetfillopacity{0.516016}%
\pgfsetlinewidth{1.003750pt}%
\definecolor{currentstroke}{rgb}{0.121569,0.466667,0.705882}%
\pgfsetstrokecolor{currentstroke}%
\pgfsetstrokeopacity{0.516016}%
\pgfsetdash{}{0pt}%
\pgfpathmoveto{\pgfqpoint{2.584485in}{2.985246in}}%
\pgfpathcurveto{\pgfqpoint{2.592721in}{2.985246in}}{\pgfqpoint{2.600621in}{2.988518in}}{\pgfqpoint{2.606445in}{2.994342in}}%
\pgfpathcurveto{\pgfqpoint{2.612269in}{3.000166in}}{\pgfqpoint{2.615542in}{3.008066in}}{\pgfqpoint{2.615542in}{3.016302in}}%
\pgfpathcurveto{\pgfqpoint{2.615542in}{3.024538in}}{\pgfqpoint{2.612269in}{3.032438in}}{\pgfqpoint{2.606445in}{3.038262in}}%
\pgfpathcurveto{\pgfqpoint{2.600621in}{3.044086in}}{\pgfqpoint{2.592721in}{3.047359in}}{\pgfqpoint{2.584485in}{3.047359in}}%
\pgfpathcurveto{\pgfqpoint{2.576249in}{3.047359in}}{\pgfqpoint{2.568349in}{3.044086in}}{\pgfqpoint{2.562525in}{3.038262in}}%
\pgfpathcurveto{\pgfqpoint{2.556701in}{3.032438in}}{\pgfqpoint{2.553429in}{3.024538in}}{\pgfqpoint{2.553429in}{3.016302in}}%
\pgfpathcurveto{\pgfqpoint{2.553429in}{3.008066in}}{\pgfqpoint{2.556701in}{3.000166in}}{\pgfqpoint{2.562525in}{2.994342in}}%
\pgfpathcurveto{\pgfqpoint{2.568349in}{2.988518in}}{\pgfqpoint{2.576249in}{2.985246in}}{\pgfqpoint{2.584485in}{2.985246in}}%
\pgfpathclose%
\pgfusepath{stroke,fill}%
\end{pgfscope}%
\begin{pgfscope}%
\pgfpathrectangle{\pgfqpoint{0.100000in}{0.220728in}}{\pgfqpoint{3.696000in}{3.696000in}}%
\pgfusepath{clip}%
\pgfsetbuttcap%
\pgfsetroundjoin%
\definecolor{currentfill}{rgb}{0.121569,0.466667,0.705882}%
\pgfsetfillcolor{currentfill}%
\pgfsetfillopacity{0.516488}%
\pgfsetlinewidth{1.003750pt}%
\definecolor{currentstroke}{rgb}{0.121569,0.466667,0.705882}%
\pgfsetstrokecolor{currentstroke}%
\pgfsetstrokeopacity{0.516488}%
\pgfsetdash{}{0pt}%
\pgfpathmoveto{\pgfqpoint{0.978285in}{1.734013in}}%
\pgfpathcurveto{\pgfqpoint{0.986521in}{1.734013in}}{\pgfqpoint{0.994421in}{1.737285in}}{\pgfqpoint{1.000245in}{1.743109in}}%
\pgfpathcurveto{\pgfqpoint{1.006069in}{1.748933in}}{\pgfqpoint{1.009342in}{1.756833in}}{\pgfqpoint{1.009342in}{1.765069in}}%
\pgfpathcurveto{\pgfqpoint{1.009342in}{1.773306in}}{\pgfqpoint{1.006069in}{1.781206in}}{\pgfqpoint{1.000245in}{1.787030in}}%
\pgfpathcurveto{\pgfqpoint{0.994421in}{1.792853in}}{\pgfqpoint{0.986521in}{1.796126in}}{\pgfqpoint{0.978285in}{1.796126in}}%
\pgfpathcurveto{\pgfqpoint{0.970049in}{1.796126in}}{\pgfqpoint{0.962149in}{1.792853in}}{\pgfqpoint{0.956325in}{1.787030in}}%
\pgfpathcurveto{\pgfqpoint{0.950501in}{1.781206in}}{\pgfqpoint{0.947229in}{1.773306in}}{\pgfqpoint{0.947229in}{1.765069in}}%
\pgfpathcurveto{\pgfqpoint{0.947229in}{1.756833in}}{\pgfqpoint{0.950501in}{1.748933in}}{\pgfqpoint{0.956325in}{1.743109in}}%
\pgfpathcurveto{\pgfqpoint{0.962149in}{1.737285in}}{\pgfqpoint{0.970049in}{1.734013in}}{\pgfqpoint{0.978285in}{1.734013in}}%
\pgfpathclose%
\pgfusepath{stroke,fill}%
\end{pgfscope}%
\begin{pgfscope}%
\pgfpathrectangle{\pgfqpoint{0.100000in}{0.220728in}}{\pgfqpoint{3.696000in}{3.696000in}}%
\pgfusepath{clip}%
\pgfsetbuttcap%
\pgfsetroundjoin%
\definecolor{currentfill}{rgb}{0.121569,0.466667,0.705882}%
\pgfsetfillcolor{currentfill}%
\pgfsetfillopacity{0.521194}%
\pgfsetlinewidth{1.003750pt}%
\definecolor{currentstroke}{rgb}{0.121569,0.466667,0.705882}%
\pgfsetstrokecolor{currentstroke}%
\pgfsetstrokeopacity{0.521194}%
\pgfsetdash{}{0pt}%
\pgfpathmoveto{\pgfqpoint{2.604319in}{2.981664in}}%
\pgfpathcurveto{\pgfqpoint{2.612555in}{2.981664in}}{\pgfqpoint{2.620455in}{2.984936in}}{\pgfqpoint{2.626279in}{2.990760in}}%
\pgfpathcurveto{\pgfqpoint{2.632103in}{2.996584in}}{\pgfqpoint{2.635376in}{3.004484in}}{\pgfqpoint{2.635376in}{3.012720in}}%
\pgfpathcurveto{\pgfqpoint{2.635376in}{3.020957in}}{\pgfqpoint{2.632103in}{3.028857in}}{\pgfqpoint{2.626279in}{3.034681in}}%
\pgfpathcurveto{\pgfqpoint{2.620455in}{3.040505in}}{\pgfqpoint{2.612555in}{3.043777in}}{\pgfqpoint{2.604319in}{3.043777in}}%
\pgfpathcurveto{\pgfqpoint{2.596083in}{3.043777in}}{\pgfqpoint{2.588183in}{3.040505in}}{\pgfqpoint{2.582359in}{3.034681in}}%
\pgfpathcurveto{\pgfqpoint{2.576535in}{3.028857in}}{\pgfqpoint{2.573263in}{3.020957in}}{\pgfqpoint{2.573263in}{3.012720in}}%
\pgfpathcurveto{\pgfqpoint{2.573263in}{3.004484in}}{\pgfqpoint{2.576535in}{2.996584in}}{\pgfqpoint{2.582359in}{2.990760in}}%
\pgfpathcurveto{\pgfqpoint{2.588183in}{2.984936in}}{\pgfqpoint{2.596083in}{2.981664in}}{\pgfqpoint{2.604319in}{2.981664in}}%
\pgfpathclose%
\pgfusepath{stroke,fill}%
\end{pgfscope}%
\begin{pgfscope}%
\pgfpathrectangle{\pgfqpoint{0.100000in}{0.220728in}}{\pgfqpoint{3.696000in}{3.696000in}}%
\pgfusepath{clip}%
\pgfsetbuttcap%
\pgfsetroundjoin%
\definecolor{currentfill}{rgb}{0.121569,0.466667,0.705882}%
\pgfsetfillcolor{currentfill}%
\pgfsetfillopacity{0.522390}%
\pgfsetlinewidth{1.003750pt}%
\definecolor{currentstroke}{rgb}{0.121569,0.466667,0.705882}%
\pgfsetstrokecolor{currentstroke}%
\pgfsetstrokeopacity{0.522390}%
\pgfsetdash{}{0pt}%
\pgfpathmoveto{\pgfqpoint{0.948010in}{1.698003in}}%
\pgfpathcurveto{\pgfqpoint{0.956246in}{1.698003in}}{\pgfqpoint{0.964146in}{1.701275in}}{\pgfqpoint{0.969970in}{1.707099in}}%
\pgfpathcurveto{\pgfqpoint{0.975794in}{1.712923in}}{\pgfqpoint{0.979066in}{1.720823in}}{\pgfqpoint{0.979066in}{1.729059in}}%
\pgfpathcurveto{\pgfqpoint{0.979066in}{1.737295in}}{\pgfqpoint{0.975794in}{1.745195in}}{\pgfqpoint{0.969970in}{1.751019in}}%
\pgfpathcurveto{\pgfqpoint{0.964146in}{1.756843in}}{\pgfqpoint{0.956246in}{1.760116in}}{\pgfqpoint{0.948010in}{1.760116in}}%
\pgfpathcurveto{\pgfqpoint{0.939774in}{1.760116in}}{\pgfqpoint{0.931874in}{1.756843in}}{\pgfqpoint{0.926050in}{1.751019in}}%
\pgfpathcurveto{\pgfqpoint{0.920226in}{1.745195in}}{\pgfqpoint{0.916953in}{1.737295in}}{\pgfqpoint{0.916953in}{1.729059in}}%
\pgfpathcurveto{\pgfqpoint{0.916953in}{1.720823in}}{\pgfqpoint{0.920226in}{1.712923in}}{\pgfqpoint{0.926050in}{1.707099in}}%
\pgfpathcurveto{\pgfqpoint{0.931874in}{1.701275in}}{\pgfqpoint{0.939774in}{1.698003in}}{\pgfqpoint{0.948010in}{1.698003in}}%
\pgfpathclose%
\pgfusepath{stroke,fill}%
\end{pgfscope}%
\begin{pgfscope}%
\pgfpathrectangle{\pgfqpoint{0.100000in}{0.220728in}}{\pgfqpoint{3.696000in}{3.696000in}}%
\pgfusepath{clip}%
\pgfsetbuttcap%
\pgfsetroundjoin%
\definecolor{currentfill}{rgb}{0.121569,0.466667,0.705882}%
\pgfsetfillcolor{currentfill}%
\pgfsetfillopacity{0.524320}%
\pgfsetlinewidth{1.003750pt}%
\definecolor{currentstroke}{rgb}{0.121569,0.466667,0.705882}%
\pgfsetstrokecolor{currentstroke}%
\pgfsetstrokeopacity{0.524320}%
\pgfsetdash{}{0pt}%
\pgfpathmoveto{\pgfqpoint{2.614749in}{2.979799in}}%
\pgfpathcurveto{\pgfqpoint{2.622985in}{2.979799in}}{\pgfqpoint{2.630885in}{2.983072in}}{\pgfqpoint{2.636709in}{2.988895in}}%
\pgfpathcurveto{\pgfqpoint{2.642533in}{2.994719in}}{\pgfqpoint{2.645805in}{3.002619in}}{\pgfqpoint{2.645805in}{3.010856in}}%
\pgfpathcurveto{\pgfqpoint{2.645805in}{3.019092in}}{\pgfqpoint{2.642533in}{3.026992in}}{\pgfqpoint{2.636709in}{3.032816in}}%
\pgfpathcurveto{\pgfqpoint{2.630885in}{3.038640in}}{\pgfqpoint{2.622985in}{3.041912in}}{\pgfqpoint{2.614749in}{3.041912in}}%
\pgfpathcurveto{\pgfqpoint{2.606513in}{3.041912in}}{\pgfqpoint{2.598613in}{3.038640in}}{\pgfqpoint{2.592789in}{3.032816in}}%
\pgfpathcurveto{\pgfqpoint{2.586965in}{3.026992in}}{\pgfqpoint{2.583692in}{3.019092in}}{\pgfqpoint{2.583692in}{3.010856in}}%
\pgfpathcurveto{\pgfqpoint{2.583692in}{3.002619in}}{\pgfqpoint{2.586965in}{2.994719in}}{\pgfqpoint{2.592789in}{2.988895in}}%
\pgfpathcurveto{\pgfqpoint{2.598613in}{2.983072in}}{\pgfqpoint{2.606513in}{2.979799in}}{\pgfqpoint{2.614749in}{2.979799in}}%
\pgfpathclose%
\pgfusepath{stroke,fill}%
\end{pgfscope}%
\begin{pgfscope}%
\pgfpathrectangle{\pgfqpoint{0.100000in}{0.220728in}}{\pgfqpoint{3.696000in}{3.696000in}}%
\pgfusepath{clip}%
\pgfsetbuttcap%
\pgfsetroundjoin%
\definecolor{currentfill}{rgb}{0.121569,0.466667,0.705882}%
\pgfsetfillcolor{currentfill}%
\pgfsetfillopacity{0.525776}%
\pgfsetlinewidth{1.003750pt}%
\definecolor{currentstroke}{rgb}{0.121569,0.466667,0.705882}%
\pgfsetstrokecolor{currentstroke}%
\pgfsetstrokeopacity{0.525776}%
\pgfsetdash{}{0pt}%
\pgfpathmoveto{\pgfqpoint{2.630927in}{2.979790in}}%
\pgfpathcurveto{\pgfqpoint{2.639163in}{2.979790in}}{\pgfqpoint{2.647064in}{2.983062in}}{\pgfqpoint{2.652887in}{2.988886in}}%
\pgfpathcurveto{\pgfqpoint{2.658711in}{2.994710in}}{\pgfqpoint{2.661984in}{3.002610in}}{\pgfqpoint{2.661984in}{3.010847in}}%
\pgfpathcurveto{\pgfqpoint{2.661984in}{3.019083in}}{\pgfqpoint{2.658711in}{3.026983in}}{\pgfqpoint{2.652887in}{3.032807in}}%
\pgfpathcurveto{\pgfqpoint{2.647064in}{3.038631in}}{\pgfqpoint{2.639163in}{3.041903in}}{\pgfqpoint{2.630927in}{3.041903in}}%
\pgfpathcurveto{\pgfqpoint{2.622691in}{3.041903in}}{\pgfqpoint{2.614791in}{3.038631in}}{\pgfqpoint{2.608967in}{3.032807in}}%
\pgfpathcurveto{\pgfqpoint{2.603143in}{3.026983in}}{\pgfqpoint{2.599871in}{3.019083in}}{\pgfqpoint{2.599871in}{3.010847in}}%
\pgfpathcurveto{\pgfqpoint{2.599871in}{3.002610in}}{\pgfqpoint{2.603143in}{2.994710in}}{\pgfqpoint{2.608967in}{2.988886in}}%
\pgfpathcurveto{\pgfqpoint{2.614791in}{2.983062in}}{\pgfqpoint{2.622691in}{2.979790in}}{\pgfqpoint{2.630927in}{2.979790in}}%
\pgfpathclose%
\pgfusepath{stroke,fill}%
\end{pgfscope}%
\begin{pgfscope}%
\pgfpathrectangle{\pgfqpoint{0.100000in}{0.220728in}}{\pgfqpoint{3.696000in}{3.696000in}}%
\pgfusepath{clip}%
\pgfsetbuttcap%
\pgfsetroundjoin%
\definecolor{currentfill}{rgb}{0.121569,0.466667,0.705882}%
\pgfsetfillcolor{currentfill}%
\pgfsetfillopacity{0.525839}%
\pgfsetlinewidth{1.003750pt}%
\definecolor{currentstroke}{rgb}{0.121569,0.466667,0.705882}%
\pgfsetstrokecolor{currentstroke}%
\pgfsetstrokeopacity{0.525839}%
\pgfsetdash{}{0pt}%
\pgfpathmoveto{\pgfqpoint{2.621005in}{2.979156in}}%
\pgfpathcurveto{\pgfqpoint{2.629241in}{2.979156in}}{\pgfqpoint{2.637141in}{2.982429in}}{\pgfqpoint{2.642965in}{2.988253in}}%
\pgfpathcurveto{\pgfqpoint{2.648789in}{2.994077in}}{\pgfqpoint{2.652061in}{3.001977in}}{\pgfqpoint{2.652061in}{3.010213in}}%
\pgfpathcurveto{\pgfqpoint{2.652061in}{3.018449in}}{\pgfqpoint{2.648789in}{3.026349in}}{\pgfqpoint{2.642965in}{3.032173in}}%
\pgfpathcurveto{\pgfqpoint{2.637141in}{3.037997in}}{\pgfqpoint{2.629241in}{3.041269in}}{\pgfqpoint{2.621005in}{3.041269in}}%
\pgfpathcurveto{\pgfqpoint{2.612769in}{3.041269in}}{\pgfqpoint{2.604869in}{3.037997in}}{\pgfqpoint{2.599045in}{3.032173in}}%
\pgfpathcurveto{\pgfqpoint{2.593221in}{3.026349in}}{\pgfqpoint{2.589948in}{3.018449in}}{\pgfqpoint{2.589948in}{3.010213in}}%
\pgfpathcurveto{\pgfqpoint{2.589948in}{3.001977in}}{\pgfqpoint{2.593221in}{2.994077in}}{\pgfqpoint{2.599045in}{2.988253in}}%
\pgfpathcurveto{\pgfqpoint{2.604869in}{2.982429in}}{\pgfqpoint{2.612769in}{2.979156in}}{\pgfqpoint{2.621005in}{2.979156in}}%
\pgfpathclose%
\pgfusepath{stroke,fill}%
\end{pgfscope}%
\begin{pgfscope}%
\pgfpathrectangle{\pgfqpoint{0.100000in}{0.220728in}}{\pgfqpoint{3.696000in}{3.696000in}}%
\pgfusepath{clip}%
\pgfsetbuttcap%
\pgfsetroundjoin%
\definecolor{currentfill}{rgb}{0.121569,0.466667,0.705882}%
\pgfsetfillcolor{currentfill}%
\pgfsetfillopacity{0.529374}%
\pgfsetlinewidth{1.003750pt}%
\definecolor{currentstroke}{rgb}{0.121569,0.466667,0.705882}%
\pgfsetstrokecolor{currentstroke}%
\pgfsetstrokeopacity{0.529374}%
\pgfsetdash{}{0pt}%
\pgfpathmoveto{\pgfqpoint{0.938171in}{1.663251in}}%
\pgfpathcurveto{\pgfqpoint{0.946407in}{1.663251in}}{\pgfqpoint{0.954307in}{1.666523in}}{\pgfqpoint{0.960131in}{1.672347in}}%
\pgfpathcurveto{\pgfqpoint{0.965955in}{1.678171in}}{\pgfqpoint{0.969227in}{1.686071in}}{\pgfqpoint{0.969227in}{1.694307in}}%
\pgfpathcurveto{\pgfqpoint{0.969227in}{1.702544in}}{\pgfqpoint{0.965955in}{1.710444in}}{\pgfqpoint{0.960131in}{1.716268in}}%
\pgfpathcurveto{\pgfqpoint{0.954307in}{1.722091in}}{\pgfqpoint{0.946407in}{1.725364in}}{\pgfqpoint{0.938171in}{1.725364in}}%
\pgfpathcurveto{\pgfqpoint{0.929935in}{1.725364in}}{\pgfqpoint{0.922034in}{1.722091in}}{\pgfqpoint{0.916211in}{1.716268in}}%
\pgfpathcurveto{\pgfqpoint{0.910387in}{1.710444in}}{\pgfqpoint{0.907114in}{1.702544in}}{\pgfqpoint{0.907114in}{1.694307in}}%
\pgfpathcurveto{\pgfqpoint{0.907114in}{1.686071in}}{\pgfqpoint{0.910387in}{1.678171in}}{\pgfqpoint{0.916211in}{1.672347in}}%
\pgfpathcurveto{\pgfqpoint{0.922034in}{1.666523in}}{\pgfqpoint{0.929935in}{1.663251in}}{\pgfqpoint{0.938171in}{1.663251in}}%
\pgfpathclose%
\pgfusepath{stroke,fill}%
\end{pgfscope}%
\begin{pgfscope}%
\pgfpathrectangle{\pgfqpoint{0.100000in}{0.220728in}}{\pgfqpoint{3.696000in}{3.696000in}}%
\pgfusepath{clip}%
\pgfsetbuttcap%
\pgfsetroundjoin%
\definecolor{currentfill}{rgb}{0.121569,0.466667,0.705882}%
\pgfsetfillcolor{currentfill}%
\pgfsetfillopacity{0.529962}%
\pgfsetlinewidth{1.003750pt}%
\definecolor{currentstroke}{rgb}{0.121569,0.466667,0.705882}%
\pgfsetstrokecolor{currentstroke}%
\pgfsetstrokeopacity{0.529962}%
\pgfsetdash{}{0pt}%
\pgfpathmoveto{\pgfqpoint{2.645184in}{2.979162in}}%
\pgfpathcurveto{\pgfqpoint{2.653420in}{2.979162in}}{\pgfqpoint{2.661320in}{2.982435in}}{\pgfqpoint{2.667144in}{2.988259in}}%
\pgfpathcurveto{\pgfqpoint{2.672968in}{2.994083in}}{\pgfqpoint{2.676240in}{3.001983in}}{\pgfqpoint{2.676240in}{3.010219in}}%
\pgfpathcurveto{\pgfqpoint{2.676240in}{3.018455in}}{\pgfqpoint{2.672968in}{3.026355in}}{\pgfqpoint{2.667144in}{3.032179in}}%
\pgfpathcurveto{\pgfqpoint{2.661320in}{3.038003in}}{\pgfqpoint{2.653420in}{3.041275in}}{\pgfqpoint{2.645184in}{3.041275in}}%
\pgfpathcurveto{\pgfqpoint{2.636947in}{3.041275in}}{\pgfqpoint{2.629047in}{3.038003in}}{\pgfqpoint{2.623223in}{3.032179in}}%
\pgfpathcurveto{\pgfqpoint{2.617399in}{3.026355in}}{\pgfqpoint{2.614127in}{3.018455in}}{\pgfqpoint{2.614127in}{3.010219in}}%
\pgfpathcurveto{\pgfqpoint{2.614127in}{3.001983in}}{\pgfqpoint{2.617399in}{2.994083in}}{\pgfqpoint{2.623223in}{2.988259in}}%
\pgfpathcurveto{\pgfqpoint{2.629047in}{2.982435in}}{\pgfqpoint{2.636947in}{2.979162in}}{\pgfqpoint{2.645184in}{2.979162in}}%
\pgfpathclose%
\pgfusepath{stroke,fill}%
\end{pgfscope}%
\begin{pgfscope}%
\pgfpathrectangle{\pgfqpoint{0.100000in}{0.220728in}}{\pgfqpoint{3.696000in}{3.696000in}}%
\pgfusepath{clip}%
\pgfsetbuttcap%
\pgfsetroundjoin%
\definecolor{currentfill}{rgb}{0.121569,0.466667,0.705882}%
\pgfsetfillcolor{currentfill}%
\pgfsetfillopacity{0.532502}%
\pgfsetlinewidth{1.003750pt}%
\definecolor{currentstroke}{rgb}{0.121569,0.466667,0.705882}%
\pgfsetstrokecolor{currentstroke}%
\pgfsetstrokeopacity{0.532502}%
\pgfsetdash{}{0pt}%
\pgfpathmoveto{\pgfqpoint{0.921844in}{1.642740in}}%
\pgfpathcurveto{\pgfqpoint{0.930080in}{1.642740in}}{\pgfqpoint{0.937980in}{1.646013in}}{\pgfqpoint{0.943804in}{1.651837in}}%
\pgfpathcurveto{\pgfqpoint{0.949628in}{1.657660in}}{\pgfqpoint{0.952900in}{1.665560in}}{\pgfqpoint{0.952900in}{1.673797in}}%
\pgfpathcurveto{\pgfqpoint{0.952900in}{1.682033in}}{\pgfqpoint{0.949628in}{1.689933in}}{\pgfqpoint{0.943804in}{1.695757in}}%
\pgfpathcurveto{\pgfqpoint{0.937980in}{1.701581in}}{\pgfqpoint{0.930080in}{1.704853in}}{\pgfqpoint{0.921844in}{1.704853in}}%
\pgfpathcurveto{\pgfqpoint{0.913607in}{1.704853in}}{\pgfqpoint{0.905707in}{1.701581in}}{\pgfqpoint{0.899883in}{1.695757in}}%
\pgfpathcurveto{\pgfqpoint{0.894060in}{1.689933in}}{\pgfqpoint{0.890787in}{1.682033in}}{\pgfqpoint{0.890787in}{1.673797in}}%
\pgfpathcurveto{\pgfqpoint{0.890787in}{1.665560in}}{\pgfqpoint{0.894060in}{1.657660in}}{\pgfqpoint{0.899883in}{1.651837in}}%
\pgfpathcurveto{\pgfqpoint{0.905707in}{1.646013in}}{\pgfqpoint{0.913607in}{1.642740in}}{\pgfqpoint{0.921844in}{1.642740in}}%
\pgfpathclose%
\pgfusepath{stroke,fill}%
\end{pgfscope}%
\begin{pgfscope}%
\pgfpathrectangle{\pgfqpoint{0.100000in}{0.220728in}}{\pgfqpoint{3.696000in}{3.696000in}}%
\pgfusepath{clip}%
\pgfsetbuttcap%
\pgfsetroundjoin%
\definecolor{currentfill}{rgb}{0.121569,0.466667,0.705882}%
\pgfsetfillcolor{currentfill}%
\pgfsetfillopacity{0.533357}%
\pgfsetlinewidth{1.003750pt}%
\definecolor{currentstroke}{rgb}{0.121569,0.466667,0.705882}%
\pgfsetstrokecolor{currentstroke}%
\pgfsetstrokeopacity{0.533357}%
\pgfsetdash{}{0pt}%
\pgfpathmoveto{\pgfqpoint{2.664278in}{2.977869in}}%
\pgfpathcurveto{\pgfqpoint{2.672515in}{2.977869in}}{\pgfqpoint{2.680415in}{2.981141in}}{\pgfqpoint{2.686239in}{2.986965in}}%
\pgfpathcurveto{\pgfqpoint{2.692062in}{2.992789in}}{\pgfqpoint{2.695335in}{3.000689in}}{\pgfqpoint{2.695335in}{3.008925in}}%
\pgfpathcurveto{\pgfqpoint{2.695335in}{3.017161in}}{\pgfqpoint{2.692062in}{3.025061in}}{\pgfqpoint{2.686239in}{3.030885in}}%
\pgfpathcurveto{\pgfqpoint{2.680415in}{3.036709in}}{\pgfqpoint{2.672515in}{3.039982in}}{\pgfqpoint{2.664278in}{3.039982in}}%
\pgfpathcurveto{\pgfqpoint{2.656042in}{3.039982in}}{\pgfqpoint{2.648142in}{3.036709in}}{\pgfqpoint{2.642318in}{3.030885in}}%
\pgfpathcurveto{\pgfqpoint{2.636494in}{3.025061in}}{\pgfqpoint{2.633222in}{3.017161in}}{\pgfqpoint{2.633222in}{3.008925in}}%
\pgfpathcurveto{\pgfqpoint{2.633222in}{3.000689in}}{\pgfqpoint{2.636494in}{2.992789in}}{\pgfqpoint{2.642318in}{2.986965in}}%
\pgfpathcurveto{\pgfqpoint{2.648142in}{2.981141in}}{\pgfqpoint{2.656042in}{2.977869in}}{\pgfqpoint{2.664278in}{2.977869in}}%
\pgfpathclose%
\pgfusepath{stroke,fill}%
\end{pgfscope}%
\begin{pgfscope}%
\pgfpathrectangle{\pgfqpoint{0.100000in}{0.220728in}}{\pgfqpoint{3.696000in}{3.696000in}}%
\pgfusepath{clip}%
\pgfsetbuttcap%
\pgfsetroundjoin%
\definecolor{currentfill}{rgb}{0.121569,0.466667,0.705882}%
\pgfsetfillcolor{currentfill}%
\pgfsetfillopacity{0.534709}%
\pgfsetlinewidth{1.003750pt}%
\definecolor{currentstroke}{rgb}{0.121569,0.466667,0.705882}%
\pgfsetstrokecolor{currentstroke}%
\pgfsetstrokeopacity{0.534709}%
\pgfsetdash{}{0pt}%
\pgfpathmoveto{\pgfqpoint{0.916152in}{1.630536in}}%
\pgfpathcurveto{\pgfqpoint{0.924389in}{1.630536in}}{\pgfqpoint{0.932289in}{1.633809in}}{\pgfqpoint{0.938113in}{1.639633in}}%
\pgfpathcurveto{\pgfqpoint{0.943937in}{1.645457in}}{\pgfqpoint{0.947209in}{1.653357in}}{\pgfqpoint{0.947209in}{1.661593in}}%
\pgfpathcurveto{\pgfqpoint{0.947209in}{1.669829in}}{\pgfqpoint{0.943937in}{1.677729in}}{\pgfqpoint{0.938113in}{1.683553in}}%
\pgfpathcurveto{\pgfqpoint{0.932289in}{1.689377in}}{\pgfqpoint{0.924389in}{1.692649in}}{\pgfqpoint{0.916152in}{1.692649in}}%
\pgfpathcurveto{\pgfqpoint{0.907916in}{1.692649in}}{\pgfqpoint{0.900016in}{1.689377in}}{\pgfqpoint{0.894192in}{1.683553in}}%
\pgfpathcurveto{\pgfqpoint{0.888368in}{1.677729in}}{\pgfqpoint{0.885096in}{1.669829in}}{\pgfqpoint{0.885096in}{1.661593in}}%
\pgfpathcurveto{\pgfqpoint{0.885096in}{1.653357in}}{\pgfqpoint{0.888368in}{1.645457in}}{\pgfqpoint{0.894192in}{1.639633in}}%
\pgfpathcurveto{\pgfqpoint{0.900016in}{1.633809in}}{\pgfqpoint{0.907916in}{1.630536in}}{\pgfqpoint{0.916152in}{1.630536in}}%
\pgfpathclose%
\pgfusepath{stroke,fill}%
\end{pgfscope}%
\begin{pgfscope}%
\pgfpathrectangle{\pgfqpoint{0.100000in}{0.220728in}}{\pgfqpoint{3.696000in}{3.696000in}}%
\pgfusepath{clip}%
\pgfsetbuttcap%
\pgfsetroundjoin%
\definecolor{currentfill}{rgb}{0.121569,0.466667,0.705882}%
\pgfsetfillcolor{currentfill}%
\pgfsetfillopacity{0.535583}%
\pgfsetlinewidth{1.003750pt}%
\definecolor{currentstroke}{rgb}{0.121569,0.466667,0.705882}%
\pgfsetstrokecolor{currentstroke}%
\pgfsetstrokeopacity{0.535583}%
\pgfsetdash{}{0pt}%
\pgfpathmoveto{\pgfqpoint{0.913032in}{1.625031in}}%
\pgfpathcurveto{\pgfqpoint{0.921269in}{1.625031in}}{\pgfqpoint{0.929169in}{1.628303in}}{\pgfqpoint{0.934993in}{1.634127in}}%
\pgfpathcurveto{\pgfqpoint{0.940817in}{1.639951in}}{\pgfqpoint{0.944089in}{1.647851in}}{\pgfqpoint{0.944089in}{1.656087in}}%
\pgfpathcurveto{\pgfqpoint{0.944089in}{1.664324in}}{\pgfqpoint{0.940817in}{1.672224in}}{\pgfqpoint{0.934993in}{1.678048in}}%
\pgfpathcurveto{\pgfqpoint{0.929169in}{1.683872in}}{\pgfqpoint{0.921269in}{1.687144in}}{\pgfqpoint{0.913032in}{1.687144in}}%
\pgfpathcurveto{\pgfqpoint{0.904796in}{1.687144in}}{\pgfqpoint{0.896896in}{1.683872in}}{\pgfqpoint{0.891072in}{1.678048in}}%
\pgfpathcurveto{\pgfqpoint{0.885248in}{1.672224in}}{\pgfqpoint{0.881976in}{1.664324in}}{\pgfqpoint{0.881976in}{1.656087in}}%
\pgfpathcurveto{\pgfqpoint{0.881976in}{1.647851in}}{\pgfqpoint{0.885248in}{1.639951in}}{\pgfqpoint{0.891072in}{1.634127in}}%
\pgfpathcurveto{\pgfqpoint{0.896896in}{1.628303in}}{\pgfqpoint{0.904796in}{1.625031in}}{\pgfqpoint{0.913032in}{1.625031in}}%
\pgfpathclose%
\pgfusepath{stroke,fill}%
\end{pgfscope}%
\begin{pgfscope}%
\pgfpathrectangle{\pgfqpoint{0.100000in}{0.220728in}}{\pgfqpoint{3.696000in}{3.696000in}}%
\pgfusepath{clip}%
\pgfsetbuttcap%
\pgfsetroundjoin%
\definecolor{currentfill}{rgb}{0.121569,0.466667,0.705882}%
\pgfsetfillcolor{currentfill}%
\pgfsetfillopacity{0.537249}%
\pgfsetlinewidth{1.003750pt}%
\definecolor{currentstroke}{rgb}{0.121569,0.466667,0.705882}%
\pgfsetstrokecolor{currentstroke}%
\pgfsetstrokeopacity{0.537249}%
\pgfsetdash{}{0pt}%
\pgfpathmoveto{\pgfqpoint{0.907282in}{1.615435in}}%
\pgfpathcurveto{\pgfqpoint{0.915518in}{1.615435in}}{\pgfqpoint{0.923418in}{1.618707in}}{\pgfqpoint{0.929242in}{1.624531in}}%
\pgfpathcurveto{\pgfqpoint{0.935066in}{1.630355in}}{\pgfqpoint{0.938338in}{1.638255in}}{\pgfqpoint{0.938338in}{1.646491in}}%
\pgfpathcurveto{\pgfqpoint{0.938338in}{1.654728in}}{\pgfqpoint{0.935066in}{1.662628in}}{\pgfqpoint{0.929242in}{1.668452in}}%
\pgfpathcurveto{\pgfqpoint{0.923418in}{1.674275in}}{\pgfqpoint{0.915518in}{1.677548in}}{\pgfqpoint{0.907282in}{1.677548in}}%
\pgfpathcurveto{\pgfqpoint{0.899045in}{1.677548in}}{\pgfqpoint{0.891145in}{1.674275in}}{\pgfqpoint{0.885322in}{1.668452in}}%
\pgfpathcurveto{\pgfqpoint{0.879498in}{1.662628in}}{\pgfqpoint{0.876225in}{1.654728in}}{\pgfqpoint{0.876225in}{1.646491in}}%
\pgfpathcurveto{\pgfqpoint{0.876225in}{1.638255in}}{\pgfqpoint{0.879498in}{1.630355in}}{\pgfqpoint{0.885322in}{1.624531in}}%
\pgfpathcurveto{\pgfqpoint{0.891145in}{1.618707in}}{\pgfqpoint{0.899045in}{1.615435in}}{\pgfqpoint{0.907282in}{1.615435in}}%
\pgfpathclose%
\pgfusepath{stroke,fill}%
\end{pgfscope}%
\begin{pgfscope}%
\pgfpathrectangle{\pgfqpoint{0.100000in}{0.220728in}}{\pgfqpoint{3.696000in}{3.696000in}}%
\pgfusepath{clip}%
\pgfsetbuttcap%
\pgfsetroundjoin%
\definecolor{currentfill}{rgb}{0.121569,0.466667,0.705882}%
\pgfsetfillcolor{currentfill}%
\pgfsetfillopacity{0.540258}%
\pgfsetlinewidth{1.003750pt}%
\definecolor{currentstroke}{rgb}{0.121569,0.466667,0.705882}%
\pgfsetstrokecolor{currentstroke}%
\pgfsetstrokeopacity{0.540258}%
\pgfsetdash{}{0pt}%
\pgfpathmoveto{\pgfqpoint{2.682332in}{2.974968in}}%
\pgfpathcurveto{\pgfqpoint{2.690568in}{2.974968in}}{\pgfqpoint{2.698468in}{2.978240in}}{\pgfqpoint{2.704292in}{2.984064in}}%
\pgfpathcurveto{\pgfqpoint{2.710116in}{2.989888in}}{\pgfqpoint{2.713389in}{2.997788in}}{\pgfqpoint{2.713389in}{3.006024in}}%
\pgfpathcurveto{\pgfqpoint{2.713389in}{3.014260in}}{\pgfqpoint{2.710116in}{3.022161in}}{\pgfqpoint{2.704292in}{3.027984in}}%
\pgfpathcurveto{\pgfqpoint{2.698468in}{3.033808in}}{\pgfqpoint{2.690568in}{3.037081in}}{\pgfqpoint{2.682332in}{3.037081in}}%
\pgfpathcurveto{\pgfqpoint{2.674096in}{3.037081in}}{\pgfqpoint{2.666196in}{3.033808in}}{\pgfqpoint{2.660372in}{3.027984in}}%
\pgfpathcurveto{\pgfqpoint{2.654548in}{3.022161in}}{\pgfqpoint{2.651276in}{3.014260in}}{\pgfqpoint{2.651276in}{3.006024in}}%
\pgfpathcurveto{\pgfqpoint{2.651276in}{2.997788in}}{\pgfqpoint{2.654548in}{2.989888in}}{\pgfqpoint{2.660372in}{2.984064in}}%
\pgfpathcurveto{\pgfqpoint{2.666196in}{2.978240in}}{\pgfqpoint{2.674096in}{2.974968in}}{\pgfqpoint{2.682332in}{2.974968in}}%
\pgfpathclose%
\pgfusepath{stroke,fill}%
\end{pgfscope}%
\begin{pgfscope}%
\pgfpathrectangle{\pgfqpoint{0.100000in}{0.220728in}}{\pgfqpoint{3.696000in}{3.696000in}}%
\pgfusepath{clip}%
\pgfsetbuttcap%
\pgfsetroundjoin%
\definecolor{currentfill}{rgb}{0.121569,0.466667,0.705882}%
\pgfsetfillcolor{currentfill}%
\pgfsetfillopacity{0.540809}%
\pgfsetlinewidth{1.003750pt}%
\definecolor{currentstroke}{rgb}{0.121569,0.466667,0.705882}%
\pgfsetstrokecolor{currentstroke}%
\pgfsetstrokeopacity{0.540809}%
\pgfsetdash{}{0pt}%
\pgfpathmoveto{\pgfqpoint{0.899425in}{1.597815in}}%
\pgfpathcurveto{\pgfqpoint{0.907662in}{1.597815in}}{\pgfqpoint{0.915562in}{1.601087in}}{\pgfqpoint{0.921386in}{1.606911in}}%
\pgfpathcurveto{\pgfqpoint{0.927209in}{1.612735in}}{\pgfqpoint{0.930482in}{1.620635in}}{\pgfqpoint{0.930482in}{1.628872in}}%
\pgfpathcurveto{\pgfqpoint{0.930482in}{1.637108in}}{\pgfqpoint{0.927209in}{1.645008in}}{\pgfqpoint{0.921386in}{1.650832in}}%
\pgfpathcurveto{\pgfqpoint{0.915562in}{1.656656in}}{\pgfqpoint{0.907662in}{1.659928in}}{\pgfqpoint{0.899425in}{1.659928in}}%
\pgfpathcurveto{\pgfqpoint{0.891189in}{1.659928in}}{\pgfqpoint{0.883289in}{1.656656in}}{\pgfqpoint{0.877465in}{1.650832in}}%
\pgfpathcurveto{\pgfqpoint{0.871641in}{1.645008in}}{\pgfqpoint{0.868369in}{1.637108in}}{\pgfqpoint{0.868369in}{1.628872in}}%
\pgfpathcurveto{\pgfqpoint{0.868369in}{1.620635in}}{\pgfqpoint{0.871641in}{1.612735in}}{\pgfqpoint{0.877465in}{1.606911in}}%
\pgfpathcurveto{\pgfqpoint{0.883289in}{1.601087in}}{\pgfqpoint{0.891189in}{1.597815in}}{\pgfqpoint{0.899425in}{1.597815in}}%
\pgfpathclose%
\pgfusepath{stroke,fill}%
\end{pgfscope}%
\begin{pgfscope}%
\pgfpathrectangle{\pgfqpoint{0.100000in}{0.220728in}}{\pgfqpoint{3.696000in}{3.696000in}}%
\pgfusepath{clip}%
\pgfsetbuttcap%
\pgfsetroundjoin%
\definecolor{currentfill}{rgb}{0.121569,0.466667,0.705882}%
\pgfsetfillcolor{currentfill}%
\pgfsetfillopacity{0.542914}%
\pgfsetlinewidth{1.003750pt}%
\definecolor{currentstroke}{rgb}{0.121569,0.466667,0.705882}%
\pgfsetstrokecolor{currentstroke}%
\pgfsetstrokeopacity{0.542914}%
\pgfsetdash{}{0pt}%
\pgfpathmoveto{\pgfqpoint{2.694538in}{2.973424in}}%
\pgfpathcurveto{\pgfqpoint{2.702774in}{2.973424in}}{\pgfqpoint{2.710674in}{2.976696in}}{\pgfqpoint{2.716498in}{2.982520in}}%
\pgfpathcurveto{\pgfqpoint{2.722322in}{2.988344in}}{\pgfqpoint{2.725595in}{2.996244in}}{\pgfqpoint{2.725595in}{3.004480in}}%
\pgfpathcurveto{\pgfqpoint{2.725595in}{3.012717in}}{\pgfqpoint{2.722322in}{3.020617in}}{\pgfqpoint{2.716498in}{3.026441in}}%
\pgfpathcurveto{\pgfqpoint{2.710674in}{3.032264in}}{\pgfqpoint{2.702774in}{3.035537in}}{\pgfqpoint{2.694538in}{3.035537in}}%
\pgfpathcurveto{\pgfqpoint{2.686302in}{3.035537in}}{\pgfqpoint{2.678402in}{3.032264in}}{\pgfqpoint{2.672578in}{3.026441in}}%
\pgfpathcurveto{\pgfqpoint{2.666754in}{3.020617in}}{\pgfqpoint{2.663482in}{3.012717in}}{\pgfqpoint{2.663482in}{3.004480in}}%
\pgfpathcurveto{\pgfqpoint{2.663482in}{2.996244in}}{\pgfqpoint{2.666754in}{2.988344in}}{\pgfqpoint{2.672578in}{2.982520in}}%
\pgfpathcurveto{\pgfqpoint{2.678402in}{2.976696in}}{\pgfqpoint{2.686302in}{2.973424in}}{\pgfqpoint{2.694538in}{2.973424in}}%
\pgfpathclose%
\pgfusepath{stroke,fill}%
\end{pgfscope}%
\begin{pgfscope}%
\pgfpathrectangle{\pgfqpoint{0.100000in}{0.220728in}}{\pgfqpoint{3.696000in}{3.696000in}}%
\pgfusepath{clip}%
\pgfsetbuttcap%
\pgfsetroundjoin%
\definecolor{currentfill}{rgb}{0.121569,0.466667,0.705882}%
\pgfsetfillcolor{currentfill}%
\pgfsetfillopacity{0.545933}%
\pgfsetlinewidth{1.003750pt}%
\definecolor{currentstroke}{rgb}{0.121569,0.466667,0.705882}%
\pgfsetstrokecolor{currentstroke}%
\pgfsetstrokeopacity{0.545933}%
\pgfsetdash{}{0pt}%
\pgfpathmoveto{\pgfqpoint{0.877551in}{1.568139in}}%
\pgfpathcurveto{\pgfqpoint{0.885787in}{1.568139in}}{\pgfqpoint{0.893687in}{1.571412in}}{\pgfqpoint{0.899511in}{1.577235in}}%
\pgfpathcurveto{\pgfqpoint{0.905335in}{1.583059in}}{\pgfqpoint{0.908607in}{1.590959in}}{\pgfqpoint{0.908607in}{1.599196in}}%
\pgfpathcurveto{\pgfqpoint{0.908607in}{1.607432in}}{\pgfqpoint{0.905335in}{1.615332in}}{\pgfqpoint{0.899511in}{1.621156in}}%
\pgfpathcurveto{\pgfqpoint{0.893687in}{1.626980in}}{\pgfqpoint{0.885787in}{1.630252in}}{\pgfqpoint{0.877551in}{1.630252in}}%
\pgfpathcurveto{\pgfqpoint{0.869314in}{1.630252in}}{\pgfqpoint{0.861414in}{1.626980in}}{\pgfqpoint{0.855590in}{1.621156in}}%
\pgfpathcurveto{\pgfqpoint{0.849766in}{1.615332in}}{\pgfqpoint{0.846494in}{1.607432in}}{\pgfqpoint{0.846494in}{1.599196in}}%
\pgfpathcurveto{\pgfqpoint{0.846494in}{1.590959in}}{\pgfqpoint{0.849766in}{1.583059in}}{\pgfqpoint{0.855590in}{1.577235in}}%
\pgfpathcurveto{\pgfqpoint{0.861414in}{1.571412in}}{\pgfqpoint{0.869314in}{1.568139in}}{\pgfqpoint{0.877551in}{1.568139in}}%
\pgfpathclose%
\pgfusepath{stroke,fill}%
\end{pgfscope}%
\begin{pgfscope}%
\pgfpathrectangle{\pgfqpoint{0.100000in}{0.220728in}}{\pgfqpoint{3.696000in}{3.696000in}}%
\pgfusepath{clip}%
\pgfsetbuttcap%
\pgfsetroundjoin%
\definecolor{currentfill}{rgb}{0.121569,0.466667,0.705882}%
\pgfsetfillcolor{currentfill}%
\pgfsetfillopacity{0.546657}%
\pgfsetlinewidth{1.003750pt}%
\definecolor{currentstroke}{rgb}{0.121569,0.466667,0.705882}%
\pgfsetstrokecolor{currentstroke}%
\pgfsetstrokeopacity{0.546657}%
\pgfsetdash{}{0pt}%
\pgfpathmoveto{\pgfqpoint{2.709388in}{2.971926in}}%
\pgfpathcurveto{\pgfqpoint{2.717624in}{2.971926in}}{\pgfqpoint{2.725524in}{2.975198in}}{\pgfqpoint{2.731348in}{2.981022in}}%
\pgfpathcurveto{\pgfqpoint{2.737172in}{2.986846in}}{\pgfqpoint{2.740445in}{2.994746in}}{\pgfqpoint{2.740445in}{3.002982in}}%
\pgfpathcurveto{\pgfqpoint{2.740445in}{3.011219in}}{\pgfqpoint{2.737172in}{3.019119in}}{\pgfqpoint{2.731348in}{3.024943in}}%
\pgfpathcurveto{\pgfqpoint{2.725524in}{3.030767in}}{\pgfqpoint{2.717624in}{3.034039in}}{\pgfqpoint{2.709388in}{3.034039in}}%
\pgfpathcurveto{\pgfqpoint{2.701152in}{3.034039in}}{\pgfqpoint{2.693252in}{3.030767in}}{\pgfqpoint{2.687428in}{3.024943in}}%
\pgfpathcurveto{\pgfqpoint{2.681604in}{3.019119in}}{\pgfqpoint{2.678332in}{3.011219in}}{\pgfqpoint{2.678332in}{3.002982in}}%
\pgfpathcurveto{\pgfqpoint{2.678332in}{2.994746in}}{\pgfqpoint{2.681604in}{2.986846in}}{\pgfqpoint{2.687428in}{2.981022in}}%
\pgfpathcurveto{\pgfqpoint{2.693252in}{2.975198in}}{\pgfqpoint{2.701152in}{2.971926in}}{\pgfqpoint{2.709388in}{2.971926in}}%
\pgfpathclose%
\pgfusepath{stroke,fill}%
\end{pgfscope}%
\begin{pgfscope}%
\pgfpathrectangle{\pgfqpoint{0.100000in}{0.220728in}}{\pgfqpoint{3.696000in}{3.696000in}}%
\pgfusepath{clip}%
\pgfsetbuttcap%
\pgfsetroundjoin%
\definecolor{currentfill}{rgb}{0.121569,0.466667,0.705882}%
\pgfsetfillcolor{currentfill}%
\pgfsetfillopacity{0.550881}%
\pgfsetlinewidth{1.003750pt}%
\definecolor{currentstroke}{rgb}{0.121569,0.466667,0.705882}%
\pgfsetstrokecolor{currentstroke}%
\pgfsetstrokeopacity{0.550881}%
\pgfsetdash{}{0pt}%
\pgfpathmoveto{\pgfqpoint{2.727295in}{2.969421in}}%
\pgfpathcurveto{\pgfqpoint{2.735532in}{2.969421in}}{\pgfqpoint{2.743432in}{2.972694in}}{\pgfqpoint{2.749256in}{2.978517in}}%
\pgfpathcurveto{\pgfqpoint{2.755080in}{2.984341in}}{\pgfqpoint{2.758352in}{2.992241in}}{\pgfqpoint{2.758352in}{3.000478in}}%
\pgfpathcurveto{\pgfqpoint{2.758352in}{3.008714in}}{\pgfqpoint{2.755080in}{3.016614in}}{\pgfqpoint{2.749256in}{3.022438in}}%
\pgfpathcurveto{\pgfqpoint{2.743432in}{3.028262in}}{\pgfqpoint{2.735532in}{3.031534in}}{\pgfqpoint{2.727295in}{3.031534in}}%
\pgfpathcurveto{\pgfqpoint{2.719059in}{3.031534in}}{\pgfqpoint{2.711159in}{3.028262in}}{\pgfqpoint{2.705335in}{3.022438in}}%
\pgfpathcurveto{\pgfqpoint{2.699511in}{3.016614in}}{\pgfqpoint{2.696239in}{3.008714in}}{\pgfqpoint{2.696239in}{3.000478in}}%
\pgfpathcurveto{\pgfqpoint{2.696239in}{2.992241in}}{\pgfqpoint{2.699511in}{2.984341in}}{\pgfqpoint{2.705335in}{2.978517in}}%
\pgfpathcurveto{\pgfqpoint{2.711159in}{2.972694in}}{\pgfqpoint{2.719059in}{2.969421in}}{\pgfqpoint{2.727295in}{2.969421in}}%
\pgfpathclose%
\pgfusepath{stroke,fill}%
\end{pgfscope}%
\begin{pgfscope}%
\pgfpathrectangle{\pgfqpoint{0.100000in}{0.220728in}}{\pgfqpoint{3.696000in}{3.696000in}}%
\pgfusepath{clip}%
\pgfsetbuttcap%
\pgfsetroundjoin%
\definecolor{currentfill}{rgb}{0.121569,0.466667,0.705882}%
\pgfsetfillcolor{currentfill}%
\pgfsetfillopacity{0.551259}%
\pgfsetlinewidth{1.003750pt}%
\definecolor{currentstroke}{rgb}{0.121569,0.466667,0.705882}%
\pgfsetstrokecolor{currentstroke}%
\pgfsetstrokeopacity{0.551259}%
\pgfsetdash{}{0pt}%
\pgfpathmoveto{\pgfqpoint{0.869845in}{1.544945in}}%
\pgfpathcurveto{\pgfqpoint{0.878081in}{1.544945in}}{\pgfqpoint{0.885981in}{1.548218in}}{\pgfqpoint{0.891805in}{1.554042in}}%
\pgfpathcurveto{\pgfqpoint{0.897629in}{1.559865in}}{\pgfqpoint{0.900901in}{1.567765in}}{\pgfqpoint{0.900901in}{1.576002in}}%
\pgfpathcurveto{\pgfqpoint{0.900901in}{1.584238in}}{\pgfqpoint{0.897629in}{1.592138in}}{\pgfqpoint{0.891805in}{1.597962in}}%
\pgfpathcurveto{\pgfqpoint{0.885981in}{1.603786in}}{\pgfqpoint{0.878081in}{1.607058in}}{\pgfqpoint{0.869845in}{1.607058in}}%
\pgfpathcurveto{\pgfqpoint{0.861609in}{1.607058in}}{\pgfqpoint{0.853708in}{1.603786in}}{\pgfqpoint{0.847885in}{1.597962in}}%
\pgfpathcurveto{\pgfqpoint{0.842061in}{1.592138in}}{\pgfqpoint{0.838788in}{1.584238in}}{\pgfqpoint{0.838788in}{1.576002in}}%
\pgfpathcurveto{\pgfqpoint{0.838788in}{1.567765in}}{\pgfqpoint{0.842061in}{1.559865in}}{\pgfqpoint{0.847885in}{1.554042in}}%
\pgfpathcurveto{\pgfqpoint{0.853708in}{1.548218in}}{\pgfqpoint{0.861609in}{1.544945in}}{\pgfqpoint{0.869845in}{1.544945in}}%
\pgfpathclose%
\pgfusepath{stroke,fill}%
\end{pgfscope}%
\begin{pgfscope}%
\pgfpathrectangle{\pgfqpoint{0.100000in}{0.220728in}}{\pgfqpoint{3.696000in}{3.696000in}}%
\pgfusepath{clip}%
\pgfsetbuttcap%
\pgfsetroundjoin%
\definecolor{currentfill}{rgb}{0.121569,0.466667,0.705882}%
\pgfsetfillcolor{currentfill}%
\pgfsetfillopacity{0.553263}%
\pgfsetlinewidth{1.003750pt}%
\definecolor{currentstroke}{rgb}{0.121569,0.466667,0.705882}%
\pgfsetstrokecolor{currentstroke}%
\pgfsetstrokeopacity{0.553263}%
\pgfsetdash{}{0pt}%
\pgfpathmoveto{\pgfqpoint{2.737015in}{2.967908in}}%
\pgfpathcurveto{\pgfqpoint{2.745251in}{2.967908in}}{\pgfqpoint{2.753151in}{2.971180in}}{\pgfqpoint{2.758975in}{2.977004in}}%
\pgfpathcurveto{\pgfqpoint{2.764799in}{2.982828in}}{\pgfqpoint{2.768071in}{2.990728in}}{\pgfqpoint{2.768071in}{2.998965in}}%
\pgfpathcurveto{\pgfqpoint{2.768071in}{3.007201in}}{\pgfqpoint{2.764799in}{3.015101in}}{\pgfqpoint{2.758975in}{3.020925in}}%
\pgfpathcurveto{\pgfqpoint{2.753151in}{3.026749in}}{\pgfqpoint{2.745251in}{3.030021in}}{\pgfqpoint{2.737015in}{3.030021in}}%
\pgfpathcurveto{\pgfqpoint{2.728779in}{3.030021in}}{\pgfqpoint{2.720878in}{3.026749in}}{\pgfqpoint{2.715055in}{3.020925in}}%
\pgfpathcurveto{\pgfqpoint{2.709231in}{3.015101in}}{\pgfqpoint{2.705958in}{3.007201in}}{\pgfqpoint{2.705958in}{2.998965in}}%
\pgfpathcurveto{\pgfqpoint{2.705958in}{2.990728in}}{\pgfqpoint{2.709231in}{2.982828in}}{\pgfqpoint{2.715055in}{2.977004in}}%
\pgfpathcurveto{\pgfqpoint{2.720878in}{2.971180in}}{\pgfqpoint{2.728779in}{2.967908in}}{\pgfqpoint{2.737015in}{2.967908in}}%
\pgfpathclose%
\pgfusepath{stroke,fill}%
\end{pgfscope}%
\begin{pgfscope}%
\pgfpathrectangle{\pgfqpoint{0.100000in}{0.220728in}}{\pgfqpoint{3.696000in}{3.696000in}}%
\pgfusepath{clip}%
\pgfsetbuttcap%
\pgfsetroundjoin%
\definecolor{currentfill}{rgb}{0.121569,0.466667,0.705882}%
\pgfsetfillcolor{currentfill}%
\pgfsetfillopacity{0.554727}%
\pgfsetlinewidth{1.003750pt}%
\definecolor{currentstroke}{rgb}{0.121569,0.466667,0.705882}%
\pgfsetstrokecolor{currentstroke}%
\pgfsetstrokeopacity{0.554727}%
\pgfsetdash{}{0pt}%
\pgfpathmoveto{\pgfqpoint{0.857642in}{1.525532in}}%
\pgfpathcurveto{\pgfqpoint{0.865878in}{1.525532in}}{\pgfqpoint{0.873778in}{1.528804in}}{\pgfqpoint{0.879602in}{1.534628in}}%
\pgfpathcurveto{\pgfqpoint{0.885426in}{1.540452in}}{\pgfqpoint{0.888698in}{1.548352in}}{\pgfqpoint{0.888698in}{1.556588in}}%
\pgfpathcurveto{\pgfqpoint{0.888698in}{1.564825in}}{\pgfqpoint{0.885426in}{1.572725in}}{\pgfqpoint{0.879602in}{1.578549in}}%
\pgfpathcurveto{\pgfqpoint{0.873778in}{1.584372in}}{\pgfqpoint{0.865878in}{1.587645in}}{\pgfqpoint{0.857642in}{1.587645in}}%
\pgfpathcurveto{\pgfqpoint{0.849406in}{1.587645in}}{\pgfqpoint{0.841506in}{1.584372in}}{\pgfqpoint{0.835682in}{1.578549in}}%
\pgfpathcurveto{\pgfqpoint{0.829858in}{1.572725in}}{\pgfqpoint{0.826585in}{1.564825in}}{\pgfqpoint{0.826585in}{1.556588in}}%
\pgfpathcurveto{\pgfqpoint{0.826585in}{1.548352in}}{\pgfqpoint{0.829858in}{1.540452in}}{\pgfqpoint{0.835682in}{1.534628in}}%
\pgfpathcurveto{\pgfqpoint{0.841506in}{1.528804in}}{\pgfqpoint{0.849406in}{1.525532in}}{\pgfqpoint{0.857642in}{1.525532in}}%
\pgfpathclose%
\pgfusepath{stroke,fill}%
\end{pgfscope}%
\begin{pgfscope}%
\pgfpathrectangle{\pgfqpoint{0.100000in}{0.220728in}}{\pgfqpoint{3.696000in}{3.696000in}}%
\pgfusepath{clip}%
\pgfsetbuttcap%
\pgfsetroundjoin%
\definecolor{currentfill}{rgb}{0.121569,0.466667,0.705882}%
\pgfsetfillcolor{currentfill}%
\pgfsetfillopacity{0.556256}%
\pgfsetlinewidth{1.003750pt}%
\definecolor{currentstroke}{rgb}{0.121569,0.466667,0.705882}%
\pgfsetstrokecolor{currentstroke}%
\pgfsetstrokeopacity{0.556256}%
\pgfsetdash{}{0pt}%
\pgfpathmoveto{\pgfqpoint{0.854061in}{1.518114in}}%
\pgfpathcurveto{\pgfqpoint{0.862297in}{1.518114in}}{\pgfqpoint{0.870197in}{1.521386in}}{\pgfqpoint{0.876021in}{1.527210in}}%
\pgfpathcurveto{\pgfqpoint{0.881845in}{1.533034in}}{\pgfqpoint{0.885118in}{1.540934in}}{\pgfqpoint{0.885118in}{1.549171in}}%
\pgfpathcurveto{\pgfqpoint{0.885118in}{1.557407in}}{\pgfqpoint{0.881845in}{1.565307in}}{\pgfqpoint{0.876021in}{1.571131in}}%
\pgfpathcurveto{\pgfqpoint{0.870197in}{1.576955in}}{\pgfqpoint{0.862297in}{1.580227in}}{\pgfqpoint{0.854061in}{1.580227in}}%
\pgfpathcurveto{\pgfqpoint{0.845825in}{1.580227in}}{\pgfqpoint{0.837925in}{1.576955in}}{\pgfqpoint{0.832101in}{1.571131in}}%
\pgfpathcurveto{\pgfqpoint{0.826277in}{1.565307in}}{\pgfqpoint{0.823005in}{1.557407in}}{\pgfqpoint{0.823005in}{1.549171in}}%
\pgfpathcurveto{\pgfqpoint{0.823005in}{1.540934in}}{\pgfqpoint{0.826277in}{1.533034in}}{\pgfqpoint{0.832101in}{1.527210in}}%
\pgfpathcurveto{\pgfqpoint{0.837925in}{1.521386in}}{\pgfqpoint{0.845825in}{1.518114in}}{\pgfqpoint{0.854061in}{1.518114in}}%
\pgfpathclose%
\pgfusepath{stroke,fill}%
\end{pgfscope}%
\begin{pgfscope}%
\pgfpathrectangle{\pgfqpoint{0.100000in}{0.220728in}}{\pgfqpoint{3.696000in}{3.696000in}}%
\pgfusepath{clip}%
\pgfsetbuttcap%
\pgfsetroundjoin%
\definecolor{currentfill}{rgb}{0.121569,0.466667,0.705882}%
\pgfsetfillcolor{currentfill}%
\pgfsetfillopacity{0.556921}%
\pgfsetlinewidth{1.003750pt}%
\definecolor{currentstroke}{rgb}{0.121569,0.466667,0.705882}%
\pgfsetstrokecolor{currentstroke}%
\pgfsetstrokeopacity{0.556921}%
\pgfsetdash{}{0pt}%
\pgfpathmoveto{\pgfqpoint{2.752662in}{2.966200in}}%
\pgfpathcurveto{\pgfqpoint{2.760898in}{2.966200in}}{\pgfqpoint{2.768798in}{2.969472in}}{\pgfqpoint{2.774622in}{2.975296in}}%
\pgfpathcurveto{\pgfqpoint{2.780446in}{2.981120in}}{\pgfqpoint{2.783719in}{2.989020in}}{\pgfqpoint{2.783719in}{2.997256in}}%
\pgfpathcurveto{\pgfqpoint{2.783719in}{3.005493in}}{\pgfqpoint{2.780446in}{3.013393in}}{\pgfqpoint{2.774622in}{3.019217in}}%
\pgfpathcurveto{\pgfqpoint{2.768798in}{3.025041in}}{\pgfqpoint{2.760898in}{3.028313in}}{\pgfqpoint{2.752662in}{3.028313in}}%
\pgfpathcurveto{\pgfqpoint{2.744426in}{3.028313in}}{\pgfqpoint{2.736526in}{3.025041in}}{\pgfqpoint{2.730702in}{3.019217in}}%
\pgfpathcurveto{\pgfqpoint{2.724878in}{3.013393in}}{\pgfqpoint{2.721606in}{3.005493in}}{\pgfqpoint{2.721606in}{2.997256in}}%
\pgfpathcurveto{\pgfqpoint{2.721606in}{2.989020in}}{\pgfqpoint{2.724878in}{2.981120in}}{\pgfqpoint{2.730702in}{2.975296in}}%
\pgfpathcurveto{\pgfqpoint{2.736526in}{2.969472in}}{\pgfqpoint{2.744426in}{2.966200in}}{\pgfqpoint{2.752662in}{2.966200in}}%
\pgfpathclose%
\pgfusepath{stroke,fill}%
\end{pgfscope}%
\begin{pgfscope}%
\pgfpathrectangle{\pgfqpoint{0.100000in}{0.220728in}}{\pgfqpoint{3.696000in}{3.696000in}}%
\pgfusepath{clip}%
\pgfsetbuttcap%
\pgfsetroundjoin%
\definecolor{currentfill}{rgb}{0.121569,0.466667,0.705882}%
\pgfsetfillcolor{currentfill}%
\pgfsetfillopacity{0.558789}%
\pgfsetlinewidth{1.003750pt}%
\definecolor{currentstroke}{rgb}{0.121569,0.466667,0.705882}%
\pgfsetstrokecolor{currentstroke}%
\pgfsetstrokeopacity{0.558789}%
\pgfsetdash{}{0pt}%
\pgfpathmoveto{\pgfqpoint{2.760981in}{2.963711in}}%
\pgfpathcurveto{\pgfqpoint{2.769217in}{2.963711in}}{\pgfqpoint{2.777117in}{2.966983in}}{\pgfqpoint{2.782941in}{2.972807in}}%
\pgfpathcurveto{\pgfqpoint{2.788765in}{2.978631in}}{\pgfqpoint{2.792037in}{2.986531in}}{\pgfqpoint{2.792037in}{2.994767in}}%
\pgfpathcurveto{\pgfqpoint{2.792037in}{3.003004in}}{\pgfqpoint{2.788765in}{3.010904in}}{\pgfqpoint{2.782941in}{3.016728in}}%
\pgfpathcurveto{\pgfqpoint{2.777117in}{3.022552in}}{\pgfqpoint{2.769217in}{3.025824in}}{\pgfqpoint{2.760981in}{3.025824in}}%
\pgfpathcurveto{\pgfqpoint{2.752745in}{3.025824in}}{\pgfqpoint{2.744845in}{3.022552in}}{\pgfqpoint{2.739021in}{3.016728in}}%
\pgfpathcurveto{\pgfqpoint{2.733197in}{3.010904in}}{\pgfqpoint{2.729924in}{3.003004in}}{\pgfqpoint{2.729924in}{2.994767in}}%
\pgfpathcurveto{\pgfqpoint{2.729924in}{2.986531in}}{\pgfqpoint{2.733197in}{2.978631in}}{\pgfqpoint{2.739021in}{2.972807in}}%
\pgfpathcurveto{\pgfqpoint{2.744845in}{2.966983in}}{\pgfqpoint{2.752745in}{2.963711in}}{\pgfqpoint{2.760981in}{2.963711in}}%
\pgfpathclose%
\pgfusepath{stroke,fill}%
\end{pgfscope}%
\begin{pgfscope}%
\pgfpathrectangle{\pgfqpoint{0.100000in}{0.220728in}}{\pgfqpoint{3.696000in}{3.696000in}}%
\pgfusepath{clip}%
\pgfsetbuttcap%
\pgfsetroundjoin%
\definecolor{currentfill}{rgb}{0.121569,0.466667,0.705882}%
\pgfsetfillcolor{currentfill}%
\pgfsetfillopacity{0.559037}%
\pgfsetlinewidth{1.003750pt}%
\definecolor{currentstroke}{rgb}{0.121569,0.466667,0.705882}%
\pgfsetstrokecolor{currentstroke}%
\pgfsetstrokeopacity{0.559037}%
\pgfsetdash{}{0pt}%
\pgfpathmoveto{\pgfqpoint{0.847374in}{1.504810in}}%
\pgfpathcurveto{\pgfqpoint{0.855610in}{1.504810in}}{\pgfqpoint{0.863510in}{1.508082in}}{\pgfqpoint{0.869334in}{1.513906in}}%
\pgfpathcurveto{\pgfqpoint{0.875158in}{1.519730in}}{\pgfqpoint{0.878430in}{1.527630in}}{\pgfqpoint{0.878430in}{1.535867in}}%
\pgfpathcurveto{\pgfqpoint{0.878430in}{1.544103in}}{\pgfqpoint{0.875158in}{1.552003in}}{\pgfqpoint{0.869334in}{1.557827in}}%
\pgfpathcurveto{\pgfqpoint{0.863510in}{1.563651in}}{\pgfqpoint{0.855610in}{1.566923in}}{\pgfqpoint{0.847374in}{1.566923in}}%
\pgfpathcurveto{\pgfqpoint{0.839138in}{1.566923in}}{\pgfqpoint{0.831238in}{1.563651in}}{\pgfqpoint{0.825414in}{1.557827in}}%
\pgfpathcurveto{\pgfqpoint{0.819590in}{1.552003in}}{\pgfqpoint{0.816317in}{1.544103in}}{\pgfqpoint{0.816317in}{1.535867in}}%
\pgfpathcurveto{\pgfqpoint{0.816317in}{1.527630in}}{\pgfqpoint{0.819590in}{1.519730in}}{\pgfqpoint{0.825414in}{1.513906in}}%
\pgfpathcurveto{\pgfqpoint{0.831238in}{1.508082in}}{\pgfqpoint{0.839138in}{1.504810in}}{\pgfqpoint{0.847374in}{1.504810in}}%
\pgfpathclose%
\pgfusepath{stroke,fill}%
\end{pgfscope}%
\begin{pgfscope}%
\pgfpathrectangle{\pgfqpoint{0.100000in}{0.220728in}}{\pgfqpoint{3.696000in}{3.696000in}}%
\pgfusepath{clip}%
\pgfsetbuttcap%
\pgfsetroundjoin%
\definecolor{currentfill}{rgb}{0.121569,0.466667,0.705882}%
\pgfsetfillcolor{currentfill}%
\pgfsetfillopacity{0.560102}%
\pgfsetlinewidth{1.003750pt}%
\definecolor{currentstroke}{rgb}{0.121569,0.466667,0.705882}%
\pgfsetstrokecolor{currentstroke}%
\pgfsetstrokeopacity{0.560102}%
\pgfsetdash{}{0pt}%
\pgfpathmoveto{\pgfqpoint{2.765414in}{2.963327in}}%
\pgfpathcurveto{\pgfqpoint{2.773650in}{2.963327in}}{\pgfqpoint{2.781550in}{2.966599in}}{\pgfqpoint{2.787374in}{2.972423in}}%
\pgfpathcurveto{\pgfqpoint{2.793198in}{2.978247in}}{\pgfqpoint{2.796471in}{2.986147in}}{\pgfqpoint{2.796471in}{2.994383in}}%
\pgfpathcurveto{\pgfqpoint{2.796471in}{3.002620in}}{\pgfqpoint{2.793198in}{3.010520in}}{\pgfqpoint{2.787374in}{3.016344in}}%
\pgfpathcurveto{\pgfqpoint{2.781550in}{3.022168in}}{\pgfqpoint{2.773650in}{3.025440in}}{\pgfqpoint{2.765414in}{3.025440in}}%
\pgfpathcurveto{\pgfqpoint{2.757178in}{3.025440in}}{\pgfqpoint{2.749278in}{3.022168in}}{\pgfqpoint{2.743454in}{3.016344in}}%
\pgfpathcurveto{\pgfqpoint{2.737630in}{3.010520in}}{\pgfqpoint{2.734358in}{3.002620in}}{\pgfqpoint{2.734358in}{2.994383in}}%
\pgfpathcurveto{\pgfqpoint{2.734358in}{2.986147in}}{\pgfqpoint{2.737630in}{2.978247in}}{\pgfqpoint{2.743454in}{2.972423in}}%
\pgfpathcurveto{\pgfqpoint{2.749278in}{2.966599in}}{\pgfqpoint{2.757178in}{2.963327in}}{\pgfqpoint{2.765414in}{2.963327in}}%
\pgfpathclose%
\pgfusepath{stroke,fill}%
\end{pgfscope}%
\begin{pgfscope}%
\pgfpathrectangle{\pgfqpoint{0.100000in}{0.220728in}}{\pgfqpoint{3.696000in}{3.696000in}}%
\pgfusepath{clip}%
\pgfsetbuttcap%
\pgfsetroundjoin%
\definecolor{currentfill}{rgb}{0.121569,0.466667,0.705882}%
\pgfsetfillcolor{currentfill}%
\pgfsetfillopacity{0.561604}%
\pgfsetlinewidth{1.003750pt}%
\definecolor{currentstroke}{rgb}{0.121569,0.466667,0.705882}%
\pgfsetstrokecolor{currentstroke}%
\pgfsetstrokeopacity{0.561604}%
\pgfsetdash{}{0pt}%
\pgfpathmoveto{\pgfqpoint{2.771947in}{2.962542in}}%
\pgfpathcurveto{\pgfqpoint{2.780183in}{2.962542in}}{\pgfqpoint{2.788083in}{2.965814in}}{\pgfqpoint{2.793907in}{2.971638in}}%
\pgfpathcurveto{\pgfqpoint{2.799731in}{2.977462in}}{\pgfqpoint{2.803003in}{2.985362in}}{\pgfqpoint{2.803003in}{2.993599in}}%
\pgfpathcurveto{\pgfqpoint{2.803003in}{3.001835in}}{\pgfqpoint{2.799731in}{3.009735in}}{\pgfqpoint{2.793907in}{3.015559in}}%
\pgfpathcurveto{\pgfqpoint{2.788083in}{3.021383in}}{\pgfqpoint{2.780183in}{3.024655in}}{\pgfqpoint{2.771947in}{3.024655in}}%
\pgfpathcurveto{\pgfqpoint{2.763710in}{3.024655in}}{\pgfqpoint{2.755810in}{3.021383in}}{\pgfqpoint{2.749986in}{3.015559in}}%
\pgfpathcurveto{\pgfqpoint{2.744162in}{3.009735in}}{\pgfqpoint{2.740890in}{3.001835in}}{\pgfqpoint{2.740890in}{2.993599in}}%
\pgfpathcurveto{\pgfqpoint{2.740890in}{2.985362in}}{\pgfqpoint{2.744162in}{2.977462in}}{\pgfqpoint{2.749986in}{2.971638in}}%
\pgfpathcurveto{\pgfqpoint{2.755810in}{2.965814in}}{\pgfqpoint{2.763710in}{2.962542in}}{\pgfqpoint{2.771947in}{2.962542in}}%
\pgfpathclose%
\pgfusepath{stroke,fill}%
\end{pgfscope}%
\begin{pgfscope}%
\pgfpathrectangle{\pgfqpoint{0.100000in}{0.220728in}}{\pgfqpoint{3.696000in}{3.696000in}}%
\pgfusepath{clip}%
\pgfsetbuttcap%
\pgfsetroundjoin%
\definecolor{currentfill}{rgb}{0.121569,0.466667,0.705882}%
\pgfsetfillcolor{currentfill}%
\pgfsetfillopacity{0.563168}%
\pgfsetlinewidth{1.003750pt}%
\definecolor{currentstroke}{rgb}{0.121569,0.466667,0.705882}%
\pgfsetstrokecolor{currentstroke}%
\pgfsetstrokeopacity{0.563168}%
\pgfsetdash{}{0pt}%
\pgfpathmoveto{\pgfqpoint{0.829748in}{1.483154in}}%
\pgfpathcurveto{\pgfqpoint{0.837985in}{1.483154in}}{\pgfqpoint{0.845885in}{1.486426in}}{\pgfqpoint{0.851709in}{1.492250in}}%
\pgfpathcurveto{\pgfqpoint{0.857533in}{1.498074in}}{\pgfqpoint{0.860805in}{1.505974in}}{\pgfqpoint{0.860805in}{1.514210in}}%
\pgfpathcurveto{\pgfqpoint{0.860805in}{1.522447in}}{\pgfqpoint{0.857533in}{1.530347in}}{\pgfqpoint{0.851709in}{1.536171in}}%
\pgfpathcurveto{\pgfqpoint{0.845885in}{1.541994in}}{\pgfqpoint{0.837985in}{1.545267in}}{\pgfqpoint{0.829748in}{1.545267in}}%
\pgfpathcurveto{\pgfqpoint{0.821512in}{1.545267in}}{\pgfqpoint{0.813612in}{1.541994in}}{\pgfqpoint{0.807788in}{1.536171in}}%
\pgfpathcurveto{\pgfqpoint{0.801964in}{1.530347in}}{\pgfqpoint{0.798692in}{1.522447in}}{\pgfqpoint{0.798692in}{1.514210in}}%
\pgfpathcurveto{\pgfqpoint{0.798692in}{1.505974in}}{\pgfqpoint{0.801964in}{1.498074in}}{\pgfqpoint{0.807788in}{1.492250in}}%
\pgfpathcurveto{\pgfqpoint{0.813612in}{1.486426in}}{\pgfqpoint{0.821512in}{1.483154in}}{\pgfqpoint{0.829748in}{1.483154in}}%
\pgfpathclose%
\pgfusepath{stroke,fill}%
\end{pgfscope}%
\begin{pgfscope}%
\pgfpathrectangle{\pgfqpoint{0.100000in}{0.220728in}}{\pgfqpoint{3.696000in}{3.696000in}}%
\pgfusepath{clip}%
\pgfsetbuttcap%
\pgfsetroundjoin%
\definecolor{currentfill}{rgb}{0.121569,0.466667,0.705882}%
\pgfsetfillcolor{currentfill}%
\pgfsetfillopacity{0.563993}%
\pgfsetlinewidth{1.003750pt}%
\definecolor{currentstroke}{rgb}{0.121569,0.466667,0.705882}%
\pgfsetstrokecolor{currentstroke}%
\pgfsetstrokeopacity{0.563993}%
\pgfsetdash{}{0pt}%
\pgfpathmoveto{\pgfqpoint{2.780013in}{2.962110in}}%
\pgfpathcurveto{\pgfqpoint{2.788249in}{2.962110in}}{\pgfqpoint{2.796149in}{2.965382in}}{\pgfqpoint{2.801973in}{2.971206in}}%
\pgfpathcurveto{\pgfqpoint{2.807797in}{2.977030in}}{\pgfqpoint{2.811069in}{2.984930in}}{\pgfqpoint{2.811069in}{2.993167in}}%
\pgfpathcurveto{\pgfqpoint{2.811069in}{3.001403in}}{\pgfqpoint{2.807797in}{3.009303in}}{\pgfqpoint{2.801973in}{3.015127in}}%
\pgfpathcurveto{\pgfqpoint{2.796149in}{3.020951in}}{\pgfqpoint{2.788249in}{3.024223in}}{\pgfqpoint{2.780013in}{3.024223in}}%
\pgfpathcurveto{\pgfqpoint{2.771776in}{3.024223in}}{\pgfqpoint{2.763876in}{3.020951in}}{\pgfqpoint{2.758052in}{3.015127in}}%
\pgfpathcurveto{\pgfqpoint{2.752228in}{3.009303in}}{\pgfqpoint{2.748956in}{3.001403in}}{\pgfqpoint{2.748956in}{2.993167in}}%
\pgfpathcurveto{\pgfqpoint{2.748956in}{2.984930in}}{\pgfqpoint{2.752228in}{2.977030in}}{\pgfqpoint{2.758052in}{2.971206in}}%
\pgfpathcurveto{\pgfqpoint{2.763876in}{2.965382in}}{\pgfqpoint{2.771776in}{2.962110in}}{\pgfqpoint{2.780013in}{2.962110in}}%
\pgfpathclose%
\pgfusepath{stroke,fill}%
\end{pgfscope}%
\begin{pgfscope}%
\pgfpathrectangle{\pgfqpoint{0.100000in}{0.220728in}}{\pgfqpoint{3.696000in}{3.696000in}}%
\pgfusepath{clip}%
\pgfsetbuttcap%
\pgfsetroundjoin%
\definecolor{currentfill}{rgb}{0.121569,0.466667,0.705882}%
\pgfsetfillcolor{currentfill}%
\pgfsetfillopacity{0.566224}%
\pgfsetlinewidth{1.003750pt}%
\definecolor{currentstroke}{rgb}{0.121569,0.466667,0.705882}%
\pgfsetstrokecolor{currentstroke}%
\pgfsetstrokeopacity{0.566224}%
\pgfsetdash{}{0pt}%
\pgfpathmoveto{\pgfqpoint{2.793267in}{2.960216in}}%
\pgfpathcurveto{\pgfqpoint{2.801503in}{2.960216in}}{\pgfqpoint{2.809403in}{2.963488in}}{\pgfqpoint{2.815227in}{2.969312in}}%
\pgfpathcurveto{\pgfqpoint{2.821051in}{2.975136in}}{\pgfqpoint{2.824323in}{2.983036in}}{\pgfqpoint{2.824323in}{2.991273in}}%
\pgfpathcurveto{\pgfqpoint{2.824323in}{2.999509in}}{\pgfqpoint{2.821051in}{3.007409in}}{\pgfqpoint{2.815227in}{3.013233in}}%
\pgfpathcurveto{\pgfqpoint{2.809403in}{3.019057in}}{\pgfqpoint{2.801503in}{3.022329in}}{\pgfqpoint{2.793267in}{3.022329in}}%
\pgfpathcurveto{\pgfqpoint{2.785030in}{3.022329in}}{\pgfqpoint{2.777130in}{3.019057in}}{\pgfqpoint{2.771306in}{3.013233in}}%
\pgfpathcurveto{\pgfqpoint{2.765483in}{3.007409in}}{\pgfqpoint{2.762210in}{2.999509in}}{\pgfqpoint{2.762210in}{2.991273in}}%
\pgfpathcurveto{\pgfqpoint{2.762210in}{2.983036in}}{\pgfqpoint{2.765483in}{2.975136in}}{\pgfqpoint{2.771306in}{2.969312in}}%
\pgfpathcurveto{\pgfqpoint{2.777130in}{2.963488in}}{\pgfqpoint{2.785030in}{2.960216in}}{\pgfqpoint{2.793267in}{2.960216in}}%
\pgfpathclose%
\pgfusepath{stroke,fill}%
\end{pgfscope}%
\begin{pgfscope}%
\pgfpathrectangle{\pgfqpoint{0.100000in}{0.220728in}}{\pgfqpoint{3.696000in}{3.696000in}}%
\pgfusepath{clip}%
\pgfsetbuttcap%
\pgfsetroundjoin%
\definecolor{currentfill}{rgb}{0.121569,0.466667,0.705882}%
\pgfsetfillcolor{currentfill}%
\pgfsetfillopacity{0.568190}%
\pgfsetlinewidth{1.003750pt}%
\definecolor{currentstroke}{rgb}{0.121569,0.466667,0.705882}%
\pgfsetstrokecolor{currentstroke}%
\pgfsetstrokeopacity{0.568190}%
\pgfsetdash{}{0pt}%
\pgfpathmoveto{\pgfqpoint{2.799099in}{2.958628in}}%
\pgfpathcurveto{\pgfqpoint{2.807335in}{2.958628in}}{\pgfqpoint{2.815235in}{2.961901in}}{\pgfqpoint{2.821059in}{2.967724in}}%
\pgfpathcurveto{\pgfqpoint{2.826883in}{2.973548in}}{\pgfqpoint{2.830155in}{2.981448in}}{\pgfqpoint{2.830155in}{2.989685in}}%
\pgfpathcurveto{\pgfqpoint{2.830155in}{2.997921in}}{\pgfqpoint{2.826883in}{3.005821in}}{\pgfqpoint{2.821059in}{3.011645in}}%
\pgfpathcurveto{\pgfqpoint{2.815235in}{3.017469in}}{\pgfqpoint{2.807335in}{3.020741in}}{\pgfqpoint{2.799099in}{3.020741in}}%
\pgfpathcurveto{\pgfqpoint{2.790863in}{3.020741in}}{\pgfqpoint{2.782962in}{3.017469in}}{\pgfqpoint{2.777139in}{3.011645in}}%
\pgfpathcurveto{\pgfqpoint{2.771315in}{3.005821in}}{\pgfqpoint{2.768042in}{2.997921in}}{\pgfqpoint{2.768042in}{2.989685in}}%
\pgfpathcurveto{\pgfqpoint{2.768042in}{2.981448in}}{\pgfqpoint{2.771315in}{2.973548in}}{\pgfqpoint{2.777139in}{2.967724in}}%
\pgfpathcurveto{\pgfqpoint{2.782962in}{2.961901in}}{\pgfqpoint{2.790863in}{2.958628in}}{\pgfqpoint{2.799099in}{2.958628in}}%
\pgfpathclose%
\pgfusepath{stroke,fill}%
\end{pgfscope}%
\begin{pgfscope}%
\pgfpathrectangle{\pgfqpoint{0.100000in}{0.220728in}}{\pgfqpoint{3.696000in}{3.696000in}}%
\pgfusepath{clip}%
\pgfsetbuttcap%
\pgfsetroundjoin%
\definecolor{currentfill}{rgb}{0.121569,0.466667,0.705882}%
\pgfsetfillcolor{currentfill}%
\pgfsetfillopacity{0.569044}%
\pgfsetlinewidth{1.003750pt}%
\definecolor{currentstroke}{rgb}{0.121569,0.466667,0.705882}%
\pgfsetstrokecolor{currentstroke}%
\pgfsetstrokeopacity{0.569044}%
\pgfsetdash{}{0pt}%
\pgfpathmoveto{\pgfqpoint{2.802894in}{2.958136in}}%
\pgfpathcurveto{\pgfqpoint{2.811130in}{2.958136in}}{\pgfqpoint{2.819030in}{2.961409in}}{\pgfqpoint{2.824854in}{2.967233in}}%
\pgfpathcurveto{\pgfqpoint{2.830678in}{2.973056in}}{\pgfqpoint{2.833950in}{2.980957in}}{\pgfqpoint{2.833950in}{2.989193in}}%
\pgfpathcurveto{\pgfqpoint{2.833950in}{2.997429in}}{\pgfqpoint{2.830678in}{3.005329in}}{\pgfqpoint{2.824854in}{3.011153in}}%
\pgfpathcurveto{\pgfqpoint{2.819030in}{3.016977in}}{\pgfqpoint{2.811130in}{3.020249in}}{\pgfqpoint{2.802894in}{3.020249in}}%
\pgfpathcurveto{\pgfqpoint{2.794657in}{3.020249in}}{\pgfqpoint{2.786757in}{3.016977in}}{\pgfqpoint{2.780933in}{3.011153in}}%
\pgfpathcurveto{\pgfqpoint{2.775109in}{3.005329in}}{\pgfqpoint{2.771837in}{2.997429in}}{\pgfqpoint{2.771837in}{2.989193in}}%
\pgfpathcurveto{\pgfqpoint{2.771837in}{2.980957in}}{\pgfqpoint{2.775109in}{2.973056in}}{\pgfqpoint{2.780933in}{2.967233in}}%
\pgfpathcurveto{\pgfqpoint{2.786757in}{2.961409in}}{\pgfqpoint{2.794657in}{2.958136in}}{\pgfqpoint{2.802894in}{2.958136in}}%
\pgfpathclose%
\pgfusepath{stroke,fill}%
\end{pgfscope}%
\begin{pgfscope}%
\pgfpathrectangle{\pgfqpoint{0.100000in}{0.220728in}}{\pgfqpoint{3.696000in}{3.696000in}}%
\pgfusepath{clip}%
\pgfsetbuttcap%
\pgfsetroundjoin%
\definecolor{currentfill}{rgb}{0.121569,0.466667,0.705882}%
\pgfsetfillcolor{currentfill}%
\pgfsetfillopacity{0.569586}%
\pgfsetlinewidth{1.003750pt}%
\definecolor{currentstroke}{rgb}{0.121569,0.466667,0.705882}%
\pgfsetstrokecolor{currentstroke}%
\pgfsetstrokeopacity{0.569586}%
\pgfsetdash{}{0pt}%
\pgfpathmoveto{\pgfqpoint{2.804876in}{2.957903in}}%
\pgfpathcurveto{\pgfqpoint{2.813112in}{2.957903in}}{\pgfqpoint{2.821012in}{2.961176in}}{\pgfqpoint{2.826836in}{2.967000in}}%
\pgfpathcurveto{\pgfqpoint{2.832660in}{2.972823in}}{\pgfqpoint{2.835932in}{2.980724in}}{\pgfqpoint{2.835932in}{2.988960in}}%
\pgfpathcurveto{\pgfqpoint{2.835932in}{2.997196in}}{\pgfqpoint{2.832660in}{3.005096in}}{\pgfqpoint{2.826836in}{3.010920in}}%
\pgfpathcurveto{\pgfqpoint{2.821012in}{3.016744in}}{\pgfqpoint{2.813112in}{3.020016in}}{\pgfqpoint{2.804876in}{3.020016in}}%
\pgfpathcurveto{\pgfqpoint{2.796640in}{3.020016in}}{\pgfqpoint{2.788740in}{3.016744in}}{\pgfqpoint{2.782916in}{3.010920in}}%
\pgfpathcurveto{\pgfqpoint{2.777092in}{3.005096in}}{\pgfqpoint{2.773819in}{2.997196in}}{\pgfqpoint{2.773819in}{2.988960in}}%
\pgfpathcurveto{\pgfqpoint{2.773819in}{2.980724in}}{\pgfqpoint{2.777092in}{2.972823in}}{\pgfqpoint{2.782916in}{2.967000in}}%
\pgfpathcurveto{\pgfqpoint{2.788740in}{2.961176in}}{\pgfqpoint{2.796640in}{2.957903in}}{\pgfqpoint{2.804876in}{2.957903in}}%
\pgfpathclose%
\pgfusepath{stroke,fill}%
\end{pgfscope}%
\begin{pgfscope}%
\pgfpathrectangle{\pgfqpoint{0.100000in}{0.220728in}}{\pgfqpoint{3.696000in}{3.696000in}}%
\pgfusepath{clip}%
\pgfsetbuttcap%
\pgfsetroundjoin%
\definecolor{currentfill}{rgb}{0.121569,0.466667,0.705882}%
\pgfsetfillcolor{currentfill}%
\pgfsetfillopacity{0.571399}%
\pgfsetlinewidth{1.003750pt}%
\definecolor{currentstroke}{rgb}{0.121569,0.466667,0.705882}%
\pgfsetstrokecolor{currentstroke}%
\pgfsetstrokeopacity{0.571399}%
\pgfsetdash{}{0pt}%
\pgfpathmoveto{\pgfqpoint{2.811542in}{2.956823in}}%
\pgfpathcurveto{\pgfqpoint{2.819779in}{2.956823in}}{\pgfqpoint{2.827679in}{2.960095in}}{\pgfqpoint{2.833503in}{2.965919in}}%
\pgfpathcurveto{\pgfqpoint{2.839326in}{2.971743in}}{\pgfqpoint{2.842599in}{2.979643in}}{\pgfqpoint{2.842599in}{2.987879in}}%
\pgfpathcurveto{\pgfqpoint{2.842599in}{2.996116in}}{\pgfqpoint{2.839326in}{3.004016in}}{\pgfqpoint{2.833503in}{3.009840in}}%
\pgfpathcurveto{\pgfqpoint{2.827679in}{3.015664in}}{\pgfqpoint{2.819779in}{3.018936in}}{\pgfqpoint{2.811542in}{3.018936in}}%
\pgfpathcurveto{\pgfqpoint{2.803306in}{3.018936in}}{\pgfqpoint{2.795406in}{3.015664in}}{\pgfqpoint{2.789582in}{3.009840in}}%
\pgfpathcurveto{\pgfqpoint{2.783758in}{3.004016in}}{\pgfqpoint{2.780486in}{2.996116in}}{\pgfqpoint{2.780486in}{2.987879in}}%
\pgfpathcurveto{\pgfqpoint{2.780486in}{2.979643in}}{\pgfqpoint{2.783758in}{2.971743in}}{\pgfqpoint{2.789582in}{2.965919in}}%
\pgfpathcurveto{\pgfqpoint{2.795406in}{2.960095in}}{\pgfqpoint{2.803306in}{2.956823in}}{\pgfqpoint{2.811542in}{2.956823in}}%
\pgfpathclose%
\pgfusepath{stroke,fill}%
\end{pgfscope}%
\begin{pgfscope}%
\pgfpathrectangle{\pgfqpoint{0.100000in}{0.220728in}}{\pgfqpoint{3.696000in}{3.696000in}}%
\pgfusepath{clip}%
\pgfsetbuttcap%
\pgfsetroundjoin%
\definecolor{currentfill}{rgb}{0.121569,0.466667,0.705882}%
\pgfsetfillcolor{currentfill}%
\pgfsetfillopacity{0.572891}%
\pgfsetlinewidth{1.003750pt}%
\definecolor{currentstroke}{rgb}{0.121569,0.466667,0.705882}%
\pgfsetstrokecolor{currentstroke}%
\pgfsetstrokeopacity{0.572891}%
\pgfsetdash{}{0pt}%
\pgfpathmoveto{\pgfqpoint{0.810121in}{1.439347in}}%
\pgfpathcurveto{\pgfqpoint{0.818357in}{1.439347in}}{\pgfqpoint{0.826257in}{1.442620in}}{\pgfqpoint{0.832081in}{1.448444in}}%
\pgfpathcurveto{\pgfqpoint{0.837905in}{1.454268in}}{\pgfqpoint{0.841177in}{1.462168in}}{\pgfqpoint{0.841177in}{1.470404in}}%
\pgfpathcurveto{\pgfqpoint{0.841177in}{1.478640in}}{\pgfqpoint{0.837905in}{1.486540in}}{\pgfqpoint{0.832081in}{1.492364in}}%
\pgfpathcurveto{\pgfqpoint{0.826257in}{1.498188in}}{\pgfqpoint{0.818357in}{1.501460in}}{\pgfqpoint{0.810121in}{1.501460in}}%
\pgfpathcurveto{\pgfqpoint{0.801884in}{1.501460in}}{\pgfqpoint{0.793984in}{1.498188in}}{\pgfqpoint{0.788160in}{1.492364in}}%
\pgfpathcurveto{\pgfqpoint{0.782336in}{1.486540in}}{\pgfqpoint{0.779064in}{1.478640in}}{\pgfqpoint{0.779064in}{1.470404in}}%
\pgfpathcurveto{\pgfqpoint{0.779064in}{1.462168in}}{\pgfqpoint{0.782336in}{1.454268in}}{\pgfqpoint{0.788160in}{1.448444in}}%
\pgfpathcurveto{\pgfqpoint{0.793984in}{1.442620in}}{\pgfqpoint{0.801884in}{1.439347in}}{\pgfqpoint{0.810121in}{1.439347in}}%
\pgfpathclose%
\pgfusepath{stroke,fill}%
\end{pgfscope}%
\begin{pgfscope}%
\pgfpathrectangle{\pgfqpoint{0.100000in}{0.220728in}}{\pgfqpoint{3.696000in}{3.696000in}}%
\pgfusepath{clip}%
\pgfsetbuttcap%
\pgfsetroundjoin%
\definecolor{currentfill}{rgb}{0.121569,0.466667,0.705882}%
\pgfsetfillcolor{currentfill}%
\pgfsetfillopacity{0.573785}%
\pgfsetlinewidth{1.003750pt}%
\definecolor{currentstroke}{rgb}{0.121569,0.466667,0.705882}%
\pgfsetstrokecolor{currentstroke}%
\pgfsetstrokeopacity{0.573785}%
\pgfsetdash{}{0pt}%
\pgfpathmoveto{\pgfqpoint{2.825554in}{2.954728in}}%
\pgfpathcurveto{\pgfqpoint{2.833790in}{2.954728in}}{\pgfqpoint{2.841690in}{2.958000in}}{\pgfqpoint{2.847514in}{2.963824in}}%
\pgfpathcurveto{\pgfqpoint{2.853338in}{2.969648in}}{\pgfqpoint{2.856610in}{2.977548in}}{\pgfqpoint{2.856610in}{2.985785in}}%
\pgfpathcurveto{\pgfqpoint{2.856610in}{2.994021in}}{\pgfqpoint{2.853338in}{3.001921in}}{\pgfqpoint{2.847514in}{3.007745in}}%
\pgfpathcurveto{\pgfqpoint{2.841690in}{3.013569in}}{\pgfqpoint{2.833790in}{3.016841in}}{\pgfqpoint{2.825554in}{3.016841in}}%
\pgfpathcurveto{\pgfqpoint{2.817317in}{3.016841in}}{\pgfqpoint{2.809417in}{3.013569in}}{\pgfqpoint{2.803593in}{3.007745in}}%
\pgfpathcurveto{\pgfqpoint{2.797770in}{3.001921in}}{\pgfqpoint{2.794497in}{2.994021in}}{\pgfqpoint{2.794497in}{2.985785in}}%
\pgfpathcurveto{\pgfqpoint{2.794497in}{2.977548in}}{\pgfqpoint{2.797770in}{2.969648in}}{\pgfqpoint{2.803593in}{2.963824in}}%
\pgfpathcurveto{\pgfqpoint{2.809417in}{2.958000in}}{\pgfqpoint{2.817317in}{2.954728in}}{\pgfqpoint{2.825554in}{2.954728in}}%
\pgfpathclose%
\pgfusepath{stroke,fill}%
\end{pgfscope}%
\begin{pgfscope}%
\pgfpathrectangle{\pgfqpoint{0.100000in}{0.220728in}}{\pgfqpoint{3.696000in}{3.696000in}}%
\pgfusepath{clip}%
\pgfsetbuttcap%
\pgfsetroundjoin%
\definecolor{currentfill}{rgb}{0.121569,0.466667,0.705882}%
\pgfsetfillcolor{currentfill}%
\pgfsetfillopacity{0.579053}%
\pgfsetlinewidth{1.003750pt}%
\definecolor{currentstroke}{rgb}{0.121569,0.466667,0.705882}%
\pgfsetstrokecolor{currentstroke}%
\pgfsetstrokeopacity{0.579053}%
\pgfsetdash{}{0pt}%
\pgfpathmoveto{\pgfqpoint{2.844434in}{2.952950in}}%
\pgfpathcurveto{\pgfqpoint{2.852671in}{2.952950in}}{\pgfqpoint{2.860571in}{2.956222in}}{\pgfqpoint{2.866395in}{2.962046in}}%
\pgfpathcurveto{\pgfqpoint{2.872218in}{2.967870in}}{\pgfqpoint{2.875491in}{2.975770in}}{\pgfqpoint{2.875491in}{2.984007in}}%
\pgfpathcurveto{\pgfqpoint{2.875491in}{2.992243in}}{\pgfqpoint{2.872218in}{3.000143in}}{\pgfqpoint{2.866395in}{3.005967in}}%
\pgfpathcurveto{\pgfqpoint{2.860571in}{3.011791in}}{\pgfqpoint{2.852671in}{3.015063in}}{\pgfqpoint{2.844434in}{3.015063in}}%
\pgfpathcurveto{\pgfqpoint{2.836198in}{3.015063in}}{\pgfqpoint{2.828298in}{3.011791in}}{\pgfqpoint{2.822474in}{3.005967in}}%
\pgfpathcurveto{\pgfqpoint{2.816650in}{3.000143in}}{\pgfqpoint{2.813378in}{2.992243in}}{\pgfqpoint{2.813378in}{2.984007in}}%
\pgfpathcurveto{\pgfqpoint{2.813378in}{2.975770in}}{\pgfqpoint{2.816650in}{2.967870in}}{\pgfqpoint{2.822474in}{2.962046in}}%
\pgfpathcurveto{\pgfqpoint{2.828298in}{2.956222in}}{\pgfqpoint{2.836198in}{2.952950in}}{\pgfqpoint{2.844434in}{2.952950in}}%
\pgfpathclose%
\pgfusepath{stroke,fill}%
\end{pgfscope}%
\begin{pgfscope}%
\pgfpathrectangle{\pgfqpoint{0.100000in}{0.220728in}}{\pgfqpoint{3.696000in}{3.696000in}}%
\pgfusepath{clip}%
\pgfsetbuttcap%
\pgfsetroundjoin%
\definecolor{currentfill}{rgb}{0.121569,0.466667,0.705882}%
\pgfsetfillcolor{currentfill}%
\pgfsetfillopacity{0.579448}%
\pgfsetlinewidth{1.003750pt}%
\definecolor{currentstroke}{rgb}{0.121569,0.466667,0.705882}%
\pgfsetstrokecolor{currentstroke}%
\pgfsetstrokeopacity{0.579448}%
\pgfsetdash{}{0pt}%
\pgfpathmoveto{\pgfqpoint{0.780815in}{1.401739in}}%
\pgfpathcurveto{\pgfqpoint{0.789051in}{1.401739in}}{\pgfqpoint{0.796951in}{1.405011in}}{\pgfqpoint{0.802775in}{1.410835in}}%
\pgfpathcurveto{\pgfqpoint{0.808599in}{1.416659in}}{\pgfqpoint{0.811871in}{1.424559in}}{\pgfqpoint{0.811871in}{1.432796in}}%
\pgfpathcurveto{\pgfqpoint{0.811871in}{1.441032in}}{\pgfqpoint{0.808599in}{1.448932in}}{\pgfqpoint{0.802775in}{1.454756in}}%
\pgfpathcurveto{\pgfqpoint{0.796951in}{1.460580in}}{\pgfqpoint{0.789051in}{1.463852in}}{\pgfqpoint{0.780815in}{1.463852in}}%
\pgfpathcurveto{\pgfqpoint{0.772578in}{1.463852in}}{\pgfqpoint{0.764678in}{1.460580in}}{\pgfqpoint{0.758854in}{1.454756in}}%
\pgfpathcurveto{\pgfqpoint{0.753030in}{1.448932in}}{\pgfqpoint{0.749758in}{1.441032in}}{\pgfqpoint{0.749758in}{1.432796in}}%
\pgfpathcurveto{\pgfqpoint{0.749758in}{1.424559in}}{\pgfqpoint{0.753030in}{1.416659in}}{\pgfqpoint{0.758854in}{1.410835in}}%
\pgfpathcurveto{\pgfqpoint{0.764678in}{1.405011in}}{\pgfqpoint{0.772578in}{1.401739in}}{\pgfqpoint{0.780815in}{1.401739in}}%
\pgfpathclose%
\pgfusepath{stroke,fill}%
\end{pgfscope}%
\begin{pgfscope}%
\pgfpathrectangle{\pgfqpoint{0.100000in}{0.220728in}}{\pgfqpoint{3.696000in}{3.696000in}}%
\pgfusepath{clip}%
\pgfsetbuttcap%
\pgfsetroundjoin%
\definecolor{currentfill}{rgb}{0.121569,0.466667,0.705882}%
\pgfsetfillcolor{currentfill}%
\pgfsetfillopacity{0.583062}%
\pgfsetlinewidth{1.003750pt}%
\definecolor{currentstroke}{rgb}{0.121569,0.466667,0.705882}%
\pgfsetstrokecolor{currentstroke}%
\pgfsetstrokeopacity{0.583062}%
\pgfsetdash{}{0pt}%
\pgfpathmoveto{\pgfqpoint{2.868918in}{2.949765in}}%
\pgfpathcurveto{\pgfqpoint{2.877154in}{2.949765in}}{\pgfqpoint{2.885054in}{2.953037in}}{\pgfqpoint{2.890878in}{2.958861in}}%
\pgfpathcurveto{\pgfqpoint{2.896702in}{2.964685in}}{\pgfqpoint{2.899974in}{2.972585in}}{\pgfqpoint{2.899974in}{2.980821in}}%
\pgfpathcurveto{\pgfqpoint{2.899974in}{2.989058in}}{\pgfqpoint{2.896702in}{2.996958in}}{\pgfqpoint{2.890878in}{3.002782in}}%
\pgfpathcurveto{\pgfqpoint{2.885054in}{3.008605in}}{\pgfqpoint{2.877154in}{3.011878in}}{\pgfqpoint{2.868918in}{3.011878in}}%
\pgfpathcurveto{\pgfqpoint{2.860682in}{3.011878in}}{\pgfqpoint{2.852781in}{3.008605in}}{\pgfqpoint{2.846958in}{3.002782in}}%
\pgfpathcurveto{\pgfqpoint{2.841134in}{2.996958in}}{\pgfqpoint{2.837861in}{2.989058in}}{\pgfqpoint{2.837861in}{2.980821in}}%
\pgfpathcurveto{\pgfqpoint{2.837861in}{2.972585in}}{\pgfqpoint{2.841134in}{2.964685in}}{\pgfqpoint{2.846958in}{2.958861in}}%
\pgfpathcurveto{\pgfqpoint{2.852781in}{2.953037in}}{\pgfqpoint{2.860682in}{2.949765in}}{\pgfqpoint{2.868918in}{2.949765in}}%
\pgfpathclose%
\pgfusepath{stroke,fill}%
\end{pgfscope}%
\begin{pgfscope}%
\pgfpathrectangle{\pgfqpoint{0.100000in}{0.220728in}}{\pgfqpoint{3.696000in}{3.696000in}}%
\pgfusepath{clip}%
\pgfsetbuttcap%
\pgfsetroundjoin%
\definecolor{currentfill}{rgb}{0.121569,0.466667,0.705882}%
\pgfsetfillcolor{currentfill}%
\pgfsetfillopacity{0.586465}%
\pgfsetlinewidth{1.003750pt}%
\definecolor{currentstroke}{rgb}{0.121569,0.466667,0.705882}%
\pgfsetstrokecolor{currentstroke}%
\pgfsetstrokeopacity{0.586465}%
\pgfsetdash{}{0pt}%
\pgfpathmoveto{\pgfqpoint{0.770520in}{1.369896in}}%
\pgfpathcurveto{\pgfqpoint{0.778756in}{1.369896in}}{\pgfqpoint{0.786656in}{1.373168in}}{\pgfqpoint{0.792480in}{1.378992in}}%
\pgfpathcurveto{\pgfqpoint{0.798304in}{1.384816in}}{\pgfqpoint{0.801577in}{1.392716in}}{\pgfqpoint{0.801577in}{1.400952in}}%
\pgfpathcurveto{\pgfqpoint{0.801577in}{1.409189in}}{\pgfqpoint{0.798304in}{1.417089in}}{\pgfqpoint{0.792480in}{1.422913in}}%
\pgfpathcurveto{\pgfqpoint{0.786656in}{1.428737in}}{\pgfqpoint{0.778756in}{1.432009in}}{\pgfqpoint{0.770520in}{1.432009in}}%
\pgfpathcurveto{\pgfqpoint{0.762284in}{1.432009in}}{\pgfqpoint{0.754384in}{1.428737in}}{\pgfqpoint{0.748560in}{1.422913in}}%
\pgfpathcurveto{\pgfqpoint{0.742736in}{1.417089in}}{\pgfqpoint{0.739464in}{1.409189in}}{\pgfqpoint{0.739464in}{1.400952in}}%
\pgfpathcurveto{\pgfqpoint{0.739464in}{1.392716in}}{\pgfqpoint{0.742736in}{1.384816in}}{\pgfqpoint{0.748560in}{1.378992in}}%
\pgfpathcurveto{\pgfqpoint{0.754384in}{1.373168in}}{\pgfqpoint{0.762284in}{1.369896in}}{\pgfqpoint{0.770520in}{1.369896in}}%
\pgfpathclose%
\pgfusepath{stroke,fill}%
\end{pgfscope}%
\begin{pgfscope}%
\pgfpathrectangle{\pgfqpoint{0.100000in}{0.220728in}}{\pgfqpoint{3.696000in}{3.696000in}}%
\pgfusepath{clip}%
\pgfsetbuttcap%
\pgfsetroundjoin%
\definecolor{currentfill}{rgb}{0.121569,0.466667,0.705882}%
\pgfsetfillcolor{currentfill}%
\pgfsetfillopacity{0.586895}%
\pgfsetlinewidth{1.003750pt}%
\definecolor{currentstroke}{rgb}{0.121569,0.466667,0.705882}%
\pgfsetstrokecolor{currentstroke}%
\pgfsetstrokeopacity{0.586895}%
\pgfsetdash{}{0pt}%
\pgfpathmoveto{\pgfqpoint{2.879509in}{2.947649in}}%
\pgfpathcurveto{\pgfqpoint{2.887745in}{2.947649in}}{\pgfqpoint{2.895645in}{2.950921in}}{\pgfqpoint{2.901469in}{2.956745in}}%
\pgfpathcurveto{\pgfqpoint{2.907293in}{2.962569in}}{\pgfqpoint{2.910565in}{2.970469in}}{\pgfqpoint{2.910565in}{2.978705in}}%
\pgfpathcurveto{\pgfqpoint{2.910565in}{2.986942in}}{\pgfqpoint{2.907293in}{2.994842in}}{\pgfqpoint{2.901469in}{3.000666in}}%
\pgfpathcurveto{\pgfqpoint{2.895645in}{3.006490in}}{\pgfqpoint{2.887745in}{3.009762in}}{\pgfqpoint{2.879509in}{3.009762in}}%
\pgfpathcurveto{\pgfqpoint{2.871272in}{3.009762in}}{\pgfqpoint{2.863372in}{3.006490in}}{\pgfqpoint{2.857548in}{3.000666in}}%
\pgfpathcurveto{\pgfqpoint{2.851724in}{2.994842in}}{\pgfqpoint{2.848452in}{2.986942in}}{\pgfqpoint{2.848452in}{2.978705in}}%
\pgfpathcurveto{\pgfqpoint{2.848452in}{2.970469in}}{\pgfqpoint{2.851724in}{2.962569in}}{\pgfqpoint{2.857548in}{2.956745in}}%
\pgfpathcurveto{\pgfqpoint{2.863372in}{2.950921in}}{\pgfqpoint{2.871272in}{2.947649in}}{\pgfqpoint{2.879509in}{2.947649in}}%
\pgfpathclose%
\pgfusepath{stroke,fill}%
\end{pgfscope}%
\begin{pgfscope}%
\pgfpathrectangle{\pgfqpoint{0.100000in}{0.220728in}}{\pgfqpoint{3.696000in}{3.696000in}}%
\pgfusepath{clip}%
\pgfsetbuttcap%
\pgfsetroundjoin%
\definecolor{currentfill}{rgb}{0.121569,0.466667,0.705882}%
\pgfsetfillcolor{currentfill}%
\pgfsetfillopacity{0.587768}%
\pgfsetlinewidth{1.003750pt}%
\definecolor{currentstroke}{rgb}{0.121569,0.466667,0.705882}%
\pgfsetstrokecolor{currentstroke}%
\pgfsetstrokeopacity{0.587768}%
\pgfsetdash{}{0pt}%
\pgfpathmoveto{\pgfqpoint{0.854595in}{1.389376in}}%
\pgfpathcurveto{\pgfqpoint{0.862831in}{1.389376in}}{\pgfqpoint{0.870732in}{1.392648in}}{\pgfqpoint{0.876555in}{1.398472in}}%
\pgfpathcurveto{\pgfqpoint{0.882379in}{1.404296in}}{\pgfqpoint{0.885652in}{1.412196in}}{\pgfqpoint{0.885652in}{1.420432in}}%
\pgfpathcurveto{\pgfqpoint{0.885652in}{1.428669in}}{\pgfqpoint{0.882379in}{1.436569in}}{\pgfqpoint{0.876555in}{1.442393in}}%
\pgfpathcurveto{\pgfqpoint{0.870732in}{1.448217in}}{\pgfqpoint{0.862831in}{1.451489in}}{\pgfqpoint{0.854595in}{1.451489in}}%
\pgfpathcurveto{\pgfqpoint{0.846359in}{1.451489in}}{\pgfqpoint{0.838459in}{1.448217in}}{\pgfqpoint{0.832635in}{1.442393in}}%
\pgfpathcurveto{\pgfqpoint{0.826811in}{1.436569in}}{\pgfqpoint{0.823539in}{1.428669in}}{\pgfqpoint{0.823539in}{1.420432in}}%
\pgfpathcurveto{\pgfqpoint{0.823539in}{1.412196in}}{\pgfqpoint{0.826811in}{1.404296in}}{\pgfqpoint{0.832635in}{1.398472in}}%
\pgfpathcurveto{\pgfqpoint{0.838459in}{1.392648in}}{\pgfqpoint{0.846359in}{1.389376in}}{\pgfqpoint{0.854595in}{1.389376in}}%
\pgfpathclose%
\pgfusepath{stroke,fill}%
\end{pgfscope}%
\begin{pgfscope}%
\pgfpathrectangle{\pgfqpoint{0.100000in}{0.220728in}}{\pgfqpoint{3.696000in}{3.696000in}}%
\pgfusepath{clip}%
\pgfsetbuttcap%
\pgfsetroundjoin%
\definecolor{currentfill}{rgb}{0.121569,0.466667,0.705882}%
\pgfsetfillcolor{currentfill}%
\pgfsetfillopacity{0.588311}%
\pgfsetlinewidth{1.003750pt}%
\definecolor{currentstroke}{rgb}{0.121569,0.466667,0.705882}%
\pgfsetstrokecolor{currentstroke}%
\pgfsetstrokeopacity{0.588311}%
\pgfsetdash{}{0pt}%
\pgfpathmoveto{\pgfqpoint{0.852136in}{1.386212in}}%
\pgfpathcurveto{\pgfqpoint{0.860372in}{1.386212in}}{\pgfqpoint{0.868272in}{1.389484in}}{\pgfqpoint{0.874096in}{1.395308in}}%
\pgfpathcurveto{\pgfqpoint{0.879920in}{1.401132in}}{\pgfqpoint{0.883192in}{1.409032in}}{\pgfqpoint{0.883192in}{1.417269in}}%
\pgfpathcurveto{\pgfqpoint{0.883192in}{1.425505in}}{\pgfqpoint{0.879920in}{1.433405in}}{\pgfqpoint{0.874096in}{1.439229in}}%
\pgfpathcurveto{\pgfqpoint{0.868272in}{1.445053in}}{\pgfqpoint{0.860372in}{1.448325in}}{\pgfqpoint{0.852136in}{1.448325in}}%
\pgfpathcurveto{\pgfqpoint{0.843899in}{1.448325in}}{\pgfqpoint{0.835999in}{1.445053in}}{\pgfqpoint{0.830175in}{1.439229in}}%
\pgfpathcurveto{\pgfqpoint{0.824351in}{1.433405in}}{\pgfqpoint{0.821079in}{1.425505in}}{\pgfqpoint{0.821079in}{1.417269in}}%
\pgfpathcurveto{\pgfqpoint{0.821079in}{1.409032in}}{\pgfqpoint{0.824351in}{1.401132in}}{\pgfqpoint{0.830175in}{1.395308in}}%
\pgfpathcurveto{\pgfqpoint{0.835999in}{1.389484in}}{\pgfqpoint{0.843899in}{1.386212in}}{\pgfqpoint{0.852136in}{1.386212in}}%
\pgfpathclose%
\pgfusepath{stroke,fill}%
\end{pgfscope}%
\begin{pgfscope}%
\pgfpathrectangle{\pgfqpoint{0.100000in}{0.220728in}}{\pgfqpoint{3.696000in}{3.696000in}}%
\pgfusepath{clip}%
\pgfsetbuttcap%
\pgfsetroundjoin%
\definecolor{currentfill}{rgb}{0.121569,0.466667,0.705882}%
\pgfsetfillcolor{currentfill}%
\pgfsetfillopacity{0.589676}%
\pgfsetlinewidth{1.003750pt}%
\definecolor{currentstroke}{rgb}{0.121569,0.466667,0.705882}%
\pgfsetstrokecolor{currentstroke}%
\pgfsetstrokeopacity{0.589676}%
\pgfsetdash{}{0pt}%
\pgfpathmoveto{\pgfqpoint{0.844956in}{1.378766in}}%
\pgfpathcurveto{\pgfqpoint{0.853193in}{1.378766in}}{\pgfqpoint{0.861093in}{1.382039in}}{\pgfqpoint{0.866917in}{1.387863in}}%
\pgfpathcurveto{\pgfqpoint{0.872740in}{1.393687in}}{\pgfqpoint{0.876013in}{1.401587in}}{\pgfqpoint{0.876013in}{1.409823in}}%
\pgfpathcurveto{\pgfqpoint{0.876013in}{1.418059in}}{\pgfqpoint{0.872740in}{1.425959in}}{\pgfqpoint{0.866917in}{1.431783in}}%
\pgfpathcurveto{\pgfqpoint{0.861093in}{1.437607in}}{\pgfqpoint{0.853193in}{1.440879in}}{\pgfqpoint{0.844956in}{1.440879in}}%
\pgfpathcurveto{\pgfqpoint{0.836720in}{1.440879in}}{\pgfqpoint{0.828820in}{1.437607in}}{\pgfqpoint{0.822996in}{1.431783in}}%
\pgfpathcurveto{\pgfqpoint{0.817172in}{1.425959in}}{\pgfqpoint{0.813900in}{1.418059in}}{\pgfqpoint{0.813900in}{1.409823in}}%
\pgfpathcurveto{\pgfqpoint{0.813900in}{1.401587in}}{\pgfqpoint{0.817172in}{1.393687in}}{\pgfqpoint{0.822996in}{1.387863in}}%
\pgfpathcurveto{\pgfqpoint{0.828820in}{1.382039in}}{\pgfqpoint{0.836720in}{1.378766in}}{\pgfqpoint{0.844956in}{1.378766in}}%
\pgfpathclose%
\pgfusepath{stroke,fill}%
\end{pgfscope}%
\begin{pgfscope}%
\pgfpathrectangle{\pgfqpoint{0.100000in}{0.220728in}}{\pgfqpoint{3.696000in}{3.696000in}}%
\pgfusepath{clip}%
\pgfsetbuttcap%
\pgfsetroundjoin%
\definecolor{currentfill}{rgb}{0.121569,0.466667,0.705882}%
\pgfsetfillcolor{currentfill}%
\pgfsetfillopacity{0.590207}%
\pgfsetlinewidth{1.003750pt}%
\definecolor{currentstroke}{rgb}{0.121569,0.466667,0.705882}%
\pgfsetstrokecolor{currentstroke}%
\pgfsetstrokeopacity{0.590207}%
\pgfsetdash{}{0pt}%
\pgfpathmoveto{\pgfqpoint{0.755558in}{1.351183in}}%
\pgfpathcurveto{\pgfqpoint{0.763794in}{1.351183in}}{\pgfqpoint{0.771694in}{1.354455in}}{\pgfqpoint{0.777518in}{1.360279in}}%
\pgfpathcurveto{\pgfqpoint{0.783342in}{1.366103in}}{\pgfqpoint{0.786615in}{1.374003in}}{\pgfqpoint{0.786615in}{1.382240in}}%
\pgfpathcurveto{\pgfqpoint{0.786615in}{1.390476in}}{\pgfqpoint{0.783342in}{1.398376in}}{\pgfqpoint{0.777518in}{1.404200in}}%
\pgfpathcurveto{\pgfqpoint{0.771694in}{1.410024in}}{\pgfqpoint{0.763794in}{1.413296in}}{\pgfqpoint{0.755558in}{1.413296in}}%
\pgfpathcurveto{\pgfqpoint{0.747322in}{1.413296in}}{\pgfqpoint{0.739422in}{1.410024in}}{\pgfqpoint{0.733598in}{1.404200in}}%
\pgfpathcurveto{\pgfqpoint{0.727774in}{1.398376in}}{\pgfqpoint{0.724502in}{1.390476in}}{\pgfqpoint{0.724502in}{1.382240in}}%
\pgfpathcurveto{\pgfqpoint{0.724502in}{1.374003in}}{\pgfqpoint{0.727774in}{1.366103in}}{\pgfqpoint{0.733598in}{1.360279in}}%
\pgfpathcurveto{\pgfqpoint{0.739422in}{1.354455in}}{\pgfqpoint{0.747322in}{1.351183in}}{\pgfqpoint{0.755558in}{1.351183in}}%
\pgfpathclose%
\pgfusepath{stroke,fill}%
\end{pgfscope}%
\begin{pgfscope}%
\pgfpathrectangle{\pgfqpoint{0.100000in}{0.220728in}}{\pgfqpoint{3.696000in}{3.696000in}}%
\pgfusepath{clip}%
\pgfsetbuttcap%
\pgfsetroundjoin%
\definecolor{currentfill}{rgb}{0.121569,0.466667,0.705882}%
\pgfsetfillcolor{currentfill}%
\pgfsetfillopacity{0.590921}%
\pgfsetlinewidth{1.003750pt}%
\definecolor{currentstroke}{rgb}{0.121569,0.466667,0.705882}%
\pgfsetstrokecolor{currentstroke}%
\pgfsetstrokeopacity{0.590921}%
\pgfsetdash{}{0pt}%
\pgfpathmoveto{\pgfqpoint{2.895674in}{2.945232in}}%
\pgfpathcurveto{\pgfqpoint{2.903910in}{2.945232in}}{\pgfqpoint{2.911810in}{2.948504in}}{\pgfqpoint{2.917634in}{2.954328in}}%
\pgfpathcurveto{\pgfqpoint{2.923458in}{2.960152in}}{\pgfqpoint{2.926731in}{2.968052in}}{\pgfqpoint{2.926731in}{2.976289in}}%
\pgfpathcurveto{\pgfqpoint{2.926731in}{2.984525in}}{\pgfqpoint{2.923458in}{2.992425in}}{\pgfqpoint{2.917634in}{2.998249in}}%
\pgfpathcurveto{\pgfqpoint{2.911810in}{3.004073in}}{\pgfqpoint{2.903910in}{3.007345in}}{\pgfqpoint{2.895674in}{3.007345in}}%
\pgfpathcurveto{\pgfqpoint{2.887438in}{3.007345in}}{\pgfqpoint{2.879538in}{3.004073in}}{\pgfqpoint{2.873714in}{2.998249in}}%
\pgfpathcurveto{\pgfqpoint{2.867890in}{2.992425in}}{\pgfqpoint{2.864618in}{2.984525in}}{\pgfqpoint{2.864618in}{2.976289in}}%
\pgfpathcurveto{\pgfqpoint{2.864618in}{2.968052in}}{\pgfqpoint{2.867890in}{2.960152in}}{\pgfqpoint{2.873714in}{2.954328in}}%
\pgfpathcurveto{\pgfqpoint{2.879538in}{2.948504in}}{\pgfqpoint{2.887438in}{2.945232in}}{\pgfqpoint{2.895674in}{2.945232in}}%
\pgfpathclose%
\pgfusepath{stroke,fill}%
\end{pgfscope}%
\begin{pgfscope}%
\pgfpathrectangle{\pgfqpoint{0.100000in}{0.220728in}}{\pgfqpoint{3.696000in}{3.696000in}}%
\pgfusepath{clip}%
\pgfsetbuttcap%
\pgfsetroundjoin%
\definecolor{currentfill}{rgb}{0.121569,0.466667,0.705882}%
\pgfsetfillcolor{currentfill}%
\pgfsetfillopacity{0.592149}%
\pgfsetlinewidth{1.003750pt}%
\definecolor{currentstroke}{rgb}{0.121569,0.466667,0.705882}%
\pgfsetstrokecolor{currentstroke}%
\pgfsetstrokeopacity{0.592149}%
\pgfsetdash{}{0pt}%
\pgfpathmoveto{\pgfqpoint{0.832530in}{1.365748in}}%
\pgfpathcurveto{\pgfqpoint{0.840766in}{1.365748in}}{\pgfqpoint{0.848666in}{1.369020in}}{\pgfqpoint{0.854490in}{1.374844in}}%
\pgfpathcurveto{\pgfqpoint{0.860314in}{1.380668in}}{\pgfqpoint{0.863586in}{1.388568in}}{\pgfqpoint{0.863586in}{1.396804in}}%
\pgfpathcurveto{\pgfqpoint{0.863586in}{1.405041in}}{\pgfqpoint{0.860314in}{1.412941in}}{\pgfqpoint{0.854490in}{1.418765in}}%
\pgfpathcurveto{\pgfqpoint{0.848666in}{1.424589in}}{\pgfqpoint{0.840766in}{1.427861in}}{\pgfqpoint{0.832530in}{1.427861in}}%
\pgfpathcurveto{\pgfqpoint{0.824293in}{1.427861in}}{\pgfqpoint{0.816393in}{1.424589in}}{\pgfqpoint{0.810569in}{1.418765in}}%
\pgfpathcurveto{\pgfqpoint{0.804745in}{1.412941in}}{\pgfqpoint{0.801473in}{1.405041in}}{\pgfqpoint{0.801473in}{1.396804in}}%
\pgfpathcurveto{\pgfqpoint{0.801473in}{1.388568in}}{\pgfqpoint{0.804745in}{1.380668in}}{\pgfqpoint{0.810569in}{1.374844in}}%
\pgfpathcurveto{\pgfqpoint{0.816393in}{1.369020in}}{\pgfqpoint{0.824293in}{1.365748in}}{\pgfqpoint{0.832530in}{1.365748in}}%
\pgfpathclose%
\pgfusepath{stroke,fill}%
\end{pgfscope}%
\begin{pgfscope}%
\pgfpathrectangle{\pgfqpoint{0.100000in}{0.220728in}}{\pgfqpoint{3.696000in}{3.696000in}}%
\pgfusepath{clip}%
\pgfsetbuttcap%
\pgfsetroundjoin%
\definecolor{currentfill}{rgb}{0.121569,0.466667,0.705882}%
\pgfsetfillcolor{currentfill}%
\pgfsetfillopacity{0.592912}%
\pgfsetlinewidth{1.003750pt}%
\definecolor{currentstroke}{rgb}{0.121569,0.466667,0.705882}%
\pgfsetstrokecolor{currentstroke}%
\pgfsetstrokeopacity{0.592912}%
\pgfsetdash{}{0pt}%
\pgfpathmoveto{\pgfqpoint{0.750216in}{1.339636in}}%
\pgfpathcurveto{\pgfqpoint{0.758452in}{1.339636in}}{\pgfqpoint{0.766352in}{1.342908in}}{\pgfqpoint{0.772176in}{1.348732in}}%
\pgfpathcurveto{\pgfqpoint{0.778000in}{1.354556in}}{\pgfqpoint{0.781272in}{1.362456in}}{\pgfqpoint{0.781272in}{1.370693in}}%
\pgfpathcurveto{\pgfqpoint{0.781272in}{1.378929in}}{\pgfqpoint{0.778000in}{1.386829in}}{\pgfqpoint{0.772176in}{1.392653in}}%
\pgfpathcurveto{\pgfqpoint{0.766352in}{1.398477in}}{\pgfqpoint{0.758452in}{1.401749in}}{\pgfqpoint{0.750216in}{1.401749in}}%
\pgfpathcurveto{\pgfqpoint{0.741980in}{1.401749in}}{\pgfqpoint{0.734079in}{1.398477in}}{\pgfqpoint{0.728256in}{1.392653in}}%
\pgfpathcurveto{\pgfqpoint{0.722432in}{1.386829in}}{\pgfqpoint{0.719159in}{1.378929in}}{\pgfqpoint{0.719159in}{1.370693in}}%
\pgfpathcurveto{\pgfqpoint{0.719159in}{1.362456in}}{\pgfqpoint{0.722432in}{1.354556in}}{\pgfqpoint{0.728256in}{1.348732in}}%
\pgfpathcurveto{\pgfqpoint{0.734079in}{1.342908in}}{\pgfqpoint{0.741980in}{1.339636in}}{\pgfqpoint{0.750216in}{1.339636in}}%
\pgfpathclose%
\pgfusepath{stroke,fill}%
\end{pgfscope}%
\begin{pgfscope}%
\pgfpathrectangle{\pgfqpoint{0.100000in}{0.220728in}}{\pgfqpoint{3.696000in}{3.696000in}}%
\pgfusepath{clip}%
\pgfsetbuttcap%
\pgfsetroundjoin%
\definecolor{currentfill}{rgb}{0.121569,0.466667,0.705882}%
\pgfsetfillcolor{currentfill}%
\pgfsetfillopacity{0.593468}%
\pgfsetlinewidth{1.003750pt}%
\definecolor{currentstroke}{rgb}{0.121569,0.466667,0.705882}%
\pgfsetstrokecolor{currentstroke}%
\pgfsetstrokeopacity{0.593468}%
\pgfsetdash{}{0pt}%
\pgfpathmoveto{\pgfqpoint{2.904129in}{2.944292in}}%
\pgfpathcurveto{\pgfqpoint{2.912366in}{2.944292in}}{\pgfqpoint{2.920266in}{2.947564in}}{\pgfqpoint{2.926090in}{2.953388in}}%
\pgfpathcurveto{\pgfqpoint{2.931914in}{2.959212in}}{\pgfqpoint{2.935186in}{2.967112in}}{\pgfqpoint{2.935186in}{2.975348in}}%
\pgfpathcurveto{\pgfqpoint{2.935186in}{2.983585in}}{\pgfqpoint{2.931914in}{2.991485in}}{\pgfqpoint{2.926090in}{2.997309in}}%
\pgfpathcurveto{\pgfqpoint{2.920266in}{3.003133in}}{\pgfqpoint{2.912366in}{3.006405in}}{\pgfqpoint{2.904129in}{3.006405in}}%
\pgfpathcurveto{\pgfqpoint{2.895893in}{3.006405in}}{\pgfqpoint{2.887993in}{3.003133in}}{\pgfqpoint{2.882169in}{2.997309in}}%
\pgfpathcurveto{\pgfqpoint{2.876345in}{2.991485in}}{\pgfqpoint{2.873073in}{2.983585in}}{\pgfqpoint{2.873073in}{2.975348in}}%
\pgfpathcurveto{\pgfqpoint{2.873073in}{2.967112in}}{\pgfqpoint{2.876345in}{2.959212in}}{\pgfqpoint{2.882169in}{2.953388in}}%
\pgfpathcurveto{\pgfqpoint{2.887993in}{2.947564in}}{\pgfqpoint{2.895893in}{2.944292in}}{\pgfqpoint{2.904129in}{2.944292in}}%
\pgfpathclose%
\pgfusepath{stroke,fill}%
\end{pgfscope}%
\begin{pgfscope}%
\pgfpathrectangle{\pgfqpoint{0.100000in}{0.220728in}}{\pgfqpoint{3.696000in}{3.696000in}}%
\pgfusepath{clip}%
\pgfsetbuttcap%
\pgfsetroundjoin%
\definecolor{currentfill}{rgb}{0.121569,0.466667,0.705882}%
\pgfsetfillcolor{currentfill}%
\pgfsetfillopacity{0.593894}%
\pgfsetlinewidth{1.003750pt}%
\definecolor{currentstroke}{rgb}{0.121569,0.466667,0.705882}%
\pgfsetstrokecolor{currentstroke}%
\pgfsetstrokeopacity{0.593894}%
\pgfsetdash{}{0pt}%
\pgfpathmoveto{\pgfqpoint{0.746950in}{1.334703in}}%
\pgfpathcurveto{\pgfqpoint{0.755186in}{1.334703in}}{\pgfqpoint{0.763086in}{1.337976in}}{\pgfqpoint{0.768910in}{1.343800in}}%
\pgfpathcurveto{\pgfqpoint{0.774734in}{1.349624in}}{\pgfqpoint{0.778006in}{1.357524in}}{\pgfqpoint{0.778006in}{1.365760in}}%
\pgfpathcurveto{\pgfqpoint{0.778006in}{1.373996in}}{\pgfqpoint{0.774734in}{1.381896in}}{\pgfqpoint{0.768910in}{1.387720in}}%
\pgfpathcurveto{\pgfqpoint{0.763086in}{1.393544in}}{\pgfqpoint{0.755186in}{1.396816in}}{\pgfqpoint{0.746950in}{1.396816in}}%
\pgfpathcurveto{\pgfqpoint{0.738714in}{1.396816in}}{\pgfqpoint{0.730814in}{1.393544in}}{\pgfqpoint{0.724990in}{1.387720in}}%
\pgfpathcurveto{\pgfqpoint{0.719166in}{1.381896in}}{\pgfqpoint{0.715893in}{1.373996in}}{\pgfqpoint{0.715893in}{1.365760in}}%
\pgfpathcurveto{\pgfqpoint{0.715893in}{1.357524in}}{\pgfqpoint{0.719166in}{1.349624in}}{\pgfqpoint{0.724990in}{1.343800in}}%
\pgfpathcurveto{\pgfqpoint{0.730814in}{1.337976in}}{\pgfqpoint{0.738714in}{1.334703in}}{\pgfqpoint{0.746950in}{1.334703in}}%
\pgfpathclose%
\pgfusepath{stroke,fill}%
\end{pgfscope}%
\begin{pgfscope}%
\pgfpathrectangle{\pgfqpoint{0.100000in}{0.220728in}}{\pgfqpoint{3.696000in}{3.696000in}}%
\pgfusepath{clip}%
\pgfsetbuttcap%
\pgfsetroundjoin%
\definecolor{currentfill}{rgb}{0.121569,0.466667,0.705882}%
\pgfsetfillcolor{currentfill}%
\pgfsetfillopacity{0.595125}%
\pgfsetlinewidth{1.003750pt}%
\definecolor{currentstroke}{rgb}{0.121569,0.466667,0.705882}%
\pgfsetstrokecolor{currentstroke}%
\pgfsetstrokeopacity{0.595125}%
\pgfsetdash{}{0pt}%
\pgfpathmoveto{\pgfqpoint{0.816546in}{1.345988in}}%
\pgfpathcurveto{\pgfqpoint{0.824782in}{1.345988in}}{\pgfqpoint{0.832682in}{1.349261in}}{\pgfqpoint{0.838506in}{1.355085in}}%
\pgfpathcurveto{\pgfqpoint{0.844330in}{1.360909in}}{\pgfqpoint{0.847602in}{1.368809in}}{\pgfqpoint{0.847602in}{1.377045in}}%
\pgfpathcurveto{\pgfqpoint{0.847602in}{1.385281in}}{\pgfqpoint{0.844330in}{1.393181in}}{\pgfqpoint{0.838506in}{1.399005in}}%
\pgfpathcurveto{\pgfqpoint{0.832682in}{1.404829in}}{\pgfqpoint{0.824782in}{1.408101in}}{\pgfqpoint{0.816546in}{1.408101in}}%
\pgfpathcurveto{\pgfqpoint{0.808309in}{1.408101in}}{\pgfqpoint{0.800409in}{1.404829in}}{\pgfqpoint{0.794585in}{1.399005in}}%
\pgfpathcurveto{\pgfqpoint{0.788762in}{1.393181in}}{\pgfqpoint{0.785489in}{1.385281in}}{\pgfqpoint{0.785489in}{1.377045in}}%
\pgfpathcurveto{\pgfqpoint{0.785489in}{1.368809in}}{\pgfqpoint{0.788762in}{1.360909in}}{\pgfqpoint{0.794585in}{1.355085in}}%
\pgfpathcurveto{\pgfqpoint{0.800409in}{1.349261in}}{\pgfqpoint{0.808309in}{1.345988in}}{\pgfqpoint{0.816546in}{1.345988in}}%
\pgfpathclose%
\pgfusepath{stroke,fill}%
\end{pgfscope}%
\begin{pgfscope}%
\pgfpathrectangle{\pgfqpoint{0.100000in}{0.220728in}}{\pgfqpoint{3.696000in}{3.696000in}}%
\pgfusepath{clip}%
\pgfsetbuttcap%
\pgfsetroundjoin%
\definecolor{currentfill}{rgb}{0.121569,0.466667,0.705882}%
\pgfsetfillcolor{currentfill}%
\pgfsetfillopacity{0.595819}%
\pgfsetlinewidth{1.003750pt}%
\definecolor{currentstroke}{rgb}{0.121569,0.466667,0.705882}%
\pgfsetstrokecolor{currentstroke}%
\pgfsetstrokeopacity{0.595819}%
\pgfsetdash{}{0pt}%
\pgfpathmoveto{\pgfqpoint{0.741806in}{1.325437in}}%
\pgfpathcurveto{\pgfqpoint{0.750042in}{1.325437in}}{\pgfqpoint{0.757942in}{1.328710in}}{\pgfqpoint{0.763766in}{1.334533in}}%
\pgfpathcurveto{\pgfqpoint{0.769590in}{1.340357in}}{\pgfqpoint{0.772863in}{1.348257in}}{\pgfqpoint{0.772863in}{1.356494in}}%
\pgfpathcurveto{\pgfqpoint{0.772863in}{1.364730in}}{\pgfqpoint{0.769590in}{1.372630in}}{\pgfqpoint{0.763766in}{1.378454in}}%
\pgfpathcurveto{\pgfqpoint{0.757942in}{1.384278in}}{\pgfqpoint{0.750042in}{1.387550in}}{\pgfqpoint{0.741806in}{1.387550in}}%
\pgfpathcurveto{\pgfqpoint{0.733570in}{1.387550in}}{\pgfqpoint{0.725670in}{1.384278in}}{\pgfqpoint{0.719846in}{1.378454in}}%
\pgfpathcurveto{\pgfqpoint{0.714022in}{1.372630in}}{\pgfqpoint{0.710750in}{1.364730in}}{\pgfqpoint{0.710750in}{1.356494in}}%
\pgfpathcurveto{\pgfqpoint{0.710750in}{1.348257in}}{\pgfqpoint{0.714022in}{1.340357in}}{\pgfqpoint{0.719846in}{1.334533in}}%
\pgfpathcurveto{\pgfqpoint{0.725670in}{1.328710in}}{\pgfqpoint{0.733570in}{1.325437in}}{\pgfqpoint{0.741806in}{1.325437in}}%
\pgfpathclose%
\pgfusepath{stroke,fill}%
\end{pgfscope}%
\begin{pgfscope}%
\pgfpathrectangle{\pgfqpoint{0.100000in}{0.220728in}}{\pgfqpoint{3.696000in}{3.696000in}}%
\pgfusepath{clip}%
\pgfsetbuttcap%
\pgfsetroundjoin%
\definecolor{currentfill}{rgb}{0.121569,0.466667,0.705882}%
\pgfsetfillcolor{currentfill}%
\pgfsetfillopacity{0.596136}%
\pgfsetlinewidth{1.003750pt}%
\definecolor{currentstroke}{rgb}{0.121569,0.466667,0.705882}%
\pgfsetstrokecolor{currentstroke}%
\pgfsetstrokeopacity{0.596136}%
\pgfsetdash{}{0pt}%
\pgfpathmoveto{\pgfqpoint{2.918091in}{2.942910in}}%
\pgfpathcurveto{\pgfqpoint{2.926327in}{2.942910in}}{\pgfqpoint{2.934227in}{2.946182in}}{\pgfqpoint{2.940051in}{2.952006in}}%
\pgfpathcurveto{\pgfqpoint{2.945875in}{2.957830in}}{\pgfqpoint{2.949147in}{2.965730in}}{\pgfqpoint{2.949147in}{2.973967in}}%
\pgfpathcurveto{\pgfqpoint{2.949147in}{2.982203in}}{\pgfqpoint{2.945875in}{2.990103in}}{\pgfqpoint{2.940051in}{2.995927in}}%
\pgfpathcurveto{\pgfqpoint{2.934227in}{3.001751in}}{\pgfqpoint{2.926327in}{3.005023in}}{\pgfqpoint{2.918091in}{3.005023in}}%
\pgfpathcurveto{\pgfqpoint{2.909855in}{3.005023in}}{\pgfqpoint{2.901955in}{3.001751in}}{\pgfqpoint{2.896131in}{2.995927in}}%
\pgfpathcurveto{\pgfqpoint{2.890307in}{2.990103in}}{\pgfqpoint{2.887034in}{2.982203in}}{\pgfqpoint{2.887034in}{2.973967in}}%
\pgfpathcurveto{\pgfqpoint{2.887034in}{2.965730in}}{\pgfqpoint{2.890307in}{2.957830in}}{\pgfqpoint{2.896131in}{2.952006in}}%
\pgfpathcurveto{\pgfqpoint{2.901955in}{2.946182in}}{\pgfqpoint{2.909855in}{2.942910in}}{\pgfqpoint{2.918091in}{2.942910in}}%
\pgfpathclose%
\pgfusepath{stroke,fill}%
\end{pgfscope}%
\begin{pgfscope}%
\pgfpathrectangle{\pgfqpoint{0.100000in}{0.220728in}}{\pgfqpoint{3.696000in}{3.696000in}}%
\pgfusepath{clip}%
\pgfsetbuttcap%
\pgfsetroundjoin%
\definecolor{currentfill}{rgb}{0.121569,0.466667,0.705882}%
\pgfsetfillcolor{currentfill}%
\pgfsetfillopacity{0.598993}%
\pgfsetlinewidth{1.003750pt}%
\definecolor{currentstroke}{rgb}{0.121569,0.466667,0.705882}%
\pgfsetstrokecolor{currentstroke}%
\pgfsetstrokeopacity{0.598993}%
\pgfsetdash{}{0pt}%
\pgfpathmoveto{\pgfqpoint{0.795940in}{1.322167in}}%
\pgfpathcurveto{\pgfqpoint{0.804177in}{1.322167in}}{\pgfqpoint{0.812077in}{1.325439in}}{\pgfqpoint{0.817901in}{1.331263in}}%
\pgfpathcurveto{\pgfqpoint{0.823724in}{1.337087in}}{\pgfqpoint{0.826997in}{1.344987in}}{\pgfqpoint{0.826997in}{1.353223in}}%
\pgfpathcurveto{\pgfqpoint{0.826997in}{1.361460in}}{\pgfqpoint{0.823724in}{1.369360in}}{\pgfqpoint{0.817901in}{1.375184in}}%
\pgfpathcurveto{\pgfqpoint{0.812077in}{1.381008in}}{\pgfqpoint{0.804177in}{1.384280in}}{\pgfqpoint{0.795940in}{1.384280in}}%
\pgfpathcurveto{\pgfqpoint{0.787704in}{1.384280in}}{\pgfqpoint{0.779804in}{1.381008in}}{\pgfqpoint{0.773980in}{1.375184in}}%
\pgfpathcurveto{\pgfqpoint{0.768156in}{1.369360in}}{\pgfqpoint{0.764884in}{1.361460in}}{\pgfqpoint{0.764884in}{1.353223in}}%
\pgfpathcurveto{\pgfqpoint{0.764884in}{1.344987in}}{\pgfqpoint{0.768156in}{1.337087in}}{\pgfqpoint{0.773980in}{1.331263in}}%
\pgfpathcurveto{\pgfqpoint{0.779804in}{1.325439in}}{\pgfqpoint{0.787704in}{1.322167in}}{\pgfqpoint{0.795940in}{1.322167in}}%
\pgfpathclose%
\pgfusepath{stroke,fill}%
\end{pgfscope}%
\begin{pgfscope}%
\pgfpathrectangle{\pgfqpoint{0.100000in}{0.220728in}}{\pgfqpoint{3.696000in}{3.696000in}}%
\pgfusepath{clip}%
\pgfsetbuttcap%
\pgfsetroundjoin%
\definecolor{currentfill}{rgb}{0.121569,0.466667,0.705882}%
\pgfsetfillcolor{currentfill}%
\pgfsetfillopacity{0.599504}%
\pgfsetlinewidth{1.003750pt}%
\definecolor{currentstroke}{rgb}{0.121569,0.466667,0.705882}%
\pgfsetstrokecolor{currentstroke}%
\pgfsetstrokeopacity{0.599504}%
\pgfsetdash{}{0pt}%
\pgfpathmoveto{\pgfqpoint{0.733353in}{1.308564in}}%
\pgfpathcurveto{\pgfqpoint{0.741590in}{1.308564in}}{\pgfqpoint{0.749490in}{1.311837in}}{\pgfqpoint{0.755314in}{1.317660in}}%
\pgfpathcurveto{\pgfqpoint{0.761138in}{1.323484in}}{\pgfqpoint{0.764410in}{1.331384in}}{\pgfqpoint{0.764410in}{1.339621in}}%
\pgfpathcurveto{\pgfqpoint{0.764410in}{1.347857in}}{\pgfqpoint{0.761138in}{1.355757in}}{\pgfqpoint{0.755314in}{1.361581in}}%
\pgfpathcurveto{\pgfqpoint{0.749490in}{1.367405in}}{\pgfqpoint{0.741590in}{1.370677in}}{\pgfqpoint{0.733353in}{1.370677in}}%
\pgfpathcurveto{\pgfqpoint{0.725117in}{1.370677in}}{\pgfqpoint{0.717217in}{1.367405in}}{\pgfqpoint{0.711393in}{1.361581in}}%
\pgfpathcurveto{\pgfqpoint{0.705569in}{1.355757in}}{\pgfqpoint{0.702297in}{1.347857in}}{\pgfqpoint{0.702297in}{1.339621in}}%
\pgfpathcurveto{\pgfqpoint{0.702297in}{1.331384in}}{\pgfqpoint{0.705569in}{1.323484in}}{\pgfqpoint{0.711393in}{1.317660in}}%
\pgfpathcurveto{\pgfqpoint{0.717217in}{1.311837in}}{\pgfqpoint{0.725117in}{1.308564in}}{\pgfqpoint{0.733353in}{1.308564in}}%
\pgfpathclose%
\pgfusepath{stroke,fill}%
\end{pgfscope}%
\begin{pgfscope}%
\pgfpathrectangle{\pgfqpoint{0.100000in}{0.220728in}}{\pgfqpoint{3.696000in}{3.696000in}}%
\pgfusepath{clip}%
\pgfsetbuttcap%
\pgfsetroundjoin%
\definecolor{currentfill}{rgb}{0.121569,0.466667,0.705882}%
\pgfsetfillcolor{currentfill}%
\pgfsetfillopacity{0.600694}%
\pgfsetlinewidth{1.003750pt}%
\definecolor{currentstroke}{rgb}{0.121569,0.466667,0.705882}%
\pgfsetstrokecolor{currentstroke}%
\pgfsetstrokeopacity{0.600694}%
\pgfsetdash{}{0pt}%
\pgfpathmoveto{\pgfqpoint{2.934184in}{2.940932in}}%
\pgfpathcurveto{\pgfqpoint{2.942420in}{2.940932in}}{\pgfqpoint{2.950320in}{2.944205in}}{\pgfqpoint{2.956144in}{2.950029in}}%
\pgfpathcurveto{\pgfqpoint{2.961968in}{2.955853in}}{\pgfqpoint{2.965240in}{2.963753in}}{\pgfqpoint{2.965240in}{2.971989in}}%
\pgfpathcurveto{\pgfqpoint{2.965240in}{2.980225in}}{\pgfqpoint{2.961968in}{2.988125in}}{\pgfqpoint{2.956144in}{2.993949in}}%
\pgfpathcurveto{\pgfqpoint{2.950320in}{2.999773in}}{\pgfqpoint{2.942420in}{3.003045in}}{\pgfqpoint{2.934184in}{3.003045in}}%
\pgfpathcurveto{\pgfqpoint{2.925947in}{3.003045in}}{\pgfqpoint{2.918047in}{2.999773in}}{\pgfqpoint{2.912223in}{2.993949in}}%
\pgfpathcurveto{\pgfqpoint{2.906399in}{2.988125in}}{\pgfqpoint{2.903127in}{2.980225in}}{\pgfqpoint{2.903127in}{2.971989in}}%
\pgfpathcurveto{\pgfqpoint{2.903127in}{2.963753in}}{\pgfqpoint{2.906399in}{2.955853in}}{\pgfqpoint{2.912223in}{2.950029in}}%
\pgfpathcurveto{\pgfqpoint{2.918047in}{2.944205in}}{\pgfqpoint{2.925947in}{2.940932in}}{\pgfqpoint{2.934184in}{2.940932in}}%
\pgfpathclose%
\pgfusepath{stroke,fill}%
\end{pgfscope}%
\begin{pgfscope}%
\pgfpathrectangle{\pgfqpoint{0.100000in}{0.220728in}}{\pgfqpoint{3.696000in}{3.696000in}}%
\pgfusepath{clip}%
\pgfsetbuttcap%
\pgfsetroundjoin%
\definecolor{currentfill}{rgb}{0.121569,0.466667,0.705882}%
\pgfsetfillcolor{currentfill}%
\pgfsetfillopacity{0.603014}%
\pgfsetlinewidth{1.003750pt}%
\definecolor{currentstroke}{rgb}{0.121569,0.466667,0.705882}%
\pgfsetstrokecolor{currentstroke}%
\pgfsetstrokeopacity{0.603014}%
\pgfsetdash{}{0pt}%
\pgfpathmoveto{\pgfqpoint{2.943320in}{2.939701in}}%
\pgfpathcurveto{\pgfqpoint{2.951557in}{2.939701in}}{\pgfqpoint{2.959457in}{2.942974in}}{\pgfqpoint{2.965281in}{2.948798in}}%
\pgfpathcurveto{\pgfqpoint{2.971104in}{2.954622in}}{\pgfqpoint{2.974377in}{2.962522in}}{\pgfqpoint{2.974377in}{2.970758in}}%
\pgfpathcurveto{\pgfqpoint{2.974377in}{2.978994in}}{\pgfqpoint{2.971104in}{2.986894in}}{\pgfqpoint{2.965281in}{2.992718in}}%
\pgfpathcurveto{\pgfqpoint{2.959457in}{2.998542in}}{\pgfqpoint{2.951557in}{3.001814in}}{\pgfqpoint{2.943320in}{3.001814in}}%
\pgfpathcurveto{\pgfqpoint{2.935084in}{3.001814in}}{\pgfqpoint{2.927184in}{2.998542in}}{\pgfqpoint{2.921360in}{2.992718in}}%
\pgfpathcurveto{\pgfqpoint{2.915536in}{2.986894in}}{\pgfqpoint{2.912264in}{2.978994in}}{\pgfqpoint{2.912264in}{2.970758in}}%
\pgfpathcurveto{\pgfqpoint{2.912264in}{2.962522in}}{\pgfqpoint{2.915536in}{2.954622in}}{\pgfqpoint{2.921360in}{2.948798in}}%
\pgfpathcurveto{\pgfqpoint{2.927184in}{2.942974in}}{\pgfqpoint{2.935084in}{2.939701in}}{\pgfqpoint{2.943320in}{2.939701in}}%
\pgfpathclose%
\pgfusepath{stroke,fill}%
\end{pgfscope}%
\begin{pgfscope}%
\pgfpathrectangle{\pgfqpoint{0.100000in}{0.220728in}}{\pgfqpoint{3.696000in}{3.696000in}}%
\pgfusepath{clip}%
\pgfsetbuttcap%
\pgfsetroundjoin%
\definecolor{currentfill}{rgb}{0.121569,0.466667,0.705882}%
\pgfsetfillcolor{currentfill}%
\pgfsetfillopacity{0.603216}%
\pgfsetlinewidth{1.003750pt}%
\definecolor{currentstroke}{rgb}{0.121569,0.466667,0.705882}%
\pgfsetstrokecolor{currentstroke}%
\pgfsetstrokeopacity{0.603216}%
\pgfsetdash{}{0pt}%
\pgfpathmoveto{\pgfqpoint{0.770798in}{1.296168in}}%
\pgfpathcurveto{\pgfqpoint{0.779035in}{1.296168in}}{\pgfqpoint{0.786935in}{1.299441in}}{\pgfqpoint{0.792759in}{1.305265in}}%
\pgfpathcurveto{\pgfqpoint{0.798583in}{1.311088in}}{\pgfqpoint{0.801855in}{1.318988in}}{\pgfqpoint{0.801855in}{1.327225in}}%
\pgfpathcurveto{\pgfqpoint{0.801855in}{1.335461in}}{\pgfqpoint{0.798583in}{1.343361in}}{\pgfqpoint{0.792759in}{1.349185in}}%
\pgfpathcurveto{\pgfqpoint{0.786935in}{1.355009in}}{\pgfqpoint{0.779035in}{1.358281in}}{\pgfqpoint{0.770798in}{1.358281in}}%
\pgfpathcurveto{\pgfqpoint{0.762562in}{1.358281in}}{\pgfqpoint{0.754662in}{1.355009in}}{\pgfqpoint{0.748838in}{1.349185in}}%
\pgfpathcurveto{\pgfqpoint{0.743014in}{1.343361in}}{\pgfqpoint{0.739742in}{1.335461in}}{\pgfqpoint{0.739742in}{1.327225in}}%
\pgfpathcurveto{\pgfqpoint{0.739742in}{1.318988in}}{\pgfqpoint{0.743014in}{1.311088in}}{\pgfqpoint{0.748838in}{1.305265in}}%
\pgfpathcurveto{\pgfqpoint{0.754662in}{1.299441in}}{\pgfqpoint{0.762562in}{1.296168in}}{\pgfqpoint{0.770798in}{1.296168in}}%
\pgfpathclose%
\pgfusepath{stroke,fill}%
\end{pgfscope}%
\begin{pgfscope}%
\pgfpathrectangle{\pgfqpoint{0.100000in}{0.220728in}}{\pgfqpoint{3.696000in}{3.696000in}}%
\pgfusepath{clip}%
\pgfsetbuttcap%
\pgfsetroundjoin%
\definecolor{currentfill}{rgb}{0.121569,0.466667,0.705882}%
\pgfsetfillcolor{currentfill}%
\pgfsetfillopacity{0.604513}%
\pgfsetlinewidth{1.003750pt}%
\definecolor{currentstroke}{rgb}{0.121569,0.466667,0.705882}%
\pgfsetstrokecolor{currentstroke}%
\pgfsetstrokeopacity{0.604513}%
\pgfsetdash{}{0pt}%
\pgfpathmoveto{\pgfqpoint{0.711218in}{1.278103in}}%
\pgfpathcurveto{\pgfqpoint{0.719454in}{1.278103in}}{\pgfqpoint{0.727354in}{1.281375in}}{\pgfqpoint{0.733178in}{1.287199in}}%
\pgfpathcurveto{\pgfqpoint{0.739002in}{1.293023in}}{\pgfqpoint{0.742274in}{1.300923in}}{\pgfqpoint{0.742274in}{1.309159in}}%
\pgfpathcurveto{\pgfqpoint{0.742274in}{1.317395in}}{\pgfqpoint{0.739002in}{1.325295in}}{\pgfqpoint{0.733178in}{1.331119in}}%
\pgfpathcurveto{\pgfqpoint{0.727354in}{1.336943in}}{\pgfqpoint{0.719454in}{1.340216in}}{\pgfqpoint{0.711218in}{1.340216in}}%
\pgfpathcurveto{\pgfqpoint{0.702981in}{1.340216in}}{\pgfqpoint{0.695081in}{1.336943in}}{\pgfqpoint{0.689257in}{1.331119in}}%
\pgfpathcurveto{\pgfqpoint{0.683434in}{1.325295in}}{\pgfqpoint{0.680161in}{1.317395in}}{\pgfqpoint{0.680161in}{1.309159in}}%
\pgfpathcurveto{\pgfqpoint{0.680161in}{1.300923in}}{\pgfqpoint{0.683434in}{1.293023in}}{\pgfqpoint{0.689257in}{1.287199in}}%
\pgfpathcurveto{\pgfqpoint{0.695081in}{1.281375in}}{\pgfqpoint{0.702981in}{1.278103in}}{\pgfqpoint{0.711218in}{1.278103in}}%
\pgfpathclose%
\pgfusepath{stroke,fill}%
\end{pgfscope}%
\begin{pgfscope}%
\pgfpathrectangle{\pgfqpoint{0.100000in}{0.220728in}}{\pgfqpoint{3.696000in}{3.696000in}}%
\pgfusepath{clip}%
\pgfsetbuttcap%
\pgfsetroundjoin%
\definecolor{currentfill}{rgb}{0.121569,0.466667,0.705882}%
\pgfsetfillcolor{currentfill}%
\pgfsetfillopacity{0.606234}%
\pgfsetlinewidth{1.003750pt}%
\definecolor{currentstroke}{rgb}{0.121569,0.466667,0.705882}%
\pgfsetstrokecolor{currentstroke}%
\pgfsetstrokeopacity{0.606234}%
\pgfsetdash{}{0pt}%
\pgfpathmoveto{\pgfqpoint{2.956866in}{2.935981in}}%
\pgfpathcurveto{\pgfqpoint{2.965103in}{2.935981in}}{\pgfqpoint{2.973003in}{2.939253in}}{\pgfqpoint{2.978827in}{2.945077in}}%
\pgfpathcurveto{\pgfqpoint{2.984650in}{2.950901in}}{\pgfqpoint{2.987923in}{2.958801in}}{\pgfqpoint{2.987923in}{2.967037in}}%
\pgfpathcurveto{\pgfqpoint{2.987923in}{2.975273in}}{\pgfqpoint{2.984650in}{2.983174in}}{\pgfqpoint{2.978827in}{2.988997in}}%
\pgfpathcurveto{\pgfqpoint{2.973003in}{2.994821in}}{\pgfqpoint{2.965103in}{2.998094in}}{\pgfqpoint{2.956866in}{2.998094in}}%
\pgfpathcurveto{\pgfqpoint{2.948630in}{2.998094in}}{\pgfqpoint{2.940730in}{2.994821in}}{\pgfqpoint{2.934906in}{2.988997in}}%
\pgfpathcurveto{\pgfqpoint{2.929082in}{2.983174in}}{\pgfqpoint{2.925810in}{2.975273in}}{\pgfqpoint{2.925810in}{2.967037in}}%
\pgfpathcurveto{\pgfqpoint{2.925810in}{2.958801in}}{\pgfqpoint{2.929082in}{2.950901in}}{\pgfqpoint{2.934906in}{2.945077in}}%
\pgfpathcurveto{\pgfqpoint{2.940730in}{2.939253in}}{\pgfqpoint{2.948630in}{2.935981in}}{\pgfqpoint{2.956866in}{2.935981in}}%
\pgfpathclose%
\pgfusepath{stroke,fill}%
\end{pgfscope}%
\begin{pgfscope}%
\pgfpathrectangle{\pgfqpoint{0.100000in}{0.220728in}}{\pgfqpoint{3.696000in}{3.696000in}}%
\pgfusepath{clip}%
\pgfsetbuttcap%
\pgfsetroundjoin%
\definecolor{currentfill}{rgb}{0.121569,0.466667,0.705882}%
\pgfsetfillcolor{currentfill}%
\pgfsetfillopacity{0.607868}%
\pgfsetlinewidth{1.003750pt}%
\definecolor{currentstroke}{rgb}{0.121569,0.466667,0.705882}%
\pgfsetstrokecolor{currentstroke}%
\pgfsetstrokeopacity{0.607868}%
\pgfsetdash{}{0pt}%
\pgfpathmoveto{\pgfqpoint{0.743806in}{1.265575in}}%
\pgfpathcurveto{\pgfqpoint{0.752042in}{1.265575in}}{\pgfqpoint{0.759942in}{1.268847in}}{\pgfqpoint{0.765766in}{1.274671in}}%
\pgfpathcurveto{\pgfqpoint{0.771590in}{1.280495in}}{\pgfqpoint{0.774862in}{1.288395in}}{\pgfqpoint{0.774862in}{1.296631in}}%
\pgfpathcurveto{\pgfqpoint{0.774862in}{1.304868in}}{\pgfqpoint{0.771590in}{1.312768in}}{\pgfqpoint{0.765766in}{1.318592in}}%
\pgfpathcurveto{\pgfqpoint{0.759942in}{1.324416in}}{\pgfqpoint{0.752042in}{1.327688in}}{\pgfqpoint{0.743806in}{1.327688in}}%
\pgfpathcurveto{\pgfqpoint{0.735570in}{1.327688in}}{\pgfqpoint{0.727670in}{1.324416in}}{\pgfqpoint{0.721846in}{1.318592in}}%
\pgfpathcurveto{\pgfqpoint{0.716022in}{1.312768in}}{\pgfqpoint{0.712749in}{1.304868in}}{\pgfqpoint{0.712749in}{1.296631in}}%
\pgfpathcurveto{\pgfqpoint{0.712749in}{1.288395in}}{\pgfqpoint{0.716022in}{1.280495in}}{\pgfqpoint{0.721846in}{1.274671in}}%
\pgfpathcurveto{\pgfqpoint{0.727670in}{1.268847in}}{\pgfqpoint{0.735570in}{1.265575in}}{\pgfqpoint{0.743806in}{1.265575in}}%
\pgfpathclose%
\pgfusepath{stroke,fill}%
\end{pgfscope}%
\begin{pgfscope}%
\pgfpathrectangle{\pgfqpoint{0.100000in}{0.220728in}}{\pgfqpoint{3.696000in}{3.696000in}}%
\pgfusepath{clip}%
\pgfsetbuttcap%
\pgfsetroundjoin%
\definecolor{currentfill}{rgb}{0.121569,0.466667,0.705882}%
\pgfsetfillcolor{currentfill}%
\pgfsetfillopacity{0.608173}%
\pgfsetlinewidth{1.003750pt}%
\definecolor{currentstroke}{rgb}{0.121569,0.466667,0.705882}%
\pgfsetstrokecolor{currentstroke}%
\pgfsetstrokeopacity{0.608173}%
\pgfsetdash{}{0pt}%
\pgfpathmoveto{\pgfqpoint{2.964366in}{2.934829in}}%
\pgfpathcurveto{\pgfqpoint{2.972602in}{2.934829in}}{\pgfqpoint{2.980503in}{2.938102in}}{\pgfqpoint{2.986326in}{2.943926in}}%
\pgfpathcurveto{\pgfqpoint{2.992150in}{2.949750in}}{\pgfqpoint{2.995423in}{2.957650in}}{\pgfqpoint{2.995423in}{2.965886in}}%
\pgfpathcurveto{\pgfqpoint{2.995423in}{2.974122in}}{\pgfqpoint{2.992150in}{2.982022in}}{\pgfqpoint{2.986326in}{2.987846in}}%
\pgfpathcurveto{\pgfqpoint{2.980503in}{2.993670in}}{\pgfqpoint{2.972602in}{2.996942in}}{\pgfqpoint{2.964366in}{2.996942in}}%
\pgfpathcurveto{\pgfqpoint{2.956130in}{2.996942in}}{\pgfqpoint{2.948230in}{2.993670in}}{\pgfqpoint{2.942406in}{2.987846in}}%
\pgfpathcurveto{\pgfqpoint{2.936582in}{2.982022in}}{\pgfqpoint{2.933310in}{2.974122in}}{\pgfqpoint{2.933310in}{2.965886in}}%
\pgfpathcurveto{\pgfqpoint{2.933310in}{2.957650in}}{\pgfqpoint{2.936582in}{2.949750in}}{\pgfqpoint{2.942406in}{2.943926in}}%
\pgfpathcurveto{\pgfqpoint{2.948230in}{2.938102in}}{\pgfqpoint{2.956130in}{2.934829in}}{\pgfqpoint{2.964366in}{2.934829in}}%
\pgfpathclose%
\pgfusepath{stroke,fill}%
\end{pgfscope}%
\begin{pgfscope}%
\pgfpathrectangle{\pgfqpoint{0.100000in}{0.220728in}}{\pgfqpoint{3.696000in}{3.696000in}}%
\pgfusepath{clip}%
\pgfsetbuttcap%
\pgfsetroundjoin%
\definecolor{currentfill}{rgb}{0.121569,0.466667,0.705882}%
\pgfsetfillcolor{currentfill}%
\pgfsetfillopacity{0.610027}%
\pgfsetlinewidth{1.003750pt}%
\definecolor{currentstroke}{rgb}{0.121569,0.466667,0.705882}%
\pgfsetstrokecolor{currentstroke}%
\pgfsetstrokeopacity{0.610027}%
\pgfsetdash{}{0pt}%
\pgfpathmoveto{\pgfqpoint{0.702529in}{1.250729in}}%
\pgfpathcurveto{\pgfqpoint{0.710765in}{1.250729in}}{\pgfqpoint{0.718665in}{1.254001in}}{\pgfqpoint{0.724489in}{1.259825in}}%
\pgfpathcurveto{\pgfqpoint{0.730313in}{1.265649in}}{\pgfqpoint{0.733586in}{1.273549in}}{\pgfqpoint{0.733586in}{1.281786in}}%
\pgfpathcurveto{\pgfqpoint{0.733586in}{1.290022in}}{\pgfqpoint{0.730313in}{1.297922in}}{\pgfqpoint{0.724489in}{1.303746in}}%
\pgfpathcurveto{\pgfqpoint{0.718665in}{1.309570in}}{\pgfqpoint{0.710765in}{1.312842in}}{\pgfqpoint{0.702529in}{1.312842in}}%
\pgfpathcurveto{\pgfqpoint{0.694293in}{1.312842in}}{\pgfqpoint{0.686393in}{1.309570in}}{\pgfqpoint{0.680569in}{1.303746in}}%
\pgfpathcurveto{\pgfqpoint{0.674745in}{1.297922in}}{\pgfqpoint{0.671473in}{1.290022in}}{\pgfqpoint{0.671473in}{1.281786in}}%
\pgfpathcurveto{\pgfqpoint{0.671473in}{1.273549in}}{\pgfqpoint{0.674745in}{1.265649in}}{\pgfqpoint{0.680569in}{1.259825in}}%
\pgfpathcurveto{\pgfqpoint{0.686393in}{1.254001in}}{\pgfqpoint{0.694293in}{1.250729in}}{\pgfqpoint{0.702529in}{1.250729in}}%
\pgfpathclose%
\pgfusepath{stroke,fill}%
\end{pgfscope}%
\begin{pgfscope}%
\pgfpathrectangle{\pgfqpoint{0.100000in}{0.220728in}}{\pgfqpoint{3.696000in}{3.696000in}}%
\pgfusepath{clip}%
\pgfsetbuttcap%
\pgfsetroundjoin%
\definecolor{currentfill}{rgb}{0.121569,0.466667,0.705882}%
\pgfsetfillcolor{currentfill}%
\pgfsetfillopacity{0.610516}%
\pgfsetlinewidth{1.003750pt}%
\definecolor{currentstroke}{rgb}{0.121569,0.466667,0.705882}%
\pgfsetstrokecolor{currentstroke}%
\pgfsetstrokeopacity{0.610516}%
\pgfsetdash{}{0pt}%
\pgfpathmoveto{\pgfqpoint{2.975252in}{2.933521in}}%
\pgfpathcurveto{\pgfqpoint{2.983488in}{2.933521in}}{\pgfqpoint{2.991388in}{2.936793in}}{\pgfqpoint{2.997212in}{2.942617in}}%
\pgfpathcurveto{\pgfqpoint{3.003036in}{2.948441in}}{\pgfqpoint{3.006308in}{2.956341in}}{\pgfqpoint{3.006308in}{2.964577in}}%
\pgfpathcurveto{\pgfqpoint{3.006308in}{2.972814in}}{\pgfqpoint{3.003036in}{2.980714in}}{\pgfqpoint{2.997212in}{2.986537in}}%
\pgfpathcurveto{\pgfqpoint{2.991388in}{2.992361in}}{\pgfqpoint{2.983488in}{2.995634in}}{\pgfqpoint{2.975252in}{2.995634in}}%
\pgfpathcurveto{\pgfqpoint{2.967016in}{2.995634in}}{\pgfqpoint{2.959115in}{2.992361in}}{\pgfqpoint{2.953292in}{2.986537in}}%
\pgfpathcurveto{\pgfqpoint{2.947468in}{2.980714in}}{\pgfqpoint{2.944195in}{2.972814in}}{\pgfqpoint{2.944195in}{2.964577in}}%
\pgfpathcurveto{\pgfqpoint{2.944195in}{2.956341in}}{\pgfqpoint{2.947468in}{2.948441in}}{\pgfqpoint{2.953292in}{2.942617in}}%
\pgfpathcurveto{\pgfqpoint{2.959115in}{2.936793in}}{\pgfqpoint{2.967016in}{2.933521in}}{\pgfqpoint{2.975252in}{2.933521in}}%
\pgfpathclose%
\pgfusepath{stroke,fill}%
\end{pgfscope}%
\begin{pgfscope}%
\pgfpathrectangle{\pgfqpoint{0.100000in}{0.220728in}}{\pgfqpoint{3.696000in}{3.696000in}}%
\pgfusepath{clip}%
\pgfsetbuttcap%
\pgfsetroundjoin%
\definecolor{currentfill}{rgb}{0.121569,0.466667,0.705882}%
\pgfsetfillcolor{currentfill}%
\pgfsetfillopacity{0.612794}%
\pgfsetlinewidth{1.003750pt}%
\definecolor{currentstroke}{rgb}{0.121569,0.466667,0.705882}%
\pgfsetstrokecolor{currentstroke}%
\pgfsetstrokeopacity{0.612794}%
\pgfsetdash{}{0pt}%
\pgfpathmoveto{\pgfqpoint{0.690628in}{1.232030in}}%
\pgfpathcurveto{\pgfqpoint{0.698865in}{1.232030in}}{\pgfqpoint{0.706765in}{1.235302in}}{\pgfqpoint{0.712589in}{1.241126in}}%
\pgfpathcurveto{\pgfqpoint{0.718413in}{1.246950in}}{\pgfqpoint{0.721685in}{1.254850in}}{\pgfqpoint{0.721685in}{1.263086in}}%
\pgfpathcurveto{\pgfqpoint{0.721685in}{1.271323in}}{\pgfqpoint{0.718413in}{1.279223in}}{\pgfqpoint{0.712589in}{1.285047in}}%
\pgfpathcurveto{\pgfqpoint{0.706765in}{1.290871in}}{\pgfqpoint{0.698865in}{1.294143in}}{\pgfqpoint{0.690628in}{1.294143in}}%
\pgfpathcurveto{\pgfqpoint{0.682392in}{1.294143in}}{\pgfqpoint{0.674492in}{1.290871in}}{\pgfqpoint{0.668668in}{1.285047in}}%
\pgfpathcurveto{\pgfqpoint{0.662844in}{1.279223in}}{\pgfqpoint{0.659572in}{1.271323in}}{\pgfqpoint{0.659572in}{1.263086in}}%
\pgfpathcurveto{\pgfqpoint{0.659572in}{1.254850in}}{\pgfqpoint{0.662844in}{1.246950in}}{\pgfqpoint{0.668668in}{1.241126in}}%
\pgfpathcurveto{\pgfqpoint{0.674492in}{1.235302in}}{\pgfqpoint{0.682392in}{1.232030in}}{\pgfqpoint{0.690628in}{1.232030in}}%
\pgfpathclose%
\pgfusepath{stroke,fill}%
\end{pgfscope}%
\begin{pgfscope}%
\pgfpathrectangle{\pgfqpoint{0.100000in}{0.220728in}}{\pgfqpoint{3.696000in}{3.696000in}}%
\pgfusepath{clip}%
\pgfsetbuttcap%
\pgfsetroundjoin%
\definecolor{currentfill}{rgb}{0.121569,0.466667,0.705882}%
\pgfsetfillcolor{currentfill}%
\pgfsetfillopacity{0.613578}%
\pgfsetlinewidth{1.003750pt}%
\definecolor{currentstroke}{rgb}{0.121569,0.466667,0.705882}%
\pgfsetstrokecolor{currentstroke}%
\pgfsetstrokeopacity{0.613578}%
\pgfsetdash{}{0pt}%
\pgfpathmoveto{\pgfqpoint{0.714371in}{1.238128in}}%
\pgfpathcurveto{\pgfqpoint{0.722607in}{1.238128in}}{\pgfqpoint{0.730507in}{1.241400in}}{\pgfqpoint{0.736331in}{1.247224in}}%
\pgfpathcurveto{\pgfqpoint{0.742155in}{1.253048in}}{\pgfqpoint{0.745427in}{1.260948in}}{\pgfqpoint{0.745427in}{1.269184in}}%
\pgfpathcurveto{\pgfqpoint{0.745427in}{1.277421in}}{\pgfqpoint{0.742155in}{1.285321in}}{\pgfqpoint{0.736331in}{1.291145in}}%
\pgfpathcurveto{\pgfqpoint{0.730507in}{1.296969in}}{\pgfqpoint{0.722607in}{1.300241in}}{\pgfqpoint{0.714371in}{1.300241in}}%
\pgfpathcurveto{\pgfqpoint{0.706135in}{1.300241in}}{\pgfqpoint{0.698234in}{1.296969in}}{\pgfqpoint{0.692411in}{1.291145in}}%
\pgfpathcurveto{\pgfqpoint{0.686587in}{1.285321in}}{\pgfqpoint{0.683314in}{1.277421in}}{\pgfqpoint{0.683314in}{1.269184in}}%
\pgfpathcurveto{\pgfqpoint{0.683314in}{1.260948in}}{\pgfqpoint{0.686587in}{1.253048in}}{\pgfqpoint{0.692411in}{1.247224in}}%
\pgfpathcurveto{\pgfqpoint{0.698234in}{1.241400in}}{\pgfqpoint{0.706135in}{1.238128in}}{\pgfqpoint{0.714371in}{1.238128in}}%
\pgfpathclose%
\pgfusepath{stroke,fill}%
\end{pgfscope}%
\begin{pgfscope}%
\pgfpathrectangle{\pgfqpoint{0.100000in}{0.220728in}}{\pgfqpoint{3.696000in}{3.696000in}}%
\pgfusepath{clip}%
\pgfsetbuttcap%
\pgfsetroundjoin%
\definecolor{currentfill}{rgb}{0.121569,0.466667,0.705882}%
\pgfsetfillcolor{currentfill}%
\pgfsetfillopacity{0.614366}%
\pgfsetlinewidth{1.003750pt}%
\definecolor{currentstroke}{rgb}{0.121569,0.466667,0.705882}%
\pgfsetstrokecolor{currentstroke}%
\pgfsetstrokeopacity{0.614366}%
\pgfsetdash{}{0pt}%
\pgfpathmoveto{\pgfqpoint{0.686478in}{1.224753in}}%
\pgfpathcurveto{\pgfqpoint{0.694714in}{1.224753in}}{\pgfqpoint{0.702614in}{1.228025in}}{\pgfqpoint{0.708438in}{1.233849in}}%
\pgfpathcurveto{\pgfqpoint{0.714262in}{1.239673in}}{\pgfqpoint{0.717534in}{1.247573in}}{\pgfqpoint{0.717534in}{1.255810in}}%
\pgfpathcurveto{\pgfqpoint{0.717534in}{1.264046in}}{\pgfqpoint{0.714262in}{1.271946in}}{\pgfqpoint{0.708438in}{1.277770in}}%
\pgfpathcurveto{\pgfqpoint{0.702614in}{1.283594in}}{\pgfqpoint{0.694714in}{1.286866in}}{\pgfqpoint{0.686478in}{1.286866in}}%
\pgfpathcurveto{\pgfqpoint{0.678242in}{1.286866in}}{\pgfqpoint{0.670342in}{1.283594in}}{\pgfqpoint{0.664518in}{1.277770in}}%
\pgfpathcurveto{\pgfqpoint{0.658694in}{1.271946in}}{\pgfqpoint{0.655421in}{1.264046in}}{\pgfqpoint{0.655421in}{1.255810in}}%
\pgfpathcurveto{\pgfqpoint{0.655421in}{1.247573in}}{\pgfqpoint{0.658694in}{1.239673in}}{\pgfqpoint{0.664518in}{1.233849in}}%
\pgfpathcurveto{\pgfqpoint{0.670342in}{1.228025in}}{\pgfqpoint{0.678242in}{1.224753in}}{\pgfqpoint{0.686478in}{1.224753in}}%
\pgfpathclose%
\pgfusepath{stroke,fill}%
\end{pgfscope}%
\begin{pgfscope}%
\pgfpathrectangle{\pgfqpoint{0.100000in}{0.220728in}}{\pgfqpoint{3.696000in}{3.696000in}}%
\pgfusepath{clip}%
\pgfsetbuttcap%
\pgfsetroundjoin%
\definecolor{currentfill}{rgb}{0.121569,0.466667,0.705882}%
\pgfsetfillcolor{currentfill}%
\pgfsetfillopacity{0.614856}%
\pgfsetlinewidth{1.003750pt}%
\definecolor{currentstroke}{rgb}{0.121569,0.466667,0.705882}%
\pgfsetstrokecolor{currentstroke}%
\pgfsetstrokeopacity{0.614856}%
\pgfsetdash{}{0pt}%
\pgfpathmoveto{\pgfqpoint{2.993130in}{2.932168in}}%
\pgfpathcurveto{\pgfqpoint{3.001367in}{2.932168in}}{\pgfqpoint{3.009267in}{2.935441in}}{\pgfqpoint{3.015091in}{2.941265in}}%
\pgfpathcurveto{\pgfqpoint{3.020915in}{2.947088in}}{\pgfqpoint{3.024187in}{2.954989in}}{\pgfqpoint{3.024187in}{2.963225in}}%
\pgfpathcurveto{\pgfqpoint{3.024187in}{2.971461in}}{\pgfqpoint{3.020915in}{2.979361in}}{\pgfqpoint{3.015091in}{2.985185in}}%
\pgfpathcurveto{\pgfqpoint{3.009267in}{2.991009in}}{\pgfqpoint{3.001367in}{2.994281in}}{\pgfqpoint{2.993130in}{2.994281in}}%
\pgfpathcurveto{\pgfqpoint{2.984894in}{2.994281in}}{\pgfqpoint{2.976994in}{2.991009in}}{\pgfqpoint{2.971170in}{2.985185in}}%
\pgfpathcurveto{\pgfqpoint{2.965346in}{2.979361in}}{\pgfqpoint{2.962074in}{2.971461in}}{\pgfqpoint{2.962074in}{2.963225in}}%
\pgfpathcurveto{\pgfqpoint{2.962074in}{2.954989in}}{\pgfqpoint{2.965346in}{2.947088in}}{\pgfqpoint{2.971170in}{2.941265in}}%
\pgfpathcurveto{\pgfqpoint{2.976994in}{2.935441in}}{\pgfqpoint{2.984894in}{2.932168in}}{\pgfqpoint{2.993130in}{2.932168in}}%
\pgfpathclose%
\pgfusepath{stroke,fill}%
\end{pgfscope}%
\begin{pgfscope}%
\pgfpathrectangle{\pgfqpoint{0.100000in}{0.220728in}}{\pgfqpoint{3.696000in}{3.696000in}}%
\pgfusepath{clip}%
\pgfsetbuttcap%
\pgfsetroundjoin%
\definecolor{currentfill}{rgb}{0.121569,0.466667,0.705882}%
\pgfsetfillcolor{currentfill}%
\pgfsetfillopacity{0.615262}%
\pgfsetlinewidth{1.003750pt}%
\definecolor{currentstroke}{rgb}{0.121569,0.466667,0.705882}%
\pgfsetstrokecolor{currentstroke}%
\pgfsetstrokeopacity{0.615262}%
\pgfsetdash{}{0pt}%
\pgfpathmoveto{\pgfqpoint{0.684262in}{1.220743in}}%
\pgfpathcurveto{\pgfqpoint{0.692499in}{1.220743in}}{\pgfqpoint{0.700399in}{1.224015in}}{\pgfqpoint{0.706223in}{1.229839in}}%
\pgfpathcurveto{\pgfqpoint{0.712047in}{1.235663in}}{\pgfqpoint{0.715319in}{1.243563in}}{\pgfqpoint{0.715319in}{1.251799in}}%
\pgfpathcurveto{\pgfqpoint{0.715319in}{1.260035in}}{\pgfqpoint{0.712047in}{1.267935in}}{\pgfqpoint{0.706223in}{1.273759in}}%
\pgfpathcurveto{\pgfqpoint{0.700399in}{1.279583in}}{\pgfqpoint{0.692499in}{1.282856in}}{\pgfqpoint{0.684262in}{1.282856in}}%
\pgfpathcurveto{\pgfqpoint{0.676026in}{1.282856in}}{\pgfqpoint{0.668126in}{1.279583in}}{\pgfqpoint{0.662302in}{1.273759in}}%
\pgfpathcurveto{\pgfqpoint{0.656478in}{1.267935in}}{\pgfqpoint{0.653206in}{1.260035in}}{\pgfqpoint{0.653206in}{1.251799in}}%
\pgfpathcurveto{\pgfqpoint{0.653206in}{1.243563in}}{\pgfqpoint{0.656478in}{1.235663in}}{\pgfqpoint{0.662302in}{1.229839in}}%
\pgfpathcurveto{\pgfqpoint{0.668126in}{1.224015in}}{\pgfqpoint{0.676026in}{1.220743in}}{\pgfqpoint{0.684262in}{1.220743in}}%
\pgfpathclose%
\pgfusepath{stroke,fill}%
\end{pgfscope}%
\begin{pgfscope}%
\pgfpathrectangle{\pgfqpoint{0.100000in}{0.220728in}}{\pgfqpoint{3.696000in}{3.696000in}}%
\pgfusepath{clip}%
\pgfsetbuttcap%
\pgfsetroundjoin%
\definecolor{currentfill}{rgb}{0.121569,0.466667,0.705882}%
\pgfsetfillcolor{currentfill}%
\pgfsetfillopacity{0.616401}%
\pgfsetlinewidth{1.003750pt}%
\definecolor{currentstroke}{rgb}{0.121569,0.466667,0.705882}%
\pgfsetstrokecolor{currentstroke}%
\pgfsetstrokeopacity{0.616401}%
\pgfsetdash{}{0pt}%
\pgfpathmoveto{\pgfqpoint{0.678016in}{1.214504in}}%
\pgfpathcurveto{\pgfqpoint{0.686252in}{1.214504in}}{\pgfqpoint{0.694152in}{1.217776in}}{\pgfqpoint{0.699976in}{1.223600in}}%
\pgfpathcurveto{\pgfqpoint{0.705800in}{1.229424in}}{\pgfqpoint{0.709072in}{1.237324in}}{\pgfqpoint{0.709072in}{1.245560in}}%
\pgfpathcurveto{\pgfqpoint{0.709072in}{1.253797in}}{\pgfqpoint{0.705800in}{1.261697in}}{\pgfqpoint{0.699976in}{1.267521in}}%
\pgfpathcurveto{\pgfqpoint{0.694152in}{1.273345in}}{\pgfqpoint{0.686252in}{1.276617in}}{\pgfqpoint{0.678016in}{1.276617in}}%
\pgfpathcurveto{\pgfqpoint{0.669780in}{1.276617in}}{\pgfqpoint{0.661880in}{1.273345in}}{\pgfqpoint{0.656056in}{1.267521in}}%
\pgfpathcurveto{\pgfqpoint{0.650232in}{1.261697in}}{\pgfqpoint{0.646959in}{1.253797in}}{\pgfqpoint{0.646959in}{1.245560in}}%
\pgfpathcurveto{\pgfqpoint{0.646959in}{1.237324in}}{\pgfqpoint{0.650232in}{1.229424in}}{\pgfqpoint{0.656056in}{1.223600in}}%
\pgfpathcurveto{\pgfqpoint{0.661880in}{1.217776in}}{\pgfqpoint{0.669780in}{1.214504in}}{\pgfqpoint{0.678016in}{1.214504in}}%
\pgfpathclose%
\pgfusepath{stroke,fill}%
\end{pgfscope}%
\begin{pgfscope}%
\pgfpathrectangle{\pgfqpoint{0.100000in}{0.220728in}}{\pgfqpoint{3.696000in}{3.696000in}}%
\pgfusepath{clip}%
\pgfsetbuttcap%
\pgfsetroundjoin%
\definecolor{currentfill}{rgb}{0.121569,0.466667,0.705882}%
\pgfsetfillcolor{currentfill}%
\pgfsetfillopacity{0.616405}%
\pgfsetlinewidth{1.003750pt}%
\definecolor{currentstroke}{rgb}{0.121569,0.466667,0.705882}%
\pgfsetstrokecolor{currentstroke}%
\pgfsetstrokeopacity{0.616405}%
\pgfsetdash{}{0pt}%
\pgfpathmoveto{\pgfqpoint{0.678010in}{1.214488in}}%
\pgfpathcurveto{\pgfqpoint{0.686246in}{1.214488in}}{\pgfqpoint{0.694146in}{1.217760in}}{\pgfqpoint{0.699970in}{1.223584in}}%
\pgfpathcurveto{\pgfqpoint{0.705794in}{1.229408in}}{\pgfqpoint{0.709067in}{1.237308in}}{\pgfqpoint{0.709067in}{1.245544in}}%
\pgfpathcurveto{\pgfqpoint{0.709067in}{1.253780in}}{\pgfqpoint{0.705794in}{1.261680in}}{\pgfqpoint{0.699970in}{1.267504in}}%
\pgfpathcurveto{\pgfqpoint{0.694146in}{1.273328in}}{\pgfqpoint{0.686246in}{1.276601in}}{\pgfqpoint{0.678010in}{1.276601in}}%
\pgfpathcurveto{\pgfqpoint{0.669774in}{1.276601in}}{\pgfqpoint{0.661874in}{1.273328in}}{\pgfqpoint{0.656050in}{1.267504in}}%
\pgfpathcurveto{\pgfqpoint{0.650226in}{1.261680in}}{\pgfqpoint{0.646954in}{1.253780in}}{\pgfqpoint{0.646954in}{1.245544in}}%
\pgfpathcurveto{\pgfqpoint{0.646954in}{1.237308in}}{\pgfqpoint{0.650226in}{1.229408in}}{\pgfqpoint{0.656050in}{1.223584in}}%
\pgfpathcurveto{\pgfqpoint{0.661874in}{1.217760in}}{\pgfqpoint{0.669774in}{1.214488in}}{\pgfqpoint{0.678010in}{1.214488in}}%
\pgfpathclose%
\pgfusepath{stroke,fill}%
\end{pgfscope}%
\begin{pgfscope}%
\pgfpathrectangle{\pgfqpoint{0.100000in}{0.220728in}}{\pgfqpoint{3.696000in}{3.696000in}}%
\pgfusepath{clip}%
\pgfsetbuttcap%
\pgfsetroundjoin%
\definecolor{currentfill}{rgb}{0.121569,0.466667,0.705882}%
\pgfsetfillcolor{currentfill}%
\pgfsetfillopacity{0.616406}%
\pgfsetlinewidth{1.003750pt}%
\definecolor{currentstroke}{rgb}{0.121569,0.466667,0.705882}%
\pgfsetstrokecolor{currentstroke}%
\pgfsetstrokeopacity{0.616406}%
\pgfsetdash{}{0pt}%
\pgfpathmoveto{\pgfqpoint{0.698593in}{1.220695in}}%
\pgfpathcurveto{\pgfqpoint{0.706829in}{1.220695in}}{\pgfqpoint{0.714729in}{1.223968in}}{\pgfqpoint{0.720553in}{1.229792in}}%
\pgfpathcurveto{\pgfqpoint{0.726377in}{1.235615in}}{\pgfqpoint{0.729649in}{1.243516in}}{\pgfqpoint{0.729649in}{1.251752in}}%
\pgfpathcurveto{\pgfqpoint{0.729649in}{1.259988in}}{\pgfqpoint{0.726377in}{1.267888in}}{\pgfqpoint{0.720553in}{1.273712in}}%
\pgfpathcurveto{\pgfqpoint{0.714729in}{1.279536in}}{\pgfqpoint{0.706829in}{1.282808in}}{\pgfqpoint{0.698593in}{1.282808in}}%
\pgfpathcurveto{\pgfqpoint{0.690357in}{1.282808in}}{\pgfqpoint{0.682456in}{1.279536in}}{\pgfqpoint{0.676633in}{1.273712in}}%
\pgfpathcurveto{\pgfqpoint{0.670809in}{1.267888in}}{\pgfqpoint{0.667536in}{1.259988in}}{\pgfqpoint{0.667536in}{1.251752in}}%
\pgfpathcurveto{\pgfqpoint{0.667536in}{1.243516in}}{\pgfqpoint{0.670809in}{1.235615in}}{\pgfqpoint{0.676633in}{1.229792in}}%
\pgfpathcurveto{\pgfqpoint{0.682456in}{1.223968in}}{\pgfqpoint{0.690357in}{1.220695in}}{\pgfqpoint{0.698593in}{1.220695in}}%
\pgfpathclose%
\pgfusepath{stroke,fill}%
\end{pgfscope}%
\begin{pgfscope}%
\pgfpathrectangle{\pgfqpoint{0.100000in}{0.220728in}}{\pgfqpoint{3.696000in}{3.696000in}}%
\pgfusepath{clip}%
\pgfsetbuttcap%
\pgfsetroundjoin%
\definecolor{currentfill}{rgb}{0.121569,0.466667,0.705882}%
\pgfsetfillcolor{currentfill}%
\pgfsetfillopacity{0.616411}%
\pgfsetlinewidth{1.003750pt}%
\definecolor{currentstroke}{rgb}{0.121569,0.466667,0.705882}%
\pgfsetstrokecolor{currentstroke}%
\pgfsetstrokeopacity{0.616411}%
\pgfsetdash{}{0pt}%
\pgfpathmoveto{\pgfqpoint{0.677994in}{1.214459in}}%
\pgfpathcurveto{\pgfqpoint{0.686230in}{1.214459in}}{\pgfqpoint{0.694130in}{1.217731in}}{\pgfqpoint{0.699954in}{1.223555in}}%
\pgfpathcurveto{\pgfqpoint{0.705778in}{1.229379in}}{\pgfqpoint{0.709050in}{1.237279in}}{\pgfqpoint{0.709050in}{1.245515in}}%
\pgfpathcurveto{\pgfqpoint{0.709050in}{1.253751in}}{\pgfqpoint{0.705778in}{1.261651in}}{\pgfqpoint{0.699954in}{1.267475in}}%
\pgfpathcurveto{\pgfqpoint{0.694130in}{1.273299in}}{\pgfqpoint{0.686230in}{1.276572in}}{\pgfqpoint{0.677994in}{1.276572in}}%
\pgfpathcurveto{\pgfqpoint{0.669758in}{1.276572in}}{\pgfqpoint{0.661858in}{1.273299in}}{\pgfqpoint{0.656034in}{1.267475in}}%
\pgfpathcurveto{\pgfqpoint{0.650210in}{1.261651in}}{\pgfqpoint{0.646937in}{1.253751in}}{\pgfqpoint{0.646937in}{1.245515in}}%
\pgfpathcurveto{\pgfqpoint{0.646937in}{1.237279in}}{\pgfqpoint{0.650210in}{1.229379in}}{\pgfqpoint{0.656034in}{1.223555in}}%
\pgfpathcurveto{\pgfqpoint{0.661858in}{1.217731in}}{\pgfqpoint{0.669758in}{1.214459in}}{\pgfqpoint{0.677994in}{1.214459in}}%
\pgfpathclose%
\pgfusepath{stroke,fill}%
\end{pgfscope}%
\begin{pgfscope}%
\pgfpathrectangle{\pgfqpoint{0.100000in}{0.220728in}}{\pgfqpoint{3.696000in}{3.696000in}}%
\pgfusepath{clip}%
\pgfsetbuttcap%
\pgfsetroundjoin%
\definecolor{currentfill}{rgb}{0.121569,0.466667,0.705882}%
\pgfsetfillcolor{currentfill}%
\pgfsetfillopacity{0.616423}%
\pgfsetlinewidth{1.003750pt}%
\definecolor{currentstroke}{rgb}{0.121569,0.466667,0.705882}%
\pgfsetstrokecolor{currentstroke}%
\pgfsetstrokeopacity{0.616423}%
\pgfsetdash{}{0pt}%
\pgfpathmoveto{\pgfqpoint{0.677963in}{1.214406in}}%
\pgfpathcurveto{\pgfqpoint{0.686200in}{1.214406in}}{\pgfqpoint{0.694100in}{1.217679in}}{\pgfqpoint{0.699924in}{1.223503in}}%
\pgfpathcurveto{\pgfqpoint{0.705747in}{1.229327in}}{\pgfqpoint{0.709020in}{1.237227in}}{\pgfqpoint{0.709020in}{1.245463in}}%
\pgfpathcurveto{\pgfqpoint{0.709020in}{1.253699in}}{\pgfqpoint{0.705747in}{1.261599in}}{\pgfqpoint{0.699924in}{1.267423in}}%
\pgfpathcurveto{\pgfqpoint{0.694100in}{1.273247in}}{\pgfqpoint{0.686200in}{1.276519in}}{\pgfqpoint{0.677963in}{1.276519in}}%
\pgfpathcurveto{\pgfqpoint{0.669727in}{1.276519in}}{\pgfqpoint{0.661827in}{1.273247in}}{\pgfqpoint{0.656003in}{1.267423in}}%
\pgfpathcurveto{\pgfqpoint{0.650179in}{1.261599in}}{\pgfqpoint{0.646907in}{1.253699in}}{\pgfqpoint{0.646907in}{1.245463in}}%
\pgfpathcurveto{\pgfqpoint{0.646907in}{1.237227in}}{\pgfqpoint{0.650179in}{1.229327in}}{\pgfqpoint{0.656003in}{1.223503in}}%
\pgfpathcurveto{\pgfqpoint{0.661827in}{1.217679in}}{\pgfqpoint{0.669727in}{1.214406in}}{\pgfqpoint{0.677963in}{1.214406in}}%
\pgfpathclose%
\pgfusepath{stroke,fill}%
\end{pgfscope}%
\begin{pgfscope}%
\pgfpathrectangle{\pgfqpoint{0.100000in}{0.220728in}}{\pgfqpoint{3.696000in}{3.696000in}}%
\pgfusepath{clip}%
\pgfsetbuttcap%
\pgfsetroundjoin%
\definecolor{currentfill}{rgb}{0.121569,0.466667,0.705882}%
\pgfsetfillcolor{currentfill}%
\pgfsetfillopacity{0.616443}%
\pgfsetlinewidth{1.003750pt}%
\definecolor{currentstroke}{rgb}{0.121569,0.466667,0.705882}%
\pgfsetstrokecolor{currentstroke}%
\pgfsetstrokeopacity{0.616443}%
\pgfsetdash{}{0pt}%
\pgfpathmoveto{\pgfqpoint{0.677904in}{1.214313in}}%
\pgfpathcurveto{\pgfqpoint{0.686140in}{1.214313in}}{\pgfqpoint{0.694040in}{1.217586in}}{\pgfqpoint{0.699864in}{1.223410in}}%
\pgfpathcurveto{\pgfqpoint{0.705688in}{1.229233in}}{\pgfqpoint{0.708961in}{1.237133in}}{\pgfqpoint{0.708961in}{1.245370in}}%
\pgfpathcurveto{\pgfqpoint{0.708961in}{1.253606in}}{\pgfqpoint{0.705688in}{1.261506in}}{\pgfqpoint{0.699864in}{1.267330in}}%
\pgfpathcurveto{\pgfqpoint{0.694040in}{1.273154in}}{\pgfqpoint{0.686140in}{1.276426in}}{\pgfqpoint{0.677904in}{1.276426in}}%
\pgfpathcurveto{\pgfqpoint{0.669668in}{1.276426in}}{\pgfqpoint{0.661768in}{1.273154in}}{\pgfqpoint{0.655944in}{1.267330in}}%
\pgfpathcurveto{\pgfqpoint{0.650120in}{1.261506in}}{\pgfqpoint{0.646848in}{1.253606in}}{\pgfqpoint{0.646848in}{1.245370in}}%
\pgfpathcurveto{\pgfqpoint{0.646848in}{1.237133in}}{\pgfqpoint{0.650120in}{1.229233in}}{\pgfqpoint{0.655944in}{1.223410in}}%
\pgfpathcurveto{\pgfqpoint{0.661768in}{1.217586in}}{\pgfqpoint{0.669668in}{1.214313in}}{\pgfqpoint{0.677904in}{1.214313in}}%
\pgfpathclose%
\pgfusepath{stroke,fill}%
\end{pgfscope}%
\begin{pgfscope}%
\pgfpathrectangle{\pgfqpoint{0.100000in}{0.220728in}}{\pgfqpoint{3.696000in}{3.696000in}}%
\pgfusepath{clip}%
\pgfsetbuttcap%
\pgfsetroundjoin%
\definecolor{currentfill}{rgb}{0.121569,0.466667,0.705882}%
\pgfsetfillcolor{currentfill}%
\pgfsetfillopacity{0.616479}%
\pgfsetlinewidth{1.003750pt}%
\definecolor{currentstroke}{rgb}{0.121569,0.466667,0.705882}%
\pgfsetstrokecolor{currentstroke}%
\pgfsetstrokeopacity{0.616479}%
\pgfsetdash{}{0pt}%
\pgfpathmoveto{\pgfqpoint{0.677800in}{1.214136in}}%
\pgfpathcurveto{\pgfqpoint{0.686036in}{1.214136in}}{\pgfqpoint{0.693936in}{1.217408in}}{\pgfqpoint{0.699760in}{1.223232in}}%
\pgfpathcurveto{\pgfqpoint{0.705584in}{1.229056in}}{\pgfqpoint{0.708856in}{1.236956in}}{\pgfqpoint{0.708856in}{1.245192in}}%
\pgfpathcurveto{\pgfqpoint{0.708856in}{1.253429in}}{\pgfqpoint{0.705584in}{1.261329in}}{\pgfqpoint{0.699760in}{1.267153in}}%
\pgfpathcurveto{\pgfqpoint{0.693936in}{1.272976in}}{\pgfqpoint{0.686036in}{1.276249in}}{\pgfqpoint{0.677800in}{1.276249in}}%
\pgfpathcurveto{\pgfqpoint{0.669564in}{1.276249in}}{\pgfqpoint{0.661664in}{1.272976in}}{\pgfqpoint{0.655840in}{1.267153in}}%
\pgfpathcurveto{\pgfqpoint{0.650016in}{1.261329in}}{\pgfqpoint{0.646743in}{1.253429in}}{\pgfqpoint{0.646743in}{1.245192in}}%
\pgfpathcurveto{\pgfqpoint{0.646743in}{1.236956in}}{\pgfqpoint{0.650016in}{1.229056in}}{\pgfqpoint{0.655840in}{1.223232in}}%
\pgfpathcurveto{\pgfqpoint{0.661664in}{1.217408in}}{\pgfqpoint{0.669564in}{1.214136in}}{\pgfqpoint{0.677800in}{1.214136in}}%
\pgfpathclose%
\pgfusepath{stroke,fill}%
\end{pgfscope}%
\begin{pgfscope}%
\pgfpathrectangle{\pgfqpoint{0.100000in}{0.220728in}}{\pgfqpoint{3.696000in}{3.696000in}}%
\pgfusepath{clip}%
\pgfsetbuttcap%
\pgfsetroundjoin%
\definecolor{currentfill}{rgb}{0.121569,0.466667,0.705882}%
\pgfsetfillcolor{currentfill}%
\pgfsetfillopacity{0.616543}%
\pgfsetlinewidth{1.003750pt}%
\definecolor{currentstroke}{rgb}{0.121569,0.466667,0.705882}%
\pgfsetstrokecolor{currentstroke}%
\pgfsetstrokeopacity{0.616543}%
\pgfsetdash{}{0pt}%
\pgfpathmoveto{\pgfqpoint{0.677596in}{1.213827in}}%
\pgfpathcurveto{\pgfqpoint{0.685833in}{1.213827in}}{\pgfqpoint{0.693733in}{1.217100in}}{\pgfqpoint{0.699556in}{1.222924in}}%
\pgfpathcurveto{\pgfqpoint{0.705380in}{1.228748in}}{\pgfqpoint{0.708653in}{1.236648in}}{\pgfqpoint{0.708653in}{1.244884in}}%
\pgfpathcurveto{\pgfqpoint{0.708653in}{1.253120in}}{\pgfqpoint{0.705380in}{1.261020in}}{\pgfqpoint{0.699556in}{1.266844in}}%
\pgfpathcurveto{\pgfqpoint{0.693733in}{1.272668in}}{\pgfqpoint{0.685833in}{1.275940in}}{\pgfqpoint{0.677596in}{1.275940in}}%
\pgfpathcurveto{\pgfqpoint{0.669360in}{1.275940in}}{\pgfqpoint{0.661460in}{1.272668in}}{\pgfqpoint{0.655636in}{1.266844in}}%
\pgfpathcurveto{\pgfqpoint{0.649812in}{1.261020in}}{\pgfqpoint{0.646540in}{1.253120in}}{\pgfqpoint{0.646540in}{1.244884in}}%
\pgfpathcurveto{\pgfqpoint{0.646540in}{1.236648in}}{\pgfqpoint{0.649812in}{1.228748in}}{\pgfqpoint{0.655636in}{1.222924in}}%
\pgfpathcurveto{\pgfqpoint{0.661460in}{1.217100in}}{\pgfqpoint{0.669360in}{1.213827in}}{\pgfqpoint{0.677596in}{1.213827in}}%
\pgfpathclose%
\pgfusepath{stroke,fill}%
\end{pgfscope}%
\begin{pgfscope}%
\pgfpathrectangle{\pgfqpoint{0.100000in}{0.220728in}}{\pgfqpoint{3.696000in}{3.696000in}}%
\pgfusepath{clip}%
\pgfsetbuttcap%
\pgfsetroundjoin%
\definecolor{currentfill}{rgb}{0.121569,0.466667,0.705882}%
\pgfsetfillcolor{currentfill}%
\pgfsetfillopacity{0.616660}%
\pgfsetlinewidth{1.003750pt}%
\definecolor{currentstroke}{rgb}{0.121569,0.466667,0.705882}%
\pgfsetstrokecolor{currentstroke}%
\pgfsetstrokeopacity{0.616660}%
\pgfsetdash{}{0pt}%
\pgfpathmoveto{\pgfqpoint{0.677227in}{1.213261in}}%
\pgfpathcurveto{\pgfqpoint{0.685463in}{1.213261in}}{\pgfqpoint{0.693363in}{1.216533in}}{\pgfqpoint{0.699187in}{1.222357in}}%
\pgfpathcurveto{\pgfqpoint{0.705011in}{1.228181in}}{\pgfqpoint{0.708283in}{1.236081in}}{\pgfqpoint{0.708283in}{1.244317in}}%
\pgfpathcurveto{\pgfqpoint{0.708283in}{1.252554in}}{\pgfqpoint{0.705011in}{1.260454in}}{\pgfqpoint{0.699187in}{1.266278in}}%
\pgfpathcurveto{\pgfqpoint{0.693363in}{1.272102in}}{\pgfqpoint{0.685463in}{1.275374in}}{\pgfqpoint{0.677227in}{1.275374in}}%
\pgfpathcurveto{\pgfqpoint{0.668990in}{1.275374in}}{\pgfqpoint{0.661090in}{1.272102in}}{\pgfqpoint{0.655266in}{1.266278in}}%
\pgfpathcurveto{\pgfqpoint{0.649442in}{1.260454in}}{\pgfqpoint{0.646170in}{1.252554in}}{\pgfqpoint{0.646170in}{1.244317in}}%
\pgfpathcurveto{\pgfqpoint{0.646170in}{1.236081in}}{\pgfqpoint{0.649442in}{1.228181in}}{\pgfqpoint{0.655266in}{1.222357in}}%
\pgfpathcurveto{\pgfqpoint{0.661090in}{1.216533in}}{\pgfqpoint{0.668990in}{1.213261in}}{\pgfqpoint{0.677227in}{1.213261in}}%
\pgfpathclose%
\pgfusepath{stroke,fill}%
\end{pgfscope}%
\begin{pgfscope}%
\pgfpathrectangle{\pgfqpoint{0.100000in}{0.220728in}}{\pgfqpoint{3.696000in}{3.696000in}}%
\pgfusepath{clip}%
\pgfsetbuttcap%
\pgfsetroundjoin%
\definecolor{currentfill}{rgb}{0.121569,0.466667,0.705882}%
\pgfsetfillcolor{currentfill}%
\pgfsetfillopacity{0.616876}%
\pgfsetlinewidth{1.003750pt}%
\definecolor{currentstroke}{rgb}{0.121569,0.466667,0.705882}%
\pgfsetstrokecolor{currentstroke}%
\pgfsetstrokeopacity{0.616876}%
\pgfsetdash{}{0pt}%
\pgfpathmoveto{\pgfqpoint{0.676568in}{1.212238in}}%
\pgfpathcurveto{\pgfqpoint{0.684804in}{1.212238in}}{\pgfqpoint{0.692704in}{1.215510in}}{\pgfqpoint{0.698528in}{1.221334in}}%
\pgfpathcurveto{\pgfqpoint{0.704352in}{1.227158in}}{\pgfqpoint{0.707624in}{1.235058in}}{\pgfqpoint{0.707624in}{1.243294in}}%
\pgfpathcurveto{\pgfqpoint{0.707624in}{1.251531in}}{\pgfqpoint{0.704352in}{1.259431in}}{\pgfqpoint{0.698528in}{1.265255in}}%
\pgfpathcurveto{\pgfqpoint{0.692704in}{1.271079in}}{\pgfqpoint{0.684804in}{1.274351in}}{\pgfqpoint{0.676568in}{1.274351in}}%
\pgfpathcurveto{\pgfqpoint{0.668332in}{1.274351in}}{\pgfqpoint{0.660431in}{1.271079in}}{\pgfqpoint{0.654608in}{1.265255in}}%
\pgfpathcurveto{\pgfqpoint{0.648784in}{1.259431in}}{\pgfqpoint{0.645511in}{1.251531in}}{\pgfqpoint{0.645511in}{1.243294in}}%
\pgfpathcurveto{\pgfqpoint{0.645511in}{1.235058in}}{\pgfqpoint{0.648784in}{1.227158in}}{\pgfqpoint{0.654608in}{1.221334in}}%
\pgfpathcurveto{\pgfqpoint{0.660431in}{1.215510in}}{\pgfqpoint{0.668332in}{1.212238in}}{\pgfqpoint{0.676568in}{1.212238in}}%
\pgfpathclose%
\pgfusepath{stroke,fill}%
\end{pgfscope}%
\begin{pgfscope}%
\pgfpathrectangle{\pgfqpoint{0.100000in}{0.220728in}}{\pgfqpoint{3.696000in}{3.696000in}}%
\pgfusepath{clip}%
\pgfsetbuttcap%
\pgfsetroundjoin%
\definecolor{currentfill}{rgb}{0.121569,0.466667,0.705882}%
\pgfsetfillcolor{currentfill}%
\pgfsetfillopacity{0.617299}%
\pgfsetlinewidth{1.003750pt}%
\definecolor{currentstroke}{rgb}{0.121569,0.466667,0.705882}%
\pgfsetstrokecolor{currentstroke}%
\pgfsetstrokeopacity{0.617299}%
\pgfsetdash{}{0pt}%
\pgfpathmoveto{\pgfqpoint{0.675405in}{1.210476in}}%
\pgfpathcurveto{\pgfqpoint{0.683642in}{1.210476in}}{\pgfqpoint{0.691542in}{1.213748in}}{\pgfqpoint{0.697366in}{1.219572in}}%
\pgfpathcurveto{\pgfqpoint{0.703190in}{1.225396in}}{\pgfqpoint{0.706462in}{1.233296in}}{\pgfqpoint{0.706462in}{1.241532in}}%
\pgfpathcurveto{\pgfqpoint{0.706462in}{1.249769in}}{\pgfqpoint{0.703190in}{1.257669in}}{\pgfqpoint{0.697366in}{1.263493in}}%
\pgfpathcurveto{\pgfqpoint{0.691542in}{1.269317in}}{\pgfqpoint{0.683642in}{1.272589in}}{\pgfqpoint{0.675405in}{1.272589in}}%
\pgfpathcurveto{\pgfqpoint{0.667169in}{1.272589in}}{\pgfqpoint{0.659269in}{1.269317in}}{\pgfqpoint{0.653445in}{1.263493in}}%
\pgfpathcurveto{\pgfqpoint{0.647621in}{1.257669in}}{\pgfqpoint{0.644349in}{1.249769in}}{\pgfqpoint{0.644349in}{1.241532in}}%
\pgfpathcurveto{\pgfqpoint{0.644349in}{1.233296in}}{\pgfqpoint{0.647621in}{1.225396in}}{\pgfqpoint{0.653445in}{1.219572in}}%
\pgfpathcurveto{\pgfqpoint{0.659269in}{1.213748in}}{\pgfqpoint{0.667169in}{1.210476in}}{\pgfqpoint{0.675405in}{1.210476in}}%
\pgfpathclose%
\pgfusepath{stroke,fill}%
\end{pgfscope}%
\begin{pgfscope}%
\pgfpathrectangle{\pgfqpoint{0.100000in}{0.220728in}}{\pgfqpoint{3.696000in}{3.696000in}}%
\pgfusepath{clip}%
\pgfsetbuttcap%
\pgfsetroundjoin%
\definecolor{currentfill}{rgb}{0.121569,0.466667,0.705882}%
\pgfsetfillcolor{currentfill}%
\pgfsetfillopacity{0.617299}%
\pgfsetlinewidth{1.003750pt}%
\definecolor{currentstroke}{rgb}{0.121569,0.466667,0.705882}%
\pgfsetstrokecolor{currentstroke}%
\pgfsetstrokeopacity{0.617299}%
\pgfsetdash{}{0pt}%
\pgfpathmoveto{\pgfqpoint{0.675405in}{1.210475in}}%
\pgfpathcurveto{\pgfqpoint{0.683641in}{1.210475in}}{\pgfqpoint{0.691541in}{1.213747in}}{\pgfqpoint{0.697365in}{1.219571in}}%
\pgfpathcurveto{\pgfqpoint{0.703189in}{1.225395in}}{\pgfqpoint{0.706461in}{1.233295in}}{\pgfqpoint{0.706461in}{1.241532in}}%
\pgfpathcurveto{\pgfqpoint{0.706461in}{1.249768in}}{\pgfqpoint{0.703189in}{1.257668in}}{\pgfqpoint{0.697365in}{1.263492in}}%
\pgfpathcurveto{\pgfqpoint{0.691541in}{1.269316in}}{\pgfqpoint{0.683641in}{1.272588in}}{\pgfqpoint{0.675405in}{1.272588in}}%
\pgfpathcurveto{\pgfqpoint{0.667169in}{1.272588in}}{\pgfqpoint{0.659268in}{1.269316in}}{\pgfqpoint{0.653445in}{1.263492in}}%
\pgfpathcurveto{\pgfqpoint{0.647621in}{1.257668in}}{\pgfqpoint{0.644348in}{1.249768in}}{\pgfqpoint{0.644348in}{1.241532in}}%
\pgfpathcurveto{\pgfqpoint{0.644348in}{1.233295in}}{\pgfqpoint{0.647621in}{1.225395in}}{\pgfqpoint{0.653445in}{1.219571in}}%
\pgfpathcurveto{\pgfqpoint{0.659268in}{1.213747in}}{\pgfqpoint{0.667169in}{1.210475in}}{\pgfqpoint{0.675405in}{1.210475in}}%
\pgfpathclose%
\pgfusepath{stroke,fill}%
\end{pgfscope}%
\begin{pgfscope}%
\pgfpathrectangle{\pgfqpoint{0.100000in}{0.220728in}}{\pgfqpoint{3.696000in}{3.696000in}}%
\pgfusepath{clip}%
\pgfsetbuttcap%
\pgfsetroundjoin%
\definecolor{currentfill}{rgb}{0.121569,0.466667,0.705882}%
\pgfsetfillcolor{currentfill}%
\pgfsetfillopacity{0.617299}%
\pgfsetlinewidth{1.003750pt}%
\definecolor{currentstroke}{rgb}{0.121569,0.466667,0.705882}%
\pgfsetstrokecolor{currentstroke}%
\pgfsetstrokeopacity{0.617299}%
\pgfsetdash{}{0pt}%
\pgfpathmoveto{\pgfqpoint{0.675404in}{1.210473in}}%
\pgfpathcurveto{\pgfqpoint{0.683640in}{1.210473in}}{\pgfqpoint{0.691540in}{1.213746in}}{\pgfqpoint{0.697364in}{1.219570in}}%
\pgfpathcurveto{\pgfqpoint{0.703188in}{1.225394in}}{\pgfqpoint{0.706460in}{1.233294in}}{\pgfqpoint{0.706460in}{1.241530in}}%
\pgfpathcurveto{\pgfqpoint{0.706460in}{1.249766in}}{\pgfqpoint{0.703188in}{1.257666in}}{\pgfqpoint{0.697364in}{1.263490in}}%
\pgfpathcurveto{\pgfqpoint{0.691540in}{1.269314in}}{\pgfqpoint{0.683640in}{1.272586in}}{\pgfqpoint{0.675404in}{1.272586in}}%
\pgfpathcurveto{\pgfqpoint{0.667168in}{1.272586in}}{\pgfqpoint{0.659268in}{1.269314in}}{\pgfqpoint{0.653444in}{1.263490in}}%
\pgfpathcurveto{\pgfqpoint{0.647620in}{1.257666in}}{\pgfqpoint{0.644347in}{1.249766in}}{\pgfqpoint{0.644347in}{1.241530in}}%
\pgfpathcurveto{\pgfqpoint{0.644347in}{1.233294in}}{\pgfqpoint{0.647620in}{1.225394in}}{\pgfqpoint{0.653444in}{1.219570in}}%
\pgfpathcurveto{\pgfqpoint{0.659268in}{1.213746in}}{\pgfqpoint{0.667168in}{1.210473in}}{\pgfqpoint{0.675404in}{1.210473in}}%
\pgfpathclose%
\pgfusepath{stroke,fill}%
\end{pgfscope}%
\begin{pgfscope}%
\pgfpathrectangle{\pgfqpoint{0.100000in}{0.220728in}}{\pgfqpoint{3.696000in}{3.696000in}}%
\pgfusepath{clip}%
\pgfsetbuttcap%
\pgfsetroundjoin%
\definecolor{currentfill}{rgb}{0.121569,0.466667,0.705882}%
\pgfsetfillcolor{currentfill}%
\pgfsetfillopacity{0.617300}%
\pgfsetlinewidth{1.003750pt}%
\definecolor{currentstroke}{rgb}{0.121569,0.466667,0.705882}%
\pgfsetstrokecolor{currentstroke}%
\pgfsetstrokeopacity{0.617300}%
\pgfsetdash{}{0pt}%
\pgfpathmoveto{\pgfqpoint{0.675402in}{1.210471in}}%
\pgfpathcurveto{\pgfqpoint{0.683639in}{1.210471in}}{\pgfqpoint{0.691539in}{1.213743in}}{\pgfqpoint{0.697363in}{1.219567in}}%
\pgfpathcurveto{\pgfqpoint{0.703186in}{1.225391in}}{\pgfqpoint{0.706459in}{1.233291in}}{\pgfqpoint{0.706459in}{1.241527in}}%
\pgfpathcurveto{\pgfqpoint{0.706459in}{1.249763in}}{\pgfqpoint{0.703186in}{1.257663in}}{\pgfqpoint{0.697363in}{1.263487in}}%
\pgfpathcurveto{\pgfqpoint{0.691539in}{1.269311in}}{\pgfqpoint{0.683639in}{1.272584in}}{\pgfqpoint{0.675402in}{1.272584in}}%
\pgfpathcurveto{\pgfqpoint{0.667166in}{1.272584in}}{\pgfqpoint{0.659266in}{1.269311in}}{\pgfqpoint{0.653442in}{1.263487in}}%
\pgfpathcurveto{\pgfqpoint{0.647618in}{1.257663in}}{\pgfqpoint{0.644346in}{1.249763in}}{\pgfqpoint{0.644346in}{1.241527in}}%
\pgfpathcurveto{\pgfqpoint{0.644346in}{1.233291in}}{\pgfqpoint{0.647618in}{1.225391in}}{\pgfqpoint{0.653442in}{1.219567in}}%
\pgfpathcurveto{\pgfqpoint{0.659266in}{1.213743in}}{\pgfqpoint{0.667166in}{1.210471in}}{\pgfqpoint{0.675402in}{1.210471in}}%
\pgfpathclose%
\pgfusepath{stroke,fill}%
\end{pgfscope}%
\begin{pgfscope}%
\pgfpathrectangle{\pgfqpoint{0.100000in}{0.220728in}}{\pgfqpoint{3.696000in}{3.696000in}}%
\pgfusepath{clip}%
\pgfsetbuttcap%
\pgfsetroundjoin%
\definecolor{currentfill}{rgb}{0.121569,0.466667,0.705882}%
\pgfsetfillcolor{currentfill}%
\pgfsetfillopacity{0.617301}%
\pgfsetlinewidth{1.003750pt}%
\definecolor{currentstroke}{rgb}{0.121569,0.466667,0.705882}%
\pgfsetstrokecolor{currentstroke}%
\pgfsetstrokeopacity{0.617301}%
\pgfsetdash{}{0pt}%
\pgfpathmoveto{\pgfqpoint{0.675399in}{1.210465in}}%
\pgfpathcurveto{\pgfqpoint{0.683636in}{1.210465in}}{\pgfqpoint{0.691536in}{1.213737in}}{\pgfqpoint{0.697360in}{1.219561in}}%
\pgfpathcurveto{\pgfqpoint{0.703183in}{1.225385in}}{\pgfqpoint{0.706456in}{1.233285in}}{\pgfqpoint{0.706456in}{1.241522in}}%
\pgfpathcurveto{\pgfqpoint{0.706456in}{1.249758in}}{\pgfqpoint{0.703183in}{1.257658in}}{\pgfqpoint{0.697360in}{1.263482in}}%
\pgfpathcurveto{\pgfqpoint{0.691536in}{1.269306in}}{\pgfqpoint{0.683636in}{1.272578in}}{\pgfqpoint{0.675399in}{1.272578in}}%
\pgfpathcurveto{\pgfqpoint{0.667163in}{1.272578in}}{\pgfqpoint{0.659263in}{1.269306in}}{\pgfqpoint{0.653439in}{1.263482in}}%
\pgfpathcurveto{\pgfqpoint{0.647615in}{1.257658in}}{\pgfqpoint{0.644343in}{1.249758in}}{\pgfqpoint{0.644343in}{1.241522in}}%
\pgfpathcurveto{\pgfqpoint{0.644343in}{1.233285in}}{\pgfqpoint{0.647615in}{1.225385in}}{\pgfqpoint{0.653439in}{1.219561in}}%
\pgfpathcurveto{\pgfqpoint{0.659263in}{1.213737in}}{\pgfqpoint{0.667163in}{1.210465in}}{\pgfqpoint{0.675399in}{1.210465in}}%
\pgfpathclose%
\pgfusepath{stroke,fill}%
\end{pgfscope}%
\begin{pgfscope}%
\pgfpathrectangle{\pgfqpoint{0.100000in}{0.220728in}}{\pgfqpoint{3.696000in}{3.696000in}}%
\pgfusepath{clip}%
\pgfsetbuttcap%
\pgfsetroundjoin%
\definecolor{currentfill}{rgb}{0.121569,0.466667,0.705882}%
\pgfsetfillcolor{currentfill}%
\pgfsetfillopacity{0.617303}%
\pgfsetlinewidth{1.003750pt}%
\definecolor{currentstroke}{rgb}{0.121569,0.466667,0.705882}%
\pgfsetstrokecolor{currentstroke}%
\pgfsetstrokeopacity{0.617303}%
\pgfsetdash{}{0pt}%
\pgfpathmoveto{\pgfqpoint{0.675394in}{1.210456in}}%
\pgfpathcurveto{\pgfqpoint{0.683630in}{1.210456in}}{\pgfqpoint{0.691530in}{1.213728in}}{\pgfqpoint{0.697354in}{1.219552in}}%
\pgfpathcurveto{\pgfqpoint{0.703178in}{1.225376in}}{\pgfqpoint{0.706450in}{1.233276in}}{\pgfqpoint{0.706450in}{1.241512in}}%
\pgfpathcurveto{\pgfqpoint{0.706450in}{1.249748in}}{\pgfqpoint{0.703178in}{1.257648in}}{\pgfqpoint{0.697354in}{1.263472in}}%
\pgfpathcurveto{\pgfqpoint{0.691530in}{1.269296in}}{\pgfqpoint{0.683630in}{1.272569in}}{\pgfqpoint{0.675394in}{1.272569in}}%
\pgfpathcurveto{\pgfqpoint{0.667158in}{1.272569in}}{\pgfqpoint{0.659257in}{1.269296in}}{\pgfqpoint{0.653434in}{1.263472in}}%
\pgfpathcurveto{\pgfqpoint{0.647610in}{1.257648in}}{\pgfqpoint{0.644337in}{1.249748in}}{\pgfqpoint{0.644337in}{1.241512in}}%
\pgfpathcurveto{\pgfqpoint{0.644337in}{1.233276in}}{\pgfqpoint{0.647610in}{1.225376in}}{\pgfqpoint{0.653434in}{1.219552in}}%
\pgfpathcurveto{\pgfqpoint{0.659257in}{1.213728in}}{\pgfqpoint{0.667158in}{1.210456in}}{\pgfqpoint{0.675394in}{1.210456in}}%
\pgfpathclose%
\pgfusepath{stroke,fill}%
\end{pgfscope}%
\begin{pgfscope}%
\pgfpathrectangle{\pgfqpoint{0.100000in}{0.220728in}}{\pgfqpoint{3.696000in}{3.696000in}}%
\pgfusepath{clip}%
\pgfsetbuttcap%
\pgfsetroundjoin%
\definecolor{currentfill}{rgb}{0.121569,0.466667,0.705882}%
\pgfsetfillcolor{currentfill}%
\pgfsetfillopacity{0.617307}%
\pgfsetlinewidth{1.003750pt}%
\definecolor{currentstroke}{rgb}{0.121569,0.466667,0.705882}%
\pgfsetstrokecolor{currentstroke}%
\pgfsetstrokeopacity{0.617307}%
\pgfsetdash{}{0pt}%
\pgfpathmoveto{\pgfqpoint{0.675383in}{1.210437in}}%
\pgfpathcurveto{\pgfqpoint{0.683620in}{1.210437in}}{\pgfqpoint{0.691520in}{1.213709in}}{\pgfqpoint{0.697344in}{1.219533in}}%
\pgfpathcurveto{\pgfqpoint{0.703168in}{1.225357in}}{\pgfqpoint{0.706440in}{1.233257in}}{\pgfqpoint{0.706440in}{1.241494in}}%
\pgfpathcurveto{\pgfqpoint{0.706440in}{1.249730in}}{\pgfqpoint{0.703168in}{1.257630in}}{\pgfqpoint{0.697344in}{1.263454in}}%
\pgfpathcurveto{\pgfqpoint{0.691520in}{1.269278in}}{\pgfqpoint{0.683620in}{1.272550in}}{\pgfqpoint{0.675383in}{1.272550in}}%
\pgfpathcurveto{\pgfqpoint{0.667147in}{1.272550in}}{\pgfqpoint{0.659247in}{1.269278in}}{\pgfqpoint{0.653423in}{1.263454in}}%
\pgfpathcurveto{\pgfqpoint{0.647599in}{1.257630in}}{\pgfqpoint{0.644327in}{1.249730in}}{\pgfqpoint{0.644327in}{1.241494in}}%
\pgfpathcurveto{\pgfqpoint{0.644327in}{1.233257in}}{\pgfqpoint{0.647599in}{1.225357in}}{\pgfqpoint{0.653423in}{1.219533in}}%
\pgfpathcurveto{\pgfqpoint{0.659247in}{1.213709in}}{\pgfqpoint{0.667147in}{1.210437in}}{\pgfqpoint{0.675383in}{1.210437in}}%
\pgfpathclose%
\pgfusepath{stroke,fill}%
\end{pgfscope}%
\begin{pgfscope}%
\pgfpathrectangle{\pgfqpoint{0.100000in}{0.220728in}}{\pgfqpoint{3.696000in}{3.696000in}}%
\pgfusepath{clip}%
\pgfsetbuttcap%
\pgfsetroundjoin%
\definecolor{currentfill}{rgb}{0.121569,0.466667,0.705882}%
\pgfsetfillcolor{currentfill}%
\pgfsetfillopacity{0.617313}%
\pgfsetlinewidth{1.003750pt}%
\definecolor{currentstroke}{rgb}{0.121569,0.466667,0.705882}%
\pgfsetstrokecolor{currentstroke}%
\pgfsetstrokeopacity{0.617313}%
\pgfsetdash{}{0pt}%
\pgfpathmoveto{\pgfqpoint{0.675366in}{1.210402in}}%
\pgfpathcurveto{\pgfqpoint{0.683602in}{1.210402in}}{\pgfqpoint{0.691502in}{1.213674in}}{\pgfqpoint{0.697326in}{1.219498in}}%
\pgfpathcurveto{\pgfqpoint{0.703150in}{1.225322in}}{\pgfqpoint{0.706422in}{1.233222in}}{\pgfqpoint{0.706422in}{1.241458in}}%
\pgfpathcurveto{\pgfqpoint{0.706422in}{1.249695in}}{\pgfqpoint{0.703150in}{1.257595in}}{\pgfqpoint{0.697326in}{1.263419in}}%
\pgfpathcurveto{\pgfqpoint{0.691502in}{1.269243in}}{\pgfqpoint{0.683602in}{1.272515in}}{\pgfqpoint{0.675366in}{1.272515in}}%
\pgfpathcurveto{\pgfqpoint{0.667130in}{1.272515in}}{\pgfqpoint{0.659230in}{1.269243in}}{\pgfqpoint{0.653406in}{1.263419in}}%
\pgfpathcurveto{\pgfqpoint{0.647582in}{1.257595in}}{\pgfqpoint{0.644309in}{1.249695in}}{\pgfqpoint{0.644309in}{1.241458in}}%
\pgfpathcurveto{\pgfqpoint{0.644309in}{1.233222in}}{\pgfqpoint{0.647582in}{1.225322in}}{\pgfqpoint{0.653406in}{1.219498in}}%
\pgfpathcurveto{\pgfqpoint{0.659230in}{1.213674in}}{\pgfqpoint{0.667130in}{1.210402in}}{\pgfqpoint{0.675366in}{1.210402in}}%
\pgfpathclose%
\pgfusepath{stroke,fill}%
\end{pgfscope}%
\begin{pgfscope}%
\pgfpathrectangle{\pgfqpoint{0.100000in}{0.220728in}}{\pgfqpoint{3.696000in}{3.696000in}}%
\pgfusepath{clip}%
\pgfsetbuttcap%
\pgfsetroundjoin%
\definecolor{currentfill}{rgb}{0.121569,0.466667,0.705882}%
\pgfsetfillcolor{currentfill}%
\pgfsetfillopacity{0.617324}%
\pgfsetlinewidth{1.003750pt}%
\definecolor{currentstroke}{rgb}{0.121569,0.466667,0.705882}%
\pgfsetstrokecolor{currentstroke}%
\pgfsetstrokeopacity{0.617324}%
\pgfsetdash{}{0pt}%
\pgfpathmoveto{\pgfqpoint{0.675330in}{1.210340in}}%
\pgfpathcurveto{\pgfqpoint{0.683566in}{1.210340in}}{\pgfqpoint{0.691466in}{1.213612in}}{\pgfqpoint{0.697290in}{1.219436in}}%
\pgfpathcurveto{\pgfqpoint{0.703114in}{1.225260in}}{\pgfqpoint{0.706386in}{1.233160in}}{\pgfqpoint{0.706386in}{1.241396in}}%
\pgfpathcurveto{\pgfqpoint{0.706386in}{1.249632in}}{\pgfqpoint{0.703114in}{1.257532in}}{\pgfqpoint{0.697290in}{1.263356in}}%
\pgfpathcurveto{\pgfqpoint{0.691466in}{1.269180in}}{\pgfqpoint{0.683566in}{1.272453in}}{\pgfqpoint{0.675330in}{1.272453in}}%
\pgfpathcurveto{\pgfqpoint{0.667093in}{1.272453in}}{\pgfqpoint{0.659193in}{1.269180in}}{\pgfqpoint{0.653369in}{1.263356in}}%
\pgfpathcurveto{\pgfqpoint{0.647545in}{1.257532in}}{\pgfqpoint{0.644273in}{1.249632in}}{\pgfqpoint{0.644273in}{1.241396in}}%
\pgfpathcurveto{\pgfqpoint{0.644273in}{1.233160in}}{\pgfqpoint{0.647545in}{1.225260in}}{\pgfqpoint{0.653369in}{1.219436in}}%
\pgfpathcurveto{\pgfqpoint{0.659193in}{1.213612in}}{\pgfqpoint{0.667093in}{1.210340in}}{\pgfqpoint{0.675330in}{1.210340in}}%
\pgfpathclose%
\pgfusepath{stroke,fill}%
\end{pgfscope}%
\begin{pgfscope}%
\pgfpathrectangle{\pgfqpoint{0.100000in}{0.220728in}}{\pgfqpoint{3.696000in}{3.696000in}}%
\pgfusepath{clip}%
\pgfsetbuttcap%
\pgfsetroundjoin%
\definecolor{currentfill}{rgb}{0.121569,0.466667,0.705882}%
\pgfsetfillcolor{currentfill}%
\pgfsetfillopacity{0.617343}%
\pgfsetlinewidth{1.003750pt}%
\definecolor{currentstroke}{rgb}{0.121569,0.466667,0.705882}%
\pgfsetstrokecolor{currentstroke}%
\pgfsetstrokeopacity{0.617343}%
\pgfsetdash{}{0pt}%
\pgfpathmoveto{\pgfqpoint{0.675279in}{1.210209in}}%
\pgfpathcurveto{\pgfqpoint{0.683516in}{1.210209in}}{\pgfqpoint{0.691416in}{1.213481in}}{\pgfqpoint{0.697240in}{1.219305in}}%
\pgfpathcurveto{\pgfqpoint{0.703064in}{1.225129in}}{\pgfqpoint{0.706336in}{1.233029in}}{\pgfqpoint{0.706336in}{1.241265in}}%
\pgfpathcurveto{\pgfqpoint{0.706336in}{1.249502in}}{\pgfqpoint{0.703064in}{1.257402in}}{\pgfqpoint{0.697240in}{1.263226in}}%
\pgfpathcurveto{\pgfqpoint{0.691416in}{1.269050in}}{\pgfqpoint{0.683516in}{1.272322in}}{\pgfqpoint{0.675279in}{1.272322in}}%
\pgfpathcurveto{\pgfqpoint{0.667043in}{1.272322in}}{\pgfqpoint{0.659143in}{1.269050in}}{\pgfqpoint{0.653319in}{1.263226in}}%
\pgfpathcurveto{\pgfqpoint{0.647495in}{1.257402in}}{\pgfqpoint{0.644223in}{1.249502in}}{\pgfqpoint{0.644223in}{1.241265in}}%
\pgfpathcurveto{\pgfqpoint{0.644223in}{1.233029in}}{\pgfqpoint{0.647495in}{1.225129in}}{\pgfqpoint{0.653319in}{1.219305in}}%
\pgfpathcurveto{\pgfqpoint{0.659143in}{1.213481in}}{\pgfqpoint{0.667043in}{1.210209in}}{\pgfqpoint{0.675279in}{1.210209in}}%
\pgfpathclose%
\pgfusepath{stroke,fill}%
\end{pgfscope}%
\begin{pgfscope}%
\pgfpathrectangle{\pgfqpoint{0.100000in}{0.220728in}}{\pgfqpoint{3.696000in}{3.696000in}}%
\pgfusepath{clip}%
\pgfsetbuttcap%
\pgfsetroundjoin%
\definecolor{currentfill}{rgb}{0.121569,0.466667,0.705882}%
\pgfsetfillcolor{currentfill}%
\pgfsetfillopacity{0.617379}%
\pgfsetlinewidth{1.003750pt}%
\definecolor{currentstroke}{rgb}{0.121569,0.466667,0.705882}%
\pgfsetstrokecolor{currentstroke}%
\pgfsetstrokeopacity{0.617379}%
\pgfsetdash{}{0pt}%
\pgfpathmoveto{\pgfqpoint{0.675194in}{1.209977in}}%
\pgfpathcurveto{\pgfqpoint{0.683430in}{1.209977in}}{\pgfqpoint{0.691330in}{1.213249in}}{\pgfqpoint{0.697154in}{1.219073in}}%
\pgfpathcurveto{\pgfqpoint{0.702978in}{1.224897in}}{\pgfqpoint{0.706250in}{1.232797in}}{\pgfqpoint{0.706250in}{1.241033in}}%
\pgfpathcurveto{\pgfqpoint{0.706250in}{1.249270in}}{\pgfqpoint{0.702978in}{1.257170in}}{\pgfqpoint{0.697154in}{1.262994in}}%
\pgfpathcurveto{\pgfqpoint{0.691330in}{1.268817in}}{\pgfqpoint{0.683430in}{1.272090in}}{\pgfqpoint{0.675194in}{1.272090in}}%
\pgfpathcurveto{\pgfqpoint{0.666958in}{1.272090in}}{\pgfqpoint{0.659057in}{1.268817in}}{\pgfqpoint{0.653234in}{1.262994in}}%
\pgfpathcurveto{\pgfqpoint{0.647410in}{1.257170in}}{\pgfqpoint{0.644137in}{1.249270in}}{\pgfqpoint{0.644137in}{1.241033in}}%
\pgfpathcurveto{\pgfqpoint{0.644137in}{1.232797in}}{\pgfqpoint{0.647410in}{1.224897in}}{\pgfqpoint{0.653234in}{1.219073in}}%
\pgfpathcurveto{\pgfqpoint{0.659057in}{1.213249in}}{\pgfqpoint{0.666958in}{1.209977in}}{\pgfqpoint{0.675194in}{1.209977in}}%
\pgfpathclose%
\pgfusepath{stroke,fill}%
\end{pgfscope}%
\begin{pgfscope}%
\pgfpathrectangle{\pgfqpoint{0.100000in}{0.220728in}}{\pgfqpoint{3.696000in}{3.696000in}}%
\pgfusepath{clip}%
\pgfsetbuttcap%
\pgfsetroundjoin%
\definecolor{currentfill}{rgb}{0.121569,0.466667,0.705882}%
\pgfsetfillcolor{currentfill}%
\pgfsetfillopacity{0.617446}%
\pgfsetlinewidth{1.003750pt}%
\definecolor{currentstroke}{rgb}{0.121569,0.466667,0.705882}%
\pgfsetstrokecolor{currentstroke}%
\pgfsetstrokeopacity{0.617446}%
\pgfsetdash{}{0pt}%
\pgfpathmoveto{\pgfqpoint{0.675026in}{1.209560in}}%
\pgfpathcurveto{\pgfqpoint{0.683262in}{1.209560in}}{\pgfqpoint{0.691162in}{1.212833in}}{\pgfqpoint{0.696986in}{1.218657in}}%
\pgfpathcurveto{\pgfqpoint{0.702810in}{1.224480in}}{\pgfqpoint{0.706082in}{1.232381in}}{\pgfqpoint{0.706082in}{1.240617in}}%
\pgfpathcurveto{\pgfqpoint{0.706082in}{1.248853in}}{\pgfqpoint{0.702810in}{1.256753in}}{\pgfqpoint{0.696986in}{1.262577in}}%
\pgfpathcurveto{\pgfqpoint{0.691162in}{1.268401in}}{\pgfqpoint{0.683262in}{1.271673in}}{\pgfqpoint{0.675026in}{1.271673in}}%
\pgfpathcurveto{\pgfqpoint{0.666789in}{1.271673in}}{\pgfqpoint{0.658889in}{1.268401in}}{\pgfqpoint{0.653065in}{1.262577in}}%
\pgfpathcurveto{\pgfqpoint{0.647241in}{1.256753in}}{\pgfqpoint{0.643969in}{1.248853in}}{\pgfqpoint{0.643969in}{1.240617in}}%
\pgfpathcurveto{\pgfqpoint{0.643969in}{1.232381in}}{\pgfqpoint{0.647241in}{1.224480in}}{\pgfqpoint{0.653065in}{1.218657in}}%
\pgfpathcurveto{\pgfqpoint{0.658889in}{1.212833in}}{\pgfqpoint{0.666789in}{1.209560in}}{\pgfqpoint{0.675026in}{1.209560in}}%
\pgfpathclose%
\pgfusepath{stroke,fill}%
\end{pgfscope}%
\begin{pgfscope}%
\pgfpathrectangle{\pgfqpoint{0.100000in}{0.220728in}}{\pgfqpoint{3.696000in}{3.696000in}}%
\pgfusepath{clip}%
\pgfsetbuttcap%
\pgfsetroundjoin%
\definecolor{currentfill}{rgb}{0.121569,0.466667,0.705882}%
\pgfsetfillcolor{currentfill}%
\pgfsetfillopacity{0.617587}%
\pgfsetlinewidth{1.003750pt}%
\definecolor{currentstroke}{rgb}{0.121569,0.466667,0.705882}%
\pgfsetstrokecolor{currentstroke}%
\pgfsetstrokeopacity{0.617587}%
\pgfsetdash{}{0pt}%
\pgfpathmoveto{\pgfqpoint{0.674784in}{1.208859in}}%
\pgfpathcurveto{\pgfqpoint{0.683020in}{1.208859in}}{\pgfqpoint{0.690920in}{1.212132in}}{\pgfqpoint{0.696744in}{1.217956in}}%
\pgfpathcurveto{\pgfqpoint{0.702568in}{1.223780in}}{\pgfqpoint{0.705841in}{1.231680in}}{\pgfqpoint{0.705841in}{1.239916in}}%
\pgfpathcurveto{\pgfqpoint{0.705841in}{1.248152in}}{\pgfqpoint{0.702568in}{1.256052in}}{\pgfqpoint{0.696744in}{1.261876in}}%
\pgfpathcurveto{\pgfqpoint{0.690920in}{1.267700in}}{\pgfqpoint{0.683020in}{1.270972in}}{\pgfqpoint{0.674784in}{1.270972in}}%
\pgfpathcurveto{\pgfqpoint{0.666548in}{1.270972in}}{\pgfqpoint{0.658648in}{1.267700in}}{\pgfqpoint{0.652824in}{1.261876in}}%
\pgfpathcurveto{\pgfqpoint{0.647000in}{1.256052in}}{\pgfqpoint{0.643728in}{1.248152in}}{\pgfqpoint{0.643728in}{1.239916in}}%
\pgfpathcurveto{\pgfqpoint{0.643728in}{1.231680in}}{\pgfqpoint{0.647000in}{1.223780in}}{\pgfqpoint{0.652824in}{1.217956in}}%
\pgfpathcurveto{\pgfqpoint{0.658648in}{1.212132in}}{\pgfqpoint{0.666548in}{1.208859in}}{\pgfqpoint{0.674784in}{1.208859in}}%
\pgfpathclose%
\pgfusepath{stroke,fill}%
\end{pgfscope}%
\begin{pgfscope}%
\pgfpathrectangle{\pgfqpoint{0.100000in}{0.220728in}}{\pgfqpoint{3.696000in}{3.696000in}}%
\pgfusepath{clip}%
\pgfsetbuttcap%
\pgfsetroundjoin%
\definecolor{currentfill}{rgb}{0.121569,0.466667,0.705882}%
\pgfsetfillcolor{currentfill}%
\pgfsetfillopacity{0.617849}%
\pgfsetlinewidth{1.003750pt}%
\definecolor{currentstroke}{rgb}{0.121569,0.466667,0.705882}%
\pgfsetstrokecolor{currentstroke}%
\pgfsetstrokeopacity{0.617849}%
\pgfsetdash{}{0pt}%
\pgfpathmoveto{\pgfqpoint{0.674191in}{1.207721in}}%
\pgfpathcurveto{\pgfqpoint{0.682427in}{1.207721in}}{\pgfqpoint{0.690327in}{1.210993in}}{\pgfqpoint{0.696151in}{1.216817in}}%
\pgfpathcurveto{\pgfqpoint{0.701975in}{1.222641in}}{\pgfqpoint{0.705248in}{1.230541in}}{\pgfqpoint{0.705248in}{1.238777in}}%
\pgfpathcurveto{\pgfqpoint{0.705248in}{1.247013in}}{\pgfqpoint{0.701975in}{1.254914in}}{\pgfqpoint{0.696151in}{1.260737in}}%
\pgfpathcurveto{\pgfqpoint{0.690327in}{1.266561in}}{\pgfqpoint{0.682427in}{1.269834in}}{\pgfqpoint{0.674191in}{1.269834in}}%
\pgfpathcurveto{\pgfqpoint{0.665955in}{1.269834in}}{\pgfqpoint{0.658055in}{1.266561in}}{\pgfqpoint{0.652231in}{1.260737in}}%
\pgfpathcurveto{\pgfqpoint{0.646407in}{1.254914in}}{\pgfqpoint{0.643135in}{1.247013in}}{\pgfqpoint{0.643135in}{1.238777in}}%
\pgfpathcurveto{\pgfqpoint{0.643135in}{1.230541in}}{\pgfqpoint{0.646407in}{1.222641in}}{\pgfqpoint{0.652231in}{1.216817in}}%
\pgfpathcurveto{\pgfqpoint{0.658055in}{1.210993in}}{\pgfqpoint{0.665955in}{1.207721in}}{\pgfqpoint{0.674191in}{1.207721in}}%
\pgfpathclose%
\pgfusepath{stroke,fill}%
\end{pgfscope}%
\begin{pgfscope}%
\pgfpathrectangle{\pgfqpoint{0.100000in}{0.220728in}}{\pgfqpoint{3.696000in}{3.696000in}}%
\pgfusepath{clip}%
\pgfsetbuttcap%
\pgfsetroundjoin%
\definecolor{currentfill}{rgb}{0.121569,0.466667,0.705882}%
\pgfsetfillcolor{currentfill}%
\pgfsetfillopacity{0.617997}%
\pgfsetlinewidth{1.003750pt}%
\definecolor{currentstroke}{rgb}{0.121569,0.466667,0.705882}%
\pgfsetstrokecolor{currentstroke}%
\pgfsetstrokeopacity{0.617997}%
\pgfsetdash{}{0pt}%
\pgfpathmoveto{\pgfqpoint{0.689870in}{1.211340in}}%
\pgfpathcurveto{\pgfqpoint{0.698106in}{1.211340in}}{\pgfqpoint{0.706006in}{1.214612in}}{\pgfqpoint{0.711830in}{1.220436in}}%
\pgfpathcurveto{\pgfqpoint{0.717654in}{1.226260in}}{\pgfqpoint{0.720926in}{1.234160in}}{\pgfqpoint{0.720926in}{1.242396in}}%
\pgfpathcurveto{\pgfqpoint{0.720926in}{1.250633in}}{\pgfqpoint{0.717654in}{1.258533in}}{\pgfqpoint{0.711830in}{1.264357in}}%
\pgfpathcurveto{\pgfqpoint{0.706006in}{1.270181in}}{\pgfqpoint{0.698106in}{1.273453in}}{\pgfqpoint{0.689870in}{1.273453in}}%
\pgfpathcurveto{\pgfqpoint{0.681633in}{1.273453in}}{\pgfqpoint{0.673733in}{1.270181in}}{\pgfqpoint{0.667909in}{1.264357in}}%
\pgfpathcurveto{\pgfqpoint{0.662085in}{1.258533in}}{\pgfqpoint{0.658813in}{1.250633in}}{\pgfqpoint{0.658813in}{1.242396in}}%
\pgfpathcurveto{\pgfqpoint{0.658813in}{1.234160in}}{\pgfqpoint{0.662085in}{1.226260in}}{\pgfqpoint{0.667909in}{1.220436in}}%
\pgfpathcurveto{\pgfqpoint{0.673733in}{1.214612in}}{\pgfqpoint{0.681633in}{1.211340in}}{\pgfqpoint{0.689870in}{1.211340in}}%
\pgfpathclose%
\pgfusepath{stroke,fill}%
\end{pgfscope}%
\begin{pgfscope}%
\pgfpathrectangle{\pgfqpoint{0.100000in}{0.220728in}}{\pgfqpoint{3.696000in}{3.696000in}}%
\pgfusepath{clip}%
\pgfsetbuttcap%
\pgfsetroundjoin%
\definecolor{currentfill}{rgb}{0.121569,0.466667,0.705882}%
\pgfsetfillcolor{currentfill}%
\pgfsetfillopacity{0.618322}%
\pgfsetlinewidth{1.003750pt}%
\definecolor{currentstroke}{rgb}{0.121569,0.466667,0.705882}%
\pgfsetstrokecolor{currentstroke}%
\pgfsetstrokeopacity{0.618322}%
\pgfsetdash{}{0pt}%
\pgfpathmoveto{\pgfqpoint{0.673127in}{1.205612in}}%
\pgfpathcurveto{\pgfqpoint{0.681363in}{1.205612in}}{\pgfqpoint{0.689263in}{1.208885in}}{\pgfqpoint{0.695087in}{1.214708in}}%
\pgfpathcurveto{\pgfqpoint{0.700911in}{1.220532in}}{\pgfqpoint{0.704183in}{1.228432in}}{\pgfqpoint{0.704183in}{1.236669in}}%
\pgfpathcurveto{\pgfqpoint{0.704183in}{1.244905in}}{\pgfqpoint{0.700911in}{1.252805in}}{\pgfqpoint{0.695087in}{1.258629in}}%
\pgfpathcurveto{\pgfqpoint{0.689263in}{1.264453in}}{\pgfqpoint{0.681363in}{1.267725in}}{\pgfqpoint{0.673127in}{1.267725in}}%
\pgfpathcurveto{\pgfqpoint{0.664891in}{1.267725in}}{\pgfqpoint{0.656991in}{1.264453in}}{\pgfqpoint{0.651167in}{1.258629in}}%
\pgfpathcurveto{\pgfqpoint{0.645343in}{1.252805in}}{\pgfqpoint{0.642070in}{1.244905in}}{\pgfqpoint{0.642070in}{1.236669in}}%
\pgfpathcurveto{\pgfqpoint{0.642070in}{1.228432in}}{\pgfqpoint{0.645343in}{1.220532in}}{\pgfqpoint{0.651167in}{1.214708in}}%
\pgfpathcurveto{\pgfqpoint{0.656991in}{1.208885in}}{\pgfqpoint{0.664891in}{1.205612in}}{\pgfqpoint{0.673127in}{1.205612in}}%
\pgfpathclose%
\pgfusepath{stroke,fill}%
\end{pgfscope}%
\begin{pgfscope}%
\pgfpathrectangle{\pgfqpoint{0.100000in}{0.220728in}}{\pgfqpoint{3.696000in}{3.696000in}}%
\pgfusepath{clip}%
\pgfsetbuttcap%
\pgfsetroundjoin%
\definecolor{currentfill}{rgb}{0.121569,0.466667,0.705882}%
\pgfsetfillcolor{currentfill}%
\pgfsetfillopacity{0.618322}%
\pgfsetlinewidth{1.003750pt}%
\definecolor{currentstroke}{rgb}{0.121569,0.466667,0.705882}%
\pgfsetstrokecolor{currentstroke}%
\pgfsetstrokeopacity{0.618322}%
\pgfsetdash{}{0pt}%
\pgfpathmoveto{\pgfqpoint{0.673126in}{1.205611in}}%
\pgfpathcurveto{\pgfqpoint{0.681363in}{1.205611in}}{\pgfqpoint{0.689263in}{1.208883in}}{\pgfqpoint{0.695087in}{1.214707in}}%
\pgfpathcurveto{\pgfqpoint{0.700911in}{1.220531in}}{\pgfqpoint{0.704183in}{1.228431in}}{\pgfqpoint{0.704183in}{1.236668in}}%
\pgfpathcurveto{\pgfqpoint{0.704183in}{1.244904in}}{\pgfqpoint{0.700911in}{1.252804in}}{\pgfqpoint{0.695087in}{1.258628in}}%
\pgfpathcurveto{\pgfqpoint{0.689263in}{1.264452in}}{\pgfqpoint{0.681363in}{1.267724in}}{\pgfqpoint{0.673126in}{1.267724in}}%
\pgfpathcurveto{\pgfqpoint{0.664890in}{1.267724in}}{\pgfqpoint{0.656990in}{1.264452in}}{\pgfqpoint{0.651166in}{1.258628in}}%
\pgfpathcurveto{\pgfqpoint{0.645342in}{1.252804in}}{\pgfqpoint{0.642070in}{1.244904in}}{\pgfqpoint{0.642070in}{1.236668in}}%
\pgfpathcurveto{\pgfqpoint{0.642070in}{1.228431in}}{\pgfqpoint{0.645342in}{1.220531in}}{\pgfqpoint{0.651166in}{1.214707in}}%
\pgfpathcurveto{\pgfqpoint{0.656990in}{1.208883in}}{\pgfqpoint{0.664890in}{1.205611in}}{\pgfqpoint{0.673126in}{1.205611in}}%
\pgfpathclose%
\pgfusepath{stroke,fill}%
\end{pgfscope}%
\begin{pgfscope}%
\pgfpathrectangle{\pgfqpoint{0.100000in}{0.220728in}}{\pgfqpoint{3.696000in}{3.696000in}}%
\pgfusepath{clip}%
\pgfsetbuttcap%
\pgfsetroundjoin%
\definecolor{currentfill}{rgb}{0.121569,0.466667,0.705882}%
\pgfsetfillcolor{currentfill}%
\pgfsetfillopacity{0.618322}%
\pgfsetlinewidth{1.003750pt}%
\definecolor{currentstroke}{rgb}{0.121569,0.466667,0.705882}%
\pgfsetstrokecolor{currentstroke}%
\pgfsetstrokeopacity{0.618322}%
\pgfsetdash{}{0pt}%
\pgfpathmoveto{\pgfqpoint{0.673125in}{1.205609in}}%
\pgfpathcurveto{\pgfqpoint{0.681362in}{1.205609in}}{\pgfqpoint{0.689262in}{1.208881in}}{\pgfqpoint{0.695086in}{1.214705in}}%
\pgfpathcurveto{\pgfqpoint{0.700910in}{1.220529in}}{\pgfqpoint{0.704182in}{1.228429in}}{\pgfqpoint{0.704182in}{1.236666in}}%
\pgfpathcurveto{\pgfqpoint{0.704182in}{1.244902in}}{\pgfqpoint{0.700910in}{1.252802in}}{\pgfqpoint{0.695086in}{1.258626in}}%
\pgfpathcurveto{\pgfqpoint{0.689262in}{1.264450in}}{\pgfqpoint{0.681362in}{1.267722in}}{\pgfqpoint{0.673125in}{1.267722in}}%
\pgfpathcurveto{\pgfqpoint{0.664889in}{1.267722in}}{\pgfqpoint{0.656989in}{1.264450in}}{\pgfqpoint{0.651165in}{1.258626in}}%
\pgfpathcurveto{\pgfqpoint{0.645341in}{1.252802in}}{\pgfqpoint{0.642069in}{1.244902in}}{\pgfqpoint{0.642069in}{1.236666in}}%
\pgfpathcurveto{\pgfqpoint{0.642069in}{1.228429in}}{\pgfqpoint{0.645341in}{1.220529in}}{\pgfqpoint{0.651165in}{1.214705in}}%
\pgfpathcurveto{\pgfqpoint{0.656989in}{1.208881in}}{\pgfqpoint{0.664889in}{1.205609in}}{\pgfqpoint{0.673125in}{1.205609in}}%
\pgfpathclose%
\pgfusepath{stroke,fill}%
\end{pgfscope}%
\begin{pgfscope}%
\pgfpathrectangle{\pgfqpoint{0.100000in}{0.220728in}}{\pgfqpoint{3.696000in}{3.696000in}}%
\pgfusepath{clip}%
\pgfsetbuttcap%
\pgfsetroundjoin%
\definecolor{currentfill}{rgb}{0.121569,0.466667,0.705882}%
\pgfsetfillcolor{currentfill}%
\pgfsetfillopacity{0.618323}%
\pgfsetlinewidth{1.003750pt}%
\definecolor{currentstroke}{rgb}{0.121569,0.466667,0.705882}%
\pgfsetstrokecolor{currentstroke}%
\pgfsetstrokeopacity{0.618323}%
\pgfsetdash{}{0pt}%
\pgfpathmoveto{\pgfqpoint{0.673124in}{1.205605in}}%
\pgfpathcurveto{\pgfqpoint{0.681360in}{1.205605in}}{\pgfqpoint{0.689260in}{1.208878in}}{\pgfqpoint{0.695084in}{1.214702in}}%
\pgfpathcurveto{\pgfqpoint{0.700908in}{1.220525in}}{\pgfqpoint{0.704180in}{1.228426in}}{\pgfqpoint{0.704180in}{1.236662in}}%
\pgfpathcurveto{\pgfqpoint{0.704180in}{1.244898in}}{\pgfqpoint{0.700908in}{1.252798in}}{\pgfqpoint{0.695084in}{1.258622in}}%
\pgfpathcurveto{\pgfqpoint{0.689260in}{1.264446in}}{\pgfqpoint{0.681360in}{1.267718in}}{\pgfqpoint{0.673124in}{1.267718in}}%
\pgfpathcurveto{\pgfqpoint{0.664887in}{1.267718in}}{\pgfqpoint{0.656987in}{1.264446in}}{\pgfqpoint{0.651163in}{1.258622in}}%
\pgfpathcurveto{\pgfqpoint{0.645339in}{1.252798in}}{\pgfqpoint{0.642067in}{1.244898in}}{\pgfqpoint{0.642067in}{1.236662in}}%
\pgfpathcurveto{\pgfqpoint{0.642067in}{1.228426in}}{\pgfqpoint{0.645339in}{1.220525in}}{\pgfqpoint{0.651163in}{1.214702in}}%
\pgfpathcurveto{\pgfqpoint{0.656987in}{1.208878in}}{\pgfqpoint{0.664887in}{1.205605in}}{\pgfqpoint{0.673124in}{1.205605in}}%
\pgfpathclose%
\pgfusepath{stroke,fill}%
\end{pgfscope}%
\begin{pgfscope}%
\pgfpathrectangle{\pgfqpoint{0.100000in}{0.220728in}}{\pgfqpoint{3.696000in}{3.696000in}}%
\pgfusepath{clip}%
\pgfsetbuttcap%
\pgfsetroundjoin%
\definecolor{currentfill}{rgb}{0.121569,0.466667,0.705882}%
\pgfsetfillcolor{currentfill}%
\pgfsetfillopacity{0.618324}%
\pgfsetlinewidth{1.003750pt}%
\definecolor{currentstroke}{rgb}{0.121569,0.466667,0.705882}%
\pgfsetstrokecolor{currentstroke}%
\pgfsetstrokeopacity{0.618324}%
\pgfsetdash{}{0pt}%
\pgfpathmoveto{\pgfqpoint{0.673120in}{1.205599in}}%
\pgfpathcurveto{\pgfqpoint{0.681356in}{1.205599in}}{\pgfqpoint{0.689256in}{1.208871in}}{\pgfqpoint{0.695080in}{1.214695in}}%
\pgfpathcurveto{\pgfqpoint{0.700904in}{1.220519in}}{\pgfqpoint{0.704176in}{1.228419in}}{\pgfqpoint{0.704176in}{1.236655in}}%
\pgfpathcurveto{\pgfqpoint{0.704176in}{1.244892in}}{\pgfqpoint{0.700904in}{1.252792in}}{\pgfqpoint{0.695080in}{1.258616in}}%
\pgfpathcurveto{\pgfqpoint{0.689256in}{1.264440in}}{\pgfqpoint{0.681356in}{1.267712in}}{\pgfqpoint{0.673120in}{1.267712in}}%
\pgfpathcurveto{\pgfqpoint{0.664884in}{1.267712in}}{\pgfqpoint{0.656984in}{1.264440in}}{\pgfqpoint{0.651160in}{1.258616in}}%
\pgfpathcurveto{\pgfqpoint{0.645336in}{1.252792in}}{\pgfqpoint{0.642063in}{1.244892in}}{\pgfqpoint{0.642063in}{1.236655in}}%
\pgfpathcurveto{\pgfqpoint{0.642063in}{1.228419in}}{\pgfqpoint{0.645336in}{1.220519in}}{\pgfqpoint{0.651160in}{1.214695in}}%
\pgfpathcurveto{\pgfqpoint{0.656984in}{1.208871in}}{\pgfqpoint{0.664884in}{1.205599in}}{\pgfqpoint{0.673120in}{1.205599in}}%
\pgfpathclose%
\pgfusepath{stroke,fill}%
\end{pgfscope}%
\begin{pgfscope}%
\pgfpathrectangle{\pgfqpoint{0.100000in}{0.220728in}}{\pgfqpoint{3.696000in}{3.696000in}}%
\pgfusepath{clip}%
\pgfsetbuttcap%
\pgfsetroundjoin%
\definecolor{currentfill}{rgb}{0.121569,0.466667,0.705882}%
\pgfsetfillcolor{currentfill}%
\pgfsetfillopacity{0.618326}%
\pgfsetlinewidth{1.003750pt}%
\definecolor{currentstroke}{rgb}{0.121569,0.466667,0.705882}%
\pgfsetstrokecolor{currentstroke}%
\pgfsetstrokeopacity{0.618326}%
\pgfsetdash{}{0pt}%
\pgfpathmoveto{\pgfqpoint{0.673113in}{1.205586in}}%
\pgfpathcurveto{\pgfqpoint{0.681349in}{1.205586in}}{\pgfqpoint{0.689250in}{1.208858in}}{\pgfqpoint{0.695073in}{1.214682in}}%
\pgfpathcurveto{\pgfqpoint{0.700897in}{1.220506in}}{\pgfqpoint{0.704170in}{1.228406in}}{\pgfqpoint{0.704170in}{1.236643in}}%
\pgfpathcurveto{\pgfqpoint{0.704170in}{1.244879in}}{\pgfqpoint{0.700897in}{1.252779in}}{\pgfqpoint{0.695073in}{1.258603in}}%
\pgfpathcurveto{\pgfqpoint{0.689250in}{1.264427in}}{\pgfqpoint{0.681349in}{1.267699in}}{\pgfqpoint{0.673113in}{1.267699in}}%
\pgfpathcurveto{\pgfqpoint{0.664877in}{1.267699in}}{\pgfqpoint{0.656977in}{1.264427in}}{\pgfqpoint{0.651153in}{1.258603in}}%
\pgfpathcurveto{\pgfqpoint{0.645329in}{1.252779in}}{\pgfqpoint{0.642057in}{1.244879in}}{\pgfqpoint{0.642057in}{1.236643in}}%
\pgfpathcurveto{\pgfqpoint{0.642057in}{1.228406in}}{\pgfqpoint{0.645329in}{1.220506in}}{\pgfqpoint{0.651153in}{1.214682in}}%
\pgfpathcurveto{\pgfqpoint{0.656977in}{1.208858in}}{\pgfqpoint{0.664877in}{1.205586in}}{\pgfqpoint{0.673113in}{1.205586in}}%
\pgfpathclose%
\pgfusepath{stroke,fill}%
\end{pgfscope}%
\begin{pgfscope}%
\pgfpathrectangle{\pgfqpoint{0.100000in}{0.220728in}}{\pgfqpoint{3.696000in}{3.696000in}}%
\pgfusepath{clip}%
\pgfsetbuttcap%
\pgfsetroundjoin%
\definecolor{currentfill}{rgb}{0.121569,0.466667,0.705882}%
\pgfsetfillcolor{currentfill}%
\pgfsetfillopacity{0.618329}%
\pgfsetlinewidth{1.003750pt}%
\definecolor{currentstroke}{rgb}{0.121569,0.466667,0.705882}%
\pgfsetstrokecolor{currentstroke}%
\pgfsetstrokeopacity{0.618329}%
\pgfsetdash{}{0pt}%
\pgfpathmoveto{\pgfqpoint{0.673100in}{1.205562in}}%
\pgfpathcurveto{\pgfqpoint{0.681337in}{1.205562in}}{\pgfqpoint{0.689237in}{1.208835in}}{\pgfqpoint{0.695061in}{1.214659in}}%
\pgfpathcurveto{\pgfqpoint{0.700884in}{1.220483in}}{\pgfqpoint{0.704157in}{1.228383in}}{\pgfqpoint{0.704157in}{1.236619in}}%
\pgfpathcurveto{\pgfqpoint{0.704157in}{1.244855in}}{\pgfqpoint{0.700884in}{1.252755in}}{\pgfqpoint{0.695061in}{1.258579in}}%
\pgfpathcurveto{\pgfqpoint{0.689237in}{1.264403in}}{\pgfqpoint{0.681337in}{1.267675in}}{\pgfqpoint{0.673100in}{1.267675in}}%
\pgfpathcurveto{\pgfqpoint{0.664864in}{1.267675in}}{\pgfqpoint{0.656964in}{1.264403in}}{\pgfqpoint{0.651140in}{1.258579in}}%
\pgfpathcurveto{\pgfqpoint{0.645316in}{1.252755in}}{\pgfqpoint{0.642044in}{1.244855in}}{\pgfqpoint{0.642044in}{1.236619in}}%
\pgfpathcurveto{\pgfqpoint{0.642044in}{1.228383in}}{\pgfqpoint{0.645316in}{1.220483in}}{\pgfqpoint{0.651140in}{1.214659in}}%
\pgfpathcurveto{\pgfqpoint{0.656964in}{1.208835in}}{\pgfqpoint{0.664864in}{1.205562in}}{\pgfqpoint{0.673100in}{1.205562in}}%
\pgfpathclose%
\pgfusepath{stroke,fill}%
\end{pgfscope}%
\begin{pgfscope}%
\pgfpathrectangle{\pgfqpoint{0.100000in}{0.220728in}}{\pgfqpoint{3.696000in}{3.696000in}}%
\pgfusepath{clip}%
\pgfsetbuttcap%
\pgfsetroundjoin%
\definecolor{currentfill}{rgb}{0.121569,0.466667,0.705882}%
\pgfsetfillcolor{currentfill}%
\pgfsetfillopacity{0.618336}%
\pgfsetlinewidth{1.003750pt}%
\definecolor{currentstroke}{rgb}{0.121569,0.466667,0.705882}%
\pgfsetstrokecolor{currentstroke}%
\pgfsetstrokeopacity{0.618336}%
\pgfsetdash{}{0pt}%
\pgfpathmoveto{\pgfqpoint{0.673080in}{1.205519in}}%
\pgfpathcurveto{\pgfqpoint{0.681316in}{1.205519in}}{\pgfqpoint{0.689216in}{1.208791in}}{\pgfqpoint{0.695040in}{1.214615in}}%
\pgfpathcurveto{\pgfqpoint{0.700864in}{1.220439in}}{\pgfqpoint{0.704136in}{1.228339in}}{\pgfqpoint{0.704136in}{1.236576in}}%
\pgfpathcurveto{\pgfqpoint{0.704136in}{1.244812in}}{\pgfqpoint{0.700864in}{1.252712in}}{\pgfqpoint{0.695040in}{1.258536in}}%
\pgfpathcurveto{\pgfqpoint{0.689216in}{1.264360in}}{\pgfqpoint{0.681316in}{1.267632in}}{\pgfqpoint{0.673080in}{1.267632in}}%
\pgfpathcurveto{\pgfqpoint{0.664843in}{1.267632in}}{\pgfqpoint{0.656943in}{1.264360in}}{\pgfqpoint{0.651119in}{1.258536in}}%
\pgfpathcurveto{\pgfqpoint{0.645296in}{1.252712in}}{\pgfqpoint{0.642023in}{1.244812in}}{\pgfqpoint{0.642023in}{1.236576in}}%
\pgfpathcurveto{\pgfqpoint{0.642023in}{1.228339in}}{\pgfqpoint{0.645296in}{1.220439in}}{\pgfqpoint{0.651119in}{1.214615in}}%
\pgfpathcurveto{\pgfqpoint{0.656943in}{1.208791in}}{\pgfqpoint{0.664843in}{1.205519in}}{\pgfqpoint{0.673080in}{1.205519in}}%
\pgfpathclose%
\pgfusepath{stroke,fill}%
\end{pgfscope}%
\begin{pgfscope}%
\pgfpathrectangle{\pgfqpoint{0.100000in}{0.220728in}}{\pgfqpoint{3.696000in}{3.696000in}}%
\pgfusepath{clip}%
\pgfsetbuttcap%
\pgfsetroundjoin%
\definecolor{currentfill}{rgb}{0.121569,0.466667,0.705882}%
\pgfsetfillcolor{currentfill}%
\pgfsetfillopacity{0.618346}%
\pgfsetlinewidth{1.003750pt}%
\definecolor{currentstroke}{rgb}{0.121569,0.466667,0.705882}%
\pgfsetstrokecolor{currentstroke}%
\pgfsetstrokeopacity{0.618346}%
\pgfsetdash{}{0pt}%
\pgfpathmoveto{\pgfqpoint{0.673040in}{1.205437in}}%
\pgfpathcurveto{\pgfqpoint{0.681276in}{1.205437in}}{\pgfqpoint{0.689176in}{1.208709in}}{\pgfqpoint{0.695000in}{1.214533in}}%
\pgfpathcurveto{\pgfqpoint{0.700824in}{1.220357in}}{\pgfqpoint{0.704096in}{1.228257in}}{\pgfqpoint{0.704096in}{1.236494in}}%
\pgfpathcurveto{\pgfqpoint{0.704096in}{1.244730in}}{\pgfqpoint{0.700824in}{1.252630in}}{\pgfqpoint{0.695000in}{1.258454in}}%
\pgfpathcurveto{\pgfqpoint{0.689176in}{1.264278in}}{\pgfqpoint{0.681276in}{1.267550in}}{\pgfqpoint{0.673040in}{1.267550in}}%
\pgfpathcurveto{\pgfqpoint{0.664803in}{1.267550in}}{\pgfqpoint{0.656903in}{1.264278in}}{\pgfqpoint{0.651079in}{1.258454in}}%
\pgfpathcurveto{\pgfqpoint{0.645256in}{1.252630in}}{\pgfqpoint{0.641983in}{1.244730in}}{\pgfqpoint{0.641983in}{1.236494in}}%
\pgfpathcurveto{\pgfqpoint{0.641983in}{1.228257in}}{\pgfqpoint{0.645256in}{1.220357in}}{\pgfqpoint{0.651079in}{1.214533in}}%
\pgfpathcurveto{\pgfqpoint{0.656903in}{1.208709in}}{\pgfqpoint{0.664803in}{1.205437in}}{\pgfqpoint{0.673040in}{1.205437in}}%
\pgfpathclose%
\pgfusepath{stroke,fill}%
\end{pgfscope}%
\begin{pgfscope}%
\pgfpathrectangle{\pgfqpoint{0.100000in}{0.220728in}}{\pgfqpoint{3.696000in}{3.696000in}}%
\pgfusepath{clip}%
\pgfsetbuttcap%
\pgfsetroundjoin%
\definecolor{currentfill}{rgb}{0.121569,0.466667,0.705882}%
\pgfsetfillcolor{currentfill}%
\pgfsetfillopacity{0.618367}%
\pgfsetlinewidth{1.003750pt}%
\definecolor{currentstroke}{rgb}{0.121569,0.466667,0.705882}%
\pgfsetstrokecolor{currentstroke}%
\pgfsetstrokeopacity{0.618367}%
\pgfsetdash{}{0pt}%
\pgfpathmoveto{\pgfqpoint{0.672962in}{1.205300in}}%
\pgfpathcurveto{\pgfqpoint{0.681198in}{1.205300in}}{\pgfqpoint{0.689098in}{1.208573in}}{\pgfqpoint{0.694922in}{1.214396in}}%
\pgfpathcurveto{\pgfqpoint{0.700746in}{1.220220in}}{\pgfqpoint{0.704018in}{1.228120in}}{\pgfqpoint{0.704018in}{1.236357in}}%
\pgfpathcurveto{\pgfqpoint{0.704018in}{1.244593in}}{\pgfqpoint{0.700746in}{1.252493in}}{\pgfqpoint{0.694922in}{1.258317in}}%
\pgfpathcurveto{\pgfqpoint{0.689098in}{1.264141in}}{\pgfqpoint{0.681198in}{1.267413in}}{\pgfqpoint{0.672962in}{1.267413in}}%
\pgfpathcurveto{\pgfqpoint{0.664726in}{1.267413in}}{\pgfqpoint{0.656826in}{1.264141in}}{\pgfqpoint{0.651002in}{1.258317in}}%
\pgfpathcurveto{\pgfqpoint{0.645178in}{1.252493in}}{\pgfqpoint{0.641905in}{1.244593in}}{\pgfqpoint{0.641905in}{1.236357in}}%
\pgfpathcurveto{\pgfqpoint{0.641905in}{1.228120in}}{\pgfqpoint{0.645178in}{1.220220in}}{\pgfqpoint{0.651002in}{1.214396in}}%
\pgfpathcurveto{\pgfqpoint{0.656826in}{1.208573in}}{\pgfqpoint{0.664726in}{1.205300in}}{\pgfqpoint{0.672962in}{1.205300in}}%
\pgfpathclose%
\pgfusepath{stroke,fill}%
\end{pgfscope}%
\begin{pgfscope}%
\pgfpathrectangle{\pgfqpoint{0.100000in}{0.220728in}}{\pgfqpoint{3.696000in}{3.696000in}}%
\pgfusepath{clip}%
\pgfsetbuttcap%
\pgfsetroundjoin%
\definecolor{currentfill}{rgb}{0.121569,0.466667,0.705882}%
\pgfsetfillcolor{currentfill}%
\pgfsetfillopacity{0.618405}%
\pgfsetlinewidth{1.003750pt}%
\definecolor{currentstroke}{rgb}{0.121569,0.466667,0.705882}%
\pgfsetstrokecolor{currentstroke}%
\pgfsetstrokeopacity{0.618405}%
\pgfsetdash{}{0pt}%
\pgfpathmoveto{\pgfqpoint{0.672832in}{1.205039in}}%
\pgfpathcurveto{\pgfqpoint{0.681068in}{1.205039in}}{\pgfqpoint{0.688968in}{1.208311in}}{\pgfqpoint{0.694792in}{1.214135in}}%
\pgfpathcurveto{\pgfqpoint{0.700616in}{1.219959in}}{\pgfqpoint{0.703888in}{1.227859in}}{\pgfqpoint{0.703888in}{1.236095in}}%
\pgfpathcurveto{\pgfqpoint{0.703888in}{1.244331in}}{\pgfqpoint{0.700616in}{1.252231in}}{\pgfqpoint{0.694792in}{1.258055in}}%
\pgfpathcurveto{\pgfqpoint{0.688968in}{1.263879in}}{\pgfqpoint{0.681068in}{1.267152in}}{\pgfqpoint{0.672832in}{1.267152in}}%
\pgfpathcurveto{\pgfqpoint{0.664595in}{1.267152in}}{\pgfqpoint{0.656695in}{1.263879in}}{\pgfqpoint{0.650871in}{1.258055in}}%
\pgfpathcurveto{\pgfqpoint{0.645047in}{1.252231in}}{\pgfqpoint{0.641775in}{1.244331in}}{\pgfqpoint{0.641775in}{1.236095in}}%
\pgfpathcurveto{\pgfqpoint{0.641775in}{1.227859in}}{\pgfqpoint{0.645047in}{1.219959in}}{\pgfqpoint{0.650871in}{1.214135in}}%
\pgfpathcurveto{\pgfqpoint{0.656695in}{1.208311in}}{\pgfqpoint{0.664595in}{1.205039in}}{\pgfqpoint{0.672832in}{1.205039in}}%
\pgfpathclose%
\pgfusepath{stroke,fill}%
\end{pgfscope}%
\begin{pgfscope}%
\pgfpathrectangle{\pgfqpoint{0.100000in}{0.220728in}}{\pgfqpoint{3.696000in}{3.696000in}}%
\pgfusepath{clip}%
\pgfsetbuttcap%
\pgfsetroundjoin%
\definecolor{currentfill}{rgb}{0.121569,0.466667,0.705882}%
\pgfsetfillcolor{currentfill}%
\pgfsetfillopacity{0.618465}%
\pgfsetlinewidth{1.003750pt}%
\definecolor{currentstroke}{rgb}{0.121569,0.466667,0.705882}%
\pgfsetstrokecolor{currentstroke}%
\pgfsetstrokeopacity{0.618465}%
\pgfsetdash{}{0pt}%
\pgfpathmoveto{\pgfqpoint{0.672557in}{1.204574in}}%
\pgfpathcurveto{\pgfqpoint{0.680793in}{1.204574in}}{\pgfqpoint{0.688693in}{1.207846in}}{\pgfqpoint{0.694517in}{1.213670in}}%
\pgfpathcurveto{\pgfqpoint{0.700341in}{1.219494in}}{\pgfqpoint{0.703613in}{1.227394in}}{\pgfqpoint{0.703613in}{1.235630in}}%
\pgfpathcurveto{\pgfqpoint{0.703613in}{1.243866in}}{\pgfqpoint{0.700341in}{1.251767in}}{\pgfqpoint{0.694517in}{1.257590in}}%
\pgfpathcurveto{\pgfqpoint{0.688693in}{1.263414in}}{\pgfqpoint{0.680793in}{1.266687in}}{\pgfqpoint{0.672557in}{1.266687in}}%
\pgfpathcurveto{\pgfqpoint{0.664321in}{1.266687in}}{\pgfqpoint{0.656421in}{1.263414in}}{\pgfqpoint{0.650597in}{1.257590in}}%
\pgfpathcurveto{\pgfqpoint{0.644773in}{1.251767in}}{\pgfqpoint{0.641500in}{1.243866in}}{\pgfqpoint{0.641500in}{1.235630in}}%
\pgfpathcurveto{\pgfqpoint{0.641500in}{1.227394in}}{\pgfqpoint{0.644773in}{1.219494in}}{\pgfqpoint{0.650597in}{1.213670in}}%
\pgfpathcurveto{\pgfqpoint{0.656421in}{1.207846in}}{\pgfqpoint{0.664321in}{1.204574in}}{\pgfqpoint{0.672557in}{1.204574in}}%
\pgfpathclose%
\pgfusepath{stroke,fill}%
\end{pgfscope}%
\begin{pgfscope}%
\pgfpathrectangle{\pgfqpoint{0.100000in}{0.220728in}}{\pgfqpoint{3.696000in}{3.696000in}}%
\pgfusepath{clip}%
\pgfsetbuttcap%
\pgfsetroundjoin%
\definecolor{currentfill}{rgb}{0.121569,0.466667,0.705882}%
\pgfsetfillcolor{currentfill}%
\pgfsetfillopacity{0.618583}%
\pgfsetlinewidth{1.003750pt}%
\definecolor{currentstroke}{rgb}{0.121569,0.466667,0.705882}%
\pgfsetstrokecolor{currentstroke}%
\pgfsetstrokeopacity{0.618583}%
\pgfsetdash{}{0pt}%
\pgfpathmoveto{\pgfqpoint{0.672093in}{1.203721in}}%
\pgfpathcurveto{\pgfqpoint{0.680329in}{1.203721in}}{\pgfqpoint{0.688229in}{1.206994in}}{\pgfqpoint{0.694053in}{1.212818in}}%
\pgfpathcurveto{\pgfqpoint{0.699877in}{1.218642in}}{\pgfqpoint{0.703149in}{1.226542in}}{\pgfqpoint{0.703149in}{1.234778in}}%
\pgfpathcurveto{\pgfqpoint{0.703149in}{1.243014in}}{\pgfqpoint{0.699877in}{1.250914in}}{\pgfqpoint{0.694053in}{1.256738in}}%
\pgfpathcurveto{\pgfqpoint{0.688229in}{1.262562in}}{\pgfqpoint{0.680329in}{1.265834in}}{\pgfqpoint{0.672093in}{1.265834in}}%
\pgfpathcurveto{\pgfqpoint{0.663856in}{1.265834in}}{\pgfqpoint{0.655956in}{1.262562in}}{\pgfqpoint{0.650132in}{1.256738in}}%
\pgfpathcurveto{\pgfqpoint{0.644308in}{1.250914in}}{\pgfqpoint{0.641036in}{1.243014in}}{\pgfqpoint{0.641036in}{1.234778in}}%
\pgfpathcurveto{\pgfqpoint{0.641036in}{1.226542in}}{\pgfqpoint{0.644308in}{1.218642in}}{\pgfqpoint{0.650132in}{1.212818in}}%
\pgfpathcurveto{\pgfqpoint{0.655956in}{1.206994in}}{\pgfqpoint{0.663856in}{1.203721in}}{\pgfqpoint{0.672093in}{1.203721in}}%
\pgfpathclose%
\pgfusepath{stroke,fill}%
\end{pgfscope}%
\begin{pgfscope}%
\pgfpathrectangle{\pgfqpoint{0.100000in}{0.220728in}}{\pgfqpoint{3.696000in}{3.696000in}}%
\pgfusepath{clip}%
\pgfsetbuttcap%
\pgfsetroundjoin%
\definecolor{currentfill}{rgb}{0.121569,0.466667,0.705882}%
\pgfsetfillcolor{currentfill}%
\pgfsetfillopacity{0.618811}%
\pgfsetlinewidth{1.003750pt}%
\definecolor{currentstroke}{rgb}{0.121569,0.466667,0.705882}%
\pgfsetstrokecolor{currentstroke}%
\pgfsetstrokeopacity{0.618811}%
\pgfsetdash{}{0pt}%
\pgfpathmoveto{\pgfqpoint{0.671296in}{1.202161in}}%
\pgfpathcurveto{\pgfqpoint{0.679532in}{1.202161in}}{\pgfqpoint{0.687432in}{1.205434in}}{\pgfqpoint{0.693256in}{1.211258in}}%
\pgfpathcurveto{\pgfqpoint{0.699080in}{1.217082in}}{\pgfqpoint{0.702352in}{1.224982in}}{\pgfqpoint{0.702352in}{1.233218in}}%
\pgfpathcurveto{\pgfqpoint{0.702352in}{1.241454in}}{\pgfqpoint{0.699080in}{1.249354in}}{\pgfqpoint{0.693256in}{1.255178in}}%
\pgfpathcurveto{\pgfqpoint{0.687432in}{1.261002in}}{\pgfqpoint{0.679532in}{1.264274in}}{\pgfqpoint{0.671296in}{1.264274in}}%
\pgfpathcurveto{\pgfqpoint{0.663060in}{1.264274in}}{\pgfqpoint{0.655159in}{1.261002in}}{\pgfqpoint{0.649336in}{1.255178in}}%
\pgfpathcurveto{\pgfqpoint{0.643512in}{1.249354in}}{\pgfqpoint{0.640239in}{1.241454in}}{\pgfqpoint{0.640239in}{1.233218in}}%
\pgfpathcurveto{\pgfqpoint{0.640239in}{1.224982in}}{\pgfqpoint{0.643512in}{1.217082in}}{\pgfqpoint{0.649336in}{1.211258in}}%
\pgfpathcurveto{\pgfqpoint{0.655159in}{1.205434in}}{\pgfqpoint{0.663060in}{1.202161in}}{\pgfqpoint{0.671296in}{1.202161in}}%
\pgfpathclose%
\pgfusepath{stroke,fill}%
\end{pgfscope}%
\begin{pgfscope}%
\pgfpathrectangle{\pgfqpoint{0.100000in}{0.220728in}}{\pgfqpoint{3.696000in}{3.696000in}}%
\pgfusepath{clip}%
\pgfsetbuttcap%
\pgfsetroundjoin%
\definecolor{currentfill}{rgb}{0.121569,0.466667,0.705882}%
\pgfsetfillcolor{currentfill}%
\pgfsetfillopacity{0.618844}%
\pgfsetlinewidth{1.003750pt}%
\definecolor{currentstroke}{rgb}{0.121569,0.466667,0.705882}%
\pgfsetstrokecolor{currentstroke}%
\pgfsetstrokeopacity{0.618844}%
\pgfsetdash{}{0pt}%
\pgfpathmoveto{\pgfqpoint{0.685043in}{1.206123in}}%
\pgfpathcurveto{\pgfqpoint{0.693280in}{1.206123in}}{\pgfqpoint{0.701180in}{1.209395in}}{\pgfqpoint{0.707004in}{1.215219in}}%
\pgfpathcurveto{\pgfqpoint{0.712827in}{1.221043in}}{\pgfqpoint{0.716100in}{1.228943in}}{\pgfqpoint{0.716100in}{1.237179in}}%
\pgfpathcurveto{\pgfqpoint{0.716100in}{1.245416in}}{\pgfqpoint{0.712827in}{1.253316in}}{\pgfqpoint{0.707004in}{1.259140in}}%
\pgfpathcurveto{\pgfqpoint{0.701180in}{1.264964in}}{\pgfqpoint{0.693280in}{1.268236in}}{\pgfqpoint{0.685043in}{1.268236in}}%
\pgfpathcurveto{\pgfqpoint{0.676807in}{1.268236in}}{\pgfqpoint{0.668907in}{1.264964in}}{\pgfqpoint{0.663083in}{1.259140in}}%
\pgfpathcurveto{\pgfqpoint{0.657259in}{1.253316in}}{\pgfqpoint{0.653987in}{1.245416in}}{\pgfqpoint{0.653987in}{1.237179in}}%
\pgfpathcurveto{\pgfqpoint{0.653987in}{1.228943in}}{\pgfqpoint{0.657259in}{1.221043in}}{\pgfqpoint{0.663083in}{1.215219in}}%
\pgfpathcurveto{\pgfqpoint{0.668907in}{1.209395in}}{\pgfqpoint{0.676807in}{1.206123in}}{\pgfqpoint{0.685043in}{1.206123in}}%
\pgfpathclose%
\pgfusepath{stroke,fill}%
\end{pgfscope}%
\begin{pgfscope}%
\pgfpathrectangle{\pgfqpoint{0.100000in}{0.220728in}}{\pgfqpoint{3.696000in}{3.696000in}}%
\pgfusepath{clip}%
\pgfsetbuttcap%
\pgfsetroundjoin%
\definecolor{currentfill}{rgb}{0.121569,0.466667,0.705882}%
\pgfsetfillcolor{currentfill}%
\pgfsetfillopacity{0.619207}%
\pgfsetlinewidth{1.003750pt}%
\definecolor{currentstroke}{rgb}{0.121569,0.466667,0.705882}%
\pgfsetstrokecolor{currentstroke}%
\pgfsetstrokeopacity{0.619207}%
\pgfsetdash{}{0pt}%
\pgfpathmoveto{\pgfqpoint{0.669849in}{1.199241in}}%
\pgfpathcurveto{\pgfqpoint{0.678085in}{1.199241in}}{\pgfqpoint{0.685985in}{1.202513in}}{\pgfqpoint{0.691809in}{1.208337in}}%
\pgfpathcurveto{\pgfqpoint{0.697633in}{1.214161in}}{\pgfqpoint{0.700906in}{1.222061in}}{\pgfqpoint{0.700906in}{1.230297in}}%
\pgfpathcurveto{\pgfqpoint{0.700906in}{1.238533in}}{\pgfqpoint{0.697633in}{1.246433in}}{\pgfqpoint{0.691809in}{1.252257in}}%
\pgfpathcurveto{\pgfqpoint{0.685985in}{1.258081in}}{\pgfqpoint{0.678085in}{1.261354in}}{\pgfqpoint{0.669849in}{1.261354in}}%
\pgfpathcurveto{\pgfqpoint{0.661613in}{1.261354in}}{\pgfqpoint{0.653713in}{1.258081in}}{\pgfqpoint{0.647889in}{1.252257in}}%
\pgfpathcurveto{\pgfqpoint{0.642065in}{1.246433in}}{\pgfqpoint{0.638793in}{1.238533in}}{\pgfqpoint{0.638793in}{1.230297in}}%
\pgfpathcurveto{\pgfqpoint{0.638793in}{1.222061in}}{\pgfqpoint{0.642065in}{1.214161in}}{\pgfqpoint{0.647889in}{1.208337in}}%
\pgfpathcurveto{\pgfqpoint{0.653713in}{1.202513in}}{\pgfqpoint{0.661613in}{1.199241in}}{\pgfqpoint{0.669849in}{1.199241in}}%
\pgfpathclose%
\pgfusepath{stroke,fill}%
\end{pgfscope}%
\begin{pgfscope}%
\pgfpathrectangle{\pgfqpoint{0.100000in}{0.220728in}}{\pgfqpoint{3.696000in}{3.696000in}}%
\pgfusepath{clip}%
\pgfsetbuttcap%
\pgfsetroundjoin%
\definecolor{currentfill}{rgb}{0.121569,0.466667,0.705882}%
\pgfsetfillcolor{currentfill}%
\pgfsetfillopacity{0.619210}%
\pgfsetlinewidth{1.003750pt}%
\definecolor{currentstroke}{rgb}{0.121569,0.466667,0.705882}%
\pgfsetstrokecolor{currentstroke}%
\pgfsetstrokeopacity{0.619210}%
\pgfsetdash{}{0pt}%
\pgfpathmoveto{\pgfqpoint{0.669840in}{1.199219in}}%
\pgfpathcurveto{\pgfqpoint{0.678076in}{1.199219in}}{\pgfqpoint{0.685976in}{1.202491in}}{\pgfqpoint{0.691800in}{1.208315in}}%
\pgfpathcurveto{\pgfqpoint{0.697624in}{1.214139in}}{\pgfqpoint{0.700896in}{1.222039in}}{\pgfqpoint{0.700896in}{1.230275in}}%
\pgfpathcurveto{\pgfqpoint{0.700896in}{1.238512in}}{\pgfqpoint{0.697624in}{1.246412in}}{\pgfqpoint{0.691800in}{1.252236in}}%
\pgfpathcurveto{\pgfqpoint{0.685976in}{1.258060in}}{\pgfqpoint{0.678076in}{1.261332in}}{\pgfqpoint{0.669840in}{1.261332in}}%
\pgfpathcurveto{\pgfqpoint{0.661603in}{1.261332in}}{\pgfqpoint{0.653703in}{1.258060in}}{\pgfqpoint{0.647879in}{1.252236in}}%
\pgfpathcurveto{\pgfqpoint{0.642055in}{1.246412in}}{\pgfqpoint{0.638783in}{1.238512in}}{\pgfqpoint{0.638783in}{1.230275in}}%
\pgfpathcurveto{\pgfqpoint{0.638783in}{1.222039in}}{\pgfqpoint{0.642055in}{1.214139in}}{\pgfqpoint{0.647879in}{1.208315in}}%
\pgfpathcurveto{\pgfqpoint{0.653703in}{1.202491in}}{\pgfqpoint{0.661603in}{1.199219in}}{\pgfqpoint{0.669840in}{1.199219in}}%
\pgfpathclose%
\pgfusepath{stroke,fill}%
\end{pgfscope}%
\begin{pgfscope}%
\pgfpathrectangle{\pgfqpoint{0.100000in}{0.220728in}}{\pgfqpoint{3.696000in}{3.696000in}}%
\pgfusepath{clip}%
\pgfsetbuttcap%
\pgfsetroundjoin%
\definecolor{currentfill}{rgb}{0.121569,0.466667,0.705882}%
\pgfsetfillcolor{currentfill}%
\pgfsetfillopacity{0.619217}%
\pgfsetlinewidth{1.003750pt}%
\definecolor{currentstroke}{rgb}{0.121569,0.466667,0.705882}%
\pgfsetstrokecolor{currentstroke}%
\pgfsetstrokeopacity{0.619217}%
\pgfsetdash{}{0pt}%
\pgfpathmoveto{\pgfqpoint{0.669831in}{1.199178in}}%
\pgfpathcurveto{\pgfqpoint{0.678067in}{1.199178in}}{\pgfqpoint{0.685967in}{1.202451in}}{\pgfqpoint{0.691791in}{1.208275in}}%
\pgfpathcurveto{\pgfqpoint{0.697615in}{1.214098in}}{\pgfqpoint{0.700887in}{1.221999in}}{\pgfqpoint{0.700887in}{1.230235in}}%
\pgfpathcurveto{\pgfqpoint{0.700887in}{1.238471in}}{\pgfqpoint{0.697615in}{1.246371in}}{\pgfqpoint{0.691791in}{1.252195in}}%
\pgfpathcurveto{\pgfqpoint{0.685967in}{1.258019in}}{\pgfqpoint{0.678067in}{1.261291in}}{\pgfqpoint{0.669831in}{1.261291in}}%
\pgfpathcurveto{\pgfqpoint{0.661595in}{1.261291in}}{\pgfqpoint{0.653695in}{1.258019in}}{\pgfqpoint{0.647871in}{1.252195in}}%
\pgfpathcurveto{\pgfqpoint{0.642047in}{1.246371in}}{\pgfqpoint{0.638774in}{1.238471in}}{\pgfqpoint{0.638774in}{1.230235in}}%
\pgfpathcurveto{\pgfqpoint{0.638774in}{1.221999in}}{\pgfqpoint{0.642047in}{1.214098in}}{\pgfqpoint{0.647871in}{1.208275in}}%
\pgfpathcurveto{\pgfqpoint{0.653695in}{1.202451in}}{\pgfqpoint{0.661595in}{1.199178in}}{\pgfqpoint{0.669831in}{1.199178in}}%
\pgfpathclose%
\pgfusepath{stroke,fill}%
\end{pgfscope}%
\begin{pgfscope}%
\pgfpathrectangle{\pgfqpoint{0.100000in}{0.220728in}}{\pgfqpoint{3.696000in}{3.696000in}}%
\pgfusepath{clip}%
\pgfsetbuttcap%
\pgfsetroundjoin%
\definecolor{currentfill}{rgb}{0.121569,0.466667,0.705882}%
\pgfsetfillcolor{currentfill}%
\pgfsetfillopacity{0.619231}%
\pgfsetlinewidth{1.003750pt}%
\definecolor{currentstroke}{rgb}{0.121569,0.466667,0.705882}%
\pgfsetstrokecolor{currentstroke}%
\pgfsetstrokeopacity{0.619231}%
\pgfsetdash{}{0pt}%
\pgfpathmoveto{\pgfqpoint{0.669833in}{1.199113in}}%
\pgfpathcurveto{\pgfqpoint{0.678069in}{1.199113in}}{\pgfqpoint{0.685969in}{1.202386in}}{\pgfqpoint{0.691793in}{1.208210in}}%
\pgfpathcurveto{\pgfqpoint{0.697617in}{1.214033in}}{\pgfqpoint{0.700889in}{1.221933in}}{\pgfqpoint{0.700889in}{1.230170in}}%
\pgfpathcurveto{\pgfqpoint{0.700889in}{1.238406in}}{\pgfqpoint{0.697617in}{1.246306in}}{\pgfqpoint{0.691793in}{1.252130in}}%
\pgfpathcurveto{\pgfqpoint{0.685969in}{1.257954in}}{\pgfqpoint{0.678069in}{1.261226in}}{\pgfqpoint{0.669833in}{1.261226in}}%
\pgfpathcurveto{\pgfqpoint{0.661597in}{1.261226in}}{\pgfqpoint{0.653697in}{1.257954in}}{\pgfqpoint{0.647873in}{1.252130in}}%
\pgfpathcurveto{\pgfqpoint{0.642049in}{1.246306in}}{\pgfqpoint{0.638776in}{1.238406in}}{\pgfqpoint{0.638776in}{1.230170in}}%
\pgfpathcurveto{\pgfqpoint{0.638776in}{1.221933in}}{\pgfqpoint{0.642049in}{1.214033in}}{\pgfqpoint{0.647873in}{1.208210in}}%
\pgfpathcurveto{\pgfqpoint{0.653697in}{1.202386in}}{\pgfqpoint{0.661597in}{1.199113in}}{\pgfqpoint{0.669833in}{1.199113in}}%
\pgfpathclose%
\pgfusepath{stroke,fill}%
\end{pgfscope}%
\begin{pgfscope}%
\pgfpathrectangle{\pgfqpoint{0.100000in}{0.220728in}}{\pgfqpoint{3.696000in}{3.696000in}}%
\pgfusepath{clip}%
\pgfsetbuttcap%
\pgfsetroundjoin%
\definecolor{currentfill}{rgb}{0.121569,0.466667,0.705882}%
\pgfsetfillcolor{currentfill}%
\pgfsetfillopacity{0.619255}%
\pgfsetlinewidth{1.003750pt}%
\definecolor{currentstroke}{rgb}{0.121569,0.466667,0.705882}%
\pgfsetstrokecolor{currentstroke}%
\pgfsetstrokeopacity{0.619255}%
\pgfsetdash{}{0pt}%
\pgfpathmoveto{\pgfqpoint{0.669865in}{1.199002in}}%
\pgfpathcurveto{\pgfqpoint{0.678102in}{1.199002in}}{\pgfqpoint{0.686002in}{1.202274in}}{\pgfqpoint{0.691826in}{1.208098in}}%
\pgfpathcurveto{\pgfqpoint{0.697650in}{1.213922in}}{\pgfqpoint{0.700922in}{1.221822in}}{\pgfqpoint{0.700922in}{1.230058in}}%
\pgfpathcurveto{\pgfqpoint{0.700922in}{1.238294in}}{\pgfqpoint{0.697650in}{1.246194in}}{\pgfqpoint{0.691826in}{1.252018in}}%
\pgfpathcurveto{\pgfqpoint{0.686002in}{1.257842in}}{\pgfqpoint{0.678102in}{1.261115in}}{\pgfqpoint{0.669865in}{1.261115in}}%
\pgfpathcurveto{\pgfqpoint{0.661629in}{1.261115in}}{\pgfqpoint{0.653729in}{1.257842in}}{\pgfqpoint{0.647905in}{1.252018in}}%
\pgfpathcurveto{\pgfqpoint{0.642081in}{1.246194in}}{\pgfqpoint{0.638809in}{1.238294in}}{\pgfqpoint{0.638809in}{1.230058in}}%
\pgfpathcurveto{\pgfqpoint{0.638809in}{1.221822in}}{\pgfqpoint{0.642081in}{1.213922in}}{\pgfqpoint{0.647905in}{1.208098in}}%
\pgfpathcurveto{\pgfqpoint{0.653729in}{1.202274in}}{\pgfqpoint{0.661629in}{1.199002in}}{\pgfqpoint{0.669865in}{1.199002in}}%
\pgfpathclose%
\pgfusepath{stroke,fill}%
\end{pgfscope}%
\begin{pgfscope}%
\pgfpathrectangle{\pgfqpoint{0.100000in}{0.220728in}}{\pgfqpoint{3.696000in}{3.696000in}}%
\pgfusepath{clip}%
\pgfsetbuttcap%
\pgfsetroundjoin%
\definecolor{currentfill}{rgb}{0.121569,0.466667,0.705882}%
\pgfsetfillcolor{currentfill}%
\pgfsetfillopacity{0.619294}%
\pgfsetlinewidth{1.003750pt}%
\definecolor{currentstroke}{rgb}{0.121569,0.466667,0.705882}%
\pgfsetstrokecolor{currentstroke}%
\pgfsetstrokeopacity{0.619294}%
\pgfsetdash{}{0pt}%
\pgfpathmoveto{\pgfqpoint{0.669974in}{1.198832in}}%
\pgfpathcurveto{\pgfqpoint{0.678210in}{1.198832in}}{\pgfqpoint{0.686111in}{1.202104in}}{\pgfqpoint{0.691934in}{1.207928in}}%
\pgfpathcurveto{\pgfqpoint{0.697758in}{1.213752in}}{\pgfqpoint{0.701031in}{1.221652in}}{\pgfqpoint{0.701031in}{1.229888in}}%
\pgfpathcurveto{\pgfqpoint{0.701031in}{1.238125in}}{\pgfqpoint{0.697758in}{1.246025in}}{\pgfqpoint{0.691934in}{1.251848in}}%
\pgfpathcurveto{\pgfqpoint{0.686111in}{1.257672in}}{\pgfqpoint{0.678210in}{1.260945in}}{\pgfqpoint{0.669974in}{1.260945in}}%
\pgfpathcurveto{\pgfqpoint{0.661738in}{1.260945in}}{\pgfqpoint{0.653838in}{1.257672in}}{\pgfqpoint{0.648014in}{1.251848in}}%
\pgfpathcurveto{\pgfqpoint{0.642190in}{1.246025in}}{\pgfqpoint{0.638918in}{1.238125in}}{\pgfqpoint{0.638918in}{1.229888in}}%
\pgfpathcurveto{\pgfqpoint{0.638918in}{1.221652in}}{\pgfqpoint{0.642190in}{1.213752in}}{\pgfqpoint{0.648014in}{1.207928in}}%
\pgfpathcurveto{\pgfqpoint{0.653838in}{1.202104in}}{\pgfqpoint{0.661738in}{1.198832in}}{\pgfqpoint{0.669974in}{1.198832in}}%
\pgfpathclose%
\pgfusepath{stroke,fill}%
\end{pgfscope}%
\begin{pgfscope}%
\pgfpathrectangle{\pgfqpoint{0.100000in}{0.220728in}}{\pgfqpoint{3.696000in}{3.696000in}}%
\pgfusepath{clip}%
\pgfsetbuttcap%
\pgfsetroundjoin%
\definecolor{currentfill}{rgb}{0.121569,0.466667,0.705882}%
\pgfsetfillcolor{currentfill}%
\pgfsetfillopacity{0.619315}%
\pgfsetlinewidth{1.003750pt}%
\definecolor{currentstroke}{rgb}{0.121569,0.466667,0.705882}%
\pgfsetstrokecolor{currentstroke}%
\pgfsetstrokeopacity{0.619315}%
\pgfsetdash{}{0pt}%
\pgfpathmoveto{\pgfqpoint{0.682427in}{1.203196in}}%
\pgfpathcurveto{\pgfqpoint{0.690663in}{1.203196in}}{\pgfqpoint{0.698563in}{1.206469in}}{\pgfqpoint{0.704387in}{1.212293in}}%
\pgfpathcurveto{\pgfqpoint{0.710211in}{1.218117in}}{\pgfqpoint{0.713483in}{1.226017in}}{\pgfqpoint{0.713483in}{1.234253in}}%
\pgfpathcurveto{\pgfqpoint{0.713483in}{1.242489in}}{\pgfqpoint{0.710211in}{1.250389in}}{\pgfqpoint{0.704387in}{1.256213in}}%
\pgfpathcurveto{\pgfqpoint{0.698563in}{1.262037in}}{\pgfqpoint{0.690663in}{1.265309in}}{\pgfqpoint{0.682427in}{1.265309in}}%
\pgfpathcurveto{\pgfqpoint{0.674190in}{1.265309in}}{\pgfqpoint{0.666290in}{1.262037in}}{\pgfqpoint{0.660466in}{1.256213in}}%
\pgfpathcurveto{\pgfqpoint{0.654642in}{1.250389in}}{\pgfqpoint{0.651370in}{1.242489in}}{\pgfqpoint{0.651370in}{1.234253in}}%
\pgfpathcurveto{\pgfqpoint{0.651370in}{1.226017in}}{\pgfqpoint{0.654642in}{1.218117in}}{\pgfqpoint{0.660466in}{1.212293in}}%
\pgfpathcurveto{\pgfqpoint{0.666290in}{1.206469in}}{\pgfqpoint{0.674190in}{1.203196in}}{\pgfqpoint{0.682427in}{1.203196in}}%
\pgfpathclose%
\pgfusepath{stroke,fill}%
\end{pgfscope}%
\begin{pgfscope}%
\pgfpathrectangle{\pgfqpoint{0.100000in}{0.220728in}}{\pgfqpoint{3.696000in}{3.696000in}}%
\pgfusepath{clip}%
\pgfsetbuttcap%
\pgfsetroundjoin%
\definecolor{currentfill}{rgb}{0.121569,0.466667,0.705882}%
\pgfsetfillcolor{currentfill}%
\pgfsetfillopacity{0.619352}%
\pgfsetlinewidth{1.003750pt}%
\definecolor{currentstroke}{rgb}{0.121569,0.466667,0.705882}%
\pgfsetstrokecolor{currentstroke}%
\pgfsetstrokeopacity{0.619352}%
\pgfsetdash{}{0pt}%
\pgfpathmoveto{\pgfqpoint{0.670252in}{1.198599in}}%
\pgfpathcurveto{\pgfqpoint{0.678488in}{1.198599in}}{\pgfqpoint{0.686388in}{1.201872in}}{\pgfqpoint{0.692212in}{1.207696in}}%
\pgfpathcurveto{\pgfqpoint{0.698036in}{1.213520in}}{\pgfqpoint{0.701309in}{1.221420in}}{\pgfqpoint{0.701309in}{1.229656in}}%
\pgfpathcurveto{\pgfqpoint{0.701309in}{1.237892in}}{\pgfqpoint{0.698036in}{1.245792in}}{\pgfqpoint{0.692212in}{1.251616in}}%
\pgfpathcurveto{\pgfqpoint{0.686388in}{1.257440in}}{\pgfqpoint{0.678488in}{1.260712in}}{\pgfqpoint{0.670252in}{1.260712in}}%
\pgfpathcurveto{\pgfqpoint{0.662016in}{1.260712in}}{\pgfqpoint{0.654116in}{1.257440in}}{\pgfqpoint{0.648292in}{1.251616in}}%
\pgfpathcurveto{\pgfqpoint{0.642468in}{1.245792in}}{\pgfqpoint{0.639196in}{1.237892in}}{\pgfqpoint{0.639196in}{1.229656in}}%
\pgfpathcurveto{\pgfqpoint{0.639196in}{1.221420in}}{\pgfqpoint{0.642468in}{1.213520in}}{\pgfqpoint{0.648292in}{1.207696in}}%
\pgfpathcurveto{\pgfqpoint{0.654116in}{1.201872in}}{\pgfqpoint{0.662016in}{1.198599in}}{\pgfqpoint{0.670252in}{1.198599in}}%
\pgfpathclose%
\pgfusepath{stroke,fill}%
\end{pgfscope}%
\begin{pgfscope}%
\pgfpathrectangle{\pgfqpoint{0.100000in}{0.220728in}}{\pgfqpoint{3.696000in}{3.696000in}}%
\pgfusepath{clip}%
\pgfsetbuttcap%
\pgfsetroundjoin%
\definecolor{currentfill}{rgb}{0.121569,0.466667,0.705882}%
\pgfsetfillcolor{currentfill}%
\pgfsetfillopacity{0.619384}%
\pgfsetlinewidth{1.003750pt}%
\definecolor{currentstroke}{rgb}{0.121569,0.466667,0.705882}%
\pgfsetstrokecolor{currentstroke}%
\pgfsetstrokeopacity{0.619384}%
\pgfsetdash{}{0pt}%
\pgfpathmoveto{\pgfqpoint{3.013487in}{2.930249in}}%
\pgfpathcurveto{\pgfqpoint{3.021723in}{2.930249in}}{\pgfqpoint{3.029623in}{2.933522in}}{\pgfqpoint{3.035447in}{2.939346in}}%
\pgfpathcurveto{\pgfqpoint{3.041271in}{2.945169in}}{\pgfqpoint{3.044543in}{2.953069in}}{\pgfqpoint{3.044543in}{2.961306in}}%
\pgfpathcurveto{\pgfqpoint{3.044543in}{2.969542in}}{\pgfqpoint{3.041271in}{2.977442in}}{\pgfqpoint{3.035447in}{2.983266in}}%
\pgfpathcurveto{\pgfqpoint{3.029623in}{2.989090in}}{\pgfqpoint{3.021723in}{2.992362in}}{\pgfqpoint{3.013487in}{2.992362in}}%
\pgfpathcurveto{\pgfqpoint{3.005251in}{2.992362in}}{\pgfqpoint{2.997351in}{2.989090in}}{\pgfqpoint{2.991527in}{2.983266in}}%
\pgfpathcurveto{\pgfqpoint{2.985703in}{2.977442in}}{\pgfqpoint{2.982430in}{2.969542in}}{\pgfqpoint{2.982430in}{2.961306in}}%
\pgfpathcurveto{\pgfqpoint{2.982430in}{2.953069in}}{\pgfqpoint{2.985703in}{2.945169in}}{\pgfqpoint{2.991527in}{2.939346in}}%
\pgfpathcurveto{\pgfqpoint{2.997351in}{2.933522in}}{\pgfqpoint{3.005251in}{2.930249in}}{\pgfqpoint{3.013487in}{2.930249in}}%
\pgfpathclose%
\pgfusepath{stroke,fill}%
\end{pgfscope}%
\begin{pgfscope}%
\pgfpathrectangle{\pgfqpoint{0.100000in}{0.220728in}}{\pgfqpoint{3.696000in}{3.696000in}}%
\pgfusepath{clip}%
\pgfsetbuttcap%
\pgfsetroundjoin%
\definecolor{currentfill}{rgb}{0.121569,0.466667,0.705882}%
\pgfsetfillcolor{currentfill}%
\pgfsetfillopacity{0.619437}%
\pgfsetlinewidth{1.003750pt}%
\definecolor{currentstroke}{rgb}{0.121569,0.466667,0.705882}%
\pgfsetstrokecolor{currentstroke}%
\pgfsetstrokeopacity{0.619437}%
\pgfsetdash{}{0pt}%
\pgfpathmoveto{\pgfqpoint{0.670911in}{1.198627in}}%
\pgfpathcurveto{\pgfqpoint{0.679148in}{1.198627in}}{\pgfqpoint{0.687048in}{1.201899in}}{\pgfqpoint{0.692872in}{1.207723in}}%
\pgfpathcurveto{\pgfqpoint{0.698696in}{1.213547in}}{\pgfqpoint{0.701968in}{1.221447in}}{\pgfqpoint{0.701968in}{1.229683in}}%
\pgfpathcurveto{\pgfqpoint{0.701968in}{1.237919in}}{\pgfqpoint{0.698696in}{1.245819in}}{\pgfqpoint{0.692872in}{1.251643in}}%
\pgfpathcurveto{\pgfqpoint{0.687048in}{1.257467in}}{\pgfqpoint{0.679148in}{1.260740in}}{\pgfqpoint{0.670911in}{1.260740in}}%
\pgfpathcurveto{\pgfqpoint{0.662675in}{1.260740in}}{\pgfqpoint{0.654775in}{1.257467in}}{\pgfqpoint{0.648951in}{1.251643in}}%
\pgfpathcurveto{\pgfqpoint{0.643127in}{1.245819in}}{\pgfqpoint{0.639855in}{1.237919in}}{\pgfqpoint{0.639855in}{1.229683in}}%
\pgfpathcurveto{\pgfqpoint{0.639855in}{1.221447in}}{\pgfqpoint{0.643127in}{1.213547in}}{\pgfqpoint{0.648951in}{1.207723in}}%
\pgfpathcurveto{\pgfqpoint{0.654775in}{1.201899in}}{\pgfqpoint{0.662675in}{1.198627in}}{\pgfqpoint{0.670911in}{1.198627in}}%
\pgfpathclose%
\pgfusepath{stroke,fill}%
\end{pgfscope}%
\begin{pgfscope}%
\pgfpathrectangle{\pgfqpoint{0.100000in}{0.220728in}}{\pgfqpoint{3.696000in}{3.696000in}}%
\pgfusepath{clip}%
\pgfsetbuttcap%
\pgfsetroundjoin%
\definecolor{currentfill}{rgb}{0.121569,0.466667,0.705882}%
\pgfsetfillcolor{currentfill}%
\pgfsetfillopacity{0.619568}%
\pgfsetlinewidth{1.003750pt}%
\definecolor{currentstroke}{rgb}{0.121569,0.466667,0.705882}%
\pgfsetstrokecolor{currentstroke}%
\pgfsetstrokeopacity{0.619568}%
\pgfsetdash{}{0pt}%
\pgfpathmoveto{\pgfqpoint{0.680991in}{1.201549in}}%
\pgfpathcurveto{\pgfqpoint{0.689227in}{1.201549in}}{\pgfqpoint{0.697127in}{1.204821in}}{\pgfqpoint{0.702951in}{1.210645in}}%
\pgfpathcurveto{\pgfqpoint{0.708775in}{1.216469in}}{\pgfqpoint{0.712047in}{1.224369in}}{\pgfqpoint{0.712047in}{1.232605in}}%
\pgfpathcurveto{\pgfqpoint{0.712047in}{1.240842in}}{\pgfqpoint{0.708775in}{1.248742in}}{\pgfqpoint{0.702951in}{1.254566in}}%
\pgfpathcurveto{\pgfqpoint{0.697127in}{1.260389in}}{\pgfqpoint{0.689227in}{1.263662in}}{\pgfqpoint{0.680991in}{1.263662in}}%
\pgfpathcurveto{\pgfqpoint{0.672754in}{1.263662in}}{\pgfqpoint{0.664854in}{1.260389in}}{\pgfqpoint{0.659030in}{1.254566in}}%
\pgfpathcurveto{\pgfqpoint{0.653206in}{1.248742in}}{\pgfqpoint{0.649934in}{1.240842in}}{\pgfqpoint{0.649934in}{1.232605in}}%
\pgfpathcurveto{\pgfqpoint{0.649934in}{1.224369in}}{\pgfqpoint{0.653206in}{1.216469in}}{\pgfqpoint{0.659030in}{1.210645in}}%
\pgfpathcurveto{\pgfqpoint{0.664854in}{1.204821in}}{\pgfqpoint{0.672754in}{1.201549in}}{\pgfqpoint{0.680991in}{1.201549in}}%
\pgfpathclose%
\pgfusepath{stroke,fill}%
\end{pgfscope}%
\begin{pgfscope}%
\pgfpathrectangle{\pgfqpoint{0.100000in}{0.220728in}}{\pgfqpoint{3.696000in}{3.696000in}}%
\pgfusepath{clip}%
\pgfsetbuttcap%
\pgfsetroundjoin%
\definecolor{currentfill}{rgb}{0.121569,0.466667,0.705882}%
\pgfsetfillcolor{currentfill}%
\pgfsetfillopacity{0.619570}%
\pgfsetlinewidth{1.003750pt}%
\definecolor{currentstroke}{rgb}{0.121569,0.466667,0.705882}%
\pgfsetstrokecolor{currentstroke}%
\pgfsetstrokeopacity{0.619570}%
\pgfsetdash{}{0pt}%
\pgfpathmoveto{\pgfqpoint{0.672084in}{1.198409in}}%
\pgfpathcurveto{\pgfqpoint{0.680320in}{1.198409in}}{\pgfqpoint{0.688220in}{1.201681in}}{\pgfqpoint{0.694044in}{1.207505in}}%
\pgfpathcurveto{\pgfqpoint{0.699868in}{1.213329in}}{\pgfqpoint{0.703140in}{1.221229in}}{\pgfqpoint{0.703140in}{1.229466in}}%
\pgfpathcurveto{\pgfqpoint{0.703140in}{1.237702in}}{\pgfqpoint{0.699868in}{1.245602in}}{\pgfqpoint{0.694044in}{1.251426in}}%
\pgfpathcurveto{\pgfqpoint{0.688220in}{1.257250in}}{\pgfqpoint{0.680320in}{1.260522in}}{\pgfqpoint{0.672084in}{1.260522in}}%
\pgfpathcurveto{\pgfqpoint{0.663848in}{1.260522in}}{\pgfqpoint{0.655947in}{1.257250in}}{\pgfqpoint{0.650124in}{1.251426in}}%
\pgfpathcurveto{\pgfqpoint{0.644300in}{1.245602in}}{\pgfqpoint{0.641027in}{1.237702in}}{\pgfqpoint{0.641027in}{1.229466in}}%
\pgfpathcurveto{\pgfqpoint{0.641027in}{1.221229in}}{\pgfqpoint{0.644300in}{1.213329in}}{\pgfqpoint{0.650124in}{1.207505in}}%
\pgfpathcurveto{\pgfqpoint{0.655947in}{1.201681in}}{\pgfqpoint{0.663848in}{1.198409in}}{\pgfqpoint{0.672084in}{1.198409in}}%
\pgfpathclose%
\pgfusepath{stroke,fill}%
\end{pgfscope}%
\begin{pgfscope}%
\pgfpathrectangle{\pgfqpoint{0.100000in}{0.220728in}}{\pgfqpoint{3.696000in}{3.696000in}}%
\pgfusepath{clip}%
\pgfsetbuttcap%
\pgfsetroundjoin%
\definecolor{currentfill}{rgb}{0.121569,0.466667,0.705882}%
\pgfsetfillcolor{currentfill}%
\pgfsetfillopacity{0.619704}%
\pgfsetlinewidth{1.003750pt}%
\definecolor{currentstroke}{rgb}{0.121569,0.466667,0.705882}%
\pgfsetstrokecolor{currentstroke}%
\pgfsetstrokeopacity{0.619704}%
\pgfsetdash{}{0pt}%
\pgfpathmoveto{\pgfqpoint{0.680173in}{1.200692in}}%
\pgfpathcurveto{\pgfqpoint{0.688409in}{1.200692in}}{\pgfqpoint{0.696309in}{1.203964in}}{\pgfqpoint{0.702133in}{1.209788in}}%
\pgfpathcurveto{\pgfqpoint{0.707957in}{1.215612in}}{\pgfqpoint{0.711229in}{1.223512in}}{\pgfqpoint{0.711229in}{1.231748in}}%
\pgfpathcurveto{\pgfqpoint{0.711229in}{1.239985in}}{\pgfqpoint{0.707957in}{1.247885in}}{\pgfqpoint{0.702133in}{1.253709in}}%
\pgfpathcurveto{\pgfqpoint{0.696309in}{1.259533in}}{\pgfqpoint{0.688409in}{1.262805in}}{\pgfqpoint{0.680173in}{1.262805in}}%
\pgfpathcurveto{\pgfqpoint{0.671936in}{1.262805in}}{\pgfqpoint{0.664036in}{1.259533in}}{\pgfqpoint{0.658212in}{1.253709in}}%
\pgfpathcurveto{\pgfqpoint{0.652388in}{1.247885in}}{\pgfqpoint{0.649116in}{1.239985in}}{\pgfqpoint{0.649116in}{1.231748in}}%
\pgfpathcurveto{\pgfqpoint{0.649116in}{1.223512in}}{\pgfqpoint{0.652388in}{1.215612in}}{\pgfqpoint{0.658212in}{1.209788in}}%
\pgfpathcurveto{\pgfqpoint{0.664036in}{1.203964in}}{\pgfqpoint{0.671936in}{1.200692in}}{\pgfqpoint{0.680173in}{1.200692in}}%
\pgfpathclose%
\pgfusepath{stroke,fill}%
\end{pgfscope}%
\begin{pgfscope}%
\pgfpathrectangle{\pgfqpoint{0.100000in}{0.220728in}}{\pgfqpoint{3.696000in}{3.696000in}}%
\pgfusepath{clip}%
\pgfsetbuttcap%
\pgfsetroundjoin%
\definecolor{currentfill}{rgb}{0.121569,0.466667,0.705882}%
\pgfsetfillcolor{currentfill}%
\pgfsetfillopacity{0.619760}%
\pgfsetlinewidth{1.003750pt}%
\definecolor{currentstroke}{rgb}{0.121569,0.466667,0.705882}%
\pgfsetstrokecolor{currentstroke}%
\pgfsetstrokeopacity{0.619760}%
\pgfsetdash{}{0pt}%
\pgfpathmoveto{\pgfqpoint{0.674287in}{1.198128in}}%
\pgfpathcurveto{\pgfqpoint{0.682523in}{1.198128in}}{\pgfqpoint{0.690423in}{1.201400in}}{\pgfqpoint{0.696247in}{1.207224in}}%
\pgfpathcurveto{\pgfqpoint{0.702071in}{1.213048in}}{\pgfqpoint{0.705344in}{1.220948in}}{\pgfqpoint{0.705344in}{1.229184in}}%
\pgfpathcurveto{\pgfqpoint{0.705344in}{1.237421in}}{\pgfqpoint{0.702071in}{1.245321in}}{\pgfqpoint{0.696247in}{1.251145in}}%
\pgfpathcurveto{\pgfqpoint{0.690423in}{1.256969in}}{\pgfqpoint{0.682523in}{1.260241in}}{\pgfqpoint{0.674287in}{1.260241in}}%
\pgfpathcurveto{\pgfqpoint{0.666051in}{1.260241in}}{\pgfqpoint{0.658151in}{1.256969in}}{\pgfqpoint{0.652327in}{1.251145in}}%
\pgfpathcurveto{\pgfqpoint{0.646503in}{1.245321in}}{\pgfqpoint{0.643231in}{1.237421in}}{\pgfqpoint{0.643231in}{1.229184in}}%
\pgfpathcurveto{\pgfqpoint{0.643231in}{1.220948in}}{\pgfqpoint{0.646503in}{1.213048in}}{\pgfqpoint{0.652327in}{1.207224in}}%
\pgfpathcurveto{\pgfqpoint{0.658151in}{1.201400in}}{\pgfqpoint{0.666051in}{1.198128in}}{\pgfqpoint{0.674287in}{1.198128in}}%
\pgfpathclose%
\pgfusepath{stroke,fill}%
\end{pgfscope}%
\begin{pgfscope}%
\pgfpathrectangle{\pgfqpoint{0.100000in}{0.220728in}}{\pgfqpoint{3.696000in}{3.696000in}}%
\pgfusepath{clip}%
\pgfsetbuttcap%
\pgfsetroundjoin%
\definecolor{currentfill}{rgb}{0.121569,0.466667,0.705882}%
\pgfsetfillcolor{currentfill}%
\pgfsetfillopacity{0.619767}%
\pgfsetlinewidth{1.003750pt}%
\definecolor{currentstroke}{rgb}{0.121569,0.466667,0.705882}%
\pgfsetstrokecolor{currentstroke}%
\pgfsetstrokeopacity{0.619767}%
\pgfsetdash{}{0pt}%
\pgfpathmoveto{\pgfqpoint{0.674400in}{1.198139in}}%
\pgfpathcurveto{\pgfqpoint{0.682636in}{1.198139in}}{\pgfqpoint{0.690536in}{1.201411in}}{\pgfqpoint{0.696360in}{1.207235in}}%
\pgfpathcurveto{\pgfqpoint{0.702184in}{1.213059in}}{\pgfqpoint{0.705457in}{1.220959in}}{\pgfqpoint{0.705457in}{1.229195in}}%
\pgfpathcurveto{\pgfqpoint{0.705457in}{1.237431in}}{\pgfqpoint{0.702184in}{1.245332in}}{\pgfqpoint{0.696360in}{1.251155in}}%
\pgfpathcurveto{\pgfqpoint{0.690536in}{1.256979in}}{\pgfqpoint{0.682636in}{1.260252in}}{\pgfqpoint{0.674400in}{1.260252in}}%
\pgfpathcurveto{\pgfqpoint{0.666164in}{1.260252in}}{\pgfqpoint{0.658264in}{1.256979in}}{\pgfqpoint{0.652440in}{1.251155in}}%
\pgfpathcurveto{\pgfqpoint{0.646616in}{1.245332in}}{\pgfqpoint{0.643344in}{1.237431in}}{\pgfqpoint{0.643344in}{1.229195in}}%
\pgfpathcurveto{\pgfqpoint{0.643344in}{1.220959in}}{\pgfqpoint{0.646616in}{1.213059in}}{\pgfqpoint{0.652440in}{1.207235in}}%
\pgfpathcurveto{\pgfqpoint{0.658264in}{1.201411in}}{\pgfqpoint{0.666164in}{1.198139in}}{\pgfqpoint{0.674400in}{1.198139in}}%
\pgfpathclose%
\pgfusepath{stroke,fill}%
\end{pgfscope}%
\begin{pgfscope}%
\pgfpathrectangle{\pgfqpoint{0.100000in}{0.220728in}}{\pgfqpoint{3.696000in}{3.696000in}}%
\pgfusepath{clip}%
\pgfsetbuttcap%
\pgfsetroundjoin%
\definecolor{currentfill}{rgb}{0.121569,0.466667,0.705882}%
\pgfsetfillcolor{currentfill}%
\pgfsetfillopacity{0.619777}%
\pgfsetlinewidth{1.003750pt}%
\definecolor{currentstroke}{rgb}{0.121569,0.466667,0.705882}%
\pgfsetstrokecolor{currentstroke}%
\pgfsetstrokeopacity{0.619777}%
\pgfsetdash{}{0pt}%
\pgfpathmoveto{\pgfqpoint{0.674607in}{1.198183in}}%
\pgfpathcurveto{\pgfqpoint{0.682844in}{1.198183in}}{\pgfqpoint{0.690744in}{1.201455in}}{\pgfqpoint{0.696567in}{1.207279in}}%
\pgfpathcurveto{\pgfqpoint{0.702391in}{1.213103in}}{\pgfqpoint{0.705664in}{1.221003in}}{\pgfqpoint{0.705664in}{1.229240in}}%
\pgfpathcurveto{\pgfqpoint{0.705664in}{1.237476in}}{\pgfqpoint{0.702391in}{1.245376in}}{\pgfqpoint{0.696567in}{1.251200in}}%
\pgfpathcurveto{\pgfqpoint{0.690744in}{1.257024in}}{\pgfqpoint{0.682844in}{1.260296in}}{\pgfqpoint{0.674607in}{1.260296in}}%
\pgfpathcurveto{\pgfqpoint{0.666371in}{1.260296in}}{\pgfqpoint{0.658471in}{1.257024in}}{\pgfqpoint{0.652647in}{1.251200in}}%
\pgfpathcurveto{\pgfqpoint{0.646823in}{1.245376in}}{\pgfqpoint{0.643551in}{1.237476in}}{\pgfqpoint{0.643551in}{1.229240in}}%
\pgfpathcurveto{\pgfqpoint{0.643551in}{1.221003in}}{\pgfqpoint{0.646823in}{1.213103in}}{\pgfqpoint{0.652647in}{1.207279in}}%
\pgfpathcurveto{\pgfqpoint{0.658471in}{1.201455in}}{\pgfqpoint{0.666371in}{1.198183in}}{\pgfqpoint{0.674607in}{1.198183in}}%
\pgfpathclose%
\pgfusepath{stroke,fill}%
\end{pgfscope}%
\begin{pgfscope}%
\pgfpathrectangle{\pgfqpoint{0.100000in}{0.220728in}}{\pgfqpoint{3.696000in}{3.696000in}}%
\pgfusepath{clip}%
\pgfsetbuttcap%
\pgfsetroundjoin%
\definecolor{currentfill}{rgb}{0.121569,0.466667,0.705882}%
\pgfsetfillcolor{currentfill}%
\pgfsetfillopacity{0.619780}%
\pgfsetlinewidth{1.003750pt}%
\definecolor{currentstroke}{rgb}{0.121569,0.466667,0.705882}%
\pgfsetstrokecolor{currentstroke}%
\pgfsetstrokeopacity{0.619780}%
\pgfsetdash{}{0pt}%
\pgfpathmoveto{\pgfqpoint{0.679736in}{1.200201in}}%
\pgfpathcurveto{\pgfqpoint{0.687972in}{1.200201in}}{\pgfqpoint{0.695872in}{1.203473in}}{\pgfqpoint{0.701696in}{1.209297in}}%
\pgfpathcurveto{\pgfqpoint{0.707520in}{1.215121in}}{\pgfqpoint{0.710792in}{1.223021in}}{\pgfqpoint{0.710792in}{1.231258in}}%
\pgfpathcurveto{\pgfqpoint{0.710792in}{1.239494in}}{\pgfqpoint{0.707520in}{1.247394in}}{\pgfqpoint{0.701696in}{1.253218in}}%
\pgfpathcurveto{\pgfqpoint{0.695872in}{1.259042in}}{\pgfqpoint{0.687972in}{1.262314in}}{\pgfqpoint{0.679736in}{1.262314in}}%
\pgfpathcurveto{\pgfqpoint{0.671499in}{1.262314in}}{\pgfqpoint{0.663599in}{1.259042in}}{\pgfqpoint{0.657775in}{1.253218in}}%
\pgfpathcurveto{\pgfqpoint{0.651951in}{1.247394in}}{\pgfqpoint{0.648679in}{1.239494in}}{\pgfqpoint{0.648679in}{1.231258in}}%
\pgfpathcurveto{\pgfqpoint{0.648679in}{1.223021in}}{\pgfqpoint{0.651951in}{1.215121in}}{\pgfqpoint{0.657775in}{1.209297in}}%
\pgfpathcurveto{\pgfqpoint{0.663599in}{1.203473in}}{\pgfqpoint{0.671499in}{1.200201in}}{\pgfqpoint{0.679736in}{1.200201in}}%
\pgfpathclose%
\pgfusepath{stroke,fill}%
\end{pgfscope}%
\begin{pgfscope}%
\pgfpathrectangle{\pgfqpoint{0.100000in}{0.220728in}}{\pgfqpoint{3.696000in}{3.696000in}}%
\pgfusepath{clip}%
\pgfsetbuttcap%
\pgfsetroundjoin%
\definecolor{currentfill}{rgb}{0.121569,0.466667,0.705882}%
\pgfsetfillcolor{currentfill}%
\pgfsetfillopacity{0.619789}%
\pgfsetlinewidth{1.003750pt}%
\definecolor{currentstroke}{rgb}{0.121569,0.466667,0.705882}%
\pgfsetstrokecolor{currentstroke}%
\pgfsetstrokeopacity{0.619789}%
\pgfsetdash{}{0pt}%
\pgfpathmoveto{\pgfqpoint{0.674986in}{1.198287in}}%
\pgfpathcurveto{\pgfqpoint{0.683222in}{1.198287in}}{\pgfqpoint{0.691122in}{1.201560in}}{\pgfqpoint{0.696946in}{1.207384in}}%
\pgfpathcurveto{\pgfqpoint{0.702770in}{1.213208in}}{\pgfqpoint{0.706042in}{1.221108in}}{\pgfqpoint{0.706042in}{1.229344in}}%
\pgfpathcurveto{\pgfqpoint{0.706042in}{1.237580in}}{\pgfqpoint{0.702770in}{1.245480in}}{\pgfqpoint{0.696946in}{1.251304in}}%
\pgfpathcurveto{\pgfqpoint{0.691122in}{1.257128in}}{\pgfqpoint{0.683222in}{1.260400in}}{\pgfqpoint{0.674986in}{1.260400in}}%
\pgfpathcurveto{\pgfqpoint{0.666750in}{1.260400in}}{\pgfqpoint{0.658850in}{1.257128in}}{\pgfqpoint{0.653026in}{1.251304in}}%
\pgfpathcurveto{\pgfqpoint{0.647202in}{1.245480in}}{\pgfqpoint{0.643929in}{1.237580in}}{\pgfqpoint{0.643929in}{1.229344in}}%
\pgfpathcurveto{\pgfqpoint{0.643929in}{1.221108in}}{\pgfqpoint{0.647202in}{1.213208in}}{\pgfqpoint{0.653026in}{1.207384in}}%
\pgfpathcurveto{\pgfqpoint{0.658850in}{1.201560in}}{\pgfqpoint{0.666750in}{1.198287in}}{\pgfqpoint{0.674986in}{1.198287in}}%
\pgfpathclose%
\pgfusepath{stroke,fill}%
\end{pgfscope}%
\begin{pgfscope}%
\pgfpathrectangle{\pgfqpoint{0.100000in}{0.220728in}}{\pgfqpoint{3.696000in}{3.696000in}}%
\pgfusepath{clip}%
\pgfsetbuttcap%
\pgfsetroundjoin%
\definecolor{currentfill}{rgb}{0.121569,0.466667,0.705882}%
\pgfsetfillcolor{currentfill}%
\pgfsetfillopacity{0.619815}%
\pgfsetlinewidth{1.003750pt}%
\definecolor{currentstroke}{rgb}{0.121569,0.466667,0.705882}%
\pgfsetstrokecolor{currentstroke}%
\pgfsetstrokeopacity{0.619815}%
\pgfsetdash{}{0pt}%
\pgfpathmoveto{\pgfqpoint{0.675675in}{1.198466in}}%
\pgfpathcurveto{\pgfqpoint{0.683911in}{1.198466in}}{\pgfqpoint{0.691811in}{1.201738in}}{\pgfqpoint{0.697635in}{1.207562in}}%
\pgfpathcurveto{\pgfqpoint{0.703459in}{1.213386in}}{\pgfqpoint{0.706731in}{1.221286in}}{\pgfqpoint{0.706731in}{1.229523in}}%
\pgfpathcurveto{\pgfqpoint{0.706731in}{1.237759in}}{\pgfqpoint{0.703459in}{1.245659in}}{\pgfqpoint{0.697635in}{1.251483in}}%
\pgfpathcurveto{\pgfqpoint{0.691811in}{1.257307in}}{\pgfqpoint{0.683911in}{1.260579in}}{\pgfqpoint{0.675675in}{1.260579in}}%
\pgfpathcurveto{\pgfqpoint{0.667438in}{1.260579in}}{\pgfqpoint{0.659538in}{1.257307in}}{\pgfqpoint{0.653714in}{1.251483in}}%
\pgfpathcurveto{\pgfqpoint{0.647890in}{1.245659in}}{\pgfqpoint{0.644618in}{1.237759in}}{\pgfqpoint{0.644618in}{1.229523in}}%
\pgfpathcurveto{\pgfqpoint{0.644618in}{1.221286in}}{\pgfqpoint{0.647890in}{1.213386in}}{\pgfqpoint{0.653714in}{1.207562in}}%
\pgfpathcurveto{\pgfqpoint{0.659538in}{1.201738in}}{\pgfqpoint{0.667438in}{1.198466in}}{\pgfqpoint{0.675675in}{1.198466in}}%
\pgfpathclose%
\pgfusepath{stroke,fill}%
\end{pgfscope}%
\begin{pgfscope}%
\pgfpathrectangle{\pgfqpoint{0.100000in}{0.220728in}}{\pgfqpoint{3.696000in}{3.696000in}}%
\pgfusepath{clip}%
\pgfsetbuttcap%
\pgfsetroundjoin%
\definecolor{currentfill}{rgb}{0.121569,0.466667,0.705882}%
\pgfsetfillcolor{currentfill}%
\pgfsetfillopacity{0.619821}%
\pgfsetlinewidth{1.003750pt}%
\definecolor{currentstroke}{rgb}{0.121569,0.466667,0.705882}%
\pgfsetstrokecolor{currentstroke}%
\pgfsetstrokeopacity{0.619821}%
\pgfsetdash{}{0pt}%
\pgfpathmoveto{\pgfqpoint{0.679489in}{1.199936in}}%
\pgfpathcurveto{\pgfqpoint{0.687726in}{1.199936in}}{\pgfqpoint{0.695626in}{1.203209in}}{\pgfqpoint{0.701450in}{1.209032in}}%
\pgfpathcurveto{\pgfqpoint{0.707274in}{1.214856in}}{\pgfqpoint{0.710546in}{1.222756in}}{\pgfqpoint{0.710546in}{1.230993in}}%
\pgfpathcurveto{\pgfqpoint{0.710546in}{1.239229in}}{\pgfqpoint{0.707274in}{1.247129in}}{\pgfqpoint{0.701450in}{1.252953in}}%
\pgfpathcurveto{\pgfqpoint{0.695626in}{1.258777in}}{\pgfqpoint{0.687726in}{1.262049in}}{\pgfqpoint{0.679489in}{1.262049in}}%
\pgfpathcurveto{\pgfqpoint{0.671253in}{1.262049in}}{\pgfqpoint{0.663353in}{1.258777in}}{\pgfqpoint{0.657529in}{1.252953in}}%
\pgfpathcurveto{\pgfqpoint{0.651705in}{1.247129in}}{\pgfqpoint{0.648433in}{1.239229in}}{\pgfqpoint{0.648433in}{1.230993in}}%
\pgfpathcurveto{\pgfqpoint{0.648433in}{1.222756in}}{\pgfqpoint{0.651705in}{1.214856in}}{\pgfqpoint{0.657529in}{1.209032in}}%
\pgfpathcurveto{\pgfqpoint{0.663353in}{1.203209in}}{\pgfqpoint{0.671253in}{1.199936in}}{\pgfqpoint{0.679489in}{1.199936in}}%
\pgfpathclose%
\pgfusepath{stroke,fill}%
\end{pgfscope}%
\begin{pgfscope}%
\pgfpathrectangle{\pgfqpoint{0.100000in}{0.220728in}}{\pgfqpoint{3.696000in}{3.696000in}}%
\pgfusepath{clip}%
\pgfsetbuttcap%
\pgfsetroundjoin%
\definecolor{currentfill}{rgb}{0.121569,0.466667,0.705882}%
\pgfsetfillcolor{currentfill}%
\pgfsetfillopacity{0.619845}%
\pgfsetlinewidth{1.003750pt}%
\definecolor{currentstroke}{rgb}{0.121569,0.466667,0.705882}%
\pgfsetstrokecolor{currentstroke}%
\pgfsetstrokeopacity{0.619845}%
\pgfsetdash{}{0pt}%
\pgfpathmoveto{\pgfqpoint{0.679356in}{1.199796in}}%
\pgfpathcurveto{\pgfqpoint{0.687592in}{1.199796in}}{\pgfqpoint{0.695492in}{1.203068in}}{\pgfqpoint{0.701316in}{1.208892in}}%
\pgfpathcurveto{\pgfqpoint{0.707140in}{1.214716in}}{\pgfqpoint{0.710412in}{1.222616in}}{\pgfqpoint{0.710412in}{1.230853in}}%
\pgfpathcurveto{\pgfqpoint{0.710412in}{1.239089in}}{\pgfqpoint{0.707140in}{1.246989in}}{\pgfqpoint{0.701316in}{1.252813in}}%
\pgfpathcurveto{\pgfqpoint{0.695492in}{1.258637in}}{\pgfqpoint{0.687592in}{1.261909in}}{\pgfqpoint{0.679356in}{1.261909in}}%
\pgfpathcurveto{\pgfqpoint{0.671119in}{1.261909in}}{\pgfqpoint{0.663219in}{1.258637in}}{\pgfqpoint{0.657395in}{1.252813in}}%
\pgfpathcurveto{\pgfqpoint{0.651571in}{1.246989in}}{\pgfqpoint{0.648299in}{1.239089in}}{\pgfqpoint{0.648299in}{1.230853in}}%
\pgfpathcurveto{\pgfqpoint{0.648299in}{1.222616in}}{\pgfqpoint{0.651571in}{1.214716in}}{\pgfqpoint{0.657395in}{1.208892in}}%
\pgfpathcurveto{\pgfqpoint{0.663219in}{1.203068in}}{\pgfqpoint{0.671119in}{1.199796in}}{\pgfqpoint{0.679356in}{1.199796in}}%
\pgfpathclose%
\pgfusepath{stroke,fill}%
\end{pgfscope}%
\begin{pgfscope}%
\pgfpathrectangle{\pgfqpoint{0.100000in}{0.220728in}}{\pgfqpoint{3.696000in}{3.696000in}}%
\pgfusepath{clip}%
\pgfsetbuttcap%
\pgfsetroundjoin%
\definecolor{currentfill}{rgb}{0.121569,0.466667,0.705882}%
\pgfsetfillcolor{currentfill}%
\pgfsetfillopacity{0.619855}%
\pgfsetlinewidth{1.003750pt}%
\definecolor{currentstroke}{rgb}{0.121569,0.466667,0.705882}%
\pgfsetstrokecolor{currentstroke}%
\pgfsetstrokeopacity{0.619855}%
\pgfsetdash{}{0pt}%
\pgfpathmoveto{\pgfqpoint{0.676925in}{1.198831in}}%
\pgfpathcurveto{\pgfqpoint{0.685162in}{1.198831in}}{\pgfqpoint{0.693062in}{1.202103in}}{\pgfqpoint{0.698886in}{1.207927in}}%
\pgfpathcurveto{\pgfqpoint{0.704710in}{1.213751in}}{\pgfqpoint{0.707982in}{1.221651in}}{\pgfqpoint{0.707982in}{1.229887in}}%
\pgfpathcurveto{\pgfqpoint{0.707982in}{1.238124in}}{\pgfqpoint{0.704710in}{1.246024in}}{\pgfqpoint{0.698886in}{1.251848in}}%
\pgfpathcurveto{\pgfqpoint{0.693062in}{1.257672in}}{\pgfqpoint{0.685162in}{1.260944in}}{\pgfqpoint{0.676925in}{1.260944in}}%
\pgfpathcurveto{\pgfqpoint{0.668689in}{1.260944in}}{\pgfqpoint{0.660789in}{1.257672in}}{\pgfqpoint{0.654965in}{1.251848in}}%
\pgfpathcurveto{\pgfqpoint{0.649141in}{1.246024in}}{\pgfqpoint{0.645869in}{1.238124in}}{\pgfqpoint{0.645869in}{1.229887in}}%
\pgfpathcurveto{\pgfqpoint{0.645869in}{1.221651in}}{\pgfqpoint{0.649141in}{1.213751in}}{\pgfqpoint{0.654965in}{1.207927in}}%
\pgfpathcurveto{\pgfqpoint{0.660789in}{1.202103in}}{\pgfqpoint{0.668689in}{1.198831in}}{\pgfqpoint{0.676925in}{1.198831in}}%
\pgfpathclose%
\pgfusepath{stroke,fill}%
\end{pgfscope}%
\begin{pgfscope}%
\pgfpathrectangle{\pgfqpoint{0.100000in}{0.220728in}}{\pgfqpoint{3.696000in}{3.696000in}}%
\pgfusepath{clip}%
\pgfsetbuttcap%
\pgfsetroundjoin%
\definecolor{currentfill}{rgb}{0.121569,0.466667,0.705882}%
\pgfsetfillcolor{currentfill}%
\pgfsetfillopacity{0.619860}%
\pgfsetlinewidth{1.003750pt}%
\definecolor{currentstroke}{rgb}{0.121569,0.466667,0.705882}%
\pgfsetstrokecolor{currentstroke}%
\pgfsetstrokeopacity{0.619860}%
\pgfsetdash{}{0pt}%
\pgfpathmoveto{\pgfqpoint{0.679285in}{1.199719in}}%
\pgfpathcurveto{\pgfqpoint{0.687521in}{1.199719in}}{\pgfqpoint{0.695421in}{1.202992in}}{\pgfqpoint{0.701245in}{1.208816in}}%
\pgfpathcurveto{\pgfqpoint{0.707069in}{1.214640in}}{\pgfqpoint{0.710341in}{1.222540in}}{\pgfqpoint{0.710341in}{1.230776in}}%
\pgfpathcurveto{\pgfqpoint{0.710341in}{1.239012in}}{\pgfqpoint{0.707069in}{1.246912in}}{\pgfqpoint{0.701245in}{1.252736in}}%
\pgfpathcurveto{\pgfqpoint{0.695421in}{1.258560in}}{\pgfqpoint{0.687521in}{1.261832in}}{\pgfqpoint{0.679285in}{1.261832in}}%
\pgfpathcurveto{\pgfqpoint{0.671048in}{1.261832in}}{\pgfqpoint{0.663148in}{1.258560in}}{\pgfqpoint{0.657324in}{1.252736in}}%
\pgfpathcurveto{\pgfqpoint{0.651500in}{1.246912in}}{\pgfqpoint{0.648228in}{1.239012in}}{\pgfqpoint{0.648228in}{1.230776in}}%
\pgfpathcurveto{\pgfqpoint{0.648228in}{1.222540in}}{\pgfqpoint{0.651500in}{1.214640in}}{\pgfqpoint{0.657324in}{1.208816in}}%
\pgfpathcurveto{\pgfqpoint{0.663148in}{1.202992in}}{\pgfqpoint{0.671048in}{1.199719in}}{\pgfqpoint{0.679285in}{1.199719in}}%
\pgfpathclose%
\pgfusepath{stroke,fill}%
\end{pgfscope}%
\begin{pgfscope}%
\pgfpathrectangle{\pgfqpoint{0.100000in}{0.220728in}}{\pgfqpoint{3.696000in}{3.696000in}}%
\pgfusepath{clip}%
\pgfsetbuttcap%
\pgfsetroundjoin%
\definecolor{currentfill}{rgb}{0.121569,0.466667,0.705882}%
\pgfsetfillcolor{currentfill}%
\pgfsetfillopacity{0.619868}%
\pgfsetlinewidth{1.003750pt}%
\definecolor{currentstroke}{rgb}{0.121569,0.466667,0.705882}%
\pgfsetstrokecolor{currentstroke}%
\pgfsetstrokeopacity{0.619868}%
\pgfsetdash{}{0pt}%
\pgfpathmoveto{\pgfqpoint{0.679246in}{1.199675in}}%
\pgfpathcurveto{\pgfqpoint{0.687482in}{1.199675in}}{\pgfqpoint{0.695382in}{1.202947in}}{\pgfqpoint{0.701206in}{1.208771in}}%
\pgfpathcurveto{\pgfqpoint{0.707030in}{1.214595in}}{\pgfqpoint{0.710303in}{1.222495in}}{\pgfqpoint{0.710303in}{1.230732in}}%
\pgfpathcurveto{\pgfqpoint{0.710303in}{1.238968in}}{\pgfqpoint{0.707030in}{1.246868in}}{\pgfqpoint{0.701206in}{1.252692in}}%
\pgfpathcurveto{\pgfqpoint{0.695382in}{1.258516in}}{\pgfqpoint{0.687482in}{1.261788in}}{\pgfqpoint{0.679246in}{1.261788in}}%
\pgfpathcurveto{\pgfqpoint{0.671010in}{1.261788in}}{\pgfqpoint{0.663110in}{1.258516in}}{\pgfqpoint{0.657286in}{1.252692in}}%
\pgfpathcurveto{\pgfqpoint{0.651462in}{1.246868in}}{\pgfqpoint{0.648190in}{1.238968in}}{\pgfqpoint{0.648190in}{1.230732in}}%
\pgfpathcurveto{\pgfqpoint{0.648190in}{1.222495in}}{\pgfqpoint{0.651462in}{1.214595in}}{\pgfqpoint{0.657286in}{1.208771in}}%
\pgfpathcurveto{\pgfqpoint{0.663110in}{1.202947in}}{\pgfqpoint{0.671010in}{1.199675in}}{\pgfqpoint{0.679246in}{1.199675in}}%
\pgfpathclose%
\pgfusepath{stroke,fill}%
\end{pgfscope}%
\begin{pgfscope}%
\pgfpathrectangle{\pgfqpoint{0.100000in}{0.220728in}}{\pgfqpoint{3.696000in}{3.696000in}}%
\pgfusepath{clip}%
\pgfsetbuttcap%
\pgfsetroundjoin%
\definecolor{currentfill}{rgb}{0.121569,0.466667,0.705882}%
\pgfsetfillcolor{currentfill}%
\pgfsetfillopacity{0.619872}%
\pgfsetlinewidth{1.003750pt}%
\definecolor{currentstroke}{rgb}{0.121569,0.466667,0.705882}%
\pgfsetstrokecolor{currentstroke}%
\pgfsetstrokeopacity{0.619872}%
\pgfsetdash{}{0pt}%
\pgfpathmoveto{\pgfqpoint{0.679225in}{1.199650in}}%
\pgfpathcurveto{\pgfqpoint{0.687461in}{1.199650in}}{\pgfqpoint{0.695361in}{1.202923in}}{\pgfqpoint{0.701185in}{1.208747in}}%
\pgfpathcurveto{\pgfqpoint{0.707009in}{1.214570in}}{\pgfqpoint{0.710281in}{1.222471in}}{\pgfqpoint{0.710281in}{1.230707in}}%
\pgfpathcurveto{\pgfqpoint{0.710281in}{1.238943in}}{\pgfqpoint{0.707009in}{1.246843in}}{\pgfqpoint{0.701185in}{1.252667in}}%
\pgfpathcurveto{\pgfqpoint{0.695361in}{1.258491in}}{\pgfqpoint{0.687461in}{1.261763in}}{\pgfqpoint{0.679225in}{1.261763in}}%
\pgfpathcurveto{\pgfqpoint{0.670989in}{1.261763in}}{\pgfqpoint{0.663089in}{1.258491in}}{\pgfqpoint{0.657265in}{1.252667in}}%
\pgfpathcurveto{\pgfqpoint{0.651441in}{1.246843in}}{\pgfqpoint{0.648168in}{1.238943in}}{\pgfqpoint{0.648168in}{1.230707in}}%
\pgfpathcurveto{\pgfqpoint{0.648168in}{1.222471in}}{\pgfqpoint{0.651441in}{1.214570in}}{\pgfqpoint{0.657265in}{1.208747in}}%
\pgfpathcurveto{\pgfqpoint{0.663089in}{1.202923in}}{\pgfqpoint{0.670989in}{1.199650in}}{\pgfqpoint{0.679225in}{1.199650in}}%
\pgfpathclose%
\pgfusepath{stroke,fill}%
\end{pgfscope}%
\begin{pgfscope}%
\pgfpathrectangle{\pgfqpoint{0.100000in}{0.220728in}}{\pgfqpoint{3.696000in}{3.696000in}}%
\pgfusepath{clip}%
\pgfsetbuttcap%
\pgfsetroundjoin%
\definecolor{currentfill}{rgb}{0.121569,0.466667,0.705882}%
\pgfsetfillcolor{currentfill}%
\pgfsetfillopacity{0.619875}%
\pgfsetlinewidth{1.003750pt}%
\definecolor{currentstroke}{rgb}{0.121569,0.466667,0.705882}%
\pgfsetstrokecolor{currentstroke}%
\pgfsetstrokeopacity{0.619875}%
\pgfsetdash{}{0pt}%
\pgfpathmoveto{\pgfqpoint{0.679213in}{1.199637in}}%
\pgfpathcurveto{\pgfqpoint{0.687450in}{1.199637in}}{\pgfqpoint{0.695350in}{1.202909in}}{\pgfqpoint{0.701174in}{1.208733in}}%
\pgfpathcurveto{\pgfqpoint{0.706998in}{1.214557in}}{\pgfqpoint{0.710270in}{1.222457in}}{\pgfqpoint{0.710270in}{1.230694in}}%
\pgfpathcurveto{\pgfqpoint{0.710270in}{1.238930in}}{\pgfqpoint{0.706998in}{1.246830in}}{\pgfqpoint{0.701174in}{1.252654in}}%
\pgfpathcurveto{\pgfqpoint{0.695350in}{1.258478in}}{\pgfqpoint{0.687450in}{1.261750in}}{\pgfqpoint{0.679213in}{1.261750in}}%
\pgfpathcurveto{\pgfqpoint{0.670977in}{1.261750in}}{\pgfqpoint{0.663077in}{1.258478in}}{\pgfqpoint{0.657253in}{1.252654in}}%
\pgfpathcurveto{\pgfqpoint{0.651429in}{1.246830in}}{\pgfqpoint{0.648157in}{1.238930in}}{\pgfqpoint{0.648157in}{1.230694in}}%
\pgfpathcurveto{\pgfqpoint{0.648157in}{1.222457in}}{\pgfqpoint{0.651429in}{1.214557in}}{\pgfqpoint{0.657253in}{1.208733in}}%
\pgfpathcurveto{\pgfqpoint{0.663077in}{1.202909in}}{\pgfqpoint{0.670977in}{1.199637in}}{\pgfqpoint{0.679213in}{1.199637in}}%
\pgfpathclose%
\pgfusepath{stroke,fill}%
\end{pgfscope}%
\begin{pgfscope}%
\pgfpathrectangle{\pgfqpoint{0.100000in}{0.220728in}}{\pgfqpoint{3.696000in}{3.696000in}}%
\pgfusepath{clip}%
\pgfsetbuttcap%
\pgfsetroundjoin%
\definecolor{currentfill}{rgb}{0.121569,0.466667,0.705882}%
\pgfsetfillcolor{currentfill}%
\pgfsetfillopacity{0.619876}%
\pgfsetlinewidth{1.003750pt}%
\definecolor{currentstroke}{rgb}{0.121569,0.466667,0.705882}%
\pgfsetstrokecolor{currentstroke}%
\pgfsetstrokeopacity{0.619876}%
\pgfsetdash{}{0pt}%
\pgfpathmoveto{\pgfqpoint{0.679207in}{1.199630in}}%
\pgfpathcurveto{\pgfqpoint{0.687443in}{1.199630in}}{\pgfqpoint{0.695343in}{1.202902in}}{\pgfqpoint{0.701167in}{1.208726in}}%
\pgfpathcurveto{\pgfqpoint{0.706991in}{1.214550in}}{\pgfqpoint{0.710263in}{1.222450in}}{\pgfqpoint{0.710263in}{1.230686in}}%
\pgfpathcurveto{\pgfqpoint{0.710263in}{1.238923in}}{\pgfqpoint{0.706991in}{1.246823in}}{\pgfqpoint{0.701167in}{1.252647in}}%
\pgfpathcurveto{\pgfqpoint{0.695343in}{1.258471in}}{\pgfqpoint{0.687443in}{1.261743in}}{\pgfqpoint{0.679207in}{1.261743in}}%
\pgfpathcurveto{\pgfqpoint{0.670971in}{1.261743in}}{\pgfqpoint{0.663071in}{1.258471in}}{\pgfqpoint{0.657247in}{1.252647in}}%
\pgfpathcurveto{\pgfqpoint{0.651423in}{1.246823in}}{\pgfqpoint{0.648150in}{1.238923in}}{\pgfqpoint{0.648150in}{1.230686in}}%
\pgfpathcurveto{\pgfqpoint{0.648150in}{1.222450in}}{\pgfqpoint{0.651423in}{1.214550in}}{\pgfqpoint{0.657247in}{1.208726in}}%
\pgfpathcurveto{\pgfqpoint{0.663071in}{1.202902in}}{\pgfqpoint{0.670971in}{1.199630in}}{\pgfqpoint{0.679207in}{1.199630in}}%
\pgfpathclose%
\pgfusepath{stroke,fill}%
\end{pgfscope}%
\begin{pgfscope}%
\pgfpathrectangle{\pgfqpoint{0.100000in}{0.220728in}}{\pgfqpoint{3.696000in}{3.696000in}}%
\pgfusepath{clip}%
\pgfsetbuttcap%
\pgfsetroundjoin%
\definecolor{currentfill}{rgb}{0.121569,0.466667,0.705882}%
\pgfsetfillcolor{currentfill}%
\pgfsetfillopacity{0.619877}%
\pgfsetlinewidth{1.003750pt}%
\definecolor{currentstroke}{rgb}{0.121569,0.466667,0.705882}%
\pgfsetstrokecolor{currentstroke}%
\pgfsetstrokeopacity{0.619877}%
\pgfsetdash{}{0pt}%
\pgfpathmoveto{\pgfqpoint{0.679203in}{1.199626in}}%
\pgfpathcurveto{\pgfqpoint{0.687440in}{1.199626in}}{\pgfqpoint{0.695340in}{1.202898in}}{\pgfqpoint{0.701164in}{1.208722in}}%
\pgfpathcurveto{\pgfqpoint{0.706988in}{1.214546in}}{\pgfqpoint{0.710260in}{1.222446in}}{\pgfqpoint{0.710260in}{1.230682in}}%
\pgfpathcurveto{\pgfqpoint{0.710260in}{1.238919in}}{\pgfqpoint{0.706988in}{1.246819in}}{\pgfqpoint{0.701164in}{1.252643in}}%
\pgfpathcurveto{\pgfqpoint{0.695340in}{1.258466in}}{\pgfqpoint{0.687440in}{1.261739in}}{\pgfqpoint{0.679203in}{1.261739in}}%
\pgfpathcurveto{\pgfqpoint{0.670967in}{1.261739in}}{\pgfqpoint{0.663067in}{1.258466in}}{\pgfqpoint{0.657243in}{1.252643in}}%
\pgfpathcurveto{\pgfqpoint{0.651419in}{1.246819in}}{\pgfqpoint{0.648147in}{1.238919in}}{\pgfqpoint{0.648147in}{1.230682in}}%
\pgfpathcurveto{\pgfqpoint{0.648147in}{1.222446in}}{\pgfqpoint{0.651419in}{1.214546in}}{\pgfqpoint{0.657243in}{1.208722in}}%
\pgfpathcurveto{\pgfqpoint{0.663067in}{1.202898in}}{\pgfqpoint{0.670967in}{1.199626in}}{\pgfqpoint{0.679203in}{1.199626in}}%
\pgfpathclose%
\pgfusepath{stroke,fill}%
\end{pgfscope}%
\begin{pgfscope}%
\pgfpathrectangle{\pgfqpoint{0.100000in}{0.220728in}}{\pgfqpoint{3.696000in}{3.696000in}}%
\pgfusepath{clip}%
\pgfsetbuttcap%
\pgfsetroundjoin%
\definecolor{currentfill}{rgb}{0.121569,0.466667,0.705882}%
\pgfsetfillcolor{currentfill}%
\pgfsetfillopacity{0.619877}%
\pgfsetlinewidth{1.003750pt}%
\definecolor{currentstroke}{rgb}{0.121569,0.466667,0.705882}%
\pgfsetstrokecolor{currentstroke}%
\pgfsetstrokeopacity{0.619877}%
\pgfsetdash{}{0pt}%
\pgfpathmoveto{\pgfqpoint{0.679201in}{1.199624in}}%
\pgfpathcurveto{\pgfqpoint{0.687438in}{1.199624in}}{\pgfqpoint{0.695338in}{1.202896in}}{\pgfqpoint{0.701162in}{1.208720in}}%
\pgfpathcurveto{\pgfqpoint{0.706986in}{1.214544in}}{\pgfqpoint{0.710258in}{1.222444in}}{\pgfqpoint{0.710258in}{1.230680in}}%
\pgfpathcurveto{\pgfqpoint{0.710258in}{1.238916in}}{\pgfqpoint{0.706986in}{1.246816in}}{\pgfqpoint{0.701162in}{1.252640in}}%
\pgfpathcurveto{\pgfqpoint{0.695338in}{1.258464in}}{\pgfqpoint{0.687438in}{1.261737in}}{\pgfqpoint{0.679201in}{1.261737in}}%
\pgfpathcurveto{\pgfqpoint{0.670965in}{1.261737in}}{\pgfqpoint{0.663065in}{1.258464in}}{\pgfqpoint{0.657241in}{1.252640in}}%
\pgfpathcurveto{\pgfqpoint{0.651417in}{1.246816in}}{\pgfqpoint{0.648145in}{1.238916in}}{\pgfqpoint{0.648145in}{1.230680in}}%
\pgfpathcurveto{\pgfqpoint{0.648145in}{1.222444in}}{\pgfqpoint{0.651417in}{1.214544in}}{\pgfqpoint{0.657241in}{1.208720in}}%
\pgfpathcurveto{\pgfqpoint{0.663065in}{1.202896in}}{\pgfqpoint{0.670965in}{1.199624in}}{\pgfqpoint{0.679201in}{1.199624in}}%
\pgfpathclose%
\pgfusepath{stroke,fill}%
\end{pgfscope}%
\begin{pgfscope}%
\pgfpathrectangle{\pgfqpoint{0.100000in}{0.220728in}}{\pgfqpoint{3.696000in}{3.696000in}}%
\pgfusepath{clip}%
\pgfsetbuttcap%
\pgfsetroundjoin%
\definecolor{currentfill}{rgb}{0.121569,0.466667,0.705882}%
\pgfsetfillcolor{currentfill}%
\pgfsetfillopacity{0.619877}%
\pgfsetlinewidth{1.003750pt}%
\definecolor{currentstroke}{rgb}{0.121569,0.466667,0.705882}%
\pgfsetstrokecolor{currentstroke}%
\pgfsetstrokeopacity{0.619877}%
\pgfsetdash{}{0pt}%
\pgfpathmoveto{\pgfqpoint{0.679200in}{1.199622in}}%
\pgfpathcurveto{\pgfqpoint{0.687437in}{1.199622in}}{\pgfqpoint{0.695337in}{1.202895in}}{\pgfqpoint{0.701161in}{1.208719in}}%
\pgfpathcurveto{\pgfqpoint{0.706985in}{1.214543in}}{\pgfqpoint{0.710257in}{1.222443in}}{\pgfqpoint{0.710257in}{1.230679in}}%
\pgfpathcurveto{\pgfqpoint{0.710257in}{1.238915in}}{\pgfqpoint{0.706985in}{1.246815in}}{\pgfqpoint{0.701161in}{1.252639in}}%
\pgfpathcurveto{\pgfqpoint{0.695337in}{1.258463in}}{\pgfqpoint{0.687437in}{1.261735in}}{\pgfqpoint{0.679200in}{1.261735in}}%
\pgfpathcurveto{\pgfqpoint{0.670964in}{1.261735in}}{\pgfqpoint{0.663064in}{1.258463in}}{\pgfqpoint{0.657240in}{1.252639in}}%
\pgfpathcurveto{\pgfqpoint{0.651416in}{1.246815in}}{\pgfqpoint{0.648144in}{1.238915in}}{\pgfqpoint{0.648144in}{1.230679in}}%
\pgfpathcurveto{\pgfqpoint{0.648144in}{1.222443in}}{\pgfqpoint{0.651416in}{1.214543in}}{\pgfqpoint{0.657240in}{1.208719in}}%
\pgfpathcurveto{\pgfqpoint{0.663064in}{1.202895in}}{\pgfqpoint{0.670964in}{1.199622in}}{\pgfqpoint{0.679200in}{1.199622in}}%
\pgfpathclose%
\pgfusepath{stroke,fill}%
\end{pgfscope}%
\begin{pgfscope}%
\pgfpathrectangle{\pgfqpoint{0.100000in}{0.220728in}}{\pgfqpoint{3.696000in}{3.696000in}}%
\pgfusepath{clip}%
\pgfsetbuttcap%
\pgfsetroundjoin%
\definecolor{currentfill}{rgb}{0.121569,0.466667,0.705882}%
\pgfsetfillcolor{currentfill}%
\pgfsetfillopacity{0.619877}%
\pgfsetlinewidth{1.003750pt}%
\definecolor{currentstroke}{rgb}{0.121569,0.466667,0.705882}%
\pgfsetstrokecolor{currentstroke}%
\pgfsetstrokeopacity{0.619877}%
\pgfsetdash{}{0pt}%
\pgfpathmoveto{\pgfqpoint{0.679200in}{1.199622in}}%
\pgfpathcurveto{\pgfqpoint{0.687436in}{1.199622in}}{\pgfqpoint{0.695336in}{1.202894in}}{\pgfqpoint{0.701160in}{1.208718in}}%
\pgfpathcurveto{\pgfqpoint{0.706984in}{1.214542in}}{\pgfqpoint{0.710256in}{1.222442in}}{\pgfqpoint{0.710256in}{1.230678in}}%
\pgfpathcurveto{\pgfqpoint{0.710256in}{1.238914in}}{\pgfqpoint{0.706984in}{1.246815in}}{\pgfqpoint{0.701160in}{1.252638in}}%
\pgfpathcurveto{\pgfqpoint{0.695336in}{1.258462in}}{\pgfqpoint{0.687436in}{1.261735in}}{\pgfqpoint{0.679200in}{1.261735in}}%
\pgfpathcurveto{\pgfqpoint{0.670963in}{1.261735in}}{\pgfqpoint{0.663063in}{1.258462in}}{\pgfqpoint{0.657239in}{1.252638in}}%
\pgfpathcurveto{\pgfqpoint{0.651416in}{1.246815in}}{\pgfqpoint{0.648143in}{1.238914in}}{\pgfqpoint{0.648143in}{1.230678in}}%
\pgfpathcurveto{\pgfqpoint{0.648143in}{1.222442in}}{\pgfqpoint{0.651416in}{1.214542in}}{\pgfqpoint{0.657239in}{1.208718in}}%
\pgfpathcurveto{\pgfqpoint{0.663063in}{1.202894in}}{\pgfqpoint{0.670963in}{1.199622in}}{\pgfqpoint{0.679200in}{1.199622in}}%
\pgfpathclose%
\pgfusepath{stroke,fill}%
\end{pgfscope}%
\begin{pgfscope}%
\pgfpathrectangle{\pgfqpoint{0.100000in}{0.220728in}}{\pgfqpoint{3.696000in}{3.696000in}}%
\pgfusepath{clip}%
\pgfsetbuttcap%
\pgfsetroundjoin%
\definecolor{currentfill}{rgb}{0.121569,0.466667,0.705882}%
\pgfsetfillcolor{currentfill}%
\pgfsetfillopacity{0.619877}%
\pgfsetlinewidth{1.003750pt}%
\definecolor{currentstroke}{rgb}{0.121569,0.466667,0.705882}%
\pgfsetstrokecolor{currentstroke}%
\pgfsetstrokeopacity{0.619877}%
\pgfsetdash{}{0pt}%
\pgfpathmoveto{\pgfqpoint{0.679199in}{1.199621in}}%
\pgfpathcurveto{\pgfqpoint{0.687436in}{1.199621in}}{\pgfqpoint{0.695336in}{1.202894in}}{\pgfqpoint{0.701160in}{1.208718in}}%
\pgfpathcurveto{\pgfqpoint{0.706984in}{1.214542in}}{\pgfqpoint{0.710256in}{1.222442in}}{\pgfqpoint{0.710256in}{1.230678in}}%
\pgfpathcurveto{\pgfqpoint{0.710256in}{1.238914in}}{\pgfqpoint{0.706984in}{1.246814in}}{\pgfqpoint{0.701160in}{1.252638in}}%
\pgfpathcurveto{\pgfqpoint{0.695336in}{1.258462in}}{\pgfqpoint{0.687436in}{1.261734in}}{\pgfqpoint{0.679199in}{1.261734in}}%
\pgfpathcurveto{\pgfqpoint{0.670963in}{1.261734in}}{\pgfqpoint{0.663063in}{1.258462in}}{\pgfqpoint{0.657239in}{1.252638in}}%
\pgfpathcurveto{\pgfqpoint{0.651415in}{1.246814in}}{\pgfqpoint{0.648143in}{1.238914in}}{\pgfqpoint{0.648143in}{1.230678in}}%
\pgfpathcurveto{\pgfqpoint{0.648143in}{1.222442in}}{\pgfqpoint{0.651415in}{1.214542in}}{\pgfqpoint{0.657239in}{1.208718in}}%
\pgfpathcurveto{\pgfqpoint{0.663063in}{1.202894in}}{\pgfqpoint{0.670963in}{1.199621in}}{\pgfqpoint{0.679199in}{1.199621in}}%
\pgfpathclose%
\pgfusepath{stroke,fill}%
\end{pgfscope}%
\begin{pgfscope}%
\pgfpathrectangle{\pgfqpoint{0.100000in}{0.220728in}}{\pgfqpoint{3.696000in}{3.696000in}}%
\pgfusepath{clip}%
\pgfsetbuttcap%
\pgfsetroundjoin%
\definecolor{currentfill}{rgb}{0.121569,0.466667,0.705882}%
\pgfsetfillcolor{currentfill}%
\pgfsetfillopacity{0.619877}%
\pgfsetlinewidth{1.003750pt}%
\definecolor{currentstroke}{rgb}{0.121569,0.466667,0.705882}%
\pgfsetstrokecolor{currentstroke}%
\pgfsetstrokeopacity{0.619877}%
\pgfsetdash{}{0pt}%
\pgfpathmoveto{\pgfqpoint{0.679199in}{1.199621in}}%
\pgfpathcurveto{\pgfqpoint{0.687435in}{1.199621in}}{\pgfqpoint{0.695336in}{1.202893in}}{\pgfqpoint{0.701159in}{1.208717in}}%
\pgfpathcurveto{\pgfqpoint{0.706983in}{1.214541in}}{\pgfqpoint{0.710256in}{1.222441in}}{\pgfqpoint{0.710256in}{1.230678in}}%
\pgfpathcurveto{\pgfqpoint{0.710256in}{1.238914in}}{\pgfqpoint{0.706983in}{1.246814in}}{\pgfqpoint{0.701159in}{1.252638in}}%
\pgfpathcurveto{\pgfqpoint{0.695336in}{1.258462in}}{\pgfqpoint{0.687435in}{1.261734in}}{\pgfqpoint{0.679199in}{1.261734in}}%
\pgfpathcurveto{\pgfqpoint{0.670963in}{1.261734in}}{\pgfqpoint{0.663063in}{1.258462in}}{\pgfqpoint{0.657239in}{1.252638in}}%
\pgfpathcurveto{\pgfqpoint{0.651415in}{1.246814in}}{\pgfqpoint{0.648143in}{1.238914in}}{\pgfqpoint{0.648143in}{1.230678in}}%
\pgfpathcurveto{\pgfqpoint{0.648143in}{1.222441in}}{\pgfqpoint{0.651415in}{1.214541in}}{\pgfqpoint{0.657239in}{1.208717in}}%
\pgfpathcurveto{\pgfqpoint{0.663063in}{1.202893in}}{\pgfqpoint{0.670963in}{1.199621in}}{\pgfqpoint{0.679199in}{1.199621in}}%
\pgfpathclose%
\pgfusepath{stroke,fill}%
\end{pgfscope}%
\begin{pgfscope}%
\pgfpathrectangle{\pgfqpoint{0.100000in}{0.220728in}}{\pgfqpoint{3.696000in}{3.696000in}}%
\pgfusepath{clip}%
\pgfsetbuttcap%
\pgfsetroundjoin%
\definecolor{currentfill}{rgb}{0.121569,0.466667,0.705882}%
\pgfsetfillcolor{currentfill}%
\pgfsetfillopacity{0.619877}%
\pgfsetlinewidth{1.003750pt}%
\definecolor{currentstroke}{rgb}{0.121569,0.466667,0.705882}%
\pgfsetstrokecolor{currentstroke}%
\pgfsetstrokeopacity{0.619877}%
\pgfsetdash{}{0pt}%
\pgfpathmoveto{\pgfqpoint{0.679199in}{1.199621in}}%
\pgfpathcurveto{\pgfqpoint{0.687435in}{1.199621in}}{\pgfqpoint{0.695335in}{1.202893in}}{\pgfqpoint{0.701159in}{1.208717in}}%
\pgfpathcurveto{\pgfqpoint{0.706983in}{1.214541in}}{\pgfqpoint{0.710256in}{1.222441in}}{\pgfqpoint{0.710256in}{1.230678in}}%
\pgfpathcurveto{\pgfqpoint{0.710256in}{1.238914in}}{\pgfqpoint{0.706983in}{1.246814in}}{\pgfqpoint{0.701159in}{1.252638in}}%
\pgfpathcurveto{\pgfqpoint{0.695335in}{1.258462in}}{\pgfqpoint{0.687435in}{1.261734in}}{\pgfqpoint{0.679199in}{1.261734in}}%
\pgfpathcurveto{\pgfqpoint{0.670963in}{1.261734in}}{\pgfqpoint{0.663063in}{1.258462in}}{\pgfqpoint{0.657239in}{1.252638in}}%
\pgfpathcurveto{\pgfqpoint{0.651415in}{1.246814in}}{\pgfqpoint{0.648143in}{1.238914in}}{\pgfqpoint{0.648143in}{1.230678in}}%
\pgfpathcurveto{\pgfqpoint{0.648143in}{1.222441in}}{\pgfqpoint{0.651415in}{1.214541in}}{\pgfqpoint{0.657239in}{1.208717in}}%
\pgfpathcurveto{\pgfqpoint{0.663063in}{1.202893in}}{\pgfqpoint{0.670963in}{1.199621in}}{\pgfqpoint{0.679199in}{1.199621in}}%
\pgfpathclose%
\pgfusepath{stroke,fill}%
\end{pgfscope}%
\begin{pgfscope}%
\pgfpathrectangle{\pgfqpoint{0.100000in}{0.220728in}}{\pgfqpoint{3.696000in}{3.696000in}}%
\pgfusepath{clip}%
\pgfsetbuttcap%
\pgfsetroundjoin%
\definecolor{currentfill}{rgb}{0.121569,0.466667,0.705882}%
\pgfsetfillcolor{currentfill}%
\pgfsetfillopacity{0.619877}%
\pgfsetlinewidth{1.003750pt}%
\definecolor{currentstroke}{rgb}{0.121569,0.466667,0.705882}%
\pgfsetstrokecolor{currentstroke}%
\pgfsetstrokeopacity{0.619877}%
\pgfsetdash{}{0pt}%
\pgfpathmoveto{\pgfqpoint{0.679199in}{1.199621in}}%
\pgfpathcurveto{\pgfqpoint{0.687435in}{1.199621in}}{\pgfqpoint{0.695335in}{1.202893in}}{\pgfqpoint{0.701159in}{1.208717in}}%
\pgfpathcurveto{\pgfqpoint{0.706983in}{1.214541in}}{\pgfqpoint{0.710256in}{1.222441in}}{\pgfqpoint{0.710256in}{1.230678in}}%
\pgfpathcurveto{\pgfqpoint{0.710256in}{1.238914in}}{\pgfqpoint{0.706983in}{1.246814in}}{\pgfqpoint{0.701159in}{1.252638in}}%
\pgfpathcurveto{\pgfqpoint{0.695335in}{1.258462in}}{\pgfqpoint{0.687435in}{1.261734in}}{\pgfqpoint{0.679199in}{1.261734in}}%
\pgfpathcurveto{\pgfqpoint{0.670963in}{1.261734in}}{\pgfqpoint{0.663063in}{1.258462in}}{\pgfqpoint{0.657239in}{1.252638in}}%
\pgfpathcurveto{\pgfqpoint{0.651415in}{1.246814in}}{\pgfqpoint{0.648143in}{1.238914in}}{\pgfqpoint{0.648143in}{1.230678in}}%
\pgfpathcurveto{\pgfqpoint{0.648143in}{1.222441in}}{\pgfqpoint{0.651415in}{1.214541in}}{\pgfqpoint{0.657239in}{1.208717in}}%
\pgfpathcurveto{\pgfqpoint{0.663063in}{1.202893in}}{\pgfqpoint{0.670963in}{1.199621in}}{\pgfqpoint{0.679199in}{1.199621in}}%
\pgfpathclose%
\pgfusepath{stroke,fill}%
\end{pgfscope}%
\begin{pgfscope}%
\pgfpathrectangle{\pgfqpoint{0.100000in}{0.220728in}}{\pgfqpoint{3.696000in}{3.696000in}}%
\pgfusepath{clip}%
\pgfsetbuttcap%
\pgfsetroundjoin%
\definecolor{currentfill}{rgb}{0.121569,0.466667,0.705882}%
\pgfsetfillcolor{currentfill}%
\pgfsetfillopacity{0.619877}%
\pgfsetlinewidth{1.003750pt}%
\definecolor{currentstroke}{rgb}{0.121569,0.466667,0.705882}%
\pgfsetstrokecolor{currentstroke}%
\pgfsetstrokeopacity{0.619877}%
\pgfsetdash{}{0pt}%
\pgfpathmoveto{\pgfqpoint{0.679199in}{1.199621in}}%
\pgfpathcurveto{\pgfqpoint{0.687435in}{1.199621in}}{\pgfqpoint{0.695335in}{1.202893in}}{\pgfqpoint{0.701159in}{1.208717in}}%
\pgfpathcurveto{\pgfqpoint{0.706983in}{1.214541in}}{\pgfqpoint{0.710256in}{1.222441in}}{\pgfqpoint{0.710256in}{1.230677in}}%
\pgfpathcurveto{\pgfqpoint{0.710256in}{1.238914in}}{\pgfqpoint{0.706983in}{1.246814in}}{\pgfqpoint{0.701159in}{1.252638in}}%
\pgfpathcurveto{\pgfqpoint{0.695335in}{1.258462in}}{\pgfqpoint{0.687435in}{1.261734in}}{\pgfqpoint{0.679199in}{1.261734in}}%
\pgfpathcurveto{\pgfqpoint{0.670963in}{1.261734in}}{\pgfqpoint{0.663063in}{1.258462in}}{\pgfqpoint{0.657239in}{1.252638in}}%
\pgfpathcurveto{\pgfqpoint{0.651415in}{1.246814in}}{\pgfqpoint{0.648143in}{1.238914in}}{\pgfqpoint{0.648143in}{1.230677in}}%
\pgfpathcurveto{\pgfqpoint{0.648143in}{1.222441in}}{\pgfqpoint{0.651415in}{1.214541in}}{\pgfqpoint{0.657239in}{1.208717in}}%
\pgfpathcurveto{\pgfqpoint{0.663063in}{1.202893in}}{\pgfqpoint{0.670963in}{1.199621in}}{\pgfqpoint{0.679199in}{1.199621in}}%
\pgfpathclose%
\pgfusepath{stroke,fill}%
\end{pgfscope}%
\begin{pgfscope}%
\pgfpathrectangle{\pgfqpoint{0.100000in}{0.220728in}}{\pgfqpoint{3.696000in}{3.696000in}}%
\pgfusepath{clip}%
\pgfsetbuttcap%
\pgfsetroundjoin%
\definecolor{currentfill}{rgb}{0.121569,0.466667,0.705882}%
\pgfsetfillcolor{currentfill}%
\pgfsetfillopacity{0.619877}%
\pgfsetlinewidth{1.003750pt}%
\definecolor{currentstroke}{rgb}{0.121569,0.466667,0.705882}%
\pgfsetstrokecolor{currentstroke}%
\pgfsetstrokeopacity{0.619877}%
\pgfsetdash{}{0pt}%
\pgfpathmoveto{\pgfqpoint{0.679199in}{1.199621in}}%
\pgfpathcurveto{\pgfqpoint{0.687435in}{1.199621in}}{\pgfqpoint{0.695335in}{1.202893in}}{\pgfqpoint{0.701159in}{1.208717in}}%
\pgfpathcurveto{\pgfqpoint{0.706983in}{1.214541in}}{\pgfqpoint{0.710256in}{1.222441in}}{\pgfqpoint{0.710256in}{1.230677in}}%
\pgfpathcurveto{\pgfqpoint{0.710256in}{1.238914in}}{\pgfqpoint{0.706983in}{1.246814in}}{\pgfqpoint{0.701159in}{1.252638in}}%
\pgfpathcurveto{\pgfqpoint{0.695335in}{1.258462in}}{\pgfqpoint{0.687435in}{1.261734in}}{\pgfqpoint{0.679199in}{1.261734in}}%
\pgfpathcurveto{\pgfqpoint{0.670963in}{1.261734in}}{\pgfqpoint{0.663063in}{1.258462in}}{\pgfqpoint{0.657239in}{1.252638in}}%
\pgfpathcurveto{\pgfqpoint{0.651415in}{1.246814in}}{\pgfqpoint{0.648143in}{1.238914in}}{\pgfqpoint{0.648143in}{1.230677in}}%
\pgfpathcurveto{\pgfqpoint{0.648143in}{1.222441in}}{\pgfqpoint{0.651415in}{1.214541in}}{\pgfqpoint{0.657239in}{1.208717in}}%
\pgfpathcurveto{\pgfqpoint{0.663063in}{1.202893in}}{\pgfqpoint{0.670963in}{1.199621in}}{\pgfqpoint{0.679199in}{1.199621in}}%
\pgfpathclose%
\pgfusepath{stroke,fill}%
\end{pgfscope}%
\begin{pgfscope}%
\pgfpathrectangle{\pgfqpoint{0.100000in}{0.220728in}}{\pgfqpoint{3.696000in}{3.696000in}}%
\pgfusepath{clip}%
\pgfsetbuttcap%
\pgfsetroundjoin%
\definecolor{currentfill}{rgb}{0.121569,0.466667,0.705882}%
\pgfsetfillcolor{currentfill}%
\pgfsetfillopacity{0.619877}%
\pgfsetlinewidth{1.003750pt}%
\definecolor{currentstroke}{rgb}{0.121569,0.466667,0.705882}%
\pgfsetstrokecolor{currentstroke}%
\pgfsetstrokeopacity{0.619877}%
\pgfsetdash{}{0pt}%
\pgfpathmoveto{\pgfqpoint{0.679199in}{1.199621in}}%
\pgfpathcurveto{\pgfqpoint{0.687435in}{1.199621in}}{\pgfqpoint{0.695335in}{1.202893in}}{\pgfqpoint{0.701159in}{1.208717in}}%
\pgfpathcurveto{\pgfqpoint{0.706983in}{1.214541in}}{\pgfqpoint{0.710255in}{1.222441in}}{\pgfqpoint{0.710255in}{1.230677in}}%
\pgfpathcurveto{\pgfqpoint{0.710255in}{1.238914in}}{\pgfqpoint{0.706983in}{1.246814in}}{\pgfqpoint{0.701159in}{1.252638in}}%
\pgfpathcurveto{\pgfqpoint{0.695335in}{1.258462in}}{\pgfqpoint{0.687435in}{1.261734in}}{\pgfqpoint{0.679199in}{1.261734in}}%
\pgfpathcurveto{\pgfqpoint{0.670963in}{1.261734in}}{\pgfqpoint{0.663063in}{1.258462in}}{\pgfqpoint{0.657239in}{1.252638in}}%
\pgfpathcurveto{\pgfqpoint{0.651415in}{1.246814in}}{\pgfqpoint{0.648143in}{1.238914in}}{\pgfqpoint{0.648143in}{1.230677in}}%
\pgfpathcurveto{\pgfqpoint{0.648143in}{1.222441in}}{\pgfqpoint{0.651415in}{1.214541in}}{\pgfqpoint{0.657239in}{1.208717in}}%
\pgfpathcurveto{\pgfqpoint{0.663063in}{1.202893in}}{\pgfqpoint{0.670963in}{1.199621in}}{\pgfqpoint{0.679199in}{1.199621in}}%
\pgfpathclose%
\pgfusepath{stroke,fill}%
\end{pgfscope}%
\begin{pgfscope}%
\pgfpathrectangle{\pgfqpoint{0.100000in}{0.220728in}}{\pgfqpoint{3.696000in}{3.696000in}}%
\pgfusepath{clip}%
\pgfsetbuttcap%
\pgfsetroundjoin%
\definecolor{currentfill}{rgb}{0.121569,0.466667,0.705882}%
\pgfsetfillcolor{currentfill}%
\pgfsetfillopacity{0.619877}%
\pgfsetlinewidth{1.003750pt}%
\definecolor{currentstroke}{rgb}{0.121569,0.466667,0.705882}%
\pgfsetstrokecolor{currentstroke}%
\pgfsetstrokeopacity{0.619877}%
\pgfsetdash{}{0pt}%
\pgfpathmoveto{\pgfqpoint{0.679199in}{1.199621in}}%
\pgfpathcurveto{\pgfqpoint{0.687435in}{1.199621in}}{\pgfqpoint{0.695335in}{1.202893in}}{\pgfqpoint{0.701159in}{1.208717in}}%
\pgfpathcurveto{\pgfqpoint{0.706983in}{1.214541in}}{\pgfqpoint{0.710255in}{1.222441in}}{\pgfqpoint{0.710255in}{1.230677in}}%
\pgfpathcurveto{\pgfqpoint{0.710255in}{1.238914in}}{\pgfqpoint{0.706983in}{1.246814in}}{\pgfqpoint{0.701159in}{1.252638in}}%
\pgfpathcurveto{\pgfqpoint{0.695335in}{1.258462in}}{\pgfqpoint{0.687435in}{1.261734in}}{\pgfqpoint{0.679199in}{1.261734in}}%
\pgfpathcurveto{\pgfqpoint{0.670963in}{1.261734in}}{\pgfqpoint{0.663063in}{1.258462in}}{\pgfqpoint{0.657239in}{1.252638in}}%
\pgfpathcurveto{\pgfqpoint{0.651415in}{1.246814in}}{\pgfqpoint{0.648142in}{1.238914in}}{\pgfqpoint{0.648142in}{1.230677in}}%
\pgfpathcurveto{\pgfqpoint{0.648142in}{1.222441in}}{\pgfqpoint{0.651415in}{1.214541in}}{\pgfqpoint{0.657239in}{1.208717in}}%
\pgfpathcurveto{\pgfqpoint{0.663063in}{1.202893in}}{\pgfqpoint{0.670963in}{1.199621in}}{\pgfqpoint{0.679199in}{1.199621in}}%
\pgfpathclose%
\pgfusepath{stroke,fill}%
\end{pgfscope}%
\begin{pgfscope}%
\pgfpathrectangle{\pgfqpoint{0.100000in}{0.220728in}}{\pgfqpoint{3.696000in}{3.696000in}}%
\pgfusepath{clip}%
\pgfsetbuttcap%
\pgfsetroundjoin%
\definecolor{currentfill}{rgb}{0.121569,0.466667,0.705882}%
\pgfsetfillcolor{currentfill}%
\pgfsetfillopacity{0.619878}%
\pgfsetlinewidth{1.003750pt}%
\definecolor{currentstroke}{rgb}{0.121569,0.466667,0.705882}%
\pgfsetstrokecolor{currentstroke}%
\pgfsetstrokeopacity{0.619878}%
\pgfsetdash{}{0pt}%
\pgfpathmoveto{\pgfqpoint{0.679199in}{1.199621in}}%
\pgfpathcurveto{\pgfqpoint{0.687435in}{1.199621in}}{\pgfqpoint{0.695335in}{1.202893in}}{\pgfqpoint{0.701159in}{1.208717in}}%
\pgfpathcurveto{\pgfqpoint{0.706983in}{1.214541in}}{\pgfqpoint{0.710255in}{1.222441in}}{\pgfqpoint{0.710255in}{1.230677in}}%
\pgfpathcurveto{\pgfqpoint{0.710255in}{1.238914in}}{\pgfqpoint{0.706983in}{1.246814in}}{\pgfqpoint{0.701159in}{1.252638in}}%
\pgfpathcurveto{\pgfqpoint{0.695335in}{1.258462in}}{\pgfqpoint{0.687435in}{1.261734in}}{\pgfqpoint{0.679199in}{1.261734in}}%
\pgfpathcurveto{\pgfqpoint{0.670963in}{1.261734in}}{\pgfqpoint{0.663063in}{1.258462in}}{\pgfqpoint{0.657239in}{1.252638in}}%
\pgfpathcurveto{\pgfqpoint{0.651415in}{1.246814in}}{\pgfqpoint{0.648142in}{1.238914in}}{\pgfqpoint{0.648142in}{1.230677in}}%
\pgfpathcurveto{\pgfqpoint{0.648142in}{1.222441in}}{\pgfqpoint{0.651415in}{1.214541in}}{\pgfqpoint{0.657239in}{1.208717in}}%
\pgfpathcurveto{\pgfqpoint{0.663063in}{1.202893in}}{\pgfqpoint{0.670963in}{1.199621in}}{\pgfqpoint{0.679199in}{1.199621in}}%
\pgfpathclose%
\pgfusepath{stroke,fill}%
\end{pgfscope}%
\begin{pgfscope}%
\pgfpathrectangle{\pgfqpoint{0.100000in}{0.220728in}}{\pgfqpoint{3.696000in}{3.696000in}}%
\pgfusepath{clip}%
\pgfsetbuttcap%
\pgfsetroundjoin%
\definecolor{currentfill}{rgb}{0.121569,0.466667,0.705882}%
\pgfsetfillcolor{currentfill}%
\pgfsetfillopacity{0.619878}%
\pgfsetlinewidth{1.003750pt}%
\definecolor{currentstroke}{rgb}{0.121569,0.466667,0.705882}%
\pgfsetstrokecolor{currentstroke}%
\pgfsetstrokeopacity{0.619878}%
\pgfsetdash{}{0pt}%
\pgfpathmoveto{\pgfqpoint{0.679199in}{1.199621in}}%
\pgfpathcurveto{\pgfqpoint{0.687435in}{1.199621in}}{\pgfqpoint{0.695335in}{1.202893in}}{\pgfqpoint{0.701159in}{1.208717in}}%
\pgfpathcurveto{\pgfqpoint{0.706983in}{1.214541in}}{\pgfqpoint{0.710255in}{1.222441in}}{\pgfqpoint{0.710255in}{1.230677in}}%
\pgfpathcurveto{\pgfqpoint{0.710255in}{1.238914in}}{\pgfqpoint{0.706983in}{1.246814in}}{\pgfqpoint{0.701159in}{1.252638in}}%
\pgfpathcurveto{\pgfqpoint{0.695335in}{1.258462in}}{\pgfqpoint{0.687435in}{1.261734in}}{\pgfqpoint{0.679199in}{1.261734in}}%
\pgfpathcurveto{\pgfqpoint{0.670963in}{1.261734in}}{\pgfqpoint{0.663063in}{1.258462in}}{\pgfqpoint{0.657239in}{1.252638in}}%
\pgfpathcurveto{\pgfqpoint{0.651415in}{1.246814in}}{\pgfqpoint{0.648142in}{1.238914in}}{\pgfqpoint{0.648142in}{1.230677in}}%
\pgfpathcurveto{\pgfqpoint{0.648142in}{1.222441in}}{\pgfqpoint{0.651415in}{1.214541in}}{\pgfqpoint{0.657239in}{1.208717in}}%
\pgfpathcurveto{\pgfqpoint{0.663063in}{1.202893in}}{\pgfqpoint{0.670963in}{1.199621in}}{\pgfqpoint{0.679199in}{1.199621in}}%
\pgfpathclose%
\pgfusepath{stroke,fill}%
\end{pgfscope}%
\begin{pgfscope}%
\pgfpathrectangle{\pgfqpoint{0.100000in}{0.220728in}}{\pgfqpoint{3.696000in}{3.696000in}}%
\pgfusepath{clip}%
\pgfsetbuttcap%
\pgfsetroundjoin%
\definecolor{currentfill}{rgb}{0.121569,0.466667,0.705882}%
\pgfsetfillcolor{currentfill}%
\pgfsetfillopacity{0.619878}%
\pgfsetlinewidth{1.003750pt}%
\definecolor{currentstroke}{rgb}{0.121569,0.466667,0.705882}%
\pgfsetstrokecolor{currentstroke}%
\pgfsetstrokeopacity{0.619878}%
\pgfsetdash{}{0pt}%
\pgfpathmoveto{\pgfqpoint{0.679199in}{1.199621in}}%
\pgfpathcurveto{\pgfqpoint{0.687435in}{1.199621in}}{\pgfqpoint{0.695335in}{1.202893in}}{\pgfqpoint{0.701159in}{1.208717in}}%
\pgfpathcurveto{\pgfqpoint{0.706983in}{1.214541in}}{\pgfqpoint{0.710255in}{1.222441in}}{\pgfqpoint{0.710255in}{1.230677in}}%
\pgfpathcurveto{\pgfqpoint{0.710255in}{1.238914in}}{\pgfqpoint{0.706983in}{1.246814in}}{\pgfqpoint{0.701159in}{1.252638in}}%
\pgfpathcurveto{\pgfqpoint{0.695335in}{1.258462in}}{\pgfqpoint{0.687435in}{1.261734in}}{\pgfqpoint{0.679199in}{1.261734in}}%
\pgfpathcurveto{\pgfqpoint{0.670963in}{1.261734in}}{\pgfqpoint{0.663063in}{1.258462in}}{\pgfqpoint{0.657239in}{1.252638in}}%
\pgfpathcurveto{\pgfqpoint{0.651415in}{1.246814in}}{\pgfqpoint{0.648142in}{1.238914in}}{\pgfqpoint{0.648142in}{1.230677in}}%
\pgfpathcurveto{\pgfqpoint{0.648142in}{1.222441in}}{\pgfqpoint{0.651415in}{1.214541in}}{\pgfqpoint{0.657239in}{1.208717in}}%
\pgfpathcurveto{\pgfqpoint{0.663063in}{1.202893in}}{\pgfqpoint{0.670963in}{1.199621in}}{\pgfqpoint{0.679199in}{1.199621in}}%
\pgfpathclose%
\pgfusepath{stroke,fill}%
\end{pgfscope}%
\begin{pgfscope}%
\pgfpathrectangle{\pgfqpoint{0.100000in}{0.220728in}}{\pgfqpoint{3.696000in}{3.696000in}}%
\pgfusepath{clip}%
\pgfsetbuttcap%
\pgfsetroundjoin%
\definecolor{currentfill}{rgb}{0.121569,0.466667,0.705882}%
\pgfsetfillcolor{currentfill}%
\pgfsetfillopacity{0.619878}%
\pgfsetlinewidth{1.003750pt}%
\definecolor{currentstroke}{rgb}{0.121569,0.466667,0.705882}%
\pgfsetstrokecolor{currentstroke}%
\pgfsetstrokeopacity{0.619878}%
\pgfsetdash{}{0pt}%
\pgfpathmoveto{\pgfqpoint{0.679199in}{1.199621in}}%
\pgfpathcurveto{\pgfqpoint{0.687435in}{1.199621in}}{\pgfqpoint{0.695335in}{1.202893in}}{\pgfqpoint{0.701159in}{1.208717in}}%
\pgfpathcurveto{\pgfqpoint{0.706983in}{1.214541in}}{\pgfqpoint{0.710255in}{1.222441in}}{\pgfqpoint{0.710255in}{1.230677in}}%
\pgfpathcurveto{\pgfqpoint{0.710255in}{1.238914in}}{\pgfqpoint{0.706983in}{1.246814in}}{\pgfqpoint{0.701159in}{1.252638in}}%
\pgfpathcurveto{\pgfqpoint{0.695335in}{1.258462in}}{\pgfqpoint{0.687435in}{1.261734in}}{\pgfqpoint{0.679199in}{1.261734in}}%
\pgfpathcurveto{\pgfqpoint{0.670963in}{1.261734in}}{\pgfqpoint{0.663063in}{1.258462in}}{\pgfqpoint{0.657239in}{1.252638in}}%
\pgfpathcurveto{\pgfqpoint{0.651415in}{1.246814in}}{\pgfqpoint{0.648142in}{1.238914in}}{\pgfqpoint{0.648142in}{1.230677in}}%
\pgfpathcurveto{\pgfqpoint{0.648142in}{1.222441in}}{\pgfqpoint{0.651415in}{1.214541in}}{\pgfqpoint{0.657239in}{1.208717in}}%
\pgfpathcurveto{\pgfqpoint{0.663063in}{1.202893in}}{\pgfqpoint{0.670963in}{1.199621in}}{\pgfqpoint{0.679199in}{1.199621in}}%
\pgfpathclose%
\pgfusepath{stroke,fill}%
\end{pgfscope}%
\begin{pgfscope}%
\pgfpathrectangle{\pgfqpoint{0.100000in}{0.220728in}}{\pgfqpoint{3.696000in}{3.696000in}}%
\pgfusepath{clip}%
\pgfsetbuttcap%
\pgfsetroundjoin%
\definecolor{currentfill}{rgb}{0.121569,0.466667,0.705882}%
\pgfsetfillcolor{currentfill}%
\pgfsetfillopacity{0.619878}%
\pgfsetlinewidth{1.003750pt}%
\definecolor{currentstroke}{rgb}{0.121569,0.466667,0.705882}%
\pgfsetstrokecolor{currentstroke}%
\pgfsetstrokeopacity{0.619878}%
\pgfsetdash{}{0pt}%
\pgfpathmoveto{\pgfqpoint{0.679199in}{1.199621in}}%
\pgfpathcurveto{\pgfqpoint{0.687435in}{1.199621in}}{\pgfqpoint{0.695335in}{1.202893in}}{\pgfqpoint{0.701159in}{1.208717in}}%
\pgfpathcurveto{\pgfqpoint{0.706983in}{1.214541in}}{\pgfqpoint{0.710255in}{1.222441in}}{\pgfqpoint{0.710255in}{1.230677in}}%
\pgfpathcurveto{\pgfqpoint{0.710255in}{1.238914in}}{\pgfqpoint{0.706983in}{1.246814in}}{\pgfqpoint{0.701159in}{1.252638in}}%
\pgfpathcurveto{\pgfqpoint{0.695335in}{1.258462in}}{\pgfqpoint{0.687435in}{1.261734in}}{\pgfqpoint{0.679199in}{1.261734in}}%
\pgfpathcurveto{\pgfqpoint{0.670963in}{1.261734in}}{\pgfqpoint{0.663063in}{1.258462in}}{\pgfqpoint{0.657239in}{1.252638in}}%
\pgfpathcurveto{\pgfqpoint{0.651415in}{1.246814in}}{\pgfqpoint{0.648142in}{1.238914in}}{\pgfqpoint{0.648142in}{1.230677in}}%
\pgfpathcurveto{\pgfqpoint{0.648142in}{1.222441in}}{\pgfqpoint{0.651415in}{1.214541in}}{\pgfqpoint{0.657239in}{1.208717in}}%
\pgfpathcurveto{\pgfqpoint{0.663063in}{1.202893in}}{\pgfqpoint{0.670963in}{1.199621in}}{\pgfqpoint{0.679199in}{1.199621in}}%
\pgfpathclose%
\pgfusepath{stroke,fill}%
\end{pgfscope}%
\begin{pgfscope}%
\pgfpathrectangle{\pgfqpoint{0.100000in}{0.220728in}}{\pgfqpoint{3.696000in}{3.696000in}}%
\pgfusepath{clip}%
\pgfsetbuttcap%
\pgfsetroundjoin%
\definecolor{currentfill}{rgb}{0.121569,0.466667,0.705882}%
\pgfsetfillcolor{currentfill}%
\pgfsetfillopacity{0.619878}%
\pgfsetlinewidth{1.003750pt}%
\definecolor{currentstroke}{rgb}{0.121569,0.466667,0.705882}%
\pgfsetstrokecolor{currentstroke}%
\pgfsetstrokeopacity{0.619878}%
\pgfsetdash{}{0pt}%
\pgfpathmoveto{\pgfqpoint{0.679199in}{1.199621in}}%
\pgfpathcurveto{\pgfqpoint{0.687435in}{1.199621in}}{\pgfqpoint{0.695335in}{1.202893in}}{\pgfqpoint{0.701159in}{1.208717in}}%
\pgfpathcurveto{\pgfqpoint{0.706983in}{1.214541in}}{\pgfqpoint{0.710255in}{1.222441in}}{\pgfqpoint{0.710255in}{1.230677in}}%
\pgfpathcurveto{\pgfqpoint{0.710255in}{1.238914in}}{\pgfqpoint{0.706983in}{1.246814in}}{\pgfqpoint{0.701159in}{1.252638in}}%
\pgfpathcurveto{\pgfqpoint{0.695335in}{1.258462in}}{\pgfqpoint{0.687435in}{1.261734in}}{\pgfqpoint{0.679199in}{1.261734in}}%
\pgfpathcurveto{\pgfqpoint{0.670963in}{1.261734in}}{\pgfqpoint{0.663063in}{1.258462in}}{\pgfqpoint{0.657239in}{1.252638in}}%
\pgfpathcurveto{\pgfqpoint{0.651415in}{1.246814in}}{\pgfqpoint{0.648142in}{1.238914in}}{\pgfqpoint{0.648142in}{1.230677in}}%
\pgfpathcurveto{\pgfqpoint{0.648142in}{1.222441in}}{\pgfqpoint{0.651415in}{1.214541in}}{\pgfqpoint{0.657239in}{1.208717in}}%
\pgfpathcurveto{\pgfqpoint{0.663063in}{1.202893in}}{\pgfqpoint{0.670963in}{1.199621in}}{\pgfqpoint{0.679199in}{1.199621in}}%
\pgfpathclose%
\pgfusepath{stroke,fill}%
\end{pgfscope}%
\begin{pgfscope}%
\pgfpathrectangle{\pgfqpoint{0.100000in}{0.220728in}}{\pgfqpoint{3.696000in}{3.696000in}}%
\pgfusepath{clip}%
\pgfsetbuttcap%
\pgfsetroundjoin%
\definecolor{currentfill}{rgb}{0.121569,0.466667,0.705882}%
\pgfsetfillcolor{currentfill}%
\pgfsetfillopacity{0.619878}%
\pgfsetlinewidth{1.003750pt}%
\definecolor{currentstroke}{rgb}{0.121569,0.466667,0.705882}%
\pgfsetstrokecolor{currentstroke}%
\pgfsetstrokeopacity{0.619878}%
\pgfsetdash{}{0pt}%
\pgfpathmoveto{\pgfqpoint{0.679199in}{1.199621in}}%
\pgfpathcurveto{\pgfqpoint{0.687435in}{1.199621in}}{\pgfqpoint{0.695335in}{1.202893in}}{\pgfqpoint{0.701159in}{1.208717in}}%
\pgfpathcurveto{\pgfqpoint{0.706983in}{1.214541in}}{\pgfqpoint{0.710255in}{1.222441in}}{\pgfqpoint{0.710255in}{1.230677in}}%
\pgfpathcurveto{\pgfqpoint{0.710255in}{1.238914in}}{\pgfqpoint{0.706983in}{1.246814in}}{\pgfqpoint{0.701159in}{1.252638in}}%
\pgfpathcurveto{\pgfqpoint{0.695335in}{1.258462in}}{\pgfqpoint{0.687435in}{1.261734in}}{\pgfqpoint{0.679199in}{1.261734in}}%
\pgfpathcurveto{\pgfqpoint{0.670963in}{1.261734in}}{\pgfqpoint{0.663063in}{1.258462in}}{\pgfqpoint{0.657239in}{1.252638in}}%
\pgfpathcurveto{\pgfqpoint{0.651415in}{1.246814in}}{\pgfqpoint{0.648142in}{1.238914in}}{\pgfqpoint{0.648142in}{1.230677in}}%
\pgfpathcurveto{\pgfqpoint{0.648142in}{1.222441in}}{\pgfqpoint{0.651415in}{1.214541in}}{\pgfqpoint{0.657239in}{1.208717in}}%
\pgfpathcurveto{\pgfqpoint{0.663063in}{1.202893in}}{\pgfqpoint{0.670963in}{1.199621in}}{\pgfqpoint{0.679199in}{1.199621in}}%
\pgfpathclose%
\pgfusepath{stroke,fill}%
\end{pgfscope}%
\begin{pgfscope}%
\pgfpathrectangle{\pgfqpoint{0.100000in}{0.220728in}}{\pgfqpoint{3.696000in}{3.696000in}}%
\pgfusepath{clip}%
\pgfsetbuttcap%
\pgfsetroundjoin%
\definecolor{currentfill}{rgb}{0.121569,0.466667,0.705882}%
\pgfsetfillcolor{currentfill}%
\pgfsetfillopacity{0.619878}%
\pgfsetlinewidth{1.003750pt}%
\definecolor{currentstroke}{rgb}{0.121569,0.466667,0.705882}%
\pgfsetstrokecolor{currentstroke}%
\pgfsetstrokeopacity{0.619878}%
\pgfsetdash{}{0pt}%
\pgfpathmoveto{\pgfqpoint{0.679199in}{1.199621in}}%
\pgfpathcurveto{\pgfqpoint{0.687435in}{1.199621in}}{\pgfqpoint{0.695335in}{1.202893in}}{\pgfqpoint{0.701159in}{1.208717in}}%
\pgfpathcurveto{\pgfqpoint{0.706983in}{1.214541in}}{\pgfqpoint{0.710255in}{1.222441in}}{\pgfqpoint{0.710255in}{1.230677in}}%
\pgfpathcurveto{\pgfqpoint{0.710255in}{1.238914in}}{\pgfqpoint{0.706983in}{1.246814in}}{\pgfqpoint{0.701159in}{1.252638in}}%
\pgfpathcurveto{\pgfqpoint{0.695335in}{1.258462in}}{\pgfqpoint{0.687435in}{1.261734in}}{\pgfqpoint{0.679199in}{1.261734in}}%
\pgfpathcurveto{\pgfqpoint{0.670963in}{1.261734in}}{\pgfqpoint{0.663063in}{1.258462in}}{\pgfqpoint{0.657239in}{1.252638in}}%
\pgfpathcurveto{\pgfqpoint{0.651415in}{1.246814in}}{\pgfqpoint{0.648142in}{1.238914in}}{\pgfqpoint{0.648142in}{1.230677in}}%
\pgfpathcurveto{\pgfqpoint{0.648142in}{1.222441in}}{\pgfqpoint{0.651415in}{1.214541in}}{\pgfqpoint{0.657239in}{1.208717in}}%
\pgfpathcurveto{\pgfqpoint{0.663063in}{1.202893in}}{\pgfqpoint{0.670963in}{1.199621in}}{\pgfqpoint{0.679199in}{1.199621in}}%
\pgfpathclose%
\pgfusepath{stroke,fill}%
\end{pgfscope}%
\begin{pgfscope}%
\pgfpathrectangle{\pgfqpoint{0.100000in}{0.220728in}}{\pgfqpoint{3.696000in}{3.696000in}}%
\pgfusepath{clip}%
\pgfsetbuttcap%
\pgfsetroundjoin%
\definecolor{currentfill}{rgb}{0.121569,0.466667,0.705882}%
\pgfsetfillcolor{currentfill}%
\pgfsetfillopacity{0.619878}%
\pgfsetlinewidth{1.003750pt}%
\definecolor{currentstroke}{rgb}{0.121569,0.466667,0.705882}%
\pgfsetstrokecolor{currentstroke}%
\pgfsetstrokeopacity{0.619878}%
\pgfsetdash{}{0pt}%
\pgfpathmoveto{\pgfqpoint{0.679199in}{1.199621in}}%
\pgfpathcurveto{\pgfqpoint{0.687435in}{1.199621in}}{\pgfqpoint{0.695335in}{1.202893in}}{\pgfqpoint{0.701159in}{1.208717in}}%
\pgfpathcurveto{\pgfqpoint{0.706983in}{1.214541in}}{\pgfqpoint{0.710255in}{1.222441in}}{\pgfqpoint{0.710255in}{1.230677in}}%
\pgfpathcurveto{\pgfqpoint{0.710255in}{1.238914in}}{\pgfqpoint{0.706983in}{1.246814in}}{\pgfqpoint{0.701159in}{1.252638in}}%
\pgfpathcurveto{\pgfqpoint{0.695335in}{1.258462in}}{\pgfqpoint{0.687435in}{1.261734in}}{\pgfqpoint{0.679199in}{1.261734in}}%
\pgfpathcurveto{\pgfqpoint{0.670963in}{1.261734in}}{\pgfqpoint{0.663063in}{1.258462in}}{\pgfqpoint{0.657239in}{1.252638in}}%
\pgfpathcurveto{\pgfqpoint{0.651415in}{1.246814in}}{\pgfqpoint{0.648142in}{1.238914in}}{\pgfqpoint{0.648142in}{1.230677in}}%
\pgfpathcurveto{\pgfqpoint{0.648142in}{1.222441in}}{\pgfqpoint{0.651415in}{1.214541in}}{\pgfqpoint{0.657239in}{1.208717in}}%
\pgfpathcurveto{\pgfqpoint{0.663063in}{1.202893in}}{\pgfqpoint{0.670963in}{1.199621in}}{\pgfqpoint{0.679199in}{1.199621in}}%
\pgfpathclose%
\pgfusepath{stroke,fill}%
\end{pgfscope}%
\begin{pgfscope}%
\pgfpathrectangle{\pgfqpoint{0.100000in}{0.220728in}}{\pgfqpoint{3.696000in}{3.696000in}}%
\pgfusepath{clip}%
\pgfsetbuttcap%
\pgfsetroundjoin%
\definecolor{currentfill}{rgb}{0.121569,0.466667,0.705882}%
\pgfsetfillcolor{currentfill}%
\pgfsetfillopacity{0.619878}%
\pgfsetlinewidth{1.003750pt}%
\definecolor{currentstroke}{rgb}{0.121569,0.466667,0.705882}%
\pgfsetstrokecolor{currentstroke}%
\pgfsetstrokeopacity{0.619878}%
\pgfsetdash{}{0pt}%
\pgfpathmoveto{\pgfqpoint{0.679199in}{1.199621in}}%
\pgfpathcurveto{\pgfqpoint{0.687435in}{1.199621in}}{\pgfqpoint{0.695335in}{1.202893in}}{\pgfqpoint{0.701159in}{1.208717in}}%
\pgfpathcurveto{\pgfqpoint{0.706983in}{1.214541in}}{\pgfqpoint{0.710255in}{1.222441in}}{\pgfqpoint{0.710255in}{1.230677in}}%
\pgfpathcurveto{\pgfqpoint{0.710255in}{1.238914in}}{\pgfqpoint{0.706983in}{1.246814in}}{\pgfqpoint{0.701159in}{1.252638in}}%
\pgfpathcurveto{\pgfqpoint{0.695335in}{1.258462in}}{\pgfqpoint{0.687435in}{1.261734in}}{\pgfqpoint{0.679199in}{1.261734in}}%
\pgfpathcurveto{\pgfqpoint{0.670963in}{1.261734in}}{\pgfqpoint{0.663063in}{1.258462in}}{\pgfqpoint{0.657239in}{1.252638in}}%
\pgfpathcurveto{\pgfqpoint{0.651415in}{1.246814in}}{\pgfqpoint{0.648142in}{1.238914in}}{\pgfqpoint{0.648142in}{1.230677in}}%
\pgfpathcurveto{\pgfqpoint{0.648142in}{1.222441in}}{\pgfqpoint{0.651415in}{1.214541in}}{\pgfqpoint{0.657239in}{1.208717in}}%
\pgfpathcurveto{\pgfqpoint{0.663063in}{1.202893in}}{\pgfqpoint{0.670963in}{1.199621in}}{\pgfqpoint{0.679199in}{1.199621in}}%
\pgfpathclose%
\pgfusepath{stroke,fill}%
\end{pgfscope}%
\begin{pgfscope}%
\pgfpathrectangle{\pgfqpoint{0.100000in}{0.220728in}}{\pgfqpoint{3.696000in}{3.696000in}}%
\pgfusepath{clip}%
\pgfsetbuttcap%
\pgfsetroundjoin%
\definecolor{currentfill}{rgb}{0.121569,0.466667,0.705882}%
\pgfsetfillcolor{currentfill}%
\pgfsetfillopacity{0.619878}%
\pgfsetlinewidth{1.003750pt}%
\definecolor{currentstroke}{rgb}{0.121569,0.466667,0.705882}%
\pgfsetstrokecolor{currentstroke}%
\pgfsetstrokeopacity{0.619878}%
\pgfsetdash{}{0pt}%
\pgfpathmoveto{\pgfqpoint{0.679199in}{1.199621in}}%
\pgfpathcurveto{\pgfqpoint{0.687435in}{1.199621in}}{\pgfqpoint{0.695335in}{1.202893in}}{\pgfqpoint{0.701159in}{1.208717in}}%
\pgfpathcurveto{\pgfqpoint{0.706983in}{1.214541in}}{\pgfqpoint{0.710255in}{1.222441in}}{\pgfqpoint{0.710255in}{1.230677in}}%
\pgfpathcurveto{\pgfqpoint{0.710255in}{1.238914in}}{\pgfqpoint{0.706983in}{1.246814in}}{\pgfqpoint{0.701159in}{1.252638in}}%
\pgfpathcurveto{\pgfqpoint{0.695335in}{1.258462in}}{\pgfqpoint{0.687435in}{1.261734in}}{\pgfqpoint{0.679199in}{1.261734in}}%
\pgfpathcurveto{\pgfqpoint{0.670963in}{1.261734in}}{\pgfqpoint{0.663063in}{1.258462in}}{\pgfqpoint{0.657239in}{1.252638in}}%
\pgfpathcurveto{\pgfqpoint{0.651415in}{1.246814in}}{\pgfqpoint{0.648142in}{1.238914in}}{\pgfqpoint{0.648142in}{1.230677in}}%
\pgfpathcurveto{\pgfqpoint{0.648142in}{1.222441in}}{\pgfqpoint{0.651415in}{1.214541in}}{\pgfqpoint{0.657239in}{1.208717in}}%
\pgfpathcurveto{\pgfqpoint{0.663063in}{1.202893in}}{\pgfqpoint{0.670963in}{1.199621in}}{\pgfqpoint{0.679199in}{1.199621in}}%
\pgfpathclose%
\pgfusepath{stroke,fill}%
\end{pgfscope}%
\begin{pgfscope}%
\pgfpathrectangle{\pgfqpoint{0.100000in}{0.220728in}}{\pgfqpoint{3.696000in}{3.696000in}}%
\pgfusepath{clip}%
\pgfsetbuttcap%
\pgfsetroundjoin%
\definecolor{currentfill}{rgb}{0.121569,0.466667,0.705882}%
\pgfsetfillcolor{currentfill}%
\pgfsetfillopacity{0.619878}%
\pgfsetlinewidth{1.003750pt}%
\definecolor{currentstroke}{rgb}{0.121569,0.466667,0.705882}%
\pgfsetstrokecolor{currentstroke}%
\pgfsetstrokeopacity{0.619878}%
\pgfsetdash{}{0pt}%
\pgfpathmoveto{\pgfqpoint{0.679199in}{1.199621in}}%
\pgfpathcurveto{\pgfqpoint{0.687435in}{1.199621in}}{\pgfqpoint{0.695335in}{1.202893in}}{\pgfqpoint{0.701159in}{1.208717in}}%
\pgfpathcurveto{\pgfqpoint{0.706983in}{1.214541in}}{\pgfqpoint{0.710255in}{1.222441in}}{\pgfqpoint{0.710255in}{1.230677in}}%
\pgfpathcurveto{\pgfqpoint{0.710255in}{1.238914in}}{\pgfqpoint{0.706983in}{1.246814in}}{\pgfqpoint{0.701159in}{1.252638in}}%
\pgfpathcurveto{\pgfqpoint{0.695335in}{1.258462in}}{\pgfqpoint{0.687435in}{1.261734in}}{\pgfqpoint{0.679199in}{1.261734in}}%
\pgfpathcurveto{\pgfqpoint{0.670963in}{1.261734in}}{\pgfqpoint{0.663063in}{1.258462in}}{\pgfqpoint{0.657239in}{1.252638in}}%
\pgfpathcurveto{\pgfqpoint{0.651415in}{1.246814in}}{\pgfqpoint{0.648142in}{1.238914in}}{\pgfqpoint{0.648142in}{1.230677in}}%
\pgfpathcurveto{\pgfqpoint{0.648142in}{1.222441in}}{\pgfqpoint{0.651415in}{1.214541in}}{\pgfqpoint{0.657239in}{1.208717in}}%
\pgfpathcurveto{\pgfqpoint{0.663063in}{1.202893in}}{\pgfqpoint{0.670963in}{1.199621in}}{\pgfqpoint{0.679199in}{1.199621in}}%
\pgfpathclose%
\pgfusepath{stroke,fill}%
\end{pgfscope}%
\begin{pgfscope}%
\pgfpathrectangle{\pgfqpoint{0.100000in}{0.220728in}}{\pgfqpoint{3.696000in}{3.696000in}}%
\pgfusepath{clip}%
\pgfsetbuttcap%
\pgfsetroundjoin%
\definecolor{currentfill}{rgb}{0.121569,0.466667,0.705882}%
\pgfsetfillcolor{currentfill}%
\pgfsetfillopacity{0.619878}%
\pgfsetlinewidth{1.003750pt}%
\definecolor{currentstroke}{rgb}{0.121569,0.466667,0.705882}%
\pgfsetstrokecolor{currentstroke}%
\pgfsetstrokeopacity{0.619878}%
\pgfsetdash{}{0pt}%
\pgfpathmoveto{\pgfqpoint{0.679199in}{1.199621in}}%
\pgfpathcurveto{\pgfqpoint{0.687435in}{1.199621in}}{\pgfqpoint{0.695335in}{1.202893in}}{\pgfqpoint{0.701159in}{1.208717in}}%
\pgfpathcurveto{\pgfqpoint{0.706983in}{1.214541in}}{\pgfqpoint{0.710255in}{1.222441in}}{\pgfqpoint{0.710255in}{1.230677in}}%
\pgfpathcurveto{\pgfqpoint{0.710255in}{1.238914in}}{\pgfqpoint{0.706983in}{1.246814in}}{\pgfqpoint{0.701159in}{1.252638in}}%
\pgfpathcurveto{\pgfqpoint{0.695335in}{1.258462in}}{\pgfqpoint{0.687435in}{1.261734in}}{\pgfqpoint{0.679199in}{1.261734in}}%
\pgfpathcurveto{\pgfqpoint{0.670963in}{1.261734in}}{\pgfqpoint{0.663063in}{1.258462in}}{\pgfqpoint{0.657239in}{1.252638in}}%
\pgfpathcurveto{\pgfqpoint{0.651415in}{1.246814in}}{\pgfqpoint{0.648142in}{1.238914in}}{\pgfqpoint{0.648142in}{1.230677in}}%
\pgfpathcurveto{\pgfqpoint{0.648142in}{1.222441in}}{\pgfqpoint{0.651415in}{1.214541in}}{\pgfqpoint{0.657239in}{1.208717in}}%
\pgfpathcurveto{\pgfqpoint{0.663063in}{1.202893in}}{\pgfqpoint{0.670963in}{1.199621in}}{\pgfqpoint{0.679199in}{1.199621in}}%
\pgfpathclose%
\pgfusepath{stroke,fill}%
\end{pgfscope}%
\begin{pgfscope}%
\pgfpathrectangle{\pgfqpoint{0.100000in}{0.220728in}}{\pgfqpoint{3.696000in}{3.696000in}}%
\pgfusepath{clip}%
\pgfsetbuttcap%
\pgfsetroundjoin%
\definecolor{currentfill}{rgb}{0.121569,0.466667,0.705882}%
\pgfsetfillcolor{currentfill}%
\pgfsetfillopacity{0.619878}%
\pgfsetlinewidth{1.003750pt}%
\definecolor{currentstroke}{rgb}{0.121569,0.466667,0.705882}%
\pgfsetstrokecolor{currentstroke}%
\pgfsetstrokeopacity{0.619878}%
\pgfsetdash{}{0pt}%
\pgfpathmoveto{\pgfqpoint{0.679199in}{1.199621in}}%
\pgfpathcurveto{\pgfqpoint{0.687435in}{1.199621in}}{\pgfqpoint{0.695335in}{1.202893in}}{\pgfqpoint{0.701159in}{1.208717in}}%
\pgfpathcurveto{\pgfqpoint{0.706983in}{1.214541in}}{\pgfqpoint{0.710255in}{1.222441in}}{\pgfqpoint{0.710255in}{1.230677in}}%
\pgfpathcurveto{\pgfqpoint{0.710255in}{1.238914in}}{\pgfqpoint{0.706983in}{1.246814in}}{\pgfqpoint{0.701159in}{1.252638in}}%
\pgfpathcurveto{\pgfqpoint{0.695335in}{1.258462in}}{\pgfqpoint{0.687435in}{1.261734in}}{\pgfqpoint{0.679199in}{1.261734in}}%
\pgfpathcurveto{\pgfqpoint{0.670963in}{1.261734in}}{\pgfqpoint{0.663063in}{1.258462in}}{\pgfqpoint{0.657239in}{1.252638in}}%
\pgfpathcurveto{\pgfqpoint{0.651415in}{1.246814in}}{\pgfqpoint{0.648142in}{1.238914in}}{\pgfqpoint{0.648142in}{1.230677in}}%
\pgfpathcurveto{\pgfqpoint{0.648142in}{1.222441in}}{\pgfqpoint{0.651415in}{1.214541in}}{\pgfqpoint{0.657239in}{1.208717in}}%
\pgfpathcurveto{\pgfqpoint{0.663063in}{1.202893in}}{\pgfqpoint{0.670963in}{1.199621in}}{\pgfqpoint{0.679199in}{1.199621in}}%
\pgfpathclose%
\pgfusepath{stroke,fill}%
\end{pgfscope}%
\begin{pgfscope}%
\pgfpathrectangle{\pgfqpoint{0.100000in}{0.220728in}}{\pgfqpoint{3.696000in}{3.696000in}}%
\pgfusepath{clip}%
\pgfsetbuttcap%
\pgfsetroundjoin%
\definecolor{currentfill}{rgb}{0.121569,0.466667,0.705882}%
\pgfsetfillcolor{currentfill}%
\pgfsetfillopacity{0.619878}%
\pgfsetlinewidth{1.003750pt}%
\definecolor{currentstroke}{rgb}{0.121569,0.466667,0.705882}%
\pgfsetstrokecolor{currentstroke}%
\pgfsetstrokeopacity{0.619878}%
\pgfsetdash{}{0pt}%
\pgfpathmoveto{\pgfqpoint{0.679199in}{1.199621in}}%
\pgfpathcurveto{\pgfqpoint{0.687435in}{1.199621in}}{\pgfqpoint{0.695335in}{1.202893in}}{\pgfqpoint{0.701159in}{1.208717in}}%
\pgfpathcurveto{\pgfqpoint{0.706983in}{1.214541in}}{\pgfqpoint{0.710255in}{1.222441in}}{\pgfqpoint{0.710255in}{1.230677in}}%
\pgfpathcurveto{\pgfqpoint{0.710255in}{1.238914in}}{\pgfqpoint{0.706983in}{1.246814in}}{\pgfqpoint{0.701159in}{1.252638in}}%
\pgfpathcurveto{\pgfqpoint{0.695335in}{1.258462in}}{\pgfqpoint{0.687435in}{1.261734in}}{\pgfqpoint{0.679199in}{1.261734in}}%
\pgfpathcurveto{\pgfqpoint{0.670963in}{1.261734in}}{\pgfqpoint{0.663063in}{1.258462in}}{\pgfqpoint{0.657239in}{1.252638in}}%
\pgfpathcurveto{\pgfqpoint{0.651415in}{1.246814in}}{\pgfqpoint{0.648142in}{1.238914in}}{\pgfqpoint{0.648142in}{1.230677in}}%
\pgfpathcurveto{\pgfqpoint{0.648142in}{1.222441in}}{\pgfqpoint{0.651415in}{1.214541in}}{\pgfqpoint{0.657239in}{1.208717in}}%
\pgfpathcurveto{\pgfqpoint{0.663063in}{1.202893in}}{\pgfqpoint{0.670963in}{1.199621in}}{\pgfqpoint{0.679199in}{1.199621in}}%
\pgfpathclose%
\pgfusepath{stroke,fill}%
\end{pgfscope}%
\begin{pgfscope}%
\pgfpathrectangle{\pgfqpoint{0.100000in}{0.220728in}}{\pgfqpoint{3.696000in}{3.696000in}}%
\pgfusepath{clip}%
\pgfsetbuttcap%
\pgfsetroundjoin%
\definecolor{currentfill}{rgb}{0.121569,0.466667,0.705882}%
\pgfsetfillcolor{currentfill}%
\pgfsetfillopacity{0.619878}%
\pgfsetlinewidth{1.003750pt}%
\definecolor{currentstroke}{rgb}{0.121569,0.466667,0.705882}%
\pgfsetstrokecolor{currentstroke}%
\pgfsetstrokeopacity{0.619878}%
\pgfsetdash{}{0pt}%
\pgfpathmoveto{\pgfqpoint{0.679199in}{1.199621in}}%
\pgfpathcurveto{\pgfqpoint{0.687435in}{1.199621in}}{\pgfqpoint{0.695335in}{1.202893in}}{\pgfqpoint{0.701159in}{1.208717in}}%
\pgfpathcurveto{\pgfqpoint{0.706983in}{1.214541in}}{\pgfqpoint{0.710255in}{1.222441in}}{\pgfqpoint{0.710255in}{1.230677in}}%
\pgfpathcurveto{\pgfqpoint{0.710255in}{1.238914in}}{\pgfqpoint{0.706983in}{1.246814in}}{\pgfqpoint{0.701159in}{1.252638in}}%
\pgfpathcurveto{\pgfqpoint{0.695335in}{1.258462in}}{\pgfqpoint{0.687435in}{1.261734in}}{\pgfqpoint{0.679199in}{1.261734in}}%
\pgfpathcurveto{\pgfqpoint{0.670963in}{1.261734in}}{\pgfqpoint{0.663063in}{1.258462in}}{\pgfqpoint{0.657239in}{1.252638in}}%
\pgfpathcurveto{\pgfqpoint{0.651415in}{1.246814in}}{\pgfqpoint{0.648142in}{1.238914in}}{\pgfqpoint{0.648142in}{1.230677in}}%
\pgfpathcurveto{\pgfqpoint{0.648142in}{1.222441in}}{\pgfqpoint{0.651415in}{1.214541in}}{\pgfqpoint{0.657239in}{1.208717in}}%
\pgfpathcurveto{\pgfqpoint{0.663063in}{1.202893in}}{\pgfqpoint{0.670963in}{1.199621in}}{\pgfqpoint{0.679199in}{1.199621in}}%
\pgfpathclose%
\pgfusepath{stroke,fill}%
\end{pgfscope}%
\begin{pgfscope}%
\pgfpathrectangle{\pgfqpoint{0.100000in}{0.220728in}}{\pgfqpoint{3.696000in}{3.696000in}}%
\pgfusepath{clip}%
\pgfsetbuttcap%
\pgfsetroundjoin%
\definecolor{currentfill}{rgb}{0.121569,0.466667,0.705882}%
\pgfsetfillcolor{currentfill}%
\pgfsetfillopacity{0.619878}%
\pgfsetlinewidth{1.003750pt}%
\definecolor{currentstroke}{rgb}{0.121569,0.466667,0.705882}%
\pgfsetstrokecolor{currentstroke}%
\pgfsetstrokeopacity{0.619878}%
\pgfsetdash{}{0pt}%
\pgfpathmoveto{\pgfqpoint{0.679199in}{1.199621in}}%
\pgfpathcurveto{\pgfqpoint{0.687435in}{1.199621in}}{\pgfqpoint{0.695335in}{1.202893in}}{\pgfqpoint{0.701159in}{1.208717in}}%
\pgfpathcurveto{\pgfqpoint{0.706983in}{1.214541in}}{\pgfqpoint{0.710255in}{1.222441in}}{\pgfqpoint{0.710255in}{1.230677in}}%
\pgfpathcurveto{\pgfqpoint{0.710255in}{1.238914in}}{\pgfqpoint{0.706983in}{1.246814in}}{\pgfqpoint{0.701159in}{1.252638in}}%
\pgfpathcurveto{\pgfqpoint{0.695335in}{1.258462in}}{\pgfqpoint{0.687435in}{1.261734in}}{\pgfqpoint{0.679199in}{1.261734in}}%
\pgfpathcurveto{\pgfqpoint{0.670963in}{1.261734in}}{\pgfqpoint{0.663063in}{1.258462in}}{\pgfqpoint{0.657239in}{1.252638in}}%
\pgfpathcurveto{\pgfqpoint{0.651415in}{1.246814in}}{\pgfqpoint{0.648142in}{1.238914in}}{\pgfqpoint{0.648142in}{1.230677in}}%
\pgfpathcurveto{\pgfqpoint{0.648142in}{1.222441in}}{\pgfqpoint{0.651415in}{1.214541in}}{\pgfqpoint{0.657239in}{1.208717in}}%
\pgfpathcurveto{\pgfqpoint{0.663063in}{1.202893in}}{\pgfqpoint{0.670963in}{1.199621in}}{\pgfqpoint{0.679199in}{1.199621in}}%
\pgfpathclose%
\pgfusepath{stroke,fill}%
\end{pgfscope}%
\begin{pgfscope}%
\pgfpathrectangle{\pgfqpoint{0.100000in}{0.220728in}}{\pgfqpoint{3.696000in}{3.696000in}}%
\pgfusepath{clip}%
\pgfsetbuttcap%
\pgfsetroundjoin%
\definecolor{currentfill}{rgb}{0.121569,0.466667,0.705882}%
\pgfsetfillcolor{currentfill}%
\pgfsetfillopacity{0.619878}%
\pgfsetlinewidth{1.003750pt}%
\definecolor{currentstroke}{rgb}{0.121569,0.466667,0.705882}%
\pgfsetstrokecolor{currentstroke}%
\pgfsetstrokeopacity{0.619878}%
\pgfsetdash{}{0pt}%
\pgfpathmoveto{\pgfqpoint{0.679199in}{1.199621in}}%
\pgfpathcurveto{\pgfqpoint{0.687435in}{1.199621in}}{\pgfqpoint{0.695335in}{1.202893in}}{\pgfqpoint{0.701159in}{1.208717in}}%
\pgfpathcurveto{\pgfqpoint{0.706983in}{1.214541in}}{\pgfqpoint{0.710255in}{1.222441in}}{\pgfqpoint{0.710255in}{1.230677in}}%
\pgfpathcurveto{\pgfqpoint{0.710255in}{1.238914in}}{\pgfqpoint{0.706983in}{1.246814in}}{\pgfqpoint{0.701159in}{1.252638in}}%
\pgfpathcurveto{\pgfqpoint{0.695335in}{1.258462in}}{\pgfqpoint{0.687435in}{1.261734in}}{\pgfqpoint{0.679199in}{1.261734in}}%
\pgfpathcurveto{\pgfqpoint{0.670963in}{1.261734in}}{\pgfqpoint{0.663063in}{1.258462in}}{\pgfqpoint{0.657239in}{1.252638in}}%
\pgfpathcurveto{\pgfqpoint{0.651415in}{1.246814in}}{\pgfqpoint{0.648142in}{1.238914in}}{\pgfqpoint{0.648142in}{1.230677in}}%
\pgfpathcurveto{\pgfqpoint{0.648142in}{1.222441in}}{\pgfqpoint{0.651415in}{1.214541in}}{\pgfqpoint{0.657239in}{1.208717in}}%
\pgfpathcurveto{\pgfqpoint{0.663063in}{1.202893in}}{\pgfqpoint{0.670963in}{1.199621in}}{\pgfqpoint{0.679199in}{1.199621in}}%
\pgfpathclose%
\pgfusepath{stroke,fill}%
\end{pgfscope}%
\begin{pgfscope}%
\pgfpathrectangle{\pgfqpoint{0.100000in}{0.220728in}}{\pgfqpoint{3.696000in}{3.696000in}}%
\pgfusepath{clip}%
\pgfsetbuttcap%
\pgfsetroundjoin%
\definecolor{currentfill}{rgb}{0.121569,0.466667,0.705882}%
\pgfsetfillcolor{currentfill}%
\pgfsetfillopacity{0.619878}%
\pgfsetlinewidth{1.003750pt}%
\definecolor{currentstroke}{rgb}{0.121569,0.466667,0.705882}%
\pgfsetstrokecolor{currentstroke}%
\pgfsetstrokeopacity{0.619878}%
\pgfsetdash{}{0pt}%
\pgfpathmoveto{\pgfqpoint{0.679199in}{1.199621in}}%
\pgfpathcurveto{\pgfqpoint{0.687435in}{1.199621in}}{\pgfqpoint{0.695335in}{1.202893in}}{\pgfqpoint{0.701159in}{1.208717in}}%
\pgfpathcurveto{\pgfqpoint{0.706983in}{1.214541in}}{\pgfqpoint{0.710255in}{1.222441in}}{\pgfqpoint{0.710255in}{1.230677in}}%
\pgfpathcurveto{\pgfqpoint{0.710255in}{1.238914in}}{\pgfqpoint{0.706983in}{1.246814in}}{\pgfqpoint{0.701159in}{1.252638in}}%
\pgfpathcurveto{\pgfqpoint{0.695335in}{1.258462in}}{\pgfqpoint{0.687435in}{1.261734in}}{\pgfqpoint{0.679199in}{1.261734in}}%
\pgfpathcurveto{\pgfqpoint{0.670963in}{1.261734in}}{\pgfqpoint{0.663063in}{1.258462in}}{\pgfqpoint{0.657239in}{1.252638in}}%
\pgfpathcurveto{\pgfqpoint{0.651415in}{1.246814in}}{\pgfqpoint{0.648142in}{1.238914in}}{\pgfqpoint{0.648142in}{1.230677in}}%
\pgfpathcurveto{\pgfqpoint{0.648142in}{1.222441in}}{\pgfqpoint{0.651415in}{1.214541in}}{\pgfqpoint{0.657239in}{1.208717in}}%
\pgfpathcurveto{\pgfqpoint{0.663063in}{1.202893in}}{\pgfqpoint{0.670963in}{1.199621in}}{\pgfqpoint{0.679199in}{1.199621in}}%
\pgfpathclose%
\pgfusepath{stroke,fill}%
\end{pgfscope}%
\begin{pgfscope}%
\pgfpathrectangle{\pgfqpoint{0.100000in}{0.220728in}}{\pgfqpoint{3.696000in}{3.696000in}}%
\pgfusepath{clip}%
\pgfsetbuttcap%
\pgfsetroundjoin%
\definecolor{currentfill}{rgb}{0.121569,0.466667,0.705882}%
\pgfsetfillcolor{currentfill}%
\pgfsetfillopacity{0.619878}%
\pgfsetlinewidth{1.003750pt}%
\definecolor{currentstroke}{rgb}{0.121569,0.466667,0.705882}%
\pgfsetstrokecolor{currentstroke}%
\pgfsetstrokeopacity{0.619878}%
\pgfsetdash{}{0pt}%
\pgfpathmoveto{\pgfqpoint{0.679199in}{1.199621in}}%
\pgfpathcurveto{\pgfqpoint{0.687435in}{1.199621in}}{\pgfqpoint{0.695335in}{1.202893in}}{\pgfqpoint{0.701159in}{1.208717in}}%
\pgfpathcurveto{\pgfqpoint{0.706983in}{1.214541in}}{\pgfqpoint{0.710255in}{1.222441in}}{\pgfqpoint{0.710255in}{1.230677in}}%
\pgfpathcurveto{\pgfqpoint{0.710255in}{1.238914in}}{\pgfqpoint{0.706983in}{1.246814in}}{\pgfqpoint{0.701159in}{1.252638in}}%
\pgfpathcurveto{\pgfqpoint{0.695335in}{1.258462in}}{\pgfqpoint{0.687435in}{1.261734in}}{\pgfqpoint{0.679199in}{1.261734in}}%
\pgfpathcurveto{\pgfqpoint{0.670963in}{1.261734in}}{\pgfqpoint{0.663063in}{1.258462in}}{\pgfqpoint{0.657239in}{1.252638in}}%
\pgfpathcurveto{\pgfqpoint{0.651415in}{1.246814in}}{\pgfqpoint{0.648142in}{1.238914in}}{\pgfqpoint{0.648142in}{1.230677in}}%
\pgfpathcurveto{\pgfqpoint{0.648142in}{1.222441in}}{\pgfqpoint{0.651415in}{1.214541in}}{\pgfqpoint{0.657239in}{1.208717in}}%
\pgfpathcurveto{\pgfqpoint{0.663063in}{1.202893in}}{\pgfqpoint{0.670963in}{1.199621in}}{\pgfqpoint{0.679199in}{1.199621in}}%
\pgfpathclose%
\pgfusepath{stroke,fill}%
\end{pgfscope}%
\begin{pgfscope}%
\pgfpathrectangle{\pgfqpoint{0.100000in}{0.220728in}}{\pgfqpoint{3.696000in}{3.696000in}}%
\pgfusepath{clip}%
\pgfsetbuttcap%
\pgfsetroundjoin%
\definecolor{currentfill}{rgb}{0.121569,0.466667,0.705882}%
\pgfsetfillcolor{currentfill}%
\pgfsetfillopacity{0.619878}%
\pgfsetlinewidth{1.003750pt}%
\definecolor{currentstroke}{rgb}{0.121569,0.466667,0.705882}%
\pgfsetstrokecolor{currentstroke}%
\pgfsetstrokeopacity{0.619878}%
\pgfsetdash{}{0pt}%
\pgfpathmoveto{\pgfqpoint{0.679199in}{1.199621in}}%
\pgfpathcurveto{\pgfqpoint{0.687435in}{1.199621in}}{\pgfqpoint{0.695335in}{1.202893in}}{\pgfqpoint{0.701159in}{1.208717in}}%
\pgfpathcurveto{\pgfqpoint{0.706983in}{1.214541in}}{\pgfqpoint{0.710255in}{1.222441in}}{\pgfqpoint{0.710255in}{1.230677in}}%
\pgfpathcurveto{\pgfqpoint{0.710255in}{1.238914in}}{\pgfqpoint{0.706983in}{1.246814in}}{\pgfqpoint{0.701159in}{1.252638in}}%
\pgfpathcurveto{\pgfqpoint{0.695335in}{1.258462in}}{\pgfqpoint{0.687435in}{1.261734in}}{\pgfqpoint{0.679199in}{1.261734in}}%
\pgfpathcurveto{\pgfqpoint{0.670963in}{1.261734in}}{\pgfqpoint{0.663063in}{1.258462in}}{\pgfqpoint{0.657239in}{1.252638in}}%
\pgfpathcurveto{\pgfqpoint{0.651415in}{1.246814in}}{\pgfqpoint{0.648142in}{1.238914in}}{\pgfqpoint{0.648142in}{1.230677in}}%
\pgfpathcurveto{\pgfqpoint{0.648142in}{1.222441in}}{\pgfqpoint{0.651415in}{1.214541in}}{\pgfqpoint{0.657239in}{1.208717in}}%
\pgfpathcurveto{\pgfqpoint{0.663063in}{1.202893in}}{\pgfqpoint{0.670963in}{1.199621in}}{\pgfqpoint{0.679199in}{1.199621in}}%
\pgfpathclose%
\pgfusepath{stroke,fill}%
\end{pgfscope}%
\begin{pgfscope}%
\pgfpathrectangle{\pgfqpoint{0.100000in}{0.220728in}}{\pgfqpoint{3.696000in}{3.696000in}}%
\pgfusepath{clip}%
\pgfsetbuttcap%
\pgfsetroundjoin%
\definecolor{currentfill}{rgb}{0.121569,0.466667,0.705882}%
\pgfsetfillcolor{currentfill}%
\pgfsetfillopacity{0.619878}%
\pgfsetlinewidth{1.003750pt}%
\definecolor{currentstroke}{rgb}{0.121569,0.466667,0.705882}%
\pgfsetstrokecolor{currentstroke}%
\pgfsetstrokeopacity{0.619878}%
\pgfsetdash{}{0pt}%
\pgfpathmoveto{\pgfqpoint{0.679199in}{1.199621in}}%
\pgfpathcurveto{\pgfqpoint{0.687435in}{1.199621in}}{\pgfqpoint{0.695335in}{1.202893in}}{\pgfqpoint{0.701159in}{1.208717in}}%
\pgfpathcurveto{\pgfqpoint{0.706983in}{1.214541in}}{\pgfqpoint{0.710255in}{1.222441in}}{\pgfqpoint{0.710255in}{1.230677in}}%
\pgfpathcurveto{\pgfqpoint{0.710255in}{1.238914in}}{\pgfqpoint{0.706983in}{1.246814in}}{\pgfqpoint{0.701159in}{1.252638in}}%
\pgfpathcurveto{\pgfqpoint{0.695335in}{1.258462in}}{\pgfqpoint{0.687435in}{1.261734in}}{\pgfqpoint{0.679199in}{1.261734in}}%
\pgfpathcurveto{\pgfqpoint{0.670963in}{1.261734in}}{\pgfqpoint{0.663063in}{1.258462in}}{\pgfqpoint{0.657239in}{1.252638in}}%
\pgfpathcurveto{\pgfqpoint{0.651415in}{1.246814in}}{\pgfqpoint{0.648142in}{1.238914in}}{\pgfqpoint{0.648142in}{1.230677in}}%
\pgfpathcurveto{\pgfqpoint{0.648142in}{1.222441in}}{\pgfqpoint{0.651415in}{1.214541in}}{\pgfqpoint{0.657239in}{1.208717in}}%
\pgfpathcurveto{\pgfqpoint{0.663063in}{1.202893in}}{\pgfqpoint{0.670963in}{1.199621in}}{\pgfqpoint{0.679199in}{1.199621in}}%
\pgfpathclose%
\pgfusepath{stroke,fill}%
\end{pgfscope}%
\begin{pgfscope}%
\pgfpathrectangle{\pgfqpoint{0.100000in}{0.220728in}}{\pgfqpoint{3.696000in}{3.696000in}}%
\pgfusepath{clip}%
\pgfsetbuttcap%
\pgfsetroundjoin%
\definecolor{currentfill}{rgb}{0.121569,0.466667,0.705882}%
\pgfsetfillcolor{currentfill}%
\pgfsetfillopacity{0.619878}%
\pgfsetlinewidth{1.003750pt}%
\definecolor{currentstroke}{rgb}{0.121569,0.466667,0.705882}%
\pgfsetstrokecolor{currentstroke}%
\pgfsetstrokeopacity{0.619878}%
\pgfsetdash{}{0pt}%
\pgfpathmoveto{\pgfqpoint{0.679199in}{1.199621in}}%
\pgfpathcurveto{\pgfqpoint{0.687435in}{1.199621in}}{\pgfqpoint{0.695335in}{1.202893in}}{\pgfqpoint{0.701159in}{1.208717in}}%
\pgfpathcurveto{\pgfqpoint{0.706983in}{1.214541in}}{\pgfqpoint{0.710255in}{1.222441in}}{\pgfqpoint{0.710255in}{1.230677in}}%
\pgfpathcurveto{\pgfqpoint{0.710255in}{1.238914in}}{\pgfqpoint{0.706983in}{1.246814in}}{\pgfqpoint{0.701159in}{1.252638in}}%
\pgfpathcurveto{\pgfqpoint{0.695335in}{1.258462in}}{\pgfqpoint{0.687435in}{1.261734in}}{\pgfqpoint{0.679199in}{1.261734in}}%
\pgfpathcurveto{\pgfqpoint{0.670963in}{1.261734in}}{\pgfqpoint{0.663063in}{1.258462in}}{\pgfqpoint{0.657239in}{1.252638in}}%
\pgfpathcurveto{\pgfqpoint{0.651415in}{1.246814in}}{\pgfqpoint{0.648142in}{1.238914in}}{\pgfqpoint{0.648142in}{1.230677in}}%
\pgfpathcurveto{\pgfqpoint{0.648142in}{1.222441in}}{\pgfqpoint{0.651415in}{1.214541in}}{\pgfqpoint{0.657239in}{1.208717in}}%
\pgfpathcurveto{\pgfqpoint{0.663063in}{1.202893in}}{\pgfqpoint{0.670963in}{1.199621in}}{\pgfqpoint{0.679199in}{1.199621in}}%
\pgfpathclose%
\pgfusepath{stroke,fill}%
\end{pgfscope}%
\begin{pgfscope}%
\pgfpathrectangle{\pgfqpoint{0.100000in}{0.220728in}}{\pgfqpoint{3.696000in}{3.696000in}}%
\pgfusepath{clip}%
\pgfsetbuttcap%
\pgfsetroundjoin%
\definecolor{currentfill}{rgb}{0.121569,0.466667,0.705882}%
\pgfsetfillcolor{currentfill}%
\pgfsetfillopacity{0.619878}%
\pgfsetlinewidth{1.003750pt}%
\definecolor{currentstroke}{rgb}{0.121569,0.466667,0.705882}%
\pgfsetstrokecolor{currentstroke}%
\pgfsetstrokeopacity{0.619878}%
\pgfsetdash{}{0pt}%
\pgfpathmoveto{\pgfqpoint{0.679199in}{1.199621in}}%
\pgfpathcurveto{\pgfqpoint{0.687435in}{1.199621in}}{\pgfqpoint{0.695335in}{1.202893in}}{\pgfqpoint{0.701159in}{1.208717in}}%
\pgfpathcurveto{\pgfqpoint{0.706983in}{1.214541in}}{\pgfqpoint{0.710255in}{1.222441in}}{\pgfqpoint{0.710255in}{1.230677in}}%
\pgfpathcurveto{\pgfqpoint{0.710255in}{1.238914in}}{\pgfqpoint{0.706983in}{1.246814in}}{\pgfqpoint{0.701159in}{1.252638in}}%
\pgfpathcurveto{\pgfqpoint{0.695335in}{1.258462in}}{\pgfqpoint{0.687435in}{1.261734in}}{\pgfqpoint{0.679199in}{1.261734in}}%
\pgfpathcurveto{\pgfqpoint{0.670963in}{1.261734in}}{\pgfqpoint{0.663063in}{1.258462in}}{\pgfqpoint{0.657239in}{1.252638in}}%
\pgfpathcurveto{\pgfqpoint{0.651415in}{1.246814in}}{\pgfqpoint{0.648142in}{1.238914in}}{\pgfqpoint{0.648142in}{1.230677in}}%
\pgfpathcurveto{\pgfqpoint{0.648142in}{1.222441in}}{\pgfqpoint{0.651415in}{1.214541in}}{\pgfqpoint{0.657239in}{1.208717in}}%
\pgfpathcurveto{\pgfqpoint{0.663063in}{1.202893in}}{\pgfqpoint{0.670963in}{1.199621in}}{\pgfqpoint{0.679199in}{1.199621in}}%
\pgfpathclose%
\pgfusepath{stroke,fill}%
\end{pgfscope}%
\begin{pgfscope}%
\pgfpathrectangle{\pgfqpoint{0.100000in}{0.220728in}}{\pgfqpoint{3.696000in}{3.696000in}}%
\pgfusepath{clip}%
\pgfsetbuttcap%
\pgfsetroundjoin%
\definecolor{currentfill}{rgb}{0.121569,0.466667,0.705882}%
\pgfsetfillcolor{currentfill}%
\pgfsetfillopacity{0.619878}%
\pgfsetlinewidth{1.003750pt}%
\definecolor{currentstroke}{rgb}{0.121569,0.466667,0.705882}%
\pgfsetstrokecolor{currentstroke}%
\pgfsetstrokeopacity{0.619878}%
\pgfsetdash{}{0pt}%
\pgfpathmoveto{\pgfqpoint{0.679199in}{1.199621in}}%
\pgfpathcurveto{\pgfqpoint{0.687435in}{1.199621in}}{\pgfqpoint{0.695335in}{1.202893in}}{\pgfqpoint{0.701159in}{1.208717in}}%
\pgfpathcurveto{\pgfqpoint{0.706983in}{1.214541in}}{\pgfqpoint{0.710255in}{1.222441in}}{\pgfqpoint{0.710255in}{1.230677in}}%
\pgfpathcurveto{\pgfqpoint{0.710255in}{1.238914in}}{\pgfqpoint{0.706983in}{1.246814in}}{\pgfqpoint{0.701159in}{1.252638in}}%
\pgfpathcurveto{\pgfqpoint{0.695335in}{1.258462in}}{\pgfqpoint{0.687435in}{1.261734in}}{\pgfqpoint{0.679199in}{1.261734in}}%
\pgfpathcurveto{\pgfqpoint{0.670963in}{1.261734in}}{\pgfqpoint{0.663063in}{1.258462in}}{\pgfqpoint{0.657239in}{1.252638in}}%
\pgfpathcurveto{\pgfqpoint{0.651415in}{1.246814in}}{\pgfqpoint{0.648142in}{1.238914in}}{\pgfqpoint{0.648142in}{1.230677in}}%
\pgfpathcurveto{\pgfqpoint{0.648142in}{1.222441in}}{\pgfqpoint{0.651415in}{1.214541in}}{\pgfqpoint{0.657239in}{1.208717in}}%
\pgfpathcurveto{\pgfqpoint{0.663063in}{1.202893in}}{\pgfqpoint{0.670963in}{1.199621in}}{\pgfqpoint{0.679199in}{1.199621in}}%
\pgfpathclose%
\pgfusepath{stroke,fill}%
\end{pgfscope}%
\begin{pgfscope}%
\pgfpathrectangle{\pgfqpoint{0.100000in}{0.220728in}}{\pgfqpoint{3.696000in}{3.696000in}}%
\pgfusepath{clip}%
\pgfsetbuttcap%
\pgfsetroundjoin%
\definecolor{currentfill}{rgb}{0.121569,0.466667,0.705882}%
\pgfsetfillcolor{currentfill}%
\pgfsetfillopacity{0.619878}%
\pgfsetlinewidth{1.003750pt}%
\definecolor{currentstroke}{rgb}{0.121569,0.466667,0.705882}%
\pgfsetstrokecolor{currentstroke}%
\pgfsetstrokeopacity{0.619878}%
\pgfsetdash{}{0pt}%
\pgfpathmoveto{\pgfqpoint{0.679199in}{1.199621in}}%
\pgfpathcurveto{\pgfqpoint{0.687435in}{1.199621in}}{\pgfqpoint{0.695335in}{1.202893in}}{\pgfqpoint{0.701159in}{1.208717in}}%
\pgfpathcurveto{\pgfqpoint{0.706983in}{1.214541in}}{\pgfqpoint{0.710255in}{1.222441in}}{\pgfqpoint{0.710255in}{1.230677in}}%
\pgfpathcurveto{\pgfqpoint{0.710255in}{1.238914in}}{\pgfqpoint{0.706983in}{1.246814in}}{\pgfqpoint{0.701159in}{1.252638in}}%
\pgfpathcurveto{\pgfqpoint{0.695335in}{1.258462in}}{\pgfqpoint{0.687435in}{1.261734in}}{\pgfqpoint{0.679199in}{1.261734in}}%
\pgfpathcurveto{\pgfqpoint{0.670963in}{1.261734in}}{\pgfqpoint{0.663063in}{1.258462in}}{\pgfqpoint{0.657239in}{1.252638in}}%
\pgfpathcurveto{\pgfqpoint{0.651415in}{1.246814in}}{\pgfqpoint{0.648142in}{1.238914in}}{\pgfqpoint{0.648142in}{1.230677in}}%
\pgfpathcurveto{\pgfqpoint{0.648142in}{1.222441in}}{\pgfqpoint{0.651415in}{1.214541in}}{\pgfqpoint{0.657239in}{1.208717in}}%
\pgfpathcurveto{\pgfqpoint{0.663063in}{1.202893in}}{\pgfqpoint{0.670963in}{1.199621in}}{\pgfqpoint{0.679199in}{1.199621in}}%
\pgfpathclose%
\pgfusepath{stroke,fill}%
\end{pgfscope}%
\begin{pgfscope}%
\pgfpathrectangle{\pgfqpoint{0.100000in}{0.220728in}}{\pgfqpoint{3.696000in}{3.696000in}}%
\pgfusepath{clip}%
\pgfsetbuttcap%
\pgfsetroundjoin%
\definecolor{currentfill}{rgb}{0.121569,0.466667,0.705882}%
\pgfsetfillcolor{currentfill}%
\pgfsetfillopacity{0.619878}%
\pgfsetlinewidth{1.003750pt}%
\definecolor{currentstroke}{rgb}{0.121569,0.466667,0.705882}%
\pgfsetstrokecolor{currentstroke}%
\pgfsetstrokeopacity{0.619878}%
\pgfsetdash{}{0pt}%
\pgfpathmoveto{\pgfqpoint{0.679199in}{1.199621in}}%
\pgfpathcurveto{\pgfqpoint{0.687435in}{1.199621in}}{\pgfqpoint{0.695335in}{1.202893in}}{\pgfqpoint{0.701159in}{1.208717in}}%
\pgfpathcurveto{\pgfqpoint{0.706983in}{1.214541in}}{\pgfqpoint{0.710255in}{1.222441in}}{\pgfqpoint{0.710255in}{1.230677in}}%
\pgfpathcurveto{\pgfqpoint{0.710255in}{1.238914in}}{\pgfqpoint{0.706983in}{1.246814in}}{\pgfqpoint{0.701159in}{1.252638in}}%
\pgfpathcurveto{\pgfqpoint{0.695335in}{1.258462in}}{\pgfqpoint{0.687435in}{1.261734in}}{\pgfqpoint{0.679199in}{1.261734in}}%
\pgfpathcurveto{\pgfqpoint{0.670963in}{1.261734in}}{\pgfqpoint{0.663063in}{1.258462in}}{\pgfqpoint{0.657239in}{1.252638in}}%
\pgfpathcurveto{\pgfqpoint{0.651415in}{1.246814in}}{\pgfqpoint{0.648142in}{1.238914in}}{\pgfqpoint{0.648142in}{1.230677in}}%
\pgfpathcurveto{\pgfqpoint{0.648142in}{1.222441in}}{\pgfqpoint{0.651415in}{1.214541in}}{\pgfqpoint{0.657239in}{1.208717in}}%
\pgfpathcurveto{\pgfqpoint{0.663063in}{1.202893in}}{\pgfqpoint{0.670963in}{1.199621in}}{\pgfqpoint{0.679199in}{1.199621in}}%
\pgfpathclose%
\pgfusepath{stroke,fill}%
\end{pgfscope}%
\begin{pgfscope}%
\pgfpathrectangle{\pgfqpoint{0.100000in}{0.220728in}}{\pgfqpoint{3.696000in}{3.696000in}}%
\pgfusepath{clip}%
\pgfsetbuttcap%
\pgfsetroundjoin%
\definecolor{currentfill}{rgb}{0.121569,0.466667,0.705882}%
\pgfsetfillcolor{currentfill}%
\pgfsetfillopacity{0.619878}%
\pgfsetlinewidth{1.003750pt}%
\definecolor{currentstroke}{rgb}{0.121569,0.466667,0.705882}%
\pgfsetstrokecolor{currentstroke}%
\pgfsetstrokeopacity{0.619878}%
\pgfsetdash{}{0pt}%
\pgfpathmoveto{\pgfqpoint{0.679199in}{1.199621in}}%
\pgfpathcurveto{\pgfqpoint{0.687435in}{1.199621in}}{\pgfqpoint{0.695335in}{1.202893in}}{\pgfqpoint{0.701159in}{1.208717in}}%
\pgfpathcurveto{\pgfqpoint{0.706983in}{1.214541in}}{\pgfqpoint{0.710255in}{1.222441in}}{\pgfqpoint{0.710255in}{1.230677in}}%
\pgfpathcurveto{\pgfqpoint{0.710255in}{1.238914in}}{\pgfqpoint{0.706983in}{1.246814in}}{\pgfqpoint{0.701159in}{1.252638in}}%
\pgfpathcurveto{\pgfqpoint{0.695335in}{1.258462in}}{\pgfqpoint{0.687435in}{1.261734in}}{\pgfqpoint{0.679199in}{1.261734in}}%
\pgfpathcurveto{\pgfqpoint{0.670963in}{1.261734in}}{\pgfqpoint{0.663063in}{1.258462in}}{\pgfqpoint{0.657239in}{1.252638in}}%
\pgfpathcurveto{\pgfqpoint{0.651415in}{1.246814in}}{\pgfqpoint{0.648142in}{1.238914in}}{\pgfqpoint{0.648142in}{1.230677in}}%
\pgfpathcurveto{\pgfqpoint{0.648142in}{1.222441in}}{\pgfqpoint{0.651415in}{1.214541in}}{\pgfqpoint{0.657239in}{1.208717in}}%
\pgfpathcurveto{\pgfqpoint{0.663063in}{1.202893in}}{\pgfqpoint{0.670963in}{1.199621in}}{\pgfqpoint{0.679199in}{1.199621in}}%
\pgfpathclose%
\pgfusepath{stroke,fill}%
\end{pgfscope}%
\begin{pgfscope}%
\pgfpathrectangle{\pgfqpoint{0.100000in}{0.220728in}}{\pgfqpoint{3.696000in}{3.696000in}}%
\pgfusepath{clip}%
\pgfsetbuttcap%
\pgfsetroundjoin%
\definecolor{currentfill}{rgb}{0.121569,0.466667,0.705882}%
\pgfsetfillcolor{currentfill}%
\pgfsetfillopacity{0.619878}%
\pgfsetlinewidth{1.003750pt}%
\definecolor{currentstroke}{rgb}{0.121569,0.466667,0.705882}%
\pgfsetstrokecolor{currentstroke}%
\pgfsetstrokeopacity{0.619878}%
\pgfsetdash{}{0pt}%
\pgfpathmoveto{\pgfqpoint{0.679199in}{1.199621in}}%
\pgfpathcurveto{\pgfqpoint{0.687435in}{1.199621in}}{\pgfqpoint{0.695335in}{1.202893in}}{\pgfqpoint{0.701159in}{1.208717in}}%
\pgfpathcurveto{\pgfqpoint{0.706983in}{1.214541in}}{\pgfqpoint{0.710255in}{1.222441in}}{\pgfqpoint{0.710255in}{1.230677in}}%
\pgfpathcurveto{\pgfqpoint{0.710255in}{1.238914in}}{\pgfqpoint{0.706983in}{1.246814in}}{\pgfqpoint{0.701159in}{1.252638in}}%
\pgfpathcurveto{\pgfqpoint{0.695335in}{1.258462in}}{\pgfqpoint{0.687435in}{1.261734in}}{\pgfqpoint{0.679199in}{1.261734in}}%
\pgfpathcurveto{\pgfqpoint{0.670963in}{1.261734in}}{\pgfqpoint{0.663063in}{1.258462in}}{\pgfqpoint{0.657239in}{1.252638in}}%
\pgfpathcurveto{\pgfqpoint{0.651415in}{1.246814in}}{\pgfqpoint{0.648142in}{1.238914in}}{\pgfqpoint{0.648142in}{1.230677in}}%
\pgfpathcurveto{\pgfqpoint{0.648142in}{1.222441in}}{\pgfqpoint{0.651415in}{1.214541in}}{\pgfqpoint{0.657239in}{1.208717in}}%
\pgfpathcurveto{\pgfqpoint{0.663063in}{1.202893in}}{\pgfqpoint{0.670963in}{1.199621in}}{\pgfqpoint{0.679199in}{1.199621in}}%
\pgfpathclose%
\pgfusepath{stroke,fill}%
\end{pgfscope}%
\begin{pgfscope}%
\pgfpathrectangle{\pgfqpoint{0.100000in}{0.220728in}}{\pgfqpoint{3.696000in}{3.696000in}}%
\pgfusepath{clip}%
\pgfsetbuttcap%
\pgfsetroundjoin%
\definecolor{currentfill}{rgb}{0.121569,0.466667,0.705882}%
\pgfsetfillcolor{currentfill}%
\pgfsetfillopacity{0.619878}%
\pgfsetlinewidth{1.003750pt}%
\definecolor{currentstroke}{rgb}{0.121569,0.466667,0.705882}%
\pgfsetstrokecolor{currentstroke}%
\pgfsetstrokeopacity{0.619878}%
\pgfsetdash{}{0pt}%
\pgfpathmoveto{\pgfqpoint{0.679199in}{1.199621in}}%
\pgfpathcurveto{\pgfqpoint{0.687435in}{1.199621in}}{\pgfqpoint{0.695335in}{1.202893in}}{\pgfqpoint{0.701159in}{1.208717in}}%
\pgfpathcurveto{\pgfqpoint{0.706983in}{1.214541in}}{\pgfqpoint{0.710255in}{1.222441in}}{\pgfqpoint{0.710255in}{1.230677in}}%
\pgfpathcurveto{\pgfqpoint{0.710255in}{1.238914in}}{\pgfqpoint{0.706983in}{1.246814in}}{\pgfqpoint{0.701159in}{1.252638in}}%
\pgfpathcurveto{\pgfqpoint{0.695335in}{1.258462in}}{\pgfqpoint{0.687435in}{1.261734in}}{\pgfqpoint{0.679199in}{1.261734in}}%
\pgfpathcurveto{\pgfqpoint{0.670963in}{1.261734in}}{\pgfqpoint{0.663063in}{1.258462in}}{\pgfqpoint{0.657239in}{1.252638in}}%
\pgfpathcurveto{\pgfqpoint{0.651415in}{1.246814in}}{\pgfqpoint{0.648142in}{1.238914in}}{\pgfqpoint{0.648142in}{1.230677in}}%
\pgfpathcurveto{\pgfqpoint{0.648142in}{1.222441in}}{\pgfqpoint{0.651415in}{1.214541in}}{\pgfqpoint{0.657239in}{1.208717in}}%
\pgfpathcurveto{\pgfqpoint{0.663063in}{1.202893in}}{\pgfqpoint{0.670963in}{1.199621in}}{\pgfqpoint{0.679199in}{1.199621in}}%
\pgfpathclose%
\pgfusepath{stroke,fill}%
\end{pgfscope}%
\begin{pgfscope}%
\pgfpathrectangle{\pgfqpoint{0.100000in}{0.220728in}}{\pgfqpoint{3.696000in}{3.696000in}}%
\pgfusepath{clip}%
\pgfsetbuttcap%
\pgfsetroundjoin%
\definecolor{currentfill}{rgb}{0.121569,0.466667,0.705882}%
\pgfsetfillcolor{currentfill}%
\pgfsetfillopacity{0.619878}%
\pgfsetlinewidth{1.003750pt}%
\definecolor{currentstroke}{rgb}{0.121569,0.466667,0.705882}%
\pgfsetstrokecolor{currentstroke}%
\pgfsetstrokeopacity{0.619878}%
\pgfsetdash{}{0pt}%
\pgfpathmoveto{\pgfqpoint{0.679199in}{1.199621in}}%
\pgfpathcurveto{\pgfqpoint{0.687435in}{1.199621in}}{\pgfqpoint{0.695335in}{1.202893in}}{\pgfqpoint{0.701159in}{1.208717in}}%
\pgfpathcurveto{\pgfqpoint{0.706983in}{1.214541in}}{\pgfqpoint{0.710255in}{1.222441in}}{\pgfqpoint{0.710255in}{1.230677in}}%
\pgfpathcurveto{\pgfqpoint{0.710255in}{1.238914in}}{\pgfqpoint{0.706983in}{1.246814in}}{\pgfqpoint{0.701159in}{1.252638in}}%
\pgfpathcurveto{\pgfqpoint{0.695335in}{1.258462in}}{\pgfqpoint{0.687435in}{1.261734in}}{\pgfqpoint{0.679199in}{1.261734in}}%
\pgfpathcurveto{\pgfqpoint{0.670963in}{1.261734in}}{\pgfqpoint{0.663063in}{1.258462in}}{\pgfqpoint{0.657239in}{1.252638in}}%
\pgfpathcurveto{\pgfqpoint{0.651415in}{1.246814in}}{\pgfqpoint{0.648142in}{1.238914in}}{\pgfqpoint{0.648142in}{1.230677in}}%
\pgfpathcurveto{\pgfqpoint{0.648142in}{1.222441in}}{\pgfqpoint{0.651415in}{1.214541in}}{\pgfqpoint{0.657239in}{1.208717in}}%
\pgfpathcurveto{\pgfqpoint{0.663063in}{1.202893in}}{\pgfqpoint{0.670963in}{1.199621in}}{\pgfqpoint{0.679199in}{1.199621in}}%
\pgfpathclose%
\pgfusepath{stroke,fill}%
\end{pgfscope}%
\begin{pgfscope}%
\pgfpathrectangle{\pgfqpoint{0.100000in}{0.220728in}}{\pgfqpoint{3.696000in}{3.696000in}}%
\pgfusepath{clip}%
\pgfsetbuttcap%
\pgfsetroundjoin%
\definecolor{currentfill}{rgb}{0.121569,0.466667,0.705882}%
\pgfsetfillcolor{currentfill}%
\pgfsetfillopacity{0.619878}%
\pgfsetlinewidth{1.003750pt}%
\definecolor{currentstroke}{rgb}{0.121569,0.466667,0.705882}%
\pgfsetstrokecolor{currentstroke}%
\pgfsetstrokeopacity{0.619878}%
\pgfsetdash{}{0pt}%
\pgfpathmoveto{\pgfqpoint{0.679199in}{1.199621in}}%
\pgfpathcurveto{\pgfqpoint{0.687435in}{1.199621in}}{\pgfqpoint{0.695335in}{1.202893in}}{\pgfqpoint{0.701159in}{1.208717in}}%
\pgfpathcurveto{\pgfqpoint{0.706983in}{1.214541in}}{\pgfqpoint{0.710255in}{1.222441in}}{\pgfqpoint{0.710255in}{1.230677in}}%
\pgfpathcurveto{\pgfqpoint{0.710255in}{1.238914in}}{\pgfqpoint{0.706983in}{1.246814in}}{\pgfqpoint{0.701159in}{1.252638in}}%
\pgfpathcurveto{\pgfqpoint{0.695335in}{1.258462in}}{\pgfqpoint{0.687435in}{1.261734in}}{\pgfqpoint{0.679199in}{1.261734in}}%
\pgfpathcurveto{\pgfqpoint{0.670963in}{1.261734in}}{\pgfqpoint{0.663063in}{1.258462in}}{\pgfqpoint{0.657239in}{1.252638in}}%
\pgfpathcurveto{\pgfqpoint{0.651415in}{1.246814in}}{\pgfqpoint{0.648142in}{1.238914in}}{\pgfqpoint{0.648142in}{1.230677in}}%
\pgfpathcurveto{\pgfqpoint{0.648142in}{1.222441in}}{\pgfqpoint{0.651415in}{1.214541in}}{\pgfqpoint{0.657239in}{1.208717in}}%
\pgfpathcurveto{\pgfqpoint{0.663063in}{1.202893in}}{\pgfqpoint{0.670963in}{1.199621in}}{\pgfqpoint{0.679199in}{1.199621in}}%
\pgfpathclose%
\pgfusepath{stroke,fill}%
\end{pgfscope}%
\begin{pgfscope}%
\pgfpathrectangle{\pgfqpoint{0.100000in}{0.220728in}}{\pgfqpoint{3.696000in}{3.696000in}}%
\pgfusepath{clip}%
\pgfsetbuttcap%
\pgfsetroundjoin%
\definecolor{currentfill}{rgb}{0.121569,0.466667,0.705882}%
\pgfsetfillcolor{currentfill}%
\pgfsetfillopacity{0.619878}%
\pgfsetlinewidth{1.003750pt}%
\definecolor{currentstroke}{rgb}{0.121569,0.466667,0.705882}%
\pgfsetstrokecolor{currentstroke}%
\pgfsetstrokeopacity{0.619878}%
\pgfsetdash{}{0pt}%
\pgfpathmoveto{\pgfqpoint{0.679199in}{1.199621in}}%
\pgfpathcurveto{\pgfqpoint{0.687435in}{1.199621in}}{\pgfqpoint{0.695335in}{1.202893in}}{\pgfqpoint{0.701159in}{1.208717in}}%
\pgfpathcurveto{\pgfqpoint{0.706983in}{1.214541in}}{\pgfqpoint{0.710255in}{1.222441in}}{\pgfqpoint{0.710255in}{1.230677in}}%
\pgfpathcurveto{\pgfqpoint{0.710255in}{1.238914in}}{\pgfqpoint{0.706983in}{1.246814in}}{\pgfqpoint{0.701159in}{1.252638in}}%
\pgfpathcurveto{\pgfqpoint{0.695335in}{1.258462in}}{\pgfqpoint{0.687435in}{1.261734in}}{\pgfqpoint{0.679199in}{1.261734in}}%
\pgfpathcurveto{\pgfqpoint{0.670963in}{1.261734in}}{\pgfqpoint{0.663063in}{1.258462in}}{\pgfqpoint{0.657239in}{1.252638in}}%
\pgfpathcurveto{\pgfqpoint{0.651415in}{1.246814in}}{\pgfqpoint{0.648142in}{1.238914in}}{\pgfqpoint{0.648142in}{1.230677in}}%
\pgfpathcurveto{\pgfqpoint{0.648142in}{1.222441in}}{\pgfqpoint{0.651415in}{1.214541in}}{\pgfqpoint{0.657239in}{1.208717in}}%
\pgfpathcurveto{\pgfqpoint{0.663063in}{1.202893in}}{\pgfqpoint{0.670963in}{1.199621in}}{\pgfqpoint{0.679199in}{1.199621in}}%
\pgfpathclose%
\pgfusepath{stroke,fill}%
\end{pgfscope}%
\begin{pgfscope}%
\pgfpathrectangle{\pgfqpoint{0.100000in}{0.220728in}}{\pgfqpoint{3.696000in}{3.696000in}}%
\pgfusepath{clip}%
\pgfsetbuttcap%
\pgfsetroundjoin%
\definecolor{currentfill}{rgb}{0.121569,0.466667,0.705882}%
\pgfsetfillcolor{currentfill}%
\pgfsetfillopacity{0.619878}%
\pgfsetlinewidth{1.003750pt}%
\definecolor{currentstroke}{rgb}{0.121569,0.466667,0.705882}%
\pgfsetstrokecolor{currentstroke}%
\pgfsetstrokeopacity{0.619878}%
\pgfsetdash{}{0pt}%
\pgfpathmoveto{\pgfqpoint{0.679199in}{1.199621in}}%
\pgfpathcurveto{\pgfqpoint{0.687435in}{1.199621in}}{\pgfqpoint{0.695335in}{1.202893in}}{\pgfqpoint{0.701159in}{1.208717in}}%
\pgfpathcurveto{\pgfqpoint{0.706983in}{1.214541in}}{\pgfqpoint{0.710255in}{1.222441in}}{\pgfqpoint{0.710255in}{1.230677in}}%
\pgfpathcurveto{\pgfqpoint{0.710255in}{1.238914in}}{\pgfqpoint{0.706983in}{1.246814in}}{\pgfqpoint{0.701159in}{1.252638in}}%
\pgfpathcurveto{\pgfqpoint{0.695335in}{1.258462in}}{\pgfqpoint{0.687435in}{1.261734in}}{\pgfqpoint{0.679199in}{1.261734in}}%
\pgfpathcurveto{\pgfqpoint{0.670963in}{1.261734in}}{\pgfqpoint{0.663063in}{1.258462in}}{\pgfqpoint{0.657239in}{1.252638in}}%
\pgfpathcurveto{\pgfqpoint{0.651415in}{1.246814in}}{\pgfqpoint{0.648142in}{1.238914in}}{\pgfqpoint{0.648142in}{1.230677in}}%
\pgfpathcurveto{\pgfqpoint{0.648142in}{1.222441in}}{\pgfqpoint{0.651415in}{1.214541in}}{\pgfqpoint{0.657239in}{1.208717in}}%
\pgfpathcurveto{\pgfqpoint{0.663063in}{1.202893in}}{\pgfqpoint{0.670963in}{1.199621in}}{\pgfqpoint{0.679199in}{1.199621in}}%
\pgfpathclose%
\pgfusepath{stroke,fill}%
\end{pgfscope}%
\begin{pgfscope}%
\pgfpathrectangle{\pgfqpoint{0.100000in}{0.220728in}}{\pgfqpoint{3.696000in}{3.696000in}}%
\pgfusepath{clip}%
\pgfsetbuttcap%
\pgfsetroundjoin%
\definecolor{currentfill}{rgb}{0.121569,0.466667,0.705882}%
\pgfsetfillcolor{currentfill}%
\pgfsetfillopacity{0.619878}%
\pgfsetlinewidth{1.003750pt}%
\definecolor{currentstroke}{rgb}{0.121569,0.466667,0.705882}%
\pgfsetstrokecolor{currentstroke}%
\pgfsetstrokeopacity{0.619878}%
\pgfsetdash{}{0pt}%
\pgfpathmoveto{\pgfqpoint{0.679199in}{1.199621in}}%
\pgfpathcurveto{\pgfqpoint{0.687435in}{1.199621in}}{\pgfqpoint{0.695335in}{1.202893in}}{\pgfqpoint{0.701159in}{1.208717in}}%
\pgfpathcurveto{\pgfqpoint{0.706983in}{1.214541in}}{\pgfqpoint{0.710255in}{1.222441in}}{\pgfqpoint{0.710255in}{1.230677in}}%
\pgfpathcurveto{\pgfqpoint{0.710255in}{1.238914in}}{\pgfqpoint{0.706983in}{1.246814in}}{\pgfqpoint{0.701159in}{1.252638in}}%
\pgfpathcurveto{\pgfqpoint{0.695335in}{1.258462in}}{\pgfqpoint{0.687435in}{1.261734in}}{\pgfqpoint{0.679199in}{1.261734in}}%
\pgfpathcurveto{\pgfqpoint{0.670963in}{1.261734in}}{\pgfqpoint{0.663063in}{1.258462in}}{\pgfqpoint{0.657239in}{1.252638in}}%
\pgfpathcurveto{\pgfqpoint{0.651415in}{1.246814in}}{\pgfqpoint{0.648142in}{1.238914in}}{\pgfqpoint{0.648142in}{1.230677in}}%
\pgfpathcurveto{\pgfqpoint{0.648142in}{1.222441in}}{\pgfqpoint{0.651415in}{1.214541in}}{\pgfqpoint{0.657239in}{1.208717in}}%
\pgfpathcurveto{\pgfqpoint{0.663063in}{1.202893in}}{\pgfqpoint{0.670963in}{1.199621in}}{\pgfqpoint{0.679199in}{1.199621in}}%
\pgfpathclose%
\pgfusepath{stroke,fill}%
\end{pgfscope}%
\begin{pgfscope}%
\pgfpathrectangle{\pgfqpoint{0.100000in}{0.220728in}}{\pgfqpoint{3.696000in}{3.696000in}}%
\pgfusepath{clip}%
\pgfsetbuttcap%
\pgfsetroundjoin%
\definecolor{currentfill}{rgb}{0.121569,0.466667,0.705882}%
\pgfsetfillcolor{currentfill}%
\pgfsetfillopacity{0.619878}%
\pgfsetlinewidth{1.003750pt}%
\definecolor{currentstroke}{rgb}{0.121569,0.466667,0.705882}%
\pgfsetstrokecolor{currentstroke}%
\pgfsetstrokeopacity{0.619878}%
\pgfsetdash{}{0pt}%
\pgfpathmoveto{\pgfqpoint{0.679199in}{1.199621in}}%
\pgfpathcurveto{\pgfqpoint{0.687435in}{1.199621in}}{\pgfqpoint{0.695335in}{1.202893in}}{\pgfqpoint{0.701159in}{1.208717in}}%
\pgfpathcurveto{\pgfqpoint{0.706983in}{1.214541in}}{\pgfqpoint{0.710255in}{1.222441in}}{\pgfqpoint{0.710255in}{1.230677in}}%
\pgfpathcurveto{\pgfqpoint{0.710255in}{1.238914in}}{\pgfqpoint{0.706983in}{1.246814in}}{\pgfqpoint{0.701159in}{1.252638in}}%
\pgfpathcurveto{\pgfqpoint{0.695335in}{1.258462in}}{\pgfqpoint{0.687435in}{1.261734in}}{\pgfqpoint{0.679199in}{1.261734in}}%
\pgfpathcurveto{\pgfqpoint{0.670963in}{1.261734in}}{\pgfqpoint{0.663063in}{1.258462in}}{\pgfqpoint{0.657239in}{1.252638in}}%
\pgfpathcurveto{\pgfqpoint{0.651415in}{1.246814in}}{\pgfqpoint{0.648142in}{1.238914in}}{\pgfqpoint{0.648142in}{1.230677in}}%
\pgfpathcurveto{\pgfqpoint{0.648142in}{1.222441in}}{\pgfqpoint{0.651415in}{1.214541in}}{\pgfqpoint{0.657239in}{1.208717in}}%
\pgfpathcurveto{\pgfqpoint{0.663063in}{1.202893in}}{\pgfqpoint{0.670963in}{1.199621in}}{\pgfqpoint{0.679199in}{1.199621in}}%
\pgfpathclose%
\pgfusepath{stroke,fill}%
\end{pgfscope}%
\begin{pgfscope}%
\pgfpathrectangle{\pgfqpoint{0.100000in}{0.220728in}}{\pgfqpoint{3.696000in}{3.696000in}}%
\pgfusepath{clip}%
\pgfsetbuttcap%
\pgfsetroundjoin%
\definecolor{currentfill}{rgb}{0.121569,0.466667,0.705882}%
\pgfsetfillcolor{currentfill}%
\pgfsetfillopacity{0.619878}%
\pgfsetlinewidth{1.003750pt}%
\definecolor{currentstroke}{rgb}{0.121569,0.466667,0.705882}%
\pgfsetstrokecolor{currentstroke}%
\pgfsetstrokeopacity{0.619878}%
\pgfsetdash{}{0pt}%
\pgfpathmoveto{\pgfqpoint{0.679199in}{1.199621in}}%
\pgfpathcurveto{\pgfqpoint{0.687435in}{1.199621in}}{\pgfqpoint{0.695335in}{1.202893in}}{\pgfqpoint{0.701159in}{1.208717in}}%
\pgfpathcurveto{\pgfqpoint{0.706983in}{1.214541in}}{\pgfqpoint{0.710255in}{1.222441in}}{\pgfqpoint{0.710255in}{1.230677in}}%
\pgfpathcurveto{\pgfqpoint{0.710255in}{1.238914in}}{\pgfqpoint{0.706983in}{1.246814in}}{\pgfqpoint{0.701159in}{1.252638in}}%
\pgfpathcurveto{\pgfqpoint{0.695335in}{1.258462in}}{\pgfqpoint{0.687435in}{1.261734in}}{\pgfqpoint{0.679199in}{1.261734in}}%
\pgfpathcurveto{\pgfqpoint{0.670963in}{1.261734in}}{\pgfqpoint{0.663063in}{1.258462in}}{\pgfqpoint{0.657239in}{1.252638in}}%
\pgfpathcurveto{\pgfqpoint{0.651415in}{1.246814in}}{\pgfqpoint{0.648142in}{1.238914in}}{\pgfqpoint{0.648142in}{1.230677in}}%
\pgfpathcurveto{\pgfqpoint{0.648142in}{1.222441in}}{\pgfqpoint{0.651415in}{1.214541in}}{\pgfqpoint{0.657239in}{1.208717in}}%
\pgfpathcurveto{\pgfqpoint{0.663063in}{1.202893in}}{\pgfqpoint{0.670963in}{1.199621in}}{\pgfqpoint{0.679199in}{1.199621in}}%
\pgfpathclose%
\pgfusepath{stroke,fill}%
\end{pgfscope}%
\begin{pgfscope}%
\pgfpathrectangle{\pgfqpoint{0.100000in}{0.220728in}}{\pgfqpoint{3.696000in}{3.696000in}}%
\pgfusepath{clip}%
\pgfsetbuttcap%
\pgfsetroundjoin%
\definecolor{currentfill}{rgb}{0.121569,0.466667,0.705882}%
\pgfsetfillcolor{currentfill}%
\pgfsetfillopacity{0.619878}%
\pgfsetlinewidth{1.003750pt}%
\definecolor{currentstroke}{rgb}{0.121569,0.466667,0.705882}%
\pgfsetstrokecolor{currentstroke}%
\pgfsetstrokeopacity{0.619878}%
\pgfsetdash{}{0pt}%
\pgfpathmoveto{\pgfqpoint{0.679199in}{1.199621in}}%
\pgfpathcurveto{\pgfqpoint{0.687435in}{1.199621in}}{\pgfqpoint{0.695335in}{1.202893in}}{\pgfqpoint{0.701159in}{1.208717in}}%
\pgfpathcurveto{\pgfqpoint{0.706983in}{1.214541in}}{\pgfqpoint{0.710255in}{1.222441in}}{\pgfqpoint{0.710255in}{1.230677in}}%
\pgfpathcurveto{\pgfqpoint{0.710255in}{1.238914in}}{\pgfqpoint{0.706983in}{1.246814in}}{\pgfqpoint{0.701159in}{1.252638in}}%
\pgfpathcurveto{\pgfqpoint{0.695335in}{1.258462in}}{\pgfqpoint{0.687435in}{1.261734in}}{\pgfqpoint{0.679199in}{1.261734in}}%
\pgfpathcurveto{\pgfqpoint{0.670963in}{1.261734in}}{\pgfqpoint{0.663063in}{1.258462in}}{\pgfqpoint{0.657239in}{1.252638in}}%
\pgfpathcurveto{\pgfqpoint{0.651415in}{1.246814in}}{\pgfqpoint{0.648142in}{1.238914in}}{\pgfqpoint{0.648142in}{1.230677in}}%
\pgfpathcurveto{\pgfqpoint{0.648142in}{1.222441in}}{\pgfqpoint{0.651415in}{1.214541in}}{\pgfqpoint{0.657239in}{1.208717in}}%
\pgfpathcurveto{\pgfqpoint{0.663063in}{1.202893in}}{\pgfqpoint{0.670963in}{1.199621in}}{\pgfqpoint{0.679199in}{1.199621in}}%
\pgfpathclose%
\pgfusepath{stroke,fill}%
\end{pgfscope}%
\begin{pgfscope}%
\pgfpathrectangle{\pgfqpoint{0.100000in}{0.220728in}}{\pgfqpoint{3.696000in}{3.696000in}}%
\pgfusepath{clip}%
\pgfsetbuttcap%
\pgfsetroundjoin%
\definecolor{currentfill}{rgb}{0.121569,0.466667,0.705882}%
\pgfsetfillcolor{currentfill}%
\pgfsetfillopacity{0.619878}%
\pgfsetlinewidth{1.003750pt}%
\definecolor{currentstroke}{rgb}{0.121569,0.466667,0.705882}%
\pgfsetstrokecolor{currentstroke}%
\pgfsetstrokeopacity{0.619878}%
\pgfsetdash{}{0pt}%
\pgfpathmoveto{\pgfqpoint{0.679199in}{1.199621in}}%
\pgfpathcurveto{\pgfqpoint{0.687435in}{1.199621in}}{\pgfqpoint{0.695335in}{1.202893in}}{\pgfqpoint{0.701159in}{1.208717in}}%
\pgfpathcurveto{\pgfqpoint{0.706983in}{1.214541in}}{\pgfqpoint{0.710255in}{1.222441in}}{\pgfqpoint{0.710255in}{1.230677in}}%
\pgfpathcurveto{\pgfqpoint{0.710255in}{1.238914in}}{\pgfqpoint{0.706983in}{1.246814in}}{\pgfqpoint{0.701159in}{1.252638in}}%
\pgfpathcurveto{\pgfqpoint{0.695335in}{1.258462in}}{\pgfqpoint{0.687435in}{1.261734in}}{\pgfqpoint{0.679199in}{1.261734in}}%
\pgfpathcurveto{\pgfqpoint{0.670963in}{1.261734in}}{\pgfqpoint{0.663063in}{1.258462in}}{\pgfqpoint{0.657239in}{1.252638in}}%
\pgfpathcurveto{\pgfqpoint{0.651415in}{1.246814in}}{\pgfqpoint{0.648142in}{1.238914in}}{\pgfqpoint{0.648142in}{1.230677in}}%
\pgfpathcurveto{\pgfqpoint{0.648142in}{1.222441in}}{\pgfqpoint{0.651415in}{1.214541in}}{\pgfqpoint{0.657239in}{1.208717in}}%
\pgfpathcurveto{\pgfqpoint{0.663063in}{1.202893in}}{\pgfqpoint{0.670963in}{1.199621in}}{\pgfqpoint{0.679199in}{1.199621in}}%
\pgfpathclose%
\pgfusepath{stroke,fill}%
\end{pgfscope}%
\begin{pgfscope}%
\pgfpathrectangle{\pgfqpoint{0.100000in}{0.220728in}}{\pgfqpoint{3.696000in}{3.696000in}}%
\pgfusepath{clip}%
\pgfsetbuttcap%
\pgfsetroundjoin%
\definecolor{currentfill}{rgb}{0.121569,0.466667,0.705882}%
\pgfsetfillcolor{currentfill}%
\pgfsetfillopacity{0.619878}%
\pgfsetlinewidth{1.003750pt}%
\definecolor{currentstroke}{rgb}{0.121569,0.466667,0.705882}%
\pgfsetstrokecolor{currentstroke}%
\pgfsetstrokeopacity{0.619878}%
\pgfsetdash{}{0pt}%
\pgfpathmoveto{\pgfqpoint{0.679199in}{1.199621in}}%
\pgfpathcurveto{\pgfqpoint{0.687435in}{1.199621in}}{\pgfqpoint{0.695335in}{1.202893in}}{\pgfqpoint{0.701159in}{1.208717in}}%
\pgfpathcurveto{\pgfqpoint{0.706983in}{1.214541in}}{\pgfqpoint{0.710255in}{1.222441in}}{\pgfqpoint{0.710255in}{1.230677in}}%
\pgfpathcurveto{\pgfqpoint{0.710255in}{1.238914in}}{\pgfqpoint{0.706983in}{1.246814in}}{\pgfqpoint{0.701159in}{1.252638in}}%
\pgfpathcurveto{\pgfqpoint{0.695335in}{1.258462in}}{\pgfqpoint{0.687435in}{1.261734in}}{\pgfqpoint{0.679199in}{1.261734in}}%
\pgfpathcurveto{\pgfqpoint{0.670963in}{1.261734in}}{\pgfqpoint{0.663063in}{1.258462in}}{\pgfqpoint{0.657239in}{1.252638in}}%
\pgfpathcurveto{\pgfqpoint{0.651415in}{1.246814in}}{\pgfqpoint{0.648142in}{1.238914in}}{\pgfqpoint{0.648142in}{1.230677in}}%
\pgfpathcurveto{\pgfqpoint{0.648142in}{1.222441in}}{\pgfqpoint{0.651415in}{1.214541in}}{\pgfqpoint{0.657239in}{1.208717in}}%
\pgfpathcurveto{\pgfqpoint{0.663063in}{1.202893in}}{\pgfqpoint{0.670963in}{1.199621in}}{\pgfqpoint{0.679199in}{1.199621in}}%
\pgfpathclose%
\pgfusepath{stroke,fill}%
\end{pgfscope}%
\begin{pgfscope}%
\pgfpathrectangle{\pgfqpoint{0.100000in}{0.220728in}}{\pgfqpoint{3.696000in}{3.696000in}}%
\pgfusepath{clip}%
\pgfsetbuttcap%
\pgfsetroundjoin%
\definecolor{currentfill}{rgb}{0.121569,0.466667,0.705882}%
\pgfsetfillcolor{currentfill}%
\pgfsetfillopacity{0.619878}%
\pgfsetlinewidth{1.003750pt}%
\definecolor{currentstroke}{rgb}{0.121569,0.466667,0.705882}%
\pgfsetstrokecolor{currentstroke}%
\pgfsetstrokeopacity{0.619878}%
\pgfsetdash{}{0pt}%
\pgfpathmoveto{\pgfqpoint{0.679199in}{1.199621in}}%
\pgfpathcurveto{\pgfqpoint{0.687435in}{1.199621in}}{\pgfqpoint{0.695335in}{1.202893in}}{\pgfqpoint{0.701159in}{1.208717in}}%
\pgfpathcurveto{\pgfqpoint{0.706983in}{1.214541in}}{\pgfqpoint{0.710255in}{1.222441in}}{\pgfqpoint{0.710255in}{1.230677in}}%
\pgfpathcurveto{\pgfqpoint{0.710255in}{1.238914in}}{\pgfqpoint{0.706983in}{1.246814in}}{\pgfqpoint{0.701159in}{1.252638in}}%
\pgfpathcurveto{\pgfqpoint{0.695335in}{1.258462in}}{\pgfqpoint{0.687435in}{1.261734in}}{\pgfqpoint{0.679199in}{1.261734in}}%
\pgfpathcurveto{\pgfqpoint{0.670963in}{1.261734in}}{\pgfqpoint{0.663063in}{1.258462in}}{\pgfqpoint{0.657239in}{1.252638in}}%
\pgfpathcurveto{\pgfqpoint{0.651415in}{1.246814in}}{\pgfqpoint{0.648142in}{1.238914in}}{\pgfqpoint{0.648142in}{1.230677in}}%
\pgfpathcurveto{\pgfqpoint{0.648142in}{1.222441in}}{\pgfqpoint{0.651415in}{1.214541in}}{\pgfqpoint{0.657239in}{1.208717in}}%
\pgfpathcurveto{\pgfqpoint{0.663063in}{1.202893in}}{\pgfqpoint{0.670963in}{1.199621in}}{\pgfqpoint{0.679199in}{1.199621in}}%
\pgfpathclose%
\pgfusepath{stroke,fill}%
\end{pgfscope}%
\begin{pgfscope}%
\pgfpathrectangle{\pgfqpoint{0.100000in}{0.220728in}}{\pgfqpoint{3.696000in}{3.696000in}}%
\pgfusepath{clip}%
\pgfsetbuttcap%
\pgfsetroundjoin%
\definecolor{currentfill}{rgb}{0.121569,0.466667,0.705882}%
\pgfsetfillcolor{currentfill}%
\pgfsetfillopacity{0.625635}%
\pgfsetlinewidth{1.003750pt}%
\definecolor{currentstroke}{rgb}{0.121569,0.466667,0.705882}%
\pgfsetstrokecolor{currentstroke}%
\pgfsetstrokeopacity{0.625635}%
\pgfsetdash{}{0pt}%
\pgfpathmoveto{\pgfqpoint{3.035698in}{2.926457in}}%
\pgfpathcurveto{\pgfqpoint{3.043934in}{2.926457in}}{\pgfqpoint{3.051834in}{2.929730in}}{\pgfqpoint{3.057658in}{2.935554in}}%
\pgfpathcurveto{\pgfqpoint{3.063482in}{2.941378in}}{\pgfqpoint{3.066754in}{2.949278in}}{\pgfqpoint{3.066754in}{2.957514in}}%
\pgfpathcurveto{\pgfqpoint{3.066754in}{2.965750in}}{\pgfqpoint{3.063482in}{2.973650in}}{\pgfqpoint{3.057658in}{2.979474in}}%
\pgfpathcurveto{\pgfqpoint{3.051834in}{2.985298in}}{\pgfqpoint{3.043934in}{2.988570in}}{\pgfqpoint{3.035698in}{2.988570in}}%
\pgfpathcurveto{\pgfqpoint{3.027461in}{2.988570in}}{\pgfqpoint{3.019561in}{2.985298in}}{\pgfqpoint{3.013737in}{2.979474in}}%
\pgfpathcurveto{\pgfqpoint{3.007913in}{2.973650in}}{\pgfqpoint{3.004641in}{2.965750in}}{\pgfqpoint{3.004641in}{2.957514in}}%
\pgfpathcurveto{\pgfqpoint{3.004641in}{2.949278in}}{\pgfqpoint{3.007913in}{2.941378in}}{\pgfqpoint{3.013737in}{2.935554in}}%
\pgfpathcurveto{\pgfqpoint{3.019561in}{2.929730in}}{\pgfqpoint{3.027461in}{2.926457in}}{\pgfqpoint{3.035698in}{2.926457in}}%
\pgfpathclose%
\pgfusepath{stroke,fill}%
\end{pgfscope}%
\begin{pgfscope}%
\pgfpathrectangle{\pgfqpoint{0.100000in}{0.220728in}}{\pgfqpoint{3.696000in}{3.696000in}}%
\pgfusepath{clip}%
\pgfsetbuttcap%
\pgfsetroundjoin%
\definecolor{currentfill}{rgb}{0.121569,0.466667,0.705882}%
\pgfsetfillcolor{currentfill}%
\pgfsetfillopacity{0.632462}%
\pgfsetlinewidth{1.003750pt}%
\definecolor{currentstroke}{rgb}{0.121569,0.466667,0.705882}%
\pgfsetstrokecolor{currentstroke}%
\pgfsetstrokeopacity{0.632462}%
\pgfsetdash{}{0pt}%
\pgfpathmoveto{\pgfqpoint{3.062807in}{2.922132in}}%
\pgfpathcurveto{\pgfqpoint{3.071044in}{2.922132in}}{\pgfqpoint{3.078944in}{2.925404in}}{\pgfqpoint{3.084768in}{2.931228in}}%
\pgfpathcurveto{\pgfqpoint{3.090591in}{2.937052in}}{\pgfqpoint{3.093864in}{2.944952in}}{\pgfqpoint{3.093864in}{2.953188in}}%
\pgfpathcurveto{\pgfqpoint{3.093864in}{2.961425in}}{\pgfqpoint{3.090591in}{2.969325in}}{\pgfqpoint{3.084768in}{2.975148in}}%
\pgfpathcurveto{\pgfqpoint{3.078944in}{2.980972in}}{\pgfqpoint{3.071044in}{2.984245in}}{\pgfqpoint{3.062807in}{2.984245in}}%
\pgfpathcurveto{\pgfqpoint{3.054571in}{2.984245in}}{\pgfqpoint{3.046671in}{2.980972in}}{\pgfqpoint{3.040847in}{2.975148in}}%
\pgfpathcurveto{\pgfqpoint{3.035023in}{2.969325in}}{\pgfqpoint{3.031751in}{2.961425in}}{\pgfqpoint{3.031751in}{2.953188in}}%
\pgfpathcurveto{\pgfqpoint{3.031751in}{2.944952in}}{\pgfqpoint{3.035023in}{2.937052in}}{\pgfqpoint{3.040847in}{2.931228in}}%
\pgfpathcurveto{\pgfqpoint{3.046671in}{2.925404in}}{\pgfqpoint{3.054571in}{2.922132in}}{\pgfqpoint{3.062807in}{2.922132in}}%
\pgfpathclose%
\pgfusepath{stroke,fill}%
\end{pgfscope}%
\begin{pgfscope}%
\pgfpathrectangle{\pgfqpoint{0.100000in}{0.220728in}}{\pgfqpoint{3.696000in}{3.696000in}}%
\pgfusepath{clip}%
\pgfsetbuttcap%
\pgfsetroundjoin%
\definecolor{currentfill}{rgb}{0.121569,0.466667,0.705882}%
\pgfsetfillcolor{currentfill}%
\pgfsetfillopacity{0.640688}%
\pgfsetlinewidth{1.003750pt}%
\definecolor{currentstroke}{rgb}{0.121569,0.466667,0.705882}%
\pgfsetstrokecolor{currentstroke}%
\pgfsetstrokeopacity{0.640688}%
\pgfsetdash{}{0pt}%
\pgfpathmoveto{\pgfqpoint{3.093070in}{2.918389in}}%
\pgfpathcurveto{\pgfqpoint{3.101307in}{2.918389in}}{\pgfqpoint{3.109207in}{2.921662in}}{\pgfqpoint{3.115031in}{2.927486in}}%
\pgfpathcurveto{\pgfqpoint{3.120855in}{2.933310in}}{\pgfqpoint{3.124127in}{2.941210in}}{\pgfqpoint{3.124127in}{2.949446in}}%
\pgfpathcurveto{\pgfqpoint{3.124127in}{2.957682in}}{\pgfqpoint{3.120855in}{2.965582in}}{\pgfqpoint{3.115031in}{2.971406in}}%
\pgfpathcurveto{\pgfqpoint{3.109207in}{2.977230in}}{\pgfqpoint{3.101307in}{2.980502in}}{\pgfqpoint{3.093070in}{2.980502in}}%
\pgfpathcurveto{\pgfqpoint{3.084834in}{2.980502in}}{\pgfqpoint{3.076934in}{2.977230in}}{\pgfqpoint{3.071110in}{2.971406in}}%
\pgfpathcurveto{\pgfqpoint{3.065286in}{2.965582in}}{\pgfqpoint{3.062014in}{2.957682in}}{\pgfqpoint{3.062014in}{2.949446in}}%
\pgfpathcurveto{\pgfqpoint{3.062014in}{2.941210in}}{\pgfqpoint{3.065286in}{2.933310in}}{\pgfqpoint{3.071110in}{2.927486in}}%
\pgfpathcurveto{\pgfqpoint{3.076934in}{2.921662in}}{\pgfqpoint{3.084834in}{2.918389in}}{\pgfqpoint{3.093070in}{2.918389in}}%
\pgfpathclose%
\pgfusepath{stroke,fill}%
\end{pgfscope}%
\begin{pgfscope}%
\pgfpathrectangle{\pgfqpoint{0.100000in}{0.220728in}}{\pgfqpoint{3.696000in}{3.696000in}}%
\pgfusepath{clip}%
\pgfsetbuttcap%
\pgfsetroundjoin%
\definecolor{currentfill}{rgb}{0.121569,0.466667,0.705882}%
\pgfsetfillcolor{currentfill}%
\pgfsetfillopacity{0.650681}%
\pgfsetlinewidth{1.003750pt}%
\definecolor{currentstroke}{rgb}{0.121569,0.466667,0.705882}%
\pgfsetstrokecolor{currentstroke}%
\pgfsetstrokeopacity{0.650681}%
\pgfsetdash{}{0pt}%
\pgfpathmoveto{\pgfqpoint{3.130267in}{2.912318in}}%
\pgfpathcurveto{\pgfqpoint{3.138503in}{2.912318in}}{\pgfqpoint{3.146403in}{2.915590in}}{\pgfqpoint{3.152227in}{2.921414in}}%
\pgfpathcurveto{\pgfqpoint{3.158051in}{2.927238in}}{\pgfqpoint{3.161324in}{2.935138in}}{\pgfqpoint{3.161324in}{2.943374in}}%
\pgfpathcurveto{\pgfqpoint{3.161324in}{2.951611in}}{\pgfqpoint{3.158051in}{2.959511in}}{\pgfqpoint{3.152227in}{2.965335in}}%
\pgfpathcurveto{\pgfqpoint{3.146403in}{2.971159in}}{\pgfqpoint{3.138503in}{2.974431in}}{\pgfqpoint{3.130267in}{2.974431in}}%
\pgfpathcurveto{\pgfqpoint{3.122031in}{2.974431in}}{\pgfqpoint{3.114131in}{2.971159in}}{\pgfqpoint{3.108307in}{2.965335in}}%
\pgfpathcurveto{\pgfqpoint{3.102483in}{2.959511in}}{\pgfqpoint{3.099211in}{2.951611in}}{\pgfqpoint{3.099211in}{2.943374in}}%
\pgfpathcurveto{\pgfqpoint{3.099211in}{2.935138in}}{\pgfqpoint{3.102483in}{2.927238in}}{\pgfqpoint{3.108307in}{2.921414in}}%
\pgfpathcurveto{\pgfqpoint{3.114131in}{2.915590in}}{\pgfqpoint{3.122031in}{2.912318in}}{\pgfqpoint{3.130267in}{2.912318in}}%
\pgfpathclose%
\pgfusepath{stroke,fill}%
\end{pgfscope}%
\begin{pgfscope}%
\pgfpathrectangle{\pgfqpoint{0.100000in}{0.220728in}}{\pgfqpoint{3.696000in}{3.696000in}}%
\pgfusepath{clip}%
\pgfsetbuttcap%
\pgfsetroundjoin%
\definecolor{currentfill}{rgb}{0.121569,0.466667,0.705882}%
\pgfsetfillcolor{currentfill}%
\pgfsetfillopacity{0.659505}%
\pgfsetlinewidth{1.003750pt}%
\definecolor{currentstroke}{rgb}{0.121569,0.466667,0.705882}%
\pgfsetstrokecolor{currentstroke}%
\pgfsetstrokeopacity{0.659505}%
\pgfsetdash{}{0pt}%
\pgfpathmoveto{\pgfqpoint{3.173077in}{2.903326in}}%
\pgfpathcurveto{\pgfqpoint{3.181313in}{2.903326in}}{\pgfqpoint{3.189213in}{2.906598in}}{\pgfqpoint{3.195037in}{2.912422in}}%
\pgfpathcurveto{\pgfqpoint{3.200861in}{2.918246in}}{\pgfqpoint{3.204133in}{2.926146in}}{\pgfqpoint{3.204133in}{2.934382in}}%
\pgfpathcurveto{\pgfqpoint{3.204133in}{2.942618in}}{\pgfqpoint{3.200861in}{2.950518in}}{\pgfqpoint{3.195037in}{2.956342in}}%
\pgfpathcurveto{\pgfqpoint{3.189213in}{2.962166in}}{\pgfqpoint{3.181313in}{2.965439in}}{\pgfqpoint{3.173077in}{2.965439in}}%
\pgfpathcurveto{\pgfqpoint{3.164841in}{2.965439in}}{\pgfqpoint{3.156940in}{2.962166in}}{\pgfqpoint{3.151117in}{2.956342in}}%
\pgfpathcurveto{\pgfqpoint{3.145293in}{2.950518in}}{\pgfqpoint{3.142020in}{2.942618in}}{\pgfqpoint{3.142020in}{2.934382in}}%
\pgfpathcurveto{\pgfqpoint{3.142020in}{2.926146in}}{\pgfqpoint{3.145293in}{2.918246in}}{\pgfqpoint{3.151117in}{2.912422in}}%
\pgfpathcurveto{\pgfqpoint{3.156940in}{2.906598in}}{\pgfqpoint{3.164841in}{2.903326in}}{\pgfqpoint{3.173077in}{2.903326in}}%
\pgfpathclose%
\pgfusepath{stroke,fill}%
\end{pgfscope}%
\begin{pgfscope}%
\pgfpathrectangle{\pgfqpoint{0.100000in}{0.220728in}}{\pgfqpoint{3.696000in}{3.696000in}}%
\pgfusepath{clip}%
\pgfsetbuttcap%
\pgfsetroundjoin%
\definecolor{currentfill}{rgb}{0.121569,0.466667,0.705882}%
\pgfsetfillcolor{currentfill}%
\pgfsetfillopacity{0.672322}%
\pgfsetlinewidth{1.003750pt}%
\definecolor{currentstroke}{rgb}{0.121569,0.466667,0.705882}%
\pgfsetstrokecolor{currentstroke}%
\pgfsetstrokeopacity{0.672322}%
\pgfsetdash{}{0pt}%
\pgfpathmoveto{\pgfqpoint{3.212527in}{2.895769in}}%
\pgfpathcurveto{\pgfqpoint{3.220763in}{2.895769in}}{\pgfqpoint{3.228663in}{2.899042in}}{\pgfqpoint{3.234487in}{2.904866in}}%
\pgfpathcurveto{\pgfqpoint{3.240311in}{2.910690in}}{\pgfqpoint{3.243583in}{2.918590in}}{\pgfqpoint{3.243583in}{2.926826in}}%
\pgfpathcurveto{\pgfqpoint{3.243583in}{2.935062in}}{\pgfqpoint{3.240311in}{2.942962in}}{\pgfqpoint{3.234487in}{2.948786in}}%
\pgfpathcurveto{\pgfqpoint{3.228663in}{2.954610in}}{\pgfqpoint{3.220763in}{2.957882in}}{\pgfqpoint{3.212527in}{2.957882in}}%
\pgfpathcurveto{\pgfqpoint{3.204290in}{2.957882in}}{\pgfqpoint{3.196390in}{2.954610in}}{\pgfqpoint{3.190566in}{2.948786in}}%
\pgfpathcurveto{\pgfqpoint{3.184742in}{2.942962in}}{\pgfqpoint{3.181470in}{2.935062in}}{\pgfqpoint{3.181470in}{2.926826in}}%
\pgfpathcurveto{\pgfqpoint{3.181470in}{2.918590in}}{\pgfqpoint{3.184742in}{2.910690in}}{\pgfqpoint{3.190566in}{2.904866in}}%
\pgfpathcurveto{\pgfqpoint{3.196390in}{2.899042in}}{\pgfqpoint{3.204290in}{2.895769in}}{\pgfqpoint{3.212527in}{2.895769in}}%
\pgfpathclose%
\pgfusepath{stroke,fill}%
\end{pgfscope}%
\begin{pgfscope}%
\pgfpathrectangle{\pgfqpoint{0.100000in}{0.220728in}}{\pgfqpoint{3.696000in}{3.696000in}}%
\pgfusepath{clip}%
\pgfsetbuttcap%
\pgfsetroundjoin%
\definecolor{currentfill}{rgb}{0.121569,0.466667,0.705882}%
\pgfsetfillcolor{currentfill}%
\pgfsetfillopacity{0.677811}%
\pgfsetlinewidth{1.003750pt}%
\definecolor{currentstroke}{rgb}{0.121569,0.466667,0.705882}%
\pgfsetstrokecolor{currentstroke}%
\pgfsetstrokeopacity{0.677811}%
\pgfsetdash{}{0pt}%
\pgfpathmoveto{\pgfqpoint{3.237616in}{2.893315in}}%
\pgfpathcurveto{\pgfqpoint{3.245852in}{2.893315in}}{\pgfqpoint{3.253752in}{2.896587in}}{\pgfqpoint{3.259576in}{2.902411in}}%
\pgfpathcurveto{\pgfqpoint{3.265400in}{2.908235in}}{\pgfqpoint{3.268672in}{2.916135in}}{\pgfqpoint{3.268672in}{2.924371in}}%
\pgfpathcurveto{\pgfqpoint{3.268672in}{2.932607in}}{\pgfqpoint{3.265400in}{2.940507in}}{\pgfqpoint{3.259576in}{2.946331in}}%
\pgfpathcurveto{\pgfqpoint{3.253752in}{2.952155in}}{\pgfqpoint{3.245852in}{2.955428in}}{\pgfqpoint{3.237616in}{2.955428in}}%
\pgfpathcurveto{\pgfqpoint{3.229379in}{2.955428in}}{\pgfqpoint{3.221479in}{2.952155in}}{\pgfqpoint{3.215655in}{2.946331in}}%
\pgfpathcurveto{\pgfqpoint{3.209831in}{2.940507in}}{\pgfqpoint{3.206559in}{2.932607in}}{\pgfqpoint{3.206559in}{2.924371in}}%
\pgfpathcurveto{\pgfqpoint{3.206559in}{2.916135in}}{\pgfqpoint{3.209831in}{2.908235in}}{\pgfqpoint{3.215655in}{2.902411in}}%
\pgfpathcurveto{\pgfqpoint{3.221479in}{2.896587in}}{\pgfqpoint{3.229379in}{2.893315in}}{\pgfqpoint{3.237616in}{2.893315in}}%
\pgfpathclose%
\pgfusepath{stroke,fill}%
\end{pgfscope}%
\begin{pgfscope}%
\pgfpathrectangle{\pgfqpoint{0.100000in}{0.220728in}}{\pgfqpoint{3.696000in}{3.696000in}}%
\pgfusepath{clip}%
\pgfsetbuttcap%
\pgfsetroundjoin%
\definecolor{currentfill}{rgb}{0.121569,0.466667,0.705882}%
\pgfsetfillcolor{currentfill}%
\pgfsetfillopacity{0.684326}%
\pgfsetlinewidth{1.003750pt}%
\definecolor{currentstroke}{rgb}{0.121569,0.466667,0.705882}%
\pgfsetstrokecolor{currentstroke}%
\pgfsetstrokeopacity{0.684326}%
\pgfsetdash{}{0pt}%
\pgfpathmoveto{\pgfqpoint{3.263794in}{2.888407in}}%
\pgfpathcurveto{\pgfqpoint{3.272030in}{2.888407in}}{\pgfqpoint{3.279930in}{2.891679in}}{\pgfqpoint{3.285754in}{2.897503in}}%
\pgfpathcurveto{\pgfqpoint{3.291578in}{2.903327in}}{\pgfqpoint{3.294850in}{2.911227in}}{\pgfqpoint{3.294850in}{2.919463in}}%
\pgfpathcurveto{\pgfqpoint{3.294850in}{2.927700in}}{\pgfqpoint{3.291578in}{2.935600in}}{\pgfqpoint{3.285754in}{2.941424in}}%
\pgfpathcurveto{\pgfqpoint{3.279930in}{2.947248in}}{\pgfqpoint{3.272030in}{2.950520in}}{\pgfqpoint{3.263794in}{2.950520in}}%
\pgfpathcurveto{\pgfqpoint{3.255557in}{2.950520in}}{\pgfqpoint{3.247657in}{2.947248in}}{\pgfqpoint{3.241833in}{2.941424in}}%
\pgfpathcurveto{\pgfqpoint{3.236009in}{2.935600in}}{\pgfqpoint{3.232737in}{2.927700in}}{\pgfqpoint{3.232737in}{2.919463in}}%
\pgfpathcurveto{\pgfqpoint{3.232737in}{2.911227in}}{\pgfqpoint{3.236009in}{2.903327in}}{\pgfqpoint{3.241833in}{2.897503in}}%
\pgfpathcurveto{\pgfqpoint{3.247657in}{2.891679in}}{\pgfqpoint{3.255557in}{2.888407in}}{\pgfqpoint{3.263794in}{2.888407in}}%
\pgfpathclose%
\pgfusepath{stroke,fill}%
\end{pgfscope}%
\begin{pgfscope}%
\pgfpathrectangle{\pgfqpoint{0.100000in}{0.220728in}}{\pgfqpoint{3.696000in}{3.696000in}}%
\pgfusepath{clip}%
\pgfsetbuttcap%
\pgfsetroundjoin%
\definecolor{currentfill}{rgb}{0.121569,0.466667,0.705882}%
\pgfsetfillcolor{currentfill}%
\pgfsetfillopacity{0.688178}%
\pgfsetlinewidth{1.003750pt}%
\definecolor{currentstroke}{rgb}{0.121569,0.466667,0.705882}%
\pgfsetstrokecolor{currentstroke}%
\pgfsetstrokeopacity{0.688178}%
\pgfsetdash{}{0pt}%
\pgfpathmoveto{\pgfqpoint{3.277560in}{2.885208in}}%
\pgfpathcurveto{\pgfqpoint{3.285797in}{2.885208in}}{\pgfqpoint{3.293697in}{2.888480in}}{\pgfqpoint{3.299521in}{2.894304in}}%
\pgfpathcurveto{\pgfqpoint{3.305345in}{2.900128in}}{\pgfqpoint{3.308617in}{2.908028in}}{\pgfqpoint{3.308617in}{2.916264in}}%
\pgfpathcurveto{\pgfqpoint{3.308617in}{2.924501in}}{\pgfqpoint{3.305345in}{2.932401in}}{\pgfqpoint{3.299521in}{2.938225in}}%
\pgfpathcurveto{\pgfqpoint{3.293697in}{2.944049in}}{\pgfqpoint{3.285797in}{2.947321in}}{\pgfqpoint{3.277560in}{2.947321in}}%
\pgfpathcurveto{\pgfqpoint{3.269324in}{2.947321in}}{\pgfqpoint{3.261424in}{2.944049in}}{\pgfqpoint{3.255600in}{2.938225in}}%
\pgfpathcurveto{\pgfqpoint{3.249776in}{2.932401in}}{\pgfqpoint{3.246504in}{2.924501in}}{\pgfqpoint{3.246504in}{2.916264in}}%
\pgfpathcurveto{\pgfqpoint{3.246504in}{2.908028in}}{\pgfqpoint{3.249776in}{2.900128in}}{\pgfqpoint{3.255600in}{2.894304in}}%
\pgfpathcurveto{\pgfqpoint{3.261424in}{2.888480in}}{\pgfqpoint{3.269324in}{2.885208in}}{\pgfqpoint{3.277560in}{2.885208in}}%
\pgfpathclose%
\pgfusepath{stroke,fill}%
\end{pgfscope}%
\begin{pgfscope}%
\pgfpathrectangle{\pgfqpoint{0.100000in}{0.220728in}}{\pgfqpoint{3.696000in}{3.696000in}}%
\pgfusepath{clip}%
\pgfsetbuttcap%
\pgfsetroundjoin%
\definecolor{currentfill}{rgb}{0.121569,0.466667,0.705882}%
\pgfsetfillcolor{currentfill}%
\pgfsetfillopacity{0.693181}%
\pgfsetlinewidth{1.003750pt}%
\definecolor{currentstroke}{rgb}{0.121569,0.466667,0.705882}%
\pgfsetstrokecolor{currentstroke}%
\pgfsetstrokeopacity{0.693181}%
\pgfsetdash{}{0pt}%
\pgfpathmoveto{\pgfqpoint{3.296852in}{2.881813in}}%
\pgfpathcurveto{\pgfqpoint{3.305089in}{2.881813in}}{\pgfqpoint{3.312989in}{2.885085in}}{\pgfqpoint{3.318813in}{2.890909in}}%
\pgfpathcurveto{\pgfqpoint{3.324637in}{2.896733in}}{\pgfqpoint{3.327909in}{2.904633in}}{\pgfqpoint{3.327909in}{2.912869in}}%
\pgfpathcurveto{\pgfqpoint{3.327909in}{2.921106in}}{\pgfqpoint{3.324637in}{2.929006in}}{\pgfqpoint{3.318813in}{2.934830in}}%
\pgfpathcurveto{\pgfqpoint{3.312989in}{2.940653in}}{\pgfqpoint{3.305089in}{2.943926in}}{\pgfqpoint{3.296852in}{2.943926in}}%
\pgfpathcurveto{\pgfqpoint{3.288616in}{2.943926in}}{\pgfqpoint{3.280716in}{2.940653in}}{\pgfqpoint{3.274892in}{2.934830in}}%
\pgfpathcurveto{\pgfqpoint{3.269068in}{2.929006in}}{\pgfqpoint{3.265796in}{2.921106in}}{\pgfqpoint{3.265796in}{2.912869in}}%
\pgfpathcurveto{\pgfqpoint{3.265796in}{2.904633in}}{\pgfqpoint{3.269068in}{2.896733in}}{\pgfqpoint{3.274892in}{2.890909in}}%
\pgfpathcurveto{\pgfqpoint{3.280716in}{2.885085in}}{\pgfqpoint{3.288616in}{2.881813in}}{\pgfqpoint{3.296852in}{2.881813in}}%
\pgfpathclose%
\pgfusepath{stroke,fill}%
\end{pgfscope}%
\begin{pgfscope}%
\pgfpathrectangle{\pgfqpoint{0.100000in}{0.220728in}}{\pgfqpoint{3.696000in}{3.696000in}}%
\pgfusepath{clip}%
\pgfsetbuttcap%
\pgfsetroundjoin%
\definecolor{currentfill}{rgb}{0.121569,0.466667,0.705882}%
\pgfsetfillcolor{currentfill}%
\pgfsetfillopacity{0.695179}%
\pgfsetlinewidth{1.003750pt}%
\definecolor{currentstroke}{rgb}{0.121569,0.466667,0.705882}%
\pgfsetstrokecolor{currentstroke}%
\pgfsetstrokeopacity{0.695179}%
\pgfsetdash{}{0pt}%
\pgfpathmoveto{\pgfqpoint{0.754893in}{2.319267in}}%
\pgfpathcurveto{\pgfqpoint{0.763129in}{2.319267in}}{\pgfqpoint{0.771029in}{2.322540in}}{\pgfqpoint{0.776853in}{2.328364in}}%
\pgfpathcurveto{\pgfqpoint{0.782677in}{2.334188in}}{\pgfqpoint{0.785949in}{2.342088in}}{\pgfqpoint{0.785949in}{2.350324in}}%
\pgfpathcurveto{\pgfqpoint{0.785949in}{2.358560in}}{\pgfqpoint{0.782677in}{2.366460in}}{\pgfqpoint{0.776853in}{2.372284in}}%
\pgfpathcurveto{\pgfqpoint{0.771029in}{2.378108in}}{\pgfqpoint{0.763129in}{2.381380in}}{\pgfqpoint{0.754893in}{2.381380in}}%
\pgfpathcurveto{\pgfqpoint{0.746657in}{2.381380in}}{\pgfqpoint{0.738757in}{2.378108in}}{\pgfqpoint{0.732933in}{2.372284in}}%
\pgfpathcurveto{\pgfqpoint{0.727109in}{2.366460in}}{\pgfqpoint{0.723836in}{2.358560in}}{\pgfqpoint{0.723836in}{2.350324in}}%
\pgfpathcurveto{\pgfqpoint{0.723836in}{2.342088in}}{\pgfqpoint{0.727109in}{2.334188in}}{\pgfqpoint{0.732933in}{2.328364in}}%
\pgfpathcurveto{\pgfqpoint{0.738757in}{2.322540in}}{\pgfqpoint{0.746657in}{2.319267in}}{\pgfqpoint{0.754893in}{2.319267in}}%
\pgfpathclose%
\pgfusepath{stroke,fill}%
\end{pgfscope}%
\begin{pgfscope}%
\pgfpathrectangle{\pgfqpoint{0.100000in}{0.220728in}}{\pgfqpoint{3.696000in}{3.696000in}}%
\pgfusepath{clip}%
\pgfsetbuttcap%
\pgfsetroundjoin%
\definecolor{currentfill}{rgb}{0.121569,0.466667,0.705882}%
\pgfsetfillcolor{currentfill}%
\pgfsetfillopacity{0.696043}%
\pgfsetlinewidth{1.003750pt}%
\definecolor{currentstroke}{rgb}{0.121569,0.466667,0.705882}%
\pgfsetstrokecolor{currentstroke}%
\pgfsetstrokeopacity{0.696043}%
\pgfsetdash{}{0pt}%
\pgfpathmoveto{\pgfqpoint{3.306938in}{2.879078in}}%
\pgfpathcurveto{\pgfqpoint{3.315174in}{2.879078in}}{\pgfqpoint{3.323075in}{2.882350in}}{\pgfqpoint{3.328898in}{2.888174in}}%
\pgfpathcurveto{\pgfqpoint{3.334722in}{2.893998in}}{\pgfqpoint{3.337995in}{2.901898in}}{\pgfqpoint{3.337995in}{2.910134in}}%
\pgfpathcurveto{\pgfqpoint{3.337995in}{2.918371in}}{\pgfqpoint{3.334722in}{2.926271in}}{\pgfqpoint{3.328898in}{2.932095in}}%
\pgfpathcurveto{\pgfqpoint{3.323075in}{2.937918in}}{\pgfqpoint{3.315174in}{2.941191in}}{\pgfqpoint{3.306938in}{2.941191in}}%
\pgfpathcurveto{\pgfqpoint{3.298702in}{2.941191in}}{\pgfqpoint{3.290802in}{2.937918in}}{\pgfqpoint{3.284978in}{2.932095in}}%
\pgfpathcurveto{\pgfqpoint{3.279154in}{2.926271in}}{\pgfqpoint{3.275882in}{2.918371in}}{\pgfqpoint{3.275882in}{2.910134in}}%
\pgfpathcurveto{\pgfqpoint{3.275882in}{2.901898in}}{\pgfqpoint{3.279154in}{2.893998in}}{\pgfqpoint{3.284978in}{2.888174in}}%
\pgfpathcurveto{\pgfqpoint{3.290802in}{2.882350in}}{\pgfqpoint{3.298702in}{2.879078in}}{\pgfqpoint{3.306938in}{2.879078in}}%
\pgfpathclose%
\pgfusepath{stroke,fill}%
\end{pgfscope}%
\begin{pgfscope}%
\pgfpathrectangle{\pgfqpoint{0.100000in}{0.220728in}}{\pgfqpoint{3.696000in}{3.696000in}}%
\pgfusepath{clip}%
\pgfsetbuttcap%
\pgfsetroundjoin%
\definecolor{currentfill}{rgb}{0.121569,0.466667,0.705882}%
\pgfsetfillcolor{currentfill}%
\pgfsetfillopacity{0.697714}%
\pgfsetlinewidth{1.003750pt}%
\definecolor{currentstroke}{rgb}{0.121569,0.466667,0.705882}%
\pgfsetstrokecolor{currentstroke}%
\pgfsetstrokeopacity{0.697714}%
\pgfsetdash{}{0pt}%
\pgfpathmoveto{\pgfqpoint{3.311522in}{2.876114in}}%
\pgfpathcurveto{\pgfqpoint{3.319758in}{2.876114in}}{\pgfqpoint{3.327658in}{2.879386in}}{\pgfqpoint{3.333482in}{2.885210in}}%
\pgfpathcurveto{\pgfqpoint{3.339306in}{2.891034in}}{\pgfqpoint{3.342578in}{2.898934in}}{\pgfqpoint{3.342578in}{2.907170in}}%
\pgfpathcurveto{\pgfqpoint{3.342578in}{2.915406in}}{\pgfqpoint{3.339306in}{2.923307in}}{\pgfqpoint{3.333482in}{2.929130in}}%
\pgfpathcurveto{\pgfqpoint{3.327658in}{2.934954in}}{\pgfqpoint{3.319758in}{2.938227in}}{\pgfqpoint{3.311522in}{2.938227in}}%
\pgfpathcurveto{\pgfqpoint{3.303285in}{2.938227in}}{\pgfqpoint{3.295385in}{2.934954in}}{\pgfqpoint{3.289561in}{2.929130in}}%
\pgfpathcurveto{\pgfqpoint{3.283738in}{2.923307in}}{\pgfqpoint{3.280465in}{2.915406in}}{\pgfqpoint{3.280465in}{2.907170in}}%
\pgfpathcurveto{\pgfqpoint{3.280465in}{2.898934in}}{\pgfqpoint{3.283738in}{2.891034in}}{\pgfqpoint{3.289561in}{2.885210in}}%
\pgfpathcurveto{\pgfqpoint{3.295385in}{2.879386in}}{\pgfqpoint{3.303285in}{2.876114in}}{\pgfqpoint{3.311522in}{2.876114in}}%
\pgfpathclose%
\pgfusepath{stroke,fill}%
\end{pgfscope}%
\begin{pgfscope}%
\pgfpathrectangle{\pgfqpoint{0.100000in}{0.220728in}}{\pgfqpoint{3.696000in}{3.696000in}}%
\pgfusepath{clip}%
\pgfsetbuttcap%
\pgfsetroundjoin%
\definecolor{currentfill}{rgb}{0.121569,0.466667,0.705882}%
\pgfsetfillcolor{currentfill}%
\pgfsetfillopacity{0.698231}%
\pgfsetlinewidth{1.003750pt}%
\definecolor{currentstroke}{rgb}{0.121569,0.466667,0.705882}%
\pgfsetstrokecolor{currentstroke}%
\pgfsetstrokeopacity{0.698231}%
\pgfsetdash{}{0pt}%
\pgfpathmoveto{\pgfqpoint{0.775879in}{2.309403in}}%
\pgfpathcurveto{\pgfqpoint{0.784115in}{2.309403in}}{\pgfqpoint{0.792015in}{2.312675in}}{\pgfqpoint{0.797839in}{2.318499in}}%
\pgfpathcurveto{\pgfqpoint{0.803663in}{2.324323in}}{\pgfqpoint{0.806935in}{2.332223in}}{\pgfqpoint{0.806935in}{2.340460in}}%
\pgfpathcurveto{\pgfqpoint{0.806935in}{2.348696in}}{\pgfqpoint{0.803663in}{2.356596in}}{\pgfqpoint{0.797839in}{2.362420in}}%
\pgfpathcurveto{\pgfqpoint{0.792015in}{2.368244in}}{\pgfqpoint{0.784115in}{2.371516in}}{\pgfqpoint{0.775879in}{2.371516in}}%
\pgfpathcurveto{\pgfqpoint{0.767643in}{2.371516in}}{\pgfqpoint{0.759743in}{2.368244in}}{\pgfqpoint{0.753919in}{2.362420in}}%
\pgfpathcurveto{\pgfqpoint{0.748095in}{2.356596in}}{\pgfqpoint{0.744822in}{2.348696in}}{\pgfqpoint{0.744822in}{2.340460in}}%
\pgfpathcurveto{\pgfqpoint{0.744822in}{2.332223in}}{\pgfqpoint{0.748095in}{2.324323in}}{\pgfqpoint{0.753919in}{2.318499in}}%
\pgfpathcurveto{\pgfqpoint{0.759743in}{2.312675in}}{\pgfqpoint{0.767643in}{2.309403in}}{\pgfqpoint{0.775879in}{2.309403in}}%
\pgfpathclose%
\pgfusepath{stroke,fill}%
\end{pgfscope}%
\begin{pgfscope}%
\pgfpathrectangle{\pgfqpoint{0.100000in}{0.220728in}}{\pgfqpoint{3.696000in}{3.696000in}}%
\pgfusepath{clip}%
\pgfsetbuttcap%
\pgfsetroundjoin%
\definecolor{currentfill}{rgb}{0.121569,0.466667,0.705882}%
\pgfsetfillcolor{currentfill}%
\pgfsetfillopacity{0.699878}%
\pgfsetlinewidth{1.003750pt}%
\definecolor{currentstroke}{rgb}{0.121569,0.466667,0.705882}%
\pgfsetstrokecolor{currentstroke}%
\pgfsetstrokeopacity{0.699878}%
\pgfsetdash{}{0pt}%
\pgfpathmoveto{\pgfqpoint{3.314132in}{2.867047in}}%
\pgfpathcurveto{\pgfqpoint{3.322369in}{2.867047in}}{\pgfqpoint{3.330269in}{2.870319in}}{\pgfqpoint{3.336093in}{2.876143in}}%
\pgfpathcurveto{\pgfqpoint{3.341917in}{2.881967in}}{\pgfqpoint{3.345189in}{2.889867in}}{\pgfqpoint{3.345189in}{2.898104in}}%
\pgfpathcurveto{\pgfqpoint{3.345189in}{2.906340in}}{\pgfqpoint{3.341917in}{2.914240in}}{\pgfqpoint{3.336093in}{2.920064in}}%
\pgfpathcurveto{\pgfqpoint{3.330269in}{2.925888in}}{\pgfqpoint{3.322369in}{2.929160in}}{\pgfqpoint{3.314132in}{2.929160in}}%
\pgfpathcurveto{\pgfqpoint{3.305896in}{2.929160in}}{\pgfqpoint{3.297996in}{2.925888in}}{\pgfqpoint{3.292172in}{2.920064in}}%
\pgfpathcurveto{\pgfqpoint{3.286348in}{2.914240in}}{\pgfqpoint{3.283076in}{2.906340in}}{\pgfqpoint{3.283076in}{2.898104in}}%
\pgfpathcurveto{\pgfqpoint{3.283076in}{2.889867in}}{\pgfqpoint{3.286348in}{2.881967in}}{\pgfqpoint{3.292172in}{2.876143in}}%
\pgfpathcurveto{\pgfqpoint{3.297996in}{2.870319in}}{\pgfqpoint{3.305896in}{2.867047in}}{\pgfqpoint{3.314132in}{2.867047in}}%
\pgfpathclose%
\pgfusepath{stroke,fill}%
\end{pgfscope}%
\begin{pgfscope}%
\pgfpathrectangle{\pgfqpoint{0.100000in}{0.220728in}}{\pgfqpoint{3.696000in}{3.696000in}}%
\pgfusepath{clip}%
\pgfsetbuttcap%
\pgfsetroundjoin%
\definecolor{currentfill}{rgb}{0.121569,0.466667,0.705882}%
\pgfsetfillcolor{currentfill}%
\pgfsetfillopacity{0.701050}%
\pgfsetlinewidth{1.003750pt}%
\definecolor{currentstroke}{rgb}{0.121569,0.466667,0.705882}%
\pgfsetstrokecolor{currentstroke}%
\pgfsetstrokeopacity{0.701050}%
\pgfsetdash{}{0pt}%
\pgfpathmoveto{\pgfqpoint{0.786823in}{2.288000in}}%
\pgfpathcurveto{\pgfqpoint{0.795059in}{2.288000in}}{\pgfqpoint{0.802960in}{2.291272in}}{\pgfqpoint{0.808783in}{2.297096in}}%
\pgfpathcurveto{\pgfqpoint{0.814607in}{2.302920in}}{\pgfqpoint{0.817880in}{2.310820in}}{\pgfqpoint{0.817880in}{2.319057in}}%
\pgfpathcurveto{\pgfqpoint{0.817880in}{2.327293in}}{\pgfqpoint{0.814607in}{2.335193in}}{\pgfqpoint{0.808783in}{2.341017in}}%
\pgfpathcurveto{\pgfqpoint{0.802960in}{2.346841in}}{\pgfqpoint{0.795059in}{2.350113in}}{\pgfqpoint{0.786823in}{2.350113in}}%
\pgfpathcurveto{\pgfqpoint{0.778587in}{2.350113in}}{\pgfqpoint{0.770687in}{2.346841in}}{\pgfqpoint{0.764863in}{2.341017in}}%
\pgfpathcurveto{\pgfqpoint{0.759039in}{2.335193in}}{\pgfqpoint{0.755767in}{2.327293in}}{\pgfqpoint{0.755767in}{2.319057in}}%
\pgfpathcurveto{\pgfqpoint{0.755767in}{2.310820in}}{\pgfqpoint{0.759039in}{2.302920in}}{\pgfqpoint{0.764863in}{2.297096in}}%
\pgfpathcurveto{\pgfqpoint{0.770687in}{2.291272in}}{\pgfqpoint{0.778587in}{2.288000in}}{\pgfqpoint{0.786823in}{2.288000in}}%
\pgfpathclose%
\pgfusepath{stroke,fill}%
\end{pgfscope}%
\begin{pgfscope}%
\pgfpathrectangle{\pgfqpoint{0.100000in}{0.220728in}}{\pgfqpoint{3.696000in}{3.696000in}}%
\pgfusepath{clip}%
\pgfsetbuttcap%
\pgfsetroundjoin%
\definecolor{currentfill}{rgb}{0.121569,0.466667,0.705882}%
\pgfsetfillcolor{currentfill}%
\pgfsetfillopacity{0.702979}%
\pgfsetlinewidth{1.003750pt}%
\definecolor{currentstroke}{rgb}{0.121569,0.466667,0.705882}%
\pgfsetstrokecolor{currentstroke}%
\pgfsetstrokeopacity{0.702979}%
\pgfsetdash{}{0pt}%
\pgfpathmoveto{\pgfqpoint{0.803873in}{2.287549in}}%
\pgfpathcurveto{\pgfqpoint{0.812109in}{2.287549in}}{\pgfqpoint{0.820009in}{2.290821in}}{\pgfqpoint{0.825833in}{2.296645in}}%
\pgfpathcurveto{\pgfqpoint{0.831657in}{2.302469in}}{\pgfqpoint{0.834929in}{2.310369in}}{\pgfqpoint{0.834929in}{2.318606in}}%
\pgfpathcurveto{\pgfqpoint{0.834929in}{2.326842in}}{\pgfqpoint{0.831657in}{2.334742in}}{\pgfqpoint{0.825833in}{2.340566in}}%
\pgfpathcurveto{\pgfqpoint{0.820009in}{2.346390in}}{\pgfqpoint{0.812109in}{2.349662in}}{\pgfqpoint{0.803873in}{2.349662in}}%
\pgfpathcurveto{\pgfqpoint{0.795636in}{2.349662in}}{\pgfqpoint{0.787736in}{2.346390in}}{\pgfqpoint{0.781912in}{2.340566in}}%
\pgfpathcurveto{\pgfqpoint{0.776088in}{2.334742in}}{\pgfqpoint{0.772816in}{2.326842in}}{\pgfqpoint{0.772816in}{2.318606in}}%
\pgfpathcurveto{\pgfqpoint{0.772816in}{2.310369in}}{\pgfqpoint{0.776088in}{2.302469in}}{\pgfqpoint{0.781912in}{2.296645in}}%
\pgfpathcurveto{\pgfqpoint{0.787736in}{2.290821in}}{\pgfqpoint{0.795636in}{2.287549in}}{\pgfqpoint{0.803873in}{2.287549in}}%
\pgfpathclose%
\pgfusepath{stroke,fill}%
\end{pgfscope}%
\begin{pgfscope}%
\pgfpathrectangle{\pgfqpoint{0.100000in}{0.220728in}}{\pgfqpoint{3.696000in}{3.696000in}}%
\pgfusepath{clip}%
\pgfsetbuttcap%
\pgfsetroundjoin%
\definecolor{currentfill}{rgb}{0.121569,0.466667,0.705882}%
\pgfsetfillcolor{currentfill}%
\pgfsetfillopacity{0.703039}%
\pgfsetlinewidth{1.003750pt}%
\definecolor{currentstroke}{rgb}{0.121569,0.466667,0.705882}%
\pgfsetstrokecolor{currentstroke}%
\pgfsetstrokeopacity{0.703039}%
\pgfsetdash{}{0pt}%
\pgfpathmoveto{\pgfqpoint{3.315105in}{2.856479in}}%
\pgfpathcurveto{\pgfqpoint{3.323341in}{2.856479in}}{\pgfqpoint{3.331241in}{2.859752in}}{\pgfqpoint{3.337065in}{2.865576in}}%
\pgfpathcurveto{\pgfqpoint{3.342889in}{2.871400in}}{\pgfqpoint{3.346161in}{2.879300in}}{\pgfqpoint{3.346161in}{2.887536in}}%
\pgfpathcurveto{\pgfqpoint{3.346161in}{2.895772in}}{\pgfqpoint{3.342889in}{2.903672in}}{\pgfqpoint{3.337065in}{2.909496in}}%
\pgfpathcurveto{\pgfqpoint{3.331241in}{2.915320in}}{\pgfqpoint{3.323341in}{2.918592in}}{\pgfqpoint{3.315105in}{2.918592in}}%
\pgfpathcurveto{\pgfqpoint{3.306868in}{2.918592in}}{\pgfqpoint{3.298968in}{2.915320in}}{\pgfqpoint{3.293144in}{2.909496in}}%
\pgfpathcurveto{\pgfqpoint{3.287321in}{2.903672in}}{\pgfqpoint{3.284048in}{2.895772in}}{\pgfqpoint{3.284048in}{2.887536in}}%
\pgfpathcurveto{\pgfqpoint{3.284048in}{2.879300in}}{\pgfqpoint{3.287321in}{2.871400in}}{\pgfqpoint{3.293144in}{2.865576in}}%
\pgfpathcurveto{\pgfqpoint{3.298968in}{2.859752in}}{\pgfqpoint{3.306868in}{2.856479in}}{\pgfqpoint{3.315105in}{2.856479in}}%
\pgfpathclose%
\pgfusepath{stroke,fill}%
\end{pgfscope}%
\begin{pgfscope}%
\pgfpathrectangle{\pgfqpoint{0.100000in}{0.220728in}}{\pgfqpoint{3.696000in}{3.696000in}}%
\pgfusepath{clip}%
\pgfsetbuttcap%
\pgfsetroundjoin%
\definecolor{currentfill}{rgb}{0.121569,0.466667,0.705882}%
\pgfsetfillcolor{currentfill}%
\pgfsetfillopacity{0.704660}%
\pgfsetlinewidth{1.003750pt}%
\definecolor{currentstroke}{rgb}{0.121569,0.466667,0.705882}%
\pgfsetstrokecolor{currentstroke}%
\pgfsetstrokeopacity{0.704660}%
\pgfsetdash{}{0pt}%
\pgfpathmoveto{\pgfqpoint{3.314643in}{2.850054in}}%
\pgfpathcurveto{\pgfqpoint{3.322880in}{2.850054in}}{\pgfqpoint{3.330780in}{2.853326in}}{\pgfqpoint{3.336604in}{2.859150in}}%
\pgfpathcurveto{\pgfqpoint{3.342428in}{2.864974in}}{\pgfqpoint{3.345700in}{2.872874in}}{\pgfqpoint{3.345700in}{2.881111in}}%
\pgfpathcurveto{\pgfqpoint{3.345700in}{2.889347in}}{\pgfqpoint{3.342428in}{2.897247in}}{\pgfqpoint{3.336604in}{2.903071in}}%
\pgfpathcurveto{\pgfqpoint{3.330780in}{2.908895in}}{\pgfqpoint{3.322880in}{2.912167in}}{\pgfqpoint{3.314643in}{2.912167in}}%
\pgfpathcurveto{\pgfqpoint{3.306407in}{2.912167in}}{\pgfqpoint{3.298507in}{2.908895in}}{\pgfqpoint{3.292683in}{2.903071in}}%
\pgfpathcurveto{\pgfqpoint{3.286859in}{2.897247in}}{\pgfqpoint{3.283587in}{2.889347in}}{\pgfqpoint{3.283587in}{2.881111in}}%
\pgfpathcurveto{\pgfqpoint{3.283587in}{2.872874in}}{\pgfqpoint{3.286859in}{2.864974in}}{\pgfqpoint{3.292683in}{2.859150in}}%
\pgfpathcurveto{\pgfqpoint{3.298507in}{2.853326in}}{\pgfqpoint{3.306407in}{2.850054in}}{\pgfqpoint{3.314643in}{2.850054in}}%
\pgfpathclose%
\pgfusepath{stroke,fill}%
\end{pgfscope}%
\begin{pgfscope}%
\pgfpathrectangle{\pgfqpoint{0.100000in}{0.220728in}}{\pgfqpoint{3.696000in}{3.696000in}}%
\pgfusepath{clip}%
\pgfsetbuttcap%
\pgfsetroundjoin%
\definecolor{currentfill}{rgb}{0.121569,0.466667,0.705882}%
\pgfsetfillcolor{currentfill}%
\pgfsetfillopacity{0.705293}%
\pgfsetlinewidth{1.003750pt}%
\definecolor{currentstroke}{rgb}{0.121569,0.466667,0.705882}%
\pgfsetstrokecolor{currentstroke}%
\pgfsetstrokeopacity{0.705293}%
\pgfsetdash{}{0pt}%
\pgfpathmoveto{\pgfqpoint{0.811111in}{2.270819in}}%
\pgfpathcurveto{\pgfqpoint{0.819348in}{2.270819in}}{\pgfqpoint{0.827248in}{2.274092in}}{\pgfqpoint{0.833072in}{2.279916in}}%
\pgfpathcurveto{\pgfqpoint{0.838896in}{2.285740in}}{\pgfqpoint{0.842168in}{2.293640in}}{\pgfqpoint{0.842168in}{2.301876in}}%
\pgfpathcurveto{\pgfqpoint{0.842168in}{2.310112in}}{\pgfqpoint{0.838896in}{2.318012in}}{\pgfqpoint{0.833072in}{2.323836in}}%
\pgfpathcurveto{\pgfqpoint{0.827248in}{2.329660in}}{\pgfqpoint{0.819348in}{2.332932in}}{\pgfqpoint{0.811111in}{2.332932in}}%
\pgfpathcurveto{\pgfqpoint{0.802875in}{2.332932in}}{\pgfqpoint{0.794975in}{2.329660in}}{\pgfqpoint{0.789151in}{2.323836in}}%
\pgfpathcurveto{\pgfqpoint{0.783327in}{2.318012in}}{\pgfqpoint{0.780055in}{2.310112in}}{\pgfqpoint{0.780055in}{2.301876in}}%
\pgfpathcurveto{\pgfqpoint{0.780055in}{2.293640in}}{\pgfqpoint{0.783327in}{2.285740in}}{\pgfqpoint{0.789151in}{2.279916in}}%
\pgfpathcurveto{\pgfqpoint{0.794975in}{2.274092in}}{\pgfqpoint{0.802875in}{2.270819in}}{\pgfqpoint{0.811111in}{2.270819in}}%
\pgfpathclose%
\pgfusepath{stroke,fill}%
\end{pgfscope}%
\begin{pgfscope}%
\pgfpathrectangle{\pgfqpoint{0.100000in}{0.220728in}}{\pgfqpoint{3.696000in}{3.696000in}}%
\pgfusepath{clip}%
\pgfsetbuttcap%
\pgfsetroundjoin%
\definecolor{currentfill}{rgb}{0.121569,0.466667,0.705882}%
\pgfsetfillcolor{currentfill}%
\pgfsetfillopacity{0.706004}%
\pgfsetlinewidth{1.003750pt}%
\definecolor{currentstroke}{rgb}{0.121569,0.466667,0.705882}%
\pgfsetstrokecolor{currentstroke}%
\pgfsetstrokeopacity{0.706004}%
\pgfsetdash{}{0pt}%
\pgfpathmoveto{\pgfqpoint{0.824289in}{2.281766in}}%
\pgfpathcurveto{\pgfqpoint{0.832525in}{2.281766in}}{\pgfqpoint{0.840425in}{2.285039in}}{\pgfqpoint{0.846249in}{2.290863in}}%
\pgfpathcurveto{\pgfqpoint{0.852073in}{2.296687in}}{\pgfqpoint{0.855346in}{2.304587in}}{\pgfqpoint{0.855346in}{2.312823in}}%
\pgfpathcurveto{\pgfqpoint{0.855346in}{2.321059in}}{\pgfqpoint{0.852073in}{2.328959in}}{\pgfqpoint{0.846249in}{2.334783in}}%
\pgfpathcurveto{\pgfqpoint{0.840425in}{2.340607in}}{\pgfqpoint{0.832525in}{2.343879in}}{\pgfqpoint{0.824289in}{2.343879in}}%
\pgfpathcurveto{\pgfqpoint{0.816053in}{2.343879in}}{\pgfqpoint{0.808153in}{2.340607in}}{\pgfqpoint{0.802329in}{2.334783in}}%
\pgfpathcurveto{\pgfqpoint{0.796505in}{2.328959in}}{\pgfqpoint{0.793233in}{2.321059in}}{\pgfqpoint{0.793233in}{2.312823in}}%
\pgfpathcurveto{\pgfqpoint{0.793233in}{2.304587in}}{\pgfqpoint{0.796505in}{2.296687in}}{\pgfqpoint{0.802329in}{2.290863in}}%
\pgfpathcurveto{\pgfqpoint{0.808153in}{2.285039in}}{\pgfqpoint{0.816053in}{2.281766in}}{\pgfqpoint{0.824289in}{2.281766in}}%
\pgfpathclose%
\pgfusepath{stroke,fill}%
\end{pgfscope}%
\begin{pgfscope}%
\pgfpathrectangle{\pgfqpoint{0.100000in}{0.220728in}}{\pgfqpoint{3.696000in}{3.696000in}}%
\pgfusepath{clip}%
\pgfsetbuttcap%
\pgfsetroundjoin%
\definecolor{currentfill}{rgb}{0.121569,0.466667,0.705882}%
\pgfsetfillcolor{currentfill}%
\pgfsetfillopacity{0.706865}%
\pgfsetlinewidth{1.003750pt}%
\definecolor{currentstroke}{rgb}{0.121569,0.466667,0.705882}%
\pgfsetstrokecolor{currentstroke}%
\pgfsetstrokeopacity{0.706865}%
\pgfsetdash{}{0pt}%
\pgfpathmoveto{\pgfqpoint{3.311924in}{2.842080in}}%
\pgfpathcurveto{\pgfqpoint{3.320160in}{2.842080in}}{\pgfqpoint{3.328060in}{2.845353in}}{\pgfqpoint{3.333884in}{2.851177in}}%
\pgfpathcurveto{\pgfqpoint{3.339708in}{2.857001in}}{\pgfqpoint{3.342980in}{2.864901in}}{\pgfqpoint{3.342980in}{2.873137in}}%
\pgfpathcurveto{\pgfqpoint{3.342980in}{2.881373in}}{\pgfqpoint{3.339708in}{2.889273in}}{\pgfqpoint{3.333884in}{2.895097in}}%
\pgfpathcurveto{\pgfqpoint{3.328060in}{2.900921in}}{\pgfqpoint{3.320160in}{2.904193in}}{\pgfqpoint{3.311924in}{2.904193in}}%
\pgfpathcurveto{\pgfqpoint{3.303687in}{2.904193in}}{\pgfqpoint{3.295787in}{2.900921in}}{\pgfqpoint{3.289963in}{2.895097in}}%
\pgfpathcurveto{\pgfqpoint{3.284140in}{2.889273in}}{\pgfqpoint{3.280867in}{2.881373in}}{\pgfqpoint{3.280867in}{2.873137in}}%
\pgfpathcurveto{\pgfqpoint{3.280867in}{2.864901in}}{\pgfqpoint{3.284140in}{2.857001in}}{\pgfqpoint{3.289963in}{2.851177in}}%
\pgfpathcurveto{\pgfqpoint{3.295787in}{2.845353in}}{\pgfqpoint{3.303687in}{2.842080in}}{\pgfqpoint{3.311924in}{2.842080in}}%
\pgfpathclose%
\pgfusepath{stroke,fill}%
\end{pgfscope}%
\begin{pgfscope}%
\pgfpathrectangle{\pgfqpoint{0.100000in}{0.220728in}}{\pgfqpoint{3.696000in}{3.696000in}}%
\pgfusepath{clip}%
\pgfsetbuttcap%
\pgfsetroundjoin%
\definecolor{currentfill}{rgb}{0.121569,0.466667,0.705882}%
\pgfsetfillcolor{currentfill}%
\pgfsetfillopacity{0.707351}%
\pgfsetlinewidth{1.003750pt}%
\definecolor{currentstroke}{rgb}{0.121569,0.466667,0.705882}%
\pgfsetstrokecolor{currentstroke}%
\pgfsetstrokeopacity{0.707351}%
\pgfsetdash{}{0pt}%
\pgfpathmoveto{\pgfqpoint{0.832055in}{2.275121in}}%
\pgfpathcurveto{\pgfqpoint{0.840291in}{2.275121in}}{\pgfqpoint{0.848191in}{2.278393in}}{\pgfqpoint{0.854015in}{2.284217in}}%
\pgfpathcurveto{\pgfqpoint{0.859839in}{2.290041in}}{\pgfqpoint{0.863112in}{2.297941in}}{\pgfqpoint{0.863112in}{2.306178in}}%
\pgfpathcurveto{\pgfqpoint{0.863112in}{2.314414in}}{\pgfqpoint{0.859839in}{2.322314in}}{\pgfqpoint{0.854015in}{2.328138in}}%
\pgfpathcurveto{\pgfqpoint{0.848191in}{2.333962in}}{\pgfqpoint{0.840291in}{2.337234in}}{\pgfqpoint{0.832055in}{2.337234in}}%
\pgfpathcurveto{\pgfqpoint{0.823819in}{2.337234in}}{\pgfqpoint{0.815919in}{2.333962in}}{\pgfqpoint{0.810095in}{2.328138in}}%
\pgfpathcurveto{\pgfqpoint{0.804271in}{2.322314in}}{\pgfqpoint{0.800999in}{2.314414in}}{\pgfqpoint{0.800999in}{2.306178in}}%
\pgfpathcurveto{\pgfqpoint{0.800999in}{2.297941in}}{\pgfqpoint{0.804271in}{2.290041in}}{\pgfqpoint{0.810095in}{2.284217in}}%
\pgfpathcurveto{\pgfqpoint{0.815919in}{2.278393in}}{\pgfqpoint{0.823819in}{2.275121in}}{\pgfqpoint{0.832055in}{2.275121in}}%
\pgfpathclose%
\pgfusepath{stroke,fill}%
\end{pgfscope}%
\begin{pgfscope}%
\pgfpathrectangle{\pgfqpoint{0.100000in}{0.220728in}}{\pgfqpoint{3.696000in}{3.696000in}}%
\pgfusepath{clip}%
\pgfsetbuttcap%
\pgfsetroundjoin%
\definecolor{currentfill}{rgb}{0.121569,0.466667,0.705882}%
\pgfsetfillcolor{currentfill}%
\pgfsetfillopacity{0.709212}%
\pgfsetlinewidth{1.003750pt}%
\definecolor{currentstroke}{rgb}{0.121569,0.466667,0.705882}%
\pgfsetstrokecolor{currentstroke}%
\pgfsetstrokeopacity{0.709212}%
\pgfsetdash{}{0pt}%
\pgfpathmoveto{\pgfqpoint{3.305062in}{2.831243in}}%
\pgfpathcurveto{\pgfqpoint{3.313299in}{2.831243in}}{\pgfqpoint{3.321199in}{2.834515in}}{\pgfqpoint{3.327023in}{2.840339in}}%
\pgfpathcurveto{\pgfqpoint{3.332847in}{2.846163in}}{\pgfqpoint{3.336119in}{2.854063in}}{\pgfqpoint{3.336119in}{2.862299in}}%
\pgfpathcurveto{\pgfqpoint{3.336119in}{2.870536in}}{\pgfqpoint{3.332847in}{2.878436in}}{\pgfqpoint{3.327023in}{2.884260in}}%
\pgfpathcurveto{\pgfqpoint{3.321199in}{2.890084in}}{\pgfqpoint{3.313299in}{2.893356in}}{\pgfqpoint{3.305062in}{2.893356in}}%
\pgfpathcurveto{\pgfqpoint{3.296826in}{2.893356in}}{\pgfqpoint{3.288926in}{2.890084in}}{\pgfqpoint{3.283102in}{2.884260in}}%
\pgfpathcurveto{\pgfqpoint{3.277278in}{2.878436in}}{\pgfqpoint{3.274006in}{2.870536in}}{\pgfqpoint{3.274006in}{2.862299in}}%
\pgfpathcurveto{\pgfqpoint{3.274006in}{2.854063in}}{\pgfqpoint{3.277278in}{2.846163in}}{\pgfqpoint{3.283102in}{2.840339in}}%
\pgfpathcurveto{\pgfqpoint{3.288926in}{2.834515in}}{\pgfqpoint{3.296826in}{2.831243in}}{\pgfqpoint{3.305062in}{2.831243in}}%
\pgfpathclose%
\pgfusepath{stroke,fill}%
\end{pgfscope}%
\begin{pgfscope}%
\pgfpathrectangle{\pgfqpoint{0.100000in}{0.220728in}}{\pgfqpoint{3.696000in}{3.696000in}}%
\pgfusepath{clip}%
\pgfsetbuttcap%
\pgfsetroundjoin%
\definecolor{currentfill}{rgb}{0.121569,0.466667,0.705882}%
\pgfsetfillcolor{currentfill}%
\pgfsetfillopacity{0.709981}%
\pgfsetlinewidth{1.003750pt}%
\definecolor{currentstroke}{rgb}{0.121569,0.466667,0.705882}%
\pgfsetstrokecolor{currentstroke}%
\pgfsetstrokeopacity{0.709981}%
\pgfsetdash{}{0pt}%
\pgfpathmoveto{\pgfqpoint{0.845581in}{2.262526in}}%
\pgfpathcurveto{\pgfqpoint{0.853817in}{2.262526in}}{\pgfqpoint{0.861717in}{2.265798in}}{\pgfqpoint{0.867541in}{2.271622in}}%
\pgfpathcurveto{\pgfqpoint{0.873365in}{2.277446in}}{\pgfqpoint{0.876637in}{2.285346in}}{\pgfqpoint{0.876637in}{2.293582in}}%
\pgfpathcurveto{\pgfqpoint{0.876637in}{2.301819in}}{\pgfqpoint{0.873365in}{2.309719in}}{\pgfqpoint{0.867541in}{2.315543in}}%
\pgfpathcurveto{\pgfqpoint{0.861717in}{2.321366in}}{\pgfqpoint{0.853817in}{2.324639in}}{\pgfqpoint{0.845581in}{2.324639in}}%
\pgfpathcurveto{\pgfqpoint{0.837344in}{2.324639in}}{\pgfqpoint{0.829444in}{2.321366in}}{\pgfqpoint{0.823620in}{2.315543in}}%
\pgfpathcurveto{\pgfqpoint{0.817796in}{2.309719in}}{\pgfqpoint{0.814524in}{2.301819in}}{\pgfqpoint{0.814524in}{2.293582in}}%
\pgfpathcurveto{\pgfqpoint{0.814524in}{2.285346in}}{\pgfqpoint{0.817796in}{2.277446in}}{\pgfqpoint{0.823620in}{2.271622in}}%
\pgfpathcurveto{\pgfqpoint{0.829444in}{2.265798in}}{\pgfqpoint{0.837344in}{2.262526in}}{\pgfqpoint{0.845581in}{2.262526in}}%
\pgfpathclose%
\pgfusepath{stroke,fill}%
\end{pgfscope}%
\begin{pgfscope}%
\pgfpathrectangle{\pgfqpoint{0.100000in}{0.220728in}}{\pgfqpoint{3.696000in}{3.696000in}}%
\pgfusepath{clip}%
\pgfsetbuttcap%
\pgfsetroundjoin%
\definecolor{currentfill}{rgb}{0.121569,0.466667,0.705882}%
\pgfsetfillcolor{currentfill}%
\pgfsetfillopacity{0.711292}%
\pgfsetlinewidth{1.003750pt}%
\definecolor{currentstroke}{rgb}{0.121569,0.466667,0.705882}%
\pgfsetstrokecolor{currentstroke}%
\pgfsetstrokeopacity{0.711292}%
\pgfsetdash{}{0pt}%
\pgfpathmoveto{\pgfqpoint{0.856261in}{2.264164in}}%
\pgfpathcurveto{\pgfqpoint{0.864497in}{2.264164in}}{\pgfqpoint{0.872397in}{2.267437in}}{\pgfqpoint{0.878221in}{2.273261in}}%
\pgfpathcurveto{\pgfqpoint{0.884045in}{2.279085in}}{\pgfqpoint{0.887317in}{2.286985in}}{\pgfqpoint{0.887317in}{2.295221in}}%
\pgfpathcurveto{\pgfqpoint{0.887317in}{2.303457in}}{\pgfqpoint{0.884045in}{2.311357in}}{\pgfqpoint{0.878221in}{2.317181in}}%
\pgfpathcurveto{\pgfqpoint{0.872397in}{2.323005in}}{\pgfqpoint{0.864497in}{2.326277in}}{\pgfqpoint{0.856261in}{2.326277in}}%
\pgfpathcurveto{\pgfqpoint{0.848024in}{2.326277in}}{\pgfqpoint{0.840124in}{2.323005in}}{\pgfqpoint{0.834300in}{2.317181in}}%
\pgfpathcurveto{\pgfqpoint{0.828476in}{2.311357in}}{\pgfqpoint{0.825204in}{2.303457in}}{\pgfqpoint{0.825204in}{2.295221in}}%
\pgfpathcurveto{\pgfqpoint{0.825204in}{2.286985in}}{\pgfqpoint{0.828476in}{2.279085in}}{\pgfqpoint{0.834300in}{2.273261in}}%
\pgfpathcurveto{\pgfqpoint{0.840124in}{2.267437in}}{\pgfqpoint{0.848024in}{2.264164in}}{\pgfqpoint{0.856261in}{2.264164in}}%
\pgfpathclose%
\pgfusepath{stroke,fill}%
\end{pgfscope}%
\begin{pgfscope}%
\pgfpathrectangle{\pgfqpoint{0.100000in}{0.220728in}}{\pgfqpoint{3.696000in}{3.696000in}}%
\pgfusepath{clip}%
\pgfsetbuttcap%
\pgfsetroundjoin%
\definecolor{currentfill}{rgb}{0.121569,0.466667,0.705882}%
\pgfsetfillcolor{currentfill}%
\pgfsetfillopacity{0.712197}%
\pgfsetlinewidth{1.003750pt}%
\definecolor{currentstroke}{rgb}{0.121569,0.466667,0.705882}%
\pgfsetstrokecolor{currentstroke}%
\pgfsetstrokeopacity{0.712197}%
\pgfsetdash{}{0pt}%
\pgfpathmoveto{\pgfqpoint{0.858387in}{2.259116in}}%
\pgfpathcurveto{\pgfqpoint{0.866624in}{2.259116in}}{\pgfqpoint{0.874524in}{2.262389in}}{\pgfqpoint{0.880348in}{2.268213in}}%
\pgfpathcurveto{\pgfqpoint{0.886172in}{2.274037in}}{\pgfqpoint{0.889444in}{2.281937in}}{\pgfqpoint{0.889444in}{2.290173in}}%
\pgfpathcurveto{\pgfqpoint{0.889444in}{2.298409in}}{\pgfqpoint{0.886172in}{2.306309in}}{\pgfqpoint{0.880348in}{2.312133in}}%
\pgfpathcurveto{\pgfqpoint{0.874524in}{2.317957in}}{\pgfqpoint{0.866624in}{2.321229in}}{\pgfqpoint{0.858387in}{2.321229in}}%
\pgfpathcurveto{\pgfqpoint{0.850151in}{2.321229in}}{\pgfqpoint{0.842251in}{2.317957in}}{\pgfqpoint{0.836427in}{2.312133in}}%
\pgfpathcurveto{\pgfqpoint{0.830603in}{2.306309in}}{\pgfqpoint{0.827331in}{2.298409in}}{\pgfqpoint{0.827331in}{2.290173in}}%
\pgfpathcurveto{\pgfqpoint{0.827331in}{2.281937in}}{\pgfqpoint{0.830603in}{2.274037in}}{\pgfqpoint{0.836427in}{2.268213in}}%
\pgfpathcurveto{\pgfqpoint{0.842251in}{2.262389in}}{\pgfqpoint{0.850151in}{2.259116in}}{\pgfqpoint{0.858387in}{2.259116in}}%
\pgfpathclose%
\pgfusepath{stroke,fill}%
\end{pgfscope}%
\begin{pgfscope}%
\pgfpathrectangle{\pgfqpoint{0.100000in}{0.220728in}}{\pgfqpoint{3.696000in}{3.696000in}}%
\pgfusepath{clip}%
\pgfsetbuttcap%
\pgfsetroundjoin%
\definecolor{currentfill}{rgb}{0.121569,0.466667,0.705882}%
\pgfsetfillcolor{currentfill}%
\pgfsetfillopacity{0.712387}%
\pgfsetlinewidth{1.003750pt}%
\definecolor{currentstroke}{rgb}{0.121569,0.466667,0.705882}%
\pgfsetstrokecolor{currentstroke}%
\pgfsetstrokeopacity{0.712387}%
\pgfsetdash{}{0pt}%
\pgfpathmoveto{\pgfqpoint{3.296522in}{2.818015in}}%
\pgfpathcurveto{\pgfqpoint{3.304758in}{2.818015in}}{\pgfqpoint{3.312658in}{2.821287in}}{\pgfqpoint{3.318482in}{2.827111in}}%
\pgfpathcurveto{\pgfqpoint{3.324306in}{2.832935in}}{\pgfqpoint{3.327578in}{2.840835in}}{\pgfqpoint{3.327578in}{2.849072in}}%
\pgfpathcurveto{\pgfqpoint{3.327578in}{2.857308in}}{\pgfqpoint{3.324306in}{2.865208in}}{\pgfqpoint{3.318482in}{2.871032in}}%
\pgfpathcurveto{\pgfqpoint{3.312658in}{2.876856in}}{\pgfqpoint{3.304758in}{2.880128in}}{\pgfqpoint{3.296522in}{2.880128in}}%
\pgfpathcurveto{\pgfqpoint{3.288285in}{2.880128in}}{\pgfqpoint{3.280385in}{2.876856in}}{\pgfqpoint{3.274561in}{2.871032in}}%
\pgfpathcurveto{\pgfqpoint{3.268737in}{2.865208in}}{\pgfqpoint{3.265465in}{2.857308in}}{\pgfqpoint{3.265465in}{2.849072in}}%
\pgfpathcurveto{\pgfqpoint{3.265465in}{2.840835in}}{\pgfqpoint{3.268737in}{2.832935in}}{\pgfqpoint{3.274561in}{2.827111in}}%
\pgfpathcurveto{\pgfqpoint{3.280385in}{2.821287in}}{\pgfqpoint{3.288285in}{2.818015in}}{\pgfqpoint{3.296522in}{2.818015in}}%
\pgfpathclose%
\pgfusepath{stroke,fill}%
\end{pgfscope}%
\begin{pgfscope}%
\pgfpathrectangle{\pgfqpoint{0.100000in}{0.220728in}}{\pgfqpoint{3.696000in}{3.696000in}}%
\pgfusepath{clip}%
\pgfsetbuttcap%
\pgfsetroundjoin%
\definecolor{currentfill}{rgb}{0.121569,0.466667,0.705882}%
\pgfsetfillcolor{currentfill}%
\pgfsetfillopacity{0.713754}%
\pgfsetlinewidth{1.003750pt}%
\definecolor{currentstroke}{rgb}{0.121569,0.466667,0.705882}%
\pgfsetstrokecolor{currentstroke}%
\pgfsetstrokeopacity{0.713754}%
\pgfsetdash{}{0pt}%
\pgfpathmoveto{\pgfqpoint{0.864646in}{2.253092in}}%
\pgfpathcurveto{\pgfqpoint{0.872883in}{2.253092in}}{\pgfqpoint{0.880783in}{2.256365in}}{\pgfqpoint{0.886607in}{2.262188in}}%
\pgfpathcurveto{\pgfqpoint{0.892431in}{2.268012in}}{\pgfqpoint{0.895703in}{2.275912in}}{\pgfqpoint{0.895703in}{2.284149in}}%
\pgfpathcurveto{\pgfqpoint{0.895703in}{2.292385in}}{\pgfqpoint{0.892431in}{2.300285in}}{\pgfqpoint{0.886607in}{2.306109in}}%
\pgfpathcurveto{\pgfqpoint{0.880783in}{2.311933in}}{\pgfqpoint{0.872883in}{2.315205in}}{\pgfqpoint{0.864646in}{2.315205in}}%
\pgfpathcurveto{\pgfqpoint{0.856410in}{2.315205in}}{\pgfqpoint{0.848510in}{2.311933in}}{\pgfqpoint{0.842686in}{2.306109in}}%
\pgfpathcurveto{\pgfqpoint{0.836862in}{2.300285in}}{\pgfqpoint{0.833590in}{2.292385in}}{\pgfqpoint{0.833590in}{2.284149in}}%
\pgfpathcurveto{\pgfqpoint{0.833590in}{2.275912in}}{\pgfqpoint{0.836862in}{2.268012in}}{\pgfqpoint{0.842686in}{2.262188in}}%
\pgfpathcurveto{\pgfqpoint{0.848510in}{2.256365in}}{\pgfqpoint{0.856410in}{2.253092in}}{\pgfqpoint{0.864646in}{2.253092in}}%
\pgfpathclose%
\pgfusepath{stroke,fill}%
\end{pgfscope}%
\begin{pgfscope}%
\pgfpathrectangle{\pgfqpoint{0.100000in}{0.220728in}}{\pgfqpoint{3.696000in}{3.696000in}}%
\pgfusepath{clip}%
\pgfsetbuttcap%
\pgfsetroundjoin%
\definecolor{currentfill}{rgb}{0.121569,0.466667,0.705882}%
\pgfsetfillcolor{currentfill}%
\pgfsetfillopacity{0.714838}%
\pgfsetlinewidth{1.003750pt}%
\definecolor{currentstroke}{rgb}{0.121569,0.466667,0.705882}%
\pgfsetstrokecolor{currentstroke}%
\pgfsetstrokeopacity{0.714838}%
\pgfsetdash{}{0pt}%
\pgfpathmoveto{\pgfqpoint{0.869069in}{2.247114in}}%
\pgfpathcurveto{\pgfqpoint{0.877306in}{2.247114in}}{\pgfqpoint{0.885206in}{2.250386in}}{\pgfqpoint{0.891030in}{2.256210in}}%
\pgfpathcurveto{\pgfqpoint{0.896854in}{2.262034in}}{\pgfqpoint{0.900126in}{2.269934in}}{\pgfqpoint{0.900126in}{2.278170in}}%
\pgfpathcurveto{\pgfqpoint{0.900126in}{2.286406in}}{\pgfqpoint{0.896854in}{2.294306in}}{\pgfqpoint{0.891030in}{2.300130in}}%
\pgfpathcurveto{\pgfqpoint{0.885206in}{2.305954in}}{\pgfqpoint{0.877306in}{2.309227in}}{\pgfqpoint{0.869069in}{2.309227in}}%
\pgfpathcurveto{\pgfqpoint{0.860833in}{2.309227in}}{\pgfqpoint{0.852933in}{2.305954in}}{\pgfqpoint{0.847109in}{2.300130in}}%
\pgfpathcurveto{\pgfqpoint{0.841285in}{2.294306in}}{\pgfqpoint{0.838013in}{2.286406in}}{\pgfqpoint{0.838013in}{2.278170in}}%
\pgfpathcurveto{\pgfqpoint{0.838013in}{2.269934in}}{\pgfqpoint{0.841285in}{2.262034in}}{\pgfqpoint{0.847109in}{2.256210in}}%
\pgfpathcurveto{\pgfqpoint{0.852933in}{2.250386in}}{\pgfqpoint{0.860833in}{2.247114in}}{\pgfqpoint{0.869069in}{2.247114in}}%
\pgfpathclose%
\pgfusepath{stroke,fill}%
\end{pgfscope}%
\begin{pgfscope}%
\pgfpathrectangle{\pgfqpoint{0.100000in}{0.220728in}}{\pgfqpoint{3.696000in}{3.696000in}}%
\pgfusepath{clip}%
\pgfsetbuttcap%
\pgfsetroundjoin%
\definecolor{currentfill}{rgb}{0.121569,0.466667,0.705882}%
\pgfsetfillcolor{currentfill}%
\pgfsetfillopacity{0.715024}%
\pgfsetlinewidth{1.003750pt}%
\definecolor{currentstroke}{rgb}{0.121569,0.466667,0.705882}%
\pgfsetstrokecolor{currentstroke}%
\pgfsetstrokeopacity{0.715024}%
\pgfsetdash{}{0pt}%
\pgfpathmoveto{\pgfqpoint{0.869748in}{2.246163in}}%
\pgfpathcurveto{\pgfqpoint{0.877985in}{2.246163in}}{\pgfqpoint{0.885885in}{2.249435in}}{\pgfqpoint{0.891709in}{2.255259in}}%
\pgfpathcurveto{\pgfqpoint{0.897532in}{2.261083in}}{\pgfqpoint{0.900805in}{2.268983in}}{\pgfqpoint{0.900805in}{2.277219in}}%
\pgfpathcurveto{\pgfqpoint{0.900805in}{2.285456in}}{\pgfqpoint{0.897532in}{2.293356in}}{\pgfqpoint{0.891709in}{2.299180in}}%
\pgfpathcurveto{\pgfqpoint{0.885885in}{2.305004in}}{\pgfqpoint{0.877985in}{2.308276in}}{\pgfqpoint{0.869748in}{2.308276in}}%
\pgfpathcurveto{\pgfqpoint{0.861512in}{2.308276in}}{\pgfqpoint{0.853612in}{2.305004in}}{\pgfqpoint{0.847788in}{2.299180in}}%
\pgfpathcurveto{\pgfqpoint{0.841964in}{2.293356in}}{\pgfqpoint{0.838692in}{2.285456in}}{\pgfqpoint{0.838692in}{2.277219in}}%
\pgfpathcurveto{\pgfqpoint{0.838692in}{2.268983in}}{\pgfqpoint{0.841964in}{2.261083in}}{\pgfqpoint{0.847788in}{2.255259in}}%
\pgfpathcurveto{\pgfqpoint{0.853612in}{2.249435in}}{\pgfqpoint{0.861512in}{2.246163in}}{\pgfqpoint{0.869748in}{2.246163in}}%
\pgfpathclose%
\pgfusepath{stroke,fill}%
\end{pgfscope}%
\begin{pgfscope}%
\pgfpathrectangle{\pgfqpoint{0.100000in}{0.220728in}}{\pgfqpoint{3.696000in}{3.696000in}}%
\pgfusepath{clip}%
\pgfsetbuttcap%
\pgfsetroundjoin%
\definecolor{currentfill}{rgb}{0.121569,0.466667,0.705882}%
\pgfsetfillcolor{currentfill}%
\pgfsetfillopacity{0.715367}%
\pgfsetlinewidth{1.003750pt}%
\definecolor{currentstroke}{rgb}{0.121569,0.466667,0.705882}%
\pgfsetstrokecolor{currentstroke}%
\pgfsetstrokeopacity{0.715367}%
\pgfsetdash{}{0pt}%
\pgfpathmoveto{\pgfqpoint{0.871196in}{2.244855in}}%
\pgfpathcurveto{\pgfqpoint{0.879432in}{2.244855in}}{\pgfqpoint{0.887332in}{2.248128in}}{\pgfqpoint{0.893156in}{2.253951in}}%
\pgfpathcurveto{\pgfqpoint{0.898980in}{2.259775in}}{\pgfqpoint{0.902252in}{2.267675in}}{\pgfqpoint{0.902252in}{2.275912in}}%
\pgfpathcurveto{\pgfqpoint{0.902252in}{2.284148in}}{\pgfqpoint{0.898980in}{2.292048in}}{\pgfqpoint{0.893156in}{2.297872in}}%
\pgfpathcurveto{\pgfqpoint{0.887332in}{2.303696in}}{\pgfqpoint{0.879432in}{2.306968in}}{\pgfqpoint{0.871196in}{2.306968in}}%
\pgfpathcurveto{\pgfqpoint{0.862960in}{2.306968in}}{\pgfqpoint{0.855060in}{2.303696in}}{\pgfqpoint{0.849236in}{2.297872in}}%
\pgfpathcurveto{\pgfqpoint{0.843412in}{2.292048in}}{\pgfqpoint{0.840139in}{2.284148in}}{\pgfqpoint{0.840139in}{2.275912in}}%
\pgfpathcurveto{\pgfqpoint{0.840139in}{2.267675in}}{\pgfqpoint{0.843412in}{2.259775in}}{\pgfqpoint{0.849236in}{2.253951in}}%
\pgfpathcurveto{\pgfqpoint{0.855060in}{2.248128in}}{\pgfqpoint{0.862960in}{2.244855in}}{\pgfqpoint{0.871196in}{2.244855in}}%
\pgfpathclose%
\pgfusepath{stroke,fill}%
\end{pgfscope}%
\begin{pgfscope}%
\pgfpathrectangle{\pgfqpoint{0.100000in}{0.220728in}}{\pgfqpoint{3.696000in}{3.696000in}}%
\pgfusepath{clip}%
\pgfsetbuttcap%
\pgfsetroundjoin%
\definecolor{currentfill}{rgb}{0.121569,0.466667,0.705882}%
\pgfsetfillcolor{currentfill}%
\pgfsetfillopacity{0.715709}%
\pgfsetlinewidth{1.003750pt}%
\definecolor{currentstroke}{rgb}{0.121569,0.466667,0.705882}%
\pgfsetstrokecolor{currentstroke}%
\pgfsetstrokeopacity{0.715709}%
\pgfsetdash{}{0pt}%
\pgfpathmoveto{\pgfqpoint{3.284508in}{2.799530in}}%
\pgfpathcurveto{\pgfqpoint{3.292744in}{2.799530in}}{\pgfqpoint{3.300644in}{2.802802in}}{\pgfqpoint{3.306468in}{2.808626in}}%
\pgfpathcurveto{\pgfqpoint{3.312292in}{2.814450in}}{\pgfqpoint{3.315565in}{2.822350in}}{\pgfqpoint{3.315565in}{2.830586in}}%
\pgfpathcurveto{\pgfqpoint{3.315565in}{2.838823in}}{\pgfqpoint{3.312292in}{2.846723in}}{\pgfqpoint{3.306468in}{2.852547in}}%
\pgfpathcurveto{\pgfqpoint{3.300644in}{2.858371in}}{\pgfqpoint{3.292744in}{2.861643in}}{\pgfqpoint{3.284508in}{2.861643in}}%
\pgfpathcurveto{\pgfqpoint{3.276272in}{2.861643in}}{\pgfqpoint{3.268372in}{2.858371in}}{\pgfqpoint{3.262548in}{2.852547in}}%
\pgfpathcurveto{\pgfqpoint{3.256724in}{2.846723in}}{\pgfqpoint{3.253452in}{2.838823in}}{\pgfqpoint{3.253452in}{2.830586in}}%
\pgfpathcurveto{\pgfqpoint{3.253452in}{2.822350in}}{\pgfqpoint{3.256724in}{2.814450in}}{\pgfqpoint{3.262548in}{2.808626in}}%
\pgfpathcurveto{\pgfqpoint{3.268372in}{2.802802in}}{\pgfqpoint{3.276272in}{2.799530in}}{\pgfqpoint{3.284508in}{2.799530in}}%
\pgfpathclose%
\pgfusepath{stroke,fill}%
\end{pgfscope}%
\begin{pgfscope}%
\pgfpathrectangle{\pgfqpoint{0.100000in}{0.220728in}}{\pgfqpoint{3.696000in}{3.696000in}}%
\pgfusepath{clip}%
\pgfsetbuttcap%
\pgfsetroundjoin%
\definecolor{currentfill}{rgb}{0.121569,0.466667,0.705882}%
\pgfsetfillcolor{currentfill}%
\pgfsetfillopacity{0.715969}%
\pgfsetlinewidth{1.003750pt}%
\definecolor{currentstroke}{rgb}{0.121569,0.466667,0.705882}%
\pgfsetstrokecolor{currentstroke}%
\pgfsetstrokeopacity{0.715969}%
\pgfsetdash{}{0pt}%
\pgfpathmoveto{\pgfqpoint{0.873526in}{2.241795in}}%
\pgfpathcurveto{\pgfqpoint{0.881762in}{2.241795in}}{\pgfqpoint{0.889662in}{2.245067in}}{\pgfqpoint{0.895486in}{2.250891in}}%
\pgfpathcurveto{\pgfqpoint{0.901310in}{2.256715in}}{\pgfqpoint{0.904582in}{2.264615in}}{\pgfqpoint{0.904582in}{2.272852in}}%
\pgfpathcurveto{\pgfqpoint{0.904582in}{2.281088in}}{\pgfqpoint{0.901310in}{2.288988in}}{\pgfqpoint{0.895486in}{2.294812in}}%
\pgfpathcurveto{\pgfqpoint{0.889662in}{2.300636in}}{\pgfqpoint{0.881762in}{2.303908in}}{\pgfqpoint{0.873526in}{2.303908in}}%
\pgfpathcurveto{\pgfqpoint{0.865290in}{2.303908in}}{\pgfqpoint{0.857390in}{2.300636in}}{\pgfqpoint{0.851566in}{2.294812in}}%
\pgfpathcurveto{\pgfqpoint{0.845742in}{2.288988in}}{\pgfqpoint{0.842469in}{2.281088in}}{\pgfqpoint{0.842469in}{2.272852in}}%
\pgfpathcurveto{\pgfqpoint{0.842469in}{2.264615in}}{\pgfqpoint{0.845742in}{2.256715in}}{\pgfqpoint{0.851566in}{2.250891in}}%
\pgfpathcurveto{\pgfqpoint{0.857390in}{2.245067in}}{\pgfqpoint{0.865290in}{2.241795in}}{\pgfqpoint{0.873526in}{2.241795in}}%
\pgfpathclose%
\pgfusepath{stroke,fill}%
\end{pgfscope}%
\begin{pgfscope}%
\pgfpathrectangle{\pgfqpoint{0.100000in}{0.220728in}}{\pgfqpoint{3.696000in}{3.696000in}}%
\pgfusepath{clip}%
\pgfsetbuttcap%
\pgfsetroundjoin%
\definecolor{currentfill}{rgb}{0.121569,0.466667,0.705882}%
\pgfsetfillcolor{currentfill}%
\pgfsetfillopacity{0.717087}%
\pgfsetlinewidth{1.003750pt}%
\definecolor{currentstroke}{rgb}{0.121569,0.466667,0.705882}%
\pgfsetstrokecolor{currentstroke}%
\pgfsetstrokeopacity{0.717087}%
\pgfsetdash{}{0pt}%
\pgfpathmoveto{\pgfqpoint{0.878185in}{2.237111in}}%
\pgfpathcurveto{\pgfqpoint{0.886421in}{2.237111in}}{\pgfqpoint{0.894321in}{2.240383in}}{\pgfqpoint{0.900145in}{2.246207in}}%
\pgfpathcurveto{\pgfqpoint{0.905969in}{2.252031in}}{\pgfqpoint{0.909241in}{2.259931in}}{\pgfqpoint{0.909241in}{2.268168in}}%
\pgfpathcurveto{\pgfqpoint{0.909241in}{2.276404in}}{\pgfqpoint{0.905969in}{2.284304in}}{\pgfqpoint{0.900145in}{2.290128in}}%
\pgfpathcurveto{\pgfqpoint{0.894321in}{2.295952in}}{\pgfqpoint{0.886421in}{2.299224in}}{\pgfqpoint{0.878185in}{2.299224in}}%
\pgfpathcurveto{\pgfqpoint{0.869949in}{2.299224in}}{\pgfqpoint{0.862049in}{2.295952in}}{\pgfqpoint{0.856225in}{2.290128in}}%
\pgfpathcurveto{\pgfqpoint{0.850401in}{2.284304in}}{\pgfqpoint{0.847128in}{2.276404in}}{\pgfqpoint{0.847128in}{2.268168in}}%
\pgfpathcurveto{\pgfqpoint{0.847128in}{2.259931in}}{\pgfqpoint{0.850401in}{2.252031in}}{\pgfqpoint{0.856225in}{2.246207in}}%
\pgfpathcurveto{\pgfqpoint{0.862049in}{2.240383in}}{\pgfqpoint{0.869949in}{2.237111in}}{\pgfqpoint{0.878185in}{2.237111in}}%
\pgfpathclose%
\pgfusepath{stroke,fill}%
\end{pgfscope}%
\begin{pgfscope}%
\pgfpathrectangle{\pgfqpoint{0.100000in}{0.220728in}}{\pgfqpoint{3.696000in}{3.696000in}}%
\pgfusepath{clip}%
\pgfsetbuttcap%
\pgfsetroundjoin%
\definecolor{currentfill}{rgb}{0.121569,0.466667,0.705882}%
\pgfsetfillcolor{currentfill}%
\pgfsetfillopacity{0.717495}%
\pgfsetlinewidth{1.003750pt}%
\definecolor{currentstroke}{rgb}{0.121569,0.466667,0.705882}%
\pgfsetstrokecolor{currentstroke}%
\pgfsetstrokeopacity{0.717495}%
\pgfsetdash{}{0pt}%
\pgfpathmoveto{\pgfqpoint{3.277599in}{2.789595in}}%
\pgfpathcurveto{\pgfqpoint{3.285835in}{2.789595in}}{\pgfqpoint{3.293735in}{2.792867in}}{\pgfqpoint{3.299559in}{2.798691in}}%
\pgfpathcurveto{\pgfqpoint{3.305383in}{2.804515in}}{\pgfqpoint{3.308656in}{2.812415in}}{\pgfqpoint{3.308656in}{2.820651in}}%
\pgfpathcurveto{\pgfqpoint{3.308656in}{2.828887in}}{\pgfqpoint{3.305383in}{2.836787in}}{\pgfqpoint{3.299559in}{2.842611in}}%
\pgfpathcurveto{\pgfqpoint{3.293735in}{2.848435in}}{\pgfqpoint{3.285835in}{2.851708in}}{\pgfqpoint{3.277599in}{2.851708in}}%
\pgfpathcurveto{\pgfqpoint{3.269363in}{2.851708in}}{\pgfqpoint{3.261463in}{2.848435in}}{\pgfqpoint{3.255639in}{2.842611in}}%
\pgfpathcurveto{\pgfqpoint{3.249815in}{2.836787in}}{\pgfqpoint{3.246543in}{2.828887in}}{\pgfqpoint{3.246543in}{2.820651in}}%
\pgfpathcurveto{\pgfqpoint{3.246543in}{2.812415in}}{\pgfqpoint{3.249815in}{2.804515in}}{\pgfqpoint{3.255639in}{2.798691in}}%
\pgfpathcurveto{\pgfqpoint{3.261463in}{2.792867in}}{\pgfqpoint{3.269363in}{2.789595in}}{\pgfqpoint{3.277599in}{2.789595in}}%
\pgfpathclose%
\pgfusepath{stroke,fill}%
\end{pgfscope}%
\begin{pgfscope}%
\pgfpathrectangle{\pgfqpoint{0.100000in}{0.220728in}}{\pgfqpoint{3.696000in}{3.696000in}}%
\pgfusepath{clip}%
\pgfsetbuttcap%
\pgfsetroundjoin%
\definecolor{currentfill}{rgb}{0.121569,0.466667,0.705882}%
\pgfsetfillcolor{currentfill}%
\pgfsetfillopacity{0.718470}%
\pgfsetlinewidth{1.003750pt}%
\definecolor{currentstroke}{rgb}{0.121569,0.466667,0.705882}%
\pgfsetstrokecolor{currentstroke}%
\pgfsetstrokeopacity{0.718470}%
\pgfsetdash{}{0pt}%
\pgfpathmoveto{\pgfqpoint{3.274036in}{2.783723in}}%
\pgfpathcurveto{\pgfqpoint{3.282273in}{2.783723in}}{\pgfqpoint{3.290173in}{2.786995in}}{\pgfqpoint{3.295997in}{2.792819in}}%
\pgfpathcurveto{\pgfqpoint{3.301820in}{2.798643in}}{\pgfqpoint{3.305093in}{2.806543in}}{\pgfqpoint{3.305093in}{2.814779in}}%
\pgfpathcurveto{\pgfqpoint{3.305093in}{2.823016in}}{\pgfqpoint{3.301820in}{2.830916in}}{\pgfqpoint{3.295997in}{2.836740in}}%
\pgfpathcurveto{\pgfqpoint{3.290173in}{2.842564in}}{\pgfqpoint{3.282273in}{2.845836in}}{\pgfqpoint{3.274036in}{2.845836in}}%
\pgfpathcurveto{\pgfqpoint{3.265800in}{2.845836in}}{\pgfqpoint{3.257900in}{2.842564in}}{\pgfqpoint{3.252076in}{2.836740in}}%
\pgfpathcurveto{\pgfqpoint{3.246252in}{2.830916in}}{\pgfqpoint{3.242980in}{2.823016in}}{\pgfqpoint{3.242980in}{2.814779in}}%
\pgfpathcurveto{\pgfqpoint{3.242980in}{2.806543in}}{\pgfqpoint{3.246252in}{2.798643in}}{\pgfqpoint{3.252076in}{2.792819in}}%
\pgfpathcurveto{\pgfqpoint{3.257900in}{2.786995in}}{\pgfqpoint{3.265800in}{2.783723in}}{\pgfqpoint{3.274036in}{2.783723in}}%
\pgfpathclose%
\pgfusepath{stroke,fill}%
\end{pgfscope}%
\begin{pgfscope}%
\pgfpathrectangle{\pgfqpoint{0.100000in}{0.220728in}}{\pgfqpoint{3.696000in}{3.696000in}}%
\pgfusepath{clip}%
\pgfsetbuttcap%
\pgfsetroundjoin%
\definecolor{currentfill}{rgb}{0.121569,0.466667,0.705882}%
\pgfsetfillcolor{currentfill}%
\pgfsetfillopacity{0.719060}%
\pgfsetlinewidth{1.003750pt}%
\definecolor{currentstroke}{rgb}{0.121569,0.466667,0.705882}%
\pgfsetstrokecolor{currentstroke}%
\pgfsetstrokeopacity{0.719060}%
\pgfsetdash{}{0pt}%
\pgfpathmoveto{\pgfqpoint{0.887763in}{2.230858in}}%
\pgfpathcurveto{\pgfqpoint{0.896000in}{2.230858in}}{\pgfqpoint{0.903900in}{2.234130in}}{\pgfqpoint{0.909724in}{2.239954in}}%
\pgfpathcurveto{\pgfqpoint{0.915548in}{2.245778in}}{\pgfqpoint{0.918820in}{2.253678in}}{\pgfqpoint{0.918820in}{2.261915in}}%
\pgfpathcurveto{\pgfqpoint{0.918820in}{2.270151in}}{\pgfqpoint{0.915548in}{2.278051in}}{\pgfqpoint{0.909724in}{2.283875in}}%
\pgfpathcurveto{\pgfqpoint{0.903900in}{2.289699in}}{\pgfqpoint{0.896000in}{2.292971in}}{\pgfqpoint{0.887763in}{2.292971in}}%
\pgfpathcurveto{\pgfqpoint{0.879527in}{2.292971in}}{\pgfqpoint{0.871627in}{2.289699in}}{\pgfqpoint{0.865803in}{2.283875in}}%
\pgfpathcurveto{\pgfqpoint{0.859979in}{2.278051in}}{\pgfqpoint{0.856707in}{2.270151in}}{\pgfqpoint{0.856707in}{2.261915in}}%
\pgfpathcurveto{\pgfqpoint{0.856707in}{2.253678in}}{\pgfqpoint{0.859979in}{2.245778in}}{\pgfqpoint{0.865803in}{2.239954in}}%
\pgfpathcurveto{\pgfqpoint{0.871627in}{2.234130in}}{\pgfqpoint{0.879527in}{2.230858in}}{\pgfqpoint{0.887763in}{2.230858in}}%
\pgfpathclose%
\pgfusepath{stroke,fill}%
\end{pgfscope}%
\begin{pgfscope}%
\pgfpathrectangle{\pgfqpoint{0.100000in}{0.220728in}}{\pgfqpoint{3.696000in}{3.696000in}}%
\pgfusepath{clip}%
\pgfsetbuttcap%
\pgfsetroundjoin%
\definecolor{currentfill}{rgb}{0.121569,0.466667,0.705882}%
\pgfsetfillcolor{currentfill}%
\pgfsetfillopacity{0.719870}%
\pgfsetlinewidth{1.003750pt}%
\definecolor{currentstroke}{rgb}{0.121569,0.466667,0.705882}%
\pgfsetstrokecolor{currentstroke}%
\pgfsetstrokeopacity{0.719870}%
\pgfsetdash{}{0pt}%
\pgfpathmoveto{\pgfqpoint{3.268763in}{2.775934in}}%
\pgfpathcurveto{\pgfqpoint{3.277000in}{2.775934in}}{\pgfqpoint{3.284900in}{2.779207in}}{\pgfqpoint{3.290724in}{2.785030in}}%
\pgfpathcurveto{\pgfqpoint{3.296548in}{2.790854in}}{\pgfqpoint{3.299820in}{2.798754in}}{\pgfqpoint{3.299820in}{2.806991in}}%
\pgfpathcurveto{\pgfqpoint{3.299820in}{2.815227in}}{\pgfqpoint{3.296548in}{2.823127in}}{\pgfqpoint{3.290724in}{2.828951in}}%
\pgfpathcurveto{\pgfqpoint{3.284900in}{2.834775in}}{\pgfqpoint{3.277000in}{2.838047in}}{\pgfqpoint{3.268763in}{2.838047in}}%
\pgfpathcurveto{\pgfqpoint{3.260527in}{2.838047in}}{\pgfqpoint{3.252627in}{2.834775in}}{\pgfqpoint{3.246803in}{2.828951in}}%
\pgfpathcurveto{\pgfqpoint{3.240979in}{2.823127in}}{\pgfqpoint{3.237707in}{2.815227in}}{\pgfqpoint{3.237707in}{2.806991in}}%
\pgfpathcurveto{\pgfqpoint{3.237707in}{2.798754in}}{\pgfqpoint{3.240979in}{2.790854in}}{\pgfqpoint{3.246803in}{2.785030in}}%
\pgfpathcurveto{\pgfqpoint{3.252627in}{2.779207in}}{\pgfqpoint{3.260527in}{2.775934in}}{\pgfqpoint{3.268763in}{2.775934in}}%
\pgfpathclose%
\pgfusepath{stroke,fill}%
\end{pgfscope}%
\begin{pgfscope}%
\pgfpathrectangle{\pgfqpoint{0.100000in}{0.220728in}}{\pgfqpoint{3.696000in}{3.696000in}}%
\pgfusepath{clip}%
\pgfsetbuttcap%
\pgfsetroundjoin%
\definecolor{currentfill}{rgb}{0.121569,0.466667,0.705882}%
\pgfsetfillcolor{currentfill}%
\pgfsetfillopacity{0.720727}%
\pgfsetlinewidth{1.003750pt}%
\definecolor{currentstroke}{rgb}{0.121569,0.466667,0.705882}%
\pgfsetstrokecolor{currentstroke}%
\pgfsetstrokeopacity{0.720727}%
\pgfsetdash{}{0pt}%
\pgfpathmoveto{\pgfqpoint{3.266119in}{2.771724in}}%
\pgfpathcurveto{\pgfqpoint{3.274355in}{2.771724in}}{\pgfqpoint{3.282255in}{2.774996in}}{\pgfqpoint{3.288079in}{2.780820in}}%
\pgfpathcurveto{\pgfqpoint{3.293903in}{2.786644in}}{\pgfqpoint{3.297175in}{2.794544in}}{\pgfqpoint{3.297175in}{2.802780in}}%
\pgfpathcurveto{\pgfqpoint{3.297175in}{2.811017in}}{\pgfqpoint{3.293903in}{2.818917in}}{\pgfqpoint{3.288079in}{2.824741in}}%
\pgfpathcurveto{\pgfqpoint{3.282255in}{2.830565in}}{\pgfqpoint{3.274355in}{2.833837in}}{\pgfqpoint{3.266119in}{2.833837in}}%
\pgfpathcurveto{\pgfqpoint{3.257882in}{2.833837in}}{\pgfqpoint{3.249982in}{2.830565in}}{\pgfqpoint{3.244158in}{2.824741in}}%
\pgfpathcurveto{\pgfqpoint{3.238334in}{2.818917in}}{\pgfqpoint{3.235062in}{2.811017in}}{\pgfqpoint{3.235062in}{2.802780in}}%
\pgfpathcurveto{\pgfqpoint{3.235062in}{2.794544in}}{\pgfqpoint{3.238334in}{2.786644in}}{\pgfqpoint{3.244158in}{2.780820in}}%
\pgfpathcurveto{\pgfqpoint{3.249982in}{2.774996in}}{\pgfqpoint{3.257882in}{2.771724in}}{\pgfqpoint{3.266119in}{2.771724in}}%
\pgfpathclose%
\pgfusepath{stroke,fill}%
\end{pgfscope}%
\begin{pgfscope}%
\pgfpathrectangle{\pgfqpoint{0.100000in}{0.220728in}}{\pgfqpoint{3.696000in}{3.696000in}}%
\pgfusepath{clip}%
\pgfsetbuttcap%
\pgfsetroundjoin%
\definecolor{currentfill}{rgb}{0.121569,0.466667,0.705882}%
\pgfsetfillcolor{currentfill}%
\pgfsetfillopacity{0.721973}%
\pgfsetlinewidth{1.003750pt}%
\definecolor{currentstroke}{rgb}{0.121569,0.466667,0.705882}%
\pgfsetstrokecolor{currentstroke}%
\pgfsetstrokeopacity{0.721973}%
\pgfsetdash{}{0pt}%
\pgfpathmoveto{\pgfqpoint{3.260894in}{2.764115in}}%
\pgfpathcurveto{\pgfqpoint{3.269131in}{2.764115in}}{\pgfqpoint{3.277031in}{2.767388in}}{\pgfqpoint{3.282855in}{2.773212in}}%
\pgfpathcurveto{\pgfqpoint{3.288679in}{2.779036in}}{\pgfqpoint{3.291951in}{2.786936in}}{\pgfqpoint{3.291951in}{2.795172in}}%
\pgfpathcurveto{\pgfqpoint{3.291951in}{2.803408in}}{\pgfqpoint{3.288679in}{2.811308in}}{\pgfqpoint{3.282855in}{2.817132in}}%
\pgfpathcurveto{\pgfqpoint{3.277031in}{2.822956in}}{\pgfqpoint{3.269131in}{2.826228in}}{\pgfqpoint{3.260894in}{2.826228in}}%
\pgfpathcurveto{\pgfqpoint{3.252658in}{2.826228in}}{\pgfqpoint{3.244758in}{2.822956in}}{\pgfqpoint{3.238934in}{2.817132in}}%
\pgfpathcurveto{\pgfqpoint{3.233110in}{2.811308in}}{\pgfqpoint{3.229838in}{2.803408in}}{\pgfqpoint{3.229838in}{2.795172in}}%
\pgfpathcurveto{\pgfqpoint{3.229838in}{2.786936in}}{\pgfqpoint{3.233110in}{2.779036in}}{\pgfqpoint{3.238934in}{2.773212in}}%
\pgfpathcurveto{\pgfqpoint{3.244758in}{2.767388in}}{\pgfqpoint{3.252658in}{2.764115in}}{\pgfqpoint{3.260894in}{2.764115in}}%
\pgfpathclose%
\pgfusepath{stroke,fill}%
\end{pgfscope}%
\begin{pgfscope}%
\pgfpathrectangle{\pgfqpoint{0.100000in}{0.220728in}}{\pgfqpoint{3.696000in}{3.696000in}}%
\pgfusepath{clip}%
\pgfsetbuttcap%
\pgfsetroundjoin%
\definecolor{currentfill}{rgb}{0.121569,0.466667,0.705882}%
\pgfsetfillcolor{currentfill}%
\pgfsetfillopacity{0.722655}%
\pgfsetlinewidth{1.003750pt}%
\definecolor{currentstroke}{rgb}{0.121569,0.466667,0.705882}%
\pgfsetstrokecolor{currentstroke}%
\pgfsetstrokeopacity{0.722655}%
\pgfsetdash{}{0pt}%
\pgfpathmoveto{\pgfqpoint{0.905104in}{2.219218in}}%
\pgfpathcurveto{\pgfqpoint{0.913340in}{2.219218in}}{\pgfqpoint{0.921240in}{2.222490in}}{\pgfqpoint{0.927064in}{2.228314in}}%
\pgfpathcurveto{\pgfqpoint{0.932888in}{2.234138in}}{\pgfqpoint{0.936160in}{2.242038in}}{\pgfqpoint{0.936160in}{2.250274in}}%
\pgfpathcurveto{\pgfqpoint{0.936160in}{2.258510in}}{\pgfqpoint{0.932888in}{2.266410in}}{\pgfqpoint{0.927064in}{2.272234in}}%
\pgfpathcurveto{\pgfqpoint{0.921240in}{2.278058in}}{\pgfqpoint{0.913340in}{2.281331in}}{\pgfqpoint{0.905104in}{2.281331in}}%
\pgfpathcurveto{\pgfqpoint{0.896868in}{2.281331in}}{\pgfqpoint{0.888968in}{2.278058in}}{\pgfqpoint{0.883144in}{2.272234in}}%
\pgfpathcurveto{\pgfqpoint{0.877320in}{2.266410in}}{\pgfqpoint{0.874047in}{2.258510in}}{\pgfqpoint{0.874047in}{2.250274in}}%
\pgfpathcurveto{\pgfqpoint{0.874047in}{2.242038in}}{\pgfqpoint{0.877320in}{2.234138in}}{\pgfqpoint{0.883144in}{2.228314in}}%
\pgfpathcurveto{\pgfqpoint{0.888968in}{2.222490in}}{\pgfqpoint{0.896868in}{2.219218in}}{\pgfqpoint{0.905104in}{2.219218in}}%
\pgfpathclose%
\pgfusepath{stroke,fill}%
\end{pgfscope}%
\begin{pgfscope}%
\pgfpathrectangle{\pgfqpoint{0.100000in}{0.220728in}}{\pgfqpoint{3.696000in}{3.696000in}}%
\pgfusepath{clip}%
\pgfsetbuttcap%
\pgfsetroundjoin%
\definecolor{currentfill}{rgb}{0.121569,0.466667,0.705882}%
\pgfsetfillcolor{currentfill}%
\pgfsetfillopacity{0.724114}%
\pgfsetlinewidth{1.003750pt}%
\definecolor{currentstroke}{rgb}{0.121569,0.466667,0.705882}%
\pgfsetstrokecolor{currentstroke}%
\pgfsetstrokeopacity{0.724114}%
\pgfsetdash{}{0pt}%
\pgfpathmoveto{\pgfqpoint{3.254299in}{2.753137in}}%
\pgfpathcurveto{\pgfqpoint{3.262535in}{2.753137in}}{\pgfqpoint{3.270435in}{2.756409in}}{\pgfqpoint{3.276259in}{2.762233in}}%
\pgfpathcurveto{\pgfqpoint{3.282083in}{2.768057in}}{\pgfqpoint{3.285355in}{2.775957in}}{\pgfqpoint{3.285355in}{2.784193in}}%
\pgfpathcurveto{\pgfqpoint{3.285355in}{2.792430in}}{\pgfqpoint{3.282083in}{2.800330in}}{\pgfqpoint{3.276259in}{2.806154in}}%
\pgfpathcurveto{\pgfqpoint{3.270435in}{2.811977in}}{\pgfqpoint{3.262535in}{2.815250in}}{\pgfqpoint{3.254299in}{2.815250in}}%
\pgfpathcurveto{\pgfqpoint{3.246063in}{2.815250in}}{\pgfqpoint{3.238163in}{2.811977in}}{\pgfqpoint{3.232339in}{2.806154in}}%
\pgfpathcurveto{\pgfqpoint{3.226515in}{2.800330in}}{\pgfqpoint{3.223242in}{2.792430in}}{\pgfqpoint{3.223242in}{2.784193in}}%
\pgfpathcurveto{\pgfqpoint{3.223242in}{2.775957in}}{\pgfqpoint{3.226515in}{2.768057in}}{\pgfqpoint{3.232339in}{2.762233in}}%
\pgfpathcurveto{\pgfqpoint{3.238163in}{2.756409in}}{\pgfqpoint{3.246063in}{2.753137in}}{\pgfqpoint{3.254299in}{2.753137in}}%
\pgfpathclose%
\pgfusepath{stroke,fill}%
\end{pgfscope}%
\begin{pgfscope}%
\pgfpathrectangle{\pgfqpoint{0.100000in}{0.220728in}}{\pgfqpoint{3.696000in}{3.696000in}}%
\pgfusepath{clip}%
\pgfsetbuttcap%
\pgfsetroundjoin%
\definecolor{currentfill}{rgb}{0.121569,0.466667,0.705882}%
\pgfsetfillcolor{currentfill}%
\pgfsetfillopacity{0.725134}%
\pgfsetlinewidth{1.003750pt}%
\definecolor{currentstroke}{rgb}{0.121569,0.466667,0.705882}%
\pgfsetstrokecolor{currentstroke}%
\pgfsetstrokeopacity{0.725134}%
\pgfsetdash{}{0pt}%
\pgfpathmoveto{\pgfqpoint{3.249939in}{2.747384in}}%
\pgfpathcurveto{\pgfqpoint{3.258175in}{2.747384in}}{\pgfqpoint{3.266075in}{2.750656in}}{\pgfqpoint{3.271899in}{2.756480in}}%
\pgfpathcurveto{\pgfqpoint{3.277723in}{2.762304in}}{\pgfqpoint{3.280995in}{2.770204in}}{\pgfqpoint{3.280995in}{2.778440in}}%
\pgfpathcurveto{\pgfqpoint{3.280995in}{2.786677in}}{\pgfqpoint{3.277723in}{2.794577in}}{\pgfqpoint{3.271899in}{2.800401in}}%
\pgfpathcurveto{\pgfqpoint{3.266075in}{2.806225in}}{\pgfqpoint{3.258175in}{2.809497in}}{\pgfqpoint{3.249939in}{2.809497in}}%
\pgfpathcurveto{\pgfqpoint{3.241702in}{2.809497in}}{\pgfqpoint{3.233802in}{2.806225in}}{\pgfqpoint{3.227978in}{2.800401in}}%
\pgfpathcurveto{\pgfqpoint{3.222154in}{2.794577in}}{\pgfqpoint{3.218882in}{2.786677in}}{\pgfqpoint{3.218882in}{2.778440in}}%
\pgfpathcurveto{\pgfqpoint{3.218882in}{2.770204in}}{\pgfqpoint{3.222154in}{2.762304in}}{\pgfqpoint{3.227978in}{2.756480in}}%
\pgfpathcurveto{\pgfqpoint{3.233802in}{2.750656in}}{\pgfqpoint{3.241702in}{2.747384in}}{\pgfqpoint{3.249939in}{2.747384in}}%
\pgfpathclose%
\pgfusepath{stroke,fill}%
\end{pgfscope}%
\begin{pgfscope}%
\pgfpathrectangle{\pgfqpoint{0.100000in}{0.220728in}}{\pgfqpoint{3.696000in}{3.696000in}}%
\pgfusepath{clip}%
\pgfsetbuttcap%
\pgfsetroundjoin%
\definecolor{currentfill}{rgb}{0.121569,0.466667,0.705882}%
\pgfsetfillcolor{currentfill}%
\pgfsetfillopacity{0.725702}%
\pgfsetlinewidth{1.003750pt}%
\definecolor{currentstroke}{rgb}{0.121569,0.466667,0.705882}%
\pgfsetstrokecolor{currentstroke}%
\pgfsetstrokeopacity{0.725702}%
\pgfsetdash{}{0pt}%
\pgfpathmoveto{\pgfqpoint{3.247626in}{2.744111in}}%
\pgfpathcurveto{\pgfqpoint{3.255862in}{2.744111in}}{\pgfqpoint{3.263762in}{2.747383in}}{\pgfqpoint{3.269586in}{2.753207in}}%
\pgfpathcurveto{\pgfqpoint{3.275410in}{2.759031in}}{\pgfqpoint{3.278682in}{2.766931in}}{\pgfqpoint{3.278682in}{2.775168in}}%
\pgfpathcurveto{\pgfqpoint{3.278682in}{2.783404in}}{\pgfqpoint{3.275410in}{2.791304in}}{\pgfqpoint{3.269586in}{2.797128in}}%
\pgfpathcurveto{\pgfqpoint{3.263762in}{2.802952in}}{\pgfqpoint{3.255862in}{2.806224in}}{\pgfqpoint{3.247626in}{2.806224in}}%
\pgfpathcurveto{\pgfqpoint{3.239390in}{2.806224in}}{\pgfqpoint{3.231490in}{2.802952in}}{\pgfqpoint{3.225666in}{2.797128in}}%
\pgfpathcurveto{\pgfqpoint{3.219842in}{2.791304in}}{\pgfqpoint{3.216569in}{2.783404in}}{\pgfqpoint{3.216569in}{2.775168in}}%
\pgfpathcurveto{\pgfqpoint{3.216569in}{2.766931in}}{\pgfqpoint{3.219842in}{2.759031in}}{\pgfqpoint{3.225666in}{2.753207in}}%
\pgfpathcurveto{\pgfqpoint{3.231490in}{2.747383in}}{\pgfqpoint{3.239390in}{2.744111in}}{\pgfqpoint{3.247626in}{2.744111in}}%
\pgfpathclose%
\pgfusepath{stroke,fill}%
\end{pgfscope}%
\begin{pgfscope}%
\pgfpathrectangle{\pgfqpoint{0.100000in}{0.220728in}}{\pgfqpoint{3.696000in}{3.696000in}}%
\pgfusepath{clip}%
\pgfsetbuttcap%
\pgfsetroundjoin%
\definecolor{currentfill}{rgb}{0.121569,0.466667,0.705882}%
\pgfsetfillcolor{currentfill}%
\pgfsetfillopacity{0.725921}%
\pgfsetlinewidth{1.003750pt}%
\definecolor{currentstroke}{rgb}{0.121569,0.466667,0.705882}%
\pgfsetstrokecolor{currentstroke}%
\pgfsetstrokeopacity{0.725921}%
\pgfsetdash{}{0pt}%
\pgfpathmoveto{\pgfqpoint{0.920563in}{2.207875in}}%
\pgfpathcurveto{\pgfqpoint{0.928800in}{2.207875in}}{\pgfqpoint{0.936700in}{2.211147in}}{\pgfqpoint{0.942524in}{2.216971in}}%
\pgfpathcurveto{\pgfqpoint{0.948348in}{2.222795in}}{\pgfqpoint{0.951620in}{2.230695in}}{\pgfqpoint{0.951620in}{2.238931in}}%
\pgfpathcurveto{\pgfqpoint{0.951620in}{2.247168in}}{\pgfqpoint{0.948348in}{2.255068in}}{\pgfqpoint{0.942524in}{2.260892in}}%
\pgfpathcurveto{\pgfqpoint{0.936700in}{2.266716in}}{\pgfqpoint{0.928800in}{2.269988in}}{\pgfqpoint{0.920563in}{2.269988in}}%
\pgfpathcurveto{\pgfqpoint{0.912327in}{2.269988in}}{\pgfqpoint{0.904427in}{2.266716in}}{\pgfqpoint{0.898603in}{2.260892in}}%
\pgfpathcurveto{\pgfqpoint{0.892779in}{2.255068in}}{\pgfqpoint{0.889507in}{2.247168in}}{\pgfqpoint{0.889507in}{2.238931in}}%
\pgfpathcurveto{\pgfqpoint{0.889507in}{2.230695in}}{\pgfqpoint{0.892779in}{2.222795in}}{\pgfqpoint{0.898603in}{2.216971in}}%
\pgfpathcurveto{\pgfqpoint{0.904427in}{2.211147in}}{\pgfqpoint{0.912327in}{2.207875in}}{\pgfqpoint{0.920563in}{2.207875in}}%
\pgfpathclose%
\pgfusepath{stroke,fill}%
\end{pgfscope}%
\begin{pgfscope}%
\pgfpathrectangle{\pgfqpoint{0.100000in}{0.220728in}}{\pgfqpoint{3.696000in}{3.696000in}}%
\pgfusepath{clip}%
\pgfsetbuttcap%
\pgfsetroundjoin%
\definecolor{currentfill}{rgb}{0.121569,0.466667,0.705882}%
\pgfsetfillcolor{currentfill}%
\pgfsetfillopacity{0.725996}%
\pgfsetlinewidth{1.003750pt}%
\definecolor{currentstroke}{rgb}{0.121569,0.466667,0.705882}%
\pgfsetstrokecolor{currentstroke}%
\pgfsetstrokeopacity{0.725996}%
\pgfsetdash{}{0pt}%
\pgfpathmoveto{\pgfqpoint{3.246356in}{2.742211in}}%
\pgfpathcurveto{\pgfqpoint{3.254592in}{2.742211in}}{\pgfqpoint{3.262492in}{2.745483in}}{\pgfqpoint{3.268316in}{2.751307in}}%
\pgfpathcurveto{\pgfqpoint{3.274140in}{2.757131in}}{\pgfqpoint{3.277412in}{2.765031in}}{\pgfqpoint{3.277412in}{2.773267in}}%
\pgfpathcurveto{\pgfqpoint{3.277412in}{2.781503in}}{\pgfqpoint{3.274140in}{2.789403in}}{\pgfqpoint{3.268316in}{2.795227in}}%
\pgfpathcurveto{\pgfqpoint{3.262492in}{2.801051in}}{\pgfqpoint{3.254592in}{2.804324in}}{\pgfqpoint{3.246356in}{2.804324in}}%
\pgfpathcurveto{\pgfqpoint{3.238119in}{2.804324in}}{\pgfqpoint{3.230219in}{2.801051in}}{\pgfqpoint{3.224395in}{2.795227in}}%
\pgfpathcurveto{\pgfqpoint{3.218571in}{2.789403in}}{\pgfqpoint{3.215299in}{2.781503in}}{\pgfqpoint{3.215299in}{2.773267in}}%
\pgfpathcurveto{\pgfqpoint{3.215299in}{2.765031in}}{\pgfqpoint{3.218571in}{2.757131in}}{\pgfqpoint{3.224395in}{2.751307in}}%
\pgfpathcurveto{\pgfqpoint{3.230219in}{2.745483in}}{\pgfqpoint{3.238119in}{2.742211in}}{\pgfqpoint{3.246356in}{2.742211in}}%
\pgfpathclose%
\pgfusepath{stroke,fill}%
\end{pgfscope}%
\begin{pgfscope}%
\pgfpathrectangle{\pgfqpoint{0.100000in}{0.220728in}}{\pgfqpoint{3.696000in}{3.696000in}}%
\pgfusepath{clip}%
\pgfsetbuttcap%
\pgfsetroundjoin%
\definecolor{currentfill}{rgb}{0.121569,0.466667,0.705882}%
\pgfsetfillcolor{currentfill}%
\pgfsetfillopacity{0.726184}%
\pgfsetlinewidth{1.003750pt}%
\definecolor{currentstroke}{rgb}{0.121569,0.466667,0.705882}%
\pgfsetstrokecolor{currentstroke}%
\pgfsetstrokeopacity{0.726184}%
\pgfsetdash{}{0pt}%
\pgfpathmoveto{\pgfqpoint{3.245649in}{2.741314in}}%
\pgfpathcurveto{\pgfqpoint{3.253886in}{2.741314in}}{\pgfqpoint{3.261786in}{2.744587in}}{\pgfqpoint{3.267610in}{2.750411in}}%
\pgfpathcurveto{\pgfqpoint{3.273433in}{2.756235in}}{\pgfqpoint{3.276706in}{2.764135in}}{\pgfqpoint{3.276706in}{2.772371in}}%
\pgfpathcurveto{\pgfqpoint{3.276706in}{2.780607in}}{\pgfqpoint{3.273433in}{2.788507in}}{\pgfqpoint{3.267610in}{2.794331in}}%
\pgfpathcurveto{\pgfqpoint{3.261786in}{2.800155in}}{\pgfqpoint{3.253886in}{2.803427in}}{\pgfqpoint{3.245649in}{2.803427in}}%
\pgfpathcurveto{\pgfqpoint{3.237413in}{2.803427in}}{\pgfqpoint{3.229513in}{2.800155in}}{\pgfqpoint{3.223689in}{2.794331in}}%
\pgfpathcurveto{\pgfqpoint{3.217865in}{2.788507in}}{\pgfqpoint{3.214593in}{2.780607in}}{\pgfqpoint{3.214593in}{2.772371in}}%
\pgfpathcurveto{\pgfqpoint{3.214593in}{2.764135in}}{\pgfqpoint{3.217865in}{2.756235in}}{\pgfqpoint{3.223689in}{2.750411in}}%
\pgfpathcurveto{\pgfqpoint{3.229513in}{2.744587in}}{\pgfqpoint{3.237413in}{2.741314in}}{\pgfqpoint{3.245649in}{2.741314in}}%
\pgfpathclose%
\pgfusepath{stroke,fill}%
\end{pgfscope}%
\begin{pgfscope}%
\pgfpathrectangle{\pgfqpoint{0.100000in}{0.220728in}}{\pgfqpoint{3.696000in}{3.696000in}}%
\pgfusepath{clip}%
\pgfsetbuttcap%
\pgfsetroundjoin%
\definecolor{currentfill}{rgb}{0.121569,0.466667,0.705882}%
\pgfsetfillcolor{currentfill}%
\pgfsetfillopacity{0.726276}%
\pgfsetlinewidth{1.003750pt}%
\definecolor{currentstroke}{rgb}{0.121569,0.466667,0.705882}%
\pgfsetstrokecolor{currentstroke}%
\pgfsetstrokeopacity{0.726276}%
\pgfsetdash{}{0pt}%
\pgfpathmoveto{\pgfqpoint{3.245269in}{2.740747in}}%
\pgfpathcurveto{\pgfqpoint{3.253505in}{2.740747in}}{\pgfqpoint{3.261405in}{2.744019in}}{\pgfqpoint{3.267229in}{2.749843in}}%
\pgfpathcurveto{\pgfqpoint{3.273053in}{2.755667in}}{\pgfqpoint{3.276326in}{2.763567in}}{\pgfqpoint{3.276326in}{2.771803in}}%
\pgfpathcurveto{\pgfqpoint{3.276326in}{2.780040in}}{\pgfqpoint{3.273053in}{2.787940in}}{\pgfqpoint{3.267229in}{2.793764in}}%
\pgfpathcurveto{\pgfqpoint{3.261405in}{2.799588in}}{\pgfqpoint{3.253505in}{2.802860in}}{\pgfqpoint{3.245269in}{2.802860in}}%
\pgfpathcurveto{\pgfqpoint{3.237033in}{2.802860in}}{\pgfqpoint{3.229133in}{2.799588in}}{\pgfqpoint{3.223309in}{2.793764in}}%
\pgfpathcurveto{\pgfqpoint{3.217485in}{2.787940in}}{\pgfqpoint{3.214213in}{2.780040in}}{\pgfqpoint{3.214213in}{2.771803in}}%
\pgfpathcurveto{\pgfqpoint{3.214213in}{2.763567in}}{\pgfqpoint{3.217485in}{2.755667in}}{\pgfqpoint{3.223309in}{2.749843in}}%
\pgfpathcurveto{\pgfqpoint{3.229133in}{2.744019in}}{\pgfqpoint{3.237033in}{2.740747in}}{\pgfqpoint{3.245269in}{2.740747in}}%
\pgfpathclose%
\pgfusepath{stroke,fill}%
\end{pgfscope}%
\begin{pgfscope}%
\pgfpathrectangle{\pgfqpoint{0.100000in}{0.220728in}}{\pgfqpoint{3.696000in}{3.696000in}}%
\pgfusepath{clip}%
\pgfsetbuttcap%
\pgfsetroundjoin%
\definecolor{currentfill}{rgb}{0.121569,0.466667,0.705882}%
\pgfsetfillcolor{currentfill}%
\pgfsetfillopacity{0.726323}%
\pgfsetlinewidth{1.003750pt}%
\definecolor{currentstroke}{rgb}{0.121569,0.466667,0.705882}%
\pgfsetstrokecolor{currentstroke}%
\pgfsetstrokeopacity{0.726323}%
\pgfsetdash{}{0pt}%
\pgfpathmoveto{\pgfqpoint{3.245031in}{2.740471in}}%
\pgfpathcurveto{\pgfqpoint{3.253268in}{2.740471in}}{\pgfqpoint{3.261168in}{2.743743in}}{\pgfqpoint{3.266992in}{2.749567in}}%
\pgfpathcurveto{\pgfqpoint{3.272816in}{2.755391in}}{\pgfqpoint{3.276088in}{2.763291in}}{\pgfqpoint{3.276088in}{2.771527in}}%
\pgfpathcurveto{\pgfqpoint{3.276088in}{2.779764in}}{\pgfqpoint{3.272816in}{2.787664in}}{\pgfqpoint{3.266992in}{2.793488in}}%
\pgfpathcurveto{\pgfqpoint{3.261168in}{2.799312in}}{\pgfqpoint{3.253268in}{2.802584in}}{\pgfqpoint{3.245031in}{2.802584in}}%
\pgfpathcurveto{\pgfqpoint{3.236795in}{2.802584in}}{\pgfqpoint{3.228895in}{2.799312in}}{\pgfqpoint{3.223071in}{2.793488in}}%
\pgfpathcurveto{\pgfqpoint{3.217247in}{2.787664in}}{\pgfqpoint{3.213975in}{2.779764in}}{\pgfqpoint{3.213975in}{2.771527in}}%
\pgfpathcurveto{\pgfqpoint{3.213975in}{2.763291in}}{\pgfqpoint{3.217247in}{2.755391in}}{\pgfqpoint{3.223071in}{2.749567in}}%
\pgfpathcurveto{\pgfqpoint{3.228895in}{2.743743in}}{\pgfqpoint{3.236795in}{2.740471in}}{\pgfqpoint{3.245031in}{2.740471in}}%
\pgfpathclose%
\pgfusepath{stroke,fill}%
\end{pgfscope}%
\begin{pgfscope}%
\pgfpathrectangle{\pgfqpoint{0.100000in}{0.220728in}}{\pgfqpoint{3.696000in}{3.696000in}}%
\pgfusepath{clip}%
\pgfsetbuttcap%
\pgfsetroundjoin%
\definecolor{currentfill}{rgb}{0.121569,0.466667,0.705882}%
\pgfsetfillcolor{currentfill}%
\pgfsetfillopacity{0.726747}%
\pgfsetlinewidth{1.003750pt}%
\definecolor{currentstroke}{rgb}{0.121569,0.466667,0.705882}%
\pgfsetstrokecolor{currentstroke}%
\pgfsetstrokeopacity{0.726747}%
\pgfsetdash{}{0pt}%
\pgfpathmoveto{\pgfqpoint{3.243620in}{2.738161in}}%
\pgfpathcurveto{\pgfqpoint{3.251856in}{2.738161in}}{\pgfqpoint{3.259756in}{2.741433in}}{\pgfqpoint{3.265580in}{2.747257in}}%
\pgfpathcurveto{\pgfqpoint{3.271404in}{2.753081in}}{\pgfqpoint{3.274676in}{2.760981in}}{\pgfqpoint{3.274676in}{2.769217in}}%
\pgfpathcurveto{\pgfqpoint{3.274676in}{2.777454in}}{\pgfqpoint{3.271404in}{2.785354in}}{\pgfqpoint{3.265580in}{2.791178in}}%
\pgfpathcurveto{\pgfqpoint{3.259756in}{2.797001in}}{\pgfqpoint{3.251856in}{2.800274in}}{\pgfqpoint{3.243620in}{2.800274in}}%
\pgfpathcurveto{\pgfqpoint{3.235384in}{2.800274in}}{\pgfqpoint{3.227483in}{2.797001in}}{\pgfqpoint{3.221660in}{2.791178in}}%
\pgfpathcurveto{\pgfqpoint{3.215836in}{2.785354in}}{\pgfqpoint{3.212563in}{2.777454in}}{\pgfqpoint{3.212563in}{2.769217in}}%
\pgfpathcurveto{\pgfqpoint{3.212563in}{2.760981in}}{\pgfqpoint{3.215836in}{2.753081in}}{\pgfqpoint{3.221660in}{2.747257in}}%
\pgfpathcurveto{\pgfqpoint{3.227483in}{2.741433in}}{\pgfqpoint{3.235384in}{2.738161in}}{\pgfqpoint{3.243620in}{2.738161in}}%
\pgfpathclose%
\pgfusepath{stroke,fill}%
\end{pgfscope}%
\begin{pgfscope}%
\pgfpathrectangle{\pgfqpoint{0.100000in}{0.220728in}}{\pgfqpoint{3.696000in}{3.696000in}}%
\pgfusepath{clip}%
\pgfsetbuttcap%
\pgfsetroundjoin%
\definecolor{currentfill}{rgb}{0.121569,0.466667,0.705882}%
\pgfsetfillcolor{currentfill}%
\pgfsetfillopacity{0.727914}%
\pgfsetlinewidth{1.003750pt}%
\definecolor{currentstroke}{rgb}{0.121569,0.466667,0.705882}%
\pgfsetstrokecolor{currentstroke}%
\pgfsetstrokeopacity{0.727914}%
\pgfsetdash{}{0pt}%
\pgfpathmoveto{\pgfqpoint{3.238242in}{2.732022in}}%
\pgfpathcurveto{\pgfqpoint{3.246478in}{2.732022in}}{\pgfqpoint{3.254378in}{2.735295in}}{\pgfqpoint{3.260202in}{2.741118in}}%
\pgfpathcurveto{\pgfqpoint{3.266026in}{2.746942in}}{\pgfqpoint{3.269299in}{2.754842in}}{\pgfqpoint{3.269299in}{2.763079in}}%
\pgfpathcurveto{\pgfqpoint{3.269299in}{2.771315in}}{\pgfqpoint{3.266026in}{2.779215in}}{\pgfqpoint{3.260202in}{2.785039in}}%
\pgfpathcurveto{\pgfqpoint{3.254378in}{2.790863in}}{\pgfqpoint{3.246478in}{2.794135in}}{\pgfqpoint{3.238242in}{2.794135in}}%
\pgfpathcurveto{\pgfqpoint{3.230006in}{2.794135in}}{\pgfqpoint{3.222106in}{2.790863in}}{\pgfqpoint{3.216282in}{2.785039in}}%
\pgfpathcurveto{\pgfqpoint{3.210458in}{2.779215in}}{\pgfqpoint{3.207186in}{2.771315in}}{\pgfqpoint{3.207186in}{2.763079in}}%
\pgfpathcurveto{\pgfqpoint{3.207186in}{2.754842in}}{\pgfqpoint{3.210458in}{2.746942in}}{\pgfqpoint{3.216282in}{2.741118in}}%
\pgfpathcurveto{\pgfqpoint{3.222106in}{2.735295in}}{\pgfqpoint{3.230006in}{2.732022in}}{\pgfqpoint{3.238242in}{2.732022in}}%
\pgfpathclose%
\pgfusepath{stroke,fill}%
\end{pgfscope}%
\begin{pgfscope}%
\pgfpathrectangle{\pgfqpoint{0.100000in}{0.220728in}}{\pgfqpoint{3.696000in}{3.696000in}}%
\pgfusepath{clip}%
\pgfsetbuttcap%
\pgfsetroundjoin%
\definecolor{currentfill}{rgb}{0.121569,0.466667,0.705882}%
\pgfsetfillcolor{currentfill}%
\pgfsetfillopacity{0.728848}%
\pgfsetlinewidth{1.003750pt}%
\definecolor{currentstroke}{rgb}{0.121569,0.466667,0.705882}%
\pgfsetstrokecolor{currentstroke}%
\pgfsetstrokeopacity{0.728848}%
\pgfsetdash{}{0pt}%
\pgfpathmoveto{\pgfqpoint{0.934862in}{2.200692in}}%
\pgfpathcurveto{\pgfqpoint{0.943098in}{2.200692in}}{\pgfqpoint{0.950998in}{2.203965in}}{\pgfqpoint{0.956822in}{2.209789in}}%
\pgfpathcurveto{\pgfqpoint{0.962646in}{2.215613in}}{\pgfqpoint{0.965918in}{2.223513in}}{\pgfqpoint{0.965918in}{2.231749in}}%
\pgfpathcurveto{\pgfqpoint{0.965918in}{2.239985in}}{\pgfqpoint{0.962646in}{2.247885in}}{\pgfqpoint{0.956822in}{2.253709in}}%
\pgfpathcurveto{\pgfqpoint{0.950998in}{2.259533in}}{\pgfqpoint{0.943098in}{2.262805in}}{\pgfqpoint{0.934862in}{2.262805in}}%
\pgfpathcurveto{\pgfqpoint{0.926625in}{2.262805in}}{\pgfqpoint{0.918725in}{2.259533in}}{\pgfqpoint{0.912901in}{2.253709in}}%
\pgfpathcurveto{\pgfqpoint{0.907077in}{2.247885in}}{\pgfqpoint{0.903805in}{2.239985in}}{\pgfqpoint{0.903805in}{2.231749in}}%
\pgfpathcurveto{\pgfqpoint{0.903805in}{2.223513in}}{\pgfqpoint{0.907077in}{2.215613in}}{\pgfqpoint{0.912901in}{2.209789in}}%
\pgfpathcurveto{\pgfqpoint{0.918725in}{2.203965in}}{\pgfqpoint{0.926625in}{2.200692in}}{\pgfqpoint{0.934862in}{2.200692in}}%
\pgfpathclose%
\pgfusepath{stroke,fill}%
\end{pgfscope}%
\begin{pgfscope}%
\pgfpathrectangle{\pgfqpoint{0.100000in}{0.220728in}}{\pgfqpoint{3.696000in}{3.696000in}}%
\pgfusepath{clip}%
\pgfsetbuttcap%
\pgfsetroundjoin%
\definecolor{currentfill}{rgb}{0.121569,0.466667,0.705882}%
\pgfsetfillcolor{currentfill}%
\pgfsetfillopacity{0.729993}%
\pgfsetlinewidth{1.003750pt}%
\definecolor{currentstroke}{rgb}{0.121569,0.466667,0.705882}%
\pgfsetstrokecolor{currentstroke}%
\pgfsetstrokeopacity{0.729993}%
\pgfsetdash{}{0pt}%
\pgfpathmoveto{\pgfqpoint{3.230944in}{2.720034in}}%
\pgfpathcurveto{\pgfqpoint{3.239180in}{2.720034in}}{\pgfqpoint{3.247080in}{2.723306in}}{\pgfqpoint{3.252904in}{2.729130in}}%
\pgfpathcurveto{\pgfqpoint{3.258728in}{2.734954in}}{\pgfqpoint{3.262000in}{2.742854in}}{\pgfqpoint{3.262000in}{2.751090in}}%
\pgfpathcurveto{\pgfqpoint{3.262000in}{2.759326in}}{\pgfqpoint{3.258728in}{2.767226in}}{\pgfqpoint{3.252904in}{2.773050in}}%
\pgfpathcurveto{\pgfqpoint{3.247080in}{2.778874in}}{\pgfqpoint{3.239180in}{2.782147in}}{\pgfqpoint{3.230944in}{2.782147in}}%
\pgfpathcurveto{\pgfqpoint{3.222707in}{2.782147in}}{\pgfqpoint{3.214807in}{2.778874in}}{\pgfqpoint{3.208983in}{2.773050in}}%
\pgfpathcurveto{\pgfqpoint{3.203159in}{2.767226in}}{\pgfqpoint{3.199887in}{2.759326in}}{\pgfqpoint{3.199887in}{2.751090in}}%
\pgfpathcurveto{\pgfqpoint{3.199887in}{2.742854in}}{\pgfqpoint{3.203159in}{2.734954in}}{\pgfqpoint{3.208983in}{2.729130in}}%
\pgfpathcurveto{\pgfqpoint{3.214807in}{2.723306in}}{\pgfqpoint{3.222707in}{2.720034in}}{\pgfqpoint{3.230944in}{2.720034in}}%
\pgfpathclose%
\pgfusepath{stroke,fill}%
\end{pgfscope}%
\begin{pgfscope}%
\pgfpathrectangle{\pgfqpoint{0.100000in}{0.220728in}}{\pgfqpoint{3.696000in}{3.696000in}}%
\pgfusepath{clip}%
\pgfsetbuttcap%
\pgfsetroundjoin%
\definecolor{currentfill}{rgb}{0.121569,0.466667,0.705882}%
\pgfsetfillcolor{currentfill}%
\pgfsetfillopacity{0.731026}%
\pgfsetlinewidth{1.003750pt}%
\definecolor{currentstroke}{rgb}{0.121569,0.466667,0.705882}%
\pgfsetstrokecolor{currentstroke}%
\pgfsetstrokeopacity{0.731026}%
\pgfsetdash{}{0pt}%
\pgfpathmoveto{\pgfqpoint{0.945180in}{2.193091in}}%
\pgfpathcurveto{\pgfqpoint{0.953416in}{2.193091in}}{\pgfqpoint{0.961317in}{2.196363in}}{\pgfqpoint{0.967140in}{2.202187in}}%
\pgfpathcurveto{\pgfqpoint{0.972964in}{2.208011in}}{\pgfqpoint{0.976237in}{2.215911in}}{\pgfqpoint{0.976237in}{2.224148in}}%
\pgfpathcurveto{\pgfqpoint{0.976237in}{2.232384in}}{\pgfqpoint{0.972964in}{2.240284in}}{\pgfqpoint{0.967140in}{2.246108in}}%
\pgfpathcurveto{\pgfqpoint{0.961317in}{2.251932in}}{\pgfqpoint{0.953416in}{2.255204in}}{\pgfqpoint{0.945180in}{2.255204in}}%
\pgfpathcurveto{\pgfqpoint{0.936944in}{2.255204in}}{\pgfqpoint{0.929044in}{2.251932in}}{\pgfqpoint{0.923220in}{2.246108in}}%
\pgfpathcurveto{\pgfqpoint{0.917396in}{2.240284in}}{\pgfqpoint{0.914124in}{2.232384in}}{\pgfqpoint{0.914124in}{2.224148in}}%
\pgfpathcurveto{\pgfqpoint{0.914124in}{2.215911in}}{\pgfqpoint{0.917396in}{2.208011in}}{\pgfqpoint{0.923220in}{2.202187in}}%
\pgfpathcurveto{\pgfqpoint{0.929044in}{2.196363in}}{\pgfqpoint{0.936944in}{2.193091in}}{\pgfqpoint{0.945180in}{2.193091in}}%
\pgfpathclose%
\pgfusepath{stroke,fill}%
\end{pgfscope}%
\begin{pgfscope}%
\pgfpathrectangle{\pgfqpoint{0.100000in}{0.220728in}}{\pgfqpoint{3.696000in}{3.696000in}}%
\pgfusepath{clip}%
\pgfsetbuttcap%
\pgfsetroundjoin%
\definecolor{currentfill}{rgb}{0.121569,0.466667,0.705882}%
\pgfsetfillcolor{currentfill}%
\pgfsetfillopacity{0.732715}%
\pgfsetlinewidth{1.003750pt}%
\definecolor{currentstroke}{rgb}{0.121569,0.466667,0.705882}%
\pgfsetstrokecolor{currentstroke}%
\pgfsetstrokeopacity{0.732715}%
\pgfsetdash{}{0pt}%
\pgfpathmoveto{\pgfqpoint{0.953874in}{2.188824in}}%
\pgfpathcurveto{\pgfqpoint{0.962111in}{2.188824in}}{\pgfqpoint{0.970011in}{2.192097in}}{\pgfqpoint{0.975835in}{2.197921in}}%
\pgfpathcurveto{\pgfqpoint{0.981659in}{2.203744in}}{\pgfqpoint{0.984931in}{2.211645in}}{\pgfqpoint{0.984931in}{2.219881in}}%
\pgfpathcurveto{\pgfqpoint{0.984931in}{2.228117in}}{\pgfqpoint{0.981659in}{2.236017in}}{\pgfqpoint{0.975835in}{2.241841in}}%
\pgfpathcurveto{\pgfqpoint{0.970011in}{2.247665in}}{\pgfqpoint{0.962111in}{2.250937in}}{\pgfqpoint{0.953874in}{2.250937in}}%
\pgfpathcurveto{\pgfqpoint{0.945638in}{2.250937in}}{\pgfqpoint{0.937738in}{2.247665in}}{\pgfqpoint{0.931914in}{2.241841in}}%
\pgfpathcurveto{\pgfqpoint{0.926090in}{2.236017in}}{\pgfqpoint{0.922818in}{2.228117in}}{\pgfqpoint{0.922818in}{2.219881in}}%
\pgfpathcurveto{\pgfqpoint{0.922818in}{2.211645in}}{\pgfqpoint{0.926090in}{2.203744in}}{\pgfqpoint{0.931914in}{2.197921in}}%
\pgfpathcurveto{\pgfqpoint{0.937738in}{2.192097in}}{\pgfqpoint{0.945638in}{2.188824in}}{\pgfqpoint{0.953874in}{2.188824in}}%
\pgfpathclose%
\pgfusepath{stroke,fill}%
\end{pgfscope}%
\begin{pgfscope}%
\pgfpathrectangle{\pgfqpoint{0.100000in}{0.220728in}}{\pgfqpoint{3.696000in}{3.696000in}}%
\pgfusepath{clip}%
\pgfsetbuttcap%
\pgfsetroundjoin%
\definecolor{currentfill}{rgb}{0.121569,0.466667,0.705882}%
\pgfsetfillcolor{currentfill}%
\pgfsetfillopacity{0.733072}%
\pgfsetlinewidth{1.003750pt}%
\definecolor{currentstroke}{rgb}{0.121569,0.466667,0.705882}%
\pgfsetstrokecolor{currentstroke}%
\pgfsetstrokeopacity{0.733072}%
\pgfsetdash{}{0pt}%
\pgfpathmoveto{\pgfqpoint{3.219356in}{2.705179in}}%
\pgfpathcurveto{\pgfqpoint{3.227592in}{2.705179in}}{\pgfqpoint{3.235492in}{2.708452in}}{\pgfqpoint{3.241316in}{2.714275in}}%
\pgfpathcurveto{\pgfqpoint{3.247140in}{2.720099in}}{\pgfqpoint{3.250412in}{2.727999in}}{\pgfqpoint{3.250412in}{2.736236in}}%
\pgfpathcurveto{\pgfqpoint{3.250412in}{2.744472in}}{\pgfqpoint{3.247140in}{2.752372in}}{\pgfqpoint{3.241316in}{2.758196in}}%
\pgfpathcurveto{\pgfqpoint{3.235492in}{2.764020in}}{\pgfqpoint{3.227592in}{2.767292in}}{\pgfqpoint{3.219356in}{2.767292in}}%
\pgfpathcurveto{\pgfqpoint{3.211120in}{2.767292in}}{\pgfqpoint{3.203219in}{2.764020in}}{\pgfqpoint{3.197396in}{2.758196in}}%
\pgfpathcurveto{\pgfqpoint{3.191572in}{2.752372in}}{\pgfqpoint{3.188299in}{2.744472in}}{\pgfqpoint{3.188299in}{2.736236in}}%
\pgfpathcurveto{\pgfqpoint{3.188299in}{2.727999in}}{\pgfqpoint{3.191572in}{2.720099in}}{\pgfqpoint{3.197396in}{2.714275in}}%
\pgfpathcurveto{\pgfqpoint{3.203219in}{2.708452in}}{\pgfqpoint{3.211120in}{2.705179in}}{\pgfqpoint{3.219356in}{2.705179in}}%
\pgfpathclose%
\pgfusepath{stroke,fill}%
\end{pgfscope}%
\begin{pgfscope}%
\pgfpathrectangle{\pgfqpoint{0.100000in}{0.220728in}}{\pgfqpoint{3.696000in}{3.696000in}}%
\pgfusepath{clip}%
\pgfsetbuttcap%
\pgfsetroundjoin%
\definecolor{currentfill}{rgb}{0.121569,0.466667,0.705882}%
\pgfsetfillcolor{currentfill}%
\pgfsetfillopacity{0.733642}%
\pgfsetlinewidth{1.003750pt}%
\definecolor{currentstroke}{rgb}{0.121569,0.466667,0.705882}%
\pgfsetstrokecolor{currentstroke}%
\pgfsetstrokeopacity{0.733642}%
\pgfsetdash{}{0pt}%
\pgfpathmoveto{\pgfqpoint{0.958367in}{2.185430in}}%
\pgfpathcurveto{\pgfqpoint{0.966603in}{2.185430in}}{\pgfqpoint{0.974503in}{2.188702in}}{\pgfqpoint{0.980327in}{2.194526in}}%
\pgfpathcurveto{\pgfqpoint{0.986151in}{2.200350in}}{\pgfqpoint{0.989423in}{2.208250in}}{\pgfqpoint{0.989423in}{2.216486in}}%
\pgfpathcurveto{\pgfqpoint{0.989423in}{2.224723in}}{\pgfqpoint{0.986151in}{2.232623in}}{\pgfqpoint{0.980327in}{2.238447in}}%
\pgfpathcurveto{\pgfqpoint{0.974503in}{2.244271in}}{\pgfqpoint{0.966603in}{2.247543in}}{\pgfqpoint{0.958367in}{2.247543in}}%
\pgfpathcurveto{\pgfqpoint{0.950130in}{2.247543in}}{\pgfqpoint{0.942230in}{2.244271in}}{\pgfqpoint{0.936406in}{2.238447in}}%
\pgfpathcurveto{\pgfqpoint{0.930582in}{2.232623in}}{\pgfqpoint{0.927310in}{2.224723in}}{\pgfqpoint{0.927310in}{2.216486in}}%
\pgfpathcurveto{\pgfqpoint{0.927310in}{2.208250in}}{\pgfqpoint{0.930582in}{2.200350in}}{\pgfqpoint{0.936406in}{2.194526in}}%
\pgfpathcurveto{\pgfqpoint{0.942230in}{2.188702in}}{\pgfqpoint{0.950130in}{2.185430in}}{\pgfqpoint{0.958367in}{2.185430in}}%
\pgfpathclose%
\pgfusepath{stroke,fill}%
\end{pgfscope}%
\begin{pgfscope}%
\pgfpathrectangle{\pgfqpoint{0.100000in}{0.220728in}}{\pgfqpoint{3.696000in}{3.696000in}}%
\pgfusepath{clip}%
\pgfsetbuttcap%
\pgfsetroundjoin%
\definecolor{currentfill}{rgb}{0.121569,0.466667,0.705882}%
\pgfsetfillcolor{currentfill}%
\pgfsetfillopacity{0.735439}%
\pgfsetlinewidth{1.003750pt}%
\definecolor{currentstroke}{rgb}{0.121569,0.466667,0.705882}%
\pgfsetstrokecolor{currentstroke}%
\pgfsetstrokeopacity{0.735439}%
\pgfsetdash{}{0pt}%
\pgfpathmoveto{\pgfqpoint{0.966564in}{2.179814in}}%
\pgfpathcurveto{\pgfqpoint{0.974800in}{2.179814in}}{\pgfqpoint{0.982700in}{2.183086in}}{\pgfqpoint{0.988524in}{2.188910in}}%
\pgfpathcurveto{\pgfqpoint{0.994348in}{2.194734in}}{\pgfqpoint{0.997620in}{2.202634in}}{\pgfqpoint{0.997620in}{2.210871in}}%
\pgfpathcurveto{\pgfqpoint{0.997620in}{2.219107in}}{\pgfqpoint{0.994348in}{2.227007in}}{\pgfqpoint{0.988524in}{2.232831in}}%
\pgfpathcurveto{\pgfqpoint{0.982700in}{2.238655in}}{\pgfqpoint{0.974800in}{2.241927in}}{\pgfqpoint{0.966564in}{2.241927in}}%
\pgfpathcurveto{\pgfqpoint{0.958327in}{2.241927in}}{\pgfqpoint{0.950427in}{2.238655in}}{\pgfqpoint{0.944603in}{2.232831in}}%
\pgfpathcurveto{\pgfqpoint{0.938780in}{2.227007in}}{\pgfqpoint{0.935507in}{2.219107in}}{\pgfqpoint{0.935507in}{2.210871in}}%
\pgfpathcurveto{\pgfqpoint{0.935507in}{2.202634in}}{\pgfqpoint{0.938780in}{2.194734in}}{\pgfqpoint{0.944603in}{2.188910in}}%
\pgfpathcurveto{\pgfqpoint{0.950427in}{2.183086in}}{\pgfqpoint{0.958327in}{2.179814in}}{\pgfqpoint{0.966564in}{2.179814in}}%
\pgfpathclose%
\pgfusepath{stroke,fill}%
\end{pgfscope}%
\begin{pgfscope}%
\pgfpathrectangle{\pgfqpoint{0.100000in}{0.220728in}}{\pgfqpoint{3.696000in}{3.696000in}}%
\pgfusepath{clip}%
\pgfsetbuttcap%
\pgfsetroundjoin%
\definecolor{currentfill}{rgb}{0.121569,0.466667,0.705882}%
\pgfsetfillcolor{currentfill}%
\pgfsetfillopacity{0.737077}%
\pgfsetlinewidth{1.003750pt}%
\definecolor{currentstroke}{rgb}{0.121569,0.466667,0.705882}%
\pgfsetstrokecolor{currentstroke}%
\pgfsetstrokeopacity{0.737077}%
\pgfsetdash{}{0pt}%
\pgfpathmoveto{\pgfqpoint{3.206166in}{2.684778in}}%
\pgfpathcurveto{\pgfqpoint{3.214403in}{2.684778in}}{\pgfqpoint{3.222303in}{2.688050in}}{\pgfqpoint{3.228127in}{2.693874in}}%
\pgfpathcurveto{\pgfqpoint{3.233951in}{2.699698in}}{\pgfqpoint{3.237223in}{2.707598in}}{\pgfqpoint{3.237223in}{2.715834in}}%
\pgfpathcurveto{\pgfqpoint{3.237223in}{2.724071in}}{\pgfqpoint{3.233951in}{2.731971in}}{\pgfqpoint{3.228127in}{2.737795in}}%
\pgfpathcurveto{\pgfqpoint{3.222303in}{2.743618in}}{\pgfqpoint{3.214403in}{2.746891in}}{\pgfqpoint{3.206166in}{2.746891in}}%
\pgfpathcurveto{\pgfqpoint{3.197930in}{2.746891in}}{\pgfqpoint{3.190030in}{2.743618in}}{\pgfqpoint{3.184206in}{2.737795in}}%
\pgfpathcurveto{\pgfqpoint{3.178382in}{2.731971in}}{\pgfqpoint{3.175110in}{2.724071in}}{\pgfqpoint{3.175110in}{2.715834in}}%
\pgfpathcurveto{\pgfqpoint{3.175110in}{2.707598in}}{\pgfqpoint{3.178382in}{2.699698in}}{\pgfqpoint{3.184206in}{2.693874in}}%
\pgfpathcurveto{\pgfqpoint{3.190030in}{2.688050in}}{\pgfqpoint{3.197930in}{2.684778in}}{\pgfqpoint{3.206166in}{2.684778in}}%
\pgfpathclose%
\pgfusepath{stroke,fill}%
\end{pgfscope}%
\begin{pgfscope}%
\pgfpathrectangle{\pgfqpoint{0.100000in}{0.220728in}}{\pgfqpoint{3.696000in}{3.696000in}}%
\pgfusepath{clip}%
\pgfsetbuttcap%
\pgfsetroundjoin%
\definecolor{currentfill}{rgb}{0.121569,0.466667,0.705882}%
\pgfsetfillcolor{currentfill}%
\pgfsetfillopacity{0.738756}%
\pgfsetlinewidth{1.003750pt}%
\definecolor{currentstroke}{rgb}{0.121569,0.466667,0.705882}%
\pgfsetstrokecolor{currentstroke}%
\pgfsetstrokeopacity{0.738756}%
\pgfsetdash{}{0pt}%
\pgfpathmoveto{\pgfqpoint{0.979293in}{2.165271in}}%
\pgfpathcurveto{\pgfqpoint{0.987530in}{2.165271in}}{\pgfqpoint{0.995430in}{2.168543in}}{\pgfqpoint{1.001254in}{2.174367in}}%
\pgfpathcurveto{\pgfqpoint{1.007077in}{2.180191in}}{\pgfqpoint{1.010350in}{2.188091in}}{\pgfqpoint{1.010350in}{2.196327in}}%
\pgfpathcurveto{\pgfqpoint{1.010350in}{2.204564in}}{\pgfqpoint{1.007077in}{2.212464in}}{\pgfqpoint{1.001254in}{2.218288in}}%
\pgfpathcurveto{\pgfqpoint{0.995430in}{2.224111in}}{\pgfqpoint{0.987530in}{2.227384in}}{\pgfqpoint{0.979293in}{2.227384in}}%
\pgfpathcurveto{\pgfqpoint{0.971057in}{2.227384in}}{\pgfqpoint{0.963157in}{2.224111in}}{\pgfqpoint{0.957333in}{2.218288in}}%
\pgfpathcurveto{\pgfqpoint{0.951509in}{2.212464in}}{\pgfqpoint{0.948237in}{2.204564in}}{\pgfqpoint{0.948237in}{2.196327in}}%
\pgfpathcurveto{\pgfqpoint{0.948237in}{2.188091in}}{\pgfqpoint{0.951509in}{2.180191in}}{\pgfqpoint{0.957333in}{2.174367in}}%
\pgfpathcurveto{\pgfqpoint{0.963157in}{2.168543in}}{\pgfqpoint{0.971057in}{2.165271in}}{\pgfqpoint{0.979293in}{2.165271in}}%
\pgfpathclose%
\pgfusepath{stroke,fill}%
\end{pgfscope}%
\begin{pgfscope}%
\pgfpathrectangle{\pgfqpoint{0.100000in}{0.220728in}}{\pgfqpoint{3.696000in}{3.696000in}}%
\pgfusepath{clip}%
\pgfsetbuttcap%
\pgfsetroundjoin%
\definecolor{currentfill}{rgb}{0.121569,0.466667,0.705882}%
\pgfsetfillcolor{currentfill}%
\pgfsetfillopacity{0.741149}%
\pgfsetlinewidth{1.003750pt}%
\definecolor{currentstroke}{rgb}{0.121569,0.466667,0.705882}%
\pgfsetstrokecolor{currentstroke}%
\pgfsetstrokeopacity{0.741149}%
\pgfsetdash{}{0pt}%
\pgfpathmoveto{\pgfqpoint{3.187946in}{2.661715in}}%
\pgfpathcurveto{\pgfqpoint{3.196182in}{2.661715in}}{\pgfqpoint{3.204082in}{2.664987in}}{\pgfqpoint{3.209906in}{2.670811in}}%
\pgfpathcurveto{\pgfqpoint{3.215730in}{2.676635in}}{\pgfqpoint{3.219002in}{2.684535in}}{\pgfqpoint{3.219002in}{2.692772in}}%
\pgfpathcurveto{\pgfqpoint{3.219002in}{2.701008in}}{\pgfqpoint{3.215730in}{2.708908in}}{\pgfqpoint{3.209906in}{2.714732in}}%
\pgfpathcurveto{\pgfqpoint{3.204082in}{2.720556in}}{\pgfqpoint{3.196182in}{2.723828in}}{\pgfqpoint{3.187946in}{2.723828in}}%
\pgfpathcurveto{\pgfqpoint{3.179709in}{2.723828in}}{\pgfqpoint{3.171809in}{2.720556in}}{\pgfqpoint{3.165985in}{2.714732in}}%
\pgfpathcurveto{\pgfqpoint{3.160161in}{2.708908in}}{\pgfqpoint{3.156889in}{2.701008in}}{\pgfqpoint{3.156889in}{2.692772in}}%
\pgfpathcurveto{\pgfqpoint{3.156889in}{2.684535in}}{\pgfqpoint{3.160161in}{2.676635in}}{\pgfqpoint{3.165985in}{2.670811in}}%
\pgfpathcurveto{\pgfqpoint{3.171809in}{2.664987in}}{\pgfqpoint{3.179709in}{2.661715in}}{\pgfqpoint{3.187946in}{2.661715in}}%
\pgfpathclose%
\pgfusepath{stroke,fill}%
\end{pgfscope}%
\begin{pgfscope}%
\pgfpathrectangle{\pgfqpoint{0.100000in}{0.220728in}}{\pgfqpoint{3.696000in}{3.696000in}}%
\pgfusepath{clip}%
\pgfsetbuttcap%
\pgfsetroundjoin%
\definecolor{currentfill}{rgb}{0.121569,0.466667,0.705882}%
\pgfsetfillcolor{currentfill}%
\pgfsetfillopacity{0.741723}%
\pgfsetlinewidth{1.003750pt}%
\definecolor{currentstroke}{rgb}{0.121569,0.466667,0.705882}%
\pgfsetstrokecolor{currentstroke}%
\pgfsetstrokeopacity{0.741723}%
\pgfsetdash{}{0pt}%
\pgfpathmoveto{\pgfqpoint{0.991296in}{2.154851in}}%
\pgfpathcurveto{\pgfqpoint{0.999532in}{2.154851in}}{\pgfqpoint{1.007432in}{2.158123in}}{\pgfqpoint{1.013256in}{2.163947in}}%
\pgfpathcurveto{\pgfqpoint{1.019080in}{2.169771in}}{\pgfqpoint{1.022352in}{2.177671in}}{\pgfqpoint{1.022352in}{2.185907in}}%
\pgfpathcurveto{\pgfqpoint{1.022352in}{2.194144in}}{\pgfqpoint{1.019080in}{2.202044in}}{\pgfqpoint{1.013256in}{2.207868in}}%
\pgfpathcurveto{\pgfqpoint{1.007432in}{2.213692in}}{\pgfqpoint{0.999532in}{2.216964in}}{\pgfqpoint{0.991296in}{2.216964in}}%
\pgfpathcurveto{\pgfqpoint{0.983059in}{2.216964in}}{\pgfqpoint{0.975159in}{2.213692in}}{\pgfqpoint{0.969335in}{2.207868in}}%
\pgfpathcurveto{\pgfqpoint{0.963511in}{2.202044in}}{\pgfqpoint{0.960239in}{2.194144in}}{\pgfqpoint{0.960239in}{2.185907in}}%
\pgfpathcurveto{\pgfqpoint{0.960239in}{2.177671in}}{\pgfqpoint{0.963511in}{2.169771in}}{\pgfqpoint{0.969335in}{2.163947in}}%
\pgfpathcurveto{\pgfqpoint{0.975159in}{2.158123in}}{\pgfqpoint{0.983059in}{2.154851in}}{\pgfqpoint{0.991296in}{2.154851in}}%
\pgfpathclose%
\pgfusepath{stroke,fill}%
\end{pgfscope}%
\begin{pgfscope}%
\pgfpathrectangle{\pgfqpoint{0.100000in}{0.220728in}}{\pgfqpoint{3.696000in}{3.696000in}}%
\pgfusepath{clip}%
\pgfsetbuttcap%
\pgfsetroundjoin%
\definecolor{currentfill}{rgb}{0.121569,0.466667,0.705882}%
\pgfsetfillcolor{currentfill}%
\pgfsetfillopacity{0.743986}%
\pgfsetlinewidth{1.003750pt}%
\definecolor{currentstroke}{rgb}{0.121569,0.466667,0.705882}%
\pgfsetstrokecolor{currentstroke}%
\pgfsetstrokeopacity{0.743986}%
\pgfsetdash{}{0pt}%
\pgfpathmoveto{\pgfqpoint{1.001523in}{2.149184in}}%
\pgfpathcurveto{\pgfqpoint{1.009759in}{2.149184in}}{\pgfqpoint{1.017659in}{2.152456in}}{\pgfqpoint{1.023483in}{2.158280in}}%
\pgfpathcurveto{\pgfqpoint{1.029307in}{2.164104in}}{\pgfqpoint{1.032579in}{2.172004in}}{\pgfqpoint{1.032579in}{2.180241in}}%
\pgfpathcurveto{\pgfqpoint{1.032579in}{2.188477in}}{\pgfqpoint{1.029307in}{2.196377in}}{\pgfqpoint{1.023483in}{2.202201in}}%
\pgfpathcurveto{\pgfqpoint{1.017659in}{2.208025in}}{\pgfqpoint{1.009759in}{2.211297in}}{\pgfqpoint{1.001523in}{2.211297in}}%
\pgfpathcurveto{\pgfqpoint{0.993286in}{2.211297in}}{\pgfqpoint{0.985386in}{2.208025in}}{\pgfqpoint{0.979562in}{2.202201in}}%
\pgfpathcurveto{\pgfqpoint{0.973739in}{2.196377in}}{\pgfqpoint{0.970466in}{2.188477in}}{\pgfqpoint{0.970466in}{2.180241in}}%
\pgfpathcurveto{\pgfqpoint{0.970466in}{2.172004in}}{\pgfqpoint{0.973739in}{2.164104in}}{\pgfqpoint{0.979562in}{2.158280in}}%
\pgfpathcurveto{\pgfqpoint{0.985386in}{2.152456in}}{\pgfqpoint{0.993286in}{2.149184in}}{\pgfqpoint{1.001523in}{2.149184in}}%
\pgfpathclose%
\pgfusepath{stroke,fill}%
\end{pgfscope}%
\begin{pgfscope}%
\pgfpathrectangle{\pgfqpoint{0.100000in}{0.220728in}}{\pgfqpoint{3.696000in}{3.696000in}}%
\pgfusepath{clip}%
\pgfsetbuttcap%
\pgfsetroundjoin%
\definecolor{currentfill}{rgb}{0.121569,0.466667,0.705882}%
\pgfsetfillcolor{currentfill}%
\pgfsetfillopacity{0.745450}%
\pgfsetlinewidth{1.003750pt}%
\definecolor{currentstroke}{rgb}{0.121569,0.466667,0.705882}%
\pgfsetstrokecolor{currentstroke}%
\pgfsetstrokeopacity{0.745450}%
\pgfsetdash{}{0pt}%
\pgfpathmoveto{\pgfqpoint{1.007544in}{2.143400in}}%
\pgfpathcurveto{\pgfqpoint{1.015780in}{2.143400in}}{\pgfqpoint{1.023680in}{2.146673in}}{\pgfqpoint{1.029504in}{2.152497in}}%
\pgfpathcurveto{\pgfqpoint{1.035328in}{2.158321in}}{\pgfqpoint{1.038600in}{2.166221in}}{\pgfqpoint{1.038600in}{2.174457in}}%
\pgfpathcurveto{\pgfqpoint{1.038600in}{2.182693in}}{\pgfqpoint{1.035328in}{2.190593in}}{\pgfqpoint{1.029504in}{2.196417in}}%
\pgfpathcurveto{\pgfqpoint{1.023680in}{2.202241in}}{\pgfqpoint{1.015780in}{2.205513in}}{\pgfqpoint{1.007544in}{2.205513in}}%
\pgfpathcurveto{\pgfqpoint{0.999307in}{2.205513in}}{\pgfqpoint{0.991407in}{2.202241in}}{\pgfqpoint{0.985583in}{2.196417in}}%
\pgfpathcurveto{\pgfqpoint{0.979759in}{2.190593in}}{\pgfqpoint{0.976487in}{2.182693in}}{\pgfqpoint{0.976487in}{2.174457in}}%
\pgfpathcurveto{\pgfqpoint{0.976487in}{2.166221in}}{\pgfqpoint{0.979759in}{2.158321in}}{\pgfqpoint{0.985583in}{2.152497in}}%
\pgfpathcurveto{\pgfqpoint{0.991407in}{2.146673in}}{\pgfqpoint{0.999307in}{2.143400in}}{\pgfqpoint{1.007544in}{2.143400in}}%
\pgfpathclose%
\pgfusepath{stroke,fill}%
\end{pgfscope}%
\begin{pgfscope}%
\pgfpathrectangle{\pgfqpoint{0.100000in}{0.220728in}}{\pgfqpoint{3.696000in}{3.696000in}}%
\pgfusepath{clip}%
\pgfsetbuttcap%
\pgfsetroundjoin%
\definecolor{currentfill}{rgb}{0.121569,0.466667,0.705882}%
\pgfsetfillcolor{currentfill}%
\pgfsetfillopacity{0.746187}%
\pgfsetlinewidth{1.003750pt}%
\definecolor{currentstroke}{rgb}{0.121569,0.466667,0.705882}%
\pgfsetstrokecolor{currentstroke}%
\pgfsetstrokeopacity{0.746187}%
\pgfsetdash{}{0pt}%
\pgfpathmoveto{\pgfqpoint{1.010753in}{2.139983in}}%
\pgfpathcurveto{\pgfqpoint{1.018989in}{2.139983in}}{\pgfqpoint{1.026889in}{2.143255in}}{\pgfqpoint{1.032713in}{2.149079in}}%
\pgfpathcurveto{\pgfqpoint{1.038537in}{2.154903in}}{\pgfqpoint{1.041809in}{2.162803in}}{\pgfqpoint{1.041809in}{2.171040in}}%
\pgfpathcurveto{\pgfqpoint{1.041809in}{2.179276in}}{\pgfqpoint{1.038537in}{2.187176in}}{\pgfqpoint{1.032713in}{2.193000in}}%
\pgfpathcurveto{\pgfqpoint{1.026889in}{2.198824in}}{\pgfqpoint{1.018989in}{2.202096in}}{\pgfqpoint{1.010753in}{2.202096in}}%
\pgfpathcurveto{\pgfqpoint{1.002516in}{2.202096in}}{\pgfqpoint{0.994616in}{2.198824in}}{\pgfqpoint{0.988792in}{2.193000in}}%
\pgfpathcurveto{\pgfqpoint{0.982968in}{2.187176in}}{\pgfqpoint{0.979696in}{2.179276in}}{\pgfqpoint{0.979696in}{2.171040in}}%
\pgfpathcurveto{\pgfqpoint{0.979696in}{2.162803in}}{\pgfqpoint{0.982968in}{2.154903in}}{\pgfqpoint{0.988792in}{2.149079in}}%
\pgfpathcurveto{\pgfqpoint{0.994616in}{2.143255in}}{\pgfqpoint{1.002516in}{2.139983in}}{\pgfqpoint{1.010753in}{2.139983in}}%
\pgfpathclose%
\pgfusepath{stroke,fill}%
\end{pgfscope}%
\begin{pgfscope}%
\pgfpathrectangle{\pgfqpoint{0.100000in}{0.220728in}}{\pgfqpoint{3.696000in}{3.696000in}}%
\pgfusepath{clip}%
\pgfsetbuttcap%
\pgfsetroundjoin%
\definecolor{currentfill}{rgb}{0.121569,0.466667,0.705882}%
\pgfsetfillcolor{currentfill}%
\pgfsetfillopacity{0.746599}%
\pgfsetlinewidth{1.003750pt}%
\definecolor{currentstroke}{rgb}{0.121569,0.466667,0.705882}%
\pgfsetstrokecolor{currentstroke}%
\pgfsetstrokeopacity{0.746599}%
\pgfsetdash{}{0pt}%
\pgfpathmoveto{\pgfqpoint{1.012533in}{2.138229in}}%
\pgfpathcurveto{\pgfqpoint{1.020769in}{2.138229in}}{\pgfqpoint{1.028669in}{2.141501in}}{\pgfqpoint{1.034493in}{2.147325in}}%
\pgfpathcurveto{\pgfqpoint{1.040317in}{2.153149in}}{\pgfqpoint{1.043589in}{2.161049in}}{\pgfqpoint{1.043589in}{2.169286in}}%
\pgfpathcurveto{\pgfqpoint{1.043589in}{2.177522in}}{\pgfqpoint{1.040317in}{2.185422in}}{\pgfqpoint{1.034493in}{2.191246in}}%
\pgfpathcurveto{\pgfqpoint{1.028669in}{2.197070in}}{\pgfqpoint{1.020769in}{2.200342in}}{\pgfqpoint{1.012533in}{2.200342in}}%
\pgfpathcurveto{\pgfqpoint{1.004296in}{2.200342in}}{\pgfqpoint{0.996396in}{2.197070in}}{\pgfqpoint{0.990572in}{2.191246in}}%
\pgfpathcurveto{\pgfqpoint{0.984748in}{2.185422in}}{\pgfqpoint{0.981476in}{2.177522in}}{\pgfqpoint{0.981476in}{2.169286in}}%
\pgfpathcurveto{\pgfqpoint{0.981476in}{2.161049in}}{\pgfqpoint{0.984748in}{2.153149in}}{\pgfqpoint{0.990572in}{2.147325in}}%
\pgfpathcurveto{\pgfqpoint{0.996396in}{2.141501in}}{\pgfqpoint{1.004296in}{2.138229in}}{\pgfqpoint{1.012533in}{2.138229in}}%
\pgfpathclose%
\pgfusepath{stroke,fill}%
\end{pgfscope}%
\begin{pgfscope}%
\pgfpathrectangle{\pgfqpoint{0.100000in}{0.220728in}}{\pgfqpoint{3.696000in}{3.696000in}}%
\pgfusepath{clip}%
\pgfsetbuttcap%
\pgfsetroundjoin%
\definecolor{currentfill}{rgb}{0.121569,0.466667,0.705882}%
\pgfsetfillcolor{currentfill}%
\pgfsetfillopacity{0.747202}%
\pgfsetlinewidth{1.003750pt}%
\definecolor{currentstroke}{rgb}{0.121569,0.466667,0.705882}%
\pgfsetstrokecolor{currentstroke}%
\pgfsetstrokeopacity{0.747202}%
\pgfsetdash{}{0pt}%
\pgfpathmoveto{\pgfqpoint{3.171045in}{2.632595in}}%
\pgfpathcurveto{\pgfqpoint{3.179281in}{2.632595in}}{\pgfqpoint{3.187181in}{2.635868in}}{\pgfqpoint{3.193005in}{2.641692in}}%
\pgfpathcurveto{\pgfqpoint{3.198829in}{2.647515in}}{\pgfqpoint{3.202101in}{2.655416in}}{\pgfqpoint{3.202101in}{2.663652in}}%
\pgfpathcurveto{\pgfqpoint{3.202101in}{2.671888in}}{\pgfqpoint{3.198829in}{2.679788in}}{\pgfqpoint{3.193005in}{2.685612in}}%
\pgfpathcurveto{\pgfqpoint{3.187181in}{2.691436in}}{\pgfqpoint{3.179281in}{2.694708in}}{\pgfqpoint{3.171045in}{2.694708in}}%
\pgfpathcurveto{\pgfqpoint{3.162809in}{2.694708in}}{\pgfqpoint{3.154909in}{2.691436in}}{\pgfqpoint{3.149085in}{2.685612in}}%
\pgfpathcurveto{\pgfqpoint{3.143261in}{2.679788in}}{\pgfqpoint{3.139988in}{2.671888in}}{\pgfqpoint{3.139988in}{2.663652in}}%
\pgfpathcurveto{\pgfqpoint{3.139988in}{2.655416in}}{\pgfqpoint{3.143261in}{2.647515in}}{\pgfqpoint{3.149085in}{2.641692in}}%
\pgfpathcurveto{\pgfqpoint{3.154909in}{2.635868in}}{\pgfqpoint{3.162809in}{2.632595in}}{\pgfqpoint{3.171045in}{2.632595in}}%
\pgfpathclose%
\pgfusepath{stroke,fill}%
\end{pgfscope}%
\begin{pgfscope}%
\pgfpathrectangle{\pgfqpoint{0.100000in}{0.220728in}}{\pgfqpoint{3.696000in}{3.696000in}}%
\pgfusepath{clip}%
\pgfsetbuttcap%
\pgfsetroundjoin%
\definecolor{currentfill}{rgb}{0.121569,0.466667,0.705882}%
\pgfsetfillcolor{currentfill}%
\pgfsetfillopacity{0.747321}%
\pgfsetlinewidth{1.003750pt}%
\definecolor{currentstroke}{rgb}{0.121569,0.466667,0.705882}%
\pgfsetstrokecolor{currentstroke}%
\pgfsetstrokeopacity{0.747321}%
\pgfsetdash{}{0pt}%
\pgfpathmoveto{\pgfqpoint{1.015725in}{2.134812in}}%
\pgfpathcurveto{\pgfqpoint{1.023962in}{2.134812in}}{\pgfqpoint{1.031862in}{2.138085in}}{\pgfqpoint{1.037686in}{2.143909in}}%
\pgfpathcurveto{\pgfqpoint{1.043510in}{2.149733in}}{\pgfqpoint{1.046782in}{2.157633in}}{\pgfqpoint{1.046782in}{2.165869in}}%
\pgfpathcurveto{\pgfqpoint{1.046782in}{2.174105in}}{\pgfqpoint{1.043510in}{2.182005in}}{\pgfqpoint{1.037686in}{2.187829in}}%
\pgfpathcurveto{\pgfqpoint{1.031862in}{2.193653in}}{\pgfqpoint{1.023962in}{2.196925in}}{\pgfqpoint{1.015725in}{2.196925in}}%
\pgfpathcurveto{\pgfqpoint{1.007489in}{2.196925in}}{\pgfqpoint{0.999589in}{2.193653in}}{\pgfqpoint{0.993765in}{2.187829in}}%
\pgfpathcurveto{\pgfqpoint{0.987941in}{2.182005in}}{\pgfqpoint{0.984669in}{2.174105in}}{\pgfqpoint{0.984669in}{2.165869in}}%
\pgfpathcurveto{\pgfqpoint{0.984669in}{2.157633in}}{\pgfqpoint{0.987941in}{2.149733in}}{\pgfqpoint{0.993765in}{2.143909in}}%
\pgfpathcurveto{\pgfqpoint{0.999589in}{2.138085in}}{\pgfqpoint{1.007489in}{2.134812in}}{\pgfqpoint{1.015725in}{2.134812in}}%
\pgfpathclose%
\pgfusepath{stroke,fill}%
\end{pgfscope}%
\begin{pgfscope}%
\pgfpathrectangle{\pgfqpoint{0.100000in}{0.220728in}}{\pgfqpoint{3.696000in}{3.696000in}}%
\pgfusepath{clip}%
\pgfsetbuttcap%
\pgfsetroundjoin%
\definecolor{currentfill}{rgb}{0.121569,0.466667,0.705882}%
\pgfsetfillcolor{currentfill}%
\pgfsetfillopacity{0.748505}%
\pgfsetlinewidth{1.003750pt}%
\definecolor{currentstroke}{rgb}{0.121569,0.466667,0.705882}%
\pgfsetstrokecolor{currentstroke}%
\pgfsetstrokeopacity{0.748505}%
\pgfsetdash{}{0pt}%
\pgfpathmoveto{\pgfqpoint{1.022185in}{2.129435in}}%
\pgfpathcurveto{\pgfqpoint{1.030421in}{2.129435in}}{\pgfqpoint{1.038321in}{2.132707in}}{\pgfqpoint{1.044145in}{2.138531in}}%
\pgfpathcurveto{\pgfqpoint{1.049969in}{2.144355in}}{\pgfqpoint{1.053241in}{2.152255in}}{\pgfqpoint{1.053241in}{2.160491in}}%
\pgfpathcurveto{\pgfqpoint{1.053241in}{2.168727in}}{\pgfqpoint{1.049969in}{2.176627in}}{\pgfqpoint{1.044145in}{2.182451in}}%
\pgfpathcurveto{\pgfqpoint{1.038321in}{2.188275in}}{\pgfqpoint{1.030421in}{2.191548in}}{\pgfqpoint{1.022185in}{2.191548in}}%
\pgfpathcurveto{\pgfqpoint{1.013948in}{2.191548in}}{\pgfqpoint{1.006048in}{2.188275in}}{\pgfqpoint{1.000224in}{2.182451in}}%
\pgfpathcurveto{\pgfqpoint{0.994400in}{2.176627in}}{\pgfqpoint{0.991128in}{2.168727in}}{\pgfqpoint{0.991128in}{2.160491in}}%
\pgfpathcurveto{\pgfqpoint{0.991128in}{2.152255in}}{\pgfqpoint{0.994400in}{2.144355in}}{\pgfqpoint{1.000224in}{2.138531in}}%
\pgfpathcurveto{\pgfqpoint{1.006048in}{2.132707in}}{\pgfqpoint{1.013948in}{2.129435in}}{\pgfqpoint{1.022185in}{2.129435in}}%
\pgfpathclose%
\pgfusepath{stroke,fill}%
\end{pgfscope}%
\begin{pgfscope}%
\pgfpathrectangle{\pgfqpoint{0.100000in}{0.220728in}}{\pgfqpoint{3.696000in}{3.696000in}}%
\pgfusepath{clip}%
\pgfsetbuttcap%
\pgfsetroundjoin%
\definecolor{currentfill}{rgb}{0.121569,0.466667,0.705882}%
\pgfsetfillcolor{currentfill}%
\pgfsetfillopacity{0.750860}%
\pgfsetlinewidth{1.003750pt}%
\definecolor{currentstroke}{rgb}{0.121569,0.466667,0.705882}%
\pgfsetstrokecolor{currentstroke}%
\pgfsetstrokeopacity{0.750860}%
\pgfsetdash{}{0pt}%
\pgfpathmoveto{\pgfqpoint{1.033710in}{2.120055in}}%
\pgfpathcurveto{\pgfqpoint{1.041946in}{2.120055in}}{\pgfqpoint{1.049846in}{2.123328in}}{\pgfqpoint{1.055670in}{2.129152in}}%
\pgfpathcurveto{\pgfqpoint{1.061494in}{2.134976in}}{\pgfqpoint{1.064766in}{2.142876in}}{\pgfqpoint{1.064766in}{2.151112in}}%
\pgfpathcurveto{\pgfqpoint{1.064766in}{2.159348in}}{\pgfqpoint{1.061494in}{2.167248in}}{\pgfqpoint{1.055670in}{2.173072in}}%
\pgfpathcurveto{\pgfqpoint{1.049846in}{2.178896in}}{\pgfqpoint{1.041946in}{2.182168in}}{\pgfqpoint{1.033710in}{2.182168in}}%
\pgfpathcurveto{\pgfqpoint{1.025473in}{2.182168in}}{\pgfqpoint{1.017573in}{2.178896in}}{\pgfqpoint{1.011749in}{2.173072in}}%
\pgfpathcurveto{\pgfqpoint{1.005925in}{2.167248in}}{\pgfqpoint{1.002653in}{2.159348in}}{\pgfqpoint{1.002653in}{2.151112in}}%
\pgfpathcurveto{\pgfqpoint{1.002653in}{2.142876in}}{\pgfqpoint{1.005925in}{2.134976in}}{\pgfqpoint{1.011749in}{2.129152in}}%
\pgfpathcurveto{\pgfqpoint{1.017573in}{2.123328in}}{\pgfqpoint{1.025473in}{2.120055in}}{\pgfqpoint{1.033710in}{2.120055in}}%
\pgfpathclose%
\pgfusepath{stroke,fill}%
\end{pgfscope}%
\begin{pgfscope}%
\pgfpathrectangle{\pgfqpoint{0.100000in}{0.220728in}}{\pgfqpoint{3.696000in}{3.696000in}}%
\pgfusepath{clip}%
\pgfsetbuttcap%
\pgfsetroundjoin%
\definecolor{currentfill}{rgb}{0.121569,0.466667,0.705882}%
\pgfsetfillcolor{currentfill}%
\pgfsetfillopacity{0.753239}%
\pgfsetlinewidth{1.003750pt}%
\definecolor{currentstroke}{rgb}{0.121569,0.466667,0.705882}%
\pgfsetstrokecolor{currentstroke}%
\pgfsetstrokeopacity{0.753239}%
\pgfsetdash{}{0pt}%
\pgfpathmoveto{\pgfqpoint{3.147593in}{2.600819in}}%
\pgfpathcurveto{\pgfqpoint{3.155830in}{2.600819in}}{\pgfqpoint{3.163730in}{2.604092in}}{\pgfqpoint{3.169554in}{2.609915in}}%
\pgfpathcurveto{\pgfqpoint{3.175378in}{2.615739in}}{\pgfqpoint{3.178650in}{2.623639in}}{\pgfqpoint{3.178650in}{2.631876in}}%
\pgfpathcurveto{\pgfqpoint{3.178650in}{2.640112in}}{\pgfqpoint{3.175378in}{2.648012in}}{\pgfqpoint{3.169554in}{2.653836in}}%
\pgfpathcurveto{\pgfqpoint{3.163730in}{2.659660in}}{\pgfqpoint{3.155830in}{2.662932in}}{\pgfqpoint{3.147593in}{2.662932in}}%
\pgfpathcurveto{\pgfqpoint{3.139357in}{2.662932in}}{\pgfqpoint{3.131457in}{2.659660in}}{\pgfqpoint{3.125633in}{2.653836in}}%
\pgfpathcurveto{\pgfqpoint{3.119809in}{2.648012in}}{\pgfqpoint{3.116537in}{2.640112in}}{\pgfqpoint{3.116537in}{2.631876in}}%
\pgfpathcurveto{\pgfqpoint{3.116537in}{2.623639in}}{\pgfqpoint{3.119809in}{2.615739in}}{\pgfqpoint{3.125633in}{2.609915in}}%
\pgfpathcurveto{\pgfqpoint{3.131457in}{2.604092in}}{\pgfqpoint{3.139357in}{2.600819in}}{\pgfqpoint{3.147593in}{2.600819in}}%
\pgfpathclose%
\pgfusepath{stroke,fill}%
\end{pgfscope}%
\begin{pgfscope}%
\pgfpathrectangle{\pgfqpoint{0.100000in}{0.220728in}}{\pgfqpoint{3.696000in}{3.696000in}}%
\pgfusepath{clip}%
\pgfsetbuttcap%
\pgfsetroundjoin%
\definecolor{currentfill}{rgb}{0.121569,0.466667,0.705882}%
\pgfsetfillcolor{currentfill}%
\pgfsetfillopacity{0.754993}%
\pgfsetlinewidth{1.003750pt}%
\definecolor{currentstroke}{rgb}{0.121569,0.466667,0.705882}%
\pgfsetstrokecolor{currentstroke}%
\pgfsetstrokeopacity{0.754993}%
\pgfsetdash{}{0pt}%
\pgfpathmoveto{\pgfqpoint{1.056653in}{2.107467in}}%
\pgfpathcurveto{\pgfqpoint{1.064889in}{2.107467in}}{\pgfqpoint{1.072789in}{2.110739in}}{\pgfqpoint{1.078613in}{2.116563in}}%
\pgfpathcurveto{\pgfqpoint{1.084437in}{2.122387in}}{\pgfqpoint{1.087709in}{2.130287in}}{\pgfqpoint{1.087709in}{2.138524in}}%
\pgfpathcurveto{\pgfqpoint{1.087709in}{2.146760in}}{\pgfqpoint{1.084437in}{2.154660in}}{\pgfqpoint{1.078613in}{2.160484in}}%
\pgfpathcurveto{\pgfqpoint{1.072789in}{2.166308in}}{\pgfqpoint{1.064889in}{2.169580in}}{\pgfqpoint{1.056653in}{2.169580in}}%
\pgfpathcurveto{\pgfqpoint{1.048416in}{2.169580in}}{\pgfqpoint{1.040516in}{2.166308in}}{\pgfqpoint{1.034692in}{2.160484in}}%
\pgfpathcurveto{\pgfqpoint{1.028868in}{2.154660in}}{\pgfqpoint{1.025596in}{2.146760in}}{\pgfqpoint{1.025596in}{2.138524in}}%
\pgfpathcurveto{\pgfqpoint{1.025596in}{2.130287in}}{\pgfqpoint{1.028868in}{2.122387in}}{\pgfqpoint{1.034692in}{2.116563in}}%
\pgfpathcurveto{\pgfqpoint{1.040516in}{2.110739in}}{\pgfqpoint{1.048416in}{2.107467in}}{\pgfqpoint{1.056653in}{2.107467in}}%
\pgfpathclose%
\pgfusepath{stroke,fill}%
\end{pgfscope}%
\begin{pgfscope}%
\pgfpathrectangle{\pgfqpoint{0.100000in}{0.220728in}}{\pgfqpoint{3.696000in}{3.696000in}}%
\pgfusepath{clip}%
\pgfsetbuttcap%
\pgfsetroundjoin%
\definecolor{currentfill}{rgb}{0.121569,0.466667,0.705882}%
\pgfsetfillcolor{currentfill}%
\pgfsetfillopacity{0.758293}%
\pgfsetlinewidth{1.003750pt}%
\definecolor{currentstroke}{rgb}{0.121569,0.466667,0.705882}%
\pgfsetstrokecolor{currentstroke}%
\pgfsetstrokeopacity{0.758293}%
\pgfsetdash{}{0pt}%
\pgfpathmoveto{\pgfqpoint{1.075398in}{2.101519in}}%
\pgfpathcurveto{\pgfqpoint{1.083634in}{2.101519in}}{\pgfqpoint{1.091534in}{2.104791in}}{\pgfqpoint{1.097358in}{2.110615in}}%
\pgfpathcurveto{\pgfqpoint{1.103182in}{2.116439in}}{\pgfqpoint{1.106454in}{2.124339in}}{\pgfqpoint{1.106454in}{2.132575in}}%
\pgfpathcurveto{\pgfqpoint{1.106454in}{2.140811in}}{\pgfqpoint{1.103182in}{2.148711in}}{\pgfqpoint{1.097358in}{2.154535in}}%
\pgfpathcurveto{\pgfqpoint{1.091534in}{2.160359in}}{\pgfqpoint{1.083634in}{2.163632in}}{\pgfqpoint{1.075398in}{2.163632in}}%
\pgfpathcurveto{\pgfqpoint{1.067161in}{2.163632in}}{\pgfqpoint{1.059261in}{2.160359in}}{\pgfqpoint{1.053437in}{2.154535in}}%
\pgfpathcurveto{\pgfqpoint{1.047614in}{2.148711in}}{\pgfqpoint{1.044341in}{2.140811in}}{\pgfqpoint{1.044341in}{2.132575in}}%
\pgfpathcurveto{\pgfqpoint{1.044341in}{2.124339in}}{\pgfqpoint{1.047614in}{2.116439in}}{\pgfqpoint{1.053437in}{2.110615in}}%
\pgfpathcurveto{\pgfqpoint{1.059261in}{2.104791in}}{\pgfqpoint{1.067161in}{2.101519in}}{\pgfqpoint{1.075398in}{2.101519in}}%
\pgfpathclose%
\pgfusepath{stroke,fill}%
\end{pgfscope}%
\begin{pgfscope}%
\pgfpathrectangle{\pgfqpoint{0.100000in}{0.220728in}}{\pgfqpoint{3.696000in}{3.696000in}}%
\pgfusepath{clip}%
\pgfsetbuttcap%
\pgfsetroundjoin%
\definecolor{currentfill}{rgb}{0.121569,0.466667,0.705882}%
\pgfsetfillcolor{currentfill}%
\pgfsetfillopacity{0.760074}%
\pgfsetlinewidth{1.003750pt}%
\definecolor{currentstroke}{rgb}{0.121569,0.466667,0.705882}%
\pgfsetstrokecolor{currentstroke}%
\pgfsetstrokeopacity{0.760074}%
\pgfsetdash{}{0pt}%
\pgfpathmoveto{\pgfqpoint{3.123285in}{2.563669in}}%
\pgfpathcurveto{\pgfqpoint{3.131522in}{2.563669in}}{\pgfqpoint{3.139422in}{2.566942in}}{\pgfqpoint{3.145246in}{2.572766in}}%
\pgfpathcurveto{\pgfqpoint{3.151070in}{2.578590in}}{\pgfqpoint{3.154342in}{2.586490in}}{\pgfqpoint{3.154342in}{2.594726in}}%
\pgfpathcurveto{\pgfqpoint{3.154342in}{2.602962in}}{\pgfqpoint{3.151070in}{2.610862in}}{\pgfqpoint{3.145246in}{2.616686in}}%
\pgfpathcurveto{\pgfqpoint{3.139422in}{2.622510in}}{\pgfqpoint{3.131522in}{2.625782in}}{\pgfqpoint{3.123285in}{2.625782in}}%
\pgfpathcurveto{\pgfqpoint{3.115049in}{2.625782in}}{\pgfqpoint{3.107149in}{2.622510in}}{\pgfqpoint{3.101325in}{2.616686in}}%
\pgfpathcurveto{\pgfqpoint{3.095501in}{2.610862in}}{\pgfqpoint{3.092229in}{2.602962in}}{\pgfqpoint{3.092229in}{2.594726in}}%
\pgfpathcurveto{\pgfqpoint{3.092229in}{2.586490in}}{\pgfqpoint{3.095501in}{2.578590in}}{\pgfqpoint{3.101325in}{2.572766in}}%
\pgfpathcurveto{\pgfqpoint{3.107149in}{2.566942in}}{\pgfqpoint{3.115049in}{2.563669in}}{\pgfqpoint{3.123285in}{2.563669in}}%
\pgfpathclose%
\pgfusepath{stroke,fill}%
\end{pgfscope}%
\begin{pgfscope}%
\pgfpathrectangle{\pgfqpoint{0.100000in}{0.220728in}}{\pgfqpoint{3.696000in}{3.696000in}}%
\pgfusepath{clip}%
\pgfsetbuttcap%
\pgfsetroundjoin%
\definecolor{currentfill}{rgb}{0.121569,0.466667,0.705882}%
\pgfsetfillcolor{currentfill}%
\pgfsetfillopacity{0.760529}%
\pgfsetlinewidth{1.003750pt}%
\definecolor{currentstroke}{rgb}{0.121569,0.466667,0.705882}%
\pgfsetstrokecolor{currentstroke}%
\pgfsetstrokeopacity{0.760529}%
\pgfsetdash{}{0pt}%
\pgfpathmoveto{\pgfqpoint{1.087690in}{2.094646in}}%
\pgfpathcurveto{\pgfqpoint{1.095927in}{2.094646in}}{\pgfqpoint{1.103827in}{2.097918in}}{\pgfqpoint{1.109651in}{2.103742in}}%
\pgfpathcurveto{\pgfqpoint{1.115475in}{2.109566in}}{\pgfqpoint{1.118747in}{2.117466in}}{\pgfqpoint{1.118747in}{2.125702in}}%
\pgfpathcurveto{\pgfqpoint{1.118747in}{2.133938in}}{\pgfqpoint{1.115475in}{2.141838in}}{\pgfqpoint{1.109651in}{2.147662in}}%
\pgfpathcurveto{\pgfqpoint{1.103827in}{2.153486in}}{\pgfqpoint{1.095927in}{2.156759in}}{\pgfqpoint{1.087690in}{2.156759in}}%
\pgfpathcurveto{\pgfqpoint{1.079454in}{2.156759in}}{\pgfqpoint{1.071554in}{2.153486in}}{\pgfqpoint{1.065730in}{2.147662in}}%
\pgfpathcurveto{\pgfqpoint{1.059906in}{2.141838in}}{\pgfqpoint{1.056634in}{2.133938in}}{\pgfqpoint{1.056634in}{2.125702in}}%
\pgfpathcurveto{\pgfqpoint{1.056634in}{2.117466in}}{\pgfqpoint{1.059906in}{2.109566in}}{\pgfqpoint{1.065730in}{2.103742in}}%
\pgfpathcurveto{\pgfqpoint{1.071554in}{2.097918in}}{\pgfqpoint{1.079454in}{2.094646in}}{\pgfqpoint{1.087690in}{2.094646in}}%
\pgfpathclose%
\pgfusepath{stroke,fill}%
\end{pgfscope}%
\begin{pgfscope}%
\pgfpathrectangle{\pgfqpoint{0.100000in}{0.220728in}}{\pgfqpoint{3.696000in}{3.696000in}}%
\pgfusepath{clip}%
\pgfsetbuttcap%
\pgfsetroundjoin%
\definecolor{currentfill}{rgb}{0.121569,0.466667,0.705882}%
\pgfsetfillcolor{currentfill}%
\pgfsetfillopacity{0.762062}%
\pgfsetlinewidth{1.003750pt}%
\definecolor{currentstroke}{rgb}{0.121569,0.466667,0.705882}%
\pgfsetstrokecolor{currentstroke}%
\pgfsetstrokeopacity{0.762062}%
\pgfsetdash{}{0pt}%
\pgfpathmoveto{\pgfqpoint{1.094792in}{2.090292in}}%
\pgfpathcurveto{\pgfqpoint{1.103028in}{2.090292in}}{\pgfqpoint{1.110928in}{2.093564in}}{\pgfqpoint{1.116752in}{2.099388in}}%
\pgfpathcurveto{\pgfqpoint{1.122576in}{2.105212in}}{\pgfqpoint{1.125848in}{2.113112in}}{\pgfqpoint{1.125848in}{2.121348in}}%
\pgfpathcurveto{\pgfqpoint{1.125848in}{2.129584in}}{\pgfqpoint{1.122576in}{2.137485in}}{\pgfqpoint{1.116752in}{2.143308in}}%
\pgfpathcurveto{\pgfqpoint{1.110928in}{2.149132in}}{\pgfqpoint{1.103028in}{2.152405in}}{\pgfqpoint{1.094792in}{2.152405in}}%
\pgfpathcurveto{\pgfqpoint{1.086555in}{2.152405in}}{\pgfqpoint{1.078655in}{2.149132in}}{\pgfqpoint{1.072831in}{2.143308in}}%
\pgfpathcurveto{\pgfqpoint{1.067007in}{2.137485in}}{\pgfqpoint{1.063735in}{2.129584in}}{\pgfqpoint{1.063735in}{2.121348in}}%
\pgfpathcurveto{\pgfqpoint{1.063735in}{2.113112in}}{\pgfqpoint{1.067007in}{2.105212in}}{\pgfqpoint{1.072831in}{2.099388in}}%
\pgfpathcurveto{\pgfqpoint{1.078655in}{2.093564in}}{\pgfqpoint{1.086555in}{2.090292in}}{\pgfqpoint{1.094792in}{2.090292in}}%
\pgfpathclose%
\pgfusepath{stroke,fill}%
\end{pgfscope}%
\begin{pgfscope}%
\pgfpathrectangle{\pgfqpoint{0.100000in}{0.220728in}}{\pgfqpoint{3.696000in}{3.696000in}}%
\pgfusepath{clip}%
\pgfsetbuttcap%
\pgfsetroundjoin%
\definecolor{currentfill}{rgb}{0.121569,0.466667,0.705882}%
\pgfsetfillcolor{currentfill}%
\pgfsetfillopacity{0.762769}%
\pgfsetlinewidth{1.003750pt}%
\definecolor{currentstroke}{rgb}{0.121569,0.466667,0.705882}%
\pgfsetstrokecolor{currentstroke}%
\pgfsetstrokeopacity{0.762769}%
\pgfsetdash{}{0pt}%
\pgfpathmoveto{\pgfqpoint{1.098355in}{2.088691in}}%
\pgfpathcurveto{\pgfqpoint{1.106592in}{2.088691in}}{\pgfqpoint{1.114492in}{2.091964in}}{\pgfqpoint{1.120316in}{2.097788in}}%
\pgfpathcurveto{\pgfqpoint{1.126139in}{2.103612in}}{\pgfqpoint{1.129412in}{2.111512in}}{\pgfqpoint{1.129412in}{2.119748in}}%
\pgfpathcurveto{\pgfqpoint{1.129412in}{2.127984in}}{\pgfqpoint{1.126139in}{2.135884in}}{\pgfqpoint{1.120316in}{2.141708in}}%
\pgfpathcurveto{\pgfqpoint{1.114492in}{2.147532in}}{\pgfqpoint{1.106592in}{2.150804in}}{\pgfqpoint{1.098355in}{2.150804in}}%
\pgfpathcurveto{\pgfqpoint{1.090119in}{2.150804in}}{\pgfqpoint{1.082219in}{2.147532in}}{\pgfqpoint{1.076395in}{2.141708in}}%
\pgfpathcurveto{\pgfqpoint{1.070571in}{2.135884in}}{\pgfqpoint{1.067299in}{2.127984in}}{\pgfqpoint{1.067299in}{2.119748in}}%
\pgfpathcurveto{\pgfqpoint{1.067299in}{2.111512in}}{\pgfqpoint{1.070571in}{2.103612in}}{\pgfqpoint{1.076395in}{2.097788in}}%
\pgfpathcurveto{\pgfqpoint{1.082219in}{2.091964in}}{\pgfqpoint{1.090119in}{2.088691in}}{\pgfqpoint{1.098355in}{2.088691in}}%
\pgfpathclose%
\pgfusepath{stroke,fill}%
\end{pgfscope}%
\begin{pgfscope}%
\pgfpathrectangle{\pgfqpoint{0.100000in}{0.220728in}}{\pgfqpoint{3.696000in}{3.696000in}}%
\pgfusepath{clip}%
\pgfsetbuttcap%
\pgfsetroundjoin%
\definecolor{currentfill}{rgb}{0.121569,0.466667,0.705882}%
\pgfsetfillcolor{currentfill}%
\pgfsetfillopacity{0.763747}%
\pgfsetlinewidth{1.003750pt}%
\definecolor{currentstroke}{rgb}{0.121569,0.466667,0.705882}%
\pgfsetstrokecolor{currentstroke}%
\pgfsetstrokeopacity{0.763747}%
\pgfsetdash{}{0pt}%
\pgfpathmoveto{\pgfqpoint{3.109113in}{2.543924in}}%
\pgfpathcurveto{\pgfqpoint{3.117349in}{2.543924in}}{\pgfqpoint{3.125249in}{2.547197in}}{\pgfqpoint{3.131073in}{2.553020in}}%
\pgfpathcurveto{\pgfqpoint{3.136897in}{2.558844in}}{\pgfqpoint{3.140169in}{2.566744in}}{\pgfqpoint{3.140169in}{2.574981in}}%
\pgfpathcurveto{\pgfqpoint{3.140169in}{2.583217in}}{\pgfqpoint{3.136897in}{2.591117in}}{\pgfqpoint{3.131073in}{2.596941in}}%
\pgfpathcurveto{\pgfqpoint{3.125249in}{2.602765in}}{\pgfqpoint{3.117349in}{2.606037in}}{\pgfqpoint{3.109113in}{2.606037in}}%
\pgfpathcurveto{\pgfqpoint{3.100876in}{2.606037in}}{\pgfqpoint{3.092976in}{2.602765in}}{\pgfqpoint{3.087152in}{2.596941in}}%
\pgfpathcurveto{\pgfqpoint{3.081328in}{2.591117in}}{\pgfqpoint{3.078056in}{2.583217in}}{\pgfqpoint{3.078056in}{2.574981in}}%
\pgfpathcurveto{\pgfqpoint{3.078056in}{2.566744in}}{\pgfqpoint{3.081328in}{2.558844in}}{\pgfqpoint{3.087152in}{2.553020in}}%
\pgfpathcurveto{\pgfqpoint{3.092976in}{2.547197in}}{\pgfqpoint{3.100876in}{2.543924in}}{\pgfqpoint{3.109113in}{2.543924in}}%
\pgfpathclose%
\pgfusepath{stroke,fill}%
\end{pgfscope}%
\begin{pgfscope}%
\pgfpathrectangle{\pgfqpoint{0.100000in}{0.220728in}}{\pgfqpoint{3.696000in}{3.696000in}}%
\pgfusepath{clip}%
\pgfsetbuttcap%
\pgfsetroundjoin%
\definecolor{currentfill}{rgb}{0.121569,0.466667,0.705882}%
\pgfsetfillcolor{currentfill}%
\pgfsetfillopacity{0.764057}%
\pgfsetlinewidth{1.003750pt}%
\definecolor{currentstroke}{rgb}{0.121569,0.466667,0.705882}%
\pgfsetstrokecolor{currentstroke}%
\pgfsetstrokeopacity{0.764057}%
\pgfsetdash{}{0pt}%
\pgfpathmoveto{\pgfqpoint{1.105155in}{2.086932in}}%
\pgfpathcurveto{\pgfqpoint{1.113392in}{2.086932in}}{\pgfqpoint{1.121292in}{2.090205in}}{\pgfqpoint{1.127116in}{2.096029in}}%
\pgfpathcurveto{\pgfqpoint{1.132939in}{2.101852in}}{\pgfqpoint{1.136212in}{2.109753in}}{\pgfqpoint{1.136212in}{2.117989in}}%
\pgfpathcurveto{\pgfqpoint{1.136212in}{2.126225in}}{\pgfqpoint{1.132939in}{2.134125in}}{\pgfqpoint{1.127116in}{2.139949in}}%
\pgfpathcurveto{\pgfqpoint{1.121292in}{2.145773in}}{\pgfqpoint{1.113392in}{2.149045in}}{\pgfqpoint{1.105155in}{2.149045in}}%
\pgfpathcurveto{\pgfqpoint{1.096919in}{2.149045in}}{\pgfqpoint{1.089019in}{2.145773in}}{\pgfqpoint{1.083195in}{2.139949in}}%
\pgfpathcurveto{\pgfqpoint{1.077371in}{2.134125in}}{\pgfqpoint{1.074099in}{2.126225in}}{\pgfqpoint{1.074099in}{2.117989in}}%
\pgfpathcurveto{\pgfqpoint{1.074099in}{2.109753in}}{\pgfqpoint{1.077371in}{2.101852in}}{\pgfqpoint{1.083195in}{2.096029in}}%
\pgfpathcurveto{\pgfqpoint{1.089019in}{2.090205in}}{\pgfqpoint{1.096919in}{2.086932in}}{\pgfqpoint{1.105155in}{2.086932in}}%
\pgfpathclose%
\pgfusepath{stroke,fill}%
\end{pgfscope}%
\begin{pgfscope}%
\pgfpathrectangle{\pgfqpoint{0.100000in}{0.220728in}}{\pgfqpoint{3.696000in}{3.696000in}}%
\pgfusepath{clip}%
\pgfsetbuttcap%
\pgfsetroundjoin%
\definecolor{currentfill}{rgb}{0.121569,0.466667,0.705882}%
\pgfsetfillcolor{currentfill}%
\pgfsetfillopacity{0.766108}%
\pgfsetlinewidth{1.003750pt}%
\definecolor{currentstroke}{rgb}{0.121569,0.466667,0.705882}%
\pgfsetstrokecolor{currentstroke}%
\pgfsetstrokeopacity{0.766108}%
\pgfsetdash{}{0pt}%
\pgfpathmoveto{\pgfqpoint{3.102056in}{2.533676in}}%
\pgfpathcurveto{\pgfqpoint{3.110292in}{2.533676in}}{\pgfqpoint{3.118192in}{2.536948in}}{\pgfqpoint{3.124016in}{2.542772in}}%
\pgfpathcurveto{\pgfqpoint{3.129840in}{2.548596in}}{\pgfqpoint{3.133112in}{2.556496in}}{\pgfqpoint{3.133112in}{2.564732in}}%
\pgfpathcurveto{\pgfqpoint{3.133112in}{2.572969in}}{\pgfqpoint{3.129840in}{2.580869in}}{\pgfqpoint{3.124016in}{2.586693in}}%
\pgfpathcurveto{\pgfqpoint{3.118192in}{2.592517in}}{\pgfqpoint{3.110292in}{2.595789in}}{\pgfqpoint{3.102056in}{2.595789in}}%
\pgfpathcurveto{\pgfqpoint{3.093819in}{2.595789in}}{\pgfqpoint{3.085919in}{2.592517in}}{\pgfqpoint{3.080095in}{2.586693in}}%
\pgfpathcurveto{\pgfqpoint{3.074271in}{2.580869in}}{\pgfqpoint{3.070999in}{2.572969in}}{\pgfqpoint{3.070999in}{2.564732in}}%
\pgfpathcurveto{\pgfqpoint{3.070999in}{2.556496in}}{\pgfqpoint{3.074271in}{2.548596in}}{\pgfqpoint{3.080095in}{2.542772in}}%
\pgfpathcurveto{\pgfqpoint{3.085919in}{2.536948in}}{\pgfqpoint{3.093819in}{2.533676in}}{\pgfqpoint{3.102056in}{2.533676in}}%
\pgfpathclose%
\pgfusepath{stroke,fill}%
\end{pgfscope}%
\begin{pgfscope}%
\pgfpathrectangle{\pgfqpoint{0.100000in}{0.220728in}}{\pgfqpoint{3.696000in}{3.696000in}}%
\pgfusepath{clip}%
\pgfsetbuttcap%
\pgfsetroundjoin%
\definecolor{currentfill}{rgb}{0.121569,0.466667,0.705882}%
\pgfsetfillcolor{currentfill}%
\pgfsetfillopacity{0.766182}%
\pgfsetlinewidth{1.003750pt}%
\definecolor{currentstroke}{rgb}{0.121569,0.466667,0.705882}%
\pgfsetstrokecolor{currentstroke}%
\pgfsetstrokeopacity{0.766182}%
\pgfsetdash{}{0pt}%
\pgfpathmoveto{\pgfqpoint{1.117221in}{2.081598in}}%
\pgfpathcurveto{\pgfqpoint{1.125458in}{2.081598in}}{\pgfqpoint{1.133358in}{2.084871in}}{\pgfqpoint{1.139182in}{2.090694in}}%
\pgfpathcurveto{\pgfqpoint{1.145006in}{2.096518in}}{\pgfqpoint{1.148278in}{2.104418in}}{\pgfqpoint{1.148278in}{2.112655in}}%
\pgfpathcurveto{\pgfqpoint{1.148278in}{2.120891in}}{\pgfqpoint{1.145006in}{2.128791in}}{\pgfqpoint{1.139182in}{2.134615in}}%
\pgfpathcurveto{\pgfqpoint{1.133358in}{2.140439in}}{\pgfqpoint{1.125458in}{2.143711in}}{\pgfqpoint{1.117221in}{2.143711in}}%
\pgfpathcurveto{\pgfqpoint{1.108985in}{2.143711in}}{\pgfqpoint{1.101085in}{2.140439in}}{\pgfqpoint{1.095261in}{2.134615in}}%
\pgfpathcurveto{\pgfqpoint{1.089437in}{2.128791in}}{\pgfqpoint{1.086165in}{2.120891in}}{\pgfqpoint{1.086165in}{2.112655in}}%
\pgfpathcurveto{\pgfqpoint{1.086165in}{2.104418in}}{\pgfqpoint{1.089437in}{2.096518in}}{\pgfqpoint{1.095261in}{2.090694in}}%
\pgfpathcurveto{\pgfqpoint{1.101085in}{2.084871in}}{\pgfqpoint{1.108985in}{2.081598in}}{\pgfqpoint{1.117221in}{2.081598in}}%
\pgfpathclose%
\pgfusepath{stroke,fill}%
\end{pgfscope}%
\begin{pgfscope}%
\pgfpathrectangle{\pgfqpoint{0.100000in}{0.220728in}}{\pgfqpoint{3.696000in}{3.696000in}}%
\pgfusepath{clip}%
\pgfsetbuttcap%
\pgfsetroundjoin%
\definecolor{currentfill}{rgb}{0.121569,0.466667,0.705882}%
\pgfsetfillcolor{currentfill}%
\pgfsetfillopacity{0.767253}%
\pgfsetlinewidth{1.003750pt}%
\definecolor{currentstroke}{rgb}{0.121569,0.466667,0.705882}%
\pgfsetstrokecolor{currentstroke}%
\pgfsetstrokeopacity{0.767253}%
\pgfsetdash{}{0pt}%
\pgfpathmoveto{\pgfqpoint{3.097962in}{2.527545in}}%
\pgfpathcurveto{\pgfqpoint{3.106198in}{2.527545in}}{\pgfqpoint{3.114098in}{2.530817in}}{\pgfqpoint{3.119922in}{2.536641in}}%
\pgfpathcurveto{\pgfqpoint{3.125746in}{2.542465in}}{\pgfqpoint{3.129019in}{2.550365in}}{\pgfqpoint{3.129019in}{2.558601in}}%
\pgfpathcurveto{\pgfqpoint{3.129019in}{2.566837in}}{\pgfqpoint{3.125746in}{2.574737in}}{\pgfqpoint{3.119922in}{2.580561in}}%
\pgfpathcurveto{\pgfqpoint{3.114098in}{2.586385in}}{\pgfqpoint{3.106198in}{2.589658in}}{\pgfqpoint{3.097962in}{2.589658in}}%
\pgfpathcurveto{\pgfqpoint{3.089726in}{2.589658in}}{\pgfqpoint{3.081826in}{2.586385in}}{\pgfqpoint{3.076002in}{2.580561in}}%
\pgfpathcurveto{\pgfqpoint{3.070178in}{2.574737in}}{\pgfqpoint{3.066906in}{2.566837in}}{\pgfqpoint{3.066906in}{2.558601in}}%
\pgfpathcurveto{\pgfqpoint{3.066906in}{2.550365in}}{\pgfqpoint{3.070178in}{2.542465in}}{\pgfqpoint{3.076002in}{2.536641in}}%
\pgfpathcurveto{\pgfqpoint{3.081826in}{2.530817in}}{\pgfqpoint{3.089726in}{2.527545in}}{\pgfqpoint{3.097962in}{2.527545in}}%
\pgfpathclose%
\pgfusepath{stroke,fill}%
\end{pgfscope}%
\begin{pgfscope}%
\pgfpathrectangle{\pgfqpoint{0.100000in}{0.220728in}}{\pgfqpoint{3.696000in}{3.696000in}}%
\pgfusepath{clip}%
\pgfsetbuttcap%
\pgfsetroundjoin%
\definecolor{currentfill}{rgb}{0.121569,0.466667,0.705882}%
\pgfsetfillcolor{currentfill}%
\pgfsetfillopacity{0.767912}%
\pgfsetlinewidth{1.003750pt}%
\definecolor{currentstroke}{rgb}{0.121569,0.466667,0.705882}%
\pgfsetstrokecolor{currentstroke}%
\pgfsetstrokeopacity{0.767912}%
\pgfsetdash{}{0pt}%
\pgfpathmoveto{\pgfqpoint{3.095654in}{2.524409in}}%
\pgfpathcurveto{\pgfqpoint{3.103890in}{2.524409in}}{\pgfqpoint{3.111790in}{2.527681in}}{\pgfqpoint{3.117614in}{2.533505in}}%
\pgfpathcurveto{\pgfqpoint{3.123438in}{2.539329in}}{\pgfqpoint{3.126710in}{2.547229in}}{\pgfqpoint{3.126710in}{2.555465in}}%
\pgfpathcurveto{\pgfqpoint{3.126710in}{2.563701in}}{\pgfqpoint{3.123438in}{2.571601in}}{\pgfqpoint{3.117614in}{2.577425in}}%
\pgfpathcurveto{\pgfqpoint{3.111790in}{2.583249in}}{\pgfqpoint{3.103890in}{2.586522in}}{\pgfqpoint{3.095654in}{2.586522in}}%
\pgfpathcurveto{\pgfqpoint{3.087417in}{2.586522in}}{\pgfqpoint{3.079517in}{2.583249in}}{\pgfqpoint{3.073693in}{2.577425in}}%
\pgfpathcurveto{\pgfqpoint{3.067869in}{2.571601in}}{\pgfqpoint{3.064597in}{2.563701in}}{\pgfqpoint{3.064597in}{2.555465in}}%
\pgfpathcurveto{\pgfqpoint{3.064597in}{2.547229in}}{\pgfqpoint{3.067869in}{2.539329in}}{\pgfqpoint{3.073693in}{2.533505in}}%
\pgfpathcurveto{\pgfqpoint{3.079517in}{2.527681in}}{\pgfqpoint{3.087417in}{2.524409in}}{\pgfqpoint{3.095654in}{2.524409in}}%
\pgfpathclose%
\pgfusepath{stroke,fill}%
\end{pgfscope}%
\begin{pgfscope}%
\pgfpathrectangle{\pgfqpoint{0.100000in}{0.220728in}}{\pgfqpoint{3.696000in}{3.696000in}}%
\pgfusepath{clip}%
\pgfsetbuttcap%
\pgfsetroundjoin%
\definecolor{currentfill}{rgb}{0.121569,0.466667,0.705882}%
\pgfsetfillcolor{currentfill}%
\pgfsetfillopacity{0.768284}%
\pgfsetlinewidth{1.003750pt}%
\definecolor{currentstroke}{rgb}{0.121569,0.466667,0.705882}%
\pgfsetstrokecolor{currentstroke}%
\pgfsetstrokeopacity{0.768284}%
\pgfsetdash{}{0pt}%
\pgfpathmoveto{\pgfqpoint{3.094457in}{2.522621in}}%
\pgfpathcurveto{\pgfqpoint{3.102693in}{2.522621in}}{\pgfqpoint{3.110593in}{2.525893in}}{\pgfqpoint{3.116417in}{2.531717in}}%
\pgfpathcurveto{\pgfqpoint{3.122241in}{2.537541in}}{\pgfqpoint{3.125513in}{2.545441in}}{\pgfqpoint{3.125513in}{2.553677in}}%
\pgfpathcurveto{\pgfqpoint{3.125513in}{2.561914in}}{\pgfqpoint{3.122241in}{2.569814in}}{\pgfqpoint{3.116417in}{2.575637in}}%
\pgfpathcurveto{\pgfqpoint{3.110593in}{2.581461in}}{\pgfqpoint{3.102693in}{2.584734in}}{\pgfqpoint{3.094457in}{2.584734in}}%
\pgfpathcurveto{\pgfqpoint{3.086220in}{2.584734in}}{\pgfqpoint{3.078320in}{2.581461in}}{\pgfqpoint{3.072497in}{2.575637in}}%
\pgfpathcurveto{\pgfqpoint{3.066673in}{2.569814in}}{\pgfqpoint{3.063400in}{2.561914in}}{\pgfqpoint{3.063400in}{2.553677in}}%
\pgfpathcurveto{\pgfqpoint{3.063400in}{2.545441in}}{\pgfqpoint{3.066673in}{2.537541in}}{\pgfqpoint{3.072497in}{2.531717in}}%
\pgfpathcurveto{\pgfqpoint{3.078320in}{2.525893in}}{\pgfqpoint{3.086220in}{2.522621in}}{\pgfqpoint{3.094457in}{2.522621in}}%
\pgfpathclose%
\pgfusepath{stroke,fill}%
\end{pgfscope}%
\begin{pgfscope}%
\pgfpathrectangle{\pgfqpoint{0.100000in}{0.220728in}}{\pgfqpoint{3.696000in}{3.696000in}}%
\pgfusepath{clip}%
\pgfsetbuttcap%
\pgfsetroundjoin%
\definecolor{currentfill}{rgb}{0.121569,0.466667,0.705882}%
\pgfsetfillcolor{currentfill}%
\pgfsetfillopacity{0.768462}%
\pgfsetlinewidth{1.003750pt}%
\definecolor{currentstroke}{rgb}{0.121569,0.466667,0.705882}%
\pgfsetstrokecolor{currentstroke}%
\pgfsetstrokeopacity{0.768462}%
\pgfsetdash{}{0pt}%
\pgfpathmoveto{\pgfqpoint{3.093676in}{2.521706in}}%
\pgfpathcurveto{\pgfqpoint{3.101912in}{2.521706in}}{\pgfqpoint{3.109812in}{2.524978in}}{\pgfqpoint{3.115636in}{2.530802in}}%
\pgfpathcurveto{\pgfqpoint{3.121460in}{2.536626in}}{\pgfqpoint{3.124732in}{2.544526in}}{\pgfqpoint{3.124732in}{2.552762in}}%
\pgfpathcurveto{\pgfqpoint{3.124732in}{2.560999in}}{\pgfqpoint{3.121460in}{2.568899in}}{\pgfqpoint{3.115636in}{2.574723in}}%
\pgfpathcurveto{\pgfqpoint{3.109812in}{2.580546in}}{\pgfqpoint{3.101912in}{2.583819in}}{\pgfqpoint{3.093676in}{2.583819in}}%
\pgfpathcurveto{\pgfqpoint{3.085440in}{2.583819in}}{\pgfqpoint{3.077539in}{2.580546in}}{\pgfqpoint{3.071716in}{2.574723in}}%
\pgfpathcurveto{\pgfqpoint{3.065892in}{2.568899in}}{\pgfqpoint{3.062619in}{2.560999in}}{\pgfqpoint{3.062619in}{2.552762in}}%
\pgfpathcurveto{\pgfqpoint{3.062619in}{2.544526in}}{\pgfqpoint{3.065892in}{2.536626in}}{\pgfqpoint{3.071716in}{2.530802in}}%
\pgfpathcurveto{\pgfqpoint{3.077539in}{2.524978in}}{\pgfqpoint{3.085440in}{2.521706in}}{\pgfqpoint{3.093676in}{2.521706in}}%
\pgfpathclose%
\pgfusepath{stroke,fill}%
\end{pgfscope}%
\begin{pgfscope}%
\pgfpathrectangle{\pgfqpoint{0.100000in}{0.220728in}}{\pgfqpoint{3.696000in}{3.696000in}}%
\pgfusepath{clip}%
\pgfsetbuttcap%
\pgfsetroundjoin%
\definecolor{currentfill}{rgb}{0.121569,0.466667,0.705882}%
\pgfsetfillcolor{currentfill}%
\pgfsetfillopacity{0.768565}%
\pgfsetlinewidth{1.003750pt}%
\definecolor{currentstroke}{rgb}{0.121569,0.466667,0.705882}%
\pgfsetstrokecolor{currentstroke}%
\pgfsetstrokeopacity{0.768565}%
\pgfsetdash{}{0pt}%
\pgfpathmoveto{\pgfqpoint{3.093293in}{2.521147in}}%
\pgfpathcurveto{\pgfqpoint{3.101529in}{2.521147in}}{\pgfqpoint{3.109429in}{2.524419in}}{\pgfqpoint{3.115253in}{2.530243in}}%
\pgfpathcurveto{\pgfqpoint{3.121077in}{2.536067in}}{\pgfqpoint{3.124350in}{2.543967in}}{\pgfqpoint{3.124350in}{2.552203in}}%
\pgfpathcurveto{\pgfqpoint{3.124350in}{2.560440in}}{\pgfqpoint{3.121077in}{2.568340in}}{\pgfqpoint{3.115253in}{2.574164in}}%
\pgfpathcurveto{\pgfqpoint{3.109429in}{2.579988in}}{\pgfqpoint{3.101529in}{2.583260in}}{\pgfqpoint{3.093293in}{2.583260in}}%
\pgfpathcurveto{\pgfqpoint{3.085057in}{2.583260in}}{\pgfqpoint{3.077157in}{2.579988in}}{\pgfqpoint{3.071333in}{2.574164in}}%
\pgfpathcurveto{\pgfqpoint{3.065509in}{2.568340in}}{\pgfqpoint{3.062237in}{2.560440in}}{\pgfqpoint{3.062237in}{2.552203in}}%
\pgfpathcurveto{\pgfqpoint{3.062237in}{2.543967in}}{\pgfqpoint{3.065509in}{2.536067in}}{\pgfqpoint{3.071333in}{2.530243in}}%
\pgfpathcurveto{\pgfqpoint{3.077157in}{2.524419in}}{\pgfqpoint{3.085057in}{2.521147in}}{\pgfqpoint{3.093293in}{2.521147in}}%
\pgfpathclose%
\pgfusepath{stroke,fill}%
\end{pgfscope}%
\begin{pgfscope}%
\pgfpathrectangle{\pgfqpoint{0.100000in}{0.220728in}}{\pgfqpoint{3.696000in}{3.696000in}}%
\pgfusepath{clip}%
\pgfsetbuttcap%
\pgfsetroundjoin%
\definecolor{currentfill}{rgb}{0.121569,0.466667,0.705882}%
\pgfsetfillcolor{currentfill}%
\pgfsetfillopacity{0.768626}%
\pgfsetlinewidth{1.003750pt}%
\definecolor{currentstroke}{rgb}{0.121569,0.466667,0.705882}%
\pgfsetstrokecolor{currentstroke}%
\pgfsetstrokeopacity{0.768626}%
\pgfsetdash{}{0pt}%
\pgfpathmoveto{\pgfqpoint{3.093065in}{2.520890in}}%
\pgfpathcurveto{\pgfqpoint{3.101302in}{2.520890in}}{\pgfqpoint{3.109202in}{2.524162in}}{\pgfqpoint{3.115026in}{2.529986in}}%
\pgfpathcurveto{\pgfqpoint{3.120850in}{2.535810in}}{\pgfqpoint{3.124122in}{2.543710in}}{\pgfqpoint{3.124122in}{2.551946in}}%
\pgfpathcurveto{\pgfqpoint{3.124122in}{2.560183in}}{\pgfqpoint{3.120850in}{2.568083in}}{\pgfqpoint{3.115026in}{2.573907in}}%
\pgfpathcurveto{\pgfqpoint{3.109202in}{2.579731in}}{\pgfqpoint{3.101302in}{2.583003in}}{\pgfqpoint{3.093065in}{2.583003in}}%
\pgfpathcurveto{\pgfqpoint{3.084829in}{2.583003in}}{\pgfqpoint{3.076929in}{2.579731in}}{\pgfqpoint{3.071105in}{2.573907in}}%
\pgfpathcurveto{\pgfqpoint{3.065281in}{2.568083in}}{\pgfqpoint{3.062009in}{2.560183in}}{\pgfqpoint{3.062009in}{2.551946in}}%
\pgfpathcurveto{\pgfqpoint{3.062009in}{2.543710in}}{\pgfqpoint{3.065281in}{2.535810in}}{\pgfqpoint{3.071105in}{2.529986in}}%
\pgfpathcurveto{\pgfqpoint{3.076929in}{2.524162in}}{\pgfqpoint{3.084829in}{2.520890in}}{\pgfqpoint{3.093065in}{2.520890in}}%
\pgfpathclose%
\pgfusepath{stroke,fill}%
\end{pgfscope}%
\begin{pgfscope}%
\pgfpathrectangle{\pgfqpoint{0.100000in}{0.220728in}}{\pgfqpoint{3.696000in}{3.696000in}}%
\pgfusepath{clip}%
\pgfsetbuttcap%
\pgfsetroundjoin%
\definecolor{currentfill}{rgb}{0.121569,0.466667,0.705882}%
\pgfsetfillcolor{currentfill}%
\pgfsetfillopacity{0.769318}%
\pgfsetlinewidth{1.003750pt}%
\definecolor{currentstroke}{rgb}{0.121569,0.466667,0.705882}%
\pgfsetstrokecolor{currentstroke}%
\pgfsetstrokeopacity{0.769318}%
\pgfsetdash{}{0pt}%
\pgfpathmoveto{\pgfqpoint{3.090934in}{2.517199in}}%
\pgfpathcurveto{\pgfqpoint{3.099170in}{2.517199in}}{\pgfqpoint{3.107070in}{2.520471in}}{\pgfqpoint{3.112894in}{2.526295in}}%
\pgfpathcurveto{\pgfqpoint{3.118718in}{2.532119in}}{\pgfqpoint{3.121990in}{2.540019in}}{\pgfqpoint{3.121990in}{2.548256in}}%
\pgfpathcurveto{\pgfqpoint{3.121990in}{2.556492in}}{\pgfqpoint{3.118718in}{2.564392in}}{\pgfqpoint{3.112894in}{2.570216in}}%
\pgfpathcurveto{\pgfqpoint{3.107070in}{2.576040in}}{\pgfqpoint{3.099170in}{2.579312in}}{\pgfqpoint{3.090934in}{2.579312in}}%
\pgfpathcurveto{\pgfqpoint{3.082697in}{2.579312in}}{\pgfqpoint{3.074797in}{2.576040in}}{\pgfqpoint{3.068974in}{2.570216in}}%
\pgfpathcurveto{\pgfqpoint{3.063150in}{2.564392in}}{\pgfqpoint{3.059877in}{2.556492in}}{\pgfqpoint{3.059877in}{2.548256in}}%
\pgfpathcurveto{\pgfqpoint{3.059877in}{2.540019in}}{\pgfqpoint{3.063150in}{2.532119in}}{\pgfqpoint{3.068974in}{2.526295in}}%
\pgfpathcurveto{\pgfqpoint{3.074797in}{2.520471in}}{\pgfqpoint{3.082697in}{2.517199in}}{\pgfqpoint{3.090934in}{2.517199in}}%
\pgfpathclose%
\pgfusepath{stroke,fill}%
\end{pgfscope}%
\begin{pgfscope}%
\pgfpathrectangle{\pgfqpoint{0.100000in}{0.220728in}}{\pgfqpoint{3.696000in}{3.696000in}}%
\pgfusepath{clip}%
\pgfsetbuttcap%
\pgfsetroundjoin%
\definecolor{currentfill}{rgb}{0.121569,0.466667,0.705882}%
\pgfsetfillcolor{currentfill}%
\pgfsetfillopacity{0.770074}%
\pgfsetlinewidth{1.003750pt}%
\definecolor{currentstroke}{rgb}{0.121569,0.466667,0.705882}%
\pgfsetstrokecolor{currentstroke}%
\pgfsetstrokeopacity{0.770074}%
\pgfsetdash{}{0pt}%
\pgfpathmoveto{\pgfqpoint{1.138150in}{2.068882in}}%
\pgfpathcurveto{\pgfqpoint{1.146386in}{2.068882in}}{\pgfqpoint{1.154286in}{2.072154in}}{\pgfqpoint{1.160110in}{2.077978in}}%
\pgfpathcurveto{\pgfqpoint{1.165934in}{2.083802in}}{\pgfqpoint{1.169206in}{2.091702in}}{\pgfqpoint{1.169206in}{2.099938in}}%
\pgfpathcurveto{\pgfqpoint{1.169206in}{2.108175in}}{\pgfqpoint{1.165934in}{2.116075in}}{\pgfqpoint{1.160110in}{2.121899in}}%
\pgfpathcurveto{\pgfqpoint{1.154286in}{2.127723in}}{\pgfqpoint{1.146386in}{2.130995in}}{\pgfqpoint{1.138150in}{2.130995in}}%
\pgfpathcurveto{\pgfqpoint{1.129914in}{2.130995in}}{\pgfqpoint{1.122014in}{2.127723in}}{\pgfqpoint{1.116190in}{2.121899in}}%
\pgfpathcurveto{\pgfqpoint{1.110366in}{2.116075in}}{\pgfqpoint{1.107093in}{2.108175in}}{\pgfqpoint{1.107093in}{2.099938in}}%
\pgfpathcurveto{\pgfqpoint{1.107093in}{2.091702in}}{\pgfqpoint{1.110366in}{2.083802in}}{\pgfqpoint{1.116190in}{2.077978in}}%
\pgfpathcurveto{\pgfqpoint{1.122014in}{2.072154in}}{\pgfqpoint{1.129914in}{2.068882in}}{\pgfqpoint{1.138150in}{2.068882in}}%
\pgfpathclose%
\pgfusepath{stroke,fill}%
\end{pgfscope}%
\begin{pgfscope}%
\pgfpathrectangle{\pgfqpoint{0.100000in}{0.220728in}}{\pgfqpoint{3.696000in}{3.696000in}}%
\pgfusepath{clip}%
\pgfsetbuttcap%
\pgfsetroundjoin%
\definecolor{currentfill}{rgb}{0.121569,0.466667,0.705882}%
\pgfsetfillcolor{currentfill}%
\pgfsetfillopacity{0.770809}%
\pgfsetlinewidth{1.003750pt}%
\definecolor{currentstroke}{rgb}{0.121569,0.466667,0.705882}%
\pgfsetstrokecolor{currentstroke}%
\pgfsetstrokeopacity{0.770809}%
\pgfsetdash{}{0pt}%
\pgfpathmoveto{\pgfqpoint{3.084046in}{2.508414in}}%
\pgfpathcurveto{\pgfqpoint{3.092282in}{2.508414in}}{\pgfqpoint{3.100182in}{2.511687in}}{\pgfqpoint{3.106006in}{2.517511in}}%
\pgfpathcurveto{\pgfqpoint{3.111830in}{2.523335in}}{\pgfqpoint{3.115102in}{2.531235in}}{\pgfqpoint{3.115102in}{2.539471in}}%
\pgfpathcurveto{\pgfqpoint{3.115102in}{2.547707in}}{\pgfqpoint{3.111830in}{2.555607in}}{\pgfqpoint{3.106006in}{2.561431in}}%
\pgfpathcurveto{\pgfqpoint{3.100182in}{2.567255in}}{\pgfqpoint{3.092282in}{2.570527in}}{\pgfqpoint{3.084046in}{2.570527in}}%
\pgfpathcurveto{\pgfqpoint{3.075809in}{2.570527in}}{\pgfqpoint{3.067909in}{2.567255in}}{\pgfqpoint{3.062085in}{2.561431in}}%
\pgfpathcurveto{\pgfqpoint{3.056262in}{2.555607in}}{\pgfqpoint{3.052989in}{2.547707in}}{\pgfqpoint{3.052989in}{2.539471in}}%
\pgfpathcurveto{\pgfqpoint{3.052989in}{2.531235in}}{\pgfqpoint{3.056262in}{2.523335in}}{\pgfqpoint{3.062085in}{2.517511in}}%
\pgfpathcurveto{\pgfqpoint{3.067909in}{2.511687in}}{\pgfqpoint{3.075809in}{2.508414in}}{\pgfqpoint{3.084046in}{2.508414in}}%
\pgfpathclose%
\pgfusepath{stroke,fill}%
\end{pgfscope}%
\begin{pgfscope}%
\pgfpathrectangle{\pgfqpoint{0.100000in}{0.220728in}}{\pgfqpoint{3.696000in}{3.696000in}}%
\pgfusepath{clip}%
\pgfsetbuttcap%
\pgfsetroundjoin%
\definecolor{currentfill}{rgb}{0.121569,0.466667,0.705882}%
\pgfsetfillcolor{currentfill}%
\pgfsetfillopacity{0.771780}%
\pgfsetlinewidth{1.003750pt}%
\definecolor{currentstroke}{rgb}{0.121569,0.466667,0.705882}%
\pgfsetstrokecolor{currentstroke}%
\pgfsetstrokeopacity{0.771780}%
\pgfsetdash{}{0pt}%
\pgfpathmoveto{\pgfqpoint{3.080965in}{2.503192in}}%
\pgfpathcurveto{\pgfqpoint{3.089201in}{2.503192in}}{\pgfqpoint{3.097101in}{2.506464in}}{\pgfqpoint{3.102925in}{2.512288in}}%
\pgfpathcurveto{\pgfqpoint{3.108749in}{2.518112in}}{\pgfqpoint{3.112021in}{2.526012in}}{\pgfqpoint{3.112021in}{2.534248in}}%
\pgfpathcurveto{\pgfqpoint{3.112021in}{2.542485in}}{\pgfqpoint{3.108749in}{2.550385in}}{\pgfqpoint{3.102925in}{2.556209in}}%
\pgfpathcurveto{\pgfqpoint{3.097101in}{2.562033in}}{\pgfqpoint{3.089201in}{2.565305in}}{\pgfqpoint{3.080965in}{2.565305in}}%
\pgfpathcurveto{\pgfqpoint{3.072729in}{2.565305in}}{\pgfqpoint{3.064829in}{2.562033in}}{\pgfqpoint{3.059005in}{2.556209in}}%
\pgfpathcurveto{\pgfqpoint{3.053181in}{2.550385in}}{\pgfqpoint{3.049908in}{2.542485in}}{\pgfqpoint{3.049908in}{2.534248in}}%
\pgfpathcurveto{\pgfqpoint{3.049908in}{2.526012in}}{\pgfqpoint{3.053181in}{2.518112in}}{\pgfqpoint{3.059005in}{2.512288in}}%
\pgfpathcurveto{\pgfqpoint{3.064829in}{2.506464in}}{\pgfqpoint{3.072729in}{2.503192in}}{\pgfqpoint{3.080965in}{2.503192in}}%
\pgfpathclose%
\pgfusepath{stroke,fill}%
\end{pgfscope}%
\begin{pgfscope}%
\pgfpathrectangle{\pgfqpoint{0.100000in}{0.220728in}}{\pgfqpoint{3.696000in}{3.696000in}}%
\pgfusepath{clip}%
\pgfsetbuttcap%
\pgfsetroundjoin%
\definecolor{currentfill}{rgb}{0.121569,0.466667,0.705882}%
\pgfsetfillcolor{currentfill}%
\pgfsetfillopacity{0.773420}%
\pgfsetlinewidth{1.003750pt}%
\definecolor{currentstroke}{rgb}{0.121569,0.466667,0.705882}%
\pgfsetstrokecolor{currentstroke}%
\pgfsetstrokeopacity{0.773420}%
\pgfsetdash{}{0pt}%
\pgfpathmoveto{\pgfqpoint{3.074175in}{2.494498in}}%
\pgfpathcurveto{\pgfqpoint{3.082412in}{2.494498in}}{\pgfqpoint{3.090312in}{2.497771in}}{\pgfqpoint{3.096136in}{2.503595in}}%
\pgfpathcurveto{\pgfqpoint{3.101960in}{2.509418in}}{\pgfqpoint{3.105232in}{2.517318in}}{\pgfqpoint{3.105232in}{2.525555in}}%
\pgfpathcurveto{\pgfqpoint{3.105232in}{2.533791in}}{\pgfqpoint{3.101960in}{2.541691in}}{\pgfqpoint{3.096136in}{2.547515in}}%
\pgfpathcurveto{\pgfqpoint{3.090312in}{2.553339in}}{\pgfqpoint{3.082412in}{2.556611in}}{\pgfqpoint{3.074175in}{2.556611in}}%
\pgfpathcurveto{\pgfqpoint{3.065939in}{2.556611in}}{\pgfqpoint{3.058039in}{2.553339in}}{\pgfqpoint{3.052215in}{2.547515in}}%
\pgfpathcurveto{\pgfqpoint{3.046391in}{2.541691in}}{\pgfqpoint{3.043119in}{2.533791in}}{\pgfqpoint{3.043119in}{2.525555in}}%
\pgfpathcurveto{\pgfqpoint{3.043119in}{2.517318in}}{\pgfqpoint{3.046391in}{2.509418in}}{\pgfqpoint{3.052215in}{2.503595in}}%
\pgfpathcurveto{\pgfqpoint{3.058039in}{2.497771in}}{\pgfqpoint{3.065939in}{2.494498in}}{\pgfqpoint{3.074175in}{2.494498in}}%
\pgfpathclose%
\pgfusepath{stroke,fill}%
\end{pgfscope}%
\begin{pgfscope}%
\pgfpathrectangle{\pgfqpoint{0.100000in}{0.220728in}}{\pgfqpoint{3.696000in}{3.696000in}}%
\pgfusepath{clip}%
\pgfsetbuttcap%
\pgfsetroundjoin%
\definecolor{currentfill}{rgb}{0.121569,0.466667,0.705882}%
\pgfsetfillcolor{currentfill}%
\pgfsetfillopacity{0.773648}%
\pgfsetlinewidth{1.003750pt}%
\definecolor{currentstroke}{rgb}{0.121569,0.466667,0.705882}%
\pgfsetstrokecolor{currentstroke}%
\pgfsetstrokeopacity{0.773648}%
\pgfsetdash{}{0pt}%
\pgfpathmoveto{\pgfqpoint{1.156960in}{2.058441in}}%
\pgfpathcurveto{\pgfqpoint{1.165197in}{2.058441in}}{\pgfqpoint{1.173097in}{2.061714in}}{\pgfqpoint{1.178921in}{2.067538in}}%
\pgfpathcurveto{\pgfqpoint{1.184744in}{2.073362in}}{\pgfqpoint{1.188017in}{2.081262in}}{\pgfqpoint{1.188017in}{2.089498in}}%
\pgfpathcurveto{\pgfqpoint{1.188017in}{2.097734in}}{\pgfqpoint{1.184744in}{2.105634in}}{\pgfqpoint{1.178921in}{2.111458in}}%
\pgfpathcurveto{\pgfqpoint{1.173097in}{2.117282in}}{\pgfqpoint{1.165197in}{2.120554in}}{\pgfqpoint{1.156960in}{2.120554in}}%
\pgfpathcurveto{\pgfqpoint{1.148724in}{2.120554in}}{\pgfqpoint{1.140824in}{2.117282in}}{\pgfqpoint{1.135000in}{2.111458in}}%
\pgfpathcurveto{\pgfqpoint{1.129176in}{2.105634in}}{\pgfqpoint{1.125904in}{2.097734in}}{\pgfqpoint{1.125904in}{2.089498in}}%
\pgfpathcurveto{\pgfqpoint{1.125904in}{2.081262in}}{\pgfqpoint{1.129176in}{2.073362in}}{\pgfqpoint{1.135000in}{2.067538in}}%
\pgfpathcurveto{\pgfqpoint{1.140824in}{2.061714in}}{\pgfqpoint{1.148724in}{2.058441in}}{\pgfqpoint{1.156960in}{2.058441in}}%
\pgfpathclose%
\pgfusepath{stroke,fill}%
\end{pgfscope}%
\begin{pgfscope}%
\pgfpathrectangle{\pgfqpoint{0.100000in}{0.220728in}}{\pgfqpoint{3.696000in}{3.696000in}}%
\pgfusepath{clip}%
\pgfsetbuttcap%
\pgfsetroundjoin%
\definecolor{currentfill}{rgb}{0.121569,0.466667,0.705882}%
\pgfsetfillcolor{currentfill}%
\pgfsetfillopacity{0.775925}%
\pgfsetlinewidth{1.003750pt}%
\definecolor{currentstroke}{rgb}{0.121569,0.466667,0.705882}%
\pgfsetstrokecolor{currentstroke}%
\pgfsetstrokeopacity{0.775925}%
\pgfsetdash{}{0pt}%
\pgfpathmoveto{\pgfqpoint{3.066792in}{2.482130in}}%
\pgfpathcurveto{\pgfqpoint{3.075029in}{2.482130in}}{\pgfqpoint{3.082929in}{2.485403in}}{\pgfqpoint{3.088753in}{2.491227in}}%
\pgfpathcurveto{\pgfqpoint{3.094577in}{2.497051in}}{\pgfqpoint{3.097849in}{2.504951in}}{\pgfqpoint{3.097849in}{2.513187in}}%
\pgfpathcurveto{\pgfqpoint{3.097849in}{2.521423in}}{\pgfqpoint{3.094577in}{2.529323in}}{\pgfqpoint{3.088753in}{2.535147in}}%
\pgfpathcurveto{\pgfqpoint{3.082929in}{2.540971in}}{\pgfqpoint{3.075029in}{2.544243in}}{\pgfqpoint{3.066792in}{2.544243in}}%
\pgfpathcurveto{\pgfqpoint{3.058556in}{2.544243in}}{\pgfqpoint{3.050656in}{2.540971in}}{\pgfqpoint{3.044832in}{2.535147in}}%
\pgfpathcurveto{\pgfqpoint{3.039008in}{2.529323in}}{\pgfqpoint{3.035736in}{2.521423in}}{\pgfqpoint{3.035736in}{2.513187in}}%
\pgfpathcurveto{\pgfqpoint{3.035736in}{2.504951in}}{\pgfqpoint{3.039008in}{2.497051in}}{\pgfqpoint{3.044832in}{2.491227in}}%
\pgfpathcurveto{\pgfqpoint{3.050656in}{2.485403in}}{\pgfqpoint{3.058556in}{2.482130in}}{\pgfqpoint{3.066792in}{2.482130in}}%
\pgfpathclose%
\pgfusepath{stroke,fill}%
\end{pgfscope}%
\begin{pgfscope}%
\pgfpathrectangle{\pgfqpoint{0.100000in}{0.220728in}}{\pgfqpoint{3.696000in}{3.696000in}}%
\pgfusepath{clip}%
\pgfsetbuttcap%
\pgfsetroundjoin%
\definecolor{currentfill}{rgb}{0.121569,0.466667,0.705882}%
\pgfsetfillcolor{currentfill}%
\pgfsetfillopacity{0.776769}%
\pgfsetlinewidth{1.003750pt}%
\definecolor{currentstroke}{rgb}{0.121569,0.466667,0.705882}%
\pgfsetstrokecolor{currentstroke}%
\pgfsetstrokeopacity{0.776769}%
\pgfsetdash{}{0pt}%
\pgfpathmoveto{\pgfqpoint{1.173422in}{2.053411in}}%
\pgfpathcurveto{\pgfqpoint{1.181658in}{2.053411in}}{\pgfqpoint{1.189559in}{2.056683in}}{\pgfqpoint{1.195382in}{2.062507in}}%
\pgfpathcurveto{\pgfqpoint{1.201206in}{2.068331in}}{\pgfqpoint{1.204479in}{2.076231in}}{\pgfqpoint{1.204479in}{2.084468in}}%
\pgfpathcurveto{\pgfqpoint{1.204479in}{2.092704in}}{\pgfqpoint{1.201206in}{2.100604in}}{\pgfqpoint{1.195382in}{2.106428in}}%
\pgfpathcurveto{\pgfqpoint{1.189559in}{2.112252in}}{\pgfqpoint{1.181658in}{2.115524in}}{\pgfqpoint{1.173422in}{2.115524in}}%
\pgfpathcurveto{\pgfqpoint{1.165186in}{2.115524in}}{\pgfqpoint{1.157286in}{2.112252in}}{\pgfqpoint{1.151462in}{2.106428in}}%
\pgfpathcurveto{\pgfqpoint{1.145638in}{2.100604in}}{\pgfqpoint{1.142366in}{2.092704in}}{\pgfqpoint{1.142366in}{2.084468in}}%
\pgfpathcurveto{\pgfqpoint{1.142366in}{2.076231in}}{\pgfqpoint{1.145638in}{2.068331in}}{\pgfqpoint{1.151462in}{2.062507in}}%
\pgfpathcurveto{\pgfqpoint{1.157286in}{2.056683in}}{\pgfqpoint{1.165186in}{2.053411in}}{\pgfqpoint{1.173422in}{2.053411in}}%
\pgfpathclose%
\pgfusepath{stroke,fill}%
\end{pgfscope}%
\begin{pgfscope}%
\pgfpathrectangle{\pgfqpoint{0.100000in}{0.220728in}}{\pgfqpoint{3.696000in}{3.696000in}}%
\pgfusepath{clip}%
\pgfsetbuttcap%
\pgfsetroundjoin%
\definecolor{currentfill}{rgb}{0.121569,0.466667,0.705882}%
\pgfsetfillcolor{currentfill}%
\pgfsetfillopacity{0.778518}%
\pgfsetlinewidth{1.003750pt}%
\definecolor{currentstroke}{rgb}{0.121569,0.466667,0.705882}%
\pgfsetstrokecolor{currentstroke}%
\pgfsetstrokeopacity{0.778518}%
\pgfsetdash{}{0pt}%
\pgfpathmoveto{\pgfqpoint{1.184725in}{2.050359in}}%
\pgfpathcurveto{\pgfqpoint{1.192961in}{2.050359in}}{\pgfqpoint{1.200861in}{2.053632in}}{\pgfqpoint{1.206685in}{2.059455in}}%
\pgfpathcurveto{\pgfqpoint{1.212509in}{2.065279in}}{\pgfqpoint{1.215782in}{2.073179in}}{\pgfqpoint{1.215782in}{2.081416in}}%
\pgfpathcurveto{\pgfqpoint{1.215782in}{2.089652in}}{\pgfqpoint{1.212509in}{2.097552in}}{\pgfqpoint{1.206685in}{2.103376in}}%
\pgfpathcurveto{\pgfqpoint{1.200861in}{2.109200in}}{\pgfqpoint{1.192961in}{2.112472in}}{\pgfqpoint{1.184725in}{2.112472in}}%
\pgfpathcurveto{\pgfqpoint{1.176489in}{2.112472in}}{\pgfqpoint{1.168589in}{2.109200in}}{\pgfqpoint{1.162765in}{2.103376in}}%
\pgfpathcurveto{\pgfqpoint{1.156941in}{2.097552in}}{\pgfqpoint{1.153669in}{2.089652in}}{\pgfqpoint{1.153669in}{2.081416in}}%
\pgfpathcurveto{\pgfqpoint{1.153669in}{2.073179in}}{\pgfqpoint{1.156941in}{2.065279in}}{\pgfqpoint{1.162765in}{2.059455in}}%
\pgfpathcurveto{\pgfqpoint{1.168589in}{2.053632in}}{\pgfqpoint{1.176489in}{2.050359in}}{\pgfqpoint{1.184725in}{2.050359in}}%
\pgfpathclose%
\pgfusepath{stroke,fill}%
\end{pgfscope}%
\begin{pgfscope}%
\pgfpathrectangle{\pgfqpoint{0.100000in}{0.220728in}}{\pgfqpoint{3.696000in}{3.696000in}}%
\pgfusepath{clip}%
\pgfsetbuttcap%
\pgfsetroundjoin%
\definecolor{currentfill}{rgb}{0.121569,0.466667,0.705882}%
\pgfsetfillcolor{currentfill}%
\pgfsetfillopacity{0.779318}%
\pgfsetlinewidth{1.003750pt}%
\definecolor{currentstroke}{rgb}{0.121569,0.466667,0.705882}%
\pgfsetstrokecolor{currentstroke}%
\pgfsetstrokeopacity{0.779318}%
\pgfsetdash{}{0pt}%
\pgfpathmoveto{\pgfqpoint{3.055192in}{2.467006in}}%
\pgfpathcurveto{\pgfqpoint{3.063428in}{2.467006in}}{\pgfqpoint{3.071328in}{2.470278in}}{\pgfqpoint{3.077152in}{2.476102in}}%
\pgfpathcurveto{\pgfqpoint{3.082976in}{2.481926in}}{\pgfqpoint{3.086248in}{2.489826in}}{\pgfqpoint{3.086248in}{2.498062in}}%
\pgfpathcurveto{\pgfqpoint{3.086248in}{2.506298in}}{\pgfqpoint{3.082976in}{2.514198in}}{\pgfqpoint{3.077152in}{2.520022in}}%
\pgfpathcurveto{\pgfqpoint{3.071328in}{2.525846in}}{\pgfqpoint{3.063428in}{2.529119in}}{\pgfqpoint{3.055192in}{2.529119in}}%
\pgfpathcurveto{\pgfqpoint{3.046956in}{2.529119in}}{\pgfqpoint{3.039056in}{2.525846in}}{\pgfqpoint{3.033232in}{2.520022in}}%
\pgfpathcurveto{\pgfqpoint{3.027408in}{2.514198in}}{\pgfqpoint{3.024135in}{2.506298in}}{\pgfqpoint{3.024135in}{2.498062in}}%
\pgfpathcurveto{\pgfqpoint{3.024135in}{2.489826in}}{\pgfqpoint{3.027408in}{2.481926in}}{\pgfqpoint{3.033232in}{2.476102in}}%
\pgfpathcurveto{\pgfqpoint{3.039056in}{2.470278in}}{\pgfqpoint{3.046956in}{2.467006in}}{\pgfqpoint{3.055192in}{2.467006in}}%
\pgfpathclose%
\pgfusepath{stroke,fill}%
\end{pgfscope}%
\begin{pgfscope}%
\pgfpathrectangle{\pgfqpoint{0.100000in}{0.220728in}}{\pgfqpoint{3.696000in}{3.696000in}}%
\pgfusepath{clip}%
\pgfsetbuttcap%
\pgfsetroundjoin%
\definecolor{currentfill}{rgb}{0.121569,0.466667,0.705882}%
\pgfsetfillcolor{currentfill}%
\pgfsetfillopacity{0.779756}%
\pgfsetlinewidth{1.003750pt}%
\definecolor{currentstroke}{rgb}{0.121569,0.466667,0.705882}%
\pgfsetstrokecolor{currentstroke}%
\pgfsetstrokeopacity{0.779756}%
\pgfsetdash{}{0pt}%
\pgfpathmoveto{\pgfqpoint{1.191034in}{2.045586in}}%
\pgfpathcurveto{\pgfqpoint{1.199270in}{2.045586in}}{\pgfqpoint{1.207170in}{2.048858in}}{\pgfqpoint{1.212994in}{2.054682in}}%
\pgfpathcurveto{\pgfqpoint{1.218818in}{2.060506in}}{\pgfqpoint{1.222091in}{2.068406in}}{\pgfqpoint{1.222091in}{2.076643in}}%
\pgfpathcurveto{\pgfqpoint{1.222091in}{2.084879in}}{\pgfqpoint{1.218818in}{2.092779in}}{\pgfqpoint{1.212994in}{2.098603in}}%
\pgfpathcurveto{\pgfqpoint{1.207170in}{2.104427in}}{\pgfqpoint{1.199270in}{2.107699in}}{\pgfqpoint{1.191034in}{2.107699in}}%
\pgfpathcurveto{\pgfqpoint{1.182798in}{2.107699in}}{\pgfqpoint{1.174898in}{2.104427in}}{\pgfqpoint{1.169074in}{2.098603in}}%
\pgfpathcurveto{\pgfqpoint{1.163250in}{2.092779in}}{\pgfqpoint{1.159978in}{2.084879in}}{\pgfqpoint{1.159978in}{2.076643in}}%
\pgfpathcurveto{\pgfqpoint{1.159978in}{2.068406in}}{\pgfqpoint{1.163250in}{2.060506in}}{\pgfqpoint{1.169074in}{2.054682in}}%
\pgfpathcurveto{\pgfqpoint{1.174898in}{2.048858in}}{\pgfqpoint{1.182798in}{2.045586in}}{\pgfqpoint{1.191034in}{2.045586in}}%
\pgfpathclose%
\pgfusepath{stroke,fill}%
\end{pgfscope}%
\begin{pgfscope}%
\pgfpathrectangle{\pgfqpoint{0.100000in}{0.220728in}}{\pgfqpoint{3.696000in}{3.696000in}}%
\pgfusepath{clip}%
\pgfsetbuttcap%
\pgfsetroundjoin%
\definecolor{currentfill}{rgb}{0.121569,0.466667,0.705882}%
\pgfsetfillcolor{currentfill}%
\pgfsetfillopacity{0.782023}%
\pgfsetlinewidth{1.003750pt}%
\definecolor{currentstroke}{rgb}{0.121569,0.466667,0.705882}%
\pgfsetstrokecolor{currentstroke}%
\pgfsetstrokeopacity{0.782023}%
\pgfsetdash{}{0pt}%
\pgfpathmoveto{\pgfqpoint{1.203375in}{2.039439in}}%
\pgfpathcurveto{\pgfqpoint{1.211611in}{2.039439in}}{\pgfqpoint{1.219512in}{2.042712in}}{\pgfqpoint{1.225335in}{2.048536in}}%
\pgfpathcurveto{\pgfqpoint{1.231159in}{2.054359in}}{\pgfqpoint{1.234432in}{2.062260in}}{\pgfqpoint{1.234432in}{2.070496in}}%
\pgfpathcurveto{\pgfqpoint{1.234432in}{2.078732in}}{\pgfqpoint{1.231159in}{2.086632in}}{\pgfqpoint{1.225335in}{2.092456in}}%
\pgfpathcurveto{\pgfqpoint{1.219512in}{2.098280in}}{\pgfqpoint{1.211611in}{2.101552in}}{\pgfqpoint{1.203375in}{2.101552in}}%
\pgfpathcurveto{\pgfqpoint{1.195139in}{2.101552in}}{\pgfqpoint{1.187239in}{2.098280in}}{\pgfqpoint{1.181415in}{2.092456in}}%
\pgfpathcurveto{\pgfqpoint{1.175591in}{2.086632in}}{\pgfqpoint{1.172319in}{2.078732in}}{\pgfqpoint{1.172319in}{2.070496in}}%
\pgfpathcurveto{\pgfqpoint{1.172319in}{2.062260in}}{\pgfqpoint{1.175591in}{2.054359in}}{\pgfqpoint{1.181415in}{2.048536in}}%
\pgfpathcurveto{\pgfqpoint{1.187239in}{2.042712in}}{\pgfqpoint{1.195139in}{2.039439in}}{\pgfqpoint{1.203375in}{2.039439in}}%
\pgfpathclose%
\pgfusepath{stroke,fill}%
\end{pgfscope}%
\begin{pgfscope}%
\pgfpathrectangle{\pgfqpoint{0.100000in}{0.220728in}}{\pgfqpoint{3.696000in}{3.696000in}}%
\pgfusepath{clip}%
\pgfsetbuttcap%
\pgfsetroundjoin%
\definecolor{currentfill}{rgb}{0.121569,0.466667,0.705882}%
\pgfsetfillcolor{currentfill}%
\pgfsetfillopacity{0.783403}%
\pgfsetlinewidth{1.003750pt}%
\definecolor{currentstroke}{rgb}{0.121569,0.466667,0.705882}%
\pgfsetstrokecolor{currentstroke}%
\pgfsetstrokeopacity{0.783403}%
\pgfsetdash{}{0pt}%
\pgfpathmoveto{\pgfqpoint{3.043571in}{2.448822in}}%
\pgfpathcurveto{\pgfqpoint{3.051807in}{2.448822in}}{\pgfqpoint{3.059707in}{2.452094in}}{\pgfqpoint{3.065531in}{2.457918in}}%
\pgfpathcurveto{\pgfqpoint{3.071355in}{2.463742in}}{\pgfqpoint{3.074628in}{2.471642in}}{\pgfqpoint{3.074628in}{2.479878in}}%
\pgfpathcurveto{\pgfqpoint{3.074628in}{2.488115in}}{\pgfqpoint{3.071355in}{2.496015in}}{\pgfqpoint{3.065531in}{2.501839in}}%
\pgfpathcurveto{\pgfqpoint{3.059707in}{2.507663in}}{\pgfqpoint{3.051807in}{2.510935in}}{\pgfqpoint{3.043571in}{2.510935in}}%
\pgfpathcurveto{\pgfqpoint{3.035335in}{2.510935in}}{\pgfqpoint{3.027435in}{2.507663in}}{\pgfqpoint{3.021611in}{2.501839in}}%
\pgfpathcurveto{\pgfqpoint{3.015787in}{2.496015in}}{\pgfqpoint{3.012515in}{2.488115in}}{\pgfqpoint{3.012515in}{2.479878in}}%
\pgfpathcurveto{\pgfqpoint{3.012515in}{2.471642in}}{\pgfqpoint{3.015787in}{2.463742in}}{\pgfqpoint{3.021611in}{2.457918in}}%
\pgfpathcurveto{\pgfqpoint{3.027435in}{2.452094in}}{\pgfqpoint{3.035335in}{2.448822in}}{\pgfqpoint{3.043571in}{2.448822in}}%
\pgfpathclose%
\pgfusepath{stroke,fill}%
\end{pgfscope}%
\begin{pgfscope}%
\pgfpathrectangle{\pgfqpoint{0.100000in}{0.220728in}}{\pgfqpoint{3.696000in}{3.696000in}}%
\pgfusepath{clip}%
\pgfsetbuttcap%
\pgfsetroundjoin%
\definecolor{currentfill}{rgb}{0.121569,0.466667,0.705882}%
\pgfsetfillcolor{currentfill}%
\pgfsetfillopacity{0.785416}%
\pgfsetlinewidth{1.003750pt}%
\definecolor{currentstroke}{rgb}{0.121569,0.466667,0.705882}%
\pgfsetstrokecolor{currentstroke}%
\pgfsetstrokeopacity{0.785416}%
\pgfsetdash{}{0pt}%
\pgfpathmoveto{\pgfqpoint{3.036538in}{2.438562in}}%
\pgfpathcurveto{\pgfqpoint{3.044775in}{2.438562in}}{\pgfqpoint{3.052675in}{2.441835in}}{\pgfqpoint{3.058499in}{2.447659in}}%
\pgfpathcurveto{\pgfqpoint{3.064323in}{2.453482in}}{\pgfqpoint{3.067595in}{2.461383in}}{\pgfqpoint{3.067595in}{2.469619in}}%
\pgfpathcurveto{\pgfqpoint{3.067595in}{2.477855in}}{\pgfqpoint{3.064323in}{2.485755in}}{\pgfqpoint{3.058499in}{2.491579in}}%
\pgfpathcurveto{\pgfqpoint{3.052675in}{2.497403in}}{\pgfqpoint{3.044775in}{2.500675in}}{\pgfqpoint{3.036538in}{2.500675in}}%
\pgfpathcurveto{\pgfqpoint{3.028302in}{2.500675in}}{\pgfqpoint{3.020402in}{2.497403in}}{\pgfqpoint{3.014578in}{2.491579in}}%
\pgfpathcurveto{\pgfqpoint{3.008754in}{2.485755in}}{\pgfqpoint{3.005482in}{2.477855in}}{\pgfqpoint{3.005482in}{2.469619in}}%
\pgfpathcurveto{\pgfqpoint{3.005482in}{2.461383in}}{\pgfqpoint{3.008754in}{2.453482in}}{\pgfqpoint{3.014578in}{2.447659in}}%
\pgfpathcurveto{\pgfqpoint{3.020402in}{2.441835in}}{\pgfqpoint{3.028302in}{2.438562in}}{\pgfqpoint{3.036538in}{2.438562in}}%
\pgfpathclose%
\pgfusepath{stroke,fill}%
\end{pgfscope}%
\begin{pgfscope}%
\pgfpathrectangle{\pgfqpoint{0.100000in}{0.220728in}}{\pgfqpoint{3.696000in}{3.696000in}}%
\pgfusepath{clip}%
\pgfsetbuttcap%
\pgfsetroundjoin%
\definecolor{currentfill}{rgb}{0.121569,0.466667,0.705882}%
\pgfsetfillcolor{currentfill}%
\pgfsetfillopacity{0.785732}%
\pgfsetlinewidth{1.003750pt}%
\definecolor{currentstroke}{rgb}{0.121569,0.466667,0.705882}%
\pgfsetstrokecolor{currentstroke}%
\pgfsetstrokeopacity{0.785732}%
\pgfsetdash{}{0pt}%
\pgfpathmoveto{\pgfqpoint{1.226379in}{2.027813in}}%
\pgfpathcurveto{\pgfqpoint{1.234615in}{2.027813in}}{\pgfqpoint{1.242515in}{2.031085in}}{\pgfqpoint{1.248339in}{2.036909in}}%
\pgfpathcurveto{\pgfqpoint{1.254163in}{2.042733in}}{\pgfqpoint{1.257435in}{2.050633in}}{\pgfqpoint{1.257435in}{2.058870in}}%
\pgfpathcurveto{\pgfqpoint{1.257435in}{2.067106in}}{\pgfqpoint{1.254163in}{2.075006in}}{\pgfqpoint{1.248339in}{2.080830in}}%
\pgfpathcurveto{\pgfqpoint{1.242515in}{2.086654in}}{\pgfqpoint{1.234615in}{2.089926in}}{\pgfqpoint{1.226379in}{2.089926in}}%
\pgfpathcurveto{\pgfqpoint{1.218143in}{2.089926in}}{\pgfqpoint{1.210243in}{2.086654in}}{\pgfqpoint{1.204419in}{2.080830in}}%
\pgfpathcurveto{\pgfqpoint{1.198595in}{2.075006in}}{\pgfqpoint{1.195322in}{2.067106in}}{\pgfqpoint{1.195322in}{2.058870in}}%
\pgfpathcurveto{\pgfqpoint{1.195322in}{2.050633in}}{\pgfqpoint{1.198595in}{2.042733in}}{\pgfqpoint{1.204419in}{2.036909in}}%
\pgfpathcurveto{\pgfqpoint{1.210243in}{2.031085in}}{\pgfqpoint{1.218143in}{2.027813in}}{\pgfqpoint{1.226379in}{2.027813in}}%
\pgfpathclose%
\pgfusepath{stroke,fill}%
\end{pgfscope}%
\begin{pgfscope}%
\pgfpathrectangle{\pgfqpoint{0.100000in}{0.220728in}}{\pgfqpoint{3.696000in}{3.696000in}}%
\pgfusepath{clip}%
\pgfsetbuttcap%
\pgfsetroundjoin%
\definecolor{currentfill}{rgb}{0.121569,0.466667,0.705882}%
\pgfsetfillcolor{currentfill}%
\pgfsetfillopacity{0.786548}%
\pgfsetlinewidth{1.003750pt}%
\definecolor{currentstroke}{rgb}{0.121569,0.466667,0.705882}%
\pgfsetstrokecolor{currentstroke}%
\pgfsetstrokeopacity{0.786548}%
\pgfsetdash{}{0pt}%
\pgfpathmoveto{\pgfqpoint{3.032664in}{2.433053in}}%
\pgfpathcurveto{\pgfqpoint{3.040900in}{2.433053in}}{\pgfqpoint{3.048800in}{2.436325in}}{\pgfqpoint{3.054624in}{2.442149in}}%
\pgfpathcurveto{\pgfqpoint{3.060448in}{2.447973in}}{\pgfqpoint{3.063720in}{2.455873in}}{\pgfqpoint{3.063720in}{2.464109in}}%
\pgfpathcurveto{\pgfqpoint{3.063720in}{2.472345in}}{\pgfqpoint{3.060448in}{2.480245in}}{\pgfqpoint{3.054624in}{2.486069in}}%
\pgfpathcurveto{\pgfqpoint{3.048800in}{2.491893in}}{\pgfqpoint{3.040900in}{2.495166in}}{\pgfqpoint{3.032664in}{2.495166in}}%
\pgfpathcurveto{\pgfqpoint{3.024427in}{2.495166in}}{\pgfqpoint{3.016527in}{2.491893in}}{\pgfqpoint{3.010703in}{2.486069in}}%
\pgfpathcurveto{\pgfqpoint{3.004880in}{2.480245in}}{\pgfqpoint{3.001607in}{2.472345in}}{\pgfqpoint{3.001607in}{2.464109in}}%
\pgfpathcurveto{\pgfqpoint{3.001607in}{2.455873in}}{\pgfqpoint{3.004880in}{2.447973in}}{\pgfqpoint{3.010703in}{2.442149in}}%
\pgfpathcurveto{\pgfqpoint{3.016527in}{2.436325in}}{\pgfqpoint{3.024427in}{2.433053in}}{\pgfqpoint{3.032664in}{2.433053in}}%
\pgfpathclose%
\pgfusepath{stroke,fill}%
\end{pgfscope}%
\begin{pgfscope}%
\pgfpathrectangle{\pgfqpoint{0.100000in}{0.220728in}}{\pgfqpoint{3.696000in}{3.696000in}}%
\pgfusepath{clip}%
\pgfsetbuttcap%
\pgfsetroundjoin%
\definecolor{currentfill}{rgb}{0.121569,0.466667,0.705882}%
\pgfsetfillcolor{currentfill}%
\pgfsetfillopacity{0.787235}%
\pgfsetlinewidth{1.003750pt}%
\definecolor{currentstroke}{rgb}{0.121569,0.466667,0.705882}%
\pgfsetstrokecolor{currentstroke}%
\pgfsetstrokeopacity{0.787235}%
\pgfsetdash{}{0pt}%
\pgfpathmoveto{\pgfqpoint{3.030750in}{2.430032in}}%
\pgfpathcurveto{\pgfqpoint{3.038986in}{2.430032in}}{\pgfqpoint{3.046886in}{2.433304in}}{\pgfqpoint{3.052710in}{2.439128in}}%
\pgfpathcurveto{\pgfqpoint{3.058534in}{2.444952in}}{\pgfqpoint{3.061806in}{2.452852in}}{\pgfqpoint{3.061806in}{2.461089in}}%
\pgfpathcurveto{\pgfqpoint{3.061806in}{2.469325in}}{\pgfqpoint{3.058534in}{2.477225in}}{\pgfqpoint{3.052710in}{2.483049in}}%
\pgfpathcurveto{\pgfqpoint{3.046886in}{2.488873in}}{\pgfqpoint{3.038986in}{2.492145in}}{\pgfqpoint{3.030750in}{2.492145in}}%
\pgfpathcurveto{\pgfqpoint{3.022514in}{2.492145in}}{\pgfqpoint{3.014614in}{2.488873in}}{\pgfqpoint{3.008790in}{2.483049in}}%
\pgfpathcurveto{\pgfqpoint{3.002966in}{2.477225in}}{\pgfqpoint{2.999693in}{2.469325in}}{\pgfqpoint{2.999693in}{2.461089in}}%
\pgfpathcurveto{\pgfqpoint{2.999693in}{2.452852in}}{\pgfqpoint{3.002966in}{2.444952in}}{\pgfqpoint{3.008790in}{2.439128in}}%
\pgfpathcurveto{\pgfqpoint{3.014614in}{2.433304in}}{\pgfqpoint{3.022514in}{2.430032in}}{\pgfqpoint{3.030750in}{2.430032in}}%
\pgfpathclose%
\pgfusepath{stroke,fill}%
\end{pgfscope}%
\begin{pgfscope}%
\pgfpathrectangle{\pgfqpoint{0.100000in}{0.220728in}}{\pgfqpoint{3.696000in}{3.696000in}}%
\pgfusepath{clip}%
\pgfsetbuttcap%
\pgfsetroundjoin%
\definecolor{currentfill}{rgb}{0.121569,0.466667,0.705882}%
\pgfsetfillcolor{currentfill}%
\pgfsetfillopacity{0.787589}%
\pgfsetlinewidth{1.003750pt}%
\definecolor{currentstroke}{rgb}{0.121569,0.466667,0.705882}%
\pgfsetstrokecolor{currentstroke}%
\pgfsetstrokeopacity{0.787589}%
\pgfsetdash{}{0pt}%
\pgfpathmoveto{\pgfqpoint{3.029540in}{2.428475in}}%
\pgfpathcurveto{\pgfqpoint{3.037776in}{2.428475in}}{\pgfqpoint{3.045676in}{2.431748in}}{\pgfqpoint{3.051500in}{2.437572in}}%
\pgfpathcurveto{\pgfqpoint{3.057324in}{2.443396in}}{\pgfqpoint{3.060596in}{2.451296in}}{\pgfqpoint{3.060596in}{2.459532in}}%
\pgfpathcurveto{\pgfqpoint{3.060596in}{2.467768in}}{\pgfqpoint{3.057324in}{2.475668in}}{\pgfqpoint{3.051500in}{2.481492in}}%
\pgfpathcurveto{\pgfqpoint{3.045676in}{2.487316in}}{\pgfqpoint{3.037776in}{2.490588in}}{\pgfqpoint{3.029540in}{2.490588in}}%
\pgfpathcurveto{\pgfqpoint{3.021303in}{2.490588in}}{\pgfqpoint{3.013403in}{2.487316in}}{\pgfqpoint{3.007579in}{2.481492in}}%
\pgfpathcurveto{\pgfqpoint{3.001756in}{2.475668in}}{\pgfqpoint{2.998483in}{2.467768in}}{\pgfqpoint{2.998483in}{2.459532in}}%
\pgfpathcurveto{\pgfqpoint{2.998483in}{2.451296in}}{\pgfqpoint{3.001756in}{2.443396in}}{\pgfqpoint{3.007579in}{2.437572in}}%
\pgfpathcurveto{\pgfqpoint{3.013403in}{2.431748in}}{\pgfqpoint{3.021303in}{2.428475in}}{\pgfqpoint{3.029540in}{2.428475in}}%
\pgfpathclose%
\pgfusepath{stroke,fill}%
\end{pgfscope}%
\begin{pgfscope}%
\pgfpathrectangle{\pgfqpoint{0.100000in}{0.220728in}}{\pgfqpoint{3.696000in}{3.696000in}}%
\pgfusepath{clip}%
\pgfsetbuttcap%
\pgfsetroundjoin%
\definecolor{currentfill}{rgb}{0.121569,0.466667,0.705882}%
\pgfsetfillcolor{currentfill}%
\pgfsetfillopacity{0.787793}%
\pgfsetlinewidth{1.003750pt}%
\definecolor{currentstroke}{rgb}{0.121569,0.466667,0.705882}%
\pgfsetstrokecolor{currentstroke}%
\pgfsetstrokeopacity{0.787793}%
\pgfsetdash{}{0pt}%
\pgfpathmoveto{\pgfqpoint{3.028984in}{2.427520in}}%
\pgfpathcurveto{\pgfqpoint{3.037220in}{2.427520in}}{\pgfqpoint{3.045120in}{2.430792in}}{\pgfqpoint{3.050944in}{2.436616in}}%
\pgfpathcurveto{\pgfqpoint{3.056768in}{2.442440in}}{\pgfqpoint{3.060040in}{2.450340in}}{\pgfqpoint{3.060040in}{2.458577in}}%
\pgfpathcurveto{\pgfqpoint{3.060040in}{2.466813in}}{\pgfqpoint{3.056768in}{2.474713in}}{\pgfqpoint{3.050944in}{2.480537in}}%
\pgfpathcurveto{\pgfqpoint{3.045120in}{2.486361in}}{\pgfqpoint{3.037220in}{2.489633in}}{\pgfqpoint{3.028984in}{2.489633in}}%
\pgfpathcurveto{\pgfqpoint{3.020747in}{2.489633in}}{\pgfqpoint{3.012847in}{2.486361in}}{\pgfqpoint{3.007023in}{2.480537in}}%
\pgfpathcurveto{\pgfqpoint{3.001200in}{2.474713in}}{\pgfqpoint{2.997927in}{2.466813in}}{\pgfqpoint{2.997927in}{2.458577in}}%
\pgfpathcurveto{\pgfqpoint{2.997927in}{2.450340in}}{\pgfqpoint{3.001200in}{2.442440in}}{\pgfqpoint{3.007023in}{2.436616in}}%
\pgfpathcurveto{\pgfqpoint{3.012847in}{2.430792in}}{\pgfqpoint{3.020747in}{2.427520in}}{\pgfqpoint{3.028984in}{2.427520in}}%
\pgfpathclose%
\pgfusepath{stroke,fill}%
\end{pgfscope}%
\begin{pgfscope}%
\pgfpathrectangle{\pgfqpoint{0.100000in}{0.220728in}}{\pgfqpoint{3.696000in}{3.696000in}}%
\pgfusepath{clip}%
\pgfsetbuttcap%
\pgfsetroundjoin%
\definecolor{currentfill}{rgb}{0.121569,0.466667,0.705882}%
\pgfsetfillcolor{currentfill}%
\pgfsetfillopacity{0.788627}%
\pgfsetlinewidth{1.003750pt}%
\definecolor{currentstroke}{rgb}{0.121569,0.466667,0.705882}%
\pgfsetstrokecolor{currentstroke}%
\pgfsetstrokeopacity{0.788627}%
\pgfsetdash{}{0pt}%
\pgfpathmoveto{\pgfqpoint{3.025894in}{2.424005in}}%
\pgfpathcurveto{\pgfqpoint{3.034130in}{2.424005in}}{\pgfqpoint{3.042030in}{2.427277in}}{\pgfqpoint{3.047854in}{2.433101in}}%
\pgfpathcurveto{\pgfqpoint{3.053678in}{2.438925in}}{\pgfqpoint{3.056950in}{2.446825in}}{\pgfqpoint{3.056950in}{2.455061in}}%
\pgfpathcurveto{\pgfqpoint{3.056950in}{2.463297in}}{\pgfqpoint{3.053678in}{2.471198in}}{\pgfqpoint{3.047854in}{2.477021in}}%
\pgfpathcurveto{\pgfqpoint{3.042030in}{2.482845in}}{\pgfqpoint{3.034130in}{2.486118in}}{\pgfqpoint{3.025894in}{2.486118in}}%
\pgfpathcurveto{\pgfqpoint{3.017657in}{2.486118in}}{\pgfqpoint{3.009757in}{2.482845in}}{\pgfqpoint{3.003933in}{2.477021in}}%
\pgfpathcurveto{\pgfqpoint{2.998110in}{2.471198in}}{\pgfqpoint{2.994837in}{2.463297in}}{\pgfqpoint{2.994837in}{2.455061in}}%
\pgfpathcurveto{\pgfqpoint{2.994837in}{2.446825in}}{\pgfqpoint{2.998110in}{2.438925in}}{\pgfqpoint{3.003933in}{2.433101in}}%
\pgfpathcurveto{\pgfqpoint{3.009757in}{2.427277in}}{\pgfqpoint{3.017657in}{2.424005in}}{\pgfqpoint{3.025894in}{2.424005in}}%
\pgfpathclose%
\pgfusepath{stroke,fill}%
\end{pgfscope}%
\begin{pgfscope}%
\pgfpathrectangle{\pgfqpoint{0.100000in}{0.220728in}}{\pgfqpoint{3.696000in}{3.696000in}}%
\pgfusepath{clip}%
\pgfsetbuttcap%
\pgfsetroundjoin%
\definecolor{currentfill}{rgb}{0.121569,0.466667,0.705882}%
\pgfsetfillcolor{currentfill}%
\pgfsetfillopacity{0.789001}%
\pgfsetlinewidth{1.003750pt}%
\definecolor{currentstroke}{rgb}{0.121569,0.466667,0.705882}%
\pgfsetstrokecolor{currentstroke}%
\pgfsetstrokeopacity{0.789001}%
\pgfsetdash{}{0pt}%
\pgfpathmoveto{\pgfqpoint{1.248181in}{2.019327in}}%
\pgfpathcurveto{\pgfqpoint{1.256418in}{2.019327in}}{\pgfqpoint{1.264318in}{2.022599in}}{\pgfqpoint{1.270142in}{2.028423in}}%
\pgfpathcurveto{\pgfqpoint{1.275965in}{2.034247in}}{\pgfqpoint{1.279238in}{2.042147in}}{\pgfqpoint{1.279238in}{2.050383in}}%
\pgfpathcurveto{\pgfqpoint{1.279238in}{2.058620in}}{\pgfqpoint{1.275965in}{2.066520in}}{\pgfqpoint{1.270142in}{2.072344in}}%
\pgfpathcurveto{\pgfqpoint{1.264318in}{2.078167in}}{\pgfqpoint{1.256418in}{2.081440in}}{\pgfqpoint{1.248181in}{2.081440in}}%
\pgfpathcurveto{\pgfqpoint{1.239945in}{2.081440in}}{\pgfqpoint{1.232045in}{2.078167in}}{\pgfqpoint{1.226221in}{2.072344in}}%
\pgfpathcurveto{\pgfqpoint{1.220397in}{2.066520in}}{\pgfqpoint{1.217125in}{2.058620in}}{\pgfqpoint{1.217125in}{2.050383in}}%
\pgfpathcurveto{\pgfqpoint{1.217125in}{2.042147in}}{\pgfqpoint{1.220397in}{2.034247in}}{\pgfqpoint{1.226221in}{2.028423in}}%
\pgfpathcurveto{\pgfqpoint{1.232045in}{2.022599in}}{\pgfqpoint{1.239945in}{2.019327in}}{\pgfqpoint{1.248181in}{2.019327in}}%
\pgfpathclose%
\pgfusepath{stroke,fill}%
\end{pgfscope}%
\begin{pgfscope}%
\pgfpathrectangle{\pgfqpoint{0.100000in}{0.220728in}}{\pgfqpoint{3.696000in}{3.696000in}}%
\pgfusepath{clip}%
\pgfsetbuttcap%
\pgfsetroundjoin%
\definecolor{currentfill}{rgb}{0.121569,0.466667,0.705882}%
\pgfsetfillcolor{currentfill}%
\pgfsetfillopacity{0.789107}%
\pgfsetlinewidth{1.003750pt}%
\definecolor{currentstroke}{rgb}{0.121569,0.466667,0.705882}%
\pgfsetstrokecolor{currentstroke}%
\pgfsetstrokeopacity{0.789107}%
\pgfsetdash{}{0pt}%
\pgfpathmoveto{\pgfqpoint{3.024556in}{2.421632in}}%
\pgfpathcurveto{\pgfqpoint{3.032792in}{2.421632in}}{\pgfqpoint{3.040692in}{2.424904in}}{\pgfqpoint{3.046516in}{2.430728in}}%
\pgfpathcurveto{\pgfqpoint{3.052340in}{2.436552in}}{\pgfqpoint{3.055612in}{2.444452in}}{\pgfqpoint{3.055612in}{2.452688in}}%
\pgfpathcurveto{\pgfqpoint{3.055612in}{2.460925in}}{\pgfqpoint{3.052340in}{2.468825in}}{\pgfqpoint{3.046516in}{2.474648in}}%
\pgfpathcurveto{\pgfqpoint{3.040692in}{2.480472in}}{\pgfqpoint{3.032792in}{2.483745in}}{\pgfqpoint{3.024556in}{2.483745in}}%
\pgfpathcurveto{\pgfqpoint{3.016319in}{2.483745in}}{\pgfqpoint{3.008419in}{2.480472in}}{\pgfqpoint{3.002595in}{2.474648in}}%
\pgfpathcurveto{\pgfqpoint{2.996771in}{2.468825in}}{\pgfqpoint{2.993499in}{2.460925in}}{\pgfqpoint{2.993499in}{2.452688in}}%
\pgfpathcurveto{\pgfqpoint{2.993499in}{2.444452in}}{\pgfqpoint{2.996771in}{2.436552in}}{\pgfqpoint{3.002595in}{2.430728in}}%
\pgfpathcurveto{\pgfqpoint{3.008419in}{2.424904in}}{\pgfqpoint{3.016319in}{2.421632in}}{\pgfqpoint{3.024556in}{2.421632in}}%
\pgfpathclose%
\pgfusepath{stroke,fill}%
\end{pgfscope}%
\begin{pgfscope}%
\pgfpathrectangle{\pgfqpoint{0.100000in}{0.220728in}}{\pgfqpoint{3.696000in}{3.696000in}}%
\pgfusepath{clip}%
\pgfsetbuttcap%
\pgfsetroundjoin%
\definecolor{currentfill}{rgb}{0.121569,0.466667,0.705882}%
\pgfsetfillcolor{currentfill}%
\pgfsetfillopacity{0.790391}%
\pgfsetlinewidth{1.003750pt}%
\definecolor{currentstroke}{rgb}{0.121569,0.466667,0.705882}%
\pgfsetstrokecolor{currentstroke}%
\pgfsetstrokeopacity{0.790391}%
\pgfsetdash{}{0pt}%
\pgfpathmoveto{\pgfqpoint{3.019687in}{2.415518in}}%
\pgfpathcurveto{\pgfqpoint{3.027924in}{2.415518in}}{\pgfqpoint{3.035824in}{2.418790in}}{\pgfqpoint{3.041648in}{2.424614in}}%
\pgfpathcurveto{\pgfqpoint{3.047472in}{2.430438in}}{\pgfqpoint{3.050744in}{2.438338in}}{\pgfqpoint{3.050744in}{2.446574in}}%
\pgfpathcurveto{\pgfqpoint{3.050744in}{2.454810in}}{\pgfqpoint{3.047472in}{2.462711in}}{\pgfqpoint{3.041648in}{2.468534in}}%
\pgfpathcurveto{\pgfqpoint{3.035824in}{2.474358in}}{\pgfqpoint{3.027924in}{2.477631in}}{\pgfqpoint{3.019687in}{2.477631in}}%
\pgfpathcurveto{\pgfqpoint{3.011451in}{2.477631in}}{\pgfqpoint{3.003551in}{2.474358in}}{\pgfqpoint{2.997727in}{2.468534in}}%
\pgfpathcurveto{\pgfqpoint{2.991903in}{2.462711in}}{\pgfqpoint{2.988631in}{2.454810in}}{\pgfqpoint{2.988631in}{2.446574in}}%
\pgfpathcurveto{\pgfqpoint{2.988631in}{2.438338in}}{\pgfqpoint{2.991903in}{2.430438in}}{\pgfqpoint{2.997727in}{2.424614in}}%
\pgfpathcurveto{\pgfqpoint{3.003551in}{2.418790in}}{\pgfqpoint{3.011451in}{2.415518in}}{\pgfqpoint{3.019687in}{2.415518in}}%
\pgfpathclose%
\pgfusepath{stroke,fill}%
\end{pgfscope}%
\begin{pgfscope}%
\pgfpathrectangle{\pgfqpoint{0.100000in}{0.220728in}}{\pgfqpoint{3.696000in}{3.696000in}}%
\pgfusepath{clip}%
\pgfsetbuttcap%
\pgfsetroundjoin%
\definecolor{currentfill}{rgb}{0.121569,0.466667,0.705882}%
\pgfsetfillcolor{currentfill}%
\pgfsetfillopacity{0.792204}%
\pgfsetlinewidth{1.003750pt}%
\definecolor{currentstroke}{rgb}{0.121569,0.466667,0.705882}%
\pgfsetstrokecolor{currentstroke}%
\pgfsetstrokeopacity{0.792204}%
\pgfsetdash{}{0pt}%
\pgfpathmoveto{\pgfqpoint{3.014584in}{2.405583in}}%
\pgfpathcurveto{\pgfqpoint{3.022821in}{2.405583in}}{\pgfqpoint{3.030721in}{2.408856in}}{\pgfqpoint{3.036545in}{2.414679in}}%
\pgfpathcurveto{\pgfqpoint{3.042369in}{2.420503in}}{\pgfqpoint{3.045641in}{2.428403in}}{\pgfqpoint{3.045641in}{2.436640in}}%
\pgfpathcurveto{\pgfqpoint{3.045641in}{2.444876in}}{\pgfqpoint{3.042369in}{2.452776in}}{\pgfqpoint{3.036545in}{2.458600in}}%
\pgfpathcurveto{\pgfqpoint{3.030721in}{2.464424in}}{\pgfqpoint{3.022821in}{2.467696in}}{\pgfqpoint{3.014584in}{2.467696in}}%
\pgfpathcurveto{\pgfqpoint{3.006348in}{2.467696in}}{\pgfqpoint{2.998448in}{2.464424in}}{\pgfqpoint{2.992624in}{2.458600in}}%
\pgfpathcurveto{\pgfqpoint{2.986800in}{2.452776in}}{\pgfqpoint{2.983528in}{2.444876in}}{\pgfqpoint{2.983528in}{2.436640in}}%
\pgfpathcurveto{\pgfqpoint{2.983528in}{2.428403in}}{\pgfqpoint{2.986800in}{2.420503in}}{\pgfqpoint{2.992624in}{2.414679in}}%
\pgfpathcurveto{\pgfqpoint{2.998448in}{2.408856in}}{\pgfqpoint{3.006348in}{2.405583in}}{\pgfqpoint{3.014584in}{2.405583in}}%
\pgfpathclose%
\pgfusepath{stroke,fill}%
\end{pgfscope}%
\begin{pgfscope}%
\pgfpathrectangle{\pgfqpoint{0.100000in}{0.220728in}}{\pgfqpoint{3.696000in}{3.696000in}}%
\pgfusepath{clip}%
\pgfsetbuttcap%
\pgfsetroundjoin%
\definecolor{currentfill}{rgb}{0.121569,0.466667,0.705882}%
\pgfsetfillcolor{currentfill}%
\pgfsetfillopacity{0.792504}%
\pgfsetlinewidth{1.003750pt}%
\definecolor{currentstroke}{rgb}{0.121569,0.466667,0.705882}%
\pgfsetstrokecolor{currentstroke}%
\pgfsetstrokeopacity{0.792504}%
\pgfsetdash{}{0pt}%
\pgfpathmoveto{\pgfqpoint{1.265641in}{2.009789in}}%
\pgfpathcurveto{\pgfqpoint{1.273877in}{2.009789in}}{\pgfqpoint{1.281778in}{2.013061in}}{\pgfqpoint{1.287601in}{2.018885in}}%
\pgfpathcurveto{\pgfqpoint{1.293425in}{2.024709in}}{\pgfqpoint{1.296698in}{2.032609in}}{\pgfqpoint{1.296698in}{2.040845in}}%
\pgfpathcurveto{\pgfqpoint{1.296698in}{2.049082in}}{\pgfqpoint{1.293425in}{2.056982in}}{\pgfqpoint{1.287601in}{2.062806in}}%
\pgfpathcurveto{\pgfqpoint{1.281778in}{2.068630in}}{\pgfqpoint{1.273877in}{2.071902in}}{\pgfqpoint{1.265641in}{2.071902in}}%
\pgfpathcurveto{\pgfqpoint{1.257405in}{2.071902in}}{\pgfqpoint{1.249505in}{2.068630in}}{\pgfqpoint{1.243681in}{2.062806in}}%
\pgfpathcurveto{\pgfqpoint{1.237857in}{2.056982in}}{\pgfqpoint{1.234585in}{2.049082in}}{\pgfqpoint{1.234585in}{2.040845in}}%
\pgfpathcurveto{\pgfqpoint{1.234585in}{2.032609in}}{\pgfqpoint{1.237857in}{2.024709in}}{\pgfqpoint{1.243681in}{2.018885in}}%
\pgfpathcurveto{\pgfqpoint{1.249505in}{2.013061in}}{\pgfqpoint{1.257405in}{2.009789in}}{\pgfqpoint{1.265641in}{2.009789in}}%
\pgfpathclose%
\pgfusepath{stroke,fill}%
\end{pgfscope}%
\begin{pgfscope}%
\pgfpathrectangle{\pgfqpoint{0.100000in}{0.220728in}}{\pgfqpoint{3.696000in}{3.696000in}}%
\pgfusepath{clip}%
\pgfsetbuttcap%
\pgfsetroundjoin%
\definecolor{currentfill}{rgb}{0.121569,0.466667,0.705882}%
\pgfsetfillcolor{currentfill}%
\pgfsetfillopacity{0.794428}%
\pgfsetlinewidth{1.003750pt}%
\definecolor{currentstroke}{rgb}{0.121569,0.466667,0.705882}%
\pgfsetstrokecolor{currentstroke}%
\pgfsetstrokeopacity{0.794428}%
\pgfsetdash{}{0pt}%
\pgfpathmoveto{\pgfqpoint{3.005319in}{2.392272in}}%
\pgfpathcurveto{\pgfqpoint{3.013555in}{2.392272in}}{\pgfqpoint{3.021455in}{2.395544in}}{\pgfqpoint{3.027279in}{2.401368in}}%
\pgfpathcurveto{\pgfqpoint{3.033103in}{2.407192in}}{\pgfqpoint{3.036375in}{2.415092in}}{\pgfqpoint{3.036375in}{2.423328in}}%
\pgfpathcurveto{\pgfqpoint{3.036375in}{2.431564in}}{\pgfqpoint{3.033103in}{2.439465in}}{\pgfqpoint{3.027279in}{2.445288in}}%
\pgfpathcurveto{\pgfqpoint{3.021455in}{2.451112in}}{\pgfqpoint{3.013555in}{2.454385in}}{\pgfqpoint{3.005319in}{2.454385in}}%
\pgfpathcurveto{\pgfqpoint{2.997082in}{2.454385in}}{\pgfqpoint{2.989182in}{2.451112in}}{\pgfqpoint{2.983358in}{2.445288in}}%
\pgfpathcurveto{\pgfqpoint{2.977534in}{2.439465in}}{\pgfqpoint{2.974262in}{2.431564in}}{\pgfqpoint{2.974262in}{2.423328in}}%
\pgfpathcurveto{\pgfqpoint{2.974262in}{2.415092in}}{\pgfqpoint{2.977534in}{2.407192in}}{\pgfqpoint{2.983358in}{2.401368in}}%
\pgfpathcurveto{\pgfqpoint{2.989182in}{2.395544in}}{\pgfqpoint{2.997082in}{2.392272in}}{\pgfqpoint{3.005319in}{2.392272in}}%
\pgfpathclose%
\pgfusepath{stroke,fill}%
\end{pgfscope}%
\begin{pgfscope}%
\pgfpathrectangle{\pgfqpoint{0.100000in}{0.220728in}}{\pgfqpoint{3.696000in}{3.696000in}}%
\pgfusepath{clip}%
\pgfsetbuttcap%
\pgfsetroundjoin%
\definecolor{currentfill}{rgb}{0.121569,0.466667,0.705882}%
\pgfsetfillcolor{currentfill}%
\pgfsetfillopacity{0.795323}%
\pgfsetlinewidth{1.003750pt}%
\definecolor{currentstroke}{rgb}{0.121569,0.466667,0.705882}%
\pgfsetstrokecolor{currentstroke}%
\pgfsetstrokeopacity{0.795323}%
\pgfsetdash{}{0pt}%
\pgfpathmoveto{\pgfqpoint{1.281922in}{2.001740in}}%
\pgfpathcurveto{\pgfqpoint{1.290158in}{2.001740in}}{\pgfqpoint{1.298058in}{2.005012in}}{\pgfqpoint{1.303882in}{2.010836in}}%
\pgfpathcurveto{\pgfqpoint{1.309706in}{2.016660in}}{\pgfqpoint{1.312978in}{2.024560in}}{\pgfqpoint{1.312978in}{2.032797in}}%
\pgfpathcurveto{\pgfqpoint{1.312978in}{2.041033in}}{\pgfqpoint{1.309706in}{2.048933in}}{\pgfqpoint{1.303882in}{2.054757in}}%
\pgfpathcurveto{\pgfqpoint{1.298058in}{2.060581in}}{\pgfqpoint{1.290158in}{2.063853in}}{\pgfqpoint{1.281922in}{2.063853in}}%
\pgfpathcurveto{\pgfqpoint{1.273686in}{2.063853in}}{\pgfqpoint{1.265785in}{2.060581in}}{\pgfqpoint{1.259962in}{2.054757in}}%
\pgfpathcurveto{\pgfqpoint{1.254138in}{2.048933in}}{\pgfqpoint{1.250865in}{2.041033in}}{\pgfqpoint{1.250865in}{2.032797in}}%
\pgfpathcurveto{\pgfqpoint{1.250865in}{2.024560in}}{\pgfqpoint{1.254138in}{2.016660in}}{\pgfqpoint{1.259962in}{2.010836in}}%
\pgfpathcurveto{\pgfqpoint{1.265785in}{2.005012in}}{\pgfqpoint{1.273686in}{2.001740in}}{\pgfqpoint{1.281922in}{2.001740in}}%
\pgfpathclose%
\pgfusepath{stroke,fill}%
\end{pgfscope}%
\begin{pgfscope}%
\pgfpathrectangle{\pgfqpoint{0.100000in}{0.220728in}}{\pgfqpoint{3.696000in}{3.696000in}}%
\pgfusepath{clip}%
\pgfsetbuttcap%
\pgfsetroundjoin%
\definecolor{currentfill}{rgb}{0.121569,0.466667,0.705882}%
\pgfsetfillcolor{currentfill}%
\pgfsetfillopacity{0.797072}%
\pgfsetlinewidth{1.003750pt}%
\definecolor{currentstroke}{rgb}{0.121569,0.466667,0.705882}%
\pgfsetstrokecolor{currentstroke}%
\pgfsetstrokeopacity{0.797072}%
\pgfsetdash{}{0pt}%
\pgfpathmoveto{\pgfqpoint{1.292660in}{1.998071in}}%
\pgfpathcurveto{\pgfqpoint{1.300896in}{1.998071in}}{\pgfqpoint{1.308796in}{2.001343in}}{\pgfqpoint{1.314620in}{2.007167in}}%
\pgfpathcurveto{\pgfqpoint{1.320444in}{2.012991in}}{\pgfqpoint{1.323716in}{2.020891in}}{\pgfqpoint{1.323716in}{2.029127in}}%
\pgfpathcurveto{\pgfqpoint{1.323716in}{2.037363in}}{\pgfqpoint{1.320444in}{2.045263in}}{\pgfqpoint{1.314620in}{2.051087in}}%
\pgfpathcurveto{\pgfqpoint{1.308796in}{2.056911in}}{\pgfqpoint{1.300896in}{2.060184in}}{\pgfqpoint{1.292660in}{2.060184in}}%
\pgfpathcurveto{\pgfqpoint{1.284424in}{2.060184in}}{\pgfqpoint{1.276524in}{2.056911in}}{\pgfqpoint{1.270700in}{2.051087in}}%
\pgfpathcurveto{\pgfqpoint{1.264876in}{2.045263in}}{\pgfqpoint{1.261603in}{2.037363in}}{\pgfqpoint{1.261603in}{2.029127in}}%
\pgfpathcurveto{\pgfqpoint{1.261603in}{2.020891in}}{\pgfqpoint{1.264876in}{2.012991in}}{\pgfqpoint{1.270700in}{2.007167in}}%
\pgfpathcurveto{\pgfqpoint{1.276524in}{2.001343in}}{\pgfqpoint{1.284424in}{1.998071in}}{\pgfqpoint{1.292660in}{1.998071in}}%
\pgfpathclose%
\pgfusepath{stroke,fill}%
\end{pgfscope}%
\begin{pgfscope}%
\pgfpathrectangle{\pgfqpoint{0.100000in}{0.220728in}}{\pgfqpoint{3.696000in}{3.696000in}}%
\pgfusepath{clip}%
\pgfsetbuttcap%
\pgfsetroundjoin%
\definecolor{currentfill}{rgb}{0.121569,0.466667,0.705882}%
\pgfsetfillcolor{currentfill}%
\pgfsetfillopacity{0.797933}%
\pgfsetlinewidth{1.003750pt}%
\definecolor{currentstroke}{rgb}{0.121569,0.466667,0.705882}%
\pgfsetstrokecolor{currentstroke}%
\pgfsetstrokeopacity{0.797933}%
\pgfsetdash{}{0pt}%
\pgfpathmoveto{\pgfqpoint{1.297348in}{1.995620in}}%
\pgfpathcurveto{\pgfqpoint{1.305584in}{1.995620in}}{\pgfqpoint{1.313484in}{1.998892in}}{\pgfqpoint{1.319308in}{2.004716in}}%
\pgfpathcurveto{\pgfqpoint{1.325132in}{2.010540in}}{\pgfqpoint{1.328404in}{2.018440in}}{\pgfqpoint{1.328404in}{2.026676in}}%
\pgfpathcurveto{\pgfqpoint{1.328404in}{2.034912in}}{\pgfqpoint{1.325132in}{2.042812in}}{\pgfqpoint{1.319308in}{2.048636in}}%
\pgfpathcurveto{\pgfqpoint{1.313484in}{2.054460in}}{\pgfqpoint{1.305584in}{2.057733in}}{\pgfqpoint{1.297348in}{2.057733in}}%
\pgfpathcurveto{\pgfqpoint{1.289111in}{2.057733in}}{\pgfqpoint{1.281211in}{2.054460in}}{\pgfqpoint{1.275387in}{2.048636in}}%
\pgfpathcurveto{\pgfqpoint{1.269563in}{2.042812in}}{\pgfqpoint{1.266291in}{2.034912in}}{\pgfqpoint{1.266291in}{2.026676in}}%
\pgfpathcurveto{\pgfqpoint{1.266291in}{2.018440in}}{\pgfqpoint{1.269563in}{2.010540in}}{\pgfqpoint{1.275387in}{2.004716in}}%
\pgfpathcurveto{\pgfqpoint{1.281211in}{1.998892in}}{\pgfqpoint{1.289111in}{1.995620in}}{\pgfqpoint{1.297348in}{1.995620in}}%
\pgfpathclose%
\pgfusepath{stroke,fill}%
\end{pgfscope}%
\begin{pgfscope}%
\pgfpathrectangle{\pgfqpoint{0.100000in}{0.220728in}}{\pgfqpoint{3.696000in}{3.696000in}}%
\pgfusepath{clip}%
\pgfsetbuttcap%
\pgfsetroundjoin%
\definecolor{currentfill}{rgb}{0.121569,0.466667,0.705882}%
\pgfsetfillcolor{currentfill}%
\pgfsetfillopacity{0.798063}%
\pgfsetlinewidth{1.003750pt}%
\definecolor{currentstroke}{rgb}{0.121569,0.466667,0.705882}%
\pgfsetstrokecolor{currentstroke}%
\pgfsetstrokeopacity{0.798063}%
\pgfsetdash{}{0pt}%
\pgfpathmoveto{\pgfqpoint{2.995727in}{2.374541in}}%
\pgfpathcurveto{\pgfqpoint{3.003963in}{2.374541in}}{\pgfqpoint{3.011863in}{2.377813in}}{\pgfqpoint{3.017687in}{2.383637in}}%
\pgfpathcurveto{\pgfqpoint{3.023511in}{2.389461in}}{\pgfqpoint{3.026783in}{2.397361in}}{\pgfqpoint{3.026783in}{2.405597in}}%
\pgfpathcurveto{\pgfqpoint{3.026783in}{2.413833in}}{\pgfqpoint{3.023511in}{2.421733in}}{\pgfqpoint{3.017687in}{2.427557in}}%
\pgfpathcurveto{\pgfqpoint{3.011863in}{2.433381in}}{\pgfqpoint{3.003963in}{2.436654in}}{\pgfqpoint{2.995727in}{2.436654in}}%
\pgfpathcurveto{\pgfqpoint{2.987490in}{2.436654in}}{\pgfqpoint{2.979590in}{2.433381in}}{\pgfqpoint{2.973766in}{2.427557in}}%
\pgfpathcurveto{\pgfqpoint{2.967942in}{2.421733in}}{\pgfqpoint{2.964670in}{2.413833in}}{\pgfqpoint{2.964670in}{2.405597in}}%
\pgfpathcurveto{\pgfqpoint{2.964670in}{2.397361in}}{\pgfqpoint{2.967942in}{2.389461in}}{\pgfqpoint{2.973766in}{2.383637in}}%
\pgfpathcurveto{\pgfqpoint{2.979590in}{2.377813in}}{\pgfqpoint{2.987490in}{2.374541in}}{\pgfqpoint{2.995727in}{2.374541in}}%
\pgfpathclose%
\pgfusepath{stroke,fill}%
\end{pgfscope}%
\begin{pgfscope}%
\pgfpathrectangle{\pgfqpoint{0.100000in}{0.220728in}}{\pgfqpoint{3.696000in}{3.696000in}}%
\pgfusepath{clip}%
\pgfsetbuttcap%
\pgfsetroundjoin%
\definecolor{currentfill}{rgb}{0.121569,0.466667,0.705882}%
\pgfsetfillcolor{currentfill}%
\pgfsetfillopacity{0.798094}%
\pgfsetlinewidth{1.003750pt}%
\definecolor{currentstroke}{rgb}{0.121569,0.466667,0.705882}%
\pgfsetstrokecolor{currentstroke}%
\pgfsetstrokeopacity{0.798094}%
\pgfsetdash{}{0pt}%
\pgfpathmoveto{\pgfqpoint{1.298454in}{1.995458in}}%
\pgfpathcurveto{\pgfqpoint{1.306690in}{1.995458in}}{\pgfqpoint{1.314590in}{1.998731in}}{\pgfqpoint{1.320414in}{2.004554in}}%
\pgfpathcurveto{\pgfqpoint{1.326238in}{2.010378in}}{\pgfqpoint{1.329510in}{2.018278in}}{\pgfqpoint{1.329510in}{2.026515in}}%
\pgfpathcurveto{\pgfqpoint{1.329510in}{2.034751in}}{\pgfqpoint{1.326238in}{2.042651in}}{\pgfqpoint{1.320414in}{2.048475in}}%
\pgfpathcurveto{\pgfqpoint{1.314590in}{2.054299in}}{\pgfqpoint{1.306690in}{2.057571in}}{\pgfqpoint{1.298454in}{2.057571in}}%
\pgfpathcurveto{\pgfqpoint{1.290217in}{2.057571in}}{\pgfqpoint{1.282317in}{2.054299in}}{\pgfqpoint{1.276493in}{2.048475in}}%
\pgfpathcurveto{\pgfqpoint{1.270669in}{2.042651in}}{\pgfqpoint{1.267397in}{2.034751in}}{\pgfqpoint{1.267397in}{2.026515in}}%
\pgfpathcurveto{\pgfqpoint{1.267397in}{2.018278in}}{\pgfqpoint{1.270669in}{2.010378in}}{\pgfqpoint{1.276493in}{2.004554in}}%
\pgfpathcurveto{\pgfqpoint{1.282317in}{1.998731in}}{\pgfqpoint{1.290217in}{1.995458in}}{\pgfqpoint{1.298454in}{1.995458in}}%
\pgfpathclose%
\pgfusepath{stroke,fill}%
\end{pgfscope}%
\begin{pgfscope}%
\pgfpathrectangle{\pgfqpoint{0.100000in}{0.220728in}}{\pgfqpoint{3.696000in}{3.696000in}}%
\pgfusepath{clip}%
\pgfsetbuttcap%
\pgfsetroundjoin%
\definecolor{currentfill}{rgb}{0.121569,0.466667,0.705882}%
\pgfsetfillcolor{currentfill}%
\pgfsetfillopacity{0.798389}%
\pgfsetlinewidth{1.003750pt}%
\definecolor{currentstroke}{rgb}{0.121569,0.466667,0.705882}%
\pgfsetstrokecolor{currentstroke}%
\pgfsetstrokeopacity{0.798389}%
\pgfsetdash{}{0pt}%
\pgfpathmoveto{\pgfqpoint{1.300146in}{1.994113in}}%
\pgfpathcurveto{\pgfqpoint{1.308383in}{1.994113in}}{\pgfqpoint{1.316283in}{1.997385in}}{\pgfqpoint{1.322107in}{2.003209in}}%
\pgfpathcurveto{\pgfqpoint{1.327931in}{2.009033in}}{\pgfqpoint{1.331203in}{2.016933in}}{\pgfqpoint{1.331203in}{2.025169in}}%
\pgfpathcurveto{\pgfqpoint{1.331203in}{2.033405in}}{\pgfqpoint{1.327931in}{2.041305in}}{\pgfqpoint{1.322107in}{2.047129in}}%
\pgfpathcurveto{\pgfqpoint{1.316283in}{2.052953in}}{\pgfqpoint{1.308383in}{2.056226in}}{\pgfqpoint{1.300146in}{2.056226in}}%
\pgfpathcurveto{\pgfqpoint{1.291910in}{2.056226in}}{\pgfqpoint{1.284010in}{2.052953in}}{\pgfqpoint{1.278186in}{2.047129in}}%
\pgfpathcurveto{\pgfqpoint{1.272362in}{2.041305in}}{\pgfqpoint{1.269090in}{2.033405in}}{\pgfqpoint{1.269090in}{2.025169in}}%
\pgfpathcurveto{\pgfqpoint{1.269090in}{2.016933in}}{\pgfqpoint{1.272362in}{2.009033in}}{\pgfqpoint{1.278186in}{2.003209in}}%
\pgfpathcurveto{\pgfqpoint{1.284010in}{1.997385in}}{\pgfqpoint{1.291910in}{1.994113in}}{\pgfqpoint{1.300146in}{1.994113in}}%
\pgfpathclose%
\pgfusepath{stroke,fill}%
\end{pgfscope}%
\begin{pgfscope}%
\pgfpathrectangle{\pgfqpoint{0.100000in}{0.220728in}}{\pgfqpoint{3.696000in}{3.696000in}}%
\pgfusepath{clip}%
\pgfsetbuttcap%
\pgfsetroundjoin%
\definecolor{currentfill}{rgb}{0.121569,0.466667,0.705882}%
\pgfsetfillcolor{currentfill}%
\pgfsetfillopacity{0.799005}%
\pgfsetlinewidth{1.003750pt}%
\definecolor{currentstroke}{rgb}{0.121569,0.466667,0.705882}%
\pgfsetstrokecolor{currentstroke}%
\pgfsetstrokeopacity{0.799005}%
\pgfsetdash{}{0pt}%
\pgfpathmoveto{\pgfqpoint{1.303655in}{1.993419in}}%
\pgfpathcurveto{\pgfqpoint{1.311891in}{1.993419in}}{\pgfqpoint{1.319791in}{1.996691in}}{\pgfqpoint{1.325615in}{2.002515in}}%
\pgfpathcurveto{\pgfqpoint{1.331439in}{2.008339in}}{\pgfqpoint{1.334712in}{2.016239in}}{\pgfqpoint{1.334712in}{2.024475in}}%
\pgfpathcurveto{\pgfqpoint{1.334712in}{2.032712in}}{\pgfqpoint{1.331439in}{2.040612in}}{\pgfqpoint{1.325615in}{2.046436in}}%
\pgfpathcurveto{\pgfqpoint{1.319791in}{2.052259in}}{\pgfqpoint{1.311891in}{2.055532in}}{\pgfqpoint{1.303655in}{2.055532in}}%
\pgfpathcurveto{\pgfqpoint{1.295419in}{2.055532in}}{\pgfqpoint{1.287519in}{2.052259in}}{\pgfqpoint{1.281695in}{2.046436in}}%
\pgfpathcurveto{\pgfqpoint{1.275871in}{2.040612in}}{\pgfqpoint{1.272599in}{2.032712in}}{\pgfqpoint{1.272599in}{2.024475in}}%
\pgfpathcurveto{\pgfqpoint{1.272599in}{2.016239in}}{\pgfqpoint{1.275871in}{2.008339in}}{\pgfqpoint{1.281695in}{2.002515in}}%
\pgfpathcurveto{\pgfqpoint{1.287519in}{1.996691in}}{\pgfqpoint{1.295419in}{1.993419in}}{\pgfqpoint{1.303655in}{1.993419in}}%
\pgfpathclose%
\pgfusepath{stroke,fill}%
\end{pgfscope}%
\begin{pgfscope}%
\pgfpathrectangle{\pgfqpoint{0.100000in}{0.220728in}}{\pgfqpoint{3.696000in}{3.696000in}}%
\pgfusepath{clip}%
\pgfsetbuttcap%
\pgfsetroundjoin%
\definecolor{currentfill}{rgb}{0.121569,0.466667,0.705882}%
\pgfsetfillcolor{currentfill}%
\pgfsetfillopacity{0.800031}%
\pgfsetlinewidth{1.003750pt}%
\definecolor{currentstroke}{rgb}{0.121569,0.466667,0.705882}%
\pgfsetstrokecolor{currentstroke}%
\pgfsetstrokeopacity{0.800031}%
\pgfsetdash{}{0pt}%
\pgfpathmoveto{\pgfqpoint{1.309460in}{1.989710in}}%
\pgfpathcurveto{\pgfqpoint{1.317696in}{1.989710in}}{\pgfqpoint{1.325596in}{1.992982in}}{\pgfqpoint{1.331420in}{1.998806in}}%
\pgfpathcurveto{\pgfqpoint{1.337244in}{2.004630in}}{\pgfqpoint{1.340516in}{2.012530in}}{\pgfqpoint{1.340516in}{2.020767in}}%
\pgfpathcurveto{\pgfqpoint{1.340516in}{2.029003in}}{\pgfqpoint{1.337244in}{2.036903in}}{\pgfqpoint{1.331420in}{2.042727in}}%
\pgfpathcurveto{\pgfqpoint{1.325596in}{2.048551in}}{\pgfqpoint{1.317696in}{2.051823in}}{\pgfqpoint{1.309460in}{2.051823in}}%
\pgfpathcurveto{\pgfqpoint{1.301224in}{2.051823in}}{\pgfqpoint{1.293324in}{2.048551in}}{\pgfqpoint{1.287500in}{2.042727in}}%
\pgfpathcurveto{\pgfqpoint{1.281676in}{2.036903in}}{\pgfqpoint{1.278403in}{2.029003in}}{\pgfqpoint{1.278403in}{2.020767in}}%
\pgfpathcurveto{\pgfqpoint{1.278403in}{2.012530in}}{\pgfqpoint{1.281676in}{2.004630in}}{\pgfqpoint{1.287500in}{1.998806in}}%
\pgfpathcurveto{\pgfqpoint{1.293324in}{1.992982in}}{\pgfqpoint{1.301224in}{1.989710in}}{\pgfqpoint{1.309460in}{1.989710in}}%
\pgfpathclose%
\pgfusepath{stroke,fill}%
\end{pgfscope}%
\begin{pgfscope}%
\pgfpathrectangle{\pgfqpoint{0.100000in}{0.220728in}}{\pgfqpoint{3.696000in}{3.696000in}}%
\pgfusepath{clip}%
\pgfsetbuttcap%
\pgfsetroundjoin%
\definecolor{currentfill}{rgb}{0.121569,0.466667,0.705882}%
\pgfsetfillcolor{currentfill}%
\pgfsetfillopacity{0.801752}%
\pgfsetlinewidth{1.003750pt}%
\definecolor{currentstroke}{rgb}{0.121569,0.466667,0.705882}%
\pgfsetstrokecolor{currentstroke}%
\pgfsetstrokeopacity{0.801752}%
\pgfsetdash{}{0pt}%
\pgfpathmoveto{\pgfqpoint{2.981172in}{2.354298in}}%
\pgfpathcurveto{\pgfqpoint{2.989408in}{2.354298in}}{\pgfqpoint{2.997309in}{2.357570in}}{\pgfqpoint{3.003132in}{2.363394in}}%
\pgfpathcurveto{\pgfqpoint{3.008956in}{2.369218in}}{\pgfqpoint{3.012229in}{2.377118in}}{\pgfqpoint{3.012229in}{2.385354in}}%
\pgfpathcurveto{\pgfqpoint{3.012229in}{2.393590in}}{\pgfqpoint{3.008956in}{2.401490in}}{\pgfqpoint{3.003132in}{2.407314in}}%
\pgfpathcurveto{\pgfqpoint{2.997309in}{2.413138in}}{\pgfqpoint{2.989408in}{2.416411in}}{\pgfqpoint{2.981172in}{2.416411in}}%
\pgfpathcurveto{\pgfqpoint{2.972936in}{2.416411in}}{\pgfqpoint{2.965036in}{2.413138in}}{\pgfqpoint{2.959212in}{2.407314in}}%
\pgfpathcurveto{\pgfqpoint{2.953388in}{2.401490in}}{\pgfqpoint{2.950116in}{2.393590in}}{\pgfqpoint{2.950116in}{2.385354in}}%
\pgfpathcurveto{\pgfqpoint{2.950116in}{2.377118in}}{\pgfqpoint{2.953388in}{2.369218in}}{\pgfqpoint{2.959212in}{2.363394in}}%
\pgfpathcurveto{\pgfqpoint{2.965036in}{2.357570in}}{\pgfqpoint{2.972936in}{2.354298in}}{\pgfqpoint{2.981172in}{2.354298in}}%
\pgfpathclose%
\pgfusepath{stroke,fill}%
\end{pgfscope}%
\begin{pgfscope}%
\pgfpathrectangle{\pgfqpoint{0.100000in}{0.220728in}}{\pgfqpoint{3.696000in}{3.696000in}}%
\pgfusepath{clip}%
\pgfsetbuttcap%
\pgfsetroundjoin%
\definecolor{currentfill}{rgb}{0.121569,0.466667,0.705882}%
\pgfsetfillcolor{currentfill}%
\pgfsetfillopacity{0.801801}%
\pgfsetlinewidth{1.003750pt}%
\definecolor{currentstroke}{rgb}{0.121569,0.466667,0.705882}%
\pgfsetstrokecolor{currentstroke}%
\pgfsetstrokeopacity{0.801801}%
\pgfsetdash{}{0pt}%
\pgfpathmoveto{\pgfqpoint{1.320780in}{1.985063in}}%
\pgfpathcurveto{\pgfqpoint{1.329016in}{1.985063in}}{\pgfqpoint{1.336916in}{1.988336in}}{\pgfqpoint{1.342740in}{1.994160in}}%
\pgfpathcurveto{\pgfqpoint{1.348564in}{1.999984in}}{\pgfqpoint{1.351836in}{2.007884in}}{\pgfqpoint{1.351836in}{2.016120in}}%
\pgfpathcurveto{\pgfqpoint{1.351836in}{2.024356in}}{\pgfqpoint{1.348564in}{2.032256in}}{\pgfqpoint{1.342740in}{2.038080in}}%
\pgfpathcurveto{\pgfqpoint{1.336916in}{2.043904in}}{\pgfqpoint{1.329016in}{2.047176in}}{\pgfqpoint{1.320780in}{2.047176in}}%
\pgfpathcurveto{\pgfqpoint{1.312544in}{2.047176in}}{\pgfqpoint{1.304644in}{2.043904in}}{\pgfqpoint{1.298820in}{2.038080in}}%
\pgfpathcurveto{\pgfqpoint{1.292996in}{2.032256in}}{\pgfqpoint{1.289723in}{2.024356in}}{\pgfqpoint{1.289723in}{2.016120in}}%
\pgfpathcurveto{\pgfqpoint{1.289723in}{2.007884in}}{\pgfqpoint{1.292996in}{1.999984in}}{\pgfqpoint{1.298820in}{1.994160in}}%
\pgfpathcurveto{\pgfqpoint{1.304644in}{1.988336in}}{\pgfqpoint{1.312544in}{1.985063in}}{\pgfqpoint{1.320780in}{1.985063in}}%
\pgfpathclose%
\pgfusepath{stroke,fill}%
\end{pgfscope}%
\begin{pgfscope}%
\pgfpathrectangle{\pgfqpoint{0.100000in}{0.220728in}}{\pgfqpoint{3.696000in}{3.696000in}}%
\pgfusepath{clip}%
\pgfsetbuttcap%
\pgfsetroundjoin%
\definecolor{currentfill}{rgb}{0.121569,0.466667,0.705882}%
\pgfsetfillcolor{currentfill}%
\pgfsetfillopacity{0.804264}%
\pgfsetlinewidth{1.003750pt}%
\definecolor{currentstroke}{rgb}{0.121569,0.466667,0.705882}%
\pgfsetstrokecolor{currentstroke}%
\pgfsetstrokeopacity{0.804264}%
\pgfsetdash{}{0pt}%
\pgfpathmoveto{\pgfqpoint{2.974544in}{2.343583in}}%
\pgfpathcurveto{\pgfqpoint{2.982780in}{2.343583in}}{\pgfqpoint{2.990680in}{2.346855in}}{\pgfqpoint{2.996504in}{2.352679in}}%
\pgfpathcurveto{\pgfqpoint{3.002328in}{2.358503in}}{\pgfqpoint{3.005600in}{2.366403in}}{\pgfqpoint{3.005600in}{2.374640in}}%
\pgfpathcurveto{\pgfqpoint{3.005600in}{2.382876in}}{\pgfqpoint{3.002328in}{2.390776in}}{\pgfqpoint{2.996504in}{2.396600in}}%
\pgfpathcurveto{\pgfqpoint{2.990680in}{2.402424in}}{\pgfqpoint{2.982780in}{2.405696in}}{\pgfqpoint{2.974544in}{2.405696in}}%
\pgfpathcurveto{\pgfqpoint{2.966307in}{2.405696in}}{\pgfqpoint{2.958407in}{2.402424in}}{\pgfqpoint{2.952583in}{2.396600in}}%
\pgfpathcurveto{\pgfqpoint{2.946760in}{2.390776in}}{\pgfqpoint{2.943487in}{2.382876in}}{\pgfqpoint{2.943487in}{2.374640in}}%
\pgfpathcurveto{\pgfqpoint{2.943487in}{2.366403in}}{\pgfqpoint{2.946760in}{2.358503in}}{\pgfqpoint{2.952583in}{2.352679in}}%
\pgfpathcurveto{\pgfqpoint{2.958407in}{2.346855in}}{\pgfqpoint{2.966307in}{2.343583in}}{\pgfqpoint{2.974544in}{2.343583in}}%
\pgfpathclose%
\pgfusepath{stroke,fill}%
\end{pgfscope}%
\begin{pgfscope}%
\pgfpathrectangle{\pgfqpoint{0.100000in}{0.220728in}}{\pgfqpoint{3.696000in}{3.696000in}}%
\pgfusepath{clip}%
\pgfsetbuttcap%
\pgfsetroundjoin%
\definecolor{currentfill}{rgb}{0.121569,0.466667,0.705882}%
\pgfsetfillcolor{currentfill}%
\pgfsetfillopacity{0.805125}%
\pgfsetlinewidth{1.003750pt}%
\definecolor{currentstroke}{rgb}{0.121569,0.466667,0.705882}%
\pgfsetstrokecolor{currentstroke}%
\pgfsetstrokeopacity{0.805125}%
\pgfsetdash{}{0pt}%
\pgfpathmoveto{\pgfqpoint{1.340080in}{1.972848in}}%
\pgfpathcurveto{\pgfqpoint{1.348316in}{1.972848in}}{\pgfqpoint{1.356216in}{1.976121in}}{\pgfqpoint{1.362040in}{1.981944in}}%
\pgfpathcurveto{\pgfqpoint{1.367864in}{1.987768in}}{\pgfqpoint{1.371136in}{1.995668in}}{\pgfqpoint{1.371136in}{2.003905in}}%
\pgfpathcurveto{\pgfqpoint{1.371136in}{2.012141in}}{\pgfqpoint{1.367864in}{2.020041in}}{\pgfqpoint{1.362040in}{2.025865in}}%
\pgfpathcurveto{\pgfqpoint{1.356216in}{2.031689in}}{\pgfqpoint{1.348316in}{2.034961in}}{\pgfqpoint{1.340080in}{2.034961in}}%
\pgfpathcurveto{\pgfqpoint{1.331843in}{2.034961in}}{\pgfqpoint{1.323943in}{2.031689in}}{\pgfqpoint{1.318119in}{2.025865in}}%
\pgfpathcurveto{\pgfqpoint{1.312295in}{2.020041in}}{\pgfqpoint{1.309023in}{2.012141in}}{\pgfqpoint{1.309023in}{2.003905in}}%
\pgfpathcurveto{\pgfqpoint{1.309023in}{1.995668in}}{\pgfqpoint{1.312295in}{1.987768in}}{\pgfqpoint{1.318119in}{1.981944in}}%
\pgfpathcurveto{\pgfqpoint{1.323943in}{1.976121in}}{\pgfqpoint{1.331843in}{1.972848in}}{\pgfqpoint{1.340080in}{1.972848in}}%
\pgfpathclose%
\pgfusepath{stroke,fill}%
\end{pgfscope}%
\begin{pgfscope}%
\pgfpathrectangle{\pgfqpoint{0.100000in}{0.220728in}}{\pgfqpoint{3.696000in}{3.696000in}}%
\pgfusepath{clip}%
\pgfsetbuttcap%
\pgfsetroundjoin%
\definecolor{currentfill}{rgb}{0.121569,0.466667,0.705882}%
\pgfsetfillcolor{currentfill}%
\pgfsetfillopacity{0.805570}%
\pgfsetlinewidth{1.003750pt}%
\definecolor{currentstroke}{rgb}{0.121569,0.466667,0.705882}%
\pgfsetstrokecolor{currentstroke}%
\pgfsetstrokeopacity{0.805570}%
\pgfsetdash{}{0pt}%
\pgfpathmoveto{\pgfqpoint{2.970775in}{2.337514in}}%
\pgfpathcurveto{\pgfqpoint{2.979011in}{2.337514in}}{\pgfqpoint{2.986911in}{2.340786in}}{\pgfqpoint{2.992735in}{2.346610in}}%
\pgfpathcurveto{\pgfqpoint{2.998559in}{2.352434in}}{\pgfqpoint{3.001832in}{2.360334in}}{\pgfqpoint{3.001832in}{2.368570in}}%
\pgfpathcurveto{\pgfqpoint{3.001832in}{2.376806in}}{\pgfqpoint{2.998559in}{2.384706in}}{\pgfqpoint{2.992735in}{2.390530in}}%
\pgfpathcurveto{\pgfqpoint{2.986911in}{2.396354in}}{\pgfqpoint{2.979011in}{2.399627in}}{\pgfqpoint{2.970775in}{2.399627in}}%
\pgfpathcurveto{\pgfqpoint{2.962539in}{2.399627in}}{\pgfqpoint{2.954639in}{2.396354in}}{\pgfqpoint{2.948815in}{2.390530in}}%
\pgfpathcurveto{\pgfqpoint{2.942991in}{2.384706in}}{\pgfqpoint{2.939719in}{2.376806in}}{\pgfqpoint{2.939719in}{2.368570in}}%
\pgfpathcurveto{\pgfqpoint{2.939719in}{2.360334in}}{\pgfqpoint{2.942991in}{2.352434in}}{\pgfqpoint{2.948815in}{2.346610in}}%
\pgfpathcurveto{\pgfqpoint{2.954639in}{2.340786in}}{\pgfqpoint{2.962539in}{2.337514in}}{\pgfqpoint{2.970775in}{2.337514in}}%
\pgfpathclose%
\pgfusepath{stroke,fill}%
\end{pgfscope}%
\begin{pgfscope}%
\pgfpathrectangle{\pgfqpoint{0.100000in}{0.220728in}}{\pgfqpoint{3.696000in}{3.696000in}}%
\pgfusepath{clip}%
\pgfsetbuttcap%
\pgfsetroundjoin%
\definecolor{currentfill}{rgb}{0.121569,0.466667,0.705882}%
\pgfsetfillcolor{currentfill}%
\pgfsetfillopacity{0.806224}%
\pgfsetlinewidth{1.003750pt}%
\definecolor{currentstroke}{rgb}{0.121569,0.466667,0.705882}%
\pgfsetstrokecolor{currentstroke}%
\pgfsetstrokeopacity{0.806224}%
\pgfsetdash{}{0pt}%
\pgfpathmoveto{\pgfqpoint{2.968488in}{2.334124in}}%
\pgfpathcurveto{\pgfqpoint{2.976724in}{2.334124in}}{\pgfqpoint{2.984624in}{2.337397in}}{\pgfqpoint{2.990448in}{2.343221in}}%
\pgfpathcurveto{\pgfqpoint{2.996272in}{2.349045in}}{\pgfqpoint{2.999544in}{2.356945in}}{\pgfqpoint{2.999544in}{2.365181in}}%
\pgfpathcurveto{\pgfqpoint{2.999544in}{2.373417in}}{\pgfqpoint{2.996272in}{2.381317in}}{\pgfqpoint{2.990448in}{2.387141in}}%
\pgfpathcurveto{\pgfqpoint{2.984624in}{2.392965in}}{\pgfqpoint{2.976724in}{2.396237in}}{\pgfqpoint{2.968488in}{2.396237in}}%
\pgfpathcurveto{\pgfqpoint{2.960252in}{2.396237in}}{\pgfqpoint{2.952351in}{2.392965in}}{\pgfqpoint{2.946528in}{2.387141in}}%
\pgfpathcurveto{\pgfqpoint{2.940704in}{2.381317in}}{\pgfqpoint{2.937431in}{2.373417in}}{\pgfqpoint{2.937431in}{2.365181in}}%
\pgfpathcurveto{\pgfqpoint{2.937431in}{2.356945in}}{\pgfqpoint{2.940704in}{2.349045in}}{\pgfqpoint{2.946528in}{2.343221in}}%
\pgfpathcurveto{\pgfqpoint{2.952351in}{2.337397in}}{\pgfqpoint{2.960252in}{2.334124in}}{\pgfqpoint{2.968488in}{2.334124in}}%
\pgfpathclose%
\pgfusepath{stroke,fill}%
\end{pgfscope}%
\begin{pgfscope}%
\pgfpathrectangle{\pgfqpoint{0.100000in}{0.220728in}}{\pgfqpoint{3.696000in}{3.696000in}}%
\pgfusepath{clip}%
\pgfsetbuttcap%
\pgfsetroundjoin%
\definecolor{currentfill}{rgb}{0.121569,0.466667,0.705882}%
\pgfsetfillcolor{currentfill}%
\pgfsetfillopacity{0.806622}%
\pgfsetlinewidth{1.003750pt}%
\definecolor{currentstroke}{rgb}{0.121569,0.466667,0.705882}%
\pgfsetstrokecolor{currentstroke}%
\pgfsetstrokeopacity{0.806622}%
\pgfsetdash{}{0pt}%
\pgfpathmoveto{\pgfqpoint{2.967442in}{2.332168in}}%
\pgfpathcurveto{\pgfqpoint{2.975679in}{2.332168in}}{\pgfqpoint{2.983579in}{2.335440in}}{\pgfqpoint{2.989403in}{2.341264in}}%
\pgfpathcurveto{\pgfqpoint{2.995227in}{2.347088in}}{\pgfqpoint{2.998499in}{2.354988in}}{\pgfqpoint{2.998499in}{2.363224in}}%
\pgfpathcurveto{\pgfqpoint{2.998499in}{2.371461in}}{\pgfqpoint{2.995227in}{2.379361in}}{\pgfqpoint{2.989403in}{2.385185in}}%
\pgfpathcurveto{\pgfqpoint{2.983579in}{2.391009in}}{\pgfqpoint{2.975679in}{2.394281in}}{\pgfqpoint{2.967442in}{2.394281in}}%
\pgfpathcurveto{\pgfqpoint{2.959206in}{2.394281in}}{\pgfqpoint{2.951306in}{2.391009in}}{\pgfqpoint{2.945482in}{2.385185in}}%
\pgfpathcurveto{\pgfqpoint{2.939658in}{2.379361in}}{\pgfqpoint{2.936386in}{2.371461in}}{\pgfqpoint{2.936386in}{2.363224in}}%
\pgfpathcurveto{\pgfqpoint{2.936386in}{2.354988in}}{\pgfqpoint{2.939658in}{2.347088in}}{\pgfqpoint{2.945482in}{2.341264in}}%
\pgfpathcurveto{\pgfqpoint{2.951306in}{2.335440in}}{\pgfqpoint{2.959206in}{2.332168in}}{\pgfqpoint{2.967442in}{2.332168in}}%
\pgfpathclose%
\pgfusepath{stroke,fill}%
\end{pgfscope}%
\begin{pgfscope}%
\pgfpathrectangle{\pgfqpoint{0.100000in}{0.220728in}}{\pgfqpoint{3.696000in}{3.696000in}}%
\pgfusepath{clip}%
\pgfsetbuttcap%
\pgfsetroundjoin%
\definecolor{currentfill}{rgb}{0.121569,0.466667,0.705882}%
\pgfsetfillcolor{currentfill}%
\pgfsetfillopacity{0.807512}%
\pgfsetlinewidth{1.003750pt}%
\definecolor{currentstroke}{rgb}{0.121569,0.466667,0.705882}%
\pgfsetstrokecolor{currentstroke}%
\pgfsetstrokeopacity{0.807512}%
\pgfsetdash{}{0pt}%
\pgfpathmoveto{\pgfqpoint{2.964401in}{2.328206in}}%
\pgfpathcurveto{\pgfqpoint{2.972638in}{2.328206in}}{\pgfqpoint{2.980538in}{2.331478in}}{\pgfqpoint{2.986362in}{2.337302in}}%
\pgfpathcurveto{\pgfqpoint{2.992185in}{2.343126in}}{\pgfqpoint{2.995458in}{2.351026in}}{\pgfqpoint{2.995458in}{2.359262in}}%
\pgfpathcurveto{\pgfqpoint{2.995458in}{2.367498in}}{\pgfqpoint{2.992185in}{2.375398in}}{\pgfqpoint{2.986362in}{2.381222in}}%
\pgfpathcurveto{\pgfqpoint{2.980538in}{2.387046in}}{\pgfqpoint{2.972638in}{2.390319in}}{\pgfqpoint{2.964401in}{2.390319in}}%
\pgfpathcurveto{\pgfqpoint{2.956165in}{2.390319in}}{\pgfqpoint{2.948265in}{2.387046in}}{\pgfqpoint{2.942441in}{2.381222in}}%
\pgfpathcurveto{\pgfqpoint{2.936617in}{2.375398in}}{\pgfqpoint{2.933345in}{2.367498in}}{\pgfqpoint{2.933345in}{2.359262in}}%
\pgfpathcurveto{\pgfqpoint{2.933345in}{2.351026in}}{\pgfqpoint{2.936617in}{2.343126in}}{\pgfqpoint{2.942441in}{2.337302in}}%
\pgfpathcurveto{\pgfqpoint{2.948265in}{2.331478in}}{\pgfqpoint{2.956165in}{2.328206in}}{\pgfqpoint{2.964401in}{2.328206in}}%
\pgfpathclose%
\pgfusepath{stroke,fill}%
\end{pgfscope}%
\begin{pgfscope}%
\pgfpathrectangle{\pgfqpoint{0.100000in}{0.220728in}}{\pgfqpoint{3.696000in}{3.696000in}}%
\pgfusepath{clip}%
\pgfsetbuttcap%
\pgfsetroundjoin%
\definecolor{currentfill}{rgb}{0.121569,0.466667,0.705882}%
\pgfsetfillcolor{currentfill}%
\pgfsetfillopacity{0.808846}%
\pgfsetlinewidth{1.003750pt}%
\definecolor{currentstroke}{rgb}{0.121569,0.466667,0.705882}%
\pgfsetstrokecolor{currentstroke}%
\pgfsetstrokeopacity{0.808846}%
\pgfsetdash{}{0pt}%
\pgfpathmoveto{\pgfqpoint{2.960964in}{2.321657in}}%
\pgfpathcurveto{\pgfqpoint{2.969200in}{2.321657in}}{\pgfqpoint{2.977100in}{2.324929in}}{\pgfqpoint{2.982924in}{2.330753in}}%
\pgfpathcurveto{\pgfqpoint{2.988748in}{2.336577in}}{\pgfqpoint{2.992021in}{2.344477in}}{\pgfqpoint{2.992021in}{2.352713in}}%
\pgfpathcurveto{\pgfqpoint{2.992021in}{2.360950in}}{\pgfqpoint{2.988748in}{2.368850in}}{\pgfqpoint{2.982924in}{2.374674in}}%
\pgfpathcurveto{\pgfqpoint{2.977100in}{2.380498in}}{\pgfqpoint{2.969200in}{2.383770in}}{\pgfqpoint{2.960964in}{2.383770in}}%
\pgfpathcurveto{\pgfqpoint{2.952728in}{2.383770in}}{\pgfqpoint{2.944828in}{2.380498in}}{\pgfqpoint{2.939004in}{2.374674in}}%
\pgfpathcurveto{\pgfqpoint{2.933180in}{2.368850in}}{\pgfqpoint{2.929908in}{2.360950in}}{\pgfqpoint{2.929908in}{2.352713in}}%
\pgfpathcurveto{\pgfqpoint{2.929908in}{2.344477in}}{\pgfqpoint{2.933180in}{2.336577in}}{\pgfqpoint{2.939004in}{2.330753in}}%
\pgfpathcurveto{\pgfqpoint{2.944828in}{2.324929in}}{\pgfqpoint{2.952728in}{2.321657in}}{\pgfqpoint{2.960964in}{2.321657in}}%
\pgfpathclose%
\pgfusepath{stroke,fill}%
\end{pgfscope}%
\begin{pgfscope}%
\pgfpathrectangle{\pgfqpoint{0.100000in}{0.220728in}}{\pgfqpoint{3.696000in}{3.696000in}}%
\pgfusepath{clip}%
\pgfsetbuttcap%
\pgfsetroundjoin%
\definecolor{currentfill}{rgb}{0.121569,0.466667,0.705882}%
\pgfsetfillcolor{currentfill}%
\pgfsetfillopacity{0.810998}%
\pgfsetlinewidth{1.003750pt}%
\definecolor{currentstroke}{rgb}{0.121569,0.466667,0.705882}%
\pgfsetstrokecolor{currentstroke}%
\pgfsetstrokeopacity{0.810998}%
\pgfsetdash{}{0pt}%
\pgfpathmoveto{\pgfqpoint{2.953597in}{2.311944in}}%
\pgfpathcurveto{\pgfqpoint{2.961833in}{2.311944in}}{\pgfqpoint{2.969733in}{2.315216in}}{\pgfqpoint{2.975557in}{2.321040in}}%
\pgfpathcurveto{\pgfqpoint{2.981381in}{2.326864in}}{\pgfqpoint{2.984653in}{2.334764in}}{\pgfqpoint{2.984653in}{2.343000in}}%
\pgfpathcurveto{\pgfqpoint{2.984653in}{2.351237in}}{\pgfqpoint{2.981381in}{2.359137in}}{\pgfqpoint{2.975557in}{2.364961in}}%
\pgfpathcurveto{\pgfqpoint{2.969733in}{2.370785in}}{\pgfqpoint{2.961833in}{2.374057in}}{\pgfqpoint{2.953597in}{2.374057in}}%
\pgfpathcurveto{\pgfqpoint{2.945360in}{2.374057in}}{\pgfqpoint{2.937460in}{2.370785in}}{\pgfqpoint{2.931636in}{2.364961in}}%
\pgfpathcurveto{\pgfqpoint{2.925812in}{2.359137in}}{\pgfqpoint{2.922540in}{2.351237in}}{\pgfqpoint{2.922540in}{2.343000in}}%
\pgfpathcurveto{\pgfqpoint{2.922540in}{2.334764in}}{\pgfqpoint{2.925812in}{2.326864in}}{\pgfqpoint{2.931636in}{2.321040in}}%
\pgfpathcurveto{\pgfqpoint{2.937460in}{2.315216in}}{\pgfqpoint{2.945360in}{2.311944in}}{\pgfqpoint{2.953597in}{2.311944in}}%
\pgfpathclose%
\pgfusepath{stroke,fill}%
\end{pgfscope}%
\begin{pgfscope}%
\pgfpathrectangle{\pgfqpoint{0.100000in}{0.220728in}}{\pgfqpoint{3.696000in}{3.696000in}}%
\pgfusepath{clip}%
\pgfsetbuttcap%
\pgfsetroundjoin%
\definecolor{currentfill}{rgb}{0.121569,0.466667,0.705882}%
\pgfsetfillcolor{currentfill}%
\pgfsetfillopacity{0.811674}%
\pgfsetlinewidth{1.003750pt}%
\definecolor{currentstroke}{rgb}{0.121569,0.466667,0.705882}%
\pgfsetstrokecolor{currentstroke}%
\pgfsetstrokeopacity{0.811674}%
\pgfsetdash{}{0pt}%
\pgfpathmoveto{\pgfqpoint{1.373675in}{1.949428in}}%
\pgfpathcurveto{\pgfqpoint{1.381912in}{1.949428in}}{\pgfqpoint{1.389812in}{1.952700in}}{\pgfqpoint{1.395636in}{1.958524in}}%
\pgfpathcurveto{\pgfqpoint{1.401459in}{1.964348in}}{\pgfqpoint{1.404732in}{1.972248in}}{\pgfqpoint{1.404732in}{1.980484in}}%
\pgfpathcurveto{\pgfqpoint{1.404732in}{1.988721in}}{\pgfqpoint{1.401459in}{1.996621in}}{\pgfqpoint{1.395636in}{2.002445in}}%
\pgfpathcurveto{\pgfqpoint{1.389812in}{2.008268in}}{\pgfqpoint{1.381912in}{2.011541in}}{\pgfqpoint{1.373675in}{2.011541in}}%
\pgfpathcurveto{\pgfqpoint{1.365439in}{2.011541in}}{\pgfqpoint{1.357539in}{2.008268in}}{\pgfqpoint{1.351715in}{2.002445in}}%
\pgfpathcurveto{\pgfqpoint{1.345891in}{1.996621in}}{\pgfqpoint{1.342619in}{1.988721in}}{\pgfqpoint{1.342619in}{1.980484in}}%
\pgfpathcurveto{\pgfqpoint{1.342619in}{1.972248in}}{\pgfqpoint{1.345891in}{1.964348in}}{\pgfqpoint{1.351715in}{1.958524in}}%
\pgfpathcurveto{\pgfqpoint{1.357539in}{1.952700in}}{\pgfqpoint{1.365439in}{1.949428in}}{\pgfqpoint{1.373675in}{1.949428in}}%
\pgfpathclose%
\pgfusepath{stroke,fill}%
\end{pgfscope}%
\begin{pgfscope}%
\pgfpathrectangle{\pgfqpoint{0.100000in}{0.220728in}}{\pgfqpoint{3.696000in}{3.696000in}}%
\pgfusepath{clip}%
\pgfsetbuttcap%
\pgfsetroundjoin%
\definecolor{currentfill}{rgb}{0.121569,0.466667,0.705882}%
\pgfsetfillcolor{currentfill}%
\pgfsetfillopacity{0.812179}%
\pgfsetlinewidth{1.003750pt}%
\definecolor{currentstroke}{rgb}{0.121569,0.466667,0.705882}%
\pgfsetstrokecolor{currentstroke}%
\pgfsetstrokeopacity{0.812179}%
\pgfsetdash{}{0pt}%
\pgfpathmoveto{\pgfqpoint{2.950296in}{2.305589in}}%
\pgfpathcurveto{\pgfqpoint{2.958533in}{2.305589in}}{\pgfqpoint{2.966433in}{2.308862in}}{\pgfqpoint{2.972257in}{2.314686in}}%
\pgfpathcurveto{\pgfqpoint{2.978080in}{2.320510in}}{\pgfqpoint{2.981353in}{2.328410in}}{\pgfqpoint{2.981353in}{2.336646in}}%
\pgfpathcurveto{\pgfqpoint{2.981353in}{2.344882in}}{\pgfqpoint{2.978080in}{2.352782in}}{\pgfqpoint{2.972257in}{2.358606in}}%
\pgfpathcurveto{\pgfqpoint{2.966433in}{2.364430in}}{\pgfqpoint{2.958533in}{2.367702in}}{\pgfqpoint{2.950296in}{2.367702in}}%
\pgfpathcurveto{\pgfqpoint{2.942060in}{2.367702in}}{\pgfqpoint{2.934160in}{2.364430in}}{\pgfqpoint{2.928336in}{2.358606in}}%
\pgfpathcurveto{\pgfqpoint{2.922512in}{2.352782in}}{\pgfqpoint{2.919240in}{2.344882in}}{\pgfqpoint{2.919240in}{2.336646in}}%
\pgfpathcurveto{\pgfqpoint{2.919240in}{2.328410in}}{\pgfqpoint{2.922512in}{2.320510in}}{\pgfqpoint{2.928336in}{2.314686in}}%
\pgfpathcurveto{\pgfqpoint{2.934160in}{2.308862in}}{\pgfqpoint{2.942060in}{2.305589in}}{\pgfqpoint{2.950296in}{2.305589in}}%
\pgfpathclose%
\pgfusepath{stroke,fill}%
\end{pgfscope}%
\begin{pgfscope}%
\pgfpathrectangle{\pgfqpoint{0.100000in}{0.220728in}}{\pgfqpoint{3.696000in}{3.696000in}}%
\pgfusepath{clip}%
\pgfsetbuttcap%
\pgfsetroundjoin%
\definecolor{currentfill}{rgb}{0.121569,0.466667,0.705882}%
\pgfsetfillcolor{currentfill}%
\pgfsetfillopacity{0.813875}%
\pgfsetlinewidth{1.003750pt}%
\definecolor{currentstroke}{rgb}{0.121569,0.466667,0.705882}%
\pgfsetstrokecolor{currentstroke}%
\pgfsetstrokeopacity{0.813875}%
\pgfsetdash{}{0pt}%
\pgfpathmoveto{\pgfqpoint{2.944005in}{2.296623in}}%
\pgfpathcurveto{\pgfqpoint{2.952241in}{2.296623in}}{\pgfqpoint{2.960141in}{2.299895in}}{\pgfqpoint{2.965965in}{2.305719in}}%
\pgfpathcurveto{\pgfqpoint{2.971789in}{2.311543in}}{\pgfqpoint{2.975061in}{2.319443in}}{\pgfqpoint{2.975061in}{2.327680in}}%
\pgfpathcurveto{\pgfqpoint{2.975061in}{2.335916in}}{\pgfqpoint{2.971789in}{2.343816in}}{\pgfqpoint{2.965965in}{2.349640in}}%
\pgfpathcurveto{\pgfqpoint{2.960141in}{2.355464in}}{\pgfqpoint{2.952241in}{2.358736in}}{\pgfqpoint{2.944005in}{2.358736in}}%
\pgfpathcurveto{\pgfqpoint{2.935769in}{2.358736in}}{\pgfqpoint{2.927868in}{2.355464in}}{\pgfqpoint{2.922045in}{2.349640in}}%
\pgfpathcurveto{\pgfqpoint{2.916221in}{2.343816in}}{\pgfqpoint{2.912948in}{2.335916in}}{\pgfqpoint{2.912948in}{2.327680in}}%
\pgfpathcurveto{\pgfqpoint{2.912948in}{2.319443in}}{\pgfqpoint{2.916221in}{2.311543in}}{\pgfqpoint{2.922045in}{2.305719in}}%
\pgfpathcurveto{\pgfqpoint{2.927868in}{2.299895in}}{\pgfqpoint{2.935769in}{2.296623in}}{\pgfqpoint{2.944005in}{2.296623in}}%
\pgfpathclose%
\pgfusepath{stroke,fill}%
\end{pgfscope}%
\begin{pgfscope}%
\pgfpathrectangle{\pgfqpoint{0.100000in}{0.220728in}}{\pgfqpoint{3.696000in}{3.696000in}}%
\pgfusepath{clip}%
\pgfsetbuttcap%
\pgfsetroundjoin%
\definecolor{currentfill}{rgb}{0.121569,0.466667,0.705882}%
\pgfsetfillcolor{currentfill}%
\pgfsetfillopacity{0.815036}%
\pgfsetlinewidth{1.003750pt}%
\definecolor{currentstroke}{rgb}{0.121569,0.466667,0.705882}%
\pgfsetstrokecolor{currentstroke}%
\pgfsetstrokeopacity{0.815036}%
\pgfsetdash{}{0pt}%
\pgfpathmoveto{\pgfqpoint{2.941197in}{2.291954in}}%
\pgfpathcurveto{\pgfqpoint{2.949433in}{2.291954in}}{\pgfqpoint{2.957333in}{2.295227in}}{\pgfqpoint{2.963157in}{2.301051in}}%
\pgfpathcurveto{\pgfqpoint{2.968981in}{2.306875in}}{\pgfqpoint{2.972253in}{2.314775in}}{\pgfqpoint{2.972253in}{2.323011in}}%
\pgfpathcurveto{\pgfqpoint{2.972253in}{2.331247in}}{\pgfqpoint{2.968981in}{2.339147in}}{\pgfqpoint{2.963157in}{2.344971in}}%
\pgfpathcurveto{\pgfqpoint{2.957333in}{2.350795in}}{\pgfqpoint{2.949433in}{2.354067in}}{\pgfqpoint{2.941197in}{2.354067in}}%
\pgfpathcurveto{\pgfqpoint{2.932961in}{2.354067in}}{\pgfqpoint{2.925061in}{2.350795in}}{\pgfqpoint{2.919237in}{2.344971in}}%
\pgfpathcurveto{\pgfqpoint{2.913413in}{2.339147in}}{\pgfqpoint{2.910140in}{2.331247in}}{\pgfqpoint{2.910140in}{2.323011in}}%
\pgfpathcurveto{\pgfqpoint{2.910140in}{2.314775in}}{\pgfqpoint{2.913413in}{2.306875in}}{\pgfqpoint{2.919237in}{2.301051in}}%
\pgfpathcurveto{\pgfqpoint{2.925061in}{2.295227in}}{\pgfqpoint{2.932961in}{2.291954in}}{\pgfqpoint{2.941197in}{2.291954in}}%
\pgfpathclose%
\pgfusepath{stroke,fill}%
\end{pgfscope}%
\begin{pgfscope}%
\pgfpathrectangle{\pgfqpoint{0.100000in}{0.220728in}}{\pgfqpoint{3.696000in}{3.696000in}}%
\pgfusepath{clip}%
\pgfsetbuttcap%
\pgfsetroundjoin%
\definecolor{currentfill}{rgb}{0.121569,0.466667,0.705882}%
\pgfsetfillcolor{currentfill}%
\pgfsetfillopacity{0.815619}%
\pgfsetlinewidth{1.003750pt}%
\definecolor{currentstroke}{rgb}{0.121569,0.466667,0.705882}%
\pgfsetstrokecolor{currentstroke}%
\pgfsetstrokeopacity{0.815619}%
\pgfsetdash{}{0pt}%
\pgfpathmoveto{\pgfqpoint{2.939589in}{2.289178in}}%
\pgfpathcurveto{\pgfqpoint{2.947825in}{2.289178in}}{\pgfqpoint{2.955725in}{2.292450in}}{\pgfqpoint{2.961549in}{2.298274in}}%
\pgfpathcurveto{\pgfqpoint{2.967373in}{2.304098in}}{\pgfqpoint{2.970646in}{2.311998in}}{\pgfqpoint{2.970646in}{2.320234in}}%
\pgfpathcurveto{\pgfqpoint{2.970646in}{2.328470in}}{\pgfqpoint{2.967373in}{2.336370in}}{\pgfqpoint{2.961549in}{2.342194in}}%
\pgfpathcurveto{\pgfqpoint{2.955725in}{2.348018in}}{\pgfqpoint{2.947825in}{2.351291in}}{\pgfqpoint{2.939589in}{2.351291in}}%
\pgfpathcurveto{\pgfqpoint{2.931353in}{2.351291in}}{\pgfqpoint{2.923453in}{2.348018in}}{\pgfqpoint{2.917629in}{2.342194in}}%
\pgfpathcurveto{\pgfqpoint{2.911805in}{2.336370in}}{\pgfqpoint{2.908533in}{2.328470in}}{\pgfqpoint{2.908533in}{2.320234in}}%
\pgfpathcurveto{\pgfqpoint{2.908533in}{2.311998in}}{\pgfqpoint{2.911805in}{2.304098in}}{\pgfqpoint{2.917629in}{2.298274in}}%
\pgfpathcurveto{\pgfqpoint{2.923453in}{2.292450in}}{\pgfqpoint{2.931353in}{2.289178in}}{\pgfqpoint{2.939589in}{2.289178in}}%
\pgfpathclose%
\pgfusepath{stroke,fill}%
\end{pgfscope}%
\begin{pgfscope}%
\pgfpathrectangle{\pgfqpoint{0.100000in}{0.220728in}}{\pgfqpoint{3.696000in}{3.696000in}}%
\pgfusepath{clip}%
\pgfsetbuttcap%
\pgfsetroundjoin%
\definecolor{currentfill}{rgb}{0.121569,0.466667,0.705882}%
\pgfsetfillcolor{currentfill}%
\pgfsetfillopacity{0.815931}%
\pgfsetlinewidth{1.003750pt}%
\definecolor{currentstroke}{rgb}{0.121569,0.466667,0.705882}%
\pgfsetstrokecolor{currentstroke}%
\pgfsetstrokeopacity{0.815931}%
\pgfsetdash{}{0pt}%
\pgfpathmoveto{\pgfqpoint{2.938626in}{2.287716in}}%
\pgfpathcurveto{\pgfqpoint{2.946862in}{2.287716in}}{\pgfqpoint{2.954762in}{2.290989in}}{\pgfqpoint{2.960586in}{2.296813in}}%
\pgfpathcurveto{\pgfqpoint{2.966410in}{2.302636in}}{\pgfqpoint{2.969682in}{2.310537in}}{\pgfqpoint{2.969682in}{2.318773in}}%
\pgfpathcurveto{\pgfqpoint{2.969682in}{2.327009in}}{\pgfqpoint{2.966410in}{2.334909in}}{\pgfqpoint{2.960586in}{2.340733in}}%
\pgfpathcurveto{\pgfqpoint{2.954762in}{2.346557in}}{\pgfqpoint{2.946862in}{2.349829in}}{\pgfqpoint{2.938626in}{2.349829in}}%
\pgfpathcurveto{\pgfqpoint{2.930390in}{2.349829in}}{\pgfqpoint{2.922490in}{2.346557in}}{\pgfqpoint{2.916666in}{2.340733in}}%
\pgfpathcurveto{\pgfqpoint{2.910842in}{2.334909in}}{\pgfqpoint{2.907569in}{2.327009in}}{\pgfqpoint{2.907569in}{2.318773in}}%
\pgfpathcurveto{\pgfqpoint{2.907569in}{2.310537in}}{\pgfqpoint{2.910842in}{2.302636in}}{\pgfqpoint{2.916666in}{2.296813in}}%
\pgfpathcurveto{\pgfqpoint{2.922490in}{2.290989in}}{\pgfqpoint{2.930390in}{2.287716in}}{\pgfqpoint{2.938626in}{2.287716in}}%
\pgfpathclose%
\pgfusepath{stroke,fill}%
\end{pgfscope}%
\begin{pgfscope}%
\pgfpathrectangle{\pgfqpoint{0.100000in}{0.220728in}}{\pgfqpoint{3.696000in}{3.696000in}}%
\pgfusepath{clip}%
\pgfsetbuttcap%
\pgfsetroundjoin%
\definecolor{currentfill}{rgb}{0.121569,0.466667,0.705882}%
\pgfsetfillcolor{currentfill}%
\pgfsetfillopacity{0.816109}%
\pgfsetlinewidth{1.003750pt}%
\definecolor{currentstroke}{rgb}{0.121569,0.466667,0.705882}%
\pgfsetstrokecolor{currentstroke}%
\pgfsetstrokeopacity{0.816109}%
\pgfsetdash{}{0pt}%
\pgfpathmoveto{\pgfqpoint{2.938130in}{2.286902in}}%
\pgfpathcurveto{\pgfqpoint{2.946366in}{2.286902in}}{\pgfqpoint{2.954266in}{2.290174in}}{\pgfqpoint{2.960090in}{2.295998in}}%
\pgfpathcurveto{\pgfqpoint{2.965914in}{2.301822in}}{\pgfqpoint{2.969186in}{2.309722in}}{\pgfqpoint{2.969186in}{2.317958in}}%
\pgfpathcurveto{\pgfqpoint{2.969186in}{2.326194in}}{\pgfqpoint{2.965914in}{2.334094in}}{\pgfqpoint{2.960090in}{2.339918in}}%
\pgfpathcurveto{\pgfqpoint{2.954266in}{2.345742in}}{\pgfqpoint{2.946366in}{2.349015in}}{\pgfqpoint{2.938130in}{2.349015in}}%
\pgfpathcurveto{\pgfqpoint{2.929894in}{2.349015in}}{\pgfqpoint{2.921994in}{2.345742in}}{\pgfqpoint{2.916170in}{2.339918in}}%
\pgfpathcurveto{\pgfqpoint{2.910346in}{2.334094in}}{\pgfqpoint{2.907073in}{2.326194in}}{\pgfqpoint{2.907073in}{2.317958in}}%
\pgfpathcurveto{\pgfqpoint{2.907073in}{2.309722in}}{\pgfqpoint{2.910346in}{2.301822in}}{\pgfqpoint{2.916170in}{2.295998in}}%
\pgfpathcurveto{\pgfqpoint{2.921994in}{2.290174in}}{\pgfqpoint{2.929894in}{2.286902in}}{\pgfqpoint{2.938130in}{2.286902in}}%
\pgfpathclose%
\pgfusepath{stroke,fill}%
\end{pgfscope}%
\begin{pgfscope}%
\pgfpathrectangle{\pgfqpoint{0.100000in}{0.220728in}}{\pgfqpoint{3.696000in}{3.696000in}}%
\pgfusepath{clip}%
\pgfsetbuttcap%
\pgfsetroundjoin%
\definecolor{currentfill}{rgb}{0.121569,0.466667,0.705882}%
\pgfsetfillcolor{currentfill}%
\pgfsetfillopacity{0.816202}%
\pgfsetlinewidth{1.003750pt}%
\definecolor{currentstroke}{rgb}{0.121569,0.466667,0.705882}%
\pgfsetstrokecolor{currentstroke}%
\pgfsetstrokeopacity{0.816202}%
\pgfsetdash{}{0pt}%
\pgfpathmoveto{\pgfqpoint{2.937801in}{2.286503in}}%
\pgfpathcurveto{\pgfqpoint{2.946037in}{2.286503in}}{\pgfqpoint{2.953937in}{2.289776in}}{\pgfqpoint{2.959761in}{2.295600in}}%
\pgfpathcurveto{\pgfqpoint{2.965585in}{2.301424in}}{\pgfqpoint{2.968857in}{2.309324in}}{\pgfqpoint{2.968857in}{2.317560in}}%
\pgfpathcurveto{\pgfqpoint{2.968857in}{2.325796in}}{\pgfqpoint{2.965585in}{2.333696in}}{\pgfqpoint{2.959761in}{2.339520in}}%
\pgfpathcurveto{\pgfqpoint{2.953937in}{2.345344in}}{\pgfqpoint{2.946037in}{2.348616in}}{\pgfqpoint{2.937801in}{2.348616in}}%
\pgfpathcurveto{\pgfqpoint{2.929564in}{2.348616in}}{\pgfqpoint{2.921664in}{2.345344in}}{\pgfqpoint{2.915840in}{2.339520in}}%
\pgfpathcurveto{\pgfqpoint{2.910016in}{2.333696in}}{\pgfqpoint{2.906744in}{2.325796in}}{\pgfqpoint{2.906744in}{2.317560in}}%
\pgfpathcurveto{\pgfqpoint{2.906744in}{2.309324in}}{\pgfqpoint{2.910016in}{2.301424in}}{\pgfqpoint{2.915840in}{2.295600in}}%
\pgfpathcurveto{\pgfqpoint{2.921664in}{2.289776in}}{\pgfqpoint{2.929564in}{2.286503in}}{\pgfqpoint{2.937801in}{2.286503in}}%
\pgfpathclose%
\pgfusepath{stroke,fill}%
\end{pgfscope}%
\begin{pgfscope}%
\pgfpathrectangle{\pgfqpoint{0.100000in}{0.220728in}}{\pgfqpoint{3.696000in}{3.696000in}}%
\pgfusepath{clip}%
\pgfsetbuttcap%
\pgfsetroundjoin%
\definecolor{currentfill}{rgb}{0.121569,0.466667,0.705882}%
\pgfsetfillcolor{currentfill}%
\pgfsetfillopacity{0.816826}%
\pgfsetlinewidth{1.003750pt}%
\definecolor{currentstroke}{rgb}{0.121569,0.466667,0.705882}%
\pgfsetstrokecolor{currentstroke}%
\pgfsetstrokeopacity{0.816826}%
\pgfsetdash{}{0pt}%
\pgfpathmoveto{\pgfqpoint{2.936486in}{2.283958in}}%
\pgfpathcurveto{\pgfqpoint{2.944723in}{2.283958in}}{\pgfqpoint{2.952623in}{2.287231in}}{\pgfqpoint{2.958447in}{2.293055in}}%
\pgfpathcurveto{\pgfqpoint{2.964270in}{2.298879in}}{\pgfqpoint{2.967543in}{2.306779in}}{\pgfqpoint{2.967543in}{2.315015in}}%
\pgfpathcurveto{\pgfqpoint{2.967543in}{2.323251in}}{\pgfqpoint{2.964270in}{2.331151in}}{\pgfqpoint{2.958447in}{2.336975in}}%
\pgfpathcurveto{\pgfqpoint{2.952623in}{2.342799in}}{\pgfqpoint{2.944723in}{2.346071in}}{\pgfqpoint{2.936486in}{2.346071in}}%
\pgfpathcurveto{\pgfqpoint{2.928250in}{2.346071in}}{\pgfqpoint{2.920350in}{2.342799in}}{\pgfqpoint{2.914526in}{2.336975in}}%
\pgfpathcurveto{\pgfqpoint{2.908702in}{2.331151in}}{\pgfqpoint{2.905430in}{2.323251in}}{\pgfqpoint{2.905430in}{2.315015in}}%
\pgfpathcurveto{\pgfqpoint{2.905430in}{2.306779in}}{\pgfqpoint{2.908702in}{2.298879in}}{\pgfqpoint{2.914526in}{2.293055in}}%
\pgfpathcurveto{\pgfqpoint{2.920350in}{2.287231in}}{\pgfqpoint{2.928250in}{2.283958in}}{\pgfqpoint{2.936486in}{2.283958in}}%
\pgfpathclose%
\pgfusepath{stroke,fill}%
\end{pgfscope}%
\begin{pgfscope}%
\pgfpathrectangle{\pgfqpoint{0.100000in}{0.220728in}}{\pgfqpoint{3.696000in}{3.696000in}}%
\pgfusepath{clip}%
\pgfsetbuttcap%
\pgfsetroundjoin%
\definecolor{currentfill}{rgb}{0.121569,0.466667,0.705882}%
\pgfsetfillcolor{currentfill}%
\pgfsetfillopacity{0.817653}%
\pgfsetlinewidth{1.003750pt}%
\definecolor{currentstroke}{rgb}{0.121569,0.466667,0.705882}%
\pgfsetstrokecolor{currentstroke}%
\pgfsetstrokeopacity{0.817653}%
\pgfsetdash{}{0pt}%
\pgfpathmoveto{\pgfqpoint{1.407366in}{1.930617in}}%
\pgfpathcurveto{\pgfqpoint{1.415602in}{1.930617in}}{\pgfqpoint{1.423502in}{1.933889in}}{\pgfqpoint{1.429326in}{1.939713in}}%
\pgfpathcurveto{\pgfqpoint{1.435150in}{1.945537in}}{\pgfqpoint{1.438423in}{1.953437in}}{\pgfqpoint{1.438423in}{1.961673in}}%
\pgfpathcurveto{\pgfqpoint{1.438423in}{1.969909in}}{\pgfqpoint{1.435150in}{1.977809in}}{\pgfqpoint{1.429326in}{1.983633in}}%
\pgfpathcurveto{\pgfqpoint{1.423502in}{1.989457in}}{\pgfqpoint{1.415602in}{1.992730in}}{\pgfqpoint{1.407366in}{1.992730in}}%
\pgfpathcurveto{\pgfqpoint{1.399130in}{1.992730in}}{\pgfqpoint{1.391230in}{1.989457in}}{\pgfqpoint{1.385406in}{1.983633in}}%
\pgfpathcurveto{\pgfqpoint{1.379582in}{1.977809in}}{\pgfqpoint{1.376310in}{1.969909in}}{\pgfqpoint{1.376310in}{1.961673in}}%
\pgfpathcurveto{\pgfqpoint{1.376310in}{1.953437in}}{\pgfqpoint{1.379582in}{1.945537in}}{\pgfqpoint{1.385406in}{1.939713in}}%
\pgfpathcurveto{\pgfqpoint{1.391230in}{1.933889in}}{\pgfqpoint{1.399130in}{1.930617in}}{\pgfqpoint{1.407366in}{1.930617in}}%
\pgfpathclose%
\pgfusepath{stroke,fill}%
\end{pgfscope}%
\begin{pgfscope}%
\pgfpathrectangle{\pgfqpoint{0.100000in}{0.220728in}}{\pgfqpoint{3.696000in}{3.696000in}}%
\pgfusepath{clip}%
\pgfsetbuttcap%
\pgfsetroundjoin%
\definecolor{currentfill}{rgb}{0.121569,0.466667,0.705882}%
\pgfsetfillcolor{currentfill}%
\pgfsetfillopacity{0.817976}%
\pgfsetlinewidth{1.003750pt}%
\definecolor{currentstroke}{rgb}{0.121569,0.466667,0.705882}%
\pgfsetstrokecolor{currentstroke}%
\pgfsetstrokeopacity{0.817976}%
\pgfsetdash{}{0pt}%
\pgfpathmoveto{\pgfqpoint{2.932053in}{2.278612in}}%
\pgfpathcurveto{\pgfqpoint{2.940289in}{2.278612in}}{\pgfqpoint{2.948189in}{2.281885in}}{\pgfqpoint{2.954013in}{2.287709in}}%
\pgfpathcurveto{\pgfqpoint{2.959837in}{2.293532in}}{\pgfqpoint{2.963110in}{2.301433in}}{\pgfqpoint{2.963110in}{2.309669in}}%
\pgfpathcurveto{\pgfqpoint{2.963110in}{2.317905in}}{\pgfqpoint{2.959837in}{2.325805in}}{\pgfqpoint{2.954013in}{2.331629in}}%
\pgfpathcurveto{\pgfqpoint{2.948189in}{2.337453in}}{\pgfqpoint{2.940289in}{2.340725in}}{\pgfqpoint{2.932053in}{2.340725in}}%
\pgfpathcurveto{\pgfqpoint{2.923817in}{2.340725in}}{\pgfqpoint{2.915917in}{2.337453in}}{\pgfqpoint{2.910093in}{2.331629in}}%
\pgfpathcurveto{\pgfqpoint{2.904269in}{2.325805in}}{\pgfqpoint{2.900997in}{2.317905in}}{\pgfqpoint{2.900997in}{2.309669in}}%
\pgfpathcurveto{\pgfqpoint{2.900997in}{2.301433in}}{\pgfqpoint{2.904269in}{2.293532in}}{\pgfqpoint{2.910093in}{2.287709in}}%
\pgfpathcurveto{\pgfqpoint{2.915917in}{2.281885in}}{\pgfqpoint{2.923817in}{2.278612in}}{\pgfqpoint{2.932053in}{2.278612in}}%
\pgfpathclose%
\pgfusepath{stroke,fill}%
\end{pgfscope}%
\begin{pgfscope}%
\pgfpathrectangle{\pgfqpoint{0.100000in}{0.220728in}}{\pgfqpoint{3.696000in}{3.696000in}}%
\pgfusepath{clip}%
\pgfsetbuttcap%
\pgfsetroundjoin%
\definecolor{currentfill}{rgb}{0.121569,0.466667,0.705882}%
\pgfsetfillcolor{currentfill}%
\pgfsetfillopacity{0.820459}%
\pgfsetlinewidth{1.003750pt}%
\definecolor{currentstroke}{rgb}{0.121569,0.466667,0.705882}%
\pgfsetstrokecolor{currentstroke}%
\pgfsetstrokeopacity{0.820459}%
\pgfsetdash{}{0pt}%
\pgfpathmoveto{\pgfqpoint{2.926164in}{2.266973in}}%
\pgfpathcurveto{\pgfqpoint{2.934400in}{2.266973in}}{\pgfqpoint{2.942300in}{2.270245in}}{\pgfqpoint{2.948124in}{2.276069in}}%
\pgfpathcurveto{\pgfqpoint{2.953948in}{2.281893in}}{\pgfqpoint{2.957220in}{2.289793in}}{\pgfqpoint{2.957220in}{2.298029in}}%
\pgfpathcurveto{\pgfqpoint{2.957220in}{2.306266in}}{\pgfqpoint{2.953948in}{2.314166in}}{\pgfqpoint{2.948124in}{2.319990in}}%
\pgfpathcurveto{\pgfqpoint{2.942300in}{2.325814in}}{\pgfqpoint{2.934400in}{2.329086in}}{\pgfqpoint{2.926164in}{2.329086in}}%
\pgfpathcurveto{\pgfqpoint{2.917927in}{2.329086in}}{\pgfqpoint{2.910027in}{2.325814in}}{\pgfqpoint{2.904203in}{2.319990in}}%
\pgfpathcurveto{\pgfqpoint{2.898379in}{2.314166in}}{\pgfqpoint{2.895107in}{2.306266in}}{\pgfqpoint{2.895107in}{2.298029in}}%
\pgfpathcurveto{\pgfqpoint{2.895107in}{2.289793in}}{\pgfqpoint{2.898379in}{2.281893in}}{\pgfqpoint{2.904203in}{2.276069in}}%
\pgfpathcurveto{\pgfqpoint{2.910027in}{2.270245in}}{\pgfqpoint{2.917927in}{2.266973in}}{\pgfqpoint{2.926164in}{2.266973in}}%
\pgfpathclose%
\pgfusepath{stroke,fill}%
\end{pgfscope}%
\begin{pgfscope}%
\pgfpathrectangle{\pgfqpoint{0.100000in}{0.220728in}}{\pgfqpoint{3.696000in}{3.696000in}}%
\pgfusepath{clip}%
\pgfsetbuttcap%
\pgfsetroundjoin%
\definecolor{currentfill}{rgb}{0.121569,0.466667,0.705882}%
\pgfsetfillcolor{currentfill}%
\pgfsetfillopacity{0.822390}%
\pgfsetlinewidth{1.003750pt}%
\definecolor{currentstroke}{rgb}{0.121569,0.466667,0.705882}%
\pgfsetstrokecolor{currentstroke}%
\pgfsetstrokeopacity{0.822390}%
\pgfsetdash{}{0pt}%
\pgfpathmoveto{\pgfqpoint{1.436953in}{1.910253in}}%
\pgfpathcurveto{\pgfqpoint{1.445189in}{1.910253in}}{\pgfqpoint{1.453089in}{1.913525in}}{\pgfqpoint{1.458913in}{1.919349in}}%
\pgfpathcurveto{\pgfqpoint{1.464737in}{1.925173in}}{\pgfqpoint{1.468009in}{1.933073in}}{\pgfqpoint{1.468009in}{1.941309in}}%
\pgfpathcurveto{\pgfqpoint{1.468009in}{1.949545in}}{\pgfqpoint{1.464737in}{1.957445in}}{\pgfqpoint{1.458913in}{1.963269in}}%
\pgfpathcurveto{\pgfqpoint{1.453089in}{1.969093in}}{\pgfqpoint{1.445189in}{1.972366in}}{\pgfqpoint{1.436953in}{1.972366in}}%
\pgfpathcurveto{\pgfqpoint{1.428716in}{1.972366in}}{\pgfqpoint{1.420816in}{1.969093in}}{\pgfqpoint{1.414992in}{1.963269in}}%
\pgfpathcurveto{\pgfqpoint{1.409168in}{1.957445in}}{\pgfqpoint{1.405896in}{1.949545in}}{\pgfqpoint{1.405896in}{1.941309in}}%
\pgfpathcurveto{\pgfqpoint{1.405896in}{1.933073in}}{\pgfqpoint{1.409168in}{1.925173in}}{\pgfqpoint{1.414992in}{1.919349in}}%
\pgfpathcurveto{\pgfqpoint{1.420816in}{1.913525in}}{\pgfqpoint{1.428716in}{1.910253in}}{\pgfqpoint{1.436953in}{1.910253in}}%
\pgfpathclose%
\pgfusepath{stroke,fill}%
\end{pgfscope}%
\begin{pgfscope}%
\pgfpathrectangle{\pgfqpoint{0.100000in}{0.220728in}}{\pgfqpoint{3.696000in}{3.696000in}}%
\pgfusepath{clip}%
\pgfsetbuttcap%
\pgfsetroundjoin%
\definecolor{currentfill}{rgb}{0.121569,0.466667,0.705882}%
\pgfsetfillcolor{currentfill}%
\pgfsetfillopacity{0.823732}%
\pgfsetlinewidth{1.003750pt}%
\definecolor{currentstroke}{rgb}{0.121569,0.466667,0.705882}%
\pgfsetstrokecolor{currentstroke}%
\pgfsetstrokeopacity{0.823732}%
\pgfsetdash{}{0pt}%
\pgfpathmoveto{\pgfqpoint{2.914043in}{2.251229in}}%
\pgfpathcurveto{\pgfqpoint{2.922279in}{2.251229in}}{\pgfqpoint{2.930179in}{2.254501in}}{\pgfqpoint{2.936003in}{2.260325in}}%
\pgfpathcurveto{\pgfqpoint{2.941827in}{2.266149in}}{\pgfqpoint{2.945099in}{2.274049in}}{\pgfqpoint{2.945099in}{2.282286in}}%
\pgfpathcurveto{\pgfqpoint{2.945099in}{2.290522in}}{\pgfqpoint{2.941827in}{2.298422in}}{\pgfqpoint{2.936003in}{2.304246in}}%
\pgfpathcurveto{\pgfqpoint{2.930179in}{2.310070in}}{\pgfqpoint{2.922279in}{2.313342in}}{\pgfqpoint{2.914043in}{2.313342in}}%
\pgfpathcurveto{\pgfqpoint{2.905806in}{2.313342in}}{\pgfqpoint{2.897906in}{2.310070in}}{\pgfqpoint{2.892082in}{2.304246in}}%
\pgfpathcurveto{\pgfqpoint{2.886259in}{2.298422in}}{\pgfqpoint{2.882986in}{2.290522in}}{\pgfqpoint{2.882986in}{2.282286in}}%
\pgfpathcurveto{\pgfqpoint{2.882986in}{2.274049in}}{\pgfqpoint{2.886259in}{2.266149in}}{\pgfqpoint{2.892082in}{2.260325in}}%
\pgfpathcurveto{\pgfqpoint{2.897906in}{2.254501in}}{\pgfqpoint{2.905806in}{2.251229in}}{\pgfqpoint{2.914043in}{2.251229in}}%
\pgfpathclose%
\pgfusepath{stroke,fill}%
\end{pgfscope}%
\begin{pgfscope}%
\pgfpathrectangle{\pgfqpoint{0.100000in}{0.220728in}}{\pgfqpoint{3.696000in}{3.696000in}}%
\pgfusepath{clip}%
\pgfsetbuttcap%
\pgfsetroundjoin%
\definecolor{currentfill}{rgb}{0.121569,0.466667,0.705882}%
\pgfsetfillcolor{currentfill}%
\pgfsetfillopacity{0.826249}%
\pgfsetlinewidth{1.003750pt}%
\definecolor{currentstroke}{rgb}{0.121569,0.466667,0.705882}%
\pgfsetstrokecolor{currentstroke}%
\pgfsetstrokeopacity{0.826249}%
\pgfsetdash{}{0pt}%
\pgfpathmoveto{\pgfqpoint{1.458352in}{1.899037in}}%
\pgfpathcurveto{\pgfqpoint{1.466588in}{1.899037in}}{\pgfqpoint{1.474488in}{1.902309in}}{\pgfqpoint{1.480312in}{1.908133in}}%
\pgfpathcurveto{\pgfqpoint{1.486136in}{1.913957in}}{\pgfqpoint{1.489409in}{1.921857in}}{\pgfqpoint{1.489409in}{1.930093in}}%
\pgfpathcurveto{\pgfqpoint{1.489409in}{1.938329in}}{\pgfqpoint{1.486136in}{1.946229in}}{\pgfqpoint{1.480312in}{1.952053in}}%
\pgfpathcurveto{\pgfqpoint{1.474488in}{1.957877in}}{\pgfqpoint{1.466588in}{1.961150in}}{\pgfqpoint{1.458352in}{1.961150in}}%
\pgfpathcurveto{\pgfqpoint{1.450116in}{1.961150in}}{\pgfqpoint{1.442216in}{1.957877in}}{\pgfqpoint{1.436392in}{1.952053in}}%
\pgfpathcurveto{\pgfqpoint{1.430568in}{1.946229in}}{\pgfqpoint{1.427296in}{1.938329in}}{\pgfqpoint{1.427296in}{1.930093in}}%
\pgfpathcurveto{\pgfqpoint{1.427296in}{1.921857in}}{\pgfqpoint{1.430568in}{1.913957in}}{\pgfqpoint{1.436392in}{1.908133in}}%
\pgfpathcurveto{\pgfqpoint{1.442216in}{1.902309in}}{\pgfqpoint{1.450116in}{1.899037in}}{\pgfqpoint{1.458352in}{1.899037in}}%
\pgfpathclose%
\pgfusepath{stroke,fill}%
\end{pgfscope}%
\begin{pgfscope}%
\pgfpathrectangle{\pgfqpoint{0.100000in}{0.220728in}}{\pgfqpoint{3.696000in}{3.696000in}}%
\pgfusepath{clip}%
\pgfsetbuttcap%
\pgfsetroundjoin%
\definecolor{currentfill}{rgb}{0.121569,0.466667,0.705882}%
\pgfsetfillcolor{currentfill}%
\pgfsetfillopacity{0.828234}%
\pgfsetlinewidth{1.003750pt}%
\definecolor{currentstroke}{rgb}{0.121569,0.466667,0.705882}%
\pgfsetstrokecolor{currentstroke}%
\pgfsetstrokeopacity{0.828234}%
\pgfsetdash{}{0pt}%
\pgfpathmoveto{\pgfqpoint{2.901987in}{2.230559in}}%
\pgfpathcurveto{\pgfqpoint{2.910223in}{2.230559in}}{\pgfqpoint{2.918123in}{2.233831in}}{\pgfqpoint{2.923947in}{2.239655in}}%
\pgfpathcurveto{\pgfqpoint{2.929771in}{2.245479in}}{\pgfqpoint{2.933043in}{2.253379in}}{\pgfqpoint{2.933043in}{2.261615in}}%
\pgfpathcurveto{\pgfqpoint{2.933043in}{2.269852in}}{\pgfqpoint{2.929771in}{2.277752in}}{\pgfqpoint{2.923947in}{2.283576in}}%
\pgfpathcurveto{\pgfqpoint{2.918123in}{2.289400in}}{\pgfqpoint{2.910223in}{2.292672in}}{\pgfqpoint{2.901987in}{2.292672in}}%
\pgfpathcurveto{\pgfqpoint{2.893751in}{2.292672in}}{\pgfqpoint{2.885851in}{2.289400in}}{\pgfqpoint{2.880027in}{2.283576in}}%
\pgfpathcurveto{\pgfqpoint{2.874203in}{2.277752in}}{\pgfqpoint{2.870930in}{2.269852in}}{\pgfqpoint{2.870930in}{2.261615in}}%
\pgfpathcurveto{\pgfqpoint{2.870930in}{2.253379in}}{\pgfqpoint{2.874203in}{2.245479in}}{\pgfqpoint{2.880027in}{2.239655in}}%
\pgfpathcurveto{\pgfqpoint{2.885851in}{2.233831in}}{\pgfqpoint{2.893751in}{2.230559in}}{\pgfqpoint{2.901987in}{2.230559in}}%
\pgfpathclose%
\pgfusepath{stroke,fill}%
\end{pgfscope}%
\begin{pgfscope}%
\pgfpathrectangle{\pgfqpoint{0.100000in}{0.220728in}}{\pgfqpoint{3.696000in}{3.696000in}}%
\pgfusepath{clip}%
\pgfsetbuttcap%
\pgfsetroundjoin%
\definecolor{currentfill}{rgb}{0.121569,0.466667,0.705882}%
\pgfsetfillcolor{currentfill}%
\pgfsetfillopacity{0.829601}%
\pgfsetlinewidth{1.003750pt}%
\definecolor{currentstroke}{rgb}{0.121569,0.466667,0.705882}%
\pgfsetstrokecolor{currentstroke}%
\pgfsetstrokeopacity{0.829601}%
\pgfsetdash{}{0pt}%
\pgfpathmoveto{\pgfqpoint{1.477025in}{1.891045in}}%
\pgfpathcurveto{\pgfqpoint{1.485261in}{1.891045in}}{\pgfqpoint{1.493161in}{1.894318in}}{\pgfqpoint{1.498985in}{1.900142in}}%
\pgfpathcurveto{\pgfqpoint{1.504809in}{1.905966in}}{\pgfqpoint{1.508081in}{1.913866in}}{\pgfqpoint{1.508081in}{1.922102in}}%
\pgfpathcurveto{\pgfqpoint{1.508081in}{1.930338in}}{\pgfqpoint{1.504809in}{1.938238in}}{\pgfqpoint{1.498985in}{1.944062in}}%
\pgfpathcurveto{\pgfqpoint{1.493161in}{1.949886in}}{\pgfqpoint{1.485261in}{1.953158in}}{\pgfqpoint{1.477025in}{1.953158in}}%
\pgfpathcurveto{\pgfqpoint{1.468788in}{1.953158in}}{\pgfqpoint{1.460888in}{1.949886in}}{\pgfqpoint{1.455064in}{1.944062in}}%
\pgfpathcurveto{\pgfqpoint{1.449240in}{1.938238in}}{\pgfqpoint{1.445968in}{1.930338in}}{\pgfqpoint{1.445968in}{1.922102in}}%
\pgfpathcurveto{\pgfqpoint{1.445968in}{1.913866in}}{\pgfqpoint{1.449240in}{1.905966in}}{\pgfqpoint{1.455064in}{1.900142in}}%
\pgfpathcurveto{\pgfqpoint{1.460888in}{1.894318in}}{\pgfqpoint{1.468788in}{1.891045in}}{\pgfqpoint{1.477025in}{1.891045in}}%
\pgfpathclose%
\pgfusepath{stroke,fill}%
\end{pgfscope}%
\begin{pgfscope}%
\pgfpathrectangle{\pgfqpoint{0.100000in}{0.220728in}}{\pgfqpoint{3.696000in}{3.696000in}}%
\pgfusepath{clip}%
\pgfsetbuttcap%
\pgfsetroundjoin%
\definecolor{currentfill}{rgb}{0.121569,0.466667,0.705882}%
\pgfsetfillcolor{currentfill}%
\pgfsetfillopacity{0.832502}%
\pgfsetlinewidth{1.003750pt}%
\definecolor{currentstroke}{rgb}{0.121569,0.466667,0.705882}%
\pgfsetstrokecolor{currentstroke}%
\pgfsetstrokeopacity{0.832502}%
\pgfsetdash{}{0pt}%
\pgfpathmoveto{\pgfqpoint{1.493409in}{1.887834in}}%
\pgfpathcurveto{\pgfqpoint{1.501646in}{1.887834in}}{\pgfqpoint{1.509546in}{1.891106in}}{\pgfqpoint{1.515370in}{1.896930in}}%
\pgfpathcurveto{\pgfqpoint{1.521194in}{1.902754in}}{\pgfqpoint{1.524466in}{1.910654in}}{\pgfqpoint{1.524466in}{1.918890in}}%
\pgfpathcurveto{\pgfqpoint{1.524466in}{1.927126in}}{\pgfqpoint{1.521194in}{1.935026in}}{\pgfqpoint{1.515370in}{1.940850in}}%
\pgfpathcurveto{\pgfqpoint{1.509546in}{1.946674in}}{\pgfqpoint{1.501646in}{1.949947in}}{\pgfqpoint{1.493409in}{1.949947in}}%
\pgfpathcurveto{\pgfqpoint{1.485173in}{1.949947in}}{\pgfqpoint{1.477273in}{1.946674in}}{\pgfqpoint{1.471449in}{1.940850in}}%
\pgfpathcurveto{\pgfqpoint{1.465625in}{1.935026in}}{\pgfqpoint{1.462353in}{1.927126in}}{\pgfqpoint{1.462353in}{1.918890in}}%
\pgfpathcurveto{\pgfqpoint{1.462353in}{1.910654in}}{\pgfqpoint{1.465625in}{1.902754in}}{\pgfqpoint{1.471449in}{1.896930in}}%
\pgfpathcurveto{\pgfqpoint{1.477273in}{1.891106in}}{\pgfqpoint{1.485173in}{1.887834in}}{\pgfqpoint{1.493409in}{1.887834in}}%
\pgfpathclose%
\pgfusepath{stroke,fill}%
\end{pgfscope}%
\begin{pgfscope}%
\pgfpathrectangle{\pgfqpoint{0.100000in}{0.220728in}}{\pgfqpoint{3.696000in}{3.696000in}}%
\pgfusepath{clip}%
\pgfsetbuttcap%
\pgfsetroundjoin%
\definecolor{currentfill}{rgb}{0.121569,0.466667,0.705882}%
\pgfsetfillcolor{currentfill}%
\pgfsetfillopacity{0.832862}%
\pgfsetlinewidth{1.003750pt}%
\definecolor{currentstroke}{rgb}{0.121569,0.466667,0.705882}%
\pgfsetstrokecolor{currentstroke}%
\pgfsetstrokeopacity{0.832862}%
\pgfsetdash{}{0pt}%
\pgfpathmoveto{\pgfqpoint{2.885490in}{2.207684in}}%
\pgfpathcurveto{\pgfqpoint{2.893726in}{2.207684in}}{\pgfqpoint{2.901626in}{2.210956in}}{\pgfqpoint{2.907450in}{2.216780in}}%
\pgfpathcurveto{\pgfqpoint{2.913274in}{2.222604in}}{\pgfqpoint{2.916546in}{2.230504in}}{\pgfqpoint{2.916546in}{2.238741in}}%
\pgfpathcurveto{\pgfqpoint{2.916546in}{2.246977in}}{\pgfqpoint{2.913274in}{2.254877in}}{\pgfqpoint{2.907450in}{2.260701in}}%
\pgfpathcurveto{\pgfqpoint{2.901626in}{2.266525in}}{\pgfqpoint{2.893726in}{2.269797in}}{\pgfqpoint{2.885490in}{2.269797in}}%
\pgfpathcurveto{\pgfqpoint{2.877254in}{2.269797in}}{\pgfqpoint{2.869354in}{2.266525in}}{\pgfqpoint{2.863530in}{2.260701in}}%
\pgfpathcurveto{\pgfqpoint{2.857706in}{2.254877in}}{\pgfqpoint{2.854433in}{2.246977in}}{\pgfqpoint{2.854433in}{2.238741in}}%
\pgfpathcurveto{\pgfqpoint{2.854433in}{2.230504in}}{\pgfqpoint{2.857706in}{2.222604in}}{\pgfqpoint{2.863530in}{2.216780in}}%
\pgfpathcurveto{\pgfqpoint{2.869354in}{2.210956in}}{\pgfqpoint{2.877254in}{2.207684in}}{\pgfqpoint{2.885490in}{2.207684in}}%
\pgfpathclose%
\pgfusepath{stroke,fill}%
\end{pgfscope}%
\begin{pgfscope}%
\pgfpathrectangle{\pgfqpoint{0.100000in}{0.220728in}}{\pgfqpoint{3.696000in}{3.696000in}}%
\pgfusepath{clip}%
\pgfsetbuttcap%
\pgfsetroundjoin%
\definecolor{currentfill}{rgb}{0.121569,0.466667,0.705882}%
\pgfsetfillcolor{currentfill}%
\pgfsetfillopacity{0.834690}%
\pgfsetlinewidth{1.003750pt}%
\definecolor{currentstroke}{rgb}{0.121569,0.466667,0.705882}%
\pgfsetstrokecolor{currentstroke}%
\pgfsetstrokeopacity{0.834690}%
\pgfsetdash{}{0pt}%
\pgfpathmoveto{\pgfqpoint{1.506606in}{1.880158in}}%
\pgfpathcurveto{\pgfqpoint{1.514842in}{1.880158in}}{\pgfqpoint{1.522742in}{1.883430in}}{\pgfqpoint{1.528566in}{1.889254in}}%
\pgfpathcurveto{\pgfqpoint{1.534390in}{1.895078in}}{\pgfqpoint{1.537663in}{1.902978in}}{\pgfqpoint{1.537663in}{1.911214in}}%
\pgfpathcurveto{\pgfqpoint{1.537663in}{1.919451in}}{\pgfqpoint{1.534390in}{1.927351in}}{\pgfqpoint{1.528566in}{1.933175in}}%
\pgfpathcurveto{\pgfqpoint{1.522742in}{1.938999in}}{\pgfqpoint{1.514842in}{1.942271in}}{\pgfqpoint{1.506606in}{1.942271in}}%
\pgfpathcurveto{\pgfqpoint{1.498370in}{1.942271in}}{\pgfqpoint{1.490470in}{1.938999in}}{\pgfqpoint{1.484646in}{1.933175in}}%
\pgfpathcurveto{\pgfqpoint{1.478822in}{1.927351in}}{\pgfqpoint{1.475550in}{1.919451in}}{\pgfqpoint{1.475550in}{1.911214in}}%
\pgfpathcurveto{\pgfqpoint{1.475550in}{1.902978in}}{\pgfqpoint{1.478822in}{1.895078in}}{\pgfqpoint{1.484646in}{1.889254in}}%
\pgfpathcurveto{\pgfqpoint{1.490470in}{1.883430in}}{\pgfqpoint{1.498370in}{1.880158in}}{\pgfqpoint{1.506606in}{1.880158in}}%
\pgfpathclose%
\pgfusepath{stroke,fill}%
\end{pgfscope}%
\begin{pgfscope}%
\pgfpathrectangle{\pgfqpoint{0.100000in}{0.220728in}}{\pgfqpoint{3.696000in}{3.696000in}}%
\pgfusepath{clip}%
\pgfsetbuttcap%
\pgfsetroundjoin%
\definecolor{currentfill}{rgb}{0.121569,0.466667,0.705882}%
\pgfsetfillcolor{currentfill}%
\pgfsetfillopacity{0.838565}%
\pgfsetlinewidth{1.003750pt}%
\definecolor{currentstroke}{rgb}{0.121569,0.466667,0.705882}%
\pgfsetstrokecolor{currentstroke}%
\pgfsetstrokeopacity{0.838565}%
\pgfsetdash{}{0pt}%
\pgfpathmoveto{\pgfqpoint{2.870149in}{2.182457in}}%
\pgfpathcurveto{\pgfqpoint{2.878385in}{2.182457in}}{\pgfqpoint{2.886285in}{2.185729in}}{\pgfqpoint{2.892109in}{2.191553in}}%
\pgfpathcurveto{\pgfqpoint{2.897933in}{2.197377in}}{\pgfqpoint{2.901206in}{2.205277in}}{\pgfqpoint{2.901206in}{2.213514in}}%
\pgfpathcurveto{\pgfqpoint{2.901206in}{2.221750in}}{\pgfqpoint{2.897933in}{2.229650in}}{\pgfqpoint{2.892109in}{2.235474in}}%
\pgfpathcurveto{\pgfqpoint{2.886285in}{2.241298in}}{\pgfqpoint{2.878385in}{2.244570in}}{\pgfqpoint{2.870149in}{2.244570in}}%
\pgfpathcurveto{\pgfqpoint{2.861913in}{2.244570in}}{\pgfqpoint{2.854013in}{2.241298in}}{\pgfqpoint{2.848189in}{2.235474in}}%
\pgfpathcurveto{\pgfqpoint{2.842365in}{2.229650in}}{\pgfqpoint{2.839093in}{2.221750in}}{\pgfqpoint{2.839093in}{2.213514in}}%
\pgfpathcurveto{\pgfqpoint{2.839093in}{2.205277in}}{\pgfqpoint{2.842365in}{2.197377in}}{\pgfqpoint{2.848189in}{2.191553in}}%
\pgfpathcurveto{\pgfqpoint{2.854013in}{2.185729in}}{\pgfqpoint{2.861913in}{2.182457in}}{\pgfqpoint{2.870149in}{2.182457in}}%
\pgfpathclose%
\pgfusepath{stroke,fill}%
\end{pgfscope}%
\begin{pgfscope}%
\pgfpathrectangle{\pgfqpoint{0.100000in}{0.220728in}}{\pgfqpoint{3.696000in}{3.696000in}}%
\pgfusepath{clip}%
\pgfsetbuttcap%
\pgfsetroundjoin%
\definecolor{currentfill}{rgb}{0.121569,0.466667,0.705882}%
\pgfsetfillcolor{currentfill}%
\pgfsetfillopacity{0.838925}%
\pgfsetlinewidth{1.003750pt}%
\definecolor{currentstroke}{rgb}{0.121569,0.466667,0.705882}%
\pgfsetstrokecolor{currentstroke}%
\pgfsetstrokeopacity{0.838925}%
\pgfsetdash{}{0pt}%
\pgfpathmoveto{\pgfqpoint{1.530527in}{1.867081in}}%
\pgfpathcurveto{\pgfqpoint{1.538763in}{1.867081in}}{\pgfqpoint{1.546663in}{1.870353in}}{\pgfqpoint{1.552487in}{1.876177in}}%
\pgfpathcurveto{\pgfqpoint{1.558311in}{1.882001in}}{\pgfqpoint{1.561583in}{1.889901in}}{\pgfqpoint{1.561583in}{1.898137in}}%
\pgfpathcurveto{\pgfqpoint{1.561583in}{1.906374in}}{\pgfqpoint{1.558311in}{1.914274in}}{\pgfqpoint{1.552487in}{1.920098in}}%
\pgfpathcurveto{\pgfqpoint{1.546663in}{1.925922in}}{\pgfqpoint{1.538763in}{1.929194in}}{\pgfqpoint{1.530527in}{1.929194in}}%
\pgfpathcurveto{\pgfqpoint{1.522291in}{1.929194in}}{\pgfqpoint{1.514390in}{1.925922in}}{\pgfqpoint{1.508567in}{1.920098in}}%
\pgfpathcurveto{\pgfqpoint{1.502743in}{1.914274in}}{\pgfqpoint{1.499470in}{1.906374in}}{\pgfqpoint{1.499470in}{1.898137in}}%
\pgfpathcurveto{\pgfqpoint{1.499470in}{1.889901in}}{\pgfqpoint{1.502743in}{1.882001in}}{\pgfqpoint{1.508567in}{1.876177in}}%
\pgfpathcurveto{\pgfqpoint{1.514390in}{1.870353in}}{\pgfqpoint{1.522291in}{1.867081in}}{\pgfqpoint{1.530527in}{1.867081in}}%
\pgfpathclose%
\pgfusepath{stroke,fill}%
\end{pgfscope}%
\begin{pgfscope}%
\pgfpathrectangle{\pgfqpoint{0.100000in}{0.220728in}}{\pgfqpoint{3.696000in}{3.696000in}}%
\pgfusepath{clip}%
\pgfsetbuttcap%
\pgfsetroundjoin%
\definecolor{currentfill}{rgb}{0.121569,0.466667,0.705882}%
\pgfsetfillcolor{currentfill}%
\pgfsetfillopacity{0.841281}%
\pgfsetlinewidth{1.003750pt}%
\definecolor{currentstroke}{rgb}{0.121569,0.466667,0.705882}%
\pgfsetstrokecolor{currentstroke}%
\pgfsetstrokeopacity{0.841281}%
\pgfsetdash{}{0pt}%
\pgfpathmoveto{\pgfqpoint{2.860345in}{2.168333in}}%
\pgfpathcurveto{\pgfqpoint{2.868581in}{2.168333in}}{\pgfqpoint{2.876481in}{2.171606in}}{\pgfqpoint{2.882305in}{2.177430in}}%
\pgfpathcurveto{\pgfqpoint{2.888129in}{2.183254in}}{\pgfqpoint{2.891401in}{2.191154in}}{\pgfqpoint{2.891401in}{2.199390in}}%
\pgfpathcurveto{\pgfqpoint{2.891401in}{2.207626in}}{\pgfqpoint{2.888129in}{2.215526in}}{\pgfqpoint{2.882305in}{2.221350in}}%
\pgfpathcurveto{\pgfqpoint{2.876481in}{2.227174in}}{\pgfqpoint{2.868581in}{2.230446in}}{\pgfqpoint{2.860345in}{2.230446in}}%
\pgfpathcurveto{\pgfqpoint{2.852109in}{2.230446in}}{\pgfqpoint{2.844209in}{2.227174in}}{\pgfqpoint{2.838385in}{2.221350in}}%
\pgfpathcurveto{\pgfqpoint{2.832561in}{2.215526in}}{\pgfqpoint{2.829288in}{2.207626in}}{\pgfqpoint{2.829288in}{2.199390in}}%
\pgfpathcurveto{\pgfqpoint{2.829288in}{2.191154in}}{\pgfqpoint{2.832561in}{2.183254in}}{\pgfqpoint{2.838385in}{2.177430in}}%
\pgfpathcurveto{\pgfqpoint{2.844209in}{2.171606in}}{\pgfqpoint{2.852109in}{2.168333in}}{\pgfqpoint{2.860345in}{2.168333in}}%
\pgfpathclose%
\pgfusepath{stroke,fill}%
\end{pgfscope}%
\begin{pgfscope}%
\pgfpathrectangle{\pgfqpoint{0.100000in}{0.220728in}}{\pgfqpoint{3.696000in}{3.696000in}}%
\pgfusepath{clip}%
\pgfsetbuttcap%
\pgfsetroundjoin%
\definecolor{currentfill}{rgb}{0.121569,0.466667,0.705882}%
\pgfsetfillcolor{currentfill}%
\pgfsetfillopacity{0.842761}%
\pgfsetlinewidth{1.003750pt}%
\definecolor{currentstroke}{rgb}{0.121569,0.466667,0.705882}%
\pgfsetstrokecolor{currentstroke}%
\pgfsetstrokeopacity{0.842761}%
\pgfsetdash{}{0pt}%
\pgfpathmoveto{\pgfqpoint{2.854773in}{2.160819in}}%
\pgfpathcurveto{\pgfqpoint{2.863009in}{2.160819in}}{\pgfqpoint{2.870909in}{2.164091in}}{\pgfqpoint{2.876733in}{2.169915in}}%
\pgfpathcurveto{\pgfqpoint{2.882557in}{2.175739in}}{\pgfqpoint{2.885829in}{2.183639in}}{\pgfqpoint{2.885829in}{2.191875in}}%
\pgfpathcurveto{\pgfqpoint{2.885829in}{2.200112in}}{\pgfqpoint{2.882557in}{2.208012in}}{\pgfqpoint{2.876733in}{2.213836in}}%
\pgfpathcurveto{\pgfqpoint{2.870909in}{2.219660in}}{\pgfqpoint{2.863009in}{2.222932in}}{\pgfqpoint{2.854773in}{2.222932in}}%
\pgfpathcurveto{\pgfqpoint{2.846536in}{2.222932in}}{\pgfqpoint{2.838636in}{2.219660in}}{\pgfqpoint{2.832812in}{2.213836in}}%
\pgfpathcurveto{\pgfqpoint{2.826988in}{2.208012in}}{\pgfqpoint{2.823716in}{2.200112in}}{\pgfqpoint{2.823716in}{2.191875in}}%
\pgfpathcurveto{\pgfqpoint{2.823716in}{2.183639in}}{\pgfqpoint{2.826988in}{2.175739in}}{\pgfqpoint{2.832812in}{2.169915in}}%
\pgfpathcurveto{\pgfqpoint{2.838636in}{2.164091in}}{\pgfqpoint{2.846536in}{2.160819in}}{\pgfqpoint{2.854773in}{2.160819in}}%
\pgfpathclose%
\pgfusepath{stroke,fill}%
\end{pgfscope}%
\begin{pgfscope}%
\pgfpathrectangle{\pgfqpoint{0.100000in}{0.220728in}}{\pgfqpoint{3.696000in}{3.696000in}}%
\pgfusepath{clip}%
\pgfsetbuttcap%
\pgfsetroundjoin%
\definecolor{currentfill}{rgb}{0.121569,0.466667,0.705882}%
\pgfsetfillcolor{currentfill}%
\pgfsetfillopacity{0.843641}%
\pgfsetlinewidth{1.003750pt}%
\definecolor{currentstroke}{rgb}{0.121569,0.466667,0.705882}%
\pgfsetstrokecolor{currentstroke}%
\pgfsetstrokeopacity{0.843641}%
\pgfsetdash{}{0pt}%
\pgfpathmoveto{\pgfqpoint{2.852069in}{2.156411in}}%
\pgfpathcurveto{\pgfqpoint{2.860305in}{2.156411in}}{\pgfqpoint{2.868205in}{2.159684in}}{\pgfqpoint{2.874029in}{2.165508in}}%
\pgfpathcurveto{\pgfqpoint{2.879853in}{2.171331in}}{\pgfqpoint{2.883125in}{2.179232in}}{\pgfqpoint{2.883125in}{2.187468in}}%
\pgfpathcurveto{\pgfqpoint{2.883125in}{2.195704in}}{\pgfqpoint{2.879853in}{2.203604in}}{\pgfqpoint{2.874029in}{2.209428in}}%
\pgfpathcurveto{\pgfqpoint{2.868205in}{2.215252in}}{\pgfqpoint{2.860305in}{2.218524in}}{\pgfqpoint{2.852069in}{2.218524in}}%
\pgfpathcurveto{\pgfqpoint{2.843832in}{2.218524in}}{\pgfqpoint{2.835932in}{2.215252in}}{\pgfqpoint{2.830108in}{2.209428in}}%
\pgfpathcurveto{\pgfqpoint{2.824285in}{2.203604in}}{\pgfqpoint{2.821012in}{2.195704in}}{\pgfqpoint{2.821012in}{2.187468in}}%
\pgfpathcurveto{\pgfqpoint{2.821012in}{2.179232in}}{\pgfqpoint{2.824285in}{2.171331in}}{\pgfqpoint{2.830108in}{2.165508in}}%
\pgfpathcurveto{\pgfqpoint{2.835932in}{2.159684in}}{\pgfqpoint{2.843832in}{2.156411in}}{\pgfqpoint{2.852069in}{2.156411in}}%
\pgfpathclose%
\pgfusepath{stroke,fill}%
\end{pgfscope}%
\begin{pgfscope}%
\pgfpathrectangle{\pgfqpoint{0.100000in}{0.220728in}}{\pgfqpoint{3.696000in}{3.696000in}}%
\pgfusepath{clip}%
\pgfsetbuttcap%
\pgfsetroundjoin%
\definecolor{currentfill}{rgb}{0.121569,0.466667,0.705882}%
\pgfsetfillcolor{currentfill}%
\pgfsetfillopacity{0.844069}%
\pgfsetlinewidth{1.003750pt}%
\definecolor{currentstroke}{rgb}{0.121569,0.466667,0.705882}%
\pgfsetstrokecolor{currentstroke}%
\pgfsetstrokeopacity{0.844069}%
\pgfsetdash{}{0pt}%
\pgfpathmoveto{\pgfqpoint{2.850305in}{2.154153in}}%
\pgfpathcurveto{\pgfqpoint{2.858541in}{2.154153in}}{\pgfqpoint{2.866441in}{2.157425in}}{\pgfqpoint{2.872265in}{2.163249in}}%
\pgfpathcurveto{\pgfqpoint{2.878089in}{2.169073in}}{\pgfqpoint{2.881362in}{2.176973in}}{\pgfqpoint{2.881362in}{2.185209in}}%
\pgfpathcurveto{\pgfqpoint{2.881362in}{2.193445in}}{\pgfqpoint{2.878089in}{2.201345in}}{\pgfqpoint{2.872265in}{2.207169in}}%
\pgfpathcurveto{\pgfqpoint{2.866441in}{2.212993in}}{\pgfqpoint{2.858541in}{2.216266in}}{\pgfqpoint{2.850305in}{2.216266in}}%
\pgfpathcurveto{\pgfqpoint{2.842069in}{2.216266in}}{\pgfqpoint{2.834169in}{2.212993in}}{\pgfqpoint{2.828345in}{2.207169in}}%
\pgfpathcurveto{\pgfqpoint{2.822521in}{2.201345in}}{\pgfqpoint{2.819249in}{2.193445in}}{\pgfqpoint{2.819249in}{2.185209in}}%
\pgfpathcurveto{\pgfqpoint{2.819249in}{2.176973in}}{\pgfqpoint{2.822521in}{2.169073in}}{\pgfqpoint{2.828345in}{2.163249in}}%
\pgfpathcurveto{\pgfqpoint{2.834169in}{2.157425in}}{\pgfqpoint{2.842069in}{2.154153in}}{\pgfqpoint{2.850305in}{2.154153in}}%
\pgfpathclose%
\pgfusepath{stroke,fill}%
\end{pgfscope}%
\begin{pgfscope}%
\pgfpathrectangle{\pgfqpoint{0.100000in}{0.220728in}}{\pgfqpoint{3.696000in}{3.696000in}}%
\pgfusepath{clip}%
\pgfsetbuttcap%
\pgfsetroundjoin%
\definecolor{currentfill}{rgb}{0.121569,0.466667,0.705882}%
\pgfsetfillcolor{currentfill}%
\pgfsetfillopacity{0.844352}%
\pgfsetlinewidth{1.003750pt}%
\definecolor{currentstroke}{rgb}{0.121569,0.466667,0.705882}%
\pgfsetstrokecolor{currentstroke}%
\pgfsetstrokeopacity{0.844352}%
\pgfsetdash{}{0pt}%
\pgfpathmoveto{\pgfqpoint{2.849563in}{2.152811in}}%
\pgfpathcurveto{\pgfqpoint{2.857800in}{2.152811in}}{\pgfqpoint{2.865700in}{2.156084in}}{\pgfqpoint{2.871524in}{2.161908in}}%
\pgfpathcurveto{\pgfqpoint{2.877348in}{2.167732in}}{\pgfqpoint{2.880620in}{2.175632in}}{\pgfqpoint{2.880620in}{2.183868in}}%
\pgfpathcurveto{\pgfqpoint{2.880620in}{2.192104in}}{\pgfqpoint{2.877348in}{2.200004in}}{\pgfqpoint{2.871524in}{2.205828in}}%
\pgfpathcurveto{\pgfqpoint{2.865700in}{2.211652in}}{\pgfqpoint{2.857800in}{2.214924in}}{\pgfqpoint{2.849563in}{2.214924in}}%
\pgfpathcurveto{\pgfqpoint{2.841327in}{2.214924in}}{\pgfqpoint{2.833427in}{2.211652in}}{\pgfqpoint{2.827603in}{2.205828in}}%
\pgfpathcurveto{\pgfqpoint{2.821779in}{2.200004in}}{\pgfqpoint{2.818507in}{2.192104in}}{\pgfqpoint{2.818507in}{2.183868in}}%
\pgfpathcurveto{\pgfqpoint{2.818507in}{2.175632in}}{\pgfqpoint{2.821779in}{2.167732in}}{\pgfqpoint{2.827603in}{2.161908in}}%
\pgfpathcurveto{\pgfqpoint{2.833427in}{2.156084in}}{\pgfqpoint{2.841327in}{2.152811in}}{\pgfqpoint{2.849563in}{2.152811in}}%
\pgfpathclose%
\pgfusepath{stroke,fill}%
\end{pgfscope}%
\begin{pgfscope}%
\pgfpathrectangle{\pgfqpoint{0.100000in}{0.220728in}}{\pgfqpoint{3.696000in}{3.696000in}}%
\pgfusepath{clip}%
\pgfsetbuttcap%
\pgfsetroundjoin%
\definecolor{currentfill}{rgb}{0.121569,0.466667,0.705882}%
\pgfsetfillcolor{currentfill}%
\pgfsetfillopacity{0.845254}%
\pgfsetlinewidth{1.003750pt}%
\definecolor{currentstroke}{rgb}{0.121569,0.466667,0.705882}%
\pgfsetstrokecolor{currentstroke}%
\pgfsetstrokeopacity{0.845254}%
\pgfsetdash{}{0pt}%
\pgfpathmoveto{\pgfqpoint{2.846166in}{2.148315in}}%
\pgfpathcurveto{\pgfqpoint{2.854402in}{2.148315in}}{\pgfqpoint{2.862302in}{2.151588in}}{\pgfqpoint{2.868126in}{2.157412in}}%
\pgfpathcurveto{\pgfqpoint{2.873950in}{2.163236in}}{\pgfqpoint{2.877222in}{2.171136in}}{\pgfqpoint{2.877222in}{2.179372in}}%
\pgfpathcurveto{\pgfqpoint{2.877222in}{2.187608in}}{\pgfqpoint{2.873950in}{2.195508in}}{\pgfqpoint{2.868126in}{2.201332in}}%
\pgfpathcurveto{\pgfqpoint{2.862302in}{2.207156in}}{\pgfqpoint{2.854402in}{2.210428in}}{\pgfqpoint{2.846166in}{2.210428in}}%
\pgfpathcurveto{\pgfqpoint{2.837930in}{2.210428in}}{\pgfqpoint{2.830030in}{2.207156in}}{\pgfqpoint{2.824206in}{2.201332in}}%
\pgfpathcurveto{\pgfqpoint{2.818382in}{2.195508in}}{\pgfqpoint{2.815109in}{2.187608in}}{\pgfqpoint{2.815109in}{2.179372in}}%
\pgfpathcurveto{\pgfqpoint{2.815109in}{2.171136in}}{\pgfqpoint{2.818382in}{2.163236in}}{\pgfqpoint{2.824206in}{2.157412in}}%
\pgfpathcurveto{\pgfqpoint{2.830030in}{2.151588in}}{\pgfqpoint{2.837930in}{2.148315in}}{\pgfqpoint{2.846166in}{2.148315in}}%
\pgfpathclose%
\pgfusepath{stroke,fill}%
\end{pgfscope}%
\begin{pgfscope}%
\pgfpathrectangle{\pgfqpoint{0.100000in}{0.220728in}}{\pgfqpoint{3.696000in}{3.696000in}}%
\pgfusepath{clip}%
\pgfsetbuttcap%
\pgfsetroundjoin%
\definecolor{currentfill}{rgb}{0.121569,0.466667,0.705882}%
\pgfsetfillcolor{currentfill}%
\pgfsetfillopacity{0.846696}%
\pgfsetlinewidth{1.003750pt}%
\definecolor{currentstroke}{rgb}{0.121569,0.466667,0.705882}%
\pgfsetstrokecolor{currentstroke}%
\pgfsetstrokeopacity{0.846696}%
\pgfsetdash{}{0pt}%
\pgfpathmoveto{\pgfqpoint{2.842069in}{2.140088in}}%
\pgfpathcurveto{\pgfqpoint{2.850306in}{2.140088in}}{\pgfqpoint{2.858206in}{2.143360in}}{\pgfqpoint{2.864030in}{2.149184in}}%
\pgfpathcurveto{\pgfqpoint{2.869853in}{2.155008in}}{\pgfqpoint{2.873126in}{2.162908in}}{\pgfqpoint{2.873126in}{2.171144in}}%
\pgfpathcurveto{\pgfqpoint{2.873126in}{2.179381in}}{\pgfqpoint{2.869853in}{2.187281in}}{\pgfqpoint{2.864030in}{2.193104in}}%
\pgfpathcurveto{\pgfqpoint{2.858206in}{2.198928in}}{\pgfqpoint{2.850306in}{2.202201in}}{\pgfqpoint{2.842069in}{2.202201in}}%
\pgfpathcurveto{\pgfqpoint{2.833833in}{2.202201in}}{\pgfqpoint{2.825933in}{2.198928in}}{\pgfqpoint{2.820109in}{2.193104in}}%
\pgfpathcurveto{\pgfqpoint{2.814285in}{2.187281in}}{\pgfqpoint{2.811013in}{2.179381in}}{\pgfqpoint{2.811013in}{2.171144in}}%
\pgfpathcurveto{\pgfqpoint{2.811013in}{2.162908in}}{\pgfqpoint{2.814285in}{2.155008in}}{\pgfqpoint{2.820109in}{2.149184in}}%
\pgfpathcurveto{\pgfqpoint{2.825933in}{2.143360in}}{\pgfqpoint{2.833833in}{2.140088in}}{\pgfqpoint{2.842069in}{2.140088in}}%
\pgfpathclose%
\pgfusepath{stroke,fill}%
\end{pgfscope}%
\begin{pgfscope}%
\pgfpathrectangle{\pgfqpoint{0.100000in}{0.220728in}}{\pgfqpoint{3.696000in}{3.696000in}}%
\pgfusepath{clip}%
\pgfsetbuttcap%
\pgfsetroundjoin%
\definecolor{currentfill}{rgb}{0.121569,0.466667,0.705882}%
\pgfsetfillcolor{currentfill}%
\pgfsetfillopacity{0.847424}%
\pgfsetlinewidth{1.003750pt}%
\definecolor{currentstroke}{rgb}{0.121569,0.466667,0.705882}%
\pgfsetstrokecolor{currentstroke}%
\pgfsetstrokeopacity{0.847424}%
\pgfsetdash{}{0pt}%
\pgfpathmoveto{\pgfqpoint{1.574362in}{1.848498in}}%
\pgfpathcurveto{\pgfqpoint{1.582599in}{1.848498in}}{\pgfqpoint{1.590499in}{1.851770in}}{\pgfqpoint{1.596323in}{1.857594in}}%
\pgfpathcurveto{\pgfqpoint{1.602147in}{1.863418in}}{\pgfqpoint{1.605419in}{1.871318in}}{\pgfqpoint{1.605419in}{1.879554in}}%
\pgfpathcurveto{\pgfqpoint{1.605419in}{1.887790in}}{\pgfqpoint{1.602147in}{1.895691in}}{\pgfqpoint{1.596323in}{1.901514in}}%
\pgfpathcurveto{\pgfqpoint{1.590499in}{1.907338in}}{\pgfqpoint{1.582599in}{1.910611in}}{\pgfqpoint{1.574362in}{1.910611in}}%
\pgfpathcurveto{\pgfqpoint{1.566126in}{1.910611in}}{\pgfqpoint{1.558226in}{1.907338in}}{\pgfqpoint{1.552402in}{1.901514in}}%
\pgfpathcurveto{\pgfqpoint{1.546578in}{1.895691in}}{\pgfqpoint{1.543306in}{1.887790in}}{\pgfqpoint{1.543306in}{1.879554in}}%
\pgfpathcurveto{\pgfqpoint{1.543306in}{1.871318in}}{\pgfqpoint{1.546578in}{1.863418in}}{\pgfqpoint{1.552402in}{1.857594in}}%
\pgfpathcurveto{\pgfqpoint{1.558226in}{1.851770in}}{\pgfqpoint{1.566126in}{1.848498in}}{\pgfqpoint{1.574362in}{1.848498in}}%
\pgfpathclose%
\pgfusepath{stroke,fill}%
\end{pgfscope}%
\begin{pgfscope}%
\pgfpathrectangle{\pgfqpoint{0.100000in}{0.220728in}}{\pgfqpoint{3.696000in}{3.696000in}}%
\pgfusepath{clip}%
\pgfsetbuttcap%
\pgfsetroundjoin%
\definecolor{currentfill}{rgb}{0.121569,0.466667,0.705882}%
\pgfsetfillcolor{currentfill}%
\pgfsetfillopacity{0.848940}%
\pgfsetlinewidth{1.003750pt}%
\definecolor{currentstroke}{rgb}{0.121569,0.466667,0.705882}%
\pgfsetstrokecolor{currentstroke}%
\pgfsetstrokeopacity{0.848940}%
\pgfsetdash{}{0pt}%
\pgfpathmoveto{\pgfqpoint{2.834196in}{2.128461in}}%
\pgfpathcurveto{\pgfqpoint{2.842432in}{2.128461in}}{\pgfqpoint{2.850332in}{2.131733in}}{\pgfqpoint{2.856156in}{2.137557in}}%
\pgfpathcurveto{\pgfqpoint{2.861980in}{2.143381in}}{\pgfqpoint{2.865252in}{2.151281in}}{\pgfqpoint{2.865252in}{2.159518in}}%
\pgfpathcurveto{\pgfqpoint{2.865252in}{2.167754in}}{\pgfqpoint{2.861980in}{2.175654in}}{\pgfqpoint{2.856156in}{2.181478in}}%
\pgfpathcurveto{\pgfqpoint{2.850332in}{2.187302in}}{\pgfqpoint{2.842432in}{2.190574in}}{\pgfqpoint{2.834196in}{2.190574in}}%
\pgfpathcurveto{\pgfqpoint{2.825959in}{2.190574in}}{\pgfqpoint{2.818059in}{2.187302in}}{\pgfqpoint{2.812235in}{2.181478in}}%
\pgfpathcurveto{\pgfqpoint{2.806411in}{2.175654in}}{\pgfqpoint{2.803139in}{2.167754in}}{\pgfqpoint{2.803139in}{2.159518in}}%
\pgfpathcurveto{\pgfqpoint{2.803139in}{2.151281in}}{\pgfqpoint{2.806411in}{2.143381in}}{\pgfqpoint{2.812235in}{2.137557in}}%
\pgfpathcurveto{\pgfqpoint{2.818059in}{2.131733in}}{\pgfqpoint{2.825959in}{2.128461in}}{\pgfqpoint{2.834196in}{2.128461in}}%
\pgfpathclose%
\pgfusepath{stroke,fill}%
\end{pgfscope}%
\begin{pgfscope}%
\pgfpathrectangle{\pgfqpoint{0.100000in}{0.220728in}}{\pgfqpoint{3.696000in}{3.696000in}}%
\pgfusepath{clip}%
\pgfsetbuttcap%
\pgfsetroundjoin%
\definecolor{currentfill}{rgb}{0.121569,0.466667,0.705882}%
\pgfsetfillcolor{currentfill}%
\pgfsetfillopacity{0.852167}%
\pgfsetlinewidth{1.003750pt}%
\definecolor{currentstroke}{rgb}{0.121569,0.466667,0.705882}%
\pgfsetstrokecolor{currentstroke}%
\pgfsetstrokeopacity{0.852167}%
\pgfsetdash{}{0pt}%
\pgfpathmoveto{\pgfqpoint{2.826405in}{2.114412in}}%
\pgfpathcurveto{\pgfqpoint{2.834641in}{2.114412in}}{\pgfqpoint{2.842541in}{2.117685in}}{\pgfqpoint{2.848365in}{2.123509in}}%
\pgfpathcurveto{\pgfqpoint{2.854189in}{2.129333in}}{\pgfqpoint{2.857462in}{2.137233in}}{\pgfqpoint{2.857462in}{2.145469in}}%
\pgfpathcurveto{\pgfqpoint{2.857462in}{2.153705in}}{\pgfqpoint{2.854189in}{2.161605in}}{\pgfqpoint{2.848365in}{2.167429in}}%
\pgfpathcurveto{\pgfqpoint{2.842541in}{2.173253in}}{\pgfqpoint{2.834641in}{2.176525in}}{\pgfqpoint{2.826405in}{2.176525in}}%
\pgfpathcurveto{\pgfqpoint{2.818169in}{2.176525in}}{\pgfqpoint{2.810269in}{2.173253in}}{\pgfqpoint{2.804445in}{2.167429in}}%
\pgfpathcurveto{\pgfqpoint{2.798621in}{2.161605in}}{\pgfqpoint{2.795349in}{2.153705in}}{\pgfqpoint{2.795349in}{2.145469in}}%
\pgfpathcurveto{\pgfqpoint{2.795349in}{2.137233in}}{\pgfqpoint{2.798621in}{2.129333in}}{\pgfqpoint{2.804445in}{2.123509in}}%
\pgfpathcurveto{\pgfqpoint{2.810269in}{2.117685in}}{\pgfqpoint{2.818169in}{2.114412in}}{\pgfqpoint{2.826405in}{2.114412in}}%
\pgfpathclose%
\pgfusepath{stroke,fill}%
\end{pgfscope}%
\begin{pgfscope}%
\pgfpathrectangle{\pgfqpoint{0.100000in}{0.220728in}}{\pgfqpoint{3.696000in}{3.696000in}}%
\pgfusepath{clip}%
\pgfsetbuttcap%
\pgfsetroundjoin%
\definecolor{currentfill}{rgb}{0.121569,0.466667,0.705882}%
\pgfsetfillcolor{currentfill}%
\pgfsetfillopacity{0.855106}%
\pgfsetlinewidth{1.003750pt}%
\definecolor{currentstroke}{rgb}{0.121569,0.466667,0.705882}%
\pgfsetstrokecolor{currentstroke}%
\pgfsetstrokeopacity{0.855106}%
\pgfsetdash{}{0pt}%
\pgfpathmoveto{\pgfqpoint{1.616057in}{1.826684in}}%
\pgfpathcurveto{\pgfqpoint{1.624293in}{1.826684in}}{\pgfqpoint{1.632193in}{1.829956in}}{\pgfqpoint{1.638017in}{1.835780in}}%
\pgfpathcurveto{\pgfqpoint{1.643841in}{1.841604in}}{\pgfqpoint{1.647113in}{1.849504in}}{\pgfqpoint{1.647113in}{1.857741in}}%
\pgfpathcurveto{\pgfqpoint{1.647113in}{1.865977in}}{\pgfqpoint{1.643841in}{1.873877in}}{\pgfqpoint{1.638017in}{1.879701in}}%
\pgfpathcurveto{\pgfqpoint{1.632193in}{1.885525in}}{\pgfqpoint{1.624293in}{1.888797in}}{\pgfqpoint{1.616057in}{1.888797in}}%
\pgfpathcurveto{\pgfqpoint{1.607821in}{1.888797in}}{\pgfqpoint{1.599921in}{1.885525in}}{\pgfqpoint{1.594097in}{1.879701in}}%
\pgfpathcurveto{\pgfqpoint{1.588273in}{1.873877in}}{\pgfqpoint{1.585000in}{1.865977in}}{\pgfqpoint{1.585000in}{1.857741in}}%
\pgfpathcurveto{\pgfqpoint{1.585000in}{1.849504in}}{\pgfqpoint{1.588273in}{1.841604in}}{\pgfqpoint{1.594097in}{1.835780in}}%
\pgfpathcurveto{\pgfqpoint{1.599921in}{1.829956in}}{\pgfqpoint{1.607821in}{1.826684in}}{\pgfqpoint{1.616057in}{1.826684in}}%
\pgfpathclose%
\pgfusepath{stroke,fill}%
\end{pgfscope}%
\begin{pgfscope}%
\pgfpathrectangle{\pgfqpoint{0.100000in}{0.220728in}}{\pgfqpoint{3.696000in}{3.696000in}}%
\pgfusepath{clip}%
\pgfsetbuttcap%
\pgfsetroundjoin%
\definecolor{currentfill}{rgb}{0.121569,0.466667,0.705882}%
\pgfsetfillcolor{currentfill}%
\pgfsetfillopacity{0.855484}%
\pgfsetlinewidth{1.003750pt}%
\definecolor{currentstroke}{rgb}{0.121569,0.466667,0.705882}%
\pgfsetstrokecolor{currentstroke}%
\pgfsetstrokeopacity{0.855484}%
\pgfsetdash{}{0pt}%
\pgfpathmoveto{\pgfqpoint{2.815089in}{2.097585in}}%
\pgfpathcurveto{\pgfqpoint{2.823325in}{2.097585in}}{\pgfqpoint{2.831225in}{2.100857in}}{\pgfqpoint{2.837049in}{2.106681in}}%
\pgfpathcurveto{\pgfqpoint{2.842873in}{2.112505in}}{\pgfqpoint{2.846145in}{2.120405in}}{\pgfqpoint{2.846145in}{2.128642in}}%
\pgfpathcurveto{\pgfqpoint{2.846145in}{2.136878in}}{\pgfqpoint{2.842873in}{2.144778in}}{\pgfqpoint{2.837049in}{2.150602in}}%
\pgfpathcurveto{\pgfqpoint{2.831225in}{2.156426in}}{\pgfqpoint{2.823325in}{2.159698in}}{\pgfqpoint{2.815089in}{2.159698in}}%
\pgfpathcurveto{\pgfqpoint{2.806852in}{2.159698in}}{\pgfqpoint{2.798952in}{2.156426in}}{\pgfqpoint{2.793128in}{2.150602in}}%
\pgfpathcurveto{\pgfqpoint{2.787304in}{2.144778in}}{\pgfqpoint{2.784032in}{2.136878in}}{\pgfqpoint{2.784032in}{2.128642in}}%
\pgfpathcurveto{\pgfqpoint{2.784032in}{2.120405in}}{\pgfqpoint{2.787304in}{2.112505in}}{\pgfqpoint{2.793128in}{2.106681in}}%
\pgfpathcurveto{\pgfqpoint{2.798952in}{2.100857in}}{\pgfqpoint{2.806852in}{2.097585in}}{\pgfqpoint{2.815089in}{2.097585in}}%
\pgfpathclose%
\pgfusepath{stroke,fill}%
\end{pgfscope}%
\begin{pgfscope}%
\pgfpathrectangle{\pgfqpoint{0.100000in}{0.220728in}}{\pgfqpoint{3.696000in}{3.696000in}}%
\pgfusepath{clip}%
\pgfsetbuttcap%
\pgfsetroundjoin%
\definecolor{currentfill}{rgb}{0.121569,0.466667,0.705882}%
\pgfsetfillcolor{currentfill}%
\pgfsetfillopacity{0.859472}%
\pgfsetlinewidth{1.003750pt}%
\definecolor{currentstroke}{rgb}{0.121569,0.466667,0.705882}%
\pgfsetstrokecolor{currentstroke}%
\pgfsetstrokeopacity{0.859472}%
\pgfsetdash{}{0pt}%
\pgfpathmoveto{\pgfqpoint{2.803080in}{2.079305in}}%
\pgfpathcurveto{\pgfqpoint{2.811316in}{2.079305in}}{\pgfqpoint{2.819216in}{2.082578in}}{\pgfqpoint{2.825040in}{2.088402in}}%
\pgfpathcurveto{\pgfqpoint{2.830864in}{2.094226in}}{\pgfqpoint{2.834136in}{2.102126in}}{\pgfqpoint{2.834136in}{2.110362in}}%
\pgfpathcurveto{\pgfqpoint{2.834136in}{2.118598in}}{\pgfqpoint{2.830864in}{2.126498in}}{\pgfqpoint{2.825040in}{2.132322in}}%
\pgfpathcurveto{\pgfqpoint{2.819216in}{2.138146in}}{\pgfqpoint{2.811316in}{2.141418in}}{\pgfqpoint{2.803080in}{2.141418in}}%
\pgfpathcurveto{\pgfqpoint{2.794844in}{2.141418in}}{\pgfqpoint{2.786944in}{2.138146in}}{\pgfqpoint{2.781120in}{2.132322in}}%
\pgfpathcurveto{\pgfqpoint{2.775296in}{2.126498in}}{\pgfqpoint{2.772023in}{2.118598in}}{\pgfqpoint{2.772023in}{2.110362in}}%
\pgfpathcurveto{\pgfqpoint{2.772023in}{2.102126in}}{\pgfqpoint{2.775296in}{2.094226in}}{\pgfqpoint{2.781120in}{2.088402in}}%
\pgfpathcurveto{\pgfqpoint{2.786944in}{2.082578in}}{\pgfqpoint{2.794844in}{2.079305in}}{\pgfqpoint{2.803080in}{2.079305in}}%
\pgfpathclose%
\pgfusepath{stroke,fill}%
\end{pgfscope}%
\begin{pgfscope}%
\pgfpathrectangle{\pgfqpoint{0.100000in}{0.220728in}}{\pgfqpoint{3.696000in}{3.696000in}}%
\pgfusepath{clip}%
\pgfsetbuttcap%
\pgfsetroundjoin%
\definecolor{currentfill}{rgb}{0.121569,0.466667,0.705882}%
\pgfsetfillcolor{currentfill}%
\pgfsetfillopacity{0.861675}%
\pgfsetlinewidth{1.003750pt}%
\definecolor{currentstroke}{rgb}{0.121569,0.466667,0.705882}%
\pgfsetstrokecolor{currentstroke}%
\pgfsetstrokeopacity{0.861675}%
\pgfsetdash{}{0pt}%
\pgfpathmoveto{\pgfqpoint{2.796655in}{2.069031in}}%
\pgfpathcurveto{\pgfqpoint{2.804891in}{2.069031in}}{\pgfqpoint{2.812791in}{2.072303in}}{\pgfqpoint{2.818615in}{2.078127in}}%
\pgfpathcurveto{\pgfqpoint{2.824439in}{2.083951in}}{\pgfqpoint{2.827711in}{2.091851in}}{\pgfqpoint{2.827711in}{2.100088in}}%
\pgfpathcurveto{\pgfqpoint{2.827711in}{2.108324in}}{\pgfqpoint{2.824439in}{2.116224in}}{\pgfqpoint{2.818615in}{2.122048in}}%
\pgfpathcurveto{\pgfqpoint{2.812791in}{2.127872in}}{\pgfqpoint{2.804891in}{2.131144in}}{\pgfqpoint{2.796655in}{2.131144in}}%
\pgfpathcurveto{\pgfqpoint{2.788419in}{2.131144in}}{\pgfqpoint{2.780519in}{2.127872in}}{\pgfqpoint{2.774695in}{2.122048in}}%
\pgfpathcurveto{\pgfqpoint{2.768871in}{2.116224in}}{\pgfqpoint{2.765598in}{2.108324in}}{\pgfqpoint{2.765598in}{2.100088in}}%
\pgfpathcurveto{\pgfqpoint{2.765598in}{2.091851in}}{\pgfqpoint{2.768871in}{2.083951in}}{\pgfqpoint{2.774695in}{2.078127in}}%
\pgfpathcurveto{\pgfqpoint{2.780519in}{2.072303in}}{\pgfqpoint{2.788419in}{2.069031in}}{\pgfqpoint{2.796655in}{2.069031in}}%
\pgfpathclose%
\pgfusepath{stroke,fill}%
\end{pgfscope}%
\begin{pgfscope}%
\pgfpathrectangle{\pgfqpoint{0.100000in}{0.220728in}}{\pgfqpoint{3.696000in}{3.696000in}}%
\pgfusepath{clip}%
\pgfsetbuttcap%
\pgfsetroundjoin%
\definecolor{currentfill}{rgb}{0.121569,0.466667,0.705882}%
\pgfsetfillcolor{currentfill}%
\pgfsetfillopacity{0.862330}%
\pgfsetlinewidth{1.003750pt}%
\definecolor{currentstroke}{rgb}{0.121569,0.466667,0.705882}%
\pgfsetstrokecolor{currentstroke}%
\pgfsetstrokeopacity{0.862330}%
\pgfsetdash{}{0pt}%
\pgfpathmoveto{\pgfqpoint{1.653350in}{1.807400in}}%
\pgfpathcurveto{\pgfqpoint{1.661586in}{1.807400in}}{\pgfqpoint{1.669486in}{1.810672in}}{\pgfqpoint{1.675310in}{1.816496in}}%
\pgfpathcurveto{\pgfqpoint{1.681134in}{1.822320in}}{\pgfqpoint{1.684406in}{1.830220in}}{\pgfqpoint{1.684406in}{1.838457in}}%
\pgfpathcurveto{\pgfqpoint{1.684406in}{1.846693in}}{\pgfqpoint{1.681134in}{1.854593in}}{\pgfqpoint{1.675310in}{1.860417in}}%
\pgfpathcurveto{\pgfqpoint{1.669486in}{1.866241in}}{\pgfqpoint{1.661586in}{1.869513in}}{\pgfqpoint{1.653350in}{1.869513in}}%
\pgfpathcurveto{\pgfqpoint{1.645114in}{1.869513in}}{\pgfqpoint{1.637214in}{1.866241in}}{\pgfqpoint{1.631390in}{1.860417in}}%
\pgfpathcurveto{\pgfqpoint{1.625566in}{1.854593in}}{\pgfqpoint{1.622293in}{1.846693in}}{\pgfqpoint{1.622293in}{1.838457in}}%
\pgfpathcurveto{\pgfqpoint{1.622293in}{1.830220in}}{\pgfqpoint{1.625566in}{1.822320in}}{\pgfqpoint{1.631390in}{1.816496in}}%
\pgfpathcurveto{\pgfqpoint{1.637214in}{1.810672in}}{\pgfqpoint{1.645114in}{1.807400in}}{\pgfqpoint{1.653350in}{1.807400in}}%
\pgfpathclose%
\pgfusepath{stroke,fill}%
\end{pgfscope}%
\begin{pgfscope}%
\pgfpathrectangle{\pgfqpoint{0.100000in}{0.220728in}}{\pgfqpoint{3.696000in}{3.696000in}}%
\pgfusepath{clip}%
\pgfsetbuttcap%
\pgfsetroundjoin%
\definecolor{currentfill}{rgb}{0.121569,0.466667,0.705882}%
\pgfsetfillcolor{currentfill}%
\pgfsetfillopacity{0.862791}%
\pgfsetlinewidth{1.003750pt}%
\definecolor{currentstroke}{rgb}{0.121569,0.466667,0.705882}%
\pgfsetstrokecolor{currentstroke}%
\pgfsetstrokeopacity{0.862791}%
\pgfsetdash{}{0pt}%
\pgfpathmoveto{\pgfqpoint{2.792796in}{2.063361in}}%
\pgfpathcurveto{\pgfqpoint{2.801032in}{2.063361in}}{\pgfqpoint{2.808933in}{2.066633in}}{\pgfqpoint{2.814756in}{2.072457in}}%
\pgfpathcurveto{\pgfqpoint{2.820580in}{2.078281in}}{\pgfqpoint{2.823853in}{2.086181in}}{\pgfqpoint{2.823853in}{2.094417in}}%
\pgfpathcurveto{\pgfqpoint{2.823853in}{2.102654in}}{\pgfqpoint{2.820580in}{2.110554in}}{\pgfqpoint{2.814756in}{2.116378in}}%
\pgfpathcurveto{\pgfqpoint{2.808933in}{2.122202in}}{\pgfqpoint{2.801032in}{2.125474in}}{\pgfqpoint{2.792796in}{2.125474in}}%
\pgfpathcurveto{\pgfqpoint{2.784560in}{2.125474in}}{\pgfqpoint{2.776660in}{2.122202in}}{\pgfqpoint{2.770836in}{2.116378in}}%
\pgfpathcurveto{\pgfqpoint{2.765012in}{2.110554in}}{\pgfqpoint{2.761740in}{2.102654in}}{\pgfqpoint{2.761740in}{2.094417in}}%
\pgfpathcurveto{\pgfqpoint{2.761740in}{2.086181in}}{\pgfqpoint{2.765012in}{2.078281in}}{\pgfqpoint{2.770836in}{2.072457in}}%
\pgfpathcurveto{\pgfqpoint{2.776660in}{2.066633in}}{\pgfqpoint{2.784560in}{2.063361in}}{\pgfqpoint{2.792796in}{2.063361in}}%
\pgfpathclose%
\pgfusepath{stroke,fill}%
\end{pgfscope}%
\begin{pgfscope}%
\pgfpathrectangle{\pgfqpoint{0.100000in}{0.220728in}}{\pgfqpoint{3.696000in}{3.696000in}}%
\pgfusepath{clip}%
\pgfsetbuttcap%
\pgfsetroundjoin%
\definecolor{currentfill}{rgb}{0.121569,0.466667,0.705882}%
\pgfsetfillcolor{currentfill}%
\pgfsetfillopacity{0.863427}%
\pgfsetlinewidth{1.003750pt}%
\definecolor{currentstroke}{rgb}{0.121569,0.466667,0.705882}%
\pgfsetstrokecolor{currentstroke}%
\pgfsetstrokeopacity{0.863427}%
\pgfsetdash{}{0pt}%
\pgfpathmoveto{\pgfqpoint{2.790820in}{2.060144in}}%
\pgfpathcurveto{\pgfqpoint{2.799056in}{2.060144in}}{\pgfqpoint{2.806957in}{2.063416in}}{\pgfqpoint{2.812780in}{2.069240in}}%
\pgfpathcurveto{\pgfqpoint{2.818604in}{2.075064in}}{\pgfqpoint{2.821877in}{2.082964in}}{\pgfqpoint{2.821877in}{2.091201in}}%
\pgfpathcurveto{\pgfqpoint{2.821877in}{2.099437in}}{\pgfqpoint{2.818604in}{2.107337in}}{\pgfqpoint{2.812780in}{2.113161in}}%
\pgfpathcurveto{\pgfqpoint{2.806957in}{2.118985in}}{\pgfqpoint{2.799056in}{2.122257in}}{\pgfqpoint{2.790820in}{2.122257in}}%
\pgfpathcurveto{\pgfqpoint{2.782584in}{2.122257in}}{\pgfqpoint{2.774684in}{2.118985in}}{\pgfqpoint{2.768860in}{2.113161in}}%
\pgfpathcurveto{\pgfqpoint{2.763036in}{2.107337in}}{\pgfqpoint{2.759764in}{2.099437in}}{\pgfqpoint{2.759764in}{2.091201in}}%
\pgfpathcurveto{\pgfqpoint{2.759764in}{2.082964in}}{\pgfqpoint{2.763036in}{2.075064in}}{\pgfqpoint{2.768860in}{2.069240in}}%
\pgfpathcurveto{\pgfqpoint{2.774684in}{2.063416in}}{\pgfqpoint{2.782584in}{2.060144in}}{\pgfqpoint{2.790820in}{2.060144in}}%
\pgfpathclose%
\pgfusepath{stroke,fill}%
\end{pgfscope}%
\begin{pgfscope}%
\pgfpathrectangle{\pgfqpoint{0.100000in}{0.220728in}}{\pgfqpoint{3.696000in}{3.696000in}}%
\pgfusepath{clip}%
\pgfsetbuttcap%
\pgfsetroundjoin%
\definecolor{currentfill}{rgb}{0.121569,0.466667,0.705882}%
\pgfsetfillcolor{currentfill}%
\pgfsetfillopacity{0.863727}%
\pgfsetlinewidth{1.003750pt}%
\definecolor{currentstroke}{rgb}{0.121569,0.466667,0.705882}%
\pgfsetstrokecolor{currentstroke}%
\pgfsetstrokeopacity{0.863727}%
\pgfsetdash{}{0pt}%
\pgfpathmoveto{\pgfqpoint{2.789534in}{2.058431in}}%
\pgfpathcurveto{\pgfqpoint{2.797770in}{2.058431in}}{\pgfqpoint{2.805670in}{2.061704in}}{\pgfqpoint{2.811494in}{2.067528in}}%
\pgfpathcurveto{\pgfqpoint{2.817318in}{2.073352in}}{\pgfqpoint{2.820590in}{2.081252in}}{\pgfqpoint{2.820590in}{2.089488in}}%
\pgfpathcurveto{\pgfqpoint{2.820590in}{2.097724in}}{\pgfqpoint{2.817318in}{2.105624in}}{\pgfqpoint{2.811494in}{2.111448in}}%
\pgfpathcurveto{\pgfqpoint{2.805670in}{2.117272in}}{\pgfqpoint{2.797770in}{2.120544in}}{\pgfqpoint{2.789534in}{2.120544in}}%
\pgfpathcurveto{\pgfqpoint{2.781298in}{2.120544in}}{\pgfqpoint{2.773398in}{2.117272in}}{\pgfqpoint{2.767574in}{2.111448in}}%
\pgfpathcurveto{\pgfqpoint{2.761750in}{2.105624in}}{\pgfqpoint{2.758477in}{2.097724in}}{\pgfqpoint{2.758477in}{2.089488in}}%
\pgfpathcurveto{\pgfqpoint{2.758477in}{2.081252in}}{\pgfqpoint{2.761750in}{2.073352in}}{\pgfqpoint{2.767574in}{2.067528in}}%
\pgfpathcurveto{\pgfqpoint{2.773398in}{2.061704in}}{\pgfqpoint{2.781298in}{2.058431in}}{\pgfqpoint{2.789534in}{2.058431in}}%
\pgfpathclose%
\pgfusepath{stroke,fill}%
\end{pgfscope}%
\begin{pgfscope}%
\pgfpathrectangle{\pgfqpoint{0.100000in}{0.220728in}}{\pgfqpoint{3.696000in}{3.696000in}}%
\pgfusepath{clip}%
\pgfsetbuttcap%
\pgfsetroundjoin%
\definecolor{currentfill}{rgb}{0.121569,0.466667,0.705882}%
\pgfsetfillcolor{currentfill}%
\pgfsetfillopacity{0.865197}%
\pgfsetlinewidth{1.003750pt}%
\definecolor{currentstroke}{rgb}{0.121569,0.466667,0.705882}%
\pgfsetstrokecolor{currentstroke}%
\pgfsetstrokeopacity{0.865197}%
\pgfsetdash{}{0pt}%
\pgfpathmoveto{\pgfqpoint{2.785570in}{2.050439in}}%
\pgfpathcurveto{\pgfqpoint{2.793806in}{2.050439in}}{\pgfqpoint{2.801706in}{2.053711in}}{\pgfqpoint{2.807530in}{2.059535in}}%
\pgfpathcurveto{\pgfqpoint{2.813354in}{2.065359in}}{\pgfqpoint{2.816626in}{2.073259in}}{\pgfqpoint{2.816626in}{2.081495in}}%
\pgfpathcurveto{\pgfqpoint{2.816626in}{2.089732in}}{\pgfqpoint{2.813354in}{2.097632in}}{\pgfqpoint{2.807530in}{2.103456in}}%
\pgfpathcurveto{\pgfqpoint{2.801706in}{2.109280in}}{\pgfqpoint{2.793806in}{2.112552in}}{\pgfqpoint{2.785570in}{2.112552in}}%
\pgfpathcurveto{\pgfqpoint{2.777334in}{2.112552in}}{\pgfqpoint{2.769434in}{2.109280in}}{\pgfqpoint{2.763610in}{2.103456in}}%
\pgfpathcurveto{\pgfqpoint{2.757786in}{2.097632in}}{\pgfqpoint{2.754513in}{2.089732in}}{\pgfqpoint{2.754513in}{2.081495in}}%
\pgfpathcurveto{\pgfqpoint{2.754513in}{2.073259in}}{\pgfqpoint{2.757786in}{2.065359in}}{\pgfqpoint{2.763610in}{2.059535in}}%
\pgfpathcurveto{\pgfqpoint{2.769434in}{2.053711in}}{\pgfqpoint{2.777334in}{2.050439in}}{\pgfqpoint{2.785570in}{2.050439in}}%
\pgfpathclose%
\pgfusepath{stroke,fill}%
\end{pgfscope}%
\begin{pgfscope}%
\pgfpathrectangle{\pgfqpoint{0.100000in}{0.220728in}}{\pgfqpoint{3.696000in}{3.696000in}}%
\pgfusepath{clip}%
\pgfsetbuttcap%
\pgfsetroundjoin%
\definecolor{currentfill}{rgb}{0.121569,0.466667,0.705882}%
\pgfsetfillcolor{currentfill}%
\pgfsetfillopacity{0.867475}%
\pgfsetlinewidth{1.003750pt}%
\definecolor{currentstroke}{rgb}{0.121569,0.466667,0.705882}%
\pgfsetstrokecolor{currentstroke}%
\pgfsetstrokeopacity{0.867475}%
\pgfsetdash{}{0pt}%
\pgfpathmoveto{\pgfqpoint{2.777215in}{2.038303in}}%
\pgfpathcurveto{\pgfqpoint{2.785451in}{2.038303in}}{\pgfqpoint{2.793351in}{2.041575in}}{\pgfqpoint{2.799175in}{2.047399in}}%
\pgfpathcurveto{\pgfqpoint{2.804999in}{2.053223in}}{\pgfqpoint{2.808271in}{2.061123in}}{\pgfqpoint{2.808271in}{2.069360in}}%
\pgfpathcurveto{\pgfqpoint{2.808271in}{2.077596in}}{\pgfqpoint{2.804999in}{2.085496in}}{\pgfqpoint{2.799175in}{2.091320in}}%
\pgfpathcurveto{\pgfqpoint{2.793351in}{2.097144in}}{\pgfqpoint{2.785451in}{2.100416in}}{\pgfqpoint{2.777215in}{2.100416in}}%
\pgfpathcurveto{\pgfqpoint{2.768978in}{2.100416in}}{\pgfqpoint{2.761078in}{2.097144in}}{\pgfqpoint{2.755254in}{2.091320in}}%
\pgfpathcurveto{\pgfqpoint{2.749430in}{2.085496in}}{\pgfqpoint{2.746158in}{2.077596in}}{\pgfqpoint{2.746158in}{2.069360in}}%
\pgfpathcurveto{\pgfqpoint{2.746158in}{2.061123in}}{\pgfqpoint{2.749430in}{2.053223in}}{\pgfqpoint{2.755254in}{2.047399in}}%
\pgfpathcurveto{\pgfqpoint{2.761078in}{2.041575in}}{\pgfqpoint{2.768978in}{2.038303in}}{\pgfqpoint{2.777215in}{2.038303in}}%
\pgfpathclose%
\pgfusepath{stroke,fill}%
\end{pgfscope}%
\begin{pgfscope}%
\pgfpathrectangle{\pgfqpoint{0.100000in}{0.220728in}}{\pgfqpoint{3.696000in}{3.696000in}}%
\pgfusepath{clip}%
\pgfsetbuttcap%
\pgfsetroundjoin%
\definecolor{currentfill}{rgb}{0.121569,0.466667,0.705882}%
\pgfsetfillcolor{currentfill}%
\pgfsetfillopacity{0.868125}%
\pgfsetlinewidth{1.003750pt}%
\definecolor{currentstroke}{rgb}{0.121569,0.466667,0.705882}%
\pgfsetstrokecolor{currentstroke}%
\pgfsetstrokeopacity{0.868125}%
\pgfsetdash{}{0pt}%
\pgfpathmoveto{\pgfqpoint{1.684981in}{1.783268in}}%
\pgfpathcurveto{\pgfqpoint{1.693217in}{1.783268in}}{\pgfqpoint{1.701117in}{1.786540in}}{\pgfqpoint{1.706941in}{1.792364in}}%
\pgfpathcurveto{\pgfqpoint{1.712765in}{1.798188in}}{\pgfqpoint{1.716037in}{1.806088in}}{\pgfqpoint{1.716037in}{1.814324in}}%
\pgfpathcurveto{\pgfqpoint{1.716037in}{1.822560in}}{\pgfqpoint{1.712765in}{1.830460in}}{\pgfqpoint{1.706941in}{1.836284in}}%
\pgfpathcurveto{\pgfqpoint{1.701117in}{1.842108in}}{\pgfqpoint{1.693217in}{1.845381in}}{\pgfqpoint{1.684981in}{1.845381in}}%
\pgfpathcurveto{\pgfqpoint{1.676744in}{1.845381in}}{\pgfqpoint{1.668844in}{1.842108in}}{\pgfqpoint{1.663020in}{1.836284in}}%
\pgfpathcurveto{\pgfqpoint{1.657196in}{1.830460in}}{\pgfqpoint{1.653924in}{1.822560in}}{\pgfqpoint{1.653924in}{1.814324in}}%
\pgfpathcurveto{\pgfqpoint{1.653924in}{1.806088in}}{\pgfqpoint{1.657196in}{1.798188in}}{\pgfqpoint{1.663020in}{1.792364in}}%
\pgfpathcurveto{\pgfqpoint{1.668844in}{1.786540in}}{\pgfqpoint{1.676744in}{1.783268in}}{\pgfqpoint{1.684981in}{1.783268in}}%
\pgfpathclose%
\pgfusepath{stroke,fill}%
\end{pgfscope}%
\begin{pgfscope}%
\pgfpathrectangle{\pgfqpoint{0.100000in}{0.220728in}}{\pgfqpoint{3.696000in}{3.696000in}}%
\pgfusepath{clip}%
\pgfsetbuttcap%
\pgfsetroundjoin%
\definecolor{currentfill}{rgb}{0.121569,0.466667,0.705882}%
\pgfsetfillcolor{currentfill}%
\pgfsetfillopacity{0.870756}%
\pgfsetlinewidth{1.003750pt}%
\definecolor{currentstroke}{rgb}{0.121569,0.466667,0.705882}%
\pgfsetstrokecolor{currentstroke}%
\pgfsetstrokeopacity{0.870756}%
\pgfsetdash{}{0pt}%
\pgfpathmoveto{\pgfqpoint{2.767389in}{2.020767in}}%
\pgfpathcurveto{\pgfqpoint{2.775626in}{2.020767in}}{\pgfqpoint{2.783526in}{2.024039in}}{\pgfqpoint{2.789349in}{2.029863in}}%
\pgfpathcurveto{\pgfqpoint{2.795173in}{2.035687in}}{\pgfqpoint{2.798446in}{2.043587in}}{\pgfqpoint{2.798446in}{2.051823in}}%
\pgfpathcurveto{\pgfqpoint{2.798446in}{2.060059in}}{\pgfqpoint{2.795173in}{2.067959in}}{\pgfqpoint{2.789349in}{2.073783in}}%
\pgfpathcurveto{\pgfqpoint{2.783526in}{2.079607in}}{\pgfqpoint{2.775626in}{2.082880in}}{\pgfqpoint{2.767389in}{2.082880in}}%
\pgfpathcurveto{\pgfqpoint{2.759153in}{2.082880in}}{\pgfqpoint{2.751253in}{2.079607in}}{\pgfqpoint{2.745429in}{2.073783in}}%
\pgfpathcurveto{\pgfqpoint{2.739605in}{2.067959in}}{\pgfqpoint{2.736333in}{2.060059in}}{\pgfqpoint{2.736333in}{2.051823in}}%
\pgfpathcurveto{\pgfqpoint{2.736333in}{2.043587in}}{\pgfqpoint{2.739605in}{2.035687in}}{\pgfqpoint{2.745429in}{2.029863in}}%
\pgfpathcurveto{\pgfqpoint{2.751253in}{2.024039in}}{\pgfqpoint{2.759153in}{2.020767in}}{\pgfqpoint{2.767389in}{2.020767in}}%
\pgfpathclose%
\pgfusepath{stroke,fill}%
\end{pgfscope}%
\begin{pgfscope}%
\pgfpathrectangle{\pgfqpoint{0.100000in}{0.220728in}}{\pgfqpoint{3.696000in}{3.696000in}}%
\pgfusepath{clip}%
\pgfsetbuttcap%
\pgfsetroundjoin%
\definecolor{currentfill}{rgb}{0.121569,0.466667,0.705882}%
\pgfsetfillcolor{currentfill}%
\pgfsetfillopacity{0.873372}%
\pgfsetlinewidth{1.003750pt}%
\definecolor{currentstroke}{rgb}{0.121569,0.466667,0.705882}%
\pgfsetstrokecolor{currentstroke}%
\pgfsetstrokeopacity{0.873372}%
\pgfsetdash{}{0pt}%
\pgfpathmoveto{\pgfqpoint{1.713514in}{1.768386in}}%
\pgfpathcurveto{\pgfqpoint{1.721750in}{1.768386in}}{\pgfqpoint{1.729650in}{1.771658in}}{\pgfqpoint{1.735474in}{1.777482in}}%
\pgfpathcurveto{\pgfqpoint{1.741298in}{1.783306in}}{\pgfqpoint{1.744570in}{1.791206in}}{\pgfqpoint{1.744570in}{1.799442in}}%
\pgfpathcurveto{\pgfqpoint{1.744570in}{1.807679in}}{\pgfqpoint{1.741298in}{1.815579in}}{\pgfqpoint{1.735474in}{1.821403in}}%
\pgfpathcurveto{\pgfqpoint{1.729650in}{1.827227in}}{\pgfqpoint{1.721750in}{1.830499in}}{\pgfqpoint{1.713514in}{1.830499in}}%
\pgfpathcurveto{\pgfqpoint{1.705278in}{1.830499in}}{\pgfqpoint{1.697378in}{1.827227in}}{\pgfqpoint{1.691554in}{1.821403in}}%
\pgfpathcurveto{\pgfqpoint{1.685730in}{1.815579in}}{\pgfqpoint{1.682457in}{1.807679in}}{\pgfqpoint{1.682457in}{1.799442in}}%
\pgfpathcurveto{\pgfqpoint{1.682457in}{1.791206in}}{\pgfqpoint{1.685730in}{1.783306in}}{\pgfqpoint{1.691554in}{1.777482in}}%
\pgfpathcurveto{\pgfqpoint{1.697378in}{1.771658in}}{\pgfqpoint{1.705278in}{1.768386in}}{\pgfqpoint{1.713514in}{1.768386in}}%
\pgfpathclose%
\pgfusepath{stroke,fill}%
\end{pgfscope}%
\begin{pgfscope}%
\pgfpathrectangle{\pgfqpoint{0.100000in}{0.220728in}}{\pgfqpoint{3.696000in}{3.696000in}}%
\pgfusepath{clip}%
\pgfsetbuttcap%
\pgfsetroundjoin%
\definecolor{currentfill}{rgb}{0.121569,0.466667,0.705882}%
\pgfsetfillcolor{currentfill}%
\pgfsetfillopacity{0.874488}%
\pgfsetlinewidth{1.003750pt}%
\definecolor{currentstroke}{rgb}{0.121569,0.466667,0.705882}%
\pgfsetstrokecolor{currentstroke}%
\pgfsetstrokeopacity{0.874488}%
\pgfsetdash{}{0pt}%
\pgfpathmoveto{\pgfqpoint{2.752576in}{2.001372in}}%
\pgfpathcurveto{\pgfqpoint{2.760812in}{2.001372in}}{\pgfqpoint{2.768713in}{2.004645in}}{\pgfqpoint{2.774536in}{2.010469in}}%
\pgfpathcurveto{\pgfqpoint{2.780360in}{2.016292in}}{\pgfqpoint{2.783633in}{2.024193in}}{\pgfqpoint{2.783633in}{2.032429in}}%
\pgfpathcurveto{\pgfqpoint{2.783633in}{2.040665in}}{\pgfqpoint{2.780360in}{2.048565in}}{\pgfqpoint{2.774536in}{2.054389in}}%
\pgfpathcurveto{\pgfqpoint{2.768713in}{2.060213in}}{\pgfqpoint{2.760812in}{2.063485in}}{\pgfqpoint{2.752576in}{2.063485in}}%
\pgfpathcurveto{\pgfqpoint{2.744340in}{2.063485in}}{\pgfqpoint{2.736440in}{2.060213in}}{\pgfqpoint{2.730616in}{2.054389in}}%
\pgfpathcurveto{\pgfqpoint{2.724792in}{2.048565in}}{\pgfqpoint{2.721520in}{2.040665in}}{\pgfqpoint{2.721520in}{2.032429in}}%
\pgfpathcurveto{\pgfqpoint{2.721520in}{2.024193in}}{\pgfqpoint{2.724792in}{2.016292in}}{\pgfqpoint{2.730616in}{2.010469in}}%
\pgfpathcurveto{\pgfqpoint{2.736440in}{2.004645in}}{\pgfqpoint{2.744340in}{2.001372in}}{\pgfqpoint{2.752576in}{2.001372in}}%
\pgfpathclose%
\pgfusepath{stroke,fill}%
\end{pgfscope}%
\begin{pgfscope}%
\pgfpathrectangle{\pgfqpoint{0.100000in}{0.220728in}}{\pgfqpoint{3.696000in}{3.696000in}}%
\pgfusepath{clip}%
\pgfsetbuttcap%
\pgfsetroundjoin%
\definecolor{currentfill}{rgb}{0.121569,0.466667,0.705882}%
\pgfsetfillcolor{currentfill}%
\pgfsetfillopacity{0.877584}%
\pgfsetlinewidth{1.003750pt}%
\definecolor{currentstroke}{rgb}{0.121569,0.466667,0.705882}%
\pgfsetstrokecolor{currentstroke}%
\pgfsetstrokeopacity{0.877584}%
\pgfsetdash{}{0pt}%
\pgfpathmoveto{\pgfqpoint{1.735548in}{1.757318in}}%
\pgfpathcurveto{\pgfqpoint{1.743785in}{1.757318in}}{\pgfqpoint{1.751685in}{1.760590in}}{\pgfqpoint{1.757509in}{1.766414in}}%
\pgfpathcurveto{\pgfqpoint{1.763333in}{1.772238in}}{\pgfqpoint{1.766605in}{1.780138in}}{\pgfqpoint{1.766605in}{1.788374in}}%
\pgfpathcurveto{\pgfqpoint{1.766605in}{1.796610in}}{\pgfqpoint{1.763333in}{1.804510in}}{\pgfqpoint{1.757509in}{1.810334in}}%
\pgfpathcurveto{\pgfqpoint{1.751685in}{1.816158in}}{\pgfqpoint{1.743785in}{1.819431in}}{\pgfqpoint{1.735548in}{1.819431in}}%
\pgfpathcurveto{\pgfqpoint{1.727312in}{1.819431in}}{\pgfqpoint{1.719412in}{1.816158in}}{\pgfqpoint{1.713588in}{1.810334in}}%
\pgfpathcurveto{\pgfqpoint{1.707764in}{1.804510in}}{\pgfqpoint{1.704492in}{1.796610in}}{\pgfqpoint{1.704492in}{1.788374in}}%
\pgfpathcurveto{\pgfqpoint{1.704492in}{1.780138in}}{\pgfqpoint{1.707764in}{1.772238in}}{\pgfqpoint{1.713588in}{1.766414in}}%
\pgfpathcurveto{\pgfqpoint{1.719412in}{1.760590in}}{\pgfqpoint{1.727312in}{1.757318in}}{\pgfqpoint{1.735548in}{1.757318in}}%
\pgfpathclose%
\pgfusepath{stroke,fill}%
\end{pgfscope}%
\begin{pgfscope}%
\pgfpathrectangle{\pgfqpoint{0.100000in}{0.220728in}}{\pgfqpoint{3.696000in}{3.696000in}}%
\pgfusepath{clip}%
\pgfsetbuttcap%
\pgfsetroundjoin%
\definecolor{currentfill}{rgb}{0.121569,0.466667,0.705882}%
\pgfsetfillcolor{currentfill}%
\pgfsetfillopacity{0.879607}%
\pgfsetlinewidth{1.003750pt}%
\definecolor{currentstroke}{rgb}{0.121569,0.466667,0.705882}%
\pgfsetstrokecolor{currentstroke}%
\pgfsetstrokeopacity{0.879607}%
\pgfsetdash{}{0pt}%
\pgfpathmoveto{\pgfqpoint{2.738348in}{1.976149in}}%
\pgfpathcurveto{\pgfqpoint{2.746584in}{1.976149in}}{\pgfqpoint{2.754484in}{1.979422in}}{\pgfqpoint{2.760308in}{1.985245in}}%
\pgfpathcurveto{\pgfqpoint{2.766132in}{1.991069in}}{\pgfqpoint{2.769404in}{1.998969in}}{\pgfqpoint{2.769404in}{2.007206in}}%
\pgfpathcurveto{\pgfqpoint{2.769404in}{2.015442in}}{\pgfqpoint{2.766132in}{2.023342in}}{\pgfqpoint{2.760308in}{2.029166in}}%
\pgfpathcurveto{\pgfqpoint{2.754484in}{2.034990in}}{\pgfqpoint{2.746584in}{2.038262in}}{\pgfqpoint{2.738348in}{2.038262in}}%
\pgfpathcurveto{\pgfqpoint{2.730112in}{2.038262in}}{\pgfqpoint{2.722212in}{2.034990in}}{\pgfqpoint{2.716388in}{2.029166in}}%
\pgfpathcurveto{\pgfqpoint{2.710564in}{2.023342in}}{\pgfqpoint{2.707291in}{2.015442in}}{\pgfqpoint{2.707291in}{2.007206in}}%
\pgfpathcurveto{\pgfqpoint{2.707291in}{1.998969in}}{\pgfqpoint{2.710564in}{1.991069in}}{\pgfqpoint{2.716388in}{1.985245in}}%
\pgfpathcurveto{\pgfqpoint{2.722212in}{1.979422in}}{\pgfqpoint{2.730112in}{1.976149in}}{\pgfqpoint{2.738348in}{1.976149in}}%
\pgfpathclose%
\pgfusepath{stroke,fill}%
\end{pgfscope}%
\begin{pgfscope}%
\pgfpathrectangle{\pgfqpoint{0.100000in}{0.220728in}}{\pgfqpoint{3.696000in}{3.696000in}}%
\pgfusepath{clip}%
\pgfsetbuttcap%
\pgfsetroundjoin%
\definecolor{currentfill}{rgb}{0.121569,0.466667,0.705882}%
\pgfsetfillcolor{currentfill}%
\pgfsetfillopacity{0.881199}%
\pgfsetlinewidth{1.003750pt}%
\definecolor{currentstroke}{rgb}{0.121569,0.466667,0.705882}%
\pgfsetstrokecolor{currentstroke}%
\pgfsetstrokeopacity{0.881199}%
\pgfsetdash{}{0pt}%
\pgfpathmoveto{\pgfqpoint{1.754474in}{1.748034in}}%
\pgfpathcurveto{\pgfqpoint{1.762710in}{1.748034in}}{\pgfqpoint{1.770610in}{1.751306in}}{\pgfqpoint{1.776434in}{1.757130in}}%
\pgfpathcurveto{\pgfqpoint{1.782258in}{1.762954in}}{\pgfqpoint{1.785530in}{1.770854in}}{\pgfqpoint{1.785530in}{1.779090in}}%
\pgfpathcurveto{\pgfqpoint{1.785530in}{1.787326in}}{\pgfqpoint{1.782258in}{1.795227in}}{\pgfqpoint{1.776434in}{1.801050in}}%
\pgfpathcurveto{\pgfqpoint{1.770610in}{1.806874in}}{\pgfqpoint{1.762710in}{1.810147in}}{\pgfqpoint{1.754474in}{1.810147in}}%
\pgfpathcurveto{\pgfqpoint{1.746238in}{1.810147in}}{\pgfqpoint{1.738337in}{1.806874in}}{\pgfqpoint{1.732514in}{1.801050in}}%
\pgfpathcurveto{\pgfqpoint{1.726690in}{1.795227in}}{\pgfqpoint{1.723417in}{1.787326in}}{\pgfqpoint{1.723417in}{1.779090in}}%
\pgfpathcurveto{\pgfqpoint{1.723417in}{1.770854in}}{\pgfqpoint{1.726690in}{1.762954in}}{\pgfqpoint{1.732514in}{1.757130in}}%
\pgfpathcurveto{\pgfqpoint{1.738337in}{1.751306in}}{\pgfqpoint{1.746238in}{1.748034in}}{\pgfqpoint{1.754474in}{1.748034in}}%
\pgfpathclose%
\pgfusepath{stroke,fill}%
\end{pgfscope}%
\begin{pgfscope}%
\pgfpathrectangle{\pgfqpoint{0.100000in}{0.220728in}}{\pgfqpoint{3.696000in}{3.696000in}}%
\pgfusepath{clip}%
\pgfsetbuttcap%
\pgfsetroundjoin%
\definecolor{currentfill}{rgb}{0.121569,0.466667,0.705882}%
\pgfsetfillcolor{currentfill}%
\pgfsetfillopacity{0.883777}%
\pgfsetlinewidth{1.003750pt}%
\definecolor{currentstroke}{rgb}{0.121569,0.466667,0.705882}%
\pgfsetstrokecolor{currentstroke}%
\pgfsetstrokeopacity{0.883777}%
\pgfsetdash{}{0pt}%
\pgfpathmoveto{\pgfqpoint{1.768765in}{1.737278in}}%
\pgfpathcurveto{\pgfqpoint{1.777001in}{1.737278in}}{\pgfqpoint{1.784902in}{1.740551in}}{\pgfqpoint{1.790725in}{1.746375in}}%
\pgfpathcurveto{\pgfqpoint{1.796549in}{1.752199in}}{\pgfqpoint{1.799822in}{1.760099in}}{\pgfqpoint{1.799822in}{1.768335in}}%
\pgfpathcurveto{\pgfqpoint{1.799822in}{1.776571in}}{\pgfqpoint{1.796549in}{1.784471in}}{\pgfqpoint{1.790725in}{1.790295in}}%
\pgfpathcurveto{\pgfqpoint{1.784902in}{1.796119in}}{\pgfqpoint{1.777001in}{1.799391in}}{\pgfqpoint{1.768765in}{1.799391in}}%
\pgfpathcurveto{\pgfqpoint{1.760529in}{1.799391in}}{\pgfqpoint{1.752629in}{1.796119in}}{\pgfqpoint{1.746805in}{1.790295in}}%
\pgfpathcurveto{\pgfqpoint{1.740981in}{1.784471in}}{\pgfqpoint{1.737709in}{1.776571in}}{\pgfqpoint{1.737709in}{1.768335in}}%
\pgfpathcurveto{\pgfqpoint{1.737709in}{1.760099in}}{\pgfqpoint{1.740981in}{1.752199in}}{\pgfqpoint{1.746805in}{1.746375in}}%
\pgfpathcurveto{\pgfqpoint{1.752629in}{1.740551in}}{\pgfqpoint{1.760529in}{1.737278in}}{\pgfqpoint{1.768765in}{1.737278in}}%
\pgfpathclose%
\pgfusepath{stroke,fill}%
\end{pgfscope}%
\begin{pgfscope}%
\pgfpathrectangle{\pgfqpoint{0.100000in}{0.220728in}}{\pgfqpoint{3.696000in}{3.696000in}}%
\pgfusepath{clip}%
\pgfsetbuttcap%
\pgfsetroundjoin%
\definecolor{currentfill}{rgb}{0.121569,0.466667,0.705882}%
\pgfsetfillcolor{currentfill}%
\pgfsetfillopacity{0.884690}%
\pgfsetlinewidth{1.003750pt}%
\definecolor{currentstroke}{rgb}{0.121569,0.466667,0.705882}%
\pgfsetstrokecolor{currentstroke}%
\pgfsetstrokeopacity{0.884690}%
\pgfsetdash{}{0pt}%
\pgfpathmoveto{\pgfqpoint{2.719958in}{1.950503in}}%
\pgfpathcurveto{\pgfqpoint{2.728194in}{1.950503in}}{\pgfqpoint{2.736094in}{1.953775in}}{\pgfqpoint{2.741918in}{1.959599in}}%
\pgfpathcurveto{\pgfqpoint{2.747742in}{1.965423in}}{\pgfqpoint{2.751014in}{1.973323in}}{\pgfqpoint{2.751014in}{1.981559in}}%
\pgfpathcurveto{\pgfqpoint{2.751014in}{1.989796in}}{\pgfqpoint{2.747742in}{1.997696in}}{\pgfqpoint{2.741918in}{2.003520in}}%
\pgfpathcurveto{\pgfqpoint{2.736094in}{2.009344in}}{\pgfqpoint{2.728194in}{2.012616in}}{\pgfqpoint{2.719958in}{2.012616in}}%
\pgfpathcurveto{\pgfqpoint{2.711721in}{2.012616in}}{\pgfqpoint{2.703821in}{2.009344in}}{\pgfqpoint{2.697997in}{2.003520in}}%
\pgfpathcurveto{\pgfqpoint{2.692173in}{1.997696in}}{\pgfqpoint{2.688901in}{1.989796in}}{\pgfqpoint{2.688901in}{1.981559in}}%
\pgfpathcurveto{\pgfqpoint{2.688901in}{1.973323in}}{\pgfqpoint{2.692173in}{1.965423in}}{\pgfqpoint{2.697997in}{1.959599in}}%
\pgfpathcurveto{\pgfqpoint{2.703821in}{1.953775in}}{\pgfqpoint{2.711721in}{1.950503in}}{\pgfqpoint{2.719958in}{1.950503in}}%
\pgfpathclose%
\pgfusepath{stroke,fill}%
\end{pgfscope}%
\begin{pgfscope}%
\pgfpathrectangle{\pgfqpoint{0.100000in}{0.220728in}}{\pgfqpoint{3.696000in}{3.696000in}}%
\pgfusepath{clip}%
\pgfsetbuttcap%
\pgfsetroundjoin%
\definecolor{currentfill}{rgb}{0.121569,0.466667,0.705882}%
\pgfsetfillcolor{currentfill}%
\pgfsetfillopacity{0.886261}%
\pgfsetlinewidth{1.003750pt}%
\definecolor{currentstroke}{rgb}{0.121569,0.466667,0.705882}%
\pgfsetstrokecolor{currentstroke}%
\pgfsetstrokeopacity{0.886261}%
\pgfsetdash{}{0pt}%
\pgfpathmoveto{\pgfqpoint{1.781823in}{1.731043in}}%
\pgfpathcurveto{\pgfqpoint{1.790059in}{1.731043in}}{\pgfqpoint{1.797959in}{1.734315in}}{\pgfqpoint{1.803783in}{1.740139in}}%
\pgfpathcurveto{\pgfqpoint{1.809607in}{1.745963in}}{\pgfqpoint{1.812879in}{1.753863in}}{\pgfqpoint{1.812879in}{1.762099in}}%
\pgfpathcurveto{\pgfqpoint{1.812879in}{1.770336in}}{\pgfqpoint{1.809607in}{1.778236in}}{\pgfqpoint{1.803783in}{1.784060in}}%
\pgfpathcurveto{\pgfqpoint{1.797959in}{1.789884in}}{\pgfqpoint{1.790059in}{1.793156in}}{\pgfqpoint{1.781823in}{1.793156in}}%
\pgfpathcurveto{\pgfqpoint{1.773586in}{1.793156in}}{\pgfqpoint{1.765686in}{1.789884in}}{\pgfqpoint{1.759862in}{1.784060in}}%
\pgfpathcurveto{\pgfqpoint{1.754038in}{1.778236in}}{\pgfqpoint{1.750766in}{1.770336in}}{\pgfqpoint{1.750766in}{1.762099in}}%
\pgfpathcurveto{\pgfqpoint{1.750766in}{1.753863in}}{\pgfqpoint{1.754038in}{1.745963in}}{\pgfqpoint{1.759862in}{1.740139in}}%
\pgfpathcurveto{\pgfqpoint{1.765686in}{1.734315in}}{\pgfqpoint{1.773586in}{1.731043in}}{\pgfqpoint{1.781823in}{1.731043in}}%
\pgfpathclose%
\pgfusepath{stroke,fill}%
\end{pgfscope}%
\begin{pgfscope}%
\pgfpathrectangle{\pgfqpoint{0.100000in}{0.220728in}}{\pgfqpoint{3.696000in}{3.696000in}}%
\pgfusepath{clip}%
\pgfsetbuttcap%
\pgfsetroundjoin%
\definecolor{currentfill}{rgb}{0.121569,0.466667,0.705882}%
\pgfsetfillcolor{currentfill}%
\pgfsetfillopacity{0.887851}%
\pgfsetlinewidth{1.003750pt}%
\definecolor{currentstroke}{rgb}{0.121569,0.466667,0.705882}%
\pgfsetstrokecolor{currentstroke}%
\pgfsetstrokeopacity{0.887851}%
\pgfsetdash{}{0pt}%
\pgfpathmoveto{\pgfqpoint{2.710986in}{1.936517in}}%
\pgfpathcurveto{\pgfqpoint{2.719222in}{1.936517in}}{\pgfqpoint{2.727122in}{1.939789in}}{\pgfqpoint{2.732946in}{1.945613in}}%
\pgfpathcurveto{\pgfqpoint{2.738770in}{1.951437in}}{\pgfqpoint{2.742042in}{1.959337in}}{\pgfqpoint{2.742042in}{1.967573in}}%
\pgfpathcurveto{\pgfqpoint{2.742042in}{1.975809in}}{\pgfqpoint{2.738770in}{1.983709in}}{\pgfqpoint{2.732946in}{1.989533in}}%
\pgfpathcurveto{\pgfqpoint{2.727122in}{1.995357in}}{\pgfqpoint{2.719222in}{1.998630in}}{\pgfqpoint{2.710986in}{1.998630in}}%
\pgfpathcurveto{\pgfqpoint{2.702749in}{1.998630in}}{\pgfqpoint{2.694849in}{1.995357in}}{\pgfqpoint{2.689025in}{1.989533in}}%
\pgfpathcurveto{\pgfqpoint{2.683201in}{1.983709in}}{\pgfqpoint{2.679929in}{1.975809in}}{\pgfqpoint{2.679929in}{1.967573in}}%
\pgfpathcurveto{\pgfqpoint{2.679929in}{1.959337in}}{\pgfqpoint{2.683201in}{1.951437in}}{\pgfqpoint{2.689025in}{1.945613in}}%
\pgfpathcurveto{\pgfqpoint{2.694849in}{1.939789in}}{\pgfqpoint{2.702749in}{1.936517in}}{\pgfqpoint{2.710986in}{1.936517in}}%
\pgfpathclose%
\pgfusepath{stroke,fill}%
\end{pgfscope}%
\begin{pgfscope}%
\pgfpathrectangle{\pgfqpoint{0.100000in}{0.220728in}}{\pgfqpoint{3.696000in}{3.696000in}}%
\pgfusepath{clip}%
\pgfsetbuttcap%
\pgfsetroundjoin%
\definecolor{currentfill}{rgb}{0.121569,0.466667,0.705882}%
\pgfsetfillcolor{currentfill}%
\pgfsetfillopacity{0.888322}%
\pgfsetlinewidth{1.003750pt}%
\definecolor{currentstroke}{rgb}{0.121569,0.466667,0.705882}%
\pgfsetstrokecolor{currentstroke}%
\pgfsetstrokeopacity{0.888322}%
\pgfsetdash{}{0pt}%
\pgfpathmoveto{\pgfqpoint{1.793426in}{1.726730in}}%
\pgfpathcurveto{\pgfqpoint{1.801662in}{1.726730in}}{\pgfqpoint{1.809562in}{1.730002in}}{\pgfqpoint{1.815386in}{1.735826in}}%
\pgfpathcurveto{\pgfqpoint{1.821210in}{1.741650in}}{\pgfqpoint{1.824482in}{1.749550in}}{\pgfqpoint{1.824482in}{1.757786in}}%
\pgfpathcurveto{\pgfqpoint{1.824482in}{1.766023in}}{\pgfqpoint{1.821210in}{1.773923in}}{\pgfqpoint{1.815386in}{1.779747in}}%
\pgfpathcurveto{\pgfqpoint{1.809562in}{1.785570in}}{\pgfqpoint{1.801662in}{1.788843in}}{\pgfqpoint{1.793426in}{1.788843in}}%
\pgfpathcurveto{\pgfqpoint{1.785189in}{1.788843in}}{\pgfqpoint{1.777289in}{1.785570in}}{\pgfqpoint{1.771465in}{1.779747in}}%
\pgfpathcurveto{\pgfqpoint{1.765642in}{1.773923in}}{\pgfqpoint{1.762369in}{1.766023in}}{\pgfqpoint{1.762369in}{1.757786in}}%
\pgfpathcurveto{\pgfqpoint{1.762369in}{1.749550in}}{\pgfqpoint{1.765642in}{1.741650in}}{\pgfqpoint{1.771465in}{1.735826in}}%
\pgfpathcurveto{\pgfqpoint{1.777289in}{1.730002in}}{\pgfqpoint{1.785189in}{1.726730in}}{\pgfqpoint{1.793426in}{1.726730in}}%
\pgfpathclose%
\pgfusepath{stroke,fill}%
\end{pgfscope}%
\begin{pgfscope}%
\pgfpathrectangle{\pgfqpoint{0.100000in}{0.220728in}}{\pgfqpoint{3.696000in}{3.696000in}}%
\pgfusepath{clip}%
\pgfsetbuttcap%
\pgfsetroundjoin%
\definecolor{currentfill}{rgb}{0.121569,0.466667,0.705882}%
\pgfsetfillcolor{currentfill}%
\pgfsetfillopacity{0.889424}%
\pgfsetlinewidth{1.003750pt}%
\definecolor{currentstroke}{rgb}{0.121569,0.466667,0.705882}%
\pgfsetstrokecolor{currentstroke}%
\pgfsetstrokeopacity{0.889424}%
\pgfsetdash{}{0pt}%
\pgfpathmoveto{\pgfqpoint{2.706029in}{1.928048in}}%
\pgfpathcurveto{\pgfqpoint{2.714266in}{1.928048in}}{\pgfqpoint{2.722166in}{1.931321in}}{\pgfqpoint{2.727990in}{1.937145in}}%
\pgfpathcurveto{\pgfqpoint{2.733814in}{1.942969in}}{\pgfqpoint{2.737086in}{1.950869in}}{\pgfqpoint{2.737086in}{1.959105in}}%
\pgfpathcurveto{\pgfqpoint{2.737086in}{1.967341in}}{\pgfqpoint{2.733814in}{1.975241in}}{\pgfqpoint{2.727990in}{1.981065in}}%
\pgfpathcurveto{\pgfqpoint{2.722166in}{1.986889in}}{\pgfqpoint{2.714266in}{1.990161in}}{\pgfqpoint{2.706029in}{1.990161in}}%
\pgfpathcurveto{\pgfqpoint{2.697793in}{1.990161in}}{\pgfqpoint{2.689893in}{1.986889in}}{\pgfqpoint{2.684069in}{1.981065in}}%
\pgfpathcurveto{\pgfqpoint{2.678245in}{1.975241in}}{\pgfqpoint{2.674973in}{1.967341in}}{\pgfqpoint{2.674973in}{1.959105in}}%
\pgfpathcurveto{\pgfqpoint{2.674973in}{1.950869in}}{\pgfqpoint{2.678245in}{1.942969in}}{\pgfqpoint{2.684069in}{1.937145in}}%
\pgfpathcurveto{\pgfqpoint{2.689893in}{1.931321in}}{\pgfqpoint{2.697793in}{1.928048in}}{\pgfqpoint{2.706029in}{1.928048in}}%
\pgfpathclose%
\pgfusepath{stroke,fill}%
\end{pgfscope}%
\begin{pgfscope}%
\pgfpathrectangle{\pgfqpoint{0.100000in}{0.220728in}}{\pgfqpoint{3.696000in}{3.696000in}}%
\pgfusepath{clip}%
\pgfsetbuttcap%
\pgfsetroundjoin%
\definecolor{currentfill}{rgb}{0.121569,0.466667,0.705882}%
\pgfsetfillcolor{currentfill}%
\pgfsetfillopacity{0.889976}%
\pgfsetlinewidth{1.003750pt}%
\definecolor{currentstroke}{rgb}{0.121569,0.466667,0.705882}%
\pgfsetstrokecolor{currentstroke}%
\pgfsetstrokeopacity{0.889976}%
\pgfsetdash{}{0pt}%
\pgfpathmoveto{\pgfqpoint{1.802344in}{1.722287in}}%
\pgfpathcurveto{\pgfqpoint{1.810581in}{1.722287in}}{\pgfqpoint{1.818481in}{1.725560in}}{\pgfqpoint{1.824305in}{1.731384in}}%
\pgfpathcurveto{\pgfqpoint{1.830129in}{1.737208in}}{\pgfqpoint{1.833401in}{1.745108in}}{\pgfqpoint{1.833401in}{1.753344in}}%
\pgfpathcurveto{\pgfqpoint{1.833401in}{1.761580in}}{\pgfqpoint{1.830129in}{1.769480in}}{\pgfqpoint{1.824305in}{1.775304in}}%
\pgfpathcurveto{\pgfqpoint{1.818481in}{1.781128in}}{\pgfqpoint{1.810581in}{1.784400in}}{\pgfqpoint{1.802344in}{1.784400in}}%
\pgfpathcurveto{\pgfqpoint{1.794108in}{1.784400in}}{\pgfqpoint{1.786208in}{1.781128in}}{\pgfqpoint{1.780384in}{1.775304in}}%
\pgfpathcurveto{\pgfqpoint{1.774560in}{1.769480in}}{\pgfqpoint{1.771288in}{1.761580in}}{\pgfqpoint{1.771288in}{1.753344in}}%
\pgfpathcurveto{\pgfqpoint{1.771288in}{1.745108in}}{\pgfqpoint{1.774560in}{1.737208in}}{\pgfqpoint{1.780384in}{1.731384in}}%
\pgfpathcurveto{\pgfqpoint{1.786208in}{1.725560in}}{\pgfqpoint{1.794108in}{1.722287in}}{\pgfqpoint{1.802344in}{1.722287in}}%
\pgfpathclose%
\pgfusepath{stroke,fill}%
\end{pgfscope}%
\begin{pgfscope}%
\pgfpathrectangle{\pgfqpoint{0.100000in}{0.220728in}}{\pgfqpoint{3.696000in}{3.696000in}}%
\pgfusepath{clip}%
\pgfsetbuttcap%
\pgfsetroundjoin%
\definecolor{currentfill}{rgb}{0.121569,0.466667,0.705882}%
\pgfsetfillcolor{currentfill}%
\pgfsetfillopacity{0.890271}%
\pgfsetlinewidth{1.003750pt}%
\definecolor{currentstroke}{rgb}{0.121569,0.466667,0.705882}%
\pgfsetstrokecolor{currentstroke}%
\pgfsetstrokeopacity{0.890271}%
\pgfsetdash{}{0pt}%
\pgfpathmoveto{\pgfqpoint{2.702841in}{1.923926in}}%
\pgfpathcurveto{\pgfqpoint{2.711077in}{1.923926in}}{\pgfqpoint{2.718978in}{1.927198in}}{\pgfqpoint{2.724801in}{1.933022in}}%
\pgfpathcurveto{\pgfqpoint{2.730625in}{1.938846in}}{\pgfqpoint{2.733898in}{1.946746in}}{\pgfqpoint{2.733898in}{1.954982in}}%
\pgfpathcurveto{\pgfqpoint{2.733898in}{1.963219in}}{\pgfqpoint{2.730625in}{1.971119in}}{\pgfqpoint{2.724801in}{1.976943in}}%
\pgfpathcurveto{\pgfqpoint{2.718978in}{1.982767in}}{\pgfqpoint{2.711077in}{1.986039in}}{\pgfqpoint{2.702841in}{1.986039in}}%
\pgfpathcurveto{\pgfqpoint{2.694605in}{1.986039in}}{\pgfqpoint{2.686705in}{1.982767in}}{\pgfqpoint{2.680881in}{1.976943in}}%
\pgfpathcurveto{\pgfqpoint{2.675057in}{1.971119in}}{\pgfqpoint{2.671785in}{1.963219in}}{\pgfqpoint{2.671785in}{1.954982in}}%
\pgfpathcurveto{\pgfqpoint{2.671785in}{1.946746in}}{\pgfqpoint{2.675057in}{1.938846in}}{\pgfqpoint{2.680881in}{1.933022in}}%
\pgfpathcurveto{\pgfqpoint{2.686705in}{1.927198in}}{\pgfqpoint{2.694605in}{1.923926in}}{\pgfqpoint{2.702841in}{1.923926in}}%
\pgfpathclose%
\pgfusepath{stroke,fill}%
\end{pgfscope}%
\begin{pgfscope}%
\pgfpathrectangle{\pgfqpoint{0.100000in}{0.220728in}}{\pgfqpoint{3.696000in}{3.696000in}}%
\pgfusepath{clip}%
\pgfsetbuttcap%
\pgfsetroundjoin%
\definecolor{currentfill}{rgb}{0.121569,0.466667,0.705882}%
\pgfsetfillcolor{currentfill}%
\pgfsetfillopacity{0.890748}%
\pgfsetlinewidth{1.003750pt}%
\definecolor{currentstroke}{rgb}{0.121569,0.466667,0.705882}%
\pgfsetstrokecolor{currentstroke}%
\pgfsetstrokeopacity{0.890748}%
\pgfsetdash{}{0pt}%
\pgfpathmoveto{\pgfqpoint{2.701303in}{1.921410in}}%
\pgfpathcurveto{\pgfqpoint{2.709539in}{1.921410in}}{\pgfqpoint{2.717439in}{1.924683in}}{\pgfqpoint{2.723263in}{1.930507in}}%
\pgfpathcurveto{\pgfqpoint{2.729087in}{1.936330in}}{\pgfqpoint{2.732359in}{1.944231in}}{\pgfqpoint{2.732359in}{1.952467in}}%
\pgfpathcurveto{\pgfqpoint{2.732359in}{1.960703in}}{\pgfqpoint{2.729087in}{1.968603in}}{\pgfqpoint{2.723263in}{1.974427in}}%
\pgfpathcurveto{\pgfqpoint{2.717439in}{1.980251in}}{\pgfqpoint{2.709539in}{1.983523in}}{\pgfqpoint{2.701303in}{1.983523in}}%
\pgfpathcurveto{\pgfqpoint{2.693066in}{1.983523in}}{\pgfqpoint{2.685166in}{1.980251in}}{\pgfqpoint{2.679342in}{1.974427in}}%
\pgfpathcurveto{\pgfqpoint{2.673518in}{1.968603in}}{\pgfqpoint{2.670246in}{1.960703in}}{\pgfqpoint{2.670246in}{1.952467in}}%
\pgfpathcurveto{\pgfqpoint{2.670246in}{1.944231in}}{\pgfqpoint{2.673518in}{1.936330in}}{\pgfqpoint{2.679342in}{1.930507in}}%
\pgfpathcurveto{\pgfqpoint{2.685166in}{1.924683in}}{\pgfqpoint{2.693066in}{1.921410in}}{\pgfqpoint{2.701303in}{1.921410in}}%
\pgfpathclose%
\pgfusepath{stroke,fill}%
\end{pgfscope}%
\begin{pgfscope}%
\pgfpathrectangle{\pgfqpoint{0.100000in}{0.220728in}}{\pgfqpoint{3.696000in}{3.696000in}}%
\pgfusepath{clip}%
\pgfsetbuttcap%
\pgfsetroundjoin%
\definecolor{currentfill}{rgb}{0.121569,0.466667,0.705882}%
\pgfsetfillcolor{currentfill}%
\pgfsetfillopacity{0.891023}%
\pgfsetlinewidth{1.003750pt}%
\definecolor{currentstroke}{rgb}{0.121569,0.466667,0.705882}%
\pgfsetstrokecolor{currentstroke}%
\pgfsetstrokeopacity{0.891023}%
\pgfsetdash{}{0pt}%
\pgfpathmoveto{\pgfqpoint{2.700407in}{1.920154in}}%
\pgfpathcurveto{\pgfqpoint{2.708643in}{1.920154in}}{\pgfqpoint{2.716544in}{1.923427in}}{\pgfqpoint{2.722367in}{1.929251in}}%
\pgfpathcurveto{\pgfqpoint{2.728191in}{1.935074in}}{\pgfqpoint{2.731464in}{1.942974in}}{\pgfqpoint{2.731464in}{1.951211in}}%
\pgfpathcurveto{\pgfqpoint{2.731464in}{1.959447in}}{\pgfqpoint{2.728191in}{1.967347in}}{\pgfqpoint{2.722367in}{1.973171in}}%
\pgfpathcurveto{\pgfqpoint{2.716544in}{1.978995in}}{\pgfqpoint{2.708643in}{1.982267in}}{\pgfqpoint{2.700407in}{1.982267in}}%
\pgfpathcurveto{\pgfqpoint{2.692171in}{1.982267in}}{\pgfqpoint{2.684271in}{1.978995in}}{\pgfqpoint{2.678447in}{1.973171in}}%
\pgfpathcurveto{\pgfqpoint{2.672623in}{1.967347in}}{\pgfqpoint{2.669351in}{1.959447in}}{\pgfqpoint{2.669351in}{1.951211in}}%
\pgfpathcurveto{\pgfqpoint{2.669351in}{1.942974in}}{\pgfqpoint{2.672623in}{1.935074in}}{\pgfqpoint{2.678447in}{1.929251in}}%
\pgfpathcurveto{\pgfqpoint{2.684271in}{1.923427in}}{\pgfqpoint{2.692171in}{1.920154in}}{\pgfqpoint{2.700407in}{1.920154in}}%
\pgfpathclose%
\pgfusepath{stroke,fill}%
\end{pgfscope}%
\begin{pgfscope}%
\pgfpathrectangle{\pgfqpoint{0.100000in}{0.220728in}}{\pgfqpoint{3.696000in}{3.696000in}}%
\pgfusepath{clip}%
\pgfsetbuttcap%
\pgfsetroundjoin%
\definecolor{currentfill}{rgb}{0.121569,0.466667,0.705882}%
\pgfsetfillcolor{currentfill}%
\pgfsetfillopacity{0.891172}%
\pgfsetlinewidth{1.003750pt}%
\definecolor{currentstroke}{rgb}{0.121569,0.466667,0.705882}%
\pgfsetstrokecolor{currentstroke}%
\pgfsetstrokeopacity{0.891172}%
\pgfsetdash{}{0pt}%
\pgfpathmoveto{\pgfqpoint{2.700002in}{1.919345in}}%
\pgfpathcurveto{\pgfqpoint{2.708238in}{1.919345in}}{\pgfqpoint{2.716138in}{1.922617in}}{\pgfqpoint{2.721962in}{1.928441in}}%
\pgfpathcurveto{\pgfqpoint{2.727786in}{1.934265in}}{\pgfqpoint{2.731058in}{1.942165in}}{\pgfqpoint{2.731058in}{1.950402in}}%
\pgfpathcurveto{\pgfqpoint{2.731058in}{1.958638in}}{\pgfqpoint{2.727786in}{1.966538in}}{\pgfqpoint{2.721962in}{1.972362in}}%
\pgfpathcurveto{\pgfqpoint{2.716138in}{1.978186in}}{\pgfqpoint{2.708238in}{1.981458in}}{\pgfqpoint{2.700002in}{1.981458in}}%
\pgfpathcurveto{\pgfqpoint{2.691766in}{1.981458in}}{\pgfqpoint{2.683866in}{1.978186in}}{\pgfqpoint{2.678042in}{1.972362in}}%
\pgfpathcurveto{\pgfqpoint{2.672218in}{1.966538in}}{\pgfqpoint{2.668945in}{1.958638in}}{\pgfqpoint{2.668945in}{1.950402in}}%
\pgfpathcurveto{\pgfqpoint{2.668945in}{1.942165in}}{\pgfqpoint{2.672218in}{1.934265in}}{\pgfqpoint{2.678042in}{1.928441in}}%
\pgfpathcurveto{\pgfqpoint{2.683866in}{1.922617in}}{\pgfqpoint{2.691766in}{1.919345in}}{\pgfqpoint{2.700002in}{1.919345in}}%
\pgfpathclose%
\pgfusepath{stroke,fill}%
\end{pgfscope}%
\begin{pgfscope}%
\pgfpathrectangle{\pgfqpoint{0.100000in}{0.220728in}}{\pgfqpoint{3.696000in}{3.696000in}}%
\pgfusepath{clip}%
\pgfsetbuttcap%
\pgfsetroundjoin%
\definecolor{currentfill}{rgb}{0.121569,0.466667,0.705882}%
\pgfsetfillcolor{currentfill}%
\pgfsetfillopacity{0.892303}%
\pgfsetlinewidth{1.003750pt}%
\definecolor{currentstroke}{rgb}{0.121569,0.466667,0.705882}%
\pgfsetstrokecolor{currentstroke}%
\pgfsetstrokeopacity{0.892303}%
\pgfsetdash{}{0pt}%
\pgfpathmoveto{\pgfqpoint{2.695755in}{1.914038in}}%
\pgfpathcurveto{\pgfqpoint{2.703991in}{1.914038in}}{\pgfqpoint{2.711891in}{1.917310in}}{\pgfqpoint{2.717715in}{1.923134in}}%
\pgfpathcurveto{\pgfqpoint{2.723539in}{1.928958in}}{\pgfqpoint{2.726811in}{1.936858in}}{\pgfqpoint{2.726811in}{1.945094in}}%
\pgfpathcurveto{\pgfqpoint{2.726811in}{1.953331in}}{\pgfqpoint{2.723539in}{1.961231in}}{\pgfqpoint{2.717715in}{1.967055in}}%
\pgfpathcurveto{\pgfqpoint{2.711891in}{1.972879in}}{\pgfqpoint{2.703991in}{1.976151in}}{\pgfqpoint{2.695755in}{1.976151in}}%
\pgfpathcurveto{\pgfqpoint{2.687519in}{1.976151in}}{\pgfqpoint{2.679618in}{1.972879in}}{\pgfqpoint{2.673795in}{1.967055in}}%
\pgfpathcurveto{\pgfqpoint{2.667971in}{1.961231in}}{\pgfqpoint{2.664698in}{1.953331in}}{\pgfqpoint{2.664698in}{1.945094in}}%
\pgfpathcurveto{\pgfqpoint{2.664698in}{1.936858in}}{\pgfqpoint{2.667971in}{1.928958in}}{\pgfqpoint{2.673795in}{1.923134in}}%
\pgfpathcurveto{\pgfqpoint{2.679618in}{1.917310in}}{\pgfqpoint{2.687519in}{1.914038in}}{\pgfqpoint{2.695755in}{1.914038in}}%
\pgfpathclose%
\pgfusepath{stroke,fill}%
\end{pgfscope}%
\begin{pgfscope}%
\pgfpathrectangle{\pgfqpoint{0.100000in}{0.220728in}}{\pgfqpoint{3.696000in}{3.696000in}}%
\pgfusepath{clip}%
\pgfsetbuttcap%
\pgfsetroundjoin%
\definecolor{currentfill}{rgb}{0.121569,0.466667,0.705882}%
\pgfsetfillcolor{currentfill}%
\pgfsetfillopacity{0.893174}%
\pgfsetlinewidth{1.003750pt}%
\definecolor{currentstroke}{rgb}{0.121569,0.466667,0.705882}%
\pgfsetstrokecolor{currentstroke}%
\pgfsetstrokeopacity{0.893174}%
\pgfsetdash{}{0pt}%
\pgfpathmoveto{\pgfqpoint{1.819268in}{1.717450in}}%
\pgfpathcurveto{\pgfqpoint{1.827504in}{1.717450in}}{\pgfqpoint{1.835404in}{1.720722in}}{\pgfqpoint{1.841228in}{1.726546in}}%
\pgfpathcurveto{\pgfqpoint{1.847052in}{1.732370in}}{\pgfqpoint{1.850324in}{1.740270in}}{\pgfqpoint{1.850324in}{1.748507in}}%
\pgfpathcurveto{\pgfqpoint{1.850324in}{1.756743in}}{\pgfqpoint{1.847052in}{1.764643in}}{\pgfqpoint{1.841228in}{1.770467in}}%
\pgfpathcurveto{\pgfqpoint{1.835404in}{1.776291in}}{\pgfqpoint{1.827504in}{1.779563in}}{\pgfqpoint{1.819268in}{1.779563in}}%
\pgfpathcurveto{\pgfqpoint{1.811031in}{1.779563in}}{\pgfqpoint{1.803131in}{1.776291in}}{\pgfqpoint{1.797307in}{1.770467in}}%
\pgfpathcurveto{\pgfqpoint{1.791484in}{1.764643in}}{\pgfqpoint{1.788211in}{1.756743in}}{\pgfqpoint{1.788211in}{1.748507in}}%
\pgfpathcurveto{\pgfqpoint{1.788211in}{1.740270in}}{\pgfqpoint{1.791484in}{1.732370in}}{\pgfqpoint{1.797307in}{1.726546in}}%
\pgfpathcurveto{\pgfqpoint{1.803131in}{1.720722in}}{\pgfqpoint{1.811031in}{1.717450in}}{\pgfqpoint{1.819268in}{1.717450in}}%
\pgfpathclose%
\pgfusepath{stroke,fill}%
\end{pgfscope}%
\begin{pgfscope}%
\pgfpathrectangle{\pgfqpoint{0.100000in}{0.220728in}}{\pgfqpoint{3.696000in}{3.696000in}}%
\pgfusepath{clip}%
\pgfsetbuttcap%
\pgfsetroundjoin%
\definecolor{currentfill}{rgb}{0.121569,0.466667,0.705882}%
\pgfsetfillcolor{currentfill}%
\pgfsetfillopacity{0.894208}%
\pgfsetlinewidth{1.003750pt}%
\definecolor{currentstroke}{rgb}{0.121569,0.466667,0.705882}%
\pgfsetstrokecolor{currentstroke}%
\pgfsetstrokeopacity{0.894208}%
\pgfsetdash{}{0pt}%
\pgfpathmoveto{\pgfqpoint{2.690072in}{1.902961in}}%
\pgfpathcurveto{\pgfqpoint{2.698309in}{1.902961in}}{\pgfqpoint{2.706209in}{1.906234in}}{\pgfqpoint{2.712033in}{1.912058in}}%
\pgfpathcurveto{\pgfqpoint{2.717857in}{1.917882in}}{\pgfqpoint{2.721129in}{1.925782in}}{\pgfqpoint{2.721129in}{1.934018in}}%
\pgfpathcurveto{\pgfqpoint{2.721129in}{1.942254in}}{\pgfqpoint{2.717857in}{1.950154in}}{\pgfqpoint{2.712033in}{1.955978in}}%
\pgfpathcurveto{\pgfqpoint{2.706209in}{1.961802in}}{\pgfqpoint{2.698309in}{1.965074in}}{\pgfqpoint{2.690072in}{1.965074in}}%
\pgfpathcurveto{\pgfqpoint{2.681836in}{1.965074in}}{\pgfqpoint{2.673936in}{1.961802in}}{\pgfqpoint{2.668112in}{1.955978in}}%
\pgfpathcurveto{\pgfqpoint{2.662288in}{1.950154in}}{\pgfqpoint{2.659016in}{1.942254in}}{\pgfqpoint{2.659016in}{1.934018in}}%
\pgfpathcurveto{\pgfqpoint{2.659016in}{1.925782in}}{\pgfqpoint{2.662288in}{1.917882in}}{\pgfqpoint{2.668112in}{1.912058in}}%
\pgfpathcurveto{\pgfqpoint{2.673936in}{1.906234in}}{\pgfqpoint{2.681836in}{1.902961in}}{\pgfqpoint{2.690072in}{1.902961in}}%
\pgfpathclose%
\pgfusepath{stroke,fill}%
\end{pgfscope}%
\begin{pgfscope}%
\pgfpathrectangle{\pgfqpoint{0.100000in}{0.220728in}}{\pgfqpoint{3.696000in}{3.696000in}}%
\pgfusepath{clip}%
\pgfsetbuttcap%
\pgfsetroundjoin%
\definecolor{currentfill}{rgb}{0.121569,0.466667,0.705882}%
\pgfsetfillcolor{currentfill}%
\pgfsetfillopacity{0.898038}%
\pgfsetlinewidth{1.003750pt}%
\definecolor{currentstroke}{rgb}{0.121569,0.466667,0.705882}%
\pgfsetstrokecolor{currentstroke}%
\pgfsetstrokeopacity{0.898038}%
\pgfsetdash{}{0pt}%
\pgfpathmoveto{\pgfqpoint{2.678907in}{1.888559in}}%
\pgfpathcurveto{\pgfqpoint{2.687143in}{1.888559in}}{\pgfqpoint{2.695043in}{1.891831in}}{\pgfqpoint{2.700867in}{1.897655in}}%
\pgfpathcurveto{\pgfqpoint{2.706691in}{1.903479in}}{\pgfqpoint{2.709963in}{1.911379in}}{\pgfqpoint{2.709963in}{1.919615in}}%
\pgfpathcurveto{\pgfqpoint{2.709963in}{1.927851in}}{\pgfqpoint{2.706691in}{1.935751in}}{\pgfqpoint{2.700867in}{1.941575in}}%
\pgfpathcurveto{\pgfqpoint{2.695043in}{1.947399in}}{\pgfqpoint{2.687143in}{1.950672in}}{\pgfqpoint{2.678907in}{1.950672in}}%
\pgfpathcurveto{\pgfqpoint{2.670670in}{1.950672in}}{\pgfqpoint{2.662770in}{1.947399in}}{\pgfqpoint{2.656946in}{1.941575in}}%
\pgfpathcurveto{\pgfqpoint{2.651122in}{1.935751in}}{\pgfqpoint{2.647850in}{1.927851in}}{\pgfqpoint{2.647850in}{1.919615in}}%
\pgfpathcurveto{\pgfqpoint{2.647850in}{1.911379in}}{\pgfqpoint{2.651122in}{1.903479in}}{\pgfqpoint{2.656946in}{1.897655in}}%
\pgfpathcurveto{\pgfqpoint{2.662770in}{1.891831in}}{\pgfqpoint{2.670670in}{1.888559in}}{\pgfqpoint{2.678907in}{1.888559in}}%
\pgfpathclose%
\pgfusepath{stroke,fill}%
\end{pgfscope}%
\begin{pgfscope}%
\pgfpathrectangle{\pgfqpoint{0.100000in}{0.220728in}}{\pgfqpoint{3.696000in}{3.696000in}}%
\pgfusepath{clip}%
\pgfsetbuttcap%
\pgfsetroundjoin%
\definecolor{currentfill}{rgb}{0.121569,0.466667,0.705882}%
\pgfsetfillcolor{currentfill}%
\pgfsetfillopacity{0.898538}%
\pgfsetlinewidth{1.003750pt}%
\definecolor{currentstroke}{rgb}{0.121569,0.466667,0.705882}%
\pgfsetstrokecolor{currentstroke}%
\pgfsetstrokeopacity{0.898538}%
\pgfsetdash{}{0pt}%
\pgfpathmoveto{\pgfqpoint{1.845619in}{1.694033in}}%
\pgfpathcurveto{\pgfqpoint{1.853855in}{1.694033in}}{\pgfqpoint{1.861756in}{1.697306in}}{\pgfqpoint{1.867579in}{1.703130in}}%
\pgfpathcurveto{\pgfqpoint{1.873403in}{1.708953in}}{\pgfqpoint{1.876676in}{1.716854in}}{\pgfqpoint{1.876676in}{1.725090in}}%
\pgfpathcurveto{\pgfqpoint{1.876676in}{1.733326in}}{\pgfqpoint{1.873403in}{1.741226in}}{\pgfqpoint{1.867579in}{1.747050in}}%
\pgfpathcurveto{\pgfqpoint{1.861756in}{1.752874in}}{\pgfqpoint{1.853855in}{1.756146in}}{\pgfqpoint{1.845619in}{1.756146in}}%
\pgfpathcurveto{\pgfqpoint{1.837383in}{1.756146in}}{\pgfqpoint{1.829483in}{1.752874in}}{\pgfqpoint{1.823659in}{1.747050in}}%
\pgfpathcurveto{\pgfqpoint{1.817835in}{1.741226in}}{\pgfqpoint{1.814563in}{1.733326in}}{\pgfqpoint{1.814563in}{1.725090in}}%
\pgfpathcurveto{\pgfqpoint{1.814563in}{1.716854in}}{\pgfqpoint{1.817835in}{1.708953in}}{\pgfqpoint{1.823659in}{1.703130in}}%
\pgfpathcurveto{\pgfqpoint{1.829483in}{1.697306in}}{\pgfqpoint{1.837383in}{1.694033in}}{\pgfqpoint{1.845619in}{1.694033in}}%
\pgfpathclose%
\pgfusepath{stroke,fill}%
\end{pgfscope}%
\begin{pgfscope}%
\pgfpathrectangle{\pgfqpoint{0.100000in}{0.220728in}}{\pgfqpoint{3.696000in}{3.696000in}}%
\pgfusepath{clip}%
\pgfsetbuttcap%
\pgfsetroundjoin%
\definecolor{currentfill}{rgb}{0.121569,0.466667,0.705882}%
\pgfsetfillcolor{currentfill}%
\pgfsetfillopacity{0.901505}%
\pgfsetlinewidth{1.003750pt}%
\definecolor{currentstroke}{rgb}{0.121569,0.466667,0.705882}%
\pgfsetstrokecolor{currentstroke}%
\pgfsetstrokeopacity{0.901505}%
\pgfsetdash{}{0pt}%
\pgfpathmoveto{\pgfqpoint{2.667353in}{1.867003in}}%
\pgfpathcurveto{\pgfqpoint{2.675589in}{1.867003in}}{\pgfqpoint{2.683489in}{1.870275in}}{\pgfqpoint{2.689313in}{1.876099in}}%
\pgfpathcurveto{\pgfqpoint{2.695137in}{1.881923in}}{\pgfqpoint{2.698409in}{1.889823in}}{\pgfqpoint{2.698409in}{1.898059in}}%
\pgfpathcurveto{\pgfqpoint{2.698409in}{1.906296in}}{\pgfqpoint{2.695137in}{1.914196in}}{\pgfqpoint{2.689313in}{1.920020in}}%
\pgfpathcurveto{\pgfqpoint{2.683489in}{1.925844in}}{\pgfqpoint{2.675589in}{1.929116in}}{\pgfqpoint{2.667353in}{1.929116in}}%
\pgfpathcurveto{\pgfqpoint{2.659117in}{1.929116in}}{\pgfqpoint{2.651216in}{1.925844in}}{\pgfqpoint{2.645393in}{1.920020in}}%
\pgfpathcurveto{\pgfqpoint{2.639569in}{1.914196in}}{\pgfqpoint{2.636296in}{1.906296in}}{\pgfqpoint{2.636296in}{1.898059in}}%
\pgfpathcurveto{\pgfqpoint{2.636296in}{1.889823in}}{\pgfqpoint{2.639569in}{1.881923in}}{\pgfqpoint{2.645393in}{1.876099in}}%
\pgfpathcurveto{\pgfqpoint{2.651216in}{1.870275in}}{\pgfqpoint{2.659117in}{1.867003in}}{\pgfqpoint{2.667353in}{1.867003in}}%
\pgfpathclose%
\pgfusepath{stroke,fill}%
\end{pgfscope}%
\begin{pgfscope}%
\pgfpathrectangle{\pgfqpoint{0.100000in}{0.220728in}}{\pgfqpoint{3.696000in}{3.696000in}}%
\pgfusepath{clip}%
\pgfsetbuttcap%
\pgfsetroundjoin%
\definecolor{currentfill}{rgb}{0.121569,0.466667,0.705882}%
\pgfsetfillcolor{currentfill}%
\pgfsetfillopacity{0.903370}%
\pgfsetlinewidth{1.003750pt}%
\definecolor{currentstroke}{rgb}{0.121569,0.466667,0.705882}%
\pgfsetstrokecolor{currentstroke}%
\pgfsetstrokeopacity{0.903370}%
\pgfsetdash{}{0pt}%
\pgfpathmoveto{\pgfqpoint{1.870981in}{1.682917in}}%
\pgfpathcurveto{\pgfqpoint{1.879217in}{1.682917in}}{\pgfqpoint{1.887117in}{1.686189in}}{\pgfqpoint{1.892941in}{1.692013in}}%
\pgfpathcurveto{\pgfqpoint{1.898765in}{1.697837in}}{\pgfqpoint{1.902037in}{1.705737in}}{\pgfqpoint{1.902037in}{1.713973in}}%
\pgfpathcurveto{\pgfqpoint{1.902037in}{1.722210in}}{\pgfqpoint{1.898765in}{1.730110in}}{\pgfqpoint{1.892941in}{1.735934in}}%
\pgfpathcurveto{\pgfqpoint{1.887117in}{1.741757in}}{\pgfqpoint{1.879217in}{1.745030in}}{\pgfqpoint{1.870981in}{1.745030in}}%
\pgfpathcurveto{\pgfqpoint{1.862745in}{1.745030in}}{\pgfqpoint{1.854845in}{1.741757in}}{\pgfqpoint{1.849021in}{1.735934in}}%
\pgfpathcurveto{\pgfqpoint{1.843197in}{1.730110in}}{\pgfqpoint{1.839924in}{1.722210in}}{\pgfqpoint{1.839924in}{1.713973in}}%
\pgfpathcurveto{\pgfqpoint{1.839924in}{1.705737in}}{\pgfqpoint{1.843197in}{1.697837in}}{\pgfqpoint{1.849021in}{1.692013in}}%
\pgfpathcurveto{\pgfqpoint{1.854845in}{1.686189in}}{\pgfqpoint{1.862745in}{1.682917in}}{\pgfqpoint{1.870981in}{1.682917in}}%
\pgfpathclose%
\pgfusepath{stroke,fill}%
\end{pgfscope}%
\begin{pgfscope}%
\pgfpathrectangle{\pgfqpoint{0.100000in}{0.220728in}}{\pgfqpoint{3.696000in}{3.696000in}}%
\pgfusepath{clip}%
\pgfsetbuttcap%
\pgfsetroundjoin%
\definecolor{currentfill}{rgb}{0.121569,0.466667,0.705882}%
\pgfsetfillcolor{currentfill}%
\pgfsetfillopacity{0.905897}%
\pgfsetlinewidth{1.003750pt}%
\definecolor{currentstroke}{rgb}{0.121569,0.466667,0.705882}%
\pgfsetstrokecolor{currentstroke}%
\pgfsetstrokeopacity{0.905897}%
\pgfsetdash{}{0pt}%
\pgfpathmoveto{\pgfqpoint{2.651120in}{1.843788in}}%
\pgfpathcurveto{\pgfqpoint{2.659356in}{1.843788in}}{\pgfqpoint{2.667256in}{1.847061in}}{\pgfqpoint{2.673080in}{1.852885in}}%
\pgfpathcurveto{\pgfqpoint{2.678904in}{1.858709in}}{\pgfqpoint{2.682176in}{1.866609in}}{\pgfqpoint{2.682176in}{1.874845in}}%
\pgfpathcurveto{\pgfqpoint{2.682176in}{1.883081in}}{\pgfqpoint{2.678904in}{1.890981in}}{\pgfqpoint{2.673080in}{1.896805in}}%
\pgfpathcurveto{\pgfqpoint{2.667256in}{1.902629in}}{\pgfqpoint{2.659356in}{1.905901in}}{\pgfqpoint{2.651120in}{1.905901in}}%
\pgfpathcurveto{\pgfqpoint{2.642884in}{1.905901in}}{\pgfqpoint{2.634984in}{1.902629in}}{\pgfqpoint{2.629160in}{1.896805in}}%
\pgfpathcurveto{\pgfqpoint{2.623336in}{1.890981in}}{\pgfqpoint{2.620063in}{1.883081in}}{\pgfqpoint{2.620063in}{1.874845in}}%
\pgfpathcurveto{\pgfqpoint{2.620063in}{1.866609in}}{\pgfqpoint{2.623336in}{1.858709in}}{\pgfqpoint{2.629160in}{1.852885in}}%
\pgfpathcurveto{\pgfqpoint{2.634984in}{1.847061in}}{\pgfqpoint{2.642884in}{1.843788in}}{\pgfqpoint{2.651120in}{1.843788in}}%
\pgfpathclose%
\pgfusepath{stroke,fill}%
\end{pgfscope}%
\begin{pgfscope}%
\pgfpathrectangle{\pgfqpoint{0.100000in}{0.220728in}}{\pgfqpoint{3.696000in}{3.696000in}}%
\pgfusepath{clip}%
\pgfsetbuttcap%
\pgfsetroundjoin%
\definecolor{currentfill}{rgb}{0.121569,0.466667,0.705882}%
\pgfsetfillcolor{currentfill}%
\pgfsetfillopacity{0.906457}%
\pgfsetlinewidth{1.003750pt}%
\definecolor{currentstroke}{rgb}{0.121569,0.466667,0.705882}%
\pgfsetstrokecolor{currentstroke}%
\pgfsetstrokeopacity{0.906457}%
\pgfsetdash{}{0pt}%
\pgfpathmoveto{\pgfqpoint{1.888988in}{1.670333in}}%
\pgfpathcurveto{\pgfqpoint{1.897224in}{1.670333in}}{\pgfqpoint{1.905124in}{1.673605in}}{\pgfqpoint{1.910948in}{1.679429in}}%
\pgfpathcurveto{\pgfqpoint{1.916772in}{1.685253in}}{\pgfqpoint{1.920044in}{1.693153in}}{\pgfqpoint{1.920044in}{1.701389in}}%
\pgfpathcurveto{\pgfqpoint{1.920044in}{1.709626in}}{\pgfqpoint{1.916772in}{1.717526in}}{\pgfqpoint{1.910948in}{1.723350in}}%
\pgfpathcurveto{\pgfqpoint{1.905124in}{1.729173in}}{\pgfqpoint{1.897224in}{1.732446in}}{\pgfqpoint{1.888988in}{1.732446in}}%
\pgfpathcurveto{\pgfqpoint{1.880752in}{1.732446in}}{\pgfqpoint{1.872852in}{1.729173in}}{\pgfqpoint{1.867028in}{1.723350in}}%
\pgfpathcurveto{\pgfqpoint{1.861204in}{1.717526in}}{\pgfqpoint{1.857931in}{1.709626in}}{\pgfqpoint{1.857931in}{1.701389in}}%
\pgfpathcurveto{\pgfqpoint{1.857931in}{1.693153in}}{\pgfqpoint{1.861204in}{1.685253in}}{\pgfqpoint{1.867028in}{1.679429in}}%
\pgfpathcurveto{\pgfqpoint{1.872852in}{1.673605in}}{\pgfqpoint{1.880752in}{1.670333in}}{\pgfqpoint{1.888988in}{1.670333in}}%
\pgfpathclose%
\pgfusepath{stroke,fill}%
\end{pgfscope}%
\begin{pgfscope}%
\pgfpathrectangle{\pgfqpoint{0.100000in}{0.220728in}}{\pgfqpoint{3.696000in}{3.696000in}}%
\pgfusepath{clip}%
\pgfsetbuttcap%
\pgfsetroundjoin%
\definecolor{currentfill}{rgb}{0.121569,0.466667,0.705882}%
\pgfsetfillcolor{currentfill}%
\pgfsetfillopacity{0.909294}%
\pgfsetlinewidth{1.003750pt}%
\definecolor{currentstroke}{rgb}{0.121569,0.466667,0.705882}%
\pgfsetstrokecolor{currentstroke}%
\pgfsetstrokeopacity{0.909294}%
\pgfsetdash{}{0pt}%
\pgfpathmoveto{\pgfqpoint{1.902352in}{1.662177in}}%
\pgfpathcurveto{\pgfqpoint{1.910588in}{1.662177in}}{\pgfqpoint{1.918488in}{1.665450in}}{\pgfqpoint{1.924312in}{1.671274in}}%
\pgfpathcurveto{\pgfqpoint{1.930136in}{1.677098in}}{\pgfqpoint{1.933408in}{1.684998in}}{\pgfqpoint{1.933408in}{1.693234in}}%
\pgfpathcurveto{\pgfqpoint{1.933408in}{1.701470in}}{\pgfqpoint{1.930136in}{1.709370in}}{\pgfqpoint{1.924312in}{1.715194in}}%
\pgfpathcurveto{\pgfqpoint{1.918488in}{1.721018in}}{\pgfqpoint{1.910588in}{1.724290in}}{\pgfqpoint{1.902352in}{1.724290in}}%
\pgfpathcurveto{\pgfqpoint{1.894115in}{1.724290in}}{\pgfqpoint{1.886215in}{1.721018in}}{\pgfqpoint{1.880391in}{1.715194in}}%
\pgfpathcurveto{\pgfqpoint{1.874567in}{1.709370in}}{\pgfqpoint{1.871295in}{1.701470in}}{\pgfqpoint{1.871295in}{1.693234in}}%
\pgfpathcurveto{\pgfqpoint{1.871295in}{1.684998in}}{\pgfqpoint{1.874567in}{1.677098in}}{\pgfqpoint{1.880391in}{1.671274in}}%
\pgfpathcurveto{\pgfqpoint{1.886215in}{1.665450in}}{\pgfqpoint{1.894115in}{1.662177in}}{\pgfqpoint{1.902352in}{1.662177in}}%
\pgfpathclose%
\pgfusepath{stroke,fill}%
\end{pgfscope}%
\begin{pgfscope}%
\pgfpathrectangle{\pgfqpoint{0.100000in}{0.220728in}}{\pgfqpoint{3.696000in}{3.696000in}}%
\pgfusepath{clip}%
\pgfsetbuttcap%
\pgfsetroundjoin%
\definecolor{currentfill}{rgb}{0.121569,0.466667,0.705882}%
\pgfsetfillcolor{currentfill}%
\pgfsetfillopacity{0.911522}%
\pgfsetlinewidth{1.003750pt}%
\definecolor{currentstroke}{rgb}{0.121569,0.466667,0.705882}%
\pgfsetstrokecolor{currentstroke}%
\pgfsetstrokeopacity{0.911522}%
\pgfsetdash{}{0pt}%
\pgfpathmoveto{\pgfqpoint{1.912203in}{1.653512in}}%
\pgfpathcurveto{\pgfqpoint{1.920439in}{1.653512in}}{\pgfqpoint{1.928339in}{1.656784in}}{\pgfqpoint{1.934163in}{1.662608in}}%
\pgfpathcurveto{\pgfqpoint{1.939987in}{1.668432in}}{\pgfqpoint{1.943259in}{1.676332in}}{\pgfqpoint{1.943259in}{1.684568in}}%
\pgfpathcurveto{\pgfqpoint{1.943259in}{1.692805in}}{\pgfqpoint{1.939987in}{1.700705in}}{\pgfqpoint{1.934163in}{1.706529in}}%
\pgfpathcurveto{\pgfqpoint{1.928339in}{1.712353in}}{\pgfqpoint{1.920439in}{1.715625in}}{\pgfqpoint{1.912203in}{1.715625in}}%
\pgfpathcurveto{\pgfqpoint{1.903967in}{1.715625in}}{\pgfqpoint{1.896066in}{1.712353in}}{\pgfqpoint{1.890243in}{1.706529in}}%
\pgfpathcurveto{\pgfqpoint{1.884419in}{1.700705in}}{\pgfqpoint{1.881146in}{1.692805in}}{\pgfqpoint{1.881146in}{1.684568in}}%
\pgfpathcurveto{\pgfqpoint{1.881146in}{1.676332in}}{\pgfqpoint{1.884419in}{1.668432in}}{\pgfqpoint{1.890243in}{1.662608in}}%
\pgfpathcurveto{\pgfqpoint{1.896066in}{1.656784in}}{\pgfqpoint{1.903967in}{1.653512in}}{\pgfqpoint{1.912203in}{1.653512in}}%
\pgfpathclose%
\pgfusepath{stroke,fill}%
\end{pgfscope}%
\begin{pgfscope}%
\pgfpathrectangle{\pgfqpoint{0.100000in}{0.220728in}}{\pgfqpoint{3.696000in}{3.696000in}}%
\pgfusepath{clip}%
\pgfsetbuttcap%
\pgfsetroundjoin%
\definecolor{currentfill}{rgb}{0.121569,0.466667,0.705882}%
\pgfsetfillcolor{currentfill}%
\pgfsetfillopacity{0.911530}%
\pgfsetlinewidth{1.003750pt}%
\definecolor{currentstroke}{rgb}{0.121569,0.466667,0.705882}%
\pgfsetstrokecolor{currentstroke}%
\pgfsetstrokeopacity{0.911530}%
\pgfsetdash{}{0pt}%
\pgfpathmoveto{\pgfqpoint{2.636435in}{1.816999in}}%
\pgfpathcurveto{\pgfqpoint{2.644671in}{1.816999in}}{\pgfqpoint{2.652571in}{1.820271in}}{\pgfqpoint{2.658395in}{1.826095in}}%
\pgfpathcurveto{\pgfqpoint{2.664219in}{1.831919in}}{\pgfqpoint{2.667492in}{1.839819in}}{\pgfqpoint{2.667492in}{1.848056in}}%
\pgfpathcurveto{\pgfqpoint{2.667492in}{1.856292in}}{\pgfqpoint{2.664219in}{1.864192in}}{\pgfqpoint{2.658395in}{1.870016in}}%
\pgfpathcurveto{\pgfqpoint{2.652571in}{1.875840in}}{\pgfqpoint{2.644671in}{1.879112in}}{\pgfqpoint{2.636435in}{1.879112in}}%
\pgfpathcurveto{\pgfqpoint{2.628199in}{1.879112in}}{\pgfqpoint{2.620299in}{1.875840in}}{\pgfqpoint{2.614475in}{1.870016in}}%
\pgfpathcurveto{\pgfqpoint{2.608651in}{1.864192in}}{\pgfqpoint{2.605379in}{1.856292in}}{\pgfqpoint{2.605379in}{1.848056in}}%
\pgfpathcurveto{\pgfqpoint{2.605379in}{1.839819in}}{\pgfqpoint{2.608651in}{1.831919in}}{\pgfqpoint{2.614475in}{1.826095in}}%
\pgfpathcurveto{\pgfqpoint{2.620299in}{1.820271in}}{\pgfqpoint{2.628199in}{1.816999in}}{\pgfqpoint{2.636435in}{1.816999in}}%
\pgfpathclose%
\pgfusepath{stroke,fill}%
\end{pgfscope}%
\begin{pgfscope}%
\pgfpathrectangle{\pgfqpoint{0.100000in}{0.220728in}}{\pgfqpoint{3.696000in}{3.696000in}}%
\pgfusepath{clip}%
\pgfsetbuttcap%
\pgfsetroundjoin%
\definecolor{currentfill}{rgb}{0.121569,0.466667,0.705882}%
\pgfsetfillcolor{currentfill}%
\pgfsetfillopacity{0.912671}%
\pgfsetlinewidth{1.003750pt}%
\definecolor{currentstroke}{rgb}{0.121569,0.466667,0.705882}%
\pgfsetstrokecolor{currentstroke}%
\pgfsetstrokeopacity{0.912671}%
\pgfsetdash{}{0pt}%
\pgfpathmoveto{\pgfqpoint{1.918699in}{1.650904in}}%
\pgfpathcurveto{\pgfqpoint{1.926935in}{1.650904in}}{\pgfqpoint{1.934835in}{1.654176in}}{\pgfqpoint{1.940659in}{1.660000in}}%
\pgfpathcurveto{\pgfqpoint{1.946483in}{1.665824in}}{\pgfqpoint{1.949756in}{1.673724in}}{\pgfqpoint{1.949756in}{1.681960in}}%
\pgfpathcurveto{\pgfqpoint{1.949756in}{1.690196in}}{\pgfqpoint{1.946483in}{1.698096in}}{\pgfqpoint{1.940659in}{1.703920in}}%
\pgfpathcurveto{\pgfqpoint{1.934835in}{1.709744in}}{\pgfqpoint{1.926935in}{1.713017in}}{\pgfqpoint{1.918699in}{1.713017in}}%
\pgfpathcurveto{\pgfqpoint{1.910463in}{1.713017in}}{\pgfqpoint{1.902563in}{1.709744in}}{\pgfqpoint{1.896739in}{1.703920in}}%
\pgfpathcurveto{\pgfqpoint{1.890915in}{1.698096in}}{\pgfqpoint{1.887643in}{1.690196in}}{\pgfqpoint{1.887643in}{1.681960in}}%
\pgfpathcurveto{\pgfqpoint{1.887643in}{1.673724in}}{\pgfqpoint{1.890915in}{1.665824in}}{\pgfqpoint{1.896739in}{1.660000in}}%
\pgfpathcurveto{\pgfqpoint{1.902563in}{1.654176in}}{\pgfqpoint{1.910463in}{1.650904in}}{\pgfqpoint{1.918699in}{1.650904in}}%
\pgfpathclose%
\pgfusepath{stroke,fill}%
\end{pgfscope}%
\begin{pgfscope}%
\pgfpathrectangle{\pgfqpoint{0.100000in}{0.220728in}}{\pgfqpoint{3.696000in}{3.696000in}}%
\pgfusepath{clip}%
\pgfsetbuttcap%
\pgfsetroundjoin%
\definecolor{currentfill}{rgb}{0.121569,0.466667,0.705882}%
\pgfsetfillcolor{currentfill}%
\pgfsetfillopacity{0.914530}%
\pgfsetlinewidth{1.003750pt}%
\definecolor{currentstroke}{rgb}{0.121569,0.466667,0.705882}%
\pgfsetstrokecolor{currentstroke}%
\pgfsetstrokeopacity{0.914530}%
\pgfsetdash{}{0pt}%
\pgfpathmoveto{\pgfqpoint{2.627258in}{1.803069in}}%
\pgfpathcurveto{\pgfqpoint{2.635494in}{1.803069in}}{\pgfqpoint{2.643394in}{1.806341in}}{\pgfqpoint{2.649218in}{1.812165in}}%
\pgfpathcurveto{\pgfqpoint{2.655042in}{1.817989in}}{\pgfqpoint{2.658314in}{1.825889in}}{\pgfqpoint{2.658314in}{1.834125in}}%
\pgfpathcurveto{\pgfqpoint{2.658314in}{1.842362in}}{\pgfqpoint{2.655042in}{1.850262in}}{\pgfqpoint{2.649218in}{1.856086in}}%
\pgfpathcurveto{\pgfqpoint{2.643394in}{1.861910in}}{\pgfqpoint{2.635494in}{1.865182in}}{\pgfqpoint{2.627258in}{1.865182in}}%
\pgfpathcurveto{\pgfqpoint{2.619021in}{1.865182in}}{\pgfqpoint{2.611121in}{1.861910in}}{\pgfqpoint{2.605297in}{1.856086in}}%
\pgfpathcurveto{\pgfqpoint{2.599473in}{1.850262in}}{\pgfqpoint{2.596201in}{1.842362in}}{\pgfqpoint{2.596201in}{1.834125in}}%
\pgfpathcurveto{\pgfqpoint{2.596201in}{1.825889in}}{\pgfqpoint{2.599473in}{1.817989in}}{\pgfqpoint{2.605297in}{1.812165in}}%
\pgfpathcurveto{\pgfqpoint{2.611121in}{1.806341in}}{\pgfqpoint{2.619021in}{1.803069in}}{\pgfqpoint{2.627258in}{1.803069in}}%
\pgfpathclose%
\pgfusepath{stroke,fill}%
\end{pgfscope}%
\begin{pgfscope}%
\pgfpathrectangle{\pgfqpoint{0.100000in}{0.220728in}}{\pgfqpoint{3.696000in}{3.696000in}}%
\pgfusepath{clip}%
\pgfsetbuttcap%
\pgfsetroundjoin%
\definecolor{currentfill}{rgb}{0.121569,0.466667,0.705882}%
\pgfsetfillcolor{currentfill}%
\pgfsetfillopacity{0.914847}%
\pgfsetlinewidth{1.003750pt}%
\definecolor{currentstroke}{rgb}{0.121569,0.466667,0.705882}%
\pgfsetstrokecolor{currentstroke}%
\pgfsetstrokeopacity{0.914847}%
\pgfsetdash{}{0pt}%
\pgfpathmoveto{\pgfqpoint{1.929276in}{1.642977in}}%
\pgfpathcurveto{\pgfqpoint{1.937512in}{1.642977in}}{\pgfqpoint{1.945412in}{1.646250in}}{\pgfqpoint{1.951236in}{1.652073in}}%
\pgfpathcurveto{\pgfqpoint{1.957060in}{1.657897in}}{\pgfqpoint{1.960333in}{1.665797in}}{\pgfqpoint{1.960333in}{1.674034in}}%
\pgfpathcurveto{\pgfqpoint{1.960333in}{1.682270in}}{\pgfqpoint{1.957060in}{1.690170in}}{\pgfqpoint{1.951236in}{1.695994in}}%
\pgfpathcurveto{\pgfqpoint{1.945412in}{1.701818in}}{\pgfqpoint{1.937512in}{1.705090in}}{\pgfqpoint{1.929276in}{1.705090in}}%
\pgfpathcurveto{\pgfqpoint{1.921040in}{1.705090in}}{\pgfqpoint{1.913140in}{1.701818in}}{\pgfqpoint{1.907316in}{1.695994in}}%
\pgfpathcurveto{\pgfqpoint{1.901492in}{1.690170in}}{\pgfqpoint{1.898220in}{1.682270in}}{\pgfqpoint{1.898220in}{1.674034in}}%
\pgfpathcurveto{\pgfqpoint{1.898220in}{1.665797in}}{\pgfqpoint{1.901492in}{1.657897in}}{\pgfqpoint{1.907316in}{1.652073in}}%
\pgfpathcurveto{\pgfqpoint{1.913140in}{1.646250in}}{\pgfqpoint{1.921040in}{1.642977in}}{\pgfqpoint{1.929276in}{1.642977in}}%
\pgfpathclose%
\pgfusepath{stroke,fill}%
\end{pgfscope}%
\begin{pgfscope}%
\pgfpathrectangle{\pgfqpoint{0.100000in}{0.220728in}}{\pgfqpoint{3.696000in}{3.696000in}}%
\pgfusepath{clip}%
\pgfsetbuttcap%
\pgfsetroundjoin%
\definecolor{currentfill}{rgb}{0.121569,0.466667,0.705882}%
\pgfsetfillcolor{currentfill}%
\pgfsetfillopacity{0.916174}%
\pgfsetlinewidth{1.003750pt}%
\definecolor{currentstroke}{rgb}{0.121569,0.466667,0.705882}%
\pgfsetstrokecolor{currentstroke}%
\pgfsetstrokeopacity{0.916174}%
\pgfsetdash{}{0pt}%
\pgfpathmoveto{\pgfqpoint{2.622399in}{1.795082in}}%
\pgfpathcurveto{\pgfqpoint{2.630636in}{1.795082in}}{\pgfqpoint{2.638536in}{1.798355in}}{\pgfqpoint{2.644360in}{1.804179in}}%
\pgfpathcurveto{\pgfqpoint{2.650184in}{1.810003in}}{\pgfqpoint{2.653456in}{1.817903in}}{\pgfqpoint{2.653456in}{1.826139in}}%
\pgfpathcurveto{\pgfqpoint{2.653456in}{1.834375in}}{\pgfqpoint{2.650184in}{1.842275in}}{\pgfqpoint{2.644360in}{1.848099in}}%
\pgfpathcurveto{\pgfqpoint{2.638536in}{1.853923in}}{\pgfqpoint{2.630636in}{1.857195in}}{\pgfqpoint{2.622399in}{1.857195in}}%
\pgfpathcurveto{\pgfqpoint{2.614163in}{1.857195in}}{\pgfqpoint{2.606263in}{1.853923in}}{\pgfqpoint{2.600439in}{1.848099in}}%
\pgfpathcurveto{\pgfqpoint{2.594615in}{1.842275in}}{\pgfqpoint{2.591343in}{1.834375in}}{\pgfqpoint{2.591343in}{1.826139in}}%
\pgfpathcurveto{\pgfqpoint{2.591343in}{1.817903in}}{\pgfqpoint{2.594615in}{1.810003in}}{\pgfqpoint{2.600439in}{1.804179in}}%
\pgfpathcurveto{\pgfqpoint{2.606263in}{1.798355in}}{\pgfqpoint{2.614163in}{1.795082in}}{\pgfqpoint{2.622399in}{1.795082in}}%
\pgfpathclose%
\pgfusepath{stroke,fill}%
\end{pgfscope}%
\begin{pgfscope}%
\pgfpathrectangle{\pgfqpoint{0.100000in}{0.220728in}}{\pgfqpoint{3.696000in}{3.696000in}}%
\pgfusepath{clip}%
\pgfsetbuttcap%
\pgfsetroundjoin%
\definecolor{currentfill}{rgb}{0.121569,0.466667,0.705882}%
\pgfsetfillcolor{currentfill}%
\pgfsetfillopacity{0.917129}%
\pgfsetlinewidth{1.003750pt}%
\definecolor{currentstroke}{rgb}{0.121569,0.466667,0.705882}%
\pgfsetstrokecolor{currentstroke}%
\pgfsetstrokeopacity{0.917129}%
\pgfsetdash{}{0pt}%
\pgfpathmoveto{\pgfqpoint{2.619800in}{1.790867in}}%
\pgfpathcurveto{\pgfqpoint{2.628036in}{1.790867in}}{\pgfqpoint{2.635936in}{1.794139in}}{\pgfqpoint{2.641760in}{1.799963in}}%
\pgfpathcurveto{\pgfqpoint{2.647584in}{1.805787in}}{\pgfqpoint{2.650856in}{1.813687in}}{\pgfqpoint{2.650856in}{1.821923in}}%
\pgfpathcurveto{\pgfqpoint{2.650856in}{1.830159in}}{\pgfqpoint{2.647584in}{1.838060in}}{\pgfqpoint{2.641760in}{1.843883in}}%
\pgfpathcurveto{\pgfqpoint{2.635936in}{1.849707in}}{\pgfqpoint{2.628036in}{1.852980in}}{\pgfqpoint{2.619800in}{1.852980in}}%
\pgfpathcurveto{\pgfqpoint{2.611564in}{1.852980in}}{\pgfqpoint{2.603664in}{1.849707in}}{\pgfqpoint{2.597840in}{1.843883in}}%
\pgfpathcurveto{\pgfqpoint{2.592016in}{1.838060in}}{\pgfqpoint{2.588743in}{1.830159in}}{\pgfqpoint{2.588743in}{1.821923in}}%
\pgfpathcurveto{\pgfqpoint{2.588743in}{1.813687in}}{\pgfqpoint{2.592016in}{1.805787in}}{\pgfqpoint{2.597840in}{1.799963in}}%
\pgfpathcurveto{\pgfqpoint{2.603664in}{1.794139in}}{\pgfqpoint{2.611564in}{1.790867in}}{\pgfqpoint{2.619800in}{1.790867in}}%
\pgfpathclose%
\pgfusepath{stroke,fill}%
\end{pgfscope}%
\begin{pgfscope}%
\pgfpathrectangle{\pgfqpoint{0.100000in}{0.220728in}}{\pgfqpoint{3.696000in}{3.696000in}}%
\pgfusepath{clip}%
\pgfsetbuttcap%
\pgfsetroundjoin%
\definecolor{currentfill}{rgb}{0.121569,0.466667,0.705882}%
\pgfsetfillcolor{currentfill}%
\pgfsetfillopacity{0.917571}%
\pgfsetlinewidth{1.003750pt}%
\definecolor{currentstroke}{rgb}{0.121569,0.466667,0.705882}%
\pgfsetstrokecolor{currentstroke}%
\pgfsetstrokeopacity{0.917571}%
\pgfsetdash{}{0pt}%
\pgfpathmoveto{\pgfqpoint{2.618047in}{1.788615in}}%
\pgfpathcurveto{\pgfqpoint{2.626284in}{1.788615in}}{\pgfqpoint{2.634184in}{1.791888in}}{\pgfqpoint{2.640008in}{1.797712in}}%
\pgfpathcurveto{\pgfqpoint{2.645832in}{1.803536in}}{\pgfqpoint{2.649104in}{1.811436in}}{\pgfqpoint{2.649104in}{1.819672in}}%
\pgfpathcurveto{\pgfqpoint{2.649104in}{1.827908in}}{\pgfqpoint{2.645832in}{1.835808in}}{\pgfqpoint{2.640008in}{1.841632in}}%
\pgfpathcurveto{\pgfqpoint{2.634184in}{1.847456in}}{\pgfqpoint{2.626284in}{1.850728in}}{\pgfqpoint{2.618047in}{1.850728in}}%
\pgfpathcurveto{\pgfqpoint{2.609811in}{1.850728in}}{\pgfqpoint{2.601911in}{1.847456in}}{\pgfqpoint{2.596087in}{1.841632in}}%
\pgfpathcurveto{\pgfqpoint{2.590263in}{1.835808in}}{\pgfqpoint{2.586991in}{1.827908in}}{\pgfqpoint{2.586991in}{1.819672in}}%
\pgfpathcurveto{\pgfqpoint{2.586991in}{1.811436in}}{\pgfqpoint{2.590263in}{1.803536in}}{\pgfqpoint{2.596087in}{1.797712in}}%
\pgfpathcurveto{\pgfqpoint{2.601911in}{1.791888in}}{\pgfqpoint{2.609811in}{1.788615in}}{\pgfqpoint{2.618047in}{1.788615in}}%
\pgfpathclose%
\pgfusepath{stroke,fill}%
\end{pgfscope}%
\begin{pgfscope}%
\pgfpathrectangle{\pgfqpoint{0.100000in}{0.220728in}}{\pgfqpoint{3.696000in}{3.696000in}}%
\pgfusepath{clip}%
\pgfsetbuttcap%
\pgfsetroundjoin%
\definecolor{currentfill}{rgb}{0.121569,0.466667,0.705882}%
\pgfsetfillcolor{currentfill}%
\pgfsetfillopacity{0.917834}%
\pgfsetlinewidth{1.003750pt}%
\definecolor{currentstroke}{rgb}{0.121569,0.466667,0.705882}%
\pgfsetstrokecolor{currentstroke}%
\pgfsetstrokeopacity{0.917834}%
\pgfsetdash{}{0pt}%
\pgfpathmoveto{\pgfqpoint{2.617193in}{1.787307in}}%
\pgfpathcurveto{\pgfqpoint{2.625429in}{1.787307in}}{\pgfqpoint{2.633329in}{1.790580in}}{\pgfqpoint{2.639153in}{1.796404in}}%
\pgfpathcurveto{\pgfqpoint{2.644977in}{1.802228in}}{\pgfqpoint{2.648249in}{1.810128in}}{\pgfqpoint{2.648249in}{1.818364in}}%
\pgfpathcurveto{\pgfqpoint{2.648249in}{1.826600in}}{\pgfqpoint{2.644977in}{1.834500in}}{\pgfqpoint{2.639153in}{1.840324in}}%
\pgfpathcurveto{\pgfqpoint{2.633329in}{1.846148in}}{\pgfqpoint{2.625429in}{1.849420in}}{\pgfqpoint{2.617193in}{1.849420in}}%
\pgfpathcurveto{\pgfqpoint{2.608957in}{1.849420in}}{\pgfqpoint{2.601057in}{1.846148in}}{\pgfqpoint{2.595233in}{1.840324in}}%
\pgfpathcurveto{\pgfqpoint{2.589409in}{1.834500in}}{\pgfqpoint{2.586136in}{1.826600in}}{\pgfqpoint{2.586136in}{1.818364in}}%
\pgfpathcurveto{\pgfqpoint{2.586136in}{1.810128in}}{\pgfqpoint{2.589409in}{1.802228in}}{\pgfqpoint{2.595233in}{1.796404in}}%
\pgfpathcurveto{\pgfqpoint{2.601057in}{1.790580in}}{\pgfqpoint{2.608957in}{1.787307in}}{\pgfqpoint{2.617193in}{1.787307in}}%
\pgfpathclose%
\pgfusepath{stroke,fill}%
\end{pgfscope}%
\begin{pgfscope}%
\pgfpathrectangle{\pgfqpoint{0.100000in}{0.220728in}}{\pgfqpoint{3.696000in}{3.696000in}}%
\pgfusepath{clip}%
\pgfsetbuttcap%
\pgfsetroundjoin%
\definecolor{currentfill}{rgb}{0.121569,0.466667,0.705882}%
\pgfsetfillcolor{currentfill}%
\pgfsetfillopacity{0.918624}%
\pgfsetlinewidth{1.003750pt}%
\definecolor{currentstroke}{rgb}{0.121569,0.466667,0.705882}%
\pgfsetstrokecolor{currentstroke}%
\pgfsetstrokeopacity{0.918624}%
\pgfsetdash{}{0pt}%
\pgfpathmoveto{\pgfqpoint{2.614177in}{1.783938in}}%
\pgfpathcurveto{\pgfqpoint{2.622413in}{1.783938in}}{\pgfqpoint{2.630313in}{1.787210in}}{\pgfqpoint{2.636137in}{1.793034in}}%
\pgfpathcurveto{\pgfqpoint{2.641961in}{1.798858in}}{\pgfqpoint{2.645234in}{1.806758in}}{\pgfqpoint{2.645234in}{1.814995in}}%
\pgfpathcurveto{\pgfqpoint{2.645234in}{1.823231in}}{\pgfqpoint{2.641961in}{1.831131in}}{\pgfqpoint{2.636137in}{1.836955in}}%
\pgfpathcurveto{\pgfqpoint{2.630313in}{1.842779in}}{\pgfqpoint{2.622413in}{1.846051in}}{\pgfqpoint{2.614177in}{1.846051in}}%
\pgfpathcurveto{\pgfqpoint{2.605941in}{1.846051in}}{\pgfqpoint{2.598041in}{1.842779in}}{\pgfqpoint{2.592217in}{1.836955in}}%
\pgfpathcurveto{\pgfqpoint{2.586393in}{1.831131in}}{\pgfqpoint{2.583121in}{1.823231in}}{\pgfqpoint{2.583121in}{1.814995in}}%
\pgfpathcurveto{\pgfqpoint{2.583121in}{1.806758in}}{\pgfqpoint{2.586393in}{1.798858in}}{\pgfqpoint{2.592217in}{1.793034in}}%
\pgfpathcurveto{\pgfqpoint{2.598041in}{1.787210in}}{\pgfqpoint{2.605941in}{1.783938in}}{\pgfqpoint{2.614177in}{1.783938in}}%
\pgfpathclose%
\pgfusepath{stroke,fill}%
\end{pgfscope}%
\begin{pgfscope}%
\pgfpathrectangle{\pgfqpoint{0.100000in}{0.220728in}}{\pgfqpoint{3.696000in}{3.696000in}}%
\pgfusepath{clip}%
\pgfsetbuttcap%
\pgfsetroundjoin%
\definecolor{currentfill}{rgb}{0.121569,0.466667,0.705882}%
\pgfsetfillcolor{currentfill}%
\pgfsetfillopacity{0.919108}%
\pgfsetlinewidth{1.003750pt}%
\definecolor{currentstroke}{rgb}{0.121569,0.466667,0.705882}%
\pgfsetstrokecolor{currentstroke}%
\pgfsetstrokeopacity{0.919108}%
\pgfsetdash{}{0pt}%
\pgfpathmoveto{\pgfqpoint{1.948522in}{1.629923in}}%
\pgfpathcurveto{\pgfqpoint{1.956758in}{1.629923in}}{\pgfqpoint{1.964658in}{1.633195in}}{\pgfqpoint{1.970482in}{1.639019in}}%
\pgfpathcurveto{\pgfqpoint{1.976306in}{1.644843in}}{\pgfqpoint{1.979578in}{1.652743in}}{\pgfqpoint{1.979578in}{1.660979in}}%
\pgfpathcurveto{\pgfqpoint{1.979578in}{1.669216in}}{\pgfqpoint{1.976306in}{1.677116in}}{\pgfqpoint{1.970482in}{1.682939in}}%
\pgfpathcurveto{\pgfqpoint{1.964658in}{1.688763in}}{\pgfqpoint{1.956758in}{1.692036in}}{\pgfqpoint{1.948522in}{1.692036in}}%
\pgfpathcurveto{\pgfqpoint{1.940285in}{1.692036in}}{\pgfqpoint{1.932385in}{1.688763in}}{\pgfqpoint{1.926561in}{1.682939in}}%
\pgfpathcurveto{\pgfqpoint{1.920738in}{1.677116in}}{\pgfqpoint{1.917465in}{1.669216in}}{\pgfqpoint{1.917465in}{1.660979in}}%
\pgfpathcurveto{\pgfqpoint{1.917465in}{1.652743in}}{\pgfqpoint{1.920738in}{1.644843in}}{\pgfqpoint{1.926561in}{1.639019in}}%
\pgfpathcurveto{\pgfqpoint{1.932385in}{1.633195in}}{\pgfqpoint{1.940285in}{1.629923in}}{\pgfqpoint{1.948522in}{1.629923in}}%
\pgfpathclose%
\pgfusepath{stroke,fill}%
\end{pgfscope}%
\begin{pgfscope}%
\pgfpathrectangle{\pgfqpoint{0.100000in}{0.220728in}}{\pgfqpoint{3.696000in}{3.696000in}}%
\pgfusepath{clip}%
\pgfsetbuttcap%
\pgfsetroundjoin%
\definecolor{currentfill}{rgb}{0.121569,0.466667,0.705882}%
\pgfsetfillcolor{currentfill}%
\pgfsetfillopacity{0.920162}%
\pgfsetlinewidth{1.003750pt}%
\definecolor{currentstroke}{rgb}{0.121569,0.466667,0.705882}%
\pgfsetstrokecolor{currentstroke}%
\pgfsetstrokeopacity{0.920162}%
\pgfsetdash{}{0pt}%
\pgfpathmoveto{\pgfqpoint{2.609941in}{1.775621in}}%
\pgfpathcurveto{\pgfqpoint{2.618178in}{1.775621in}}{\pgfqpoint{2.626078in}{1.778893in}}{\pgfqpoint{2.631902in}{1.784717in}}%
\pgfpathcurveto{\pgfqpoint{2.637726in}{1.790541in}}{\pgfqpoint{2.640998in}{1.798441in}}{\pgfqpoint{2.640998in}{1.806678in}}%
\pgfpathcurveto{\pgfqpoint{2.640998in}{1.814914in}}{\pgfqpoint{2.637726in}{1.822814in}}{\pgfqpoint{2.631902in}{1.828638in}}%
\pgfpathcurveto{\pgfqpoint{2.626078in}{1.834462in}}{\pgfqpoint{2.618178in}{1.837734in}}{\pgfqpoint{2.609941in}{1.837734in}}%
\pgfpathcurveto{\pgfqpoint{2.601705in}{1.837734in}}{\pgfqpoint{2.593805in}{1.834462in}}{\pgfqpoint{2.587981in}{1.828638in}}%
\pgfpathcurveto{\pgfqpoint{2.582157in}{1.822814in}}{\pgfqpoint{2.578885in}{1.814914in}}{\pgfqpoint{2.578885in}{1.806678in}}%
\pgfpathcurveto{\pgfqpoint{2.578885in}{1.798441in}}{\pgfqpoint{2.582157in}{1.790541in}}{\pgfqpoint{2.587981in}{1.784717in}}%
\pgfpathcurveto{\pgfqpoint{2.593805in}{1.778893in}}{\pgfqpoint{2.601705in}{1.775621in}}{\pgfqpoint{2.609941in}{1.775621in}}%
\pgfpathclose%
\pgfusepath{stroke,fill}%
\end{pgfscope}%
\begin{pgfscope}%
\pgfpathrectangle{\pgfqpoint{0.100000in}{0.220728in}}{\pgfqpoint{3.696000in}{3.696000in}}%
\pgfusepath{clip}%
\pgfsetbuttcap%
\pgfsetroundjoin%
\definecolor{currentfill}{rgb}{0.121569,0.466667,0.705882}%
\pgfsetfillcolor{currentfill}%
\pgfsetfillopacity{0.922313}%
\pgfsetlinewidth{1.003750pt}%
\definecolor{currentstroke}{rgb}{0.121569,0.466667,0.705882}%
\pgfsetstrokecolor{currentstroke}%
\pgfsetstrokeopacity{0.922313}%
\pgfsetdash{}{0pt}%
\pgfpathmoveto{\pgfqpoint{2.602591in}{1.766162in}}%
\pgfpathcurveto{\pgfqpoint{2.610828in}{1.766162in}}{\pgfqpoint{2.618728in}{1.769434in}}{\pgfqpoint{2.624552in}{1.775258in}}%
\pgfpathcurveto{\pgfqpoint{2.630376in}{1.781082in}}{\pgfqpoint{2.633648in}{1.788982in}}{\pgfqpoint{2.633648in}{1.797219in}}%
\pgfpathcurveto{\pgfqpoint{2.633648in}{1.805455in}}{\pgfqpoint{2.630376in}{1.813355in}}{\pgfqpoint{2.624552in}{1.819179in}}%
\pgfpathcurveto{\pgfqpoint{2.618728in}{1.825003in}}{\pgfqpoint{2.610828in}{1.828275in}}{\pgfqpoint{2.602591in}{1.828275in}}%
\pgfpathcurveto{\pgfqpoint{2.594355in}{1.828275in}}{\pgfqpoint{2.586455in}{1.825003in}}{\pgfqpoint{2.580631in}{1.819179in}}%
\pgfpathcurveto{\pgfqpoint{2.574807in}{1.813355in}}{\pgfqpoint{2.571535in}{1.805455in}}{\pgfqpoint{2.571535in}{1.797219in}}%
\pgfpathcurveto{\pgfqpoint{2.571535in}{1.788982in}}{\pgfqpoint{2.574807in}{1.781082in}}{\pgfqpoint{2.580631in}{1.775258in}}%
\pgfpathcurveto{\pgfqpoint{2.586455in}{1.769434in}}{\pgfqpoint{2.594355in}{1.766162in}}{\pgfqpoint{2.602591in}{1.766162in}}%
\pgfpathclose%
\pgfusepath{stroke,fill}%
\end{pgfscope}%
\begin{pgfscope}%
\pgfpathrectangle{\pgfqpoint{0.100000in}{0.220728in}}{\pgfqpoint{3.696000in}{3.696000in}}%
\pgfusepath{clip}%
\pgfsetbuttcap%
\pgfsetroundjoin%
\definecolor{currentfill}{rgb}{0.121569,0.466667,0.705882}%
\pgfsetfillcolor{currentfill}%
\pgfsetfillopacity{0.925471}%
\pgfsetlinewidth{1.003750pt}%
\definecolor{currentstroke}{rgb}{0.121569,0.466667,0.705882}%
\pgfsetstrokecolor{currentstroke}%
\pgfsetstrokeopacity{0.925471}%
\pgfsetdash{}{0pt}%
\pgfpathmoveto{\pgfqpoint{2.594113in}{1.752388in}}%
\pgfpathcurveto{\pgfqpoint{2.602349in}{1.752388in}}{\pgfqpoint{2.610249in}{1.755660in}}{\pgfqpoint{2.616073in}{1.761484in}}%
\pgfpathcurveto{\pgfqpoint{2.621897in}{1.767308in}}{\pgfqpoint{2.625170in}{1.775208in}}{\pgfqpoint{2.625170in}{1.783444in}}%
\pgfpathcurveto{\pgfqpoint{2.625170in}{1.791680in}}{\pgfqpoint{2.621897in}{1.799580in}}{\pgfqpoint{2.616073in}{1.805404in}}%
\pgfpathcurveto{\pgfqpoint{2.610249in}{1.811228in}}{\pgfqpoint{2.602349in}{1.814501in}}{\pgfqpoint{2.594113in}{1.814501in}}%
\pgfpathcurveto{\pgfqpoint{2.585877in}{1.814501in}}{\pgfqpoint{2.577977in}{1.811228in}}{\pgfqpoint{2.572153in}{1.805404in}}%
\pgfpathcurveto{\pgfqpoint{2.566329in}{1.799580in}}{\pgfqpoint{2.563057in}{1.791680in}}{\pgfqpoint{2.563057in}{1.783444in}}%
\pgfpathcurveto{\pgfqpoint{2.563057in}{1.775208in}}{\pgfqpoint{2.566329in}{1.767308in}}{\pgfqpoint{2.572153in}{1.761484in}}%
\pgfpathcurveto{\pgfqpoint{2.577977in}{1.755660in}}{\pgfqpoint{2.585877in}{1.752388in}}{\pgfqpoint{2.594113in}{1.752388in}}%
\pgfpathclose%
\pgfusepath{stroke,fill}%
\end{pgfscope}%
\begin{pgfscope}%
\pgfpathrectangle{\pgfqpoint{0.100000in}{0.220728in}}{\pgfqpoint{3.696000in}{3.696000in}}%
\pgfusepath{clip}%
\pgfsetbuttcap%
\pgfsetroundjoin%
\definecolor{currentfill}{rgb}{0.121569,0.466667,0.705882}%
\pgfsetfillcolor{currentfill}%
\pgfsetfillopacity{0.925568}%
\pgfsetlinewidth{1.003750pt}%
\definecolor{currentstroke}{rgb}{0.121569,0.466667,0.705882}%
\pgfsetstrokecolor{currentstroke}%
\pgfsetstrokeopacity{0.925568}%
\pgfsetdash{}{0pt}%
\pgfpathmoveto{\pgfqpoint{1.985308in}{1.604329in}}%
\pgfpathcurveto{\pgfqpoint{1.993544in}{1.604329in}}{\pgfqpoint{2.001444in}{1.607601in}}{\pgfqpoint{2.007268in}{1.613425in}}%
\pgfpathcurveto{\pgfqpoint{2.013092in}{1.619249in}}{\pgfqpoint{2.016364in}{1.627149in}}{\pgfqpoint{2.016364in}{1.635385in}}%
\pgfpathcurveto{\pgfqpoint{2.016364in}{1.643622in}}{\pgfqpoint{2.013092in}{1.651522in}}{\pgfqpoint{2.007268in}{1.657346in}}%
\pgfpathcurveto{\pgfqpoint{2.001444in}{1.663170in}}{\pgfqpoint{1.993544in}{1.666442in}}{\pgfqpoint{1.985308in}{1.666442in}}%
\pgfpathcurveto{\pgfqpoint{1.977072in}{1.666442in}}{\pgfqpoint{1.969172in}{1.663170in}}{\pgfqpoint{1.963348in}{1.657346in}}%
\pgfpathcurveto{\pgfqpoint{1.957524in}{1.651522in}}{\pgfqpoint{1.954251in}{1.643622in}}{\pgfqpoint{1.954251in}{1.635385in}}%
\pgfpathcurveto{\pgfqpoint{1.954251in}{1.627149in}}{\pgfqpoint{1.957524in}{1.619249in}}{\pgfqpoint{1.963348in}{1.613425in}}%
\pgfpathcurveto{\pgfqpoint{1.969172in}{1.607601in}}{\pgfqpoint{1.977072in}{1.604329in}}{\pgfqpoint{1.985308in}{1.604329in}}%
\pgfpathclose%
\pgfusepath{stroke,fill}%
\end{pgfscope}%
\begin{pgfscope}%
\pgfpathrectangle{\pgfqpoint{0.100000in}{0.220728in}}{\pgfqpoint{3.696000in}{3.696000in}}%
\pgfusepath{clip}%
\pgfsetbuttcap%
\pgfsetroundjoin%
\definecolor{currentfill}{rgb}{0.121569,0.466667,0.705882}%
\pgfsetfillcolor{currentfill}%
\pgfsetfillopacity{0.928931}%
\pgfsetlinewidth{1.003750pt}%
\definecolor{currentstroke}{rgb}{0.121569,0.466667,0.705882}%
\pgfsetstrokecolor{currentstroke}%
\pgfsetstrokeopacity{0.928931}%
\pgfsetdash{}{0pt}%
\pgfpathmoveto{\pgfqpoint{2.582389in}{1.737301in}}%
\pgfpathcurveto{\pgfqpoint{2.590625in}{1.737301in}}{\pgfqpoint{2.598525in}{1.740574in}}{\pgfqpoint{2.604349in}{1.746398in}}%
\pgfpathcurveto{\pgfqpoint{2.610173in}{1.752222in}}{\pgfqpoint{2.613445in}{1.760122in}}{\pgfqpoint{2.613445in}{1.768358in}}%
\pgfpathcurveto{\pgfqpoint{2.613445in}{1.776594in}}{\pgfqpoint{2.610173in}{1.784494in}}{\pgfqpoint{2.604349in}{1.790318in}}%
\pgfpathcurveto{\pgfqpoint{2.598525in}{1.796142in}}{\pgfqpoint{2.590625in}{1.799414in}}{\pgfqpoint{2.582389in}{1.799414in}}%
\pgfpathcurveto{\pgfqpoint{2.574152in}{1.799414in}}{\pgfqpoint{2.566252in}{1.796142in}}{\pgfqpoint{2.560428in}{1.790318in}}%
\pgfpathcurveto{\pgfqpoint{2.554604in}{1.784494in}}{\pgfqpoint{2.551332in}{1.776594in}}{\pgfqpoint{2.551332in}{1.768358in}}%
\pgfpathcurveto{\pgfqpoint{2.551332in}{1.760122in}}{\pgfqpoint{2.554604in}{1.752222in}}{\pgfqpoint{2.560428in}{1.746398in}}%
\pgfpathcurveto{\pgfqpoint{2.566252in}{1.740574in}}{\pgfqpoint{2.574152in}{1.737301in}}{\pgfqpoint{2.582389in}{1.737301in}}%
\pgfpathclose%
\pgfusepath{stroke,fill}%
\end{pgfscope}%
\begin{pgfscope}%
\pgfpathrectangle{\pgfqpoint{0.100000in}{0.220728in}}{\pgfqpoint{3.696000in}{3.696000in}}%
\pgfusepath{clip}%
\pgfsetbuttcap%
\pgfsetroundjoin%
\definecolor{currentfill}{rgb}{0.121569,0.466667,0.705882}%
\pgfsetfillcolor{currentfill}%
\pgfsetfillopacity{0.932260}%
\pgfsetlinewidth{1.003750pt}%
\definecolor{currentstroke}{rgb}{0.121569,0.466667,0.705882}%
\pgfsetstrokecolor{currentstroke}%
\pgfsetstrokeopacity{0.932260}%
\pgfsetdash{}{0pt}%
\pgfpathmoveto{\pgfqpoint{2.022224in}{1.590686in}}%
\pgfpathcurveto{\pgfqpoint{2.030460in}{1.590686in}}{\pgfqpoint{2.038360in}{1.593958in}}{\pgfqpoint{2.044184in}{1.599782in}}%
\pgfpathcurveto{\pgfqpoint{2.050008in}{1.605606in}}{\pgfqpoint{2.053280in}{1.613506in}}{\pgfqpoint{2.053280in}{1.621742in}}%
\pgfpathcurveto{\pgfqpoint{2.053280in}{1.629978in}}{\pgfqpoint{2.050008in}{1.637878in}}{\pgfqpoint{2.044184in}{1.643702in}}%
\pgfpathcurveto{\pgfqpoint{2.038360in}{1.649526in}}{\pgfqpoint{2.030460in}{1.652799in}}{\pgfqpoint{2.022224in}{1.652799in}}%
\pgfpathcurveto{\pgfqpoint{2.013988in}{1.652799in}}{\pgfqpoint{2.006088in}{1.649526in}}{\pgfqpoint{2.000264in}{1.643702in}}%
\pgfpathcurveto{\pgfqpoint{1.994440in}{1.637878in}}{\pgfqpoint{1.991167in}{1.629978in}}{\pgfqpoint{1.991167in}{1.621742in}}%
\pgfpathcurveto{\pgfqpoint{1.991167in}{1.613506in}}{\pgfqpoint{1.994440in}{1.605606in}}{\pgfqpoint{2.000264in}{1.599782in}}%
\pgfpathcurveto{\pgfqpoint{2.006088in}{1.593958in}}{\pgfqpoint{2.013988in}{1.590686in}}{\pgfqpoint{2.022224in}{1.590686in}}%
\pgfpathclose%
\pgfusepath{stroke,fill}%
\end{pgfscope}%
\begin{pgfscope}%
\pgfpathrectangle{\pgfqpoint{0.100000in}{0.220728in}}{\pgfqpoint{3.696000in}{3.696000in}}%
\pgfusepath{clip}%
\pgfsetbuttcap%
\pgfsetroundjoin%
\definecolor{currentfill}{rgb}{0.121569,0.466667,0.705882}%
\pgfsetfillcolor{currentfill}%
\pgfsetfillopacity{0.933238}%
\pgfsetlinewidth{1.003750pt}%
\definecolor{currentstroke}{rgb}{0.121569,0.466667,0.705882}%
\pgfsetstrokecolor{currentstroke}%
\pgfsetstrokeopacity{0.933238}%
\pgfsetdash{}{0pt}%
\pgfpathmoveto{\pgfqpoint{2.570996in}{1.719649in}}%
\pgfpathcurveto{\pgfqpoint{2.579232in}{1.719649in}}{\pgfqpoint{2.587132in}{1.722922in}}{\pgfqpoint{2.592956in}{1.728746in}}%
\pgfpathcurveto{\pgfqpoint{2.598780in}{1.734570in}}{\pgfqpoint{2.602052in}{1.742470in}}{\pgfqpoint{2.602052in}{1.750706in}}%
\pgfpathcurveto{\pgfqpoint{2.602052in}{1.758942in}}{\pgfqpoint{2.598780in}{1.766842in}}{\pgfqpoint{2.592956in}{1.772666in}}%
\pgfpathcurveto{\pgfqpoint{2.587132in}{1.778490in}}{\pgfqpoint{2.579232in}{1.781762in}}{\pgfqpoint{2.570996in}{1.781762in}}%
\pgfpathcurveto{\pgfqpoint{2.562760in}{1.781762in}}{\pgfqpoint{2.554860in}{1.778490in}}{\pgfqpoint{2.549036in}{1.772666in}}%
\pgfpathcurveto{\pgfqpoint{2.543212in}{1.766842in}}{\pgfqpoint{2.539939in}{1.758942in}}{\pgfqpoint{2.539939in}{1.750706in}}%
\pgfpathcurveto{\pgfqpoint{2.539939in}{1.742470in}}{\pgfqpoint{2.543212in}{1.734570in}}{\pgfqpoint{2.549036in}{1.728746in}}%
\pgfpathcurveto{\pgfqpoint{2.554860in}{1.722922in}}{\pgfqpoint{2.562760in}{1.719649in}}{\pgfqpoint{2.570996in}{1.719649in}}%
\pgfpathclose%
\pgfusepath{stroke,fill}%
\end{pgfscope}%
\begin{pgfscope}%
\pgfpathrectangle{\pgfqpoint{0.100000in}{0.220728in}}{\pgfqpoint{3.696000in}{3.696000in}}%
\pgfusepath{clip}%
\pgfsetbuttcap%
\pgfsetroundjoin%
\definecolor{currentfill}{rgb}{0.121569,0.466667,0.705882}%
\pgfsetfillcolor{currentfill}%
\pgfsetfillopacity{0.935322}%
\pgfsetlinewidth{1.003750pt}%
\definecolor{currentstroke}{rgb}{0.121569,0.466667,0.705882}%
\pgfsetstrokecolor{currentstroke}%
\pgfsetstrokeopacity{0.935322}%
\pgfsetdash{}{0pt}%
\pgfpathmoveto{\pgfqpoint{2.564396in}{1.708999in}}%
\pgfpathcurveto{\pgfqpoint{2.572633in}{1.708999in}}{\pgfqpoint{2.580533in}{1.712272in}}{\pgfqpoint{2.586357in}{1.718096in}}%
\pgfpathcurveto{\pgfqpoint{2.592180in}{1.723919in}}{\pgfqpoint{2.595453in}{1.731820in}}{\pgfqpoint{2.595453in}{1.740056in}}%
\pgfpathcurveto{\pgfqpoint{2.595453in}{1.748292in}}{\pgfqpoint{2.592180in}{1.756192in}}{\pgfqpoint{2.586357in}{1.762016in}}%
\pgfpathcurveto{\pgfqpoint{2.580533in}{1.767840in}}{\pgfqpoint{2.572633in}{1.771112in}}{\pgfqpoint{2.564396in}{1.771112in}}%
\pgfpathcurveto{\pgfqpoint{2.556160in}{1.771112in}}{\pgfqpoint{2.548260in}{1.767840in}}{\pgfqpoint{2.542436in}{1.762016in}}%
\pgfpathcurveto{\pgfqpoint{2.536612in}{1.756192in}}{\pgfqpoint{2.533340in}{1.748292in}}{\pgfqpoint{2.533340in}{1.740056in}}%
\pgfpathcurveto{\pgfqpoint{2.533340in}{1.731820in}}{\pgfqpoint{2.536612in}{1.723919in}}{\pgfqpoint{2.542436in}{1.718096in}}%
\pgfpathcurveto{\pgfqpoint{2.548260in}{1.712272in}}{\pgfqpoint{2.556160in}{1.708999in}}{\pgfqpoint{2.564396in}{1.708999in}}%
\pgfpathclose%
\pgfusepath{stroke,fill}%
\end{pgfscope}%
\begin{pgfscope}%
\pgfpathrectangle{\pgfqpoint{0.100000in}{0.220728in}}{\pgfqpoint{3.696000in}{3.696000in}}%
\pgfusepath{clip}%
\pgfsetbuttcap%
\pgfsetroundjoin%
\definecolor{currentfill}{rgb}{0.121569,0.466667,0.705882}%
\pgfsetfillcolor{currentfill}%
\pgfsetfillopacity{0.936599}%
\pgfsetlinewidth{1.003750pt}%
\definecolor{currentstroke}{rgb}{0.121569,0.466667,0.705882}%
\pgfsetstrokecolor{currentstroke}%
\pgfsetstrokeopacity{0.936599}%
\pgfsetdash{}{0pt}%
\pgfpathmoveto{\pgfqpoint{2.560737in}{1.703740in}}%
\pgfpathcurveto{\pgfqpoint{2.568974in}{1.703740in}}{\pgfqpoint{2.576874in}{1.707013in}}{\pgfqpoint{2.582698in}{1.712837in}}%
\pgfpathcurveto{\pgfqpoint{2.588521in}{1.718661in}}{\pgfqpoint{2.591794in}{1.726561in}}{\pgfqpoint{2.591794in}{1.734797in}}%
\pgfpathcurveto{\pgfqpoint{2.591794in}{1.743033in}}{\pgfqpoint{2.588521in}{1.750933in}}{\pgfqpoint{2.582698in}{1.756757in}}%
\pgfpathcurveto{\pgfqpoint{2.576874in}{1.762581in}}{\pgfqpoint{2.568974in}{1.765853in}}{\pgfqpoint{2.560737in}{1.765853in}}%
\pgfpathcurveto{\pgfqpoint{2.552501in}{1.765853in}}{\pgfqpoint{2.544601in}{1.762581in}}{\pgfqpoint{2.538777in}{1.756757in}}%
\pgfpathcurveto{\pgfqpoint{2.532953in}{1.750933in}}{\pgfqpoint{2.529681in}{1.743033in}}{\pgfqpoint{2.529681in}{1.734797in}}%
\pgfpathcurveto{\pgfqpoint{2.529681in}{1.726561in}}{\pgfqpoint{2.532953in}{1.718661in}}{\pgfqpoint{2.538777in}{1.712837in}}%
\pgfpathcurveto{\pgfqpoint{2.544601in}{1.707013in}}{\pgfqpoint{2.552501in}{1.703740in}}{\pgfqpoint{2.560737in}{1.703740in}}%
\pgfpathclose%
\pgfusepath{stroke,fill}%
\end{pgfscope}%
\begin{pgfscope}%
\pgfpathrectangle{\pgfqpoint{0.100000in}{0.220728in}}{\pgfqpoint{3.696000in}{3.696000in}}%
\pgfusepath{clip}%
\pgfsetbuttcap%
\pgfsetroundjoin%
\definecolor{currentfill}{rgb}{0.121569,0.466667,0.705882}%
\pgfsetfillcolor{currentfill}%
\pgfsetfillopacity{0.937279}%
\pgfsetlinewidth{1.003750pt}%
\definecolor{currentstroke}{rgb}{0.121569,0.466667,0.705882}%
\pgfsetstrokecolor{currentstroke}%
\pgfsetstrokeopacity{0.937279}%
\pgfsetdash{}{0pt}%
\pgfpathmoveto{\pgfqpoint{2.558712in}{1.700785in}}%
\pgfpathcurveto{\pgfqpoint{2.566948in}{1.700785in}}{\pgfqpoint{2.574849in}{1.704057in}}{\pgfqpoint{2.580672in}{1.709881in}}%
\pgfpathcurveto{\pgfqpoint{2.586496in}{1.715705in}}{\pgfqpoint{2.589769in}{1.723605in}}{\pgfqpoint{2.589769in}{1.731841in}}%
\pgfpathcurveto{\pgfqpoint{2.589769in}{1.740078in}}{\pgfqpoint{2.586496in}{1.747978in}}{\pgfqpoint{2.580672in}{1.753802in}}%
\pgfpathcurveto{\pgfqpoint{2.574849in}{1.759626in}}{\pgfqpoint{2.566948in}{1.762898in}}{\pgfqpoint{2.558712in}{1.762898in}}%
\pgfpathcurveto{\pgfqpoint{2.550476in}{1.762898in}}{\pgfqpoint{2.542576in}{1.759626in}}{\pgfqpoint{2.536752in}{1.753802in}}%
\pgfpathcurveto{\pgfqpoint{2.530928in}{1.747978in}}{\pgfqpoint{2.527656in}{1.740078in}}{\pgfqpoint{2.527656in}{1.731841in}}%
\pgfpathcurveto{\pgfqpoint{2.527656in}{1.723605in}}{\pgfqpoint{2.530928in}{1.715705in}}{\pgfqpoint{2.536752in}{1.709881in}}%
\pgfpathcurveto{\pgfqpoint{2.542576in}{1.704057in}}{\pgfqpoint{2.550476in}{1.700785in}}{\pgfqpoint{2.558712in}{1.700785in}}%
\pgfpathclose%
\pgfusepath{stroke,fill}%
\end{pgfscope}%
\begin{pgfscope}%
\pgfpathrectangle{\pgfqpoint{0.100000in}{0.220728in}}{\pgfqpoint{3.696000in}{3.696000in}}%
\pgfusepath{clip}%
\pgfsetbuttcap%
\pgfsetroundjoin%
\definecolor{currentfill}{rgb}{0.121569,0.466667,0.705882}%
\pgfsetfillcolor{currentfill}%
\pgfsetfillopacity{0.937627}%
\pgfsetlinewidth{1.003750pt}%
\definecolor{currentstroke}{rgb}{0.121569,0.466667,0.705882}%
\pgfsetstrokecolor{currentstroke}%
\pgfsetstrokeopacity{0.937627}%
\pgfsetdash{}{0pt}%
\pgfpathmoveto{\pgfqpoint{2.557483in}{1.699201in}}%
\pgfpathcurveto{\pgfqpoint{2.565720in}{1.699201in}}{\pgfqpoint{2.573620in}{1.702473in}}{\pgfqpoint{2.579444in}{1.708297in}}%
\pgfpathcurveto{\pgfqpoint{2.585268in}{1.714121in}}{\pgfqpoint{2.588540in}{1.722021in}}{\pgfqpoint{2.588540in}{1.730257in}}%
\pgfpathcurveto{\pgfqpoint{2.588540in}{1.738494in}}{\pgfqpoint{2.585268in}{1.746394in}}{\pgfqpoint{2.579444in}{1.752218in}}%
\pgfpathcurveto{\pgfqpoint{2.573620in}{1.758042in}}{\pgfqpoint{2.565720in}{1.761314in}}{\pgfqpoint{2.557483in}{1.761314in}}%
\pgfpathcurveto{\pgfqpoint{2.549247in}{1.761314in}}{\pgfqpoint{2.541347in}{1.758042in}}{\pgfqpoint{2.535523in}{1.752218in}}%
\pgfpathcurveto{\pgfqpoint{2.529699in}{1.746394in}}{\pgfqpoint{2.526427in}{1.738494in}}{\pgfqpoint{2.526427in}{1.730257in}}%
\pgfpathcurveto{\pgfqpoint{2.526427in}{1.722021in}}{\pgfqpoint{2.529699in}{1.714121in}}{\pgfqpoint{2.535523in}{1.708297in}}%
\pgfpathcurveto{\pgfqpoint{2.541347in}{1.702473in}}{\pgfqpoint{2.549247in}{1.699201in}}{\pgfqpoint{2.557483in}{1.699201in}}%
\pgfpathclose%
\pgfusepath{stroke,fill}%
\end{pgfscope}%
\begin{pgfscope}%
\pgfpathrectangle{\pgfqpoint{0.100000in}{0.220728in}}{\pgfqpoint{3.696000in}{3.696000in}}%
\pgfusepath{clip}%
\pgfsetbuttcap%
\pgfsetroundjoin%
\definecolor{currentfill}{rgb}{0.121569,0.466667,0.705882}%
\pgfsetfillcolor{currentfill}%
\pgfsetfillopacity{0.937828}%
\pgfsetlinewidth{1.003750pt}%
\definecolor{currentstroke}{rgb}{0.121569,0.466667,0.705882}%
\pgfsetstrokecolor{currentstroke}%
\pgfsetstrokeopacity{0.937828}%
\pgfsetdash{}{0pt}%
\pgfpathmoveto{\pgfqpoint{2.556924in}{1.698216in}}%
\pgfpathcurveto{\pgfqpoint{2.565160in}{1.698216in}}{\pgfqpoint{2.573060in}{1.701488in}}{\pgfqpoint{2.578884in}{1.707312in}}%
\pgfpathcurveto{\pgfqpoint{2.584708in}{1.713136in}}{\pgfqpoint{2.587980in}{1.721036in}}{\pgfqpoint{2.587980in}{1.729272in}}%
\pgfpathcurveto{\pgfqpoint{2.587980in}{1.737509in}}{\pgfqpoint{2.584708in}{1.745409in}}{\pgfqpoint{2.578884in}{1.751233in}}%
\pgfpathcurveto{\pgfqpoint{2.573060in}{1.757057in}}{\pgfqpoint{2.565160in}{1.760329in}}{\pgfqpoint{2.556924in}{1.760329in}}%
\pgfpathcurveto{\pgfqpoint{2.548687in}{1.760329in}}{\pgfqpoint{2.540787in}{1.757057in}}{\pgfqpoint{2.534963in}{1.751233in}}%
\pgfpathcurveto{\pgfqpoint{2.529140in}{1.745409in}}{\pgfqpoint{2.525867in}{1.737509in}}{\pgfqpoint{2.525867in}{1.729272in}}%
\pgfpathcurveto{\pgfqpoint{2.525867in}{1.721036in}}{\pgfqpoint{2.529140in}{1.713136in}}{\pgfqpoint{2.534963in}{1.707312in}}%
\pgfpathcurveto{\pgfqpoint{2.540787in}{1.701488in}}{\pgfqpoint{2.548687in}{1.698216in}}{\pgfqpoint{2.556924in}{1.698216in}}%
\pgfpathclose%
\pgfusepath{stroke,fill}%
\end{pgfscope}%
\begin{pgfscope}%
\pgfpathrectangle{\pgfqpoint{0.100000in}{0.220728in}}{\pgfqpoint{3.696000in}{3.696000in}}%
\pgfusepath{clip}%
\pgfsetbuttcap%
\pgfsetroundjoin%
\definecolor{currentfill}{rgb}{0.121569,0.466667,0.705882}%
\pgfsetfillcolor{currentfill}%
\pgfsetfillopacity{0.938304}%
\pgfsetlinewidth{1.003750pt}%
\definecolor{currentstroke}{rgb}{0.121569,0.466667,0.705882}%
\pgfsetstrokecolor{currentstroke}%
\pgfsetstrokeopacity{0.938304}%
\pgfsetdash{}{0pt}%
\pgfpathmoveto{\pgfqpoint{2.054475in}{1.576800in}}%
\pgfpathcurveto{\pgfqpoint{2.062711in}{1.576800in}}{\pgfqpoint{2.070611in}{1.580072in}}{\pgfqpoint{2.076435in}{1.585896in}}%
\pgfpathcurveto{\pgfqpoint{2.082259in}{1.591720in}}{\pgfqpoint{2.085531in}{1.599620in}}{\pgfqpoint{2.085531in}{1.607856in}}%
\pgfpathcurveto{\pgfqpoint{2.085531in}{1.616093in}}{\pgfqpoint{2.082259in}{1.623993in}}{\pgfqpoint{2.076435in}{1.629817in}}%
\pgfpathcurveto{\pgfqpoint{2.070611in}{1.635641in}}{\pgfqpoint{2.062711in}{1.638913in}}{\pgfqpoint{2.054475in}{1.638913in}}%
\pgfpathcurveto{\pgfqpoint{2.046238in}{1.638913in}}{\pgfqpoint{2.038338in}{1.635641in}}{\pgfqpoint{2.032515in}{1.629817in}}%
\pgfpathcurveto{\pgfqpoint{2.026691in}{1.623993in}}{\pgfqpoint{2.023418in}{1.616093in}}{\pgfqpoint{2.023418in}{1.607856in}}%
\pgfpathcurveto{\pgfqpoint{2.023418in}{1.599620in}}{\pgfqpoint{2.026691in}{1.591720in}}{\pgfqpoint{2.032515in}{1.585896in}}%
\pgfpathcurveto{\pgfqpoint{2.038338in}{1.580072in}}{\pgfqpoint{2.046238in}{1.576800in}}{\pgfqpoint{2.054475in}{1.576800in}}%
\pgfpathclose%
\pgfusepath{stroke,fill}%
\end{pgfscope}%
\begin{pgfscope}%
\pgfpathrectangle{\pgfqpoint{0.100000in}{0.220728in}}{\pgfqpoint{3.696000in}{3.696000in}}%
\pgfusepath{clip}%
\pgfsetbuttcap%
\pgfsetroundjoin%
\definecolor{currentfill}{rgb}{0.121569,0.466667,0.705882}%
\pgfsetfillcolor{currentfill}%
\pgfsetfillopacity{0.939105}%
\pgfsetlinewidth{1.003750pt}%
\definecolor{currentstroke}{rgb}{0.121569,0.466667,0.705882}%
\pgfsetstrokecolor{currentstroke}%
\pgfsetstrokeopacity{0.939105}%
\pgfsetdash{}{0pt}%
\pgfpathmoveto{\pgfqpoint{2.552428in}{1.692273in}}%
\pgfpathcurveto{\pgfqpoint{2.560664in}{1.692273in}}{\pgfqpoint{2.568564in}{1.695545in}}{\pgfqpoint{2.574388in}{1.701369in}}%
\pgfpathcurveto{\pgfqpoint{2.580212in}{1.707193in}}{\pgfqpoint{2.583485in}{1.715093in}}{\pgfqpoint{2.583485in}{1.723329in}}%
\pgfpathcurveto{\pgfqpoint{2.583485in}{1.731566in}}{\pgfqpoint{2.580212in}{1.739466in}}{\pgfqpoint{2.574388in}{1.745290in}}%
\pgfpathcurveto{\pgfqpoint{2.568564in}{1.751113in}}{\pgfqpoint{2.560664in}{1.754386in}}{\pgfqpoint{2.552428in}{1.754386in}}%
\pgfpathcurveto{\pgfqpoint{2.544192in}{1.754386in}}{\pgfqpoint{2.536292in}{1.751113in}}{\pgfqpoint{2.530468in}{1.745290in}}%
\pgfpathcurveto{\pgfqpoint{2.524644in}{1.739466in}}{\pgfqpoint{2.521372in}{1.731566in}}{\pgfqpoint{2.521372in}{1.723329in}}%
\pgfpathcurveto{\pgfqpoint{2.521372in}{1.715093in}}{\pgfqpoint{2.524644in}{1.707193in}}{\pgfqpoint{2.530468in}{1.701369in}}%
\pgfpathcurveto{\pgfqpoint{2.536292in}{1.695545in}}{\pgfqpoint{2.544192in}{1.692273in}}{\pgfqpoint{2.552428in}{1.692273in}}%
\pgfpathclose%
\pgfusepath{stroke,fill}%
\end{pgfscope}%
\begin{pgfscope}%
\pgfpathrectangle{\pgfqpoint{0.100000in}{0.220728in}}{\pgfqpoint{3.696000in}{3.696000in}}%
\pgfusepath{clip}%
\pgfsetbuttcap%
\pgfsetroundjoin%
\definecolor{currentfill}{rgb}{0.121569,0.466667,0.705882}%
\pgfsetfillcolor{currentfill}%
\pgfsetfillopacity{0.940805}%
\pgfsetlinewidth{1.003750pt}%
\definecolor{currentstroke}{rgb}{0.121569,0.466667,0.705882}%
\pgfsetstrokecolor{currentstroke}%
\pgfsetstrokeopacity{0.940805}%
\pgfsetdash{}{0pt}%
\pgfpathmoveto{\pgfqpoint{2.546587in}{1.682010in}}%
\pgfpathcurveto{\pgfqpoint{2.554823in}{1.682010in}}{\pgfqpoint{2.562723in}{1.685282in}}{\pgfqpoint{2.568547in}{1.691106in}}%
\pgfpathcurveto{\pgfqpoint{2.574371in}{1.696930in}}{\pgfqpoint{2.577643in}{1.704830in}}{\pgfqpoint{2.577643in}{1.713066in}}%
\pgfpathcurveto{\pgfqpoint{2.577643in}{1.721302in}}{\pgfqpoint{2.574371in}{1.729202in}}{\pgfqpoint{2.568547in}{1.735026in}}%
\pgfpathcurveto{\pgfqpoint{2.562723in}{1.740850in}}{\pgfqpoint{2.554823in}{1.744123in}}{\pgfqpoint{2.546587in}{1.744123in}}%
\pgfpathcurveto{\pgfqpoint{2.538350in}{1.744123in}}{\pgfqpoint{2.530450in}{1.740850in}}{\pgfqpoint{2.524626in}{1.735026in}}%
\pgfpathcurveto{\pgfqpoint{2.518802in}{1.729202in}}{\pgfqpoint{2.515530in}{1.721302in}}{\pgfqpoint{2.515530in}{1.713066in}}%
\pgfpathcurveto{\pgfqpoint{2.515530in}{1.704830in}}{\pgfqpoint{2.518802in}{1.696930in}}{\pgfqpoint{2.524626in}{1.691106in}}%
\pgfpathcurveto{\pgfqpoint{2.530450in}{1.685282in}}{\pgfqpoint{2.538350in}{1.682010in}}{\pgfqpoint{2.546587in}{1.682010in}}%
\pgfpathclose%
\pgfusepath{stroke,fill}%
\end{pgfscope}%
\begin{pgfscope}%
\pgfpathrectangle{\pgfqpoint{0.100000in}{0.220728in}}{\pgfqpoint{3.696000in}{3.696000in}}%
\pgfusepath{clip}%
\pgfsetbuttcap%
\pgfsetroundjoin%
\definecolor{currentfill}{rgb}{0.121569,0.466667,0.705882}%
\pgfsetfillcolor{currentfill}%
\pgfsetfillopacity{0.943187}%
\pgfsetlinewidth{1.003750pt}%
\definecolor{currentstroke}{rgb}{0.121569,0.466667,0.705882}%
\pgfsetstrokecolor{currentstroke}%
\pgfsetstrokeopacity{0.943187}%
\pgfsetdash{}{0pt}%
\pgfpathmoveto{\pgfqpoint{2.078117in}{1.564670in}}%
\pgfpathcurveto{\pgfqpoint{2.086353in}{1.564670in}}{\pgfqpoint{2.094253in}{1.567943in}}{\pgfqpoint{2.100077in}{1.573767in}}%
\pgfpathcurveto{\pgfqpoint{2.105901in}{1.579591in}}{\pgfqpoint{2.109173in}{1.587491in}}{\pgfqpoint{2.109173in}{1.595727in}}%
\pgfpathcurveto{\pgfqpoint{2.109173in}{1.603963in}}{\pgfqpoint{2.105901in}{1.611863in}}{\pgfqpoint{2.100077in}{1.617687in}}%
\pgfpathcurveto{\pgfqpoint{2.094253in}{1.623511in}}{\pgfqpoint{2.086353in}{1.626783in}}{\pgfqpoint{2.078117in}{1.626783in}}%
\pgfpathcurveto{\pgfqpoint{2.069880in}{1.626783in}}{\pgfqpoint{2.061980in}{1.623511in}}{\pgfqpoint{2.056156in}{1.617687in}}%
\pgfpathcurveto{\pgfqpoint{2.050333in}{1.611863in}}{\pgfqpoint{2.047060in}{1.603963in}}{\pgfqpoint{2.047060in}{1.595727in}}%
\pgfpathcurveto{\pgfqpoint{2.047060in}{1.587491in}}{\pgfqpoint{2.050333in}{1.579591in}}{\pgfqpoint{2.056156in}{1.573767in}}%
\pgfpathcurveto{\pgfqpoint{2.061980in}{1.567943in}}{\pgfqpoint{2.069880in}{1.564670in}}{\pgfqpoint{2.078117in}{1.564670in}}%
\pgfpathclose%
\pgfusepath{stroke,fill}%
\end{pgfscope}%
\begin{pgfscope}%
\pgfpathrectangle{\pgfqpoint{0.100000in}{0.220728in}}{\pgfqpoint{3.696000in}{3.696000in}}%
\pgfusepath{clip}%
\pgfsetbuttcap%
\pgfsetroundjoin%
\definecolor{currentfill}{rgb}{0.121569,0.466667,0.705882}%
\pgfsetfillcolor{currentfill}%
\pgfsetfillopacity{0.943458}%
\pgfsetlinewidth{1.003750pt}%
\definecolor{currentstroke}{rgb}{0.121569,0.466667,0.705882}%
\pgfsetstrokecolor{currentstroke}%
\pgfsetstrokeopacity{0.943458}%
\pgfsetdash{}{0pt}%
\pgfpathmoveto{\pgfqpoint{2.537491in}{1.669683in}}%
\pgfpathcurveto{\pgfqpoint{2.545727in}{1.669683in}}{\pgfqpoint{2.553627in}{1.672955in}}{\pgfqpoint{2.559451in}{1.678779in}}%
\pgfpathcurveto{\pgfqpoint{2.565275in}{1.684603in}}{\pgfqpoint{2.568547in}{1.692503in}}{\pgfqpoint{2.568547in}{1.700739in}}%
\pgfpathcurveto{\pgfqpoint{2.568547in}{1.708975in}}{\pgfqpoint{2.565275in}{1.716875in}}{\pgfqpoint{2.559451in}{1.722699in}}%
\pgfpathcurveto{\pgfqpoint{2.553627in}{1.728523in}}{\pgfqpoint{2.545727in}{1.731796in}}{\pgfqpoint{2.537491in}{1.731796in}}%
\pgfpathcurveto{\pgfqpoint{2.529254in}{1.731796in}}{\pgfqpoint{2.521354in}{1.728523in}}{\pgfqpoint{2.515530in}{1.722699in}}%
\pgfpathcurveto{\pgfqpoint{2.509706in}{1.716875in}}{\pgfqpoint{2.506434in}{1.708975in}}{\pgfqpoint{2.506434in}{1.700739in}}%
\pgfpathcurveto{\pgfqpoint{2.506434in}{1.692503in}}{\pgfqpoint{2.509706in}{1.684603in}}{\pgfqpoint{2.515530in}{1.678779in}}%
\pgfpathcurveto{\pgfqpoint{2.521354in}{1.672955in}}{\pgfqpoint{2.529254in}{1.669683in}}{\pgfqpoint{2.537491in}{1.669683in}}%
\pgfpathclose%
\pgfusepath{stroke,fill}%
\end{pgfscope}%
\begin{pgfscope}%
\pgfpathrectangle{\pgfqpoint{0.100000in}{0.220728in}}{\pgfqpoint{3.696000in}{3.696000in}}%
\pgfusepath{clip}%
\pgfsetbuttcap%
\pgfsetroundjoin%
\definecolor{currentfill}{rgb}{0.121569,0.466667,0.705882}%
\pgfsetfillcolor{currentfill}%
\pgfsetfillopacity{0.945005}%
\pgfsetlinewidth{1.003750pt}%
\definecolor{currentstroke}{rgb}{0.121569,0.466667,0.705882}%
\pgfsetstrokecolor{currentstroke}%
\pgfsetstrokeopacity{0.945005}%
\pgfsetdash{}{0pt}%
\pgfpathmoveto{\pgfqpoint{2.533269in}{1.662196in}}%
\pgfpathcurveto{\pgfqpoint{2.541506in}{1.662196in}}{\pgfqpoint{2.549406in}{1.665468in}}{\pgfqpoint{2.555230in}{1.671292in}}%
\pgfpathcurveto{\pgfqpoint{2.561054in}{1.677116in}}{\pgfqpoint{2.564326in}{1.685016in}}{\pgfqpoint{2.564326in}{1.693252in}}%
\pgfpathcurveto{\pgfqpoint{2.564326in}{1.701489in}}{\pgfqpoint{2.561054in}{1.709389in}}{\pgfqpoint{2.555230in}{1.715213in}}%
\pgfpathcurveto{\pgfqpoint{2.549406in}{1.721037in}}{\pgfqpoint{2.541506in}{1.724309in}}{\pgfqpoint{2.533269in}{1.724309in}}%
\pgfpathcurveto{\pgfqpoint{2.525033in}{1.724309in}}{\pgfqpoint{2.517133in}{1.721037in}}{\pgfqpoint{2.511309in}{1.715213in}}%
\pgfpathcurveto{\pgfqpoint{2.505485in}{1.709389in}}{\pgfqpoint{2.502213in}{1.701489in}}{\pgfqpoint{2.502213in}{1.693252in}}%
\pgfpathcurveto{\pgfqpoint{2.502213in}{1.685016in}}{\pgfqpoint{2.505485in}{1.677116in}}{\pgfqpoint{2.511309in}{1.671292in}}%
\pgfpathcurveto{\pgfqpoint{2.517133in}{1.665468in}}{\pgfqpoint{2.525033in}{1.662196in}}{\pgfqpoint{2.533269in}{1.662196in}}%
\pgfpathclose%
\pgfusepath{stroke,fill}%
\end{pgfscope}%
\begin{pgfscope}%
\pgfpathrectangle{\pgfqpoint{0.100000in}{0.220728in}}{\pgfqpoint{3.696000in}{3.696000in}}%
\pgfusepath{clip}%
\pgfsetbuttcap%
\pgfsetroundjoin%
\definecolor{currentfill}{rgb}{0.121569,0.466667,0.705882}%
\pgfsetfillcolor{currentfill}%
\pgfsetfillopacity{0.945846}%
\pgfsetlinewidth{1.003750pt}%
\definecolor{currentstroke}{rgb}{0.121569,0.466667,0.705882}%
\pgfsetstrokecolor{currentstroke}%
\pgfsetstrokeopacity{0.945846}%
\pgfsetdash{}{0pt}%
\pgfpathmoveto{\pgfqpoint{2.530713in}{1.658370in}}%
\pgfpathcurveto{\pgfqpoint{2.538950in}{1.658370in}}{\pgfqpoint{2.546850in}{1.661643in}}{\pgfqpoint{2.552674in}{1.667467in}}%
\pgfpathcurveto{\pgfqpoint{2.558497in}{1.673291in}}{\pgfqpoint{2.561770in}{1.681191in}}{\pgfqpoint{2.561770in}{1.689427in}}%
\pgfpathcurveto{\pgfqpoint{2.561770in}{1.697663in}}{\pgfqpoint{2.558497in}{1.705563in}}{\pgfqpoint{2.552674in}{1.711387in}}%
\pgfpathcurveto{\pgfqpoint{2.546850in}{1.717211in}}{\pgfqpoint{2.538950in}{1.720483in}}{\pgfqpoint{2.530713in}{1.720483in}}%
\pgfpathcurveto{\pgfqpoint{2.522477in}{1.720483in}}{\pgfqpoint{2.514577in}{1.717211in}}{\pgfqpoint{2.508753in}{1.711387in}}%
\pgfpathcurveto{\pgfqpoint{2.502929in}{1.705563in}}{\pgfqpoint{2.499657in}{1.697663in}}{\pgfqpoint{2.499657in}{1.689427in}}%
\pgfpathcurveto{\pgfqpoint{2.499657in}{1.681191in}}{\pgfqpoint{2.502929in}{1.673291in}}{\pgfqpoint{2.508753in}{1.667467in}}%
\pgfpathcurveto{\pgfqpoint{2.514577in}{1.661643in}}{\pgfqpoint{2.522477in}{1.658370in}}{\pgfqpoint{2.530713in}{1.658370in}}%
\pgfpathclose%
\pgfusepath{stroke,fill}%
\end{pgfscope}%
\begin{pgfscope}%
\pgfpathrectangle{\pgfqpoint{0.100000in}{0.220728in}}{\pgfqpoint{3.696000in}{3.696000in}}%
\pgfusepath{clip}%
\pgfsetbuttcap%
\pgfsetroundjoin%
\definecolor{currentfill}{rgb}{0.121569,0.466667,0.705882}%
\pgfsetfillcolor{currentfill}%
\pgfsetfillopacity{0.946283}%
\pgfsetlinewidth{1.003750pt}%
\definecolor{currentstroke}{rgb}{0.121569,0.466667,0.705882}%
\pgfsetstrokecolor{currentstroke}%
\pgfsetstrokeopacity{0.946283}%
\pgfsetdash{}{0pt}%
\pgfpathmoveto{\pgfqpoint{2.529200in}{1.656283in}}%
\pgfpathcurveto{\pgfqpoint{2.537437in}{1.656283in}}{\pgfqpoint{2.545337in}{1.659555in}}{\pgfqpoint{2.551161in}{1.665379in}}%
\pgfpathcurveto{\pgfqpoint{2.556985in}{1.671203in}}{\pgfqpoint{2.560257in}{1.679103in}}{\pgfqpoint{2.560257in}{1.687339in}}%
\pgfpathcurveto{\pgfqpoint{2.560257in}{1.695575in}}{\pgfqpoint{2.556985in}{1.703475in}}{\pgfqpoint{2.551161in}{1.709299in}}%
\pgfpathcurveto{\pgfqpoint{2.545337in}{1.715123in}}{\pgfqpoint{2.537437in}{1.718396in}}{\pgfqpoint{2.529200in}{1.718396in}}%
\pgfpathcurveto{\pgfqpoint{2.520964in}{1.718396in}}{\pgfqpoint{2.513064in}{1.715123in}}{\pgfqpoint{2.507240in}{1.709299in}}%
\pgfpathcurveto{\pgfqpoint{2.501416in}{1.703475in}}{\pgfqpoint{2.498144in}{1.695575in}}{\pgfqpoint{2.498144in}{1.687339in}}%
\pgfpathcurveto{\pgfqpoint{2.498144in}{1.679103in}}{\pgfqpoint{2.501416in}{1.671203in}}{\pgfqpoint{2.507240in}{1.665379in}}%
\pgfpathcurveto{\pgfqpoint{2.513064in}{1.659555in}}{\pgfqpoint{2.520964in}{1.656283in}}{\pgfqpoint{2.529200in}{1.656283in}}%
\pgfpathclose%
\pgfusepath{stroke,fill}%
\end{pgfscope}%
\begin{pgfscope}%
\pgfpathrectangle{\pgfqpoint{0.100000in}{0.220728in}}{\pgfqpoint{3.696000in}{3.696000in}}%
\pgfusepath{clip}%
\pgfsetbuttcap%
\pgfsetroundjoin%
\definecolor{currentfill}{rgb}{0.121569,0.466667,0.705882}%
\pgfsetfillcolor{currentfill}%
\pgfsetfillopacity{0.946547}%
\pgfsetlinewidth{1.003750pt}%
\definecolor{currentstroke}{rgb}{0.121569,0.466667,0.705882}%
\pgfsetstrokecolor{currentstroke}%
\pgfsetstrokeopacity{0.946547}%
\pgfsetdash{}{0pt}%
\pgfpathmoveto{\pgfqpoint{2.095235in}{1.557576in}}%
\pgfpathcurveto{\pgfqpoint{2.103472in}{1.557576in}}{\pgfqpoint{2.111372in}{1.560848in}}{\pgfqpoint{2.117196in}{1.566672in}}%
\pgfpathcurveto{\pgfqpoint{2.123020in}{1.572496in}}{\pgfqpoint{2.126292in}{1.580396in}}{\pgfqpoint{2.126292in}{1.588633in}}%
\pgfpathcurveto{\pgfqpoint{2.126292in}{1.596869in}}{\pgfqpoint{2.123020in}{1.604769in}}{\pgfqpoint{2.117196in}{1.610593in}}%
\pgfpathcurveto{\pgfqpoint{2.111372in}{1.616417in}}{\pgfqpoint{2.103472in}{1.619689in}}{\pgfqpoint{2.095235in}{1.619689in}}%
\pgfpathcurveto{\pgfqpoint{2.086999in}{1.619689in}}{\pgfqpoint{2.079099in}{1.616417in}}{\pgfqpoint{2.073275in}{1.610593in}}%
\pgfpathcurveto{\pgfqpoint{2.067451in}{1.604769in}}{\pgfqpoint{2.064179in}{1.596869in}}{\pgfqpoint{2.064179in}{1.588633in}}%
\pgfpathcurveto{\pgfqpoint{2.064179in}{1.580396in}}{\pgfqpoint{2.067451in}{1.572496in}}{\pgfqpoint{2.073275in}{1.566672in}}%
\pgfpathcurveto{\pgfqpoint{2.079099in}{1.560848in}}{\pgfqpoint{2.086999in}{1.557576in}}{\pgfqpoint{2.095235in}{1.557576in}}%
\pgfpathclose%
\pgfusepath{stroke,fill}%
\end{pgfscope}%
\begin{pgfscope}%
\pgfpathrectangle{\pgfqpoint{0.100000in}{0.220728in}}{\pgfqpoint{3.696000in}{3.696000in}}%
\pgfusepath{clip}%
\pgfsetbuttcap%
\pgfsetroundjoin%
\definecolor{currentfill}{rgb}{0.121569,0.466667,0.705882}%
\pgfsetfillcolor{currentfill}%
\pgfsetfillopacity{0.946551}%
\pgfsetlinewidth{1.003750pt}%
\definecolor{currentstroke}{rgb}{0.121569,0.466667,0.705882}%
\pgfsetstrokecolor{currentstroke}%
\pgfsetstrokeopacity{0.946551}%
\pgfsetdash{}{0pt}%
\pgfpathmoveto{\pgfqpoint{2.528472in}{1.655125in}}%
\pgfpathcurveto{\pgfqpoint{2.536708in}{1.655125in}}{\pgfqpoint{2.544608in}{1.658398in}}{\pgfqpoint{2.550432in}{1.664222in}}%
\pgfpathcurveto{\pgfqpoint{2.556256in}{1.670046in}}{\pgfqpoint{2.559528in}{1.677946in}}{\pgfqpoint{2.559528in}{1.686182in}}%
\pgfpathcurveto{\pgfqpoint{2.559528in}{1.694418in}}{\pgfqpoint{2.556256in}{1.702318in}}{\pgfqpoint{2.550432in}{1.708142in}}%
\pgfpathcurveto{\pgfqpoint{2.544608in}{1.713966in}}{\pgfqpoint{2.536708in}{1.717238in}}{\pgfqpoint{2.528472in}{1.717238in}}%
\pgfpathcurveto{\pgfqpoint{2.520236in}{1.717238in}}{\pgfqpoint{2.512335in}{1.713966in}}{\pgfqpoint{2.506512in}{1.708142in}}%
\pgfpathcurveto{\pgfqpoint{2.500688in}{1.702318in}}{\pgfqpoint{2.497415in}{1.694418in}}{\pgfqpoint{2.497415in}{1.686182in}}%
\pgfpathcurveto{\pgfqpoint{2.497415in}{1.677946in}}{\pgfqpoint{2.500688in}{1.670046in}}{\pgfqpoint{2.506512in}{1.664222in}}%
\pgfpathcurveto{\pgfqpoint{2.512335in}{1.658398in}}{\pgfqpoint{2.520236in}{1.655125in}}{\pgfqpoint{2.528472in}{1.655125in}}%
\pgfpathclose%
\pgfusepath{stroke,fill}%
\end{pgfscope}%
\begin{pgfscope}%
\pgfpathrectangle{\pgfqpoint{0.100000in}{0.220728in}}{\pgfqpoint{3.696000in}{3.696000in}}%
\pgfusepath{clip}%
\pgfsetbuttcap%
\pgfsetroundjoin%
\definecolor{currentfill}{rgb}{0.121569,0.466667,0.705882}%
\pgfsetfillcolor{currentfill}%
\pgfsetfillopacity{0.946678}%
\pgfsetlinewidth{1.003750pt}%
\definecolor{currentstroke}{rgb}{0.121569,0.466667,0.705882}%
\pgfsetstrokecolor{currentstroke}%
\pgfsetstrokeopacity{0.946678}%
\pgfsetdash{}{0pt}%
\pgfpathmoveto{\pgfqpoint{2.527992in}{1.654502in}}%
\pgfpathcurveto{\pgfqpoint{2.536228in}{1.654502in}}{\pgfqpoint{2.544128in}{1.657774in}}{\pgfqpoint{2.549952in}{1.663598in}}%
\pgfpathcurveto{\pgfqpoint{2.555776in}{1.669422in}}{\pgfqpoint{2.559049in}{1.677322in}}{\pgfqpoint{2.559049in}{1.685558in}}%
\pgfpathcurveto{\pgfqpoint{2.559049in}{1.693794in}}{\pgfqpoint{2.555776in}{1.701694in}}{\pgfqpoint{2.549952in}{1.707518in}}%
\pgfpathcurveto{\pgfqpoint{2.544128in}{1.713342in}}{\pgfqpoint{2.536228in}{1.716615in}}{\pgfqpoint{2.527992in}{1.716615in}}%
\pgfpathcurveto{\pgfqpoint{2.519756in}{1.716615in}}{\pgfqpoint{2.511856in}{1.713342in}}{\pgfqpoint{2.506032in}{1.707518in}}%
\pgfpathcurveto{\pgfqpoint{2.500208in}{1.701694in}}{\pgfqpoint{2.496936in}{1.693794in}}{\pgfqpoint{2.496936in}{1.685558in}}%
\pgfpathcurveto{\pgfqpoint{2.496936in}{1.677322in}}{\pgfqpoint{2.500208in}{1.669422in}}{\pgfqpoint{2.506032in}{1.663598in}}%
\pgfpathcurveto{\pgfqpoint{2.511856in}{1.657774in}}{\pgfqpoint{2.519756in}{1.654502in}}{\pgfqpoint{2.527992in}{1.654502in}}%
\pgfpathclose%
\pgfusepath{stroke,fill}%
\end{pgfscope}%
\begin{pgfscope}%
\pgfpathrectangle{\pgfqpoint{0.100000in}{0.220728in}}{\pgfqpoint{3.696000in}{3.696000in}}%
\pgfusepath{clip}%
\pgfsetbuttcap%
\pgfsetroundjoin%
\definecolor{currentfill}{rgb}{0.121569,0.466667,0.705882}%
\pgfsetfillcolor{currentfill}%
\pgfsetfillopacity{0.946750}%
\pgfsetlinewidth{1.003750pt}%
\definecolor{currentstroke}{rgb}{0.121569,0.466667,0.705882}%
\pgfsetstrokecolor{currentstroke}%
\pgfsetstrokeopacity{0.946750}%
\pgfsetdash{}{0pt}%
\pgfpathmoveto{\pgfqpoint{2.527770in}{1.654112in}}%
\pgfpathcurveto{\pgfqpoint{2.536007in}{1.654112in}}{\pgfqpoint{2.543907in}{1.657384in}}{\pgfqpoint{2.549730in}{1.663208in}}%
\pgfpathcurveto{\pgfqpoint{2.555554in}{1.669032in}}{\pgfqpoint{2.558827in}{1.676932in}}{\pgfqpoint{2.558827in}{1.685168in}}%
\pgfpathcurveto{\pgfqpoint{2.558827in}{1.693405in}}{\pgfqpoint{2.555554in}{1.701305in}}{\pgfqpoint{2.549730in}{1.707129in}}%
\pgfpathcurveto{\pgfqpoint{2.543907in}{1.712953in}}{\pgfqpoint{2.536007in}{1.716225in}}{\pgfqpoint{2.527770in}{1.716225in}}%
\pgfpathcurveto{\pgfqpoint{2.519534in}{1.716225in}}{\pgfqpoint{2.511634in}{1.712953in}}{\pgfqpoint{2.505810in}{1.707129in}}%
\pgfpathcurveto{\pgfqpoint{2.499986in}{1.701305in}}{\pgfqpoint{2.496714in}{1.693405in}}{\pgfqpoint{2.496714in}{1.685168in}}%
\pgfpathcurveto{\pgfqpoint{2.496714in}{1.676932in}}{\pgfqpoint{2.499986in}{1.669032in}}{\pgfqpoint{2.505810in}{1.663208in}}%
\pgfpathcurveto{\pgfqpoint{2.511634in}{1.657384in}}{\pgfqpoint{2.519534in}{1.654112in}}{\pgfqpoint{2.527770in}{1.654112in}}%
\pgfpathclose%
\pgfusepath{stroke,fill}%
\end{pgfscope}%
\begin{pgfscope}%
\pgfpathrectangle{\pgfqpoint{0.100000in}{0.220728in}}{\pgfqpoint{3.696000in}{3.696000in}}%
\pgfusepath{clip}%
\pgfsetbuttcap%
\pgfsetroundjoin%
\definecolor{currentfill}{rgb}{0.121569,0.466667,0.705882}%
\pgfsetfillcolor{currentfill}%
\pgfsetfillopacity{0.947575}%
\pgfsetlinewidth{1.003750pt}%
\definecolor{currentstroke}{rgb}{0.121569,0.466667,0.705882}%
\pgfsetstrokecolor{currentstroke}%
\pgfsetstrokeopacity{0.947575}%
\pgfsetdash{}{0pt}%
\pgfpathmoveto{\pgfqpoint{2.525322in}{1.651009in}}%
\pgfpathcurveto{\pgfqpoint{2.533558in}{1.651009in}}{\pgfqpoint{2.541458in}{1.654282in}}{\pgfqpoint{2.547282in}{1.660106in}}%
\pgfpathcurveto{\pgfqpoint{2.553106in}{1.665929in}}{\pgfqpoint{2.556378in}{1.673829in}}{\pgfqpoint{2.556378in}{1.682066in}}%
\pgfpathcurveto{\pgfqpoint{2.556378in}{1.690302in}}{\pgfqpoint{2.553106in}{1.698202in}}{\pgfqpoint{2.547282in}{1.704026in}}%
\pgfpathcurveto{\pgfqpoint{2.541458in}{1.709850in}}{\pgfqpoint{2.533558in}{1.713122in}}{\pgfqpoint{2.525322in}{1.713122in}}%
\pgfpathcurveto{\pgfqpoint{2.517086in}{1.713122in}}{\pgfqpoint{2.509185in}{1.709850in}}{\pgfqpoint{2.503362in}{1.704026in}}%
\pgfpathcurveto{\pgfqpoint{2.497538in}{1.698202in}}{\pgfqpoint{2.494265in}{1.690302in}}{\pgfqpoint{2.494265in}{1.682066in}}%
\pgfpathcurveto{\pgfqpoint{2.494265in}{1.673829in}}{\pgfqpoint{2.497538in}{1.665929in}}{\pgfqpoint{2.503362in}{1.660106in}}%
\pgfpathcurveto{\pgfqpoint{2.509185in}{1.654282in}}{\pgfqpoint{2.517086in}{1.651009in}}{\pgfqpoint{2.525322in}{1.651009in}}%
\pgfpathclose%
\pgfusepath{stroke,fill}%
\end{pgfscope}%
\begin{pgfscope}%
\pgfpathrectangle{\pgfqpoint{0.100000in}{0.220728in}}{\pgfqpoint{3.696000in}{3.696000in}}%
\pgfusepath{clip}%
\pgfsetbuttcap%
\pgfsetroundjoin%
\definecolor{currentfill}{rgb}{0.121569,0.466667,0.705882}%
\pgfsetfillcolor{currentfill}%
\pgfsetfillopacity{0.949110}%
\pgfsetlinewidth{1.003750pt}%
\definecolor{currentstroke}{rgb}{0.121569,0.466667,0.705882}%
\pgfsetstrokecolor{currentstroke}%
\pgfsetstrokeopacity{0.949110}%
\pgfsetdash{}{0pt}%
\pgfpathmoveto{\pgfqpoint{2.521607in}{1.643301in}}%
\pgfpathcurveto{\pgfqpoint{2.529843in}{1.643301in}}{\pgfqpoint{2.537743in}{1.646573in}}{\pgfqpoint{2.543567in}{1.652397in}}%
\pgfpathcurveto{\pgfqpoint{2.549391in}{1.658221in}}{\pgfqpoint{2.552663in}{1.666121in}}{\pgfqpoint{2.552663in}{1.674357in}}%
\pgfpathcurveto{\pgfqpoint{2.552663in}{1.682593in}}{\pgfqpoint{2.549391in}{1.690493in}}{\pgfqpoint{2.543567in}{1.696317in}}%
\pgfpathcurveto{\pgfqpoint{2.537743in}{1.702141in}}{\pgfqpoint{2.529843in}{1.705414in}}{\pgfqpoint{2.521607in}{1.705414in}}%
\pgfpathcurveto{\pgfqpoint{2.513371in}{1.705414in}}{\pgfqpoint{2.505471in}{1.702141in}}{\pgfqpoint{2.499647in}{1.696317in}}%
\pgfpathcurveto{\pgfqpoint{2.493823in}{1.690493in}}{\pgfqpoint{2.490550in}{1.682593in}}{\pgfqpoint{2.490550in}{1.674357in}}%
\pgfpathcurveto{\pgfqpoint{2.490550in}{1.666121in}}{\pgfqpoint{2.493823in}{1.658221in}}{\pgfqpoint{2.499647in}{1.652397in}}%
\pgfpathcurveto{\pgfqpoint{2.505471in}{1.646573in}}{\pgfqpoint{2.513371in}{1.643301in}}{\pgfqpoint{2.521607in}{1.643301in}}%
\pgfpathclose%
\pgfusepath{stroke,fill}%
\end{pgfscope}%
\begin{pgfscope}%
\pgfpathrectangle{\pgfqpoint{0.100000in}{0.220728in}}{\pgfqpoint{3.696000in}{3.696000in}}%
\pgfusepath{clip}%
\pgfsetbuttcap%
\pgfsetroundjoin%
\definecolor{currentfill}{rgb}{0.121569,0.466667,0.705882}%
\pgfsetfillcolor{currentfill}%
\pgfsetfillopacity{0.949179}%
\pgfsetlinewidth{1.003750pt}%
\definecolor{currentstroke}{rgb}{0.121569,0.466667,0.705882}%
\pgfsetstrokecolor{currentstroke}%
\pgfsetstrokeopacity{0.949179}%
\pgfsetdash{}{0pt}%
\pgfpathmoveto{\pgfqpoint{2.107860in}{1.551769in}}%
\pgfpathcurveto{\pgfqpoint{2.116097in}{1.551769in}}{\pgfqpoint{2.123997in}{1.555042in}}{\pgfqpoint{2.129821in}{1.560866in}}%
\pgfpathcurveto{\pgfqpoint{2.135644in}{1.566690in}}{\pgfqpoint{2.138917in}{1.574590in}}{\pgfqpoint{2.138917in}{1.582826in}}%
\pgfpathcurveto{\pgfqpoint{2.138917in}{1.591062in}}{\pgfqpoint{2.135644in}{1.598962in}}{\pgfqpoint{2.129821in}{1.604786in}}%
\pgfpathcurveto{\pgfqpoint{2.123997in}{1.610610in}}{\pgfqpoint{2.116097in}{1.613882in}}{\pgfqpoint{2.107860in}{1.613882in}}%
\pgfpathcurveto{\pgfqpoint{2.099624in}{1.613882in}}{\pgfqpoint{2.091724in}{1.610610in}}{\pgfqpoint{2.085900in}{1.604786in}}%
\pgfpathcurveto{\pgfqpoint{2.080076in}{1.598962in}}{\pgfqpoint{2.076804in}{1.591062in}}{\pgfqpoint{2.076804in}{1.582826in}}%
\pgfpathcurveto{\pgfqpoint{2.076804in}{1.574590in}}{\pgfqpoint{2.080076in}{1.566690in}}{\pgfqpoint{2.085900in}{1.560866in}}%
\pgfpathcurveto{\pgfqpoint{2.091724in}{1.555042in}}{\pgfqpoint{2.099624in}{1.551769in}}{\pgfqpoint{2.107860in}{1.551769in}}%
\pgfpathclose%
\pgfusepath{stroke,fill}%
\end{pgfscope}%
\begin{pgfscope}%
\pgfpathrectangle{\pgfqpoint{0.100000in}{0.220728in}}{\pgfqpoint{3.696000in}{3.696000in}}%
\pgfusepath{clip}%
\pgfsetbuttcap%
\pgfsetroundjoin%
\definecolor{currentfill}{rgb}{0.121569,0.466667,0.705882}%
\pgfsetfillcolor{currentfill}%
\pgfsetfillopacity{0.950921}%
\pgfsetlinewidth{1.003750pt}%
\definecolor{currentstroke}{rgb}{0.121569,0.466667,0.705882}%
\pgfsetstrokecolor{currentstroke}%
\pgfsetstrokeopacity{0.950921}%
\pgfsetdash{}{0pt}%
\pgfpathmoveto{\pgfqpoint{2.117611in}{1.549032in}}%
\pgfpathcurveto{\pgfqpoint{2.125847in}{1.549032in}}{\pgfqpoint{2.133748in}{1.552304in}}{\pgfqpoint{2.139571in}{1.558128in}}%
\pgfpathcurveto{\pgfqpoint{2.145395in}{1.563952in}}{\pgfqpoint{2.148668in}{1.571852in}}{\pgfqpoint{2.148668in}{1.580088in}}%
\pgfpathcurveto{\pgfqpoint{2.148668in}{1.588324in}}{\pgfqpoint{2.145395in}{1.596224in}}{\pgfqpoint{2.139571in}{1.602048in}}%
\pgfpathcurveto{\pgfqpoint{2.133748in}{1.607872in}}{\pgfqpoint{2.125847in}{1.611145in}}{\pgfqpoint{2.117611in}{1.611145in}}%
\pgfpathcurveto{\pgfqpoint{2.109375in}{1.611145in}}{\pgfqpoint{2.101475in}{1.607872in}}{\pgfqpoint{2.095651in}{1.602048in}}%
\pgfpathcurveto{\pgfqpoint{2.089827in}{1.596224in}}{\pgfqpoint{2.086555in}{1.588324in}}{\pgfqpoint{2.086555in}{1.580088in}}%
\pgfpathcurveto{\pgfqpoint{2.086555in}{1.571852in}}{\pgfqpoint{2.089827in}{1.563952in}}{\pgfqpoint{2.095651in}{1.558128in}}%
\pgfpathcurveto{\pgfqpoint{2.101475in}{1.552304in}}{\pgfqpoint{2.109375in}{1.549032in}}{\pgfqpoint{2.117611in}{1.549032in}}%
\pgfpathclose%
\pgfusepath{stroke,fill}%
\end{pgfscope}%
\begin{pgfscope}%
\pgfpathrectangle{\pgfqpoint{0.100000in}{0.220728in}}{\pgfqpoint{3.696000in}{3.696000in}}%
\pgfusepath{clip}%
\pgfsetbuttcap%
\pgfsetroundjoin%
\definecolor{currentfill}{rgb}{0.121569,0.466667,0.705882}%
\pgfsetfillcolor{currentfill}%
\pgfsetfillopacity{0.951482}%
\pgfsetlinewidth{1.003750pt}%
\definecolor{currentstroke}{rgb}{0.121569,0.466667,0.705882}%
\pgfsetstrokecolor{currentstroke}%
\pgfsetstrokeopacity{0.951482}%
\pgfsetdash{}{0pt}%
\pgfpathmoveto{\pgfqpoint{2.512740in}{1.632334in}}%
\pgfpathcurveto{\pgfqpoint{2.520977in}{1.632334in}}{\pgfqpoint{2.528877in}{1.635607in}}{\pgfqpoint{2.534701in}{1.641431in}}%
\pgfpathcurveto{\pgfqpoint{2.540525in}{1.647255in}}{\pgfqpoint{2.543797in}{1.655155in}}{\pgfqpoint{2.543797in}{1.663391in}}%
\pgfpathcurveto{\pgfqpoint{2.543797in}{1.671627in}}{\pgfqpoint{2.540525in}{1.679527in}}{\pgfqpoint{2.534701in}{1.685351in}}%
\pgfpathcurveto{\pgfqpoint{2.528877in}{1.691175in}}{\pgfqpoint{2.520977in}{1.694447in}}{\pgfqpoint{2.512740in}{1.694447in}}%
\pgfpathcurveto{\pgfqpoint{2.504504in}{1.694447in}}{\pgfqpoint{2.496604in}{1.691175in}}{\pgfqpoint{2.490780in}{1.685351in}}%
\pgfpathcurveto{\pgfqpoint{2.484956in}{1.679527in}}{\pgfqpoint{2.481684in}{1.671627in}}{\pgfqpoint{2.481684in}{1.663391in}}%
\pgfpathcurveto{\pgfqpoint{2.481684in}{1.655155in}}{\pgfqpoint{2.484956in}{1.647255in}}{\pgfqpoint{2.490780in}{1.641431in}}%
\pgfpathcurveto{\pgfqpoint{2.496604in}{1.635607in}}{\pgfqpoint{2.504504in}{1.632334in}}{\pgfqpoint{2.512740in}{1.632334in}}%
\pgfpathclose%
\pgfusepath{stroke,fill}%
\end{pgfscope}%
\begin{pgfscope}%
\pgfpathrectangle{\pgfqpoint{0.100000in}{0.220728in}}{\pgfqpoint{3.696000in}{3.696000in}}%
\pgfusepath{clip}%
\pgfsetbuttcap%
\pgfsetroundjoin%
\definecolor{currentfill}{rgb}{0.121569,0.466667,0.705882}%
\pgfsetfillcolor{currentfill}%
\pgfsetfillopacity{0.952291}%
\pgfsetlinewidth{1.003750pt}%
\definecolor{currentstroke}{rgb}{0.121569,0.466667,0.705882}%
\pgfsetstrokecolor{currentstroke}%
\pgfsetstrokeopacity{0.952291}%
\pgfsetdash{}{0pt}%
\pgfpathmoveto{\pgfqpoint{2.124054in}{1.543860in}}%
\pgfpathcurveto{\pgfqpoint{2.132290in}{1.543860in}}{\pgfqpoint{2.140190in}{1.547132in}}{\pgfqpoint{2.146014in}{1.552956in}}%
\pgfpathcurveto{\pgfqpoint{2.151838in}{1.558780in}}{\pgfqpoint{2.155110in}{1.566680in}}{\pgfqpoint{2.155110in}{1.574916in}}%
\pgfpathcurveto{\pgfqpoint{2.155110in}{1.583153in}}{\pgfqpoint{2.151838in}{1.591053in}}{\pgfqpoint{2.146014in}{1.596877in}}%
\pgfpathcurveto{\pgfqpoint{2.140190in}{1.602700in}}{\pgfqpoint{2.132290in}{1.605973in}}{\pgfqpoint{2.124054in}{1.605973in}}%
\pgfpathcurveto{\pgfqpoint{2.115818in}{1.605973in}}{\pgfqpoint{2.107917in}{1.602700in}}{\pgfqpoint{2.102094in}{1.596877in}}%
\pgfpathcurveto{\pgfqpoint{2.096270in}{1.591053in}}{\pgfqpoint{2.092997in}{1.583153in}}{\pgfqpoint{2.092997in}{1.574916in}}%
\pgfpathcurveto{\pgfqpoint{2.092997in}{1.566680in}}{\pgfqpoint{2.096270in}{1.558780in}}{\pgfqpoint{2.102094in}{1.552956in}}%
\pgfpathcurveto{\pgfqpoint{2.107917in}{1.547132in}}{\pgfqpoint{2.115818in}{1.543860in}}{\pgfqpoint{2.124054in}{1.543860in}}%
\pgfpathclose%
\pgfusepath{stroke,fill}%
\end{pgfscope}%
\begin{pgfscope}%
\pgfpathrectangle{\pgfqpoint{0.100000in}{0.220728in}}{\pgfqpoint{3.696000in}{3.696000in}}%
\pgfusepath{clip}%
\pgfsetbuttcap%
\pgfsetroundjoin%
\definecolor{currentfill}{rgb}{0.121569,0.466667,0.705882}%
\pgfsetfillcolor{currentfill}%
\pgfsetfillopacity{0.954856}%
\pgfsetlinewidth{1.003750pt}%
\definecolor{currentstroke}{rgb}{0.121569,0.466667,0.705882}%
\pgfsetstrokecolor{currentstroke}%
\pgfsetstrokeopacity{0.954856}%
\pgfsetdash{}{0pt}%
\pgfpathmoveto{\pgfqpoint{2.137661in}{1.540123in}}%
\pgfpathcurveto{\pgfqpoint{2.145897in}{1.540123in}}{\pgfqpoint{2.153797in}{1.543396in}}{\pgfqpoint{2.159621in}{1.549220in}}%
\pgfpathcurveto{\pgfqpoint{2.165445in}{1.555044in}}{\pgfqpoint{2.168717in}{1.562944in}}{\pgfqpoint{2.168717in}{1.571180in}}%
\pgfpathcurveto{\pgfqpoint{2.168717in}{1.579416in}}{\pgfqpoint{2.165445in}{1.587316in}}{\pgfqpoint{2.159621in}{1.593140in}}%
\pgfpathcurveto{\pgfqpoint{2.153797in}{1.598964in}}{\pgfqpoint{2.145897in}{1.602236in}}{\pgfqpoint{2.137661in}{1.602236in}}%
\pgfpathcurveto{\pgfqpoint{2.129424in}{1.602236in}}{\pgfqpoint{2.121524in}{1.598964in}}{\pgfqpoint{2.115700in}{1.593140in}}%
\pgfpathcurveto{\pgfqpoint{2.109877in}{1.587316in}}{\pgfqpoint{2.106604in}{1.579416in}}{\pgfqpoint{2.106604in}{1.571180in}}%
\pgfpathcurveto{\pgfqpoint{2.106604in}{1.562944in}}{\pgfqpoint{2.109877in}{1.555044in}}{\pgfqpoint{2.115700in}{1.549220in}}%
\pgfpathcurveto{\pgfqpoint{2.121524in}{1.543396in}}{\pgfqpoint{2.129424in}{1.540123in}}{\pgfqpoint{2.137661in}{1.540123in}}%
\pgfpathclose%
\pgfusepath{stroke,fill}%
\end{pgfscope}%
\begin{pgfscope}%
\pgfpathrectangle{\pgfqpoint{0.100000in}{0.220728in}}{\pgfqpoint{3.696000in}{3.696000in}}%
\pgfusepath{clip}%
\pgfsetbuttcap%
\pgfsetroundjoin%
\definecolor{currentfill}{rgb}{0.121569,0.466667,0.705882}%
\pgfsetfillcolor{currentfill}%
\pgfsetfillopacity{0.955511}%
\pgfsetlinewidth{1.003750pt}%
\definecolor{currentstroke}{rgb}{0.121569,0.466667,0.705882}%
\pgfsetstrokecolor{currentstroke}%
\pgfsetstrokeopacity{0.955511}%
\pgfsetdash{}{0pt}%
\pgfpathmoveto{\pgfqpoint{2.503258in}{1.614091in}}%
\pgfpathcurveto{\pgfqpoint{2.511494in}{1.614091in}}{\pgfqpoint{2.519394in}{1.617363in}}{\pgfqpoint{2.525218in}{1.623187in}}%
\pgfpathcurveto{\pgfqpoint{2.531042in}{1.629011in}}{\pgfqpoint{2.534314in}{1.636911in}}{\pgfqpoint{2.534314in}{1.645147in}}%
\pgfpathcurveto{\pgfqpoint{2.534314in}{1.653383in}}{\pgfqpoint{2.531042in}{1.661283in}}{\pgfqpoint{2.525218in}{1.667107in}}%
\pgfpathcurveto{\pgfqpoint{2.519394in}{1.672931in}}{\pgfqpoint{2.511494in}{1.676204in}}{\pgfqpoint{2.503258in}{1.676204in}}%
\pgfpathcurveto{\pgfqpoint{2.495021in}{1.676204in}}{\pgfqpoint{2.487121in}{1.672931in}}{\pgfqpoint{2.481297in}{1.667107in}}%
\pgfpathcurveto{\pgfqpoint{2.475473in}{1.661283in}}{\pgfqpoint{2.472201in}{1.653383in}}{\pgfqpoint{2.472201in}{1.645147in}}%
\pgfpathcurveto{\pgfqpoint{2.472201in}{1.636911in}}{\pgfqpoint{2.475473in}{1.629011in}}{\pgfqpoint{2.481297in}{1.623187in}}%
\pgfpathcurveto{\pgfqpoint{2.487121in}{1.617363in}}{\pgfqpoint{2.495021in}{1.614091in}}{\pgfqpoint{2.503258in}{1.614091in}}%
\pgfpathclose%
\pgfusepath{stroke,fill}%
\end{pgfscope}%
\begin{pgfscope}%
\pgfpathrectangle{\pgfqpoint{0.100000in}{0.220728in}}{\pgfqpoint{3.696000in}{3.696000in}}%
\pgfusepath{clip}%
\pgfsetbuttcap%
\pgfsetroundjoin%
\definecolor{currentfill}{rgb}{0.121569,0.466667,0.705882}%
\pgfsetfillcolor{currentfill}%
\pgfsetfillopacity{0.958823}%
\pgfsetlinewidth{1.003750pt}%
\definecolor{currentstroke}{rgb}{0.121569,0.466667,0.705882}%
\pgfsetstrokecolor{currentstroke}%
\pgfsetstrokeopacity{0.958823}%
\pgfsetdash{}{0pt}%
\pgfpathmoveto{\pgfqpoint{2.160544in}{1.524389in}}%
\pgfpathcurveto{\pgfqpoint{2.168780in}{1.524389in}}{\pgfqpoint{2.176680in}{1.527661in}}{\pgfqpoint{2.182504in}{1.533485in}}%
\pgfpathcurveto{\pgfqpoint{2.188328in}{1.539309in}}{\pgfqpoint{2.191601in}{1.547209in}}{\pgfqpoint{2.191601in}{1.555446in}}%
\pgfpathcurveto{\pgfqpoint{2.191601in}{1.563682in}}{\pgfqpoint{2.188328in}{1.571582in}}{\pgfqpoint{2.182504in}{1.577406in}}%
\pgfpathcurveto{\pgfqpoint{2.176680in}{1.583230in}}{\pgfqpoint{2.168780in}{1.586502in}}{\pgfqpoint{2.160544in}{1.586502in}}%
\pgfpathcurveto{\pgfqpoint{2.152308in}{1.586502in}}{\pgfqpoint{2.144408in}{1.583230in}}{\pgfqpoint{2.138584in}{1.577406in}}%
\pgfpathcurveto{\pgfqpoint{2.132760in}{1.571582in}}{\pgfqpoint{2.129488in}{1.563682in}}{\pgfqpoint{2.129488in}{1.555446in}}%
\pgfpathcurveto{\pgfqpoint{2.129488in}{1.547209in}}{\pgfqpoint{2.132760in}{1.539309in}}{\pgfqpoint{2.138584in}{1.533485in}}%
\pgfpathcurveto{\pgfqpoint{2.144408in}{1.527661in}}{\pgfqpoint{2.152308in}{1.524389in}}{\pgfqpoint{2.160544in}{1.524389in}}%
\pgfpathclose%
\pgfusepath{stroke,fill}%
\end{pgfscope}%
\begin{pgfscope}%
\pgfpathrectangle{\pgfqpoint{0.100000in}{0.220728in}}{\pgfqpoint{3.696000in}{3.696000in}}%
\pgfusepath{clip}%
\pgfsetbuttcap%
\pgfsetroundjoin%
\definecolor{currentfill}{rgb}{0.121569,0.466667,0.705882}%
\pgfsetfillcolor{currentfill}%
\pgfsetfillopacity{0.959279}%
\pgfsetlinewidth{1.003750pt}%
\definecolor{currentstroke}{rgb}{0.121569,0.466667,0.705882}%
\pgfsetstrokecolor{currentstroke}%
\pgfsetstrokeopacity{0.959279}%
\pgfsetdash{}{0pt}%
\pgfpathmoveto{\pgfqpoint{2.488956in}{1.592917in}}%
\pgfpathcurveto{\pgfqpoint{2.497193in}{1.592917in}}{\pgfqpoint{2.505093in}{1.596189in}}{\pgfqpoint{2.510917in}{1.602013in}}%
\pgfpathcurveto{\pgfqpoint{2.516741in}{1.607837in}}{\pgfqpoint{2.520013in}{1.615737in}}{\pgfqpoint{2.520013in}{1.623973in}}%
\pgfpathcurveto{\pgfqpoint{2.520013in}{1.632210in}}{\pgfqpoint{2.516741in}{1.640110in}}{\pgfqpoint{2.510917in}{1.645934in}}%
\pgfpathcurveto{\pgfqpoint{2.505093in}{1.651758in}}{\pgfqpoint{2.497193in}{1.655030in}}{\pgfqpoint{2.488956in}{1.655030in}}%
\pgfpathcurveto{\pgfqpoint{2.480720in}{1.655030in}}{\pgfqpoint{2.472820in}{1.651758in}}{\pgfqpoint{2.466996in}{1.645934in}}%
\pgfpathcurveto{\pgfqpoint{2.461172in}{1.640110in}}{\pgfqpoint{2.457900in}{1.632210in}}{\pgfqpoint{2.457900in}{1.623973in}}%
\pgfpathcurveto{\pgfqpoint{2.457900in}{1.615737in}}{\pgfqpoint{2.461172in}{1.607837in}}{\pgfqpoint{2.466996in}{1.602013in}}%
\pgfpathcurveto{\pgfqpoint{2.472820in}{1.596189in}}{\pgfqpoint{2.480720in}{1.592917in}}{\pgfqpoint{2.488956in}{1.592917in}}%
\pgfpathclose%
\pgfusepath{stroke,fill}%
\end{pgfscope}%
\begin{pgfscope}%
\pgfpathrectangle{\pgfqpoint{0.100000in}{0.220728in}}{\pgfqpoint{3.696000in}{3.696000in}}%
\pgfusepath{clip}%
\pgfsetbuttcap%
\pgfsetroundjoin%
\definecolor{currentfill}{rgb}{0.121569,0.466667,0.705882}%
\pgfsetfillcolor{currentfill}%
\pgfsetfillopacity{0.963333}%
\pgfsetlinewidth{1.003750pt}%
\definecolor{currentstroke}{rgb}{0.121569,0.466667,0.705882}%
\pgfsetstrokecolor{currentstroke}%
\pgfsetstrokeopacity{0.963333}%
\pgfsetdash{}{0pt}%
\pgfpathmoveto{\pgfqpoint{2.181700in}{1.516259in}}%
\pgfpathcurveto{\pgfqpoint{2.189937in}{1.516259in}}{\pgfqpoint{2.197837in}{1.519531in}}{\pgfqpoint{2.203661in}{1.525355in}}%
\pgfpathcurveto{\pgfqpoint{2.209485in}{1.531179in}}{\pgfqpoint{2.212757in}{1.539079in}}{\pgfqpoint{2.212757in}{1.547315in}}%
\pgfpathcurveto{\pgfqpoint{2.212757in}{1.555552in}}{\pgfqpoint{2.209485in}{1.563452in}}{\pgfqpoint{2.203661in}{1.569276in}}%
\pgfpathcurveto{\pgfqpoint{2.197837in}{1.575100in}}{\pgfqpoint{2.189937in}{1.578372in}}{\pgfqpoint{2.181700in}{1.578372in}}%
\pgfpathcurveto{\pgfqpoint{2.173464in}{1.578372in}}{\pgfqpoint{2.165564in}{1.575100in}}{\pgfqpoint{2.159740in}{1.569276in}}%
\pgfpathcurveto{\pgfqpoint{2.153916in}{1.563452in}}{\pgfqpoint{2.150644in}{1.555552in}}{\pgfqpoint{2.150644in}{1.547315in}}%
\pgfpathcurveto{\pgfqpoint{2.150644in}{1.539079in}}{\pgfqpoint{2.153916in}{1.531179in}}{\pgfqpoint{2.159740in}{1.525355in}}%
\pgfpathcurveto{\pgfqpoint{2.165564in}{1.519531in}}{\pgfqpoint{2.173464in}{1.516259in}}{\pgfqpoint{2.181700in}{1.516259in}}%
\pgfpathclose%
\pgfusepath{stroke,fill}%
\end{pgfscope}%
\begin{pgfscope}%
\pgfpathrectangle{\pgfqpoint{0.100000in}{0.220728in}}{\pgfqpoint{3.696000in}{3.696000in}}%
\pgfusepath{clip}%
\pgfsetbuttcap%
\pgfsetroundjoin%
\definecolor{currentfill}{rgb}{0.121569,0.466667,0.705882}%
\pgfsetfillcolor{currentfill}%
\pgfsetfillopacity{0.964419}%
\pgfsetlinewidth{1.003750pt}%
\definecolor{currentstroke}{rgb}{0.121569,0.466667,0.705882}%
\pgfsetstrokecolor{currentstroke}%
\pgfsetstrokeopacity{0.964419}%
\pgfsetdash{}{0pt}%
\pgfpathmoveto{\pgfqpoint{2.475215in}{1.568504in}}%
\pgfpathcurveto{\pgfqpoint{2.483451in}{1.568504in}}{\pgfqpoint{2.491351in}{1.571776in}}{\pgfqpoint{2.497175in}{1.577600in}}%
\pgfpathcurveto{\pgfqpoint{2.502999in}{1.583424in}}{\pgfqpoint{2.506271in}{1.591324in}}{\pgfqpoint{2.506271in}{1.599560in}}%
\pgfpathcurveto{\pgfqpoint{2.506271in}{1.607797in}}{\pgfqpoint{2.502999in}{1.615697in}}{\pgfqpoint{2.497175in}{1.621521in}}%
\pgfpathcurveto{\pgfqpoint{2.491351in}{1.627345in}}{\pgfqpoint{2.483451in}{1.630617in}}{\pgfqpoint{2.475215in}{1.630617in}}%
\pgfpathcurveto{\pgfqpoint{2.466978in}{1.630617in}}{\pgfqpoint{2.459078in}{1.627345in}}{\pgfqpoint{2.453254in}{1.621521in}}%
\pgfpathcurveto{\pgfqpoint{2.447430in}{1.615697in}}{\pgfqpoint{2.444158in}{1.607797in}}{\pgfqpoint{2.444158in}{1.599560in}}%
\pgfpathcurveto{\pgfqpoint{2.444158in}{1.591324in}}{\pgfqpoint{2.447430in}{1.583424in}}{\pgfqpoint{2.453254in}{1.577600in}}%
\pgfpathcurveto{\pgfqpoint{2.459078in}{1.571776in}}{\pgfqpoint{2.466978in}{1.568504in}}{\pgfqpoint{2.475215in}{1.568504in}}%
\pgfpathclose%
\pgfusepath{stroke,fill}%
\end{pgfscope}%
\begin{pgfscope}%
\pgfpathrectangle{\pgfqpoint{0.100000in}{0.220728in}}{\pgfqpoint{3.696000in}{3.696000in}}%
\pgfusepath{clip}%
\pgfsetbuttcap%
\pgfsetroundjoin%
\definecolor{currentfill}{rgb}{0.121569,0.466667,0.705882}%
\pgfsetfillcolor{currentfill}%
\pgfsetfillopacity{0.966835}%
\pgfsetlinewidth{1.003750pt}%
\definecolor{currentstroke}{rgb}{0.121569,0.466667,0.705882}%
\pgfsetstrokecolor{currentstroke}%
\pgfsetstrokeopacity{0.966835}%
\pgfsetdash{}{0pt}%
\pgfpathmoveto{\pgfqpoint{2.466436in}{1.554680in}}%
\pgfpathcurveto{\pgfqpoint{2.474672in}{1.554680in}}{\pgfqpoint{2.482572in}{1.557952in}}{\pgfqpoint{2.488396in}{1.563776in}}%
\pgfpathcurveto{\pgfqpoint{2.494220in}{1.569600in}}{\pgfqpoint{2.497492in}{1.577500in}}{\pgfqpoint{2.497492in}{1.585737in}}%
\pgfpathcurveto{\pgfqpoint{2.497492in}{1.593973in}}{\pgfqpoint{2.494220in}{1.601873in}}{\pgfqpoint{2.488396in}{1.607697in}}%
\pgfpathcurveto{\pgfqpoint{2.482572in}{1.613521in}}{\pgfqpoint{2.474672in}{1.616793in}}{\pgfqpoint{2.466436in}{1.616793in}}%
\pgfpathcurveto{\pgfqpoint{2.458199in}{1.616793in}}{\pgfqpoint{2.450299in}{1.613521in}}{\pgfqpoint{2.444475in}{1.607697in}}%
\pgfpathcurveto{\pgfqpoint{2.438651in}{1.601873in}}{\pgfqpoint{2.435379in}{1.593973in}}{\pgfqpoint{2.435379in}{1.585737in}}%
\pgfpathcurveto{\pgfqpoint{2.435379in}{1.577500in}}{\pgfqpoint{2.438651in}{1.569600in}}{\pgfqpoint{2.444475in}{1.563776in}}%
\pgfpathcurveto{\pgfqpoint{2.450299in}{1.557952in}}{\pgfqpoint{2.458199in}{1.554680in}}{\pgfqpoint{2.466436in}{1.554680in}}%
\pgfpathclose%
\pgfusepath{stroke,fill}%
\end{pgfscope}%
\begin{pgfscope}%
\pgfpathrectangle{\pgfqpoint{0.100000in}{0.220728in}}{\pgfqpoint{3.696000in}{3.696000in}}%
\pgfusepath{clip}%
\pgfsetbuttcap%
\pgfsetroundjoin%
\definecolor{currentfill}{rgb}{0.121569,0.466667,0.705882}%
\pgfsetfillcolor{currentfill}%
\pgfsetfillopacity{0.967044}%
\pgfsetlinewidth{1.003750pt}%
\definecolor{currentstroke}{rgb}{0.121569,0.466667,0.705882}%
\pgfsetstrokecolor{currentstroke}%
\pgfsetstrokeopacity{0.967044}%
\pgfsetdash{}{0pt}%
\pgfpathmoveto{\pgfqpoint{2.197571in}{1.506587in}}%
\pgfpathcurveto{\pgfqpoint{2.205807in}{1.506587in}}{\pgfqpoint{2.213707in}{1.509859in}}{\pgfqpoint{2.219531in}{1.515683in}}%
\pgfpathcurveto{\pgfqpoint{2.225355in}{1.521507in}}{\pgfqpoint{2.228627in}{1.529407in}}{\pgfqpoint{2.228627in}{1.537644in}}%
\pgfpathcurveto{\pgfqpoint{2.228627in}{1.545880in}}{\pgfqpoint{2.225355in}{1.553780in}}{\pgfqpoint{2.219531in}{1.559604in}}%
\pgfpathcurveto{\pgfqpoint{2.213707in}{1.565428in}}{\pgfqpoint{2.205807in}{1.568700in}}{\pgfqpoint{2.197571in}{1.568700in}}%
\pgfpathcurveto{\pgfqpoint{2.189335in}{1.568700in}}{\pgfqpoint{2.181434in}{1.565428in}}{\pgfqpoint{2.175611in}{1.559604in}}%
\pgfpathcurveto{\pgfqpoint{2.169787in}{1.553780in}}{\pgfqpoint{2.166514in}{1.545880in}}{\pgfqpoint{2.166514in}{1.537644in}}%
\pgfpathcurveto{\pgfqpoint{2.166514in}{1.529407in}}{\pgfqpoint{2.169787in}{1.521507in}}{\pgfqpoint{2.175611in}{1.515683in}}%
\pgfpathcurveto{\pgfqpoint{2.181434in}{1.509859in}}{\pgfqpoint{2.189335in}{1.506587in}}{\pgfqpoint{2.197571in}{1.506587in}}%
\pgfpathclose%
\pgfusepath{stroke,fill}%
\end{pgfscope}%
\begin{pgfscope}%
\pgfpathrectangle{\pgfqpoint{0.100000in}{0.220728in}}{\pgfqpoint{3.696000in}{3.696000in}}%
\pgfusepath{clip}%
\pgfsetbuttcap%
\pgfsetroundjoin%
\definecolor{currentfill}{rgb}{0.121569,0.466667,0.705882}%
\pgfsetfillcolor{currentfill}%
\pgfsetfillopacity{0.968249}%
\pgfsetlinewidth{1.003750pt}%
\definecolor{currentstroke}{rgb}{0.121569,0.466667,0.705882}%
\pgfsetstrokecolor{currentstroke}%
\pgfsetstrokeopacity{0.968249}%
\pgfsetdash{}{0pt}%
\pgfpathmoveto{\pgfqpoint{2.461430in}{1.547726in}}%
\pgfpathcurveto{\pgfqpoint{2.469666in}{1.547726in}}{\pgfqpoint{2.477566in}{1.550998in}}{\pgfqpoint{2.483390in}{1.556822in}}%
\pgfpathcurveto{\pgfqpoint{2.489214in}{1.562646in}}{\pgfqpoint{2.492486in}{1.570546in}}{\pgfqpoint{2.492486in}{1.578783in}}%
\pgfpathcurveto{\pgfqpoint{2.492486in}{1.587019in}}{\pgfqpoint{2.489214in}{1.594919in}}{\pgfqpoint{2.483390in}{1.600743in}}%
\pgfpathcurveto{\pgfqpoint{2.477566in}{1.606567in}}{\pgfqpoint{2.469666in}{1.609839in}}{\pgfqpoint{2.461430in}{1.609839in}}%
\pgfpathcurveto{\pgfqpoint{2.453194in}{1.609839in}}{\pgfqpoint{2.445294in}{1.606567in}}{\pgfqpoint{2.439470in}{1.600743in}}%
\pgfpathcurveto{\pgfqpoint{2.433646in}{1.594919in}}{\pgfqpoint{2.430373in}{1.587019in}}{\pgfqpoint{2.430373in}{1.578783in}}%
\pgfpathcurveto{\pgfqpoint{2.430373in}{1.570546in}}{\pgfqpoint{2.433646in}{1.562646in}}{\pgfqpoint{2.439470in}{1.556822in}}%
\pgfpathcurveto{\pgfqpoint{2.445294in}{1.550998in}}{\pgfqpoint{2.453194in}{1.547726in}}{\pgfqpoint{2.461430in}{1.547726in}}%
\pgfpathclose%
\pgfusepath{stroke,fill}%
\end{pgfscope}%
\begin{pgfscope}%
\pgfpathrectangle{\pgfqpoint{0.100000in}{0.220728in}}{\pgfqpoint{3.696000in}{3.696000in}}%
\pgfusepath{clip}%
\pgfsetbuttcap%
\pgfsetroundjoin%
\definecolor{currentfill}{rgb}{0.121569,0.466667,0.705882}%
\pgfsetfillcolor{currentfill}%
\pgfsetfillopacity{0.969044}%
\pgfsetlinewidth{1.003750pt}%
\definecolor{currentstroke}{rgb}{0.121569,0.466667,0.705882}%
\pgfsetstrokecolor{currentstroke}%
\pgfsetstrokeopacity{0.969044}%
\pgfsetdash{}{0pt}%
\pgfpathmoveto{\pgfqpoint{2.459049in}{1.543493in}}%
\pgfpathcurveto{\pgfqpoint{2.467286in}{1.543493in}}{\pgfqpoint{2.475186in}{1.546765in}}{\pgfqpoint{2.481009in}{1.552589in}}%
\pgfpathcurveto{\pgfqpoint{2.486833in}{1.558413in}}{\pgfqpoint{2.490106in}{1.566313in}}{\pgfqpoint{2.490106in}{1.574550in}}%
\pgfpathcurveto{\pgfqpoint{2.490106in}{1.582786in}}{\pgfqpoint{2.486833in}{1.590686in}}{\pgfqpoint{2.481009in}{1.596510in}}%
\pgfpathcurveto{\pgfqpoint{2.475186in}{1.602334in}}{\pgfqpoint{2.467286in}{1.605606in}}{\pgfqpoint{2.459049in}{1.605606in}}%
\pgfpathcurveto{\pgfqpoint{2.450813in}{1.605606in}}{\pgfqpoint{2.442913in}{1.602334in}}{\pgfqpoint{2.437089in}{1.596510in}}%
\pgfpathcurveto{\pgfqpoint{2.431265in}{1.590686in}}{\pgfqpoint{2.427993in}{1.582786in}}{\pgfqpoint{2.427993in}{1.574550in}}%
\pgfpathcurveto{\pgfqpoint{2.427993in}{1.566313in}}{\pgfqpoint{2.431265in}{1.558413in}}{\pgfqpoint{2.437089in}{1.552589in}}%
\pgfpathcurveto{\pgfqpoint{2.442913in}{1.546765in}}{\pgfqpoint{2.450813in}{1.543493in}}{\pgfqpoint{2.459049in}{1.543493in}}%
\pgfpathclose%
\pgfusepath{stroke,fill}%
\end{pgfscope}%
\begin{pgfscope}%
\pgfpathrectangle{\pgfqpoint{0.100000in}{0.220728in}}{\pgfqpoint{3.696000in}{3.696000in}}%
\pgfusepath{clip}%
\pgfsetbuttcap%
\pgfsetroundjoin%
\definecolor{currentfill}{rgb}{0.121569,0.466667,0.705882}%
\pgfsetfillcolor{currentfill}%
\pgfsetfillopacity{0.969496}%
\pgfsetlinewidth{1.003750pt}%
\definecolor{currentstroke}{rgb}{0.121569,0.466667,0.705882}%
\pgfsetstrokecolor{currentstroke}%
\pgfsetstrokeopacity{0.969496}%
\pgfsetdash{}{0pt}%
\pgfpathmoveto{\pgfqpoint{2.457512in}{1.541543in}}%
\pgfpathcurveto{\pgfqpoint{2.465748in}{1.541543in}}{\pgfqpoint{2.473648in}{1.544815in}}{\pgfqpoint{2.479472in}{1.550639in}}%
\pgfpathcurveto{\pgfqpoint{2.485296in}{1.556463in}}{\pgfqpoint{2.488569in}{1.564363in}}{\pgfqpoint{2.488569in}{1.572599in}}%
\pgfpathcurveto{\pgfqpoint{2.488569in}{1.580836in}}{\pgfqpoint{2.485296in}{1.588736in}}{\pgfqpoint{2.479472in}{1.594560in}}%
\pgfpathcurveto{\pgfqpoint{2.473648in}{1.600383in}}{\pgfqpoint{2.465748in}{1.603656in}}{\pgfqpoint{2.457512in}{1.603656in}}%
\pgfpathcurveto{\pgfqpoint{2.449276in}{1.603656in}}{\pgfqpoint{2.441376in}{1.600383in}}{\pgfqpoint{2.435552in}{1.594560in}}%
\pgfpathcurveto{\pgfqpoint{2.429728in}{1.588736in}}{\pgfqpoint{2.426456in}{1.580836in}}{\pgfqpoint{2.426456in}{1.572599in}}%
\pgfpathcurveto{\pgfqpoint{2.426456in}{1.564363in}}{\pgfqpoint{2.429728in}{1.556463in}}{\pgfqpoint{2.435552in}{1.550639in}}%
\pgfpathcurveto{\pgfqpoint{2.441376in}{1.544815in}}{\pgfqpoint{2.449276in}{1.541543in}}{\pgfqpoint{2.457512in}{1.541543in}}%
\pgfpathclose%
\pgfusepath{stroke,fill}%
\end{pgfscope}%
\begin{pgfscope}%
\pgfpathrectangle{\pgfqpoint{0.100000in}{0.220728in}}{\pgfqpoint{3.696000in}{3.696000in}}%
\pgfusepath{clip}%
\pgfsetbuttcap%
\pgfsetroundjoin%
\definecolor{currentfill}{rgb}{0.121569,0.466667,0.705882}%
\pgfsetfillcolor{currentfill}%
\pgfsetfillopacity{0.969744}%
\pgfsetlinewidth{1.003750pt}%
\definecolor{currentstroke}{rgb}{0.121569,0.466667,0.705882}%
\pgfsetstrokecolor{currentstroke}%
\pgfsetstrokeopacity{0.969744}%
\pgfsetdash{}{0pt}%
\pgfpathmoveto{\pgfqpoint{2.456786in}{1.540297in}}%
\pgfpathcurveto{\pgfqpoint{2.465022in}{1.540297in}}{\pgfqpoint{2.472922in}{1.543569in}}{\pgfqpoint{2.478746in}{1.549393in}}%
\pgfpathcurveto{\pgfqpoint{2.484570in}{1.555217in}}{\pgfqpoint{2.487843in}{1.563117in}}{\pgfqpoint{2.487843in}{1.571354in}}%
\pgfpathcurveto{\pgfqpoint{2.487843in}{1.579590in}}{\pgfqpoint{2.484570in}{1.587490in}}{\pgfqpoint{2.478746in}{1.593314in}}%
\pgfpathcurveto{\pgfqpoint{2.472922in}{1.599138in}}{\pgfqpoint{2.465022in}{1.602410in}}{\pgfqpoint{2.456786in}{1.602410in}}%
\pgfpathcurveto{\pgfqpoint{2.448550in}{1.602410in}}{\pgfqpoint{2.440650in}{1.599138in}}{\pgfqpoint{2.434826in}{1.593314in}}%
\pgfpathcurveto{\pgfqpoint{2.429002in}{1.587490in}}{\pgfqpoint{2.425730in}{1.579590in}}{\pgfqpoint{2.425730in}{1.571354in}}%
\pgfpathcurveto{\pgfqpoint{2.425730in}{1.563117in}}{\pgfqpoint{2.429002in}{1.555217in}}{\pgfqpoint{2.434826in}{1.549393in}}%
\pgfpathcurveto{\pgfqpoint{2.440650in}{1.543569in}}{\pgfqpoint{2.448550in}{1.540297in}}{\pgfqpoint{2.456786in}{1.540297in}}%
\pgfpathclose%
\pgfusepath{stroke,fill}%
\end{pgfscope}%
\begin{pgfscope}%
\pgfpathrectangle{\pgfqpoint{0.100000in}{0.220728in}}{\pgfqpoint{3.696000in}{3.696000in}}%
\pgfusepath{clip}%
\pgfsetbuttcap%
\pgfsetroundjoin%
\definecolor{currentfill}{rgb}{0.121569,0.466667,0.705882}%
\pgfsetfillcolor{currentfill}%
\pgfsetfillopacity{0.970786}%
\pgfsetlinewidth{1.003750pt}%
\definecolor{currentstroke}{rgb}{0.121569,0.466667,0.705882}%
\pgfsetstrokecolor{currentstroke}%
\pgfsetstrokeopacity{0.970786}%
\pgfsetdash{}{0pt}%
\pgfpathmoveto{\pgfqpoint{2.452776in}{1.535103in}}%
\pgfpathcurveto{\pgfqpoint{2.461013in}{1.535103in}}{\pgfqpoint{2.468913in}{1.538375in}}{\pgfqpoint{2.474737in}{1.544199in}}%
\pgfpathcurveto{\pgfqpoint{2.480561in}{1.550023in}}{\pgfqpoint{2.483833in}{1.557923in}}{\pgfqpoint{2.483833in}{1.566159in}}%
\pgfpathcurveto{\pgfqpoint{2.483833in}{1.574396in}}{\pgfqpoint{2.480561in}{1.582296in}}{\pgfqpoint{2.474737in}{1.588120in}}%
\pgfpathcurveto{\pgfqpoint{2.468913in}{1.593944in}}{\pgfqpoint{2.461013in}{1.597216in}}{\pgfqpoint{2.452776in}{1.597216in}}%
\pgfpathcurveto{\pgfqpoint{2.444540in}{1.597216in}}{\pgfqpoint{2.436640in}{1.593944in}}{\pgfqpoint{2.430816in}{1.588120in}}%
\pgfpathcurveto{\pgfqpoint{2.424992in}{1.582296in}}{\pgfqpoint{2.421720in}{1.574396in}}{\pgfqpoint{2.421720in}{1.566159in}}%
\pgfpathcurveto{\pgfqpoint{2.421720in}{1.557923in}}{\pgfqpoint{2.424992in}{1.550023in}}{\pgfqpoint{2.430816in}{1.544199in}}%
\pgfpathcurveto{\pgfqpoint{2.436640in}{1.538375in}}{\pgfqpoint{2.444540in}{1.535103in}}{\pgfqpoint{2.452776in}{1.535103in}}%
\pgfpathclose%
\pgfusepath{stroke,fill}%
\end{pgfscope}%
\begin{pgfscope}%
\pgfpathrectangle{\pgfqpoint{0.100000in}{0.220728in}}{\pgfqpoint{3.696000in}{3.696000in}}%
\pgfusepath{clip}%
\pgfsetbuttcap%
\pgfsetroundjoin%
\definecolor{currentfill}{rgb}{0.121569,0.466667,0.705882}%
\pgfsetfillcolor{currentfill}%
\pgfsetfillopacity{0.972322}%
\pgfsetlinewidth{1.003750pt}%
\definecolor{currentstroke}{rgb}{0.121569,0.466667,0.705882}%
\pgfsetstrokecolor{currentstroke}%
\pgfsetstrokeopacity{0.972322}%
\pgfsetdash{}{0pt}%
\pgfpathmoveto{\pgfqpoint{2.448253in}{1.526927in}}%
\pgfpathcurveto{\pgfqpoint{2.456489in}{1.526927in}}{\pgfqpoint{2.464389in}{1.530199in}}{\pgfqpoint{2.470213in}{1.536023in}}%
\pgfpathcurveto{\pgfqpoint{2.476037in}{1.541847in}}{\pgfqpoint{2.479309in}{1.549747in}}{\pgfqpoint{2.479309in}{1.557983in}}%
\pgfpathcurveto{\pgfqpoint{2.479309in}{1.566219in}}{\pgfqpoint{2.476037in}{1.574119in}}{\pgfqpoint{2.470213in}{1.579943in}}%
\pgfpathcurveto{\pgfqpoint{2.464389in}{1.585767in}}{\pgfqpoint{2.456489in}{1.589040in}}{\pgfqpoint{2.448253in}{1.589040in}}%
\pgfpathcurveto{\pgfqpoint{2.440017in}{1.589040in}}{\pgfqpoint{2.432117in}{1.585767in}}{\pgfqpoint{2.426293in}{1.579943in}}%
\pgfpathcurveto{\pgfqpoint{2.420469in}{1.574119in}}{\pgfqpoint{2.417196in}{1.566219in}}{\pgfqpoint{2.417196in}{1.557983in}}%
\pgfpathcurveto{\pgfqpoint{2.417196in}{1.549747in}}{\pgfqpoint{2.420469in}{1.541847in}}{\pgfqpoint{2.426293in}{1.536023in}}%
\pgfpathcurveto{\pgfqpoint{2.432117in}{1.530199in}}{\pgfqpoint{2.440017in}{1.526927in}}{\pgfqpoint{2.448253in}{1.526927in}}%
\pgfpathclose%
\pgfusepath{stroke,fill}%
\end{pgfscope}%
\begin{pgfscope}%
\pgfpathrectangle{\pgfqpoint{0.100000in}{0.220728in}}{\pgfqpoint{3.696000in}{3.696000in}}%
\pgfusepath{clip}%
\pgfsetbuttcap%
\pgfsetroundjoin%
\definecolor{currentfill}{rgb}{0.121569,0.466667,0.705882}%
\pgfsetfillcolor{currentfill}%
\pgfsetfillopacity{0.973221}%
\pgfsetlinewidth{1.003750pt}%
\definecolor{currentstroke}{rgb}{0.121569,0.466667,0.705882}%
\pgfsetstrokecolor{currentstroke}%
\pgfsetstrokeopacity{0.973221}%
\pgfsetdash{}{0pt}%
\pgfpathmoveto{\pgfqpoint{2.226433in}{1.486341in}}%
\pgfpathcurveto{\pgfqpoint{2.234669in}{1.486341in}}{\pgfqpoint{2.242569in}{1.489613in}}{\pgfqpoint{2.248393in}{1.495437in}}%
\pgfpathcurveto{\pgfqpoint{2.254217in}{1.501261in}}{\pgfqpoint{2.257489in}{1.509161in}}{\pgfqpoint{2.257489in}{1.517397in}}%
\pgfpathcurveto{\pgfqpoint{2.257489in}{1.525634in}}{\pgfqpoint{2.254217in}{1.533534in}}{\pgfqpoint{2.248393in}{1.539358in}}%
\pgfpathcurveto{\pgfqpoint{2.242569in}{1.545182in}}{\pgfqpoint{2.234669in}{1.548454in}}{\pgfqpoint{2.226433in}{1.548454in}}%
\pgfpathcurveto{\pgfqpoint{2.218197in}{1.548454in}}{\pgfqpoint{2.210297in}{1.545182in}}{\pgfqpoint{2.204473in}{1.539358in}}%
\pgfpathcurveto{\pgfqpoint{2.198649in}{1.533534in}}{\pgfqpoint{2.195376in}{1.525634in}}{\pgfqpoint{2.195376in}{1.517397in}}%
\pgfpathcurveto{\pgfqpoint{2.195376in}{1.509161in}}{\pgfqpoint{2.198649in}{1.501261in}}{\pgfqpoint{2.204473in}{1.495437in}}%
\pgfpathcurveto{\pgfqpoint{2.210297in}{1.489613in}}{\pgfqpoint{2.218197in}{1.486341in}}{\pgfqpoint{2.226433in}{1.486341in}}%
\pgfpathclose%
\pgfusepath{stroke,fill}%
\end{pgfscope}%
\begin{pgfscope}%
\pgfpathrectangle{\pgfqpoint{0.100000in}{0.220728in}}{\pgfqpoint{3.696000in}{3.696000in}}%
\pgfusepath{clip}%
\pgfsetbuttcap%
\pgfsetroundjoin%
\definecolor{currentfill}{rgb}{0.121569,0.466667,0.705882}%
\pgfsetfillcolor{currentfill}%
\pgfsetfillopacity{0.974496}%
\pgfsetlinewidth{1.003750pt}%
\definecolor{currentstroke}{rgb}{0.121569,0.466667,0.705882}%
\pgfsetstrokecolor{currentstroke}%
\pgfsetstrokeopacity{0.974496}%
\pgfsetdash{}{0pt}%
\pgfpathmoveto{\pgfqpoint{2.440096in}{1.515632in}}%
\pgfpathcurveto{\pgfqpoint{2.448333in}{1.515632in}}{\pgfqpoint{2.456233in}{1.518904in}}{\pgfqpoint{2.462057in}{1.524728in}}%
\pgfpathcurveto{\pgfqpoint{2.467881in}{1.530552in}}{\pgfqpoint{2.471153in}{1.538452in}}{\pgfqpoint{2.471153in}{1.546688in}}%
\pgfpathcurveto{\pgfqpoint{2.471153in}{1.554924in}}{\pgfqpoint{2.467881in}{1.562825in}}{\pgfqpoint{2.462057in}{1.568648in}}%
\pgfpathcurveto{\pgfqpoint{2.456233in}{1.574472in}}{\pgfqpoint{2.448333in}{1.577745in}}{\pgfqpoint{2.440096in}{1.577745in}}%
\pgfpathcurveto{\pgfqpoint{2.431860in}{1.577745in}}{\pgfqpoint{2.423960in}{1.574472in}}{\pgfqpoint{2.418136in}{1.568648in}}%
\pgfpathcurveto{\pgfqpoint{2.412312in}{1.562825in}}{\pgfqpoint{2.409040in}{1.554924in}}{\pgfqpoint{2.409040in}{1.546688in}}%
\pgfpathcurveto{\pgfqpoint{2.409040in}{1.538452in}}{\pgfqpoint{2.412312in}{1.530552in}}{\pgfqpoint{2.418136in}{1.524728in}}%
\pgfpathcurveto{\pgfqpoint{2.423960in}{1.518904in}}{\pgfqpoint{2.431860in}{1.515632in}}{\pgfqpoint{2.440096in}{1.515632in}}%
\pgfpathclose%
\pgfusepath{stroke,fill}%
\end{pgfscope}%
\begin{pgfscope}%
\pgfpathrectangle{\pgfqpoint{0.100000in}{0.220728in}}{\pgfqpoint{3.696000in}{3.696000in}}%
\pgfusepath{clip}%
\pgfsetbuttcap%
\pgfsetroundjoin%
\definecolor{currentfill}{rgb}{0.121569,0.466667,0.705882}%
\pgfsetfillcolor{currentfill}%
\pgfsetfillopacity{0.975721}%
\pgfsetlinewidth{1.003750pt}%
\definecolor{currentstroke}{rgb}{0.121569,0.466667,0.705882}%
\pgfsetstrokecolor{currentstroke}%
\pgfsetstrokeopacity{0.975721}%
\pgfsetdash{}{0pt}%
\pgfpathmoveto{\pgfqpoint{2.436060in}{1.508922in}}%
\pgfpathcurveto{\pgfqpoint{2.444296in}{1.508922in}}{\pgfqpoint{2.452196in}{1.512194in}}{\pgfqpoint{2.458020in}{1.518018in}}%
\pgfpathcurveto{\pgfqpoint{2.463844in}{1.523842in}}{\pgfqpoint{2.467116in}{1.531742in}}{\pgfqpoint{2.467116in}{1.539979in}}%
\pgfpathcurveto{\pgfqpoint{2.467116in}{1.548215in}}{\pgfqpoint{2.463844in}{1.556115in}}{\pgfqpoint{2.458020in}{1.561939in}}%
\pgfpathcurveto{\pgfqpoint{2.452196in}{1.567763in}}{\pgfqpoint{2.444296in}{1.571035in}}{\pgfqpoint{2.436060in}{1.571035in}}%
\pgfpathcurveto{\pgfqpoint{2.427824in}{1.571035in}}{\pgfqpoint{2.419924in}{1.567763in}}{\pgfqpoint{2.414100in}{1.561939in}}%
\pgfpathcurveto{\pgfqpoint{2.408276in}{1.556115in}}{\pgfqpoint{2.405003in}{1.548215in}}{\pgfqpoint{2.405003in}{1.539979in}}%
\pgfpathcurveto{\pgfqpoint{2.405003in}{1.531742in}}{\pgfqpoint{2.408276in}{1.523842in}}{\pgfqpoint{2.414100in}{1.518018in}}%
\pgfpathcurveto{\pgfqpoint{2.419924in}{1.512194in}}{\pgfqpoint{2.427824in}{1.508922in}}{\pgfqpoint{2.436060in}{1.508922in}}%
\pgfpathclose%
\pgfusepath{stroke,fill}%
\end{pgfscope}%
\begin{pgfscope}%
\pgfpathrectangle{\pgfqpoint{0.100000in}{0.220728in}}{\pgfqpoint{3.696000in}{3.696000in}}%
\pgfusepath{clip}%
\pgfsetbuttcap%
\pgfsetroundjoin%
\definecolor{currentfill}{rgb}{0.121569,0.466667,0.705882}%
\pgfsetfillcolor{currentfill}%
\pgfsetfillopacity{0.976409}%
\pgfsetlinewidth{1.003750pt}%
\definecolor{currentstroke}{rgb}{0.121569,0.466667,0.705882}%
\pgfsetstrokecolor{currentstroke}%
\pgfsetstrokeopacity{0.976409}%
\pgfsetdash{}{0pt}%
\pgfpathmoveto{\pgfqpoint{2.433828in}{1.505288in}}%
\pgfpathcurveto{\pgfqpoint{2.442064in}{1.505288in}}{\pgfqpoint{2.449964in}{1.508561in}}{\pgfqpoint{2.455788in}{1.514385in}}%
\pgfpathcurveto{\pgfqpoint{2.461612in}{1.520209in}}{\pgfqpoint{2.464885in}{1.528109in}}{\pgfqpoint{2.464885in}{1.536345in}}%
\pgfpathcurveto{\pgfqpoint{2.464885in}{1.544581in}}{\pgfqpoint{2.461612in}{1.552481in}}{\pgfqpoint{2.455788in}{1.558305in}}%
\pgfpathcurveto{\pgfqpoint{2.449964in}{1.564129in}}{\pgfqpoint{2.442064in}{1.567401in}}{\pgfqpoint{2.433828in}{1.567401in}}%
\pgfpathcurveto{\pgfqpoint{2.425592in}{1.567401in}}{\pgfqpoint{2.417692in}{1.564129in}}{\pgfqpoint{2.411868in}{1.558305in}}%
\pgfpathcurveto{\pgfqpoint{2.406044in}{1.552481in}}{\pgfqpoint{2.402772in}{1.544581in}}{\pgfqpoint{2.402772in}{1.536345in}}%
\pgfpathcurveto{\pgfqpoint{2.402772in}{1.528109in}}{\pgfqpoint{2.406044in}{1.520209in}}{\pgfqpoint{2.411868in}{1.514385in}}%
\pgfpathcurveto{\pgfqpoint{2.417692in}{1.508561in}}{\pgfqpoint{2.425592in}{1.505288in}}{\pgfqpoint{2.433828in}{1.505288in}}%
\pgfpathclose%
\pgfusepath{stroke,fill}%
\end{pgfscope}%
\begin{pgfscope}%
\pgfpathrectangle{\pgfqpoint{0.100000in}{0.220728in}}{\pgfqpoint{3.696000in}{3.696000in}}%
\pgfusepath{clip}%
\pgfsetbuttcap%
\pgfsetroundjoin%
\definecolor{currentfill}{rgb}{0.121569,0.466667,0.705882}%
\pgfsetfillcolor{currentfill}%
\pgfsetfillopacity{0.976774}%
\pgfsetlinewidth{1.003750pt}%
\definecolor{currentstroke}{rgb}{0.121569,0.466667,0.705882}%
\pgfsetstrokecolor{currentstroke}%
\pgfsetstrokeopacity{0.976774}%
\pgfsetdash{}{0pt}%
\pgfpathmoveto{\pgfqpoint{2.432421in}{1.503501in}}%
\pgfpathcurveto{\pgfqpoint{2.440657in}{1.503501in}}{\pgfqpoint{2.448557in}{1.506773in}}{\pgfqpoint{2.454381in}{1.512597in}}%
\pgfpathcurveto{\pgfqpoint{2.460205in}{1.518421in}}{\pgfqpoint{2.463477in}{1.526321in}}{\pgfqpoint{2.463477in}{1.534558in}}%
\pgfpathcurveto{\pgfqpoint{2.463477in}{1.542794in}}{\pgfqpoint{2.460205in}{1.550694in}}{\pgfqpoint{2.454381in}{1.556518in}}%
\pgfpathcurveto{\pgfqpoint{2.448557in}{1.562342in}}{\pgfqpoint{2.440657in}{1.565614in}}{\pgfqpoint{2.432421in}{1.565614in}}%
\pgfpathcurveto{\pgfqpoint{2.424185in}{1.565614in}}{\pgfqpoint{2.416285in}{1.562342in}}{\pgfqpoint{2.410461in}{1.556518in}}%
\pgfpathcurveto{\pgfqpoint{2.404637in}{1.550694in}}{\pgfqpoint{2.401364in}{1.542794in}}{\pgfqpoint{2.401364in}{1.534558in}}%
\pgfpathcurveto{\pgfqpoint{2.401364in}{1.526321in}}{\pgfqpoint{2.404637in}{1.518421in}}{\pgfqpoint{2.410461in}{1.512597in}}%
\pgfpathcurveto{\pgfqpoint{2.416285in}{1.506773in}}{\pgfqpoint{2.424185in}{1.503501in}}{\pgfqpoint{2.432421in}{1.503501in}}%
\pgfpathclose%
\pgfusepath{stroke,fill}%
\end{pgfscope}%
\begin{pgfscope}%
\pgfpathrectangle{\pgfqpoint{0.100000in}{0.220728in}}{\pgfqpoint{3.696000in}{3.696000in}}%
\pgfusepath{clip}%
\pgfsetbuttcap%
\pgfsetroundjoin%
\definecolor{currentfill}{rgb}{0.121569,0.466667,0.705882}%
\pgfsetfillcolor{currentfill}%
\pgfsetfillopacity{0.976996}%
\pgfsetlinewidth{1.003750pt}%
\definecolor{currentstroke}{rgb}{0.121569,0.466667,0.705882}%
\pgfsetstrokecolor{currentstroke}%
\pgfsetstrokeopacity{0.976996}%
\pgfsetdash{}{0pt}%
\pgfpathmoveto{\pgfqpoint{2.431798in}{1.502398in}}%
\pgfpathcurveto{\pgfqpoint{2.440034in}{1.502398in}}{\pgfqpoint{2.447934in}{1.505670in}}{\pgfqpoint{2.453758in}{1.511494in}}%
\pgfpathcurveto{\pgfqpoint{2.459582in}{1.517318in}}{\pgfqpoint{2.462854in}{1.525218in}}{\pgfqpoint{2.462854in}{1.533454in}}%
\pgfpathcurveto{\pgfqpoint{2.462854in}{1.541690in}}{\pgfqpoint{2.459582in}{1.549590in}}{\pgfqpoint{2.453758in}{1.555414in}}%
\pgfpathcurveto{\pgfqpoint{2.447934in}{1.561238in}}{\pgfqpoint{2.440034in}{1.564511in}}{\pgfqpoint{2.431798in}{1.564511in}}%
\pgfpathcurveto{\pgfqpoint{2.423561in}{1.564511in}}{\pgfqpoint{2.415661in}{1.561238in}}{\pgfqpoint{2.409837in}{1.555414in}}%
\pgfpathcurveto{\pgfqpoint{2.404013in}{1.549590in}}{\pgfqpoint{2.400741in}{1.541690in}}{\pgfqpoint{2.400741in}{1.533454in}}%
\pgfpathcurveto{\pgfqpoint{2.400741in}{1.525218in}}{\pgfqpoint{2.404013in}{1.517318in}}{\pgfqpoint{2.409837in}{1.511494in}}%
\pgfpathcurveto{\pgfqpoint{2.415661in}{1.505670in}}{\pgfqpoint{2.423561in}{1.502398in}}{\pgfqpoint{2.431798in}{1.502398in}}%
\pgfpathclose%
\pgfusepath{stroke,fill}%
\end{pgfscope}%
\begin{pgfscope}%
\pgfpathrectangle{\pgfqpoint{0.100000in}{0.220728in}}{\pgfqpoint{3.696000in}{3.696000in}}%
\pgfusepath{clip}%
\pgfsetbuttcap%
\pgfsetroundjoin%
\definecolor{currentfill}{rgb}{0.121569,0.466667,0.705882}%
\pgfsetfillcolor{currentfill}%
\pgfsetfillopacity{0.977106}%
\pgfsetlinewidth{1.003750pt}%
\definecolor{currentstroke}{rgb}{0.121569,0.466667,0.705882}%
\pgfsetstrokecolor{currentstroke}%
\pgfsetstrokeopacity{0.977106}%
\pgfsetdash{}{0pt}%
\pgfpathmoveto{\pgfqpoint{2.431366in}{1.501862in}}%
\pgfpathcurveto{\pgfqpoint{2.439602in}{1.501862in}}{\pgfqpoint{2.447502in}{1.505134in}}{\pgfqpoint{2.453326in}{1.510958in}}%
\pgfpathcurveto{\pgfqpoint{2.459150in}{1.516782in}}{\pgfqpoint{2.462423in}{1.524682in}}{\pgfqpoint{2.462423in}{1.532918in}}%
\pgfpathcurveto{\pgfqpoint{2.462423in}{1.541154in}}{\pgfqpoint{2.459150in}{1.549054in}}{\pgfqpoint{2.453326in}{1.554878in}}%
\pgfpathcurveto{\pgfqpoint{2.447502in}{1.560702in}}{\pgfqpoint{2.439602in}{1.563975in}}{\pgfqpoint{2.431366in}{1.563975in}}%
\pgfpathcurveto{\pgfqpoint{2.423130in}{1.563975in}}{\pgfqpoint{2.415230in}{1.560702in}}{\pgfqpoint{2.409406in}{1.554878in}}%
\pgfpathcurveto{\pgfqpoint{2.403582in}{1.549054in}}{\pgfqpoint{2.400310in}{1.541154in}}{\pgfqpoint{2.400310in}{1.532918in}}%
\pgfpathcurveto{\pgfqpoint{2.400310in}{1.524682in}}{\pgfqpoint{2.403582in}{1.516782in}}{\pgfqpoint{2.409406in}{1.510958in}}%
\pgfpathcurveto{\pgfqpoint{2.415230in}{1.505134in}}{\pgfqpoint{2.423130in}{1.501862in}}{\pgfqpoint{2.431366in}{1.501862in}}%
\pgfpathclose%
\pgfusepath{stroke,fill}%
\end{pgfscope}%
\begin{pgfscope}%
\pgfpathrectangle{\pgfqpoint{0.100000in}{0.220728in}}{\pgfqpoint{3.696000in}{3.696000in}}%
\pgfusepath{clip}%
\pgfsetbuttcap%
\pgfsetroundjoin%
\definecolor{currentfill}{rgb}{0.121569,0.466667,0.705882}%
\pgfsetfillcolor{currentfill}%
\pgfsetfillopacity{0.977178}%
\pgfsetlinewidth{1.003750pt}%
\definecolor{currentstroke}{rgb}{0.121569,0.466667,0.705882}%
\pgfsetstrokecolor{currentstroke}%
\pgfsetstrokeopacity{0.977178}%
\pgfsetdash{}{0pt}%
\pgfpathmoveto{\pgfqpoint{2.431188in}{1.501536in}}%
\pgfpathcurveto{\pgfqpoint{2.439424in}{1.501536in}}{\pgfqpoint{2.447324in}{1.504809in}}{\pgfqpoint{2.453148in}{1.510633in}}%
\pgfpathcurveto{\pgfqpoint{2.458972in}{1.516457in}}{\pgfqpoint{2.462244in}{1.524357in}}{\pgfqpoint{2.462244in}{1.532593in}}%
\pgfpathcurveto{\pgfqpoint{2.462244in}{1.540829in}}{\pgfqpoint{2.458972in}{1.548729in}}{\pgfqpoint{2.453148in}{1.554553in}}%
\pgfpathcurveto{\pgfqpoint{2.447324in}{1.560377in}}{\pgfqpoint{2.439424in}{1.563649in}}{\pgfqpoint{2.431188in}{1.563649in}}%
\pgfpathcurveto{\pgfqpoint{2.422952in}{1.563649in}}{\pgfqpoint{2.415052in}{1.560377in}}{\pgfqpoint{2.409228in}{1.554553in}}%
\pgfpathcurveto{\pgfqpoint{2.403404in}{1.548729in}}{\pgfqpoint{2.400131in}{1.540829in}}{\pgfqpoint{2.400131in}{1.532593in}}%
\pgfpathcurveto{\pgfqpoint{2.400131in}{1.524357in}}{\pgfqpoint{2.403404in}{1.516457in}}{\pgfqpoint{2.409228in}{1.510633in}}%
\pgfpathcurveto{\pgfqpoint{2.415052in}{1.504809in}}{\pgfqpoint{2.422952in}{1.501536in}}{\pgfqpoint{2.431188in}{1.501536in}}%
\pgfpathclose%
\pgfusepath{stroke,fill}%
\end{pgfscope}%
\begin{pgfscope}%
\pgfpathrectangle{\pgfqpoint{0.100000in}{0.220728in}}{\pgfqpoint{3.696000in}{3.696000in}}%
\pgfusepath{clip}%
\pgfsetbuttcap%
\pgfsetroundjoin%
\definecolor{currentfill}{rgb}{0.121569,0.466667,0.705882}%
\pgfsetfillcolor{currentfill}%
\pgfsetfillopacity{0.977927}%
\pgfsetlinewidth{1.003750pt}%
\definecolor{currentstroke}{rgb}{0.121569,0.466667,0.705882}%
\pgfsetstrokecolor{currentstroke}%
\pgfsetstrokeopacity{0.977927}%
\pgfsetdash{}{0pt}%
\pgfpathmoveto{\pgfqpoint{2.428405in}{1.498159in}}%
\pgfpathcurveto{\pgfqpoint{2.436641in}{1.498159in}}{\pgfqpoint{2.444541in}{1.501431in}}{\pgfqpoint{2.450365in}{1.507255in}}%
\pgfpathcurveto{\pgfqpoint{2.456189in}{1.513079in}}{\pgfqpoint{2.459462in}{1.520979in}}{\pgfqpoint{2.459462in}{1.529216in}}%
\pgfpathcurveto{\pgfqpoint{2.459462in}{1.537452in}}{\pgfqpoint{2.456189in}{1.545352in}}{\pgfqpoint{2.450365in}{1.551176in}}%
\pgfpathcurveto{\pgfqpoint{2.444541in}{1.557000in}}{\pgfqpoint{2.436641in}{1.560272in}}{\pgfqpoint{2.428405in}{1.560272in}}%
\pgfpathcurveto{\pgfqpoint{2.420169in}{1.560272in}}{\pgfqpoint{2.412269in}{1.557000in}}{\pgfqpoint{2.406445in}{1.551176in}}%
\pgfpathcurveto{\pgfqpoint{2.400621in}{1.545352in}}{\pgfqpoint{2.397349in}{1.537452in}}{\pgfqpoint{2.397349in}{1.529216in}}%
\pgfpathcurveto{\pgfqpoint{2.397349in}{1.520979in}}{\pgfqpoint{2.400621in}{1.513079in}}{\pgfqpoint{2.406445in}{1.507255in}}%
\pgfpathcurveto{\pgfqpoint{2.412269in}{1.501431in}}{\pgfqpoint{2.420169in}{1.498159in}}{\pgfqpoint{2.428405in}{1.498159in}}%
\pgfpathclose%
\pgfusepath{stroke,fill}%
\end{pgfscope}%
\begin{pgfscope}%
\pgfpathrectangle{\pgfqpoint{0.100000in}{0.220728in}}{\pgfqpoint{3.696000in}{3.696000in}}%
\pgfusepath{clip}%
\pgfsetbuttcap%
\pgfsetroundjoin%
\definecolor{currentfill}{rgb}{0.121569,0.466667,0.705882}%
\pgfsetfillcolor{currentfill}%
\pgfsetfillopacity{0.978629}%
\pgfsetlinewidth{1.003750pt}%
\definecolor{currentstroke}{rgb}{0.121569,0.466667,0.705882}%
\pgfsetstrokecolor{currentstroke}%
\pgfsetstrokeopacity{0.978629}%
\pgfsetdash{}{0pt}%
\pgfpathmoveto{\pgfqpoint{2.249916in}{1.470923in}}%
\pgfpathcurveto{\pgfqpoint{2.258153in}{1.470923in}}{\pgfqpoint{2.266053in}{1.474195in}}{\pgfqpoint{2.271877in}{1.480019in}}%
\pgfpathcurveto{\pgfqpoint{2.277701in}{1.485843in}}{\pgfqpoint{2.280973in}{1.493743in}}{\pgfqpoint{2.280973in}{1.501979in}}%
\pgfpathcurveto{\pgfqpoint{2.280973in}{1.510216in}}{\pgfqpoint{2.277701in}{1.518116in}}{\pgfqpoint{2.271877in}{1.523939in}}%
\pgfpathcurveto{\pgfqpoint{2.266053in}{1.529763in}}{\pgfqpoint{2.258153in}{1.533036in}}{\pgfqpoint{2.249916in}{1.533036in}}%
\pgfpathcurveto{\pgfqpoint{2.241680in}{1.533036in}}{\pgfqpoint{2.233780in}{1.529763in}}{\pgfqpoint{2.227956in}{1.523939in}}%
\pgfpathcurveto{\pgfqpoint{2.222132in}{1.518116in}}{\pgfqpoint{2.218860in}{1.510216in}}{\pgfqpoint{2.218860in}{1.501979in}}%
\pgfpathcurveto{\pgfqpoint{2.218860in}{1.493743in}}{\pgfqpoint{2.222132in}{1.485843in}}{\pgfqpoint{2.227956in}{1.480019in}}%
\pgfpathcurveto{\pgfqpoint{2.233780in}{1.474195in}}{\pgfqpoint{2.241680in}{1.470923in}}{\pgfqpoint{2.249916in}{1.470923in}}%
\pgfpathclose%
\pgfusepath{stroke,fill}%
\end{pgfscope}%
\begin{pgfscope}%
\pgfpathrectangle{\pgfqpoint{0.100000in}{0.220728in}}{\pgfqpoint{3.696000in}{3.696000in}}%
\pgfusepath{clip}%
\pgfsetbuttcap%
\pgfsetroundjoin%
\definecolor{currentfill}{rgb}{0.121569,0.466667,0.705882}%
\pgfsetfillcolor{currentfill}%
\pgfsetfillopacity{0.979238}%
\pgfsetlinewidth{1.003750pt}%
\definecolor{currentstroke}{rgb}{0.121569,0.466667,0.705882}%
\pgfsetstrokecolor{currentstroke}%
\pgfsetstrokeopacity{0.979238}%
\pgfsetdash{}{0pt}%
\pgfpathmoveto{\pgfqpoint{2.424860in}{1.491335in}}%
\pgfpathcurveto{\pgfqpoint{2.433096in}{1.491335in}}{\pgfqpoint{2.440997in}{1.494607in}}{\pgfqpoint{2.446820in}{1.500431in}}%
\pgfpathcurveto{\pgfqpoint{2.452644in}{1.506255in}}{\pgfqpoint{2.455917in}{1.514155in}}{\pgfqpoint{2.455917in}{1.522391in}}%
\pgfpathcurveto{\pgfqpoint{2.455917in}{1.530627in}}{\pgfqpoint{2.452644in}{1.538527in}}{\pgfqpoint{2.446820in}{1.544351in}}%
\pgfpathcurveto{\pgfqpoint{2.440997in}{1.550175in}}{\pgfqpoint{2.433096in}{1.553448in}}{\pgfqpoint{2.424860in}{1.553448in}}%
\pgfpathcurveto{\pgfqpoint{2.416624in}{1.553448in}}{\pgfqpoint{2.408724in}{1.550175in}}{\pgfqpoint{2.402900in}{1.544351in}}%
\pgfpathcurveto{\pgfqpoint{2.397076in}{1.538527in}}{\pgfqpoint{2.393804in}{1.530627in}}{\pgfqpoint{2.393804in}{1.522391in}}%
\pgfpathcurveto{\pgfqpoint{2.393804in}{1.514155in}}{\pgfqpoint{2.397076in}{1.506255in}}{\pgfqpoint{2.402900in}{1.500431in}}%
\pgfpathcurveto{\pgfqpoint{2.408724in}{1.494607in}}{\pgfqpoint{2.416624in}{1.491335in}}{\pgfqpoint{2.424860in}{1.491335in}}%
\pgfpathclose%
\pgfusepath{stroke,fill}%
\end{pgfscope}%
\begin{pgfscope}%
\pgfpathrectangle{\pgfqpoint{0.100000in}{0.220728in}}{\pgfqpoint{3.696000in}{3.696000in}}%
\pgfusepath{clip}%
\pgfsetbuttcap%
\pgfsetroundjoin%
\definecolor{currentfill}{rgb}{0.121569,0.466667,0.705882}%
\pgfsetfillcolor{currentfill}%
\pgfsetfillopacity{0.981276}%
\pgfsetlinewidth{1.003750pt}%
\definecolor{currentstroke}{rgb}{0.121569,0.466667,0.705882}%
\pgfsetstrokecolor{currentstroke}%
\pgfsetstrokeopacity{0.981276}%
\pgfsetdash{}{0pt}%
\pgfpathmoveto{\pgfqpoint{2.418160in}{1.481913in}}%
\pgfpathcurveto{\pgfqpoint{2.426397in}{1.481913in}}{\pgfqpoint{2.434297in}{1.485185in}}{\pgfqpoint{2.440121in}{1.491009in}}%
\pgfpathcurveto{\pgfqpoint{2.445945in}{1.496833in}}{\pgfqpoint{2.449217in}{1.504733in}}{\pgfqpoint{2.449217in}{1.512969in}}%
\pgfpathcurveto{\pgfqpoint{2.449217in}{1.521206in}}{\pgfqpoint{2.445945in}{1.529106in}}{\pgfqpoint{2.440121in}{1.534930in}}%
\pgfpathcurveto{\pgfqpoint{2.434297in}{1.540754in}}{\pgfqpoint{2.426397in}{1.544026in}}{\pgfqpoint{2.418160in}{1.544026in}}%
\pgfpathcurveto{\pgfqpoint{2.409924in}{1.544026in}}{\pgfqpoint{2.402024in}{1.540754in}}{\pgfqpoint{2.396200in}{1.534930in}}%
\pgfpathcurveto{\pgfqpoint{2.390376in}{1.529106in}}{\pgfqpoint{2.387104in}{1.521206in}}{\pgfqpoint{2.387104in}{1.512969in}}%
\pgfpathcurveto{\pgfqpoint{2.387104in}{1.504733in}}{\pgfqpoint{2.390376in}{1.496833in}}{\pgfqpoint{2.396200in}{1.491009in}}%
\pgfpathcurveto{\pgfqpoint{2.402024in}{1.485185in}}{\pgfqpoint{2.409924in}{1.481913in}}{\pgfqpoint{2.418160in}{1.481913in}}%
\pgfpathclose%
\pgfusepath{stroke,fill}%
\end{pgfscope}%
\begin{pgfscope}%
\pgfpathrectangle{\pgfqpoint{0.100000in}{0.220728in}}{\pgfqpoint{3.696000in}{3.696000in}}%
\pgfusepath{clip}%
\pgfsetbuttcap%
\pgfsetroundjoin%
\definecolor{currentfill}{rgb}{0.121569,0.466667,0.705882}%
\pgfsetfillcolor{currentfill}%
\pgfsetfillopacity{0.982941}%
\pgfsetlinewidth{1.003750pt}%
\definecolor{currentstroke}{rgb}{0.121569,0.466667,0.705882}%
\pgfsetstrokecolor{currentstroke}%
\pgfsetstrokeopacity{0.982941}%
\pgfsetdash{}{0pt}%
\pgfpathmoveto{\pgfqpoint{2.269808in}{1.458130in}}%
\pgfpathcurveto{\pgfqpoint{2.278044in}{1.458130in}}{\pgfqpoint{2.285944in}{1.461402in}}{\pgfqpoint{2.291768in}{1.467226in}}%
\pgfpathcurveto{\pgfqpoint{2.297592in}{1.473050in}}{\pgfqpoint{2.300864in}{1.480950in}}{\pgfqpoint{2.300864in}{1.489186in}}%
\pgfpathcurveto{\pgfqpoint{2.300864in}{1.497422in}}{\pgfqpoint{2.297592in}{1.505322in}}{\pgfqpoint{2.291768in}{1.511146in}}%
\pgfpathcurveto{\pgfqpoint{2.285944in}{1.516970in}}{\pgfqpoint{2.278044in}{1.520243in}}{\pgfqpoint{2.269808in}{1.520243in}}%
\pgfpathcurveto{\pgfqpoint{2.261572in}{1.520243in}}{\pgfqpoint{2.253672in}{1.516970in}}{\pgfqpoint{2.247848in}{1.511146in}}%
\pgfpathcurveto{\pgfqpoint{2.242024in}{1.505322in}}{\pgfqpoint{2.238751in}{1.497422in}}{\pgfqpoint{2.238751in}{1.489186in}}%
\pgfpathcurveto{\pgfqpoint{2.238751in}{1.480950in}}{\pgfqpoint{2.242024in}{1.473050in}}{\pgfqpoint{2.247848in}{1.467226in}}%
\pgfpathcurveto{\pgfqpoint{2.253672in}{1.461402in}}{\pgfqpoint{2.261572in}{1.458130in}}{\pgfqpoint{2.269808in}{1.458130in}}%
\pgfpathclose%
\pgfusepath{stroke,fill}%
\end{pgfscope}%
\begin{pgfscope}%
\pgfpathrectangle{\pgfqpoint{0.100000in}{0.220728in}}{\pgfqpoint{3.696000in}{3.696000in}}%
\pgfusepath{clip}%
\pgfsetbuttcap%
\pgfsetroundjoin%
\definecolor{currentfill}{rgb}{0.121569,0.466667,0.705882}%
\pgfsetfillcolor{currentfill}%
\pgfsetfillopacity{0.983968}%
\pgfsetlinewidth{1.003750pt}%
\definecolor{currentstroke}{rgb}{0.121569,0.466667,0.705882}%
\pgfsetstrokecolor{currentstroke}%
\pgfsetstrokeopacity{0.983968}%
\pgfsetdash{}{0pt}%
\pgfpathmoveto{\pgfqpoint{2.411094in}{1.470002in}}%
\pgfpathcurveto{\pgfqpoint{2.419330in}{1.470002in}}{\pgfqpoint{2.427230in}{1.473274in}}{\pgfqpoint{2.433054in}{1.479098in}}%
\pgfpathcurveto{\pgfqpoint{2.438878in}{1.484922in}}{\pgfqpoint{2.442150in}{1.492822in}}{\pgfqpoint{2.442150in}{1.501058in}}%
\pgfpathcurveto{\pgfqpoint{2.442150in}{1.509295in}}{\pgfqpoint{2.438878in}{1.517195in}}{\pgfqpoint{2.433054in}{1.523019in}}%
\pgfpathcurveto{\pgfqpoint{2.427230in}{1.528843in}}{\pgfqpoint{2.419330in}{1.532115in}}{\pgfqpoint{2.411094in}{1.532115in}}%
\pgfpathcurveto{\pgfqpoint{2.402857in}{1.532115in}}{\pgfqpoint{2.394957in}{1.528843in}}{\pgfqpoint{2.389133in}{1.523019in}}%
\pgfpathcurveto{\pgfqpoint{2.383309in}{1.517195in}}{\pgfqpoint{2.380037in}{1.509295in}}{\pgfqpoint{2.380037in}{1.501058in}}%
\pgfpathcurveto{\pgfqpoint{2.380037in}{1.492822in}}{\pgfqpoint{2.383309in}{1.484922in}}{\pgfqpoint{2.389133in}{1.479098in}}%
\pgfpathcurveto{\pgfqpoint{2.394957in}{1.473274in}}{\pgfqpoint{2.402857in}{1.470002in}}{\pgfqpoint{2.411094in}{1.470002in}}%
\pgfpathclose%
\pgfusepath{stroke,fill}%
\end{pgfscope}%
\begin{pgfscope}%
\pgfpathrectangle{\pgfqpoint{0.100000in}{0.220728in}}{\pgfqpoint{3.696000in}{3.696000in}}%
\pgfusepath{clip}%
\pgfsetbuttcap%
\pgfsetroundjoin%
\definecolor{currentfill}{rgb}{0.121569,0.466667,0.705882}%
\pgfsetfillcolor{currentfill}%
\pgfsetfillopacity{0.985302}%
\pgfsetlinewidth{1.003750pt}%
\definecolor{currentstroke}{rgb}{0.121569,0.466667,0.705882}%
\pgfsetstrokecolor{currentstroke}%
\pgfsetstrokeopacity{0.985302}%
\pgfsetdash{}{0pt}%
\pgfpathmoveto{\pgfqpoint{2.406849in}{1.463267in}}%
\pgfpathcurveto{\pgfqpoint{2.415085in}{1.463267in}}{\pgfqpoint{2.422985in}{1.466539in}}{\pgfqpoint{2.428809in}{1.472363in}}%
\pgfpathcurveto{\pgfqpoint{2.434633in}{1.478187in}}{\pgfqpoint{2.437905in}{1.486087in}}{\pgfqpoint{2.437905in}{1.494323in}}%
\pgfpathcurveto{\pgfqpoint{2.437905in}{1.502560in}}{\pgfqpoint{2.434633in}{1.510460in}}{\pgfqpoint{2.428809in}{1.516284in}}%
\pgfpathcurveto{\pgfqpoint{2.422985in}{1.522108in}}{\pgfqpoint{2.415085in}{1.525380in}}{\pgfqpoint{2.406849in}{1.525380in}}%
\pgfpathcurveto{\pgfqpoint{2.398613in}{1.525380in}}{\pgfqpoint{2.390713in}{1.522108in}}{\pgfqpoint{2.384889in}{1.516284in}}%
\pgfpathcurveto{\pgfqpoint{2.379065in}{1.510460in}}{\pgfqpoint{2.375792in}{1.502560in}}{\pgfqpoint{2.375792in}{1.494323in}}%
\pgfpathcurveto{\pgfqpoint{2.375792in}{1.486087in}}{\pgfqpoint{2.379065in}{1.478187in}}{\pgfqpoint{2.384889in}{1.472363in}}%
\pgfpathcurveto{\pgfqpoint{2.390713in}{1.466539in}}{\pgfqpoint{2.398613in}{1.463267in}}{\pgfqpoint{2.406849in}{1.463267in}}%
\pgfpathclose%
\pgfusepath{stroke,fill}%
\end{pgfscope}%
\begin{pgfscope}%
\pgfpathrectangle{\pgfqpoint{0.100000in}{0.220728in}}{\pgfqpoint{3.696000in}{3.696000in}}%
\pgfusepath{clip}%
\pgfsetbuttcap%
\pgfsetroundjoin%
\definecolor{currentfill}{rgb}{0.121569,0.466667,0.705882}%
\pgfsetfillcolor{currentfill}%
\pgfsetfillopacity{0.985426}%
\pgfsetlinewidth{1.003750pt}%
\definecolor{currentstroke}{rgb}{0.121569,0.466667,0.705882}%
\pgfsetstrokecolor{currentstroke}%
\pgfsetstrokeopacity{0.985426}%
\pgfsetdash{}{0pt}%
\pgfpathmoveto{\pgfqpoint{2.282246in}{1.454459in}}%
\pgfpathcurveto{\pgfqpoint{2.290482in}{1.454459in}}{\pgfqpoint{2.298382in}{1.457732in}}{\pgfqpoint{2.304206in}{1.463556in}}%
\pgfpathcurveto{\pgfqpoint{2.310030in}{1.469380in}}{\pgfqpoint{2.313302in}{1.477280in}}{\pgfqpoint{2.313302in}{1.485516in}}%
\pgfpathcurveto{\pgfqpoint{2.313302in}{1.493752in}}{\pgfqpoint{2.310030in}{1.501652in}}{\pgfqpoint{2.304206in}{1.507476in}}%
\pgfpathcurveto{\pgfqpoint{2.298382in}{1.513300in}}{\pgfqpoint{2.290482in}{1.516572in}}{\pgfqpoint{2.282246in}{1.516572in}}%
\pgfpathcurveto{\pgfqpoint{2.274010in}{1.516572in}}{\pgfqpoint{2.266110in}{1.513300in}}{\pgfqpoint{2.260286in}{1.507476in}}%
\pgfpathcurveto{\pgfqpoint{2.254462in}{1.501652in}}{\pgfqpoint{2.251189in}{1.493752in}}{\pgfqpoint{2.251189in}{1.485516in}}%
\pgfpathcurveto{\pgfqpoint{2.251189in}{1.477280in}}{\pgfqpoint{2.254462in}{1.469380in}}{\pgfqpoint{2.260286in}{1.463556in}}%
\pgfpathcurveto{\pgfqpoint{2.266110in}{1.457732in}}{\pgfqpoint{2.274010in}{1.454459in}}{\pgfqpoint{2.282246in}{1.454459in}}%
\pgfpathclose%
\pgfusepath{stroke,fill}%
\end{pgfscope}%
\begin{pgfscope}%
\pgfpathrectangle{\pgfqpoint{0.100000in}{0.220728in}}{\pgfqpoint{3.696000in}{3.696000in}}%
\pgfusepath{clip}%
\pgfsetbuttcap%
\pgfsetroundjoin%
\definecolor{currentfill}{rgb}{0.121569,0.466667,0.705882}%
\pgfsetfillcolor{currentfill}%
\pgfsetfillopacity{0.986068}%
\pgfsetlinewidth{1.003750pt}%
\definecolor{currentstroke}{rgb}{0.121569,0.466667,0.705882}%
\pgfsetstrokecolor{currentstroke}%
\pgfsetstrokeopacity{0.986068}%
\pgfsetdash{}{0pt}%
\pgfpathmoveto{\pgfqpoint{2.404636in}{1.459533in}}%
\pgfpathcurveto{\pgfqpoint{2.412872in}{1.459533in}}{\pgfqpoint{2.420772in}{1.462806in}}{\pgfqpoint{2.426596in}{1.468630in}}%
\pgfpathcurveto{\pgfqpoint{2.432420in}{1.474454in}}{\pgfqpoint{2.435692in}{1.482354in}}{\pgfqpoint{2.435692in}{1.490590in}}%
\pgfpathcurveto{\pgfqpoint{2.435692in}{1.498826in}}{\pgfqpoint{2.432420in}{1.506726in}}{\pgfqpoint{2.426596in}{1.512550in}}%
\pgfpathcurveto{\pgfqpoint{2.420772in}{1.518374in}}{\pgfqpoint{2.412872in}{1.521646in}}{\pgfqpoint{2.404636in}{1.521646in}}%
\pgfpathcurveto{\pgfqpoint{2.396399in}{1.521646in}}{\pgfqpoint{2.388499in}{1.518374in}}{\pgfqpoint{2.382675in}{1.512550in}}%
\pgfpathcurveto{\pgfqpoint{2.376852in}{1.506726in}}{\pgfqpoint{2.373579in}{1.498826in}}{\pgfqpoint{2.373579in}{1.490590in}}%
\pgfpathcurveto{\pgfqpoint{2.373579in}{1.482354in}}{\pgfqpoint{2.376852in}{1.474454in}}{\pgfqpoint{2.382675in}{1.468630in}}%
\pgfpathcurveto{\pgfqpoint{2.388499in}{1.462806in}}{\pgfqpoint{2.396399in}{1.459533in}}{\pgfqpoint{2.404636in}{1.459533in}}%
\pgfpathclose%
\pgfusepath{stroke,fill}%
\end{pgfscope}%
\begin{pgfscope}%
\pgfpathrectangle{\pgfqpoint{0.100000in}{0.220728in}}{\pgfqpoint{3.696000in}{3.696000in}}%
\pgfusepath{clip}%
\pgfsetbuttcap%
\pgfsetroundjoin%
\definecolor{currentfill}{rgb}{0.121569,0.466667,0.705882}%
\pgfsetfillcolor{currentfill}%
\pgfsetfillopacity{0.986468}%
\pgfsetlinewidth{1.003750pt}%
\definecolor{currentstroke}{rgb}{0.121569,0.466667,0.705882}%
\pgfsetstrokecolor{currentstroke}%
\pgfsetstrokeopacity{0.986468}%
\pgfsetdash{}{0pt}%
\pgfpathmoveto{\pgfqpoint{2.403318in}{1.457513in}}%
\pgfpathcurveto{\pgfqpoint{2.411554in}{1.457513in}}{\pgfqpoint{2.419454in}{1.460786in}}{\pgfqpoint{2.425278in}{1.466609in}}%
\pgfpathcurveto{\pgfqpoint{2.431102in}{1.472433in}}{\pgfqpoint{2.434374in}{1.480333in}}{\pgfqpoint{2.434374in}{1.488570in}}%
\pgfpathcurveto{\pgfqpoint{2.434374in}{1.496806in}}{\pgfqpoint{2.431102in}{1.504706in}}{\pgfqpoint{2.425278in}{1.510530in}}%
\pgfpathcurveto{\pgfqpoint{2.419454in}{1.516354in}}{\pgfqpoint{2.411554in}{1.519626in}}{\pgfqpoint{2.403318in}{1.519626in}}%
\pgfpathcurveto{\pgfqpoint{2.395081in}{1.519626in}}{\pgfqpoint{2.387181in}{1.516354in}}{\pgfqpoint{2.381357in}{1.510530in}}%
\pgfpathcurveto{\pgfqpoint{2.375533in}{1.504706in}}{\pgfqpoint{2.372261in}{1.496806in}}{\pgfqpoint{2.372261in}{1.488570in}}%
\pgfpathcurveto{\pgfqpoint{2.372261in}{1.480333in}}{\pgfqpoint{2.375533in}{1.472433in}}{\pgfqpoint{2.381357in}{1.466609in}}%
\pgfpathcurveto{\pgfqpoint{2.387181in}{1.460786in}}{\pgfqpoint{2.395081in}{1.457513in}}{\pgfqpoint{2.403318in}{1.457513in}}%
\pgfpathclose%
\pgfusepath{stroke,fill}%
\end{pgfscope}%
\begin{pgfscope}%
\pgfpathrectangle{\pgfqpoint{0.100000in}{0.220728in}}{\pgfqpoint{3.696000in}{3.696000in}}%
\pgfusepath{clip}%
\pgfsetbuttcap%
\pgfsetroundjoin%
\definecolor{currentfill}{rgb}{0.121569,0.466667,0.705882}%
\pgfsetfillcolor{currentfill}%
\pgfsetfillopacity{0.986702}%
\pgfsetlinewidth{1.003750pt}%
\definecolor{currentstroke}{rgb}{0.121569,0.466667,0.705882}%
\pgfsetstrokecolor{currentstroke}%
\pgfsetstrokeopacity{0.986702}%
\pgfsetdash{}{0pt}%
\pgfpathmoveto{\pgfqpoint{2.402559in}{1.456525in}}%
\pgfpathcurveto{\pgfqpoint{2.410795in}{1.456525in}}{\pgfqpoint{2.418696in}{1.459797in}}{\pgfqpoint{2.424519in}{1.465621in}}%
\pgfpathcurveto{\pgfqpoint{2.430343in}{1.471445in}}{\pgfqpoint{2.433616in}{1.479345in}}{\pgfqpoint{2.433616in}{1.487582in}}%
\pgfpathcurveto{\pgfqpoint{2.433616in}{1.495818in}}{\pgfqpoint{2.430343in}{1.503718in}}{\pgfqpoint{2.424519in}{1.509542in}}%
\pgfpathcurveto{\pgfqpoint{2.418696in}{1.515366in}}{\pgfqpoint{2.410795in}{1.518638in}}{\pgfqpoint{2.402559in}{1.518638in}}%
\pgfpathcurveto{\pgfqpoint{2.394323in}{1.518638in}}{\pgfqpoint{2.386423in}{1.515366in}}{\pgfqpoint{2.380599in}{1.509542in}}%
\pgfpathcurveto{\pgfqpoint{2.374775in}{1.503718in}}{\pgfqpoint{2.371503in}{1.495818in}}{\pgfqpoint{2.371503in}{1.487582in}}%
\pgfpathcurveto{\pgfqpoint{2.371503in}{1.479345in}}{\pgfqpoint{2.374775in}{1.471445in}}{\pgfqpoint{2.380599in}{1.465621in}}%
\pgfpathcurveto{\pgfqpoint{2.386423in}{1.459797in}}{\pgfqpoint{2.394323in}{1.456525in}}{\pgfqpoint{2.402559in}{1.456525in}}%
\pgfpathclose%
\pgfusepath{stroke,fill}%
\end{pgfscope}%
\begin{pgfscope}%
\pgfpathrectangle{\pgfqpoint{0.100000in}{0.220728in}}{\pgfqpoint{3.696000in}{3.696000in}}%
\pgfusepath{clip}%
\pgfsetbuttcap%
\pgfsetroundjoin%
\definecolor{currentfill}{rgb}{0.121569,0.466667,0.705882}%
\pgfsetfillcolor{currentfill}%
\pgfsetfillopacity{0.986817}%
\pgfsetlinewidth{1.003750pt}%
\definecolor{currentstroke}{rgb}{0.121569,0.466667,0.705882}%
\pgfsetstrokecolor{currentstroke}%
\pgfsetstrokeopacity{0.986817}%
\pgfsetdash{}{0pt}%
\pgfpathmoveto{\pgfqpoint{2.402209in}{1.455823in}}%
\pgfpathcurveto{\pgfqpoint{2.410445in}{1.455823in}}{\pgfqpoint{2.418345in}{1.459096in}}{\pgfqpoint{2.424169in}{1.464920in}}%
\pgfpathcurveto{\pgfqpoint{2.429993in}{1.470744in}}{\pgfqpoint{2.433265in}{1.478644in}}{\pgfqpoint{2.433265in}{1.486880in}}%
\pgfpathcurveto{\pgfqpoint{2.433265in}{1.495116in}}{\pgfqpoint{2.429993in}{1.503016in}}{\pgfqpoint{2.424169in}{1.508840in}}%
\pgfpathcurveto{\pgfqpoint{2.418345in}{1.514664in}}{\pgfqpoint{2.410445in}{1.517936in}}{\pgfqpoint{2.402209in}{1.517936in}}%
\pgfpathcurveto{\pgfqpoint{2.393972in}{1.517936in}}{\pgfqpoint{2.386072in}{1.514664in}}{\pgfqpoint{2.380248in}{1.508840in}}%
\pgfpathcurveto{\pgfqpoint{2.374424in}{1.503016in}}{\pgfqpoint{2.371152in}{1.495116in}}{\pgfqpoint{2.371152in}{1.486880in}}%
\pgfpathcurveto{\pgfqpoint{2.371152in}{1.478644in}}{\pgfqpoint{2.374424in}{1.470744in}}{\pgfqpoint{2.380248in}{1.464920in}}%
\pgfpathcurveto{\pgfqpoint{2.386072in}{1.459096in}}{\pgfqpoint{2.393972in}{1.455823in}}{\pgfqpoint{2.402209in}{1.455823in}}%
\pgfpathclose%
\pgfusepath{stroke,fill}%
\end{pgfscope}%
\begin{pgfscope}%
\pgfpathrectangle{\pgfqpoint{0.100000in}{0.220728in}}{\pgfqpoint{3.696000in}{3.696000in}}%
\pgfusepath{clip}%
\pgfsetbuttcap%
\pgfsetroundjoin%
\definecolor{currentfill}{rgb}{0.121569,0.466667,0.705882}%
\pgfsetfillcolor{currentfill}%
\pgfsetfillopacity{0.987250}%
\pgfsetlinewidth{1.003750pt}%
\definecolor{currentstroke}{rgb}{0.121569,0.466667,0.705882}%
\pgfsetstrokecolor{currentstroke}%
\pgfsetstrokeopacity{0.987250}%
\pgfsetdash{}{0pt}%
\pgfpathmoveto{\pgfqpoint{2.289558in}{1.447590in}}%
\pgfpathcurveto{\pgfqpoint{2.297794in}{1.447590in}}{\pgfqpoint{2.305694in}{1.450862in}}{\pgfqpoint{2.311518in}{1.456686in}}%
\pgfpathcurveto{\pgfqpoint{2.317342in}{1.462510in}}{\pgfqpoint{2.320614in}{1.470410in}}{\pgfqpoint{2.320614in}{1.478646in}}%
\pgfpathcurveto{\pgfqpoint{2.320614in}{1.486882in}}{\pgfqpoint{2.317342in}{1.494782in}}{\pgfqpoint{2.311518in}{1.500606in}}%
\pgfpathcurveto{\pgfqpoint{2.305694in}{1.506430in}}{\pgfqpoint{2.297794in}{1.509703in}}{\pgfqpoint{2.289558in}{1.509703in}}%
\pgfpathcurveto{\pgfqpoint{2.281322in}{1.509703in}}{\pgfqpoint{2.273422in}{1.506430in}}{\pgfqpoint{2.267598in}{1.500606in}}%
\pgfpathcurveto{\pgfqpoint{2.261774in}{1.494782in}}{\pgfqpoint{2.258501in}{1.486882in}}{\pgfqpoint{2.258501in}{1.478646in}}%
\pgfpathcurveto{\pgfqpoint{2.258501in}{1.470410in}}{\pgfqpoint{2.261774in}{1.462510in}}{\pgfqpoint{2.267598in}{1.456686in}}%
\pgfpathcurveto{\pgfqpoint{2.273422in}{1.450862in}}{\pgfqpoint{2.281322in}{1.447590in}}{\pgfqpoint{2.289558in}{1.447590in}}%
\pgfpathclose%
\pgfusepath{stroke,fill}%
\end{pgfscope}%
\begin{pgfscope}%
\pgfpathrectangle{\pgfqpoint{0.100000in}{0.220728in}}{\pgfqpoint{3.696000in}{3.696000in}}%
\pgfusepath{clip}%
\pgfsetbuttcap%
\pgfsetroundjoin%
\definecolor{currentfill}{rgb}{0.121569,0.466667,0.705882}%
\pgfsetfillcolor{currentfill}%
\pgfsetfillopacity{0.987763}%
\pgfsetlinewidth{1.003750pt}%
\definecolor{currentstroke}{rgb}{0.121569,0.466667,0.705882}%
\pgfsetstrokecolor{currentstroke}%
\pgfsetstrokeopacity{0.987763}%
\pgfsetdash{}{0pt}%
\pgfpathmoveto{\pgfqpoint{2.291376in}{1.446091in}}%
\pgfpathcurveto{\pgfqpoint{2.299613in}{1.446091in}}{\pgfqpoint{2.307513in}{1.449364in}}{\pgfqpoint{2.313337in}{1.455188in}}%
\pgfpathcurveto{\pgfqpoint{2.319160in}{1.461012in}}{\pgfqpoint{2.322433in}{1.468912in}}{\pgfqpoint{2.322433in}{1.477148in}}%
\pgfpathcurveto{\pgfqpoint{2.322433in}{1.485384in}}{\pgfqpoint{2.319160in}{1.493284in}}{\pgfqpoint{2.313337in}{1.499108in}}%
\pgfpathcurveto{\pgfqpoint{2.307513in}{1.504932in}}{\pgfqpoint{2.299613in}{1.508204in}}{\pgfqpoint{2.291376in}{1.508204in}}%
\pgfpathcurveto{\pgfqpoint{2.283140in}{1.508204in}}{\pgfqpoint{2.275240in}{1.504932in}}{\pgfqpoint{2.269416in}{1.499108in}}%
\pgfpathcurveto{\pgfqpoint{2.263592in}{1.493284in}}{\pgfqpoint{2.260320in}{1.485384in}}{\pgfqpoint{2.260320in}{1.477148in}}%
\pgfpathcurveto{\pgfqpoint{2.260320in}{1.468912in}}{\pgfqpoint{2.263592in}{1.461012in}}{\pgfqpoint{2.269416in}{1.455188in}}%
\pgfpathcurveto{\pgfqpoint{2.275240in}{1.449364in}}{\pgfqpoint{2.283140in}{1.446091in}}{\pgfqpoint{2.291376in}{1.446091in}}%
\pgfpathclose%
\pgfusepath{stroke,fill}%
\end{pgfscope}%
\begin{pgfscope}%
\pgfpathrectangle{\pgfqpoint{0.100000in}{0.220728in}}{\pgfqpoint{3.696000in}{3.696000in}}%
\pgfusepath{clip}%
\pgfsetbuttcap%
\pgfsetroundjoin%
\definecolor{currentfill}{rgb}{0.121569,0.466667,0.705882}%
\pgfsetfillcolor{currentfill}%
\pgfsetfillopacity{0.987804}%
\pgfsetlinewidth{1.003750pt}%
\definecolor{currentstroke}{rgb}{0.121569,0.466667,0.705882}%
\pgfsetstrokecolor{currentstroke}%
\pgfsetstrokeopacity{0.987804}%
\pgfsetdash{}{0pt}%
\pgfpathmoveto{\pgfqpoint{2.398849in}{1.451234in}}%
\pgfpathcurveto{\pgfqpoint{2.407085in}{1.451234in}}{\pgfqpoint{2.414985in}{1.454506in}}{\pgfqpoint{2.420809in}{1.460330in}}%
\pgfpathcurveto{\pgfqpoint{2.426633in}{1.466154in}}{\pgfqpoint{2.429905in}{1.474054in}}{\pgfqpoint{2.429905in}{1.482291in}}%
\pgfpathcurveto{\pgfqpoint{2.429905in}{1.490527in}}{\pgfqpoint{2.426633in}{1.498427in}}{\pgfqpoint{2.420809in}{1.504251in}}%
\pgfpathcurveto{\pgfqpoint{2.414985in}{1.510075in}}{\pgfqpoint{2.407085in}{1.513347in}}{\pgfqpoint{2.398849in}{1.513347in}}%
\pgfpathcurveto{\pgfqpoint{2.390613in}{1.513347in}}{\pgfqpoint{2.382713in}{1.510075in}}{\pgfqpoint{2.376889in}{1.504251in}}%
\pgfpathcurveto{\pgfqpoint{2.371065in}{1.498427in}}{\pgfqpoint{2.367792in}{1.490527in}}{\pgfqpoint{2.367792in}{1.482291in}}%
\pgfpathcurveto{\pgfqpoint{2.367792in}{1.474054in}}{\pgfqpoint{2.371065in}{1.466154in}}{\pgfqpoint{2.376889in}{1.460330in}}%
\pgfpathcurveto{\pgfqpoint{2.382713in}{1.454506in}}{\pgfqpoint{2.390613in}{1.451234in}}{\pgfqpoint{2.398849in}{1.451234in}}%
\pgfpathclose%
\pgfusepath{stroke,fill}%
\end{pgfscope}%
\begin{pgfscope}%
\pgfpathrectangle{\pgfqpoint{0.100000in}{0.220728in}}{\pgfqpoint{3.696000in}{3.696000in}}%
\pgfusepath{clip}%
\pgfsetbuttcap%
\pgfsetroundjoin%
\definecolor{currentfill}{rgb}{0.121569,0.466667,0.705882}%
\pgfsetfillcolor{currentfill}%
\pgfsetfillopacity{0.988380}%
\pgfsetlinewidth{1.003750pt}%
\definecolor{currentstroke}{rgb}{0.121569,0.466667,0.705882}%
\pgfsetstrokecolor{currentstroke}%
\pgfsetstrokeopacity{0.988380}%
\pgfsetdash{}{0pt}%
\pgfpathmoveto{\pgfqpoint{2.397337in}{1.448397in}}%
\pgfpathcurveto{\pgfqpoint{2.405573in}{1.448397in}}{\pgfqpoint{2.413473in}{1.451670in}}{\pgfqpoint{2.419297in}{1.457494in}}%
\pgfpathcurveto{\pgfqpoint{2.425121in}{1.463318in}}{\pgfqpoint{2.428393in}{1.471218in}}{\pgfqpoint{2.428393in}{1.479454in}}%
\pgfpathcurveto{\pgfqpoint{2.428393in}{1.487690in}}{\pgfqpoint{2.425121in}{1.495590in}}{\pgfqpoint{2.419297in}{1.501414in}}%
\pgfpathcurveto{\pgfqpoint{2.413473in}{1.507238in}}{\pgfqpoint{2.405573in}{1.510510in}}{\pgfqpoint{2.397337in}{1.510510in}}%
\pgfpathcurveto{\pgfqpoint{2.389100in}{1.510510in}}{\pgfqpoint{2.381200in}{1.507238in}}{\pgfqpoint{2.375376in}{1.501414in}}%
\pgfpathcurveto{\pgfqpoint{2.369552in}{1.495590in}}{\pgfqpoint{2.366280in}{1.487690in}}{\pgfqpoint{2.366280in}{1.479454in}}%
\pgfpathcurveto{\pgfqpoint{2.366280in}{1.471218in}}{\pgfqpoint{2.369552in}{1.463318in}}{\pgfqpoint{2.375376in}{1.457494in}}%
\pgfpathcurveto{\pgfqpoint{2.381200in}{1.451670in}}{\pgfqpoint{2.389100in}{1.448397in}}{\pgfqpoint{2.397337in}{1.448397in}}%
\pgfpathclose%
\pgfusepath{stroke,fill}%
\end{pgfscope}%
\begin{pgfscope}%
\pgfpathrectangle{\pgfqpoint{0.100000in}{0.220728in}}{\pgfqpoint{3.696000in}{3.696000in}}%
\pgfusepath{clip}%
\pgfsetbuttcap%
\pgfsetroundjoin%
\definecolor{currentfill}{rgb}{0.121569,0.466667,0.705882}%
\pgfsetfillcolor{currentfill}%
\pgfsetfillopacity{0.988655}%
\pgfsetlinewidth{1.003750pt}%
\definecolor{currentstroke}{rgb}{0.121569,0.466667,0.705882}%
\pgfsetstrokecolor{currentstroke}%
\pgfsetstrokeopacity{0.988655}%
\pgfsetdash{}{0pt}%
\pgfpathmoveto{\pgfqpoint{2.293931in}{1.441910in}}%
\pgfpathcurveto{\pgfqpoint{2.302167in}{1.441910in}}{\pgfqpoint{2.310067in}{1.445183in}}{\pgfqpoint{2.315891in}{1.451006in}}%
\pgfpathcurveto{\pgfqpoint{2.321715in}{1.456830in}}{\pgfqpoint{2.324987in}{1.464730in}}{\pgfqpoint{2.324987in}{1.472967in}}%
\pgfpathcurveto{\pgfqpoint{2.324987in}{1.481203in}}{\pgfqpoint{2.321715in}{1.489103in}}{\pgfqpoint{2.315891in}{1.494927in}}%
\pgfpathcurveto{\pgfqpoint{2.310067in}{1.500751in}}{\pgfqpoint{2.302167in}{1.504023in}}{\pgfqpoint{2.293931in}{1.504023in}}%
\pgfpathcurveto{\pgfqpoint{2.285695in}{1.504023in}}{\pgfqpoint{2.277795in}{1.500751in}}{\pgfqpoint{2.271971in}{1.494927in}}%
\pgfpathcurveto{\pgfqpoint{2.266147in}{1.489103in}}{\pgfqpoint{2.262874in}{1.481203in}}{\pgfqpoint{2.262874in}{1.472967in}}%
\pgfpathcurveto{\pgfqpoint{2.262874in}{1.464730in}}{\pgfqpoint{2.266147in}{1.456830in}}{\pgfqpoint{2.271971in}{1.451006in}}%
\pgfpathcurveto{\pgfqpoint{2.277795in}{1.445183in}}{\pgfqpoint{2.285695in}{1.441910in}}{\pgfqpoint{2.293931in}{1.441910in}}%
\pgfpathclose%
\pgfusepath{stroke,fill}%
\end{pgfscope}%
\begin{pgfscope}%
\pgfpathrectangle{\pgfqpoint{0.100000in}{0.220728in}}{\pgfqpoint{3.696000in}{3.696000in}}%
\pgfusepath{clip}%
\pgfsetbuttcap%
\pgfsetroundjoin%
\definecolor{currentfill}{rgb}{0.121569,0.466667,0.705882}%
\pgfsetfillcolor{currentfill}%
\pgfsetfillopacity{0.989558}%
\pgfsetlinewidth{1.003750pt}%
\definecolor{currentstroke}{rgb}{0.121569,0.466667,0.705882}%
\pgfsetstrokecolor{currentstroke}%
\pgfsetstrokeopacity{0.989558}%
\pgfsetdash{}{0pt}%
\pgfpathmoveto{\pgfqpoint{2.393332in}{1.442929in}}%
\pgfpathcurveto{\pgfqpoint{2.401569in}{1.442929in}}{\pgfqpoint{2.409469in}{1.446201in}}{\pgfqpoint{2.415293in}{1.452025in}}%
\pgfpathcurveto{\pgfqpoint{2.421117in}{1.457849in}}{\pgfqpoint{2.424389in}{1.465749in}}{\pgfqpoint{2.424389in}{1.473985in}}%
\pgfpathcurveto{\pgfqpoint{2.424389in}{1.482222in}}{\pgfqpoint{2.421117in}{1.490122in}}{\pgfqpoint{2.415293in}{1.495946in}}%
\pgfpathcurveto{\pgfqpoint{2.409469in}{1.501770in}}{\pgfqpoint{2.401569in}{1.505042in}}{\pgfqpoint{2.393332in}{1.505042in}}%
\pgfpathcurveto{\pgfqpoint{2.385096in}{1.505042in}}{\pgfqpoint{2.377196in}{1.501770in}}{\pgfqpoint{2.371372in}{1.495946in}}%
\pgfpathcurveto{\pgfqpoint{2.365548in}{1.490122in}}{\pgfqpoint{2.362276in}{1.482222in}}{\pgfqpoint{2.362276in}{1.473985in}}%
\pgfpathcurveto{\pgfqpoint{2.362276in}{1.465749in}}{\pgfqpoint{2.365548in}{1.457849in}}{\pgfqpoint{2.371372in}{1.452025in}}%
\pgfpathcurveto{\pgfqpoint{2.377196in}{1.446201in}}{\pgfqpoint{2.385096in}{1.442929in}}{\pgfqpoint{2.393332in}{1.442929in}}%
\pgfpathclose%
\pgfusepath{stroke,fill}%
\end{pgfscope}%
\begin{pgfscope}%
\pgfpathrectangle{\pgfqpoint{0.100000in}{0.220728in}}{\pgfqpoint{3.696000in}{3.696000in}}%
\pgfusepath{clip}%
\pgfsetbuttcap%
\pgfsetroundjoin%
\definecolor{currentfill}{rgb}{0.121569,0.466667,0.705882}%
\pgfsetfillcolor{currentfill}%
\pgfsetfillopacity{0.990186}%
\pgfsetlinewidth{1.003750pt}%
\definecolor{currentstroke}{rgb}{0.121569,0.466667,0.705882}%
\pgfsetstrokecolor{currentstroke}%
\pgfsetstrokeopacity{0.990186}%
\pgfsetdash{}{0pt}%
\pgfpathmoveto{\pgfqpoint{2.297471in}{1.432695in}}%
\pgfpathcurveto{\pgfqpoint{2.305708in}{1.432695in}}{\pgfqpoint{2.313608in}{1.435967in}}{\pgfqpoint{2.319432in}{1.441791in}}%
\pgfpathcurveto{\pgfqpoint{2.325256in}{1.447615in}}{\pgfqpoint{2.328528in}{1.455515in}}{\pgfqpoint{2.328528in}{1.463751in}}%
\pgfpathcurveto{\pgfqpoint{2.328528in}{1.471987in}}{\pgfqpoint{2.325256in}{1.479888in}}{\pgfqpoint{2.319432in}{1.485711in}}%
\pgfpathcurveto{\pgfqpoint{2.313608in}{1.491535in}}{\pgfqpoint{2.305708in}{1.494808in}}{\pgfqpoint{2.297471in}{1.494808in}}%
\pgfpathcurveto{\pgfqpoint{2.289235in}{1.494808in}}{\pgfqpoint{2.281335in}{1.491535in}}{\pgfqpoint{2.275511in}{1.485711in}}%
\pgfpathcurveto{\pgfqpoint{2.269687in}{1.479888in}}{\pgfqpoint{2.266415in}{1.471987in}}{\pgfqpoint{2.266415in}{1.463751in}}%
\pgfpathcurveto{\pgfqpoint{2.266415in}{1.455515in}}{\pgfqpoint{2.269687in}{1.447615in}}{\pgfqpoint{2.275511in}{1.441791in}}%
\pgfpathcurveto{\pgfqpoint{2.281335in}{1.435967in}}{\pgfqpoint{2.289235in}{1.432695in}}{\pgfqpoint{2.297471in}{1.432695in}}%
\pgfpathclose%
\pgfusepath{stroke,fill}%
\end{pgfscope}%
\begin{pgfscope}%
\pgfpathrectangle{\pgfqpoint{0.100000in}{0.220728in}}{\pgfqpoint{3.696000in}{3.696000in}}%
\pgfusepath{clip}%
\pgfsetbuttcap%
\pgfsetroundjoin%
\definecolor{currentfill}{rgb}{0.121569,0.466667,0.705882}%
\pgfsetfillcolor{currentfill}%
\pgfsetfillopacity{0.990230}%
\pgfsetlinewidth{1.003750pt}%
\definecolor{currentstroke}{rgb}{0.121569,0.466667,0.705882}%
\pgfsetstrokecolor{currentstroke}%
\pgfsetstrokeopacity{0.990230}%
\pgfsetdash{}{0pt}%
\pgfpathmoveto{\pgfqpoint{2.391362in}{1.439678in}}%
\pgfpathcurveto{\pgfqpoint{2.399598in}{1.439678in}}{\pgfqpoint{2.407498in}{1.442950in}}{\pgfqpoint{2.413322in}{1.448774in}}%
\pgfpathcurveto{\pgfqpoint{2.419146in}{1.454598in}}{\pgfqpoint{2.422418in}{1.462498in}}{\pgfqpoint{2.422418in}{1.470734in}}%
\pgfpathcurveto{\pgfqpoint{2.422418in}{1.478970in}}{\pgfqpoint{2.419146in}{1.486870in}}{\pgfqpoint{2.413322in}{1.492694in}}%
\pgfpathcurveto{\pgfqpoint{2.407498in}{1.498518in}}{\pgfqpoint{2.399598in}{1.501791in}}{\pgfqpoint{2.391362in}{1.501791in}}%
\pgfpathcurveto{\pgfqpoint{2.383125in}{1.501791in}}{\pgfqpoint{2.375225in}{1.498518in}}{\pgfqpoint{2.369401in}{1.492694in}}%
\pgfpathcurveto{\pgfqpoint{2.363578in}{1.486870in}}{\pgfqpoint{2.360305in}{1.478970in}}{\pgfqpoint{2.360305in}{1.470734in}}%
\pgfpathcurveto{\pgfqpoint{2.360305in}{1.462498in}}{\pgfqpoint{2.363578in}{1.454598in}}{\pgfqpoint{2.369401in}{1.448774in}}%
\pgfpathcurveto{\pgfqpoint{2.375225in}{1.442950in}}{\pgfqpoint{2.383125in}{1.439678in}}{\pgfqpoint{2.391362in}{1.439678in}}%
\pgfpathclose%
\pgfusepath{stroke,fill}%
\end{pgfscope}%
\begin{pgfscope}%
\pgfpathrectangle{\pgfqpoint{0.100000in}{0.220728in}}{\pgfqpoint{3.696000in}{3.696000in}}%
\pgfusepath{clip}%
\pgfsetbuttcap%
\pgfsetroundjoin%
\definecolor{currentfill}{rgb}{0.121569,0.466667,0.705882}%
\pgfsetfillcolor{currentfill}%
\pgfsetfillopacity{0.990554}%
\pgfsetlinewidth{1.003750pt}%
\definecolor{currentstroke}{rgb}{0.121569,0.466667,0.705882}%
\pgfsetstrokecolor{currentstroke}%
\pgfsetstrokeopacity{0.990554}%
\pgfsetdash{}{0pt}%
\pgfpathmoveto{\pgfqpoint{2.390191in}{1.437795in}}%
\pgfpathcurveto{\pgfqpoint{2.398427in}{1.437795in}}{\pgfqpoint{2.406327in}{1.441068in}}{\pgfqpoint{2.412151in}{1.446892in}}%
\pgfpathcurveto{\pgfqpoint{2.417975in}{1.452716in}}{\pgfqpoint{2.421247in}{1.460616in}}{\pgfqpoint{2.421247in}{1.468852in}}%
\pgfpathcurveto{\pgfqpoint{2.421247in}{1.477088in}}{\pgfqpoint{2.417975in}{1.484988in}}{\pgfqpoint{2.412151in}{1.490812in}}%
\pgfpathcurveto{\pgfqpoint{2.406327in}{1.496636in}}{\pgfqpoint{2.398427in}{1.499908in}}{\pgfqpoint{2.390191in}{1.499908in}}%
\pgfpathcurveto{\pgfqpoint{2.381954in}{1.499908in}}{\pgfqpoint{2.374054in}{1.496636in}}{\pgfqpoint{2.368230in}{1.490812in}}%
\pgfpathcurveto{\pgfqpoint{2.362406in}{1.484988in}}{\pgfqpoint{2.359134in}{1.477088in}}{\pgfqpoint{2.359134in}{1.468852in}}%
\pgfpathcurveto{\pgfqpoint{2.359134in}{1.460616in}}{\pgfqpoint{2.362406in}{1.452716in}}{\pgfqpoint{2.368230in}{1.446892in}}%
\pgfpathcurveto{\pgfqpoint{2.374054in}{1.441068in}}{\pgfqpoint{2.381954in}{1.437795in}}{\pgfqpoint{2.390191in}{1.437795in}}%
\pgfpathclose%
\pgfusepath{stroke,fill}%
\end{pgfscope}%
\begin{pgfscope}%
\pgfpathrectangle{\pgfqpoint{0.100000in}{0.220728in}}{\pgfqpoint{3.696000in}{3.696000in}}%
\pgfusepath{clip}%
\pgfsetbuttcap%
\pgfsetroundjoin%
\definecolor{currentfill}{rgb}{0.121569,0.466667,0.705882}%
\pgfsetfillcolor{currentfill}%
\pgfsetfillopacity{0.990761}%
\pgfsetlinewidth{1.003750pt}%
\definecolor{currentstroke}{rgb}{0.121569,0.466667,0.705882}%
\pgfsetstrokecolor{currentstroke}%
\pgfsetstrokeopacity{0.990761}%
\pgfsetdash{}{0pt}%
\pgfpathmoveto{\pgfqpoint{2.389517in}{1.436934in}}%
\pgfpathcurveto{\pgfqpoint{2.397754in}{1.436934in}}{\pgfqpoint{2.405654in}{1.440207in}}{\pgfqpoint{2.411478in}{1.446031in}}%
\pgfpathcurveto{\pgfqpoint{2.417302in}{1.451855in}}{\pgfqpoint{2.420574in}{1.459755in}}{\pgfqpoint{2.420574in}{1.467991in}}%
\pgfpathcurveto{\pgfqpoint{2.420574in}{1.476227in}}{\pgfqpoint{2.417302in}{1.484127in}}{\pgfqpoint{2.411478in}{1.489951in}}%
\pgfpathcurveto{\pgfqpoint{2.405654in}{1.495775in}}{\pgfqpoint{2.397754in}{1.499047in}}{\pgfqpoint{2.389517in}{1.499047in}}%
\pgfpathcurveto{\pgfqpoint{2.381281in}{1.499047in}}{\pgfqpoint{2.373381in}{1.495775in}}{\pgfqpoint{2.367557in}{1.489951in}}%
\pgfpathcurveto{\pgfqpoint{2.361733in}{1.484127in}}{\pgfqpoint{2.358461in}{1.476227in}}{\pgfqpoint{2.358461in}{1.467991in}}%
\pgfpathcurveto{\pgfqpoint{2.358461in}{1.459755in}}{\pgfqpoint{2.361733in}{1.451855in}}{\pgfqpoint{2.367557in}{1.446031in}}%
\pgfpathcurveto{\pgfqpoint{2.373381in}{1.440207in}}{\pgfqpoint{2.381281in}{1.436934in}}{\pgfqpoint{2.389517in}{1.436934in}}%
\pgfpathclose%
\pgfusepath{stroke,fill}%
\end{pgfscope}%
\begin{pgfscope}%
\pgfpathrectangle{\pgfqpoint{0.100000in}{0.220728in}}{\pgfqpoint{3.696000in}{3.696000in}}%
\pgfusepath{clip}%
\pgfsetbuttcap%
\pgfsetroundjoin%
\definecolor{currentfill}{rgb}{0.121569,0.466667,0.705882}%
\pgfsetfillcolor{currentfill}%
\pgfsetfillopacity{0.990860}%
\pgfsetlinewidth{1.003750pt}%
\definecolor{currentstroke}{rgb}{0.121569,0.466667,0.705882}%
\pgfsetstrokecolor{currentstroke}%
\pgfsetstrokeopacity{0.990860}%
\pgfsetdash{}{0pt}%
\pgfpathmoveto{\pgfqpoint{2.299212in}{1.428944in}}%
\pgfpathcurveto{\pgfqpoint{2.307449in}{1.428944in}}{\pgfqpoint{2.315349in}{1.432216in}}{\pgfqpoint{2.321173in}{1.438040in}}%
\pgfpathcurveto{\pgfqpoint{2.326996in}{1.443864in}}{\pgfqpoint{2.330269in}{1.451764in}}{\pgfqpoint{2.330269in}{1.460000in}}%
\pgfpathcurveto{\pgfqpoint{2.330269in}{1.468237in}}{\pgfqpoint{2.326996in}{1.476137in}}{\pgfqpoint{2.321173in}{1.481961in}}%
\pgfpathcurveto{\pgfqpoint{2.315349in}{1.487785in}}{\pgfqpoint{2.307449in}{1.491057in}}{\pgfqpoint{2.299212in}{1.491057in}}%
\pgfpathcurveto{\pgfqpoint{2.290976in}{1.491057in}}{\pgfqpoint{2.283076in}{1.487785in}}{\pgfqpoint{2.277252in}{1.481961in}}%
\pgfpathcurveto{\pgfqpoint{2.271428in}{1.476137in}}{\pgfqpoint{2.268156in}{1.468237in}}{\pgfqpoint{2.268156in}{1.460000in}}%
\pgfpathcurveto{\pgfqpoint{2.268156in}{1.451764in}}{\pgfqpoint{2.271428in}{1.443864in}}{\pgfqpoint{2.277252in}{1.438040in}}%
\pgfpathcurveto{\pgfqpoint{2.283076in}{1.432216in}}{\pgfqpoint{2.290976in}{1.428944in}}{\pgfqpoint{2.299212in}{1.428944in}}%
\pgfpathclose%
\pgfusepath{stroke,fill}%
\end{pgfscope}%
\begin{pgfscope}%
\pgfpathrectangle{\pgfqpoint{0.100000in}{0.220728in}}{\pgfqpoint{3.696000in}{3.696000in}}%
\pgfusepath{clip}%
\pgfsetbuttcap%
\pgfsetroundjoin%
\definecolor{currentfill}{rgb}{0.121569,0.466667,0.705882}%
\pgfsetfillcolor{currentfill}%
\pgfsetfillopacity{0.990877}%
\pgfsetlinewidth{1.003750pt}%
\definecolor{currentstroke}{rgb}{0.121569,0.466667,0.705882}%
\pgfsetstrokecolor{currentstroke}%
\pgfsetstrokeopacity{0.990877}%
\pgfsetdash{}{0pt}%
\pgfpathmoveto{\pgfqpoint{2.389188in}{1.436411in}}%
\pgfpathcurveto{\pgfqpoint{2.397424in}{1.436411in}}{\pgfqpoint{2.405324in}{1.439684in}}{\pgfqpoint{2.411148in}{1.445508in}}%
\pgfpathcurveto{\pgfqpoint{2.416972in}{1.451331in}}{\pgfqpoint{2.420245in}{1.459232in}}{\pgfqpoint{2.420245in}{1.467468in}}%
\pgfpathcurveto{\pgfqpoint{2.420245in}{1.475704in}}{\pgfqpoint{2.416972in}{1.483604in}}{\pgfqpoint{2.411148in}{1.489428in}}%
\pgfpathcurveto{\pgfqpoint{2.405324in}{1.495252in}}{\pgfqpoint{2.397424in}{1.498524in}}{\pgfqpoint{2.389188in}{1.498524in}}%
\pgfpathcurveto{\pgfqpoint{2.380952in}{1.498524in}}{\pgfqpoint{2.373052in}{1.495252in}}{\pgfqpoint{2.367228in}{1.489428in}}%
\pgfpathcurveto{\pgfqpoint{2.361404in}{1.483604in}}{\pgfqpoint{2.358132in}{1.475704in}}{\pgfqpoint{2.358132in}{1.467468in}}%
\pgfpathcurveto{\pgfqpoint{2.358132in}{1.459232in}}{\pgfqpoint{2.361404in}{1.451331in}}{\pgfqpoint{2.367228in}{1.445508in}}%
\pgfpathcurveto{\pgfqpoint{2.373052in}{1.439684in}}{\pgfqpoint{2.380952in}{1.436411in}}{\pgfqpoint{2.389188in}{1.436411in}}%
\pgfpathclose%
\pgfusepath{stroke,fill}%
\end{pgfscope}%
\begin{pgfscope}%
\pgfpathrectangle{\pgfqpoint{0.100000in}{0.220728in}}{\pgfqpoint{3.696000in}{3.696000in}}%
\pgfusepath{clip}%
\pgfsetbuttcap%
\pgfsetroundjoin%
\definecolor{currentfill}{rgb}{0.121569,0.466667,0.705882}%
\pgfsetfillcolor{currentfill}%
\pgfsetfillopacity{0.990935}%
\pgfsetlinewidth{1.003750pt}%
\definecolor{currentstroke}{rgb}{0.121569,0.466667,0.705882}%
\pgfsetstrokecolor{currentstroke}%
\pgfsetstrokeopacity{0.990935}%
\pgfsetdash{}{0pt}%
\pgfpathmoveto{\pgfqpoint{2.388985in}{1.436131in}}%
\pgfpathcurveto{\pgfqpoint{2.397221in}{1.436131in}}{\pgfqpoint{2.405121in}{1.439404in}}{\pgfqpoint{2.410945in}{1.445227in}}%
\pgfpathcurveto{\pgfqpoint{2.416769in}{1.451051in}}{\pgfqpoint{2.420041in}{1.458951in}}{\pgfqpoint{2.420041in}{1.467188in}}%
\pgfpathcurveto{\pgfqpoint{2.420041in}{1.475424in}}{\pgfqpoint{2.416769in}{1.483324in}}{\pgfqpoint{2.410945in}{1.489148in}}%
\pgfpathcurveto{\pgfqpoint{2.405121in}{1.494972in}}{\pgfqpoint{2.397221in}{1.498244in}}{\pgfqpoint{2.388985in}{1.498244in}}%
\pgfpathcurveto{\pgfqpoint{2.380748in}{1.498244in}}{\pgfqpoint{2.372848in}{1.494972in}}{\pgfqpoint{2.367024in}{1.489148in}}%
\pgfpathcurveto{\pgfqpoint{2.361201in}{1.483324in}}{\pgfqpoint{2.357928in}{1.475424in}}{\pgfqpoint{2.357928in}{1.467188in}}%
\pgfpathcurveto{\pgfqpoint{2.357928in}{1.458951in}}{\pgfqpoint{2.361201in}{1.451051in}}{\pgfqpoint{2.367024in}{1.445227in}}%
\pgfpathcurveto{\pgfqpoint{2.372848in}{1.439404in}}{\pgfqpoint{2.380748in}{1.436131in}}{\pgfqpoint{2.388985in}{1.436131in}}%
\pgfpathclose%
\pgfusepath{stroke,fill}%
\end{pgfscope}%
\begin{pgfscope}%
\pgfpathrectangle{\pgfqpoint{0.100000in}{0.220728in}}{\pgfqpoint{3.696000in}{3.696000in}}%
\pgfusepath{clip}%
\pgfsetbuttcap%
\pgfsetroundjoin%
\definecolor{currentfill}{rgb}{0.121569,0.466667,0.705882}%
\pgfsetfillcolor{currentfill}%
\pgfsetfillopacity{0.990967}%
\pgfsetlinewidth{1.003750pt}%
\definecolor{currentstroke}{rgb}{0.121569,0.466667,0.705882}%
\pgfsetstrokecolor{currentstroke}%
\pgfsetstrokeopacity{0.990967}%
\pgfsetdash{}{0pt}%
\pgfpathmoveto{\pgfqpoint{2.388890in}{1.435953in}}%
\pgfpathcurveto{\pgfqpoint{2.397126in}{1.435953in}}{\pgfqpoint{2.405026in}{1.439225in}}{\pgfqpoint{2.410850in}{1.445049in}}%
\pgfpathcurveto{\pgfqpoint{2.416674in}{1.450873in}}{\pgfqpoint{2.419947in}{1.458773in}}{\pgfqpoint{2.419947in}{1.467009in}}%
\pgfpathcurveto{\pgfqpoint{2.419947in}{1.475245in}}{\pgfqpoint{2.416674in}{1.483146in}}{\pgfqpoint{2.410850in}{1.488969in}}%
\pgfpathcurveto{\pgfqpoint{2.405026in}{1.494793in}}{\pgfqpoint{2.397126in}{1.498066in}}{\pgfqpoint{2.388890in}{1.498066in}}%
\pgfpathcurveto{\pgfqpoint{2.380654in}{1.498066in}}{\pgfqpoint{2.372754in}{1.494793in}}{\pgfqpoint{2.366930in}{1.488969in}}%
\pgfpathcurveto{\pgfqpoint{2.361106in}{1.483146in}}{\pgfqpoint{2.357834in}{1.475245in}}{\pgfqpoint{2.357834in}{1.467009in}}%
\pgfpathcurveto{\pgfqpoint{2.357834in}{1.458773in}}{\pgfqpoint{2.361106in}{1.450873in}}{\pgfqpoint{2.366930in}{1.445049in}}%
\pgfpathcurveto{\pgfqpoint{2.372754in}{1.439225in}}{\pgfqpoint{2.380654in}{1.435953in}}{\pgfqpoint{2.388890in}{1.435953in}}%
\pgfpathclose%
\pgfusepath{stroke,fill}%
\end{pgfscope}%
\begin{pgfscope}%
\pgfpathrectangle{\pgfqpoint{0.100000in}{0.220728in}}{\pgfqpoint{3.696000in}{3.696000in}}%
\pgfusepath{clip}%
\pgfsetbuttcap%
\pgfsetroundjoin%
\definecolor{currentfill}{rgb}{0.121569,0.466667,0.705882}%
\pgfsetfillcolor{currentfill}%
\pgfsetfillopacity{0.991408}%
\pgfsetlinewidth{1.003750pt}%
\definecolor{currentstroke}{rgb}{0.121569,0.466667,0.705882}%
\pgfsetstrokecolor{currentstroke}%
\pgfsetstrokeopacity{0.991408}%
\pgfsetdash{}{0pt}%
\pgfpathmoveto{\pgfqpoint{2.387294in}{1.433680in}}%
\pgfpathcurveto{\pgfqpoint{2.395530in}{1.433680in}}{\pgfqpoint{2.403430in}{1.436952in}}{\pgfqpoint{2.409254in}{1.442776in}}%
\pgfpathcurveto{\pgfqpoint{2.415078in}{1.448600in}}{\pgfqpoint{2.418350in}{1.456500in}}{\pgfqpoint{2.418350in}{1.464736in}}%
\pgfpathcurveto{\pgfqpoint{2.418350in}{1.472972in}}{\pgfqpoint{2.415078in}{1.480873in}}{\pgfqpoint{2.409254in}{1.486696in}}%
\pgfpathcurveto{\pgfqpoint{2.403430in}{1.492520in}}{\pgfqpoint{2.395530in}{1.495793in}}{\pgfqpoint{2.387294in}{1.495793in}}%
\pgfpathcurveto{\pgfqpoint{2.379057in}{1.495793in}}{\pgfqpoint{2.371157in}{1.492520in}}{\pgfqpoint{2.365333in}{1.486696in}}%
\pgfpathcurveto{\pgfqpoint{2.359510in}{1.480873in}}{\pgfqpoint{2.356237in}{1.472972in}}{\pgfqpoint{2.356237in}{1.464736in}}%
\pgfpathcurveto{\pgfqpoint{2.356237in}{1.456500in}}{\pgfqpoint{2.359510in}{1.448600in}}{\pgfqpoint{2.365333in}{1.442776in}}%
\pgfpathcurveto{\pgfqpoint{2.371157in}{1.436952in}}{\pgfqpoint{2.379057in}{1.433680in}}{\pgfqpoint{2.387294in}{1.433680in}}%
\pgfpathclose%
\pgfusepath{stroke,fill}%
\end{pgfscope}%
\begin{pgfscope}%
\pgfpathrectangle{\pgfqpoint{0.100000in}{0.220728in}}{\pgfqpoint{3.696000in}{3.696000in}}%
\pgfusepath{clip}%
\pgfsetbuttcap%
\pgfsetroundjoin%
\definecolor{currentfill}{rgb}{0.121569,0.466667,0.705882}%
\pgfsetfillcolor{currentfill}%
\pgfsetfillopacity{0.991675}%
\pgfsetlinewidth{1.003750pt}%
\definecolor{currentstroke}{rgb}{0.121569,0.466667,0.705882}%
\pgfsetstrokecolor{currentstroke}%
\pgfsetstrokeopacity{0.991675}%
\pgfsetdash{}{0pt}%
\pgfpathmoveto{\pgfqpoint{2.386552in}{1.432363in}}%
\pgfpathcurveto{\pgfqpoint{2.394789in}{1.432363in}}{\pgfqpoint{2.402689in}{1.435635in}}{\pgfqpoint{2.408513in}{1.441459in}}%
\pgfpathcurveto{\pgfqpoint{2.414336in}{1.447283in}}{\pgfqpoint{2.417609in}{1.455183in}}{\pgfqpoint{2.417609in}{1.463419in}}%
\pgfpathcurveto{\pgfqpoint{2.417609in}{1.471655in}}{\pgfqpoint{2.414336in}{1.479555in}}{\pgfqpoint{2.408513in}{1.485379in}}%
\pgfpathcurveto{\pgfqpoint{2.402689in}{1.491203in}}{\pgfqpoint{2.394789in}{1.494476in}}{\pgfqpoint{2.386552in}{1.494476in}}%
\pgfpathcurveto{\pgfqpoint{2.378316in}{1.494476in}}{\pgfqpoint{2.370416in}{1.491203in}}{\pgfqpoint{2.364592in}{1.485379in}}%
\pgfpathcurveto{\pgfqpoint{2.358768in}{1.479555in}}{\pgfqpoint{2.355496in}{1.471655in}}{\pgfqpoint{2.355496in}{1.463419in}}%
\pgfpathcurveto{\pgfqpoint{2.355496in}{1.455183in}}{\pgfqpoint{2.358768in}{1.447283in}}{\pgfqpoint{2.364592in}{1.441459in}}%
\pgfpathcurveto{\pgfqpoint{2.370416in}{1.435635in}}{\pgfqpoint{2.378316in}{1.432363in}}{\pgfqpoint{2.386552in}{1.432363in}}%
\pgfpathclose%
\pgfusepath{stroke,fill}%
\end{pgfscope}%
\begin{pgfscope}%
\pgfpathrectangle{\pgfqpoint{0.100000in}{0.220728in}}{\pgfqpoint{3.696000in}{3.696000in}}%
\pgfusepath{clip}%
\pgfsetbuttcap%
\pgfsetroundjoin%
\definecolor{currentfill}{rgb}{0.121569,0.466667,0.705882}%
\pgfsetfillcolor{currentfill}%
\pgfsetfillopacity{0.991810}%
\pgfsetlinewidth{1.003750pt}%
\definecolor{currentstroke}{rgb}{0.121569,0.466667,0.705882}%
\pgfsetstrokecolor{currentstroke}%
\pgfsetstrokeopacity{0.991810}%
\pgfsetdash{}{0pt}%
\pgfpathmoveto{\pgfqpoint{2.386073in}{1.431683in}}%
\pgfpathcurveto{\pgfqpoint{2.394309in}{1.431683in}}{\pgfqpoint{2.402209in}{1.434955in}}{\pgfqpoint{2.408033in}{1.440779in}}%
\pgfpathcurveto{\pgfqpoint{2.413857in}{1.446603in}}{\pgfqpoint{2.417130in}{1.454503in}}{\pgfqpoint{2.417130in}{1.462739in}}%
\pgfpathcurveto{\pgfqpoint{2.417130in}{1.470976in}}{\pgfqpoint{2.413857in}{1.478876in}}{\pgfqpoint{2.408033in}{1.484700in}}%
\pgfpathcurveto{\pgfqpoint{2.402209in}{1.490524in}}{\pgfqpoint{2.394309in}{1.493796in}}{\pgfqpoint{2.386073in}{1.493796in}}%
\pgfpathcurveto{\pgfqpoint{2.377837in}{1.493796in}}{\pgfqpoint{2.369937in}{1.490524in}}{\pgfqpoint{2.364113in}{1.484700in}}%
\pgfpathcurveto{\pgfqpoint{2.358289in}{1.478876in}}{\pgfqpoint{2.355017in}{1.470976in}}{\pgfqpoint{2.355017in}{1.462739in}}%
\pgfpathcurveto{\pgfqpoint{2.355017in}{1.454503in}}{\pgfqpoint{2.358289in}{1.446603in}}{\pgfqpoint{2.364113in}{1.440779in}}%
\pgfpathcurveto{\pgfqpoint{2.369937in}{1.434955in}}{\pgfqpoint{2.377837in}{1.431683in}}{\pgfqpoint{2.386073in}{1.431683in}}%
\pgfpathclose%
\pgfusepath{stroke,fill}%
\end{pgfscope}%
\begin{pgfscope}%
\pgfpathrectangle{\pgfqpoint{0.100000in}{0.220728in}}{\pgfqpoint{3.696000in}{3.696000in}}%
\pgfusepath{clip}%
\pgfsetbuttcap%
\pgfsetroundjoin%
\definecolor{currentfill}{rgb}{0.121569,0.466667,0.705882}%
\pgfsetfillcolor{currentfill}%
\pgfsetfillopacity{0.991892}%
\pgfsetlinewidth{1.003750pt}%
\definecolor{currentstroke}{rgb}{0.121569,0.466667,0.705882}%
\pgfsetstrokecolor{currentstroke}%
\pgfsetstrokeopacity{0.991892}%
\pgfsetdash{}{0pt}%
\pgfpathmoveto{\pgfqpoint{2.385834in}{1.431310in}}%
\pgfpathcurveto{\pgfqpoint{2.394070in}{1.431310in}}{\pgfqpoint{2.401970in}{1.434583in}}{\pgfqpoint{2.407794in}{1.440407in}}%
\pgfpathcurveto{\pgfqpoint{2.413618in}{1.446231in}}{\pgfqpoint{2.416890in}{1.454131in}}{\pgfqpoint{2.416890in}{1.462367in}}%
\pgfpathcurveto{\pgfqpoint{2.416890in}{1.470603in}}{\pgfqpoint{2.413618in}{1.478503in}}{\pgfqpoint{2.407794in}{1.484327in}}%
\pgfpathcurveto{\pgfqpoint{2.401970in}{1.490151in}}{\pgfqpoint{2.394070in}{1.493423in}}{\pgfqpoint{2.385834in}{1.493423in}}%
\pgfpathcurveto{\pgfqpoint{2.377598in}{1.493423in}}{\pgfqpoint{2.369698in}{1.490151in}}{\pgfqpoint{2.363874in}{1.484327in}}%
\pgfpathcurveto{\pgfqpoint{2.358050in}{1.478503in}}{\pgfqpoint{2.354777in}{1.470603in}}{\pgfqpoint{2.354777in}{1.462367in}}%
\pgfpathcurveto{\pgfqpoint{2.354777in}{1.454131in}}{\pgfqpoint{2.358050in}{1.446231in}}{\pgfqpoint{2.363874in}{1.440407in}}%
\pgfpathcurveto{\pgfqpoint{2.369698in}{1.434583in}}{\pgfqpoint{2.377598in}{1.431310in}}{\pgfqpoint{2.385834in}{1.431310in}}%
\pgfpathclose%
\pgfusepath{stroke,fill}%
\end{pgfscope}%
\begin{pgfscope}%
\pgfpathrectangle{\pgfqpoint{0.100000in}{0.220728in}}{\pgfqpoint{3.696000in}{3.696000in}}%
\pgfusepath{clip}%
\pgfsetbuttcap%
\pgfsetroundjoin%
\definecolor{currentfill}{rgb}{0.121569,0.466667,0.705882}%
\pgfsetfillcolor{currentfill}%
\pgfsetfillopacity{0.991935}%
\pgfsetlinewidth{1.003750pt}%
\definecolor{currentstroke}{rgb}{0.121569,0.466667,0.705882}%
\pgfsetstrokecolor{currentstroke}%
\pgfsetstrokeopacity{0.991935}%
\pgfsetdash{}{0pt}%
\pgfpathmoveto{\pgfqpoint{2.385696in}{1.431101in}}%
\pgfpathcurveto{\pgfqpoint{2.393932in}{1.431101in}}{\pgfqpoint{2.401832in}{1.434373in}}{\pgfqpoint{2.407656in}{1.440197in}}%
\pgfpathcurveto{\pgfqpoint{2.413480in}{1.446021in}}{\pgfqpoint{2.416752in}{1.453921in}}{\pgfqpoint{2.416752in}{1.462158in}}%
\pgfpathcurveto{\pgfqpoint{2.416752in}{1.470394in}}{\pgfqpoint{2.413480in}{1.478294in}}{\pgfqpoint{2.407656in}{1.484118in}}%
\pgfpathcurveto{\pgfqpoint{2.401832in}{1.489942in}}{\pgfqpoint{2.393932in}{1.493214in}}{\pgfqpoint{2.385696in}{1.493214in}}%
\pgfpathcurveto{\pgfqpoint{2.377459in}{1.493214in}}{\pgfqpoint{2.369559in}{1.489942in}}{\pgfqpoint{2.363735in}{1.484118in}}%
\pgfpathcurveto{\pgfqpoint{2.357911in}{1.478294in}}{\pgfqpoint{2.354639in}{1.470394in}}{\pgfqpoint{2.354639in}{1.462158in}}%
\pgfpathcurveto{\pgfqpoint{2.354639in}{1.453921in}}{\pgfqpoint{2.357911in}{1.446021in}}{\pgfqpoint{2.363735in}{1.440197in}}%
\pgfpathcurveto{\pgfqpoint{2.369559in}{1.434373in}}{\pgfqpoint{2.377459in}{1.431101in}}{\pgfqpoint{2.385696in}{1.431101in}}%
\pgfpathclose%
\pgfusepath{stroke,fill}%
\end{pgfscope}%
\begin{pgfscope}%
\pgfpathrectangle{\pgfqpoint{0.100000in}{0.220728in}}{\pgfqpoint{3.696000in}{3.696000in}}%
\pgfusepath{clip}%
\pgfsetbuttcap%
\pgfsetroundjoin%
\definecolor{currentfill}{rgb}{0.121569,0.466667,0.705882}%
\pgfsetfillcolor{currentfill}%
\pgfsetfillopacity{0.991958}%
\pgfsetlinewidth{1.003750pt}%
\definecolor{currentstroke}{rgb}{0.121569,0.466667,0.705882}%
\pgfsetstrokecolor{currentstroke}%
\pgfsetstrokeopacity{0.991958}%
\pgfsetdash{}{0pt}%
\pgfpathmoveto{\pgfqpoint{2.385616in}{1.430993in}}%
\pgfpathcurveto{\pgfqpoint{2.393852in}{1.430993in}}{\pgfqpoint{2.401752in}{1.434265in}}{\pgfqpoint{2.407576in}{1.440089in}}%
\pgfpathcurveto{\pgfqpoint{2.413400in}{1.445913in}}{\pgfqpoint{2.416672in}{1.453813in}}{\pgfqpoint{2.416672in}{1.462050in}}%
\pgfpathcurveto{\pgfqpoint{2.416672in}{1.470286in}}{\pgfqpoint{2.413400in}{1.478186in}}{\pgfqpoint{2.407576in}{1.484010in}}%
\pgfpathcurveto{\pgfqpoint{2.401752in}{1.489834in}}{\pgfqpoint{2.393852in}{1.493106in}}{\pgfqpoint{2.385616in}{1.493106in}}%
\pgfpathcurveto{\pgfqpoint{2.377379in}{1.493106in}}{\pgfqpoint{2.369479in}{1.489834in}}{\pgfqpoint{2.363655in}{1.484010in}}%
\pgfpathcurveto{\pgfqpoint{2.357831in}{1.478186in}}{\pgfqpoint{2.354559in}{1.470286in}}{\pgfqpoint{2.354559in}{1.462050in}}%
\pgfpathcurveto{\pgfqpoint{2.354559in}{1.453813in}}{\pgfqpoint{2.357831in}{1.445913in}}{\pgfqpoint{2.363655in}{1.440089in}}%
\pgfpathcurveto{\pgfqpoint{2.369479in}{1.434265in}}{\pgfqpoint{2.377379in}{1.430993in}}{\pgfqpoint{2.385616in}{1.430993in}}%
\pgfpathclose%
\pgfusepath{stroke,fill}%
\end{pgfscope}%
\begin{pgfscope}%
\pgfpathrectangle{\pgfqpoint{0.100000in}{0.220728in}}{\pgfqpoint{3.696000in}{3.696000in}}%
\pgfusepath{clip}%
\pgfsetbuttcap%
\pgfsetroundjoin%
\definecolor{currentfill}{rgb}{0.121569,0.466667,0.705882}%
\pgfsetfillcolor{currentfill}%
\pgfsetfillopacity{0.992042}%
\pgfsetlinewidth{1.003750pt}%
\definecolor{currentstroke}{rgb}{0.121569,0.466667,0.705882}%
\pgfsetstrokecolor{currentstroke}%
\pgfsetstrokeopacity{0.992042}%
\pgfsetdash{}{0pt}%
\pgfpathmoveto{\pgfqpoint{2.302305in}{1.421818in}}%
\pgfpathcurveto{\pgfqpoint{2.310541in}{1.421818in}}{\pgfqpoint{2.318442in}{1.425090in}}{\pgfqpoint{2.324265in}{1.430914in}}%
\pgfpathcurveto{\pgfqpoint{2.330089in}{1.436738in}}{\pgfqpoint{2.333362in}{1.444638in}}{\pgfqpoint{2.333362in}{1.452874in}}%
\pgfpathcurveto{\pgfqpoint{2.333362in}{1.461110in}}{\pgfqpoint{2.330089in}{1.469011in}}{\pgfqpoint{2.324265in}{1.474834in}}%
\pgfpathcurveto{\pgfqpoint{2.318442in}{1.480658in}}{\pgfqpoint{2.310541in}{1.483931in}}{\pgfqpoint{2.302305in}{1.483931in}}%
\pgfpathcurveto{\pgfqpoint{2.294069in}{1.483931in}}{\pgfqpoint{2.286169in}{1.480658in}}{\pgfqpoint{2.280345in}{1.474834in}}%
\pgfpathcurveto{\pgfqpoint{2.274521in}{1.469011in}}{\pgfqpoint{2.271249in}{1.461110in}}{\pgfqpoint{2.271249in}{1.452874in}}%
\pgfpathcurveto{\pgfqpoint{2.271249in}{1.444638in}}{\pgfqpoint{2.274521in}{1.436738in}}{\pgfqpoint{2.280345in}{1.430914in}}%
\pgfpathcurveto{\pgfqpoint{2.286169in}{1.425090in}}{\pgfqpoint{2.294069in}{1.421818in}}{\pgfqpoint{2.302305in}{1.421818in}}%
\pgfpathclose%
\pgfusepath{stroke,fill}%
\end{pgfscope}%
\begin{pgfscope}%
\pgfpathrectangle{\pgfqpoint{0.100000in}{0.220728in}}{\pgfqpoint{3.696000in}{3.696000in}}%
\pgfusepath{clip}%
\pgfsetbuttcap%
\pgfsetroundjoin%
\definecolor{currentfill}{rgb}{0.121569,0.466667,0.705882}%
\pgfsetfillcolor{currentfill}%
\pgfsetfillopacity{0.992518}%
\pgfsetlinewidth{1.003750pt}%
\definecolor{currentstroke}{rgb}{0.121569,0.466667,0.705882}%
\pgfsetstrokecolor{currentstroke}%
\pgfsetstrokeopacity{0.992518}%
\pgfsetdash{}{0pt}%
\pgfpathmoveto{\pgfqpoint{2.384137in}{1.428185in}}%
\pgfpathcurveto{\pgfqpoint{2.392373in}{1.428185in}}{\pgfqpoint{2.400273in}{1.431457in}}{\pgfqpoint{2.406097in}{1.437281in}}%
\pgfpathcurveto{\pgfqpoint{2.411921in}{1.443105in}}{\pgfqpoint{2.415193in}{1.451005in}}{\pgfqpoint{2.415193in}{1.459241in}}%
\pgfpathcurveto{\pgfqpoint{2.415193in}{1.467478in}}{\pgfqpoint{2.411921in}{1.475378in}}{\pgfqpoint{2.406097in}{1.481202in}}%
\pgfpathcurveto{\pgfqpoint{2.400273in}{1.487026in}}{\pgfqpoint{2.392373in}{1.490298in}}{\pgfqpoint{2.384137in}{1.490298in}}%
\pgfpathcurveto{\pgfqpoint{2.375901in}{1.490298in}}{\pgfqpoint{2.368001in}{1.487026in}}{\pgfqpoint{2.362177in}{1.481202in}}%
\pgfpathcurveto{\pgfqpoint{2.356353in}{1.475378in}}{\pgfqpoint{2.353080in}{1.467478in}}{\pgfqpoint{2.353080in}{1.459241in}}%
\pgfpathcurveto{\pgfqpoint{2.353080in}{1.451005in}}{\pgfqpoint{2.356353in}{1.443105in}}{\pgfqpoint{2.362177in}{1.437281in}}%
\pgfpathcurveto{\pgfqpoint{2.368001in}{1.431457in}}{\pgfqpoint{2.375901in}{1.428185in}}{\pgfqpoint{2.384137in}{1.428185in}}%
\pgfpathclose%
\pgfusepath{stroke,fill}%
\end{pgfscope}%
\begin{pgfscope}%
\pgfpathrectangle{\pgfqpoint{0.100000in}{0.220728in}}{\pgfqpoint{3.696000in}{3.696000in}}%
\pgfusepath{clip}%
\pgfsetbuttcap%
\pgfsetroundjoin%
\definecolor{currentfill}{rgb}{0.121569,0.466667,0.705882}%
\pgfsetfillcolor{currentfill}%
\pgfsetfillopacity{0.993690}%
\pgfsetlinewidth{1.003750pt}%
\definecolor{currentstroke}{rgb}{0.121569,0.466667,0.705882}%
\pgfsetstrokecolor{currentstroke}%
\pgfsetstrokeopacity{0.993690}%
\pgfsetdash{}{0pt}%
\pgfpathmoveto{\pgfqpoint{2.380098in}{1.422269in}}%
\pgfpathcurveto{\pgfqpoint{2.388334in}{1.422269in}}{\pgfqpoint{2.396234in}{1.425541in}}{\pgfqpoint{2.402058in}{1.431365in}}%
\pgfpathcurveto{\pgfqpoint{2.407882in}{1.437189in}}{\pgfqpoint{2.411154in}{1.445089in}}{\pgfqpoint{2.411154in}{1.453326in}}%
\pgfpathcurveto{\pgfqpoint{2.411154in}{1.461562in}}{\pgfqpoint{2.407882in}{1.469462in}}{\pgfqpoint{2.402058in}{1.475286in}}%
\pgfpathcurveto{\pgfqpoint{2.396234in}{1.481110in}}{\pgfqpoint{2.388334in}{1.484382in}}{\pgfqpoint{2.380098in}{1.484382in}}%
\pgfpathcurveto{\pgfqpoint{2.371862in}{1.484382in}}{\pgfqpoint{2.363962in}{1.481110in}}{\pgfqpoint{2.358138in}{1.475286in}}%
\pgfpathcurveto{\pgfqpoint{2.352314in}{1.469462in}}{\pgfqpoint{2.349041in}{1.461562in}}{\pgfqpoint{2.349041in}{1.453326in}}%
\pgfpathcurveto{\pgfqpoint{2.349041in}{1.445089in}}{\pgfqpoint{2.352314in}{1.437189in}}{\pgfqpoint{2.358138in}{1.431365in}}%
\pgfpathcurveto{\pgfqpoint{2.363962in}{1.425541in}}{\pgfqpoint{2.371862in}{1.422269in}}{\pgfqpoint{2.380098in}{1.422269in}}%
\pgfpathclose%
\pgfusepath{stroke,fill}%
\end{pgfscope}%
\begin{pgfscope}%
\pgfpathrectangle{\pgfqpoint{0.100000in}{0.220728in}}{\pgfqpoint{3.696000in}{3.696000in}}%
\pgfusepath{clip}%
\pgfsetbuttcap%
\pgfsetroundjoin%
\definecolor{currentfill}{rgb}{0.121569,0.466667,0.705882}%
\pgfsetfillcolor{currentfill}%
\pgfsetfillopacity{0.994093}%
\pgfsetlinewidth{1.003750pt}%
\definecolor{currentstroke}{rgb}{0.121569,0.466667,0.705882}%
\pgfsetstrokecolor{currentstroke}%
\pgfsetstrokeopacity{0.994093}%
\pgfsetdash{}{0pt}%
\pgfpathmoveto{\pgfqpoint{2.308865in}{1.409550in}}%
\pgfpathcurveto{\pgfqpoint{2.317102in}{1.409550in}}{\pgfqpoint{2.325002in}{1.412822in}}{\pgfqpoint{2.330826in}{1.418646in}}%
\pgfpathcurveto{\pgfqpoint{2.336650in}{1.424470in}}{\pgfqpoint{2.339922in}{1.432370in}}{\pgfqpoint{2.339922in}{1.440606in}}%
\pgfpathcurveto{\pgfqpoint{2.339922in}{1.448843in}}{\pgfqpoint{2.336650in}{1.456743in}}{\pgfqpoint{2.330826in}{1.462567in}}%
\pgfpathcurveto{\pgfqpoint{2.325002in}{1.468391in}}{\pgfqpoint{2.317102in}{1.471663in}}{\pgfqpoint{2.308865in}{1.471663in}}%
\pgfpathcurveto{\pgfqpoint{2.300629in}{1.471663in}}{\pgfqpoint{2.292729in}{1.468391in}}{\pgfqpoint{2.286905in}{1.462567in}}%
\pgfpathcurveto{\pgfqpoint{2.281081in}{1.456743in}}{\pgfqpoint{2.277809in}{1.448843in}}{\pgfqpoint{2.277809in}{1.440606in}}%
\pgfpathcurveto{\pgfqpoint{2.277809in}{1.432370in}}{\pgfqpoint{2.281081in}{1.424470in}}{\pgfqpoint{2.286905in}{1.418646in}}%
\pgfpathcurveto{\pgfqpoint{2.292729in}{1.412822in}}{\pgfqpoint{2.300629in}{1.409550in}}{\pgfqpoint{2.308865in}{1.409550in}}%
\pgfpathclose%
\pgfusepath{stroke,fill}%
\end{pgfscope}%
\begin{pgfscope}%
\pgfpathrectangle{\pgfqpoint{0.100000in}{0.220728in}}{\pgfqpoint{3.696000in}{3.696000in}}%
\pgfusepath{clip}%
\pgfsetbuttcap%
\pgfsetroundjoin%
\definecolor{currentfill}{rgb}{0.121569,0.466667,0.705882}%
\pgfsetfillcolor{currentfill}%
\pgfsetfillopacity{0.995778}%
\pgfsetlinewidth{1.003750pt}%
\definecolor{currentstroke}{rgb}{0.121569,0.466667,0.705882}%
\pgfsetstrokecolor{currentstroke}%
\pgfsetstrokeopacity{0.995778}%
\pgfsetdash{}{0pt}%
\pgfpathmoveto{\pgfqpoint{2.374647in}{1.412424in}}%
\pgfpathcurveto{\pgfqpoint{2.382883in}{1.412424in}}{\pgfqpoint{2.390783in}{1.415697in}}{\pgfqpoint{2.396607in}{1.421520in}}%
\pgfpathcurveto{\pgfqpoint{2.402431in}{1.427344in}}{\pgfqpoint{2.405704in}{1.435244in}}{\pgfqpoint{2.405704in}{1.443481in}}%
\pgfpathcurveto{\pgfqpoint{2.405704in}{1.451717in}}{\pgfqpoint{2.402431in}{1.459617in}}{\pgfqpoint{2.396607in}{1.465441in}}%
\pgfpathcurveto{\pgfqpoint{2.390783in}{1.471265in}}{\pgfqpoint{2.382883in}{1.474537in}}{\pgfqpoint{2.374647in}{1.474537in}}%
\pgfpathcurveto{\pgfqpoint{2.366411in}{1.474537in}}{\pgfqpoint{2.358511in}{1.471265in}}{\pgfqpoint{2.352687in}{1.465441in}}%
\pgfpathcurveto{\pgfqpoint{2.346863in}{1.459617in}}{\pgfqpoint{2.343591in}{1.451717in}}{\pgfqpoint{2.343591in}{1.443481in}}%
\pgfpathcurveto{\pgfqpoint{2.343591in}{1.435244in}}{\pgfqpoint{2.346863in}{1.427344in}}{\pgfqpoint{2.352687in}{1.421520in}}%
\pgfpathcurveto{\pgfqpoint{2.358511in}{1.415697in}}{\pgfqpoint{2.366411in}{1.412424in}}{\pgfqpoint{2.374647in}{1.412424in}}%
\pgfpathclose%
\pgfusepath{stroke,fill}%
\end{pgfscope}%
\begin{pgfscope}%
\pgfpathrectangle{\pgfqpoint{0.100000in}{0.220728in}}{\pgfqpoint{3.696000in}{3.696000in}}%
\pgfusepath{clip}%
\pgfsetbuttcap%
\pgfsetroundjoin%
\definecolor{currentfill}{rgb}{0.121569,0.466667,0.705882}%
\pgfsetfillcolor{currentfill}%
\pgfsetfillopacity{0.996794}%
\pgfsetlinewidth{1.003750pt}%
\definecolor{currentstroke}{rgb}{0.121569,0.466667,0.705882}%
\pgfsetstrokecolor{currentstroke}%
\pgfsetstrokeopacity{0.996794}%
\pgfsetdash{}{0pt}%
\pgfpathmoveto{\pgfqpoint{2.371126in}{1.407042in}}%
\pgfpathcurveto{\pgfqpoint{2.379362in}{1.407042in}}{\pgfqpoint{2.387262in}{1.410314in}}{\pgfqpoint{2.393086in}{1.416138in}}%
\pgfpathcurveto{\pgfqpoint{2.398910in}{1.421962in}}{\pgfqpoint{2.402183in}{1.429862in}}{\pgfqpoint{2.402183in}{1.438098in}}%
\pgfpathcurveto{\pgfqpoint{2.402183in}{1.446335in}}{\pgfqpoint{2.398910in}{1.454235in}}{\pgfqpoint{2.393086in}{1.460059in}}%
\pgfpathcurveto{\pgfqpoint{2.387262in}{1.465882in}}{\pgfqpoint{2.379362in}{1.469155in}}{\pgfqpoint{2.371126in}{1.469155in}}%
\pgfpathcurveto{\pgfqpoint{2.362890in}{1.469155in}}{\pgfqpoint{2.354990in}{1.465882in}}{\pgfqpoint{2.349166in}{1.460059in}}%
\pgfpathcurveto{\pgfqpoint{2.343342in}{1.454235in}}{\pgfqpoint{2.340070in}{1.446335in}}{\pgfqpoint{2.340070in}{1.438098in}}%
\pgfpathcurveto{\pgfqpoint{2.340070in}{1.429862in}}{\pgfqpoint{2.343342in}{1.421962in}}{\pgfqpoint{2.349166in}{1.416138in}}%
\pgfpathcurveto{\pgfqpoint{2.354990in}{1.410314in}}{\pgfqpoint{2.362890in}{1.407042in}}{\pgfqpoint{2.371126in}{1.407042in}}%
\pgfpathclose%
\pgfusepath{stroke,fill}%
\end{pgfscope}%
\begin{pgfscope}%
\pgfpathrectangle{\pgfqpoint{0.100000in}{0.220728in}}{\pgfqpoint{3.696000in}{3.696000in}}%
\pgfusepath{clip}%
\pgfsetbuttcap%
\pgfsetroundjoin%
\definecolor{currentfill}{rgb}{0.121569,0.466667,0.705882}%
\pgfsetfillcolor{currentfill}%
\pgfsetfillopacity{0.997400}%
\pgfsetlinewidth{1.003750pt}%
\definecolor{currentstroke}{rgb}{0.121569,0.466667,0.705882}%
\pgfsetstrokecolor{currentstroke}%
\pgfsetstrokeopacity{0.997400}%
\pgfsetdash{}{0pt}%
\pgfpathmoveto{\pgfqpoint{2.369173in}{1.404328in}}%
\pgfpathcurveto{\pgfqpoint{2.377409in}{1.404328in}}{\pgfqpoint{2.385309in}{1.407600in}}{\pgfqpoint{2.391133in}{1.413424in}}%
\pgfpathcurveto{\pgfqpoint{2.396957in}{1.419248in}}{\pgfqpoint{2.400230in}{1.427148in}}{\pgfqpoint{2.400230in}{1.435385in}}%
\pgfpathcurveto{\pgfqpoint{2.400230in}{1.443621in}}{\pgfqpoint{2.396957in}{1.451521in}}{\pgfqpoint{2.391133in}{1.457345in}}%
\pgfpathcurveto{\pgfqpoint{2.385309in}{1.463169in}}{\pgfqpoint{2.377409in}{1.466441in}}{\pgfqpoint{2.369173in}{1.466441in}}%
\pgfpathcurveto{\pgfqpoint{2.360937in}{1.466441in}}{\pgfqpoint{2.353037in}{1.463169in}}{\pgfqpoint{2.347213in}{1.457345in}}%
\pgfpathcurveto{\pgfqpoint{2.341389in}{1.451521in}}{\pgfqpoint{2.338117in}{1.443621in}}{\pgfqpoint{2.338117in}{1.435385in}}%
\pgfpathcurveto{\pgfqpoint{2.338117in}{1.427148in}}{\pgfqpoint{2.341389in}{1.419248in}}{\pgfqpoint{2.347213in}{1.413424in}}%
\pgfpathcurveto{\pgfqpoint{2.353037in}{1.407600in}}{\pgfqpoint{2.360937in}{1.404328in}}{\pgfqpoint{2.369173in}{1.404328in}}%
\pgfpathclose%
\pgfusepath{stroke,fill}%
\end{pgfscope}%
\begin{pgfscope}%
\pgfpathrectangle{\pgfqpoint{0.100000in}{0.220728in}}{\pgfqpoint{3.696000in}{3.696000in}}%
\pgfusepath{clip}%
\pgfsetbuttcap%
\pgfsetroundjoin%
\definecolor{currentfill}{rgb}{0.121569,0.466667,0.705882}%
\pgfsetfillcolor{currentfill}%
\pgfsetfillopacity{0.997523}%
\pgfsetlinewidth{1.003750pt}%
\definecolor{currentstroke}{rgb}{0.121569,0.466667,0.705882}%
\pgfsetstrokecolor{currentstroke}%
\pgfsetstrokeopacity{0.997523}%
\pgfsetdash{}{0pt}%
\pgfpathmoveto{\pgfqpoint{2.324529in}{1.391990in}}%
\pgfpathcurveto{\pgfqpoint{2.332766in}{1.391990in}}{\pgfqpoint{2.340666in}{1.395262in}}{\pgfqpoint{2.346490in}{1.401086in}}%
\pgfpathcurveto{\pgfqpoint{2.352314in}{1.406910in}}{\pgfqpoint{2.355586in}{1.414810in}}{\pgfqpoint{2.355586in}{1.423047in}}%
\pgfpathcurveto{\pgfqpoint{2.355586in}{1.431283in}}{\pgfqpoint{2.352314in}{1.439183in}}{\pgfqpoint{2.346490in}{1.445007in}}%
\pgfpathcurveto{\pgfqpoint{2.340666in}{1.450831in}}{\pgfqpoint{2.332766in}{1.454103in}}{\pgfqpoint{2.324529in}{1.454103in}}%
\pgfpathcurveto{\pgfqpoint{2.316293in}{1.454103in}}{\pgfqpoint{2.308393in}{1.450831in}}{\pgfqpoint{2.302569in}{1.445007in}}%
\pgfpathcurveto{\pgfqpoint{2.296745in}{1.439183in}}{\pgfqpoint{2.293473in}{1.431283in}}{\pgfqpoint{2.293473in}{1.423047in}}%
\pgfpathcurveto{\pgfqpoint{2.293473in}{1.414810in}}{\pgfqpoint{2.296745in}{1.406910in}}{\pgfqpoint{2.302569in}{1.401086in}}%
\pgfpathcurveto{\pgfqpoint{2.308393in}{1.395262in}}{\pgfqpoint{2.316293in}{1.391990in}}{\pgfqpoint{2.324529in}{1.391990in}}%
\pgfpathclose%
\pgfusepath{stroke,fill}%
\end{pgfscope}%
\begin{pgfscope}%
\pgfpathrectangle{\pgfqpoint{0.100000in}{0.220728in}}{\pgfqpoint{3.696000in}{3.696000in}}%
\pgfusepath{clip}%
\pgfsetbuttcap%
\pgfsetroundjoin%
\definecolor{currentfill}{rgb}{0.121569,0.466667,0.705882}%
\pgfsetfillcolor{currentfill}%
\pgfsetfillopacity{0.997704}%
\pgfsetlinewidth{1.003750pt}%
\definecolor{currentstroke}{rgb}{0.121569,0.466667,0.705882}%
\pgfsetstrokecolor{currentstroke}%
\pgfsetstrokeopacity{0.997704}%
\pgfsetdash{}{0pt}%
\pgfpathmoveto{\pgfqpoint{2.368122in}{1.402668in}}%
\pgfpathcurveto{\pgfqpoint{2.376358in}{1.402668in}}{\pgfqpoint{2.384258in}{1.405941in}}{\pgfqpoint{2.390082in}{1.411764in}}%
\pgfpathcurveto{\pgfqpoint{2.395906in}{1.417588in}}{\pgfqpoint{2.399178in}{1.425488in}}{\pgfqpoint{2.399178in}{1.433725in}}%
\pgfpathcurveto{\pgfqpoint{2.399178in}{1.441961in}}{\pgfqpoint{2.395906in}{1.449861in}}{\pgfqpoint{2.390082in}{1.455685in}}%
\pgfpathcurveto{\pgfqpoint{2.384258in}{1.461509in}}{\pgfqpoint{2.376358in}{1.464781in}}{\pgfqpoint{2.368122in}{1.464781in}}%
\pgfpathcurveto{\pgfqpoint{2.359886in}{1.464781in}}{\pgfqpoint{2.351986in}{1.461509in}}{\pgfqpoint{2.346162in}{1.455685in}}%
\pgfpathcurveto{\pgfqpoint{2.340338in}{1.449861in}}{\pgfqpoint{2.337065in}{1.441961in}}{\pgfqpoint{2.337065in}{1.433725in}}%
\pgfpathcurveto{\pgfqpoint{2.337065in}{1.425488in}}{\pgfqpoint{2.340338in}{1.417588in}}{\pgfqpoint{2.346162in}{1.411764in}}%
\pgfpathcurveto{\pgfqpoint{2.351986in}{1.405941in}}{\pgfqpoint{2.359886in}{1.402668in}}{\pgfqpoint{2.368122in}{1.402668in}}%
\pgfpathclose%
\pgfusepath{stroke,fill}%
\end{pgfscope}%
\begin{pgfscope}%
\pgfpathrectangle{\pgfqpoint{0.100000in}{0.220728in}}{\pgfqpoint{3.696000in}{3.696000in}}%
\pgfusepath{clip}%
\pgfsetbuttcap%
\pgfsetroundjoin%
\definecolor{currentfill}{rgb}{0.121569,0.466667,0.705882}%
\pgfsetfillcolor{currentfill}%
\pgfsetfillopacity{0.997887}%
\pgfsetlinewidth{1.003750pt}%
\definecolor{currentstroke}{rgb}{0.121569,0.466667,0.705882}%
\pgfsetstrokecolor{currentstroke}%
\pgfsetstrokeopacity{0.997887}%
\pgfsetdash{}{0pt}%
\pgfpathmoveto{\pgfqpoint{2.367546in}{1.401816in}}%
\pgfpathcurveto{\pgfqpoint{2.375782in}{1.401816in}}{\pgfqpoint{2.383683in}{1.405089in}}{\pgfqpoint{2.389506in}{1.410912in}}%
\pgfpathcurveto{\pgfqpoint{2.395330in}{1.416736in}}{\pgfqpoint{2.398603in}{1.424636in}}{\pgfqpoint{2.398603in}{1.432873in}}%
\pgfpathcurveto{\pgfqpoint{2.398603in}{1.441109in}}{\pgfqpoint{2.395330in}{1.449009in}}{\pgfqpoint{2.389506in}{1.454833in}}%
\pgfpathcurveto{\pgfqpoint{2.383683in}{1.460657in}}{\pgfqpoint{2.375782in}{1.463929in}}{\pgfqpoint{2.367546in}{1.463929in}}%
\pgfpathcurveto{\pgfqpoint{2.359310in}{1.463929in}}{\pgfqpoint{2.351410in}{1.460657in}}{\pgfqpoint{2.345586in}{1.454833in}}%
\pgfpathcurveto{\pgfqpoint{2.339762in}{1.449009in}}{\pgfqpoint{2.336490in}{1.441109in}}{\pgfqpoint{2.336490in}{1.432873in}}%
\pgfpathcurveto{\pgfqpoint{2.336490in}{1.424636in}}{\pgfqpoint{2.339762in}{1.416736in}}{\pgfqpoint{2.345586in}{1.410912in}}%
\pgfpathcurveto{\pgfqpoint{2.351410in}{1.405089in}}{\pgfqpoint{2.359310in}{1.401816in}}{\pgfqpoint{2.367546in}{1.401816in}}%
\pgfpathclose%
\pgfusepath{stroke,fill}%
\end{pgfscope}%
\begin{pgfscope}%
\pgfpathrectangle{\pgfqpoint{0.100000in}{0.220728in}}{\pgfqpoint{3.696000in}{3.696000in}}%
\pgfusepath{clip}%
\pgfsetbuttcap%
\pgfsetroundjoin%
\definecolor{currentfill}{rgb}{0.121569,0.466667,0.705882}%
\pgfsetfillcolor{currentfill}%
\pgfsetfillopacity{0.997990}%
\pgfsetlinewidth{1.003750pt}%
\definecolor{currentstroke}{rgb}{0.121569,0.466667,0.705882}%
\pgfsetstrokecolor{currentstroke}%
\pgfsetstrokeopacity{0.997990}%
\pgfsetdash{}{0pt}%
\pgfpathmoveto{\pgfqpoint{2.367246in}{1.401335in}}%
\pgfpathcurveto{\pgfqpoint{2.375482in}{1.401335in}}{\pgfqpoint{2.383383in}{1.404607in}}{\pgfqpoint{2.389206in}{1.410431in}}%
\pgfpathcurveto{\pgfqpoint{2.395030in}{1.416255in}}{\pgfqpoint{2.398303in}{1.424155in}}{\pgfqpoint{2.398303in}{1.432391in}}%
\pgfpathcurveto{\pgfqpoint{2.398303in}{1.440627in}}{\pgfqpoint{2.395030in}{1.448527in}}{\pgfqpoint{2.389206in}{1.454351in}}%
\pgfpathcurveto{\pgfqpoint{2.383383in}{1.460175in}}{\pgfqpoint{2.375482in}{1.463448in}}{\pgfqpoint{2.367246in}{1.463448in}}%
\pgfpathcurveto{\pgfqpoint{2.359010in}{1.463448in}}{\pgfqpoint{2.351110in}{1.460175in}}{\pgfqpoint{2.345286in}{1.454351in}}%
\pgfpathcurveto{\pgfqpoint{2.339462in}{1.448527in}}{\pgfqpoint{2.336190in}{1.440627in}}{\pgfqpoint{2.336190in}{1.432391in}}%
\pgfpathcurveto{\pgfqpoint{2.336190in}{1.424155in}}{\pgfqpoint{2.339462in}{1.416255in}}{\pgfqpoint{2.345286in}{1.410431in}}%
\pgfpathcurveto{\pgfqpoint{2.351110in}{1.404607in}}{\pgfqpoint{2.359010in}{1.401335in}}{\pgfqpoint{2.367246in}{1.401335in}}%
\pgfpathclose%
\pgfusepath{stroke,fill}%
\end{pgfscope}%
\begin{pgfscope}%
\pgfpathrectangle{\pgfqpoint{0.100000in}{0.220728in}}{\pgfqpoint{3.696000in}{3.696000in}}%
\pgfusepath{clip}%
\pgfsetbuttcap%
\pgfsetroundjoin%
\definecolor{currentfill}{rgb}{0.121569,0.466667,0.705882}%
\pgfsetfillcolor{currentfill}%
\pgfsetfillopacity{0.998043}%
\pgfsetlinewidth{1.003750pt}%
\definecolor{currentstroke}{rgb}{0.121569,0.466667,0.705882}%
\pgfsetstrokecolor{currentstroke}%
\pgfsetstrokeopacity{0.998043}%
\pgfsetdash{}{0pt}%
\pgfpathmoveto{\pgfqpoint{2.367060in}{1.401088in}}%
\pgfpathcurveto{\pgfqpoint{2.375296in}{1.401088in}}{\pgfqpoint{2.383196in}{1.404361in}}{\pgfqpoint{2.389020in}{1.410184in}}%
\pgfpathcurveto{\pgfqpoint{2.394844in}{1.416008in}}{\pgfqpoint{2.398117in}{1.423908in}}{\pgfqpoint{2.398117in}{1.432145in}}%
\pgfpathcurveto{\pgfqpoint{2.398117in}{1.440381in}}{\pgfqpoint{2.394844in}{1.448281in}}{\pgfqpoint{2.389020in}{1.454105in}}%
\pgfpathcurveto{\pgfqpoint{2.383196in}{1.459929in}}{\pgfqpoint{2.375296in}{1.463201in}}{\pgfqpoint{2.367060in}{1.463201in}}%
\pgfpathcurveto{\pgfqpoint{2.358824in}{1.463201in}}{\pgfqpoint{2.350924in}{1.459929in}}{\pgfqpoint{2.345100in}{1.454105in}}%
\pgfpathcurveto{\pgfqpoint{2.339276in}{1.448281in}}{\pgfqpoint{2.336004in}{1.440381in}}{\pgfqpoint{2.336004in}{1.432145in}}%
\pgfpathcurveto{\pgfqpoint{2.336004in}{1.423908in}}{\pgfqpoint{2.339276in}{1.416008in}}{\pgfqpoint{2.345100in}{1.410184in}}%
\pgfpathcurveto{\pgfqpoint{2.350924in}{1.404361in}}{\pgfqpoint{2.358824in}{1.401088in}}{\pgfqpoint{2.367060in}{1.401088in}}%
\pgfpathclose%
\pgfusepath{stroke,fill}%
\end{pgfscope}%
\begin{pgfscope}%
\pgfpathrectangle{\pgfqpoint{0.100000in}{0.220728in}}{\pgfqpoint{3.696000in}{3.696000in}}%
\pgfusepath{clip}%
\pgfsetbuttcap%
\pgfsetroundjoin%
\definecolor{currentfill}{rgb}{0.121569,0.466667,0.705882}%
\pgfsetfillcolor{currentfill}%
\pgfsetfillopacity{0.998074}%
\pgfsetlinewidth{1.003750pt}%
\definecolor{currentstroke}{rgb}{0.121569,0.466667,0.705882}%
\pgfsetstrokecolor{currentstroke}%
\pgfsetstrokeopacity{0.998074}%
\pgfsetdash{}{0pt}%
\pgfpathmoveto{\pgfqpoint{2.366973in}{1.400938in}}%
\pgfpathcurveto{\pgfqpoint{2.375209in}{1.400938in}}{\pgfqpoint{2.383109in}{1.404210in}}{\pgfqpoint{2.388933in}{1.410034in}}%
\pgfpathcurveto{\pgfqpoint{2.394757in}{1.415858in}}{\pgfqpoint{2.398029in}{1.423758in}}{\pgfqpoint{2.398029in}{1.431994in}}%
\pgfpathcurveto{\pgfqpoint{2.398029in}{1.440230in}}{\pgfqpoint{2.394757in}{1.448131in}}{\pgfqpoint{2.388933in}{1.453954in}}%
\pgfpathcurveto{\pgfqpoint{2.383109in}{1.459778in}}{\pgfqpoint{2.375209in}{1.463051in}}{\pgfqpoint{2.366973in}{1.463051in}}%
\pgfpathcurveto{\pgfqpoint{2.358737in}{1.463051in}}{\pgfqpoint{2.350836in}{1.459778in}}{\pgfqpoint{2.345013in}{1.453954in}}%
\pgfpathcurveto{\pgfqpoint{2.339189in}{1.448131in}}{\pgfqpoint{2.335916in}{1.440230in}}{\pgfqpoint{2.335916in}{1.431994in}}%
\pgfpathcurveto{\pgfqpoint{2.335916in}{1.423758in}}{\pgfqpoint{2.339189in}{1.415858in}}{\pgfqpoint{2.345013in}{1.410034in}}%
\pgfpathcurveto{\pgfqpoint{2.350836in}{1.404210in}}{\pgfqpoint{2.358737in}{1.400938in}}{\pgfqpoint{2.366973in}{1.400938in}}%
\pgfpathclose%
\pgfusepath{stroke,fill}%
\end{pgfscope}%
\begin{pgfscope}%
\pgfpathrectangle{\pgfqpoint{0.100000in}{0.220728in}}{\pgfqpoint{3.696000in}{3.696000in}}%
\pgfusepath{clip}%
\pgfsetbuttcap%
\pgfsetroundjoin%
\definecolor{currentfill}{rgb}{0.121569,0.466667,0.705882}%
\pgfsetfillcolor{currentfill}%
\pgfsetfillopacity{0.998091}%
\pgfsetlinewidth{1.003750pt}%
\definecolor{currentstroke}{rgb}{0.121569,0.466667,0.705882}%
\pgfsetstrokecolor{currentstroke}%
\pgfsetstrokeopacity{0.998091}%
\pgfsetdash{}{0pt}%
\pgfpathmoveto{\pgfqpoint{2.366922in}{1.400858in}}%
\pgfpathcurveto{\pgfqpoint{2.375158in}{1.400858in}}{\pgfqpoint{2.383058in}{1.404131in}}{\pgfqpoint{2.388882in}{1.409954in}}%
\pgfpathcurveto{\pgfqpoint{2.394706in}{1.415778in}}{\pgfqpoint{2.397978in}{1.423678in}}{\pgfqpoint{2.397978in}{1.431915in}}%
\pgfpathcurveto{\pgfqpoint{2.397978in}{1.440151in}}{\pgfqpoint{2.394706in}{1.448051in}}{\pgfqpoint{2.388882in}{1.453875in}}%
\pgfpathcurveto{\pgfqpoint{2.383058in}{1.459699in}}{\pgfqpoint{2.375158in}{1.462971in}}{\pgfqpoint{2.366922in}{1.462971in}}%
\pgfpathcurveto{\pgfqpoint{2.358686in}{1.462971in}}{\pgfqpoint{2.350785in}{1.459699in}}{\pgfqpoint{2.344962in}{1.453875in}}%
\pgfpathcurveto{\pgfqpoint{2.339138in}{1.448051in}}{\pgfqpoint{2.335865in}{1.440151in}}{\pgfqpoint{2.335865in}{1.431915in}}%
\pgfpathcurveto{\pgfqpoint{2.335865in}{1.423678in}}{\pgfqpoint{2.339138in}{1.415778in}}{\pgfqpoint{2.344962in}{1.409954in}}%
\pgfpathcurveto{\pgfqpoint{2.350785in}{1.404131in}}{\pgfqpoint{2.358686in}{1.400858in}}{\pgfqpoint{2.366922in}{1.400858in}}%
\pgfpathclose%
\pgfusepath{stroke,fill}%
\end{pgfscope}%
\begin{pgfscope}%
\pgfpathrectangle{\pgfqpoint{0.100000in}{0.220728in}}{\pgfqpoint{3.696000in}{3.696000in}}%
\pgfusepath{clip}%
\pgfsetbuttcap%
\pgfsetroundjoin%
\definecolor{currentfill}{rgb}{0.121569,0.466667,0.705882}%
\pgfsetfillcolor{currentfill}%
\pgfsetfillopacity{0.998100}%
\pgfsetlinewidth{1.003750pt}%
\definecolor{currentstroke}{rgb}{0.121569,0.466667,0.705882}%
\pgfsetstrokecolor{currentstroke}%
\pgfsetstrokeopacity{0.998100}%
\pgfsetdash{}{0pt}%
\pgfpathmoveto{\pgfqpoint{2.366892in}{1.400813in}}%
\pgfpathcurveto{\pgfqpoint{2.375129in}{1.400813in}}{\pgfqpoint{2.383029in}{1.404086in}}{\pgfqpoint{2.388853in}{1.409910in}}%
\pgfpathcurveto{\pgfqpoint{2.394677in}{1.415734in}}{\pgfqpoint{2.397949in}{1.423634in}}{\pgfqpoint{2.397949in}{1.431870in}}%
\pgfpathcurveto{\pgfqpoint{2.397949in}{1.440106in}}{\pgfqpoint{2.394677in}{1.448006in}}{\pgfqpoint{2.388853in}{1.453830in}}%
\pgfpathcurveto{\pgfqpoint{2.383029in}{1.459654in}}{\pgfqpoint{2.375129in}{1.462926in}}{\pgfqpoint{2.366892in}{1.462926in}}%
\pgfpathcurveto{\pgfqpoint{2.358656in}{1.462926in}}{\pgfqpoint{2.350756in}{1.459654in}}{\pgfqpoint{2.344932in}{1.453830in}}%
\pgfpathcurveto{\pgfqpoint{2.339108in}{1.448006in}}{\pgfqpoint{2.335836in}{1.440106in}}{\pgfqpoint{2.335836in}{1.431870in}}%
\pgfpathcurveto{\pgfqpoint{2.335836in}{1.423634in}}{\pgfqpoint{2.339108in}{1.415734in}}{\pgfqpoint{2.344932in}{1.409910in}}%
\pgfpathcurveto{\pgfqpoint{2.350756in}{1.404086in}}{\pgfqpoint{2.358656in}{1.400813in}}{\pgfqpoint{2.366892in}{1.400813in}}%
\pgfpathclose%
\pgfusepath{stroke,fill}%
\end{pgfscope}%
\begin{pgfscope}%
\pgfpathrectangle{\pgfqpoint{0.100000in}{0.220728in}}{\pgfqpoint{3.696000in}{3.696000in}}%
\pgfusepath{clip}%
\pgfsetbuttcap%
\pgfsetroundjoin%
\definecolor{currentfill}{rgb}{0.121569,0.466667,0.705882}%
\pgfsetfillcolor{currentfill}%
\pgfsetfillopacity{0.998104}%
\pgfsetlinewidth{1.003750pt}%
\definecolor{currentstroke}{rgb}{0.121569,0.466667,0.705882}%
\pgfsetstrokecolor{currentstroke}%
\pgfsetstrokeopacity{0.998104}%
\pgfsetdash{}{0pt}%
\pgfpathmoveto{\pgfqpoint{2.366877in}{1.400789in}}%
\pgfpathcurveto{\pgfqpoint{2.375113in}{1.400789in}}{\pgfqpoint{2.383013in}{1.404061in}}{\pgfqpoint{2.388837in}{1.409885in}}%
\pgfpathcurveto{\pgfqpoint{2.394661in}{1.415709in}}{\pgfqpoint{2.397933in}{1.423609in}}{\pgfqpoint{2.397933in}{1.431845in}}%
\pgfpathcurveto{\pgfqpoint{2.397933in}{1.440082in}}{\pgfqpoint{2.394661in}{1.447982in}}{\pgfqpoint{2.388837in}{1.453806in}}%
\pgfpathcurveto{\pgfqpoint{2.383013in}{1.459630in}}{\pgfqpoint{2.375113in}{1.462902in}}{\pgfqpoint{2.366877in}{1.462902in}}%
\pgfpathcurveto{\pgfqpoint{2.358640in}{1.462902in}}{\pgfqpoint{2.350740in}{1.459630in}}{\pgfqpoint{2.344916in}{1.453806in}}%
\pgfpathcurveto{\pgfqpoint{2.339092in}{1.447982in}}{\pgfqpoint{2.335820in}{1.440082in}}{\pgfqpoint{2.335820in}{1.431845in}}%
\pgfpathcurveto{\pgfqpoint{2.335820in}{1.423609in}}{\pgfqpoint{2.339092in}{1.415709in}}{\pgfqpoint{2.344916in}{1.409885in}}%
\pgfpathcurveto{\pgfqpoint{2.350740in}{1.404061in}}{\pgfqpoint{2.358640in}{1.400789in}}{\pgfqpoint{2.366877in}{1.400789in}}%
\pgfpathclose%
\pgfusepath{stroke,fill}%
\end{pgfscope}%
\begin{pgfscope}%
\pgfpathrectangle{\pgfqpoint{0.100000in}{0.220728in}}{\pgfqpoint{3.696000in}{3.696000in}}%
\pgfusepath{clip}%
\pgfsetbuttcap%
\pgfsetroundjoin%
\definecolor{currentfill}{rgb}{0.121569,0.466667,0.705882}%
\pgfsetfillcolor{currentfill}%
\pgfsetfillopacity{0.998107}%
\pgfsetlinewidth{1.003750pt}%
\definecolor{currentstroke}{rgb}{0.121569,0.466667,0.705882}%
\pgfsetstrokecolor{currentstroke}%
\pgfsetstrokeopacity{0.998107}%
\pgfsetdash{}{0pt}%
\pgfpathmoveto{\pgfqpoint{2.366868in}{1.400775in}}%
\pgfpathcurveto{\pgfqpoint{2.375104in}{1.400775in}}{\pgfqpoint{2.383004in}{1.404047in}}{\pgfqpoint{2.388828in}{1.409871in}}%
\pgfpathcurveto{\pgfqpoint{2.394652in}{1.415695in}}{\pgfqpoint{2.397924in}{1.423595in}}{\pgfqpoint{2.397924in}{1.431831in}}%
\pgfpathcurveto{\pgfqpoint{2.397924in}{1.440068in}}{\pgfqpoint{2.394652in}{1.447968in}}{\pgfqpoint{2.388828in}{1.453792in}}%
\pgfpathcurveto{\pgfqpoint{2.383004in}{1.459616in}}{\pgfqpoint{2.375104in}{1.462888in}}{\pgfqpoint{2.366868in}{1.462888in}}%
\pgfpathcurveto{\pgfqpoint{2.358631in}{1.462888in}}{\pgfqpoint{2.350731in}{1.459616in}}{\pgfqpoint{2.344907in}{1.453792in}}%
\pgfpathcurveto{\pgfqpoint{2.339083in}{1.447968in}}{\pgfqpoint{2.335811in}{1.440068in}}{\pgfqpoint{2.335811in}{1.431831in}}%
\pgfpathcurveto{\pgfqpoint{2.335811in}{1.423595in}}{\pgfqpoint{2.339083in}{1.415695in}}{\pgfqpoint{2.344907in}{1.409871in}}%
\pgfpathcurveto{\pgfqpoint{2.350731in}{1.404047in}}{\pgfqpoint{2.358631in}{1.400775in}}{\pgfqpoint{2.366868in}{1.400775in}}%
\pgfpathclose%
\pgfusepath{stroke,fill}%
\end{pgfscope}%
\begin{pgfscope}%
\pgfpathrectangle{\pgfqpoint{0.100000in}{0.220728in}}{\pgfqpoint{3.696000in}{3.696000in}}%
\pgfusepath{clip}%
\pgfsetbuttcap%
\pgfsetroundjoin%
\definecolor{currentfill}{rgb}{0.121569,0.466667,0.705882}%
\pgfsetfillcolor{currentfill}%
\pgfsetfillopacity{0.998108}%
\pgfsetlinewidth{1.003750pt}%
\definecolor{currentstroke}{rgb}{0.121569,0.466667,0.705882}%
\pgfsetstrokecolor{currentstroke}%
\pgfsetstrokeopacity{0.998108}%
\pgfsetdash{}{0pt}%
\pgfpathmoveto{\pgfqpoint{2.366863in}{1.400767in}}%
\pgfpathcurveto{\pgfqpoint{2.375099in}{1.400767in}}{\pgfqpoint{2.382999in}{1.404040in}}{\pgfqpoint{2.388823in}{1.409864in}}%
\pgfpathcurveto{\pgfqpoint{2.394647in}{1.415688in}}{\pgfqpoint{2.397919in}{1.423588in}}{\pgfqpoint{2.397919in}{1.431824in}}%
\pgfpathcurveto{\pgfqpoint{2.397919in}{1.440060in}}{\pgfqpoint{2.394647in}{1.447960in}}{\pgfqpoint{2.388823in}{1.453784in}}%
\pgfpathcurveto{\pgfqpoint{2.382999in}{1.459608in}}{\pgfqpoint{2.375099in}{1.462880in}}{\pgfqpoint{2.366863in}{1.462880in}}%
\pgfpathcurveto{\pgfqpoint{2.358626in}{1.462880in}}{\pgfqpoint{2.350726in}{1.459608in}}{\pgfqpoint{2.344902in}{1.453784in}}%
\pgfpathcurveto{\pgfqpoint{2.339078in}{1.447960in}}{\pgfqpoint{2.335806in}{1.440060in}}{\pgfqpoint{2.335806in}{1.431824in}}%
\pgfpathcurveto{\pgfqpoint{2.335806in}{1.423588in}}{\pgfqpoint{2.339078in}{1.415688in}}{\pgfqpoint{2.344902in}{1.409864in}}%
\pgfpathcurveto{\pgfqpoint{2.350726in}{1.404040in}}{\pgfqpoint{2.358626in}{1.400767in}}{\pgfqpoint{2.366863in}{1.400767in}}%
\pgfpathclose%
\pgfusepath{stroke,fill}%
\end{pgfscope}%
\begin{pgfscope}%
\pgfpathrectangle{\pgfqpoint{0.100000in}{0.220728in}}{\pgfqpoint{3.696000in}{3.696000in}}%
\pgfusepath{clip}%
\pgfsetbuttcap%
\pgfsetroundjoin%
\definecolor{currentfill}{rgb}{0.121569,0.466667,0.705882}%
\pgfsetfillcolor{currentfill}%
\pgfsetfillopacity{0.998470}%
\pgfsetlinewidth{1.003750pt}%
\definecolor{currentstroke}{rgb}{0.121569,0.466667,0.705882}%
\pgfsetstrokecolor{currentstroke}%
\pgfsetstrokeopacity{0.998470}%
\pgfsetdash{}{0pt}%
\pgfpathmoveto{\pgfqpoint{2.365150in}{1.398716in}}%
\pgfpathcurveto{\pgfqpoint{2.373386in}{1.398716in}}{\pgfqpoint{2.381286in}{1.401988in}}{\pgfqpoint{2.387110in}{1.407812in}}%
\pgfpathcurveto{\pgfqpoint{2.392934in}{1.413636in}}{\pgfqpoint{2.396207in}{1.421536in}}{\pgfqpoint{2.396207in}{1.429773in}}%
\pgfpathcurveto{\pgfqpoint{2.396207in}{1.438009in}}{\pgfqpoint{2.392934in}{1.445909in}}{\pgfqpoint{2.387110in}{1.451733in}}%
\pgfpathcurveto{\pgfqpoint{2.381286in}{1.457557in}}{\pgfqpoint{2.373386in}{1.460829in}}{\pgfqpoint{2.365150in}{1.460829in}}%
\pgfpathcurveto{\pgfqpoint{2.356914in}{1.460829in}}{\pgfqpoint{2.349014in}{1.457557in}}{\pgfqpoint{2.343190in}{1.451733in}}%
\pgfpathcurveto{\pgfqpoint{2.337366in}{1.445909in}}{\pgfqpoint{2.334094in}{1.438009in}}{\pgfqpoint{2.334094in}{1.429773in}}%
\pgfpathcurveto{\pgfqpoint{2.334094in}{1.421536in}}{\pgfqpoint{2.337366in}{1.413636in}}{\pgfqpoint{2.343190in}{1.407812in}}%
\pgfpathcurveto{\pgfqpoint{2.349014in}{1.401988in}}{\pgfqpoint{2.356914in}{1.398716in}}{\pgfqpoint{2.365150in}{1.398716in}}%
\pgfpathclose%
\pgfusepath{stroke,fill}%
\end{pgfscope}%
\begin{pgfscope}%
\pgfpathrectangle{\pgfqpoint{0.100000in}{0.220728in}}{\pgfqpoint{3.696000in}{3.696000in}}%
\pgfusepath{clip}%
\pgfsetbuttcap%
\pgfsetroundjoin%
\definecolor{currentfill}{rgb}{0.121569,0.466667,0.705882}%
\pgfsetfillcolor{currentfill}%
\pgfsetfillopacity{0.999070}%
\pgfsetlinewidth{1.003750pt}%
\definecolor{currentstroke}{rgb}{0.121569,0.466667,0.705882}%
\pgfsetstrokecolor{currentstroke}%
\pgfsetstrokeopacity{0.999070}%
\pgfsetdash{}{0pt}%
\pgfpathmoveto{\pgfqpoint{2.361458in}{1.394390in}}%
\pgfpathcurveto{\pgfqpoint{2.369694in}{1.394390in}}{\pgfqpoint{2.377594in}{1.397663in}}{\pgfqpoint{2.383418in}{1.403487in}}%
\pgfpathcurveto{\pgfqpoint{2.389242in}{1.409311in}}{\pgfqpoint{2.392514in}{1.417211in}}{\pgfqpoint{2.392514in}{1.425447in}}%
\pgfpathcurveto{\pgfqpoint{2.392514in}{1.433683in}}{\pgfqpoint{2.389242in}{1.441583in}}{\pgfqpoint{2.383418in}{1.447407in}}%
\pgfpathcurveto{\pgfqpoint{2.377594in}{1.453231in}}{\pgfqpoint{2.369694in}{1.456503in}}{\pgfqpoint{2.361458in}{1.456503in}}%
\pgfpathcurveto{\pgfqpoint{2.353221in}{1.456503in}}{\pgfqpoint{2.345321in}{1.453231in}}{\pgfqpoint{2.339498in}{1.447407in}}%
\pgfpathcurveto{\pgfqpoint{2.333674in}{1.441583in}}{\pgfqpoint{2.330401in}{1.433683in}}{\pgfqpoint{2.330401in}{1.425447in}}%
\pgfpathcurveto{\pgfqpoint{2.330401in}{1.417211in}}{\pgfqpoint{2.333674in}{1.409311in}}{\pgfqpoint{2.339498in}{1.403487in}}%
\pgfpathcurveto{\pgfqpoint{2.345321in}{1.397663in}}{\pgfqpoint{2.353221in}{1.394390in}}{\pgfqpoint{2.361458in}{1.394390in}}%
\pgfpathclose%
\pgfusepath{stroke,fill}%
\end{pgfscope}%
\begin{pgfscope}%
\pgfpathrectangle{\pgfqpoint{0.100000in}{0.220728in}}{\pgfqpoint{3.696000in}{3.696000in}}%
\pgfusepath{clip}%
\pgfsetbuttcap%
\pgfsetroundjoin%
\definecolor{currentfill}{rgb}{0.121569,0.466667,0.705882}%
\pgfsetfillcolor{currentfill}%
\pgfsetfillopacity{0.999379}%
\pgfsetlinewidth{1.003750pt}%
\definecolor{currentstroke}{rgb}{0.121569,0.466667,0.705882}%
\pgfsetstrokecolor{currentstroke}%
\pgfsetstrokeopacity{0.999379}%
\pgfsetdash{}{0pt}%
\pgfpathmoveto{\pgfqpoint{2.339696in}{1.384634in}}%
\pgfpathcurveto{\pgfqpoint{2.347933in}{1.384634in}}{\pgfqpoint{2.355833in}{1.387907in}}{\pgfqpoint{2.361657in}{1.393731in}}%
\pgfpathcurveto{\pgfqpoint{2.367480in}{1.399555in}}{\pgfqpoint{2.370753in}{1.407455in}}{\pgfqpoint{2.370753in}{1.415691in}}%
\pgfpathcurveto{\pgfqpoint{2.370753in}{1.423927in}}{\pgfqpoint{2.367480in}{1.431827in}}{\pgfqpoint{2.361657in}{1.437651in}}%
\pgfpathcurveto{\pgfqpoint{2.355833in}{1.443475in}}{\pgfqpoint{2.347933in}{1.446747in}}{\pgfqpoint{2.339696in}{1.446747in}}%
\pgfpathcurveto{\pgfqpoint{2.331460in}{1.446747in}}{\pgfqpoint{2.323560in}{1.443475in}}{\pgfqpoint{2.317736in}{1.437651in}}%
\pgfpathcurveto{\pgfqpoint{2.311912in}{1.431827in}}{\pgfqpoint{2.308640in}{1.423927in}}{\pgfqpoint{2.308640in}{1.415691in}}%
\pgfpathcurveto{\pgfqpoint{2.308640in}{1.407455in}}{\pgfqpoint{2.311912in}{1.399555in}}{\pgfqpoint{2.317736in}{1.393731in}}%
\pgfpathcurveto{\pgfqpoint{2.323560in}{1.387907in}}{\pgfqpoint{2.331460in}{1.384634in}}{\pgfqpoint{2.339696in}{1.384634in}}%
\pgfpathclose%
\pgfusepath{stroke,fill}%
\end{pgfscope}%
\begin{pgfscope}%
\pgfpathrectangle{\pgfqpoint{0.100000in}{0.220728in}}{\pgfqpoint{3.696000in}{3.696000in}}%
\pgfusepath{clip}%
\pgfsetbuttcap%
\pgfsetroundjoin%
\definecolor{currentfill}{rgb}{0.121569,0.466667,0.705882}%
\pgfsetfillcolor{currentfill}%
\pgfsetfillopacity{0.999917}%
\pgfsetlinewidth{1.003750pt}%
\definecolor{currentstroke}{rgb}{0.121569,0.466667,0.705882}%
\pgfsetstrokecolor{currentstroke}%
\pgfsetstrokeopacity{0.999917}%
\pgfsetdash{}{0pt}%
\pgfpathmoveto{\pgfqpoint{2.354383in}{1.388350in}}%
\pgfpathcurveto{\pgfqpoint{2.362619in}{1.388350in}}{\pgfqpoint{2.370519in}{1.391622in}}{\pgfqpoint{2.376343in}{1.397446in}}%
\pgfpathcurveto{\pgfqpoint{2.382167in}{1.403270in}}{\pgfqpoint{2.385439in}{1.411170in}}{\pgfqpoint{2.385439in}{1.419406in}}%
\pgfpathcurveto{\pgfqpoint{2.385439in}{1.427642in}}{\pgfqpoint{2.382167in}{1.435542in}}{\pgfqpoint{2.376343in}{1.441366in}}%
\pgfpathcurveto{\pgfqpoint{2.370519in}{1.447190in}}{\pgfqpoint{2.362619in}{1.450463in}}{\pgfqpoint{2.354383in}{1.450463in}}%
\pgfpathcurveto{\pgfqpoint{2.346147in}{1.450463in}}{\pgfqpoint{2.338246in}{1.447190in}}{\pgfqpoint{2.332423in}{1.441366in}}%
\pgfpathcurveto{\pgfqpoint{2.326599in}{1.435542in}}{\pgfqpoint{2.323326in}{1.427642in}}{\pgfqpoint{2.323326in}{1.419406in}}%
\pgfpathcurveto{\pgfqpoint{2.323326in}{1.411170in}}{\pgfqpoint{2.326599in}{1.403270in}}{\pgfqpoint{2.332423in}{1.397446in}}%
\pgfpathcurveto{\pgfqpoint{2.338246in}{1.391622in}}{\pgfqpoint{2.346147in}{1.388350in}}{\pgfqpoint{2.354383in}{1.388350in}}%
\pgfpathclose%
\pgfusepath{stroke,fill}%
\end{pgfscope}%
\begin{pgfscope}%
\pgfpathrectangle{\pgfqpoint{0.100000in}{0.220728in}}{\pgfqpoint{3.696000in}{3.696000in}}%
\pgfusepath{clip}%
\pgfsetbuttcap%
\pgfsetroundjoin%
\definecolor{currentfill}{rgb}{0.121569,0.466667,0.705882}%
\pgfsetfillcolor{currentfill}%
\pgfsetlinewidth{1.003750pt}%
\definecolor{currentstroke}{rgb}{0.121569,0.466667,0.705882}%
\pgfsetstrokecolor{currentstroke}%
\pgfsetdash{}{0pt}%
\pgfpathmoveto{\pgfqpoint{2.349891in}{1.385917in}}%
\pgfpathcurveto{\pgfqpoint{2.358127in}{1.385917in}}{\pgfqpoint{2.366027in}{1.389189in}}{\pgfqpoint{2.371851in}{1.395013in}}%
\pgfpathcurveto{\pgfqpoint{2.377675in}{1.400837in}}{\pgfqpoint{2.380947in}{1.408737in}}{\pgfqpoint{2.380947in}{1.416973in}}%
\pgfpathcurveto{\pgfqpoint{2.380947in}{1.425209in}}{\pgfqpoint{2.377675in}{1.433109in}}{\pgfqpoint{2.371851in}{1.438933in}}%
\pgfpathcurveto{\pgfqpoint{2.366027in}{1.444757in}}{\pgfqpoint{2.358127in}{1.448030in}}{\pgfqpoint{2.349891in}{1.448030in}}%
\pgfpathcurveto{\pgfqpoint{2.341655in}{1.448030in}}{\pgfqpoint{2.333755in}{1.444757in}}{\pgfqpoint{2.327931in}{1.438933in}}%
\pgfpathcurveto{\pgfqpoint{2.322107in}{1.433109in}}{\pgfqpoint{2.318834in}{1.425209in}}{\pgfqpoint{2.318834in}{1.416973in}}%
\pgfpathcurveto{\pgfqpoint{2.318834in}{1.408737in}}{\pgfqpoint{2.322107in}{1.400837in}}{\pgfqpoint{2.327931in}{1.395013in}}%
\pgfpathcurveto{\pgfqpoint{2.333755in}{1.389189in}}{\pgfqpoint{2.341655in}{1.385917in}}{\pgfqpoint{2.349891in}{1.385917in}}%
\pgfpathclose%
\pgfusepath{stroke,fill}%
\end{pgfscope}%
\begin{pgfscope}%
\definecolor{textcolor}{rgb}{0.000000,0.000000,0.000000}%
\pgfsetstrokecolor{textcolor}%
\pgfsetfillcolor{textcolor}%
\pgftext[x=1.948000in,y=4.000061in,,base]{\color{textcolor}\sffamily\fontsize{12.000000}{14.400000}\selectfont FLAE}%
\end{pgfscope}%
\begin{pgfscope}%
\pgfpathrectangle{\pgfqpoint{0.100000in}{0.220728in}}{\pgfqpoint{3.696000in}{3.696000in}}%
\pgfusepath{clip}%
\pgfsetbuttcap%
\pgfsetroundjoin%
\definecolor{currentfill}{rgb}{1.000000,0.498039,0.054902}%
\pgfsetfillcolor{currentfill}%
\pgfsetfillopacity{0.300000}%
\pgfsetlinewidth{1.003750pt}%
\definecolor{currentstroke}{rgb}{1.000000,0.498039,0.054902}%
\pgfsetstrokecolor{currentstroke}%
\pgfsetstrokeopacity{0.300000}%
\pgfsetdash{}{0pt}%
\pgfpathmoveto{\pgfqpoint{1.654126in}{3.126024in}}%
\pgfpathcurveto{\pgfqpoint{1.662362in}{3.126024in}}{\pgfqpoint{1.670263in}{3.129297in}}{\pgfqpoint{1.676086in}{3.135121in}}%
\pgfpathcurveto{\pgfqpoint{1.681910in}{3.140945in}}{\pgfqpoint{1.685183in}{3.148845in}}{\pgfqpoint{1.685183in}{3.157081in}}%
\pgfpathcurveto{\pgfqpoint{1.685183in}{3.165317in}}{\pgfqpoint{1.681910in}{3.173217in}}{\pgfqpoint{1.676086in}{3.179041in}}%
\pgfpathcurveto{\pgfqpoint{1.670263in}{3.184865in}}{\pgfqpoint{1.662362in}{3.188137in}}{\pgfqpoint{1.654126in}{3.188137in}}%
\pgfpathcurveto{\pgfqpoint{1.645890in}{3.188137in}}{\pgfqpoint{1.637990in}{3.184865in}}{\pgfqpoint{1.632166in}{3.179041in}}%
\pgfpathcurveto{\pgfqpoint{1.626342in}{3.173217in}}{\pgfqpoint{1.623070in}{3.165317in}}{\pgfqpoint{1.623070in}{3.157081in}}%
\pgfpathcurveto{\pgfqpoint{1.623070in}{3.148845in}}{\pgfqpoint{1.626342in}{3.140945in}}{\pgfqpoint{1.632166in}{3.135121in}}%
\pgfpathcurveto{\pgfqpoint{1.637990in}{3.129297in}}{\pgfqpoint{1.645890in}{3.126024in}}{\pgfqpoint{1.654126in}{3.126024in}}%
\pgfpathclose%
\pgfusepath{stroke,fill}%
\end{pgfscope}%
\begin{pgfscope}%
\pgfpathrectangle{\pgfqpoint{0.100000in}{0.220728in}}{\pgfqpoint{3.696000in}{3.696000in}}%
\pgfusepath{clip}%
\pgfsetbuttcap%
\pgfsetroundjoin%
\definecolor{currentfill}{rgb}{1.000000,0.498039,0.054902}%
\pgfsetfillcolor{currentfill}%
\pgfsetfillopacity{0.587662}%
\pgfsetlinewidth{1.003750pt}%
\definecolor{currentstroke}{rgb}{1.000000,0.498039,0.054902}%
\pgfsetstrokecolor{currentstroke}%
\pgfsetstrokeopacity{0.587662}%
\pgfsetdash{}{0pt}%
\pgfpathmoveto{\pgfqpoint{0.854595in}{1.389376in}}%
\pgfpathcurveto{\pgfqpoint{0.862831in}{1.389376in}}{\pgfqpoint{0.870732in}{1.392648in}}{\pgfqpoint{0.876555in}{1.398472in}}%
\pgfpathcurveto{\pgfqpoint{0.882379in}{1.404296in}}{\pgfqpoint{0.885652in}{1.412196in}}{\pgfqpoint{0.885652in}{1.420432in}}%
\pgfpathcurveto{\pgfqpoint{0.885652in}{1.428669in}}{\pgfqpoint{0.882379in}{1.436569in}}{\pgfqpoint{0.876555in}{1.442393in}}%
\pgfpathcurveto{\pgfqpoint{0.870732in}{1.448217in}}{\pgfqpoint{0.862831in}{1.451489in}}{\pgfqpoint{0.854595in}{1.451489in}}%
\pgfpathcurveto{\pgfqpoint{0.846359in}{1.451489in}}{\pgfqpoint{0.838459in}{1.448217in}}{\pgfqpoint{0.832635in}{1.442393in}}%
\pgfpathcurveto{\pgfqpoint{0.826811in}{1.436569in}}{\pgfqpoint{0.823539in}{1.428669in}}{\pgfqpoint{0.823539in}{1.420432in}}%
\pgfpathcurveto{\pgfqpoint{0.823539in}{1.412196in}}{\pgfqpoint{0.826811in}{1.404296in}}{\pgfqpoint{0.832635in}{1.398472in}}%
\pgfpathcurveto{\pgfqpoint{0.838459in}{1.392648in}}{\pgfqpoint{0.846359in}{1.389376in}}{\pgfqpoint{0.854595in}{1.389376in}}%
\pgfpathclose%
\pgfusepath{stroke,fill}%
\end{pgfscope}%
\begin{pgfscope}%
\pgfpathrectangle{\pgfqpoint{0.100000in}{0.220728in}}{\pgfqpoint{3.696000in}{3.696000in}}%
\pgfusepath{clip}%
\pgfsetbuttcap%
\pgfsetroundjoin%
\definecolor{currentfill}{rgb}{1.000000,0.498039,0.054902}%
\pgfsetfillcolor{currentfill}%
\pgfsetfillopacity{0.695101}%
\pgfsetlinewidth{1.003750pt}%
\definecolor{currentstroke}{rgb}{1.000000,0.498039,0.054902}%
\pgfsetstrokecolor{currentstroke}%
\pgfsetstrokeopacity{0.695101}%
\pgfsetdash{}{0pt}%
\pgfpathmoveto{\pgfqpoint{0.754893in}{2.319267in}}%
\pgfpathcurveto{\pgfqpoint{0.763129in}{2.319267in}}{\pgfqpoint{0.771029in}{2.322540in}}{\pgfqpoint{0.776853in}{2.328364in}}%
\pgfpathcurveto{\pgfqpoint{0.782677in}{2.334188in}}{\pgfqpoint{0.785949in}{2.342088in}}{\pgfqpoint{0.785949in}{2.350324in}}%
\pgfpathcurveto{\pgfqpoint{0.785949in}{2.358560in}}{\pgfqpoint{0.782677in}{2.366460in}}{\pgfqpoint{0.776853in}{2.372284in}}%
\pgfpathcurveto{\pgfqpoint{0.771029in}{2.378108in}}{\pgfqpoint{0.763129in}{2.381380in}}{\pgfqpoint{0.754893in}{2.381380in}}%
\pgfpathcurveto{\pgfqpoint{0.746657in}{2.381380in}}{\pgfqpoint{0.738757in}{2.378108in}}{\pgfqpoint{0.732933in}{2.372284in}}%
\pgfpathcurveto{\pgfqpoint{0.727109in}{2.366460in}}{\pgfqpoint{0.723836in}{2.358560in}}{\pgfqpoint{0.723836in}{2.350324in}}%
\pgfpathcurveto{\pgfqpoint{0.723836in}{2.342088in}}{\pgfqpoint{0.727109in}{2.334188in}}{\pgfqpoint{0.732933in}{2.328364in}}%
\pgfpathcurveto{\pgfqpoint{0.738757in}{2.322540in}}{\pgfqpoint{0.746657in}{2.319267in}}{\pgfqpoint{0.754893in}{2.319267in}}%
\pgfpathclose%
\pgfusepath{stroke,fill}%
\end{pgfscope}%
\begin{pgfscope}%
\pgfpathrectangle{\pgfqpoint{0.100000in}{0.220728in}}{\pgfqpoint{3.696000in}{3.696000in}}%
\pgfusepath{clip}%
\pgfsetbuttcap%
\pgfsetroundjoin%
\definecolor{currentfill}{rgb}{1.000000,0.498039,0.054902}%
\pgfsetfillcolor{currentfill}%
\pgfsetfillopacity{0.697636}%
\pgfsetlinewidth{1.003750pt}%
\definecolor{currentstroke}{rgb}{1.000000,0.498039,0.054902}%
\pgfsetstrokecolor{currentstroke}%
\pgfsetstrokeopacity{0.697636}%
\pgfsetdash{}{0pt}%
\pgfpathmoveto{\pgfqpoint{3.311522in}{2.876114in}}%
\pgfpathcurveto{\pgfqpoint{3.319758in}{2.876114in}}{\pgfqpoint{3.327658in}{2.879386in}}{\pgfqpoint{3.333482in}{2.885210in}}%
\pgfpathcurveto{\pgfqpoint{3.339306in}{2.891034in}}{\pgfqpoint{3.342578in}{2.898934in}}{\pgfqpoint{3.342578in}{2.907170in}}%
\pgfpathcurveto{\pgfqpoint{3.342578in}{2.915406in}}{\pgfqpoint{3.339306in}{2.923307in}}{\pgfqpoint{3.333482in}{2.929130in}}%
\pgfpathcurveto{\pgfqpoint{3.327658in}{2.934954in}}{\pgfqpoint{3.319758in}{2.938227in}}{\pgfqpoint{3.311522in}{2.938227in}}%
\pgfpathcurveto{\pgfqpoint{3.303285in}{2.938227in}}{\pgfqpoint{3.295385in}{2.934954in}}{\pgfqpoint{3.289561in}{2.929130in}}%
\pgfpathcurveto{\pgfqpoint{3.283738in}{2.923307in}}{\pgfqpoint{3.280465in}{2.915406in}}{\pgfqpoint{3.280465in}{2.907170in}}%
\pgfpathcurveto{\pgfqpoint{3.280465in}{2.898934in}}{\pgfqpoint{3.283738in}{2.891034in}}{\pgfqpoint{3.289561in}{2.885210in}}%
\pgfpathcurveto{\pgfqpoint{3.295385in}{2.879386in}}{\pgfqpoint{3.303285in}{2.876114in}}{\pgfqpoint{3.311522in}{2.876114in}}%
\pgfpathclose%
\pgfusepath{stroke,fill}%
\end{pgfscope}%
\begin{pgfscope}%
\pgfpathrectangle{\pgfqpoint{0.100000in}{0.220728in}}{\pgfqpoint{3.696000in}{3.696000in}}%
\pgfusepath{clip}%
\pgfsetbuttcap%
\pgfsetroundjoin%
\definecolor{currentfill}{rgb}{1.000000,0.498039,0.054902}%
\pgfsetfillcolor{currentfill}%
\pgfsetlinewidth{1.003750pt}%
\definecolor{currentstroke}{rgb}{1.000000,0.498039,0.054902}%
\pgfsetstrokecolor{currentstroke}%
\pgfsetdash{}{0pt}%
\pgfpathmoveto{\pgfqpoint{2.349891in}{1.385917in}}%
\pgfpathcurveto{\pgfqpoint{2.358127in}{1.385917in}}{\pgfqpoint{2.366027in}{1.389189in}}{\pgfqpoint{2.371851in}{1.395013in}}%
\pgfpathcurveto{\pgfqpoint{2.377675in}{1.400837in}}{\pgfqpoint{2.380947in}{1.408737in}}{\pgfqpoint{2.380947in}{1.416973in}}%
\pgfpathcurveto{\pgfqpoint{2.380947in}{1.425209in}}{\pgfqpoint{2.377675in}{1.433109in}}{\pgfqpoint{2.371851in}{1.438933in}}%
\pgfpathcurveto{\pgfqpoint{2.366027in}{1.444757in}}{\pgfqpoint{2.358127in}{1.448030in}}{\pgfqpoint{2.349891in}{1.448030in}}%
\pgfpathcurveto{\pgfqpoint{2.341655in}{1.448030in}}{\pgfqpoint{2.333755in}{1.444757in}}{\pgfqpoint{2.327931in}{1.438933in}}%
\pgfpathcurveto{\pgfqpoint{2.322107in}{1.433109in}}{\pgfqpoint{2.318834in}{1.425209in}}{\pgfqpoint{2.318834in}{1.416973in}}%
\pgfpathcurveto{\pgfqpoint{2.318834in}{1.408737in}}{\pgfqpoint{2.322107in}{1.400837in}}{\pgfqpoint{2.327931in}{1.395013in}}%
\pgfpathcurveto{\pgfqpoint{2.333755in}{1.389189in}}{\pgfqpoint{2.341655in}{1.385917in}}{\pgfqpoint{2.349891in}{1.385917in}}%
\pgfpathclose%
\pgfusepath{stroke,fill}%
\end{pgfscope}%
\begin{pgfscope}%
\pgfsetbuttcap%
\pgfsetmiterjoin%
\definecolor{currentfill}{rgb}{1.000000,1.000000,1.000000}%
\pgfsetfillcolor{currentfill}%
\pgfsetfillopacity{0.800000}%
\pgfsetlinewidth{1.003750pt}%
\definecolor{currentstroke}{rgb}{0.800000,0.800000,0.800000}%
\pgfsetstrokecolor{currentstroke}%
\pgfsetstrokeopacity{0.800000}%
\pgfsetdash{}{0pt}%
\pgfpathmoveto{\pgfqpoint{1.958421in}{3.194045in}}%
\pgfpathlineto{\pgfqpoint{3.698778in}{3.194045in}}%
\pgfpathquadraticcurveto{\pgfqpoint{3.726556in}{3.194045in}}{\pgfqpoint{3.726556in}{3.221823in}}%
\pgfpathlineto{\pgfqpoint{3.726556in}{3.819506in}}%
\pgfpathquadraticcurveto{\pgfqpoint{3.726556in}{3.847284in}}{\pgfqpoint{3.698778in}{3.847284in}}%
\pgfpathlineto{\pgfqpoint{1.958421in}{3.847284in}}%
\pgfpathquadraticcurveto{\pgfqpoint{1.930644in}{3.847284in}}{\pgfqpoint{1.930644in}{3.819506in}}%
\pgfpathlineto{\pgfqpoint{1.930644in}{3.221823in}}%
\pgfpathquadraticcurveto{\pgfqpoint{1.930644in}{3.194045in}}{\pgfqpoint{1.958421in}{3.194045in}}%
\pgfpathclose%
\pgfusepath{stroke,fill}%
\end{pgfscope}%
\begin{pgfscope}%
\pgfsetrectcap%
\pgfsetroundjoin%
\pgfsetlinewidth{1.505625pt}%
\definecolor{currentstroke}{rgb}{0.121569,0.466667,0.705882}%
\pgfsetstrokecolor{currentstroke}%
\pgfsetdash{}{0pt}%
\pgfpathmoveto{\pgfqpoint{1.986199in}{3.734816in}}%
\pgfpathlineto{\pgfqpoint{2.263977in}{3.734816in}}%
\pgfusepath{stroke}%
\end{pgfscope}%
\begin{pgfscope}%
\definecolor{textcolor}{rgb}{0.000000,0.000000,0.000000}%
\pgfsetstrokecolor{textcolor}%
\pgfsetfillcolor{textcolor}%
\pgftext[x=2.375088in,y=3.686205in,left,base]{\color{textcolor}\sffamily\fontsize{10.000000}{12.000000}\selectfont Ground truth}%
\end{pgfscope}%
\begin{pgfscope}%
\pgfsetbuttcap%
\pgfsetroundjoin%
\definecolor{currentfill}{rgb}{0.121569,0.466667,0.705882}%
\pgfsetfillcolor{currentfill}%
\pgfsetlinewidth{1.003750pt}%
\definecolor{currentstroke}{rgb}{0.121569,0.466667,0.705882}%
\pgfsetstrokecolor{currentstroke}%
\pgfsetdash{}{0pt}%
\pgfsys@defobject{currentmarker}{\pgfqpoint{-0.031056in}{-0.031056in}}{\pgfqpoint{0.031056in}{0.031056in}}{%
\pgfpathmoveto{\pgfqpoint{0.000000in}{-0.031056in}}%
\pgfpathcurveto{\pgfqpoint{0.008236in}{-0.031056in}}{\pgfqpoint{0.016136in}{-0.027784in}}{\pgfqpoint{0.021960in}{-0.021960in}}%
\pgfpathcurveto{\pgfqpoint{0.027784in}{-0.016136in}}{\pgfqpoint{0.031056in}{-0.008236in}}{\pgfqpoint{0.031056in}{0.000000in}}%
\pgfpathcurveto{\pgfqpoint{0.031056in}{0.008236in}}{\pgfqpoint{0.027784in}{0.016136in}}{\pgfqpoint{0.021960in}{0.021960in}}%
\pgfpathcurveto{\pgfqpoint{0.016136in}{0.027784in}}{\pgfqpoint{0.008236in}{0.031056in}}{\pgfqpoint{0.000000in}{0.031056in}}%
\pgfpathcurveto{\pgfqpoint{-0.008236in}{0.031056in}}{\pgfqpoint{-0.016136in}{0.027784in}}{\pgfqpoint{-0.021960in}{0.021960in}}%
\pgfpathcurveto{\pgfqpoint{-0.027784in}{0.016136in}}{\pgfqpoint{-0.031056in}{0.008236in}}{\pgfqpoint{-0.031056in}{0.000000in}}%
\pgfpathcurveto{\pgfqpoint{-0.031056in}{-0.008236in}}{\pgfqpoint{-0.027784in}{-0.016136in}}{\pgfqpoint{-0.021960in}{-0.021960in}}%
\pgfpathcurveto{\pgfqpoint{-0.016136in}{-0.027784in}}{\pgfqpoint{-0.008236in}{-0.031056in}}{\pgfqpoint{0.000000in}{-0.031056in}}%
\pgfpathclose%
\pgfusepath{stroke,fill}%
}%
\begin{pgfscope}%
\pgfsys@transformshift{2.125088in}{3.518806in}%
\pgfsys@useobject{currentmarker}{}%
\end{pgfscope}%
\end{pgfscope}%
\begin{pgfscope}%
\definecolor{textcolor}{rgb}{0.000000,0.000000,0.000000}%
\pgfsetstrokecolor{textcolor}%
\pgfsetfillcolor{textcolor}%
\pgftext[x=2.375088in,y=3.482348in,left,base]{\color{textcolor}\sffamily\fontsize{10.000000}{12.000000}\selectfont Estimated position}%
\end{pgfscope}%
\begin{pgfscope}%
\pgfsetbuttcap%
\pgfsetroundjoin%
\definecolor{currentfill}{rgb}{1.000000,0.498039,0.054902}%
\pgfsetfillcolor{currentfill}%
\pgfsetlinewidth{1.003750pt}%
\definecolor{currentstroke}{rgb}{1.000000,0.498039,0.054902}%
\pgfsetstrokecolor{currentstroke}%
\pgfsetdash{}{0pt}%
\pgfsys@defobject{currentmarker}{\pgfqpoint{-0.031056in}{-0.031056in}}{\pgfqpoint{0.031056in}{0.031056in}}{%
\pgfpathmoveto{\pgfqpoint{0.000000in}{-0.031056in}}%
\pgfpathcurveto{\pgfqpoint{0.008236in}{-0.031056in}}{\pgfqpoint{0.016136in}{-0.027784in}}{\pgfqpoint{0.021960in}{-0.021960in}}%
\pgfpathcurveto{\pgfqpoint{0.027784in}{-0.016136in}}{\pgfqpoint{0.031056in}{-0.008236in}}{\pgfqpoint{0.031056in}{0.000000in}}%
\pgfpathcurveto{\pgfqpoint{0.031056in}{0.008236in}}{\pgfqpoint{0.027784in}{0.016136in}}{\pgfqpoint{0.021960in}{0.021960in}}%
\pgfpathcurveto{\pgfqpoint{0.016136in}{0.027784in}}{\pgfqpoint{0.008236in}{0.031056in}}{\pgfqpoint{0.000000in}{0.031056in}}%
\pgfpathcurveto{\pgfqpoint{-0.008236in}{0.031056in}}{\pgfqpoint{-0.016136in}{0.027784in}}{\pgfqpoint{-0.021960in}{0.021960in}}%
\pgfpathcurveto{\pgfqpoint{-0.027784in}{0.016136in}}{\pgfqpoint{-0.031056in}{0.008236in}}{\pgfqpoint{-0.031056in}{0.000000in}}%
\pgfpathcurveto{\pgfqpoint{-0.031056in}{-0.008236in}}{\pgfqpoint{-0.027784in}{-0.016136in}}{\pgfqpoint{-0.021960in}{-0.021960in}}%
\pgfpathcurveto{\pgfqpoint{-0.016136in}{-0.027784in}}{\pgfqpoint{-0.008236in}{-0.031056in}}{\pgfqpoint{0.000000in}{-0.031056in}}%
\pgfpathclose%
\pgfusepath{stroke,fill}%
}%
\begin{pgfscope}%
\pgfsys@transformshift{2.125088in}{3.314949in}%
\pgfsys@useobject{currentmarker}{}%
\end{pgfscope}%
\end{pgfscope}%
\begin{pgfscope}%
\definecolor{textcolor}{rgb}{0.000000,0.000000,0.000000}%
\pgfsetstrokecolor{textcolor}%
\pgfsetfillcolor{textcolor}%
\pgftext[x=2.375088in,y=3.278491in,left,base]{\color{textcolor}\sffamily\fontsize{10.000000}{12.000000}\selectfont Estimated turn}%
\end{pgfscope}%
\end{pgfpicture}%
\makeatother%
\endgroup%
}
%         \caption{MPU-9250 Breakout}
%         \label{fig:square43D}
%     \end{subfigure}
%     \caption{Position estimation by the best performing algorithms in the 4-meter line experiment.}
%     \label{fig:square4}
% \end{figure}

% \subsubsection{16 meter}

% For the 16-meter line experiment, the Mahony algorithm which had the lowest displacement error with an average of 0.48 meters (16\% of error margin), and ROLEQ with an average of 0.24 meters of turn error (7\% of error margin).

% \begin{figure}[!h]
%     \centering
%     \begin{table}[H]
    \begin{center}
        \begin{tabular}[t]{lcccc}
            \hline
            Algorithm                   & Displacement Error[$m$] & Displacement Error[\%]      & Turn Error[$m$]  & Turn Error[\%]             \\
            \hline 
            AngularRate            & 29.43  & 45.99 & 35.94 & 56.16              \\            AQUA            & 19.83  & 30.99 & 19.41 & 30.33              \\            Complementary            & 10.27  & 16.05 & 10.00 & 15.62              \\            Davenport            & 12.95  & 20.23 & 12.29 & 19.20              \\            EKF            & 11.49  & 17.95 & 8.76 & 13.68              \\            FAMC            & 29.13  & 45.52 & 34.52 & 53.94              \\            FLAE            & 13.22  & 20.66 & 12.28 & 19.19              \\            Fourati            & 28.71  & 44.85 & 34.08 & 53.24              \\            Madgwick            & 12.60  & 19.69 & 9.32 & 14.56              \\            Mahony            & 11.57  & 18.08 & 10.81 & 16.90              \\            OLEQ            & 12.31  & 19.24 & 10.24 & 15.99              \\            QUEST            & 25.38  & 39.66 & 31.03 & 48.49              \\            ROLEQ            & 12.50  & 19.53 & 11.54 & 18.03              \\            SAAM            & 13.39  & 20.92 & 12.53 & 19.58              \\            Tilt            & 13.39  & 20.92 & 12.53 & 19.58              \\
            \hline
            Average & 17.08 & 26.69 & 17.68 & 27.63
        \end{tabular}
        \caption{Accelerometer Specifications. }
        \label{tab:accelerometer_specification}
    \end{center}
\end{table}
% \end{figure}

% \begin{figure}[!h]
%     \centering
%     \begin{subfigure}{0.49\textwidth}
%         \centering
%         \resizebox{1\linewidth}{!}{%% Creator: Matplotlib, PGF backend
%%
%% To include the figure in your LaTeX document, write
%%   \input{<filename>.pgf}
%%
%% Make sure the required packages are loaded in your preamble
%%   \usepackage{pgf}
%%
%% and, on pdftex
%%   \usepackage[utf8]{inputenc}\DeclareUnicodeCharacter{2212}{-}
%%
%% or, on luatex and xetex
%%   \usepackage{unicode-math}
%%
%% Figures using additional raster images can only be included by \input if
%% they are in the same directory as the main LaTeX file. For loading figures
%% from other directories you can use the `import` package
%%   \usepackage{import}
%%
%% and then include the figures with
%%   \import{<path to file>}{<filename>.pgf}
%%
%% Matplotlib used the following preamble
%%   \usepackage{fontspec}
%%
\begingroup%
\makeatletter%
\begin{pgfpicture}%
\pgfpathrectangle{\pgfpointorigin}{\pgfqpoint{4.342355in}{4.207622in}}%
\pgfusepath{use as bounding box, clip}%
\begin{pgfscope}%
\pgfsetbuttcap%
\pgfsetmiterjoin%
\definecolor{currentfill}{rgb}{1.000000,1.000000,1.000000}%
\pgfsetfillcolor{currentfill}%
\pgfsetlinewidth{0.000000pt}%
\definecolor{currentstroke}{rgb}{1.000000,1.000000,1.000000}%
\pgfsetstrokecolor{currentstroke}%
\pgfsetdash{}{0pt}%
\pgfpathmoveto{\pgfqpoint{0.000000in}{-0.000000in}}%
\pgfpathlineto{\pgfqpoint{4.342355in}{-0.000000in}}%
\pgfpathlineto{\pgfqpoint{4.342355in}{4.207622in}}%
\pgfpathlineto{\pgfqpoint{0.000000in}{4.207622in}}%
\pgfpathclose%
\pgfusepath{fill}%
\end{pgfscope}%
\begin{pgfscope}%
\pgfsetbuttcap%
\pgfsetmiterjoin%
\definecolor{currentfill}{rgb}{1.000000,1.000000,1.000000}%
\pgfsetfillcolor{currentfill}%
\pgfsetlinewidth{0.000000pt}%
\definecolor{currentstroke}{rgb}{0.000000,0.000000,0.000000}%
\pgfsetstrokecolor{currentstroke}%
\pgfsetstrokeopacity{0.000000}%
\pgfsetdash{}{0pt}%
\pgfpathmoveto{\pgfqpoint{0.100000in}{0.212622in}}%
\pgfpathlineto{\pgfqpoint{3.796000in}{0.212622in}}%
\pgfpathlineto{\pgfqpoint{3.796000in}{3.908622in}}%
\pgfpathlineto{\pgfqpoint{0.100000in}{3.908622in}}%
\pgfpathclose%
\pgfusepath{fill}%
\end{pgfscope}%
\begin{pgfscope}%
\pgfsetbuttcap%
\pgfsetmiterjoin%
\definecolor{currentfill}{rgb}{0.950000,0.950000,0.950000}%
\pgfsetfillcolor{currentfill}%
\pgfsetfillopacity{0.500000}%
\pgfsetlinewidth{1.003750pt}%
\definecolor{currentstroke}{rgb}{0.950000,0.950000,0.950000}%
\pgfsetstrokecolor{currentstroke}%
\pgfsetstrokeopacity{0.500000}%
\pgfsetdash{}{0pt}%
\pgfpathmoveto{\pgfqpoint{0.379073in}{1.123938in}}%
\pgfpathlineto{\pgfqpoint{1.599613in}{2.147018in}}%
\pgfpathlineto{\pgfqpoint{1.582647in}{3.622484in}}%
\pgfpathlineto{\pgfqpoint{0.303698in}{2.689165in}}%
\pgfusepath{stroke,fill}%
\end{pgfscope}%
\begin{pgfscope}%
\pgfsetbuttcap%
\pgfsetmiterjoin%
\definecolor{currentfill}{rgb}{0.900000,0.900000,0.900000}%
\pgfsetfillcolor{currentfill}%
\pgfsetfillopacity{0.500000}%
\pgfsetlinewidth{1.003750pt}%
\definecolor{currentstroke}{rgb}{0.900000,0.900000,0.900000}%
\pgfsetstrokecolor{currentstroke}%
\pgfsetstrokeopacity{0.500000}%
\pgfsetdash{}{0pt}%
\pgfpathmoveto{\pgfqpoint{1.599613in}{2.147018in}}%
\pgfpathlineto{\pgfqpoint{3.558144in}{1.577751in}}%
\pgfpathlineto{\pgfqpoint{3.628038in}{3.104037in}}%
\pgfpathlineto{\pgfqpoint{1.582647in}{3.622484in}}%
\pgfusepath{stroke,fill}%
\end{pgfscope}%
\begin{pgfscope}%
\pgfsetbuttcap%
\pgfsetmiterjoin%
\definecolor{currentfill}{rgb}{0.925000,0.925000,0.925000}%
\pgfsetfillcolor{currentfill}%
\pgfsetfillopacity{0.500000}%
\pgfsetlinewidth{1.003750pt}%
\definecolor{currentstroke}{rgb}{0.925000,0.925000,0.925000}%
\pgfsetstrokecolor{currentstroke}%
\pgfsetstrokeopacity{0.500000}%
\pgfsetdash{}{0pt}%
\pgfpathmoveto{\pgfqpoint{0.379073in}{1.123938in}}%
\pgfpathlineto{\pgfqpoint{2.455212in}{0.445871in}}%
\pgfpathlineto{\pgfqpoint{3.558144in}{1.577751in}}%
\pgfpathlineto{\pgfqpoint{1.599613in}{2.147018in}}%
\pgfusepath{stroke,fill}%
\end{pgfscope}%
\begin{pgfscope}%
\pgfsetrectcap%
\pgfsetroundjoin%
\pgfsetlinewidth{0.803000pt}%
\definecolor{currentstroke}{rgb}{0.000000,0.000000,0.000000}%
\pgfsetstrokecolor{currentstroke}%
\pgfsetdash{}{0pt}%
\pgfpathmoveto{\pgfqpoint{0.379073in}{1.123938in}}%
\pgfpathlineto{\pgfqpoint{2.455212in}{0.445871in}}%
\pgfusepath{stroke}%
\end{pgfscope}%
\begin{pgfscope}%
\definecolor{textcolor}{rgb}{0.000000,0.000000,0.000000}%
\pgfsetstrokecolor{textcolor}%
\pgfsetfillcolor{textcolor}%
\pgftext[x=0.730374in, y=0.408886in, left, base,rotate=341.912962]{\color{textcolor}\rmfamily\fontsize{10.000000}{12.000000}\selectfont Position X [\(\displaystyle m\)]}%
\end{pgfscope}%
\begin{pgfscope}%
\pgfsetbuttcap%
\pgfsetroundjoin%
\pgfsetlinewidth{0.803000pt}%
\definecolor{currentstroke}{rgb}{0.690196,0.690196,0.690196}%
\pgfsetstrokecolor{currentstroke}%
\pgfsetdash{}{0pt}%
\pgfpathmoveto{\pgfqpoint{0.729096in}{1.009620in}}%
\pgfpathlineto{\pgfqpoint{1.931022in}{2.050691in}}%
\pgfpathlineto{\pgfqpoint{1.928147in}{3.534909in}}%
\pgfusepath{stroke}%
\end{pgfscope}%
\begin{pgfscope}%
\pgfsetbuttcap%
\pgfsetroundjoin%
\pgfsetlinewidth{0.803000pt}%
\definecolor{currentstroke}{rgb}{0.690196,0.690196,0.690196}%
\pgfsetstrokecolor{currentstroke}%
\pgfsetdash{}{0pt}%
\pgfpathmoveto{\pgfqpoint{1.166492in}{0.866767in}}%
\pgfpathlineto{\pgfqpoint{2.344463in}{1.930520in}}%
\pgfpathlineto{\pgfqpoint{2.359513in}{3.425571in}}%
\pgfusepath{stroke}%
\end{pgfscope}%
\begin{pgfscope}%
\pgfsetbuttcap%
\pgfsetroundjoin%
\pgfsetlinewidth{0.803000pt}%
\definecolor{currentstroke}{rgb}{0.690196,0.690196,0.690196}%
\pgfsetstrokecolor{currentstroke}%
\pgfsetdash{}{0pt}%
\pgfpathmoveto{\pgfqpoint{1.613890in}{0.720646in}}%
\pgfpathlineto{\pgfqpoint{2.766564in}{1.807832in}}%
\pgfpathlineto{\pgfqpoint{2.800311in}{3.313841in}}%
\pgfusepath{stroke}%
\end{pgfscope}%
\begin{pgfscope}%
\pgfsetbuttcap%
\pgfsetroundjoin%
\pgfsetlinewidth{0.803000pt}%
\definecolor{currentstroke}{rgb}{0.690196,0.690196,0.690196}%
\pgfsetstrokecolor{currentstroke}%
\pgfsetdash{}{0pt}%
\pgfpathmoveto{\pgfqpoint{2.071639in}{0.571146in}}%
\pgfpathlineto{\pgfqpoint{3.197600in}{1.682547in}}%
\pgfpathlineto{\pgfqpoint{3.250854in}{3.199642in}}%
\pgfusepath{stroke}%
\end{pgfscope}%
\begin{pgfscope}%
\pgfsetrectcap%
\pgfsetroundjoin%
\pgfsetlinewidth{0.803000pt}%
\definecolor{currentstroke}{rgb}{0.000000,0.000000,0.000000}%
\pgfsetstrokecolor{currentstroke}%
\pgfsetdash{}{0pt}%
\pgfpathmoveto{\pgfqpoint{0.739568in}{1.018690in}}%
\pgfpathlineto{\pgfqpoint{0.708109in}{0.991441in}}%
\pgfusepath{stroke}%
\end{pgfscope}%
\begin{pgfscope}%
\definecolor{textcolor}{rgb}{0.000000,0.000000,0.000000}%
\pgfsetstrokecolor{textcolor}%
\pgfsetfillcolor{textcolor}%
\pgftext[x=0.624753in,y=0.789865in,,top]{\color{textcolor}\rmfamily\fontsize{10.000000}{12.000000}\selectfont \(\displaystyle {0}\)}%
\end{pgfscope}%
\begin{pgfscope}%
\pgfsetrectcap%
\pgfsetroundjoin%
\pgfsetlinewidth{0.803000pt}%
\definecolor{currentstroke}{rgb}{0.000000,0.000000,0.000000}%
\pgfsetstrokecolor{currentstroke}%
\pgfsetdash{}{0pt}%
\pgfpathmoveto{\pgfqpoint{1.176764in}{0.876043in}}%
\pgfpathlineto{\pgfqpoint{1.145903in}{0.848174in}}%
\pgfusepath{stroke}%
\end{pgfscope}%
\begin{pgfscope}%
\definecolor{textcolor}{rgb}{0.000000,0.000000,0.000000}%
\pgfsetstrokecolor{textcolor}%
\pgfsetfillcolor{textcolor}%
\pgftext[x=1.062618in,y=0.643977in,,top]{\color{textcolor}\rmfamily\fontsize{10.000000}{12.000000}\selectfont \(\displaystyle {5}\)}%
\end{pgfscope}%
\begin{pgfscope}%
\pgfsetrectcap%
\pgfsetroundjoin%
\pgfsetlinewidth{0.803000pt}%
\definecolor{currentstroke}{rgb}{0.000000,0.000000,0.000000}%
\pgfsetstrokecolor{currentstroke}%
\pgfsetdash{}{0pt}%
\pgfpathmoveto{\pgfqpoint{1.623951in}{0.730136in}}%
\pgfpathlineto{\pgfqpoint{1.593724in}{0.701626in}}%
\pgfusepath{stroke}%
\end{pgfscope}%
\begin{pgfscope}%
\definecolor{textcolor}{rgb}{0.000000,0.000000,0.000000}%
\pgfsetstrokecolor{textcolor}%
\pgfsetfillcolor{textcolor}%
\pgftext[x=1.510535in,y=0.494741in,,top]{\color{textcolor}\rmfamily\fontsize{10.000000}{12.000000}\selectfont \(\displaystyle {10}\)}%
\end{pgfscope}%
\begin{pgfscope}%
\pgfsetrectcap%
\pgfsetroundjoin%
\pgfsetlinewidth{0.803000pt}%
\definecolor{currentstroke}{rgb}{0.000000,0.000000,0.000000}%
\pgfsetstrokecolor{currentstroke}%
\pgfsetdash{}{0pt}%
\pgfpathmoveto{\pgfqpoint{2.081476in}{0.580856in}}%
\pgfpathlineto{\pgfqpoint{2.051920in}{0.551682in}}%
\pgfusepath{stroke}%
\end{pgfscope}%
\begin{pgfscope}%
\definecolor{textcolor}{rgb}{0.000000,0.000000,0.000000}%
\pgfsetstrokecolor{textcolor}%
\pgfsetfillcolor{textcolor}%
\pgftext[x=1.968855in,y=0.342038in,,top]{\color{textcolor}\rmfamily\fontsize{10.000000}{12.000000}\selectfont \(\displaystyle {15}\)}%
\end{pgfscope}%
\begin{pgfscope}%
\pgfsetrectcap%
\pgfsetroundjoin%
\pgfsetlinewidth{0.803000pt}%
\definecolor{currentstroke}{rgb}{0.000000,0.000000,0.000000}%
\pgfsetstrokecolor{currentstroke}%
\pgfsetdash{}{0pt}%
\pgfpathmoveto{\pgfqpoint{3.558144in}{1.577751in}}%
\pgfpathlineto{\pgfqpoint{2.455212in}{0.445871in}}%
\pgfusepath{stroke}%
\end{pgfscope}%
\begin{pgfscope}%
\definecolor{textcolor}{rgb}{0.000000,0.000000,0.000000}%
\pgfsetstrokecolor{textcolor}%
\pgfsetfillcolor{textcolor}%
\pgftext[x=3.120747in, y=0.305657in, left, base,rotate=45.742112]{\color{textcolor}\rmfamily\fontsize{10.000000}{12.000000}\selectfont Position Y [\(\displaystyle m\)]}%
\end{pgfscope}%
\begin{pgfscope}%
\pgfsetbuttcap%
\pgfsetroundjoin%
\pgfsetlinewidth{0.803000pt}%
\definecolor{currentstroke}{rgb}{0.690196,0.690196,0.690196}%
\pgfsetstrokecolor{currentstroke}%
\pgfsetdash{}{0pt}%
\pgfpathmoveto{\pgfqpoint{0.544927in}{2.865203in}}%
\pgfpathlineto{\pgfqpoint{0.608577in}{1.316312in}}%
\pgfpathlineto{\pgfqpoint{2.663346in}{0.659468in}}%
\pgfusepath{stroke}%
\end{pgfscope}%
\begin{pgfscope}%
\pgfsetbuttcap%
\pgfsetroundjoin%
\pgfsetlinewidth{0.803000pt}%
\definecolor{currentstroke}{rgb}{0.690196,0.690196,0.690196}%
\pgfsetstrokecolor{currentstroke}%
\pgfsetdash{}{0pt}%
\pgfpathmoveto{\pgfqpoint{0.831044in}{3.073998in}}%
\pgfpathlineto{\pgfqpoint{0.881213in}{1.544841in}}%
\pgfpathlineto{\pgfqpoint{2.910147in}{0.912746in}}%
\pgfusepath{stroke}%
\end{pgfscope}%
\begin{pgfscope}%
\pgfsetbuttcap%
\pgfsetroundjoin%
\pgfsetlinewidth{0.803000pt}%
\definecolor{currentstroke}{rgb}{0.690196,0.690196,0.690196}%
\pgfsetstrokecolor{currentstroke}%
\pgfsetdash{}{0pt}%
\pgfpathmoveto{\pgfqpoint{1.106287in}{3.274858in}}%
\pgfpathlineto{\pgfqpoint{1.143926in}{1.765052in}}%
\pgfpathlineto{\pgfqpoint{3.147504in}{1.156332in}}%
\pgfusepath{stroke}%
\end{pgfscope}%
\begin{pgfscope}%
\pgfsetbuttcap%
\pgfsetroundjoin%
\pgfsetlinewidth{0.803000pt}%
\definecolor{currentstroke}{rgb}{0.690196,0.690196,0.690196}%
\pgfsetstrokecolor{currentstroke}%
\pgfsetdash{}{0pt}%
\pgfpathmoveto{\pgfqpoint{1.371267in}{3.468228in}}%
\pgfpathlineto{\pgfqpoint{1.397247in}{1.977391in}}%
\pgfpathlineto{\pgfqpoint{3.375950in}{1.390774in}}%
\pgfusepath{stroke}%
\end{pgfscope}%
\begin{pgfscope}%
\pgfsetrectcap%
\pgfsetroundjoin%
\pgfsetlinewidth{0.803000pt}%
\definecolor{currentstroke}{rgb}{0.000000,0.000000,0.000000}%
\pgfsetstrokecolor{currentstroke}%
\pgfsetdash{}{0pt}%
\pgfpathmoveto{\pgfqpoint{2.646040in}{0.665000in}}%
\pgfpathlineto{\pgfqpoint{2.698004in}{0.648389in}}%
\pgfusepath{stroke}%
\end{pgfscope}%
\begin{pgfscope}%
\definecolor{textcolor}{rgb}{0.000000,0.000000,0.000000}%
\pgfsetstrokecolor{textcolor}%
\pgfsetfillcolor{textcolor}%
\pgftext[x=2.840476in,y=0.475036in,,top]{\color{textcolor}\rmfamily\fontsize{10.000000}{12.000000}\selectfont \(\displaystyle {0}\)}%
\end{pgfscope}%
\begin{pgfscope}%
\pgfsetrectcap%
\pgfsetroundjoin%
\pgfsetlinewidth{0.803000pt}%
\definecolor{currentstroke}{rgb}{0.000000,0.000000,0.000000}%
\pgfsetstrokecolor{currentstroke}%
\pgfsetdash{}{0pt}%
\pgfpathmoveto{\pgfqpoint{2.893074in}{0.918064in}}%
\pgfpathlineto{\pgfqpoint{2.944334in}{0.902095in}}%
\pgfusepath{stroke}%
\end{pgfscope}%
\begin{pgfscope}%
\definecolor{textcolor}{rgb}{0.000000,0.000000,0.000000}%
\pgfsetstrokecolor{textcolor}%
\pgfsetfillcolor{textcolor}%
\pgftext[x=3.083964in,y=0.732061in,,top]{\color{textcolor}\rmfamily\fontsize{10.000000}{12.000000}\selectfont \(\displaystyle {5}\)}%
\end{pgfscope}%
\begin{pgfscope}%
\pgfsetrectcap%
\pgfsetroundjoin%
\pgfsetlinewidth{0.803000pt}%
\definecolor{currentstroke}{rgb}{0.000000,0.000000,0.000000}%
\pgfsetstrokecolor{currentstroke}%
\pgfsetdash{}{0pt}%
\pgfpathmoveto{\pgfqpoint{3.130661in}{1.161450in}}%
\pgfpathlineto{\pgfqpoint{3.181231in}{1.146086in}}%
\pgfusepath{stroke}%
\end{pgfscope}%
\begin{pgfscope}%
\definecolor{textcolor}{rgb}{0.000000,0.000000,0.000000}%
\pgfsetstrokecolor{textcolor}%
\pgfsetfillcolor{textcolor}%
\pgftext[x=3.318129in,y=0.979245in,,top]{\color{textcolor}\rmfamily\fontsize{10.000000}{12.000000}\selectfont \(\displaystyle {10}\)}%
\end{pgfscope}%
\begin{pgfscope}%
\pgfsetrectcap%
\pgfsetroundjoin%
\pgfsetlinewidth{0.803000pt}%
\definecolor{currentstroke}{rgb}{0.000000,0.000000,0.000000}%
\pgfsetstrokecolor{currentstroke}%
\pgfsetdash{}{0pt}%
\pgfpathmoveto{\pgfqpoint{3.359331in}{1.395701in}}%
\pgfpathlineto{\pgfqpoint{3.409227in}{1.380908in}}%
\pgfusepath{stroke}%
\end{pgfscope}%
\begin{pgfscope}%
\definecolor{textcolor}{rgb}{0.000000,0.000000,0.000000}%
\pgfsetstrokecolor{textcolor}%
\pgfsetfillcolor{textcolor}%
\pgftext[x=3.543497in,y=1.217144in,,top]{\color{textcolor}\rmfamily\fontsize{10.000000}{12.000000}\selectfont \(\displaystyle {15}\)}%
\end{pgfscope}%
\begin{pgfscope}%
\pgfsetrectcap%
\pgfsetroundjoin%
\pgfsetlinewidth{0.803000pt}%
\definecolor{currentstroke}{rgb}{0.000000,0.000000,0.000000}%
\pgfsetstrokecolor{currentstroke}%
\pgfsetdash{}{0pt}%
\pgfpathmoveto{\pgfqpoint{3.558144in}{1.577751in}}%
\pgfpathlineto{\pgfqpoint{3.628038in}{3.104037in}}%
\pgfusepath{stroke}%
\end{pgfscope}%
\begin{pgfscope}%
\definecolor{textcolor}{rgb}{0.000000,0.000000,0.000000}%
\pgfsetstrokecolor{textcolor}%
\pgfsetfillcolor{textcolor}%
\pgftext[x=4.167903in, y=1.963517in, left, base,rotate=87.378092]{\color{textcolor}\rmfamily\fontsize{10.000000}{12.000000}\selectfont Position Z [\(\displaystyle m\)]}%
\end{pgfscope}%
\begin{pgfscope}%
\pgfsetbuttcap%
\pgfsetroundjoin%
\pgfsetlinewidth{0.803000pt}%
\definecolor{currentstroke}{rgb}{0.690196,0.690196,0.690196}%
\pgfsetstrokecolor{currentstroke}%
\pgfsetdash{}{0pt}%
\pgfpathmoveto{\pgfqpoint{3.562413in}{1.670968in}}%
\pgfpathlineto{\pgfqpoint{1.598575in}{2.237310in}}%
\pgfpathlineto{\pgfqpoint{0.374477in}{1.219382in}}%
\pgfusepath{stroke}%
\end{pgfscope}%
\begin{pgfscope}%
\pgfsetbuttcap%
\pgfsetroundjoin%
\pgfsetlinewidth{0.803000pt}%
\definecolor{currentstroke}{rgb}{0.690196,0.690196,0.690196}%
\pgfsetstrokecolor{currentstroke}%
\pgfsetdash{}{0pt}%
\pgfpathmoveto{\pgfqpoint{3.572337in}{1.887689in}}%
\pgfpathlineto{\pgfqpoint{1.596162in}{2.447141in}}%
\pgfpathlineto{\pgfqpoint{0.363788in}{1.441357in}}%
\pgfusepath{stroke}%
\end{pgfscope}%
\begin{pgfscope}%
\pgfsetbuttcap%
\pgfsetroundjoin%
\pgfsetlinewidth{0.803000pt}%
\definecolor{currentstroke}{rgb}{0.690196,0.690196,0.690196}%
\pgfsetstrokecolor{currentstroke}%
\pgfsetdash{}{0pt}%
\pgfpathmoveto{\pgfqpoint{3.582388in}{2.107177in}}%
\pgfpathlineto{\pgfqpoint{1.593720in}{2.659522in}}%
\pgfpathlineto{\pgfqpoint{0.352956in}{1.666274in}}%
\pgfusepath{stroke}%
\end{pgfscope}%
\begin{pgfscope}%
\pgfsetbuttcap%
\pgfsetroundjoin%
\pgfsetlinewidth{0.803000pt}%
\definecolor{currentstroke}{rgb}{0.690196,0.690196,0.690196}%
\pgfsetstrokecolor{currentstroke}%
\pgfsetdash{}{0pt}%
\pgfpathmoveto{\pgfqpoint{3.592569in}{2.329486in}}%
\pgfpathlineto{\pgfqpoint{1.591248in}{2.874502in}}%
\pgfpathlineto{\pgfqpoint{0.341981in}{1.894193in}}%
\pgfusepath{stroke}%
\end{pgfscope}%
\begin{pgfscope}%
\pgfsetbuttcap%
\pgfsetroundjoin%
\pgfsetlinewidth{0.803000pt}%
\definecolor{currentstroke}{rgb}{0.690196,0.690196,0.690196}%
\pgfsetstrokecolor{currentstroke}%
\pgfsetdash{}{0pt}%
\pgfpathmoveto{\pgfqpoint{3.602880in}{2.554670in}}%
\pgfpathlineto{\pgfqpoint{1.588745in}{3.092126in}}%
\pgfpathlineto{\pgfqpoint{0.330858in}{2.125173in}}%
\pgfusepath{stroke}%
\end{pgfscope}%
\begin{pgfscope}%
\pgfsetbuttcap%
\pgfsetroundjoin%
\pgfsetlinewidth{0.803000pt}%
\definecolor{currentstroke}{rgb}{0.690196,0.690196,0.690196}%
\pgfsetstrokecolor{currentstroke}%
\pgfsetdash{}{0pt}%
\pgfpathmoveto{\pgfqpoint{3.613326in}{2.782785in}}%
\pgfpathlineto{\pgfqpoint{1.586212in}{3.312446in}}%
\pgfpathlineto{\pgfqpoint{0.319584in}{2.359278in}}%
\pgfusepath{stroke}%
\end{pgfscope}%
\begin{pgfscope}%
\pgfsetbuttcap%
\pgfsetroundjoin%
\pgfsetlinewidth{0.803000pt}%
\definecolor{currentstroke}{rgb}{0.690196,0.690196,0.690196}%
\pgfsetstrokecolor{currentstroke}%
\pgfsetdash{}{0pt}%
\pgfpathmoveto{\pgfqpoint{3.623909in}{3.013889in}}%
\pgfpathlineto{\pgfqpoint{1.583647in}{3.535511in}}%
\pgfpathlineto{\pgfqpoint{0.308157in}{2.596570in}}%
\pgfusepath{stroke}%
\end{pgfscope}%
\begin{pgfscope}%
\pgfsetrectcap%
\pgfsetroundjoin%
\pgfsetlinewidth{0.803000pt}%
\definecolor{currentstroke}{rgb}{0.000000,0.000000,0.000000}%
\pgfsetstrokecolor{currentstroke}%
\pgfsetdash{}{0pt}%
\pgfpathmoveto{\pgfqpoint{3.545929in}{1.675722in}}%
\pgfpathlineto{\pgfqpoint{3.595421in}{1.661449in}}%
\pgfusepath{stroke}%
\end{pgfscope}%
\begin{pgfscope}%
\definecolor{textcolor}{rgb}{0.000000,0.000000,0.000000}%
\pgfsetstrokecolor{textcolor}%
\pgfsetfillcolor{textcolor}%
\pgftext[x=3.816545in,y=1.706967in,,top]{\color{textcolor}\rmfamily\fontsize{10.000000}{12.000000}\selectfont \(\displaystyle {0}\)}%
\end{pgfscope}%
\begin{pgfscope}%
\pgfsetrectcap%
\pgfsetroundjoin%
\pgfsetlinewidth{0.803000pt}%
\definecolor{currentstroke}{rgb}{0.000000,0.000000,0.000000}%
\pgfsetstrokecolor{currentstroke}%
\pgfsetdash{}{0pt}%
\pgfpathmoveto{\pgfqpoint{3.555745in}{1.892386in}}%
\pgfpathlineto{\pgfqpoint{3.605562in}{1.878283in}}%
\pgfusepath{stroke}%
\end{pgfscope}%
\begin{pgfscope}%
\definecolor{textcolor}{rgb}{0.000000,0.000000,0.000000}%
\pgfsetstrokecolor{textcolor}%
\pgfsetfillcolor{textcolor}%
\pgftext[x=3.828046in,y=1.923260in,,top]{\color{textcolor}\rmfamily\fontsize{10.000000}{12.000000}\selectfont \(\displaystyle {1}\)}%
\end{pgfscope}%
\begin{pgfscope}%
\pgfsetrectcap%
\pgfsetroundjoin%
\pgfsetlinewidth{0.803000pt}%
\definecolor{currentstroke}{rgb}{0.000000,0.000000,0.000000}%
\pgfsetstrokecolor{currentstroke}%
\pgfsetdash{}{0pt}%
\pgfpathmoveto{\pgfqpoint{3.565686in}{2.111816in}}%
\pgfpathlineto{\pgfqpoint{3.615834in}{2.097888in}}%
\pgfusepath{stroke}%
\end{pgfscope}%
\begin{pgfscope}%
\definecolor{textcolor}{rgb}{0.000000,0.000000,0.000000}%
\pgfsetstrokecolor{textcolor}%
\pgfsetfillcolor{textcolor}%
\pgftext[x=3.839694in,y=2.142306in,,top]{\color{textcolor}\rmfamily\fontsize{10.000000}{12.000000}\selectfont \(\displaystyle {2}\)}%
\end{pgfscope}%
\begin{pgfscope}%
\pgfsetrectcap%
\pgfsetroundjoin%
\pgfsetlinewidth{0.803000pt}%
\definecolor{currentstroke}{rgb}{0.000000,0.000000,0.000000}%
\pgfsetstrokecolor{currentstroke}%
\pgfsetdash{}{0pt}%
\pgfpathmoveto{\pgfqpoint{3.575755in}{2.334065in}}%
\pgfpathlineto{\pgfqpoint{3.626237in}{2.320317in}}%
\pgfusepath{stroke}%
\end{pgfscope}%
\begin{pgfscope}%
\definecolor{textcolor}{rgb}{0.000000,0.000000,0.000000}%
\pgfsetstrokecolor{textcolor}%
\pgfsetfillcolor{textcolor}%
\pgftext[x=3.851491in,y=2.364159in,,top]{\color{textcolor}\rmfamily\fontsize{10.000000}{12.000000}\selectfont \(\displaystyle {3}\)}%
\end{pgfscope}%
\begin{pgfscope}%
\pgfsetrectcap%
\pgfsetroundjoin%
\pgfsetlinewidth{0.803000pt}%
\definecolor{currentstroke}{rgb}{0.000000,0.000000,0.000000}%
\pgfsetstrokecolor{currentstroke}%
\pgfsetdash{}{0pt}%
\pgfpathmoveto{\pgfqpoint{3.585954in}{2.559187in}}%
\pgfpathlineto{\pgfqpoint{3.636775in}{2.545625in}}%
\pgfusepath{stroke}%
\end{pgfscope}%
\begin{pgfscope}%
\definecolor{textcolor}{rgb}{0.000000,0.000000,0.000000}%
\pgfsetstrokecolor{textcolor}%
\pgfsetfillcolor{textcolor}%
\pgftext[x=3.863441in,y=2.588872in,,top]{\color{textcolor}\rmfamily\fontsize{10.000000}{12.000000}\selectfont \(\displaystyle {4}\)}%
\end{pgfscope}%
\begin{pgfscope}%
\pgfsetrectcap%
\pgfsetroundjoin%
\pgfsetlinewidth{0.803000pt}%
\definecolor{currentstroke}{rgb}{0.000000,0.000000,0.000000}%
\pgfsetstrokecolor{currentstroke}%
\pgfsetdash{}{0pt}%
\pgfpathmoveto{\pgfqpoint{3.596285in}{2.787238in}}%
\pgfpathlineto{\pgfqpoint{3.647451in}{2.773869in}}%
\pgfusepath{stroke}%
\end{pgfscope}%
\begin{pgfscope}%
\definecolor{textcolor}{rgb}{0.000000,0.000000,0.000000}%
\pgfsetstrokecolor{textcolor}%
\pgfsetfillcolor{textcolor}%
\pgftext[x=3.875545in,y=2.816501in,,top]{\color{textcolor}\rmfamily\fontsize{10.000000}{12.000000}\selectfont \(\displaystyle {5}\)}%
\end{pgfscope}%
\begin{pgfscope}%
\pgfsetrectcap%
\pgfsetroundjoin%
\pgfsetlinewidth{0.803000pt}%
\definecolor{currentstroke}{rgb}{0.000000,0.000000,0.000000}%
\pgfsetstrokecolor{currentstroke}%
\pgfsetdash{}{0pt}%
\pgfpathmoveto{\pgfqpoint{3.606752in}{3.018276in}}%
\pgfpathlineto{\pgfqpoint{3.658266in}{3.005105in}}%
\pgfusepath{stroke}%
\end{pgfscope}%
\begin{pgfscope}%
\definecolor{textcolor}{rgb}{0.000000,0.000000,0.000000}%
\pgfsetstrokecolor{textcolor}%
\pgfsetfillcolor{textcolor}%
\pgftext[x=3.887807in,y=3.047103in,,top]{\color{textcolor}\rmfamily\fontsize{10.000000}{12.000000}\selectfont \(\displaystyle {6}\)}%
\end{pgfscope}%
\begin{pgfscope}%
\pgfpathrectangle{\pgfqpoint{0.100000in}{0.212622in}}{\pgfqpoint{3.696000in}{3.696000in}}%
\pgfusepath{clip}%
\pgfsetrectcap%
\pgfsetroundjoin%
\pgfsetlinewidth{1.505625pt}%
\definecolor{currentstroke}{rgb}{0.121569,0.466667,0.705882}%
\pgfsetstrokecolor{currentstroke}%
\pgfsetdash{}{0pt}%
\pgfpathmoveto{\pgfqpoint{0.952296in}{1.300547in}}%
\pgfpathlineto{\pgfqpoint{1.780088in}{2.011930in}}%
\pgfpathlineto{\pgfqpoint{3.148269in}{1.611409in}}%
\pgfpathlineto{\pgfqpoint{2.376900in}{0.848592in}}%
\pgfpathlineto{\pgfqpoint{0.952296in}{1.300547in}}%
\pgfusepath{stroke}%
\end{pgfscope}%
\begin{pgfscope}%
\pgfpathrectangle{\pgfqpoint{0.100000in}{0.212622in}}{\pgfqpoint{3.696000in}{3.696000in}}%
\pgfusepath{clip}%
\pgfsetrectcap%
\pgfsetroundjoin%
\pgfsetlinewidth{1.505625pt}%
\definecolor{currentstroke}{rgb}{1.000000,0.000000,0.000000}%
\pgfsetstrokecolor{currentstroke}%
\pgfsetdash{}{0pt}%
\pgfpathmoveto{\pgfqpoint{0.952296in}{1.300547in}}%
\pgfpathlineto{\pgfqpoint{0.952296in}{1.300547in}}%
\pgfusepath{stroke}%
\end{pgfscope}%
\begin{pgfscope}%
\pgfpathrectangle{\pgfqpoint{0.100000in}{0.212622in}}{\pgfqpoint{3.696000in}{3.696000in}}%
\pgfusepath{clip}%
\pgfsetrectcap%
\pgfsetroundjoin%
\pgfsetlinewidth{1.505625pt}%
\definecolor{currentstroke}{rgb}{1.000000,0.000000,0.000000}%
\pgfsetstrokecolor{currentstroke}%
\pgfsetdash{}{0pt}%
\pgfpathmoveto{\pgfqpoint{0.950950in}{1.302067in}}%
\pgfpathlineto{\pgfqpoint{0.952296in}{1.300547in}}%
\pgfusepath{stroke}%
\end{pgfscope}%
\begin{pgfscope}%
\pgfpathrectangle{\pgfqpoint{0.100000in}{0.212622in}}{\pgfqpoint{3.696000in}{3.696000in}}%
\pgfusepath{clip}%
\pgfsetrectcap%
\pgfsetroundjoin%
\pgfsetlinewidth{1.505625pt}%
\definecolor{currentstroke}{rgb}{1.000000,0.000000,0.000000}%
\pgfsetstrokecolor{currentstroke}%
\pgfsetdash{}{0pt}%
\pgfpathmoveto{\pgfqpoint{0.948252in}{1.304961in}}%
\pgfpathlineto{\pgfqpoint{0.952296in}{1.300547in}}%
\pgfusepath{stroke}%
\end{pgfscope}%
\begin{pgfscope}%
\pgfpathrectangle{\pgfqpoint{0.100000in}{0.212622in}}{\pgfqpoint{3.696000in}{3.696000in}}%
\pgfusepath{clip}%
\pgfsetrectcap%
\pgfsetroundjoin%
\pgfsetlinewidth{1.505625pt}%
\definecolor{currentstroke}{rgb}{1.000000,0.000000,0.000000}%
\pgfsetstrokecolor{currentstroke}%
\pgfsetdash{}{0pt}%
\pgfpathmoveto{\pgfqpoint{0.944313in}{1.308964in}}%
\pgfpathlineto{\pgfqpoint{0.952296in}{1.300547in}}%
\pgfusepath{stroke}%
\end{pgfscope}%
\begin{pgfscope}%
\pgfpathrectangle{\pgfqpoint{0.100000in}{0.212622in}}{\pgfqpoint{3.696000in}{3.696000in}}%
\pgfusepath{clip}%
\pgfsetrectcap%
\pgfsetroundjoin%
\pgfsetlinewidth{1.505625pt}%
\definecolor{currentstroke}{rgb}{1.000000,0.000000,0.000000}%
\pgfsetstrokecolor{currentstroke}%
\pgfsetdash{}{0pt}%
\pgfpathmoveto{\pgfqpoint{0.939412in}{1.313684in}}%
\pgfpathlineto{\pgfqpoint{0.952296in}{1.300547in}}%
\pgfusepath{stroke}%
\end{pgfscope}%
\begin{pgfscope}%
\pgfpathrectangle{\pgfqpoint{0.100000in}{0.212622in}}{\pgfqpoint{3.696000in}{3.696000in}}%
\pgfusepath{clip}%
\pgfsetrectcap%
\pgfsetroundjoin%
\pgfsetlinewidth{1.505625pt}%
\definecolor{currentstroke}{rgb}{1.000000,0.000000,0.000000}%
\pgfsetstrokecolor{currentstroke}%
\pgfsetdash{}{0pt}%
\pgfpathmoveto{\pgfqpoint{0.936666in}{1.316184in}}%
\pgfpathlineto{\pgfqpoint{0.952296in}{1.300547in}}%
\pgfusepath{stroke}%
\end{pgfscope}%
\begin{pgfscope}%
\pgfpathrectangle{\pgfqpoint{0.100000in}{0.212622in}}{\pgfqpoint{3.696000in}{3.696000in}}%
\pgfusepath{clip}%
\pgfsetrectcap%
\pgfsetroundjoin%
\pgfsetlinewidth{1.505625pt}%
\definecolor{currentstroke}{rgb}{1.000000,0.000000,0.000000}%
\pgfsetstrokecolor{currentstroke}%
\pgfsetdash{}{0pt}%
\pgfpathmoveto{\pgfqpoint{0.935130in}{1.317502in}}%
\pgfpathlineto{\pgfqpoint{0.952296in}{1.300547in}}%
\pgfusepath{stroke}%
\end{pgfscope}%
\begin{pgfscope}%
\pgfpathrectangle{\pgfqpoint{0.100000in}{0.212622in}}{\pgfqpoint{3.696000in}{3.696000in}}%
\pgfusepath{clip}%
\pgfsetrectcap%
\pgfsetroundjoin%
\pgfsetlinewidth{1.505625pt}%
\definecolor{currentstroke}{rgb}{1.000000,0.000000,0.000000}%
\pgfsetstrokecolor{currentstroke}%
\pgfsetdash{}{0pt}%
\pgfpathmoveto{\pgfqpoint{0.934272in}{1.318194in}}%
\pgfpathlineto{\pgfqpoint{0.952296in}{1.300547in}}%
\pgfusepath{stroke}%
\end{pgfscope}%
\begin{pgfscope}%
\pgfpathrectangle{\pgfqpoint{0.100000in}{0.212622in}}{\pgfqpoint{3.696000in}{3.696000in}}%
\pgfusepath{clip}%
\pgfsetrectcap%
\pgfsetroundjoin%
\pgfsetlinewidth{1.505625pt}%
\definecolor{currentstroke}{rgb}{1.000000,0.000000,0.000000}%
\pgfsetstrokecolor{currentstroke}%
\pgfsetdash{}{0pt}%
\pgfpathmoveto{\pgfqpoint{0.933793in}{1.318558in}}%
\pgfpathlineto{\pgfqpoint{0.952296in}{1.300547in}}%
\pgfusepath{stroke}%
\end{pgfscope}%
\begin{pgfscope}%
\pgfpathrectangle{\pgfqpoint{0.100000in}{0.212622in}}{\pgfqpoint{3.696000in}{3.696000in}}%
\pgfusepath{clip}%
\pgfsetrectcap%
\pgfsetroundjoin%
\pgfsetlinewidth{1.505625pt}%
\definecolor{currentstroke}{rgb}{1.000000,0.000000,0.000000}%
\pgfsetstrokecolor{currentstroke}%
\pgfsetdash{}{0pt}%
\pgfpathmoveto{\pgfqpoint{0.933527in}{1.318747in}}%
\pgfpathlineto{\pgfqpoint{0.952296in}{1.300547in}}%
\pgfusepath{stroke}%
\end{pgfscope}%
\begin{pgfscope}%
\pgfpathrectangle{\pgfqpoint{0.100000in}{0.212622in}}{\pgfqpoint{3.696000in}{3.696000in}}%
\pgfusepath{clip}%
\pgfsetrectcap%
\pgfsetroundjoin%
\pgfsetlinewidth{1.505625pt}%
\definecolor{currentstroke}{rgb}{1.000000,0.000000,0.000000}%
\pgfsetstrokecolor{currentstroke}%
\pgfsetdash{}{0pt}%
\pgfpathmoveto{\pgfqpoint{0.933379in}{1.318845in}}%
\pgfpathlineto{\pgfqpoint{0.952296in}{1.300547in}}%
\pgfusepath{stroke}%
\end{pgfscope}%
\begin{pgfscope}%
\pgfpathrectangle{\pgfqpoint{0.100000in}{0.212622in}}{\pgfqpoint{3.696000in}{3.696000in}}%
\pgfusepath{clip}%
\pgfsetrectcap%
\pgfsetroundjoin%
\pgfsetlinewidth{1.505625pt}%
\definecolor{currentstroke}{rgb}{1.000000,0.000000,0.000000}%
\pgfsetstrokecolor{currentstroke}%
\pgfsetdash{}{0pt}%
\pgfpathmoveto{\pgfqpoint{0.933297in}{1.318896in}}%
\pgfpathlineto{\pgfqpoint{0.952296in}{1.300547in}}%
\pgfusepath{stroke}%
\end{pgfscope}%
\begin{pgfscope}%
\pgfpathrectangle{\pgfqpoint{0.100000in}{0.212622in}}{\pgfqpoint{3.696000in}{3.696000in}}%
\pgfusepath{clip}%
\pgfsetrectcap%
\pgfsetroundjoin%
\pgfsetlinewidth{1.505625pt}%
\definecolor{currentstroke}{rgb}{1.000000,0.000000,0.000000}%
\pgfsetstrokecolor{currentstroke}%
\pgfsetdash{}{0pt}%
\pgfpathmoveto{\pgfqpoint{0.933252in}{1.318922in}}%
\pgfpathlineto{\pgfqpoint{0.952296in}{1.300547in}}%
\pgfusepath{stroke}%
\end{pgfscope}%
\begin{pgfscope}%
\pgfpathrectangle{\pgfqpoint{0.100000in}{0.212622in}}{\pgfqpoint{3.696000in}{3.696000in}}%
\pgfusepath{clip}%
\pgfsetrectcap%
\pgfsetroundjoin%
\pgfsetlinewidth{1.505625pt}%
\definecolor{currentstroke}{rgb}{1.000000,0.000000,0.000000}%
\pgfsetstrokecolor{currentstroke}%
\pgfsetdash{}{0pt}%
\pgfpathmoveto{\pgfqpoint{0.933226in}{1.318935in}}%
\pgfpathlineto{\pgfqpoint{0.952296in}{1.300547in}}%
\pgfusepath{stroke}%
\end{pgfscope}%
\begin{pgfscope}%
\pgfpathrectangle{\pgfqpoint{0.100000in}{0.212622in}}{\pgfqpoint{3.696000in}{3.696000in}}%
\pgfusepath{clip}%
\pgfsetrectcap%
\pgfsetroundjoin%
\pgfsetlinewidth{1.505625pt}%
\definecolor{currentstroke}{rgb}{1.000000,0.000000,0.000000}%
\pgfsetstrokecolor{currentstroke}%
\pgfsetdash{}{0pt}%
\pgfpathmoveto{\pgfqpoint{0.933213in}{1.318942in}}%
\pgfpathlineto{\pgfqpoint{0.952296in}{1.300547in}}%
\pgfusepath{stroke}%
\end{pgfscope}%
\begin{pgfscope}%
\pgfpathrectangle{\pgfqpoint{0.100000in}{0.212622in}}{\pgfqpoint{3.696000in}{3.696000in}}%
\pgfusepath{clip}%
\pgfsetrectcap%
\pgfsetroundjoin%
\pgfsetlinewidth{1.505625pt}%
\definecolor{currentstroke}{rgb}{1.000000,0.000000,0.000000}%
\pgfsetstrokecolor{currentstroke}%
\pgfsetdash{}{0pt}%
\pgfpathmoveto{\pgfqpoint{0.933205in}{1.318945in}}%
\pgfpathlineto{\pgfqpoint{0.952296in}{1.300547in}}%
\pgfusepath{stroke}%
\end{pgfscope}%
\begin{pgfscope}%
\pgfpathrectangle{\pgfqpoint{0.100000in}{0.212622in}}{\pgfqpoint{3.696000in}{3.696000in}}%
\pgfusepath{clip}%
\pgfsetrectcap%
\pgfsetroundjoin%
\pgfsetlinewidth{1.505625pt}%
\definecolor{currentstroke}{rgb}{1.000000,0.000000,0.000000}%
\pgfsetstrokecolor{currentstroke}%
\pgfsetdash{}{0pt}%
\pgfpathmoveto{\pgfqpoint{0.933201in}{1.318947in}}%
\pgfpathlineto{\pgfqpoint{0.952296in}{1.300547in}}%
\pgfusepath{stroke}%
\end{pgfscope}%
\begin{pgfscope}%
\pgfpathrectangle{\pgfqpoint{0.100000in}{0.212622in}}{\pgfqpoint{3.696000in}{3.696000in}}%
\pgfusepath{clip}%
\pgfsetrectcap%
\pgfsetroundjoin%
\pgfsetlinewidth{1.505625pt}%
\definecolor{currentstroke}{rgb}{1.000000,0.000000,0.000000}%
\pgfsetstrokecolor{currentstroke}%
\pgfsetdash{}{0pt}%
\pgfpathmoveto{\pgfqpoint{0.933199in}{1.318948in}}%
\pgfpathlineto{\pgfqpoint{0.952296in}{1.300547in}}%
\pgfusepath{stroke}%
\end{pgfscope}%
\begin{pgfscope}%
\pgfpathrectangle{\pgfqpoint{0.100000in}{0.212622in}}{\pgfqpoint{3.696000in}{3.696000in}}%
\pgfusepath{clip}%
\pgfsetrectcap%
\pgfsetroundjoin%
\pgfsetlinewidth{1.505625pt}%
\definecolor{currentstroke}{rgb}{1.000000,0.000000,0.000000}%
\pgfsetstrokecolor{currentstroke}%
\pgfsetdash{}{0pt}%
\pgfpathmoveto{\pgfqpoint{0.933197in}{1.318948in}}%
\pgfpathlineto{\pgfqpoint{0.952296in}{1.300547in}}%
\pgfusepath{stroke}%
\end{pgfscope}%
\begin{pgfscope}%
\pgfpathrectangle{\pgfqpoint{0.100000in}{0.212622in}}{\pgfqpoint{3.696000in}{3.696000in}}%
\pgfusepath{clip}%
\pgfsetrectcap%
\pgfsetroundjoin%
\pgfsetlinewidth{1.505625pt}%
\definecolor{currentstroke}{rgb}{1.000000,0.000000,0.000000}%
\pgfsetstrokecolor{currentstroke}%
\pgfsetdash{}{0pt}%
\pgfpathmoveto{\pgfqpoint{0.933197in}{1.318948in}}%
\pgfpathlineto{\pgfqpoint{0.952296in}{1.300547in}}%
\pgfusepath{stroke}%
\end{pgfscope}%
\begin{pgfscope}%
\pgfpathrectangle{\pgfqpoint{0.100000in}{0.212622in}}{\pgfqpoint{3.696000in}{3.696000in}}%
\pgfusepath{clip}%
\pgfsetrectcap%
\pgfsetroundjoin%
\pgfsetlinewidth{1.505625pt}%
\definecolor{currentstroke}{rgb}{1.000000,0.000000,0.000000}%
\pgfsetstrokecolor{currentstroke}%
\pgfsetdash{}{0pt}%
\pgfpathmoveto{\pgfqpoint{0.933196in}{1.318948in}}%
\pgfpathlineto{\pgfqpoint{0.952296in}{1.300547in}}%
\pgfusepath{stroke}%
\end{pgfscope}%
\begin{pgfscope}%
\pgfpathrectangle{\pgfqpoint{0.100000in}{0.212622in}}{\pgfqpoint{3.696000in}{3.696000in}}%
\pgfusepath{clip}%
\pgfsetrectcap%
\pgfsetroundjoin%
\pgfsetlinewidth{1.505625pt}%
\definecolor{currentstroke}{rgb}{1.000000,0.000000,0.000000}%
\pgfsetstrokecolor{currentstroke}%
\pgfsetdash{}{0pt}%
\pgfpathmoveto{\pgfqpoint{0.933196in}{1.318949in}}%
\pgfpathlineto{\pgfqpoint{0.952296in}{1.300547in}}%
\pgfusepath{stroke}%
\end{pgfscope}%
\begin{pgfscope}%
\pgfpathrectangle{\pgfqpoint{0.100000in}{0.212622in}}{\pgfqpoint{3.696000in}{3.696000in}}%
\pgfusepath{clip}%
\pgfsetrectcap%
\pgfsetroundjoin%
\pgfsetlinewidth{1.505625pt}%
\definecolor{currentstroke}{rgb}{1.000000,0.000000,0.000000}%
\pgfsetstrokecolor{currentstroke}%
\pgfsetdash{}{0pt}%
\pgfpathmoveto{\pgfqpoint{0.933196in}{1.318949in}}%
\pgfpathlineto{\pgfqpoint{0.952296in}{1.300547in}}%
\pgfusepath{stroke}%
\end{pgfscope}%
\begin{pgfscope}%
\pgfpathrectangle{\pgfqpoint{0.100000in}{0.212622in}}{\pgfqpoint{3.696000in}{3.696000in}}%
\pgfusepath{clip}%
\pgfsetrectcap%
\pgfsetroundjoin%
\pgfsetlinewidth{1.505625pt}%
\definecolor{currentstroke}{rgb}{1.000000,0.000000,0.000000}%
\pgfsetstrokecolor{currentstroke}%
\pgfsetdash{}{0pt}%
\pgfpathmoveto{\pgfqpoint{0.933196in}{1.318949in}}%
\pgfpathlineto{\pgfqpoint{0.952296in}{1.300547in}}%
\pgfusepath{stroke}%
\end{pgfscope}%
\begin{pgfscope}%
\pgfpathrectangle{\pgfqpoint{0.100000in}{0.212622in}}{\pgfqpoint{3.696000in}{3.696000in}}%
\pgfusepath{clip}%
\pgfsetrectcap%
\pgfsetroundjoin%
\pgfsetlinewidth{1.505625pt}%
\definecolor{currentstroke}{rgb}{1.000000,0.000000,0.000000}%
\pgfsetstrokecolor{currentstroke}%
\pgfsetdash{}{0pt}%
\pgfpathmoveto{\pgfqpoint{0.933196in}{1.318949in}}%
\pgfpathlineto{\pgfqpoint{0.952296in}{1.300547in}}%
\pgfusepath{stroke}%
\end{pgfscope}%
\begin{pgfscope}%
\pgfpathrectangle{\pgfqpoint{0.100000in}{0.212622in}}{\pgfqpoint{3.696000in}{3.696000in}}%
\pgfusepath{clip}%
\pgfsetrectcap%
\pgfsetroundjoin%
\pgfsetlinewidth{1.505625pt}%
\definecolor{currentstroke}{rgb}{1.000000,0.000000,0.000000}%
\pgfsetstrokecolor{currentstroke}%
\pgfsetdash{}{0pt}%
\pgfpathmoveto{\pgfqpoint{0.933196in}{1.318949in}}%
\pgfpathlineto{\pgfqpoint{0.952296in}{1.300547in}}%
\pgfusepath{stroke}%
\end{pgfscope}%
\begin{pgfscope}%
\pgfpathrectangle{\pgfqpoint{0.100000in}{0.212622in}}{\pgfqpoint{3.696000in}{3.696000in}}%
\pgfusepath{clip}%
\pgfsetrectcap%
\pgfsetroundjoin%
\pgfsetlinewidth{1.505625pt}%
\definecolor{currentstroke}{rgb}{1.000000,0.000000,0.000000}%
\pgfsetstrokecolor{currentstroke}%
\pgfsetdash{}{0pt}%
\pgfpathmoveto{\pgfqpoint{0.933196in}{1.318949in}}%
\pgfpathlineto{\pgfqpoint{0.952296in}{1.300547in}}%
\pgfusepath{stroke}%
\end{pgfscope}%
\begin{pgfscope}%
\pgfpathrectangle{\pgfqpoint{0.100000in}{0.212622in}}{\pgfqpoint{3.696000in}{3.696000in}}%
\pgfusepath{clip}%
\pgfsetrectcap%
\pgfsetroundjoin%
\pgfsetlinewidth{1.505625pt}%
\definecolor{currentstroke}{rgb}{1.000000,0.000000,0.000000}%
\pgfsetstrokecolor{currentstroke}%
\pgfsetdash{}{0pt}%
\pgfpathmoveto{\pgfqpoint{0.933196in}{1.318949in}}%
\pgfpathlineto{\pgfqpoint{0.952296in}{1.300547in}}%
\pgfusepath{stroke}%
\end{pgfscope}%
\begin{pgfscope}%
\pgfpathrectangle{\pgfqpoint{0.100000in}{0.212622in}}{\pgfqpoint{3.696000in}{3.696000in}}%
\pgfusepath{clip}%
\pgfsetrectcap%
\pgfsetroundjoin%
\pgfsetlinewidth{1.505625pt}%
\definecolor{currentstroke}{rgb}{1.000000,0.000000,0.000000}%
\pgfsetstrokecolor{currentstroke}%
\pgfsetdash{}{0pt}%
\pgfpathmoveto{\pgfqpoint{0.933196in}{1.318949in}}%
\pgfpathlineto{\pgfqpoint{0.952296in}{1.300547in}}%
\pgfusepath{stroke}%
\end{pgfscope}%
\begin{pgfscope}%
\pgfpathrectangle{\pgfqpoint{0.100000in}{0.212622in}}{\pgfqpoint{3.696000in}{3.696000in}}%
\pgfusepath{clip}%
\pgfsetrectcap%
\pgfsetroundjoin%
\pgfsetlinewidth{1.505625pt}%
\definecolor{currentstroke}{rgb}{1.000000,0.000000,0.000000}%
\pgfsetstrokecolor{currentstroke}%
\pgfsetdash{}{0pt}%
\pgfpathmoveto{\pgfqpoint{0.933196in}{1.318949in}}%
\pgfpathlineto{\pgfqpoint{0.952296in}{1.300547in}}%
\pgfusepath{stroke}%
\end{pgfscope}%
\begin{pgfscope}%
\pgfpathrectangle{\pgfqpoint{0.100000in}{0.212622in}}{\pgfqpoint{3.696000in}{3.696000in}}%
\pgfusepath{clip}%
\pgfsetrectcap%
\pgfsetroundjoin%
\pgfsetlinewidth{1.505625pt}%
\definecolor{currentstroke}{rgb}{1.000000,0.000000,0.000000}%
\pgfsetstrokecolor{currentstroke}%
\pgfsetdash{}{0pt}%
\pgfpathmoveto{\pgfqpoint{0.933196in}{1.318949in}}%
\pgfpathlineto{\pgfqpoint{0.952296in}{1.300547in}}%
\pgfusepath{stroke}%
\end{pgfscope}%
\begin{pgfscope}%
\pgfpathrectangle{\pgfqpoint{0.100000in}{0.212622in}}{\pgfqpoint{3.696000in}{3.696000in}}%
\pgfusepath{clip}%
\pgfsetrectcap%
\pgfsetroundjoin%
\pgfsetlinewidth{1.505625pt}%
\definecolor{currentstroke}{rgb}{1.000000,0.000000,0.000000}%
\pgfsetstrokecolor{currentstroke}%
\pgfsetdash{}{0pt}%
\pgfpathmoveto{\pgfqpoint{0.933196in}{1.318949in}}%
\pgfpathlineto{\pgfqpoint{0.952296in}{1.300547in}}%
\pgfusepath{stroke}%
\end{pgfscope}%
\begin{pgfscope}%
\pgfpathrectangle{\pgfqpoint{0.100000in}{0.212622in}}{\pgfqpoint{3.696000in}{3.696000in}}%
\pgfusepath{clip}%
\pgfsetrectcap%
\pgfsetroundjoin%
\pgfsetlinewidth{1.505625pt}%
\definecolor{currentstroke}{rgb}{1.000000,0.000000,0.000000}%
\pgfsetstrokecolor{currentstroke}%
\pgfsetdash{}{0pt}%
\pgfpathmoveto{\pgfqpoint{0.933196in}{1.318949in}}%
\pgfpathlineto{\pgfqpoint{0.952296in}{1.300547in}}%
\pgfusepath{stroke}%
\end{pgfscope}%
\begin{pgfscope}%
\pgfpathrectangle{\pgfqpoint{0.100000in}{0.212622in}}{\pgfqpoint{3.696000in}{3.696000in}}%
\pgfusepath{clip}%
\pgfsetrectcap%
\pgfsetroundjoin%
\pgfsetlinewidth{1.505625pt}%
\definecolor{currentstroke}{rgb}{1.000000,0.000000,0.000000}%
\pgfsetstrokecolor{currentstroke}%
\pgfsetdash{}{0pt}%
\pgfpathmoveto{\pgfqpoint{0.933196in}{1.318949in}}%
\pgfpathlineto{\pgfqpoint{0.952296in}{1.300547in}}%
\pgfusepath{stroke}%
\end{pgfscope}%
\begin{pgfscope}%
\pgfpathrectangle{\pgfqpoint{0.100000in}{0.212622in}}{\pgfqpoint{3.696000in}{3.696000in}}%
\pgfusepath{clip}%
\pgfsetrectcap%
\pgfsetroundjoin%
\pgfsetlinewidth{1.505625pt}%
\definecolor{currentstroke}{rgb}{1.000000,0.000000,0.000000}%
\pgfsetstrokecolor{currentstroke}%
\pgfsetdash{}{0pt}%
\pgfpathmoveto{\pgfqpoint{0.933196in}{1.318949in}}%
\pgfpathlineto{\pgfqpoint{0.952296in}{1.300547in}}%
\pgfusepath{stroke}%
\end{pgfscope}%
\begin{pgfscope}%
\pgfpathrectangle{\pgfqpoint{0.100000in}{0.212622in}}{\pgfqpoint{3.696000in}{3.696000in}}%
\pgfusepath{clip}%
\pgfsetrectcap%
\pgfsetroundjoin%
\pgfsetlinewidth{1.505625pt}%
\definecolor{currentstroke}{rgb}{1.000000,0.000000,0.000000}%
\pgfsetstrokecolor{currentstroke}%
\pgfsetdash{}{0pt}%
\pgfpathmoveto{\pgfqpoint{0.933196in}{1.318949in}}%
\pgfpathlineto{\pgfqpoint{0.952296in}{1.300547in}}%
\pgfusepath{stroke}%
\end{pgfscope}%
\begin{pgfscope}%
\pgfpathrectangle{\pgfqpoint{0.100000in}{0.212622in}}{\pgfqpoint{3.696000in}{3.696000in}}%
\pgfusepath{clip}%
\pgfsetrectcap%
\pgfsetroundjoin%
\pgfsetlinewidth{1.505625pt}%
\definecolor{currentstroke}{rgb}{1.000000,0.000000,0.000000}%
\pgfsetstrokecolor{currentstroke}%
\pgfsetdash{}{0pt}%
\pgfpathmoveto{\pgfqpoint{0.933196in}{1.318949in}}%
\pgfpathlineto{\pgfqpoint{0.952296in}{1.300547in}}%
\pgfusepath{stroke}%
\end{pgfscope}%
\begin{pgfscope}%
\pgfpathrectangle{\pgfqpoint{0.100000in}{0.212622in}}{\pgfqpoint{3.696000in}{3.696000in}}%
\pgfusepath{clip}%
\pgfsetrectcap%
\pgfsetroundjoin%
\pgfsetlinewidth{1.505625pt}%
\definecolor{currentstroke}{rgb}{1.000000,0.000000,0.000000}%
\pgfsetstrokecolor{currentstroke}%
\pgfsetdash{}{0pt}%
\pgfpathmoveto{\pgfqpoint{0.933196in}{1.318949in}}%
\pgfpathlineto{\pgfqpoint{0.952296in}{1.300547in}}%
\pgfusepath{stroke}%
\end{pgfscope}%
\begin{pgfscope}%
\pgfpathrectangle{\pgfqpoint{0.100000in}{0.212622in}}{\pgfqpoint{3.696000in}{3.696000in}}%
\pgfusepath{clip}%
\pgfsetrectcap%
\pgfsetroundjoin%
\pgfsetlinewidth{1.505625pt}%
\definecolor{currentstroke}{rgb}{1.000000,0.000000,0.000000}%
\pgfsetstrokecolor{currentstroke}%
\pgfsetdash{}{0pt}%
\pgfpathmoveto{\pgfqpoint{0.933196in}{1.318949in}}%
\pgfpathlineto{\pgfqpoint{0.952296in}{1.300547in}}%
\pgfusepath{stroke}%
\end{pgfscope}%
\begin{pgfscope}%
\pgfpathrectangle{\pgfqpoint{0.100000in}{0.212622in}}{\pgfqpoint{3.696000in}{3.696000in}}%
\pgfusepath{clip}%
\pgfsetrectcap%
\pgfsetroundjoin%
\pgfsetlinewidth{1.505625pt}%
\definecolor{currentstroke}{rgb}{1.000000,0.000000,0.000000}%
\pgfsetstrokecolor{currentstroke}%
\pgfsetdash{}{0pt}%
\pgfpathmoveto{\pgfqpoint{0.933196in}{1.318949in}}%
\pgfpathlineto{\pgfqpoint{0.952296in}{1.300547in}}%
\pgfusepath{stroke}%
\end{pgfscope}%
\begin{pgfscope}%
\pgfpathrectangle{\pgfqpoint{0.100000in}{0.212622in}}{\pgfqpoint{3.696000in}{3.696000in}}%
\pgfusepath{clip}%
\pgfsetrectcap%
\pgfsetroundjoin%
\pgfsetlinewidth{1.505625pt}%
\definecolor{currentstroke}{rgb}{1.000000,0.000000,0.000000}%
\pgfsetstrokecolor{currentstroke}%
\pgfsetdash{}{0pt}%
\pgfpathmoveto{\pgfqpoint{0.933196in}{1.318949in}}%
\pgfpathlineto{\pgfqpoint{0.952296in}{1.300547in}}%
\pgfusepath{stroke}%
\end{pgfscope}%
\begin{pgfscope}%
\pgfpathrectangle{\pgfqpoint{0.100000in}{0.212622in}}{\pgfqpoint{3.696000in}{3.696000in}}%
\pgfusepath{clip}%
\pgfsetrectcap%
\pgfsetroundjoin%
\pgfsetlinewidth{1.505625pt}%
\definecolor{currentstroke}{rgb}{1.000000,0.000000,0.000000}%
\pgfsetstrokecolor{currentstroke}%
\pgfsetdash{}{0pt}%
\pgfpathmoveto{\pgfqpoint{0.933196in}{1.318949in}}%
\pgfpathlineto{\pgfqpoint{0.952296in}{1.300547in}}%
\pgfusepath{stroke}%
\end{pgfscope}%
\begin{pgfscope}%
\pgfpathrectangle{\pgfqpoint{0.100000in}{0.212622in}}{\pgfqpoint{3.696000in}{3.696000in}}%
\pgfusepath{clip}%
\pgfsetrectcap%
\pgfsetroundjoin%
\pgfsetlinewidth{1.505625pt}%
\definecolor{currentstroke}{rgb}{1.000000,0.000000,0.000000}%
\pgfsetstrokecolor{currentstroke}%
\pgfsetdash{}{0pt}%
\pgfpathmoveto{\pgfqpoint{0.933196in}{1.318949in}}%
\pgfpathlineto{\pgfqpoint{0.952296in}{1.300547in}}%
\pgfusepath{stroke}%
\end{pgfscope}%
\begin{pgfscope}%
\pgfpathrectangle{\pgfqpoint{0.100000in}{0.212622in}}{\pgfqpoint{3.696000in}{3.696000in}}%
\pgfusepath{clip}%
\pgfsetrectcap%
\pgfsetroundjoin%
\pgfsetlinewidth{1.505625pt}%
\definecolor{currentstroke}{rgb}{1.000000,0.000000,0.000000}%
\pgfsetstrokecolor{currentstroke}%
\pgfsetdash{}{0pt}%
\pgfpathmoveto{\pgfqpoint{0.933196in}{1.318949in}}%
\pgfpathlineto{\pgfqpoint{0.952296in}{1.300547in}}%
\pgfusepath{stroke}%
\end{pgfscope}%
\begin{pgfscope}%
\pgfpathrectangle{\pgfqpoint{0.100000in}{0.212622in}}{\pgfqpoint{3.696000in}{3.696000in}}%
\pgfusepath{clip}%
\pgfsetrectcap%
\pgfsetroundjoin%
\pgfsetlinewidth{1.505625pt}%
\definecolor{currentstroke}{rgb}{1.000000,0.000000,0.000000}%
\pgfsetstrokecolor{currentstroke}%
\pgfsetdash{}{0pt}%
\pgfpathmoveto{\pgfqpoint{0.933196in}{1.318949in}}%
\pgfpathlineto{\pgfqpoint{0.952296in}{1.300547in}}%
\pgfusepath{stroke}%
\end{pgfscope}%
\begin{pgfscope}%
\pgfpathrectangle{\pgfqpoint{0.100000in}{0.212622in}}{\pgfqpoint{3.696000in}{3.696000in}}%
\pgfusepath{clip}%
\pgfsetrectcap%
\pgfsetroundjoin%
\pgfsetlinewidth{1.505625pt}%
\definecolor{currentstroke}{rgb}{1.000000,0.000000,0.000000}%
\pgfsetstrokecolor{currentstroke}%
\pgfsetdash{}{0pt}%
\pgfpathmoveto{\pgfqpoint{0.933196in}{1.318949in}}%
\pgfpathlineto{\pgfqpoint{0.952296in}{1.300547in}}%
\pgfusepath{stroke}%
\end{pgfscope}%
\begin{pgfscope}%
\pgfpathrectangle{\pgfqpoint{0.100000in}{0.212622in}}{\pgfqpoint{3.696000in}{3.696000in}}%
\pgfusepath{clip}%
\pgfsetrectcap%
\pgfsetroundjoin%
\pgfsetlinewidth{1.505625pt}%
\definecolor{currentstroke}{rgb}{1.000000,0.000000,0.000000}%
\pgfsetstrokecolor{currentstroke}%
\pgfsetdash{}{0pt}%
\pgfpathmoveto{\pgfqpoint{0.933196in}{1.318949in}}%
\pgfpathlineto{\pgfqpoint{0.952296in}{1.300547in}}%
\pgfusepath{stroke}%
\end{pgfscope}%
\begin{pgfscope}%
\pgfpathrectangle{\pgfqpoint{0.100000in}{0.212622in}}{\pgfqpoint{3.696000in}{3.696000in}}%
\pgfusepath{clip}%
\pgfsetrectcap%
\pgfsetroundjoin%
\pgfsetlinewidth{1.505625pt}%
\definecolor{currentstroke}{rgb}{1.000000,0.000000,0.000000}%
\pgfsetstrokecolor{currentstroke}%
\pgfsetdash{}{0pt}%
\pgfpathmoveto{\pgfqpoint{0.933196in}{1.318949in}}%
\pgfpathlineto{\pgfqpoint{0.952296in}{1.300547in}}%
\pgfusepath{stroke}%
\end{pgfscope}%
\begin{pgfscope}%
\pgfpathrectangle{\pgfqpoint{0.100000in}{0.212622in}}{\pgfqpoint{3.696000in}{3.696000in}}%
\pgfusepath{clip}%
\pgfsetrectcap%
\pgfsetroundjoin%
\pgfsetlinewidth{1.505625pt}%
\definecolor{currentstroke}{rgb}{1.000000,0.000000,0.000000}%
\pgfsetstrokecolor{currentstroke}%
\pgfsetdash{}{0pt}%
\pgfpathmoveto{\pgfqpoint{0.933196in}{1.318949in}}%
\pgfpathlineto{\pgfqpoint{0.952296in}{1.300547in}}%
\pgfusepath{stroke}%
\end{pgfscope}%
\begin{pgfscope}%
\pgfpathrectangle{\pgfqpoint{0.100000in}{0.212622in}}{\pgfqpoint{3.696000in}{3.696000in}}%
\pgfusepath{clip}%
\pgfsetrectcap%
\pgfsetroundjoin%
\pgfsetlinewidth{1.505625pt}%
\definecolor{currentstroke}{rgb}{1.000000,0.000000,0.000000}%
\pgfsetstrokecolor{currentstroke}%
\pgfsetdash{}{0pt}%
\pgfpathmoveto{\pgfqpoint{0.933196in}{1.318949in}}%
\pgfpathlineto{\pgfqpoint{0.952296in}{1.300547in}}%
\pgfusepath{stroke}%
\end{pgfscope}%
\begin{pgfscope}%
\pgfpathrectangle{\pgfqpoint{0.100000in}{0.212622in}}{\pgfqpoint{3.696000in}{3.696000in}}%
\pgfusepath{clip}%
\pgfsetrectcap%
\pgfsetroundjoin%
\pgfsetlinewidth{1.505625pt}%
\definecolor{currentstroke}{rgb}{1.000000,0.000000,0.000000}%
\pgfsetstrokecolor{currentstroke}%
\pgfsetdash{}{0pt}%
\pgfpathmoveto{\pgfqpoint{0.933196in}{1.318949in}}%
\pgfpathlineto{\pgfqpoint{0.952296in}{1.300547in}}%
\pgfusepath{stroke}%
\end{pgfscope}%
\begin{pgfscope}%
\pgfpathrectangle{\pgfqpoint{0.100000in}{0.212622in}}{\pgfqpoint{3.696000in}{3.696000in}}%
\pgfusepath{clip}%
\pgfsetrectcap%
\pgfsetroundjoin%
\pgfsetlinewidth{1.505625pt}%
\definecolor{currentstroke}{rgb}{1.000000,0.000000,0.000000}%
\pgfsetstrokecolor{currentstroke}%
\pgfsetdash{}{0pt}%
\pgfpathmoveto{\pgfqpoint{0.933196in}{1.318949in}}%
\pgfpathlineto{\pgfqpoint{0.952296in}{1.300547in}}%
\pgfusepath{stroke}%
\end{pgfscope}%
\begin{pgfscope}%
\pgfpathrectangle{\pgfqpoint{0.100000in}{0.212622in}}{\pgfqpoint{3.696000in}{3.696000in}}%
\pgfusepath{clip}%
\pgfsetrectcap%
\pgfsetroundjoin%
\pgfsetlinewidth{1.505625pt}%
\definecolor{currentstroke}{rgb}{1.000000,0.000000,0.000000}%
\pgfsetstrokecolor{currentstroke}%
\pgfsetdash{}{0pt}%
\pgfpathmoveto{\pgfqpoint{0.933196in}{1.318949in}}%
\pgfpathlineto{\pgfqpoint{0.952296in}{1.300547in}}%
\pgfusepath{stroke}%
\end{pgfscope}%
\begin{pgfscope}%
\pgfpathrectangle{\pgfqpoint{0.100000in}{0.212622in}}{\pgfqpoint{3.696000in}{3.696000in}}%
\pgfusepath{clip}%
\pgfsetrectcap%
\pgfsetroundjoin%
\pgfsetlinewidth{1.505625pt}%
\definecolor{currentstroke}{rgb}{1.000000,0.000000,0.000000}%
\pgfsetstrokecolor{currentstroke}%
\pgfsetdash{}{0pt}%
\pgfpathmoveto{\pgfqpoint{0.933196in}{1.318949in}}%
\pgfpathlineto{\pgfqpoint{0.952296in}{1.300547in}}%
\pgfusepath{stroke}%
\end{pgfscope}%
\begin{pgfscope}%
\pgfpathrectangle{\pgfqpoint{0.100000in}{0.212622in}}{\pgfqpoint{3.696000in}{3.696000in}}%
\pgfusepath{clip}%
\pgfsetrectcap%
\pgfsetroundjoin%
\pgfsetlinewidth{1.505625pt}%
\definecolor{currentstroke}{rgb}{1.000000,0.000000,0.000000}%
\pgfsetstrokecolor{currentstroke}%
\pgfsetdash{}{0pt}%
\pgfpathmoveto{\pgfqpoint{0.933196in}{1.318949in}}%
\pgfpathlineto{\pgfqpoint{0.952296in}{1.300547in}}%
\pgfusepath{stroke}%
\end{pgfscope}%
\begin{pgfscope}%
\pgfpathrectangle{\pgfqpoint{0.100000in}{0.212622in}}{\pgfqpoint{3.696000in}{3.696000in}}%
\pgfusepath{clip}%
\pgfsetrectcap%
\pgfsetroundjoin%
\pgfsetlinewidth{1.505625pt}%
\definecolor{currentstroke}{rgb}{1.000000,0.000000,0.000000}%
\pgfsetstrokecolor{currentstroke}%
\pgfsetdash{}{0pt}%
\pgfpathmoveto{\pgfqpoint{0.933196in}{1.318949in}}%
\pgfpathlineto{\pgfqpoint{0.952296in}{1.300547in}}%
\pgfusepath{stroke}%
\end{pgfscope}%
\begin{pgfscope}%
\pgfpathrectangle{\pgfqpoint{0.100000in}{0.212622in}}{\pgfqpoint{3.696000in}{3.696000in}}%
\pgfusepath{clip}%
\pgfsetrectcap%
\pgfsetroundjoin%
\pgfsetlinewidth{1.505625pt}%
\definecolor{currentstroke}{rgb}{1.000000,0.000000,0.000000}%
\pgfsetstrokecolor{currentstroke}%
\pgfsetdash{}{0pt}%
\pgfpathmoveto{\pgfqpoint{0.933196in}{1.318949in}}%
\pgfpathlineto{\pgfqpoint{0.952296in}{1.300547in}}%
\pgfusepath{stroke}%
\end{pgfscope}%
\begin{pgfscope}%
\pgfpathrectangle{\pgfqpoint{0.100000in}{0.212622in}}{\pgfqpoint{3.696000in}{3.696000in}}%
\pgfusepath{clip}%
\pgfsetrectcap%
\pgfsetroundjoin%
\pgfsetlinewidth{1.505625pt}%
\definecolor{currentstroke}{rgb}{1.000000,0.000000,0.000000}%
\pgfsetstrokecolor{currentstroke}%
\pgfsetdash{}{0pt}%
\pgfpathmoveto{\pgfqpoint{0.933196in}{1.318949in}}%
\pgfpathlineto{\pgfqpoint{0.952296in}{1.300547in}}%
\pgfusepath{stroke}%
\end{pgfscope}%
\begin{pgfscope}%
\pgfpathrectangle{\pgfqpoint{0.100000in}{0.212622in}}{\pgfqpoint{3.696000in}{3.696000in}}%
\pgfusepath{clip}%
\pgfsetrectcap%
\pgfsetroundjoin%
\pgfsetlinewidth{1.505625pt}%
\definecolor{currentstroke}{rgb}{1.000000,0.000000,0.000000}%
\pgfsetstrokecolor{currentstroke}%
\pgfsetdash{}{0pt}%
\pgfpathmoveto{\pgfqpoint{0.933196in}{1.318949in}}%
\pgfpathlineto{\pgfqpoint{0.952296in}{1.300547in}}%
\pgfusepath{stroke}%
\end{pgfscope}%
\begin{pgfscope}%
\pgfpathrectangle{\pgfqpoint{0.100000in}{0.212622in}}{\pgfqpoint{3.696000in}{3.696000in}}%
\pgfusepath{clip}%
\pgfsetrectcap%
\pgfsetroundjoin%
\pgfsetlinewidth{1.505625pt}%
\definecolor{currentstroke}{rgb}{1.000000,0.000000,0.000000}%
\pgfsetstrokecolor{currentstroke}%
\pgfsetdash{}{0pt}%
\pgfpathmoveto{\pgfqpoint{0.933196in}{1.318949in}}%
\pgfpathlineto{\pgfqpoint{0.952296in}{1.300547in}}%
\pgfusepath{stroke}%
\end{pgfscope}%
\begin{pgfscope}%
\pgfpathrectangle{\pgfqpoint{0.100000in}{0.212622in}}{\pgfqpoint{3.696000in}{3.696000in}}%
\pgfusepath{clip}%
\pgfsetrectcap%
\pgfsetroundjoin%
\pgfsetlinewidth{1.505625pt}%
\definecolor{currentstroke}{rgb}{1.000000,0.000000,0.000000}%
\pgfsetstrokecolor{currentstroke}%
\pgfsetdash{}{0pt}%
\pgfpathmoveto{\pgfqpoint{0.933196in}{1.318949in}}%
\pgfpathlineto{\pgfqpoint{0.952296in}{1.300547in}}%
\pgfusepath{stroke}%
\end{pgfscope}%
\begin{pgfscope}%
\pgfpathrectangle{\pgfqpoint{0.100000in}{0.212622in}}{\pgfqpoint{3.696000in}{3.696000in}}%
\pgfusepath{clip}%
\pgfsetrectcap%
\pgfsetroundjoin%
\pgfsetlinewidth{1.505625pt}%
\definecolor{currentstroke}{rgb}{1.000000,0.000000,0.000000}%
\pgfsetstrokecolor{currentstroke}%
\pgfsetdash{}{0pt}%
\pgfpathmoveto{\pgfqpoint{0.933196in}{1.318949in}}%
\pgfpathlineto{\pgfqpoint{0.952296in}{1.300547in}}%
\pgfusepath{stroke}%
\end{pgfscope}%
\begin{pgfscope}%
\pgfpathrectangle{\pgfqpoint{0.100000in}{0.212622in}}{\pgfqpoint{3.696000in}{3.696000in}}%
\pgfusepath{clip}%
\pgfsetrectcap%
\pgfsetroundjoin%
\pgfsetlinewidth{1.505625pt}%
\definecolor{currentstroke}{rgb}{1.000000,0.000000,0.000000}%
\pgfsetstrokecolor{currentstroke}%
\pgfsetdash{}{0pt}%
\pgfpathmoveto{\pgfqpoint{0.933196in}{1.318949in}}%
\pgfpathlineto{\pgfqpoint{0.952296in}{1.300547in}}%
\pgfusepath{stroke}%
\end{pgfscope}%
\begin{pgfscope}%
\pgfpathrectangle{\pgfqpoint{0.100000in}{0.212622in}}{\pgfqpoint{3.696000in}{3.696000in}}%
\pgfusepath{clip}%
\pgfsetrectcap%
\pgfsetroundjoin%
\pgfsetlinewidth{1.505625pt}%
\definecolor{currentstroke}{rgb}{1.000000,0.000000,0.000000}%
\pgfsetstrokecolor{currentstroke}%
\pgfsetdash{}{0pt}%
\pgfpathmoveto{\pgfqpoint{0.933196in}{1.318949in}}%
\pgfpathlineto{\pgfqpoint{0.952296in}{1.300547in}}%
\pgfusepath{stroke}%
\end{pgfscope}%
\begin{pgfscope}%
\pgfpathrectangle{\pgfqpoint{0.100000in}{0.212622in}}{\pgfqpoint{3.696000in}{3.696000in}}%
\pgfusepath{clip}%
\pgfsetrectcap%
\pgfsetroundjoin%
\pgfsetlinewidth{1.505625pt}%
\definecolor{currentstroke}{rgb}{1.000000,0.000000,0.000000}%
\pgfsetstrokecolor{currentstroke}%
\pgfsetdash{}{0pt}%
\pgfpathmoveto{\pgfqpoint{0.933196in}{1.318949in}}%
\pgfpathlineto{\pgfqpoint{0.952296in}{1.300547in}}%
\pgfusepath{stroke}%
\end{pgfscope}%
\begin{pgfscope}%
\pgfpathrectangle{\pgfqpoint{0.100000in}{0.212622in}}{\pgfqpoint{3.696000in}{3.696000in}}%
\pgfusepath{clip}%
\pgfsetrectcap%
\pgfsetroundjoin%
\pgfsetlinewidth{1.505625pt}%
\definecolor{currentstroke}{rgb}{1.000000,0.000000,0.000000}%
\pgfsetstrokecolor{currentstroke}%
\pgfsetdash{}{0pt}%
\pgfpathmoveto{\pgfqpoint{0.933196in}{1.318949in}}%
\pgfpathlineto{\pgfqpoint{0.952296in}{1.300547in}}%
\pgfusepath{stroke}%
\end{pgfscope}%
\begin{pgfscope}%
\pgfpathrectangle{\pgfqpoint{0.100000in}{0.212622in}}{\pgfqpoint{3.696000in}{3.696000in}}%
\pgfusepath{clip}%
\pgfsetrectcap%
\pgfsetroundjoin%
\pgfsetlinewidth{1.505625pt}%
\definecolor{currentstroke}{rgb}{1.000000,0.000000,0.000000}%
\pgfsetstrokecolor{currentstroke}%
\pgfsetdash{}{0pt}%
\pgfpathmoveto{\pgfqpoint{0.933196in}{1.318949in}}%
\pgfpathlineto{\pgfqpoint{0.952296in}{1.300547in}}%
\pgfusepath{stroke}%
\end{pgfscope}%
\begin{pgfscope}%
\pgfpathrectangle{\pgfqpoint{0.100000in}{0.212622in}}{\pgfqpoint{3.696000in}{3.696000in}}%
\pgfusepath{clip}%
\pgfsetrectcap%
\pgfsetroundjoin%
\pgfsetlinewidth{1.505625pt}%
\definecolor{currentstroke}{rgb}{1.000000,0.000000,0.000000}%
\pgfsetstrokecolor{currentstroke}%
\pgfsetdash{}{0pt}%
\pgfpathmoveto{\pgfqpoint{0.933196in}{1.318949in}}%
\pgfpathlineto{\pgfqpoint{0.952296in}{1.300547in}}%
\pgfusepath{stroke}%
\end{pgfscope}%
\begin{pgfscope}%
\pgfpathrectangle{\pgfqpoint{0.100000in}{0.212622in}}{\pgfqpoint{3.696000in}{3.696000in}}%
\pgfusepath{clip}%
\pgfsetrectcap%
\pgfsetroundjoin%
\pgfsetlinewidth{1.505625pt}%
\definecolor{currentstroke}{rgb}{1.000000,0.000000,0.000000}%
\pgfsetstrokecolor{currentstroke}%
\pgfsetdash{}{0pt}%
\pgfpathmoveto{\pgfqpoint{0.933196in}{1.318949in}}%
\pgfpathlineto{\pgfqpoint{0.952296in}{1.300547in}}%
\pgfusepath{stroke}%
\end{pgfscope}%
\begin{pgfscope}%
\pgfpathrectangle{\pgfqpoint{0.100000in}{0.212622in}}{\pgfqpoint{3.696000in}{3.696000in}}%
\pgfusepath{clip}%
\pgfsetrectcap%
\pgfsetroundjoin%
\pgfsetlinewidth{1.505625pt}%
\definecolor{currentstroke}{rgb}{1.000000,0.000000,0.000000}%
\pgfsetstrokecolor{currentstroke}%
\pgfsetdash{}{0pt}%
\pgfpathmoveto{\pgfqpoint{0.933196in}{1.318949in}}%
\pgfpathlineto{\pgfqpoint{0.952296in}{1.300547in}}%
\pgfusepath{stroke}%
\end{pgfscope}%
\begin{pgfscope}%
\pgfpathrectangle{\pgfqpoint{0.100000in}{0.212622in}}{\pgfqpoint{3.696000in}{3.696000in}}%
\pgfusepath{clip}%
\pgfsetrectcap%
\pgfsetroundjoin%
\pgfsetlinewidth{1.505625pt}%
\definecolor{currentstroke}{rgb}{1.000000,0.000000,0.000000}%
\pgfsetstrokecolor{currentstroke}%
\pgfsetdash{}{0pt}%
\pgfpathmoveto{\pgfqpoint{0.933196in}{1.318949in}}%
\pgfpathlineto{\pgfqpoint{0.952296in}{1.300547in}}%
\pgfusepath{stroke}%
\end{pgfscope}%
\begin{pgfscope}%
\pgfpathrectangle{\pgfqpoint{0.100000in}{0.212622in}}{\pgfqpoint{3.696000in}{3.696000in}}%
\pgfusepath{clip}%
\pgfsetrectcap%
\pgfsetroundjoin%
\pgfsetlinewidth{1.505625pt}%
\definecolor{currentstroke}{rgb}{1.000000,0.000000,0.000000}%
\pgfsetstrokecolor{currentstroke}%
\pgfsetdash{}{0pt}%
\pgfpathmoveto{\pgfqpoint{0.933196in}{1.318949in}}%
\pgfpathlineto{\pgfqpoint{0.952296in}{1.300547in}}%
\pgfusepath{stroke}%
\end{pgfscope}%
\begin{pgfscope}%
\pgfpathrectangle{\pgfqpoint{0.100000in}{0.212622in}}{\pgfqpoint{3.696000in}{3.696000in}}%
\pgfusepath{clip}%
\pgfsetrectcap%
\pgfsetroundjoin%
\pgfsetlinewidth{1.505625pt}%
\definecolor{currentstroke}{rgb}{1.000000,0.000000,0.000000}%
\pgfsetstrokecolor{currentstroke}%
\pgfsetdash{}{0pt}%
\pgfpathmoveto{\pgfqpoint{0.933196in}{1.318949in}}%
\pgfpathlineto{\pgfqpoint{0.952296in}{1.300547in}}%
\pgfusepath{stroke}%
\end{pgfscope}%
\begin{pgfscope}%
\pgfpathrectangle{\pgfqpoint{0.100000in}{0.212622in}}{\pgfqpoint{3.696000in}{3.696000in}}%
\pgfusepath{clip}%
\pgfsetrectcap%
\pgfsetroundjoin%
\pgfsetlinewidth{1.505625pt}%
\definecolor{currentstroke}{rgb}{1.000000,0.000000,0.000000}%
\pgfsetstrokecolor{currentstroke}%
\pgfsetdash{}{0pt}%
\pgfpathmoveto{\pgfqpoint{0.933196in}{1.318949in}}%
\pgfpathlineto{\pgfqpoint{0.952296in}{1.300547in}}%
\pgfusepath{stroke}%
\end{pgfscope}%
\begin{pgfscope}%
\pgfpathrectangle{\pgfqpoint{0.100000in}{0.212622in}}{\pgfqpoint{3.696000in}{3.696000in}}%
\pgfusepath{clip}%
\pgfsetrectcap%
\pgfsetroundjoin%
\pgfsetlinewidth{1.505625pt}%
\definecolor{currentstroke}{rgb}{1.000000,0.000000,0.000000}%
\pgfsetstrokecolor{currentstroke}%
\pgfsetdash{}{0pt}%
\pgfpathmoveto{\pgfqpoint{0.933196in}{1.318949in}}%
\pgfpathlineto{\pgfqpoint{0.952296in}{1.300547in}}%
\pgfusepath{stroke}%
\end{pgfscope}%
\begin{pgfscope}%
\pgfpathrectangle{\pgfqpoint{0.100000in}{0.212622in}}{\pgfqpoint{3.696000in}{3.696000in}}%
\pgfusepath{clip}%
\pgfsetrectcap%
\pgfsetroundjoin%
\pgfsetlinewidth{1.505625pt}%
\definecolor{currentstroke}{rgb}{1.000000,0.000000,0.000000}%
\pgfsetstrokecolor{currentstroke}%
\pgfsetdash{}{0pt}%
\pgfpathmoveto{\pgfqpoint{0.933196in}{1.318949in}}%
\pgfpathlineto{\pgfqpoint{0.952296in}{1.300547in}}%
\pgfusepath{stroke}%
\end{pgfscope}%
\begin{pgfscope}%
\pgfpathrectangle{\pgfqpoint{0.100000in}{0.212622in}}{\pgfqpoint{3.696000in}{3.696000in}}%
\pgfusepath{clip}%
\pgfsetrectcap%
\pgfsetroundjoin%
\pgfsetlinewidth{1.505625pt}%
\definecolor{currentstroke}{rgb}{1.000000,0.000000,0.000000}%
\pgfsetstrokecolor{currentstroke}%
\pgfsetdash{}{0pt}%
\pgfpathmoveto{\pgfqpoint{0.933196in}{1.318949in}}%
\pgfpathlineto{\pgfqpoint{0.952296in}{1.300547in}}%
\pgfusepath{stroke}%
\end{pgfscope}%
\begin{pgfscope}%
\pgfpathrectangle{\pgfqpoint{0.100000in}{0.212622in}}{\pgfqpoint{3.696000in}{3.696000in}}%
\pgfusepath{clip}%
\pgfsetrectcap%
\pgfsetroundjoin%
\pgfsetlinewidth{1.505625pt}%
\definecolor{currentstroke}{rgb}{1.000000,0.000000,0.000000}%
\pgfsetstrokecolor{currentstroke}%
\pgfsetdash{}{0pt}%
\pgfpathmoveto{\pgfqpoint{0.933196in}{1.318949in}}%
\pgfpathlineto{\pgfqpoint{0.952296in}{1.300547in}}%
\pgfusepath{stroke}%
\end{pgfscope}%
\begin{pgfscope}%
\pgfpathrectangle{\pgfqpoint{0.100000in}{0.212622in}}{\pgfqpoint{3.696000in}{3.696000in}}%
\pgfusepath{clip}%
\pgfsetrectcap%
\pgfsetroundjoin%
\pgfsetlinewidth{1.505625pt}%
\definecolor{currentstroke}{rgb}{1.000000,0.000000,0.000000}%
\pgfsetstrokecolor{currentstroke}%
\pgfsetdash{}{0pt}%
\pgfpathmoveto{\pgfqpoint{0.933196in}{1.318949in}}%
\pgfpathlineto{\pgfqpoint{0.952296in}{1.300547in}}%
\pgfusepath{stroke}%
\end{pgfscope}%
\begin{pgfscope}%
\pgfpathrectangle{\pgfqpoint{0.100000in}{0.212622in}}{\pgfqpoint{3.696000in}{3.696000in}}%
\pgfusepath{clip}%
\pgfsetrectcap%
\pgfsetroundjoin%
\pgfsetlinewidth{1.505625pt}%
\definecolor{currentstroke}{rgb}{1.000000,0.000000,0.000000}%
\pgfsetstrokecolor{currentstroke}%
\pgfsetdash{}{0pt}%
\pgfpathmoveto{\pgfqpoint{0.933196in}{1.318949in}}%
\pgfpathlineto{\pgfqpoint{0.952296in}{1.300547in}}%
\pgfusepath{stroke}%
\end{pgfscope}%
\begin{pgfscope}%
\pgfpathrectangle{\pgfqpoint{0.100000in}{0.212622in}}{\pgfqpoint{3.696000in}{3.696000in}}%
\pgfusepath{clip}%
\pgfsetrectcap%
\pgfsetroundjoin%
\pgfsetlinewidth{1.505625pt}%
\definecolor{currentstroke}{rgb}{1.000000,0.000000,0.000000}%
\pgfsetstrokecolor{currentstroke}%
\pgfsetdash{}{0pt}%
\pgfpathmoveto{\pgfqpoint{0.933196in}{1.318949in}}%
\pgfpathlineto{\pgfqpoint{0.952296in}{1.300547in}}%
\pgfusepath{stroke}%
\end{pgfscope}%
\begin{pgfscope}%
\pgfpathrectangle{\pgfqpoint{0.100000in}{0.212622in}}{\pgfqpoint{3.696000in}{3.696000in}}%
\pgfusepath{clip}%
\pgfsetrectcap%
\pgfsetroundjoin%
\pgfsetlinewidth{1.505625pt}%
\definecolor{currentstroke}{rgb}{1.000000,0.000000,0.000000}%
\pgfsetstrokecolor{currentstroke}%
\pgfsetdash{}{0pt}%
\pgfpathmoveto{\pgfqpoint{0.933196in}{1.318949in}}%
\pgfpathlineto{\pgfqpoint{0.952296in}{1.300547in}}%
\pgfusepath{stroke}%
\end{pgfscope}%
\begin{pgfscope}%
\pgfpathrectangle{\pgfqpoint{0.100000in}{0.212622in}}{\pgfqpoint{3.696000in}{3.696000in}}%
\pgfusepath{clip}%
\pgfsetrectcap%
\pgfsetroundjoin%
\pgfsetlinewidth{1.505625pt}%
\definecolor{currentstroke}{rgb}{1.000000,0.000000,0.000000}%
\pgfsetstrokecolor{currentstroke}%
\pgfsetdash{}{0pt}%
\pgfpathmoveto{\pgfqpoint{0.933196in}{1.318949in}}%
\pgfpathlineto{\pgfqpoint{0.952296in}{1.300547in}}%
\pgfusepath{stroke}%
\end{pgfscope}%
\begin{pgfscope}%
\pgfpathrectangle{\pgfqpoint{0.100000in}{0.212622in}}{\pgfqpoint{3.696000in}{3.696000in}}%
\pgfusepath{clip}%
\pgfsetrectcap%
\pgfsetroundjoin%
\pgfsetlinewidth{1.505625pt}%
\definecolor{currentstroke}{rgb}{1.000000,0.000000,0.000000}%
\pgfsetstrokecolor{currentstroke}%
\pgfsetdash{}{0pt}%
\pgfpathmoveto{\pgfqpoint{0.933196in}{1.318949in}}%
\pgfpathlineto{\pgfqpoint{0.952296in}{1.300547in}}%
\pgfusepath{stroke}%
\end{pgfscope}%
\begin{pgfscope}%
\pgfpathrectangle{\pgfqpoint{0.100000in}{0.212622in}}{\pgfqpoint{3.696000in}{3.696000in}}%
\pgfusepath{clip}%
\pgfsetrectcap%
\pgfsetroundjoin%
\pgfsetlinewidth{1.505625pt}%
\definecolor{currentstroke}{rgb}{1.000000,0.000000,0.000000}%
\pgfsetstrokecolor{currentstroke}%
\pgfsetdash{}{0pt}%
\pgfpathmoveto{\pgfqpoint{0.933196in}{1.318949in}}%
\pgfpathlineto{\pgfqpoint{0.952296in}{1.300547in}}%
\pgfusepath{stroke}%
\end{pgfscope}%
\begin{pgfscope}%
\pgfpathrectangle{\pgfqpoint{0.100000in}{0.212622in}}{\pgfqpoint{3.696000in}{3.696000in}}%
\pgfusepath{clip}%
\pgfsetrectcap%
\pgfsetroundjoin%
\pgfsetlinewidth{1.505625pt}%
\definecolor{currentstroke}{rgb}{1.000000,0.000000,0.000000}%
\pgfsetstrokecolor{currentstroke}%
\pgfsetdash{}{0pt}%
\pgfpathmoveto{\pgfqpoint{0.933196in}{1.318949in}}%
\pgfpathlineto{\pgfqpoint{0.952296in}{1.300547in}}%
\pgfusepath{stroke}%
\end{pgfscope}%
\begin{pgfscope}%
\pgfpathrectangle{\pgfqpoint{0.100000in}{0.212622in}}{\pgfqpoint{3.696000in}{3.696000in}}%
\pgfusepath{clip}%
\pgfsetrectcap%
\pgfsetroundjoin%
\pgfsetlinewidth{1.505625pt}%
\definecolor{currentstroke}{rgb}{1.000000,0.000000,0.000000}%
\pgfsetstrokecolor{currentstroke}%
\pgfsetdash{}{0pt}%
\pgfpathmoveto{\pgfqpoint{0.933196in}{1.318949in}}%
\pgfpathlineto{\pgfqpoint{0.952296in}{1.300547in}}%
\pgfusepath{stroke}%
\end{pgfscope}%
\begin{pgfscope}%
\pgfpathrectangle{\pgfqpoint{0.100000in}{0.212622in}}{\pgfqpoint{3.696000in}{3.696000in}}%
\pgfusepath{clip}%
\pgfsetrectcap%
\pgfsetroundjoin%
\pgfsetlinewidth{1.505625pt}%
\definecolor{currentstroke}{rgb}{1.000000,0.000000,0.000000}%
\pgfsetstrokecolor{currentstroke}%
\pgfsetdash{}{0pt}%
\pgfpathmoveto{\pgfqpoint{0.933196in}{1.318949in}}%
\pgfpathlineto{\pgfqpoint{0.952296in}{1.300547in}}%
\pgfusepath{stroke}%
\end{pgfscope}%
\begin{pgfscope}%
\pgfpathrectangle{\pgfqpoint{0.100000in}{0.212622in}}{\pgfqpoint{3.696000in}{3.696000in}}%
\pgfusepath{clip}%
\pgfsetrectcap%
\pgfsetroundjoin%
\pgfsetlinewidth{1.505625pt}%
\definecolor{currentstroke}{rgb}{1.000000,0.000000,0.000000}%
\pgfsetstrokecolor{currentstroke}%
\pgfsetdash{}{0pt}%
\pgfpathmoveto{\pgfqpoint{0.933196in}{1.318949in}}%
\pgfpathlineto{\pgfqpoint{0.952296in}{1.300547in}}%
\pgfusepath{stroke}%
\end{pgfscope}%
\begin{pgfscope}%
\pgfpathrectangle{\pgfqpoint{0.100000in}{0.212622in}}{\pgfqpoint{3.696000in}{3.696000in}}%
\pgfusepath{clip}%
\pgfsetrectcap%
\pgfsetroundjoin%
\pgfsetlinewidth{1.505625pt}%
\definecolor{currentstroke}{rgb}{1.000000,0.000000,0.000000}%
\pgfsetstrokecolor{currentstroke}%
\pgfsetdash{}{0pt}%
\pgfpathmoveto{\pgfqpoint{0.933196in}{1.318949in}}%
\pgfpathlineto{\pgfqpoint{0.952296in}{1.300547in}}%
\pgfusepath{stroke}%
\end{pgfscope}%
\begin{pgfscope}%
\pgfpathrectangle{\pgfqpoint{0.100000in}{0.212622in}}{\pgfqpoint{3.696000in}{3.696000in}}%
\pgfusepath{clip}%
\pgfsetrectcap%
\pgfsetroundjoin%
\pgfsetlinewidth{1.505625pt}%
\definecolor{currentstroke}{rgb}{1.000000,0.000000,0.000000}%
\pgfsetstrokecolor{currentstroke}%
\pgfsetdash{}{0pt}%
\pgfpathmoveto{\pgfqpoint{0.933196in}{1.318949in}}%
\pgfpathlineto{\pgfqpoint{0.952296in}{1.300547in}}%
\pgfusepath{stroke}%
\end{pgfscope}%
\begin{pgfscope}%
\pgfpathrectangle{\pgfqpoint{0.100000in}{0.212622in}}{\pgfqpoint{3.696000in}{3.696000in}}%
\pgfusepath{clip}%
\pgfsetrectcap%
\pgfsetroundjoin%
\pgfsetlinewidth{1.505625pt}%
\definecolor{currentstroke}{rgb}{1.000000,0.000000,0.000000}%
\pgfsetstrokecolor{currentstroke}%
\pgfsetdash{}{0pt}%
\pgfpathmoveto{\pgfqpoint{0.933196in}{1.318949in}}%
\pgfpathlineto{\pgfqpoint{0.952296in}{1.300547in}}%
\pgfusepath{stroke}%
\end{pgfscope}%
\begin{pgfscope}%
\pgfpathrectangle{\pgfqpoint{0.100000in}{0.212622in}}{\pgfqpoint{3.696000in}{3.696000in}}%
\pgfusepath{clip}%
\pgfsetrectcap%
\pgfsetroundjoin%
\pgfsetlinewidth{1.505625pt}%
\definecolor{currentstroke}{rgb}{1.000000,0.000000,0.000000}%
\pgfsetstrokecolor{currentstroke}%
\pgfsetdash{}{0pt}%
\pgfpathmoveto{\pgfqpoint{0.933196in}{1.318949in}}%
\pgfpathlineto{\pgfqpoint{0.952296in}{1.300547in}}%
\pgfusepath{stroke}%
\end{pgfscope}%
\begin{pgfscope}%
\pgfpathrectangle{\pgfqpoint{0.100000in}{0.212622in}}{\pgfqpoint{3.696000in}{3.696000in}}%
\pgfusepath{clip}%
\pgfsetrectcap%
\pgfsetroundjoin%
\pgfsetlinewidth{1.505625pt}%
\definecolor{currentstroke}{rgb}{1.000000,0.000000,0.000000}%
\pgfsetstrokecolor{currentstroke}%
\pgfsetdash{}{0pt}%
\pgfpathmoveto{\pgfqpoint{0.933196in}{1.318949in}}%
\pgfpathlineto{\pgfqpoint{0.952296in}{1.300547in}}%
\pgfusepath{stroke}%
\end{pgfscope}%
\begin{pgfscope}%
\pgfpathrectangle{\pgfqpoint{0.100000in}{0.212622in}}{\pgfqpoint{3.696000in}{3.696000in}}%
\pgfusepath{clip}%
\pgfsetrectcap%
\pgfsetroundjoin%
\pgfsetlinewidth{1.505625pt}%
\definecolor{currentstroke}{rgb}{1.000000,0.000000,0.000000}%
\pgfsetstrokecolor{currentstroke}%
\pgfsetdash{}{0pt}%
\pgfpathmoveto{\pgfqpoint{0.933196in}{1.318949in}}%
\pgfpathlineto{\pgfqpoint{0.952296in}{1.300547in}}%
\pgfusepath{stroke}%
\end{pgfscope}%
\begin{pgfscope}%
\pgfpathrectangle{\pgfqpoint{0.100000in}{0.212622in}}{\pgfqpoint{3.696000in}{3.696000in}}%
\pgfusepath{clip}%
\pgfsetrectcap%
\pgfsetroundjoin%
\pgfsetlinewidth{1.505625pt}%
\definecolor{currentstroke}{rgb}{1.000000,0.000000,0.000000}%
\pgfsetstrokecolor{currentstroke}%
\pgfsetdash{}{0pt}%
\pgfpathmoveto{\pgfqpoint{0.933196in}{1.318949in}}%
\pgfpathlineto{\pgfqpoint{0.952296in}{1.300547in}}%
\pgfusepath{stroke}%
\end{pgfscope}%
\begin{pgfscope}%
\pgfpathrectangle{\pgfqpoint{0.100000in}{0.212622in}}{\pgfqpoint{3.696000in}{3.696000in}}%
\pgfusepath{clip}%
\pgfsetrectcap%
\pgfsetroundjoin%
\pgfsetlinewidth{1.505625pt}%
\definecolor{currentstroke}{rgb}{1.000000,0.000000,0.000000}%
\pgfsetstrokecolor{currentstroke}%
\pgfsetdash{}{0pt}%
\pgfpathmoveto{\pgfqpoint{0.933196in}{1.318949in}}%
\pgfpathlineto{\pgfqpoint{0.952296in}{1.300547in}}%
\pgfusepath{stroke}%
\end{pgfscope}%
\begin{pgfscope}%
\pgfpathrectangle{\pgfqpoint{0.100000in}{0.212622in}}{\pgfqpoint{3.696000in}{3.696000in}}%
\pgfusepath{clip}%
\pgfsetrectcap%
\pgfsetroundjoin%
\pgfsetlinewidth{1.505625pt}%
\definecolor{currentstroke}{rgb}{1.000000,0.000000,0.000000}%
\pgfsetstrokecolor{currentstroke}%
\pgfsetdash{}{0pt}%
\pgfpathmoveto{\pgfqpoint{0.933196in}{1.318949in}}%
\pgfpathlineto{\pgfqpoint{0.952296in}{1.300547in}}%
\pgfusepath{stroke}%
\end{pgfscope}%
\begin{pgfscope}%
\pgfpathrectangle{\pgfqpoint{0.100000in}{0.212622in}}{\pgfqpoint{3.696000in}{3.696000in}}%
\pgfusepath{clip}%
\pgfsetrectcap%
\pgfsetroundjoin%
\pgfsetlinewidth{1.505625pt}%
\definecolor{currentstroke}{rgb}{1.000000,0.000000,0.000000}%
\pgfsetstrokecolor{currentstroke}%
\pgfsetdash{}{0pt}%
\pgfpathmoveto{\pgfqpoint{0.933196in}{1.318949in}}%
\pgfpathlineto{\pgfqpoint{0.952296in}{1.300547in}}%
\pgfusepath{stroke}%
\end{pgfscope}%
\begin{pgfscope}%
\pgfpathrectangle{\pgfqpoint{0.100000in}{0.212622in}}{\pgfqpoint{3.696000in}{3.696000in}}%
\pgfusepath{clip}%
\pgfsetrectcap%
\pgfsetroundjoin%
\pgfsetlinewidth{1.505625pt}%
\definecolor{currentstroke}{rgb}{1.000000,0.000000,0.000000}%
\pgfsetstrokecolor{currentstroke}%
\pgfsetdash{}{0pt}%
\pgfpathmoveto{\pgfqpoint{0.933196in}{1.318949in}}%
\pgfpathlineto{\pgfqpoint{0.952296in}{1.300547in}}%
\pgfusepath{stroke}%
\end{pgfscope}%
\begin{pgfscope}%
\pgfpathrectangle{\pgfqpoint{0.100000in}{0.212622in}}{\pgfqpoint{3.696000in}{3.696000in}}%
\pgfusepath{clip}%
\pgfsetrectcap%
\pgfsetroundjoin%
\pgfsetlinewidth{1.505625pt}%
\definecolor{currentstroke}{rgb}{1.000000,0.000000,0.000000}%
\pgfsetstrokecolor{currentstroke}%
\pgfsetdash{}{0pt}%
\pgfpathmoveto{\pgfqpoint{0.933196in}{1.318949in}}%
\pgfpathlineto{\pgfqpoint{0.952296in}{1.300547in}}%
\pgfusepath{stroke}%
\end{pgfscope}%
\begin{pgfscope}%
\pgfpathrectangle{\pgfqpoint{0.100000in}{0.212622in}}{\pgfqpoint{3.696000in}{3.696000in}}%
\pgfusepath{clip}%
\pgfsetrectcap%
\pgfsetroundjoin%
\pgfsetlinewidth{1.505625pt}%
\definecolor{currentstroke}{rgb}{1.000000,0.000000,0.000000}%
\pgfsetstrokecolor{currentstroke}%
\pgfsetdash{}{0pt}%
\pgfpathmoveto{\pgfqpoint{0.933196in}{1.318949in}}%
\pgfpathlineto{\pgfqpoint{0.952296in}{1.300547in}}%
\pgfusepath{stroke}%
\end{pgfscope}%
\begin{pgfscope}%
\pgfpathrectangle{\pgfqpoint{0.100000in}{0.212622in}}{\pgfqpoint{3.696000in}{3.696000in}}%
\pgfusepath{clip}%
\pgfsetrectcap%
\pgfsetroundjoin%
\pgfsetlinewidth{1.505625pt}%
\definecolor{currentstroke}{rgb}{1.000000,0.000000,0.000000}%
\pgfsetstrokecolor{currentstroke}%
\pgfsetdash{}{0pt}%
\pgfpathmoveto{\pgfqpoint{0.933196in}{1.318949in}}%
\pgfpathlineto{\pgfqpoint{0.952296in}{1.300547in}}%
\pgfusepath{stroke}%
\end{pgfscope}%
\begin{pgfscope}%
\pgfpathrectangle{\pgfqpoint{0.100000in}{0.212622in}}{\pgfqpoint{3.696000in}{3.696000in}}%
\pgfusepath{clip}%
\pgfsetrectcap%
\pgfsetroundjoin%
\pgfsetlinewidth{1.505625pt}%
\definecolor{currentstroke}{rgb}{1.000000,0.000000,0.000000}%
\pgfsetstrokecolor{currentstroke}%
\pgfsetdash{}{0pt}%
\pgfpathmoveto{\pgfqpoint{0.933196in}{1.318949in}}%
\pgfpathlineto{\pgfqpoint{0.952296in}{1.300547in}}%
\pgfusepath{stroke}%
\end{pgfscope}%
\begin{pgfscope}%
\pgfpathrectangle{\pgfqpoint{0.100000in}{0.212622in}}{\pgfqpoint{3.696000in}{3.696000in}}%
\pgfusepath{clip}%
\pgfsetrectcap%
\pgfsetroundjoin%
\pgfsetlinewidth{1.505625pt}%
\definecolor{currentstroke}{rgb}{1.000000,0.000000,0.000000}%
\pgfsetstrokecolor{currentstroke}%
\pgfsetdash{}{0pt}%
\pgfpathmoveto{\pgfqpoint{0.933196in}{1.318949in}}%
\pgfpathlineto{\pgfqpoint{0.952296in}{1.300547in}}%
\pgfusepath{stroke}%
\end{pgfscope}%
\begin{pgfscope}%
\pgfpathrectangle{\pgfqpoint{0.100000in}{0.212622in}}{\pgfqpoint{3.696000in}{3.696000in}}%
\pgfusepath{clip}%
\pgfsetrectcap%
\pgfsetroundjoin%
\pgfsetlinewidth{1.505625pt}%
\definecolor{currentstroke}{rgb}{1.000000,0.000000,0.000000}%
\pgfsetstrokecolor{currentstroke}%
\pgfsetdash{}{0pt}%
\pgfpathmoveto{\pgfqpoint{0.933196in}{1.318949in}}%
\pgfpathlineto{\pgfqpoint{0.952296in}{1.300547in}}%
\pgfusepath{stroke}%
\end{pgfscope}%
\begin{pgfscope}%
\pgfpathrectangle{\pgfqpoint{0.100000in}{0.212622in}}{\pgfqpoint{3.696000in}{3.696000in}}%
\pgfusepath{clip}%
\pgfsetrectcap%
\pgfsetroundjoin%
\pgfsetlinewidth{1.505625pt}%
\definecolor{currentstroke}{rgb}{1.000000,0.000000,0.000000}%
\pgfsetstrokecolor{currentstroke}%
\pgfsetdash{}{0pt}%
\pgfpathmoveto{\pgfqpoint{0.933196in}{1.318949in}}%
\pgfpathlineto{\pgfqpoint{0.952296in}{1.300547in}}%
\pgfusepath{stroke}%
\end{pgfscope}%
\begin{pgfscope}%
\pgfpathrectangle{\pgfqpoint{0.100000in}{0.212622in}}{\pgfqpoint{3.696000in}{3.696000in}}%
\pgfusepath{clip}%
\pgfsetrectcap%
\pgfsetroundjoin%
\pgfsetlinewidth{1.505625pt}%
\definecolor{currentstroke}{rgb}{1.000000,0.000000,0.000000}%
\pgfsetstrokecolor{currentstroke}%
\pgfsetdash{}{0pt}%
\pgfpathmoveto{\pgfqpoint{0.932684in}{1.319053in}}%
\pgfpathlineto{\pgfqpoint{0.952296in}{1.300547in}}%
\pgfusepath{stroke}%
\end{pgfscope}%
\begin{pgfscope}%
\pgfpathrectangle{\pgfqpoint{0.100000in}{0.212622in}}{\pgfqpoint{3.696000in}{3.696000in}}%
\pgfusepath{clip}%
\pgfsetrectcap%
\pgfsetroundjoin%
\pgfsetlinewidth{1.505625pt}%
\definecolor{currentstroke}{rgb}{1.000000,0.000000,0.000000}%
\pgfsetstrokecolor{currentstroke}%
\pgfsetdash{}{0pt}%
\pgfpathmoveto{\pgfqpoint{0.931049in}{1.319389in}}%
\pgfpathlineto{\pgfqpoint{0.952296in}{1.300547in}}%
\pgfusepath{stroke}%
\end{pgfscope}%
\begin{pgfscope}%
\pgfpathrectangle{\pgfqpoint{0.100000in}{0.212622in}}{\pgfqpoint{3.696000in}{3.696000in}}%
\pgfusepath{clip}%
\pgfsetrectcap%
\pgfsetroundjoin%
\pgfsetlinewidth{1.505625pt}%
\definecolor{currentstroke}{rgb}{1.000000,0.000000,0.000000}%
\pgfsetstrokecolor{currentstroke}%
\pgfsetdash{}{0pt}%
\pgfpathmoveto{\pgfqpoint{0.930124in}{1.319603in}}%
\pgfpathlineto{\pgfqpoint{0.952296in}{1.300547in}}%
\pgfusepath{stroke}%
\end{pgfscope}%
\begin{pgfscope}%
\pgfpathrectangle{\pgfqpoint{0.100000in}{0.212622in}}{\pgfqpoint{3.696000in}{3.696000in}}%
\pgfusepath{clip}%
\pgfsetrectcap%
\pgfsetroundjoin%
\pgfsetlinewidth{1.505625pt}%
\definecolor{currentstroke}{rgb}{1.000000,0.000000,0.000000}%
\pgfsetstrokecolor{currentstroke}%
\pgfsetdash{}{0pt}%
\pgfpathmoveto{\pgfqpoint{0.928628in}{1.320050in}}%
\pgfpathlineto{\pgfqpoint{0.952296in}{1.300547in}}%
\pgfusepath{stroke}%
\end{pgfscope}%
\begin{pgfscope}%
\pgfpathrectangle{\pgfqpoint{0.100000in}{0.212622in}}{\pgfqpoint{3.696000in}{3.696000in}}%
\pgfusepath{clip}%
\pgfsetrectcap%
\pgfsetroundjoin%
\pgfsetlinewidth{1.505625pt}%
\definecolor{currentstroke}{rgb}{1.000000,0.000000,0.000000}%
\pgfsetstrokecolor{currentstroke}%
\pgfsetdash{}{0pt}%
\pgfpathmoveto{\pgfqpoint{0.925728in}{1.321146in}}%
\pgfpathlineto{\pgfqpoint{0.952296in}{1.300547in}}%
\pgfusepath{stroke}%
\end{pgfscope}%
\begin{pgfscope}%
\pgfpathrectangle{\pgfqpoint{0.100000in}{0.212622in}}{\pgfqpoint{3.696000in}{3.696000in}}%
\pgfusepath{clip}%
\pgfsetrectcap%
\pgfsetroundjoin%
\pgfsetlinewidth{1.505625pt}%
\definecolor{currentstroke}{rgb}{1.000000,0.000000,0.000000}%
\pgfsetstrokecolor{currentstroke}%
\pgfsetdash{}{0pt}%
\pgfpathmoveto{\pgfqpoint{0.924114in}{1.321883in}}%
\pgfpathlineto{\pgfqpoint{0.952296in}{1.300547in}}%
\pgfusepath{stroke}%
\end{pgfscope}%
\begin{pgfscope}%
\pgfpathrectangle{\pgfqpoint{0.100000in}{0.212622in}}{\pgfqpoint{3.696000in}{3.696000in}}%
\pgfusepath{clip}%
\pgfsetrectcap%
\pgfsetroundjoin%
\pgfsetlinewidth{1.505625pt}%
\definecolor{currentstroke}{rgb}{1.000000,0.000000,0.000000}%
\pgfsetstrokecolor{currentstroke}%
\pgfsetdash{}{0pt}%
\pgfpathmoveto{\pgfqpoint{0.921766in}{1.323160in}}%
\pgfpathlineto{\pgfqpoint{0.952296in}{1.300547in}}%
\pgfusepath{stroke}%
\end{pgfscope}%
\begin{pgfscope}%
\pgfpathrectangle{\pgfqpoint{0.100000in}{0.212622in}}{\pgfqpoint{3.696000in}{3.696000in}}%
\pgfusepath{clip}%
\pgfsetrectcap%
\pgfsetroundjoin%
\pgfsetlinewidth{1.505625pt}%
\definecolor{currentstroke}{rgb}{1.000000,0.000000,0.000000}%
\pgfsetstrokecolor{currentstroke}%
\pgfsetdash{}{0pt}%
\pgfpathmoveto{\pgfqpoint{0.918828in}{1.325037in}}%
\pgfpathlineto{\pgfqpoint{0.952296in}{1.300547in}}%
\pgfusepath{stroke}%
\end{pgfscope}%
\begin{pgfscope}%
\pgfpathrectangle{\pgfqpoint{0.100000in}{0.212622in}}{\pgfqpoint{3.696000in}{3.696000in}}%
\pgfusepath{clip}%
\pgfsetrectcap%
\pgfsetroundjoin%
\pgfsetlinewidth{1.505625pt}%
\definecolor{currentstroke}{rgb}{1.000000,0.000000,0.000000}%
\pgfsetstrokecolor{currentstroke}%
\pgfsetdash{}{0pt}%
\pgfpathmoveto{\pgfqpoint{0.914788in}{1.327975in}}%
\pgfpathlineto{\pgfqpoint{0.952296in}{1.300547in}}%
\pgfusepath{stroke}%
\end{pgfscope}%
\begin{pgfscope}%
\pgfpathrectangle{\pgfqpoint{0.100000in}{0.212622in}}{\pgfqpoint{3.696000in}{3.696000in}}%
\pgfusepath{clip}%
\pgfsetrectcap%
\pgfsetroundjoin%
\pgfsetlinewidth{1.505625pt}%
\definecolor{currentstroke}{rgb}{1.000000,0.000000,0.000000}%
\pgfsetstrokecolor{currentstroke}%
\pgfsetdash{}{0pt}%
\pgfpathmoveto{\pgfqpoint{0.908969in}{1.332801in}}%
\pgfpathlineto{\pgfqpoint{0.952296in}{1.300547in}}%
\pgfusepath{stroke}%
\end{pgfscope}%
\begin{pgfscope}%
\pgfpathrectangle{\pgfqpoint{0.100000in}{0.212622in}}{\pgfqpoint{3.696000in}{3.696000in}}%
\pgfusepath{clip}%
\pgfsetrectcap%
\pgfsetroundjoin%
\pgfsetlinewidth{1.505625pt}%
\definecolor{currentstroke}{rgb}{1.000000,0.000000,0.000000}%
\pgfsetstrokecolor{currentstroke}%
\pgfsetdash{}{0pt}%
\pgfpathmoveto{\pgfqpoint{0.901830in}{1.339322in}}%
\pgfpathlineto{\pgfqpoint{0.952296in}{1.300547in}}%
\pgfusepath{stroke}%
\end{pgfscope}%
\begin{pgfscope}%
\pgfpathrectangle{\pgfqpoint{0.100000in}{0.212622in}}{\pgfqpoint{3.696000in}{3.696000in}}%
\pgfusepath{clip}%
\pgfsetrectcap%
\pgfsetroundjoin%
\pgfsetlinewidth{1.505625pt}%
\definecolor{currentstroke}{rgb}{1.000000,0.000000,0.000000}%
\pgfsetstrokecolor{currentstroke}%
\pgfsetdash{}{0pt}%
\pgfpathmoveto{\pgfqpoint{0.893374in}{1.347760in}}%
\pgfpathlineto{\pgfqpoint{0.952296in}{1.300547in}}%
\pgfusepath{stroke}%
\end{pgfscope}%
\begin{pgfscope}%
\pgfpathrectangle{\pgfqpoint{0.100000in}{0.212622in}}{\pgfqpoint{3.696000in}{3.696000in}}%
\pgfusepath{clip}%
\pgfsetrectcap%
\pgfsetroundjoin%
\pgfsetlinewidth{1.505625pt}%
\definecolor{currentstroke}{rgb}{1.000000,0.000000,0.000000}%
\pgfsetstrokecolor{currentstroke}%
\pgfsetdash{}{0pt}%
\pgfpathmoveto{\pgfqpoint{0.883794in}{1.357663in}}%
\pgfpathlineto{\pgfqpoint{0.952296in}{1.300547in}}%
\pgfusepath{stroke}%
\end{pgfscope}%
\begin{pgfscope}%
\pgfpathrectangle{\pgfqpoint{0.100000in}{0.212622in}}{\pgfqpoint{3.696000in}{3.696000in}}%
\pgfusepath{clip}%
\pgfsetrectcap%
\pgfsetroundjoin%
\pgfsetlinewidth{1.505625pt}%
\definecolor{currentstroke}{rgb}{1.000000,0.000000,0.000000}%
\pgfsetstrokecolor{currentstroke}%
\pgfsetdash{}{0pt}%
\pgfpathmoveto{\pgfqpoint{0.878525in}{1.363208in}}%
\pgfpathlineto{\pgfqpoint{0.952296in}{1.300547in}}%
\pgfusepath{stroke}%
\end{pgfscope}%
\begin{pgfscope}%
\pgfpathrectangle{\pgfqpoint{0.100000in}{0.212622in}}{\pgfqpoint{3.696000in}{3.696000in}}%
\pgfusepath{clip}%
\pgfsetrectcap%
\pgfsetroundjoin%
\pgfsetlinewidth{1.505625pt}%
\definecolor{currentstroke}{rgb}{1.000000,0.000000,0.000000}%
\pgfsetstrokecolor{currentstroke}%
\pgfsetdash{}{0pt}%
\pgfpathmoveto{\pgfqpoint{0.875628in}{1.366474in}}%
\pgfpathlineto{\pgfqpoint{0.952296in}{1.300547in}}%
\pgfusepath{stroke}%
\end{pgfscope}%
\begin{pgfscope}%
\pgfpathrectangle{\pgfqpoint{0.100000in}{0.212622in}}{\pgfqpoint{3.696000in}{3.696000in}}%
\pgfusepath{clip}%
\pgfsetrectcap%
\pgfsetroundjoin%
\pgfsetlinewidth{1.505625pt}%
\definecolor{currentstroke}{rgb}{1.000000,0.000000,0.000000}%
\pgfsetstrokecolor{currentstroke}%
\pgfsetdash{}{0pt}%
\pgfpathmoveto{\pgfqpoint{0.874039in}{1.368389in}}%
\pgfpathlineto{\pgfqpoint{0.952296in}{1.300547in}}%
\pgfusepath{stroke}%
\end{pgfscope}%
\begin{pgfscope}%
\pgfpathrectangle{\pgfqpoint{0.100000in}{0.212622in}}{\pgfqpoint{3.696000in}{3.696000in}}%
\pgfusepath{clip}%
\pgfsetrectcap%
\pgfsetroundjoin%
\pgfsetlinewidth{1.505625pt}%
\definecolor{currentstroke}{rgb}{1.000000,0.000000,0.000000}%
\pgfsetstrokecolor{currentstroke}%
\pgfsetdash{}{0pt}%
\pgfpathmoveto{\pgfqpoint{0.873170in}{1.369494in}}%
\pgfpathlineto{\pgfqpoint{0.952296in}{1.300547in}}%
\pgfusepath{stroke}%
\end{pgfscope}%
\begin{pgfscope}%
\pgfpathrectangle{\pgfqpoint{0.100000in}{0.212622in}}{\pgfqpoint{3.696000in}{3.696000in}}%
\pgfusepath{clip}%
\pgfsetrectcap%
\pgfsetroundjoin%
\pgfsetlinewidth{1.505625pt}%
\definecolor{currentstroke}{rgb}{1.000000,0.000000,0.000000}%
\pgfsetstrokecolor{currentstroke}%
\pgfsetdash{}{0pt}%
\pgfpathmoveto{\pgfqpoint{0.872697in}{1.370128in}}%
\pgfpathlineto{\pgfqpoint{0.952296in}{1.300547in}}%
\pgfusepath{stroke}%
\end{pgfscope}%
\begin{pgfscope}%
\pgfpathrectangle{\pgfqpoint{0.100000in}{0.212622in}}{\pgfqpoint{3.696000in}{3.696000in}}%
\pgfusepath{clip}%
\pgfsetrectcap%
\pgfsetroundjoin%
\pgfsetlinewidth{1.505625pt}%
\definecolor{currentstroke}{rgb}{1.000000,0.000000,0.000000}%
\pgfsetstrokecolor{currentstroke}%
\pgfsetdash{}{0pt}%
\pgfpathmoveto{\pgfqpoint{0.872439in}{1.370487in}}%
\pgfpathlineto{\pgfqpoint{0.952296in}{1.300547in}}%
\pgfusepath{stroke}%
\end{pgfscope}%
\begin{pgfscope}%
\pgfpathrectangle{\pgfqpoint{0.100000in}{0.212622in}}{\pgfqpoint{3.696000in}{3.696000in}}%
\pgfusepath{clip}%
\pgfsetrectcap%
\pgfsetroundjoin%
\pgfsetlinewidth{1.505625pt}%
\definecolor{currentstroke}{rgb}{1.000000,0.000000,0.000000}%
\pgfsetstrokecolor{currentstroke}%
\pgfsetdash{}{0pt}%
\pgfpathmoveto{\pgfqpoint{0.872300in}{1.370688in}}%
\pgfpathlineto{\pgfqpoint{0.952296in}{1.300547in}}%
\pgfusepath{stroke}%
\end{pgfscope}%
\begin{pgfscope}%
\pgfpathrectangle{\pgfqpoint{0.100000in}{0.212622in}}{\pgfqpoint{3.696000in}{3.696000in}}%
\pgfusepath{clip}%
\pgfsetrectcap%
\pgfsetroundjoin%
\pgfsetlinewidth{1.505625pt}%
\definecolor{currentstroke}{rgb}{1.000000,0.000000,0.000000}%
\pgfsetstrokecolor{currentstroke}%
\pgfsetdash{}{0pt}%
\pgfpathmoveto{\pgfqpoint{0.872225in}{1.370800in}}%
\pgfpathlineto{\pgfqpoint{0.952296in}{1.300547in}}%
\pgfusepath{stroke}%
\end{pgfscope}%
\begin{pgfscope}%
\pgfpathrectangle{\pgfqpoint{0.100000in}{0.212622in}}{\pgfqpoint{3.696000in}{3.696000in}}%
\pgfusepath{clip}%
\pgfsetrectcap%
\pgfsetroundjoin%
\pgfsetlinewidth{1.505625pt}%
\definecolor{currentstroke}{rgb}{1.000000,0.000000,0.000000}%
\pgfsetstrokecolor{currentstroke}%
\pgfsetdash{}{0pt}%
\pgfpathmoveto{\pgfqpoint{0.872184in}{1.370862in}}%
\pgfpathlineto{\pgfqpoint{0.952296in}{1.300547in}}%
\pgfusepath{stroke}%
\end{pgfscope}%
\begin{pgfscope}%
\pgfpathrectangle{\pgfqpoint{0.100000in}{0.212622in}}{\pgfqpoint{3.696000in}{3.696000in}}%
\pgfusepath{clip}%
\pgfsetrectcap%
\pgfsetroundjoin%
\pgfsetlinewidth{1.505625pt}%
\definecolor{currentstroke}{rgb}{1.000000,0.000000,0.000000}%
\pgfsetstrokecolor{currentstroke}%
\pgfsetdash{}{0pt}%
\pgfpathmoveto{\pgfqpoint{0.871618in}{1.371755in}}%
\pgfpathlineto{\pgfqpoint{0.952296in}{1.300547in}}%
\pgfusepath{stroke}%
\end{pgfscope}%
\begin{pgfscope}%
\pgfpathrectangle{\pgfqpoint{0.100000in}{0.212622in}}{\pgfqpoint{3.696000in}{3.696000in}}%
\pgfusepath{clip}%
\pgfsetrectcap%
\pgfsetroundjoin%
\pgfsetlinewidth{1.505625pt}%
\definecolor{currentstroke}{rgb}{1.000000,0.000000,0.000000}%
\pgfsetstrokecolor{currentstroke}%
\pgfsetdash{}{0pt}%
\pgfpathmoveto{\pgfqpoint{0.871314in}{1.372243in}}%
\pgfpathlineto{\pgfqpoint{0.952296in}{1.300547in}}%
\pgfusepath{stroke}%
\end{pgfscope}%
\begin{pgfscope}%
\pgfpathrectangle{\pgfqpoint{0.100000in}{0.212622in}}{\pgfqpoint{3.696000in}{3.696000in}}%
\pgfusepath{clip}%
\pgfsetrectcap%
\pgfsetroundjoin%
\pgfsetlinewidth{1.505625pt}%
\definecolor{currentstroke}{rgb}{1.000000,0.000000,0.000000}%
\pgfsetstrokecolor{currentstroke}%
\pgfsetdash{}{0pt}%
\pgfpathmoveto{\pgfqpoint{0.870477in}{1.373631in}}%
\pgfpathlineto{\pgfqpoint{0.952296in}{1.300547in}}%
\pgfusepath{stroke}%
\end{pgfscope}%
\begin{pgfscope}%
\pgfpathrectangle{\pgfqpoint{0.100000in}{0.212622in}}{\pgfqpoint{3.696000in}{3.696000in}}%
\pgfusepath{clip}%
\pgfsetrectcap%
\pgfsetroundjoin%
\pgfsetlinewidth{1.505625pt}%
\definecolor{currentstroke}{rgb}{1.000000,0.000000,0.000000}%
\pgfsetstrokecolor{currentstroke}%
\pgfsetdash{}{0pt}%
\pgfpathmoveto{\pgfqpoint{0.870033in}{1.374387in}}%
\pgfpathlineto{\pgfqpoint{0.952296in}{1.300547in}}%
\pgfusepath{stroke}%
\end{pgfscope}%
\begin{pgfscope}%
\pgfpathrectangle{\pgfqpoint{0.100000in}{0.212622in}}{\pgfqpoint{3.696000in}{3.696000in}}%
\pgfusepath{clip}%
\pgfsetrectcap%
\pgfsetroundjoin%
\pgfsetlinewidth{1.505625pt}%
\definecolor{currentstroke}{rgb}{1.000000,0.000000,0.000000}%
\pgfsetstrokecolor{currentstroke}%
\pgfsetdash{}{0pt}%
\pgfpathmoveto{\pgfqpoint{0.869798in}{1.374800in}}%
\pgfpathlineto{\pgfqpoint{0.952296in}{1.300547in}}%
\pgfusepath{stroke}%
\end{pgfscope}%
\begin{pgfscope}%
\pgfpathrectangle{\pgfqpoint{0.100000in}{0.212622in}}{\pgfqpoint{3.696000in}{3.696000in}}%
\pgfusepath{clip}%
\pgfsetrectcap%
\pgfsetroundjoin%
\pgfsetlinewidth{1.505625pt}%
\definecolor{currentstroke}{rgb}{1.000000,0.000000,0.000000}%
\pgfsetstrokecolor{currentstroke}%
\pgfsetdash{}{0pt}%
\pgfpathmoveto{\pgfqpoint{0.869675in}{1.375026in}}%
\pgfpathlineto{\pgfqpoint{0.952296in}{1.300547in}}%
\pgfusepath{stroke}%
\end{pgfscope}%
\begin{pgfscope}%
\pgfpathrectangle{\pgfqpoint{0.100000in}{0.212622in}}{\pgfqpoint{3.696000in}{3.696000in}}%
\pgfusepath{clip}%
\pgfsetrectcap%
\pgfsetroundjoin%
\pgfsetlinewidth{1.505625pt}%
\definecolor{currentstroke}{rgb}{1.000000,0.000000,0.000000}%
\pgfsetstrokecolor{currentstroke}%
\pgfsetdash{}{0pt}%
\pgfpathmoveto{\pgfqpoint{0.869009in}{1.376302in}}%
\pgfpathlineto{\pgfqpoint{0.952296in}{1.300547in}}%
\pgfusepath{stroke}%
\end{pgfscope}%
\begin{pgfscope}%
\pgfpathrectangle{\pgfqpoint{0.100000in}{0.212622in}}{\pgfqpoint{3.696000in}{3.696000in}}%
\pgfusepath{clip}%
\pgfsetrectcap%
\pgfsetroundjoin%
\pgfsetlinewidth{1.505625pt}%
\definecolor{currentstroke}{rgb}{1.000000,0.000000,0.000000}%
\pgfsetstrokecolor{currentstroke}%
\pgfsetdash{}{0pt}%
\pgfpathmoveto{\pgfqpoint{0.868663in}{1.376986in}}%
\pgfpathlineto{\pgfqpoint{0.952296in}{1.300547in}}%
\pgfusepath{stroke}%
\end{pgfscope}%
\begin{pgfscope}%
\pgfpathrectangle{\pgfqpoint{0.100000in}{0.212622in}}{\pgfqpoint{3.696000in}{3.696000in}}%
\pgfusepath{clip}%
\pgfsetrectcap%
\pgfsetroundjoin%
\pgfsetlinewidth{1.505625pt}%
\definecolor{currentstroke}{rgb}{1.000000,0.000000,0.000000}%
\pgfsetstrokecolor{currentstroke}%
\pgfsetdash{}{0pt}%
\pgfpathmoveto{\pgfqpoint{0.867740in}{1.378920in}}%
\pgfpathlineto{\pgfqpoint{0.952296in}{1.300547in}}%
\pgfusepath{stroke}%
\end{pgfscope}%
\begin{pgfscope}%
\pgfpathrectangle{\pgfqpoint{0.100000in}{0.212622in}}{\pgfqpoint{3.696000in}{3.696000in}}%
\pgfusepath{clip}%
\pgfsetrectcap%
\pgfsetroundjoin%
\pgfsetlinewidth{1.505625pt}%
\definecolor{currentstroke}{rgb}{1.000000,0.000000,0.000000}%
\pgfsetstrokecolor{currentstroke}%
\pgfsetdash{}{0pt}%
\pgfpathmoveto{\pgfqpoint{0.867264in}{1.379993in}}%
\pgfpathlineto{\pgfqpoint{0.952296in}{1.300547in}}%
\pgfusepath{stroke}%
\end{pgfscope}%
\begin{pgfscope}%
\pgfpathrectangle{\pgfqpoint{0.100000in}{0.212622in}}{\pgfqpoint{3.696000in}{3.696000in}}%
\pgfusepath{clip}%
\pgfsetrectcap%
\pgfsetroundjoin%
\pgfsetlinewidth{1.505625pt}%
\definecolor{currentstroke}{rgb}{1.000000,0.000000,0.000000}%
\pgfsetstrokecolor{currentstroke}%
\pgfsetdash{}{0pt}%
\pgfpathmoveto{\pgfqpoint{0.865527in}{1.384093in}}%
\pgfpathlineto{\pgfqpoint{0.952296in}{1.300547in}}%
\pgfusepath{stroke}%
\end{pgfscope}%
\begin{pgfscope}%
\pgfpathrectangle{\pgfqpoint{0.100000in}{0.212622in}}{\pgfqpoint{3.696000in}{3.696000in}}%
\pgfusepath{clip}%
\pgfsetrectcap%
\pgfsetroundjoin%
\pgfsetlinewidth{1.505625pt}%
\definecolor{currentstroke}{rgb}{1.000000,0.000000,0.000000}%
\pgfsetstrokecolor{currentstroke}%
\pgfsetdash{}{0pt}%
\pgfpathmoveto{\pgfqpoint{0.862582in}{1.391563in}}%
\pgfpathlineto{\pgfqpoint{0.952296in}{1.300547in}}%
\pgfusepath{stroke}%
\end{pgfscope}%
\begin{pgfscope}%
\pgfpathrectangle{\pgfqpoint{0.100000in}{0.212622in}}{\pgfqpoint{3.696000in}{3.696000in}}%
\pgfusepath{clip}%
\pgfsetrectcap%
\pgfsetroundjoin%
\pgfsetlinewidth{1.505625pt}%
\definecolor{currentstroke}{rgb}{1.000000,0.000000,0.000000}%
\pgfsetstrokecolor{currentstroke}%
\pgfsetdash{}{0pt}%
\pgfpathmoveto{\pgfqpoint{0.861065in}{1.395584in}}%
\pgfpathlineto{\pgfqpoint{0.952296in}{1.300547in}}%
\pgfusepath{stroke}%
\end{pgfscope}%
\begin{pgfscope}%
\pgfpathrectangle{\pgfqpoint{0.100000in}{0.212622in}}{\pgfqpoint{3.696000in}{3.696000in}}%
\pgfusepath{clip}%
\pgfsetrectcap%
\pgfsetroundjoin%
\pgfsetlinewidth{1.505625pt}%
\definecolor{currentstroke}{rgb}{1.000000,0.000000,0.000000}%
\pgfsetstrokecolor{currentstroke}%
\pgfsetdash{}{0pt}%
\pgfpathmoveto{\pgfqpoint{0.860300in}{1.397746in}}%
\pgfpathlineto{\pgfqpoint{0.952296in}{1.300547in}}%
\pgfusepath{stroke}%
\end{pgfscope}%
\begin{pgfscope}%
\pgfpathrectangle{\pgfqpoint{0.100000in}{0.212622in}}{\pgfqpoint{3.696000in}{3.696000in}}%
\pgfusepath{clip}%
\pgfsetrectcap%
\pgfsetroundjoin%
\pgfsetlinewidth{1.505625pt}%
\definecolor{currentstroke}{rgb}{1.000000,0.000000,0.000000}%
\pgfsetstrokecolor{currentstroke}%
\pgfsetdash{}{0pt}%
\pgfpathmoveto{\pgfqpoint{0.859079in}{1.401381in}}%
\pgfpathlineto{\pgfqpoint{0.952296in}{1.300547in}}%
\pgfusepath{stroke}%
\end{pgfscope}%
\begin{pgfscope}%
\pgfpathrectangle{\pgfqpoint{0.100000in}{0.212622in}}{\pgfqpoint{3.696000in}{3.696000in}}%
\pgfusepath{clip}%
\pgfsetrectcap%
\pgfsetroundjoin%
\pgfsetlinewidth{1.505625pt}%
\definecolor{currentstroke}{rgb}{1.000000,0.000000,0.000000}%
\pgfsetstrokecolor{currentstroke}%
\pgfsetdash{}{0pt}%
\pgfpathmoveto{\pgfqpoint{0.858488in}{1.403329in}}%
\pgfpathlineto{\pgfqpoint{0.952296in}{1.300547in}}%
\pgfusepath{stroke}%
\end{pgfscope}%
\begin{pgfscope}%
\pgfpathrectangle{\pgfqpoint{0.100000in}{0.212622in}}{\pgfqpoint{3.696000in}{3.696000in}}%
\pgfusepath{clip}%
\pgfsetrectcap%
\pgfsetroundjoin%
\pgfsetlinewidth{1.505625pt}%
\definecolor{currentstroke}{rgb}{1.000000,0.000000,0.000000}%
\pgfsetstrokecolor{currentstroke}%
\pgfsetdash{}{0pt}%
\pgfpathmoveto{\pgfqpoint{0.857549in}{1.406726in}}%
\pgfpathlineto{\pgfqpoint{0.952296in}{1.300547in}}%
\pgfusepath{stroke}%
\end{pgfscope}%
\begin{pgfscope}%
\pgfpathrectangle{\pgfqpoint{0.100000in}{0.212622in}}{\pgfqpoint{3.696000in}{3.696000in}}%
\pgfusepath{clip}%
\pgfsetrectcap%
\pgfsetroundjoin%
\pgfsetlinewidth{1.505625pt}%
\definecolor{currentstroke}{rgb}{1.000000,0.000000,0.000000}%
\pgfsetstrokecolor{currentstroke}%
\pgfsetdash{}{0pt}%
\pgfpathmoveto{\pgfqpoint{0.856250in}{1.412150in}}%
\pgfpathlineto{\pgfqpoint{0.952296in}{1.300547in}}%
\pgfusepath{stroke}%
\end{pgfscope}%
\begin{pgfscope}%
\pgfpathrectangle{\pgfqpoint{0.100000in}{0.212622in}}{\pgfqpoint{3.696000in}{3.696000in}}%
\pgfusepath{clip}%
\pgfsetrectcap%
\pgfsetroundjoin%
\pgfsetlinewidth{1.505625pt}%
\definecolor{currentstroke}{rgb}{1.000000,0.000000,0.000000}%
\pgfsetstrokecolor{currentstroke}%
\pgfsetdash{}{0pt}%
\pgfpathmoveto{\pgfqpoint{0.854825in}{1.418753in}}%
\pgfpathlineto{\pgfqpoint{0.952296in}{1.300547in}}%
\pgfusepath{stroke}%
\end{pgfscope}%
\begin{pgfscope}%
\pgfpathrectangle{\pgfqpoint{0.100000in}{0.212622in}}{\pgfqpoint{3.696000in}{3.696000in}}%
\pgfusepath{clip}%
\pgfsetrectcap%
\pgfsetroundjoin%
\pgfsetlinewidth{1.505625pt}%
\definecolor{currentstroke}{rgb}{1.000000,0.000000,0.000000}%
\pgfsetstrokecolor{currentstroke}%
\pgfsetdash{}{0pt}%
\pgfpathmoveto{\pgfqpoint{0.854219in}{1.422269in}}%
\pgfpathlineto{\pgfqpoint{0.952296in}{1.300547in}}%
\pgfusepath{stroke}%
\end{pgfscope}%
\begin{pgfscope}%
\pgfpathrectangle{\pgfqpoint{0.100000in}{0.212622in}}{\pgfqpoint{3.696000in}{3.696000in}}%
\pgfusepath{clip}%
\pgfsetrectcap%
\pgfsetroundjoin%
\pgfsetlinewidth{1.505625pt}%
\definecolor{currentstroke}{rgb}{1.000000,0.000000,0.000000}%
\pgfsetstrokecolor{currentstroke}%
\pgfsetdash{}{0pt}%
\pgfpathmoveto{\pgfqpoint{0.853354in}{1.427801in}}%
\pgfpathlineto{\pgfqpoint{0.952296in}{1.300547in}}%
\pgfusepath{stroke}%
\end{pgfscope}%
\begin{pgfscope}%
\pgfpathrectangle{\pgfqpoint{0.100000in}{0.212622in}}{\pgfqpoint{3.696000in}{3.696000in}}%
\pgfusepath{clip}%
\pgfsetrectcap%
\pgfsetroundjoin%
\pgfsetlinewidth{1.505625pt}%
\definecolor{currentstroke}{rgb}{1.000000,0.000000,0.000000}%
\pgfsetstrokecolor{currentstroke}%
\pgfsetdash{}{0pt}%
\pgfpathmoveto{\pgfqpoint{0.852506in}{1.435025in}}%
\pgfpathlineto{\pgfqpoint{0.952296in}{1.300547in}}%
\pgfusepath{stroke}%
\end{pgfscope}%
\begin{pgfscope}%
\pgfpathrectangle{\pgfqpoint{0.100000in}{0.212622in}}{\pgfqpoint{3.696000in}{3.696000in}}%
\pgfusepath{clip}%
\pgfsetrectcap%
\pgfsetroundjoin%
\pgfsetlinewidth{1.505625pt}%
\definecolor{currentstroke}{rgb}{1.000000,0.000000,0.000000}%
\pgfsetstrokecolor{currentstroke}%
\pgfsetdash{}{0pt}%
\pgfpathmoveto{\pgfqpoint{0.851545in}{1.444188in}}%
\pgfpathlineto{\pgfqpoint{0.952296in}{1.300547in}}%
\pgfusepath{stroke}%
\end{pgfscope}%
\begin{pgfscope}%
\pgfpathrectangle{\pgfqpoint{0.100000in}{0.212622in}}{\pgfqpoint{3.696000in}{3.696000in}}%
\pgfusepath{clip}%
\pgfsetrectcap%
\pgfsetroundjoin%
\pgfsetlinewidth{1.505625pt}%
\definecolor{currentstroke}{rgb}{1.000000,0.000000,0.000000}%
\pgfsetstrokecolor{currentstroke}%
\pgfsetdash{}{0pt}%
\pgfpathmoveto{\pgfqpoint{0.850875in}{1.455929in}}%
\pgfpathlineto{\pgfqpoint{0.952296in}{1.300547in}}%
\pgfusepath{stroke}%
\end{pgfscope}%
\begin{pgfscope}%
\pgfpathrectangle{\pgfqpoint{0.100000in}{0.212622in}}{\pgfqpoint{3.696000in}{3.696000in}}%
\pgfusepath{clip}%
\pgfsetrectcap%
\pgfsetroundjoin%
\pgfsetlinewidth{1.505625pt}%
\definecolor{currentstroke}{rgb}{1.000000,0.000000,0.000000}%
\pgfsetstrokecolor{currentstroke}%
\pgfsetdash{}{0pt}%
\pgfpathmoveto{\pgfqpoint{0.850262in}{1.469643in}}%
\pgfpathlineto{\pgfqpoint{0.961164in}{1.308167in}}%
\pgfusepath{stroke}%
\end{pgfscope}%
\begin{pgfscope}%
\pgfpathrectangle{\pgfqpoint{0.100000in}{0.212622in}}{\pgfqpoint{3.696000in}{3.696000in}}%
\pgfusepath{clip}%
\pgfsetrectcap%
\pgfsetroundjoin%
\pgfsetlinewidth{1.505625pt}%
\definecolor{currentstroke}{rgb}{1.000000,0.000000,0.000000}%
\pgfsetstrokecolor{currentstroke}%
\pgfsetdash{}{0pt}%
\pgfpathmoveto{\pgfqpoint{0.850435in}{1.484214in}}%
\pgfpathlineto{\pgfqpoint{0.970020in}{1.315778in}}%
\pgfusepath{stroke}%
\end{pgfscope}%
\begin{pgfscope}%
\pgfpathrectangle{\pgfqpoint{0.100000in}{0.212622in}}{\pgfqpoint{3.696000in}{3.696000in}}%
\pgfusepath{clip}%
\pgfsetrectcap%
\pgfsetroundjoin%
\pgfsetlinewidth{1.505625pt}%
\definecolor{currentstroke}{rgb}{1.000000,0.000000,0.000000}%
\pgfsetstrokecolor{currentstroke}%
\pgfsetdash{}{0pt}%
\pgfpathmoveto{\pgfqpoint{0.850787in}{1.500465in}}%
\pgfpathlineto{\pgfqpoint{0.978866in}{1.323380in}}%
\pgfusepath{stroke}%
\end{pgfscope}%
\begin{pgfscope}%
\pgfpathrectangle{\pgfqpoint{0.100000in}{0.212622in}}{\pgfqpoint{3.696000in}{3.696000in}}%
\pgfusepath{clip}%
\pgfsetrectcap%
\pgfsetroundjoin%
\pgfsetlinewidth{1.505625pt}%
\definecolor{currentstroke}{rgb}{1.000000,0.000000,0.000000}%
\pgfsetstrokecolor{currentstroke}%
\pgfsetdash{}{0pt}%
\pgfpathmoveto{\pgfqpoint{0.851675in}{1.508936in}}%
\pgfpathlineto{\pgfqpoint{0.978866in}{1.323380in}}%
\pgfusepath{stroke}%
\end{pgfscope}%
\begin{pgfscope}%
\pgfpathrectangle{\pgfqpoint{0.100000in}{0.212622in}}{\pgfqpoint{3.696000in}{3.696000in}}%
\pgfusepath{clip}%
\pgfsetrectcap%
\pgfsetroundjoin%
\pgfsetlinewidth{1.505625pt}%
\definecolor{currentstroke}{rgb}{1.000000,0.000000,0.000000}%
\pgfsetstrokecolor{currentstroke}%
\pgfsetdash{}{0pt}%
\pgfpathmoveto{\pgfqpoint{0.852901in}{1.518700in}}%
\pgfpathlineto{\pgfqpoint{0.987700in}{1.330972in}}%
\pgfusepath{stroke}%
\end{pgfscope}%
\begin{pgfscope}%
\pgfpathrectangle{\pgfqpoint{0.100000in}{0.212622in}}{\pgfqpoint{3.696000in}{3.696000in}}%
\pgfusepath{clip}%
\pgfsetrectcap%
\pgfsetroundjoin%
\pgfsetlinewidth{1.505625pt}%
\definecolor{currentstroke}{rgb}{1.000000,0.000000,0.000000}%
\pgfsetstrokecolor{currentstroke}%
\pgfsetdash{}{0pt}%
\pgfpathmoveto{\pgfqpoint{0.854010in}{1.523755in}}%
\pgfpathlineto{\pgfqpoint{0.987700in}{1.330972in}}%
\pgfusepath{stroke}%
\end{pgfscope}%
\begin{pgfscope}%
\pgfpathrectangle{\pgfqpoint{0.100000in}{0.212622in}}{\pgfqpoint{3.696000in}{3.696000in}}%
\pgfusepath{clip}%
\pgfsetrectcap%
\pgfsetroundjoin%
\pgfsetlinewidth{1.505625pt}%
\definecolor{currentstroke}{rgb}{1.000000,0.000000,0.000000}%
\pgfsetstrokecolor{currentstroke}%
\pgfsetdash{}{0pt}%
\pgfpathmoveto{\pgfqpoint{0.855625in}{1.530229in}}%
\pgfpathlineto{\pgfqpoint{0.996524in}{1.338555in}}%
\pgfusepath{stroke}%
\end{pgfscope}%
\begin{pgfscope}%
\pgfpathrectangle{\pgfqpoint{0.100000in}{0.212622in}}{\pgfqpoint{3.696000in}{3.696000in}}%
\pgfusepath{clip}%
\pgfsetrectcap%
\pgfsetroundjoin%
\pgfsetlinewidth{1.505625pt}%
\definecolor{currentstroke}{rgb}{1.000000,0.000000,0.000000}%
\pgfsetstrokecolor{currentstroke}%
\pgfsetdash{}{0pt}%
\pgfpathmoveto{\pgfqpoint{0.857884in}{1.537150in}}%
\pgfpathlineto{\pgfqpoint{0.996524in}{1.338555in}}%
\pgfusepath{stroke}%
\end{pgfscope}%
\begin{pgfscope}%
\pgfpathrectangle{\pgfqpoint{0.100000in}{0.212622in}}{\pgfqpoint{3.696000in}{3.696000in}}%
\pgfusepath{clip}%
\pgfsetrectcap%
\pgfsetroundjoin%
\pgfsetlinewidth{1.505625pt}%
\definecolor{currentstroke}{rgb}{1.000000,0.000000,0.000000}%
\pgfsetstrokecolor{currentstroke}%
\pgfsetdash{}{0pt}%
\pgfpathmoveto{\pgfqpoint{0.860725in}{1.545404in}}%
\pgfpathlineto{\pgfqpoint{1.005337in}{1.346129in}}%
\pgfusepath{stroke}%
\end{pgfscope}%
\begin{pgfscope}%
\pgfpathrectangle{\pgfqpoint{0.100000in}{0.212622in}}{\pgfqpoint{3.696000in}{3.696000in}}%
\pgfusepath{clip}%
\pgfsetrectcap%
\pgfsetroundjoin%
\pgfsetlinewidth{1.505625pt}%
\definecolor{currentstroke}{rgb}{1.000000,0.000000,0.000000}%
\pgfsetstrokecolor{currentstroke}%
\pgfsetdash{}{0pt}%
\pgfpathmoveto{\pgfqpoint{0.862544in}{1.549906in}}%
\pgfpathlineto{\pgfqpoint{1.005337in}{1.346129in}}%
\pgfusepath{stroke}%
\end{pgfscope}%
\begin{pgfscope}%
\pgfpathrectangle{\pgfqpoint{0.100000in}{0.212622in}}{\pgfqpoint{3.696000in}{3.696000in}}%
\pgfusepath{clip}%
\pgfsetrectcap%
\pgfsetroundjoin%
\pgfsetlinewidth{1.505625pt}%
\definecolor{currentstroke}{rgb}{1.000000,0.000000,0.000000}%
\pgfsetstrokecolor{currentstroke}%
\pgfsetdash{}{0pt}%
\pgfpathmoveto{\pgfqpoint{0.865007in}{1.555949in}}%
\pgfpathlineto{\pgfqpoint{1.014139in}{1.353693in}}%
\pgfusepath{stroke}%
\end{pgfscope}%
\begin{pgfscope}%
\pgfpathrectangle{\pgfqpoint{0.100000in}{0.212622in}}{\pgfqpoint{3.696000in}{3.696000in}}%
\pgfusepath{clip}%
\pgfsetrectcap%
\pgfsetroundjoin%
\pgfsetlinewidth{1.505625pt}%
\definecolor{currentstroke}{rgb}{1.000000,0.000000,0.000000}%
\pgfsetstrokecolor{currentstroke}%
\pgfsetdash{}{0pt}%
\pgfpathmoveto{\pgfqpoint{0.866548in}{1.559217in}}%
\pgfpathlineto{\pgfqpoint{1.014139in}{1.353693in}}%
\pgfusepath{stroke}%
\end{pgfscope}%
\begin{pgfscope}%
\pgfpathrectangle{\pgfqpoint{0.100000in}{0.212622in}}{\pgfqpoint{3.696000in}{3.696000in}}%
\pgfusepath{clip}%
\pgfsetrectcap%
\pgfsetroundjoin%
\pgfsetlinewidth{1.505625pt}%
\definecolor{currentstroke}{rgb}{1.000000,0.000000,0.000000}%
\pgfsetstrokecolor{currentstroke}%
\pgfsetdash{}{0pt}%
\pgfpathmoveto{\pgfqpoint{0.868753in}{1.564024in}}%
\pgfpathlineto{\pgfqpoint{1.014139in}{1.353693in}}%
\pgfusepath{stroke}%
\end{pgfscope}%
\begin{pgfscope}%
\pgfpathrectangle{\pgfqpoint{0.100000in}{0.212622in}}{\pgfqpoint{3.696000in}{3.696000in}}%
\pgfusepath{clip}%
\pgfsetrectcap%
\pgfsetroundjoin%
\pgfsetlinewidth{1.505625pt}%
\definecolor{currentstroke}{rgb}{1.000000,0.000000,0.000000}%
\pgfsetstrokecolor{currentstroke}%
\pgfsetdash{}{0pt}%
\pgfpathmoveto{\pgfqpoint{0.870192in}{1.566529in}}%
\pgfpathlineto{\pgfqpoint{1.014139in}{1.353693in}}%
\pgfusepath{stroke}%
\end{pgfscope}%
\begin{pgfscope}%
\pgfpathrectangle{\pgfqpoint{0.100000in}{0.212622in}}{\pgfqpoint{3.696000in}{3.696000in}}%
\pgfusepath{clip}%
\pgfsetrectcap%
\pgfsetroundjoin%
\pgfsetlinewidth{1.505625pt}%
\definecolor{currentstroke}{rgb}{1.000000,0.000000,0.000000}%
\pgfsetstrokecolor{currentstroke}%
\pgfsetdash{}{0pt}%
\pgfpathmoveto{\pgfqpoint{0.870947in}{1.567932in}}%
\pgfpathlineto{\pgfqpoint{1.022930in}{1.361248in}}%
\pgfusepath{stroke}%
\end{pgfscope}%
\begin{pgfscope}%
\pgfpathrectangle{\pgfqpoint{0.100000in}{0.212622in}}{\pgfqpoint{3.696000in}{3.696000in}}%
\pgfusepath{clip}%
\pgfsetrectcap%
\pgfsetroundjoin%
\pgfsetlinewidth{1.505625pt}%
\definecolor{currentstroke}{rgb}{1.000000,0.000000,0.000000}%
\pgfsetstrokecolor{currentstroke}%
\pgfsetdash{}{0pt}%
\pgfpathmoveto{\pgfqpoint{0.871428in}{1.568629in}}%
\pgfpathlineto{\pgfqpoint{1.022930in}{1.361248in}}%
\pgfusepath{stroke}%
\end{pgfscope}%
\begin{pgfscope}%
\pgfpathrectangle{\pgfqpoint{0.100000in}{0.212622in}}{\pgfqpoint{3.696000in}{3.696000in}}%
\pgfusepath{clip}%
\pgfsetrectcap%
\pgfsetroundjoin%
\pgfsetlinewidth{1.505625pt}%
\definecolor{currentstroke}{rgb}{1.000000,0.000000,0.000000}%
\pgfsetstrokecolor{currentstroke}%
\pgfsetdash{}{0pt}%
\pgfpathmoveto{\pgfqpoint{0.871699in}{1.569048in}}%
\pgfpathlineto{\pgfqpoint{1.022930in}{1.361248in}}%
\pgfusepath{stroke}%
\end{pgfscope}%
\begin{pgfscope}%
\pgfpathrectangle{\pgfqpoint{0.100000in}{0.212622in}}{\pgfqpoint{3.696000in}{3.696000in}}%
\pgfusepath{clip}%
\pgfsetrectcap%
\pgfsetroundjoin%
\pgfsetlinewidth{1.505625pt}%
\definecolor{currentstroke}{rgb}{1.000000,0.000000,0.000000}%
\pgfsetstrokecolor{currentstroke}%
\pgfsetdash{}{0pt}%
\pgfpathmoveto{\pgfqpoint{0.872282in}{1.569893in}}%
\pgfpathlineto{\pgfqpoint{1.022930in}{1.361248in}}%
\pgfusepath{stroke}%
\end{pgfscope}%
\begin{pgfscope}%
\pgfpathrectangle{\pgfqpoint{0.100000in}{0.212622in}}{\pgfqpoint{3.696000in}{3.696000in}}%
\pgfusepath{clip}%
\pgfsetrectcap%
\pgfsetroundjoin%
\pgfsetlinewidth{1.505625pt}%
\definecolor{currentstroke}{rgb}{1.000000,0.000000,0.000000}%
\pgfsetstrokecolor{currentstroke}%
\pgfsetdash{}{0pt}%
\pgfpathmoveto{\pgfqpoint{0.873464in}{1.571673in}}%
\pgfpathlineto{\pgfqpoint{1.022930in}{1.361248in}}%
\pgfusepath{stroke}%
\end{pgfscope}%
\begin{pgfscope}%
\pgfpathrectangle{\pgfqpoint{0.100000in}{0.212622in}}{\pgfqpoint{3.696000in}{3.696000in}}%
\pgfusepath{clip}%
\pgfsetrectcap%
\pgfsetroundjoin%
\pgfsetlinewidth{1.505625pt}%
\definecolor{currentstroke}{rgb}{1.000000,0.000000,0.000000}%
\pgfsetstrokecolor{currentstroke}%
\pgfsetdash{}{0pt}%
\pgfpathmoveto{\pgfqpoint{0.874106in}{1.572621in}}%
\pgfpathlineto{\pgfqpoint{1.022930in}{1.361248in}}%
\pgfusepath{stroke}%
\end{pgfscope}%
\begin{pgfscope}%
\pgfpathrectangle{\pgfqpoint{0.100000in}{0.212622in}}{\pgfqpoint{3.696000in}{3.696000in}}%
\pgfusepath{clip}%
\pgfsetrectcap%
\pgfsetroundjoin%
\pgfsetlinewidth{1.505625pt}%
\definecolor{currentstroke}{rgb}{1.000000,0.000000,0.000000}%
\pgfsetstrokecolor{currentstroke}%
\pgfsetdash{}{0pt}%
\pgfpathmoveto{\pgfqpoint{0.874461in}{1.573164in}}%
\pgfpathlineto{\pgfqpoint{1.022930in}{1.361248in}}%
\pgfusepath{stroke}%
\end{pgfscope}%
\begin{pgfscope}%
\pgfpathrectangle{\pgfqpoint{0.100000in}{0.212622in}}{\pgfqpoint{3.696000in}{3.696000in}}%
\pgfusepath{clip}%
\pgfsetrectcap%
\pgfsetroundjoin%
\pgfsetlinewidth{1.505625pt}%
\definecolor{currentstroke}{rgb}{1.000000,0.000000,0.000000}%
\pgfsetstrokecolor{currentstroke}%
\pgfsetdash{}{0pt}%
\pgfpathmoveto{\pgfqpoint{0.875873in}{1.575326in}}%
\pgfpathlineto{\pgfqpoint{1.022930in}{1.361248in}}%
\pgfusepath{stroke}%
\end{pgfscope}%
\begin{pgfscope}%
\pgfpathrectangle{\pgfqpoint{0.100000in}{0.212622in}}{\pgfqpoint{3.696000in}{3.696000in}}%
\pgfusepath{clip}%
\pgfsetrectcap%
\pgfsetroundjoin%
\pgfsetlinewidth{1.505625pt}%
\definecolor{currentstroke}{rgb}{1.000000,0.000000,0.000000}%
\pgfsetstrokecolor{currentstroke}%
\pgfsetdash{}{0pt}%
\pgfpathmoveto{\pgfqpoint{0.878210in}{1.578978in}}%
\pgfpathlineto{\pgfqpoint{1.022930in}{1.361248in}}%
\pgfusepath{stroke}%
\end{pgfscope}%
\begin{pgfscope}%
\pgfpathrectangle{\pgfqpoint{0.100000in}{0.212622in}}{\pgfqpoint{3.696000in}{3.696000in}}%
\pgfusepath{clip}%
\pgfsetrectcap%
\pgfsetroundjoin%
\pgfsetlinewidth{1.505625pt}%
\definecolor{currentstroke}{rgb}{1.000000,0.000000,0.000000}%
\pgfsetstrokecolor{currentstroke}%
\pgfsetdash{}{0pt}%
\pgfpathmoveto{\pgfqpoint{0.881865in}{1.584796in}}%
\pgfpathlineto{\pgfqpoint{1.031710in}{1.368794in}}%
\pgfusepath{stroke}%
\end{pgfscope}%
\begin{pgfscope}%
\pgfpathrectangle{\pgfqpoint{0.100000in}{0.212622in}}{\pgfqpoint{3.696000in}{3.696000in}}%
\pgfusepath{clip}%
\pgfsetrectcap%
\pgfsetroundjoin%
\pgfsetlinewidth{1.505625pt}%
\definecolor{currentstroke}{rgb}{1.000000,0.000000,0.000000}%
\pgfsetstrokecolor{currentstroke}%
\pgfsetdash{}{0pt}%
\pgfpathmoveto{\pgfqpoint{0.886590in}{1.591691in}}%
\pgfpathlineto{\pgfqpoint{1.040480in}{1.376330in}}%
\pgfusepath{stroke}%
\end{pgfscope}%
\begin{pgfscope}%
\pgfpathrectangle{\pgfqpoint{0.100000in}{0.212622in}}{\pgfqpoint{3.696000in}{3.696000in}}%
\pgfusepath{clip}%
\pgfsetrectcap%
\pgfsetroundjoin%
\pgfsetlinewidth{1.505625pt}%
\definecolor{currentstroke}{rgb}{1.000000,0.000000,0.000000}%
\pgfsetstrokecolor{currentstroke}%
\pgfsetdash{}{0pt}%
\pgfpathmoveto{\pgfqpoint{0.892161in}{1.600056in}}%
\pgfpathlineto{\pgfqpoint{1.040480in}{1.376330in}}%
\pgfusepath{stroke}%
\end{pgfscope}%
\begin{pgfscope}%
\pgfpathrectangle{\pgfqpoint{0.100000in}{0.212622in}}{\pgfqpoint{3.696000in}{3.696000in}}%
\pgfusepath{clip}%
\pgfsetrectcap%
\pgfsetroundjoin%
\pgfsetlinewidth{1.505625pt}%
\definecolor{currentstroke}{rgb}{1.000000,0.000000,0.000000}%
\pgfsetstrokecolor{currentstroke}%
\pgfsetdash{}{0pt}%
\pgfpathmoveto{\pgfqpoint{0.898606in}{1.609629in}}%
\pgfpathlineto{\pgfqpoint{1.049239in}{1.383857in}}%
\pgfusepath{stroke}%
\end{pgfscope}%
\begin{pgfscope}%
\pgfpathrectangle{\pgfqpoint{0.100000in}{0.212622in}}{\pgfqpoint{3.696000in}{3.696000in}}%
\pgfusepath{clip}%
\pgfsetrectcap%
\pgfsetroundjoin%
\pgfsetlinewidth{1.505625pt}%
\definecolor{currentstroke}{rgb}{1.000000,0.000000,0.000000}%
\pgfsetstrokecolor{currentstroke}%
\pgfsetdash{}{0pt}%
\pgfpathmoveto{\pgfqpoint{0.905750in}{1.620277in}}%
\pgfpathlineto{\pgfqpoint{1.057987in}{1.391375in}}%
\pgfusepath{stroke}%
\end{pgfscope}%
\begin{pgfscope}%
\pgfpathrectangle{\pgfqpoint{0.100000in}{0.212622in}}{\pgfqpoint{3.696000in}{3.696000in}}%
\pgfusepath{clip}%
\pgfsetrectcap%
\pgfsetroundjoin%
\pgfsetlinewidth{1.505625pt}%
\definecolor{currentstroke}{rgb}{1.000000,0.000000,0.000000}%
\pgfsetstrokecolor{currentstroke}%
\pgfsetdash{}{0pt}%
\pgfpathmoveto{\pgfqpoint{0.909798in}{1.627442in}}%
\pgfpathlineto{\pgfqpoint{1.066724in}{1.398884in}}%
\pgfusepath{stroke}%
\end{pgfscope}%
\begin{pgfscope}%
\pgfpathrectangle{\pgfqpoint{0.100000in}{0.212622in}}{\pgfqpoint{3.696000in}{3.696000in}}%
\pgfusepath{clip}%
\pgfsetrectcap%
\pgfsetroundjoin%
\pgfsetlinewidth{1.505625pt}%
\definecolor{currentstroke}{rgb}{1.000000,0.000000,0.000000}%
\pgfsetstrokecolor{currentstroke}%
\pgfsetdash{}{0pt}%
\pgfpathmoveto{\pgfqpoint{0.912144in}{1.630827in}}%
\pgfpathlineto{\pgfqpoint{1.066724in}{1.398884in}}%
\pgfusepath{stroke}%
\end{pgfscope}%
\begin{pgfscope}%
\pgfpathrectangle{\pgfqpoint{0.100000in}{0.212622in}}{\pgfqpoint{3.696000in}{3.696000in}}%
\pgfusepath{clip}%
\pgfsetrectcap%
\pgfsetroundjoin%
\pgfsetlinewidth{1.505625pt}%
\definecolor{currentstroke}{rgb}{1.000000,0.000000,0.000000}%
\pgfsetstrokecolor{currentstroke}%
\pgfsetdash{}{0pt}%
\pgfpathmoveto{\pgfqpoint{0.914903in}{1.634583in}}%
\pgfpathlineto{\pgfqpoint{1.066724in}{1.398884in}}%
\pgfusepath{stroke}%
\end{pgfscope}%
\begin{pgfscope}%
\pgfpathrectangle{\pgfqpoint{0.100000in}{0.212622in}}{\pgfqpoint{3.696000in}{3.696000in}}%
\pgfusepath{clip}%
\pgfsetrectcap%
\pgfsetroundjoin%
\pgfsetlinewidth{1.505625pt}%
\definecolor{currentstroke}{rgb}{1.000000,0.000000,0.000000}%
\pgfsetstrokecolor{currentstroke}%
\pgfsetdash{}{0pt}%
\pgfpathmoveto{\pgfqpoint{0.918050in}{1.639093in}}%
\pgfpathlineto{\pgfqpoint{1.075451in}{1.406383in}}%
\pgfusepath{stroke}%
\end{pgfscope}%
\begin{pgfscope}%
\pgfpathrectangle{\pgfqpoint{0.100000in}{0.212622in}}{\pgfqpoint{3.696000in}{3.696000in}}%
\pgfusepath{clip}%
\pgfsetrectcap%
\pgfsetroundjoin%
\pgfsetlinewidth{1.505625pt}%
\definecolor{currentstroke}{rgb}{1.000000,0.000000,0.000000}%
\pgfsetstrokecolor{currentstroke}%
\pgfsetdash{}{0pt}%
\pgfpathmoveto{\pgfqpoint{0.921545in}{1.643715in}}%
\pgfpathlineto{\pgfqpoint{1.075451in}{1.406383in}}%
\pgfusepath{stroke}%
\end{pgfscope}%
\begin{pgfscope}%
\pgfpathrectangle{\pgfqpoint{0.100000in}{0.212622in}}{\pgfqpoint{3.696000in}{3.696000in}}%
\pgfusepath{clip}%
\pgfsetrectcap%
\pgfsetroundjoin%
\pgfsetlinewidth{1.505625pt}%
\definecolor{currentstroke}{rgb}{1.000000,0.000000,0.000000}%
\pgfsetstrokecolor{currentstroke}%
\pgfsetdash{}{0pt}%
\pgfpathmoveto{\pgfqpoint{0.925302in}{1.648786in}}%
\pgfpathlineto{\pgfqpoint{1.075451in}{1.406383in}}%
\pgfusepath{stroke}%
\end{pgfscope}%
\begin{pgfscope}%
\pgfpathrectangle{\pgfqpoint{0.100000in}{0.212622in}}{\pgfqpoint{3.696000in}{3.696000in}}%
\pgfusepath{clip}%
\pgfsetrectcap%
\pgfsetroundjoin%
\pgfsetlinewidth{1.505625pt}%
\definecolor{currentstroke}{rgb}{1.000000,0.000000,0.000000}%
\pgfsetstrokecolor{currentstroke}%
\pgfsetdash{}{0pt}%
\pgfpathmoveto{\pgfqpoint{0.927372in}{1.651427in}}%
\pgfpathlineto{\pgfqpoint{1.084167in}{1.413874in}}%
\pgfusepath{stroke}%
\end{pgfscope}%
\begin{pgfscope}%
\pgfpathrectangle{\pgfqpoint{0.100000in}{0.212622in}}{\pgfqpoint{3.696000in}{3.696000in}}%
\pgfusepath{clip}%
\pgfsetrectcap%
\pgfsetroundjoin%
\pgfsetlinewidth{1.505625pt}%
\definecolor{currentstroke}{rgb}{1.000000,0.000000,0.000000}%
\pgfsetstrokecolor{currentstroke}%
\pgfsetdash{}{0pt}%
\pgfpathmoveto{\pgfqpoint{0.928492in}{1.652982in}}%
\pgfpathlineto{\pgfqpoint{1.084167in}{1.413874in}}%
\pgfusepath{stroke}%
\end{pgfscope}%
\begin{pgfscope}%
\pgfpathrectangle{\pgfqpoint{0.100000in}{0.212622in}}{\pgfqpoint{3.696000in}{3.696000in}}%
\pgfusepath{clip}%
\pgfsetrectcap%
\pgfsetroundjoin%
\pgfsetlinewidth{1.505625pt}%
\definecolor{currentstroke}{rgb}{1.000000,0.000000,0.000000}%
\pgfsetstrokecolor{currentstroke}%
\pgfsetdash{}{0pt}%
\pgfpathmoveto{\pgfqpoint{0.929865in}{1.654871in}}%
\pgfpathlineto{\pgfqpoint{1.084167in}{1.413874in}}%
\pgfusepath{stroke}%
\end{pgfscope}%
\begin{pgfscope}%
\pgfpathrectangle{\pgfqpoint{0.100000in}{0.212622in}}{\pgfqpoint{3.696000in}{3.696000in}}%
\pgfusepath{clip}%
\pgfsetrectcap%
\pgfsetroundjoin%
\pgfsetlinewidth{1.505625pt}%
\definecolor{currentstroke}{rgb}{1.000000,0.000000,0.000000}%
\pgfsetstrokecolor{currentstroke}%
\pgfsetdash{}{0pt}%
\pgfpathmoveto{\pgfqpoint{0.931629in}{1.657372in}}%
\pgfpathlineto{\pgfqpoint{1.084167in}{1.413874in}}%
\pgfusepath{stroke}%
\end{pgfscope}%
\begin{pgfscope}%
\pgfpathrectangle{\pgfqpoint{0.100000in}{0.212622in}}{\pgfqpoint{3.696000in}{3.696000in}}%
\pgfusepath{clip}%
\pgfsetrectcap%
\pgfsetroundjoin%
\pgfsetlinewidth{1.505625pt}%
\definecolor{currentstroke}{rgb}{1.000000,0.000000,0.000000}%
\pgfsetstrokecolor{currentstroke}%
\pgfsetdash{}{0pt}%
\pgfpathmoveto{\pgfqpoint{0.934045in}{1.660859in}}%
\pgfpathlineto{\pgfqpoint{1.092872in}{1.421355in}}%
\pgfusepath{stroke}%
\end{pgfscope}%
\begin{pgfscope}%
\pgfpathrectangle{\pgfqpoint{0.100000in}{0.212622in}}{\pgfqpoint{3.696000in}{3.696000in}}%
\pgfusepath{clip}%
\pgfsetrectcap%
\pgfsetroundjoin%
\pgfsetlinewidth{1.505625pt}%
\definecolor{currentstroke}{rgb}{1.000000,0.000000,0.000000}%
\pgfsetstrokecolor{currentstroke}%
\pgfsetdash{}{0pt}%
\pgfpathmoveto{\pgfqpoint{0.937769in}{1.666335in}}%
\pgfpathlineto{\pgfqpoint{1.092872in}{1.421355in}}%
\pgfusepath{stroke}%
\end{pgfscope}%
\begin{pgfscope}%
\pgfpathrectangle{\pgfqpoint{0.100000in}{0.212622in}}{\pgfqpoint{3.696000in}{3.696000in}}%
\pgfusepath{clip}%
\pgfsetrectcap%
\pgfsetroundjoin%
\pgfsetlinewidth{1.505625pt}%
\definecolor{currentstroke}{rgb}{1.000000,0.000000,0.000000}%
\pgfsetstrokecolor{currentstroke}%
\pgfsetdash{}{0pt}%
\pgfpathmoveto{\pgfqpoint{0.942501in}{1.673383in}}%
\pgfpathlineto{\pgfqpoint{1.101567in}{1.428827in}}%
\pgfusepath{stroke}%
\end{pgfscope}%
\begin{pgfscope}%
\pgfpathrectangle{\pgfqpoint{0.100000in}{0.212622in}}{\pgfqpoint{3.696000in}{3.696000in}}%
\pgfusepath{clip}%
\pgfsetrectcap%
\pgfsetroundjoin%
\pgfsetlinewidth{1.505625pt}%
\definecolor{currentstroke}{rgb}{1.000000,0.000000,0.000000}%
\pgfsetstrokecolor{currentstroke}%
\pgfsetdash{}{0pt}%
\pgfpathmoveto{\pgfqpoint{0.948230in}{1.681587in}}%
\pgfpathlineto{\pgfqpoint{1.101567in}{1.428827in}}%
\pgfusepath{stroke}%
\end{pgfscope}%
\begin{pgfscope}%
\pgfpathrectangle{\pgfqpoint{0.100000in}{0.212622in}}{\pgfqpoint{3.696000in}{3.696000in}}%
\pgfusepath{clip}%
\pgfsetrectcap%
\pgfsetroundjoin%
\pgfsetlinewidth{1.505625pt}%
\definecolor{currentstroke}{rgb}{1.000000,0.000000,0.000000}%
\pgfsetstrokecolor{currentstroke}%
\pgfsetdash{}{0pt}%
\pgfpathmoveto{\pgfqpoint{0.954672in}{1.691737in}}%
\pgfpathlineto{\pgfqpoint{1.110251in}{1.436290in}}%
\pgfusepath{stroke}%
\end{pgfscope}%
\begin{pgfscope}%
\pgfpathrectangle{\pgfqpoint{0.100000in}{0.212622in}}{\pgfqpoint{3.696000in}{3.696000in}}%
\pgfusepath{clip}%
\pgfsetrectcap%
\pgfsetroundjoin%
\pgfsetlinewidth{1.505625pt}%
\definecolor{currentstroke}{rgb}{1.000000,0.000000,0.000000}%
\pgfsetstrokecolor{currentstroke}%
\pgfsetdash{}{0pt}%
\pgfpathmoveto{\pgfqpoint{0.962198in}{1.702156in}}%
\pgfpathlineto{\pgfqpoint{1.118925in}{1.443743in}}%
\pgfusepath{stroke}%
\end{pgfscope}%
\begin{pgfscope}%
\pgfpathrectangle{\pgfqpoint{0.100000in}{0.212622in}}{\pgfqpoint{3.696000in}{3.696000in}}%
\pgfusepath{clip}%
\pgfsetrectcap%
\pgfsetroundjoin%
\pgfsetlinewidth{1.505625pt}%
\definecolor{currentstroke}{rgb}{1.000000,0.000000,0.000000}%
\pgfsetstrokecolor{currentstroke}%
\pgfsetdash{}{0pt}%
\pgfpathmoveto{\pgfqpoint{0.966162in}{1.707939in}}%
\pgfpathlineto{\pgfqpoint{1.127588in}{1.451188in}}%
\pgfusepath{stroke}%
\end{pgfscope}%
\begin{pgfscope}%
\pgfpathrectangle{\pgfqpoint{0.100000in}{0.212622in}}{\pgfqpoint{3.696000in}{3.696000in}}%
\pgfusepath{clip}%
\pgfsetrectcap%
\pgfsetroundjoin%
\pgfsetlinewidth{1.505625pt}%
\definecolor{currentstroke}{rgb}{1.000000,0.000000,0.000000}%
\pgfsetstrokecolor{currentstroke}%
\pgfsetdash{}{0pt}%
\pgfpathmoveto{\pgfqpoint{0.968530in}{1.710667in}}%
\pgfpathlineto{\pgfqpoint{1.127588in}{1.451188in}}%
\pgfusepath{stroke}%
\end{pgfscope}%
\begin{pgfscope}%
\pgfpathrectangle{\pgfqpoint{0.100000in}{0.212622in}}{\pgfqpoint{3.696000in}{3.696000in}}%
\pgfusepath{clip}%
\pgfsetrectcap%
\pgfsetroundjoin%
\pgfsetlinewidth{1.505625pt}%
\definecolor{currentstroke}{rgb}{1.000000,0.000000,0.000000}%
\pgfsetstrokecolor{currentstroke}%
\pgfsetdash{}{0pt}%
\pgfpathmoveto{\pgfqpoint{0.971470in}{1.714105in}}%
\pgfpathlineto{\pgfqpoint{1.127588in}{1.451188in}}%
\pgfusepath{stroke}%
\end{pgfscope}%
\begin{pgfscope}%
\pgfpathrectangle{\pgfqpoint{0.100000in}{0.212622in}}{\pgfqpoint{3.696000in}{3.696000in}}%
\pgfusepath{clip}%
\pgfsetrectcap%
\pgfsetroundjoin%
\pgfsetlinewidth{1.505625pt}%
\definecolor{currentstroke}{rgb}{1.000000,0.000000,0.000000}%
\pgfsetstrokecolor{currentstroke}%
\pgfsetdash{}{0pt}%
\pgfpathmoveto{\pgfqpoint{0.974755in}{1.717805in}}%
\pgfpathlineto{\pgfqpoint{1.136240in}{1.458624in}}%
\pgfusepath{stroke}%
\end{pgfscope}%
\begin{pgfscope}%
\pgfpathrectangle{\pgfqpoint{0.100000in}{0.212622in}}{\pgfqpoint{3.696000in}{3.696000in}}%
\pgfusepath{clip}%
\pgfsetrectcap%
\pgfsetroundjoin%
\pgfsetlinewidth{1.505625pt}%
\definecolor{currentstroke}{rgb}{1.000000,0.000000,0.000000}%
\pgfsetstrokecolor{currentstroke}%
\pgfsetdash{}{0pt}%
\pgfpathmoveto{\pgfqpoint{0.976613in}{1.719957in}}%
\pgfpathlineto{\pgfqpoint{1.136240in}{1.458624in}}%
\pgfusepath{stroke}%
\end{pgfscope}%
\begin{pgfscope}%
\pgfpathrectangle{\pgfqpoint{0.100000in}{0.212622in}}{\pgfqpoint{3.696000in}{3.696000in}}%
\pgfusepath{clip}%
\pgfsetrectcap%
\pgfsetroundjoin%
\pgfsetlinewidth{1.505625pt}%
\definecolor{currentstroke}{rgb}{1.000000,0.000000,0.000000}%
\pgfsetstrokecolor{currentstroke}%
\pgfsetdash{}{0pt}%
\pgfpathmoveto{\pgfqpoint{0.980114in}{1.723928in}}%
\pgfpathlineto{\pgfqpoint{1.136240in}{1.458624in}}%
\pgfusepath{stroke}%
\end{pgfscope}%
\begin{pgfscope}%
\pgfpathrectangle{\pgfqpoint{0.100000in}{0.212622in}}{\pgfqpoint{3.696000in}{3.696000in}}%
\pgfusepath{clip}%
\pgfsetrectcap%
\pgfsetroundjoin%
\pgfsetlinewidth{1.505625pt}%
\definecolor{currentstroke}{rgb}{1.000000,0.000000,0.000000}%
\pgfsetstrokecolor{currentstroke}%
\pgfsetdash{}{0pt}%
\pgfpathmoveto{\pgfqpoint{0.984734in}{1.729305in}}%
\pgfpathlineto{\pgfqpoint{1.144882in}{1.466050in}}%
\pgfusepath{stroke}%
\end{pgfscope}%
\begin{pgfscope}%
\pgfpathrectangle{\pgfqpoint{0.100000in}{0.212622in}}{\pgfqpoint{3.696000in}{3.696000in}}%
\pgfusepath{clip}%
\pgfsetrectcap%
\pgfsetroundjoin%
\pgfsetlinewidth{1.505625pt}%
\definecolor{currentstroke}{rgb}{1.000000,0.000000,0.000000}%
\pgfsetstrokecolor{currentstroke}%
\pgfsetdash{}{0pt}%
\pgfpathmoveto{\pgfqpoint{0.991102in}{1.736867in}}%
\pgfpathlineto{\pgfqpoint{1.144882in}{1.466050in}}%
\pgfusepath{stroke}%
\end{pgfscope}%
\begin{pgfscope}%
\pgfpathrectangle{\pgfqpoint{0.100000in}{0.212622in}}{\pgfqpoint{3.696000in}{3.696000in}}%
\pgfusepath{clip}%
\pgfsetrectcap%
\pgfsetroundjoin%
\pgfsetlinewidth{1.505625pt}%
\definecolor{currentstroke}{rgb}{1.000000,0.000000,0.000000}%
\pgfsetstrokecolor{currentstroke}%
\pgfsetdash{}{0pt}%
\pgfpathmoveto{\pgfqpoint{0.998253in}{1.744614in}}%
\pgfpathlineto{\pgfqpoint{1.153513in}{1.473468in}}%
\pgfusepath{stroke}%
\end{pgfscope}%
\begin{pgfscope}%
\pgfpathrectangle{\pgfqpoint{0.100000in}{0.212622in}}{\pgfqpoint{3.696000in}{3.696000in}}%
\pgfusepath{clip}%
\pgfsetrectcap%
\pgfsetroundjoin%
\pgfsetlinewidth{1.505625pt}%
\definecolor{currentstroke}{rgb}{1.000000,0.000000,0.000000}%
\pgfsetstrokecolor{currentstroke}%
\pgfsetdash{}{0pt}%
\pgfpathmoveto{\pgfqpoint{1.007127in}{1.754211in}}%
\pgfpathlineto{\pgfqpoint{1.162134in}{1.480876in}}%
\pgfusepath{stroke}%
\end{pgfscope}%
\begin{pgfscope}%
\pgfpathrectangle{\pgfqpoint{0.100000in}{0.212622in}}{\pgfqpoint{3.696000in}{3.696000in}}%
\pgfusepath{clip}%
\pgfsetrectcap%
\pgfsetroundjoin%
\pgfsetlinewidth{1.505625pt}%
\definecolor{currentstroke}{rgb}{1.000000,0.000000,0.000000}%
\pgfsetstrokecolor{currentstroke}%
\pgfsetdash{}{0pt}%
\pgfpathmoveto{\pgfqpoint{1.012288in}{1.759057in}}%
\pgfpathlineto{\pgfqpoint{1.170744in}{1.488276in}}%
\pgfusepath{stroke}%
\end{pgfscope}%
\begin{pgfscope}%
\pgfpathrectangle{\pgfqpoint{0.100000in}{0.212622in}}{\pgfqpoint{3.696000in}{3.696000in}}%
\pgfusepath{clip}%
\pgfsetrectcap%
\pgfsetroundjoin%
\pgfsetlinewidth{1.505625pt}%
\definecolor{currentstroke}{rgb}{1.000000,0.000000,0.000000}%
\pgfsetstrokecolor{currentstroke}%
\pgfsetdash{}{0pt}%
\pgfpathmoveto{\pgfqpoint{1.018260in}{1.765350in}}%
\pgfpathlineto{\pgfqpoint{1.170744in}{1.488276in}}%
\pgfusepath{stroke}%
\end{pgfscope}%
\begin{pgfscope}%
\pgfpathrectangle{\pgfqpoint{0.100000in}{0.212622in}}{\pgfqpoint{3.696000in}{3.696000in}}%
\pgfusepath{clip}%
\pgfsetrectcap%
\pgfsetroundjoin%
\pgfsetlinewidth{1.505625pt}%
\definecolor{currentstroke}{rgb}{1.000000,0.000000,0.000000}%
\pgfsetstrokecolor{currentstroke}%
\pgfsetdash{}{0pt}%
\pgfpathmoveto{\pgfqpoint{1.021411in}{1.768420in}}%
\pgfpathlineto{\pgfqpoint{1.179344in}{1.495666in}}%
\pgfusepath{stroke}%
\end{pgfscope}%
\begin{pgfscope}%
\pgfpathrectangle{\pgfqpoint{0.100000in}{0.212622in}}{\pgfqpoint{3.696000in}{3.696000in}}%
\pgfusepath{clip}%
\pgfsetrectcap%
\pgfsetroundjoin%
\pgfsetlinewidth{1.505625pt}%
\definecolor{currentstroke}{rgb}{1.000000,0.000000,0.000000}%
\pgfsetstrokecolor{currentstroke}%
\pgfsetdash{}{0pt}%
\pgfpathmoveto{\pgfqpoint{1.023176in}{1.770222in}}%
\pgfpathlineto{\pgfqpoint{1.179344in}{1.495666in}}%
\pgfusepath{stroke}%
\end{pgfscope}%
\begin{pgfscope}%
\pgfpathrectangle{\pgfqpoint{0.100000in}{0.212622in}}{\pgfqpoint{3.696000in}{3.696000in}}%
\pgfusepath{clip}%
\pgfsetrectcap%
\pgfsetroundjoin%
\pgfsetlinewidth{1.505625pt}%
\definecolor{currentstroke}{rgb}{1.000000,0.000000,0.000000}%
\pgfsetstrokecolor{currentstroke}%
\pgfsetdash{}{0pt}%
\pgfpathmoveto{\pgfqpoint{1.025419in}{1.772262in}}%
\pgfpathlineto{\pgfqpoint{1.179344in}{1.495666in}}%
\pgfusepath{stroke}%
\end{pgfscope}%
\begin{pgfscope}%
\pgfpathrectangle{\pgfqpoint{0.100000in}{0.212622in}}{\pgfqpoint{3.696000in}{3.696000in}}%
\pgfusepath{clip}%
\pgfsetrectcap%
\pgfsetroundjoin%
\pgfsetlinewidth{1.505625pt}%
\definecolor{currentstroke}{rgb}{1.000000,0.000000,0.000000}%
\pgfsetstrokecolor{currentstroke}%
\pgfsetdash{}{0pt}%
\pgfpathmoveto{\pgfqpoint{1.028307in}{1.775029in}}%
\pgfpathlineto{\pgfqpoint{1.187933in}{1.503047in}}%
\pgfusepath{stroke}%
\end{pgfscope}%
\begin{pgfscope}%
\pgfpathrectangle{\pgfqpoint{0.100000in}{0.212622in}}{\pgfqpoint{3.696000in}{3.696000in}}%
\pgfusepath{clip}%
\pgfsetrectcap%
\pgfsetroundjoin%
\pgfsetlinewidth{1.505625pt}%
\definecolor{currentstroke}{rgb}{1.000000,0.000000,0.000000}%
\pgfsetstrokecolor{currentstroke}%
\pgfsetdash{}{0pt}%
\pgfpathmoveto{\pgfqpoint{1.029876in}{1.776517in}}%
\pgfpathlineto{\pgfqpoint{1.187933in}{1.503047in}}%
\pgfusepath{stroke}%
\end{pgfscope}%
\begin{pgfscope}%
\pgfpathrectangle{\pgfqpoint{0.100000in}{0.212622in}}{\pgfqpoint{3.696000in}{3.696000in}}%
\pgfusepath{clip}%
\pgfsetrectcap%
\pgfsetroundjoin%
\pgfsetlinewidth{1.505625pt}%
\definecolor{currentstroke}{rgb}{1.000000,0.000000,0.000000}%
\pgfsetstrokecolor{currentstroke}%
\pgfsetdash{}{0pt}%
\pgfpathmoveto{\pgfqpoint{1.032266in}{1.778872in}}%
\pgfpathlineto{\pgfqpoint{1.187933in}{1.503047in}}%
\pgfusepath{stroke}%
\end{pgfscope}%
\begin{pgfscope}%
\pgfpathrectangle{\pgfqpoint{0.100000in}{0.212622in}}{\pgfqpoint{3.696000in}{3.696000in}}%
\pgfusepath{clip}%
\pgfsetrectcap%
\pgfsetroundjoin%
\pgfsetlinewidth{1.505625pt}%
\definecolor{currentstroke}{rgb}{1.000000,0.000000,0.000000}%
\pgfsetstrokecolor{currentstroke}%
\pgfsetdash{}{0pt}%
\pgfpathmoveto{\pgfqpoint{1.035343in}{1.781911in}}%
\pgfpathlineto{\pgfqpoint{1.187933in}{1.503047in}}%
\pgfusepath{stroke}%
\end{pgfscope}%
\begin{pgfscope}%
\pgfpathrectangle{\pgfqpoint{0.100000in}{0.212622in}}{\pgfqpoint{3.696000in}{3.696000in}}%
\pgfusepath{clip}%
\pgfsetrectcap%
\pgfsetroundjoin%
\pgfsetlinewidth{1.505625pt}%
\definecolor{currentstroke}{rgb}{1.000000,0.000000,0.000000}%
\pgfsetstrokecolor{currentstroke}%
\pgfsetdash{}{0pt}%
\pgfpathmoveto{\pgfqpoint{1.039235in}{1.785936in}}%
\pgfpathlineto{\pgfqpoint{1.196512in}{1.510420in}}%
\pgfusepath{stroke}%
\end{pgfscope}%
\begin{pgfscope}%
\pgfpathrectangle{\pgfqpoint{0.100000in}{0.212622in}}{\pgfqpoint{3.696000in}{3.696000in}}%
\pgfusepath{clip}%
\pgfsetrectcap%
\pgfsetroundjoin%
\pgfsetlinewidth{1.505625pt}%
\definecolor{currentstroke}{rgb}{1.000000,0.000000,0.000000}%
\pgfsetstrokecolor{currentstroke}%
\pgfsetdash{}{0pt}%
\pgfpathmoveto{\pgfqpoint{1.044234in}{1.791217in}}%
\pgfpathlineto{\pgfqpoint{1.196512in}{1.510420in}}%
\pgfusepath{stroke}%
\end{pgfscope}%
\begin{pgfscope}%
\pgfpathrectangle{\pgfqpoint{0.100000in}{0.212622in}}{\pgfqpoint{3.696000in}{3.696000in}}%
\pgfusepath{clip}%
\pgfsetrectcap%
\pgfsetroundjoin%
\pgfsetlinewidth{1.505625pt}%
\definecolor{currentstroke}{rgb}{1.000000,0.000000,0.000000}%
\pgfsetstrokecolor{currentstroke}%
\pgfsetdash{}{0pt}%
\pgfpathmoveto{\pgfqpoint{1.051224in}{1.798473in}}%
\pgfpathlineto{\pgfqpoint{1.205081in}{1.517784in}}%
\pgfusepath{stroke}%
\end{pgfscope}%
\begin{pgfscope}%
\pgfpathrectangle{\pgfqpoint{0.100000in}{0.212622in}}{\pgfqpoint{3.696000in}{3.696000in}}%
\pgfusepath{clip}%
\pgfsetrectcap%
\pgfsetroundjoin%
\pgfsetlinewidth{1.505625pt}%
\definecolor{currentstroke}{rgb}{1.000000,0.000000,0.000000}%
\pgfsetstrokecolor{currentstroke}%
\pgfsetdash{}{0pt}%
\pgfpathmoveto{\pgfqpoint{1.059569in}{1.807498in}}%
\pgfpathlineto{\pgfqpoint{1.213639in}{1.525138in}}%
\pgfusepath{stroke}%
\end{pgfscope}%
\begin{pgfscope}%
\pgfpathrectangle{\pgfqpoint{0.100000in}{0.212622in}}{\pgfqpoint{3.696000in}{3.696000in}}%
\pgfusepath{clip}%
\pgfsetrectcap%
\pgfsetroundjoin%
\pgfsetlinewidth{1.505625pt}%
\definecolor{currentstroke}{rgb}{1.000000,0.000000,0.000000}%
\pgfsetstrokecolor{currentstroke}%
\pgfsetdash{}{0pt}%
\pgfpathmoveto{\pgfqpoint{1.068948in}{1.817218in}}%
\pgfpathlineto{\pgfqpoint{1.222187in}{1.532484in}}%
\pgfusepath{stroke}%
\end{pgfscope}%
\begin{pgfscope}%
\pgfpathrectangle{\pgfqpoint{0.100000in}{0.212622in}}{\pgfqpoint{3.696000in}{3.696000in}}%
\pgfusepath{clip}%
\pgfsetrectcap%
\pgfsetroundjoin%
\pgfsetlinewidth{1.505625pt}%
\definecolor{currentstroke}{rgb}{1.000000,0.000000,0.000000}%
\pgfsetstrokecolor{currentstroke}%
\pgfsetdash{}{0pt}%
\pgfpathmoveto{\pgfqpoint{1.078630in}{1.830308in}}%
\pgfpathlineto{\pgfqpoint{1.230724in}{1.539821in}}%
\pgfusepath{stroke}%
\end{pgfscope}%
\begin{pgfscope}%
\pgfpathrectangle{\pgfqpoint{0.100000in}{0.212622in}}{\pgfqpoint{3.696000in}{3.696000in}}%
\pgfusepath{clip}%
\pgfsetrectcap%
\pgfsetroundjoin%
\pgfsetlinewidth{1.505625pt}%
\definecolor{currentstroke}{rgb}{1.000000,0.000000,0.000000}%
\pgfsetstrokecolor{currentstroke}%
\pgfsetdash{}{0pt}%
\pgfpathmoveto{\pgfqpoint{1.088590in}{1.844187in}}%
\pgfpathlineto{\pgfqpoint{1.247768in}{1.554468in}}%
\pgfusepath{stroke}%
\end{pgfscope}%
\begin{pgfscope}%
\pgfpathrectangle{\pgfqpoint{0.100000in}{0.212622in}}{\pgfqpoint{3.696000in}{3.696000in}}%
\pgfusepath{clip}%
\pgfsetrectcap%
\pgfsetroundjoin%
\pgfsetlinewidth{1.505625pt}%
\definecolor{currentstroke}{rgb}{1.000000,0.000000,0.000000}%
\pgfsetstrokecolor{currentstroke}%
\pgfsetdash{}{0pt}%
\pgfpathmoveto{\pgfqpoint{1.099068in}{1.858926in}}%
\pgfpathlineto{\pgfqpoint{1.256274in}{1.561778in}}%
\pgfusepath{stroke}%
\end{pgfscope}%
\begin{pgfscope}%
\pgfpathrectangle{\pgfqpoint{0.100000in}{0.212622in}}{\pgfqpoint{3.696000in}{3.696000in}}%
\pgfusepath{clip}%
\pgfsetrectcap%
\pgfsetroundjoin%
\pgfsetlinewidth{1.505625pt}%
\definecolor{currentstroke}{rgb}{1.000000,0.000000,0.000000}%
\pgfsetstrokecolor{currentstroke}%
\pgfsetdash{}{0pt}%
\pgfpathmoveto{\pgfqpoint{1.109595in}{1.873521in}}%
\pgfpathlineto{\pgfqpoint{1.264770in}{1.569079in}}%
\pgfusepath{stroke}%
\end{pgfscope}%
\begin{pgfscope}%
\pgfpathrectangle{\pgfqpoint{0.100000in}{0.212622in}}{\pgfqpoint{3.696000in}{3.696000in}}%
\pgfusepath{clip}%
\pgfsetrectcap%
\pgfsetroundjoin%
\pgfsetlinewidth{1.505625pt}%
\definecolor{currentstroke}{rgb}{1.000000,0.000000,0.000000}%
\pgfsetstrokecolor{currentstroke}%
\pgfsetdash{}{0pt}%
\pgfpathmoveto{\pgfqpoint{1.121500in}{1.890286in}}%
\pgfpathlineto{\pgfqpoint{1.281732in}{1.583656in}}%
\pgfusepath{stroke}%
\end{pgfscope}%
\begin{pgfscope}%
\pgfpathrectangle{\pgfqpoint{0.100000in}{0.212622in}}{\pgfqpoint{3.696000in}{3.696000in}}%
\pgfusepath{clip}%
\pgfsetrectcap%
\pgfsetroundjoin%
\pgfsetlinewidth{1.505625pt}%
\definecolor{currentstroke}{rgb}{1.000000,0.000000,0.000000}%
\pgfsetstrokecolor{currentstroke}%
\pgfsetdash{}{0pt}%
\pgfpathmoveto{\pgfqpoint{1.133602in}{1.907169in}}%
\pgfpathlineto{\pgfqpoint{1.290197in}{1.590931in}}%
\pgfusepath{stroke}%
\end{pgfscope}%
\begin{pgfscope}%
\pgfpathrectangle{\pgfqpoint{0.100000in}{0.212622in}}{\pgfqpoint{3.696000in}{3.696000in}}%
\pgfusepath{clip}%
\pgfsetrectcap%
\pgfsetroundjoin%
\pgfsetlinewidth{1.505625pt}%
\definecolor{currentstroke}{rgb}{1.000000,0.000000,0.000000}%
\pgfsetstrokecolor{currentstroke}%
\pgfsetdash{}{0pt}%
\pgfpathmoveto{\pgfqpoint{1.147391in}{1.925786in}}%
\pgfpathlineto{\pgfqpoint{1.307098in}{1.605454in}}%
\pgfusepath{stroke}%
\end{pgfscope}%
\begin{pgfscope}%
\pgfpathrectangle{\pgfqpoint{0.100000in}{0.212622in}}{\pgfqpoint{3.696000in}{3.696000in}}%
\pgfusepath{clip}%
\pgfsetrectcap%
\pgfsetroundjoin%
\pgfsetlinewidth{1.505625pt}%
\definecolor{currentstroke}{rgb}{1.000000,0.000000,0.000000}%
\pgfsetstrokecolor{currentstroke}%
\pgfsetdash{}{0pt}%
\pgfpathmoveto{\pgfqpoint{1.161900in}{1.945838in}}%
\pgfpathlineto{\pgfqpoint{1.323957in}{1.619943in}}%
\pgfusepath{stroke}%
\end{pgfscope}%
\begin{pgfscope}%
\pgfpathrectangle{\pgfqpoint{0.100000in}{0.212622in}}{\pgfqpoint{3.696000in}{3.696000in}}%
\pgfusepath{clip}%
\pgfsetrectcap%
\pgfsetroundjoin%
\pgfsetlinewidth{1.505625pt}%
\definecolor{currentstroke}{rgb}{1.000000,0.000000,0.000000}%
\pgfsetstrokecolor{currentstroke}%
\pgfsetdash{}{0pt}%
\pgfpathmoveto{\pgfqpoint{1.178250in}{1.967197in}}%
\pgfpathlineto{\pgfqpoint{1.340776in}{1.634397in}}%
\pgfusepath{stroke}%
\end{pgfscope}%
\begin{pgfscope}%
\pgfpathrectangle{\pgfqpoint{0.100000in}{0.212622in}}{\pgfqpoint{3.696000in}{3.696000in}}%
\pgfusepath{clip}%
\pgfsetrectcap%
\pgfsetroundjoin%
\pgfsetlinewidth{1.505625pt}%
\definecolor{currentstroke}{rgb}{1.000000,0.000000,0.000000}%
\pgfsetstrokecolor{currentstroke}%
\pgfsetdash{}{0pt}%
\pgfpathmoveto{\pgfqpoint{1.195256in}{1.988468in}}%
\pgfpathlineto{\pgfqpoint{1.357555in}{1.648816in}}%
\pgfusepath{stroke}%
\end{pgfscope}%
\begin{pgfscope}%
\pgfpathrectangle{\pgfqpoint{0.100000in}{0.212622in}}{\pgfqpoint{3.696000in}{3.696000in}}%
\pgfusepath{clip}%
\pgfsetrectcap%
\pgfsetroundjoin%
\pgfsetlinewidth{1.505625pt}%
\definecolor{currentstroke}{rgb}{1.000000,0.000000,0.000000}%
\pgfsetstrokecolor{currentstroke}%
\pgfsetdash{}{0pt}%
\pgfpathmoveto{\pgfqpoint{1.205128in}{2.000697in}}%
\pgfpathlineto{\pgfqpoint{1.365929in}{1.656013in}}%
\pgfusepath{stroke}%
\end{pgfscope}%
\begin{pgfscope}%
\pgfpathrectangle{\pgfqpoint{0.100000in}{0.212622in}}{\pgfqpoint{3.696000in}{3.696000in}}%
\pgfusepath{clip}%
\pgfsetrectcap%
\pgfsetroundjoin%
\pgfsetlinewidth{1.505625pt}%
\definecolor{currentstroke}{rgb}{1.000000,0.000000,0.000000}%
\pgfsetstrokecolor{currentstroke}%
\pgfsetdash{}{0pt}%
\pgfpathmoveto{\pgfqpoint{1.215751in}{2.013540in}}%
\pgfpathlineto{\pgfqpoint{1.374293in}{1.663201in}}%
\pgfusepath{stroke}%
\end{pgfscope}%
\begin{pgfscope}%
\pgfpathrectangle{\pgfqpoint{0.100000in}{0.212622in}}{\pgfqpoint{3.696000in}{3.696000in}}%
\pgfusepath{clip}%
\pgfsetrectcap%
\pgfsetroundjoin%
\pgfsetlinewidth{1.505625pt}%
\definecolor{currentstroke}{rgb}{1.000000,0.000000,0.000000}%
\pgfsetstrokecolor{currentstroke}%
\pgfsetdash{}{0pt}%
\pgfpathmoveto{\pgfqpoint{1.221469in}{2.020464in}}%
\pgfpathlineto{\pgfqpoint{1.382647in}{1.670380in}}%
\pgfusepath{stroke}%
\end{pgfscope}%
\begin{pgfscope}%
\pgfpathrectangle{\pgfqpoint{0.100000in}{0.212622in}}{\pgfqpoint{3.696000in}{3.696000in}}%
\pgfusepath{clip}%
\pgfsetrectcap%
\pgfsetroundjoin%
\pgfsetlinewidth{1.505625pt}%
\definecolor{currentstroke}{rgb}{1.000000,0.000000,0.000000}%
\pgfsetstrokecolor{currentstroke}%
\pgfsetdash{}{0pt}%
\pgfpathmoveto{\pgfqpoint{1.227780in}{2.028188in}}%
\pgfpathlineto{\pgfqpoint{1.390992in}{1.677551in}}%
\pgfusepath{stroke}%
\end{pgfscope}%
\begin{pgfscope}%
\pgfpathrectangle{\pgfqpoint{0.100000in}{0.212622in}}{\pgfqpoint{3.696000in}{3.696000in}}%
\pgfusepath{clip}%
\pgfsetrectcap%
\pgfsetroundjoin%
\pgfsetlinewidth{1.505625pt}%
\definecolor{currentstroke}{rgb}{1.000000,0.000000,0.000000}%
\pgfsetstrokecolor{currentstroke}%
\pgfsetdash{}{0pt}%
\pgfpathmoveto{\pgfqpoint{1.234507in}{2.036310in}}%
\pgfpathlineto{\pgfqpoint{1.390992in}{1.677551in}}%
\pgfusepath{stroke}%
\end{pgfscope}%
\begin{pgfscope}%
\pgfpathrectangle{\pgfqpoint{0.100000in}{0.212622in}}{\pgfqpoint{3.696000in}{3.696000in}}%
\pgfusepath{clip}%
\pgfsetrectcap%
\pgfsetroundjoin%
\pgfsetlinewidth{1.505625pt}%
\definecolor{currentstroke}{rgb}{1.000000,0.000000,0.000000}%
\pgfsetstrokecolor{currentstroke}%
\pgfsetdash{}{0pt}%
\pgfpathmoveto{\pgfqpoint{1.242648in}{2.046048in}}%
\pgfpathlineto{\pgfqpoint{1.399326in}{1.684713in}}%
\pgfusepath{stroke}%
\end{pgfscope}%
\begin{pgfscope}%
\pgfpathrectangle{\pgfqpoint{0.100000in}{0.212622in}}{\pgfqpoint{3.696000in}{3.696000in}}%
\pgfusepath{clip}%
\pgfsetrectcap%
\pgfsetroundjoin%
\pgfsetlinewidth{1.505625pt}%
\definecolor{currentstroke}{rgb}{1.000000,0.000000,0.000000}%
\pgfsetstrokecolor{currentstroke}%
\pgfsetdash{}{0pt}%
\pgfpathmoveto{\pgfqpoint{1.251760in}{2.056829in}}%
\pgfpathlineto{\pgfqpoint{1.407650in}{1.691866in}}%
\pgfusepath{stroke}%
\end{pgfscope}%
\begin{pgfscope}%
\pgfpathrectangle{\pgfqpoint{0.100000in}{0.212622in}}{\pgfqpoint{3.696000in}{3.696000in}}%
\pgfusepath{clip}%
\pgfsetrectcap%
\pgfsetroundjoin%
\pgfsetlinewidth{1.505625pt}%
\definecolor{currentstroke}{rgb}{1.000000,0.000000,0.000000}%
\pgfsetstrokecolor{currentstroke}%
\pgfsetdash{}{0pt}%
\pgfpathmoveto{\pgfqpoint{1.261908in}{2.068509in}}%
\pgfpathlineto{\pgfqpoint{1.424269in}{1.706148in}}%
\pgfusepath{stroke}%
\end{pgfscope}%
\begin{pgfscope}%
\pgfpathrectangle{\pgfqpoint{0.100000in}{0.212622in}}{\pgfqpoint{3.696000in}{3.696000in}}%
\pgfusepath{clip}%
\pgfsetrectcap%
\pgfsetroundjoin%
\pgfsetlinewidth{1.505625pt}%
\definecolor{currentstroke}{rgb}{1.000000,0.000000,0.000000}%
\pgfsetstrokecolor{currentstroke}%
\pgfsetdash{}{0pt}%
\pgfpathmoveto{\pgfqpoint{1.272185in}{2.080233in}}%
\pgfpathlineto{\pgfqpoint{1.432563in}{1.713276in}}%
\pgfusepath{stroke}%
\end{pgfscope}%
\begin{pgfscope}%
\pgfpathrectangle{\pgfqpoint{0.100000in}{0.212622in}}{\pgfqpoint{3.696000in}{3.696000in}}%
\pgfusepath{clip}%
\pgfsetrectcap%
\pgfsetroundjoin%
\pgfsetlinewidth{1.505625pt}%
\definecolor{currentstroke}{rgb}{1.000000,0.000000,0.000000}%
\pgfsetstrokecolor{currentstroke}%
\pgfsetdash{}{0pt}%
\pgfpathmoveto{\pgfqpoint{1.282669in}{2.092750in}}%
\pgfpathlineto{\pgfqpoint{1.440848in}{1.720396in}}%
\pgfusepath{stroke}%
\end{pgfscope}%
\begin{pgfscope}%
\pgfpathrectangle{\pgfqpoint{0.100000in}{0.212622in}}{\pgfqpoint{3.696000in}{3.696000in}}%
\pgfusepath{clip}%
\pgfsetrectcap%
\pgfsetroundjoin%
\pgfsetlinewidth{1.505625pt}%
\definecolor{currentstroke}{rgb}{1.000000,0.000000,0.000000}%
\pgfsetstrokecolor{currentstroke}%
\pgfsetdash{}{0pt}%
\pgfpathmoveto{\pgfqpoint{1.293735in}{2.106196in}}%
\pgfpathlineto{\pgfqpoint{1.457387in}{1.734609in}}%
\pgfusepath{stroke}%
\end{pgfscope}%
\begin{pgfscope}%
\pgfpathrectangle{\pgfqpoint{0.100000in}{0.212622in}}{\pgfqpoint{3.696000in}{3.696000in}}%
\pgfusepath{clip}%
\pgfsetrectcap%
\pgfsetroundjoin%
\pgfsetlinewidth{1.505625pt}%
\definecolor{currentstroke}{rgb}{1.000000,0.000000,0.000000}%
\pgfsetstrokecolor{currentstroke}%
\pgfsetdash{}{0pt}%
\pgfpathmoveto{\pgfqpoint{1.299705in}{2.113340in}}%
\pgfpathlineto{\pgfqpoint{1.457387in}{1.734609in}}%
\pgfusepath{stroke}%
\end{pgfscope}%
\begin{pgfscope}%
\pgfpathrectangle{\pgfqpoint{0.100000in}{0.212622in}}{\pgfqpoint{3.696000in}{3.696000in}}%
\pgfusepath{clip}%
\pgfsetrectcap%
\pgfsetroundjoin%
\pgfsetlinewidth{1.505625pt}%
\definecolor{currentstroke}{rgb}{1.000000,0.000000,0.000000}%
\pgfsetstrokecolor{currentstroke}%
\pgfsetdash{}{0pt}%
\pgfpathmoveto{\pgfqpoint{1.303081in}{2.117316in}}%
\pgfpathlineto{\pgfqpoint{1.465642in}{1.741703in}}%
\pgfusepath{stroke}%
\end{pgfscope}%
\begin{pgfscope}%
\pgfpathrectangle{\pgfqpoint{0.100000in}{0.212622in}}{\pgfqpoint{3.696000in}{3.696000in}}%
\pgfusepath{clip}%
\pgfsetrectcap%
\pgfsetroundjoin%
\pgfsetlinewidth{1.505625pt}%
\definecolor{currentstroke}{rgb}{1.000000,0.000000,0.000000}%
\pgfsetstrokecolor{currentstroke}%
\pgfsetdash{}{0pt}%
\pgfpathmoveto{\pgfqpoint{1.307192in}{2.122366in}}%
\pgfpathlineto{\pgfqpoint{1.465642in}{1.741703in}}%
\pgfusepath{stroke}%
\end{pgfscope}%
\begin{pgfscope}%
\pgfpathrectangle{\pgfqpoint{0.100000in}{0.212622in}}{\pgfqpoint{3.696000in}{3.696000in}}%
\pgfusepath{clip}%
\pgfsetrectcap%
\pgfsetroundjoin%
\pgfsetlinewidth{1.505625pt}%
\definecolor{currentstroke}{rgb}{1.000000,0.000000,0.000000}%
\pgfsetstrokecolor{currentstroke}%
\pgfsetdash{}{0pt}%
\pgfpathmoveto{\pgfqpoint{1.312187in}{2.128493in}}%
\pgfpathlineto{\pgfqpoint{1.473887in}{1.748789in}}%
\pgfusepath{stroke}%
\end{pgfscope}%
\begin{pgfscope}%
\pgfpathrectangle{\pgfqpoint{0.100000in}{0.212622in}}{\pgfqpoint{3.696000in}{3.696000in}}%
\pgfusepath{clip}%
\pgfsetrectcap%
\pgfsetroundjoin%
\pgfsetlinewidth{1.505625pt}%
\definecolor{currentstroke}{rgb}{1.000000,0.000000,0.000000}%
\pgfsetstrokecolor{currentstroke}%
\pgfsetdash{}{0pt}%
\pgfpathmoveto{\pgfqpoint{1.318408in}{2.136394in}}%
\pgfpathlineto{\pgfqpoint{1.482122in}{1.755866in}}%
\pgfusepath{stroke}%
\end{pgfscope}%
\begin{pgfscope}%
\pgfpathrectangle{\pgfqpoint{0.100000in}{0.212622in}}{\pgfqpoint{3.696000in}{3.696000in}}%
\pgfusepath{clip}%
\pgfsetrectcap%
\pgfsetroundjoin%
\pgfsetlinewidth{1.505625pt}%
\definecolor{currentstroke}{rgb}{1.000000,0.000000,0.000000}%
\pgfsetstrokecolor{currentstroke}%
\pgfsetdash{}{0pt}%
\pgfpathmoveto{\pgfqpoint{1.325139in}{2.144634in}}%
\pgfpathlineto{\pgfqpoint{1.490348in}{1.762935in}}%
\pgfusepath{stroke}%
\end{pgfscope}%
\begin{pgfscope}%
\pgfpathrectangle{\pgfqpoint{0.100000in}{0.212622in}}{\pgfqpoint{3.696000in}{3.696000in}}%
\pgfusepath{clip}%
\pgfsetrectcap%
\pgfsetroundjoin%
\pgfsetlinewidth{1.505625pt}%
\definecolor{currentstroke}{rgb}{1.000000,0.000000,0.000000}%
\pgfsetstrokecolor{currentstroke}%
\pgfsetdash{}{0pt}%
\pgfpathmoveto{\pgfqpoint{1.331758in}{2.155516in}}%
\pgfpathlineto{\pgfqpoint{1.498563in}{1.769995in}}%
\pgfusepath{stroke}%
\end{pgfscope}%
\begin{pgfscope}%
\pgfpathrectangle{\pgfqpoint{0.100000in}{0.212622in}}{\pgfqpoint{3.696000in}{3.696000in}}%
\pgfusepath{clip}%
\pgfsetrectcap%
\pgfsetroundjoin%
\pgfsetlinewidth{1.505625pt}%
\definecolor{currentstroke}{rgb}{1.000000,0.000000,0.000000}%
\pgfsetstrokecolor{currentstroke}%
\pgfsetdash{}{0pt}%
\pgfpathmoveto{\pgfqpoint{1.338641in}{2.166075in}}%
\pgfpathlineto{\pgfqpoint{1.506769in}{1.777047in}}%
\pgfusepath{stroke}%
\end{pgfscope}%
\begin{pgfscope}%
\pgfpathrectangle{\pgfqpoint{0.100000in}{0.212622in}}{\pgfqpoint{3.696000in}{3.696000in}}%
\pgfusepath{clip}%
\pgfsetrectcap%
\pgfsetroundjoin%
\pgfsetlinewidth{1.505625pt}%
\definecolor{currentstroke}{rgb}{1.000000,0.000000,0.000000}%
\pgfsetstrokecolor{currentstroke}%
\pgfsetdash{}{0pt}%
\pgfpathmoveto{\pgfqpoint{1.346288in}{2.177067in}}%
\pgfpathlineto{\pgfqpoint{1.514966in}{1.784091in}}%
\pgfusepath{stroke}%
\end{pgfscope}%
\begin{pgfscope}%
\pgfpathrectangle{\pgfqpoint{0.100000in}{0.212622in}}{\pgfqpoint{3.696000in}{3.696000in}}%
\pgfusepath{clip}%
\pgfsetrectcap%
\pgfsetroundjoin%
\pgfsetlinewidth{1.505625pt}%
\definecolor{currentstroke}{rgb}{1.000000,0.000000,0.000000}%
\pgfsetstrokecolor{currentstroke}%
\pgfsetdash{}{0pt}%
\pgfpathmoveto{\pgfqpoint{1.354420in}{2.188050in}}%
\pgfpathlineto{\pgfqpoint{1.523152in}{1.791126in}}%
\pgfusepath{stroke}%
\end{pgfscope}%
\begin{pgfscope}%
\pgfpathrectangle{\pgfqpoint{0.100000in}{0.212622in}}{\pgfqpoint{3.696000in}{3.696000in}}%
\pgfusepath{clip}%
\pgfsetrectcap%
\pgfsetroundjoin%
\pgfsetlinewidth{1.505625pt}%
\definecolor{currentstroke}{rgb}{1.000000,0.000000,0.000000}%
\pgfsetstrokecolor{currentstroke}%
\pgfsetdash{}{0pt}%
\pgfpathmoveto{\pgfqpoint{1.363218in}{2.200380in}}%
\pgfpathlineto{\pgfqpoint{1.531329in}{1.798153in}}%
\pgfusepath{stroke}%
\end{pgfscope}%
\begin{pgfscope}%
\pgfpathrectangle{\pgfqpoint{0.100000in}{0.212622in}}{\pgfqpoint{3.696000in}{3.696000in}}%
\pgfusepath{clip}%
\pgfsetrectcap%
\pgfsetroundjoin%
\pgfsetlinewidth{1.505625pt}%
\definecolor{currentstroke}{rgb}{1.000000,0.000000,0.000000}%
\pgfsetstrokecolor{currentstroke}%
\pgfsetdash{}{0pt}%
\pgfpathmoveto{\pgfqpoint{1.368014in}{2.206958in}}%
\pgfpathlineto{\pgfqpoint{1.539496in}{1.805172in}}%
\pgfusepath{stroke}%
\end{pgfscope}%
\begin{pgfscope}%
\pgfpathrectangle{\pgfqpoint{0.100000in}{0.212622in}}{\pgfqpoint{3.696000in}{3.696000in}}%
\pgfusepath{clip}%
\pgfsetrectcap%
\pgfsetroundjoin%
\pgfsetlinewidth{1.505625pt}%
\definecolor{currentstroke}{rgb}{1.000000,0.000000,0.000000}%
\pgfsetstrokecolor{currentstroke}%
\pgfsetdash{}{0pt}%
\pgfpathmoveto{\pgfqpoint{1.373987in}{2.214969in}}%
\pgfpathlineto{\pgfqpoint{1.539496in}{1.805172in}}%
\pgfusepath{stroke}%
\end{pgfscope}%
\begin{pgfscope}%
\pgfpathrectangle{\pgfqpoint{0.100000in}{0.212622in}}{\pgfqpoint{3.696000in}{3.696000in}}%
\pgfusepath{clip}%
\pgfsetrectcap%
\pgfsetroundjoin%
\pgfsetlinewidth{1.505625pt}%
\definecolor{currentstroke}{rgb}{1.000000,0.000000,0.000000}%
\pgfsetstrokecolor{currentstroke}%
\pgfsetdash{}{0pt}%
\pgfpathmoveto{\pgfqpoint{1.380586in}{2.224126in}}%
\pgfpathlineto{\pgfqpoint{1.547653in}{1.812182in}}%
\pgfusepath{stroke}%
\end{pgfscope}%
\begin{pgfscope}%
\pgfpathrectangle{\pgfqpoint{0.100000in}{0.212622in}}{\pgfqpoint{3.696000in}{3.696000in}}%
\pgfusepath{clip}%
\pgfsetrectcap%
\pgfsetroundjoin%
\pgfsetlinewidth{1.505625pt}%
\definecolor{currentstroke}{rgb}{1.000000,0.000000,0.000000}%
\pgfsetstrokecolor{currentstroke}%
\pgfsetdash{}{0pt}%
\pgfpathmoveto{\pgfqpoint{1.388426in}{2.234455in}}%
\pgfpathlineto{\pgfqpoint{1.555801in}{1.819184in}}%
\pgfusepath{stroke}%
\end{pgfscope}%
\begin{pgfscope}%
\pgfpathrectangle{\pgfqpoint{0.100000in}{0.212622in}}{\pgfqpoint{3.696000in}{3.696000in}}%
\pgfusepath{clip}%
\pgfsetrectcap%
\pgfsetroundjoin%
\pgfsetlinewidth{1.505625pt}%
\definecolor{currentstroke}{rgb}{1.000000,0.000000,0.000000}%
\pgfsetstrokecolor{currentstroke}%
\pgfsetdash{}{0pt}%
\pgfpathmoveto{\pgfqpoint{1.396164in}{2.245221in}}%
\pgfpathlineto{\pgfqpoint{1.563939in}{1.826177in}}%
\pgfusepath{stroke}%
\end{pgfscope}%
\begin{pgfscope}%
\pgfpathrectangle{\pgfqpoint{0.100000in}{0.212622in}}{\pgfqpoint{3.696000in}{3.696000in}}%
\pgfusepath{clip}%
\pgfsetrectcap%
\pgfsetroundjoin%
\pgfsetlinewidth{1.505625pt}%
\definecolor{currentstroke}{rgb}{1.000000,0.000000,0.000000}%
\pgfsetstrokecolor{currentstroke}%
\pgfsetdash{}{0pt}%
\pgfpathmoveto{\pgfqpoint{1.404637in}{2.256639in}}%
\pgfpathlineto{\pgfqpoint{1.572068in}{1.833163in}}%
\pgfusepath{stroke}%
\end{pgfscope}%
\begin{pgfscope}%
\pgfpathrectangle{\pgfqpoint{0.100000in}{0.212622in}}{\pgfqpoint{3.696000in}{3.696000in}}%
\pgfusepath{clip}%
\pgfsetrectcap%
\pgfsetroundjoin%
\pgfsetlinewidth{1.505625pt}%
\definecolor{currentstroke}{rgb}{1.000000,0.000000,0.000000}%
\pgfsetstrokecolor{currentstroke}%
\pgfsetdash{}{0pt}%
\pgfpathmoveto{\pgfqpoint{1.413847in}{2.269547in}}%
\pgfpathlineto{\pgfqpoint{1.588296in}{1.847109in}}%
\pgfusepath{stroke}%
\end{pgfscope}%
\begin{pgfscope}%
\pgfpathrectangle{\pgfqpoint{0.100000in}{0.212622in}}{\pgfqpoint{3.696000in}{3.696000in}}%
\pgfusepath{clip}%
\pgfsetrectcap%
\pgfsetroundjoin%
\pgfsetlinewidth{1.505625pt}%
\definecolor{currentstroke}{rgb}{1.000000,0.000000,0.000000}%
\pgfsetstrokecolor{currentstroke}%
\pgfsetdash{}{0pt}%
\pgfpathmoveto{\pgfqpoint{1.423284in}{2.282553in}}%
\pgfpathlineto{\pgfqpoint{1.596396in}{1.854070in}}%
\pgfusepath{stroke}%
\end{pgfscope}%
\begin{pgfscope}%
\pgfpathrectangle{\pgfqpoint{0.100000in}{0.212622in}}{\pgfqpoint{3.696000in}{3.696000in}}%
\pgfusepath{clip}%
\pgfsetrectcap%
\pgfsetroundjoin%
\pgfsetlinewidth{1.505625pt}%
\definecolor{currentstroke}{rgb}{1.000000,0.000000,0.000000}%
\pgfsetstrokecolor{currentstroke}%
\pgfsetdash{}{0pt}%
\pgfpathmoveto{\pgfqpoint{1.432727in}{2.295030in}}%
\pgfpathlineto{\pgfqpoint{1.604486in}{1.861022in}}%
\pgfusepath{stroke}%
\end{pgfscope}%
\begin{pgfscope}%
\pgfpathrectangle{\pgfqpoint{0.100000in}{0.212622in}}{\pgfqpoint{3.696000in}{3.696000in}}%
\pgfusepath{clip}%
\pgfsetrectcap%
\pgfsetroundjoin%
\pgfsetlinewidth{1.505625pt}%
\definecolor{currentstroke}{rgb}{1.000000,0.000000,0.000000}%
\pgfsetstrokecolor{currentstroke}%
\pgfsetdash{}{0pt}%
\pgfpathmoveto{\pgfqpoint{1.437708in}{2.301957in}}%
\pgfpathlineto{\pgfqpoint{1.612567in}{1.867967in}}%
\pgfusepath{stroke}%
\end{pgfscope}%
\begin{pgfscope}%
\pgfpathrectangle{\pgfqpoint{0.100000in}{0.212622in}}{\pgfqpoint{3.696000in}{3.696000in}}%
\pgfusepath{clip}%
\pgfsetrectcap%
\pgfsetroundjoin%
\pgfsetlinewidth{1.505625pt}%
\definecolor{currentstroke}{rgb}{1.000000,0.000000,0.000000}%
\pgfsetstrokecolor{currentstroke}%
\pgfsetdash{}{0pt}%
\pgfpathmoveto{\pgfqpoint{1.443264in}{2.309598in}}%
\pgfpathlineto{\pgfqpoint{1.620638in}{1.874903in}}%
\pgfusepath{stroke}%
\end{pgfscope}%
\begin{pgfscope}%
\pgfpathrectangle{\pgfqpoint{0.100000in}{0.212622in}}{\pgfqpoint{3.696000in}{3.696000in}}%
\pgfusepath{clip}%
\pgfsetrectcap%
\pgfsetroundjoin%
\pgfsetlinewidth{1.505625pt}%
\definecolor{currentstroke}{rgb}{1.000000,0.000000,0.000000}%
\pgfsetstrokecolor{currentstroke}%
\pgfsetdash{}{0pt}%
\pgfpathmoveto{\pgfqpoint{1.449070in}{2.317616in}}%
\pgfpathlineto{\pgfqpoint{1.620638in}{1.874903in}}%
\pgfusepath{stroke}%
\end{pgfscope}%
\begin{pgfscope}%
\pgfpathrectangle{\pgfqpoint{0.100000in}{0.212622in}}{\pgfqpoint{3.696000in}{3.696000in}}%
\pgfusepath{clip}%
\pgfsetrectcap%
\pgfsetroundjoin%
\pgfsetlinewidth{1.505625pt}%
\definecolor{currentstroke}{rgb}{1.000000,0.000000,0.000000}%
\pgfsetstrokecolor{currentstroke}%
\pgfsetdash{}{0pt}%
\pgfpathmoveto{\pgfqpoint{1.456108in}{2.327666in}}%
\pgfpathlineto{\pgfqpoint{1.628700in}{1.881831in}}%
\pgfusepath{stroke}%
\end{pgfscope}%
\begin{pgfscope}%
\pgfpathrectangle{\pgfqpoint{0.100000in}{0.212622in}}{\pgfqpoint{3.696000in}{3.696000in}}%
\pgfusepath{clip}%
\pgfsetrectcap%
\pgfsetroundjoin%
\pgfsetlinewidth{1.505625pt}%
\definecolor{currentstroke}{rgb}{1.000000,0.000000,0.000000}%
\pgfsetstrokecolor{currentstroke}%
\pgfsetdash{}{0pt}%
\pgfpathmoveto{\pgfqpoint{1.463275in}{2.338268in}}%
\pgfpathlineto{\pgfqpoint{1.636752in}{1.888751in}}%
\pgfusepath{stroke}%
\end{pgfscope}%
\begin{pgfscope}%
\pgfpathrectangle{\pgfqpoint{0.100000in}{0.212622in}}{\pgfqpoint{3.696000in}{3.696000in}}%
\pgfusepath{clip}%
\pgfsetrectcap%
\pgfsetroundjoin%
\pgfsetlinewidth{1.505625pt}%
\definecolor{currentstroke}{rgb}{1.000000,0.000000,0.000000}%
\pgfsetstrokecolor{currentstroke}%
\pgfsetdash{}{0pt}%
\pgfpathmoveto{\pgfqpoint{1.471699in}{2.350530in}}%
\pgfpathlineto{\pgfqpoint{1.652828in}{1.902566in}}%
\pgfusepath{stroke}%
\end{pgfscope}%
\begin{pgfscope}%
\pgfpathrectangle{\pgfqpoint{0.100000in}{0.212622in}}{\pgfqpoint{3.696000in}{3.696000in}}%
\pgfusepath{clip}%
\pgfsetrectcap%
\pgfsetroundjoin%
\pgfsetlinewidth{1.505625pt}%
\definecolor{currentstroke}{rgb}{1.000000,0.000000,0.000000}%
\pgfsetstrokecolor{currentstroke}%
\pgfsetdash{}{0pt}%
\pgfpathmoveto{\pgfqpoint{1.480246in}{2.363417in}}%
\pgfpathlineto{\pgfqpoint{1.660851in}{1.909461in}}%
\pgfusepath{stroke}%
\end{pgfscope}%
\begin{pgfscope}%
\pgfpathrectangle{\pgfqpoint{0.100000in}{0.212622in}}{\pgfqpoint{3.696000in}{3.696000in}}%
\pgfusepath{clip}%
\pgfsetrectcap%
\pgfsetroundjoin%
\pgfsetlinewidth{1.505625pt}%
\definecolor{currentstroke}{rgb}{1.000000,0.000000,0.000000}%
\pgfsetstrokecolor{currentstroke}%
\pgfsetdash{}{0pt}%
\pgfpathmoveto{\pgfqpoint{1.488531in}{2.378748in}}%
\pgfpathlineto{\pgfqpoint{1.668866in}{1.916349in}}%
\pgfusepath{stroke}%
\end{pgfscope}%
\begin{pgfscope}%
\pgfpathrectangle{\pgfqpoint{0.100000in}{0.212622in}}{\pgfqpoint{3.696000in}{3.696000in}}%
\pgfusepath{clip}%
\pgfsetrectcap%
\pgfsetroundjoin%
\pgfsetlinewidth{1.505625pt}%
\definecolor{currentstroke}{rgb}{1.000000,0.000000,0.000000}%
\pgfsetstrokecolor{currentstroke}%
\pgfsetdash{}{0pt}%
\pgfpathmoveto{\pgfqpoint{1.497654in}{2.392716in}}%
\pgfpathlineto{\pgfqpoint{1.684866in}{1.930099in}}%
\pgfusepath{stroke}%
\end{pgfscope}%
\begin{pgfscope}%
\pgfpathrectangle{\pgfqpoint{0.100000in}{0.212622in}}{\pgfqpoint{3.696000in}{3.696000in}}%
\pgfusepath{clip}%
\pgfsetrectcap%
\pgfsetroundjoin%
\pgfsetlinewidth{1.505625pt}%
\definecolor{currentstroke}{rgb}{1.000000,0.000000,0.000000}%
\pgfsetstrokecolor{currentstroke}%
\pgfsetdash{}{0pt}%
\pgfpathmoveto{\pgfqpoint{1.502522in}{2.400076in}}%
\pgfpathlineto{\pgfqpoint{1.692853in}{1.936962in}}%
\pgfusepath{stroke}%
\end{pgfscope}%
\begin{pgfscope}%
\pgfpathrectangle{\pgfqpoint{0.100000in}{0.212622in}}{\pgfqpoint{3.696000in}{3.696000in}}%
\pgfusepath{clip}%
\pgfsetrectcap%
\pgfsetroundjoin%
\pgfsetlinewidth{1.505625pt}%
\definecolor{currentstroke}{rgb}{1.000000,0.000000,0.000000}%
\pgfsetstrokecolor{currentstroke}%
\pgfsetdash{}{0pt}%
\pgfpathmoveto{\pgfqpoint{1.505257in}{2.404213in}}%
\pgfpathlineto{\pgfqpoint{1.692853in}{1.936962in}}%
\pgfusepath{stroke}%
\end{pgfscope}%
\begin{pgfscope}%
\pgfpathrectangle{\pgfqpoint{0.100000in}{0.212622in}}{\pgfqpoint{3.696000in}{3.696000in}}%
\pgfusepath{clip}%
\pgfsetrectcap%
\pgfsetroundjoin%
\pgfsetlinewidth{1.505625pt}%
\definecolor{currentstroke}{rgb}{1.000000,0.000000,0.000000}%
\pgfsetstrokecolor{currentstroke}%
\pgfsetdash{}{0pt}%
\pgfpathmoveto{\pgfqpoint{1.506710in}{2.406490in}}%
\pgfpathlineto{\pgfqpoint{1.692853in}{1.936962in}}%
\pgfusepath{stroke}%
\end{pgfscope}%
\begin{pgfscope}%
\pgfpathrectangle{\pgfqpoint{0.100000in}{0.212622in}}{\pgfqpoint{3.696000in}{3.696000in}}%
\pgfusepath{clip}%
\pgfsetrectcap%
\pgfsetroundjoin%
\pgfsetlinewidth{1.505625pt}%
\definecolor{currentstroke}{rgb}{1.000000,0.000000,0.000000}%
\pgfsetstrokecolor{currentstroke}%
\pgfsetdash{}{0pt}%
\pgfpathmoveto{\pgfqpoint{1.509355in}{2.410737in}}%
\pgfpathlineto{\pgfqpoint{1.700830in}{1.943818in}}%
\pgfusepath{stroke}%
\end{pgfscope}%
\begin{pgfscope}%
\pgfpathrectangle{\pgfqpoint{0.100000in}{0.212622in}}{\pgfqpoint{3.696000in}{3.696000in}}%
\pgfusepath{clip}%
\pgfsetrectcap%
\pgfsetroundjoin%
\pgfsetlinewidth{1.505625pt}%
\definecolor{currentstroke}{rgb}{1.000000,0.000000,0.000000}%
\pgfsetstrokecolor{currentstroke}%
\pgfsetdash{}{0pt}%
\pgfpathmoveto{\pgfqpoint{1.512311in}{2.415800in}}%
\pgfpathlineto{\pgfqpoint{1.700830in}{1.943818in}}%
\pgfusepath{stroke}%
\end{pgfscope}%
\begin{pgfscope}%
\pgfpathrectangle{\pgfqpoint{0.100000in}{0.212622in}}{\pgfqpoint{3.696000in}{3.696000in}}%
\pgfusepath{clip}%
\pgfsetrectcap%
\pgfsetroundjoin%
\pgfsetlinewidth{1.505625pt}%
\definecolor{currentstroke}{rgb}{1.000000,0.000000,0.000000}%
\pgfsetstrokecolor{currentstroke}%
\pgfsetdash{}{0pt}%
\pgfpathmoveto{\pgfqpoint{1.516391in}{2.422165in}}%
\pgfpathlineto{\pgfqpoint{1.708797in}{1.950665in}}%
\pgfusepath{stroke}%
\end{pgfscope}%
\begin{pgfscope}%
\pgfpathrectangle{\pgfqpoint{0.100000in}{0.212622in}}{\pgfqpoint{3.696000in}{3.696000in}}%
\pgfusepath{clip}%
\pgfsetrectcap%
\pgfsetroundjoin%
\pgfsetlinewidth{1.505625pt}%
\definecolor{currentstroke}{rgb}{1.000000,0.000000,0.000000}%
\pgfsetstrokecolor{currentstroke}%
\pgfsetdash{}{0pt}%
\pgfpathmoveto{\pgfqpoint{1.518365in}{2.425924in}}%
\pgfpathlineto{\pgfqpoint{1.708797in}{1.950665in}}%
\pgfusepath{stroke}%
\end{pgfscope}%
\begin{pgfscope}%
\pgfpathrectangle{\pgfqpoint{0.100000in}{0.212622in}}{\pgfqpoint{3.696000in}{3.696000in}}%
\pgfusepath{clip}%
\pgfsetrectcap%
\pgfsetroundjoin%
\pgfsetlinewidth{1.505625pt}%
\definecolor{currentstroke}{rgb}{1.000000,0.000000,0.000000}%
\pgfsetstrokecolor{currentstroke}%
\pgfsetdash{}{0pt}%
\pgfpathmoveto{\pgfqpoint{1.519474in}{2.428090in}}%
\pgfpathlineto{\pgfqpoint{1.708797in}{1.950665in}}%
\pgfusepath{stroke}%
\end{pgfscope}%
\begin{pgfscope}%
\pgfpathrectangle{\pgfqpoint{0.100000in}{0.212622in}}{\pgfqpoint{3.696000in}{3.696000in}}%
\pgfusepath{clip}%
\pgfsetrectcap%
\pgfsetroundjoin%
\pgfsetlinewidth{1.505625pt}%
\definecolor{currentstroke}{rgb}{1.000000,0.000000,0.000000}%
\pgfsetstrokecolor{currentstroke}%
\pgfsetdash{}{0pt}%
\pgfpathmoveto{\pgfqpoint{1.521042in}{2.430988in}}%
\pgfpathlineto{\pgfqpoint{1.716755in}{1.957504in}}%
\pgfusepath{stroke}%
\end{pgfscope}%
\begin{pgfscope}%
\pgfpathrectangle{\pgfqpoint{0.100000in}{0.212622in}}{\pgfqpoint{3.696000in}{3.696000in}}%
\pgfusepath{clip}%
\pgfsetrectcap%
\pgfsetroundjoin%
\pgfsetlinewidth{1.505625pt}%
\definecolor{currentstroke}{rgb}{1.000000,0.000000,0.000000}%
\pgfsetstrokecolor{currentstroke}%
\pgfsetdash{}{0pt}%
\pgfpathmoveto{\pgfqpoint{1.523060in}{2.434689in}}%
\pgfpathlineto{\pgfqpoint{1.716755in}{1.957504in}}%
\pgfusepath{stroke}%
\end{pgfscope}%
\begin{pgfscope}%
\pgfpathrectangle{\pgfqpoint{0.100000in}{0.212622in}}{\pgfqpoint{3.696000in}{3.696000in}}%
\pgfusepath{clip}%
\pgfsetrectcap%
\pgfsetroundjoin%
\pgfsetlinewidth{1.505625pt}%
\definecolor{currentstroke}{rgb}{1.000000,0.000000,0.000000}%
\pgfsetstrokecolor{currentstroke}%
\pgfsetdash{}{0pt}%
\pgfpathmoveto{\pgfqpoint{1.526110in}{2.440036in}}%
\pgfpathlineto{\pgfqpoint{1.716755in}{1.957504in}}%
\pgfusepath{stroke}%
\end{pgfscope}%
\begin{pgfscope}%
\pgfpathrectangle{\pgfqpoint{0.100000in}{0.212622in}}{\pgfqpoint{3.696000in}{3.696000in}}%
\pgfusepath{clip}%
\pgfsetrectcap%
\pgfsetroundjoin%
\pgfsetlinewidth{1.505625pt}%
\definecolor{currentstroke}{rgb}{1.000000,0.000000,0.000000}%
\pgfsetstrokecolor{currentstroke}%
\pgfsetdash{}{0pt}%
\pgfpathmoveto{\pgfqpoint{1.529638in}{2.446379in}}%
\pgfpathlineto{\pgfqpoint{1.724704in}{1.964335in}}%
\pgfusepath{stroke}%
\end{pgfscope}%
\begin{pgfscope}%
\pgfpathrectangle{\pgfqpoint{0.100000in}{0.212622in}}{\pgfqpoint{3.696000in}{3.696000in}}%
\pgfusepath{clip}%
\pgfsetrectcap%
\pgfsetroundjoin%
\pgfsetlinewidth{1.505625pt}%
\definecolor{currentstroke}{rgb}{1.000000,0.000000,0.000000}%
\pgfsetstrokecolor{currentstroke}%
\pgfsetdash{}{0pt}%
\pgfpathmoveto{\pgfqpoint{1.534141in}{2.453481in}}%
\pgfpathlineto{\pgfqpoint{1.732644in}{1.971158in}}%
\pgfusepath{stroke}%
\end{pgfscope}%
\begin{pgfscope}%
\pgfpathrectangle{\pgfqpoint{0.100000in}{0.212622in}}{\pgfqpoint{3.696000in}{3.696000in}}%
\pgfusepath{clip}%
\pgfsetrectcap%
\pgfsetroundjoin%
\pgfsetlinewidth{1.505625pt}%
\definecolor{currentstroke}{rgb}{1.000000,0.000000,0.000000}%
\pgfsetstrokecolor{currentstroke}%
\pgfsetdash{}{0pt}%
\pgfpathmoveto{\pgfqpoint{1.536641in}{2.457362in}}%
\pgfpathlineto{\pgfqpoint{1.732644in}{1.971158in}}%
\pgfusepath{stroke}%
\end{pgfscope}%
\begin{pgfscope}%
\pgfpathrectangle{\pgfqpoint{0.100000in}{0.212622in}}{\pgfqpoint{3.696000in}{3.696000in}}%
\pgfusepath{clip}%
\pgfsetrectcap%
\pgfsetroundjoin%
\pgfsetlinewidth{1.505625pt}%
\definecolor{currentstroke}{rgb}{1.000000,0.000000,0.000000}%
\pgfsetstrokecolor{currentstroke}%
\pgfsetdash{}{0pt}%
\pgfpathmoveto{\pgfqpoint{1.538001in}{2.459507in}}%
\pgfpathlineto{\pgfqpoint{1.732644in}{1.971158in}}%
\pgfusepath{stroke}%
\end{pgfscope}%
\begin{pgfscope}%
\pgfpathrectangle{\pgfqpoint{0.100000in}{0.212622in}}{\pgfqpoint{3.696000in}{3.696000in}}%
\pgfusepath{clip}%
\pgfsetrectcap%
\pgfsetroundjoin%
\pgfsetlinewidth{1.505625pt}%
\definecolor{currentstroke}{rgb}{1.000000,0.000000,0.000000}%
\pgfsetstrokecolor{currentstroke}%
\pgfsetdash{}{0pt}%
\pgfpathmoveto{\pgfqpoint{1.538761in}{2.460692in}}%
\pgfpathlineto{\pgfqpoint{1.732644in}{1.971158in}}%
\pgfusepath{stroke}%
\end{pgfscope}%
\begin{pgfscope}%
\pgfpathrectangle{\pgfqpoint{0.100000in}{0.212622in}}{\pgfqpoint{3.696000in}{3.696000in}}%
\pgfusepath{clip}%
\pgfsetrectcap%
\pgfsetroundjoin%
\pgfsetlinewidth{1.505625pt}%
\definecolor{currentstroke}{rgb}{1.000000,0.000000,0.000000}%
\pgfsetstrokecolor{currentstroke}%
\pgfsetdash{}{0pt}%
\pgfpathmoveto{\pgfqpoint{1.539160in}{2.461343in}}%
\pgfpathlineto{\pgfqpoint{1.732644in}{1.971158in}}%
\pgfusepath{stroke}%
\end{pgfscope}%
\begin{pgfscope}%
\pgfpathrectangle{\pgfqpoint{0.100000in}{0.212622in}}{\pgfqpoint{3.696000in}{3.696000in}}%
\pgfusepath{clip}%
\pgfsetrectcap%
\pgfsetroundjoin%
\pgfsetlinewidth{1.505625pt}%
\definecolor{currentstroke}{rgb}{1.000000,0.000000,0.000000}%
\pgfsetstrokecolor{currentstroke}%
\pgfsetdash{}{0pt}%
\pgfpathmoveto{\pgfqpoint{1.540248in}{2.463106in}}%
\pgfpathlineto{\pgfqpoint{1.740574in}{1.977973in}}%
\pgfusepath{stroke}%
\end{pgfscope}%
\begin{pgfscope}%
\pgfpathrectangle{\pgfqpoint{0.100000in}{0.212622in}}{\pgfqpoint{3.696000in}{3.696000in}}%
\pgfusepath{clip}%
\pgfsetrectcap%
\pgfsetroundjoin%
\pgfsetlinewidth{1.505625pt}%
\definecolor{currentstroke}{rgb}{1.000000,0.000000,0.000000}%
\pgfsetstrokecolor{currentstroke}%
\pgfsetdash{}{0pt}%
\pgfpathmoveto{\pgfqpoint{1.542095in}{2.466293in}}%
\pgfpathlineto{\pgfqpoint{1.740574in}{1.977973in}}%
\pgfusepath{stroke}%
\end{pgfscope}%
\begin{pgfscope}%
\pgfpathrectangle{\pgfqpoint{0.100000in}{0.212622in}}{\pgfqpoint{3.696000in}{3.696000in}}%
\pgfusepath{clip}%
\pgfsetrectcap%
\pgfsetroundjoin%
\pgfsetlinewidth{1.505625pt}%
\definecolor{currentstroke}{rgb}{1.000000,0.000000,0.000000}%
\pgfsetstrokecolor{currentstroke}%
\pgfsetdash{}{0pt}%
\pgfpathmoveto{\pgfqpoint{1.544273in}{2.470583in}}%
\pgfpathlineto{\pgfqpoint{1.740574in}{1.977973in}}%
\pgfusepath{stroke}%
\end{pgfscope}%
\begin{pgfscope}%
\pgfpathrectangle{\pgfqpoint{0.100000in}{0.212622in}}{\pgfqpoint{3.696000in}{3.696000in}}%
\pgfusepath{clip}%
\pgfsetrectcap%
\pgfsetroundjoin%
\pgfsetlinewidth{1.505625pt}%
\definecolor{currentstroke}{rgb}{1.000000,0.000000,0.000000}%
\pgfsetstrokecolor{currentstroke}%
\pgfsetdash{}{0pt}%
\pgfpathmoveto{\pgfqpoint{1.545502in}{2.472925in}}%
\pgfpathlineto{\pgfqpoint{1.748495in}{1.984780in}}%
\pgfusepath{stroke}%
\end{pgfscope}%
\begin{pgfscope}%
\pgfpathrectangle{\pgfqpoint{0.100000in}{0.212622in}}{\pgfqpoint{3.696000in}{3.696000in}}%
\pgfusepath{clip}%
\pgfsetrectcap%
\pgfsetroundjoin%
\pgfsetlinewidth{1.505625pt}%
\definecolor{currentstroke}{rgb}{1.000000,0.000000,0.000000}%
\pgfsetstrokecolor{currentstroke}%
\pgfsetdash{}{0pt}%
\pgfpathmoveto{\pgfqpoint{1.546174in}{2.474156in}}%
\pgfpathlineto{\pgfqpoint{1.748495in}{1.984780in}}%
\pgfusepath{stroke}%
\end{pgfscope}%
\begin{pgfscope}%
\pgfpathrectangle{\pgfqpoint{0.100000in}{0.212622in}}{\pgfqpoint{3.696000in}{3.696000in}}%
\pgfusepath{clip}%
\pgfsetrectcap%
\pgfsetroundjoin%
\pgfsetlinewidth{1.505625pt}%
\definecolor{currentstroke}{rgb}{1.000000,0.000000,0.000000}%
\pgfsetstrokecolor{currentstroke}%
\pgfsetdash{}{0pt}%
\pgfpathmoveto{\pgfqpoint{1.546561in}{2.474850in}}%
\pgfpathlineto{\pgfqpoint{1.748495in}{1.984780in}}%
\pgfusepath{stroke}%
\end{pgfscope}%
\begin{pgfscope}%
\pgfpathrectangle{\pgfqpoint{0.100000in}{0.212622in}}{\pgfqpoint{3.696000in}{3.696000in}}%
\pgfusepath{clip}%
\pgfsetrectcap%
\pgfsetroundjoin%
\pgfsetlinewidth{1.505625pt}%
\definecolor{currentstroke}{rgb}{1.000000,0.000000,0.000000}%
\pgfsetstrokecolor{currentstroke}%
\pgfsetdash{}{0pt}%
\pgfpathmoveto{\pgfqpoint{1.547609in}{2.476786in}}%
\pgfpathlineto{\pgfqpoint{1.748495in}{1.984780in}}%
\pgfusepath{stroke}%
\end{pgfscope}%
\begin{pgfscope}%
\pgfpathrectangle{\pgfqpoint{0.100000in}{0.212622in}}{\pgfqpoint{3.696000in}{3.696000in}}%
\pgfusepath{clip}%
\pgfsetrectcap%
\pgfsetroundjoin%
\pgfsetlinewidth{1.505625pt}%
\definecolor{currentstroke}{rgb}{1.000000,0.000000,0.000000}%
\pgfsetstrokecolor{currentstroke}%
\pgfsetdash{}{0pt}%
\pgfpathmoveto{\pgfqpoint{1.549195in}{2.479373in}}%
\pgfpathlineto{\pgfqpoint{1.748495in}{1.984780in}}%
\pgfusepath{stroke}%
\end{pgfscope}%
\begin{pgfscope}%
\pgfpathrectangle{\pgfqpoint{0.100000in}{0.212622in}}{\pgfqpoint{3.696000in}{3.696000in}}%
\pgfusepath{clip}%
\pgfsetrectcap%
\pgfsetroundjoin%
\pgfsetlinewidth{1.505625pt}%
\definecolor{currentstroke}{rgb}{1.000000,0.000000,0.000000}%
\pgfsetstrokecolor{currentstroke}%
\pgfsetdash{}{0pt}%
\pgfpathmoveto{\pgfqpoint{1.551297in}{2.483261in}}%
\pgfpathlineto{\pgfqpoint{1.756407in}{1.991580in}}%
\pgfusepath{stroke}%
\end{pgfscope}%
\begin{pgfscope}%
\pgfpathrectangle{\pgfqpoint{0.100000in}{0.212622in}}{\pgfqpoint{3.696000in}{3.696000in}}%
\pgfusepath{clip}%
\pgfsetrectcap%
\pgfsetroundjoin%
\pgfsetlinewidth{1.505625pt}%
\definecolor{currentstroke}{rgb}{1.000000,0.000000,0.000000}%
\pgfsetstrokecolor{currentstroke}%
\pgfsetdash{}{0pt}%
\pgfpathmoveto{\pgfqpoint{1.553899in}{2.487994in}}%
\pgfpathlineto{\pgfqpoint{1.756407in}{1.991580in}}%
\pgfusepath{stroke}%
\end{pgfscope}%
\begin{pgfscope}%
\pgfpathrectangle{\pgfqpoint{0.100000in}{0.212622in}}{\pgfqpoint{3.696000in}{3.696000in}}%
\pgfusepath{clip}%
\pgfsetrectcap%
\pgfsetroundjoin%
\pgfsetlinewidth{1.505625pt}%
\definecolor{currentstroke}{rgb}{1.000000,0.000000,0.000000}%
\pgfsetstrokecolor{currentstroke}%
\pgfsetdash{}{0pt}%
\pgfpathmoveto{\pgfqpoint{1.555262in}{2.490632in}}%
\pgfpathlineto{\pgfqpoint{1.756407in}{1.991580in}}%
\pgfusepath{stroke}%
\end{pgfscope}%
\begin{pgfscope}%
\pgfpathrectangle{\pgfqpoint{0.100000in}{0.212622in}}{\pgfqpoint{3.696000in}{3.696000in}}%
\pgfusepath{clip}%
\pgfsetrectcap%
\pgfsetroundjoin%
\pgfsetlinewidth{1.505625pt}%
\definecolor{currentstroke}{rgb}{1.000000,0.000000,0.000000}%
\pgfsetstrokecolor{currentstroke}%
\pgfsetdash{}{0pt}%
\pgfpathmoveto{\pgfqpoint{1.555985in}{2.492027in}}%
\pgfpathlineto{\pgfqpoint{1.756407in}{1.991580in}}%
\pgfusepath{stroke}%
\end{pgfscope}%
\begin{pgfscope}%
\pgfpathrectangle{\pgfqpoint{0.100000in}{0.212622in}}{\pgfqpoint{3.696000in}{3.696000in}}%
\pgfusepath{clip}%
\pgfsetrectcap%
\pgfsetroundjoin%
\pgfsetlinewidth{1.505625pt}%
\definecolor{currentstroke}{rgb}{1.000000,0.000000,0.000000}%
\pgfsetstrokecolor{currentstroke}%
\pgfsetdash{}{0pt}%
\pgfpathmoveto{\pgfqpoint{1.556388in}{2.492819in}}%
\pgfpathlineto{\pgfqpoint{1.756407in}{1.991580in}}%
\pgfusepath{stroke}%
\end{pgfscope}%
\begin{pgfscope}%
\pgfpathrectangle{\pgfqpoint{0.100000in}{0.212622in}}{\pgfqpoint{3.696000in}{3.696000in}}%
\pgfusepath{clip}%
\pgfsetrectcap%
\pgfsetroundjoin%
\pgfsetlinewidth{1.505625pt}%
\definecolor{currentstroke}{rgb}{1.000000,0.000000,0.000000}%
\pgfsetstrokecolor{currentstroke}%
\pgfsetdash{}{0pt}%
\pgfpathmoveto{\pgfqpoint{1.557486in}{2.495086in}}%
\pgfpathlineto{\pgfqpoint{1.764310in}{1.998371in}}%
\pgfusepath{stroke}%
\end{pgfscope}%
\begin{pgfscope}%
\pgfpathrectangle{\pgfqpoint{0.100000in}{0.212622in}}{\pgfqpoint{3.696000in}{3.696000in}}%
\pgfusepath{clip}%
\pgfsetrectcap%
\pgfsetroundjoin%
\pgfsetlinewidth{1.505625pt}%
\definecolor{currentstroke}{rgb}{1.000000,0.000000,0.000000}%
\pgfsetstrokecolor{currentstroke}%
\pgfsetdash{}{0pt}%
\pgfpathmoveto{\pgfqpoint{1.558143in}{2.496332in}}%
\pgfpathlineto{\pgfqpoint{1.764310in}{1.998371in}}%
\pgfusepath{stroke}%
\end{pgfscope}%
\begin{pgfscope}%
\pgfpathrectangle{\pgfqpoint{0.100000in}{0.212622in}}{\pgfqpoint{3.696000in}{3.696000in}}%
\pgfusepath{clip}%
\pgfsetrectcap%
\pgfsetroundjoin%
\pgfsetlinewidth{1.505625pt}%
\definecolor{currentstroke}{rgb}{1.000000,0.000000,0.000000}%
\pgfsetstrokecolor{currentstroke}%
\pgfsetdash{}{0pt}%
\pgfpathmoveto{\pgfqpoint{1.558477in}{2.497038in}}%
\pgfpathlineto{\pgfqpoint{1.764310in}{1.998371in}}%
\pgfusepath{stroke}%
\end{pgfscope}%
\begin{pgfscope}%
\pgfpathrectangle{\pgfqpoint{0.100000in}{0.212622in}}{\pgfqpoint{3.696000in}{3.696000in}}%
\pgfusepath{clip}%
\pgfsetrectcap%
\pgfsetroundjoin%
\pgfsetlinewidth{1.505625pt}%
\definecolor{currentstroke}{rgb}{1.000000,0.000000,0.000000}%
\pgfsetstrokecolor{currentstroke}%
\pgfsetdash{}{0pt}%
\pgfpathmoveto{\pgfqpoint{1.558665in}{2.497420in}}%
\pgfpathlineto{\pgfqpoint{1.764310in}{1.998371in}}%
\pgfusepath{stroke}%
\end{pgfscope}%
\begin{pgfscope}%
\pgfpathrectangle{\pgfqpoint{0.100000in}{0.212622in}}{\pgfqpoint{3.696000in}{3.696000in}}%
\pgfusepath{clip}%
\pgfsetrectcap%
\pgfsetroundjoin%
\pgfsetlinewidth{1.505625pt}%
\definecolor{currentstroke}{rgb}{1.000000,0.000000,0.000000}%
\pgfsetstrokecolor{currentstroke}%
\pgfsetdash{}{0pt}%
\pgfpathmoveto{\pgfqpoint{1.558777in}{2.497630in}}%
\pgfpathlineto{\pgfqpoint{1.764310in}{1.998371in}}%
\pgfusepath{stroke}%
\end{pgfscope}%
\begin{pgfscope}%
\pgfpathrectangle{\pgfqpoint{0.100000in}{0.212622in}}{\pgfqpoint{3.696000in}{3.696000in}}%
\pgfusepath{clip}%
\pgfsetrectcap%
\pgfsetroundjoin%
\pgfsetlinewidth{1.505625pt}%
\definecolor{currentstroke}{rgb}{1.000000,0.000000,0.000000}%
\pgfsetstrokecolor{currentstroke}%
\pgfsetdash{}{0pt}%
\pgfpathmoveto{\pgfqpoint{1.559146in}{2.498271in}}%
\pgfpathlineto{\pgfqpoint{1.764310in}{1.998371in}}%
\pgfusepath{stroke}%
\end{pgfscope}%
\begin{pgfscope}%
\pgfpathrectangle{\pgfqpoint{0.100000in}{0.212622in}}{\pgfqpoint{3.696000in}{3.696000in}}%
\pgfusepath{clip}%
\pgfsetrectcap%
\pgfsetroundjoin%
\pgfsetlinewidth{1.505625pt}%
\definecolor{currentstroke}{rgb}{1.000000,0.000000,0.000000}%
\pgfsetstrokecolor{currentstroke}%
\pgfsetdash{}{0pt}%
\pgfpathmoveto{\pgfqpoint{1.559362in}{2.498604in}}%
\pgfpathlineto{\pgfqpoint{1.764310in}{1.998371in}}%
\pgfusepath{stroke}%
\end{pgfscope}%
\begin{pgfscope}%
\pgfpathrectangle{\pgfqpoint{0.100000in}{0.212622in}}{\pgfqpoint{3.696000in}{3.696000in}}%
\pgfusepath{clip}%
\pgfsetrectcap%
\pgfsetroundjoin%
\pgfsetlinewidth{1.505625pt}%
\definecolor{currentstroke}{rgb}{1.000000,0.000000,0.000000}%
\pgfsetstrokecolor{currentstroke}%
\pgfsetdash{}{0pt}%
\pgfpathmoveto{\pgfqpoint{1.560113in}{2.499706in}}%
\pgfpathlineto{\pgfqpoint{1.764310in}{1.998371in}}%
\pgfusepath{stroke}%
\end{pgfscope}%
\begin{pgfscope}%
\pgfpathrectangle{\pgfqpoint{0.100000in}{0.212622in}}{\pgfqpoint{3.696000in}{3.696000in}}%
\pgfusepath{clip}%
\pgfsetrectcap%
\pgfsetroundjoin%
\pgfsetlinewidth{1.505625pt}%
\definecolor{currentstroke}{rgb}{1.000000,0.000000,0.000000}%
\pgfsetstrokecolor{currentstroke}%
\pgfsetdash{}{0pt}%
\pgfpathmoveto{\pgfqpoint{1.561190in}{2.501205in}}%
\pgfpathlineto{\pgfqpoint{1.764310in}{1.998371in}}%
\pgfusepath{stroke}%
\end{pgfscope}%
\begin{pgfscope}%
\pgfpathrectangle{\pgfqpoint{0.100000in}{0.212622in}}{\pgfqpoint{3.696000in}{3.696000in}}%
\pgfusepath{clip}%
\pgfsetrectcap%
\pgfsetroundjoin%
\pgfsetlinewidth{1.505625pt}%
\definecolor{currentstroke}{rgb}{1.000000,0.000000,0.000000}%
\pgfsetstrokecolor{currentstroke}%
\pgfsetdash{}{0pt}%
\pgfpathmoveto{\pgfqpoint{1.561801in}{2.502028in}}%
\pgfpathlineto{\pgfqpoint{1.764310in}{1.998371in}}%
\pgfusepath{stroke}%
\end{pgfscope}%
\begin{pgfscope}%
\pgfpathrectangle{\pgfqpoint{0.100000in}{0.212622in}}{\pgfqpoint{3.696000in}{3.696000in}}%
\pgfusepath{clip}%
\pgfsetrectcap%
\pgfsetroundjoin%
\pgfsetlinewidth{1.505625pt}%
\definecolor{currentstroke}{rgb}{1.000000,0.000000,0.000000}%
\pgfsetstrokecolor{currentstroke}%
\pgfsetdash{}{0pt}%
\pgfpathmoveto{\pgfqpoint{1.562142in}{2.502495in}}%
\pgfpathlineto{\pgfqpoint{1.764310in}{1.998371in}}%
\pgfusepath{stroke}%
\end{pgfscope}%
\begin{pgfscope}%
\pgfpathrectangle{\pgfqpoint{0.100000in}{0.212622in}}{\pgfqpoint{3.696000in}{3.696000in}}%
\pgfusepath{clip}%
\pgfsetrectcap%
\pgfsetroundjoin%
\pgfsetlinewidth{1.505625pt}%
\definecolor{currentstroke}{rgb}{1.000000,0.000000,0.000000}%
\pgfsetstrokecolor{currentstroke}%
\pgfsetdash{}{0pt}%
\pgfpathmoveto{\pgfqpoint{1.562821in}{2.503448in}}%
\pgfpathlineto{\pgfqpoint{1.764310in}{1.998371in}}%
\pgfusepath{stroke}%
\end{pgfscope}%
\begin{pgfscope}%
\pgfpathrectangle{\pgfqpoint{0.100000in}{0.212622in}}{\pgfqpoint{3.696000in}{3.696000in}}%
\pgfusepath{clip}%
\pgfsetrectcap%
\pgfsetroundjoin%
\pgfsetlinewidth{1.505625pt}%
\definecolor{currentstroke}{rgb}{1.000000,0.000000,0.000000}%
\pgfsetstrokecolor{currentstroke}%
\pgfsetdash{}{0pt}%
\pgfpathmoveto{\pgfqpoint{1.563921in}{2.505221in}}%
\pgfpathlineto{\pgfqpoint{1.772203in}{2.005154in}}%
\pgfusepath{stroke}%
\end{pgfscope}%
\begin{pgfscope}%
\pgfpathrectangle{\pgfqpoint{0.100000in}{0.212622in}}{\pgfqpoint{3.696000in}{3.696000in}}%
\pgfusepath{clip}%
\pgfsetrectcap%
\pgfsetroundjoin%
\pgfsetlinewidth{1.505625pt}%
\definecolor{currentstroke}{rgb}{1.000000,0.000000,0.000000}%
\pgfsetstrokecolor{currentstroke}%
\pgfsetdash{}{0pt}%
\pgfpathmoveto{\pgfqpoint{1.564556in}{2.506133in}}%
\pgfpathlineto{\pgfqpoint{1.772203in}{2.005154in}}%
\pgfusepath{stroke}%
\end{pgfscope}%
\begin{pgfscope}%
\pgfpathrectangle{\pgfqpoint{0.100000in}{0.212622in}}{\pgfqpoint{3.696000in}{3.696000in}}%
\pgfusepath{clip}%
\pgfsetrectcap%
\pgfsetroundjoin%
\pgfsetlinewidth{1.505625pt}%
\definecolor{currentstroke}{rgb}{1.000000,0.000000,0.000000}%
\pgfsetstrokecolor{currentstroke}%
\pgfsetdash{}{0pt}%
\pgfpathmoveto{\pgfqpoint{1.564870in}{2.506698in}}%
\pgfpathlineto{\pgfqpoint{1.772203in}{2.005154in}}%
\pgfusepath{stroke}%
\end{pgfscope}%
\begin{pgfscope}%
\pgfpathrectangle{\pgfqpoint{0.100000in}{0.212622in}}{\pgfqpoint{3.696000in}{3.696000in}}%
\pgfusepath{clip}%
\pgfsetrectcap%
\pgfsetroundjoin%
\pgfsetlinewidth{1.505625pt}%
\definecolor{currentstroke}{rgb}{1.000000,0.000000,0.000000}%
\pgfsetstrokecolor{currentstroke}%
\pgfsetdash{}{0pt}%
\pgfpathmoveto{\pgfqpoint{1.565053in}{2.506945in}}%
\pgfpathlineto{\pgfqpoint{1.772203in}{2.005154in}}%
\pgfusepath{stroke}%
\end{pgfscope}%
\begin{pgfscope}%
\pgfpathrectangle{\pgfqpoint{0.100000in}{0.212622in}}{\pgfqpoint{3.696000in}{3.696000in}}%
\pgfusepath{clip}%
\pgfsetrectcap%
\pgfsetroundjoin%
\pgfsetlinewidth{1.505625pt}%
\definecolor{currentstroke}{rgb}{1.000000,0.000000,0.000000}%
\pgfsetstrokecolor{currentstroke}%
\pgfsetdash{}{0pt}%
\pgfpathmoveto{\pgfqpoint{1.565144in}{2.507113in}}%
\pgfpathlineto{\pgfqpoint{1.772203in}{2.005154in}}%
\pgfusepath{stroke}%
\end{pgfscope}%
\begin{pgfscope}%
\pgfpathrectangle{\pgfqpoint{0.100000in}{0.212622in}}{\pgfqpoint{3.696000in}{3.696000in}}%
\pgfusepath{clip}%
\pgfsetrectcap%
\pgfsetroundjoin%
\pgfsetlinewidth{1.505625pt}%
\definecolor{currentstroke}{rgb}{1.000000,0.000000,0.000000}%
\pgfsetstrokecolor{currentstroke}%
\pgfsetdash{}{0pt}%
\pgfpathmoveto{\pgfqpoint{1.565197in}{2.507197in}}%
\pgfpathlineto{\pgfqpoint{1.772203in}{2.005154in}}%
\pgfusepath{stroke}%
\end{pgfscope}%
\begin{pgfscope}%
\pgfpathrectangle{\pgfqpoint{0.100000in}{0.212622in}}{\pgfqpoint{3.696000in}{3.696000in}}%
\pgfusepath{clip}%
\pgfsetrectcap%
\pgfsetroundjoin%
\pgfsetlinewidth{1.505625pt}%
\definecolor{currentstroke}{rgb}{1.000000,0.000000,0.000000}%
\pgfsetstrokecolor{currentstroke}%
\pgfsetdash{}{0pt}%
\pgfpathmoveto{\pgfqpoint{1.565490in}{2.507684in}}%
\pgfpathlineto{\pgfqpoint{1.772203in}{2.005154in}}%
\pgfusepath{stroke}%
\end{pgfscope}%
\begin{pgfscope}%
\pgfpathrectangle{\pgfqpoint{0.100000in}{0.212622in}}{\pgfqpoint{3.696000in}{3.696000in}}%
\pgfusepath{clip}%
\pgfsetrectcap%
\pgfsetroundjoin%
\pgfsetlinewidth{1.505625pt}%
\definecolor{currentstroke}{rgb}{1.000000,0.000000,0.000000}%
\pgfsetstrokecolor{currentstroke}%
\pgfsetdash{}{0pt}%
\pgfpathmoveto{\pgfqpoint{1.566009in}{2.508519in}}%
\pgfpathlineto{\pgfqpoint{1.772203in}{2.005154in}}%
\pgfusepath{stroke}%
\end{pgfscope}%
\begin{pgfscope}%
\pgfpathrectangle{\pgfqpoint{0.100000in}{0.212622in}}{\pgfqpoint{3.696000in}{3.696000in}}%
\pgfusepath{clip}%
\pgfsetrectcap%
\pgfsetroundjoin%
\pgfsetlinewidth{1.505625pt}%
\definecolor{currentstroke}{rgb}{1.000000,0.000000,0.000000}%
\pgfsetstrokecolor{currentstroke}%
\pgfsetdash{}{0pt}%
\pgfpathmoveto{\pgfqpoint{1.567250in}{2.510602in}}%
\pgfpathlineto{\pgfqpoint{1.772203in}{2.005154in}}%
\pgfusepath{stroke}%
\end{pgfscope}%
\begin{pgfscope}%
\pgfpathrectangle{\pgfqpoint{0.100000in}{0.212622in}}{\pgfqpoint{3.696000in}{3.696000in}}%
\pgfusepath{clip}%
\pgfsetrectcap%
\pgfsetroundjoin%
\pgfsetlinewidth{1.505625pt}%
\definecolor{currentstroke}{rgb}{1.000000,0.000000,0.000000}%
\pgfsetstrokecolor{currentstroke}%
\pgfsetdash{}{0pt}%
\pgfpathmoveto{\pgfqpoint{1.568740in}{2.513044in}}%
\pgfpathlineto{\pgfqpoint{1.772203in}{2.005154in}}%
\pgfusepath{stroke}%
\end{pgfscope}%
\begin{pgfscope}%
\pgfpathrectangle{\pgfqpoint{0.100000in}{0.212622in}}{\pgfqpoint{3.696000in}{3.696000in}}%
\pgfusepath{clip}%
\pgfsetrectcap%
\pgfsetroundjoin%
\pgfsetlinewidth{1.505625pt}%
\definecolor{currentstroke}{rgb}{1.000000,0.000000,0.000000}%
\pgfsetstrokecolor{currentstroke}%
\pgfsetdash{}{0pt}%
\pgfpathmoveto{\pgfqpoint{1.570496in}{2.515820in}}%
\pgfpathlineto{\pgfqpoint{1.780088in}{2.011930in}}%
\pgfusepath{stroke}%
\end{pgfscope}%
\begin{pgfscope}%
\pgfpathrectangle{\pgfqpoint{0.100000in}{0.212622in}}{\pgfqpoint{3.696000in}{3.696000in}}%
\pgfusepath{clip}%
\pgfsetrectcap%
\pgfsetroundjoin%
\pgfsetlinewidth{1.505625pt}%
\definecolor{currentstroke}{rgb}{1.000000,0.000000,0.000000}%
\pgfsetstrokecolor{currentstroke}%
\pgfsetdash{}{0pt}%
\pgfpathmoveto{\pgfqpoint{1.571485in}{2.517344in}}%
\pgfpathlineto{\pgfqpoint{1.780088in}{2.011930in}}%
\pgfusepath{stroke}%
\end{pgfscope}%
\begin{pgfscope}%
\pgfpathrectangle{\pgfqpoint{0.100000in}{0.212622in}}{\pgfqpoint{3.696000in}{3.696000in}}%
\pgfusepath{clip}%
\pgfsetrectcap%
\pgfsetroundjoin%
\pgfsetlinewidth{1.505625pt}%
\definecolor{currentstroke}{rgb}{1.000000,0.000000,0.000000}%
\pgfsetstrokecolor{currentstroke}%
\pgfsetdash{}{0pt}%
\pgfpathmoveto{\pgfqpoint{1.572051in}{2.518215in}}%
\pgfpathlineto{\pgfqpoint{1.780088in}{2.011930in}}%
\pgfusepath{stroke}%
\end{pgfscope}%
\begin{pgfscope}%
\pgfpathrectangle{\pgfqpoint{0.100000in}{0.212622in}}{\pgfqpoint{3.696000in}{3.696000in}}%
\pgfusepath{clip}%
\pgfsetrectcap%
\pgfsetroundjoin%
\pgfsetlinewidth{1.505625pt}%
\definecolor{currentstroke}{rgb}{1.000000,0.000000,0.000000}%
\pgfsetstrokecolor{currentstroke}%
\pgfsetdash{}{0pt}%
\pgfpathmoveto{\pgfqpoint{1.572904in}{2.519568in}}%
\pgfpathlineto{\pgfqpoint{1.780088in}{2.011930in}}%
\pgfusepath{stroke}%
\end{pgfscope}%
\begin{pgfscope}%
\pgfpathrectangle{\pgfqpoint{0.100000in}{0.212622in}}{\pgfqpoint{3.696000in}{3.696000in}}%
\pgfusepath{clip}%
\pgfsetrectcap%
\pgfsetroundjoin%
\pgfsetlinewidth{1.505625pt}%
\definecolor{currentstroke}{rgb}{1.000000,0.000000,0.000000}%
\pgfsetstrokecolor{currentstroke}%
\pgfsetdash{}{0pt}%
\pgfpathmoveto{\pgfqpoint{1.574461in}{2.521892in}}%
\pgfpathlineto{\pgfqpoint{1.780088in}{2.011930in}}%
\pgfusepath{stroke}%
\end{pgfscope}%
\begin{pgfscope}%
\pgfpathrectangle{\pgfqpoint{0.100000in}{0.212622in}}{\pgfqpoint{3.696000in}{3.696000in}}%
\pgfusepath{clip}%
\pgfsetrectcap%
\pgfsetroundjoin%
\pgfsetlinewidth{1.505625pt}%
\definecolor{currentstroke}{rgb}{1.000000,0.000000,0.000000}%
\pgfsetstrokecolor{currentstroke}%
\pgfsetdash{}{0pt}%
\pgfpathmoveto{\pgfqpoint{1.575332in}{2.523185in}}%
\pgfpathlineto{\pgfqpoint{1.780088in}{2.011930in}}%
\pgfusepath{stroke}%
\end{pgfscope}%
\begin{pgfscope}%
\pgfpathrectangle{\pgfqpoint{0.100000in}{0.212622in}}{\pgfqpoint{3.696000in}{3.696000in}}%
\pgfusepath{clip}%
\pgfsetrectcap%
\pgfsetroundjoin%
\pgfsetlinewidth{1.505625pt}%
\definecolor{currentstroke}{rgb}{1.000000,0.000000,0.000000}%
\pgfsetstrokecolor{currentstroke}%
\pgfsetdash{}{0pt}%
\pgfpathmoveto{\pgfqpoint{1.575801in}{2.523900in}}%
\pgfpathlineto{\pgfqpoint{1.780088in}{2.011930in}}%
\pgfusepath{stroke}%
\end{pgfscope}%
\begin{pgfscope}%
\pgfpathrectangle{\pgfqpoint{0.100000in}{0.212622in}}{\pgfqpoint{3.696000in}{3.696000in}}%
\pgfusepath{clip}%
\pgfsetrectcap%
\pgfsetroundjoin%
\pgfsetlinewidth{1.505625pt}%
\definecolor{currentstroke}{rgb}{1.000000,0.000000,0.000000}%
\pgfsetstrokecolor{currentstroke}%
\pgfsetdash{}{0pt}%
\pgfpathmoveto{\pgfqpoint{1.576068in}{2.524296in}}%
\pgfpathlineto{\pgfqpoint{1.780088in}{2.011930in}}%
\pgfusepath{stroke}%
\end{pgfscope}%
\begin{pgfscope}%
\pgfpathrectangle{\pgfqpoint{0.100000in}{0.212622in}}{\pgfqpoint{3.696000in}{3.696000in}}%
\pgfusepath{clip}%
\pgfsetrectcap%
\pgfsetroundjoin%
\pgfsetlinewidth{1.505625pt}%
\definecolor{currentstroke}{rgb}{1.000000,0.000000,0.000000}%
\pgfsetstrokecolor{currentstroke}%
\pgfsetdash{}{0pt}%
\pgfpathmoveto{\pgfqpoint{1.577209in}{2.526177in}}%
\pgfpathlineto{\pgfqpoint{1.780088in}{2.011930in}}%
\pgfusepath{stroke}%
\end{pgfscope}%
\begin{pgfscope}%
\pgfpathrectangle{\pgfqpoint{0.100000in}{0.212622in}}{\pgfqpoint{3.696000in}{3.696000in}}%
\pgfusepath{clip}%
\pgfsetrectcap%
\pgfsetroundjoin%
\pgfsetlinewidth{1.505625pt}%
\definecolor{currentstroke}{rgb}{1.000000,0.000000,0.000000}%
\pgfsetstrokecolor{currentstroke}%
\pgfsetdash{}{0pt}%
\pgfpathmoveto{\pgfqpoint{1.578553in}{2.528461in}}%
\pgfpathlineto{\pgfqpoint{1.780088in}{2.011930in}}%
\pgfusepath{stroke}%
\end{pgfscope}%
\begin{pgfscope}%
\pgfpathrectangle{\pgfqpoint{0.100000in}{0.212622in}}{\pgfqpoint{3.696000in}{3.696000in}}%
\pgfusepath{clip}%
\pgfsetrectcap%
\pgfsetroundjoin%
\pgfsetlinewidth{1.505625pt}%
\definecolor{currentstroke}{rgb}{1.000000,0.000000,0.000000}%
\pgfsetstrokecolor{currentstroke}%
\pgfsetdash{}{0pt}%
\pgfpathmoveto{\pgfqpoint{1.580206in}{2.531305in}}%
\pgfpathlineto{\pgfqpoint{1.780088in}{2.011930in}}%
\pgfusepath{stroke}%
\end{pgfscope}%
\begin{pgfscope}%
\pgfpathrectangle{\pgfqpoint{0.100000in}{0.212622in}}{\pgfqpoint{3.696000in}{3.696000in}}%
\pgfusepath{clip}%
\pgfsetrectcap%
\pgfsetroundjoin%
\pgfsetlinewidth{1.505625pt}%
\definecolor{currentstroke}{rgb}{1.000000,0.000000,0.000000}%
\pgfsetstrokecolor{currentstroke}%
\pgfsetdash{}{0pt}%
\pgfpathmoveto{\pgfqpoint{1.582353in}{2.535130in}}%
\pgfpathlineto{\pgfqpoint{1.780088in}{2.011930in}}%
\pgfusepath{stroke}%
\end{pgfscope}%
\begin{pgfscope}%
\pgfpathrectangle{\pgfqpoint{0.100000in}{0.212622in}}{\pgfqpoint{3.696000in}{3.696000in}}%
\pgfusepath{clip}%
\pgfsetrectcap%
\pgfsetroundjoin%
\pgfsetlinewidth{1.505625pt}%
\definecolor{currentstroke}{rgb}{1.000000,0.000000,0.000000}%
\pgfsetstrokecolor{currentstroke}%
\pgfsetdash{}{0pt}%
\pgfpathmoveto{\pgfqpoint{1.584881in}{2.539182in}}%
\pgfpathlineto{\pgfqpoint{1.780088in}{2.011930in}}%
\pgfusepath{stroke}%
\end{pgfscope}%
\begin{pgfscope}%
\pgfpathrectangle{\pgfqpoint{0.100000in}{0.212622in}}{\pgfqpoint{3.696000in}{3.696000in}}%
\pgfusepath{clip}%
\pgfsetrectcap%
\pgfsetroundjoin%
\pgfsetlinewidth{1.505625pt}%
\definecolor{currentstroke}{rgb}{1.000000,0.000000,0.000000}%
\pgfsetstrokecolor{currentstroke}%
\pgfsetdash{}{0pt}%
\pgfpathmoveto{\pgfqpoint{1.586298in}{2.541444in}}%
\pgfpathlineto{\pgfqpoint{1.780088in}{2.011930in}}%
\pgfusepath{stroke}%
\end{pgfscope}%
\begin{pgfscope}%
\pgfpathrectangle{\pgfqpoint{0.100000in}{0.212622in}}{\pgfqpoint{3.696000in}{3.696000in}}%
\pgfusepath{clip}%
\pgfsetrectcap%
\pgfsetroundjoin%
\pgfsetlinewidth{1.505625pt}%
\definecolor{currentstroke}{rgb}{1.000000,0.000000,0.000000}%
\pgfsetstrokecolor{currentstroke}%
\pgfsetdash{}{0pt}%
\pgfpathmoveto{\pgfqpoint{1.587876in}{2.544085in}}%
\pgfpathlineto{\pgfqpoint{1.780088in}{2.011930in}}%
\pgfusepath{stroke}%
\end{pgfscope}%
\begin{pgfscope}%
\pgfpathrectangle{\pgfqpoint{0.100000in}{0.212622in}}{\pgfqpoint{3.696000in}{3.696000in}}%
\pgfusepath{clip}%
\pgfsetrectcap%
\pgfsetroundjoin%
\pgfsetlinewidth{1.505625pt}%
\definecolor{currentstroke}{rgb}{1.000000,0.000000,0.000000}%
\pgfsetstrokecolor{currentstroke}%
\pgfsetdash{}{0pt}%
\pgfpathmoveto{\pgfqpoint{1.590363in}{2.548038in}}%
\pgfpathlineto{\pgfqpoint{1.780088in}{2.011930in}}%
\pgfusepath{stroke}%
\end{pgfscope}%
\begin{pgfscope}%
\pgfpathrectangle{\pgfqpoint{0.100000in}{0.212622in}}{\pgfqpoint{3.696000in}{3.696000in}}%
\pgfusepath{clip}%
\pgfsetrectcap%
\pgfsetroundjoin%
\pgfsetlinewidth{1.505625pt}%
\definecolor{currentstroke}{rgb}{1.000000,0.000000,0.000000}%
\pgfsetstrokecolor{currentstroke}%
\pgfsetdash{}{0pt}%
\pgfpathmoveto{\pgfqpoint{1.593276in}{2.553136in}}%
\pgfpathlineto{\pgfqpoint{1.780088in}{2.011930in}}%
\pgfusepath{stroke}%
\end{pgfscope}%
\begin{pgfscope}%
\pgfpathrectangle{\pgfqpoint{0.100000in}{0.212622in}}{\pgfqpoint{3.696000in}{3.696000in}}%
\pgfusepath{clip}%
\pgfsetrectcap%
\pgfsetroundjoin%
\pgfsetlinewidth{1.505625pt}%
\definecolor{currentstroke}{rgb}{1.000000,0.000000,0.000000}%
\pgfsetstrokecolor{currentstroke}%
\pgfsetdash{}{0pt}%
\pgfpathmoveto{\pgfqpoint{1.594932in}{2.555929in}}%
\pgfpathlineto{\pgfqpoint{1.780088in}{2.011930in}}%
\pgfusepath{stroke}%
\end{pgfscope}%
\begin{pgfscope}%
\pgfpathrectangle{\pgfqpoint{0.100000in}{0.212622in}}{\pgfqpoint{3.696000in}{3.696000in}}%
\pgfusepath{clip}%
\pgfsetrectcap%
\pgfsetroundjoin%
\pgfsetlinewidth{1.505625pt}%
\definecolor{currentstroke}{rgb}{1.000000,0.000000,0.000000}%
\pgfsetstrokecolor{currentstroke}%
\pgfsetdash{}{0pt}%
\pgfpathmoveto{\pgfqpoint{1.596965in}{2.559275in}}%
\pgfpathlineto{\pgfqpoint{1.780088in}{2.011930in}}%
\pgfusepath{stroke}%
\end{pgfscope}%
\begin{pgfscope}%
\pgfpathrectangle{\pgfqpoint{0.100000in}{0.212622in}}{\pgfqpoint{3.696000in}{3.696000in}}%
\pgfusepath{clip}%
\pgfsetrectcap%
\pgfsetroundjoin%
\pgfsetlinewidth{1.505625pt}%
\definecolor{currentstroke}{rgb}{1.000000,0.000000,0.000000}%
\pgfsetstrokecolor{currentstroke}%
\pgfsetdash{}{0pt}%
\pgfpathmoveto{\pgfqpoint{1.598029in}{2.561140in}}%
\pgfpathlineto{\pgfqpoint{1.780088in}{2.011930in}}%
\pgfusepath{stroke}%
\end{pgfscope}%
\begin{pgfscope}%
\pgfpathrectangle{\pgfqpoint{0.100000in}{0.212622in}}{\pgfqpoint{3.696000in}{3.696000in}}%
\pgfusepath{clip}%
\pgfsetrectcap%
\pgfsetroundjoin%
\pgfsetlinewidth{1.505625pt}%
\definecolor{currentstroke}{rgb}{1.000000,0.000000,0.000000}%
\pgfsetstrokecolor{currentstroke}%
\pgfsetdash{}{0pt}%
\pgfpathmoveto{\pgfqpoint{1.599797in}{2.563728in}}%
\pgfpathlineto{\pgfqpoint{1.780088in}{2.011930in}}%
\pgfusepath{stroke}%
\end{pgfscope}%
\begin{pgfscope}%
\pgfpathrectangle{\pgfqpoint{0.100000in}{0.212622in}}{\pgfqpoint{3.696000in}{3.696000in}}%
\pgfusepath{clip}%
\pgfsetrectcap%
\pgfsetroundjoin%
\pgfsetlinewidth{1.505625pt}%
\definecolor{currentstroke}{rgb}{1.000000,0.000000,0.000000}%
\pgfsetstrokecolor{currentstroke}%
\pgfsetdash{}{0pt}%
\pgfpathmoveto{\pgfqpoint{1.600800in}{2.565167in}}%
\pgfpathlineto{\pgfqpoint{1.780088in}{2.011930in}}%
\pgfusepath{stroke}%
\end{pgfscope}%
\begin{pgfscope}%
\pgfpathrectangle{\pgfqpoint{0.100000in}{0.212622in}}{\pgfqpoint{3.696000in}{3.696000in}}%
\pgfusepath{clip}%
\pgfsetrectcap%
\pgfsetroundjoin%
\pgfsetlinewidth{1.505625pt}%
\definecolor{currentstroke}{rgb}{1.000000,0.000000,0.000000}%
\pgfsetstrokecolor{currentstroke}%
\pgfsetdash{}{0pt}%
\pgfpathmoveto{\pgfqpoint{1.601329in}{2.565960in}}%
\pgfpathlineto{\pgfqpoint{1.780088in}{2.011930in}}%
\pgfusepath{stroke}%
\end{pgfscope}%
\begin{pgfscope}%
\pgfpathrectangle{\pgfqpoint{0.100000in}{0.212622in}}{\pgfqpoint{3.696000in}{3.696000in}}%
\pgfusepath{clip}%
\pgfsetrectcap%
\pgfsetroundjoin%
\pgfsetlinewidth{1.505625pt}%
\definecolor{currentstroke}{rgb}{1.000000,0.000000,0.000000}%
\pgfsetstrokecolor{currentstroke}%
\pgfsetdash{}{0pt}%
\pgfpathmoveto{\pgfqpoint{1.602619in}{2.567857in}}%
\pgfpathlineto{\pgfqpoint{1.780088in}{2.011930in}}%
\pgfusepath{stroke}%
\end{pgfscope}%
\begin{pgfscope}%
\pgfpathrectangle{\pgfqpoint{0.100000in}{0.212622in}}{\pgfqpoint{3.696000in}{3.696000in}}%
\pgfusepath{clip}%
\pgfsetrectcap%
\pgfsetroundjoin%
\pgfsetlinewidth{1.505625pt}%
\definecolor{currentstroke}{rgb}{1.000000,0.000000,0.000000}%
\pgfsetstrokecolor{currentstroke}%
\pgfsetdash{}{0pt}%
\pgfpathmoveto{\pgfqpoint{1.603283in}{2.568907in}}%
\pgfpathlineto{\pgfqpoint{1.780088in}{2.011930in}}%
\pgfusepath{stroke}%
\end{pgfscope}%
\begin{pgfscope}%
\pgfpathrectangle{\pgfqpoint{0.100000in}{0.212622in}}{\pgfqpoint{3.696000in}{3.696000in}}%
\pgfusepath{clip}%
\pgfsetrectcap%
\pgfsetroundjoin%
\pgfsetlinewidth{1.505625pt}%
\definecolor{currentstroke}{rgb}{1.000000,0.000000,0.000000}%
\pgfsetstrokecolor{currentstroke}%
\pgfsetdash{}{0pt}%
\pgfpathmoveto{\pgfqpoint{1.604139in}{2.570267in}}%
\pgfpathlineto{\pgfqpoint{1.780088in}{2.011930in}}%
\pgfusepath{stroke}%
\end{pgfscope}%
\begin{pgfscope}%
\pgfpathrectangle{\pgfqpoint{0.100000in}{0.212622in}}{\pgfqpoint{3.696000in}{3.696000in}}%
\pgfusepath{clip}%
\pgfsetrectcap%
\pgfsetroundjoin%
\pgfsetlinewidth{1.505625pt}%
\definecolor{currentstroke}{rgb}{1.000000,0.000000,0.000000}%
\pgfsetstrokecolor{currentstroke}%
\pgfsetdash{}{0pt}%
\pgfpathmoveto{\pgfqpoint{1.604621in}{2.571017in}}%
\pgfpathlineto{\pgfqpoint{1.780088in}{2.011930in}}%
\pgfusepath{stroke}%
\end{pgfscope}%
\begin{pgfscope}%
\pgfpathrectangle{\pgfqpoint{0.100000in}{0.212622in}}{\pgfqpoint{3.696000in}{3.696000in}}%
\pgfusepath{clip}%
\pgfsetrectcap%
\pgfsetroundjoin%
\pgfsetlinewidth{1.505625pt}%
\definecolor{currentstroke}{rgb}{1.000000,0.000000,0.000000}%
\pgfsetstrokecolor{currentstroke}%
\pgfsetdash{}{0pt}%
\pgfpathmoveto{\pgfqpoint{1.605842in}{2.573091in}}%
\pgfpathlineto{\pgfqpoint{1.780088in}{2.011930in}}%
\pgfusepath{stroke}%
\end{pgfscope}%
\begin{pgfscope}%
\pgfpathrectangle{\pgfqpoint{0.100000in}{0.212622in}}{\pgfqpoint{3.696000in}{3.696000in}}%
\pgfusepath{clip}%
\pgfsetrectcap%
\pgfsetroundjoin%
\pgfsetlinewidth{1.505625pt}%
\definecolor{currentstroke}{rgb}{1.000000,0.000000,0.000000}%
\pgfsetstrokecolor{currentstroke}%
\pgfsetdash{}{0pt}%
\pgfpathmoveto{\pgfqpoint{1.606537in}{2.574171in}}%
\pgfpathlineto{\pgfqpoint{1.780088in}{2.011930in}}%
\pgfusepath{stroke}%
\end{pgfscope}%
\begin{pgfscope}%
\pgfpathrectangle{\pgfqpoint{0.100000in}{0.212622in}}{\pgfqpoint{3.696000in}{3.696000in}}%
\pgfusepath{clip}%
\pgfsetrectcap%
\pgfsetroundjoin%
\pgfsetlinewidth{1.505625pt}%
\definecolor{currentstroke}{rgb}{1.000000,0.000000,0.000000}%
\pgfsetstrokecolor{currentstroke}%
\pgfsetdash{}{0pt}%
\pgfpathmoveto{\pgfqpoint{1.607638in}{2.575842in}}%
\pgfpathlineto{\pgfqpoint{1.780088in}{2.011930in}}%
\pgfusepath{stroke}%
\end{pgfscope}%
\begin{pgfscope}%
\pgfpathrectangle{\pgfqpoint{0.100000in}{0.212622in}}{\pgfqpoint{3.696000in}{3.696000in}}%
\pgfusepath{clip}%
\pgfsetrectcap%
\pgfsetroundjoin%
\pgfsetlinewidth{1.505625pt}%
\definecolor{currentstroke}{rgb}{1.000000,0.000000,0.000000}%
\pgfsetstrokecolor{currentstroke}%
\pgfsetdash{}{0pt}%
\pgfpathmoveto{\pgfqpoint{1.608237in}{2.576790in}}%
\pgfpathlineto{\pgfqpoint{1.780088in}{2.011930in}}%
\pgfusepath{stroke}%
\end{pgfscope}%
\begin{pgfscope}%
\pgfpathrectangle{\pgfqpoint{0.100000in}{0.212622in}}{\pgfqpoint{3.696000in}{3.696000in}}%
\pgfusepath{clip}%
\pgfsetrectcap%
\pgfsetroundjoin%
\pgfsetlinewidth{1.505625pt}%
\definecolor{currentstroke}{rgb}{1.000000,0.000000,0.000000}%
\pgfsetstrokecolor{currentstroke}%
\pgfsetdash{}{0pt}%
\pgfpathmoveto{\pgfqpoint{1.609497in}{2.578639in}}%
\pgfpathlineto{\pgfqpoint{1.780088in}{2.011930in}}%
\pgfusepath{stroke}%
\end{pgfscope}%
\begin{pgfscope}%
\pgfpathrectangle{\pgfqpoint{0.100000in}{0.212622in}}{\pgfqpoint{3.696000in}{3.696000in}}%
\pgfusepath{clip}%
\pgfsetrectcap%
\pgfsetroundjoin%
\pgfsetlinewidth{1.505625pt}%
\definecolor{currentstroke}{rgb}{1.000000,0.000000,0.000000}%
\pgfsetstrokecolor{currentstroke}%
\pgfsetdash{}{0pt}%
\pgfpathmoveto{\pgfqpoint{1.611188in}{2.581062in}}%
\pgfpathlineto{\pgfqpoint{1.780088in}{2.011930in}}%
\pgfusepath{stroke}%
\end{pgfscope}%
\begin{pgfscope}%
\pgfpathrectangle{\pgfqpoint{0.100000in}{0.212622in}}{\pgfqpoint{3.696000in}{3.696000in}}%
\pgfusepath{clip}%
\pgfsetrectcap%
\pgfsetroundjoin%
\pgfsetlinewidth{1.505625pt}%
\definecolor{currentstroke}{rgb}{1.000000,0.000000,0.000000}%
\pgfsetstrokecolor{currentstroke}%
\pgfsetdash{}{0pt}%
\pgfpathmoveto{\pgfqpoint{1.612081in}{2.582339in}}%
\pgfpathlineto{\pgfqpoint{1.780088in}{2.011930in}}%
\pgfusepath{stroke}%
\end{pgfscope}%
\begin{pgfscope}%
\pgfpathrectangle{\pgfqpoint{0.100000in}{0.212622in}}{\pgfqpoint{3.696000in}{3.696000in}}%
\pgfusepath{clip}%
\pgfsetrectcap%
\pgfsetroundjoin%
\pgfsetlinewidth{1.505625pt}%
\definecolor{currentstroke}{rgb}{1.000000,0.000000,0.000000}%
\pgfsetstrokecolor{currentstroke}%
\pgfsetdash{}{0pt}%
\pgfpathmoveto{\pgfqpoint{1.612550in}{2.583158in}}%
\pgfpathlineto{\pgfqpoint{1.780088in}{2.011930in}}%
\pgfusepath{stroke}%
\end{pgfscope}%
\begin{pgfscope}%
\pgfpathrectangle{\pgfqpoint{0.100000in}{0.212622in}}{\pgfqpoint{3.696000in}{3.696000in}}%
\pgfusepath{clip}%
\pgfsetrectcap%
\pgfsetroundjoin%
\pgfsetlinewidth{1.505625pt}%
\definecolor{currentstroke}{rgb}{1.000000,0.000000,0.000000}%
\pgfsetstrokecolor{currentstroke}%
\pgfsetdash{}{0pt}%
\pgfpathmoveto{\pgfqpoint{1.612795in}{2.583553in}}%
\pgfpathlineto{\pgfqpoint{1.780088in}{2.011930in}}%
\pgfusepath{stroke}%
\end{pgfscope}%
\begin{pgfscope}%
\pgfpathrectangle{\pgfqpoint{0.100000in}{0.212622in}}{\pgfqpoint{3.696000in}{3.696000in}}%
\pgfusepath{clip}%
\pgfsetrectcap%
\pgfsetroundjoin%
\pgfsetlinewidth{1.505625pt}%
\definecolor{currentstroke}{rgb}{1.000000,0.000000,0.000000}%
\pgfsetstrokecolor{currentstroke}%
\pgfsetdash{}{0pt}%
\pgfpathmoveto{\pgfqpoint{1.613283in}{2.584290in}}%
\pgfpathlineto{\pgfqpoint{1.780088in}{2.011930in}}%
\pgfusepath{stroke}%
\end{pgfscope}%
\begin{pgfscope}%
\pgfpathrectangle{\pgfqpoint{0.100000in}{0.212622in}}{\pgfqpoint{3.696000in}{3.696000in}}%
\pgfusepath{clip}%
\pgfsetrectcap%
\pgfsetroundjoin%
\pgfsetlinewidth{1.505625pt}%
\definecolor{currentstroke}{rgb}{1.000000,0.000000,0.000000}%
\pgfsetstrokecolor{currentstroke}%
\pgfsetdash{}{0pt}%
\pgfpathmoveto{\pgfqpoint{1.613574in}{2.584756in}}%
\pgfpathlineto{\pgfqpoint{1.780088in}{2.011930in}}%
\pgfusepath{stroke}%
\end{pgfscope}%
\begin{pgfscope}%
\pgfpathrectangle{\pgfqpoint{0.100000in}{0.212622in}}{\pgfqpoint{3.696000in}{3.696000in}}%
\pgfusepath{clip}%
\pgfsetrectcap%
\pgfsetroundjoin%
\pgfsetlinewidth{1.505625pt}%
\definecolor{currentstroke}{rgb}{1.000000,0.000000,0.000000}%
\pgfsetstrokecolor{currentstroke}%
\pgfsetdash{}{0pt}%
\pgfpathmoveto{\pgfqpoint{1.613749in}{2.585024in}}%
\pgfpathlineto{\pgfqpoint{1.780088in}{2.011930in}}%
\pgfusepath{stroke}%
\end{pgfscope}%
\begin{pgfscope}%
\pgfpathrectangle{\pgfqpoint{0.100000in}{0.212622in}}{\pgfqpoint{3.696000in}{3.696000in}}%
\pgfusepath{clip}%
\pgfsetrectcap%
\pgfsetroundjoin%
\pgfsetlinewidth{1.505625pt}%
\definecolor{currentstroke}{rgb}{1.000000,0.000000,0.000000}%
\pgfsetstrokecolor{currentstroke}%
\pgfsetdash{}{0pt}%
\pgfpathmoveto{\pgfqpoint{1.614279in}{2.585725in}}%
\pgfpathlineto{\pgfqpoint{1.780088in}{2.011930in}}%
\pgfusepath{stroke}%
\end{pgfscope}%
\begin{pgfscope}%
\pgfpathrectangle{\pgfqpoint{0.100000in}{0.212622in}}{\pgfqpoint{3.696000in}{3.696000in}}%
\pgfusepath{clip}%
\pgfsetrectcap%
\pgfsetroundjoin%
\pgfsetlinewidth{1.505625pt}%
\definecolor{currentstroke}{rgb}{1.000000,0.000000,0.000000}%
\pgfsetstrokecolor{currentstroke}%
\pgfsetdash{}{0pt}%
\pgfpathmoveto{\pgfqpoint{1.615741in}{2.587480in}}%
\pgfpathlineto{\pgfqpoint{1.780088in}{2.011930in}}%
\pgfusepath{stroke}%
\end{pgfscope}%
\begin{pgfscope}%
\pgfpathrectangle{\pgfqpoint{0.100000in}{0.212622in}}{\pgfqpoint{3.696000in}{3.696000in}}%
\pgfusepath{clip}%
\pgfsetrectcap%
\pgfsetroundjoin%
\pgfsetlinewidth{1.505625pt}%
\definecolor{currentstroke}{rgb}{1.000000,0.000000,0.000000}%
\pgfsetstrokecolor{currentstroke}%
\pgfsetdash{}{0pt}%
\pgfpathmoveto{\pgfqpoint{1.617758in}{2.589542in}}%
\pgfpathlineto{\pgfqpoint{1.780088in}{2.011930in}}%
\pgfusepath{stroke}%
\end{pgfscope}%
\begin{pgfscope}%
\pgfpathrectangle{\pgfqpoint{0.100000in}{0.212622in}}{\pgfqpoint{3.696000in}{3.696000in}}%
\pgfusepath{clip}%
\pgfsetrectcap%
\pgfsetroundjoin%
\pgfsetlinewidth{1.505625pt}%
\definecolor{currentstroke}{rgb}{1.000000,0.000000,0.000000}%
\pgfsetstrokecolor{currentstroke}%
\pgfsetdash{}{0pt}%
\pgfpathmoveto{\pgfqpoint{1.618921in}{2.590577in}}%
\pgfpathlineto{\pgfqpoint{1.780088in}{2.011930in}}%
\pgfusepath{stroke}%
\end{pgfscope}%
\begin{pgfscope}%
\pgfpathrectangle{\pgfqpoint{0.100000in}{0.212622in}}{\pgfqpoint{3.696000in}{3.696000in}}%
\pgfusepath{clip}%
\pgfsetrectcap%
\pgfsetroundjoin%
\pgfsetlinewidth{1.505625pt}%
\definecolor{currentstroke}{rgb}{1.000000,0.000000,0.000000}%
\pgfsetstrokecolor{currentstroke}%
\pgfsetdash{}{0pt}%
\pgfpathmoveto{\pgfqpoint{1.620640in}{2.591907in}}%
\pgfpathlineto{\pgfqpoint{1.780088in}{2.011930in}}%
\pgfusepath{stroke}%
\end{pgfscope}%
\begin{pgfscope}%
\pgfpathrectangle{\pgfqpoint{0.100000in}{0.212622in}}{\pgfqpoint{3.696000in}{3.696000in}}%
\pgfusepath{clip}%
\pgfsetrectcap%
\pgfsetroundjoin%
\pgfsetlinewidth{1.505625pt}%
\definecolor{currentstroke}{rgb}{1.000000,0.000000,0.000000}%
\pgfsetstrokecolor{currentstroke}%
\pgfsetdash{}{0pt}%
\pgfpathmoveto{\pgfqpoint{1.622971in}{2.593278in}}%
\pgfpathlineto{\pgfqpoint{1.780088in}{2.011930in}}%
\pgfusepath{stroke}%
\end{pgfscope}%
\begin{pgfscope}%
\pgfpathrectangle{\pgfqpoint{0.100000in}{0.212622in}}{\pgfqpoint{3.696000in}{3.696000in}}%
\pgfusepath{clip}%
\pgfsetrectcap%
\pgfsetroundjoin%
\pgfsetlinewidth{1.505625pt}%
\definecolor{currentstroke}{rgb}{1.000000,0.000000,0.000000}%
\pgfsetstrokecolor{currentstroke}%
\pgfsetdash{}{0pt}%
\pgfpathmoveto{\pgfqpoint{1.624311in}{2.593874in}}%
\pgfpathlineto{\pgfqpoint{1.780088in}{2.011930in}}%
\pgfusepath{stroke}%
\end{pgfscope}%
\begin{pgfscope}%
\pgfpathrectangle{\pgfqpoint{0.100000in}{0.212622in}}{\pgfqpoint{3.696000in}{3.696000in}}%
\pgfusepath{clip}%
\pgfsetrectcap%
\pgfsetroundjoin%
\pgfsetlinewidth{1.505625pt}%
\definecolor{currentstroke}{rgb}{1.000000,0.000000,0.000000}%
\pgfsetstrokecolor{currentstroke}%
\pgfsetdash{}{0pt}%
\pgfpathmoveto{\pgfqpoint{1.625083in}{2.594122in}}%
\pgfpathlineto{\pgfqpoint{1.780088in}{2.011930in}}%
\pgfusepath{stroke}%
\end{pgfscope}%
\begin{pgfscope}%
\pgfpathrectangle{\pgfqpoint{0.100000in}{0.212622in}}{\pgfqpoint{3.696000in}{3.696000in}}%
\pgfusepath{clip}%
\pgfsetrectcap%
\pgfsetroundjoin%
\pgfsetlinewidth{1.505625pt}%
\definecolor{currentstroke}{rgb}{1.000000,0.000000,0.000000}%
\pgfsetstrokecolor{currentstroke}%
\pgfsetdash{}{0pt}%
\pgfpathmoveto{\pgfqpoint{1.625529in}{2.594204in}}%
\pgfpathlineto{\pgfqpoint{1.780088in}{2.011930in}}%
\pgfusepath{stroke}%
\end{pgfscope}%
\begin{pgfscope}%
\pgfpathrectangle{\pgfqpoint{0.100000in}{0.212622in}}{\pgfqpoint{3.696000in}{3.696000in}}%
\pgfusepath{clip}%
\pgfsetrectcap%
\pgfsetroundjoin%
\pgfsetlinewidth{1.505625pt}%
\definecolor{currentstroke}{rgb}{1.000000,0.000000,0.000000}%
\pgfsetstrokecolor{currentstroke}%
\pgfsetdash{}{0pt}%
\pgfpathmoveto{\pgfqpoint{1.626399in}{2.594244in}}%
\pgfpathlineto{\pgfqpoint{1.780088in}{2.011930in}}%
\pgfusepath{stroke}%
\end{pgfscope}%
\begin{pgfscope}%
\pgfpathrectangle{\pgfqpoint{0.100000in}{0.212622in}}{\pgfqpoint{3.696000in}{3.696000in}}%
\pgfusepath{clip}%
\pgfsetrectcap%
\pgfsetroundjoin%
\pgfsetlinewidth{1.505625pt}%
\definecolor{currentstroke}{rgb}{1.000000,0.000000,0.000000}%
\pgfsetstrokecolor{currentstroke}%
\pgfsetdash{}{0pt}%
\pgfpathmoveto{\pgfqpoint{1.626898in}{2.594212in}}%
\pgfpathlineto{\pgfqpoint{1.780088in}{2.011930in}}%
\pgfusepath{stroke}%
\end{pgfscope}%
\begin{pgfscope}%
\pgfpathrectangle{\pgfqpoint{0.100000in}{0.212622in}}{\pgfqpoint{3.696000in}{3.696000in}}%
\pgfusepath{clip}%
\pgfsetrectcap%
\pgfsetroundjoin%
\pgfsetlinewidth{1.505625pt}%
\definecolor{currentstroke}{rgb}{1.000000,0.000000,0.000000}%
\pgfsetstrokecolor{currentstroke}%
\pgfsetdash{}{0pt}%
\pgfpathmoveto{\pgfqpoint{1.627183in}{2.594173in}}%
\pgfpathlineto{\pgfqpoint{1.780088in}{2.011930in}}%
\pgfusepath{stroke}%
\end{pgfscope}%
\begin{pgfscope}%
\pgfpathrectangle{\pgfqpoint{0.100000in}{0.212622in}}{\pgfqpoint{3.696000in}{3.696000in}}%
\pgfusepath{clip}%
\pgfsetrectcap%
\pgfsetroundjoin%
\pgfsetlinewidth{1.505625pt}%
\definecolor{currentstroke}{rgb}{1.000000,0.000000,0.000000}%
\pgfsetstrokecolor{currentstroke}%
\pgfsetdash{}{0pt}%
\pgfpathmoveto{\pgfqpoint{1.627344in}{2.594139in}}%
\pgfpathlineto{\pgfqpoint{1.780088in}{2.011930in}}%
\pgfusepath{stroke}%
\end{pgfscope}%
\begin{pgfscope}%
\pgfpathrectangle{\pgfqpoint{0.100000in}{0.212622in}}{\pgfqpoint{3.696000in}{3.696000in}}%
\pgfusepath{clip}%
\pgfsetrectcap%
\pgfsetroundjoin%
\pgfsetlinewidth{1.505625pt}%
\definecolor{currentstroke}{rgb}{1.000000,0.000000,0.000000}%
\pgfsetstrokecolor{currentstroke}%
\pgfsetdash{}{0pt}%
\pgfpathmoveto{\pgfqpoint{1.628022in}{2.593990in}}%
\pgfpathlineto{\pgfqpoint{1.780088in}{2.011930in}}%
\pgfusepath{stroke}%
\end{pgfscope}%
\begin{pgfscope}%
\pgfpathrectangle{\pgfqpoint{0.100000in}{0.212622in}}{\pgfqpoint{3.696000in}{3.696000in}}%
\pgfusepath{clip}%
\pgfsetrectcap%
\pgfsetroundjoin%
\pgfsetlinewidth{1.505625pt}%
\definecolor{currentstroke}{rgb}{1.000000,0.000000,0.000000}%
\pgfsetstrokecolor{currentstroke}%
\pgfsetdash{}{0pt}%
\pgfpathmoveto{\pgfqpoint{1.630851in}{2.593093in}}%
\pgfpathlineto{\pgfqpoint{1.780088in}{2.011930in}}%
\pgfusepath{stroke}%
\end{pgfscope}%
\begin{pgfscope}%
\pgfpathrectangle{\pgfqpoint{0.100000in}{0.212622in}}{\pgfqpoint{3.696000in}{3.696000in}}%
\pgfusepath{clip}%
\pgfsetrectcap%
\pgfsetroundjoin%
\pgfsetlinewidth{1.505625pt}%
\definecolor{currentstroke}{rgb}{1.000000,0.000000,0.000000}%
\pgfsetstrokecolor{currentstroke}%
\pgfsetdash{}{0pt}%
\pgfpathmoveto{\pgfqpoint{1.634803in}{2.591804in}}%
\pgfpathlineto{\pgfqpoint{1.780088in}{2.011930in}}%
\pgfusepath{stroke}%
\end{pgfscope}%
\begin{pgfscope}%
\pgfpathrectangle{\pgfqpoint{0.100000in}{0.212622in}}{\pgfqpoint{3.696000in}{3.696000in}}%
\pgfusepath{clip}%
\pgfsetrectcap%
\pgfsetroundjoin%
\pgfsetlinewidth{1.505625pt}%
\definecolor{currentstroke}{rgb}{1.000000,0.000000,0.000000}%
\pgfsetstrokecolor{currentstroke}%
\pgfsetdash{}{0pt}%
\pgfpathmoveto{\pgfqpoint{1.639993in}{2.590466in}}%
\pgfpathlineto{\pgfqpoint{1.780088in}{2.011930in}}%
\pgfusepath{stroke}%
\end{pgfscope}%
\begin{pgfscope}%
\pgfpathrectangle{\pgfqpoint{0.100000in}{0.212622in}}{\pgfqpoint{3.696000in}{3.696000in}}%
\pgfusepath{clip}%
\pgfsetrectcap%
\pgfsetroundjoin%
\pgfsetlinewidth{1.505625pt}%
\definecolor{currentstroke}{rgb}{1.000000,0.000000,0.000000}%
\pgfsetstrokecolor{currentstroke}%
\pgfsetdash{}{0pt}%
\pgfpathmoveto{\pgfqpoint{1.645976in}{2.588859in}}%
\pgfpathlineto{\pgfqpoint{1.780088in}{2.011930in}}%
\pgfusepath{stroke}%
\end{pgfscope}%
\begin{pgfscope}%
\pgfpathrectangle{\pgfqpoint{0.100000in}{0.212622in}}{\pgfqpoint{3.696000in}{3.696000in}}%
\pgfusepath{clip}%
\pgfsetrectcap%
\pgfsetroundjoin%
\pgfsetlinewidth{1.505625pt}%
\definecolor{currentstroke}{rgb}{1.000000,0.000000,0.000000}%
\pgfsetstrokecolor{currentstroke}%
\pgfsetdash{}{0pt}%
\pgfpathmoveto{\pgfqpoint{1.649283in}{2.588043in}}%
\pgfpathlineto{\pgfqpoint{1.780088in}{2.011930in}}%
\pgfusepath{stroke}%
\end{pgfscope}%
\begin{pgfscope}%
\pgfpathrectangle{\pgfqpoint{0.100000in}{0.212622in}}{\pgfqpoint{3.696000in}{3.696000in}}%
\pgfusepath{clip}%
\pgfsetrectcap%
\pgfsetroundjoin%
\pgfsetlinewidth{1.505625pt}%
\definecolor{currentstroke}{rgb}{1.000000,0.000000,0.000000}%
\pgfsetstrokecolor{currentstroke}%
\pgfsetdash{}{0pt}%
\pgfpathmoveto{\pgfqpoint{1.651119in}{2.587543in}}%
\pgfpathlineto{\pgfqpoint{1.780088in}{2.011930in}}%
\pgfusepath{stroke}%
\end{pgfscope}%
\begin{pgfscope}%
\pgfpathrectangle{\pgfqpoint{0.100000in}{0.212622in}}{\pgfqpoint{3.696000in}{3.696000in}}%
\pgfusepath{clip}%
\pgfsetrectcap%
\pgfsetroundjoin%
\pgfsetlinewidth{1.505625pt}%
\definecolor{currentstroke}{rgb}{1.000000,0.000000,0.000000}%
\pgfsetstrokecolor{currentstroke}%
\pgfsetdash{}{0pt}%
\pgfpathmoveto{\pgfqpoint{1.653519in}{2.586894in}}%
\pgfpathlineto{\pgfqpoint{1.780088in}{2.011930in}}%
\pgfusepath{stroke}%
\end{pgfscope}%
\begin{pgfscope}%
\pgfpathrectangle{\pgfqpoint{0.100000in}{0.212622in}}{\pgfqpoint{3.696000in}{3.696000in}}%
\pgfusepath{clip}%
\pgfsetrectcap%
\pgfsetroundjoin%
\pgfsetlinewidth{1.505625pt}%
\definecolor{currentstroke}{rgb}{1.000000,0.000000,0.000000}%
\pgfsetstrokecolor{currentstroke}%
\pgfsetdash{}{0pt}%
\pgfpathmoveto{\pgfqpoint{1.654843in}{2.586514in}}%
\pgfpathlineto{\pgfqpoint{1.780088in}{2.011930in}}%
\pgfusepath{stroke}%
\end{pgfscope}%
\begin{pgfscope}%
\pgfpathrectangle{\pgfqpoint{0.100000in}{0.212622in}}{\pgfqpoint{3.696000in}{3.696000in}}%
\pgfusepath{clip}%
\pgfsetrectcap%
\pgfsetroundjoin%
\pgfsetlinewidth{1.505625pt}%
\definecolor{currentstroke}{rgb}{1.000000,0.000000,0.000000}%
\pgfsetstrokecolor{currentstroke}%
\pgfsetdash{}{0pt}%
\pgfpathmoveto{\pgfqpoint{1.655570in}{2.586306in}}%
\pgfpathlineto{\pgfqpoint{1.780088in}{2.011930in}}%
\pgfusepath{stroke}%
\end{pgfscope}%
\begin{pgfscope}%
\pgfpathrectangle{\pgfqpoint{0.100000in}{0.212622in}}{\pgfqpoint{3.696000in}{3.696000in}}%
\pgfusepath{clip}%
\pgfsetrectcap%
\pgfsetroundjoin%
\pgfsetlinewidth{1.505625pt}%
\definecolor{currentstroke}{rgb}{1.000000,0.000000,0.000000}%
\pgfsetstrokecolor{currentstroke}%
\pgfsetdash{}{0pt}%
\pgfpathmoveto{\pgfqpoint{1.657550in}{2.585832in}}%
\pgfpathlineto{\pgfqpoint{1.780088in}{2.011930in}}%
\pgfusepath{stroke}%
\end{pgfscope}%
\begin{pgfscope}%
\pgfpathrectangle{\pgfqpoint{0.100000in}{0.212622in}}{\pgfqpoint{3.696000in}{3.696000in}}%
\pgfusepath{clip}%
\pgfsetrectcap%
\pgfsetroundjoin%
\pgfsetlinewidth{1.505625pt}%
\definecolor{currentstroke}{rgb}{1.000000,0.000000,0.000000}%
\pgfsetstrokecolor{currentstroke}%
\pgfsetdash{}{0pt}%
\pgfpathmoveto{\pgfqpoint{1.658632in}{2.585492in}}%
\pgfpathlineto{\pgfqpoint{1.780088in}{2.011930in}}%
\pgfusepath{stroke}%
\end{pgfscope}%
\begin{pgfscope}%
\pgfpathrectangle{\pgfqpoint{0.100000in}{0.212622in}}{\pgfqpoint{3.696000in}{3.696000in}}%
\pgfusepath{clip}%
\pgfsetrectcap%
\pgfsetroundjoin%
\pgfsetlinewidth{1.505625pt}%
\definecolor{currentstroke}{rgb}{1.000000,0.000000,0.000000}%
\pgfsetstrokecolor{currentstroke}%
\pgfsetdash{}{0pt}%
\pgfpathmoveto{\pgfqpoint{1.663026in}{2.584083in}}%
\pgfpathlineto{\pgfqpoint{1.780088in}{2.011930in}}%
\pgfusepath{stroke}%
\end{pgfscope}%
\begin{pgfscope}%
\pgfpathrectangle{\pgfqpoint{0.100000in}{0.212622in}}{\pgfqpoint{3.696000in}{3.696000in}}%
\pgfusepath{clip}%
\pgfsetrectcap%
\pgfsetroundjoin%
\pgfsetlinewidth{1.505625pt}%
\definecolor{currentstroke}{rgb}{1.000000,0.000000,0.000000}%
\pgfsetstrokecolor{currentstroke}%
\pgfsetdash{}{0pt}%
\pgfpathmoveto{\pgfqpoint{1.668109in}{2.581930in}}%
\pgfpathlineto{\pgfqpoint{1.780088in}{2.011930in}}%
\pgfusepath{stroke}%
\end{pgfscope}%
\begin{pgfscope}%
\pgfpathrectangle{\pgfqpoint{0.100000in}{0.212622in}}{\pgfqpoint{3.696000in}{3.696000in}}%
\pgfusepath{clip}%
\pgfsetrectcap%
\pgfsetroundjoin%
\pgfsetlinewidth{1.505625pt}%
\definecolor{currentstroke}{rgb}{1.000000,0.000000,0.000000}%
\pgfsetstrokecolor{currentstroke}%
\pgfsetdash{}{0pt}%
\pgfpathmoveto{\pgfqpoint{1.670887in}{2.581036in}}%
\pgfpathlineto{\pgfqpoint{1.780088in}{2.011930in}}%
\pgfusepath{stroke}%
\end{pgfscope}%
\begin{pgfscope}%
\pgfpathrectangle{\pgfqpoint{0.100000in}{0.212622in}}{\pgfqpoint{3.696000in}{3.696000in}}%
\pgfusepath{clip}%
\pgfsetrectcap%
\pgfsetroundjoin%
\pgfsetlinewidth{1.505625pt}%
\definecolor{currentstroke}{rgb}{1.000000,0.000000,0.000000}%
\pgfsetstrokecolor{currentstroke}%
\pgfsetdash{}{0pt}%
\pgfpathmoveto{\pgfqpoint{1.674718in}{2.579718in}}%
\pgfpathlineto{\pgfqpoint{1.780088in}{2.011930in}}%
\pgfusepath{stroke}%
\end{pgfscope}%
\begin{pgfscope}%
\pgfpathrectangle{\pgfqpoint{0.100000in}{0.212622in}}{\pgfqpoint{3.696000in}{3.696000in}}%
\pgfusepath{clip}%
\pgfsetrectcap%
\pgfsetroundjoin%
\pgfsetlinewidth{1.505625pt}%
\definecolor{currentstroke}{rgb}{1.000000,0.000000,0.000000}%
\pgfsetstrokecolor{currentstroke}%
\pgfsetdash{}{0pt}%
\pgfpathmoveto{\pgfqpoint{1.681040in}{2.577356in}}%
\pgfpathlineto{\pgfqpoint{1.780088in}{2.011930in}}%
\pgfusepath{stroke}%
\end{pgfscope}%
\begin{pgfscope}%
\pgfpathrectangle{\pgfqpoint{0.100000in}{0.212622in}}{\pgfqpoint{3.696000in}{3.696000in}}%
\pgfusepath{clip}%
\pgfsetrectcap%
\pgfsetroundjoin%
\pgfsetlinewidth{1.505625pt}%
\definecolor{currentstroke}{rgb}{1.000000,0.000000,0.000000}%
\pgfsetstrokecolor{currentstroke}%
\pgfsetdash{}{0pt}%
\pgfpathmoveto{\pgfqpoint{1.684480in}{2.576554in}}%
\pgfpathlineto{\pgfqpoint{1.780088in}{2.011930in}}%
\pgfusepath{stroke}%
\end{pgfscope}%
\begin{pgfscope}%
\pgfpathrectangle{\pgfqpoint{0.100000in}{0.212622in}}{\pgfqpoint{3.696000in}{3.696000in}}%
\pgfusepath{clip}%
\pgfsetrectcap%
\pgfsetroundjoin%
\pgfsetlinewidth{1.505625pt}%
\definecolor{currentstroke}{rgb}{1.000000,0.000000,0.000000}%
\pgfsetstrokecolor{currentstroke}%
\pgfsetdash{}{0pt}%
\pgfpathmoveto{\pgfqpoint{1.688936in}{2.575657in}}%
\pgfpathlineto{\pgfqpoint{1.780088in}{2.011930in}}%
\pgfusepath{stroke}%
\end{pgfscope}%
\begin{pgfscope}%
\pgfpathrectangle{\pgfqpoint{0.100000in}{0.212622in}}{\pgfqpoint{3.696000in}{3.696000in}}%
\pgfusepath{clip}%
\pgfsetrectcap%
\pgfsetroundjoin%
\pgfsetlinewidth{1.505625pt}%
\definecolor{currentstroke}{rgb}{1.000000,0.000000,0.000000}%
\pgfsetstrokecolor{currentstroke}%
\pgfsetdash{}{0pt}%
\pgfpathmoveto{\pgfqpoint{1.694850in}{2.574470in}}%
\pgfpathlineto{\pgfqpoint{1.780088in}{2.011930in}}%
\pgfusepath{stroke}%
\end{pgfscope}%
\begin{pgfscope}%
\pgfpathrectangle{\pgfqpoint{0.100000in}{0.212622in}}{\pgfqpoint{3.696000in}{3.696000in}}%
\pgfusepath{clip}%
\pgfsetrectcap%
\pgfsetroundjoin%
\pgfsetlinewidth{1.505625pt}%
\definecolor{currentstroke}{rgb}{1.000000,0.000000,0.000000}%
\pgfsetstrokecolor{currentstroke}%
\pgfsetdash{}{0pt}%
\pgfpathmoveto{\pgfqpoint{1.704325in}{2.572204in}}%
\pgfpathlineto{\pgfqpoint{1.780088in}{2.011930in}}%
\pgfusepath{stroke}%
\end{pgfscope}%
\begin{pgfscope}%
\pgfpathrectangle{\pgfqpoint{0.100000in}{0.212622in}}{\pgfqpoint{3.696000in}{3.696000in}}%
\pgfusepath{clip}%
\pgfsetrectcap%
\pgfsetroundjoin%
\pgfsetlinewidth{1.505625pt}%
\definecolor{currentstroke}{rgb}{1.000000,0.000000,0.000000}%
\pgfsetstrokecolor{currentstroke}%
\pgfsetdash{}{0pt}%
\pgfpathmoveto{\pgfqpoint{1.714401in}{2.569100in}}%
\pgfpathlineto{\pgfqpoint{1.780088in}{2.011930in}}%
\pgfusepath{stroke}%
\end{pgfscope}%
\begin{pgfscope}%
\pgfpathrectangle{\pgfqpoint{0.100000in}{0.212622in}}{\pgfqpoint{3.696000in}{3.696000in}}%
\pgfusepath{clip}%
\pgfsetrectcap%
\pgfsetroundjoin%
\pgfsetlinewidth{1.505625pt}%
\definecolor{currentstroke}{rgb}{1.000000,0.000000,0.000000}%
\pgfsetstrokecolor{currentstroke}%
\pgfsetdash{}{0pt}%
\pgfpathmoveto{\pgfqpoint{1.725386in}{2.566109in}}%
\pgfpathlineto{\pgfqpoint{1.780088in}{2.011930in}}%
\pgfusepath{stroke}%
\end{pgfscope}%
\begin{pgfscope}%
\pgfpathrectangle{\pgfqpoint{0.100000in}{0.212622in}}{\pgfqpoint{3.696000in}{3.696000in}}%
\pgfusepath{clip}%
\pgfsetrectcap%
\pgfsetroundjoin%
\pgfsetlinewidth{1.505625pt}%
\definecolor{currentstroke}{rgb}{1.000000,0.000000,0.000000}%
\pgfsetstrokecolor{currentstroke}%
\pgfsetdash{}{0pt}%
\pgfpathmoveto{\pgfqpoint{1.736842in}{2.563207in}}%
\pgfpathlineto{\pgfqpoint{1.780088in}{2.011930in}}%
\pgfusepath{stroke}%
\end{pgfscope}%
\begin{pgfscope}%
\pgfpathrectangle{\pgfqpoint{0.100000in}{0.212622in}}{\pgfqpoint{3.696000in}{3.696000in}}%
\pgfusepath{clip}%
\pgfsetrectcap%
\pgfsetroundjoin%
\pgfsetlinewidth{1.505625pt}%
\definecolor{currentstroke}{rgb}{1.000000,0.000000,0.000000}%
\pgfsetstrokecolor{currentstroke}%
\pgfsetdash{}{0pt}%
\pgfpathmoveto{\pgfqpoint{1.750602in}{2.561463in}}%
\pgfpathlineto{\pgfqpoint{1.780088in}{2.011930in}}%
\pgfusepath{stroke}%
\end{pgfscope}%
\begin{pgfscope}%
\pgfpathrectangle{\pgfqpoint{0.100000in}{0.212622in}}{\pgfqpoint{3.696000in}{3.696000in}}%
\pgfusepath{clip}%
\pgfsetrectcap%
\pgfsetroundjoin%
\pgfsetlinewidth{1.505625pt}%
\definecolor{currentstroke}{rgb}{1.000000,0.000000,0.000000}%
\pgfsetstrokecolor{currentstroke}%
\pgfsetdash{}{0pt}%
\pgfpathmoveto{\pgfqpoint{1.758097in}{2.561142in}}%
\pgfpathlineto{\pgfqpoint{1.780088in}{2.011930in}}%
\pgfusepath{stroke}%
\end{pgfscope}%
\begin{pgfscope}%
\pgfpathrectangle{\pgfqpoint{0.100000in}{0.212622in}}{\pgfqpoint{3.696000in}{3.696000in}}%
\pgfusepath{clip}%
\pgfsetrectcap%
\pgfsetroundjoin%
\pgfsetlinewidth{1.505625pt}%
\definecolor{currentstroke}{rgb}{1.000000,0.000000,0.000000}%
\pgfsetstrokecolor{currentstroke}%
\pgfsetdash{}{0pt}%
\pgfpathmoveto{\pgfqpoint{1.766567in}{2.560609in}}%
\pgfpathlineto{\pgfqpoint{1.780088in}{2.011930in}}%
\pgfusepath{stroke}%
\end{pgfscope}%
\begin{pgfscope}%
\pgfpathrectangle{\pgfqpoint{0.100000in}{0.212622in}}{\pgfqpoint{3.696000in}{3.696000in}}%
\pgfusepath{clip}%
\pgfsetrectcap%
\pgfsetroundjoin%
\pgfsetlinewidth{1.505625pt}%
\definecolor{currentstroke}{rgb}{1.000000,0.000000,0.000000}%
\pgfsetstrokecolor{currentstroke}%
\pgfsetdash{}{0pt}%
\pgfpathmoveto{\pgfqpoint{1.775798in}{2.560088in}}%
\pgfpathlineto{\pgfqpoint{1.780088in}{2.011930in}}%
\pgfusepath{stroke}%
\end{pgfscope}%
\begin{pgfscope}%
\pgfpathrectangle{\pgfqpoint{0.100000in}{0.212622in}}{\pgfqpoint{3.696000in}{3.696000in}}%
\pgfusepath{clip}%
\pgfsetrectcap%
\pgfsetroundjoin%
\pgfsetlinewidth{1.505625pt}%
\definecolor{currentstroke}{rgb}{1.000000,0.000000,0.000000}%
\pgfsetstrokecolor{currentstroke}%
\pgfsetdash{}{0pt}%
\pgfpathmoveto{\pgfqpoint{1.788894in}{2.558837in}}%
\pgfpathlineto{\pgfqpoint{1.780088in}{2.011930in}}%
\pgfusepath{stroke}%
\end{pgfscope}%
\begin{pgfscope}%
\pgfpathrectangle{\pgfqpoint{0.100000in}{0.212622in}}{\pgfqpoint{3.696000in}{3.696000in}}%
\pgfusepath{clip}%
\pgfsetrectcap%
\pgfsetroundjoin%
\pgfsetlinewidth{1.505625pt}%
\definecolor{currentstroke}{rgb}{1.000000,0.000000,0.000000}%
\pgfsetstrokecolor{currentstroke}%
\pgfsetdash{}{0pt}%
\pgfpathmoveto{\pgfqpoint{1.803623in}{2.557349in}}%
\pgfpathlineto{\pgfqpoint{1.780088in}{2.011930in}}%
\pgfusepath{stroke}%
\end{pgfscope}%
\begin{pgfscope}%
\pgfpathrectangle{\pgfqpoint{0.100000in}{0.212622in}}{\pgfqpoint{3.696000in}{3.696000in}}%
\pgfusepath{clip}%
\pgfsetrectcap%
\pgfsetroundjoin%
\pgfsetlinewidth{1.505625pt}%
\definecolor{currentstroke}{rgb}{1.000000,0.000000,0.000000}%
\pgfsetstrokecolor{currentstroke}%
\pgfsetdash{}{0pt}%
\pgfpathmoveto{\pgfqpoint{1.819160in}{2.556350in}}%
\pgfpathlineto{\pgfqpoint{1.780088in}{2.011930in}}%
\pgfusepath{stroke}%
\end{pgfscope}%
\begin{pgfscope}%
\pgfpathrectangle{\pgfqpoint{0.100000in}{0.212622in}}{\pgfqpoint{3.696000in}{3.696000in}}%
\pgfusepath{clip}%
\pgfsetrectcap%
\pgfsetroundjoin%
\pgfsetlinewidth{1.505625pt}%
\definecolor{currentstroke}{rgb}{1.000000,0.000000,0.000000}%
\pgfsetstrokecolor{currentstroke}%
\pgfsetdash{}{0pt}%
\pgfpathmoveto{\pgfqpoint{1.835147in}{2.554196in}}%
\pgfpathlineto{\pgfqpoint{1.780088in}{2.011930in}}%
\pgfusepath{stroke}%
\end{pgfscope}%
\begin{pgfscope}%
\pgfpathrectangle{\pgfqpoint{0.100000in}{0.212622in}}{\pgfqpoint{3.696000in}{3.696000in}}%
\pgfusepath{clip}%
\pgfsetrectcap%
\pgfsetroundjoin%
\pgfsetlinewidth{1.505625pt}%
\definecolor{currentstroke}{rgb}{1.000000,0.000000,0.000000}%
\pgfsetstrokecolor{currentstroke}%
\pgfsetdash{}{0pt}%
\pgfpathmoveto{\pgfqpoint{1.853172in}{2.551062in}}%
\pgfpathlineto{\pgfqpoint{1.780088in}{2.011930in}}%
\pgfusepath{stroke}%
\end{pgfscope}%
\begin{pgfscope}%
\pgfpathrectangle{\pgfqpoint{0.100000in}{0.212622in}}{\pgfqpoint{3.696000in}{3.696000in}}%
\pgfusepath{clip}%
\pgfsetrectcap%
\pgfsetroundjoin%
\pgfsetlinewidth{1.505625pt}%
\definecolor{currentstroke}{rgb}{1.000000,0.000000,0.000000}%
\pgfsetstrokecolor{currentstroke}%
\pgfsetdash{}{0pt}%
\pgfpathmoveto{\pgfqpoint{1.863193in}{2.549899in}}%
\pgfpathlineto{\pgfqpoint{1.780088in}{2.011930in}}%
\pgfusepath{stroke}%
\end{pgfscope}%
\begin{pgfscope}%
\pgfpathrectangle{\pgfqpoint{0.100000in}{0.212622in}}{\pgfqpoint{3.696000in}{3.696000in}}%
\pgfusepath{clip}%
\pgfsetrectcap%
\pgfsetroundjoin%
\pgfsetlinewidth{1.505625pt}%
\definecolor{currentstroke}{rgb}{1.000000,0.000000,0.000000}%
\pgfsetstrokecolor{currentstroke}%
\pgfsetdash{}{0pt}%
\pgfpathmoveto{\pgfqpoint{1.873994in}{2.548927in}}%
\pgfpathlineto{\pgfqpoint{1.793452in}{2.008018in}}%
\pgfusepath{stroke}%
\end{pgfscope}%
\begin{pgfscope}%
\pgfpathrectangle{\pgfqpoint{0.100000in}{0.212622in}}{\pgfqpoint{3.696000in}{3.696000in}}%
\pgfusepath{clip}%
\pgfsetrectcap%
\pgfsetroundjoin%
\pgfsetlinewidth{1.505625pt}%
\definecolor{currentstroke}{rgb}{1.000000,0.000000,0.000000}%
\pgfsetstrokecolor{currentstroke}%
\pgfsetdash{}{0pt}%
\pgfpathmoveto{\pgfqpoint{1.879873in}{2.548201in}}%
\pgfpathlineto{\pgfqpoint{1.806826in}{2.004103in}}%
\pgfusepath{stroke}%
\end{pgfscope}%
\begin{pgfscope}%
\pgfpathrectangle{\pgfqpoint{0.100000in}{0.212622in}}{\pgfqpoint{3.696000in}{3.696000in}}%
\pgfusepath{clip}%
\pgfsetrectcap%
\pgfsetroundjoin%
\pgfsetlinewidth{1.505625pt}%
\definecolor{currentstroke}{rgb}{1.000000,0.000000,0.000000}%
\pgfsetstrokecolor{currentstroke}%
\pgfsetdash{}{0pt}%
\pgfpathmoveto{\pgfqpoint{1.888251in}{2.546781in}}%
\pgfpathlineto{\pgfqpoint{1.806826in}{2.004103in}}%
\pgfusepath{stroke}%
\end{pgfscope}%
\begin{pgfscope}%
\pgfpathrectangle{\pgfqpoint{0.100000in}{0.212622in}}{\pgfqpoint{3.696000in}{3.696000in}}%
\pgfusepath{clip}%
\pgfsetrectcap%
\pgfsetroundjoin%
\pgfsetlinewidth{1.505625pt}%
\definecolor{currentstroke}{rgb}{1.000000,0.000000,0.000000}%
\pgfsetstrokecolor{currentstroke}%
\pgfsetdash{}{0pt}%
\pgfpathmoveto{\pgfqpoint{1.897857in}{2.545345in}}%
\pgfpathlineto{\pgfqpoint{1.820208in}{2.000185in}}%
\pgfusepath{stroke}%
\end{pgfscope}%
\begin{pgfscope}%
\pgfpathrectangle{\pgfqpoint{0.100000in}{0.212622in}}{\pgfqpoint{3.696000in}{3.696000in}}%
\pgfusepath{clip}%
\pgfsetrectcap%
\pgfsetroundjoin%
\pgfsetlinewidth{1.505625pt}%
\definecolor{currentstroke}{rgb}{1.000000,0.000000,0.000000}%
\pgfsetstrokecolor{currentstroke}%
\pgfsetdash{}{0pt}%
\pgfpathmoveto{\pgfqpoint{1.909247in}{2.544079in}}%
\pgfpathlineto{\pgfqpoint{1.833600in}{1.996265in}}%
\pgfusepath{stroke}%
\end{pgfscope}%
\begin{pgfscope}%
\pgfpathrectangle{\pgfqpoint{0.100000in}{0.212622in}}{\pgfqpoint{3.696000in}{3.696000in}}%
\pgfusepath{clip}%
\pgfsetrectcap%
\pgfsetroundjoin%
\pgfsetlinewidth{1.505625pt}%
\definecolor{currentstroke}{rgb}{1.000000,0.000000,0.000000}%
\pgfsetstrokecolor{currentstroke}%
\pgfsetdash{}{0pt}%
\pgfpathmoveto{\pgfqpoint{1.921442in}{2.542925in}}%
\pgfpathlineto{\pgfqpoint{1.847000in}{1.992342in}}%
\pgfusepath{stroke}%
\end{pgfscope}%
\begin{pgfscope}%
\pgfpathrectangle{\pgfqpoint{0.100000in}{0.212622in}}{\pgfqpoint{3.696000in}{3.696000in}}%
\pgfusepath{clip}%
\pgfsetrectcap%
\pgfsetroundjoin%
\pgfsetlinewidth{1.505625pt}%
\definecolor{currentstroke}{rgb}{1.000000,0.000000,0.000000}%
\pgfsetstrokecolor{currentstroke}%
\pgfsetdash{}{0pt}%
\pgfpathmoveto{\pgfqpoint{1.935793in}{2.541170in}}%
\pgfpathlineto{\pgfqpoint{1.860410in}{1.988416in}}%
\pgfusepath{stroke}%
\end{pgfscope}%
\begin{pgfscope}%
\pgfpathrectangle{\pgfqpoint{0.100000in}{0.212622in}}{\pgfqpoint{3.696000in}{3.696000in}}%
\pgfusepath{clip}%
\pgfsetrectcap%
\pgfsetroundjoin%
\pgfsetlinewidth{1.505625pt}%
\definecolor{currentstroke}{rgb}{1.000000,0.000000,0.000000}%
\pgfsetstrokecolor{currentstroke}%
\pgfsetdash{}{0pt}%
\pgfpathmoveto{\pgfqpoint{1.950600in}{2.541158in}}%
\pgfpathlineto{\pgfqpoint{1.873829in}{1.984488in}}%
\pgfusepath{stroke}%
\end{pgfscope}%
\begin{pgfscope}%
\pgfpathrectangle{\pgfqpoint{0.100000in}{0.212622in}}{\pgfqpoint{3.696000in}{3.696000in}}%
\pgfusepath{clip}%
\pgfsetrectcap%
\pgfsetroundjoin%
\pgfsetlinewidth{1.505625pt}%
\definecolor{currentstroke}{rgb}{1.000000,0.000000,0.000000}%
\pgfsetstrokecolor{currentstroke}%
\pgfsetdash{}{0pt}%
\pgfpathmoveto{\pgfqpoint{1.958676in}{2.540108in}}%
\pgfpathlineto{\pgfqpoint{1.887256in}{1.980557in}}%
\pgfusepath{stroke}%
\end{pgfscope}%
\begin{pgfscope}%
\pgfpathrectangle{\pgfqpoint{0.100000in}{0.212622in}}{\pgfqpoint{3.696000in}{3.696000in}}%
\pgfusepath{clip}%
\pgfsetrectcap%
\pgfsetroundjoin%
\pgfsetlinewidth{1.505625pt}%
\definecolor{currentstroke}{rgb}{1.000000,0.000000,0.000000}%
\pgfsetstrokecolor{currentstroke}%
\pgfsetdash{}{0pt}%
\pgfpathmoveto{\pgfqpoint{1.967232in}{2.539411in}}%
\pgfpathlineto{\pgfqpoint{1.887256in}{1.980557in}}%
\pgfusepath{stroke}%
\end{pgfscope}%
\begin{pgfscope}%
\pgfpathrectangle{\pgfqpoint{0.100000in}{0.212622in}}{\pgfqpoint{3.696000in}{3.696000in}}%
\pgfusepath{clip}%
\pgfsetrectcap%
\pgfsetroundjoin%
\pgfsetlinewidth{1.505625pt}%
\definecolor{currentstroke}{rgb}{1.000000,0.000000,0.000000}%
\pgfsetstrokecolor{currentstroke}%
\pgfsetdash{}{0pt}%
\pgfpathmoveto{\pgfqpoint{1.971928in}{2.538839in}}%
\pgfpathlineto{\pgfqpoint{1.900693in}{1.976624in}}%
\pgfusepath{stroke}%
\end{pgfscope}%
\begin{pgfscope}%
\pgfpathrectangle{\pgfqpoint{0.100000in}{0.212622in}}{\pgfqpoint{3.696000in}{3.696000in}}%
\pgfusepath{clip}%
\pgfsetrectcap%
\pgfsetroundjoin%
\pgfsetlinewidth{1.505625pt}%
\definecolor{currentstroke}{rgb}{1.000000,0.000000,0.000000}%
\pgfsetstrokecolor{currentstroke}%
\pgfsetdash{}{0pt}%
\pgfpathmoveto{\pgfqpoint{1.979137in}{2.538108in}}%
\pgfpathlineto{\pgfqpoint{1.900693in}{1.976624in}}%
\pgfusepath{stroke}%
\end{pgfscope}%
\begin{pgfscope}%
\pgfpathrectangle{\pgfqpoint{0.100000in}{0.212622in}}{\pgfqpoint{3.696000in}{3.696000in}}%
\pgfusepath{clip}%
\pgfsetrectcap%
\pgfsetroundjoin%
\pgfsetlinewidth{1.505625pt}%
\definecolor{currentstroke}{rgb}{1.000000,0.000000,0.000000}%
\pgfsetstrokecolor{currentstroke}%
\pgfsetdash{}{0pt}%
\pgfpathmoveto{\pgfqpoint{1.986862in}{2.537954in}}%
\pgfpathlineto{\pgfqpoint{1.914139in}{1.972688in}}%
\pgfusepath{stroke}%
\end{pgfscope}%
\begin{pgfscope}%
\pgfpathrectangle{\pgfqpoint{0.100000in}{0.212622in}}{\pgfqpoint{3.696000in}{3.696000in}}%
\pgfusepath{clip}%
\pgfsetrectcap%
\pgfsetroundjoin%
\pgfsetlinewidth{1.505625pt}%
\definecolor{currentstroke}{rgb}{1.000000,0.000000,0.000000}%
\pgfsetstrokecolor{currentstroke}%
\pgfsetdash{}{0pt}%
\pgfpathmoveto{\pgfqpoint{1.995356in}{2.537234in}}%
\pgfpathlineto{\pgfqpoint{1.914139in}{1.972688in}}%
\pgfusepath{stroke}%
\end{pgfscope}%
\begin{pgfscope}%
\pgfpathrectangle{\pgfqpoint{0.100000in}{0.212622in}}{\pgfqpoint{3.696000in}{3.696000in}}%
\pgfusepath{clip}%
\pgfsetrectcap%
\pgfsetroundjoin%
\pgfsetlinewidth{1.505625pt}%
\definecolor{currentstroke}{rgb}{1.000000,0.000000,0.000000}%
\pgfsetstrokecolor{currentstroke}%
\pgfsetdash{}{0pt}%
\pgfpathmoveto{\pgfqpoint{1.999999in}{2.536812in}}%
\pgfpathlineto{\pgfqpoint{1.927594in}{1.968749in}}%
\pgfusepath{stroke}%
\end{pgfscope}%
\begin{pgfscope}%
\pgfpathrectangle{\pgfqpoint{0.100000in}{0.212622in}}{\pgfqpoint{3.696000in}{3.696000in}}%
\pgfusepath{clip}%
\pgfsetrectcap%
\pgfsetroundjoin%
\pgfsetlinewidth{1.505625pt}%
\definecolor{currentstroke}{rgb}{1.000000,0.000000,0.000000}%
\pgfsetstrokecolor{currentstroke}%
\pgfsetdash{}{0pt}%
\pgfpathmoveto{\pgfqpoint{2.007024in}{2.535569in}}%
\pgfpathlineto{\pgfqpoint{1.927594in}{1.968749in}}%
\pgfusepath{stroke}%
\end{pgfscope}%
\begin{pgfscope}%
\pgfpathrectangle{\pgfqpoint{0.100000in}{0.212622in}}{\pgfqpoint{3.696000in}{3.696000in}}%
\pgfusepath{clip}%
\pgfsetrectcap%
\pgfsetroundjoin%
\pgfsetlinewidth{1.505625pt}%
\definecolor{currentstroke}{rgb}{1.000000,0.000000,0.000000}%
\pgfsetstrokecolor{currentstroke}%
\pgfsetdash{}{0pt}%
\pgfpathmoveto{\pgfqpoint{2.014807in}{2.534727in}}%
\pgfpathlineto{\pgfqpoint{1.941058in}{1.964808in}}%
\pgfusepath{stroke}%
\end{pgfscope}%
\begin{pgfscope}%
\pgfpathrectangle{\pgfqpoint{0.100000in}{0.212622in}}{\pgfqpoint{3.696000in}{3.696000in}}%
\pgfusepath{clip}%
\pgfsetrectcap%
\pgfsetroundjoin%
\pgfsetlinewidth{1.505625pt}%
\definecolor{currentstroke}{rgb}{1.000000,0.000000,0.000000}%
\pgfsetstrokecolor{currentstroke}%
\pgfsetdash{}{0pt}%
\pgfpathmoveto{\pgfqpoint{2.023242in}{2.534151in}}%
\pgfpathlineto{\pgfqpoint{1.954531in}{1.960864in}}%
\pgfusepath{stroke}%
\end{pgfscope}%
\begin{pgfscope}%
\pgfpathrectangle{\pgfqpoint{0.100000in}{0.212622in}}{\pgfqpoint{3.696000in}{3.696000in}}%
\pgfusepath{clip}%
\pgfsetrectcap%
\pgfsetroundjoin%
\pgfsetlinewidth{1.505625pt}%
\definecolor{currentstroke}{rgb}{1.000000,0.000000,0.000000}%
\pgfsetstrokecolor{currentstroke}%
\pgfsetdash{}{0pt}%
\pgfpathmoveto{\pgfqpoint{2.032304in}{2.533261in}}%
\pgfpathlineto{\pgfqpoint{1.954531in}{1.960864in}}%
\pgfusepath{stroke}%
\end{pgfscope}%
\begin{pgfscope}%
\pgfpathrectangle{\pgfqpoint{0.100000in}{0.212622in}}{\pgfqpoint{3.696000in}{3.696000in}}%
\pgfusepath{clip}%
\pgfsetrectcap%
\pgfsetroundjoin%
\pgfsetlinewidth{1.505625pt}%
\definecolor{currentstroke}{rgb}{1.000000,0.000000,0.000000}%
\pgfsetstrokecolor{currentstroke}%
\pgfsetdash{}{0pt}%
\pgfpathmoveto{\pgfqpoint{2.043560in}{2.530458in}}%
\pgfpathlineto{\pgfqpoint{1.968013in}{1.956917in}}%
\pgfusepath{stroke}%
\end{pgfscope}%
\begin{pgfscope}%
\pgfpathrectangle{\pgfqpoint{0.100000in}{0.212622in}}{\pgfqpoint{3.696000in}{3.696000in}}%
\pgfusepath{clip}%
\pgfsetrectcap%
\pgfsetroundjoin%
\pgfsetlinewidth{1.505625pt}%
\definecolor{currentstroke}{rgb}{1.000000,0.000000,0.000000}%
\pgfsetstrokecolor{currentstroke}%
\pgfsetdash{}{0pt}%
\pgfpathmoveto{\pgfqpoint{2.049941in}{2.529095in}}%
\pgfpathlineto{\pgfqpoint{1.981504in}{1.952967in}}%
\pgfusepath{stroke}%
\end{pgfscope}%
\begin{pgfscope}%
\pgfpathrectangle{\pgfqpoint{0.100000in}{0.212622in}}{\pgfqpoint{3.696000in}{3.696000in}}%
\pgfusepath{clip}%
\pgfsetrectcap%
\pgfsetroundjoin%
\pgfsetlinewidth{1.505625pt}%
\definecolor{currentstroke}{rgb}{1.000000,0.000000,0.000000}%
\pgfsetstrokecolor{currentstroke}%
\pgfsetdash{}{0pt}%
\pgfpathmoveto{\pgfqpoint{2.057297in}{2.527705in}}%
\pgfpathlineto{\pgfqpoint{1.981504in}{1.952967in}}%
\pgfusepath{stroke}%
\end{pgfscope}%
\begin{pgfscope}%
\pgfpathrectangle{\pgfqpoint{0.100000in}{0.212622in}}{\pgfqpoint{3.696000in}{3.696000in}}%
\pgfusepath{clip}%
\pgfsetrectcap%
\pgfsetroundjoin%
\pgfsetlinewidth{1.505625pt}%
\definecolor{currentstroke}{rgb}{1.000000,0.000000,0.000000}%
\pgfsetstrokecolor{currentstroke}%
\pgfsetdash{}{0pt}%
\pgfpathmoveto{\pgfqpoint{2.066090in}{2.526546in}}%
\pgfpathlineto{\pgfqpoint{1.995004in}{1.949015in}}%
\pgfusepath{stroke}%
\end{pgfscope}%
\begin{pgfscope}%
\pgfpathrectangle{\pgfqpoint{0.100000in}{0.212622in}}{\pgfqpoint{3.696000in}{3.696000in}}%
\pgfusepath{clip}%
\pgfsetrectcap%
\pgfsetroundjoin%
\pgfsetlinewidth{1.505625pt}%
\definecolor{currentstroke}{rgb}{1.000000,0.000000,0.000000}%
\pgfsetstrokecolor{currentstroke}%
\pgfsetdash{}{0pt}%
\pgfpathmoveto{\pgfqpoint{2.077973in}{2.523099in}}%
\pgfpathlineto{\pgfqpoint{2.008514in}{1.945060in}}%
\pgfusepath{stroke}%
\end{pgfscope}%
\begin{pgfscope}%
\pgfpathrectangle{\pgfqpoint{0.100000in}{0.212622in}}{\pgfqpoint{3.696000in}{3.696000in}}%
\pgfusepath{clip}%
\pgfsetrectcap%
\pgfsetroundjoin%
\pgfsetlinewidth{1.505625pt}%
\definecolor{currentstroke}{rgb}{1.000000,0.000000,0.000000}%
\pgfsetstrokecolor{currentstroke}%
\pgfsetdash{}{0pt}%
\pgfpathmoveto{\pgfqpoint{2.091253in}{2.518611in}}%
\pgfpathlineto{\pgfqpoint{2.022033in}{1.941103in}}%
\pgfusepath{stroke}%
\end{pgfscope}%
\begin{pgfscope}%
\pgfpathrectangle{\pgfqpoint{0.100000in}{0.212622in}}{\pgfqpoint{3.696000in}{3.696000in}}%
\pgfusepath{clip}%
\pgfsetrectcap%
\pgfsetroundjoin%
\pgfsetlinewidth{1.505625pt}%
\definecolor{currentstroke}{rgb}{1.000000,0.000000,0.000000}%
\pgfsetstrokecolor{currentstroke}%
\pgfsetdash{}{0pt}%
\pgfpathmoveto{\pgfqpoint{2.105587in}{2.515900in}}%
\pgfpathlineto{\pgfqpoint{2.035560in}{1.937143in}}%
\pgfusepath{stroke}%
\end{pgfscope}%
\begin{pgfscope}%
\pgfpathrectangle{\pgfqpoint{0.100000in}{0.212622in}}{\pgfqpoint{3.696000in}{3.696000in}}%
\pgfusepath{clip}%
\pgfsetrectcap%
\pgfsetroundjoin%
\pgfsetlinewidth{1.505625pt}%
\definecolor{currentstroke}{rgb}{1.000000,0.000000,0.000000}%
\pgfsetstrokecolor{currentstroke}%
\pgfsetdash{}{0pt}%
\pgfpathmoveto{\pgfqpoint{2.120380in}{2.512854in}}%
\pgfpathlineto{\pgfqpoint{2.049097in}{1.933180in}}%
\pgfusepath{stroke}%
\end{pgfscope}%
\begin{pgfscope}%
\pgfpathrectangle{\pgfqpoint{0.100000in}{0.212622in}}{\pgfqpoint{3.696000in}{3.696000in}}%
\pgfusepath{clip}%
\pgfsetrectcap%
\pgfsetroundjoin%
\pgfsetlinewidth{1.505625pt}%
\definecolor{currentstroke}{rgb}{1.000000,0.000000,0.000000}%
\pgfsetstrokecolor{currentstroke}%
\pgfsetdash{}{0pt}%
\pgfpathmoveto{\pgfqpoint{2.136676in}{2.508435in}}%
\pgfpathlineto{\pgfqpoint{2.062643in}{1.929215in}}%
\pgfusepath{stroke}%
\end{pgfscope}%
\begin{pgfscope}%
\pgfpathrectangle{\pgfqpoint{0.100000in}{0.212622in}}{\pgfqpoint{3.696000in}{3.696000in}}%
\pgfusepath{clip}%
\pgfsetrectcap%
\pgfsetroundjoin%
\pgfsetlinewidth{1.505625pt}%
\definecolor{currentstroke}{rgb}{1.000000,0.000000,0.000000}%
\pgfsetstrokecolor{currentstroke}%
\pgfsetdash{}{0pt}%
\pgfpathmoveto{\pgfqpoint{2.153319in}{2.501401in}}%
\pgfpathlineto{\pgfqpoint{2.076199in}{1.925246in}}%
\pgfusepath{stroke}%
\end{pgfscope}%
\begin{pgfscope}%
\pgfpathrectangle{\pgfqpoint{0.100000in}{0.212622in}}{\pgfqpoint{3.696000in}{3.696000in}}%
\pgfusepath{clip}%
\pgfsetrectcap%
\pgfsetroundjoin%
\pgfsetlinewidth{1.505625pt}%
\definecolor{currentstroke}{rgb}{1.000000,0.000000,0.000000}%
\pgfsetstrokecolor{currentstroke}%
\pgfsetdash{}{0pt}%
\pgfpathmoveto{\pgfqpoint{2.162653in}{2.497881in}}%
\pgfpathlineto{\pgfqpoint{2.089763in}{1.921276in}}%
\pgfusepath{stroke}%
\end{pgfscope}%
\begin{pgfscope}%
\pgfpathrectangle{\pgfqpoint{0.100000in}{0.212622in}}{\pgfqpoint{3.696000in}{3.696000in}}%
\pgfusepath{clip}%
\pgfsetrectcap%
\pgfsetroundjoin%
\pgfsetlinewidth{1.505625pt}%
\definecolor{currentstroke}{rgb}{1.000000,0.000000,0.000000}%
\pgfsetstrokecolor{currentstroke}%
\pgfsetdash{}{0pt}%
\pgfpathmoveto{\pgfqpoint{2.167840in}{2.496363in}}%
\pgfpathlineto{\pgfqpoint{2.103337in}{1.917302in}}%
\pgfusepath{stroke}%
\end{pgfscope}%
\begin{pgfscope}%
\pgfpathrectangle{\pgfqpoint{0.100000in}{0.212622in}}{\pgfqpoint{3.696000in}{3.696000in}}%
\pgfusepath{clip}%
\pgfsetrectcap%
\pgfsetroundjoin%
\pgfsetlinewidth{1.505625pt}%
\definecolor{currentstroke}{rgb}{1.000000,0.000000,0.000000}%
\pgfsetstrokecolor{currentstroke}%
\pgfsetdash{}{0pt}%
\pgfpathmoveto{\pgfqpoint{2.173877in}{2.494671in}}%
\pgfpathlineto{\pgfqpoint{2.103337in}{1.917302in}}%
\pgfusepath{stroke}%
\end{pgfscope}%
\begin{pgfscope}%
\pgfpathrectangle{\pgfqpoint{0.100000in}{0.212622in}}{\pgfqpoint{3.696000in}{3.696000in}}%
\pgfusepath{clip}%
\pgfsetrectcap%
\pgfsetroundjoin%
\pgfsetlinewidth{1.505625pt}%
\definecolor{currentstroke}{rgb}{1.000000,0.000000,0.000000}%
\pgfsetstrokecolor{currentstroke}%
\pgfsetdash{}{0pt}%
\pgfpathmoveto{\pgfqpoint{2.180867in}{2.492597in}}%
\pgfpathlineto{\pgfqpoint{2.103337in}{1.917302in}}%
\pgfusepath{stroke}%
\end{pgfscope}%
\begin{pgfscope}%
\pgfpathrectangle{\pgfqpoint{0.100000in}{0.212622in}}{\pgfqpoint{3.696000in}{3.696000in}}%
\pgfusepath{clip}%
\pgfsetrectcap%
\pgfsetroundjoin%
\pgfsetlinewidth{1.505625pt}%
\definecolor{currentstroke}{rgb}{1.000000,0.000000,0.000000}%
\pgfsetstrokecolor{currentstroke}%
\pgfsetdash{}{0pt}%
\pgfpathmoveto{\pgfqpoint{2.184822in}{2.492217in}}%
\pgfpathlineto{\pgfqpoint{2.116920in}{1.913326in}}%
\pgfusepath{stroke}%
\end{pgfscope}%
\begin{pgfscope}%
\pgfpathrectangle{\pgfqpoint{0.100000in}{0.212622in}}{\pgfqpoint{3.696000in}{3.696000in}}%
\pgfusepath{clip}%
\pgfsetrectcap%
\pgfsetroundjoin%
\pgfsetlinewidth{1.505625pt}%
\definecolor{currentstroke}{rgb}{1.000000,0.000000,0.000000}%
\pgfsetstrokecolor{currentstroke}%
\pgfsetdash{}{0pt}%
\pgfpathmoveto{\pgfqpoint{2.186947in}{2.491636in}}%
\pgfpathlineto{\pgfqpoint{2.116920in}{1.913326in}}%
\pgfusepath{stroke}%
\end{pgfscope}%
\begin{pgfscope}%
\pgfpathrectangle{\pgfqpoint{0.100000in}{0.212622in}}{\pgfqpoint{3.696000in}{3.696000in}}%
\pgfusepath{clip}%
\pgfsetrectcap%
\pgfsetroundjoin%
\pgfsetlinewidth{1.505625pt}%
\definecolor{currentstroke}{rgb}{1.000000,0.000000,0.000000}%
\pgfsetstrokecolor{currentstroke}%
\pgfsetdash{}{0pt}%
\pgfpathmoveto{\pgfqpoint{2.189774in}{2.491010in}}%
\pgfpathlineto{\pgfqpoint{2.116920in}{1.913326in}}%
\pgfusepath{stroke}%
\end{pgfscope}%
\begin{pgfscope}%
\pgfpathrectangle{\pgfqpoint{0.100000in}{0.212622in}}{\pgfqpoint{3.696000in}{3.696000in}}%
\pgfusepath{clip}%
\pgfsetrectcap%
\pgfsetroundjoin%
\pgfsetlinewidth{1.505625pt}%
\definecolor{currentstroke}{rgb}{1.000000,0.000000,0.000000}%
\pgfsetstrokecolor{currentstroke}%
\pgfsetdash{}{0pt}%
\pgfpathmoveto{\pgfqpoint{2.191321in}{2.490636in}}%
\pgfpathlineto{\pgfqpoint{2.116920in}{1.913326in}}%
\pgfusepath{stroke}%
\end{pgfscope}%
\begin{pgfscope}%
\pgfpathrectangle{\pgfqpoint{0.100000in}{0.212622in}}{\pgfqpoint{3.696000in}{3.696000in}}%
\pgfusepath{clip}%
\pgfsetrectcap%
\pgfsetroundjoin%
\pgfsetlinewidth{1.505625pt}%
\definecolor{currentstroke}{rgb}{1.000000,0.000000,0.000000}%
\pgfsetstrokecolor{currentstroke}%
\pgfsetdash{}{0pt}%
\pgfpathmoveto{\pgfqpoint{2.195054in}{2.489655in}}%
\pgfpathlineto{\pgfqpoint{2.130512in}{1.909347in}}%
\pgfusepath{stroke}%
\end{pgfscope}%
\begin{pgfscope}%
\pgfpathrectangle{\pgfqpoint{0.100000in}{0.212622in}}{\pgfqpoint{3.696000in}{3.696000in}}%
\pgfusepath{clip}%
\pgfsetrectcap%
\pgfsetroundjoin%
\pgfsetlinewidth{1.505625pt}%
\definecolor{currentstroke}{rgb}{1.000000,0.000000,0.000000}%
\pgfsetstrokecolor{currentstroke}%
\pgfsetdash{}{0pt}%
\pgfpathmoveto{\pgfqpoint{2.197133in}{2.489305in}}%
\pgfpathlineto{\pgfqpoint{2.130512in}{1.909347in}}%
\pgfusepath{stroke}%
\end{pgfscope}%
\begin{pgfscope}%
\pgfpathrectangle{\pgfqpoint{0.100000in}{0.212622in}}{\pgfqpoint{3.696000in}{3.696000in}}%
\pgfusepath{clip}%
\pgfsetrectcap%
\pgfsetroundjoin%
\pgfsetlinewidth{1.505625pt}%
\definecolor{currentstroke}{rgb}{1.000000,0.000000,0.000000}%
\pgfsetstrokecolor{currentstroke}%
\pgfsetdash{}{0pt}%
\pgfpathmoveto{\pgfqpoint{2.198271in}{2.489141in}}%
\pgfpathlineto{\pgfqpoint{2.130512in}{1.909347in}}%
\pgfusepath{stroke}%
\end{pgfscope}%
\begin{pgfscope}%
\pgfpathrectangle{\pgfqpoint{0.100000in}{0.212622in}}{\pgfqpoint{3.696000in}{3.696000in}}%
\pgfusepath{clip}%
\pgfsetrectcap%
\pgfsetroundjoin%
\pgfsetlinewidth{1.505625pt}%
\definecolor{currentstroke}{rgb}{1.000000,0.000000,0.000000}%
\pgfsetstrokecolor{currentstroke}%
\pgfsetdash{}{0pt}%
\pgfpathmoveto{\pgfqpoint{2.201140in}{2.488755in}}%
\pgfpathlineto{\pgfqpoint{2.130512in}{1.909347in}}%
\pgfusepath{stroke}%
\end{pgfscope}%
\begin{pgfscope}%
\pgfpathrectangle{\pgfqpoint{0.100000in}{0.212622in}}{\pgfqpoint{3.696000in}{3.696000in}}%
\pgfusepath{clip}%
\pgfsetrectcap%
\pgfsetroundjoin%
\pgfsetlinewidth{1.505625pt}%
\definecolor{currentstroke}{rgb}{1.000000,0.000000,0.000000}%
\pgfsetstrokecolor{currentstroke}%
\pgfsetdash{}{0pt}%
\pgfpathmoveto{\pgfqpoint{2.205734in}{2.488045in}}%
\pgfpathlineto{\pgfqpoint{2.130512in}{1.909347in}}%
\pgfusepath{stroke}%
\end{pgfscope}%
\begin{pgfscope}%
\pgfpathrectangle{\pgfqpoint{0.100000in}{0.212622in}}{\pgfqpoint{3.696000in}{3.696000in}}%
\pgfusepath{clip}%
\pgfsetrectcap%
\pgfsetroundjoin%
\pgfsetlinewidth{1.505625pt}%
\definecolor{currentstroke}{rgb}{1.000000,0.000000,0.000000}%
\pgfsetstrokecolor{currentstroke}%
\pgfsetdash{}{0pt}%
\pgfpathmoveto{\pgfqpoint{2.208268in}{2.487587in}}%
\pgfpathlineto{\pgfqpoint{2.144113in}{1.905365in}}%
\pgfusepath{stroke}%
\end{pgfscope}%
\begin{pgfscope}%
\pgfpathrectangle{\pgfqpoint{0.100000in}{0.212622in}}{\pgfqpoint{3.696000in}{3.696000in}}%
\pgfusepath{clip}%
\pgfsetrectcap%
\pgfsetroundjoin%
\pgfsetlinewidth{1.505625pt}%
\definecolor{currentstroke}{rgb}{1.000000,0.000000,0.000000}%
\pgfsetstrokecolor{currentstroke}%
\pgfsetdash{}{0pt}%
\pgfpathmoveto{\pgfqpoint{2.212320in}{2.486145in}}%
\pgfpathlineto{\pgfqpoint{2.144113in}{1.905365in}}%
\pgfusepath{stroke}%
\end{pgfscope}%
\begin{pgfscope}%
\pgfpathrectangle{\pgfqpoint{0.100000in}{0.212622in}}{\pgfqpoint{3.696000in}{3.696000in}}%
\pgfusepath{clip}%
\pgfsetrectcap%
\pgfsetroundjoin%
\pgfsetlinewidth{1.505625pt}%
\definecolor{currentstroke}{rgb}{1.000000,0.000000,0.000000}%
\pgfsetstrokecolor{currentstroke}%
\pgfsetdash{}{0pt}%
\pgfpathmoveto{\pgfqpoint{2.218089in}{2.484349in}}%
\pgfpathlineto{\pgfqpoint{2.144113in}{1.905365in}}%
\pgfusepath{stroke}%
\end{pgfscope}%
\begin{pgfscope}%
\pgfpathrectangle{\pgfqpoint{0.100000in}{0.212622in}}{\pgfqpoint{3.696000in}{3.696000in}}%
\pgfusepath{clip}%
\pgfsetrectcap%
\pgfsetroundjoin%
\pgfsetlinewidth{1.505625pt}%
\definecolor{currentstroke}{rgb}{1.000000,0.000000,0.000000}%
\pgfsetstrokecolor{currentstroke}%
\pgfsetdash{}{0pt}%
\pgfpathmoveto{\pgfqpoint{2.224771in}{2.482786in}}%
\pgfpathlineto{\pgfqpoint{2.157724in}{1.901381in}}%
\pgfusepath{stroke}%
\end{pgfscope}%
\begin{pgfscope}%
\pgfpathrectangle{\pgfqpoint{0.100000in}{0.212622in}}{\pgfqpoint{3.696000in}{3.696000in}}%
\pgfusepath{clip}%
\pgfsetrectcap%
\pgfsetroundjoin%
\pgfsetlinewidth{1.505625pt}%
\definecolor{currentstroke}{rgb}{1.000000,0.000000,0.000000}%
\pgfsetstrokecolor{currentstroke}%
\pgfsetdash{}{0pt}%
\pgfpathmoveto{\pgfqpoint{2.232035in}{2.480740in}}%
\pgfpathlineto{\pgfqpoint{2.157724in}{1.901381in}}%
\pgfusepath{stroke}%
\end{pgfscope}%
\begin{pgfscope}%
\pgfpathrectangle{\pgfqpoint{0.100000in}{0.212622in}}{\pgfqpoint{3.696000in}{3.696000in}}%
\pgfusepath{clip}%
\pgfsetrectcap%
\pgfsetroundjoin%
\pgfsetlinewidth{1.505625pt}%
\definecolor{currentstroke}{rgb}{1.000000,0.000000,0.000000}%
\pgfsetstrokecolor{currentstroke}%
\pgfsetdash{}{0pt}%
\pgfpathmoveto{\pgfqpoint{2.240607in}{2.479723in}}%
\pgfpathlineto{\pgfqpoint{2.171343in}{1.897394in}}%
\pgfusepath{stroke}%
\end{pgfscope}%
\begin{pgfscope}%
\pgfpathrectangle{\pgfqpoint{0.100000in}{0.212622in}}{\pgfqpoint{3.696000in}{3.696000in}}%
\pgfusepath{clip}%
\pgfsetrectcap%
\pgfsetroundjoin%
\pgfsetlinewidth{1.505625pt}%
\definecolor{currentstroke}{rgb}{1.000000,0.000000,0.000000}%
\pgfsetstrokecolor{currentstroke}%
\pgfsetdash{}{0pt}%
\pgfpathmoveto{\pgfqpoint{2.249636in}{2.478668in}}%
\pgfpathlineto{\pgfqpoint{2.184972in}{1.893404in}}%
\pgfusepath{stroke}%
\end{pgfscope}%
\begin{pgfscope}%
\pgfpathrectangle{\pgfqpoint{0.100000in}{0.212622in}}{\pgfqpoint{3.696000in}{3.696000in}}%
\pgfusepath{clip}%
\pgfsetrectcap%
\pgfsetroundjoin%
\pgfsetlinewidth{1.505625pt}%
\definecolor{currentstroke}{rgb}{1.000000,0.000000,0.000000}%
\pgfsetstrokecolor{currentstroke}%
\pgfsetdash{}{0pt}%
\pgfpathmoveto{\pgfqpoint{2.260523in}{2.476828in}}%
\pgfpathlineto{\pgfqpoint{2.184972in}{1.893404in}}%
\pgfusepath{stroke}%
\end{pgfscope}%
\begin{pgfscope}%
\pgfpathrectangle{\pgfqpoint{0.100000in}{0.212622in}}{\pgfqpoint{3.696000in}{3.696000in}}%
\pgfusepath{clip}%
\pgfsetrectcap%
\pgfsetroundjoin%
\pgfsetlinewidth{1.505625pt}%
\definecolor{currentstroke}{rgb}{1.000000,0.000000,0.000000}%
\pgfsetstrokecolor{currentstroke}%
\pgfsetdash{}{0pt}%
\pgfpathmoveto{\pgfqpoint{2.271800in}{2.473982in}}%
\pgfpathlineto{\pgfqpoint{2.198611in}{1.889412in}}%
\pgfusepath{stroke}%
\end{pgfscope}%
\begin{pgfscope}%
\pgfpathrectangle{\pgfqpoint{0.100000in}{0.212622in}}{\pgfqpoint{3.696000in}{3.696000in}}%
\pgfusepath{clip}%
\pgfsetrectcap%
\pgfsetroundjoin%
\pgfsetlinewidth{1.505625pt}%
\definecolor{currentstroke}{rgb}{1.000000,0.000000,0.000000}%
\pgfsetstrokecolor{currentstroke}%
\pgfsetdash{}{0pt}%
\pgfpathmoveto{\pgfqpoint{2.285343in}{2.472840in}}%
\pgfpathlineto{\pgfqpoint{2.212258in}{1.885416in}}%
\pgfusepath{stroke}%
\end{pgfscope}%
\begin{pgfscope}%
\pgfpathrectangle{\pgfqpoint{0.100000in}{0.212622in}}{\pgfqpoint{3.696000in}{3.696000in}}%
\pgfusepath{clip}%
\pgfsetrectcap%
\pgfsetroundjoin%
\pgfsetlinewidth{1.505625pt}%
\definecolor{currentstroke}{rgb}{1.000000,0.000000,0.000000}%
\pgfsetstrokecolor{currentstroke}%
\pgfsetdash{}{0pt}%
\pgfpathmoveto{\pgfqpoint{2.292730in}{2.471619in}}%
\pgfpathlineto{\pgfqpoint{2.225915in}{1.881418in}}%
\pgfusepath{stroke}%
\end{pgfscope}%
\begin{pgfscope}%
\pgfpathrectangle{\pgfqpoint{0.100000in}{0.212622in}}{\pgfqpoint{3.696000in}{3.696000in}}%
\pgfusepath{clip}%
\pgfsetrectcap%
\pgfsetroundjoin%
\pgfsetlinewidth{1.505625pt}%
\definecolor{currentstroke}{rgb}{1.000000,0.000000,0.000000}%
\pgfsetstrokecolor{currentstroke}%
\pgfsetdash{}{0pt}%
\pgfpathmoveto{\pgfqpoint{2.296810in}{2.471005in}}%
\pgfpathlineto{\pgfqpoint{2.225915in}{1.881418in}}%
\pgfusepath{stroke}%
\end{pgfscope}%
\begin{pgfscope}%
\pgfpathrectangle{\pgfqpoint{0.100000in}{0.212622in}}{\pgfqpoint{3.696000in}{3.696000in}}%
\pgfusepath{clip}%
\pgfsetrectcap%
\pgfsetroundjoin%
\pgfsetlinewidth{1.505625pt}%
\definecolor{currentstroke}{rgb}{1.000000,0.000000,0.000000}%
\pgfsetstrokecolor{currentstroke}%
\pgfsetdash{}{0pt}%
\pgfpathmoveto{\pgfqpoint{2.301993in}{2.469865in}}%
\pgfpathlineto{\pgfqpoint{2.239581in}{1.877418in}}%
\pgfusepath{stroke}%
\end{pgfscope}%
\begin{pgfscope}%
\pgfpathrectangle{\pgfqpoint{0.100000in}{0.212622in}}{\pgfqpoint{3.696000in}{3.696000in}}%
\pgfusepath{clip}%
\pgfsetrectcap%
\pgfsetroundjoin%
\pgfsetlinewidth{1.505625pt}%
\definecolor{currentstroke}{rgb}{1.000000,0.000000,0.000000}%
\pgfsetstrokecolor{currentstroke}%
\pgfsetdash{}{0pt}%
\pgfpathmoveto{\pgfqpoint{2.304869in}{2.469406in}}%
\pgfpathlineto{\pgfqpoint{2.239581in}{1.877418in}}%
\pgfusepath{stroke}%
\end{pgfscope}%
\begin{pgfscope}%
\pgfpathrectangle{\pgfqpoint{0.100000in}{0.212622in}}{\pgfqpoint{3.696000in}{3.696000in}}%
\pgfusepath{clip}%
\pgfsetrectcap%
\pgfsetroundjoin%
\pgfsetlinewidth{1.505625pt}%
\definecolor{currentstroke}{rgb}{1.000000,0.000000,0.000000}%
\pgfsetstrokecolor{currentstroke}%
\pgfsetdash{}{0pt}%
\pgfpathmoveto{\pgfqpoint{2.306438in}{2.469251in}}%
\pgfpathlineto{\pgfqpoint{2.239581in}{1.877418in}}%
\pgfusepath{stroke}%
\end{pgfscope}%
\begin{pgfscope}%
\pgfpathrectangle{\pgfqpoint{0.100000in}{0.212622in}}{\pgfqpoint{3.696000in}{3.696000in}}%
\pgfusepath{clip}%
\pgfsetrectcap%
\pgfsetroundjoin%
\pgfsetlinewidth{1.505625pt}%
\definecolor{currentstroke}{rgb}{1.000000,0.000000,0.000000}%
\pgfsetstrokecolor{currentstroke}%
\pgfsetdash{}{0pt}%
\pgfpathmoveto{\pgfqpoint{2.309608in}{2.468783in}}%
\pgfpathlineto{\pgfqpoint{2.239581in}{1.877418in}}%
\pgfusepath{stroke}%
\end{pgfscope}%
\begin{pgfscope}%
\pgfpathrectangle{\pgfqpoint{0.100000in}{0.212622in}}{\pgfqpoint{3.696000in}{3.696000in}}%
\pgfusepath{clip}%
\pgfsetrectcap%
\pgfsetroundjoin%
\pgfsetlinewidth{1.505625pt}%
\definecolor{currentstroke}{rgb}{1.000000,0.000000,0.000000}%
\pgfsetstrokecolor{currentstroke}%
\pgfsetdash{}{0pt}%
\pgfpathmoveto{\pgfqpoint{2.311369in}{2.468439in}}%
\pgfpathlineto{\pgfqpoint{2.239581in}{1.877418in}}%
\pgfusepath{stroke}%
\end{pgfscope}%
\begin{pgfscope}%
\pgfpathrectangle{\pgfqpoint{0.100000in}{0.212622in}}{\pgfqpoint{3.696000in}{3.696000in}}%
\pgfusepath{clip}%
\pgfsetrectcap%
\pgfsetroundjoin%
\pgfsetlinewidth{1.505625pt}%
\definecolor{currentstroke}{rgb}{1.000000,0.000000,0.000000}%
\pgfsetstrokecolor{currentstroke}%
\pgfsetdash{}{0pt}%
\pgfpathmoveto{\pgfqpoint{2.313912in}{2.467492in}}%
\pgfpathlineto{\pgfqpoint{2.239581in}{1.877418in}}%
\pgfusepath{stroke}%
\end{pgfscope}%
\begin{pgfscope}%
\pgfpathrectangle{\pgfqpoint{0.100000in}{0.212622in}}{\pgfqpoint{3.696000in}{3.696000in}}%
\pgfusepath{clip}%
\pgfsetrectcap%
\pgfsetroundjoin%
\pgfsetlinewidth{1.505625pt}%
\definecolor{currentstroke}{rgb}{1.000000,0.000000,0.000000}%
\pgfsetstrokecolor{currentstroke}%
\pgfsetdash{}{0pt}%
\pgfpathmoveto{\pgfqpoint{2.315384in}{2.467130in}}%
\pgfpathlineto{\pgfqpoint{2.253257in}{1.873414in}}%
\pgfusepath{stroke}%
\end{pgfscope}%
\begin{pgfscope}%
\pgfpathrectangle{\pgfqpoint{0.100000in}{0.212622in}}{\pgfqpoint{3.696000in}{3.696000in}}%
\pgfusepath{clip}%
\pgfsetrectcap%
\pgfsetroundjoin%
\pgfsetlinewidth{1.505625pt}%
\definecolor{currentstroke}{rgb}{1.000000,0.000000,0.000000}%
\pgfsetstrokecolor{currentstroke}%
\pgfsetdash{}{0pt}%
\pgfpathmoveto{\pgfqpoint{2.316207in}{2.467048in}}%
\pgfpathlineto{\pgfqpoint{2.253257in}{1.873414in}}%
\pgfusepath{stroke}%
\end{pgfscope}%
\begin{pgfscope}%
\pgfpathrectangle{\pgfqpoint{0.100000in}{0.212622in}}{\pgfqpoint{3.696000in}{3.696000in}}%
\pgfusepath{clip}%
\pgfsetrectcap%
\pgfsetroundjoin%
\pgfsetlinewidth{1.505625pt}%
\definecolor{currentstroke}{rgb}{1.000000,0.000000,0.000000}%
\pgfsetstrokecolor{currentstroke}%
\pgfsetdash{}{0pt}%
\pgfpathmoveto{\pgfqpoint{2.318626in}{2.466741in}}%
\pgfpathlineto{\pgfqpoint{2.253257in}{1.873414in}}%
\pgfusepath{stroke}%
\end{pgfscope}%
\begin{pgfscope}%
\pgfpathrectangle{\pgfqpoint{0.100000in}{0.212622in}}{\pgfqpoint{3.696000in}{3.696000in}}%
\pgfusepath{clip}%
\pgfsetrectcap%
\pgfsetroundjoin%
\pgfsetlinewidth{1.505625pt}%
\definecolor{currentstroke}{rgb}{1.000000,0.000000,0.000000}%
\pgfsetstrokecolor{currentstroke}%
\pgfsetdash{}{0pt}%
\pgfpathmoveto{\pgfqpoint{2.323038in}{2.465626in}}%
\pgfpathlineto{\pgfqpoint{2.253257in}{1.873414in}}%
\pgfusepath{stroke}%
\end{pgfscope}%
\begin{pgfscope}%
\pgfpathrectangle{\pgfqpoint{0.100000in}{0.212622in}}{\pgfqpoint{3.696000in}{3.696000in}}%
\pgfusepath{clip}%
\pgfsetrectcap%
\pgfsetroundjoin%
\pgfsetlinewidth{1.505625pt}%
\definecolor{currentstroke}{rgb}{1.000000,0.000000,0.000000}%
\pgfsetstrokecolor{currentstroke}%
\pgfsetdash{}{0pt}%
\pgfpathmoveto{\pgfqpoint{2.328113in}{2.464024in}}%
\pgfpathlineto{\pgfqpoint{2.266941in}{1.869408in}}%
\pgfusepath{stroke}%
\end{pgfscope}%
\begin{pgfscope}%
\pgfpathrectangle{\pgfqpoint{0.100000in}{0.212622in}}{\pgfqpoint{3.696000in}{3.696000in}}%
\pgfusepath{clip}%
\pgfsetrectcap%
\pgfsetroundjoin%
\pgfsetlinewidth{1.505625pt}%
\definecolor{currentstroke}{rgb}{1.000000,0.000000,0.000000}%
\pgfsetstrokecolor{currentstroke}%
\pgfsetdash{}{0pt}%
\pgfpathmoveto{\pgfqpoint{2.333654in}{2.461613in}}%
\pgfpathlineto{\pgfqpoint{2.266941in}{1.869408in}}%
\pgfusepath{stroke}%
\end{pgfscope}%
\begin{pgfscope}%
\pgfpathrectangle{\pgfqpoint{0.100000in}{0.212622in}}{\pgfqpoint{3.696000in}{3.696000in}}%
\pgfusepath{clip}%
\pgfsetrectcap%
\pgfsetroundjoin%
\pgfsetlinewidth{1.505625pt}%
\definecolor{currentstroke}{rgb}{1.000000,0.000000,0.000000}%
\pgfsetstrokecolor{currentstroke}%
\pgfsetdash{}{0pt}%
\pgfpathmoveto{\pgfqpoint{2.340067in}{2.459492in}}%
\pgfpathlineto{\pgfqpoint{2.266941in}{1.869408in}}%
\pgfusepath{stroke}%
\end{pgfscope}%
\begin{pgfscope}%
\pgfpathrectangle{\pgfqpoint{0.100000in}{0.212622in}}{\pgfqpoint{3.696000in}{3.696000in}}%
\pgfusepath{clip}%
\pgfsetrectcap%
\pgfsetroundjoin%
\pgfsetlinewidth{1.505625pt}%
\definecolor{currentstroke}{rgb}{1.000000,0.000000,0.000000}%
\pgfsetstrokecolor{currentstroke}%
\pgfsetdash{}{0pt}%
\pgfpathmoveto{\pgfqpoint{2.343620in}{2.458619in}}%
\pgfpathlineto{\pgfqpoint{2.280636in}{1.865400in}}%
\pgfusepath{stroke}%
\end{pgfscope}%
\begin{pgfscope}%
\pgfpathrectangle{\pgfqpoint{0.100000in}{0.212622in}}{\pgfqpoint{3.696000in}{3.696000in}}%
\pgfusepath{clip}%
\pgfsetrectcap%
\pgfsetroundjoin%
\pgfsetlinewidth{1.505625pt}%
\definecolor{currentstroke}{rgb}{1.000000,0.000000,0.000000}%
\pgfsetstrokecolor{currentstroke}%
\pgfsetdash{}{0pt}%
\pgfpathmoveto{\pgfqpoint{2.347875in}{2.457190in}}%
\pgfpathlineto{\pgfqpoint{2.280636in}{1.865400in}}%
\pgfusepath{stroke}%
\end{pgfscope}%
\begin{pgfscope}%
\pgfpathrectangle{\pgfqpoint{0.100000in}{0.212622in}}{\pgfqpoint{3.696000in}{3.696000in}}%
\pgfusepath{clip}%
\pgfsetrectcap%
\pgfsetroundjoin%
\pgfsetlinewidth{1.505625pt}%
\definecolor{currentstroke}{rgb}{1.000000,0.000000,0.000000}%
\pgfsetstrokecolor{currentstroke}%
\pgfsetdash{}{0pt}%
\pgfpathmoveto{\pgfqpoint{2.354631in}{2.455903in}}%
\pgfpathlineto{\pgfqpoint{2.294339in}{1.861388in}}%
\pgfusepath{stroke}%
\end{pgfscope}%
\begin{pgfscope}%
\pgfpathrectangle{\pgfqpoint{0.100000in}{0.212622in}}{\pgfqpoint{3.696000in}{3.696000in}}%
\pgfusepath{clip}%
\pgfsetrectcap%
\pgfsetroundjoin%
\pgfsetlinewidth{1.505625pt}%
\definecolor{currentstroke}{rgb}{1.000000,0.000000,0.000000}%
\pgfsetstrokecolor{currentstroke}%
\pgfsetdash{}{0pt}%
\pgfpathmoveto{\pgfqpoint{2.361945in}{2.454326in}}%
\pgfpathlineto{\pgfqpoint{2.294339in}{1.861388in}}%
\pgfusepath{stroke}%
\end{pgfscope}%
\begin{pgfscope}%
\pgfpathrectangle{\pgfqpoint{0.100000in}{0.212622in}}{\pgfqpoint{3.696000in}{3.696000in}}%
\pgfusepath{clip}%
\pgfsetrectcap%
\pgfsetroundjoin%
\pgfsetlinewidth{1.505625pt}%
\definecolor{currentstroke}{rgb}{1.000000,0.000000,0.000000}%
\pgfsetstrokecolor{currentstroke}%
\pgfsetdash{}{0pt}%
\pgfpathmoveto{\pgfqpoint{2.365957in}{2.453592in}}%
\pgfpathlineto{\pgfqpoint{2.294339in}{1.861388in}}%
\pgfusepath{stroke}%
\end{pgfscope}%
\begin{pgfscope}%
\pgfpathrectangle{\pgfqpoint{0.100000in}{0.212622in}}{\pgfqpoint{3.696000in}{3.696000in}}%
\pgfusepath{clip}%
\pgfsetrectcap%
\pgfsetroundjoin%
\pgfsetlinewidth{1.505625pt}%
\definecolor{currentstroke}{rgb}{1.000000,0.000000,0.000000}%
\pgfsetstrokecolor{currentstroke}%
\pgfsetdash{}{0pt}%
\pgfpathmoveto{\pgfqpoint{2.370894in}{2.452258in}}%
\pgfpathlineto{\pgfqpoint{2.308052in}{1.857374in}}%
\pgfusepath{stroke}%
\end{pgfscope}%
\begin{pgfscope}%
\pgfpathrectangle{\pgfqpoint{0.100000in}{0.212622in}}{\pgfqpoint{3.696000in}{3.696000in}}%
\pgfusepath{clip}%
\pgfsetrectcap%
\pgfsetroundjoin%
\pgfsetlinewidth{1.505625pt}%
\definecolor{currentstroke}{rgb}{1.000000,0.000000,0.000000}%
\pgfsetstrokecolor{currentstroke}%
\pgfsetdash{}{0pt}%
\pgfpathmoveto{\pgfqpoint{2.373660in}{2.452062in}}%
\pgfpathlineto{\pgfqpoint{2.308052in}{1.857374in}}%
\pgfusepath{stroke}%
\end{pgfscope}%
\begin{pgfscope}%
\pgfpathrectangle{\pgfqpoint{0.100000in}{0.212622in}}{\pgfqpoint{3.696000in}{3.696000in}}%
\pgfusepath{clip}%
\pgfsetrectcap%
\pgfsetroundjoin%
\pgfsetlinewidth{1.505625pt}%
\definecolor{currentstroke}{rgb}{1.000000,0.000000,0.000000}%
\pgfsetstrokecolor{currentstroke}%
\pgfsetdash{}{0pt}%
\pgfpathmoveto{\pgfqpoint{2.377114in}{2.452058in}}%
\pgfpathlineto{\pgfqpoint{2.308052in}{1.857374in}}%
\pgfusepath{stroke}%
\end{pgfscope}%
\begin{pgfscope}%
\pgfpathrectangle{\pgfqpoint{0.100000in}{0.212622in}}{\pgfqpoint{3.696000in}{3.696000in}}%
\pgfusepath{clip}%
\pgfsetrectcap%
\pgfsetroundjoin%
\pgfsetlinewidth{1.505625pt}%
\definecolor{currentstroke}{rgb}{1.000000,0.000000,0.000000}%
\pgfsetstrokecolor{currentstroke}%
\pgfsetdash{}{0pt}%
\pgfpathmoveto{\pgfqpoint{2.378984in}{2.451853in}}%
\pgfpathlineto{\pgfqpoint{2.308052in}{1.857374in}}%
\pgfusepath{stroke}%
\end{pgfscope}%
\begin{pgfscope}%
\pgfpathrectangle{\pgfqpoint{0.100000in}{0.212622in}}{\pgfqpoint{3.696000in}{3.696000in}}%
\pgfusepath{clip}%
\pgfsetrectcap%
\pgfsetroundjoin%
\pgfsetlinewidth{1.505625pt}%
\definecolor{currentstroke}{rgb}{1.000000,0.000000,0.000000}%
\pgfsetstrokecolor{currentstroke}%
\pgfsetdash{}{0pt}%
\pgfpathmoveto{\pgfqpoint{2.381624in}{2.451287in}}%
\pgfpathlineto{\pgfqpoint{2.321774in}{1.853357in}}%
\pgfusepath{stroke}%
\end{pgfscope}%
\begin{pgfscope}%
\pgfpathrectangle{\pgfqpoint{0.100000in}{0.212622in}}{\pgfqpoint{3.696000in}{3.696000in}}%
\pgfusepath{clip}%
\pgfsetrectcap%
\pgfsetroundjoin%
\pgfsetlinewidth{1.505625pt}%
\definecolor{currentstroke}{rgb}{1.000000,0.000000,0.000000}%
\pgfsetstrokecolor{currentstroke}%
\pgfsetdash{}{0pt}%
\pgfpathmoveto{\pgfqpoint{2.385096in}{2.450468in}}%
\pgfpathlineto{\pgfqpoint{2.321774in}{1.853357in}}%
\pgfusepath{stroke}%
\end{pgfscope}%
\begin{pgfscope}%
\pgfpathrectangle{\pgfqpoint{0.100000in}{0.212622in}}{\pgfqpoint{3.696000in}{3.696000in}}%
\pgfusepath{clip}%
\pgfsetrectcap%
\pgfsetroundjoin%
\pgfsetlinewidth{1.505625pt}%
\definecolor{currentstroke}{rgb}{1.000000,0.000000,0.000000}%
\pgfsetstrokecolor{currentstroke}%
\pgfsetdash{}{0pt}%
\pgfpathmoveto{\pgfqpoint{2.387057in}{2.450218in}}%
\pgfpathlineto{\pgfqpoint{2.321774in}{1.853357in}}%
\pgfusepath{stroke}%
\end{pgfscope}%
\begin{pgfscope}%
\pgfpathrectangle{\pgfqpoint{0.100000in}{0.212622in}}{\pgfqpoint{3.696000in}{3.696000in}}%
\pgfusepath{clip}%
\pgfsetrectcap%
\pgfsetroundjoin%
\pgfsetlinewidth{1.505625pt}%
\definecolor{currentstroke}{rgb}{1.000000,0.000000,0.000000}%
\pgfsetstrokecolor{currentstroke}%
\pgfsetdash{}{0pt}%
\pgfpathmoveto{\pgfqpoint{2.389780in}{2.449976in}}%
\pgfpathlineto{\pgfqpoint{2.321774in}{1.853357in}}%
\pgfusepath{stroke}%
\end{pgfscope}%
\begin{pgfscope}%
\pgfpathrectangle{\pgfqpoint{0.100000in}{0.212622in}}{\pgfqpoint{3.696000in}{3.696000in}}%
\pgfusepath{clip}%
\pgfsetrectcap%
\pgfsetroundjoin%
\pgfsetlinewidth{1.505625pt}%
\definecolor{currentstroke}{rgb}{1.000000,0.000000,0.000000}%
\pgfsetstrokecolor{currentstroke}%
\pgfsetdash{}{0pt}%
\pgfpathmoveto{\pgfqpoint{2.393835in}{2.449410in}}%
\pgfpathlineto{\pgfqpoint{2.321774in}{1.853357in}}%
\pgfusepath{stroke}%
\end{pgfscope}%
\begin{pgfscope}%
\pgfpathrectangle{\pgfqpoint{0.100000in}{0.212622in}}{\pgfqpoint{3.696000in}{3.696000in}}%
\pgfusepath{clip}%
\pgfsetrectcap%
\pgfsetroundjoin%
\pgfsetlinewidth{1.505625pt}%
\definecolor{currentstroke}{rgb}{1.000000,0.000000,0.000000}%
\pgfsetstrokecolor{currentstroke}%
\pgfsetdash{}{0pt}%
\pgfpathmoveto{\pgfqpoint{2.400401in}{2.448276in}}%
\pgfpathlineto{\pgfqpoint{2.335505in}{1.849337in}}%
\pgfusepath{stroke}%
\end{pgfscope}%
\begin{pgfscope}%
\pgfpathrectangle{\pgfqpoint{0.100000in}{0.212622in}}{\pgfqpoint{3.696000in}{3.696000in}}%
\pgfusepath{clip}%
\pgfsetrectcap%
\pgfsetroundjoin%
\pgfsetlinewidth{1.505625pt}%
\definecolor{currentstroke}{rgb}{1.000000,0.000000,0.000000}%
\pgfsetstrokecolor{currentstroke}%
\pgfsetdash{}{0pt}%
\pgfpathmoveto{\pgfqpoint{2.408036in}{2.447410in}}%
\pgfpathlineto{\pgfqpoint{2.335505in}{1.849337in}}%
\pgfusepath{stroke}%
\end{pgfscope}%
\begin{pgfscope}%
\pgfpathrectangle{\pgfqpoint{0.100000in}{0.212622in}}{\pgfqpoint{3.696000in}{3.696000in}}%
\pgfusepath{clip}%
\pgfsetrectcap%
\pgfsetroundjoin%
\pgfsetlinewidth{1.505625pt}%
\definecolor{currentstroke}{rgb}{1.000000,0.000000,0.000000}%
\pgfsetstrokecolor{currentstroke}%
\pgfsetdash{}{0pt}%
\pgfpathmoveto{\pgfqpoint{2.416472in}{2.445772in}}%
\pgfpathlineto{\pgfqpoint{2.349246in}{1.845315in}}%
\pgfusepath{stroke}%
\end{pgfscope}%
\begin{pgfscope}%
\pgfpathrectangle{\pgfqpoint{0.100000in}{0.212622in}}{\pgfqpoint{3.696000in}{3.696000in}}%
\pgfusepath{clip}%
\pgfsetrectcap%
\pgfsetroundjoin%
\pgfsetlinewidth{1.505625pt}%
\definecolor{currentstroke}{rgb}{1.000000,0.000000,0.000000}%
\pgfsetstrokecolor{currentstroke}%
\pgfsetdash{}{0pt}%
\pgfpathmoveto{\pgfqpoint{2.426432in}{2.444617in}}%
\pgfpathlineto{\pgfqpoint{2.362996in}{1.841289in}}%
\pgfusepath{stroke}%
\end{pgfscope}%
\begin{pgfscope}%
\pgfpathrectangle{\pgfqpoint{0.100000in}{0.212622in}}{\pgfqpoint{3.696000in}{3.696000in}}%
\pgfusepath{clip}%
\pgfsetrectcap%
\pgfsetroundjoin%
\pgfsetlinewidth{1.505625pt}%
\definecolor{currentstroke}{rgb}{1.000000,0.000000,0.000000}%
\pgfsetstrokecolor{currentstroke}%
\pgfsetdash{}{0pt}%
\pgfpathmoveto{\pgfqpoint{2.437556in}{2.443308in}}%
\pgfpathlineto{\pgfqpoint{2.376756in}{1.837261in}}%
\pgfusepath{stroke}%
\end{pgfscope}%
\begin{pgfscope}%
\pgfpathrectangle{\pgfqpoint{0.100000in}{0.212622in}}{\pgfqpoint{3.696000in}{3.696000in}}%
\pgfusepath{clip}%
\pgfsetrectcap%
\pgfsetroundjoin%
\pgfsetlinewidth{1.505625pt}%
\definecolor{currentstroke}{rgb}{1.000000,0.000000,0.000000}%
\pgfsetstrokecolor{currentstroke}%
\pgfsetdash{}{0pt}%
\pgfpathmoveto{\pgfqpoint{2.443606in}{2.442414in}}%
\pgfpathlineto{\pgfqpoint{2.376756in}{1.837261in}}%
\pgfusepath{stroke}%
\end{pgfscope}%
\begin{pgfscope}%
\pgfpathrectangle{\pgfqpoint{0.100000in}{0.212622in}}{\pgfqpoint{3.696000in}{3.696000in}}%
\pgfusepath{clip}%
\pgfsetrectcap%
\pgfsetroundjoin%
\pgfsetlinewidth{1.505625pt}%
\definecolor{currentstroke}{rgb}{1.000000,0.000000,0.000000}%
\pgfsetstrokecolor{currentstroke}%
\pgfsetdash{}{0pt}%
\pgfpathmoveto{\pgfqpoint{2.450513in}{2.441371in}}%
\pgfpathlineto{\pgfqpoint{2.390525in}{1.833231in}}%
\pgfusepath{stroke}%
\end{pgfscope}%
\begin{pgfscope}%
\pgfpathrectangle{\pgfqpoint{0.100000in}{0.212622in}}{\pgfqpoint{3.696000in}{3.696000in}}%
\pgfusepath{clip}%
\pgfsetrectcap%
\pgfsetroundjoin%
\pgfsetlinewidth{1.505625pt}%
\definecolor{currentstroke}{rgb}{1.000000,0.000000,0.000000}%
\pgfsetstrokecolor{currentstroke}%
\pgfsetdash{}{0pt}%
\pgfpathmoveto{\pgfqpoint{2.458613in}{2.439814in}}%
\pgfpathlineto{\pgfqpoint{2.390525in}{1.833231in}}%
\pgfusepath{stroke}%
\end{pgfscope}%
\begin{pgfscope}%
\pgfpathrectangle{\pgfqpoint{0.100000in}{0.212622in}}{\pgfqpoint{3.696000in}{3.696000in}}%
\pgfusepath{clip}%
\pgfsetrectcap%
\pgfsetroundjoin%
\pgfsetlinewidth{1.505625pt}%
\definecolor{currentstroke}{rgb}{1.000000,0.000000,0.000000}%
\pgfsetstrokecolor{currentstroke}%
\pgfsetdash{}{0pt}%
\pgfpathmoveto{\pgfqpoint{2.467997in}{2.438506in}}%
\pgfpathlineto{\pgfqpoint{2.404303in}{1.829197in}}%
\pgfusepath{stroke}%
\end{pgfscope}%
\begin{pgfscope}%
\pgfpathrectangle{\pgfqpoint{0.100000in}{0.212622in}}{\pgfqpoint{3.696000in}{3.696000in}}%
\pgfusepath{clip}%
\pgfsetrectcap%
\pgfsetroundjoin%
\pgfsetlinewidth{1.505625pt}%
\definecolor{currentstroke}{rgb}{1.000000,0.000000,0.000000}%
\pgfsetstrokecolor{currentstroke}%
\pgfsetdash{}{0pt}%
\pgfpathmoveto{\pgfqpoint{2.478148in}{2.435450in}}%
\pgfpathlineto{\pgfqpoint{2.418091in}{1.825161in}}%
\pgfusepath{stroke}%
\end{pgfscope}%
\begin{pgfscope}%
\pgfpathrectangle{\pgfqpoint{0.100000in}{0.212622in}}{\pgfqpoint{3.696000in}{3.696000in}}%
\pgfusepath{clip}%
\pgfsetrectcap%
\pgfsetroundjoin%
\pgfsetlinewidth{1.505625pt}%
\definecolor{currentstroke}{rgb}{1.000000,0.000000,0.000000}%
\pgfsetstrokecolor{currentstroke}%
\pgfsetdash{}{0pt}%
\pgfpathmoveto{\pgfqpoint{2.490704in}{2.433115in}}%
\pgfpathlineto{\pgfqpoint{2.431888in}{1.821122in}}%
\pgfusepath{stroke}%
\end{pgfscope}%
\begin{pgfscope}%
\pgfpathrectangle{\pgfqpoint{0.100000in}{0.212622in}}{\pgfqpoint{3.696000in}{3.696000in}}%
\pgfusepath{clip}%
\pgfsetrectcap%
\pgfsetroundjoin%
\pgfsetlinewidth{1.505625pt}%
\definecolor{currentstroke}{rgb}{1.000000,0.000000,0.000000}%
\pgfsetstrokecolor{currentstroke}%
\pgfsetdash{}{0pt}%
\pgfpathmoveto{\pgfqpoint{2.503940in}{2.428568in}}%
\pgfpathlineto{\pgfqpoint{2.445694in}{1.817080in}}%
\pgfusepath{stroke}%
\end{pgfscope}%
\begin{pgfscope}%
\pgfpathrectangle{\pgfqpoint{0.100000in}{0.212622in}}{\pgfqpoint{3.696000in}{3.696000in}}%
\pgfusepath{clip}%
\pgfsetrectcap%
\pgfsetroundjoin%
\pgfsetlinewidth{1.505625pt}%
\definecolor{currentstroke}{rgb}{1.000000,0.000000,0.000000}%
\pgfsetstrokecolor{currentstroke}%
\pgfsetdash{}{0pt}%
\pgfpathmoveto{\pgfqpoint{2.511318in}{2.426594in}}%
\pgfpathlineto{\pgfqpoint{2.445694in}{1.817080in}}%
\pgfusepath{stroke}%
\end{pgfscope}%
\begin{pgfscope}%
\pgfpathrectangle{\pgfqpoint{0.100000in}{0.212622in}}{\pgfqpoint{3.696000in}{3.696000in}}%
\pgfusepath{clip}%
\pgfsetrectcap%
\pgfsetroundjoin%
\pgfsetlinewidth{1.505625pt}%
\definecolor{currentstroke}{rgb}{1.000000,0.000000,0.000000}%
\pgfsetstrokecolor{currentstroke}%
\pgfsetdash{}{0pt}%
\pgfpathmoveto{\pgfqpoint{2.519589in}{2.425034in}}%
\pgfpathlineto{\pgfqpoint{2.459510in}{1.813036in}}%
\pgfusepath{stroke}%
\end{pgfscope}%
\begin{pgfscope}%
\pgfpathrectangle{\pgfqpoint{0.100000in}{0.212622in}}{\pgfqpoint{3.696000in}{3.696000in}}%
\pgfusepath{clip}%
\pgfsetrectcap%
\pgfsetroundjoin%
\pgfsetlinewidth{1.505625pt}%
\definecolor{currentstroke}{rgb}{1.000000,0.000000,0.000000}%
\pgfsetstrokecolor{currentstroke}%
\pgfsetdash{}{0pt}%
\pgfpathmoveto{\pgfqpoint{2.529615in}{2.422778in}}%
\pgfpathlineto{\pgfqpoint{2.473336in}{1.808989in}}%
\pgfusepath{stroke}%
\end{pgfscope}%
\begin{pgfscope}%
\pgfpathrectangle{\pgfqpoint{0.100000in}{0.212622in}}{\pgfqpoint{3.696000in}{3.696000in}}%
\pgfusepath{clip}%
\pgfsetrectcap%
\pgfsetroundjoin%
\pgfsetlinewidth{1.505625pt}%
\definecolor{currentstroke}{rgb}{1.000000,0.000000,0.000000}%
\pgfsetstrokecolor{currentstroke}%
\pgfsetdash{}{0pt}%
\pgfpathmoveto{\pgfqpoint{2.535169in}{2.420860in}}%
\pgfpathlineto{\pgfqpoint{2.473336in}{1.808989in}}%
\pgfusepath{stroke}%
\end{pgfscope}%
\begin{pgfscope}%
\pgfpathrectangle{\pgfqpoint{0.100000in}{0.212622in}}{\pgfqpoint{3.696000in}{3.696000in}}%
\pgfusepath{clip}%
\pgfsetrectcap%
\pgfsetroundjoin%
\pgfsetlinewidth{1.505625pt}%
\definecolor{currentstroke}{rgb}{1.000000,0.000000,0.000000}%
\pgfsetstrokecolor{currentstroke}%
\pgfsetdash{}{0pt}%
\pgfpathmoveto{\pgfqpoint{2.541809in}{2.419574in}}%
\pgfpathlineto{\pgfqpoint{2.473336in}{1.808989in}}%
\pgfusepath{stroke}%
\end{pgfscope}%
\begin{pgfscope}%
\pgfpathrectangle{\pgfqpoint{0.100000in}{0.212622in}}{\pgfqpoint{3.696000in}{3.696000in}}%
\pgfusepath{clip}%
\pgfsetrectcap%
\pgfsetroundjoin%
\pgfsetlinewidth{1.505625pt}%
\definecolor{currentstroke}{rgb}{1.000000,0.000000,0.000000}%
\pgfsetstrokecolor{currentstroke}%
\pgfsetdash{}{0pt}%
\pgfpathmoveto{\pgfqpoint{2.550300in}{2.417731in}}%
\pgfpathlineto{\pgfqpoint{2.487171in}{1.804938in}}%
\pgfusepath{stroke}%
\end{pgfscope}%
\begin{pgfscope}%
\pgfpathrectangle{\pgfqpoint{0.100000in}{0.212622in}}{\pgfqpoint{3.696000in}{3.696000in}}%
\pgfusepath{clip}%
\pgfsetrectcap%
\pgfsetroundjoin%
\pgfsetlinewidth{1.505625pt}%
\definecolor{currentstroke}{rgb}{1.000000,0.000000,0.000000}%
\pgfsetstrokecolor{currentstroke}%
\pgfsetdash{}{0pt}%
\pgfpathmoveto{\pgfqpoint{2.561242in}{2.417471in}}%
\pgfpathlineto{\pgfqpoint{2.501015in}{1.800886in}}%
\pgfusepath{stroke}%
\end{pgfscope}%
\begin{pgfscope}%
\pgfpathrectangle{\pgfqpoint{0.100000in}{0.212622in}}{\pgfqpoint{3.696000in}{3.696000in}}%
\pgfusepath{clip}%
\pgfsetrectcap%
\pgfsetroundjoin%
\pgfsetlinewidth{1.505625pt}%
\definecolor{currentstroke}{rgb}{1.000000,0.000000,0.000000}%
\pgfsetstrokecolor{currentstroke}%
\pgfsetdash{}{0pt}%
\pgfpathmoveto{\pgfqpoint{2.574813in}{2.416161in}}%
\pgfpathlineto{\pgfqpoint{2.514869in}{1.796830in}}%
\pgfusepath{stroke}%
\end{pgfscope}%
\begin{pgfscope}%
\pgfpathrectangle{\pgfqpoint{0.100000in}{0.212622in}}{\pgfqpoint{3.696000in}{3.696000in}}%
\pgfusepath{clip}%
\pgfsetrectcap%
\pgfsetroundjoin%
\pgfsetlinewidth{1.505625pt}%
\definecolor{currentstroke}{rgb}{1.000000,0.000000,0.000000}%
\pgfsetstrokecolor{currentstroke}%
\pgfsetdash{}{0pt}%
\pgfpathmoveto{\pgfqpoint{2.588914in}{2.414496in}}%
\pgfpathlineto{\pgfqpoint{2.528733in}{1.792772in}}%
\pgfusepath{stroke}%
\end{pgfscope}%
\begin{pgfscope}%
\pgfpathrectangle{\pgfqpoint{0.100000in}{0.212622in}}{\pgfqpoint{3.696000in}{3.696000in}}%
\pgfusepath{clip}%
\pgfsetrectcap%
\pgfsetroundjoin%
\pgfsetlinewidth{1.505625pt}%
\definecolor{currentstroke}{rgb}{1.000000,0.000000,0.000000}%
\pgfsetstrokecolor{currentstroke}%
\pgfsetdash{}{0pt}%
\pgfpathmoveto{\pgfqpoint{2.596670in}{2.413358in}}%
\pgfpathlineto{\pgfqpoint{2.528733in}{1.792772in}}%
\pgfusepath{stroke}%
\end{pgfscope}%
\begin{pgfscope}%
\pgfpathrectangle{\pgfqpoint{0.100000in}{0.212622in}}{\pgfqpoint{3.696000in}{3.696000in}}%
\pgfusepath{clip}%
\pgfsetrectcap%
\pgfsetroundjoin%
\pgfsetlinewidth{1.505625pt}%
\definecolor{currentstroke}{rgb}{1.000000,0.000000,0.000000}%
\pgfsetstrokecolor{currentstroke}%
\pgfsetdash{}{0pt}%
\pgfpathmoveto{\pgfqpoint{2.604807in}{2.410390in}}%
\pgfpathlineto{\pgfqpoint{2.542606in}{1.788711in}}%
\pgfusepath{stroke}%
\end{pgfscope}%
\begin{pgfscope}%
\pgfpathrectangle{\pgfqpoint{0.100000in}{0.212622in}}{\pgfqpoint{3.696000in}{3.696000in}}%
\pgfusepath{clip}%
\pgfsetrectcap%
\pgfsetroundjoin%
\pgfsetlinewidth{1.505625pt}%
\definecolor{currentstroke}{rgb}{1.000000,0.000000,0.000000}%
\pgfsetstrokecolor{currentstroke}%
\pgfsetdash{}{0pt}%
\pgfpathmoveto{\pgfqpoint{2.614074in}{2.407565in}}%
\pgfpathlineto{\pgfqpoint{2.556488in}{1.784647in}}%
\pgfusepath{stroke}%
\end{pgfscope}%
\begin{pgfscope}%
\pgfpathrectangle{\pgfqpoint{0.100000in}{0.212622in}}{\pgfqpoint{3.696000in}{3.696000in}}%
\pgfusepath{clip}%
\pgfsetrectcap%
\pgfsetroundjoin%
\pgfsetlinewidth{1.505625pt}%
\definecolor{currentstroke}{rgb}{1.000000,0.000000,0.000000}%
\pgfsetstrokecolor{currentstroke}%
\pgfsetdash{}{0pt}%
\pgfpathmoveto{\pgfqpoint{2.619263in}{2.406736in}}%
\pgfpathlineto{\pgfqpoint{2.556488in}{1.784647in}}%
\pgfusepath{stroke}%
\end{pgfscope}%
\begin{pgfscope}%
\pgfpathrectangle{\pgfqpoint{0.100000in}{0.212622in}}{\pgfqpoint{3.696000in}{3.696000in}}%
\pgfusepath{clip}%
\pgfsetrectcap%
\pgfsetroundjoin%
\pgfsetlinewidth{1.505625pt}%
\definecolor{currentstroke}{rgb}{1.000000,0.000000,0.000000}%
\pgfsetstrokecolor{currentstroke}%
\pgfsetdash{}{0pt}%
\pgfpathmoveto{\pgfqpoint{2.625875in}{2.405614in}}%
\pgfpathlineto{\pgfqpoint{2.570380in}{1.780580in}}%
\pgfusepath{stroke}%
\end{pgfscope}%
\begin{pgfscope}%
\pgfpathrectangle{\pgfqpoint{0.100000in}{0.212622in}}{\pgfqpoint{3.696000in}{3.696000in}}%
\pgfusepath{clip}%
\pgfsetrectcap%
\pgfsetroundjoin%
\pgfsetlinewidth{1.505625pt}%
\definecolor{currentstroke}{rgb}{1.000000,0.000000,0.000000}%
\pgfsetstrokecolor{currentstroke}%
\pgfsetdash{}{0pt}%
\pgfpathmoveto{\pgfqpoint{2.634704in}{2.404104in}}%
\pgfpathlineto{\pgfqpoint{2.570380in}{1.780580in}}%
\pgfusepath{stroke}%
\end{pgfscope}%
\begin{pgfscope}%
\pgfpathrectangle{\pgfqpoint{0.100000in}{0.212622in}}{\pgfqpoint{3.696000in}{3.696000in}}%
\pgfusepath{clip}%
\pgfsetrectcap%
\pgfsetroundjoin%
\pgfsetlinewidth{1.505625pt}%
\definecolor{currentstroke}{rgb}{1.000000,0.000000,0.000000}%
\pgfsetstrokecolor{currentstroke}%
\pgfsetdash{}{0pt}%
\pgfpathmoveto{\pgfqpoint{2.644999in}{2.401565in}}%
\pgfpathlineto{\pgfqpoint{2.584281in}{1.776510in}}%
\pgfusepath{stroke}%
\end{pgfscope}%
\begin{pgfscope}%
\pgfpathrectangle{\pgfqpoint{0.100000in}{0.212622in}}{\pgfqpoint{3.696000in}{3.696000in}}%
\pgfusepath{clip}%
\pgfsetrectcap%
\pgfsetroundjoin%
\pgfsetlinewidth{1.505625pt}%
\definecolor{currentstroke}{rgb}{1.000000,0.000000,0.000000}%
\pgfsetstrokecolor{currentstroke}%
\pgfsetdash{}{0pt}%
\pgfpathmoveto{\pgfqpoint{2.650633in}{2.399914in}}%
\pgfpathlineto{\pgfqpoint{2.584281in}{1.776510in}}%
\pgfusepath{stroke}%
\end{pgfscope}%
\begin{pgfscope}%
\pgfpathrectangle{\pgfqpoint{0.100000in}{0.212622in}}{\pgfqpoint{3.696000in}{3.696000in}}%
\pgfusepath{clip}%
\pgfsetrectcap%
\pgfsetroundjoin%
\pgfsetlinewidth{1.505625pt}%
\definecolor{currentstroke}{rgb}{1.000000,0.000000,0.000000}%
\pgfsetstrokecolor{currentstroke}%
\pgfsetdash{}{0pt}%
\pgfpathmoveto{\pgfqpoint{2.658235in}{2.398920in}}%
\pgfpathlineto{\pgfqpoint{2.598192in}{1.772438in}}%
\pgfusepath{stroke}%
\end{pgfscope}%
\begin{pgfscope}%
\pgfpathrectangle{\pgfqpoint{0.100000in}{0.212622in}}{\pgfqpoint{3.696000in}{3.696000in}}%
\pgfusepath{clip}%
\pgfsetrectcap%
\pgfsetroundjoin%
\pgfsetlinewidth{1.505625pt}%
\definecolor{currentstroke}{rgb}{1.000000,0.000000,0.000000}%
\pgfsetstrokecolor{currentstroke}%
\pgfsetdash{}{0pt}%
\pgfpathmoveto{\pgfqpoint{2.667349in}{2.397130in}}%
\pgfpathlineto{\pgfqpoint{2.612113in}{1.768363in}}%
\pgfusepath{stroke}%
\end{pgfscope}%
\begin{pgfscope}%
\pgfpathrectangle{\pgfqpoint{0.100000in}{0.212622in}}{\pgfqpoint{3.696000in}{3.696000in}}%
\pgfusepath{clip}%
\pgfsetrectcap%
\pgfsetroundjoin%
\pgfsetlinewidth{1.505625pt}%
\definecolor{currentstroke}{rgb}{1.000000,0.000000,0.000000}%
\pgfsetstrokecolor{currentstroke}%
\pgfsetdash{}{0pt}%
\pgfpathmoveto{\pgfqpoint{2.672359in}{2.395991in}}%
\pgfpathlineto{\pgfqpoint{2.612113in}{1.768363in}}%
\pgfusepath{stroke}%
\end{pgfscope}%
\begin{pgfscope}%
\pgfpathrectangle{\pgfqpoint{0.100000in}{0.212622in}}{\pgfqpoint{3.696000in}{3.696000in}}%
\pgfusepath{clip}%
\pgfsetrectcap%
\pgfsetroundjoin%
\pgfsetlinewidth{1.505625pt}%
\definecolor{currentstroke}{rgb}{1.000000,0.000000,0.000000}%
\pgfsetstrokecolor{currentstroke}%
\pgfsetdash{}{0pt}%
\pgfpathmoveto{\pgfqpoint{2.679304in}{2.395553in}}%
\pgfpathlineto{\pgfqpoint{2.612113in}{1.768363in}}%
\pgfusepath{stroke}%
\end{pgfscope}%
\begin{pgfscope}%
\pgfpathrectangle{\pgfqpoint{0.100000in}{0.212622in}}{\pgfqpoint{3.696000in}{3.696000in}}%
\pgfusepath{clip}%
\pgfsetrectcap%
\pgfsetroundjoin%
\pgfsetlinewidth{1.505625pt}%
\definecolor{currentstroke}{rgb}{1.000000,0.000000,0.000000}%
\pgfsetstrokecolor{currentstroke}%
\pgfsetdash{}{0pt}%
\pgfpathmoveto{\pgfqpoint{2.683097in}{2.395203in}}%
\pgfpathlineto{\pgfqpoint{2.626043in}{1.764285in}}%
\pgfusepath{stroke}%
\end{pgfscope}%
\begin{pgfscope}%
\pgfpathrectangle{\pgfqpoint{0.100000in}{0.212622in}}{\pgfqpoint{3.696000in}{3.696000in}}%
\pgfusepath{clip}%
\pgfsetrectcap%
\pgfsetroundjoin%
\pgfsetlinewidth{1.505625pt}%
\definecolor{currentstroke}{rgb}{1.000000,0.000000,0.000000}%
\pgfsetstrokecolor{currentstroke}%
\pgfsetdash{}{0pt}%
\pgfpathmoveto{\pgfqpoint{2.687695in}{2.394684in}}%
\pgfpathlineto{\pgfqpoint{2.626043in}{1.764285in}}%
\pgfusepath{stroke}%
\end{pgfscope}%
\begin{pgfscope}%
\pgfpathrectangle{\pgfqpoint{0.100000in}{0.212622in}}{\pgfqpoint{3.696000in}{3.696000in}}%
\pgfusepath{clip}%
\pgfsetrectcap%
\pgfsetroundjoin%
\pgfsetlinewidth{1.505625pt}%
\definecolor{currentstroke}{rgb}{1.000000,0.000000,0.000000}%
\pgfsetstrokecolor{currentstroke}%
\pgfsetdash{}{0pt}%
\pgfpathmoveto{\pgfqpoint{2.690186in}{2.394170in}}%
\pgfpathlineto{\pgfqpoint{2.626043in}{1.764285in}}%
\pgfusepath{stroke}%
\end{pgfscope}%
\begin{pgfscope}%
\pgfpathrectangle{\pgfqpoint{0.100000in}{0.212622in}}{\pgfqpoint{3.696000in}{3.696000in}}%
\pgfusepath{clip}%
\pgfsetrectcap%
\pgfsetroundjoin%
\pgfsetlinewidth{1.505625pt}%
\definecolor{currentstroke}{rgb}{1.000000,0.000000,0.000000}%
\pgfsetstrokecolor{currentstroke}%
\pgfsetdash{}{0pt}%
\pgfpathmoveto{\pgfqpoint{2.694040in}{2.393740in}}%
\pgfpathlineto{\pgfqpoint{2.626043in}{1.764285in}}%
\pgfusepath{stroke}%
\end{pgfscope}%
\begin{pgfscope}%
\pgfpathrectangle{\pgfqpoint{0.100000in}{0.212622in}}{\pgfqpoint{3.696000in}{3.696000in}}%
\pgfusepath{clip}%
\pgfsetrectcap%
\pgfsetroundjoin%
\pgfsetlinewidth{1.505625pt}%
\definecolor{currentstroke}{rgb}{1.000000,0.000000,0.000000}%
\pgfsetstrokecolor{currentstroke}%
\pgfsetdash{}{0pt}%
\pgfpathmoveto{\pgfqpoint{2.699014in}{2.393397in}}%
\pgfpathlineto{\pgfqpoint{2.639983in}{1.760204in}}%
\pgfusepath{stroke}%
\end{pgfscope}%
\begin{pgfscope}%
\pgfpathrectangle{\pgfqpoint{0.100000in}{0.212622in}}{\pgfqpoint{3.696000in}{3.696000in}}%
\pgfusepath{clip}%
\pgfsetrectcap%
\pgfsetroundjoin%
\pgfsetlinewidth{1.505625pt}%
\definecolor{currentstroke}{rgb}{1.000000,0.000000,0.000000}%
\pgfsetstrokecolor{currentstroke}%
\pgfsetdash{}{0pt}%
\pgfpathmoveto{\pgfqpoint{2.701729in}{2.393126in}}%
\pgfpathlineto{\pgfqpoint{2.639983in}{1.760204in}}%
\pgfusepath{stroke}%
\end{pgfscope}%
\begin{pgfscope}%
\pgfpathrectangle{\pgfqpoint{0.100000in}{0.212622in}}{\pgfqpoint{3.696000in}{3.696000in}}%
\pgfusepath{clip}%
\pgfsetrectcap%
\pgfsetroundjoin%
\pgfsetlinewidth{1.505625pt}%
\definecolor{currentstroke}{rgb}{1.000000,0.000000,0.000000}%
\pgfsetstrokecolor{currentstroke}%
\pgfsetdash{}{0pt}%
\pgfpathmoveto{\pgfqpoint{2.705469in}{2.392802in}}%
\pgfpathlineto{\pgfqpoint{2.639983in}{1.760204in}}%
\pgfusepath{stroke}%
\end{pgfscope}%
\begin{pgfscope}%
\pgfpathrectangle{\pgfqpoint{0.100000in}{0.212622in}}{\pgfqpoint{3.696000in}{3.696000in}}%
\pgfusepath{clip}%
\pgfsetrectcap%
\pgfsetroundjoin%
\pgfsetlinewidth{1.505625pt}%
\definecolor{currentstroke}{rgb}{1.000000,0.000000,0.000000}%
\pgfsetstrokecolor{currentstroke}%
\pgfsetdash{}{0pt}%
\pgfpathmoveto{\pgfqpoint{2.707466in}{2.392222in}}%
\pgfpathlineto{\pgfqpoint{2.639983in}{1.760204in}}%
\pgfusepath{stroke}%
\end{pgfscope}%
\begin{pgfscope}%
\pgfpathrectangle{\pgfqpoint{0.100000in}{0.212622in}}{\pgfqpoint{3.696000in}{3.696000in}}%
\pgfusepath{clip}%
\pgfsetrectcap%
\pgfsetroundjoin%
\pgfsetlinewidth{1.505625pt}%
\definecolor{currentstroke}{rgb}{1.000000,0.000000,0.000000}%
\pgfsetstrokecolor{currentstroke}%
\pgfsetdash{}{0pt}%
\pgfpathmoveto{\pgfqpoint{2.710494in}{2.391461in}}%
\pgfpathlineto{\pgfqpoint{2.653932in}{1.756121in}}%
\pgfusepath{stroke}%
\end{pgfscope}%
\begin{pgfscope}%
\pgfpathrectangle{\pgfqpoint{0.100000in}{0.212622in}}{\pgfqpoint{3.696000in}{3.696000in}}%
\pgfusepath{clip}%
\pgfsetrectcap%
\pgfsetroundjoin%
\pgfsetlinewidth{1.505625pt}%
\definecolor{currentstroke}{rgb}{1.000000,0.000000,0.000000}%
\pgfsetstrokecolor{currentstroke}%
\pgfsetdash{}{0pt}%
\pgfpathmoveto{\pgfqpoint{2.714085in}{2.390931in}}%
\pgfpathlineto{\pgfqpoint{2.653932in}{1.756121in}}%
\pgfusepath{stroke}%
\end{pgfscope}%
\begin{pgfscope}%
\pgfpathrectangle{\pgfqpoint{0.100000in}{0.212622in}}{\pgfqpoint{3.696000in}{3.696000in}}%
\pgfusepath{clip}%
\pgfsetrectcap%
\pgfsetroundjoin%
\pgfsetlinewidth{1.505625pt}%
\definecolor{currentstroke}{rgb}{1.000000,0.000000,0.000000}%
\pgfsetstrokecolor{currentstroke}%
\pgfsetdash{}{0pt}%
\pgfpathmoveto{\pgfqpoint{2.716025in}{2.390417in}}%
\pgfpathlineto{\pgfqpoint{2.653932in}{1.756121in}}%
\pgfusepath{stroke}%
\end{pgfscope}%
\begin{pgfscope}%
\pgfpathrectangle{\pgfqpoint{0.100000in}{0.212622in}}{\pgfqpoint{3.696000in}{3.696000in}}%
\pgfusepath{clip}%
\pgfsetrectcap%
\pgfsetroundjoin%
\pgfsetlinewidth{1.505625pt}%
\definecolor{currentstroke}{rgb}{1.000000,0.000000,0.000000}%
\pgfsetstrokecolor{currentstroke}%
\pgfsetdash{}{0pt}%
\pgfpathmoveto{\pgfqpoint{2.717097in}{2.390061in}}%
\pgfpathlineto{\pgfqpoint{2.653932in}{1.756121in}}%
\pgfusepath{stroke}%
\end{pgfscope}%
\begin{pgfscope}%
\pgfpathrectangle{\pgfqpoint{0.100000in}{0.212622in}}{\pgfqpoint{3.696000in}{3.696000in}}%
\pgfusepath{clip}%
\pgfsetrectcap%
\pgfsetroundjoin%
\pgfsetlinewidth{1.505625pt}%
\definecolor{currentstroke}{rgb}{1.000000,0.000000,0.000000}%
\pgfsetstrokecolor{currentstroke}%
\pgfsetdash{}{0pt}%
\pgfpathmoveto{\pgfqpoint{2.719118in}{2.389616in}}%
\pgfpathlineto{\pgfqpoint{2.653932in}{1.756121in}}%
\pgfusepath{stroke}%
\end{pgfscope}%
\begin{pgfscope}%
\pgfpathrectangle{\pgfqpoint{0.100000in}{0.212622in}}{\pgfqpoint{3.696000in}{3.696000in}}%
\pgfusepath{clip}%
\pgfsetrectcap%
\pgfsetroundjoin%
\pgfsetlinewidth{1.505625pt}%
\definecolor{currentstroke}{rgb}{1.000000,0.000000,0.000000}%
\pgfsetstrokecolor{currentstroke}%
\pgfsetdash{}{0pt}%
\pgfpathmoveto{\pgfqpoint{2.720221in}{2.389170in}}%
\pgfpathlineto{\pgfqpoint{2.653932in}{1.756121in}}%
\pgfusepath{stroke}%
\end{pgfscope}%
\begin{pgfscope}%
\pgfpathrectangle{\pgfqpoint{0.100000in}{0.212622in}}{\pgfqpoint{3.696000in}{3.696000in}}%
\pgfusepath{clip}%
\pgfsetrectcap%
\pgfsetroundjoin%
\pgfsetlinewidth{1.505625pt}%
\definecolor{currentstroke}{rgb}{1.000000,0.000000,0.000000}%
\pgfsetstrokecolor{currentstroke}%
\pgfsetdash{}{0pt}%
\pgfpathmoveto{\pgfqpoint{2.723056in}{2.388510in}}%
\pgfpathlineto{\pgfqpoint{2.667891in}{1.752034in}}%
\pgfusepath{stroke}%
\end{pgfscope}%
\begin{pgfscope}%
\pgfpathrectangle{\pgfqpoint{0.100000in}{0.212622in}}{\pgfqpoint{3.696000in}{3.696000in}}%
\pgfusepath{clip}%
\pgfsetrectcap%
\pgfsetroundjoin%
\pgfsetlinewidth{1.505625pt}%
\definecolor{currentstroke}{rgb}{1.000000,0.000000,0.000000}%
\pgfsetstrokecolor{currentstroke}%
\pgfsetdash{}{0pt}%
\pgfpathmoveto{\pgfqpoint{2.726392in}{2.388006in}}%
\pgfpathlineto{\pgfqpoint{2.667891in}{1.752034in}}%
\pgfusepath{stroke}%
\end{pgfscope}%
\begin{pgfscope}%
\pgfpathrectangle{\pgfqpoint{0.100000in}{0.212622in}}{\pgfqpoint{3.696000in}{3.696000in}}%
\pgfusepath{clip}%
\pgfsetrectcap%
\pgfsetroundjoin%
\pgfsetlinewidth{1.505625pt}%
\definecolor{currentstroke}{rgb}{1.000000,0.000000,0.000000}%
\pgfsetstrokecolor{currentstroke}%
\pgfsetdash{}{0pt}%
\pgfpathmoveto{\pgfqpoint{2.730697in}{2.387262in}}%
\pgfpathlineto{\pgfqpoint{2.667891in}{1.752034in}}%
\pgfusepath{stroke}%
\end{pgfscope}%
\begin{pgfscope}%
\pgfpathrectangle{\pgfqpoint{0.100000in}{0.212622in}}{\pgfqpoint{3.696000in}{3.696000in}}%
\pgfusepath{clip}%
\pgfsetrectcap%
\pgfsetroundjoin%
\pgfsetlinewidth{1.505625pt}%
\definecolor{currentstroke}{rgb}{1.000000,0.000000,0.000000}%
\pgfsetstrokecolor{currentstroke}%
\pgfsetdash{}{0pt}%
\pgfpathmoveto{\pgfqpoint{2.736078in}{2.386325in}}%
\pgfpathlineto{\pgfqpoint{2.667891in}{1.752034in}}%
\pgfusepath{stroke}%
\end{pgfscope}%
\begin{pgfscope}%
\pgfpathrectangle{\pgfqpoint{0.100000in}{0.212622in}}{\pgfqpoint{3.696000in}{3.696000in}}%
\pgfusepath{clip}%
\pgfsetrectcap%
\pgfsetroundjoin%
\pgfsetlinewidth{1.505625pt}%
\definecolor{currentstroke}{rgb}{1.000000,0.000000,0.000000}%
\pgfsetstrokecolor{currentstroke}%
\pgfsetdash{}{0pt}%
\pgfpathmoveto{\pgfqpoint{2.743348in}{2.385254in}}%
\pgfpathlineto{\pgfqpoint{2.681859in}{1.747945in}}%
\pgfusepath{stroke}%
\end{pgfscope}%
\begin{pgfscope}%
\pgfpathrectangle{\pgfqpoint{0.100000in}{0.212622in}}{\pgfqpoint{3.696000in}{3.696000in}}%
\pgfusepath{clip}%
\pgfsetrectcap%
\pgfsetroundjoin%
\pgfsetlinewidth{1.505625pt}%
\definecolor{currentstroke}{rgb}{1.000000,0.000000,0.000000}%
\pgfsetstrokecolor{currentstroke}%
\pgfsetdash{}{0pt}%
\pgfpathmoveto{\pgfqpoint{2.751329in}{2.384225in}}%
\pgfpathlineto{\pgfqpoint{2.695838in}{1.743853in}}%
\pgfusepath{stroke}%
\end{pgfscope}%
\begin{pgfscope}%
\pgfpathrectangle{\pgfqpoint{0.100000in}{0.212622in}}{\pgfqpoint{3.696000in}{3.696000in}}%
\pgfusepath{clip}%
\pgfsetrectcap%
\pgfsetroundjoin%
\pgfsetlinewidth{1.505625pt}%
\definecolor{currentstroke}{rgb}{1.000000,0.000000,0.000000}%
\pgfsetstrokecolor{currentstroke}%
\pgfsetdash{}{0pt}%
\pgfpathmoveto{\pgfqpoint{2.755763in}{2.384007in}}%
\pgfpathlineto{\pgfqpoint{2.695838in}{1.743853in}}%
\pgfusepath{stroke}%
\end{pgfscope}%
\begin{pgfscope}%
\pgfpathrectangle{\pgfqpoint{0.100000in}{0.212622in}}{\pgfqpoint{3.696000in}{3.696000in}}%
\pgfusepath{clip}%
\pgfsetrectcap%
\pgfsetroundjoin%
\pgfsetlinewidth{1.505625pt}%
\definecolor{currentstroke}{rgb}{1.000000,0.000000,0.000000}%
\pgfsetstrokecolor{currentstroke}%
\pgfsetdash{}{0pt}%
\pgfpathmoveto{\pgfqpoint{2.758175in}{2.383466in}}%
\pgfpathlineto{\pgfqpoint{2.695838in}{1.743853in}}%
\pgfusepath{stroke}%
\end{pgfscope}%
\begin{pgfscope}%
\pgfpathrectangle{\pgfqpoint{0.100000in}{0.212622in}}{\pgfqpoint{3.696000in}{3.696000in}}%
\pgfusepath{clip}%
\pgfsetrectcap%
\pgfsetroundjoin%
\pgfsetlinewidth{1.505625pt}%
\definecolor{currentstroke}{rgb}{1.000000,0.000000,0.000000}%
\pgfsetstrokecolor{currentstroke}%
\pgfsetdash{}{0pt}%
\pgfpathmoveto{\pgfqpoint{2.762098in}{2.381948in}}%
\pgfpathlineto{\pgfqpoint{2.695838in}{1.743853in}}%
\pgfusepath{stroke}%
\end{pgfscope}%
\begin{pgfscope}%
\pgfpathrectangle{\pgfqpoint{0.100000in}{0.212622in}}{\pgfqpoint{3.696000in}{3.696000in}}%
\pgfusepath{clip}%
\pgfsetrectcap%
\pgfsetroundjoin%
\pgfsetlinewidth{1.505625pt}%
\definecolor{currentstroke}{rgb}{1.000000,0.000000,0.000000}%
\pgfsetstrokecolor{currentstroke}%
\pgfsetdash{}{0pt}%
\pgfpathmoveto{\pgfqpoint{2.767403in}{2.380505in}}%
\pgfpathlineto{\pgfqpoint{2.709825in}{1.739759in}}%
\pgfusepath{stroke}%
\end{pgfscope}%
\begin{pgfscope}%
\pgfpathrectangle{\pgfqpoint{0.100000in}{0.212622in}}{\pgfqpoint{3.696000in}{3.696000in}}%
\pgfusepath{clip}%
\pgfsetrectcap%
\pgfsetroundjoin%
\pgfsetlinewidth{1.505625pt}%
\definecolor{currentstroke}{rgb}{1.000000,0.000000,0.000000}%
\pgfsetstrokecolor{currentstroke}%
\pgfsetdash{}{0pt}%
\pgfpathmoveto{\pgfqpoint{2.770334in}{2.379876in}}%
\pgfpathlineto{\pgfqpoint{2.709825in}{1.739759in}}%
\pgfusepath{stroke}%
\end{pgfscope}%
\begin{pgfscope}%
\pgfpathrectangle{\pgfqpoint{0.100000in}{0.212622in}}{\pgfqpoint{3.696000in}{3.696000in}}%
\pgfusepath{clip}%
\pgfsetrectcap%
\pgfsetroundjoin%
\pgfsetlinewidth{1.505625pt}%
\definecolor{currentstroke}{rgb}{1.000000,0.000000,0.000000}%
\pgfsetstrokecolor{currentstroke}%
\pgfsetdash{}{0pt}%
\pgfpathmoveto{\pgfqpoint{2.774472in}{2.379351in}}%
\pgfpathlineto{\pgfqpoint{2.709825in}{1.739759in}}%
\pgfusepath{stroke}%
\end{pgfscope}%
\begin{pgfscope}%
\pgfpathrectangle{\pgfqpoint{0.100000in}{0.212622in}}{\pgfqpoint{3.696000in}{3.696000in}}%
\pgfusepath{clip}%
\pgfsetrectcap%
\pgfsetroundjoin%
\pgfsetlinewidth{1.505625pt}%
\definecolor{currentstroke}{rgb}{1.000000,0.000000,0.000000}%
\pgfsetstrokecolor{currentstroke}%
\pgfsetdash{}{0pt}%
\pgfpathmoveto{\pgfqpoint{2.779343in}{2.379030in}}%
\pgfpathlineto{\pgfqpoint{2.723823in}{1.735661in}}%
\pgfusepath{stroke}%
\end{pgfscope}%
\begin{pgfscope}%
\pgfpathrectangle{\pgfqpoint{0.100000in}{0.212622in}}{\pgfqpoint{3.696000in}{3.696000in}}%
\pgfusepath{clip}%
\pgfsetrectcap%
\pgfsetroundjoin%
\pgfsetlinewidth{1.505625pt}%
\definecolor{currentstroke}{rgb}{1.000000,0.000000,0.000000}%
\pgfsetstrokecolor{currentstroke}%
\pgfsetdash{}{0pt}%
\pgfpathmoveto{\pgfqpoint{2.784951in}{2.378424in}}%
\pgfpathlineto{\pgfqpoint{2.723823in}{1.735661in}}%
\pgfusepath{stroke}%
\end{pgfscope}%
\begin{pgfscope}%
\pgfpathrectangle{\pgfqpoint{0.100000in}{0.212622in}}{\pgfqpoint{3.696000in}{3.696000in}}%
\pgfusepath{clip}%
\pgfsetrectcap%
\pgfsetroundjoin%
\pgfsetlinewidth{1.505625pt}%
\definecolor{currentstroke}{rgb}{1.000000,0.000000,0.000000}%
\pgfsetstrokecolor{currentstroke}%
\pgfsetdash{}{0pt}%
\pgfpathmoveto{\pgfqpoint{2.791400in}{2.377455in}}%
\pgfpathlineto{\pgfqpoint{2.723823in}{1.735661in}}%
\pgfusepath{stroke}%
\end{pgfscope}%
\begin{pgfscope}%
\pgfpathrectangle{\pgfqpoint{0.100000in}{0.212622in}}{\pgfqpoint{3.696000in}{3.696000in}}%
\pgfusepath{clip}%
\pgfsetrectcap%
\pgfsetroundjoin%
\pgfsetlinewidth{1.505625pt}%
\definecolor{currentstroke}{rgb}{1.000000,0.000000,0.000000}%
\pgfsetstrokecolor{currentstroke}%
\pgfsetdash{}{0pt}%
\pgfpathmoveto{\pgfqpoint{2.800398in}{2.377851in}}%
\pgfpathlineto{\pgfqpoint{2.737829in}{1.731561in}}%
\pgfusepath{stroke}%
\end{pgfscope}%
\begin{pgfscope}%
\pgfpathrectangle{\pgfqpoint{0.100000in}{0.212622in}}{\pgfqpoint{3.696000in}{3.696000in}}%
\pgfusepath{clip}%
\pgfsetrectcap%
\pgfsetroundjoin%
\pgfsetlinewidth{1.505625pt}%
\definecolor{currentstroke}{rgb}{1.000000,0.000000,0.000000}%
\pgfsetstrokecolor{currentstroke}%
\pgfsetdash{}{0pt}%
\pgfpathmoveto{\pgfqpoint{2.810829in}{2.377487in}}%
\pgfpathlineto{\pgfqpoint{2.751846in}{1.727458in}}%
\pgfusepath{stroke}%
\end{pgfscope}%
\begin{pgfscope}%
\pgfpathrectangle{\pgfqpoint{0.100000in}{0.212622in}}{\pgfqpoint{3.696000in}{3.696000in}}%
\pgfusepath{clip}%
\pgfsetrectcap%
\pgfsetroundjoin%
\pgfsetlinewidth{1.505625pt}%
\definecolor{currentstroke}{rgb}{1.000000,0.000000,0.000000}%
\pgfsetstrokecolor{currentstroke}%
\pgfsetdash{}{0pt}%
\pgfpathmoveto{\pgfqpoint{2.822154in}{2.376214in}}%
\pgfpathlineto{\pgfqpoint{2.765872in}{1.723351in}}%
\pgfusepath{stroke}%
\end{pgfscope}%
\begin{pgfscope}%
\pgfpathrectangle{\pgfqpoint{0.100000in}{0.212622in}}{\pgfqpoint{3.696000in}{3.696000in}}%
\pgfusepath{clip}%
\pgfsetrectcap%
\pgfsetroundjoin%
\pgfsetlinewidth{1.505625pt}%
\definecolor{currentstroke}{rgb}{1.000000,0.000000,0.000000}%
\pgfsetstrokecolor{currentstroke}%
\pgfsetdash{}{0pt}%
\pgfpathmoveto{\pgfqpoint{2.828434in}{2.374823in}}%
\pgfpathlineto{\pgfqpoint{2.765872in}{1.723351in}}%
\pgfusepath{stroke}%
\end{pgfscope}%
\begin{pgfscope}%
\pgfpathrectangle{\pgfqpoint{0.100000in}{0.212622in}}{\pgfqpoint{3.696000in}{3.696000in}}%
\pgfusepath{clip}%
\pgfsetrectcap%
\pgfsetroundjoin%
\pgfsetlinewidth{1.505625pt}%
\definecolor{currentstroke}{rgb}{1.000000,0.000000,0.000000}%
\pgfsetstrokecolor{currentstroke}%
\pgfsetdash{}{0pt}%
\pgfpathmoveto{\pgfqpoint{2.835645in}{2.373052in}}%
\pgfpathlineto{\pgfqpoint{2.765872in}{1.723351in}}%
\pgfusepath{stroke}%
\end{pgfscope}%
\begin{pgfscope}%
\pgfpathrectangle{\pgfqpoint{0.100000in}{0.212622in}}{\pgfqpoint{3.696000in}{3.696000in}}%
\pgfusepath{clip}%
\pgfsetrectcap%
\pgfsetroundjoin%
\pgfsetlinewidth{1.505625pt}%
\definecolor{currentstroke}{rgb}{1.000000,0.000000,0.000000}%
\pgfsetstrokecolor{currentstroke}%
\pgfsetdash{}{0pt}%
\pgfpathmoveto{\pgfqpoint{2.844339in}{2.371382in}}%
\pgfpathlineto{\pgfqpoint{2.779908in}{1.719243in}}%
\pgfusepath{stroke}%
\end{pgfscope}%
\begin{pgfscope}%
\pgfpathrectangle{\pgfqpoint{0.100000in}{0.212622in}}{\pgfqpoint{3.696000in}{3.696000in}}%
\pgfusepath{clip}%
\pgfsetrectcap%
\pgfsetroundjoin%
\pgfsetlinewidth{1.505625pt}%
\definecolor{currentstroke}{rgb}{1.000000,0.000000,0.000000}%
\pgfsetstrokecolor{currentstroke}%
\pgfsetdash{}{0pt}%
\pgfpathmoveto{\pgfqpoint{2.853844in}{2.369659in}}%
\pgfpathlineto{\pgfqpoint{2.793954in}{1.715131in}}%
\pgfusepath{stroke}%
\end{pgfscope}%
\begin{pgfscope}%
\pgfpathrectangle{\pgfqpoint{0.100000in}{0.212622in}}{\pgfqpoint{3.696000in}{3.696000in}}%
\pgfusepath{clip}%
\pgfsetrectcap%
\pgfsetroundjoin%
\pgfsetlinewidth{1.505625pt}%
\definecolor{currentstroke}{rgb}{1.000000,0.000000,0.000000}%
\pgfsetstrokecolor{currentstroke}%
\pgfsetdash{}{0pt}%
\pgfpathmoveto{\pgfqpoint{2.864777in}{2.366188in}}%
\pgfpathlineto{\pgfqpoint{2.808009in}{1.711016in}}%
\pgfusepath{stroke}%
\end{pgfscope}%
\begin{pgfscope}%
\pgfpathrectangle{\pgfqpoint{0.100000in}{0.212622in}}{\pgfqpoint{3.696000in}{3.696000in}}%
\pgfusepath{clip}%
\pgfsetrectcap%
\pgfsetroundjoin%
\pgfsetlinewidth{1.505625pt}%
\definecolor{currentstroke}{rgb}{1.000000,0.000000,0.000000}%
\pgfsetstrokecolor{currentstroke}%
\pgfsetdash{}{0pt}%
\pgfpathmoveto{\pgfqpoint{2.876374in}{2.363039in}}%
\pgfpathlineto{\pgfqpoint{2.808009in}{1.711016in}}%
\pgfusepath{stroke}%
\end{pgfscope}%
\begin{pgfscope}%
\pgfpathrectangle{\pgfqpoint{0.100000in}{0.212622in}}{\pgfqpoint{3.696000in}{3.696000in}}%
\pgfusepath{clip}%
\pgfsetrectcap%
\pgfsetroundjoin%
\pgfsetlinewidth{1.505625pt}%
\definecolor{currentstroke}{rgb}{1.000000,0.000000,0.000000}%
\pgfsetstrokecolor{currentstroke}%
\pgfsetdash{}{0pt}%
\pgfpathmoveto{\pgfqpoint{2.882842in}{2.361781in}}%
\pgfpathlineto{\pgfqpoint{2.822074in}{1.706899in}}%
\pgfusepath{stroke}%
\end{pgfscope}%
\begin{pgfscope}%
\pgfpathrectangle{\pgfqpoint{0.100000in}{0.212622in}}{\pgfqpoint{3.696000in}{3.696000in}}%
\pgfusepath{clip}%
\pgfsetrectcap%
\pgfsetroundjoin%
\pgfsetlinewidth{1.505625pt}%
\definecolor{currentstroke}{rgb}{1.000000,0.000000,0.000000}%
\pgfsetstrokecolor{currentstroke}%
\pgfsetdash{}{0pt}%
\pgfpathmoveto{\pgfqpoint{2.891173in}{2.359831in}}%
\pgfpathlineto{\pgfqpoint{2.822074in}{1.706899in}}%
\pgfusepath{stroke}%
\end{pgfscope}%
\begin{pgfscope}%
\pgfpathrectangle{\pgfqpoint{0.100000in}{0.212622in}}{\pgfqpoint{3.696000in}{3.696000in}}%
\pgfusepath{clip}%
\pgfsetrectcap%
\pgfsetroundjoin%
\pgfsetlinewidth{1.505625pt}%
\definecolor{currentstroke}{rgb}{1.000000,0.000000,0.000000}%
\pgfsetstrokecolor{currentstroke}%
\pgfsetdash{}{0pt}%
\pgfpathmoveto{\pgfqpoint{2.895786in}{2.358986in}}%
\pgfpathlineto{\pgfqpoint{2.836149in}{1.702779in}}%
\pgfusepath{stroke}%
\end{pgfscope}%
\begin{pgfscope}%
\pgfpathrectangle{\pgfqpoint{0.100000in}{0.212622in}}{\pgfqpoint{3.696000in}{3.696000in}}%
\pgfusepath{clip}%
\pgfsetrectcap%
\pgfsetroundjoin%
\pgfsetlinewidth{1.505625pt}%
\definecolor{currentstroke}{rgb}{1.000000,0.000000,0.000000}%
\pgfsetstrokecolor{currentstroke}%
\pgfsetdash{}{0pt}%
\pgfpathmoveto{\pgfqpoint{2.901555in}{2.358064in}}%
\pgfpathlineto{\pgfqpoint{2.836149in}{1.702779in}}%
\pgfusepath{stroke}%
\end{pgfscope}%
\begin{pgfscope}%
\pgfpathrectangle{\pgfqpoint{0.100000in}{0.212622in}}{\pgfqpoint{3.696000in}{3.696000in}}%
\pgfusepath{clip}%
\pgfsetrectcap%
\pgfsetroundjoin%
\pgfsetlinewidth{1.505625pt}%
\definecolor{currentstroke}{rgb}{1.000000,0.000000,0.000000}%
\pgfsetstrokecolor{currentstroke}%
\pgfsetdash{}{0pt}%
\pgfpathmoveto{\pgfqpoint{2.909838in}{2.356333in}}%
\pgfpathlineto{\pgfqpoint{2.850233in}{1.698656in}}%
\pgfusepath{stroke}%
\end{pgfscope}%
\begin{pgfscope}%
\pgfpathrectangle{\pgfqpoint{0.100000in}{0.212622in}}{\pgfqpoint{3.696000in}{3.696000in}}%
\pgfusepath{clip}%
\pgfsetrectcap%
\pgfsetroundjoin%
\pgfsetlinewidth{1.505625pt}%
\definecolor{currentstroke}{rgb}{1.000000,0.000000,0.000000}%
\pgfsetstrokecolor{currentstroke}%
\pgfsetdash{}{0pt}%
\pgfpathmoveto{\pgfqpoint{2.914400in}{2.355548in}}%
\pgfpathlineto{\pgfqpoint{2.850233in}{1.698656in}}%
\pgfusepath{stroke}%
\end{pgfscope}%
\begin{pgfscope}%
\pgfpathrectangle{\pgfqpoint{0.100000in}{0.212622in}}{\pgfqpoint{3.696000in}{3.696000in}}%
\pgfusepath{clip}%
\pgfsetrectcap%
\pgfsetroundjoin%
\pgfsetlinewidth{1.505625pt}%
\definecolor{currentstroke}{rgb}{1.000000,0.000000,0.000000}%
\pgfsetstrokecolor{currentstroke}%
\pgfsetdash{}{0pt}%
\pgfpathmoveto{\pgfqpoint{2.919669in}{2.354780in}}%
\pgfpathlineto{\pgfqpoint{2.850233in}{1.698656in}}%
\pgfusepath{stroke}%
\end{pgfscope}%
\begin{pgfscope}%
\pgfpathrectangle{\pgfqpoint{0.100000in}{0.212622in}}{\pgfqpoint{3.696000in}{3.696000in}}%
\pgfusepath{clip}%
\pgfsetrectcap%
\pgfsetroundjoin%
\pgfsetlinewidth{1.505625pt}%
\definecolor{currentstroke}{rgb}{1.000000,0.000000,0.000000}%
\pgfsetstrokecolor{currentstroke}%
\pgfsetdash{}{0pt}%
\pgfpathmoveto{\pgfqpoint{2.922577in}{2.354370in}}%
\pgfpathlineto{\pgfqpoint{2.864327in}{1.694530in}}%
\pgfusepath{stroke}%
\end{pgfscope}%
\begin{pgfscope}%
\pgfpathrectangle{\pgfqpoint{0.100000in}{0.212622in}}{\pgfqpoint{3.696000in}{3.696000in}}%
\pgfusepath{clip}%
\pgfsetrectcap%
\pgfsetroundjoin%
\pgfsetlinewidth{1.505625pt}%
\definecolor{currentstroke}{rgb}{1.000000,0.000000,0.000000}%
\pgfsetstrokecolor{currentstroke}%
\pgfsetdash{}{0pt}%
\pgfpathmoveto{\pgfqpoint{2.927369in}{2.352919in}}%
\pgfpathlineto{\pgfqpoint{2.864327in}{1.694530in}}%
\pgfusepath{stroke}%
\end{pgfscope}%
\begin{pgfscope}%
\pgfpathrectangle{\pgfqpoint{0.100000in}{0.212622in}}{\pgfqpoint{3.696000in}{3.696000in}}%
\pgfusepath{clip}%
\pgfsetrectcap%
\pgfsetroundjoin%
\pgfsetlinewidth{1.505625pt}%
\definecolor{currentstroke}{rgb}{1.000000,0.000000,0.000000}%
\pgfsetstrokecolor{currentstroke}%
\pgfsetdash{}{0pt}%
\pgfpathmoveto{\pgfqpoint{2.933834in}{2.351106in}}%
\pgfpathlineto{\pgfqpoint{2.864327in}{1.694530in}}%
\pgfusepath{stroke}%
\end{pgfscope}%
\begin{pgfscope}%
\pgfpathrectangle{\pgfqpoint{0.100000in}{0.212622in}}{\pgfqpoint{3.696000in}{3.696000in}}%
\pgfusepath{clip}%
\pgfsetrectcap%
\pgfsetroundjoin%
\pgfsetlinewidth{1.505625pt}%
\definecolor{currentstroke}{rgb}{1.000000,0.000000,0.000000}%
\pgfsetstrokecolor{currentstroke}%
\pgfsetdash{}{0pt}%
\pgfpathmoveto{\pgfqpoint{2.941528in}{2.349641in}}%
\pgfpathlineto{\pgfqpoint{2.878431in}{1.690401in}}%
\pgfusepath{stroke}%
\end{pgfscope}%
\begin{pgfscope}%
\pgfpathrectangle{\pgfqpoint{0.100000in}{0.212622in}}{\pgfqpoint{3.696000in}{3.696000in}}%
\pgfusepath{clip}%
\pgfsetrectcap%
\pgfsetroundjoin%
\pgfsetlinewidth{1.505625pt}%
\definecolor{currentstroke}{rgb}{1.000000,0.000000,0.000000}%
\pgfsetstrokecolor{currentstroke}%
\pgfsetdash{}{0pt}%
\pgfpathmoveto{\pgfqpoint{2.950613in}{2.347464in}}%
\pgfpathlineto{\pgfqpoint{2.892545in}{1.686269in}}%
\pgfusepath{stroke}%
\end{pgfscope}%
\begin{pgfscope}%
\pgfpathrectangle{\pgfqpoint{0.100000in}{0.212622in}}{\pgfqpoint{3.696000in}{3.696000in}}%
\pgfusepath{clip}%
\pgfsetrectcap%
\pgfsetroundjoin%
\pgfsetlinewidth{1.505625pt}%
\definecolor{currentstroke}{rgb}{1.000000,0.000000,0.000000}%
\pgfsetstrokecolor{currentstroke}%
\pgfsetdash{}{0pt}%
\pgfpathmoveto{\pgfqpoint{2.961889in}{2.344016in}}%
\pgfpathlineto{\pgfqpoint{2.892545in}{1.686269in}}%
\pgfusepath{stroke}%
\end{pgfscope}%
\begin{pgfscope}%
\pgfpathrectangle{\pgfqpoint{0.100000in}{0.212622in}}{\pgfqpoint{3.696000in}{3.696000in}}%
\pgfusepath{clip}%
\pgfsetrectcap%
\pgfsetroundjoin%
\pgfsetlinewidth{1.505625pt}%
\definecolor{currentstroke}{rgb}{1.000000,0.000000,0.000000}%
\pgfsetstrokecolor{currentstroke}%
\pgfsetdash{}{0pt}%
\pgfpathmoveto{\pgfqpoint{2.974386in}{2.341683in}}%
\pgfpathlineto{\pgfqpoint{2.906668in}{1.682135in}}%
\pgfusepath{stroke}%
\end{pgfscope}%
\begin{pgfscope}%
\pgfpathrectangle{\pgfqpoint{0.100000in}{0.212622in}}{\pgfqpoint{3.696000in}{3.696000in}}%
\pgfusepath{clip}%
\pgfsetrectcap%
\pgfsetroundjoin%
\pgfsetlinewidth{1.505625pt}%
\definecolor{currentstroke}{rgb}{1.000000,0.000000,0.000000}%
\pgfsetstrokecolor{currentstroke}%
\pgfsetdash{}{0pt}%
\pgfpathmoveto{\pgfqpoint{2.988846in}{2.339094in}}%
\pgfpathlineto{\pgfqpoint{2.920801in}{1.677998in}}%
\pgfusepath{stroke}%
\end{pgfscope}%
\begin{pgfscope}%
\pgfpathrectangle{\pgfqpoint{0.100000in}{0.212622in}}{\pgfqpoint{3.696000in}{3.696000in}}%
\pgfusepath{clip}%
\pgfsetrectcap%
\pgfsetroundjoin%
\pgfsetlinewidth{1.505625pt}%
\definecolor{currentstroke}{rgb}{1.000000,0.000000,0.000000}%
\pgfsetstrokecolor{currentstroke}%
\pgfsetdash{}{0pt}%
\pgfpathmoveto{\pgfqpoint{3.003822in}{2.336740in}}%
\pgfpathlineto{\pgfqpoint{2.934944in}{1.673857in}}%
\pgfusepath{stroke}%
\end{pgfscope}%
\begin{pgfscope}%
\pgfpathrectangle{\pgfqpoint{0.100000in}{0.212622in}}{\pgfqpoint{3.696000in}{3.696000in}}%
\pgfusepath{clip}%
\pgfsetrectcap%
\pgfsetroundjoin%
\pgfsetlinewidth{1.505625pt}%
\definecolor{currentstroke}{rgb}{1.000000,0.000000,0.000000}%
\pgfsetstrokecolor{currentstroke}%
\pgfsetdash{}{0pt}%
\pgfpathmoveto{\pgfqpoint{3.020708in}{2.334303in}}%
\pgfpathlineto{\pgfqpoint{2.949097in}{1.669714in}}%
\pgfusepath{stroke}%
\end{pgfscope}%
\begin{pgfscope}%
\pgfpathrectangle{\pgfqpoint{0.100000in}{0.212622in}}{\pgfqpoint{3.696000in}{3.696000in}}%
\pgfusepath{clip}%
\pgfsetrectcap%
\pgfsetroundjoin%
\pgfsetlinewidth{1.505625pt}%
\definecolor{currentstroke}{rgb}{1.000000,0.000000,0.000000}%
\pgfsetstrokecolor{currentstroke}%
\pgfsetdash{}{0pt}%
\pgfpathmoveto{\pgfqpoint{3.038867in}{2.332246in}}%
\pgfpathlineto{\pgfqpoint{2.977432in}{1.661419in}}%
\pgfusepath{stroke}%
\end{pgfscope}%
\begin{pgfscope}%
\pgfpathrectangle{\pgfqpoint{0.100000in}{0.212622in}}{\pgfqpoint{3.696000in}{3.696000in}}%
\pgfusepath{clip}%
\pgfsetrectcap%
\pgfsetroundjoin%
\pgfsetlinewidth{1.505625pt}%
\definecolor{currentstroke}{rgb}{1.000000,0.000000,0.000000}%
\pgfsetstrokecolor{currentstroke}%
\pgfsetdash{}{0pt}%
\pgfpathmoveto{\pgfqpoint{3.059121in}{2.328690in}}%
\pgfpathlineto{\pgfqpoint{2.991614in}{1.657268in}}%
\pgfusepath{stroke}%
\end{pgfscope}%
\begin{pgfscope}%
\pgfpathrectangle{\pgfqpoint{0.100000in}{0.212622in}}{\pgfqpoint{3.696000in}{3.696000in}}%
\pgfusepath{clip}%
\pgfsetrectcap%
\pgfsetroundjoin%
\pgfsetlinewidth{1.505625pt}%
\definecolor{currentstroke}{rgb}{1.000000,0.000000,0.000000}%
\pgfsetstrokecolor{currentstroke}%
\pgfsetdash{}{0pt}%
\pgfpathmoveto{\pgfqpoint{3.070373in}{2.326879in}}%
\pgfpathlineto{\pgfqpoint{3.005806in}{1.653113in}}%
\pgfusepath{stroke}%
\end{pgfscope}%
\begin{pgfscope}%
\pgfpathrectangle{\pgfqpoint{0.100000in}{0.212622in}}{\pgfqpoint{3.696000in}{3.696000in}}%
\pgfusepath{clip}%
\pgfsetrectcap%
\pgfsetroundjoin%
\pgfsetlinewidth{1.505625pt}%
\definecolor{currentstroke}{rgb}{1.000000,0.000000,0.000000}%
\pgfsetstrokecolor{currentstroke}%
\pgfsetdash{}{0pt}%
\pgfpathmoveto{\pgfqpoint{3.076526in}{2.325641in}}%
\pgfpathlineto{\pgfqpoint{3.005806in}{1.653113in}}%
\pgfusepath{stroke}%
\end{pgfscope}%
\begin{pgfscope}%
\pgfpathrectangle{\pgfqpoint{0.100000in}{0.212622in}}{\pgfqpoint{3.696000in}{3.696000in}}%
\pgfusepath{clip}%
\pgfsetrectcap%
\pgfsetroundjoin%
\pgfsetlinewidth{1.505625pt}%
\definecolor{currentstroke}{rgb}{1.000000,0.000000,0.000000}%
\pgfsetstrokecolor{currentstroke}%
\pgfsetdash{}{0pt}%
\pgfpathmoveto{\pgfqpoint{3.084570in}{2.324324in}}%
\pgfpathlineto{\pgfqpoint{3.020008in}{1.648956in}}%
\pgfusepath{stroke}%
\end{pgfscope}%
\begin{pgfscope}%
\pgfpathrectangle{\pgfqpoint{0.100000in}{0.212622in}}{\pgfqpoint{3.696000in}{3.696000in}}%
\pgfusepath{clip}%
\pgfsetrectcap%
\pgfsetroundjoin%
\pgfsetlinewidth{1.505625pt}%
\definecolor{currentstroke}{rgb}{1.000000,0.000000,0.000000}%
\pgfsetstrokecolor{currentstroke}%
\pgfsetdash{}{0pt}%
\pgfpathmoveto{\pgfqpoint{3.095654in}{2.321143in}}%
\pgfpathlineto{\pgfqpoint{3.034220in}{1.644795in}}%
\pgfusepath{stroke}%
\end{pgfscope}%
\begin{pgfscope}%
\pgfpathrectangle{\pgfqpoint{0.100000in}{0.212622in}}{\pgfqpoint{3.696000in}{3.696000in}}%
\pgfusepath{clip}%
\pgfsetrectcap%
\pgfsetroundjoin%
\pgfsetlinewidth{1.505625pt}%
\definecolor{currentstroke}{rgb}{1.000000,0.000000,0.000000}%
\pgfsetstrokecolor{currentstroke}%
\pgfsetdash{}{0pt}%
\pgfpathmoveto{\pgfqpoint{3.107206in}{2.315871in}}%
\pgfpathlineto{\pgfqpoint{3.034220in}{1.644795in}}%
\pgfusepath{stroke}%
\end{pgfscope}%
\begin{pgfscope}%
\pgfpathrectangle{\pgfqpoint{0.100000in}{0.212622in}}{\pgfqpoint{3.696000in}{3.696000in}}%
\pgfusepath{clip}%
\pgfsetrectcap%
\pgfsetroundjoin%
\pgfsetlinewidth{1.505625pt}%
\definecolor{currentstroke}{rgb}{1.000000,0.000000,0.000000}%
\pgfsetstrokecolor{currentstroke}%
\pgfsetdash{}{0pt}%
\pgfpathmoveto{\pgfqpoint{3.113663in}{2.313590in}}%
\pgfpathlineto{\pgfqpoint{3.048441in}{1.640632in}}%
\pgfusepath{stroke}%
\end{pgfscope}%
\begin{pgfscope}%
\pgfpathrectangle{\pgfqpoint{0.100000in}{0.212622in}}{\pgfqpoint{3.696000in}{3.696000in}}%
\pgfusepath{clip}%
\pgfsetrectcap%
\pgfsetroundjoin%
\pgfsetlinewidth{1.505625pt}%
\definecolor{currentstroke}{rgb}{1.000000,0.000000,0.000000}%
\pgfsetstrokecolor{currentstroke}%
\pgfsetdash{}{0pt}%
\pgfpathmoveto{\pgfqpoint{3.121298in}{2.311419in}}%
\pgfpathlineto{\pgfqpoint{3.048441in}{1.640632in}}%
\pgfusepath{stroke}%
\end{pgfscope}%
\begin{pgfscope}%
\pgfpathrectangle{\pgfqpoint{0.100000in}{0.212622in}}{\pgfqpoint{3.696000in}{3.696000in}}%
\pgfusepath{clip}%
\pgfsetrectcap%
\pgfsetroundjoin%
\pgfsetlinewidth{1.505625pt}%
\definecolor{currentstroke}{rgb}{1.000000,0.000000,0.000000}%
\pgfsetstrokecolor{currentstroke}%
\pgfsetdash{}{0pt}%
\pgfpathmoveto{\pgfqpoint{3.130978in}{2.310240in}}%
\pgfpathlineto{\pgfqpoint{3.062673in}{1.636466in}}%
\pgfusepath{stroke}%
\end{pgfscope}%
\begin{pgfscope}%
\pgfpathrectangle{\pgfqpoint{0.100000in}{0.212622in}}{\pgfqpoint{3.696000in}{3.696000in}}%
\pgfusepath{clip}%
\pgfsetrectcap%
\pgfsetroundjoin%
\pgfsetlinewidth{1.505625pt}%
\definecolor{currentstroke}{rgb}{1.000000,0.000000,0.000000}%
\pgfsetstrokecolor{currentstroke}%
\pgfsetdash{}{0pt}%
\pgfpathmoveto{\pgfqpoint{3.141720in}{2.308984in}}%
\pgfpathlineto{\pgfqpoint{3.076914in}{1.632297in}}%
\pgfusepath{stroke}%
\end{pgfscope}%
\begin{pgfscope}%
\pgfpathrectangle{\pgfqpoint{0.100000in}{0.212622in}}{\pgfqpoint{3.696000in}{3.696000in}}%
\pgfusepath{clip}%
\pgfsetrectcap%
\pgfsetroundjoin%
\pgfsetlinewidth{1.505625pt}%
\definecolor{currentstroke}{rgb}{1.000000,0.000000,0.000000}%
\pgfsetstrokecolor{currentstroke}%
\pgfsetdash{}{0pt}%
\pgfpathmoveto{\pgfqpoint{3.153310in}{2.307246in}}%
\pgfpathlineto{\pgfqpoint{3.091165in}{1.628125in}}%
\pgfusepath{stroke}%
\end{pgfscope}%
\begin{pgfscope}%
\pgfpathrectangle{\pgfqpoint{0.100000in}{0.212622in}}{\pgfqpoint{3.696000in}{3.696000in}}%
\pgfusepath{clip}%
\pgfsetrectcap%
\pgfsetroundjoin%
\pgfsetlinewidth{1.505625pt}%
\definecolor{currentstroke}{rgb}{1.000000,0.000000,0.000000}%
\pgfsetstrokecolor{currentstroke}%
\pgfsetdash{}{0pt}%
\pgfpathmoveto{\pgfqpoint{3.165434in}{2.306119in}}%
\pgfpathlineto{\pgfqpoint{3.091165in}{1.628125in}}%
\pgfusepath{stroke}%
\end{pgfscope}%
\begin{pgfscope}%
\pgfpathrectangle{\pgfqpoint{0.100000in}{0.212622in}}{\pgfqpoint{3.696000in}{3.696000in}}%
\pgfusepath{clip}%
\pgfsetrectcap%
\pgfsetroundjoin%
\pgfsetlinewidth{1.505625pt}%
\definecolor{currentstroke}{rgb}{1.000000,0.000000,0.000000}%
\pgfsetstrokecolor{currentstroke}%
\pgfsetdash{}{0pt}%
\pgfpathmoveto{\pgfqpoint{3.178213in}{2.303781in}}%
\pgfpathlineto{\pgfqpoint{3.105426in}{1.623950in}}%
\pgfusepath{stroke}%
\end{pgfscope}%
\begin{pgfscope}%
\pgfpathrectangle{\pgfqpoint{0.100000in}{0.212622in}}{\pgfqpoint{3.696000in}{3.696000in}}%
\pgfusepath{clip}%
\pgfsetrectcap%
\pgfsetroundjoin%
\pgfsetlinewidth{1.505625pt}%
\definecolor{currentstroke}{rgb}{1.000000,0.000000,0.000000}%
\pgfsetstrokecolor{currentstroke}%
\pgfsetdash{}{0pt}%
\pgfpathmoveto{\pgfqpoint{3.192363in}{2.301918in}}%
\pgfpathlineto{\pgfqpoint{3.119697in}{1.619773in}}%
\pgfusepath{stroke}%
\end{pgfscope}%
\begin{pgfscope}%
\pgfpathrectangle{\pgfqpoint{0.100000in}{0.212622in}}{\pgfqpoint{3.696000in}{3.696000in}}%
\pgfusepath{clip}%
\pgfsetrectcap%
\pgfsetroundjoin%
\pgfsetlinewidth{1.505625pt}%
\definecolor{currentstroke}{rgb}{1.000000,0.000000,0.000000}%
\pgfsetstrokecolor{currentstroke}%
\pgfsetdash{}{0pt}%
\pgfpathmoveto{\pgfqpoint{3.206885in}{2.298332in}}%
\pgfpathlineto{\pgfqpoint{3.133978in}{1.615592in}}%
\pgfusepath{stroke}%
\end{pgfscope}%
\begin{pgfscope}%
\pgfpathrectangle{\pgfqpoint{0.100000in}{0.212622in}}{\pgfqpoint{3.696000in}{3.696000in}}%
\pgfusepath{clip}%
\pgfsetrectcap%
\pgfsetroundjoin%
\pgfsetlinewidth{1.505625pt}%
\definecolor{currentstroke}{rgb}{1.000000,0.000000,0.000000}%
\pgfsetstrokecolor{currentstroke}%
\pgfsetdash{}{0pt}%
\pgfpathmoveto{\pgfqpoint{3.223075in}{2.297781in}}%
\pgfpathlineto{\pgfqpoint{3.148269in}{1.611409in}}%
\pgfusepath{stroke}%
\end{pgfscope}%
\begin{pgfscope}%
\pgfpathrectangle{\pgfqpoint{0.100000in}{0.212622in}}{\pgfqpoint{3.696000in}{3.696000in}}%
\pgfusepath{clip}%
\pgfsetrectcap%
\pgfsetroundjoin%
\pgfsetlinewidth{1.505625pt}%
\definecolor{currentstroke}{rgb}{1.000000,0.000000,0.000000}%
\pgfsetstrokecolor{currentstroke}%
\pgfsetdash{}{0pt}%
\pgfpathmoveto{\pgfqpoint{3.239825in}{2.295736in}}%
\pgfpathlineto{\pgfqpoint{3.148269in}{1.611409in}}%
\pgfusepath{stroke}%
\end{pgfscope}%
\begin{pgfscope}%
\pgfpathrectangle{\pgfqpoint{0.100000in}{0.212622in}}{\pgfqpoint{3.696000in}{3.696000in}}%
\pgfusepath{clip}%
\pgfsetrectcap%
\pgfsetroundjoin%
\pgfsetlinewidth{1.505625pt}%
\definecolor{currentstroke}{rgb}{1.000000,0.000000,0.000000}%
\pgfsetstrokecolor{currentstroke}%
\pgfsetdash{}{0pt}%
\pgfpathmoveto{\pgfqpoint{3.248986in}{2.294356in}}%
\pgfpathlineto{\pgfqpoint{3.148269in}{1.611409in}}%
\pgfusepath{stroke}%
\end{pgfscope}%
\begin{pgfscope}%
\pgfpathrectangle{\pgfqpoint{0.100000in}{0.212622in}}{\pgfqpoint{3.696000in}{3.696000in}}%
\pgfusepath{clip}%
\pgfsetrectcap%
\pgfsetroundjoin%
\pgfsetlinewidth{1.505625pt}%
\definecolor{currentstroke}{rgb}{1.000000,0.000000,0.000000}%
\pgfsetstrokecolor{currentstroke}%
\pgfsetdash{}{0pt}%
\pgfpathmoveto{\pgfqpoint{3.253938in}{2.293006in}}%
\pgfpathlineto{\pgfqpoint{3.148269in}{1.611409in}}%
\pgfusepath{stroke}%
\end{pgfscope}%
\begin{pgfscope}%
\pgfpathrectangle{\pgfqpoint{0.100000in}{0.212622in}}{\pgfqpoint{3.696000in}{3.696000in}}%
\pgfusepath{clip}%
\pgfsetrectcap%
\pgfsetroundjoin%
\pgfsetlinewidth{1.505625pt}%
\definecolor{currentstroke}{rgb}{1.000000,0.000000,0.000000}%
\pgfsetstrokecolor{currentstroke}%
\pgfsetdash{}{0pt}%
\pgfpathmoveto{\pgfqpoint{3.259931in}{2.291427in}}%
\pgfpathlineto{\pgfqpoint{3.148269in}{1.611409in}}%
\pgfusepath{stroke}%
\end{pgfscope}%
\begin{pgfscope}%
\pgfpathrectangle{\pgfqpoint{0.100000in}{0.212622in}}{\pgfqpoint{3.696000in}{3.696000in}}%
\pgfusepath{clip}%
\pgfsetrectcap%
\pgfsetroundjoin%
\pgfsetlinewidth{1.505625pt}%
\definecolor{currentstroke}{rgb}{1.000000,0.000000,0.000000}%
\pgfsetstrokecolor{currentstroke}%
\pgfsetdash{}{0pt}%
\pgfpathmoveto{\pgfqpoint{3.263255in}{2.291226in}}%
\pgfpathlineto{\pgfqpoint{3.148269in}{1.611409in}}%
\pgfusepath{stroke}%
\end{pgfscope}%
\begin{pgfscope}%
\pgfpathrectangle{\pgfqpoint{0.100000in}{0.212622in}}{\pgfqpoint{3.696000in}{3.696000in}}%
\pgfusepath{clip}%
\pgfsetrectcap%
\pgfsetroundjoin%
\pgfsetlinewidth{1.505625pt}%
\definecolor{currentstroke}{rgb}{1.000000,0.000000,0.000000}%
\pgfsetstrokecolor{currentstroke}%
\pgfsetdash{}{0pt}%
\pgfpathmoveto{\pgfqpoint{3.267357in}{2.290512in}}%
\pgfpathlineto{\pgfqpoint{3.148269in}{1.611409in}}%
\pgfusepath{stroke}%
\end{pgfscope}%
\begin{pgfscope}%
\pgfpathrectangle{\pgfqpoint{0.100000in}{0.212622in}}{\pgfqpoint{3.696000in}{3.696000in}}%
\pgfusepath{clip}%
\pgfsetrectcap%
\pgfsetroundjoin%
\pgfsetlinewidth{1.505625pt}%
\definecolor{currentstroke}{rgb}{1.000000,0.000000,0.000000}%
\pgfsetstrokecolor{currentstroke}%
\pgfsetdash{}{0pt}%
\pgfpathmoveto{\pgfqpoint{3.269614in}{2.290454in}}%
\pgfpathlineto{\pgfqpoint{3.148269in}{1.611409in}}%
\pgfusepath{stroke}%
\end{pgfscope}%
\begin{pgfscope}%
\pgfpathrectangle{\pgfqpoint{0.100000in}{0.212622in}}{\pgfqpoint{3.696000in}{3.696000in}}%
\pgfusepath{clip}%
\pgfsetrectcap%
\pgfsetroundjoin%
\pgfsetlinewidth{1.505625pt}%
\definecolor{currentstroke}{rgb}{1.000000,0.000000,0.000000}%
\pgfsetstrokecolor{currentstroke}%
\pgfsetdash{}{0pt}%
\pgfpathmoveto{\pgfqpoint{3.270859in}{2.290664in}}%
\pgfpathlineto{\pgfqpoint{3.148269in}{1.611409in}}%
\pgfusepath{stroke}%
\end{pgfscope}%
\begin{pgfscope}%
\pgfpathrectangle{\pgfqpoint{0.100000in}{0.212622in}}{\pgfqpoint{3.696000in}{3.696000in}}%
\pgfusepath{clip}%
\pgfsetrectcap%
\pgfsetroundjoin%
\pgfsetlinewidth{1.505625pt}%
\definecolor{currentstroke}{rgb}{1.000000,0.000000,0.000000}%
\pgfsetstrokecolor{currentstroke}%
\pgfsetdash{}{0pt}%
\pgfpathmoveto{\pgfqpoint{3.271524in}{2.290650in}}%
\pgfpathlineto{\pgfqpoint{3.148269in}{1.611409in}}%
\pgfusepath{stroke}%
\end{pgfscope}%
\begin{pgfscope}%
\pgfpathrectangle{\pgfqpoint{0.100000in}{0.212622in}}{\pgfqpoint{3.696000in}{3.696000in}}%
\pgfusepath{clip}%
\pgfsetrectcap%
\pgfsetroundjoin%
\pgfsetlinewidth{1.505625pt}%
\definecolor{currentstroke}{rgb}{1.000000,0.000000,0.000000}%
\pgfsetstrokecolor{currentstroke}%
\pgfsetdash{}{0pt}%
\pgfpathmoveto{\pgfqpoint{3.272956in}{2.290360in}}%
\pgfpathlineto{\pgfqpoint{3.148269in}{1.611409in}}%
\pgfusepath{stroke}%
\end{pgfscope}%
\begin{pgfscope}%
\pgfpathrectangle{\pgfqpoint{0.100000in}{0.212622in}}{\pgfqpoint{3.696000in}{3.696000in}}%
\pgfusepath{clip}%
\pgfsetrectcap%
\pgfsetroundjoin%
\pgfsetlinewidth{1.505625pt}%
\definecolor{currentstroke}{rgb}{1.000000,0.000000,0.000000}%
\pgfsetstrokecolor{currentstroke}%
\pgfsetdash{}{0pt}%
\pgfpathmoveto{\pgfqpoint{3.275017in}{2.289599in}}%
\pgfpathlineto{\pgfqpoint{3.148269in}{1.611409in}}%
\pgfusepath{stroke}%
\end{pgfscope}%
\begin{pgfscope}%
\pgfpathrectangle{\pgfqpoint{0.100000in}{0.212622in}}{\pgfqpoint{3.696000in}{3.696000in}}%
\pgfusepath{clip}%
\pgfsetrectcap%
\pgfsetroundjoin%
\pgfsetlinewidth{1.505625pt}%
\definecolor{currentstroke}{rgb}{1.000000,0.000000,0.000000}%
\pgfsetstrokecolor{currentstroke}%
\pgfsetdash{}{0pt}%
\pgfpathmoveto{\pgfqpoint{3.276098in}{2.289176in}}%
\pgfpathlineto{\pgfqpoint{3.148269in}{1.611409in}}%
\pgfusepath{stroke}%
\end{pgfscope}%
\begin{pgfscope}%
\pgfpathrectangle{\pgfqpoint{0.100000in}{0.212622in}}{\pgfqpoint{3.696000in}{3.696000in}}%
\pgfusepath{clip}%
\pgfsetrectcap%
\pgfsetroundjoin%
\pgfsetlinewidth{1.505625pt}%
\definecolor{currentstroke}{rgb}{1.000000,0.000000,0.000000}%
\pgfsetstrokecolor{currentstroke}%
\pgfsetdash{}{0pt}%
\pgfpathmoveto{\pgfqpoint{3.276664in}{2.288858in}}%
\pgfpathlineto{\pgfqpoint{3.148269in}{1.611409in}}%
\pgfusepath{stroke}%
\end{pgfscope}%
\begin{pgfscope}%
\pgfpathrectangle{\pgfqpoint{0.100000in}{0.212622in}}{\pgfqpoint{3.696000in}{3.696000in}}%
\pgfusepath{clip}%
\pgfsetrectcap%
\pgfsetroundjoin%
\pgfsetlinewidth{1.505625pt}%
\definecolor{currentstroke}{rgb}{1.000000,0.000000,0.000000}%
\pgfsetstrokecolor{currentstroke}%
\pgfsetdash{}{0pt}%
\pgfpathmoveto{\pgfqpoint{3.276963in}{2.288664in}}%
\pgfpathlineto{\pgfqpoint{3.148269in}{1.611409in}}%
\pgfusepath{stroke}%
\end{pgfscope}%
\begin{pgfscope}%
\pgfpathrectangle{\pgfqpoint{0.100000in}{0.212622in}}{\pgfqpoint{3.696000in}{3.696000in}}%
\pgfusepath{clip}%
\pgfsetrectcap%
\pgfsetroundjoin%
\pgfsetlinewidth{1.505625pt}%
\definecolor{currentstroke}{rgb}{1.000000,0.000000,0.000000}%
\pgfsetstrokecolor{currentstroke}%
\pgfsetdash{}{0pt}%
\pgfpathmoveto{\pgfqpoint{3.277911in}{2.287958in}}%
\pgfpathlineto{\pgfqpoint{3.148269in}{1.611409in}}%
\pgfusepath{stroke}%
\end{pgfscope}%
\begin{pgfscope}%
\pgfpathrectangle{\pgfqpoint{0.100000in}{0.212622in}}{\pgfqpoint{3.696000in}{3.696000in}}%
\pgfusepath{clip}%
\pgfsetrectcap%
\pgfsetroundjoin%
\pgfsetlinewidth{1.505625pt}%
\definecolor{currentstroke}{rgb}{1.000000,0.000000,0.000000}%
\pgfsetstrokecolor{currentstroke}%
\pgfsetdash{}{0pt}%
\pgfpathmoveto{\pgfqpoint{3.279288in}{2.286734in}}%
\pgfpathlineto{\pgfqpoint{3.148269in}{1.611409in}}%
\pgfusepath{stroke}%
\end{pgfscope}%
\begin{pgfscope}%
\pgfpathrectangle{\pgfqpoint{0.100000in}{0.212622in}}{\pgfqpoint{3.696000in}{3.696000in}}%
\pgfusepath{clip}%
\pgfsetrectcap%
\pgfsetroundjoin%
\pgfsetlinewidth{1.505625pt}%
\definecolor{currentstroke}{rgb}{1.000000,0.000000,0.000000}%
\pgfsetstrokecolor{currentstroke}%
\pgfsetdash{}{0pt}%
\pgfpathmoveto{\pgfqpoint{3.281491in}{2.284478in}}%
\pgfpathlineto{\pgfqpoint{3.148269in}{1.611409in}}%
\pgfusepath{stroke}%
\end{pgfscope}%
\begin{pgfscope}%
\pgfpathrectangle{\pgfqpoint{0.100000in}{0.212622in}}{\pgfqpoint{3.696000in}{3.696000in}}%
\pgfusepath{clip}%
\pgfsetrectcap%
\pgfsetroundjoin%
\pgfsetlinewidth{1.505625pt}%
\definecolor{currentstroke}{rgb}{1.000000,0.000000,0.000000}%
\pgfsetstrokecolor{currentstroke}%
\pgfsetdash{}{0pt}%
\pgfpathmoveto{\pgfqpoint{3.283735in}{2.281733in}}%
\pgfpathlineto{\pgfqpoint{3.148269in}{1.611409in}}%
\pgfusepath{stroke}%
\end{pgfscope}%
\begin{pgfscope}%
\pgfpathrectangle{\pgfqpoint{0.100000in}{0.212622in}}{\pgfqpoint{3.696000in}{3.696000in}}%
\pgfusepath{clip}%
\pgfsetrectcap%
\pgfsetroundjoin%
\pgfsetlinewidth{1.505625pt}%
\definecolor{currentstroke}{rgb}{1.000000,0.000000,0.000000}%
\pgfsetstrokecolor{currentstroke}%
\pgfsetdash{}{0pt}%
\pgfpathmoveto{\pgfqpoint{3.284793in}{2.280201in}}%
\pgfpathlineto{\pgfqpoint{3.148269in}{1.611409in}}%
\pgfusepath{stroke}%
\end{pgfscope}%
\begin{pgfscope}%
\pgfpathrectangle{\pgfqpoint{0.100000in}{0.212622in}}{\pgfqpoint{3.696000in}{3.696000in}}%
\pgfusepath{clip}%
\pgfsetrectcap%
\pgfsetroundjoin%
\pgfsetlinewidth{1.505625pt}%
\definecolor{currentstroke}{rgb}{1.000000,0.000000,0.000000}%
\pgfsetstrokecolor{currentstroke}%
\pgfsetdash{}{0pt}%
\pgfpathmoveto{\pgfqpoint{3.285275in}{2.279345in}}%
\pgfpathlineto{\pgfqpoint{3.148269in}{1.611409in}}%
\pgfusepath{stroke}%
\end{pgfscope}%
\begin{pgfscope}%
\pgfpathrectangle{\pgfqpoint{0.100000in}{0.212622in}}{\pgfqpoint{3.696000in}{3.696000in}}%
\pgfusepath{clip}%
\pgfsetrectcap%
\pgfsetroundjoin%
\pgfsetlinewidth{1.505625pt}%
\definecolor{currentstroke}{rgb}{1.000000,0.000000,0.000000}%
\pgfsetstrokecolor{currentstroke}%
\pgfsetdash{}{0pt}%
\pgfpathmoveto{\pgfqpoint{3.285486in}{2.278890in}}%
\pgfpathlineto{\pgfqpoint{3.148269in}{1.611409in}}%
\pgfusepath{stroke}%
\end{pgfscope}%
\begin{pgfscope}%
\pgfpathrectangle{\pgfqpoint{0.100000in}{0.212622in}}{\pgfqpoint{3.696000in}{3.696000in}}%
\pgfusepath{clip}%
\pgfsetrectcap%
\pgfsetroundjoin%
\pgfsetlinewidth{1.505625pt}%
\definecolor{currentstroke}{rgb}{1.000000,0.000000,0.000000}%
\pgfsetstrokecolor{currentstroke}%
\pgfsetdash{}{0pt}%
\pgfpathmoveto{\pgfqpoint{3.285575in}{2.278647in}}%
\pgfpathlineto{\pgfqpoint{3.148269in}{1.611409in}}%
\pgfusepath{stroke}%
\end{pgfscope}%
\begin{pgfscope}%
\pgfpathrectangle{\pgfqpoint{0.100000in}{0.212622in}}{\pgfqpoint{3.696000in}{3.696000in}}%
\pgfusepath{clip}%
\pgfsetrectcap%
\pgfsetroundjoin%
\pgfsetlinewidth{1.505625pt}%
\definecolor{currentstroke}{rgb}{1.000000,0.000000,0.000000}%
\pgfsetstrokecolor{currentstroke}%
\pgfsetdash{}{0pt}%
\pgfpathmoveto{\pgfqpoint{3.285608in}{2.278518in}}%
\pgfpathlineto{\pgfqpoint{3.148269in}{1.611409in}}%
\pgfusepath{stroke}%
\end{pgfscope}%
\begin{pgfscope}%
\pgfpathrectangle{\pgfqpoint{0.100000in}{0.212622in}}{\pgfqpoint{3.696000in}{3.696000in}}%
\pgfusepath{clip}%
\pgfsetrectcap%
\pgfsetroundjoin%
\pgfsetlinewidth{1.505625pt}%
\definecolor{currentstroke}{rgb}{1.000000,0.000000,0.000000}%
\pgfsetstrokecolor{currentstroke}%
\pgfsetdash{}{0pt}%
\pgfpathmoveto{\pgfqpoint{3.285618in}{2.278450in}}%
\pgfpathlineto{\pgfqpoint{3.148269in}{1.611409in}}%
\pgfusepath{stroke}%
\end{pgfscope}%
\begin{pgfscope}%
\pgfpathrectangle{\pgfqpoint{0.100000in}{0.212622in}}{\pgfqpoint{3.696000in}{3.696000in}}%
\pgfusepath{clip}%
\pgfsetrectcap%
\pgfsetroundjoin%
\pgfsetlinewidth{1.505625pt}%
\definecolor{currentstroke}{rgb}{1.000000,0.000000,0.000000}%
\pgfsetstrokecolor{currentstroke}%
\pgfsetdash{}{0pt}%
\pgfpathmoveto{\pgfqpoint{3.285619in}{2.278414in}}%
\pgfpathlineto{\pgfqpoint{3.148269in}{1.611409in}}%
\pgfusepath{stroke}%
\end{pgfscope}%
\begin{pgfscope}%
\pgfpathrectangle{\pgfqpoint{0.100000in}{0.212622in}}{\pgfqpoint{3.696000in}{3.696000in}}%
\pgfusepath{clip}%
\pgfsetrectcap%
\pgfsetroundjoin%
\pgfsetlinewidth{1.505625pt}%
\definecolor{currentstroke}{rgb}{1.000000,0.000000,0.000000}%
\pgfsetstrokecolor{currentstroke}%
\pgfsetdash{}{0pt}%
\pgfpathmoveto{\pgfqpoint{3.285584in}{2.278113in}}%
\pgfpathlineto{\pgfqpoint{3.148269in}{1.611409in}}%
\pgfusepath{stroke}%
\end{pgfscope}%
\begin{pgfscope}%
\pgfpathrectangle{\pgfqpoint{0.100000in}{0.212622in}}{\pgfqpoint{3.696000in}{3.696000in}}%
\pgfusepath{clip}%
\pgfsetrectcap%
\pgfsetroundjoin%
\pgfsetlinewidth{1.505625pt}%
\definecolor{currentstroke}{rgb}{1.000000,0.000000,0.000000}%
\pgfsetstrokecolor{currentstroke}%
\pgfsetdash{}{0pt}%
\pgfpathmoveto{\pgfqpoint{3.285342in}{2.277154in}}%
\pgfpathlineto{\pgfqpoint{3.148269in}{1.611409in}}%
\pgfusepath{stroke}%
\end{pgfscope}%
\begin{pgfscope}%
\pgfpathrectangle{\pgfqpoint{0.100000in}{0.212622in}}{\pgfqpoint{3.696000in}{3.696000in}}%
\pgfusepath{clip}%
\pgfsetrectcap%
\pgfsetroundjoin%
\pgfsetlinewidth{1.505625pt}%
\definecolor{currentstroke}{rgb}{1.000000,0.000000,0.000000}%
\pgfsetstrokecolor{currentstroke}%
\pgfsetdash{}{0pt}%
\pgfpathmoveto{\pgfqpoint{3.284888in}{2.276053in}}%
\pgfpathlineto{\pgfqpoint{3.148269in}{1.611409in}}%
\pgfusepath{stroke}%
\end{pgfscope}%
\begin{pgfscope}%
\pgfpathrectangle{\pgfqpoint{0.100000in}{0.212622in}}{\pgfqpoint{3.696000in}{3.696000in}}%
\pgfusepath{clip}%
\pgfsetrectcap%
\pgfsetroundjoin%
\pgfsetlinewidth{1.505625pt}%
\definecolor{currentstroke}{rgb}{1.000000,0.000000,0.000000}%
\pgfsetstrokecolor{currentstroke}%
\pgfsetdash{}{0pt}%
\pgfpathmoveto{\pgfqpoint{3.284560in}{2.275497in}}%
\pgfpathlineto{\pgfqpoint{3.148269in}{1.611409in}}%
\pgfusepath{stroke}%
\end{pgfscope}%
\begin{pgfscope}%
\pgfpathrectangle{\pgfqpoint{0.100000in}{0.212622in}}{\pgfqpoint{3.696000in}{3.696000in}}%
\pgfusepath{clip}%
\pgfsetrectcap%
\pgfsetroundjoin%
\pgfsetlinewidth{1.505625pt}%
\definecolor{currentstroke}{rgb}{1.000000,0.000000,0.000000}%
\pgfsetstrokecolor{currentstroke}%
\pgfsetdash{}{0pt}%
\pgfpathmoveto{\pgfqpoint{3.284364in}{2.275204in}}%
\pgfpathlineto{\pgfqpoint{3.148269in}{1.611409in}}%
\pgfusepath{stroke}%
\end{pgfscope}%
\begin{pgfscope}%
\pgfpathrectangle{\pgfqpoint{0.100000in}{0.212622in}}{\pgfqpoint{3.696000in}{3.696000in}}%
\pgfusepath{clip}%
\pgfsetrectcap%
\pgfsetroundjoin%
\pgfsetlinewidth{1.505625pt}%
\definecolor{currentstroke}{rgb}{1.000000,0.000000,0.000000}%
\pgfsetstrokecolor{currentstroke}%
\pgfsetdash{}{0pt}%
\pgfpathmoveto{\pgfqpoint{3.283696in}{2.274572in}}%
\pgfpathlineto{\pgfqpoint{3.148269in}{1.611409in}}%
\pgfusepath{stroke}%
\end{pgfscope}%
\begin{pgfscope}%
\pgfpathrectangle{\pgfqpoint{0.100000in}{0.212622in}}{\pgfqpoint{3.696000in}{3.696000in}}%
\pgfusepath{clip}%
\pgfsetrectcap%
\pgfsetroundjoin%
\pgfsetlinewidth{1.505625pt}%
\definecolor{currentstroke}{rgb}{1.000000,0.000000,0.000000}%
\pgfsetstrokecolor{currentstroke}%
\pgfsetdash{}{0pt}%
\pgfpathmoveto{\pgfqpoint{3.282597in}{2.273915in}}%
\pgfpathlineto{\pgfqpoint{3.148269in}{1.611409in}}%
\pgfusepath{stroke}%
\end{pgfscope}%
\begin{pgfscope}%
\pgfpathrectangle{\pgfqpoint{0.100000in}{0.212622in}}{\pgfqpoint{3.696000in}{3.696000in}}%
\pgfusepath{clip}%
\pgfsetrectcap%
\pgfsetroundjoin%
\pgfsetlinewidth{1.505625pt}%
\definecolor{currentstroke}{rgb}{1.000000,0.000000,0.000000}%
\pgfsetstrokecolor{currentstroke}%
\pgfsetdash{}{0pt}%
\pgfpathmoveto{\pgfqpoint{3.281250in}{2.273287in}}%
\pgfpathlineto{\pgfqpoint{3.148269in}{1.611409in}}%
\pgfusepath{stroke}%
\end{pgfscope}%
\begin{pgfscope}%
\pgfpathrectangle{\pgfqpoint{0.100000in}{0.212622in}}{\pgfqpoint{3.696000in}{3.696000in}}%
\pgfusepath{clip}%
\pgfsetrectcap%
\pgfsetroundjoin%
\pgfsetlinewidth{1.505625pt}%
\definecolor{currentstroke}{rgb}{1.000000,0.000000,0.000000}%
\pgfsetstrokecolor{currentstroke}%
\pgfsetdash{}{0pt}%
\pgfpathmoveto{\pgfqpoint{3.279474in}{2.272479in}}%
\pgfpathlineto{\pgfqpoint{3.148269in}{1.611409in}}%
\pgfusepath{stroke}%
\end{pgfscope}%
\begin{pgfscope}%
\pgfpathrectangle{\pgfqpoint{0.100000in}{0.212622in}}{\pgfqpoint{3.696000in}{3.696000in}}%
\pgfusepath{clip}%
\pgfsetrectcap%
\pgfsetroundjoin%
\pgfsetlinewidth{1.505625pt}%
\definecolor{currentstroke}{rgb}{1.000000,0.000000,0.000000}%
\pgfsetstrokecolor{currentstroke}%
\pgfsetdash{}{0pt}%
\pgfpathmoveto{\pgfqpoint{3.276942in}{2.271116in}}%
\pgfpathlineto{\pgfqpoint{3.148269in}{1.611409in}}%
\pgfusepath{stroke}%
\end{pgfscope}%
\begin{pgfscope}%
\pgfpathrectangle{\pgfqpoint{0.100000in}{0.212622in}}{\pgfqpoint{3.696000in}{3.696000in}}%
\pgfusepath{clip}%
\pgfsetrectcap%
\pgfsetroundjoin%
\pgfsetlinewidth{1.505625pt}%
\definecolor{currentstroke}{rgb}{1.000000,0.000000,0.000000}%
\pgfsetstrokecolor{currentstroke}%
\pgfsetdash{}{0pt}%
\pgfpathmoveto{\pgfqpoint{3.275355in}{2.270571in}}%
\pgfpathlineto{\pgfqpoint{3.148269in}{1.611409in}}%
\pgfusepath{stroke}%
\end{pgfscope}%
\begin{pgfscope}%
\pgfpathrectangle{\pgfqpoint{0.100000in}{0.212622in}}{\pgfqpoint{3.696000in}{3.696000in}}%
\pgfusepath{clip}%
\pgfsetrectcap%
\pgfsetroundjoin%
\pgfsetlinewidth{1.505625pt}%
\definecolor{currentstroke}{rgb}{1.000000,0.000000,0.000000}%
\pgfsetstrokecolor{currentstroke}%
\pgfsetdash{}{0pt}%
\pgfpathmoveto{\pgfqpoint{3.273339in}{2.269732in}}%
\pgfpathlineto{\pgfqpoint{3.148269in}{1.611409in}}%
\pgfusepath{stroke}%
\end{pgfscope}%
\begin{pgfscope}%
\pgfpathrectangle{\pgfqpoint{0.100000in}{0.212622in}}{\pgfqpoint{3.696000in}{3.696000in}}%
\pgfusepath{clip}%
\pgfsetrectcap%
\pgfsetroundjoin%
\pgfsetlinewidth{1.505625pt}%
\definecolor{currentstroke}{rgb}{1.000000,0.000000,0.000000}%
\pgfsetstrokecolor{currentstroke}%
\pgfsetdash{}{0pt}%
\pgfpathmoveto{\pgfqpoint{3.271155in}{2.268777in}}%
\pgfpathlineto{\pgfqpoint{3.148269in}{1.611409in}}%
\pgfusepath{stroke}%
\end{pgfscope}%
\begin{pgfscope}%
\pgfpathrectangle{\pgfqpoint{0.100000in}{0.212622in}}{\pgfqpoint{3.696000in}{3.696000in}}%
\pgfusepath{clip}%
\pgfsetrectcap%
\pgfsetroundjoin%
\pgfsetlinewidth{1.505625pt}%
\definecolor{currentstroke}{rgb}{1.000000,0.000000,0.000000}%
\pgfsetstrokecolor{currentstroke}%
\pgfsetdash{}{0pt}%
\pgfpathmoveto{\pgfqpoint{3.268364in}{2.267781in}}%
\pgfpathlineto{\pgfqpoint{3.148269in}{1.611409in}}%
\pgfusepath{stroke}%
\end{pgfscope}%
\begin{pgfscope}%
\pgfpathrectangle{\pgfqpoint{0.100000in}{0.212622in}}{\pgfqpoint{3.696000in}{3.696000in}}%
\pgfusepath{clip}%
\pgfsetrectcap%
\pgfsetroundjoin%
\pgfsetlinewidth{1.505625pt}%
\definecolor{currentstroke}{rgb}{1.000000,0.000000,0.000000}%
\pgfsetstrokecolor{currentstroke}%
\pgfsetdash{}{0pt}%
\pgfpathmoveto{\pgfqpoint{3.265231in}{2.267704in}}%
\pgfpathlineto{\pgfqpoint{3.148269in}{1.611409in}}%
\pgfusepath{stroke}%
\end{pgfscope}%
\begin{pgfscope}%
\pgfpathrectangle{\pgfqpoint{0.100000in}{0.212622in}}{\pgfqpoint{3.696000in}{3.696000in}}%
\pgfusepath{clip}%
\pgfsetrectcap%
\pgfsetroundjoin%
\pgfsetlinewidth{1.505625pt}%
\definecolor{currentstroke}{rgb}{1.000000,0.000000,0.000000}%
\pgfsetstrokecolor{currentstroke}%
\pgfsetdash{}{0pt}%
\pgfpathmoveto{\pgfqpoint{3.261897in}{2.267544in}}%
\pgfpathlineto{\pgfqpoint{3.148269in}{1.611409in}}%
\pgfusepath{stroke}%
\end{pgfscope}%
\begin{pgfscope}%
\pgfpathrectangle{\pgfqpoint{0.100000in}{0.212622in}}{\pgfqpoint{3.696000in}{3.696000in}}%
\pgfusepath{clip}%
\pgfsetrectcap%
\pgfsetroundjoin%
\pgfsetlinewidth{1.505625pt}%
\definecolor{currentstroke}{rgb}{1.000000,0.000000,0.000000}%
\pgfsetstrokecolor{currentstroke}%
\pgfsetdash{}{0pt}%
\pgfpathmoveto{\pgfqpoint{3.258040in}{2.266732in}}%
\pgfpathlineto{\pgfqpoint{3.148269in}{1.611409in}}%
\pgfusepath{stroke}%
\end{pgfscope}%
\begin{pgfscope}%
\pgfpathrectangle{\pgfqpoint{0.100000in}{0.212622in}}{\pgfqpoint{3.696000in}{3.696000in}}%
\pgfusepath{clip}%
\pgfsetrectcap%
\pgfsetroundjoin%
\pgfsetlinewidth{1.505625pt}%
\definecolor{currentstroke}{rgb}{1.000000,0.000000,0.000000}%
\pgfsetstrokecolor{currentstroke}%
\pgfsetdash{}{0pt}%
\pgfpathmoveto{\pgfqpoint{3.253524in}{2.265312in}}%
\pgfpathlineto{\pgfqpoint{3.148269in}{1.611409in}}%
\pgfusepath{stroke}%
\end{pgfscope}%
\begin{pgfscope}%
\pgfpathrectangle{\pgfqpoint{0.100000in}{0.212622in}}{\pgfqpoint{3.696000in}{3.696000in}}%
\pgfusepath{clip}%
\pgfsetrectcap%
\pgfsetroundjoin%
\pgfsetlinewidth{1.505625pt}%
\definecolor{currentstroke}{rgb}{1.000000,0.000000,0.000000}%
\pgfsetstrokecolor{currentstroke}%
\pgfsetdash{}{0pt}%
\pgfpathmoveto{\pgfqpoint{3.250995in}{2.264415in}}%
\pgfpathlineto{\pgfqpoint{3.148269in}{1.611409in}}%
\pgfusepath{stroke}%
\end{pgfscope}%
\begin{pgfscope}%
\pgfpathrectangle{\pgfqpoint{0.100000in}{0.212622in}}{\pgfqpoint{3.696000in}{3.696000in}}%
\pgfusepath{clip}%
\pgfsetrectcap%
\pgfsetroundjoin%
\pgfsetlinewidth{1.505625pt}%
\definecolor{currentstroke}{rgb}{1.000000,0.000000,0.000000}%
\pgfsetstrokecolor{currentstroke}%
\pgfsetdash{}{0pt}%
\pgfpathmoveto{\pgfqpoint{3.249535in}{2.263728in}}%
\pgfpathlineto{\pgfqpoint{3.148269in}{1.611409in}}%
\pgfusepath{stroke}%
\end{pgfscope}%
\begin{pgfscope}%
\pgfpathrectangle{\pgfqpoint{0.100000in}{0.212622in}}{\pgfqpoint{3.696000in}{3.696000in}}%
\pgfusepath{clip}%
\pgfsetrectcap%
\pgfsetroundjoin%
\pgfsetlinewidth{1.505625pt}%
\definecolor{currentstroke}{rgb}{1.000000,0.000000,0.000000}%
\pgfsetstrokecolor{currentstroke}%
\pgfsetdash{}{0pt}%
\pgfpathmoveto{\pgfqpoint{3.248765in}{2.263388in}}%
\pgfpathlineto{\pgfqpoint{3.148269in}{1.611409in}}%
\pgfusepath{stroke}%
\end{pgfscope}%
\begin{pgfscope}%
\pgfpathrectangle{\pgfqpoint{0.100000in}{0.212622in}}{\pgfqpoint{3.696000in}{3.696000in}}%
\pgfusepath{clip}%
\pgfsetrectcap%
\pgfsetroundjoin%
\pgfsetlinewidth{1.505625pt}%
\definecolor{currentstroke}{rgb}{1.000000,0.000000,0.000000}%
\pgfsetstrokecolor{currentstroke}%
\pgfsetdash{}{0pt}%
\pgfpathmoveto{\pgfqpoint{3.246927in}{2.262416in}}%
\pgfpathlineto{\pgfqpoint{3.148269in}{1.611409in}}%
\pgfusepath{stroke}%
\end{pgfscope}%
\begin{pgfscope}%
\pgfpathrectangle{\pgfqpoint{0.100000in}{0.212622in}}{\pgfqpoint{3.696000in}{3.696000in}}%
\pgfusepath{clip}%
\pgfsetrectcap%
\pgfsetroundjoin%
\pgfsetlinewidth{1.505625pt}%
\definecolor{currentstroke}{rgb}{1.000000,0.000000,0.000000}%
\pgfsetstrokecolor{currentstroke}%
\pgfsetdash{}{0pt}%
\pgfpathmoveto{\pgfqpoint{3.245902in}{2.261947in}}%
\pgfpathlineto{\pgfqpoint{3.148269in}{1.611409in}}%
\pgfusepath{stroke}%
\end{pgfscope}%
\begin{pgfscope}%
\pgfpathrectangle{\pgfqpoint{0.100000in}{0.212622in}}{\pgfqpoint{3.696000in}{3.696000in}}%
\pgfusepath{clip}%
\pgfsetrectcap%
\pgfsetroundjoin%
\pgfsetlinewidth{1.505625pt}%
\definecolor{currentstroke}{rgb}{1.000000,0.000000,0.000000}%
\pgfsetstrokecolor{currentstroke}%
\pgfsetdash{}{0pt}%
\pgfpathmoveto{\pgfqpoint{3.245357in}{2.261707in}}%
\pgfpathlineto{\pgfqpoint{3.148269in}{1.611409in}}%
\pgfusepath{stroke}%
\end{pgfscope}%
\begin{pgfscope}%
\pgfpathrectangle{\pgfqpoint{0.100000in}{0.212622in}}{\pgfqpoint{3.696000in}{3.696000in}}%
\pgfusepath{clip}%
\pgfsetrectcap%
\pgfsetroundjoin%
\pgfsetlinewidth{1.505625pt}%
\definecolor{currentstroke}{rgb}{1.000000,0.000000,0.000000}%
\pgfsetstrokecolor{currentstroke}%
\pgfsetdash{}{0pt}%
\pgfpathmoveto{\pgfqpoint{3.245046in}{2.261597in}}%
\pgfpathlineto{\pgfqpoint{3.148269in}{1.611409in}}%
\pgfusepath{stroke}%
\end{pgfscope}%
\begin{pgfscope}%
\pgfpathrectangle{\pgfqpoint{0.100000in}{0.212622in}}{\pgfqpoint{3.696000in}{3.696000in}}%
\pgfusepath{clip}%
\pgfsetrectcap%
\pgfsetroundjoin%
\pgfsetlinewidth{1.505625pt}%
\definecolor{currentstroke}{rgb}{1.000000,0.000000,0.000000}%
\pgfsetstrokecolor{currentstroke}%
\pgfsetdash{}{0pt}%
\pgfpathmoveto{\pgfqpoint{3.243978in}{2.261223in}}%
\pgfpathlineto{\pgfqpoint{3.140937in}{1.604158in}}%
\pgfusepath{stroke}%
\end{pgfscope}%
\begin{pgfscope}%
\pgfpathrectangle{\pgfqpoint{0.100000in}{0.212622in}}{\pgfqpoint{3.696000in}{3.696000in}}%
\pgfusepath{clip}%
\pgfsetrectcap%
\pgfsetroundjoin%
\pgfsetlinewidth{1.505625pt}%
\definecolor{currentstroke}{rgb}{1.000000,0.000000,0.000000}%
\pgfsetstrokecolor{currentstroke}%
\pgfsetdash{}{0pt}%
\pgfpathmoveto{\pgfqpoint{3.242685in}{2.260597in}}%
\pgfpathlineto{\pgfqpoint{3.140937in}{1.604158in}}%
\pgfusepath{stroke}%
\end{pgfscope}%
\begin{pgfscope}%
\pgfpathrectangle{\pgfqpoint{0.100000in}{0.212622in}}{\pgfqpoint{3.696000in}{3.696000in}}%
\pgfusepath{clip}%
\pgfsetrectcap%
\pgfsetroundjoin%
\pgfsetlinewidth{1.505625pt}%
\definecolor{currentstroke}{rgb}{1.000000,0.000000,0.000000}%
\pgfsetstrokecolor{currentstroke}%
\pgfsetdash{}{0pt}%
\pgfpathmoveto{\pgfqpoint{3.241941in}{2.260212in}}%
\pgfpathlineto{\pgfqpoint{3.140937in}{1.604158in}}%
\pgfusepath{stroke}%
\end{pgfscope}%
\begin{pgfscope}%
\pgfpathrectangle{\pgfqpoint{0.100000in}{0.212622in}}{\pgfqpoint{3.696000in}{3.696000in}}%
\pgfusepath{clip}%
\pgfsetrectcap%
\pgfsetroundjoin%
\pgfsetlinewidth{1.505625pt}%
\definecolor{currentstroke}{rgb}{1.000000,0.000000,0.000000}%
\pgfsetstrokecolor{currentstroke}%
\pgfsetdash{}{0pt}%
\pgfpathmoveto{\pgfqpoint{3.240810in}{2.259627in}}%
\pgfpathlineto{\pgfqpoint{3.140937in}{1.604158in}}%
\pgfusepath{stroke}%
\end{pgfscope}%
\begin{pgfscope}%
\pgfpathrectangle{\pgfqpoint{0.100000in}{0.212622in}}{\pgfqpoint{3.696000in}{3.696000in}}%
\pgfusepath{clip}%
\pgfsetrectcap%
\pgfsetroundjoin%
\pgfsetlinewidth{1.505625pt}%
\definecolor{currentstroke}{rgb}{1.000000,0.000000,0.000000}%
\pgfsetstrokecolor{currentstroke}%
\pgfsetdash{}{0pt}%
\pgfpathmoveto{\pgfqpoint{3.238724in}{2.259026in}}%
\pgfpathlineto{\pgfqpoint{3.140937in}{1.604158in}}%
\pgfusepath{stroke}%
\end{pgfscope}%
\begin{pgfscope}%
\pgfpathrectangle{\pgfqpoint{0.100000in}{0.212622in}}{\pgfqpoint{3.696000in}{3.696000in}}%
\pgfusepath{clip}%
\pgfsetrectcap%
\pgfsetroundjoin%
\pgfsetlinewidth{1.505625pt}%
\definecolor{currentstroke}{rgb}{1.000000,0.000000,0.000000}%
\pgfsetstrokecolor{currentstroke}%
\pgfsetdash{}{0pt}%
\pgfpathmoveto{\pgfqpoint{3.237618in}{2.258755in}}%
\pgfpathlineto{\pgfqpoint{3.140937in}{1.604158in}}%
\pgfusepath{stroke}%
\end{pgfscope}%
\begin{pgfscope}%
\pgfpathrectangle{\pgfqpoint{0.100000in}{0.212622in}}{\pgfqpoint{3.696000in}{3.696000in}}%
\pgfusepath{clip}%
\pgfsetrectcap%
\pgfsetroundjoin%
\pgfsetlinewidth{1.505625pt}%
\definecolor{currentstroke}{rgb}{1.000000,0.000000,0.000000}%
\pgfsetstrokecolor{currentstroke}%
\pgfsetdash{}{0pt}%
\pgfpathmoveto{\pgfqpoint{3.237036in}{2.258621in}}%
\pgfpathlineto{\pgfqpoint{3.133597in}{1.596899in}}%
\pgfusepath{stroke}%
\end{pgfscope}%
\begin{pgfscope}%
\pgfpathrectangle{\pgfqpoint{0.100000in}{0.212622in}}{\pgfqpoint{3.696000in}{3.696000in}}%
\pgfusepath{clip}%
\pgfsetrectcap%
\pgfsetroundjoin%
\pgfsetlinewidth{1.505625pt}%
\definecolor{currentstroke}{rgb}{1.000000,0.000000,0.000000}%
\pgfsetstrokecolor{currentstroke}%
\pgfsetdash{}{0pt}%
\pgfpathmoveto{\pgfqpoint{3.236692in}{2.258560in}}%
\pgfpathlineto{\pgfqpoint{3.133597in}{1.596899in}}%
\pgfusepath{stroke}%
\end{pgfscope}%
\begin{pgfscope}%
\pgfpathrectangle{\pgfqpoint{0.100000in}{0.212622in}}{\pgfqpoint{3.696000in}{3.696000in}}%
\pgfusepath{clip}%
\pgfsetrectcap%
\pgfsetroundjoin%
\pgfsetlinewidth{1.505625pt}%
\definecolor{currentstroke}{rgb}{1.000000,0.000000,0.000000}%
\pgfsetstrokecolor{currentstroke}%
\pgfsetdash{}{0pt}%
\pgfpathmoveto{\pgfqpoint{3.235712in}{2.258341in}}%
\pgfpathlineto{\pgfqpoint{3.133597in}{1.596899in}}%
\pgfusepath{stroke}%
\end{pgfscope}%
\begin{pgfscope}%
\pgfpathrectangle{\pgfqpoint{0.100000in}{0.212622in}}{\pgfqpoint{3.696000in}{3.696000in}}%
\pgfusepath{clip}%
\pgfsetrectcap%
\pgfsetroundjoin%
\pgfsetlinewidth{1.505625pt}%
\definecolor{currentstroke}{rgb}{1.000000,0.000000,0.000000}%
\pgfsetstrokecolor{currentstroke}%
\pgfsetdash{}{0pt}%
\pgfpathmoveto{\pgfqpoint{3.235147in}{2.258300in}}%
\pgfpathlineto{\pgfqpoint{3.133597in}{1.596899in}}%
\pgfusepath{stroke}%
\end{pgfscope}%
\begin{pgfscope}%
\pgfpathrectangle{\pgfqpoint{0.100000in}{0.212622in}}{\pgfqpoint{3.696000in}{3.696000in}}%
\pgfusepath{clip}%
\pgfsetrectcap%
\pgfsetroundjoin%
\pgfsetlinewidth{1.505625pt}%
\definecolor{currentstroke}{rgb}{1.000000,0.000000,0.000000}%
\pgfsetstrokecolor{currentstroke}%
\pgfsetdash{}{0pt}%
\pgfpathmoveto{\pgfqpoint{3.234826in}{2.258232in}}%
\pgfpathlineto{\pgfqpoint{3.133597in}{1.596899in}}%
\pgfusepath{stroke}%
\end{pgfscope}%
\begin{pgfscope}%
\pgfpathrectangle{\pgfqpoint{0.100000in}{0.212622in}}{\pgfqpoint{3.696000in}{3.696000in}}%
\pgfusepath{clip}%
\pgfsetrectcap%
\pgfsetroundjoin%
\pgfsetlinewidth{1.505625pt}%
\definecolor{currentstroke}{rgb}{1.000000,0.000000,0.000000}%
\pgfsetstrokecolor{currentstroke}%
\pgfsetdash{}{0pt}%
\pgfpathmoveto{\pgfqpoint{3.234658in}{2.258195in}}%
\pgfpathlineto{\pgfqpoint{3.133597in}{1.596899in}}%
\pgfusepath{stroke}%
\end{pgfscope}%
\begin{pgfscope}%
\pgfpathrectangle{\pgfqpoint{0.100000in}{0.212622in}}{\pgfqpoint{3.696000in}{3.696000in}}%
\pgfusepath{clip}%
\pgfsetrectcap%
\pgfsetroundjoin%
\pgfsetlinewidth{1.505625pt}%
\definecolor{currentstroke}{rgb}{1.000000,0.000000,0.000000}%
\pgfsetstrokecolor{currentstroke}%
\pgfsetdash{}{0pt}%
\pgfpathmoveto{\pgfqpoint{3.233425in}{2.257889in}}%
\pgfpathlineto{\pgfqpoint{3.133597in}{1.596899in}}%
\pgfusepath{stroke}%
\end{pgfscope}%
\begin{pgfscope}%
\pgfpathrectangle{\pgfqpoint{0.100000in}{0.212622in}}{\pgfqpoint{3.696000in}{3.696000in}}%
\pgfusepath{clip}%
\pgfsetrectcap%
\pgfsetroundjoin%
\pgfsetlinewidth{1.505625pt}%
\definecolor{currentstroke}{rgb}{1.000000,0.000000,0.000000}%
\pgfsetstrokecolor{currentstroke}%
\pgfsetdash{}{0pt}%
\pgfpathmoveto{\pgfqpoint{3.232741in}{2.257632in}}%
\pgfpathlineto{\pgfqpoint{3.133597in}{1.596899in}}%
\pgfusepath{stroke}%
\end{pgfscope}%
\begin{pgfscope}%
\pgfpathrectangle{\pgfqpoint{0.100000in}{0.212622in}}{\pgfqpoint{3.696000in}{3.696000in}}%
\pgfusepath{clip}%
\pgfsetrectcap%
\pgfsetroundjoin%
\pgfsetlinewidth{1.505625pt}%
\definecolor{currentstroke}{rgb}{1.000000,0.000000,0.000000}%
\pgfsetstrokecolor{currentstroke}%
\pgfsetdash{}{0pt}%
\pgfpathmoveto{\pgfqpoint{3.231821in}{2.257380in}}%
\pgfpathlineto{\pgfqpoint{3.133597in}{1.596899in}}%
\pgfusepath{stroke}%
\end{pgfscope}%
\begin{pgfscope}%
\pgfpathrectangle{\pgfqpoint{0.100000in}{0.212622in}}{\pgfqpoint{3.696000in}{3.696000in}}%
\pgfusepath{clip}%
\pgfsetrectcap%
\pgfsetroundjoin%
\pgfsetlinewidth{1.505625pt}%
\definecolor{currentstroke}{rgb}{1.000000,0.000000,0.000000}%
\pgfsetstrokecolor{currentstroke}%
\pgfsetdash{}{0pt}%
\pgfpathmoveto{\pgfqpoint{3.231291in}{2.257196in}}%
\pgfpathlineto{\pgfqpoint{3.133597in}{1.596899in}}%
\pgfusepath{stroke}%
\end{pgfscope}%
\begin{pgfscope}%
\pgfpathrectangle{\pgfqpoint{0.100000in}{0.212622in}}{\pgfqpoint{3.696000in}{3.696000in}}%
\pgfusepath{clip}%
\pgfsetrectcap%
\pgfsetroundjoin%
\pgfsetlinewidth{1.505625pt}%
\definecolor{currentstroke}{rgb}{1.000000,0.000000,0.000000}%
\pgfsetstrokecolor{currentstroke}%
\pgfsetdash{}{0pt}%
\pgfpathmoveto{\pgfqpoint{3.230033in}{2.256795in}}%
\pgfpathlineto{\pgfqpoint{3.133597in}{1.596899in}}%
\pgfusepath{stroke}%
\end{pgfscope}%
\begin{pgfscope}%
\pgfpathrectangle{\pgfqpoint{0.100000in}{0.212622in}}{\pgfqpoint{3.696000in}{3.696000in}}%
\pgfusepath{clip}%
\pgfsetrectcap%
\pgfsetroundjoin%
\pgfsetlinewidth{1.505625pt}%
\definecolor{currentstroke}{rgb}{1.000000,0.000000,0.000000}%
\pgfsetstrokecolor{currentstroke}%
\pgfsetdash{}{0pt}%
\pgfpathmoveto{\pgfqpoint{3.229364in}{2.256704in}}%
\pgfpathlineto{\pgfqpoint{3.126248in}{1.589632in}}%
\pgfusepath{stroke}%
\end{pgfscope}%
\begin{pgfscope}%
\pgfpathrectangle{\pgfqpoint{0.100000in}{0.212622in}}{\pgfqpoint{3.696000in}{3.696000in}}%
\pgfusepath{clip}%
\pgfsetrectcap%
\pgfsetroundjoin%
\pgfsetlinewidth{1.505625pt}%
\definecolor{currentstroke}{rgb}{1.000000,0.000000,0.000000}%
\pgfsetstrokecolor{currentstroke}%
\pgfsetdash{}{0pt}%
\pgfpathmoveto{\pgfqpoint{3.228967in}{2.256586in}}%
\pgfpathlineto{\pgfqpoint{3.126248in}{1.589632in}}%
\pgfusepath{stroke}%
\end{pgfscope}%
\begin{pgfscope}%
\pgfpathrectangle{\pgfqpoint{0.100000in}{0.212622in}}{\pgfqpoint{3.696000in}{3.696000in}}%
\pgfusepath{clip}%
\pgfsetrectcap%
\pgfsetroundjoin%
\pgfsetlinewidth{1.505625pt}%
\definecolor{currentstroke}{rgb}{1.000000,0.000000,0.000000}%
\pgfsetstrokecolor{currentstroke}%
\pgfsetdash{}{0pt}%
\pgfpathmoveto{\pgfqpoint{3.228156in}{2.256298in}}%
\pgfpathlineto{\pgfqpoint{3.126248in}{1.589632in}}%
\pgfusepath{stroke}%
\end{pgfscope}%
\begin{pgfscope}%
\pgfpathrectangle{\pgfqpoint{0.100000in}{0.212622in}}{\pgfqpoint{3.696000in}{3.696000in}}%
\pgfusepath{clip}%
\pgfsetrectcap%
\pgfsetroundjoin%
\pgfsetlinewidth{1.505625pt}%
\definecolor{currentstroke}{rgb}{1.000000,0.000000,0.000000}%
\pgfsetstrokecolor{currentstroke}%
\pgfsetdash{}{0pt}%
\pgfpathmoveto{\pgfqpoint{3.226671in}{2.256023in}}%
\pgfpathlineto{\pgfqpoint{3.126248in}{1.589632in}}%
\pgfusepath{stroke}%
\end{pgfscope}%
\begin{pgfscope}%
\pgfpathrectangle{\pgfqpoint{0.100000in}{0.212622in}}{\pgfqpoint{3.696000in}{3.696000in}}%
\pgfusepath{clip}%
\pgfsetrectcap%
\pgfsetroundjoin%
\pgfsetlinewidth{1.505625pt}%
\definecolor{currentstroke}{rgb}{1.000000,0.000000,0.000000}%
\pgfsetstrokecolor{currentstroke}%
\pgfsetdash{}{0pt}%
\pgfpathmoveto{\pgfqpoint{3.225846in}{2.255756in}}%
\pgfpathlineto{\pgfqpoint{3.126248in}{1.589632in}}%
\pgfusepath{stroke}%
\end{pgfscope}%
\begin{pgfscope}%
\pgfpathrectangle{\pgfqpoint{0.100000in}{0.212622in}}{\pgfqpoint{3.696000in}{3.696000in}}%
\pgfusepath{clip}%
\pgfsetrectcap%
\pgfsetroundjoin%
\pgfsetlinewidth{1.505625pt}%
\definecolor{currentstroke}{rgb}{1.000000,0.000000,0.000000}%
\pgfsetstrokecolor{currentstroke}%
\pgfsetdash{}{0pt}%
\pgfpathmoveto{\pgfqpoint{3.225416in}{2.255620in}}%
\pgfpathlineto{\pgfqpoint{3.126248in}{1.589632in}}%
\pgfusepath{stroke}%
\end{pgfscope}%
\begin{pgfscope}%
\pgfpathrectangle{\pgfqpoint{0.100000in}{0.212622in}}{\pgfqpoint{3.696000in}{3.696000in}}%
\pgfusepath{clip}%
\pgfsetrectcap%
\pgfsetroundjoin%
\pgfsetlinewidth{1.505625pt}%
\definecolor{currentstroke}{rgb}{1.000000,0.000000,0.000000}%
\pgfsetstrokecolor{currentstroke}%
\pgfsetdash{}{0pt}%
\pgfpathmoveto{\pgfqpoint{3.224462in}{2.255263in}}%
\pgfpathlineto{\pgfqpoint{3.126248in}{1.589632in}}%
\pgfusepath{stroke}%
\end{pgfscope}%
\begin{pgfscope}%
\pgfpathrectangle{\pgfqpoint{0.100000in}{0.212622in}}{\pgfqpoint{3.696000in}{3.696000in}}%
\pgfusepath{clip}%
\pgfsetrectcap%
\pgfsetroundjoin%
\pgfsetlinewidth{1.505625pt}%
\definecolor{currentstroke}{rgb}{1.000000,0.000000,0.000000}%
\pgfsetstrokecolor{currentstroke}%
\pgfsetdash{}{0pt}%
\pgfpathmoveto{\pgfqpoint{3.222855in}{2.254588in}}%
\pgfpathlineto{\pgfqpoint{3.126248in}{1.589632in}}%
\pgfusepath{stroke}%
\end{pgfscope}%
\begin{pgfscope}%
\pgfpathrectangle{\pgfqpoint{0.100000in}{0.212622in}}{\pgfqpoint{3.696000in}{3.696000in}}%
\pgfusepath{clip}%
\pgfsetrectcap%
\pgfsetroundjoin%
\pgfsetlinewidth{1.505625pt}%
\definecolor{currentstroke}{rgb}{1.000000,0.000000,0.000000}%
\pgfsetstrokecolor{currentstroke}%
\pgfsetdash{}{0pt}%
\pgfpathmoveto{\pgfqpoint{3.220887in}{2.253943in}}%
\pgfpathlineto{\pgfqpoint{3.118890in}{1.582355in}}%
\pgfusepath{stroke}%
\end{pgfscope}%
\begin{pgfscope}%
\pgfpathrectangle{\pgfqpoint{0.100000in}{0.212622in}}{\pgfqpoint{3.696000in}{3.696000in}}%
\pgfusepath{clip}%
\pgfsetrectcap%
\pgfsetroundjoin%
\pgfsetlinewidth{1.505625pt}%
\definecolor{currentstroke}{rgb}{1.000000,0.000000,0.000000}%
\pgfsetstrokecolor{currentstroke}%
\pgfsetdash{}{0pt}%
\pgfpathmoveto{\pgfqpoint{3.219745in}{2.253548in}}%
\pgfpathlineto{\pgfqpoint{3.118890in}{1.582355in}}%
\pgfusepath{stroke}%
\end{pgfscope}%
\begin{pgfscope}%
\pgfpathrectangle{\pgfqpoint{0.100000in}{0.212622in}}{\pgfqpoint{3.696000in}{3.696000in}}%
\pgfusepath{clip}%
\pgfsetrectcap%
\pgfsetroundjoin%
\pgfsetlinewidth{1.505625pt}%
\definecolor{currentstroke}{rgb}{1.000000,0.000000,0.000000}%
\pgfsetstrokecolor{currentstroke}%
\pgfsetdash{}{0pt}%
\pgfpathmoveto{\pgfqpoint{3.219153in}{2.253319in}}%
\pgfpathlineto{\pgfqpoint{3.118890in}{1.582355in}}%
\pgfusepath{stroke}%
\end{pgfscope}%
\begin{pgfscope}%
\pgfpathrectangle{\pgfqpoint{0.100000in}{0.212622in}}{\pgfqpoint{3.696000in}{3.696000in}}%
\pgfusepath{clip}%
\pgfsetrectcap%
\pgfsetroundjoin%
\pgfsetlinewidth{1.505625pt}%
\definecolor{currentstroke}{rgb}{1.000000,0.000000,0.000000}%
\pgfsetstrokecolor{currentstroke}%
\pgfsetdash{}{0pt}%
\pgfpathmoveto{\pgfqpoint{3.217707in}{2.252683in}}%
\pgfpathlineto{\pgfqpoint{3.118890in}{1.582355in}}%
\pgfusepath{stroke}%
\end{pgfscope}%
\begin{pgfscope}%
\pgfpathrectangle{\pgfqpoint{0.100000in}{0.212622in}}{\pgfqpoint{3.696000in}{3.696000in}}%
\pgfusepath{clip}%
\pgfsetrectcap%
\pgfsetroundjoin%
\pgfsetlinewidth{1.505625pt}%
\definecolor{currentstroke}{rgb}{1.000000,0.000000,0.000000}%
\pgfsetstrokecolor{currentstroke}%
\pgfsetdash{}{0pt}%
\pgfpathmoveto{\pgfqpoint{3.216925in}{2.252381in}}%
\pgfpathlineto{\pgfqpoint{3.118890in}{1.582355in}}%
\pgfusepath{stroke}%
\end{pgfscope}%
\begin{pgfscope}%
\pgfpathrectangle{\pgfqpoint{0.100000in}{0.212622in}}{\pgfqpoint{3.696000in}{3.696000in}}%
\pgfusepath{clip}%
\pgfsetrectcap%
\pgfsetroundjoin%
\pgfsetlinewidth{1.505625pt}%
\definecolor{currentstroke}{rgb}{1.000000,0.000000,0.000000}%
\pgfsetstrokecolor{currentstroke}%
\pgfsetdash{}{0pt}%
\pgfpathmoveto{\pgfqpoint{3.216520in}{2.252225in}}%
\pgfpathlineto{\pgfqpoint{3.118890in}{1.582355in}}%
\pgfusepath{stroke}%
\end{pgfscope}%
\begin{pgfscope}%
\pgfpathrectangle{\pgfqpoint{0.100000in}{0.212622in}}{\pgfqpoint{3.696000in}{3.696000in}}%
\pgfusepath{clip}%
\pgfsetrectcap%
\pgfsetroundjoin%
\pgfsetlinewidth{1.505625pt}%
\definecolor{currentstroke}{rgb}{1.000000,0.000000,0.000000}%
\pgfsetstrokecolor{currentstroke}%
\pgfsetdash{}{0pt}%
\pgfpathmoveto{\pgfqpoint{3.216283in}{2.252113in}}%
\pgfpathlineto{\pgfqpoint{3.118890in}{1.582355in}}%
\pgfusepath{stroke}%
\end{pgfscope}%
\begin{pgfscope}%
\pgfpathrectangle{\pgfqpoint{0.100000in}{0.212622in}}{\pgfqpoint{3.696000in}{3.696000in}}%
\pgfusepath{clip}%
\pgfsetrectcap%
\pgfsetroundjoin%
\pgfsetlinewidth{1.505625pt}%
\definecolor{currentstroke}{rgb}{1.000000,0.000000,0.000000}%
\pgfsetstrokecolor{currentstroke}%
\pgfsetdash{}{0pt}%
\pgfpathmoveto{\pgfqpoint{3.214963in}{2.251585in}}%
\pgfpathlineto{\pgfqpoint{3.118890in}{1.582355in}}%
\pgfusepath{stroke}%
\end{pgfscope}%
\begin{pgfscope}%
\pgfpathrectangle{\pgfqpoint{0.100000in}{0.212622in}}{\pgfqpoint{3.696000in}{3.696000in}}%
\pgfusepath{clip}%
\pgfsetrectcap%
\pgfsetroundjoin%
\pgfsetlinewidth{1.505625pt}%
\definecolor{currentstroke}{rgb}{1.000000,0.000000,0.000000}%
\pgfsetstrokecolor{currentstroke}%
\pgfsetdash{}{0pt}%
\pgfpathmoveto{\pgfqpoint{3.214219in}{2.251256in}}%
\pgfpathlineto{\pgfqpoint{3.111523in}{1.575070in}}%
\pgfusepath{stroke}%
\end{pgfscope}%
\begin{pgfscope}%
\pgfpathrectangle{\pgfqpoint{0.100000in}{0.212622in}}{\pgfqpoint{3.696000in}{3.696000in}}%
\pgfusepath{clip}%
\pgfsetrectcap%
\pgfsetroundjoin%
\pgfsetlinewidth{1.505625pt}%
\definecolor{currentstroke}{rgb}{1.000000,0.000000,0.000000}%
\pgfsetstrokecolor{currentstroke}%
\pgfsetdash{}{0pt}%
\pgfpathmoveto{\pgfqpoint{3.213790in}{2.251063in}}%
\pgfpathlineto{\pgfqpoint{3.111523in}{1.575070in}}%
\pgfusepath{stroke}%
\end{pgfscope}%
\begin{pgfscope}%
\pgfpathrectangle{\pgfqpoint{0.100000in}{0.212622in}}{\pgfqpoint{3.696000in}{3.696000in}}%
\pgfusepath{clip}%
\pgfsetrectcap%
\pgfsetroundjoin%
\pgfsetlinewidth{1.505625pt}%
\definecolor{currentstroke}{rgb}{1.000000,0.000000,0.000000}%
\pgfsetstrokecolor{currentstroke}%
\pgfsetdash{}{0pt}%
\pgfpathmoveto{\pgfqpoint{3.213562in}{2.250957in}}%
\pgfpathlineto{\pgfqpoint{3.111523in}{1.575070in}}%
\pgfusepath{stroke}%
\end{pgfscope}%
\begin{pgfscope}%
\pgfpathrectangle{\pgfqpoint{0.100000in}{0.212622in}}{\pgfqpoint{3.696000in}{3.696000in}}%
\pgfusepath{clip}%
\pgfsetrectcap%
\pgfsetroundjoin%
\pgfsetlinewidth{1.505625pt}%
\definecolor{currentstroke}{rgb}{1.000000,0.000000,0.000000}%
\pgfsetstrokecolor{currentstroke}%
\pgfsetdash{}{0pt}%
\pgfpathmoveto{\pgfqpoint{3.212529in}{2.250590in}}%
\pgfpathlineto{\pgfqpoint{3.111523in}{1.575070in}}%
\pgfusepath{stroke}%
\end{pgfscope}%
\begin{pgfscope}%
\pgfpathrectangle{\pgfqpoint{0.100000in}{0.212622in}}{\pgfqpoint{3.696000in}{3.696000in}}%
\pgfusepath{clip}%
\pgfsetrectcap%
\pgfsetroundjoin%
\pgfsetlinewidth{1.505625pt}%
\definecolor{currentstroke}{rgb}{1.000000,0.000000,0.000000}%
\pgfsetstrokecolor{currentstroke}%
\pgfsetdash{}{0pt}%
\pgfpathmoveto{\pgfqpoint{3.211951in}{2.250386in}}%
\pgfpathlineto{\pgfqpoint{3.111523in}{1.575070in}}%
\pgfusepath{stroke}%
\end{pgfscope}%
\begin{pgfscope}%
\pgfpathrectangle{\pgfqpoint{0.100000in}{0.212622in}}{\pgfqpoint{3.696000in}{3.696000in}}%
\pgfusepath{clip}%
\pgfsetrectcap%
\pgfsetroundjoin%
\pgfsetlinewidth{1.505625pt}%
\definecolor{currentstroke}{rgb}{1.000000,0.000000,0.000000}%
\pgfsetstrokecolor{currentstroke}%
\pgfsetdash{}{0pt}%
\pgfpathmoveto{\pgfqpoint{3.211652in}{2.250288in}}%
\pgfpathlineto{\pgfqpoint{3.111523in}{1.575070in}}%
\pgfusepath{stroke}%
\end{pgfscope}%
\begin{pgfscope}%
\pgfpathrectangle{\pgfqpoint{0.100000in}{0.212622in}}{\pgfqpoint{3.696000in}{3.696000in}}%
\pgfusepath{clip}%
\pgfsetrectcap%
\pgfsetroundjoin%
\pgfsetlinewidth{1.505625pt}%
\definecolor{currentstroke}{rgb}{1.000000,0.000000,0.000000}%
\pgfsetstrokecolor{currentstroke}%
\pgfsetdash{}{0pt}%
\pgfpathmoveto{\pgfqpoint{3.210921in}{2.250207in}}%
\pgfpathlineto{\pgfqpoint{3.111523in}{1.575070in}}%
\pgfusepath{stroke}%
\end{pgfscope}%
\begin{pgfscope}%
\pgfpathrectangle{\pgfqpoint{0.100000in}{0.212622in}}{\pgfqpoint{3.696000in}{3.696000in}}%
\pgfusepath{clip}%
\pgfsetrectcap%
\pgfsetroundjoin%
\pgfsetlinewidth{1.505625pt}%
\definecolor{currentstroke}{rgb}{1.000000,0.000000,0.000000}%
\pgfsetstrokecolor{currentstroke}%
\pgfsetdash{}{0pt}%
\pgfpathmoveto{\pgfqpoint{3.209717in}{2.249982in}}%
\pgfpathlineto{\pgfqpoint{3.111523in}{1.575070in}}%
\pgfusepath{stroke}%
\end{pgfscope}%
\begin{pgfscope}%
\pgfpathrectangle{\pgfqpoint{0.100000in}{0.212622in}}{\pgfqpoint{3.696000in}{3.696000in}}%
\pgfusepath{clip}%
\pgfsetrectcap%
\pgfsetroundjoin%
\pgfsetlinewidth{1.505625pt}%
\definecolor{currentstroke}{rgb}{1.000000,0.000000,0.000000}%
\pgfsetstrokecolor{currentstroke}%
\pgfsetdash{}{0pt}%
\pgfpathmoveto{\pgfqpoint{3.209093in}{2.249811in}}%
\pgfpathlineto{\pgfqpoint{3.111523in}{1.575070in}}%
\pgfusepath{stroke}%
\end{pgfscope}%
\begin{pgfscope}%
\pgfpathrectangle{\pgfqpoint{0.100000in}{0.212622in}}{\pgfqpoint{3.696000in}{3.696000in}}%
\pgfusepath{clip}%
\pgfsetrectcap%
\pgfsetroundjoin%
\pgfsetlinewidth{1.505625pt}%
\definecolor{currentstroke}{rgb}{1.000000,0.000000,0.000000}%
\pgfsetstrokecolor{currentstroke}%
\pgfsetdash{}{0pt}%
\pgfpathmoveto{\pgfqpoint{3.207598in}{2.249435in}}%
\pgfpathlineto{\pgfqpoint{3.111523in}{1.575070in}}%
\pgfusepath{stroke}%
\end{pgfscope}%
\begin{pgfscope}%
\pgfpathrectangle{\pgfqpoint{0.100000in}{0.212622in}}{\pgfqpoint{3.696000in}{3.696000in}}%
\pgfusepath{clip}%
\pgfsetrectcap%
\pgfsetroundjoin%
\pgfsetlinewidth{1.505625pt}%
\definecolor{currentstroke}{rgb}{1.000000,0.000000,0.000000}%
\pgfsetstrokecolor{currentstroke}%
\pgfsetdash{}{0pt}%
\pgfpathmoveto{\pgfqpoint{3.205869in}{2.248929in}}%
\pgfpathlineto{\pgfqpoint{3.104147in}{1.567776in}}%
\pgfusepath{stroke}%
\end{pgfscope}%
\begin{pgfscope}%
\pgfpathrectangle{\pgfqpoint{0.100000in}{0.212622in}}{\pgfqpoint{3.696000in}{3.696000in}}%
\pgfusepath{clip}%
\pgfsetrectcap%
\pgfsetroundjoin%
\pgfsetlinewidth{1.505625pt}%
\definecolor{currentstroke}{rgb}{1.000000,0.000000,0.000000}%
\pgfsetstrokecolor{currentstroke}%
\pgfsetdash{}{0pt}%
\pgfpathmoveto{\pgfqpoint{3.204949in}{2.248784in}}%
\pgfpathlineto{\pgfqpoint{3.104147in}{1.567776in}}%
\pgfusepath{stroke}%
\end{pgfscope}%
\begin{pgfscope}%
\pgfpathrectangle{\pgfqpoint{0.100000in}{0.212622in}}{\pgfqpoint{3.696000in}{3.696000in}}%
\pgfusepath{clip}%
\pgfsetrectcap%
\pgfsetroundjoin%
\pgfsetlinewidth{1.505625pt}%
\definecolor{currentstroke}{rgb}{1.000000,0.000000,0.000000}%
\pgfsetstrokecolor{currentstroke}%
\pgfsetdash{}{0pt}%
\pgfpathmoveto{\pgfqpoint{3.204414in}{2.248589in}}%
\pgfpathlineto{\pgfqpoint{3.104147in}{1.567776in}}%
\pgfusepath{stroke}%
\end{pgfscope}%
\begin{pgfscope}%
\pgfpathrectangle{\pgfqpoint{0.100000in}{0.212622in}}{\pgfqpoint{3.696000in}{3.696000in}}%
\pgfusepath{clip}%
\pgfsetrectcap%
\pgfsetroundjoin%
\pgfsetlinewidth{1.505625pt}%
\definecolor{currentstroke}{rgb}{1.000000,0.000000,0.000000}%
\pgfsetstrokecolor{currentstroke}%
\pgfsetdash{}{0pt}%
\pgfpathmoveto{\pgfqpoint{3.202746in}{2.247674in}}%
\pgfpathlineto{\pgfqpoint{3.104147in}{1.567776in}}%
\pgfusepath{stroke}%
\end{pgfscope}%
\begin{pgfscope}%
\pgfpathrectangle{\pgfqpoint{0.100000in}{0.212622in}}{\pgfqpoint{3.696000in}{3.696000in}}%
\pgfusepath{clip}%
\pgfsetrectcap%
\pgfsetroundjoin%
\pgfsetlinewidth{1.505625pt}%
\definecolor{currentstroke}{rgb}{1.000000,0.000000,0.000000}%
\pgfsetstrokecolor{currentstroke}%
\pgfsetdash{}{0pt}%
\pgfpathmoveto{\pgfqpoint{3.201895in}{2.247490in}}%
\pgfpathlineto{\pgfqpoint{3.104147in}{1.567776in}}%
\pgfusepath{stroke}%
\end{pgfscope}%
\begin{pgfscope}%
\pgfpathrectangle{\pgfqpoint{0.100000in}{0.212622in}}{\pgfqpoint{3.696000in}{3.696000in}}%
\pgfusepath{clip}%
\pgfsetrectcap%
\pgfsetroundjoin%
\pgfsetlinewidth{1.505625pt}%
\definecolor{currentstroke}{rgb}{1.000000,0.000000,0.000000}%
\pgfsetstrokecolor{currentstroke}%
\pgfsetdash{}{0pt}%
\pgfpathmoveto{\pgfqpoint{3.200721in}{2.247051in}}%
\pgfpathlineto{\pgfqpoint{3.104147in}{1.567776in}}%
\pgfusepath{stroke}%
\end{pgfscope}%
\begin{pgfscope}%
\pgfpathrectangle{\pgfqpoint{0.100000in}{0.212622in}}{\pgfqpoint{3.696000in}{3.696000in}}%
\pgfusepath{clip}%
\pgfsetrectcap%
\pgfsetroundjoin%
\pgfsetlinewidth{1.505625pt}%
\definecolor{currentstroke}{rgb}{1.000000,0.000000,0.000000}%
\pgfsetstrokecolor{currentstroke}%
\pgfsetdash{}{0pt}%
\pgfpathmoveto{\pgfqpoint{3.200080in}{2.246788in}}%
\pgfpathlineto{\pgfqpoint{3.104147in}{1.567776in}}%
\pgfusepath{stroke}%
\end{pgfscope}%
\begin{pgfscope}%
\pgfpathrectangle{\pgfqpoint{0.100000in}{0.212622in}}{\pgfqpoint{3.696000in}{3.696000in}}%
\pgfusepath{clip}%
\pgfsetrectcap%
\pgfsetroundjoin%
\pgfsetlinewidth{1.505625pt}%
\definecolor{currentstroke}{rgb}{1.000000,0.000000,0.000000}%
\pgfsetstrokecolor{currentstroke}%
\pgfsetdash{}{0pt}%
\pgfpathmoveto{\pgfqpoint{3.199126in}{2.246289in}}%
\pgfpathlineto{\pgfqpoint{3.096762in}{1.560473in}}%
\pgfusepath{stroke}%
\end{pgfscope}%
\begin{pgfscope}%
\pgfpathrectangle{\pgfqpoint{0.100000in}{0.212622in}}{\pgfqpoint{3.696000in}{3.696000in}}%
\pgfusepath{clip}%
\pgfsetrectcap%
\pgfsetroundjoin%
\pgfsetlinewidth{1.505625pt}%
\definecolor{currentstroke}{rgb}{1.000000,0.000000,0.000000}%
\pgfsetstrokecolor{currentstroke}%
\pgfsetdash{}{0pt}%
\pgfpathmoveto{\pgfqpoint{3.197465in}{2.245984in}}%
\pgfpathlineto{\pgfqpoint{3.096762in}{1.560473in}}%
\pgfusepath{stroke}%
\end{pgfscope}%
\begin{pgfscope}%
\pgfpathrectangle{\pgfqpoint{0.100000in}{0.212622in}}{\pgfqpoint{3.696000in}{3.696000in}}%
\pgfusepath{clip}%
\pgfsetrectcap%
\pgfsetroundjoin%
\pgfsetlinewidth{1.505625pt}%
\definecolor{currentstroke}{rgb}{1.000000,0.000000,0.000000}%
\pgfsetstrokecolor{currentstroke}%
\pgfsetdash{}{0pt}%
\pgfpathmoveto{\pgfqpoint{3.196594in}{2.245976in}}%
\pgfpathlineto{\pgfqpoint{3.096762in}{1.560473in}}%
\pgfusepath{stroke}%
\end{pgfscope}%
\begin{pgfscope}%
\pgfpathrectangle{\pgfqpoint{0.100000in}{0.212622in}}{\pgfqpoint{3.696000in}{3.696000in}}%
\pgfusepath{clip}%
\pgfsetrectcap%
\pgfsetroundjoin%
\pgfsetlinewidth{1.505625pt}%
\definecolor{currentstroke}{rgb}{1.000000,0.000000,0.000000}%
\pgfsetstrokecolor{currentstroke}%
\pgfsetdash{}{0pt}%
\pgfpathmoveto{\pgfqpoint{3.196090in}{2.245898in}}%
\pgfpathlineto{\pgfqpoint{3.096762in}{1.560473in}}%
\pgfusepath{stroke}%
\end{pgfscope}%
\begin{pgfscope}%
\pgfpathrectangle{\pgfqpoint{0.100000in}{0.212622in}}{\pgfqpoint{3.696000in}{3.696000in}}%
\pgfusepath{clip}%
\pgfsetrectcap%
\pgfsetroundjoin%
\pgfsetlinewidth{1.505625pt}%
\definecolor{currentstroke}{rgb}{1.000000,0.000000,0.000000}%
\pgfsetstrokecolor{currentstroke}%
\pgfsetdash{}{0pt}%
\pgfpathmoveto{\pgfqpoint{3.194995in}{2.245641in}}%
\pgfpathlineto{\pgfqpoint{3.096762in}{1.560473in}}%
\pgfusepath{stroke}%
\end{pgfscope}%
\begin{pgfscope}%
\pgfpathrectangle{\pgfqpoint{0.100000in}{0.212622in}}{\pgfqpoint{3.696000in}{3.696000in}}%
\pgfusepath{clip}%
\pgfsetrectcap%
\pgfsetroundjoin%
\pgfsetlinewidth{1.505625pt}%
\definecolor{currentstroke}{rgb}{1.000000,0.000000,0.000000}%
\pgfsetstrokecolor{currentstroke}%
\pgfsetdash{}{0pt}%
\pgfpathmoveto{\pgfqpoint{3.193385in}{2.245874in}}%
\pgfpathlineto{\pgfqpoint{3.096762in}{1.560473in}}%
\pgfusepath{stroke}%
\end{pgfscope}%
\begin{pgfscope}%
\pgfpathrectangle{\pgfqpoint{0.100000in}{0.212622in}}{\pgfqpoint{3.696000in}{3.696000in}}%
\pgfusepath{clip}%
\pgfsetrectcap%
\pgfsetroundjoin%
\pgfsetlinewidth{1.505625pt}%
\definecolor{currentstroke}{rgb}{1.000000,0.000000,0.000000}%
\pgfsetstrokecolor{currentstroke}%
\pgfsetdash{}{0pt}%
\pgfpathmoveto{\pgfqpoint{3.192518in}{2.245837in}}%
\pgfpathlineto{\pgfqpoint{3.096762in}{1.560473in}}%
\pgfusepath{stroke}%
\end{pgfscope}%
\begin{pgfscope}%
\pgfpathrectangle{\pgfqpoint{0.100000in}{0.212622in}}{\pgfqpoint{3.696000in}{3.696000in}}%
\pgfusepath{clip}%
\pgfsetrectcap%
\pgfsetroundjoin%
\pgfsetlinewidth{1.505625pt}%
\definecolor{currentstroke}{rgb}{1.000000,0.000000,0.000000}%
\pgfsetstrokecolor{currentstroke}%
\pgfsetdash{}{0pt}%
\pgfpathmoveto{\pgfqpoint{3.192069in}{2.245760in}}%
\pgfpathlineto{\pgfqpoint{3.089368in}{1.553161in}}%
\pgfusepath{stroke}%
\end{pgfscope}%
\begin{pgfscope}%
\pgfpathrectangle{\pgfqpoint{0.100000in}{0.212622in}}{\pgfqpoint{3.696000in}{3.696000in}}%
\pgfusepath{clip}%
\pgfsetrectcap%
\pgfsetroundjoin%
\pgfsetlinewidth{1.505625pt}%
\definecolor{currentstroke}{rgb}{1.000000,0.000000,0.000000}%
\pgfsetstrokecolor{currentstroke}%
\pgfsetdash{}{0pt}%
\pgfpathmoveto{\pgfqpoint{3.191290in}{2.245590in}}%
\pgfpathlineto{\pgfqpoint{3.089368in}{1.553161in}}%
\pgfusepath{stroke}%
\end{pgfscope}%
\begin{pgfscope}%
\pgfpathrectangle{\pgfqpoint{0.100000in}{0.212622in}}{\pgfqpoint{3.696000in}{3.696000in}}%
\pgfusepath{clip}%
\pgfsetrectcap%
\pgfsetroundjoin%
\pgfsetlinewidth{1.505625pt}%
\definecolor{currentstroke}{rgb}{1.000000,0.000000,0.000000}%
\pgfsetstrokecolor{currentstroke}%
\pgfsetdash{}{0pt}%
\pgfpathmoveto{\pgfqpoint{3.189655in}{2.244569in}}%
\pgfpathlineto{\pgfqpoint{3.089368in}{1.553161in}}%
\pgfusepath{stroke}%
\end{pgfscope}%
\begin{pgfscope}%
\pgfpathrectangle{\pgfqpoint{0.100000in}{0.212622in}}{\pgfqpoint{3.696000in}{3.696000in}}%
\pgfusepath{clip}%
\pgfsetrectcap%
\pgfsetroundjoin%
\pgfsetlinewidth{1.505625pt}%
\definecolor{currentstroke}{rgb}{1.000000,0.000000,0.000000}%
\pgfsetstrokecolor{currentstroke}%
\pgfsetdash{}{0pt}%
\pgfpathmoveto{\pgfqpoint{3.187861in}{2.243691in}}%
\pgfpathlineto{\pgfqpoint{3.089368in}{1.553161in}}%
\pgfusepath{stroke}%
\end{pgfscope}%
\begin{pgfscope}%
\pgfpathrectangle{\pgfqpoint{0.100000in}{0.212622in}}{\pgfqpoint{3.696000in}{3.696000in}}%
\pgfusepath{clip}%
\pgfsetrectcap%
\pgfsetroundjoin%
\pgfsetlinewidth{1.505625pt}%
\definecolor{currentstroke}{rgb}{1.000000,0.000000,0.000000}%
\pgfsetstrokecolor{currentstroke}%
\pgfsetdash{}{0pt}%
\pgfpathmoveto{\pgfqpoint{3.186774in}{2.243121in}}%
\pgfpathlineto{\pgfqpoint{3.089368in}{1.553161in}}%
\pgfusepath{stroke}%
\end{pgfscope}%
\begin{pgfscope}%
\pgfpathrectangle{\pgfqpoint{0.100000in}{0.212622in}}{\pgfqpoint{3.696000in}{3.696000in}}%
\pgfusepath{clip}%
\pgfsetrectcap%
\pgfsetroundjoin%
\pgfsetlinewidth{1.505625pt}%
\definecolor{currentstroke}{rgb}{1.000000,0.000000,0.000000}%
\pgfsetstrokecolor{currentstroke}%
\pgfsetdash{}{0pt}%
\pgfpathmoveto{\pgfqpoint{3.185001in}{2.242293in}}%
\pgfpathlineto{\pgfqpoint{3.081965in}{1.545840in}}%
\pgfusepath{stroke}%
\end{pgfscope}%
\begin{pgfscope}%
\pgfpathrectangle{\pgfqpoint{0.100000in}{0.212622in}}{\pgfqpoint{3.696000in}{3.696000in}}%
\pgfusepath{clip}%
\pgfsetrectcap%
\pgfsetroundjoin%
\pgfsetlinewidth{1.505625pt}%
\definecolor{currentstroke}{rgb}{1.000000,0.000000,0.000000}%
\pgfsetstrokecolor{currentstroke}%
\pgfsetdash{}{0pt}%
\pgfpathmoveto{\pgfqpoint{3.184011in}{2.242096in}}%
\pgfpathlineto{\pgfqpoint{3.081965in}{1.545840in}}%
\pgfusepath{stroke}%
\end{pgfscope}%
\begin{pgfscope}%
\pgfpathrectangle{\pgfqpoint{0.100000in}{0.212622in}}{\pgfqpoint{3.696000in}{3.696000in}}%
\pgfusepath{clip}%
\pgfsetrectcap%
\pgfsetroundjoin%
\pgfsetlinewidth{1.505625pt}%
\definecolor{currentstroke}{rgb}{1.000000,0.000000,0.000000}%
\pgfsetstrokecolor{currentstroke}%
\pgfsetdash{}{0pt}%
\pgfpathmoveto{\pgfqpoint{3.183432in}{2.241917in}}%
\pgfpathlineto{\pgfqpoint{3.081965in}{1.545840in}}%
\pgfusepath{stroke}%
\end{pgfscope}%
\begin{pgfscope}%
\pgfpathrectangle{\pgfqpoint{0.100000in}{0.212622in}}{\pgfqpoint{3.696000in}{3.696000in}}%
\pgfusepath{clip}%
\pgfsetrectcap%
\pgfsetroundjoin%
\pgfsetlinewidth{1.505625pt}%
\definecolor{currentstroke}{rgb}{1.000000,0.000000,0.000000}%
\pgfsetstrokecolor{currentstroke}%
\pgfsetdash{}{0pt}%
\pgfpathmoveto{\pgfqpoint{3.183119in}{2.241814in}}%
\pgfpathlineto{\pgfqpoint{3.081965in}{1.545840in}}%
\pgfusepath{stroke}%
\end{pgfscope}%
\begin{pgfscope}%
\pgfpathrectangle{\pgfqpoint{0.100000in}{0.212622in}}{\pgfqpoint{3.696000in}{3.696000in}}%
\pgfusepath{clip}%
\pgfsetrectcap%
\pgfsetroundjoin%
\pgfsetlinewidth{1.505625pt}%
\definecolor{currentstroke}{rgb}{1.000000,0.000000,0.000000}%
\pgfsetstrokecolor{currentstroke}%
\pgfsetdash{}{0pt}%
\pgfpathmoveto{\pgfqpoint{3.181693in}{2.241373in}}%
\pgfpathlineto{\pgfqpoint{3.081965in}{1.545840in}}%
\pgfusepath{stroke}%
\end{pgfscope}%
\begin{pgfscope}%
\pgfpathrectangle{\pgfqpoint{0.100000in}{0.212622in}}{\pgfqpoint{3.696000in}{3.696000in}}%
\pgfusepath{clip}%
\pgfsetrectcap%
\pgfsetroundjoin%
\pgfsetlinewidth{1.505625pt}%
\definecolor{currentstroke}{rgb}{1.000000,0.000000,0.000000}%
\pgfsetstrokecolor{currentstroke}%
\pgfsetdash{}{0pt}%
\pgfpathmoveto{\pgfqpoint{3.180901in}{2.240958in}}%
\pgfpathlineto{\pgfqpoint{3.081965in}{1.545840in}}%
\pgfusepath{stroke}%
\end{pgfscope}%
\begin{pgfscope}%
\pgfpathrectangle{\pgfqpoint{0.100000in}{0.212622in}}{\pgfqpoint{3.696000in}{3.696000in}}%
\pgfusepath{clip}%
\pgfsetrectcap%
\pgfsetroundjoin%
\pgfsetlinewidth{1.505625pt}%
\definecolor{currentstroke}{rgb}{1.000000,0.000000,0.000000}%
\pgfsetstrokecolor{currentstroke}%
\pgfsetdash{}{0pt}%
\pgfpathmoveto{\pgfqpoint{3.180487in}{2.240778in}}%
\pgfpathlineto{\pgfqpoint{3.081965in}{1.545840in}}%
\pgfusepath{stroke}%
\end{pgfscope}%
\begin{pgfscope}%
\pgfpathrectangle{\pgfqpoint{0.100000in}{0.212622in}}{\pgfqpoint{3.696000in}{3.696000in}}%
\pgfusepath{clip}%
\pgfsetrectcap%
\pgfsetroundjoin%
\pgfsetlinewidth{1.505625pt}%
\definecolor{currentstroke}{rgb}{1.000000,0.000000,0.000000}%
\pgfsetstrokecolor{currentstroke}%
\pgfsetdash{}{0pt}%
\pgfpathmoveto{\pgfqpoint{3.179785in}{2.240763in}}%
\pgfpathlineto{\pgfqpoint{3.081965in}{1.545840in}}%
\pgfusepath{stroke}%
\end{pgfscope}%
\begin{pgfscope}%
\pgfpathrectangle{\pgfqpoint{0.100000in}{0.212622in}}{\pgfqpoint{3.696000in}{3.696000in}}%
\pgfusepath{clip}%
\pgfsetrectcap%
\pgfsetroundjoin%
\pgfsetlinewidth{1.505625pt}%
\definecolor{currentstroke}{rgb}{1.000000,0.000000,0.000000}%
\pgfsetstrokecolor{currentstroke}%
\pgfsetdash{}{0pt}%
\pgfpathmoveto{\pgfqpoint{3.178671in}{2.240488in}}%
\pgfpathlineto{\pgfqpoint{3.081965in}{1.545840in}}%
\pgfusepath{stroke}%
\end{pgfscope}%
\begin{pgfscope}%
\pgfpathrectangle{\pgfqpoint{0.100000in}{0.212622in}}{\pgfqpoint{3.696000in}{3.696000in}}%
\pgfusepath{clip}%
\pgfsetrectcap%
\pgfsetroundjoin%
\pgfsetlinewidth{1.505625pt}%
\definecolor{currentstroke}{rgb}{1.000000,0.000000,0.000000}%
\pgfsetstrokecolor{currentstroke}%
\pgfsetdash{}{0pt}%
\pgfpathmoveto{\pgfqpoint{3.177346in}{2.240076in}}%
\pgfpathlineto{\pgfqpoint{3.074554in}{1.538511in}}%
\pgfusepath{stroke}%
\end{pgfscope}%
\begin{pgfscope}%
\pgfpathrectangle{\pgfqpoint{0.100000in}{0.212622in}}{\pgfqpoint{3.696000in}{3.696000in}}%
\pgfusepath{clip}%
\pgfsetrectcap%
\pgfsetroundjoin%
\pgfsetlinewidth{1.505625pt}%
\definecolor{currentstroke}{rgb}{1.000000,0.000000,0.000000}%
\pgfsetstrokecolor{currentstroke}%
\pgfsetdash{}{0pt}%
\pgfpathmoveto{\pgfqpoint{3.175362in}{2.240069in}}%
\pgfpathlineto{\pgfqpoint{3.074554in}{1.538511in}}%
\pgfusepath{stroke}%
\end{pgfscope}%
\begin{pgfscope}%
\pgfpathrectangle{\pgfqpoint{0.100000in}{0.212622in}}{\pgfqpoint{3.696000in}{3.696000in}}%
\pgfusepath{clip}%
\pgfsetrectcap%
\pgfsetroundjoin%
\pgfsetlinewidth{1.505625pt}%
\definecolor{currentstroke}{rgb}{1.000000,0.000000,0.000000}%
\pgfsetstrokecolor{currentstroke}%
\pgfsetdash{}{0pt}%
\pgfpathmoveto{\pgfqpoint{3.174180in}{2.239880in}}%
\pgfpathlineto{\pgfqpoint{3.074554in}{1.538511in}}%
\pgfusepath{stroke}%
\end{pgfscope}%
\begin{pgfscope}%
\pgfpathrectangle{\pgfqpoint{0.100000in}{0.212622in}}{\pgfqpoint{3.696000in}{3.696000in}}%
\pgfusepath{clip}%
\pgfsetrectcap%
\pgfsetroundjoin%
\pgfsetlinewidth{1.505625pt}%
\definecolor{currentstroke}{rgb}{1.000000,0.000000,0.000000}%
\pgfsetstrokecolor{currentstroke}%
\pgfsetdash{}{0pt}%
\pgfpathmoveto{\pgfqpoint{3.172755in}{2.239618in}}%
\pgfpathlineto{\pgfqpoint{3.074554in}{1.538511in}}%
\pgfusepath{stroke}%
\end{pgfscope}%
\begin{pgfscope}%
\pgfpathrectangle{\pgfqpoint{0.100000in}{0.212622in}}{\pgfqpoint{3.696000in}{3.696000in}}%
\pgfusepath{clip}%
\pgfsetrectcap%
\pgfsetroundjoin%
\pgfsetlinewidth{1.505625pt}%
\definecolor{currentstroke}{rgb}{1.000000,0.000000,0.000000}%
\pgfsetstrokecolor{currentstroke}%
\pgfsetdash{}{0pt}%
\pgfpathmoveto{\pgfqpoint{3.170213in}{2.239974in}}%
\pgfpathlineto{\pgfqpoint{3.067133in}{1.531172in}}%
\pgfusepath{stroke}%
\end{pgfscope}%
\begin{pgfscope}%
\pgfpathrectangle{\pgfqpoint{0.100000in}{0.212622in}}{\pgfqpoint{3.696000in}{3.696000in}}%
\pgfusepath{clip}%
\pgfsetrectcap%
\pgfsetroundjoin%
\pgfsetlinewidth{1.505625pt}%
\definecolor{currentstroke}{rgb}{1.000000,0.000000,0.000000}%
\pgfsetstrokecolor{currentstroke}%
\pgfsetdash{}{0pt}%
\pgfpathmoveto{\pgfqpoint{3.168699in}{2.239943in}}%
\pgfpathlineto{\pgfqpoint{3.067133in}{1.531172in}}%
\pgfusepath{stroke}%
\end{pgfscope}%
\begin{pgfscope}%
\pgfpathrectangle{\pgfqpoint{0.100000in}{0.212622in}}{\pgfqpoint{3.696000in}{3.696000in}}%
\pgfusepath{clip}%
\pgfsetrectcap%
\pgfsetroundjoin%
\pgfsetlinewidth{1.505625pt}%
\definecolor{currentstroke}{rgb}{1.000000,0.000000,0.000000}%
\pgfsetstrokecolor{currentstroke}%
\pgfsetdash{}{0pt}%
\pgfpathmoveto{\pgfqpoint{3.167877in}{2.239906in}}%
\pgfpathlineto{\pgfqpoint{3.067133in}{1.531172in}}%
\pgfusepath{stroke}%
\end{pgfscope}%
\begin{pgfscope}%
\pgfpathrectangle{\pgfqpoint{0.100000in}{0.212622in}}{\pgfqpoint{3.696000in}{3.696000in}}%
\pgfusepath{clip}%
\pgfsetrectcap%
\pgfsetroundjoin%
\pgfsetlinewidth{1.505625pt}%
\definecolor{currentstroke}{rgb}{1.000000,0.000000,0.000000}%
\pgfsetstrokecolor{currentstroke}%
\pgfsetdash{}{0pt}%
\pgfpathmoveto{\pgfqpoint{3.165512in}{2.239681in}}%
\pgfpathlineto{\pgfqpoint{3.067133in}{1.531172in}}%
\pgfusepath{stroke}%
\end{pgfscope}%
\begin{pgfscope}%
\pgfpathrectangle{\pgfqpoint{0.100000in}{0.212622in}}{\pgfqpoint{3.696000in}{3.696000in}}%
\pgfusepath{clip}%
\pgfsetrectcap%
\pgfsetroundjoin%
\pgfsetlinewidth{1.505625pt}%
\definecolor{currentstroke}{rgb}{1.000000,0.000000,0.000000}%
\pgfsetstrokecolor{currentstroke}%
\pgfsetdash{}{0pt}%
\pgfpathmoveto{\pgfqpoint{3.162696in}{2.239124in}}%
\pgfpathlineto{\pgfqpoint{3.067133in}{1.531172in}}%
\pgfusepath{stroke}%
\end{pgfscope}%
\begin{pgfscope}%
\pgfpathrectangle{\pgfqpoint{0.100000in}{0.212622in}}{\pgfqpoint{3.696000in}{3.696000in}}%
\pgfusepath{clip}%
\pgfsetrectcap%
\pgfsetroundjoin%
\pgfsetlinewidth{1.505625pt}%
\definecolor{currentstroke}{rgb}{1.000000,0.000000,0.000000}%
\pgfsetstrokecolor{currentstroke}%
\pgfsetdash{}{0pt}%
\pgfpathmoveto{\pgfqpoint{3.161232in}{2.238949in}}%
\pgfpathlineto{\pgfqpoint{3.059703in}{1.523825in}}%
\pgfusepath{stroke}%
\end{pgfscope}%
\begin{pgfscope}%
\pgfpathrectangle{\pgfqpoint{0.100000in}{0.212622in}}{\pgfqpoint{3.696000in}{3.696000in}}%
\pgfusepath{clip}%
\pgfsetrectcap%
\pgfsetroundjoin%
\pgfsetlinewidth{1.505625pt}%
\definecolor{currentstroke}{rgb}{1.000000,0.000000,0.000000}%
\pgfsetstrokecolor{currentstroke}%
\pgfsetdash{}{0pt}%
\pgfpathmoveto{\pgfqpoint{3.159062in}{2.238367in}}%
\pgfpathlineto{\pgfqpoint{3.059703in}{1.523825in}}%
\pgfusepath{stroke}%
\end{pgfscope}%
\begin{pgfscope}%
\pgfpathrectangle{\pgfqpoint{0.100000in}{0.212622in}}{\pgfqpoint{3.696000in}{3.696000in}}%
\pgfusepath{clip}%
\pgfsetrectcap%
\pgfsetroundjoin%
\pgfsetlinewidth{1.505625pt}%
\definecolor{currentstroke}{rgb}{1.000000,0.000000,0.000000}%
\pgfsetstrokecolor{currentstroke}%
\pgfsetdash{}{0pt}%
\pgfpathmoveto{\pgfqpoint{3.155663in}{2.237141in}}%
\pgfpathlineto{\pgfqpoint{3.059703in}{1.523825in}}%
\pgfusepath{stroke}%
\end{pgfscope}%
\begin{pgfscope}%
\pgfpathrectangle{\pgfqpoint{0.100000in}{0.212622in}}{\pgfqpoint{3.696000in}{3.696000in}}%
\pgfusepath{clip}%
\pgfsetrectcap%
\pgfsetroundjoin%
\pgfsetlinewidth{1.505625pt}%
\definecolor{currentstroke}{rgb}{1.000000,0.000000,0.000000}%
\pgfsetstrokecolor{currentstroke}%
\pgfsetdash{}{0pt}%
\pgfpathmoveto{\pgfqpoint{3.151357in}{2.236730in}}%
\pgfpathlineto{\pgfqpoint{3.052265in}{1.516469in}}%
\pgfusepath{stroke}%
\end{pgfscope}%
\begin{pgfscope}%
\pgfpathrectangle{\pgfqpoint{0.100000in}{0.212622in}}{\pgfqpoint{3.696000in}{3.696000in}}%
\pgfusepath{clip}%
\pgfsetrectcap%
\pgfsetroundjoin%
\pgfsetlinewidth{1.505625pt}%
\definecolor{currentstroke}{rgb}{1.000000,0.000000,0.000000}%
\pgfsetstrokecolor{currentstroke}%
\pgfsetdash{}{0pt}%
\pgfpathmoveto{\pgfqpoint{3.146475in}{2.235739in}}%
\pgfpathlineto{\pgfqpoint{3.052265in}{1.516469in}}%
\pgfusepath{stroke}%
\end{pgfscope}%
\begin{pgfscope}%
\pgfpathrectangle{\pgfqpoint{0.100000in}{0.212622in}}{\pgfqpoint{3.696000in}{3.696000in}}%
\pgfusepath{clip}%
\pgfsetrectcap%
\pgfsetroundjoin%
\pgfsetlinewidth{1.505625pt}%
\definecolor{currentstroke}{rgb}{1.000000,0.000000,0.000000}%
\pgfsetstrokecolor{currentstroke}%
\pgfsetdash{}{0pt}%
\pgfpathmoveto{\pgfqpoint{3.143910in}{2.235234in}}%
\pgfpathlineto{\pgfqpoint{3.044817in}{1.509104in}}%
\pgfusepath{stroke}%
\end{pgfscope}%
\begin{pgfscope}%
\pgfpathrectangle{\pgfqpoint{0.100000in}{0.212622in}}{\pgfqpoint{3.696000in}{3.696000in}}%
\pgfusepath{clip}%
\pgfsetrectcap%
\pgfsetroundjoin%
\pgfsetlinewidth{1.505625pt}%
\definecolor{currentstroke}{rgb}{1.000000,0.000000,0.000000}%
\pgfsetstrokecolor{currentstroke}%
\pgfsetdash{}{0pt}%
\pgfpathmoveto{\pgfqpoint{3.139771in}{2.234530in}}%
\pgfpathlineto{\pgfqpoint{3.044817in}{1.509104in}}%
\pgfusepath{stroke}%
\end{pgfscope}%
\begin{pgfscope}%
\pgfpathrectangle{\pgfqpoint{0.100000in}{0.212622in}}{\pgfqpoint{3.696000in}{3.696000in}}%
\pgfusepath{clip}%
\pgfsetrectcap%
\pgfsetroundjoin%
\pgfsetlinewidth{1.505625pt}%
\definecolor{currentstroke}{rgb}{1.000000,0.000000,0.000000}%
\pgfsetstrokecolor{currentstroke}%
\pgfsetdash{}{0pt}%
\pgfpathmoveto{\pgfqpoint{3.135099in}{2.232529in}}%
\pgfpathlineto{\pgfqpoint{3.037360in}{1.501730in}}%
\pgfusepath{stroke}%
\end{pgfscope}%
\begin{pgfscope}%
\pgfpathrectangle{\pgfqpoint{0.100000in}{0.212622in}}{\pgfqpoint{3.696000in}{3.696000in}}%
\pgfusepath{clip}%
\pgfsetrectcap%
\pgfsetroundjoin%
\pgfsetlinewidth{1.505625pt}%
\definecolor{currentstroke}{rgb}{1.000000,0.000000,0.000000}%
\pgfsetstrokecolor{currentstroke}%
\pgfsetdash{}{0pt}%
\pgfpathmoveto{\pgfqpoint{3.132561in}{2.231897in}}%
\pgfpathlineto{\pgfqpoint{3.037360in}{1.501730in}}%
\pgfusepath{stroke}%
\end{pgfscope}%
\begin{pgfscope}%
\pgfpathrectangle{\pgfqpoint{0.100000in}{0.212622in}}{\pgfqpoint{3.696000in}{3.696000in}}%
\pgfusepath{clip}%
\pgfsetrectcap%
\pgfsetroundjoin%
\pgfsetlinewidth{1.505625pt}%
\definecolor{currentstroke}{rgb}{1.000000,0.000000,0.000000}%
\pgfsetstrokecolor{currentstroke}%
\pgfsetdash{}{0pt}%
\pgfpathmoveto{\pgfqpoint{3.131066in}{2.231363in}}%
\pgfpathlineto{\pgfqpoint{3.037360in}{1.501730in}}%
\pgfusepath{stroke}%
\end{pgfscope}%
\begin{pgfscope}%
\pgfpathrectangle{\pgfqpoint{0.100000in}{0.212622in}}{\pgfqpoint{3.696000in}{3.696000in}}%
\pgfusepath{clip}%
\pgfsetrectcap%
\pgfsetroundjoin%
\pgfsetlinewidth{1.505625pt}%
\definecolor{currentstroke}{rgb}{1.000000,0.000000,0.000000}%
\pgfsetstrokecolor{currentstroke}%
\pgfsetdash{}{0pt}%
\pgfpathmoveto{\pgfqpoint{3.128746in}{2.230336in}}%
\pgfpathlineto{\pgfqpoint{3.037360in}{1.501730in}}%
\pgfusepath{stroke}%
\end{pgfscope}%
\begin{pgfscope}%
\pgfpathrectangle{\pgfqpoint{0.100000in}{0.212622in}}{\pgfqpoint{3.696000in}{3.696000in}}%
\pgfusepath{clip}%
\pgfsetrectcap%
\pgfsetroundjoin%
\pgfsetlinewidth{1.505625pt}%
\definecolor{currentstroke}{rgb}{1.000000,0.000000,0.000000}%
\pgfsetstrokecolor{currentstroke}%
\pgfsetdash{}{0pt}%
\pgfpathmoveto{\pgfqpoint{3.127408in}{2.230201in}}%
\pgfpathlineto{\pgfqpoint{3.029894in}{1.494346in}}%
\pgfusepath{stroke}%
\end{pgfscope}%
\begin{pgfscope}%
\pgfpathrectangle{\pgfqpoint{0.100000in}{0.212622in}}{\pgfqpoint{3.696000in}{3.696000in}}%
\pgfusepath{clip}%
\pgfsetrectcap%
\pgfsetroundjoin%
\pgfsetlinewidth{1.505625pt}%
\definecolor{currentstroke}{rgb}{1.000000,0.000000,0.000000}%
\pgfsetstrokecolor{currentstroke}%
\pgfsetdash{}{0pt}%
\pgfpathmoveto{\pgfqpoint{3.125626in}{2.229818in}}%
\pgfpathlineto{\pgfqpoint{3.029894in}{1.494346in}}%
\pgfusepath{stroke}%
\end{pgfscope}%
\begin{pgfscope}%
\pgfpathrectangle{\pgfqpoint{0.100000in}{0.212622in}}{\pgfqpoint{3.696000in}{3.696000in}}%
\pgfusepath{clip}%
\pgfsetrectcap%
\pgfsetroundjoin%
\pgfsetlinewidth{1.505625pt}%
\definecolor{currentstroke}{rgb}{1.000000,0.000000,0.000000}%
\pgfsetstrokecolor{currentstroke}%
\pgfsetdash{}{0pt}%
\pgfpathmoveto{\pgfqpoint{3.124698in}{2.229664in}}%
\pgfpathlineto{\pgfqpoint{3.029894in}{1.494346in}}%
\pgfusepath{stroke}%
\end{pgfscope}%
\begin{pgfscope}%
\pgfpathrectangle{\pgfqpoint{0.100000in}{0.212622in}}{\pgfqpoint{3.696000in}{3.696000in}}%
\pgfusepath{clip}%
\pgfsetrectcap%
\pgfsetroundjoin%
\pgfsetlinewidth{1.505625pt}%
\definecolor{currentstroke}{rgb}{1.000000,0.000000,0.000000}%
\pgfsetstrokecolor{currentstroke}%
\pgfsetdash{}{0pt}%
\pgfpathmoveto{\pgfqpoint{3.123098in}{2.229162in}}%
\pgfpathlineto{\pgfqpoint{3.029894in}{1.494346in}}%
\pgfusepath{stroke}%
\end{pgfscope}%
\begin{pgfscope}%
\pgfpathrectangle{\pgfqpoint{0.100000in}{0.212622in}}{\pgfqpoint{3.696000in}{3.696000in}}%
\pgfusepath{clip}%
\pgfsetrectcap%
\pgfsetroundjoin%
\pgfsetlinewidth{1.505625pt}%
\definecolor{currentstroke}{rgb}{1.000000,0.000000,0.000000}%
\pgfsetstrokecolor{currentstroke}%
\pgfsetdash{}{0pt}%
\pgfpathmoveto{\pgfqpoint{3.120653in}{2.227986in}}%
\pgfpathlineto{\pgfqpoint{3.029894in}{1.494346in}}%
\pgfusepath{stroke}%
\end{pgfscope}%
\begin{pgfscope}%
\pgfpathrectangle{\pgfqpoint{0.100000in}{0.212622in}}{\pgfqpoint{3.696000in}{3.696000in}}%
\pgfusepath{clip}%
\pgfsetrectcap%
\pgfsetroundjoin%
\pgfsetlinewidth{1.505625pt}%
\definecolor{currentstroke}{rgb}{1.000000,0.000000,0.000000}%
\pgfsetstrokecolor{currentstroke}%
\pgfsetdash{}{0pt}%
\pgfpathmoveto{\pgfqpoint{3.119274in}{2.227755in}}%
\pgfpathlineto{\pgfqpoint{3.022419in}{1.486954in}}%
\pgfusepath{stroke}%
\end{pgfscope}%
\begin{pgfscope}%
\pgfpathrectangle{\pgfqpoint{0.100000in}{0.212622in}}{\pgfqpoint{3.696000in}{3.696000in}}%
\pgfusepath{clip}%
\pgfsetrectcap%
\pgfsetroundjoin%
\pgfsetlinewidth{1.505625pt}%
\definecolor{currentstroke}{rgb}{1.000000,0.000000,0.000000}%
\pgfsetstrokecolor{currentstroke}%
\pgfsetdash{}{0pt}%
\pgfpathmoveto{\pgfqpoint{3.118483in}{2.227540in}}%
\pgfpathlineto{\pgfqpoint{3.022419in}{1.486954in}}%
\pgfusepath{stroke}%
\end{pgfscope}%
\begin{pgfscope}%
\pgfpathrectangle{\pgfqpoint{0.100000in}{0.212622in}}{\pgfqpoint{3.696000in}{3.696000in}}%
\pgfusepath{clip}%
\pgfsetrectcap%
\pgfsetroundjoin%
\pgfsetlinewidth{1.505625pt}%
\definecolor{currentstroke}{rgb}{1.000000,0.000000,0.000000}%
\pgfsetstrokecolor{currentstroke}%
\pgfsetdash{}{0pt}%
\pgfpathmoveto{\pgfqpoint{3.118074in}{2.227499in}}%
\pgfpathlineto{\pgfqpoint{3.022419in}{1.486954in}}%
\pgfusepath{stroke}%
\end{pgfscope}%
\begin{pgfscope}%
\pgfpathrectangle{\pgfqpoint{0.100000in}{0.212622in}}{\pgfqpoint{3.696000in}{3.696000in}}%
\pgfusepath{clip}%
\pgfsetrectcap%
\pgfsetroundjoin%
\pgfsetlinewidth{1.505625pt}%
\definecolor{currentstroke}{rgb}{1.000000,0.000000,0.000000}%
\pgfsetstrokecolor{currentstroke}%
\pgfsetdash{}{0pt}%
\pgfpathmoveto{\pgfqpoint{3.117057in}{2.227140in}}%
\pgfpathlineto{\pgfqpoint{3.022419in}{1.486954in}}%
\pgfusepath{stroke}%
\end{pgfscope}%
\begin{pgfscope}%
\pgfpathrectangle{\pgfqpoint{0.100000in}{0.212622in}}{\pgfqpoint{3.696000in}{3.696000in}}%
\pgfusepath{clip}%
\pgfsetrectcap%
\pgfsetroundjoin%
\pgfsetlinewidth{1.505625pt}%
\definecolor{currentstroke}{rgb}{1.000000,0.000000,0.000000}%
\pgfsetstrokecolor{currentstroke}%
\pgfsetdash{}{0pt}%
\pgfpathmoveto{\pgfqpoint{3.115546in}{2.226764in}}%
\pgfpathlineto{\pgfqpoint{3.022419in}{1.486954in}}%
\pgfusepath{stroke}%
\end{pgfscope}%
\begin{pgfscope}%
\pgfpathrectangle{\pgfqpoint{0.100000in}{0.212622in}}{\pgfqpoint{3.696000in}{3.696000in}}%
\pgfusepath{clip}%
\pgfsetrectcap%
\pgfsetroundjoin%
\pgfsetlinewidth{1.505625pt}%
\definecolor{currentstroke}{rgb}{1.000000,0.000000,0.000000}%
\pgfsetstrokecolor{currentstroke}%
\pgfsetdash{}{0pt}%
\pgfpathmoveto{\pgfqpoint{3.113439in}{2.226725in}}%
\pgfpathlineto{\pgfqpoint{3.022419in}{1.486954in}}%
\pgfusepath{stroke}%
\end{pgfscope}%
\begin{pgfscope}%
\pgfpathrectangle{\pgfqpoint{0.100000in}{0.212622in}}{\pgfqpoint{3.696000in}{3.696000in}}%
\pgfusepath{clip}%
\pgfsetrectcap%
\pgfsetroundjoin%
\pgfsetlinewidth{1.505625pt}%
\definecolor{currentstroke}{rgb}{1.000000,0.000000,0.000000}%
\pgfsetstrokecolor{currentstroke}%
\pgfsetdash{}{0pt}%
\pgfpathmoveto{\pgfqpoint{3.111016in}{2.226101in}}%
\pgfpathlineto{\pgfqpoint{3.014935in}{1.479553in}}%
\pgfusepath{stroke}%
\end{pgfscope}%
\begin{pgfscope}%
\pgfpathrectangle{\pgfqpoint{0.100000in}{0.212622in}}{\pgfqpoint{3.696000in}{3.696000in}}%
\pgfusepath{clip}%
\pgfsetrectcap%
\pgfsetroundjoin%
\pgfsetlinewidth{1.505625pt}%
\definecolor{currentstroke}{rgb}{1.000000,0.000000,0.000000}%
\pgfsetstrokecolor{currentstroke}%
\pgfsetdash{}{0pt}%
\pgfpathmoveto{\pgfqpoint{3.109719in}{2.225831in}}%
\pgfpathlineto{\pgfqpoint{3.014935in}{1.479553in}}%
\pgfusepath{stroke}%
\end{pgfscope}%
\begin{pgfscope}%
\pgfpathrectangle{\pgfqpoint{0.100000in}{0.212622in}}{\pgfqpoint{3.696000in}{3.696000in}}%
\pgfusepath{clip}%
\pgfsetrectcap%
\pgfsetroundjoin%
\pgfsetlinewidth{1.505625pt}%
\definecolor{currentstroke}{rgb}{1.000000,0.000000,0.000000}%
\pgfsetstrokecolor{currentstroke}%
\pgfsetdash{}{0pt}%
\pgfpathmoveto{\pgfqpoint{3.108034in}{2.225262in}}%
\pgfpathlineto{\pgfqpoint{3.014935in}{1.479553in}}%
\pgfusepath{stroke}%
\end{pgfscope}%
\begin{pgfscope}%
\pgfpathrectangle{\pgfqpoint{0.100000in}{0.212622in}}{\pgfqpoint{3.696000in}{3.696000in}}%
\pgfusepath{clip}%
\pgfsetrectcap%
\pgfsetroundjoin%
\pgfsetlinewidth{1.505625pt}%
\definecolor{currentstroke}{rgb}{1.000000,0.000000,0.000000}%
\pgfsetstrokecolor{currentstroke}%
\pgfsetdash{}{0pt}%
\pgfpathmoveto{\pgfqpoint{3.105749in}{2.224293in}}%
\pgfpathlineto{\pgfqpoint{3.014935in}{1.479553in}}%
\pgfusepath{stroke}%
\end{pgfscope}%
\begin{pgfscope}%
\pgfpathrectangle{\pgfqpoint{0.100000in}{0.212622in}}{\pgfqpoint{3.696000in}{3.696000in}}%
\pgfusepath{clip}%
\pgfsetrectcap%
\pgfsetroundjoin%
\pgfsetlinewidth{1.505625pt}%
\definecolor{currentstroke}{rgb}{1.000000,0.000000,0.000000}%
\pgfsetstrokecolor{currentstroke}%
\pgfsetdash{}{0pt}%
\pgfpathmoveto{\pgfqpoint{3.104511in}{2.223952in}}%
\pgfpathlineto{\pgfqpoint{3.014935in}{1.479553in}}%
\pgfusepath{stroke}%
\end{pgfscope}%
\begin{pgfscope}%
\pgfpathrectangle{\pgfqpoint{0.100000in}{0.212622in}}{\pgfqpoint{3.696000in}{3.696000in}}%
\pgfusepath{clip}%
\pgfsetrectcap%
\pgfsetroundjoin%
\pgfsetlinewidth{1.505625pt}%
\definecolor{currentstroke}{rgb}{1.000000,0.000000,0.000000}%
\pgfsetstrokecolor{currentstroke}%
\pgfsetdash{}{0pt}%
\pgfpathmoveto{\pgfqpoint{3.103802in}{2.223689in}}%
\pgfpathlineto{\pgfqpoint{3.007442in}{1.472143in}}%
\pgfusepath{stroke}%
\end{pgfscope}%
\begin{pgfscope}%
\pgfpathrectangle{\pgfqpoint{0.100000in}{0.212622in}}{\pgfqpoint{3.696000in}{3.696000in}}%
\pgfusepath{clip}%
\pgfsetrectcap%
\pgfsetroundjoin%
\pgfsetlinewidth{1.505625pt}%
\definecolor{currentstroke}{rgb}{1.000000,0.000000,0.000000}%
\pgfsetstrokecolor{currentstroke}%
\pgfsetdash{}{0pt}%
\pgfpathmoveto{\pgfqpoint{3.103445in}{2.223565in}}%
\pgfpathlineto{\pgfqpoint{3.007442in}{1.472143in}}%
\pgfusepath{stroke}%
\end{pgfscope}%
\begin{pgfscope}%
\pgfpathrectangle{\pgfqpoint{0.100000in}{0.212622in}}{\pgfqpoint{3.696000in}{3.696000in}}%
\pgfusepath{clip}%
\pgfsetrectcap%
\pgfsetroundjoin%
\pgfsetlinewidth{1.505625pt}%
\definecolor{currentstroke}{rgb}{1.000000,0.000000,0.000000}%
\pgfsetstrokecolor{currentstroke}%
\pgfsetdash{}{0pt}%
\pgfpathmoveto{\pgfqpoint{3.102049in}{2.223149in}}%
\pgfpathlineto{\pgfqpoint{3.007442in}{1.472143in}}%
\pgfusepath{stroke}%
\end{pgfscope}%
\begin{pgfscope}%
\pgfpathrectangle{\pgfqpoint{0.100000in}{0.212622in}}{\pgfqpoint{3.696000in}{3.696000in}}%
\pgfusepath{clip}%
\pgfsetrectcap%
\pgfsetroundjoin%
\pgfsetlinewidth{1.505625pt}%
\definecolor{currentstroke}{rgb}{1.000000,0.000000,0.000000}%
\pgfsetstrokecolor{currentstroke}%
\pgfsetdash{}{0pt}%
\pgfpathmoveto{\pgfqpoint{3.100212in}{2.222686in}}%
\pgfpathlineto{\pgfqpoint{3.007442in}{1.472143in}}%
\pgfusepath{stroke}%
\end{pgfscope}%
\begin{pgfscope}%
\pgfpathrectangle{\pgfqpoint{0.100000in}{0.212622in}}{\pgfqpoint{3.696000in}{3.696000in}}%
\pgfusepath{clip}%
\pgfsetrectcap%
\pgfsetroundjoin%
\pgfsetlinewidth{1.505625pt}%
\definecolor{currentstroke}{rgb}{1.000000,0.000000,0.000000}%
\pgfsetstrokecolor{currentstroke}%
\pgfsetdash{}{0pt}%
\pgfpathmoveto{\pgfqpoint{3.099257in}{2.222612in}}%
\pgfpathlineto{\pgfqpoint{3.007442in}{1.472143in}}%
\pgfusepath{stroke}%
\end{pgfscope}%
\begin{pgfscope}%
\pgfpathrectangle{\pgfqpoint{0.100000in}{0.212622in}}{\pgfqpoint{3.696000in}{3.696000in}}%
\pgfusepath{clip}%
\pgfsetrectcap%
\pgfsetroundjoin%
\pgfsetlinewidth{1.505625pt}%
\definecolor{currentstroke}{rgb}{1.000000,0.000000,0.000000}%
\pgfsetstrokecolor{currentstroke}%
\pgfsetdash{}{0pt}%
\pgfpathmoveto{\pgfqpoint{3.098698in}{2.222477in}}%
\pgfpathlineto{\pgfqpoint{3.007442in}{1.472143in}}%
\pgfusepath{stroke}%
\end{pgfscope}%
\begin{pgfscope}%
\pgfpathrectangle{\pgfqpoint{0.100000in}{0.212622in}}{\pgfqpoint{3.696000in}{3.696000in}}%
\pgfusepath{clip}%
\pgfsetrectcap%
\pgfsetroundjoin%
\pgfsetlinewidth{1.505625pt}%
\definecolor{currentstroke}{rgb}{1.000000,0.000000,0.000000}%
\pgfsetstrokecolor{currentstroke}%
\pgfsetdash{}{0pt}%
\pgfpathmoveto{\pgfqpoint{3.097316in}{2.222031in}}%
\pgfpathlineto{\pgfqpoint{3.007442in}{1.472143in}}%
\pgfusepath{stroke}%
\end{pgfscope}%
\begin{pgfscope}%
\pgfpathrectangle{\pgfqpoint{0.100000in}{0.212622in}}{\pgfqpoint{3.696000in}{3.696000in}}%
\pgfusepath{clip}%
\pgfsetrectcap%
\pgfsetroundjoin%
\pgfsetlinewidth{1.505625pt}%
\definecolor{currentstroke}{rgb}{1.000000,0.000000,0.000000}%
\pgfsetstrokecolor{currentstroke}%
\pgfsetdash{}{0pt}%
\pgfpathmoveto{\pgfqpoint{3.096518in}{2.221947in}}%
\pgfpathlineto{\pgfqpoint{3.007442in}{1.472143in}}%
\pgfusepath{stroke}%
\end{pgfscope}%
\begin{pgfscope}%
\pgfpathrectangle{\pgfqpoint{0.100000in}{0.212622in}}{\pgfqpoint{3.696000in}{3.696000in}}%
\pgfusepath{clip}%
\pgfsetrectcap%
\pgfsetroundjoin%
\pgfsetlinewidth{1.505625pt}%
\definecolor{currentstroke}{rgb}{1.000000,0.000000,0.000000}%
\pgfsetstrokecolor{currentstroke}%
\pgfsetdash{}{0pt}%
\pgfpathmoveto{\pgfqpoint{3.096063in}{2.221849in}}%
\pgfpathlineto{\pgfqpoint{2.999940in}{1.464724in}}%
\pgfusepath{stroke}%
\end{pgfscope}%
\begin{pgfscope}%
\pgfpathrectangle{\pgfqpoint{0.100000in}{0.212622in}}{\pgfqpoint{3.696000in}{3.696000in}}%
\pgfusepath{clip}%
\pgfsetrectcap%
\pgfsetroundjoin%
\pgfsetlinewidth{1.505625pt}%
\definecolor{currentstroke}{rgb}{1.000000,0.000000,0.000000}%
\pgfsetstrokecolor{currentstroke}%
\pgfsetdash{}{0pt}%
\pgfpathmoveto{\pgfqpoint{3.095396in}{2.221685in}}%
\pgfpathlineto{\pgfqpoint{2.999940in}{1.464724in}}%
\pgfusepath{stroke}%
\end{pgfscope}%
\begin{pgfscope}%
\pgfpathrectangle{\pgfqpoint{0.100000in}{0.212622in}}{\pgfqpoint{3.696000in}{3.696000in}}%
\pgfusepath{clip}%
\pgfsetrectcap%
\pgfsetroundjoin%
\pgfsetlinewidth{1.505625pt}%
\definecolor{currentstroke}{rgb}{1.000000,0.000000,0.000000}%
\pgfsetstrokecolor{currentstroke}%
\pgfsetdash{}{0pt}%
\pgfpathmoveto{\pgfqpoint{3.094010in}{2.221291in}}%
\pgfpathlineto{\pgfqpoint{2.999940in}{1.464724in}}%
\pgfusepath{stroke}%
\end{pgfscope}%
\begin{pgfscope}%
\pgfpathrectangle{\pgfqpoint{0.100000in}{0.212622in}}{\pgfqpoint{3.696000in}{3.696000in}}%
\pgfusepath{clip}%
\pgfsetrectcap%
\pgfsetroundjoin%
\pgfsetlinewidth{1.505625pt}%
\definecolor{currentstroke}{rgb}{1.000000,0.000000,0.000000}%
\pgfsetstrokecolor{currentstroke}%
\pgfsetdash{}{0pt}%
\pgfpathmoveto{\pgfqpoint{3.092416in}{2.220795in}}%
\pgfpathlineto{\pgfqpoint{2.999940in}{1.464724in}}%
\pgfusepath{stroke}%
\end{pgfscope}%
\begin{pgfscope}%
\pgfpathrectangle{\pgfqpoint{0.100000in}{0.212622in}}{\pgfqpoint{3.696000in}{3.696000in}}%
\pgfusepath{clip}%
\pgfsetrectcap%
\pgfsetroundjoin%
\pgfsetlinewidth{1.505625pt}%
\definecolor{currentstroke}{rgb}{1.000000,0.000000,0.000000}%
\pgfsetstrokecolor{currentstroke}%
\pgfsetdash{}{0pt}%
\pgfpathmoveto{\pgfqpoint{3.091521in}{2.220618in}}%
\pgfpathlineto{\pgfqpoint{2.999940in}{1.464724in}}%
\pgfusepath{stroke}%
\end{pgfscope}%
\begin{pgfscope}%
\pgfpathrectangle{\pgfqpoint{0.100000in}{0.212622in}}{\pgfqpoint{3.696000in}{3.696000in}}%
\pgfusepath{clip}%
\pgfsetrectcap%
\pgfsetroundjoin%
\pgfsetlinewidth{1.505625pt}%
\definecolor{currentstroke}{rgb}{1.000000,0.000000,0.000000}%
\pgfsetstrokecolor{currentstroke}%
\pgfsetdash{}{0pt}%
\pgfpathmoveto{\pgfqpoint{3.090988in}{2.220483in}}%
\pgfpathlineto{\pgfqpoint{2.999940in}{1.464724in}}%
\pgfusepath{stroke}%
\end{pgfscope}%
\begin{pgfscope}%
\pgfpathrectangle{\pgfqpoint{0.100000in}{0.212622in}}{\pgfqpoint{3.696000in}{3.696000in}}%
\pgfusepath{clip}%
\pgfsetrectcap%
\pgfsetroundjoin%
\pgfsetlinewidth{1.505625pt}%
\definecolor{currentstroke}{rgb}{1.000000,0.000000,0.000000}%
\pgfsetstrokecolor{currentstroke}%
\pgfsetdash{}{0pt}%
\pgfpathmoveto{\pgfqpoint{3.090690in}{2.220397in}}%
\pgfpathlineto{\pgfqpoint{2.999940in}{1.464724in}}%
\pgfusepath{stroke}%
\end{pgfscope}%
\begin{pgfscope}%
\pgfpathrectangle{\pgfqpoint{0.100000in}{0.212622in}}{\pgfqpoint{3.696000in}{3.696000in}}%
\pgfusepath{clip}%
\pgfsetrectcap%
\pgfsetroundjoin%
\pgfsetlinewidth{1.505625pt}%
\definecolor{currentstroke}{rgb}{1.000000,0.000000,0.000000}%
\pgfsetstrokecolor{currentstroke}%
\pgfsetdash{}{0pt}%
\pgfpathmoveto{\pgfqpoint{3.089988in}{2.220064in}}%
\pgfpathlineto{\pgfqpoint{2.999940in}{1.464724in}}%
\pgfusepath{stroke}%
\end{pgfscope}%
\begin{pgfscope}%
\pgfpathrectangle{\pgfqpoint{0.100000in}{0.212622in}}{\pgfqpoint{3.696000in}{3.696000in}}%
\pgfusepath{clip}%
\pgfsetrectcap%
\pgfsetroundjoin%
\pgfsetlinewidth{1.505625pt}%
\definecolor{currentstroke}{rgb}{1.000000,0.000000,0.000000}%
\pgfsetstrokecolor{currentstroke}%
\pgfsetdash{}{0pt}%
\pgfpathmoveto{\pgfqpoint{3.089584in}{2.219947in}}%
\pgfpathlineto{\pgfqpoint{2.999940in}{1.464724in}}%
\pgfusepath{stroke}%
\end{pgfscope}%
\begin{pgfscope}%
\pgfpathrectangle{\pgfqpoint{0.100000in}{0.212622in}}{\pgfqpoint{3.696000in}{3.696000in}}%
\pgfusepath{clip}%
\pgfsetrectcap%
\pgfsetroundjoin%
\pgfsetlinewidth{1.505625pt}%
\definecolor{currentstroke}{rgb}{1.000000,0.000000,0.000000}%
\pgfsetstrokecolor{currentstroke}%
\pgfsetdash{}{0pt}%
\pgfpathmoveto{\pgfqpoint{3.089371in}{2.219912in}}%
\pgfpathlineto{\pgfqpoint{2.999940in}{1.464724in}}%
\pgfusepath{stroke}%
\end{pgfscope}%
\begin{pgfscope}%
\pgfpathrectangle{\pgfqpoint{0.100000in}{0.212622in}}{\pgfqpoint{3.696000in}{3.696000in}}%
\pgfusepath{clip}%
\pgfsetrectcap%
\pgfsetroundjoin%
\pgfsetlinewidth{1.505625pt}%
\definecolor{currentstroke}{rgb}{1.000000,0.000000,0.000000}%
\pgfsetstrokecolor{currentstroke}%
\pgfsetdash{}{0pt}%
\pgfpathmoveto{\pgfqpoint{3.089253in}{2.219891in}}%
\pgfpathlineto{\pgfqpoint{2.992428in}{1.457296in}}%
\pgfusepath{stroke}%
\end{pgfscope}%
\begin{pgfscope}%
\pgfpathrectangle{\pgfqpoint{0.100000in}{0.212622in}}{\pgfqpoint{3.696000in}{3.696000in}}%
\pgfusepath{clip}%
\pgfsetrectcap%
\pgfsetroundjoin%
\pgfsetlinewidth{1.505625pt}%
\definecolor{currentstroke}{rgb}{1.000000,0.000000,0.000000}%
\pgfsetstrokecolor{currentstroke}%
\pgfsetdash{}{0pt}%
\pgfpathmoveto{\pgfqpoint{3.089188in}{2.219875in}}%
\pgfpathlineto{\pgfqpoint{2.992428in}{1.457296in}}%
\pgfusepath{stroke}%
\end{pgfscope}%
\begin{pgfscope}%
\pgfpathrectangle{\pgfqpoint{0.100000in}{0.212622in}}{\pgfqpoint{3.696000in}{3.696000in}}%
\pgfusepath{clip}%
\pgfsetrectcap%
\pgfsetroundjoin%
\pgfsetlinewidth{1.505625pt}%
\definecolor{currentstroke}{rgb}{1.000000,0.000000,0.000000}%
\pgfsetstrokecolor{currentstroke}%
\pgfsetdash{}{0pt}%
\pgfpathmoveto{\pgfqpoint{3.089150in}{2.219870in}}%
\pgfpathlineto{\pgfqpoint{2.992428in}{1.457296in}}%
\pgfusepath{stroke}%
\end{pgfscope}%
\begin{pgfscope}%
\pgfpathrectangle{\pgfqpoint{0.100000in}{0.212622in}}{\pgfqpoint{3.696000in}{3.696000in}}%
\pgfusepath{clip}%
\pgfsetrectcap%
\pgfsetroundjoin%
\pgfsetlinewidth{1.505625pt}%
\definecolor{currentstroke}{rgb}{1.000000,0.000000,0.000000}%
\pgfsetstrokecolor{currentstroke}%
\pgfsetdash{}{0pt}%
\pgfpathmoveto{\pgfqpoint{3.089130in}{2.219863in}}%
\pgfpathlineto{\pgfqpoint{2.992428in}{1.457296in}}%
\pgfusepath{stroke}%
\end{pgfscope}%
\begin{pgfscope}%
\pgfpathrectangle{\pgfqpoint{0.100000in}{0.212622in}}{\pgfqpoint{3.696000in}{3.696000in}}%
\pgfusepath{clip}%
\pgfsetrectcap%
\pgfsetroundjoin%
\pgfsetlinewidth{1.505625pt}%
\definecolor{currentstroke}{rgb}{1.000000,0.000000,0.000000}%
\pgfsetstrokecolor{currentstroke}%
\pgfsetdash{}{0pt}%
\pgfpathmoveto{\pgfqpoint{3.089120in}{2.219862in}}%
\pgfpathlineto{\pgfqpoint{2.992428in}{1.457296in}}%
\pgfusepath{stroke}%
\end{pgfscope}%
\begin{pgfscope}%
\pgfpathrectangle{\pgfqpoint{0.100000in}{0.212622in}}{\pgfqpoint{3.696000in}{3.696000in}}%
\pgfusepath{clip}%
\pgfsetrectcap%
\pgfsetroundjoin%
\pgfsetlinewidth{1.505625pt}%
\definecolor{currentstroke}{rgb}{1.000000,0.000000,0.000000}%
\pgfsetstrokecolor{currentstroke}%
\pgfsetdash{}{0pt}%
\pgfpathmoveto{\pgfqpoint{3.089114in}{2.219860in}}%
\pgfpathlineto{\pgfqpoint{2.992428in}{1.457296in}}%
\pgfusepath{stroke}%
\end{pgfscope}%
\begin{pgfscope}%
\pgfpathrectangle{\pgfqpoint{0.100000in}{0.212622in}}{\pgfqpoint{3.696000in}{3.696000in}}%
\pgfusepath{clip}%
\pgfsetrectcap%
\pgfsetroundjoin%
\pgfsetlinewidth{1.505625pt}%
\definecolor{currentstroke}{rgb}{1.000000,0.000000,0.000000}%
\pgfsetstrokecolor{currentstroke}%
\pgfsetdash{}{0pt}%
\pgfpathmoveto{\pgfqpoint{3.088603in}{2.219679in}}%
\pgfpathlineto{\pgfqpoint{2.992428in}{1.457296in}}%
\pgfusepath{stroke}%
\end{pgfscope}%
\begin{pgfscope}%
\pgfpathrectangle{\pgfqpoint{0.100000in}{0.212622in}}{\pgfqpoint{3.696000in}{3.696000in}}%
\pgfusepath{clip}%
\pgfsetrectcap%
\pgfsetroundjoin%
\pgfsetlinewidth{1.505625pt}%
\definecolor{currentstroke}{rgb}{1.000000,0.000000,0.000000}%
\pgfsetstrokecolor{currentstroke}%
\pgfsetdash{}{0pt}%
\pgfpathmoveto{\pgfqpoint{3.088327in}{2.219618in}}%
\pgfpathlineto{\pgfqpoint{2.992428in}{1.457296in}}%
\pgfusepath{stroke}%
\end{pgfscope}%
\begin{pgfscope}%
\pgfpathrectangle{\pgfqpoint{0.100000in}{0.212622in}}{\pgfqpoint{3.696000in}{3.696000in}}%
\pgfusepath{clip}%
\pgfsetrectcap%
\pgfsetroundjoin%
\pgfsetlinewidth{1.505625pt}%
\definecolor{currentstroke}{rgb}{1.000000,0.000000,0.000000}%
\pgfsetstrokecolor{currentstroke}%
\pgfsetdash{}{0pt}%
\pgfpathmoveto{\pgfqpoint{3.088166in}{2.219566in}}%
\pgfpathlineto{\pgfqpoint{2.992428in}{1.457296in}}%
\pgfusepath{stroke}%
\end{pgfscope}%
\begin{pgfscope}%
\pgfpathrectangle{\pgfqpoint{0.100000in}{0.212622in}}{\pgfqpoint{3.696000in}{3.696000in}}%
\pgfusepath{clip}%
\pgfsetrectcap%
\pgfsetroundjoin%
\pgfsetlinewidth{1.505625pt}%
\definecolor{currentstroke}{rgb}{1.000000,0.000000,0.000000}%
\pgfsetstrokecolor{currentstroke}%
\pgfsetdash{}{0pt}%
\pgfpathmoveto{\pgfqpoint{3.088084in}{2.219533in}}%
\pgfpathlineto{\pgfqpoint{2.992428in}{1.457296in}}%
\pgfusepath{stroke}%
\end{pgfscope}%
\begin{pgfscope}%
\pgfpathrectangle{\pgfqpoint{0.100000in}{0.212622in}}{\pgfqpoint{3.696000in}{3.696000in}}%
\pgfusepath{clip}%
\pgfsetrectcap%
\pgfsetroundjoin%
\pgfsetlinewidth{1.505625pt}%
\definecolor{currentstroke}{rgb}{1.000000,0.000000,0.000000}%
\pgfsetstrokecolor{currentstroke}%
\pgfsetdash{}{0pt}%
\pgfpathmoveto{\pgfqpoint{3.087369in}{2.219326in}}%
\pgfpathlineto{\pgfqpoint{2.992428in}{1.457296in}}%
\pgfusepath{stroke}%
\end{pgfscope}%
\begin{pgfscope}%
\pgfpathrectangle{\pgfqpoint{0.100000in}{0.212622in}}{\pgfqpoint{3.696000in}{3.696000in}}%
\pgfusepath{clip}%
\pgfsetrectcap%
\pgfsetroundjoin%
\pgfsetlinewidth{1.505625pt}%
\definecolor{currentstroke}{rgb}{1.000000,0.000000,0.000000}%
\pgfsetstrokecolor{currentstroke}%
\pgfsetdash{}{0pt}%
\pgfpathmoveto{\pgfqpoint{3.086968in}{2.219194in}}%
\pgfpathlineto{\pgfqpoint{2.992428in}{1.457296in}}%
\pgfusepath{stroke}%
\end{pgfscope}%
\begin{pgfscope}%
\pgfpathrectangle{\pgfqpoint{0.100000in}{0.212622in}}{\pgfqpoint{3.696000in}{3.696000in}}%
\pgfusepath{clip}%
\pgfsetrectcap%
\pgfsetroundjoin%
\pgfsetlinewidth{1.505625pt}%
\definecolor{currentstroke}{rgb}{1.000000,0.000000,0.000000}%
\pgfsetstrokecolor{currentstroke}%
\pgfsetdash{}{0pt}%
\pgfpathmoveto{\pgfqpoint{3.086757in}{2.219122in}}%
\pgfpathlineto{\pgfqpoint{2.992428in}{1.457296in}}%
\pgfusepath{stroke}%
\end{pgfscope}%
\begin{pgfscope}%
\pgfpathrectangle{\pgfqpoint{0.100000in}{0.212622in}}{\pgfqpoint{3.696000in}{3.696000in}}%
\pgfusepath{clip}%
\pgfsetrectcap%
\pgfsetroundjoin%
\pgfsetlinewidth{1.505625pt}%
\definecolor{currentstroke}{rgb}{1.000000,0.000000,0.000000}%
\pgfsetstrokecolor{currentstroke}%
\pgfsetdash{}{0pt}%
\pgfpathmoveto{\pgfqpoint{3.086131in}{2.218850in}}%
\pgfpathlineto{\pgfqpoint{2.992428in}{1.457296in}}%
\pgfusepath{stroke}%
\end{pgfscope}%
\begin{pgfscope}%
\pgfpathrectangle{\pgfqpoint{0.100000in}{0.212622in}}{\pgfqpoint{3.696000in}{3.696000in}}%
\pgfusepath{clip}%
\pgfsetrectcap%
\pgfsetroundjoin%
\pgfsetlinewidth{1.505625pt}%
\definecolor{currentstroke}{rgb}{1.000000,0.000000,0.000000}%
\pgfsetstrokecolor{currentstroke}%
\pgfsetdash{}{0pt}%
\pgfpathmoveto{\pgfqpoint{3.085081in}{2.218368in}}%
\pgfpathlineto{\pgfqpoint{2.992428in}{1.457296in}}%
\pgfusepath{stroke}%
\end{pgfscope}%
\begin{pgfscope}%
\pgfpathrectangle{\pgfqpoint{0.100000in}{0.212622in}}{\pgfqpoint{3.696000in}{3.696000in}}%
\pgfusepath{clip}%
\pgfsetrectcap%
\pgfsetroundjoin%
\pgfsetlinewidth{1.505625pt}%
\definecolor{currentstroke}{rgb}{1.000000,0.000000,0.000000}%
\pgfsetstrokecolor{currentstroke}%
\pgfsetdash{}{0pt}%
\pgfpathmoveto{\pgfqpoint{3.084501in}{2.218236in}}%
\pgfpathlineto{\pgfqpoint{2.992428in}{1.457296in}}%
\pgfusepath{stroke}%
\end{pgfscope}%
\begin{pgfscope}%
\pgfpathrectangle{\pgfqpoint{0.100000in}{0.212622in}}{\pgfqpoint{3.696000in}{3.696000in}}%
\pgfusepath{clip}%
\pgfsetrectcap%
\pgfsetroundjoin%
\pgfsetlinewidth{1.505625pt}%
\definecolor{currentstroke}{rgb}{1.000000,0.000000,0.000000}%
\pgfsetstrokecolor{currentstroke}%
\pgfsetdash{}{0pt}%
\pgfpathmoveto{\pgfqpoint{3.084165in}{2.218126in}}%
\pgfpathlineto{\pgfqpoint{2.992428in}{1.457296in}}%
\pgfusepath{stroke}%
\end{pgfscope}%
\begin{pgfscope}%
\pgfpathrectangle{\pgfqpoint{0.100000in}{0.212622in}}{\pgfqpoint{3.696000in}{3.696000in}}%
\pgfusepath{clip}%
\pgfsetrectcap%
\pgfsetroundjoin%
\pgfsetlinewidth{1.505625pt}%
\definecolor{currentstroke}{rgb}{1.000000,0.000000,0.000000}%
\pgfsetstrokecolor{currentstroke}%
\pgfsetdash{}{0pt}%
\pgfpathmoveto{\pgfqpoint{3.083993in}{2.218059in}}%
\pgfpathlineto{\pgfqpoint{2.992428in}{1.457296in}}%
\pgfusepath{stroke}%
\end{pgfscope}%
\begin{pgfscope}%
\pgfpathrectangle{\pgfqpoint{0.100000in}{0.212622in}}{\pgfqpoint{3.696000in}{3.696000in}}%
\pgfusepath{clip}%
\pgfsetrectcap%
\pgfsetroundjoin%
\pgfsetlinewidth{1.505625pt}%
\definecolor{currentstroke}{rgb}{1.000000,0.000000,0.000000}%
\pgfsetstrokecolor{currentstroke}%
\pgfsetdash{}{0pt}%
\pgfpathmoveto{\pgfqpoint{3.083344in}{2.217943in}}%
\pgfpathlineto{\pgfqpoint{2.992428in}{1.457296in}}%
\pgfusepath{stroke}%
\end{pgfscope}%
\begin{pgfscope}%
\pgfpathrectangle{\pgfqpoint{0.100000in}{0.212622in}}{\pgfqpoint{3.696000in}{3.696000in}}%
\pgfusepath{clip}%
\pgfsetrectcap%
\pgfsetroundjoin%
\pgfsetlinewidth{1.505625pt}%
\definecolor{currentstroke}{rgb}{1.000000,0.000000,0.000000}%
\pgfsetstrokecolor{currentstroke}%
\pgfsetdash{}{0pt}%
\pgfpathmoveto{\pgfqpoint{3.082967in}{2.217831in}}%
\pgfpathlineto{\pgfqpoint{2.992428in}{1.457296in}}%
\pgfusepath{stroke}%
\end{pgfscope}%
\begin{pgfscope}%
\pgfpathrectangle{\pgfqpoint{0.100000in}{0.212622in}}{\pgfqpoint{3.696000in}{3.696000in}}%
\pgfusepath{clip}%
\pgfsetrectcap%
\pgfsetroundjoin%
\pgfsetlinewidth{1.505625pt}%
\definecolor{currentstroke}{rgb}{1.000000,0.000000,0.000000}%
\pgfsetstrokecolor{currentstroke}%
\pgfsetdash{}{0pt}%
\pgfpathmoveto{\pgfqpoint{3.082768in}{2.217768in}}%
\pgfpathlineto{\pgfqpoint{2.992428in}{1.457296in}}%
\pgfusepath{stroke}%
\end{pgfscope}%
\begin{pgfscope}%
\pgfpathrectangle{\pgfqpoint{0.100000in}{0.212622in}}{\pgfqpoint{3.696000in}{3.696000in}}%
\pgfusepath{clip}%
\pgfsetrectcap%
\pgfsetroundjoin%
\pgfsetlinewidth{1.505625pt}%
\definecolor{currentstroke}{rgb}{1.000000,0.000000,0.000000}%
\pgfsetstrokecolor{currentstroke}%
\pgfsetdash{}{0pt}%
\pgfpathmoveto{\pgfqpoint{3.082653in}{2.217719in}}%
\pgfpathlineto{\pgfqpoint{2.992428in}{1.457296in}}%
\pgfusepath{stroke}%
\end{pgfscope}%
\begin{pgfscope}%
\pgfpathrectangle{\pgfqpoint{0.100000in}{0.212622in}}{\pgfqpoint{3.696000in}{3.696000in}}%
\pgfusepath{clip}%
\pgfsetrectcap%
\pgfsetroundjoin%
\pgfsetlinewidth{1.505625pt}%
\definecolor{currentstroke}{rgb}{1.000000,0.000000,0.000000}%
\pgfsetstrokecolor{currentstroke}%
\pgfsetdash{}{0pt}%
\pgfpathmoveto{\pgfqpoint{3.081763in}{2.217623in}}%
\pgfpathlineto{\pgfqpoint{2.984908in}{1.449859in}}%
\pgfusepath{stroke}%
\end{pgfscope}%
\begin{pgfscope}%
\pgfpathrectangle{\pgfqpoint{0.100000in}{0.212622in}}{\pgfqpoint{3.696000in}{3.696000in}}%
\pgfusepath{clip}%
\pgfsetrectcap%
\pgfsetroundjoin%
\pgfsetlinewidth{1.505625pt}%
\definecolor{currentstroke}{rgb}{1.000000,0.000000,0.000000}%
\pgfsetstrokecolor{currentstroke}%
\pgfsetdash{}{0pt}%
\pgfpathmoveto{\pgfqpoint{3.081300in}{2.217604in}}%
\pgfpathlineto{\pgfqpoint{2.984908in}{1.449859in}}%
\pgfusepath{stroke}%
\end{pgfscope}%
\begin{pgfscope}%
\pgfpathrectangle{\pgfqpoint{0.100000in}{0.212622in}}{\pgfqpoint{3.696000in}{3.696000in}}%
\pgfusepath{clip}%
\pgfsetrectcap%
\pgfsetroundjoin%
\pgfsetlinewidth{1.505625pt}%
\definecolor{currentstroke}{rgb}{1.000000,0.000000,0.000000}%
\pgfsetstrokecolor{currentstroke}%
\pgfsetdash{}{0pt}%
\pgfpathmoveto{\pgfqpoint{3.081032in}{2.217565in}}%
\pgfpathlineto{\pgfqpoint{2.984908in}{1.449859in}}%
\pgfusepath{stroke}%
\end{pgfscope}%
\begin{pgfscope}%
\pgfpathrectangle{\pgfqpoint{0.100000in}{0.212622in}}{\pgfqpoint{3.696000in}{3.696000in}}%
\pgfusepath{clip}%
\pgfsetrectcap%
\pgfsetroundjoin%
\pgfsetlinewidth{1.505625pt}%
\definecolor{currentstroke}{rgb}{1.000000,0.000000,0.000000}%
\pgfsetstrokecolor{currentstroke}%
\pgfsetdash{}{0pt}%
\pgfpathmoveto{\pgfqpoint{3.080230in}{2.217335in}}%
\pgfpathlineto{\pgfqpoint{2.984908in}{1.449859in}}%
\pgfusepath{stroke}%
\end{pgfscope}%
\begin{pgfscope}%
\pgfpathrectangle{\pgfqpoint{0.100000in}{0.212622in}}{\pgfqpoint{3.696000in}{3.696000in}}%
\pgfusepath{clip}%
\pgfsetrectcap%
\pgfsetroundjoin%
\pgfsetlinewidth{1.505625pt}%
\definecolor{currentstroke}{rgb}{1.000000,0.000000,0.000000}%
\pgfsetstrokecolor{currentstroke}%
\pgfsetdash{}{0pt}%
\pgfpathmoveto{\pgfqpoint{3.079107in}{2.217313in}}%
\pgfpathlineto{\pgfqpoint{2.984908in}{1.449859in}}%
\pgfusepath{stroke}%
\end{pgfscope}%
\begin{pgfscope}%
\pgfpathrectangle{\pgfqpoint{0.100000in}{0.212622in}}{\pgfqpoint{3.696000in}{3.696000in}}%
\pgfusepath{clip}%
\pgfsetrectcap%
\pgfsetroundjoin%
\pgfsetlinewidth{1.505625pt}%
\definecolor{currentstroke}{rgb}{1.000000,0.000000,0.000000}%
\pgfsetstrokecolor{currentstroke}%
\pgfsetdash{}{0pt}%
\pgfpathmoveto{\pgfqpoint{3.078455in}{2.217190in}}%
\pgfpathlineto{\pgfqpoint{2.984908in}{1.449859in}}%
\pgfusepath{stroke}%
\end{pgfscope}%
\begin{pgfscope}%
\pgfpathrectangle{\pgfqpoint{0.100000in}{0.212622in}}{\pgfqpoint{3.696000in}{3.696000in}}%
\pgfusepath{clip}%
\pgfsetrectcap%
\pgfsetroundjoin%
\pgfsetlinewidth{1.505625pt}%
\definecolor{currentstroke}{rgb}{1.000000,0.000000,0.000000}%
\pgfsetstrokecolor{currentstroke}%
\pgfsetdash{}{0pt}%
\pgfpathmoveto{\pgfqpoint{3.078119in}{2.217067in}}%
\pgfpathlineto{\pgfqpoint{2.984908in}{1.449859in}}%
\pgfusepath{stroke}%
\end{pgfscope}%
\begin{pgfscope}%
\pgfpathrectangle{\pgfqpoint{0.100000in}{0.212622in}}{\pgfqpoint{3.696000in}{3.696000in}}%
\pgfusepath{clip}%
\pgfsetrectcap%
\pgfsetroundjoin%
\pgfsetlinewidth{1.505625pt}%
\definecolor{currentstroke}{rgb}{1.000000,0.000000,0.000000}%
\pgfsetstrokecolor{currentstroke}%
\pgfsetdash{}{0pt}%
\pgfpathmoveto{\pgfqpoint{3.077256in}{2.216868in}}%
\pgfpathlineto{\pgfqpoint{2.984908in}{1.449859in}}%
\pgfusepath{stroke}%
\end{pgfscope}%
\begin{pgfscope}%
\pgfpathrectangle{\pgfqpoint{0.100000in}{0.212622in}}{\pgfqpoint{3.696000in}{3.696000in}}%
\pgfusepath{clip}%
\pgfsetrectcap%
\pgfsetroundjoin%
\pgfsetlinewidth{1.505625pt}%
\definecolor{currentstroke}{rgb}{1.000000,0.000000,0.000000}%
\pgfsetstrokecolor{currentstroke}%
\pgfsetdash{}{0pt}%
\pgfpathmoveto{\pgfqpoint{3.076782in}{2.216719in}}%
\pgfpathlineto{\pgfqpoint{2.984908in}{1.449859in}}%
\pgfusepath{stroke}%
\end{pgfscope}%
\begin{pgfscope}%
\pgfpathrectangle{\pgfqpoint{0.100000in}{0.212622in}}{\pgfqpoint{3.696000in}{3.696000in}}%
\pgfusepath{clip}%
\pgfsetrectcap%
\pgfsetroundjoin%
\pgfsetlinewidth{1.505625pt}%
\definecolor{currentstroke}{rgb}{1.000000,0.000000,0.000000}%
\pgfsetstrokecolor{currentstroke}%
\pgfsetdash{}{0pt}%
\pgfpathmoveto{\pgfqpoint{3.076056in}{2.216447in}}%
\pgfpathlineto{\pgfqpoint{2.984908in}{1.449859in}}%
\pgfusepath{stroke}%
\end{pgfscope}%
\begin{pgfscope}%
\pgfpathrectangle{\pgfqpoint{0.100000in}{0.212622in}}{\pgfqpoint{3.696000in}{3.696000in}}%
\pgfusepath{clip}%
\pgfsetrectcap%
\pgfsetroundjoin%
\pgfsetlinewidth{1.505625pt}%
\definecolor{currentstroke}{rgb}{1.000000,0.000000,0.000000}%
\pgfsetstrokecolor{currentstroke}%
\pgfsetdash{}{0pt}%
\pgfpathmoveto{\pgfqpoint{3.074680in}{2.215917in}}%
\pgfpathlineto{\pgfqpoint{2.977378in}{1.442412in}}%
\pgfusepath{stroke}%
\end{pgfscope}%
\begin{pgfscope}%
\pgfpathrectangle{\pgfqpoint{0.100000in}{0.212622in}}{\pgfqpoint{3.696000in}{3.696000in}}%
\pgfusepath{clip}%
\pgfsetrectcap%
\pgfsetroundjoin%
\pgfsetlinewidth{1.505625pt}%
\definecolor{currentstroke}{rgb}{1.000000,0.000000,0.000000}%
\pgfsetstrokecolor{currentstroke}%
\pgfsetdash{}{0pt}%
\pgfpathmoveto{\pgfqpoint{3.073938in}{2.215645in}}%
\pgfpathlineto{\pgfqpoint{2.977378in}{1.442412in}}%
\pgfusepath{stroke}%
\end{pgfscope}%
\begin{pgfscope}%
\pgfpathrectangle{\pgfqpoint{0.100000in}{0.212622in}}{\pgfqpoint{3.696000in}{3.696000in}}%
\pgfusepath{clip}%
\pgfsetrectcap%
\pgfsetroundjoin%
\pgfsetlinewidth{1.505625pt}%
\definecolor{currentstroke}{rgb}{1.000000,0.000000,0.000000}%
\pgfsetstrokecolor{currentstroke}%
\pgfsetdash{}{0pt}%
\pgfpathmoveto{\pgfqpoint{3.073558in}{2.215485in}}%
\pgfpathlineto{\pgfqpoint{2.977378in}{1.442412in}}%
\pgfusepath{stroke}%
\end{pgfscope}%
\begin{pgfscope}%
\pgfpathrectangle{\pgfqpoint{0.100000in}{0.212622in}}{\pgfqpoint{3.696000in}{3.696000in}}%
\pgfusepath{clip}%
\pgfsetrectcap%
\pgfsetroundjoin%
\pgfsetlinewidth{1.505625pt}%
\definecolor{currentstroke}{rgb}{1.000000,0.000000,0.000000}%
\pgfsetstrokecolor{currentstroke}%
\pgfsetdash{}{0pt}%
\pgfpathmoveto{\pgfqpoint{3.073338in}{2.215400in}}%
\pgfpathlineto{\pgfqpoint{2.977378in}{1.442412in}}%
\pgfusepath{stroke}%
\end{pgfscope}%
\begin{pgfscope}%
\pgfpathrectangle{\pgfqpoint{0.100000in}{0.212622in}}{\pgfqpoint{3.696000in}{3.696000in}}%
\pgfusepath{clip}%
\pgfsetrectcap%
\pgfsetroundjoin%
\pgfsetlinewidth{1.505625pt}%
\definecolor{currentstroke}{rgb}{1.000000,0.000000,0.000000}%
\pgfsetstrokecolor{currentstroke}%
\pgfsetdash{}{0pt}%
\pgfpathmoveto{\pgfqpoint{3.072743in}{2.215093in}}%
\pgfpathlineto{\pgfqpoint{2.977378in}{1.442412in}}%
\pgfusepath{stroke}%
\end{pgfscope}%
\begin{pgfscope}%
\pgfpathrectangle{\pgfqpoint{0.100000in}{0.212622in}}{\pgfqpoint{3.696000in}{3.696000in}}%
\pgfusepath{clip}%
\pgfsetrectcap%
\pgfsetroundjoin%
\pgfsetlinewidth{1.505625pt}%
\definecolor{currentstroke}{rgb}{1.000000,0.000000,0.000000}%
\pgfsetstrokecolor{currentstroke}%
\pgfsetdash{}{0pt}%
\pgfpathmoveto{\pgfqpoint{3.072420in}{2.214958in}}%
\pgfpathlineto{\pgfqpoint{2.977378in}{1.442412in}}%
\pgfusepath{stroke}%
\end{pgfscope}%
\begin{pgfscope}%
\pgfpathrectangle{\pgfqpoint{0.100000in}{0.212622in}}{\pgfqpoint{3.696000in}{3.696000in}}%
\pgfusepath{clip}%
\pgfsetrectcap%
\pgfsetroundjoin%
\pgfsetlinewidth{1.505625pt}%
\definecolor{currentstroke}{rgb}{1.000000,0.000000,0.000000}%
\pgfsetstrokecolor{currentstroke}%
\pgfsetdash{}{0pt}%
\pgfpathmoveto{\pgfqpoint{3.072228in}{2.214864in}}%
\pgfpathlineto{\pgfqpoint{2.977378in}{1.442412in}}%
\pgfusepath{stroke}%
\end{pgfscope}%
\begin{pgfscope}%
\pgfpathrectangle{\pgfqpoint{0.100000in}{0.212622in}}{\pgfqpoint{3.696000in}{3.696000in}}%
\pgfusepath{clip}%
\pgfsetrectcap%
\pgfsetroundjoin%
\pgfsetlinewidth{1.505625pt}%
\definecolor{currentstroke}{rgb}{1.000000,0.000000,0.000000}%
\pgfsetstrokecolor{currentstroke}%
\pgfsetdash{}{0pt}%
\pgfpathmoveto{\pgfqpoint{3.071590in}{2.214504in}}%
\pgfpathlineto{\pgfqpoint{2.977378in}{1.442412in}}%
\pgfusepath{stroke}%
\end{pgfscope}%
\begin{pgfscope}%
\pgfpathrectangle{\pgfqpoint{0.100000in}{0.212622in}}{\pgfqpoint{3.696000in}{3.696000in}}%
\pgfusepath{clip}%
\pgfsetrectcap%
\pgfsetroundjoin%
\pgfsetlinewidth{1.505625pt}%
\definecolor{currentstroke}{rgb}{1.000000,0.000000,0.000000}%
\pgfsetstrokecolor{currentstroke}%
\pgfsetdash{}{0pt}%
\pgfpathmoveto{\pgfqpoint{3.070641in}{2.214155in}}%
\pgfpathlineto{\pgfqpoint{2.977378in}{1.442412in}}%
\pgfusepath{stroke}%
\end{pgfscope}%
\begin{pgfscope}%
\pgfpathrectangle{\pgfqpoint{0.100000in}{0.212622in}}{\pgfqpoint{3.696000in}{3.696000in}}%
\pgfusepath{clip}%
\pgfsetrectcap%
\pgfsetroundjoin%
\pgfsetlinewidth{1.505625pt}%
\definecolor{currentstroke}{rgb}{1.000000,0.000000,0.000000}%
\pgfsetstrokecolor{currentstroke}%
\pgfsetdash{}{0pt}%
\pgfpathmoveto{\pgfqpoint{3.069359in}{2.213523in}}%
\pgfpathlineto{\pgfqpoint{2.977378in}{1.442412in}}%
\pgfusepath{stroke}%
\end{pgfscope}%
\begin{pgfscope}%
\pgfpathrectangle{\pgfqpoint{0.100000in}{0.212622in}}{\pgfqpoint{3.696000in}{3.696000in}}%
\pgfusepath{clip}%
\pgfsetrectcap%
\pgfsetroundjoin%
\pgfsetlinewidth{1.505625pt}%
\definecolor{currentstroke}{rgb}{1.000000,0.000000,0.000000}%
\pgfsetstrokecolor{currentstroke}%
\pgfsetdash{}{0pt}%
\pgfpathmoveto{\pgfqpoint{3.067288in}{2.212388in}}%
\pgfpathlineto{\pgfqpoint{2.969839in}{1.434957in}}%
\pgfusepath{stroke}%
\end{pgfscope}%
\begin{pgfscope}%
\pgfpathrectangle{\pgfqpoint{0.100000in}{0.212622in}}{\pgfqpoint{3.696000in}{3.696000in}}%
\pgfusepath{clip}%
\pgfsetrectcap%
\pgfsetroundjoin%
\pgfsetlinewidth{1.505625pt}%
\definecolor{currentstroke}{rgb}{1.000000,0.000000,0.000000}%
\pgfsetstrokecolor{currentstroke}%
\pgfsetdash{}{0pt}%
\pgfpathmoveto{\pgfqpoint{3.066170in}{2.211805in}}%
\pgfpathlineto{\pgfqpoint{2.969839in}{1.434957in}}%
\pgfusepath{stroke}%
\end{pgfscope}%
\begin{pgfscope}%
\pgfpathrectangle{\pgfqpoint{0.100000in}{0.212622in}}{\pgfqpoint{3.696000in}{3.696000in}}%
\pgfusepath{clip}%
\pgfsetrectcap%
\pgfsetroundjoin%
\pgfsetlinewidth{1.505625pt}%
\definecolor{currentstroke}{rgb}{1.000000,0.000000,0.000000}%
\pgfsetstrokecolor{currentstroke}%
\pgfsetdash{}{0pt}%
\pgfpathmoveto{\pgfqpoint{3.065518in}{2.211454in}}%
\pgfpathlineto{\pgfqpoint{2.969839in}{1.434957in}}%
\pgfusepath{stroke}%
\end{pgfscope}%
\begin{pgfscope}%
\pgfpathrectangle{\pgfqpoint{0.100000in}{0.212622in}}{\pgfqpoint{3.696000in}{3.696000in}}%
\pgfusepath{clip}%
\pgfsetrectcap%
\pgfsetroundjoin%
\pgfsetlinewidth{1.505625pt}%
\definecolor{currentstroke}{rgb}{1.000000,0.000000,0.000000}%
\pgfsetstrokecolor{currentstroke}%
\pgfsetdash{}{0pt}%
\pgfpathmoveto{\pgfqpoint{3.065149in}{2.211266in}}%
\pgfpathlineto{\pgfqpoint{2.969839in}{1.434957in}}%
\pgfusepath{stroke}%
\end{pgfscope}%
\begin{pgfscope}%
\pgfpathrectangle{\pgfqpoint{0.100000in}{0.212622in}}{\pgfqpoint{3.696000in}{3.696000in}}%
\pgfusepath{clip}%
\pgfsetrectcap%
\pgfsetroundjoin%
\pgfsetlinewidth{1.505625pt}%
\definecolor{currentstroke}{rgb}{1.000000,0.000000,0.000000}%
\pgfsetstrokecolor{currentstroke}%
\pgfsetdash{}{0pt}%
\pgfpathmoveto{\pgfqpoint{3.064397in}{2.211200in}}%
\pgfpathlineto{\pgfqpoint{2.969839in}{1.434957in}}%
\pgfusepath{stroke}%
\end{pgfscope}%
\begin{pgfscope}%
\pgfpathrectangle{\pgfqpoint{0.100000in}{0.212622in}}{\pgfqpoint{3.696000in}{3.696000in}}%
\pgfusepath{clip}%
\pgfsetrectcap%
\pgfsetroundjoin%
\pgfsetlinewidth{1.505625pt}%
\definecolor{currentstroke}{rgb}{1.000000,0.000000,0.000000}%
\pgfsetstrokecolor{currentstroke}%
\pgfsetdash{}{0pt}%
\pgfpathmoveto{\pgfqpoint{3.063963in}{2.211111in}}%
\pgfpathlineto{\pgfqpoint{2.969839in}{1.434957in}}%
\pgfusepath{stroke}%
\end{pgfscope}%
\begin{pgfscope}%
\pgfpathrectangle{\pgfqpoint{0.100000in}{0.212622in}}{\pgfqpoint{3.696000in}{3.696000in}}%
\pgfusepath{clip}%
\pgfsetrectcap%
\pgfsetroundjoin%
\pgfsetlinewidth{1.505625pt}%
\definecolor{currentstroke}{rgb}{1.000000,0.000000,0.000000}%
\pgfsetstrokecolor{currentstroke}%
\pgfsetdash{}{0pt}%
\pgfpathmoveto{\pgfqpoint{3.063732in}{2.211017in}}%
\pgfpathlineto{\pgfqpoint{2.969839in}{1.434957in}}%
\pgfusepath{stroke}%
\end{pgfscope}%
\begin{pgfscope}%
\pgfpathrectangle{\pgfqpoint{0.100000in}{0.212622in}}{\pgfqpoint{3.696000in}{3.696000in}}%
\pgfusepath{clip}%
\pgfsetrectcap%
\pgfsetroundjoin%
\pgfsetlinewidth{1.505625pt}%
\definecolor{currentstroke}{rgb}{1.000000,0.000000,0.000000}%
\pgfsetstrokecolor{currentstroke}%
\pgfsetdash{}{0pt}%
\pgfpathmoveto{\pgfqpoint{3.062830in}{2.210689in}}%
\pgfpathlineto{\pgfqpoint{2.969839in}{1.434957in}}%
\pgfusepath{stroke}%
\end{pgfscope}%
\begin{pgfscope}%
\pgfpathrectangle{\pgfqpoint{0.100000in}{0.212622in}}{\pgfqpoint{3.696000in}{3.696000in}}%
\pgfusepath{clip}%
\pgfsetrectcap%
\pgfsetroundjoin%
\pgfsetlinewidth{1.505625pt}%
\definecolor{currentstroke}{rgb}{1.000000,0.000000,0.000000}%
\pgfsetstrokecolor{currentstroke}%
\pgfsetdash{}{0pt}%
\pgfpathmoveto{\pgfqpoint{3.062323in}{2.210465in}}%
\pgfpathlineto{\pgfqpoint{2.969839in}{1.434957in}}%
\pgfusepath{stroke}%
\end{pgfscope}%
\begin{pgfscope}%
\pgfpathrectangle{\pgfqpoint{0.100000in}{0.212622in}}{\pgfqpoint{3.696000in}{3.696000in}}%
\pgfusepath{clip}%
\pgfsetrectcap%
\pgfsetroundjoin%
\pgfsetlinewidth{1.505625pt}%
\definecolor{currentstroke}{rgb}{1.000000,0.000000,0.000000}%
\pgfsetstrokecolor{currentstroke}%
\pgfsetdash{}{0pt}%
\pgfpathmoveto{\pgfqpoint{3.062056in}{2.210319in}}%
\pgfpathlineto{\pgfqpoint{2.969839in}{1.434957in}}%
\pgfusepath{stroke}%
\end{pgfscope}%
\begin{pgfscope}%
\pgfpathrectangle{\pgfqpoint{0.100000in}{0.212622in}}{\pgfqpoint{3.696000in}{3.696000in}}%
\pgfusepath{clip}%
\pgfsetrectcap%
\pgfsetroundjoin%
\pgfsetlinewidth{1.505625pt}%
\definecolor{currentstroke}{rgb}{1.000000,0.000000,0.000000}%
\pgfsetstrokecolor{currentstroke}%
\pgfsetdash{}{0pt}%
\pgfpathmoveto{\pgfqpoint{3.061527in}{2.210075in}}%
\pgfpathlineto{\pgfqpoint{2.962291in}{1.427493in}}%
\pgfusepath{stroke}%
\end{pgfscope}%
\begin{pgfscope}%
\pgfpathrectangle{\pgfqpoint{0.100000in}{0.212622in}}{\pgfqpoint{3.696000in}{3.696000in}}%
\pgfusepath{clip}%
\pgfsetrectcap%
\pgfsetroundjoin%
\pgfsetlinewidth{1.505625pt}%
\definecolor{currentstroke}{rgb}{1.000000,0.000000,0.000000}%
\pgfsetstrokecolor{currentstroke}%
\pgfsetdash{}{0pt}%
\pgfpathmoveto{\pgfqpoint{3.060673in}{2.209660in}}%
\pgfpathlineto{\pgfqpoint{2.962291in}{1.427493in}}%
\pgfusepath{stroke}%
\end{pgfscope}%
\begin{pgfscope}%
\pgfpathrectangle{\pgfqpoint{0.100000in}{0.212622in}}{\pgfqpoint{3.696000in}{3.696000in}}%
\pgfusepath{clip}%
\pgfsetrectcap%
\pgfsetroundjoin%
\pgfsetlinewidth{1.505625pt}%
\definecolor{currentstroke}{rgb}{1.000000,0.000000,0.000000}%
\pgfsetstrokecolor{currentstroke}%
\pgfsetdash{}{0pt}%
\pgfpathmoveto{\pgfqpoint{3.059578in}{2.209133in}}%
\pgfpathlineto{\pgfqpoint{2.962291in}{1.427493in}}%
\pgfusepath{stroke}%
\end{pgfscope}%
\begin{pgfscope}%
\pgfpathrectangle{\pgfqpoint{0.100000in}{0.212622in}}{\pgfqpoint{3.696000in}{3.696000in}}%
\pgfusepath{clip}%
\pgfsetrectcap%
\pgfsetroundjoin%
\pgfsetlinewidth{1.505625pt}%
\definecolor{currentstroke}{rgb}{1.000000,0.000000,0.000000}%
\pgfsetstrokecolor{currentstroke}%
\pgfsetdash{}{0pt}%
\pgfpathmoveto{\pgfqpoint{3.058182in}{2.208421in}}%
\pgfpathlineto{\pgfqpoint{2.962291in}{1.427493in}}%
\pgfusepath{stroke}%
\end{pgfscope}%
\begin{pgfscope}%
\pgfpathrectangle{\pgfqpoint{0.100000in}{0.212622in}}{\pgfqpoint{3.696000in}{3.696000in}}%
\pgfusepath{clip}%
\pgfsetrectcap%
\pgfsetroundjoin%
\pgfsetlinewidth{1.505625pt}%
\definecolor{currentstroke}{rgb}{1.000000,0.000000,0.000000}%
\pgfsetstrokecolor{currentstroke}%
\pgfsetdash{}{0pt}%
\pgfpathmoveto{\pgfqpoint{3.056338in}{2.207444in}}%
\pgfpathlineto{\pgfqpoint{2.962291in}{1.427493in}}%
\pgfusepath{stroke}%
\end{pgfscope}%
\begin{pgfscope}%
\pgfpathrectangle{\pgfqpoint{0.100000in}{0.212622in}}{\pgfqpoint{3.696000in}{3.696000in}}%
\pgfusepath{clip}%
\pgfsetrectcap%
\pgfsetroundjoin%
\pgfsetlinewidth{1.505625pt}%
\definecolor{currentstroke}{rgb}{1.000000,0.000000,0.000000}%
\pgfsetstrokecolor{currentstroke}%
\pgfsetdash{}{0pt}%
\pgfpathmoveto{\pgfqpoint{3.054361in}{2.206490in}}%
\pgfpathlineto{\pgfqpoint{2.954733in}{1.420019in}}%
\pgfusepath{stroke}%
\end{pgfscope}%
\begin{pgfscope}%
\pgfpathrectangle{\pgfqpoint{0.100000in}{0.212622in}}{\pgfqpoint{3.696000in}{3.696000in}}%
\pgfusepath{clip}%
\pgfsetrectcap%
\pgfsetroundjoin%
\pgfsetlinewidth{1.505625pt}%
\definecolor{currentstroke}{rgb}{1.000000,0.000000,0.000000}%
\pgfsetstrokecolor{currentstroke}%
\pgfsetdash{}{0pt}%
\pgfpathmoveto{\pgfqpoint{3.051872in}{2.205214in}}%
\pgfpathlineto{\pgfqpoint{2.954733in}{1.420019in}}%
\pgfusepath{stroke}%
\end{pgfscope}%
\begin{pgfscope}%
\pgfpathrectangle{\pgfqpoint{0.100000in}{0.212622in}}{\pgfqpoint{3.696000in}{3.696000in}}%
\pgfusepath{clip}%
\pgfsetrectcap%
\pgfsetroundjoin%
\pgfsetlinewidth{1.505625pt}%
\definecolor{currentstroke}{rgb}{1.000000,0.000000,0.000000}%
\pgfsetstrokecolor{currentstroke}%
\pgfsetdash{}{0pt}%
\pgfpathmoveto{\pgfqpoint{3.049057in}{2.203795in}}%
\pgfpathlineto{\pgfqpoint{2.947167in}{1.412536in}}%
\pgfusepath{stroke}%
\end{pgfscope}%
\begin{pgfscope}%
\pgfpathrectangle{\pgfqpoint{0.100000in}{0.212622in}}{\pgfqpoint{3.696000in}{3.696000in}}%
\pgfusepath{clip}%
\pgfsetrectcap%
\pgfsetroundjoin%
\pgfsetlinewidth{1.505625pt}%
\definecolor{currentstroke}{rgb}{1.000000,0.000000,0.000000}%
\pgfsetstrokecolor{currentstroke}%
\pgfsetdash{}{0pt}%
\pgfpathmoveto{\pgfqpoint{3.047449in}{2.203331in}}%
\pgfpathlineto{\pgfqpoint{2.947167in}{1.412536in}}%
\pgfusepath{stroke}%
\end{pgfscope}%
\begin{pgfscope}%
\pgfpathrectangle{\pgfqpoint{0.100000in}{0.212622in}}{\pgfqpoint{3.696000in}{3.696000in}}%
\pgfusepath{clip}%
\pgfsetrectcap%
\pgfsetroundjoin%
\pgfsetlinewidth{1.505625pt}%
\definecolor{currentstroke}{rgb}{1.000000,0.000000,0.000000}%
\pgfsetstrokecolor{currentstroke}%
\pgfsetdash{}{0pt}%
\pgfpathmoveto{\pgfqpoint{3.046505in}{2.202959in}}%
\pgfpathlineto{\pgfqpoint{2.947167in}{1.412536in}}%
\pgfusepath{stroke}%
\end{pgfscope}%
\begin{pgfscope}%
\pgfpathrectangle{\pgfqpoint{0.100000in}{0.212622in}}{\pgfqpoint{3.696000in}{3.696000in}}%
\pgfusepath{clip}%
\pgfsetrectcap%
\pgfsetroundjoin%
\pgfsetlinewidth{1.505625pt}%
\definecolor{currentstroke}{rgb}{1.000000,0.000000,0.000000}%
\pgfsetstrokecolor{currentstroke}%
\pgfsetdash{}{0pt}%
\pgfpathmoveto{\pgfqpoint{3.046023in}{2.202712in}}%
\pgfpathlineto{\pgfqpoint{2.947167in}{1.412536in}}%
\pgfusepath{stroke}%
\end{pgfscope}%
\begin{pgfscope}%
\pgfpathrectangle{\pgfqpoint{0.100000in}{0.212622in}}{\pgfqpoint{3.696000in}{3.696000in}}%
\pgfusepath{clip}%
\pgfsetrectcap%
\pgfsetroundjoin%
\pgfsetlinewidth{1.505625pt}%
\definecolor{currentstroke}{rgb}{1.000000,0.000000,0.000000}%
\pgfsetstrokecolor{currentstroke}%
\pgfsetdash{}{0pt}%
\pgfpathmoveto{\pgfqpoint{3.044929in}{2.202251in}}%
\pgfpathlineto{\pgfqpoint{2.947167in}{1.412536in}}%
\pgfusepath{stroke}%
\end{pgfscope}%
\begin{pgfscope}%
\pgfpathrectangle{\pgfqpoint{0.100000in}{0.212622in}}{\pgfqpoint{3.696000in}{3.696000in}}%
\pgfusepath{clip}%
\pgfsetrectcap%
\pgfsetroundjoin%
\pgfsetlinewidth{1.505625pt}%
\definecolor{currentstroke}{rgb}{1.000000,0.000000,0.000000}%
\pgfsetstrokecolor{currentstroke}%
\pgfsetdash{}{0pt}%
\pgfpathmoveto{\pgfqpoint{3.044309in}{2.202022in}}%
\pgfpathlineto{\pgfqpoint{2.947167in}{1.412536in}}%
\pgfusepath{stroke}%
\end{pgfscope}%
\begin{pgfscope}%
\pgfpathrectangle{\pgfqpoint{0.100000in}{0.212622in}}{\pgfqpoint{3.696000in}{3.696000in}}%
\pgfusepath{clip}%
\pgfsetrectcap%
\pgfsetroundjoin%
\pgfsetlinewidth{1.505625pt}%
\definecolor{currentstroke}{rgb}{1.000000,0.000000,0.000000}%
\pgfsetstrokecolor{currentstroke}%
\pgfsetdash{}{0pt}%
\pgfpathmoveto{\pgfqpoint{3.043987in}{2.201873in}}%
\pgfpathlineto{\pgfqpoint{2.947167in}{1.412536in}}%
\pgfusepath{stroke}%
\end{pgfscope}%
\begin{pgfscope}%
\pgfpathrectangle{\pgfqpoint{0.100000in}{0.212622in}}{\pgfqpoint{3.696000in}{3.696000in}}%
\pgfusepath{clip}%
\pgfsetrectcap%
\pgfsetroundjoin%
\pgfsetlinewidth{1.505625pt}%
\definecolor{currentstroke}{rgb}{1.000000,0.000000,0.000000}%
\pgfsetstrokecolor{currentstroke}%
\pgfsetdash{}{0pt}%
\pgfpathmoveto{\pgfqpoint{3.043325in}{2.201534in}}%
\pgfpathlineto{\pgfqpoint{2.947167in}{1.412536in}}%
\pgfusepath{stroke}%
\end{pgfscope}%
\begin{pgfscope}%
\pgfpathrectangle{\pgfqpoint{0.100000in}{0.212622in}}{\pgfqpoint{3.696000in}{3.696000in}}%
\pgfusepath{clip}%
\pgfsetrectcap%
\pgfsetroundjoin%
\pgfsetlinewidth{1.505625pt}%
\definecolor{currentstroke}{rgb}{1.000000,0.000000,0.000000}%
\pgfsetstrokecolor{currentstroke}%
\pgfsetdash{}{0pt}%
\pgfpathmoveto{\pgfqpoint{3.042937in}{2.201324in}}%
\pgfpathlineto{\pgfqpoint{2.947167in}{1.412536in}}%
\pgfusepath{stroke}%
\end{pgfscope}%
\begin{pgfscope}%
\pgfpathrectangle{\pgfqpoint{0.100000in}{0.212622in}}{\pgfqpoint{3.696000in}{3.696000in}}%
\pgfusepath{clip}%
\pgfsetrectcap%
\pgfsetroundjoin%
\pgfsetlinewidth{1.505625pt}%
\definecolor{currentstroke}{rgb}{1.000000,0.000000,0.000000}%
\pgfsetstrokecolor{currentstroke}%
\pgfsetdash{}{0pt}%
\pgfpathmoveto{\pgfqpoint{3.042735in}{2.201202in}}%
\pgfpathlineto{\pgfqpoint{2.939591in}{1.405044in}}%
\pgfusepath{stroke}%
\end{pgfscope}%
\begin{pgfscope}%
\pgfpathrectangle{\pgfqpoint{0.100000in}{0.212622in}}{\pgfqpoint{3.696000in}{3.696000in}}%
\pgfusepath{clip}%
\pgfsetrectcap%
\pgfsetroundjoin%
\pgfsetlinewidth{1.505625pt}%
\definecolor{currentstroke}{rgb}{1.000000,0.000000,0.000000}%
\pgfsetstrokecolor{currentstroke}%
\pgfsetdash{}{0pt}%
\pgfpathmoveto{\pgfqpoint{3.042174in}{2.200913in}}%
\pgfpathlineto{\pgfqpoint{2.939591in}{1.405044in}}%
\pgfusepath{stroke}%
\end{pgfscope}%
\begin{pgfscope}%
\pgfpathrectangle{\pgfqpoint{0.100000in}{0.212622in}}{\pgfqpoint{3.696000in}{3.696000in}}%
\pgfusepath{clip}%
\pgfsetrectcap%
\pgfsetroundjoin%
\pgfsetlinewidth{1.505625pt}%
\definecolor{currentstroke}{rgb}{1.000000,0.000000,0.000000}%
\pgfsetstrokecolor{currentstroke}%
\pgfsetdash{}{0pt}%
\pgfpathmoveto{\pgfqpoint{3.041853in}{2.200816in}}%
\pgfpathlineto{\pgfqpoint{2.939591in}{1.405044in}}%
\pgfusepath{stroke}%
\end{pgfscope}%
\begin{pgfscope}%
\pgfpathrectangle{\pgfqpoint{0.100000in}{0.212622in}}{\pgfqpoint{3.696000in}{3.696000in}}%
\pgfusepath{clip}%
\pgfsetrectcap%
\pgfsetroundjoin%
\pgfsetlinewidth{1.505625pt}%
\definecolor{currentstroke}{rgb}{1.000000,0.000000,0.000000}%
\pgfsetstrokecolor{currentstroke}%
\pgfsetdash{}{0pt}%
\pgfpathmoveto{\pgfqpoint{3.041690in}{2.200761in}}%
\pgfpathlineto{\pgfqpoint{2.939591in}{1.405044in}}%
\pgfusepath{stroke}%
\end{pgfscope}%
\begin{pgfscope}%
\pgfpathrectangle{\pgfqpoint{0.100000in}{0.212622in}}{\pgfqpoint{3.696000in}{3.696000in}}%
\pgfusepath{clip}%
\pgfsetrectcap%
\pgfsetroundjoin%
\pgfsetlinewidth{1.505625pt}%
\definecolor{currentstroke}{rgb}{1.000000,0.000000,0.000000}%
\pgfsetstrokecolor{currentstroke}%
\pgfsetdash{}{0pt}%
\pgfpathmoveto{\pgfqpoint{3.041593in}{2.200720in}}%
\pgfpathlineto{\pgfqpoint{2.939591in}{1.405044in}}%
\pgfusepath{stroke}%
\end{pgfscope}%
\begin{pgfscope}%
\pgfpathrectangle{\pgfqpoint{0.100000in}{0.212622in}}{\pgfqpoint{3.696000in}{3.696000in}}%
\pgfusepath{clip}%
\pgfsetrectcap%
\pgfsetroundjoin%
\pgfsetlinewidth{1.505625pt}%
\definecolor{currentstroke}{rgb}{1.000000,0.000000,0.000000}%
\pgfsetstrokecolor{currentstroke}%
\pgfsetdash{}{0pt}%
\pgfpathmoveto{\pgfqpoint{3.040956in}{2.200372in}}%
\pgfpathlineto{\pgfqpoint{2.939591in}{1.405044in}}%
\pgfusepath{stroke}%
\end{pgfscope}%
\begin{pgfscope}%
\pgfpathrectangle{\pgfqpoint{0.100000in}{0.212622in}}{\pgfqpoint{3.696000in}{3.696000in}}%
\pgfusepath{clip}%
\pgfsetrectcap%
\pgfsetroundjoin%
\pgfsetlinewidth{1.505625pt}%
\definecolor{currentstroke}{rgb}{1.000000,0.000000,0.000000}%
\pgfsetstrokecolor{currentstroke}%
\pgfsetdash{}{0pt}%
\pgfpathmoveto{\pgfqpoint{3.040058in}{2.200054in}}%
\pgfpathlineto{\pgfqpoint{2.939591in}{1.405044in}}%
\pgfusepath{stroke}%
\end{pgfscope}%
\begin{pgfscope}%
\pgfpathrectangle{\pgfqpoint{0.100000in}{0.212622in}}{\pgfqpoint{3.696000in}{3.696000in}}%
\pgfusepath{clip}%
\pgfsetrectcap%
\pgfsetroundjoin%
\pgfsetlinewidth{1.505625pt}%
\definecolor{currentstroke}{rgb}{1.000000,0.000000,0.000000}%
\pgfsetstrokecolor{currentstroke}%
\pgfsetdash{}{0pt}%
\pgfpathmoveto{\pgfqpoint{3.038893in}{2.199409in}}%
\pgfpathlineto{\pgfqpoint{2.939591in}{1.405044in}}%
\pgfusepath{stroke}%
\end{pgfscope}%
\begin{pgfscope}%
\pgfpathrectangle{\pgfqpoint{0.100000in}{0.212622in}}{\pgfqpoint{3.696000in}{3.696000in}}%
\pgfusepath{clip}%
\pgfsetrectcap%
\pgfsetroundjoin%
\pgfsetlinewidth{1.505625pt}%
\definecolor{currentstroke}{rgb}{1.000000,0.000000,0.000000}%
\pgfsetstrokecolor{currentstroke}%
\pgfsetdash{}{0pt}%
\pgfpathmoveto{\pgfqpoint{3.037310in}{2.198484in}}%
\pgfpathlineto{\pgfqpoint{2.939591in}{1.405044in}}%
\pgfusepath{stroke}%
\end{pgfscope}%
\begin{pgfscope}%
\pgfpathrectangle{\pgfqpoint{0.100000in}{0.212622in}}{\pgfqpoint{3.696000in}{3.696000in}}%
\pgfusepath{clip}%
\pgfsetrectcap%
\pgfsetroundjoin%
\pgfsetlinewidth{1.505625pt}%
\definecolor{currentstroke}{rgb}{1.000000,0.000000,0.000000}%
\pgfsetstrokecolor{currentstroke}%
\pgfsetdash{}{0pt}%
\pgfpathmoveto{\pgfqpoint{3.035222in}{2.197647in}}%
\pgfpathlineto{\pgfqpoint{2.932006in}{1.397543in}}%
\pgfusepath{stroke}%
\end{pgfscope}%
\begin{pgfscope}%
\pgfpathrectangle{\pgfqpoint{0.100000in}{0.212622in}}{\pgfqpoint{3.696000in}{3.696000in}}%
\pgfusepath{clip}%
\pgfsetrectcap%
\pgfsetroundjoin%
\pgfsetlinewidth{1.505625pt}%
\definecolor{currentstroke}{rgb}{1.000000,0.000000,0.000000}%
\pgfsetstrokecolor{currentstroke}%
\pgfsetdash{}{0pt}%
\pgfpathmoveto{\pgfqpoint{3.034038in}{2.197138in}}%
\pgfpathlineto{\pgfqpoint{2.932006in}{1.397543in}}%
\pgfusepath{stroke}%
\end{pgfscope}%
\begin{pgfscope}%
\pgfpathrectangle{\pgfqpoint{0.100000in}{0.212622in}}{\pgfqpoint{3.696000in}{3.696000in}}%
\pgfusepath{clip}%
\pgfsetrectcap%
\pgfsetroundjoin%
\pgfsetlinewidth{1.505625pt}%
\definecolor{currentstroke}{rgb}{1.000000,0.000000,0.000000}%
\pgfsetstrokecolor{currentstroke}%
\pgfsetdash{}{0pt}%
\pgfpathmoveto{\pgfqpoint{3.032649in}{2.196406in}}%
\pgfpathlineto{\pgfqpoint{2.932006in}{1.397543in}}%
\pgfusepath{stroke}%
\end{pgfscope}%
\begin{pgfscope}%
\pgfpathrectangle{\pgfqpoint{0.100000in}{0.212622in}}{\pgfqpoint{3.696000in}{3.696000in}}%
\pgfusepath{clip}%
\pgfsetrectcap%
\pgfsetroundjoin%
\pgfsetlinewidth{1.505625pt}%
\definecolor{currentstroke}{rgb}{1.000000,0.000000,0.000000}%
\pgfsetstrokecolor{currentstroke}%
\pgfsetdash{}{0pt}%
\pgfpathmoveto{\pgfqpoint{3.030227in}{2.195453in}}%
\pgfpathlineto{\pgfqpoint{2.932006in}{1.397543in}}%
\pgfusepath{stroke}%
\end{pgfscope}%
\begin{pgfscope}%
\pgfpathrectangle{\pgfqpoint{0.100000in}{0.212622in}}{\pgfqpoint{3.696000in}{3.696000in}}%
\pgfusepath{clip}%
\pgfsetrectcap%
\pgfsetroundjoin%
\pgfsetlinewidth{1.505625pt}%
\definecolor{currentstroke}{rgb}{1.000000,0.000000,0.000000}%
\pgfsetstrokecolor{currentstroke}%
\pgfsetdash{}{0pt}%
\pgfpathmoveto{\pgfqpoint{3.028894in}{2.194882in}}%
\pgfpathlineto{\pgfqpoint{2.924411in}{1.390033in}}%
\pgfusepath{stroke}%
\end{pgfscope}%
\begin{pgfscope}%
\pgfpathrectangle{\pgfqpoint{0.100000in}{0.212622in}}{\pgfqpoint{3.696000in}{3.696000in}}%
\pgfusepath{clip}%
\pgfsetrectcap%
\pgfsetroundjoin%
\pgfsetlinewidth{1.505625pt}%
\definecolor{currentstroke}{rgb}{1.000000,0.000000,0.000000}%
\pgfsetstrokecolor{currentstroke}%
\pgfsetdash{}{0pt}%
\pgfpathmoveto{\pgfqpoint{3.028220in}{2.194549in}}%
\pgfpathlineto{\pgfqpoint{2.924411in}{1.390033in}}%
\pgfusepath{stroke}%
\end{pgfscope}%
\begin{pgfscope}%
\pgfpathrectangle{\pgfqpoint{0.100000in}{0.212622in}}{\pgfqpoint{3.696000in}{3.696000in}}%
\pgfusepath{clip}%
\pgfsetrectcap%
\pgfsetroundjoin%
\pgfsetlinewidth{1.505625pt}%
\definecolor{currentstroke}{rgb}{1.000000,0.000000,0.000000}%
\pgfsetstrokecolor{currentstroke}%
\pgfsetdash{}{0pt}%
\pgfpathmoveto{\pgfqpoint{3.027165in}{2.194069in}}%
\pgfpathlineto{\pgfqpoint{2.924411in}{1.390033in}}%
\pgfusepath{stroke}%
\end{pgfscope}%
\begin{pgfscope}%
\pgfpathrectangle{\pgfqpoint{0.100000in}{0.212622in}}{\pgfqpoint{3.696000in}{3.696000in}}%
\pgfusepath{clip}%
\pgfsetrectcap%
\pgfsetroundjoin%
\pgfsetlinewidth{1.505625pt}%
\definecolor{currentstroke}{rgb}{1.000000,0.000000,0.000000}%
\pgfsetstrokecolor{currentstroke}%
\pgfsetdash{}{0pt}%
\pgfpathmoveto{\pgfqpoint{3.025760in}{2.193511in}}%
\pgfpathlineto{\pgfqpoint{2.924411in}{1.390033in}}%
\pgfusepath{stroke}%
\end{pgfscope}%
\begin{pgfscope}%
\pgfpathrectangle{\pgfqpoint{0.100000in}{0.212622in}}{\pgfqpoint{3.696000in}{3.696000in}}%
\pgfusepath{clip}%
\pgfsetrectcap%
\pgfsetroundjoin%
\pgfsetlinewidth{1.505625pt}%
\definecolor{currentstroke}{rgb}{1.000000,0.000000,0.000000}%
\pgfsetstrokecolor{currentstroke}%
\pgfsetdash{}{0pt}%
\pgfpathmoveto{\pgfqpoint{3.025020in}{2.193214in}}%
\pgfpathlineto{\pgfqpoint{2.924411in}{1.390033in}}%
\pgfusepath{stroke}%
\end{pgfscope}%
\begin{pgfscope}%
\pgfpathrectangle{\pgfqpoint{0.100000in}{0.212622in}}{\pgfqpoint{3.696000in}{3.696000in}}%
\pgfusepath{clip}%
\pgfsetrectcap%
\pgfsetroundjoin%
\pgfsetlinewidth{1.505625pt}%
\definecolor{currentstroke}{rgb}{1.000000,0.000000,0.000000}%
\pgfsetstrokecolor{currentstroke}%
\pgfsetdash{}{0pt}%
\pgfpathmoveto{\pgfqpoint{3.024566in}{2.192988in}}%
\pgfpathlineto{\pgfqpoint{2.924411in}{1.390033in}}%
\pgfusepath{stroke}%
\end{pgfscope}%
\begin{pgfscope}%
\pgfpathrectangle{\pgfqpoint{0.100000in}{0.212622in}}{\pgfqpoint{3.696000in}{3.696000in}}%
\pgfusepath{clip}%
\pgfsetrectcap%
\pgfsetroundjoin%
\pgfsetlinewidth{1.505625pt}%
\definecolor{currentstroke}{rgb}{1.000000,0.000000,0.000000}%
\pgfsetstrokecolor{currentstroke}%
\pgfsetdash{}{0pt}%
\pgfpathmoveto{\pgfqpoint{3.023317in}{2.192693in}}%
\pgfpathlineto{\pgfqpoint{2.924411in}{1.390033in}}%
\pgfusepath{stroke}%
\end{pgfscope}%
\begin{pgfscope}%
\pgfpathrectangle{\pgfqpoint{0.100000in}{0.212622in}}{\pgfqpoint{3.696000in}{3.696000in}}%
\pgfusepath{clip}%
\pgfsetrectcap%
\pgfsetroundjoin%
\pgfsetlinewidth{1.505625pt}%
\definecolor{currentstroke}{rgb}{1.000000,0.000000,0.000000}%
\pgfsetstrokecolor{currentstroke}%
\pgfsetdash{}{0pt}%
\pgfpathmoveto{\pgfqpoint{3.022676in}{2.192581in}}%
\pgfpathlineto{\pgfqpoint{2.916807in}{1.382513in}}%
\pgfusepath{stroke}%
\end{pgfscope}%
\begin{pgfscope}%
\pgfpathrectangle{\pgfqpoint{0.100000in}{0.212622in}}{\pgfqpoint{3.696000in}{3.696000in}}%
\pgfusepath{clip}%
\pgfsetrectcap%
\pgfsetroundjoin%
\pgfsetlinewidth{1.505625pt}%
\definecolor{currentstroke}{rgb}{1.000000,0.000000,0.000000}%
\pgfsetstrokecolor{currentstroke}%
\pgfsetdash{}{0pt}%
\pgfpathmoveto{\pgfqpoint{3.022311in}{2.192458in}}%
\pgfpathlineto{\pgfqpoint{2.916807in}{1.382513in}}%
\pgfusepath{stroke}%
\end{pgfscope}%
\begin{pgfscope}%
\pgfpathrectangle{\pgfqpoint{0.100000in}{0.212622in}}{\pgfqpoint{3.696000in}{3.696000in}}%
\pgfusepath{clip}%
\pgfsetrectcap%
\pgfsetroundjoin%
\pgfsetlinewidth{1.505625pt}%
\definecolor{currentstroke}{rgb}{1.000000,0.000000,0.000000}%
\pgfsetstrokecolor{currentstroke}%
\pgfsetdash{}{0pt}%
\pgfpathmoveto{\pgfqpoint{3.021071in}{2.191915in}}%
\pgfpathlineto{\pgfqpoint{2.916807in}{1.382513in}}%
\pgfusepath{stroke}%
\end{pgfscope}%
\begin{pgfscope}%
\pgfpathrectangle{\pgfqpoint{0.100000in}{0.212622in}}{\pgfqpoint{3.696000in}{3.696000in}}%
\pgfusepath{clip}%
\pgfsetrectcap%
\pgfsetroundjoin%
\pgfsetlinewidth{1.505625pt}%
\definecolor{currentstroke}{rgb}{1.000000,0.000000,0.000000}%
\pgfsetstrokecolor{currentstroke}%
\pgfsetdash{}{0pt}%
\pgfpathmoveto{\pgfqpoint{3.019575in}{2.191215in}}%
\pgfpathlineto{\pgfqpoint{2.916807in}{1.382513in}}%
\pgfusepath{stroke}%
\end{pgfscope}%
\begin{pgfscope}%
\pgfpathrectangle{\pgfqpoint{0.100000in}{0.212622in}}{\pgfqpoint{3.696000in}{3.696000in}}%
\pgfusepath{clip}%
\pgfsetrectcap%
\pgfsetroundjoin%
\pgfsetlinewidth{1.505625pt}%
\definecolor{currentstroke}{rgb}{1.000000,0.000000,0.000000}%
\pgfsetstrokecolor{currentstroke}%
\pgfsetdash{}{0pt}%
\pgfpathmoveto{\pgfqpoint{3.017242in}{2.190284in}}%
\pgfpathlineto{\pgfqpoint{2.916807in}{1.382513in}}%
\pgfusepath{stroke}%
\end{pgfscope}%
\begin{pgfscope}%
\pgfpathrectangle{\pgfqpoint{0.100000in}{0.212622in}}{\pgfqpoint{3.696000in}{3.696000in}}%
\pgfusepath{clip}%
\pgfsetrectcap%
\pgfsetroundjoin%
\pgfsetlinewidth{1.505625pt}%
\definecolor{currentstroke}{rgb}{1.000000,0.000000,0.000000}%
\pgfsetstrokecolor{currentstroke}%
\pgfsetdash{}{0pt}%
\pgfpathmoveto{\pgfqpoint{3.014318in}{2.188929in}}%
\pgfpathlineto{\pgfqpoint{2.909194in}{1.374984in}}%
\pgfusepath{stroke}%
\end{pgfscope}%
\begin{pgfscope}%
\pgfpathrectangle{\pgfqpoint{0.100000in}{0.212622in}}{\pgfqpoint{3.696000in}{3.696000in}}%
\pgfusepath{clip}%
\pgfsetrectcap%
\pgfsetroundjoin%
\pgfsetlinewidth{1.505625pt}%
\definecolor{currentstroke}{rgb}{1.000000,0.000000,0.000000}%
\pgfsetstrokecolor{currentstroke}%
\pgfsetdash{}{0pt}%
\pgfpathmoveto{\pgfqpoint{3.012810in}{2.188264in}}%
\pgfpathlineto{\pgfqpoint{2.909194in}{1.374984in}}%
\pgfusepath{stroke}%
\end{pgfscope}%
\begin{pgfscope}%
\pgfpathrectangle{\pgfqpoint{0.100000in}{0.212622in}}{\pgfqpoint{3.696000in}{3.696000in}}%
\pgfusepath{clip}%
\pgfsetrectcap%
\pgfsetroundjoin%
\pgfsetlinewidth{1.505625pt}%
\definecolor{currentstroke}{rgb}{1.000000,0.000000,0.000000}%
\pgfsetstrokecolor{currentstroke}%
\pgfsetdash{}{0pt}%
\pgfpathmoveto{\pgfqpoint{3.010908in}{2.187315in}}%
\pgfpathlineto{\pgfqpoint{2.909194in}{1.374984in}}%
\pgfusepath{stroke}%
\end{pgfscope}%
\begin{pgfscope}%
\pgfpathrectangle{\pgfqpoint{0.100000in}{0.212622in}}{\pgfqpoint{3.696000in}{3.696000in}}%
\pgfusepath{clip}%
\pgfsetrectcap%
\pgfsetroundjoin%
\pgfsetlinewidth{1.505625pt}%
\definecolor{currentstroke}{rgb}{1.000000,0.000000,0.000000}%
\pgfsetstrokecolor{currentstroke}%
\pgfsetdash{}{0pt}%
\pgfpathmoveto{\pgfqpoint{3.008782in}{2.186337in}}%
\pgfpathlineto{\pgfqpoint{2.901571in}{1.367446in}}%
\pgfusepath{stroke}%
\end{pgfscope}%
\begin{pgfscope}%
\pgfpathrectangle{\pgfqpoint{0.100000in}{0.212622in}}{\pgfqpoint{3.696000in}{3.696000in}}%
\pgfusepath{clip}%
\pgfsetrectcap%
\pgfsetroundjoin%
\pgfsetlinewidth{1.505625pt}%
\definecolor{currentstroke}{rgb}{1.000000,0.000000,0.000000}%
\pgfsetstrokecolor{currentstroke}%
\pgfsetdash{}{0pt}%
\pgfpathmoveto{\pgfqpoint{3.006500in}{2.185394in}}%
\pgfpathlineto{\pgfqpoint{2.901571in}{1.367446in}}%
\pgfusepath{stroke}%
\end{pgfscope}%
\begin{pgfscope}%
\pgfpathrectangle{\pgfqpoint{0.100000in}{0.212622in}}{\pgfqpoint{3.696000in}{3.696000in}}%
\pgfusepath{clip}%
\pgfsetrectcap%
\pgfsetroundjoin%
\pgfsetlinewidth{1.505625pt}%
\definecolor{currentstroke}{rgb}{1.000000,0.000000,0.000000}%
\pgfsetstrokecolor{currentstroke}%
\pgfsetdash{}{0pt}%
\pgfpathmoveto{\pgfqpoint{3.005156in}{2.184776in}}%
\pgfpathlineto{\pgfqpoint{2.901571in}{1.367446in}}%
\pgfusepath{stroke}%
\end{pgfscope}%
\begin{pgfscope}%
\pgfpathrectangle{\pgfqpoint{0.100000in}{0.212622in}}{\pgfqpoint{3.696000in}{3.696000in}}%
\pgfusepath{clip}%
\pgfsetrectcap%
\pgfsetroundjoin%
\pgfsetlinewidth{1.505625pt}%
\definecolor{currentstroke}{rgb}{1.000000,0.000000,0.000000}%
\pgfsetstrokecolor{currentstroke}%
\pgfsetdash{}{0pt}%
\pgfpathmoveto{\pgfqpoint{3.003266in}{2.183785in}}%
\pgfpathlineto{\pgfqpoint{2.893939in}{1.359899in}}%
\pgfusepath{stroke}%
\end{pgfscope}%
\begin{pgfscope}%
\pgfpathrectangle{\pgfqpoint{0.100000in}{0.212622in}}{\pgfqpoint{3.696000in}{3.696000in}}%
\pgfusepath{clip}%
\pgfsetrectcap%
\pgfsetroundjoin%
\pgfsetlinewidth{1.505625pt}%
\definecolor{currentstroke}{rgb}{1.000000,0.000000,0.000000}%
\pgfsetstrokecolor{currentstroke}%
\pgfsetdash{}{0pt}%
\pgfpathmoveto{\pgfqpoint{3.002256in}{2.183479in}}%
\pgfpathlineto{\pgfqpoint{2.893939in}{1.359899in}}%
\pgfusepath{stroke}%
\end{pgfscope}%
\begin{pgfscope}%
\pgfpathrectangle{\pgfqpoint{0.100000in}{0.212622in}}{\pgfqpoint{3.696000in}{3.696000in}}%
\pgfusepath{clip}%
\pgfsetrectcap%
\pgfsetroundjoin%
\pgfsetlinewidth{1.505625pt}%
\definecolor{currentstroke}{rgb}{1.000000,0.000000,0.000000}%
\pgfsetstrokecolor{currentstroke}%
\pgfsetdash{}{0pt}%
\pgfpathmoveto{\pgfqpoint{3.000857in}{2.182866in}}%
\pgfpathlineto{\pgfqpoint{2.893939in}{1.359899in}}%
\pgfusepath{stroke}%
\end{pgfscope}%
\begin{pgfscope}%
\pgfpathrectangle{\pgfqpoint{0.100000in}{0.212622in}}{\pgfqpoint{3.696000in}{3.696000in}}%
\pgfusepath{clip}%
\pgfsetrectcap%
\pgfsetroundjoin%
\pgfsetlinewidth{1.505625pt}%
\definecolor{currentstroke}{rgb}{1.000000,0.000000,0.000000}%
\pgfsetstrokecolor{currentstroke}%
\pgfsetdash{}{0pt}%
\pgfpathmoveto{\pgfqpoint{2.999197in}{2.182052in}}%
\pgfpathlineto{\pgfqpoint{2.893939in}{1.359899in}}%
\pgfusepath{stroke}%
\end{pgfscope}%
\begin{pgfscope}%
\pgfpathrectangle{\pgfqpoint{0.100000in}{0.212622in}}{\pgfqpoint{3.696000in}{3.696000in}}%
\pgfusepath{clip}%
\pgfsetrectcap%
\pgfsetroundjoin%
\pgfsetlinewidth{1.505625pt}%
\definecolor{currentstroke}{rgb}{1.000000,0.000000,0.000000}%
\pgfsetstrokecolor{currentstroke}%
\pgfsetdash{}{0pt}%
\pgfpathmoveto{\pgfqpoint{2.998236in}{2.181587in}}%
\pgfpathlineto{\pgfqpoint{2.893939in}{1.359899in}}%
\pgfusepath{stroke}%
\end{pgfscope}%
\begin{pgfscope}%
\pgfpathrectangle{\pgfqpoint{0.100000in}{0.212622in}}{\pgfqpoint{3.696000in}{3.696000in}}%
\pgfusepath{clip}%
\pgfsetrectcap%
\pgfsetroundjoin%
\pgfsetlinewidth{1.505625pt}%
\definecolor{currentstroke}{rgb}{1.000000,0.000000,0.000000}%
\pgfsetstrokecolor{currentstroke}%
\pgfsetdash{}{0pt}%
\pgfpathmoveto{\pgfqpoint{2.997668in}{2.181295in}}%
\pgfpathlineto{\pgfqpoint{2.893939in}{1.359899in}}%
\pgfusepath{stroke}%
\end{pgfscope}%
\begin{pgfscope}%
\pgfpathrectangle{\pgfqpoint{0.100000in}{0.212622in}}{\pgfqpoint{3.696000in}{3.696000in}}%
\pgfusepath{clip}%
\pgfsetrectcap%
\pgfsetroundjoin%
\pgfsetlinewidth{1.505625pt}%
\definecolor{currentstroke}{rgb}{1.000000,0.000000,0.000000}%
\pgfsetstrokecolor{currentstroke}%
\pgfsetdash{}{0pt}%
\pgfpathmoveto{\pgfqpoint{2.997366in}{2.181109in}}%
\pgfpathlineto{\pgfqpoint{2.893939in}{1.359899in}}%
\pgfusepath{stroke}%
\end{pgfscope}%
\begin{pgfscope}%
\pgfpathrectangle{\pgfqpoint{0.100000in}{0.212622in}}{\pgfqpoint{3.696000in}{3.696000in}}%
\pgfusepath{clip}%
\pgfsetrectcap%
\pgfsetroundjoin%
\pgfsetlinewidth{1.505625pt}%
\definecolor{currentstroke}{rgb}{1.000000,0.000000,0.000000}%
\pgfsetstrokecolor{currentstroke}%
\pgfsetdash{}{0pt}%
\pgfpathmoveto{\pgfqpoint{2.996563in}{2.180709in}}%
\pgfpathlineto{\pgfqpoint{2.886298in}{1.352342in}}%
\pgfusepath{stroke}%
\end{pgfscope}%
\begin{pgfscope}%
\pgfpathrectangle{\pgfqpoint{0.100000in}{0.212622in}}{\pgfqpoint{3.696000in}{3.696000in}}%
\pgfusepath{clip}%
\pgfsetrectcap%
\pgfsetroundjoin%
\pgfsetlinewidth{1.505625pt}%
\definecolor{currentstroke}{rgb}{1.000000,0.000000,0.000000}%
\pgfsetstrokecolor{currentstroke}%
\pgfsetdash{}{0pt}%
\pgfpathmoveto{\pgfqpoint{2.996101in}{2.180492in}}%
\pgfpathlineto{\pgfqpoint{2.886298in}{1.352342in}}%
\pgfusepath{stroke}%
\end{pgfscope}%
\begin{pgfscope}%
\pgfpathrectangle{\pgfqpoint{0.100000in}{0.212622in}}{\pgfqpoint{3.696000in}{3.696000in}}%
\pgfusepath{clip}%
\pgfsetrectcap%
\pgfsetroundjoin%
\pgfsetlinewidth{1.505625pt}%
\definecolor{currentstroke}{rgb}{1.000000,0.000000,0.000000}%
\pgfsetstrokecolor{currentstroke}%
\pgfsetdash{}{0pt}%
\pgfpathmoveto{\pgfqpoint{2.995440in}{2.180138in}}%
\pgfpathlineto{\pgfqpoint{2.886298in}{1.352342in}}%
\pgfusepath{stroke}%
\end{pgfscope}%
\begin{pgfscope}%
\pgfpathrectangle{\pgfqpoint{0.100000in}{0.212622in}}{\pgfqpoint{3.696000in}{3.696000in}}%
\pgfusepath{clip}%
\pgfsetrectcap%
\pgfsetroundjoin%
\pgfsetlinewidth{1.505625pt}%
\definecolor{currentstroke}{rgb}{1.000000,0.000000,0.000000}%
\pgfsetstrokecolor{currentstroke}%
\pgfsetdash{}{0pt}%
\pgfpathmoveto{\pgfqpoint{2.994382in}{2.179774in}}%
\pgfpathlineto{\pgfqpoint{2.886298in}{1.352342in}}%
\pgfusepath{stroke}%
\end{pgfscope}%
\begin{pgfscope}%
\pgfpathrectangle{\pgfqpoint{0.100000in}{0.212622in}}{\pgfqpoint{3.696000in}{3.696000in}}%
\pgfusepath{clip}%
\pgfsetrectcap%
\pgfsetroundjoin%
\pgfsetlinewidth{1.505625pt}%
\definecolor{currentstroke}{rgb}{1.000000,0.000000,0.000000}%
\pgfsetstrokecolor{currentstroke}%
\pgfsetdash{}{0pt}%
\pgfpathmoveto{\pgfqpoint{2.993808in}{2.179547in}}%
\pgfpathlineto{\pgfqpoint{2.886298in}{1.352342in}}%
\pgfusepath{stroke}%
\end{pgfscope}%
\begin{pgfscope}%
\pgfpathrectangle{\pgfqpoint{0.100000in}{0.212622in}}{\pgfqpoint{3.696000in}{3.696000in}}%
\pgfusepath{clip}%
\pgfsetrectcap%
\pgfsetroundjoin%
\pgfsetlinewidth{1.505625pt}%
\definecolor{currentstroke}{rgb}{1.000000,0.000000,0.000000}%
\pgfsetstrokecolor{currentstroke}%
\pgfsetdash{}{0pt}%
\pgfpathmoveto{\pgfqpoint{2.993508in}{2.179432in}}%
\pgfpathlineto{\pgfqpoint{2.886298in}{1.352342in}}%
\pgfusepath{stroke}%
\end{pgfscope}%
\begin{pgfscope}%
\pgfpathrectangle{\pgfqpoint{0.100000in}{0.212622in}}{\pgfqpoint{3.696000in}{3.696000in}}%
\pgfusepath{clip}%
\pgfsetrectcap%
\pgfsetroundjoin%
\pgfsetlinewidth{1.505625pt}%
\definecolor{currentstroke}{rgb}{1.000000,0.000000,0.000000}%
\pgfsetstrokecolor{currentstroke}%
\pgfsetdash{}{0pt}%
\pgfpathmoveto{\pgfqpoint{2.993328in}{2.179353in}}%
\pgfpathlineto{\pgfqpoint{2.886298in}{1.352342in}}%
\pgfusepath{stroke}%
\end{pgfscope}%
\begin{pgfscope}%
\pgfpathrectangle{\pgfqpoint{0.100000in}{0.212622in}}{\pgfqpoint{3.696000in}{3.696000in}}%
\pgfusepath{clip}%
\pgfsetrectcap%
\pgfsetroundjoin%
\pgfsetlinewidth{1.505625pt}%
\definecolor{currentstroke}{rgb}{1.000000,0.000000,0.000000}%
\pgfsetstrokecolor{currentstroke}%
\pgfsetdash{}{0pt}%
\pgfpathmoveto{\pgfqpoint{2.992719in}{2.179220in}}%
\pgfpathlineto{\pgfqpoint{2.886298in}{1.352342in}}%
\pgfusepath{stroke}%
\end{pgfscope}%
\begin{pgfscope}%
\pgfpathrectangle{\pgfqpoint{0.100000in}{0.212622in}}{\pgfqpoint{3.696000in}{3.696000in}}%
\pgfusepath{clip}%
\pgfsetrectcap%
\pgfsetroundjoin%
\pgfsetlinewidth{1.505625pt}%
\definecolor{currentstroke}{rgb}{1.000000,0.000000,0.000000}%
\pgfsetstrokecolor{currentstroke}%
\pgfsetdash{}{0pt}%
\pgfpathmoveto{\pgfqpoint{2.991875in}{2.179053in}}%
\pgfpathlineto{\pgfqpoint{2.886298in}{1.352342in}}%
\pgfusepath{stroke}%
\end{pgfscope}%
\begin{pgfscope}%
\pgfpathrectangle{\pgfqpoint{0.100000in}{0.212622in}}{\pgfqpoint{3.696000in}{3.696000in}}%
\pgfusepath{clip}%
\pgfsetrectcap%
\pgfsetroundjoin%
\pgfsetlinewidth{1.505625pt}%
\definecolor{currentstroke}{rgb}{1.000000,0.000000,0.000000}%
\pgfsetstrokecolor{currentstroke}%
\pgfsetdash{}{0pt}%
\pgfpathmoveto{\pgfqpoint{2.990746in}{2.178689in}}%
\pgfpathlineto{\pgfqpoint{2.886298in}{1.352342in}}%
\pgfusepath{stroke}%
\end{pgfscope}%
\begin{pgfscope}%
\pgfpathrectangle{\pgfqpoint{0.100000in}{0.212622in}}{\pgfqpoint{3.696000in}{3.696000in}}%
\pgfusepath{clip}%
\pgfsetrectcap%
\pgfsetroundjoin%
\pgfsetlinewidth{1.505625pt}%
\definecolor{currentstroke}{rgb}{1.000000,0.000000,0.000000}%
\pgfsetstrokecolor{currentstroke}%
\pgfsetdash{}{0pt}%
\pgfpathmoveto{\pgfqpoint{2.989163in}{2.177722in}}%
\pgfpathlineto{\pgfqpoint{2.878647in}{1.344776in}}%
\pgfusepath{stroke}%
\end{pgfscope}%
\begin{pgfscope}%
\pgfpathrectangle{\pgfqpoint{0.100000in}{0.212622in}}{\pgfqpoint{3.696000in}{3.696000in}}%
\pgfusepath{clip}%
\pgfsetrectcap%
\pgfsetroundjoin%
\pgfsetlinewidth{1.505625pt}%
\definecolor{currentstroke}{rgb}{1.000000,0.000000,0.000000}%
\pgfsetstrokecolor{currentstroke}%
\pgfsetdash{}{0pt}%
\pgfpathmoveto{\pgfqpoint{2.987238in}{2.176707in}}%
\pgfpathlineto{\pgfqpoint{2.878647in}{1.344776in}}%
\pgfusepath{stroke}%
\end{pgfscope}%
\begin{pgfscope}%
\pgfpathrectangle{\pgfqpoint{0.100000in}{0.212622in}}{\pgfqpoint{3.696000in}{3.696000in}}%
\pgfusepath{clip}%
\pgfsetrectcap%
\pgfsetroundjoin%
\pgfsetlinewidth{1.505625pt}%
\definecolor{currentstroke}{rgb}{1.000000,0.000000,0.000000}%
\pgfsetstrokecolor{currentstroke}%
\pgfsetdash{}{0pt}%
\pgfpathmoveto{\pgfqpoint{2.984894in}{2.175337in}}%
\pgfpathlineto{\pgfqpoint{2.878647in}{1.344776in}}%
\pgfusepath{stroke}%
\end{pgfscope}%
\begin{pgfscope}%
\pgfpathrectangle{\pgfqpoint{0.100000in}{0.212622in}}{\pgfqpoint{3.696000in}{3.696000in}}%
\pgfusepath{clip}%
\pgfsetrectcap%
\pgfsetroundjoin%
\pgfsetlinewidth{1.505625pt}%
\definecolor{currentstroke}{rgb}{1.000000,0.000000,0.000000}%
\pgfsetstrokecolor{currentstroke}%
\pgfsetdash{}{0pt}%
\pgfpathmoveto{\pgfqpoint{2.982260in}{2.173321in}}%
\pgfpathlineto{\pgfqpoint{2.870987in}{1.337201in}}%
\pgfusepath{stroke}%
\end{pgfscope}%
\begin{pgfscope}%
\pgfpathrectangle{\pgfqpoint{0.100000in}{0.212622in}}{\pgfqpoint{3.696000in}{3.696000in}}%
\pgfusepath{clip}%
\pgfsetrectcap%
\pgfsetroundjoin%
\pgfsetlinewidth{1.505625pt}%
\definecolor{currentstroke}{rgb}{1.000000,0.000000,0.000000}%
\pgfsetstrokecolor{currentstroke}%
\pgfsetdash{}{0pt}%
\pgfpathmoveto{\pgfqpoint{2.980826in}{2.172657in}}%
\pgfpathlineto{\pgfqpoint{2.870987in}{1.337201in}}%
\pgfusepath{stroke}%
\end{pgfscope}%
\begin{pgfscope}%
\pgfpathrectangle{\pgfqpoint{0.100000in}{0.212622in}}{\pgfqpoint{3.696000in}{3.696000in}}%
\pgfusepath{clip}%
\pgfsetrectcap%
\pgfsetroundjoin%
\pgfsetlinewidth{1.505625pt}%
\definecolor{currentstroke}{rgb}{1.000000,0.000000,0.000000}%
\pgfsetstrokecolor{currentstroke}%
\pgfsetdash{}{0pt}%
\pgfpathmoveto{\pgfqpoint{2.979971in}{2.172224in}}%
\pgfpathlineto{\pgfqpoint{2.870987in}{1.337201in}}%
\pgfusepath{stroke}%
\end{pgfscope}%
\begin{pgfscope}%
\pgfpathrectangle{\pgfqpoint{0.100000in}{0.212622in}}{\pgfqpoint{3.696000in}{3.696000in}}%
\pgfusepath{clip}%
\pgfsetrectcap%
\pgfsetroundjoin%
\pgfsetlinewidth{1.505625pt}%
\definecolor{currentstroke}{rgb}{1.000000,0.000000,0.000000}%
\pgfsetstrokecolor{currentstroke}%
\pgfsetdash{}{0pt}%
\pgfpathmoveto{\pgfqpoint{2.979529in}{2.171969in}}%
\pgfpathlineto{\pgfqpoint{2.870987in}{1.337201in}}%
\pgfusepath{stroke}%
\end{pgfscope}%
\begin{pgfscope}%
\pgfpathrectangle{\pgfqpoint{0.100000in}{0.212622in}}{\pgfqpoint{3.696000in}{3.696000in}}%
\pgfusepath{clip}%
\pgfsetrectcap%
\pgfsetroundjoin%
\pgfsetlinewidth{1.505625pt}%
\definecolor{currentstroke}{rgb}{1.000000,0.000000,0.000000}%
\pgfsetstrokecolor{currentstroke}%
\pgfsetdash{}{0pt}%
\pgfpathmoveto{\pgfqpoint{2.978516in}{2.171419in}}%
\pgfpathlineto{\pgfqpoint{2.870987in}{1.337201in}}%
\pgfusepath{stroke}%
\end{pgfscope}%
\begin{pgfscope}%
\pgfpathrectangle{\pgfqpoint{0.100000in}{0.212622in}}{\pgfqpoint{3.696000in}{3.696000in}}%
\pgfusepath{clip}%
\pgfsetrectcap%
\pgfsetroundjoin%
\pgfsetlinewidth{1.505625pt}%
\definecolor{currentstroke}{rgb}{1.000000,0.000000,0.000000}%
\pgfsetstrokecolor{currentstroke}%
\pgfsetdash{}{0pt}%
\pgfpathmoveto{\pgfqpoint{2.977936in}{2.171100in}}%
\pgfpathlineto{\pgfqpoint{2.870987in}{1.337201in}}%
\pgfusepath{stroke}%
\end{pgfscope}%
\begin{pgfscope}%
\pgfpathrectangle{\pgfqpoint{0.100000in}{0.212622in}}{\pgfqpoint{3.696000in}{3.696000in}}%
\pgfusepath{clip}%
\pgfsetrectcap%
\pgfsetroundjoin%
\pgfsetlinewidth{1.505625pt}%
\definecolor{currentstroke}{rgb}{1.000000,0.000000,0.000000}%
\pgfsetstrokecolor{currentstroke}%
\pgfsetdash{}{0pt}%
\pgfpathmoveto{\pgfqpoint{2.977646in}{2.170933in}}%
\pgfpathlineto{\pgfqpoint{2.870987in}{1.337201in}}%
\pgfusepath{stroke}%
\end{pgfscope}%
\begin{pgfscope}%
\pgfpathrectangle{\pgfqpoint{0.100000in}{0.212622in}}{\pgfqpoint{3.696000in}{3.696000in}}%
\pgfusepath{clip}%
\pgfsetrectcap%
\pgfsetroundjoin%
\pgfsetlinewidth{1.505625pt}%
\definecolor{currentstroke}{rgb}{1.000000,0.000000,0.000000}%
\pgfsetstrokecolor{currentstroke}%
\pgfsetdash{}{0pt}%
\pgfpathmoveto{\pgfqpoint{2.976745in}{2.170354in}}%
\pgfpathlineto{\pgfqpoint{2.863318in}{1.329617in}}%
\pgfusepath{stroke}%
\end{pgfscope}%
\begin{pgfscope}%
\pgfpathrectangle{\pgfqpoint{0.100000in}{0.212622in}}{\pgfqpoint{3.696000in}{3.696000in}}%
\pgfusepath{clip}%
\pgfsetrectcap%
\pgfsetroundjoin%
\pgfsetlinewidth{1.505625pt}%
\definecolor{currentstroke}{rgb}{1.000000,0.000000,0.000000}%
\pgfsetstrokecolor{currentstroke}%
\pgfsetdash{}{0pt}%
\pgfpathmoveto{\pgfqpoint{2.975164in}{2.169532in}}%
\pgfpathlineto{\pgfqpoint{2.863318in}{1.329617in}}%
\pgfusepath{stroke}%
\end{pgfscope}%
\begin{pgfscope}%
\pgfpathrectangle{\pgfqpoint{0.100000in}{0.212622in}}{\pgfqpoint{3.696000in}{3.696000in}}%
\pgfusepath{clip}%
\pgfsetrectcap%
\pgfsetroundjoin%
\pgfsetlinewidth{1.505625pt}%
\definecolor{currentstroke}{rgb}{1.000000,0.000000,0.000000}%
\pgfsetstrokecolor{currentstroke}%
\pgfsetdash{}{0pt}%
\pgfpathmoveto{\pgfqpoint{2.973328in}{2.168667in}}%
\pgfpathlineto{\pgfqpoint{2.863318in}{1.329617in}}%
\pgfusepath{stroke}%
\end{pgfscope}%
\begin{pgfscope}%
\pgfpathrectangle{\pgfqpoint{0.100000in}{0.212622in}}{\pgfqpoint{3.696000in}{3.696000in}}%
\pgfusepath{clip}%
\pgfsetrectcap%
\pgfsetroundjoin%
\pgfsetlinewidth{1.505625pt}%
\definecolor{currentstroke}{rgb}{1.000000,0.000000,0.000000}%
\pgfsetstrokecolor{currentstroke}%
\pgfsetdash{}{0pt}%
\pgfpathmoveto{\pgfqpoint{2.971198in}{2.167280in}}%
\pgfpathlineto{\pgfqpoint{2.863318in}{1.329617in}}%
\pgfusepath{stroke}%
\end{pgfscope}%
\begin{pgfscope}%
\pgfpathrectangle{\pgfqpoint{0.100000in}{0.212622in}}{\pgfqpoint{3.696000in}{3.696000in}}%
\pgfusepath{clip}%
\pgfsetrectcap%
\pgfsetroundjoin%
\pgfsetlinewidth{1.505625pt}%
\definecolor{currentstroke}{rgb}{1.000000,0.000000,0.000000}%
\pgfsetstrokecolor{currentstroke}%
\pgfsetdash{}{0pt}%
\pgfpathmoveto{\pgfqpoint{2.968389in}{2.165307in}}%
\pgfpathlineto{\pgfqpoint{2.855639in}{1.322023in}}%
\pgfusepath{stroke}%
\end{pgfscope}%
\begin{pgfscope}%
\pgfpathrectangle{\pgfqpoint{0.100000in}{0.212622in}}{\pgfqpoint{3.696000in}{3.696000in}}%
\pgfusepath{clip}%
\pgfsetrectcap%
\pgfsetroundjoin%
\pgfsetlinewidth{1.505625pt}%
\definecolor{currentstroke}{rgb}{1.000000,0.000000,0.000000}%
\pgfsetstrokecolor{currentstroke}%
\pgfsetdash{}{0pt}%
\pgfpathmoveto{\pgfqpoint{2.966905in}{2.164463in}}%
\pgfpathlineto{\pgfqpoint{2.855639in}{1.322023in}}%
\pgfusepath{stroke}%
\end{pgfscope}%
\begin{pgfscope}%
\pgfpathrectangle{\pgfqpoint{0.100000in}{0.212622in}}{\pgfqpoint{3.696000in}{3.696000in}}%
\pgfusepath{clip}%
\pgfsetrectcap%
\pgfsetroundjoin%
\pgfsetlinewidth{1.505625pt}%
\definecolor{currentstroke}{rgb}{1.000000,0.000000,0.000000}%
\pgfsetstrokecolor{currentstroke}%
\pgfsetdash{}{0pt}%
\pgfpathmoveto{\pgfqpoint{2.966029in}{2.163953in}}%
\pgfpathlineto{\pgfqpoint{2.855639in}{1.322023in}}%
\pgfusepath{stroke}%
\end{pgfscope}%
\begin{pgfscope}%
\pgfpathrectangle{\pgfqpoint{0.100000in}{0.212622in}}{\pgfqpoint{3.696000in}{3.696000in}}%
\pgfusepath{clip}%
\pgfsetrectcap%
\pgfsetroundjoin%
\pgfsetlinewidth{1.505625pt}%
\definecolor{currentstroke}{rgb}{1.000000,0.000000,0.000000}%
\pgfsetstrokecolor{currentstroke}%
\pgfsetdash{}{0pt}%
\pgfpathmoveto{\pgfqpoint{2.965582in}{2.163665in}}%
\pgfpathlineto{\pgfqpoint{2.855639in}{1.322023in}}%
\pgfusepath{stroke}%
\end{pgfscope}%
\begin{pgfscope}%
\pgfpathrectangle{\pgfqpoint{0.100000in}{0.212622in}}{\pgfqpoint{3.696000in}{3.696000in}}%
\pgfusepath{clip}%
\pgfsetrectcap%
\pgfsetroundjoin%
\pgfsetlinewidth{1.505625pt}%
\definecolor{currentstroke}{rgb}{1.000000,0.000000,0.000000}%
\pgfsetstrokecolor{currentstroke}%
\pgfsetdash{}{0pt}%
\pgfpathmoveto{\pgfqpoint{2.964291in}{2.162941in}}%
\pgfpathlineto{\pgfqpoint{2.855639in}{1.322023in}}%
\pgfusepath{stroke}%
\end{pgfscope}%
\begin{pgfscope}%
\pgfpathrectangle{\pgfqpoint{0.100000in}{0.212622in}}{\pgfqpoint{3.696000in}{3.696000in}}%
\pgfusepath{clip}%
\pgfsetrectcap%
\pgfsetroundjoin%
\pgfsetlinewidth{1.505625pt}%
\definecolor{currentstroke}{rgb}{1.000000,0.000000,0.000000}%
\pgfsetstrokecolor{currentstroke}%
\pgfsetdash{}{0pt}%
\pgfpathmoveto{\pgfqpoint{2.963569in}{2.162533in}}%
\pgfpathlineto{\pgfqpoint{2.847950in}{1.314419in}}%
\pgfusepath{stroke}%
\end{pgfscope}%
\begin{pgfscope}%
\pgfpathrectangle{\pgfqpoint{0.100000in}{0.212622in}}{\pgfqpoint{3.696000in}{3.696000in}}%
\pgfusepath{clip}%
\pgfsetrectcap%
\pgfsetroundjoin%
\pgfsetlinewidth{1.505625pt}%
\definecolor{currentstroke}{rgb}{1.000000,0.000000,0.000000}%
\pgfsetstrokecolor{currentstroke}%
\pgfsetdash{}{0pt}%
\pgfpathmoveto{\pgfqpoint{2.962622in}{2.161977in}}%
\pgfpathlineto{\pgfqpoint{2.847950in}{1.314419in}}%
\pgfusepath{stroke}%
\end{pgfscope}%
\begin{pgfscope}%
\pgfpathrectangle{\pgfqpoint{0.100000in}{0.212622in}}{\pgfqpoint{3.696000in}{3.696000in}}%
\pgfusepath{clip}%
\pgfsetrectcap%
\pgfsetroundjoin%
\pgfsetlinewidth{1.505625pt}%
\definecolor{currentstroke}{rgb}{1.000000,0.000000,0.000000}%
\pgfsetstrokecolor{currentstroke}%
\pgfsetdash{}{0pt}%
\pgfpathmoveto{\pgfqpoint{2.962053in}{2.161688in}}%
\pgfpathlineto{\pgfqpoint{2.847950in}{1.314419in}}%
\pgfusepath{stroke}%
\end{pgfscope}%
\begin{pgfscope}%
\pgfpathrectangle{\pgfqpoint{0.100000in}{0.212622in}}{\pgfqpoint{3.696000in}{3.696000in}}%
\pgfusepath{clip}%
\pgfsetrectcap%
\pgfsetroundjoin%
\pgfsetlinewidth{1.505625pt}%
\definecolor{currentstroke}{rgb}{1.000000,0.000000,0.000000}%
\pgfsetstrokecolor{currentstroke}%
\pgfsetdash{}{0pt}%
\pgfpathmoveto{\pgfqpoint{2.960972in}{2.161004in}}%
\pgfpathlineto{\pgfqpoint{2.847950in}{1.314419in}}%
\pgfusepath{stroke}%
\end{pgfscope}%
\begin{pgfscope}%
\pgfpathrectangle{\pgfqpoint{0.100000in}{0.212622in}}{\pgfqpoint{3.696000in}{3.696000in}}%
\pgfusepath{clip}%
\pgfsetrectcap%
\pgfsetroundjoin%
\pgfsetlinewidth{1.505625pt}%
\definecolor{currentstroke}{rgb}{1.000000,0.000000,0.000000}%
\pgfsetstrokecolor{currentstroke}%
\pgfsetdash{}{0pt}%
\pgfpathmoveto{\pgfqpoint{2.959675in}{2.160406in}}%
\pgfpathlineto{\pgfqpoint{2.847950in}{1.314419in}}%
\pgfusepath{stroke}%
\end{pgfscope}%
\begin{pgfscope}%
\pgfpathrectangle{\pgfqpoint{0.100000in}{0.212622in}}{\pgfqpoint{3.696000in}{3.696000in}}%
\pgfusepath{clip}%
\pgfsetrectcap%
\pgfsetroundjoin%
\pgfsetlinewidth{1.505625pt}%
\definecolor{currentstroke}{rgb}{1.000000,0.000000,0.000000}%
\pgfsetstrokecolor{currentstroke}%
\pgfsetdash{}{0pt}%
\pgfpathmoveto{\pgfqpoint{2.958045in}{2.159526in}}%
\pgfpathlineto{\pgfqpoint{2.847950in}{1.314419in}}%
\pgfusepath{stroke}%
\end{pgfscope}%
\begin{pgfscope}%
\pgfpathrectangle{\pgfqpoint{0.100000in}{0.212622in}}{\pgfqpoint{3.696000in}{3.696000in}}%
\pgfusepath{clip}%
\pgfsetrectcap%
\pgfsetroundjoin%
\pgfsetlinewidth{1.505625pt}%
\definecolor{currentstroke}{rgb}{1.000000,0.000000,0.000000}%
\pgfsetstrokecolor{currentstroke}%
\pgfsetdash{}{0pt}%
\pgfpathmoveto{\pgfqpoint{2.956193in}{2.158462in}}%
\pgfpathlineto{\pgfqpoint{2.840252in}{1.306807in}}%
\pgfusepath{stroke}%
\end{pgfscope}%
\begin{pgfscope}%
\pgfpathrectangle{\pgfqpoint{0.100000in}{0.212622in}}{\pgfqpoint{3.696000in}{3.696000in}}%
\pgfusepath{clip}%
\pgfsetrectcap%
\pgfsetroundjoin%
\pgfsetlinewidth{1.505625pt}%
\definecolor{currentstroke}{rgb}{1.000000,0.000000,0.000000}%
\pgfsetstrokecolor{currentstroke}%
\pgfsetdash{}{0pt}%
\pgfpathmoveto{\pgfqpoint{2.953673in}{2.157097in}}%
\pgfpathlineto{\pgfqpoint{2.840252in}{1.306807in}}%
\pgfusepath{stroke}%
\end{pgfscope}%
\begin{pgfscope}%
\pgfpathrectangle{\pgfqpoint{0.100000in}{0.212622in}}{\pgfqpoint{3.696000in}{3.696000in}}%
\pgfusepath{clip}%
\pgfsetrectcap%
\pgfsetroundjoin%
\pgfsetlinewidth{1.505625pt}%
\definecolor{currentstroke}{rgb}{1.000000,0.000000,0.000000}%
\pgfsetstrokecolor{currentstroke}%
\pgfsetdash{}{0pt}%
\pgfpathmoveto{\pgfqpoint{2.950783in}{2.155931in}}%
\pgfpathlineto{\pgfqpoint{2.840252in}{1.306807in}}%
\pgfusepath{stroke}%
\end{pgfscope}%
\begin{pgfscope}%
\pgfpathrectangle{\pgfqpoint{0.100000in}{0.212622in}}{\pgfqpoint{3.696000in}{3.696000in}}%
\pgfusepath{clip}%
\pgfsetrectcap%
\pgfsetroundjoin%
\pgfsetlinewidth{1.505625pt}%
\definecolor{currentstroke}{rgb}{1.000000,0.000000,0.000000}%
\pgfsetstrokecolor{currentstroke}%
\pgfsetdash{}{0pt}%
\pgfpathmoveto{\pgfqpoint{2.947802in}{2.154655in}}%
\pgfpathlineto{\pgfqpoint{2.832545in}{1.299185in}}%
\pgfusepath{stroke}%
\end{pgfscope}%
\begin{pgfscope}%
\pgfpathrectangle{\pgfqpoint{0.100000in}{0.212622in}}{\pgfqpoint{3.696000in}{3.696000in}}%
\pgfusepath{clip}%
\pgfsetrectcap%
\pgfsetroundjoin%
\pgfsetlinewidth{1.505625pt}%
\definecolor{currentstroke}{rgb}{1.000000,0.000000,0.000000}%
\pgfsetstrokecolor{currentstroke}%
\pgfsetdash{}{0pt}%
\pgfpathmoveto{\pgfqpoint{2.946070in}{2.153678in}}%
\pgfpathlineto{\pgfqpoint{2.832545in}{1.299185in}}%
\pgfusepath{stroke}%
\end{pgfscope}%
\begin{pgfscope}%
\pgfpathrectangle{\pgfqpoint{0.100000in}{0.212622in}}{\pgfqpoint{3.696000in}{3.696000in}}%
\pgfusepath{clip}%
\pgfsetrectcap%
\pgfsetroundjoin%
\pgfsetlinewidth{1.505625pt}%
\definecolor{currentstroke}{rgb}{1.000000,0.000000,0.000000}%
\pgfsetstrokecolor{currentstroke}%
\pgfsetdash{}{0pt}%
\pgfpathmoveto{\pgfqpoint{2.943673in}{2.152117in}}%
\pgfpathlineto{\pgfqpoint{2.824827in}{1.291553in}}%
\pgfusepath{stroke}%
\end{pgfscope}%
\begin{pgfscope}%
\pgfpathrectangle{\pgfqpoint{0.100000in}{0.212622in}}{\pgfqpoint{3.696000in}{3.696000in}}%
\pgfusepath{clip}%
\pgfsetrectcap%
\pgfsetroundjoin%
\pgfsetlinewidth{1.505625pt}%
\definecolor{currentstroke}{rgb}{1.000000,0.000000,0.000000}%
\pgfsetstrokecolor{currentstroke}%
\pgfsetdash{}{0pt}%
\pgfpathmoveto{\pgfqpoint{2.942361in}{2.151431in}}%
\pgfpathlineto{\pgfqpoint{2.824827in}{1.291553in}}%
\pgfusepath{stroke}%
\end{pgfscope}%
\begin{pgfscope}%
\pgfpathrectangle{\pgfqpoint{0.100000in}{0.212622in}}{\pgfqpoint{3.696000in}{3.696000in}}%
\pgfusepath{clip}%
\pgfsetrectcap%
\pgfsetroundjoin%
\pgfsetlinewidth{1.505625pt}%
\definecolor{currentstroke}{rgb}{1.000000,0.000000,0.000000}%
\pgfsetstrokecolor{currentstroke}%
\pgfsetdash{}{0pt}%
\pgfpathmoveto{\pgfqpoint{2.941578in}{2.150997in}}%
\pgfpathlineto{\pgfqpoint{2.824827in}{1.291553in}}%
\pgfusepath{stroke}%
\end{pgfscope}%
\begin{pgfscope}%
\pgfpathrectangle{\pgfqpoint{0.100000in}{0.212622in}}{\pgfqpoint{3.696000in}{3.696000in}}%
\pgfusepath{clip}%
\pgfsetrectcap%
\pgfsetroundjoin%
\pgfsetlinewidth{1.505625pt}%
\definecolor{currentstroke}{rgb}{1.000000,0.000000,0.000000}%
\pgfsetstrokecolor{currentstroke}%
\pgfsetdash{}{0pt}%
\pgfpathmoveto{\pgfqpoint{2.941174in}{2.150758in}}%
\pgfpathlineto{\pgfqpoint{2.824827in}{1.291553in}}%
\pgfusepath{stroke}%
\end{pgfscope}%
\begin{pgfscope}%
\pgfpathrectangle{\pgfqpoint{0.100000in}{0.212622in}}{\pgfqpoint{3.696000in}{3.696000in}}%
\pgfusepath{clip}%
\pgfsetrectcap%
\pgfsetroundjoin%
\pgfsetlinewidth{1.505625pt}%
\definecolor{currentstroke}{rgb}{1.000000,0.000000,0.000000}%
\pgfsetstrokecolor{currentstroke}%
\pgfsetdash{}{0pt}%
\pgfpathmoveto{\pgfqpoint{2.940194in}{2.150177in}}%
\pgfpathlineto{\pgfqpoint{2.824827in}{1.291553in}}%
\pgfusepath{stroke}%
\end{pgfscope}%
\begin{pgfscope}%
\pgfpathrectangle{\pgfqpoint{0.100000in}{0.212622in}}{\pgfqpoint{3.696000in}{3.696000in}}%
\pgfusepath{clip}%
\pgfsetrectcap%
\pgfsetroundjoin%
\pgfsetlinewidth{1.505625pt}%
\definecolor{currentstroke}{rgb}{1.000000,0.000000,0.000000}%
\pgfsetstrokecolor{currentstroke}%
\pgfsetdash{}{0pt}%
\pgfpathmoveto{\pgfqpoint{2.938739in}{2.149282in}}%
\pgfpathlineto{\pgfqpoint{2.824827in}{1.291553in}}%
\pgfusepath{stroke}%
\end{pgfscope}%
\begin{pgfscope}%
\pgfpathrectangle{\pgfqpoint{0.100000in}{0.212622in}}{\pgfqpoint{3.696000in}{3.696000in}}%
\pgfusepath{clip}%
\pgfsetrectcap%
\pgfsetroundjoin%
\pgfsetlinewidth{1.505625pt}%
\definecolor{currentstroke}{rgb}{1.000000,0.000000,0.000000}%
\pgfsetstrokecolor{currentstroke}%
\pgfsetdash{}{0pt}%
\pgfpathmoveto{\pgfqpoint{2.937140in}{2.148420in}}%
\pgfpathlineto{\pgfqpoint{2.817101in}{1.283912in}}%
\pgfusepath{stroke}%
\end{pgfscope}%
\begin{pgfscope}%
\pgfpathrectangle{\pgfqpoint{0.100000in}{0.212622in}}{\pgfqpoint{3.696000in}{3.696000in}}%
\pgfusepath{clip}%
\pgfsetrectcap%
\pgfsetroundjoin%
\pgfsetlinewidth{1.505625pt}%
\definecolor{currentstroke}{rgb}{1.000000,0.000000,0.000000}%
\pgfsetstrokecolor{currentstroke}%
\pgfsetdash{}{0pt}%
\pgfpathmoveto{\pgfqpoint{2.935219in}{2.147272in}}%
\pgfpathlineto{\pgfqpoint{2.817101in}{1.283912in}}%
\pgfusepath{stroke}%
\end{pgfscope}%
\begin{pgfscope}%
\pgfpathrectangle{\pgfqpoint{0.100000in}{0.212622in}}{\pgfqpoint{3.696000in}{3.696000in}}%
\pgfusepath{clip}%
\pgfsetrectcap%
\pgfsetroundjoin%
\pgfsetlinewidth{1.505625pt}%
\definecolor{currentstroke}{rgb}{1.000000,0.000000,0.000000}%
\pgfsetstrokecolor{currentstroke}%
\pgfsetdash{}{0pt}%
\pgfpathmoveto{\pgfqpoint{2.932771in}{2.145665in}}%
\pgfpathlineto{\pgfqpoint{2.817101in}{1.283912in}}%
\pgfusepath{stroke}%
\end{pgfscope}%
\begin{pgfscope}%
\pgfpathrectangle{\pgfqpoint{0.100000in}{0.212622in}}{\pgfqpoint{3.696000in}{3.696000in}}%
\pgfusepath{clip}%
\pgfsetrectcap%
\pgfsetroundjoin%
\pgfsetlinewidth{1.505625pt}%
\definecolor{currentstroke}{rgb}{1.000000,0.000000,0.000000}%
\pgfsetstrokecolor{currentstroke}%
\pgfsetdash{}{0pt}%
\pgfpathmoveto{\pgfqpoint{2.930099in}{2.144368in}}%
\pgfpathlineto{\pgfqpoint{2.809365in}{1.276262in}}%
\pgfusepath{stroke}%
\end{pgfscope}%
\begin{pgfscope}%
\pgfpathrectangle{\pgfqpoint{0.100000in}{0.212622in}}{\pgfqpoint{3.696000in}{3.696000in}}%
\pgfusepath{clip}%
\pgfsetrectcap%
\pgfsetroundjoin%
\pgfsetlinewidth{1.505625pt}%
\definecolor{currentstroke}{rgb}{1.000000,0.000000,0.000000}%
\pgfsetstrokecolor{currentstroke}%
\pgfsetdash{}{0pt}%
\pgfpathmoveto{\pgfqpoint{2.928518in}{2.143467in}}%
\pgfpathlineto{\pgfqpoint{2.809365in}{1.276262in}}%
\pgfusepath{stroke}%
\end{pgfscope}%
\begin{pgfscope}%
\pgfpathrectangle{\pgfqpoint{0.100000in}{0.212622in}}{\pgfqpoint{3.696000in}{3.696000in}}%
\pgfusepath{clip}%
\pgfsetrectcap%
\pgfsetroundjoin%
\pgfsetlinewidth{1.505625pt}%
\definecolor{currentstroke}{rgb}{1.000000,0.000000,0.000000}%
\pgfsetstrokecolor{currentstroke}%
\pgfsetdash{}{0pt}%
\pgfpathmoveto{\pgfqpoint{2.926851in}{2.142341in}}%
\pgfpathlineto{\pgfqpoint{2.809365in}{1.276262in}}%
\pgfusepath{stroke}%
\end{pgfscope}%
\begin{pgfscope}%
\pgfpathrectangle{\pgfqpoint{0.100000in}{0.212622in}}{\pgfqpoint{3.696000in}{3.696000in}}%
\pgfusepath{clip}%
\pgfsetrectcap%
\pgfsetroundjoin%
\pgfsetlinewidth{1.505625pt}%
\definecolor{currentstroke}{rgb}{1.000000,0.000000,0.000000}%
\pgfsetstrokecolor{currentstroke}%
\pgfsetdash{}{0pt}%
\pgfpathmoveto{\pgfqpoint{2.924818in}{2.140644in}}%
\pgfpathlineto{\pgfqpoint{2.809365in}{1.276262in}}%
\pgfusepath{stroke}%
\end{pgfscope}%
\begin{pgfscope}%
\pgfpathrectangle{\pgfqpoint{0.100000in}{0.212622in}}{\pgfqpoint{3.696000in}{3.696000in}}%
\pgfusepath{clip}%
\pgfsetrectcap%
\pgfsetroundjoin%
\pgfsetlinewidth{1.505625pt}%
\definecolor{currentstroke}{rgb}{1.000000,0.000000,0.000000}%
\pgfsetstrokecolor{currentstroke}%
\pgfsetdash{}{0pt}%
\pgfpathmoveto{\pgfqpoint{2.923644in}{2.139688in}}%
\pgfpathlineto{\pgfqpoint{2.801619in}{1.268602in}}%
\pgfusepath{stroke}%
\end{pgfscope}%
\begin{pgfscope}%
\pgfpathrectangle{\pgfqpoint{0.100000in}{0.212622in}}{\pgfqpoint{3.696000in}{3.696000in}}%
\pgfusepath{clip}%
\pgfsetrectcap%
\pgfsetroundjoin%
\pgfsetlinewidth{1.505625pt}%
\definecolor{currentstroke}{rgb}{1.000000,0.000000,0.000000}%
\pgfsetstrokecolor{currentstroke}%
\pgfsetdash{}{0pt}%
\pgfpathmoveto{\pgfqpoint{2.923043in}{2.139170in}}%
\pgfpathlineto{\pgfqpoint{2.801619in}{1.268602in}}%
\pgfusepath{stroke}%
\end{pgfscope}%
\begin{pgfscope}%
\pgfpathrectangle{\pgfqpoint{0.100000in}{0.212622in}}{\pgfqpoint{3.696000in}{3.696000in}}%
\pgfusepath{clip}%
\pgfsetrectcap%
\pgfsetroundjoin%
\pgfsetlinewidth{1.505625pt}%
\definecolor{currentstroke}{rgb}{1.000000,0.000000,0.000000}%
\pgfsetstrokecolor{currentstroke}%
\pgfsetdash{}{0pt}%
\pgfpathmoveto{\pgfqpoint{2.922678in}{2.138877in}}%
\pgfpathlineto{\pgfqpoint{2.801619in}{1.268602in}}%
\pgfusepath{stroke}%
\end{pgfscope}%
\begin{pgfscope}%
\pgfpathrectangle{\pgfqpoint{0.100000in}{0.212622in}}{\pgfqpoint{3.696000in}{3.696000in}}%
\pgfusepath{clip}%
\pgfsetrectcap%
\pgfsetroundjoin%
\pgfsetlinewidth{1.505625pt}%
\definecolor{currentstroke}{rgb}{1.000000,0.000000,0.000000}%
\pgfsetstrokecolor{currentstroke}%
\pgfsetdash{}{0pt}%
\pgfpathmoveto{\pgfqpoint{2.921845in}{2.138108in}}%
\pgfpathlineto{\pgfqpoint{2.801619in}{1.268602in}}%
\pgfusepath{stroke}%
\end{pgfscope}%
\begin{pgfscope}%
\pgfpathrectangle{\pgfqpoint{0.100000in}{0.212622in}}{\pgfqpoint{3.696000in}{3.696000in}}%
\pgfusepath{clip}%
\pgfsetrectcap%
\pgfsetroundjoin%
\pgfsetlinewidth{1.505625pt}%
\definecolor{currentstroke}{rgb}{1.000000,0.000000,0.000000}%
\pgfsetstrokecolor{currentstroke}%
\pgfsetdash{}{0pt}%
\pgfpathmoveto{\pgfqpoint{2.921407in}{2.137772in}}%
\pgfpathlineto{\pgfqpoint{2.801619in}{1.268602in}}%
\pgfusepath{stroke}%
\end{pgfscope}%
\begin{pgfscope}%
\pgfpathrectangle{\pgfqpoint{0.100000in}{0.212622in}}{\pgfqpoint{3.696000in}{3.696000in}}%
\pgfusepath{clip}%
\pgfsetrectcap%
\pgfsetroundjoin%
\pgfsetlinewidth{1.505625pt}%
\definecolor{currentstroke}{rgb}{1.000000,0.000000,0.000000}%
\pgfsetstrokecolor{currentstroke}%
\pgfsetdash{}{0pt}%
\pgfpathmoveto{\pgfqpoint{2.920543in}{2.137090in}}%
\pgfpathlineto{\pgfqpoint{2.801619in}{1.268602in}}%
\pgfusepath{stroke}%
\end{pgfscope}%
\begin{pgfscope}%
\pgfpathrectangle{\pgfqpoint{0.100000in}{0.212622in}}{\pgfqpoint{3.696000in}{3.696000in}}%
\pgfusepath{clip}%
\pgfsetrectcap%
\pgfsetroundjoin%
\pgfsetlinewidth{1.505625pt}%
\definecolor{currentstroke}{rgb}{1.000000,0.000000,0.000000}%
\pgfsetstrokecolor{currentstroke}%
\pgfsetdash{}{0pt}%
\pgfpathmoveto{\pgfqpoint{2.919062in}{2.135877in}}%
\pgfpathlineto{\pgfqpoint{2.801619in}{1.268602in}}%
\pgfusepath{stroke}%
\end{pgfscope}%
\begin{pgfscope}%
\pgfpathrectangle{\pgfqpoint{0.100000in}{0.212622in}}{\pgfqpoint{3.696000in}{3.696000in}}%
\pgfusepath{clip}%
\pgfsetrectcap%
\pgfsetroundjoin%
\pgfsetlinewidth{1.505625pt}%
\definecolor{currentstroke}{rgb}{1.000000,0.000000,0.000000}%
\pgfsetstrokecolor{currentstroke}%
\pgfsetdash{}{0pt}%
\pgfpathmoveto{\pgfqpoint{2.918130in}{2.135155in}}%
\pgfpathlineto{\pgfqpoint{2.793864in}{1.260933in}}%
\pgfusepath{stroke}%
\end{pgfscope}%
\begin{pgfscope}%
\pgfpathrectangle{\pgfqpoint{0.100000in}{0.212622in}}{\pgfqpoint{3.696000in}{3.696000in}}%
\pgfusepath{clip}%
\pgfsetrectcap%
\pgfsetroundjoin%
\pgfsetlinewidth{1.505625pt}%
\definecolor{currentstroke}{rgb}{1.000000,0.000000,0.000000}%
\pgfsetstrokecolor{currentstroke}%
\pgfsetdash{}{0pt}%
\pgfpathmoveto{\pgfqpoint{2.916852in}{2.134187in}}%
\pgfpathlineto{\pgfqpoint{2.793864in}{1.260933in}}%
\pgfusepath{stroke}%
\end{pgfscope}%
\begin{pgfscope}%
\pgfpathrectangle{\pgfqpoint{0.100000in}{0.212622in}}{\pgfqpoint{3.696000in}{3.696000in}}%
\pgfusepath{clip}%
\pgfsetrectcap%
\pgfsetroundjoin%
\pgfsetlinewidth{1.505625pt}%
\definecolor{currentstroke}{rgb}{1.000000,0.000000,0.000000}%
\pgfsetstrokecolor{currentstroke}%
\pgfsetdash{}{0pt}%
\pgfpathmoveto{\pgfqpoint{2.916205in}{2.133661in}}%
\pgfpathlineto{\pgfqpoint{2.793864in}{1.260933in}}%
\pgfusepath{stroke}%
\end{pgfscope}%
\begin{pgfscope}%
\pgfpathrectangle{\pgfqpoint{0.100000in}{0.212622in}}{\pgfqpoint{3.696000in}{3.696000in}}%
\pgfusepath{clip}%
\pgfsetrectcap%
\pgfsetroundjoin%
\pgfsetlinewidth{1.505625pt}%
\definecolor{currentstroke}{rgb}{1.000000,0.000000,0.000000}%
\pgfsetstrokecolor{currentstroke}%
\pgfsetdash{}{0pt}%
\pgfpathmoveto{\pgfqpoint{2.915105in}{2.132885in}}%
\pgfpathlineto{\pgfqpoint{2.793864in}{1.260933in}}%
\pgfusepath{stroke}%
\end{pgfscope}%
\begin{pgfscope}%
\pgfpathrectangle{\pgfqpoint{0.100000in}{0.212622in}}{\pgfqpoint{3.696000in}{3.696000in}}%
\pgfusepath{clip}%
\pgfsetrectcap%
\pgfsetroundjoin%
\pgfsetlinewidth{1.505625pt}%
\definecolor{currentstroke}{rgb}{1.000000,0.000000,0.000000}%
\pgfsetstrokecolor{currentstroke}%
\pgfsetdash{}{0pt}%
\pgfpathmoveto{\pgfqpoint{2.913716in}{2.131805in}}%
\pgfpathlineto{\pgfqpoint{2.793864in}{1.260933in}}%
\pgfusepath{stroke}%
\end{pgfscope}%
\begin{pgfscope}%
\pgfpathrectangle{\pgfqpoint{0.100000in}{0.212622in}}{\pgfqpoint{3.696000in}{3.696000in}}%
\pgfusepath{clip}%
\pgfsetrectcap%
\pgfsetroundjoin%
\pgfsetlinewidth{1.505625pt}%
\definecolor{currentstroke}{rgb}{1.000000,0.000000,0.000000}%
\pgfsetstrokecolor{currentstroke}%
\pgfsetdash{}{0pt}%
\pgfpathmoveto{\pgfqpoint{2.912223in}{2.130798in}}%
\pgfpathlineto{\pgfqpoint{2.786099in}{1.253254in}}%
\pgfusepath{stroke}%
\end{pgfscope}%
\begin{pgfscope}%
\pgfpathrectangle{\pgfqpoint{0.100000in}{0.212622in}}{\pgfqpoint{3.696000in}{3.696000in}}%
\pgfusepath{clip}%
\pgfsetrectcap%
\pgfsetroundjoin%
\pgfsetlinewidth{1.505625pt}%
\definecolor{currentstroke}{rgb}{1.000000,0.000000,0.000000}%
\pgfsetstrokecolor{currentstroke}%
\pgfsetdash{}{0pt}%
\pgfpathmoveto{\pgfqpoint{2.910238in}{2.129501in}}%
\pgfpathlineto{\pgfqpoint{2.786099in}{1.253254in}}%
\pgfusepath{stroke}%
\end{pgfscope}%
\begin{pgfscope}%
\pgfpathrectangle{\pgfqpoint{0.100000in}{0.212622in}}{\pgfqpoint{3.696000in}{3.696000in}}%
\pgfusepath{clip}%
\pgfsetrectcap%
\pgfsetroundjoin%
\pgfsetlinewidth{1.505625pt}%
\definecolor{currentstroke}{rgb}{1.000000,0.000000,0.000000}%
\pgfsetstrokecolor{currentstroke}%
\pgfsetdash{}{0pt}%
\pgfpathmoveto{\pgfqpoint{2.907467in}{2.127146in}}%
\pgfpathlineto{\pgfqpoint{2.786099in}{1.253254in}}%
\pgfusepath{stroke}%
\end{pgfscope}%
\begin{pgfscope}%
\pgfpathrectangle{\pgfqpoint{0.100000in}{0.212622in}}{\pgfqpoint{3.696000in}{3.696000in}}%
\pgfusepath{clip}%
\pgfsetrectcap%
\pgfsetroundjoin%
\pgfsetlinewidth{1.505625pt}%
\definecolor{currentstroke}{rgb}{1.000000,0.000000,0.000000}%
\pgfsetstrokecolor{currentstroke}%
\pgfsetdash{}{0pt}%
\pgfpathmoveto{\pgfqpoint{2.906006in}{2.126119in}}%
\pgfpathlineto{\pgfqpoint{2.778324in}{1.245565in}}%
\pgfusepath{stroke}%
\end{pgfscope}%
\begin{pgfscope}%
\pgfpathrectangle{\pgfqpoint{0.100000in}{0.212622in}}{\pgfqpoint{3.696000in}{3.696000in}}%
\pgfusepath{clip}%
\pgfsetrectcap%
\pgfsetroundjoin%
\pgfsetlinewidth{1.505625pt}%
\definecolor{currentstroke}{rgb}{1.000000,0.000000,0.000000}%
\pgfsetstrokecolor{currentstroke}%
\pgfsetdash{}{0pt}%
\pgfpathmoveto{\pgfqpoint{2.905122in}{2.125512in}}%
\pgfpathlineto{\pgfqpoint{2.778324in}{1.245565in}}%
\pgfusepath{stroke}%
\end{pgfscope}%
\begin{pgfscope}%
\pgfpathrectangle{\pgfqpoint{0.100000in}{0.212622in}}{\pgfqpoint{3.696000in}{3.696000in}}%
\pgfusepath{clip}%
\pgfsetrectcap%
\pgfsetroundjoin%
\pgfsetlinewidth{1.505625pt}%
\definecolor{currentstroke}{rgb}{1.000000,0.000000,0.000000}%
\pgfsetstrokecolor{currentstroke}%
\pgfsetdash{}{0pt}%
\pgfpathmoveto{\pgfqpoint{2.903308in}{2.124307in}}%
\pgfpathlineto{\pgfqpoint{2.778324in}{1.245565in}}%
\pgfusepath{stroke}%
\end{pgfscope}%
\begin{pgfscope}%
\pgfpathrectangle{\pgfqpoint{0.100000in}{0.212622in}}{\pgfqpoint{3.696000in}{3.696000in}}%
\pgfusepath{clip}%
\pgfsetrectcap%
\pgfsetroundjoin%
\pgfsetlinewidth{1.505625pt}%
\definecolor{currentstroke}{rgb}{1.000000,0.000000,0.000000}%
\pgfsetstrokecolor{currentstroke}%
\pgfsetdash{}{0pt}%
\pgfpathmoveto{\pgfqpoint{2.902373in}{2.123535in}}%
\pgfpathlineto{\pgfqpoint{2.778324in}{1.245565in}}%
\pgfusepath{stroke}%
\end{pgfscope}%
\begin{pgfscope}%
\pgfpathrectangle{\pgfqpoint{0.100000in}{0.212622in}}{\pgfqpoint{3.696000in}{3.696000in}}%
\pgfusepath{clip}%
\pgfsetrectcap%
\pgfsetroundjoin%
\pgfsetlinewidth{1.505625pt}%
\definecolor{currentstroke}{rgb}{1.000000,0.000000,0.000000}%
\pgfsetstrokecolor{currentstroke}%
\pgfsetdash{}{0pt}%
\pgfpathmoveto{\pgfqpoint{2.901789in}{2.123239in}}%
\pgfpathlineto{\pgfqpoint{2.778324in}{1.245565in}}%
\pgfusepath{stroke}%
\end{pgfscope}%
\begin{pgfscope}%
\pgfpathrectangle{\pgfqpoint{0.100000in}{0.212622in}}{\pgfqpoint{3.696000in}{3.696000in}}%
\pgfusepath{clip}%
\pgfsetrectcap%
\pgfsetroundjoin%
\pgfsetlinewidth{1.505625pt}%
\definecolor{currentstroke}{rgb}{1.000000,0.000000,0.000000}%
\pgfsetstrokecolor{currentstroke}%
\pgfsetdash{}{0pt}%
\pgfpathmoveto{\pgfqpoint{2.899921in}{2.122207in}}%
\pgfpathlineto{\pgfqpoint{2.770540in}{1.237867in}}%
\pgfusepath{stroke}%
\end{pgfscope}%
\begin{pgfscope}%
\pgfpathrectangle{\pgfqpoint{0.100000in}{0.212622in}}{\pgfqpoint{3.696000in}{3.696000in}}%
\pgfusepath{clip}%
\pgfsetrectcap%
\pgfsetroundjoin%
\pgfsetlinewidth{1.505625pt}%
\definecolor{currentstroke}{rgb}{1.000000,0.000000,0.000000}%
\pgfsetstrokecolor{currentstroke}%
\pgfsetdash{}{0pt}%
\pgfpathmoveto{\pgfqpoint{2.897246in}{2.120215in}}%
\pgfpathlineto{\pgfqpoint{2.770540in}{1.237867in}}%
\pgfusepath{stroke}%
\end{pgfscope}%
\begin{pgfscope}%
\pgfpathrectangle{\pgfqpoint{0.100000in}{0.212622in}}{\pgfqpoint{3.696000in}{3.696000in}}%
\pgfusepath{clip}%
\pgfsetrectcap%
\pgfsetroundjoin%
\pgfsetlinewidth{1.505625pt}%
\definecolor{currentstroke}{rgb}{1.000000,0.000000,0.000000}%
\pgfsetstrokecolor{currentstroke}%
\pgfsetdash{}{0pt}%
\pgfpathmoveto{\pgfqpoint{2.893312in}{2.117221in}}%
\pgfpathlineto{\pgfqpoint{2.770540in}{1.237867in}}%
\pgfusepath{stroke}%
\end{pgfscope}%
\begin{pgfscope}%
\pgfpathrectangle{\pgfqpoint{0.100000in}{0.212622in}}{\pgfqpoint{3.696000in}{3.696000in}}%
\pgfusepath{clip}%
\pgfsetrectcap%
\pgfsetroundjoin%
\pgfsetlinewidth{1.505625pt}%
\definecolor{currentstroke}{rgb}{1.000000,0.000000,0.000000}%
\pgfsetstrokecolor{currentstroke}%
\pgfsetdash{}{0pt}%
\pgfpathmoveto{\pgfqpoint{2.890974in}{2.115426in}}%
\pgfpathlineto{\pgfqpoint{2.762746in}{1.230160in}}%
\pgfusepath{stroke}%
\end{pgfscope}%
\begin{pgfscope}%
\pgfpathrectangle{\pgfqpoint{0.100000in}{0.212622in}}{\pgfqpoint{3.696000in}{3.696000in}}%
\pgfusepath{clip}%
\pgfsetrectcap%
\pgfsetroundjoin%
\pgfsetlinewidth{1.505625pt}%
\definecolor{currentstroke}{rgb}{1.000000,0.000000,0.000000}%
\pgfsetstrokecolor{currentstroke}%
\pgfsetdash{}{0pt}%
\pgfpathmoveto{\pgfqpoint{2.887233in}{2.113144in}}%
\pgfpathlineto{\pgfqpoint{2.762746in}{1.230160in}}%
\pgfusepath{stroke}%
\end{pgfscope}%
\begin{pgfscope}%
\pgfpathrectangle{\pgfqpoint{0.100000in}{0.212622in}}{\pgfqpoint{3.696000in}{3.696000in}}%
\pgfusepath{clip}%
\pgfsetrectcap%
\pgfsetroundjoin%
\pgfsetlinewidth{1.505625pt}%
\definecolor{currentstroke}{rgb}{1.000000,0.000000,0.000000}%
\pgfsetstrokecolor{currentstroke}%
\pgfsetdash{}{0pt}%
\pgfpathmoveto{\pgfqpoint{2.883149in}{2.110998in}}%
\pgfpathlineto{\pgfqpoint{2.754942in}{1.222443in}}%
\pgfusepath{stroke}%
\end{pgfscope}%
\begin{pgfscope}%
\pgfpathrectangle{\pgfqpoint{0.100000in}{0.212622in}}{\pgfqpoint{3.696000in}{3.696000in}}%
\pgfusepath{clip}%
\pgfsetrectcap%
\pgfsetroundjoin%
\pgfsetlinewidth{1.505625pt}%
\definecolor{currentstroke}{rgb}{1.000000,0.000000,0.000000}%
\pgfsetstrokecolor{currentstroke}%
\pgfsetdash{}{0pt}%
\pgfpathmoveto{\pgfqpoint{2.880672in}{2.109650in}}%
\pgfpathlineto{\pgfqpoint{2.754942in}{1.222443in}}%
\pgfusepath{stroke}%
\end{pgfscope}%
\begin{pgfscope}%
\pgfpathrectangle{\pgfqpoint{0.100000in}{0.212622in}}{\pgfqpoint{3.696000in}{3.696000in}}%
\pgfusepath{clip}%
\pgfsetrectcap%
\pgfsetroundjoin%
\pgfsetlinewidth{1.505625pt}%
\definecolor{currentstroke}{rgb}{1.000000,0.000000,0.000000}%
\pgfsetstrokecolor{currentstroke}%
\pgfsetdash{}{0pt}%
\pgfpathmoveto{\pgfqpoint{2.875569in}{2.105388in}}%
\pgfpathlineto{\pgfqpoint{2.747129in}{1.214716in}}%
\pgfusepath{stroke}%
\end{pgfscope}%
\begin{pgfscope}%
\pgfpathrectangle{\pgfqpoint{0.100000in}{0.212622in}}{\pgfqpoint{3.696000in}{3.696000in}}%
\pgfusepath{clip}%
\pgfsetrectcap%
\pgfsetroundjoin%
\pgfsetlinewidth{1.505625pt}%
\definecolor{currentstroke}{rgb}{1.000000,0.000000,0.000000}%
\pgfsetstrokecolor{currentstroke}%
\pgfsetdash{}{0pt}%
\pgfpathmoveto{\pgfqpoint{2.872771in}{2.103445in}}%
\pgfpathlineto{\pgfqpoint{2.747129in}{1.214716in}}%
\pgfusepath{stroke}%
\end{pgfscope}%
\begin{pgfscope}%
\pgfpathrectangle{\pgfqpoint{0.100000in}{0.212622in}}{\pgfqpoint{3.696000in}{3.696000in}}%
\pgfusepath{clip}%
\pgfsetrectcap%
\pgfsetroundjoin%
\pgfsetlinewidth{1.505625pt}%
\definecolor{currentstroke}{rgb}{1.000000,0.000000,0.000000}%
\pgfsetstrokecolor{currentstroke}%
\pgfsetdash{}{0pt}%
\pgfpathmoveto{\pgfqpoint{2.869321in}{2.100933in}}%
\pgfpathlineto{\pgfqpoint{2.739306in}{1.206980in}}%
\pgfusepath{stroke}%
\end{pgfscope}%
\begin{pgfscope}%
\pgfpathrectangle{\pgfqpoint{0.100000in}{0.212622in}}{\pgfqpoint{3.696000in}{3.696000in}}%
\pgfusepath{clip}%
\pgfsetrectcap%
\pgfsetroundjoin%
\pgfsetlinewidth{1.505625pt}%
\definecolor{currentstroke}{rgb}{1.000000,0.000000,0.000000}%
\pgfsetstrokecolor{currentstroke}%
\pgfsetdash{}{0pt}%
\pgfpathmoveto{\pgfqpoint{2.864746in}{2.098169in}}%
\pgfpathlineto{\pgfqpoint{2.739306in}{1.206980in}}%
\pgfusepath{stroke}%
\end{pgfscope}%
\begin{pgfscope}%
\pgfpathrectangle{\pgfqpoint{0.100000in}{0.212622in}}{\pgfqpoint{3.696000in}{3.696000in}}%
\pgfusepath{clip}%
\pgfsetrectcap%
\pgfsetroundjoin%
\pgfsetlinewidth{1.505625pt}%
\definecolor{currentstroke}{rgb}{1.000000,0.000000,0.000000}%
\pgfsetstrokecolor{currentstroke}%
\pgfsetdash{}{0pt}%
\pgfpathmoveto{\pgfqpoint{2.858747in}{2.095454in}}%
\pgfpathlineto{\pgfqpoint{2.731473in}{1.199234in}}%
\pgfusepath{stroke}%
\end{pgfscope}%
\begin{pgfscope}%
\pgfpathrectangle{\pgfqpoint{0.100000in}{0.212622in}}{\pgfqpoint{3.696000in}{3.696000in}}%
\pgfusepath{clip}%
\pgfsetrectcap%
\pgfsetroundjoin%
\pgfsetlinewidth{1.505625pt}%
\definecolor{currentstroke}{rgb}{1.000000,0.000000,0.000000}%
\pgfsetstrokecolor{currentstroke}%
\pgfsetdash{}{0pt}%
\pgfpathmoveto{\pgfqpoint{2.855390in}{2.093604in}}%
\pgfpathlineto{\pgfqpoint{2.731473in}{1.199234in}}%
\pgfusepath{stroke}%
\end{pgfscope}%
\begin{pgfscope}%
\pgfpathrectangle{\pgfqpoint{0.100000in}{0.212622in}}{\pgfqpoint{3.696000in}{3.696000in}}%
\pgfusepath{clip}%
\pgfsetrectcap%
\pgfsetroundjoin%
\pgfsetlinewidth{1.505625pt}%
\definecolor{currentstroke}{rgb}{1.000000,0.000000,0.000000}%
\pgfsetstrokecolor{currentstroke}%
\pgfsetdash{}{0pt}%
\pgfpathmoveto{\pgfqpoint{2.853679in}{2.092575in}}%
\pgfpathlineto{\pgfqpoint{2.723631in}{1.191478in}}%
\pgfusepath{stroke}%
\end{pgfscope}%
\begin{pgfscope}%
\pgfpathrectangle{\pgfqpoint{0.100000in}{0.212622in}}{\pgfqpoint{3.696000in}{3.696000in}}%
\pgfusepath{clip}%
\pgfsetrectcap%
\pgfsetroundjoin%
\pgfsetlinewidth{1.505625pt}%
\definecolor{currentstroke}{rgb}{1.000000,0.000000,0.000000}%
\pgfsetstrokecolor{currentstroke}%
\pgfsetdash{}{0pt}%
\pgfpathmoveto{\pgfqpoint{2.849919in}{2.090584in}}%
\pgfpathlineto{\pgfqpoint{2.723631in}{1.191478in}}%
\pgfusepath{stroke}%
\end{pgfscope}%
\begin{pgfscope}%
\pgfpathrectangle{\pgfqpoint{0.100000in}{0.212622in}}{\pgfqpoint{3.696000in}{3.696000in}}%
\pgfusepath{clip}%
\pgfsetrectcap%
\pgfsetroundjoin%
\pgfsetlinewidth{1.505625pt}%
\definecolor{currentstroke}{rgb}{1.000000,0.000000,0.000000}%
\pgfsetstrokecolor{currentstroke}%
\pgfsetdash{}{0pt}%
\pgfpathmoveto{\pgfqpoint{2.845967in}{2.087604in}}%
\pgfpathlineto{\pgfqpoint{2.715778in}{1.183713in}}%
\pgfusepath{stroke}%
\end{pgfscope}%
\begin{pgfscope}%
\pgfpathrectangle{\pgfqpoint{0.100000in}{0.212622in}}{\pgfqpoint{3.696000in}{3.696000in}}%
\pgfusepath{clip}%
\pgfsetrectcap%
\pgfsetroundjoin%
\pgfsetlinewidth{1.505625pt}%
\definecolor{currentstroke}{rgb}{1.000000,0.000000,0.000000}%
\pgfsetstrokecolor{currentstroke}%
\pgfsetdash{}{0pt}%
\pgfpathmoveto{\pgfqpoint{2.841991in}{2.084517in}}%
\pgfpathlineto{\pgfqpoint{2.715778in}{1.183713in}}%
\pgfusepath{stroke}%
\end{pgfscope}%
\begin{pgfscope}%
\pgfpathrectangle{\pgfqpoint{0.100000in}{0.212622in}}{\pgfqpoint{3.696000in}{3.696000in}}%
\pgfusepath{clip}%
\pgfsetrectcap%
\pgfsetroundjoin%
\pgfsetlinewidth{1.505625pt}%
\definecolor{currentstroke}{rgb}{1.000000,0.000000,0.000000}%
\pgfsetstrokecolor{currentstroke}%
\pgfsetdash{}{0pt}%
\pgfpathmoveto{\pgfqpoint{2.839655in}{2.082940in}}%
\pgfpathlineto{\pgfqpoint{2.715778in}{1.183713in}}%
\pgfusepath{stroke}%
\end{pgfscope}%
\begin{pgfscope}%
\pgfpathrectangle{\pgfqpoint{0.100000in}{0.212622in}}{\pgfqpoint{3.696000in}{3.696000in}}%
\pgfusepath{clip}%
\pgfsetrectcap%
\pgfsetroundjoin%
\pgfsetlinewidth{1.505625pt}%
\definecolor{currentstroke}{rgb}{1.000000,0.000000,0.000000}%
\pgfsetstrokecolor{currentstroke}%
\pgfsetdash{}{0pt}%
\pgfpathmoveto{\pgfqpoint{2.835675in}{2.080059in}}%
\pgfpathlineto{\pgfqpoint{2.707916in}{1.175938in}}%
\pgfusepath{stroke}%
\end{pgfscope}%
\begin{pgfscope}%
\pgfpathrectangle{\pgfqpoint{0.100000in}{0.212622in}}{\pgfqpoint{3.696000in}{3.696000in}}%
\pgfusepath{clip}%
\pgfsetrectcap%
\pgfsetroundjoin%
\pgfsetlinewidth{1.505625pt}%
\definecolor{currentstroke}{rgb}{1.000000,0.000000,0.000000}%
\pgfsetstrokecolor{currentstroke}%
\pgfsetdash{}{0pt}%
\pgfpathmoveto{\pgfqpoint{2.831551in}{2.077337in}}%
\pgfpathlineto{\pgfqpoint{2.707916in}{1.175938in}}%
\pgfusepath{stroke}%
\end{pgfscope}%
\begin{pgfscope}%
\pgfpathrectangle{\pgfqpoint{0.100000in}{0.212622in}}{\pgfqpoint{3.696000in}{3.696000in}}%
\pgfusepath{clip}%
\pgfsetrectcap%
\pgfsetroundjoin%
\pgfsetlinewidth{1.505625pt}%
\definecolor{currentstroke}{rgb}{1.000000,0.000000,0.000000}%
\pgfsetstrokecolor{currentstroke}%
\pgfsetdash{}{0pt}%
\pgfpathmoveto{\pgfqpoint{2.829289in}{2.075698in}}%
\pgfpathlineto{\pgfqpoint{2.700044in}{1.168153in}}%
\pgfusepath{stroke}%
\end{pgfscope}%
\begin{pgfscope}%
\pgfpathrectangle{\pgfqpoint{0.100000in}{0.212622in}}{\pgfqpoint{3.696000in}{3.696000in}}%
\pgfusepath{clip}%
\pgfsetrectcap%
\pgfsetroundjoin%
\pgfsetlinewidth{1.505625pt}%
\definecolor{currentstroke}{rgb}{1.000000,0.000000,0.000000}%
\pgfsetstrokecolor{currentstroke}%
\pgfsetdash{}{0pt}%
\pgfpathmoveto{\pgfqpoint{2.825866in}{2.072510in}}%
\pgfpathlineto{\pgfqpoint{2.700044in}{1.168153in}}%
\pgfusepath{stroke}%
\end{pgfscope}%
\begin{pgfscope}%
\pgfpathrectangle{\pgfqpoint{0.100000in}{0.212622in}}{\pgfqpoint{3.696000in}{3.696000in}}%
\pgfusepath{clip}%
\pgfsetrectcap%
\pgfsetroundjoin%
\pgfsetlinewidth{1.505625pt}%
\definecolor{currentstroke}{rgb}{1.000000,0.000000,0.000000}%
\pgfsetstrokecolor{currentstroke}%
\pgfsetdash{}{0pt}%
\pgfpathmoveto{\pgfqpoint{2.822332in}{2.069422in}}%
\pgfpathlineto{\pgfqpoint{2.692162in}{1.160359in}}%
\pgfusepath{stroke}%
\end{pgfscope}%
\begin{pgfscope}%
\pgfpathrectangle{\pgfqpoint{0.100000in}{0.212622in}}{\pgfqpoint{3.696000in}{3.696000in}}%
\pgfusepath{clip}%
\pgfsetrectcap%
\pgfsetroundjoin%
\pgfsetlinewidth{1.505625pt}%
\definecolor{currentstroke}{rgb}{1.000000,0.000000,0.000000}%
\pgfsetstrokecolor{currentstroke}%
\pgfsetdash{}{0pt}%
\pgfpathmoveto{\pgfqpoint{2.818235in}{2.065687in}}%
\pgfpathlineto{\pgfqpoint{2.692162in}{1.160359in}}%
\pgfusepath{stroke}%
\end{pgfscope}%
\begin{pgfscope}%
\pgfpathrectangle{\pgfqpoint{0.100000in}{0.212622in}}{\pgfqpoint{3.696000in}{3.696000in}}%
\pgfusepath{clip}%
\pgfsetrectcap%
\pgfsetroundjoin%
\pgfsetlinewidth{1.505625pt}%
\definecolor{currentstroke}{rgb}{1.000000,0.000000,0.000000}%
\pgfsetstrokecolor{currentstroke}%
\pgfsetdash{}{0pt}%
\pgfpathmoveto{\pgfqpoint{2.812493in}{2.059907in}}%
\pgfpathlineto{\pgfqpoint{2.684271in}{1.152555in}}%
\pgfusepath{stroke}%
\end{pgfscope}%
\begin{pgfscope}%
\pgfpathrectangle{\pgfqpoint{0.100000in}{0.212622in}}{\pgfqpoint{3.696000in}{3.696000in}}%
\pgfusepath{clip}%
\pgfsetrectcap%
\pgfsetroundjoin%
\pgfsetlinewidth{1.505625pt}%
\definecolor{currentstroke}{rgb}{1.000000,0.000000,0.000000}%
\pgfsetstrokecolor{currentstroke}%
\pgfsetdash{}{0pt}%
\pgfpathmoveto{\pgfqpoint{2.809427in}{2.057198in}}%
\pgfpathlineto{\pgfqpoint{2.684271in}{1.152555in}}%
\pgfusepath{stroke}%
\end{pgfscope}%
\begin{pgfscope}%
\pgfpathrectangle{\pgfqpoint{0.100000in}{0.212622in}}{\pgfqpoint{3.696000in}{3.696000in}}%
\pgfusepath{clip}%
\pgfsetrectcap%
\pgfsetroundjoin%
\pgfsetlinewidth{1.505625pt}%
\definecolor{currentstroke}{rgb}{1.000000,0.000000,0.000000}%
\pgfsetstrokecolor{currentstroke}%
\pgfsetdash{}{0pt}%
\pgfpathmoveto{\pgfqpoint{2.805844in}{2.053880in}}%
\pgfpathlineto{\pgfqpoint{2.676369in}{1.144741in}}%
\pgfusepath{stroke}%
\end{pgfscope}%
\begin{pgfscope}%
\pgfpathrectangle{\pgfqpoint{0.100000in}{0.212622in}}{\pgfqpoint{3.696000in}{3.696000in}}%
\pgfusepath{clip}%
\pgfsetrectcap%
\pgfsetroundjoin%
\pgfsetlinewidth{1.505625pt}%
\definecolor{currentstroke}{rgb}{1.000000,0.000000,0.000000}%
\pgfsetstrokecolor{currentstroke}%
\pgfsetdash{}{0pt}%
\pgfpathmoveto{\pgfqpoint{2.801561in}{2.049017in}}%
\pgfpathlineto{\pgfqpoint{2.676369in}{1.144741in}}%
\pgfusepath{stroke}%
\end{pgfscope}%
\begin{pgfscope}%
\pgfpathrectangle{\pgfqpoint{0.100000in}{0.212622in}}{\pgfqpoint{3.696000in}{3.696000in}}%
\pgfusepath{clip}%
\pgfsetrectcap%
\pgfsetroundjoin%
\pgfsetlinewidth{1.505625pt}%
\definecolor{currentstroke}{rgb}{1.000000,0.000000,0.000000}%
\pgfsetstrokecolor{currentstroke}%
\pgfsetdash{}{0pt}%
\pgfpathmoveto{\pgfqpoint{2.796943in}{2.045036in}}%
\pgfpathlineto{\pgfqpoint{2.668458in}{1.136917in}}%
\pgfusepath{stroke}%
\end{pgfscope}%
\begin{pgfscope}%
\pgfpathrectangle{\pgfqpoint{0.100000in}{0.212622in}}{\pgfqpoint{3.696000in}{3.696000in}}%
\pgfusepath{clip}%
\pgfsetrectcap%
\pgfsetroundjoin%
\pgfsetlinewidth{1.505625pt}%
\definecolor{currentstroke}{rgb}{1.000000,0.000000,0.000000}%
\pgfsetstrokecolor{currentstroke}%
\pgfsetdash{}{0pt}%
\pgfpathmoveto{\pgfqpoint{2.794245in}{2.042481in}}%
\pgfpathlineto{\pgfqpoint{2.668458in}{1.136917in}}%
\pgfusepath{stroke}%
\end{pgfscope}%
\begin{pgfscope}%
\pgfpathrectangle{\pgfqpoint{0.100000in}{0.212622in}}{\pgfqpoint{3.696000in}{3.696000in}}%
\pgfusepath{clip}%
\pgfsetrectcap%
\pgfsetroundjoin%
\pgfsetlinewidth{1.505625pt}%
\definecolor{currentstroke}{rgb}{1.000000,0.000000,0.000000}%
\pgfsetstrokecolor{currentstroke}%
\pgfsetdash{}{0pt}%
\pgfpathmoveto{\pgfqpoint{2.792895in}{2.041225in}}%
\pgfpathlineto{\pgfqpoint{2.660536in}{1.129084in}}%
\pgfusepath{stroke}%
\end{pgfscope}%
\begin{pgfscope}%
\pgfpathrectangle{\pgfqpoint{0.100000in}{0.212622in}}{\pgfqpoint{3.696000in}{3.696000in}}%
\pgfusepath{clip}%
\pgfsetrectcap%
\pgfsetroundjoin%
\pgfsetlinewidth{1.505625pt}%
\definecolor{currentstroke}{rgb}{1.000000,0.000000,0.000000}%
\pgfsetstrokecolor{currentstroke}%
\pgfsetdash{}{0pt}%
\pgfpathmoveto{\pgfqpoint{2.789742in}{2.038109in}}%
\pgfpathlineto{\pgfqpoint{2.660536in}{1.129084in}}%
\pgfusepath{stroke}%
\end{pgfscope}%
\begin{pgfscope}%
\pgfpathrectangle{\pgfqpoint{0.100000in}{0.212622in}}{\pgfqpoint{3.696000in}{3.696000in}}%
\pgfusepath{clip}%
\pgfsetrectcap%
\pgfsetroundjoin%
\pgfsetlinewidth{1.505625pt}%
\definecolor{currentstroke}{rgb}{1.000000,0.000000,0.000000}%
\pgfsetstrokecolor{currentstroke}%
\pgfsetdash{}{0pt}%
\pgfpathmoveto{\pgfqpoint{2.786140in}{2.034578in}}%
\pgfpathlineto{\pgfqpoint{2.660536in}{1.129084in}}%
\pgfusepath{stroke}%
\end{pgfscope}%
\begin{pgfscope}%
\pgfpathrectangle{\pgfqpoint{0.100000in}{0.212622in}}{\pgfqpoint{3.696000in}{3.696000in}}%
\pgfusepath{clip}%
\pgfsetrectcap%
\pgfsetroundjoin%
\pgfsetlinewidth{1.505625pt}%
\definecolor{currentstroke}{rgb}{1.000000,0.000000,0.000000}%
\pgfsetstrokecolor{currentstroke}%
\pgfsetdash{}{0pt}%
\pgfpathmoveto{\pgfqpoint{2.784320in}{2.032688in}}%
\pgfpathlineto{\pgfqpoint{2.652605in}{1.121240in}}%
\pgfusepath{stroke}%
\end{pgfscope}%
\begin{pgfscope}%
\pgfpathrectangle{\pgfqpoint{0.100000in}{0.212622in}}{\pgfqpoint{3.696000in}{3.696000in}}%
\pgfusepath{clip}%
\pgfsetrectcap%
\pgfsetroundjoin%
\pgfsetlinewidth{1.505625pt}%
\definecolor{currentstroke}{rgb}{1.000000,0.000000,0.000000}%
\pgfsetstrokecolor{currentstroke}%
\pgfsetdash{}{0pt}%
\pgfpathmoveto{\pgfqpoint{2.781548in}{2.029990in}}%
\pgfpathlineto{\pgfqpoint{2.652605in}{1.121240in}}%
\pgfusepath{stroke}%
\end{pgfscope}%
\begin{pgfscope}%
\pgfpathrectangle{\pgfqpoint{0.100000in}{0.212622in}}{\pgfqpoint{3.696000in}{3.696000in}}%
\pgfusepath{clip}%
\pgfsetrectcap%
\pgfsetroundjoin%
\pgfsetlinewidth{1.505625pt}%
\definecolor{currentstroke}{rgb}{1.000000,0.000000,0.000000}%
\pgfsetstrokecolor{currentstroke}%
\pgfsetdash{}{0pt}%
\pgfpathmoveto{\pgfqpoint{2.778122in}{2.026479in}}%
\pgfpathlineto{\pgfqpoint{2.644664in}{1.113387in}}%
\pgfusepath{stroke}%
\end{pgfscope}%
\begin{pgfscope}%
\pgfpathrectangle{\pgfqpoint{0.100000in}{0.212622in}}{\pgfqpoint{3.696000in}{3.696000in}}%
\pgfusepath{clip}%
\pgfsetrectcap%
\pgfsetroundjoin%
\pgfsetlinewidth{1.505625pt}%
\definecolor{currentstroke}{rgb}{1.000000,0.000000,0.000000}%
\pgfsetstrokecolor{currentstroke}%
\pgfsetdash{}{0pt}%
\pgfpathmoveto{\pgfqpoint{2.776261in}{2.024972in}}%
\pgfpathlineto{\pgfqpoint{2.644664in}{1.113387in}}%
\pgfusepath{stroke}%
\end{pgfscope}%
\begin{pgfscope}%
\pgfpathrectangle{\pgfqpoint{0.100000in}{0.212622in}}{\pgfqpoint{3.696000in}{3.696000in}}%
\pgfusepath{clip}%
\pgfsetrectcap%
\pgfsetroundjoin%
\pgfsetlinewidth{1.505625pt}%
\definecolor{currentstroke}{rgb}{1.000000,0.000000,0.000000}%
\pgfsetstrokecolor{currentstroke}%
\pgfsetdash{}{0pt}%
\pgfpathmoveto{\pgfqpoint{2.773991in}{2.022923in}}%
\pgfpathlineto{\pgfqpoint{2.644664in}{1.113387in}}%
\pgfusepath{stroke}%
\end{pgfscope}%
\begin{pgfscope}%
\pgfpathrectangle{\pgfqpoint{0.100000in}{0.212622in}}{\pgfqpoint{3.696000in}{3.696000in}}%
\pgfusepath{clip}%
\pgfsetrectcap%
\pgfsetroundjoin%
\pgfsetlinewidth{1.505625pt}%
\definecolor{currentstroke}{rgb}{1.000000,0.000000,0.000000}%
\pgfsetstrokecolor{currentstroke}%
\pgfsetdash{}{0pt}%
\pgfpathmoveto{\pgfqpoint{2.770292in}{2.019451in}}%
\pgfpathlineto{\pgfqpoint{2.636713in}{1.105524in}}%
\pgfusepath{stroke}%
\end{pgfscope}%
\begin{pgfscope}%
\pgfpathrectangle{\pgfqpoint{0.100000in}{0.212622in}}{\pgfqpoint{3.696000in}{3.696000in}}%
\pgfusepath{clip}%
\pgfsetrectcap%
\pgfsetroundjoin%
\pgfsetlinewidth{1.505625pt}%
\definecolor{currentstroke}{rgb}{1.000000,0.000000,0.000000}%
\pgfsetstrokecolor{currentstroke}%
\pgfsetdash{}{0pt}%
\pgfpathmoveto{\pgfqpoint{2.766298in}{2.015088in}}%
\pgfpathlineto{\pgfqpoint{2.636713in}{1.105524in}}%
\pgfusepath{stroke}%
\end{pgfscope}%
\begin{pgfscope}%
\pgfpathrectangle{\pgfqpoint{0.100000in}{0.212622in}}{\pgfqpoint{3.696000in}{3.696000in}}%
\pgfusepath{clip}%
\pgfsetrectcap%
\pgfsetroundjoin%
\pgfsetlinewidth{1.505625pt}%
\definecolor{currentstroke}{rgb}{1.000000,0.000000,0.000000}%
\pgfsetstrokecolor{currentstroke}%
\pgfsetdash{}{0pt}%
\pgfpathmoveto{\pgfqpoint{2.761697in}{2.010127in}}%
\pgfpathlineto{\pgfqpoint{2.628752in}{1.097651in}}%
\pgfusepath{stroke}%
\end{pgfscope}%
\begin{pgfscope}%
\pgfpathrectangle{\pgfqpoint{0.100000in}{0.212622in}}{\pgfqpoint{3.696000in}{3.696000in}}%
\pgfusepath{clip}%
\pgfsetrectcap%
\pgfsetroundjoin%
\pgfsetlinewidth{1.505625pt}%
\definecolor{currentstroke}{rgb}{1.000000,0.000000,0.000000}%
\pgfsetstrokecolor{currentstroke}%
\pgfsetdash{}{0pt}%
\pgfpathmoveto{\pgfqpoint{2.759346in}{2.007488in}}%
\pgfpathlineto{\pgfqpoint{2.628752in}{1.097651in}}%
\pgfusepath{stroke}%
\end{pgfscope}%
\begin{pgfscope}%
\pgfpathrectangle{\pgfqpoint{0.100000in}{0.212622in}}{\pgfqpoint{3.696000in}{3.696000in}}%
\pgfusepath{clip}%
\pgfsetrectcap%
\pgfsetroundjoin%
\pgfsetlinewidth{1.505625pt}%
\definecolor{currentstroke}{rgb}{1.000000,0.000000,0.000000}%
\pgfsetstrokecolor{currentstroke}%
\pgfsetdash{}{0pt}%
\pgfpathmoveto{\pgfqpoint{2.755073in}{2.002749in}}%
\pgfpathlineto{\pgfqpoint{2.620781in}{1.089769in}}%
\pgfusepath{stroke}%
\end{pgfscope}%
\begin{pgfscope}%
\pgfpathrectangle{\pgfqpoint{0.100000in}{0.212622in}}{\pgfqpoint{3.696000in}{3.696000in}}%
\pgfusepath{clip}%
\pgfsetrectcap%
\pgfsetroundjoin%
\pgfsetlinewidth{1.505625pt}%
\definecolor{currentstroke}{rgb}{1.000000,0.000000,0.000000}%
\pgfsetstrokecolor{currentstroke}%
\pgfsetdash{}{0pt}%
\pgfpathmoveto{\pgfqpoint{2.750496in}{1.998145in}}%
\pgfpathlineto{\pgfqpoint{2.620781in}{1.089769in}}%
\pgfusepath{stroke}%
\end{pgfscope}%
\begin{pgfscope}%
\pgfpathrectangle{\pgfqpoint{0.100000in}{0.212622in}}{\pgfqpoint{3.696000in}{3.696000in}}%
\pgfusepath{clip}%
\pgfsetrectcap%
\pgfsetroundjoin%
\pgfsetlinewidth{1.505625pt}%
\definecolor{currentstroke}{rgb}{1.000000,0.000000,0.000000}%
\pgfsetstrokecolor{currentstroke}%
\pgfsetdash{}{0pt}%
\pgfpathmoveto{\pgfqpoint{2.748098in}{1.995805in}}%
\pgfpathlineto{\pgfqpoint{2.612799in}{1.081876in}}%
\pgfusepath{stroke}%
\end{pgfscope}%
\begin{pgfscope}%
\pgfpathrectangle{\pgfqpoint{0.100000in}{0.212622in}}{\pgfqpoint{3.696000in}{3.696000in}}%
\pgfusepath{clip}%
\pgfsetrectcap%
\pgfsetroundjoin%
\pgfsetlinewidth{1.505625pt}%
\definecolor{currentstroke}{rgb}{1.000000,0.000000,0.000000}%
\pgfsetstrokecolor{currentstroke}%
\pgfsetdash{}{0pt}%
\pgfpathmoveto{\pgfqpoint{2.746694in}{1.994460in}}%
\pgfpathlineto{\pgfqpoint{2.612799in}{1.081876in}}%
\pgfusepath{stroke}%
\end{pgfscope}%
\begin{pgfscope}%
\pgfpathrectangle{\pgfqpoint{0.100000in}{0.212622in}}{\pgfqpoint{3.696000in}{3.696000in}}%
\pgfusepath{clip}%
\pgfsetrectcap%
\pgfsetroundjoin%
\pgfsetlinewidth{1.505625pt}%
\definecolor{currentstroke}{rgb}{1.000000,0.000000,0.000000}%
\pgfsetstrokecolor{currentstroke}%
\pgfsetdash{}{0pt}%
\pgfpathmoveto{\pgfqpoint{2.744033in}{1.991966in}}%
\pgfpathlineto{\pgfqpoint{2.612799in}{1.081876in}}%
\pgfusepath{stroke}%
\end{pgfscope}%
\begin{pgfscope}%
\pgfpathrectangle{\pgfqpoint{0.100000in}{0.212622in}}{\pgfqpoint{3.696000in}{3.696000in}}%
\pgfusepath{clip}%
\pgfsetrectcap%
\pgfsetroundjoin%
\pgfsetlinewidth{1.505625pt}%
\definecolor{currentstroke}{rgb}{1.000000,0.000000,0.000000}%
\pgfsetstrokecolor{currentstroke}%
\pgfsetdash{}{0pt}%
\pgfpathmoveto{\pgfqpoint{2.741068in}{1.988998in}}%
\pgfpathlineto{\pgfqpoint{2.604808in}{1.073973in}}%
\pgfusepath{stroke}%
\end{pgfscope}%
\begin{pgfscope}%
\pgfpathrectangle{\pgfqpoint{0.100000in}{0.212622in}}{\pgfqpoint{3.696000in}{3.696000in}}%
\pgfusepath{clip}%
\pgfsetrectcap%
\pgfsetroundjoin%
\pgfsetlinewidth{1.505625pt}%
\definecolor{currentstroke}{rgb}{1.000000,0.000000,0.000000}%
\pgfsetstrokecolor{currentstroke}%
\pgfsetdash{}{0pt}%
\pgfpathmoveto{\pgfqpoint{2.737367in}{1.985087in}}%
\pgfpathlineto{\pgfqpoint{2.604808in}{1.073973in}}%
\pgfusepath{stroke}%
\end{pgfscope}%
\begin{pgfscope}%
\pgfpathrectangle{\pgfqpoint{0.100000in}{0.212622in}}{\pgfqpoint{3.696000in}{3.696000in}}%
\pgfusepath{clip}%
\pgfsetrectcap%
\pgfsetroundjoin%
\pgfsetlinewidth{1.505625pt}%
\definecolor{currentstroke}{rgb}{1.000000,0.000000,0.000000}%
\pgfsetstrokecolor{currentstroke}%
\pgfsetdash{}{0pt}%
\pgfpathmoveto{\pgfqpoint{2.735558in}{1.983211in}}%
\pgfpathlineto{\pgfqpoint{2.596807in}{1.066061in}}%
\pgfusepath{stroke}%
\end{pgfscope}%
\begin{pgfscope}%
\pgfpathrectangle{\pgfqpoint{0.100000in}{0.212622in}}{\pgfqpoint{3.696000in}{3.696000in}}%
\pgfusepath{clip}%
\pgfsetrectcap%
\pgfsetroundjoin%
\pgfsetlinewidth{1.505625pt}%
\definecolor{currentstroke}{rgb}{1.000000,0.000000,0.000000}%
\pgfsetstrokecolor{currentstroke}%
\pgfsetdash{}{0pt}%
\pgfpathmoveto{\pgfqpoint{2.731884in}{1.979862in}}%
\pgfpathlineto{\pgfqpoint{2.596807in}{1.066061in}}%
\pgfusepath{stroke}%
\end{pgfscope}%
\begin{pgfscope}%
\pgfpathrectangle{\pgfqpoint{0.100000in}{0.212622in}}{\pgfqpoint{3.696000in}{3.696000in}}%
\pgfusepath{clip}%
\pgfsetrectcap%
\pgfsetroundjoin%
\pgfsetlinewidth{1.505625pt}%
\definecolor{currentstroke}{rgb}{1.000000,0.000000,0.000000}%
\pgfsetstrokecolor{currentstroke}%
\pgfsetdash{}{0pt}%
\pgfpathmoveto{\pgfqpoint{2.729818in}{1.977721in}}%
\pgfpathlineto{\pgfqpoint{2.596807in}{1.066061in}}%
\pgfusepath{stroke}%
\end{pgfscope}%
\begin{pgfscope}%
\pgfpathrectangle{\pgfqpoint{0.100000in}{0.212622in}}{\pgfqpoint{3.696000in}{3.696000in}}%
\pgfusepath{clip}%
\pgfsetrectcap%
\pgfsetroundjoin%
\pgfsetlinewidth{1.505625pt}%
\definecolor{currentstroke}{rgb}{1.000000,0.000000,0.000000}%
\pgfsetstrokecolor{currentstroke}%
\pgfsetdash{}{0pt}%
\pgfpathmoveto{\pgfqpoint{2.728799in}{1.976636in}}%
\pgfpathlineto{\pgfqpoint{2.596807in}{1.066061in}}%
\pgfusepath{stroke}%
\end{pgfscope}%
\begin{pgfscope}%
\pgfpathrectangle{\pgfqpoint{0.100000in}{0.212622in}}{\pgfqpoint{3.696000in}{3.696000in}}%
\pgfusepath{clip}%
\pgfsetrectcap%
\pgfsetroundjoin%
\pgfsetlinewidth{1.505625pt}%
\definecolor{currentstroke}{rgb}{1.000000,0.000000,0.000000}%
\pgfsetstrokecolor{currentstroke}%
\pgfsetdash{}{0pt}%
\pgfpathmoveto{\pgfqpoint{2.726534in}{1.973765in}}%
\pgfpathlineto{\pgfqpoint{2.588796in}{1.058138in}}%
\pgfusepath{stroke}%
\end{pgfscope}%
\begin{pgfscope}%
\pgfpathrectangle{\pgfqpoint{0.100000in}{0.212622in}}{\pgfqpoint{3.696000in}{3.696000in}}%
\pgfusepath{clip}%
\pgfsetrectcap%
\pgfsetroundjoin%
\pgfsetlinewidth{1.505625pt}%
\definecolor{currentstroke}{rgb}{1.000000,0.000000,0.000000}%
\pgfsetstrokecolor{currentstroke}%
\pgfsetdash{}{0pt}%
\pgfpathmoveto{\pgfqpoint{2.723897in}{1.970440in}}%
\pgfpathlineto{\pgfqpoint{2.588796in}{1.058138in}}%
\pgfusepath{stroke}%
\end{pgfscope}%
\begin{pgfscope}%
\pgfpathrectangle{\pgfqpoint{0.100000in}{0.212622in}}{\pgfqpoint{3.696000in}{3.696000in}}%
\pgfusepath{clip}%
\pgfsetrectcap%
\pgfsetroundjoin%
\pgfsetlinewidth{1.505625pt}%
\definecolor{currentstroke}{rgb}{1.000000,0.000000,0.000000}%
\pgfsetstrokecolor{currentstroke}%
\pgfsetdash{}{0pt}%
\pgfpathmoveto{\pgfqpoint{2.722538in}{1.968897in}}%
\pgfpathlineto{\pgfqpoint{2.588796in}{1.058138in}}%
\pgfusepath{stroke}%
\end{pgfscope}%
\begin{pgfscope}%
\pgfpathrectangle{\pgfqpoint{0.100000in}{0.212622in}}{\pgfqpoint{3.696000in}{3.696000in}}%
\pgfusepath{clip}%
\pgfsetrectcap%
\pgfsetroundjoin%
\pgfsetlinewidth{1.505625pt}%
\definecolor{currentstroke}{rgb}{1.000000,0.000000,0.000000}%
\pgfsetstrokecolor{currentstroke}%
\pgfsetdash{}{0pt}%
\pgfpathmoveto{\pgfqpoint{2.720253in}{1.966266in}}%
\pgfpathlineto{\pgfqpoint{2.580774in}{1.050206in}}%
\pgfusepath{stroke}%
\end{pgfscope}%
\begin{pgfscope}%
\pgfpathrectangle{\pgfqpoint{0.100000in}{0.212622in}}{\pgfqpoint{3.696000in}{3.696000in}}%
\pgfusepath{clip}%
\pgfsetrectcap%
\pgfsetroundjoin%
\pgfsetlinewidth{1.505625pt}%
\definecolor{currentstroke}{rgb}{1.000000,0.000000,0.000000}%
\pgfsetstrokecolor{currentstroke}%
\pgfsetdash{}{0pt}%
\pgfpathmoveto{\pgfqpoint{2.717499in}{1.962726in}}%
\pgfpathlineto{\pgfqpoint{2.580774in}{1.050206in}}%
\pgfusepath{stroke}%
\end{pgfscope}%
\begin{pgfscope}%
\pgfpathrectangle{\pgfqpoint{0.100000in}{0.212622in}}{\pgfqpoint{3.696000in}{3.696000in}}%
\pgfusepath{clip}%
\pgfsetrectcap%
\pgfsetroundjoin%
\pgfsetlinewidth{1.505625pt}%
\definecolor{currentstroke}{rgb}{1.000000,0.000000,0.000000}%
\pgfsetstrokecolor{currentstroke}%
\pgfsetdash{}{0pt}%
\pgfpathmoveto{\pgfqpoint{2.715987in}{1.961286in}}%
\pgfpathlineto{\pgfqpoint{2.580774in}{1.050206in}}%
\pgfusepath{stroke}%
\end{pgfscope}%
\begin{pgfscope}%
\pgfpathrectangle{\pgfqpoint{0.100000in}{0.212622in}}{\pgfqpoint{3.696000in}{3.696000in}}%
\pgfusepath{clip}%
\pgfsetrectcap%
\pgfsetroundjoin%
\pgfsetlinewidth{1.505625pt}%
\definecolor{currentstroke}{rgb}{1.000000,0.000000,0.000000}%
\pgfsetstrokecolor{currentstroke}%
\pgfsetdash{}{0pt}%
\pgfpathmoveto{\pgfqpoint{2.715084in}{1.960360in}}%
\pgfpathlineto{\pgfqpoint{2.580774in}{1.050206in}}%
\pgfusepath{stroke}%
\end{pgfscope}%
\begin{pgfscope}%
\pgfpathrectangle{\pgfqpoint{0.100000in}{0.212622in}}{\pgfqpoint{3.696000in}{3.696000in}}%
\pgfusepath{clip}%
\pgfsetrectcap%
\pgfsetroundjoin%
\pgfsetlinewidth{1.505625pt}%
\definecolor{currentstroke}{rgb}{1.000000,0.000000,0.000000}%
\pgfsetstrokecolor{currentstroke}%
\pgfsetdash{}{0pt}%
\pgfpathmoveto{\pgfqpoint{2.713602in}{1.958719in}}%
\pgfpathlineto{\pgfqpoint{2.572743in}{1.042263in}}%
\pgfusepath{stroke}%
\end{pgfscope}%
\begin{pgfscope}%
\pgfpathrectangle{\pgfqpoint{0.100000in}{0.212622in}}{\pgfqpoint{3.696000in}{3.696000in}}%
\pgfusepath{clip}%
\pgfsetrectcap%
\pgfsetroundjoin%
\pgfsetlinewidth{1.505625pt}%
\definecolor{currentstroke}{rgb}{1.000000,0.000000,0.000000}%
\pgfsetstrokecolor{currentstroke}%
\pgfsetdash{}{0pt}%
\pgfpathmoveto{\pgfqpoint{2.711072in}{1.956000in}}%
\pgfpathlineto{\pgfqpoint{2.572743in}{1.042263in}}%
\pgfusepath{stroke}%
\end{pgfscope}%
\begin{pgfscope}%
\pgfpathrectangle{\pgfqpoint{0.100000in}{0.212622in}}{\pgfqpoint{3.696000in}{3.696000in}}%
\pgfusepath{clip}%
\pgfsetrectcap%
\pgfsetroundjoin%
\pgfsetlinewidth{1.505625pt}%
\definecolor{currentstroke}{rgb}{1.000000,0.000000,0.000000}%
\pgfsetstrokecolor{currentstroke}%
\pgfsetdash{}{0pt}%
\pgfpathmoveto{\pgfqpoint{2.707789in}{1.952752in}}%
\pgfpathlineto{\pgfqpoint{2.572743in}{1.042263in}}%
\pgfusepath{stroke}%
\end{pgfscope}%
\begin{pgfscope}%
\pgfpathrectangle{\pgfqpoint{0.100000in}{0.212622in}}{\pgfqpoint{3.696000in}{3.696000in}}%
\pgfusepath{clip}%
\pgfsetrectcap%
\pgfsetroundjoin%
\pgfsetlinewidth{1.505625pt}%
\definecolor{currentstroke}{rgb}{1.000000,0.000000,0.000000}%
\pgfsetstrokecolor{currentstroke}%
\pgfsetdash{}{0pt}%
\pgfpathmoveto{\pgfqpoint{2.704443in}{1.949476in}}%
\pgfpathlineto{\pgfqpoint{2.564701in}{1.034311in}}%
\pgfusepath{stroke}%
\end{pgfscope}%
\begin{pgfscope}%
\pgfpathrectangle{\pgfqpoint{0.100000in}{0.212622in}}{\pgfqpoint{3.696000in}{3.696000in}}%
\pgfusepath{clip}%
\pgfsetrectcap%
\pgfsetroundjoin%
\pgfsetlinewidth{1.505625pt}%
\definecolor{currentstroke}{rgb}{1.000000,0.000000,0.000000}%
\pgfsetstrokecolor{currentstroke}%
\pgfsetdash{}{0pt}%
\pgfpathmoveto{\pgfqpoint{2.698984in}{1.944286in}}%
\pgfpathlineto{\pgfqpoint{2.556649in}{1.026348in}}%
\pgfusepath{stroke}%
\end{pgfscope}%
\begin{pgfscope}%
\pgfpathrectangle{\pgfqpoint{0.100000in}{0.212622in}}{\pgfqpoint{3.696000in}{3.696000in}}%
\pgfusepath{clip}%
\pgfsetrectcap%
\pgfsetroundjoin%
\pgfsetlinewidth{1.505625pt}%
\definecolor{currentstroke}{rgb}{1.000000,0.000000,0.000000}%
\pgfsetstrokecolor{currentstroke}%
\pgfsetdash{}{0pt}%
\pgfpathmoveto{\pgfqpoint{2.692778in}{1.938190in}}%
\pgfpathlineto{\pgfqpoint{2.556649in}{1.026348in}}%
\pgfusepath{stroke}%
\end{pgfscope}%
\begin{pgfscope}%
\pgfpathrectangle{\pgfqpoint{0.100000in}{0.212622in}}{\pgfqpoint{3.696000in}{3.696000in}}%
\pgfusepath{clip}%
\pgfsetrectcap%
\pgfsetroundjoin%
\pgfsetlinewidth{1.505625pt}%
\definecolor{currentstroke}{rgb}{1.000000,0.000000,0.000000}%
\pgfsetstrokecolor{currentstroke}%
\pgfsetdash{}{0pt}%
\pgfpathmoveto{\pgfqpoint{2.686233in}{1.932425in}}%
\pgfpathlineto{\pgfqpoint{2.548587in}{1.018376in}}%
\pgfusepath{stroke}%
\end{pgfscope}%
\begin{pgfscope}%
\pgfpathrectangle{\pgfqpoint{0.100000in}{0.212622in}}{\pgfqpoint{3.696000in}{3.696000in}}%
\pgfusepath{clip}%
\pgfsetrectcap%
\pgfsetroundjoin%
\pgfsetlinewidth{1.505625pt}%
\definecolor{currentstroke}{rgb}{1.000000,0.000000,0.000000}%
\pgfsetstrokecolor{currentstroke}%
\pgfsetdash{}{0pt}%
\pgfpathmoveto{\pgfqpoint{2.682364in}{1.928806in}}%
\pgfpathlineto{\pgfqpoint{2.540515in}{1.010393in}}%
\pgfusepath{stroke}%
\end{pgfscope}%
\begin{pgfscope}%
\pgfpathrectangle{\pgfqpoint{0.100000in}{0.212622in}}{\pgfqpoint{3.696000in}{3.696000in}}%
\pgfusepath{clip}%
\pgfsetrectcap%
\pgfsetroundjoin%
\pgfsetlinewidth{1.505625pt}%
\definecolor{currentstroke}{rgb}{1.000000,0.000000,0.000000}%
\pgfsetstrokecolor{currentstroke}%
\pgfsetdash{}{0pt}%
\pgfpathmoveto{\pgfqpoint{2.678326in}{1.925369in}}%
\pgfpathlineto{\pgfqpoint{2.540515in}{1.010393in}}%
\pgfusepath{stroke}%
\end{pgfscope}%
\begin{pgfscope}%
\pgfpathrectangle{\pgfqpoint{0.100000in}{0.212622in}}{\pgfqpoint{3.696000in}{3.696000in}}%
\pgfusepath{clip}%
\pgfsetrectcap%
\pgfsetroundjoin%
\pgfsetlinewidth{1.505625pt}%
\definecolor{currentstroke}{rgb}{1.000000,0.000000,0.000000}%
\pgfsetstrokecolor{currentstroke}%
\pgfsetdash{}{0pt}%
\pgfpathmoveto{\pgfqpoint{2.672614in}{1.920024in}}%
\pgfpathlineto{\pgfqpoint{2.532432in}{1.002400in}}%
\pgfusepath{stroke}%
\end{pgfscope}%
\begin{pgfscope}%
\pgfpathrectangle{\pgfqpoint{0.100000in}{0.212622in}}{\pgfqpoint{3.696000in}{3.696000in}}%
\pgfusepath{clip}%
\pgfsetrectcap%
\pgfsetroundjoin%
\pgfsetlinewidth{1.505625pt}%
\definecolor{currentstroke}{rgb}{1.000000,0.000000,0.000000}%
\pgfsetstrokecolor{currentstroke}%
\pgfsetdash{}{0pt}%
\pgfpathmoveto{\pgfqpoint{2.666192in}{1.914441in}}%
\pgfpathlineto{\pgfqpoint{2.524340in}{0.994397in}}%
\pgfusepath{stroke}%
\end{pgfscope}%
\begin{pgfscope}%
\pgfpathrectangle{\pgfqpoint{0.100000in}{0.212622in}}{\pgfqpoint{3.696000in}{3.696000in}}%
\pgfusepath{clip}%
\pgfsetrectcap%
\pgfsetroundjoin%
\pgfsetlinewidth{1.505625pt}%
\definecolor{currentstroke}{rgb}{1.000000,0.000000,0.000000}%
\pgfsetstrokecolor{currentstroke}%
\pgfsetdash{}{0pt}%
\pgfpathmoveto{\pgfqpoint{2.659758in}{1.908673in}}%
\pgfpathlineto{\pgfqpoint{2.516237in}{0.986384in}}%
\pgfusepath{stroke}%
\end{pgfscope}%
\begin{pgfscope}%
\pgfpathrectangle{\pgfqpoint{0.100000in}{0.212622in}}{\pgfqpoint{3.696000in}{3.696000in}}%
\pgfusepath{clip}%
\pgfsetrectcap%
\pgfsetroundjoin%
\pgfsetlinewidth{1.505625pt}%
\definecolor{currentstroke}{rgb}{1.000000,0.000000,0.000000}%
\pgfsetstrokecolor{currentstroke}%
\pgfsetdash{}{0pt}%
\pgfpathmoveto{\pgfqpoint{2.656162in}{1.905056in}}%
\pgfpathlineto{\pgfqpoint{2.516237in}{0.986384in}}%
\pgfusepath{stroke}%
\end{pgfscope}%
\begin{pgfscope}%
\pgfpathrectangle{\pgfqpoint{0.100000in}{0.212622in}}{\pgfqpoint{3.696000in}{3.696000in}}%
\pgfusepath{clip}%
\pgfsetrectcap%
\pgfsetroundjoin%
\pgfsetlinewidth{1.505625pt}%
\definecolor{currentstroke}{rgb}{1.000000,0.000000,0.000000}%
\pgfsetstrokecolor{currentstroke}%
\pgfsetdash{}{0pt}%
\pgfpathmoveto{\pgfqpoint{2.651227in}{1.899460in}}%
\pgfpathlineto{\pgfqpoint{2.508124in}{0.978361in}}%
\pgfusepath{stroke}%
\end{pgfscope}%
\begin{pgfscope}%
\pgfpathrectangle{\pgfqpoint{0.100000in}{0.212622in}}{\pgfqpoint{3.696000in}{3.696000in}}%
\pgfusepath{clip}%
\pgfsetrectcap%
\pgfsetroundjoin%
\pgfsetlinewidth{1.505625pt}%
\definecolor{currentstroke}{rgb}{1.000000,0.000000,0.000000}%
\pgfsetstrokecolor{currentstroke}%
\pgfsetdash{}{0pt}%
\pgfpathmoveto{\pgfqpoint{2.648613in}{1.896799in}}%
\pgfpathlineto{\pgfqpoint{2.508124in}{0.978361in}}%
\pgfusepath{stroke}%
\end{pgfscope}%
\begin{pgfscope}%
\pgfpathrectangle{\pgfqpoint{0.100000in}{0.212622in}}{\pgfqpoint{3.696000in}{3.696000in}}%
\pgfusepath{clip}%
\pgfsetrectcap%
\pgfsetroundjoin%
\pgfsetlinewidth{1.505625pt}%
\definecolor{currentstroke}{rgb}{1.000000,0.000000,0.000000}%
\pgfsetstrokecolor{currentstroke}%
\pgfsetdash{}{0pt}%
\pgfpathmoveto{\pgfqpoint{2.647079in}{1.895102in}}%
\pgfpathlineto{\pgfqpoint{2.500000in}{0.970327in}}%
\pgfusepath{stroke}%
\end{pgfscope}%
\begin{pgfscope}%
\pgfpathrectangle{\pgfqpoint{0.100000in}{0.212622in}}{\pgfqpoint{3.696000in}{3.696000in}}%
\pgfusepath{clip}%
\pgfsetrectcap%
\pgfsetroundjoin%
\pgfsetlinewidth{1.505625pt}%
\definecolor{currentstroke}{rgb}{1.000000,0.000000,0.000000}%
\pgfsetstrokecolor{currentstroke}%
\pgfsetdash{}{0pt}%
\pgfpathmoveto{\pgfqpoint{2.645196in}{1.892835in}}%
\pgfpathlineto{\pgfqpoint{2.500000in}{0.970327in}}%
\pgfusepath{stroke}%
\end{pgfscope}%
\begin{pgfscope}%
\pgfpathrectangle{\pgfqpoint{0.100000in}{0.212622in}}{\pgfqpoint{3.696000in}{3.696000in}}%
\pgfusepath{clip}%
\pgfsetrectcap%
\pgfsetroundjoin%
\pgfsetlinewidth{1.505625pt}%
\definecolor{currentstroke}{rgb}{1.000000,0.000000,0.000000}%
\pgfsetstrokecolor{currentstroke}%
\pgfsetdash{}{0pt}%
\pgfpathmoveto{\pgfqpoint{2.642317in}{1.890102in}}%
\pgfpathlineto{\pgfqpoint{2.500000in}{0.970327in}}%
\pgfusepath{stroke}%
\end{pgfscope}%
\begin{pgfscope}%
\pgfpathrectangle{\pgfqpoint{0.100000in}{0.212622in}}{\pgfqpoint{3.696000in}{3.696000in}}%
\pgfusepath{clip}%
\pgfsetrectcap%
\pgfsetroundjoin%
\pgfsetlinewidth{1.505625pt}%
\definecolor{currentstroke}{rgb}{1.000000,0.000000,0.000000}%
\pgfsetstrokecolor{currentstroke}%
\pgfsetdash{}{0pt}%
\pgfpathmoveto{\pgfqpoint{2.640675in}{1.888417in}}%
\pgfpathlineto{\pgfqpoint{2.500000in}{0.970327in}}%
\pgfusepath{stroke}%
\end{pgfscope}%
\begin{pgfscope}%
\pgfpathrectangle{\pgfqpoint{0.100000in}{0.212622in}}{\pgfqpoint{3.696000in}{3.696000in}}%
\pgfusepath{clip}%
\pgfsetrectcap%
\pgfsetroundjoin%
\pgfsetlinewidth{1.505625pt}%
\definecolor{currentstroke}{rgb}{1.000000,0.000000,0.000000}%
\pgfsetstrokecolor{currentstroke}%
\pgfsetdash{}{0pt}%
\pgfpathmoveto{\pgfqpoint{2.638857in}{1.886544in}}%
\pgfpathlineto{\pgfqpoint{2.491866in}{0.962284in}}%
\pgfusepath{stroke}%
\end{pgfscope}%
\begin{pgfscope}%
\pgfpathrectangle{\pgfqpoint{0.100000in}{0.212622in}}{\pgfqpoint{3.696000in}{3.696000in}}%
\pgfusepath{clip}%
\pgfsetrectcap%
\pgfsetroundjoin%
\pgfsetlinewidth{1.505625pt}%
\definecolor{currentstroke}{rgb}{1.000000,0.000000,0.000000}%
\pgfsetstrokecolor{currentstroke}%
\pgfsetdash{}{0pt}%
\pgfpathmoveto{\pgfqpoint{2.635094in}{1.882602in}}%
\pgfpathlineto{\pgfqpoint{2.491866in}{0.962284in}}%
\pgfusepath{stroke}%
\end{pgfscope}%
\begin{pgfscope}%
\pgfpathrectangle{\pgfqpoint{0.100000in}{0.212622in}}{\pgfqpoint{3.696000in}{3.696000in}}%
\pgfusepath{clip}%
\pgfsetrectcap%
\pgfsetroundjoin%
\pgfsetlinewidth{1.505625pt}%
\definecolor{currentstroke}{rgb}{1.000000,0.000000,0.000000}%
\pgfsetstrokecolor{currentstroke}%
\pgfsetdash{}{0pt}%
\pgfpathmoveto{\pgfqpoint{2.631252in}{1.878804in}}%
\pgfpathlineto{\pgfqpoint{2.483722in}{0.954230in}}%
\pgfusepath{stroke}%
\end{pgfscope}%
\begin{pgfscope}%
\pgfpathrectangle{\pgfqpoint{0.100000in}{0.212622in}}{\pgfqpoint{3.696000in}{3.696000in}}%
\pgfusepath{clip}%
\pgfsetrectcap%
\pgfsetroundjoin%
\pgfsetlinewidth{1.505625pt}%
\definecolor{currentstroke}{rgb}{1.000000,0.000000,0.000000}%
\pgfsetstrokecolor{currentstroke}%
\pgfsetdash{}{0pt}%
\pgfpathmoveto{\pgfqpoint{2.629258in}{1.876802in}}%
\pgfpathlineto{\pgfqpoint{2.483722in}{0.954230in}}%
\pgfusepath{stroke}%
\end{pgfscope}%
\begin{pgfscope}%
\pgfpathrectangle{\pgfqpoint{0.100000in}{0.212622in}}{\pgfqpoint{3.696000in}{3.696000in}}%
\pgfusepath{clip}%
\pgfsetrectcap%
\pgfsetroundjoin%
\pgfsetlinewidth{1.505625pt}%
\definecolor{currentstroke}{rgb}{1.000000,0.000000,0.000000}%
\pgfsetstrokecolor{currentstroke}%
\pgfsetdash{}{0pt}%
\pgfpathmoveto{\pgfqpoint{2.626811in}{1.874398in}}%
\pgfpathlineto{\pgfqpoint{2.483722in}{0.954230in}}%
\pgfusepath{stroke}%
\end{pgfscope}%
\begin{pgfscope}%
\pgfpathrectangle{\pgfqpoint{0.100000in}{0.212622in}}{\pgfqpoint{3.696000in}{3.696000in}}%
\pgfusepath{clip}%
\pgfsetrectcap%
\pgfsetroundjoin%
\pgfsetlinewidth{1.505625pt}%
\definecolor{currentstroke}{rgb}{1.000000,0.000000,0.000000}%
\pgfsetstrokecolor{currentstroke}%
\pgfsetdash{}{0pt}%
\pgfpathmoveto{\pgfqpoint{2.623532in}{1.870633in}}%
\pgfpathlineto{\pgfqpoint{2.475568in}{0.946166in}}%
\pgfusepath{stroke}%
\end{pgfscope}%
\begin{pgfscope}%
\pgfpathrectangle{\pgfqpoint{0.100000in}{0.212622in}}{\pgfqpoint{3.696000in}{3.696000in}}%
\pgfusepath{clip}%
\pgfsetrectcap%
\pgfsetroundjoin%
\pgfsetlinewidth{1.505625pt}%
\definecolor{currentstroke}{rgb}{1.000000,0.000000,0.000000}%
\pgfsetstrokecolor{currentstroke}%
\pgfsetdash{}{0pt}%
\pgfpathmoveto{\pgfqpoint{2.621679in}{1.868740in}}%
\pgfpathlineto{\pgfqpoint{2.475568in}{0.946166in}}%
\pgfusepath{stroke}%
\end{pgfscope}%
\begin{pgfscope}%
\pgfpathrectangle{\pgfqpoint{0.100000in}{0.212622in}}{\pgfqpoint{3.696000in}{3.696000in}}%
\pgfusepath{clip}%
\pgfsetrectcap%
\pgfsetroundjoin%
\pgfsetlinewidth{1.505625pt}%
\definecolor{currentstroke}{rgb}{1.000000,0.000000,0.000000}%
\pgfsetstrokecolor{currentstroke}%
\pgfsetdash{}{0pt}%
\pgfpathmoveto{\pgfqpoint{2.620600in}{1.867611in}}%
\pgfpathlineto{\pgfqpoint{2.475568in}{0.946166in}}%
\pgfusepath{stroke}%
\end{pgfscope}%
\begin{pgfscope}%
\pgfpathrectangle{\pgfqpoint{0.100000in}{0.212622in}}{\pgfqpoint{3.696000in}{3.696000in}}%
\pgfusepath{clip}%
\pgfsetrectcap%
\pgfsetroundjoin%
\pgfsetlinewidth{1.505625pt}%
\definecolor{currentstroke}{rgb}{1.000000,0.000000,0.000000}%
\pgfsetstrokecolor{currentstroke}%
\pgfsetdash{}{0pt}%
\pgfpathmoveto{\pgfqpoint{2.619029in}{1.865869in}}%
\pgfpathlineto{\pgfqpoint{2.475568in}{0.946166in}}%
\pgfusepath{stroke}%
\end{pgfscope}%
\begin{pgfscope}%
\pgfpathrectangle{\pgfqpoint{0.100000in}{0.212622in}}{\pgfqpoint{3.696000in}{3.696000in}}%
\pgfusepath{clip}%
\pgfsetrectcap%
\pgfsetroundjoin%
\pgfsetlinewidth{1.505625pt}%
\definecolor{currentstroke}{rgb}{1.000000,0.000000,0.000000}%
\pgfsetstrokecolor{currentstroke}%
\pgfsetdash{}{0pt}%
\pgfpathmoveto{\pgfqpoint{2.616755in}{1.863237in}}%
\pgfpathlineto{\pgfqpoint{2.467403in}{0.938092in}}%
\pgfusepath{stroke}%
\end{pgfscope}%
\begin{pgfscope}%
\pgfpathrectangle{\pgfqpoint{0.100000in}{0.212622in}}{\pgfqpoint{3.696000in}{3.696000in}}%
\pgfusepath{clip}%
\pgfsetrectcap%
\pgfsetroundjoin%
\pgfsetlinewidth{1.505625pt}%
\definecolor{currentstroke}{rgb}{1.000000,0.000000,0.000000}%
\pgfsetstrokecolor{currentstroke}%
\pgfsetdash{}{0pt}%
\pgfpathmoveto{\pgfqpoint{2.613897in}{1.860201in}}%
\pgfpathlineto{\pgfqpoint{2.467403in}{0.938092in}}%
\pgfusepath{stroke}%
\end{pgfscope}%
\begin{pgfscope}%
\pgfpathrectangle{\pgfqpoint{0.100000in}{0.212622in}}{\pgfqpoint{3.696000in}{3.696000in}}%
\pgfusepath{clip}%
\pgfsetrectcap%
\pgfsetroundjoin%
\pgfsetlinewidth{1.505625pt}%
\definecolor{currentstroke}{rgb}{1.000000,0.000000,0.000000}%
\pgfsetstrokecolor{currentstroke}%
\pgfsetdash{}{0pt}%
\pgfpathmoveto{\pgfqpoint{2.611032in}{1.855989in}}%
\pgfpathlineto{\pgfqpoint{2.459228in}{0.930007in}}%
\pgfusepath{stroke}%
\end{pgfscope}%
\begin{pgfscope}%
\pgfpathrectangle{\pgfqpoint{0.100000in}{0.212622in}}{\pgfqpoint{3.696000in}{3.696000in}}%
\pgfusepath{clip}%
\pgfsetrectcap%
\pgfsetroundjoin%
\pgfsetlinewidth{1.505625pt}%
\definecolor{currentstroke}{rgb}{1.000000,0.000000,0.000000}%
\pgfsetstrokecolor{currentstroke}%
\pgfsetdash{}{0pt}%
\pgfpathmoveto{\pgfqpoint{2.607654in}{1.851551in}}%
\pgfpathlineto{\pgfqpoint{2.459228in}{0.930007in}}%
\pgfusepath{stroke}%
\end{pgfscope}%
\begin{pgfscope}%
\pgfpathrectangle{\pgfqpoint{0.100000in}{0.212622in}}{\pgfqpoint{3.696000in}{3.696000in}}%
\pgfusepath{clip}%
\pgfsetrectcap%
\pgfsetroundjoin%
\pgfsetlinewidth{1.505625pt}%
\definecolor{currentstroke}{rgb}{1.000000,0.000000,0.000000}%
\pgfsetstrokecolor{currentstroke}%
\pgfsetdash{}{0pt}%
\pgfpathmoveto{\pgfqpoint{2.603735in}{1.846921in}}%
\pgfpathlineto{\pgfqpoint{2.451042in}{0.921912in}}%
\pgfusepath{stroke}%
\end{pgfscope}%
\begin{pgfscope}%
\pgfpathrectangle{\pgfqpoint{0.100000in}{0.212622in}}{\pgfqpoint{3.696000in}{3.696000in}}%
\pgfusepath{clip}%
\pgfsetrectcap%
\pgfsetroundjoin%
\pgfsetlinewidth{1.505625pt}%
\definecolor{currentstroke}{rgb}{1.000000,0.000000,0.000000}%
\pgfsetstrokecolor{currentstroke}%
\pgfsetdash{}{0pt}%
\pgfpathmoveto{\pgfqpoint{2.598693in}{1.841093in}}%
\pgfpathlineto{\pgfqpoint{2.442846in}{0.913807in}}%
\pgfusepath{stroke}%
\end{pgfscope}%
\begin{pgfscope}%
\pgfpathrectangle{\pgfqpoint{0.100000in}{0.212622in}}{\pgfqpoint{3.696000in}{3.696000in}}%
\pgfusepath{clip}%
\pgfsetrectcap%
\pgfsetroundjoin%
\pgfsetlinewidth{1.505625pt}%
\definecolor{currentstroke}{rgb}{1.000000,0.000000,0.000000}%
\pgfsetstrokecolor{currentstroke}%
\pgfsetdash{}{0pt}%
\pgfpathmoveto{\pgfqpoint{2.591568in}{1.833721in}}%
\pgfpathlineto{\pgfqpoint{2.434640in}{0.905691in}}%
\pgfusepath{stroke}%
\end{pgfscope}%
\begin{pgfscope}%
\pgfpathrectangle{\pgfqpoint{0.100000in}{0.212622in}}{\pgfqpoint{3.696000in}{3.696000in}}%
\pgfusepath{clip}%
\pgfsetrectcap%
\pgfsetroundjoin%
\pgfsetlinewidth{1.505625pt}%
\definecolor{currentstroke}{rgb}{1.000000,0.000000,0.000000}%
\pgfsetstrokecolor{currentstroke}%
\pgfsetdash{}{0pt}%
\pgfpathmoveto{\pgfqpoint{2.583095in}{1.823820in}}%
\pgfpathlineto{\pgfqpoint{2.426423in}{0.897566in}}%
\pgfusepath{stroke}%
\end{pgfscope}%
\begin{pgfscope}%
\pgfpathrectangle{\pgfqpoint{0.100000in}{0.212622in}}{\pgfqpoint{3.696000in}{3.696000in}}%
\pgfusepath{clip}%
\pgfsetrectcap%
\pgfsetroundjoin%
\pgfsetlinewidth{1.505625pt}%
\definecolor{currentstroke}{rgb}{1.000000,0.000000,0.000000}%
\pgfsetstrokecolor{currentstroke}%
\pgfsetdash{}{0pt}%
\pgfpathmoveto{\pgfqpoint{2.573153in}{1.811956in}}%
\pgfpathlineto{\pgfqpoint{2.418195in}{0.889429in}}%
\pgfusepath{stroke}%
\end{pgfscope}%
\begin{pgfscope}%
\pgfpathrectangle{\pgfqpoint{0.100000in}{0.212622in}}{\pgfqpoint{3.696000in}{3.696000in}}%
\pgfusepath{clip}%
\pgfsetrectcap%
\pgfsetroundjoin%
\pgfsetlinewidth{1.505625pt}%
\definecolor{currentstroke}{rgb}{1.000000,0.000000,0.000000}%
\pgfsetstrokecolor{currentstroke}%
\pgfsetdash{}{0pt}%
\pgfpathmoveto{\pgfqpoint{2.561453in}{1.797594in}}%
\pgfpathlineto{\pgfqpoint{2.409957in}{0.881283in}}%
\pgfusepath{stroke}%
\end{pgfscope}%
\begin{pgfscope}%
\pgfpathrectangle{\pgfqpoint{0.100000in}{0.212622in}}{\pgfqpoint{3.696000in}{3.696000in}}%
\pgfusepath{clip}%
\pgfsetrectcap%
\pgfsetroundjoin%
\pgfsetlinewidth{1.505625pt}%
\definecolor{currentstroke}{rgb}{1.000000,0.000000,0.000000}%
\pgfsetstrokecolor{currentstroke}%
\pgfsetdash{}{0pt}%
\pgfpathmoveto{\pgfqpoint{2.555206in}{1.790224in}}%
\pgfpathlineto{\pgfqpoint{2.401709in}{0.873126in}}%
\pgfusepath{stroke}%
\end{pgfscope}%
\begin{pgfscope}%
\pgfpathrectangle{\pgfqpoint{0.100000in}{0.212622in}}{\pgfqpoint{3.696000in}{3.696000in}}%
\pgfusepath{clip}%
\pgfsetrectcap%
\pgfsetroundjoin%
\pgfsetlinewidth{1.505625pt}%
\definecolor{currentstroke}{rgb}{1.000000,0.000000,0.000000}%
\pgfsetstrokecolor{currentstroke}%
\pgfsetdash{}{0pt}%
\pgfpathmoveto{\pgfqpoint{2.551585in}{1.786134in}}%
\pgfpathlineto{\pgfqpoint{2.393450in}{0.864958in}}%
\pgfusepath{stroke}%
\end{pgfscope}%
\begin{pgfscope}%
\pgfpathrectangle{\pgfqpoint{0.100000in}{0.212622in}}{\pgfqpoint{3.696000in}{3.696000in}}%
\pgfusepath{clip}%
\pgfsetrectcap%
\pgfsetroundjoin%
\pgfsetlinewidth{1.505625pt}%
\definecolor{currentstroke}{rgb}{1.000000,0.000000,0.000000}%
\pgfsetstrokecolor{currentstroke}%
\pgfsetdash{}{0pt}%
\pgfpathmoveto{\pgfqpoint{2.547958in}{1.782172in}}%
\pgfpathlineto{\pgfqpoint{2.393450in}{0.864958in}}%
\pgfusepath{stroke}%
\end{pgfscope}%
\begin{pgfscope}%
\pgfpathrectangle{\pgfqpoint{0.100000in}{0.212622in}}{\pgfqpoint{3.696000in}{3.696000in}}%
\pgfusepath{clip}%
\pgfsetrectcap%
\pgfsetroundjoin%
\pgfsetlinewidth{1.505625pt}%
\definecolor{currentstroke}{rgb}{1.000000,0.000000,0.000000}%
\pgfsetstrokecolor{currentstroke}%
\pgfsetdash{}{0pt}%
\pgfpathmoveto{\pgfqpoint{2.541333in}{1.775261in}}%
\pgfpathlineto{\pgfqpoint{2.385180in}{0.856780in}}%
\pgfusepath{stroke}%
\end{pgfscope}%
\begin{pgfscope}%
\pgfpathrectangle{\pgfqpoint{0.100000in}{0.212622in}}{\pgfqpoint{3.696000in}{3.696000in}}%
\pgfusepath{clip}%
\pgfsetrectcap%
\pgfsetroundjoin%
\pgfsetlinewidth{1.505625pt}%
\definecolor{currentstroke}{rgb}{1.000000,0.000000,0.000000}%
\pgfsetstrokecolor{currentstroke}%
\pgfsetdash{}{0pt}%
\pgfpathmoveto{\pgfqpoint{2.534271in}{1.767726in}}%
\pgfpathlineto{\pgfqpoint{2.376900in}{0.848592in}}%
\pgfusepath{stroke}%
\end{pgfscope}%
\begin{pgfscope}%
\pgfpathrectangle{\pgfqpoint{0.100000in}{0.212622in}}{\pgfqpoint{3.696000in}{3.696000in}}%
\pgfusepath{clip}%
\pgfsetrectcap%
\pgfsetroundjoin%
\pgfsetlinewidth{1.505625pt}%
\definecolor{currentstroke}{rgb}{1.000000,0.000000,0.000000}%
\pgfsetstrokecolor{currentstroke}%
\pgfsetdash{}{0pt}%
\pgfpathmoveto{\pgfqpoint{2.525489in}{1.760066in}}%
\pgfpathlineto{\pgfqpoint{2.376900in}{0.848592in}}%
\pgfusepath{stroke}%
\end{pgfscope}%
\begin{pgfscope}%
\pgfpathrectangle{\pgfqpoint{0.100000in}{0.212622in}}{\pgfqpoint{3.696000in}{3.696000in}}%
\pgfusepath{clip}%
\pgfsetrectcap%
\pgfsetroundjoin%
\pgfsetlinewidth{1.505625pt}%
\definecolor{currentstroke}{rgb}{1.000000,0.000000,0.000000}%
\pgfsetstrokecolor{currentstroke}%
\pgfsetdash{}{0pt}%
\pgfpathmoveto{\pgfqpoint{2.516429in}{1.754322in}}%
\pgfpathlineto{\pgfqpoint{2.376900in}{0.848592in}}%
\pgfusepath{stroke}%
\end{pgfscope}%
\begin{pgfscope}%
\pgfpathrectangle{\pgfqpoint{0.100000in}{0.212622in}}{\pgfqpoint{3.696000in}{3.696000in}}%
\pgfusepath{clip}%
\pgfsetrectcap%
\pgfsetroundjoin%
\pgfsetlinewidth{1.505625pt}%
\definecolor{currentstroke}{rgb}{1.000000,0.000000,0.000000}%
\pgfsetstrokecolor{currentstroke}%
\pgfsetdash{}{0pt}%
\pgfpathmoveto{\pgfqpoint{2.511490in}{1.750681in}}%
\pgfpathlineto{\pgfqpoint{2.376900in}{0.848592in}}%
\pgfusepath{stroke}%
\end{pgfscope}%
\begin{pgfscope}%
\pgfpathrectangle{\pgfqpoint{0.100000in}{0.212622in}}{\pgfqpoint{3.696000in}{3.696000in}}%
\pgfusepath{clip}%
\pgfsetrectcap%
\pgfsetroundjoin%
\pgfsetlinewidth{1.505625pt}%
\definecolor{currentstroke}{rgb}{1.000000,0.000000,0.000000}%
\pgfsetstrokecolor{currentstroke}%
\pgfsetdash{}{0pt}%
\pgfpathmoveto{\pgfqpoint{2.508649in}{1.748401in}}%
\pgfpathlineto{\pgfqpoint{2.376900in}{0.848592in}}%
\pgfusepath{stroke}%
\end{pgfscope}%
\begin{pgfscope}%
\pgfpathrectangle{\pgfqpoint{0.100000in}{0.212622in}}{\pgfqpoint{3.696000in}{3.696000in}}%
\pgfusepath{clip}%
\pgfsetrectcap%
\pgfsetroundjoin%
\pgfsetlinewidth{1.505625pt}%
\definecolor{currentstroke}{rgb}{1.000000,0.000000,0.000000}%
\pgfsetstrokecolor{currentstroke}%
\pgfsetdash{}{0pt}%
\pgfpathmoveto{\pgfqpoint{2.507180in}{1.747140in}}%
\pgfpathlineto{\pgfqpoint{2.376900in}{0.848592in}}%
\pgfusepath{stroke}%
\end{pgfscope}%
\begin{pgfscope}%
\pgfpathrectangle{\pgfqpoint{0.100000in}{0.212622in}}{\pgfqpoint{3.696000in}{3.696000in}}%
\pgfusepath{clip}%
\pgfsetrectcap%
\pgfsetroundjoin%
\pgfsetlinewidth{1.505625pt}%
\definecolor{currentstroke}{rgb}{1.000000,0.000000,0.000000}%
\pgfsetstrokecolor{currentstroke}%
\pgfsetdash{}{0pt}%
\pgfpathmoveto{\pgfqpoint{2.503182in}{1.743562in}}%
\pgfpathlineto{\pgfqpoint{2.376900in}{0.848592in}}%
\pgfusepath{stroke}%
\end{pgfscope}%
\begin{pgfscope}%
\pgfpathrectangle{\pgfqpoint{0.100000in}{0.212622in}}{\pgfqpoint{3.696000in}{3.696000in}}%
\pgfusepath{clip}%
\pgfsetrectcap%
\pgfsetroundjoin%
\pgfsetlinewidth{1.505625pt}%
\definecolor{currentstroke}{rgb}{1.000000,0.000000,0.000000}%
\pgfsetstrokecolor{currentstroke}%
\pgfsetdash{}{0pt}%
\pgfpathmoveto{\pgfqpoint{2.498862in}{1.738983in}}%
\pgfpathlineto{\pgfqpoint{2.376900in}{0.848592in}}%
\pgfusepath{stroke}%
\end{pgfscope}%
\begin{pgfscope}%
\pgfpathrectangle{\pgfqpoint{0.100000in}{0.212622in}}{\pgfqpoint{3.696000in}{3.696000in}}%
\pgfusepath{clip}%
\pgfsetrectcap%
\pgfsetroundjoin%
\pgfsetlinewidth{1.505625pt}%
\definecolor{currentstroke}{rgb}{1.000000,0.000000,0.000000}%
\pgfsetstrokecolor{currentstroke}%
\pgfsetdash{}{0pt}%
\pgfpathmoveto{\pgfqpoint{2.493066in}{1.732558in}}%
\pgfpathlineto{\pgfqpoint{2.376900in}{0.848592in}}%
\pgfusepath{stroke}%
\end{pgfscope}%
\begin{pgfscope}%
\pgfpathrectangle{\pgfqpoint{0.100000in}{0.212622in}}{\pgfqpoint{3.696000in}{3.696000in}}%
\pgfusepath{clip}%
\pgfsetrectcap%
\pgfsetroundjoin%
\pgfsetlinewidth{1.505625pt}%
\definecolor{currentstroke}{rgb}{1.000000,0.000000,0.000000}%
\pgfsetstrokecolor{currentstroke}%
\pgfsetdash{}{0pt}%
\pgfpathmoveto{\pgfqpoint{2.487897in}{1.726105in}}%
\pgfpathlineto{\pgfqpoint{2.376900in}{0.848592in}}%
\pgfusepath{stroke}%
\end{pgfscope}%
\begin{pgfscope}%
\pgfpathrectangle{\pgfqpoint{0.100000in}{0.212622in}}{\pgfqpoint{3.696000in}{3.696000in}}%
\pgfusepath{clip}%
\pgfsetrectcap%
\pgfsetroundjoin%
\pgfsetlinewidth{1.505625pt}%
\definecolor{currentstroke}{rgb}{1.000000,0.000000,0.000000}%
\pgfsetstrokecolor{currentstroke}%
\pgfsetdash{}{0pt}%
\pgfpathmoveto{\pgfqpoint{2.482385in}{1.720395in}}%
\pgfpathlineto{\pgfqpoint{2.376900in}{0.848592in}}%
\pgfusepath{stroke}%
\end{pgfscope}%
\begin{pgfscope}%
\pgfpathrectangle{\pgfqpoint{0.100000in}{0.212622in}}{\pgfqpoint{3.696000in}{3.696000in}}%
\pgfusepath{clip}%
\pgfsetrectcap%
\pgfsetroundjoin%
\pgfsetlinewidth{1.505625pt}%
\definecolor{currentstroke}{rgb}{1.000000,0.000000,0.000000}%
\pgfsetstrokecolor{currentstroke}%
\pgfsetdash{}{0pt}%
\pgfpathmoveto{\pgfqpoint{2.479447in}{1.716158in}}%
\pgfpathlineto{\pgfqpoint{2.376900in}{0.848592in}}%
\pgfusepath{stroke}%
\end{pgfscope}%
\begin{pgfscope}%
\pgfpathrectangle{\pgfqpoint{0.100000in}{0.212622in}}{\pgfqpoint{3.696000in}{3.696000in}}%
\pgfusepath{clip}%
\pgfsetrectcap%
\pgfsetroundjoin%
\pgfsetlinewidth{1.505625pt}%
\definecolor{currentstroke}{rgb}{1.000000,0.000000,0.000000}%
\pgfsetstrokecolor{currentstroke}%
\pgfsetdash{}{0pt}%
\pgfpathmoveto{\pgfqpoint{2.477771in}{1.714142in}}%
\pgfpathlineto{\pgfqpoint{2.376900in}{0.848592in}}%
\pgfusepath{stroke}%
\end{pgfscope}%
\begin{pgfscope}%
\pgfpathrectangle{\pgfqpoint{0.100000in}{0.212622in}}{\pgfqpoint{3.696000in}{3.696000in}}%
\pgfusepath{clip}%
\pgfsetrectcap%
\pgfsetroundjoin%
\pgfsetlinewidth{1.505625pt}%
\definecolor{currentstroke}{rgb}{1.000000,0.000000,0.000000}%
\pgfsetstrokecolor{currentstroke}%
\pgfsetdash{}{0pt}%
\pgfpathmoveto{\pgfqpoint{2.475900in}{1.711501in}}%
\pgfpathlineto{\pgfqpoint{2.376900in}{0.848592in}}%
\pgfusepath{stroke}%
\end{pgfscope}%
\begin{pgfscope}%
\pgfpathrectangle{\pgfqpoint{0.100000in}{0.212622in}}{\pgfqpoint{3.696000in}{3.696000in}}%
\pgfusepath{clip}%
\pgfsetrectcap%
\pgfsetroundjoin%
\pgfsetlinewidth{1.505625pt}%
\definecolor{currentstroke}{rgb}{1.000000,0.000000,0.000000}%
\pgfsetstrokecolor{currentstroke}%
\pgfsetdash{}{0pt}%
\pgfpathmoveto{\pgfqpoint{2.474837in}{1.710163in}}%
\pgfpathlineto{\pgfqpoint{2.376900in}{0.848592in}}%
\pgfusepath{stroke}%
\end{pgfscope}%
\begin{pgfscope}%
\pgfpathrectangle{\pgfqpoint{0.100000in}{0.212622in}}{\pgfqpoint{3.696000in}{3.696000in}}%
\pgfusepath{clip}%
\pgfsetrectcap%
\pgfsetroundjoin%
\pgfsetlinewidth{1.505625pt}%
\definecolor{currentstroke}{rgb}{1.000000,0.000000,0.000000}%
\pgfsetstrokecolor{currentstroke}%
\pgfsetdash{}{0pt}%
\pgfpathmoveto{\pgfqpoint{2.474216in}{1.709346in}}%
\pgfpathlineto{\pgfqpoint{2.376900in}{0.848592in}}%
\pgfusepath{stroke}%
\end{pgfscope}%
\begin{pgfscope}%
\pgfpathrectangle{\pgfqpoint{0.100000in}{0.212622in}}{\pgfqpoint{3.696000in}{3.696000in}}%
\pgfusepath{clip}%
\pgfsetrectcap%
\pgfsetroundjoin%
\pgfsetlinewidth{1.505625pt}%
\definecolor{currentstroke}{rgb}{1.000000,0.000000,0.000000}%
\pgfsetstrokecolor{currentstroke}%
\pgfsetdash{}{0pt}%
\pgfpathmoveto{\pgfqpoint{2.473869in}{1.708975in}}%
\pgfpathlineto{\pgfqpoint{2.376900in}{0.848592in}}%
\pgfusepath{stroke}%
\end{pgfscope}%
\begin{pgfscope}%
\pgfpathrectangle{\pgfqpoint{0.100000in}{0.212622in}}{\pgfqpoint{3.696000in}{3.696000in}}%
\pgfusepath{clip}%
\pgfsetrectcap%
\pgfsetroundjoin%
\pgfsetlinewidth{1.505625pt}%
\definecolor{currentstroke}{rgb}{1.000000,0.000000,0.000000}%
\pgfsetstrokecolor{currentstroke}%
\pgfsetdash{}{0pt}%
\pgfpathmoveto{\pgfqpoint{2.473690in}{1.708780in}}%
\pgfpathlineto{\pgfqpoint{2.376900in}{0.848592in}}%
\pgfusepath{stroke}%
\end{pgfscope}%
\begin{pgfscope}%
\pgfpathrectangle{\pgfqpoint{0.100000in}{0.212622in}}{\pgfqpoint{3.696000in}{3.696000in}}%
\pgfusepath{clip}%
\pgfsetrectcap%
\pgfsetroundjoin%
\pgfsetlinewidth{1.505625pt}%
\definecolor{currentstroke}{rgb}{1.000000,0.000000,0.000000}%
\pgfsetstrokecolor{currentstroke}%
\pgfsetdash{}{0pt}%
\pgfpathmoveto{\pgfqpoint{2.473584in}{1.708677in}}%
\pgfpathlineto{\pgfqpoint{2.376900in}{0.848592in}}%
\pgfusepath{stroke}%
\end{pgfscope}%
\begin{pgfscope}%
\pgfpathrectangle{\pgfqpoint{0.100000in}{0.212622in}}{\pgfqpoint{3.696000in}{3.696000in}}%
\pgfusepath{clip}%
\pgfsetrectcap%
\pgfsetroundjoin%
\pgfsetlinewidth{1.505625pt}%
\definecolor{currentstroke}{rgb}{1.000000,0.000000,0.000000}%
\pgfsetstrokecolor{currentstroke}%
\pgfsetdash{}{0pt}%
\pgfpathmoveto{\pgfqpoint{2.473524in}{1.708627in}}%
\pgfpathlineto{\pgfqpoint{2.376900in}{0.848592in}}%
\pgfusepath{stroke}%
\end{pgfscope}%
\begin{pgfscope}%
\pgfpathrectangle{\pgfqpoint{0.100000in}{0.212622in}}{\pgfqpoint{3.696000in}{3.696000in}}%
\pgfusepath{clip}%
\pgfsetrectcap%
\pgfsetroundjoin%
\pgfsetlinewidth{1.505625pt}%
\definecolor{currentstroke}{rgb}{1.000000,0.000000,0.000000}%
\pgfsetstrokecolor{currentstroke}%
\pgfsetdash{}{0pt}%
\pgfpathmoveto{\pgfqpoint{2.473488in}{1.708605in}}%
\pgfpathlineto{\pgfqpoint{2.376900in}{0.848592in}}%
\pgfusepath{stroke}%
\end{pgfscope}%
\begin{pgfscope}%
\pgfpathrectangle{\pgfqpoint{0.100000in}{0.212622in}}{\pgfqpoint{3.696000in}{3.696000in}}%
\pgfusepath{clip}%
\pgfsetrectcap%
\pgfsetroundjoin%
\pgfsetlinewidth{1.505625pt}%
\definecolor{currentstroke}{rgb}{1.000000,0.000000,0.000000}%
\pgfsetstrokecolor{currentstroke}%
\pgfsetdash{}{0pt}%
\pgfpathmoveto{\pgfqpoint{2.473466in}{1.708592in}}%
\pgfpathlineto{\pgfqpoint{2.376900in}{0.848592in}}%
\pgfusepath{stroke}%
\end{pgfscope}%
\begin{pgfscope}%
\pgfpathrectangle{\pgfqpoint{0.100000in}{0.212622in}}{\pgfqpoint{3.696000in}{3.696000in}}%
\pgfusepath{clip}%
\pgfsetrectcap%
\pgfsetroundjoin%
\pgfsetlinewidth{1.505625pt}%
\definecolor{currentstroke}{rgb}{1.000000,0.000000,0.000000}%
\pgfsetstrokecolor{currentstroke}%
\pgfsetdash{}{0pt}%
\pgfpathmoveto{\pgfqpoint{2.473453in}{1.708586in}}%
\pgfpathlineto{\pgfqpoint{2.376900in}{0.848592in}}%
\pgfusepath{stroke}%
\end{pgfscope}%
\begin{pgfscope}%
\pgfpathrectangle{\pgfqpoint{0.100000in}{0.212622in}}{\pgfqpoint{3.696000in}{3.696000in}}%
\pgfusepath{clip}%
\pgfsetrectcap%
\pgfsetroundjoin%
\pgfsetlinewidth{1.505625pt}%
\definecolor{currentstroke}{rgb}{1.000000,0.000000,0.000000}%
\pgfsetstrokecolor{currentstroke}%
\pgfsetdash{}{0pt}%
\pgfpathmoveto{\pgfqpoint{2.471617in}{1.707771in}}%
\pgfpathlineto{\pgfqpoint{2.376900in}{0.848592in}}%
\pgfusepath{stroke}%
\end{pgfscope}%
\begin{pgfscope}%
\pgfpathrectangle{\pgfqpoint{0.100000in}{0.212622in}}{\pgfqpoint{3.696000in}{3.696000in}}%
\pgfusepath{clip}%
\pgfsetrectcap%
\pgfsetroundjoin%
\pgfsetlinewidth{1.505625pt}%
\definecolor{currentstroke}{rgb}{1.000000,0.000000,0.000000}%
\pgfsetstrokecolor{currentstroke}%
\pgfsetdash{}{0pt}%
\pgfpathmoveto{\pgfqpoint{2.470525in}{1.707432in}}%
\pgfpathlineto{\pgfqpoint{2.376900in}{0.848592in}}%
\pgfusepath{stroke}%
\end{pgfscope}%
\begin{pgfscope}%
\pgfpathrectangle{\pgfqpoint{0.100000in}{0.212622in}}{\pgfqpoint{3.696000in}{3.696000in}}%
\pgfusepath{clip}%
\pgfsetrectcap%
\pgfsetroundjoin%
\pgfsetlinewidth{1.505625pt}%
\definecolor{currentstroke}{rgb}{1.000000,0.000000,0.000000}%
\pgfsetstrokecolor{currentstroke}%
\pgfsetdash{}{0pt}%
\pgfpathmoveto{\pgfqpoint{2.468934in}{1.707139in}}%
\pgfpathlineto{\pgfqpoint{2.376900in}{0.848592in}}%
\pgfusepath{stroke}%
\end{pgfscope}%
\begin{pgfscope}%
\pgfpathrectangle{\pgfqpoint{0.100000in}{0.212622in}}{\pgfqpoint{3.696000in}{3.696000in}}%
\pgfusepath{clip}%
\pgfsetrectcap%
\pgfsetroundjoin%
\pgfsetlinewidth{1.505625pt}%
\definecolor{currentstroke}{rgb}{1.000000,0.000000,0.000000}%
\pgfsetstrokecolor{currentstroke}%
\pgfsetdash{}{0pt}%
\pgfpathmoveto{\pgfqpoint{2.466216in}{1.706927in}}%
\pgfpathlineto{\pgfqpoint{2.376900in}{0.848592in}}%
\pgfusepath{stroke}%
\end{pgfscope}%
\begin{pgfscope}%
\pgfpathrectangle{\pgfqpoint{0.100000in}{0.212622in}}{\pgfqpoint{3.696000in}{3.696000in}}%
\pgfusepath{clip}%
\pgfsetrectcap%
\pgfsetroundjoin%
\pgfsetlinewidth{1.505625pt}%
\definecolor{currentstroke}{rgb}{1.000000,0.000000,0.000000}%
\pgfsetstrokecolor{currentstroke}%
\pgfsetdash{}{0pt}%
\pgfpathmoveto{\pgfqpoint{2.464634in}{1.706984in}}%
\pgfpathlineto{\pgfqpoint{2.376900in}{0.848592in}}%
\pgfusepath{stroke}%
\end{pgfscope}%
\begin{pgfscope}%
\pgfpathrectangle{\pgfqpoint{0.100000in}{0.212622in}}{\pgfqpoint{3.696000in}{3.696000in}}%
\pgfusepath{clip}%
\pgfsetrectcap%
\pgfsetroundjoin%
\pgfsetlinewidth{1.505625pt}%
\definecolor{currentstroke}{rgb}{1.000000,0.000000,0.000000}%
\pgfsetstrokecolor{currentstroke}%
\pgfsetdash{}{0pt}%
\pgfpathmoveto{\pgfqpoint{2.462481in}{1.707310in}}%
\pgfpathlineto{\pgfqpoint{2.376900in}{0.848592in}}%
\pgfusepath{stroke}%
\end{pgfscope}%
\begin{pgfscope}%
\pgfpathrectangle{\pgfqpoint{0.100000in}{0.212622in}}{\pgfqpoint{3.696000in}{3.696000in}}%
\pgfusepath{clip}%
\pgfsetrectcap%
\pgfsetroundjoin%
\pgfsetlinewidth{1.505625pt}%
\definecolor{currentstroke}{rgb}{1.000000,0.000000,0.000000}%
\pgfsetstrokecolor{currentstroke}%
\pgfsetdash{}{0pt}%
\pgfpathmoveto{\pgfqpoint{2.461235in}{1.707609in}}%
\pgfpathlineto{\pgfqpoint{2.376900in}{0.848592in}}%
\pgfusepath{stroke}%
\end{pgfscope}%
\begin{pgfscope}%
\pgfpathrectangle{\pgfqpoint{0.100000in}{0.212622in}}{\pgfqpoint{3.696000in}{3.696000in}}%
\pgfusepath{clip}%
\pgfsetrectcap%
\pgfsetroundjoin%
\pgfsetlinewidth{1.505625pt}%
\definecolor{currentstroke}{rgb}{1.000000,0.000000,0.000000}%
\pgfsetstrokecolor{currentstroke}%
\pgfsetdash{}{0pt}%
\pgfpathmoveto{\pgfqpoint{2.460520in}{1.707830in}}%
\pgfpathlineto{\pgfqpoint{2.376900in}{0.848592in}}%
\pgfusepath{stroke}%
\end{pgfscope}%
\begin{pgfscope}%
\pgfpathrectangle{\pgfqpoint{0.100000in}{0.212622in}}{\pgfqpoint{3.696000in}{3.696000in}}%
\pgfusepath{clip}%
\pgfsetrectcap%
\pgfsetroundjoin%
\pgfsetlinewidth{1.505625pt}%
\definecolor{currentstroke}{rgb}{1.000000,0.000000,0.000000}%
\pgfsetstrokecolor{currentstroke}%
\pgfsetdash{}{0pt}%
\pgfpathmoveto{\pgfqpoint{2.460112in}{1.707985in}}%
\pgfpathlineto{\pgfqpoint{2.376900in}{0.848592in}}%
\pgfusepath{stroke}%
\end{pgfscope}%
\begin{pgfscope}%
\pgfpathrectangle{\pgfqpoint{0.100000in}{0.212622in}}{\pgfqpoint{3.696000in}{3.696000in}}%
\pgfusepath{clip}%
\pgfsetrectcap%
\pgfsetroundjoin%
\pgfsetlinewidth{1.505625pt}%
\definecolor{currentstroke}{rgb}{1.000000,0.000000,0.000000}%
\pgfsetstrokecolor{currentstroke}%
\pgfsetdash{}{0pt}%
\pgfpathmoveto{\pgfqpoint{2.459882in}{1.708089in}}%
\pgfpathlineto{\pgfqpoint{2.376900in}{0.848592in}}%
\pgfusepath{stroke}%
\end{pgfscope}%
\begin{pgfscope}%
\pgfpathrectangle{\pgfqpoint{0.100000in}{0.212622in}}{\pgfqpoint{3.696000in}{3.696000in}}%
\pgfusepath{clip}%
\pgfsetrectcap%
\pgfsetroundjoin%
\pgfsetlinewidth{1.505625pt}%
\definecolor{currentstroke}{rgb}{1.000000,0.000000,0.000000}%
\pgfsetstrokecolor{currentstroke}%
\pgfsetdash{}{0pt}%
\pgfpathmoveto{\pgfqpoint{2.459753in}{1.708155in}}%
\pgfpathlineto{\pgfqpoint{2.376900in}{0.848592in}}%
\pgfusepath{stroke}%
\end{pgfscope}%
\begin{pgfscope}%
\pgfpathrectangle{\pgfqpoint{0.100000in}{0.212622in}}{\pgfqpoint{3.696000in}{3.696000in}}%
\pgfusepath{clip}%
\pgfsetrectcap%
\pgfsetroundjoin%
\pgfsetlinewidth{1.505625pt}%
\definecolor{currentstroke}{rgb}{1.000000,0.000000,0.000000}%
\pgfsetstrokecolor{currentstroke}%
\pgfsetdash{}{0pt}%
\pgfpathmoveto{\pgfqpoint{2.459681in}{1.708193in}}%
\pgfpathlineto{\pgfqpoint{2.376900in}{0.848592in}}%
\pgfusepath{stroke}%
\end{pgfscope}%
\begin{pgfscope}%
\pgfpathrectangle{\pgfqpoint{0.100000in}{0.212622in}}{\pgfqpoint{3.696000in}{3.696000in}}%
\pgfusepath{clip}%
\pgfsetrectcap%
\pgfsetroundjoin%
\pgfsetlinewidth{1.505625pt}%
\definecolor{currentstroke}{rgb}{1.000000,0.000000,0.000000}%
\pgfsetstrokecolor{currentstroke}%
\pgfsetdash{}{0pt}%
\pgfpathmoveto{\pgfqpoint{2.456804in}{1.709854in}}%
\pgfpathlineto{\pgfqpoint{2.376900in}{0.848592in}}%
\pgfusepath{stroke}%
\end{pgfscope}%
\begin{pgfscope}%
\pgfpathrectangle{\pgfqpoint{0.100000in}{0.212622in}}{\pgfqpoint{3.696000in}{3.696000in}}%
\pgfusepath{clip}%
\pgfsetrectcap%
\pgfsetroundjoin%
\pgfsetlinewidth{1.505625pt}%
\definecolor{currentstroke}{rgb}{1.000000,0.000000,0.000000}%
\pgfsetstrokecolor{currentstroke}%
\pgfsetdash{}{0pt}%
\pgfpathmoveto{\pgfqpoint{2.451218in}{1.713295in}}%
\pgfpathlineto{\pgfqpoint{2.376900in}{0.848592in}}%
\pgfusepath{stroke}%
\end{pgfscope}%
\begin{pgfscope}%
\pgfpathrectangle{\pgfqpoint{0.100000in}{0.212622in}}{\pgfqpoint{3.696000in}{3.696000in}}%
\pgfusepath{clip}%
\pgfsetrectcap%
\pgfsetroundjoin%
\pgfsetlinewidth{1.505625pt}%
\definecolor{currentstroke}{rgb}{1.000000,0.000000,0.000000}%
\pgfsetstrokecolor{currentstroke}%
\pgfsetdash{}{0pt}%
\pgfpathmoveto{\pgfqpoint{2.443972in}{1.717691in}}%
\pgfpathlineto{\pgfqpoint{2.376900in}{0.848592in}}%
\pgfusepath{stroke}%
\end{pgfscope}%
\begin{pgfscope}%
\pgfpathrectangle{\pgfqpoint{0.100000in}{0.212622in}}{\pgfqpoint{3.696000in}{3.696000in}}%
\pgfusepath{clip}%
\pgfsetrectcap%
\pgfsetroundjoin%
\pgfsetlinewidth{1.505625pt}%
\definecolor{currentstroke}{rgb}{1.000000,0.000000,0.000000}%
\pgfsetstrokecolor{currentstroke}%
\pgfsetdash{}{0pt}%
\pgfpathmoveto{\pgfqpoint{2.435786in}{1.722360in}}%
\pgfpathlineto{\pgfqpoint{2.376900in}{0.848592in}}%
\pgfusepath{stroke}%
\end{pgfscope}%
\begin{pgfscope}%
\pgfpathrectangle{\pgfqpoint{0.100000in}{0.212622in}}{\pgfqpoint{3.696000in}{3.696000in}}%
\pgfusepath{clip}%
\pgfsetrectcap%
\pgfsetroundjoin%
\pgfsetlinewidth{1.505625pt}%
\definecolor{currentstroke}{rgb}{1.000000,0.000000,0.000000}%
\pgfsetstrokecolor{currentstroke}%
\pgfsetdash{}{0pt}%
\pgfpathmoveto{\pgfqpoint{2.426744in}{1.727583in}}%
\pgfpathlineto{\pgfqpoint{2.376900in}{0.848592in}}%
\pgfusepath{stroke}%
\end{pgfscope}%
\begin{pgfscope}%
\pgfpathrectangle{\pgfqpoint{0.100000in}{0.212622in}}{\pgfqpoint{3.696000in}{3.696000in}}%
\pgfusepath{clip}%
\pgfsetrectcap%
\pgfsetroundjoin%
\pgfsetlinewidth{1.505625pt}%
\definecolor{currentstroke}{rgb}{1.000000,0.000000,0.000000}%
\pgfsetstrokecolor{currentstroke}%
\pgfsetdash{}{0pt}%
\pgfpathmoveto{\pgfqpoint{2.414922in}{1.734235in}}%
\pgfpathlineto{\pgfqpoint{2.376900in}{0.848592in}}%
\pgfusepath{stroke}%
\end{pgfscope}%
\begin{pgfscope}%
\pgfpathrectangle{\pgfqpoint{0.100000in}{0.212622in}}{\pgfqpoint{3.696000in}{3.696000in}}%
\pgfusepath{clip}%
\pgfsetrectcap%
\pgfsetroundjoin%
\pgfsetlinewidth{1.505625pt}%
\definecolor{currentstroke}{rgb}{1.000000,0.000000,0.000000}%
\pgfsetstrokecolor{currentstroke}%
\pgfsetdash{}{0pt}%
\pgfpathmoveto{\pgfqpoint{2.399567in}{1.741988in}}%
\pgfpathlineto{\pgfqpoint{2.376900in}{0.848592in}}%
\pgfusepath{stroke}%
\end{pgfscope}%
\begin{pgfscope}%
\pgfpathrectangle{\pgfqpoint{0.100000in}{0.212622in}}{\pgfqpoint{3.696000in}{3.696000in}}%
\pgfusepath{clip}%
\pgfsetrectcap%
\pgfsetroundjoin%
\pgfsetlinewidth{1.505625pt}%
\definecolor{currentstroke}{rgb}{1.000000,0.000000,0.000000}%
\pgfsetstrokecolor{currentstroke}%
\pgfsetdash{}{0pt}%
\pgfpathmoveto{\pgfqpoint{2.383112in}{1.752000in}}%
\pgfpathlineto{\pgfqpoint{2.376900in}{0.848592in}}%
\pgfusepath{stroke}%
\end{pgfscope}%
\begin{pgfscope}%
\pgfpathrectangle{\pgfqpoint{0.100000in}{0.212622in}}{\pgfqpoint{3.696000in}{3.696000in}}%
\pgfusepath{clip}%
\pgfsetrectcap%
\pgfsetroundjoin%
\pgfsetlinewidth{1.505625pt}%
\definecolor{currentstroke}{rgb}{1.000000,0.000000,0.000000}%
\pgfsetstrokecolor{currentstroke}%
\pgfsetdash{}{0pt}%
\pgfpathmoveto{\pgfqpoint{2.374086in}{1.756937in}}%
\pgfpathlineto{\pgfqpoint{2.376900in}{0.848592in}}%
\pgfusepath{stroke}%
\end{pgfscope}%
\begin{pgfscope}%
\pgfpathrectangle{\pgfqpoint{0.100000in}{0.212622in}}{\pgfqpoint{3.696000in}{3.696000in}}%
\pgfusepath{clip}%
\pgfsetrectcap%
\pgfsetroundjoin%
\pgfsetlinewidth{1.505625pt}%
\definecolor{currentstroke}{rgb}{1.000000,0.000000,0.000000}%
\pgfsetstrokecolor{currentstroke}%
\pgfsetdash{}{0pt}%
\pgfpathmoveto{\pgfqpoint{2.362493in}{1.763258in}}%
\pgfpathlineto{\pgfqpoint{2.376900in}{0.848592in}}%
\pgfusepath{stroke}%
\end{pgfscope}%
\begin{pgfscope}%
\pgfpathrectangle{\pgfqpoint{0.100000in}{0.212622in}}{\pgfqpoint{3.696000in}{3.696000in}}%
\pgfusepath{clip}%
\pgfsetrectcap%
\pgfsetroundjoin%
\pgfsetlinewidth{1.505625pt}%
\definecolor{currentstroke}{rgb}{1.000000,0.000000,0.000000}%
\pgfsetstrokecolor{currentstroke}%
\pgfsetdash{}{0pt}%
\pgfpathmoveto{\pgfqpoint{2.347146in}{1.772710in}}%
\pgfpathlineto{\pgfqpoint{2.376900in}{0.848592in}}%
\pgfusepath{stroke}%
\end{pgfscope}%
\begin{pgfscope}%
\pgfpathrectangle{\pgfqpoint{0.100000in}{0.212622in}}{\pgfqpoint{3.696000in}{3.696000in}}%
\pgfusepath{clip}%
\pgfsetrectcap%
\pgfsetroundjoin%
\pgfsetlinewidth{1.505625pt}%
\definecolor{currentstroke}{rgb}{1.000000,0.000000,0.000000}%
\pgfsetstrokecolor{currentstroke}%
\pgfsetdash{}{0pt}%
\pgfpathmoveto{\pgfqpoint{2.330917in}{1.780891in}}%
\pgfpathlineto{\pgfqpoint{2.376900in}{0.848592in}}%
\pgfusepath{stroke}%
\end{pgfscope}%
\begin{pgfscope}%
\pgfpathrectangle{\pgfqpoint{0.100000in}{0.212622in}}{\pgfqpoint{3.696000in}{3.696000in}}%
\pgfusepath{clip}%
\pgfsetrectcap%
\pgfsetroundjoin%
\pgfsetlinewidth{1.505625pt}%
\definecolor{currentstroke}{rgb}{1.000000,0.000000,0.000000}%
\pgfsetstrokecolor{currentstroke}%
\pgfsetdash{}{0pt}%
\pgfpathmoveto{\pgfqpoint{2.313989in}{1.789205in}}%
\pgfpathlineto{\pgfqpoint{2.376900in}{0.848592in}}%
\pgfusepath{stroke}%
\end{pgfscope}%
\begin{pgfscope}%
\pgfpathrectangle{\pgfqpoint{0.100000in}{0.212622in}}{\pgfqpoint{3.696000in}{3.696000in}}%
\pgfusepath{clip}%
\pgfsetrectcap%
\pgfsetroundjoin%
\pgfsetlinewidth{1.505625pt}%
\definecolor{currentstroke}{rgb}{1.000000,0.000000,0.000000}%
\pgfsetstrokecolor{currentstroke}%
\pgfsetdash{}{0pt}%
\pgfpathmoveto{\pgfqpoint{2.293481in}{1.799162in}}%
\pgfpathlineto{\pgfqpoint{2.376900in}{0.848592in}}%
\pgfusepath{stroke}%
\end{pgfscope}%
\begin{pgfscope}%
\pgfpathrectangle{\pgfqpoint{0.100000in}{0.212622in}}{\pgfqpoint{3.696000in}{3.696000in}}%
\pgfusepath{clip}%
\pgfsetrectcap%
\pgfsetroundjoin%
\pgfsetlinewidth{1.505625pt}%
\definecolor{currentstroke}{rgb}{1.000000,0.000000,0.000000}%
\pgfsetstrokecolor{currentstroke}%
\pgfsetdash{}{0pt}%
\pgfpathmoveto{\pgfqpoint{2.271061in}{1.810752in}}%
\pgfpathlineto{\pgfqpoint{2.361989in}{0.853322in}}%
\pgfusepath{stroke}%
\end{pgfscope}%
\begin{pgfscope}%
\pgfpathrectangle{\pgfqpoint{0.100000in}{0.212622in}}{\pgfqpoint{3.696000in}{3.696000in}}%
\pgfusepath{clip}%
\pgfsetrectcap%
\pgfsetroundjoin%
\pgfsetlinewidth{1.505625pt}%
\definecolor{currentstroke}{rgb}{1.000000,0.000000,0.000000}%
\pgfsetstrokecolor{currentstroke}%
\pgfsetdash{}{0pt}%
\pgfpathmoveto{\pgfqpoint{2.258858in}{1.818738in}}%
\pgfpathlineto{\pgfqpoint{2.347088in}{0.858049in}}%
\pgfusepath{stroke}%
\end{pgfscope}%
\begin{pgfscope}%
\pgfpathrectangle{\pgfqpoint{0.100000in}{0.212622in}}{\pgfqpoint{3.696000in}{3.696000in}}%
\pgfusepath{clip}%
\pgfsetrectcap%
\pgfsetroundjoin%
\pgfsetlinewidth{1.505625pt}%
\definecolor{currentstroke}{rgb}{1.000000,0.000000,0.000000}%
\pgfsetstrokecolor{currentstroke}%
\pgfsetdash{}{0pt}%
\pgfpathmoveto{\pgfqpoint{2.252137in}{1.822586in}}%
\pgfpathlineto{\pgfqpoint{2.347088in}{0.858049in}}%
\pgfusepath{stroke}%
\end{pgfscope}%
\begin{pgfscope}%
\pgfpathrectangle{\pgfqpoint{0.100000in}{0.212622in}}{\pgfqpoint{3.696000in}{3.696000in}}%
\pgfusepath{clip}%
\pgfsetrectcap%
\pgfsetroundjoin%
\pgfsetlinewidth{1.505625pt}%
\definecolor{currentstroke}{rgb}{1.000000,0.000000,0.000000}%
\pgfsetstrokecolor{currentstroke}%
\pgfsetdash{}{0pt}%
\pgfpathmoveto{\pgfqpoint{2.243908in}{1.827132in}}%
\pgfpathlineto{\pgfqpoint{2.332199in}{0.862773in}}%
\pgfusepath{stroke}%
\end{pgfscope}%
\begin{pgfscope}%
\pgfpathrectangle{\pgfqpoint{0.100000in}{0.212622in}}{\pgfqpoint{3.696000in}{3.696000in}}%
\pgfusepath{clip}%
\pgfsetrectcap%
\pgfsetroundjoin%
\pgfsetlinewidth{1.505625pt}%
\definecolor{currentstroke}{rgb}{1.000000,0.000000,0.000000}%
\pgfsetstrokecolor{currentstroke}%
\pgfsetdash{}{0pt}%
\pgfpathmoveto{\pgfqpoint{2.232287in}{1.832636in}}%
\pgfpathlineto{\pgfqpoint{2.332199in}{0.862773in}}%
\pgfusepath{stroke}%
\end{pgfscope}%
\begin{pgfscope}%
\pgfpathrectangle{\pgfqpoint{0.100000in}{0.212622in}}{\pgfqpoint{3.696000in}{3.696000in}}%
\pgfusepath{clip}%
\pgfsetrectcap%
\pgfsetroundjoin%
\pgfsetlinewidth{1.505625pt}%
\definecolor{currentstroke}{rgb}{1.000000,0.000000,0.000000}%
\pgfsetstrokecolor{currentstroke}%
\pgfsetdash{}{0pt}%
\pgfpathmoveto{\pgfqpoint{2.218876in}{1.836736in}}%
\pgfpathlineto{\pgfqpoint{2.317321in}{0.867493in}}%
\pgfusepath{stroke}%
\end{pgfscope}%
\begin{pgfscope}%
\pgfpathrectangle{\pgfqpoint{0.100000in}{0.212622in}}{\pgfqpoint{3.696000in}{3.696000in}}%
\pgfusepath{clip}%
\pgfsetrectcap%
\pgfsetroundjoin%
\pgfsetlinewidth{1.505625pt}%
\definecolor{currentstroke}{rgb}{1.000000,0.000000,0.000000}%
\pgfsetstrokecolor{currentstroke}%
\pgfsetdash{}{0pt}%
\pgfpathmoveto{\pgfqpoint{2.211541in}{1.839159in}}%
\pgfpathlineto{\pgfqpoint{2.302454in}{0.872210in}}%
\pgfusepath{stroke}%
\end{pgfscope}%
\begin{pgfscope}%
\pgfpathrectangle{\pgfqpoint{0.100000in}{0.212622in}}{\pgfqpoint{3.696000in}{3.696000in}}%
\pgfusepath{clip}%
\pgfsetrectcap%
\pgfsetroundjoin%
\pgfsetlinewidth{1.505625pt}%
\definecolor{currentstroke}{rgb}{1.000000,0.000000,0.000000}%
\pgfsetstrokecolor{currentstroke}%
\pgfsetdash{}{0pt}%
\pgfpathmoveto{\pgfqpoint{2.203110in}{1.841983in}}%
\pgfpathlineto{\pgfqpoint{2.302454in}{0.872210in}}%
\pgfusepath{stroke}%
\end{pgfscope}%
\begin{pgfscope}%
\pgfpathrectangle{\pgfqpoint{0.100000in}{0.212622in}}{\pgfqpoint{3.696000in}{3.696000in}}%
\pgfusepath{clip}%
\pgfsetrectcap%
\pgfsetroundjoin%
\pgfsetlinewidth{1.505625pt}%
\definecolor{currentstroke}{rgb}{1.000000,0.000000,0.000000}%
\pgfsetstrokecolor{currentstroke}%
\pgfsetdash{}{0pt}%
\pgfpathmoveto{\pgfqpoint{2.192945in}{1.845977in}}%
\pgfpathlineto{\pgfqpoint{2.287597in}{0.876923in}}%
\pgfusepath{stroke}%
\end{pgfscope}%
\begin{pgfscope}%
\pgfpathrectangle{\pgfqpoint{0.100000in}{0.212622in}}{\pgfqpoint{3.696000in}{3.696000in}}%
\pgfusepath{clip}%
\pgfsetrectcap%
\pgfsetroundjoin%
\pgfsetlinewidth{1.505625pt}%
\definecolor{currentstroke}{rgb}{1.000000,0.000000,0.000000}%
\pgfsetstrokecolor{currentstroke}%
\pgfsetdash{}{0pt}%
\pgfpathmoveto{\pgfqpoint{2.181689in}{1.851720in}}%
\pgfpathlineto{\pgfqpoint{2.272752in}{0.881633in}}%
\pgfusepath{stroke}%
\end{pgfscope}%
\begin{pgfscope}%
\pgfpathrectangle{\pgfqpoint{0.100000in}{0.212622in}}{\pgfqpoint{3.696000in}{3.696000in}}%
\pgfusepath{clip}%
\pgfsetrectcap%
\pgfsetroundjoin%
\pgfsetlinewidth{1.505625pt}%
\definecolor{currentstroke}{rgb}{1.000000,0.000000,0.000000}%
\pgfsetstrokecolor{currentstroke}%
\pgfsetdash{}{0pt}%
\pgfpathmoveto{\pgfqpoint{2.169352in}{1.857145in}}%
\pgfpathlineto{\pgfqpoint{2.272752in}{0.881633in}}%
\pgfusepath{stroke}%
\end{pgfscope}%
\begin{pgfscope}%
\pgfpathrectangle{\pgfqpoint{0.100000in}{0.212622in}}{\pgfqpoint{3.696000in}{3.696000in}}%
\pgfusepath{clip}%
\pgfsetrectcap%
\pgfsetroundjoin%
\pgfsetlinewidth{1.505625pt}%
\definecolor{currentstroke}{rgb}{1.000000,0.000000,0.000000}%
\pgfsetstrokecolor{currentstroke}%
\pgfsetdash{}{0pt}%
\pgfpathmoveto{\pgfqpoint{2.162612in}{1.860242in}}%
\pgfpathlineto{\pgfqpoint{2.257918in}{0.886339in}}%
\pgfusepath{stroke}%
\end{pgfscope}%
\begin{pgfscope}%
\pgfpathrectangle{\pgfqpoint{0.100000in}{0.212622in}}{\pgfqpoint{3.696000in}{3.696000in}}%
\pgfusepath{clip}%
\pgfsetrectcap%
\pgfsetroundjoin%
\pgfsetlinewidth{1.505625pt}%
\definecolor{currentstroke}{rgb}{1.000000,0.000000,0.000000}%
\pgfsetstrokecolor{currentstroke}%
\pgfsetdash{}{0pt}%
\pgfpathmoveto{\pgfqpoint{2.152954in}{1.864362in}}%
\pgfpathlineto{\pgfqpoint{2.257918in}{0.886339in}}%
\pgfusepath{stroke}%
\end{pgfscope}%
\begin{pgfscope}%
\pgfpathrectangle{\pgfqpoint{0.100000in}{0.212622in}}{\pgfqpoint{3.696000in}{3.696000in}}%
\pgfusepath{clip}%
\pgfsetrectcap%
\pgfsetroundjoin%
\pgfsetlinewidth{1.505625pt}%
\definecolor{currentstroke}{rgb}{1.000000,0.000000,0.000000}%
\pgfsetstrokecolor{currentstroke}%
\pgfsetdash{}{0pt}%
\pgfpathmoveto{\pgfqpoint{2.141290in}{1.869911in}}%
\pgfpathlineto{\pgfqpoint{2.243094in}{0.891042in}}%
\pgfusepath{stroke}%
\end{pgfscope}%
\begin{pgfscope}%
\pgfpathrectangle{\pgfqpoint{0.100000in}{0.212622in}}{\pgfqpoint{3.696000in}{3.696000in}}%
\pgfusepath{clip}%
\pgfsetrectcap%
\pgfsetroundjoin%
\pgfsetlinewidth{1.505625pt}%
\definecolor{currentstroke}{rgb}{1.000000,0.000000,0.000000}%
\pgfsetstrokecolor{currentstroke}%
\pgfsetdash{}{0pt}%
\pgfpathmoveto{\pgfqpoint{2.128741in}{1.875677in}}%
\pgfpathlineto{\pgfqpoint{2.228282in}{0.895741in}}%
\pgfusepath{stroke}%
\end{pgfscope}%
\begin{pgfscope}%
\pgfpathrectangle{\pgfqpoint{0.100000in}{0.212622in}}{\pgfqpoint{3.696000in}{3.696000in}}%
\pgfusepath{clip}%
\pgfsetrectcap%
\pgfsetroundjoin%
\pgfsetlinewidth{1.505625pt}%
\definecolor{currentstroke}{rgb}{1.000000,0.000000,0.000000}%
\pgfsetstrokecolor{currentstroke}%
\pgfsetdash{}{0pt}%
\pgfpathmoveto{\pgfqpoint{2.115659in}{1.880970in}}%
\pgfpathlineto{\pgfqpoint{2.213480in}{0.900437in}}%
\pgfusepath{stroke}%
\end{pgfscope}%
\begin{pgfscope}%
\pgfpathrectangle{\pgfqpoint{0.100000in}{0.212622in}}{\pgfqpoint{3.696000in}{3.696000in}}%
\pgfusepath{clip}%
\pgfsetrectcap%
\pgfsetroundjoin%
\pgfsetlinewidth{1.505625pt}%
\definecolor{currentstroke}{rgb}{1.000000,0.000000,0.000000}%
\pgfsetstrokecolor{currentstroke}%
\pgfsetdash{}{0pt}%
\pgfpathmoveto{\pgfqpoint{2.100945in}{1.887341in}}%
\pgfpathlineto{\pgfqpoint{2.198689in}{0.905129in}}%
\pgfusepath{stroke}%
\end{pgfscope}%
\begin{pgfscope}%
\pgfpathrectangle{\pgfqpoint{0.100000in}{0.212622in}}{\pgfqpoint{3.696000in}{3.696000in}}%
\pgfusepath{clip}%
\pgfsetrectcap%
\pgfsetroundjoin%
\pgfsetlinewidth{1.505625pt}%
\definecolor{currentstroke}{rgb}{1.000000,0.000000,0.000000}%
\pgfsetstrokecolor{currentstroke}%
\pgfsetdash{}{0pt}%
\pgfpathmoveto{\pgfqpoint{2.082884in}{1.894812in}}%
\pgfpathlineto{\pgfqpoint{2.183910in}{0.909818in}}%
\pgfusepath{stroke}%
\end{pgfscope}%
\begin{pgfscope}%
\pgfpathrectangle{\pgfqpoint{0.100000in}{0.212622in}}{\pgfqpoint{3.696000in}{3.696000in}}%
\pgfusepath{clip}%
\pgfsetrectcap%
\pgfsetroundjoin%
\pgfsetlinewidth{1.505625pt}%
\definecolor{currentstroke}{rgb}{1.000000,0.000000,0.000000}%
\pgfsetstrokecolor{currentstroke}%
\pgfsetdash{}{0pt}%
\pgfpathmoveto{\pgfqpoint{2.063336in}{1.900303in}}%
\pgfpathlineto{\pgfqpoint{2.169141in}{0.914503in}}%
\pgfusepath{stroke}%
\end{pgfscope}%
\begin{pgfscope}%
\pgfpathrectangle{\pgfqpoint{0.100000in}{0.212622in}}{\pgfqpoint{3.696000in}{3.696000in}}%
\pgfusepath{clip}%
\pgfsetrectcap%
\pgfsetroundjoin%
\pgfsetlinewidth{1.505625pt}%
\definecolor{currentstroke}{rgb}{1.000000,0.000000,0.000000}%
\pgfsetstrokecolor{currentstroke}%
\pgfsetdash{}{0pt}%
\pgfpathmoveto{\pgfqpoint{2.052685in}{1.903972in}}%
\pgfpathlineto{\pgfqpoint{2.154382in}{0.919185in}}%
\pgfusepath{stroke}%
\end{pgfscope}%
\begin{pgfscope}%
\pgfpathrectangle{\pgfqpoint{0.100000in}{0.212622in}}{\pgfqpoint{3.696000in}{3.696000in}}%
\pgfusepath{clip}%
\pgfsetrectcap%
\pgfsetroundjoin%
\pgfsetlinewidth{1.505625pt}%
\definecolor{currentstroke}{rgb}{1.000000,0.000000,0.000000}%
\pgfsetstrokecolor{currentstroke}%
\pgfsetdash{}{0pt}%
\pgfpathmoveto{\pgfqpoint{2.046843in}{1.906144in}}%
\pgfpathlineto{\pgfqpoint{2.154382in}{0.919185in}}%
\pgfusepath{stroke}%
\end{pgfscope}%
\begin{pgfscope}%
\pgfpathrectangle{\pgfqpoint{0.100000in}{0.212622in}}{\pgfqpoint{3.696000in}{3.696000in}}%
\pgfusepath{clip}%
\pgfsetrectcap%
\pgfsetroundjoin%
\pgfsetlinewidth{1.505625pt}%
\definecolor{currentstroke}{rgb}{1.000000,0.000000,0.000000}%
\pgfsetstrokecolor{currentstroke}%
\pgfsetdash{}{0pt}%
\pgfpathmoveto{\pgfqpoint{2.039237in}{1.910272in}}%
\pgfpathlineto{\pgfqpoint{2.139635in}{0.923864in}}%
\pgfusepath{stroke}%
\end{pgfscope}%
\begin{pgfscope}%
\pgfpathrectangle{\pgfqpoint{0.100000in}{0.212622in}}{\pgfqpoint{3.696000in}{3.696000in}}%
\pgfusepath{clip}%
\pgfsetrectcap%
\pgfsetroundjoin%
\pgfsetlinewidth{1.505625pt}%
\definecolor{currentstroke}{rgb}{1.000000,0.000000,0.000000}%
\pgfsetstrokecolor{currentstroke}%
\pgfsetdash{}{0pt}%
\pgfpathmoveto{\pgfqpoint{2.030648in}{1.915135in}}%
\pgfpathlineto{\pgfqpoint{2.139635in}{0.923864in}}%
\pgfusepath{stroke}%
\end{pgfscope}%
\begin{pgfscope}%
\pgfpathrectangle{\pgfqpoint{0.100000in}{0.212622in}}{\pgfqpoint{3.696000in}{3.696000in}}%
\pgfusepath{clip}%
\pgfsetrectcap%
\pgfsetroundjoin%
\pgfsetlinewidth{1.505625pt}%
\definecolor{currentstroke}{rgb}{1.000000,0.000000,0.000000}%
\pgfsetstrokecolor{currentstroke}%
\pgfsetdash{}{0pt}%
\pgfpathmoveto{\pgfqpoint{2.021488in}{1.919477in}}%
\pgfpathlineto{\pgfqpoint{2.124899in}{0.928539in}}%
\pgfusepath{stroke}%
\end{pgfscope}%
\begin{pgfscope}%
\pgfpathrectangle{\pgfqpoint{0.100000in}{0.212622in}}{\pgfqpoint{3.696000in}{3.696000in}}%
\pgfusepath{clip}%
\pgfsetrectcap%
\pgfsetroundjoin%
\pgfsetlinewidth{1.505625pt}%
\definecolor{currentstroke}{rgb}{1.000000,0.000000,0.000000}%
\pgfsetstrokecolor{currentstroke}%
\pgfsetdash{}{0pt}%
\pgfpathmoveto{\pgfqpoint{2.011117in}{1.924603in}}%
\pgfpathlineto{\pgfqpoint{2.110173in}{0.933211in}}%
\pgfusepath{stroke}%
\end{pgfscope}%
\begin{pgfscope}%
\pgfpathrectangle{\pgfqpoint{0.100000in}{0.212622in}}{\pgfqpoint{3.696000in}{3.696000in}}%
\pgfusepath{clip}%
\pgfsetrectcap%
\pgfsetroundjoin%
\pgfsetlinewidth{1.505625pt}%
\definecolor{currentstroke}{rgb}{1.000000,0.000000,0.000000}%
\pgfsetstrokecolor{currentstroke}%
\pgfsetdash{}{0pt}%
\pgfpathmoveto{\pgfqpoint{1.997861in}{1.931058in}}%
\pgfpathlineto{\pgfqpoint{2.110173in}{0.933211in}}%
\pgfusepath{stroke}%
\end{pgfscope}%
\begin{pgfscope}%
\pgfpathrectangle{\pgfqpoint{0.100000in}{0.212622in}}{\pgfqpoint{3.696000in}{3.696000in}}%
\pgfusepath{clip}%
\pgfsetrectcap%
\pgfsetroundjoin%
\pgfsetlinewidth{1.505625pt}%
\definecolor{currentstroke}{rgb}{1.000000,0.000000,0.000000}%
\pgfsetstrokecolor{currentstroke}%
\pgfsetdash{}{0pt}%
\pgfpathmoveto{\pgfqpoint{1.981852in}{1.936022in}}%
\pgfpathlineto{\pgfqpoint{2.095459in}{0.937879in}}%
\pgfusepath{stroke}%
\end{pgfscope}%
\begin{pgfscope}%
\pgfpathrectangle{\pgfqpoint{0.100000in}{0.212622in}}{\pgfqpoint{3.696000in}{3.696000in}}%
\pgfusepath{clip}%
\pgfsetrectcap%
\pgfsetroundjoin%
\pgfsetlinewidth{1.505625pt}%
\definecolor{currentstroke}{rgb}{1.000000,0.000000,0.000000}%
\pgfsetstrokecolor{currentstroke}%
\pgfsetdash{}{0pt}%
\pgfpathmoveto{\pgfqpoint{1.973118in}{1.939523in}}%
\pgfpathlineto{\pgfqpoint{2.080755in}{0.942544in}}%
\pgfusepath{stroke}%
\end{pgfscope}%
\begin{pgfscope}%
\pgfpathrectangle{\pgfqpoint{0.100000in}{0.212622in}}{\pgfqpoint{3.696000in}{3.696000in}}%
\pgfusepath{clip}%
\pgfsetrectcap%
\pgfsetroundjoin%
\pgfsetlinewidth{1.505625pt}%
\definecolor{currentstroke}{rgb}{1.000000,0.000000,0.000000}%
\pgfsetstrokecolor{currentstroke}%
\pgfsetdash{}{0pt}%
\pgfpathmoveto{\pgfqpoint{1.963699in}{1.942783in}}%
\pgfpathlineto{\pgfqpoint{2.066061in}{0.947205in}}%
\pgfusepath{stroke}%
\end{pgfscope}%
\begin{pgfscope}%
\pgfpathrectangle{\pgfqpoint{0.100000in}{0.212622in}}{\pgfqpoint{3.696000in}{3.696000in}}%
\pgfusepath{clip}%
\pgfsetrectcap%
\pgfsetroundjoin%
\pgfsetlinewidth{1.505625pt}%
\definecolor{currentstroke}{rgb}{1.000000,0.000000,0.000000}%
\pgfsetstrokecolor{currentstroke}%
\pgfsetdash{}{0pt}%
\pgfpathmoveto{\pgfqpoint{1.952080in}{1.947418in}}%
\pgfpathlineto{\pgfqpoint{2.066061in}{0.947205in}}%
\pgfusepath{stroke}%
\end{pgfscope}%
\begin{pgfscope}%
\pgfpathrectangle{\pgfqpoint{0.100000in}{0.212622in}}{\pgfqpoint{3.696000in}{3.696000in}}%
\pgfusepath{clip}%
\pgfsetrectcap%
\pgfsetroundjoin%
\pgfsetlinewidth{1.505625pt}%
\definecolor{currentstroke}{rgb}{1.000000,0.000000,0.000000}%
\pgfsetstrokecolor{currentstroke}%
\pgfsetdash{}{0pt}%
\pgfpathmoveto{\pgfqpoint{1.938029in}{1.952188in}}%
\pgfpathlineto{\pgfqpoint{2.051379in}{0.951863in}}%
\pgfusepath{stroke}%
\end{pgfscope}%
\begin{pgfscope}%
\pgfpathrectangle{\pgfqpoint{0.100000in}{0.212622in}}{\pgfqpoint{3.696000in}{3.696000in}}%
\pgfusepath{clip}%
\pgfsetrectcap%
\pgfsetroundjoin%
\pgfsetlinewidth{1.505625pt}%
\definecolor{currentstroke}{rgb}{1.000000,0.000000,0.000000}%
\pgfsetstrokecolor{currentstroke}%
\pgfsetdash{}{0pt}%
\pgfpathmoveto{\pgfqpoint{1.923233in}{1.956563in}}%
\pgfpathlineto{\pgfqpoint{2.036708in}{0.956518in}}%
\pgfusepath{stroke}%
\end{pgfscope}%
\begin{pgfscope}%
\pgfpathrectangle{\pgfqpoint{0.100000in}{0.212622in}}{\pgfqpoint{3.696000in}{3.696000in}}%
\pgfusepath{clip}%
\pgfsetrectcap%
\pgfsetroundjoin%
\pgfsetlinewidth{1.505625pt}%
\definecolor{currentstroke}{rgb}{1.000000,0.000000,0.000000}%
\pgfsetstrokecolor{currentstroke}%
\pgfsetdash{}{0pt}%
\pgfpathmoveto{\pgfqpoint{1.907255in}{1.961377in}}%
\pgfpathlineto{\pgfqpoint{2.022047in}{0.961169in}}%
\pgfusepath{stroke}%
\end{pgfscope}%
\begin{pgfscope}%
\pgfpathrectangle{\pgfqpoint{0.100000in}{0.212622in}}{\pgfqpoint{3.696000in}{3.696000in}}%
\pgfusepath{clip}%
\pgfsetrectcap%
\pgfsetroundjoin%
\pgfsetlinewidth{1.505625pt}%
\definecolor{currentstroke}{rgb}{1.000000,0.000000,0.000000}%
\pgfsetstrokecolor{currentstroke}%
\pgfsetdash{}{0pt}%
\pgfpathmoveto{\pgfqpoint{1.890184in}{1.966388in}}%
\pgfpathlineto{\pgfqpoint{2.007397in}{0.965817in}}%
\pgfusepath{stroke}%
\end{pgfscope}%
\begin{pgfscope}%
\pgfpathrectangle{\pgfqpoint{0.100000in}{0.212622in}}{\pgfqpoint{3.696000in}{3.696000in}}%
\pgfusepath{clip}%
\pgfsetrectcap%
\pgfsetroundjoin%
\pgfsetlinewidth{1.505625pt}%
\definecolor{currentstroke}{rgb}{1.000000,0.000000,0.000000}%
\pgfsetstrokecolor{currentstroke}%
\pgfsetdash{}{0pt}%
\pgfpathmoveto{\pgfqpoint{1.871449in}{1.972743in}}%
\pgfpathlineto{\pgfqpoint{1.992757in}{0.970461in}}%
\pgfusepath{stroke}%
\end{pgfscope}%
\begin{pgfscope}%
\pgfpathrectangle{\pgfqpoint{0.100000in}{0.212622in}}{\pgfqpoint{3.696000in}{3.696000in}}%
\pgfusepath{clip}%
\pgfsetrectcap%
\pgfsetroundjoin%
\pgfsetlinewidth{1.505625pt}%
\definecolor{currentstroke}{rgb}{1.000000,0.000000,0.000000}%
\pgfsetstrokecolor{currentstroke}%
\pgfsetdash{}{0pt}%
\pgfpathmoveto{\pgfqpoint{1.852274in}{1.980972in}}%
\pgfpathlineto{\pgfqpoint{1.963511in}{0.979739in}}%
\pgfusepath{stroke}%
\end{pgfscope}%
\begin{pgfscope}%
\pgfpathrectangle{\pgfqpoint{0.100000in}{0.212622in}}{\pgfqpoint{3.696000in}{3.696000in}}%
\pgfusepath{clip}%
\pgfsetrectcap%
\pgfsetroundjoin%
\pgfsetlinewidth{1.505625pt}%
\definecolor{currentstroke}{rgb}{1.000000,0.000000,0.000000}%
\pgfsetstrokecolor{currentstroke}%
\pgfsetdash{}{0pt}%
\pgfpathmoveto{\pgfqpoint{1.832003in}{1.989384in}}%
\pgfpathlineto{\pgfqpoint{1.948903in}{0.984374in}}%
\pgfusepath{stroke}%
\end{pgfscope}%
\begin{pgfscope}%
\pgfpathrectangle{\pgfqpoint{0.100000in}{0.212622in}}{\pgfqpoint{3.696000in}{3.696000in}}%
\pgfusepath{clip}%
\pgfsetrectcap%
\pgfsetroundjoin%
\pgfsetlinewidth{1.505625pt}%
\definecolor{currentstroke}{rgb}{1.000000,0.000000,0.000000}%
\pgfsetstrokecolor{currentstroke}%
\pgfsetdash{}{0pt}%
\pgfpathmoveto{\pgfqpoint{1.807702in}{2.000304in}}%
\pgfpathlineto{\pgfqpoint{1.919721in}{0.993632in}}%
\pgfusepath{stroke}%
\end{pgfscope}%
\begin{pgfscope}%
\pgfpathrectangle{\pgfqpoint{0.100000in}{0.212622in}}{\pgfqpoint{3.696000in}{3.696000in}}%
\pgfusepath{clip}%
\pgfsetrectcap%
\pgfsetroundjoin%
\pgfsetlinewidth{1.505625pt}%
\definecolor{currentstroke}{rgb}{1.000000,0.000000,0.000000}%
\pgfsetstrokecolor{currentstroke}%
\pgfsetdash{}{0pt}%
\pgfpathmoveto{\pgfqpoint{1.780986in}{2.013392in}}%
\pgfpathlineto{\pgfqpoint{1.905146in}{0.998256in}}%
\pgfusepath{stroke}%
\end{pgfscope}%
\begin{pgfscope}%
\pgfpathrectangle{\pgfqpoint{0.100000in}{0.212622in}}{\pgfqpoint{3.696000in}{3.696000in}}%
\pgfusepath{clip}%
\pgfsetrectcap%
\pgfsetroundjoin%
\pgfsetlinewidth{1.505625pt}%
\definecolor{currentstroke}{rgb}{1.000000,0.000000,0.000000}%
\pgfsetstrokecolor{currentstroke}%
\pgfsetdash{}{0pt}%
\pgfpathmoveto{\pgfqpoint{1.766184in}{2.019654in}}%
\pgfpathlineto{\pgfqpoint{1.890581in}{1.002876in}}%
\pgfusepath{stroke}%
\end{pgfscope}%
\begin{pgfscope}%
\pgfpathrectangle{\pgfqpoint{0.100000in}{0.212622in}}{\pgfqpoint{3.696000in}{3.696000in}}%
\pgfusepath{clip}%
\pgfsetrectcap%
\pgfsetroundjoin%
\pgfsetlinewidth{1.505625pt}%
\definecolor{currentstroke}{rgb}{1.000000,0.000000,0.000000}%
\pgfsetstrokecolor{currentstroke}%
\pgfsetdash{}{0pt}%
\pgfpathmoveto{\pgfqpoint{1.749579in}{2.026558in}}%
\pgfpathlineto{\pgfqpoint{1.876027in}{1.007493in}}%
\pgfusepath{stroke}%
\end{pgfscope}%
\begin{pgfscope}%
\pgfpathrectangle{\pgfqpoint{0.100000in}{0.212622in}}{\pgfqpoint{3.696000in}{3.696000in}}%
\pgfusepath{clip}%
\pgfsetrectcap%
\pgfsetroundjoin%
\pgfsetlinewidth{1.505625pt}%
\definecolor{currentstroke}{rgb}{1.000000,0.000000,0.000000}%
\pgfsetstrokecolor{currentstroke}%
\pgfsetdash{}{0pt}%
\pgfpathmoveto{\pgfqpoint{1.730407in}{2.033351in}}%
\pgfpathlineto{\pgfqpoint{1.846951in}{1.016718in}}%
\pgfusepath{stroke}%
\end{pgfscope}%
\begin{pgfscope}%
\pgfpathrectangle{\pgfqpoint{0.100000in}{0.212622in}}{\pgfqpoint{3.696000in}{3.696000in}}%
\pgfusepath{clip}%
\pgfsetrectcap%
\pgfsetroundjoin%
\pgfsetlinewidth{1.505625pt}%
\definecolor{currentstroke}{rgb}{1.000000,0.000000,0.000000}%
\pgfsetstrokecolor{currentstroke}%
\pgfsetdash{}{0pt}%
\pgfpathmoveto{\pgfqpoint{1.708497in}{2.040082in}}%
\pgfpathlineto{\pgfqpoint{1.832429in}{1.021325in}}%
\pgfusepath{stroke}%
\end{pgfscope}%
\begin{pgfscope}%
\pgfpathrectangle{\pgfqpoint{0.100000in}{0.212622in}}{\pgfqpoint{3.696000in}{3.696000in}}%
\pgfusepath{clip}%
\pgfsetrectcap%
\pgfsetroundjoin%
\pgfsetlinewidth{1.505625pt}%
\definecolor{currentstroke}{rgb}{1.000000,0.000000,0.000000}%
\pgfsetstrokecolor{currentstroke}%
\pgfsetdash{}{0pt}%
\pgfpathmoveto{\pgfqpoint{1.685644in}{2.047963in}}%
\pgfpathlineto{\pgfqpoint{1.803417in}{1.030529in}}%
\pgfusepath{stroke}%
\end{pgfscope}%
\begin{pgfscope}%
\pgfpathrectangle{\pgfqpoint{0.100000in}{0.212622in}}{\pgfqpoint{3.696000in}{3.696000in}}%
\pgfusepath{clip}%
\pgfsetrectcap%
\pgfsetroundjoin%
\pgfsetlinewidth{1.505625pt}%
\definecolor{currentstroke}{rgb}{1.000000,0.000000,0.000000}%
\pgfsetstrokecolor{currentstroke}%
\pgfsetdash{}{0pt}%
\pgfpathmoveto{\pgfqpoint{1.659326in}{2.056785in}}%
\pgfpathlineto{\pgfqpoint{1.788926in}{1.035126in}}%
\pgfusepath{stroke}%
\end{pgfscope}%
\begin{pgfscope}%
\pgfpathrectangle{\pgfqpoint{0.100000in}{0.212622in}}{\pgfqpoint{3.696000in}{3.696000in}}%
\pgfusepath{clip}%
\pgfsetrectcap%
\pgfsetroundjoin%
\pgfsetlinewidth{1.505625pt}%
\definecolor{currentstroke}{rgb}{1.000000,0.000000,0.000000}%
\pgfsetstrokecolor{currentstroke}%
\pgfsetdash{}{0pt}%
\pgfpathmoveto{\pgfqpoint{1.631091in}{2.071774in}}%
\pgfpathlineto{\pgfqpoint{1.759977in}{1.044310in}}%
\pgfusepath{stroke}%
\end{pgfscope}%
\begin{pgfscope}%
\pgfpathrectangle{\pgfqpoint{0.100000in}{0.212622in}}{\pgfqpoint{3.696000in}{3.696000in}}%
\pgfusepath{clip}%
\pgfsetrectcap%
\pgfsetroundjoin%
\pgfsetlinewidth{1.505625pt}%
\definecolor{currentstroke}{rgb}{1.000000,0.000000,0.000000}%
\pgfsetstrokecolor{currentstroke}%
\pgfsetdash{}{0pt}%
\pgfpathmoveto{\pgfqpoint{1.601707in}{2.091767in}}%
\pgfpathlineto{\pgfqpoint{1.731070in}{1.053481in}}%
\pgfusepath{stroke}%
\end{pgfscope}%
\begin{pgfscope}%
\pgfpathrectangle{\pgfqpoint{0.100000in}{0.212622in}}{\pgfqpoint{3.696000in}{3.696000in}}%
\pgfusepath{clip}%
\pgfsetrectcap%
\pgfsetroundjoin%
\pgfsetlinewidth{1.505625pt}%
\definecolor{currentstroke}{rgb}{1.000000,0.000000,0.000000}%
\pgfsetstrokecolor{currentstroke}%
\pgfsetdash{}{0pt}%
\pgfpathmoveto{\pgfqpoint{1.585477in}{2.101742in}}%
\pgfpathlineto{\pgfqpoint{1.716633in}{1.058061in}}%
\pgfusepath{stroke}%
\end{pgfscope}%
\begin{pgfscope}%
\pgfpathrectangle{\pgfqpoint{0.100000in}{0.212622in}}{\pgfqpoint{3.696000in}{3.696000in}}%
\pgfusepath{clip}%
\pgfsetrectcap%
\pgfsetroundjoin%
\pgfsetlinewidth{1.505625pt}%
\definecolor{currentstroke}{rgb}{1.000000,0.000000,0.000000}%
\pgfsetstrokecolor{currentstroke}%
\pgfsetdash{}{0pt}%
\pgfpathmoveto{\pgfqpoint{1.567866in}{2.112552in}}%
\pgfpathlineto{\pgfqpoint{1.702206in}{1.062638in}}%
\pgfusepath{stroke}%
\end{pgfscope}%
\begin{pgfscope}%
\pgfpathrectangle{\pgfqpoint{0.100000in}{0.212622in}}{\pgfqpoint{3.696000in}{3.696000in}}%
\pgfusepath{clip}%
\pgfsetrectcap%
\pgfsetroundjoin%
\pgfsetlinewidth{1.505625pt}%
\definecolor{currentstroke}{rgb}{1.000000,0.000000,0.000000}%
\pgfsetstrokecolor{currentstroke}%
\pgfsetdash{}{0pt}%
\pgfpathmoveto{\pgfqpoint{1.545687in}{2.121737in}}%
\pgfpathlineto{\pgfqpoint{1.673383in}{1.071782in}}%
\pgfusepath{stroke}%
\end{pgfscope}%
\begin{pgfscope}%
\pgfpathrectangle{\pgfqpoint{0.100000in}{0.212622in}}{\pgfqpoint{3.696000in}{3.696000in}}%
\pgfusepath{clip}%
\pgfsetrectcap%
\pgfsetroundjoin%
\pgfsetlinewidth{1.505625pt}%
\definecolor{currentstroke}{rgb}{1.000000,0.000000,0.000000}%
\pgfsetstrokecolor{currentstroke}%
\pgfsetdash{}{0pt}%
\pgfpathmoveto{\pgfqpoint{1.522136in}{2.130011in}}%
\pgfpathlineto{\pgfqpoint{1.658987in}{1.076349in}}%
\pgfusepath{stroke}%
\end{pgfscope}%
\begin{pgfscope}%
\pgfpathrectangle{\pgfqpoint{0.100000in}{0.212622in}}{\pgfqpoint{3.696000in}{3.696000in}}%
\pgfusepath{clip}%
\pgfsetrectcap%
\pgfsetroundjoin%
\pgfsetlinewidth{1.505625pt}%
\definecolor{currentstroke}{rgb}{1.000000,0.000000,0.000000}%
\pgfsetstrokecolor{currentstroke}%
\pgfsetdash{}{0pt}%
\pgfpathmoveto{\pgfqpoint{1.509267in}{2.134851in}}%
\pgfpathlineto{\pgfqpoint{1.644602in}{1.080913in}}%
\pgfusepath{stroke}%
\end{pgfscope}%
\begin{pgfscope}%
\pgfpathrectangle{\pgfqpoint{0.100000in}{0.212622in}}{\pgfqpoint{3.696000in}{3.696000in}}%
\pgfusepath{clip}%
\pgfsetrectcap%
\pgfsetroundjoin%
\pgfsetlinewidth{1.505625pt}%
\definecolor{currentstroke}{rgb}{1.000000,0.000000,0.000000}%
\pgfsetstrokecolor{currentstroke}%
\pgfsetdash{}{0pt}%
\pgfpathmoveto{\pgfqpoint{1.502174in}{2.137653in}}%
\pgfpathlineto{\pgfqpoint{1.644602in}{1.080913in}}%
\pgfusepath{stroke}%
\end{pgfscope}%
\begin{pgfscope}%
\pgfpathrectangle{\pgfqpoint{0.100000in}{0.212622in}}{\pgfqpoint{3.696000in}{3.696000in}}%
\pgfusepath{clip}%
\pgfsetrectcap%
\pgfsetroundjoin%
\pgfsetlinewidth{1.505625pt}%
\definecolor{currentstroke}{rgb}{1.000000,0.000000,0.000000}%
\pgfsetstrokecolor{currentstroke}%
\pgfsetdash{}{0pt}%
\pgfpathmoveto{\pgfqpoint{1.493004in}{2.142114in}}%
\pgfpathlineto{\pgfqpoint{1.630227in}{1.085474in}}%
\pgfusepath{stroke}%
\end{pgfscope}%
\begin{pgfscope}%
\pgfpathrectangle{\pgfqpoint{0.100000in}{0.212622in}}{\pgfqpoint{3.696000in}{3.696000in}}%
\pgfusepath{clip}%
\pgfsetrectcap%
\pgfsetroundjoin%
\pgfsetlinewidth{1.505625pt}%
\definecolor{currentstroke}{rgb}{1.000000,0.000000,0.000000}%
\pgfsetstrokecolor{currentstroke}%
\pgfsetdash{}{0pt}%
\pgfpathmoveto{\pgfqpoint{1.481243in}{2.149708in}}%
\pgfpathlineto{\pgfqpoint{1.615862in}{1.090031in}}%
\pgfusepath{stroke}%
\end{pgfscope}%
\begin{pgfscope}%
\pgfpathrectangle{\pgfqpoint{0.100000in}{0.212622in}}{\pgfqpoint{3.696000in}{3.696000in}}%
\pgfusepath{clip}%
\pgfsetrectcap%
\pgfsetroundjoin%
\pgfsetlinewidth{1.505625pt}%
\definecolor{currentstroke}{rgb}{1.000000,0.000000,0.000000}%
\pgfsetstrokecolor{currentstroke}%
\pgfsetdash{}{0pt}%
\pgfpathmoveto{\pgfqpoint{1.474747in}{2.153011in}}%
\pgfpathlineto{\pgfqpoint{1.615862in}{1.090031in}}%
\pgfusepath{stroke}%
\end{pgfscope}%
\begin{pgfscope}%
\pgfpathrectangle{\pgfqpoint{0.100000in}{0.212622in}}{\pgfqpoint{3.696000in}{3.696000in}}%
\pgfusepath{clip}%
\pgfsetrectcap%
\pgfsetroundjoin%
\pgfsetlinewidth{1.505625pt}%
\definecolor{currentstroke}{rgb}{1.000000,0.000000,0.000000}%
\pgfsetstrokecolor{currentstroke}%
\pgfsetdash{}{0pt}%
\pgfpathmoveto{\pgfqpoint{1.467352in}{2.156716in}}%
\pgfpathlineto{\pgfqpoint{1.601508in}{1.094584in}}%
\pgfusepath{stroke}%
\end{pgfscope}%
\begin{pgfscope}%
\pgfpathrectangle{\pgfqpoint{0.100000in}{0.212622in}}{\pgfqpoint{3.696000in}{3.696000in}}%
\pgfusepath{clip}%
\pgfsetrectcap%
\pgfsetroundjoin%
\pgfsetlinewidth{1.505625pt}%
\definecolor{currentstroke}{rgb}{1.000000,0.000000,0.000000}%
\pgfsetstrokecolor{currentstroke}%
\pgfsetdash{}{0pt}%
\pgfpathmoveto{\pgfqpoint{1.458667in}{2.159964in}}%
\pgfpathlineto{\pgfqpoint{1.601508in}{1.094584in}}%
\pgfusepath{stroke}%
\end{pgfscope}%
\begin{pgfscope}%
\pgfpathrectangle{\pgfqpoint{0.100000in}{0.212622in}}{\pgfqpoint{3.696000in}{3.696000in}}%
\pgfusepath{clip}%
\pgfsetrectcap%
\pgfsetroundjoin%
\pgfsetlinewidth{1.505625pt}%
\definecolor{currentstroke}{rgb}{1.000000,0.000000,0.000000}%
\pgfsetstrokecolor{currentstroke}%
\pgfsetdash{}{0pt}%
\pgfpathmoveto{\pgfqpoint{1.448800in}{2.164566in}}%
\pgfpathlineto{\pgfqpoint{1.587165in}{1.099135in}}%
\pgfusepath{stroke}%
\end{pgfscope}%
\begin{pgfscope}%
\pgfpathrectangle{\pgfqpoint{0.100000in}{0.212622in}}{\pgfqpoint{3.696000in}{3.696000in}}%
\pgfusepath{clip}%
\pgfsetrectcap%
\pgfsetroundjoin%
\pgfsetlinewidth{1.505625pt}%
\definecolor{currentstroke}{rgb}{1.000000,0.000000,0.000000}%
\pgfsetstrokecolor{currentstroke}%
\pgfsetdash{}{0pt}%
\pgfpathmoveto{\pgfqpoint{1.437897in}{2.169944in}}%
\pgfpathlineto{\pgfqpoint{1.572832in}{1.103682in}}%
\pgfusepath{stroke}%
\end{pgfscope}%
\begin{pgfscope}%
\pgfpathrectangle{\pgfqpoint{0.100000in}{0.212622in}}{\pgfqpoint{3.696000in}{3.696000in}}%
\pgfusepath{clip}%
\pgfsetrectcap%
\pgfsetroundjoin%
\pgfsetlinewidth{1.505625pt}%
\definecolor{currentstroke}{rgb}{1.000000,0.000000,0.000000}%
\pgfsetstrokecolor{currentstroke}%
\pgfsetdash{}{0pt}%
\pgfpathmoveto{\pgfqpoint{1.431870in}{2.172564in}}%
\pgfpathlineto{\pgfqpoint{1.572832in}{1.103682in}}%
\pgfusepath{stroke}%
\end{pgfscope}%
\begin{pgfscope}%
\pgfpathrectangle{\pgfqpoint{0.100000in}{0.212622in}}{\pgfqpoint{3.696000in}{3.696000in}}%
\pgfusepath{clip}%
\pgfsetrectcap%
\pgfsetroundjoin%
\pgfsetlinewidth{1.505625pt}%
\definecolor{currentstroke}{rgb}{1.000000,0.000000,0.000000}%
\pgfsetstrokecolor{currentstroke}%
\pgfsetdash{}{0pt}%
\pgfpathmoveto{\pgfqpoint{1.424704in}{2.175610in}}%
\pgfpathlineto{\pgfqpoint{1.558509in}{1.108226in}}%
\pgfusepath{stroke}%
\end{pgfscope}%
\begin{pgfscope}%
\pgfpathrectangle{\pgfqpoint{0.100000in}{0.212622in}}{\pgfqpoint{3.696000in}{3.696000in}}%
\pgfusepath{clip}%
\pgfsetrectcap%
\pgfsetroundjoin%
\pgfsetlinewidth{1.505625pt}%
\definecolor{currentstroke}{rgb}{1.000000,0.000000,0.000000}%
\pgfsetstrokecolor{currentstroke}%
\pgfsetdash{}{0pt}%
\pgfpathmoveto{\pgfqpoint{1.415117in}{2.179365in}}%
\pgfpathlineto{\pgfqpoint{1.558509in}{1.108226in}}%
\pgfusepath{stroke}%
\end{pgfscope}%
\begin{pgfscope}%
\pgfpathrectangle{\pgfqpoint{0.100000in}{0.212622in}}{\pgfqpoint{3.696000in}{3.696000in}}%
\pgfusepath{clip}%
\pgfsetrectcap%
\pgfsetroundjoin%
\pgfsetlinewidth{1.505625pt}%
\definecolor{currentstroke}{rgb}{1.000000,0.000000,0.000000}%
\pgfsetstrokecolor{currentstroke}%
\pgfsetdash{}{0pt}%
\pgfpathmoveto{\pgfqpoint{1.403311in}{2.183914in}}%
\pgfpathlineto{\pgfqpoint{1.544197in}{1.112767in}}%
\pgfusepath{stroke}%
\end{pgfscope}%
\begin{pgfscope}%
\pgfpathrectangle{\pgfqpoint{0.100000in}{0.212622in}}{\pgfqpoint{3.696000in}{3.696000in}}%
\pgfusepath{clip}%
\pgfsetrectcap%
\pgfsetroundjoin%
\pgfsetlinewidth{1.505625pt}%
\definecolor{currentstroke}{rgb}{1.000000,0.000000,0.000000}%
\pgfsetstrokecolor{currentstroke}%
\pgfsetdash{}{0pt}%
\pgfpathmoveto{\pgfqpoint{1.391021in}{2.189305in}}%
\pgfpathlineto{\pgfqpoint{1.529894in}{1.117304in}}%
\pgfusepath{stroke}%
\end{pgfscope}%
\begin{pgfscope}%
\pgfpathrectangle{\pgfqpoint{0.100000in}{0.212622in}}{\pgfqpoint{3.696000in}{3.696000in}}%
\pgfusepath{clip}%
\pgfsetrectcap%
\pgfsetroundjoin%
\pgfsetlinewidth{1.505625pt}%
\definecolor{currentstroke}{rgb}{1.000000,0.000000,0.000000}%
\pgfsetstrokecolor{currentstroke}%
\pgfsetdash{}{0pt}%
\pgfpathmoveto{\pgfqpoint{1.377699in}{2.194553in}}%
\pgfpathlineto{\pgfqpoint{1.515603in}{1.121838in}}%
\pgfusepath{stroke}%
\end{pgfscope}%
\begin{pgfscope}%
\pgfpathrectangle{\pgfqpoint{0.100000in}{0.212622in}}{\pgfqpoint{3.696000in}{3.696000in}}%
\pgfusepath{clip}%
\pgfsetrectcap%
\pgfsetroundjoin%
\pgfsetlinewidth{1.505625pt}%
\definecolor{currentstroke}{rgb}{1.000000,0.000000,0.000000}%
\pgfsetstrokecolor{currentstroke}%
\pgfsetdash{}{0pt}%
\pgfpathmoveto{\pgfqpoint{1.361908in}{2.201544in}}%
\pgfpathlineto{\pgfqpoint{1.501321in}{1.126369in}}%
\pgfusepath{stroke}%
\end{pgfscope}%
\begin{pgfscope}%
\pgfpathrectangle{\pgfqpoint{0.100000in}{0.212622in}}{\pgfqpoint{3.696000in}{3.696000in}}%
\pgfusepath{clip}%
\pgfsetrectcap%
\pgfsetroundjoin%
\pgfsetlinewidth{1.505625pt}%
\definecolor{currentstroke}{rgb}{1.000000,0.000000,0.000000}%
\pgfsetstrokecolor{currentstroke}%
\pgfsetdash{}{0pt}%
\pgfpathmoveto{\pgfqpoint{1.343664in}{2.208943in}}%
\pgfpathlineto{\pgfqpoint{1.487050in}{1.130896in}}%
\pgfusepath{stroke}%
\end{pgfscope}%
\begin{pgfscope}%
\pgfpathrectangle{\pgfqpoint{0.100000in}{0.212622in}}{\pgfqpoint{3.696000in}{3.696000in}}%
\pgfusepath{clip}%
\pgfsetrectcap%
\pgfsetroundjoin%
\pgfsetlinewidth{1.505625pt}%
\definecolor{currentstroke}{rgb}{1.000000,0.000000,0.000000}%
\pgfsetstrokecolor{currentstroke}%
\pgfsetdash{}{0pt}%
\pgfpathmoveto{\pgfqpoint{1.323906in}{2.216450in}}%
\pgfpathlineto{\pgfqpoint{1.472790in}{1.135421in}}%
\pgfusepath{stroke}%
\end{pgfscope}%
\begin{pgfscope}%
\pgfpathrectangle{\pgfqpoint{0.100000in}{0.212622in}}{\pgfqpoint{3.696000in}{3.696000in}}%
\pgfusepath{clip}%
\pgfsetrectcap%
\pgfsetroundjoin%
\pgfsetlinewidth{1.505625pt}%
\definecolor{currentstroke}{rgb}{1.000000,0.000000,0.000000}%
\pgfsetstrokecolor{currentstroke}%
\pgfsetdash{}{0pt}%
\pgfpathmoveto{\pgfqpoint{1.303558in}{2.224317in}}%
\pgfpathlineto{\pgfqpoint{1.458539in}{1.139941in}}%
\pgfusepath{stroke}%
\end{pgfscope}%
\begin{pgfscope}%
\pgfpathrectangle{\pgfqpoint{0.100000in}{0.212622in}}{\pgfqpoint{3.696000in}{3.696000in}}%
\pgfusepath{clip}%
\pgfsetrectcap%
\pgfsetroundjoin%
\pgfsetlinewidth{1.505625pt}%
\definecolor{currentstroke}{rgb}{1.000000,0.000000,0.000000}%
\pgfsetstrokecolor{currentstroke}%
\pgfsetdash{}{0pt}%
\pgfpathmoveto{\pgfqpoint{1.281053in}{2.232828in}}%
\pgfpathlineto{\pgfqpoint{1.430069in}{1.148974in}}%
\pgfusepath{stroke}%
\end{pgfscope}%
\begin{pgfscope}%
\pgfpathrectangle{\pgfqpoint{0.100000in}{0.212622in}}{\pgfqpoint{3.696000in}{3.696000in}}%
\pgfusepath{clip}%
\pgfsetrectcap%
\pgfsetroundjoin%
\pgfsetlinewidth{1.505625pt}%
\definecolor{currentstroke}{rgb}{1.000000,0.000000,0.000000}%
\pgfsetstrokecolor{currentstroke}%
\pgfsetdash{}{0pt}%
\pgfpathmoveto{\pgfqpoint{1.255966in}{2.246901in}}%
\pgfpathlineto{\pgfqpoint{1.415850in}{1.153485in}}%
\pgfusepath{stroke}%
\end{pgfscope}%
\begin{pgfscope}%
\pgfpathrectangle{\pgfqpoint{0.100000in}{0.212622in}}{\pgfqpoint{3.696000in}{3.696000in}}%
\pgfusepath{clip}%
\pgfsetrectcap%
\pgfsetroundjoin%
\pgfsetlinewidth{1.505625pt}%
\definecolor{currentstroke}{rgb}{1.000000,0.000000,0.000000}%
\pgfsetstrokecolor{currentstroke}%
\pgfsetdash{}{0pt}%
\pgfpathmoveto{\pgfqpoint{1.229140in}{2.263478in}}%
\pgfpathlineto{\pgfqpoint{1.387441in}{1.162497in}}%
\pgfusepath{stroke}%
\end{pgfscope}%
\begin{pgfscope}%
\pgfpathrectangle{\pgfqpoint{0.100000in}{0.212622in}}{\pgfqpoint{3.696000in}{3.696000in}}%
\pgfusepath{clip}%
\pgfsetrectcap%
\pgfsetroundjoin%
\pgfsetlinewidth{1.505625pt}%
\definecolor{currentstroke}{rgb}{1.000000,0.000000,0.000000}%
\pgfsetstrokecolor{currentstroke}%
\pgfsetdash{}{0pt}%
\pgfpathmoveto{\pgfqpoint{1.214336in}{2.271780in}}%
\pgfpathlineto{\pgfqpoint{1.373253in}{1.166999in}}%
\pgfusepath{stroke}%
\end{pgfscope}%
\begin{pgfscope}%
\pgfpathrectangle{\pgfqpoint{0.100000in}{0.212622in}}{\pgfqpoint{3.696000in}{3.696000in}}%
\pgfusepath{clip}%
\pgfsetrectcap%
\pgfsetroundjoin%
\pgfsetlinewidth{1.505625pt}%
\definecolor{currentstroke}{rgb}{1.000000,0.000000,0.000000}%
\pgfsetstrokecolor{currentstroke}%
\pgfsetdash{}{0pt}%
\pgfpathmoveto{\pgfqpoint{1.195471in}{2.282387in}}%
\pgfpathlineto{\pgfqpoint{1.359074in}{1.171497in}}%
\pgfusepath{stroke}%
\end{pgfscope}%
\begin{pgfscope}%
\pgfpathrectangle{\pgfqpoint{0.100000in}{0.212622in}}{\pgfqpoint{3.696000in}{3.696000in}}%
\pgfusepath{clip}%
\pgfsetrectcap%
\pgfsetroundjoin%
\pgfsetlinewidth{1.505625pt}%
\definecolor{currentstroke}{rgb}{1.000000,0.000000,0.000000}%
\pgfsetstrokecolor{currentstroke}%
\pgfsetdash{}{0pt}%
\pgfpathmoveto{\pgfqpoint{1.172913in}{2.293152in}}%
\pgfpathlineto{\pgfqpoint{1.330748in}{1.180483in}}%
\pgfusepath{stroke}%
\end{pgfscope}%
\begin{pgfscope}%
\pgfpathrectangle{\pgfqpoint{0.100000in}{0.212622in}}{\pgfqpoint{3.696000in}{3.696000in}}%
\pgfusepath{clip}%
\pgfsetrectcap%
\pgfsetroundjoin%
\pgfsetlinewidth{1.505625pt}%
\definecolor{currentstroke}{rgb}{1.000000,0.000000,0.000000}%
\pgfsetstrokecolor{currentstroke}%
\pgfsetdash{}{0pt}%
\pgfpathmoveto{\pgfqpoint{1.147904in}{2.302828in}}%
\pgfpathlineto{\pgfqpoint{1.316600in}{1.184972in}}%
\pgfusepath{stroke}%
\end{pgfscope}%
\begin{pgfscope}%
\pgfpathrectangle{\pgfqpoint{0.100000in}{0.212622in}}{\pgfqpoint{3.696000in}{3.696000in}}%
\pgfusepath{clip}%
\pgfsetrectcap%
\pgfsetroundjoin%
\pgfsetlinewidth{1.505625pt}%
\definecolor{currentstroke}{rgb}{1.000000,0.000000,0.000000}%
\pgfsetstrokecolor{currentstroke}%
\pgfsetdash{}{0pt}%
\pgfpathmoveto{\pgfqpoint{1.134192in}{2.308345in}}%
\pgfpathlineto{\pgfqpoint{1.302462in}{1.189457in}}%
\pgfusepath{stroke}%
\end{pgfscope}%
\begin{pgfscope}%
\pgfpathrectangle{\pgfqpoint{0.100000in}{0.212622in}}{\pgfqpoint{3.696000in}{3.696000in}}%
\pgfusepath{clip}%
\pgfsetrectcap%
\pgfsetroundjoin%
\pgfsetlinewidth{1.505625pt}%
\definecolor{currentstroke}{rgb}{1.000000,0.000000,0.000000}%
\pgfsetstrokecolor{currentstroke}%
\pgfsetdash{}{0pt}%
\pgfpathmoveto{\pgfqpoint{1.120048in}{2.314044in}}%
\pgfpathlineto{\pgfqpoint{1.288334in}{1.193939in}}%
\pgfusepath{stroke}%
\end{pgfscope}%
\begin{pgfscope}%
\pgfpathrectangle{\pgfqpoint{0.100000in}{0.212622in}}{\pgfqpoint{3.696000in}{3.696000in}}%
\pgfusepath{clip}%
\pgfsetrectcap%
\pgfsetroundjoin%
\pgfsetlinewidth{1.505625pt}%
\definecolor{currentstroke}{rgb}{1.000000,0.000000,0.000000}%
\pgfsetstrokecolor{currentstroke}%
\pgfsetdash{}{0pt}%
\pgfpathmoveto{\pgfqpoint{1.102768in}{2.322683in}}%
\pgfpathlineto{\pgfqpoint{1.274217in}{1.198418in}}%
\pgfusepath{stroke}%
\end{pgfscope}%
\begin{pgfscope}%
\pgfpathrectangle{\pgfqpoint{0.100000in}{0.212622in}}{\pgfqpoint{3.696000in}{3.696000in}}%
\pgfusepath{clip}%
\pgfsetrectcap%
\pgfsetroundjoin%
\pgfsetlinewidth{1.505625pt}%
\definecolor{currentstroke}{rgb}{1.000000,0.000000,0.000000}%
\pgfsetstrokecolor{currentstroke}%
\pgfsetdash{}{0pt}%
\pgfpathmoveto{\pgfqpoint{1.083305in}{2.336437in}}%
\pgfpathlineto{\pgfqpoint{1.246012in}{1.207366in}}%
\pgfusepath{stroke}%
\end{pgfscope}%
\begin{pgfscope}%
\pgfpathrectangle{\pgfqpoint{0.100000in}{0.212622in}}{\pgfqpoint{3.696000in}{3.696000in}}%
\pgfusepath{clip}%
\pgfsetrectcap%
\pgfsetroundjoin%
\pgfsetlinewidth{1.505625pt}%
\definecolor{currentstroke}{rgb}{1.000000,0.000000,0.000000}%
\pgfsetstrokecolor{currentstroke}%
\pgfsetdash{}{0pt}%
\pgfpathmoveto{\pgfqpoint{1.063166in}{2.348931in}}%
\pgfpathlineto{\pgfqpoint{1.231925in}{1.211835in}}%
\pgfusepath{stroke}%
\end{pgfscope}%
\begin{pgfscope}%
\pgfpathrectangle{\pgfqpoint{0.100000in}{0.212622in}}{\pgfqpoint{3.696000in}{3.696000in}}%
\pgfusepath{clip}%
\pgfsetrectcap%
\pgfsetroundjoin%
\pgfsetlinewidth{1.505625pt}%
\definecolor{currentstroke}{rgb}{1.000000,0.000000,0.000000}%
\pgfsetstrokecolor{currentstroke}%
\pgfsetdash{}{0pt}%
\pgfpathmoveto{\pgfqpoint{1.042000in}{2.361621in}}%
\pgfpathlineto{\pgfqpoint{1.217849in}{1.216301in}}%
\pgfusepath{stroke}%
\end{pgfscope}%
\begin{pgfscope}%
\pgfpathrectangle{\pgfqpoint{0.100000in}{0.212622in}}{\pgfqpoint{3.696000in}{3.696000in}}%
\pgfusepath{clip}%
\pgfsetrectcap%
\pgfsetroundjoin%
\pgfsetlinewidth{1.505625pt}%
\definecolor{currentstroke}{rgb}{1.000000,0.000000,0.000000}%
\pgfsetstrokecolor{currentstroke}%
\pgfsetdash{}{0pt}%
\pgfpathmoveto{\pgfqpoint{1.019041in}{2.374135in}}%
\pgfpathlineto{\pgfqpoint{1.189725in}{1.225223in}}%
\pgfusepath{stroke}%
\end{pgfscope}%
\begin{pgfscope}%
\pgfpathrectangle{\pgfqpoint{0.100000in}{0.212622in}}{\pgfqpoint{3.696000in}{3.696000in}}%
\pgfusepath{clip}%
\pgfsetrectcap%
\pgfsetroundjoin%
\pgfsetlinewidth{1.505625pt}%
\definecolor{currentstroke}{rgb}{1.000000,0.000000,0.000000}%
\pgfsetstrokecolor{currentstroke}%
\pgfsetdash{}{0pt}%
\pgfpathmoveto{\pgfqpoint{0.992780in}{2.387539in}}%
\pgfpathlineto{\pgfqpoint{1.161642in}{1.234132in}}%
\pgfusepath{stroke}%
\end{pgfscope}%
\begin{pgfscope}%
\pgfpathrectangle{\pgfqpoint{0.100000in}{0.212622in}}{\pgfqpoint{3.696000in}{3.696000in}}%
\pgfusepath{clip}%
\pgfsetrectcap%
\pgfsetroundjoin%
\pgfsetlinewidth{1.505625pt}%
\definecolor{currentstroke}{rgb}{1.000000,0.000000,0.000000}%
\pgfsetstrokecolor{currentstroke}%
\pgfsetdash{}{0pt}%
\pgfpathmoveto{\pgfqpoint{0.965471in}{2.401605in}}%
\pgfpathlineto{\pgfqpoint{1.147616in}{1.238582in}}%
\pgfusepath{stroke}%
\end{pgfscope}%
\begin{pgfscope}%
\pgfpathrectangle{\pgfqpoint{0.100000in}{0.212622in}}{\pgfqpoint{3.696000in}{3.696000in}}%
\pgfusepath{clip}%
\pgfsetrectcap%
\pgfsetroundjoin%
\pgfsetlinewidth{1.505625pt}%
\definecolor{currentstroke}{rgb}{1.000000,0.000000,0.000000}%
\pgfsetstrokecolor{currentstroke}%
\pgfsetdash{}{0pt}%
\pgfpathmoveto{\pgfqpoint{0.950394in}{2.408966in}}%
\pgfpathlineto{\pgfqpoint{1.133599in}{1.243029in}}%
\pgfusepath{stroke}%
\end{pgfscope}%
\begin{pgfscope}%
\pgfpathrectangle{\pgfqpoint{0.100000in}{0.212622in}}{\pgfqpoint{3.696000in}{3.696000in}}%
\pgfusepath{clip}%
\pgfsetrectcap%
\pgfsetroundjoin%
\pgfsetlinewidth{1.505625pt}%
\definecolor{currentstroke}{rgb}{1.000000,0.000000,0.000000}%
\pgfsetstrokecolor{currentstroke}%
\pgfsetdash{}{0pt}%
\pgfpathmoveto{\pgfqpoint{0.934343in}{2.416894in}}%
\pgfpathlineto{\pgfqpoint{1.105597in}{1.251913in}}%
\pgfusepath{stroke}%
\end{pgfscope}%
\begin{pgfscope}%
\pgfpathrectangle{\pgfqpoint{0.100000in}{0.212622in}}{\pgfqpoint{3.696000in}{3.696000in}}%
\pgfusepath{clip}%
\pgfsetrectcap%
\pgfsetroundjoin%
\pgfsetlinewidth{1.505625pt}%
\definecolor{currentstroke}{rgb}{1.000000,0.000000,0.000000}%
\pgfsetstrokecolor{currentstroke}%
\pgfsetdash{}{0pt}%
\pgfpathmoveto{\pgfqpoint{0.913688in}{2.426069in}}%
\pgfpathlineto{\pgfqpoint{1.091610in}{1.256350in}}%
\pgfusepath{stroke}%
\end{pgfscope}%
\begin{pgfscope}%
\pgfpathrectangle{\pgfqpoint{0.100000in}{0.212622in}}{\pgfqpoint{3.696000in}{3.696000in}}%
\pgfusepath{clip}%
\pgfsetrectcap%
\pgfsetroundjoin%
\pgfsetlinewidth{1.505625pt}%
\definecolor{currentstroke}{rgb}{1.000000,0.000000,0.000000}%
\pgfsetstrokecolor{currentstroke}%
\pgfsetdash{}{0pt}%
\pgfpathmoveto{\pgfqpoint{0.892145in}{2.438423in}}%
\pgfpathlineto{\pgfqpoint{1.077634in}{1.260784in}}%
\pgfusepath{stroke}%
\end{pgfscope}%
\begin{pgfscope}%
\pgfpathrectangle{\pgfqpoint{0.100000in}{0.212622in}}{\pgfqpoint{3.696000in}{3.696000in}}%
\pgfusepath{clip}%
\pgfsetrectcap%
\pgfsetroundjoin%
\pgfsetlinewidth{1.505625pt}%
\definecolor{currentstroke}{rgb}{1.000000,0.000000,0.000000}%
\pgfsetstrokecolor{currentstroke}%
\pgfsetdash{}{0pt}%
\pgfpathmoveto{\pgfqpoint{0.870005in}{2.452487in}}%
\pgfpathlineto{\pgfqpoint{1.049711in}{1.269642in}}%
\pgfusepath{stroke}%
\end{pgfscope}%
\begin{pgfscope}%
\pgfpathrectangle{\pgfqpoint{0.100000in}{0.212622in}}{\pgfqpoint{3.696000in}{3.696000in}}%
\pgfusepath{clip}%
\pgfsetrectcap%
\pgfsetroundjoin%
\pgfsetlinewidth{1.505625pt}%
\definecolor{currentstroke}{rgb}{1.000000,0.000000,0.000000}%
\pgfsetstrokecolor{currentstroke}%
\pgfsetdash{}{0pt}%
\pgfpathmoveto{\pgfqpoint{0.857779in}{2.459721in}}%
\pgfpathlineto{\pgfqpoint{1.035765in}{1.274067in}}%
\pgfusepath{stroke}%
\end{pgfscope}%
\begin{pgfscope}%
\pgfpathrectangle{\pgfqpoint{0.100000in}{0.212622in}}{\pgfqpoint{3.696000in}{3.696000in}}%
\pgfusepath{clip}%
\pgfsetrectcap%
\pgfsetroundjoin%
\pgfsetlinewidth{1.505625pt}%
\definecolor{currentstroke}{rgb}{1.000000,0.000000,0.000000}%
\pgfsetstrokecolor{currentstroke}%
\pgfsetdash{}{0pt}%
\pgfpathmoveto{\pgfqpoint{0.842887in}{2.468507in}}%
\pgfpathlineto{\pgfqpoint{1.021829in}{1.278488in}}%
\pgfusepath{stroke}%
\end{pgfscope}%
\begin{pgfscope}%
\pgfpathrectangle{\pgfqpoint{0.100000in}{0.212622in}}{\pgfqpoint{3.696000in}{3.696000in}}%
\pgfusepath{clip}%
\pgfsetrectcap%
\pgfsetroundjoin%
\pgfsetlinewidth{1.505625pt}%
\definecolor{currentstroke}{rgb}{1.000000,0.000000,0.000000}%
\pgfsetstrokecolor{currentstroke}%
\pgfsetdash{}{0pt}%
\pgfpathmoveto{\pgfqpoint{0.825143in}{2.477030in}}%
\pgfpathlineto{\pgfqpoint{1.007902in}{1.282906in}}%
\pgfusepath{stroke}%
\end{pgfscope}%
\begin{pgfscope}%
\pgfpathrectangle{\pgfqpoint{0.100000in}{0.212622in}}{\pgfqpoint{3.696000in}{3.696000in}}%
\pgfusepath{clip}%
\pgfsetrectcap%
\pgfsetroundjoin%
\pgfsetlinewidth{1.505625pt}%
\definecolor{currentstroke}{rgb}{1.000000,0.000000,0.000000}%
\pgfsetstrokecolor{currentstroke}%
\pgfsetdash{}{0pt}%
\pgfpathmoveto{\pgfqpoint{0.806513in}{2.484929in}}%
\pgfpathlineto{\pgfqpoint{0.993986in}{1.287321in}}%
\pgfusepath{stroke}%
\end{pgfscope}%
\begin{pgfscope}%
\pgfpathrectangle{\pgfqpoint{0.100000in}{0.212622in}}{\pgfqpoint{3.696000in}{3.696000in}}%
\pgfusepath{clip}%
\pgfsetrectcap%
\pgfsetroundjoin%
\pgfsetlinewidth{1.505625pt}%
\definecolor{currentstroke}{rgb}{1.000000,0.000000,0.000000}%
\pgfsetstrokecolor{currentstroke}%
\pgfsetdash{}{0pt}%
\pgfpathmoveto{\pgfqpoint{0.796329in}{2.489315in}}%
\pgfpathlineto{\pgfqpoint{0.980079in}{1.291733in}}%
\pgfusepath{stroke}%
\end{pgfscope}%
\begin{pgfscope}%
\pgfpathrectangle{\pgfqpoint{0.100000in}{0.212622in}}{\pgfqpoint{3.696000in}{3.696000in}}%
\pgfusepath{clip}%
\pgfsetrectcap%
\pgfsetroundjoin%
\pgfsetlinewidth{1.505625pt}%
\definecolor{currentstroke}{rgb}{1.000000,0.000000,0.000000}%
\pgfsetstrokecolor{currentstroke}%
\pgfsetdash{}{0pt}%
\pgfpathmoveto{\pgfqpoint{0.785409in}{2.494011in}}%
\pgfpathlineto{\pgfqpoint{0.966183in}{1.296141in}}%
\pgfusepath{stroke}%
\end{pgfscope}%
\begin{pgfscope}%
\pgfpathrectangle{\pgfqpoint{0.100000in}{0.212622in}}{\pgfqpoint{3.696000in}{3.696000in}}%
\pgfusepath{clip}%
\pgfsetrectcap%
\pgfsetroundjoin%
\pgfsetlinewidth{1.505625pt}%
\definecolor{currentstroke}{rgb}{1.000000,0.000000,0.000000}%
\pgfsetstrokecolor{currentstroke}%
\pgfsetdash{}{0pt}%
\pgfpathmoveto{\pgfqpoint{0.772256in}{2.500480in}}%
\pgfpathlineto{\pgfqpoint{0.952296in}{1.300547in}}%
\pgfusepath{stroke}%
\end{pgfscope}%
\begin{pgfscope}%
\pgfpathrectangle{\pgfqpoint{0.100000in}{0.212622in}}{\pgfqpoint{3.696000in}{3.696000in}}%
\pgfusepath{clip}%
\pgfsetrectcap%
\pgfsetroundjoin%
\pgfsetlinewidth{1.505625pt}%
\definecolor{currentstroke}{rgb}{1.000000,0.000000,0.000000}%
\pgfsetstrokecolor{currentstroke}%
\pgfsetdash{}{0pt}%
\pgfpathmoveto{\pgfqpoint{0.756497in}{2.508595in}}%
\pgfpathlineto{\pgfqpoint{0.952296in}{1.300547in}}%
\pgfusepath{stroke}%
\end{pgfscope}%
\begin{pgfscope}%
\pgfpathrectangle{\pgfqpoint{0.100000in}{0.212622in}}{\pgfqpoint{3.696000in}{3.696000in}}%
\pgfusepath{clip}%
\pgfsetrectcap%
\pgfsetroundjoin%
\pgfsetlinewidth{1.505625pt}%
\definecolor{currentstroke}{rgb}{1.000000,0.000000,0.000000}%
\pgfsetstrokecolor{currentstroke}%
\pgfsetdash{}{0pt}%
\pgfpathmoveto{\pgfqpoint{0.747838in}{2.512809in}}%
\pgfpathlineto{\pgfqpoint{0.952296in}{1.300547in}}%
\pgfusepath{stroke}%
\end{pgfscope}%
\begin{pgfscope}%
\pgfpathrectangle{\pgfqpoint{0.100000in}{0.212622in}}{\pgfqpoint{3.696000in}{3.696000in}}%
\pgfusepath{clip}%
\pgfsetrectcap%
\pgfsetroundjoin%
\pgfsetlinewidth{1.505625pt}%
\definecolor{currentstroke}{rgb}{1.000000,0.000000,0.000000}%
\pgfsetstrokecolor{currentstroke}%
\pgfsetdash{}{0pt}%
\pgfpathmoveto{\pgfqpoint{0.736665in}{2.518437in}}%
\pgfpathlineto{\pgfqpoint{0.952296in}{1.300547in}}%
\pgfusepath{stroke}%
\end{pgfscope}%
\begin{pgfscope}%
\pgfpathrectangle{\pgfqpoint{0.100000in}{0.212622in}}{\pgfqpoint{3.696000in}{3.696000in}}%
\pgfusepath{clip}%
\pgfsetrectcap%
\pgfsetroundjoin%
\pgfsetlinewidth{1.505625pt}%
\definecolor{currentstroke}{rgb}{1.000000,0.000000,0.000000}%
\pgfsetstrokecolor{currentstroke}%
\pgfsetdash{}{0pt}%
\pgfpathmoveto{\pgfqpoint{0.722157in}{2.524553in}}%
\pgfpathlineto{\pgfqpoint{0.952296in}{1.300547in}}%
\pgfusepath{stroke}%
\end{pgfscope}%
\begin{pgfscope}%
\pgfpathrectangle{\pgfqpoint{0.100000in}{0.212622in}}{\pgfqpoint{3.696000in}{3.696000in}}%
\pgfusepath{clip}%
\pgfsetrectcap%
\pgfsetroundjoin%
\pgfsetlinewidth{1.505625pt}%
\definecolor{currentstroke}{rgb}{1.000000,0.000000,0.000000}%
\pgfsetstrokecolor{currentstroke}%
\pgfsetdash{}{0pt}%
\pgfpathmoveto{\pgfqpoint{0.705341in}{2.531464in}}%
\pgfpathlineto{\pgfqpoint{0.952296in}{1.300547in}}%
\pgfusepath{stroke}%
\end{pgfscope}%
\begin{pgfscope}%
\pgfpathrectangle{\pgfqpoint{0.100000in}{0.212622in}}{\pgfqpoint{3.696000in}{3.696000in}}%
\pgfusepath{clip}%
\pgfsetrectcap%
\pgfsetroundjoin%
\pgfsetlinewidth{1.505625pt}%
\definecolor{currentstroke}{rgb}{1.000000,0.000000,0.000000}%
\pgfsetstrokecolor{currentstroke}%
\pgfsetdash{}{0pt}%
\pgfpathmoveto{\pgfqpoint{0.696129in}{2.535790in}}%
\pgfpathlineto{\pgfqpoint{0.952296in}{1.300547in}}%
\pgfusepath{stroke}%
\end{pgfscope}%
\begin{pgfscope}%
\pgfpathrectangle{\pgfqpoint{0.100000in}{0.212622in}}{\pgfqpoint{3.696000in}{3.696000in}}%
\pgfusepath{clip}%
\pgfsetrectcap%
\pgfsetroundjoin%
\pgfsetlinewidth{1.505625pt}%
\definecolor{currentstroke}{rgb}{1.000000,0.000000,0.000000}%
\pgfsetstrokecolor{currentstroke}%
\pgfsetdash{}{0pt}%
\pgfpathmoveto{\pgfqpoint{0.686015in}{2.540340in}}%
\pgfpathlineto{\pgfqpoint{0.952296in}{1.300547in}}%
\pgfusepath{stroke}%
\end{pgfscope}%
\begin{pgfscope}%
\pgfpathrectangle{\pgfqpoint{0.100000in}{0.212622in}}{\pgfqpoint{3.696000in}{3.696000in}}%
\pgfusepath{clip}%
\pgfsetrectcap%
\pgfsetroundjoin%
\pgfsetlinewidth{1.505625pt}%
\definecolor{currentstroke}{rgb}{1.000000,0.000000,0.000000}%
\pgfsetstrokecolor{currentstroke}%
\pgfsetdash{}{0pt}%
\pgfpathmoveto{\pgfqpoint{0.673212in}{2.546769in}}%
\pgfpathlineto{\pgfqpoint{0.952296in}{1.300547in}}%
\pgfusepath{stroke}%
\end{pgfscope}%
\begin{pgfscope}%
\pgfpathrectangle{\pgfqpoint{0.100000in}{0.212622in}}{\pgfqpoint{3.696000in}{3.696000in}}%
\pgfusepath{clip}%
\pgfsetrectcap%
\pgfsetroundjoin%
\pgfsetlinewidth{1.505625pt}%
\definecolor{currentstroke}{rgb}{1.000000,0.000000,0.000000}%
\pgfsetstrokecolor{currentstroke}%
\pgfsetdash{}{0pt}%
\pgfpathmoveto{\pgfqpoint{0.658806in}{2.553508in}}%
\pgfpathlineto{\pgfqpoint{0.952296in}{1.300547in}}%
\pgfusepath{stroke}%
\end{pgfscope}%
\begin{pgfscope}%
\pgfpathrectangle{\pgfqpoint{0.100000in}{0.212622in}}{\pgfqpoint{3.696000in}{3.696000in}}%
\pgfusepath{clip}%
\pgfsetrectcap%
\pgfsetroundjoin%
\pgfsetlinewidth{1.505625pt}%
\definecolor{currentstroke}{rgb}{1.000000,0.000000,0.000000}%
\pgfsetstrokecolor{currentstroke}%
\pgfsetdash{}{0pt}%
\pgfpathmoveto{\pgfqpoint{0.641992in}{2.559328in}}%
\pgfpathlineto{\pgfqpoint{0.952296in}{1.300547in}}%
\pgfusepath{stroke}%
\end{pgfscope}%
\begin{pgfscope}%
\pgfpathrectangle{\pgfqpoint{0.100000in}{0.212622in}}{\pgfqpoint{3.696000in}{3.696000in}}%
\pgfusepath{clip}%
\pgfsetrectcap%
\pgfsetroundjoin%
\pgfsetlinewidth{1.505625pt}%
\definecolor{currentstroke}{rgb}{1.000000,0.000000,0.000000}%
\pgfsetstrokecolor{currentstroke}%
\pgfsetdash{}{0pt}%
\pgfpathmoveto{\pgfqpoint{0.632811in}{2.562739in}}%
\pgfpathlineto{\pgfqpoint{0.952296in}{1.300547in}}%
\pgfusepath{stroke}%
\end{pgfscope}%
\begin{pgfscope}%
\pgfpathrectangle{\pgfqpoint{0.100000in}{0.212622in}}{\pgfqpoint{3.696000in}{3.696000in}}%
\pgfusepath{clip}%
\pgfsetrectcap%
\pgfsetroundjoin%
\pgfsetlinewidth{1.505625pt}%
\definecolor{currentstroke}{rgb}{1.000000,0.000000,0.000000}%
\pgfsetstrokecolor{currentstroke}%
\pgfsetdash{}{0pt}%
\pgfpathmoveto{\pgfqpoint{0.621932in}{2.566959in}}%
\pgfpathlineto{\pgfqpoint{0.952296in}{1.300547in}}%
\pgfusepath{stroke}%
\end{pgfscope}%
\begin{pgfscope}%
\pgfpathrectangle{\pgfqpoint{0.100000in}{0.212622in}}{\pgfqpoint{3.696000in}{3.696000in}}%
\pgfusepath{clip}%
\pgfsetrectcap%
\pgfsetroundjoin%
\pgfsetlinewidth{1.505625pt}%
\definecolor{currentstroke}{rgb}{1.000000,0.000000,0.000000}%
\pgfsetstrokecolor{currentstroke}%
\pgfsetdash{}{0pt}%
\pgfpathmoveto{\pgfqpoint{0.610067in}{2.572263in}}%
\pgfpathlineto{\pgfqpoint{0.952296in}{1.300547in}}%
\pgfusepath{stroke}%
\end{pgfscope}%
\begin{pgfscope}%
\pgfpathrectangle{\pgfqpoint{0.100000in}{0.212622in}}{\pgfqpoint{3.696000in}{3.696000in}}%
\pgfusepath{clip}%
\pgfsetrectcap%
\pgfsetroundjoin%
\pgfsetlinewidth{1.505625pt}%
\definecolor{currentstroke}{rgb}{1.000000,0.000000,0.000000}%
\pgfsetstrokecolor{currentstroke}%
\pgfsetdash{}{0pt}%
\pgfpathmoveto{\pgfqpoint{0.595445in}{2.579172in}}%
\pgfpathlineto{\pgfqpoint{0.952296in}{1.300547in}}%
\pgfusepath{stroke}%
\end{pgfscope}%
\begin{pgfscope}%
\pgfpathrectangle{\pgfqpoint{0.100000in}{0.212622in}}{\pgfqpoint{3.696000in}{3.696000in}}%
\pgfusepath{clip}%
\pgfsetrectcap%
\pgfsetroundjoin%
\pgfsetlinewidth{1.505625pt}%
\definecolor{currentstroke}{rgb}{1.000000,0.000000,0.000000}%
\pgfsetstrokecolor{currentstroke}%
\pgfsetdash{}{0pt}%
\pgfpathmoveto{\pgfqpoint{0.580285in}{2.588310in}}%
\pgfpathlineto{\pgfqpoint{0.952296in}{1.300547in}}%
\pgfusepath{stroke}%
\end{pgfscope}%
\begin{pgfscope}%
\pgfpathrectangle{\pgfqpoint{0.100000in}{0.212622in}}{\pgfqpoint{3.696000in}{3.696000in}}%
\pgfusepath{clip}%
\pgfsetrectcap%
\pgfsetroundjoin%
\pgfsetlinewidth{1.505625pt}%
\definecolor{currentstroke}{rgb}{1.000000,0.000000,0.000000}%
\pgfsetstrokecolor{currentstroke}%
\pgfsetdash{}{0pt}%
\pgfpathmoveto{\pgfqpoint{0.571989in}{2.592658in}}%
\pgfpathlineto{\pgfqpoint{0.952296in}{1.300547in}}%
\pgfusepath{stroke}%
\end{pgfscope}%
\begin{pgfscope}%
\pgfpathrectangle{\pgfqpoint{0.100000in}{0.212622in}}{\pgfqpoint{3.696000in}{3.696000in}}%
\pgfusepath{clip}%
\pgfsetrectcap%
\pgfsetroundjoin%
\pgfsetlinewidth{1.505625pt}%
\definecolor{currentstroke}{rgb}{1.000000,0.000000,0.000000}%
\pgfsetstrokecolor{currentstroke}%
\pgfsetdash{}{0pt}%
\pgfpathmoveto{\pgfqpoint{0.567487in}{2.594486in}}%
\pgfpathlineto{\pgfqpoint{0.952296in}{1.300547in}}%
\pgfusepath{stroke}%
\end{pgfscope}%
\begin{pgfscope}%
\pgfpathrectangle{\pgfqpoint{0.100000in}{0.212622in}}{\pgfqpoint{3.696000in}{3.696000in}}%
\pgfusepath{clip}%
\pgfsetrectcap%
\pgfsetroundjoin%
\pgfsetlinewidth{1.505625pt}%
\definecolor{currentstroke}{rgb}{1.000000,0.000000,0.000000}%
\pgfsetstrokecolor{currentstroke}%
\pgfsetdash{}{0pt}%
\pgfpathmoveto{\pgfqpoint{0.565018in}{2.595292in}}%
\pgfpathlineto{\pgfqpoint{0.952296in}{1.300547in}}%
\pgfusepath{stroke}%
\end{pgfscope}%
\begin{pgfscope}%
\pgfpathrectangle{\pgfqpoint{0.100000in}{0.212622in}}{\pgfqpoint{3.696000in}{3.696000in}}%
\pgfusepath{clip}%
\pgfsetrectcap%
\pgfsetroundjoin%
\pgfsetlinewidth{1.505625pt}%
\definecolor{currentstroke}{rgb}{1.000000,0.000000,0.000000}%
\pgfsetstrokecolor{currentstroke}%
\pgfsetdash{}{0pt}%
\pgfpathmoveto{\pgfqpoint{0.561876in}{2.596946in}}%
\pgfpathlineto{\pgfqpoint{0.952296in}{1.300547in}}%
\pgfusepath{stroke}%
\end{pgfscope}%
\begin{pgfscope}%
\pgfpathrectangle{\pgfqpoint{0.100000in}{0.212622in}}{\pgfqpoint{3.696000in}{3.696000in}}%
\pgfusepath{clip}%
\pgfsetrectcap%
\pgfsetroundjoin%
\pgfsetlinewidth{1.505625pt}%
\definecolor{currentstroke}{rgb}{1.000000,0.000000,0.000000}%
\pgfsetstrokecolor{currentstroke}%
\pgfsetdash{}{0pt}%
\pgfpathmoveto{\pgfqpoint{0.560167in}{2.598211in}}%
\pgfpathlineto{\pgfqpoint{0.952296in}{1.300547in}}%
\pgfusepath{stroke}%
\end{pgfscope}%
\begin{pgfscope}%
\pgfpathrectangle{\pgfqpoint{0.100000in}{0.212622in}}{\pgfqpoint{3.696000in}{3.696000in}}%
\pgfusepath{clip}%
\pgfsetrectcap%
\pgfsetroundjoin%
\pgfsetlinewidth{1.505625pt}%
\definecolor{currentstroke}{rgb}{1.000000,0.000000,0.000000}%
\pgfsetstrokecolor{currentstroke}%
\pgfsetdash{}{0pt}%
\pgfpathmoveto{\pgfqpoint{0.555515in}{2.601147in}}%
\pgfpathlineto{\pgfqpoint{0.952296in}{1.300547in}}%
\pgfusepath{stroke}%
\end{pgfscope}%
\begin{pgfscope}%
\pgfpathrectangle{\pgfqpoint{0.100000in}{0.212622in}}{\pgfqpoint{3.696000in}{3.696000in}}%
\pgfusepath{clip}%
\pgfsetrectcap%
\pgfsetroundjoin%
\pgfsetlinewidth{1.505625pt}%
\definecolor{currentstroke}{rgb}{1.000000,0.000000,0.000000}%
\pgfsetstrokecolor{currentstroke}%
\pgfsetdash{}{0pt}%
\pgfpathmoveto{\pgfqpoint{0.549481in}{2.604238in}}%
\pgfpathlineto{\pgfqpoint{0.952296in}{1.300547in}}%
\pgfusepath{stroke}%
\end{pgfscope}%
\begin{pgfscope}%
\pgfpathrectangle{\pgfqpoint{0.100000in}{0.212622in}}{\pgfqpoint{3.696000in}{3.696000in}}%
\pgfusepath{clip}%
\pgfsetrectcap%
\pgfsetroundjoin%
\pgfsetlinewidth{1.505625pt}%
\definecolor{currentstroke}{rgb}{1.000000,0.000000,0.000000}%
\pgfsetstrokecolor{currentstroke}%
\pgfsetdash{}{0pt}%
\pgfpathmoveto{\pgfqpoint{0.542142in}{2.608352in}}%
\pgfpathlineto{\pgfqpoint{0.952296in}{1.300547in}}%
\pgfusepath{stroke}%
\end{pgfscope}%
\begin{pgfscope}%
\pgfpathrectangle{\pgfqpoint{0.100000in}{0.212622in}}{\pgfqpoint{3.696000in}{3.696000in}}%
\pgfusepath{clip}%
\pgfsetrectcap%
\pgfsetroundjoin%
\pgfsetlinewidth{1.505625pt}%
\definecolor{currentstroke}{rgb}{1.000000,0.000000,0.000000}%
\pgfsetstrokecolor{currentstroke}%
\pgfsetdash{}{0pt}%
\pgfpathmoveto{\pgfqpoint{0.538069in}{2.609981in}}%
\pgfpathlineto{\pgfqpoint{0.952296in}{1.300547in}}%
\pgfusepath{stroke}%
\end{pgfscope}%
\begin{pgfscope}%
\pgfpathrectangle{\pgfqpoint{0.100000in}{0.212622in}}{\pgfqpoint{3.696000in}{3.696000in}}%
\pgfusepath{clip}%
\pgfsetrectcap%
\pgfsetroundjoin%
\pgfsetlinewidth{1.505625pt}%
\definecolor{currentstroke}{rgb}{1.000000,0.000000,0.000000}%
\pgfsetstrokecolor{currentstroke}%
\pgfsetdash{}{0pt}%
\pgfpathmoveto{\pgfqpoint{0.533294in}{2.611363in}}%
\pgfpathlineto{\pgfqpoint{0.952296in}{1.300547in}}%
\pgfusepath{stroke}%
\end{pgfscope}%
\begin{pgfscope}%
\pgfpathrectangle{\pgfqpoint{0.100000in}{0.212622in}}{\pgfqpoint{3.696000in}{3.696000in}}%
\pgfusepath{clip}%
\pgfsetrectcap%
\pgfsetroundjoin%
\pgfsetlinewidth{1.505625pt}%
\definecolor{currentstroke}{rgb}{1.000000,0.000000,0.000000}%
\pgfsetstrokecolor{currentstroke}%
\pgfsetdash{}{0pt}%
\pgfpathmoveto{\pgfqpoint{0.530609in}{2.612005in}}%
\pgfpathlineto{\pgfqpoint{0.952296in}{1.300547in}}%
\pgfusepath{stroke}%
\end{pgfscope}%
\begin{pgfscope}%
\pgfpathrectangle{\pgfqpoint{0.100000in}{0.212622in}}{\pgfqpoint{3.696000in}{3.696000in}}%
\pgfusepath{clip}%
\pgfsetrectcap%
\pgfsetroundjoin%
\pgfsetlinewidth{1.505625pt}%
\definecolor{currentstroke}{rgb}{1.000000,0.000000,0.000000}%
\pgfsetstrokecolor{currentstroke}%
\pgfsetdash{}{0pt}%
\pgfpathmoveto{\pgfqpoint{0.527370in}{2.612780in}}%
\pgfpathlineto{\pgfqpoint{0.952296in}{1.300547in}}%
\pgfusepath{stroke}%
\end{pgfscope}%
\begin{pgfscope}%
\pgfpathrectangle{\pgfqpoint{0.100000in}{0.212622in}}{\pgfqpoint{3.696000in}{3.696000in}}%
\pgfusepath{clip}%
\pgfsetbuttcap%
\pgfsetroundjoin%
\definecolor{currentfill}{rgb}{0.121569,0.466667,0.705882}%
\pgfsetfillcolor{currentfill}%
\pgfsetfillopacity{0.300000}%
\pgfsetlinewidth{1.003750pt}%
\definecolor{currentstroke}{rgb}{0.121569,0.466667,0.705882}%
\pgfsetstrokecolor{currentstroke}%
\pgfsetstrokeopacity{0.300000}%
\pgfsetdash{}{0pt}%
\pgfpathmoveto{\pgfqpoint{1.668109in}{2.550873in}}%
\pgfpathcurveto{\pgfqpoint{1.676346in}{2.550873in}}{\pgfqpoint{1.684246in}{2.554146in}}{\pgfqpoint{1.690070in}{2.559969in}}%
\pgfpathcurveto{\pgfqpoint{1.695894in}{2.565793in}}{\pgfqpoint{1.699166in}{2.573693in}}{\pgfqpoint{1.699166in}{2.581930in}}%
\pgfpathcurveto{\pgfqpoint{1.699166in}{2.590166in}}{\pgfqpoint{1.695894in}{2.598066in}}{\pgfqpoint{1.690070in}{2.603890in}}%
\pgfpathcurveto{\pgfqpoint{1.684246in}{2.609714in}}{\pgfqpoint{1.676346in}{2.612986in}}{\pgfqpoint{1.668109in}{2.612986in}}%
\pgfpathcurveto{\pgfqpoint{1.659873in}{2.612986in}}{\pgfqpoint{1.651973in}{2.609714in}}{\pgfqpoint{1.646149in}{2.603890in}}%
\pgfpathcurveto{\pgfqpoint{1.640325in}{2.598066in}}{\pgfqpoint{1.637053in}{2.590166in}}{\pgfqpoint{1.637053in}{2.581930in}}%
\pgfpathcurveto{\pgfqpoint{1.637053in}{2.573693in}}{\pgfqpoint{1.640325in}{2.565793in}}{\pgfqpoint{1.646149in}{2.559969in}}%
\pgfpathcurveto{\pgfqpoint{1.651973in}{2.554146in}}{\pgfqpoint{1.659873in}{2.550873in}}{\pgfqpoint{1.668109in}{2.550873in}}%
\pgfpathclose%
\pgfusepath{stroke,fill}%
\end{pgfscope}%
\begin{pgfscope}%
\pgfpathrectangle{\pgfqpoint{0.100000in}{0.212622in}}{\pgfqpoint{3.696000in}{3.696000in}}%
\pgfusepath{clip}%
\pgfsetbuttcap%
\pgfsetroundjoin%
\definecolor{currentfill}{rgb}{0.121569,0.466667,0.705882}%
\pgfsetfillcolor{currentfill}%
\pgfsetfillopacity{0.300013}%
\pgfsetlinewidth{1.003750pt}%
\definecolor{currentstroke}{rgb}{0.121569,0.466667,0.705882}%
\pgfsetstrokecolor{currentstroke}%
\pgfsetstrokeopacity{0.300013}%
\pgfsetdash{}{0pt}%
\pgfpathmoveto{\pgfqpoint{1.663026in}{2.553026in}}%
\pgfpathcurveto{\pgfqpoint{1.671262in}{2.553026in}}{\pgfqpoint{1.679162in}{2.556298in}}{\pgfqpoint{1.684986in}{2.562122in}}%
\pgfpathcurveto{\pgfqpoint{1.690810in}{2.567946in}}{\pgfqpoint{1.694083in}{2.575846in}}{\pgfqpoint{1.694083in}{2.584083in}}%
\pgfpathcurveto{\pgfqpoint{1.694083in}{2.592319in}}{\pgfqpoint{1.690810in}{2.600219in}}{\pgfqpoint{1.684986in}{2.606043in}}%
\pgfpathcurveto{\pgfqpoint{1.679162in}{2.611867in}}{\pgfqpoint{1.671262in}{2.615139in}}{\pgfqpoint{1.663026in}{2.615139in}}%
\pgfpathcurveto{\pgfqpoint{1.654790in}{2.615139in}}{\pgfqpoint{1.646890in}{2.611867in}}{\pgfqpoint{1.641066in}{2.606043in}}%
\pgfpathcurveto{\pgfqpoint{1.635242in}{2.600219in}}{\pgfqpoint{1.631970in}{2.592319in}}{\pgfqpoint{1.631970in}{2.584083in}}%
\pgfpathcurveto{\pgfqpoint{1.631970in}{2.575846in}}{\pgfqpoint{1.635242in}{2.567946in}}{\pgfqpoint{1.641066in}{2.562122in}}%
\pgfpathcurveto{\pgfqpoint{1.646890in}{2.556298in}}{\pgfqpoint{1.654790in}{2.553026in}}{\pgfqpoint{1.663026in}{2.553026in}}%
\pgfpathclose%
\pgfusepath{stroke,fill}%
\end{pgfscope}%
\begin{pgfscope}%
\pgfpathrectangle{\pgfqpoint{0.100000in}{0.212622in}}{\pgfqpoint{3.696000in}{3.696000in}}%
\pgfusepath{clip}%
\pgfsetbuttcap%
\pgfsetroundjoin%
\definecolor{currentfill}{rgb}{0.121569,0.466667,0.705882}%
\pgfsetfillcolor{currentfill}%
\pgfsetfillopacity{0.300013}%
\pgfsetlinewidth{1.003750pt}%
\definecolor{currentstroke}{rgb}{0.121569,0.466667,0.705882}%
\pgfsetstrokecolor{currentstroke}%
\pgfsetstrokeopacity{0.300013}%
\pgfsetdash{}{0pt}%
\pgfpathmoveto{\pgfqpoint{1.658632in}{2.554435in}}%
\pgfpathcurveto{\pgfqpoint{1.666869in}{2.554435in}}{\pgfqpoint{1.674769in}{2.557707in}}{\pgfqpoint{1.680593in}{2.563531in}}%
\pgfpathcurveto{\pgfqpoint{1.686417in}{2.569355in}}{\pgfqpoint{1.689689in}{2.577255in}}{\pgfqpoint{1.689689in}{2.585492in}}%
\pgfpathcurveto{\pgfqpoint{1.689689in}{2.593728in}}{\pgfqpoint{1.686417in}{2.601628in}}{\pgfqpoint{1.680593in}{2.607452in}}%
\pgfpathcurveto{\pgfqpoint{1.674769in}{2.613276in}}{\pgfqpoint{1.666869in}{2.616548in}}{\pgfqpoint{1.658632in}{2.616548in}}%
\pgfpathcurveto{\pgfqpoint{1.650396in}{2.616548in}}{\pgfqpoint{1.642496in}{2.613276in}}{\pgfqpoint{1.636672in}{2.607452in}}%
\pgfpathcurveto{\pgfqpoint{1.630848in}{2.601628in}}{\pgfqpoint{1.627576in}{2.593728in}}{\pgfqpoint{1.627576in}{2.585492in}}%
\pgfpathcurveto{\pgfqpoint{1.627576in}{2.577255in}}{\pgfqpoint{1.630848in}{2.569355in}}{\pgfqpoint{1.636672in}{2.563531in}}%
\pgfpathcurveto{\pgfqpoint{1.642496in}{2.557707in}}{\pgfqpoint{1.650396in}{2.554435in}}{\pgfqpoint{1.658632in}{2.554435in}}%
\pgfpathclose%
\pgfusepath{stroke,fill}%
\end{pgfscope}%
\begin{pgfscope}%
\pgfpathrectangle{\pgfqpoint{0.100000in}{0.212622in}}{\pgfqpoint{3.696000in}{3.696000in}}%
\pgfusepath{clip}%
\pgfsetbuttcap%
\pgfsetroundjoin%
\definecolor{currentfill}{rgb}{0.121569,0.466667,0.705882}%
\pgfsetfillcolor{currentfill}%
\pgfsetfillopacity{0.300027}%
\pgfsetlinewidth{1.003750pt}%
\definecolor{currentstroke}{rgb}{0.121569,0.466667,0.705882}%
\pgfsetstrokecolor{currentstroke}%
\pgfsetstrokeopacity{0.300027}%
\pgfsetdash{}{0pt}%
\pgfpathmoveto{\pgfqpoint{1.657550in}{2.554775in}}%
\pgfpathcurveto{\pgfqpoint{1.665786in}{2.554775in}}{\pgfqpoint{1.673686in}{2.558048in}}{\pgfqpoint{1.679510in}{2.563872in}}%
\pgfpathcurveto{\pgfqpoint{1.685334in}{2.569696in}}{\pgfqpoint{1.688606in}{2.577596in}}{\pgfqpoint{1.688606in}{2.585832in}}%
\pgfpathcurveto{\pgfqpoint{1.688606in}{2.594068in}}{\pgfqpoint{1.685334in}{2.601968in}}{\pgfqpoint{1.679510in}{2.607792in}}%
\pgfpathcurveto{\pgfqpoint{1.673686in}{2.613616in}}{\pgfqpoint{1.665786in}{2.616888in}}{\pgfqpoint{1.657550in}{2.616888in}}%
\pgfpathcurveto{\pgfqpoint{1.649313in}{2.616888in}}{\pgfqpoint{1.641413in}{2.613616in}}{\pgfqpoint{1.635589in}{2.607792in}}%
\pgfpathcurveto{\pgfqpoint{1.629766in}{2.601968in}}{\pgfqpoint{1.626493in}{2.594068in}}{\pgfqpoint{1.626493in}{2.585832in}}%
\pgfpathcurveto{\pgfqpoint{1.626493in}{2.577596in}}{\pgfqpoint{1.629766in}{2.569696in}}{\pgfqpoint{1.635589in}{2.563872in}}%
\pgfpathcurveto{\pgfqpoint{1.641413in}{2.558048in}}{\pgfqpoint{1.649313in}{2.554775in}}{\pgfqpoint{1.657550in}{2.554775in}}%
\pgfpathclose%
\pgfusepath{stroke,fill}%
\end{pgfscope}%
\begin{pgfscope}%
\pgfpathrectangle{\pgfqpoint{0.100000in}{0.212622in}}{\pgfqpoint{3.696000in}{3.696000in}}%
\pgfusepath{clip}%
\pgfsetbuttcap%
\pgfsetroundjoin%
\definecolor{currentfill}{rgb}{0.121569,0.466667,0.705882}%
\pgfsetfillcolor{currentfill}%
\pgfsetfillopacity{0.300053}%
\pgfsetlinewidth{1.003750pt}%
\definecolor{currentstroke}{rgb}{0.121569,0.466667,0.705882}%
\pgfsetstrokecolor{currentstroke}%
\pgfsetstrokeopacity{0.300053}%
\pgfsetdash{}{0pt}%
\pgfpathmoveto{\pgfqpoint{1.670887in}{2.549979in}}%
\pgfpathcurveto{\pgfqpoint{1.679124in}{2.549979in}}{\pgfqpoint{1.687024in}{2.553251in}}{\pgfqpoint{1.692848in}{2.559075in}}%
\pgfpathcurveto{\pgfqpoint{1.698672in}{2.564899in}}{\pgfqpoint{1.701944in}{2.572799in}}{\pgfqpoint{1.701944in}{2.581036in}}%
\pgfpathcurveto{\pgfqpoint{1.701944in}{2.589272in}}{\pgfqpoint{1.698672in}{2.597172in}}{\pgfqpoint{1.692848in}{2.602996in}}%
\pgfpathcurveto{\pgfqpoint{1.687024in}{2.608820in}}{\pgfqpoint{1.679124in}{2.612092in}}{\pgfqpoint{1.670887in}{2.612092in}}%
\pgfpathcurveto{\pgfqpoint{1.662651in}{2.612092in}}{\pgfqpoint{1.654751in}{2.608820in}}{\pgfqpoint{1.648927in}{2.602996in}}%
\pgfpathcurveto{\pgfqpoint{1.643103in}{2.597172in}}{\pgfqpoint{1.639831in}{2.589272in}}{\pgfqpoint{1.639831in}{2.581036in}}%
\pgfpathcurveto{\pgfqpoint{1.639831in}{2.572799in}}{\pgfqpoint{1.643103in}{2.564899in}}{\pgfqpoint{1.648927in}{2.559075in}}%
\pgfpathcurveto{\pgfqpoint{1.654751in}{2.553251in}}{\pgfqpoint{1.662651in}{2.549979in}}{\pgfqpoint{1.670887in}{2.549979in}}%
\pgfpathclose%
\pgfusepath{stroke,fill}%
\end{pgfscope}%
\begin{pgfscope}%
\pgfpathrectangle{\pgfqpoint{0.100000in}{0.212622in}}{\pgfqpoint{3.696000in}{3.696000in}}%
\pgfusepath{clip}%
\pgfsetbuttcap%
\pgfsetroundjoin%
\definecolor{currentfill}{rgb}{0.121569,0.466667,0.705882}%
\pgfsetfillcolor{currentfill}%
\pgfsetfillopacity{0.300068}%
\pgfsetlinewidth{1.003750pt}%
\definecolor{currentstroke}{rgb}{0.121569,0.466667,0.705882}%
\pgfsetstrokecolor{currentstroke}%
\pgfsetstrokeopacity{0.300068}%
\pgfsetdash{}{0pt}%
\pgfpathmoveto{\pgfqpoint{1.655570in}{2.555250in}}%
\pgfpathcurveto{\pgfqpoint{1.663806in}{2.555250in}}{\pgfqpoint{1.671706in}{2.558522in}}{\pgfqpoint{1.677530in}{2.564346in}}%
\pgfpathcurveto{\pgfqpoint{1.683354in}{2.570170in}}{\pgfqpoint{1.686627in}{2.578070in}}{\pgfqpoint{1.686627in}{2.586306in}}%
\pgfpathcurveto{\pgfqpoint{1.686627in}{2.594542in}}{\pgfqpoint{1.683354in}{2.602442in}}{\pgfqpoint{1.677530in}{2.608266in}}%
\pgfpathcurveto{\pgfqpoint{1.671706in}{2.614090in}}{\pgfqpoint{1.663806in}{2.617363in}}{\pgfqpoint{1.655570in}{2.617363in}}%
\pgfpathcurveto{\pgfqpoint{1.647334in}{2.617363in}}{\pgfqpoint{1.639434in}{2.614090in}}{\pgfqpoint{1.633610in}{2.608266in}}%
\pgfpathcurveto{\pgfqpoint{1.627786in}{2.602442in}}{\pgfqpoint{1.624514in}{2.594542in}}{\pgfqpoint{1.624514in}{2.586306in}}%
\pgfpathcurveto{\pgfqpoint{1.624514in}{2.578070in}}{\pgfqpoint{1.627786in}{2.570170in}}{\pgfqpoint{1.633610in}{2.564346in}}%
\pgfpathcurveto{\pgfqpoint{1.639434in}{2.558522in}}{\pgfqpoint{1.647334in}{2.555250in}}{\pgfqpoint{1.655570in}{2.555250in}}%
\pgfpathclose%
\pgfusepath{stroke,fill}%
\end{pgfscope}%
\begin{pgfscope}%
\pgfpathrectangle{\pgfqpoint{0.100000in}{0.212622in}}{\pgfqpoint{3.696000in}{3.696000in}}%
\pgfusepath{clip}%
\pgfsetbuttcap%
\pgfsetroundjoin%
\definecolor{currentfill}{rgb}{0.121569,0.466667,0.705882}%
\pgfsetfillcolor{currentfill}%
\pgfsetfillopacity{0.300091}%
\pgfsetlinewidth{1.003750pt}%
\definecolor{currentstroke}{rgb}{0.121569,0.466667,0.705882}%
\pgfsetstrokecolor{currentstroke}%
\pgfsetstrokeopacity{0.300091}%
\pgfsetdash{}{0pt}%
\pgfpathmoveto{\pgfqpoint{1.654843in}{2.555457in}}%
\pgfpathcurveto{\pgfqpoint{1.663079in}{2.555457in}}{\pgfqpoint{1.670979in}{2.558729in}}{\pgfqpoint{1.676803in}{2.564553in}}%
\pgfpathcurveto{\pgfqpoint{1.682627in}{2.570377in}}{\pgfqpoint{1.685899in}{2.578277in}}{\pgfqpoint{1.685899in}{2.586514in}}%
\pgfpathcurveto{\pgfqpoint{1.685899in}{2.594750in}}{\pgfqpoint{1.682627in}{2.602650in}}{\pgfqpoint{1.676803in}{2.608474in}}%
\pgfpathcurveto{\pgfqpoint{1.670979in}{2.614298in}}{\pgfqpoint{1.663079in}{2.617570in}}{\pgfqpoint{1.654843in}{2.617570in}}%
\pgfpathcurveto{\pgfqpoint{1.646607in}{2.617570in}}{\pgfqpoint{1.638707in}{2.614298in}}{\pgfqpoint{1.632883in}{2.608474in}}%
\pgfpathcurveto{\pgfqpoint{1.627059in}{2.602650in}}{\pgfqpoint{1.623786in}{2.594750in}}{\pgfqpoint{1.623786in}{2.586514in}}%
\pgfpathcurveto{\pgfqpoint{1.623786in}{2.578277in}}{\pgfqpoint{1.627059in}{2.570377in}}{\pgfqpoint{1.632883in}{2.564553in}}%
\pgfpathcurveto{\pgfqpoint{1.638707in}{2.558729in}}{\pgfqpoint{1.646607in}{2.555457in}}{\pgfqpoint{1.654843in}{2.555457in}}%
\pgfpathclose%
\pgfusepath{stroke,fill}%
\end{pgfscope}%
\begin{pgfscope}%
\pgfpathrectangle{\pgfqpoint{0.100000in}{0.212622in}}{\pgfqpoint{3.696000in}{3.696000in}}%
\pgfusepath{clip}%
\pgfsetbuttcap%
\pgfsetroundjoin%
\definecolor{currentfill}{rgb}{0.121569,0.466667,0.705882}%
\pgfsetfillcolor{currentfill}%
\pgfsetfillopacity{0.300136}%
\pgfsetlinewidth{1.003750pt}%
\definecolor{currentstroke}{rgb}{0.121569,0.466667,0.705882}%
\pgfsetstrokecolor{currentstroke}%
\pgfsetstrokeopacity{0.300136}%
\pgfsetdash{}{0pt}%
\pgfpathmoveto{\pgfqpoint{1.674718in}{2.548662in}}%
\pgfpathcurveto{\pgfqpoint{1.682954in}{2.548662in}}{\pgfqpoint{1.690854in}{2.551934in}}{\pgfqpoint{1.696678in}{2.557758in}}%
\pgfpathcurveto{\pgfqpoint{1.702502in}{2.563582in}}{\pgfqpoint{1.705774in}{2.571482in}}{\pgfqpoint{1.705774in}{2.579718in}}%
\pgfpathcurveto{\pgfqpoint{1.705774in}{2.587954in}}{\pgfqpoint{1.702502in}{2.595854in}}{\pgfqpoint{1.696678in}{2.601678in}}%
\pgfpathcurveto{\pgfqpoint{1.690854in}{2.607502in}}{\pgfqpoint{1.682954in}{2.610775in}}{\pgfqpoint{1.674718in}{2.610775in}}%
\pgfpathcurveto{\pgfqpoint{1.666482in}{2.610775in}}{\pgfqpoint{1.658582in}{2.607502in}}{\pgfqpoint{1.652758in}{2.601678in}}%
\pgfpathcurveto{\pgfqpoint{1.646934in}{2.595854in}}{\pgfqpoint{1.643661in}{2.587954in}}{\pgfqpoint{1.643661in}{2.579718in}}%
\pgfpathcurveto{\pgfqpoint{1.643661in}{2.571482in}}{\pgfqpoint{1.646934in}{2.563582in}}{\pgfqpoint{1.652758in}{2.557758in}}%
\pgfpathcurveto{\pgfqpoint{1.658582in}{2.551934in}}{\pgfqpoint{1.666482in}{2.548662in}}{\pgfqpoint{1.674718in}{2.548662in}}%
\pgfpathclose%
\pgfusepath{stroke,fill}%
\end{pgfscope}%
\begin{pgfscope}%
\pgfpathrectangle{\pgfqpoint{0.100000in}{0.212622in}}{\pgfqpoint{3.696000in}{3.696000in}}%
\pgfusepath{clip}%
\pgfsetbuttcap%
\pgfsetroundjoin%
\definecolor{currentfill}{rgb}{0.121569,0.466667,0.705882}%
\pgfsetfillcolor{currentfill}%
\pgfsetfillopacity{0.300155}%
\pgfsetlinewidth{1.003750pt}%
\definecolor{currentstroke}{rgb}{0.121569,0.466667,0.705882}%
\pgfsetstrokecolor{currentstroke}%
\pgfsetstrokeopacity{0.300155}%
\pgfsetdash{}{0pt}%
\pgfpathmoveto{\pgfqpoint{1.653519in}{2.555838in}}%
\pgfpathcurveto{\pgfqpoint{1.661756in}{2.555838in}}{\pgfqpoint{1.669656in}{2.559110in}}{\pgfqpoint{1.675480in}{2.564934in}}%
\pgfpathcurveto{\pgfqpoint{1.681303in}{2.570758in}}{\pgfqpoint{1.684576in}{2.578658in}}{\pgfqpoint{1.684576in}{2.586894in}}%
\pgfpathcurveto{\pgfqpoint{1.684576in}{2.595131in}}{\pgfqpoint{1.681303in}{2.603031in}}{\pgfqpoint{1.675480in}{2.608855in}}%
\pgfpathcurveto{\pgfqpoint{1.669656in}{2.614679in}}{\pgfqpoint{1.661756in}{2.617951in}}{\pgfqpoint{1.653519in}{2.617951in}}%
\pgfpathcurveto{\pgfqpoint{1.645283in}{2.617951in}}{\pgfqpoint{1.637383in}{2.614679in}}{\pgfqpoint{1.631559in}{2.608855in}}%
\pgfpathcurveto{\pgfqpoint{1.625735in}{2.603031in}}{\pgfqpoint{1.622463in}{2.595131in}}{\pgfqpoint{1.622463in}{2.586894in}}%
\pgfpathcurveto{\pgfqpoint{1.622463in}{2.578658in}}{\pgfqpoint{1.625735in}{2.570758in}}{\pgfqpoint{1.631559in}{2.564934in}}%
\pgfpathcurveto{\pgfqpoint{1.637383in}{2.559110in}}{\pgfqpoint{1.645283in}{2.555838in}}{\pgfqpoint{1.653519in}{2.555838in}}%
\pgfpathclose%
\pgfusepath{stroke,fill}%
\end{pgfscope}%
\begin{pgfscope}%
\pgfpathrectangle{\pgfqpoint{0.100000in}{0.212622in}}{\pgfqpoint{3.696000in}{3.696000in}}%
\pgfusepath{clip}%
\pgfsetbuttcap%
\pgfsetroundjoin%
\definecolor{currentfill}{rgb}{0.121569,0.466667,0.705882}%
\pgfsetfillcolor{currentfill}%
\pgfsetfillopacity{0.300309}%
\pgfsetlinewidth{1.003750pt}%
\definecolor{currentstroke}{rgb}{0.121569,0.466667,0.705882}%
\pgfsetstrokecolor{currentstroke}%
\pgfsetstrokeopacity{0.300309}%
\pgfsetdash{}{0pt}%
\pgfpathmoveto{\pgfqpoint{1.651119in}{2.556487in}}%
\pgfpathcurveto{\pgfqpoint{1.659355in}{2.556487in}}{\pgfqpoint{1.667255in}{2.559759in}}{\pgfqpoint{1.673079in}{2.565583in}}%
\pgfpathcurveto{\pgfqpoint{1.678903in}{2.571407in}}{\pgfqpoint{1.682176in}{2.579307in}}{\pgfqpoint{1.682176in}{2.587543in}}%
\pgfpathcurveto{\pgfqpoint{1.682176in}{2.595780in}}{\pgfqpoint{1.678903in}{2.603680in}}{\pgfqpoint{1.673079in}{2.609504in}}%
\pgfpathcurveto{\pgfqpoint{1.667255in}{2.615328in}}{\pgfqpoint{1.659355in}{2.618600in}}{\pgfqpoint{1.651119in}{2.618600in}}%
\pgfpathcurveto{\pgfqpoint{1.642883in}{2.618600in}}{\pgfqpoint{1.634983in}{2.615328in}}{\pgfqpoint{1.629159in}{2.609504in}}%
\pgfpathcurveto{\pgfqpoint{1.623335in}{2.603680in}}{\pgfqpoint{1.620063in}{2.595780in}}{\pgfqpoint{1.620063in}{2.587543in}}%
\pgfpathcurveto{\pgfqpoint{1.620063in}{2.579307in}}{\pgfqpoint{1.623335in}{2.571407in}}{\pgfqpoint{1.629159in}{2.565583in}}%
\pgfpathcurveto{\pgfqpoint{1.634983in}{2.559759in}}{\pgfqpoint{1.642883in}{2.556487in}}{\pgfqpoint{1.651119in}{2.556487in}}%
\pgfpathclose%
\pgfusepath{stroke,fill}%
\end{pgfscope}%
\begin{pgfscope}%
\pgfpathrectangle{\pgfqpoint{0.100000in}{0.212622in}}{\pgfqpoint{3.696000in}{3.696000in}}%
\pgfusepath{clip}%
\pgfsetbuttcap%
\pgfsetroundjoin%
\definecolor{currentfill}{rgb}{0.121569,0.466667,0.705882}%
\pgfsetfillcolor{currentfill}%
\pgfsetfillopacity{0.300359}%
\pgfsetlinewidth{1.003750pt}%
\definecolor{currentstroke}{rgb}{0.121569,0.466667,0.705882}%
\pgfsetstrokecolor{currentstroke}%
\pgfsetstrokeopacity{0.300359}%
\pgfsetdash{}{0pt}%
\pgfpathmoveto{\pgfqpoint{1.681040in}{2.546299in}}%
\pgfpathcurveto{\pgfqpoint{1.689276in}{2.546299in}}{\pgfqpoint{1.697177in}{2.549571in}}{\pgfqpoint{1.703000in}{2.555395in}}%
\pgfpathcurveto{\pgfqpoint{1.708824in}{2.561219in}}{\pgfqpoint{1.712097in}{2.569119in}}{\pgfqpoint{1.712097in}{2.577356in}}%
\pgfpathcurveto{\pgfqpoint{1.712097in}{2.585592in}}{\pgfqpoint{1.708824in}{2.593492in}}{\pgfqpoint{1.703000in}{2.599316in}}%
\pgfpathcurveto{\pgfqpoint{1.697177in}{2.605140in}}{\pgfqpoint{1.689276in}{2.608412in}}{\pgfqpoint{1.681040in}{2.608412in}}%
\pgfpathcurveto{\pgfqpoint{1.672804in}{2.608412in}}{\pgfqpoint{1.664904in}{2.605140in}}{\pgfqpoint{1.659080in}{2.599316in}}%
\pgfpathcurveto{\pgfqpoint{1.653256in}{2.593492in}}{\pgfqpoint{1.649984in}{2.585592in}}{\pgfqpoint{1.649984in}{2.577356in}}%
\pgfpathcurveto{\pgfqpoint{1.649984in}{2.569119in}}{\pgfqpoint{1.653256in}{2.561219in}}{\pgfqpoint{1.659080in}{2.555395in}}%
\pgfpathcurveto{\pgfqpoint{1.664904in}{2.549571in}}{\pgfqpoint{1.672804in}{2.546299in}}{\pgfqpoint{1.681040in}{2.546299in}}%
\pgfpathclose%
\pgfusepath{stroke,fill}%
\end{pgfscope}%
\begin{pgfscope}%
\pgfpathrectangle{\pgfqpoint{0.100000in}{0.212622in}}{\pgfqpoint{3.696000in}{3.696000in}}%
\pgfusepath{clip}%
\pgfsetbuttcap%
\pgfsetroundjoin%
\definecolor{currentfill}{rgb}{0.121569,0.466667,0.705882}%
\pgfsetfillcolor{currentfill}%
\pgfsetfillopacity{0.300448}%
\pgfsetlinewidth{1.003750pt}%
\definecolor{currentstroke}{rgb}{0.121569,0.466667,0.705882}%
\pgfsetstrokecolor{currentstroke}%
\pgfsetstrokeopacity{0.300448}%
\pgfsetdash{}{0pt}%
\pgfpathmoveto{\pgfqpoint{1.649283in}{2.556987in}}%
\pgfpathcurveto{\pgfqpoint{1.657519in}{2.556987in}}{\pgfqpoint{1.665419in}{2.560259in}}{\pgfqpoint{1.671243in}{2.566083in}}%
\pgfpathcurveto{\pgfqpoint{1.677067in}{2.571907in}}{\pgfqpoint{1.680339in}{2.579807in}}{\pgfqpoint{1.680339in}{2.588043in}}%
\pgfpathcurveto{\pgfqpoint{1.680339in}{2.596279in}}{\pgfqpoint{1.677067in}{2.604180in}}{\pgfqpoint{1.671243in}{2.610003in}}%
\pgfpathcurveto{\pgfqpoint{1.665419in}{2.615827in}}{\pgfqpoint{1.657519in}{2.619100in}}{\pgfqpoint{1.649283in}{2.619100in}}%
\pgfpathcurveto{\pgfqpoint{1.641046in}{2.619100in}}{\pgfqpoint{1.633146in}{2.615827in}}{\pgfqpoint{1.627322in}{2.610003in}}%
\pgfpathcurveto{\pgfqpoint{1.621498in}{2.604180in}}{\pgfqpoint{1.618226in}{2.596279in}}{\pgfqpoint{1.618226in}{2.588043in}}%
\pgfpathcurveto{\pgfqpoint{1.618226in}{2.579807in}}{\pgfqpoint{1.621498in}{2.571907in}}{\pgfqpoint{1.627322in}{2.566083in}}%
\pgfpathcurveto{\pgfqpoint{1.633146in}{2.560259in}}{\pgfqpoint{1.641046in}{2.556987in}}{\pgfqpoint{1.649283in}{2.556987in}}%
\pgfpathclose%
\pgfusepath{stroke,fill}%
\end{pgfscope}%
\begin{pgfscope}%
\pgfpathrectangle{\pgfqpoint{0.100000in}{0.212622in}}{\pgfqpoint{3.696000in}{3.696000in}}%
\pgfusepath{clip}%
\pgfsetbuttcap%
\pgfsetroundjoin%
\definecolor{currentfill}{rgb}{0.121569,0.466667,0.705882}%
\pgfsetfillcolor{currentfill}%
\pgfsetfillopacity{0.300579}%
\pgfsetlinewidth{1.003750pt}%
\definecolor{currentstroke}{rgb}{0.121569,0.466667,0.705882}%
\pgfsetstrokecolor{currentstroke}%
\pgfsetstrokeopacity{0.300579}%
\pgfsetdash{}{0pt}%
\pgfpathmoveto{\pgfqpoint{1.684480in}{2.545497in}}%
\pgfpathcurveto{\pgfqpoint{1.692716in}{2.545497in}}{\pgfqpoint{1.700616in}{2.548770in}}{\pgfqpoint{1.706440in}{2.554593in}}%
\pgfpathcurveto{\pgfqpoint{1.712264in}{2.560417in}}{\pgfqpoint{1.715537in}{2.568317in}}{\pgfqpoint{1.715537in}{2.576554in}}%
\pgfpathcurveto{\pgfqpoint{1.715537in}{2.584790in}}{\pgfqpoint{1.712264in}{2.592690in}}{\pgfqpoint{1.706440in}{2.598514in}}%
\pgfpathcurveto{\pgfqpoint{1.700616in}{2.604338in}}{\pgfqpoint{1.692716in}{2.607610in}}{\pgfqpoint{1.684480in}{2.607610in}}%
\pgfpathcurveto{\pgfqpoint{1.676244in}{2.607610in}}{\pgfqpoint{1.668344in}{2.604338in}}{\pgfqpoint{1.662520in}{2.598514in}}%
\pgfpathcurveto{\pgfqpoint{1.656696in}{2.592690in}}{\pgfqpoint{1.653424in}{2.584790in}}{\pgfqpoint{1.653424in}{2.576554in}}%
\pgfpathcurveto{\pgfqpoint{1.653424in}{2.568317in}}{\pgfqpoint{1.656696in}{2.560417in}}{\pgfqpoint{1.662520in}{2.554593in}}%
\pgfpathcurveto{\pgfqpoint{1.668344in}{2.548770in}}{\pgfqpoint{1.676244in}{2.545497in}}{\pgfqpoint{1.684480in}{2.545497in}}%
\pgfpathclose%
\pgfusepath{stroke,fill}%
\end{pgfscope}%
\begin{pgfscope}%
\pgfpathrectangle{\pgfqpoint{0.100000in}{0.212622in}}{\pgfqpoint{3.696000in}{3.696000in}}%
\pgfusepath{clip}%
\pgfsetbuttcap%
\pgfsetroundjoin%
\definecolor{currentfill}{rgb}{0.121569,0.466667,0.705882}%
\pgfsetfillcolor{currentfill}%
\pgfsetfillopacity{0.300749}%
\pgfsetlinewidth{1.003750pt}%
\definecolor{currentstroke}{rgb}{0.121569,0.466667,0.705882}%
\pgfsetstrokecolor{currentstroke}%
\pgfsetstrokeopacity{0.300749}%
\pgfsetdash{}{0pt}%
\pgfpathmoveto{\pgfqpoint{1.645976in}{2.557803in}}%
\pgfpathcurveto{\pgfqpoint{1.654212in}{2.557803in}}{\pgfqpoint{1.662112in}{2.561075in}}{\pgfqpoint{1.667936in}{2.566899in}}%
\pgfpathcurveto{\pgfqpoint{1.673760in}{2.572723in}}{\pgfqpoint{1.677032in}{2.580623in}}{\pgfqpoint{1.677032in}{2.588859in}}%
\pgfpathcurveto{\pgfqpoint{1.677032in}{2.597096in}}{\pgfqpoint{1.673760in}{2.604996in}}{\pgfqpoint{1.667936in}{2.610820in}}%
\pgfpathcurveto{\pgfqpoint{1.662112in}{2.616643in}}{\pgfqpoint{1.654212in}{2.619916in}}{\pgfqpoint{1.645976in}{2.619916in}}%
\pgfpathcurveto{\pgfqpoint{1.637739in}{2.619916in}}{\pgfqpoint{1.629839in}{2.616643in}}{\pgfqpoint{1.624015in}{2.610820in}}%
\pgfpathcurveto{\pgfqpoint{1.618191in}{2.604996in}}{\pgfqpoint{1.614919in}{2.597096in}}{\pgfqpoint{1.614919in}{2.588859in}}%
\pgfpathcurveto{\pgfqpoint{1.614919in}{2.580623in}}{\pgfqpoint{1.618191in}{2.572723in}}{\pgfqpoint{1.624015in}{2.566899in}}%
\pgfpathcurveto{\pgfqpoint{1.629839in}{2.561075in}}{\pgfqpoint{1.637739in}{2.557803in}}{\pgfqpoint{1.645976in}{2.557803in}}%
\pgfpathclose%
\pgfusepath{stroke,fill}%
\end{pgfscope}%
\begin{pgfscope}%
\pgfpathrectangle{\pgfqpoint{0.100000in}{0.212622in}}{\pgfqpoint{3.696000in}{3.696000in}}%
\pgfusepath{clip}%
\pgfsetbuttcap%
\pgfsetroundjoin%
\definecolor{currentfill}{rgb}{0.121569,0.466667,0.705882}%
\pgfsetfillcolor{currentfill}%
\pgfsetfillopacity{0.300901}%
\pgfsetlinewidth{1.003750pt}%
\definecolor{currentstroke}{rgb}{0.121569,0.466667,0.705882}%
\pgfsetstrokecolor{currentstroke}%
\pgfsetstrokeopacity{0.300901}%
\pgfsetdash{}{0pt}%
\pgfpathmoveto{\pgfqpoint{1.688936in}{2.544601in}}%
\pgfpathcurveto{\pgfqpoint{1.697172in}{2.544601in}}{\pgfqpoint{1.705072in}{2.547873in}}{\pgfqpoint{1.710896in}{2.553697in}}%
\pgfpathcurveto{\pgfqpoint{1.716720in}{2.559521in}}{\pgfqpoint{1.719993in}{2.567421in}}{\pgfqpoint{1.719993in}{2.575657in}}%
\pgfpathcurveto{\pgfqpoint{1.719993in}{2.583893in}}{\pgfqpoint{1.716720in}{2.591794in}}{\pgfqpoint{1.710896in}{2.597617in}}%
\pgfpathcurveto{\pgfqpoint{1.705072in}{2.603441in}}{\pgfqpoint{1.697172in}{2.606714in}}{\pgfqpoint{1.688936in}{2.606714in}}%
\pgfpathcurveto{\pgfqpoint{1.680700in}{2.606714in}}{\pgfqpoint{1.672800in}{2.603441in}}{\pgfqpoint{1.666976in}{2.597617in}}%
\pgfpathcurveto{\pgfqpoint{1.661152in}{2.591794in}}{\pgfqpoint{1.657880in}{2.583893in}}{\pgfqpoint{1.657880in}{2.575657in}}%
\pgfpathcurveto{\pgfqpoint{1.657880in}{2.567421in}}{\pgfqpoint{1.661152in}{2.559521in}}{\pgfqpoint{1.666976in}{2.553697in}}%
\pgfpathcurveto{\pgfqpoint{1.672800in}{2.547873in}}{\pgfqpoint{1.680700in}{2.544601in}}{\pgfqpoint{1.688936in}{2.544601in}}%
\pgfpathclose%
\pgfusepath{stroke,fill}%
\end{pgfscope}%
\begin{pgfscope}%
\pgfpathrectangle{\pgfqpoint{0.100000in}{0.212622in}}{\pgfqpoint{3.696000in}{3.696000in}}%
\pgfusepath{clip}%
\pgfsetbuttcap%
\pgfsetroundjoin%
\definecolor{currentfill}{rgb}{0.121569,0.466667,0.705882}%
\pgfsetfillcolor{currentfill}%
\pgfsetfillopacity{0.301362}%
\pgfsetlinewidth{1.003750pt}%
\definecolor{currentstroke}{rgb}{0.121569,0.466667,0.705882}%
\pgfsetstrokecolor{currentstroke}%
\pgfsetstrokeopacity{0.301362}%
\pgfsetdash{}{0pt}%
\pgfpathmoveto{\pgfqpoint{1.639993in}{2.559409in}}%
\pgfpathcurveto{\pgfqpoint{1.648229in}{2.559409in}}{\pgfqpoint{1.656129in}{2.562682in}}{\pgfqpoint{1.661953in}{2.568506in}}%
\pgfpathcurveto{\pgfqpoint{1.667777in}{2.574330in}}{\pgfqpoint{1.671049in}{2.582230in}}{\pgfqpoint{1.671049in}{2.590466in}}%
\pgfpathcurveto{\pgfqpoint{1.671049in}{2.598702in}}{\pgfqpoint{1.667777in}{2.606602in}}{\pgfqpoint{1.661953in}{2.612426in}}%
\pgfpathcurveto{\pgfqpoint{1.656129in}{2.618250in}}{\pgfqpoint{1.648229in}{2.621522in}}{\pgfqpoint{1.639993in}{2.621522in}}%
\pgfpathcurveto{\pgfqpoint{1.631756in}{2.621522in}}{\pgfqpoint{1.623856in}{2.618250in}}{\pgfqpoint{1.618032in}{2.612426in}}%
\pgfpathcurveto{\pgfqpoint{1.612209in}{2.606602in}}{\pgfqpoint{1.608936in}{2.598702in}}{\pgfqpoint{1.608936in}{2.590466in}}%
\pgfpathcurveto{\pgfqpoint{1.608936in}{2.582230in}}{\pgfqpoint{1.612209in}{2.574330in}}{\pgfqpoint{1.618032in}{2.568506in}}%
\pgfpathcurveto{\pgfqpoint{1.623856in}{2.562682in}}{\pgfqpoint{1.631756in}{2.559409in}}{\pgfqpoint{1.639993in}{2.559409in}}%
\pgfpathclose%
\pgfusepath{stroke,fill}%
\end{pgfscope}%
\begin{pgfscope}%
\pgfpathrectangle{\pgfqpoint{0.100000in}{0.212622in}}{\pgfqpoint{3.696000in}{3.696000in}}%
\pgfusepath{clip}%
\pgfsetbuttcap%
\pgfsetroundjoin%
\definecolor{currentfill}{rgb}{0.121569,0.466667,0.705882}%
\pgfsetfillcolor{currentfill}%
\pgfsetfillopacity{0.301380}%
\pgfsetlinewidth{1.003750pt}%
\definecolor{currentstroke}{rgb}{0.121569,0.466667,0.705882}%
\pgfsetstrokecolor{currentstroke}%
\pgfsetstrokeopacity{0.301380}%
\pgfsetdash{}{0pt}%
\pgfpathmoveto{\pgfqpoint{1.694850in}{2.543414in}}%
\pgfpathcurveto{\pgfqpoint{1.703087in}{2.543414in}}{\pgfqpoint{1.710987in}{2.546686in}}{\pgfqpoint{1.716811in}{2.552510in}}%
\pgfpathcurveto{\pgfqpoint{1.722634in}{2.558334in}}{\pgfqpoint{1.725907in}{2.566234in}}{\pgfqpoint{1.725907in}{2.574470in}}%
\pgfpathcurveto{\pgfqpoint{1.725907in}{2.582706in}}{\pgfqpoint{1.722634in}{2.590606in}}{\pgfqpoint{1.716811in}{2.596430in}}%
\pgfpathcurveto{\pgfqpoint{1.710987in}{2.602254in}}{\pgfqpoint{1.703087in}{2.605527in}}{\pgfqpoint{1.694850in}{2.605527in}}%
\pgfpathcurveto{\pgfqpoint{1.686614in}{2.605527in}}{\pgfqpoint{1.678714in}{2.602254in}}{\pgfqpoint{1.672890in}{2.596430in}}%
\pgfpathcurveto{\pgfqpoint{1.667066in}{2.590606in}}{\pgfqpoint{1.663794in}{2.582706in}}{\pgfqpoint{1.663794in}{2.574470in}}%
\pgfpathcurveto{\pgfqpoint{1.663794in}{2.566234in}}{\pgfqpoint{1.667066in}{2.558334in}}{\pgfqpoint{1.672890in}{2.552510in}}%
\pgfpathcurveto{\pgfqpoint{1.678714in}{2.546686in}}{\pgfqpoint{1.686614in}{2.543414in}}{\pgfqpoint{1.694850in}{2.543414in}}%
\pgfpathclose%
\pgfusepath{stroke,fill}%
\end{pgfscope}%
\begin{pgfscope}%
\pgfpathrectangle{\pgfqpoint{0.100000in}{0.212622in}}{\pgfqpoint{3.696000in}{3.696000in}}%
\pgfusepath{clip}%
\pgfsetbuttcap%
\pgfsetroundjoin%
\definecolor{currentfill}{rgb}{0.121569,0.466667,0.705882}%
\pgfsetfillcolor{currentfill}%
\pgfsetfillopacity{0.301980}%
\pgfsetlinewidth{1.003750pt}%
\definecolor{currentstroke}{rgb}{0.121569,0.466667,0.705882}%
\pgfsetstrokecolor{currentstroke}%
\pgfsetstrokeopacity{0.301980}%
\pgfsetdash{}{0pt}%
\pgfpathmoveto{\pgfqpoint{1.634803in}{2.560748in}}%
\pgfpathcurveto{\pgfqpoint{1.643039in}{2.560748in}}{\pgfqpoint{1.650939in}{2.564020in}}{\pgfqpoint{1.656763in}{2.569844in}}%
\pgfpathcurveto{\pgfqpoint{1.662587in}{2.575668in}}{\pgfqpoint{1.665859in}{2.583568in}}{\pgfqpoint{1.665859in}{2.591804in}}%
\pgfpathcurveto{\pgfqpoint{1.665859in}{2.600041in}}{\pgfqpoint{1.662587in}{2.607941in}}{\pgfqpoint{1.656763in}{2.613765in}}%
\pgfpathcurveto{\pgfqpoint{1.650939in}{2.619589in}}{\pgfqpoint{1.643039in}{2.622861in}}{\pgfqpoint{1.634803in}{2.622861in}}%
\pgfpathcurveto{\pgfqpoint{1.626566in}{2.622861in}}{\pgfqpoint{1.618666in}{2.619589in}}{\pgfqpoint{1.612842in}{2.613765in}}%
\pgfpathcurveto{\pgfqpoint{1.607018in}{2.607941in}}{\pgfqpoint{1.603746in}{2.600041in}}{\pgfqpoint{1.603746in}{2.591804in}}%
\pgfpathcurveto{\pgfqpoint{1.603746in}{2.583568in}}{\pgfqpoint{1.607018in}{2.575668in}}{\pgfqpoint{1.612842in}{2.569844in}}%
\pgfpathcurveto{\pgfqpoint{1.618666in}{2.564020in}}{\pgfqpoint{1.626566in}{2.560748in}}{\pgfqpoint{1.634803in}{2.560748in}}%
\pgfpathclose%
\pgfusepath{stroke,fill}%
\end{pgfscope}%
\begin{pgfscope}%
\pgfpathrectangle{\pgfqpoint{0.100000in}{0.212622in}}{\pgfqpoint{3.696000in}{3.696000in}}%
\pgfusepath{clip}%
\pgfsetbuttcap%
\pgfsetroundjoin%
\definecolor{currentfill}{rgb}{0.121569,0.466667,0.705882}%
\pgfsetfillcolor{currentfill}%
\pgfsetfillopacity{0.302190}%
\pgfsetlinewidth{1.003750pt}%
\definecolor{currentstroke}{rgb}{0.121569,0.466667,0.705882}%
\pgfsetstrokecolor{currentstroke}%
\pgfsetstrokeopacity{0.302190}%
\pgfsetdash{}{0pt}%
\pgfpathmoveto{\pgfqpoint{1.704325in}{2.541147in}}%
\pgfpathcurveto{\pgfqpoint{1.712561in}{2.541147in}}{\pgfqpoint{1.720461in}{2.544420in}}{\pgfqpoint{1.726285in}{2.550244in}}%
\pgfpathcurveto{\pgfqpoint{1.732109in}{2.556068in}}{\pgfqpoint{1.735381in}{2.563968in}}{\pgfqpoint{1.735381in}{2.572204in}}%
\pgfpathcurveto{\pgfqpoint{1.735381in}{2.580440in}}{\pgfqpoint{1.732109in}{2.588340in}}{\pgfqpoint{1.726285in}{2.594164in}}%
\pgfpathcurveto{\pgfqpoint{1.720461in}{2.599988in}}{\pgfqpoint{1.712561in}{2.603260in}}{\pgfqpoint{1.704325in}{2.603260in}}%
\pgfpathcurveto{\pgfqpoint{1.696088in}{2.603260in}}{\pgfqpoint{1.688188in}{2.599988in}}{\pgfqpoint{1.682364in}{2.594164in}}%
\pgfpathcurveto{\pgfqpoint{1.676540in}{2.588340in}}{\pgfqpoint{1.673268in}{2.580440in}}{\pgfqpoint{1.673268in}{2.572204in}}%
\pgfpathcurveto{\pgfqpoint{1.673268in}{2.563968in}}{\pgfqpoint{1.676540in}{2.556068in}}{\pgfqpoint{1.682364in}{2.550244in}}%
\pgfpathcurveto{\pgfqpoint{1.688188in}{2.544420in}}{\pgfqpoint{1.696088in}{2.541147in}}{\pgfqpoint{1.704325in}{2.541147in}}%
\pgfpathclose%
\pgfusepath{stroke,fill}%
\end{pgfscope}%
\begin{pgfscope}%
\pgfpathrectangle{\pgfqpoint{0.100000in}{0.212622in}}{\pgfqpoint{3.696000in}{3.696000in}}%
\pgfusepath{clip}%
\pgfsetbuttcap%
\pgfsetroundjoin%
\definecolor{currentfill}{rgb}{0.121569,0.466667,0.705882}%
\pgfsetfillcolor{currentfill}%
\pgfsetfillopacity{0.302538}%
\pgfsetlinewidth{1.003750pt}%
\definecolor{currentstroke}{rgb}{0.121569,0.466667,0.705882}%
\pgfsetstrokecolor{currentstroke}%
\pgfsetstrokeopacity{0.302538}%
\pgfsetdash{}{0pt}%
\pgfpathmoveto{\pgfqpoint{1.630851in}{2.562037in}}%
\pgfpathcurveto{\pgfqpoint{1.639088in}{2.562037in}}{\pgfqpoint{1.646988in}{2.565309in}}{\pgfqpoint{1.652812in}{2.571133in}}%
\pgfpathcurveto{\pgfqpoint{1.658635in}{2.576957in}}{\pgfqpoint{1.661908in}{2.584857in}}{\pgfqpoint{1.661908in}{2.593093in}}%
\pgfpathcurveto{\pgfqpoint{1.661908in}{2.601329in}}{\pgfqpoint{1.658635in}{2.609230in}}{\pgfqpoint{1.652812in}{2.615053in}}%
\pgfpathcurveto{\pgfqpoint{1.646988in}{2.620877in}}{\pgfqpoint{1.639088in}{2.624150in}}{\pgfqpoint{1.630851in}{2.624150in}}%
\pgfpathcurveto{\pgfqpoint{1.622615in}{2.624150in}}{\pgfqpoint{1.614715in}{2.620877in}}{\pgfqpoint{1.608891in}{2.615053in}}%
\pgfpathcurveto{\pgfqpoint{1.603067in}{2.609230in}}{\pgfqpoint{1.599795in}{2.601329in}}{\pgfqpoint{1.599795in}{2.593093in}}%
\pgfpathcurveto{\pgfqpoint{1.599795in}{2.584857in}}{\pgfqpoint{1.603067in}{2.576957in}}{\pgfqpoint{1.608891in}{2.571133in}}%
\pgfpathcurveto{\pgfqpoint{1.614715in}{2.565309in}}{\pgfqpoint{1.622615in}{2.562037in}}{\pgfqpoint{1.630851in}{2.562037in}}%
\pgfpathclose%
\pgfusepath{stroke,fill}%
\end{pgfscope}%
\begin{pgfscope}%
\pgfpathrectangle{\pgfqpoint{0.100000in}{0.212622in}}{\pgfqpoint{3.696000in}{3.696000in}}%
\pgfusepath{clip}%
\pgfsetbuttcap%
\pgfsetroundjoin%
\definecolor{currentfill}{rgb}{0.121569,0.466667,0.705882}%
\pgfsetfillcolor{currentfill}%
\pgfsetfillopacity{0.302985}%
\pgfsetlinewidth{1.003750pt}%
\definecolor{currentstroke}{rgb}{0.121569,0.466667,0.705882}%
\pgfsetstrokecolor{currentstroke}%
\pgfsetstrokeopacity{0.302985}%
\pgfsetdash{}{0pt}%
\pgfpathmoveto{\pgfqpoint{1.628022in}{2.562934in}}%
\pgfpathcurveto{\pgfqpoint{1.636258in}{2.562934in}}{\pgfqpoint{1.644158in}{2.566206in}}{\pgfqpoint{1.649982in}{2.572030in}}%
\pgfpathcurveto{\pgfqpoint{1.655806in}{2.577854in}}{\pgfqpoint{1.659078in}{2.585754in}}{\pgfqpoint{1.659078in}{2.593990in}}%
\pgfpathcurveto{\pgfqpoint{1.659078in}{2.602226in}}{\pgfqpoint{1.655806in}{2.610127in}}{\pgfqpoint{1.649982in}{2.615950in}}%
\pgfpathcurveto{\pgfqpoint{1.644158in}{2.621774in}}{\pgfqpoint{1.636258in}{2.625047in}}{\pgfqpoint{1.628022in}{2.625047in}}%
\pgfpathcurveto{\pgfqpoint{1.619786in}{2.625047in}}{\pgfqpoint{1.611885in}{2.621774in}}{\pgfqpoint{1.606062in}{2.615950in}}%
\pgfpathcurveto{\pgfqpoint{1.600238in}{2.610127in}}{\pgfqpoint{1.596965in}{2.602226in}}{\pgfqpoint{1.596965in}{2.593990in}}%
\pgfpathcurveto{\pgfqpoint{1.596965in}{2.585754in}}{\pgfqpoint{1.600238in}{2.577854in}}{\pgfqpoint{1.606062in}{2.572030in}}%
\pgfpathcurveto{\pgfqpoint{1.611885in}{2.566206in}}{\pgfqpoint{1.619786in}{2.562934in}}{\pgfqpoint{1.628022in}{2.562934in}}%
\pgfpathclose%
\pgfusepath{stroke,fill}%
\end{pgfscope}%
\begin{pgfscope}%
\pgfpathrectangle{\pgfqpoint{0.100000in}{0.212622in}}{\pgfqpoint{3.696000in}{3.696000in}}%
\pgfusepath{clip}%
\pgfsetbuttcap%
\pgfsetroundjoin%
\definecolor{currentfill}{rgb}{0.121569,0.466667,0.705882}%
\pgfsetfillcolor{currentfill}%
\pgfsetfillopacity{0.303084}%
\pgfsetlinewidth{1.003750pt}%
\definecolor{currentstroke}{rgb}{0.121569,0.466667,0.705882}%
\pgfsetstrokecolor{currentstroke}%
\pgfsetstrokeopacity{0.303084}%
\pgfsetdash{}{0pt}%
\pgfpathmoveto{\pgfqpoint{1.714401in}{2.538044in}}%
\pgfpathcurveto{\pgfqpoint{1.722638in}{2.538044in}}{\pgfqpoint{1.730538in}{2.541316in}}{\pgfqpoint{1.736362in}{2.547140in}}%
\pgfpathcurveto{\pgfqpoint{1.742186in}{2.552964in}}{\pgfqpoint{1.745458in}{2.560864in}}{\pgfqpoint{1.745458in}{2.569100in}}%
\pgfpathcurveto{\pgfqpoint{1.745458in}{2.577337in}}{\pgfqpoint{1.742186in}{2.585237in}}{\pgfqpoint{1.736362in}{2.591061in}}%
\pgfpathcurveto{\pgfqpoint{1.730538in}{2.596885in}}{\pgfqpoint{1.722638in}{2.600157in}}{\pgfqpoint{1.714401in}{2.600157in}}%
\pgfpathcurveto{\pgfqpoint{1.706165in}{2.600157in}}{\pgfqpoint{1.698265in}{2.596885in}}{\pgfqpoint{1.692441in}{2.591061in}}%
\pgfpathcurveto{\pgfqpoint{1.686617in}{2.585237in}}{\pgfqpoint{1.683345in}{2.577337in}}{\pgfqpoint{1.683345in}{2.569100in}}%
\pgfpathcurveto{\pgfqpoint{1.683345in}{2.560864in}}{\pgfqpoint{1.686617in}{2.552964in}}{\pgfqpoint{1.692441in}{2.547140in}}%
\pgfpathcurveto{\pgfqpoint{1.698265in}{2.541316in}}{\pgfqpoint{1.706165in}{2.538044in}}{\pgfqpoint{1.714401in}{2.538044in}}%
\pgfpathclose%
\pgfusepath{stroke,fill}%
\end{pgfscope}%
\begin{pgfscope}%
\pgfpathrectangle{\pgfqpoint{0.100000in}{0.212622in}}{\pgfqpoint{3.696000in}{3.696000in}}%
\pgfusepath{clip}%
\pgfsetbuttcap%
\pgfsetroundjoin%
\definecolor{currentfill}{rgb}{0.121569,0.466667,0.705882}%
\pgfsetfillcolor{currentfill}%
\pgfsetfillopacity{0.303097}%
\pgfsetlinewidth{1.003750pt}%
\definecolor{currentstroke}{rgb}{0.121569,0.466667,0.705882}%
\pgfsetstrokecolor{currentstroke}%
\pgfsetstrokeopacity{0.303097}%
\pgfsetdash{}{0pt}%
\pgfpathmoveto{\pgfqpoint{1.627344in}{2.563082in}}%
\pgfpathcurveto{\pgfqpoint{1.635581in}{2.563082in}}{\pgfqpoint{1.643481in}{2.566354in}}{\pgfqpoint{1.649305in}{2.572178in}}%
\pgfpathcurveto{\pgfqpoint{1.655129in}{2.578002in}}{\pgfqpoint{1.658401in}{2.585902in}}{\pgfqpoint{1.658401in}{2.594139in}}%
\pgfpathcurveto{\pgfqpoint{1.658401in}{2.602375in}}{\pgfqpoint{1.655129in}{2.610275in}}{\pgfqpoint{1.649305in}{2.616099in}}%
\pgfpathcurveto{\pgfqpoint{1.643481in}{2.621923in}}{\pgfqpoint{1.635581in}{2.625195in}}{\pgfqpoint{1.627344in}{2.625195in}}%
\pgfpathcurveto{\pgfqpoint{1.619108in}{2.625195in}}{\pgfqpoint{1.611208in}{2.621923in}}{\pgfqpoint{1.605384in}{2.616099in}}%
\pgfpathcurveto{\pgfqpoint{1.599560in}{2.610275in}}{\pgfqpoint{1.596288in}{2.602375in}}{\pgfqpoint{1.596288in}{2.594139in}}%
\pgfpathcurveto{\pgfqpoint{1.596288in}{2.585902in}}{\pgfqpoint{1.599560in}{2.578002in}}{\pgfqpoint{1.605384in}{2.572178in}}%
\pgfpathcurveto{\pgfqpoint{1.611208in}{2.566354in}}{\pgfqpoint{1.619108in}{2.563082in}}{\pgfqpoint{1.627344in}{2.563082in}}%
\pgfpathclose%
\pgfusepath{stroke,fill}%
\end{pgfscope}%
\begin{pgfscope}%
\pgfpathrectangle{\pgfqpoint{0.100000in}{0.212622in}}{\pgfqpoint{3.696000in}{3.696000in}}%
\pgfusepath{clip}%
\pgfsetbuttcap%
\pgfsetroundjoin%
\definecolor{currentfill}{rgb}{0.121569,0.466667,0.705882}%
\pgfsetfillcolor{currentfill}%
\pgfsetfillopacity{0.303126}%
\pgfsetlinewidth{1.003750pt}%
\definecolor{currentstroke}{rgb}{0.121569,0.466667,0.705882}%
\pgfsetstrokecolor{currentstroke}%
\pgfsetstrokeopacity{0.303126}%
\pgfsetdash{}{0pt}%
\pgfpathmoveto{\pgfqpoint{1.627183in}{2.563116in}}%
\pgfpathcurveto{\pgfqpoint{1.635420in}{2.563116in}}{\pgfqpoint{1.643320in}{2.566389in}}{\pgfqpoint{1.649144in}{2.572212in}}%
\pgfpathcurveto{\pgfqpoint{1.654967in}{2.578036in}}{\pgfqpoint{1.658240in}{2.585936in}}{\pgfqpoint{1.658240in}{2.594173in}}%
\pgfpathcurveto{\pgfqpoint{1.658240in}{2.602409in}}{\pgfqpoint{1.654967in}{2.610309in}}{\pgfqpoint{1.649144in}{2.616133in}}%
\pgfpathcurveto{\pgfqpoint{1.643320in}{2.621957in}}{\pgfqpoint{1.635420in}{2.625229in}}{\pgfqpoint{1.627183in}{2.625229in}}%
\pgfpathcurveto{\pgfqpoint{1.618947in}{2.625229in}}{\pgfqpoint{1.611047in}{2.621957in}}{\pgfqpoint{1.605223in}{2.616133in}}%
\pgfpathcurveto{\pgfqpoint{1.599399in}{2.610309in}}{\pgfqpoint{1.596127in}{2.602409in}}{\pgfqpoint{1.596127in}{2.594173in}}%
\pgfpathcurveto{\pgfqpoint{1.596127in}{2.585936in}}{\pgfqpoint{1.599399in}{2.578036in}}{\pgfqpoint{1.605223in}{2.572212in}}%
\pgfpathcurveto{\pgfqpoint{1.611047in}{2.566389in}}{\pgfqpoint{1.618947in}{2.563116in}}{\pgfqpoint{1.627183in}{2.563116in}}%
\pgfpathclose%
\pgfusepath{stroke,fill}%
\end{pgfscope}%
\begin{pgfscope}%
\pgfpathrectangle{\pgfqpoint{0.100000in}{0.212622in}}{\pgfqpoint{3.696000in}{3.696000in}}%
\pgfusepath{clip}%
\pgfsetbuttcap%
\pgfsetroundjoin%
\definecolor{currentfill}{rgb}{0.121569,0.466667,0.705882}%
\pgfsetfillcolor{currentfill}%
\pgfsetfillopacity{0.303179}%
\pgfsetlinewidth{1.003750pt}%
\definecolor{currentstroke}{rgb}{0.121569,0.466667,0.705882}%
\pgfsetstrokecolor{currentstroke}%
\pgfsetstrokeopacity{0.303179}%
\pgfsetdash{}{0pt}%
\pgfpathmoveto{\pgfqpoint{1.626898in}{2.563155in}}%
\pgfpathcurveto{\pgfqpoint{1.635135in}{2.563155in}}{\pgfqpoint{1.643035in}{2.566428in}}{\pgfqpoint{1.648859in}{2.572251in}}%
\pgfpathcurveto{\pgfqpoint{1.654683in}{2.578075in}}{\pgfqpoint{1.657955in}{2.585975in}}{\pgfqpoint{1.657955in}{2.594212in}}%
\pgfpathcurveto{\pgfqpoint{1.657955in}{2.602448in}}{\pgfqpoint{1.654683in}{2.610348in}}{\pgfqpoint{1.648859in}{2.616172in}}%
\pgfpathcurveto{\pgfqpoint{1.643035in}{2.621996in}}{\pgfqpoint{1.635135in}{2.625268in}}{\pgfqpoint{1.626898in}{2.625268in}}%
\pgfpathcurveto{\pgfqpoint{1.618662in}{2.625268in}}{\pgfqpoint{1.610762in}{2.621996in}}{\pgfqpoint{1.604938in}{2.616172in}}%
\pgfpathcurveto{\pgfqpoint{1.599114in}{2.610348in}}{\pgfqpoint{1.595842in}{2.602448in}}{\pgfqpoint{1.595842in}{2.594212in}}%
\pgfpathcurveto{\pgfqpoint{1.595842in}{2.585975in}}{\pgfqpoint{1.599114in}{2.578075in}}{\pgfqpoint{1.604938in}{2.572251in}}%
\pgfpathcurveto{\pgfqpoint{1.610762in}{2.566428in}}{\pgfqpoint{1.618662in}{2.563155in}}{\pgfqpoint{1.626898in}{2.563155in}}%
\pgfpathclose%
\pgfusepath{stroke,fill}%
\end{pgfscope}%
\begin{pgfscope}%
\pgfpathrectangle{\pgfqpoint{0.100000in}{0.212622in}}{\pgfqpoint{3.696000in}{3.696000in}}%
\pgfusepath{clip}%
\pgfsetbuttcap%
\pgfsetroundjoin%
\definecolor{currentfill}{rgb}{0.121569,0.466667,0.705882}%
\pgfsetfillcolor{currentfill}%
\pgfsetfillopacity{0.303279}%
\pgfsetlinewidth{1.003750pt}%
\definecolor{currentstroke}{rgb}{0.121569,0.466667,0.705882}%
\pgfsetstrokecolor{currentstroke}%
\pgfsetstrokeopacity{0.303279}%
\pgfsetdash{}{0pt}%
\pgfpathmoveto{\pgfqpoint{1.626399in}{2.563187in}}%
\pgfpathcurveto{\pgfqpoint{1.634635in}{2.563187in}}{\pgfqpoint{1.642535in}{2.566460in}}{\pgfqpoint{1.648359in}{2.572284in}}%
\pgfpathcurveto{\pgfqpoint{1.654183in}{2.578108in}}{\pgfqpoint{1.657455in}{2.586008in}}{\pgfqpoint{1.657455in}{2.594244in}}%
\pgfpathcurveto{\pgfqpoint{1.657455in}{2.602480in}}{\pgfqpoint{1.654183in}{2.610380in}}{\pgfqpoint{1.648359in}{2.616204in}}%
\pgfpathcurveto{\pgfqpoint{1.642535in}{2.622028in}}{\pgfqpoint{1.634635in}{2.625300in}}{\pgfqpoint{1.626399in}{2.625300in}}%
\pgfpathcurveto{\pgfqpoint{1.618162in}{2.625300in}}{\pgfqpoint{1.610262in}{2.622028in}}{\pgfqpoint{1.604438in}{2.616204in}}%
\pgfpathcurveto{\pgfqpoint{1.598614in}{2.610380in}}{\pgfqpoint{1.595342in}{2.602480in}}{\pgfqpoint{1.595342in}{2.594244in}}%
\pgfpathcurveto{\pgfqpoint{1.595342in}{2.586008in}}{\pgfqpoint{1.598614in}{2.578108in}}{\pgfqpoint{1.604438in}{2.572284in}}%
\pgfpathcurveto{\pgfqpoint{1.610262in}{2.566460in}}{\pgfqpoint{1.618162in}{2.563187in}}{\pgfqpoint{1.626399in}{2.563187in}}%
\pgfpathclose%
\pgfusepath{stroke,fill}%
\end{pgfscope}%
\begin{pgfscope}%
\pgfpathrectangle{\pgfqpoint{0.100000in}{0.212622in}}{\pgfqpoint{3.696000in}{3.696000in}}%
\pgfusepath{clip}%
\pgfsetbuttcap%
\pgfsetroundjoin%
\definecolor{currentfill}{rgb}{0.121569,0.466667,0.705882}%
\pgfsetfillcolor{currentfill}%
\pgfsetfillopacity{0.303461}%
\pgfsetlinewidth{1.003750pt}%
\definecolor{currentstroke}{rgb}{0.121569,0.466667,0.705882}%
\pgfsetstrokecolor{currentstroke}%
\pgfsetstrokeopacity{0.303461}%
\pgfsetdash{}{0pt}%
\pgfpathmoveto{\pgfqpoint{1.625529in}{2.563148in}}%
\pgfpathcurveto{\pgfqpoint{1.633766in}{2.563148in}}{\pgfqpoint{1.641666in}{2.566420in}}{\pgfqpoint{1.647490in}{2.572244in}}%
\pgfpathcurveto{\pgfqpoint{1.653313in}{2.578068in}}{\pgfqpoint{1.656586in}{2.585968in}}{\pgfqpoint{1.656586in}{2.594204in}}%
\pgfpathcurveto{\pgfqpoint{1.656586in}{2.602441in}}{\pgfqpoint{1.653313in}{2.610341in}}{\pgfqpoint{1.647490in}{2.616165in}}%
\pgfpathcurveto{\pgfqpoint{1.641666in}{2.621989in}}{\pgfqpoint{1.633766in}{2.625261in}}{\pgfqpoint{1.625529in}{2.625261in}}%
\pgfpathcurveto{\pgfqpoint{1.617293in}{2.625261in}}{\pgfqpoint{1.609393in}{2.621989in}}{\pgfqpoint{1.603569in}{2.616165in}}%
\pgfpathcurveto{\pgfqpoint{1.597745in}{2.610341in}}{\pgfqpoint{1.594473in}{2.602441in}}{\pgfqpoint{1.594473in}{2.594204in}}%
\pgfpathcurveto{\pgfqpoint{1.594473in}{2.585968in}}{\pgfqpoint{1.597745in}{2.578068in}}{\pgfqpoint{1.603569in}{2.572244in}}%
\pgfpathcurveto{\pgfqpoint{1.609393in}{2.566420in}}{\pgfqpoint{1.617293in}{2.563148in}}{\pgfqpoint{1.625529in}{2.563148in}}%
\pgfpathclose%
\pgfusepath{stroke,fill}%
\end{pgfscope}%
\begin{pgfscope}%
\pgfpathrectangle{\pgfqpoint{0.100000in}{0.212622in}}{\pgfqpoint{3.696000in}{3.696000in}}%
\pgfusepath{clip}%
\pgfsetbuttcap%
\pgfsetroundjoin%
\definecolor{currentfill}{rgb}{0.121569,0.466667,0.705882}%
\pgfsetfillcolor{currentfill}%
\pgfsetfillopacity{0.303558}%
\pgfsetlinewidth{1.003750pt}%
\definecolor{currentstroke}{rgb}{0.121569,0.466667,0.705882}%
\pgfsetstrokecolor{currentstroke}%
\pgfsetstrokeopacity{0.303558}%
\pgfsetdash{}{0pt}%
\pgfpathmoveto{\pgfqpoint{1.625083in}{2.563065in}}%
\pgfpathcurveto{\pgfqpoint{1.633320in}{2.563065in}}{\pgfqpoint{1.641220in}{2.566338in}}{\pgfqpoint{1.647044in}{2.572162in}}%
\pgfpathcurveto{\pgfqpoint{1.652868in}{2.577985in}}{\pgfqpoint{1.656140in}{2.585886in}}{\pgfqpoint{1.656140in}{2.594122in}}%
\pgfpathcurveto{\pgfqpoint{1.656140in}{2.602358in}}{\pgfqpoint{1.652868in}{2.610258in}}{\pgfqpoint{1.647044in}{2.616082in}}%
\pgfpathcurveto{\pgfqpoint{1.641220in}{2.621906in}}{\pgfqpoint{1.633320in}{2.625178in}}{\pgfqpoint{1.625083in}{2.625178in}}%
\pgfpathcurveto{\pgfqpoint{1.616847in}{2.625178in}}{\pgfqpoint{1.608947in}{2.621906in}}{\pgfqpoint{1.603123in}{2.616082in}}%
\pgfpathcurveto{\pgfqpoint{1.597299in}{2.610258in}}{\pgfqpoint{1.594027in}{2.602358in}}{\pgfqpoint{1.594027in}{2.594122in}}%
\pgfpathcurveto{\pgfqpoint{1.594027in}{2.585886in}}{\pgfqpoint{1.597299in}{2.577985in}}{\pgfqpoint{1.603123in}{2.572162in}}%
\pgfpathcurveto{\pgfqpoint{1.608947in}{2.566338in}}{\pgfqpoint{1.616847in}{2.563065in}}{\pgfqpoint{1.625083in}{2.563065in}}%
\pgfpathclose%
\pgfusepath{stroke,fill}%
\end{pgfscope}%
\begin{pgfscope}%
\pgfpathrectangle{\pgfqpoint{0.100000in}{0.212622in}}{\pgfqpoint{3.696000in}{3.696000in}}%
\pgfusepath{clip}%
\pgfsetbuttcap%
\pgfsetroundjoin%
\definecolor{currentfill}{rgb}{0.121569,0.466667,0.705882}%
\pgfsetfillcolor{currentfill}%
\pgfsetfillopacity{0.303732}%
\pgfsetlinewidth{1.003750pt}%
\definecolor{currentstroke}{rgb}{0.121569,0.466667,0.705882}%
\pgfsetstrokecolor{currentstroke}%
\pgfsetstrokeopacity{0.303732}%
\pgfsetdash{}{0pt}%
\pgfpathmoveto{\pgfqpoint{1.624311in}{2.562817in}}%
\pgfpathcurveto{\pgfqpoint{1.632547in}{2.562817in}}{\pgfqpoint{1.640448in}{2.566090in}}{\pgfqpoint{1.646271in}{2.571913in}}%
\pgfpathcurveto{\pgfqpoint{1.652095in}{2.577737in}}{\pgfqpoint{1.655368in}{2.585637in}}{\pgfqpoint{1.655368in}{2.593874in}}%
\pgfpathcurveto{\pgfqpoint{1.655368in}{2.602110in}}{\pgfqpoint{1.652095in}{2.610010in}}{\pgfqpoint{1.646271in}{2.615834in}}%
\pgfpathcurveto{\pgfqpoint{1.640448in}{2.621658in}}{\pgfqpoint{1.632547in}{2.624930in}}{\pgfqpoint{1.624311in}{2.624930in}}%
\pgfpathcurveto{\pgfqpoint{1.616075in}{2.624930in}}{\pgfqpoint{1.608175in}{2.621658in}}{\pgfqpoint{1.602351in}{2.615834in}}%
\pgfpathcurveto{\pgfqpoint{1.596527in}{2.610010in}}{\pgfqpoint{1.593255in}{2.602110in}}{\pgfqpoint{1.593255in}{2.593874in}}%
\pgfpathcurveto{\pgfqpoint{1.593255in}{2.585637in}}{\pgfqpoint{1.596527in}{2.577737in}}{\pgfqpoint{1.602351in}{2.571913in}}%
\pgfpathcurveto{\pgfqpoint{1.608175in}{2.566090in}}{\pgfqpoint{1.616075in}{2.562817in}}{\pgfqpoint{1.624311in}{2.562817in}}%
\pgfpathclose%
\pgfusepath{stroke,fill}%
\end{pgfscope}%
\begin{pgfscope}%
\pgfpathrectangle{\pgfqpoint{0.100000in}{0.212622in}}{\pgfqpoint{3.696000in}{3.696000in}}%
\pgfusepath{clip}%
\pgfsetbuttcap%
\pgfsetroundjoin%
\definecolor{currentfill}{rgb}{0.121569,0.466667,0.705882}%
\pgfsetfillcolor{currentfill}%
\pgfsetfillopacity{0.304045}%
\pgfsetlinewidth{1.003750pt}%
\definecolor{currentstroke}{rgb}{0.121569,0.466667,0.705882}%
\pgfsetstrokecolor{currentstroke}%
\pgfsetstrokeopacity{0.304045}%
\pgfsetdash{}{0pt}%
\pgfpathmoveto{\pgfqpoint{1.622971in}{2.562222in}}%
\pgfpathcurveto{\pgfqpoint{1.631207in}{2.562222in}}{\pgfqpoint{1.639107in}{2.565494in}}{\pgfqpoint{1.644931in}{2.571318in}}%
\pgfpathcurveto{\pgfqpoint{1.650755in}{2.577142in}}{\pgfqpoint{1.654027in}{2.585042in}}{\pgfqpoint{1.654027in}{2.593278in}}%
\pgfpathcurveto{\pgfqpoint{1.654027in}{2.601514in}}{\pgfqpoint{1.650755in}{2.609414in}}{\pgfqpoint{1.644931in}{2.615238in}}%
\pgfpathcurveto{\pgfqpoint{1.639107in}{2.621062in}}{\pgfqpoint{1.631207in}{2.624335in}}{\pgfqpoint{1.622971in}{2.624335in}}%
\pgfpathcurveto{\pgfqpoint{1.614735in}{2.624335in}}{\pgfqpoint{1.606835in}{2.621062in}}{\pgfqpoint{1.601011in}{2.615238in}}%
\pgfpathcurveto{\pgfqpoint{1.595187in}{2.609414in}}{\pgfqpoint{1.591914in}{2.601514in}}{\pgfqpoint{1.591914in}{2.593278in}}%
\pgfpathcurveto{\pgfqpoint{1.591914in}{2.585042in}}{\pgfqpoint{1.595187in}{2.577142in}}{\pgfqpoint{1.601011in}{2.571318in}}%
\pgfpathcurveto{\pgfqpoint{1.606835in}{2.565494in}}{\pgfqpoint{1.614735in}{2.562222in}}{\pgfqpoint{1.622971in}{2.562222in}}%
\pgfpathclose%
\pgfusepath{stroke,fill}%
\end{pgfscope}%
\begin{pgfscope}%
\pgfpathrectangle{\pgfqpoint{0.100000in}{0.212622in}}{\pgfqpoint{3.696000in}{3.696000in}}%
\pgfusepath{clip}%
\pgfsetbuttcap%
\pgfsetroundjoin%
\definecolor{currentfill}{rgb}{0.121569,0.466667,0.705882}%
\pgfsetfillcolor{currentfill}%
\pgfsetfillopacity{0.304141}%
\pgfsetlinewidth{1.003750pt}%
\definecolor{currentstroke}{rgb}{0.121569,0.466667,0.705882}%
\pgfsetstrokecolor{currentstroke}%
\pgfsetstrokeopacity{0.304141}%
\pgfsetdash{}{0pt}%
\pgfpathmoveto{\pgfqpoint{1.725386in}{2.535053in}}%
\pgfpathcurveto{\pgfqpoint{1.733623in}{2.535053in}}{\pgfqpoint{1.741523in}{2.538325in}}{\pgfqpoint{1.747347in}{2.544149in}}%
\pgfpathcurveto{\pgfqpoint{1.753171in}{2.549973in}}{\pgfqpoint{1.756443in}{2.557873in}}{\pgfqpoint{1.756443in}{2.566109in}}%
\pgfpathcurveto{\pgfqpoint{1.756443in}{2.574345in}}{\pgfqpoint{1.753171in}{2.582245in}}{\pgfqpoint{1.747347in}{2.588069in}}%
\pgfpathcurveto{\pgfqpoint{1.741523in}{2.593893in}}{\pgfqpoint{1.733623in}{2.597166in}}{\pgfqpoint{1.725386in}{2.597166in}}%
\pgfpathcurveto{\pgfqpoint{1.717150in}{2.597166in}}{\pgfqpoint{1.709250in}{2.593893in}}{\pgfqpoint{1.703426in}{2.588069in}}%
\pgfpathcurveto{\pgfqpoint{1.697602in}{2.582245in}}{\pgfqpoint{1.694330in}{2.574345in}}{\pgfqpoint{1.694330in}{2.566109in}}%
\pgfpathcurveto{\pgfqpoint{1.694330in}{2.557873in}}{\pgfqpoint{1.697602in}{2.549973in}}{\pgfqpoint{1.703426in}{2.544149in}}%
\pgfpathcurveto{\pgfqpoint{1.709250in}{2.538325in}}{\pgfqpoint{1.717150in}{2.535053in}}{\pgfqpoint{1.725386in}{2.535053in}}%
\pgfpathclose%
\pgfusepath{stroke,fill}%
\end{pgfscope}%
\begin{pgfscope}%
\pgfpathrectangle{\pgfqpoint{0.100000in}{0.212622in}}{\pgfqpoint{3.696000in}{3.696000in}}%
\pgfusepath{clip}%
\pgfsetbuttcap%
\pgfsetroundjoin%
\definecolor{currentfill}{rgb}{0.121569,0.466667,0.705882}%
\pgfsetfillcolor{currentfill}%
\pgfsetfillopacity{0.304601}%
\pgfsetlinewidth{1.003750pt}%
\definecolor{currentstroke}{rgb}{0.121569,0.466667,0.705882}%
\pgfsetstrokecolor{currentstroke}%
\pgfsetstrokeopacity{0.304601}%
\pgfsetdash{}{0pt}%
\pgfpathmoveto{\pgfqpoint{1.620640in}{2.560850in}}%
\pgfpathcurveto{\pgfqpoint{1.628877in}{2.560850in}}{\pgfqpoint{1.636777in}{2.564123in}}{\pgfqpoint{1.642601in}{2.569947in}}%
\pgfpathcurveto{\pgfqpoint{1.648425in}{2.575770in}}{\pgfqpoint{1.651697in}{2.583671in}}{\pgfqpoint{1.651697in}{2.591907in}}%
\pgfpathcurveto{\pgfqpoint{1.651697in}{2.600143in}}{\pgfqpoint{1.648425in}{2.608043in}}{\pgfqpoint{1.642601in}{2.613867in}}%
\pgfpathcurveto{\pgfqpoint{1.636777in}{2.619691in}}{\pgfqpoint{1.628877in}{2.622963in}}{\pgfqpoint{1.620640in}{2.622963in}}%
\pgfpathcurveto{\pgfqpoint{1.612404in}{2.622963in}}{\pgfqpoint{1.604504in}{2.619691in}}{\pgfqpoint{1.598680in}{2.613867in}}%
\pgfpathcurveto{\pgfqpoint{1.592856in}{2.608043in}}{\pgfqpoint{1.589584in}{2.600143in}}{\pgfqpoint{1.589584in}{2.591907in}}%
\pgfpathcurveto{\pgfqpoint{1.589584in}{2.583671in}}{\pgfqpoint{1.592856in}{2.575770in}}{\pgfqpoint{1.598680in}{2.569947in}}%
\pgfpathcurveto{\pgfqpoint{1.604504in}{2.564123in}}{\pgfqpoint{1.612404in}{2.560850in}}{\pgfqpoint{1.620640in}{2.560850in}}%
\pgfpathclose%
\pgfusepath{stroke,fill}%
\end{pgfscope}%
\begin{pgfscope}%
\pgfpathrectangle{\pgfqpoint{0.100000in}{0.212622in}}{\pgfqpoint{3.696000in}{3.696000in}}%
\pgfusepath{clip}%
\pgfsetbuttcap%
\pgfsetroundjoin%
\definecolor{currentfill}{rgb}{0.121569,0.466667,0.705882}%
\pgfsetfillcolor{currentfill}%
\pgfsetfillopacity{0.305020}%
\pgfsetlinewidth{1.003750pt}%
\definecolor{currentstroke}{rgb}{0.121569,0.466667,0.705882}%
\pgfsetstrokecolor{currentstroke}%
\pgfsetstrokeopacity{0.305020}%
\pgfsetdash{}{0pt}%
\pgfpathmoveto{\pgfqpoint{1.618921in}{2.559520in}}%
\pgfpathcurveto{\pgfqpoint{1.627157in}{2.559520in}}{\pgfqpoint{1.635057in}{2.562792in}}{\pgfqpoint{1.640881in}{2.568616in}}%
\pgfpathcurveto{\pgfqpoint{1.646705in}{2.574440in}}{\pgfqpoint{1.649977in}{2.582340in}}{\pgfqpoint{1.649977in}{2.590577in}}%
\pgfpathcurveto{\pgfqpoint{1.649977in}{2.598813in}}{\pgfqpoint{1.646705in}{2.606713in}}{\pgfqpoint{1.640881in}{2.612537in}}%
\pgfpathcurveto{\pgfqpoint{1.635057in}{2.618361in}}{\pgfqpoint{1.627157in}{2.621633in}}{\pgfqpoint{1.618921in}{2.621633in}}%
\pgfpathcurveto{\pgfqpoint{1.610685in}{2.621633in}}{\pgfqpoint{1.602785in}{2.618361in}}{\pgfqpoint{1.596961in}{2.612537in}}%
\pgfpathcurveto{\pgfqpoint{1.591137in}{2.606713in}}{\pgfqpoint{1.587864in}{2.598813in}}{\pgfqpoint{1.587864in}{2.590577in}}%
\pgfpathcurveto{\pgfqpoint{1.587864in}{2.582340in}}{\pgfqpoint{1.591137in}{2.574440in}}{\pgfqpoint{1.596961in}{2.568616in}}%
\pgfpathcurveto{\pgfqpoint{1.602785in}{2.562792in}}{\pgfqpoint{1.610685in}{2.559520in}}{\pgfqpoint{1.618921in}{2.559520in}}%
\pgfpathclose%
\pgfusepath{stroke,fill}%
\end{pgfscope}%
\begin{pgfscope}%
\pgfpathrectangle{\pgfqpoint{0.100000in}{0.212622in}}{\pgfqpoint{3.696000in}{3.696000in}}%
\pgfusepath{clip}%
\pgfsetbuttcap%
\pgfsetroundjoin%
\definecolor{currentfill}{rgb}{0.121569,0.466667,0.705882}%
\pgfsetfillcolor{currentfill}%
\pgfsetfillopacity{0.305309}%
\pgfsetlinewidth{1.003750pt}%
\definecolor{currentstroke}{rgb}{0.121569,0.466667,0.705882}%
\pgfsetstrokecolor{currentstroke}%
\pgfsetstrokeopacity{0.305309}%
\pgfsetdash{}{0pt}%
\pgfpathmoveto{\pgfqpoint{1.736842in}{2.532151in}}%
\pgfpathcurveto{\pgfqpoint{1.745078in}{2.532151in}}{\pgfqpoint{1.752978in}{2.535423in}}{\pgfqpoint{1.758802in}{2.541247in}}%
\pgfpathcurveto{\pgfqpoint{1.764626in}{2.547071in}}{\pgfqpoint{1.767898in}{2.554971in}}{\pgfqpoint{1.767898in}{2.563207in}}%
\pgfpathcurveto{\pgfqpoint{1.767898in}{2.571443in}}{\pgfqpoint{1.764626in}{2.579343in}}{\pgfqpoint{1.758802in}{2.585167in}}%
\pgfpathcurveto{\pgfqpoint{1.752978in}{2.590991in}}{\pgfqpoint{1.745078in}{2.594264in}}{\pgfqpoint{1.736842in}{2.594264in}}%
\pgfpathcurveto{\pgfqpoint{1.728606in}{2.594264in}}{\pgfqpoint{1.720706in}{2.590991in}}{\pgfqpoint{1.714882in}{2.585167in}}%
\pgfpathcurveto{\pgfqpoint{1.709058in}{2.579343in}}{\pgfqpoint{1.705785in}{2.571443in}}{\pgfqpoint{1.705785in}{2.563207in}}%
\pgfpathcurveto{\pgfqpoint{1.705785in}{2.554971in}}{\pgfqpoint{1.709058in}{2.547071in}}{\pgfqpoint{1.714882in}{2.541247in}}%
\pgfpathcurveto{\pgfqpoint{1.720706in}{2.535423in}}{\pgfqpoint{1.728606in}{2.532151in}}{\pgfqpoint{1.736842in}{2.532151in}}%
\pgfpathclose%
\pgfusepath{stroke,fill}%
\end{pgfscope}%
\begin{pgfscope}%
\pgfpathrectangle{\pgfqpoint{0.100000in}{0.212622in}}{\pgfqpoint{3.696000in}{3.696000in}}%
\pgfusepath{clip}%
\pgfsetbuttcap%
\pgfsetroundjoin%
\definecolor{currentfill}{rgb}{0.121569,0.466667,0.705882}%
\pgfsetfillcolor{currentfill}%
\pgfsetfillopacity{0.305316}%
\pgfsetlinewidth{1.003750pt}%
\definecolor{currentstroke}{rgb}{0.121569,0.466667,0.705882}%
\pgfsetstrokecolor{currentstroke}%
\pgfsetstrokeopacity{0.305316}%
\pgfsetdash{}{0pt}%
\pgfpathmoveto{\pgfqpoint{1.617758in}{2.558485in}}%
\pgfpathcurveto{\pgfqpoint{1.625994in}{2.558485in}}{\pgfqpoint{1.633894in}{2.561757in}}{\pgfqpoint{1.639718in}{2.567581in}}%
\pgfpathcurveto{\pgfqpoint{1.645542in}{2.573405in}}{\pgfqpoint{1.648814in}{2.581305in}}{\pgfqpoint{1.648814in}{2.589542in}}%
\pgfpathcurveto{\pgfqpoint{1.648814in}{2.597778in}}{\pgfqpoint{1.645542in}{2.605678in}}{\pgfqpoint{1.639718in}{2.611502in}}%
\pgfpathcurveto{\pgfqpoint{1.633894in}{2.617326in}}{\pgfqpoint{1.625994in}{2.620598in}}{\pgfqpoint{1.617758in}{2.620598in}}%
\pgfpathcurveto{\pgfqpoint{1.609521in}{2.620598in}}{\pgfqpoint{1.601621in}{2.617326in}}{\pgfqpoint{1.595797in}{2.611502in}}%
\pgfpathcurveto{\pgfqpoint{1.589973in}{2.605678in}}{\pgfqpoint{1.586701in}{2.597778in}}{\pgfqpoint{1.586701in}{2.589542in}}%
\pgfpathcurveto{\pgfqpoint{1.586701in}{2.581305in}}{\pgfqpoint{1.589973in}{2.573405in}}{\pgfqpoint{1.595797in}{2.567581in}}%
\pgfpathcurveto{\pgfqpoint{1.601621in}{2.561757in}}{\pgfqpoint{1.609521in}{2.558485in}}{\pgfqpoint{1.617758in}{2.558485in}}%
\pgfpathclose%
\pgfusepath{stroke,fill}%
\end{pgfscope}%
\begin{pgfscope}%
\pgfpathrectangle{\pgfqpoint{0.100000in}{0.212622in}}{\pgfqpoint{3.696000in}{3.696000in}}%
\pgfusepath{clip}%
\pgfsetbuttcap%
\pgfsetroundjoin%
\definecolor{currentfill}{rgb}{0.121569,0.466667,0.705882}%
\pgfsetfillcolor{currentfill}%
\pgfsetfillopacity{0.305849}%
\pgfsetlinewidth{1.003750pt}%
\definecolor{currentstroke}{rgb}{0.121569,0.466667,0.705882}%
\pgfsetstrokecolor{currentstroke}%
\pgfsetstrokeopacity{0.305849}%
\pgfsetdash{}{0pt}%
\pgfpathmoveto{\pgfqpoint{1.615741in}{2.556423in}}%
\pgfpathcurveto{\pgfqpoint{1.623978in}{2.556423in}}{\pgfqpoint{1.631878in}{2.559696in}}{\pgfqpoint{1.637702in}{2.565520in}}%
\pgfpathcurveto{\pgfqpoint{1.643526in}{2.571344in}}{\pgfqpoint{1.646798in}{2.579244in}}{\pgfqpoint{1.646798in}{2.587480in}}%
\pgfpathcurveto{\pgfqpoint{1.646798in}{2.595716in}}{\pgfqpoint{1.643526in}{2.603616in}}{\pgfqpoint{1.637702in}{2.609440in}}%
\pgfpathcurveto{\pgfqpoint{1.631878in}{2.615264in}}{\pgfqpoint{1.623978in}{2.618536in}}{\pgfqpoint{1.615741in}{2.618536in}}%
\pgfpathcurveto{\pgfqpoint{1.607505in}{2.618536in}}{\pgfqpoint{1.599605in}{2.615264in}}{\pgfqpoint{1.593781in}{2.609440in}}%
\pgfpathcurveto{\pgfqpoint{1.587957in}{2.603616in}}{\pgfqpoint{1.584685in}{2.595716in}}{\pgfqpoint{1.584685in}{2.587480in}}%
\pgfpathcurveto{\pgfqpoint{1.584685in}{2.579244in}}{\pgfqpoint{1.587957in}{2.571344in}}{\pgfqpoint{1.593781in}{2.565520in}}%
\pgfpathcurveto{\pgfqpoint{1.599605in}{2.559696in}}{\pgfqpoint{1.607505in}{2.556423in}}{\pgfqpoint{1.615741in}{2.556423in}}%
\pgfpathclose%
\pgfusepath{stroke,fill}%
\end{pgfscope}%
\begin{pgfscope}%
\pgfpathrectangle{\pgfqpoint{0.100000in}{0.212622in}}{\pgfqpoint{3.696000in}{3.696000in}}%
\pgfusepath{clip}%
\pgfsetbuttcap%
\pgfsetroundjoin%
\definecolor{currentfill}{rgb}{0.121569,0.466667,0.705882}%
\pgfsetfillcolor{currentfill}%
\pgfsetfillopacity{0.306243}%
\pgfsetlinewidth{1.003750pt}%
\definecolor{currentstroke}{rgb}{0.121569,0.466667,0.705882}%
\pgfsetstrokecolor{currentstroke}%
\pgfsetstrokeopacity{0.306243}%
\pgfsetdash{}{0pt}%
\pgfpathmoveto{\pgfqpoint{1.614279in}{2.554669in}}%
\pgfpathcurveto{\pgfqpoint{1.622515in}{2.554669in}}{\pgfqpoint{1.630415in}{2.557941in}}{\pgfqpoint{1.636239in}{2.563765in}}%
\pgfpathcurveto{\pgfqpoint{1.642063in}{2.569589in}}{\pgfqpoint{1.645335in}{2.577489in}}{\pgfqpoint{1.645335in}{2.585725in}}%
\pgfpathcurveto{\pgfqpoint{1.645335in}{2.593962in}}{\pgfqpoint{1.642063in}{2.601862in}}{\pgfqpoint{1.636239in}{2.607686in}}%
\pgfpathcurveto{\pgfqpoint{1.630415in}{2.613510in}}{\pgfqpoint{1.622515in}{2.616782in}}{\pgfqpoint{1.614279in}{2.616782in}}%
\pgfpathcurveto{\pgfqpoint{1.606043in}{2.616782in}}{\pgfqpoint{1.598143in}{2.613510in}}{\pgfqpoint{1.592319in}{2.607686in}}%
\pgfpathcurveto{\pgfqpoint{1.586495in}{2.601862in}}{\pgfqpoint{1.583222in}{2.593962in}}{\pgfqpoint{1.583222in}{2.585725in}}%
\pgfpathcurveto{\pgfqpoint{1.583222in}{2.577489in}}{\pgfqpoint{1.586495in}{2.569589in}}{\pgfqpoint{1.592319in}{2.563765in}}%
\pgfpathcurveto{\pgfqpoint{1.598143in}{2.557941in}}{\pgfqpoint{1.606043in}{2.554669in}}{\pgfqpoint{1.614279in}{2.554669in}}%
\pgfpathclose%
\pgfusepath{stroke,fill}%
\end{pgfscope}%
\begin{pgfscope}%
\pgfpathrectangle{\pgfqpoint{0.100000in}{0.212622in}}{\pgfqpoint{3.696000in}{3.696000in}}%
\pgfusepath{clip}%
\pgfsetbuttcap%
\pgfsetroundjoin%
\definecolor{currentfill}{rgb}{0.121569,0.466667,0.705882}%
\pgfsetfillcolor{currentfill}%
\pgfsetfillopacity{0.306397}%
\pgfsetlinewidth{1.003750pt}%
\definecolor{currentstroke}{rgb}{0.121569,0.466667,0.705882}%
\pgfsetstrokecolor{currentstroke}%
\pgfsetstrokeopacity{0.306397}%
\pgfsetdash{}{0pt}%
\pgfpathmoveto{\pgfqpoint{1.613749in}{2.553968in}}%
\pgfpathcurveto{\pgfqpoint{1.621985in}{2.553968in}}{\pgfqpoint{1.629885in}{2.557240in}}{\pgfqpoint{1.635709in}{2.563064in}}%
\pgfpathcurveto{\pgfqpoint{1.641533in}{2.568888in}}{\pgfqpoint{1.644805in}{2.576788in}}{\pgfqpoint{1.644805in}{2.585024in}}%
\pgfpathcurveto{\pgfqpoint{1.644805in}{2.593260in}}{\pgfqpoint{1.641533in}{2.601160in}}{\pgfqpoint{1.635709in}{2.606984in}}%
\pgfpathcurveto{\pgfqpoint{1.629885in}{2.612808in}}{\pgfqpoint{1.621985in}{2.616081in}}{\pgfqpoint{1.613749in}{2.616081in}}%
\pgfpathcurveto{\pgfqpoint{1.605512in}{2.616081in}}{\pgfqpoint{1.597612in}{2.612808in}}{\pgfqpoint{1.591788in}{2.606984in}}%
\pgfpathcurveto{\pgfqpoint{1.585964in}{2.601160in}}{\pgfqpoint{1.582692in}{2.593260in}}{\pgfqpoint{1.582692in}{2.585024in}}%
\pgfpathcurveto{\pgfqpoint{1.582692in}{2.576788in}}{\pgfqpoint{1.585964in}{2.568888in}}{\pgfqpoint{1.591788in}{2.563064in}}%
\pgfpathcurveto{\pgfqpoint{1.597612in}{2.557240in}}{\pgfqpoint{1.605512in}{2.553968in}}{\pgfqpoint{1.613749in}{2.553968in}}%
\pgfpathclose%
\pgfusepath{stroke,fill}%
\end{pgfscope}%
\begin{pgfscope}%
\pgfpathrectangle{\pgfqpoint{0.100000in}{0.212622in}}{\pgfqpoint{3.696000in}{3.696000in}}%
\pgfusepath{clip}%
\pgfsetbuttcap%
\pgfsetroundjoin%
\definecolor{currentfill}{rgb}{0.121569,0.466667,0.705882}%
\pgfsetfillcolor{currentfill}%
\pgfsetfillopacity{0.306451}%
\pgfsetlinewidth{1.003750pt}%
\definecolor{currentstroke}{rgb}{0.121569,0.466667,0.705882}%
\pgfsetstrokecolor{currentstroke}%
\pgfsetstrokeopacity{0.306451}%
\pgfsetdash{}{0pt}%
\pgfpathmoveto{\pgfqpoint{1.613574in}{2.553699in}}%
\pgfpathcurveto{\pgfqpoint{1.621811in}{2.553699in}}{\pgfqpoint{1.629711in}{2.556972in}}{\pgfqpoint{1.635535in}{2.562796in}}%
\pgfpathcurveto{\pgfqpoint{1.641359in}{2.568619in}}{\pgfqpoint{1.644631in}{2.576519in}}{\pgfqpoint{1.644631in}{2.584756in}}%
\pgfpathcurveto{\pgfqpoint{1.644631in}{2.592992in}}{\pgfqpoint{1.641359in}{2.600892in}}{\pgfqpoint{1.635535in}{2.606716in}}%
\pgfpathcurveto{\pgfqpoint{1.629711in}{2.612540in}}{\pgfqpoint{1.621811in}{2.615812in}}{\pgfqpoint{1.613574in}{2.615812in}}%
\pgfpathcurveto{\pgfqpoint{1.605338in}{2.615812in}}{\pgfqpoint{1.597438in}{2.612540in}}{\pgfqpoint{1.591614in}{2.606716in}}%
\pgfpathcurveto{\pgfqpoint{1.585790in}{2.600892in}}{\pgfqpoint{1.582518in}{2.592992in}}{\pgfqpoint{1.582518in}{2.584756in}}%
\pgfpathcurveto{\pgfqpoint{1.582518in}{2.576519in}}{\pgfqpoint{1.585790in}{2.568619in}}{\pgfqpoint{1.591614in}{2.562796in}}%
\pgfpathcurveto{\pgfqpoint{1.597438in}{2.556972in}}{\pgfqpoint{1.605338in}{2.553699in}}{\pgfqpoint{1.613574in}{2.553699in}}%
\pgfpathclose%
\pgfusepath{stroke,fill}%
\end{pgfscope}%
\begin{pgfscope}%
\pgfpathrectangle{\pgfqpoint{0.100000in}{0.212622in}}{\pgfqpoint{3.696000in}{3.696000in}}%
\pgfusepath{clip}%
\pgfsetbuttcap%
\pgfsetroundjoin%
\definecolor{currentfill}{rgb}{0.121569,0.466667,0.705882}%
\pgfsetfillcolor{currentfill}%
\pgfsetfillopacity{0.306554}%
\pgfsetlinewidth{1.003750pt}%
\definecolor{currentstroke}{rgb}{0.121569,0.466667,0.705882}%
\pgfsetstrokecolor{currentstroke}%
\pgfsetstrokeopacity{0.306554}%
\pgfsetdash{}{0pt}%
\pgfpathmoveto{\pgfqpoint{1.613283in}{2.553233in}}%
\pgfpathcurveto{\pgfqpoint{1.621520in}{2.553233in}}{\pgfqpoint{1.629420in}{2.556506in}}{\pgfqpoint{1.635244in}{2.562330in}}%
\pgfpathcurveto{\pgfqpoint{1.641068in}{2.568153in}}{\pgfqpoint{1.644340in}{2.576054in}}{\pgfqpoint{1.644340in}{2.584290in}}%
\pgfpathcurveto{\pgfqpoint{1.644340in}{2.592526in}}{\pgfqpoint{1.641068in}{2.600426in}}{\pgfqpoint{1.635244in}{2.606250in}}%
\pgfpathcurveto{\pgfqpoint{1.629420in}{2.612074in}}{\pgfqpoint{1.621520in}{2.615346in}}{\pgfqpoint{1.613283in}{2.615346in}}%
\pgfpathcurveto{\pgfqpoint{1.605047in}{2.615346in}}{\pgfqpoint{1.597147in}{2.612074in}}{\pgfqpoint{1.591323in}{2.606250in}}%
\pgfpathcurveto{\pgfqpoint{1.585499in}{2.600426in}}{\pgfqpoint{1.582227in}{2.592526in}}{\pgfqpoint{1.582227in}{2.584290in}}%
\pgfpathcurveto{\pgfqpoint{1.582227in}{2.576054in}}{\pgfqpoint{1.585499in}{2.568153in}}{\pgfqpoint{1.591323in}{2.562330in}}%
\pgfpathcurveto{\pgfqpoint{1.597147in}{2.556506in}}{\pgfqpoint{1.605047in}{2.553233in}}{\pgfqpoint{1.613283in}{2.553233in}}%
\pgfpathclose%
\pgfusepath{stroke,fill}%
\end{pgfscope}%
\begin{pgfscope}%
\pgfpathrectangle{\pgfqpoint{0.100000in}{0.212622in}}{\pgfqpoint{3.696000in}{3.696000in}}%
\pgfusepath{clip}%
\pgfsetbuttcap%
\pgfsetroundjoin%
\definecolor{currentfill}{rgb}{0.121569,0.466667,0.705882}%
\pgfsetfillcolor{currentfill}%
\pgfsetfillopacity{0.306763}%
\pgfsetlinewidth{1.003750pt}%
\definecolor{currentstroke}{rgb}{0.121569,0.466667,0.705882}%
\pgfsetstrokecolor{currentstroke}%
\pgfsetstrokeopacity{0.306763}%
\pgfsetdash{}{0pt}%
\pgfpathmoveto{\pgfqpoint{1.612795in}{2.552497in}}%
\pgfpathcurveto{\pgfqpoint{1.621031in}{2.552497in}}{\pgfqpoint{1.628931in}{2.555769in}}{\pgfqpoint{1.634755in}{2.561593in}}%
\pgfpathcurveto{\pgfqpoint{1.640579in}{2.567417in}}{\pgfqpoint{1.643851in}{2.575317in}}{\pgfqpoint{1.643851in}{2.583553in}}%
\pgfpathcurveto{\pgfqpoint{1.643851in}{2.591789in}}{\pgfqpoint{1.640579in}{2.599689in}}{\pgfqpoint{1.634755in}{2.605513in}}%
\pgfpathcurveto{\pgfqpoint{1.628931in}{2.611337in}}{\pgfqpoint{1.621031in}{2.614610in}}{\pgfqpoint{1.612795in}{2.614610in}}%
\pgfpathcurveto{\pgfqpoint{1.604558in}{2.614610in}}{\pgfqpoint{1.596658in}{2.611337in}}{\pgfqpoint{1.590834in}{2.605513in}}%
\pgfpathcurveto{\pgfqpoint{1.585010in}{2.599689in}}{\pgfqpoint{1.581738in}{2.591789in}}{\pgfqpoint{1.581738in}{2.583553in}}%
\pgfpathcurveto{\pgfqpoint{1.581738in}{2.575317in}}{\pgfqpoint{1.585010in}{2.567417in}}{\pgfqpoint{1.590834in}{2.561593in}}%
\pgfpathcurveto{\pgfqpoint{1.596658in}{2.555769in}}{\pgfqpoint{1.604558in}{2.552497in}}{\pgfqpoint{1.612795in}{2.552497in}}%
\pgfpathclose%
\pgfusepath{stroke,fill}%
\end{pgfscope}%
\begin{pgfscope}%
\pgfpathrectangle{\pgfqpoint{0.100000in}{0.212622in}}{\pgfqpoint{3.696000in}{3.696000in}}%
\pgfusepath{clip}%
\pgfsetbuttcap%
\pgfsetroundjoin%
\definecolor{currentfill}{rgb}{0.121569,0.466667,0.705882}%
\pgfsetfillcolor{currentfill}%
\pgfsetfillopacity{0.306879}%
\pgfsetlinewidth{1.003750pt}%
\definecolor{currentstroke}{rgb}{0.121569,0.466667,0.705882}%
\pgfsetstrokecolor{currentstroke}%
\pgfsetstrokeopacity{0.306879}%
\pgfsetdash{}{0pt}%
\pgfpathmoveto{\pgfqpoint{1.612550in}{2.552101in}}%
\pgfpathcurveto{\pgfqpoint{1.620786in}{2.552101in}}{\pgfqpoint{1.628686in}{2.555374in}}{\pgfqpoint{1.634510in}{2.561198in}}%
\pgfpathcurveto{\pgfqpoint{1.640334in}{2.567022in}}{\pgfqpoint{1.643606in}{2.574922in}}{\pgfqpoint{1.643606in}{2.583158in}}%
\pgfpathcurveto{\pgfqpoint{1.643606in}{2.591394in}}{\pgfqpoint{1.640334in}{2.599294in}}{\pgfqpoint{1.634510in}{2.605118in}}%
\pgfpathcurveto{\pgfqpoint{1.628686in}{2.610942in}}{\pgfqpoint{1.620786in}{2.614214in}}{\pgfqpoint{1.612550in}{2.614214in}}%
\pgfpathcurveto{\pgfqpoint{1.604313in}{2.614214in}}{\pgfqpoint{1.596413in}{2.610942in}}{\pgfqpoint{1.590589in}{2.605118in}}%
\pgfpathcurveto{\pgfqpoint{1.584766in}{2.599294in}}{\pgfqpoint{1.581493in}{2.591394in}}{\pgfqpoint{1.581493in}{2.583158in}}%
\pgfpathcurveto{\pgfqpoint{1.581493in}{2.574922in}}{\pgfqpoint{1.584766in}{2.567022in}}{\pgfqpoint{1.590589in}{2.561198in}}%
\pgfpathcurveto{\pgfqpoint{1.596413in}{2.555374in}}{\pgfqpoint{1.604313in}{2.552101in}}{\pgfqpoint{1.612550in}{2.552101in}}%
\pgfpathclose%
\pgfusepath{stroke,fill}%
\end{pgfscope}%
\begin{pgfscope}%
\pgfpathrectangle{\pgfqpoint{0.100000in}{0.212622in}}{\pgfqpoint{3.696000in}{3.696000in}}%
\pgfusepath{clip}%
\pgfsetbuttcap%
\pgfsetroundjoin%
\definecolor{currentfill}{rgb}{0.121569,0.466667,0.705882}%
\pgfsetfillcolor{currentfill}%
\pgfsetfillopacity{0.306970}%
\pgfsetlinewidth{1.003750pt}%
\definecolor{currentstroke}{rgb}{0.121569,0.466667,0.705882}%
\pgfsetstrokecolor{currentstroke}%
\pgfsetstrokeopacity{0.306970}%
\pgfsetdash{}{0pt}%
\pgfpathmoveto{\pgfqpoint{1.750602in}{2.530407in}}%
\pgfpathcurveto{\pgfqpoint{1.758839in}{2.530407in}}{\pgfqpoint{1.766739in}{2.533679in}}{\pgfqpoint{1.772563in}{2.539503in}}%
\pgfpathcurveto{\pgfqpoint{1.778387in}{2.545327in}}{\pgfqpoint{1.781659in}{2.553227in}}{\pgfqpoint{1.781659in}{2.561463in}}%
\pgfpathcurveto{\pgfqpoint{1.781659in}{2.569700in}}{\pgfqpoint{1.778387in}{2.577600in}}{\pgfqpoint{1.772563in}{2.583424in}}%
\pgfpathcurveto{\pgfqpoint{1.766739in}{2.589248in}}{\pgfqpoint{1.758839in}{2.592520in}}{\pgfqpoint{1.750602in}{2.592520in}}%
\pgfpathcurveto{\pgfqpoint{1.742366in}{2.592520in}}{\pgfqpoint{1.734466in}{2.589248in}}{\pgfqpoint{1.728642in}{2.583424in}}%
\pgfpathcurveto{\pgfqpoint{1.722818in}{2.577600in}}{\pgfqpoint{1.719546in}{2.569700in}}{\pgfqpoint{1.719546in}{2.561463in}}%
\pgfpathcurveto{\pgfqpoint{1.719546in}{2.553227in}}{\pgfqpoint{1.722818in}{2.545327in}}{\pgfqpoint{1.728642in}{2.539503in}}%
\pgfpathcurveto{\pgfqpoint{1.734466in}{2.533679in}}{\pgfqpoint{1.742366in}{2.530407in}}{\pgfqpoint{1.750602in}{2.530407in}}%
\pgfpathclose%
\pgfusepath{stroke,fill}%
\end{pgfscope}%
\begin{pgfscope}%
\pgfpathrectangle{\pgfqpoint{0.100000in}{0.212622in}}{\pgfqpoint{3.696000in}{3.696000in}}%
\pgfusepath{clip}%
\pgfsetbuttcap%
\pgfsetroundjoin%
\definecolor{currentfill}{rgb}{0.121569,0.466667,0.705882}%
\pgfsetfillcolor{currentfill}%
\pgfsetfillopacity{0.307074}%
\pgfsetlinewidth{1.003750pt}%
\definecolor{currentstroke}{rgb}{0.121569,0.466667,0.705882}%
\pgfsetstrokecolor{currentstroke}%
\pgfsetstrokeopacity{0.307074}%
\pgfsetdash{}{0pt}%
\pgfpathmoveto{\pgfqpoint{1.612081in}{2.551283in}}%
\pgfpathcurveto{\pgfqpoint{1.620317in}{2.551283in}}{\pgfqpoint{1.628217in}{2.554555in}}{\pgfqpoint{1.634041in}{2.560379in}}%
\pgfpathcurveto{\pgfqpoint{1.639865in}{2.566203in}}{\pgfqpoint{1.643137in}{2.574103in}}{\pgfqpoint{1.643137in}{2.582339in}}%
\pgfpathcurveto{\pgfqpoint{1.643137in}{2.590576in}}{\pgfqpoint{1.639865in}{2.598476in}}{\pgfqpoint{1.634041in}{2.604300in}}%
\pgfpathcurveto{\pgfqpoint{1.628217in}{2.610124in}}{\pgfqpoint{1.620317in}{2.613396in}}{\pgfqpoint{1.612081in}{2.613396in}}%
\pgfpathcurveto{\pgfqpoint{1.603845in}{2.613396in}}{\pgfqpoint{1.595945in}{2.610124in}}{\pgfqpoint{1.590121in}{2.604300in}}%
\pgfpathcurveto{\pgfqpoint{1.584297in}{2.598476in}}{\pgfqpoint{1.581024in}{2.590576in}}{\pgfqpoint{1.581024in}{2.582339in}}%
\pgfpathcurveto{\pgfqpoint{1.581024in}{2.574103in}}{\pgfqpoint{1.584297in}{2.566203in}}{\pgfqpoint{1.590121in}{2.560379in}}%
\pgfpathcurveto{\pgfqpoint{1.595945in}{2.554555in}}{\pgfqpoint{1.603845in}{2.551283in}}{\pgfqpoint{1.612081in}{2.551283in}}%
\pgfpathclose%
\pgfusepath{stroke,fill}%
\end{pgfscope}%
\begin{pgfscope}%
\pgfpathrectangle{\pgfqpoint{0.100000in}{0.212622in}}{\pgfqpoint{3.696000in}{3.696000in}}%
\pgfusepath{clip}%
\pgfsetbuttcap%
\pgfsetroundjoin%
\definecolor{currentfill}{rgb}{0.121569,0.466667,0.705882}%
\pgfsetfillcolor{currentfill}%
\pgfsetfillopacity{0.307452}%
\pgfsetlinewidth{1.003750pt}%
\definecolor{currentstroke}{rgb}{0.121569,0.466667,0.705882}%
\pgfsetstrokecolor{currentstroke}%
\pgfsetstrokeopacity{0.307452}%
\pgfsetdash{}{0pt}%
\pgfpathmoveto{\pgfqpoint{1.611188in}{2.550005in}}%
\pgfpathcurveto{\pgfqpoint{1.619424in}{2.550005in}}{\pgfqpoint{1.627324in}{2.553278in}}{\pgfqpoint{1.633148in}{2.559102in}}%
\pgfpathcurveto{\pgfqpoint{1.638972in}{2.564925in}}{\pgfqpoint{1.642244in}{2.572825in}}{\pgfqpoint{1.642244in}{2.581062in}}%
\pgfpathcurveto{\pgfqpoint{1.642244in}{2.589298in}}{\pgfqpoint{1.638972in}{2.597198in}}{\pgfqpoint{1.633148in}{2.603022in}}%
\pgfpathcurveto{\pgfqpoint{1.627324in}{2.608846in}}{\pgfqpoint{1.619424in}{2.612118in}}{\pgfqpoint{1.611188in}{2.612118in}}%
\pgfpathcurveto{\pgfqpoint{1.602952in}{2.612118in}}{\pgfqpoint{1.595052in}{2.608846in}}{\pgfqpoint{1.589228in}{2.603022in}}%
\pgfpathcurveto{\pgfqpoint{1.583404in}{2.597198in}}{\pgfqpoint{1.580131in}{2.589298in}}{\pgfqpoint{1.580131in}{2.581062in}}%
\pgfpathcurveto{\pgfqpoint{1.580131in}{2.572825in}}{\pgfqpoint{1.583404in}{2.564925in}}{\pgfqpoint{1.589228in}{2.559102in}}%
\pgfpathcurveto{\pgfqpoint{1.595052in}{2.553278in}}{\pgfqpoint{1.602952in}{2.550005in}}{\pgfqpoint{1.611188in}{2.550005in}}%
\pgfpathclose%
\pgfusepath{stroke,fill}%
\end{pgfscope}%
\begin{pgfscope}%
\pgfpathrectangle{\pgfqpoint{0.100000in}{0.212622in}}{\pgfqpoint{3.696000in}{3.696000in}}%
\pgfusepath{clip}%
\pgfsetbuttcap%
\pgfsetroundjoin%
\definecolor{currentfill}{rgb}{0.121569,0.466667,0.705882}%
\pgfsetfillcolor{currentfill}%
\pgfsetfillopacity{0.308016}%
\pgfsetlinewidth{1.003750pt}%
\definecolor{currentstroke}{rgb}{0.121569,0.466667,0.705882}%
\pgfsetstrokecolor{currentstroke}%
\pgfsetstrokeopacity{0.308016}%
\pgfsetdash{}{0pt}%
\pgfpathmoveto{\pgfqpoint{1.758097in}{2.530085in}}%
\pgfpathcurveto{\pgfqpoint{1.766333in}{2.530085in}}{\pgfqpoint{1.774233in}{2.533357in}}{\pgfqpoint{1.780057in}{2.539181in}}%
\pgfpathcurveto{\pgfqpoint{1.785881in}{2.545005in}}{\pgfqpoint{1.789153in}{2.552905in}}{\pgfqpoint{1.789153in}{2.561142in}}%
\pgfpathcurveto{\pgfqpoint{1.789153in}{2.569378in}}{\pgfqpoint{1.785881in}{2.577278in}}{\pgfqpoint{1.780057in}{2.583102in}}%
\pgfpathcurveto{\pgfqpoint{1.774233in}{2.588926in}}{\pgfqpoint{1.766333in}{2.592198in}}{\pgfqpoint{1.758097in}{2.592198in}}%
\pgfpathcurveto{\pgfqpoint{1.749861in}{2.592198in}}{\pgfqpoint{1.741960in}{2.588926in}}{\pgfqpoint{1.736137in}{2.583102in}}%
\pgfpathcurveto{\pgfqpoint{1.730313in}{2.577278in}}{\pgfqpoint{1.727040in}{2.569378in}}{\pgfqpoint{1.727040in}{2.561142in}}%
\pgfpathcurveto{\pgfqpoint{1.727040in}{2.552905in}}{\pgfqpoint{1.730313in}{2.545005in}}{\pgfqpoint{1.736137in}{2.539181in}}%
\pgfpathcurveto{\pgfqpoint{1.741960in}{2.533357in}}{\pgfqpoint{1.749861in}{2.530085in}}{\pgfqpoint{1.758097in}{2.530085in}}%
\pgfpathclose%
\pgfusepath{stroke,fill}%
\end{pgfscope}%
\begin{pgfscope}%
\pgfpathrectangle{\pgfqpoint{0.100000in}{0.212622in}}{\pgfqpoint{3.696000in}{3.696000in}}%
\pgfusepath{clip}%
\pgfsetbuttcap%
\pgfsetroundjoin%
\definecolor{currentfill}{rgb}{0.121569,0.466667,0.705882}%
\pgfsetfillcolor{currentfill}%
\pgfsetfillopacity{0.308118}%
\pgfsetlinewidth{1.003750pt}%
\definecolor{currentstroke}{rgb}{0.121569,0.466667,0.705882}%
\pgfsetstrokecolor{currentstroke}%
\pgfsetstrokeopacity{0.308118}%
\pgfsetdash{}{0pt}%
\pgfpathmoveto{\pgfqpoint{1.609497in}{2.547583in}}%
\pgfpathcurveto{\pgfqpoint{1.617734in}{2.547583in}}{\pgfqpoint{1.625634in}{2.550855in}}{\pgfqpoint{1.631458in}{2.556679in}}%
\pgfpathcurveto{\pgfqpoint{1.637282in}{2.562503in}}{\pgfqpoint{1.640554in}{2.570403in}}{\pgfqpoint{1.640554in}{2.578639in}}%
\pgfpathcurveto{\pgfqpoint{1.640554in}{2.586875in}}{\pgfqpoint{1.637282in}{2.594775in}}{\pgfqpoint{1.631458in}{2.600599in}}%
\pgfpathcurveto{\pgfqpoint{1.625634in}{2.606423in}}{\pgfqpoint{1.617734in}{2.609696in}}{\pgfqpoint{1.609497in}{2.609696in}}%
\pgfpathcurveto{\pgfqpoint{1.601261in}{2.609696in}}{\pgfqpoint{1.593361in}{2.606423in}}{\pgfqpoint{1.587537in}{2.600599in}}%
\pgfpathcurveto{\pgfqpoint{1.581713in}{2.594775in}}{\pgfqpoint{1.578441in}{2.586875in}}{\pgfqpoint{1.578441in}{2.578639in}}%
\pgfpathcurveto{\pgfqpoint{1.578441in}{2.570403in}}{\pgfqpoint{1.581713in}{2.562503in}}{\pgfqpoint{1.587537in}{2.556679in}}%
\pgfpathcurveto{\pgfqpoint{1.593361in}{2.550855in}}{\pgfqpoint{1.601261in}{2.547583in}}{\pgfqpoint{1.609497in}{2.547583in}}%
\pgfpathclose%
\pgfusepath{stroke,fill}%
\end{pgfscope}%
\begin{pgfscope}%
\pgfpathrectangle{\pgfqpoint{0.100000in}{0.212622in}}{\pgfqpoint{3.696000in}{3.696000in}}%
\pgfusepath{clip}%
\pgfsetbuttcap%
\pgfsetroundjoin%
\definecolor{currentfill}{rgb}{0.121569,0.466667,0.705882}%
\pgfsetfillcolor{currentfill}%
\pgfsetfillopacity{0.308575}%
\pgfsetlinewidth{1.003750pt}%
\definecolor{currentstroke}{rgb}{0.121569,0.466667,0.705882}%
\pgfsetstrokecolor{currentstroke}%
\pgfsetstrokeopacity{0.308575}%
\pgfsetdash{}{0pt}%
\pgfpathmoveto{\pgfqpoint{1.608237in}{2.545734in}}%
\pgfpathcurveto{\pgfqpoint{1.616473in}{2.545734in}}{\pgfqpoint{1.624373in}{2.549006in}}{\pgfqpoint{1.630197in}{2.554830in}}%
\pgfpathcurveto{\pgfqpoint{1.636021in}{2.560654in}}{\pgfqpoint{1.639293in}{2.568554in}}{\pgfqpoint{1.639293in}{2.576790in}}%
\pgfpathcurveto{\pgfqpoint{1.639293in}{2.585027in}}{\pgfqpoint{1.636021in}{2.592927in}}{\pgfqpoint{1.630197in}{2.598751in}}%
\pgfpathcurveto{\pgfqpoint{1.624373in}{2.604574in}}{\pgfqpoint{1.616473in}{2.607847in}}{\pgfqpoint{1.608237in}{2.607847in}}%
\pgfpathcurveto{\pgfqpoint{1.600001in}{2.607847in}}{\pgfqpoint{1.592100in}{2.604574in}}{\pgfqpoint{1.586277in}{2.598751in}}%
\pgfpathcurveto{\pgfqpoint{1.580453in}{2.592927in}}{\pgfqpoint{1.577180in}{2.585027in}}{\pgfqpoint{1.577180in}{2.576790in}}%
\pgfpathcurveto{\pgfqpoint{1.577180in}{2.568554in}}{\pgfqpoint{1.580453in}{2.560654in}}{\pgfqpoint{1.586277in}{2.554830in}}%
\pgfpathcurveto{\pgfqpoint{1.592100in}{2.549006in}}{\pgfqpoint{1.600001in}{2.545734in}}{\pgfqpoint{1.608237in}{2.545734in}}%
\pgfpathclose%
\pgfusepath{stroke,fill}%
\end{pgfscope}%
\begin{pgfscope}%
\pgfpathrectangle{\pgfqpoint{0.100000in}{0.212622in}}{\pgfqpoint{3.696000in}{3.696000in}}%
\pgfusepath{clip}%
\pgfsetbuttcap%
\pgfsetroundjoin%
\definecolor{currentfill}{rgb}{0.121569,0.466667,0.705882}%
\pgfsetfillcolor{currentfill}%
\pgfsetfillopacity{0.308815}%
\pgfsetlinewidth{1.003750pt}%
\definecolor{currentstroke}{rgb}{0.121569,0.466667,0.705882}%
\pgfsetstrokecolor{currentstroke}%
\pgfsetstrokeopacity{0.308815}%
\pgfsetdash{}{0pt}%
\pgfpathmoveto{\pgfqpoint{1.607638in}{2.544785in}}%
\pgfpathcurveto{\pgfqpoint{1.615874in}{2.544785in}}{\pgfqpoint{1.623774in}{2.548058in}}{\pgfqpoint{1.629598in}{2.553881in}}%
\pgfpathcurveto{\pgfqpoint{1.635422in}{2.559705in}}{\pgfqpoint{1.638694in}{2.567605in}}{\pgfqpoint{1.638694in}{2.575842in}}%
\pgfpathcurveto{\pgfqpoint{1.638694in}{2.584078in}}{\pgfqpoint{1.635422in}{2.591978in}}{\pgfqpoint{1.629598in}{2.597802in}}%
\pgfpathcurveto{\pgfqpoint{1.623774in}{2.603626in}}{\pgfqpoint{1.615874in}{2.606898in}}{\pgfqpoint{1.607638in}{2.606898in}}%
\pgfpathcurveto{\pgfqpoint{1.599402in}{2.606898in}}{\pgfqpoint{1.591502in}{2.603626in}}{\pgfqpoint{1.585678in}{2.597802in}}%
\pgfpathcurveto{\pgfqpoint{1.579854in}{2.591978in}}{\pgfqpoint{1.576581in}{2.584078in}}{\pgfqpoint{1.576581in}{2.575842in}}%
\pgfpathcurveto{\pgfqpoint{1.576581in}{2.567605in}}{\pgfqpoint{1.579854in}{2.559705in}}{\pgfqpoint{1.585678in}{2.553881in}}%
\pgfpathcurveto{\pgfqpoint{1.591502in}{2.548058in}}{\pgfqpoint{1.599402in}{2.544785in}}{\pgfqpoint{1.607638in}{2.544785in}}%
\pgfpathclose%
\pgfusepath{stroke,fill}%
\end{pgfscope}%
\begin{pgfscope}%
\pgfpathrectangle{\pgfqpoint{0.100000in}{0.212622in}}{\pgfqpoint{3.696000in}{3.696000in}}%
\pgfusepath{clip}%
\pgfsetbuttcap%
\pgfsetroundjoin%
\definecolor{currentfill}{rgb}{0.121569,0.466667,0.705882}%
\pgfsetfillcolor{currentfill}%
\pgfsetfillopacity{0.309199}%
\pgfsetlinewidth{1.003750pt}%
\definecolor{currentstroke}{rgb}{0.121569,0.466667,0.705882}%
\pgfsetstrokecolor{currentstroke}%
\pgfsetstrokeopacity{0.309199}%
\pgfsetdash{}{0pt}%
\pgfpathmoveto{\pgfqpoint{1.766567in}{2.529552in}}%
\pgfpathcurveto{\pgfqpoint{1.774803in}{2.529552in}}{\pgfqpoint{1.782703in}{2.532825in}}{\pgfqpoint{1.788527in}{2.538648in}}%
\pgfpathcurveto{\pgfqpoint{1.794351in}{2.544472in}}{\pgfqpoint{1.797624in}{2.552372in}}{\pgfqpoint{1.797624in}{2.560609in}}%
\pgfpathcurveto{\pgfqpoint{1.797624in}{2.568845in}}{\pgfqpoint{1.794351in}{2.576745in}}{\pgfqpoint{1.788527in}{2.582569in}}%
\pgfpathcurveto{\pgfqpoint{1.782703in}{2.588393in}}{\pgfqpoint{1.774803in}{2.591665in}}{\pgfqpoint{1.766567in}{2.591665in}}%
\pgfpathcurveto{\pgfqpoint{1.758331in}{2.591665in}}{\pgfqpoint{1.750431in}{2.588393in}}{\pgfqpoint{1.744607in}{2.582569in}}%
\pgfpathcurveto{\pgfqpoint{1.738783in}{2.576745in}}{\pgfqpoint{1.735511in}{2.568845in}}{\pgfqpoint{1.735511in}{2.560609in}}%
\pgfpathcurveto{\pgfqpoint{1.735511in}{2.552372in}}{\pgfqpoint{1.738783in}{2.544472in}}{\pgfqpoint{1.744607in}{2.538648in}}%
\pgfpathcurveto{\pgfqpoint{1.750431in}{2.532825in}}{\pgfqpoint{1.758331in}{2.529552in}}{\pgfqpoint{1.766567in}{2.529552in}}%
\pgfpathclose%
\pgfusepath{stroke,fill}%
\end{pgfscope}%
\begin{pgfscope}%
\pgfpathrectangle{\pgfqpoint{0.100000in}{0.212622in}}{\pgfqpoint{3.696000in}{3.696000in}}%
\pgfusepath{clip}%
\pgfsetbuttcap%
\pgfsetroundjoin%
\definecolor{currentfill}{rgb}{0.121569,0.466667,0.705882}%
\pgfsetfillcolor{currentfill}%
\pgfsetfillopacity{0.309259}%
\pgfsetlinewidth{1.003750pt}%
\definecolor{currentstroke}{rgb}{0.121569,0.466667,0.705882}%
\pgfsetstrokecolor{currentstroke}%
\pgfsetstrokeopacity{0.309259}%
\pgfsetdash{}{0pt}%
\pgfpathmoveto{\pgfqpoint{1.606537in}{2.543114in}}%
\pgfpathcurveto{\pgfqpoint{1.614774in}{2.543114in}}{\pgfqpoint{1.622674in}{2.546386in}}{\pgfqpoint{1.628498in}{2.552210in}}%
\pgfpathcurveto{\pgfqpoint{1.634321in}{2.558034in}}{\pgfqpoint{1.637594in}{2.565934in}}{\pgfqpoint{1.637594in}{2.574171in}}%
\pgfpathcurveto{\pgfqpoint{1.637594in}{2.582407in}}{\pgfqpoint{1.634321in}{2.590307in}}{\pgfqpoint{1.628498in}{2.596131in}}%
\pgfpathcurveto{\pgfqpoint{1.622674in}{2.601955in}}{\pgfqpoint{1.614774in}{2.605227in}}{\pgfqpoint{1.606537in}{2.605227in}}%
\pgfpathcurveto{\pgfqpoint{1.598301in}{2.605227in}}{\pgfqpoint{1.590401in}{2.601955in}}{\pgfqpoint{1.584577in}{2.596131in}}%
\pgfpathcurveto{\pgfqpoint{1.578753in}{2.590307in}}{\pgfqpoint{1.575481in}{2.582407in}}{\pgfqpoint{1.575481in}{2.574171in}}%
\pgfpathcurveto{\pgfqpoint{1.575481in}{2.565934in}}{\pgfqpoint{1.578753in}{2.558034in}}{\pgfqpoint{1.584577in}{2.552210in}}%
\pgfpathcurveto{\pgfqpoint{1.590401in}{2.546386in}}{\pgfqpoint{1.598301in}{2.543114in}}{\pgfqpoint{1.606537in}{2.543114in}}%
\pgfpathclose%
\pgfusepath{stroke,fill}%
\end{pgfscope}%
\begin{pgfscope}%
\pgfpathrectangle{\pgfqpoint{0.100000in}{0.212622in}}{\pgfqpoint{3.696000in}{3.696000in}}%
\pgfusepath{clip}%
\pgfsetbuttcap%
\pgfsetroundjoin%
\definecolor{currentfill}{rgb}{0.121569,0.466667,0.705882}%
\pgfsetfillcolor{currentfill}%
\pgfsetfillopacity{0.309560}%
\pgfsetlinewidth{1.003750pt}%
\definecolor{currentstroke}{rgb}{0.121569,0.466667,0.705882}%
\pgfsetstrokecolor{currentstroke}%
\pgfsetstrokeopacity{0.309560}%
\pgfsetdash{}{0pt}%
\pgfpathmoveto{\pgfqpoint{1.605842in}{2.542034in}}%
\pgfpathcurveto{\pgfqpoint{1.614078in}{2.542034in}}{\pgfqpoint{1.621978in}{2.545307in}}{\pgfqpoint{1.627802in}{2.551131in}}%
\pgfpathcurveto{\pgfqpoint{1.633626in}{2.556955in}}{\pgfqpoint{1.636898in}{2.564855in}}{\pgfqpoint{1.636898in}{2.573091in}}%
\pgfpathcurveto{\pgfqpoint{1.636898in}{2.581327in}}{\pgfqpoint{1.633626in}{2.589227in}}{\pgfqpoint{1.627802in}{2.595051in}}%
\pgfpathcurveto{\pgfqpoint{1.621978in}{2.600875in}}{\pgfqpoint{1.614078in}{2.604147in}}{\pgfqpoint{1.605842in}{2.604147in}}%
\pgfpathcurveto{\pgfqpoint{1.597605in}{2.604147in}}{\pgfqpoint{1.589705in}{2.600875in}}{\pgfqpoint{1.583881in}{2.595051in}}%
\pgfpathcurveto{\pgfqpoint{1.578057in}{2.589227in}}{\pgfqpoint{1.574785in}{2.581327in}}{\pgfqpoint{1.574785in}{2.573091in}}%
\pgfpathcurveto{\pgfqpoint{1.574785in}{2.564855in}}{\pgfqpoint{1.578057in}{2.556955in}}{\pgfqpoint{1.583881in}{2.551131in}}%
\pgfpathcurveto{\pgfqpoint{1.589705in}{2.545307in}}{\pgfqpoint{1.597605in}{2.542034in}}{\pgfqpoint{1.605842in}{2.542034in}}%
\pgfpathclose%
\pgfusepath{stroke,fill}%
\end{pgfscope}%
\begin{pgfscope}%
\pgfpathrectangle{\pgfqpoint{0.100000in}{0.212622in}}{\pgfqpoint{3.696000in}{3.696000in}}%
\pgfusepath{clip}%
\pgfsetbuttcap%
\pgfsetroundjoin%
\definecolor{currentfill}{rgb}{0.121569,0.466667,0.705882}%
\pgfsetfillcolor{currentfill}%
\pgfsetfillopacity{0.310098}%
\pgfsetlinewidth{1.003750pt}%
\definecolor{currentstroke}{rgb}{0.121569,0.466667,0.705882}%
\pgfsetstrokecolor{currentstroke}%
\pgfsetstrokeopacity{0.310098}%
\pgfsetdash{}{0pt}%
\pgfpathmoveto{\pgfqpoint{1.604621in}{2.539961in}}%
\pgfpathcurveto{\pgfqpoint{1.612858in}{2.539961in}}{\pgfqpoint{1.620758in}{2.543233in}}{\pgfqpoint{1.626582in}{2.549057in}}%
\pgfpathcurveto{\pgfqpoint{1.632406in}{2.554881in}}{\pgfqpoint{1.635678in}{2.562781in}}{\pgfqpoint{1.635678in}{2.571017in}}%
\pgfpathcurveto{\pgfqpoint{1.635678in}{2.579254in}}{\pgfqpoint{1.632406in}{2.587154in}}{\pgfqpoint{1.626582in}{2.592978in}}%
\pgfpathcurveto{\pgfqpoint{1.620758in}{2.598802in}}{\pgfqpoint{1.612858in}{2.602074in}}{\pgfqpoint{1.604621in}{2.602074in}}%
\pgfpathcurveto{\pgfqpoint{1.596385in}{2.602074in}}{\pgfqpoint{1.588485in}{2.598802in}}{\pgfqpoint{1.582661in}{2.592978in}}%
\pgfpathcurveto{\pgfqpoint{1.576837in}{2.587154in}}{\pgfqpoint{1.573565in}{2.579254in}}{\pgfqpoint{1.573565in}{2.571017in}}%
\pgfpathcurveto{\pgfqpoint{1.573565in}{2.562781in}}{\pgfqpoint{1.576837in}{2.554881in}}{\pgfqpoint{1.582661in}{2.549057in}}%
\pgfpathcurveto{\pgfqpoint{1.588485in}{2.543233in}}{\pgfqpoint{1.596385in}{2.539961in}}{\pgfqpoint{1.604621in}{2.539961in}}%
\pgfpathclose%
\pgfusepath{stroke,fill}%
\end{pgfscope}%
\begin{pgfscope}%
\pgfpathrectangle{\pgfqpoint{0.100000in}{0.212622in}}{\pgfqpoint{3.696000in}{3.696000in}}%
\pgfusepath{clip}%
\pgfsetbuttcap%
\pgfsetroundjoin%
\definecolor{currentfill}{rgb}{0.121569,0.466667,0.705882}%
\pgfsetfillcolor{currentfill}%
\pgfsetfillopacity{0.310293}%
\pgfsetlinewidth{1.003750pt}%
\definecolor{currentstroke}{rgb}{0.121569,0.466667,0.705882}%
\pgfsetstrokecolor{currentstroke}%
\pgfsetstrokeopacity{0.310293}%
\pgfsetdash{}{0pt}%
\pgfpathmoveto{\pgfqpoint{1.604139in}{2.539210in}}%
\pgfpathcurveto{\pgfqpoint{1.612375in}{2.539210in}}{\pgfqpoint{1.620275in}{2.542483in}}{\pgfqpoint{1.626099in}{2.548307in}}%
\pgfpathcurveto{\pgfqpoint{1.631923in}{2.554131in}}{\pgfqpoint{1.635196in}{2.562031in}}{\pgfqpoint{1.635196in}{2.570267in}}%
\pgfpathcurveto{\pgfqpoint{1.635196in}{2.578503in}}{\pgfqpoint{1.631923in}{2.586403in}}{\pgfqpoint{1.626099in}{2.592227in}}%
\pgfpathcurveto{\pgfqpoint{1.620275in}{2.598051in}}{\pgfqpoint{1.612375in}{2.601323in}}{\pgfqpoint{1.604139in}{2.601323in}}%
\pgfpathcurveto{\pgfqpoint{1.595903in}{2.601323in}}{\pgfqpoint{1.588003in}{2.598051in}}{\pgfqpoint{1.582179in}{2.592227in}}%
\pgfpathcurveto{\pgfqpoint{1.576355in}{2.586403in}}{\pgfqpoint{1.573083in}{2.578503in}}{\pgfqpoint{1.573083in}{2.570267in}}%
\pgfpathcurveto{\pgfqpoint{1.573083in}{2.562031in}}{\pgfqpoint{1.576355in}{2.554131in}}{\pgfqpoint{1.582179in}{2.548307in}}%
\pgfpathcurveto{\pgfqpoint{1.588003in}{2.542483in}}{\pgfqpoint{1.595903in}{2.539210in}}{\pgfqpoint{1.604139in}{2.539210in}}%
\pgfpathclose%
\pgfusepath{stroke,fill}%
\end{pgfscope}%
\begin{pgfscope}%
\pgfpathrectangle{\pgfqpoint{0.100000in}{0.212622in}}{\pgfqpoint{3.696000in}{3.696000in}}%
\pgfusepath{clip}%
\pgfsetbuttcap%
\pgfsetroundjoin%
\definecolor{currentfill}{rgb}{0.121569,0.466667,0.705882}%
\pgfsetfillcolor{currentfill}%
\pgfsetfillopacity{0.310578}%
\pgfsetlinewidth{1.003750pt}%
\definecolor{currentstroke}{rgb}{0.121569,0.466667,0.705882}%
\pgfsetstrokecolor{currentstroke}%
\pgfsetstrokeopacity{0.310578}%
\pgfsetdash{}{0pt}%
\pgfpathmoveto{\pgfqpoint{1.775798in}{2.529031in}}%
\pgfpathcurveto{\pgfqpoint{1.784034in}{2.529031in}}{\pgfqpoint{1.791934in}{2.532304in}}{\pgfqpoint{1.797758in}{2.538128in}}%
\pgfpathcurveto{\pgfqpoint{1.803582in}{2.543952in}}{\pgfqpoint{1.806855in}{2.551852in}}{\pgfqpoint{1.806855in}{2.560088in}}%
\pgfpathcurveto{\pgfqpoint{1.806855in}{2.568324in}}{\pgfqpoint{1.803582in}{2.576224in}}{\pgfqpoint{1.797758in}{2.582048in}}%
\pgfpathcurveto{\pgfqpoint{1.791934in}{2.587872in}}{\pgfqpoint{1.784034in}{2.591144in}}{\pgfqpoint{1.775798in}{2.591144in}}%
\pgfpathcurveto{\pgfqpoint{1.767562in}{2.591144in}}{\pgfqpoint{1.759662in}{2.587872in}}{\pgfqpoint{1.753838in}{2.582048in}}%
\pgfpathcurveto{\pgfqpoint{1.748014in}{2.576224in}}{\pgfqpoint{1.744742in}{2.568324in}}{\pgfqpoint{1.744742in}{2.560088in}}%
\pgfpathcurveto{\pgfqpoint{1.744742in}{2.551852in}}{\pgfqpoint{1.748014in}{2.543952in}}{\pgfqpoint{1.753838in}{2.538128in}}%
\pgfpathcurveto{\pgfqpoint{1.759662in}{2.532304in}}{\pgfqpoint{1.767562in}{2.529031in}}{\pgfqpoint{1.775798in}{2.529031in}}%
\pgfpathclose%
\pgfusepath{stroke,fill}%
\end{pgfscope}%
\begin{pgfscope}%
\pgfpathrectangle{\pgfqpoint{0.100000in}{0.212622in}}{\pgfqpoint{3.696000in}{3.696000in}}%
\pgfusepath{clip}%
\pgfsetbuttcap%
\pgfsetroundjoin%
\definecolor{currentfill}{rgb}{0.121569,0.466667,0.705882}%
\pgfsetfillcolor{currentfill}%
\pgfsetfillopacity{0.310651}%
\pgfsetlinewidth{1.003750pt}%
\definecolor{currentstroke}{rgb}{0.121569,0.466667,0.705882}%
\pgfsetstrokecolor{currentstroke}%
\pgfsetstrokeopacity{0.310651}%
\pgfsetdash{}{0pt}%
\pgfpathmoveto{\pgfqpoint{1.603283in}{2.537851in}}%
\pgfpathcurveto{\pgfqpoint{1.611519in}{2.537851in}}{\pgfqpoint{1.619419in}{2.541123in}}{\pgfqpoint{1.625243in}{2.546947in}}%
\pgfpathcurveto{\pgfqpoint{1.631067in}{2.552771in}}{\pgfqpoint{1.634339in}{2.560671in}}{\pgfqpoint{1.634339in}{2.568907in}}%
\pgfpathcurveto{\pgfqpoint{1.634339in}{2.577144in}}{\pgfqpoint{1.631067in}{2.585044in}}{\pgfqpoint{1.625243in}{2.590868in}}%
\pgfpathcurveto{\pgfqpoint{1.619419in}{2.596692in}}{\pgfqpoint{1.611519in}{2.599964in}}{\pgfqpoint{1.603283in}{2.599964in}}%
\pgfpathcurveto{\pgfqpoint{1.595046in}{2.599964in}}{\pgfqpoint{1.587146in}{2.596692in}}{\pgfqpoint{1.581322in}{2.590868in}}%
\pgfpathcurveto{\pgfqpoint{1.575498in}{2.585044in}}{\pgfqpoint{1.572226in}{2.577144in}}{\pgfqpoint{1.572226in}{2.568907in}}%
\pgfpathcurveto{\pgfqpoint{1.572226in}{2.560671in}}{\pgfqpoint{1.575498in}{2.552771in}}{\pgfqpoint{1.581322in}{2.546947in}}%
\pgfpathcurveto{\pgfqpoint{1.587146in}{2.541123in}}{\pgfqpoint{1.595046in}{2.537851in}}{\pgfqpoint{1.603283in}{2.537851in}}%
\pgfpathclose%
\pgfusepath{stroke,fill}%
\end{pgfscope}%
\begin{pgfscope}%
\pgfpathrectangle{\pgfqpoint{0.100000in}{0.212622in}}{\pgfqpoint{3.696000in}{3.696000in}}%
\pgfusepath{clip}%
\pgfsetbuttcap%
\pgfsetroundjoin%
\definecolor{currentfill}{rgb}{0.121569,0.466667,0.705882}%
\pgfsetfillcolor{currentfill}%
\pgfsetfillopacity{0.310918}%
\pgfsetlinewidth{1.003750pt}%
\definecolor{currentstroke}{rgb}{0.121569,0.466667,0.705882}%
\pgfsetstrokecolor{currentstroke}%
\pgfsetstrokeopacity{0.310918}%
\pgfsetdash{}{0pt}%
\pgfpathmoveto{\pgfqpoint{1.602619in}{2.536801in}}%
\pgfpathcurveto{\pgfqpoint{1.610855in}{2.536801in}}{\pgfqpoint{1.618755in}{2.540073in}}{\pgfqpoint{1.624579in}{2.545897in}}%
\pgfpathcurveto{\pgfqpoint{1.630403in}{2.551721in}}{\pgfqpoint{1.633675in}{2.559621in}}{\pgfqpoint{1.633675in}{2.567857in}}%
\pgfpathcurveto{\pgfqpoint{1.633675in}{2.576093in}}{\pgfqpoint{1.630403in}{2.583994in}}{\pgfqpoint{1.624579in}{2.589817in}}%
\pgfpathcurveto{\pgfqpoint{1.618755in}{2.595641in}}{\pgfqpoint{1.610855in}{2.598914in}}{\pgfqpoint{1.602619in}{2.598914in}}%
\pgfpathcurveto{\pgfqpoint{1.594383in}{2.598914in}}{\pgfqpoint{1.586482in}{2.595641in}}{\pgfqpoint{1.580659in}{2.589817in}}%
\pgfpathcurveto{\pgfqpoint{1.574835in}{2.583994in}}{\pgfqpoint{1.571562in}{2.576093in}}{\pgfqpoint{1.571562in}{2.567857in}}%
\pgfpathcurveto{\pgfqpoint{1.571562in}{2.559621in}}{\pgfqpoint{1.574835in}{2.551721in}}{\pgfqpoint{1.580659in}{2.545897in}}%
\pgfpathcurveto{\pgfqpoint{1.586482in}{2.540073in}}{\pgfqpoint{1.594383in}{2.536801in}}{\pgfqpoint{1.602619in}{2.536801in}}%
\pgfpathclose%
\pgfusepath{stroke,fill}%
\end{pgfscope}%
\begin{pgfscope}%
\pgfpathrectangle{\pgfqpoint{0.100000in}{0.212622in}}{\pgfqpoint{3.696000in}{3.696000in}}%
\pgfusepath{clip}%
\pgfsetbuttcap%
\pgfsetroundjoin%
\definecolor{currentfill}{rgb}{0.121569,0.466667,0.705882}%
\pgfsetfillcolor{currentfill}%
\pgfsetfillopacity{0.311395}%
\pgfsetlinewidth{1.003750pt}%
\definecolor{currentstroke}{rgb}{0.121569,0.466667,0.705882}%
\pgfsetstrokecolor{currentstroke}%
\pgfsetstrokeopacity{0.311395}%
\pgfsetdash{}{0pt}%
\pgfpathmoveto{\pgfqpoint{1.601329in}{2.534904in}}%
\pgfpathcurveto{\pgfqpoint{1.609565in}{2.534904in}}{\pgfqpoint{1.617465in}{2.538176in}}{\pgfqpoint{1.623289in}{2.544000in}}%
\pgfpathcurveto{\pgfqpoint{1.629113in}{2.549824in}}{\pgfqpoint{1.632385in}{2.557724in}}{\pgfqpoint{1.632385in}{2.565960in}}%
\pgfpathcurveto{\pgfqpoint{1.632385in}{2.574197in}}{\pgfqpoint{1.629113in}{2.582097in}}{\pgfqpoint{1.623289in}{2.587921in}}%
\pgfpathcurveto{\pgfqpoint{1.617465in}{2.593744in}}{\pgfqpoint{1.609565in}{2.597017in}}{\pgfqpoint{1.601329in}{2.597017in}}%
\pgfpathcurveto{\pgfqpoint{1.593092in}{2.597017in}}{\pgfqpoint{1.585192in}{2.593744in}}{\pgfqpoint{1.579368in}{2.587921in}}%
\pgfpathcurveto{\pgfqpoint{1.573544in}{2.582097in}}{\pgfqpoint{1.570272in}{2.574197in}}{\pgfqpoint{1.570272in}{2.565960in}}%
\pgfpathcurveto{\pgfqpoint{1.570272in}{2.557724in}}{\pgfqpoint{1.573544in}{2.549824in}}{\pgfqpoint{1.579368in}{2.544000in}}%
\pgfpathcurveto{\pgfqpoint{1.585192in}{2.538176in}}{\pgfqpoint{1.593092in}{2.534904in}}{\pgfqpoint{1.601329in}{2.534904in}}%
\pgfpathclose%
\pgfusepath{stroke,fill}%
\end{pgfscope}%
\begin{pgfscope}%
\pgfpathrectangle{\pgfqpoint{0.100000in}{0.212622in}}{\pgfqpoint{3.696000in}{3.696000in}}%
\pgfusepath{clip}%
\pgfsetbuttcap%
\pgfsetroundjoin%
\definecolor{currentfill}{rgb}{0.121569,0.466667,0.705882}%
\pgfsetfillcolor{currentfill}%
\pgfsetfillopacity{0.311601}%
\pgfsetlinewidth{1.003750pt}%
\definecolor{currentstroke}{rgb}{0.121569,0.466667,0.705882}%
\pgfsetstrokecolor{currentstroke}%
\pgfsetstrokeopacity{0.311601}%
\pgfsetdash{}{0pt}%
\pgfpathmoveto{\pgfqpoint{1.600800in}{2.534110in}}%
\pgfpathcurveto{\pgfqpoint{1.609037in}{2.534110in}}{\pgfqpoint{1.616937in}{2.537383in}}{\pgfqpoint{1.622761in}{2.543207in}}%
\pgfpathcurveto{\pgfqpoint{1.628585in}{2.549031in}}{\pgfqpoint{1.631857in}{2.556931in}}{\pgfqpoint{1.631857in}{2.565167in}}%
\pgfpathcurveto{\pgfqpoint{1.631857in}{2.573403in}}{\pgfqpoint{1.628585in}{2.581303in}}{\pgfqpoint{1.622761in}{2.587127in}}%
\pgfpathcurveto{\pgfqpoint{1.616937in}{2.592951in}}{\pgfqpoint{1.609037in}{2.596223in}}{\pgfqpoint{1.600800in}{2.596223in}}%
\pgfpathcurveto{\pgfqpoint{1.592564in}{2.596223in}}{\pgfqpoint{1.584664in}{2.592951in}}{\pgfqpoint{1.578840in}{2.587127in}}%
\pgfpathcurveto{\pgfqpoint{1.573016in}{2.581303in}}{\pgfqpoint{1.569744in}{2.573403in}}{\pgfqpoint{1.569744in}{2.565167in}}%
\pgfpathcurveto{\pgfqpoint{1.569744in}{2.556931in}}{\pgfqpoint{1.573016in}{2.549031in}}{\pgfqpoint{1.578840in}{2.543207in}}%
\pgfpathcurveto{\pgfqpoint{1.584664in}{2.537383in}}{\pgfqpoint{1.592564in}{2.534110in}}{\pgfqpoint{1.600800in}{2.534110in}}%
\pgfpathclose%
\pgfusepath{stroke,fill}%
\end{pgfscope}%
\begin{pgfscope}%
\pgfpathrectangle{\pgfqpoint{0.100000in}{0.212622in}}{\pgfqpoint{3.696000in}{3.696000in}}%
\pgfusepath{clip}%
\pgfsetbuttcap%
\pgfsetroundjoin%
\definecolor{currentfill}{rgb}{0.121569,0.466667,0.705882}%
\pgfsetfillcolor{currentfill}%
\pgfsetfillopacity{0.311970}%
\pgfsetlinewidth{1.003750pt}%
\definecolor{currentstroke}{rgb}{0.121569,0.466667,0.705882}%
\pgfsetstrokecolor{currentstroke}%
\pgfsetstrokeopacity{0.311970}%
\pgfsetdash{}{0pt}%
\pgfpathmoveto{\pgfqpoint{1.599797in}{2.532672in}}%
\pgfpathcurveto{\pgfqpoint{1.608033in}{2.532672in}}{\pgfqpoint{1.615933in}{2.535944in}}{\pgfqpoint{1.621757in}{2.541768in}}%
\pgfpathcurveto{\pgfqpoint{1.627581in}{2.547592in}}{\pgfqpoint{1.630854in}{2.555492in}}{\pgfqpoint{1.630854in}{2.563728in}}%
\pgfpathcurveto{\pgfqpoint{1.630854in}{2.571964in}}{\pgfqpoint{1.627581in}{2.579864in}}{\pgfqpoint{1.621757in}{2.585688in}}%
\pgfpathcurveto{\pgfqpoint{1.615933in}{2.591512in}}{\pgfqpoint{1.608033in}{2.594785in}}{\pgfqpoint{1.599797in}{2.594785in}}%
\pgfpathcurveto{\pgfqpoint{1.591561in}{2.594785in}}{\pgfqpoint{1.583661in}{2.591512in}}{\pgfqpoint{1.577837in}{2.585688in}}%
\pgfpathcurveto{\pgfqpoint{1.572013in}{2.579864in}}{\pgfqpoint{1.568741in}{2.571964in}}{\pgfqpoint{1.568741in}{2.563728in}}%
\pgfpathcurveto{\pgfqpoint{1.568741in}{2.555492in}}{\pgfqpoint{1.572013in}{2.547592in}}{\pgfqpoint{1.577837in}{2.541768in}}%
\pgfpathcurveto{\pgfqpoint{1.583661in}{2.535944in}}{\pgfqpoint{1.591561in}{2.532672in}}{\pgfqpoint{1.599797in}{2.532672in}}%
\pgfpathclose%
\pgfusepath{stroke,fill}%
\end{pgfscope}%
\begin{pgfscope}%
\pgfpathrectangle{\pgfqpoint{0.100000in}{0.212622in}}{\pgfqpoint{3.696000in}{3.696000in}}%
\pgfusepath{clip}%
\pgfsetbuttcap%
\pgfsetroundjoin%
\definecolor{currentfill}{rgb}{0.121569,0.466667,0.705882}%
\pgfsetfillcolor{currentfill}%
\pgfsetfillopacity{0.312531}%
\pgfsetlinewidth{1.003750pt}%
\definecolor{currentstroke}{rgb}{0.121569,0.466667,0.705882}%
\pgfsetstrokecolor{currentstroke}%
\pgfsetstrokeopacity{0.312531}%
\pgfsetdash{}{0pt}%
\pgfpathmoveto{\pgfqpoint{1.788894in}{2.527781in}}%
\pgfpathcurveto{\pgfqpoint{1.797130in}{2.527781in}}{\pgfqpoint{1.805030in}{2.531053in}}{\pgfqpoint{1.810854in}{2.536877in}}%
\pgfpathcurveto{\pgfqpoint{1.816678in}{2.542701in}}{\pgfqpoint{1.819950in}{2.550601in}}{\pgfqpoint{1.819950in}{2.558837in}}%
\pgfpathcurveto{\pgfqpoint{1.819950in}{2.567074in}}{\pgfqpoint{1.816678in}{2.574974in}}{\pgfqpoint{1.810854in}{2.580798in}}%
\pgfpathcurveto{\pgfqpoint{1.805030in}{2.586622in}}{\pgfqpoint{1.797130in}{2.589894in}}{\pgfqpoint{1.788894in}{2.589894in}}%
\pgfpathcurveto{\pgfqpoint{1.780658in}{2.589894in}}{\pgfqpoint{1.772758in}{2.586622in}}{\pgfqpoint{1.766934in}{2.580798in}}%
\pgfpathcurveto{\pgfqpoint{1.761110in}{2.574974in}}{\pgfqpoint{1.757837in}{2.567074in}}{\pgfqpoint{1.757837in}{2.558837in}}%
\pgfpathcurveto{\pgfqpoint{1.757837in}{2.550601in}}{\pgfqpoint{1.761110in}{2.542701in}}{\pgfqpoint{1.766934in}{2.536877in}}%
\pgfpathcurveto{\pgfqpoint{1.772758in}{2.531053in}}{\pgfqpoint{1.780658in}{2.527781in}}{\pgfqpoint{1.788894in}{2.527781in}}%
\pgfpathclose%
\pgfusepath{stroke,fill}%
\end{pgfscope}%
\begin{pgfscope}%
\pgfpathrectangle{\pgfqpoint{0.100000in}{0.212622in}}{\pgfqpoint{3.696000in}{3.696000in}}%
\pgfusepath{clip}%
\pgfsetbuttcap%
\pgfsetroundjoin%
\definecolor{currentfill}{rgb}{0.121569,0.466667,0.705882}%
\pgfsetfillcolor{currentfill}%
\pgfsetfillopacity{0.312654}%
\pgfsetlinewidth{1.003750pt}%
\definecolor{currentstroke}{rgb}{0.121569,0.466667,0.705882}%
\pgfsetstrokecolor{currentstroke}%
\pgfsetstrokeopacity{0.312654}%
\pgfsetdash{}{0pt}%
\pgfpathmoveto{\pgfqpoint{1.598029in}{2.530084in}}%
\pgfpathcurveto{\pgfqpoint{1.606265in}{2.530084in}}{\pgfqpoint{1.614165in}{2.533356in}}{\pgfqpoint{1.619989in}{2.539180in}}%
\pgfpathcurveto{\pgfqpoint{1.625813in}{2.545004in}}{\pgfqpoint{1.629085in}{2.552904in}}{\pgfqpoint{1.629085in}{2.561140in}}%
\pgfpathcurveto{\pgfqpoint{1.629085in}{2.569377in}}{\pgfqpoint{1.625813in}{2.577277in}}{\pgfqpoint{1.619989in}{2.583101in}}%
\pgfpathcurveto{\pgfqpoint{1.614165in}{2.588924in}}{\pgfqpoint{1.606265in}{2.592197in}}{\pgfqpoint{1.598029in}{2.592197in}}%
\pgfpathcurveto{\pgfqpoint{1.589793in}{2.592197in}}{\pgfqpoint{1.581893in}{2.588924in}}{\pgfqpoint{1.576069in}{2.583101in}}%
\pgfpathcurveto{\pgfqpoint{1.570245in}{2.577277in}}{\pgfqpoint{1.566972in}{2.569377in}}{\pgfqpoint{1.566972in}{2.561140in}}%
\pgfpathcurveto{\pgfqpoint{1.566972in}{2.552904in}}{\pgfqpoint{1.570245in}{2.545004in}}{\pgfqpoint{1.576069in}{2.539180in}}%
\pgfpathcurveto{\pgfqpoint{1.581893in}{2.533356in}}{\pgfqpoint{1.589793in}{2.530084in}}{\pgfqpoint{1.598029in}{2.530084in}}%
\pgfpathclose%
\pgfusepath{stroke,fill}%
\end{pgfscope}%
\begin{pgfscope}%
\pgfpathrectangle{\pgfqpoint{0.100000in}{0.212622in}}{\pgfqpoint{3.696000in}{3.696000in}}%
\pgfusepath{clip}%
\pgfsetbuttcap%
\pgfsetroundjoin%
\definecolor{currentfill}{rgb}{0.121569,0.466667,0.705882}%
\pgfsetfillcolor{currentfill}%
\pgfsetfillopacity{0.313122}%
\pgfsetlinewidth{1.003750pt}%
\definecolor{currentstroke}{rgb}{0.121569,0.466667,0.705882}%
\pgfsetstrokecolor{currentstroke}%
\pgfsetstrokeopacity{0.313122}%
\pgfsetdash{}{0pt}%
\pgfpathmoveto{\pgfqpoint{1.596965in}{2.528218in}}%
\pgfpathcurveto{\pgfqpoint{1.605202in}{2.528218in}}{\pgfqpoint{1.613102in}{2.531491in}}{\pgfqpoint{1.618925in}{2.537315in}}%
\pgfpathcurveto{\pgfqpoint{1.624749in}{2.543138in}}{\pgfqpoint{1.628022in}{2.551039in}}{\pgfqpoint{1.628022in}{2.559275in}}%
\pgfpathcurveto{\pgfqpoint{1.628022in}{2.567511in}}{\pgfqpoint{1.624749in}{2.575411in}}{\pgfqpoint{1.618925in}{2.581235in}}%
\pgfpathcurveto{\pgfqpoint{1.613102in}{2.587059in}}{\pgfqpoint{1.605202in}{2.590331in}}{\pgfqpoint{1.596965in}{2.590331in}}%
\pgfpathcurveto{\pgfqpoint{1.588729in}{2.590331in}}{\pgfqpoint{1.580829in}{2.587059in}}{\pgfqpoint{1.575005in}{2.581235in}}%
\pgfpathcurveto{\pgfqpoint{1.569181in}{2.575411in}}{\pgfqpoint{1.565909in}{2.567511in}}{\pgfqpoint{1.565909in}{2.559275in}}%
\pgfpathcurveto{\pgfqpoint{1.565909in}{2.551039in}}{\pgfqpoint{1.569181in}{2.543138in}}{\pgfqpoint{1.575005in}{2.537315in}}%
\pgfpathcurveto{\pgfqpoint{1.580829in}{2.531491in}}{\pgfqpoint{1.588729in}{2.528218in}}{\pgfqpoint{1.596965in}{2.528218in}}%
\pgfpathclose%
\pgfusepath{stroke,fill}%
\end{pgfscope}%
\begin{pgfscope}%
\pgfpathrectangle{\pgfqpoint{0.100000in}{0.212622in}}{\pgfqpoint{3.696000in}{3.696000in}}%
\pgfusepath{clip}%
\pgfsetbuttcap%
\pgfsetroundjoin%
\definecolor{currentfill}{rgb}{0.121569,0.466667,0.705882}%
\pgfsetfillcolor{currentfill}%
\pgfsetfillopacity{0.313968}%
\pgfsetlinewidth{1.003750pt}%
\definecolor{currentstroke}{rgb}{0.121569,0.466667,0.705882}%
\pgfsetstrokecolor{currentstroke}%
\pgfsetstrokeopacity{0.313968}%
\pgfsetdash{}{0pt}%
\pgfpathmoveto{\pgfqpoint{1.594932in}{2.524873in}}%
\pgfpathcurveto{\pgfqpoint{1.603168in}{2.524873in}}{\pgfqpoint{1.611068in}{2.528145in}}{\pgfqpoint{1.616892in}{2.533969in}}%
\pgfpathcurveto{\pgfqpoint{1.622716in}{2.539793in}}{\pgfqpoint{1.625989in}{2.547693in}}{\pgfqpoint{1.625989in}{2.555929in}}%
\pgfpathcurveto{\pgfqpoint{1.625989in}{2.564165in}}{\pgfqpoint{1.622716in}{2.572065in}}{\pgfqpoint{1.616892in}{2.577889in}}%
\pgfpathcurveto{\pgfqpoint{1.611068in}{2.583713in}}{\pgfqpoint{1.603168in}{2.586986in}}{\pgfqpoint{1.594932in}{2.586986in}}%
\pgfpathcurveto{\pgfqpoint{1.586696in}{2.586986in}}{\pgfqpoint{1.578796in}{2.583713in}}{\pgfqpoint{1.572972in}{2.577889in}}%
\pgfpathcurveto{\pgfqpoint{1.567148in}{2.572065in}}{\pgfqpoint{1.563876in}{2.564165in}}{\pgfqpoint{1.563876in}{2.555929in}}%
\pgfpathcurveto{\pgfqpoint{1.563876in}{2.547693in}}{\pgfqpoint{1.567148in}{2.539793in}}{\pgfqpoint{1.572972in}{2.533969in}}%
\pgfpathcurveto{\pgfqpoint{1.578796in}{2.528145in}}{\pgfqpoint{1.586696in}{2.524873in}}{\pgfqpoint{1.594932in}{2.524873in}}%
\pgfpathclose%
\pgfusepath{stroke,fill}%
\end{pgfscope}%
\begin{pgfscope}%
\pgfpathrectangle{\pgfqpoint{0.100000in}{0.212622in}}{\pgfqpoint{3.696000in}{3.696000in}}%
\pgfusepath{clip}%
\pgfsetbuttcap%
\pgfsetroundjoin%
\definecolor{currentfill}{rgb}{0.121569,0.466667,0.705882}%
\pgfsetfillcolor{currentfill}%
\pgfsetfillopacity{0.314669}%
\pgfsetlinewidth{1.003750pt}%
\definecolor{currentstroke}{rgb}{0.121569,0.466667,0.705882}%
\pgfsetstrokecolor{currentstroke}%
\pgfsetstrokeopacity{0.314669}%
\pgfsetdash{}{0pt}%
\pgfpathmoveto{\pgfqpoint{1.803623in}{2.526293in}}%
\pgfpathcurveto{\pgfqpoint{1.811859in}{2.526293in}}{\pgfqpoint{1.819759in}{2.529565in}}{\pgfqpoint{1.825583in}{2.535389in}}%
\pgfpathcurveto{\pgfqpoint{1.831407in}{2.541213in}}{\pgfqpoint{1.834679in}{2.549113in}}{\pgfqpoint{1.834679in}{2.557349in}}%
\pgfpathcurveto{\pgfqpoint{1.834679in}{2.565585in}}{\pgfqpoint{1.831407in}{2.573485in}}{\pgfqpoint{1.825583in}{2.579309in}}%
\pgfpathcurveto{\pgfqpoint{1.819759in}{2.585133in}}{\pgfqpoint{1.811859in}{2.588406in}}{\pgfqpoint{1.803623in}{2.588406in}}%
\pgfpathcurveto{\pgfqpoint{1.795387in}{2.588406in}}{\pgfqpoint{1.787487in}{2.585133in}}{\pgfqpoint{1.781663in}{2.579309in}}%
\pgfpathcurveto{\pgfqpoint{1.775839in}{2.573485in}}{\pgfqpoint{1.772566in}{2.565585in}}{\pgfqpoint{1.772566in}{2.557349in}}%
\pgfpathcurveto{\pgfqpoint{1.772566in}{2.549113in}}{\pgfqpoint{1.775839in}{2.541213in}}{\pgfqpoint{1.781663in}{2.535389in}}%
\pgfpathcurveto{\pgfqpoint{1.787487in}{2.529565in}}{\pgfqpoint{1.795387in}{2.526293in}}{\pgfqpoint{1.803623in}{2.526293in}}%
\pgfpathclose%
\pgfusepath{stroke,fill}%
\end{pgfscope}%
\begin{pgfscope}%
\pgfpathrectangle{\pgfqpoint{0.100000in}{0.212622in}}{\pgfqpoint{3.696000in}{3.696000in}}%
\pgfusepath{clip}%
\pgfsetbuttcap%
\pgfsetroundjoin%
\definecolor{currentfill}{rgb}{0.121569,0.466667,0.705882}%
\pgfsetfillcolor{currentfill}%
\pgfsetfillopacity{0.314705}%
\pgfsetlinewidth{1.003750pt}%
\definecolor{currentstroke}{rgb}{0.121569,0.466667,0.705882}%
\pgfsetstrokecolor{currentstroke}%
\pgfsetstrokeopacity{0.314705}%
\pgfsetdash{}{0pt}%
\pgfpathmoveto{\pgfqpoint{1.593276in}{2.522079in}}%
\pgfpathcurveto{\pgfqpoint{1.601512in}{2.522079in}}{\pgfqpoint{1.609412in}{2.525352in}}{\pgfqpoint{1.615236in}{2.531176in}}%
\pgfpathcurveto{\pgfqpoint{1.621060in}{2.537000in}}{\pgfqpoint{1.624333in}{2.544900in}}{\pgfqpoint{1.624333in}{2.553136in}}%
\pgfpathcurveto{\pgfqpoint{1.624333in}{2.561372in}}{\pgfqpoint{1.621060in}{2.569272in}}{\pgfqpoint{1.615236in}{2.575096in}}%
\pgfpathcurveto{\pgfqpoint{1.609412in}{2.580920in}}{\pgfqpoint{1.601512in}{2.584192in}}{\pgfqpoint{1.593276in}{2.584192in}}%
\pgfpathcurveto{\pgfqpoint{1.585040in}{2.584192in}}{\pgfqpoint{1.577140in}{2.580920in}}{\pgfqpoint{1.571316in}{2.575096in}}%
\pgfpathcurveto{\pgfqpoint{1.565492in}{2.569272in}}{\pgfqpoint{1.562220in}{2.561372in}}{\pgfqpoint{1.562220in}{2.553136in}}%
\pgfpathcurveto{\pgfqpoint{1.562220in}{2.544900in}}{\pgfqpoint{1.565492in}{2.537000in}}{\pgfqpoint{1.571316in}{2.531176in}}%
\pgfpathcurveto{\pgfqpoint{1.577140in}{2.525352in}}{\pgfqpoint{1.585040in}{2.522079in}}{\pgfqpoint{1.593276in}{2.522079in}}%
\pgfpathclose%
\pgfusepath{stroke,fill}%
\end{pgfscope}%
\begin{pgfscope}%
\pgfpathrectangle{\pgfqpoint{0.100000in}{0.212622in}}{\pgfqpoint{3.696000in}{3.696000in}}%
\pgfusepath{clip}%
\pgfsetbuttcap%
\pgfsetroundjoin%
\definecolor{currentfill}{rgb}{0.121569,0.466667,0.705882}%
\pgfsetfillcolor{currentfill}%
\pgfsetfillopacity{0.316054}%
\pgfsetlinewidth{1.003750pt}%
\definecolor{currentstroke}{rgb}{0.121569,0.466667,0.705882}%
\pgfsetstrokecolor{currentstroke}%
\pgfsetstrokeopacity{0.316054}%
\pgfsetdash{}{0pt}%
\pgfpathmoveto{\pgfqpoint{1.590363in}{2.516981in}}%
\pgfpathcurveto{\pgfqpoint{1.598599in}{2.516981in}}{\pgfqpoint{1.606499in}{2.520253in}}{\pgfqpoint{1.612323in}{2.526077in}}%
\pgfpathcurveto{\pgfqpoint{1.618147in}{2.531901in}}{\pgfqpoint{1.621420in}{2.539801in}}{\pgfqpoint{1.621420in}{2.548038in}}%
\pgfpathcurveto{\pgfqpoint{1.621420in}{2.556274in}}{\pgfqpoint{1.618147in}{2.564174in}}{\pgfqpoint{1.612323in}{2.569998in}}%
\pgfpathcurveto{\pgfqpoint{1.606499in}{2.575822in}}{\pgfqpoint{1.598599in}{2.579094in}}{\pgfqpoint{1.590363in}{2.579094in}}%
\pgfpathcurveto{\pgfqpoint{1.582127in}{2.579094in}}{\pgfqpoint{1.574227in}{2.575822in}}{\pgfqpoint{1.568403in}{2.569998in}}%
\pgfpathcurveto{\pgfqpoint{1.562579in}{2.564174in}}{\pgfqpoint{1.559307in}{2.556274in}}{\pgfqpoint{1.559307in}{2.548038in}}%
\pgfpathcurveto{\pgfqpoint{1.559307in}{2.539801in}}{\pgfqpoint{1.562579in}{2.531901in}}{\pgfqpoint{1.568403in}{2.526077in}}%
\pgfpathcurveto{\pgfqpoint{1.574227in}{2.520253in}}{\pgfqpoint{1.582127in}{2.516981in}}{\pgfqpoint{1.590363in}{2.516981in}}%
\pgfpathclose%
\pgfusepath{stroke,fill}%
\end{pgfscope}%
\begin{pgfscope}%
\pgfpathrectangle{\pgfqpoint{0.100000in}{0.212622in}}{\pgfqpoint{3.696000in}{3.696000in}}%
\pgfusepath{clip}%
\pgfsetbuttcap%
\pgfsetroundjoin%
\definecolor{currentfill}{rgb}{0.121569,0.466667,0.705882}%
\pgfsetfillcolor{currentfill}%
\pgfsetfillopacity{0.317056}%
\pgfsetlinewidth{1.003750pt}%
\definecolor{currentstroke}{rgb}{0.121569,0.466667,0.705882}%
\pgfsetstrokecolor{currentstroke}%
\pgfsetstrokeopacity{0.317056}%
\pgfsetdash{}{0pt}%
\pgfpathmoveto{\pgfqpoint{1.587876in}{2.513028in}}%
\pgfpathcurveto{\pgfqpoint{1.596112in}{2.513028in}}{\pgfqpoint{1.604012in}{2.516300in}}{\pgfqpoint{1.609836in}{2.522124in}}%
\pgfpathcurveto{\pgfqpoint{1.615660in}{2.527948in}}{\pgfqpoint{1.618932in}{2.535848in}}{\pgfqpoint{1.618932in}{2.544085in}}%
\pgfpathcurveto{\pgfqpoint{1.618932in}{2.552321in}}{\pgfqpoint{1.615660in}{2.560221in}}{\pgfqpoint{1.609836in}{2.566045in}}%
\pgfpathcurveto{\pgfqpoint{1.604012in}{2.571869in}}{\pgfqpoint{1.596112in}{2.575141in}}{\pgfqpoint{1.587876in}{2.575141in}}%
\pgfpathcurveto{\pgfqpoint{1.579640in}{2.575141in}}{\pgfqpoint{1.571740in}{2.571869in}}{\pgfqpoint{1.565916in}{2.566045in}}%
\pgfpathcurveto{\pgfqpoint{1.560092in}{2.560221in}}{\pgfqpoint{1.556819in}{2.552321in}}{\pgfqpoint{1.556819in}{2.544085in}}%
\pgfpathcurveto{\pgfqpoint{1.556819in}{2.535848in}}{\pgfqpoint{1.560092in}{2.527948in}}{\pgfqpoint{1.565916in}{2.522124in}}%
\pgfpathcurveto{\pgfqpoint{1.571740in}{2.516300in}}{\pgfqpoint{1.579640in}{2.513028in}}{\pgfqpoint{1.587876in}{2.513028in}}%
\pgfpathclose%
\pgfusepath{stroke,fill}%
\end{pgfscope}%
\begin{pgfscope}%
\pgfpathrectangle{\pgfqpoint{0.100000in}{0.212622in}}{\pgfqpoint{3.696000in}{3.696000in}}%
\pgfusepath{clip}%
\pgfsetbuttcap%
\pgfsetroundjoin%
\definecolor{currentfill}{rgb}{0.121569,0.466667,0.705882}%
\pgfsetfillcolor{currentfill}%
\pgfsetfillopacity{0.317057}%
\pgfsetlinewidth{1.003750pt}%
\definecolor{currentstroke}{rgb}{0.121569,0.466667,0.705882}%
\pgfsetstrokecolor{currentstroke}%
\pgfsetstrokeopacity{0.317057}%
\pgfsetdash{}{0pt}%
\pgfpathmoveto{\pgfqpoint{1.819160in}{2.525293in}}%
\pgfpathcurveto{\pgfqpoint{1.827397in}{2.525293in}}{\pgfqpoint{1.835297in}{2.528565in}}{\pgfqpoint{1.841121in}{2.534389in}}%
\pgfpathcurveto{\pgfqpoint{1.846945in}{2.540213in}}{\pgfqpoint{1.850217in}{2.548113in}}{\pgfqpoint{1.850217in}{2.556350in}}%
\pgfpathcurveto{\pgfqpoint{1.850217in}{2.564586in}}{\pgfqpoint{1.846945in}{2.572486in}}{\pgfqpoint{1.841121in}{2.578310in}}%
\pgfpathcurveto{\pgfqpoint{1.835297in}{2.584134in}}{\pgfqpoint{1.827397in}{2.587406in}}{\pgfqpoint{1.819160in}{2.587406in}}%
\pgfpathcurveto{\pgfqpoint{1.810924in}{2.587406in}}{\pgfqpoint{1.803024in}{2.584134in}}{\pgfqpoint{1.797200in}{2.578310in}}%
\pgfpathcurveto{\pgfqpoint{1.791376in}{2.572486in}}{\pgfqpoint{1.788104in}{2.564586in}}{\pgfqpoint{1.788104in}{2.556350in}}%
\pgfpathcurveto{\pgfqpoint{1.788104in}{2.548113in}}{\pgfqpoint{1.791376in}{2.540213in}}{\pgfqpoint{1.797200in}{2.534389in}}%
\pgfpathcurveto{\pgfqpoint{1.803024in}{2.528565in}}{\pgfqpoint{1.810924in}{2.525293in}}{\pgfqpoint{1.819160in}{2.525293in}}%
\pgfpathclose%
\pgfusepath{stroke,fill}%
\end{pgfscope}%
\begin{pgfscope}%
\pgfpathrectangle{\pgfqpoint{0.100000in}{0.212622in}}{\pgfqpoint{3.696000in}{3.696000in}}%
\pgfusepath{clip}%
\pgfsetbuttcap%
\pgfsetroundjoin%
\definecolor{currentfill}{rgb}{0.121569,0.466667,0.705882}%
\pgfsetfillcolor{currentfill}%
\pgfsetfillopacity{0.317733}%
\pgfsetlinewidth{1.003750pt}%
\definecolor{currentstroke}{rgb}{0.121569,0.466667,0.705882}%
\pgfsetstrokecolor{currentstroke}%
\pgfsetstrokeopacity{0.317733}%
\pgfsetdash{}{0pt}%
\pgfpathmoveto{\pgfqpoint{1.586298in}{2.510387in}}%
\pgfpathcurveto{\pgfqpoint{1.594534in}{2.510387in}}{\pgfqpoint{1.602434in}{2.513660in}}{\pgfqpoint{1.608258in}{2.519484in}}%
\pgfpathcurveto{\pgfqpoint{1.614082in}{2.525308in}}{\pgfqpoint{1.617354in}{2.533208in}}{\pgfqpoint{1.617354in}{2.541444in}}%
\pgfpathcurveto{\pgfqpoint{1.617354in}{2.549680in}}{\pgfqpoint{1.614082in}{2.557580in}}{\pgfqpoint{1.608258in}{2.563404in}}%
\pgfpathcurveto{\pgfqpoint{1.602434in}{2.569228in}}{\pgfqpoint{1.594534in}{2.572500in}}{\pgfqpoint{1.586298in}{2.572500in}}%
\pgfpathcurveto{\pgfqpoint{1.578061in}{2.572500in}}{\pgfqpoint{1.570161in}{2.569228in}}{\pgfqpoint{1.564337in}{2.563404in}}%
\pgfpathcurveto{\pgfqpoint{1.558513in}{2.557580in}}{\pgfqpoint{1.555241in}{2.549680in}}{\pgfqpoint{1.555241in}{2.541444in}}%
\pgfpathcurveto{\pgfqpoint{1.555241in}{2.533208in}}{\pgfqpoint{1.558513in}{2.525308in}}{\pgfqpoint{1.564337in}{2.519484in}}%
\pgfpathcurveto{\pgfqpoint{1.570161in}{2.513660in}}{\pgfqpoint{1.578061in}{2.510387in}}{\pgfqpoint{1.586298in}{2.510387in}}%
\pgfpathclose%
\pgfusepath{stroke,fill}%
\end{pgfscope}%
\begin{pgfscope}%
\pgfpathrectangle{\pgfqpoint{0.100000in}{0.212622in}}{\pgfqpoint{3.696000in}{3.696000in}}%
\pgfusepath{clip}%
\pgfsetbuttcap%
\pgfsetroundjoin%
\definecolor{currentfill}{rgb}{0.121569,0.466667,0.705882}%
\pgfsetfillcolor{currentfill}%
\pgfsetfillopacity{0.318320}%
\pgfsetlinewidth{1.003750pt}%
\definecolor{currentstroke}{rgb}{0.121569,0.466667,0.705882}%
\pgfsetstrokecolor{currentstroke}%
\pgfsetstrokeopacity{0.318320}%
\pgfsetdash{}{0pt}%
\pgfpathmoveto{\pgfqpoint{1.584881in}{2.508125in}}%
\pgfpathcurveto{\pgfqpoint{1.593117in}{2.508125in}}{\pgfqpoint{1.601017in}{2.511398in}}{\pgfqpoint{1.606841in}{2.517222in}}%
\pgfpathcurveto{\pgfqpoint{1.612665in}{2.523045in}}{\pgfqpoint{1.615937in}{2.530946in}}{\pgfqpoint{1.615937in}{2.539182in}}%
\pgfpathcurveto{\pgfqpoint{1.615937in}{2.547418in}}{\pgfqpoint{1.612665in}{2.555318in}}{\pgfqpoint{1.606841in}{2.561142in}}%
\pgfpathcurveto{\pgfqpoint{1.601017in}{2.566966in}}{\pgfqpoint{1.593117in}{2.570238in}}{\pgfqpoint{1.584881in}{2.570238in}}%
\pgfpathcurveto{\pgfqpoint{1.576644in}{2.570238in}}{\pgfqpoint{1.568744in}{2.566966in}}{\pgfqpoint{1.562920in}{2.561142in}}%
\pgfpathcurveto{\pgfqpoint{1.557097in}{2.555318in}}{\pgfqpoint{1.553824in}{2.547418in}}{\pgfqpoint{1.553824in}{2.539182in}}%
\pgfpathcurveto{\pgfqpoint{1.553824in}{2.530946in}}{\pgfqpoint{1.557097in}{2.523045in}}{\pgfqpoint{1.562920in}{2.517222in}}%
\pgfpathcurveto{\pgfqpoint{1.568744in}{2.511398in}}{\pgfqpoint{1.576644in}{2.508125in}}{\pgfqpoint{1.584881in}{2.508125in}}%
\pgfpathclose%
\pgfusepath{stroke,fill}%
\end{pgfscope}%
\begin{pgfscope}%
\pgfpathrectangle{\pgfqpoint{0.100000in}{0.212622in}}{\pgfqpoint{3.696000in}{3.696000in}}%
\pgfusepath{clip}%
\pgfsetbuttcap%
\pgfsetroundjoin%
\definecolor{currentfill}{rgb}{0.121569,0.466667,0.705882}%
\pgfsetfillcolor{currentfill}%
\pgfsetfillopacity{0.319404}%
\pgfsetlinewidth{1.003750pt}%
\definecolor{currentstroke}{rgb}{0.121569,0.466667,0.705882}%
\pgfsetstrokecolor{currentstroke}%
\pgfsetstrokeopacity{0.319404}%
\pgfsetdash{}{0pt}%
\pgfpathmoveto{\pgfqpoint{1.582353in}{2.504074in}}%
\pgfpathcurveto{\pgfqpoint{1.590590in}{2.504074in}}{\pgfqpoint{1.598490in}{2.507346in}}{\pgfqpoint{1.604314in}{2.513170in}}%
\pgfpathcurveto{\pgfqpoint{1.610138in}{2.518994in}}{\pgfqpoint{1.613410in}{2.526894in}}{\pgfqpoint{1.613410in}{2.535130in}}%
\pgfpathcurveto{\pgfqpoint{1.613410in}{2.543367in}}{\pgfqpoint{1.610138in}{2.551267in}}{\pgfqpoint{1.604314in}{2.557091in}}%
\pgfpathcurveto{\pgfqpoint{1.598490in}{2.562914in}}{\pgfqpoint{1.590590in}{2.566187in}}{\pgfqpoint{1.582353in}{2.566187in}}%
\pgfpathcurveto{\pgfqpoint{1.574117in}{2.566187in}}{\pgfqpoint{1.566217in}{2.562914in}}{\pgfqpoint{1.560393in}{2.557091in}}%
\pgfpathcurveto{\pgfqpoint{1.554569in}{2.551267in}}{\pgfqpoint{1.551297in}{2.543367in}}{\pgfqpoint{1.551297in}{2.535130in}}%
\pgfpathcurveto{\pgfqpoint{1.551297in}{2.526894in}}{\pgfqpoint{1.554569in}{2.518994in}}{\pgfqpoint{1.560393in}{2.513170in}}%
\pgfpathcurveto{\pgfqpoint{1.566217in}{2.507346in}}{\pgfqpoint{1.574117in}{2.504074in}}{\pgfqpoint{1.582353in}{2.504074in}}%
\pgfpathclose%
\pgfusepath{stroke,fill}%
\end{pgfscope}%
\begin{pgfscope}%
\pgfpathrectangle{\pgfqpoint{0.100000in}{0.212622in}}{\pgfqpoint{3.696000in}{3.696000in}}%
\pgfusepath{clip}%
\pgfsetbuttcap%
\pgfsetroundjoin%
\definecolor{currentfill}{rgb}{0.121569,0.466667,0.705882}%
\pgfsetfillcolor{currentfill}%
\pgfsetfillopacity{0.319414}%
\pgfsetlinewidth{1.003750pt}%
\definecolor{currentstroke}{rgb}{0.121569,0.466667,0.705882}%
\pgfsetstrokecolor{currentstroke}%
\pgfsetstrokeopacity{0.319414}%
\pgfsetdash{}{0pt}%
\pgfpathmoveto{\pgfqpoint{1.835147in}{2.523140in}}%
\pgfpathcurveto{\pgfqpoint{1.843383in}{2.523140in}}{\pgfqpoint{1.851284in}{2.526412in}}{\pgfqpoint{1.857107in}{2.532236in}}%
\pgfpathcurveto{\pgfqpoint{1.862931in}{2.538060in}}{\pgfqpoint{1.866204in}{2.545960in}}{\pgfqpoint{1.866204in}{2.554196in}}%
\pgfpathcurveto{\pgfqpoint{1.866204in}{2.562432in}}{\pgfqpoint{1.862931in}{2.570332in}}{\pgfqpoint{1.857107in}{2.576156in}}%
\pgfpathcurveto{\pgfqpoint{1.851284in}{2.581980in}}{\pgfqpoint{1.843383in}{2.585253in}}{\pgfqpoint{1.835147in}{2.585253in}}%
\pgfpathcurveto{\pgfqpoint{1.826911in}{2.585253in}}{\pgfqpoint{1.819011in}{2.581980in}}{\pgfqpoint{1.813187in}{2.576156in}}%
\pgfpathcurveto{\pgfqpoint{1.807363in}{2.570332in}}{\pgfqpoint{1.804091in}{2.562432in}}{\pgfqpoint{1.804091in}{2.554196in}}%
\pgfpathcurveto{\pgfqpoint{1.804091in}{2.545960in}}{\pgfqpoint{1.807363in}{2.538060in}}{\pgfqpoint{1.813187in}{2.532236in}}%
\pgfpathcurveto{\pgfqpoint{1.819011in}{2.526412in}}{\pgfqpoint{1.826911in}{2.523140in}}{\pgfqpoint{1.835147in}{2.523140in}}%
\pgfpathclose%
\pgfusepath{stroke,fill}%
\end{pgfscope}%
\begin{pgfscope}%
\pgfpathrectangle{\pgfqpoint{0.100000in}{0.212622in}}{\pgfqpoint{3.696000in}{3.696000in}}%
\pgfusepath{clip}%
\pgfsetbuttcap%
\pgfsetroundjoin%
\definecolor{currentfill}{rgb}{0.121569,0.466667,0.705882}%
\pgfsetfillcolor{currentfill}%
\pgfsetfillopacity{0.320409}%
\pgfsetlinewidth{1.003750pt}%
\definecolor{currentstroke}{rgb}{0.121569,0.466667,0.705882}%
\pgfsetstrokecolor{currentstroke}%
\pgfsetstrokeopacity{0.320409}%
\pgfsetdash{}{0pt}%
\pgfpathmoveto{\pgfqpoint{1.580206in}{2.500248in}}%
\pgfpathcurveto{\pgfqpoint{1.588443in}{2.500248in}}{\pgfqpoint{1.596343in}{2.503520in}}{\pgfqpoint{1.602167in}{2.509344in}}%
\pgfpathcurveto{\pgfqpoint{1.607991in}{2.515168in}}{\pgfqpoint{1.611263in}{2.523068in}}{\pgfqpoint{1.611263in}{2.531305in}}%
\pgfpathcurveto{\pgfqpoint{1.611263in}{2.539541in}}{\pgfqpoint{1.607991in}{2.547441in}}{\pgfqpoint{1.602167in}{2.553265in}}%
\pgfpathcurveto{\pgfqpoint{1.596343in}{2.559089in}}{\pgfqpoint{1.588443in}{2.562361in}}{\pgfqpoint{1.580206in}{2.562361in}}%
\pgfpathcurveto{\pgfqpoint{1.571970in}{2.562361in}}{\pgfqpoint{1.564070in}{2.559089in}}{\pgfqpoint{1.558246in}{2.553265in}}%
\pgfpathcurveto{\pgfqpoint{1.552422in}{2.547441in}}{\pgfqpoint{1.549150in}{2.539541in}}{\pgfqpoint{1.549150in}{2.531305in}}%
\pgfpathcurveto{\pgfqpoint{1.549150in}{2.523068in}}{\pgfqpoint{1.552422in}{2.515168in}}{\pgfqpoint{1.558246in}{2.509344in}}%
\pgfpathcurveto{\pgfqpoint{1.564070in}{2.503520in}}{\pgfqpoint{1.571970in}{2.500248in}}{\pgfqpoint{1.580206in}{2.500248in}}%
\pgfpathclose%
\pgfusepath{stroke,fill}%
\end{pgfscope}%
\begin{pgfscope}%
\pgfpathrectangle{\pgfqpoint{0.100000in}{0.212622in}}{\pgfqpoint{3.696000in}{3.696000in}}%
\pgfusepath{clip}%
\pgfsetbuttcap%
\pgfsetroundjoin%
\definecolor{currentfill}{rgb}{0.121569,0.466667,0.705882}%
\pgfsetfillcolor{currentfill}%
\pgfsetfillopacity{0.321152}%
\pgfsetlinewidth{1.003750pt}%
\definecolor{currentstroke}{rgb}{0.121569,0.466667,0.705882}%
\pgfsetstrokecolor{currentstroke}%
\pgfsetstrokeopacity{0.321152}%
\pgfsetdash{}{0pt}%
\pgfpathmoveto{\pgfqpoint{1.578553in}{2.497405in}}%
\pgfpathcurveto{\pgfqpoint{1.586789in}{2.497405in}}{\pgfqpoint{1.594689in}{2.500677in}}{\pgfqpoint{1.600513in}{2.506501in}}%
\pgfpathcurveto{\pgfqpoint{1.606337in}{2.512325in}}{\pgfqpoint{1.609609in}{2.520225in}}{\pgfqpoint{1.609609in}{2.528461in}}%
\pgfpathcurveto{\pgfqpoint{1.609609in}{2.536698in}}{\pgfqpoint{1.606337in}{2.544598in}}{\pgfqpoint{1.600513in}{2.550422in}}%
\pgfpathcurveto{\pgfqpoint{1.594689in}{2.556246in}}{\pgfqpoint{1.586789in}{2.559518in}}{\pgfqpoint{1.578553in}{2.559518in}}%
\pgfpathcurveto{\pgfqpoint{1.570317in}{2.559518in}}{\pgfqpoint{1.562416in}{2.556246in}}{\pgfqpoint{1.556593in}{2.550422in}}%
\pgfpathcurveto{\pgfqpoint{1.550769in}{2.544598in}}{\pgfqpoint{1.547496in}{2.536698in}}{\pgfqpoint{1.547496in}{2.528461in}}%
\pgfpathcurveto{\pgfqpoint{1.547496in}{2.520225in}}{\pgfqpoint{1.550769in}{2.512325in}}{\pgfqpoint{1.556593in}{2.506501in}}%
\pgfpathcurveto{\pgfqpoint{1.562416in}{2.500677in}}{\pgfqpoint{1.570317in}{2.497405in}}{\pgfqpoint{1.578553in}{2.497405in}}%
\pgfpathclose%
\pgfusepath{stroke,fill}%
\end{pgfscope}%
\begin{pgfscope}%
\pgfpathrectangle{\pgfqpoint{0.100000in}{0.212622in}}{\pgfqpoint{3.696000in}{3.696000in}}%
\pgfusepath{clip}%
\pgfsetbuttcap%
\pgfsetroundjoin%
\definecolor{currentfill}{rgb}{0.121569,0.466667,0.705882}%
\pgfsetfillcolor{currentfill}%
\pgfsetfillopacity{0.321774}%
\pgfsetlinewidth{1.003750pt}%
\definecolor{currentstroke}{rgb}{0.121569,0.466667,0.705882}%
\pgfsetstrokecolor{currentstroke}%
\pgfsetstrokeopacity{0.321774}%
\pgfsetdash{}{0pt}%
\pgfpathmoveto{\pgfqpoint{1.577209in}{2.495120in}}%
\pgfpathcurveto{\pgfqpoint{1.585446in}{2.495120in}}{\pgfqpoint{1.593346in}{2.498393in}}{\pgfqpoint{1.599170in}{2.504217in}}%
\pgfpathcurveto{\pgfqpoint{1.604993in}{2.510041in}}{\pgfqpoint{1.608266in}{2.517941in}}{\pgfqpoint{1.608266in}{2.526177in}}%
\pgfpathcurveto{\pgfqpoint{1.608266in}{2.534413in}}{\pgfqpoint{1.604993in}{2.542313in}}{\pgfqpoint{1.599170in}{2.548137in}}%
\pgfpathcurveto{\pgfqpoint{1.593346in}{2.553961in}}{\pgfqpoint{1.585446in}{2.557233in}}{\pgfqpoint{1.577209in}{2.557233in}}%
\pgfpathcurveto{\pgfqpoint{1.568973in}{2.557233in}}{\pgfqpoint{1.561073in}{2.553961in}}{\pgfqpoint{1.555249in}{2.548137in}}%
\pgfpathcurveto{\pgfqpoint{1.549425in}{2.542313in}}{\pgfqpoint{1.546153in}{2.534413in}}{\pgfqpoint{1.546153in}{2.526177in}}%
\pgfpathcurveto{\pgfqpoint{1.546153in}{2.517941in}}{\pgfqpoint{1.549425in}{2.510041in}}{\pgfqpoint{1.555249in}{2.504217in}}%
\pgfpathcurveto{\pgfqpoint{1.561073in}{2.498393in}}{\pgfqpoint{1.568973in}{2.495120in}}{\pgfqpoint{1.577209in}{2.495120in}}%
\pgfpathclose%
\pgfusepath{stroke,fill}%
\end{pgfscope}%
\begin{pgfscope}%
\pgfpathrectangle{\pgfqpoint{0.100000in}{0.212622in}}{\pgfqpoint{3.696000in}{3.696000in}}%
\pgfusepath{clip}%
\pgfsetbuttcap%
\pgfsetroundjoin%
\definecolor{currentfill}{rgb}{0.121569,0.466667,0.705882}%
\pgfsetfillcolor{currentfill}%
\pgfsetfillopacity{0.322186}%
\pgfsetlinewidth{1.003750pt}%
\definecolor{currentstroke}{rgb}{0.121569,0.466667,0.705882}%
\pgfsetstrokecolor{currentstroke}%
\pgfsetstrokeopacity{0.322186}%
\pgfsetdash{}{0pt}%
\pgfpathmoveto{\pgfqpoint{1.853172in}{2.520005in}}%
\pgfpathcurveto{\pgfqpoint{1.861409in}{2.520005in}}{\pgfqpoint{1.869309in}{2.523278in}}{\pgfqpoint{1.875133in}{2.529102in}}%
\pgfpathcurveto{\pgfqpoint{1.880957in}{2.534926in}}{\pgfqpoint{1.884229in}{2.542826in}}{\pgfqpoint{1.884229in}{2.551062in}}%
\pgfpathcurveto{\pgfqpoint{1.884229in}{2.559298in}}{\pgfqpoint{1.880957in}{2.567198in}}{\pgfqpoint{1.875133in}{2.573022in}}%
\pgfpathcurveto{\pgfqpoint{1.869309in}{2.578846in}}{\pgfqpoint{1.861409in}{2.582118in}}{\pgfqpoint{1.853172in}{2.582118in}}%
\pgfpathcurveto{\pgfqpoint{1.844936in}{2.582118in}}{\pgfqpoint{1.837036in}{2.578846in}}{\pgfqpoint{1.831212in}{2.573022in}}%
\pgfpathcurveto{\pgfqpoint{1.825388in}{2.567198in}}{\pgfqpoint{1.822116in}{2.559298in}}{\pgfqpoint{1.822116in}{2.551062in}}%
\pgfpathcurveto{\pgfqpoint{1.822116in}{2.542826in}}{\pgfqpoint{1.825388in}{2.534926in}}{\pgfqpoint{1.831212in}{2.529102in}}%
\pgfpathcurveto{\pgfqpoint{1.837036in}{2.523278in}}{\pgfqpoint{1.844936in}{2.520005in}}{\pgfqpoint{1.853172in}{2.520005in}}%
\pgfpathclose%
\pgfusepath{stroke,fill}%
\end{pgfscope}%
\begin{pgfscope}%
\pgfpathrectangle{\pgfqpoint{0.100000in}{0.212622in}}{\pgfqpoint{3.696000in}{3.696000in}}%
\pgfusepath{clip}%
\pgfsetbuttcap%
\pgfsetroundjoin%
\definecolor{currentfill}{rgb}{0.121569,0.466667,0.705882}%
\pgfsetfillcolor{currentfill}%
\pgfsetfillopacity{0.322283}%
\pgfsetlinewidth{1.003750pt}%
\definecolor{currentstroke}{rgb}{0.121569,0.466667,0.705882}%
\pgfsetstrokecolor{currentstroke}%
\pgfsetstrokeopacity{0.322283}%
\pgfsetdash{}{0pt}%
\pgfpathmoveto{\pgfqpoint{1.576068in}{2.493240in}}%
\pgfpathcurveto{\pgfqpoint{1.584304in}{2.493240in}}{\pgfqpoint{1.592204in}{2.496512in}}{\pgfqpoint{1.598028in}{2.502336in}}%
\pgfpathcurveto{\pgfqpoint{1.603852in}{2.508160in}}{\pgfqpoint{1.607124in}{2.516060in}}{\pgfqpoint{1.607124in}{2.524296in}}%
\pgfpathcurveto{\pgfqpoint{1.607124in}{2.532533in}}{\pgfqpoint{1.603852in}{2.540433in}}{\pgfqpoint{1.598028in}{2.546257in}}%
\pgfpathcurveto{\pgfqpoint{1.592204in}{2.552081in}}{\pgfqpoint{1.584304in}{2.555353in}}{\pgfqpoint{1.576068in}{2.555353in}}%
\pgfpathcurveto{\pgfqpoint{1.567831in}{2.555353in}}{\pgfqpoint{1.559931in}{2.552081in}}{\pgfqpoint{1.554108in}{2.546257in}}%
\pgfpathcurveto{\pgfqpoint{1.548284in}{2.540433in}}{\pgfqpoint{1.545011in}{2.532533in}}{\pgfqpoint{1.545011in}{2.524296in}}%
\pgfpathcurveto{\pgfqpoint{1.545011in}{2.516060in}}{\pgfqpoint{1.548284in}{2.508160in}}{\pgfqpoint{1.554108in}{2.502336in}}%
\pgfpathcurveto{\pgfqpoint{1.559931in}{2.496512in}}{\pgfqpoint{1.567831in}{2.493240in}}{\pgfqpoint{1.576068in}{2.493240in}}%
\pgfpathclose%
\pgfusepath{stroke,fill}%
\end{pgfscope}%
\begin{pgfscope}%
\pgfpathrectangle{\pgfqpoint{0.100000in}{0.212622in}}{\pgfqpoint{3.696000in}{3.696000in}}%
\pgfusepath{clip}%
\pgfsetbuttcap%
\pgfsetroundjoin%
\definecolor{currentfill}{rgb}{0.121569,0.466667,0.705882}%
\pgfsetfillcolor{currentfill}%
\pgfsetfillopacity{0.322387}%
\pgfsetlinewidth{1.003750pt}%
\definecolor{currentstroke}{rgb}{0.121569,0.466667,0.705882}%
\pgfsetstrokecolor{currentstroke}%
\pgfsetstrokeopacity{0.322387}%
\pgfsetdash{}{0pt}%
\pgfpathmoveto{\pgfqpoint{1.575801in}{2.492843in}}%
\pgfpathcurveto{\pgfqpoint{1.584037in}{2.492843in}}{\pgfqpoint{1.591937in}{2.496116in}}{\pgfqpoint{1.597761in}{2.501940in}}%
\pgfpathcurveto{\pgfqpoint{1.603585in}{2.507764in}}{\pgfqpoint{1.606857in}{2.515664in}}{\pgfqpoint{1.606857in}{2.523900in}}%
\pgfpathcurveto{\pgfqpoint{1.606857in}{2.532136in}}{\pgfqpoint{1.603585in}{2.540036in}}{\pgfqpoint{1.597761in}{2.545860in}}%
\pgfpathcurveto{\pgfqpoint{1.591937in}{2.551684in}}{\pgfqpoint{1.584037in}{2.554956in}}{\pgfqpoint{1.575801in}{2.554956in}}%
\pgfpathcurveto{\pgfqpoint{1.567564in}{2.554956in}}{\pgfqpoint{1.559664in}{2.551684in}}{\pgfqpoint{1.553841in}{2.545860in}}%
\pgfpathcurveto{\pgfqpoint{1.548017in}{2.540036in}}{\pgfqpoint{1.544744in}{2.532136in}}{\pgfqpoint{1.544744in}{2.523900in}}%
\pgfpathcurveto{\pgfqpoint{1.544744in}{2.515664in}}{\pgfqpoint{1.548017in}{2.507764in}}{\pgfqpoint{1.553841in}{2.501940in}}%
\pgfpathcurveto{\pgfqpoint{1.559664in}{2.496116in}}{\pgfqpoint{1.567564in}{2.492843in}}{\pgfqpoint{1.575801in}{2.492843in}}%
\pgfpathclose%
\pgfusepath{stroke,fill}%
\end{pgfscope}%
\begin{pgfscope}%
\pgfpathrectangle{\pgfqpoint{0.100000in}{0.212622in}}{\pgfqpoint{3.696000in}{3.696000in}}%
\pgfusepath{clip}%
\pgfsetbuttcap%
\pgfsetroundjoin%
\definecolor{currentfill}{rgb}{0.121569,0.466667,0.705882}%
\pgfsetfillcolor{currentfill}%
\pgfsetfillopacity{0.322579}%
\pgfsetlinewidth{1.003750pt}%
\definecolor{currentstroke}{rgb}{0.121569,0.466667,0.705882}%
\pgfsetstrokecolor{currentstroke}%
\pgfsetstrokeopacity{0.322579}%
\pgfsetdash{}{0pt}%
\pgfpathmoveto{\pgfqpoint{1.575332in}{2.492128in}}%
\pgfpathcurveto{\pgfqpoint{1.583569in}{2.492128in}}{\pgfqpoint{1.591469in}{2.495400in}}{\pgfqpoint{1.597293in}{2.501224in}}%
\pgfpathcurveto{\pgfqpoint{1.603116in}{2.507048in}}{\pgfqpoint{1.606389in}{2.514948in}}{\pgfqpoint{1.606389in}{2.523185in}}%
\pgfpathcurveto{\pgfqpoint{1.606389in}{2.531421in}}{\pgfqpoint{1.603116in}{2.539321in}}{\pgfqpoint{1.597293in}{2.545145in}}%
\pgfpathcurveto{\pgfqpoint{1.591469in}{2.550969in}}{\pgfqpoint{1.583569in}{2.554241in}}{\pgfqpoint{1.575332in}{2.554241in}}%
\pgfpathcurveto{\pgfqpoint{1.567096in}{2.554241in}}{\pgfqpoint{1.559196in}{2.550969in}}{\pgfqpoint{1.553372in}{2.545145in}}%
\pgfpathcurveto{\pgfqpoint{1.547548in}{2.539321in}}{\pgfqpoint{1.544276in}{2.531421in}}{\pgfqpoint{1.544276in}{2.523185in}}%
\pgfpathcurveto{\pgfqpoint{1.544276in}{2.514948in}}{\pgfqpoint{1.547548in}{2.507048in}}{\pgfqpoint{1.553372in}{2.501224in}}%
\pgfpathcurveto{\pgfqpoint{1.559196in}{2.495400in}}{\pgfqpoint{1.567096in}{2.492128in}}{\pgfqpoint{1.575332in}{2.492128in}}%
\pgfpathclose%
\pgfusepath{stroke,fill}%
\end{pgfscope}%
\begin{pgfscope}%
\pgfpathrectangle{\pgfqpoint{0.100000in}{0.212622in}}{\pgfqpoint{3.696000in}{3.696000in}}%
\pgfusepath{clip}%
\pgfsetbuttcap%
\pgfsetroundjoin%
\definecolor{currentfill}{rgb}{0.121569,0.466667,0.705882}%
\pgfsetfillcolor{currentfill}%
\pgfsetfillopacity{0.322926}%
\pgfsetlinewidth{1.003750pt}%
\definecolor{currentstroke}{rgb}{0.121569,0.466667,0.705882}%
\pgfsetstrokecolor{currentstroke}%
\pgfsetstrokeopacity{0.322926}%
\pgfsetdash{}{0pt}%
\pgfpathmoveto{\pgfqpoint{1.574461in}{2.490835in}}%
\pgfpathcurveto{\pgfqpoint{1.582698in}{2.490835in}}{\pgfqpoint{1.590598in}{2.494107in}}{\pgfqpoint{1.596422in}{2.499931in}}%
\pgfpathcurveto{\pgfqpoint{1.602246in}{2.505755in}}{\pgfqpoint{1.605518in}{2.513655in}}{\pgfqpoint{1.605518in}{2.521892in}}%
\pgfpathcurveto{\pgfqpoint{1.605518in}{2.530128in}}{\pgfqpoint{1.602246in}{2.538028in}}{\pgfqpoint{1.596422in}{2.543852in}}%
\pgfpathcurveto{\pgfqpoint{1.590598in}{2.549676in}}{\pgfqpoint{1.582698in}{2.552948in}}{\pgfqpoint{1.574461in}{2.552948in}}%
\pgfpathcurveto{\pgfqpoint{1.566225in}{2.552948in}}{\pgfqpoint{1.558325in}{2.549676in}}{\pgfqpoint{1.552501in}{2.543852in}}%
\pgfpathcurveto{\pgfqpoint{1.546677in}{2.538028in}}{\pgfqpoint{1.543405in}{2.530128in}}{\pgfqpoint{1.543405in}{2.521892in}}%
\pgfpathcurveto{\pgfqpoint{1.543405in}{2.513655in}}{\pgfqpoint{1.546677in}{2.505755in}}{\pgfqpoint{1.552501in}{2.499931in}}%
\pgfpathcurveto{\pgfqpoint{1.558325in}{2.494107in}}{\pgfqpoint{1.566225in}{2.490835in}}{\pgfqpoint{1.574461in}{2.490835in}}%
\pgfpathclose%
\pgfusepath{stroke,fill}%
\end{pgfscope}%
\begin{pgfscope}%
\pgfpathrectangle{\pgfqpoint{0.100000in}{0.212622in}}{\pgfqpoint{3.696000in}{3.696000in}}%
\pgfusepath{clip}%
\pgfsetbuttcap%
\pgfsetroundjoin%
\definecolor{currentfill}{rgb}{0.121569,0.466667,0.705882}%
\pgfsetfillcolor{currentfill}%
\pgfsetfillopacity{0.323564}%
\pgfsetlinewidth{1.003750pt}%
\definecolor{currentstroke}{rgb}{0.121569,0.466667,0.705882}%
\pgfsetstrokecolor{currentstroke}%
\pgfsetstrokeopacity{0.323564}%
\pgfsetdash{}{0pt}%
\pgfpathmoveto{\pgfqpoint{1.572904in}{2.488512in}}%
\pgfpathcurveto{\pgfqpoint{1.581141in}{2.488512in}}{\pgfqpoint{1.589041in}{2.491784in}}{\pgfqpoint{1.594865in}{2.497608in}}%
\pgfpathcurveto{\pgfqpoint{1.600689in}{2.503432in}}{\pgfqpoint{1.603961in}{2.511332in}}{\pgfqpoint{1.603961in}{2.519568in}}%
\pgfpathcurveto{\pgfqpoint{1.603961in}{2.527804in}}{\pgfqpoint{1.600689in}{2.535704in}}{\pgfqpoint{1.594865in}{2.541528in}}%
\pgfpathcurveto{\pgfqpoint{1.589041in}{2.547352in}}{\pgfqpoint{1.581141in}{2.550625in}}{\pgfqpoint{1.572904in}{2.550625in}}%
\pgfpathcurveto{\pgfqpoint{1.564668in}{2.550625in}}{\pgfqpoint{1.556768in}{2.547352in}}{\pgfqpoint{1.550944in}{2.541528in}}%
\pgfpathcurveto{\pgfqpoint{1.545120in}{2.535704in}}{\pgfqpoint{1.541848in}{2.527804in}}{\pgfqpoint{1.541848in}{2.519568in}}%
\pgfpathcurveto{\pgfqpoint{1.541848in}{2.511332in}}{\pgfqpoint{1.545120in}{2.503432in}}{\pgfqpoint{1.550944in}{2.497608in}}%
\pgfpathcurveto{\pgfqpoint{1.556768in}{2.491784in}}{\pgfqpoint{1.564668in}{2.488512in}}{\pgfqpoint{1.572904in}{2.488512in}}%
\pgfpathclose%
\pgfusepath{stroke,fill}%
\end{pgfscope}%
\begin{pgfscope}%
\pgfpathrectangle{\pgfqpoint{0.100000in}{0.212622in}}{\pgfqpoint{3.696000in}{3.696000in}}%
\pgfusepath{clip}%
\pgfsetbuttcap%
\pgfsetroundjoin%
\definecolor{currentfill}{rgb}{0.121569,0.466667,0.705882}%
\pgfsetfillcolor{currentfill}%
\pgfsetfillopacity{0.323747}%
\pgfsetlinewidth{1.003750pt}%
\definecolor{currentstroke}{rgb}{0.121569,0.466667,0.705882}%
\pgfsetstrokecolor{currentstroke}%
\pgfsetstrokeopacity{0.323747}%
\pgfsetdash{}{0pt}%
\pgfpathmoveto{\pgfqpoint{1.863193in}{2.518843in}}%
\pgfpathcurveto{\pgfqpoint{1.871429in}{2.518843in}}{\pgfqpoint{1.879329in}{2.522115in}}{\pgfqpoint{1.885153in}{2.527939in}}%
\pgfpathcurveto{\pgfqpoint{1.890977in}{2.533763in}}{\pgfqpoint{1.894249in}{2.541663in}}{\pgfqpoint{1.894249in}{2.549899in}}%
\pgfpathcurveto{\pgfqpoint{1.894249in}{2.558136in}}{\pgfqpoint{1.890977in}{2.566036in}}{\pgfqpoint{1.885153in}{2.571860in}}%
\pgfpathcurveto{\pgfqpoint{1.879329in}{2.577684in}}{\pgfqpoint{1.871429in}{2.580956in}}{\pgfqpoint{1.863193in}{2.580956in}}%
\pgfpathcurveto{\pgfqpoint{1.854956in}{2.580956in}}{\pgfqpoint{1.847056in}{2.577684in}}{\pgfqpoint{1.841232in}{2.571860in}}%
\pgfpathcurveto{\pgfqpoint{1.835408in}{2.566036in}}{\pgfqpoint{1.832136in}{2.558136in}}{\pgfqpoint{1.832136in}{2.549899in}}%
\pgfpathcurveto{\pgfqpoint{1.832136in}{2.541663in}}{\pgfqpoint{1.835408in}{2.533763in}}{\pgfqpoint{1.841232in}{2.527939in}}%
\pgfpathcurveto{\pgfqpoint{1.847056in}{2.522115in}}{\pgfqpoint{1.854956in}{2.518843in}}{\pgfqpoint{1.863193in}{2.518843in}}%
\pgfpathclose%
\pgfusepath{stroke,fill}%
\end{pgfscope}%
\begin{pgfscope}%
\pgfpathrectangle{\pgfqpoint{0.100000in}{0.212622in}}{\pgfqpoint{3.696000in}{3.696000in}}%
\pgfusepath{clip}%
\pgfsetbuttcap%
\pgfsetroundjoin%
\definecolor{currentfill}{rgb}{0.121569,0.466667,0.705882}%
\pgfsetfillcolor{currentfill}%
\pgfsetfillopacity{0.323952}%
\pgfsetlinewidth{1.003750pt}%
\definecolor{currentstroke}{rgb}{0.121569,0.466667,0.705882}%
\pgfsetstrokecolor{currentstroke}%
\pgfsetstrokeopacity{0.323952}%
\pgfsetdash{}{0pt}%
\pgfpathmoveto{\pgfqpoint{1.572051in}{2.487158in}}%
\pgfpathcurveto{\pgfqpoint{1.580287in}{2.487158in}}{\pgfqpoint{1.588187in}{2.490431in}}{\pgfqpoint{1.594011in}{2.496255in}}%
\pgfpathcurveto{\pgfqpoint{1.599835in}{2.502078in}}{\pgfqpoint{1.603108in}{2.509979in}}{\pgfqpoint{1.603108in}{2.518215in}}%
\pgfpathcurveto{\pgfqpoint{1.603108in}{2.526451in}}{\pgfqpoint{1.599835in}{2.534351in}}{\pgfqpoint{1.594011in}{2.540175in}}%
\pgfpathcurveto{\pgfqpoint{1.588187in}{2.545999in}}{\pgfqpoint{1.580287in}{2.549271in}}{\pgfqpoint{1.572051in}{2.549271in}}%
\pgfpathcurveto{\pgfqpoint{1.563815in}{2.549271in}}{\pgfqpoint{1.555915in}{2.545999in}}{\pgfqpoint{1.550091in}{2.540175in}}%
\pgfpathcurveto{\pgfqpoint{1.544267in}{2.534351in}}{\pgfqpoint{1.540995in}{2.526451in}}{\pgfqpoint{1.540995in}{2.518215in}}%
\pgfpathcurveto{\pgfqpoint{1.540995in}{2.509979in}}{\pgfqpoint{1.544267in}{2.502078in}}{\pgfqpoint{1.550091in}{2.496255in}}%
\pgfpathcurveto{\pgfqpoint{1.555915in}{2.490431in}}{\pgfqpoint{1.563815in}{2.487158in}}{\pgfqpoint{1.572051in}{2.487158in}}%
\pgfpathclose%
\pgfusepath{stroke,fill}%
\end{pgfscope}%
\begin{pgfscope}%
\pgfpathrectangle{\pgfqpoint{0.100000in}{0.212622in}}{\pgfqpoint{3.696000in}{3.696000in}}%
\pgfusepath{clip}%
\pgfsetbuttcap%
\pgfsetroundjoin%
\definecolor{currentfill}{rgb}{0.121569,0.466667,0.705882}%
\pgfsetfillcolor{currentfill}%
\pgfsetfillopacity{0.324201}%
\pgfsetlinewidth{1.003750pt}%
\definecolor{currentstroke}{rgb}{0.121569,0.466667,0.705882}%
\pgfsetstrokecolor{currentstroke}%
\pgfsetstrokeopacity{0.324201}%
\pgfsetdash{}{0pt}%
\pgfpathmoveto{\pgfqpoint{1.571485in}{2.486288in}}%
\pgfpathcurveto{\pgfqpoint{1.579722in}{2.486288in}}{\pgfqpoint{1.587622in}{2.489560in}}{\pgfqpoint{1.593446in}{2.495384in}}%
\pgfpathcurveto{\pgfqpoint{1.599270in}{2.501208in}}{\pgfqpoint{1.602542in}{2.509108in}}{\pgfqpoint{1.602542in}{2.517344in}}%
\pgfpathcurveto{\pgfqpoint{1.602542in}{2.525580in}}{\pgfqpoint{1.599270in}{2.533481in}}{\pgfqpoint{1.593446in}{2.539304in}}%
\pgfpathcurveto{\pgfqpoint{1.587622in}{2.545128in}}{\pgfqpoint{1.579722in}{2.548401in}}{\pgfqpoint{1.571485in}{2.548401in}}%
\pgfpathcurveto{\pgfqpoint{1.563249in}{2.548401in}}{\pgfqpoint{1.555349in}{2.545128in}}{\pgfqpoint{1.549525in}{2.539304in}}%
\pgfpathcurveto{\pgfqpoint{1.543701in}{2.533481in}}{\pgfqpoint{1.540429in}{2.525580in}}{\pgfqpoint{1.540429in}{2.517344in}}%
\pgfpathcurveto{\pgfqpoint{1.540429in}{2.509108in}}{\pgfqpoint{1.543701in}{2.501208in}}{\pgfqpoint{1.549525in}{2.495384in}}%
\pgfpathcurveto{\pgfqpoint{1.555349in}{2.489560in}}{\pgfqpoint{1.563249in}{2.486288in}}{\pgfqpoint{1.571485in}{2.486288in}}%
\pgfpathclose%
\pgfusepath{stroke,fill}%
\end{pgfscope}%
\begin{pgfscope}%
\pgfpathrectangle{\pgfqpoint{0.100000in}{0.212622in}}{\pgfqpoint{3.696000in}{3.696000in}}%
\pgfusepath{clip}%
\pgfsetbuttcap%
\pgfsetroundjoin%
\definecolor{currentfill}{rgb}{0.121569,0.466667,0.705882}%
\pgfsetfillcolor{currentfill}%
\pgfsetfillopacity{0.324668}%
\pgfsetlinewidth{1.003750pt}%
\definecolor{currentstroke}{rgb}{0.121569,0.466667,0.705882}%
\pgfsetstrokecolor{currentstroke}%
\pgfsetstrokeopacity{0.324668}%
\pgfsetdash{}{0pt}%
\pgfpathmoveto{\pgfqpoint{1.570496in}{2.484764in}}%
\pgfpathcurveto{\pgfqpoint{1.578732in}{2.484764in}}{\pgfqpoint{1.586632in}{2.488036in}}{\pgfqpoint{1.592456in}{2.493860in}}%
\pgfpathcurveto{\pgfqpoint{1.598280in}{2.499684in}}{\pgfqpoint{1.601552in}{2.507584in}}{\pgfqpoint{1.601552in}{2.515820in}}%
\pgfpathcurveto{\pgfqpoint{1.601552in}{2.524056in}}{\pgfqpoint{1.598280in}{2.531957in}}{\pgfqpoint{1.592456in}{2.537780in}}%
\pgfpathcurveto{\pgfqpoint{1.586632in}{2.543604in}}{\pgfqpoint{1.578732in}{2.546877in}}{\pgfqpoint{1.570496in}{2.546877in}}%
\pgfpathcurveto{\pgfqpoint{1.562259in}{2.546877in}}{\pgfqpoint{1.554359in}{2.543604in}}{\pgfqpoint{1.548535in}{2.537780in}}%
\pgfpathcurveto{\pgfqpoint{1.542712in}{2.531957in}}{\pgfqpoint{1.539439in}{2.524056in}}{\pgfqpoint{1.539439in}{2.515820in}}%
\pgfpathcurveto{\pgfqpoint{1.539439in}{2.507584in}}{\pgfqpoint{1.542712in}{2.499684in}}{\pgfqpoint{1.548535in}{2.493860in}}%
\pgfpathcurveto{\pgfqpoint{1.554359in}{2.488036in}}{\pgfqpoint{1.562259in}{2.484764in}}{\pgfqpoint{1.570496in}{2.484764in}}%
\pgfpathclose%
\pgfusepath{stroke,fill}%
\end{pgfscope}%
\begin{pgfscope}%
\pgfpathrectangle{\pgfqpoint{0.100000in}{0.212622in}}{\pgfqpoint{3.696000in}{3.696000in}}%
\pgfusepath{clip}%
\pgfsetbuttcap%
\pgfsetroundjoin%
\definecolor{currentfill}{rgb}{0.121569,0.466667,0.705882}%
\pgfsetfillcolor{currentfill}%
\pgfsetfillopacity{0.325499}%
\pgfsetlinewidth{1.003750pt}%
\definecolor{currentstroke}{rgb}{0.121569,0.466667,0.705882}%
\pgfsetstrokecolor{currentstroke}%
\pgfsetstrokeopacity{0.325499}%
\pgfsetdash{}{0pt}%
\pgfpathmoveto{\pgfqpoint{1.873994in}{2.517870in}}%
\pgfpathcurveto{\pgfqpoint{1.882231in}{2.517870in}}{\pgfqpoint{1.890131in}{2.521142in}}{\pgfqpoint{1.895955in}{2.526966in}}%
\pgfpathcurveto{\pgfqpoint{1.901778in}{2.532790in}}{\pgfqpoint{1.905051in}{2.540690in}}{\pgfqpoint{1.905051in}{2.548927in}}%
\pgfpathcurveto{\pgfqpoint{1.905051in}{2.557163in}}{\pgfqpoint{1.901778in}{2.565063in}}{\pgfqpoint{1.895955in}{2.570887in}}%
\pgfpathcurveto{\pgfqpoint{1.890131in}{2.576711in}}{\pgfqpoint{1.882231in}{2.579983in}}{\pgfqpoint{1.873994in}{2.579983in}}%
\pgfpathcurveto{\pgfqpoint{1.865758in}{2.579983in}}{\pgfqpoint{1.857858in}{2.576711in}}{\pgfqpoint{1.852034in}{2.570887in}}%
\pgfpathcurveto{\pgfqpoint{1.846210in}{2.565063in}}{\pgfqpoint{1.842938in}{2.557163in}}{\pgfqpoint{1.842938in}{2.548927in}}%
\pgfpathcurveto{\pgfqpoint{1.842938in}{2.540690in}}{\pgfqpoint{1.846210in}{2.532790in}}{\pgfqpoint{1.852034in}{2.526966in}}%
\pgfpathcurveto{\pgfqpoint{1.857858in}{2.521142in}}{\pgfqpoint{1.865758in}{2.517870in}}{\pgfqpoint{1.873994in}{2.517870in}}%
\pgfpathclose%
\pgfusepath{stroke,fill}%
\end{pgfscope}%
\begin{pgfscope}%
\pgfpathrectangle{\pgfqpoint{0.100000in}{0.212622in}}{\pgfqpoint{3.696000in}{3.696000in}}%
\pgfusepath{clip}%
\pgfsetbuttcap%
\pgfsetroundjoin%
\definecolor{currentfill}{rgb}{0.121569,0.466667,0.705882}%
\pgfsetfillcolor{currentfill}%
\pgfsetfillopacity{0.325521}%
\pgfsetlinewidth{1.003750pt}%
\definecolor{currentstroke}{rgb}{0.121569,0.466667,0.705882}%
\pgfsetstrokecolor{currentstroke}%
\pgfsetstrokeopacity{0.325521}%
\pgfsetdash{}{0pt}%
\pgfpathmoveto{\pgfqpoint{1.568740in}{2.481988in}}%
\pgfpathcurveto{\pgfqpoint{1.576976in}{2.481988in}}{\pgfqpoint{1.584876in}{2.485260in}}{\pgfqpoint{1.590700in}{2.491084in}}%
\pgfpathcurveto{\pgfqpoint{1.596524in}{2.496908in}}{\pgfqpoint{1.599796in}{2.504808in}}{\pgfqpoint{1.599796in}{2.513044in}}%
\pgfpathcurveto{\pgfqpoint{1.599796in}{2.521281in}}{\pgfqpoint{1.596524in}{2.529181in}}{\pgfqpoint{1.590700in}{2.535005in}}%
\pgfpathcurveto{\pgfqpoint{1.584876in}{2.540829in}}{\pgfqpoint{1.576976in}{2.544101in}}{\pgfqpoint{1.568740in}{2.544101in}}%
\pgfpathcurveto{\pgfqpoint{1.560504in}{2.544101in}}{\pgfqpoint{1.552604in}{2.540829in}}{\pgfqpoint{1.546780in}{2.535005in}}%
\pgfpathcurveto{\pgfqpoint{1.540956in}{2.529181in}}{\pgfqpoint{1.537683in}{2.521281in}}{\pgfqpoint{1.537683in}{2.513044in}}%
\pgfpathcurveto{\pgfqpoint{1.537683in}{2.504808in}}{\pgfqpoint{1.540956in}{2.496908in}}{\pgfqpoint{1.546780in}{2.491084in}}%
\pgfpathcurveto{\pgfqpoint{1.552604in}{2.485260in}}{\pgfqpoint{1.560504in}{2.481988in}}{\pgfqpoint{1.568740in}{2.481988in}}%
\pgfpathclose%
\pgfusepath{stroke,fill}%
\end{pgfscope}%
\begin{pgfscope}%
\pgfpathrectangle{\pgfqpoint{0.100000in}{0.212622in}}{\pgfqpoint{3.696000in}{3.696000in}}%
\pgfusepath{clip}%
\pgfsetbuttcap%
\pgfsetroundjoin%
\definecolor{currentfill}{rgb}{0.121569,0.466667,0.705882}%
\pgfsetfillcolor{currentfill}%
\pgfsetfillopacity{0.326134}%
\pgfsetlinewidth{1.003750pt}%
\definecolor{currentstroke}{rgb}{0.121569,0.466667,0.705882}%
\pgfsetstrokecolor{currentstroke}%
\pgfsetstrokeopacity{0.326134}%
\pgfsetdash{}{0pt}%
\pgfpathmoveto{\pgfqpoint{1.567250in}{2.479546in}}%
\pgfpathcurveto{\pgfqpoint{1.575486in}{2.479546in}}{\pgfqpoint{1.583386in}{2.482818in}}{\pgfqpoint{1.589210in}{2.488642in}}%
\pgfpathcurveto{\pgfqpoint{1.595034in}{2.494466in}}{\pgfqpoint{1.598306in}{2.502366in}}{\pgfqpoint{1.598306in}{2.510602in}}%
\pgfpathcurveto{\pgfqpoint{1.598306in}{2.518839in}}{\pgfqpoint{1.595034in}{2.526739in}}{\pgfqpoint{1.589210in}{2.532563in}}%
\pgfpathcurveto{\pgfqpoint{1.583386in}{2.538387in}}{\pgfqpoint{1.575486in}{2.541659in}}{\pgfqpoint{1.567250in}{2.541659in}}%
\pgfpathcurveto{\pgfqpoint{1.559014in}{2.541659in}}{\pgfqpoint{1.551114in}{2.538387in}}{\pgfqpoint{1.545290in}{2.532563in}}%
\pgfpathcurveto{\pgfqpoint{1.539466in}{2.526739in}}{\pgfqpoint{1.536193in}{2.518839in}}{\pgfqpoint{1.536193in}{2.510602in}}%
\pgfpathcurveto{\pgfqpoint{1.536193in}{2.502366in}}{\pgfqpoint{1.539466in}{2.494466in}}{\pgfqpoint{1.545290in}{2.488642in}}%
\pgfpathcurveto{\pgfqpoint{1.551114in}{2.482818in}}{\pgfqpoint{1.559014in}{2.479546in}}{\pgfqpoint{1.567250in}{2.479546in}}%
\pgfpathclose%
\pgfusepath{stroke,fill}%
\end{pgfscope}%
\begin{pgfscope}%
\pgfpathrectangle{\pgfqpoint{0.100000in}{0.212622in}}{\pgfqpoint{3.696000in}{3.696000in}}%
\pgfusepath{clip}%
\pgfsetbuttcap%
\pgfsetroundjoin%
\definecolor{currentfill}{rgb}{0.121569,0.466667,0.705882}%
\pgfsetfillcolor{currentfill}%
\pgfsetfillopacity{0.326467}%
\pgfsetlinewidth{1.003750pt}%
\definecolor{currentstroke}{rgb}{0.121569,0.466667,0.705882}%
\pgfsetstrokecolor{currentstroke}%
\pgfsetstrokeopacity{0.326467}%
\pgfsetdash{}{0pt}%
\pgfpathmoveto{\pgfqpoint{1.879873in}{2.517144in}}%
\pgfpathcurveto{\pgfqpoint{1.888109in}{2.517144in}}{\pgfqpoint{1.896009in}{2.520416in}}{\pgfqpoint{1.901833in}{2.526240in}}%
\pgfpathcurveto{\pgfqpoint{1.907657in}{2.532064in}}{\pgfqpoint{1.910929in}{2.539964in}}{\pgfqpoint{1.910929in}{2.548201in}}%
\pgfpathcurveto{\pgfqpoint{1.910929in}{2.556437in}}{\pgfqpoint{1.907657in}{2.564337in}}{\pgfqpoint{1.901833in}{2.570161in}}%
\pgfpathcurveto{\pgfqpoint{1.896009in}{2.575985in}}{\pgfqpoint{1.888109in}{2.579257in}}{\pgfqpoint{1.879873in}{2.579257in}}%
\pgfpathcurveto{\pgfqpoint{1.871636in}{2.579257in}}{\pgfqpoint{1.863736in}{2.575985in}}{\pgfqpoint{1.857912in}{2.570161in}}%
\pgfpathcurveto{\pgfqpoint{1.852088in}{2.564337in}}{\pgfqpoint{1.848816in}{2.556437in}}{\pgfqpoint{1.848816in}{2.548201in}}%
\pgfpathcurveto{\pgfqpoint{1.848816in}{2.539964in}}{\pgfqpoint{1.852088in}{2.532064in}}{\pgfqpoint{1.857912in}{2.526240in}}%
\pgfpathcurveto{\pgfqpoint{1.863736in}{2.520416in}}{\pgfqpoint{1.871636in}{2.517144in}}{\pgfqpoint{1.879873in}{2.517144in}}%
\pgfpathclose%
\pgfusepath{stroke,fill}%
\end{pgfscope}%
\begin{pgfscope}%
\pgfpathrectangle{\pgfqpoint{0.100000in}{0.212622in}}{\pgfqpoint{3.696000in}{3.696000in}}%
\pgfusepath{clip}%
\pgfsetbuttcap%
\pgfsetroundjoin%
\definecolor{currentfill}{rgb}{0.121569,0.466667,0.705882}%
\pgfsetfillcolor{currentfill}%
\pgfsetfillopacity{0.326663}%
\pgfsetlinewidth{1.003750pt}%
\definecolor{currentstroke}{rgb}{0.121569,0.466667,0.705882}%
\pgfsetstrokecolor{currentstroke}%
\pgfsetstrokeopacity{0.326663}%
\pgfsetdash{}{0pt}%
\pgfpathmoveto{\pgfqpoint{1.566009in}{2.477463in}}%
\pgfpathcurveto{\pgfqpoint{1.574245in}{2.477463in}}{\pgfqpoint{1.582145in}{2.480735in}}{\pgfqpoint{1.587969in}{2.486559in}}%
\pgfpathcurveto{\pgfqpoint{1.593793in}{2.492383in}}{\pgfqpoint{1.597065in}{2.500283in}}{\pgfqpoint{1.597065in}{2.508519in}}%
\pgfpathcurveto{\pgfqpoint{1.597065in}{2.516756in}}{\pgfqpoint{1.593793in}{2.524656in}}{\pgfqpoint{1.587969in}{2.530480in}}%
\pgfpathcurveto{\pgfqpoint{1.582145in}{2.536304in}}{\pgfqpoint{1.574245in}{2.539576in}}{\pgfqpoint{1.566009in}{2.539576in}}%
\pgfpathcurveto{\pgfqpoint{1.557772in}{2.539576in}}{\pgfqpoint{1.549872in}{2.536304in}}{\pgfqpoint{1.544048in}{2.530480in}}%
\pgfpathcurveto{\pgfqpoint{1.538224in}{2.524656in}}{\pgfqpoint{1.534952in}{2.516756in}}{\pgfqpoint{1.534952in}{2.508519in}}%
\pgfpathcurveto{\pgfqpoint{1.534952in}{2.500283in}}{\pgfqpoint{1.538224in}{2.492383in}}{\pgfqpoint{1.544048in}{2.486559in}}%
\pgfpathcurveto{\pgfqpoint{1.549872in}{2.480735in}}{\pgfqpoint{1.557772in}{2.477463in}}{\pgfqpoint{1.566009in}{2.477463in}}%
\pgfpathclose%
\pgfusepath{stroke,fill}%
\end{pgfscope}%
\begin{pgfscope}%
\pgfpathrectangle{\pgfqpoint{0.100000in}{0.212622in}}{\pgfqpoint{3.696000in}{3.696000in}}%
\pgfusepath{clip}%
\pgfsetbuttcap%
\pgfsetroundjoin%
\definecolor{currentfill}{rgb}{0.121569,0.466667,0.705882}%
\pgfsetfillcolor{currentfill}%
\pgfsetfillopacity{0.326876}%
\pgfsetlinewidth{1.003750pt}%
\definecolor{currentstroke}{rgb}{0.121569,0.466667,0.705882}%
\pgfsetstrokecolor{currentstroke}%
\pgfsetstrokeopacity{0.326876}%
\pgfsetdash{}{0pt}%
\pgfpathmoveto{\pgfqpoint{1.565490in}{2.476627in}}%
\pgfpathcurveto{\pgfqpoint{1.573726in}{2.476627in}}{\pgfqpoint{1.581626in}{2.479900in}}{\pgfqpoint{1.587450in}{2.485724in}}%
\pgfpathcurveto{\pgfqpoint{1.593274in}{2.491548in}}{\pgfqpoint{1.596546in}{2.499448in}}{\pgfqpoint{1.596546in}{2.507684in}}%
\pgfpathcurveto{\pgfqpoint{1.596546in}{2.515920in}}{\pgfqpoint{1.593274in}{2.523820in}}{\pgfqpoint{1.587450in}{2.529644in}}%
\pgfpathcurveto{\pgfqpoint{1.581626in}{2.535468in}}{\pgfqpoint{1.573726in}{2.538740in}}{\pgfqpoint{1.565490in}{2.538740in}}%
\pgfpathcurveto{\pgfqpoint{1.557254in}{2.538740in}}{\pgfqpoint{1.549354in}{2.535468in}}{\pgfqpoint{1.543530in}{2.529644in}}%
\pgfpathcurveto{\pgfqpoint{1.537706in}{2.523820in}}{\pgfqpoint{1.534433in}{2.515920in}}{\pgfqpoint{1.534433in}{2.507684in}}%
\pgfpathcurveto{\pgfqpoint{1.534433in}{2.499448in}}{\pgfqpoint{1.537706in}{2.491548in}}{\pgfqpoint{1.543530in}{2.485724in}}%
\pgfpathcurveto{\pgfqpoint{1.549354in}{2.479900in}}{\pgfqpoint{1.557254in}{2.476627in}}{\pgfqpoint{1.565490in}{2.476627in}}%
\pgfpathclose%
\pgfusepath{stroke,fill}%
\end{pgfscope}%
\begin{pgfscope}%
\pgfpathrectangle{\pgfqpoint{0.100000in}{0.212622in}}{\pgfqpoint{3.696000in}{3.696000in}}%
\pgfusepath{clip}%
\pgfsetbuttcap%
\pgfsetroundjoin%
\definecolor{currentfill}{rgb}{0.121569,0.466667,0.705882}%
\pgfsetfillcolor{currentfill}%
\pgfsetfillopacity{0.327004}%
\pgfsetlinewidth{1.003750pt}%
\definecolor{currentstroke}{rgb}{0.121569,0.466667,0.705882}%
\pgfsetstrokecolor{currentstroke}%
\pgfsetstrokeopacity{0.327004}%
\pgfsetdash{}{0pt}%
\pgfpathmoveto{\pgfqpoint{1.565197in}{2.476140in}}%
\pgfpathcurveto{\pgfqpoint{1.573433in}{2.476140in}}{\pgfqpoint{1.581333in}{2.479413in}}{\pgfqpoint{1.587157in}{2.485237in}}%
\pgfpathcurveto{\pgfqpoint{1.592981in}{2.491060in}}{\pgfqpoint{1.596253in}{2.498961in}}{\pgfqpoint{1.596253in}{2.507197in}}%
\pgfpathcurveto{\pgfqpoint{1.596253in}{2.515433in}}{\pgfqpoint{1.592981in}{2.523333in}}{\pgfqpoint{1.587157in}{2.529157in}}%
\pgfpathcurveto{\pgfqpoint{1.581333in}{2.534981in}}{\pgfqpoint{1.573433in}{2.538253in}}{\pgfqpoint{1.565197in}{2.538253in}}%
\pgfpathcurveto{\pgfqpoint{1.556960in}{2.538253in}}{\pgfqpoint{1.549060in}{2.534981in}}{\pgfqpoint{1.543236in}{2.529157in}}%
\pgfpathcurveto{\pgfqpoint{1.537412in}{2.523333in}}{\pgfqpoint{1.534140in}{2.515433in}}{\pgfqpoint{1.534140in}{2.507197in}}%
\pgfpathcurveto{\pgfqpoint{1.534140in}{2.498961in}}{\pgfqpoint{1.537412in}{2.491060in}}{\pgfqpoint{1.543236in}{2.485237in}}%
\pgfpathcurveto{\pgfqpoint{1.549060in}{2.479413in}}{\pgfqpoint{1.556960in}{2.476140in}}{\pgfqpoint{1.565197in}{2.476140in}}%
\pgfpathclose%
\pgfusepath{stroke,fill}%
\end{pgfscope}%
\begin{pgfscope}%
\pgfpathrectangle{\pgfqpoint{0.100000in}{0.212622in}}{\pgfqpoint{3.696000in}{3.696000in}}%
\pgfusepath{clip}%
\pgfsetbuttcap%
\pgfsetroundjoin%
\definecolor{currentfill}{rgb}{0.121569,0.466667,0.705882}%
\pgfsetfillcolor{currentfill}%
\pgfsetfillopacity{0.327026}%
\pgfsetlinewidth{1.003750pt}%
\definecolor{currentstroke}{rgb}{0.121569,0.466667,0.705882}%
\pgfsetstrokecolor{currentstroke}%
\pgfsetstrokeopacity{0.327026}%
\pgfsetdash{}{0pt}%
\pgfpathmoveto{\pgfqpoint{1.565144in}{2.476056in}}%
\pgfpathcurveto{\pgfqpoint{1.573380in}{2.476056in}}{\pgfqpoint{1.581280in}{2.479329in}}{\pgfqpoint{1.587104in}{2.485153in}}%
\pgfpathcurveto{\pgfqpoint{1.592928in}{2.490977in}}{\pgfqpoint{1.596200in}{2.498877in}}{\pgfqpoint{1.596200in}{2.507113in}}%
\pgfpathcurveto{\pgfqpoint{1.596200in}{2.515349in}}{\pgfqpoint{1.592928in}{2.523249in}}{\pgfqpoint{1.587104in}{2.529073in}}%
\pgfpathcurveto{\pgfqpoint{1.581280in}{2.534897in}}{\pgfqpoint{1.573380in}{2.538169in}}{\pgfqpoint{1.565144in}{2.538169in}}%
\pgfpathcurveto{\pgfqpoint{1.556908in}{2.538169in}}{\pgfqpoint{1.549007in}{2.534897in}}{\pgfqpoint{1.543184in}{2.529073in}}%
\pgfpathcurveto{\pgfqpoint{1.537360in}{2.523249in}}{\pgfqpoint{1.534087in}{2.515349in}}{\pgfqpoint{1.534087in}{2.507113in}}%
\pgfpathcurveto{\pgfqpoint{1.534087in}{2.498877in}}{\pgfqpoint{1.537360in}{2.490977in}}{\pgfqpoint{1.543184in}{2.485153in}}%
\pgfpathcurveto{\pgfqpoint{1.549007in}{2.479329in}}{\pgfqpoint{1.556908in}{2.476056in}}{\pgfqpoint{1.565144in}{2.476056in}}%
\pgfpathclose%
\pgfusepath{stroke,fill}%
\end{pgfscope}%
\begin{pgfscope}%
\pgfpathrectangle{\pgfqpoint{0.100000in}{0.212622in}}{\pgfqpoint{3.696000in}{3.696000in}}%
\pgfusepath{clip}%
\pgfsetbuttcap%
\pgfsetroundjoin%
\definecolor{currentfill}{rgb}{0.121569,0.466667,0.705882}%
\pgfsetfillcolor{currentfill}%
\pgfsetfillopacity{0.327064}%
\pgfsetlinewidth{1.003750pt}%
\definecolor{currentstroke}{rgb}{0.121569,0.466667,0.705882}%
\pgfsetstrokecolor{currentstroke}%
\pgfsetstrokeopacity{0.327064}%
\pgfsetdash{}{0pt}%
\pgfpathmoveto{\pgfqpoint{1.565053in}{2.475889in}}%
\pgfpathcurveto{\pgfqpoint{1.573289in}{2.475889in}}{\pgfqpoint{1.581189in}{2.479161in}}{\pgfqpoint{1.587013in}{2.484985in}}%
\pgfpathcurveto{\pgfqpoint{1.592837in}{2.490809in}}{\pgfqpoint{1.596109in}{2.498709in}}{\pgfqpoint{1.596109in}{2.506945in}}%
\pgfpathcurveto{\pgfqpoint{1.596109in}{2.515182in}}{\pgfqpoint{1.592837in}{2.523082in}}{\pgfqpoint{1.587013in}{2.528906in}}%
\pgfpathcurveto{\pgfqpoint{1.581189in}{2.534730in}}{\pgfqpoint{1.573289in}{2.538002in}}{\pgfqpoint{1.565053in}{2.538002in}}%
\pgfpathcurveto{\pgfqpoint{1.556816in}{2.538002in}}{\pgfqpoint{1.548916in}{2.534730in}}{\pgfqpoint{1.543092in}{2.528906in}}%
\pgfpathcurveto{\pgfqpoint{1.537269in}{2.523082in}}{\pgfqpoint{1.533996in}{2.515182in}}{\pgfqpoint{1.533996in}{2.506945in}}%
\pgfpathcurveto{\pgfqpoint{1.533996in}{2.498709in}}{\pgfqpoint{1.537269in}{2.490809in}}{\pgfqpoint{1.543092in}{2.484985in}}%
\pgfpathcurveto{\pgfqpoint{1.548916in}{2.479161in}}{\pgfqpoint{1.556816in}{2.475889in}}{\pgfqpoint{1.565053in}{2.475889in}}%
\pgfpathclose%
\pgfusepath{stroke,fill}%
\end{pgfscope}%
\begin{pgfscope}%
\pgfpathrectangle{\pgfqpoint{0.100000in}{0.212622in}}{\pgfqpoint{3.696000in}{3.696000in}}%
\pgfusepath{clip}%
\pgfsetbuttcap%
\pgfsetroundjoin%
\definecolor{currentfill}{rgb}{0.121569,0.466667,0.705882}%
\pgfsetfillcolor{currentfill}%
\pgfsetfillopacity{0.327140}%
\pgfsetlinewidth{1.003750pt}%
\definecolor{currentstroke}{rgb}{0.121569,0.466667,0.705882}%
\pgfsetstrokecolor{currentstroke}%
\pgfsetstrokeopacity{0.327140}%
\pgfsetdash{}{0pt}%
\pgfpathmoveto{\pgfqpoint{1.564870in}{2.475641in}}%
\pgfpathcurveto{\pgfqpoint{1.573106in}{2.475641in}}{\pgfqpoint{1.581006in}{2.478913in}}{\pgfqpoint{1.586830in}{2.484737in}}%
\pgfpathcurveto{\pgfqpoint{1.592654in}{2.490561in}}{\pgfqpoint{1.595926in}{2.498461in}}{\pgfqpoint{1.595926in}{2.506698in}}%
\pgfpathcurveto{\pgfqpoint{1.595926in}{2.514934in}}{\pgfqpoint{1.592654in}{2.522834in}}{\pgfqpoint{1.586830in}{2.528658in}}%
\pgfpathcurveto{\pgfqpoint{1.581006in}{2.534482in}}{\pgfqpoint{1.573106in}{2.537754in}}{\pgfqpoint{1.564870in}{2.537754in}}%
\pgfpathcurveto{\pgfqpoint{1.556633in}{2.537754in}}{\pgfqpoint{1.548733in}{2.534482in}}{\pgfqpoint{1.542909in}{2.528658in}}%
\pgfpathcurveto{\pgfqpoint{1.537085in}{2.522834in}}{\pgfqpoint{1.533813in}{2.514934in}}{\pgfqpoint{1.533813in}{2.506698in}}%
\pgfpathcurveto{\pgfqpoint{1.533813in}{2.498461in}}{\pgfqpoint{1.537085in}{2.490561in}}{\pgfqpoint{1.542909in}{2.484737in}}%
\pgfpathcurveto{\pgfqpoint{1.548733in}{2.478913in}}{\pgfqpoint{1.556633in}{2.475641in}}{\pgfqpoint{1.564870in}{2.475641in}}%
\pgfpathclose%
\pgfusepath{stroke,fill}%
\end{pgfscope}%
\begin{pgfscope}%
\pgfpathrectangle{\pgfqpoint{0.100000in}{0.212622in}}{\pgfqpoint{3.696000in}{3.696000in}}%
\pgfusepath{clip}%
\pgfsetbuttcap%
\pgfsetroundjoin%
\definecolor{currentfill}{rgb}{0.121569,0.466667,0.705882}%
\pgfsetfillcolor{currentfill}%
\pgfsetfillopacity{0.327264}%
\pgfsetlinewidth{1.003750pt}%
\definecolor{currentstroke}{rgb}{0.121569,0.466667,0.705882}%
\pgfsetstrokecolor{currentstroke}%
\pgfsetstrokeopacity{0.327264}%
\pgfsetdash{}{0pt}%
\pgfpathmoveto{\pgfqpoint{1.564556in}{2.475077in}}%
\pgfpathcurveto{\pgfqpoint{1.572792in}{2.475077in}}{\pgfqpoint{1.580692in}{2.478349in}}{\pgfqpoint{1.586516in}{2.484173in}}%
\pgfpathcurveto{\pgfqpoint{1.592340in}{2.489997in}}{\pgfqpoint{1.595613in}{2.497897in}}{\pgfqpoint{1.595613in}{2.506133in}}%
\pgfpathcurveto{\pgfqpoint{1.595613in}{2.514369in}}{\pgfqpoint{1.592340in}{2.522269in}}{\pgfqpoint{1.586516in}{2.528093in}}%
\pgfpathcurveto{\pgfqpoint{1.580692in}{2.533917in}}{\pgfqpoint{1.572792in}{2.537190in}}{\pgfqpoint{1.564556in}{2.537190in}}%
\pgfpathcurveto{\pgfqpoint{1.556320in}{2.537190in}}{\pgfqpoint{1.548420in}{2.533917in}}{\pgfqpoint{1.542596in}{2.528093in}}%
\pgfpathcurveto{\pgfqpoint{1.536772in}{2.522269in}}{\pgfqpoint{1.533500in}{2.514369in}}{\pgfqpoint{1.533500in}{2.506133in}}%
\pgfpathcurveto{\pgfqpoint{1.533500in}{2.497897in}}{\pgfqpoint{1.536772in}{2.489997in}}{\pgfqpoint{1.542596in}{2.484173in}}%
\pgfpathcurveto{\pgfqpoint{1.548420in}{2.478349in}}{\pgfqpoint{1.556320in}{2.475077in}}{\pgfqpoint{1.564556in}{2.475077in}}%
\pgfpathclose%
\pgfusepath{stroke,fill}%
\end{pgfscope}%
\begin{pgfscope}%
\pgfpathrectangle{\pgfqpoint{0.100000in}{0.212622in}}{\pgfqpoint{3.696000in}{3.696000in}}%
\pgfusepath{clip}%
\pgfsetbuttcap%
\pgfsetroundjoin%
\definecolor{currentfill}{rgb}{0.121569,0.466667,0.705882}%
\pgfsetfillcolor{currentfill}%
\pgfsetfillopacity{0.327497}%
\pgfsetlinewidth{1.003750pt}%
\definecolor{currentstroke}{rgb}{0.121569,0.466667,0.705882}%
\pgfsetstrokecolor{currentstroke}%
\pgfsetstrokeopacity{0.327497}%
\pgfsetdash{}{0pt}%
\pgfpathmoveto{\pgfqpoint{1.563921in}{2.474164in}}%
\pgfpathcurveto{\pgfqpoint{1.572158in}{2.474164in}}{\pgfqpoint{1.580058in}{2.477436in}}{\pgfqpoint{1.585882in}{2.483260in}}%
\pgfpathcurveto{\pgfqpoint{1.591705in}{2.489084in}}{\pgfqpoint{1.594978in}{2.496984in}}{\pgfqpoint{1.594978in}{2.505221in}}%
\pgfpathcurveto{\pgfqpoint{1.594978in}{2.513457in}}{\pgfqpoint{1.591705in}{2.521357in}}{\pgfqpoint{1.585882in}{2.527181in}}%
\pgfpathcurveto{\pgfqpoint{1.580058in}{2.533005in}}{\pgfqpoint{1.572158in}{2.536277in}}{\pgfqpoint{1.563921in}{2.536277in}}%
\pgfpathcurveto{\pgfqpoint{1.555685in}{2.536277in}}{\pgfqpoint{1.547785in}{2.533005in}}{\pgfqpoint{1.541961in}{2.527181in}}%
\pgfpathcurveto{\pgfqpoint{1.536137in}{2.521357in}}{\pgfqpoint{1.532865in}{2.513457in}}{\pgfqpoint{1.532865in}{2.505221in}}%
\pgfpathcurveto{\pgfqpoint{1.532865in}{2.496984in}}{\pgfqpoint{1.536137in}{2.489084in}}{\pgfqpoint{1.541961in}{2.483260in}}%
\pgfpathcurveto{\pgfqpoint{1.547785in}{2.477436in}}{\pgfqpoint{1.555685in}{2.474164in}}{\pgfqpoint{1.563921in}{2.474164in}}%
\pgfpathclose%
\pgfusepath{stroke,fill}%
\end{pgfscope}%
\begin{pgfscope}%
\pgfpathrectangle{\pgfqpoint{0.100000in}{0.212622in}}{\pgfqpoint{3.696000in}{3.696000in}}%
\pgfusepath{clip}%
\pgfsetbuttcap%
\pgfsetroundjoin%
\definecolor{currentfill}{rgb}{0.121569,0.466667,0.705882}%
\pgfsetfillcolor{currentfill}%
\pgfsetfillopacity{0.327811}%
\pgfsetlinewidth{1.003750pt}%
\definecolor{currentstroke}{rgb}{0.121569,0.466667,0.705882}%
\pgfsetstrokecolor{currentstroke}%
\pgfsetstrokeopacity{0.327811}%
\pgfsetdash{}{0pt}%
\pgfpathmoveto{\pgfqpoint{1.888251in}{2.515725in}}%
\pgfpathcurveto{\pgfqpoint{1.896487in}{2.515725in}}{\pgfqpoint{1.904387in}{2.518997in}}{\pgfqpoint{1.910211in}{2.524821in}}%
\pgfpathcurveto{\pgfqpoint{1.916035in}{2.530645in}}{\pgfqpoint{1.919307in}{2.538545in}}{\pgfqpoint{1.919307in}{2.546781in}}%
\pgfpathcurveto{\pgfqpoint{1.919307in}{2.555017in}}{\pgfqpoint{1.916035in}{2.562917in}}{\pgfqpoint{1.910211in}{2.568741in}}%
\pgfpathcurveto{\pgfqpoint{1.904387in}{2.574565in}}{\pgfqpoint{1.896487in}{2.577838in}}{\pgfqpoint{1.888251in}{2.577838in}}%
\pgfpathcurveto{\pgfqpoint{1.880014in}{2.577838in}}{\pgfqpoint{1.872114in}{2.574565in}}{\pgfqpoint{1.866290in}{2.568741in}}%
\pgfpathcurveto{\pgfqpoint{1.860467in}{2.562917in}}{\pgfqpoint{1.857194in}{2.555017in}}{\pgfqpoint{1.857194in}{2.546781in}}%
\pgfpathcurveto{\pgfqpoint{1.857194in}{2.538545in}}{\pgfqpoint{1.860467in}{2.530645in}}{\pgfqpoint{1.866290in}{2.524821in}}%
\pgfpathcurveto{\pgfqpoint{1.872114in}{2.518997in}}{\pgfqpoint{1.880014in}{2.515725in}}{\pgfqpoint{1.888251in}{2.515725in}}%
\pgfpathclose%
\pgfusepath{stroke,fill}%
\end{pgfscope}%
\begin{pgfscope}%
\pgfpathrectangle{\pgfqpoint{0.100000in}{0.212622in}}{\pgfqpoint{3.696000in}{3.696000in}}%
\pgfusepath{clip}%
\pgfsetbuttcap%
\pgfsetroundjoin%
\definecolor{currentfill}{rgb}{0.121569,0.466667,0.705882}%
\pgfsetfillcolor{currentfill}%
\pgfsetfillopacity{0.327912}%
\pgfsetlinewidth{1.003750pt}%
\definecolor{currentstroke}{rgb}{0.121569,0.466667,0.705882}%
\pgfsetstrokecolor{currentstroke}%
\pgfsetstrokeopacity{0.327912}%
\pgfsetdash{}{0pt}%
\pgfpathmoveto{\pgfqpoint{1.562821in}{2.472391in}}%
\pgfpathcurveto{\pgfqpoint{1.571058in}{2.472391in}}{\pgfqpoint{1.578958in}{2.475663in}}{\pgfqpoint{1.584782in}{2.481487in}}%
\pgfpathcurveto{\pgfqpoint{1.590606in}{2.487311in}}{\pgfqpoint{1.593878in}{2.495211in}}{\pgfqpoint{1.593878in}{2.503448in}}%
\pgfpathcurveto{\pgfqpoint{1.593878in}{2.511684in}}{\pgfqpoint{1.590606in}{2.519584in}}{\pgfqpoint{1.584782in}{2.525408in}}%
\pgfpathcurveto{\pgfqpoint{1.578958in}{2.531232in}}{\pgfqpoint{1.571058in}{2.534504in}}{\pgfqpoint{1.562821in}{2.534504in}}%
\pgfpathcurveto{\pgfqpoint{1.554585in}{2.534504in}}{\pgfqpoint{1.546685in}{2.531232in}}{\pgfqpoint{1.540861in}{2.525408in}}%
\pgfpathcurveto{\pgfqpoint{1.535037in}{2.519584in}}{\pgfqpoint{1.531765in}{2.511684in}}{\pgfqpoint{1.531765in}{2.503448in}}%
\pgfpathcurveto{\pgfqpoint{1.531765in}{2.495211in}}{\pgfqpoint{1.535037in}{2.487311in}}{\pgfqpoint{1.540861in}{2.481487in}}%
\pgfpathcurveto{\pgfqpoint{1.546685in}{2.475663in}}{\pgfqpoint{1.554585in}{2.472391in}}{\pgfqpoint{1.562821in}{2.472391in}}%
\pgfpathclose%
\pgfusepath{stroke,fill}%
\end{pgfscope}%
\begin{pgfscope}%
\pgfpathrectangle{\pgfqpoint{0.100000in}{0.212622in}}{\pgfqpoint{3.696000in}{3.696000in}}%
\pgfusepath{clip}%
\pgfsetbuttcap%
\pgfsetroundjoin%
\definecolor{currentfill}{rgb}{0.121569,0.466667,0.705882}%
\pgfsetfillcolor{currentfill}%
\pgfsetfillopacity{0.328156}%
\pgfsetlinewidth{1.003750pt}%
\definecolor{currentstroke}{rgb}{0.121569,0.466667,0.705882}%
\pgfsetstrokecolor{currentstroke}%
\pgfsetstrokeopacity{0.328156}%
\pgfsetdash{}{0pt}%
\pgfpathmoveto{\pgfqpoint{1.562142in}{2.471439in}}%
\pgfpathcurveto{\pgfqpoint{1.570378in}{2.471439in}}{\pgfqpoint{1.578278in}{2.474711in}}{\pgfqpoint{1.584102in}{2.480535in}}%
\pgfpathcurveto{\pgfqpoint{1.589926in}{2.486359in}}{\pgfqpoint{1.593198in}{2.494259in}}{\pgfqpoint{1.593198in}{2.502495in}}%
\pgfpathcurveto{\pgfqpoint{1.593198in}{2.510732in}}{\pgfqpoint{1.589926in}{2.518632in}}{\pgfqpoint{1.584102in}{2.524456in}}%
\pgfpathcurveto{\pgfqpoint{1.578278in}{2.530279in}}{\pgfqpoint{1.570378in}{2.533552in}}{\pgfqpoint{1.562142in}{2.533552in}}%
\pgfpathcurveto{\pgfqpoint{1.553905in}{2.533552in}}{\pgfqpoint{1.546005in}{2.530279in}}{\pgfqpoint{1.540181in}{2.524456in}}%
\pgfpathcurveto{\pgfqpoint{1.534357in}{2.518632in}}{\pgfqpoint{1.531085in}{2.510732in}}{\pgfqpoint{1.531085in}{2.502495in}}%
\pgfpathcurveto{\pgfqpoint{1.531085in}{2.494259in}}{\pgfqpoint{1.534357in}{2.486359in}}{\pgfqpoint{1.540181in}{2.480535in}}%
\pgfpathcurveto{\pgfqpoint{1.546005in}{2.474711in}}{\pgfqpoint{1.553905in}{2.471439in}}{\pgfqpoint{1.562142in}{2.471439in}}%
\pgfpathclose%
\pgfusepath{stroke,fill}%
\end{pgfscope}%
\begin{pgfscope}%
\pgfpathrectangle{\pgfqpoint{0.100000in}{0.212622in}}{\pgfqpoint{3.696000in}{3.696000in}}%
\pgfusepath{clip}%
\pgfsetbuttcap%
\pgfsetroundjoin%
\definecolor{currentfill}{rgb}{0.121569,0.466667,0.705882}%
\pgfsetfillcolor{currentfill}%
\pgfsetfillopacity{0.328282}%
\pgfsetlinewidth{1.003750pt}%
\definecolor{currentstroke}{rgb}{0.121569,0.466667,0.705882}%
\pgfsetstrokecolor{currentstroke}%
\pgfsetstrokeopacity{0.328282}%
\pgfsetdash{}{0pt}%
\pgfpathmoveto{\pgfqpoint{1.561801in}{2.470971in}}%
\pgfpathcurveto{\pgfqpoint{1.570037in}{2.470971in}}{\pgfqpoint{1.577937in}{2.474244in}}{\pgfqpoint{1.583761in}{2.480068in}}%
\pgfpathcurveto{\pgfqpoint{1.589585in}{2.485892in}}{\pgfqpoint{1.592857in}{2.493792in}}{\pgfqpoint{1.592857in}{2.502028in}}%
\pgfpathcurveto{\pgfqpoint{1.592857in}{2.510264in}}{\pgfqpoint{1.589585in}{2.518164in}}{\pgfqpoint{1.583761in}{2.523988in}}%
\pgfpathcurveto{\pgfqpoint{1.577937in}{2.529812in}}{\pgfqpoint{1.570037in}{2.533084in}}{\pgfqpoint{1.561801in}{2.533084in}}%
\pgfpathcurveto{\pgfqpoint{1.553564in}{2.533084in}}{\pgfqpoint{1.545664in}{2.529812in}}{\pgfqpoint{1.539840in}{2.523988in}}%
\pgfpathcurveto{\pgfqpoint{1.534017in}{2.518164in}}{\pgfqpoint{1.530744in}{2.510264in}}{\pgfqpoint{1.530744in}{2.502028in}}%
\pgfpathcurveto{\pgfqpoint{1.530744in}{2.493792in}}{\pgfqpoint{1.534017in}{2.485892in}}{\pgfqpoint{1.539840in}{2.480068in}}%
\pgfpathcurveto{\pgfqpoint{1.545664in}{2.474244in}}{\pgfqpoint{1.553564in}{2.470971in}}{\pgfqpoint{1.561801in}{2.470971in}}%
\pgfpathclose%
\pgfusepath{stroke,fill}%
\end{pgfscope}%
\begin{pgfscope}%
\pgfpathrectangle{\pgfqpoint{0.100000in}{0.212622in}}{\pgfqpoint{3.696000in}{3.696000in}}%
\pgfusepath{clip}%
\pgfsetbuttcap%
\pgfsetroundjoin%
\definecolor{currentfill}{rgb}{0.121569,0.466667,0.705882}%
\pgfsetfillcolor{currentfill}%
\pgfsetfillopacity{0.328517}%
\pgfsetlinewidth{1.003750pt}%
\definecolor{currentstroke}{rgb}{0.121569,0.466667,0.705882}%
\pgfsetstrokecolor{currentstroke}%
\pgfsetstrokeopacity{0.328517}%
\pgfsetdash{}{0pt}%
\pgfpathmoveto{\pgfqpoint{1.561190in}{2.470149in}}%
\pgfpathcurveto{\pgfqpoint{1.569426in}{2.470149in}}{\pgfqpoint{1.577326in}{2.473421in}}{\pgfqpoint{1.583150in}{2.479245in}}%
\pgfpathcurveto{\pgfqpoint{1.588974in}{2.485069in}}{\pgfqpoint{1.592246in}{2.492969in}}{\pgfqpoint{1.592246in}{2.501205in}}%
\pgfpathcurveto{\pgfqpoint{1.592246in}{2.509441in}}{\pgfqpoint{1.588974in}{2.517341in}}{\pgfqpoint{1.583150in}{2.523165in}}%
\pgfpathcurveto{\pgfqpoint{1.577326in}{2.528989in}}{\pgfqpoint{1.569426in}{2.532262in}}{\pgfqpoint{1.561190in}{2.532262in}}%
\pgfpathcurveto{\pgfqpoint{1.552954in}{2.532262in}}{\pgfqpoint{1.545053in}{2.528989in}}{\pgfqpoint{1.539230in}{2.523165in}}%
\pgfpathcurveto{\pgfqpoint{1.533406in}{2.517341in}}{\pgfqpoint{1.530133in}{2.509441in}}{\pgfqpoint{1.530133in}{2.501205in}}%
\pgfpathcurveto{\pgfqpoint{1.530133in}{2.492969in}}{\pgfqpoint{1.533406in}{2.485069in}}{\pgfqpoint{1.539230in}{2.479245in}}%
\pgfpathcurveto{\pgfqpoint{1.545053in}{2.473421in}}{\pgfqpoint{1.552954in}{2.470149in}}{\pgfqpoint{1.561190in}{2.470149in}}%
\pgfpathclose%
\pgfusepath{stroke,fill}%
\end{pgfscope}%
\begin{pgfscope}%
\pgfpathrectangle{\pgfqpoint{0.100000in}{0.212622in}}{\pgfqpoint{3.696000in}{3.696000in}}%
\pgfusepath{clip}%
\pgfsetbuttcap%
\pgfsetroundjoin%
\definecolor{currentfill}{rgb}{0.121569,0.466667,0.705882}%
\pgfsetfillcolor{currentfill}%
\pgfsetfillopacity{0.328947}%
\pgfsetlinewidth{1.003750pt}%
\definecolor{currentstroke}{rgb}{0.121569,0.466667,0.705882}%
\pgfsetstrokecolor{currentstroke}%
\pgfsetstrokeopacity{0.328947}%
\pgfsetdash{}{0pt}%
\pgfpathmoveto{\pgfqpoint{1.560113in}{2.468650in}}%
\pgfpathcurveto{\pgfqpoint{1.568349in}{2.468650in}}{\pgfqpoint{1.576249in}{2.471922in}}{\pgfqpoint{1.582073in}{2.477746in}}%
\pgfpathcurveto{\pgfqpoint{1.587897in}{2.483570in}}{\pgfqpoint{1.591169in}{2.491470in}}{\pgfqpoint{1.591169in}{2.499706in}}%
\pgfpathcurveto{\pgfqpoint{1.591169in}{2.507943in}}{\pgfqpoint{1.587897in}{2.515843in}}{\pgfqpoint{1.582073in}{2.521667in}}%
\pgfpathcurveto{\pgfqpoint{1.576249in}{2.527491in}}{\pgfqpoint{1.568349in}{2.530763in}}{\pgfqpoint{1.560113in}{2.530763in}}%
\pgfpathcurveto{\pgfqpoint{1.551876in}{2.530763in}}{\pgfqpoint{1.543976in}{2.527491in}}{\pgfqpoint{1.538152in}{2.521667in}}%
\pgfpathcurveto{\pgfqpoint{1.532328in}{2.515843in}}{\pgfqpoint{1.529056in}{2.507943in}}{\pgfqpoint{1.529056in}{2.499706in}}%
\pgfpathcurveto{\pgfqpoint{1.529056in}{2.491470in}}{\pgfqpoint{1.532328in}{2.483570in}}{\pgfqpoint{1.538152in}{2.477746in}}%
\pgfpathcurveto{\pgfqpoint{1.543976in}{2.471922in}}{\pgfqpoint{1.551876in}{2.468650in}}{\pgfqpoint{1.560113in}{2.468650in}}%
\pgfpathclose%
\pgfusepath{stroke,fill}%
\end{pgfscope}%
\begin{pgfscope}%
\pgfpathrectangle{\pgfqpoint{0.100000in}{0.212622in}}{\pgfqpoint{3.696000in}{3.696000in}}%
\pgfusepath{clip}%
\pgfsetbuttcap%
\pgfsetroundjoin%
\definecolor{currentfill}{rgb}{0.121569,0.466667,0.705882}%
\pgfsetfillcolor{currentfill}%
\pgfsetfillopacity{0.329247}%
\pgfsetlinewidth{1.003750pt}%
\definecolor{currentstroke}{rgb}{0.121569,0.466667,0.705882}%
\pgfsetstrokecolor{currentstroke}%
\pgfsetstrokeopacity{0.329247}%
\pgfsetdash{}{0pt}%
\pgfpathmoveto{\pgfqpoint{1.559362in}{2.467547in}}%
\pgfpathcurveto{\pgfqpoint{1.567598in}{2.467547in}}{\pgfqpoint{1.575498in}{2.470819in}}{\pgfqpoint{1.581322in}{2.476643in}}%
\pgfpathcurveto{\pgfqpoint{1.587146in}{2.482467in}}{\pgfqpoint{1.590419in}{2.490367in}}{\pgfqpoint{1.590419in}{2.498604in}}%
\pgfpathcurveto{\pgfqpoint{1.590419in}{2.506840in}}{\pgfqpoint{1.587146in}{2.514740in}}{\pgfqpoint{1.581322in}{2.520564in}}%
\pgfpathcurveto{\pgfqpoint{1.575498in}{2.526388in}}{\pgfqpoint{1.567598in}{2.529660in}}{\pgfqpoint{1.559362in}{2.529660in}}%
\pgfpathcurveto{\pgfqpoint{1.551126in}{2.529660in}}{\pgfqpoint{1.543226in}{2.526388in}}{\pgfqpoint{1.537402in}{2.520564in}}%
\pgfpathcurveto{\pgfqpoint{1.531578in}{2.514740in}}{\pgfqpoint{1.528306in}{2.506840in}}{\pgfqpoint{1.528306in}{2.498604in}}%
\pgfpathcurveto{\pgfqpoint{1.528306in}{2.490367in}}{\pgfqpoint{1.531578in}{2.482467in}}{\pgfqpoint{1.537402in}{2.476643in}}%
\pgfpathcurveto{\pgfqpoint{1.543226in}{2.470819in}}{\pgfqpoint{1.551126in}{2.467547in}}{\pgfqpoint{1.559362in}{2.467547in}}%
\pgfpathclose%
\pgfusepath{stroke,fill}%
\end{pgfscope}%
\begin{pgfscope}%
\pgfpathrectangle{\pgfqpoint{0.100000in}{0.212622in}}{\pgfqpoint{3.696000in}{3.696000in}}%
\pgfusepath{clip}%
\pgfsetbuttcap%
\pgfsetroundjoin%
\definecolor{currentfill}{rgb}{0.121569,0.466667,0.705882}%
\pgfsetfillcolor{currentfill}%
\pgfsetfillopacity{0.329336}%
\pgfsetlinewidth{1.003750pt}%
\definecolor{currentstroke}{rgb}{0.121569,0.466667,0.705882}%
\pgfsetstrokecolor{currentstroke}%
\pgfsetstrokeopacity{0.329336}%
\pgfsetdash{}{0pt}%
\pgfpathmoveto{\pgfqpoint{1.559146in}{2.467214in}}%
\pgfpathcurveto{\pgfqpoint{1.567382in}{2.467214in}}{\pgfqpoint{1.575282in}{2.470486in}}{\pgfqpoint{1.581106in}{2.476310in}}%
\pgfpathcurveto{\pgfqpoint{1.586930in}{2.482134in}}{\pgfqpoint{1.590202in}{2.490034in}}{\pgfqpoint{1.590202in}{2.498271in}}%
\pgfpathcurveto{\pgfqpoint{1.590202in}{2.506507in}}{\pgfqpoint{1.586930in}{2.514407in}}{\pgfqpoint{1.581106in}{2.520231in}}%
\pgfpathcurveto{\pgfqpoint{1.575282in}{2.526055in}}{\pgfqpoint{1.567382in}{2.529327in}}{\pgfqpoint{1.559146in}{2.529327in}}%
\pgfpathcurveto{\pgfqpoint{1.550909in}{2.529327in}}{\pgfqpoint{1.543009in}{2.526055in}}{\pgfqpoint{1.537185in}{2.520231in}}%
\pgfpathcurveto{\pgfqpoint{1.531361in}{2.514407in}}{\pgfqpoint{1.528089in}{2.506507in}}{\pgfqpoint{1.528089in}{2.498271in}}%
\pgfpathcurveto{\pgfqpoint{1.528089in}{2.490034in}}{\pgfqpoint{1.531361in}{2.482134in}}{\pgfqpoint{1.537185in}{2.476310in}}%
\pgfpathcurveto{\pgfqpoint{1.543009in}{2.470486in}}{\pgfqpoint{1.550909in}{2.467214in}}{\pgfqpoint{1.559146in}{2.467214in}}%
\pgfpathclose%
\pgfusepath{stroke,fill}%
\end{pgfscope}%
\begin{pgfscope}%
\pgfpathrectangle{\pgfqpoint{0.100000in}{0.212622in}}{\pgfqpoint{3.696000in}{3.696000in}}%
\pgfusepath{clip}%
\pgfsetbuttcap%
\pgfsetroundjoin%
\definecolor{currentfill}{rgb}{0.121569,0.466667,0.705882}%
\pgfsetfillcolor{currentfill}%
\pgfsetfillopacity{0.329344}%
\pgfsetlinewidth{1.003750pt}%
\definecolor{currentstroke}{rgb}{0.121569,0.466667,0.705882}%
\pgfsetstrokecolor{currentstroke}%
\pgfsetstrokeopacity{0.329344}%
\pgfsetdash{}{0pt}%
\pgfpathmoveto{\pgfqpoint{1.897857in}{2.514288in}}%
\pgfpathcurveto{\pgfqpoint{1.906093in}{2.514288in}}{\pgfqpoint{1.913993in}{2.517561in}}{\pgfqpoint{1.919817in}{2.523385in}}%
\pgfpathcurveto{\pgfqpoint{1.925641in}{2.529209in}}{\pgfqpoint{1.928914in}{2.537109in}}{\pgfqpoint{1.928914in}{2.545345in}}%
\pgfpathcurveto{\pgfqpoint{1.928914in}{2.553581in}}{\pgfqpoint{1.925641in}{2.561481in}}{\pgfqpoint{1.919817in}{2.567305in}}%
\pgfpathcurveto{\pgfqpoint{1.913993in}{2.573129in}}{\pgfqpoint{1.906093in}{2.576401in}}{\pgfqpoint{1.897857in}{2.576401in}}%
\pgfpathcurveto{\pgfqpoint{1.889621in}{2.576401in}}{\pgfqpoint{1.881721in}{2.573129in}}{\pgfqpoint{1.875897in}{2.567305in}}%
\pgfpathcurveto{\pgfqpoint{1.870073in}{2.561481in}}{\pgfqpoint{1.866801in}{2.553581in}}{\pgfqpoint{1.866801in}{2.545345in}}%
\pgfpathcurveto{\pgfqpoint{1.866801in}{2.537109in}}{\pgfqpoint{1.870073in}{2.529209in}}{\pgfqpoint{1.875897in}{2.523385in}}%
\pgfpathcurveto{\pgfqpoint{1.881721in}{2.517561in}}{\pgfqpoint{1.889621in}{2.514288in}}{\pgfqpoint{1.897857in}{2.514288in}}%
\pgfpathclose%
\pgfusepath{stroke,fill}%
\end{pgfscope}%
\begin{pgfscope}%
\pgfpathrectangle{\pgfqpoint{0.100000in}{0.212622in}}{\pgfqpoint{3.696000in}{3.696000in}}%
\pgfusepath{clip}%
\pgfsetbuttcap%
\pgfsetroundjoin%
\definecolor{currentfill}{rgb}{0.121569,0.466667,0.705882}%
\pgfsetfillcolor{currentfill}%
\pgfsetfillopacity{0.329496}%
\pgfsetlinewidth{1.003750pt}%
\definecolor{currentstroke}{rgb}{0.121569,0.466667,0.705882}%
\pgfsetstrokecolor{currentstroke}%
\pgfsetstrokeopacity{0.329496}%
\pgfsetdash{}{0pt}%
\pgfpathmoveto{\pgfqpoint{1.558777in}{2.466573in}}%
\pgfpathcurveto{\pgfqpoint{1.567013in}{2.466573in}}{\pgfqpoint{1.574913in}{2.469846in}}{\pgfqpoint{1.580737in}{2.475670in}}%
\pgfpathcurveto{\pgfqpoint{1.586561in}{2.481494in}}{\pgfqpoint{1.589834in}{2.489394in}}{\pgfqpoint{1.589834in}{2.497630in}}%
\pgfpathcurveto{\pgfqpoint{1.589834in}{2.505866in}}{\pgfqpoint{1.586561in}{2.513766in}}{\pgfqpoint{1.580737in}{2.519590in}}%
\pgfpathcurveto{\pgfqpoint{1.574913in}{2.525414in}}{\pgfqpoint{1.567013in}{2.528686in}}{\pgfqpoint{1.558777in}{2.528686in}}%
\pgfpathcurveto{\pgfqpoint{1.550541in}{2.528686in}}{\pgfqpoint{1.542641in}{2.525414in}}{\pgfqpoint{1.536817in}{2.519590in}}%
\pgfpathcurveto{\pgfqpoint{1.530993in}{2.513766in}}{\pgfqpoint{1.527721in}{2.505866in}}{\pgfqpoint{1.527721in}{2.497630in}}%
\pgfpathcurveto{\pgfqpoint{1.527721in}{2.489394in}}{\pgfqpoint{1.530993in}{2.481494in}}{\pgfqpoint{1.536817in}{2.475670in}}%
\pgfpathcurveto{\pgfqpoint{1.542641in}{2.469846in}}{\pgfqpoint{1.550541in}{2.466573in}}{\pgfqpoint{1.558777in}{2.466573in}}%
\pgfpathclose%
\pgfusepath{stroke,fill}%
\end{pgfscope}%
\begin{pgfscope}%
\pgfpathrectangle{\pgfqpoint{0.100000in}{0.212622in}}{\pgfqpoint{3.696000in}{3.696000in}}%
\pgfusepath{clip}%
\pgfsetbuttcap%
\pgfsetroundjoin%
\definecolor{currentfill}{rgb}{0.121569,0.466667,0.705882}%
\pgfsetfillcolor{currentfill}%
\pgfsetfillopacity{0.329546}%
\pgfsetlinewidth{1.003750pt}%
\definecolor{currentstroke}{rgb}{0.121569,0.466667,0.705882}%
\pgfsetstrokecolor{currentstroke}%
\pgfsetstrokeopacity{0.329546}%
\pgfsetdash{}{0pt}%
\pgfpathmoveto{\pgfqpoint{1.558665in}{2.466364in}}%
\pgfpathcurveto{\pgfqpoint{1.566902in}{2.466364in}}{\pgfqpoint{1.574802in}{2.469636in}}{\pgfqpoint{1.580626in}{2.475460in}}%
\pgfpathcurveto{\pgfqpoint{1.586450in}{2.481284in}}{\pgfqpoint{1.589722in}{2.489184in}}{\pgfqpoint{1.589722in}{2.497420in}}%
\pgfpathcurveto{\pgfqpoint{1.589722in}{2.505657in}}{\pgfqpoint{1.586450in}{2.513557in}}{\pgfqpoint{1.580626in}{2.519380in}}%
\pgfpathcurveto{\pgfqpoint{1.574802in}{2.525204in}}{\pgfqpoint{1.566902in}{2.528477in}}{\pgfqpoint{1.558665in}{2.528477in}}%
\pgfpathcurveto{\pgfqpoint{1.550429in}{2.528477in}}{\pgfqpoint{1.542529in}{2.525204in}}{\pgfqpoint{1.536705in}{2.519380in}}%
\pgfpathcurveto{\pgfqpoint{1.530881in}{2.513557in}}{\pgfqpoint{1.527609in}{2.505657in}}{\pgfqpoint{1.527609in}{2.497420in}}%
\pgfpathcurveto{\pgfqpoint{1.527609in}{2.489184in}}{\pgfqpoint{1.530881in}{2.481284in}}{\pgfqpoint{1.536705in}{2.475460in}}%
\pgfpathcurveto{\pgfqpoint{1.542529in}{2.469636in}}{\pgfqpoint{1.550429in}{2.466364in}}{\pgfqpoint{1.558665in}{2.466364in}}%
\pgfpathclose%
\pgfusepath{stroke,fill}%
\end{pgfscope}%
\begin{pgfscope}%
\pgfpathrectangle{\pgfqpoint{0.100000in}{0.212622in}}{\pgfqpoint{3.696000in}{3.696000in}}%
\pgfusepath{clip}%
\pgfsetbuttcap%
\pgfsetroundjoin%
\definecolor{currentfill}{rgb}{0.121569,0.466667,0.705882}%
\pgfsetfillcolor{currentfill}%
\pgfsetfillopacity{0.329639}%
\pgfsetlinewidth{1.003750pt}%
\definecolor{currentstroke}{rgb}{0.121569,0.466667,0.705882}%
\pgfsetstrokecolor{currentstroke}%
\pgfsetstrokeopacity{0.329639}%
\pgfsetdash{}{0pt}%
\pgfpathmoveto{\pgfqpoint{1.558477in}{2.465982in}}%
\pgfpathcurveto{\pgfqpoint{1.566714in}{2.465982in}}{\pgfqpoint{1.574614in}{2.469254in}}{\pgfqpoint{1.580437in}{2.475078in}}%
\pgfpathcurveto{\pgfqpoint{1.586261in}{2.480902in}}{\pgfqpoint{1.589534in}{2.488802in}}{\pgfqpoint{1.589534in}{2.497038in}}%
\pgfpathcurveto{\pgfqpoint{1.589534in}{2.505274in}}{\pgfqpoint{1.586261in}{2.513174in}}{\pgfqpoint{1.580437in}{2.518998in}}%
\pgfpathcurveto{\pgfqpoint{1.574614in}{2.524822in}}{\pgfqpoint{1.566714in}{2.528095in}}{\pgfqpoint{1.558477in}{2.528095in}}%
\pgfpathcurveto{\pgfqpoint{1.550241in}{2.528095in}}{\pgfqpoint{1.542341in}{2.524822in}}{\pgfqpoint{1.536517in}{2.518998in}}%
\pgfpathcurveto{\pgfqpoint{1.530693in}{2.513174in}}{\pgfqpoint{1.527421in}{2.505274in}}{\pgfqpoint{1.527421in}{2.497038in}}%
\pgfpathcurveto{\pgfqpoint{1.527421in}{2.488802in}}{\pgfqpoint{1.530693in}{2.480902in}}{\pgfqpoint{1.536517in}{2.475078in}}%
\pgfpathcurveto{\pgfqpoint{1.542341in}{2.469254in}}{\pgfqpoint{1.550241in}{2.465982in}}{\pgfqpoint{1.558477in}{2.465982in}}%
\pgfpathclose%
\pgfusepath{stroke,fill}%
\end{pgfscope}%
\begin{pgfscope}%
\pgfpathrectangle{\pgfqpoint{0.100000in}{0.212622in}}{\pgfqpoint{3.696000in}{3.696000in}}%
\pgfusepath{clip}%
\pgfsetbuttcap%
\pgfsetroundjoin%
\definecolor{currentfill}{rgb}{0.121569,0.466667,0.705882}%
\pgfsetfillcolor{currentfill}%
\pgfsetfillopacity{0.329807}%
\pgfsetlinewidth{1.003750pt}%
\definecolor{currentstroke}{rgb}{0.121569,0.466667,0.705882}%
\pgfsetstrokecolor{currentstroke}%
\pgfsetstrokeopacity{0.329807}%
\pgfsetdash{}{0pt}%
\pgfpathmoveto{\pgfqpoint{1.558143in}{2.465276in}}%
\pgfpathcurveto{\pgfqpoint{1.566379in}{2.465276in}}{\pgfqpoint{1.574279in}{2.468548in}}{\pgfqpoint{1.580103in}{2.474372in}}%
\pgfpathcurveto{\pgfqpoint{1.585927in}{2.480196in}}{\pgfqpoint{1.589199in}{2.488096in}}{\pgfqpoint{1.589199in}{2.496332in}}%
\pgfpathcurveto{\pgfqpoint{1.589199in}{2.504568in}}{\pgfqpoint{1.585927in}{2.512468in}}{\pgfqpoint{1.580103in}{2.518292in}}%
\pgfpathcurveto{\pgfqpoint{1.574279in}{2.524116in}}{\pgfqpoint{1.566379in}{2.527389in}}{\pgfqpoint{1.558143in}{2.527389in}}%
\pgfpathcurveto{\pgfqpoint{1.549906in}{2.527389in}}{\pgfqpoint{1.542006in}{2.524116in}}{\pgfqpoint{1.536182in}{2.518292in}}%
\pgfpathcurveto{\pgfqpoint{1.530358in}{2.512468in}}{\pgfqpoint{1.527086in}{2.504568in}}{\pgfqpoint{1.527086in}{2.496332in}}%
\pgfpathcurveto{\pgfqpoint{1.527086in}{2.488096in}}{\pgfqpoint{1.530358in}{2.480196in}}{\pgfqpoint{1.536182in}{2.474372in}}%
\pgfpathcurveto{\pgfqpoint{1.542006in}{2.468548in}}{\pgfqpoint{1.549906in}{2.465276in}}{\pgfqpoint{1.558143in}{2.465276in}}%
\pgfpathclose%
\pgfusepath{stroke,fill}%
\end{pgfscope}%
\begin{pgfscope}%
\pgfpathrectangle{\pgfqpoint{0.100000in}{0.212622in}}{\pgfqpoint{3.696000in}{3.696000in}}%
\pgfusepath{clip}%
\pgfsetbuttcap%
\pgfsetroundjoin%
\definecolor{currentfill}{rgb}{0.121569,0.466667,0.705882}%
\pgfsetfillcolor{currentfill}%
\pgfsetfillopacity{0.330113}%
\pgfsetlinewidth{1.003750pt}%
\definecolor{currentstroke}{rgb}{0.121569,0.466667,0.705882}%
\pgfsetstrokecolor{currentstroke}%
\pgfsetstrokeopacity{0.330113}%
\pgfsetdash{}{0pt}%
\pgfpathmoveto{\pgfqpoint{1.557486in}{2.464030in}}%
\pgfpathcurveto{\pgfqpoint{1.565723in}{2.464030in}}{\pgfqpoint{1.573623in}{2.467302in}}{\pgfqpoint{1.579447in}{2.473126in}}%
\pgfpathcurveto{\pgfqpoint{1.585271in}{2.478950in}}{\pgfqpoint{1.588543in}{2.486850in}}{\pgfqpoint{1.588543in}{2.495086in}}%
\pgfpathcurveto{\pgfqpoint{1.588543in}{2.503322in}}{\pgfqpoint{1.585271in}{2.511222in}}{\pgfqpoint{1.579447in}{2.517046in}}%
\pgfpathcurveto{\pgfqpoint{1.573623in}{2.522870in}}{\pgfqpoint{1.565723in}{2.526143in}}{\pgfqpoint{1.557486in}{2.526143in}}%
\pgfpathcurveto{\pgfqpoint{1.549250in}{2.526143in}}{\pgfqpoint{1.541350in}{2.522870in}}{\pgfqpoint{1.535526in}{2.517046in}}%
\pgfpathcurveto{\pgfqpoint{1.529702in}{2.511222in}}{\pgfqpoint{1.526430in}{2.503322in}}{\pgfqpoint{1.526430in}{2.495086in}}%
\pgfpathcurveto{\pgfqpoint{1.526430in}{2.486850in}}{\pgfqpoint{1.529702in}{2.478950in}}{\pgfqpoint{1.535526in}{2.473126in}}%
\pgfpathcurveto{\pgfqpoint{1.541350in}{2.467302in}}{\pgfqpoint{1.549250in}{2.464030in}}{\pgfqpoint{1.557486in}{2.464030in}}%
\pgfpathclose%
\pgfusepath{stroke,fill}%
\end{pgfscope}%
\begin{pgfscope}%
\pgfpathrectangle{\pgfqpoint{0.100000in}{0.212622in}}{\pgfqpoint{3.696000in}{3.696000in}}%
\pgfusepath{clip}%
\pgfsetbuttcap%
\pgfsetroundjoin%
\definecolor{currentfill}{rgb}{0.121569,0.466667,0.705882}%
\pgfsetfillcolor{currentfill}%
\pgfsetfillopacity{0.330679}%
\pgfsetlinewidth{1.003750pt}%
\definecolor{currentstroke}{rgb}{0.121569,0.466667,0.705882}%
\pgfsetstrokecolor{currentstroke}%
\pgfsetstrokeopacity{0.330679}%
\pgfsetdash{}{0pt}%
\pgfpathmoveto{\pgfqpoint{1.556388in}{2.461763in}}%
\pgfpathcurveto{\pgfqpoint{1.564624in}{2.461763in}}{\pgfqpoint{1.572524in}{2.465035in}}{\pgfqpoint{1.578348in}{2.470859in}}%
\pgfpathcurveto{\pgfqpoint{1.584172in}{2.476683in}}{\pgfqpoint{1.587444in}{2.484583in}}{\pgfqpoint{1.587444in}{2.492819in}}%
\pgfpathcurveto{\pgfqpoint{1.587444in}{2.501055in}}{\pgfqpoint{1.584172in}{2.508955in}}{\pgfqpoint{1.578348in}{2.514779in}}%
\pgfpathcurveto{\pgfqpoint{1.572524in}{2.520603in}}{\pgfqpoint{1.564624in}{2.523876in}}{\pgfqpoint{1.556388in}{2.523876in}}%
\pgfpathcurveto{\pgfqpoint{1.548152in}{2.523876in}}{\pgfqpoint{1.540251in}{2.520603in}}{\pgfqpoint{1.534428in}{2.514779in}}%
\pgfpathcurveto{\pgfqpoint{1.528604in}{2.508955in}}{\pgfqpoint{1.525331in}{2.501055in}}{\pgfqpoint{1.525331in}{2.492819in}}%
\pgfpathcurveto{\pgfqpoint{1.525331in}{2.484583in}}{\pgfqpoint{1.528604in}{2.476683in}}{\pgfqpoint{1.534428in}{2.470859in}}%
\pgfpathcurveto{\pgfqpoint{1.540251in}{2.465035in}}{\pgfqpoint{1.548152in}{2.461763in}}{\pgfqpoint{1.556388in}{2.461763in}}%
\pgfpathclose%
\pgfusepath{stroke,fill}%
\end{pgfscope}%
\begin{pgfscope}%
\pgfpathrectangle{\pgfqpoint{0.100000in}{0.212622in}}{\pgfqpoint{3.696000in}{3.696000in}}%
\pgfusepath{clip}%
\pgfsetbuttcap%
\pgfsetroundjoin%
\definecolor{currentfill}{rgb}{0.121569,0.466667,0.705882}%
\pgfsetfillcolor{currentfill}%
\pgfsetfillopacity{0.330869}%
\pgfsetlinewidth{1.003750pt}%
\definecolor{currentstroke}{rgb}{0.121569,0.466667,0.705882}%
\pgfsetstrokecolor{currentstroke}%
\pgfsetstrokeopacity{0.330869}%
\pgfsetdash{}{0pt}%
\pgfpathmoveto{\pgfqpoint{1.555985in}{2.460970in}}%
\pgfpathcurveto{\pgfqpoint{1.564221in}{2.460970in}}{\pgfqpoint{1.572121in}{2.464243in}}{\pgfqpoint{1.577945in}{2.470066in}}%
\pgfpathcurveto{\pgfqpoint{1.583769in}{2.475890in}}{\pgfqpoint{1.587041in}{2.483790in}}{\pgfqpoint{1.587041in}{2.492027in}}%
\pgfpathcurveto{\pgfqpoint{1.587041in}{2.500263in}}{\pgfqpoint{1.583769in}{2.508163in}}{\pgfqpoint{1.577945in}{2.513987in}}%
\pgfpathcurveto{\pgfqpoint{1.572121in}{2.519811in}}{\pgfqpoint{1.564221in}{2.523083in}}{\pgfqpoint{1.555985in}{2.523083in}}%
\pgfpathcurveto{\pgfqpoint{1.547749in}{2.523083in}}{\pgfqpoint{1.539849in}{2.519811in}}{\pgfqpoint{1.534025in}{2.513987in}}%
\pgfpathcurveto{\pgfqpoint{1.528201in}{2.508163in}}{\pgfqpoint{1.524928in}{2.500263in}}{\pgfqpoint{1.524928in}{2.492027in}}%
\pgfpathcurveto{\pgfqpoint{1.524928in}{2.483790in}}{\pgfqpoint{1.528201in}{2.475890in}}{\pgfqpoint{1.534025in}{2.470066in}}%
\pgfpathcurveto{\pgfqpoint{1.539849in}{2.464243in}}{\pgfqpoint{1.547749in}{2.460970in}}{\pgfqpoint{1.555985in}{2.460970in}}%
\pgfpathclose%
\pgfusepath{stroke,fill}%
\end{pgfscope}%
\begin{pgfscope}%
\pgfpathrectangle{\pgfqpoint{0.100000in}{0.212622in}}{\pgfqpoint{3.696000in}{3.696000in}}%
\pgfusepath{clip}%
\pgfsetbuttcap%
\pgfsetroundjoin%
\definecolor{currentfill}{rgb}{0.121569,0.466667,0.705882}%
\pgfsetfillcolor{currentfill}%
\pgfsetfillopacity{0.331203}%
\pgfsetlinewidth{1.003750pt}%
\definecolor{currentstroke}{rgb}{0.121569,0.466667,0.705882}%
\pgfsetstrokecolor{currentstroke}%
\pgfsetstrokeopacity{0.331203}%
\pgfsetdash{}{0pt}%
\pgfpathmoveto{\pgfqpoint{1.909247in}{2.513022in}}%
\pgfpathcurveto{\pgfqpoint{1.917484in}{2.513022in}}{\pgfqpoint{1.925384in}{2.516294in}}{\pgfqpoint{1.931208in}{2.522118in}}%
\pgfpathcurveto{\pgfqpoint{1.937032in}{2.527942in}}{\pgfqpoint{1.940304in}{2.535842in}}{\pgfqpoint{1.940304in}{2.544079in}}%
\pgfpathcurveto{\pgfqpoint{1.940304in}{2.552315in}}{\pgfqpoint{1.937032in}{2.560215in}}{\pgfqpoint{1.931208in}{2.566039in}}%
\pgfpathcurveto{\pgfqpoint{1.925384in}{2.571863in}}{\pgfqpoint{1.917484in}{2.575135in}}{\pgfqpoint{1.909247in}{2.575135in}}%
\pgfpathcurveto{\pgfqpoint{1.901011in}{2.575135in}}{\pgfqpoint{1.893111in}{2.571863in}}{\pgfqpoint{1.887287in}{2.566039in}}%
\pgfpathcurveto{\pgfqpoint{1.881463in}{2.560215in}}{\pgfqpoint{1.878191in}{2.552315in}}{\pgfqpoint{1.878191in}{2.544079in}}%
\pgfpathcurveto{\pgfqpoint{1.878191in}{2.535842in}}{\pgfqpoint{1.881463in}{2.527942in}}{\pgfqpoint{1.887287in}{2.522118in}}%
\pgfpathcurveto{\pgfqpoint{1.893111in}{2.516294in}}{\pgfqpoint{1.901011in}{2.513022in}}{\pgfqpoint{1.909247in}{2.513022in}}%
\pgfpathclose%
\pgfusepath{stroke,fill}%
\end{pgfscope}%
\begin{pgfscope}%
\pgfpathrectangle{\pgfqpoint{0.100000in}{0.212622in}}{\pgfqpoint{3.696000in}{3.696000in}}%
\pgfusepath{clip}%
\pgfsetbuttcap%
\pgfsetroundjoin%
\definecolor{currentfill}{rgb}{0.121569,0.466667,0.705882}%
\pgfsetfillcolor{currentfill}%
\pgfsetfillopacity{0.331222}%
\pgfsetlinewidth{1.003750pt}%
\definecolor{currentstroke}{rgb}{0.121569,0.466667,0.705882}%
\pgfsetstrokecolor{currentstroke}%
\pgfsetstrokeopacity{0.331222}%
\pgfsetdash{}{0pt}%
\pgfpathmoveto{\pgfqpoint{1.555262in}{2.459576in}}%
\pgfpathcurveto{\pgfqpoint{1.563498in}{2.459576in}}{\pgfqpoint{1.571398in}{2.462848in}}{\pgfqpoint{1.577222in}{2.468672in}}%
\pgfpathcurveto{\pgfqpoint{1.583046in}{2.474496in}}{\pgfqpoint{1.586318in}{2.482396in}}{\pgfqpoint{1.586318in}{2.490632in}}%
\pgfpathcurveto{\pgfqpoint{1.586318in}{2.498868in}}{\pgfqpoint{1.583046in}{2.506768in}}{\pgfqpoint{1.577222in}{2.512592in}}%
\pgfpathcurveto{\pgfqpoint{1.571398in}{2.518416in}}{\pgfqpoint{1.563498in}{2.521689in}}{\pgfqpoint{1.555262in}{2.521689in}}%
\pgfpathcurveto{\pgfqpoint{1.547025in}{2.521689in}}{\pgfqpoint{1.539125in}{2.518416in}}{\pgfqpoint{1.533301in}{2.512592in}}%
\pgfpathcurveto{\pgfqpoint{1.527477in}{2.506768in}}{\pgfqpoint{1.524205in}{2.498868in}}{\pgfqpoint{1.524205in}{2.490632in}}%
\pgfpathcurveto{\pgfqpoint{1.524205in}{2.482396in}}{\pgfqpoint{1.527477in}{2.474496in}}{\pgfqpoint{1.533301in}{2.468672in}}%
\pgfpathcurveto{\pgfqpoint{1.539125in}{2.462848in}}{\pgfqpoint{1.547025in}{2.459576in}}{\pgfqpoint{1.555262in}{2.459576in}}%
\pgfpathclose%
\pgfusepath{stroke,fill}%
\end{pgfscope}%
\begin{pgfscope}%
\pgfpathrectangle{\pgfqpoint{0.100000in}{0.212622in}}{\pgfqpoint{3.696000in}{3.696000in}}%
\pgfusepath{clip}%
\pgfsetbuttcap%
\pgfsetroundjoin%
\definecolor{currentfill}{rgb}{0.121569,0.466667,0.705882}%
\pgfsetfillcolor{currentfill}%
\pgfsetfillopacity{0.331845}%
\pgfsetlinewidth{1.003750pt}%
\definecolor{currentstroke}{rgb}{0.121569,0.466667,0.705882}%
\pgfsetstrokecolor{currentstroke}%
\pgfsetstrokeopacity{0.331845}%
\pgfsetdash{}{0pt}%
\pgfpathmoveto{\pgfqpoint{1.553899in}{2.456937in}}%
\pgfpathcurveto{\pgfqpoint{1.562136in}{2.456937in}}{\pgfqpoint{1.570036in}{2.460209in}}{\pgfqpoint{1.575860in}{2.466033in}}%
\pgfpathcurveto{\pgfqpoint{1.581683in}{2.471857in}}{\pgfqpoint{1.584956in}{2.479757in}}{\pgfqpoint{1.584956in}{2.487994in}}%
\pgfpathcurveto{\pgfqpoint{1.584956in}{2.496230in}}{\pgfqpoint{1.581683in}{2.504130in}}{\pgfqpoint{1.575860in}{2.509954in}}%
\pgfpathcurveto{\pgfqpoint{1.570036in}{2.515778in}}{\pgfqpoint{1.562136in}{2.519050in}}{\pgfqpoint{1.553899in}{2.519050in}}%
\pgfpathcurveto{\pgfqpoint{1.545663in}{2.519050in}}{\pgfqpoint{1.537763in}{2.515778in}}{\pgfqpoint{1.531939in}{2.509954in}}%
\pgfpathcurveto{\pgfqpoint{1.526115in}{2.504130in}}{\pgfqpoint{1.522843in}{2.496230in}}{\pgfqpoint{1.522843in}{2.487994in}}%
\pgfpathcurveto{\pgfqpoint{1.522843in}{2.479757in}}{\pgfqpoint{1.526115in}{2.471857in}}{\pgfqpoint{1.531939in}{2.466033in}}%
\pgfpathcurveto{\pgfqpoint{1.537763in}{2.460209in}}{\pgfqpoint{1.545663in}{2.456937in}}{\pgfqpoint{1.553899in}{2.456937in}}%
\pgfpathclose%
\pgfusepath{stroke,fill}%
\end{pgfscope}%
\begin{pgfscope}%
\pgfpathrectangle{\pgfqpoint{0.100000in}{0.212622in}}{\pgfqpoint{3.696000in}{3.696000in}}%
\pgfusepath{clip}%
\pgfsetbuttcap%
\pgfsetroundjoin%
\definecolor{currentfill}{rgb}{0.121569,0.466667,0.705882}%
\pgfsetfillcolor{currentfill}%
\pgfsetfillopacity{0.332973}%
\pgfsetlinewidth{1.003750pt}%
\definecolor{currentstroke}{rgb}{0.121569,0.466667,0.705882}%
\pgfsetstrokecolor{currentstroke}%
\pgfsetstrokeopacity{0.332973}%
\pgfsetdash{}{0pt}%
\pgfpathmoveto{\pgfqpoint{1.551297in}{2.452205in}}%
\pgfpathcurveto{\pgfqpoint{1.559533in}{2.452205in}}{\pgfqpoint{1.567433in}{2.455477in}}{\pgfqpoint{1.573257in}{2.461301in}}%
\pgfpathcurveto{\pgfqpoint{1.579081in}{2.467125in}}{\pgfqpoint{1.582354in}{2.475025in}}{\pgfqpoint{1.582354in}{2.483261in}}%
\pgfpathcurveto{\pgfqpoint{1.582354in}{2.491498in}}{\pgfqpoint{1.579081in}{2.499398in}}{\pgfqpoint{1.573257in}{2.505221in}}%
\pgfpathcurveto{\pgfqpoint{1.567433in}{2.511045in}}{\pgfqpoint{1.559533in}{2.514318in}}{\pgfqpoint{1.551297in}{2.514318in}}%
\pgfpathcurveto{\pgfqpoint{1.543061in}{2.514318in}}{\pgfqpoint{1.535161in}{2.511045in}}{\pgfqpoint{1.529337in}{2.505221in}}%
\pgfpathcurveto{\pgfqpoint{1.523513in}{2.499398in}}{\pgfqpoint{1.520241in}{2.491498in}}{\pgfqpoint{1.520241in}{2.483261in}}%
\pgfpathcurveto{\pgfqpoint{1.520241in}{2.475025in}}{\pgfqpoint{1.523513in}{2.467125in}}{\pgfqpoint{1.529337in}{2.461301in}}%
\pgfpathcurveto{\pgfqpoint{1.535161in}{2.455477in}}{\pgfqpoint{1.543061in}{2.452205in}}{\pgfqpoint{1.551297in}{2.452205in}}%
\pgfpathclose%
\pgfusepath{stroke,fill}%
\end{pgfscope}%
\begin{pgfscope}%
\pgfpathrectangle{\pgfqpoint{0.100000in}{0.212622in}}{\pgfqpoint{3.696000in}{3.696000in}}%
\pgfusepath{clip}%
\pgfsetbuttcap%
\pgfsetroundjoin%
\definecolor{currentfill}{rgb}{0.121569,0.466667,0.705882}%
\pgfsetfillcolor{currentfill}%
\pgfsetfillopacity{0.333212}%
\pgfsetlinewidth{1.003750pt}%
\definecolor{currentstroke}{rgb}{0.121569,0.466667,0.705882}%
\pgfsetstrokecolor{currentstroke}%
\pgfsetstrokeopacity{0.333212}%
\pgfsetdash{}{0pt}%
\pgfpathmoveto{\pgfqpoint{1.921442in}{2.511869in}}%
\pgfpathcurveto{\pgfqpoint{1.929678in}{2.511869in}}{\pgfqpoint{1.937578in}{2.515141in}}{\pgfqpoint{1.943402in}{2.520965in}}%
\pgfpathcurveto{\pgfqpoint{1.949226in}{2.526789in}}{\pgfqpoint{1.952498in}{2.534689in}}{\pgfqpoint{1.952498in}{2.542925in}}%
\pgfpathcurveto{\pgfqpoint{1.952498in}{2.551162in}}{\pgfqpoint{1.949226in}{2.559062in}}{\pgfqpoint{1.943402in}{2.564886in}}%
\pgfpathcurveto{\pgfqpoint{1.937578in}{2.570710in}}{\pgfqpoint{1.929678in}{2.573982in}}{\pgfqpoint{1.921442in}{2.573982in}}%
\pgfpathcurveto{\pgfqpoint{1.913206in}{2.573982in}}{\pgfqpoint{1.905306in}{2.570710in}}{\pgfqpoint{1.899482in}{2.564886in}}%
\pgfpathcurveto{\pgfqpoint{1.893658in}{2.559062in}}{\pgfqpoint{1.890385in}{2.551162in}}{\pgfqpoint{1.890385in}{2.542925in}}%
\pgfpathcurveto{\pgfqpoint{1.890385in}{2.534689in}}{\pgfqpoint{1.893658in}{2.526789in}}{\pgfqpoint{1.899482in}{2.520965in}}%
\pgfpathcurveto{\pgfqpoint{1.905306in}{2.515141in}}{\pgfqpoint{1.913206in}{2.511869in}}{\pgfqpoint{1.921442in}{2.511869in}}%
\pgfpathclose%
\pgfusepath{stroke,fill}%
\end{pgfscope}%
\begin{pgfscope}%
\pgfpathrectangle{\pgfqpoint{0.100000in}{0.212622in}}{\pgfqpoint{3.696000in}{3.696000in}}%
\pgfusepath{clip}%
\pgfsetbuttcap%
\pgfsetroundjoin%
\definecolor{currentfill}{rgb}{0.121569,0.466667,0.705882}%
\pgfsetfillcolor{currentfill}%
\pgfsetfillopacity{0.334003}%
\pgfsetlinewidth{1.003750pt}%
\definecolor{currentstroke}{rgb}{0.121569,0.466667,0.705882}%
\pgfsetstrokecolor{currentstroke}%
\pgfsetstrokeopacity{0.334003}%
\pgfsetdash{}{0pt}%
\pgfpathmoveto{\pgfqpoint{1.549195in}{2.448316in}}%
\pgfpathcurveto{\pgfqpoint{1.557431in}{2.448316in}}{\pgfqpoint{1.565331in}{2.451589in}}{\pgfqpoint{1.571155in}{2.457413in}}%
\pgfpathcurveto{\pgfqpoint{1.576979in}{2.463236in}}{\pgfqpoint{1.580252in}{2.471137in}}{\pgfqpoint{1.580252in}{2.479373in}}%
\pgfpathcurveto{\pgfqpoint{1.580252in}{2.487609in}}{\pgfqpoint{1.576979in}{2.495509in}}{\pgfqpoint{1.571155in}{2.501333in}}%
\pgfpathcurveto{\pgfqpoint{1.565331in}{2.507157in}}{\pgfqpoint{1.557431in}{2.510429in}}{\pgfqpoint{1.549195in}{2.510429in}}%
\pgfpathcurveto{\pgfqpoint{1.540959in}{2.510429in}}{\pgfqpoint{1.533059in}{2.507157in}}{\pgfqpoint{1.527235in}{2.501333in}}%
\pgfpathcurveto{\pgfqpoint{1.521411in}{2.495509in}}{\pgfqpoint{1.518139in}{2.487609in}}{\pgfqpoint{1.518139in}{2.479373in}}%
\pgfpathcurveto{\pgfqpoint{1.518139in}{2.471137in}}{\pgfqpoint{1.521411in}{2.463236in}}{\pgfqpoint{1.527235in}{2.457413in}}%
\pgfpathcurveto{\pgfqpoint{1.533059in}{2.451589in}}{\pgfqpoint{1.540959in}{2.448316in}}{\pgfqpoint{1.549195in}{2.448316in}}%
\pgfpathclose%
\pgfusepath{stroke,fill}%
\end{pgfscope}%
\begin{pgfscope}%
\pgfpathrectangle{\pgfqpoint{0.100000in}{0.212622in}}{\pgfqpoint{3.696000in}{3.696000in}}%
\pgfusepath{clip}%
\pgfsetbuttcap%
\pgfsetroundjoin%
\definecolor{currentfill}{rgb}{0.121569,0.466667,0.705882}%
\pgfsetfillcolor{currentfill}%
\pgfsetfillopacity{0.334742}%
\pgfsetlinewidth{1.003750pt}%
\definecolor{currentstroke}{rgb}{0.121569,0.466667,0.705882}%
\pgfsetstrokecolor{currentstroke}%
\pgfsetstrokeopacity{0.334742}%
\pgfsetdash{}{0pt}%
\pgfpathmoveto{\pgfqpoint{1.547609in}{2.445729in}}%
\pgfpathcurveto{\pgfqpoint{1.555845in}{2.445729in}}{\pgfqpoint{1.563745in}{2.449002in}}{\pgfqpoint{1.569569in}{2.454826in}}%
\pgfpathcurveto{\pgfqpoint{1.575393in}{2.460649in}}{\pgfqpoint{1.578665in}{2.468550in}}{\pgfqpoint{1.578665in}{2.476786in}}%
\pgfpathcurveto{\pgfqpoint{1.578665in}{2.485022in}}{\pgfqpoint{1.575393in}{2.492922in}}{\pgfqpoint{1.569569in}{2.498746in}}%
\pgfpathcurveto{\pgfqpoint{1.563745in}{2.504570in}}{\pgfqpoint{1.555845in}{2.507842in}}{\pgfqpoint{1.547609in}{2.507842in}}%
\pgfpathcurveto{\pgfqpoint{1.539372in}{2.507842in}}{\pgfqpoint{1.531472in}{2.504570in}}{\pgfqpoint{1.525648in}{2.498746in}}%
\pgfpathcurveto{\pgfqpoint{1.519825in}{2.492922in}}{\pgfqpoint{1.516552in}{2.485022in}}{\pgfqpoint{1.516552in}{2.476786in}}%
\pgfpathcurveto{\pgfqpoint{1.516552in}{2.468550in}}{\pgfqpoint{1.519825in}{2.460649in}}{\pgfqpoint{1.525648in}{2.454826in}}%
\pgfpathcurveto{\pgfqpoint{1.531472in}{2.449002in}}{\pgfqpoint{1.539372in}{2.445729in}}{\pgfqpoint{1.547609in}{2.445729in}}%
\pgfpathclose%
\pgfusepath{stroke,fill}%
\end{pgfscope}%
\begin{pgfscope}%
\pgfpathrectangle{\pgfqpoint{0.100000in}{0.212622in}}{\pgfqpoint{3.696000in}{3.696000in}}%
\pgfusepath{clip}%
\pgfsetbuttcap%
\pgfsetroundjoin%
\definecolor{currentfill}{rgb}{0.121569,0.466667,0.705882}%
\pgfsetfillcolor{currentfill}%
\pgfsetfillopacity{0.335265}%
\pgfsetlinewidth{1.003750pt}%
\definecolor{currentstroke}{rgb}{0.121569,0.466667,0.705882}%
\pgfsetstrokecolor{currentstroke}%
\pgfsetstrokeopacity{0.335265}%
\pgfsetdash{}{0pt}%
\pgfpathmoveto{\pgfqpoint{1.546561in}{2.443794in}}%
\pgfpathcurveto{\pgfqpoint{1.554797in}{2.443794in}}{\pgfqpoint{1.562698in}{2.447066in}}{\pgfqpoint{1.568521in}{2.452890in}}%
\pgfpathcurveto{\pgfqpoint{1.574345in}{2.458714in}}{\pgfqpoint{1.577618in}{2.466614in}}{\pgfqpoint{1.577618in}{2.474850in}}%
\pgfpathcurveto{\pgfqpoint{1.577618in}{2.483086in}}{\pgfqpoint{1.574345in}{2.490986in}}{\pgfqpoint{1.568521in}{2.496810in}}%
\pgfpathcurveto{\pgfqpoint{1.562698in}{2.502634in}}{\pgfqpoint{1.554797in}{2.505907in}}{\pgfqpoint{1.546561in}{2.505907in}}%
\pgfpathcurveto{\pgfqpoint{1.538325in}{2.505907in}}{\pgfqpoint{1.530425in}{2.502634in}}{\pgfqpoint{1.524601in}{2.496810in}}%
\pgfpathcurveto{\pgfqpoint{1.518777in}{2.490986in}}{\pgfqpoint{1.515505in}{2.483086in}}{\pgfqpoint{1.515505in}{2.474850in}}%
\pgfpathcurveto{\pgfqpoint{1.515505in}{2.466614in}}{\pgfqpoint{1.518777in}{2.458714in}}{\pgfqpoint{1.524601in}{2.452890in}}%
\pgfpathcurveto{\pgfqpoint{1.530425in}{2.447066in}}{\pgfqpoint{1.538325in}{2.443794in}}{\pgfqpoint{1.546561in}{2.443794in}}%
\pgfpathclose%
\pgfusepath{stroke,fill}%
\end{pgfscope}%
\begin{pgfscope}%
\pgfpathrectangle{\pgfqpoint{0.100000in}{0.212622in}}{\pgfqpoint{3.696000in}{3.696000in}}%
\pgfusepath{clip}%
\pgfsetbuttcap%
\pgfsetroundjoin%
\definecolor{currentfill}{rgb}{0.121569,0.466667,0.705882}%
\pgfsetfillcolor{currentfill}%
\pgfsetfillopacity{0.335446}%
\pgfsetlinewidth{1.003750pt}%
\definecolor{currentstroke}{rgb}{0.121569,0.466667,0.705882}%
\pgfsetstrokecolor{currentstroke}%
\pgfsetstrokeopacity{0.335446}%
\pgfsetdash{}{0pt}%
\pgfpathmoveto{\pgfqpoint{1.546174in}{2.443100in}}%
\pgfpathcurveto{\pgfqpoint{1.554410in}{2.443100in}}{\pgfqpoint{1.562310in}{2.446372in}}{\pgfqpoint{1.568134in}{2.452196in}}%
\pgfpathcurveto{\pgfqpoint{1.573958in}{2.458020in}}{\pgfqpoint{1.577231in}{2.465920in}}{\pgfqpoint{1.577231in}{2.474156in}}%
\pgfpathcurveto{\pgfqpoint{1.577231in}{2.482392in}}{\pgfqpoint{1.573958in}{2.490292in}}{\pgfqpoint{1.568134in}{2.496116in}}%
\pgfpathcurveto{\pgfqpoint{1.562310in}{2.501940in}}{\pgfqpoint{1.554410in}{2.505213in}}{\pgfqpoint{1.546174in}{2.505213in}}%
\pgfpathcurveto{\pgfqpoint{1.537938in}{2.505213in}}{\pgfqpoint{1.530038in}{2.501940in}}{\pgfqpoint{1.524214in}{2.496116in}}%
\pgfpathcurveto{\pgfqpoint{1.518390in}{2.490292in}}{\pgfqpoint{1.515118in}{2.482392in}}{\pgfqpoint{1.515118in}{2.474156in}}%
\pgfpathcurveto{\pgfqpoint{1.515118in}{2.465920in}}{\pgfqpoint{1.518390in}{2.458020in}}{\pgfqpoint{1.524214in}{2.452196in}}%
\pgfpathcurveto{\pgfqpoint{1.530038in}{2.446372in}}{\pgfqpoint{1.537938in}{2.443100in}}{\pgfqpoint{1.546174in}{2.443100in}}%
\pgfpathclose%
\pgfusepath{stroke,fill}%
\end{pgfscope}%
\begin{pgfscope}%
\pgfpathrectangle{\pgfqpoint{0.100000in}{0.212622in}}{\pgfqpoint{3.696000in}{3.696000in}}%
\pgfusepath{clip}%
\pgfsetbuttcap%
\pgfsetroundjoin%
\definecolor{currentfill}{rgb}{0.121569,0.466667,0.705882}%
\pgfsetfillcolor{currentfill}%
\pgfsetfillopacity{0.335535}%
\pgfsetlinewidth{1.003750pt}%
\definecolor{currentstroke}{rgb}{0.121569,0.466667,0.705882}%
\pgfsetstrokecolor{currentstroke}%
\pgfsetstrokeopacity{0.335535}%
\pgfsetdash{}{0pt}%
\pgfpathmoveto{\pgfqpoint{1.935793in}{2.510113in}}%
\pgfpathcurveto{\pgfqpoint{1.944029in}{2.510113in}}{\pgfqpoint{1.951929in}{2.513385in}}{\pgfqpoint{1.957753in}{2.519209in}}%
\pgfpathcurveto{\pgfqpoint{1.963577in}{2.525033in}}{\pgfqpoint{1.966849in}{2.532933in}}{\pgfqpoint{1.966849in}{2.541170in}}%
\pgfpathcurveto{\pgfqpoint{1.966849in}{2.549406in}}{\pgfqpoint{1.963577in}{2.557306in}}{\pgfqpoint{1.957753in}{2.563130in}}%
\pgfpathcurveto{\pgfqpoint{1.951929in}{2.568954in}}{\pgfqpoint{1.944029in}{2.572226in}}{\pgfqpoint{1.935793in}{2.572226in}}%
\pgfpathcurveto{\pgfqpoint{1.927557in}{2.572226in}}{\pgfqpoint{1.919657in}{2.568954in}}{\pgfqpoint{1.913833in}{2.563130in}}%
\pgfpathcurveto{\pgfqpoint{1.908009in}{2.557306in}}{\pgfqpoint{1.904736in}{2.549406in}}{\pgfqpoint{1.904736in}{2.541170in}}%
\pgfpathcurveto{\pgfqpoint{1.904736in}{2.532933in}}{\pgfqpoint{1.908009in}{2.525033in}}{\pgfqpoint{1.913833in}{2.519209in}}%
\pgfpathcurveto{\pgfqpoint{1.919657in}{2.513385in}}{\pgfqpoint{1.927557in}{2.510113in}}{\pgfqpoint{1.935793in}{2.510113in}}%
\pgfpathclose%
\pgfusepath{stroke,fill}%
\end{pgfscope}%
\begin{pgfscope}%
\pgfpathrectangle{\pgfqpoint{0.100000in}{0.212622in}}{\pgfqpoint{3.696000in}{3.696000in}}%
\pgfusepath{clip}%
\pgfsetbuttcap%
\pgfsetroundjoin%
\definecolor{currentfill}{rgb}{0.121569,0.466667,0.705882}%
\pgfsetfillcolor{currentfill}%
\pgfsetfillopacity{0.335781}%
\pgfsetlinewidth{1.003750pt}%
\definecolor{currentstroke}{rgb}{0.121569,0.466667,0.705882}%
\pgfsetstrokecolor{currentstroke}%
\pgfsetstrokeopacity{0.335781}%
\pgfsetdash{}{0pt}%
\pgfpathmoveto{\pgfqpoint{1.545502in}{2.441869in}}%
\pgfpathcurveto{\pgfqpoint{1.553738in}{2.441869in}}{\pgfqpoint{1.561638in}{2.445141in}}{\pgfqpoint{1.567462in}{2.450965in}}%
\pgfpathcurveto{\pgfqpoint{1.573286in}{2.456789in}}{\pgfqpoint{1.576558in}{2.464689in}}{\pgfqpoint{1.576558in}{2.472925in}}%
\pgfpathcurveto{\pgfqpoint{1.576558in}{2.481162in}}{\pgfqpoint{1.573286in}{2.489062in}}{\pgfqpoint{1.567462in}{2.494886in}}%
\pgfpathcurveto{\pgfqpoint{1.561638in}{2.500709in}}{\pgfqpoint{1.553738in}{2.503982in}}{\pgfqpoint{1.545502in}{2.503982in}}%
\pgfpathcurveto{\pgfqpoint{1.537265in}{2.503982in}}{\pgfqpoint{1.529365in}{2.500709in}}{\pgfqpoint{1.523541in}{2.494886in}}%
\pgfpathcurveto{\pgfqpoint{1.517717in}{2.489062in}}{\pgfqpoint{1.514445in}{2.481162in}}{\pgfqpoint{1.514445in}{2.472925in}}%
\pgfpathcurveto{\pgfqpoint{1.514445in}{2.464689in}}{\pgfqpoint{1.517717in}{2.456789in}}{\pgfqpoint{1.523541in}{2.450965in}}%
\pgfpathcurveto{\pgfqpoint{1.529365in}{2.445141in}}{\pgfqpoint{1.537265in}{2.441869in}}{\pgfqpoint{1.545502in}{2.441869in}}%
\pgfpathclose%
\pgfusepath{stroke,fill}%
\end{pgfscope}%
\begin{pgfscope}%
\pgfpathrectangle{\pgfqpoint{0.100000in}{0.212622in}}{\pgfqpoint{3.696000in}{3.696000in}}%
\pgfusepath{clip}%
\pgfsetbuttcap%
\pgfsetroundjoin%
\definecolor{currentfill}{rgb}{0.121569,0.466667,0.705882}%
\pgfsetfillcolor{currentfill}%
\pgfsetfillopacity{0.336376}%
\pgfsetlinewidth{1.003750pt}%
\definecolor{currentstroke}{rgb}{0.121569,0.466667,0.705882}%
\pgfsetstrokecolor{currentstroke}%
\pgfsetstrokeopacity{0.336376}%
\pgfsetdash{}{0pt}%
\pgfpathmoveto{\pgfqpoint{1.544273in}{2.439527in}}%
\pgfpathcurveto{\pgfqpoint{1.552510in}{2.439527in}}{\pgfqpoint{1.560410in}{2.442799in}}{\pgfqpoint{1.566234in}{2.448623in}}%
\pgfpathcurveto{\pgfqpoint{1.572058in}{2.454447in}}{\pgfqpoint{1.575330in}{2.462347in}}{\pgfqpoint{1.575330in}{2.470583in}}%
\pgfpathcurveto{\pgfqpoint{1.575330in}{2.478819in}}{\pgfqpoint{1.572058in}{2.486720in}}{\pgfqpoint{1.566234in}{2.492543in}}%
\pgfpathcurveto{\pgfqpoint{1.560410in}{2.498367in}}{\pgfqpoint{1.552510in}{2.501640in}}{\pgfqpoint{1.544273in}{2.501640in}}%
\pgfpathcurveto{\pgfqpoint{1.536037in}{2.501640in}}{\pgfqpoint{1.528137in}{2.498367in}}{\pgfqpoint{1.522313in}{2.492543in}}%
\pgfpathcurveto{\pgfqpoint{1.516489in}{2.486720in}}{\pgfqpoint{1.513217in}{2.478819in}}{\pgfqpoint{1.513217in}{2.470583in}}%
\pgfpathcurveto{\pgfqpoint{1.513217in}{2.462347in}}{\pgfqpoint{1.516489in}{2.454447in}}{\pgfqpoint{1.522313in}{2.448623in}}%
\pgfpathcurveto{\pgfqpoint{1.528137in}{2.442799in}}{\pgfqpoint{1.536037in}{2.439527in}}{\pgfqpoint{1.544273in}{2.439527in}}%
\pgfpathclose%
\pgfusepath{stroke,fill}%
\end{pgfscope}%
\begin{pgfscope}%
\pgfpathrectangle{\pgfqpoint{0.100000in}{0.212622in}}{\pgfqpoint{3.696000in}{3.696000in}}%
\pgfusepath{clip}%
\pgfsetbuttcap%
\pgfsetroundjoin%
\definecolor{currentfill}{rgb}{0.121569,0.466667,0.705882}%
\pgfsetfillcolor{currentfill}%
\pgfsetfillopacity{0.337458}%
\pgfsetlinewidth{1.003750pt}%
\definecolor{currentstroke}{rgb}{0.121569,0.466667,0.705882}%
\pgfsetstrokecolor{currentstroke}%
\pgfsetstrokeopacity{0.337458}%
\pgfsetdash{}{0pt}%
\pgfpathmoveto{\pgfqpoint{1.542095in}{2.435237in}}%
\pgfpathcurveto{\pgfqpoint{1.550332in}{2.435237in}}{\pgfqpoint{1.558232in}{2.438509in}}{\pgfqpoint{1.564056in}{2.444333in}}%
\pgfpathcurveto{\pgfqpoint{1.569880in}{2.450157in}}{\pgfqpoint{1.573152in}{2.458057in}}{\pgfqpoint{1.573152in}{2.466293in}}%
\pgfpathcurveto{\pgfqpoint{1.573152in}{2.474529in}}{\pgfqpoint{1.569880in}{2.482429in}}{\pgfqpoint{1.564056in}{2.488253in}}%
\pgfpathcurveto{\pgfqpoint{1.558232in}{2.494077in}}{\pgfqpoint{1.550332in}{2.497350in}}{\pgfqpoint{1.542095in}{2.497350in}}%
\pgfpathcurveto{\pgfqpoint{1.533859in}{2.497350in}}{\pgfqpoint{1.525959in}{2.494077in}}{\pgfqpoint{1.520135in}{2.488253in}}%
\pgfpathcurveto{\pgfqpoint{1.514311in}{2.482429in}}{\pgfqpoint{1.511039in}{2.474529in}}{\pgfqpoint{1.511039in}{2.466293in}}%
\pgfpathcurveto{\pgfqpoint{1.511039in}{2.458057in}}{\pgfqpoint{1.514311in}{2.450157in}}{\pgfqpoint{1.520135in}{2.444333in}}%
\pgfpathcurveto{\pgfqpoint{1.525959in}{2.438509in}}{\pgfqpoint{1.533859in}{2.435237in}}{\pgfqpoint{1.542095in}{2.435237in}}%
\pgfpathclose%
\pgfusepath{stroke,fill}%
\end{pgfscope}%
\begin{pgfscope}%
\pgfpathrectangle{\pgfqpoint{0.100000in}{0.212622in}}{\pgfqpoint{3.696000in}{3.696000in}}%
\pgfusepath{clip}%
\pgfsetbuttcap%
\pgfsetroundjoin%
\definecolor{currentfill}{rgb}{0.121569,0.466667,0.705882}%
\pgfsetfillcolor{currentfill}%
\pgfsetfillopacity{0.338186}%
\pgfsetlinewidth{1.003750pt}%
\definecolor{currentstroke}{rgb}{0.121569,0.466667,0.705882}%
\pgfsetstrokecolor{currentstroke}%
\pgfsetstrokeopacity{0.338186}%
\pgfsetdash{}{0pt}%
\pgfpathmoveto{\pgfqpoint{1.950600in}{2.510102in}}%
\pgfpathcurveto{\pgfqpoint{1.958837in}{2.510102in}}{\pgfqpoint{1.966737in}{2.513374in}}{\pgfqpoint{1.972561in}{2.519198in}}%
\pgfpathcurveto{\pgfqpoint{1.978385in}{2.525022in}}{\pgfqpoint{1.981657in}{2.532922in}}{\pgfqpoint{1.981657in}{2.541158in}}%
\pgfpathcurveto{\pgfqpoint{1.981657in}{2.549395in}}{\pgfqpoint{1.978385in}{2.557295in}}{\pgfqpoint{1.972561in}{2.563119in}}%
\pgfpathcurveto{\pgfqpoint{1.966737in}{2.568943in}}{\pgfqpoint{1.958837in}{2.572215in}}{\pgfqpoint{1.950600in}{2.572215in}}%
\pgfpathcurveto{\pgfqpoint{1.942364in}{2.572215in}}{\pgfqpoint{1.934464in}{2.568943in}}{\pgfqpoint{1.928640in}{2.563119in}}%
\pgfpathcurveto{\pgfqpoint{1.922816in}{2.557295in}}{\pgfqpoint{1.919544in}{2.549395in}}{\pgfqpoint{1.919544in}{2.541158in}}%
\pgfpathcurveto{\pgfqpoint{1.919544in}{2.532922in}}{\pgfqpoint{1.922816in}{2.525022in}}{\pgfqpoint{1.928640in}{2.519198in}}%
\pgfpathcurveto{\pgfqpoint{1.934464in}{2.513374in}}{\pgfqpoint{1.942364in}{2.510102in}}{\pgfqpoint{1.950600in}{2.510102in}}%
\pgfpathclose%
\pgfusepath{stroke,fill}%
\end{pgfscope}%
\begin{pgfscope}%
\pgfpathrectangle{\pgfqpoint{0.100000in}{0.212622in}}{\pgfqpoint{3.696000in}{3.696000in}}%
\pgfusepath{clip}%
\pgfsetbuttcap%
\pgfsetroundjoin%
\definecolor{currentfill}{rgb}{0.121569,0.466667,0.705882}%
\pgfsetfillcolor{currentfill}%
\pgfsetfillopacity{0.338266}%
\pgfsetlinewidth{1.003750pt}%
\definecolor{currentstroke}{rgb}{0.121569,0.466667,0.705882}%
\pgfsetstrokecolor{currentstroke}%
\pgfsetstrokeopacity{0.338266}%
\pgfsetdash{}{0pt}%
\pgfpathmoveto{\pgfqpoint{1.540248in}{2.432049in}}%
\pgfpathcurveto{\pgfqpoint{1.548485in}{2.432049in}}{\pgfqpoint{1.556385in}{2.435322in}}{\pgfqpoint{1.562209in}{2.441146in}}%
\pgfpathcurveto{\pgfqpoint{1.568032in}{2.446970in}}{\pgfqpoint{1.571305in}{2.454870in}}{\pgfqpoint{1.571305in}{2.463106in}}%
\pgfpathcurveto{\pgfqpoint{1.571305in}{2.471342in}}{\pgfqpoint{1.568032in}{2.479242in}}{\pgfqpoint{1.562209in}{2.485066in}}%
\pgfpathcurveto{\pgfqpoint{1.556385in}{2.490890in}}{\pgfqpoint{1.548485in}{2.494162in}}{\pgfqpoint{1.540248in}{2.494162in}}%
\pgfpathcurveto{\pgfqpoint{1.532012in}{2.494162in}}{\pgfqpoint{1.524112in}{2.490890in}}{\pgfqpoint{1.518288in}{2.485066in}}%
\pgfpathcurveto{\pgfqpoint{1.512464in}{2.479242in}}{\pgfqpoint{1.509192in}{2.471342in}}{\pgfqpoint{1.509192in}{2.463106in}}%
\pgfpathcurveto{\pgfqpoint{1.509192in}{2.454870in}}{\pgfqpoint{1.512464in}{2.446970in}}{\pgfqpoint{1.518288in}{2.441146in}}%
\pgfpathcurveto{\pgfqpoint{1.524112in}{2.435322in}}{\pgfqpoint{1.532012in}{2.432049in}}{\pgfqpoint{1.540248in}{2.432049in}}%
\pgfpathclose%
\pgfusepath{stroke,fill}%
\end{pgfscope}%
\begin{pgfscope}%
\pgfpathrectangle{\pgfqpoint{0.100000in}{0.212622in}}{\pgfqpoint{3.696000in}{3.696000in}}%
\pgfusepath{clip}%
\pgfsetbuttcap%
\pgfsetroundjoin%
\definecolor{currentfill}{rgb}{0.121569,0.466667,0.705882}%
\pgfsetfillcolor{currentfill}%
\pgfsetfillopacity{0.338718}%
\pgfsetlinewidth{1.003750pt}%
\definecolor{currentstroke}{rgb}{0.121569,0.466667,0.705882}%
\pgfsetstrokecolor{currentstroke}%
\pgfsetstrokeopacity{0.338718}%
\pgfsetdash{}{0pt}%
\pgfpathmoveto{\pgfqpoint{1.539160in}{2.430287in}}%
\pgfpathcurveto{\pgfqpoint{1.547396in}{2.430287in}}{\pgfqpoint{1.555296in}{2.433559in}}{\pgfqpoint{1.561120in}{2.439383in}}%
\pgfpathcurveto{\pgfqpoint{1.566944in}{2.445207in}}{\pgfqpoint{1.570216in}{2.453107in}}{\pgfqpoint{1.570216in}{2.461343in}}%
\pgfpathcurveto{\pgfqpoint{1.570216in}{2.469580in}}{\pgfqpoint{1.566944in}{2.477480in}}{\pgfqpoint{1.561120in}{2.483304in}}%
\pgfpathcurveto{\pgfqpoint{1.555296in}{2.489128in}}{\pgfqpoint{1.547396in}{2.492400in}}{\pgfqpoint{1.539160in}{2.492400in}}%
\pgfpathcurveto{\pgfqpoint{1.530923in}{2.492400in}}{\pgfqpoint{1.523023in}{2.489128in}}{\pgfqpoint{1.517199in}{2.483304in}}%
\pgfpathcurveto{\pgfqpoint{1.511375in}{2.477480in}}{\pgfqpoint{1.508103in}{2.469580in}}{\pgfqpoint{1.508103in}{2.461343in}}%
\pgfpathcurveto{\pgfqpoint{1.508103in}{2.453107in}}{\pgfqpoint{1.511375in}{2.445207in}}{\pgfqpoint{1.517199in}{2.439383in}}%
\pgfpathcurveto{\pgfqpoint{1.523023in}{2.433559in}}{\pgfqpoint{1.530923in}{2.430287in}}{\pgfqpoint{1.539160in}{2.430287in}}%
\pgfpathclose%
\pgfusepath{stroke,fill}%
\end{pgfscope}%
\begin{pgfscope}%
\pgfpathrectangle{\pgfqpoint{0.100000in}{0.212622in}}{\pgfqpoint{3.696000in}{3.696000in}}%
\pgfusepath{clip}%
\pgfsetbuttcap%
\pgfsetroundjoin%
\definecolor{currentfill}{rgb}{0.121569,0.466667,0.705882}%
\pgfsetfillcolor{currentfill}%
\pgfsetfillopacity{0.338887}%
\pgfsetlinewidth{1.003750pt}%
\definecolor{currentstroke}{rgb}{0.121569,0.466667,0.705882}%
\pgfsetstrokecolor{currentstroke}%
\pgfsetstrokeopacity{0.338887}%
\pgfsetdash{}{0pt}%
\pgfpathmoveto{\pgfqpoint{1.538761in}{2.429635in}}%
\pgfpathcurveto{\pgfqpoint{1.546998in}{2.429635in}}{\pgfqpoint{1.554898in}{2.432908in}}{\pgfqpoint{1.560722in}{2.438732in}}%
\pgfpathcurveto{\pgfqpoint{1.566546in}{2.444556in}}{\pgfqpoint{1.569818in}{2.452456in}}{\pgfqpoint{1.569818in}{2.460692in}}%
\pgfpathcurveto{\pgfqpoint{1.569818in}{2.468928in}}{\pgfqpoint{1.566546in}{2.476828in}}{\pgfqpoint{1.560722in}{2.482652in}}%
\pgfpathcurveto{\pgfqpoint{1.554898in}{2.488476in}}{\pgfqpoint{1.546998in}{2.491748in}}{\pgfqpoint{1.538761in}{2.491748in}}%
\pgfpathcurveto{\pgfqpoint{1.530525in}{2.491748in}}{\pgfqpoint{1.522625in}{2.488476in}}{\pgfqpoint{1.516801in}{2.482652in}}%
\pgfpathcurveto{\pgfqpoint{1.510977in}{2.476828in}}{\pgfqpoint{1.507705in}{2.468928in}}{\pgfqpoint{1.507705in}{2.460692in}}%
\pgfpathcurveto{\pgfqpoint{1.507705in}{2.452456in}}{\pgfqpoint{1.510977in}{2.444556in}}{\pgfqpoint{1.516801in}{2.438732in}}%
\pgfpathcurveto{\pgfqpoint{1.522625in}{2.432908in}}{\pgfqpoint{1.530525in}{2.429635in}}{\pgfqpoint{1.538761in}{2.429635in}}%
\pgfpathclose%
\pgfusepath{stroke,fill}%
\end{pgfscope}%
\begin{pgfscope}%
\pgfpathrectangle{\pgfqpoint{0.100000in}{0.212622in}}{\pgfqpoint{3.696000in}{3.696000in}}%
\pgfusepath{clip}%
\pgfsetbuttcap%
\pgfsetroundjoin%
\definecolor{currentfill}{rgb}{0.121569,0.466667,0.705882}%
\pgfsetfillcolor{currentfill}%
\pgfsetfillopacity{0.339190}%
\pgfsetlinewidth{1.003750pt}%
\definecolor{currentstroke}{rgb}{0.121569,0.466667,0.705882}%
\pgfsetstrokecolor{currentstroke}%
\pgfsetstrokeopacity{0.339190}%
\pgfsetdash{}{0pt}%
\pgfpathmoveto{\pgfqpoint{1.538001in}{2.428450in}}%
\pgfpathcurveto{\pgfqpoint{1.546238in}{2.428450in}}{\pgfqpoint{1.554138in}{2.431723in}}{\pgfqpoint{1.559962in}{2.437547in}}%
\pgfpathcurveto{\pgfqpoint{1.565786in}{2.443371in}}{\pgfqpoint{1.569058in}{2.451271in}}{\pgfqpoint{1.569058in}{2.459507in}}%
\pgfpathcurveto{\pgfqpoint{1.569058in}{2.467743in}}{\pgfqpoint{1.565786in}{2.475643in}}{\pgfqpoint{1.559962in}{2.481467in}}%
\pgfpathcurveto{\pgfqpoint{1.554138in}{2.487291in}}{\pgfqpoint{1.546238in}{2.490563in}}{\pgfqpoint{1.538001in}{2.490563in}}%
\pgfpathcurveto{\pgfqpoint{1.529765in}{2.490563in}}{\pgfqpoint{1.521865in}{2.487291in}}{\pgfqpoint{1.516041in}{2.481467in}}%
\pgfpathcurveto{\pgfqpoint{1.510217in}{2.475643in}}{\pgfqpoint{1.506945in}{2.467743in}}{\pgfqpoint{1.506945in}{2.459507in}}%
\pgfpathcurveto{\pgfqpoint{1.506945in}{2.451271in}}{\pgfqpoint{1.510217in}{2.443371in}}{\pgfqpoint{1.516041in}{2.437547in}}%
\pgfpathcurveto{\pgfqpoint{1.521865in}{2.431723in}}{\pgfqpoint{1.529765in}{2.428450in}}{\pgfqpoint{1.538001in}{2.428450in}}%
\pgfpathclose%
\pgfusepath{stroke,fill}%
\end{pgfscope}%
\begin{pgfscope}%
\pgfpathrectangle{\pgfqpoint{0.100000in}{0.212622in}}{\pgfqpoint{3.696000in}{3.696000in}}%
\pgfusepath{clip}%
\pgfsetbuttcap%
\pgfsetroundjoin%
\definecolor{currentfill}{rgb}{0.121569,0.466667,0.705882}%
\pgfsetfillcolor{currentfill}%
\pgfsetfillopacity{0.339531}%
\pgfsetlinewidth{1.003750pt}%
\definecolor{currentstroke}{rgb}{0.121569,0.466667,0.705882}%
\pgfsetstrokecolor{currentstroke}%
\pgfsetstrokeopacity{0.339531}%
\pgfsetdash{}{0pt}%
\pgfpathmoveto{\pgfqpoint{1.958676in}{2.509052in}}%
\pgfpathcurveto{\pgfqpoint{1.966912in}{2.509052in}}{\pgfqpoint{1.974812in}{2.512324in}}{\pgfqpoint{1.980636in}{2.518148in}}%
\pgfpathcurveto{\pgfqpoint{1.986460in}{2.523972in}}{\pgfqpoint{1.989733in}{2.531872in}}{\pgfqpoint{1.989733in}{2.540108in}}%
\pgfpathcurveto{\pgfqpoint{1.989733in}{2.548345in}}{\pgfqpoint{1.986460in}{2.556245in}}{\pgfqpoint{1.980636in}{2.562069in}}%
\pgfpathcurveto{\pgfqpoint{1.974812in}{2.567893in}}{\pgfqpoint{1.966912in}{2.571165in}}{\pgfqpoint{1.958676in}{2.571165in}}%
\pgfpathcurveto{\pgfqpoint{1.950440in}{2.571165in}}{\pgfqpoint{1.942540in}{2.567893in}}{\pgfqpoint{1.936716in}{2.562069in}}%
\pgfpathcurveto{\pgfqpoint{1.930892in}{2.556245in}}{\pgfqpoint{1.927620in}{2.548345in}}{\pgfqpoint{1.927620in}{2.540108in}}%
\pgfpathcurveto{\pgfqpoint{1.927620in}{2.531872in}}{\pgfqpoint{1.930892in}{2.523972in}}{\pgfqpoint{1.936716in}{2.518148in}}%
\pgfpathcurveto{\pgfqpoint{1.942540in}{2.512324in}}{\pgfqpoint{1.950440in}{2.509052in}}{\pgfqpoint{1.958676in}{2.509052in}}%
\pgfpathclose%
\pgfusepath{stroke,fill}%
\end{pgfscope}%
\begin{pgfscope}%
\pgfpathrectangle{\pgfqpoint{0.100000in}{0.212622in}}{\pgfqpoint{3.696000in}{3.696000in}}%
\pgfusepath{clip}%
\pgfsetbuttcap%
\pgfsetroundjoin%
\definecolor{currentfill}{rgb}{0.121569,0.466667,0.705882}%
\pgfsetfillcolor{currentfill}%
\pgfsetfillopacity{0.339745}%
\pgfsetlinewidth{1.003750pt}%
\definecolor{currentstroke}{rgb}{0.121569,0.466667,0.705882}%
\pgfsetstrokecolor{currentstroke}%
\pgfsetstrokeopacity{0.339745}%
\pgfsetdash{}{0pt}%
\pgfpathmoveto{\pgfqpoint{1.536641in}{2.426305in}}%
\pgfpathcurveto{\pgfqpoint{1.544877in}{2.426305in}}{\pgfqpoint{1.552777in}{2.429578in}}{\pgfqpoint{1.558601in}{2.435401in}}%
\pgfpathcurveto{\pgfqpoint{1.564425in}{2.441225in}}{\pgfqpoint{1.567697in}{2.449125in}}{\pgfqpoint{1.567697in}{2.457362in}}%
\pgfpathcurveto{\pgfqpoint{1.567697in}{2.465598in}}{\pgfqpoint{1.564425in}{2.473498in}}{\pgfqpoint{1.558601in}{2.479322in}}%
\pgfpathcurveto{\pgfqpoint{1.552777in}{2.485146in}}{\pgfqpoint{1.544877in}{2.488418in}}{\pgfqpoint{1.536641in}{2.488418in}}%
\pgfpathcurveto{\pgfqpoint{1.528405in}{2.488418in}}{\pgfqpoint{1.520505in}{2.485146in}}{\pgfqpoint{1.514681in}{2.479322in}}%
\pgfpathcurveto{\pgfqpoint{1.508857in}{2.473498in}}{\pgfqpoint{1.505584in}{2.465598in}}{\pgfqpoint{1.505584in}{2.457362in}}%
\pgfpathcurveto{\pgfqpoint{1.505584in}{2.449125in}}{\pgfqpoint{1.508857in}{2.441225in}}{\pgfqpoint{1.514681in}{2.435401in}}%
\pgfpathcurveto{\pgfqpoint{1.520505in}{2.429578in}}{\pgfqpoint{1.528405in}{2.426305in}}{\pgfqpoint{1.536641in}{2.426305in}}%
\pgfpathclose%
\pgfusepath{stroke,fill}%
\end{pgfscope}%
\begin{pgfscope}%
\pgfpathrectangle{\pgfqpoint{0.100000in}{0.212622in}}{\pgfqpoint{3.696000in}{3.696000in}}%
\pgfusepath{clip}%
\pgfsetbuttcap%
\pgfsetroundjoin%
\definecolor{currentfill}{rgb}{0.121569,0.466667,0.705882}%
\pgfsetfillcolor{currentfill}%
\pgfsetfillopacity{0.340754}%
\pgfsetlinewidth{1.003750pt}%
\definecolor{currentstroke}{rgb}{0.121569,0.466667,0.705882}%
\pgfsetstrokecolor{currentstroke}%
\pgfsetstrokeopacity{0.340754}%
\pgfsetdash{}{0pt}%
\pgfpathmoveto{\pgfqpoint{1.534141in}{2.422425in}}%
\pgfpathcurveto{\pgfqpoint{1.542378in}{2.422425in}}{\pgfqpoint{1.550278in}{2.425697in}}{\pgfqpoint{1.556102in}{2.431521in}}%
\pgfpathcurveto{\pgfqpoint{1.561925in}{2.437345in}}{\pgfqpoint{1.565198in}{2.445245in}}{\pgfqpoint{1.565198in}{2.453481in}}%
\pgfpathcurveto{\pgfqpoint{1.565198in}{2.461717in}}{\pgfqpoint{1.561925in}{2.469618in}}{\pgfqpoint{1.556102in}{2.475441in}}%
\pgfpathcurveto{\pgfqpoint{1.550278in}{2.481265in}}{\pgfqpoint{1.542378in}{2.484538in}}{\pgfqpoint{1.534141in}{2.484538in}}%
\pgfpathcurveto{\pgfqpoint{1.525905in}{2.484538in}}{\pgfqpoint{1.518005in}{2.481265in}}{\pgfqpoint{1.512181in}{2.475441in}}%
\pgfpathcurveto{\pgfqpoint{1.506357in}{2.469618in}}{\pgfqpoint{1.503085in}{2.461717in}}{\pgfqpoint{1.503085in}{2.453481in}}%
\pgfpathcurveto{\pgfqpoint{1.503085in}{2.445245in}}{\pgfqpoint{1.506357in}{2.437345in}}{\pgfqpoint{1.512181in}{2.431521in}}%
\pgfpathcurveto{\pgfqpoint{1.518005in}{2.425697in}}{\pgfqpoint{1.525905in}{2.422425in}}{\pgfqpoint{1.534141in}{2.422425in}}%
\pgfpathclose%
\pgfusepath{stroke,fill}%
\end{pgfscope}%
\begin{pgfscope}%
\pgfpathrectangle{\pgfqpoint{0.100000in}{0.212622in}}{\pgfqpoint{3.696000in}{3.696000in}}%
\pgfusepath{clip}%
\pgfsetbuttcap%
\pgfsetroundjoin%
\definecolor{currentfill}{rgb}{0.121569,0.466667,0.705882}%
\pgfsetfillcolor{currentfill}%
\pgfsetfillopacity{0.341006}%
\pgfsetlinewidth{1.003750pt}%
\definecolor{currentstroke}{rgb}{0.121569,0.466667,0.705882}%
\pgfsetstrokecolor{currentstroke}%
\pgfsetstrokeopacity{0.341006}%
\pgfsetdash{}{0pt}%
\pgfpathmoveto{\pgfqpoint{1.967232in}{2.508355in}}%
\pgfpathcurveto{\pgfqpoint{1.975468in}{2.508355in}}{\pgfqpoint{1.983368in}{2.511627in}}{\pgfqpoint{1.989192in}{2.517451in}}%
\pgfpathcurveto{\pgfqpoint{1.995016in}{2.523275in}}{\pgfqpoint{1.998289in}{2.531175in}}{\pgfqpoint{1.998289in}{2.539411in}}%
\pgfpathcurveto{\pgfqpoint{1.998289in}{2.547647in}}{\pgfqpoint{1.995016in}{2.555547in}}{\pgfqpoint{1.989192in}{2.561371in}}%
\pgfpathcurveto{\pgfqpoint{1.983368in}{2.567195in}}{\pgfqpoint{1.975468in}{2.570468in}}{\pgfqpoint{1.967232in}{2.570468in}}%
\pgfpathcurveto{\pgfqpoint{1.958996in}{2.570468in}}{\pgfqpoint{1.951096in}{2.567195in}}{\pgfqpoint{1.945272in}{2.561371in}}%
\pgfpathcurveto{\pgfqpoint{1.939448in}{2.555547in}}{\pgfqpoint{1.936176in}{2.547647in}}{\pgfqpoint{1.936176in}{2.539411in}}%
\pgfpathcurveto{\pgfqpoint{1.936176in}{2.531175in}}{\pgfqpoint{1.939448in}{2.523275in}}{\pgfqpoint{1.945272in}{2.517451in}}%
\pgfpathcurveto{\pgfqpoint{1.951096in}{2.511627in}}{\pgfqpoint{1.958996in}{2.508355in}}{\pgfqpoint{1.967232in}{2.508355in}}%
\pgfpathclose%
\pgfusepath{stroke,fill}%
\end{pgfscope}%
\begin{pgfscope}%
\pgfpathrectangle{\pgfqpoint{0.100000in}{0.212622in}}{\pgfqpoint{3.696000in}{3.696000in}}%
\pgfusepath{clip}%
\pgfsetbuttcap%
\pgfsetroundjoin%
\definecolor{currentfill}{rgb}{0.121569,0.466667,0.705882}%
\pgfsetfillcolor{currentfill}%
\pgfsetfillopacity{0.341796}%
\pgfsetlinewidth{1.003750pt}%
\definecolor{currentstroke}{rgb}{0.121569,0.466667,0.705882}%
\pgfsetstrokecolor{currentstroke}%
\pgfsetstrokeopacity{0.341796}%
\pgfsetdash{}{0pt}%
\pgfpathmoveto{\pgfqpoint{1.971928in}{2.507783in}}%
\pgfpathcurveto{\pgfqpoint{1.980164in}{2.507783in}}{\pgfqpoint{1.988064in}{2.511055in}}{\pgfqpoint{1.993888in}{2.516879in}}%
\pgfpathcurveto{\pgfqpoint{1.999712in}{2.522703in}}{\pgfqpoint{2.002984in}{2.530603in}}{\pgfqpoint{2.002984in}{2.538839in}}%
\pgfpathcurveto{\pgfqpoint{2.002984in}{2.547076in}}{\pgfqpoint{1.999712in}{2.554976in}}{\pgfqpoint{1.993888in}{2.560800in}}%
\pgfpathcurveto{\pgfqpoint{1.988064in}{2.566623in}}{\pgfqpoint{1.980164in}{2.569896in}}{\pgfqpoint{1.971928in}{2.569896in}}%
\pgfpathcurveto{\pgfqpoint{1.963691in}{2.569896in}}{\pgfqpoint{1.955791in}{2.566623in}}{\pgfqpoint{1.949967in}{2.560800in}}%
\pgfpathcurveto{\pgfqpoint{1.944143in}{2.554976in}}{\pgfqpoint{1.940871in}{2.547076in}}{\pgfqpoint{1.940871in}{2.538839in}}%
\pgfpathcurveto{\pgfqpoint{1.940871in}{2.530603in}}{\pgfqpoint{1.944143in}{2.522703in}}{\pgfqpoint{1.949967in}{2.516879in}}%
\pgfpathcurveto{\pgfqpoint{1.955791in}{2.511055in}}{\pgfqpoint{1.963691in}{2.507783in}}{\pgfqpoint{1.971928in}{2.507783in}}%
\pgfpathclose%
\pgfusepath{stroke,fill}%
\end{pgfscope}%
\begin{pgfscope}%
\pgfpathrectangle{\pgfqpoint{0.100000in}{0.212622in}}{\pgfqpoint{3.696000in}{3.696000in}}%
\pgfusepath{clip}%
\pgfsetbuttcap%
\pgfsetroundjoin%
\definecolor{currentfill}{rgb}{0.121569,0.466667,0.705882}%
\pgfsetfillcolor{currentfill}%
\pgfsetfillopacity{0.342590}%
\pgfsetlinewidth{1.003750pt}%
\definecolor{currentstroke}{rgb}{0.121569,0.466667,0.705882}%
\pgfsetstrokecolor{currentstroke}%
\pgfsetstrokeopacity{0.342590}%
\pgfsetdash{}{0pt}%
\pgfpathmoveto{\pgfqpoint{1.529638in}{2.415323in}}%
\pgfpathcurveto{\pgfqpoint{1.537874in}{2.415323in}}{\pgfqpoint{1.545774in}{2.418595in}}{\pgfqpoint{1.551598in}{2.424419in}}%
\pgfpathcurveto{\pgfqpoint{1.557422in}{2.430243in}}{\pgfqpoint{1.560695in}{2.438143in}}{\pgfqpoint{1.560695in}{2.446379in}}%
\pgfpathcurveto{\pgfqpoint{1.560695in}{2.454615in}}{\pgfqpoint{1.557422in}{2.462515in}}{\pgfqpoint{1.551598in}{2.468339in}}%
\pgfpathcurveto{\pgfqpoint{1.545774in}{2.474163in}}{\pgfqpoint{1.537874in}{2.477436in}}{\pgfqpoint{1.529638in}{2.477436in}}%
\pgfpathcurveto{\pgfqpoint{1.521402in}{2.477436in}}{\pgfqpoint{1.513502in}{2.474163in}}{\pgfqpoint{1.507678in}{2.468339in}}%
\pgfpathcurveto{\pgfqpoint{1.501854in}{2.462515in}}{\pgfqpoint{1.498582in}{2.454615in}}{\pgfqpoint{1.498582in}{2.446379in}}%
\pgfpathcurveto{\pgfqpoint{1.498582in}{2.438143in}}{\pgfqpoint{1.501854in}{2.430243in}}{\pgfqpoint{1.507678in}{2.424419in}}%
\pgfpathcurveto{\pgfqpoint{1.513502in}{2.418595in}}{\pgfqpoint{1.521402in}{2.415323in}}{\pgfqpoint{1.529638in}{2.415323in}}%
\pgfpathclose%
\pgfusepath{stroke,fill}%
\end{pgfscope}%
\begin{pgfscope}%
\pgfpathrectangle{\pgfqpoint{0.100000in}{0.212622in}}{\pgfqpoint{3.696000in}{3.696000in}}%
\pgfusepath{clip}%
\pgfsetbuttcap%
\pgfsetroundjoin%
\definecolor{currentfill}{rgb}{0.121569,0.466667,0.705882}%
\pgfsetfillcolor{currentfill}%
\pgfsetfillopacity{0.343001}%
\pgfsetlinewidth{1.003750pt}%
\definecolor{currentstroke}{rgb}{0.121569,0.466667,0.705882}%
\pgfsetstrokecolor{currentstroke}%
\pgfsetstrokeopacity{0.343001}%
\pgfsetdash{}{0pt}%
\pgfpathmoveto{\pgfqpoint{1.979137in}{2.507052in}}%
\pgfpathcurveto{\pgfqpoint{1.987373in}{2.507052in}}{\pgfqpoint{1.995273in}{2.510324in}}{\pgfqpoint{2.001097in}{2.516148in}}%
\pgfpathcurveto{\pgfqpoint{2.006921in}{2.521972in}}{\pgfqpoint{2.010193in}{2.529872in}}{\pgfqpoint{2.010193in}{2.538108in}}%
\pgfpathcurveto{\pgfqpoint{2.010193in}{2.546344in}}{\pgfqpoint{2.006921in}{2.554244in}}{\pgfqpoint{2.001097in}{2.560068in}}%
\pgfpathcurveto{\pgfqpoint{1.995273in}{2.565892in}}{\pgfqpoint{1.987373in}{2.569165in}}{\pgfqpoint{1.979137in}{2.569165in}}%
\pgfpathcurveto{\pgfqpoint{1.970900in}{2.569165in}}{\pgfqpoint{1.963000in}{2.565892in}}{\pgfqpoint{1.957176in}{2.560068in}}%
\pgfpathcurveto{\pgfqpoint{1.951352in}{2.554244in}}{\pgfqpoint{1.948080in}{2.546344in}}{\pgfqpoint{1.948080in}{2.538108in}}%
\pgfpathcurveto{\pgfqpoint{1.948080in}{2.529872in}}{\pgfqpoint{1.951352in}{2.521972in}}{\pgfqpoint{1.957176in}{2.516148in}}%
\pgfpathcurveto{\pgfqpoint{1.963000in}{2.510324in}}{\pgfqpoint{1.970900in}{2.507052in}}{\pgfqpoint{1.979137in}{2.507052in}}%
\pgfpathclose%
\pgfusepath{stroke,fill}%
\end{pgfscope}%
\begin{pgfscope}%
\pgfpathrectangle{\pgfqpoint{0.100000in}{0.212622in}}{\pgfqpoint{3.696000in}{3.696000in}}%
\pgfusepath{clip}%
\pgfsetbuttcap%
\pgfsetroundjoin%
\definecolor{currentfill}{rgb}{0.121569,0.466667,0.705882}%
\pgfsetfillcolor{currentfill}%
\pgfsetfillopacity{0.344148}%
\pgfsetlinewidth{1.003750pt}%
\definecolor{currentstroke}{rgb}{0.121569,0.466667,0.705882}%
\pgfsetstrokecolor{currentstroke}%
\pgfsetstrokeopacity{0.344148}%
\pgfsetdash{}{0pt}%
\pgfpathmoveto{\pgfqpoint{1.526110in}{2.408980in}}%
\pgfpathcurveto{\pgfqpoint{1.534347in}{2.408980in}}{\pgfqpoint{1.542247in}{2.412252in}}{\pgfqpoint{1.548070in}{2.418076in}}%
\pgfpathcurveto{\pgfqpoint{1.553894in}{2.423900in}}{\pgfqpoint{1.557167in}{2.431800in}}{\pgfqpoint{1.557167in}{2.440036in}}%
\pgfpathcurveto{\pgfqpoint{1.557167in}{2.448273in}}{\pgfqpoint{1.553894in}{2.456173in}}{\pgfqpoint{1.548070in}{2.461997in}}%
\pgfpathcurveto{\pgfqpoint{1.542247in}{2.467821in}}{\pgfqpoint{1.534347in}{2.471093in}}{\pgfqpoint{1.526110in}{2.471093in}}%
\pgfpathcurveto{\pgfqpoint{1.517874in}{2.471093in}}{\pgfqpoint{1.509974in}{2.467821in}}{\pgfqpoint{1.504150in}{2.461997in}}%
\pgfpathcurveto{\pgfqpoint{1.498326in}{2.456173in}}{\pgfqpoint{1.495054in}{2.448273in}}{\pgfqpoint{1.495054in}{2.440036in}}%
\pgfpathcurveto{\pgfqpoint{1.495054in}{2.431800in}}{\pgfqpoint{1.498326in}{2.423900in}}{\pgfqpoint{1.504150in}{2.418076in}}%
\pgfpathcurveto{\pgfqpoint{1.509974in}{2.412252in}}{\pgfqpoint{1.517874in}{2.408980in}}{\pgfqpoint{1.526110in}{2.408980in}}%
\pgfpathclose%
\pgfusepath{stroke,fill}%
\end{pgfscope}%
\begin{pgfscope}%
\pgfpathrectangle{\pgfqpoint{0.100000in}{0.212622in}}{\pgfqpoint{3.696000in}{3.696000in}}%
\pgfusepath{clip}%
\pgfsetbuttcap%
\pgfsetroundjoin%
\definecolor{currentfill}{rgb}{0.121569,0.466667,0.705882}%
\pgfsetfillcolor{currentfill}%
\pgfsetfillopacity{0.344426}%
\pgfsetlinewidth{1.003750pt}%
\definecolor{currentstroke}{rgb}{0.121569,0.466667,0.705882}%
\pgfsetstrokecolor{currentstroke}%
\pgfsetstrokeopacity{0.344426}%
\pgfsetdash{}{0pt}%
\pgfpathmoveto{\pgfqpoint{1.986862in}{2.506897in}}%
\pgfpathcurveto{\pgfqpoint{1.995098in}{2.506897in}}{\pgfqpoint{2.002998in}{2.510170in}}{\pgfqpoint{2.008822in}{2.515994in}}%
\pgfpathcurveto{\pgfqpoint{2.014646in}{2.521818in}}{\pgfqpoint{2.017918in}{2.529718in}}{\pgfqpoint{2.017918in}{2.537954in}}%
\pgfpathcurveto{\pgfqpoint{2.017918in}{2.546190in}}{\pgfqpoint{2.014646in}{2.554090in}}{\pgfqpoint{2.008822in}{2.559914in}}%
\pgfpathcurveto{\pgfqpoint{2.002998in}{2.565738in}}{\pgfqpoint{1.995098in}{2.569010in}}{\pgfqpoint{1.986862in}{2.569010in}}%
\pgfpathcurveto{\pgfqpoint{1.978625in}{2.569010in}}{\pgfqpoint{1.970725in}{2.565738in}}{\pgfqpoint{1.964901in}{2.559914in}}%
\pgfpathcurveto{\pgfqpoint{1.959077in}{2.554090in}}{\pgfqpoint{1.955805in}{2.546190in}}{\pgfqpoint{1.955805in}{2.537954in}}%
\pgfpathcurveto{\pgfqpoint{1.955805in}{2.529718in}}{\pgfqpoint{1.959077in}{2.521818in}}{\pgfqpoint{1.964901in}{2.515994in}}%
\pgfpathcurveto{\pgfqpoint{1.970725in}{2.510170in}}{\pgfqpoint{1.978625in}{2.506897in}}{\pgfqpoint{1.986862in}{2.506897in}}%
\pgfpathclose%
\pgfusepath{stroke,fill}%
\end{pgfscope}%
\begin{pgfscope}%
\pgfpathrectangle{\pgfqpoint{0.100000in}{0.212622in}}{\pgfqpoint{3.696000in}{3.696000in}}%
\pgfusepath{clip}%
\pgfsetbuttcap%
\pgfsetroundjoin%
\definecolor{currentfill}{rgb}{0.121569,0.466667,0.705882}%
\pgfsetfillcolor{currentfill}%
\pgfsetfillopacity{0.345437}%
\pgfsetlinewidth{1.003750pt}%
\definecolor{currentstroke}{rgb}{0.121569,0.466667,0.705882}%
\pgfsetstrokecolor{currentstroke}%
\pgfsetstrokeopacity{0.345437}%
\pgfsetdash{}{0pt}%
\pgfpathmoveto{\pgfqpoint{1.523060in}{2.403632in}}%
\pgfpathcurveto{\pgfqpoint{1.531296in}{2.403632in}}{\pgfqpoint{1.539196in}{2.406905in}}{\pgfqpoint{1.545020in}{2.412729in}}%
\pgfpathcurveto{\pgfqpoint{1.550844in}{2.418553in}}{\pgfqpoint{1.554116in}{2.426453in}}{\pgfqpoint{1.554116in}{2.434689in}}%
\pgfpathcurveto{\pgfqpoint{1.554116in}{2.442925in}}{\pgfqpoint{1.550844in}{2.450825in}}{\pgfqpoint{1.545020in}{2.456649in}}%
\pgfpathcurveto{\pgfqpoint{1.539196in}{2.462473in}}{\pgfqpoint{1.531296in}{2.465745in}}{\pgfqpoint{1.523060in}{2.465745in}}%
\pgfpathcurveto{\pgfqpoint{1.514823in}{2.465745in}}{\pgfqpoint{1.506923in}{2.462473in}}{\pgfqpoint{1.501099in}{2.456649in}}%
\pgfpathcurveto{\pgfqpoint{1.495275in}{2.450825in}}{\pgfqpoint{1.492003in}{2.442925in}}{\pgfqpoint{1.492003in}{2.434689in}}%
\pgfpathcurveto{\pgfqpoint{1.492003in}{2.426453in}}{\pgfqpoint{1.495275in}{2.418553in}}{\pgfqpoint{1.501099in}{2.412729in}}%
\pgfpathcurveto{\pgfqpoint{1.506923in}{2.406905in}}{\pgfqpoint{1.514823in}{2.403632in}}{\pgfqpoint{1.523060in}{2.403632in}}%
\pgfpathclose%
\pgfusepath{stroke,fill}%
\end{pgfscope}%
\begin{pgfscope}%
\pgfpathrectangle{\pgfqpoint{0.100000in}{0.212622in}}{\pgfqpoint{3.696000in}{3.696000in}}%
\pgfusepath{clip}%
\pgfsetbuttcap%
\pgfsetroundjoin%
\definecolor{currentfill}{rgb}{0.121569,0.466667,0.705882}%
\pgfsetfillcolor{currentfill}%
\pgfsetfillopacity{0.345939}%
\pgfsetlinewidth{1.003750pt}%
\definecolor{currentstroke}{rgb}{0.121569,0.466667,0.705882}%
\pgfsetstrokecolor{currentstroke}%
\pgfsetstrokeopacity{0.345939}%
\pgfsetdash{}{0pt}%
\pgfpathmoveto{\pgfqpoint{1.995356in}{2.506177in}}%
\pgfpathcurveto{\pgfqpoint{2.003593in}{2.506177in}}{\pgfqpoint{2.011493in}{2.509449in}}{\pgfqpoint{2.017317in}{2.515273in}}%
\pgfpathcurveto{\pgfqpoint{2.023140in}{2.521097in}}{\pgfqpoint{2.026413in}{2.528997in}}{\pgfqpoint{2.026413in}{2.537234in}}%
\pgfpathcurveto{\pgfqpoint{2.026413in}{2.545470in}}{\pgfqpoint{2.023140in}{2.553370in}}{\pgfqpoint{2.017317in}{2.559194in}}%
\pgfpathcurveto{\pgfqpoint{2.011493in}{2.565018in}}{\pgfqpoint{2.003593in}{2.568290in}}{\pgfqpoint{1.995356in}{2.568290in}}%
\pgfpathcurveto{\pgfqpoint{1.987120in}{2.568290in}}{\pgfqpoint{1.979220in}{2.565018in}}{\pgfqpoint{1.973396in}{2.559194in}}%
\pgfpathcurveto{\pgfqpoint{1.967572in}{2.553370in}}{\pgfqpoint{1.964300in}{2.545470in}}{\pgfqpoint{1.964300in}{2.537234in}}%
\pgfpathcurveto{\pgfqpoint{1.964300in}{2.528997in}}{\pgfqpoint{1.967572in}{2.521097in}}{\pgfqpoint{1.973396in}{2.515273in}}%
\pgfpathcurveto{\pgfqpoint{1.979220in}{2.509449in}}{\pgfqpoint{1.987120in}{2.506177in}}{\pgfqpoint{1.995356in}{2.506177in}}%
\pgfpathclose%
\pgfusepath{stroke,fill}%
\end{pgfscope}%
\begin{pgfscope}%
\pgfpathrectangle{\pgfqpoint{0.100000in}{0.212622in}}{\pgfqpoint{3.696000in}{3.696000in}}%
\pgfusepath{clip}%
\pgfsetbuttcap%
\pgfsetroundjoin%
\definecolor{currentfill}{rgb}{0.121569,0.466667,0.705882}%
\pgfsetfillcolor{currentfill}%
\pgfsetfillopacity{0.346363}%
\pgfsetlinewidth{1.003750pt}%
\definecolor{currentstroke}{rgb}{0.121569,0.466667,0.705882}%
\pgfsetstrokecolor{currentstroke}%
\pgfsetstrokeopacity{0.346363}%
\pgfsetdash{}{0pt}%
\pgfpathmoveto{\pgfqpoint{1.521042in}{2.399931in}}%
\pgfpathcurveto{\pgfqpoint{1.529279in}{2.399931in}}{\pgfqpoint{1.537179in}{2.403203in}}{\pgfqpoint{1.543003in}{2.409027in}}%
\pgfpathcurveto{\pgfqpoint{1.548827in}{2.414851in}}{\pgfqpoint{1.552099in}{2.422751in}}{\pgfqpoint{1.552099in}{2.430988in}}%
\pgfpathcurveto{\pgfqpoint{1.552099in}{2.439224in}}{\pgfqpoint{1.548827in}{2.447124in}}{\pgfqpoint{1.543003in}{2.452948in}}%
\pgfpathcurveto{\pgfqpoint{1.537179in}{2.458772in}}{\pgfqpoint{1.529279in}{2.462044in}}{\pgfqpoint{1.521042in}{2.462044in}}%
\pgfpathcurveto{\pgfqpoint{1.512806in}{2.462044in}}{\pgfqpoint{1.504906in}{2.458772in}}{\pgfqpoint{1.499082in}{2.452948in}}%
\pgfpathcurveto{\pgfqpoint{1.493258in}{2.447124in}}{\pgfqpoint{1.489986in}{2.439224in}}{\pgfqpoint{1.489986in}{2.430988in}}%
\pgfpathcurveto{\pgfqpoint{1.489986in}{2.422751in}}{\pgfqpoint{1.493258in}{2.414851in}}{\pgfqpoint{1.499082in}{2.409027in}}%
\pgfpathcurveto{\pgfqpoint{1.504906in}{2.403203in}}{\pgfqpoint{1.512806in}{2.399931in}}{\pgfqpoint{1.521042in}{2.399931in}}%
\pgfpathclose%
\pgfusepath{stroke,fill}%
\end{pgfscope}%
\begin{pgfscope}%
\pgfpathrectangle{\pgfqpoint{0.100000in}{0.212622in}}{\pgfqpoint{3.696000in}{3.696000in}}%
\pgfusepath{clip}%
\pgfsetbuttcap%
\pgfsetroundjoin%
\definecolor{currentfill}{rgb}{0.121569,0.466667,0.705882}%
\pgfsetfillcolor{currentfill}%
\pgfsetfillopacity{0.346782}%
\pgfsetlinewidth{1.003750pt}%
\definecolor{currentstroke}{rgb}{0.121569,0.466667,0.705882}%
\pgfsetstrokecolor{currentstroke}%
\pgfsetstrokeopacity{0.346782}%
\pgfsetdash{}{0pt}%
\pgfpathmoveto{\pgfqpoint{1.999999in}{2.505755in}}%
\pgfpathcurveto{\pgfqpoint{2.008236in}{2.505755in}}{\pgfqpoint{2.016136in}{2.509027in}}{\pgfqpoint{2.021960in}{2.514851in}}%
\pgfpathcurveto{\pgfqpoint{2.027784in}{2.520675in}}{\pgfqpoint{2.031056in}{2.528575in}}{\pgfqpoint{2.031056in}{2.536812in}}%
\pgfpathcurveto{\pgfqpoint{2.031056in}{2.545048in}}{\pgfqpoint{2.027784in}{2.552948in}}{\pgfqpoint{2.021960in}{2.558772in}}%
\pgfpathcurveto{\pgfqpoint{2.016136in}{2.564596in}}{\pgfqpoint{2.008236in}{2.567868in}}{\pgfqpoint{1.999999in}{2.567868in}}%
\pgfpathcurveto{\pgfqpoint{1.991763in}{2.567868in}}{\pgfqpoint{1.983863in}{2.564596in}}{\pgfqpoint{1.978039in}{2.558772in}}%
\pgfpathcurveto{\pgfqpoint{1.972215in}{2.552948in}}{\pgfqpoint{1.968943in}{2.545048in}}{\pgfqpoint{1.968943in}{2.536812in}}%
\pgfpathcurveto{\pgfqpoint{1.968943in}{2.528575in}}{\pgfqpoint{1.972215in}{2.520675in}}{\pgfqpoint{1.978039in}{2.514851in}}%
\pgfpathcurveto{\pgfqpoint{1.983863in}{2.509027in}}{\pgfqpoint{1.991763in}{2.505755in}}{\pgfqpoint{1.999999in}{2.505755in}}%
\pgfpathclose%
\pgfusepath{stroke,fill}%
\end{pgfscope}%
\begin{pgfscope}%
\pgfpathrectangle{\pgfqpoint{0.100000in}{0.212622in}}{\pgfqpoint{3.696000in}{3.696000in}}%
\pgfusepath{clip}%
\pgfsetbuttcap%
\pgfsetroundjoin%
\definecolor{currentfill}{rgb}{0.121569,0.466667,0.705882}%
\pgfsetfillcolor{currentfill}%
\pgfsetfillopacity{0.347069}%
\pgfsetlinewidth{1.003750pt}%
\definecolor{currentstroke}{rgb}{0.121569,0.466667,0.705882}%
\pgfsetstrokecolor{currentstroke}%
\pgfsetstrokeopacity{0.347069}%
\pgfsetdash{}{0pt}%
\pgfpathmoveto{\pgfqpoint{1.519474in}{2.397033in}}%
\pgfpathcurveto{\pgfqpoint{1.527710in}{2.397033in}}{\pgfqpoint{1.535610in}{2.400306in}}{\pgfqpoint{1.541434in}{2.406129in}}%
\pgfpathcurveto{\pgfqpoint{1.547258in}{2.411953in}}{\pgfqpoint{1.550530in}{2.419853in}}{\pgfqpoint{1.550530in}{2.428090in}}%
\pgfpathcurveto{\pgfqpoint{1.550530in}{2.436326in}}{\pgfqpoint{1.547258in}{2.444226in}}{\pgfqpoint{1.541434in}{2.450050in}}%
\pgfpathcurveto{\pgfqpoint{1.535610in}{2.455874in}}{\pgfqpoint{1.527710in}{2.459146in}}{\pgfqpoint{1.519474in}{2.459146in}}%
\pgfpathcurveto{\pgfqpoint{1.511237in}{2.459146in}}{\pgfqpoint{1.503337in}{2.455874in}}{\pgfqpoint{1.497513in}{2.450050in}}%
\pgfpathcurveto{\pgfqpoint{1.491690in}{2.444226in}}{\pgfqpoint{1.488417in}{2.436326in}}{\pgfqpoint{1.488417in}{2.428090in}}%
\pgfpathcurveto{\pgfqpoint{1.488417in}{2.419853in}}{\pgfqpoint{1.491690in}{2.411953in}}{\pgfqpoint{1.497513in}{2.406129in}}%
\pgfpathcurveto{\pgfqpoint{1.503337in}{2.400306in}}{\pgfqpoint{1.511237in}{2.397033in}}{\pgfqpoint{1.519474in}{2.397033in}}%
\pgfpathclose%
\pgfusepath{stroke,fill}%
\end{pgfscope}%
\begin{pgfscope}%
\pgfpathrectangle{\pgfqpoint{0.100000in}{0.212622in}}{\pgfqpoint{3.696000in}{3.696000in}}%
\pgfusepath{clip}%
\pgfsetbuttcap%
\pgfsetroundjoin%
\definecolor{currentfill}{rgb}{0.121569,0.466667,0.705882}%
\pgfsetfillcolor{currentfill}%
\pgfsetfillopacity{0.347604}%
\pgfsetlinewidth{1.003750pt}%
\definecolor{currentstroke}{rgb}{0.121569,0.466667,0.705882}%
\pgfsetstrokecolor{currentstroke}%
\pgfsetstrokeopacity{0.347604}%
\pgfsetdash{}{0pt}%
\pgfpathmoveto{\pgfqpoint{1.518365in}{2.394867in}}%
\pgfpathcurveto{\pgfqpoint{1.526602in}{2.394867in}}{\pgfqpoint{1.534502in}{2.398139in}}{\pgfqpoint{1.540325in}{2.403963in}}%
\pgfpathcurveto{\pgfqpoint{1.546149in}{2.409787in}}{\pgfqpoint{1.549422in}{2.417687in}}{\pgfqpoint{1.549422in}{2.425924in}}%
\pgfpathcurveto{\pgfqpoint{1.549422in}{2.434160in}}{\pgfqpoint{1.546149in}{2.442060in}}{\pgfqpoint{1.540325in}{2.447884in}}%
\pgfpathcurveto{\pgfqpoint{1.534502in}{2.453708in}}{\pgfqpoint{1.526602in}{2.456980in}}{\pgfqpoint{1.518365in}{2.456980in}}%
\pgfpathcurveto{\pgfqpoint{1.510129in}{2.456980in}}{\pgfqpoint{1.502229in}{2.453708in}}{\pgfqpoint{1.496405in}{2.447884in}}%
\pgfpathcurveto{\pgfqpoint{1.490581in}{2.442060in}}{\pgfqpoint{1.487309in}{2.434160in}}{\pgfqpoint{1.487309in}{2.425924in}}%
\pgfpathcurveto{\pgfqpoint{1.487309in}{2.417687in}}{\pgfqpoint{1.490581in}{2.409787in}}{\pgfqpoint{1.496405in}{2.403963in}}%
\pgfpathcurveto{\pgfqpoint{1.502229in}{2.398139in}}{\pgfqpoint{1.510129in}{2.394867in}}{\pgfqpoint{1.518365in}{2.394867in}}%
\pgfpathclose%
\pgfusepath{stroke,fill}%
\end{pgfscope}%
\begin{pgfscope}%
\pgfpathrectangle{\pgfqpoint{0.100000in}{0.212622in}}{\pgfqpoint{3.696000in}{3.696000in}}%
\pgfusepath{clip}%
\pgfsetbuttcap%
\pgfsetroundjoin%
\definecolor{currentfill}{rgb}{0.121569,0.466667,0.705882}%
\pgfsetfillcolor{currentfill}%
\pgfsetfillopacity{0.347961}%
\pgfsetlinewidth{1.003750pt}%
\definecolor{currentstroke}{rgb}{0.121569,0.466667,0.705882}%
\pgfsetstrokecolor{currentstroke}%
\pgfsetstrokeopacity{0.347961}%
\pgfsetdash{}{0pt}%
\pgfpathmoveto{\pgfqpoint{2.007024in}{2.504513in}}%
\pgfpathcurveto{\pgfqpoint{2.015261in}{2.504513in}}{\pgfqpoint{2.023161in}{2.507785in}}{\pgfqpoint{2.028985in}{2.513609in}}%
\pgfpathcurveto{\pgfqpoint{2.034809in}{2.519433in}}{\pgfqpoint{2.038081in}{2.527333in}}{\pgfqpoint{2.038081in}{2.535569in}}%
\pgfpathcurveto{\pgfqpoint{2.038081in}{2.543806in}}{\pgfqpoint{2.034809in}{2.551706in}}{\pgfqpoint{2.028985in}{2.557530in}}%
\pgfpathcurveto{\pgfqpoint{2.023161in}{2.563354in}}{\pgfqpoint{2.015261in}{2.566626in}}{\pgfqpoint{2.007024in}{2.566626in}}%
\pgfpathcurveto{\pgfqpoint{1.998788in}{2.566626in}}{\pgfqpoint{1.990888in}{2.563354in}}{\pgfqpoint{1.985064in}{2.557530in}}%
\pgfpathcurveto{\pgfqpoint{1.979240in}{2.551706in}}{\pgfqpoint{1.975968in}{2.543806in}}{\pgfqpoint{1.975968in}{2.535569in}}%
\pgfpathcurveto{\pgfqpoint{1.975968in}{2.527333in}}{\pgfqpoint{1.979240in}{2.519433in}}{\pgfqpoint{1.985064in}{2.513609in}}%
\pgfpathcurveto{\pgfqpoint{1.990888in}{2.507785in}}{\pgfqpoint{1.998788in}{2.504513in}}{\pgfqpoint{2.007024in}{2.504513in}}%
\pgfpathclose%
\pgfusepath{stroke,fill}%
\end{pgfscope}%
\begin{pgfscope}%
\pgfpathrectangle{\pgfqpoint{0.100000in}{0.212622in}}{\pgfqpoint{3.696000in}{3.696000in}}%
\pgfusepath{clip}%
\pgfsetbuttcap%
\pgfsetroundjoin%
\definecolor{currentfill}{rgb}{0.121569,0.466667,0.705882}%
\pgfsetfillcolor{currentfill}%
\pgfsetfillopacity{0.348607}%
\pgfsetlinewidth{1.003750pt}%
\definecolor{currentstroke}{rgb}{0.121569,0.466667,0.705882}%
\pgfsetstrokecolor{currentstroke}%
\pgfsetstrokeopacity{0.348607}%
\pgfsetdash{}{0pt}%
\pgfpathmoveto{\pgfqpoint{1.516391in}{2.391109in}}%
\pgfpathcurveto{\pgfqpoint{1.524627in}{2.391109in}}{\pgfqpoint{1.532527in}{2.394381in}}{\pgfqpoint{1.538351in}{2.400205in}}%
\pgfpathcurveto{\pgfqpoint{1.544175in}{2.406029in}}{\pgfqpoint{1.547448in}{2.413929in}}{\pgfqpoint{1.547448in}{2.422165in}}%
\pgfpathcurveto{\pgfqpoint{1.547448in}{2.430402in}}{\pgfqpoint{1.544175in}{2.438302in}}{\pgfqpoint{1.538351in}{2.444126in}}%
\pgfpathcurveto{\pgfqpoint{1.532527in}{2.449950in}}{\pgfqpoint{1.524627in}{2.453222in}}{\pgfqpoint{1.516391in}{2.453222in}}%
\pgfpathcurveto{\pgfqpoint{1.508155in}{2.453222in}}{\pgfqpoint{1.500255in}{2.449950in}}{\pgfqpoint{1.494431in}{2.444126in}}%
\pgfpathcurveto{\pgfqpoint{1.488607in}{2.438302in}}{\pgfqpoint{1.485335in}{2.430402in}}{\pgfqpoint{1.485335in}{2.422165in}}%
\pgfpathcurveto{\pgfqpoint{1.485335in}{2.413929in}}{\pgfqpoint{1.488607in}{2.406029in}}{\pgfqpoint{1.494431in}{2.400205in}}%
\pgfpathcurveto{\pgfqpoint{1.500255in}{2.394381in}}{\pgfqpoint{1.508155in}{2.391109in}}{\pgfqpoint{1.516391in}{2.391109in}}%
\pgfpathclose%
\pgfusepath{stroke,fill}%
\end{pgfscope}%
\begin{pgfscope}%
\pgfpathrectangle{\pgfqpoint{0.100000in}{0.212622in}}{\pgfqpoint{3.696000in}{3.696000in}}%
\pgfusepath{clip}%
\pgfsetbuttcap%
\pgfsetroundjoin%
\definecolor{currentfill}{rgb}{0.121569,0.466667,0.705882}%
\pgfsetfillcolor{currentfill}%
\pgfsetfillopacity{0.349374}%
\pgfsetlinewidth{1.003750pt}%
\definecolor{currentstroke}{rgb}{0.121569,0.466667,0.705882}%
\pgfsetstrokecolor{currentstroke}%
\pgfsetstrokeopacity{0.349374}%
\pgfsetdash{}{0pt}%
\pgfpathmoveto{\pgfqpoint{2.014807in}{2.503671in}}%
\pgfpathcurveto{\pgfqpoint{2.023044in}{2.503671in}}{\pgfqpoint{2.030944in}{2.506943in}}{\pgfqpoint{2.036768in}{2.512767in}}%
\pgfpathcurveto{\pgfqpoint{2.042592in}{2.518591in}}{\pgfqpoint{2.045864in}{2.526491in}}{\pgfqpoint{2.045864in}{2.534727in}}%
\pgfpathcurveto{\pgfqpoint{2.045864in}{2.542963in}}{\pgfqpoint{2.042592in}{2.550863in}}{\pgfqpoint{2.036768in}{2.556687in}}%
\pgfpathcurveto{\pgfqpoint{2.030944in}{2.562511in}}{\pgfqpoint{2.023044in}{2.565784in}}{\pgfqpoint{2.014807in}{2.565784in}}%
\pgfpathcurveto{\pgfqpoint{2.006571in}{2.565784in}}{\pgfqpoint{1.998671in}{2.562511in}}{\pgfqpoint{1.992847in}{2.556687in}}%
\pgfpathcurveto{\pgfqpoint{1.987023in}{2.550863in}}{\pgfqpoint{1.983751in}{2.542963in}}{\pgfqpoint{1.983751in}{2.534727in}}%
\pgfpathcurveto{\pgfqpoint{1.983751in}{2.526491in}}{\pgfqpoint{1.987023in}{2.518591in}}{\pgfqpoint{1.992847in}{2.512767in}}%
\pgfpathcurveto{\pgfqpoint{1.998671in}{2.506943in}}{\pgfqpoint{2.006571in}{2.503671in}}{\pgfqpoint{2.014807in}{2.503671in}}%
\pgfpathclose%
\pgfusepath{stroke,fill}%
\end{pgfscope}%
\begin{pgfscope}%
\pgfpathrectangle{\pgfqpoint{0.100000in}{0.212622in}}{\pgfqpoint{3.696000in}{3.696000in}}%
\pgfusepath{clip}%
\pgfsetbuttcap%
\pgfsetroundjoin%
\definecolor{currentfill}{rgb}{0.121569,0.466667,0.705882}%
\pgfsetfillcolor{currentfill}%
\pgfsetfillopacity{0.350447}%
\pgfsetlinewidth{1.003750pt}%
\definecolor{currentstroke}{rgb}{0.121569,0.466667,0.705882}%
\pgfsetstrokecolor{currentstroke}%
\pgfsetstrokeopacity{0.350447}%
\pgfsetdash{}{0pt}%
\pgfpathmoveto{\pgfqpoint{1.512311in}{2.384744in}}%
\pgfpathcurveto{\pgfqpoint{1.520548in}{2.384744in}}{\pgfqpoint{1.528448in}{2.388016in}}{\pgfqpoint{1.534272in}{2.393840in}}%
\pgfpathcurveto{\pgfqpoint{1.540096in}{2.399664in}}{\pgfqpoint{1.543368in}{2.407564in}}{\pgfqpoint{1.543368in}{2.415800in}}%
\pgfpathcurveto{\pgfqpoint{1.543368in}{2.424037in}}{\pgfqpoint{1.540096in}{2.431937in}}{\pgfqpoint{1.534272in}{2.437761in}}%
\pgfpathcurveto{\pgfqpoint{1.528448in}{2.443584in}}{\pgfqpoint{1.520548in}{2.446857in}}{\pgfqpoint{1.512311in}{2.446857in}}%
\pgfpathcurveto{\pgfqpoint{1.504075in}{2.446857in}}{\pgfqpoint{1.496175in}{2.443584in}}{\pgfqpoint{1.490351in}{2.437761in}}%
\pgfpathcurveto{\pgfqpoint{1.484527in}{2.431937in}}{\pgfqpoint{1.481255in}{2.424037in}}{\pgfqpoint{1.481255in}{2.415800in}}%
\pgfpathcurveto{\pgfqpoint{1.481255in}{2.407564in}}{\pgfqpoint{1.484527in}{2.399664in}}{\pgfqpoint{1.490351in}{2.393840in}}%
\pgfpathcurveto{\pgfqpoint{1.496175in}{2.388016in}}{\pgfqpoint{1.504075in}{2.384744in}}{\pgfqpoint{1.512311in}{2.384744in}}%
\pgfpathclose%
\pgfusepath{stroke,fill}%
\end{pgfscope}%
\begin{pgfscope}%
\pgfpathrectangle{\pgfqpoint{0.100000in}{0.212622in}}{\pgfqpoint{3.696000in}{3.696000in}}%
\pgfusepath{clip}%
\pgfsetbuttcap%
\pgfsetroundjoin%
\definecolor{currentfill}{rgb}{0.121569,0.466667,0.705882}%
\pgfsetfillcolor{currentfill}%
\pgfsetfillopacity{0.350923}%
\pgfsetlinewidth{1.003750pt}%
\definecolor{currentstroke}{rgb}{0.121569,0.466667,0.705882}%
\pgfsetstrokecolor{currentstroke}%
\pgfsetstrokeopacity{0.350923}%
\pgfsetdash{}{0pt}%
\pgfpathmoveto{\pgfqpoint{2.023242in}{2.503095in}}%
\pgfpathcurveto{\pgfqpoint{2.031478in}{2.503095in}}{\pgfqpoint{2.039378in}{2.506367in}}{\pgfqpoint{2.045202in}{2.512191in}}%
\pgfpathcurveto{\pgfqpoint{2.051026in}{2.518015in}}{\pgfqpoint{2.054298in}{2.525915in}}{\pgfqpoint{2.054298in}{2.534151in}}%
\pgfpathcurveto{\pgfqpoint{2.054298in}{2.542388in}}{\pgfqpoint{2.051026in}{2.550288in}}{\pgfqpoint{2.045202in}{2.556112in}}%
\pgfpathcurveto{\pgfqpoint{2.039378in}{2.561935in}}{\pgfqpoint{2.031478in}{2.565208in}}{\pgfqpoint{2.023242in}{2.565208in}}%
\pgfpathcurveto{\pgfqpoint{2.015006in}{2.565208in}}{\pgfqpoint{2.007106in}{2.561935in}}{\pgfqpoint{2.001282in}{2.556112in}}%
\pgfpathcurveto{\pgfqpoint{1.995458in}{2.550288in}}{\pgfqpoint{1.992185in}{2.542388in}}{\pgfqpoint{1.992185in}{2.534151in}}%
\pgfpathcurveto{\pgfqpoint{1.992185in}{2.525915in}}{\pgfqpoint{1.995458in}{2.518015in}}{\pgfqpoint{2.001282in}{2.512191in}}%
\pgfpathcurveto{\pgfqpoint{2.007106in}{2.506367in}}{\pgfqpoint{2.015006in}{2.503095in}}{\pgfqpoint{2.023242in}{2.503095in}}%
\pgfpathclose%
\pgfusepath{stroke,fill}%
\end{pgfscope}%
\begin{pgfscope}%
\pgfpathrectangle{\pgfqpoint{0.100000in}{0.212622in}}{\pgfqpoint{3.696000in}{3.696000in}}%
\pgfusepath{clip}%
\pgfsetbuttcap%
\pgfsetroundjoin%
\definecolor{currentfill}{rgb}{0.121569,0.466667,0.705882}%
\pgfsetfillcolor{currentfill}%
\pgfsetfillopacity{0.351893}%
\pgfsetlinewidth{1.003750pt}%
\definecolor{currentstroke}{rgb}{0.121569,0.466667,0.705882}%
\pgfsetstrokecolor{currentstroke}%
\pgfsetstrokeopacity{0.351893}%
\pgfsetdash{}{0pt}%
\pgfpathmoveto{\pgfqpoint{1.509355in}{2.379681in}}%
\pgfpathcurveto{\pgfqpoint{1.517592in}{2.379681in}}{\pgfqpoint{1.525492in}{2.382953in}}{\pgfqpoint{1.531316in}{2.388777in}}%
\pgfpathcurveto{\pgfqpoint{1.537140in}{2.394601in}}{\pgfqpoint{1.540412in}{2.402501in}}{\pgfqpoint{1.540412in}{2.410737in}}%
\pgfpathcurveto{\pgfqpoint{1.540412in}{2.418974in}}{\pgfqpoint{1.537140in}{2.426874in}}{\pgfqpoint{1.531316in}{2.432698in}}%
\pgfpathcurveto{\pgfqpoint{1.525492in}{2.438521in}}{\pgfqpoint{1.517592in}{2.441794in}}{\pgfqpoint{1.509355in}{2.441794in}}%
\pgfpathcurveto{\pgfqpoint{1.501119in}{2.441794in}}{\pgfqpoint{1.493219in}{2.438521in}}{\pgfqpoint{1.487395in}{2.432698in}}%
\pgfpathcurveto{\pgfqpoint{1.481571in}{2.426874in}}{\pgfqpoint{1.478299in}{2.418974in}}{\pgfqpoint{1.478299in}{2.410737in}}%
\pgfpathcurveto{\pgfqpoint{1.478299in}{2.402501in}}{\pgfqpoint{1.481571in}{2.394601in}}{\pgfqpoint{1.487395in}{2.388777in}}%
\pgfpathcurveto{\pgfqpoint{1.493219in}{2.382953in}}{\pgfqpoint{1.501119in}{2.379681in}}{\pgfqpoint{1.509355in}{2.379681in}}%
\pgfpathclose%
\pgfusepath{stroke,fill}%
\end{pgfscope}%
\begin{pgfscope}%
\pgfpathrectangle{\pgfqpoint{0.100000in}{0.212622in}}{\pgfqpoint{3.696000in}{3.696000in}}%
\pgfusepath{clip}%
\pgfsetbuttcap%
\pgfsetroundjoin%
\definecolor{currentfill}{rgb}{0.121569,0.466667,0.705882}%
\pgfsetfillcolor{currentfill}%
\pgfsetfillopacity{0.352551}%
\pgfsetlinewidth{1.003750pt}%
\definecolor{currentstroke}{rgb}{0.121569,0.466667,0.705882}%
\pgfsetstrokecolor{currentstroke}%
\pgfsetstrokeopacity{0.352551}%
\pgfsetdash{}{0pt}%
\pgfpathmoveto{\pgfqpoint{2.032304in}{2.502205in}}%
\pgfpathcurveto{\pgfqpoint{2.040540in}{2.502205in}}{\pgfqpoint{2.048440in}{2.505477in}}{\pgfqpoint{2.054264in}{2.511301in}}%
\pgfpathcurveto{\pgfqpoint{2.060088in}{2.517125in}}{\pgfqpoint{2.063361in}{2.525025in}}{\pgfqpoint{2.063361in}{2.533261in}}%
\pgfpathcurveto{\pgfqpoint{2.063361in}{2.541497in}}{\pgfqpoint{2.060088in}{2.549397in}}{\pgfqpoint{2.054264in}{2.555221in}}%
\pgfpathcurveto{\pgfqpoint{2.048440in}{2.561045in}}{\pgfqpoint{2.040540in}{2.564318in}}{\pgfqpoint{2.032304in}{2.564318in}}%
\pgfpathcurveto{\pgfqpoint{2.024068in}{2.564318in}}{\pgfqpoint{2.016168in}{2.561045in}}{\pgfqpoint{2.010344in}{2.555221in}}%
\pgfpathcurveto{\pgfqpoint{2.004520in}{2.549397in}}{\pgfqpoint{2.001248in}{2.541497in}}{\pgfqpoint{2.001248in}{2.533261in}}%
\pgfpathcurveto{\pgfqpoint{2.001248in}{2.525025in}}{\pgfqpoint{2.004520in}{2.517125in}}{\pgfqpoint{2.010344in}{2.511301in}}%
\pgfpathcurveto{\pgfqpoint{2.016168in}{2.505477in}}{\pgfqpoint{2.024068in}{2.502205in}}{\pgfqpoint{2.032304in}{2.502205in}}%
\pgfpathclose%
\pgfusepath{stroke,fill}%
\end{pgfscope}%
\begin{pgfscope}%
\pgfpathrectangle{\pgfqpoint{0.100000in}{0.212622in}}{\pgfqpoint{3.696000in}{3.696000in}}%
\pgfusepath{clip}%
\pgfsetbuttcap%
\pgfsetroundjoin%
\definecolor{currentfill}{rgb}{0.121569,0.466667,0.705882}%
\pgfsetfillcolor{currentfill}%
\pgfsetfillopacity{0.353085}%
\pgfsetlinewidth{1.003750pt}%
\definecolor{currentstroke}{rgb}{0.121569,0.466667,0.705882}%
\pgfsetstrokecolor{currentstroke}%
\pgfsetstrokeopacity{0.353085}%
\pgfsetdash{}{0pt}%
\pgfpathmoveto{\pgfqpoint{1.506710in}{2.375433in}}%
\pgfpathcurveto{\pgfqpoint{1.514946in}{2.375433in}}{\pgfqpoint{1.522846in}{2.378705in}}{\pgfqpoint{1.528670in}{2.384529in}}%
\pgfpathcurveto{\pgfqpoint{1.534494in}{2.390353in}}{\pgfqpoint{1.537766in}{2.398253in}}{\pgfqpoint{1.537766in}{2.406490in}}%
\pgfpathcurveto{\pgfqpoint{1.537766in}{2.414726in}}{\pgfqpoint{1.534494in}{2.422626in}}{\pgfqpoint{1.528670in}{2.428450in}}%
\pgfpathcurveto{\pgfqpoint{1.522846in}{2.434274in}}{\pgfqpoint{1.514946in}{2.437546in}}{\pgfqpoint{1.506710in}{2.437546in}}%
\pgfpathcurveto{\pgfqpoint{1.498474in}{2.437546in}}{\pgfqpoint{1.490574in}{2.434274in}}{\pgfqpoint{1.484750in}{2.428450in}}%
\pgfpathcurveto{\pgfqpoint{1.478926in}{2.422626in}}{\pgfqpoint{1.475653in}{2.414726in}}{\pgfqpoint{1.475653in}{2.406490in}}%
\pgfpathcurveto{\pgfqpoint{1.475653in}{2.398253in}}{\pgfqpoint{1.478926in}{2.390353in}}{\pgfqpoint{1.484750in}{2.384529in}}%
\pgfpathcurveto{\pgfqpoint{1.490574in}{2.378705in}}{\pgfqpoint{1.498474in}{2.375433in}}{\pgfqpoint{1.506710in}{2.375433in}}%
\pgfpathclose%
\pgfusepath{stroke,fill}%
\end{pgfscope}%
\begin{pgfscope}%
\pgfpathrectangle{\pgfqpoint{0.100000in}{0.212622in}}{\pgfqpoint{3.696000in}{3.696000in}}%
\pgfusepath{clip}%
\pgfsetbuttcap%
\pgfsetroundjoin%
\definecolor{currentfill}{rgb}{0.121569,0.466667,0.705882}%
\pgfsetfillcolor{currentfill}%
\pgfsetfillopacity{0.353756}%
\pgfsetlinewidth{1.003750pt}%
\definecolor{currentstroke}{rgb}{0.121569,0.466667,0.705882}%
\pgfsetstrokecolor{currentstroke}%
\pgfsetstrokeopacity{0.353756}%
\pgfsetdash{}{0pt}%
\pgfpathmoveto{\pgfqpoint{1.505257in}{2.373156in}}%
\pgfpathcurveto{\pgfqpoint{1.513494in}{2.373156in}}{\pgfqpoint{1.521394in}{2.376428in}}{\pgfqpoint{1.527218in}{2.382252in}}%
\pgfpathcurveto{\pgfqpoint{1.533041in}{2.388076in}}{\pgfqpoint{1.536314in}{2.395976in}}{\pgfqpoint{1.536314in}{2.404213in}}%
\pgfpathcurveto{\pgfqpoint{1.536314in}{2.412449in}}{\pgfqpoint{1.533041in}{2.420349in}}{\pgfqpoint{1.527218in}{2.426173in}}%
\pgfpathcurveto{\pgfqpoint{1.521394in}{2.431997in}}{\pgfqpoint{1.513494in}{2.435269in}}{\pgfqpoint{1.505257in}{2.435269in}}%
\pgfpathcurveto{\pgfqpoint{1.497021in}{2.435269in}}{\pgfqpoint{1.489121in}{2.431997in}}{\pgfqpoint{1.483297in}{2.426173in}}%
\pgfpathcurveto{\pgfqpoint{1.477473in}{2.420349in}}{\pgfqpoint{1.474201in}{2.412449in}}{\pgfqpoint{1.474201in}{2.404213in}}%
\pgfpathcurveto{\pgfqpoint{1.474201in}{2.395976in}}{\pgfqpoint{1.477473in}{2.388076in}}{\pgfqpoint{1.483297in}{2.382252in}}%
\pgfpathcurveto{\pgfqpoint{1.489121in}{2.376428in}}{\pgfqpoint{1.497021in}{2.373156in}}{\pgfqpoint{1.505257in}{2.373156in}}%
\pgfpathclose%
\pgfusepath{stroke,fill}%
\end{pgfscope}%
\begin{pgfscope}%
\pgfpathrectangle{\pgfqpoint{0.100000in}{0.212622in}}{\pgfqpoint{3.696000in}{3.696000in}}%
\pgfusepath{clip}%
\pgfsetbuttcap%
\pgfsetroundjoin%
\definecolor{currentfill}{rgb}{0.121569,0.466667,0.705882}%
\pgfsetfillcolor{currentfill}%
\pgfsetfillopacity{0.354414}%
\pgfsetlinewidth{1.003750pt}%
\definecolor{currentstroke}{rgb}{0.121569,0.466667,0.705882}%
\pgfsetstrokecolor{currentstroke}%
\pgfsetstrokeopacity{0.354414}%
\pgfsetdash{}{0pt}%
\pgfpathmoveto{\pgfqpoint{2.043560in}{2.499401in}}%
\pgfpathcurveto{\pgfqpoint{2.051796in}{2.499401in}}{\pgfqpoint{2.059696in}{2.502674in}}{\pgfqpoint{2.065520in}{2.508498in}}%
\pgfpathcurveto{\pgfqpoint{2.071344in}{2.514322in}}{\pgfqpoint{2.074616in}{2.522222in}}{\pgfqpoint{2.074616in}{2.530458in}}%
\pgfpathcurveto{\pgfqpoint{2.074616in}{2.538694in}}{\pgfqpoint{2.071344in}{2.546594in}}{\pgfqpoint{2.065520in}{2.552418in}}%
\pgfpathcurveto{\pgfqpoint{2.059696in}{2.558242in}}{\pgfqpoint{2.051796in}{2.561514in}}{\pgfqpoint{2.043560in}{2.561514in}}%
\pgfpathcurveto{\pgfqpoint{2.035323in}{2.561514in}}{\pgfqpoint{2.027423in}{2.558242in}}{\pgfqpoint{2.021599in}{2.552418in}}%
\pgfpathcurveto{\pgfqpoint{2.015775in}{2.546594in}}{\pgfqpoint{2.012503in}{2.538694in}}{\pgfqpoint{2.012503in}{2.530458in}}%
\pgfpathcurveto{\pgfqpoint{2.012503in}{2.522222in}}{\pgfqpoint{2.015775in}{2.514322in}}{\pgfqpoint{2.021599in}{2.508498in}}%
\pgfpathcurveto{\pgfqpoint{2.027423in}{2.502674in}}{\pgfqpoint{2.035323in}{2.499401in}}{\pgfqpoint{2.043560in}{2.499401in}}%
\pgfpathclose%
\pgfusepath{stroke,fill}%
\end{pgfscope}%
\begin{pgfscope}%
\pgfpathrectangle{\pgfqpoint{0.100000in}{0.212622in}}{\pgfqpoint{3.696000in}{3.696000in}}%
\pgfusepath{clip}%
\pgfsetbuttcap%
\pgfsetroundjoin%
\definecolor{currentfill}{rgb}{0.121569,0.466667,0.705882}%
\pgfsetfillcolor{currentfill}%
\pgfsetfillopacity{0.354967}%
\pgfsetlinewidth{1.003750pt}%
\definecolor{currentstroke}{rgb}{0.121569,0.466667,0.705882}%
\pgfsetstrokecolor{currentstroke}%
\pgfsetstrokeopacity{0.354967}%
\pgfsetdash{}{0pt}%
\pgfpathmoveto{\pgfqpoint{1.502522in}{2.369019in}}%
\pgfpathcurveto{\pgfqpoint{1.510758in}{2.369019in}}{\pgfqpoint{1.518658in}{2.372292in}}{\pgfqpoint{1.524482in}{2.378115in}}%
\pgfpathcurveto{\pgfqpoint{1.530306in}{2.383939in}}{\pgfqpoint{1.533578in}{2.391839in}}{\pgfqpoint{1.533578in}{2.400076in}}%
\pgfpathcurveto{\pgfqpoint{1.533578in}{2.408312in}}{\pgfqpoint{1.530306in}{2.416212in}}{\pgfqpoint{1.524482in}{2.422036in}}%
\pgfpathcurveto{\pgfqpoint{1.518658in}{2.427860in}}{\pgfqpoint{1.510758in}{2.431132in}}{\pgfqpoint{1.502522in}{2.431132in}}%
\pgfpathcurveto{\pgfqpoint{1.494285in}{2.431132in}}{\pgfqpoint{1.486385in}{2.427860in}}{\pgfqpoint{1.480561in}{2.422036in}}%
\pgfpathcurveto{\pgfqpoint{1.474737in}{2.416212in}}{\pgfqpoint{1.471465in}{2.408312in}}{\pgfqpoint{1.471465in}{2.400076in}}%
\pgfpathcurveto{\pgfqpoint{1.471465in}{2.391839in}}{\pgfqpoint{1.474737in}{2.383939in}}{\pgfqpoint{1.480561in}{2.378115in}}%
\pgfpathcurveto{\pgfqpoint{1.486385in}{2.372292in}}{\pgfqpoint{1.494285in}{2.369019in}}{\pgfqpoint{1.502522in}{2.369019in}}%
\pgfpathclose%
\pgfusepath{stroke,fill}%
\end{pgfscope}%
\begin{pgfscope}%
\pgfpathrectangle{\pgfqpoint{0.100000in}{0.212622in}}{\pgfqpoint{3.696000in}{3.696000in}}%
\pgfusepath{clip}%
\pgfsetbuttcap%
\pgfsetroundjoin%
\definecolor{currentfill}{rgb}{0.121569,0.466667,0.705882}%
\pgfsetfillcolor{currentfill}%
\pgfsetfillopacity{0.355385}%
\pgfsetlinewidth{1.003750pt}%
\definecolor{currentstroke}{rgb}{0.121569,0.466667,0.705882}%
\pgfsetstrokecolor{currentstroke}%
\pgfsetstrokeopacity{0.355385}%
\pgfsetdash{}{0pt}%
\pgfpathmoveto{\pgfqpoint{2.049941in}{2.498039in}}%
\pgfpathcurveto{\pgfqpoint{2.058177in}{2.498039in}}{\pgfqpoint{2.066077in}{2.501311in}}{\pgfqpoint{2.071901in}{2.507135in}}%
\pgfpathcurveto{\pgfqpoint{2.077725in}{2.512959in}}{\pgfqpoint{2.080997in}{2.520859in}}{\pgfqpoint{2.080997in}{2.529095in}}%
\pgfpathcurveto{\pgfqpoint{2.080997in}{2.537332in}}{\pgfqpoint{2.077725in}{2.545232in}}{\pgfqpoint{2.071901in}{2.551056in}}%
\pgfpathcurveto{\pgfqpoint{2.066077in}{2.556880in}}{\pgfqpoint{2.058177in}{2.560152in}}{\pgfqpoint{2.049941in}{2.560152in}}%
\pgfpathcurveto{\pgfqpoint{2.041705in}{2.560152in}}{\pgfqpoint{2.033804in}{2.556880in}}{\pgfqpoint{2.027981in}{2.551056in}}%
\pgfpathcurveto{\pgfqpoint{2.022157in}{2.545232in}}{\pgfqpoint{2.018884in}{2.537332in}}{\pgfqpoint{2.018884in}{2.529095in}}%
\pgfpathcurveto{\pgfqpoint{2.018884in}{2.520859in}}{\pgfqpoint{2.022157in}{2.512959in}}{\pgfqpoint{2.027981in}{2.507135in}}%
\pgfpathcurveto{\pgfqpoint{2.033804in}{2.501311in}}{\pgfqpoint{2.041705in}{2.498039in}}{\pgfqpoint{2.049941in}{2.498039in}}%
\pgfpathclose%
\pgfusepath{stroke,fill}%
\end{pgfscope}%
\begin{pgfscope}%
\pgfpathrectangle{\pgfqpoint{0.100000in}{0.212622in}}{\pgfqpoint{3.696000in}{3.696000in}}%
\pgfusepath{clip}%
\pgfsetbuttcap%
\pgfsetroundjoin%
\definecolor{currentfill}{rgb}{0.121569,0.466667,0.705882}%
\pgfsetfillcolor{currentfill}%
\pgfsetfillopacity{0.356524}%
\pgfsetlinewidth{1.003750pt}%
\definecolor{currentstroke}{rgb}{0.121569,0.466667,0.705882}%
\pgfsetstrokecolor{currentstroke}%
\pgfsetstrokeopacity{0.356524}%
\pgfsetdash{}{0pt}%
\pgfpathmoveto{\pgfqpoint{2.057297in}{2.496648in}}%
\pgfpathcurveto{\pgfqpoint{2.065533in}{2.496648in}}{\pgfqpoint{2.073433in}{2.499920in}}{\pgfqpoint{2.079257in}{2.505744in}}%
\pgfpathcurveto{\pgfqpoint{2.085081in}{2.511568in}}{\pgfqpoint{2.088353in}{2.519468in}}{\pgfqpoint{2.088353in}{2.527705in}}%
\pgfpathcurveto{\pgfqpoint{2.088353in}{2.535941in}}{\pgfqpoint{2.085081in}{2.543841in}}{\pgfqpoint{2.079257in}{2.549665in}}%
\pgfpathcurveto{\pgfqpoint{2.073433in}{2.555489in}}{\pgfqpoint{2.065533in}{2.558761in}}{\pgfqpoint{2.057297in}{2.558761in}}%
\pgfpathcurveto{\pgfqpoint{2.049061in}{2.558761in}}{\pgfqpoint{2.041160in}{2.555489in}}{\pgfqpoint{2.035337in}{2.549665in}}%
\pgfpathcurveto{\pgfqpoint{2.029513in}{2.543841in}}{\pgfqpoint{2.026240in}{2.535941in}}{\pgfqpoint{2.026240in}{2.527705in}}%
\pgfpathcurveto{\pgfqpoint{2.026240in}{2.519468in}}{\pgfqpoint{2.029513in}{2.511568in}}{\pgfqpoint{2.035337in}{2.505744in}}%
\pgfpathcurveto{\pgfqpoint{2.041160in}{2.499920in}}{\pgfqpoint{2.049061in}{2.496648in}}{\pgfqpoint{2.057297in}{2.496648in}}%
\pgfpathclose%
\pgfusepath{stroke,fill}%
\end{pgfscope}%
\begin{pgfscope}%
\pgfpathrectangle{\pgfqpoint{0.100000in}{0.212622in}}{\pgfqpoint{3.696000in}{3.696000in}}%
\pgfusepath{clip}%
\pgfsetbuttcap%
\pgfsetroundjoin%
\definecolor{currentfill}{rgb}{0.121569,0.466667,0.705882}%
\pgfsetfillcolor{currentfill}%
\pgfsetfillopacity{0.357208}%
\pgfsetlinewidth{1.003750pt}%
\definecolor{currentstroke}{rgb}{0.121569,0.466667,0.705882}%
\pgfsetstrokecolor{currentstroke}%
\pgfsetstrokeopacity{0.357208}%
\pgfsetdash{}{0pt}%
\pgfpathmoveto{\pgfqpoint{1.497654in}{2.361659in}}%
\pgfpathcurveto{\pgfqpoint{1.505890in}{2.361659in}}{\pgfqpoint{1.513790in}{2.364932in}}{\pgfqpoint{1.519614in}{2.370756in}}%
\pgfpathcurveto{\pgfqpoint{1.525438in}{2.376579in}}{\pgfqpoint{1.528710in}{2.384480in}}{\pgfqpoint{1.528710in}{2.392716in}}%
\pgfpathcurveto{\pgfqpoint{1.528710in}{2.400952in}}{\pgfqpoint{1.525438in}{2.408852in}}{\pgfqpoint{1.519614in}{2.414676in}}%
\pgfpathcurveto{\pgfqpoint{1.513790in}{2.420500in}}{\pgfqpoint{1.505890in}{2.423772in}}{\pgfqpoint{1.497654in}{2.423772in}}%
\pgfpathcurveto{\pgfqpoint{1.489417in}{2.423772in}}{\pgfqpoint{1.481517in}{2.420500in}}{\pgfqpoint{1.475694in}{2.414676in}}%
\pgfpathcurveto{\pgfqpoint{1.469870in}{2.408852in}}{\pgfqpoint{1.466597in}{2.400952in}}{\pgfqpoint{1.466597in}{2.392716in}}%
\pgfpathcurveto{\pgfqpoint{1.466597in}{2.384480in}}{\pgfqpoint{1.469870in}{2.376579in}}{\pgfqpoint{1.475694in}{2.370756in}}%
\pgfpathcurveto{\pgfqpoint{1.481517in}{2.364932in}}{\pgfqpoint{1.489417in}{2.361659in}}{\pgfqpoint{1.497654in}{2.361659in}}%
\pgfpathclose%
\pgfusepath{stroke,fill}%
\end{pgfscope}%
\begin{pgfscope}%
\pgfpathrectangle{\pgfqpoint{0.100000in}{0.212622in}}{\pgfqpoint{3.696000in}{3.696000in}}%
\pgfusepath{clip}%
\pgfsetbuttcap%
\pgfsetroundjoin%
\definecolor{currentfill}{rgb}{0.121569,0.466667,0.705882}%
\pgfsetfillcolor{currentfill}%
\pgfsetfillopacity{0.357978}%
\pgfsetlinewidth{1.003750pt}%
\definecolor{currentstroke}{rgb}{0.121569,0.466667,0.705882}%
\pgfsetstrokecolor{currentstroke}%
\pgfsetstrokeopacity{0.357978}%
\pgfsetdash{}{0pt}%
\pgfpathmoveto{\pgfqpoint{2.066090in}{2.495489in}}%
\pgfpathcurveto{\pgfqpoint{2.074326in}{2.495489in}}{\pgfqpoint{2.082226in}{2.498762in}}{\pgfqpoint{2.088050in}{2.504585in}}%
\pgfpathcurveto{\pgfqpoint{2.093874in}{2.510409in}}{\pgfqpoint{2.097146in}{2.518309in}}{\pgfqpoint{2.097146in}{2.526546in}}%
\pgfpathcurveto{\pgfqpoint{2.097146in}{2.534782in}}{\pgfqpoint{2.093874in}{2.542682in}}{\pgfqpoint{2.088050in}{2.548506in}}%
\pgfpathcurveto{\pgfqpoint{2.082226in}{2.554330in}}{\pgfqpoint{2.074326in}{2.557602in}}{\pgfqpoint{2.066090in}{2.557602in}}%
\pgfpathcurveto{\pgfqpoint{2.057853in}{2.557602in}}{\pgfqpoint{2.049953in}{2.554330in}}{\pgfqpoint{2.044129in}{2.548506in}}%
\pgfpathcurveto{\pgfqpoint{2.038305in}{2.542682in}}{\pgfqpoint{2.035033in}{2.534782in}}{\pgfqpoint{2.035033in}{2.526546in}}%
\pgfpathcurveto{\pgfqpoint{2.035033in}{2.518309in}}{\pgfqpoint{2.038305in}{2.510409in}}{\pgfqpoint{2.044129in}{2.504585in}}%
\pgfpathcurveto{\pgfqpoint{2.049953in}{2.498762in}}{\pgfqpoint{2.057853in}{2.495489in}}{\pgfqpoint{2.066090in}{2.495489in}}%
\pgfpathclose%
\pgfusepath{stroke,fill}%
\end{pgfscope}%
\begin{pgfscope}%
\pgfpathrectangle{\pgfqpoint{0.100000in}{0.212622in}}{\pgfqpoint{3.696000in}{3.696000in}}%
\pgfusepath{clip}%
\pgfsetbuttcap%
\pgfsetroundjoin%
\definecolor{currentfill}{rgb}{0.121569,0.466667,0.705882}%
\pgfsetfillcolor{currentfill}%
\pgfsetfillopacity{0.359781}%
\pgfsetlinewidth{1.003750pt}%
\definecolor{currentstroke}{rgb}{0.121569,0.466667,0.705882}%
\pgfsetstrokecolor{currentstroke}%
\pgfsetstrokeopacity{0.359781}%
\pgfsetdash{}{0pt}%
\pgfpathmoveto{\pgfqpoint{2.077973in}{2.492042in}}%
\pgfpathcurveto{\pgfqpoint{2.086210in}{2.492042in}}{\pgfqpoint{2.094110in}{2.495314in}}{\pgfqpoint{2.099934in}{2.501138in}}%
\pgfpathcurveto{\pgfqpoint{2.105758in}{2.506962in}}{\pgfqpoint{2.109030in}{2.514862in}}{\pgfqpoint{2.109030in}{2.523099in}}%
\pgfpathcurveto{\pgfqpoint{2.109030in}{2.531335in}}{\pgfqpoint{2.105758in}{2.539235in}}{\pgfqpoint{2.099934in}{2.545059in}}%
\pgfpathcurveto{\pgfqpoint{2.094110in}{2.550883in}}{\pgfqpoint{2.086210in}{2.554155in}}{\pgfqpoint{2.077973in}{2.554155in}}%
\pgfpathcurveto{\pgfqpoint{2.069737in}{2.554155in}}{\pgfqpoint{2.061837in}{2.550883in}}{\pgfqpoint{2.056013in}{2.545059in}}%
\pgfpathcurveto{\pgfqpoint{2.050189in}{2.539235in}}{\pgfqpoint{2.046917in}{2.531335in}}{\pgfqpoint{2.046917in}{2.523099in}}%
\pgfpathcurveto{\pgfqpoint{2.046917in}{2.514862in}}{\pgfqpoint{2.050189in}{2.506962in}}{\pgfqpoint{2.056013in}{2.501138in}}%
\pgfpathcurveto{\pgfqpoint{2.061837in}{2.495314in}}{\pgfqpoint{2.069737in}{2.492042in}}{\pgfqpoint{2.077973in}{2.492042in}}%
\pgfpathclose%
\pgfusepath{stroke,fill}%
\end{pgfscope}%
\begin{pgfscope}%
\pgfpathrectangle{\pgfqpoint{0.100000in}{0.212622in}}{\pgfqpoint{3.696000in}{3.696000in}}%
\pgfusepath{clip}%
\pgfsetbuttcap%
\pgfsetroundjoin%
\definecolor{currentfill}{rgb}{0.121569,0.466667,0.705882}%
\pgfsetfillcolor{currentfill}%
\pgfsetfillopacity{0.361171}%
\pgfsetlinewidth{1.003750pt}%
\definecolor{currentstroke}{rgb}{0.121569,0.466667,0.705882}%
\pgfsetstrokecolor{currentstroke}%
\pgfsetstrokeopacity{0.361171}%
\pgfsetdash{}{0pt}%
\pgfpathmoveto{\pgfqpoint{1.488531in}{2.347691in}}%
\pgfpathcurveto{\pgfqpoint{1.496767in}{2.347691in}}{\pgfqpoint{1.504668in}{2.350963in}}{\pgfqpoint{1.510491in}{2.356787in}}%
\pgfpathcurveto{\pgfqpoint{1.516315in}{2.362611in}}{\pgfqpoint{1.519588in}{2.370511in}}{\pgfqpoint{1.519588in}{2.378748in}}%
\pgfpathcurveto{\pgfqpoint{1.519588in}{2.386984in}}{\pgfqpoint{1.516315in}{2.394884in}}{\pgfqpoint{1.510491in}{2.400708in}}%
\pgfpathcurveto{\pgfqpoint{1.504668in}{2.406532in}}{\pgfqpoint{1.496767in}{2.409804in}}{\pgfqpoint{1.488531in}{2.409804in}}%
\pgfpathcurveto{\pgfqpoint{1.480295in}{2.409804in}}{\pgfqpoint{1.472395in}{2.406532in}}{\pgfqpoint{1.466571in}{2.400708in}}%
\pgfpathcurveto{\pgfqpoint{1.460747in}{2.394884in}}{\pgfqpoint{1.457475in}{2.386984in}}{\pgfqpoint{1.457475in}{2.378748in}}%
\pgfpathcurveto{\pgfqpoint{1.457475in}{2.370511in}}{\pgfqpoint{1.460747in}{2.362611in}}{\pgfqpoint{1.466571in}{2.356787in}}%
\pgfpathcurveto{\pgfqpoint{1.472395in}{2.350963in}}{\pgfqpoint{1.480295in}{2.347691in}}{\pgfqpoint{1.488531in}{2.347691in}}%
\pgfpathclose%
\pgfusepath{stroke,fill}%
\end{pgfscope}%
\begin{pgfscope}%
\pgfpathrectangle{\pgfqpoint{0.100000in}{0.212622in}}{\pgfqpoint{3.696000in}{3.696000in}}%
\pgfusepath{clip}%
\pgfsetbuttcap%
\pgfsetroundjoin%
\definecolor{currentfill}{rgb}{0.121569,0.466667,0.705882}%
\pgfsetfillcolor{currentfill}%
\pgfsetfillopacity{0.361672}%
\pgfsetlinewidth{1.003750pt}%
\definecolor{currentstroke}{rgb}{0.121569,0.466667,0.705882}%
\pgfsetstrokecolor{currentstroke}%
\pgfsetstrokeopacity{0.361672}%
\pgfsetdash{}{0pt}%
\pgfpathmoveto{\pgfqpoint{2.091253in}{2.487555in}}%
\pgfpathcurveto{\pgfqpoint{2.099489in}{2.487555in}}{\pgfqpoint{2.107389in}{2.490827in}}{\pgfqpoint{2.113213in}{2.496651in}}%
\pgfpathcurveto{\pgfqpoint{2.119037in}{2.502475in}}{\pgfqpoint{2.122309in}{2.510375in}}{\pgfqpoint{2.122309in}{2.518611in}}%
\pgfpathcurveto{\pgfqpoint{2.122309in}{2.526847in}}{\pgfqpoint{2.119037in}{2.534748in}}{\pgfqpoint{2.113213in}{2.540571in}}%
\pgfpathcurveto{\pgfqpoint{2.107389in}{2.546395in}}{\pgfqpoint{2.099489in}{2.549668in}}{\pgfqpoint{2.091253in}{2.549668in}}%
\pgfpathcurveto{\pgfqpoint{2.083017in}{2.549668in}}{\pgfqpoint{2.075117in}{2.546395in}}{\pgfqpoint{2.069293in}{2.540571in}}%
\pgfpathcurveto{\pgfqpoint{2.063469in}{2.534748in}}{\pgfqpoint{2.060196in}{2.526847in}}{\pgfqpoint{2.060196in}{2.518611in}}%
\pgfpathcurveto{\pgfqpoint{2.060196in}{2.510375in}}{\pgfqpoint{2.063469in}{2.502475in}}{\pgfqpoint{2.069293in}{2.496651in}}%
\pgfpathcurveto{\pgfqpoint{2.075117in}{2.490827in}}{\pgfqpoint{2.083017in}{2.487555in}}{\pgfqpoint{2.091253in}{2.487555in}}%
\pgfpathclose%
\pgfusepath{stroke,fill}%
\end{pgfscope}%
\begin{pgfscope}%
\pgfpathrectangle{\pgfqpoint{0.100000in}{0.212622in}}{\pgfqpoint{3.696000in}{3.696000in}}%
\pgfusepath{clip}%
\pgfsetbuttcap%
\pgfsetroundjoin%
\definecolor{currentfill}{rgb}{0.121569,0.466667,0.705882}%
\pgfsetfillcolor{currentfill}%
\pgfsetfillopacity{0.363938}%
\pgfsetlinewidth{1.003750pt}%
\definecolor{currentstroke}{rgb}{0.121569,0.466667,0.705882}%
\pgfsetstrokecolor{currentstroke}%
\pgfsetstrokeopacity{0.363938}%
\pgfsetdash{}{0pt}%
\pgfpathmoveto{\pgfqpoint{2.105587in}{2.484843in}}%
\pgfpathcurveto{\pgfqpoint{2.113824in}{2.484843in}}{\pgfqpoint{2.121724in}{2.488116in}}{\pgfqpoint{2.127547in}{2.493940in}}%
\pgfpathcurveto{\pgfqpoint{2.133371in}{2.499764in}}{\pgfqpoint{2.136644in}{2.507664in}}{\pgfqpoint{2.136644in}{2.515900in}}%
\pgfpathcurveto{\pgfqpoint{2.136644in}{2.524136in}}{\pgfqpoint{2.133371in}{2.532036in}}{\pgfqpoint{2.127547in}{2.537860in}}%
\pgfpathcurveto{\pgfqpoint{2.121724in}{2.543684in}}{\pgfqpoint{2.113824in}{2.546956in}}{\pgfqpoint{2.105587in}{2.546956in}}%
\pgfpathcurveto{\pgfqpoint{2.097351in}{2.546956in}}{\pgfqpoint{2.089451in}{2.543684in}}{\pgfqpoint{2.083627in}{2.537860in}}%
\pgfpathcurveto{\pgfqpoint{2.077803in}{2.532036in}}{\pgfqpoint{2.074531in}{2.524136in}}{\pgfqpoint{2.074531in}{2.515900in}}%
\pgfpathcurveto{\pgfqpoint{2.074531in}{2.507664in}}{\pgfqpoint{2.077803in}{2.499764in}}{\pgfqpoint{2.083627in}{2.493940in}}%
\pgfpathcurveto{\pgfqpoint{2.089451in}{2.488116in}}{\pgfqpoint{2.097351in}{2.484843in}}{\pgfqpoint{2.105587in}{2.484843in}}%
\pgfpathclose%
\pgfusepath{stroke,fill}%
\end{pgfscope}%
\begin{pgfscope}%
\pgfpathrectangle{\pgfqpoint{0.100000in}{0.212622in}}{\pgfqpoint{3.696000in}{3.696000in}}%
\pgfusepath{clip}%
\pgfsetbuttcap%
\pgfsetroundjoin%
\definecolor{currentfill}{rgb}{0.121569,0.466667,0.705882}%
\pgfsetfillcolor{currentfill}%
\pgfsetfillopacity{0.364863}%
\pgfsetlinewidth{1.003750pt}%
\definecolor{currentstroke}{rgb}{0.121569,0.466667,0.705882}%
\pgfsetstrokecolor{currentstroke}%
\pgfsetstrokeopacity{0.364863}%
\pgfsetdash{}{0pt}%
\pgfpathmoveto{\pgfqpoint{1.480246in}{2.332361in}}%
\pgfpathcurveto{\pgfqpoint{1.488482in}{2.332361in}}{\pgfqpoint{1.496382in}{2.335633in}}{\pgfqpoint{1.502206in}{2.341457in}}%
\pgfpathcurveto{\pgfqpoint{1.508030in}{2.347281in}}{\pgfqpoint{1.511302in}{2.355181in}}{\pgfqpoint{1.511302in}{2.363417in}}%
\pgfpathcurveto{\pgfqpoint{1.511302in}{2.371654in}}{\pgfqpoint{1.508030in}{2.379554in}}{\pgfqpoint{1.502206in}{2.385378in}}%
\pgfpathcurveto{\pgfqpoint{1.496382in}{2.391202in}}{\pgfqpoint{1.488482in}{2.394474in}}{\pgfqpoint{1.480246in}{2.394474in}}%
\pgfpathcurveto{\pgfqpoint{1.472009in}{2.394474in}}{\pgfqpoint{1.464109in}{2.391202in}}{\pgfqpoint{1.458285in}{2.385378in}}%
\pgfpathcurveto{\pgfqpoint{1.452462in}{2.379554in}}{\pgfqpoint{1.449189in}{2.371654in}}{\pgfqpoint{1.449189in}{2.363417in}}%
\pgfpathcurveto{\pgfqpoint{1.449189in}{2.355181in}}{\pgfqpoint{1.452462in}{2.347281in}}{\pgfqpoint{1.458285in}{2.341457in}}%
\pgfpathcurveto{\pgfqpoint{1.464109in}{2.335633in}}{\pgfqpoint{1.472009in}{2.332361in}}{\pgfqpoint{1.480246in}{2.332361in}}%
\pgfpathclose%
\pgfusepath{stroke,fill}%
\end{pgfscope}%
\begin{pgfscope}%
\pgfpathrectangle{\pgfqpoint{0.100000in}{0.212622in}}{\pgfqpoint{3.696000in}{3.696000in}}%
\pgfusepath{clip}%
\pgfsetbuttcap%
\pgfsetroundjoin%
\definecolor{currentfill}{rgb}{0.121569,0.466667,0.705882}%
\pgfsetfillcolor{currentfill}%
\pgfsetfillopacity{0.366242}%
\pgfsetlinewidth{1.003750pt}%
\definecolor{currentstroke}{rgb}{0.121569,0.466667,0.705882}%
\pgfsetstrokecolor{currentstroke}%
\pgfsetstrokeopacity{0.366242}%
\pgfsetdash{}{0pt}%
\pgfpathmoveto{\pgfqpoint{2.120380in}{2.481798in}}%
\pgfpathcurveto{\pgfqpoint{2.128616in}{2.481798in}}{\pgfqpoint{2.136516in}{2.485070in}}{\pgfqpoint{2.142340in}{2.490894in}}%
\pgfpathcurveto{\pgfqpoint{2.148164in}{2.496718in}}{\pgfqpoint{2.151436in}{2.504618in}}{\pgfqpoint{2.151436in}{2.512854in}}%
\pgfpathcurveto{\pgfqpoint{2.151436in}{2.521091in}}{\pgfqpoint{2.148164in}{2.528991in}}{\pgfqpoint{2.142340in}{2.534815in}}%
\pgfpathcurveto{\pgfqpoint{2.136516in}{2.540638in}}{\pgfqpoint{2.128616in}{2.543911in}}{\pgfqpoint{2.120380in}{2.543911in}}%
\pgfpathcurveto{\pgfqpoint{2.112144in}{2.543911in}}{\pgfqpoint{2.104244in}{2.540638in}}{\pgfqpoint{2.098420in}{2.534815in}}%
\pgfpathcurveto{\pgfqpoint{2.092596in}{2.528991in}}{\pgfqpoint{2.089323in}{2.521091in}}{\pgfqpoint{2.089323in}{2.512854in}}%
\pgfpathcurveto{\pgfqpoint{2.089323in}{2.504618in}}{\pgfqpoint{2.092596in}{2.496718in}}{\pgfqpoint{2.098420in}{2.490894in}}%
\pgfpathcurveto{\pgfqpoint{2.104244in}{2.485070in}}{\pgfqpoint{2.112144in}{2.481798in}}{\pgfqpoint{2.120380in}{2.481798in}}%
\pgfpathclose%
\pgfusepath{stroke,fill}%
\end{pgfscope}%
\begin{pgfscope}%
\pgfpathrectangle{\pgfqpoint{0.100000in}{0.212622in}}{\pgfqpoint{3.696000in}{3.696000in}}%
\pgfusepath{clip}%
\pgfsetbuttcap%
\pgfsetroundjoin%
\definecolor{currentfill}{rgb}{0.121569,0.466667,0.705882}%
\pgfsetfillcolor{currentfill}%
\pgfsetfillopacity{0.368471}%
\pgfsetlinewidth{1.003750pt}%
\definecolor{currentstroke}{rgb}{0.121569,0.466667,0.705882}%
\pgfsetstrokecolor{currentstroke}%
\pgfsetstrokeopacity{0.368471}%
\pgfsetdash{}{0pt}%
\pgfpathmoveto{\pgfqpoint{1.471699in}{2.319473in}}%
\pgfpathcurveto{\pgfqpoint{1.479935in}{2.319473in}}{\pgfqpoint{1.487836in}{2.322746in}}{\pgfqpoint{1.493659in}{2.328570in}}%
\pgfpathcurveto{\pgfqpoint{1.499483in}{2.334394in}}{\pgfqpoint{1.502756in}{2.342294in}}{\pgfqpoint{1.502756in}{2.350530in}}%
\pgfpathcurveto{\pgfqpoint{1.502756in}{2.358766in}}{\pgfqpoint{1.499483in}{2.366666in}}{\pgfqpoint{1.493659in}{2.372490in}}%
\pgfpathcurveto{\pgfqpoint{1.487836in}{2.378314in}}{\pgfqpoint{1.479935in}{2.381586in}}{\pgfqpoint{1.471699in}{2.381586in}}%
\pgfpathcurveto{\pgfqpoint{1.463463in}{2.381586in}}{\pgfqpoint{1.455563in}{2.378314in}}{\pgfqpoint{1.449739in}{2.372490in}}%
\pgfpathcurveto{\pgfqpoint{1.443915in}{2.366666in}}{\pgfqpoint{1.440643in}{2.358766in}}{\pgfqpoint{1.440643in}{2.350530in}}%
\pgfpathcurveto{\pgfqpoint{1.440643in}{2.342294in}}{\pgfqpoint{1.443915in}{2.334394in}}{\pgfqpoint{1.449739in}{2.328570in}}%
\pgfpathcurveto{\pgfqpoint{1.455563in}{2.322746in}}{\pgfqpoint{1.463463in}{2.319473in}}{\pgfqpoint{1.471699in}{2.319473in}}%
\pgfpathclose%
\pgfusepath{stroke,fill}%
\end{pgfscope}%
\begin{pgfscope}%
\pgfpathrectangle{\pgfqpoint{0.100000in}{0.212622in}}{\pgfqpoint{3.696000in}{3.696000in}}%
\pgfusepath{clip}%
\pgfsetbuttcap%
\pgfsetroundjoin%
\definecolor{currentfill}{rgb}{0.121569,0.466667,0.705882}%
\pgfsetfillcolor{currentfill}%
\pgfsetfillopacity{0.368767}%
\pgfsetlinewidth{1.003750pt}%
\definecolor{currentstroke}{rgb}{0.121569,0.466667,0.705882}%
\pgfsetstrokecolor{currentstroke}%
\pgfsetstrokeopacity{0.368767}%
\pgfsetdash{}{0pt}%
\pgfpathmoveto{\pgfqpoint{2.136676in}{2.477379in}}%
\pgfpathcurveto{\pgfqpoint{2.144912in}{2.477379in}}{\pgfqpoint{2.152812in}{2.480651in}}{\pgfqpoint{2.158636in}{2.486475in}}%
\pgfpathcurveto{\pgfqpoint{2.164460in}{2.492299in}}{\pgfqpoint{2.167732in}{2.500199in}}{\pgfqpoint{2.167732in}{2.508435in}}%
\pgfpathcurveto{\pgfqpoint{2.167732in}{2.516671in}}{\pgfqpoint{2.164460in}{2.524571in}}{\pgfqpoint{2.158636in}{2.530395in}}%
\pgfpathcurveto{\pgfqpoint{2.152812in}{2.536219in}}{\pgfqpoint{2.144912in}{2.539492in}}{\pgfqpoint{2.136676in}{2.539492in}}%
\pgfpathcurveto{\pgfqpoint{2.128440in}{2.539492in}}{\pgfqpoint{2.120540in}{2.536219in}}{\pgfqpoint{2.114716in}{2.530395in}}%
\pgfpathcurveto{\pgfqpoint{2.108892in}{2.524571in}}{\pgfqpoint{2.105619in}{2.516671in}}{\pgfqpoint{2.105619in}{2.508435in}}%
\pgfpathcurveto{\pgfqpoint{2.105619in}{2.500199in}}{\pgfqpoint{2.108892in}{2.492299in}}{\pgfqpoint{2.114716in}{2.486475in}}%
\pgfpathcurveto{\pgfqpoint{2.120540in}{2.480651in}}{\pgfqpoint{2.128440in}{2.477379in}}{\pgfqpoint{2.136676in}{2.477379in}}%
\pgfpathclose%
\pgfusepath{stroke,fill}%
\end{pgfscope}%
\begin{pgfscope}%
\pgfpathrectangle{\pgfqpoint{0.100000in}{0.212622in}}{\pgfqpoint{3.696000in}{3.696000in}}%
\pgfusepath{clip}%
\pgfsetbuttcap%
\pgfsetroundjoin%
\definecolor{currentfill}{rgb}{0.121569,0.466667,0.705882}%
\pgfsetfillcolor{currentfill}%
\pgfsetfillopacity{0.371137}%
\pgfsetlinewidth{1.003750pt}%
\definecolor{currentstroke}{rgb}{0.121569,0.466667,0.705882}%
\pgfsetstrokecolor{currentstroke}%
\pgfsetstrokeopacity{0.371137}%
\pgfsetdash{}{0pt}%
\pgfpathmoveto{\pgfqpoint{2.153319in}{2.470345in}}%
\pgfpathcurveto{\pgfqpoint{2.161556in}{2.470345in}}{\pgfqpoint{2.169456in}{2.473617in}}{\pgfqpoint{2.175280in}{2.479441in}}%
\pgfpathcurveto{\pgfqpoint{2.181104in}{2.485265in}}{\pgfqpoint{2.184376in}{2.493165in}}{\pgfqpoint{2.184376in}{2.501401in}}%
\pgfpathcurveto{\pgfqpoint{2.184376in}{2.509638in}}{\pgfqpoint{2.181104in}{2.517538in}}{\pgfqpoint{2.175280in}{2.523362in}}%
\pgfpathcurveto{\pgfqpoint{2.169456in}{2.529186in}}{\pgfqpoint{2.161556in}{2.532458in}}{\pgfqpoint{2.153319in}{2.532458in}}%
\pgfpathcurveto{\pgfqpoint{2.145083in}{2.532458in}}{\pgfqpoint{2.137183in}{2.529186in}}{\pgfqpoint{2.131359in}{2.523362in}}%
\pgfpathcurveto{\pgfqpoint{2.125535in}{2.517538in}}{\pgfqpoint{2.122263in}{2.509638in}}{\pgfqpoint{2.122263in}{2.501401in}}%
\pgfpathcurveto{\pgfqpoint{2.122263in}{2.493165in}}{\pgfqpoint{2.125535in}{2.485265in}}{\pgfqpoint{2.131359in}{2.479441in}}%
\pgfpathcurveto{\pgfqpoint{2.137183in}{2.473617in}}{\pgfqpoint{2.145083in}{2.470345in}}{\pgfqpoint{2.153319in}{2.470345in}}%
\pgfpathclose%
\pgfusepath{stroke,fill}%
\end{pgfscope}%
\begin{pgfscope}%
\pgfpathrectangle{\pgfqpoint{0.100000in}{0.212622in}}{\pgfqpoint{3.696000in}{3.696000in}}%
\pgfusepath{clip}%
\pgfsetbuttcap%
\pgfsetroundjoin%
\definecolor{currentfill}{rgb}{0.121569,0.466667,0.705882}%
\pgfsetfillcolor{currentfill}%
\pgfsetfillopacity{0.371792}%
\pgfsetlinewidth{1.003750pt}%
\definecolor{currentstroke}{rgb}{0.121569,0.466667,0.705882}%
\pgfsetstrokecolor{currentstroke}%
\pgfsetstrokeopacity{0.371792}%
\pgfsetdash{}{0pt}%
\pgfpathmoveto{\pgfqpoint{1.463275in}{2.307211in}}%
\pgfpathcurveto{\pgfqpoint{1.471511in}{2.307211in}}{\pgfqpoint{1.479411in}{2.310484in}}{\pgfqpoint{1.485235in}{2.316308in}}%
\pgfpathcurveto{\pgfqpoint{1.491059in}{2.322132in}}{\pgfqpoint{1.494331in}{2.330032in}}{\pgfqpoint{1.494331in}{2.338268in}}%
\pgfpathcurveto{\pgfqpoint{1.494331in}{2.346504in}}{\pgfqpoint{1.491059in}{2.354404in}}{\pgfqpoint{1.485235in}{2.360228in}}%
\pgfpathcurveto{\pgfqpoint{1.479411in}{2.366052in}}{\pgfqpoint{1.471511in}{2.369324in}}{\pgfqpoint{1.463275in}{2.369324in}}%
\pgfpathcurveto{\pgfqpoint{1.455039in}{2.369324in}}{\pgfqpoint{1.447139in}{2.366052in}}{\pgfqpoint{1.441315in}{2.360228in}}%
\pgfpathcurveto{\pgfqpoint{1.435491in}{2.354404in}}{\pgfqpoint{1.432218in}{2.346504in}}{\pgfqpoint{1.432218in}{2.338268in}}%
\pgfpathcurveto{\pgfqpoint{1.432218in}{2.330032in}}{\pgfqpoint{1.435491in}{2.322132in}}{\pgfqpoint{1.441315in}{2.316308in}}%
\pgfpathcurveto{\pgfqpoint{1.447139in}{2.310484in}}{\pgfqpoint{1.455039in}{2.307211in}}{\pgfqpoint{1.463275in}{2.307211in}}%
\pgfpathclose%
\pgfusepath{stroke,fill}%
\end{pgfscope}%
\begin{pgfscope}%
\pgfpathrectangle{\pgfqpoint{0.100000in}{0.212622in}}{\pgfqpoint{3.696000in}{3.696000in}}%
\pgfusepath{clip}%
\pgfsetbuttcap%
\pgfsetroundjoin%
\definecolor{currentfill}{rgb}{0.121569,0.466667,0.705882}%
\pgfsetfillcolor{currentfill}%
\pgfsetfillopacity{0.372418}%
\pgfsetlinewidth{1.003750pt}%
\definecolor{currentstroke}{rgb}{0.121569,0.466667,0.705882}%
\pgfsetstrokecolor{currentstroke}%
\pgfsetstrokeopacity{0.372418}%
\pgfsetdash{}{0pt}%
\pgfpathmoveto{\pgfqpoint{2.162653in}{2.466825in}}%
\pgfpathcurveto{\pgfqpoint{2.170890in}{2.466825in}}{\pgfqpoint{2.178790in}{2.470097in}}{\pgfqpoint{2.184614in}{2.475921in}}%
\pgfpathcurveto{\pgfqpoint{2.190438in}{2.481745in}}{\pgfqpoint{2.193710in}{2.489645in}}{\pgfqpoint{2.193710in}{2.497881in}}%
\pgfpathcurveto{\pgfqpoint{2.193710in}{2.506118in}}{\pgfqpoint{2.190438in}{2.514018in}}{\pgfqpoint{2.184614in}{2.519842in}}%
\pgfpathcurveto{\pgfqpoint{2.178790in}{2.525665in}}{\pgfqpoint{2.170890in}{2.528938in}}{\pgfqpoint{2.162653in}{2.528938in}}%
\pgfpathcurveto{\pgfqpoint{2.154417in}{2.528938in}}{\pgfqpoint{2.146517in}{2.525665in}}{\pgfqpoint{2.140693in}{2.519842in}}%
\pgfpathcurveto{\pgfqpoint{2.134869in}{2.514018in}}{\pgfqpoint{2.131597in}{2.506118in}}{\pgfqpoint{2.131597in}{2.497881in}}%
\pgfpathcurveto{\pgfqpoint{2.131597in}{2.489645in}}{\pgfqpoint{2.134869in}{2.481745in}}{\pgfqpoint{2.140693in}{2.475921in}}%
\pgfpathcurveto{\pgfqpoint{2.146517in}{2.470097in}}{\pgfqpoint{2.154417in}{2.466825in}}{\pgfqpoint{2.162653in}{2.466825in}}%
\pgfpathclose%
\pgfusepath{stroke,fill}%
\end{pgfscope}%
\begin{pgfscope}%
\pgfpathrectangle{\pgfqpoint{0.100000in}{0.212622in}}{\pgfqpoint{3.696000in}{3.696000in}}%
\pgfusepath{clip}%
\pgfsetbuttcap%
\pgfsetroundjoin%
\definecolor{currentfill}{rgb}{0.121569,0.466667,0.705882}%
\pgfsetfillcolor{currentfill}%
\pgfsetfillopacity{0.373162}%
\pgfsetlinewidth{1.003750pt}%
\definecolor{currentstroke}{rgb}{0.121569,0.466667,0.705882}%
\pgfsetstrokecolor{currentstroke}%
\pgfsetstrokeopacity{0.373162}%
\pgfsetdash{}{0pt}%
\pgfpathmoveto{\pgfqpoint{2.167840in}{2.465306in}}%
\pgfpathcurveto{\pgfqpoint{2.176077in}{2.465306in}}{\pgfqpoint{2.183977in}{2.468578in}}{\pgfqpoint{2.189801in}{2.474402in}}%
\pgfpathcurveto{\pgfqpoint{2.195625in}{2.480226in}}{\pgfqpoint{2.198897in}{2.488126in}}{\pgfqpoint{2.198897in}{2.496363in}}%
\pgfpathcurveto{\pgfqpoint{2.198897in}{2.504599in}}{\pgfqpoint{2.195625in}{2.512499in}}{\pgfqpoint{2.189801in}{2.518323in}}%
\pgfpathcurveto{\pgfqpoint{2.183977in}{2.524147in}}{\pgfqpoint{2.176077in}{2.527419in}}{\pgfqpoint{2.167840in}{2.527419in}}%
\pgfpathcurveto{\pgfqpoint{2.159604in}{2.527419in}}{\pgfqpoint{2.151704in}{2.524147in}}{\pgfqpoint{2.145880in}{2.518323in}}%
\pgfpathcurveto{\pgfqpoint{2.140056in}{2.512499in}}{\pgfqpoint{2.136784in}{2.504599in}}{\pgfqpoint{2.136784in}{2.496363in}}%
\pgfpathcurveto{\pgfqpoint{2.136784in}{2.488126in}}{\pgfqpoint{2.140056in}{2.480226in}}{\pgfqpoint{2.145880in}{2.474402in}}%
\pgfpathcurveto{\pgfqpoint{2.151704in}{2.468578in}}{\pgfqpoint{2.159604in}{2.465306in}}{\pgfqpoint{2.167840in}{2.465306in}}%
\pgfpathclose%
\pgfusepath{stroke,fill}%
\end{pgfscope}%
\begin{pgfscope}%
\pgfpathrectangle{\pgfqpoint{0.100000in}{0.212622in}}{\pgfqpoint{3.696000in}{3.696000in}}%
\pgfusepath{clip}%
\pgfsetbuttcap%
\pgfsetroundjoin%
\definecolor{currentfill}{rgb}{0.121569,0.466667,0.705882}%
\pgfsetfillcolor{currentfill}%
\pgfsetfillopacity{0.374034}%
\pgfsetlinewidth{1.003750pt}%
\definecolor{currentstroke}{rgb}{0.121569,0.466667,0.705882}%
\pgfsetstrokecolor{currentstroke}%
\pgfsetstrokeopacity{0.374034}%
\pgfsetdash{}{0pt}%
\pgfpathmoveto{\pgfqpoint{2.173877in}{2.463614in}}%
\pgfpathcurveto{\pgfqpoint{2.182113in}{2.463614in}}{\pgfqpoint{2.190013in}{2.466887in}}{\pgfqpoint{2.195837in}{2.472711in}}%
\pgfpathcurveto{\pgfqpoint{2.201661in}{2.478535in}}{\pgfqpoint{2.204933in}{2.486435in}}{\pgfqpoint{2.204933in}{2.494671in}}%
\pgfpathcurveto{\pgfqpoint{2.204933in}{2.502907in}}{\pgfqpoint{2.201661in}{2.510807in}}{\pgfqpoint{2.195837in}{2.516631in}}%
\pgfpathcurveto{\pgfqpoint{2.190013in}{2.522455in}}{\pgfqpoint{2.182113in}{2.525727in}}{\pgfqpoint{2.173877in}{2.525727in}}%
\pgfpathcurveto{\pgfqpoint{2.165640in}{2.525727in}}{\pgfqpoint{2.157740in}{2.522455in}}{\pgfqpoint{2.151916in}{2.516631in}}%
\pgfpathcurveto{\pgfqpoint{2.146092in}{2.510807in}}{\pgfqpoint{2.142820in}{2.502907in}}{\pgfqpoint{2.142820in}{2.494671in}}%
\pgfpathcurveto{\pgfqpoint{2.142820in}{2.486435in}}{\pgfqpoint{2.146092in}{2.478535in}}{\pgfqpoint{2.151916in}{2.472711in}}%
\pgfpathcurveto{\pgfqpoint{2.157740in}{2.466887in}}{\pgfqpoint{2.165640in}{2.463614in}}{\pgfqpoint{2.173877in}{2.463614in}}%
\pgfpathclose%
\pgfusepath{stroke,fill}%
\end{pgfscope}%
\begin{pgfscope}%
\pgfpathrectangle{\pgfqpoint{0.100000in}{0.212622in}}{\pgfqpoint{3.696000in}{3.696000in}}%
\pgfusepath{clip}%
\pgfsetbuttcap%
\pgfsetroundjoin%
\definecolor{currentfill}{rgb}{0.121569,0.466667,0.705882}%
\pgfsetfillcolor{currentfill}%
\pgfsetfillopacity{0.374730}%
\pgfsetlinewidth{1.003750pt}%
\definecolor{currentstroke}{rgb}{0.121569,0.466667,0.705882}%
\pgfsetstrokecolor{currentstroke}%
\pgfsetstrokeopacity{0.374730}%
\pgfsetdash{}{0pt}%
\pgfpathmoveto{\pgfqpoint{1.456108in}{2.296610in}}%
\pgfpathcurveto{\pgfqpoint{1.464344in}{2.296610in}}{\pgfqpoint{1.472244in}{2.299882in}}{\pgfqpoint{1.478068in}{2.305706in}}%
\pgfpathcurveto{\pgfqpoint{1.483892in}{2.311530in}}{\pgfqpoint{1.487164in}{2.319430in}}{\pgfqpoint{1.487164in}{2.327666in}}%
\pgfpathcurveto{\pgfqpoint{1.487164in}{2.335902in}}{\pgfqpoint{1.483892in}{2.343802in}}{\pgfqpoint{1.478068in}{2.349626in}}%
\pgfpathcurveto{\pgfqpoint{1.472244in}{2.355450in}}{\pgfqpoint{1.464344in}{2.358723in}}{\pgfqpoint{1.456108in}{2.358723in}}%
\pgfpathcurveto{\pgfqpoint{1.447872in}{2.358723in}}{\pgfqpoint{1.439972in}{2.355450in}}{\pgfqpoint{1.434148in}{2.349626in}}%
\pgfpathcurveto{\pgfqpoint{1.428324in}{2.343802in}}{\pgfqpoint{1.425051in}{2.335902in}}{\pgfqpoint{1.425051in}{2.327666in}}%
\pgfpathcurveto{\pgfqpoint{1.425051in}{2.319430in}}{\pgfqpoint{1.428324in}{2.311530in}}{\pgfqpoint{1.434148in}{2.305706in}}%
\pgfpathcurveto{\pgfqpoint{1.439972in}{2.299882in}}{\pgfqpoint{1.447872in}{2.296610in}}{\pgfqpoint{1.456108in}{2.296610in}}%
\pgfpathclose%
\pgfusepath{stroke,fill}%
\end{pgfscope}%
\begin{pgfscope}%
\pgfpathrectangle{\pgfqpoint{0.100000in}{0.212622in}}{\pgfqpoint{3.696000in}{3.696000in}}%
\pgfusepath{clip}%
\pgfsetbuttcap%
\pgfsetroundjoin%
\definecolor{currentfill}{rgb}{0.121569,0.466667,0.705882}%
\pgfsetfillcolor{currentfill}%
\pgfsetfillopacity{0.375028}%
\pgfsetlinewidth{1.003750pt}%
\definecolor{currentstroke}{rgb}{0.121569,0.466667,0.705882}%
\pgfsetstrokecolor{currentstroke}%
\pgfsetstrokeopacity{0.375028}%
\pgfsetdash{}{0pt}%
\pgfpathmoveto{\pgfqpoint{2.180867in}{2.461540in}}%
\pgfpathcurveto{\pgfqpoint{2.189103in}{2.461540in}}{\pgfqpoint{2.197003in}{2.464813in}}{\pgfqpoint{2.202827in}{2.470637in}}%
\pgfpathcurveto{\pgfqpoint{2.208651in}{2.476461in}}{\pgfqpoint{2.211923in}{2.484361in}}{\pgfqpoint{2.211923in}{2.492597in}}%
\pgfpathcurveto{\pgfqpoint{2.211923in}{2.500833in}}{\pgfqpoint{2.208651in}{2.508733in}}{\pgfqpoint{2.202827in}{2.514557in}}%
\pgfpathcurveto{\pgfqpoint{2.197003in}{2.520381in}}{\pgfqpoint{2.189103in}{2.523653in}}{\pgfqpoint{2.180867in}{2.523653in}}%
\pgfpathcurveto{\pgfqpoint{2.172630in}{2.523653in}}{\pgfqpoint{2.164730in}{2.520381in}}{\pgfqpoint{2.158906in}{2.514557in}}%
\pgfpathcurveto{\pgfqpoint{2.153082in}{2.508733in}}{\pgfqpoint{2.149810in}{2.500833in}}{\pgfqpoint{2.149810in}{2.492597in}}%
\pgfpathcurveto{\pgfqpoint{2.149810in}{2.484361in}}{\pgfqpoint{2.153082in}{2.476461in}}{\pgfqpoint{2.158906in}{2.470637in}}%
\pgfpathcurveto{\pgfqpoint{2.164730in}{2.464813in}}{\pgfqpoint{2.172630in}{2.461540in}}{\pgfqpoint{2.180867in}{2.461540in}}%
\pgfpathclose%
\pgfusepath{stroke,fill}%
\end{pgfscope}%
\begin{pgfscope}%
\pgfpathrectangle{\pgfqpoint{0.100000in}{0.212622in}}{\pgfqpoint{3.696000in}{3.696000in}}%
\pgfusepath{clip}%
\pgfsetbuttcap%
\pgfsetroundjoin%
\definecolor{currentfill}{rgb}{0.121569,0.466667,0.705882}%
\pgfsetfillcolor{currentfill}%
\pgfsetfillopacity{0.375635}%
\pgfsetlinewidth{1.003750pt}%
\definecolor{currentstroke}{rgb}{0.121569,0.466667,0.705882}%
\pgfsetstrokecolor{currentstroke}%
\pgfsetstrokeopacity{0.375635}%
\pgfsetdash{}{0pt}%
\pgfpathmoveto{\pgfqpoint{2.184822in}{2.461160in}}%
\pgfpathcurveto{\pgfqpoint{2.193058in}{2.461160in}}{\pgfqpoint{2.200958in}{2.464433in}}{\pgfqpoint{2.206782in}{2.470257in}}%
\pgfpathcurveto{\pgfqpoint{2.212606in}{2.476081in}}{\pgfqpoint{2.215878in}{2.483981in}}{\pgfqpoint{2.215878in}{2.492217in}}%
\pgfpathcurveto{\pgfqpoint{2.215878in}{2.500453in}}{\pgfqpoint{2.212606in}{2.508353in}}{\pgfqpoint{2.206782in}{2.514177in}}%
\pgfpathcurveto{\pgfqpoint{2.200958in}{2.520001in}}{\pgfqpoint{2.193058in}{2.523273in}}{\pgfqpoint{2.184822in}{2.523273in}}%
\pgfpathcurveto{\pgfqpoint{2.176585in}{2.523273in}}{\pgfqpoint{2.168685in}{2.520001in}}{\pgfqpoint{2.162861in}{2.514177in}}%
\pgfpathcurveto{\pgfqpoint{2.157037in}{2.508353in}}{\pgfqpoint{2.153765in}{2.500453in}}{\pgfqpoint{2.153765in}{2.492217in}}%
\pgfpathcurveto{\pgfqpoint{2.153765in}{2.483981in}}{\pgfqpoint{2.157037in}{2.476081in}}{\pgfqpoint{2.162861in}{2.470257in}}%
\pgfpathcurveto{\pgfqpoint{2.168685in}{2.464433in}}{\pgfqpoint{2.176585in}{2.461160in}}{\pgfqpoint{2.184822in}{2.461160in}}%
\pgfpathclose%
\pgfusepath{stroke,fill}%
\end{pgfscope}%
\begin{pgfscope}%
\pgfpathrectangle{\pgfqpoint{0.100000in}{0.212622in}}{\pgfqpoint{3.696000in}{3.696000in}}%
\pgfusepath{clip}%
\pgfsetbuttcap%
\pgfsetroundjoin%
\definecolor{currentfill}{rgb}{0.121569,0.466667,0.705882}%
\pgfsetfillcolor{currentfill}%
\pgfsetfillopacity{0.375938}%
\pgfsetlinewidth{1.003750pt}%
\definecolor{currentstroke}{rgb}{0.121569,0.466667,0.705882}%
\pgfsetstrokecolor{currentstroke}%
\pgfsetstrokeopacity{0.375938}%
\pgfsetdash{}{0pt}%
\pgfpathmoveto{\pgfqpoint{2.186947in}{2.460580in}}%
\pgfpathcurveto{\pgfqpoint{2.195183in}{2.460580in}}{\pgfqpoint{2.203083in}{2.463852in}}{\pgfqpoint{2.208907in}{2.469676in}}%
\pgfpathcurveto{\pgfqpoint{2.214731in}{2.475500in}}{\pgfqpoint{2.218004in}{2.483400in}}{\pgfqpoint{2.218004in}{2.491636in}}%
\pgfpathcurveto{\pgfqpoint{2.218004in}{2.499872in}}{\pgfqpoint{2.214731in}{2.507772in}}{\pgfqpoint{2.208907in}{2.513596in}}%
\pgfpathcurveto{\pgfqpoint{2.203083in}{2.519420in}}{\pgfqpoint{2.195183in}{2.522693in}}{\pgfqpoint{2.186947in}{2.522693in}}%
\pgfpathcurveto{\pgfqpoint{2.178711in}{2.522693in}}{\pgfqpoint{2.170811in}{2.519420in}}{\pgfqpoint{2.164987in}{2.513596in}}%
\pgfpathcurveto{\pgfqpoint{2.159163in}{2.507772in}}{\pgfqpoint{2.155891in}{2.499872in}}{\pgfqpoint{2.155891in}{2.491636in}}%
\pgfpathcurveto{\pgfqpoint{2.155891in}{2.483400in}}{\pgfqpoint{2.159163in}{2.475500in}}{\pgfqpoint{2.164987in}{2.469676in}}%
\pgfpathcurveto{\pgfqpoint{2.170811in}{2.463852in}}{\pgfqpoint{2.178711in}{2.460580in}}{\pgfqpoint{2.186947in}{2.460580in}}%
\pgfpathclose%
\pgfusepath{stroke,fill}%
\end{pgfscope}%
\begin{pgfscope}%
\pgfpathrectangle{\pgfqpoint{0.100000in}{0.212622in}}{\pgfqpoint{3.696000in}{3.696000in}}%
\pgfusepath{clip}%
\pgfsetbuttcap%
\pgfsetroundjoin%
\definecolor{currentfill}{rgb}{0.121569,0.466667,0.705882}%
\pgfsetfillcolor{currentfill}%
\pgfsetfillopacity{0.376362}%
\pgfsetlinewidth{1.003750pt}%
\definecolor{currentstroke}{rgb}{0.121569,0.466667,0.705882}%
\pgfsetstrokecolor{currentstroke}%
\pgfsetstrokeopacity{0.376362}%
\pgfsetdash{}{0pt}%
\pgfpathmoveto{\pgfqpoint{2.189774in}{2.459953in}}%
\pgfpathcurveto{\pgfqpoint{2.198011in}{2.459953in}}{\pgfqpoint{2.205911in}{2.463226in}}{\pgfqpoint{2.211735in}{2.469050in}}%
\pgfpathcurveto{\pgfqpoint{2.217558in}{2.474874in}}{\pgfqpoint{2.220831in}{2.482774in}}{\pgfqpoint{2.220831in}{2.491010in}}%
\pgfpathcurveto{\pgfqpoint{2.220831in}{2.499246in}}{\pgfqpoint{2.217558in}{2.507146in}}{\pgfqpoint{2.211735in}{2.512970in}}%
\pgfpathcurveto{\pgfqpoint{2.205911in}{2.518794in}}{\pgfqpoint{2.198011in}{2.522066in}}{\pgfqpoint{2.189774in}{2.522066in}}%
\pgfpathcurveto{\pgfqpoint{2.181538in}{2.522066in}}{\pgfqpoint{2.173638in}{2.518794in}}{\pgfqpoint{2.167814in}{2.512970in}}%
\pgfpathcurveto{\pgfqpoint{2.161990in}{2.507146in}}{\pgfqpoint{2.158718in}{2.499246in}}{\pgfqpoint{2.158718in}{2.491010in}}%
\pgfpathcurveto{\pgfqpoint{2.158718in}{2.482774in}}{\pgfqpoint{2.161990in}{2.474874in}}{\pgfqpoint{2.167814in}{2.469050in}}%
\pgfpathcurveto{\pgfqpoint{2.173638in}{2.463226in}}{\pgfqpoint{2.181538in}{2.459953in}}{\pgfqpoint{2.189774in}{2.459953in}}%
\pgfpathclose%
\pgfusepath{stroke,fill}%
\end{pgfscope}%
\begin{pgfscope}%
\pgfpathrectangle{\pgfqpoint{0.100000in}{0.212622in}}{\pgfqpoint{3.696000in}{3.696000in}}%
\pgfusepath{clip}%
\pgfsetbuttcap%
\pgfsetroundjoin%
\definecolor{currentfill}{rgb}{0.121569,0.466667,0.705882}%
\pgfsetfillcolor{currentfill}%
\pgfsetfillopacity{0.376594}%
\pgfsetlinewidth{1.003750pt}%
\definecolor{currentstroke}{rgb}{0.121569,0.466667,0.705882}%
\pgfsetstrokecolor{currentstroke}%
\pgfsetstrokeopacity{0.376594}%
\pgfsetdash{}{0pt}%
\pgfpathmoveto{\pgfqpoint{2.191321in}{2.459579in}}%
\pgfpathcurveto{\pgfqpoint{2.199557in}{2.459579in}}{\pgfqpoint{2.207457in}{2.462851in}}{\pgfqpoint{2.213281in}{2.468675in}}%
\pgfpathcurveto{\pgfqpoint{2.219105in}{2.474499in}}{\pgfqpoint{2.222378in}{2.482399in}}{\pgfqpoint{2.222378in}{2.490636in}}%
\pgfpathcurveto{\pgfqpoint{2.222378in}{2.498872in}}{\pgfqpoint{2.219105in}{2.506772in}}{\pgfqpoint{2.213281in}{2.512596in}}%
\pgfpathcurveto{\pgfqpoint{2.207457in}{2.518420in}}{\pgfqpoint{2.199557in}{2.521692in}}{\pgfqpoint{2.191321in}{2.521692in}}%
\pgfpathcurveto{\pgfqpoint{2.183085in}{2.521692in}}{\pgfqpoint{2.175185in}{2.518420in}}{\pgfqpoint{2.169361in}{2.512596in}}%
\pgfpathcurveto{\pgfqpoint{2.163537in}{2.506772in}}{\pgfqpoint{2.160265in}{2.498872in}}{\pgfqpoint{2.160265in}{2.490636in}}%
\pgfpathcurveto{\pgfqpoint{2.160265in}{2.482399in}}{\pgfqpoint{2.163537in}{2.474499in}}{\pgfqpoint{2.169361in}{2.468675in}}%
\pgfpathcurveto{\pgfqpoint{2.175185in}{2.462851in}}{\pgfqpoint{2.183085in}{2.459579in}}{\pgfqpoint{2.191321in}{2.459579in}}%
\pgfpathclose%
\pgfusepath{stroke,fill}%
\end{pgfscope}%
\begin{pgfscope}%
\pgfpathrectangle{\pgfqpoint{0.100000in}{0.212622in}}{\pgfqpoint{3.696000in}{3.696000in}}%
\pgfusepath{clip}%
\pgfsetbuttcap%
\pgfsetroundjoin%
\definecolor{currentfill}{rgb}{0.121569,0.466667,0.705882}%
\pgfsetfillcolor{currentfill}%
\pgfsetfillopacity{0.377160}%
\pgfsetlinewidth{1.003750pt}%
\definecolor{currentstroke}{rgb}{0.121569,0.466667,0.705882}%
\pgfsetstrokecolor{currentstroke}%
\pgfsetstrokeopacity{0.377160}%
\pgfsetdash{}{0pt}%
\pgfpathmoveto{\pgfqpoint{2.195054in}{2.458599in}}%
\pgfpathcurveto{\pgfqpoint{2.203290in}{2.458599in}}{\pgfqpoint{2.211190in}{2.461871in}}{\pgfqpoint{2.217014in}{2.467695in}}%
\pgfpathcurveto{\pgfqpoint{2.222838in}{2.473519in}}{\pgfqpoint{2.226110in}{2.481419in}}{\pgfqpoint{2.226110in}{2.489655in}}%
\pgfpathcurveto{\pgfqpoint{2.226110in}{2.497891in}}{\pgfqpoint{2.222838in}{2.505791in}}{\pgfqpoint{2.217014in}{2.511615in}}%
\pgfpathcurveto{\pgfqpoint{2.211190in}{2.517439in}}{\pgfqpoint{2.203290in}{2.520712in}}{\pgfqpoint{2.195054in}{2.520712in}}%
\pgfpathcurveto{\pgfqpoint{2.186817in}{2.520712in}}{\pgfqpoint{2.178917in}{2.517439in}}{\pgfqpoint{2.173093in}{2.511615in}}%
\pgfpathcurveto{\pgfqpoint{2.167269in}{2.505791in}}{\pgfqpoint{2.163997in}{2.497891in}}{\pgfqpoint{2.163997in}{2.489655in}}%
\pgfpathcurveto{\pgfqpoint{2.163997in}{2.481419in}}{\pgfqpoint{2.167269in}{2.473519in}}{\pgfqpoint{2.173093in}{2.467695in}}%
\pgfpathcurveto{\pgfqpoint{2.178917in}{2.461871in}}{\pgfqpoint{2.186817in}{2.458599in}}{\pgfqpoint{2.195054in}{2.458599in}}%
\pgfpathclose%
\pgfusepath{stroke,fill}%
\end{pgfscope}%
\begin{pgfscope}%
\pgfpathrectangle{\pgfqpoint{0.100000in}{0.212622in}}{\pgfqpoint{3.696000in}{3.696000in}}%
\pgfusepath{clip}%
\pgfsetbuttcap%
\pgfsetroundjoin%
\definecolor{currentfill}{rgb}{0.121569,0.466667,0.705882}%
\pgfsetfillcolor{currentfill}%
\pgfsetfillopacity{0.377429}%
\pgfsetlinewidth{1.003750pt}%
\definecolor{currentstroke}{rgb}{0.121569,0.466667,0.705882}%
\pgfsetstrokecolor{currentstroke}%
\pgfsetstrokeopacity{0.377429}%
\pgfsetdash{}{0pt}%
\pgfpathmoveto{\pgfqpoint{1.449070in}{2.286560in}}%
\pgfpathcurveto{\pgfqpoint{1.457306in}{2.286560in}}{\pgfqpoint{1.465206in}{2.289832in}}{\pgfqpoint{1.471030in}{2.295656in}}%
\pgfpathcurveto{\pgfqpoint{1.476854in}{2.301480in}}{\pgfqpoint{1.480126in}{2.309380in}}{\pgfqpoint{1.480126in}{2.317616in}}%
\pgfpathcurveto{\pgfqpoint{1.480126in}{2.325852in}}{\pgfqpoint{1.476854in}{2.333752in}}{\pgfqpoint{1.471030in}{2.339576in}}%
\pgfpathcurveto{\pgfqpoint{1.465206in}{2.345400in}}{\pgfqpoint{1.457306in}{2.348673in}}{\pgfqpoint{1.449070in}{2.348673in}}%
\pgfpathcurveto{\pgfqpoint{1.440833in}{2.348673in}}{\pgfqpoint{1.432933in}{2.345400in}}{\pgfqpoint{1.427109in}{2.339576in}}%
\pgfpathcurveto{\pgfqpoint{1.421285in}{2.333752in}}{\pgfqpoint{1.418013in}{2.325852in}}{\pgfqpoint{1.418013in}{2.317616in}}%
\pgfpathcurveto{\pgfqpoint{1.418013in}{2.309380in}}{\pgfqpoint{1.421285in}{2.301480in}}{\pgfqpoint{1.427109in}{2.295656in}}%
\pgfpathcurveto{\pgfqpoint{1.432933in}{2.289832in}}{\pgfqpoint{1.440833in}{2.286560in}}{\pgfqpoint{1.449070in}{2.286560in}}%
\pgfpathclose%
\pgfusepath{stroke,fill}%
\end{pgfscope}%
\begin{pgfscope}%
\pgfpathrectangle{\pgfqpoint{0.100000in}{0.212622in}}{\pgfqpoint{3.696000in}{3.696000in}}%
\pgfusepath{clip}%
\pgfsetbuttcap%
\pgfsetroundjoin%
\definecolor{currentfill}{rgb}{0.121569,0.466667,0.705882}%
\pgfsetfillcolor{currentfill}%
\pgfsetfillopacity{0.377488}%
\pgfsetlinewidth{1.003750pt}%
\definecolor{currentstroke}{rgb}{0.121569,0.466667,0.705882}%
\pgfsetstrokecolor{currentstroke}%
\pgfsetstrokeopacity{0.377488}%
\pgfsetdash{}{0pt}%
\pgfpathmoveto{\pgfqpoint{2.197133in}{2.458248in}}%
\pgfpathcurveto{\pgfqpoint{2.205369in}{2.458248in}}{\pgfqpoint{2.213269in}{2.461520in}}{\pgfqpoint{2.219093in}{2.467344in}}%
\pgfpathcurveto{\pgfqpoint{2.224917in}{2.473168in}}{\pgfqpoint{2.228189in}{2.481068in}}{\pgfqpoint{2.228189in}{2.489305in}}%
\pgfpathcurveto{\pgfqpoint{2.228189in}{2.497541in}}{\pgfqpoint{2.224917in}{2.505441in}}{\pgfqpoint{2.219093in}{2.511265in}}%
\pgfpathcurveto{\pgfqpoint{2.213269in}{2.517089in}}{\pgfqpoint{2.205369in}{2.520361in}}{\pgfqpoint{2.197133in}{2.520361in}}%
\pgfpathcurveto{\pgfqpoint{2.188896in}{2.520361in}}{\pgfqpoint{2.180996in}{2.517089in}}{\pgfqpoint{2.175172in}{2.511265in}}%
\pgfpathcurveto{\pgfqpoint{2.169348in}{2.505441in}}{\pgfqpoint{2.166076in}{2.497541in}}{\pgfqpoint{2.166076in}{2.489305in}}%
\pgfpathcurveto{\pgfqpoint{2.166076in}{2.481068in}}{\pgfqpoint{2.169348in}{2.473168in}}{\pgfqpoint{2.175172in}{2.467344in}}%
\pgfpathcurveto{\pgfqpoint{2.180996in}{2.461520in}}{\pgfqpoint{2.188896in}{2.458248in}}{\pgfqpoint{2.197133in}{2.458248in}}%
\pgfpathclose%
\pgfusepath{stroke,fill}%
\end{pgfscope}%
\begin{pgfscope}%
\pgfpathrectangle{\pgfqpoint{0.100000in}{0.212622in}}{\pgfqpoint{3.696000in}{3.696000in}}%
\pgfusepath{clip}%
\pgfsetbuttcap%
\pgfsetroundjoin%
\definecolor{currentfill}{rgb}{0.121569,0.466667,0.705882}%
\pgfsetfillcolor{currentfill}%
\pgfsetfillopacity{0.377675}%
\pgfsetlinewidth{1.003750pt}%
\definecolor{currentstroke}{rgb}{0.121569,0.466667,0.705882}%
\pgfsetstrokecolor{currentstroke}%
\pgfsetstrokeopacity{0.377675}%
\pgfsetdash{}{0pt}%
\pgfpathmoveto{\pgfqpoint{2.198271in}{2.458084in}}%
\pgfpathcurveto{\pgfqpoint{2.206507in}{2.458084in}}{\pgfqpoint{2.214407in}{2.461357in}}{\pgfqpoint{2.220231in}{2.467181in}}%
\pgfpathcurveto{\pgfqpoint{2.226055in}{2.473005in}}{\pgfqpoint{2.229327in}{2.480905in}}{\pgfqpoint{2.229327in}{2.489141in}}%
\pgfpathcurveto{\pgfqpoint{2.229327in}{2.497377in}}{\pgfqpoint{2.226055in}{2.505277in}}{\pgfqpoint{2.220231in}{2.511101in}}%
\pgfpathcurveto{\pgfqpoint{2.214407in}{2.516925in}}{\pgfqpoint{2.206507in}{2.520197in}}{\pgfqpoint{2.198271in}{2.520197in}}%
\pgfpathcurveto{\pgfqpoint{2.190035in}{2.520197in}}{\pgfqpoint{2.182135in}{2.516925in}}{\pgfqpoint{2.176311in}{2.511101in}}%
\pgfpathcurveto{\pgfqpoint{2.170487in}{2.505277in}}{\pgfqpoint{2.167214in}{2.497377in}}{\pgfqpoint{2.167214in}{2.489141in}}%
\pgfpathcurveto{\pgfqpoint{2.167214in}{2.480905in}}{\pgfqpoint{2.170487in}{2.473005in}}{\pgfqpoint{2.176311in}{2.467181in}}%
\pgfpathcurveto{\pgfqpoint{2.182135in}{2.461357in}}{\pgfqpoint{2.190035in}{2.458084in}}{\pgfqpoint{2.198271in}{2.458084in}}%
\pgfpathclose%
\pgfusepath{stroke,fill}%
\end{pgfscope}%
\begin{pgfscope}%
\pgfpathrectangle{\pgfqpoint{0.100000in}{0.212622in}}{\pgfqpoint{3.696000in}{3.696000in}}%
\pgfusepath{clip}%
\pgfsetbuttcap%
\pgfsetroundjoin%
\definecolor{currentfill}{rgb}{0.121569,0.466667,0.705882}%
\pgfsetfillcolor{currentfill}%
\pgfsetfillopacity{0.378168}%
\pgfsetlinewidth{1.003750pt}%
\definecolor{currentstroke}{rgb}{0.121569,0.466667,0.705882}%
\pgfsetstrokecolor{currentstroke}%
\pgfsetstrokeopacity{0.378168}%
\pgfsetdash{}{0pt}%
\pgfpathmoveto{\pgfqpoint{2.201140in}{2.457699in}}%
\pgfpathcurveto{\pgfqpoint{2.209376in}{2.457699in}}{\pgfqpoint{2.217276in}{2.460971in}}{\pgfqpoint{2.223100in}{2.466795in}}%
\pgfpathcurveto{\pgfqpoint{2.228924in}{2.472619in}}{\pgfqpoint{2.232197in}{2.480519in}}{\pgfqpoint{2.232197in}{2.488755in}}%
\pgfpathcurveto{\pgfqpoint{2.232197in}{2.496992in}}{\pgfqpoint{2.228924in}{2.504892in}}{\pgfqpoint{2.223100in}{2.510716in}}%
\pgfpathcurveto{\pgfqpoint{2.217276in}{2.516540in}}{\pgfqpoint{2.209376in}{2.519812in}}{\pgfqpoint{2.201140in}{2.519812in}}%
\pgfpathcurveto{\pgfqpoint{2.192904in}{2.519812in}}{\pgfqpoint{2.185004in}{2.516540in}}{\pgfqpoint{2.179180in}{2.510716in}}%
\pgfpathcurveto{\pgfqpoint{2.173356in}{2.504892in}}{\pgfqpoint{2.170084in}{2.496992in}}{\pgfqpoint{2.170084in}{2.488755in}}%
\pgfpathcurveto{\pgfqpoint{2.170084in}{2.480519in}}{\pgfqpoint{2.173356in}{2.472619in}}{\pgfqpoint{2.179180in}{2.466795in}}%
\pgfpathcurveto{\pgfqpoint{2.185004in}{2.460971in}}{\pgfqpoint{2.192904in}{2.457699in}}{\pgfqpoint{2.201140in}{2.457699in}}%
\pgfpathclose%
\pgfusepath{stroke,fill}%
\end{pgfscope}%
\begin{pgfscope}%
\pgfpathrectangle{\pgfqpoint{0.100000in}{0.212622in}}{\pgfqpoint{3.696000in}{3.696000in}}%
\pgfusepath{clip}%
\pgfsetbuttcap%
\pgfsetroundjoin%
\definecolor{currentfill}{rgb}{0.121569,0.466667,0.705882}%
\pgfsetfillcolor{currentfill}%
\pgfsetfillopacity{0.378944}%
\pgfsetlinewidth{1.003750pt}%
\definecolor{currentstroke}{rgb}{0.121569,0.466667,0.705882}%
\pgfsetstrokecolor{currentstroke}%
\pgfsetstrokeopacity{0.378944}%
\pgfsetdash{}{0pt}%
\pgfpathmoveto{\pgfqpoint{2.205734in}{2.456988in}}%
\pgfpathcurveto{\pgfqpoint{2.213971in}{2.456988in}}{\pgfqpoint{2.221871in}{2.460261in}}{\pgfqpoint{2.227695in}{2.466084in}}%
\pgfpathcurveto{\pgfqpoint{2.233518in}{2.471908in}}{\pgfqpoint{2.236791in}{2.479808in}}{\pgfqpoint{2.236791in}{2.488045in}}%
\pgfpathcurveto{\pgfqpoint{2.236791in}{2.496281in}}{\pgfqpoint{2.233518in}{2.504181in}}{\pgfqpoint{2.227695in}{2.510005in}}%
\pgfpathcurveto{\pgfqpoint{2.221871in}{2.515829in}}{\pgfqpoint{2.213971in}{2.519101in}}{\pgfqpoint{2.205734in}{2.519101in}}%
\pgfpathcurveto{\pgfqpoint{2.197498in}{2.519101in}}{\pgfqpoint{2.189598in}{2.515829in}}{\pgfqpoint{2.183774in}{2.510005in}}%
\pgfpathcurveto{\pgfqpoint{2.177950in}{2.504181in}}{\pgfqpoint{2.174678in}{2.496281in}}{\pgfqpoint{2.174678in}{2.488045in}}%
\pgfpathcurveto{\pgfqpoint{2.174678in}{2.479808in}}{\pgfqpoint{2.177950in}{2.471908in}}{\pgfqpoint{2.183774in}{2.466084in}}%
\pgfpathcurveto{\pgfqpoint{2.189598in}{2.460261in}}{\pgfqpoint{2.197498in}{2.456988in}}{\pgfqpoint{2.205734in}{2.456988in}}%
\pgfpathclose%
\pgfusepath{stroke,fill}%
\end{pgfscope}%
\begin{pgfscope}%
\pgfpathrectangle{\pgfqpoint{0.100000in}{0.212622in}}{\pgfqpoint{3.696000in}{3.696000in}}%
\pgfusepath{clip}%
\pgfsetbuttcap%
\pgfsetroundjoin%
\definecolor{currentfill}{rgb}{0.121569,0.466667,0.705882}%
\pgfsetfillcolor{currentfill}%
\pgfsetfillopacity{0.379357}%
\pgfsetlinewidth{1.003750pt}%
\definecolor{currentstroke}{rgb}{0.121569,0.466667,0.705882}%
\pgfsetstrokecolor{currentstroke}%
\pgfsetstrokeopacity{0.379357}%
\pgfsetdash{}{0pt}%
\pgfpathmoveto{\pgfqpoint{2.208268in}{2.456531in}}%
\pgfpathcurveto{\pgfqpoint{2.216505in}{2.456531in}}{\pgfqpoint{2.224405in}{2.459803in}}{\pgfqpoint{2.230229in}{2.465627in}}%
\pgfpathcurveto{\pgfqpoint{2.236053in}{2.471451in}}{\pgfqpoint{2.239325in}{2.479351in}}{\pgfqpoint{2.239325in}{2.487587in}}%
\pgfpathcurveto{\pgfqpoint{2.239325in}{2.495824in}}{\pgfqpoint{2.236053in}{2.503724in}}{\pgfqpoint{2.230229in}{2.509548in}}%
\pgfpathcurveto{\pgfqpoint{2.224405in}{2.515372in}}{\pgfqpoint{2.216505in}{2.518644in}}{\pgfqpoint{2.208268in}{2.518644in}}%
\pgfpathcurveto{\pgfqpoint{2.200032in}{2.518644in}}{\pgfqpoint{2.192132in}{2.515372in}}{\pgfqpoint{2.186308in}{2.509548in}}%
\pgfpathcurveto{\pgfqpoint{2.180484in}{2.503724in}}{\pgfqpoint{2.177212in}{2.495824in}}{\pgfqpoint{2.177212in}{2.487587in}}%
\pgfpathcurveto{\pgfqpoint{2.177212in}{2.479351in}}{\pgfqpoint{2.180484in}{2.471451in}}{\pgfqpoint{2.186308in}{2.465627in}}%
\pgfpathcurveto{\pgfqpoint{2.192132in}{2.459803in}}{\pgfqpoint{2.200032in}{2.456531in}}{\pgfqpoint{2.208268in}{2.456531in}}%
\pgfpathclose%
\pgfusepath{stroke,fill}%
\end{pgfscope}%
\begin{pgfscope}%
\pgfpathrectangle{\pgfqpoint{0.100000in}{0.212622in}}{\pgfqpoint{3.696000in}{3.696000in}}%
\pgfusepath{clip}%
\pgfsetbuttcap%
\pgfsetroundjoin%
\definecolor{currentfill}{rgb}{0.121569,0.466667,0.705882}%
\pgfsetfillcolor{currentfill}%
\pgfsetfillopacity{0.379754}%
\pgfsetlinewidth{1.003750pt}%
\definecolor{currentstroke}{rgb}{0.121569,0.466667,0.705882}%
\pgfsetstrokecolor{currentstroke}%
\pgfsetstrokeopacity{0.379754}%
\pgfsetdash{}{0pt}%
\pgfpathmoveto{\pgfqpoint{1.443264in}{2.278542in}}%
\pgfpathcurveto{\pgfqpoint{1.451500in}{2.278542in}}{\pgfqpoint{1.459400in}{2.281814in}}{\pgfqpoint{1.465224in}{2.287638in}}%
\pgfpathcurveto{\pgfqpoint{1.471048in}{2.293462in}}{\pgfqpoint{1.474320in}{2.301362in}}{\pgfqpoint{1.474320in}{2.309598in}}%
\pgfpathcurveto{\pgfqpoint{1.474320in}{2.317835in}}{\pgfqpoint{1.471048in}{2.325735in}}{\pgfqpoint{1.465224in}{2.331559in}}%
\pgfpathcurveto{\pgfqpoint{1.459400in}{2.337383in}}{\pgfqpoint{1.451500in}{2.340655in}}{\pgfqpoint{1.443264in}{2.340655in}}%
\pgfpathcurveto{\pgfqpoint{1.435027in}{2.340655in}}{\pgfqpoint{1.427127in}{2.337383in}}{\pgfqpoint{1.421304in}{2.331559in}}%
\pgfpathcurveto{\pgfqpoint{1.415480in}{2.325735in}}{\pgfqpoint{1.412207in}{2.317835in}}{\pgfqpoint{1.412207in}{2.309598in}}%
\pgfpathcurveto{\pgfqpoint{1.412207in}{2.301362in}}{\pgfqpoint{1.415480in}{2.293462in}}{\pgfqpoint{1.421304in}{2.287638in}}%
\pgfpathcurveto{\pgfqpoint{1.427127in}{2.281814in}}{\pgfqpoint{1.435027in}{2.278542in}}{\pgfqpoint{1.443264in}{2.278542in}}%
\pgfpathclose%
\pgfusepath{stroke,fill}%
\end{pgfscope}%
\begin{pgfscope}%
\pgfpathrectangle{\pgfqpoint{0.100000in}{0.212622in}}{\pgfqpoint{3.696000in}{3.696000in}}%
\pgfusepath{clip}%
\pgfsetbuttcap%
\pgfsetroundjoin%
\definecolor{currentfill}{rgb}{0.121569,0.466667,0.705882}%
\pgfsetfillcolor{currentfill}%
\pgfsetfillopacity{0.380028}%
\pgfsetlinewidth{1.003750pt}%
\definecolor{currentstroke}{rgb}{0.121569,0.466667,0.705882}%
\pgfsetstrokecolor{currentstroke}%
\pgfsetstrokeopacity{0.380028}%
\pgfsetdash{}{0pt}%
\pgfpathmoveto{\pgfqpoint{2.212320in}{2.455088in}}%
\pgfpathcurveto{\pgfqpoint{2.220556in}{2.455088in}}{\pgfqpoint{2.228456in}{2.458360in}}{\pgfqpoint{2.234280in}{2.464184in}}%
\pgfpathcurveto{\pgfqpoint{2.240104in}{2.470008in}}{\pgfqpoint{2.243376in}{2.477908in}}{\pgfqpoint{2.243376in}{2.486145in}}%
\pgfpathcurveto{\pgfqpoint{2.243376in}{2.494381in}}{\pgfqpoint{2.240104in}{2.502281in}}{\pgfqpoint{2.234280in}{2.508105in}}%
\pgfpathcurveto{\pgfqpoint{2.228456in}{2.513929in}}{\pgfqpoint{2.220556in}{2.517201in}}{\pgfqpoint{2.212320in}{2.517201in}}%
\pgfpathcurveto{\pgfqpoint{2.204083in}{2.517201in}}{\pgfqpoint{2.196183in}{2.513929in}}{\pgfqpoint{2.190359in}{2.508105in}}%
\pgfpathcurveto{\pgfqpoint{2.184535in}{2.502281in}}{\pgfqpoint{2.181263in}{2.494381in}}{\pgfqpoint{2.181263in}{2.486145in}}%
\pgfpathcurveto{\pgfqpoint{2.181263in}{2.477908in}}{\pgfqpoint{2.184535in}{2.470008in}}{\pgfqpoint{2.190359in}{2.464184in}}%
\pgfpathcurveto{\pgfqpoint{2.196183in}{2.458360in}}{\pgfqpoint{2.204083in}{2.455088in}}{\pgfqpoint{2.212320in}{2.455088in}}%
\pgfpathclose%
\pgfusepath{stroke,fill}%
\end{pgfscope}%
\begin{pgfscope}%
\pgfpathrectangle{\pgfqpoint{0.100000in}{0.212622in}}{\pgfqpoint{3.696000in}{3.696000in}}%
\pgfusepath{clip}%
\pgfsetbuttcap%
\pgfsetroundjoin%
\definecolor{currentfill}{rgb}{0.121569,0.466667,0.705882}%
\pgfsetfillcolor{currentfill}%
\pgfsetfillopacity{0.380984}%
\pgfsetlinewidth{1.003750pt}%
\definecolor{currentstroke}{rgb}{0.121569,0.466667,0.705882}%
\pgfsetstrokecolor{currentstroke}%
\pgfsetstrokeopacity{0.380984}%
\pgfsetdash{}{0pt}%
\pgfpathmoveto{\pgfqpoint{2.218089in}{2.453292in}}%
\pgfpathcurveto{\pgfqpoint{2.226325in}{2.453292in}}{\pgfqpoint{2.234225in}{2.456565in}}{\pgfqpoint{2.240049in}{2.462389in}}%
\pgfpathcurveto{\pgfqpoint{2.245873in}{2.468213in}}{\pgfqpoint{2.249145in}{2.476113in}}{\pgfqpoint{2.249145in}{2.484349in}}%
\pgfpathcurveto{\pgfqpoint{2.249145in}{2.492585in}}{\pgfqpoint{2.245873in}{2.500485in}}{\pgfqpoint{2.240049in}{2.506309in}}%
\pgfpathcurveto{\pgfqpoint{2.234225in}{2.512133in}}{\pgfqpoint{2.226325in}{2.515405in}}{\pgfqpoint{2.218089in}{2.515405in}}%
\pgfpathcurveto{\pgfqpoint{2.209852in}{2.515405in}}{\pgfqpoint{2.201952in}{2.512133in}}{\pgfqpoint{2.196128in}{2.506309in}}%
\pgfpathcurveto{\pgfqpoint{2.190304in}{2.500485in}}{\pgfqpoint{2.187032in}{2.492585in}}{\pgfqpoint{2.187032in}{2.484349in}}%
\pgfpathcurveto{\pgfqpoint{2.187032in}{2.476113in}}{\pgfqpoint{2.190304in}{2.468213in}}{\pgfqpoint{2.196128in}{2.462389in}}%
\pgfpathcurveto{\pgfqpoint{2.201952in}{2.456565in}}{\pgfqpoint{2.209852in}{2.453292in}}{\pgfqpoint{2.218089in}{2.453292in}}%
\pgfpathclose%
\pgfusepath{stroke,fill}%
\end{pgfscope}%
\begin{pgfscope}%
\pgfpathrectangle{\pgfqpoint{0.100000in}{0.212622in}}{\pgfqpoint{3.696000in}{3.696000in}}%
\pgfusepath{clip}%
\pgfsetbuttcap%
\pgfsetroundjoin%
\definecolor{currentfill}{rgb}{0.121569,0.466667,0.705882}%
\pgfsetfillcolor{currentfill}%
\pgfsetfillopacity{0.381886}%
\pgfsetlinewidth{1.003750pt}%
\definecolor{currentstroke}{rgb}{0.121569,0.466667,0.705882}%
\pgfsetstrokecolor{currentstroke}%
\pgfsetstrokeopacity{0.381886}%
\pgfsetdash{}{0pt}%
\pgfpathmoveto{\pgfqpoint{1.437708in}{2.270900in}}%
\pgfpathcurveto{\pgfqpoint{1.445944in}{2.270900in}}{\pgfqpoint{1.453844in}{2.274173in}}{\pgfqpoint{1.459668in}{2.279997in}}%
\pgfpathcurveto{\pgfqpoint{1.465492in}{2.285820in}}{\pgfqpoint{1.468765in}{2.293720in}}{\pgfqpoint{1.468765in}{2.301957in}}%
\pgfpathcurveto{\pgfqpoint{1.468765in}{2.310193in}}{\pgfqpoint{1.465492in}{2.318093in}}{\pgfqpoint{1.459668in}{2.323917in}}%
\pgfpathcurveto{\pgfqpoint{1.453844in}{2.329741in}}{\pgfqpoint{1.445944in}{2.333013in}}{\pgfqpoint{1.437708in}{2.333013in}}%
\pgfpathcurveto{\pgfqpoint{1.429472in}{2.333013in}}{\pgfqpoint{1.421572in}{2.329741in}}{\pgfqpoint{1.415748in}{2.323917in}}%
\pgfpathcurveto{\pgfqpoint{1.409924in}{2.318093in}}{\pgfqpoint{1.406652in}{2.310193in}}{\pgfqpoint{1.406652in}{2.301957in}}%
\pgfpathcurveto{\pgfqpoint{1.406652in}{2.293720in}}{\pgfqpoint{1.409924in}{2.285820in}}{\pgfqpoint{1.415748in}{2.279997in}}%
\pgfpathcurveto{\pgfqpoint{1.421572in}{2.274173in}}{\pgfqpoint{1.429472in}{2.270900in}}{\pgfqpoint{1.437708in}{2.270900in}}%
\pgfpathclose%
\pgfusepath{stroke,fill}%
\end{pgfscope}%
\begin{pgfscope}%
\pgfpathrectangle{\pgfqpoint{0.100000in}{0.212622in}}{\pgfqpoint{3.696000in}{3.696000in}}%
\pgfusepath{clip}%
\pgfsetbuttcap%
\pgfsetroundjoin%
\definecolor{currentfill}{rgb}{0.121569,0.466667,0.705882}%
\pgfsetfillcolor{currentfill}%
\pgfsetfillopacity{0.382121}%
\pgfsetlinewidth{1.003750pt}%
\definecolor{currentstroke}{rgb}{0.121569,0.466667,0.705882}%
\pgfsetstrokecolor{currentstroke}%
\pgfsetstrokeopacity{0.382121}%
\pgfsetdash{}{0pt}%
\pgfpathmoveto{\pgfqpoint{2.224771in}{2.451729in}}%
\pgfpathcurveto{\pgfqpoint{2.233007in}{2.451729in}}{\pgfqpoint{2.240908in}{2.455002in}}{\pgfqpoint{2.246731in}{2.460826in}}%
\pgfpathcurveto{\pgfqpoint{2.252555in}{2.466650in}}{\pgfqpoint{2.255828in}{2.474550in}}{\pgfqpoint{2.255828in}{2.482786in}}%
\pgfpathcurveto{\pgfqpoint{2.255828in}{2.491022in}}{\pgfqpoint{2.252555in}{2.498922in}}{\pgfqpoint{2.246731in}{2.504746in}}%
\pgfpathcurveto{\pgfqpoint{2.240908in}{2.510570in}}{\pgfqpoint{2.233007in}{2.513842in}}{\pgfqpoint{2.224771in}{2.513842in}}%
\pgfpathcurveto{\pgfqpoint{2.216535in}{2.513842in}}{\pgfqpoint{2.208635in}{2.510570in}}{\pgfqpoint{2.202811in}{2.504746in}}%
\pgfpathcurveto{\pgfqpoint{2.196987in}{2.498922in}}{\pgfqpoint{2.193715in}{2.491022in}}{\pgfqpoint{2.193715in}{2.482786in}}%
\pgfpathcurveto{\pgfqpoint{2.193715in}{2.474550in}}{\pgfqpoint{2.196987in}{2.466650in}}{\pgfqpoint{2.202811in}{2.460826in}}%
\pgfpathcurveto{\pgfqpoint{2.208635in}{2.455002in}}{\pgfqpoint{2.216535in}{2.451729in}}{\pgfqpoint{2.224771in}{2.451729in}}%
\pgfpathclose%
\pgfusepath{stroke,fill}%
\end{pgfscope}%
\begin{pgfscope}%
\pgfpathrectangle{\pgfqpoint{0.100000in}{0.212622in}}{\pgfqpoint{3.696000in}{3.696000in}}%
\pgfusepath{clip}%
\pgfsetbuttcap%
\pgfsetroundjoin%
\definecolor{currentfill}{rgb}{0.121569,0.466667,0.705882}%
\pgfsetfillcolor{currentfill}%
\pgfsetfillopacity{0.383307}%
\pgfsetlinewidth{1.003750pt}%
\definecolor{currentstroke}{rgb}{0.121569,0.466667,0.705882}%
\pgfsetstrokecolor{currentstroke}%
\pgfsetstrokeopacity{0.383307}%
\pgfsetdash{}{0pt}%
\pgfpathmoveto{\pgfqpoint{2.232035in}{2.449683in}}%
\pgfpathcurveto{\pgfqpoint{2.240271in}{2.449683in}}{\pgfqpoint{2.248171in}{2.452956in}}{\pgfqpoint{2.253995in}{2.458780in}}%
\pgfpathcurveto{\pgfqpoint{2.259819in}{2.464604in}}{\pgfqpoint{2.263091in}{2.472504in}}{\pgfqpoint{2.263091in}{2.480740in}}%
\pgfpathcurveto{\pgfqpoint{2.263091in}{2.488976in}}{\pgfqpoint{2.259819in}{2.496876in}}{\pgfqpoint{2.253995in}{2.502700in}}%
\pgfpathcurveto{\pgfqpoint{2.248171in}{2.508524in}}{\pgfqpoint{2.240271in}{2.511796in}}{\pgfqpoint{2.232035in}{2.511796in}}%
\pgfpathcurveto{\pgfqpoint{2.223798in}{2.511796in}}{\pgfqpoint{2.215898in}{2.508524in}}{\pgfqpoint{2.210074in}{2.502700in}}%
\pgfpathcurveto{\pgfqpoint{2.204250in}{2.496876in}}{\pgfqpoint{2.200978in}{2.488976in}}{\pgfqpoint{2.200978in}{2.480740in}}%
\pgfpathcurveto{\pgfqpoint{2.200978in}{2.472504in}}{\pgfqpoint{2.204250in}{2.464604in}}{\pgfqpoint{2.210074in}{2.458780in}}%
\pgfpathcurveto{\pgfqpoint{2.215898in}{2.452956in}}{\pgfqpoint{2.223798in}{2.449683in}}{\pgfqpoint{2.232035in}{2.449683in}}%
\pgfpathclose%
\pgfusepath{stroke,fill}%
\end{pgfscope}%
\begin{pgfscope}%
\pgfpathrectangle{\pgfqpoint{0.100000in}{0.212622in}}{\pgfqpoint{3.696000in}{3.696000in}}%
\pgfusepath{clip}%
\pgfsetbuttcap%
\pgfsetroundjoin%
\definecolor{currentfill}{rgb}{0.121569,0.466667,0.705882}%
\pgfsetfillcolor{currentfill}%
\pgfsetfillopacity{0.383874}%
\pgfsetlinewidth{1.003750pt}%
\definecolor{currentstroke}{rgb}{0.121569,0.466667,0.705882}%
\pgfsetstrokecolor{currentstroke}%
\pgfsetstrokeopacity{0.383874}%
\pgfsetdash{}{0pt}%
\pgfpathmoveto{\pgfqpoint{1.432727in}{2.263974in}}%
\pgfpathcurveto{\pgfqpoint{1.440963in}{2.263974in}}{\pgfqpoint{1.448863in}{2.267246in}}{\pgfqpoint{1.454687in}{2.273070in}}%
\pgfpathcurveto{\pgfqpoint{1.460511in}{2.278894in}}{\pgfqpoint{1.463783in}{2.286794in}}{\pgfqpoint{1.463783in}{2.295030in}}%
\pgfpathcurveto{\pgfqpoint{1.463783in}{2.303267in}}{\pgfqpoint{1.460511in}{2.311167in}}{\pgfqpoint{1.454687in}{2.316991in}}%
\pgfpathcurveto{\pgfqpoint{1.448863in}{2.322815in}}{\pgfqpoint{1.440963in}{2.326087in}}{\pgfqpoint{1.432727in}{2.326087in}}%
\pgfpathcurveto{\pgfqpoint{1.424491in}{2.326087in}}{\pgfqpoint{1.416591in}{2.322815in}}{\pgfqpoint{1.410767in}{2.316991in}}%
\pgfpathcurveto{\pgfqpoint{1.404943in}{2.311167in}}{\pgfqpoint{1.401670in}{2.303267in}}{\pgfqpoint{1.401670in}{2.295030in}}%
\pgfpathcurveto{\pgfqpoint{1.401670in}{2.286794in}}{\pgfqpoint{1.404943in}{2.278894in}}{\pgfqpoint{1.410767in}{2.273070in}}%
\pgfpathcurveto{\pgfqpoint{1.416591in}{2.267246in}}{\pgfqpoint{1.424491in}{2.263974in}}{\pgfqpoint{1.432727in}{2.263974in}}%
\pgfpathclose%
\pgfusepath{stroke,fill}%
\end{pgfscope}%
\begin{pgfscope}%
\pgfpathrectangle{\pgfqpoint{0.100000in}{0.212622in}}{\pgfqpoint{3.696000in}{3.696000in}}%
\pgfusepath{clip}%
\pgfsetbuttcap%
\pgfsetroundjoin%
\definecolor{currentfill}{rgb}{0.121569,0.466667,0.705882}%
\pgfsetfillcolor{currentfill}%
\pgfsetfillopacity{0.384678}%
\pgfsetlinewidth{1.003750pt}%
\definecolor{currentstroke}{rgb}{0.121569,0.466667,0.705882}%
\pgfsetstrokecolor{currentstroke}%
\pgfsetstrokeopacity{0.384678}%
\pgfsetdash{}{0pt}%
\pgfpathmoveto{\pgfqpoint{2.240607in}{2.448666in}}%
\pgfpathcurveto{\pgfqpoint{2.248843in}{2.448666in}}{\pgfqpoint{2.256743in}{2.451939in}}{\pgfqpoint{2.262567in}{2.457763in}}%
\pgfpathcurveto{\pgfqpoint{2.268391in}{2.463587in}}{\pgfqpoint{2.271663in}{2.471487in}}{\pgfqpoint{2.271663in}{2.479723in}}%
\pgfpathcurveto{\pgfqpoint{2.271663in}{2.487959in}}{\pgfqpoint{2.268391in}{2.495859in}}{\pgfqpoint{2.262567in}{2.501683in}}%
\pgfpathcurveto{\pgfqpoint{2.256743in}{2.507507in}}{\pgfqpoint{2.248843in}{2.510779in}}{\pgfqpoint{2.240607in}{2.510779in}}%
\pgfpathcurveto{\pgfqpoint{2.232370in}{2.510779in}}{\pgfqpoint{2.224470in}{2.507507in}}{\pgfqpoint{2.218646in}{2.501683in}}%
\pgfpathcurveto{\pgfqpoint{2.212822in}{2.495859in}}{\pgfqpoint{2.209550in}{2.487959in}}{\pgfqpoint{2.209550in}{2.479723in}}%
\pgfpathcurveto{\pgfqpoint{2.209550in}{2.471487in}}{\pgfqpoint{2.212822in}{2.463587in}}{\pgfqpoint{2.218646in}{2.457763in}}%
\pgfpathcurveto{\pgfqpoint{2.224470in}{2.451939in}}{\pgfqpoint{2.232370in}{2.448666in}}{\pgfqpoint{2.240607in}{2.448666in}}%
\pgfpathclose%
\pgfusepath{stroke,fill}%
\end{pgfscope}%
\begin{pgfscope}%
\pgfpathrectangle{\pgfqpoint{0.100000in}{0.212622in}}{\pgfqpoint{3.696000in}{3.696000in}}%
\pgfusepath{clip}%
\pgfsetbuttcap%
\pgfsetroundjoin%
\definecolor{currentfill}{rgb}{0.121569,0.466667,0.705882}%
\pgfsetfillcolor{currentfill}%
\pgfsetfillopacity{0.386109}%
\pgfsetlinewidth{1.003750pt}%
\definecolor{currentstroke}{rgb}{0.121569,0.466667,0.705882}%
\pgfsetstrokecolor{currentstroke}%
\pgfsetstrokeopacity{0.386109}%
\pgfsetdash{}{0pt}%
\pgfpathmoveto{\pgfqpoint{2.249636in}{2.447611in}}%
\pgfpathcurveto{\pgfqpoint{2.257872in}{2.447611in}}{\pgfqpoint{2.265772in}{2.450884in}}{\pgfqpoint{2.271596in}{2.456708in}}%
\pgfpathcurveto{\pgfqpoint{2.277420in}{2.462532in}}{\pgfqpoint{2.280692in}{2.470432in}}{\pgfqpoint{2.280692in}{2.478668in}}%
\pgfpathcurveto{\pgfqpoint{2.280692in}{2.486904in}}{\pgfqpoint{2.277420in}{2.494804in}}{\pgfqpoint{2.271596in}{2.500628in}}%
\pgfpathcurveto{\pgfqpoint{2.265772in}{2.506452in}}{\pgfqpoint{2.257872in}{2.509724in}}{\pgfqpoint{2.249636in}{2.509724in}}%
\pgfpathcurveto{\pgfqpoint{2.241399in}{2.509724in}}{\pgfqpoint{2.233499in}{2.506452in}}{\pgfqpoint{2.227675in}{2.500628in}}%
\pgfpathcurveto{\pgfqpoint{2.221851in}{2.494804in}}{\pgfqpoint{2.218579in}{2.486904in}}{\pgfqpoint{2.218579in}{2.478668in}}%
\pgfpathcurveto{\pgfqpoint{2.218579in}{2.470432in}}{\pgfqpoint{2.221851in}{2.462532in}}{\pgfqpoint{2.227675in}{2.456708in}}%
\pgfpathcurveto{\pgfqpoint{2.233499in}{2.450884in}}{\pgfqpoint{2.241399in}{2.447611in}}{\pgfqpoint{2.249636in}{2.447611in}}%
\pgfpathclose%
\pgfusepath{stroke,fill}%
\end{pgfscope}%
\begin{pgfscope}%
\pgfpathrectangle{\pgfqpoint{0.100000in}{0.212622in}}{\pgfqpoint{3.696000in}{3.696000in}}%
\pgfusepath{clip}%
\pgfsetbuttcap%
\pgfsetroundjoin%
\definecolor{currentfill}{rgb}{0.121569,0.466667,0.705882}%
\pgfsetfillcolor{currentfill}%
\pgfsetfillopacity{0.387457}%
\pgfsetlinewidth{1.003750pt}%
\definecolor{currentstroke}{rgb}{0.121569,0.466667,0.705882}%
\pgfsetstrokecolor{currentstroke}%
\pgfsetstrokeopacity{0.387457}%
\pgfsetdash{}{0pt}%
\pgfpathmoveto{\pgfqpoint{1.423284in}{2.251497in}}%
\pgfpathcurveto{\pgfqpoint{1.431520in}{2.251497in}}{\pgfqpoint{1.439420in}{2.254769in}}{\pgfqpoint{1.445244in}{2.260593in}}%
\pgfpathcurveto{\pgfqpoint{1.451068in}{2.266417in}}{\pgfqpoint{1.454340in}{2.274317in}}{\pgfqpoint{1.454340in}{2.282553in}}%
\pgfpathcurveto{\pgfqpoint{1.454340in}{2.290790in}}{\pgfqpoint{1.451068in}{2.298690in}}{\pgfqpoint{1.445244in}{2.304514in}}%
\pgfpathcurveto{\pgfqpoint{1.439420in}{2.310338in}}{\pgfqpoint{1.431520in}{2.313610in}}{\pgfqpoint{1.423284in}{2.313610in}}%
\pgfpathcurveto{\pgfqpoint{1.415047in}{2.313610in}}{\pgfqpoint{1.407147in}{2.310338in}}{\pgfqpoint{1.401323in}{2.304514in}}%
\pgfpathcurveto{\pgfqpoint{1.395499in}{2.298690in}}{\pgfqpoint{1.392227in}{2.290790in}}{\pgfqpoint{1.392227in}{2.282553in}}%
\pgfpathcurveto{\pgfqpoint{1.392227in}{2.274317in}}{\pgfqpoint{1.395499in}{2.266417in}}{\pgfqpoint{1.401323in}{2.260593in}}%
\pgfpathcurveto{\pgfqpoint{1.407147in}{2.254769in}}{\pgfqpoint{1.415047in}{2.251497in}}{\pgfqpoint{1.423284in}{2.251497in}}%
\pgfpathclose%
\pgfusepath{stroke,fill}%
\end{pgfscope}%
\begin{pgfscope}%
\pgfpathrectangle{\pgfqpoint{0.100000in}{0.212622in}}{\pgfqpoint{3.696000in}{3.696000in}}%
\pgfusepath{clip}%
\pgfsetbuttcap%
\pgfsetroundjoin%
\definecolor{currentfill}{rgb}{0.121569,0.466667,0.705882}%
\pgfsetfillcolor{currentfill}%
\pgfsetfillopacity{0.387811}%
\pgfsetlinewidth{1.003750pt}%
\definecolor{currentstroke}{rgb}{0.121569,0.466667,0.705882}%
\pgfsetstrokecolor{currentstroke}%
\pgfsetstrokeopacity{0.387811}%
\pgfsetdash{}{0pt}%
\pgfpathmoveto{\pgfqpoint{2.260523in}{2.445771in}}%
\pgfpathcurveto{\pgfqpoint{2.268760in}{2.445771in}}{\pgfqpoint{2.276660in}{2.449044in}}{\pgfqpoint{2.282484in}{2.454868in}}%
\pgfpathcurveto{\pgfqpoint{2.288308in}{2.460692in}}{\pgfqpoint{2.291580in}{2.468592in}}{\pgfqpoint{2.291580in}{2.476828in}}%
\pgfpathcurveto{\pgfqpoint{2.291580in}{2.485064in}}{\pgfqpoint{2.288308in}{2.492964in}}{\pgfqpoint{2.282484in}{2.498788in}}%
\pgfpathcurveto{\pgfqpoint{2.276660in}{2.504612in}}{\pgfqpoint{2.268760in}{2.507884in}}{\pgfqpoint{2.260523in}{2.507884in}}%
\pgfpathcurveto{\pgfqpoint{2.252287in}{2.507884in}}{\pgfqpoint{2.244387in}{2.504612in}}{\pgfqpoint{2.238563in}{2.498788in}}%
\pgfpathcurveto{\pgfqpoint{2.232739in}{2.492964in}}{\pgfqpoint{2.229467in}{2.485064in}}{\pgfqpoint{2.229467in}{2.476828in}}%
\pgfpathcurveto{\pgfqpoint{2.229467in}{2.468592in}}{\pgfqpoint{2.232739in}{2.460692in}}{\pgfqpoint{2.238563in}{2.454868in}}%
\pgfpathcurveto{\pgfqpoint{2.244387in}{2.449044in}}{\pgfqpoint{2.252287in}{2.445771in}}{\pgfqpoint{2.260523in}{2.445771in}}%
\pgfpathclose%
\pgfusepath{stroke,fill}%
\end{pgfscope}%
\begin{pgfscope}%
\pgfpathrectangle{\pgfqpoint{0.100000in}{0.212622in}}{\pgfqpoint{3.696000in}{3.696000in}}%
\pgfusepath{clip}%
\pgfsetbuttcap%
\pgfsetroundjoin%
\definecolor{currentfill}{rgb}{0.121569,0.466667,0.705882}%
\pgfsetfillcolor{currentfill}%
\pgfsetfillopacity{0.389573}%
\pgfsetlinewidth{1.003750pt}%
\definecolor{currentstroke}{rgb}{0.121569,0.466667,0.705882}%
\pgfsetstrokecolor{currentstroke}%
\pgfsetstrokeopacity{0.389573}%
\pgfsetdash{}{0pt}%
\pgfpathmoveto{\pgfqpoint{2.271800in}{2.442926in}}%
\pgfpathcurveto{\pgfqpoint{2.280036in}{2.442926in}}{\pgfqpoint{2.287936in}{2.446198in}}{\pgfqpoint{2.293760in}{2.452022in}}%
\pgfpathcurveto{\pgfqpoint{2.299584in}{2.457846in}}{\pgfqpoint{2.302857in}{2.465746in}}{\pgfqpoint{2.302857in}{2.473982in}}%
\pgfpathcurveto{\pgfqpoint{2.302857in}{2.482218in}}{\pgfqpoint{2.299584in}{2.490118in}}{\pgfqpoint{2.293760in}{2.495942in}}%
\pgfpathcurveto{\pgfqpoint{2.287936in}{2.501766in}}{\pgfqpoint{2.280036in}{2.505039in}}{\pgfqpoint{2.271800in}{2.505039in}}%
\pgfpathcurveto{\pgfqpoint{2.263564in}{2.505039in}}{\pgfqpoint{2.255664in}{2.501766in}}{\pgfqpoint{2.249840in}{2.495942in}}%
\pgfpathcurveto{\pgfqpoint{2.244016in}{2.490118in}}{\pgfqpoint{2.240744in}{2.482218in}}{\pgfqpoint{2.240744in}{2.473982in}}%
\pgfpathcurveto{\pgfqpoint{2.240744in}{2.465746in}}{\pgfqpoint{2.244016in}{2.457846in}}{\pgfqpoint{2.249840in}{2.452022in}}%
\pgfpathcurveto{\pgfqpoint{2.255664in}{2.446198in}}{\pgfqpoint{2.263564in}{2.442926in}}{\pgfqpoint{2.271800in}{2.442926in}}%
\pgfpathclose%
\pgfusepath{stroke,fill}%
\end{pgfscope}%
\begin{pgfscope}%
\pgfpathrectangle{\pgfqpoint{0.100000in}{0.212622in}}{\pgfqpoint{3.696000in}{3.696000in}}%
\pgfusepath{clip}%
\pgfsetbuttcap%
\pgfsetroundjoin%
\definecolor{currentfill}{rgb}{0.121569,0.466667,0.705882}%
\pgfsetfillcolor{currentfill}%
\pgfsetfillopacity{0.390629}%
\pgfsetlinewidth{1.003750pt}%
\definecolor{currentstroke}{rgb}{0.121569,0.466667,0.705882}%
\pgfsetstrokecolor{currentstroke}%
\pgfsetstrokeopacity{0.390629}%
\pgfsetdash{}{0pt}%
\pgfpathmoveto{\pgfqpoint{1.413847in}{2.238490in}}%
\pgfpathcurveto{\pgfqpoint{1.422083in}{2.238490in}}{\pgfqpoint{1.429983in}{2.241763in}}{\pgfqpoint{1.435807in}{2.247587in}}%
\pgfpathcurveto{\pgfqpoint{1.441631in}{2.253411in}}{\pgfqpoint{1.444903in}{2.261311in}}{\pgfqpoint{1.444903in}{2.269547in}}%
\pgfpathcurveto{\pgfqpoint{1.444903in}{2.277783in}}{\pgfqpoint{1.441631in}{2.285683in}}{\pgfqpoint{1.435807in}{2.291507in}}%
\pgfpathcurveto{\pgfqpoint{1.429983in}{2.297331in}}{\pgfqpoint{1.422083in}{2.300603in}}{\pgfqpoint{1.413847in}{2.300603in}}%
\pgfpathcurveto{\pgfqpoint{1.405611in}{2.300603in}}{\pgfqpoint{1.397710in}{2.297331in}}{\pgfqpoint{1.391887in}{2.291507in}}%
\pgfpathcurveto{\pgfqpoint{1.386063in}{2.285683in}}{\pgfqpoint{1.382790in}{2.277783in}}{\pgfqpoint{1.382790in}{2.269547in}}%
\pgfpathcurveto{\pgfqpoint{1.382790in}{2.261311in}}{\pgfqpoint{1.386063in}{2.253411in}}{\pgfqpoint{1.391887in}{2.247587in}}%
\pgfpathcurveto{\pgfqpoint{1.397710in}{2.241763in}}{\pgfqpoint{1.405611in}{2.238490in}}{\pgfqpoint{1.413847in}{2.238490in}}%
\pgfpathclose%
\pgfusepath{stroke,fill}%
\end{pgfscope}%
\begin{pgfscope}%
\pgfpathrectangle{\pgfqpoint{0.100000in}{0.212622in}}{\pgfqpoint{3.696000in}{3.696000in}}%
\pgfusepath{clip}%
\pgfsetbuttcap%
\pgfsetroundjoin%
\definecolor{currentfill}{rgb}{0.121569,0.466667,0.705882}%
\pgfsetfillcolor{currentfill}%
\pgfsetfillopacity{0.391989}%
\pgfsetlinewidth{1.003750pt}%
\definecolor{currentstroke}{rgb}{0.121569,0.466667,0.705882}%
\pgfsetstrokecolor{currentstroke}%
\pgfsetstrokeopacity{0.391989}%
\pgfsetdash{}{0pt}%
\pgfpathmoveto{\pgfqpoint{2.285343in}{2.441784in}}%
\pgfpathcurveto{\pgfqpoint{2.293580in}{2.441784in}}{\pgfqpoint{2.301480in}{2.445056in}}{\pgfqpoint{2.307304in}{2.450880in}}%
\pgfpathcurveto{\pgfqpoint{2.313127in}{2.456704in}}{\pgfqpoint{2.316400in}{2.464604in}}{\pgfqpoint{2.316400in}{2.472840in}}%
\pgfpathcurveto{\pgfqpoint{2.316400in}{2.481077in}}{\pgfqpoint{2.313127in}{2.488977in}}{\pgfqpoint{2.307304in}{2.494801in}}%
\pgfpathcurveto{\pgfqpoint{2.301480in}{2.500624in}}{\pgfqpoint{2.293580in}{2.503897in}}{\pgfqpoint{2.285343in}{2.503897in}}%
\pgfpathcurveto{\pgfqpoint{2.277107in}{2.503897in}}{\pgfqpoint{2.269207in}{2.500624in}}{\pgfqpoint{2.263383in}{2.494801in}}%
\pgfpathcurveto{\pgfqpoint{2.257559in}{2.488977in}}{\pgfqpoint{2.254287in}{2.481077in}}{\pgfqpoint{2.254287in}{2.472840in}}%
\pgfpathcurveto{\pgfqpoint{2.254287in}{2.464604in}}{\pgfqpoint{2.257559in}{2.456704in}}{\pgfqpoint{2.263383in}{2.450880in}}%
\pgfpathcurveto{\pgfqpoint{2.269207in}{2.445056in}}{\pgfqpoint{2.277107in}{2.441784in}}{\pgfqpoint{2.285343in}{2.441784in}}%
\pgfpathclose%
\pgfusepath{stroke,fill}%
\end{pgfscope}%
\begin{pgfscope}%
\pgfpathrectangle{\pgfqpoint{0.100000in}{0.212622in}}{\pgfqpoint{3.696000in}{3.696000in}}%
\pgfusepath{clip}%
\pgfsetbuttcap%
\pgfsetroundjoin%
\definecolor{currentfill}{rgb}{0.121569,0.466667,0.705882}%
\pgfsetfillcolor{currentfill}%
\pgfsetfillopacity{0.393260}%
\pgfsetlinewidth{1.003750pt}%
\definecolor{currentstroke}{rgb}{0.121569,0.466667,0.705882}%
\pgfsetstrokecolor{currentstroke}%
\pgfsetstrokeopacity{0.393260}%
\pgfsetdash{}{0pt}%
\pgfpathmoveto{\pgfqpoint{2.292730in}{2.440563in}}%
\pgfpathcurveto{\pgfqpoint{2.300966in}{2.440563in}}{\pgfqpoint{2.308866in}{2.443835in}}{\pgfqpoint{2.314690in}{2.449659in}}%
\pgfpathcurveto{\pgfqpoint{2.320514in}{2.455483in}}{\pgfqpoint{2.323787in}{2.463383in}}{\pgfqpoint{2.323787in}{2.471619in}}%
\pgfpathcurveto{\pgfqpoint{2.323787in}{2.479856in}}{\pgfqpoint{2.320514in}{2.487756in}}{\pgfqpoint{2.314690in}{2.493580in}}%
\pgfpathcurveto{\pgfqpoint{2.308866in}{2.499404in}}{\pgfqpoint{2.300966in}{2.502676in}}{\pgfqpoint{2.292730in}{2.502676in}}%
\pgfpathcurveto{\pgfqpoint{2.284494in}{2.502676in}}{\pgfqpoint{2.276594in}{2.499404in}}{\pgfqpoint{2.270770in}{2.493580in}}%
\pgfpathcurveto{\pgfqpoint{2.264946in}{2.487756in}}{\pgfqpoint{2.261674in}{2.479856in}}{\pgfqpoint{2.261674in}{2.471619in}}%
\pgfpathcurveto{\pgfqpoint{2.261674in}{2.463383in}}{\pgfqpoint{2.264946in}{2.455483in}}{\pgfqpoint{2.270770in}{2.449659in}}%
\pgfpathcurveto{\pgfqpoint{2.276594in}{2.443835in}}{\pgfqpoint{2.284494in}{2.440563in}}{\pgfqpoint{2.292730in}{2.440563in}}%
\pgfpathclose%
\pgfusepath{stroke,fill}%
\end{pgfscope}%
\begin{pgfscope}%
\pgfpathrectangle{\pgfqpoint{0.100000in}{0.212622in}}{\pgfqpoint{3.696000in}{3.696000in}}%
\pgfusepath{clip}%
\pgfsetbuttcap%
\pgfsetroundjoin%
\definecolor{currentfill}{rgb}{0.121569,0.466667,0.705882}%
\pgfsetfillcolor{currentfill}%
\pgfsetfillopacity{0.393688}%
\pgfsetlinewidth{1.003750pt}%
\definecolor{currentstroke}{rgb}{0.121569,0.466667,0.705882}%
\pgfsetstrokecolor{currentstroke}%
\pgfsetstrokeopacity{0.393688}%
\pgfsetdash{}{0pt}%
\pgfpathmoveto{\pgfqpoint{1.404637in}{2.225582in}}%
\pgfpathcurveto{\pgfqpoint{1.412873in}{2.225582in}}{\pgfqpoint{1.420773in}{2.228855in}}{\pgfqpoint{1.426597in}{2.234679in}}%
\pgfpathcurveto{\pgfqpoint{1.432421in}{2.240503in}}{\pgfqpoint{1.435693in}{2.248403in}}{\pgfqpoint{1.435693in}{2.256639in}}%
\pgfpathcurveto{\pgfqpoint{1.435693in}{2.264875in}}{\pgfqpoint{1.432421in}{2.272775in}}{\pgfqpoint{1.426597in}{2.278599in}}%
\pgfpathcurveto{\pgfqpoint{1.420773in}{2.284423in}}{\pgfqpoint{1.412873in}{2.287695in}}{\pgfqpoint{1.404637in}{2.287695in}}%
\pgfpathcurveto{\pgfqpoint{1.396400in}{2.287695in}}{\pgfqpoint{1.388500in}{2.284423in}}{\pgfqpoint{1.382676in}{2.278599in}}%
\pgfpathcurveto{\pgfqpoint{1.376852in}{2.272775in}}{\pgfqpoint{1.373580in}{2.264875in}}{\pgfqpoint{1.373580in}{2.256639in}}%
\pgfpathcurveto{\pgfqpoint{1.373580in}{2.248403in}}{\pgfqpoint{1.376852in}{2.240503in}}{\pgfqpoint{1.382676in}{2.234679in}}%
\pgfpathcurveto{\pgfqpoint{1.388500in}{2.228855in}}{\pgfqpoint{1.396400in}{2.225582in}}{\pgfqpoint{1.404637in}{2.225582in}}%
\pgfpathclose%
\pgfusepath{stroke,fill}%
\end{pgfscope}%
\begin{pgfscope}%
\pgfpathrectangle{\pgfqpoint{0.100000in}{0.212622in}}{\pgfqpoint{3.696000in}{3.696000in}}%
\pgfusepath{clip}%
\pgfsetbuttcap%
\pgfsetroundjoin%
\definecolor{currentfill}{rgb}{0.121569,0.466667,0.705882}%
\pgfsetfillcolor{currentfill}%
\pgfsetfillopacity{0.393961}%
\pgfsetlinewidth{1.003750pt}%
\definecolor{currentstroke}{rgb}{0.121569,0.466667,0.705882}%
\pgfsetstrokecolor{currentstroke}%
\pgfsetstrokeopacity{0.393961}%
\pgfsetdash{}{0pt}%
\pgfpathmoveto{\pgfqpoint{2.296810in}{2.439949in}}%
\pgfpathcurveto{\pgfqpoint{2.305046in}{2.439949in}}{\pgfqpoint{2.312947in}{2.443221in}}{\pgfqpoint{2.318770in}{2.449045in}}%
\pgfpathcurveto{\pgfqpoint{2.324594in}{2.454869in}}{\pgfqpoint{2.327867in}{2.462769in}}{\pgfqpoint{2.327867in}{2.471005in}}%
\pgfpathcurveto{\pgfqpoint{2.327867in}{2.479242in}}{\pgfqpoint{2.324594in}{2.487142in}}{\pgfqpoint{2.318770in}{2.492966in}}%
\pgfpathcurveto{\pgfqpoint{2.312947in}{2.498790in}}{\pgfqpoint{2.305046in}{2.502062in}}{\pgfqpoint{2.296810in}{2.502062in}}%
\pgfpathcurveto{\pgfqpoint{2.288574in}{2.502062in}}{\pgfqpoint{2.280674in}{2.498790in}}{\pgfqpoint{2.274850in}{2.492966in}}%
\pgfpathcurveto{\pgfqpoint{2.269026in}{2.487142in}}{\pgfqpoint{2.265754in}{2.479242in}}{\pgfqpoint{2.265754in}{2.471005in}}%
\pgfpathcurveto{\pgfqpoint{2.265754in}{2.462769in}}{\pgfqpoint{2.269026in}{2.454869in}}{\pgfqpoint{2.274850in}{2.449045in}}%
\pgfpathcurveto{\pgfqpoint{2.280674in}{2.443221in}}{\pgfqpoint{2.288574in}{2.439949in}}{\pgfqpoint{2.296810in}{2.439949in}}%
\pgfpathclose%
\pgfusepath{stroke,fill}%
\end{pgfscope}%
\begin{pgfscope}%
\pgfpathrectangle{\pgfqpoint{0.100000in}{0.212622in}}{\pgfqpoint{3.696000in}{3.696000in}}%
\pgfusepath{clip}%
\pgfsetbuttcap%
\pgfsetroundjoin%
\definecolor{currentfill}{rgb}{0.121569,0.466667,0.705882}%
\pgfsetfillcolor{currentfill}%
\pgfsetfillopacity{0.394821}%
\pgfsetlinewidth{1.003750pt}%
\definecolor{currentstroke}{rgb}{0.121569,0.466667,0.705882}%
\pgfsetstrokecolor{currentstroke}%
\pgfsetstrokeopacity{0.394821}%
\pgfsetdash{}{0pt}%
\pgfpathmoveto{\pgfqpoint{2.301993in}{2.438809in}}%
\pgfpathcurveto{\pgfqpoint{2.310229in}{2.438809in}}{\pgfqpoint{2.318129in}{2.442081in}}{\pgfqpoint{2.323953in}{2.447905in}}%
\pgfpathcurveto{\pgfqpoint{2.329777in}{2.453729in}}{\pgfqpoint{2.333050in}{2.461629in}}{\pgfqpoint{2.333050in}{2.469865in}}%
\pgfpathcurveto{\pgfqpoint{2.333050in}{2.478102in}}{\pgfqpoint{2.329777in}{2.486002in}}{\pgfqpoint{2.323953in}{2.491826in}}%
\pgfpathcurveto{\pgfqpoint{2.318129in}{2.497650in}}{\pgfqpoint{2.310229in}{2.500922in}}{\pgfqpoint{2.301993in}{2.500922in}}%
\pgfpathcurveto{\pgfqpoint{2.293757in}{2.500922in}}{\pgfqpoint{2.285857in}{2.497650in}}{\pgfqpoint{2.280033in}{2.491826in}}%
\pgfpathcurveto{\pgfqpoint{2.274209in}{2.486002in}}{\pgfqpoint{2.270937in}{2.478102in}}{\pgfqpoint{2.270937in}{2.469865in}}%
\pgfpathcurveto{\pgfqpoint{2.270937in}{2.461629in}}{\pgfqpoint{2.274209in}{2.453729in}}{\pgfqpoint{2.280033in}{2.447905in}}%
\pgfpathcurveto{\pgfqpoint{2.285857in}{2.442081in}}{\pgfqpoint{2.293757in}{2.438809in}}{\pgfqpoint{2.301993in}{2.438809in}}%
\pgfpathclose%
\pgfusepath{stroke,fill}%
\end{pgfscope}%
\begin{pgfscope}%
\pgfpathrectangle{\pgfqpoint{0.100000in}{0.212622in}}{\pgfqpoint{3.696000in}{3.696000in}}%
\pgfusepath{clip}%
\pgfsetbuttcap%
\pgfsetroundjoin%
\definecolor{currentfill}{rgb}{0.121569,0.466667,0.705882}%
\pgfsetfillcolor{currentfill}%
\pgfsetfillopacity{0.395308}%
\pgfsetlinewidth{1.003750pt}%
\definecolor{currentstroke}{rgb}{0.121569,0.466667,0.705882}%
\pgfsetstrokecolor{currentstroke}%
\pgfsetstrokeopacity{0.395308}%
\pgfsetdash{}{0pt}%
\pgfpathmoveto{\pgfqpoint{2.304869in}{2.438350in}}%
\pgfpathcurveto{\pgfqpoint{2.313106in}{2.438350in}}{\pgfqpoint{2.321006in}{2.441622in}}{\pgfqpoint{2.326830in}{2.447446in}}%
\pgfpathcurveto{\pgfqpoint{2.332653in}{2.453270in}}{\pgfqpoint{2.335926in}{2.461170in}}{\pgfqpoint{2.335926in}{2.469406in}}%
\pgfpathcurveto{\pgfqpoint{2.335926in}{2.477642in}}{\pgfqpoint{2.332653in}{2.485542in}}{\pgfqpoint{2.326830in}{2.491366in}}%
\pgfpathcurveto{\pgfqpoint{2.321006in}{2.497190in}}{\pgfqpoint{2.313106in}{2.500463in}}{\pgfqpoint{2.304869in}{2.500463in}}%
\pgfpathcurveto{\pgfqpoint{2.296633in}{2.500463in}}{\pgfqpoint{2.288733in}{2.497190in}}{\pgfqpoint{2.282909in}{2.491366in}}%
\pgfpathcurveto{\pgfqpoint{2.277085in}{2.485542in}}{\pgfqpoint{2.273813in}{2.477642in}}{\pgfqpoint{2.273813in}{2.469406in}}%
\pgfpathcurveto{\pgfqpoint{2.273813in}{2.461170in}}{\pgfqpoint{2.277085in}{2.453270in}}{\pgfqpoint{2.282909in}{2.447446in}}%
\pgfpathcurveto{\pgfqpoint{2.288733in}{2.441622in}}{\pgfqpoint{2.296633in}{2.438350in}}{\pgfqpoint{2.304869in}{2.438350in}}%
\pgfpathclose%
\pgfusepath{stroke,fill}%
\end{pgfscope}%
\begin{pgfscope}%
\pgfpathrectangle{\pgfqpoint{0.100000in}{0.212622in}}{\pgfqpoint{3.696000in}{3.696000in}}%
\pgfusepath{clip}%
\pgfsetbuttcap%
\pgfsetroundjoin%
\definecolor{currentfill}{rgb}{0.121569,0.466667,0.705882}%
\pgfsetfillcolor{currentfill}%
\pgfsetfillopacity{0.395597}%
\pgfsetlinewidth{1.003750pt}%
\definecolor{currentstroke}{rgb}{0.121569,0.466667,0.705882}%
\pgfsetstrokecolor{currentstroke}%
\pgfsetstrokeopacity{0.395597}%
\pgfsetdash{}{0pt}%
\pgfpathmoveto{\pgfqpoint{2.306438in}{2.438195in}}%
\pgfpathcurveto{\pgfqpoint{2.314674in}{2.438195in}}{\pgfqpoint{2.322574in}{2.441467in}}{\pgfqpoint{2.328398in}{2.447291in}}%
\pgfpathcurveto{\pgfqpoint{2.334222in}{2.453115in}}{\pgfqpoint{2.337494in}{2.461015in}}{\pgfqpoint{2.337494in}{2.469251in}}%
\pgfpathcurveto{\pgfqpoint{2.337494in}{2.477487in}}{\pgfqpoint{2.334222in}{2.485388in}}{\pgfqpoint{2.328398in}{2.491211in}}%
\pgfpathcurveto{\pgfqpoint{2.322574in}{2.497035in}}{\pgfqpoint{2.314674in}{2.500308in}}{\pgfqpoint{2.306438in}{2.500308in}}%
\pgfpathcurveto{\pgfqpoint{2.298202in}{2.500308in}}{\pgfqpoint{2.290301in}{2.497035in}}{\pgfqpoint{2.284478in}{2.491211in}}%
\pgfpathcurveto{\pgfqpoint{2.278654in}{2.485388in}}{\pgfqpoint{2.275381in}{2.477487in}}{\pgfqpoint{2.275381in}{2.469251in}}%
\pgfpathcurveto{\pgfqpoint{2.275381in}{2.461015in}}{\pgfqpoint{2.278654in}{2.453115in}}{\pgfqpoint{2.284478in}{2.447291in}}%
\pgfpathcurveto{\pgfqpoint{2.290301in}{2.441467in}}{\pgfqpoint{2.298202in}{2.438195in}}{\pgfqpoint{2.306438in}{2.438195in}}%
\pgfpathclose%
\pgfusepath{stroke,fill}%
\end{pgfscope}%
\begin{pgfscope}%
\pgfpathrectangle{\pgfqpoint{0.100000in}{0.212622in}}{\pgfqpoint{3.696000in}{3.696000in}}%
\pgfusepath{clip}%
\pgfsetbuttcap%
\pgfsetroundjoin%
\definecolor{currentfill}{rgb}{0.121569,0.466667,0.705882}%
\pgfsetfillcolor{currentfill}%
\pgfsetfillopacity{0.396174}%
\pgfsetlinewidth{1.003750pt}%
\definecolor{currentstroke}{rgb}{0.121569,0.466667,0.705882}%
\pgfsetstrokecolor{currentstroke}%
\pgfsetstrokeopacity{0.396174}%
\pgfsetdash{}{0pt}%
\pgfpathmoveto{\pgfqpoint{2.309608in}{2.437726in}}%
\pgfpathcurveto{\pgfqpoint{2.317845in}{2.437726in}}{\pgfqpoint{2.325745in}{2.440998in}}{\pgfqpoint{2.331569in}{2.446822in}}%
\pgfpathcurveto{\pgfqpoint{2.337392in}{2.452646in}}{\pgfqpoint{2.340665in}{2.460546in}}{\pgfqpoint{2.340665in}{2.468783in}}%
\pgfpathcurveto{\pgfqpoint{2.340665in}{2.477019in}}{\pgfqpoint{2.337392in}{2.484919in}}{\pgfqpoint{2.331569in}{2.490743in}}%
\pgfpathcurveto{\pgfqpoint{2.325745in}{2.496567in}}{\pgfqpoint{2.317845in}{2.499839in}}{\pgfqpoint{2.309608in}{2.499839in}}%
\pgfpathcurveto{\pgfqpoint{2.301372in}{2.499839in}}{\pgfqpoint{2.293472in}{2.496567in}}{\pgfqpoint{2.287648in}{2.490743in}}%
\pgfpathcurveto{\pgfqpoint{2.281824in}{2.484919in}}{\pgfqpoint{2.278552in}{2.477019in}}{\pgfqpoint{2.278552in}{2.468783in}}%
\pgfpathcurveto{\pgfqpoint{2.278552in}{2.460546in}}{\pgfqpoint{2.281824in}{2.452646in}}{\pgfqpoint{2.287648in}{2.446822in}}%
\pgfpathcurveto{\pgfqpoint{2.293472in}{2.440998in}}{\pgfqpoint{2.301372in}{2.437726in}}{\pgfqpoint{2.309608in}{2.437726in}}%
\pgfpathclose%
\pgfusepath{stroke,fill}%
\end{pgfscope}%
\begin{pgfscope}%
\pgfpathrectangle{\pgfqpoint{0.100000in}{0.212622in}}{\pgfqpoint{3.696000in}{3.696000in}}%
\pgfusepath{clip}%
\pgfsetbuttcap%
\pgfsetroundjoin%
\definecolor{currentfill}{rgb}{0.121569,0.466667,0.705882}%
\pgfsetfillcolor{currentfill}%
\pgfsetfillopacity{0.396471}%
\pgfsetlinewidth{1.003750pt}%
\definecolor{currentstroke}{rgb}{0.121569,0.466667,0.705882}%
\pgfsetstrokecolor{currentstroke}%
\pgfsetstrokeopacity{0.396471}%
\pgfsetdash{}{0pt}%
\pgfpathmoveto{\pgfqpoint{2.311369in}{2.437382in}}%
\pgfpathcurveto{\pgfqpoint{2.319605in}{2.437382in}}{\pgfqpoint{2.327505in}{2.440654in}}{\pgfqpoint{2.333329in}{2.446478in}}%
\pgfpathcurveto{\pgfqpoint{2.339153in}{2.452302in}}{\pgfqpoint{2.342425in}{2.460202in}}{\pgfqpoint{2.342425in}{2.468439in}}%
\pgfpathcurveto{\pgfqpoint{2.342425in}{2.476675in}}{\pgfqpoint{2.339153in}{2.484575in}}{\pgfqpoint{2.333329in}{2.490399in}}%
\pgfpathcurveto{\pgfqpoint{2.327505in}{2.496223in}}{\pgfqpoint{2.319605in}{2.499495in}}{\pgfqpoint{2.311369in}{2.499495in}}%
\pgfpathcurveto{\pgfqpoint{2.303132in}{2.499495in}}{\pgfqpoint{2.295232in}{2.496223in}}{\pgfqpoint{2.289408in}{2.490399in}}%
\pgfpathcurveto{\pgfqpoint{2.283585in}{2.484575in}}{\pgfqpoint{2.280312in}{2.476675in}}{\pgfqpoint{2.280312in}{2.468439in}}%
\pgfpathcurveto{\pgfqpoint{2.280312in}{2.460202in}}{\pgfqpoint{2.283585in}{2.452302in}}{\pgfqpoint{2.289408in}{2.446478in}}%
\pgfpathcurveto{\pgfqpoint{2.295232in}{2.440654in}}{\pgfqpoint{2.303132in}{2.437382in}}{\pgfqpoint{2.311369in}{2.437382in}}%
\pgfpathclose%
\pgfusepath{stroke,fill}%
\end{pgfscope}%
\begin{pgfscope}%
\pgfpathrectangle{\pgfqpoint{0.100000in}{0.212622in}}{\pgfqpoint{3.696000in}{3.696000in}}%
\pgfusepath{clip}%
\pgfsetbuttcap%
\pgfsetroundjoin%
\definecolor{currentfill}{rgb}{0.121569,0.466667,0.705882}%
\pgfsetfillcolor{currentfill}%
\pgfsetfillopacity{0.396613}%
\pgfsetlinewidth{1.003750pt}%
\definecolor{currentstroke}{rgb}{0.121569,0.466667,0.705882}%
\pgfsetstrokecolor{currentstroke}%
\pgfsetstrokeopacity{0.396613}%
\pgfsetdash{}{0pt}%
\pgfpathmoveto{\pgfqpoint{1.396164in}{2.214165in}}%
\pgfpathcurveto{\pgfqpoint{1.404401in}{2.214165in}}{\pgfqpoint{1.412301in}{2.217437in}}{\pgfqpoint{1.418125in}{2.223261in}}%
\pgfpathcurveto{\pgfqpoint{1.423949in}{2.229085in}}{\pgfqpoint{1.427221in}{2.236985in}}{\pgfqpoint{1.427221in}{2.245221in}}%
\pgfpathcurveto{\pgfqpoint{1.427221in}{2.253458in}}{\pgfqpoint{1.423949in}{2.261358in}}{\pgfqpoint{1.418125in}{2.267182in}}%
\pgfpathcurveto{\pgfqpoint{1.412301in}{2.273005in}}{\pgfqpoint{1.404401in}{2.276278in}}{\pgfqpoint{1.396164in}{2.276278in}}%
\pgfpathcurveto{\pgfqpoint{1.387928in}{2.276278in}}{\pgfqpoint{1.380028in}{2.273005in}}{\pgfqpoint{1.374204in}{2.267182in}}%
\pgfpathcurveto{\pgfqpoint{1.368380in}{2.261358in}}{\pgfqpoint{1.365108in}{2.253458in}}{\pgfqpoint{1.365108in}{2.245221in}}%
\pgfpathcurveto{\pgfqpoint{1.365108in}{2.236985in}}{\pgfqpoint{1.368380in}{2.229085in}}{\pgfqpoint{1.374204in}{2.223261in}}%
\pgfpathcurveto{\pgfqpoint{1.380028in}{2.217437in}}{\pgfqpoint{1.387928in}{2.214165in}}{\pgfqpoint{1.396164in}{2.214165in}}%
\pgfpathclose%
\pgfusepath{stroke,fill}%
\end{pgfscope}%
\begin{pgfscope}%
\pgfpathrectangle{\pgfqpoint{0.100000in}{0.212622in}}{\pgfqpoint{3.696000in}{3.696000in}}%
\pgfusepath{clip}%
\pgfsetbuttcap%
\pgfsetroundjoin%
\definecolor{currentfill}{rgb}{0.121569,0.466667,0.705882}%
\pgfsetfillcolor{currentfill}%
\pgfsetfillopacity{0.396854}%
\pgfsetlinewidth{1.003750pt}%
\definecolor{currentstroke}{rgb}{0.121569,0.466667,0.705882}%
\pgfsetstrokecolor{currentstroke}%
\pgfsetstrokeopacity{0.396854}%
\pgfsetdash{}{0pt}%
\pgfpathmoveto{\pgfqpoint{2.313912in}{2.436435in}}%
\pgfpathcurveto{\pgfqpoint{2.322148in}{2.436435in}}{\pgfqpoint{2.330049in}{2.439707in}}{\pgfqpoint{2.335872in}{2.445531in}}%
\pgfpathcurveto{\pgfqpoint{2.341696in}{2.451355in}}{\pgfqpoint{2.344969in}{2.459255in}}{\pgfqpoint{2.344969in}{2.467492in}}%
\pgfpathcurveto{\pgfqpoint{2.344969in}{2.475728in}}{\pgfqpoint{2.341696in}{2.483628in}}{\pgfqpoint{2.335872in}{2.489452in}}%
\pgfpathcurveto{\pgfqpoint{2.330049in}{2.495276in}}{\pgfqpoint{2.322148in}{2.498548in}}{\pgfqpoint{2.313912in}{2.498548in}}%
\pgfpathcurveto{\pgfqpoint{2.305676in}{2.498548in}}{\pgfqpoint{2.297776in}{2.495276in}}{\pgfqpoint{2.291952in}{2.489452in}}%
\pgfpathcurveto{\pgfqpoint{2.286128in}{2.483628in}}{\pgfqpoint{2.282856in}{2.475728in}}{\pgfqpoint{2.282856in}{2.467492in}}%
\pgfpathcurveto{\pgfqpoint{2.282856in}{2.459255in}}{\pgfqpoint{2.286128in}{2.451355in}}{\pgfqpoint{2.291952in}{2.445531in}}%
\pgfpathcurveto{\pgfqpoint{2.297776in}{2.439707in}}{\pgfqpoint{2.305676in}{2.436435in}}{\pgfqpoint{2.313912in}{2.436435in}}%
\pgfpathclose%
\pgfusepath{stroke,fill}%
\end{pgfscope}%
\begin{pgfscope}%
\pgfpathrectangle{\pgfqpoint{0.100000in}{0.212622in}}{\pgfqpoint{3.696000in}{3.696000in}}%
\pgfusepath{clip}%
\pgfsetbuttcap%
\pgfsetroundjoin%
\definecolor{currentfill}{rgb}{0.121569,0.466667,0.705882}%
\pgfsetfillcolor{currentfill}%
\pgfsetfillopacity{0.397055}%
\pgfsetlinewidth{1.003750pt}%
\definecolor{currentstroke}{rgb}{0.121569,0.466667,0.705882}%
\pgfsetstrokecolor{currentstroke}%
\pgfsetstrokeopacity{0.397055}%
\pgfsetdash{}{0pt}%
\pgfpathmoveto{\pgfqpoint{2.315384in}{2.436074in}}%
\pgfpathcurveto{\pgfqpoint{2.323621in}{2.436074in}}{\pgfqpoint{2.331521in}{2.439346in}}{\pgfqpoint{2.337345in}{2.445170in}}%
\pgfpathcurveto{\pgfqpoint{2.343168in}{2.450994in}}{\pgfqpoint{2.346441in}{2.458894in}}{\pgfqpoint{2.346441in}{2.467130in}}%
\pgfpathcurveto{\pgfqpoint{2.346441in}{2.475367in}}{\pgfqpoint{2.343168in}{2.483267in}}{\pgfqpoint{2.337345in}{2.489091in}}%
\pgfpathcurveto{\pgfqpoint{2.331521in}{2.494915in}}{\pgfqpoint{2.323621in}{2.498187in}}{\pgfqpoint{2.315384in}{2.498187in}}%
\pgfpathcurveto{\pgfqpoint{2.307148in}{2.498187in}}{\pgfqpoint{2.299248in}{2.494915in}}{\pgfqpoint{2.293424in}{2.489091in}}%
\pgfpathcurveto{\pgfqpoint{2.287600in}{2.483267in}}{\pgfqpoint{2.284328in}{2.475367in}}{\pgfqpoint{2.284328in}{2.467130in}}%
\pgfpathcurveto{\pgfqpoint{2.284328in}{2.458894in}}{\pgfqpoint{2.287600in}{2.450994in}}{\pgfqpoint{2.293424in}{2.445170in}}%
\pgfpathcurveto{\pgfqpoint{2.299248in}{2.439346in}}{\pgfqpoint{2.307148in}{2.436074in}}{\pgfqpoint{2.315384in}{2.436074in}}%
\pgfpathclose%
\pgfusepath{stroke,fill}%
\end{pgfscope}%
\begin{pgfscope}%
\pgfpathrectangle{\pgfqpoint{0.100000in}{0.212622in}}{\pgfqpoint{3.696000in}{3.696000in}}%
\pgfusepath{clip}%
\pgfsetbuttcap%
\pgfsetroundjoin%
\definecolor{currentfill}{rgb}{0.121569,0.466667,0.705882}%
\pgfsetfillcolor{currentfill}%
\pgfsetfillopacity{0.397176}%
\pgfsetlinewidth{1.003750pt}%
\definecolor{currentstroke}{rgb}{0.121569,0.466667,0.705882}%
\pgfsetstrokecolor{currentstroke}%
\pgfsetstrokeopacity{0.397176}%
\pgfsetdash{}{0pt}%
\pgfpathmoveto{\pgfqpoint{2.316207in}{2.435991in}}%
\pgfpathcurveto{\pgfqpoint{2.324443in}{2.435991in}}{\pgfqpoint{2.332343in}{2.439263in}}{\pgfqpoint{2.338167in}{2.445087in}}%
\pgfpathcurveto{\pgfqpoint{2.343991in}{2.450911in}}{\pgfqpoint{2.347263in}{2.458811in}}{\pgfqpoint{2.347263in}{2.467048in}}%
\pgfpathcurveto{\pgfqpoint{2.347263in}{2.475284in}}{\pgfqpoint{2.343991in}{2.483184in}}{\pgfqpoint{2.338167in}{2.489008in}}%
\pgfpathcurveto{\pgfqpoint{2.332343in}{2.494832in}}{\pgfqpoint{2.324443in}{2.498104in}}{\pgfqpoint{2.316207in}{2.498104in}}%
\pgfpathcurveto{\pgfqpoint{2.307971in}{2.498104in}}{\pgfqpoint{2.300070in}{2.494832in}}{\pgfqpoint{2.294247in}{2.489008in}}%
\pgfpathcurveto{\pgfqpoint{2.288423in}{2.483184in}}{\pgfqpoint{2.285150in}{2.475284in}}{\pgfqpoint{2.285150in}{2.467048in}}%
\pgfpathcurveto{\pgfqpoint{2.285150in}{2.458811in}}{\pgfqpoint{2.288423in}{2.450911in}}{\pgfqpoint{2.294247in}{2.445087in}}%
\pgfpathcurveto{\pgfqpoint{2.300070in}{2.439263in}}{\pgfqpoint{2.307971in}{2.435991in}}{\pgfqpoint{2.316207in}{2.435991in}}%
\pgfpathclose%
\pgfusepath{stroke,fill}%
\end{pgfscope}%
\begin{pgfscope}%
\pgfpathrectangle{\pgfqpoint{0.100000in}{0.212622in}}{\pgfqpoint{3.696000in}{3.696000in}}%
\pgfusepath{clip}%
\pgfsetbuttcap%
\pgfsetroundjoin%
\definecolor{currentfill}{rgb}{0.121569,0.466667,0.705882}%
\pgfsetfillcolor{currentfill}%
\pgfsetfillopacity{0.397529}%
\pgfsetlinewidth{1.003750pt}%
\definecolor{currentstroke}{rgb}{0.121569,0.466667,0.705882}%
\pgfsetstrokecolor{currentstroke}%
\pgfsetstrokeopacity{0.397529}%
\pgfsetdash{}{0pt}%
\pgfpathmoveto{\pgfqpoint{2.318626in}{2.435685in}}%
\pgfpathcurveto{\pgfqpoint{2.326862in}{2.435685in}}{\pgfqpoint{2.334762in}{2.438957in}}{\pgfqpoint{2.340586in}{2.444781in}}%
\pgfpathcurveto{\pgfqpoint{2.346410in}{2.450605in}}{\pgfqpoint{2.349682in}{2.458505in}}{\pgfqpoint{2.349682in}{2.466741in}}%
\pgfpathcurveto{\pgfqpoint{2.349682in}{2.474977in}}{\pgfqpoint{2.346410in}{2.482877in}}{\pgfqpoint{2.340586in}{2.488701in}}%
\pgfpathcurveto{\pgfqpoint{2.334762in}{2.494525in}}{\pgfqpoint{2.326862in}{2.497798in}}{\pgfqpoint{2.318626in}{2.497798in}}%
\pgfpathcurveto{\pgfqpoint{2.310390in}{2.497798in}}{\pgfqpoint{2.302490in}{2.494525in}}{\pgfqpoint{2.296666in}{2.488701in}}%
\pgfpathcurveto{\pgfqpoint{2.290842in}{2.482877in}}{\pgfqpoint{2.287569in}{2.474977in}}{\pgfqpoint{2.287569in}{2.466741in}}%
\pgfpathcurveto{\pgfqpoint{2.287569in}{2.458505in}}{\pgfqpoint{2.290842in}{2.450605in}}{\pgfqpoint{2.296666in}{2.444781in}}%
\pgfpathcurveto{\pgfqpoint{2.302490in}{2.438957in}}{\pgfqpoint{2.310390in}{2.435685in}}{\pgfqpoint{2.318626in}{2.435685in}}%
\pgfpathclose%
\pgfusepath{stroke,fill}%
\end{pgfscope}%
\begin{pgfscope}%
\pgfpathrectangle{\pgfqpoint{0.100000in}{0.212622in}}{\pgfqpoint{3.696000in}{3.696000in}}%
\pgfusepath{clip}%
\pgfsetbuttcap%
\pgfsetroundjoin%
\definecolor{currentfill}{rgb}{0.121569,0.466667,0.705882}%
\pgfsetfillcolor{currentfill}%
\pgfsetfillopacity{0.398162}%
\pgfsetlinewidth{1.003750pt}%
\definecolor{currentstroke}{rgb}{0.121569,0.466667,0.705882}%
\pgfsetstrokecolor{currentstroke}%
\pgfsetstrokeopacity{0.398162}%
\pgfsetdash{}{0pt}%
\pgfpathmoveto{\pgfqpoint{2.323038in}{2.434569in}}%
\pgfpathcurveto{\pgfqpoint{2.331274in}{2.434569in}}{\pgfqpoint{2.339174in}{2.437841in}}{\pgfqpoint{2.344998in}{2.443665in}}%
\pgfpathcurveto{\pgfqpoint{2.350822in}{2.449489in}}{\pgfqpoint{2.354094in}{2.457389in}}{\pgfqpoint{2.354094in}{2.465626in}}%
\pgfpathcurveto{\pgfqpoint{2.354094in}{2.473862in}}{\pgfqpoint{2.350822in}{2.481762in}}{\pgfqpoint{2.344998in}{2.487586in}}%
\pgfpathcurveto{\pgfqpoint{2.339174in}{2.493410in}}{\pgfqpoint{2.331274in}{2.496682in}}{\pgfqpoint{2.323038in}{2.496682in}}%
\pgfpathcurveto{\pgfqpoint{2.314801in}{2.496682in}}{\pgfqpoint{2.306901in}{2.493410in}}{\pgfqpoint{2.301077in}{2.487586in}}%
\pgfpathcurveto{\pgfqpoint{2.295253in}{2.481762in}}{\pgfqpoint{2.291981in}{2.473862in}}{\pgfqpoint{2.291981in}{2.465626in}}%
\pgfpathcurveto{\pgfqpoint{2.291981in}{2.457389in}}{\pgfqpoint{2.295253in}{2.449489in}}{\pgfqpoint{2.301077in}{2.443665in}}%
\pgfpathcurveto{\pgfqpoint{2.306901in}{2.437841in}}{\pgfqpoint{2.314801in}{2.434569in}}{\pgfqpoint{2.323038in}{2.434569in}}%
\pgfpathclose%
\pgfusepath{stroke,fill}%
\end{pgfscope}%
\begin{pgfscope}%
\pgfpathrectangle{\pgfqpoint{0.100000in}{0.212622in}}{\pgfqpoint{3.696000in}{3.696000in}}%
\pgfusepath{clip}%
\pgfsetbuttcap%
\pgfsetroundjoin%
\definecolor{currentfill}{rgb}{0.121569,0.466667,0.705882}%
\pgfsetfillcolor{currentfill}%
\pgfsetfillopacity{0.398883}%
\pgfsetlinewidth{1.003750pt}%
\definecolor{currentstroke}{rgb}{0.121569,0.466667,0.705882}%
\pgfsetstrokecolor{currentstroke}%
\pgfsetstrokeopacity{0.398883}%
\pgfsetdash{}{0pt}%
\pgfpathmoveto{\pgfqpoint{2.328113in}{2.432968in}}%
\pgfpathcurveto{\pgfqpoint{2.336349in}{2.432968in}}{\pgfqpoint{2.344249in}{2.436240in}}{\pgfqpoint{2.350073in}{2.442064in}}%
\pgfpathcurveto{\pgfqpoint{2.355897in}{2.447888in}}{\pgfqpoint{2.359170in}{2.455788in}}{\pgfqpoint{2.359170in}{2.464024in}}%
\pgfpathcurveto{\pgfqpoint{2.359170in}{2.472260in}}{\pgfqpoint{2.355897in}{2.480161in}}{\pgfqpoint{2.350073in}{2.485984in}}%
\pgfpathcurveto{\pgfqpoint{2.344249in}{2.491808in}}{\pgfqpoint{2.336349in}{2.495081in}}{\pgfqpoint{2.328113in}{2.495081in}}%
\pgfpathcurveto{\pgfqpoint{2.319877in}{2.495081in}}{\pgfqpoint{2.311977in}{2.491808in}}{\pgfqpoint{2.306153in}{2.485984in}}%
\pgfpathcurveto{\pgfqpoint{2.300329in}{2.480161in}}{\pgfqpoint{2.297057in}{2.472260in}}{\pgfqpoint{2.297057in}{2.464024in}}%
\pgfpathcurveto{\pgfqpoint{2.297057in}{2.455788in}}{\pgfqpoint{2.300329in}{2.447888in}}{\pgfqpoint{2.306153in}{2.442064in}}%
\pgfpathcurveto{\pgfqpoint{2.311977in}{2.436240in}}{\pgfqpoint{2.319877in}{2.432968in}}{\pgfqpoint{2.328113in}{2.432968in}}%
\pgfpathclose%
\pgfusepath{stroke,fill}%
\end{pgfscope}%
\begin{pgfscope}%
\pgfpathrectangle{\pgfqpoint{0.100000in}{0.212622in}}{\pgfqpoint{3.696000in}{3.696000in}}%
\pgfusepath{clip}%
\pgfsetbuttcap%
\pgfsetroundjoin%
\definecolor{currentfill}{rgb}{0.121569,0.466667,0.705882}%
\pgfsetfillcolor{currentfill}%
\pgfsetfillopacity{0.399395}%
\pgfsetlinewidth{1.003750pt}%
\definecolor{currentstroke}{rgb}{0.121569,0.466667,0.705882}%
\pgfsetstrokecolor{currentstroke}%
\pgfsetstrokeopacity{0.399395}%
\pgfsetdash{}{0pt}%
\pgfpathmoveto{\pgfqpoint{1.388426in}{2.203398in}}%
\pgfpathcurveto{\pgfqpoint{1.396662in}{2.203398in}}{\pgfqpoint{1.404562in}{2.206671in}}{\pgfqpoint{1.410386in}{2.212495in}}%
\pgfpathcurveto{\pgfqpoint{1.416210in}{2.218319in}}{\pgfqpoint{1.419482in}{2.226219in}}{\pgfqpoint{1.419482in}{2.234455in}}%
\pgfpathcurveto{\pgfqpoint{1.419482in}{2.242691in}}{\pgfqpoint{1.416210in}{2.250591in}}{\pgfqpoint{1.410386in}{2.256415in}}%
\pgfpathcurveto{\pgfqpoint{1.404562in}{2.262239in}}{\pgfqpoint{1.396662in}{2.265511in}}{\pgfqpoint{1.388426in}{2.265511in}}%
\pgfpathcurveto{\pgfqpoint{1.380189in}{2.265511in}}{\pgfqpoint{1.372289in}{2.262239in}}{\pgfqpoint{1.366465in}{2.256415in}}%
\pgfpathcurveto{\pgfqpoint{1.360641in}{2.250591in}}{\pgfqpoint{1.357369in}{2.242691in}}{\pgfqpoint{1.357369in}{2.234455in}}%
\pgfpathcurveto{\pgfqpoint{1.357369in}{2.226219in}}{\pgfqpoint{1.360641in}{2.218319in}}{\pgfqpoint{1.366465in}{2.212495in}}%
\pgfpathcurveto{\pgfqpoint{1.372289in}{2.206671in}}{\pgfqpoint{1.380189in}{2.203398in}}{\pgfqpoint{1.388426in}{2.203398in}}%
\pgfpathclose%
\pgfusepath{stroke,fill}%
\end{pgfscope}%
\begin{pgfscope}%
\pgfpathrectangle{\pgfqpoint{0.100000in}{0.212622in}}{\pgfqpoint{3.696000in}{3.696000in}}%
\pgfusepath{clip}%
\pgfsetbuttcap%
\pgfsetroundjoin%
\definecolor{currentfill}{rgb}{0.121569,0.466667,0.705882}%
\pgfsetfillcolor{currentfill}%
\pgfsetfillopacity{0.399627}%
\pgfsetlinewidth{1.003750pt}%
\definecolor{currentstroke}{rgb}{0.121569,0.466667,0.705882}%
\pgfsetstrokecolor{currentstroke}%
\pgfsetstrokeopacity{0.399627}%
\pgfsetdash{}{0pt}%
\pgfpathmoveto{\pgfqpoint{2.333654in}{2.430556in}}%
\pgfpathcurveto{\pgfqpoint{2.341890in}{2.430556in}}{\pgfqpoint{2.349790in}{2.433828in}}{\pgfqpoint{2.355614in}{2.439652in}}%
\pgfpathcurveto{\pgfqpoint{2.361438in}{2.445476in}}{\pgfqpoint{2.364710in}{2.453376in}}{\pgfqpoint{2.364710in}{2.461613in}}%
\pgfpathcurveto{\pgfqpoint{2.364710in}{2.469849in}}{\pgfqpoint{2.361438in}{2.477749in}}{\pgfqpoint{2.355614in}{2.483573in}}%
\pgfpathcurveto{\pgfqpoint{2.349790in}{2.489397in}}{\pgfqpoint{2.341890in}{2.492669in}}{\pgfqpoint{2.333654in}{2.492669in}}%
\pgfpathcurveto{\pgfqpoint{2.325418in}{2.492669in}}{\pgfqpoint{2.317518in}{2.489397in}}{\pgfqpoint{2.311694in}{2.483573in}}%
\pgfpathcurveto{\pgfqpoint{2.305870in}{2.477749in}}{\pgfqpoint{2.302597in}{2.469849in}}{\pgfqpoint{2.302597in}{2.461613in}}%
\pgfpathcurveto{\pgfqpoint{2.302597in}{2.453376in}}{\pgfqpoint{2.305870in}{2.445476in}}{\pgfqpoint{2.311694in}{2.439652in}}%
\pgfpathcurveto{\pgfqpoint{2.317518in}{2.433828in}}{\pgfqpoint{2.325418in}{2.430556in}}{\pgfqpoint{2.333654in}{2.430556in}}%
\pgfpathclose%
\pgfusepath{stroke,fill}%
\end{pgfscope}%
\begin{pgfscope}%
\pgfpathrectangle{\pgfqpoint{0.100000in}{0.212622in}}{\pgfqpoint{3.696000in}{3.696000in}}%
\pgfusepath{clip}%
\pgfsetbuttcap%
\pgfsetroundjoin%
\definecolor{currentfill}{rgb}{0.121569,0.466667,0.705882}%
\pgfsetfillcolor{currentfill}%
\pgfsetfillopacity{0.400543}%
\pgfsetlinewidth{1.003750pt}%
\definecolor{currentstroke}{rgb}{0.121569,0.466667,0.705882}%
\pgfsetstrokecolor{currentstroke}%
\pgfsetstrokeopacity{0.400543}%
\pgfsetdash{}{0pt}%
\pgfpathmoveto{\pgfqpoint{2.340067in}{2.428435in}}%
\pgfpathcurveto{\pgfqpoint{2.348303in}{2.428435in}}{\pgfqpoint{2.356203in}{2.431708in}}{\pgfqpoint{2.362027in}{2.437532in}}%
\pgfpathcurveto{\pgfqpoint{2.367851in}{2.443356in}}{\pgfqpoint{2.371123in}{2.451256in}}{\pgfqpoint{2.371123in}{2.459492in}}%
\pgfpathcurveto{\pgfqpoint{2.371123in}{2.467728in}}{\pgfqpoint{2.367851in}{2.475628in}}{\pgfqpoint{2.362027in}{2.481452in}}%
\pgfpathcurveto{\pgfqpoint{2.356203in}{2.487276in}}{\pgfqpoint{2.348303in}{2.490548in}}{\pgfqpoint{2.340067in}{2.490548in}}%
\pgfpathcurveto{\pgfqpoint{2.331830in}{2.490548in}}{\pgfqpoint{2.323930in}{2.487276in}}{\pgfqpoint{2.318106in}{2.481452in}}%
\pgfpathcurveto{\pgfqpoint{2.312282in}{2.475628in}}{\pgfqpoint{2.309010in}{2.467728in}}{\pgfqpoint{2.309010in}{2.459492in}}%
\pgfpathcurveto{\pgfqpoint{2.309010in}{2.451256in}}{\pgfqpoint{2.312282in}{2.443356in}}{\pgfqpoint{2.318106in}{2.437532in}}%
\pgfpathcurveto{\pgfqpoint{2.323930in}{2.431708in}}{\pgfqpoint{2.331830in}{2.428435in}}{\pgfqpoint{2.340067in}{2.428435in}}%
\pgfpathclose%
\pgfusepath{stroke,fill}%
\end{pgfscope}%
\begin{pgfscope}%
\pgfpathrectangle{\pgfqpoint{0.100000in}{0.212622in}}{\pgfqpoint{3.696000in}{3.696000in}}%
\pgfusepath{clip}%
\pgfsetbuttcap%
\pgfsetroundjoin%
\definecolor{currentfill}{rgb}{0.121569,0.466667,0.705882}%
\pgfsetfillcolor{currentfill}%
\pgfsetfillopacity{0.401080}%
\pgfsetlinewidth{1.003750pt}%
\definecolor{currentstroke}{rgb}{0.121569,0.466667,0.705882}%
\pgfsetstrokecolor{currentstroke}%
\pgfsetstrokeopacity{0.401080}%
\pgfsetdash{}{0pt}%
\pgfpathmoveto{\pgfqpoint{2.343620in}{2.427562in}}%
\pgfpathcurveto{\pgfqpoint{2.351856in}{2.427562in}}{\pgfqpoint{2.359756in}{2.430835in}}{\pgfqpoint{2.365580in}{2.436659in}}%
\pgfpathcurveto{\pgfqpoint{2.371404in}{2.442483in}}{\pgfqpoint{2.374676in}{2.450383in}}{\pgfqpoint{2.374676in}{2.458619in}}%
\pgfpathcurveto{\pgfqpoint{2.374676in}{2.466855in}}{\pgfqpoint{2.371404in}{2.474755in}}{\pgfqpoint{2.365580in}{2.480579in}}%
\pgfpathcurveto{\pgfqpoint{2.359756in}{2.486403in}}{\pgfqpoint{2.351856in}{2.489675in}}{\pgfqpoint{2.343620in}{2.489675in}}%
\pgfpathcurveto{\pgfqpoint{2.335383in}{2.489675in}}{\pgfqpoint{2.327483in}{2.486403in}}{\pgfqpoint{2.321659in}{2.480579in}}%
\pgfpathcurveto{\pgfqpoint{2.315835in}{2.474755in}}{\pgfqpoint{2.312563in}{2.466855in}}{\pgfqpoint{2.312563in}{2.458619in}}%
\pgfpathcurveto{\pgfqpoint{2.312563in}{2.450383in}}{\pgfqpoint{2.315835in}{2.442483in}}{\pgfqpoint{2.321659in}{2.436659in}}%
\pgfpathcurveto{\pgfqpoint{2.327483in}{2.430835in}}{\pgfqpoint{2.335383in}{2.427562in}}{\pgfqpoint{2.343620in}{2.427562in}}%
\pgfpathclose%
\pgfusepath{stroke,fill}%
\end{pgfscope}%
\begin{pgfscope}%
\pgfpathrectangle{\pgfqpoint{0.100000in}{0.212622in}}{\pgfqpoint{3.696000in}{3.696000in}}%
\pgfusepath{clip}%
\pgfsetbuttcap%
\pgfsetroundjoin%
\definecolor{currentfill}{rgb}{0.121569,0.466667,0.705882}%
\pgfsetfillcolor{currentfill}%
\pgfsetfillopacity{0.401685}%
\pgfsetlinewidth{1.003750pt}%
\definecolor{currentstroke}{rgb}{0.121569,0.466667,0.705882}%
\pgfsetstrokecolor{currentstroke}%
\pgfsetstrokeopacity{0.401685}%
\pgfsetdash{}{0pt}%
\pgfpathmoveto{\pgfqpoint{2.347875in}{2.426134in}}%
\pgfpathcurveto{\pgfqpoint{2.356111in}{2.426134in}}{\pgfqpoint{2.364011in}{2.429406in}}{\pgfqpoint{2.369835in}{2.435230in}}%
\pgfpathcurveto{\pgfqpoint{2.375659in}{2.441054in}}{\pgfqpoint{2.378932in}{2.448954in}}{\pgfqpoint{2.378932in}{2.457190in}}%
\pgfpathcurveto{\pgfqpoint{2.378932in}{2.465426in}}{\pgfqpoint{2.375659in}{2.473326in}}{\pgfqpoint{2.369835in}{2.479150in}}%
\pgfpathcurveto{\pgfqpoint{2.364011in}{2.484974in}}{\pgfqpoint{2.356111in}{2.488247in}}{\pgfqpoint{2.347875in}{2.488247in}}%
\pgfpathcurveto{\pgfqpoint{2.339639in}{2.488247in}}{\pgfqpoint{2.331739in}{2.484974in}}{\pgfqpoint{2.325915in}{2.479150in}}%
\pgfpathcurveto{\pgfqpoint{2.320091in}{2.473326in}}{\pgfqpoint{2.316819in}{2.465426in}}{\pgfqpoint{2.316819in}{2.457190in}}%
\pgfpathcurveto{\pgfqpoint{2.316819in}{2.448954in}}{\pgfqpoint{2.320091in}{2.441054in}}{\pgfqpoint{2.325915in}{2.435230in}}%
\pgfpathcurveto{\pgfqpoint{2.331739in}{2.429406in}}{\pgfqpoint{2.339639in}{2.426134in}}{\pgfqpoint{2.347875in}{2.426134in}}%
\pgfpathclose%
\pgfusepath{stroke,fill}%
\end{pgfscope}%
\begin{pgfscope}%
\pgfpathrectangle{\pgfqpoint{0.100000in}{0.212622in}}{\pgfqpoint{3.696000in}{3.696000in}}%
\pgfusepath{clip}%
\pgfsetbuttcap%
\pgfsetroundjoin%
\definecolor{currentfill}{rgb}{0.121569,0.466667,0.705882}%
\pgfsetfillcolor{currentfill}%
\pgfsetfillopacity{0.402035}%
\pgfsetlinewidth{1.003750pt}%
\definecolor{currentstroke}{rgb}{0.121569,0.466667,0.705882}%
\pgfsetstrokecolor{currentstroke}%
\pgfsetstrokeopacity{0.402035}%
\pgfsetdash{}{0pt}%
\pgfpathmoveto{\pgfqpoint{1.380586in}{2.193069in}}%
\pgfpathcurveto{\pgfqpoint{1.388823in}{2.193069in}}{\pgfqpoint{1.396723in}{2.196341in}}{\pgfqpoint{1.402547in}{2.202165in}}%
\pgfpathcurveto{\pgfqpoint{1.408370in}{2.207989in}}{\pgfqpoint{1.411643in}{2.215889in}}{\pgfqpoint{1.411643in}{2.224126in}}%
\pgfpathcurveto{\pgfqpoint{1.411643in}{2.232362in}}{\pgfqpoint{1.408370in}{2.240262in}}{\pgfqpoint{1.402547in}{2.246086in}}%
\pgfpathcurveto{\pgfqpoint{1.396723in}{2.251910in}}{\pgfqpoint{1.388823in}{2.255182in}}{\pgfqpoint{1.380586in}{2.255182in}}%
\pgfpathcurveto{\pgfqpoint{1.372350in}{2.255182in}}{\pgfqpoint{1.364450in}{2.251910in}}{\pgfqpoint{1.358626in}{2.246086in}}%
\pgfpathcurveto{\pgfqpoint{1.352802in}{2.240262in}}{\pgfqpoint{1.349530in}{2.232362in}}{\pgfqpoint{1.349530in}{2.224126in}}%
\pgfpathcurveto{\pgfqpoint{1.349530in}{2.215889in}}{\pgfqpoint{1.352802in}{2.207989in}}{\pgfqpoint{1.358626in}{2.202165in}}%
\pgfpathcurveto{\pgfqpoint{1.364450in}{2.196341in}}{\pgfqpoint{1.372350in}{2.193069in}}{\pgfqpoint{1.380586in}{2.193069in}}%
\pgfpathclose%
\pgfusepath{stroke,fill}%
\end{pgfscope}%
\begin{pgfscope}%
\pgfpathrectangle{\pgfqpoint{0.100000in}{0.212622in}}{\pgfqpoint{3.696000in}{3.696000in}}%
\pgfusepath{clip}%
\pgfsetbuttcap%
\pgfsetroundjoin%
\definecolor{currentfill}{rgb}{0.121569,0.466667,0.705882}%
\pgfsetfillcolor{currentfill}%
\pgfsetfillopacity{0.402768}%
\pgfsetlinewidth{1.003750pt}%
\definecolor{currentstroke}{rgb}{0.121569,0.466667,0.705882}%
\pgfsetstrokecolor{currentstroke}%
\pgfsetstrokeopacity{0.402768}%
\pgfsetdash{}{0pt}%
\pgfpathmoveto{\pgfqpoint{2.354631in}{2.424846in}}%
\pgfpathcurveto{\pgfqpoint{2.362867in}{2.424846in}}{\pgfqpoint{2.370767in}{2.428119in}}{\pgfqpoint{2.376591in}{2.433943in}}%
\pgfpathcurveto{\pgfqpoint{2.382415in}{2.439767in}}{\pgfqpoint{2.385688in}{2.447667in}}{\pgfqpoint{2.385688in}{2.455903in}}%
\pgfpathcurveto{\pgfqpoint{2.385688in}{2.464139in}}{\pgfqpoint{2.382415in}{2.472039in}}{\pgfqpoint{2.376591in}{2.477863in}}%
\pgfpathcurveto{\pgfqpoint{2.370767in}{2.483687in}}{\pgfqpoint{2.362867in}{2.486959in}}{\pgfqpoint{2.354631in}{2.486959in}}%
\pgfpathcurveto{\pgfqpoint{2.346395in}{2.486959in}}{\pgfqpoint{2.338495in}{2.483687in}}{\pgfqpoint{2.332671in}{2.477863in}}%
\pgfpathcurveto{\pgfqpoint{2.326847in}{2.472039in}}{\pgfqpoint{2.323575in}{2.464139in}}{\pgfqpoint{2.323575in}{2.455903in}}%
\pgfpathcurveto{\pgfqpoint{2.323575in}{2.447667in}}{\pgfqpoint{2.326847in}{2.439767in}}{\pgfqpoint{2.332671in}{2.433943in}}%
\pgfpathcurveto{\pgfqpoint{2.338495in}{2.428119in}}{\pgfqpoint{2.346395in}{2.424846in}}{\pgfqpoint{2.354631in}{2.424846in}}%
\pgfpathclose%
\pgfusepath{stroke,fill}%
\end{pgfscope}%
\begin{pgfscope}%
\pgfpathrectangle{\pgfqpoint{0.100000in}{0.212622in}}{\pgfqpoint{3.696000in}{3.696000in}}%
\pgfusepath{clip}%
\pgfsetbuttcap%
\pgfsetroundjoin%
\definecolor{currentfill}{rgb}{0.121569,0.466667,0.705882}%
\pgfsetfillcolor{currentfill}%
\pgfsetfillopacity{0.403936}%
\pgfsetlinewidth{1.003750pt}%
\definecolor{currentstroke}{rgb}{0.121569,0.466667,0.705882}%
\pgfsetstrokecolor{currentstroke}%
\pgfsetstrokeopacity{0.403936}%
\pgfsetdash{}{0pt}%
\pgfpathmoveto{\pgfqpoint{2.361945in}{2.423269in}}%
\pgfpathcurveto{\pgfqpoint{2.370181in}{2.423269in}}{\pgfqpoint{2.378081in}{2.426542in}}{\pgfqpoint{2.383905in}{2.432366in}}%
\pgfpathcurveto{\pgfqpoint{2.389729in}{2.438190in}}{\pgfqpoint{2.393001in}{2.446090in}}{\pgfqpoint{2.393001in}{2.454326in}}%
\pgfpathcurveto{\pgfqpoint{2.393001in}{2.462562in}}{\pgfqpoint{2.389729in}{2.470462in}}{\pgfqpoint{2.383905in}{2.476286in}}%
\pgfpathcurveto{\pgfqpoint{2.378081in}{2.482110in}}{\pgfqpoint{2.370181in}{2.485382in}}{\pgfqpoint{2.361945in}{2.485382in}}%
\pgfpathcurveto{\pgfqpoint{2.353709in}{2.485382in}}{\pgfqpoint{2.345809in}{2.482110in}}{\pgfqpoint{2.339985in}{2.476286in}}%
\pgfpathcurveto{\pgfqpoint{2.334161in}{2.470462in}}{\pgfqpoint{2.330888in}{2.462562in}}{\pgfqpoint{2.330888in}{2.454326in}}%
\pgfpathcurveto{\pgfqpoint{2.330888in}{2.446090in}}{\pgfqpoint{2.334161in}{2.438190in}}{\pgfqpoint{2.339985in}{2.432366in}}%
\pgfpathcurveto{\pgfqpoint{2.345809in}{2.426542in}}{\pgfqpoint{2.353709in}{2.423269in}}{\pgfqpoint{2.361945in}{2.423269in}}%
\pgfpathclose%
\pgfusepath{stroke,fill}%
\end{pgfscope}%
\begin{pgfscope}%
\pgfpathrectangle{\pgfqpoint{0.100000in}{0.212622in}}{\pgfqpoint{3.696000in}{3.696000in}}%
\pgfusepath{clip}%
\pgfsetbuttcap%
\pgfsetroundjoin%
\definecolor{currentfill}{rgb}{0.121569,0.466667,0.705882}%
\pgfsetfillcolor{currentfill}%
\pgfsetfillopacity{0.404430}%
\pgfsetlinewidth{1.003750pt}%
\definecolor{currentstroke}{rgb}{0.121569,0.466667,0.705882}%
\pgfsetstrokecolor{currentstroke}%
\pgfsetstrokeopacity{0.404430}%
\pgfsetdash{}{0pt}%
\pgfpathmoveto{\pgfqpoint{1.373987in}{2.183912in}}%
\pgfpathcurveto{\pgfqpoint{1.382223in}{2.183912in}}{\pgfqpoint{1.390123in}{2.187185in}}{\pgfqpoint{1.395947in}{2.193009in}}%
\pgfpathcurveto{\pgfqpoint{1.401771in}{2.198833in}}{\pgfqpoint{1.405043in}{2.206733in}}{\pgfqpoint{1.405043in}{2.214969in}}%
\pgfpathcurveto{\pgfqpoint{1.405043in}{2.223205in}}{\pgfqpoint{1.401771in}{2.231105in}}{\pgfqpoint{1.395947in}{2.236929in}}%
\pgfpathcurveto{\pgfqpoint{1.390123in}{2.242753in}}{\pgfqpoint{1.382223in}{2.246025in}}{\pgfqpoint{1.373987in}{2.246025in}}%
\pgfpathcurveto{\pgfqpoint{1.365750in}{2.246025in}}{\pgfqpoint{1.357850in}{2.242753in}}{\pgfqpoint{1.352026in}{2.236929in}}%
\pgfpathcurveto{\pgfqpoint{1.346202in}{2.231105in}}{\pgfqpoint{1.342930in}{2.223205in}}{\pgfqpoint{1.342930in}{2.214969in}}%
\pgfpathcurveto{\pgfqpoint{1.342930in}{2.206733in}}{\pgfqpoint{1.346202in}{2.198833in}}{\pgfqpoint{1.352026in}{2.193009in}}%
\pgfpathcurveto{\pgfqpoint{1.357850in}{2.187185in}}{\pgfqpoint{1.365750in}{2.183912in}}{\pgfqpoint{1.373987in}{2.183912in}}%
\pgfpathclose%
\pgfusepath{stroke,fill}%
\end{pgfscope}%
\begin{pgfscope}%
\pgfpathrectangle{\pgfqpoint{0.100000in}{0.212622in}}{\pgfqpoint{3.696000in}{3.696000in}}%
\pgfusepath{clip}%
\pgfsetbuttcap%
\pgfsetroundjoin%
\definecolor{currentfill}{rgb}{0.121569,0.466667,0.705882}%
\pgfsetfillcolor{currentfill}%
\pgfsetfillopacity{0.404604}%
\pgfsetlinewidth{1.003750pt}%
\definecolor{currentstroke}{rgb}{0.121569,0.466667,0.705882}%
\pgfsetstrokecolor{currentstroke}%
\pgfsetstrokeopacity{0.404604}%
\pgfsetdash{}{0pt}%
\pgfpathmoveto{\pgfqpoint{2.365957in}{2.422536in}}%
\pgfpathcurveto{\pgfqpoint{2.374194in}{2.422536in}}{\pgfqpoint{2.382094in}{2.425808in}}{\pgfqpoint{2.387918in}{2.431632in}}%
\pgfpathcurveto{\pgfqpoint{2.393742in}{2.437456in}}{\pgfqpoint{2.397014in}{2.445356in}}{\pgfqpoint{2.397014in}{2.453592in}}%
\pgfpathcurveto{\pgfqpoint{2.397014in}{2.461828in}}{\pgfqpoint{2.393742in}{2.469729in}}{\pgfqpoint{2.387918in}{2.475552in}}%
\pgfpathcurveto{\pgfqpoint{2.382094in}{2.481376in}}{\pgfqpoint{2.374194in}{2.484649in}}{\pgfqpoint{2.365957in}{2.484649in}}%
\pgfpathcurveto{\pgfqpoint{2.357721in}{2.484649in}}{\pgfqpoint{2.349821in}{2.481376in}}{\pgfqpoint{2.343997in}{2.475552in}}%
\pgfpathcurveto{\pgfqpoint{2.338173in}{2.469729in}}{\pgfqpoint{2.334901in}{2.461828in}}{\pgfqpoint{2.334901in}{2.453592in}}%
\pgfpathcurveto{\pgfqpoint{2.334901in}{2.445356in}}{\pgfqpoint{2.338173in}{2.437456in}}{\pgfqpoint{2.343997in}{2.431632in}}%
\pgfpathcurveto{\pgfqpoint{2.349821in}{2.425808in}}{\pgfqpoint{2.357721in}{2.422536in}}{\pgfqpoint{2.365957in}{2.422536in}}%
\pgfpathclose%
\pgfusepath{stroke,fill}%
\end{pgfscope}%
\begin{pgfscope}%
\pgfpathrectangle{\pgfqpoint{0.100000in}{0.212622in}}{\pgfqpoint{3.696000in}{3.696000in}}%
\pgfusepath{clip}%
\pgfsetbuttcap%
\pgfsetroundjoin%
\definecolor{currentfill}{rgb}{0.121569,0.466667,0.705882}%
\pgfsetfillcolor{currentfill}%
\pgfsetfillopacity{0.405390}%
\pgfsetlinewidth{1.003750pt}%
\definecolor{currentstroke}{rgb}{0.121569,0.466667,0.705882}%
\pgfsetstrokecolor{currentstroke}%
\pgfsetstrokeopacity{0.405390}%
\pgfsetdash{}{0pt}%
\pgfpathmoveto{\pgfqpoint{2.370894in}{2.421201in}}%
\pgfpathcurveto{\pgfqpoint{2.379130in}{2.421201in}}{\pgfqpoint{2.387030in}{2.424474in}}{\pgfqpoint{2.392854in}{2.430298in}}%
\pgfpathcurveto{\pgfqpoint{2.398678in}{2.436121in}}{\pgfqpoint{2.401950in}{2.444022in}}{\pgfqpoint{2.401950in}{2.452258in}}%
\pgfpathcurveto{\pgfqpoint{2.401950in}{2.460494in}}{\pgfqpoint{2.398678in}{2.468394in}}{\pgfqpoint{2.392854in}{2.474218in}}%
\pgfpathcurveto{\pgfqpoint{2.387030in}{2.480042in}}{\pgfqpoint{2.379130in}{2.483314in}}{\pgfqpoint{2.370894in}{2.483314in}}%
\pgfpathcurveto{\pgfqpoint{2.362657in}{2.483314in}}{\pgfqpoint{2.354757in}{2.480042in}}{\pgfqpoint{2.348933in}{2.474218in}}%
\pgfpathcurveto{\pgfqpoint{2.343109in}{2.468394in}}{\pgfqpoint{2.339837in}{2.460494in}}{\pgfqpoint{2.339837in}{2.452258in}}%
\pgfpathcurveto{\pgfqpoint{2.339837in}{2.444022in}}{\pgfqpoint{2.343109in}{2.436121in}}{\pgfqpoint{2.348933in}{2.430298in}}%
\pgfpathcurveto{\pgfqpoint{2.354757in}{2.424474in}}{\pgfqpoint{2.362657in}{2.421201in}}{\pgfqpoint{2.370894in}{2.421201in}}%
\pgfpathclose%
\pgfusepath{stroke,fill}%
\end{pgfscope}%
\begin{pgfscope}%
\pgfpathrectangle{\pgfqpoint{0.100000in}{0.212622in}}{\pgfqpoint{3.696000in}{3.696000in}}%
\pgfusepath{clip}%
\pgfsetbuttcap%
\pgfsetroundjoin%
\definecolor{currentfill}{rgb}{0.121569,0.466667,0.705882}%
\pgfsetfillcolor{currentfill}%
\pgfsetfillopacity{0.405881}%
\pgfsetlinewidth{1.003750pt}%
\definecolor{currentstroke}{rgb}{0.121569,0.466667,0.705882}%
\pgfsetstrokecolor{currentstroke}%
\pgfsetstrokeopacity{0.405881}%
\pgfsetdash{}{0pt}%
\pgfpathmoveto{\pgfqpoint{2.373660in}{2.421006in}}%
\pgfpathcurveto{\pgfqpoint{2.381897in}{2.421006in}}{\pgfqpoint{2.389797in}{2.424278in}}{\pgfqpoint{2.395621in}{2.430102in}}%
\pgfpathcurveto{\pgfqpoint{2.401445in}{2.435926in}}{\pgfqpoint{2.404717in}{2.443826in}}{\pgfqpoint{2.404717in}{2.452062in}}%
\pgfpathcurveto{\pgfqpoint{2.404717in}{2.460299in}}{\pgfqpoint{2.401445in}{2.468199in}}{\pgfqpoint{2.395621in}{2.474023in}}%
\pgfpathcurveto{\pgfqpoint{2.389797in}{2.479847in}}{\pgfqpoint{2.381897in}{2.483119in}}{\pgfqpoint{2.373660in}{2.483119in}}%
\pgfpathcurveto{\pgfqpoint{2.365424in}{2.483119in}}{\pgfqpoint{2.357524in}{2.479847in}}{\pgfqpoint{2.351700in}{2.474023in}}%
\pgfpathcurveto{\pgfqpoint{2.345876in}{2.468199in}}{\pgfqpoint{2.342604in}{2.460299in}}{\pgfqpoint{2.342604in}{2.452062in}}%
\pgfpathcurveto{\pgfqpoint{2.342604in}{2.443826in}}{\pgfqpoint{2.345876in}{2.435926in}}{\pgfqpoint{2.351700in}{2.430102in}}%
\pgfpathcurveto{\pgfqpoint{2.357524in}{2.424278in}}{\pgfqpoint{2.365424in}{2.421006in}}{\pgfqpoint{2.373660in}{2.421006in}}%
\pgfpathclose%
\pgfusepath{stroke,fill}%
\end{pgfscope}%
\begin{pgfscope}%
\pgfpathrectangle{\pgfqpoint{0.100000in}{0.212622in}}{\pgfqpoint{3.696000in}{3.696000in}}%
\pgfusepath{clip}%
\pgfsetbuttcap%
\pgfsetroundjoin%
\definecolor{currentfill}{rgb}{0.121569,0.466667,0.705882}%
\pgfsetfillcolor{currentfill}%
\pgfsetfillopacity{0.406492}%
\pgfsetlinewidth{1.003750pt}%
\definecolor{currentstroke}{rgb}{0.121569,0.466667,0.705882}%
\pgfsetstrokecolor{currentstroke}%
\pgfsetstrokeopacity{0.406492}%
\pgfsetdash{}{0pt}%
\pgfpathmoveto{\pgfqpoint{1.368014in}{2.175901in}}%
\pgfpathcurveto{\pgfqpoint{1.376250in}{2.175901in}}{\pgfqpoint{1.384150in}{2.179173in}}{\pgfqpoint{1.389974in}{2.184997in}}%
\pgfpathcurveto{\pgfqpoint{1.395798in}{2.190821in}}{\pgfqpoint{1.399071in}{2.198721in}}{\pgfqpoint{1.399071in}{2.206958in}}%
\pgfpathcurveto{\pgfqpoint{1.399071in}{2.215194in}}{\pgfqpoint{1.395798in}{2.223094in}}{\pgfqpoint{1.389974in}{2.228918in}}%
\pgfpathcurveto{\pgfqpoint{1.384150in}{2.234742in}}{\pgfqpoint{1.376250in}{2.238014in}}{\pgfqpoint{1.368014in}{2.238014in}}%
\pgfpathcurveto{\pgfqpoint{1.359778in}{2.238014in}}{\pgfqpoint{1.351878in}{2.234742in}}{\pgfqpoint{1.346054in}{2.228918in}}%
\pgfpathcurveto{\pgfqpoint{1.340230in}{2.223094in}}{\pgfqpoint{1.336958in}{2.215194in}}{\pgfqpoint{1.336958in}{2.206958in}}%
\pgfpathcurveto{\pgfqpoint{1.336958in}{2.198721in}}{\pgfqpoint{1.340230in}{2.190821in}}{\pgfqpoint{1.346054in}{2.184997in}}%
\pgfpathcurveto{\pgfqpoint{1.351878in}{2.179173in}}{\pgfqpoint{1.359778in}{2.175901in}}{\pgfqpoint{1.368014in}{2.175901in}}%
\pgfpathclose%
\pgfusepath{stroke,fill}%
\end{pgfscope}%
\begin{pgfscope}%
\pgfpathrectangle{\pgfqpoint{0.100000in}{0.212622in}}{\pgfqpoint{3.696000in}{3.696000in}}%
\pgfusepath{clip}%
\pgfsetbuttcap%
\pgfsetroundjoin%
\definecolor{currentfill}{rgb}{0.121569,0.466667,0.705882}%
\pgfsetfillcolor{currentfill}%
\pgfsetfillopacity{0.406547}%
\pgfsetlinewidth{1.003750pt}%
\definecolor{currentstroke}{rgb}{0.121569,0.466667,0.705882}%
\pgfsetstrokecolor{currentstroke}%
\pgfsetstrokeopacity{0.406547}%
\pgfsetdash{}{0pt}%
\pgfpathmoveto{\pgfqpoint{2.377114in}{2.421002in}}%
\pgfpathcurveto{\pgfqpoint{2.385350in}{2.421002in}}{\pgfqpoint{2.393250in}{2.424274in}}{\pgfqpoint{2.399074in}{2.430098in}}%
\pgfpathcurveto{\pgfqpoint{2.404898in}{2.435922in}}{\pgfqpoint{2.408170in}{2.443822in}}{\pgfqpoint{2.408170in}{2.452058in}}%
\pgfpathcurveto{\pgfqpoint{2.408170in}{2.460294in}}{\pgfqpoint{2.404898in}{2.468194in}}{\pgfqpoint{2.399074in}{2.474018in}}%
\pgfpathcurveto{\pgfqpoint{2.393250in}{2.479842in}}{\pgfqpoint{2.385350in}{2.483115in}}{\pgfqpoint{2.377114in}{2.483115in}}%
\pgfpathcurveto{\pgfqpoint{2.368877in}{2.483115in}}{\pgfqpoint{2.360977in}{2.479842in}}{\pgfqpoint{2.355153in}{2.474018in}}%
\pgfpathcurveto{\pgfqpoint{2.349329in}{2.468194in}}{\pgfqpoint{2.346057in}{2.460294in}}{\pgfqpoint{2.346057in}{2.452058in}}%
\pgfpathcurveto{\pgfqpoint{2.346057in}{2.443822in}}{\pgfqpoint{2.349329in}{2.435922in}}{\pgfqpoint{2.355153in}{2.430098in}}%
\pgfpathcurveto{\pgfqpoint{2.360977in}{2.424274in}}{\pgfqpoint{2.368877in}{2.421002in}}{\pgfqpoint{2.377114in}{2.421002in}}%
\pgfpathclose%
\pgfusepath{stroke,fill}%
\end{pgfscope}%
\begin{pgfscope}%
\pgfpathrectangle{\pgfqpoint{0.100000in}{0.212622in}}{\pgfqpoint{3.696000in}{3.696000in}}%
\pgfusepath{clip}%
\pgfsetbuttcap%
\pgfsetroundjoin%
\definecolor{currentfill}{rgb}{0.121569,0.466667,0.705882}%
\pgfsetfillcolor{currentfill}%
\pgfsetfillopacity{0.406896}%
\pgfsetlinewidth{1.003750pt}%
\definecolor{currentstroke}{rgb}{0.121569,0.466667,0.705882}%
\pgfsetstrokecolor{currentstroke}%
\pgfsetstrokeopacity{0.406896}%
\pgfsetdash{}{0pt}%
\pgfpathmoveto{\pgfqpoint{2.378984in}{2.420797in}}%
\pgfpathcurveto{\pgfqpoint{2.387220in}{2.420797in}}{\pgfqpoint{2.395120in}{2.424069in}}{\pgfqpoint{2.400944in}{2.429893in}}%
\pgfpathcurveto{\pgfqpoint{2.406768in}{2.435717in}}{\pgfqpoint{2.410040in}{2.443617in}}{\pgfqpoint{2.410040in}{2.451853in}}%
\pgfpathcurveto{\pgfqpoint{2.410040in}{2.460089in}}{\pgfqpoint{2.406768in}{2.467989in}}{\pgfqpoint{2.400944in}{2.473813in}}%
\pgfpathcurveto{\pgfqpoint{2.395120in}{2.479637in}}{\pgfqpoint{2.387220in}{2.482910in}}{\pgfqpoint{2.378984in}{2.482910in}}%
\pgfpathcurveto{\pgfqpoint{2.370747in}{2.482910in}}{\pgfqpoint{2.362847in}{2.479637in}}{\pgfqpoint{2.357023in}{2.473813in}}%
\pgfpathcurveto{\pgfqpoint{2.351199in}{2.467989in}}{\pgfqpoint{2.347927in}{2.460089in}}{\pgfqpoint{2.347927in}{2.451853in}}%
\pgfpathcurveto{\pgfqpoint{2.347927in}{2.443617in}}{\pgfqpoint{2.351199in}{2.435717in}}{\pgfqpoint{2.357023in}{2.429893in}}%
\pgfpathcurveto{\pgfqpoint{2.362847in}{2.424069in}}{\pgfqpoint{2.370747in}{2.420797in}}{\pgfqpoint{2.378984in}{2.420797in}}%
\pgfpathclose%
\pgfusepath{stroke,fill}%
\end{pgfscope}%
\begin{pgfscope}%
\pgfpathrectangle{\pgfqpoint{0.100000in}{0.212622in}}{\pgfqpoint{3.696000in}{3.696000in}}%
\pgfusepath{clip}%
\pgfsetbuttcap%
\pgfsetroundjoin%
\definecolor{currentfill}{rgb}{0.121569,0.466667,0.705882}%
\pgfsetfillcolor{currentfill}%
\pgfsetfillopacity{0.407376}%
\pgfsetlinewidth{1.003750pt}%
\definecolor{currentstroke}{rgb}{0.121569,0.466667,0.705882}%
\pgfsetstrokecolor{currentstroke}%
\pgfsetstrokeopacity{0.407376}%
\pgfsetdash{}{0pt}%
\pgfpathmoveto{\pgfqpoint{2.381624in}{2.420230in}}%
\pgfpathcurveto{\pgfqpoint{2.389861in}{2.420230in}}{\pgfqpoint{2.397761in}{2.423502in}}{\pgfqpoint{2.403585in}{2.429326in}}%
\pgfpathcurveto{\pgfqpoint{2.409409in}{2.435150in}}{\pgfqpoint{2.412681in}{2.443050in}}{\pgfqpoint{2.412681in}{2.451287in}}%
\pgfpathcurveto{\pgfqpoint{2.412681in}{2.459523in}}{\pgfqpoint{2.409409in}{2.467423in}}{\pgfqpoint{2.403585in}{2.473247in}}%
\pgfpathcurveto{\pgfqpoint{2.397761in}{2.479071in}}{\pgfqpoint{2.389861in}{2.482343in}}{\pgfqpoint{2.381624in}{2.482343in}}%
\pgfpathcurveto{\pgfqpoint{2.373388in}{2.482343in}}{\pgfqpoint{2.365488in}{2.479071in}}{\pgfqpoint{2.359664in}{2.473247in}}%
\pgfpathcurveto{\pgfqpoint{2.353840in}{2.467423in}}{\pgfqpoint{2.350568in}{2.459523in}}{\pgfqpoint{2.350568in}{2.451287in}}%
\pgfpathcurveto{\pgfqpoint{2.350568in}{2.443050in}}{\pgfqpoint{2.353840in}{2.435150in}}{\pgfqpoint{2.359664in}{2.429326in}}%
\pgfpathcurveto{\pgfqpoint{2.365488in}{2.423502in}}{\pgfqpoint{2.373388in}{2.420230in}}{\pgfqpoint{2.381624in}{2.420230in}}%
\pgfpathclose%
\pgfusepath{stroke,fill}%
\end{pgfscope}%
\begin{pgfscope}%
\pgfpathrectangle{\pgfqpoint{0.100000in}{0.212622in}}{\pgfqpoint{3.696000in}{3.696000in}}%
\pgfusepath{clip}%
\pgfsetbuttcap%
\pgfsetroundjoin%
\definecolor{currentfill}{rgb}{0.121569,0.466667,0.705882}%
\pgfsetfillcolor{currentfill}%
\pgfsetfillopacity{0.407956}%
\pgfsetlinewidth{1.003750pt}%
\definecolor{currentstroke}{rgb}{0.121569,0.466667,0.705882}%
\pgfsetstrokecolor{currentstroke}%
\pgfsetstrokeopacity{0.407956}%
\pgfsetdash{}{0pt}%
\pgfpathmoveto{\pgfqpoint{2.385096in}{2.419411in}}%
\pgfpathcurveto{\pgfqpoint{2.393332in}{2.419411in}}{\pgfqpoint{2.401233in}{2.422684in}}{\pgfqpoint{2.407056in}{2.428507in}}%
\pgfpathcurveto{\pgfqpoint{2.412880in}{2.434331in}}{\pgfqpoint{2.416153in}{2.442231in}}{\pgfqpoint{2.416153in}{2.450468in}}%
\pgfpathcurveto{\pgfqpoint{2.416153in}{2.458704in}}{\pgfqpoint{2.412880in}{2.466604in}}{\pgfqpoint{2.407056in}{2.472428in}}%
\pgfpathcurveto{\pgfqpoint{2.401233in}{2.478252in}}{\pgfqpoint{2.393332in}{2.481524in}}{\pgfqpoint{2.385096in}{2.481524in}}%
\pgfpathcurveto{\pgfqpoint{2.376860in}{2.481524in}}{\pgfqpoint{2.368960in}{2.478252in}}{\pgfqpoint{2.363136in}{2.472428in}}%
\pgfpathcurveto{\pgfqpoint{2.357312in}{2.466604in}}{\pgfqpoint{2.354040in}{2.458704in}}{\pgfqpoint{2.354040in}{2.450468in}}%
\pgfpathcurveto{\pgfqpoint{2.354040in}{2.442231in}}{\pgfqpoint{2.357312in}{2.434331in}}{\pgfqpoint{2.363136in}{2.428507in}}%
\pgfpathcurveto{\pgfqpoint{2.368960in}{2.422684in}}{\pgfqpoint{2.376860in}{2.419411in}}{\pgfqpoint{2.385096in}{2.419411in}}%
\pgfpathclose%
\pgfusepath{stroke,fill}%
\end{pgfscope}%
\begin{pgfscope}%
\pgfpathrectangle{\pgfqpoint{0.100000in}{0.212622in}}{\pgfqpoint{3.696000in}{3.696000in}}%
\pgfusepath{clip}%
\pgfsetbuttcap%
\pgfsetroundjoin%
\definecolor{currentfill}{rgb}{0.121569,0.466667,0.705882}%
\pgfsetfillcolor{currentfill}%
\pgfsetfillopacity{0.408246}%
\pgfsetlinewidth{1.003750pt}%
\definecolor{currentstroke}{rgb}{0.121569,0.466667,0.705882}%
\pgfsetstrokecolor{currentstroke}%
\pgfsetstrokeopacity{0.408246}%
\pgfsetdash{}{0pt}%
\pgfpathmoveto{\pgfqpoint{1.363218in}{2.169324in}}%
\pgfpathcurveto{\pgfqpoint{1.371455in}{2.169324in}}{\pgfqpoint{1.379355in}{2.172596in}}{\pgfqpoint{1.385179in}{2.178420in}}%
\pgfpathcurveto{\pgfqpoint{1.391003in}{2.184244in}}{\pgfqpoint{1.394275in}{2.192144in}}{\pgfqpoint{1.394275in}{2.200380in}}%
\pgfpathcurveto{\pgfqpoint{1.394275in}{2.208617in}}{\pgfqpoint{1.391003in}{2.216517in}}{\pgfqpoint{1.385179in}{2.222341in}}%
\pgfpathcurveto{\pgfqpoint{1.379355in}{2.228165in}}{\pgfqpoint{1.371455in}{2.231437in}}{\pgfqpoint{1.363218in}{2.231437in}}%
\pgfpathcurveto{\pgfqpoint{1.354982in}{2.231437in}}{\pgfqpoint{1.347082in}{2.228165in}}{\pgfqpoint{1.341258in}{2.222341in}}%
\pgfpathcurveto{\pgfqpoint{1.335434in}{2.216517in}}{\pgfqpoint{1.332162in}{2.208617in}}{\pgfqpoint{1.332162in}{2.200380in}}%
\pgfpathcurveto{\pgfqpoint{1.332162in}{2.192144in}}{\pgfqpoint{1.335434in}{2.184244in}}{\pgfqpoint{1.341258in}{2.178420in}}%
\pgfpathcurveto{\pgfqpoint{1.347082in}{2.172596in}}{\pgfqpoint{1.354982in}{2.169324in}}{\pgfqpoint{1.363218in}{2.169324in}}%
\pgfpathclose%
\pgfusepath{stroke,fill}%
\end{pgfscope}%
\begin{pgfscope}%
\pgfpathrectangle{\pgfqpoint{0.100000in}{0.212622in}}{\pgfqpoint{3.696000in}{3.696000in}}%
\pgfusepath{clip}%
\pgfsetbuttcap%
\pgfsetroundjoin%
\definecolor{currentfill}{rgb}{0.121569,0.466667,0.705882}%
\pgfsetfillcolor{currentfill}%
\pgfsetfillopacity{0.408283}%
\pgfsetlinewidth{1.003750pt}%
\definecolor{currentstroke}{rgb}{0.121569,0.466667,0.705882}%
\pgfsetstrokecolor{currentstroke}%
\pgfsetstrokeopacity{0.408283}%
\pgfsetdash{}{0pt}%
\pgfpathmoveto{\pgfqpoint{2.387057in}{2.419162in}}%
\pgfpathcurveto{\pgfqpoint{2.395293in}{2.419162in}}{\pgfqpoint{2.403194in}{2.422434in}}{\pgfqpoint{2.409017in}{2.428258in}}%
\pgfpathcurveto{\pgfqpoint{2.414841in}{2.434082in}}{\pgfqpoint{2.418114in}{2.441982in}}{\pgfqpoint{2.418114in}{2.450218in}}%
\pgfpathcurveto{\pgfqpoint{2.418114in}{2.458455in}}{\pgfqpoint{2.414841in}{2.466355in}}{\pgfqpoint{2.409017in}{2.472179in}}%
\pgfpathcurveto{\pgfqpoint{2.403194in}{2.478003in}}{\pgfqpoint{2.395293in}{2.481275in}}{\pgfqpoint{2.387057in}{2.481275in}}%
\pgfpathcurveto{\pgfqpoint{2.378821in}{2.481275in}}{\pgfqpoint{2.370921in}{2.478003in}}{\pgfqpoint{2.365097in}{2.472179in}}%
\pgfpathcurveto{\pgfqpoint{2.359273in}{2.466355in}}{\pgfqpoint{2.356001in}{2.458455in}}{\pgfqpoint{2.356001in}{2.450218in}}%
\pgfpathcurveto{\pgfqpoint{2.356001in}{2.441982in}}{\pgfqpoint{2.359273in}{2.434082in}}{\pgfqpoint{2.365097in}{2.428258in}}%
\pgfpathcurveto{\pgfqpoint{2.370921in}{2.422434in}}{\pgfqpoint{2.378821in}{2.419162in}}{\pgfqpoint{2.387057in}{2.419162in}}%
\pgfpathclose%
\pgfusepath{stroke,fill}%
\end{pgfscope}%
\begin{pgfscope}%
\pgfpathrectangle{\pgfqpoint{0.100000in}{0.212622in}}{\pgfqpoint{3.696000in}{3.696000in}}%
\pgfusepath{clip}%
\pgfsetbuttcap%
\pgfsetroundjoin%
\definecolor{currentfill}{rgb}{0.121569,0.466667,0.705882}%
\pgfsetfillcolor{currentfill}%
\pgfsetfillopacity{0.408749}%
\pgfsetlinewidth{1.003750pt}%
\definecolor{currentstroke}{rgb}{0.121569,0.466667,0.705882}%
\pgfsetstrokecolor{currentstroke}%
\pgfsetstrokeopacity{0.408749}%
\pgfsetdash{}{0pt}%
\pgfpathmoveto{\pgfqpoint{2.389780in}{2.418919in}}%
\pgfpathcurveto{\pgfqpoint{2.398016in}{2.418919in}}{\pgfqpoint{2.405916in}{2.422192in}}{\pgfqpoint{2.411740in}{2.428016in}}%
\pgfpathcurveto{\pgfqpoint{2.417564in}{2.433840in}}{\pgfqpoint{2.420836in}{2.441740in}}{\pgfqpoint{2.420836in}{2.449976in}}%
\pgfpathcurveto{\pgfqpoint{2.420836in}{2.458212in}}{\pgfqpoint{2.417564in}{2.466112in}}{\pgfqpoint{2.411740in}{2.471936in}}%
\pgfpathcurveto{\pgfqpoint{2.405916in}{2.477760in}}{\pgfqpoint{2.398016in}{2.481032in}}{\pgfqpoint{2.389780in}{2.481032in}}%
\pgfpathcurveto{\pgfqpoint{2.381544in}{2.481032in}}{\pgfqpoint{2.373644in}{2.477760in}}{\pgfqpoint{2.367820in}{2.471936in}}%
\pgfpathcurveto{\pgfqpoint{2.361996in}{2.466112in}}{\pgfqpoint{2.358723in}{2.458212in}}{\pgfqpoint{2.358723in}{2.449976in}}%
\pgfpathcurveto{\pgfqpoint{2.358723in}{2.441740in}}{\pgfqpoint{2.361996in}{2.433840in}}{\pgfqpoint{2.367820in}{2.428016in}}%
\pgfpathcurveto{\pgfqpoint{2.373644in}{2.422192in}}{\pgfqpoint{2.381544in}{2.418919in}}{\pgfqpoint{2.389780in}{2.418919in}}%
\pgfpathclose%
\pgfusepath{stroke,fill}%
\end{pgfscope}%
\begin{pgfscope}%
\pgfpathrectangle{\pgfqpoint{0.100000in}{0.212622in}}{\pgfqpoint{3.696000in}{3.696000in}}%
\pgfusepath{clip}%
\pgfsetbuttcap%
\pgfsetroundjoin%
\definecolor{currentfill}{rgb}{0.121569,0.466667,0.705882}%
\pgfsetfillcolor{currentfill}%
\pgfsetfillopacity{0.409420}%
\pgfsetlinewidth{1.003750pt}%
\definecolor{currentstroke}{rgb}{0.121569,0.466667,0.705882}%
\pgfsetstrokecolor{currentstroke}%
\pgfsetstrokeopacity{0.409420}%
\pgfsetdash{}{0pt}%
\pgfpathmoveto{\pgfqpoint{2.393835in}{2.418353in}}%
\pgfpathcurveto{\pgfqpoint{2.402072in}{2.418353in}}{\pgfqpoint{2.409972in}{2.421625in}}{\pgfqpoint{2.415796in}{2.427449in}}%
\pgfpathcurveto{\pgfqpoint{2.421620in}{2.433273in}}{\pgfqpoint{2.424892in}{2.441173in}}{\pgfqpoint{2.424892in}{2.449410in}}%
\pgfpathcurveto{\pgfqpoint{2.424892in}{2.457646in}}{\pgfqpoint{2.421620in}{2.465546in}}{\pgfqpoint{2.415796in}{2.471370in}}%
\pgfpathcurveto{\pgfqpoint{2.409972in}{2.477194in}}{\pgfqpoint{2.402072in}{2.480466in}}{\pgfqpoint{2.393835in}{2.480466in}}%
\pgfpathcurveto{\pgfqpoint{2.385599in}{2.480466in}}{\pgfqpoint{2.377699in}{2.477194in}}{\pgfqpoint{2.371875in}{2.471370in}}%
\pgfpathcurveto{\pgfqpoint{2.366051in}{2.465546in}}{\pgfqpoint{2.362779in}{2.457646in}}{\pgfqpoint{2.362779in}{2.449410in}}%
\pgfpathcurveto{\pgfqpoint{2.362779in}{2.441173in}}{\pgfqpoint{2.366051in}{2.433273in}}{\pgfqpoint{2.371875in}{2.427449in}}%
\pgfpathcurveto{\pgfqpoint{2.377699in}{2.421625in}}{\pgfqpoint{2.385599in}{2.418353in}}{\pgfqpoint{2.393835in}{2.418353in}}%
\pgfpathclose%
\pgfusepath{stroke,fill}%
\end{pgfscope}%
\begin{pgfscope}%
\pgfpathrectangle{\pgfqpoint{0.100000in}{0.212622in}}{\pgfqpoint{3.696000in}{3.696000in}}%
\pgfusepath{clip}%
\pgfsetbuttcap%
\pgfsetroundjoin%
\definecolor{currentfill}{rgb}{0.121569,0.466667,0.705882}%
\pgfsetfillcolor{currentfill}%
\pgfsetfillopacity{0.410500}%
\pgfsetlinewidth{1.003750pt}%
\definecolor{currentstroke}{rgb}{0.121569,0.466667,0.705882}%
\pgfsetstrokecolor{currentstroke}%
\pgfsetstrokeopacity{0.410500}%
\pgfsetdash{}{0pt}%
\pgfpathmoveto{\pgfqpoint{2.400401in}{2.417220in}}%
\pgfpathcurveto{\pgfqpoint{2.408637in}{2.417220in}}{\pgfqpoint{2.416537in}{2.420492in}}{\pgfqpoint{2.422361in}{2.426316in}}%
\pgfpathcurveto{\pgfqpoint{2.428185in}{2.432140in}}{\pgfqpoint{2.431458in}{2.440040in}}{\pgfqpoint{2.431458in}{2.448276in}}%
\pgfpathcurveto{\pgfqpoint{2.431458in}{2.456512in}}{\pgfqpoint{2.428185in}{2.464412in}}{\pgfqpoint{2.422361in}{2.470236in}}%
\pgfpathcurveto{\pgfqpoint{2.416537in}{2.476060in}}{\pgfqpoint{2.408637in}{2.479333in}}{\pgfqpoint{2.400401in}{2.479333in}}%
\pgfpathcurveto{\pgfqpoint{2.392165in}{2.479333in}}{\pgfqpoint{2.384265in}{2.476060in}}{\pgfqpoint{2.378441in}{2.470236in}}%
\pgfpathcurveto{\pgfqpoint{2.372617in}{2.464412in}}{\pgfqpoint{2.369345in}{2.456512in}}{\pgfqpoint{2.369345in}{2.448276in}}%
\pgfpathcurveto{\pgfqpoint{2.369345in}{2.440040in}}{\pgfqpoint{2.372617in}{2.432140in}}{\pgfqpoint{2.378441in}{2.426316in}}%
\pgfpathcurveto{\pgfqpoint{2.384265in}{2.420492in}}{\pgfqpoint{2.392165in}{2.417220in}}{\pgfqpoint{2.400401in}{2.417220in}}%
\pgfpathclose%
\pgfusepath{stroke,fill}%
\end{pgfscope}%
\begin{pgfscope}%
\pgfpathrectangle{\pgfqpoint{0.100000in}{0.212622in}}{\pgfqpoint{3.696000in}{3.696000in}}%
\pgfusepath{clip}%
\pgfsetbuttcap%
\pgfsetroundjoin%
\definecolor{currentfill}{rgb}{0.121569,0.466667,0.705882}%
\pgfsetfillcolor{currentfill}%
\pgfsetfillopacity{0.411373}%
\pgfsetlinewidth{1.003750pt}%
\definecolor{currentstroke}{rgb}{0.121569,0.466667,0.705882}%
\pgfsetstrokecolor{currentstroke}%
\pgfsetstrokeopacity{0.411373}%
\pgfsetdash{}{0pt}%
\pgfpathmoveto{\pgfqpoint{1.354420in}{2.156993in}}%
\pgfpathcurveto{\pgfqpoint{1.362656in}{2.156993in}}{\pgfqpoint{1.370556in}{2.160266in}}{\pgfqpoint{1.376380in}{2.166090in}}%
\pgfpathcurveto{\pgfqpoint{1.382204in}{2.171913in}}{\pgfqpoint{1.385476in}{2.179813in}}{\pgfqpoint{1.385476in}{2.188050in}}%
\pgfpathcurveto{\pgfqpoint{1.385476in}{2.196286in}}{\pgfqpoint{1.382204in}{2.204186in}}{\pgfqpoint{1.376380in}{2.210010in}}%
\pgfpathcurveto{\pgfqpoint{1.370556in}{2.215834in}}{\pgfqpoint{1.362656in}{2.219106in}}{\pgfqpoint{1.354420in}{2.219106in}}%
\pgfpathcurveto{\pgfqpoint{1.346183in}{2.219106in}}{\pgfqpoint{1.338283in}{2.215834in}}{\pgfqpoint{1.332459in}{2.210010in}}%
\pgfpathcurveto{\pgfqpoint{1.326635in}{2.204186in}}{\pgfqpoint{1.323363in}{2.196286in}}{\pgfqpoint{1.323363in}{2.188050in}}%
\pgfpathcurveto{\pgfqpoint{1.323363in}{2.179813in}}{\pgfqpoint{1.326635in}{2.171913in}}{\pgfqpoint{1.332459in}{2.166090in}}%
\pgfpathcurveto{\pgfqpoint{1.338283in}{2.160266in}}{\pgfqpoint{1.346183in}{2.156993in}}{\pgfqpoint{1.354420in}{2.156993in}}%
\pgfpathclose%
\pgfusepath{stroke,fill}%
\end{pgfscope}%
\begin{pgfscope}%
\pgfpathrectangle{\pgfqpoint{0.100000in}{0.212622in}}{\pgfqpoint{3.696000in}{3.696000in}}%
\pgfusepath{clip}%
\pgfsetbuttcap%
\pgfsetroundjoin%
\definecolor{currentfill}{rgb}{0.121569,0.466667,0.705882}%
\pgfsetfillcolor{currentfill}%
\pgfsetfillopacity{0.411789}%
\pgfsetlinewidth{1.003750pt}%
\definecolor{currentstroke}{rgb}{0.121569,0.466667,0.705882}%
\pgfsetstrokecolor{currentstroke}%
\pgfsetstrokeopacity{0.411789}%
\pgfsetdash{}{0pt}%
\pgfpathmoveto{\pgfqpoint{2.408036in}{2.416354in}}%
\pgfpathcurveto{\pgfqpoint{2.416272in}{2.416354in}}{\pgfqpoint{2.424172in}{2.419626in}}{\pgfqpoint{2.429996in}{2.425450in}}%
\pgfpathcurveto{\pgfqpoint{2.435820in}{2.431274in}}{\pgfqpoint{2.439092in}{2.439174in}}{\pgfqpoint{2.439092in}{2.447410in}}%
\pgfpathcurveto{\pgfqpoint{2.439092in}{2.455646in}}{\pgfqpoint{2.435820in}{2.463546in}}{\pgfqpoint{2.429996in}{2.469370in}}%
\pgfpathcurveto{\pgfqpoint{2.424172in}{2.475194in}}{\pgfqpoint{2.416272in}{2.478467in}}{\pgfqpoint{2.408036in}{2.478467in}}%
\pgfpathcurveto{\pgfqpoint{2.399800in}{2.478467in}}{\pgfqpoint{2.391900in}{2.475194in}}{\pgfqpoint{2.386076in}{2.469370in}}%
\pgfpathcurveto{\pgfqpoint{2.380252in}{2.463546in}}{\pgfqpoint{2.376979in}{2.455646in}}{\pgfqpoint{2.376979in}{2.447410in}}%
\pgfpathcurveto{\pgfqpoint{2.376979in}{2.439174in}}{\pgfqpoint{2.380252in}{2.431274in}}{\pgfqpoint{2.386076in}{2.425450in}}%
\pgfpathcurveto{\pgfqpoint{2.391900in}{2.419626in}}{\pgfqpoint{2.399800in}{2.416354in}}{\pgfqpoint{2.408036in}{2.416354in}}%
\pgfpathclose%
\pgfusepath{stroke,fill}%
\end{pgfscope}%
\begin{pgfscope}%
\pgfpathrectangle{\pgfqpoint{0.100000in}{0.212622in}}{\pgfqpoint{3.696000in}{3.696000in}}%
\pgfusepath{clip}%
\pgfsetbuttcap%
\pgfsetroundjoin%
\definecolor{currentfill}{rgb}{0.121569,0.466667,0.705882}%
\pgfsetfillcolor{currentfill}%
\pgfsetfillopacity{0.413154}%
\pgfsetlinewidth{1.003750pt}%
\definecolor{currentstroke}{rgb}{0.121569,0.466667,0.705882}%
\pgfsetstrokecolor{currentstroke}%
\pgfsetstrokeopacity{0.413154}%
\pgfsetdash{}{0pt}%
\pgfpathmoveto{\pgfqpoint{2.416472in}{2.414715in}}%
\pgfpathcurveto{\pgfqpoint{2.424708in}{2.414715in}}{\pgfqpoint{2.432608in}{2.417987in}}{\pgfqpoint{2.438432in}{2.423811in}}%
\pgfpathcurveto{\pgfqpoint{2.444256in}{2.429635in}}{\pgfqpoint{2.447529in}{2.437535in}}{\pgfqpoint{2.447529in}{2.445772in}}%
\pgfpathcurveto{\pgfqpoint{2.447529in}{2.454008in}}{\pgfqpoint{2.444256in}{2.461908in}}{\pgfqpoint{2.438432in}{2.467732in}}%
\pgfpathcurveto{\pgfqpoint{2.432608in}{2.473556in}}{\pgfqpoint{2.424708in}{2.476828in}}{\pgfqpoint{2.416472in}{2.476828in}}%
\pgfpathcurveto{\pgfqpoint{2.408236in}{2.476828in}}{\pgfqpoint{2.400336in}{2.473556in}}{\pgfqpoint{2.394512in}{2.467732in}}%
\pgfpathcurveto{\pgfqpoint{2.388688in}{2.461908in}}{\pgfqpoint{2.385416in}{2.454008in}}{\pgfqpoint{2.385416in}{2.445772in}}%
\pgfpathcurveto{\pgfqpoint{2.385416in}{2.437535in}}{\pgfqpoint{2.388688in}{2.429635in}}{\pgfqpoint{2.394512in}{2.423811in}}%
\pgfpathcurveto{\pgfqpoint{2.400336in}{2.417987in}}{\pgfqpoint{2.408236in}{2.414715in}}{\pgfqpoint{2.416472in}{2.414715in}}%
\pgfpathclose%
\pgfusepath{stroke,fill}%
\end{pgfscope}%
\begin{pgfscope}%
\pgfpathrectangle{\pgfqpoint{0.100000in}{0.212622in}}{\pgfqpoint{3.696000in}{3.696000in}}%
\pgfusepath{clip}%
\pgfsetbuttcap%
\pgfsetroundjoin%
\definecolor{currentfill}{rgb}{0.121569,0.466667,0.705882}%
\pgfsetfillcolor{currentfill}%
\pgfsetfillopacity{0.414378}%
\pgfsetlinewidth{1.003750pt}%
\definecolor{currentstroke}{rgb}{0.121569,0.466667,0.705882}%
\pgfsetstrokecolor{currentstroke}%
\pgfsetstrokeopacity{0.414378}%
\pgfsetdash{}{0pt}%
\pgfpathmoveto{\pgfqpoint{1.346288in}{2.146010in}}%
\pgfpathcurveto{\pgfqpoint{1.354524in}{2.146010in}}{\pgfqpoint{1.362424in}{2.149282in}}{\pgfqpoint{1.368248in}{2.155106in}}%
\pgfpathcurveto{\pgfqpoint{1.374072in}{2.160930in}}{\pgfqpoint{1.377344in}{2.168830in}}{\pgfqpoint{1.377344in}{2.177067in}}%
\pgfpathcurveto{\pgfqpoint{1.377344in}{2.185303in}}{\pgfqpoint{1.374072in}{2.193203in}}{\pgfqpoint{1.368248in}{2.199027in}}%
\pgfpathcurveto{\pgfqpoint{1.362424in}{2.204851in}}{\pgfqpoint{1.354524in}{2.208123in}}{\pgfqpoint{1.346288in}{2.208123in}}%
\pgfpathcurveto{\pgfqpoint{1.338052in}{2.208123in}}{\pgfqpoint{1.330151in}{2.204851in}}{\pgfqpoint{1.324328in}{2.199027in}}%
\pgfpathcurveto{\pgfqpoint{1.318504in}{2.193203in}}{\pgfqpoint{1.315231in}{2.185303in}}{\pgfqpoint{1.315231in}{2.177067in}}%
\pgfpathcurveto{\pgfqpoint{1.315231in}{2.168830in}}{\pgfqpoint{1.318504in}{2.160930in}}{\pgfqpoint{1.324328in}{2.155106in}}%
\pgfpathcurveto{\pgfqpoint{1.330151in}{2.149282in}}{\pgfqpoint{1.338052in}{2.146010in}}{\pgfqpoint{1.346288in}{2.146010in}}%
\pgfpathclose%
\pgfusepath{stroke,fill}%
\end{pgfscope}%
\begin{pgfscope}%
\pgfpathrectangle{\pgfqpoint{0.100000in}{0.212622in}}{\pgfqpoint{3.696000in}{3.696000in}}%
\pgfusepath{clip}%
\pgfsetbuttcap%
\pgfsetroundjoin%
\definecolor{currentfill}{rgb}{0.121569,0.466667,0.705882}%
\pgfsetfillcolor{currentfill}%
\pgfsetfillopacity{0.414853}%
\pgfsetlinewidth{1.003750pt}%
\definecolor{currentstroke}{rgb}{0.121569,0.466667,0.705882}%
\pgfsetstrokecolor{currentstroke}%
\pgfsetstrokeopacity{0.414853}%
\pgfsetdash{}{0pt}%
\pgfpathmoveto{\pgfqpoint{2.426432in}{2.413561in}}%
\pgfpathcurveto{\pgfqpoint{2.434668in}{2.413561in}}{\pgfqpoint{2.442568in}{2.416833in}}{\pgfqpoint{2.448392in}{2.422657in}}%
\pgfpathcurveto{\pgfqpoint{2.454216in}{2.428481in}}{\pgfqpoint{2.457488in}{2.436381in}}{\pgfqpoint{2.457488in}{2.444617in}}%
\pgfpathcurveto{\pgfqpoint{2.457488in}{2.452854in}}{\pgfqpoint{2.454216in}{2.460754in}}{\pgfqpoint{2.448392in}{2.466578in}}%
\pgfpathcurveto{\pgfqpoint{2.442568in}{2.472402in}}{\pgfqpoint{2.434668in}{2.475674in}}{\pgfqpoint{2.426432in}{2.475674in}}%
\pgfpathcurveto{\pgfqpoint{2.418196in}{2.475674in}}{\pgfqpoint{2.410296in}{2.472402in}}{\pgfqpoint{2.404472in}{2.466578in}}%
\pgfpathcurveto{\pgfqpoint{2.398648in}{2.460754in}}{\pgfqpoint{2.395375in}{2.452854in}}{\pgfqpoint{2.395375in}{2.444617in}}%
\pgfpathcurveto{\pgfqpoint{2.395375in}{2.436381in}}{\pgfqpoint{2.398648in}{2.428481in}}{\pgfqpoint{2.404472in}{2.422657in}}%
\pgfpathcurveto{\pgfqpoint{2.410296in}{2.416833in}}{\pgfqpoint{2.418196in}{2.413561in}}{\pgfqpoint{2.426432in}{2.413561in}}%
\pgfpathclose%
\pgfusepath{stroke,fill}%
\end{pgfscope}%
\begin{pgfscope}%
\pgfpathrectangle{\pgfqpoint{0.100000in}{0.212622in}}{\pgfqpoint{3.696000in}{3.696000in}}%
\pgfusepath{clip}%
\pgfsetbuttcap%
\pgfsetroundjoin%
\definecolor{currentfill}{rgb}{0.121569,0.466667,0.705882}%
\pgfsetfillcolor{currentfill}%
\pgfsetfillopacity{0.416772}%
\pgfsetlinewidth{1.003750pt}%
\definecolor{currentstroke}{rgb}{0.121569,0.466667,0.705882}%
\pgfsetstrokecolor{currentstroke}%
\pgfsetstrokeopacity{0.416772}%
\pgfsetdash{}{0pt}%
\pgfpathmoveto{\pgfqpoint{2.437556in}{2.412251in}}%
\pgfpathcurveto{\pgfqpoint{2.445793in}{2.412251in}}{\pgfqpoint{2.453693in}{2.415524in}}{\pgfqpoint{2.459517in}{2.421348in}}%
\pgfpathcurveto{\pgfqpoint{2.465341in}{2.427172in}}{\pgfqpoint{2.468613in}{2.435072in}}{\pgfqpoint{2.468613in}{2.443308in}}%
\pgfpathcurveto{\pgfqpoint{2.468613in}{2.451544in}}{\pgfqpoint{2.465341in}{2.459444in}}{\pgfqpoint{2.459517in}{2.465268in}}%
\pgfpathcurveto{\pgfqpoint{2.453693in}{2.471092in}}{\pgfqpoint{2.445793in}{2.474364in}}{\pgfqpoint{2.437556in}{2.474364in}}%
\pgfpathcurveto{\pgfqpoint{2.429320in}{2.474364in}}{\pgfqpoint{2.421420in}{2.471092in}}{\pgfqpoint{2.415596in}{2.465268in}}%
\pgfpathcurveto{\pgfqpoint{2.409772in}{2.459444in}}{\pgfqpoint{2.406500in}{2.451544in}}{\pgfqpoint{2.406500in}{2.443308in}}%
\pgfpathcurveto{\pgfqpoint{2.406500in}{2.435072in}}{\pgfqpoint{2.409772in}{2.427172in}}{\pgfqpoint{2.415596in}{2.421348in}}%
\pgfpathcurveto{\pgfqpoint{2.421420in}{2.415524in}}{\pgfqpoint{2.429320in}{2.412251in}}{\pgfqpoint{2.437556in}{2.412251in}}%
\pgfpathclose%
\pgfusepath{stroke,fill}%
\end{pgfscope}%
\begin{pgfscope}%
\pgfpathrectangle{\pgfqpoint{0.100000in}{0.212622in}}{\pgfqpoint{3.696000in}{3.696000in}}%
\pgfusepath{clip}%
\pgfsetbuttcap%
\pgfsetroundjoin%
\definecolor{currentfill}{rgb}{0.121569,0.466667,0.705882}%
\pgfsetfillcolor{currentfill}%
\pgfsetfillopacity{0.417117}%
\pgfsetlinewidth{1.003750pt}%
\definecolor{currentstroke}{rgb}{0.121569,0.466667,0.705882}%
\pgfsetstrokecolor{currentstroke}%
\pgfsetstrokeopacity{0.417117}%
\pgfsetdash{}{0pt}%
\pgfpathmoveto{\pgfqpoint{1.338641in}{2.135018in}}%
\pgfpathcurveto{\pgfqpoint{1.346877in}{2.135018in}}{\pgfqpoint{1.354777in}{2.138291in}}{\pgfqpoint{1.360601in}{2.144114in}}%
\pgfpathcurveto{\pgfqpoint{1.366425in}{2.149938in}}{\pgfqpoint{1.369698in}{2.157838in}}{\pgfqpoint{1.369698in}{2.166075in}}%
\pgfpathcurveto{\pgfqpoint{1.369698in}{2.174311in}}{\pgfqpoint{1.366425in}{2.182211in}}{\pgfqpoint{1.360601in}{2.188035in}}%
\pgfpathcurveto{\pgfqpoint{1.354777in}{2.193859in}}{\pgfqpoint{1.346877in}{2.197131in}}{\pgfqpoint{1.338641in}{2.197131in}}%
\pgfpathcurveto{\pgfqpoint{1.330405in}{2.197131in}}{\pgfqpoint{1.322505in}{2.193859in}}{\pgfqpoint{1.316681in}{2.188035in}}%
\pgfpathcurveto{\pgfqpoint{1.310857in}{2.182211in}}{\pgfqpoint{1.307585in}{2.174311in}}{\pgfqpoint{1.307585in}{2.166075in}}%
\pgfpathcurveto{\pgfqpoint{1.307585in}{2.157838in}}{\pgfqpoint{1.310857in}{2.149938in}}{\pgfqpoint{1.316681in}{2.144114in}}%
\pgfpathcurveto{\pgfqpoint{1.322505in}{2.138291in}}{\pgfqpoint{1.330405in}{2.135018in}}{\pgfqpoint{1.338641in}{2.135018in}}%
\pgfpathclose%
\pgfusepath{stroke,fill}%
\end{pgfscope}%
\begin{pgfscope}%
\pgfpathrectangle{\pgfqpoint{0.100000in}{0.212622in}}{\pgfqpoint{3.696000in}{3.696000in}}%
\pgfusepath{clip}%
\pgfsetbuttcap%
\pgfsetroundjoin%
\definecolor{currentfill}{rgb}{0.121569,0.466667,0.705882}%
\pgfsetfillcolor{currentfill}%
\pgfsetfillopacity{0.417835}%
\pgfsetlinewidth{1.003750pt}%
\definecolor{currentstroke}{rgb}{0.121569,0.466667,0.705882}%
\pgfsetstrokecolor{currentstroke}%
\pgfsetstrokeopacity{0.417835}%
\pgfsetdash{}{0pt}%
\pgfpathmoveto{\pgfqpoint{2.443606in}{2.411357in}}%
\pgfpathcurveto{\pgfqpoint{2.451843in}{2.411357in}}{\pgfqpoint{2.459743in}{2.414630in}}{\pgfqpoint{2.465566in}{2.420454in}}%
\pgfpathcurveto{\pgfqpoint{2.471390in}{2.426278in}}{\pgfqpoint{2.474663in}{2.434178in}}{\pgfqpoint{2.474663in}{2.442414in}}%
\pgfpathcurveto{\pgfqpoint{2.474663in}{2.450650in}}{\pgfqpoint{2.471390in}{2.458550in}}{\pgfqpoint{2.465566in}{2.464374in}}%
\pgfpathcurveto{\pgfqpoint{2.459743in}{2.470198in}}{\pgfqpoint{2.451843in}{2.473470in}}{\pgfqpoint{2.443606in}{2.473470in}}%
\pgfpathcurveto{\pgfqpoint{2.435370in}{2.473470in}}{\pgfqpoint{2.427470in}{2.470198in}}{\pgfqpoint{2.421646in}{2.464374in}}%
\pgfpathcurveto{\pgfqpoint{2.415822in}{2.458550in}}{\pgfqpoint{2.412550in}{2.450650in}}{\pgfqpoint{2.412550in}{2.442414in}}%
\pgfpathcurveto{\pgfqpoint{2.412550in}{2.434178in}}{\pgfqpoint{2.415822in}{2.426278in}}{\pgfqpoint{2.421646in}{2.420454in}}%
\pgfpathcurveto{\pgfqpoint{2.427470in}{2.414630in}}{\pgfqpoint{2.435370in}{2.411357in}}{\pgfqpoint{2.443606in}{2.411357in}}%
\pgfpathclose%
\pgfusepath{stroke,fill}%
\end{pgfscope}%
\begin{pgfscope}%
\pgfpathrectangle{\pgfqpoint{0.100000in}{0.212622in}}{\pgfqpoint{3.696000in}{3.696000in}}%
\pgfusepath{clip}%
\pgfsetbuttcap%
\pgfsetroundjoin%
\definecolor{currentfill}{rgb}{0.121569,0.466667,0.705882}%
\pgfsetfillcolor{currentfill}%
\pgfsetfillopacity{0.419014}%
\pgfsetlinewidth{1.003750pt}%
\definecolor{currentstroke}{rgb}{0.121569,0.466667,0.705882}%
\pgfsetstrokecolor{currentstroke}%
\pgfsetstrokeopacity{0.419014}%
\pgfsetdash{}{0pt}%
\pgfpathmoveto{\pgfqpoint{2.450513in}{2.410315in}}%
\pgfpathcurveto{\pgfqpoint{2.458749in}{2.410315in}}{\pgfqpoint{2.466649in}{2.413587in}}{\pgfqpoint{2.472473in}{2.419411in}}%
\pgfpathcurveto{\pgfqpoint{2.478297in}{2.425235in}}{\pgfqpoint{2.481569in}{2.433135in}}{\pgfqpoint{2.481569in}{2.441371in}}%
\pgfpathcurveto{\pgfqpoint{2.481569in}{2.449607in}}{\pgfqpoint{2.478297in}{2.457507in}}{\pgfqpoint{2.472473in}{2.463331in}}%
\pgfpathcurveto{\pgfqpoint{2.466649in}{2.469155in}}{\pgfqpoint{2.458749in}{2.472428in}}{\pgfqpoint{2.450513in}{2.472428in}}%
\pgfpathcurveto{\pgfqpoint{2.442277in}{2.472428in}}{\pgfqpoint{2.434377in}{2.469155in}}{\pgfqpoint{2.428553in}{2.463331in}}%
\pgfpathcurveto{\pgfqpoint{2.422729in}{2.457507in}}{\pgfqpoint{2.419456in}{2.449607in}}{\pgfqpoint{2.419456in}{2.441371in}}%
\pgfpathcurveto{\pgfqpoint{2.419456in}{2.433135in}}{\pgfqpoint{2.422729in}{2.425235in}}{\pgfqpoint{2.428553in}{2.419411in}}%
\pgfpathcurveto{\pgfqpoint{2.434377in}{2.413587in}}{\pgfqpoint{2.442277in}{2.410315in}}{\pgfqpoint{2.450513in}{2.410315in}}%
\pgfpathclose%
\pgfusepath{stroke,fill}%
\end{pgfscope}%
\begin{pgfscope}%
\pgfpathrectangle{\pgfqpoint{0.100000in}{0.212622in}}{\pgfqpoint{3.696000in}{3.696000in}}%
\pgfusepath{clip}%
\pgfsetbuttcap%
\pgfsetroundjoin%
\definecolor{currentfill}{rgb}{0.121569,0.466667,0.705882}%
\pgfsetfillcolor{currentfill}%
\pgfsetfillopacity{0.419690}%
\pgfsetlinewidth{1.003750pt}%
\definecolor{currentstroke}{rgb}{0.121569,0.466667,0.705882}%
\pgfsetstrokecolor{currentstroke}%
\pgfsetstrokeopacity{0.419690}%
\pgfsetdash{}{0pt}%
\pgfpathmoveto{\pgfqpoint{1.331758in}{2.124460in}}%
\pgfpathcurveto{\pgfqpoint{1.339995in}{2.124460in}}{\pgfqpoint{1.347895in}{2.127732in}}{\pgfqpoint{1.353719in}{2.133556in}}%
\pgfpathcurveto{\pgfqpoint{1.359543in}{2.139380in}}{\pgfqpoint{1.362815in}{2.147280in}}{\pgfqpoint{1.362815in}{2.155516in}}%
\pgfpathcurveto{\pgfqpoint{1.362815in}{2.163752in}}{\pgfqpoint{1.359543in}{2.171653in}}{\pgfqpoint{1.353719in}{2.177476in}}%
\pgfpathcurveto{\pgfqpoint{1.347895in}{2.183300in}}{\pgfqpoint{1.339995in}{2.186573in}}{\pgfqpoint{1.331758in}{2.186573in}}%
\pgfpathcurveto{\pgfqpoint{1.323522in}{2.186573in}}{\pgfqpoint{1.315622in}{2.183300in}}{\pgfqpoint{1.309798in}{2.177476in}}%
\pgfpathcurveto{\pgfqpoint{1.303974in}{2.171653in}}{\pgfqpoint{1.300702in}{2.163752in}}{\pgfqpoint{1.300702in}{2.155516in}}%
\pgfpathcurveto{\pgfqpoint{1.300702in}{2.147280in}}{\pgfqpoint{1.303974in}{2.139380in}}{\pgfqpoint{1.309798in}{2.133556in}}%
\pgfpathcurveto{\pgfqpoint{1.315622in}{2.127732in}}{\pgfqpoint{1.323522in}{2.124460in}}{\pgfqpoint{1.331758in}{2.124460in}}%
\pgfpathclose%
\pgfusepath{stroke,fill}%
\end{pgfscope}%
\begin{pgfscope}%
\pgfpathrectangle{\pgfqpoint{0.100000in}{0.212622in}}{\pgfqpoint{3.696000in}{3.696000in}}%
\pgfusepath{clip}%
\pgfsetbuttcap%
\pgfsetroundjoin%
\definecolor{currentfill}{rgb}{0.121569,0.466667,0.705882}%
\pgfsetfillcolor{currentfill}%
\pgfsetfillopacity{0.420367}%
\pgfsetlinewidth{1.003750pt}%
\definecolor{currentstroke}{rgb}{0.121569,0.466667,0.705882}%
\pgfsetstrokecolor{currentstroke}%
\pgfsetstrokeopacity{0.420367}%
\pgfsetdash{}{0pt}%
\pgfpathmoveto{\pgfqpoint{2.458613in}{2.408758in}}%
\pgfpathcurveto{\pgfqpoint{2.466849in}{2.408758in}}{\pgfqpoint{2.474749in}{2.412030in}}{\pgfqpoint{2.480573in}{2.417854in}}%
\pgfpathcurveto{\pgfqpoint{2.486397in}{2.423678in}}{\pgfqpoint{2.489669in}{2.431578in}}{\pgfqpoint{2.489669in}{2.439814in}}%
\pgfpathcurveto{\pgfqpoint{2.489669in}{2.448050in}}{\pgfqpoint{2.486397in}{2.455950in}}{\pgfqpoint{2.480573in}{2.461774in}}%
\pgfpathcurveto{\pgfqpoint{2.474749in}{2.467598in}}{\pgfqpoint{2.466849in}{2.470871in}}{\pgfqpoint{2.458613in}{2.470871in}}%
\pgfpathcurveto{\pgfqpoint{2.450377in}{2.470871in}}{\pgfqpoint{2.442476in}{2.467598in}}{\pgfqpoint{2.436653in}{2.461774in}}%
\pgfpathcurveto{\pgfqpoint{2.430829in}{2.455950in}}{\pgfqpoint{2.427556in}{2.448050in}}{\pgfqpoint{2.427556in}{2.439814in}}%
\pgfpathcurveto{\pgfqpoint{2.427556in}{2.431578in}}{\pgfqpoint{2.430829in}{2.423678in}}{\pgfqpoint{2.436653in}{2.417854in}}%
\pgfpathcurveto{\pgfqpoint{2.442476in}{2.412030in}}{\pgfqpoint{2.450377in}{2.408758in}}{\pgfqpoint{2.458613in}{2.408758in}}%
\pgfpathclose%
\pgfusepath{stroke,fill}%
\end{pgfscope}%
\begin{pgfscope}%
\pgfpathrectangle{\pgfqpoint{0.100000in}{0.212622in}}{\pgfqpoint{3.696000in}{3.696000in}}%
\pgfusepath{clip}%
\pgfsetbuttcap%
\pgfsetroundjoin%
\definecolor{currentfill}{rgb}{0.121569,0.466667,0.705882}%
\pgfsetfillcolor{currentfill}%
\pgfsetfillopacity{0.421950}%
\pgfsetlinewidth{1.003750pt}%
\definecolor{currentstroke}{rgb}{0.121569,0.466667,0.705882}%
\pgfsetstrokecolor{currentstroke}%
\pgfsetstrokeopacity{0.421950}%
\pgfsetdash{}{0pt}%
\pgfpathmoveto{\pgfqpoint{2.467997in}{2.407450in}}%
\pgfpathcurveto{\pgfqpoint{2.476233in}{2.407450in}}{\pgfqpoint{2.484133in}{2.410722in}}{\pgfqpoint{2.489957in}{2.416546in}}%
\pgfpathcurveto{\pgfqpoint{2.495781in}{2.422370in}}{\pgfqpoint{2.499053in}{2.430270in}}{\pgfqpoint{2.499053in}{2.438506in}}%
\pgfpathcurveto{\pgfqpoint{2.499053in}{2.446743in}}{\pgfqpoint{2.495781in}{2.454643in}}{\pgfqpoint{2.489957in}{2.460467in}}%
\pgfpathcurveto{\pgfqpoint{2.484133in}{2.466291in}}{\pgfqpoint{2.476233in}{2.469563in}}{\pgfqpoint{2.467997in}{2.469563in}}%
\pgfpathcurveto{\pgfqpoint{2.459760in}{2.469563in}}{\pgfqpoint{2.451860in}{2.466291in}}{\pgfqpoint{2.446036in}{2.460467in}}%
\pgfpathcurveto{\pgfqpoint{2.440212in}{2.454643in}}{\pgfqpoint{2.436940in}{2.446743in}}{\pgfqpoint{2.436940in}{2.438506in}}%
\pgfpathcurveto{\pgfqpoint{2.436940in}{2.430270in}}{\pgfqpoint{2.440212in}{2.422370in}}{\pgfqpoint{2.446036in}{2.416546in}}%
\pgfpathcurveto{\pgfqpoint{2.451860in}{2.410722in}}{\pgfqpoint{2.459760in}{2.407450in}}{\pgfqpoint{2.467997in}{2.407450in}}%
\pgfpathclose%
\pgfusepath{stroke,fill}%
\end{pgfscope}%
\begin{pgfscope}%
\pgfpathrectangle{\pgfqpoint{0.100000in}{0.212622in}}{\pgfqpoint{3.696000in}{3.696000in}}%
\pgfusepath{clip}%
\pgfsetbuttcap%
\pgfsetroundjoin%
\definecolor{currentfill}{rgb}{0.121569,0.466667,0.705882}%
\pgfsetfillcolor{currentfill}%
\pgfsetfillopacity{0.422058}%
\pgfsetlinewidth{1.003750pt}%
\definecolor{currentstroke}{rgb}{0.121569,0.466667,0.705882}%
\pgfsetstrokecolor{currentstroke}%
\pgfsetstrokeopacity{0.422058}%
\pgfsetdash{}{0pt}%
\pgfpathmoveto{\pgfqpoint{1.325139in}{2.113577in}}%
\pgfpathcurveto{\pgfqpoint{1.333376in}{2.113577in}}{\pgfqpoint{1.341276in}{2.116850in}}{\pgfqpoint{1.347100in}{2.122673in}}%
\pgfpathcurveto{\pgfqpoint{1.352924in}{2.128497in}}{\pgfqpoint{1.356196in}{2.136397in}}{\pgfqpoint{1.356196in}{2.144634in}}%
\pgfpathcurveto{\pgfqpoint{1.356196in}{2.152870in}}{\pgfqpoint{1.352924in}{2.160770in}}{\pgfqpoint{1.347100in}{2.166594in}}%
\pgfpathcurveto{\pgfqpoint{1.341276in}{2.172418in}}{\pgfqpoint{1.333376in}{2.175690in}}{\pgfqpoint{1.325139in}{2.175690in}}%
\pgfpathcurveto{\pgfqpoint{1.316903in}{2.175690in}}{\pgfqpoint{1.309003in}{2.172418in}}{\pgfqpoint{1.303179in}{2.166594in}}%
\pgfpathcurveto{\pgfqpoint{1.297355in}{2.160770in}}{\pgfqpoint{1.294083in}{2.152870in}}{\pgfqpoint{1.294083in}{2.144634in}}%
\pgfpathcurveto{\pgfqpoint{1.294083in}{2.136397in}}{\pgfqpoint{1.297355in}{2.128497in}}{\pgfqpoint{1.303179in}{2.122673in}}%
\pgfpathcurveto{\pgfqpoint{1.309003in}{2.116850in}}{\pgfqpoint{1.316903in}{2.113577in}}{\pgfqpoint{1.325139in}{2.113577in}}%
\pgfpathclose%
\pgfusepath{stroke,fill}%
\end{pgfscope}%
\begin{pgfscope}%
\pgfpathrectangle{\pgfqpoint{0.100000in}{0.212622in}}{\pgfqpoint{3.696000in}{3.696000in}}%
\pgfusepath{clip}%
\pgfsetbuttcap%
\pgfsetroundjoin%
\definecolor{currentfill}{rgb}{0.121569,0.466667,0.705882}%
\pgfsetfillcolor{currentfill}%
\pgfsetfillopacity{0.423504}%
\pgfsetlinewidth{1.003750pt}%
\definecolor{currentstroke}{rgb}{0.121569,0.466667,0.705882}%
\pgfsetstrokecolor{currentstroke}%
\pgfsetstrokeopacity{0.423504}%
\pgfsetdash{}{0pt}%
\pgfpathmoveto{\pgfqpoint{2.478148in}{2.404393in}}%
\pgfpathcurveto{\pgfqpoint{2.486384in}{2.404393in}}{\pgfqpoint{2.494285in}{2.407665in}}{\pgfqpoint{2.500108in}{2.413489in}}%
\pgfpathcurveto{\pgfqpoint{2.505932in}{2.419313in}}{\pgfqpoint{2.509205in}{2.427213in}}{\pgfqpoint{2.509205in}{2.435450in}}%
\pgfpathcurveto{\pgfqpoint{2.509205in}{2.443686in}}{\pgfqpoint{2.505932in}{2.451586in}}{\pgfqpoint{2.500108in}{2.457410in}}%
\pgfpathcurveto{\pgfqpoint{2.494285in}{2.463234in}}{\pgfqpoint{2.486384in}{2.466506in}}{\pgfqpoint{2.478148in}{2.466506in}}%
\pgfpathcurveto{\pgfqpoint{2.469912in}{2.466506in}}{\pgfqpoint{2.462012in}{2.463234in}}{\pgfqpoint{2.456188in}{2.457410in}}%
\pgfpathcurveto{\pgfqpoint{2.450364in}{2.451586in}}{\pgfqpoint{2.447092in}{2.443686in}}{\pgfqpoint{2.447092in}{2.435450in}}%
\pgfpathcurveto{\pgfqpoint{2.447092in}{2.427213in}}{\pgfqpoint{2.450364in}{2.419313in}}{\pgfqpoint{2.456188in}{2.413489in}}%
\pgfpathcurveto{\pgfqpoint{2.462012in}{2.407665in}}{\pgfqpoint{2.469912in}{2.404393in}}{\pgfqpoint{2.478148in}{2.404393in}}%
\pgfpathclose%
\pgfusepath{stroke,fill}%
\end{pgfscope}%
\begin{pgfscope}%
\pgfpathrectangle{\pgfqpoint{0.100000in}{0.212622in}}{\pgfqpoint{3.696000in}{3.696000in}}%
\pgfusepath{clip}%
\pgfsetbuttcap%
\pgfsetroundjoin%
\definecolor{currentfill}{rgb}{0.121569,0.466667,0.705882}%
\pgfsetfillcolor{currentfill}%
\pgfsetfillopacity{0.424380}%
\pgfsetlinewidth{1.003750pt}%
\definecolor{currentstroke}{rgb}{0.121569,0.466667,0.705882}%
\pgfsetstrokecolor{currentstroke}%
\pgfsetstrokeopacity{0.424380}%
\pgfsetdash{}{0pt}%
\pgfpathmoveto{\pgfqpoint{1.318408in}{2.105338in}}%
\pgfpathcurveto{\pgfqpoint{1.326644in}{2.105338in}}{\pgfqpoint{1.334544in}{2.108610in}}{\pgfqpoint{1.340368in}{2.114434in}}%
\pgfpathcurveto{\pgfqpoint{1.346192in}{2.120258in}}{\pgfqpoint{1.349465in}{2.128158in}}{\pgfqpoint{1.349465in}{2.136394in}}%
\pgfpathcurveto{\pgfqpoint{1.349465in}{2.144631in}}{\pgfqpoint{1.346192in}{2.152531in}}{\pgfqpoint{1.340368in}{2.158355in}}%
\pgfpathcurveto{\pgfqpoint{1.334544in}{2.164178in}}{\pgfqpoint{1.326644in}{2.167451in}}{\pgfqpoint{1.318408in}{2.167451in}}%
\pgfpathcurveto{\pgfqpoint{1.310172in}{2.167451in}}{\pgfqpoint{1.302272in}{2.164178in}}{\pgfqpoint{1.296448in}{2.158355in}}%
\pgfpathcurveto{\pgfqpoint{1.290624in}{2.152531in}}{\pgfqpoint{1.287352in}{2.144631in}}{\pgfqpoint{1.287352in}{2.136394in}}%
\pgfpathcurveto{\pgfqpoint{1.287352in}{2.128158in}}{\pgfqpoint{1.290624in}{2.120258in}}{\pgfqpoint{1.296448in}{2.114434in}}%
\pgfpathcurveto{\pgfqpoint{1.302272in}{2.108610in}}{\pgfqpoint{1.310172in}{2.105338in}}{\pgfqpoint{1.318408in}{2.105338in}}%
\pgfpathclose%
\pgfusepath{stroke,fill}%
\end{pgfscope}%
\begin{pgfscope}%
\pgfpathrectangle{\pgfqpoint{0.100000in}{0.212622in}}{\pgfqpoint{3.696000in}{3.696000in}}%
\pgfusepath{clip}%
\pgfsetbuttcap%
\pgfsetroundjoin%
\definecolor{currentfill}{rgb}{0.121569,0.466667,0.705882}%
\pgfsetfillcolor{currentfill}%
\pgfsetfillopacity{0.425443}%
\pgfsetlinewidth{1.003750pt}%
\definecolor{currentstroke}{rgb}{0.121569,0.466667,0.705882}%
\pgfsetstrokecolor{currentstroke}%
\pgfsetstrokeopacity{0.425443}%
\pgfsetdash{}{0pt}%
\pgfpathmoveto{\pgfqpoint{2.490704in}{2.402058in}}%
\pgfpathcurveto{\pgfqpoint{2.498940in}{2.402058in}}{\pgfqpoint{2.506840in}{2.405331in}}{\pgfqpoint{2.512664in}{2.411155in}}%
\pgfpathcurveto{\pgfqpoint{2.518488in}{2.416979in}}{\pgfqpoint{2.521760in}{2.424879in}}{\pgfqpoint{2.521760in}{2.433115in}}%
\pgfpathcurveto{\pgfqpoint{2.521760in}{2.441351in}}{\pgfqpoint{2.518488in}{2.449251in}}{\pgfqpoint{2.512664in}{2.455075in}}%
\pgfpathcurveto{\pgfqpoint{2.506840in}{2.460899in}}{\pgfqpoint{2.498940in}{2.464171in}}{\pgfqpoint{2.490704in}{2.464171in}}%
\pgfpathcurveto{\pgfqpoint{2.482467in}{2.464171in}}{\pgfqpoint{2.474567in}{2.460899in}}{\pgfqpoint{2.468743in}{2.455075in}}%
\pgfpathcurveto{\pgfqpoint{2.462919in}{2.449251in}}{\pgfqpoint{2.459647in}{2.441351in}}{\pgfqpoint{2.459647in}{2.433115in}}%
\pgfpathcurveto{\pgfqpoint{2.459647in}{2.424879in}}{\pgfqpoint{2.462919in}{2.416979in}}{\pgfqpoint{2.468743in}{2.411155in}}%
\pgfpathcurveto{\pgfqpoint{2.474567in}{2.405331in}}{\pgfqpoint{2.482467in}{2.402058in}}{\pgfqpoint{2.490704in}{2.402058in}}%
\pgfpathclose%
\pgfusepath{stroke,fill}%
\end{pgfscope}%
\begin{pgfscope}%
\pgfpathrectangle{\pgfqpoint{0.100000in}{0.212622in}}{\pgfqpoint{3.696000in}{3.696000in}}%
\pgfusepath{clip}%
\pgfsetbuttcap%
\pgfsetroundjoin%
\definecolor{currentfill}{rgb}{0.121569,0.466667,0.705882}%
\pgfsetfillcolor{currentfill}%
\pgfsetfillopacity{0.426627}%
\pgfsetlinewidth{1.003750pt}%
\definecolor{currentstroke}{rgb}{0.121569,0.466667,0.705882}%
\pgfsetstrokecolor{currentstroke}%
\pgfsetstrokeopacity{0.426627}%
\pgfsetdash{}{0pt}%
\pgfpathmoveto{\pgfqpoint{1.312187in}{2.097437in}}%
\pgfpathcurveto{\pgfqpoint{1.320423in}{2.097437in}}{\pgfqpoint{1.328323in}{2.100709in}}{\pgfqpoint{1.334147in}{2.106533in}}%
\pgfpathcurveto{\pgfqpoint{1.339971in}{2.112357in}}{\pgfqpoint{1.343244in}{2.120257in}}{\pgfqpoint{1.343244in}{2.128493in}}%
\pgfpathcurveto{\pgfqpoint{1.343244in}{2.136729in}}{\pgfqpoint{1.339971in}{2.144629in}}{\pgfqpoint{1.334147in}{2.150453in}}%
\pgfpathcurveto{\pgfqpoint{1.328323in}{2.156277in}}{\pgfqpoint{1.320423in}{2.159550in}}{\pgfqpoint{1.312187in}{2.159550in}}%
\pgfpathcurveto{\pgfqpoint{1.303951in}{2.159550in}}{\pgfqpoint{1.296051in}{2.156277in}}{\pgfqpoint{1.290227in}{2.150453in}}%
\pgfpathcurveto{\pgfqpoint{1.284403in}{2.144629in}}{\pgfqpoint{1.281131in}{2.136729in}}{\pgfqpoint{1.281131in}{2.128493in}}%
\pgfpathcurveto{\pgfqpoint{1.281131in}{2.120257in}}{\pgfqpoint{1.284403in}{2.112357in}}{\pgfqpoint{1.290227in}{2.106533in}}%
\pgfpathcurveto{\pgfqpoint{1.296051in}{2.100709in}}{\pgfqpoint{1.303951in}{2.097437in}}{\pgfqpoint{1.312187in}{2.097437in}}%
\pgfpathclose%
\pgfusepath{stroke,fill}%
\end{pgfscope}%
\begin{pgfscope}%
\pgfpathrectangle{\pgfqpoint{0.100000in}{0.212622in}}{\pgfqpoint{3.696000in}{3.696000in}}%
\pgfusepath{clip}%
\pgfsetbuttcap%
\pgfsetroundjoin%
\definecolor{currentfill}{rgb}{0.121569,0.466667,0.705882}%
\pgfsetfillcolor{currentfill}%
\pgfsetfillopacity{0.427619}%
\pgfsetlinewidth{1.003750pt}%
\definecolor{currentstroke}{rgb}{0.121569,0.466667,0.705882}%
\pgfsetstrokecolor{currentstroke}%
\pgfsetstrokeopacity{0.427619}%
\pgfsetdash{}{0pt}%
\pgfpathmoveto{\pgfqpoint{2.503940in}{2.397512in}}%
\pgfpathcurveto{\pgfqpoint{2.512176in}{2.397512in}}{\pgfqpoint{2.520076in}{2.400784in}}{\pgfqpoint{2.525900in}{2.406608in}}%
\pgfpathcurveto{\pgfqpoint{2.531724in}{2.412432in}}{\pgfqpoint{2.534996in}{2.420332in}}{\pgfqpoint{2.534996in}{2.428568in}}%
\pgfpathcurveto{\pgfqpoint{2.534996in}{2.436805in}}{\pgfqpoint{2.531724in}{2.444705in}}{\pgfqpoint{2.525900in}{2.450529in}}%
\pgfpathcurveto{\pgfqpoint{2.520076in}{2.456353in}}{\pgfqpoint{2.512176in}{2.459625in}}{\pgfqpoint{2.503940in}{2.459625in}}%
\pgfpathcurveto{\pgfqpoint{2.495703in}{2.459625in}}{\pgfqpoint{2.487803in}{2.456353in}}{\pgfqpoint{2.481979in}{2.450529in}}%
\pgfpathcurveto{\pgfqpoint{2.476156in}{2.444705in}}{\pgfqpoint{2.472883in}{2.436805in}}{\pgfqpoint{2.472883in}{2.428568in}}%
\pgfpathcurveto{\pgfqpoint{2.472883in}{2.420332in}}{\pgfqpoint{2.476156in}{2.412432in}}{\pgfqpoint{2.481979in}{2.406608in}}%
\pgfpathcurveto{\pgfqpoint{2.487803in}{2.400784in}}{\pgfqpoint{2.495703in}{2.397512in}}{\pgfqpoint{2.503940in}{2.397512in}}%
\pgfpathclose%
\pgfusepath{stroke,fill}%
\end{pgfscope}%
\begin{pgfscope}%
\pgfpathrectangle{\pgfqpoint{0.100000in}{0.212622in}}{\pgfqpoint{3.696000in}{3.696000in}}%
\pgfusepath{clip}%
\pgfsetbuttcap%
\pgfsetroundjoin%
\definecolor{currentfill}{rgb}{0.121569,0.466667,0.705882}%
\pgfsetfillcolor{currentfill}%
\pgfsetfillopacity{0.428356}%
\pgfsetlinewidth{1.003750pt}%
\definecolor{currentstroke}{rgb}{0.121569,0.466667,0.705882}%
\pgfsetstrokecolor{currentstroke}%
\pgfsetstrokeopacity{0.428356}%
\pgfsetdash{}{0pt}%
\pgfpathmoveto{\pgfqpoint{1.307192in}{2.091309in}}%
\pgfpathcurveto{\pgfqpoint{1.315429in}{2.091309in}}{\pgfqpoint{1.323329in}{2.094582in}}{\pgfqpoint{1.329153in}{2.100406in}}%
\pgfpathcurveto{\pgfqpoint{1.334977in}{2.106229in}}{\pgfqpoint{1.338249in}{2.114129in}}{\pgfqpoint{1.338249in}{2.122366in}}%
\pgfpathcurveto{\pgfqpoint{1.338249in}{2.130602in}}{\pgfqpoint{1.334977in}{2.138502in}}{\pgfqpoint{1.329153in}{2.144326in}}%
\pgfpathcurveto{\pgfqpoint{1.323329in}{2.150150in}}{\pgfqpoint{1.315429in}{2.153422in}}{\pgfqpoint{1.307192in}{2.153422in}}%
\pgfpathcurveto{\pgfqpoint{1.298956in}{2.153422in}}{\pgfqpoint{1.291056in}{2.150150in}}{\pgfqpoint{1.285232in}{2.144326in}}%
\pgfpathcurveto{\pgfqpoint{1.279408in}{2.138502in}}{\pgfqpoint{1.276136in}{2.130602in}}{\pgfqpoint{1.276136in}{2.122366in}}%
\pgfpathcurveto{\pgfqpoint{1.276136in}{2.114129in}}{\pgfqpoint{1.279408in}{2.106229in}}{\pgfqpoint{1.285232in}{2.100406in}}%
\pgfpathcurveto{\pgfqpoint{1.291056in}{2.094582in}}{\pgfqpoint{1.298956in}{2.091309in}}{\pgfqpoint{1.307192in}{2.091309in}}%
\pgfpathclose%
\pgfusepath{stroke,fill}%
\end{pgfscope}%
\begin{pgfscope}%
\pgfpathrectangle{\pgfqpoint{0.100000in}{0.212622in}}{\pgfqpoint{3.696000in}{3.696000in}}%
\pgfusepath{clip}%
\pgfsetbuttcap%
\pgfsetroundjoin%
\definecolor{currentfill}{rgb}{0.121569,0.466667,0.705882}%
\pgfsetfillcolor{currentfill}%
\pgfsetfillopacity{0.428858}%
\pgfsetlinewidth{1.003750pt}%
\definecolor{currentstroke}{rgb}{0.121569,0.466667,0.705882}%
\pgfsetstrokecolor{currentstroke}%
\pgfsetstrokeopacity{0.428858}%
\pgfsetdash{}{0pt}%
\pgfpathmoveto{\pgfqpoint{2.511318in}{2.395537in}}%
\pgfpathcurveto{\pgfqpoint{2.519554in}{2.395537in}}{\pgfqpoint{2.527454in}{2.398809in}}{\pgfqpoint{2.533278in}{2.404633in}}%
\pgfpathcurveto{\pgfqpoint{2.539102in}{2.410457in}}{\pgfqpoint{2.542374in}{2.418357in}}{\pgfqpoint{2.542374in}{2.426594in}}%
\pgfpathcurveto{\pgfqpoint{2.542374in}{2.434830in}}{\pgfqpoint{2.539102in}{2.442730in}}{\pgfqpoint{2.533278in}{2.448554in}}%
\pgfpathcurveto{\pgfqpoint{2.527454in}{2.454378in}}{\pgfqpoint{2.519554in}{2.457650in}}{\pgfqpoint{2.511318in}{2.457650in}}%
\pgfpathcurveto{\pgfqpoint{2.503082in}{2.457650in}}{\pgfqpoint{2.495182in}{2.454378in}}{\pgfqpoint{2.489358in}{2.448554in}}%
\pgfpathcurveto{\pgfqpoint{2.483534in}{2.442730in}}{\pgfqpoint{2.480261in}{2.434830in}}{\pgfqpoint{2.480261in}{2.426594in}}%
\pgfpathcurveto{\pgfqpoint{2.480261in}{2.418357in}}{\pgfqpoint{2.483534in}{2.410457in}}{\pgfqpoint{2.489358in}{2.404633in}}%
\pgfpathcurveto{\pgfqpoint{2.495182in}{2.398809in}}{\pgfqpoint{2.503082in}{2.395537in}}{\pgfqpoint{2.511318in}{2.395537in}}%
\pgfpathclose%
\pgfusepath{stroke,fill}%
\end{pgfscope}%
\begin{pgfscope}%
\pgfpathrectangle{\pgfqpoint{0.100000in}{0.212622in}}{\pgfqpoint{3.696000in}{3.696000in}}%
\pgfusepath{clip}%
\pgfsetbuttcap%
\pgfsetroundjoin%
\definecolor{currentfill}{rgb}{0.121569,0.466667,0.705882}%
\pgfsetfillcolor{currentfill}%
\pgfsetfillopacity{0.429822}%
\pgfsetlinewidth{1.003750pt}%
\definecolor{currentstroke}{rgb}{0.121569,0.466667,0.705882}%
\pgfsetstrokecolor{currentstroke}%
\pgfsetstrokeopacity{0.429822}%
\pgfsetdash{}{0pt}%
\pgfpathmoveto{\pgfqpoint{1.303081in}{2.086259in}}%
\pgfpathcurveto{\pgfqpoint{1.311317in}{2.086259in}}{\pgfqpoint{1.319217in}{2.089532in}}{\pgfqpoint{1.325041in}{2.095356in}}%
\pgfpathcurveto{\pgfqpoint{1.330865in}{2.101179in}}{\pgfqpoint{1.334137in}{2.109080in}}{\pgfqpoint{1.334137in}{2.117316in}}%
\pgfpathcurveto{\pgfqpoint{1.334137in}{2.125552in}}{\pgfqpoint{1.330865in}{2.133452in}}{\pgfqpoint{1.325041in}{2.139276in}}%
\pgfpathcurveto{\pgfqpoint{1.319217in}{2.145100in}}{\pgfqpoint{1.311317in}{2.148372in}}{\pgfqpoint{1.303081in}{2.148372in}}%
\pgfpathcurveto{\pgfqpoint{1.294845in}{2.148372in}}{\pgfqpoint{1.286944in}{2.145100in}}{\pgfqpoint{1.281121in}{2.139276in}}%
\pgfpathcurveto{\pgfqpoint{1.275297in}{2.133452in}}{\pgfqpoint{1.272024in}{2.125552in}}{\pgfqpoint{1.272024in}{2.117316in}}%
\pgfpathcurveto{\pgfqpoint{1.272024in}{2.109080in}}{\pgfqpoint{1.275297in}{2.101179in}}{\pgfqpoint{1.281121in}{2.095356in}}%
\pgfpathcurveto{\pgfqpoint{1.286944in}{2.089532in}}{\pgfqpoint{1.294845in}{2.086259in}}{\pgfqpoint{1.303081in}{2.086259in}}%
\pgfpathclose%
\pgfusepath{stroke,fill}%
\end{pgfscope}%
\begin{pgfscope}%
\pgfpathrectangle{\pgfqpoint{0.100000in}{0.212622in}}{\pgfqpoint{3.696000in}{3.696000in}}%
\pgfusepath{clip}%
\pgfsetbuttcap%
\pgfsetroundjoin%
\definecolor{currentfill}{rgb}{0.121569,0.466667,0.705882}%
\pgfsetfillcolor{currentfill}%
\pgfsetfillopacity{0.430268}%
\pgfsetlinewidth{1.003750pt}%
\definecolor{currentstroke}{rgb}{0.121569,0.466667,0.705882}%
\pgfsetstrokecolor{currentstroke}%
\pgfsetstrokeopacity{0.430268}%
\pgfsetdash{}{0pt}%
\pgfpathmoveto{\pgfqpoint{2.519589in}{2.393977in}}%
\pgfpathcurveto{\pgfqpoint{2.527825in}{2.393977in}}{\pgfqpoint{2.535725in}{2.397250in}}{\pgfqpoint{2.541549in}{2.403073in}}%
\pgfpathcurveto{\pgfqpoint{2.547373in}{2.408897in}}{\pgfqpoint{2.550646in}{2.416797in}}{\pgfqpoint{2.550646in}{2.425034in}}%
\pgfpathcurveto{\pgfqpoint{2.550646in}{2.433270in}}{\pgfqpoint{2.547373in}{2.441170in}}{\pgfqpoint{2.541549in}{2.446994in}}%
\pgfpathcurveto{\pgfqpoint{2.535725in}{2.452818in}}{\pgfqpoint{2.527825in}{2.456090in}}{\pgfqpoint{2.519589in}{2.456090in}}%
\pgfpathcurveto{\pgfqpoint{2.511353in}{2.456090in}}{\pgfqpoint{2.503453in}{2.452818in}}{\pgfqpoint{2.497629in}{2.446994in}}%
\pgfpathcurveto{\pgfqpoint{2.491805in}{2.441170in}}{\pgfqpoint{2.488533in}{2.433270in}}{\pgfqpoint{2.488533in}{2.425034in}}%
\pgfpathcurveto{\pgfqpoint{2.488533in}{2.416797in}}{\pgfqpoint{2.491805in}{2.408897in}}{\pgfqpoint{2.497629in}{2.403073in}}%
\pgfpathcurveto{\pgfqpoint{2.503453in}{2.397250in}}{\pgfqpoint{2.511353in}{2.393977in}}{\pgfqpoint{2.519589in}{2.393977in}}%
\pgfpathclose%
\pgfusepath{stroke,fill}%
\end{pgfscope}%
\begin{pgfscope}%
\pgfpathrectangle{\pgfqpoint{0.100000in}{0.212622in}}{\pgfqpoint{3.696000in}{3.696000in}}%
\pgfusepath{clip}%
\pgfsetbuttcap%
\pgfsetroundjoin%
\definecolor{currentfill}{rgb}{0.121569,0.466667,0.705882}%
\pgfsetfillcolor{currentfill}%
\pgfsetfillopacity{0.430972}%
\pgfsetlinewidth{1.003750pt}%
\definecolor{currentstroke}{rgb}{0.121569,0.466667,0.705882}%
\pgfsetstrokecolor{currentstroke}%
\pgfsetstrokeopacity{0.430972}%
\pgfsetdash{}{0pt}%
\pgfpathmoveto{\pgfqpoint{1.299705in}{2.082283in}}%
\pgfpathcurveto{\pgfqpoint{1.307941in}{2.082283in}}{\pgfqpoint{1.315841in}{2.085556in}}{\pgfqpoint{1.321665in}{2.091379in}}%
\pgfpathcurveto{\pgfqpoint{1.327489in}{2.097203in}}{\pgfqpoint{1.330761in}{2.105103in}}{\pgfqpoint{1.330761in}{2.113340in}}%
\pgfpathcurveto{\pgfqpoint{1.330761in}{2.121576in}}{\pgfqpoint{1.327489in}{2.129476in}}{\pgfqpoint{1.321665in}{2.135300in}}%
\pgfpathcurveto{\pgfqpoint{1.315841in}{2.141124in}}{\pgfqpoint{1.307941in}{2.144396in}}{\pgfqpoint{1.299705in}{2.144396in}}%
\pgfpathcurveto{\pgfqpoint{1.291468in}{2.144396in}}{\pgfqpoint{1.283568in}{2.141124in}}{\pgfqpoint{1.277744in}{2.135300in}}%
\pgfpathcurveto{\pgfqpoint{1.271920in}{2.129476in}}{\pgfqpoint{1.268648in}{2.121576in}}{\pgfqpoint{1.268648in}{2.113340in}}%
\pgfpathcurveto{\pgfqpoint{1.268648in}{2.105103in}}{\pgfqpoint{1.271920in}{2.097203in}}{\pgfqpoint{1.277744in}{2.091379in}}%
\pgfpathcurveto{\pgfqpoint{1.283568in}{2.085556in}}{\pgfqpoint{1.291468in}{2.082283in}}{\pgfqpoint{1.299705in}{2.082283in}}%
\pgfpathclose%
\pgfusepath{stroke,fill}%
\end{pgfscope}%
\begin{pgfscope}%
\pgfpathrectangle{\pgfqpoint{0.100000in}{0.212622in}}{\pgfqpoint{3.696000in}{3.696000in}}%
\pgfusepath{clip}%
\pgfsetbuttcap%
\pgfsetroundjoin%
\definecolor{currentfill}{rgb}{0.121569,0.466667,0.705882}%
\pgfsetfillcolor{currentfill}%
\pgfsetfillopacity{0.431929}%
\pgfsetlinewidth{1.003750pt}%
\definecolor{currentstroke}{rgb}{0.121569,0.466667,0.705882}%
\pgfsetstrokecolor{currentstroke}%
\pgfsetstrokeopacity{0.431929}%
\pgfsetdash{}{0pt}%
\pgfpathmoveto{\pgfqpoint{2.529615in}{2.391721in}}%
\pgfpathcurveto{\pgfqpoint{2.537851in}{2.391721in}}{\pgfqpoint{2.545751in}{2.394993in}}{\pgfqpoint{2.551575in}{2.400817in}}%
\pgfpathcurveto{\pgfqpoint{2.557399in}{2.406641in}}{\pgfqpoint{2.560672in}{2.414541in}}{\pgfqpoint{2.560672in}{2.422778in}}%
\pgfpathcurveto{\pgfqpoint{2.560672in}{2.431014in}}{\pgfqpoint{2.557399in}{2.438914in}}{\pgfqpoint{2.551575in}{2.444738in}}%
\pgfpathcurveto{\pgfqpoint{2.545751in}{2.450562in}}{\pgfqpoint{2.537851in}{2.453834in}}{\pgfqpoint{2.529615in}{2.453834in}}%
\pgfpathcurveto{\pgfqpoint{2.521379in}{2.453834in}}{\pgfqpoint{2.513479in}{2.450562in}}{\pgfqpoint{2.507655in}{2.444738in}}%
\pgfpathcurveto{\pgfqpoint{2.501831in}{2.438914in}}{\pgfqpoint{2.498559in}{2.431014in}}{\pgfqpoint{2.498559in}{2.422778in}}%
\pgfpathcurveto{\pgfqpoint{2.498559in}{2.414541in}}{\pgfqpoint{2.501831in}{2.406641in}}{\pgfqpoint{2.507655in}{2.400817in}}%
\pgfpathcurveto{\pgfqpoint{2.513479in}{2.394993in}}{\pgfqpoint{2.521379in}{2.391721in}}{\pgfqpoint{2.529615in}{2.391721in}}%
\pgfpathclose%
\pgfusepath{stroke,fill}%
\end{pgfscope}%
\begin{pgfscope}%
\pgfpathrectangle{\pgfqpoint{0.100000in}{0.212622in}}{\pgfqpoint{3.696000in}{3.696000in}}%
\pgfusepath{clip}%
\pgfsetbuttcap%
\pgfsetroundjoin%
\definecolor{currentfill}{rgb}{0.121569,0.466667,0.705882}%
\pgfsetfillcolor{currentfill}%
\pgfsetfillopacity{0.432723}%
\pgfsetlinewidth{1.003750pt}%
\definecolor{currentstroke}{rgb}{0.121569,0.466667,0.705882}%
\pgfsetstrokecolor{currentstroke}%
\pgfsetstrokeopacity{0.432723}%
\pgfsetdash{}{0pt}%
\pgfpathmoveto{\pgfqpoint{2.535169in}{2.389804in}}%
\pgfpathcurveto{\pgfqpoint{2.543406in}{2.389804in}}{\pgfqpoint{2.551306in}{2.393076in}}{\pgfqpoint{2.557130in}{2.398900in}}%
\pgfpathcurveto{\pgfqpoint{2.562954in}{2.404724in}}{\pgfqpoint{2.566226in}{2.412624in}}{\pgfqpoint{2.566226in}{2.420860in}}%
\pgfpathcurveto{\pgfqpoint{2.566226in}{2.429097in}}{\pgfqpoint{2.562954in}{2.436997in}}{\pgfqpoint{2.557130in}{2.442821in}}%
\pgfpathcurveto{\pgfqpoint{2.551306in}{2.448645in}}{\pgfqpoint{2.543406in}{2.451917in}}{\pgfqpoint{2.535169in}{2.451917in}}%
\pgfpathcurveto{\pgfqpoint{2.526933in}{2.451917in}}{\pgfqpoint{2.519033in}{2.448645in}}{\pgfqpoint{2.513209in}{2.442821in}}%
\pgfpathcurveto{\pgfqpoint{2.507385in}{2.436997in}}{\pgfqpoint{2.504113in}{2.429097in}}{\pgfqpoint{2.504113in}{2.420860in}}%
\pgfpathcurveto{\pgfqpoint{2.504113in}{2.412624in}}{\pgfqpoint{2.507385in}{2.404724in}}{\pgfqpoint{2.513209in}{2.398900in}}%
\pgfpathcurveto{\pgfqpoint{2.519033in}{2.393076in}}{\pgfqpoint{2.526933in}{2.389804in}}{\pgfqpoint{2.535169in}{2.389804in}}%
\pgfpathclose%
\pgfusepath{stroke,fill}%
\end{pgfscope}%
\begin{pgfscope}%
\pgfpathrectangle{\pgfqpoint{0.100000in}{0.212622in}}{\pgfqpoint{3.696000in}{3.696000in}}%
\pgfusepath{clip}%
\pgfsetbuttcap%
\pgfsetroundjoin%
\definecolor{currentfill}{rgb}{0.121569,0.466667,0.705882}%
\pgfsetfillcolor{currentfill}%
\pgfsetfillopacity{0.433105}%
\pgfsetlinewidth{1.003750pt}%
\definecolor{currentstroke}{rgb}{0.121569,0.466667,0.705882}%
\pgfsetstrokecolor{currentstroke}%
\pgfsetstrokeopacity{0.433105}%
\pgfsetdash{}{0pt}%
\pgfpathmoveto{\pgfqpoint{1.293735in}{2.075140in}}%
\pgfpathcurveto{\pgfqpoint{1.301971in}{2.075140in}}{\pgfqpoint{1.309871in}{2.078412in}}{\pgfqpoint{1.315695in}{2.084236in}}%
\pgfpathcurveto{\pgfqpoint{1.321519in}{2.090060in}}{\pgfqpoint{1.324791in}{2.097960in}}{\pgfqpoint{1.324791in}{2.106196in}}%
\pgfpathcurveto{\pgfqpoint{1.324791in}{2.114433in}}{\pgfqpoint{1.321519in}{2.122333in}}{\pgfqpoint{1.315695in}{2.128157in}}%
\pgfpathcurveto{\pgfqpoint{1.309871in}{2.133980in}}{\pgfqpoint{1.301971in}{2.137253in}}{\pgfqpoint{1.293735in}{2.137253in}}%
\pgfpathcurveto{\pgfqpoint{1.285498in}{2.137253in}}{\pgfqpoint{1.277598in}{2.133980in}}{\pgfqpoint{1.271774in}{2.128157in}}%
\pgfpathcurveto{\pgfqpoint{1.265951in}{2.122333in}}{\pgfqpoint{1.262678in}{2.114433in}}{\pgfqpoint{1.262678in}{2.106196in}}%
\pgfpathcurveto{\pgfqpoint{1.262678in}{2.097960in}}{\pgfqpoint{1.265951in}{2.090060in}}{\pgfqpoint{1.271774in}{2.084236in}}%
\pgfpathcurveto{\pgfqpoint{1.277598in}{2.078412in}}{\pgfqpoint{1.285498in}{2.075140in}}{\pgfqpoint{1.293735in}{2.075140in}}%
\pgfpathclose%
\pgfusepath{stroke,fill}%
\end{pgfscope}%
\begin{pgfscope}%
\pgfpathrectangle{\pgfqpoint{0.100000in}{0.212622in}}{\pgfqpoint{3.696000in}{3.696000in}}%
\pgfusepath{clip}%
\pgfsetbuttcap%
\pgfsetroundjoin%
\definecolor{currentfill}{rgb}{0.121569,0.466667,0.705882}%
\pgfsetfillcolor{currentfill}%
\pgfsetfillopacity{0.433763}%
\pgfsetlinewidth{1.003750pt}%
\definecolor{currentstroke}{rgb}{0.121569,0.466667,0.705882}%
\pgfsetstrokecolor{currentstroke}%
\pgfsetstrokeopacity{0.433763}%
\pgfsetdash{}{0pt}%
\pgfpathmoveto{\pgfqpoint{2.541809in}{2.388518in}}%
\pgfpathcurveto{\pgfqpoint{2.550045in}{2.388518in}}{\pgfqpoint{2.557945in}{2.391790in}}{\pgfqpoint{2.563769in}{2.397614in}}%
\pgfpathcurveto{\pgfqpoint{2.569593in}{2.403438in}}{\pgfqpoint{2.572865in}{2.411338in}}{\pgfqpoint{2.572865in}{2.419574in}}%
\pgfpathcurveto{\pgfqpoint{2.572865in}{2.427810in}}{\pgfqpoint{2.569593in}{2.435710in}}{\pgfqpoint{2.563769in}{2.441534in}}%
\pgfpathcurveto{\pgfqpoint{2.557945in}{2.447358in}}{\pgfqpoint{2.550045in}{2.450631in}}{\pgfqpoint{2.541809in}{2.450631in}}%
\pgfpathcurveto{\pgfqpoint{2.533572in}{2.450631in}}{\pgfqpoint{2.525672in}{2.447358in}}{\pgfqpoint{2.519848in}{2.441534in}}%
\pgfpathcurveto{\pgfqpoint{2.514024in}{2.435710in}}{\pgfqpoint{2.510752in}{2.427810in}}{\pgfqpoint{2.510752in}{2.419574in}}%
\pgfpathcurveto{\pgfqpoint{2.510752in}{2.411338in}}{\pgfqpoint{2.514024in}{2.403438in}}{\pgfqpoint{2.519848in}{2.397614in}}%
\pgfpathcurveto{\pgfqpoint{2.525672in}{2.391790in}}{\pgfqpoint{2.533572in}{2.388518in}}{\pgfqpoint{2.541809in}{2.388518in}}%
\pgfpathclose%
\pgfusepath{stroke,fill}%
\end{pgfscope}%
\begin{pgfscope}%
\pgfpathrectangle{\pgfqpoint{0.100000in}{0.212622in}}{\pgfqpoint{3.696000in}{3.696000in}}%
\pgfusepath{clip}%
\pgfsetbuttcap%
\pgfsetroundjoin%
\definecolor{currentfill}{rgb}{0.121569,0.466667,0.705882}%
\pgfsetfillcolor{currentfill}%
\pgfsetfillopacity{0.435081}%
\pgfsetlinewidth{1.003750pt}%
\definecolor{currentstroke}{rgb}{0.121569,0.466667,0.705882}%
\pgfsetstrokecolor{currentstroke}%
\pgfsetstrokeopacity{0.435081}%
\pgfsetdash{}{0pt}%
\pgfpathmoveto{\pgfqpoint{2.550300in}{2.386675in}}%
\pgfpathcurveto{\pgfqpoint{2.558536in}{2.386675in}}{\pgfqpoint{2.566436in}{2.389947in}}{\pgfqpoint{2.572260in}{2.395771in}}%
\pgfpathcurveto{\pgfqpoint{2.578084in}{2.401595in}}{\pgfqpoint{2.581356in}{2.409495in}}{\pgfqpoint{2.581356in}{2.417731in}}%
\pgfpathcurveto{\pgfqpoint{2.581356in}{2.425968in}}{\pgfqpoint{2.578084in}{2.433868in}}{\pgfqpoint{2.572260in}{2.439692in}}%
\pgfpathcurveto{\pgfqpoint{2.566436in}{2.445516in}}{\pgfqpoint{2.558536in}{2.448788in}}{\pgfqpoint{2.550300in}{2.448788in}}%
\pgfpathcurveto{\pgfqpoint{2.542064in}{2.448788in}}{\pgfqpoint{2.534164in}{2.445516in}}{\pgfqpoint{2.528340in}{2.439692in}}%
\pgfpathcurveto{\pgfqpoint{2.522516in}{2.433868in}}{\pgfqpoint{2.519243in}{2.425968in}}{\pgfqpoint{2.519243in}{2.417731in}}%
\pgfpathcurveto{\pgfqpoint{2.519243in}{2.409495in}}{\pgfqpoint{2.522516in}{2.401595in}}{\pgfqpoint{2.528340in}{2.395771in}}%
\pgfpathcurveto{\pgfqpoint{2.534164in}{2.389947in}}{\pgfqpoint{2.542064in}{2.386675in}}{\pgfqpoint{2.550300in}{2.386675in}}%
\pgfpathclose%
\pgfusepath{stroke,fill}%
\end{pgfscope}%
\begin{pgfscope}%
\pgfpathrectangle{\pgfqpoint{0.100000in}{0.212622in}}{\pgfqpoint{3.696000in}{3.696000in}}%
\pgfusepath{clip}%
\pgfsetbuttcap%
\pgfsetroundjoin%
\definecolor{currentfill}{rgb}{0.121569,0.466667,0.705882}%
\pgfsetfillcolor{currentfill}%
\pgfsetfillopacity{0.436889}%
\pgfsetlinewidth{1.003750pt}%
\definecolor{currentstroke}{rgb}{0.121569,0.466667,0.705882}%
\pgfsetstrokecolor{currentstroke}%
\pgfsetstrokeopacity{0.436889}%
\pgfsetdash{}{0pt}%
\pgfpathmoveto{\pgfqpoint{1.282669in}{2.061693in}}%
\pgfpathcurveto{\pgfqpoint{1.290905in}{2.061693in}}{\pgfqpoint{1.298805in}{2.064966in}}{\pgfqpoint{1.304629in}{2.070790in}}%
\pgfpathcurveto{\pgfqpoint{1.310453in}{2.076614in}}{\pgfqpoint{1.313725in}{2.084514in}}{\pgfqpoint{1.313725in}{2.092750in}}%
\pgfpathcurveto{\pgfqpoint{1.313725in}{2.100986in}}{\pgfqpoint{1.310453in}{2.108886in}}{\pgfqpoint{1.304629in}{2.114710in}}%
\pgfpathcurveto{\pgfqpoint{1.298805in}{2.120534in}}{\pgfqpoint{1.290905in}{2.123806in}}{\pgfqpoint{1.282669in}{2.123806in}}%
\pgfpathcurveto{\pgfqpoint{1.274433in}{2.123806in}}{\pgfqpoint{1.266533in}{2.120534in}}{\pgfqpoint{1.260709in}{2.114710in}}%
\pgfpathcurveto{\pgfqpoint{1.254885in}{2.108886in}}{\pgfqpoint{1.251612in}{2.100986in}}{\pgfqpoint{1.251612in}{2.092750in}}%
\pgfpathcurveto{\pgfqpoint{1.251612in}{2.084514in}}{\pgfqpoint{1.254885in}{2.076614in}}{\pgfqpoint{1.260709in}{2.070790in}}%
\pgfpathcurveto{\pgfqpoint{1.266533in}{2.064966in}}{\pgfqpoint{1.274433in}{2.061693in}}{\pgfqpoint{1.282669in}{2.061693in}}%
\pgfpathclose%
\pgfusepath{stroke,fill}%
\end{pgfscope}%
\begin{pgfscope}%
\pgfpathrectangle{\pgfqpoint{0.100000in}{0.212622in}}{\pgfqpoint{3.696000in}{3.696000in}}%
\pgfusepath{clip}%
\pgfsetbuttcap%
\pgfsetroundjoin%
\definecolor{currentfill}{rgb}{0.121569,0.466667,0.705882}%
\pgfsetfillcolor{currentfill}%
\pgfsetfillopacity{0.436960}%
\pgfsetlinewidth{1.003750pt}%
\definecolor{currentstroke}{rgb}{0.121569,0.466667,0.705882}%
\pgfsetstrokecolor{currentstroke}%
\pgfsetstrokeopacity{0.436960}%
\pgfsetdash{}{0pt}%
\pgfpathmoveto{\pgfqpoint{2.561242in}{2.386414in}}%
\pgfpathcurveto{\pgfqpoint{2.569479in}{2.386414in}}{\pgfqpoint{2.577379in}{2.389686in}}{\pgfqpoint{2.583203in}{2.395510in}}%
\pgfpathcurveto{\pgfqpoint{2.589027in}{2.401334in}}{\pgfqpoint{2.592299in}{2.409234in}}{\pgfqpoint{2.592299in}{2.417471in}}%
\pgfpathcurveto{\pgfqpoint{2.592299in}{2.425707in}}{\pgfqpoint{2.589027in}{2.433607in}}{\pgfqpoint{2.583203in}{2.439431in}}%
\pgfpathcurveto{\pgfqpoint{2.577379in}{2.445255in}}{\pgfqpoint{2.569479in}{2.448527in}}{\pgfqpoint{2.561242in}{2.448527in}}%
\pgfpathcurveto{\pgfqpoint{2.553006in}{2.448527in}}{\pgfqpoint{2.545106in}{2.445255in}}{\pgfqpoint{2.539282in}{2.439431in}}%
\pgfpathcurveto{\pgfqpoint{2.533458in}{2.433607in}}{\pgfqpoint{2.530186in}{2.425707in}}{\pgfqpoint{2.530186in}{2.417471in}}%
\pgfpathcurveto{\pgfqpoint{2.530186in}{2.409234in}}{\pgfqpoint{2.533458in}{2.401334in}}{\pgfqpoint{2.539282in}{2.395510in}}%
\pgfpathcurveto{\pgfqpoint{2.545106in}{2.389686in}}{\pgfqpoint{2.553006in}{2.386414in}}{\pgfqpoint{2.561242in}{2.386414in}}%
\pgfpathclose%
\pgfusepath{stroke,fill}%
\end{pgfscope}%
\begin{pgfscope}%
\pgfpathrectangle{\pgfqpoint{0.100000in}{0.212622in}}{\pgfqpoint{3.696000in}{3.696000in}}%
\pgfusepath{clip}%
\pgfsetbuttcap%
\pgfsetroundjoin%
\definecolor{currentfill}{rgb}{0.121569,0.466667,0.705882}%
\pgfsetfillcolor{currentfill}%
\pgfsetfillopacity{0.439262}%
\pgfsetlinewidth{1.003750pt}%
\definecolor{currentstroke}{rgb}{0.121569,0.466667,0.705882}%
\pgfsetstrokecolor{currentstroke}%
\pgfsetstrokeopacity{0.439262}%
\pgfsetdash{}{0pt}%
\pgfpathmoveto{\pgfqpoint{2.574813in}{2.385105in}}%
\pgfpathcurveto{\pgfqpoint{2.583049in}{2.385105in}}{\pgfqpoint{2.590949in}{2.388377in}}{\pgfqpoint{2.596773in}{2.394201in}}%
\pgfpathcurveto{\pgfqpoint{2.602597in}{2.400025in}}{\pgfqpoint{2.605870in}{2.407925in}}{\pgfqpoint{2.605870in}{2.416161in}}%
\pgfpathcurveto{\pgfqpoint{2.605870in}{2.424398in}}{\pgfqpoint{2.602597in}{2.432298in}}{\pgfqpoint{2.596773in}{2.438122in}}%
\pgfpathcurveto{\pgfqpoint{2.590949in}{2.443945in}}{\pgfqpoint{2.583049in}{2.447218in}}{\pgfqpoint{2.574813in}{2.447218in}}%
\pgfpathcurveto{\pgfqpoint{2.566577in}{2.447218in}}{\pgfqpoint{2.558677in}{2.443945in}}{\pgfqpoint{2.552853in}{2.438122in}}%
\pgfpathcurveto{\pgfqpoint{2.547029in}{2.432298in}}{\pgfqpoint{2.543757in}{2.424398in}}{\pgfqpoint{2.543757in}{2.416161in}}%
\pgfpathcurveto{\pgfqpoint{2.543757in}{2.407925in}}{\pgfqpoint{2.547029in}{2.400025in}}{\pgfqpoint{2.552853in}{2.394201in}}%
\pgfpathcurveto{\pgfqpoint{2.558677in}{2.388377in}}{\pgfqpoint{2.566577in}{2.385105in}}{\pgfqpoint{2.574813in}{2.385105in}}%
\pgfpathclose%
\pgfusepath{stroke,fill}%
\end{pgfscope}%
\begin{pgfscope}%
\pgfpathrectangle{\pgfqpoint{0.100000in}{0.212622in}}{\pgfqpoint{3.696000in}{3.696000in}}%
\pgfusepath{clip}%
\pgfsetbuttcap%
\pgfsetroundjoin%
\definecolor{currentfill}{rgb}{0.121569,0.466667,0.705882}%
\pgfsetfillcolor{currentfill}%
\pgfsetfillopacity{0.440553}%
\pgfsetlinewidth{1.003750pt}%
\definecolor{currentstroke}{rgb}{0.121569,0.466667,0.705882}%
\pgfsetstrokecolor{currentstroke}%
\pgfsetstrokeopacity{0.440553}%
\pgfsetdash{}{0pt}%
\pgfpathmoveto{\pgfqpoint{1.272185in}{2.049176in}}%
\pgfpathcurveto{\pgfqpoint{1.280421in}{2.049176in}}{\pgfqpoint{1.288321in}{2.052449in}}{\pgfqpoint{1.294145in}{2.058272in}}%
\pgfpathcurveto{\pgfqpoint{1.299969in}{2.064096in}}{\pgfqpoint{1.303242in}{2.071996in}}{\pgfqpoint{1.303242in}{2.080233in}}%
\pgfpathcurveto{\pgfqpoint{1.303242in}{2.088469in}}{\pgfqpoint{1.299969in}{2.096369in}}{\pgfqpoint{1.294145in}{2.102193in}}%
\pgfpathcurveto{\pgfqpoint{1.288321in}{2.108017in}}{\pgfqpoint{1.280421in}{2.111289in}}{\pgfqpoint{1.272185in}{2.111289in}}%
\pgfpathcurveto{\pgfqpoint{1.263949in}{2.111289in}}{\pgfqpoint{1.256049in}{2.108017in}}{\pgfqpoint{1.250225in}{2.102193in}}%
\pgfpathcurveto{\pgfqpoint{1.244401in}{2.096369in}}{\pgfqpoint{1.241129in}{2.088469in}}{\pgfqpoint{1.241129in}{2.080233in}}%
\pgfpathcurveto{\pgfqpoint{1.241129in}{2.071996in}}{\pgfqpoint{1.244401in}{2.064096in}}{\pgfqpoint{1.250225in}{2.058272in}}%
\pgfpathcurveto{\pgfqpoint{1.256049in}{2.052449in}}{\pgfqpoint{1.263949in}{2.049176in}}{\pgfqpoint{1.272185in}{2.049176in}}%
\pgfpathclose%
\pgfusepath{stroke,fill}%
\end{pgfscope}%
\begin{pgfscope}%
\pgfpathrectangle{\pgfqpoint{0.100000in}{0.212622in}}{\pgfqpoint{3.696000in}{3.696000in}}%
\pgfusepath{clip}%
\pgfsetbuttcap%
\pgfsetroundjoin%
\definecolor{currentfill}{rgb}{0.121569,0.466667,0.705882}%
\pgfsetfillcolor{currentfill}%
\pgfsetfillopacity{0.441696}%
\pgfsetlinewidth{1.003750pt}%
\definecolor{currentstroke}{rgb}{0.121569,0.466667,0.705882}%
\pgfsetstrokecolor{currentstroke}%
\pgfsetstrokeopacity{0.441696}%
\pgfsetdash{}{0pt}%
\pgfpathmoveto{\pgfqpoint{2.588914in}{2.383439in}}%
\pgfpathcurveto{\pgfqpoint{2.597151in}{2.383439in}}{\pgfqpoint{2.605051in}{2.386711in}}{\pgfqpoint{2.610875in}{2.392535in}}%
\pgfpathcurveto{\pgfqpoint{2.616699in}{2.398359in}}{\pgfqpoint{2.619971in}{2.406259in}}{\pgfqpoint{2.619971in}{2.414496in}}%
\pgfpathcurveto{\pgfqpoint{2.619971in}{2.422732in}}{\pgfqpoint{2.616699in}{2.430632in}}{\pgfqpoint{2.610875in}{2.436456in}}%
\pgfpathcurveto{\pgfqpoint{2.605051in}{2.442280in}}{\pgfqpoint{2.597151in}{2.445552in}}{\pgfqpoint{2.588914in}{2.445552in}}%
\pgfpathcurveto{\pgfqpoint{2.580678in}{2.445552in}}{\pgfqpoint{2.572778in}{2.442280in}}{\pgfqpoint{2.566954in}{2.436456in}}%
\pgfpathcurveto{\pgfqpoint{2.561130in}{2.430632in}}{\pgfqpoint{2.557858in}{2.422732in}}{\pgfqpoint{2.557858in}{2.414496in}}%
\pgfpathcurveto{\pgfqpoint{2.557858in}{2.406259in}}{\pgfqpoint{2.561130in}{2.398359in}}{\pgfqpoint{2.566954in}{2.392535in}}%
\pgfpathcurveto{\pgfqpoint{2.572778in}{2.386711in}}{\pgfqpoint{2.580678in}{2.383439in}}{\pgfqpoint{2.588914in}{2.383439in}}%
\pgfpathclose%
\pgfusepath{stroke,fill}%
\end{pgfscope}%
\begin{pgfscope}%
\pgfpathrectangle{\pgfqpoint{0.100000in}{0.212622in}}{\pgfqpoint{3.696000in}{3.696000in}}%
\pgfusepath{clip}%
\pgfsetbuttcap%
\pgfsetroundjoin%
\definecolor{currentfill}{rgb}{0.121569,0.466667,0.705882}%
\pgfsetfillcolor{currentfill}%
\pgfsetfillopacity{0.443005}%
\pgfsetlinewidth{1.003750pt}%
\definecolor{currentstroke}{rgb}{0.121569,0.466667,0.705882}%
\pgfsetstrokecolor{currentstroke}%
\pgfsetstrokeopacity{0.443005}%
\pgfsetdash{}{0pt}%
\pgfpathmoveto{\pgfqpoint{2.596670in}{2.382301in}}%
\pgfpathcurveto{\pgfqpoint{2.604906in}{2.382301in}}{\pgfqpoint{2.612806in}{2.385574in}}{\pgfqpoint{2.618630in}{2.391398in}}%
\pgfpathcurveto{\pgfqpoint{2.624454in}{2.397222in}}{\pgfqpoint{2.627726in}{2.405122in}}{\pgfqpoint{2.627726in}{2.413358in}}%
\pgfpathcurveto{\pgfqpoint{2.627726in}{2.421594in}}{\pgfqpoint{2.624454in}{2.429494in}}{\pgfqpoint{2.618630in}{2.435318in}}%
\pgfpathcurveto{\pgfqpoint{2.612806in}{2.441142in}}{\pgfqpoint{2.604906in}{2.444414in}}{\pgfqpoint{2.596670in}{2.444414in}}%
\pgfpathcurveto{\pgfqpoint{2.588433in}{2.444414in}}{\pgfqpoint{2.580533in}{2.441142in}}{\pgfqpoint{2.574709in}{2.435318in}}%
\pgfpathcurveto{\pgfqpoint{2.568885in}{2.429494in}}{\pgfqpoint{2.565613in}{2.421594in}}{\pgfqpoint{2.565613in}{2.413358in}}%
\pgfpathcurveto{\pgfqpoint{2.565613in}{2.405122in}}{\pgfqpoint{2.568885in}{2.397222in}}{\pgfqpoint{2.574709in}{2.391398in}}%
\pgfpathcurveto{\pgfqpoint{2.580533in}{2.385574in}}{\pgfqpoint{2.588433in}{2.382301in}}{\pgfqpoint{2.596670in}{2.382301in}}%
\pgfpathclose%
\pgfusepath{stroke,fill}%
\end{pgfscope}%
\begin{pgfscope}%
\pgfpathrectangle{\pgfqpoint{0.100000in}{0.212622in}}{\pgfqpoint{3.696000in}{3.696000in}}%
\pgfusepath{clip}%
\pgfsetbuttcap%
\pgfsetroundjoin%
\definecolor{currentfill}{rgb}{0.121569,0.466667,0.705882}%
\pgfsetfillcolor{currentfill}%
\pgfsetfillopacity{0.444173}%
\pgfsetlinewidth{1.003750pt}%
\definecolor{currentstroke}{rgb}{0.121569,0.466667,0.705882}%
\pgfsetstrokecolor{currentstroke}%
\pgfsetstrokeopacity{0.444173}%
\pgfsetdash{}{0pt}%
\pgfpathmoveto{\pgfqpoint{1.261908in}{2.037452in}}%
\pgfpathcurveto{\pgfqpoint{1.270144in}{2.037452in}}{\pgfqpoint{1.278045in}{2.040725in}}{\pgfqpoint{1.283868in}{2.046549in}}%
\pgfpathcurveto{\pgfqpoint{1.289692in}{2.052373in}}{\pgfqpoint{1.292965in}{2.060273in}}{\pgfqpoint{1.292965in}{2.068509in}}%
\pgfpathcurveto{\pgfqpoint{1.292965in}{2.076745in}}{\pgfqpoint{1.289692in}{2.084645in}}{\pgfqpoint{1.283868in}{2.090469in}}%
\pgfpathcurveto{\pgfqpoint{1.278045in}{2.096293in}}{\pgfqpoint{1.270144in}{2.099565in}}{\pgfqpoint{1.261908in}{2.099565in}}%
\pgfpathcurveto{\pgfqpoint{1.253672in}{2.099565in}}{\pgfqpoint{1.245772in}{2.096293in}}{\pgfqpoint{1.239948in}{2.090469in}}%
\pgfpathcurveto{\pgfqpoint{1.234124in}{2.084645in}}{\pgfqpoint{1.230852in}{2.076745in}}{\pgfqpoint{1.230852in}{2.068509in}}%
\pgfpathcurveto{\pgfqpoint{1.230852in}{2.060273in}}{\pgfqpoint{1.234124in}{2.052373in}}{\pgfqpoint{1.239948in}{2.046549in}}%
\pgfpathcurveto{\pgfqpoint{1.245772in}{2.040725in}}{\pgfqpoint{1.253672in}{2.037452in}}{\pgfqpoint{1.261908in}{2.037452in}}%
\pgfpathclose%
\pgfusepath{stroke,fill}%
\end{pgfscope}%
\begin{pgfscope}%
\pgfpathrectangle{\pgfqpoint{0.100000in}{0.212622in}}{\pgfqpoint{3.696000in}{3.696000in}}%
\pgfusepath{clip}%
\pgfsetbuttcap%
\pgfsetroundjoin%
\definecolor{currentfill}{rgb}{0.121569,0.466667,0.705882}%
\pgfsetfillcolor{currentfill}%
\pgfsetfillopacity{0.444311}%
\pgfsetlinewidth{1.003750pt}%
\definecolor{currentstroke}{rgb}{0.121569,0.466667,0.705882}%
\pgfsetstrokecolor{currentstroke}%
\pgfsetstrokeopacity{0.444311}%
\pgfsetdash{}{0pt}%
\pgfpathmoveto{\pgfqpoint{2.604807in}{2.379334in}}%
\pgfpathcurveto{\pgfqpoint{2.613044in}{2.379334in}}{\pgfqpoint{2.620944in}{2.382606in}}{\pgfqpoint{2.626768in}{2.388430in}}%
\pgfpathcurveto{\pgfqpoint{2.632592in}{2.394254in}}{\pgfqpoint{2.635864in}{2.402154in}}{\pgfqpoint{2.635864in}{2.410390in}}%
\pgfpathcurveto{\pgfqpoint{2.635864in}{2.418626in}}{\pgfqpoint{2.632592in}{2.426526in}}{\pgfqpoint{2.626768in}{2.432350in}}%
\pgfpathcurveto{\pgfqpoint{2.620944in}{2.438174in}}{\pgfqpoint{2.613044in}{2.441447in}}{\pgfqpoint{2.604807in}{2.441447in}}%
\pgfpathcurveto{\pgfqpoint{2.596571in}{2.441447in}}{\pgfqpoint{2.588671in}{2.438174in}}{\pgfqpoint{2.582847in}{2.432350in}}%
\pgfpathcurveto{\pgfqpoint{2.577023in}{2.426526in}}{\pgfqpoint{2.573751in}{2.418626in}}{\pgfqpoint{2.573751in}{2.410390in}}%
\pgfpathcurveto{\pgfqpoint{2.573751in}{2.402154in}}{\pgfqpoint{2.577023in}{2.394254in}}{\pgfqpoint{2.582847in}{2.388430in}}%
\pgfpathcurveto{\pgfqpoint{2.588671in}{2.382606in}}{\pgfqpoint{2.596571in}{2.379334in}}{\pgfqpoint{2.604807in}{2.379334in}}%
\pgfpathclose%
\pgfusepath{stroke,fill}%
\end{pgfscope}%
\begin{pgfscope}%
\pgfpathrectangle{\pgfqpoint{0.100000in}{0.212622in}}{\pgfqpoint{3.696000in}{3.696000in}}%
\pgfusepath{clip}%
\pgfsetbuttcap%
\pgfsetroundjoin%
\definecolor{currentfill}{rgb}{0.121569,0.466667,0.705882}%
\pgfsetfillcolor{currentfill}%
\pgfsetfillopacity{0.445691}%
\pgfsetlinewidth{1.003750pt}%
\definecolor{currentstroke}{rgb}{0.121569,0.466667,0.705882}%
\pgfsetstrokecolor{currentstroke}%
\pgfsetstrokeopacity{0.445691}%
\pgfsetdash{}{0pt}%
\pgfpathmoveto{\pgfqpoint{2.614074in}{2.376509in}}%
\pgfpathcurveto{\pgfqpoint{2.622310in}{2.376509in}}{\pgfqpoint{2.630210in}{2.379781in}}{\pgfqpoint{2.636034in}{2.385605in}}%
\pgfpathcurveto{\pgfqpoint{2.641858in}{2.391429in}}{\pgfqpoint{2.645130in}{2.399329in}}{\pgfqpoint{2.645130in}{2.407565in}}%
\pgfpathcurveto{\pgfqpoint{2.645130in}{2.415802in}}{\pgfqpoint{2.641858in}{2.423702in}}{\pgfqpoint{2.636034in}{2.429526in}}%
\pgfpathcurveto{\pgfqpoint{2.630210in}{2.435350in}}{\pgfqpoint{2.622310in}{2.438622in}}{\pgfqpoint{2.614074in}{2.438622in}}%
\pgfpathcurveto{\pgfqpoint{2.605838in}{2.438622in}}{\pgfqpoint{2.597938in}{2.435350in}}{\pgfqpoint{2.592114in}{2.429526in}}%
\pgfpathcurveto{\pgfqpoint{2.586290in}{2.423702in}}{\pgfqpoint{2.583017in}{2.415802in}}{\pgfqpoint{2.583017in}{2.407565in}}%
\pgfpathcurveto{\pgfqpoint{2.583017in}{2.399329in}}{\pgfqpoint{2.586290in}{2.391429in}}{\pgfqpoint{2.592114in}{2.385605in}}%
\pgfpathcurveto{\pgfqpoint{2.597938in}{2.379781in}}{\pgfqpoint{2.605838in}{2.376509in}}{\pgfqpoint{2.614074in}{2.376509in}}%
\pgfpathclose%
\pgfusepath{stroke,fill}%
\end{pgfscope}%
\begin{pgfscope}%
\pgfpathrectangle{\pgfqpoint{0.100000in}{0.212622in}}{\pgfqpoint{3.696000in}{3.696000in}}%
\pgfusepath{clip}%
\pgfsetbuttcap%
\pgfsetroundjoin%
\definecolor{currentfill}{rgb}{0.121569,0.466667,0.705882}%
\pgfsetfillcolor{currentfill}%
\pgfsetfillopacity{0.446520}%
\pgfsetlinewidth{1.003750pt}%
\definecolor{currentstroke}{rgb}{0.121569,0.466667,0.705882}%
\pgfsetstrokecolor{currentstroke}%
\pgfsetstrokeopacity{0.446520}%
\pgfsetdash{}{0pt}%
\pgfpathmoveto{\pgfqpoint{2.619263in}{2.375680in}}%
\pgfpathcurveto{\pgfqpoint{2.627499in}{2.375680in}}{\pgfqpoint{2.635399in}{2.378952in}}{\pgfqpoint{2.641223in}{2.384776in}}%
\pgfpathcurveto{\pgfqpoint{2.647047in}{2.390600in}}{\pgfqpoint{2.650319in}{2.398500in}}{\pgfqpoint{2.650319in}{2.406736in}}%
\pgfpathcurveto{\pgfqpoint{2.650319in}{2.414972in}}{\pgfqpoint{2.647047in}{2.422872in}}{\pgfqpoint{2.641223in}{2.428696in}}%
\pgfpathcurveto{\pgfqpoint{2.635399in}{2.434520in}}{\pgfqpoint{2.627499in}{2.437793in}}{\pgfqpoint{2.619263in}{2.437793in}}%
\pgfpathcurveto{\pgfqpoint{2.611026in}{2.437793in}}{\pgfqpoint{2.603126in}{2.434520in}}{\pgfqpoint{2.597303in}{2.428696in}}%
\pgfpathcurveto{\pgfqpoint{2.591479in}{2.422872in}}{\pgfqpoint{2.588206in}{2.414972in}}{\pgfqpoint{2.588206in}{2.406736in}}%
\pgfpathcurveto{\pgfqpoint{2.588206in}{2.398500in}}{\pgfqpoint{2.591479in}{2.390600in}}{\pgfqpoint{2.597303in}{2.384776in}}%
\pgfpathcurveto{\pgfqpoint{2.603126in}{2.378952in}}{\pgfqpoint{2.611026in}{2.375680in}}{\pgfqpoint{2.619263in}{2.375680in}}%
\pgfpathclose%
\pgfusepath{stroke,fill}%
\end{pgfscope}%
\begin{pgfscope}%
\pgfpathrectangle{\pgfqpoint{0.100000in}{0.212622in}}{\pgfqpoint{3.696000in}{3.696000in}}%
\pgfusepath{clip}%
\pgfsetbuttcap%
\pgfsetroundjoin%
\definecolor{currentfill}{rgb}{0.121569,0.466667,0.705882}%
\pgfsetfillcolor{currentfill}%
\pgfsetfillopacity{0.447228}%
\pgfsetlinewidth{1.003750pt}%
\definecolor{currentstroke}{rgb}{0.121569,0.466667,0.705882}%
\pgfsetstrokecolor{currentstroke}%
\pgfsetstrokeopacity{0.447228}%
\pgfsetdash{}{0pt}%
\pgfpathmoveto{\pgfqpoint{1.251760in}{2.025773in}}%
\pgfpathcurveto{\pgfqpoint{1.259996in}{2.025773in}}{\pgfqpoint{1.267896in}{2.029045in}}{\pgfqpoint{1.273720in}{2.034869in}}%
\pgfpathcurveto{\pgfqpoint{1.279544in}{2.040693in}}{\pgfqpoint{1.282817in}{2.048593in}}{\pgfqpoint{1.282817in}{2.056829in}}%
\pgfpathcurveto{\pgfqpoint{1.282817in}{2.065066in}}{\pgfqpoint{1.279544in}{2.072966in}}{\pgfqpoint{1.273720in}{2.078790in}}%
\pgfpathcurveto{\pgfqpoint{1.267896in}{2.084614in}}{\pgfqpoint{1.259996in}{2.087886in}}{\pgfqpoint{1.251760in}{2.087886in}}%
\pgfpathcurveto{\pgfqpoint{1.243524in}{2.087886in}}{\pgfqpoint{1.235624in}{2.084614in}}{\pgfqpoint{1.229800in}{2.078790in}}%
\pgfpathcurveto{\pgfqpoint{1.223976in}{2.072966in}}{\pgfqpoint{1.220704in}{2.065066in}}{\pgfqpoint{1.220704in}{2.056829in}}%
\pgfpathcurveto{\pgfqpoint{1.220704in}{2.048593in}}{\pgfqpoint{1.223976in}{2.040693in}}{\pgfqpoint{1.229800in}{2.034869in}}%
\pgfpathcurveto{\pgfqpoint{1.235624in}{2.029045in}}{\pgfqpoint{1.243524in}{2.025773in}}{\pgfqpoint{1.251760in}{2.025773in}}%
\pgfpathclose%
\pgfusepath{stroke,fill}%
\end{pgfscope}%
\begin{pgfscope}%
\pgfpathrectangle{\pgfqpoint{0.100000in}{0.212622in}}{\pgfqpoint{3.696000in}{3.696000in}}%
\pgfusepath{clip}%
\pgfsetbuttcap%
\pgfsetroundjoin%
\definecolor{currentfill}{rgb}{0.121569,0.466667,0.705882}%
\pgfsetfillcolor{currentfill}%
\pgfsetfillopacity{0.447593}%
\pgfsetlinewidth{1.003750pt}%
\definecolor{currentstroke}{rgb}{0.121569,0.466667,0.705882}%
\pgfsetstrokecolor{currentstroke}%
\pgfsetstrokeopacity{0.447593}%
\pgfsetdash{}{0pt}%
\pgfpathmoveto{\pgfqpoint{2.625875in}{2.374557in}}%
\pgfpathcurveto{\pgfqpoint{2.634111in}{2.374557in}}{\pgfqpoint{2.642011in}{2.377830in}}{\pgfqpoint{2.647835in}{2.383654in}}%
\pgfpathcurveto{\pgfqpoint{2.653659in}{2.389478in}}{\pgfqpoint{2.656931in}{2.397378in}}{\pgfqpoint{2.656931in}{2.405614in}}%
\pgfpathcurveto{\pgfqpoint{2.656931in}{2.413850in}}{\pgfqpoint{2.653659in}{2.421750in}}{\pgfqpoint{2.647835in}{2.427574in}}%
\pgfpathcurveto{\pgfqpoint{2.642011in}{2.433398in}}{\pgfqpoint{2.634111in}{2.436670in}}{\pgfqpoint{2.625875in}{2.436670in}}%
\pgfpathcurveto{\pgfqpoint{2.617638in}{2.436670in}}{\pgfqpoint{2.609738in}{2.433398in}}{\pgfqpoint{2.603914in}{2.427574in}}%
\pgfpathcurveto{\pgfqpoint{2.598090in}{2.421750in}}{\pgfqpoint{2.594818in}{2.413850in}}{\pgfqpoint{2.594818in}{2.405614in}}%
\pgfpathcurveto{\pgfqpoint{2.594818in}{2.397378in}}{\pgfqpoint{2.598090in}{2.389478in}}{\pgfqpoint{2.603914in}{2.383654in}}%
\pgfpathcurveto{\pgfqpoint{2.609738in}{2.377830in}}{\pgfqpoint{2.617638in}{2.374557in}}{\pgfqpoint{2.625875in}{2.374557in}}%
\pgfpathclose%
\pgfusepath{stroke,fill}%
\end{pgfscope}%
\begin{pgfscope}%
\pgfpathrectangle{\pgfqpoint{0.100000in}{0.212622in}}{\pgfqpoint{3.696000in}{3.696000in}}%
\pgfusepath{clip}%
\pgfsetbuttcap%
\pgfsetroundjoin%
\definecolor{currentfill}{rgb}{0.121569,0.466667,0.705882}%
\pgfsetfillcolor{currentfill}%
\pgfsetfillopacity{0.449005}%
\pgfsetlinewidth{1.003750pt}%
\definecolor{currentstroke}{rgb}{0.121569,0.466667,0.705882}%
\pgfsetstrokecolor{currentstroke}%
\pgfsetstrokeopacity{0.449005}%
\pgfsetdash{}{0pt}%
\pgfpathmoveto{\pgfqpoint{2.634704in}{2.373047in}}%
\pgfpathcurveto{\pgfqpoint{2.642940in}{2.373047in}}{\pgfqpoint{2.650840in}{2.376320in}}{\pgfqpoint{2.656664in}{2.382144in}}%
\pgfpathcurveto{\pgfqpoint{2.662488in}{2.387968in}}{\pgfqpoint{2.665760in}{2.395868in}}{\pgfqpoint{2.665760in}{2.404104in}}%
\pgfpathcurveto{\pgfqpoint{2.665760in}{2.412340in}}{\pgfqpoint{2.662488in}{2.420240in}}{\pgfqpoint{2.656664in}{2.426064in}}%
\pgfpathcurveto{\pgfqpoint{2.650840in}{2.431888in}}{\pgfqpoint{2.642940in}{2.435160in}}{\pgfqpoint{2.634704in}{2.435160in}}%
\pgfpathcurveto{\pgfqpoint{2.626468in}{2.435160in}}{\pgfqpoint{2.618568in}{2.431888in}}{\pgfqpoint{2.612744in}{2.426064in}}%
\pgfpathcurveto{\pgfqpoint{2.606920in}{2.420240in}}{\pgfqpoint{2.603647in}{2.412340in}}{\pgfqpoint{2.603647in}{2.404104in}}%
\pgfpathcurveto{\pgfqpoint{2.603647in}{2.395868in}}{\pgfqpoint{2.606920in}{2.387968in}}{\pgfqpoint{2.612744in}{2.382144in}}%
\pgfpathcurveto{\pgfqpoint{2.618568in}{2.376320in}}{\pgfqpoint{2.626468in}{2.373047in}}{\pgfqpoint{2.634704in}{2.373047in}}%
\pgfpathclose%
\pgfusepath{stroke,fill}%
\end{pgfscope}%
\begin{pgfscope}%
\pgfpathrectangle{\pgfqpoint{0.100000in}{0.212622in}}{\pgfqpoint{3.696000in}{3.696000in}}%
\pgfusepath{clip}%
\pgfsetbuttcap%
\pgfsetroundjoin%
\definecolor{currentfill}{rgb}{0.121569,0.466667,0.705882}%
\pgfsetfillcolor{currentfill}%
\pgfsetfillopacity{0.450100}%
\pgfsetlinewidth{1.003750pt}%
\definecolor{currentstroke}{rgb}{0.121569,0.466667,0.705882}%
\pgfsetstrokecolor{currentstroke}%
\pgfsetstrokeopacity{0.450100}%
\pgfsetdash{}{0pt}%
\pgfpathmoveto{\pgfqpoint{1.242648in}{2.014992in}}%
\pgfpathcurveto{\pgfqpoint{1.250884in}{2.014992in}}{\pgfqpoint{1.258784in}{2.018264in}}{\pgfqpoint{1.264608in}{2.024088in}}%
\pgfpathcurveto{\pgfqpoint{1.270432in}{2.029912in}}{\pgfqpoint{1.273705in}{2.037812in}}{\pgfqpoint{1.273705in}{2.046048in}}%
\pgfpathcurveto{\pgfqpoint{1.273705in}{2.054284in}}{\pgfqpoint{1.270432in}{2.062184in}}{\pgfqpoint{1.264608in}{2.068008in}}%
\pgfpathcurveto{\pgfqpoint{1.258784in}{2.073832in}}{\pgfqpoint{1.250884in}{2.077105in}}{\pgfqpoint{1.242648in}{2.077105in}}%
\pgfpathcurveto{\pgfqpoint{1.234412in}{2.077105in}}{\pgfqpoint{1.226512in}{2.073832in}}{\pgfqpoint{1.220688in}{2.068008in}}%
\pgfpathcurveto{\pgfqpoint{1.214864in}{2.062184in}}{\pgfqpoint{1.211592in}{2.054284in}}{\pgfqpoint{1.211592in}{2.046048in}}%
\pgfpathcurveto{\pgfqpoint{1.211592in}{2.037812in}}{\pgfqpoint{1.214864in}{2.029912in}}{\pgfqpoint{1.220688in}{2.024088in}}%
\pgfpathcurveto{\pgfqpoint{1.226512in}{2.018264in}}{\pgfqpoint{1.234412in}{2.014992in}}{\pgfqpoint{1.242648in}{2.014992in}}%
\pgfpathclose%
\pgfusepath{stroke,fill}%
\end{pgfscope}%
\begin{pgfscope}%
\pgfpathrectangle{\pgfqpoint{0.100000in}{0.212622in}}{\pgfqpoint{3.696000in}{3.696000in}}%
\pgfusepath{clip}%
\pgfsetbuttcap%
\pgfsetroundjoin%
\definecolor{currentfill}{rgb}{0.121569,0.466667,0.705882}%
\pgfsetfillcolor{currentfill}%
\pgfsetfillopacity{0.450368}%
\pgfsetlinewidth{1.003750pt}%
\definecolor{currentstroke}{rgb}{0.121569,0.466667,0.705882}%
\pgfsetstrokecolor{currentstroke}%
\pgfsetstrokeopacity{0.450368}%
\pgfsetdash{}{0pt}%
\pgfpathmoveto{\pgfqpoint{2.644999in}{2.370509in}}%
\pgfpathcurveto{\pgfqpoint{2.653236in}{2.370509in}}{\pgfqpoint{2.661136in}{2.373781in}}{\pgfqpoint{2.666960in}{2.379605in}}%
\pgfpathcurveto{\pgfqpoint{2.672784in}{2.385429in}}{\pgfqpoint{2.676056in}{2.393329in}}{\pgfqpoint{2.676056in}{2.401565in}}%
\pgfpathcurveto{\pgfqpoint{2.676056in}{2.409801in}}{\pgfqpoint{2.672784in}{2.417701in}}{\pgfqpoint{2.666960in}{2.423525in}}%
\pgfpathcurveto{\pgfqpoint{2.661136in}{2.429349in}}{\pgfqpoint{2.653236in}{2.432622in}}{\pgfqpoint{2.644999in}{2.432622in}}%
\pgfpathcurveto{\pgfqpoint{2.636763in}{2.432622in}}{\pgfqpoint{2.628863in}{2.429349in}}{\pgfqpoint{2.623039in}{2.423525in}}%
\pgfpathcurveto{\pgfqpoint{2.617215in}{2.417701in}}{\pgfqpoint{2.613943in}{2.409801in}}{\pgfqpoint{2.613943in}{2.401565in}}%
\pgfpathcurveto{\pgfqpoint{2.613943in}{2.393329in}}{\pgfqpoint{2.617215in}{2.385429in}}{\pgfqpoint{2.623039in}{2.379605in}}%
\pgfpathcurveto{\pgfqpoint{2.628863in}{2.373781in}}{\pgfqpoint{2.636763in}{2.370509in}}{\pgfqpoint{2.644999in}{2.370509in}}%
\pgfpathclose%
\pgfusepath{stroke,fill}%
\end{pgfscope}%
\begin{pgfscope}%
\pgfpathrectangle{\pgfqpoint{0.100000in}{0.212622in}}{\pgfqpoint{3.696000in}{3.696000in}}%
\pgfusepath{clip}%
\pgfsetbuttcap%
\pgfsetroundjoin%
\definecolor{currentfill}{rgb}{0.121569,0.466667,0.705882}%
\pgfsetfillcolor{currentfill}%
\pgfsetfillopacity{0.451094}%
\pgfsetlinewidth{1.003750pt}%
\definecolor{currentstroke}{rgb}{0.121569,0.466667,0.705882}%
\pgfsetstrokecolor{currentstroke}%
\pgfsetstrokeopacity{0.451094}%
\pgfsetdash{}{0pt}%
\pgfpathmoveto{\pgfqpoint{2.650633in}{2.368857in}}%
\pgfpathcurveto{\pgfqpoint{2.658870in}{2.368857in}}{\pgfqpoint{2.666770in}{2.372129in}}{\pgfqpoint{2.672594in}{2.377953in}}%
\pgfpathcurveto{\pgfqpoint{2.678418in}{2.383777in}}{\pgfqpoint{2.681690in}{2.391677in}}{\pgfqpoint{2.681690in}{2.399914in}}%
\pgfpathcurveto{\pgfqpoint{2.681690in}{2.408150in}}{\pgfqpoint{2.678418in}{2.416050in}}{\pgfqpoint{2.672594in}{2.421874in}}%
\pgfpathcurveto{\pgfqpoint{2.666770in}{2.427698in}}{\pgfqpoint{2.658870in}{2.430970in}}{\pgfqpoint{2.650633in}{2.430970in}}%
\pgfpathcurveto{\pgfqpoint{2.642397in}{2.430970in}}{\pgfqpoint{2.634497in}{2.427698in}}{\pgfqpoint{2.628673in}{2.421874in}}%
\pgfpathcurveto{\pgfqpoint{2.622849in}{2.416050in}}{\pgfqpoint{2.619577in}{2.408150in}}{\pgfqpoint{2.619577in}{2.399914in}}%
\pgfpathcurveto{\pgfqpoint{2.619577in}{2.391677in}}{\pgfqpoint{2.622849in}{2.383777in}}{\pgfqpoint{2.628673in}{2.377953in}}%
\pgfpathcurveto{\pgfqpoint{2.634497in}{2.372129in}}{\pgfqpoint{2.642397in}{2.368857in}}{\pgfqpoint{2.650633in}{2.368857in}}%
\pgfpathclose%
\pgfusepath{stroke,fill}%
\end{pgfscope}%
\begin{pgfscope}%
\pgfpathrectangle{\pgfqpoint{0.100000in}{0.212622in}}{\pgfqpoint{3.696000in}{3.696000in}}%
\pgfusepath{clip}%
\pgfsetbuttcap%
\pgfsetroundjoin%
\definecolor{currentfill}{rgb}{0.121569,0.466667,0.705882}%
\pgfsetfillcolor{currentfill}%
\pgfsetfillopacity{0.452201}%
\pgfsetlinewidth{1.003750pt}%
\definecolor{currentstroke}{rgb}{0.121569,0.466667,0.705882}%
\pgfsetstrokecolor{currentstroke}%
\pgfsetstrokeopacity{0.452201}%
\pgfsetdash{}{0pt}%
\pgfpathmoveto{\pgfqpoint{2.658235in}{2.367863in}}%
\pgfpathcurveto{\pgfqpoint{2.666472in}{2.367863in}}{\pgfqpoint{2.674372in}{2.371136in}}{\pgfqpoint{2.680195in}{2.376960in}}%
\pgfpathcurveto{\pgfqpoint{2.686019in}{2.382784in}}{\pgfqpoint{2.689292in}{2.390684in}}{\pgfqpoint{2.689292in}{2.398920in}}%
\pgfpathcurveto{\pgfqpoint{2.689292in}{2.407156in}}{\pgfqpoint{2.686019in}{2.415056in}}{\pgfqpoint{2.680195in}{2.420880in}}%
\pgfpathcurveto{\pgfqpoint{2.674372in}{2.426704in}}{\pgfqpoint{2.666472in}{2.429976in}}{\pgfqpoint{2.658235in}{2.429976in}}%
\pgfpathcurveto{\pgfqpoint{2.649999in}{2.429976in}}{\pgfqpoint{2.642099in}{2.426704in}}{\pgfqpoint{2.636275in}{2.420880in}}%
\pgfpathcurveto{\pgfqpoint{2.630451in}{2.415056in}}{\pgfqpoint{2.627179in}{2.407156in}}{\pgfqpoint{2.627179in}{2.398920in}}%
\pgfpathcurveto{\pgfqpoint{2.627179in}{2.390684in}}{\pgfqpoint{2.630451in}{2.382784in}}{\pgfqpoint{2.636275in}{2.376960in}}%
\pgfpathcurveto{\pgfqpoint{2.642099in}{2.371136in}}{\pgfqpoint{2.649999in}{2.367863in}}{\pgfqpoint{2.658235in}{2.367863in}}%
\pgfpathclose%
\pgfusepath{stroke,fill}%
\end{pgfscope}%
\begin{pgfscope}%
\pgfpathrectangle{\pgfqpoint{0.100000in}{0.212622in}}{\pgfqpoint{3.696000in}{3.696000in}}%
\pgfusepath{clip}%
\pgfsetbuttcap%
\pgfsetroundjoin%
\definecolor{currentfill}{rgb}{0.121569,0.466667,0.705882}%
\pgfsetfillcolor{currentfill}%
\pgfsetfillopacity{0.452569}%
\pgfsetlinewidth{1.003750pt}%
\definecolor{currentstroke}{rgb}{0.121569,0.466667,0.705882}%
\pgfsetstrokecolor{currentstroke}%
\pgfsetstrokeopacity{0.452569}%
\pgfsetdash{}{0pt}%
\pgfpathmoveto{\pgfqpoint{1.234507in}{2.005253in}}%
\pgfpathcurveto{\pgfqpoint{1.242743in}{2.005253in}}{\pgfqpoint{1.250643in}{2.008525in}}{\pgfqpoint{1.256467in}{2.014349in}}%
\pgfpathcurveto{\pgfqpoint{1.262291in}{2.020173in}}{\pgfqpoint{1.265563in}{2.028073in}}{\pgfqpoint{1.265563in}{2.036310in}}%
\pgfpathcurveto{\pgfqpoint{1.265563in}{2.044546in}}{\pgfqpoint{1.262291in}{2.052446in}}{\pgfqpoint{1.256467in}{2.058270in}}%
\pgfpathcurveto{\pgfqpoint{1.250643in}{2.064094in}}{\pgfqpoint{1.242743in}{2.067366in}}{\pgfqpoint{1.234507in}{2.067366in}}%
\pgfpathcurveto{\pgfqpoint{1.226270in}{2.067366in}}{\pgfqpoint{1.218370in}{2.064094in}}{\pgfqpoint{1.212546in}{2.058270in}}%
\pgfpathcurveto{\pgfqpoint{1.206722in}{2.052446in}}{\pgfqpoint{1.203450in}{2.044546in}}{\pgfqpoint{1.203450in}{2.036310in}}%
\pgfpathcurveto{\pgfqpoint{1.203450in}{2.028073in}}{\pgfqpoint{1.206722in}{2.020173in}}{\pgfqpoint{1.212546in}{2.014349in}}%
\pgfpathcurveto{\pgfqpoint{1.218370in}{2.008525in}}{\pgfqpoint{1.226270in}{2.005253in}}{\pgfqpoint{1.234507in}{2.005253in}}%
\pgfpathclose%
\pgfusepath{stroke,fill}%
\end{pgfscope}%
\begin{pgfscope}%
\pgfpathrectangle{\pgfqpoint{0.100000in}{0.212622in}}{\pgfqpoint{3.696000in}{3.696000in}}%
\pgfusepath{clip}%
\pgfsetbuttcap%
\pgfsetroundjoin%
\definecolor{currentfill}{rgb}{0.121569,0.466667,0.705882}%
\pgfsetfillcolor{currentfill}%
\pgfsetfillopacity{0.453522}%
\pgfsetlinewidth{1.003750pt}%
\definecolor{currentstroke}{rgb}{0.121569,0.466667,0.705882}%
\pgfsetstrokecolor{currentstroke}%
\pgfsetstrokeopacity{0.453522}%
\pgfsetdash{}{0pt}%
\pgfpathmoveto{\pgfqpoint{2.667349in}{2.366074in}}%
\pgfpathcurveto{\pgfqpoint{2.675586in}{2.366074in}}{\pgfqpoint{2.683486in}{2.369346in}}{\pgfqpoint{2.689310in}{2.375170in}}%
\pgfpathcurveto{\pgfqpoint{2.695134in}{2.380994in}}{\pgfqpoint{2.698406in}{2.388894in}}{\pgfqpoint{2.698406in}{2.397130in}}%
\pgfpathcurveto{\pgfqpoint{2.698406in}{2.405366in}}{\pgfqpoint{2.695134in}{2.413266in}}{\pgfqpoint{2.689310in}{2.419090in}}%
\pgfpathcurveto{\pgfqpoint{2.683486in}{2.424914in}}{\pgfqpoint{2.675586in}{2.428187in}}{\pgfqpoint{2.667349in}{2.428187in}}%
\pgfpathcurveto{\pgfqpoint{2.659113in}{2.428187in}}{\pgfqpoint{2.651213in}{2.424914in}}{\pgfqpoint{2.645389in}{2.419090in}}%
\pgfpathcurveto{\pgfqpoint{2.639565in}{2.413266in}}{\pgfqpoint{2.636293in}{2.405366in}}{\pgfqpoint{2.636293in}{2.397130in}}%
\pgfpathcurveto{\pgfqpoint{2.636293in}{2.388894in}}{\pgfqpoint{2.639565in}{2.380994in}}{\pgfqpoint{2.645389in}{2.375170in}}%
\pgfpathcurveto{\pgfqpoint{2.651213in}{2.369346in}}{\pgfqpoint{2.659113in}{2.366074in}}{\pgfqpoint{2.667349in}{2.366074in}}%
\pgfpathclose%
\pgfusepath{stroke,fill}%
\end{pgfscope}%
\begin{pgfscope}%
\pgfpathrectangle{\pgfqpoint{0.100000in}{0.212622in}}{\pgfqpoint{3.696000in}{3.696000in}}%
\pgfusepath{clip}%
\pgfsetbuttcap%
\pgfsetroundjoin%
\definecolor{currentfill}{rgb}{0.121569,0.466667,0.705882}%
\pgfsetfillcolor{currentfill}%
\pgfsetfillopacity{0.454228}%
\pgfsetlinewidth{1.003750pt}%
\definecolor{currentstroke}{rgb}{0.121569,0.466667,0.705882}%
\pgfsetstrokecolor{currentstroke}%
\pgfsetstrokeopacity{0.454228}%
\pgfsetdash{}{0pt}%
\pgfpathmoveto{\pgfqpoint{2.672359in}{2.364934in}}%
\pgfpathcurveto{\pgfqpoint{2.680595in}{2.364934in}}{\pgfqpoint{2.688495in}{2.368207in}}{\pgfqpoint{2.694319in}{2.374030in}}%
\pgfpathcurveto{\pgfqpoint{2.700143in}{2.379854in}}{\pgfqpoint{2.703415in}{2.387754in}}{\pgfqpoint{2.703415in}{2.395991in}}%
\pgfpathcurveto{\pgfqpoint{2.703415in}{2.404227in}}{\pgfqpoint{2.700143in}{2.412127in}}{\pgfqpoint{2.694319in}{2.417951in}}%
\pgfpathcurveto{\pgfqpoint{2.688495in}{2.423775in}}{\pgfqpoint{2.680595in}{2.427047in}}{\pgfqpoint{2.672359in}{2.427047in}}%
\pgfpathcurveto{\pgfqpoint{2.664122in}{2.427047in}}{\pgfqpoint{2.656222in}{2.423775in}}{\pgfqpoint{2.650398in}{2.417951in}}%
\pgfpathcurveto{\pgfqpoint{2.644574in}{2.412127in}}{\pgfqpoint{2.641302in}{2.404227in}}{\pgfqpoint{2.641302in}{2.395991in}}%
\pgfpathcurveto{\pgfqpoint{2.641302in}{2.387754in}}{\pgfqpoint{2.644574in}{2.379854in}}{\pgfqpoint{2.650398in}{2.374030in}}%
\pgfpathcurveto{\pgfqpoint{2.656222in}{2.368207in}}{\pgfqpoint{2.664122in}{2.364934in}}{\pgfqpoint{2.672359in}{2.364934in}}%
\pgfpathclose%
\pgfusepath{stroke,fill}%
\end{pgfscope}%
\begin{pgfscope}%
\pgfpathrectangle{\pgfqpoint{0.100000in}{0.212622in}}{\pgfqpoint{3.696000in}{3.696000in}}%
\pgfusepath{clip}%
\pgfsetbuttcap%
\pgfsetroundjoin%
\definecolor{currentfill}{rgb}{0.121569,0.466667,0.705882}%
\pgfsetfillcolor{currentfill}%
\pgfsetfillopacity{0.454721}%
\pgfsetlinewidth{1.003750pt}%
\definecolor{currentstroke}{rgb}{0.121569,0.466667,0.705882}%
\pgfsetstrokecolor{currentstroke}%
\pgfsetstrokeopacity{0.454721}%
\pgfsetdash{}{0pt}%
\pgfpathmoveto{\pgfqpoint{1.227780in}{1.997132in}}%
\pgfpathcurveto{\pgfqpoint{1.236016in}{1.997132in}}{\pgfqpoint{1.243916in}{2.000404in}}{\pgfqpoint{1.249740in}{2.006228in}}%
\pgfpathcurveto{\pgfqpoint{1.255564in}{2.012052in}}{\pgfqpoint{1.258836in}{2.019952in}}{\pgfqpoint{1.258836in}{2.028188in}}%
\pgfpathcurveto{\pgfqpoint{1.258836in}{2.036425in}}{\pgfqpoint{1.255564in}{2.044325in}}{\pgfqpoint{1.249740in}{2.050149in}}%
\pgfpathcurveto{\pgfqpoint{1.243916in}{2.055973in}}{\pgfqpoint{1.236016in}{2.059245in}}{\pgfqpoint{1.227780in}{2.059245in}}%
\pgfpathcurveto{\pgfqpoint{1.219544in}{2.059245in}}{\pgfqpoint{1.211644in}{2.055973in}}{\pgfqpoint{1.205820in}{2.050149in}}%
\pgfpathcurveto{\pgfqpoint{1.199996in}{2.044325in}}{\pgfqpoint{1.196723in}{2.036425in}}{\pgfqpoint{1.196723in}{2.028188in}}%
\pgfpathcurveto{\pgfqpoint{1.196723in}{2.019952in}}{\pgfqpoint{1.199996in}{2.012052in}}{\pgfqpoint{1.205820in}{2.006228in}}%
\pgfpathcurveto{\pgfqpoint{1.211644in}{2.000404in}}{\pgfqpoint{1.219544in}{1.997132in}}{\pgfqpoint{1.227780in}{1.997132in}}%
\pgfpathclose%
\pgfusepath{stroke,fill}%
\end{pgfscope}%
\begin{pgfscope}%
\pgfpathrectangle{\pgfqpoint{0.100000in}{0.212622in}}{\pgfqpoint{3.696000in}{3.696000in}}%
\pgfusepath{clip}%
\pgfsetbuttcap%
\pgfsetroundjoin%
\definecolor{currentfill}{rgb}{0.121569,0.466667,0.705882}%
\pgfsetfillcolor{currentfill}%
\pgfsetfillopacity{0.455323}%
\pgfsetlinewidth{1.003750pt}%
\definecolor{currentstroke}{rgb}{0.121569,0.466667,0.705882}%
\pgfsetstrokecolor{currentstroke}%
\pgfsetstrokeopacity{0.455323}%
\pgfsetdash{}{0pt}%
\pgfpathmoveto{\pgfqpoint{2.679304in}{2.364496in}}%
\pgfpathcurveto{\pgfqpoint{2.687540in}{2.364496in}}{\pgfqpoint{2.695440in}{2.367768in}}{\pgfqpoint{2.701264in}{2.373592in}}%
\pgfpathcurveto{\pgfqpoint{2.707088in}{2.379416in}}{\pgfqpoint{2.710361in}{2.387316in}}{\pgfqpoint{2.710361in}{2.395553in}}%
\pgfpathcurveto{\pgfqpoint{2.710361in}{2.403789in}}{\pgfqpoint{2.707088in}{2.411689in}}{\pgfqpoint{2.701264in}{2.417513in}}%
\pgfpathcurveto{\pgfqpoint{2.695440in}{2.423337in}}{\pgfqpoint{2.687540in}{2.426609in}}{\pgfqpoint{2.679304in}{2.426609in}}%
\pgfpathcurveto{\pgfqpoint{2.671068in}{2.426609in}}{\pgfqpoint{2.663168in}{2.423337in}}{\pgfqpoint{2.657344in}{2.417513in}}%
\pgfpathcurveto{\pgfqpoint{2.651520in}{2.411689in}}{\pgfqpoint{2.648248in}{2.403789in}}{\pgfqpoint{2.648248in}{2.395553in}}%
\pgfpathcurveto{\pgfqpoint{2.648248in}{2.387316in}}{\pgfqpoint{2.651520in}{2.379416in}}{\pgfqpoint{2.657344in}{2.373592in}}%
\pgfpathcurveto{\pgfqpoint{2.663168in}{2.367768in}}{\pgfqpoint{2.671068in}{2.364496in}}{\pgfqpoint{2.679304in}{2.364496in}}%
\pgfpathclose%
\pgfusepath{stroke,fill}%
\end{pgfscope}%
\begin{pgfscope}%
\pgfpathrectangle{\pgfqpoint{0.100000in}{0.212622in}}{\pgfqpoint{3.696000in}{3.696000in}}%
\pgfusepath{clip}%
\pgfsetbuttcap%
\pgfsetroundjoin%
\definecolor{currentfill}{rgb}{0.121569,0.466667,0.705882}%
\pgfsetfillcolor{currentfill}%
\pgfsetfillopacity{0.455923}%
\pgfsetlinewidth{1.003750pt}%
\definecolor{currentstroke}{rgb}{0.121569,0.466667,0.705882}%
\pgfsetstrokecolor{currentstroke}%
\pgfsetstrokeopacity{0.455923}%
\pgfsetdash{}{0pt}%
\pgfpathmoveto{\pgfqpoint{2.683097in}{2.364147in}}%
\pgfpathcurveto{\pgfqpoint{2.691334in}{2.364147in}}{\pgfqpoint{2.699234in}{2.367419in}}{\pgfqpoint{2.705058in}{2.373243in}}%
\pgfpathcurveto{\pgfqpoint{2.710882in}{2.379067in}}{\pgfqpoint{2.714154in}{2.386967in}}{\pgfqpoint{2.714154in}{2.395203in}}%
\pgfpathcurveto{\pgfqpoint{2.714154in}{2.403440in}}{\pgfqpoint{2.710882in}{2.411340in}}{\pgfqpoint{2.705058in}{2.417164in}}%
\pgfpathcurveto{\pgfqpoint{2.699234in}{2.422988in}}{\pgfqpoint{2.691334in}{2.426260in}}{\pgfqpoint{2.683097in}{2.426260in}}%
\pgfpathcurveto{\pgfqpoint{2.674861in}{2.426260in}}{\pgfqpoint{2.666961in}{2.422988in}}{\pgfqpoint{2.661137in}{2.417164in}}%
\pgfpathcurveto{\pgfqpoint{2.655313in}{2.411340in}}{\pgfqpoint{2.652041in}{2.403440in}}{\pgfqpoint{2.652041in}{2.395203in}}%
\pgfpathcurveto{\pgfqpoint{2.652041in}{2.386967in}}{\pgfqpoint{2.655313in}{2.379067in}}{\pgfqpoint{2.661137in}{2.373243in}}%
\pgfpathcurveto{\pgfqpoint{2.666961in}{2.367419in}}{\pgfqpoint{2.674861in}{2.364147in}}{\pgfqpoint{2.683097in}{2.364147in}}%
\pgfpathclose%
\pgfusepath{stroke,fill}%
\end{pgfscope}%
\begin{pgfscope}%
\pgfpathrectangle{\pgfqpoint{0.100000in}{0.212622in}}{\pgfqpoint{3.696000in}{3.696000in}}%
\pgfusepath{clip}%
\pgfsetbuttcap%
\pgfsetroundjoin%
\definecolor{currentfill}{rgb}{0.121569,0.466667,0.705882}%
\pgfsetfillcolor{currentfill}%
\pgfsetfillopacity{0.456619}%
\pgfsetlinewidth{1.003750pt}%
\definecolor{currentstroke}{rgb}{0.121569,0.466667,0.705882}%
\pgfsetstrokecolor{currentstroke}%
\pgfsetstrokeopacity{0.456619}%
\pgfsetdash{}{0pt}%
\pgfpathmoveto{\pgfqpoint{1.221469in}{1.989407in}}%
\pgfpathcurveto{\pgfqpoint{1.229705in}{1.989407in}}{\pgfqpoint{1.237605in}{1.992680in}}{\pgfqpoint{1.243429in}{1.998504in}}%
\pgfpathcurveto{\pgfqpoint{1.249253in}{2.004328in}}{\pgfqpoint{1.252525in}{2.012228in}}{\pgfqpoint{1.252525in}{2.020464in}}%
\pgfpathcurveto{\pgfqpoint{1.252525in}{2.028700in}}{\pgfqpoint{1.249253in}{2.036600in}}{\pgfqpoint{1.243429in}{2.042424in}}%
\pgfpathcurveto{\pgfqpoint{1.237605in}{2.048248in}}{\pgfqpoint{1.229705in}{2.051520in}}{\pgfqpoint{1.221469in}{2.051520in}}%
\pgfpathcurveto{\pgfqpoint{1.213233in}{2.051520in}}{\pgfqpoint{1.205333in}{2.048248in}}{\pgfqpoint{1.199509in}{2.042424in}}%
\pgfpathcurveto{\pgfqpoint{1.193685in}{2.036600in}}{\pgfqpoint{1.190412in}{2.028700in}}{\pgfqpoint{1.190412in}{2.020464in}}%
\pgfpathcurveto{\pgfqpoint{1.190412in}{2.012228in}}{\pgfqpoint{1.193685in}{2.004328in}}{\pgfqpoint{1.199509in}{1.998504in}}%
\pgfpathcurveto{\pgfqpoint{1.205333in}{1.992680in}}{\pgfqpoint{1.213233in}{1.989407in}}{\pgfqpoint{1.221469in}{1.989407in}}%
\pgfpathclose%
\pgfusepath{stroke,fill}%
\end{pgfscope}%
\begin{pgfscope}%
\pgfpathrectangle{\pgfqpoint{0.100000in}{0.212622in}}{\pgfqpoint{3.696000in}{3.696000in}}%
\pgfusepath{clip}%
\pgfsetbuttcap%
\pgfsetroundjoin%
\definecolor{currentfill}{rgb}{0.121569,0.466667,0.705882}%
\pgfsetfillcolor{currentfill}%
\pgfsetfillopacity{0.456620}%
\pgfsetlinewidth{1.003750pt}%
\definecolor{currentstroke}{rgb}{0.121569,0.466667,0.705882}%
\pgfsetstrokecolor{currentstroke}%
\pgfsetstrokeopacity{0.456620}%
\pgfsetdash{}{0pt}%
\pgfpathmoveto{\pgfqpoint{2.687695in}{2.363627in}}%
\pgfpathcurveto{\pgfqpoint{2.695932in}{2.363627in}}{\pgfqpoint{2.703832in}{2.366900in}}{\pgfqpoint{2.709656in}{2.372724in}}%
\pgfpathcurveto{\pgfqpoint{2.715480in}{2.378547in}}{\pgfqpoint{2.718752in}{2.386447in}}{\pgfqpoint{2.718752in}{2.394684in}}%
\pgfpathcurveto{\pgfqpoint{2.718752in}{2.402920in}}{\pgfqpoint{2.715480in}{2.410820in}}{\pgfqpoint{2.709656in}{2.416644in}}%
\pgfpathcurveto{\pgfqpoint{2.703832in}{2.422468in}}{\pgfqpoint{2.695932in}{2.425740in}}{\pgfqpoint{2.687695in}{2.425740in}}%
\pgfpathcurveto{\pgfqpoint{2.679459in}{2.425740in}}{\pgfqpoint{2.671559in}{2.422468in}}{\pgfqpoint{2.665735in}{2.416644in}}%
\pgfpathcurveto{\pgfqpoint{2.659911in}{2.410820in}}{\pgfqpoint{2.656639in}{2.402920in}}{\pgfqpoint{2.656639in}{2.394684in}}%
\pgfpathcurveto{\pgfqpoint{2.656639in}{2.386447in}}{\pgfqpoint{2.659911in}{2.378547in}}{\pgfqpoint{2.665735in}{2.372724in}}%
\pgfpathcurveto{\pgfqpoint{2.671559in}{2.366900in}}{\pgfqpoint{2.679459in}{2.363627in}}{\pgfqpoint{2.687695in}{2.363627in}}%
\pgfpathclose%
\pgfusepath{stroke,fill}%
\end{pgfscope}%
\begin{pgfscope}%
\pgfpathrectangle{\pgfqpoint{0.100000in}{0.212622in}}{\pgfqpoint{3.696000in}{3.696000in}}%
\pgfusepath{clip}%
\pgfsetbuttcap%
\pgfsetroundjoin%
\definecolor{currentfill}{rgb}{0.121569,0.466667,0.705882}%
\pgfsetfillcolor{currentfill}%
\pgfsetfillopacity{0.456988}%
\pgfsetlinewidth{1.003750pt}%
\definecolor{currentstroke}{rgb}{0.121569,0.466667,0.705882}%
\pgfsetstrokecolor{currentstroke}%
\pgfsetstrokeopacity{0.456988}%
\pgfsetdash{}{0pt}%
\pgfpathmoveto{\pgfqpoint{2.690186in}{2.363113in}}%
\pgfpathcurveto{\pgfqpoint{2.698422in}{2.363113in}}{\pgfqpoint{2.706323in}{2.366386in}}{\pgfqpoint{2.712146in}{2.372210in}}%
\pgfpathcurveto{\pgfqpoint{2.717970in}{2.378034in}}{\pgfqpoint{2.721243in}{2.385934in}}{\pgfqpoint{2.721243in}{2.394170in}}%
\pgfpathcurveto{\pgfqpoint{2.721243in}{2.402406in}}{\pgfqpoint{2.717970in}{2.410306in}}{\pgfqpoint{2.712146in}{2.416130in}}%
\pgfpathcurveto{\pgfqpoint{2.706323in}{2.421954in}}{\pgfqpoint{2.698422in}{2.425226in}}{\pgfqpoint{2.690186in}{2.425226in}}%
\pgfpathcurveto{\pgfqpoint{2.681950in}{2.425226in}}{\pgfqpoint{2.674050in}{2.421954in}}{\pgfqpoint{2.668226in}{2.416130in}}%
\pgfpathcurveto{\pgfqpoint{2.662402in}{2.410306in}}{\pgfqpoint{2.659130in}{2.402406in}}{\pgfqpoint{2.659130in}{2.394170in}}%
\pgfpathcurveto{\pgfqpoint{2.659130in}{2.385934in}}{\pgfqpoint{2.662402in}{2.378034in}}{\pgfqpoint{2.668226in}{2.372210in}}%
\pgfpathcurveto{\pgfqpoint{2.674050in}{2.366386in}}{\pgfqpoint{2.681950in}{2.363113in}}{\pgfqpoint{2.690186in}{2.363113in}}%
\pgfpathclose%
\pgfusepath{stroke,fill}%
\end{pgfscope}%
\begin{pgfscope}%
\pgfpathrectangle{\pgfqpoint{0.100000in}{0.212622in}}{\pgfqpoint{3.696000in}{3.696000in}}%
\pgfusepath{clip}%
\pgfsetbuttcap%
\pgfsetroundjoin%
\definecolor{currentfill}{rgb}{0.121569,0.466667,0.705882}%
\pgfsetfillcolor{currentfill}%
\pgfsetfillopacity{0.457562}%
\pgfsetlinewidth{1.003750pt}%
\definecolor{currentstroke}{rgb}{0.121569,0.466667,0.705882}%
\pgfsetstrokecolor{currentstroke}%
\pgfsetstrokeopacity{0.457562}%
\pgfsetdash{}{0pt}%
\pgfpathmoveto{\pgfqpoint{2.694040in}{2.362683in}}%
\pgfpathcurveto{\pgfqpoint{2.702276in}{2.362683in}}{\pgfqpoint{2.710176in}{2.365956in}}{\pgfqpoint{2.716000in}{2.371779in}}%
\pgfpathcurveto{\pgfqpoint{2.721824in}{2.377603in}}{\pgfqpoint{2.725096in}{2.385503in}}{\pgfqpoint{2.725096in}{2.393740in}}%
\pgfpathcurveto{\pgfqpoint{2.725096in}{2.401976in}}{\pgfqpoint{2.721824in}{2.409876in}}{\pgfqpoint{2.716000in}{2.415700in}}%
\pgfpathcurveto{\pgfqpoint{2.710176in}{2.421524in}}{\pgfqpoint{2.702276in}{2.424796in}}{\pgfqpoint{2.694040in}{2.424796in}}%
\pgfpathcurveto{\pgfqpoint{2.685803in}{2.424796in}}{\pgfqpoint{2.677903in}{2.421524in}}{\pgfqpoint{2.672079in}{2.415700in}}%
\pgfpathcurveto{\pgfqpoint{2.666255in}{2.409876in}}{\pgfqpoint{2.662983in}{2.401976in}}{\pgfqpoint{2.662983in}{2.393740in}}%
\pgfpathcurveto{\pgfqpoint{2.662983in}{2.385503in}}{\pgfqpoint{2.666255in}{2.377603in}}{\pgfqpoint{2.672079in}{2.371779in}}%
\pgfpathcurveto{\pgfqpoint{2.677903in}{2.365956in}}{\pgfqpoint{2.685803in}{2.362683in}}{\pgfqpoint{2.694040in}{2.362683in}}%
\pgfpathclose%
\pgfusepath{stroke,fill}%
\end{pgfscope}%
\begin{pgfscope}%
\pgfpathrectangle{\pgfqpoint{0.100000in}{0.212622in}}{\pgfqpoint{3.696000in}{3.696000in}}%
\pgfusepath{clip}%
\pgfsetbuttcap%
\pgfsetroundjoin%
\definecolor{currentfill}{rgb}{0.121569,0.466667,0.705882}%
\pgfsetfillcolor{currentfill}%
\pgfsetfillopacity{0.458341}%
\pgfsetlinewidth{1.003750pt}%
\definecolor{currentstroke}{rgb}{0.121569,0.466667,0.705882}%
\pgfsetstrokecolor{currentstroke}%
\pgfsetstrokeopacity{0.458341}%
\pgfsetdash{}{0pt}%
\pgfpathmoveto{\pgfqpoint{2.699014in}{2.362340in}}%
\pgfpathcurveto{\pgfqpoint{2.707251in}{2.362340in}}{\pgfqpoint{2.715151in}{2.365613in}}{\pgfqpoint{2.720975in}{2.371437in}}%
\pgfpathcurveto{\pgfqpoint{2.726799in}{2.377261in}}{\pgfqpoint{2.730071in}{2.385161in}}{\pgfqpoint{2.730071in}{2.393397in}}%
\pgfpathcurveto{\pgfqpoint{2.730071in}{2.401633in}}{\pgfqpoint{2.726799in}{2.409533in}}{\pgfqpoint{2.720975in}{2.415357in}}%
\pgfpathcurveto{\pgfqpoint{2.715151in}{2.421181in}}{\pgfqpoint{2.707251in}{2.424453in}}{\pgfqpoint{2.699014in}{2.424453in}}%
\pgfpathcurveto{\pgfqpoint{2.690778in}{2.424453in}}{\pgfqpoint{2.682878in}{2.421181in}}{\pgfqpoint{2.677054in}{2.415357in}}%
\pgfpathcurveto{\pgfqpoint{2.671230in}{2.409533in}}{\pgfqpoint{2.667958in}{2.401633in}}{\pgfqpoint{2.667958in}{2.393397in}}%
\pgfpathcurveto{\pgfqpoint{2.667958in}{2.385161in}}{\pgfqpoint{2.671230in}{2.377261in}}{\pgfqpoint{2.677054in}{2.371437in}}%
\pgfpathcurveto{\pgfqpoint{2.682878in}{2.365613in}}{\pgfqpoint{2.690778in}{2.362340in}}{\pgfqpoint{2.699014in}{2.362340in}}%
\pgfpathclose%
\pgfusepath{stroke,fill}%
\end{pgfscope}%
\begin{pgfscope}%
\pgfpathrectangle{\pgfqpoint{0.100000in}{0.212622in}}{\pgfqpoint{3.696000in}{3.696000in}}%
\pgfusepath{clip}%
\pgfsetbuttcap%
\pgfsetroundjoin%
\definecolor{currentfill}{rgb}{0.121569,0.466667,0.705882}%
\pgfsetfillcolor{currentfill}%
\pgfsetfillopacity{0.458417}%
\pgfsetlinewidth{1.003750pt}%
\definecolor{currentstroke}{rgb}{0.121569,0.466667,0.705882}%
\pgfsetstrokecolor{currentstroke}%
\pgfsetstrokeopacity{0.458417}%
\pgfsetdash{}{0pt}%
\pgfpathmoveto{\pgfqpoint{1.215751in}{1.982483in}}%
\pgfpathcurveto{\pgfqpoint{1.223988in}{1.982483in}}{\pgfqpoint{1.231888in}{1.985755in}}{\pgfqpoint{1.237712in}{1.991579in}}%
\pgfpathcurveto{\pgfqpoint{1.243536in}{1.997403in}}{\pgfqpoint{1.246808in}{2.005303in}}{\pgfqpoint{1.246808in}{2.013540in}}%
\pgfpathcurveto{\pgfqpoint{1.246808in}{2.021776in}}{\pgfqpoint{1.243536in}{2.029676in}}{\pgfqpoint{1.237712in}{2.035500in}}%
\pgfpathcurveto{\pgfqpoint{1.231888in}{2.041324in}}{\pgfqpoint{1.223988in}{2.044596in}}{\pgfqpoint{1.215751in}{2.044596in}}%
\pgfpathcurveto{\pgfqpoint{1.207515in}{2.044596in}}{\pgfqpoint{1.199615in}{2.041324in}}{\pgfqpoint{1.193791in}{2.035500in}}%
\pgfpathcurveto{\pgfqpoint{1.187967in}{2.029676in}}{\pgfqpoint{1.184695in}{2.021776in}}{\pgfqpoint{1.184695in}{2.013540in}}%
\pgfpathcurveto{\pgfqpoint{1.184695in}{2.005303in}}{\pgfqpoint{1.187967in}{1.997403in}}{\pgfqpoint{1.193791in}{1.991579in}}%
\pgfpathcurveto{\pgfqpoint{1.199615in}{1.985755in}}{\pgfqpoint{1.207515in}{1.982483in}}{\pgfqpoint{1.215751in}{1.982483in}}%
\pgfpathclose%
\pgfusepath{stroke,fill}%
\end{pgfscope}%
\begin{pgfscope}%
\pgfpathrectangle{\pgfqpoint{0.100000in}{0.212622in}}{\pgfqpoint{3.696000in}{3.696000in}}%
\pgfusepath{clip}%
\pgfsetbuttcap%
\pgfsetroundjoin%
\definecolor{currentfill}{rgb}{0.121569,0.466667,0.705882}%
\pgfsetfillcolor{currentfill}%
\pgfsetfillopacity{0.458768}%
\pgfsetlinewidth{1.003750pt}%
\definecolor{currentstroke}{rgb}{0.121569,0.466667,0.705882}%
\pgfsetstrokecolor{currentstroke}%
\pgfsetstrokeopacity{0.458768}%
\pgfsetdash{}{0pt}%
\pgfpathmoveto{\pgfqpoint{2.701729in}{2.362069in}}%
\pgfpathcurveto{\pgfqpoint{2.709965in}{2.362069in}}{\pgfqpoint{2.717865in}{2.365342in}}{\pgfqpoint{2.723689in}{2.371165in}}%
\pgfpathcurveto{\pgfqpoint{2.729513in}{2.376989in}}{\pgfqpoint{2.732785in}{2.384889in}}{\pgfqpoint{2.732785in}{2.393126in}}%
\pgfpathcurveto{\pgfqpoint{2.732785in}{2.401362in}}{\pgfqpoint{2.729513in}{2.409262in}}{\pgfqpoint{2.723689in}{2.415086in}}%
\pgfpathcurveto{\pgfqpoint{2.717865in}{2.420910in}}{\pgfqpoint{2.709965in}{2.424182in}}{\pgfqpoint{2.701729in}{2.424182in}}%
\pgfpathcurveto{\pgfqpoint{2.693493in}{2.424182in}}{\pgfqpoint{2.685593in}{2.420910in}}{\pgfqpoint{2.679769in}{2.415086in}}%
\pgfpathcurveto{\pgfqpoint{2.673945in}{2.409262in}}{\pgfqpoint{2.670672in}{2.401362in}}{\pgfqpoint{2.670672in}{2.393126in}}%
\pgfpathcurveto{\pgfqpoint{2.670672in}{2.384889in}}{\pgfqpoint{2.673945in}{2.376989in}}{\pgfqpoint{2.679769in}{2.371165in}}%
\pgfpathcurveto{\pgfqpoint{2.685593in}{2.365342in}}{\pgfqpoint{2.693493in}{2.362069in}}{\pgfqpoint{2.701729in}{2.362069in}}%
\pgfpathclose%
\pgfusepath{stroke,fill}%
\end{pgfscope}%
\begin{pgfscope}%
\pgfpathrectangle{\pgfqpoint{0.100000in}{0.212622in}}{\pgfqpoint{3.696000in}{3.696000in}}%
\pgfusepath{clip}%
\pgfsetbuttcap%
\pgfsetroundjoin%
\definecolor{currentfill}{rgb}{0.121569,0.466667,0.705882}%
\pgfsetfillcolor{currentfill}%
\pgfsetfillopacity{0.459375}%
\pgfsetlinewidth{1.003750pt}%
\definecolor{currentstroke}{rgb}{0.121569,0.466667,0.705882}%
\pgfsetstrokecolor{currentstroke}%
\pgfsetstrokeopacity{0.459375}%
\pgfsetdash{}{0pt}%
\pgfpathmoveto{\pgfqpoint{2.705469in}{2.361745in}}%
\pgfpathcurveto{\pgfqpoint{2.713705in}{2.361745in}}{\pgfqpoint{2.721605in}{2.365017in}}{\pgfqpoint{2.727429in}{2.370841in}}%
\pgfpathcurveto{\pgfqpoint{2.733253in}{2.376665in}}{\pgfqpoint{2.736525in}{2.384565in}}{\pgfqpoint{2.736525in}{2.392802in}}%
\pgfpathcurveto{\pgfqpoint{2.736525in}{2.401038in}}{\pgfqpoint{2.733253in}{2.408938in}}{\pgfqpoint{2.727429in}{2.414762in}}%
\pgfpathcurveto{\pgfqpoint{2.721605in}{2.420586in}}{\pgfqpoint{2.713705in}{2.423858in}}{\pgfqpoint{2.705469in}{2.423858in}}%
\pgfpathcurveto{\pgfqpoint{2.697233in}{2.423858in}}{\pgfqpoint{2.689332in}{2.420586in}}{\pgfqpoint{2.683509in}{2.414762in}}%
\pgfpathcurveto{\pgfqpoint{2.677685in}{2.408938in}}{\pgfqpoint{2.674412in}{2.401038in}}{\pgfqpoint{2.674412in}{2.392802in}}%
\pgfpathcurveto{\pgfqpoint{2.674412in}{2.384565in}}{\pgfqpoint{2.677685in}{2.376665in}}{\pgfqpoint{2.683509in}{2.370841in}}%
\pgfpathcurveto{\pgfqpoint{2.689332in}{2.365017in}}{\pgfqpoint{2.697233in}{2.361745in}}{\pgfqpoint{2.705469in}{2.361745in}}%
\pgfpathclose%
\pgfusepath{stroke,fill}%
\end{pgfscope}%
\begin{pgfscope}%
\pgfpathrectangle{\pgfqpoint{0.100000in}{0.212622in}}{\pgfqpoint{3.696000in}{3.696000in}}%
\pgfusepath{clip}%
\pgfsetbuttcap%
\pgfsetroundjoin%
\definecolor{currentfill}{rgb}{0.121569,0.466667,0.705882}%
\pgfsetfillcolor{currentfill}%
\pgfsetfillopacity{0.459676}%
\pgfsetlinewidth{1.003750pt}%
\definecolor{currentstroke}{rgb}{0.121569,0.466667,0.705882}%
\pgfsetstrokecolor{currentstroke}%
\pgfsetstrokeopacity{0.459676}%
\pgfsetdash{}{0pt}%
\pgfpathmoveto{\pgfqpoint{2.707466in}{2.361166in}}%
\pgfpathcurveto{\pgfqpoint{2.715703in}{2.361166in}}{\pgfqpoint{2.723603in}{2.364438in}}{\pgfqpoint{2.729427in}{2.370262in}}%
\pgfpathcurveto{\pgfqpoint{2.735250in}{2.376086in}}{\pgfqpoint{2.738523in}{2.383986in}}{\pgfqpoint{2.738523in}{2.392222in}}%
\pgfpathcurveto{\pgfqpoint{2.738523in}{2.400458in}}{\pgfqpoint{2.735250in}{2.408358in}}{\pgfqpoint{2.729427in}{2.414182in}}%
\pgfpathcurveto{\pgfqpoint{2.723603in}{2.420006in}}{\pgfqpoint{2.715703in}{2.423279in}}{\pgfqpoint{2.707466in}{2.423279in}}%
\pgfpathcurveto{\pgfqpoint{2.699230in}{2.423279in}}{\pgfqpoint{2.691330in}{2.420006in}}{\pgfqpoint{2.685506in}{2.414182in}}%
\pgfpathcurveto{\pgfqpoint{2.679682in}{2.408358in}}{\pgfqpoint{2.676410in}{2.400458in}}{\pgfqpoint{2.676410in}{2.392222in}}%
\pgfpathcurveto{\pgfqpoint{2.676410in}{2.383986in}}{\pgfqpoint{2.679682in}{2.376086in}}{\pgfqpoint{2.685506in}{2.370262in}}%
\pgfpathcurveto{\pgfqpoint{2.691330in}{2.364438in}}{\pgfqpoint{2.699230in}{2.361166in}}{\pgfqpoint{2.707466in}{2.361166in}}%
\pgfpathclose%
\pgfusepath{stroke,fill}%
\end{pgfscope}%
\begin{pgfscope}%
\pgfpathrectangle{\pgfqpoint{0.100000in}{0.212622in}}{\pgfqpoint{3.696000in}{3.696000in}}%
\pgfusepath{clip}%
\pgfsetbuttcap%
\pgfsetroundjoin%
\definecolor{currentfill}{rgb}{0.121569,0.466667,0.705882}%
\pgfsetfillcolor{currentfill}%
\pgfsetfillopacity{0.460141}%
\pgfsetlinewidth{1.003750pt}%
\definecolor{currentstroke}{rgb}{0.121569,0.466667,0.705882}%
\pgfsetstrokecolor{currentstroke}%
\pgfsetstrokeopacity{0.460141}%
\pgfsetdash{}{0pt}%
\pgfpathmoveto{\pgfqpoint{2.710494in}{2.360404in}}%
\pgfpathcurveto{\pgfqpoint{2.718730in}{2.360404in}}{\pgfqpoint{2.726630in}{2.363677in}}{\pgfqpoint{2.732454in}{2.369501in}}%
\pgfpathcurveto{\pgfqpoint{2.738278in}{2.375325in}}{\pgfqpoint{2.741551in}{2.383225in}}{\pgfqpoint{2.741551in}{2.391461in}}%
\pgfpathcurveto{\pgfqpoint{2.741551in}{2.399697in}}{\pgfqpoint{2.738278in}{2.407597in}}{\pgfqpoint{2.732454in}{2.413421in}}%
\pgfpathcurveto{\pgfqpoint{2.726630in}{2.419245in}}{\pgfqpoint{2.718730in}{2.422517in}}{\pgfqpoint{2.710494in}{2.422517in}}%
\pgfpathcurveto{\pgfqpoint{2.702258in}{2.422517in}}{\pgfqpoint{2.694358in}{2.419245in}}{\pgfqpoint{2.688534in}{2.413421in}}%
\pgfpathcurveto{\pgfqpoint{2.682710in}{2.407597in}}{\pgfqpoint{2.679438in}{2.399697in}}{\pgfqpoint{2.679438in}{2.391461in}}%
\pgfpathcurveto{\pgfqpoint{2.679438in}{2.383225in}}{\pgfqpoint{2.682710in}{2.375325in}}{\pgfqpoint{2.688534in}{2.369501in}}%
\pgfpathcurveto{\pgfqpoint{2.694358in}{2.363677in}}{\pgfqpoint{2.702258in}{2.360404in}}{\pgfqpoint{2.710494in}{2.360404in}}%
\pgfpathclose%
\pgfusepath{stroke,fill}%
\end{pgfscope}%
\begin{pgfscope}%
\pgfpathrectangle{\pgfqpoint{0.100000in}{0.212622in}}{\pgfqpoint{3.696000in}{3.696000in}}%
\pgfusepath{clip}%
\pgfsetbuttcap%
\pgfsetroundjoin%
\definecolor{currentfill}{rgb}{0.121569,0.466667,0.705882}%
\pgfsetfillcolor{currentfill}%
\pgfsetfillopacity{0.460728}%
\pgfsetlinewidth{1.003750pt}%
\definecolor{currentstroke}{rgb}{0.121569,0.466667,0.705882}%
\pgfsetstrokecolor{currentstroke}%
\pgfsetstrokeopacity{0.460728}%
\pgfsetdash{}{0pt}%
\pgfpathmoveto{\pgfqpoint{2.714085in}{2.359874in}}%
\pgfpathcurveto{\pgfqpoint{2.722321in}{2.359874in}}{\pgfqpoint{2.730221in}{2.363146in}}{\pgfqpoint{2.736045in}{2.368970in}}%
\pgfpathcurveto{\pgfqpoint{2.741869in}{2.374794in}}{\pgfqpoint{2.745141in}{2.382694in}}{\pgfqpoint{2.745141in}{2.390931in}}%
\pgfpathcurveto{\pgfqpoint{2.745141in}{2.399167in}}{\pgfqpoint{2.741869in}{2.407067in}}{\pgfqpoint{2.736045in}{2.412891in}}%
\pgfpathcurveto{\pgfqpoint{2.730221in}{2.418715in}}{\pgfqpoint{2.722321in}{2.421987in}}{\pgfqpoint{2.714085in}{2.421987in}}%
\pgfpathcurveto{\pgfqpoint{2.705848in}{2.421987in}}{\pgfqpoint{2.697948in}{2.418715in}}{\pgfqpoint{2.692124in}{2.412891in}}%
\pgfpathcurveto{\pgfqpoint{2.686300in}{2.407067in}}{\pgfqpoint{2.683028in}{2.399167in}}{\pgfqpoint{2.683028in}{2.390931in}}%
\pgfpathcurveto{\pgfqpoint{2.683028in}{2.382694in}}{\pgfqpoint{2.686300in}{2.374794in}}{\pgfqpoint{2.692124in}{2.368970in}}%
\pgfpathcurveto{\pgfqpoint{2.697948in}{2.363146in}}{\pgfqpoint{2.705848in}{2.359874in}}{\pgfqpoint{2.714085in}{2.359874in}}%
\pgfpathclose%
\pgfusepath{stroke,fill}%
\end{pgfscope}%
\begin{pgfscope}%
\pgfpathrectangle{\pgfqpoint{0.100000in}{0.212622in}}{\pgfqpoint{3.696000in}{3.696000in}}%
\pgfusepath{clip}%
\pgfsetbuttcap%
\pgfsetroundjoin%
\definecolor{currentfill}{rgb}{0.121569,0.466667,0.705882}%
\pgfsetfillcolor{currentfill}%
\pgfsetfillopacity{0.461033}%
\pgfsetlinewidth{1.003750pt}%
\definecolor{currentstroke}{rgb}{0.121569,0.466667,0.705882}%
\pgfsetstrokecolor{currentstroke}%
\pgfsetstrokeopacity{0.461033}%
\pgfsetdash{}{0pt}%
\pgfpathmoveto{\pgfqpoint{2.716025in}{2.359360in}}%
\pgfpathcurveto{\pgfqpoint{2.724262in}{2.359360in}}{\pgfqpoint{2.732162in}{2.362633in}}{\pgfqpoint{2.737986in}{2.368457in}}%
\pgfpathcurveto{\pgfqpoint{2.743809in}{2.374281in}}{\pgfqpoint{2.747082in}{2.382181in}}{\pgfqpoint{2.747082in}{2.390417in}}%
\pgfpathcurveto{\pgfqpoint{2.747082in}{2.398653in}}{\pgfqpoint{2.743809in}{2.406553in}}{\pgfqpoint{2.737986in}{2.412377in}}%
\pgfpathcurveto{\pgfqpoint{2.732162in}{2.418201in}}{\pgfqpoint{2.724262in}{2.421473in}}{\pgfqpoint{2.716025in}{2.421473in}}%
\pgfpathcurveto{\pgfqpoint{2.707789in}{2.421473in}}{\pgfqpoint{2.699889in}{2.418201in}}{\pgfqpoint{2.694065in}{2.412377in}}%
\pgfpathcurveto{\pgfqpoint{2.688241in}{2.406553in}}{\pgfqpoint{2.684969in}{2.398653in}}{\pgfqpoint{2.684969in}{2.390417in}}%
\pgfpathcurveto{\pgfqpoint{2.684969in}{2.382181in}}{\pgfqpoint{2.688241in}{2.374281in}}{\pgfqpoint{2.694065in}{2.368457in}}%
\pgfpathcurveto{\pgfqpoint{2.699889in}{2.362633in}}{\pgfqpoint{2.707789in}{2.359360in}}{\pgfqpoint{2.716025in}{2.359360in}}%
\pgfpathclose%
\pgfusepath{stroke,fill}%
\end{pgfscope}%
\begin{pgfscope}%
\pgfpathrectangle{\pgfqpoint{0.100000in}{0.212622in}}{\pgfqpoint{3.696000in}{3.696000in}}%
\pgfusepath{clip}%
\pgfsetbuttcap%
\pgfsetroundjoin%
\definecolor{currentfill}{rgb}{0.121569,0.466667,0.705882}%
\pgfsetfillcolor{currentfill}%
\pgfsetfillopacity{0.461187}%
\pgfsetlinewidth{1.003750pt}%
\definecolor{currentstroke}{rgb}{0.121569,0.466667,0.705882}%
\pgfsetstrokecolor{currentstroke}%
\pgfsetstrokeopacity{0.461187}%
\pgfsetdash{}{0pt}%
\pgfpathmoveto{\pgfqpoint{2.717097in}{2.359005in}}%
\pgfpathcurveto{\pgfqpoint{2.725334in}{2.359005in}}{\pgfqpoint{2.733234in}{2.362277in}}{\pgfqpoint{2.739058in}{2.368101in}}%
\pgfpathcurveto{\pgfqpoint{2.744882in}{2.373925in}}{\pgfqpoint{2.748154in}{2.381825in}}{\pgfqpoint{2.748154in}{2.390061in}}%
\pgfpathcurveto{\pgfqpoint{2.748154in}{2.398298in}}{\pgfqpoint{2.744882in}{2.406198in}}{\pgfqpoint{2.739058in}{2.412022in}}%
\pgfpathcurveto{\pgfqpoint{2.733234in}{2.417845in}}{\pgfqpoint{2.725334in}{2.421118in}}{\pgfqpoint{2.717097in}{2.421118in}}%
\pgfpathcurveto{\pgfqpoint{2.708861in}{2.421118in}}{\pgfqpoint{2.700961in}{2.417845in}}{\pgfqpoint{2.695137in}{2.412022in}}%
\pgfpathcurveto{\pgfqpoint{2.689313in}{2.406198in}}{\pgfqpoint{2.686041in}{2.398298in}}{\pgfqpoint{2.686041in}{2.390061in}}%
\pgfpathcurveto{\pgfqpoint{2.686041in}{2.381825in}}{\pgfqpoint{2.689313in}{2.373925in}}{\pgfqpoint{2.695137in}{2.368101in}}%
\pgfpathcurveto{\pgfqpoint{2.700961in}{2.362277in}}{\pgfqpoint{2.708861in}{2.359005in}}{\pgfqpoint{2.717097in}{2.359005in}}%
\pgfpathclose%
\pgfusepath{stroke,fill}%
\end{pgfscope}%
\begin{pgfscope}%
\pgfpathrectangle{\pgfqpoint{0.100000in}{0.212622in}}{\pgfqpoint{3.696000in}{3.696000in}}%
\pgfusepath{clip}%
\pgfsetbuttcap%
\pgfsetroundjoin%
\definecolor{currentfill}{rgb}{0.121569,0.466667,0.705882}%
\pgfsetfillcolor{currentfill}%
\pgfsetfillopacity{0.461492}%
\pgfsetlinewidth{1.003750pt}%
\definecolor{currentstroke}{rgb}{0.121569,0.466667,0.705882}%
\pgfsetstrokecolor{currentstroke}%
\pgfsetstrokeopacity{0.461492}%
\pgfsetdash{}{0pt}%
\pgfpathmoveto{\pgfqpoint{2.719118in}{2.358560in}}%
\pgfpathcurveto{\pgfqpoint{2.727355in}{2.358560in}}{\pgfqpoint{2.735255in}{2.361832in}}{\pgfqpoint{2.741079in}{2.367656in}}%
\pgfpathcurveto{\pgfqpoint{2.746903in}{2.373480in}}{\pgfqpoint{2.750175in}{2.381380in}}{\pgfqpoint{2.750175in}{2.389616in}}%
\pgfpathcurveto{\pgfqpoint{2.750175in}{2.397852in}}{\pgfqpoint{2.746903in}{2.405752in}}{\pgfqpoint{2.741079in}{2.411576in}}%
\pgfpathcurveto{\pgfqpoint{2.735255in}{2.417400in}}{\pgfqpoint{2.727355in}{2.420673in}}{\pgfqpoint{2.719118in}{2.420673in}}%
\pgfpathcurveto{\pgfqpoint{2.710882in}{2.420673in}}{\pgfqpoint{2.702982in}{2.417400in}}{\pgfqpoint{2.697158in}{2.411576in}}%
\pgfpathcurveto{\pgfqpoint{2.691334in}{2.405752in}}{\pgfqpoint{2.688062in}{2.397852in}}{\pgfqpoint{2.688062in}{2.389616in}}%
\pgfpathcurveto{\pgfqpoint{2.688062in}{2.381380in}}{\pgfqpoint{2.691334in}{2.373480in}}{\pgfqpoint{2.697158in}{2.367656in}}%
\pgfpathcurveto{\pgfqpoint{2.702982in}{2.361832in}}{\pgfqpoint{2.710882in}{2.358560in}}{\pgfqpoint{2.719118in}{2.358560in}}%
\pgfpathclose%
\pgfusepath{stroke,fill}%
\end{pgfscope}%
\begin{pgfscope}%
\pgfpathrectangle{\pgfqpoint{0.100000in}{0.212622in}}{\pgfqpoint{3.696000in}{3.696000in}}%
\pgfusepath{clip}%
\pgfsetbuttcap%
\pgfsetroundjoin%
\definecolor{currentfill}{rgb}{0.121569,0.466667,0.705882}%
\pgfsetfillcolor{currentfill}%
\pgfsetfillopacity{0.461613}%
\pgfsetlinewidth{1.003750pt}%
\definecolor{currentstroke}{rgb}{0.121569,0.466667,0.705882}%
\pgfsetstrokecolor{currentstroke}%
\pgfsetstrokeopacity{0.461613}%
\pgfsetdash{}{0pt}%
\pgfpathmoveto{\pgfqpoint{1.205128in}{1.969641in}}%
\pgfpathcurveto{\pgfqpoint{1.213365in}{1.969641in}}{\pgfqpoint{1.221265in}{1.972913in}}{\pgfqpoint{1.227089in}{1.978737in}}%
\pgfpathcurveto{\pgfqpoint{1.232913in}{1.984561in}}{\pgfqpoint{1.236185in}{1.992461in}}{\pgfqpoint{1.236185in}{2.000697in}}%
\pgfpathcurveto{\pgfqpoint{1.236185in}{2.008934in}}{\pgfqpoint{1.232913in}{2.016834in}}{\pgfqpoint{1.227089in}{2.022658in}}%
\pgfpathcurveto{\pgfqpoint{1.221265in}{2.028482in}}{\pgfqpoint{1.213365in}{2.031754in}}{\pgfqpoint{1.205128in}{2.031754in}}%
\pgfpathcurveto{\pgfqpoint{1.196892in}{2.031754in}}{\pgfqpoint{1.188992in}{2.028482in}}{\pgfqpoint{1.183168in}{2.022658in}}%
\pgfpathcurveto{\pgfqpoint{1.177344in}{2.016834in}}{\pgfqpoint{1.174072in}{2.008934in}}{\pgfqpoint{1.174072in}{2.000697in}}%
\pgfpathcurveto{\pgfqpoint{1.174072in}{1.992461in}}{\pgfqpoint{1.177344in}{1.984561in}}{\pgfqpoint{1.183168in}{1.978737in}}%
\pgfpathcurveto{\pgfqpoint{1.188992in}{1.972913in}}{\pgfqpoint{1.196892in}{1.969641in}}{\pgfqpoint{1.205128in}{1.969641in}}%
\pgfpathclose%
\pgfusepath{stroke,fill}%
\end{pgfscope}%
\begin{pgfscope}%
\pgfpathrectangle{\pgfqpoint{0.100000in}{0.212622in}}{\pgfqpoint{3.696000in}{3.696000in}}%
\pgfusepath{clip}%
\pgfsetbuttcap%
\pgfsetroundjoin%
\definecolor{currentfill}{rgb}{0.121569,0.466667,0.705882}%
\pgfsetfillcolor{currentfill}%
\pgfsetfillopacity{0.461633}%
\pgfsetlinewidth{1.003750pt}%
\definecolor{currentstroke}{rgb}{0.121569,0.466667,0.705882}%
\pgfsetstrokecolor{currentstroke}%
\pgfsetstrokeopacity{0.461633}%
\pgfsetdash{}{0pt}%
\pgfpathmoveto{\pgfqpoint{2.720221in}{2.358113in}}%
\pgfpathcurveto{\pgfqpoint{2.728457in}{2.358113in}}{\pgfqpoint{2.736357in}{2.361385in}}{\pgfqpoint{2.742181in}{2.367209in}}%
\pgfpathcurveto{\pgfqpoint{2.748005in}{2.373033in}}{\pgfqpoint{2.751277in}{2.380933in}}{\pgfqpoint{2.751277in}{2.389170in}}%
\pgfpathcurveto{\pgfqpoint{2.751277in}{2.397406in}}{\pgfqpoint{2.748005in}{2.405306in}}{\pgfqpoint{2.742181in}{2.411130in}}%
\pgfpathcurveto{\pgfqpoint{2.736357in}{2.416954in}}{\pgfqpoint{2.728457in}{2.420226in}}{\pgfqpoint{2.720221in}{2.420226in}}%
\pgfpathcurveto{\pgfqpoint{2.711984in}{2.420226in}}{\pgfqpoint{2.704084in}{2.416954in}}{\pgfqpoint{2.698260in}{2.411130in}}%
\pgfpathcurveto{\pgfqpoint{2.692436in}{2.405306in}}{\pgfqpoint{2.689164in}{2.397406in}}{\pgfqpoint{2.689164in}{2.389170in}}%
\pgfpathcurveto{\pgfqpoint{2.689164in}{2.380933in}}{\pgfqpoint{2.692436in}{2.373033in}}{\pgfqpoint{2.698260in}{2.367209in}}%
\pgfpathcurveto{\pgfqpoint{2.704084in}{2.361385in}}{\pgfqpoint{2.711984in}{2.358113in}}{\pgfqpoint{2.720221in}{2.358113in}}%
\pgfpathclose%
\pgfusepath{stroke,fill}%
\end{pgfscope}%
\begin{pgfscope}%
\pgfpathrectangle{\pgfqpoint{0.100000in}{0.212622in}}{\pgfqpoint{3.696000in}{3.696000in}}%
\pgfusepath{clip}%
\pgfsetbuttcap%
\pgfsetroundjoin%
\definecolor{currentfill}{rgb}{0.121569,0.466667,0.705882}%
\pgfsetfillcolor{currentfill}%
\pgfsetfillopacity{0.462033}%
\pgfsetlinewidth{1.003750pt}%
\definecolor{currentstroke}{rgb}{0.121569,0.466667,0.705882}%
\pgfsetstrokecolor{currentstroke}%
\pgfsetstrokeopacity{0.462033}%
\pgfsetdash{}{0pt}%
\pgfpathmoveto{\pgfqpoint{2.723056in}{2.357453in}}%
\pgfpathcurveto{\pgfqpoint{2.731292in}{2.357453in}}{\pgfqpoint{2.739192in}{2.360725in}}{\pgfqpoint{2.745016in}{2.366549in}}%
\pgfpathcurveto{\pgfqpoint{2.750840in}{2.372373in}}{\pgfqpoint{2.754112in}{2.380273in}}{\pgfqpoint{2.754112in}{2.388510in}}%
\pgfpathcurveto{\pgfqpoint{2.754112in}{2.396746in}}{\pgfqpoint{2.750840in}{2.404646in}}{\pgfqpoint{2.745016in}{2.410470in}}%
\pgfpathcurveto{\pgfqpoint{2.739192in}{2.416294in}}{\pgfqpoint{2.731292in}{2.419566in}}{\pgfqpoint{2.723056in}{2.419566in}}%
\pgfpathcurveto{\pgfqpoint{2.714819in}{2.419566in}}{\pgfqpoint{2.706919in}{2.416294in}}{\pgfqpoint{2.701095in}{2.410470in}}%
\pgfpathcurveto{\pgfqpoint{2.695271in}{2.404646in}}{\pgfqpoint{2.691999in}{2.396746in}}{\pgfqpoint{2.691999in}{2.388510in}}%
\pgfpathcurveto{\pgfqpoint{2.691999in}{2.380273in}}{\pgfqpoint{2.695271in}{2.372373in}}{\pgfqpoint{2.701095in}{2.366549in}}%
\pgfpathcurveto{\pgfqpoint{2.706919in}{2.360725in}}{\pgfqpoint{2.714819in}{2.357453in}}{\pgfqpoint{2.723056in}{2.357453in}}%
\pgfpathclose%
\pgfusepath{stroke,fill}%
\end{pgfscope}%
\begin{pgfscope}%
\pgfpathrectangle{\pgfqpoint{0.100000in}{0.212622in}}{\pgfqpoint{3.696000in}{3.696000in}}%
\pgfusepath{clip}%
\pgfsetbuttcap%
\pgfsetroundjoin%
\definecolor{currentfill}{rgb}{0.121569,0.466667,0.705882}%
\pgfsetfillcolor{currentfill}%
\pgfsetfillopacity{0.462548}%
\pgfsetlinewidth{1.003750pt}%
\definecolor{currentstroke}{rgb}{0.121569,0.466667,0.705882}%
\pgfsetstrokecolor{currentstroke}%
\pgfsetstrokeopacity{0.462548}%
\pgfsetdash{}{0pt}%
\pgfpathmoveto{\pgfqpoint{2.726392in}{2.356950in}}%
\pgfpathcurveto{\pgfqpoint{2.734628in}{2.356950in}}{\pgfqpoint{2.742528in}{2.360222in}}{\pgfqpoint{2.748352in}{2.366046in}}%
\pgfpathcurveto{\pgfqpoint{2.754176in}{2.371870in}}{\pgfqpoint{2.757449in}{2.379770in}}{\pgfqpoint{2.757449in}{2.388006in}}%
\pgfpathcurveto{\pgfqpoint{2.757449in}{2.396242in}}{\pgfqpoint{2.754176in}{2.404142in}}{\pgfqpoint{2.748352in}{2.409966in}}%
\pgfpathcurveto{\pgfqpoint{2.742528in}{2.415790in}}{\pgfqpoint{2.734628in}{2.419063in}}{\pgfqpoint{2.726392in}{2.419063in}}%
\pgfpathcurveto{\pgfqpoint{2.718156in}{2.419063in}}{\pgfqpoint{2.710256in}{2.415790in}}{\pgfqpoint{2.704432in}{2.409966in}}%
\pgfpathcurveto{\pgfqpoint{2.698608in}{2.404142in}}{\pgfqpoint{2.695336in}{2.396242in}}{\pgfqpoint{2.695336in}{2.388006in}}%
\pgfpathcurveto{\pgfqpoint{2.695336in}{2.379770in}}{\pgfqpoint{2.698608in}{2.371870in}}{\pgfqpoint{2.704432in}{2.366046in}}%
\pgfpathcurveto{\pgfqpoint{2.710256in}{2.360222in}}{\pgfqpoint{2.718156in}{2.356950in}}{\pgfqpoint{2.726392in}{2.356950in}}%
\pgfpathclose%
\pgfusepath{stroke,fill}%
\end{pgfscope}%
\begin{pgfscope}%
\pgfpathrectangle{\pgfqpoint{0.100000in}{0.212622in}}{\pgfqpoint{3.696000in}{3.696000in}}%
\pgfusepath{clip}%
\pgfsetbuttcap%
\pgfsetroundjoin%
\definecolor{currentfill}{rgb}{0.121569,0.466667,0.705882}%
\pgfsetfillcolor{currentfill}%
\pgfsetfillopacity{0.463214}%
\pgfsetlinewidth{1.003750pt}%
\definecolor{currentstroke}{rgb}{0.121569,0.466667,0.705882}%
\pgfsetstrokecolor{currentstroke}%
\pgfsetstrokeopacity{0.463214}%
\pgfsetdash{}{0pt}%
\pgfpathmoveto{\pgfqpoint{2.730697in}{2.356205in}}%
\pgfpathcurveto{\pgfqpoint{2.738933in}{2.356205in}}{\pgfqpoint{2.746833in}{2.359477in}}{\pgfqpoint{2.752657in}{2.365301in}}%
\pgfpathcurveto{\pgfqpoint{2.758481in}{2.371125in}}{\pgfqpoint{2.761753in}{2.379025in}}{\pgfqpoint{2.761753in}{2.387262in}}%
\pgfpathcurveto{\pgfqpoint{2.761753in}{2.395498in}}{\pgfqpoint{2.758481in}{2.403398in}}{\pgfqpoint{2.752657in}{2.409222in}}%
\pgfpathcurveto{\pgfqpoint{2.746833in}{2.415046in}}{\pgfqpoint{2.738933in}{2.418318in}}{\pgfqpoint{2.730697in}{2.418318in}}%
\pgfpathcurveto{\pgfqpoint{2.722460in}{2.418318in}}{\pgfqpoint{2.714560in}{2.415046in}}{\pgfqpoint{2.708736in}{2.409222in}}%
\pgfpathcurveto{\pgfqpoint{2.702912in}{2.403398in}}{\pgfqpoint{2.699640in}{2.395498in}}{\pgfqpoint{2.699640in}{2.387262in}}%
\pgfpathcurveto{\pgfqpoint{2.699640in}{2.379025in}}{\pgfqpoint{2.702912in}{2.371125in}}{\pgfqpoint{2.708736in}{2.365301in}}%
\pgfpathcurveto{\pgfqpoint{2.714560in}{2.359477in}}{\pgfqpoint{2.722460in}{2.356205in}}{\pgfqpoint{2.730697in}{2.356205in}}%
\pgfpathclose%
\pgfusepath{stroke,fill}%
\end{pgfscope}%
\begin{pgfscope}%
\pgfpathrectangle{\pgfqpoint{0.100000in}{0.212622in}}{\pgfqpoint{3.696000in}{3.696000in}}%
\pgfusepath{clip}%
\pgfsetbuttcap%
\pgfsetroundjoin%
\definecolor{currentfill}{rgb}{0.121569,0.466667,0.705882}%
\pgfsetfillcolor{currentfill}%
\pgfsetfillopacity{0.464026}%
\pgfsetlinewidth{1.003750pt}%
\definecolor{currentstroke}{rgb}{0.121569,0.466667,0.705882}%
\pgfsetstrokecolor{currentstroke}%
\pgfsetstrokeopacity{0.464026}%
\pgfsetdash{}{0pt}%
\pgfpathmoveto{\pgfqpoint{2.736078in}{2.355268in}}%
\pgfpathcurveto{\pgfqpoint{2.744315in}{2.355268in}}{\pgfqpoint{2.752215in}{2.358541in}}{\pgfqpoint{2.758039in}{2.364364in}}%
\pgfpathcurveto{\pgfqpoint{2.763863in}{2.370188in}}{\pgfqpoint{2.767135in}{2.378088in}}{\pgfqpoint{2.767135in}{2.386325in}}%
\pgfpathcurveto{\pgfqpoint{2.767135in}{2.394561in}}{\pgfqpoint{2.763863in}{2.402461in}}{\pgfqpoint{2.758039in}{2.408285in}}%
\pgfpathcurveto{\pgfqpoint{2.752215in}{2.414109in}}{\pgfqpoint{2.744315in}{2.417381in}}{\pgfqpoint{2.736078in}{2.417381in}}%
\pgfpathcurveto{\pgfqpoint{2.727842in}{2.417381in}}{\pgfqpoint{2.719942in}{2.414109in}}{\pgfqpoint{2.714118in}{2.408285in}}%
\pgfpathcurveto{\pgfqpoint{2.708294in}{2.402461in}}{\pgfqpoint{2.705022in}{2.394561in}}{\pgfqpoint{2.705022in}{2.386325in}}%
\pgfpathcurveto{\pgfqpoint{2.705022in}{2.378088in}}{\pgfqpoint{2.708294in}{2.370188in}}{\pgfqpoint{2.714118in}{2.364364in}}%
\pgfpathcurveto{\pgfqpoint{2.719942in}{2.358541in}}{\pgfqpoint{2.727842in}{2.355268in}}{\pgfqpoint{2.736078in}{2.355268in}}%
\pgfpathclose%
\pgfusepath{stroke,fill}%
\end{pgfscope}%
\begin{pgfscope}%
\pgfpathrectangle{\pgfqpoint{0.100000in}{0.212622in}}{\pgfqpoint{3.696000in}{3.696000in}}%
\pgfusepath{clip}%
\pgfsetbuttcap%
\pgfsetroundjoin%
\definecolor{currentfill}{rgb}{0.121569,0.466667,0.705882}%
\pgfsetfillcolor{currentfill}%
\pgfsetfillopacity{0.464657}%
\pgfsetlinewidth{1.003750pt}%
\definecolor{currentstroke}{rgb}{0.121569,0.466667,0.705882}%
\pgfsetstrokecolor{currentstroke}%
\pgfsetstrokeopacity{0.464657}%
\pgfsetdash{}{0pt}%
\pgfpathmoveto{\pgfqpoint{1.195256in}{1.957411in}}%
\pgfpathcurveto{\pgfqpoint{1.203493in}{1.957411in}}{\pgfqpoint{1.211393in}{1.960683in}}{\pgfqpoint{1.217216in}{1.966507in}}%
\pgfpathcurveto{\pgfqpoint{1.223040in}{1.972331in}}{\pgfqpoint{1.226313in}{1.980231in}}{\pgfqpoint{1.226313in}{1.988468in}}%
\pgfpathcurveto{\pgfqpoint{1.226313in}{1.996704in}}{\pgfqpoint{1.223040in}{2.004604in}}{\pgfqpoint{1.217216in}{2.010428in}}%
\pgfpathcurveto{\pgfqpoint{1.211393in}{2.016252in}}{\pgfqpoint{1.203493in}{2.019524in}}{\pgfqpoint{1.195256in}{2.019524in}}%
\pgfpathcurveto{\pgfqpoint{1.187020in}{2.019524in}}{\pgfqpoint{1.179120in}{2.016252in}}{\pgfqpoint{1.173296in}{2.010428in}}%
\pgfpathcurveto{\pgfqpoint{1.167472in}{2.004604in}}{\pgfqpoint{1.164200in}{1.996704in}}{\pgfqpoint{1.164200in}{1.988468in}}%
\pgfpathcurveto{\pgfqpoint{1.164200in}{1.980231in}}{\pgfqpoint{1.167472in}{1.972331in}}{\pgfqpoint{1.173296in}{1.966507in}}%
\pgfpathcurveto{\pgfqpoint{1.179120in}{1.960683in}}{\pgfqpoint{1.187020in}{1.957411in}}{\pgfqpoint{1.195256in}{1.957411in}}%
\pgfpathclose%
\pgfusepath{stroke,fill}%
\end{pgfscope}%
\begin{pgfscope}%
\pgfpathrectangle{\pgfqpoint{0.100000in}{0.212622in}}{\pgfqpoint{3.696000in}{3.696000in}}%
\pgfusepath{clip}%
\pgfsetbuttcap%
\pgfsetroundjoin%
\definecolor{currentfill}{rgb}{0.121569,0.466667,0.705882}%
\pgfsetfillcolor{currentfill}%
\pgfsetfillopacity{0.465155}%
\pgfsetlinewidth{1.003750pt}%
\definecolor{currentstroke}{rgb}{0.121569,0.466667,0.705882}%
\pgfsetstrokecolor{currentstroke}%
\pgfsetstrokeopacity{0.465155}%
\pgfsetdash{}{0pt}%
\pgfpathmoveto{\pgfqpoint{2.743348in}{2.354197in}}%
\pgfpathcurveto{\pgfqpoint{2.751584in}{2.354197in}}{\pgfqpoint{2.759484in}{2.357470in}}{\pgfqpoint{2.765308in}{2.363293in}}%
\pgfpathcurveto{\pgfqpoint{2.771132in}{2.369117in}}{\pgfqpoint{2.774404in}{2.377017in}}{\pgfqpoint{2.774404in}{2.385254in}}%
\pgfpathcurveto{\pgfqpoint{2.774404in}{2.393490in}}{\pgfqpoint{2.771132in}{2.401390in}}{\pgfqpoint{2.765308in}{2.407214in}}%
\pgfpathcurveto{\pgfqpoint{2.759484in}{2.413038in}}{\pgfqpoint{2.751584in}{2.416310in}}{\pgfqpoint{2.743348in}{2.416310in}}%
\pgfpathcurveto{\pgfqpoint{2.735112in}{2.416310in}}{\pgfqpoint{2.727212in}{2.413038in}}{\pgfqpoint{2.721388in}{2.407214in}}%
\pgfpathcurveto{\pgfqpoint{2.715564in}{2.401390in}}{\pgfqpoint{2.712291in}{2.393490in}}{\pgfqpoint{2.712291in}{2.385254in}}%
\pgfpathcurveto{\pgfqpoint{2.712291in}{2.377017in}}{\pgfqpoint{2.715564in}{2.369117in}}{\pgfqpoint{2.721388in}{2.363293in}}%
\pgfpathcurveto{\pgfqpoint{2.727212in}{2.357470in}}{\pgfqpoint{2.735112in}{2.354197in}}{\pgfqpoint{2.743348in}{2.354197in}}%
\pgfpathclose%
\pgfusepath{stroke,fill}%
\end{pgfscope}%
\begin{pgfscope}%
\pgfpathrectangle{\pgfqpoint{0.100000in}{0.212622in}}{\pgfqpoint{3.696000in}{3.696000in}}%
\pgfusepath{clip}%
\pgfsetbuttcap%
\pgfsetroundjoin%
\definecolor{currentfill}{rgb}{0.121569,0.466667,0.705882}%
\pgfsetfillcolor{currentfill}%
\pgfsetfillopacity{0.466395}%
\pgfsetlinewidth{1.003750pt}%
\definecolor{currentstroke}{rgb}{0.121569,0.466667,0.705882}%
\pgfsetstrokecolor{currentstroke}%
\pgfsetstrokeopacity{0.466395}%
\pgfsetdash{}{0pt}%
\pgfpathmoveto{\pgfqpoint{2.751329in}{2.353169in}}%
\pgfpathcurveto{\pgfqpoint{2.759565in}{2.353169in}}{\pgfqpoint{2.767465in}{2.356441in}}{\pgfqpoint{2.773289in}{2.362265in}}%
\pgfpathcurveto{\pgfqpoint{2.779113in}{2.368089in}}{\pgfqpoint{2.782386in}{2.375989in}}{\pgfqpoint{2.782386in}{2.384225in}}%
\pgfpathcurveto{\pgfqpoint{2.782386in}{2.392461in}}{\pgfqpoint{2.779113in}{2.400362in}}{\pgfqpoint{2.773289in}{2.406185in}}%
\pgfpathcurveto{\pgfqpoint{2.767465in}{2.412009in}}{\pgfqpoint{2.759565in}{2.415282in}}{\pgfqpoint{2.751329in}{2.415282in}}%
\pgfpathcurveto{\pgfqpoint{2.743093in}{2.415282in}}{\pgfqpoint{2.735193in}{2.412009in}}{\pgfqpoint{2.729369in}{2.406185in}}%
\pgfpathcurveto{\pgfqpoint{2.723545in}{2.400362in}}{\pgfqpoint{2.720273in}{2.392461in}}{\pgfqpoint{2.720273in}{2.384225in}}%
\pgfpathcurveto{\pgfqpoint{2.720273in}{2.375989in}}{\pgfqpoint{2.723545in}{2.368089in}}{\pgfqpoint{2.729369in}{2.362265in}}%
\pgfpathcurveto{\pgfqpoint{2.735193in}{2.356441in}}{\pgfqpoint{2.743093in}{2.353169in}}{\pgfqpoint{2.751329in}{2.353169in}}%
\pgfpathclose%
\pgfusepath{stroke,fill}%
\end{pgfscope}%
\begin{pgfscope}%
\pgfpathrectangle{\pgfqpoint{0.100000in}{0.212622in}}{\pgfqpoint{3.696000in}{3.696000in}}%
\pgfusepath{clip}%
\pgfsetbuttcap%
\pgfsetroundjoin%
\definecolor{currentfill}{rgb}{0.121569,0.466667,0.705882}%
\pgfsetfillcolor{currentfill}%
\pgfsetfillopacity{0.467110}%
\pgfsetlinewidth{1.003750pt}%
\definecolor{currentstroke}{rgb}{0.121569,0.466667,0.705882}%
\pgfsetstrokecolor{currentstroke}%
\pgfsetstrokeopacity{0.467110}%
\pgfsetdash{}{0pt}%
\pgfpathmoveto{\pgfqpoint{2.755763in}{2.352950in}}%
\pgfpathcurveto{\pgfqpoint{2.764000in}{2.352950in}}{\pgfqpoint{2.771900in}{2.356222in}}{\pgfqpoint{2.777724in}{2.362046in}}%
\pgfpathcurveto{\pgfqpoint{2.783548in}{2.367870in}}{\pgfqpoint{2.786820in}{2.375770in}}{\pgfqpoint{2.786820in}{2.384007in}}%
\pgfpathcurveto{\pgfqpoint{2.786820in}{2.392243in}}{\pgfqpoint{2.783548in}{2.400143in}}{\pgfqpoint{2.777724in}{2.405967in}}%
\pgfpathcurveto{\pgfqpoint{2.771900in}{2.411791in}}{\pgfqpoint{2.764000in}{2.415063in}}{\pgfqpoint{2.755763in}{2.415063in}}%
\pgfpathcurveto{\pgfqpoint{2.747527in}{2.415063in}}{\pgfqpoint{2.739627in}{2.411791in}}{\pgfqpoint{2.733803in}{2.405967in}}%
\pgfpathcurveto{\pgfqpoint{2.727979in}{2.400143in}}{\pgfqpoint{2.724707in}{2.392243in}}{\pgfqpoint{2.724707in}{2.384007in}}%
\pgfpathcurveto{\pgfqpoint{2.724707in}{2.375770in}}{\pgfqpoint{2.727979in}{2.367870in}}{\pgfqpoint{2.733803in}{2.362046in}}%
\pgfpathcurveto{\pgfqpoint{2.739627in}{2.356222in}}{\pgfqpoint{2.747527in}{2.352950in}}{\pgfqpoint{2.755763in}{2.352950in}}%
\pgfpathclose%
\pgfusepath{stroke,fill}%
\end{pgfscope}%
\begin{pgfscope}%
\pgfpathrectangle{\pgfqpoint{0.100000in}{0.212622in}}{\pgfqpoint{3.696000in}{3.696000in}}%
\pgfusepath{clip}%
\pgfsetbuttcap%
\pgfsetroundjoin%
\definecolor{currentfill}{rgb}{0.121569,0.466667,0.705882}%
\pgfsetfillcolor{currentfill}%
\pgfsetfillopacity{0.467451}%
\pgfsetlinewidth{1.003750pt}%
\definecolor{currentstroke}{rgb}{0.121569,0.466667,0.705882}%
\pgfsetstrokecolor{currentstroke}%
\pgfsetstrokeopacity{0.467451}%
\pgfsetdash{}{0pt}%
\pgfpathmoveto{\pgfqpoint{2.758175in}{2.352409in}}%
\pgfpathcurveto{\pgfqpoint{2.766411in}{2.352409in}}{\pgfqpoint{2.774311in}{2.355682in}}{\pgfqpoint{2.780135in}{2.361506in}}%
\pgfpathcurveto{\pgfqpoint{2.785959in}{2.367330in}}{\pgfqpoint{2.789231in}{2.375230in}}{\pgfqpoint{2.789231in}{2.383466in}}%
\pgfpathcurveto{\pgfqpoint{2.789231in}{2.391702in}}{\pgfqpoint{2.785959in}{2.399602in}}{\pgfqpoint{2.780135in}{2.405426in}}%
\pgfpathcurveto{\pgfqpoint{2.774311in}{2.411250in}}{\pgfqpoint{2.766411in}{2.414522in}}{\pgfqpoint{2.758175in}{2.414522in}}%
\pgfpathcurveto{\pgfqpoint{2.749939in}{2.414522in}}{\pgfqpoint{2.742039in}{2.411250in}}{\pgfqpoint{2.736215in}{2.405426in}}%
\pgfpathcurveto{\pgfqpoint{2.730391in}{2.399602in}}{\pgfqpoint{2.727118in}{2.391702in}}{\pgfqpoint{2.727118in}{2.383466in}}%
\pgfpathcurveto{\pgfqpoint{2.727118in}{2.375230in}}{\pgfqpoint{2.730391in}{2.367330in}}{\pgfqpoint{2.736215in}{2.361506in}}%
\pgfpathcurveto{\pgfqpoint{2.742039in}{2.355682in}}{\pgfqpoint{2.749939in}{2.352409in}}{\pgfqpoint{2.758175in}{2.352409in}}%
\pgfpathclose%
\pgfusepath{stroke,fill}%
\end{pgfscope}%
\begin{pgfscope}%
\pgfpathrectangle{\pgfqpoint{0.100000in}{0.212622in}}{\pgfqpoint{3.696000in}{3.696000in}}%
\pgfusepath{clip}%
\pgfsetbuttcap%
\pgfsetroundjoin%
\definecolor{currentfill}{rgb}{0.121569,0.466667,0.705882}%
\pgfsetfillcolor{currentfill}%
\pgfsetfillopacity{0.467899}%
\pgfsetlinewidth{1.003750pt}%
\definecolor{currentstroke}{rgb}{0.121569,0.466667,0.705882}%
\pgfsetstrokecolor{currentstroke}%
\pgfsetstrokeopacity{0.467899}%
\pgfsetdash{}{0pt}%
\pgfpathmoveto{\pgfqpoint{2.762098in}{2.350891in}}%
\pgfpathcurveto{\pgfqpoint{2.770334in}{2.350891in}}{\pgfqpoint{2.778234in}{2.354163in}}{\pgfqpoint{2.784058in}{2.359987in}}%
\pgfpathcurveto{\pgfqpoint{2.789882in}{2.365811in}}{\pgfqpoint{2.793154in}{2.373711in}}{\pgfqpoint{2.793154in}{2.381948in}}%
\pgfpathcurveto{\pgfqpoint{2.793154in}{2.390184in}}{\pgfqpoint{2.789882in}{2.398084in}}{\pgfqpoint{2.784058in}{2.403908in}}%
\pgfpathcurveto{\pgfqpoint{2.778234in}{2.409732in}}{\pgfqpoint{2.770334in}{2.413004in}}{\pgfqpoint{2.762098in}{2.413004in}}%
\pgfpathcurveto{\pgfqpoint{2.753861in}{2.413004in}}{\pgfqpoint{2.745961in}{2.409732in}}{\pgfqpoint{2.740137in}{2.403908in}}%
\pgfpathcurveto{\pgfqpoint{2.734313in}{2.398084in}}{\pgfqpoint{2.731041in}{2.390184in}}{\pgfqpoint{2.731041in}{2.381948in}}%
\pgfpathcurveto{\pgfqpoint{2.731041in}{2.373711in}}{\pgfqpoint{2.734313in}{2.365811in}}{\pgfqpoint{2.740137in}{2.359987in}}%
\pgfpathcurveto{\pgfqpoint{2.745961in}{2.354163in}}{\pgfqpoint{2.753861in}{2.350891in}}{\pgfqpoint{2.762098in}{2.350891in}}%
\pgfpathclose%
\pgfusepath{stroke,fill}%
\end{pgfscope}%
\begin{pgfscope}%
\pgfpathrectangle{\pgfqpoint{0.100000in}{0.212622in}}{\pgfqpoint{3.696000in}{3.696000in}}%
\pgfusepath{clip}%
\pgfsetbuttcap%
\pgfsetroundjoin%
\definecolor{currentfill}{rgb}{0.121569,0.466667,0.705882}%
\pgfsetfillcolor{currentfill}%
\pgfsetfillopacity{0.468560}%
\pgfsetlinewidth{1.003750pt}%
\definecolor{currentstroke}{rgb}{0.121569,0.466667,0.705882}%
\pgfsetstrokecolor{currentstroke}%
\pgfsetstrokeopacity{0.468560}%
\pgfsetdash{}{0pt}%
\pgfpathmoveto{\pgfqpoint{2.767403in}{2.349449in}}%
\pgfpathcurveto{\pgfqpoint{2.775639in}{2.349449in}}{\pgfqpoint{2.783539in}{2.352721in}}{\pgfqpoint{2.789363in}{2.358545in}}%
\pgfpathcurveto{\pgfqpoint{2.795187in}{2.364369in}}{\pgfqpoint{2.798460in}{2.372269in}}{\pgfqpoint{2.798460in}{2.380505in}}%
\pgfpathcurveto{\pgfqpoint{2.798460in}{2.388741in}}{\pgfqpoint{2.795187in}{2.396642in}}{\pgfqpoint{2.789363in}{2.402465in}}%
\pgfpathcurveto{\pgfqpoint{2.783539in}{2.408289in}}{\pgfqpoint{2.775639in}{2.411562in}}{\pgfqpoint{2.767403in}{2.411562in}}%
\pgfpathcurveto{\pgfqpoint{2.759167in}{2.411562in}}{\pgfqpoint{2.751267in}{2.408289in}}{\pgfqpoint{2.745443in}{2.402465in}}%
\pgfpathcurveto{\pgfqpoint{2.739619in}{2.396642in}}{\pgfqpoint{2.736347in}{2.388741in}}{\pgfqpoint{2.736347in}{2.380505in}}%
\pgfpathcurveto{\pgfqpoint{2.736347in}{2.372269in}}{\pgfqpoint{2.739619in}{2.364369in}}{\pgfqpoint{2.745443in}{2.358545in}}%
\pgfpathcurveto{\pgfqpoint{2.751267in}{2.352721in}}{\pgfqpoint{2.759167in}{2.349449in}}{\pgfqpoint{2.767403in}{2.349449in}}%
\pgfpathclose%
\pgfusepath{stroke,fill}%
\end{pgfscope}%
\begin{pgfscope}%
\pgfpathrectangle{\pgfqpoint{0.100000in}{0.212622in}}{\pgfqpoint{3.696000in}{3.696000in}}%
\pgfusepath{clip}%
\pgfsetbuttcap%
\pgfsetroundjoin%
\definecolor{currentfill}{rgb}{0.121569,0.466667,0.705882}%
\pgfsetfillcolor{currentfill}%
\pgfsetfillopacity{0.468942}%
\pgfsetlinewidth{1.003750pt}%
\definecolor{currentstroke}{rgb}{0.121569,0.466667,0.705882}%
\pgfsetstrokecolor{currentstroke}%
\pgfsetstrokeopacity{0.468942}%
\pgfsetdash{}{0pt}%
\pgfpathmoveto{\pgfqpoint{2.770334in}{2.348819in}}%
\pgfpathcurveto{\pgfqpoint{2.778571in}{2.348819in}}{\pgfqpoint{2.786471in}{2.352091in}}{\pgfqpoint{2.792295in}{2.357915in}}%
\pgfpathcurveto{\pgfqpoint{2.798119in}{2.363739in}}{\pgfqpoint{2.801391in}{2.371639in}}{\pgfqpoint{2.801391in}{2.379876in}}%
\pgfpathcurveto{\pgfqpoint{2.801391in}{2.388112in}}{\pgfqpoint{2.798119in}{2.396012in}}{\pgfqpoint{2.792295in}{2.401836in}}%
\pgfpathcurveto{\pgfqpoint{2.786471in}{2.407660in}}{\pgfqpoint{2.778571in}{2.410932in}}{\pgfqpoint{2.770334in}{2.410932in}}%
\pgfpathcurveto{\pgfqpoint{2.762098in}{2.410932in}}{\pgfqpoint{2.754198in}{2.407660in}}{\pgfqpoint{2.748374in}{2.401836in}}%
\pgfpathcurveto{\pgfqpoint{2.742550in}{2.396012in}}{\pgfqpoint{2.739278in}{2.388112in}}{\pgfqpoint{2.739278in}{2.379876in}}%
\pgfpathcurveto{\pgfqpoint{2.739278in}{2.371639in}}{\pgfqpoint{2.742550in}{2.363739in}}{\pgfqpoint{2.748374in}{2.357915in}}%
\pgfpathcurveto{\pgfqpoint{2.754198in}{2.352091in}}{\pgfqpoint{2.762098in}{2.348819in}}{\pgfqpoint{2.770334in}{2.348819in}}%
\pgfpathclose%
\pgfusepath{stroke,fill}%
\end{pgfscope}%
\begin{pgfscope}%
\pgfpathrectangle{\pgfqpoint{0.100000in}{0.212622in}}{\pgfqpoint{3.696000in}{3.696000in}}%
\pgfusepath{clip}%
\pgfsetbuttcap%
\pgfsetroundjoin%
\definecolor{currentfill}{rgb}{0.121569,0.466667,0.705882}%
\pgfsetfillcolor{currentfill}%
\pgfsetfillopacity{0.469554}%
\pgfsetlinewidth{1.003750pt}%
\definecolor{currentstroke}{rgb}{0.121569,0.466667,0.705882}%
\pgfsetstrokecolor{currentstroke}%
\pgfsetstrokeopacity{0.469554}%
\pgfsetdash{}{0pt}%
\pgfpathmoveto{\pgfqpoint{2.774472in}{2.348294in}}%
\pgfpathcurveto{\pgfqpoint{2.782708in}{2.348294in}}{\pgfqpoint{2.790608in}{2.351567in}}{\pgfqpoint{2.796432in}{2.357391in}}%
\pgfpathcurveto{\pgfqpoint{2.802256in}{2.363214in}}{\pgfqpoint{2.805528in}{2.371114in}}{\pgfqpoint{2.805528in}{2.379351in}}%
\pgfpathcurveto{\pgfqpoint{2.805528in}{2.387587in}}{\pgfqpoint{2.802256in}{2.395487in}}{\pgfqpoint{2.796432in}{2.401311in}}%
\pgfpathcurveto{\pgfqpoint{2.790608in}{2.407135in}}{\pgfqpoint{2.782708in}{2.410407in}}{\pgfqpoint{2.774472in}{2.410407in}}%
\pgfpathcurveto{\pgfqpoint{2.766236in}{2.410407in}}{\pgfqpoint{2.758336in}{2.407135in}}{\pgfqpoint{2.752512in}{2.401311in}}%
\pgfpathcurveto{\pgfqpoint{2.746688in}{2.395487in}}{\pgfqpoint{2.743415in}{2.387587in}}{\pgfqpoint{2.743415in}{2.379351in}}%
\pgfpathcurveto{\pgfqpoint{2.743415in}{2.371114in}}{\pgfqpoint{2.746688in}{2.363214in}}{\pgfqpoint{2.752512in}{2.357391in}}%
\pgfpathcurveto{\pgfqpoint{2.758336in}{2.351567in}}{\pgfqpoint{2.766236in}{2.348294in}}{\pgfqpoint{2.774472in}{2.348294in}}%
\pgfpathclose%
\pgfusepath{stroke,fill}%
\end{pgfscope}%
\begin{pgfscope}%
\pgfpathrectangle{\pgfqpoint{0.100000in}{0.212622in}}{\pgfqpoint{3.696000in}{3.696000in}}%
\pgfusepath{clip}%
\pgfsetbuttcap%
\pgfsetroundjoin%
\definecolor{currentfill}{rgb}{0.121569,0.466667,0.705882}%
\pgfsetfillcolor{currentfill}%
\pgfsetfillopacity{0.470312}%
\pgfsetlinewidth{1.003750pt}%
\definecolor{currentstroke}{rgb}{0.121569,0.466667,0.705882}%
\pgfsetstrokecolor{currentstroke}%
\pgfsetstrokeopacity{0.470312}%
\pgfsetdash{}{0pt}%
\pgfpathmoveto{\pgfqpoint{2.779343in}{2.347974in}}%
\pgfpathcurveto{\pgfqpoint{2.787579in}{2.347974in}}{\pgfqpoint{2.795480in}{2.351246in}}{\pgfqpoint{2.801303in}{2.357070in}}%
\pgfpathcurveto{\pgfqpoint{2.807127in}{2.362894in}}{\pgfqpoint{2.810400in}{2.370794in}}{\pgfqpoint{2.810400in}{2.379030in}}%
\pgfpathcurveto{\pgfqpoint{2.810400in}{2.387267in}}{\pgfqpoint{2.807127in}{2.395167in}}{\pgfqpoint{2.801303in}{2.400991in}}%
\pgfpathcurveto{\pgfqpoint{2.795480in}{2.406815in}}{\pgfqpoint{2.787579in}{2.410087in}}{\pgfqpoint{2.779343in}{2.410087in}}%
\pgfpathcurveto{\pgfqpoint{2.771107in}{2.410087in}}{\pgfqpoint{2.763207in}{2.406815in}}{\pgfqpoint{2.757383in}{2.400991in}}%
\pgfpathcurveto{\pgfqpoint{2.751559in}{2.395167in}}{\pgfqpoint{2.748287in}{2.387267in}}{\pgfqpoint{2.748287in}{2.379030in}}%
\pgfpathcurveto{\pgfqpoint{2.748287in}{2.370794in}}{\pgfqpoint{2.751559in}{2.362894in}}{\pgfqpoint{2.757383in}{2.357070in}}%
\pgfpathcurveto{\pgfqpoint{2.763207in}{2.351246in}}{\pgfqpoint{2.771107in}{2.347974in}}{\pgfqpoint{2.779343in}{2.347974in}}%
\pgfpathclose%
\pgfusepath{stroke,fill}%
\end{pgfscope}%
\begin{pgfscope}%
\pgfpathrectangle{\pgfqpoint{0.100000in}{0.212622in}}{\pgfqpoint{3.696000in}{3.696000in}}%
\pgfusepath{clip}%
\pgfsetbuttcap%
\pgfsetroundjoin%
\definecolor{currentfill}{rgb}{0.121569,0.466667,0.705882}%
\pgfsetfillcolor{currentfill}%
\pgfsetfillopacity{0.470522}%
\pgfsetlinewidth{1.003750pt}%
\definecolor{currentstroke}{rgb}{0.121569,0.466667,0.705882}%
\pgfsetstrokecolor{currentstroke}%
\pgfsetstrokeopacity{0.470522}%
\pgfsetdash{}{0pt}%
\pgfpathmoveto{\pgfqpoint{1.178250in}{1.936141in}}%
\pgfpathcurveto{\pgfqpoint{1.186487in}{1.936141in}}{\pgfqpoint{1.194387in}{1.939413in}}{\pgfqpoint{1.200211in}{1.945237in}}%
\pgfpathcurveto{\pgfqpoint{1.206034in}{1.951061in}}{\pgfqpoint{1.209307in}{1.958961in}}{\pgfqpoint{1.209307in}{1.967197in}}%
\pgfpathcurveto{\pgfqpoint{1.209307in}{1.975434in}}{\pgfqpoint{1.206034in}{1.983334in}}{\pgfqpoint{1.200211in}{1.989158in}}%
\pgfpathcurveto{\pgfqpoint{1.194387in}{1.994981in}}{\pgfqpoint{1.186487in}{1.998254in}}{\pgfqpoint{1.178250in}{1.998254in}}%
\pgfpathcurveto{\pgfqpoint{1.170014in}{1.998254in}}{\pgfqpoint{1.162114in}{1.994981in}}{\pgfqpoint{1.156290in}{1.989158in}}%
\pgfpathcurveto{\pgfqpoint{1.150466in}{1.983334in}}{\pgfqpoint{1.147194in}{1.975434in}}{\pgfqpoint{1.147194in}{1.967197in}}%
\pgfpathcurveto{\pgfqpoint{1.147194in}{1.958961in}}{\pgfqpoint{1.150466in}{1.951061in}}{\pgfqpoint{1.156290in}{1.945237in}}%
\pgfpathcurveto{\pgfqpoint{1.162114in}{1.939413in}}{\pgfqpoint{1.170014in}{1.936141in}}{\pgfqpoint{1.178250in}{1.936141in}}%
\pgfpathclose%
\pgfusepath{stroke,fill}%
\end{pgfscope}%
\begin{pgfscope}%
\pgfpathrectangle{\pgfqpoint{0.100000in}{0.212622in}}{\pgfqpoint{3.696000in}{3.696000in}}%
\pgfusepath{clip}%
\pgfsetbuttcap%
\pgfsetroundjoin%
\definecolor{currentfill}{rgb}{0.121569,0.466667,0.705882}%
\pgfsetfillcolor{currentfill}%
\pgfsetfillopacity{0.471181}%
\pgfsetlinewidth{1.003750pt}%
\definecolor{currentstroke}{rgb}{0.121569,0.466667,0.705882}%
\pgfsetstrokecolor{currentstroke}%
\pgfsetstrokeopacity{0.471181}%
\pgfsetdash{}{0pt}%
\pgfpathmoveto{\pgfqpoint{2.784951in}{2.347368in}}%
\pgfpathcurveto{\pgfqpoint{2.793187in}{2.347368in}}{\pgfqpoint{2.801087in}{2.350640in}}{\pgfqpoint{2.806911in}{2.356464in}}%
\pgfpathcurveto{\pgfqpoint{2.812735in}{2.362288in}}{\pgfqpoint{2.816007in}{2.370188in}}{\pgfqpoint{2.816007in}{2.378424in}}%
\pgfpathcurveto{\pgfqpoint{2.816007in}{2.386661in}}{\pgfqpoint{2.812735in}{2.394561in}}{\pgfqpoint{2.806911in}{2.400385in}}%
\pgfpathcurveto{\pgfqpoint{2.801087in}{2.406208in}}{\pgfqpoint{2.793187in}{2.409481in}}{\pgfqpoint{2.784951in}{2.409481in}}%
\pgfpathcurveto{\pgfqpoint{2.776714in}{2.409481in}}{\pgfqpoint{2.768814in}{2.406208in}}{\pgfqpoint{2.762990in}{2.400385in}}%
\pgfpathcurveto{\pgfqpoint{2.757166in}{2.394561in}}{\pgfqpoint{2.753894in}{2.386661in}}{\pgfqpoint{2.753894in}{2.378424in}}%
\pgfpathcurveto{\pgfqpoint{2.753894in}{2.370188in}}{\pgfqpoint{2.757166in}{2.362288in}}{\pgfqpoint{2.762990in}{2.356464in}}%
\pgfpathcurveto{\pgfqpoint{2.768814in}{2.350640in}}{\pgfqpoint{2.776714in}{2.347368in}}{\pgfqpoint{2.784951in}{2.347368in}}%
\pgfpathclose%
\pgfusepath{stroke,fill}%
\end{pgfscope}%
\begin{pgfscope}%
\pgfpathrectangle{\pgfqpoint{0.100000in}{0.212622in}}{\pgfqpoint{3.696000in}{3.696000in}}%
\pgfusepath{clip}%
\pgfsetbuttcap%
\pgfsetroundjoin%
\definecolor{currentfill}{rgb}{0.121569,0.466667,0.705882}%
\pgfsetfillcolor{currentfill}%
\pgfsetfillopacity{0.472153}%
\pgfsetlinewidth{1.003750pt}%
\definecolor{currentstroke}{rgb}{0.121569,0.466667,0.705882}%
\pgfsetstrokecolor{currentstroke}%
\pgfsetstrokeopacity{0.472153}%
\pgfsetdash{}{0pt}%
\pgfpathmoveto{\pgfqpoint{2.791400in}{2.346399in}}%
\pgfpathcurveto{\pgfqpoint{2.799636in}{2.346399in}}{\pgfqpoint{2.807537in}{2.349671in}}{\pgfqpoint{2.813360in}{2.355495in}}%
\pgfpathcurveto{\pgfqpoint{2.819184in}{2.361319in}}{\pgfqpoint{2.822457in}{2.369219in}}{\pgfqpoint{2.822457in}{2.377455in}}%
\pgfpathcurveto{\pgfqpoint{2.822457in}{2.385691in}}{\pgfqpoint{2.819184in}{2.393591in}}{\pgfqpoint{2.813360in}{2.399415in}}%
\pgfpathcurveto{\pgfqpoint{2.807537in}{2.405239in}}{\pgfqpoint{2.799636in}{2.408512in}}{\pgfqpoint{2.791400in}{2.408512in}}%
\pgfpathcurveto{\pgfqpoint{2.783164in}{2.408512in}}{\pgfqpoint{2.775264in}{2.405239in}}{\pgfqpoint{2.769440in}{2.399415in}}%
\pgfpathcurveto{\pgfqpoint{2.763616in}{2.393591in}}{\pgfqpoint{2.760344in}{2.385691in}}{\pgfqpoint{2.760344in}{2.377455in}}%
\pgfpathcurveto{\pgfqpoint{2.760344in}{2.369219in}}{\pgfqpoint{2.763616in}{2.361319in}}{\pgfqpoint{2.769440in}{2.355495in}}%
\pgfpathcurveto{\pgfqpoint{2.775264in}{2.349671in}}{\pgfqpoint{2.783164in}{2.346399in}}{\pgfqpoint{2.791400in}{2.346399in}}%
\pgfpathclose%
\pgfusepath{stroke,fill}%
\end{pgfscope}%
\begin{pgfscope}%
\pgfpathrectangle{\pgfqpoint{0.100000in}{0.212622in}}{\pgfqpoint{3.696000in}{3.696000in}}%
\pgfusepath{clip}%
\pgfsetbuttcap%
\pgfsetroundjoin%
\definecolor{currentfill}{rgb}{0.121569,0.466667,0.705882}%
\pgfsetfillcolor{currentfill}%
\pgfsetfillopacity{0.473664}%
\pgfsetlinewidth{1.003750pt}%
\definecolor{currentstroke}{rgb}{0.121569,0.466667,0.705882}%
\pgfsetstrokecolor{currentstroke}%
\pgfsetstrokeopacity{0.473664}%
\pgfsetdash{}{0pt}%
\pgfpathmoveto{\pgfqpoint{2.800398in}{2.346795in}}%
\pgfpathcurveto{\pgfqpoint{2.808634in}{2.346795in}}{\pgfqpoint{2.816535in}{2.350067in}}{\pgfqpoint{2.822358in}{2.355891in}}%
\pgfpathcurveto{\pgfqpoint{2.828182in}{2.361715in}}{\pgfqpoint{2.831455in}{2.369615in}}{\pgfqpoint{2.831455in}{2.377851in}}%
\pgfpathcurveto{\pgfqpoint{2.831455in}{2.386087in}}{\pgfqpoint{2.828182in}{2.393987in}}{\pgfqpoint{2.822358in}{2.399811in}}%
\pgfpathcurveto{\pgfqpoint{2.816535in}{2.405635in}}{\pgfqpoint{2.808634in}{2.408908in}}{\pgfqpoint{2.800398in}{2.408908in}}%
\pgfpathcurveto{\pgfqpoint{2.792162in}{2.408908in}}{\pgfqpoint{2.784262in}{2.405635in}}{\pgfqpoint{2.778438in}{2.399811in}}%
\pgfpathcurveto{\pgfqpoint{2.772614in}{2.393987in}}{\pgfqpoint{2.769342in}{2.386087in}}{\pgfqpoint{2.769342in}{2.377851in}}%
\pgfpathcurveto{\pgfqpoint{2.769342in}{2.369615in}}{\pgfqpoint{2.772614in}{2.361715in}}{\pgfqpoint{2.778438in}{2.355891in}}%
\pgfpathcurveto{\pgfqpoint{2.784262in}{2.350067in}}{\pgfqpoint{2.792162in}{2.346795in}}{\pgfqpoint{2.800398in}{2.346795in}}%
\pgfpathclose%
\pgfusepath{stroke,fill}%
\end{pgfscope}%
\begin{pgfscope}%
\pgfpathrectangle{\pgfqpoint{0.100000in}{0.212622in}}{\pgfqpoint{3.696000in}{3.696000in}}%
\pgfusepath{clip}%
\pgfsetbuttcap%
\pgfsetroundjoin%
\definecolor{currentfill}{rgb}{0.121569,0.466667,0.705882}%
\pgfsetfillcolor{currentfill}%
\pgfsetfillopacity{0.475365}%
\pgfsetlinewidth{1.003750pt}%
\definecolor{currentstroke}{rgb}{0.121569,0.466667,0.705882}%
\pgfsetstrokecolor{currentstroke}%
\pgfsetstrokeopacity{0.475365}%
\pgfsetdash{}{0pt}%
\pgfpathmoveto{\pgfqpoint{2.810829in}{2.346431in}}%
\pgfpathcurveto{\pgfqpoint{2.819065in}{2.346431in}}{\pgfqpoint{2.826965in}{2.349703in}}{\pgfqpoint{2.832789in}{2.355527in}}%
\pgfpathcurveto{\pgfqpoint{2.838613in}{2.361351in}}{\pgfqpoint{2.841885in}{2.369251in}}{\pgfqpoint{2.841885in}{2.377487in}}%
\pgfpathcurveto{\pgfqpoint{2.841885in}{2.385724in}}{\pgfqpoint{2.838613in}{2.393624in}}{\pgfqpoint{2.832789in}{2.399447in}}%
\pgfpathcurveto{\pgfqpoint{2.826965in}{2.405271in}}{\pgfqpoint{2.819065in}{2.408544in}}{\pgfqpoint{2.810829in}{2.408544in}}%
\pgfpathcurveto{\pgfqpoint{2.802593in}{2.408544in}}{\pgfqpoint{2.794693in}{2.405271in}}{\pgfqpoint{2.788869in}{2.399447in}}%
\pgfpathcurveto{\pgfqpoint{2.783045in}{2.393624in}}{\pgfqpoint{2.779772in}{2.385724in}}{\pgfqpoint{2.779772in}{2.377487in}}%
\pgfpathcurveto{\pgfqpoint{2.779772in}{2.369251in}}{\pgfqpoint{2.783045in}{2.361351in}}{\pgfqpoint{2.788869in}{2.355527in}}%
\pgfpathcurveto{\pgfqpoint{2.794693in}{2.349703in}}{\pgfqpoint{2.802593in}{2.346431in}}{\pgfqpoint{2.810829in}{2.346431in}}%
\pgfpathclose%
\pgfusepath{stroke,fill}%
\end{pgfscope}%
\begin{pgfscope}%
\pgfpathrectangle{\pgfqpoint{0.100000in}{0.212622in}}{\pgfqpoint{3.696000in}{3.696000in}}%
\pgfusepath{clip}%
\pgfsetbuttcap%
\pgfsetroundjoin%
\definecolor{currentfill}{rgb}{0.121569,0.466667,0.705882}%
\pgfsetfillcolor{currentfill}%
\pgfsetfillopacity{0.475892}%
\pgfsetlinewidth{1.003750pt}%
\definecolor{currentstroke}{rgb}{0.121569,0.466667,0.705882}%
\pgfsetstrokecolor{currentstroke}%
\pgfsetstrokeopacity{0.475892}%
\pgfsetdash{}{0pt}%
\pgfpathmoveto{\pgfqpoint{1.161900in}{1.914781in}}%
\pgfpathcurveto{\pgfqpoint{1.170136in}{1.914781in}}{\pgfqpoint{1.178036in}{1.918054in}}{\pgfqpoint{1.183860in}{1.923877in}}%
\pgfpathcurveto{\pgfqpoint{1.189684in}{1.929701in}}{\pgfqpoint{1.192956in}{1.937601in}}{\pgfqpoint{1.192956in}{1.945838in}}%
\pgfpathcurveto{\pgfqpoint{1.192956in}{1.954074in}}{\pgfqpoint{1.189684in}{1.961974in}}{\pgfqpoint{1.183860in}{1.967798in}}%
\pgfpathcurveto{\pgfqpoint{1.178036in}{1.973622in}}{\pgfqpoint{1.170136in}{1.976894in}}{\pgfqpoint{1.161900in}{1.976894in}}%
\pgfpathcurveto{\pgfqpoint{1.153663in}{1.976894in}}{\pgfqpoint{1.145763in}{1.973622in}}{\pgfqpoint{1.139939in}{1.967798in}}%
\pgfpathcurveto{\pgfqpoint{1.134115in}{1.961974in}}{\pgfqpoint{1.130843in}{1.954074in}}{\pgfqpoint{1.130843in}{1.945838in}}%
\pgfpathcurveto{\pgfqpoint{1.130843in}{1.937601in}}{\pgfqpoint{1.134115in}{1.929701in}}{\pgfqpoint{1.139939in}{1.923877in}}%
\pgfpathcurveto{\pgfqpoint{1.145763in}{1.918054in}}{\pgfqpoint{1.153663in}{1.914781in}}{\pgfqpoint{1.161900in}{1.914781in}}%
\pgfpathclose%
\pgfusepath{stroke,fill}%
\end{pgfscope}%
\begin{pgfscope}%
\pgfpathrectangle{\pgfqpoint{0.100000in}{0.212622in}}{\pgfqpoint{3.696000in}{3.696000in}}%
\pgfusepath{clip}%
\pgfsetbuttcap%
\pgfsetroundjoin%
\definecolor{currentfill}{rgb}{0.121569,0.466667,0.705882}%
\pgfsetfillcolor{currentfill}%
\pgfsetfillopacity{0.477152}%
\pgfsetlinewidth{1.003750pt}%
\definecolor{currentstroke}{rgb}{0.121569,0.466667,0.705882}%
\pgfsetstrokecolor{currentstroke}%
\pgfsetstrokeopacity{0.477152}%
\pgfsetdash{}{0pt}%
\pgfpathmoveto{\pgfqpoint{2.822154in}{2.345158in}}%
\pgfpathcurveto{\pgfqpoint{2.830390in}{2.345158in}}{\pgfqpoint{2.838290in}{2.348430in}}{\pgfqpoint{2.844114in}{2.354254in}}%
\pgfpathcurveto{\pgfqpoint{2.849938in}{2.360078in}}{\pgfqpoint{2.853210in}{2.367978in}}{\pgfqpoint{2.853210in}{2.376214in}}%
\pgfpathcurveto{\pgfqpoint{2.853210in}{2.384451in}}{\pgfqpoint{2.849938in}{2.392351in}}{\pgfqpoint{2.844114in}{2.398175in}}%
\pgfpathcurveto{\pgfqpoint{2.838290in}{2.403999in}}{\pgfqpoint{2.830390in}{2.407271in}}{\pgfqpoint{2.822154in}{2.407271in}}%
\pgfpathcurveto{\pgfqpoint{2.813918in}{2.407271in}}{\pgfqpoint{2.806017in}{2.403999in}}{\pgfqpoint{2.800194in}{2.398175in}}%
\pgfpathcurveto{\pgfqpoint{2.794370in}{2.392351in}}{\pgfqpoint{2.791097in}{2.384451in}}{\pgfqpoint{2.791097in}{2.376214in}}%
\pgfpathcurveto{\pgfqpoint{2.791097in}{2.367978in}}{\pgfqpoint{2.794370in}{2.360078in}}{\pgfqpoint{2.800194in}{2.354254in}}%
\pgfpathcurveto{\pgfqpoint{2.806017in}{2.348430in}}{\pgfqpoint{2.813918in}{2.345158in}}{\pgfqpoint{2.822154in}{2.345158in}}%
\pgfpathclose%
\pgfusepath{stroke,fill}%
\end{pgfscope}%
\begin{pgfscope}%
\pgfpathrectangle{\pgfqpoint{0.100000in}{0.212622in}}{\pgfqpoint{3.696000in}{3.696000in}}%
\pgfusepath{clip}%
\pgfsetbuttcap%
\pgfsetroundjoin%
\definecolor{currentfill}{rgb}{0.121569,0.466667,0.705882}%
\pgfsetfillcolor{currentfill}%
\pgfsetfillopacity{0.477999}%
\pgfsetlinewidth{1.003750pt}%
\definecolor{currentstroke}{rgb}{0.121569,0.466667,0.705882}%
\pgfsetstrokecolor{currentstroke}%
\pgfsetstrokeopacity{0.477999}%
\pgfsetdash{}{0pt}%
\pgfpathmoveto{\pgfqpoint{2.828434in}{2.343767in}}%
\pgfpathcurveto{\pgfqpoint{2.836670in}{2.343767in}}{\pgfqpoint{2.844570in}{2.347039in}}{\pgfqpoint{2.850394in}{2.352863in}}%
\pgfpathcurveto{\pgfqpoint{2.856218in}{2.358687in}}{\pgfqpoint{2.859490in}{2.366587in}}{\pgfqpoint{2.859490in}{2.374823in}}%
\pgfpathcurveto{\pgfqpoint{2.859490in}{2.383060in}}{\pgfqpoint{2.856218in}{2.390960in}}{\pgfqpoint{2.850394in}{2.396784in}}%
\pgfpathcurveto{\pgfqpoint{2.844570in}{2.402608in}}{\pgfqpoint{2.836670in}{2.405880in}}{\pgfqpoint{2.828434in}{2.405880in}}%
\pgfpathcurveto{\pgfqpoint{2.820197in}{2.405880in}}{\pgfqpoint{2.812297in}{2.402608in}}{\pgfqpoint{2.806473in}{2.396784in}}%
\pgfpathcurveto{\pgfqpoint{2.800649in}{2.390960in}}{\pgfqpoint{2.797377in}{2.383060in}}{\pgfqpoint{2.797377in}{2.374823in}}%
\pgfpathcurveto{\pgfqpoint{2.797377in}{2.366587in}}{\pgfqpoint{2.800649in}{2.358687in}}{\pgfqpoint{2.806473in}{2.352863in}}%
\pgfpathcurveto{\pgfqpoint{2.812297in}{2.347039in}}{\pgfqpoint{2.820197in}{2.343767in}}{\pgfqpoint{2.828434in}{2.343767in}}%
\pgfpathclose%
\pgfusepath{stroke,fill}%
\end{pgfscope}%
\begin{pgfscope}%
\pgfpathrectangle{\pgfqpoint{0.100000in}{0.212622in}}{\pgfqpoint{3.696000in}{3.696000in}}%
\pgfusepath{clip}%
\pgfsetbuttcap%
\pgfsetroundjoin%
\definecolor{currentfill}{rgb}{0.121569,0.466667,0.705882}%
\pgfsetfillcolor{currentfill}%
\pgfsetfillopacity{0.478953}%
\pgfsetlinewidth{1.003750pt}%
\definecolor{currentstroke}{rgb}{0.121569,0.466667,0.705882}%
\pgfsetstrokecolor{currentstroke}%
\pgfsetstrokeopacity{0.478953}%
\pgfsetdash{}{0pt}%
\pgfpathmoveto{\pgfqpoint{2.835645in}{2.341996in}}%
\pgfpathcurveto{\pgfqpoint{2.843882in}{2.341996in}}{\pgfqpoint{2.851782in}{2.345268in}}{\pgfqpoint{2.857606in}{2.351092in}}%
\pgfpathcurveto{\pgfqpoint{2.863430in}{2.356916in}}{\pgfqpoint{2.866702in}{2.364816in}}{\pgfqpoint{2.866702in}{2.373052in}}%
\pgfpathcurveto{\pgfqpoint{2.866702in}{2.381288in}}{\pgfqpoint{2.863430in}{2.389188in}}{\pgfqpoint{2.857606in}{2.395012in}}%
\pgfpathcurveto{\pgfqpoint{2.851782in}{2.400836in}}{\pgfqpoint{2.843882in}{2.404109in}}{\pgfqpoint{2.835645in}{2.404109in}}%
\pgfpathcurveto{\pgfqpoint{2.827409in}{2.404109in}}{\pgfqpoint{2.819509in}{2.400836in}}{\pgfqpoint{2.813685in}{2.395012in}}%
\pgfpathcurveto{\pgfqpoint{2.807861in}{2.389188in}}{\pgfqpoint{2.804589in}{2.381288in}}{\pgfqpoint{2.804589in}{2.373052in}}%
\pgfpathcurveto{\pgfqpoint{2.804589in}{2.364816in}}{\pgfqpoint{2.807861in}{2.356916in}}{\pgfqpoint{2.813685in}{2.351092in}}%
\pgfpathcurveto{\pgfqpoint{2.819509in}{2.345268in}}{\pgfqpoint{2.827409in}{2.341996in}}{\pgfqpoint{2.835645in}{2.341996in}}%
\pgfpathclose%
\pgfusepath{stroke,fill}%
\end{pgfscope}%
\begin{pgfscope}%
\pgfpathrectangle{\pgfqpoint{0.100000in}{0.212622in}}{\pgfqpoint{3.696000in}{3.696000in}}%
\pgfusepath{clip}%
\pgfsetbuttcap%
\pgfsetroundjoin%
\definecolor{currentfill}{rgb}{0.121569,0.466667,0.705882}%
\pgfsetfillcolor{currentfill}%
\pgfsetfillopacity{0.480158}%
\pgfsetlinewidth{1.003750pt}%
\definecolor{currentstroke}{rgb}{0.121569,0.466667,0.705882}%
\pgfsetstrokecolor{currentstroke}%
\pgfsetstrokeopacity{0.480158}%
\pgfsetdash{}{0pt}%
\pgfpathmoveto{\pgfqpoint{2.844339in}{2.340325in}}%
\pgfpathcurveto{\pgfqpoint{2.852576in}{2.340325in}}{\pgfqpoint{2.860476in}{2.343597in}}{\pgfqpoint{2.866300in}{2.349421in}}%
\pgfpathcurveto{\pgfqpoint{2.872124in}{2.355245in}}{\pgfqpoint{2.875396in}{2.363145in}}{\pgfqpoint{2.875396in}{2.371382in}}%
\pgfpathcurveto{\pgfqpoint{2.875396in}{2.379618in}}{\pgfqpoint{2.872124in}{2.387518in}}{\pgfqpoint{2.866300in}{2.393342in}}%
\pgfpathcurveto{\pgfqpoint{2.860476in}{2.399166in}}{\pgfqpoint{2.852576in}{2.402438in}}{\pgfqpoint{2.844339in}{2.402438in}}%
\pgfpathcurveto{\pgfqpoint{2.836103in}{2.402438in}}{\pgfqpoint{2.828203in}{2.399166in}}{\pgfqpoint{2.822379in}{2.393342in}}%
\pgfpathcurveto{\pgfqpoint{2.816555in}{2.387518in}}{\pgfqpoint{2.813283in}{2.379618in}}{\pgfqpoint{2.813283in}{2.371382in}}%
\pgfpathcurveto{\pgfqpoint{2.813283in}{2.363145in}}{\pgfqpoint{2.816555in}{2.355245in}}{\pgfqpoint{2.822379in}{2.349421in}}%
\pgfpathcurveto{\pgfqpoint{2.828203in}{2.343597in}}{\pgfqpoint{2.836103in}{2.340325in}}{\pgfqpoint{2.844339in}{2.340325in}}%
\pgfpathclose%
\pgfusepath{stroke,fill}%
\end{pgfscope}%
\begin{pgfscope}%
\pgfpathrectangle{\pgfqpoint{0.100000in}{0.212622in}}{\pgfqpoint{3.696000in}{3.696000in}}%
\pgfusepath{clip}%
\pgfsetbuttcap%
\pgfsetroundjoin%
\definecolor{currentfill}{rgb}{0.121569,0.466667,0.705882}%
\pgfsetfillcolor{currentfill}%
\pgfsetfillopacity{0.480941}%
\pgfsetlinewidth{1.003750pt}%
\definecolor{currentstroke}{rgb}{0.121569,0.466667,0.705882}%
\pgfsetstrokecolor{currentstroke}%
\pgfsetstrokeopacity{0.480941}%
\pgfsetdash{}{0pt}%
\pgfpathmoveto{\pgfqpoint{1.147391in}{1.894729in}}%
\pgfpathcurveto{\pgfqpoint{1.155627in}{1.894729in}}{\pgfqpoint{1.163527in}{1.898002in}}{\pgfqpoint{1.169351in}{1.903826in}}%
\pgfpathcurveto{\pgfqpoint{1.175175in}{1.909650in}}{\pgfqpoint{1.178447in}{1.917550in}}{\pgfqpoint{1.178447in}{1.925786in}}%
\pgfpathcurveto{\pgfqpoint{1.178447in}{1.934022in}}{\pgfqpoint{1.175175in}{1.941922in}}{\pgfqpoint{1.169351in}{1.947746in}}%
\pgfpathcurveto{\pgfqpoint{1.163527in}{1.953570in}}{\pgfqpoint{1.155627in}{1.956842in}}{\pgfqpoint{1.147391in}{1.956842in}}%
\pgfpathcurveto{\pgfqpoint{1.139155in}{1.956842in}}{\pgfqpoint{1.131255in}{1.953570in}}{\pgfqpoint{1.125431in}{1.947746in}}%
\pgfpathcurveto{\pgfqpoint{1.119607in}{1.941922in}}{\pgfqpoint{1.116334in}{1.934022in}}{\pgfqpoint{1.116334in}{1.925786in}}%
\pgfpathcurveto{\pgfqpoint{1.116334in}{1.917550in}}{\pgfqpoint{1.119607in}{1.909650in}}{\pgfqpoint{1.125431in}{1.903826in}}%
\pgfpathcurveto{\pgfqpoint{1.131255in}{1.898002in}}{\pgfqpoint{1.139155in}{1.894729in}}{\pgfqpoint{1.147391in}{1.894729in}}%
\pgfpathclose%
\pgfusepath{stroke,fill}%
\end{pgfscope}%
\begin{pgfscope}%
\pgfpathrectangle{\pgfqpoint{0.100000in}{0.212622in}}{\pgfqpoint{3.696000in}{3.696000in}}%
\pgfusepath{clip}%
\pgfsetbuttcap%
\pgfsetroundjoin%
\definecolor{currentfill}{rgb}{0.121569,0.466667,0.705882}%
\pgfsetfillcolor{currentfill}%
\pgfsetfillopacity{0.481515}%
\pgfsetlinewidth{1.003750pt}%
\definecolor{currentstroke}{rgb}{0.121569,0.466667,0.705882}%
\pgfsetstrokecolor{currentstroke}%
\pgfsetstrokeopacity{0.481515}%
\pgfsetdash{}{0pt}%
\pgfpathmoveto{\pgfqpoint{2.853844in}{2.338602in}}%
\pgfpathcurveto{\pgfqpoint{2.862080in}{2.338602in}}{\pgfqpoint{2.869980in}{2.341875in}}{\pgfqpoint{2.875804in}{2.347698in}}%
\pgfpathcurveto{\pgfqpoint{2.881628in}{2.353522in}}{\pgfqpoint{2.884900in}{2.361422in}}{\pgfqpoint{2.884900in}{2.369659in}}%
\pgfpathcurveto{\pgfqpoint{2.884900in}{2.377895in}}{\pgfqpoint{2.881628in}{2.385795in}}{\pgfqpoint{2.875804in}{2.391619in}}%
\pgfpathcurveto{\pgfqpoint{2.869980in}{2.397443in}}{\pgfqpoint{2.862080in}{2.400715in}}{\pgfqpoint{2.853844in}{2.400715in}}%
\pgfpathcurveto{\pgfqpoint{2.845607in}{2.400715in}}{\pgfqpoint{2.837707in}{2.397443in}}{\pgfqpoint{2.831883in}{2.391619in}}%
\pgfpathcurveto{\pgfqpoint{2.826060in}{2.385795in}}{\pgfqpoint{2.822787in}{2.377895in}}{\pgfqpoint{2.822787in}{2.369659in}}%
\pgfpathcurveto{\pgfqpoint{2.822787in}{2.361422in}}{\pgfqpoint{2.826060in}{2.353522in}}{\pgfqpoint{2.831883in}{2.347698in}}%
\pgfpathcurveto{\pgfqpoint{2.837707in}{2.341875in}}{\pgfqpoint{2.845607in}{2.338602in}}{\pgfqpoint{2.853844in}{2.338602in}}%
\pgfpathclose%
\pgfusepath{stroke,fill}%
\end{pgfscope}%
\begin{pgfscope}%
\pgfpathrectangle{\pgfqpoint{0.100000in}{0.212622in}}{\pgfqpoint{3.696000in}{3.696000in}}%
\pgfusepath{clip}%
\pgfsetbuttcap%
\pgfsetroundjoin%
\definecolor{currentfill}{rgb}{0.121569,0.466667,0.705882}%
\pgfsetfillcolor{currentfill}%
\pgfsetfillopacity{0.483078}%
\pgfsetlinewidth{1.003750pt}%
\definecolor{currentstroke}{rgb}{0.121569,0.466667,0.705882}%
\pgfsetstrokecolor{currentstroke}%
\pgfsetstrokeopacity{0.483078}%
\pgfsetdash{}{0pt}%
\pgfpathmoveto{\pgfqpoint{2.864777in}{2.335132in}}%
\pgfpathcurveto{\pgfqpoint{2.873013in}{2.335132in}}{\pgfqpoint{2.880913in}{2.338404in}}{\pgfqpoint{2.886737in}{2.344228in}}%
\pgfpathcurveto{\pgfqpoint{2.892561in}{2.350052in}}{\pgfqpoint{2.895833in}{2.357952in}}{\pgfqpoint{2.895833in}{2.366188in}}%
\pgfpathcurveto{\pgfqpoint{2.895833in}{2.374424in}}{\pgfqpoint{2.892561in}{2.382324in}}{\pgfqpoint{2.886737in}{2.388148in}}%
\pgfpathcurveto{\pgfqpoint{2.880913in}{2.393972in}}{\pgfqpoint{2.873013in}{2.397245in}}{\pgfqpoint{2.864777in}{2.397245in}}%
\pgfpathcurveto{\pgfqpoint{2.856541in}{2.397245in}}{\pgfqpoint{2.848640in}{2.393972in}}{\pgfqpoint{2.842817in}{2.388148in}}%
\pgfpathcurveto{\pgfqpoint{2.836993in}{2.382324in}}{\pgfqpoint{2.833720in}{2.374424in}}{\pgfqpoint{2.833720in}{2.366188in}}%
\pgfpathcurveto{\pgfqpoint{2.833720in}{2.357952in}}{\pgfqpoint{2.836993in}{2.350052in}}{\pgfqpoint{2.842817in}{2.344228in}}%
\pgfpathcurveto{\pgfqpoint{2.848640in}{2.338404in}}{\pgfqpoint{2.856541in}{2.335132in}}{\pgfqpoint{2.864777in}{2.335132in}}%
\pgfpathclose%
\pgfusepath{stroke,fill}%
\end{pgfscope}%
\begin{pgfscope}%
\pgfpathrectangle{\pgfqpoint{0.100000in}{0.212622in}}{\pgfqpoint{3.696000in}{3.696000in}}%
\pgfusepath{clip}%
\pgfsetbuttcap%
\pgfsetroundjoin%
\definecolor{currentfill}{rgb}{0.121569,0.466667,0.705882}%
\pgfsetfillcolor{currentfill}%
\pgfsetfillopacity{0.484754}%
\pgfsetlinewidth{1.003750pt}%
\definecolor{currentstroke}{rgb}{0.121569,0.466667,0.705882}%
\pgfsetstrokecolor{currentstroke}%
\pgfsetstrokeopacity{0.484754}%
\pgfsetdash{}{0pt}%
\pgfpathmoveto{\pgfqpoint{2.876374in}{2.331982in}}%
\pgfpathcurveto{\pgfqpoint{2.884610in}{2.331982in}}{\pgfqpoint{2.892510in}{2.335254in}}{\pgfqpoint{2.898334in}{2.341078in}}%
\pgfpathcurveto{\pgfqpoint{2.904158in}{2.346902in}}{\pgfqpoint{2.907430in}{2.354802in}}{\pgfqpoint{2.907430in}{2.363039in}}%
\pgfpathcurveto{\pgfqpoint{2.907430in}{2.371275in}}{\pgfqpoint{2.904158in}{2.379175in}}{\pgfqpoint{2.898334in}{2.384999in}}%
\pgfpathcurveto{\pgfqpoint{2.892510in}{2.390823in}}{\pgfqpoint{2.884610in}{2.394095in}}{\pgfqpoint{2.876374in}{2.394095in}}%
\pgfpathcurveto{\pgfqpoint{2.868137in}{2.394095in}}{\pgfqpoint{2.860237in}{2.390823in}}{\pgfqpoint{2.854413in}{2.384999in}}%
\pgfpathcurveto{\pgfqpoint{2.848589in}{2.379175in}}{\pgfqpoint{2.845317in}{2.371275in}}{\pgfqpoint{2.845317in}{2.363039in}}%
\pgfpathcurveto{\pgfqpoint{2.845317in}{2.354802in}}{\pgfqpoint{2.848589in}{2.346902in}}{\pgfqpoint{2.854413in}{2.341078in}}%
\pgfpathcurveto{\pgfqpoint{2.860237in}{2.335254in}}{\pgfqpoint{2.868137in}{2.331982in}}{\pgfqpoint{2.876374in}{2.331982in}}%
\pgfpathclose%
\pgfusepath{stroke,fill}%
\end{pgfscope}%
\begin{pgfscope}%
\pgfpathrectangle{\pgfqpoint{0.100000in}{0.212622in}}{\pgfqpoint{3.696000in}{3.696000in}}%
\pgfusepath{clip}%
\pgfsetbuttcap%
\pgfsetroundjoin%
\definecolor{currentfill}{rgb}{0.121569,0.466667,0.705882}%
\pgfsetfillcolor{currentfill}%
\pgfsetfillopacity{0.485367}%
\pgfsetlinewidth{1.003750pt}%
\definecolor{currentstroke}{rgb}{0.121569,0.466667,0.705882}%
\pgfsetstrokecolor{currentstroke}%
\pgfsetstrokeopacity{0.485367}%
\pgfsetdash{}{0pt}%
\pgfpathmoveto{\pgfqpoint{1.133602in}{1.876112in}}%
\pgfpathcurveto{\pgfqpoint{1.141839in}{1.876112in}}{\pgfqpoint{1.149739in}{1.879385in}}{\pgfqpoint{1.155563in}{1.885209in}}%
\pgfpathcurveto{\pgfqpoint{1.161387in}{1.891033in}}{\pgfqpoint{1.164659in}{1.898933in}}{\pgfqpoint{1.164659in}{1.907169in}}%
\pgfpathcurveto{\pgfqpoint{1.164659in}{1.915405in}}{\pgfqpoint{1.161387in}{1.923305in}}{\pgfqpoint{1.155563in}{1.929129in}}%
\pgfpathcurveto{\pgfqpoint{1.149739in}{1.934953in}}{\pgfqpoint{1.141839in}{1.938225in}}{\pgfqpoint{1.133602in}{1.938225in}}%
\pgfpathcurveto{\pgfqpoint{1.125366in}{1.938225in}}{\pgfqpoint{1.117466in}{1.934953in}}{\pgfqpoint{1.111642in}{1.929129in}}%
\pgfpathcurveto{\pgfqpoint{1.105818in}{1.923305in}}{\pgfqpoint{1.102546in}{1.915405in}}{\pgfqpoint{1.102546in}{1.907169in}}%
\pgfpathcurveto{\pgfqpoint{1.102546in}{1.898933in}}{\pgfqpoint{1.105818in}{1.891033in}}{\pgfqpoint{1.111642in}{1.885209in}}%
\pgfpathcurveto{\pgfqpoint{1.117466in}{1.879385in}}{\pgfqpoint{1.125366in}{1.876112in}}{\pgfqpoint{1.133602in}{1.876112in}}%
\pgfpathclose%
\pgfusepath{stroke,fill}%
\end{pgfscope}%
\begin{pgfscope}%
\pgfpathrectangle{\pgfqpoint{0.100000in}{0.212622in}}{\pgfqpoint{3.696000in}{3.696000in}}%
\pgfusepath{clip}%
\pgfsetbuttcap%
\pgfsetroundjoin%
\definecolor{currentfill}{rgb}{0.121569,0.466667,0.705882}%
\pgfsetfillcolor{currentfill}%
\pgfsetfillopacity{0.485708}%
\pgfsetlinewidth{1.003750pt}%
\definecolor{currentstroke}{rgb}{0.121569,0.466667,0.705882}%
\pgfsetstrokecolor{currentstroke}%
\pgfsetstrokeopacity{0.485708}%
\pgfsetdash{}{0pt}%
\pgfpathmoveto{\pgfqpoint{2.882842in}{2.330725in}}%
\pgfpathcurveto{\pgfqpoint{2.891079in}{2.330725in}}{\pgfqpoint{2.898979in}{2.333997in}}{\pgfqpoint{2.904802in}{2.339821in}}%
\pgfpathcurveto{\pgfqpoint{2.910626in}{2.345645in}}{\pgfqpoint{2.913899in}{2.353545in}}{\pgfqpoint{2.913899in}{2.361781in}}%
\pgfpathcurveto{\pgfqpoint{2.913899in}{2.370018in}}{\pgfqpoint{2.910626in}{2.377918in}}{\pgfqpoint{2.904802in}{2.383741in}}%
\pgfpathcurveto{\pgfqpoint{2.898979in}{2.389565in}}{\pgfqpoint{2.891079in}{2.392838in}}{\pgfqpoint{2.882842in}{2.392838in}}%
\pgfpathcurveto{\pgfqpoint{2.874606in}{2.392838in}}{\pgfqpoint{2.866706in}{2.389565in}}{\pgfqpoint{2.860882in}{2.383741in}}%
\pgfpathcurveto{\pgfqpoint{2.855058in}{2.377918in}}{\pgfqpoint{2.851786in}{2.370018in}}{\pgfqpoint{2.851786in}{2.361781in}}%
\pgfpathcurveto{\pgfqpoint{2.851786in}{2.353545in}}{\pgfqpoint{2.855058in}{2.345645in}}{\pgfqpoint{2.860882in}{2.339821in}}%
\pgfpathcurveto{\pgfqpoint{2.866706in}{2.333997in}}{\pgfqpoint{2.874606in}{2.330725in}}{\pgfqpoint{2.882842in}{2.330725in}}%
\pgfpathclose%
\pgfusepath{stroke,fill}%
\end{pgfscope}%
\begin{pgfscope}%
\pgfpathrectangle{\pgfqpoint{0.100000in}{0.212622in}}{\pgfqpoint{3.696000in}{3.696000in}}%
\pgfusepath{clip}%
\pgfsetbuttcap%
\pgfsetroundjoin%
\definecolor{currentfill}{rgb}{0.121569,0.466667,0.705882}%
\pgfsetfillcolor{currentfill}%
\pgfsetfillopacity{0.486785}%
\pgfsetlinewidth{1.003750pt}%
\definecolor{currentstroke}{rgb}{0.121569,0.466667,0.705882}%
\pgfsetstrokecolor{currentstroke}%
\pgfsetstrokeopacity{0.486785}%
\pgfsetdash{}{0pt}%
\pgfpathmoveto{\pgfqpoint{2.891173in}{2.328775in}}%
\pgfpathcurveto{\pgfqpoint{2.899410in}{2.328775in}}{\pgfqpoint{2.907310in}{2.332047in}}{\pgfqpoint{2.913134in}{2.337871in}}%
\pgfpathcurveto{\pgfqpoint{2.918957in}{2.343695in}}{\pgfqpoint{2.922230in}{2.351595in}}{\pgfqpoint{2.922230in}{2.359831in}}%
\pgfpathcurveto{\pgfqpoint{2.922230in}{2.368067in}}{\pgfqpoint{2.918957in}{2.375967in}}{\pgfqpoint{2.913134in}{2.381791in}}%
\pgfpathcurveto{\pgfqpoint{2.907310in}{2.387615in}}{\pgfqpoint{2.899410in}{2.390888in}}{\pgfqpoint{2.891173in}{2.390888in}}%
\pgfpathcurveto{\pgfqpoint{2.882937in}{2.390888in}}{\pgfqpoint{2.875037in}{2.387615in}}{\pgfqpoint{2.869213in}{2.381791in}}%
\pgfpathcurveto{\pgfqpoint{2.863389in}{2.375967in}}{\pgfqpoint{2.860117in}{2.368067in}}{\pgfqpoint{2.860117in}{2.359831in}}%
\pgfpathcurveto{\pgfqpoint{2.860117in}{2.351595in}}{\pgfqpoint{2.863389in}{2.343695in}}{\pgfqpoint{2.869213in}{2.337871in}}%
\pgfpathcurveto{\pgfqpoint{2.875037in}{2.332047in}}{\pgfqpoint{2.882937in}{2.328775in}}{\pgfqpoint{2.891173in}{2.328775in}}%
\pgfpathclose%
\pgfusepath{stroke,fill}%
\end{pgfscope}%
\begin{pgfscope}%
\pgfpathrectangle{\pgfqpoint{0.100000in}{0.212622in}}{\pgfqpoint{3.696000in}{3.696000in}}%
\pgfusepath{clip}%
\pgfsetbuttcap%
\pgfsetroundjoin%
\definecolor{currentfill}{rgb}{0.121569,0.466667,0.705882}%
\pgfsetfillcolor{currentfill}%
\pgfsetfillopacity{0.487397}%
\pgfsetlinewidth{1.003750pt}%
\definecolor{currentstroke}{rgb}{0.121569,0.466667,0.705882}%
\pgfsetstrokecolor{currentstroke}%
\pgfsetstrokeopacity{0.487397}%
\pgfsetdash{}{0pt}%
\pgfpathmoveto{\pgfqpoint{2.895786in}{2.327929in}}%
\pgfpathcurveto{\pgfqpoint{2.904023in}{2.327929in}}{\pgfqpoint{2.911923in}{2.331201in}}{\pgfqpoint{2.917747in}{2.337025in}}%
\pgfpathcurveto{\pgfqpoint{2.923571in}{2.342849in}}{\pgfqpoint{2.926843in}{2.350749in}}{\pgfqpoint{2.926843in}{2.358986in}}%
\pgfpathcurveto{\pgfqpoint{2.926843in}{2.367222in}}{\pgfqpoint{2.923571in}{2.375122in}}{\pgfqpoint{2.917747in}{2.380946in}}%
\pgfpathcurveto{\pgfqpoint{2.911923in}{2.386770in}}{\pgfqpoint{2.904023in}{2.390042in}}{\pgfqpoint{2.895786in}{2.390042in}}%
\pgfpathcurveto{\pgfqpoint{2.887550in}{2.390042in}}{\pgfqpoint{2.879650in}{2.386770in}}{\pgfqpoint{2.873826in}{2.380946in}}%
\pgfpathcurveto{\pgfqpoint{2.868002in}{2.375122in}}{\pgfqpoint{2.864730in}{2.367222in}}{\pgfqpoint{2.864730in}{2.358986in}}%
\pgfpathcurveto{\pgfqpoint{2.864730in}{2.350749in}}{\pgfqpoint{2.868002in}{2.342849in}}{\pgfqpoint{2.873826in}{2.337025in}}%
\pgfpathcurveto{\pgfqpoint{2.879650in}{2.331201in}}{\pgfqpoint{2.887550in}{2.327929in}}{\pgfqpoint{2.895786in}{2.327929in}}%
\pgfpathclose%
\pgfusepath{stroke,fill}%
\end{pgfscope}%
\begin{pgfscope}%
\pgfpathrectangle{\pgfqpoint{0.100000in}{0.212622in}}{\pgfqpoint{3.696000in}{3.696000in}}%
\pgfusepath{clip}%
\pgfsetbuttcap%
\pgfsetroundjoin%
\definecolor{currentfill}{rgb}{0.121569,0.466667,0.705882}%
\pgfsetfillcolor{currentfill}%
\pgfsetfillopacity{0.488200}%
\pgfsetlinewidth{1.003750pt}%
\definecolor{currentstroke}{rgb}{0.121569,0.466667,0.705882}%
\pgfsetstrokecolor{currentstroke}%
\pgfsetstrokeopacity{0.488200}%
\pgfsetdash{}{0pt}%
\pgfpathmoveto{\pgfqpoint{2.901555in}{2.327008in}}%
\pgfpathcurveto{\pgfqpoint{2.909791in}{2.327008in}}{\pgfqpoint{2.917691in}{2.330280in}}{\pgfqpoint{2.923515in}{2.336104in}}%
\pgfpathcurveto{\pgfqpoint{2.929339in}{2.341928in}}{\pgfqpoint{2.932611in}{2.349828in}}{\pgfqpoint{2.932611in}{2.358064in}}%
\pgfpathcurveto{\pgfqpoint{2.932611in}{2.366300in}}{\pgfqpoint{2.929339in}{2.374201in}}{\pgfqpoint{2.923515in}{2.380024in}}%
\pgfpathcurveto{\pgfqpoint{2.917691in}{2.385848in}}{\pgfqpoint{2.909791in}{2.389121in}}{\pgfqpoint{2.901555in}{2.389121in}}%
\pgfpathcurveto{\pgfqpoint{2.893319in}{2.389121in}}{\pgfqpoint{2.885418in}{2.385848in}}{\pgfqpoint{2.879595in}{2.380024in}}%
\pgfpathcurveto{\pgfqpoint{2.873771in}{2.374201in}}{\pgfqpoint{2.870498in}{2.366300in}}{\pgfqpoint{2.870498in}{2.358064in}}%
\pgfpathcurveto{\pgfqpoint{2.870498in}{2.349828in}}{\pgfqpoint{2.873771in}{2.341928in}}{\pgfqpoint{2.879595in}{2.336104in}}%
\pgfpathcurveto{\pgfqpoint{2.885418in}{2.330280in}}{\pgfqpoint{2.893319in}{2.327008in}}{\pgfqpoint{2.901555in}{2.327008in}}%
\pgfpathclose%
\pgfusepath{stroke,fill}%
\end{pgfscope}%
\begin{pgfscope}%
\pgfpathrectangle{\pgfqpoint{0.100000in}{0.212622in}}{\pgfqpoint{3.696000in}{3.696000in}}%
\pgfusepath{clip}%
\pgfsetbuttcap%
\pgfsetroundjoin%
\definecolor{currentfill}{rgb}{0.121569,0.466667,0.705882}%
\pgfsetfillcolor{currentfill}%
\pgfsetfillopacity{0.489352}%
\pgfsetlinewidth{1.003750pt}%
\definecolor{currentstroke}{rgb}{0.121569,0.466667,0.705882}%
\pgfsetstrokecolor{currentstroke}%
\pgfsetstrokeopacity{0.489352}%
\pgfsetdash{}{0pt}%
\pgfpathmoveto{\pgfqpoint{2.909838in}{2.325277in}}%
\pgfpathcurveto{\pgfqpoint{2.918075in}{2.325277in}}{\pgfqpoint{2.925975in}{2.328549in}}{\pgfqpoint{2.931799in}{2.334373in}}%
\pgfpathcurveto{\pgfqpoint{2.937623in}{2.340197in}}{\pgfqpoint{2.940895in}{2.348097in}}{\pgfqpoint{2.940895in}{2.356333in}}%
\pgfpathcurveto{\pgfqpoint{2.940895in}{2.364570in}}{\pgfqpoint{2.937623in}{2.372470in}}{\pgfqpoint{2.931799in}{2.378294in}}%
\pgfpathcurveto{\pgfqpoint{2.925975in}{2.384117in}}{\pgfqpoint{2.918075in}{2.387390in}}{\pgfqpoint{2.909838in}{2.387390in}}%
\pgfpathcurveto{\pgfqpoint{2.901602in}{2.387390in}}{\pgfqpoint{2.893702in}{2.384117in}}{\pgfqpoint{2.887878in}{2.378294in}}%
\pgfpathcurveto{\pgfqpoint{2.882054in}{2.372470in}}{\pgfqpoint{2.878782in}{2.364570in}}{\pgfqpoint{2.878782in}{2.356333in}}%
\pgfpathcurveto{\pgfqpoint{2.878782in}{2.348097in}}{\pgfqpoint{2.882054in}{2.340197in}}{\pgfqpoint{2.887878in}{2.334373in}}%
\pgfpathcurveto{\pgfqpoint{2.893702in}{2.328549in}}{\pgfqpoint{2.901602in}{2.325277in}}{\pgfqpoint{2.909838in}{2.325277in}}%
\pgfpathclose%
\pgfusepath{stroke,fill}%
\end{pgfscope}%
\begin{pgfscope}%
\pgfpathrectangle{\pgfqpoint{0.100000in}{0.212622in}}{\pgfqpoint{3.696000in}{3.696000in}}%
\pgfusepath{clip}%
\pgfsetbuttcap%
\pgfsetroundjoin%
\definecolor{currentfill}{rgb}{0.121569,0.466667,0.705882}%
\pgfsetfillcolor{currentfill}%
\pgfsetfillopacity{0.489444}%
\pgfsetlinewidth{1.003750pt}%
\definecolor{currentstroke}{rgb}{0.121569,0.466667,0.705882}%
\pgfsetstrokecolor{currentstroke}%
\pgfsetstrokeopacity{0.489444}%
\pgfsetdash{}{0pt}%
\pgfpathmoveto{\pgfqpoint{1.121500in}{1.859230in}}%
\pgfpathcurveto{\pgfqpoint{1.129736in}{1.859230in}}{\pgfqpoint{1.137636in}{1.862502in}}{\pgfqpoint{1.143460in}{1.868326in}}%
\pgfpathcurveto{\pgfqpoint{1.149284in}{1.874150in}}{\pgfqpoint{1.152557in}{1.882050in}}{\pgfqpoint{1.152557in}{1.890286in}}%
\pgfpathcurveto{\pgfqpoint{1.152557in}{1.898522in}}{\pgfqpoint{1.149284in}{1.906422in}}{\pgfqpoint{1.143460in}{1.912246in}}%
\pgfpathcurveto{\pgfqpoint{1.137636in}{1.918070in}}{\pgfqpoint{1.129736in}{1.921343in}}{\pgfqpoint{1.121500in}{1.921343in}}%
\pgfpathcurveto{\pgfqpoint{1.113264in}{1.921343in}}{\pgfqpoint{1.105364in}{1.918070in}}{\pgfqpoint{1.099540in}{1.912246in}}%
\pgfpathcurveto{\pgfqpoint{1.093716in}{1.906422in}}{\pgfqpoint{1.090444in}{1.898522in}}{\pgfqpoint{1.090444in}{1.890286in}}%
\pgfpathcurveto{\pgfqpoint{1.090444in}{1.882050in}}{\pgfqpoint{1.093716in}{1.874150in}}{\pgfqpoint{1.099540in}{1.868326in}}%
\pgfpathcurveto{\pgfqpoint{1.105364in}{1.862502in}}{\pgfqpoint{1.113264in}{1.859230in}}{\pgfqpoint{1.121500in}{1.859230in}}%
\pgfpathclose%
\pgfusepath{stroke,fill}%
\end{pgfscope}%
\begin{pgfscope}%
\pgfpathrectangle{\pgfqpoint{0.100000in}{0.212622in}}{\pgfqpoint{3.696000in}{3.696000in}}%
\pgfusepath{clip}%
\pgfsetbuttcap%
\pgfsetroundjoin%
\definecolor{currentfill}{rgb}{0.121569,0.466667,0.705882}%
\pgfsetfillcolor{currentfill}%
\pgfsetfillopacity{0.490011}%
\pgfsetlinewidth{1.003750pt}%
\definecolor{currentstroke}{rgb}{0.121569,0.466667,0.705882}%
\pgfsetstrokecolor{currentstroke}%
\pgfsetstrokeopacity{0.490011}%
\pgfsetdash{}{0pt}%
\pgfpathmoveto{\pgfqpoint{2.914400in}{2.324492in}}%
\pgfpathcurveto{\pgfqpoint{2.922637in}{2.324492in}}{\pgfqpoint{2.930537in}{2.327764in}}{\pgfqpoint{2.936361in}{2.333588in}}%
\pgfpathcurveto{\pgfqpoint{2.942185in}{2.339412in}}{\pgfqpoint{2.945457in}{2.347312in}}{\pgfqpoint{2.945457in}{2.355548in}}%
\pgfpathcurveto{\pgfqpoint{2.945457in}{2.363784in}}{\pgfqpoint{2.942185in}{2.371684in}}{\pgfqpoint{2.936361in}{2.377508in}}%
\pgfpathcurveto{\pgfqpoint{2.930537in}{2.383332in}}{\pgfqpoint{2.922637in}{2.386605in}}{\pgfqpoint{2.914400in}{2.386605in}}%
\pgfpathcurveto{\pgfqpoint{2.906164in}{2.386605in}}{\pgfqpoint{2.898264in}{2.383332in}}{\pgfqpoint{2.892440in}{2.377508in}}%
\pgfpathcurveto{\pgfqpoint{2.886616in}{2.371684in}}{\pgfqpoint{2.883344in}{2.363784in}}{\pgfqpoint{2.883344in}{2.355548in}}%
\pgfpathcurveto{\pgfqpoint{2.883344in}{2.347312in}}{\pgfqpoint{2.886616in}{2.339412in}}{\pgfqpoint{2.892440in}{2.333588in}}%
\pgfpathcurveto{\pgfqpoint{2.898264in}{2.327764in}}{\pgfqpoint{2.906164in}{2.324492in}}{\pgfqpoint{2.914400in}{2.324492in}}%
\pgfpathclose%
\pgfusepath{stroke,fill}%
\end{pgfscope}%
\begin{pgfscope}%
\pgfpathrectangle{\pgfqpoint{0.100000in}{0.212622in}}{\pgfqpoint{3.696000in}{3.696000in}}%
\pgfusepath{clip}%
\pgfsetbuttcap%
\pgfsetroundjoin%
\definecolor{currentfill}{rgb}{0.121569,0.466667,0.705882}%
\pgfsetfillcolor{currentfill}%
\pgfsetfillopacity{0.490806}%
\pgfsetlinewidth{1.003750pt}%
\definecolor{currentstroke}{rgb}{0.121569,0.466667,0.705882}%
\pgfsetstrokecolor{currentstroke}%
\pgfsetstrokeopacity{0.490806}%
\pgfsetdash{}{0pt}%
\pgfpathmoveto{\pgfqpoint{2.919669in}{2.323723in}}%
\pgfpathcurveto{\pgfqpoint{2.927905in}{2.323723in}}{\pgfqpoint{2.935805in}{2.326995in}}{\pgfqpoint{2.941629in}{2.332819in}}%
\pgfpathcurveto{\pgfqpoint{2.947453in}{2.338643in}}{\pgfqpoint{2.950725in}{2.346543in}}{\pgfqpoint{2.950725in}{2.354780in}}%
\pgfpathcurveto{\pgfqpoint{2.950725in}{2.363016in}}{\pgfqpoint{2.947453in}{2.370916in}}{\pgfqpoint{2.941629in}{2.376740in}}%
\pgfpathcurveto{\pgfqpoint{2.935805in}{2.382564in}}{\pgfqpoint{2.927905in}{2.385836in}}{\pgfqpoint{2.919669in}{2.385836in}}%
\pgfpathcurveto{\pgfqpoint{2.911433in}{2.385836in}}{\pgfqpoint{2.903532in}{2.382564in}}{\pgfqpoint{2.897709in}{2.376740in}}%
\pgfpathcurveto{\pgfqpoint{2.891885in}{2.370916in}}{\pgfqpoint{2.888612in}{2.363016in}}{\pgfqpoint{2.888612in}{2.354780in}}%
\pgfpathcurveto{\pgfqpoint{2.888612in}{2.346543in}}{\pgfqpoint{2.891885in}{2.338643in}}{\pgfqpoint{2.897709in}{2.332819in}}%
\pgfpathcurveto{\pgfqpoint{2.903532in}{2.326995in}}{\pgfqpoint{2.911433in}{2.323723in}}{\pgfqpoint{2.919669in}{2.323723in}}%
\pgfpathclose%
\pgfusepath{stroke,fill}%
\end{pgfscope}%
\begin{pgfscope}%
\pgfpathrectangle{\pgfqpoint{0.100000in}{0.212622in}}{\pgfqpoint{3.696000in}{3.696000in}}%
\pgfusepath{clip}%
\pgfsetbuttcap%
\pgfsetroundjoin%
\definecolor{currentfill}{rgb}{0.121569,0.466667,0.705882}%
\pgfsetfillcolor{currentfill}%
\pgfsetfillopacity{0.491241}%
\pgfsetlinewidth{1.003750pt}%
\definecolor{currentstroke}{rgb}{0.121569,0.466667,0.705882}%
\pgfsetstrokecolor{currentstroke}%
\pgfsetstrokeopacity{0.491241}%
\pgfsetdash{}{0pt}%
\pgfpathmoveto{\pgfqpoint{2.922577in}{2.323314in}}%
\pgfpathcurveto{\pgfqpoint{2.930813in}{2.323314in}}{\pgfqpoint{2.938713in}{2.326586in}}{\pgfqpoint{2.944537in}{2.332410in}}%
\pgfpathcurveto{\pgfqpoint{2.950361in}{2.338234in}}{\pgfqpoint{2.953633in}{2.346134in}}{\pgfqpoint{2.953633in}{2.354370in}}%
\pgfpathcurveto{\pgfqpoint{2.953633in}{2.362607in}}{\pgfqpoint{2.950361in}{2.370507in}}{\pgfqpoint{2.944537in}{2.376331in}}%
\pgfpathcurveto{\pgfqpoint{2.938713in}{2.382155in}}{\pgfqpoint{2.930813in}{2.385427in}}{\pgfqpoint{2.922577in}{2.385427in}}%
\pgfpathcurveto{\pgfqpoint{2.914340in}{2.385427in}}{\pgfqpoint{2.906440in}{2.382155in}}{\pgfqpoint{2.900616in}{2.376331in}}%
\pgfpathcurveto{\pgfqpoint{2.894793in}{2.370507in}}{\pgfqpoint{2.891520in}{2.362607in}}{\pgfqpoint{2.891520in}{2.354370in}}%
\pgfpathcurveto{\pgfqpoint{2.891520in}{2.346134in}}{\pgfqpoint{2.894793in}{2.338234in}}{\pgfqpoint{2.900616in}{2.332410in}}%
\pgfpathcurveto{\pgfqpoint{2.906440in}{2.326586in}}{\pgfqpoint{2.914340in}{2.323314in}}{\pgfqpoint{2.922577in}{2.323314in}}%
\pgfpathclose%
\pgfusepath{stroke,fill}%
\end{pgfscope}%
\begin{pgfscope}%
\pgfpathrectangle{\pgfqpoint{0.100000in}{0.212622in}}{\pgfqpoint{3.696000in}{3.696000in}}%
\pgfusepath{clip}%
\pgfsetbuttcap%
\pgfsetroundjoin%
\definecolor{currentfill}{rgb}{0.121569,0.466667,0.705882}%
\pgfsetfillcolor{currentfill}%
\pgfsetfillopacity{0.491884}%
\pgfsetlinewidth{1.003750pt}%
\definecolor{currentstroke}{rgb}{0.121569,0.466667,0.705882}%
\pgfsetstrokecolor{currentstroke}%
\pgfsetstrokeopacity{0.491884}%
\pgfsetdash{}{0pt}%
\pgfpathmoveto{\pgfqpoint{2.927369in}{2.321862in}}%
\pgfpathcurveto{\pgfqpoint{2.935605in}{2.321862in}}{\pgfqpoint{2.943506in}{2.325135in}}{\pgfqpoint{2.949329in}{2.330959in}}%
\pgfpathcurveto{\pgfqpoint{2.955153in}{2.336783in}}{\pgfqpoint{2.958426in}{2.344683in}}{\pgfqpoint{2.958426in}{2.352919in}}%
\pgfpathcurveto{\pgfqpoint{2.958426in}{2.361155in}}{\pgfqpoint{2.955153in}{2.369055in}}{\pgfqpoint{2.949329in}{2.374879in}}%
\pgfpathcurveto{\pgfqpoint{2.943506in}{2.380703in}}{\pgfqpoint{2.935605in}{2.383975in}}{\pgfqpoint{2.927369in}{2.383975in}}%
\pgfpathcurveto{\pgfqpoint{2.919133in}{2.383975in}}{\pgfqpoint{2.911233in}{2.380703in}}{\pgfqpoint{2.905409in}{2.374879in}}%
\pgfpathcurveto{\pgfqpoint{2.899585in}{2.369055in}}{\pgfqpoint{2.896313in}{2.361155in}}{\pgfqpoint{2.896313in}{2.352919in}}%
\pgfpathcurveto{\pgfqpoint{2.896313in}{2.344683in}}{\pgfqpoint{2.899585in}{2.336783in}}{\pgfqpoint{2.905409in}{2.330959in}}%
\pgfpathcurveto{\pgfqpoint{2.911233in}{2.325135in}}{\pgfqpoint{2.919133in}{2.321862in}}{\pgfqpoint{2.927369in}{2.321862in}}%
\pgfpathclose%
\pgfusepath{stroke,fill}%
\end{pgfscope}%
\begin{pgfscope}%
\pgfpathrectangle{\pgfqpoint{0.100000in}{0.212622in}}{\pgfqpoint{3.696000in}{3.696000in}}%
\pgfusepath{clip}%
\pgfsetbuttcap%
\pgfsetroundjoin%
\definecolor{currentfill}{rgb}{0.121569,0.466667,0.705882}%
\pgfsetfillcolor{currentfill}%
\pgfsetfillopacity{0.492760}%
\pgfsetlinewidth{1.003750pt}%
\definecolor{currentstroke}{rgb}{0.121569,0.466667,0.705882}%
\pgfsetstrokecolor{currentstroke}%
\pgfsetstrokeopacity{0.492760}%
\pgfsetdash{}{0pt}%
\pgfpathmoveto{\pgfqpoint{2.933834in}{2.320049in}}%
\pgfpathcurveto{\pgfqpoint{2.942071in}{2.320049in}}{\pgfqpoint{2.949971in}{2.323322in}}{\pgfqpoint{2.955795in}{2.329146in}}%
\pgfpathcurveto{\pgfqpoint{2.961618in}{2.334970in}}{\pgfqpoint{2.964891in}{2.342870in}}{\pgfqpoint{2.964891in}{2.351106in}}%
\pgfpathcurveto{\pgfqpoint{2.964891in}{2.359342in}}{\pgfqpoint{2.961618in}{2.367242in}}{\pgfqpoint{2.955795in}{2.373066in}}%
\pgfpathcurveto{\pgfqpoint{2.949971in}{2.378890in}}{\pgfqpoint{2.942071in}{2.382162in}}{\pgfqpoint{2.933834in}{2.382162in}}%
\pgfpathcurveto{\pgfqpoint{2.925598in}{2.382162in}}{\pgfqpoint{2.917698in}{2.378890in}}{\pgfqpoint{2.911874in}{2.373066in}}%
\pgfpathcurveto{\pgfqpoint{2.906050in}{2.367242in}}{\pgfqpoint{2.902778in}{2.359342in}}{\pgfqpoint{2.902778in}{2.351106in}}%
\pgfpathcurveto{\pgfqpoint{2.902778in}{2.342870in}}{\pgfqpoint{2.906050in}{2.334970in}}{\pgfqpoint{2.911874in}{2.329146in}}%
\pgfpathcurveto{\pgfqpoint{2.917698in}{2.323322in}}{\pgfqpoint{2.925598in}{2.320049in}}{\pgfqpoint{2.933834in}{2.320049in}}%
\pgfpathclose%
\pgfusepath{stroke,fill}%
\end{pgfscope}%
\begin{pgfscope}%
\pgfpathrectangle{\pgfqpoint{0.100000in}{0.212622in}}{\pgfqpoint{3.696000in}{3.696000in}}%
\pgfusepath{clip}%
\pgfsetbuttcap%
\pgfsetroundjoin%
\definecolor{currentfill}{rgb}{0.121569,0.466667,0.705882}%
\pgfsetfillcolor{currentfill}%
\pgfsetfillopacity{0.493134}%
\pgfsetlinewidth{1.003750pt}%
\definecolor{currentstroke}{rgb}{0.121569,0.466667,0.705882}%
\pgfsetstrokecolor{currentstroke}%
\pgfsetstrokeopacity{0.493134}%
\pgfsetdash{}{0pt}%
\pgfpathmoveto{\pgfqpoint{1.109595in}{1.842465in}}%
\pgfpathcurveto{\pgfqpoint{1.117832in}{1.842465in}}{\pgfqpoint{1.125732in}{1.845737in}}{\pgfqpoint{1.131556in}{1.851561in}}%
\pgfpathcurveto{\pgfqpoint{1.137380in}{1.857385in}}{\pgfqpoint{1.140652in}{1.865285in}}{\pgfqpoint{1.140652in}{1.873521in}}%
\pgfpathcurveto{\pgfqpoint{1.140652in}{1.881758in}}{\pgfqpoint{1.137380in}{1.889658in}}{\pgfqpoint{1.131556in}{1.895482in}}%
\pgfpathcurveto{\pgfqpoint{1.125732in}{1.901306in}}{\pgfqpoint{1.117832in}{1.904578in}}{\pgfqpoint{1.109595in}{1.904578in}}%
\pgfpathcurveto{\pgfqpoint{1.101359in}{1.904578in}}{\pgfqpoint{1.093459in}{1.901306in}}{\pgfqpoint{1.087635in}{1.895482in}}%
\pgfpathcurveto{\pgfqpoint{1.081811in}{1.889658in}}{\pgfqpoint{1.078539in}{1.881758in}}{\pgfqpoint{1.078539in}{1.873521in}}%
\pgfpathcurveto{\pgfqpoint{1.078539in}{1.865285in}}{\pgfqpoint{1.081811in}{1.857385in}}{\pgfqpoint{1.087635in}{1.851561in}}%
\pgfpathcurveto{\pgfqpoint{1.093459in}{1.845737in}}{\pgfqpoint{1.101359in}{1.842465in}}{\pgfqpoint{1.109595in}{1.842465in}}%
\pgfpathclose%
\pgfusepath{stroke,fill}%
\end{pgfscope}%
\begin{pgfscope}%
\pgfpathrectangle{\pgfqpoint{0.100000in}{0.212622in}}{\pgfqpoint{3.696000in}{3.696000in}}%
\pgfusepath{clip}%
\pgfsetbuttcap%
\pgfsetroundjoin%
\definecolor{currentfill}{rgb}{0.121569,0.466667,0.705882}%
\pgfsetfillcolor{currentfill}%
\pgfsetfillopacity{0.493848}%
\pgfsetlinewidth{1.003750pt}%
\definecolor{currentstroke}{rgb}{0.121569,0.466667,0.705882}%
\pgfsetstrokecolor{currentstroke}%
\pgfsetstrokeopacity{0.493848}%
\pgfsetdash{}{0pt}%
\pgfpathmoveto{\pgfqpoint{2.941528in}{2.318585in}}%
\pgfpathcurveto{\pgfqpoint{2.949765in}{2.318585in}}{\pgfqpoint{2.957665in}{2.321857in}}{\pgfqpoint{2.963489in}{2.327681in}}%
\pgfpathcurveto{\pgfqpoint{2.969313in}{2.333505in}}{\pgfqpoint{2.972585in}{2.341405in}}{\pgfqpoint{2.972585in}{2.349641in}}%
\pgfpathcurveto{\pgfqpoint{2.972585in}{2.357878in}}{\pgfqpoint{2.969313in}{2.365778in}}{\pgfqpoint{2.963489in}{2.371602in}}%
\pgfpathcurveto{\pgfqpoint{2.957665in}{2.377426in}}{\pgfqpoint{2.949765in}{2.380698in}}{\pgfqpoint{2.941528in}{2.380698in}}%
\pgfpathcurveto{\pgfqpoint{2.933292in}{2.380698in}}{\pgfqpoint{2.925392in}{2.377426in}}{\pgfqpoint{2.919568in}{2.371602in}}%
\pgfpathcurveto{\pgfqpoint{2.913744in}{2.365778in}}{\pgfqpoint{2.910472in}{2.357878in}}{\pgfqpoint{2.910472in}{2.349641in}}%
\pgfpathcurveto{\pgfqpoint{2.910472in}{2.341405in}}{\pgfqpoint{2.913744in}{2.333505in}}{\pgfqpoint{2.919568in}{2.327681in}}%
\pgfpathcurveto{\pgfqpoint{2.925392in}{2.321857in}}{\pgfqpoint{2.933292in}{2.318585in}}{\pgfqpoint{2.941528in}{2.318585in}}%
\pgfpathclose%
\pgfusepath{stroke,fill}%
\end{pgfscope}%
\begin{pgfscope}%
\pgfpathrectangle{\pgfqpoint{0.100000in}{0.212622in}}{\pgfqpoint{3.696000in}{3.696000in}}%
\pgfusepath{clip}%
\pgfsetbuttcap%
\pgfsetroundjoin%
\definecolor{currentfill}{rgb}{0.121569,0.466667,0.705882}%
\pgfsetfillcolor{currentfill}%
\pgfsetfillopacity{0.495040}%
\pgfsetlinewidth{1.003750pt}%
\definecolor{currentstroke}{rgb}{0.121569,0.466667,0.705882}%
\pgfsetstrokecolor{currentstroke}%
\pgfsetstrokeopacity{0.495040}%
\pgfsetdash{}{0pt}%
\pgfpathmoveto{\pgfqpoint{2.950613in}{2.316407in}}%
\pgfpathcurveto{\pgfqpoint{2.958849in}{2.316407in}}{\pgfqpoint{2.966749in}{2.319680in}}{\pgfqpoint{2.972573in}{2.325504in}}%
\pgfpathcurveto{\pgfqpoint{2.978397in}{2.331328in}}{\pgfqpoint{2.981670in}{2.339228in}}{\pgfqpoint{2.981670in}{2.347464in}}%
\pgfpathcurveto{\pgfqpoint{2.981670in}{2.355700in}}{\pgfqpoint{2.978397in}{2.363600in}}{\pgfqpoint{2.972573in}{2.369424in}}%
\pgfpathcurveto{\pgfqpoint{2.966749in}{2.375248in}}{\pgfqpoint{2.958849in}{2.378520in}}{\pgfqpoint{2.950613in}{2.378520in}}%
\pgfpathcurveto{\pgfqpoint{2.942377in}{2.378520in}}{\pgfqpoint{2.934477in}{2.375248in}}{\pgfqpoint{2.928653in}{2.369424in}}%
\pgfpathcurveto{\pgfqpoint{2.922829in}{2.363600in}}{\pgfqpoint{2.919557in}{2.355700in}}{\pgfqpoint{2.919557in}{2.347464in}}%
\pgfpathcurveto{\pgfqpoint{2.919557in}{2.339228in}}{\pgfqpoint{2.922829in}{2.331328in}}{\pgfqpoint{2.928653in}{2.325504in}}%
\pgfpathcurveto{\pgfqpoint{2.934477in}{2.319680in}}{\pgfqpoint{2.942377in}{2.316407in}}{\pgfqpoint{2.950613in}{2.316407in}}%
\pgfpathclose%
\pgfusepath{stroke,fill}%
\end{pgfscope}%
\begin{pgfscope}%
\pgfpathrectangle{\pgfqpoint{0.100000in}{0.212622in}}{\pgfqpoint{3.696000in}{3.696000in}}%
\pgfusepath{clip}%
\pgfsetbuttcap%
\pgfsetroundjoin%
\definecolor{currentfill}{rgb}{0.121569,0.466667,0.705882}%
\pgfsetfillcolor{currentfill}%
\pgfsetfillopacity{0.496445}%
\pgfsetlinewidth{1.003750pt}%
\definecolor{currentstroke}{rgb}{0.121569,0.466667,0.705882}%
\pgfsetstrokecolor{currentstroke}%
\pgfsetstrokeopacity{0.496445}%
\pgfsetdash{}{0pt}%
\pgfpathmoveto{\pgfqpoint{2.961889in}{2.312960in}}%
\pgfpathcurveto{\pgfqpoint{2.970126in}{2.312960in}}{\pgfqpoint{2.978026in}{2.316232in}}{\pgfqpoint{2.983850in}{2.322056in}}%
\pgfpathcurveto{\pgfqpoint{2.989674in}{2.327880in}}{\pgfqpoint{2.992946in}{2.335780in}}{\pgfqpoint{2.992946in}{2.344016in}}%
\pgfpathcurveto{\pgfqpoint{2.992946in}{2.352252in}}{\pgfqpoint{2.989674in}{2.360152in}}{\pgfqpoint{2.983850in}{2.365976in}}%
\pgfpathcurveto{\pgfqpoint{2.978026in}{2.371800in}}{\pgfqpoint{2.970126in}{2.375073in}}{\pgfqpoint{2.961889in}{2.375073in}}%
\pgfpathcurveto{\pgfqpoint{2.953653in}{2.375073in}}{\pgfqpoint{2.945753in}{2.371800in}}{\pgfqpoint{2.939929in}{2.365976in}}%
\pgfpathcurveto{\pgfqpoint{2.934105in}{2.360152in}}{\pgfqpoint{2.930833in}{2.352252in}}{\pgfqpoint{2.930833in}{2.344016in}}%
\pgfpathcurveto{\pgfqpoint{2.930833in}{2.335780in}}{\pgfqpoint{2.934105in}{2.327880in}}{\pgfqpoint{2.939929in}{2.322056in}}%
\pgfpathcurveto{\pgfqpoint{2.945753in}{2.316232in}}{\pgfqpoint{2.953653in}{2.312960in}}{\pgfqpoint{2.961889in}{2.312960in}}%
\pgfpathclose%
\pgfusepath{stroke,fill}%
\end{pgfscope}%
\begin{pgfscope}%
\pgfpathrectangle{\pgfqpoint{0.100000in}{0.212622in}}{\pgfqpoint{3.696000in}{3.696000in}}%
\pgfusepath{clip}%
\pgfsetbuttcap%
\pgfsetroundjoin%
\definecolor{currentfill}{rgb}{0.121569,0.466667,0.705882}%
\pgfsetfillcolor{currentfill}%
\pgfsetfillopacity{0.496501}%
\pgfsetlinewidth{1.003750pt}%
\definecolor{currentstroke}{rgb}{0.121569,0.466667,0.705882}%
\pgfsetstrokecolor{currentstroke}%
\pgfsetstrokeopacity{0.496501}%
\pgfsetdash{}{0pt}%
\pgfpathmoveto{\pgfqpoint{1.099068in}{1.827869in}}%
\pgfpathcurveto{\pgfqpoint{1.107305in}{1.827869in}}{\pgfqpoint{1.115205in}{1.831142in}}{\pgfqpoint{1.121029in}{1.836966in}}%
\pgfpathcurveto{\pgfqpoint{1.126852in}{1.842789in}}{\pgfqpoint{1.130125in}{1.850690in}}{\pgfqpoint{1.130125in}{1.858926in}}%
\pgfpathcurveto{\pgfqpoint{1.130125in}{1.867162in}}{\pgfqpoint{1.126852in}{1.875062in}}{\pgfqpoint{1.121029in}{1.880886in}}%
\pgfpathcurveto{\pgfqpoint{1.115205in}{1.886710in}}{\pgfqpoint{1.107305in}{1.889982in}}{\pgfqpoint{1.099068in}{1.889982in}}%
\pgfpathcurveto{\pgfqpoint{1.090832in}{1.889982in}}{\pgfqpoint{1.082932in}{1.886710in}}{\pgfqpoint{1.077108in}{1.880886in}}%
\pgfpathcurveto{\pgfqpoint{1.071284in}{1.875062in}}{\pgfqpoint{1.068012in}{1.867162in}}{\pgfqpoint{1.068012in}{1.858926in}}%
\pgfpathcurveto{\pgfqpoint{1.068012in}{1.850690in}}{\pgfqpoint{1.071284in}{1.842789in}}{\pgfqpoint{1.077108in}{1.836966in}}%
\pgfpathcurveto{\pgfqpoint{1.082932in}{1.831142in}}{\pgfqpoint{1.090832in}{1.827869in}}{\pgfqpoint{1.099068in}{1.827869in}}%
\pgfpathclose%
\pgfusepath{stroke,fill}%
\end{pgfscope}%
\begin{pgfscope}%
\pgfpathrectangle{\pgfqpoint{0.100000in}{0.212622in}}{\pgfqpoint{3.696000in}{3.696000in}}%
\pgfusepath{clip}%
\pgfsetbuttcap%
\pgfsetroundjoin%
\definecolor{currentfill}{rgb}{0.121569,0.466667,0.705882}%
\pgfsetfillcolor{currentfill}%
\pgfsetfillopacity{0.498151}%
\pgfsetlinewidth{1.003750pt}%
\definecolor{currentstroke}{rgb}{0.121569,0.466667,0.705882}%
\pgfsetstrokecolor{currentstroke}%
\pgfsetstrokeopacity{0.498151}%
\pgfsetdash{}{0pt}%
\pgfpathmoveto{\pgfqpoint{2.974386in}{2.310626in}}%
\pgfpathcurveto{\pgfqpoint{2.982622in}{2.310626in}}{\pgfqpoint{2.990522in}{2.313899in}}{\pgfqpoint{2.996346in}{2.319723in}}%
\pgfpathcurveto{\pgfqpoint{3.002170in}{2.325547in}}{\pgfqpoint{3.005442in}{2.333447in}}{\pgfqpoint{3.005442in}{2.341683in}}%
\pgfpathcurveto{\pgfqpoint{3.005442in}{2.349919in}}{\pgfqpoint{3.002170in}{2.357819in}}{\pgfqpoint{2.996346in}{2.363643in}}%
\pgfpathcurveto{\pgfqpoint{2.990522in}{2.369467in}}{\pgfqpoint{2.982622in}{2.372739in}}{\pgfqpoint{2.974386in}{2.372739in}}%
\pgfpathcurveto{\pgfqpoint{2.966149in}{2.372739in}}{\pgfqpoint{2.958249in}{2.369467in}}{\pgfqpoint{2.952425in}{2.363643in}}%
\pgfpathcurveto{\pgfqpoint{2.946601in}{2.357819in}}{\pgfqpoint{2.943329in}{2.349919in}}{\pgfqpoint{2.943329in}{2.341683in}}%
\pgfpathcurveto{\pgfqpoint{2.943329in}{2.333447in}}{\pgfqpoint{2.946601in}{2.325547in}}{\pgfqpoint{2.952425in}{2.319723in}}%
\pgfpathcurveto{\pgfqpoint{2.958249in}{2.313899in}}{\pgfqpoint{2.966149in}{2.310626in}}{\pgfqpoint{2.974386in}{2.310626in}}%
\pgfpathclose%
\pgfusepath{stroke,fill}%
\end{pgfscope}%
\begin{pgfscope}%
\pgfpathrectangle{\pgfqpoint{0.100000in}{0.212622in}}{\pgfqpoint{3.696000in}{3.696000in}}%
\pgfusepath{clip}%
\pgfsetbuttcap%
\pgfsetroundjoin%
\definecolor{currentfill}{rgb}{0.121569,0.466667,0.705882}%
\pgfsetfillcolor{currentfill}%
\pgfsetfillopacity{0.499673}%
\pgfsetlinewidth{1.003750pt}%
\definecolor{currentstroke}{rgb}{0.121569,0.466667,0.705882}%
\pgfsetstrokecolor{currentstroke}%
\pgfsetstrokeopacity{0.499673}%
\pgfsetdash{}{0pt}%
\pgfpathmoveto{\pgfqpoint{1.088590in}{1.813131in}}%
\pgfpathcurveto{\pgfqpoint{1.096827in}{1.813131in}}{\pgfqpoint{1.104727in}{1.816403in}}{\pgfqpoint{1.110551in}{1.822227in}}%
\pgfpathcurveto{\pgfqpoint{1.116374in}{1.828051in}}{\pgfqpoint{1.119647in}{1.835951in}}{\pgfqpoint{1.119647in}{1.844187in}}%
\pgfpathcurveto{\pgfqpoint{1.119647in}{1.852423in}}{\pgfqpoint{1.116374in}{1.860323in}}{\pgfqpoint{1.110551in}{1.866147in}}%
\pgfpathcurveto{\pgfqpoint{1.104727in}{1.871971in}}{\pgfqpoint{1.096827in}{1.875244in}}{\pgfqpoint{1.088590in}{1.875244in}}%
\pgfpathcurveto{\pgfqpoint{1.080354in}{1.875244in}}{\pgfqpoint{1.072454in}{1.871971in}}{\pgfqpoint{1.066630in}{1.866147in}}%
\pgfpathcurveto{\pgfqpoint{1.060806in}{1.860323in}}{\pgfqpoint{1.057534in}{1.852423in}}{\pgfqpoint{1.057534in}{1.844187in}}%
\pgfpathcurveto{\pgfqpoint{1.057534in}{1.835951in}}{\pgfqpoint{1.060806in}{1.828051in}}{\pgfqpoint{1.066630in}{1.822227in}}%
\pgfpathcurveto{\pgfqpoint{1.072454in}{1.816403in}}{\pgfqpoint{1.080354in}{1.813131in}}{\pgfqpoint{1.088590in}{1.813131in}}%
\pgfpathclose%
\pgfusepath{stroke,fill}%
\end{pgfscope}%
\begin{pgfscope}%
\pgfpathrectangle{\pgfqpoint{0.100000in}{0.212622in}}{\pgfqpoint{3.696000in}{3.696000in}}%
\pgfusepath{clip}%
\pgfsetbuttcap%
\pgfsetroundjoin%
\definecolor{currentfill}{rgb}{0.121569,0.466667,0.705882}%
\pgfsetfillcolor{currentfill}%
\pgfsetfillopacity{0.500154}%
\pgfsetlinewidth{1.003750pt}%
\definecolor{currentstroke}{rgb}{0.121569,0.466667,0.705882}%
\pgfsetstrokecolor{currentstroke}%
\pgfsetstrokeopacity{0.500154}%
\pgfsetdash{}{0pt}%
\pgfpathmoveto{\pgfqpoint{2.988846in}{2.308038in}}%
\pgfpathcurveto{\pgfqpoint{2.997082in}{2.308038in}}{\pgfqpoint{3.004982in}{2.311310in}}{\pgfqpoint{3.010806in}{2.317134in}}%
\pgfpathcurveto{\pgfqpoint{3.016630in}{2.322958in}}{\pgfqpoint{3.019903in}{2.330858in}}{\pgfqpoint{3.019903in}{2.339094in}}%
\pgfpathcurveto{\pgfqpoint{3.019903in}{2.347331in}}{\pgfqpoint{3.016630in}{2.355231in}}{\pgfqpoint{3.010806in}{2.361055in}}%
\pgfpathcurveto{\pgfqpoint{3.004982in}{2.366879in}}{\pgfqpoint{2.997082in}{2.370151in}}{\pgfqpoint{2.988846in}{2.370151in}}%
\pgfpathcurveto{\pgfqpoint{2.980610in}{2.370151in}}{\pgfqpoint{2.972710in}{2.366879in}}{\pgfqpoint{2.966886in}{2.361055in}}%
\pgfpathcurveto{\pgfqpoint{2.961062in}{2.355231in}}{\pgfqpoint{2.957790in}{2.347331in}}{\pgfqpoint{2.957790in}{2.339094in}}%
\pgfpathcurveto{\pgfqpoint{2.957790in}{2.330858in}}{\pgfqpoint{2.961062in}{2.322958in}}{\pgfqpoint{2.966886in}{2.317134in}}%
\pgfpathcurveto{\pgfqpoint{2.972710in}{2.311310in}}{\pgfqpoint{2.980610in}{2.308038in}}{\pgfqpoint{2.988846in}{2.308038in}}%
\pgfpathclose%
\pgfusepath{stroke,fill}%
\end{pgfscope}%
\begin{pgfscope}%
\pgfpathrectangle{\pgfqpoint{0.100000in}{0.212622in}}{\pgfqpoint{3.696000in}{3.696000in}}%
\pgfusepath{clip}%
\pgfsetbuttcap%
\pgfsetroundjoin%
\definecolor{currentfill}{rgb}{0.121569,0.466667,0.705882}%
\pgfsetfillcolor{currentfill}%
\pgfsetfillopacity{0.502341}%
\pgfsetlinewidth{1.003750pt}%
\definecolor{currentstroke}{rgb}{0.121569,0.466667,0.705882}%
\pgfsetstrokecolor{currentstroke}%
\pgfsetstrokeopacity{0.502341}%
\pgfsetdash{}{0pt}%
\pgfpathmoveto{\pgfqpoint{3.003822in}{2.305683in}}%
\pgfpathcurveto{\pgfqpoint{3.012058in}{2.305683in}}{\pgfqpoint{3.019958in}{2.308956in}}{\pgfqpoint{3.025782in}{2.314779in}}%
\pgfpathcurveto{\pgfqpoint{3.031606in}{2.320603in}}{\pgfqpoint{3.034878in}{2.328503in}}{\pgfqpoint{3.034878in}{2.336740in}}%
\pgfpathcurveto{\pgfqpoint{3.034878in}{2.344976in}}{\pgfqpoint{3.031606in}{2.352876in}}{\pgfqpoint{3.025782in}{2.358700in}}%
\pgfpathcurveto{\pgfqpoint{3.019958in}{2.364524in}}{\pgfqpoint{3.012058in}{2.367796in}}{\pgfqpoint{3.003822in}{2.367796in}}%
\pgfpathcurveto{\pgfqpoint{2.995586in}{2.367796in}}{\pgfqpoint{2.987686in}{2.364524in}}{\pgfqpoint{2.981862in}{2.358700in}}%
\pgfpathcurveto{\pgfqpoint{2.976038in}{2.352876in}}{\pgfqpoint{2.972765in}{2.344976in}}{\pgfqpoint{2.972765in}{2.336740in}}%
\pgfpathcurveto{\pgfqpoint{2.972765in}{2.328503in}}{\pgfqpoint{2.976038in}{2.320603in}}{\pgfqpoint{2.981862in}{2.314779in}}%
\pgfpathcurveto{\pgfqpoint{2.987686in}{2.308956in}}{\pgfqpoint{2.995586in}{2.305683in}}{\pgfqpoint{3.003822in}{2.305683in}}%
\pgfpathclose%
\pgfusepath{stroke,fill}%
\end{pgfscope}%
\begin{pgfscope}%
\pgfpathrectangle{\pgfqpoint{0.100000in}{0.212622in}}{\pgfqpoint{3.696000in}{3.696000in}}%
\pgfusepath{clip}%
\pgfsetbuttcap%
\pgfsetroundjoin%
\definecolor{currentfill}{rgb}{0.121569,0.466667,0.705882}%
\pgfsetfillcolor{currentfill}%
\pgfsetfillopacity{0.502688}%
\pgfsetlinewidth{1.003750pt}%
\definecolor{currentstroke}{rgb}{0.121569,0.466667,0.705882}%
\pgfsetstrokecolor{currentstroke}%
\pgfsetstrokeopacity{0.502688}%
\pgfsetdash{}{0pt}%
\pgfpathmoveto{\pgfqpoint{1.078630in}{1.799251in}}%
\pgfpathcurveto{\pgfqpoint{1.086866in}{1.799251in}}{\pgfqpoint{1.094766in}{1.802523in}}{\pgfqpoint{1.100590in}{1.808347in}}%
\pgfpathcurveto{\pgfqpoint{1.106414in}{1.814171in}}{\pgfqpoint{1.109686in}{1.822071in}}{\pgfqpoint{1.109686in}{1.830308in}}%
\pgfpathcurveto{\pgfqpoint{1.109686in}{1.838544in}}{\pgfqpoint{1.106414in}{1.846444in}}{\pgfqpoint{1.100590in}{1.852268in}}%
\pgfpathcurveto{\pgfqpoint{1.094766in}{1.858092in}}{\pgfqpoint{1.086866in}{1.861364in}}{\pgfqpoint{1.078630in}{1.861364in}}%
\pgfpathcurveto{\pgfqpoint{1.070394in}{1.861364in}}{\pgfqpoint{1.062493in}{1.858092in}}{\pgfqpoint{1.056670in}{1.852268in}}%
\pgfpathcurveto{\pgfqpoint{1.050846in}{1.846444in}}{\pgfqpoint{1.047573in}{1.838544in}}{\pgfqpoint{1.047573in}{1.830308in}}%
\pgfpathcurveto{\pgfqpoint{1.047573in}{1.822071in}}{\pgfqpoint{1.050846in}{1.814171in}}{\pgfqpoint{1.056670in}{1.808347in}}%
\pgfpathcurveto{\pgfqpoint{1.062493in}{1.802523in}}{\pgfqpoint{1.070394in}{1.799251in}}{\pgfqpoint{1.078630in}{1.799251in}}%
\pgfpathclose%
\pgfusepath{stroke,fill}%
\end{pgfscope}%
\begin{pgfscope}%
\pgfpathrectangle{\pgfqpoint{0.100000in}{0.212622in}}{\pgfqpoint{3.696000in}{3.696000in}}%
\pgfusepath{clip}%
\pgfsetbuttcap%
\pgfsetroundjoin%
\definecolor{currentfill}{rgb}{0.121569,0.466667,0.705882}%
\pgfsetfillcolor{currentfill}%
\pgfsetfillopacity{0.504717}%
\pgfsetlinewidth{1.003750pt}%
\definecolor{currentstroke}{rgb}{0.121569,0.466667,0.705882}%
\pgfsetstrokecolor{currentstroke}%
\pgfsetstrokeopacity{0.504717}%
\pgfsetdash{}{0pt}%
\pgfpathmoveto{\pgfqpoint{3.020708in}{2.303246in}}%
\pgfpathcurveto{\pgfqpoint{3.028945in}{2.303246in}}{\pgfqpoint{3.036845in}{2.306518in}}{\pgfqpoint{3.042669in}{2.312342in}}%
\pgfpathcurveto{\pgfqpoint{3.048493in}{2.318166in}}{\pgfqpoint{3.051765in}{2.326066in}}{\pgfqpoint{3.051765in}{2.334303in}}%
\pgfpathcurveto{\pgfqpoint{3.051765in}{2.342539in}}{\pgfqpoint{3.048493in}{2.350439in}}{\pgfqpoint{3.042669in}{2.356263in}}%
\pgfpathcurveto{\pgfqpoint{3.036845in}{2.362087in}}{\pgfqpoint{3.028945in}{2.365359in}}{\pgfqpoint{3.020708in}{2.365359in}}%
\pgfpathcurveto{\pgfqpoint{3.012472in}{2.365359in}}{\pgfqpoint{3.004572in}{2.362087in}}{\pgfqpoint{2.998748in}{2.356263in}}%
\pgfpathcurveto{\pgfqpoint{2.992924in}{2.350439in}}{\pgfqpoint{2.989652in}{2.342539in}}{\pgfqpoint{2.989652in}{2.334303in}}%
\pgfpathcurveto{\pgfqpoint{2.989652in}{2.326066in}}{\pgfqpoint{2.992924in}{2.318166in}}{\pgfqpoint{2.998748in}{2.312342in}}%
\pgfpathcurveto{\pgfqpoint{3.004572in}{2.306518in}}{\pgfqpoint{3.012472in}{2.303246in}}{\pgfqpoint{3.020708in}{2.303246in}}%
\pgfpathclose%
\pgfusepath{stroke,fill}%
\end{pgfscope}%
\begin{pgfscope}%
\pgfpathrectangle{\pgfqpoint{0.100000in}{0.212622in}}{\pgfqpoint{3.696000in}{3.696000in}}%
\pgfusepath{clip}%
\pgfsetbuttcap%
\pgfsetroundjoin%
\definecolor{currentfill}{rgb}{0.121569,0.466667,0.705882}%
\pgfsetfillcolor{currentfill}%
\pgfsetfillopacity{0.505623}%
\pgfsetlinewidth{1.003750pt}%
\definecolor{currentstroke}{rgb}{0.121569,0.466667,0.705882}%
\pgfsetstrokecolor{currentstroke}%
\pgfsetstrokeopacity{0.505623}%
\pgfsetdash{}{0pt}%
\pgfpathmoveto{\pgfqpoint{1.068948in}{1.786161in}}%
\pgfpathcurveto{\pgfqpoint{1.077184in}{1.786161in}}{\pgfqpoint{1.085084in}{1.789433in}}{\pgfqpoint{1.090908in}{1.795257in}}%
\pgfpathcurveto{\pgfqpoint{1.096732in}{1.801081in}}{\pgfqpoint{1.100004in}{1.808981in}}{\pgfqpoint{1.100004in}{1.817218in}}%
\pgfpathcurveto{\pgfqpoint{1.100004in}{1.825454in}}{\pgfqpoint{1.096732in}{1.833354in}}{\pgfqpoint{1.090908in}{1.839178in}}%
\pgfpathcurveto{\pgfqpoint{1.085084in}{1.845002in}}{\pgfqpoint{1.077184in}{1.848274in}}{\pgfqpoint{1.068948in}{1.848274in}}%
\pgfpathcurveto{\pgfqpoint{1.060711in}{1.848274in}}{\pgfqpoint{1.052811in}{1.845002in}}{\pgfqpoint{1.046987in}{1.839178in}}%
\pgfpathcurveto{\pgfqpoint{1.041163in}{1.833354in}}{\pgfqpoint{1.037891in}{1.825454in}}{\pgfqpoint{1.037891in}{1.817218in}}%
\pgfpathcurveto{\pgfqpoint{1.037891in}{1.808981in}}{\pgfqpoint{1.041163in}{1.801081in}}{\pgfqpoint{1.046987in}{1.795257in}}%
\pgfpathcurveto{\pgfqpoint{1.052811in}{1.789433in}}{\pgfqpoint{1.060711in}{1.786161in}}{\pgfqpoint{1.068948in}{1.786161in}}%
\pgfpathclose%
\pgfusepath{stroke,fill}%
\end{pgfscope}%
\begin{pgfscope}%
\pgfpathrectangle{\pgfqpoint{0.100000in}{0.212622in}}{\pgfqpoint{3.696000in}{3.696000in}}%
\pgfusepath{clip}%
\pgfsetbuttcap%
\pgfsetroundjoin%
\definecolor{currentfill}{rgb}{0.121569,0.466667,0.705882}%
\pgfsetfillcolor{currentfill}%
\pgfsetfillopacity{0.507379}%
\pgfsetlinewidth{1.003750pt}%
\definecolor{currentstroke}{rgb}{0.121569,0.466667,0.705882}%
\pgfsetstrokecolor{currentstroke}%
\pgfsetstrokeopacity{0.507379}%
\pgfsetdash{}{0pt}%
\pgfpathmoveto{\pgfqpoint{3.038867in}{2.301190in}}%
\pgfpathcurveto{\pgfqpoint{3.047103in}{2.301190in}}{\pgfqpoint{3.055004in}{2.304462in}}{\pgfqpoint{3.060827in}{2.310286in}}%
\pgfpathcurveto{\pgfqpoint{3.066651in}{2.316110in}}{\pgfqpoint{3.069924in}{2.324010in}}{\pgfqpoint{3.069924in}{2.332246in}}%
\pgfpathcurveto{\pgfqpoint{3.069924in}{2.340482in}}{\pgfqpoint{3.066651in}{2.348382in}}{\pgfqpoint{3.060827in}{2.354206in}}%
\pgfpathcurveto{\pgfqpoint{3.055004in}{2.360030in}}{\pgfqpoint{3.047103in}{2.363303in}}{\pgfqpoint{3.038867in}{2.363303in}}%
\pgfpathcurveto{\pgfqpoint{3.030631in}{2.363303in}}{\pgfqpoint{3.022731in}{2.360030in}}{\pgfqpoint{3.016907in}{2.354206in}}%
\pgfpathcurveto{\pgfqpoint{3.011083in}{2.348382in}}{\pgfqpoint{3.007811in}{2.340482in}}{\pgfqpoint{3.007811in}{2.332246in}}%
\pgfpathcurveto{\pgfqpoint{3.007811in}{2.324010in}}{\pgfqpoint{3.011083in}{2.316110in}}{\pgfqpoint{3.016907in}{2.310286in}}%
\pgfpathcurveto{\pgfqpoint{3.022731in}{2.304462in}}{\pgfqpoint{3.030631in}{2.301190in}}{\pgfqpoint{3.038867in}{2.301190in}}%
\pgfpathclose%
\pgfusepath{stroke,fill}%
\end{pgfscope}%
\begin{pgfscope}%
\pgfpathrectangle{\pgfqpoint{0.100000in}{0.212622in}}{\pgfqpoint{3.696000in}{3.696000in}}%
\pgfusepath{clip}%
\pgfsetbuttcap%
\pgfsetroundjoin%
\definecolor{currentfill}{rgb}{0.121569,0.466667,0.705882}%
\pgfsetfillcolor{currentfill}%
\pgfsetfillopacity{0.508506}%
\pgfsetlinewidth{1.003750pt}%
\definecolor{currentstroke}{rgb}{0.121569,0.466667,0.705882}%
\pgfsetstrokecolor{currentstroke}%
\pgfsetstrokeopacity{0.508506}%
\pgfsetdash{}{0pt}%
\pgfpathmoveto{\pgfqpoint{1.059569in}{1.776442in}}%
\pgfpathcurveto{\pgfqpoint{1.067805in}{1.776442in}}{\pgfqpoint{1.075705in}{1.779714in}}{\pgfqpoint{1.081529in}{1.785538in}}%
\pgfpathcurveto{\pgfqpoint{1.087353in}{1.791362in}}{\pgfqpoint{1.090625in}{1.799262in}}{\pgfqpoint{1.090625in}{1.807498in}}%
\pgfpathcurveto{\pgfqpoint{1.090625in}{1.815734in}}{\pgfqpoint{1.087353in}{1.823634in}}{\pgfqpoint{1.081529in}{1.829458in}}%
\pgfpathcurveto{\pgfqpoint{1.075705in}{1.835282in}}{\pgfqpoint{1.067805in}{1.838555in}}{\pgfqpoint{1.059569in}{1.838555in}}%
\pgfpathcurveto{\pgfqpoint{1.051333in}{1.838555in}}{\pgfqpoint{1.043433in}{1.835282in}}{\pgfqpoint{1.037609in}{1.829458in}}%
\pgfpathcurveto{\pgfqpoint{1.031785in}{1.823634in}}{\pgfqpoint{1.028512in}{1.815734in}}{\pgfqpoint{1.028512in}{1.807498in}}%
\pgfpathcurveto{\pgfqpoint{1.028512in}{1.799262in}}{\pgfqpoint{1.031785in}{1.791362in}}{\pgfqpoint{1.037609in}{1.785538in}}%
\pgfpathcurveto{\pgfqpoint{1.043433in}{1.779714in}}{\pgfqpoint{1.051333in}{1.776442in}}{\pgfqpoint{1.059569in}{1.776442in}}%
\pgfpathclose%
\pgfusepath{stroke,fill}%
\end{pgfscope}%
\begin{pgfscope}%
\pgfpathrectangle{\pgfqpoint{0.100000in}{0.212622in}}{\pgfqpoint{3.696000in}{3.696000in}}%
\pgfusepath{clip}%
\pgfsetbuttcap%
\pgfsetroundjoin%
\definecolor{currentfill}{rgb}{0.121569,0.466667,0.705882}%
\pgfsetfillcolor{currentfill}%
\pgfsetfillopacity{0.510246}%
\pgfsetlinewidth{1.003750pt}%
\definecolor{currentstroke}{rgb}{0.121569,0.466667,0.705882}%
\pgfsetstrokecolor{currentstroke}%
\pgfsetstrokeopacity{0.510246}%
\pgfsetdash{}{0pt}%
\pgfpathmoveto{\pgfqpoint{3.059121in}{2.297633in}}%
\pgfpathcurveto{\pgfqpoint{3.067357in}{2.297633in}}{\pgfqpoint{3.075257in}{2.300906in}}{\pgfqpoint{3.081081in}{2.306729in}}%
\pgfpathcurveto{\pgfqpoint{3.086905in}{2.312553in}}{\pgfqpoint{3.090177in}{2.320453in}}{\pgfqpoint{3.090177in}{2.328690in}}%
\pgfpathcurveto{\pgfqpoint{3.090177in}{2.336926in}}{\pgfqpoint{3.086905in}{2.344826in}}{\pgfqpoint{3.081081in}{2.350650in}}%
\pgfpathcurveto{\pgfqpoint{3.075257in}{2.356474in}}{\pgfqpoint{3.067357in}{2.359746in}}{\pgfqpoint{3.059121in}{2.359746in}}%
\pgfpathcurveto{\pgfqpoint{3.050885in}{2.359746in}}{\pgfqpoint{3.042984in}{2.356474in}}{\pgfqpoint{3.037161in}{2.350650in}}%
\pgfpathcurveto{\pgfqpoint{3.031337in}{2.344826in}}{\pgfqpoint{3.028064in}{2.336926in}}{\pgfqpoint{3.028064in}{2.328690in}}%
\pgfpathcurveto{\pgfqpoint{3.028064in}{2.320453in}}{\pgfqpoint{3.031337in}{2.312553in}}{\pgfqpoint{3.037161in}{2.306729in}}%
\pgfpathcurveto{\pgfqpoint{3.042984in}{2.300906in}}{\pgfqpoint{3.050885in}{2.297633in}}{\pgfqpoint{3.059121in}{2.297633in}}%
\pgfpathclose%
\pgfusepath{stroke,fill}%
\end{pgfscope}%
\begin{pgfscope}%
\pgfpathrectangle{\pgfqpoint{0.100000in}{0.212622in}}{\pgfqpoint{3.696000in}{3.696000in}}%
\pgfusepath{clip}%
\pgfsetbuttcap%
\pgfsetroundjoin%
\definecolor{currentfill}{rgb}{0.121569,0.466667,0.705882}%
\pgfsetfillcolor{currentfill}%
\pgfsetfillopacity{0.511266}%
\pgfsetlinewidth{1.003750pt}%
\definecolor{currentstroke}{rgb}{0.121569,0.466667,0.705882}%
\pgfsetstrokecolor{currentstroke}%
\pgfsetstrokeopacity{0.511266}%
\pgfsetdash{}{0pt}%
\pgfpathmoveto{\pgfqpoint{1.051224in}{1.767416in}}%
\pgfpathcurveto{\pgfqpoint{1.059461in}{1.767416in}}{\pgfqpoint{1.067361in}{1.770689in}}{\pgfqpoint{1.073185in}{1.776512in}}%
\pgfpathcurveto{\pgfqpoint{1.079009in}{1.782336in}}{\pgfqpoint{1.082281in}{1.790236in}}{\pgfqpoint{1.082281in}{1.798473in}}%
\pgfpathcurveto{\pgfqpoint{1.082281in}{1.806709in}}{\pgfqpoint{1.079009in}{1.814609in}}{\pgfqpoint{1.073185in}{1.820433in}}%
\pgfpathcurveto{\pgfqpoint{1.067361in}{1.826257in}}{\pgfqpoint{1.059461in}{1.829529in}}{\pgfqpoint{1.051224in}{1.829529in}}%
\pgfpathcurveto{\pgfqpoint{1.042988in}{1.829529in}}{\pgfqpoint{1.035088in}{1.826257in}}{\pgfqpoint{1.029264in}{1.820433in}}%
\pgfpathcurveto{\pgfqpoint{1.023440in}{1.814609in}}{\pgfqpoint{1.020168in}{1.806709in}}{\pgfqpoint{1.020168in}{1.798473in}}%
\pgfpathcurveto{\pgfqpoint{1.020168in}{1.790236in}}{\pgfqpoint{1.023440in}{1.782336in}}{\pgfqpoint{1.029264in}{1.776512in}}%
\pgfpathcurveto{\pgfqpoint{1.035088in}{1.770689in}}{\pgfqpoint{1.042988in}{1.767416in}}{\pgfqpoint{1.051224in}{1.767416in}}%
\pgfpathclose%
\pgfusepath{stroke,fill}%
\end{pgfscope}%
\begin{pgfscope}%
\pgfpathrectangle{\pgfqpoint{0.100000in}{0.212622in}}{\pgfqpoint{3.696000in}{3.696000in}}%
\pgfusepath{clip}%
\pgfsetbuttcap%
\pgfsetroundjoin%
\definecolor{currentfill}{rgb}{0.121569,0.466667,0.705882}%
\pgfsetfillcolor{currentfill}%
\pgfsetfillopacity{0.511788}%
\pgfsetlinewidth{1.003750pt}%
\definecolor{currentstroke}{rgb}{0.121569,0.466667,0.705882}%
\pgfsetstrokecolor{currentstroke}%
\pgfsetstrokeopacity{0.511788}%
\pgfsetdash{}{0pt}%
\pgfpathmoveto{\pgfqpoint{3.070373in}{2.295822in}}%
\pgfpathcurveto{\pgfqpoint{3.078610in}{2.295822in}}{\pgfqpoint{3.086510in}{2.299094in}}{\pgfqpoint{3.092334in}{2.304918in}}%
\pgfpathcurveto{\pgfqpoint{3.098157in}{2.310742in}}{\pgfqpoint{3.101430in}{2.318642in}}{\pgfqpoint{3.101430in}{2.326879in}}%
\pgfpathcurveto{\pgfqpoint{3.101430in}{2.335115in}}{\pgfqpoint{3.098157in}{2.343015in}}{\pgfqpoint{3.092334in}{2.348839in}}%
\pgfpathcurveto{\pgfqpoint{3.086510in}{2.354663in}}{\pgfqpoint{3.078610in}{2.357935in}}{\pgfqpoint{3.070373in}{2.357935in}}%
\pgfpathcurveto{\pgfqpoint{3.062137in}{2.357935in}}{\pgfqpoint{3.054237in}{2.354663in}}{\pgfqpoint{3.048413in}{2.348839in}}%
\pgfpathcurveto{\pgfqpoint{3.042589in}{2.343015in}}{\pgfqpoint{3.039317in}{2.335115in}}{\pgfqpoint{3.039317in}{2.326879in}}%
\pgfpathcurveto{\pgfqpoint{3.039317in}{2.318642in}}{\pgfqpoint{3.042589in}{2.310742in}}{\pgfqpoint{3.048413in}{2.304918in}}%
\pgfpathcurveto{\pgfqpoint{3.054237in}{2.299094in}}{\pgfqpoint{3.062137in}{2.295822in}}{\pgfqpoint{3.070373in}{2.295822in}}%
\pgfpathclose%
\pgfusepath{stroke,fill}%
\end{pgfscope}%
\begin{pgfscope}%
\pgfpathrectangle{\pgfqpoint{0.100000in}{0.212622in}}{\pgfqpoint{3.696000in}{3.696000in}}%
\pgfusepath{clip}%
\pgfsetbuttcap%
\pgfsetroundjoin%
\definecolor{currentfill}{rgb}{0.121569,0.466667,0.705882}%
\pgfsetfillcolor{currentfill}%
\pgfsetfillopacity{0.512620}%
\pgfsetlinewidth{1.003750pt}%
\definecolor{currentstroke}{rgb}{0.121569,0.466667,0.705882}%
\pgfsetstrokecolor{currentstroke}%
\pgfsetstrokeopacity{0.512620}%
\pgfsetdash{}{0pt}%
\pgfpathmoveto{\pgfqpoint{3.076526in}{2.294585in}}%
\pgfpathcurveto{\pgfqpoint{3.084762in}{2.294585in}}{\pgfqpoint{3.092662in}{2.297857in}}{\pgfqpoint{3.098486in}{2.303681in}}%
\pgfpathcurveto{\pgfqpoint{3.104310in}{2.309505in}}{\pgfqpoint{3.107582in}{2.317405in}}{\pgfqpoint{3.107582in}{2.325641in}}%
\pgfpathcurveto{\pgfqpoint{3.107582in}{2.333877in}}{\pgfqpoint{3.104310in}{2.341777in}}{\pgfqpoint{3.098486in}{2.347601in}}%
\pgfpathcurveto{\pgfqpoint{3.092662in}{2.353425in}}{\pgfqpoint{3.084762in}{2.356698in}}{\pgfqpoint{3.076526in}{2.356698in}}%
\pgfpathcurveto{\pgfqpoint{3.068290in}{2.356698in}}{\pgfqpoint{3.060389in}{2.353425in}}{\pgfqpoint{3.054566in}{2.347601in}}%
\pgfpathcurveto{\pgfqpoint{3.048742in}{2.341777in}}{\pgfqpoint{3.045469in}{2.333877in}}{\pgfqpoint{3.045469in}{2.325641in}}%
\pgfpathcurveto{\pgfqpoint{3.045469in}{2.317405in}}{\pgfqpoint{3.048742in}{2.309505in}}{\pgfqpoint{3.054566in}{2.303681in}}%
\pgfpathcurveto{\pgfqpoint{3.060389in}{2.297857in}}{\pgfqpoint{3.068290in}{2.294585in}}{\pgfqpoint{3.076526in}{2.294585in}}%
\pgfpathclose%
\pgfusepath{stroke,fill}%
\end{pgfscope}%
\begin{pgfscope}%
\pgfpathrectangle{\pgfqpoint{0.100000in}{0.212622in}}{\pgfqpoint{3.696000in}{3.696000in}}%
\pgfusepath{clip}%
\pgfsetbuttcap%
\pgfsetroundjoin%
\definecolor{currentfill}{rgb}{0.121569,0.466667,0.705882}%
\pgfsetfillcolor{currentfill}%
\pgfsetfillopacity{0.513393}%
\pgfsetlinewidth{1.003750pt}%
\definecolor{currentstroke}{rgb}{0.121569,0.466667,0.705882}%
\pgfsetstrokecolor{currentstroke}%
\pgfsetstrokeopacity{0.513393}%
\pgfsetdash{}{0pt}%
\pgfpathmoveto{\pgfqpoint{1.044234in}{1.760161in}}%
\pgfpathcurveto{\pgfqpoint{1.052470in}{1.760161in}}{\pgfqpoint{1.060370in}{1.763433in}}{\pgfqpoint{1.066194in}{1.769257in}}%
\pgfpathcurveto{\pgfqpoint{1.072018in}{1.775081in}}{\pgfqpoint{1.075290in}{1.782981in}}{\pgfqpoint{1.075290in}{1.791217in}}%
\pgfpathcurveto{\pgfqpoint{1.075290in}{1.799453in}}{\pgfqpoint{1.072018in}{1.807354in}}{\pgfqpoint{1.066194in}{1.813177in}}%
\pgfpathcurveto{\pgfqpoint{1.060370in}{1.819001in}}{\pgfqpoint{1.052470in}{1.822274in}}{\pgfqpoint{1.044234in}{1.822274in}}%
\pgfpathcurveto{\pgfqpoint{1.035997in}{1.822274in}}{\pgfqpoint{1.028097in}{1.819001in}}{\pgfqpoint{1.022273in}{1.813177in}}%
\pgfpathcurveto{\pgfqpoint{1.016449in}{1.807354in}}{\pgfqpoint{1.013177in}{1.799453in}}{\pgfqpoint{1.013177in}{1.791217in}}%
\pgfpathcurveto{\pgfqpoint{1.013177in}{1.782981in}}{\pgfqpoint{1.016449in}{1.775081in}}{\pgfqpoint{1.022273in}{1.769257in}}%
\pgfpathcurveto{\pgfqpoint{1.028097in}{1.763433in}}{\pgfqpoint{1.035997in}{1.760161in}}{\pgfqpoint{1.044234in}{1.760161in}}%
\pgfpathclose%
\pgfusepath{stroke,fill}%
\end{pgfscope}%
\begin{pgfscope}%
\pgfpathrectangle{\pgfqpoint{0.100000in}{0.212622in}}{\pgfqpoint{3.696000in}{3.696000in}}%
\pgfusepath{clip}%
\pgfsetbuttcap%
\pgfsetroundjoin%
\definecolor{currentfill}{rgb}{0.121569,0.466667,0.705882}%
\pgfsetfillcolor{currentfill}%
\pgfsetfillopacity{0.513725}%
\pgfsetlinewidth{1.003750pt}%
\definecolor{currentstroke}{rgb}{0.121569,0.466667,0.705882}%
\pgfsetstrokecolor{currentstroke}%
\pgfsetstrokeopacity{0.513725}%
\pgfsetdash{}{0pt}%
\pgfpathmoveto{\pgfqpoint{3.084570in}{2.293267in}}%
\pgfpathcurveto{\pgfqpoint{3.092806in}{2.293267in}}{\pgfqpoint{3.100706in}{2.296539in}}{\pgfqpoint{3.106530in}{2.302363in}}%
\pgfpathcurveto{\pgfqpoint{3.112354in}{2.308187in}}{\pgfqpoint{3.115627in}{2.316087in}}{\pgfqpoint{3.115627in}{2.324324in}}%
\pgfpathcurveto{\pgfqpoint{3.115627in}{2.332560in}}{\pgfqpoint{3.112354in}{2.340460in}}{\pgfqpoint{3.106530in}{2.346284in}}%
\pgfpathcurveto{\pgfqpoint{3.100706in}{2.352108in}}{\pgfqpoint{3.092806in}{2.355380in}}{\pgfqpoint{3.084570in}{2.355380in}}%
\pgfpathcurveto{\pgfqpoint{3.076334in}{2.355380in}}{\pgfqpoint{3.068434in}{2.352108in}}{\pgfqpoint{3.062610in}{2.346284in}}%
\pgfpathcurveto{\pgfqpoint{3.056786in}{2.340460in}}{\pgfqpoint{3.053514in}{2.332560in}}{\pgfqpoint{3.053514in}{2.324324in}}%
\pgfpathcurveto{\pgfqpoint{3.053514in}{2.316087in}}{\pgfqpoint{3.056786in}{2.308187in}}{\pgfqpoint{3.062610in}{2.302363in}}%
\pgfpathcurveto{\pgfqpoint{3.068434in}{2.296539in}}{\pgfqpoint{3.076334in}{2.293267in}}{\pgfqpoint{3.084570in}{2.293267in}}%
\pgfpathclose%
\pgfusepath{stroke,fill}%
\end{pgfscope}%
\begin{pgfscope}%
\pgfpathrectangle{\pgfqpoint{0.100000in}{0.212622in}}{\pgfqpoint{3.696000in}{3.696000in}}%
\pgfusepath{clip}%
\pgfsetbuttcap%
\pgfsetroundjoin%
\definecolor{currentfill}{rgb}{0.121569,0.466667,0.705882}%
\pgfsetfillcolor{currentfill}%
\pgfsetfillopacity{0.514989}%
\pgfsetlinewidth{1.003750pt}%
\definecolor{currentstroke}{rgb}{0.121569,0.466667,0.705882}%
\pgfsetstrokecolor{currentstroke}%
\pgfsetstrokeopacity{0.514989}%
\pgfsetdash{}{0pt}%
\pgfpathmoveto{\pgfqpoint{1.039235in}{1.754879in}}%
\pgfpathcurveto{\pgfqpoint{1.047471in}{1.754879in}}{\pgfqpoint{1.055371in}{1.758152in}}{\pgfqpoint{1.061195in}{1.763975in}}%
\pgfpathcurveto{\pgfqpoint{1.067019in}{1.769799in}}{\pgfqpoint{1.070292in}{1.777699in}}{\pgfqpoint{1.070292in}{1.785936in}}%
\pgfpathcurveto{\pgfqpoint{1.070292in}{1.794172in}}{\pgfqpoint{1.067019in}{1.802072in}}{\pgfqpoint{1.061195in}{1.807896in}}%
\pgfpathcurveto{\pgfqpoint{1.055371in}{1.813720in}}{\pgfqpoint{1.047471in}{1.816992in}}{\pgfqpoint{1.039235in}{1.816992in}}%
\pgfpathcurveto{\pgfqpoint{1.030999in}{1.816992in}}{\pgfqpoint{1.023099in}{1.813720in}}{\pgfqpoint{1.017275in}{1.807896in}}%
\pgfpathcurveto{\pgfqpoint{1.011451in}{1.802072in}}{\pgfqpoint{1.008179in}{1.794172in}}{\pgfqpoint{1.008179in}{1.785936in}}%
\pgfpathcurveto{\pgfqpoint{1.008179in}{1.777699in}}{\pgfqpoint{1.011451in}{1.769799in}}{\pgfqpoint{1.017275in}{1.763975in}}%
\pgfpathcurveto{\pgfqpoint{1.023099in}{1.758152in}}{\pgfqpoint{1.030999in}{1.754879in}}{\pgfqpoint{1.039235in}{1.754879in}}%
\pgfpathclose%
\pgfusepath{stroke,fill}%
\end{pgfscope}%
\begin{pgfscope}%
\pgfpathrectangle{\pgfqpoint{0.100000in}{0.212622in}}{\pgfqpoint{3.696000in}{3.696000in}}%
\pgfusepath{clip}%
\pgfsetbuttcap%
\pgfsetroundjoin%
\definecolor{currentfill}{rgb}{0.121569,0.466667,0.705882}%
\pgfsetfillcolor{currentfill}%
\pgfsetfillopacity{0.515149}%
\pgfsetlinewidth{1.003750pt}%
\definecolor{currentstroke}{rgb}{0.121569,0.466667,0.705882}%
\pgfsetstrokecolor{currentstroke}%
\pgfsetstrokeopacity{0.515149}%
\pgfsetdash{}{0pt}%
\pgfpathmoveto{\pgfqpoint{3.095654in}{2.290087in}}%
\pgfpathcurveto{\pgfqpoint{3.103890in}{2.290087in}}{\pgfqpoint{3.111790in}{2.293359in}}{\pgfqpoint{3.117614in}{2.299183in}}%
\pgfpathcurveto{\pgfqpoint{3.123438in}{2.305007in}}{\pgfqpoint{3.126710in}{2.312907in}}{\pgfqpoint{3.126710in}{2.321143in}}%
\pgfpathcurveto{\pgfqpoint{3.126710in}{2.329379in}}{\pgfqpoint{3.123438in}{2.337279in}}{\pgfqpoint{3.117614in}{2.343103in}}%
\pgfpathcurveto{\pgfqpoint{3.111790in}{2.348927in}}{\pgfqpoint{3.103890in}{2.352200in}}{\pgfqpoint{3.095654in}{2.352200in}}%
\pgfpathcurveto{\pgfqpoint{3.087417in}{2.352200in}}{\pgfqpoint{3.079517in}{2.348927in}}{\pgfqpoint{3.073693in}{2.343103in}}%
\pgfpathcurveto{\pgfqpoint{3.067869in}{2.337279in}}{\pgfqpoint{3.064597in}{2.329379in}}{\pgfqpoint{3.064597in}{2.321143in}}%
\pgfpathcurveto{\pgfqpoint{3.064597in}{2.312907in}}{\pgfqpoint{3.067869in}{2.305007in}}{\pgfqpoint{3.073693in}{2.299183in}}%
\pgfpathcurveto{\pgfqpoint{3.079517in}{2.293359in}}{\pgfqpoint{3.087417in}{2.290087in}}{\pgfqpoint{3.095654in}{2.290087in}}%
\pgfpathclose%
\pgfusepath{stroke,fill}%
\end{pgfscope}%
\begin{pgfscope}%
\pgfpathrectangle{\pgfqpoint{0.100000in}{0.212622in}}{\pgfqpoint{3.696000in}{3.696000in}}%
\pgfusepath{clip}%
\pgfsetbuttcap%
\pgfsetroundjoin%
\definecolor{currentfill}{rgb}{0.121569,0.466667,0.705882}%
\pgfsetfillcolor{currentfill}%
\pgfsetfillopacity{0.516147}%
\pgfsetlinewidth{1.003750pt}%
\definecolor{currentstroke}{rgb}{0.121569,0.466667,0.705882}%
\pgfsetstrokecolor{currentstroke}%
\pgfsetstrokeopacity{0.516147}%
\pgfsetdash{}{0pt}%
\pgfpathmoveto{\pgfqpoint{1.035343in}{1.750854in}}%
\pgfpathcurveto{\pgfqpoint{1.043579in}{1.750854in}}{\pgfqpoint{1.051479in}{1.754127in}}{\pgfqpoint{1.057303in}{1.759951in}}%
\pgfpathcurveto{\pgfqpoint{1.063127in}{1.765774in}}{\pgfqpoint{1.066399in}{1.773675in}}{\pgfqpoint{1.066399in}{1.781911in}}%
\pgfpathcurveto{\pgfqpoint{1.066399in}{1.790147in}}{\pgfqpoint{1.063127in}{1.798047in}}{\pgfqpoint{1.057303in}{1.803871in}}%
\pgfpathcurveto{\pgfqpoint{1.051479in}{1.809695in}}{\pgfqpoint{1.043579in}{1.812967in}}{\pgfqpoint{1.035343in}{1.812967in}}%
\pgfpathcurveto{\pgfqpoint{1.027107in}{1.812967in}}{\pgfqpoint{1.019206in}{1.809695in}}{\pgfqpoint{1.013383in}{1.803871in}}%
\pgfpathcurveto{\pgfqpoint{1.007559in}{1.798047in}}{\pgfqpoint{1.004286in}{1.790147in}}{\pgfqpoint{1.004286in}{1.781911in}}%
\pgfpathcurveto{\pgfqpoint{1.004286in}{1.773675in}}{\pgfqpoint{1.007559in}{1.765774in}}{\pgfqpoint{1.013383in}{1.759951in}}%
\pgfpathcurveto{\pgfqpoint{1.019206in}{1.754127in}}{\pgfqpoint{1.027107in}{1.750854in}}{\pgfqpoint{1.035343in}{1.750854in}}%
\pgfpathclose%
\pgfusepath{stroke,fill}%
\end{pgfscope}%
\begin{pgfscope}%
\pgfpathrectangle{\pgfqpoint{0.100000in}{0.212622in}}{\pgfqpoint{3.696000in}{3.696000in}}%
\pgfusepath{clip}%
\pgfsetbuttcap%
\pgfsetroundjoin%
\definecolor{currentfill}{rgb}{0.121569,0.466667,0.705882}%
\pgfsetfillcolor{currentfill}%
\pgfsetfillopacity{0.516347}%
\pgfsetlinewidth{1.003750pt}%
\definecolor{currentstroke}{rgb}{0.121569,0.466667,0.705882}%
\pgfsetstrokecolor{currentstroke}%
\pgfsetstrokeopacity{0.516347}%
\pgfsetdash{}{0pt}%
\pgfpathmoveto{\pgfqpoint{3.107206in}{2.284814in}}%
\pgfpathcurveto{\pgfqpoint{3.115443in}{2.284814in}}{\pgfqpoint{3.123343in}{2.288087in}}{\pgfqpoint{3.129167in}{2.293911in}}%
\pgfpathcurveto{\pgfqpoint{3.134991in}{2.299735in}}{\pgfqpoint{3.138263in}{2.307635in}}{\pgfqpoint{3.138263in}{2.315871in}}%
\pgfpathcurveto{\pgfqpoint{3.138263in}{2.324107in}}{\pgfqpoint{3.134991in}{2.332007in}}{\pgfqpoint{3.129167in}{2.337831in}}%
\pgfpathcurveto{\pgfqpoint{3.123343in}{2.343655in}}{\pgfqpoint{3.115443in}{2.346927in}}{\pgfqpoint{3.107206in}{2.346927in}}%
\pgfpathcurveto{\pgfqpoint{3.098970in}{2.346927in}}{\pgfqpoint{3.091070in}{2.343655in}}{\pgfqpoint{3.085246in}{2.337831in}}%
\pgfpathcurveto{\pgfqpoint{3.079422in}{2.332007in}}{\pgfqpoint{3.076150in}{2.324107in}}{\pgfqpoint{3.076150in}{2.315871in}}%
\pgfpathcurveto{\pgfqpoint{3.076150in}{2.307635in}}{\pgfqpoint{3.079422in}{2.299735in}}{\pgfqpoint{3.085246in}{2.293911in}}%
\pgfpathcurveto{\pgfqpoint{3.091070in}{2.288087in}}{\pgfqpoint{3.098970in}{2.284814in}}{\pgfqpoint{3.107206in}{2.284814in}}%
\pgfpathclose%
\pgfusepath{stroke,fill}%
\end{pgfscope}%
\begin{pgfscope}%
\pgfpathrectangle{\pgfqpoint{0.100000in}{0.212622in}}{\pgfqpoint{3.696000in}{3.696000in}}%
\pgfusepath{clip}%
\pgfsetbuttcap%
\pgfsetroundjoin%
\definecolor{currentfill}{rgb}{0.121569,0.466667,0.705882}%
\pgfsetfillcolor{currentfill}%
\pgfsetfillopacity{0.517051}%
\pgfsetlinewidth{1.003750pt}%
\definecolor{currentstroke}{rgb}{0.121569,0.466667,0.705882}%
\pgfsetstrokecolor{currentstroke}%
\pgfsetstrokeopacity{0.517051}%
\pgfsetdash{}{0pt}%
\pgfpathmoveto{\pgfqpoint{3.113663in}{2.282533in}}%
\pgfpathcurveto{\pgfqpoint{3.121899in}{2.282533in}}{\pgfqpoint{3.129799in}{2.285806in}}{\pgfqpoint{3.135623in}{2.291630in}}%
\pgfpathcurveto{\pgfqpoint{3.141447in}{2.297453in}}{\pgfqpoint{3.144720in}{2.305353in}}{\pgfqpoint{3.144720in}{2.313590in}}%
\pgfpathcurveto{\pgfqpoint{3.144720in}{2.321826in}}{\pgfqpoint{3.141447in}{2.329726in}}{\pgfqpoint{3.135623in}{2.335550in}}%
\pgfpathcurveto{\pgfqpoint{3.129799in}{2.341374in}}{\pgfqpoint{3.121899in}{2.344646in}}{\pgfqpoint{3.113663in}{2.344646in}}%
\pgfpathcurveto{\pgfqpoint{3.105427in}{2.344646in}}{\pgfqpoint{3.097527in}{2.341374in}}{\pgfqpoint{3.091703in}{2.335550in}}%
\pgfpathcurveto{\pgfqpoint{3.085879in}{2.329726in}}{\pgfqpoint{3.082607in}{2.321826in}}{\pgfqpoint{3.082607in}{2.313590in}}%
\pgfpathcurveto{\pgfqpoint{3.082607in}{2.305353in}}{\pgfqpoint{3.085879in}{2.297453in}}{\pgfqpoint{3.091703in}{2.291630in}}%
\pgfpathcurveto{\pgfqpoint{3.097527in}{2.285806in}}{\pgfqpoint{3.105427in}{2.282533in}}{\pgfqpoint{3.113663in}{2.282533in}}%
\pgfpathclose%
\pgfusepath{stroke,fill}%
\end{pgfscope}%
\begin{pgfscope}%
\pgfpathrectangle{\pgfqpoint{0.100000in}{0.212622in}}{\pgfqpoint{3.696000in}{3.696000in}}%
\pgfusepath{clip}%
\pgfsetbuttcap%
\pgfsetroundjoin%
\definecolor{currentfill}{rgb}{0.121569,0.466667,0.705882}%
\pgfsetfillcolor{currentfill}%
\pgfsetfillopacity{0.517108}%
\pgfsetlinewidth{1.003750pt}%
\definecolor{currentstroke}{rgb}{0.121569,0.466667,0.705882}%
\pgfsetstrokecolor{currentstroke}%
\pgfsetstrokeopacity{0.517108}%
\pgfsetdash{}{0pt}%
\pgfpathmoveto{\pgfqpoint{1.032266in}{1.747816in}}%
\pgfpathcurveto{\pgfqpoint{1.040503in}{1.747816in}}{\pgfqpoint{1.048403in}{1.751088in}}{\pgfqpoint{1.054227in}{1.756912in}}%
\pgfpathcurveto{\pgfqpoint{1.060051in}{1.762736in}}{\pgfqpoint{1.063323in}{1.770636in}}{\pgfqpoint{1.063323in}{1.778872in}}%
\pgfpathcurveto{\pgfqpoint{1.063323in}{1.787109in}}{\pgfqpoint{1.060051in}{1.795009in}}{\pgfqpoint{1.054227in}{1.800833in}}%
\pgfpathcurveto{\pgfqpoint{1.048403in}{1.806657in}}{\pgfqpoint{1.040503in}{1.809929in}}{\pgfqpoint{1.032266in}{1.809929in}}%
\pgfpathcurveto{\pgfqpoint{1.024030in}{1.809929in}}{\pgfqpoint{1.016130in}{1.806657in}}{\pgfqpoint{1.010306in}{1.800833in}}%
\pgfpathcurveto{\pgfqpoint{1.004482in}{1.795009in}}{\pgfqpoint{1.001210in}{1.787109in}}{\pgfqpoint{1.001210in}{1.778872in}}%
\pgfpathcurveto{\pgfqpoint{1.001210in}{1.770636in}}{\pgfqpoint{1.004482in}{1.762736in}}{\pgfqpoint{1.010306in}{1.756912in}}%
\pgfpathcurveto{\pgfqpoint{1.016130in}{1.751088in}}{\pgfqpoint{1.024030in}{1.747816in}}{\pgfqpoint{1.032266in}{1.747816in}}%
\pgfpathclose%
\pgfusepath{stroke,fill}%
\end{pgfscope}%
\begin{pgfscope}%
\pgfpathrectangle{\pgfqpoint{0.100000in}{0.212622in}}{\pgfqpoint{3.696000in}{3.696000in}}%
\pgfusepath{clip}%
\pgfsetbuttcap%
\pgfsetroundjoin%
\definecolor{currentfill}{rgb}{0.121569,0.466667,0.705882}%
\pgfsetfillcolor{currentfill}%
\pgfsetfillopacity{0.517830}%
\pgfsetlinewidth{1.003750pt}%
\definecolor{currentstroke}{rgb}{0.121569,0.466667,0.705882}%
\pgfsetstrokecolor{currentstroke}%
\pgfsetstrokeopacity{0.517830}%
\pgfsetdash{}{0pt}%
\pgfpathmoveto{\pgfqpoint{1.029876in}{1.745460in}}%
\pgfpathcurveto{\pgfqpoint{1.038113in}{1.745460in}}{\pgfqpoint{1.046013in}{1.748732in}}{\pgfqpoint{1.051837in}{1.754556in}}%
\pgfpathcurveto{\pgfqpoint{1.057660in}{1.760380in}}{\pgfqpoint{1.060933in}{1.768280in}}{\pgfqpoint{1.060933in}{1.776517in}}%
\pgfpathcurveto{\pgfqpoint{1.060933in}{1.784753in}}{\pgfqpoint{1.057660in}{1.792653in}}{\pgfqpoint{1.051837in}{1.798477in}}%
\pgfpathcurveto{\pgfqpoint{1.046013in}{1.804301in}}{\pgfqpoint{1.038113in}{1.807573in}}{\pgfqpoint{1.029876in}{1.807573in}}%
\pgfpathcurveto{\pgfqpoint{1.021640in}{1.807573in}}{\pgfqpoint{1.013740in}{1.804301in}}{\pgfqpoint{1.007916in}{1.798477in}}%
\pgfpathcurveto{\pgfqpoint{1.002092in}{1.792653in}}{\pgfqpoint{0.998820in}{1.784753in}}{\pgfqpoint{0.998820in}{1.776517in}}%
\pgfpathcurveto{\pgfqpoint{0.998820in}{1.768280in}}{\pgfqpoint{1.002092in}{1.760380in}}{\pgfqpoint{1.007916in}{1.754556in}}%
\pgfpathcurveto{\pgfqpoint{1.013740in}{1.748732in}}{\pgfqpoint{1.021640in}{1.745460in}}{\pgfqpoint{1.029876in}{1.745460in}}%
\pgfpathclose%
\pgfusepath{stroke,fill}%
\end{pgfscope}%
\begin{pgfscope}%
\pgfpathrectangle{\pgfqpoint{0.100000in}{0.212622in}}{\pgfqpoint{3.696000in}{3.696000in}}%
\pgfusepath{clip}%
\pgfsetbuttcap%
\pgfsetroundjoin%
\definecolor{currentfill}{rgb}{0.121569,0.466667,0.705882}%
\pgfsetfillcolor{currentfill}%
\pgfsetfillopacity{0.517935}%
\pgfsetlinewidth{1.003750pt}%
\definecolor{currentstroke}{rgb}{0.121569,0.466667,0.705882}%
\pgfsetstrokecolor{currentstroke}%
\pgfsetstrokeopacity{0.517935}%
\pgfsetdash{}{0pt}%
\pgfpathmoveto{\pgfqpoint{3.121298in}{2.280362in}}%
\pgfpathcurveto{\pgfqpoint{3.129534in}{2.280362in}}{\pgfqpoint{3.137434in}{2.283634in}}{\pgfqpoint{3.143258in}{2.289458in}}%
\pgfpathcurveto{\pgfqpoint{3.149082in}{2.295282in}}{\pgfqpoint{3.152354in}{2.303182in}}{\pgfqpoint{3.152354in}{2.311419in}}%
\pgfpathcurveto{\pgfqpoint{3.152354in}{2.319655in}}{\pgfqpoint{3.149082in}{2.327555in}}{\pgfqpoint{3.143258in}{2.333379in}}%
\pgfpathcurveto{\pgfqpoint{3.137434in}{2.339203in}}{\pgfqpoint{3.129534in}{2.342475in}}{\pgfqpoint{3.121298in}{2.342475in}}%
\pgfpathcurveto{\pgfqpoint{3.113062in}{2.342475in}}{\pgfqpoint{3.105162in}{2.339203in}}{\pgfqpoint{3.099338in}{2.333379in}}%
\pgfpathcurveto{\pgfqpoint{3.093514in}{2.327555in}}{\pgfqpoint{3.090241in}{2.319655in}}{\pgfqpoint{3.090241in}{2.311419in}}%
\pgfpathcurveto{\pgfqpoint{3.090241in}{2.303182in}}{\pgfqpoint{3.093514in}{2.295282in}}{\pgfqpoint{3.099338in}{2.289458in}}%
\pgfpathcurveto{\pgfqpoint{3.105162in}{2.283634in}}{\pgfqpoint{3.113062in}{2.280362in}}{\pgfqpoint{3.121298in}{2.280362in}}%
\pgfpathclose%
\pgfusepath{stroke,fill}%
\end{pgfscope}%
\begin{pgfscope}%
\pgfpathrectangle{\pgfqpoint{0.100000in}{0.212622in}}{\pgfqpoint{3.696000in}{3.696000in}}%
\pgfusepath{clip}%
\pgfsetbuttcap%
\pgfsetroundjoin%
\definecolor{currentfill}{rgb}{0.121569,0.466667,0.705882}%
\pgfsetfillcolor{currentfill}%
\pgfsetfillopacity{0.518337}%
\pgfsetlinewidth{1.003750pt}%
\definecolor{currentstroke}{rgb}{0.121569,0.466667,0.705882}%
\pgfsetstrokecolor{currentstroke}%
\pgfsetstrokeopacity{0.518337}%
\pgfsetdash{}{0pt}%
\pgfpathmoveto{\pgfqpoint{1.028307in}{1.743972in}}%
\pgfpathcurveto{\pgfqpoint{1.036543in}{1.743972in}}{\pgfqpoint{1.044443in}{1.747245in}}{\pgfqpoint{1.050267in}{1.753069in}}%
\pgfpathcurveto{\pgfqpoint{1.056091in}{1.758893in}}{\pgfqpoint{1.059363in}{1.766793in}}{\pgfqpoint{1.059363in}{1.775029in}}%
\pgfpathcurveto{\pgfqpoint{1.059363in}{1.783265in}}{\pgfqpoint{1.056091in}{1.791165in}}{\pgfqpoint{1.050267in}{1.796989in}}%
\pgfpathcurveto{\pgfqpoint{1.044443in}{1.802813in}}{\pgfqpoint{1.036543in}{1.806085in}}{\pgfqpoint{1.028307in}{1.806085in}}%
\pgfpathcurveto{\pgfqpoint{1.020070in}{1.806085in}}{\pgfqpoint{1.012170in}{1.802813in}}{\pgfqpoint{1.006346in}{1.796989in}}%
\pgfpathcurveto{\pgfqpoint{1.000522in}{1.791165in}}{\pgfqpoint{0.997250in}{1.783265in}}{\pgfqpoint{0.997250in}{1.775029in}}%
\pgfpathcurveto{\pgfqpoint{0.997250in}{1.766793in}}{\pgfqpoint{1.000522in}{1.758893in}}{\pgfqpoint{1.006346in}{1.753069in}}%
\pgfpathcurveto{\pgfqpoint{1.012170in}{1.747245in}}{\pgfqpoint{1.020070in}{1.743972in}}{\pgfqpoint{1.028307in}{1.743972in}}%
\pgfpathclose%
\pgfusepath{stroke,fill}%
\end{pgfscope}%
\begin{pgfscope}%
\pgfpathrectangle{\pgfqpoint{0.100000in}{0.212622in}}{\pgfqpoint{3.696000in}{3.696000in}}%
\pgfusepath{clip}%
\pgfsetbuttcap%
\pgfsetroundjoin%
\definecolor{currentfill}{rgb}{0.121569,0.466667,0.705882}%
\pgfsetfillcolor{currentfill}%
\pgfsetfillopacity{0.519204}%
\pgfsetlinewidth{1.003750pt}%
\definecolor{currentstroke}{rgb}{0.121569,0.466667,0.705882}%
\pgfsetstrokecolor{currentstroke}%
\pgfsetstrokeopacity{0.519204}%
\pgfsetdash{}{0pt}%
\pgfpathmoveto{\pgfqpoint{3.130978in}{2.279183in}}%
\pgfpathcurveto{\pgfqpoint{3.139214in}{2.279183in}}{\pgfqpoint{3.147114in}{2.282456in}}{\pgfqpoint{3.152938in}{2.288280in}}%
\pgfpathcurveto{\pgfqpoint{3.158762in}{2.294104in}}{\pgfqpoint{3.162035in}{2.302004in}}{\pgfqpoint{3.162035in}{2.310240in}}%
\pgfpathcurveto{\pgfqpoint{3.162035in}{2.318476in}}{\pgfqpoint{3.158762in}{2.326376in}}{\pgfqpoint{3.152938in}{2.332200in}}%
\pgfpathcurveto{\pgfqpoint{3.147114in}{2.338024in}}{\pgfqpoint{3.139214in}{2.341296in}}{\pgfqpoint{3.130978in}{2.341296in}}%
\pgfpathcurveto{\pgfqpoint{3.122742in}{2.341296in}}{\pgfqpoint{3.114842in}{2.338024in}}{\pgfqpoint{3.109018in}{2.332200in}}%
\pgfpathcurveto{\pgfqpoint{3.103194in}{2.326376in}}{\pgfqpoint{3.099922in}{2.318476in}}{\pgfqpoint{3.099922in}{2.310240in}}%
\pgfpathcurveto{\pgfqpoint{3.099922in}{2.302004in}}{\pgfqpoint{3.103194in}{2.294104in}}{\pgfqpoint{3.109018in}{2.288280in}}%
\pgfpathcurveto{\pgfqpoint{3.114842in}{2.282456in}}{\pgfqpoint{3.122742in}{2.279183in}}{\pgfqpoint{3.130978in}{2.279183in}}%
\pgfpathclose%
\pgfusepath{stroke,fill}%
\end{pgfscope}%
\begin{pgfscope}%
\pgfpathrectangle{\pgfqpoint{0.100000in}{0.212622in}}{\pgfqpoint{3.696000in}{3.696000in}}%
\pgfusepath{clip}%
\pgfsetbuttcap%
\pgfsetroundjoin%
\definecolor{currentfill}{rgb}{0.121569,0.466667,0.705882}%
\pgfsetfillcolor{currentfill}%
\pgfsetfillopacity{0.519242}%
\pgfsetlinewidth{1.003750pt}%
\definecolor{currentstroke}{rgb}{0.121569,0.466667,0.705882}%
\pgfsetstrokecolor{currentstroke}%
\pgfsetstrokeopacity{0.519242}%
\pgfsetdash{}{0pt}%
\pgfpathmoveto{\pgfqpoint{1.025419in}{1.741206in}}%
\pgfpathcurveto{\pgfqpoint{1.033655in}{1.741206in}}{\pgfqpoint{1.041555in}{1.744478in}}{\pgfqpoint{1.047379in}{1.750302in}}%
\pgfpathcurveto{\pgfqpoint{1.053203in}{1.756126in}}{\pgfqpoint{1.056476in}{1.764026in}}{\pgfqpoint{1.056476in}{1.772262in}}%
\pgfpathcurveto{\pgfqpoint{1.056476in}{1.780498in}}{\pgfqpoint{1.053203in}{1.788398in}}{\pgfqpoint{1.047379in}{1.794222in}}%
\pgfpathcurveto{\pgfqpoint{1.041555in}{1.800046in}}{\pgfqpoint{1.033655in}{1.803319in}}{\pgfqpoint{1.025419in}{1.803319in}}%
\pgfpathcurveto{\pgfqpoint{1.017183in}{1.803319in}}{\pgfqpoint{1.009283in}{1.800046in}}{\pgfqpoint{1.003459in}{1.794222in}}%
\pgfpathcurveto{\pgfqpoint{0.997635in}{1.788398in}}{\pgfqpoint{0.994363in}{1.780498in}}{\pgfqpoint{0.994363in}{1.772262in}}%
\pgfpathcurveto{\pgfqpoint{0.994363in}{1.764026in}}{\pgfqpoint{0.997635in}{1.756126in}}{\pgfqpoint{1.003459in}{1.750302in}}%
\pgfpathcurveto{\pgfqpoint{1.009283in}{1.744478in}}{\pgfqpoint{1.017183in}{1.741206in}}{\pgfqpoint{1.025419in}{1.741206in}}%
\pgfpathclose%
\pgfusepath{stroke,fill}%
\end{pgfscope}%
\begin{pgfscope}%
\pgfpathrectangle{\pgfqpoint{0.100000in}{0.212622in}}{\pgfqpoint{3.696000in}{3.696000in}}%
\pgfusepath{clip}%
\pgfsetbuttcap%
\pgfsetroundjoin%
\definecolor{currentfill}{rgb}{0.121569,0.466667,0.705882}%
\pgfsetfillcolor{currentfill}%
\pgfsetfillopacity{0.519991}%
\pgfsetlinewidth{1.003750pt}%
\definecolor{currentstroke}{rgb}{0.121569,0.466667,0.705882}%
\pgfsetstrokecolor{currentstroke}%
\pgfsetstrokeopacity{0.519991}%
\pgfsetdash{}{0pt}%
\pgfpathmoveto{\pgfqpoint{1.023176in}{1.739166in}}%
\pgfpathcurveto{\pgfqpoint{1.031412in}{1.739166in}}{\pgfqpoint{1.039312in}{1.742438in}}{\pgfqpoint{1.045136in}{1.748262in}}%
\pgfpathcurveto{\pgfqpoint{1.050960in}{1.754086in}}{\pgfqpoint{1.054232in}{1.761986in}}{\pgfqpoint{1.054232in}{1.770222in}}%
\pgfpathcurveto{\pgfqpoint{1.054232in}{1.778458in}}{\pgfqpoint{1.050960in}{1.786358in}}{\pgfqpoint{1.045136in}{1.792182in}}%
\pgfpathcurveto{\pgfqpoint{1.039312in}{1.798006in}}{\pgfqpoint{1.031412in}{1.801279in}}{\pgfqpoint{1.023176in}{1.801279in}}%
\pgfpathcurveto{\pgfqpoint{1.014939in}{1.801279in}}{\pgfqpoint{1.007039in}{1.798006in}}{\pgfqpoint{1.001215in}{1.792182in}}%
\pgfpathcurveto{\pgfqpoint{0.995391in}{1.786358in}}{\pgfqpoint{0.992119in}{1.778458in}}{\pgfqpoint{0.992119in}{1.770222in}}%
\pgfpathcurveto{\pgfqpoint{0.992119in}{1.761986in}}{\pgfqpoint{0.995391in}{1.754086in}}{\pgfqpoint{1.001215in}{1.748262in}}%
\pgfpathcurveto{\pgfqpoint{1.007039in}{1.742438in}}{\pgfqpoint{1.014939in}{1.739166in}}{\pgfqpoint{1.023176in}{1.739166in}}%
\pgfpathclose%
\pgfusepath{stroke,fill}%
\end{pgfscope}%
\begin{pgfscope}%
\pgfpathrectangle{\pgfqpoint{0.100000in}{0.212622in}}{\pgfqpoint{3.696000in}{3.696000in}}%
\pgfusepath{clip}%
\pgfsetbuttcap%
\pgfsetroundjoin%
\definecolor{currentfill}{rgb}{0.121569,0.466667,0.705882}%
\pgfsetfillcolor{currentfill}%
\pgfsetfillopacity{0.520568}%
\pgfsetlinewidth{1.003750pt}%
\definecolor{currentstroke}{rgb}{0.121569,0.466667,0.705882}%
\pgfsetstrokecolor{currentstroke}%
\pgfsetstrokeopacity{0.520568}%
\pgfsetdash{}{0pt}%
\pgfpathmoveto{\pgfqpoint{1.021411in}{1.737363in}}%
\pgfpathcurveto{\pgfqpoint{1.029647in}{1.737363in}}{\pgfqpoint{1.037547in}{1.740636in}}{\pgfqpoint{1.043371in}{1.746459in}}%
\pgfpathcurveto{\pgfqpoint{1.049195in}{1.752283in}}{\pgfqpoint{1.052467in}{1.760183in}}{\pgfqpoint{1.052467in}{1.768420in}}%
\pgfpathcurveto{\pgfqpoint{1.052467in}{1.776656in}}{\pgfqpoint{1.049195in}{1.784556in}}{\pgfqpoint{1.043371in}{1.790380in}}%
\pgfpathcurveto{\pgfqpoint{1.037547in}{1.796204in}}{\pgfqpoint{1.029647in}{1.799476in}}{\pgfqpoint{1.021411in}{1.799476in}}%
\pgfpathcurveto{\pgfqpoint{1.013174in}{1.799476in}}{\pgfqpoint{1.005274in}{1.796204in}}{\pgfqpoint{0.999450in}{1.790380in}}%
\pgfpathcurveto{\pgfqpoint{0.993626in}{1.784556in}}{\pgfqpoint{0.990354in}{1.776656in}}{\pgfqpoint{0.990354in}{1.768420in}}%
\pgfpathcurveto{\pgfqpoint{0.990354in}{1.760183in}}{\pgfqpoint{0.993626in}{1.752283in}}{\pgfqpoint{0.999450in}{1.746459in}}%
\pgfpathcurveto{\pgfqpoint{1.005274in}{1.740636in}}{\pgfqpoint{1.013174in}{1.737363in}}{\pgfqpoint{1.021411in}{1.737363in}}%
\pgfpathclose%
\pgfusepath{stroke,fill}%
\end{pgfscope}%
\begin{pgfscope}%
\pgfpathrectangle{\pgfqpoint{0.100000in}{0.212622in}}{\pgfqpoint{3.696000in}{3.696000in}}%
\pgfusepath{clip}%
\pgfsetbuttcap%
\pgfsetroundjoin%
\definecolor{currentfill}{rgb}{0.121569,0.466667,0.705882}%
\pgfsetfillcolor{currentfill}%
\pgfsetfillopacity{0.520668}%
\pgfsetlinewidth{1.003750pt}%
\definecolor{currentstroke}{rgb}{0.121569,0.466667,0.705882}%
\pgfsetstrokecolor{currentstroke}%
\pgfsetstrokeopacity{0.520668}%
\pgfsetdash{}{0pt}%
\pgfpathmoveto{\pgfqpoint{3.141720in}{2.277928in}}%
\pgfpathcurveto{\pgfqpoint{3.149956in}{2.277928in}}{\pgfqpoint{3.157856in}{2.281200in}}{\pgfqpoint{3.163680in}{2.287024in}}%
\pgfpathcurveto{\pgfqpoint{3.169504in}{2.292848in}}{\pgfqpoint{3.172777in}{2.300748in}}{\pgfqpoint{3.172777in}{2.308984in}}%
\pgfpathcurveto{\pgfqpoint{3.172777in}{2.317221in}}{\pgfqpoint{3.169504in}{2.325121in}}{\pgfqpoint{3.163680in}{2.330945in}}%
\pgfpathcurveto{\pgfqpoint{3.157856in}{2.336769in}}{\pgfqpoint{3.149956in}{2.340041in}}{\pgfqpoint{3.141720in}{2.340041in}}%
\pgfpathcurveto{\pgfqpoint{3.133484in}{2.340041in}}{\pgfqpoint{3.125584in}{2.336769in}}{\pgfqpoint{3.119760in}{2.330945in}}%
\pgfpathcurveto{\pgfqpoint{3.113936in}{2.325121in}}{\pgfqpoint{3.110664in}{2.317221in}}{\pgfqpoint{3.110664in}{2.308984in}}%
\pgfpathcurveto{\pgfqpoint{3.110664in}{2.300748in}}{\pgfqpoint{3.113936in}{2.292848in}}{\pgfqpoint{3.119760in}{2.287024in}}%
\pgfpathcurveto{\pgfqpoint{3.125584in}{2.281200in}}{\pgfqpoint{3.133484in}{2.277928in}}{\pgfqpoint{3.141720in}{2.277928in}}%
\pgfpathclose%
\pgfusepath{stroke,fill}%
\end{pgfscope}%
\begin{pgfscope}%
\pgfpathrectangle{\pgfqpoint{0.100000in}{0.212622in}}{\pgfqpoint{3.696000in}{3.696000in}}%
\pgfusepath{clip}%
\pgfsetbuttcap%
\pgfsetroundjoin%
\definecolor{currentfill}{rgb}{0.121569,0.466667,0.705882}%
\pgfsetfillcolor{currentfill}%
\pgfsetfillopacity{0.521662}%
\pgfsetlinewidth{1.003750pt}%
\definecolor{currentstroke}{rgb}{0.121569,0.466667,0.705882}%
\pgfsetstrokecolor{currentstroke}%
\pgfsetstrokeopacity{0.521662}%
\pgfsetdash{}{0pt}%
\pgfpathmoveto{\pgfqpoint{1.018260in}{1.734294in}}%
\pgfpathcurveto{\pgfqpoint{1.026496in}{1.734294in}}{\pgfqpoint{1.034396in}{1.737566in}}{\pgfqpoint{1.040220in}{1.743390in}}%
\pgfpathcurveto{\pgfqpoint{1.046044in}{1.749214in}}{\pgfqpoint{1.049316in}{1.757114in}}{\pgfqpoint{1.049316in}{1.765350in}}%
\pgfpathcurveto{\pgfqpoint{1.049316in}{1.773586in}}{\pgfqpoint{1.046044in}{1.781487in}}{\pgfqpoint{1.040220in}{1.787310in}}%
\pgfpathcurveto{\pgfqpoint{1.034396in}{1.793134in}}{\pgfqpoint{1.026496in}{1.796407in}}{\pgfqpoint{1.018260in}{1.796407in}}%
\pgfpathcurveto{\pgfqpoint{1.010023in}{1.796407in}}{\pgfqpoint{1.002123in}{1.793134in}}{\pgfqpoint{0.996299in}{1.787310in}}%
\pgfpathcurveto{\pgfqpoint{0.990475in}{1.781487in}}{\pgfqpoint{0.987203in}{1.773586in}}{\pgfqpoint{0.987203in}{1.765350in}}%
\pgfpathcurveto{\pgfqpoint{0.987203in}{1.757114in}}{\pgfqpoint{0.990475in}{1.749214in}}{\pgfqpoint{0.996299in}{1.743390in}}%
\pgfpathcurveto{\pgfqpoint{1.002123in}{1.737566in}}{\pgfqpoint{1.010023in}{1.734294in}}{\pgfqpoint{1.018260in}{1.734294in}}%
\pgfpathclose%
\pgfusepath{stroke,fill}%
\end{pgfscope}%
\begin{pgfscope}%
\pgfpathrectangle{\pgfqpoint{0.100000in}{0.212622in}}{\pgfqpoint{3.696000in}{3.696000in}}%
\pgfusepath{clip}%
\pgfsetbuttcap%
\pgfsetroundjoin%
\definecolor{currentfill}{rgb}{0.121569,0.466667,0.705882}%
\pgfsetfillcolor{currentfill}%
\pgfsetfillopacity{0.522217}%
\pgfsetlinewidth{1.003750pt}%
\definecolor{currentstroke}{rgb}{0.121569,0.466667,0.705882}%
\pgfsetstrokecolor{currentstroke}%
\pgfsetstrokeopacity{0.522217}%
\pgfsetdash{}{0pt}%
\pgfpathmoveto{\pgfqpoint{3.153310in}{2.276190in}}%
\pgfpathcurveto{\pgfqpoint{3.161547in}{2.276190in}}{\pgfqpoint{3.169447in}{2.279462in}}{\pgfqpoint{3.175271in}{2.285286in}}%
\pgfpathcurveto{\pgfqpoint{3.181094in}{2.291110in}}{\pgfqpoint{3.184367in}{2.299010in}}{\pgfqpoint{3.184367in}{2.307246in}}%
\pgfpathcurveto{\pgfqpoint{3.184367in}{2.315482in}}{\pgfqpoint{3.181094in}{2.323382in}}{\pgfqpoint{3.175271in}{2.329206in}}%
\pgfpathcurveto{\pgfqpoint{3.169447in}{2.335030in}}{\pgfqpoint{3.161547in}{2.338303in}}{\pgfqpoint{3.153310in}{2.338303in}}%
\pgfpathcurveto{\pgfqpoint{3.145074in}{2.338303in}}{\pgfqpoint{3.137174in}{2.335030in}}{\pgfqpoint{3.131350in}{2.329206in}}%
\pgfpathcurveto{\pgfqpoint{3.125526in}{2.323382in}}{\pgfqpoint{3.122254in}{2.315482in}}{\pgfqpoint{3.122254in}{2.307246in}}%
\pgfpathcurveto{\pgfqpoint{3.122254in}{2.299010in}}{\pgfqpoint{3.125526in}{2.291110in}}{\pgfqpoint{3.131350in}{2.285286in}}%
\pgfpathcurveto{\pgfqpoint{3.137174in}{2.279462in}}{\pgfqpoint{3.145074in}{2.276190in}}{\pgfqpoint{3.153310in}{2.276190in}}%
\pgfpathclose%
\pgfusepath{stroke,fill}%
\end{pgfscope}%
\begin{pgfscope}%
\pgfpathrectangle{\pgfqpoint{0.100000in}{0.212622in}}{\pgfqpoint{3.696000in}{3.696000in}}%
\pgfusepath{clip}%
\pgfsetbuttcap%
\pgfsetroundjoin%
\definecolor{currentfill}{rgb}{0.121569,0.466667,0.705882}%
\pgfsetfillcolor{currentfill}%
\pgfsetfillopacity{0.523494}%
\pgfsetlinewidth{1.003750pt}%
\definecolor{currentstroke}{rgb}{0.121569,0.466667,0.705882}%
\pgfsetstrokecolor{currentstroke}%
\pgfsetstrokeopacity{0.523494}%
\pgfsetdash{}{0pt}%
\pgfpathmoveto{\pgfqpoint{1.012288in}{1.728000in}}%
\pgfpathcurveto{\pgfqpoint{1.020524in}{1.728000in}}{\pgfqpoint{1.028424in}{1.731272in}}{\pgfqpoint{1.034248in}{1.737096in}}%
\pgfpathcurveto{\pgfqpoint{1.040072in}{1.742920in}}{\pgfqpoint{1.043344in}{1.750820in}}{\pgfqpoint{1.043344in}{1.759057in}}%
\pgfpathcurveto{\pgfqpoint{1.043344in}{1.767293in}}{\pgfqpoint{1.040072in}{1.775193in}}{\pgfqpoint{1.034248in}{1.781017in}}%
\pgfpathcurveto{\pgfqpoint{1.028424in}{1.786841in}}{\pgfqpoint{1.020524in}{1.790113in}}{\pgfqpoint{1.012288in}{1.790113in}}%
\pgfpathcurveto{\pgfqpoint{1.004051in}{1.790113in}}{\pgfqpoint{0.996151in}{1.786841in}}{\pgfqpoint{0.990327in}{1.781017in}}%
\pgfpathcurveto{\pgfqpoint{0.984504in}{1.775193in}}{\pgfqpoint{0.981231in}{1.767293in}}{\pgfqpoint{0.981231in}{1.759057in}}%
\pgfpathcurveto{\pgfqpoint{0.981231in}{1.750820in}}{\pgfqpoint{0.984504in}{1.742920in}}{\pgfqpoint{0.990327in}{1.737096in}}%
\pgfpathcurveto{\pgfqpoint{0.996151in}{1.731272in}}{\pgfqpoint{1.004051in}{1.728000in}}{\pgfqpoint{1.012288in}{1.728000in}}%
\pgfpathclose%
\pgfusepath{stroke,fill}%
\end{pgfscope}%
\begin{pgfscope}%
\pgfpathrectangle{\pgfqpoint{0.100000in}{0.212622in}}{\pgfqpoint{3.696000in}{3.696000in}}%
\pgfusepath{clip}%
\pgfsetbuttcap%
\pgfsetroundjoin%
\definecolor{currentfill}{rgb}{0.121569,0.466667,0.705882}%
\pgfsetfillcolor{currentfill}%
\pgfsetfillopacity{0.523906}%
\pgfsetlinewidth{1.003750pt}%
\definecolor{currentstroke}{rgb}{0.121569,0.466667,0.705882}%
\pgfsetstrokecolor{currentstroke}%
\pgfsetstrokeopacity{0.523906}%
\pgfsetdash{}{0pt}%
\pgfpathmoveto{\pgfqpoint{3.165434in}{2.275062in}}%
\pgfpathcurveto{\pgfqpoint{3.173671in}{2.275062in}}{\pgfqpoint{3.181571in}{2.278335in}}{\pgfqpoint{3.187395in}{2.284159in}}%
\pgfpathcurveto{\pgfqpoint{3.193218in}{2.289983in}}{\pgfqpoint{3.196491in}{2.297883in}}{\pgfqpoint{3.196491in}{2.306119in}}%
\pgfpathcurveto{\pgfqpoint{3.196491in}{2.314355in}}{\pgfqpoint{3.193218in}{2.322255in}}{\pgfqpoint{3.187395in}{2.328079in}}%
\pgfpathcurveto{\pgfqpoint{3.181571in}{2.333903in}}{\pgfqpoint{3.173671in}{2.337175in}}{\pgfqpoint{3.165434in}{2.337175in}}%
\pgfpathcurveto{\pgfqpoint{3.157198in}{2.337175in}}{\pgfqpoint{3.149298in}{2.333903in}}{\pgfqpoint{3.143474in}{2.328079in}}%
\pgfpathcurveto{\pgfqpoint{3.137650in}{2.322255in}}{\pgfqpoint{3.134378in}{2.314355in}}{\pgfqpoint{3.134378in}{2.306119in}}%
\pgfpathcurveto{\pgfqpoint{3.134378in}{2.297883in}}{\pgfqpoint{3.137650in}{2.289983in}}{\pgfqpoint{3.143474in}{2.284159in}}%
\pgfpathcurveto{\pgfqpoint{3.149298in}{2.278335in}}{\pgfqpoint{3.157198in}{2.275062in}}{\pgfqpoint{3.165434in}{2.275062in}}%
\pgfpathclose%
\pgfusepath{stroke,fill}%
\end{pgfscope}%
\begin{pgfscope}%
\pgfpathrectangle{\pgfqpoint{0.100000in}{0.212622in}}{\pgfqpoint{3.696000in}{3.696000in}}%
\pgfusepath{clip}%
\pgfsetbuttcap%
\pgfsetroundjoin%
\definecolor{currentfill}{rgb}{0.121569,0.466667,0.705882}%
\pgfsetfillcolor{currentfill}%
\pgfsetfillopacity{0.525016}%
\pgfsetlinewidth{1.003750pt}%
\definecolor{currentstroke}{rgb}{0.121569,0.466667,0.705882}%
\pgfsetstrokecolor{currentstroke}%
\pgfsetstrokeopacity{0.525016}%
\pgfsetdash{}{0pt}%
\pgfpathmoveto{\pgfqpoint{1.007127in}{1.723155in}}%
\pgfpathcurveto{\pgfqpoint{1.015363in}{1.723155in}}{\pgfqpoint{1.023263in}{1.726427in}}{\pgfqpoint{1.029087in}{1.732251in}}%
\pgfpathcurveto{\pgfqpoint{1.034911in}{1.738075in}}{\pgfqpoint{1.038183in}{1.745975in}}{\pgfqpoint{1.038183in}{1.754211in}}%
\pgfpathcurveto{\pgfqpoint{1.038183in}{1.762448in}}{\pgfqpoint{1.034911in}{1.770348in}}{\pgfqpoint{1.029087in}{1.776171in}}%
\pgfpathcurveto{\pgfqpoint{1.023263in}{1.781995in}}{\pgfqpoint{1.015363in}{1.785268in}}{\pgfqpoint{1.007127in}{1.785268in}}%
\pgfpathcurveto{\pgfqpoint{0.998891in}{1.785268in}}{\pgfqpoint{0.990991in}{1.781995in}}{\pgfqpoint{0.985167in}{1.776171in}}%
\pgfpathcurveto{\pgfqpoint{0.979343in}{1.770348in}}{\pgfqpoint{0.976070in}{1.762448in}}{\pgfqpoint{0.976070in}{1.754211in}}%
\pgfpathcurveto{\pgfqpoint{0.976070in}{1.745975in}}{\pgfqpoint{0.979343in}{1.738075in}}{\pgfqpoint{0.985167in}{1.732251in}}%
\pgfpathcurveto{\pgfqpoint{0.990991in}{1.726427in}}{\pgfqpoint{0.998891in}{1.723155in}}{\pgfqpoint{1.007127in}{1.723155in}}%
\pgfpathclose%
\pgfusepath{stroke,fill}%
\end{pgfscope}%
\begin{pgfscope}%
\pgfpathrectangle{\pgfqpoint{0.100000in}{0.212622in}}{\pgfqpoint{3.696000in}{3.696000in}}%
\pgfusepath{clip}%
\pgfsetbuttcap%
\pgfsetroundjoin%
\definecolor{currentfill}{rgb}{0.121569,0.466667,0.705882}%
\pgfsetfillcolor{currentfill}%
\pgfsetfillopacity{0.525780}%
\pgfsetlinewidth{1.003750pt}%
\definecolor{currentstroke}{rgb}{0.121569,0.466667,0.705882}%
\pgfsetstrokecolor{currentstroke}%
\pgfsetstrokeopacity{0.525780}%
\pgfsetdash{}{0pt}%
\pgfpathmoveto{\pgfqpoint{3.178213in}{2.272724in}}%
\pgfpathcurveto{\pgfqpoint{3.186450in}{2.272724in}}{\pgfqpoint{3.194350in}{2.275996in}}{\pgfqpoint{3.200174in}{2.281820in}}%
\pgfpathcurveto{\pgfqpoint{3.205997in}{2.287644in}}{\pgfqpoint{3.209270in}{2.295544in}}{\pgfqpoint{3.209270in}{2.303781in}}%
\pgfpathcurveto{\pgfqpoint{3.209270in}{2.312017in}}{\pgfqpoint{3.205997in}{2.319917in}}{\pgfqpoint{3.200174in}{2.325741in}}%
\pgfpathcurveto{\pgfqpoint{3.194350in}{2.331565in}}{\pgfqpoint{3.186450in}{2.334837in}}{\pgfqpoint{3.178213in}{2.334837in}}%
\pgfpathcurveto{\pgfqpoint{3.169977in}{2.334837in}}{\pgfqpoint{3.162077in}{2.331565in}}{\pgfqpoint{3.156253in}{2.325741in}}%
\pgfpathcurveto{\pgfqpoint{3.150429in}{2.319917in}}{\pgfqpoint{3.147157in}{2.312017in}}{\pgfqpoint{3.147157in}{2.303781in}}%
\pgfpathcurveto{\pgfqpoint{3.147157in}{2.295544in}}{\pgfqpoint{3.150429in}{2.287644in}}{\pgfqpoint{3.156253in}{2.281820in}}%
\pgfpathcurveto{\pgfqpoint{3.162077in}{2.275996in}}{\pgfqpoint{3.169977in}{2.272724in}}{\pgfqpoint{3.178213in}{2.272724in}}%
\pgfpathclose%
\pgfusepath{stroke,fill}%
\end{pgfscope}%
\begin{pgfscope}%
\pgfpathrectangle{\pgfqpoint{0.100000in}{0.212622in}}{\pgfqpoint{3.696000in}{3.696000in}}%
\pgfusepath{clip}%
\pgfsetbuttcap%
\pgfsetroundjoin%
\definecolor{currentfill}{rgb}{0.121569,0.466667,0.705882}%
\pgfsetfillcolor{currentfill}%
\pgfsetfillopacity{0.527765}%
\pgfsetlinewidth{1.003750pt}%
\definecolor{currentstroke}{rgb}{0.121569,0.466667,0.705882}%
\pgfsetstrokecolor{currentstroke}%
\pgfsetstrokeopacity{0.527765}%
\pgfsetdash{}{0pt}%
\pgfpathmoveto{\pgfqpoint{0.998253in}{1.713557in}}%
\pgfpathcurveto{\pgfqpoint{1.006489in}{1.713557in}}{\pgfqpoint{1.014389in}{1.716829in}}{\pgfqpoint{1.020213in}{1.722653in}}%
\pgfpathcurveto{\pgfqpoint{1.026037in}{1.728477in}}{\pgfqpoint{1.029309in}{1.736377in}}{\pgfqpoint{1.029309in}{1.744614in}}%
\pgfpathcurveto{\pgfqpoint{1.029309in}{1.752850in}}{\pgfqpoint{1.026037in}{1.760750in}}{\pgfqpoint{1.020213in}{1.766574in}}%
\pgfpathcurveto{\pgfqpoint{1.014389in}{1.772398in}}{\pgfqpoint{1.006489in}{1.775670in}}{\pgfqpoint{0.998253in}{1.775670in}}%
\pgfpathcurveto{\pgfqpoint{0.990016in}{1.775670in}}{\pgfqpoint{0.982116in}{1.772398in}}{\pgfqpoint{0.976292in}{1.766574in}}%
\pgfpathcurveto{\pgfqpoint{0.970468in}{1.760750in}}{\pgfqpoint{0.967196in}{1.752850in}}{\pgfqpoint{0.967196in}{1.744614in}}%
\pgfpathcurveto{\pgfqpoint{0.967196in}{1.736377in}}{\pgfqpoint{0.970468in}{1.728477in}}{\pgfqpoint{0.976292in}{1.722653in}}%
\pgfpathcurveto{\pgfqpoint{0.982116in}{1.716829in}}{\pgfqpoint{0.990016in}{1.713557in}}{\pgfqpoint{0.998253in}{1.713557in}}%
\pgfpathclose%
\pgfusepath{stroke,fill}%
\end{pgfscope}%
\begin{pgfscope}%
\pgfpathrectangle{\pgfqpoint{0.100000in}{0.212622in}}{\pgfqpoint{3.696000in}{3.696000in}}%
\pgfusepath{clip}%
\pgfsetbuttcap%
\pgfsetroundjoin%
\definecolor{currentfill}{rgb}{0.121569,0.466667,0.705882}%
\pgfsetfillcolor{currentfill}%
\pgfsetfillopacity{0.527884}%
\pgfsetlinewidth{1.003750pt}%
\definecolor{currentstroke}{rgb}{0.121569,0.466667,0.705882}%
\pgfsetstrokecolor{currentstroke}%
\pgfsetstrokeopacity{0.527884}%
\pgfsetdash{}{0pt}%
\pgfpathmoveto{\pgfqpoint{3.192363in}{2.270862in}}%
\pgfpathcurveto{\pgfqpoint{3.200599in}{2.270862in}}{\pgfqpoint{3.208500in}{2.274134in}}{\pgfqpoint{3.214323in}{2.279958in}}%
\pgfpathcurveto{\pgfqpoint{3.220147in}{2.285782in}}{\pgfqpoint{3.223420in}{2.293682in}}{\pgfqpoint{3.223420in}{2.301918in}}%
\pgfpathcurveto{\pgfqpoint{3.223420in}{2.310154in}}{\pgfqpoint{3.220147in}{2.318054in}}{\pgfqpoint{3.214323in}{2.323878in}}%
\pgfpathcurveto{\pgfqpoint{3.208500in}{2.329702in}}{\pgfqpoint{3.200599in}{2.332975in}}{\pgfqpoint{3.192363in}{2.332975in}}%
\pgfpathcurveto{\pgfqpoint{3.184127in}{2.332975in}}{\pgfqpoint{3.176227in}{2.329702in}}{\pgfqpoint{3.170403in}{2.323878in}}%
\pgfpathcurveto{\pgfqpoint{3.164579in}{2.318054in}}{\pgfqpoint{3.161307in}{2.310154in}}{\pgfqpoint{3.161307in}{2.301918in}}%
\pgfpathcurveto{\pgfqpoint{3.161307in}{2.293682in}}{\pgfqpoint{3.164579in}{2.285782in}}{\pgfqpoint{3.170403in}{2.279958in}}%
\pgfpathcurveto{\pgfqpoint{3.176227in}{2.274134in}}{\pgfqpoint{3.184127in}{2.270862in}}{\pgfqpoint{3.192363in}{2.270862in}}%
\pgfpathclose%
\pgfusepath{stroke,fill}%
\end{pgfscope}%
\begin{pgfscope}%
\pgfpathrectangle{\pgfqpoint{0.100000in}{0.212622in}}{\pgfqpoint{3.696000in}{3.696000in}}%
\pgfusepath{clip}%
\pgfsetbuttcap%
\pgfsetroundjoin%
\definecolor{currentfill}{rgb}{0.121569,0.466667,0.705882}%
\pgfsetfillcolor{currentfill}%
\pgfsetfillopacity{0.529903}%
\pgfsetlinewidth{1.003750pt}%
\definecolor{currentstroke}{rgb}{0.121569,0.466667,0.705882}%
\pgfsetstrokecolor{currentstroke}%
\pgfsetstrokeopacity{0.529903}%
\pgfsetdash{}{0pt}%
\pgfpathmoveto{\pgfqpoint{3.206885in}{2.267275in}}%
\pgfpathcurveto{\pgfqpoint{3.215121in}{2.267275in}}{\pgfqpoint{3.223021in}{2.270547in}}{\pgfqpoint{3.228845in}{2.276371in}}%
\pgfpathcurveto{\pgfqpoint{3.234669in}{2.282195in}}{\pgfqpoint{3.237941in}{2.290095in}}{\pgfqpoint{3.237941in}{2.298332in}}%
\pgfpathcurveto{\pgfqpoint{3.237941in}{2.306568in}}{\pgfqpoint{3.234669in}{2.314468in}}{\pgfqpoint{3.228845in}{2.320292in}}%
\pgfpathcurveto{\pgfqpoint{3.223021in}{2.326116in}}{\pgfqpoint{3.215121in}{2.329388in}}{\pgfqpoint{3.206885in}{2.329388in}}%
\pgfpathcurveto{\pgfqpoint{3.198648in}{2.329388in}}{\pgfqpoint{3.190748in}{2.326116in}}{\pgfqpoint{3.184924in}{2.320292in}}%
\pgfpathcurveto{\pgfqpoint{3.179101in}{2.314468in}}{\pgfqpoint{3.175828in}{2.306568in}}{\pgfqpoint{3.175828in}{2.298332in}}%
\pgfpathcurveto{\pgfqpoint{3.175828in}{2.290095in}}{\pgfqpoint{3.179101in}{2.282195in}}{\pgfqpoint{3.184924in}{2.276371in}}%
\pgfpathcurveto{\pgfqpoint{3.190748in}{2.270547in}}{\pgfqpoint{3.198648in}{2.267275in}}{\pgfqpoint{3.206885in}{2.267275in}}%
\pgfpathclose%
\pgfusepath{stroke,fill}%
\end{pgfscope}%
\begin{pgfscope}%
\pgfpathrectangle{\pgfqpoint{0.100000in}{0.212622in}}{\pgfqpoint{3.696000in}{3.696000in}}%
\pgfusepath{clip}%
\pgfsetbuttcap%
\pgfsetroundjoin%
\definecolor{currentfill}{rgb}{0.121569,0.466667,0.705882}%
\pgfsetfillcolor{currentfill}%
\pgfsetfillopacity{0.529969}%
\pgfsetlinewidth{1.003750pt}%
\definecolor{currentstroke}{rgb}{0.121569,0.466667,0.705882}%
\pgfsetstrokecolor{currentstroke}%
\pgfsetstrokeopacity{0.529969}%
\pgfsetdash{}{0pt}%
\pgfpathmoveto{\pgfqpoint{0.991102in}{1.705811in}}%
\pgfpathcurveto{\pgfqpoint{0.999338in}{1.705811in}}{\pgfqpoint{1.007238in}{1.709083in}}{\pgfqpoint{1.013062in}{1.714907in}}%
\pgfpathcurveto{\pgfqpoint{1.018886in}{1.720731in}}{\pgfqpoint{1.022158in}{1.728631in}}{\pgfqpoint{1.022158in}{1.736867in}}%
\pgfpathcurveto{\pgfqpoint{1.022158in}{1.745104in}}{\pgfqpoint{1.018886in}{1.753004in}}{\pgfqpoint{1.013062in}{1.758828in}}%
\pgfpathcurveto{\pgfqpoint{1.007238in}{1.764652in}}{\pgfqpoint{0.999338in}{1.767924in}}{\pgfqpoint{0.991102in}{1.767924in}}%
\pgfpathcurveto{\pgfqpoint{0.982866in}{1.767924in}}{\pgfqpoint{0.974966in}{1.764652in}}{\pgfqpoint{0.969142in}{1.758828in}}%
\pgfpathcurveto{\pgfqpoint{0.963318in}{1.753004in}}{\pgfqpoint{0.960045in}{1.745104in}}{\pgfqpoint{0.960045in}{1.736867in}}%
\pgfpathcurveto{\pgfqpoint{0.960045in}{1.728631in}}{\pgfqpoint{0.963318in}{1.720731in}}{\pgfqpoint{0.969142in}{1.714907in}}%
\pgfpathcurveto{\pgfqpoint{0.974966in}{1.709083in}}{\pgfqpoint{0.982866in}{1.705811in}}{\pgfqpoint{0.991102in}{1.705811in}}%
\pgfpathclose%
\pgfusepath{stroke,fill}%
\end{pgfscope}%
\begin{pgfscope}%
\pgfpathrectangle{\pgfqpoint{0.100000in}{0.212622in}}{\pgfqpoint{3.696000in}{3.696000in}}%
\pgfusepath{clip}%
\pgfsetbuttcap%
\pgfsetroundjoin%
\definecolor{currentfill}{rgb}{0.121569,0.466667,0.705882}%
\pgfsetfillcolor{currentfill}%
\pgfsetfillopacity{0.532051}%
\pgfsetlinewidth{1.003750pt}%
\definecolor{currentstroke}{rgb}{0.121569,0.466667,0.705882}%
\pgfsetstrokecolor{currentstroke}%
\pgfsetstrokeopacity{0.532051}%
\pgfsetdash{}{0pt}%
\pgfpathmoveto{\pgfqpoint{0.984734in}{1.698249in}}%
\pgfpathcurveto{\pgfqpoint{0.992970in}{1.698249in}}{\pgfqpoint{1.000870in}{1.701521in}}{\pgfqpoint{1.006694in}{1.707345in}}%
\pgfpathcurveto{\pgfqpoint{1.012518in}{1.713169in}}{\pgfqpoint{1.015790in}{1.721069in}}{\pgfqpoint{1.015790in}{1.729305in}}%
\pgfpathcurveto{\pgfqpoint{1.015790in}{1.737542in}}{\pgfqpoint{1.012518in}{1.745442in}}{\pgfqpoint{1.006694in}{1.751266in}}%
\pgfpathcurveto{\pgfqpoint{1.000870in}{1.757090in}}{\pgfqpoint{0.992970in}{1.760362in}}{\pgfqpoint{0.984734in}{1.760362in}}%
\pgfpathcurveto{\pgfqpoint{0.976497in}{1.760362in}}{\pgfqpoint{0.968597in}{1.757090in}}{\pgfqpoint{0.962773in}{1.751266in}}%
\pgfpathcurveto{\pgfqpoint{0.956950in}{1.745442in}}{\pgfqpoint{0.953677in}{1.737542in}}{\pgfqpoint{0.953677in}{1.729305in}}%
\pgfpathcurveto{\pgfqpoint{0.953677in}{1.721069in}}{\pgfqpoint{0.956950in}{1.713169in}}{\pgfqpoint{0.962773in}{1.707345in}}%
\pgfpathcurveto{\pgfqpoint{0.968597in}{1.701521in}}{\pgfqpoint{0.976497in}{1.698249in}}{\pgfqpoint{0.984734in}{1.698249in}}%
\pgfpathclose%
\pgfusepath{stroke,fill}%
\end{pgfscope}%
\begin{pgfscope}%
\pgfpathrectangle{\pgfqpoint{0.100000in}{0.212622in}}{\pgfqpoint{3.696000in}{3.696000in}}%
\pgfusepath{clip}%
\pgfsetbuttcap%
\pgfsetroundjoin%
\definecolor{currentfill}{rgb}{0.121569,0.466667,0.705882}%
\pgfsetfillcolor{currentfill}%
\pgfsetfillopacity{0.532376}%
\pgfsetlinewidth{1.003750pt}%
\definecolor{currentstroke}{rgb}{0.121569,0.466667,0.705882}%
\pgfsetstrokecolor{currentstroke}%
\pgfsetstrokeopacity{0.532376}%
\pgfsetdash{}{0pt}%
\pgfpathmoveto{\pgfqpoint{3.223075in}{2.266725in}}%
\pgfpathcurveto{\pgfqpoint{3.231312in}{2.266725in}}{\pgfqpoint{3.239212in}{2.269997in}}{\pgfqpoint{3.245036in}{2.275821in}}%
\pgfpathcurveto{\pgfqpoint{3.250860in}{2.281645in}}{\pgfqpoint{3.254132in}{2.289545in}}{\pgfqpoint{3.254132in}{2.297781in}}%
\pgfpathcurveto{\pgfqpoint{3.254132in}{2.306017in}}{\pgfqpoint{3.250860in}{2.313917in}}{\pgfqpoint{3.245036in}{2.319741in}}%
\pgfpathcurveto{\pgfqpoint{3.239212in}{2.325565in}}{\pgfqpoint{3.231312in}{2.328838in}}{\pgfqpoint{3.223075in}{2.328838in}}%
\pgfpathcurveto{\pgfqpoint{3.214839in}{2.328838in}}{\pgfqpoint{3.206939in}{2.325565in}}{\pgfqpoint{3.201115in}{2.319741in}}%
\pgfpathcurveto{\pgfqpoint{3.195291in}{2.313917in}}{\pgfqpoint{3.192019in}{2.306017in}}{\pgfqpoint{3.192019in}{2.297781in}}%
\pgfpathcurveto{\pgfqpoint{3.192019in}{2.289545in}}{\pgfqpoint{3.195291in}{2.281645in}}{\pgfqpoint{3.201115in}{2.275821in}}%
\pgfpathcurveto{\pgfqpoint{3.206939in}{2.269997in}}{\pgfqpoint{3.214839in}{2.266725in}}{\pgfqpoint{3.223075in}{2.266725in}}%
\pgfpathclose%
\pgfusepath{stroke,fill}%
\end{pgfscope}%
\begin{pgfscope}%
\pgfpathrectangle{\pgfqpoint{0.100000in}{0.212622in}}{\pgfqpoint{3.696000in}{3.696000in}}%
\pgfusepath{clip}%
\pgfsetbuttcap%
\pgfsetroundjoin%
\definecolor{currentfill}{rgb}{0.121569,0.466667,0.705882}%
\pgfsetfillcolor{currentfill}%
\pgfsetfillopacity{0.533469}%
\pgfsetlinewidth{1.003750pt}%
\definecolor{currentstroke}{rgb}{0.121569,0.466667,0.705882}%
\pgfsetstrokecolor{currentstroke}%
\pgfsetstrokeopacity{0.533469}%
\pgfsetdash{}{0pt}%
\pgfpathmoveto{\pgfqpoint{0.980114in}{1.692871in}}%
\pgfpathcurveto{\pgfqpoint{0.988350in}{1.692871in}}{\pgfqpoint{0.996250in}{1.696144in}}{\pgfqpoint{1.002074in}{1.701968in}}%
\pgfpathcurveto{\pgfqpoint{1.007898in}{1.707791in}}{\pgfqpoint{1.011170in}{1.715692in}}{\pgfqpoint{1.011170in}{1.723928in}}%
\pgfpathcurveto{\pgfqpoint{1.011170in}{1.732164in}}{\pgfqpoint{1.007898in}{1.740064in}}{\pgfqpoint{1.002074in}{1.745888in}}%
\pgfpathcurveto{\pgfqpoint{0.996250in}{1.751712in}}{\pgfqpoint{0.988350in}{1.754984in}}{\pgfqpoint{0.980114in}{1.754984in}}%
\pgfpathcurveto{\pgfqpoint{0.971878in}{1.754984in}}{\pgfqpoint{0.963978in}{1.751712in}}{\pgfqpoint{0.958154in}{1.745888in}}%
\pgfpathcurveto{\pgfqpoint{0.952330in}{1.740064in}}{\pgfqpoint{0.949057in}{1.732164in}}{\pgfqpoint{0.949057in}{1.723928in}}%
\pgfpathcurveto{\pgfqpoint{0.949057in}{1.715692in}}{\pgfqpoint{0.952330in}{1.707791in}}{\pgfqpoint{0.958154in}{1.701968in}}%
\pgfpathcurveto{\pgfqpoint{0.963978in}{1.696144in}}{\pgfqpoint{0.971878in}{1.692871in}}{\pgfqpoint{0.980114in}{1.692871in}}%
\pgfpathclose%
\pgfusepath{stroke,fill}%
\end{pgfscope}%
\begin{pgfscope}%
\pgfpathrectangle{\pgfqpoint{0.100000in}{0.212622in}}{\pgfqpoint{3.696000in}{3.696000in}}%
\pgfusepath{clip}%
\pgfsetbuttcap%
\pgfsetroundjoin%
\definecolor{currentfill}{rgb}{0.121569,0.466667,0.705882}%
\pgfsetfillcolor{currentfill}%
\pgfsetfillopacity{0.534603}%
\pgfsetlinewidth{1.003750pt}%
\definecolor{currentstroke}{rgb}{0.121569,0.466667,0.705882}%
\pgfsetstrokecolor{currentstroke}%
\pgfsetstrokeopacity{0.534603}%
\pgfsetdash{}{0pt}%
\pgfpathmoveto{\pgfqpoint{0.976613in}{1.688901in}}%
\pgfpathcurveto{\pgfqpoint{0.984849in}{1.688901in}}{\pgfqpoint{0.992749in}{1.692173in}}{\pgfqpoint{0.998573in}{1.697997in}}%
\pgfpathcurveto{\pgfqpoint{1.004397in}{1.703821in}}{\pgfqpoint{1.007670in}{1.711721in}}{\pgfqpoint{1.007670in}{1.719957in}}%
\pgfpathcurveto{\pgfqpoint{1.007670in}{1.728194in}}{\pgfqpoint{1.004397in}{1.736094in}}{\pgfqpoint{0.998573in}{1.741918in}}%
\pgfpathcurveto{\pgfqpoint{0.992749in}{1.747742in}}{\pgfqpoint{0.984849in}{1.751014in}}{\pgfqpoint{0.976613in}{1.751014in}}%
\pgfpathcurveto{\pgfqpoint{0.968377in}{1.751014in}}{\pgfqpoint{0.960477in}{1.747742in}}{\pgfqpoint{0.954653in}{1.741918in}}%
\pgfpathcurveto{\pgfqpoint{0.948829in}{1.736094in}}{\pgfqpoint{0.945557in}{1.728194in}}{\pgfqpoint{0.945557in}{1.719957in}}%
\pgfpathcurveto{\pgfqpoint{0.945557in}{1.711721in}}{\pgfqpoint{0.948829in}{1.703821in}}{\pgfqpoint{0.954653in}{1.697997in}}%
\pgfpathcurveto{\pgfqpoint{0.960477in}{1.692173in}}{\pgfqpoint{0.968377in}{1.688901in}}{\pgfqpoint{0.976613in}{1.688901in}}%
\pgfpathclose%
\pgfusepath{stroke,fill}%
\end{pgfscope}%
\begin{pgfscope}%
\pgfpathrectangle{\pgfqpoint{0.100000in}{0.212622in}}{\pgfqpoint{3.696000in}{3.696000in}}%
\pgfusepath{clip}%
\pgfsetbuttcap%
\pgfsetroundjoin%
\definecolor{currentfill}{rgb}{0.121569,0.466667,0.705882}%
\pgfsetfillcolor{currentfill}%
\pgfsetfillopacity{0.534833}%
\pgfsetlinewidth{1.003750pt}%
\definecolor{currentstroke}{rgb}{0.121569,0.466667,0.705882}%
\pgfsetstrokecolor{currentstroke}%
\pgfsetstrokeopacity{0.534833}%
\pgfsetdash{}{0pt}%
\pgfpathmoveto{\pgfqpoint{3.239825in}{2.264680in}}%
\pgfpathcurveto{\pgfqpoint{3.248061in}{2.264680in}}{\pgfqpoint{3.255961in}{2.267952in}}{\pgfqpoint{3.261785in}{2.273776in}}%
\pgfpathcurveto{\pgfqpoint{3.267609in}{2.279600in}}{\pgfqpoint{3.270881in}{2.287500in}}{\pgfqpoint{3.270881in}{2.295736in}}%
\pgfpathcurveto{\pgfqpoint{3.270881in}{2.303972in}}{\pgfqpoint{3.267609in}{2.311872in}}{\pgfqpoint{3.261785in}{2.317696in}}%
\pgfpathcurveto{\pgfqpoint{3.255961in}{2.323520in}}{\pgfqpoint{3.248061in}{2.326793in}}{\pgfqpoint{3.239825in}{2.326793in}}%
\pgfpathcurveto{\pgfqpoint{3.231589in}{2.326793in}}{\pgfqpoint{3.223689in}{2.323520in}}{\pgfqpoint{3.217865in}{2.317696in}}%
\pgfpathcurveto{\pgfqpoint{3.212041in}{2.311872in}}{\pgfqpoint{3.208768in}{2.303972in}}{\pgfqpoint{3.208768in}{2.295736in}}%
\pgfpathcurveto{\pgfqpoint{3.208768in}{2.287500in}}{\pgfqpoint{3.212041in}{2.279600in}}{\pgfqpoint{3.217865in}{2.273776in}}%
\pgfpathcurveto{\pgfqpoint{3.223689in}{2.267952in}}{\pgfqpoint{3.231589in}{2.264680in}}{\pgfqpoint{3.239825in}{2.264680in}}%
\pgfpathclose%
\pgfusepath{stroke,fill}%
\end{pgfscope}%
\begin{pgfscope}%
\pgfpathrectangle{\pgfqpoint{0.100000in}{0.212622in}}{\pgfqpoint{3.696000in}{3.696000in}}%
\pgfusepath{clip}%
\pgfsetbuttcap%
\pgfsetroundjoin%
\definecolor{currentfill}{rgb}{0.121569,0.466667,0.705882}%
\pgfsetfillcolor{currentfill}%
\pgfsetfillopacity{0.535178}%
\pgfsetlinewidth{1.003750pt}%
\definecolor{currentstroke}{rgb}{0.121569,0.466667,0.705882}%
\pgfsetstrokecolor{currentstroke}%
\pgfsetstrokeopacity{0.535178}%
\pgfsetdash{}{0pt}%
\pgfpathmoveto{\pgfqpoint{0.974755in}{1.686749in}}%
\pgfpathcurveto{\pgfqpoint{0.982992in}{1.686749in}}{\pgfqpoint{0.990892in}{1.690021in}}{\pgfqpoint{0.996716in}{1.695845in}}%
\pgfpathcurveto{\pgfqpoint{1.002540in}{1.701669in}}{\pgfqpoint{1.005812in}{1.709569in}}{\pgfqpoint{1.005812in}{1.717805in}}%
\pgfpathcurveto{\pgfqpoint{1.005812in}{1.726042in}}{\pgfqpoint{1.002540in}{1.733942in}}{\pgfqpoint{0.996716in}{1.739766in}}%
\pgfpathcurveto{\pgfqpoint{0.990892in}{1.745590in}}{\pgfqpoint{0.982992in}{1.748862in}}{\pgfqpoint{0.974755in}{1.748862in}}%
\pgfpathcurveto{\pgfqpoint{0.966519in}{1.748862in}}{\pgfqpoint{0.958619in}{1.745590in}}{\pgfqpoint{0.952795in}{1.739766in}}%
\pgfpathcurveto{\pgfqpoint{0.946971in}{1.733942in}}{\pgfqpoint{0.943699in}{1.726042in}}{\pgfqpoint{0.943699in}{1.717805in}}%
\pgfpathcurveto{\pgfqpoint{0.943699in}{1.709569in}}{\pgfqpoint{0.946971in}{1.701669in}}{\pgfqpoint{0.952795in}{1.695845in}}%
\pgfpathcurveto{\pgfqpoint{0.958619in}{1.690021in}}{\pgfqpoint{0.966519in}{1.686749in}}{\pgfqpoint{0.974755in}{1.686749in}}%
\pgfpathclose%
\pgfusepath{stroke,fill}%
\end{pgfscope}%
\begin{pgfscope}%
\pgfpathrectangle{\pgfqpoint{0.100000in}{0.212622in}}{\pgfqpoint{3.696000in}{3.696000in}}%
\pgfusepath{clip}%
\pgfsetbuttcap%
\pgfsetroundjoin%
\definecolor{currentfill}{rgb}{0.121569,0.466667,0.705882}%
\pgfsetfillcolor{currentfill}%
\pgfsetfillopacity{0.536176}%
\pgfsetlinewidth{1.003750pt}%
\definecolor{currentstroke}{rgb}{0.121569,0.466667,0.705882}%
\pgfsetstrokecolor{currentstroke}%
\pgfsetstrokeopacity{0.536176}%
\pgfsetdash{}{0pt}%
\pgfpathmoveto{\pgfqpoint{3.248986in}{2.263300in}}%
\pgfpathcurveto{\pgfqpoint{3.257222in}{2.263300in}}{\pgfqpoint{3.265123in}{2.266572in}}{\pgfqpoint{3.270946in}{2.272396in}}%
\pgfpathcurveto{\pgfqpoint{3.276770in}{2.278220in}}{\pgfqpoint{3.280043in}{2.286120in}}{\pgfqpoint{3.280043in}{2.294356in}}%
\pgfpathcurveto{\pgfqpoint{3.280043in}{2.302592in}}{\pgfqpoint{3.276770in}{2.310492in}}{\pgfqpoint{3.270946in}{2.316316in}}%
\pgfpathcurveto{\pgfqpoint{3.265123in}{2.322140in}}{\pgfqpoint{3.257222in}{2.325413in}}{\pgfqpoint{3.248986in}{2.325413in}}%
\pgfpathcurveto{\pgfqpoint{3.240750in}{2.325413in}}{\pgfqpoint{3.232850in}{2.322140in}}{\pgfqpoint{3.227026in}{2.316316in}}%
\pgfpathcurveto{\pgfqpoint{3.221202in}{2.310492in}}{\pgfqpoint{3.217930in}{2.302592in}}{\pgfqpoint{3.217930in}{2.294356in}}%
\pgfpathcurveto{\pgfqpoint{3.217930in}{2.286120in}}{\pgfqpoint{3.221202in}{2.278220in}}{\pgfqpoint{3.227026in}{2.272396in}}%
\pgfpathcurveto{\pgfqpoint{3.232850in}{2.266572in}}{\pgfqpoint{3.240750in}{2.263300in}}{\pgfqpoint{3.248986in}{2.263300in}}%
\pgfpathclose%
\pgfusepath{stroke,fill}%
\end{pgfscope}%
\begin{pgfscope}%
\pgfpathrectangle{\pgfqpoint{0.100000in}{0.212622in}}{\pgfqpoint{3.696000in}{3.696000in}}%
\pgfusepath{clip}%
\pgfsetbuttcap%
\pgfsetroundjoin%
\definecolor{currentfill}{rgb}{0.121569,0.466667,0.705882}%
\pgfsetfillcolor{currentfill}%
\pgfsetfillopacity{0.536273}%
\pgfsetlinewidth{1.003750pt}%
\definecolor{currentstroke}{rgb}{0.121569,0.466667,0.705882}%
\pgfsetstrokecolor{currentstroke}%
\pgfsetstrokeopacity{0.536273}%
\pgfsetdash{}{0pt}%
\pgfpathmoveto{\pgfqpoint{0.971470in}{1.683048in}}%
\pgfpathcurveto{\pgfqpoint{0.979707in}{1.683048in}}{\pgfqpoint{0.987607in}{1.686321in}}{\pgfqpoint{0.993431in}{1.692145in}}%
\pgfpathcurveto{\pgfqpoint{0.999254in}{1.697969in}}{\pgfqpoint{1.002527in}{1.705869in}}{\pgfqpoint{1.002527in}{1.714105in}}%
\pgfpathcurveto{\pgfqpoint{1.002527in}{1.722341in}}{\pgfqpoint{0.999254in}{1.730241in}}{\pgfqpoint{0.993431in}{1.736065in}}%
\pgfpathcurveto{\pgfqpoint{0.987607in}{1.741889in}}{\pgfqpoint{0.979707in}{1.745161in}}{\pgfqpoint{0.971470in}{1.745161in}}%
\pgfpathcurveto{\pgfqpoint{0.963234in}{1.745161in}}{\pgfqpoint{0.955334in}{1.741889in}}{\pgfqpoint{0.949510in}{1.736065in}}%
\pgfpathcurveto{\pgfqpoint{0.943686in}{1.730241in}}{\pgfqpoint{0.940414in}{1.722341in}}{\pgfqpoint{0.940414in}{1.714105in}}%
\pgfpathcurveto{\pgfqpoint{0.940414in}{1.705869in}}{\pgfqpoint{0.943686in}{1.697969in}}{\pgfqpoint{0.949510in}{1.692145in}}%
\pgfpathcurveto{\pgfqpoint{0.955334in}{1.686321in}}{\pgfqpoint{0.963234in}{1.683048in}}{\pgfqpoint{0.971470in}{1.683048in}}%
\pgfpathclose%
\pgfusepath{stroke,fill}%
\end{pgfscope}%
\begin{pgfscope}%
\pgfpathrectangle{\pgfqpoint{0.100000in}{0.212622in}}{\pgfqpoint{3.696000in}{3.696000in}}%
\pgfusepath{clip}%
\pgfsetbuttcap%
\pgfsetroundjoin%
\definecolor{currentfill}{rgb}{0.121569,0.466667,0.705882}%
\pgfsetfillcolor{currentfill}%
\pgfsetfillopacity{0.536865}%
\pgfsetlinewidth{1.003750pt}%
\definecolor{currentstroke}{rgb}{0.121569,0.466667,0.705882}%
\pgfsetstrokecolor{currentstroke}%
\pgfsetstrokeopacity{0.536865}%
\pgfsetdash{}{0pt}%
\pgfpathmoveto{\pgfqpoint{3.253938in}{2.261950in}}%
\pgfpathcurveto{\pgfqpoint{3.262174in}{2.261950in}}{\pgfqpoint{3.270074in}{2.265222in}}{\pgfqpoint{3.275898in}{2.271046in}}%
\pgfpathcurveto{\pgfqpoint{3.281722in}{2.276870in}}{\pgfqpoint{3.284994in}{2.284770in}}{\pgfqpoint{3.284994in}{2.293006in}}%
\pgfpathcurveto{\pgfqpoint{3.284994in}{2.301243in}}{\pgfqpoint{3.281722in}{2.309143in}}{\pgfqpoint{3.275898in}{2.314966in}}%
\pgfpathcurveto{\pgfqpoint{3.270074in}{2.320790in}}{\pgfqpoint{3.262174in}{2.324063in}}{\pgfqpoint{3.253938in}{2.324063in}}%
\pgfpathcurveto{\pgfqpoint{3.245701in}{2.324063in}}{\pgfqpoint{3.237801in}{2.320790in}}{\pgfqpoint{3.231978in}{2.314966in}}%
\pgfpathcurveto{\pgfqpoint{3.226154in}{2.309143in}}{\pgfqpoint{3.222881in}{2.301243in}}{\pgfqpoint{3.222881in}{2.293006in}}%
\pgfpathcurveto{\pgfqpoint{3.222881in}{2.284770in}}{\pgfqpoint{3.226154in}{2.276870in}}{\pgfqpoint{3.231978in}{2.271046in}}%
\pgfpathcurveto{\pgfqpoint{3.237801in}{2.265222in}}{\pgfqpoint{3.245701in}{2.261950in}}{\pgfqpoint{3.253938in}{2.261950in}}%
\pgfpathclose%
\pgfusepath{stroke,fill}%
\end{pgfscope}%
\begin{pgfscope}%
\pgfpathrectangle{\pgfqpoint{0.100000in}{0.212622in}}{\pgfqpoint{3.696000in}{3.696000in}}%
\pgfusepath{clip}%
\pgfsetbuttcap%
\pgfsetroundjoin%
\definecolor{currentfill}{rgb}{0.121569,0.466667,0.705882}%
\pgfsetfillcolor{currentfill}%
\pgfsetfillopacity{0.537230}%
\pgfsetlinewidth{1.003750pt}%
\definecolor{currentstroke}{rgb}{0.121569,0.466667,0.705882}%
\pgfsetstrokecolor{currentstroke}%
\pgfsetstrokeopacity{0.537230}%
\pgfsetdash{}{0pt}%
\pgfpathmoveto{\pgfqpoint{0.968530in}{1.679611in}}%
\pgfpathcurveto{\pgfqpoint{0.976766in}{1.679611in}}{\pgfqpoint{0.984666in}{1.682883in}}{\pgfqpoint{0.990490in}{1.688707in}}%
\pgfpathcurveto{\pgfqpoint{0.996314in}{1.694531in}}{\pgfqpoint{0.999586in}{1.702431in}}{\pgfqpoint{0.999586in}{1.710667in}}%
\pgfpathcurveto{\pgfqpoint{0.999586in}{1.718904in}}{\pgfqpoint{0.996314in}{1.726804in}}{\pgfqpoint{0.990490in}{1.732628in}}%
\pgfpathcurveto{\pgfqpoint{0.984666in}{1.738452in}}{\pgfqpoint{0.976766in}{1.741724in}}{\pgfqpoint{0.968530in}{1.741724in}}%
\pgfpathcurveto{\pgfqpoint{0.960293in}{1.741724in}}{\pgfqpoint{0.952393in}{1.738452in}}{\pgfqpoint{0.946569in}{1.732628in}}%
\pgfpathcurveto{\pgfqpoint{0.940746in}{1.726804in}}{\pgfqpoint{0.937473in}{1.718904in}}{\pgfqpoint{0.937473in}{1.710667in}}%
\pgfpathcurveto{\pgfqpoint{0.937473in}{1.702431in}}{\pgfqpoint{0.940746in}{1.694531in}}{\pgfqpoint{0.946569in}{1.688707in}}%
\pgfpathcurveto{\pgfqpoint{0.952393in}{1.682883in}}{\pgfqpoint{0.960293in}{1.679611in}}{\pgfqpoint{0.968530in}{1.679611in}}%
\pgfpathclose%
\pgfusepath{stroke,fill}%
\end{pgfscope}%
\begin{pgfscope}%
\pgfpathrectangle{\pgfqpoint{0.100000in}{0.212622in}}{\pgfqpoint{3.696000in}{3.696000in}}%
\pgfusepath{clip}%
\pgfsetbuttcap%
\pgfsetroundjoin%
\definecolor{currentfill}{rgb}{0.121569,0.466667,0.705882}%
\pgfsetfillcolor{currentfill}%
\pgfsetfillopacity{0.537674}%
\pgfsetlinewidth{1.003750pt}%
\definecolor{currentstroke}{rgb}{0.121569,0.466667,0.705882}%
\pgfsetstrokecolor{currentstroke}%
\pgfsetstrokeopacity{0.537674}%
\pgfsetdash{}{0pt}%
\pgfpathmoveto{\pgfqpoint{3.259931in}{2.260371in}}%
\pgfpathcurveto{\pgfqpoint{3.268168in}{2.260371in}}{\pgfqpoint{3.276068in}{2.263643in}}{\pgfqpoint{3.281892in}{2.269467in}}%
\pgfpathcurveto{\pgfqpoint{3.287716in}{2.275291in}}{\pgfqpoint{3.290988in}{2.283191in}}{\pgfqpoint{3.290988in}{2.291427in}}%
\pgfpathcurveto{\pgfqpoint{3.290988in}{2.299663in}}{\pgfqpoint{3.287716in}{2.307563in}}{\pgfqpoint{3.281892in}{2.313387in}}%
\pgfpathcurveto{\pgfqpoint{3.276068in}{2.319211in}}{\pgfqpoint{3.268168in}{2.322484in}}{\pgfqpoint{3.259931in}{2.322484in}}%
\pgfpathcurveto{\pgfqpoint{3.251695in}{2.322484in}}{\pgfqpoint{3.243795in}{2.319211in}}{\pgfqpoint{3.237971in}{2.313387in}}%
\pgfpathcurveto{\pgfqpoint{3.232147in}{2.307563in}}{\pgfqpoint{3.228875in}{2.299663in}}{\pgfqpoint{3.228875in}{2.291427in}}%
\pgfpathcurveto{\pgfqpoint{3.228875in}{2.283191in}}{\pgfqpoint{3.232147in}{2.275291in}}{\pgfqpoint{3.237971in}{2.269467in}}%
\pgfpathcurveto{\pgfqpoint{3.243795in}{2.263643in}}{\pgfqpoint{3.251695in}{2.260371in}}{\pgfqpoint{3.259931in}{2.260371in}}%
\pgfpathclose%
\pgfusepath{stroke,fill}%
\end{pgfscope}%
\begin{pgfscope}%
\pgfpathrectangle{\pgfqpoint{0.100000in}{0.212622in}}{\pgfqpoint{3.696000in}{3.696000in}}%
\pgfusepath{clip}%
\pgfsetbuttcap%
\pgfsetroundjoin%
\definecolor{currentfill}{rgb}{0.121569,0.466667,0.705882}%
\pgfsetfillcolor{currentfill}%
\pgfsetfillopacity{0.538024}%
\pgfsetlinewidth{1.003750pt}%
\definecolor{currentstroke}{rgb}{0.121569,0.466667,0.705882}%
\pgfsetstrokecolor{currentstroke}%
\pgfsetstrokeopacity{0.538024}%
\pgfsetdash{}{0pt}%
\pgfpathmoveto{\pgfqpoint{0.966162in}{1.676883in}}%
\pgfpathcurveto{\pgfqpoint{0.974398in}{1.676883in}}{\pgfqpoint{0.982299in}{1.680155in}}{\pgfqpoint{0.988122in}{1.685979in}}%
\pgfpathcurveto{\pgfqpoint{0.993946in}{1.691803in}}{\pgfqpoint{0.997219in}{1.699703in}}{\pgfqpoint{0.997219in}{1.707939in}}%
\pgfpathcurveto{\pgfqpoint{0.997219in}{1.716176in}}{\pgfqpoint{0.993946in}{1.724076in}}{\pgfqpoint{0.988122in}{1.729900in}}%
\pgfpathcurveto{\pgfqpoint{0.982299in}{1.735724in}}{\pgfqpoint{0.974398in}{1.738996in}}{\pgfqpoint{0.966162in}{1.738996in}}%
\pgfpathcurveto{\pgfqpoint{0.957926in}{1.738996in}}{\pgfqpoint{0.950026in}{1.735724in}}{\pgfqpoint{0.944202in}{1.729900in}}%
\pgfpathcurveto{\pgfqpoint{0.938378in}{1.724076in}}{\pgfqpoint{0.935106in}{1.716176in}}{\pgfqpoint{0.935106in}{1.707939in}}%
\pgfpathcurveto{\pgfqpoint{0.935106in}{1.699703in}}{\pgfqpoint{0.938378in}{1.691803in}}{\pgfqpoint{0.944202in}{1.685979in}}%
\pgfpathcurveto{\pgfqpoint{0.950026in}{1.680155in}}{\pgfqpoint{0.957926in}{1.676883in}}{\pgfqpoint{0.966162in}{1.676883in}}%
\pgfpathclose%
\pgfusepath{stroke,fill}%
\end{pgfscope}%
\begin{pgfscope}%
\pgfpathrectangle{\pgfqpoint{0.100000in}{0.212622in}}{\pgfqpoint{3.696000in}{3.696000in}}%
\pgfusepath{clip}%
\pgfsetbuttcap%
\pgfsetroundjoin%
\definecolor{currentfill}{rgb}{0.121569,0.466667,0.705882}%
\pgfsetfillcolor{currentfill}%
\pgfsetfillopacity{0.538219}%
\pgfsetlinewidth{1.003750pt}%
\definecolor{currentstroke}{rgb}{0.121569,0.466667,0.705882}%
\pgfsetstrokecolor{currentstroke}%
\pgfsetstrokeopacity{0.538219}%
\pgfsetdash{}{0pt}%
\pgfpathmoveto{\pgfqpoint{3.263255in}{2.260169in}}%
\pgfpathcurveto{\pgfqpoint{3.271491in}{2.260169in}}{\pgfqpoint{3.279391in}{2.263441in}}{\pgfqpoint{3.285215in}{2.269265in}}%
\pgfpathcurveto{\pgfqpoint{3.291039in}{2.275089in}}{\pgfqpoint{3.294311in}{2.282989in}}{\pgfqpoint{3.294311in}{2.291226in}}%
\pgfpathcurveto{\pgfqpoint{3.294311in}{2.299462in}}{\pgfqpoint{3.291039in}{2.307362in}}{\pgfqpoint{3.285215in}{2.313186in}}%
\pgfpathcurveto{\pgfqpoint{3.279391in}{2.319010in}}{\pgfqpoint{3.271491in}{2.322282in}}{\pgfqpoint{3.263255in}{2.322282in}}%
\pgfpathcurveto{\pgfqpoint{3.255018in}{2.322282in}}{\pgfqpoint{3.247118in}{2.319010in}}{\pgfqpoint{3.241295in}{2.313186in}}%
\pgfpathcurveto{\pgfqpoint{3.235471in}{2.307362in}}{\pgfqpoint{3.232198in}{2.299462in}}{\pgfqpoint{3.232198in}{2.291226in}}%
\pgfpathcurveto{\pgfqpoint{3.232198in}{2.282989in}}{\pgfqpoint{3.235471in}{2.275089in}}{\pgfqpoint{3.241295in}{2.269265in}}%
\pgfpathcurveto{\pgfqpoint{3.247118in}{2.263441in}}{\pgfqpoint{3.255018in}{2.260169in}}{\pgfqpoint{3.263255in}{2.260169in}}%
\pgfpathclose%
\pgfusepath{stroke,fill}%
\end{pgfscope}%
\begin{pgfscope}%
\pgfpathrectangle{\pgfqpoint{0.100000in}{0.212622in}}{\pgfqpoint{3.696000in}{3.696000in}}%
\pgfusepath{clip}%
\pgfsetbuttcap%
\pgfsetroundjoin%
\definecolor{currentfill}{rgb}{0.121569,0.466667,0.705882}%
\pgfsetfillcolor{currentfill}%
\pgfsetfillopacity{0.538881}%
\pgfsetlinewidth{1.003750pt}%
\definecolor{currentstroke}{rgb}{0.121569,0.466667,0.705882}%
\pgfsetstrokecolor{currentstroke}%
\pgfsetstrokeopacity{0.538881}%
\pgfsetdash{}{0pt}%
\pgfpathmoveto{\pgfqpoint{3.267357in}{2.259456in}}%
\pgfpathcurveto{\pgfqpoint{3.275594in}{2.259456in}}{\pgfqpoint{3.283494in}{2.262728in}}{\pgfqpoint{3.289318in}{2.268552in}}%
\pgfpathcurveto{\pgfqpoint{3.295142in}{2.274376in}}{\pgfqpoint{3.298414in}{2.282276in}}{\pgfqpoint{3.298414in}{2.290512in}}%
\pgfpathcurveto{\pgfqpoint{3.298414in}{2.298748in}}{\pgfqpoint{3.295142in}{2.306648in}}{\pgfqpoint{3.289318in}{2.312472in}}%
\pgfpathcurveto{\pgfqpoint{3.283494in}{2.318296in}}{\pgfqpoint{3.275594in}{2.321569in}}{\pgfqpoint{3.267357in}{2.321569in}}%
\pgfpathcurveto{\pgfqpoint{3.259121in}{2.321569in}}{\pgfqpoint{3.251221in}{2.318296in}}{\pgfqpoint{3.245397in}{2.312472in}}%
\pgfpathcurveto{\pgfqpoint{3.239573in}{2.306648in}}{\pgfqpoint{3.236301in}{2.298748in}}{\pgfqpoint{3.236301in}{2.290512in}}%
\pgfpathcurveto{\pgfqpoint{3.236301in}{2.282276in}}{\pgfqpoint{3.239573in}{2.274376in}}{\pgfqpoint{3.245397in}{2.268552in}}%
\pgfpathcurveto{\pgfqpoint{3.251221in}{2.262728in}}{\pgfqpoint{3.259121in}{2.259456in}}{\pgfqpoint{3.267357in}{2.259456in}}%
\pgfpathclose%
\pgfusepath{stroke,fill}%
\end{pgfscope}%
\begin{pgfscope}%
\pgfpathrectangle{\pgfqpoint{0.100000in}{0.212622in}}{\pgfqpoint{3.696000in}{3.696000in}}%
\pgfusepath{clip}%
\pgfsetbuttcap%
\pgfsetroundjoin%
\definecolor{currentfill}{rgb}{0.121569,0.466667,0.705882}%
\pgfsetfillcolor{currentfill}%
\pgfsetfillopacity{0.539302}%
\pgfsetlinewidth{1.003750pt}%
\definecolor{currentstroke}{rgb}{0.121569,0.466667,0.705882}%
\pgfsetstrokecolor{currentstroke}%
\pgfsetstrokeopacity{0.539302}%
\pgfsetdash{}{0pt}%
\pgfpathmoveto{\pgfqpoint{3.269614in}{2.259397in}}%
\pgfpathcurveto{\pgfqpoint{3.277850in}{2.259397in}}{\pgfqpoint{3.285750in}{2.262670in}}{\pgfqpoint{3.291574in}{2.268493in}}%
\pgfpathcurveto{\pgfqpoint{3.297398in}{2.274317in}}{\pgfqpoint{3.300671in}{2.282217in}}{\pgfqpoint{3.300671in}{2.290454in}}%
\pgfpathcurveto{\pgfqpoint{3.300671in}{2.298690in}}{\pgfqpoint{3.297398in}{2.306590in}}{\pgfqpoint{3.291574in}{2.312414in}}%
\pgfpathcurveto{\pgfqpoint{3.285750in}{2.318238in}}{\pgfqpoint{3.277850in}{2.321510in}}{\pgfqpoint{3.269614in}{2.321510in}}%
\pgfpathcurveto{\pgfqpoint{3.261378in}{2.321510in}}{\pgfqpoint{3.253478in}{2.318238in}}{\pgfqpoint{3.247654in}{2.312414in}}%
\pgfpathcurveto{\pgfqpoint{3.241830in}{2.306590in}}{\pgfqpoint{3.238558in}{2.298690in}}{\pgfqpoint{3.238558in}{2.290454in}}%
\pgfpathcurveto{\pgfqpoint{3.238558in}{2.282217in}}{\pgfqpoint{3.241830in}{2.274317in}}{\pgfqpoint{3.247654in}{2.268493in}}%
\pgfpathcurveto{\pgfqpoint{3.253478in}{2.262670in}}{\pgfqpoint{3.261378in}{2.259397in}}{\pgfqpoint{3.269614in}{2.259397in}}%
\pgfpathclose%
\pgfusepath{stroke,fill}%
\end{pgfscope}%
\begin{pgfscope}%
\pgfpathrectangle{\pgfqpoint{0.100000in}{0.212622in}}{\pgfqpoint{3.696000in}{3.696000in}}%
\pgfusepath{clip}%
\pgfsetbuttcap%
\pgfsetroundjoin%
\definecolor{currentfill}{rgb}{0.121569,0.466667,0.705882}%
\pgfsetfillcolor{currentfill}%
\pgfsetfillopacity{0.539397}%
\pgfsetlinewidth{1.003750pt}%
\definecolor{currentstroke}{rgb}{0.121569,0.466667,0.705882}%
\pgfsetstrokecolor{currentstroke}%
\pgfsetstrokeopacity{0.539397}%
\pgfsetdash{}{0pt}%
\pgfpathmoveto{\pgfqpoint{0.962198in}{1.671099in}}%
\pgfpathcurveto{\pgfqpoint{0.970435in}{1.671099in}}{\pgfqpoint{0.978335in}{1.674372in}}{\pgfqpoint{0.984159in}{1.680196in}}%
\pgfpathcurveto{\pgfqpoint{0.989983in}{1.686020in}}{\pgfqpoint{0.993255in}{1.693920in}}{\pgfqpoint{0.993255in}{1.702156in}}%
\pgfpathcurveto{\pgfqpoint{0.993255in}{1.710392in}}{\pgfqpoint{0.989983in}{1.718292in}}{\pgfqpoint{0.984159in}{1.724116in}}%
\pgfpathcurveto{\pgfqpoint{0.978335in}{1.729940in}}{\pgfqpoint{0.970435in}{1.733212in}}{\pgfqpoint{0.962198in}{1.733212in}}%
\pgfpathcurveto{\pgfqpoint{0.953962in}{1.733212in}}{\pgfqpoint{0.946062in}{1.729940in}}{\pgfqpoint{0.940238in}{1.724116in}}%
\pgfpathcurveto{\pgfqpoint{0.934414in}{1.718292in}}{\pgfqpoint{0.931142in}{1.710392in}}{\pgfqpoint{0.931142in}{1.702156in}}%
\pgfpathcurveto{\pgfqpoint{0.931142in}{1.693920in}}{\pgfqpoint{0.934414in}{1.686020in}}{\pgfqpoint{0.940238in}{1.680196in}}%
\pgfpathcurveto{\pgfqpoint{0.946062in}{1.674372in}}{\pgfqpoint{0.953962in}{1.671099in}}{\pgfqpoint{0.962198in}{1.671099in}}%
\pgfpathclose%
\pgfusepath{stroke,fill}%
\end{pgfscope}%
\begin{pgfscope}%
\pgfpathrectangle{\pgfqpoint{0.100000in}{0.212622in}}{\pgfqpoint{3.696000in}{3.696000in}}%
\pgfusepath{clip}%
\pgfsetbuttcap%
\pgfsetroundjoin%
\definecolor{currentfill}{rgb}{0.121569,0.466667,0.705882}%
\pgfsetfillcolor{currentfill}%
\pgfsetfillopacity{0.539572}%
\pgfsetlinewidth{1.003750pt}%
\definecolor{currentstroke}{rgb}{0.121569,0.466667,0.705882}%
\pgfsetstrokecolor{currentstroke}%
\pgfsetstrokeopacity{0.539572}%
\pgfsetdash{}{0pt}%
\pgfpathmoveto{\pgfqpoint{3.270859in}{2.259608in}}%
\pgfpathcurveto{\pgfqpoint{3.279096in}{2.259608in}}{\pgfqpoint{3.286996in}{2.262880in}}{\pgfqpoint{3.292820in}{2.268704in}}%
\pgfpathcurveto{\pgfqpoint{3.298644in}{2.274528in}}{\pgfqpoint{3.301916in}{2.282428in}}{\pgfqpoint{3.301916in}{2.290664in}}%
\pgfpathcurveto{\pgfqpoint{3.301916in}{2.298900in}}{\pgfqpoint{3.298644in}{2.306800in}}{\pgfqpoint{3.292820in}{2.312624in}}%
\pgfpathcurveto{\pgfqpoint{3.286996in}{2.318448in}}{\pgfqpoint{3.279096in}{2.321721in}}{\pgfqpoint{3.270859in}{2.321721in}}%
\pgfpathcurveto{\pgfqpoint{3.262623in}{2.321721in}}{\pgfqpoint{3.254723in}{2.318448in}}{\pgfqpoint{3.248899in}{2.312624in}}%
\pgfpathcurveto{\pgfqpoint{3.243075in}{2.306800in}}{\pgfqpoint{3.239803in}{2.298900in}}{\pgfqpoint{3.239803in}{2.290664in}}%
\pgfpathcurveto{\pgfqpoint{3.239803in}{2.282428in}}{\pgfqpoint{3.243075in}{2.274528in}}{\pgfqpoint{3.248899in}{2.268704in}}%
\pgfpathcurveto{\pgfqpoint{3.254723in}{2.262880in}}{\pgfqpoint{3.262623in}{2.259608in}}{\pgfqpoint{3.270859in}{2.259608in}}%
\pgfpathclose%
\pgfusepath{stroke,fill}%
\end{pgfscope}%
\begin{pgfscope}%
\pgfpathrectangle{\pgfqpoint{0.100000in}{0.212622in}}{\pgfqpoint{3.696000in}{3.696000in}}%
\pgfusepath{clip}%
\pgfsetbuttcap%
\pgfsetroundjoin%
\definecolor{currentfill}{rgb}{0.121569,0.466667,0.705882}%
\pgfsetfillcolor{currentfill}%
\pgfsetfillopacity{0.539709}%
\pgfsetlinewidth{1.003750pt}%
\definecolor{currentstroke}{rgb}{0.121569,0.466667,0.705882}%
\pgfsetstrokecolor{currentstroke}%
\pgfsetstrokeopacity{0.539709}%
\pgfsetdash{}{0pt}%
\pgfpathmoveto{\pgfqpoint{3.271524in}{2.259593in}}%
\pgfpathcurveto{\pgfqpoint{3.279760in}{2.259593in}}{\pgfqpoint{3.287660in}{2.262866in}}{\pgfqpoint{3.293484in}{2.268690in}}%
\pgfpathcurveto{\pgfqpoint{3.299308in}{2.274514in}}{\pgfqpoint{3.302580in}{2.282414in}}{\pgfqpoint{3.302580in}{2.290650in}}%
\pgfpathcurveto{\pgfqpoint{3.302580in}{2.298886in}}{\pgfqpoint{3.299308in}{2.306786in}}{\pgfqpoint{3.293484in}{2.312610in}}%
\pgfpathcurveto{\pgfqpoint{3.287660in}{2.318434in}}{\pgfqpoint{3.279760in}{2.321706in}}{\pgfqpoint{3.271524in}{2.321706in}}%
\pgfpathcurveto{\pgfqpoint{3.263287in}{2.321706in}}{\pgfqpoint{3.255387in}{2.318434in}}{\pgfqpoint{3.249563in}{2.312610in}}%
\pgfpathcurveto{\pgfqpoint{3.243740in}{2.306786in}}{\pgfqpoint{3.240467in}{2.298886in}}{\pgfqpoint{3.240467in}{2.290650in}}%
\pgfpathcurveto{\pgfqpoint{3.240467in}{2.282414in}}{\pgfqpoint{3.243740in}{2.274514in}}{\pgfqpoint{3.249563in}{2.268690in}}%
\pgfpathcurveto{\pgfqpoint{3.255387in}{2.262866in}}{\pgfqpoint{3.263287in}{2.259593in}}{\pgfqpoint{3.271524in}{2.259593in}}%
\pgfpathclose%
\pgfusepath{stroke,fill}%
\end{pgfscope}%
\begin{pgfscope}%
\pgfpathrectangle{\pgfqpoint{0.100000in}{0.212622in}}{\pgfqpoint{3.696000in}{3.696000in}}%
\pgfusepath{clip}%
\pgfsetbuttcap%
\pgfsetroundjoin%
\definecolor{currentfill}{rgb}{0.121569,0.466667,0.705882}%
\pgfsetfillcolor{currentfill}%
\pgfsetfillopacity{0.539995}%
\pgfsetlinewidth{1.003750pt}%
\definecolor{currentstroke}{rgb}{0.121569,0.466667,0.705882}%
\pgfsetstrokecolor{currentstroke}%
\pgfsetstrokeopacity{0.539995}%
\pgfsetdash{}{0pt}%
\pgfpathmoveto{\pgfqpoint{3.272956in}{2.259304in}}%
\pgfpathcurveto{\pgfqpoint{3.281193in}{2.259304in}}{\pgfqpoint{3.289093in}{2.262576in}}{\pgfqpoint{3.294917in}{2.268400in}}%
\pgfpathcurveto{\pgfqpoint{3.300740in}{2.274224in}}{\pgfqpoint{3.304013in}{2.282124in}}{\pgfqpoint{3.304013in}{2.290360in}}%
\pgfpathcurveto{\pgfqpoint{3.304013in}{2.298597in}}{\pgfqpoint{3.300740in}{2.306497in}}{\pgfqpoint{3.294917in}{2.312321in}}%
\pgfpathcurveto{\pgfqpoint{3.289093in}{2.318145in}}{\pgfqpoint{3.281193in}{2.321417in}}{\pgfqpoint{3.272956in}{2.321417in}}%
\pgfpathcurveto{\pgfqpoint{3.264720in}{2.321417in}}{\pgfqpoint{3.256820in}{2.318145in}}{\pgfqpoint{3.250996in}{2.312321in}}%
\pgfpathcurveto{\pgfqpoint{3.245172in}{2.306497in}}{\pgfqpoint{3.241900in}{2.298597in}}{\pgfqpoint{3.241900in}{2.290360in}}%
\pgfpathcurveto{\pgfqpoint{3.241900in}{2.282124in}}{\pgfqpoint{3.245172in}{2.274224in}}{\pgfqpoint{3.250996in}{2.268400in}}%
\pgfpathcurveto{\pgfqpoint{3.256820in}{2.262576in}}{\pgfqpoint{3.264720in}{2.259304in}}{\pgfqpoint{3.272956in}{2.259304in}}%
\pgfpathclose%
\pgfusepath{stroke,fill}%
\end{pgfscope}%
\begin{pgfscope}%
\pgfpathrectangle{\pgfqpoint{0.100000in}{0.212622in}}{\pgfqpoint{3.696000in}{3.696000in}}%
\pgfusepath{clip}%
\pgfsetbuttcap%
\pgfsetroundjoin%
\definecolor{currentfill}{rgb}{0.121569,0.466667,0.705882}%
\pgfsetfillcolor{currentfill}%
\pgfsetfillopacity{0.540421}%
\pgfsetlinewidth{1.003750pt}%
\definecolor{currentstroke}{rgb}{0.121569,0.466667,0.705882}%
\pgfsetstrokecolor{currentstroke}%
\pgfsetstrokeopacity{0.540421}%
\pgfsetdash{}{0pt}%
\pgfpathmoveto{\pgfqpoint{3.275017in}{2.258543in}}%
\pgfpathcurveto{\pgfqpoint{3.283254in}{2.258543in}}{\pgfqpoint{3.291154in}{2.261815in}}{\pgfqpoint{3.296978in}{2.267639in}}%
\pgfpathcurveto{\pgfqpoint{3.302802in}{2.273463in}}{\pgfqpoint{3.306074in}{2.281363in}}{\pgfqpoint{3.306074in}{2.289599in}}%
\pgfpathcurveto{\pgfqpoint{3.306074in}{2.297835in}}{\pgfqpoint{3.302802in}{2.305735in}}{\pgfqpoint{3.296978in}{2.311559in}}%
\pgfpathcurveto{\pgfqpoint{3.291154in}{2.317383in}}{\pgfqpoint{3.283254in}{2.320656in}}{\pgfqpoint{3.275017in}{2.320656in}}%
\pgfpathcurveto{\pgfqpoint{3.266781in}{2.320656in}}{\pgfqpoint{3.258881in}{2.317383in}}{\pgfqpoint{3.253057in}{2.311559in}}%
\pgfpathcurveto{\pgfqpoint{3.247233in}{2.305735in}}{\pgfqpoint{3.243961in}{2.297835in}}{\pgfqpoint{3.243961in}{2.289599in}}%
\pgfpathcurveto{\pgfqpoint{3.243961in}{2.281363in}}{\pgfqpoint{3.247233in}{2.273463in}}{\pgfqpoint{3.253057in}{2.267639in}}%
\pgfpathcurveto{\pgfqpoint{3.258881in}{2.261815in}}{\pgfqpoint{3.266781in}{2.258543in}}{\pgfqpoint{3.275017in}{2.258543in}}%
\pgfpathclose%
\pgfusepath{stroke,fill}%
\end{pgfscope}%
\begin{pgfscope}%
\pgfpathrectangle{\pgfqpoint{0.100000in}{0.212622in}}{\pgfqpoint{3.696000in}{3.696000in}}%
\pgfusepath{clip}%
\pgfsetbuttcap%
\pgfsetroundjoin%
\definecolor{currentfill}{rgb}{0.121569,0.466667,0.705882}%
\pgfsetfillcolor{currentfill}%
\pgfsetfillopacity{0.540672}%
\pgfsetlinewidth{1.003750pt}%
\definecolor{currentstroke}{rgb}{0.121569,0.466667,0.705882}%
\pgfsetstrokecolor{currentstroke}%
\pgfsetstrokeopacity{0.540672}%
\pgfsetdash{}{0pt}%
\pgfpathmoveto{\pgfqpoint{3.276098in}{2.258120in}}%
\pgfpathcurveto{\pgfqpoint{3.284334in}{2.258120in}}{\pgfqpoint{3.292234in}{2.261392in}}{\pgfqpoint{3.298058in}{2.267216in}}%
\pgfpathcurveto{\pgfqpoint{3.303882in}{2.273040in}}{\pgfqpoint{3.307154in}{2.280940in}}{\pgfqpoint{3.307154in}{2.289176in}}%
\pgfpathcurveto{\pgfqpoint{3.307154in}{2.297412in}}{\pgfqpoint{3.303882in}{2.305312in}}{\pgfqpoint{3.298058in}{2.311136in}}%
\pgfpathcurveto{\pgfqpoint{3.292234in}{2.316960in}}{\pgfqpoint{3.284334in}{2.320233in}}{\pgfqpoint{3.276098in}{2.320233in}}%
\pgfpathcurveto{\pgfqpoint{3.267862in}{2.320233in}}{\pgfqpoint{3.259962in}{2.316960in}}{\pgfqpoint{3.254138in}{2.311136in}}%
\pgfpathcurveto{\pgfqpoint{3.248314in}{2.305312in}}{\pgfqpoint{3.245041in}{2.297412in}}{\pgfqpoint{3.245041in}{2.289176in}}%
\pgfpathcurveto{\pgfqpoint{3.245041in}{2.280940in}}{\pgfqpoint{3.248314in}{2.273040in}}{\pgfqpoint{3.254138in}{2.267216in}}%
\pgfpathcurveto{\pgfqpoint{3.259962in}{2.261392in}}{\pgfqpoint{3.267862in}{2.258120in}}{\pgfqpoint{3.276098in}{2.258120in}}%
\pgfpathclose%
\pgfusepath{stroke,fill}%
\end{pgfscope}%
\begin{pgfscope}%
\pgfpathrectangle{\pgfqpoint{0.100000in}{0.212622in}}{\pgfqpoint{3.696000in}{3.696000in}}%
\pgfusepath{clip}%
\pgfsetbuttcap%
\pgfsetroundjoin%
\definecolor{currentfill}{rgb}{0.121569,0.466667,0.705882}%
\pgfsetfillcolor{currentfill}%
\pgfsetfillopacity{0.540804}%
\pgfsetlinewidth{1.003750pt}%
\definecolor{currentstroke}{rgb}{0.121569,0.466667,0.705882}%
\pgfsetstrokecolor{currentstroke}%
\pgfsetstrokeopacity{0.540804}%
\pgfsetdash{}{0pt}%
\pgfpathmoveto{\pgfqpoint{3.276664in}{2.257801in}}%
\pgfpathcurveto{\pgfqpoint{3.284901in}{2.257801in}}{\pgfqpoint{3.292801in}{2.261074in}}{\pgfqpoint{3.298625in}{2.266898in}}%
\pgfpathcurveto{\pgfqpoint{3.304449in}{2.272722in}}{\pgfqpoint{3.307721in}{2.280622in}}{\pgfqpoint{3.307721in}{2.288858in}}%
\pgfpathcurveto{\pgfqpoint{3.307721in}{2.297094in}}{\pgfqpoint{3.304449in}{2.304994in}}{\pgfqpoint{3.298625in}{2.310818in}}%
\pgfpathcurveto{\pgfqpoint{3.292801in}{2.316642in}}{\pgfqpoint{3.284901in}{2.319914in}}{\pgfqpoint{3.276664in}{2.319914in}}%
\pgfpathcurveto{\pgfqpoint{3.268428in}{2.319914in}}{\pgfqpoint{3.260528in}{2.316642in}}{\pgfqpoint{3.254704in}{2.310818in}}%
\pgfpathcurveto{\pgfqpoint{3.248880in}{2.304994in}}{\pgfqpoint{3.245608in}{2.297094in}}{\pgfqpoint{3.245608in}{2.288858in}}%
\pgfpathcurveto{\pgfqpoint{3.245608in}{2.280622in}}{\pgfqpoint{3.248880in}{2.272722in}}{\pgfqpoint{3.254704in}{2.266898in}}%
\pgfpathcurveto{\pgfqpoint{3.260528in}{2.261074in}}{\pgfqpoint{3.268428in}{2.257801in}}{\pgfqpoint{3.276664in}{2.257801in}}%
\pgfpathclose%
\pgfusepath{stroke,fill}%
\end{pgfscope}%
\begin{pgfscope}%
\pgfpathrectangle{\pgfqpoint{0.100000in}{0.212622in}}{\pgfqpoint{3.696000in}{3.696000in}}%
\pgfusepath{clip}%
\pgfsetbuttcap%
\pgfsetroundjoin%
\definecolor{currentfill}{rgb}{0.121569,0.466667,0.705882}%
\pgfsetfillcolor{currentfill}%
\pgfsetfillopacity{0.540878}%
\pgfsetlinewidth{1.003750pt}%
\definecolor{currentstroke}{rgb}{0.121569,0.466667,0.705882}%
\pgfsetstrokecolor{currentstroke}%
\pgfsetstrokeopacity{0.540878}%
\pgfsetdash{}{0pt}%
\pgfpathmoveto{\pgfqpoint{3.276963in}{2.257607in}}%
\pgfpathcurveto{\pgfqpoint{3.285199in}{2.257607in}}{\pgfqpoint{3.293099in}{2.260880in}}{\pgfqpoint{3.298923in}{2.266703in}}%
\pgfpathcurveto{\pgfqpoint{3.304747in}{2.272527in}}{\pgfqpoint{3.308019in}{2.280427in}}{\pgfqpoint{3.308019in}{2.288664in}}%
\pgfpathcurveto{\pgfqpoint{3.308019in}{2.296900in}}{\pgfqpoint{3.304747in}{2.304800in}}{\pgfqpoint{3.298923in}{2.310624in}}%
\pgfpathcurveto{\pgfqpoint{3.293099in}{2.316448in}}{\pgfqpoint{3.285199in}{2.319720in}}{\pgfqpoint{3.276963in}{2.319720in}}%
\pgfpathcurveto{\pgfqpoint{3.268726in}{2.319720in}}{\pgfqpoint{3.260826in}{2.316448in}}{\pgfqpoint{3.255002in}{2.310624in}}%
\pgfpathcurveto{\pgfqpoint{3.249178in}{2.304800in}}{\pgfqpoint{3.245906in}{2.296900in}}{\pgfqpoint{3.245906in}{2.288664in}}%
\pgfpathcurveto{\pgfqpoint{3.245906in}{2.280427in}}{\pgfqpoint{3.249178in}{2.272527in}}{\pgfqpoint{3.255002in}{2.266703in}}%
\pgfpathcurveto{\pgfqpoint{3.260826in}{2.260880in}}{\pgfqpoint{3.268726in}{2.257607in}}{\pgfqpoint{3.276963in}{2.257607in}}%
\pgfpathclose%
\pgfusepath{stroke,fill}%
\end{pgfscope}%
\begin{pgfscope}%
\pgfpathrectangle{\pgfqpoint{0.100000in}{0.212622in}}{\pgfqpoint{3.696000in}{3.696000in}}%
\pgfusepath{clip}%
\pgfsetbuttcap%
\pgfsetroundjoin%
\definecolor{currentfill}{rgb}{0.121569,0.466667,0.705882}%
\pgfsetfillcolor{currentfill}%
\pgfsetfillopacity{0.541147}%
\pgfsetlinewidth{1.003750pt}%
\definecolor{currentstroke}{rgb}{0.121569,0.466667,0.705882}%
\pgfsetstrokecolor{currentstroke}%
\pgfsetstrokeopacity{0.541147}%
\pgfsetdash{}{0pt}%
\pgfpathmoveto{\pgfqpoint{3.277911in}{2.256902in}}%
\pgfpathcurveto{\pgfqpoint{3.286147in}{2.256902in}}{\pgfqpoint{3.294047in}{2.260174in}}{\pgfqpoint{3.299871in}{2.265998in}}%
\pgfpathcurveto{\pgfqpoint{3.305695in}{2.271822in}}{\pgfqpoint{3.308968in}{2.279722in}}{\pgfqpoint{3.308968in}{2.287958in}}%
\pgfpathcurveto{\pgfqpoint{3.308968in}{2.296195in}}{\pgfqpoint{3.305695in}{2.304095in}}{\pgfqpoint{3.299871in}{2.309919in}}%
\pgfpathcurveto{\pgfqpoint{3.294047in}{2.315743in}}{\pgfqpoint{3.286147in}{2.319015in}}{\pgfqpoint{3.277911in}{2.319015in}}%
\pgfpathcurveto{\pgfqpoint{3.269675in}{2.319015in}}{\pgfqpoint{3.261775in}{2.315743in}}{\pgfqpoint{3.255951in}{2.309919in}}%
\pgfpathcurveto{\pgfqpoint{3.250127in}{2.304095in}}{\pgfqpoint{3.246855in}{2.296195in}}{\pgfqpoint{3.246855in}{2.287958in}}%
\pgfpathcurveto{\pgfqpoint{3.246855in}{2.279722in}}{\pgfqpoint{3.250127in}{2.271822in}}{\pgfqpoint{3.255951in}{2.265998in}}%
\pgfpathcurveto{\pgfqpoint{3.261775in}{2.260174in}}{\pgfqpoint{3.269675in}{2.256902in}}{\pgfqpoint{3.277911in}{2.256902in}}%
\pgfpathclose%
\pgfusepath{stroke,fill}%
\end{pgfscope}%
\begin{pgfscope}%
\pgfpathrectangle{\pgfqpoint{0.100000in}{0.212622in}}{\pgfqpoint{3.696000in}{3.696000in}}%
\pgfusepath{clip}%
\pgfsetbuttcap%
\pgfsetroundjoin%
\definecolor{currentfill}{rgb}{0.121569,0.466667,0.705882}%
\pgfsetfillcolor{currentfill}%
\pgfsetfillopacity{0.541596}%
\pgfsetlinewidth{1.003750pt}%
\definecolor{currentstroke}{rgb}{0.121569,0.466667,0.705882}%
\pgfsetstrokecolor{currentstroke}%
\pgfsetstrokeopacity{0.541596}%
\pgfsetdash{}{0pt}%
\pgfpathmoveto{\pgfqpoint{3.279288in}{2.255678in}}%
\pgfpathcurveto{\pgfqpoint{3.287525in}{2.255678in}}{\pgfqpoint{3.295425in}{2.258950in}}{\pgfqpoint{3.301249in}{2.264774in}}%
\pgfpathcurveto{\pgfqpoint{3.307073in}{2.270598in}}{\pgfqpoint{3.310345in}{2.278498in}}{\pgfqpoint{3.310345in}{2.286734in}}%
\pgfpathcurveto{\pgfqpoint{3.310345in}{2.294970in}}{\pgfqpoint{3.307073in}{2.302870in}}{\pgfqpoint{3.301249in}{2.308694in}}%
\pgfpathcurveto{\pgfqpoint{3.295425in}{2.314518in}}{\pgfqpoint{3.287525in}{2.317791in}}{\pgfqpoint{3.279288in}{2.317791in}}%
\pgfpathcurveto{\pgfqpoint{3.271052in}{2.317791in}}{\pgfqpoint{3.263152in}{2.314518in}}{\pgfqpoint{3.257328in}{2.308694in}}%
\pgfpathcurveto{\pgfqpoint{3.251504in}{2.302870in}}{\pgfqpoint{3.248232in}{2.294970in}}{\pgfqpoint{3.248232in}{2.286734in}}%
\pgfpathcurveto{\pgfqpoint{3.248232in}{2.278498in}}{\pgfqpoint{3.251504in}{2.270598in}}{\pgfqpoint{3.257328in}{2.264774in}}%
\pgfpathcurveto{\pgfqpoint{3.263152in}{2.258950in}}{\pgfqpoint{3.271052in}{2.255678in}}{\pgfqpoint{3.279288in}{2.255678in}}%
\pgfpathclose%
\pgfusepath{stroke,fill}%
\end{pgfscope}%
\begin{pgfscope}%
\pgfpathrectangle{\pgfqpoint{0.100000in}{0.212622in}}{\pgfqpoint{3.696000in}{3.696000in}}%
\pgfusepath{clip}%
\pgfsetbuttcap%
\pgfsetroundjoin%
\definecolor{currentfill}{rgb}{0.121569,0.466667,0.705882}%
\pgfsetfillcolor{currentfill}%
\pgfsetfillopacity{0.541860}%
\pgfsetlinewidth{1.003750pt}%
\definecolor{currentstroke}{rgb}{0.121569,0.466667,0.705882}%
\pgfsetstrokecolor{currentstroke}%
\pgfsetstrokeopacity{0.541860}%
\pgfsetdash{}{0pt}%
\pgfpathmoveto{\pgfqpoint{0.954672in}{1.660680in}}%
\pgfpathcurveto{\pgfqpoint{0.962908in}{1.660680in}}{\pgfqpoint{0.970808in}{1.663952in}}{\pgfqpoint{0.976632in}{1.669776in}}%
\pgfpathcurveto{\pgfqpoint{0.982456in}{1.675600in}}{\pgfqpoint{0.985728in}{1.683500in}}{\pgfqpoint{0.985728in}{1.691737in}}%
\pgfpathcurveto{\pgfqpoint{0.985728in}{1.699973in}}{\pgfqpoint{0.982456in}{1.707873in}}{\pgfqpoint{0.976632in}{1.713697in}}%
\pgfpathcurveto{\pgfqpoint{0.970808in}{1.719521in}}{\pgfqpoint{0.962908in}{1.722793in}}{\pgfqpoint{0.954672in}{1.722793in}}%
\pgfpathcurveto{\pgfqpoint{0.946436in}{1.722793in}}{\pgfqpoint{0.938536in}{1.719521in}}{\pgfqpoint{0.932712in}{1.713697in}}%
\pgfpathcurveto{\pgfqpoint{0.926888in}{1.707873in}}{\pgfqpoint{0.923615in}{1.699973in}}{\pgfqpoint{0.923615in}{1.691737in}}%
\pgfpathcurveto{\pgfqpoint{0.923615in}{1.683500in}}{\pgfqpoint{0.926888in}{1.675600in}}{\pgfqpoint{0.932712in}{1.669776in}}%
\pgfpathcurveto{\pgfqpoint{0.938536in}{1.663952in}}{\pgfqpoint{0.946436in}{1.660680in}}{\pgfqpoint{0.954672in}{1.660680in}}%
\pgfpathclose%
\pgfusepath{stroke,fill}%
\end{pgfscope}%
\begin{pgfscope}%
\pgfpathrectangle{\pgfqpoint{0.100000in}{0.212622in}}{\pgfqpoint{3.696000in}{3.696000in}}%
\pgfusepath{clip}%
\pgfsetbuttcap%
\pgfsetroundjoin%
\definecolor{currentfill}{rgb}{0.121569,0.466667,0.705882}%
\pgfsetfillcolor{currentfill}%
\pgfsetfillopacity{0.542431}%
\pgfsetlinewidth{1.003750pt}%
\definecolor{currentstroke}{rgb}{0.121569,0.466667,0.705882}%
\pgfsetstrokecolor{currentstroke}%
\pgfsetstrokeopacity{0.542431}%
\pgfsetdash{}{0pt}%
\pgfpathmoveto{\pgfqpoint{3.281491in}{2.253421in}}%
\pgfpathcurveto{\pgfqpoint{3.289727in}{2.253421in}}{\pgfqpoint{3.297627in}{2.256694in}}{\pgfqpoint{3.303451in}{2.262518in}}%
\pgfpathcurveto{\pgfqpoint{3.309275in}{2.268342in}}{\pgfqpoint{3.312547in}{2.276242in}}{\pgfqpoint{3.312547in}{2.284478in}}%
\pgfpathcurveto{\pgfqpoint{3.312547in}{2.292714in}}{\pgfqpoint{3.309275in}{2.300614in}}{\pgfqpoint{3.303451in}{2.306438in}}%
\pgfpathcurveto{\pgfqpoint{3.297627in}{2.312262in}}{\pgfqpoint{3.289727in}{2.315534in}}{\pgfqpoint{3.281491in}{2.315534in}}%
\pgfpathcurveto{\pgfqpoint{3.273254in}{2.315534in}}{\pgfqpoint{3.265354in}{2.312262in}}{\pgfqpoint{3.259530in}{2.306438in}}%
\pgfpathcurveto{\pgfqpoint{3.253707in}{2.300614in}}{\pgfqpoint{3.250434in}{2.292714in}}{\pgfqpoint{3.250434in}{2.284478in}}%
\pgfpathcurveto{\pgfqpoint{3.250434in}{2.276242in}}{\pgfqpoint{3.253707in}{2.268342in}}{\pgfqpoint{3.259530in}{2.262518in}}%
\pgfpathcurveto{\pgfqpoint{3.265354in}{2.256694in}}{\pgfqpoint{3.273254in}{2.253421in}}{\pgfqpoint{3.281491in}{2.253421in}}%
\pgfpathclose%
\pgfusepath{stroke,fill}%
\end{pgfscope}%
\begin{pgfscope}%
\pgfpathrectangle{\pgfqpoint{0.100000in}{0.212622in}}{\pgfqpoint{3.696000in}{3.696000in}}%
\pgfusepath{clip}%
\pgfsetbuttcap%
\pgfsetroundjoin%
\definecolor{currentfill}{rgb}{0.121569,0.466667,0.705882}%
\pgfsetfillcolor{currentfill}%
\pgfsetfillopacity{0.543427}%
\pgfsetlinewidth{1.003750pt}%
\definecolor{currentstroke}{rgb}{0.121569,0.466667,0.705882}%
\pgfsetstrokecolor{currentstroke}%
\pgfsetstrokeopacity{0.543427}%
\pgfsetdash{}{0pt}%
\pgfpathmoveto{\pgfqpoint{3.283735in}{2.250676in}}%
\pgfpathcurveto{\pgfqpoint{3.291971in}{2.250676in}}{\pgfqpoint{3.299871in}{2.253948in}}{\pgfqpoint{3.305695in}{2.259772in}}%
\pgfpathcurveto{\pgfqpoint{3.311519in}{2.265596in}}{\pgfqpoint{3.314791in}{2.273496in}}{\pgfqpoint{3.314791in}{2.281733in}}%
\pgfpathcurveto{\pgfqpoint{3.314791in}{2.289969in}}{\pgfqpoint{3.311519in}{2.297869in}}{\pgfqpoint{3.305695in}{2.303693in}}%
\pgfpathcurveto{\pgfqpoint{3.299871in}{2.309517in}}{\pgfqpoint{3.291971in}{2.312789in}}{\pgfqpoint{3.283735in}{2.312789in}}%
\pgfpathcurveto{\pgfqpoint{3.275498in}{2.312789in}}{\pgfqpoint{3.267598in}{2.309517in}}{\pgfqpoint{3.261774in}{2.303693in}}%
\pgfpathcurveto{\pgfqpoint{3.255951in}{2.297869in}}{\pgfqpoint{3.252678in}{2.289969in}}{\pgfqpoint{3.252678in}{2.281733in}}%
\pgfpathcurveto{\pgfqpoint{3.252678in}{2.273496in}}{\pgfqpoint{3.255951in}{2.265596in}}{\pgfqpoint{3.261774in}{2.259772in}}%
\pgfpathcurveto{\pgfqpoint{3.267598in}{2.253948in}}{\pgfqpoint{3.275498in}{2.250676in}}{\pgfqpoint{3.283735in}{2.250676in}}%
\pgfpathclose%
\pgfusepath{stroke,fill}%
\end{pgfscope}%
\begin{pgfscope}%
\pgfpathrectangle{\pgfqpoint{0.100000in}{0.212622in}}{\pgfqpoint{3.696000in}{3.696000in}}%
\pgfusepath{clip}%
\pgfsetbuttcap%
\pgfsetroundjoin%
\definecolor{currentfill}{rgb}{0.121569,0.466667,0.705882}%
\pgfsetfillcolor{currentfill}%
\pgfsetfillopacity{0.543993}%
\pgfsetlinewidth{1.003750pt}%
\definecolor{currentstroke}{rgb}{0.121569,0.466667,0.705882}%
\pgfsetstrokecolor{currentstroke}%
\pgfsetstrokeopacity{0.543993}%
\pgfsetdash{}{0pt}%
\pgfpathmoveto{\pgfqpoint{3.284793in}{2.249145in}}%
\pgfpathcurveto{\pgfqpoint{3.293030in}{2.249145in}}{\pgfqpoint{3.300930in}{2.252417in}}{\pgfqpoint{3.306754in}{2.258241in}}%
\pgfpathcurveto{\pgfqpoint{3.312578in}{2.264065in}}{\pgfqpoint{3.315850in}{2.271965in}}{\pgfqpoint{3.315850in}{2.280201in}}%
\pgfpathcurveto{\pgfqpoint{3.315850in}{2.288438in}}{\pgfqpoint{3.312578in}{2.296338in}}{\pgfqpoint{3.306754in}{2.302162in}}%
\pgfpathcurveto{\pgfqpoint{3.300930in}{2.307986in}}{\pgfqpoint{3.293030in}{2.311258in}}{\pgfqpoint{3.284793in}{2.311258in}}%
\pgfpathcurveto{\pgfqpoint{3.276557in}{2.311258in}}{\pgfqpoint{3.268657in}{2.307986in}}{\pgfqpoint{3.262833in}{2.302162in}}%
\pgfpathcurveto{\pgfqpoint{3.257009in}{2.296338in}}{\pgfqpoint{3.253737in}{2.288438in}}{\pgfqpoint{3.253737in}{2.280201in}}%
\pgfpathcurveto{\pgfqpoint{3.253737in}{2.271965in}}{\pgfqpoint{3.257009in}{2.264065in}}{\pgfqpoint{3.262833in}{2.258241in}}%
\pgfpathcurveto{\pgfqpoint{3.268657in}{2.252417in}}{\pgfqpoint{3.276557in}{2.249145in}}{\pgfqpoint{3.284793in}{2.249145in}}%
\pgfpathclose%
\pgfusepath{stroke,fill}%
\end{pgfscope}%
\begin{pgfscope}%
\pgfpathrectangle{\pgfqpoint{0.100000in}{0.212622in}}{\pgfqpoint{3.696000in}{3.696000in}}%
\pgfusepath{clip}%
\pgfsetbuttcap%
\pgfsetroundjoin%
\definecolor{currentfill}{rgb}{0.121569,0.466667,0.705882}%
\pgfsetfillcolor{currentfill}%
\pgfsetfillopacity{0.544125}%
\pgfsetlinewidth{1.003750pt}%
\definecolor{currentstroke}{rgb}{0.121569,0.466667,0.705882}%
\pgfsetstrokecolor{currentstroke}%
\pgfsetstrokeopacity{0.544125}%
\pgfsetdash{}{0pt}%
\pgfpathmoveto{\pgfqpoint{0.948230in}{1.650531in}}%
\pgfpathcurveto{\pgfqpoint{0.956466in}{1.650531in}}{\pgfqpoint{0.964366in}{1.653803in}}{\pgfqpoint{0.970190in}{1.659627in}}%
\pgfpathcurveto{\pgfqpoint{0.976014in}{1.665451in}}{\pgfqpoint{0.979286in}{1.673351in}}{\pgfqpoint{0.979286in}{1.681587in}}%
\pgfpathcurveto{\pgfqpoint{0.979286in}{1.689824in}}{\pgfqpoint{0.976014in}{1.697724in}}{\pgfqpoint{0.970190in}{1.703548in}}%
\pgfpathcurveto{\pgfqpoint{0.964366in}{1.709371in}}{\pgfqpoint{0.956466in}{1.712644in}}{\pgfqpoint{0.948230in}{1.712644in}}%
\pgfpathcurveto{\pgfqpoint{0.939993in}{1.712644in}}{\pgfqpoint{0.932093in}{1.709371in}}{\pgfqpoint{0.926269in}{1.703548in}}%
\pgfpathcurveto{\pgfqpoint{0.920445in}{1.697724in}}{\pgfqpoint{0.917173in}{1.689824in}}{\pgfqpoint{0.917173in}{1.681587in}}%
\pgfpathcurveto{\pgfqpoint{0.917173in}{1.673351in}}{\pgfqpoint{0.920445in}{1.665451in}}{\pgfqpoint{0.926269in}{1.659627in}}%
\pgfpathcurveto{\pgfqpoint{0.932093in}{1.653803in}}{\pgfqpoint{0.939993in}{1.650531in}}{\pgfqpoint{0.948230in}{1.650531in}}%
\pgfpathclose%
\pgfusepath{stroke,fill}%
\end{pgfscope}%
\begin{pgfscope}%
\pgfpathrectangle{\pgfqpoint{0.100000in}{0.212622in}}{\pgfqpoint{3.696000in}{3.696000in}}%
\pgfusepath{clip}%
\pgfsetbuttcap%
\pgfsetroundjoin%
\definecolor{currentfill}{rgb}{0.121569,0.466667,0.705882}%
\pgfsetfillcolor{currentfill}%
\pgfsetfillopacity{0.544312}%
\pgfsetlinewidth{1.003750pt}%
\definecolor{currentstroke}{rgb}{0.121569,0.466667,0.705882}%
\pgfsetstrokecolor{currentstroke}%
\pgfsetstrokeopacity{0.544312}%
\pgfsetdash{}{0pt}%
\pgfpathmoveto{\pgfqpoint{3.285275in}{2.248288in}}%
\pgfpathcurveto{\pgfqpoint{3.293512in}{2.248288in}}{\pgfqpoint{3.301412in}{2.251561in}}{\pgfqpoint{3.307236in}{2.257385in}}%
\pgfpathcurveto{\pgfqpoint{3.313059in}{2.263208in}}{\pgfqpoint{3.316332in}{2.271109in}}{\pgfqpoint{3.316332in}{2.279345in}}%
\pgfpathcurveto{\pgfqpoint{3.316332in}{2.287581in}}{\pgfqpoint{3.313059in}{2.295481in}}{\pgfqpoint{3.307236in}{2.301305in}}%
\pgfpathcurveto{\pgfqpoint{3.301412in}{2.307129in}}{\pgfqpoint{3.293512in}{2.310401in}}{\pgfqpoint{3.285275in}{2.310401in}}%
\pgfpathcurveto{\pgfqpoint{3.277039in}{2.310401in}}{\pgfqpoint{3.269139in}{2.307129in}}{\pgfqpoint{3.263315in}{2.301305in}}%
\pgfpathcurveto{\pgfqpoint{3.257491in}{2.295481in}}{\pgfqpoint{3.254219in}{2.287581in}}{\pgfqpoint{3.254219in}{2.279345in}}%
\pgfpathcurveto{\pgfqpoint{3.254219in}{2.271109in}}{\pgfqpoint{3.257491in}{2.263208in}}{\pgfqpoint{3.263315in}{2.257385in}}%
\pgfpathcurveto{\pgfqpoint{3.269139in}{2.251561in}}{\pgfqpoint{3.277039in}{2.248288in}}{\pgfqpoint{3.285275in}{2.248288in}}%
\pgfpathclose%
\pgfusepath{stroke,fill}%
\end{pgfscope}%
\begin{pgfscope}%
\pgfpathrectangle{\pgfqpoint{0.100000in}{0.212622in}}{\pgfqpoint{3.696000in}{3.696000in}}%
\pgfusepath{clip}%
\pgfsetbuttcap%
\pgfsetroundjoin%
\definecolor{currentfill}{rgb}{0.121569,0.466667,0.705882}%
\pgfsetfillcolor{currentfill}%
\pgfsetfillopacity{0.544494}%
\pgfsetlinewidth{1.003750pt}%
\definecolor{currentstroke}{rgb}{0.121569,0.466667,0.705882}%
\pgfsetstrokecolor{currentstroke}%
\pgfsetstrokeopacity{0.544494}%
\pgfsetdash{}{0pt}%
\pgfpathmoveto{\pgfqpoint{3.285486in}{2.247833in}}%
\pgfpathcurveto{\pgfqpoint{3.293722in}{2.247833in}}{\pgfqpoint{3.301622in}{2.251106in}}{\pgfqpoint{3.307446in}{2.256930in}}%
\pgfpathcurveto{\pgfqpoint{3.313270in}{2.262754in}}{\pgfqpoint{3.316543in}{2.270654in}}{\pgfqpoint{3.316543in}{2.278890in}}%
\pgfpathcurveto{\pgfqpoint{3.316543in}{2.287126in}}{\pgfqpoint{3.313270in}{2.295026in}}{\pgfqpoint{3.307446in}{2.300850in}}%
\pgfpathcurveto{\pgfqpoint{3.301622in}{2.306674in}}{\pgfqpoint{3.293722in}{2.309946in}}{\pgfqpoint{3.285486in}{2.309946in}}%
\pgfpathcurveto{\pgfqpoint{3.277250in}{2.309946in}}{\pgfqpoint{3.269350in}{2.306674in}}{\pgfqpoint{3.263526in}{2.300850in}}%
\pgfpathcurveto{\pgfqpoint{3.257702in}{2.295026in}}{\pgfqpoint{3.254430in}{2.287126in}}{\pgfqpoint{3.254430in}{2.278890in}}%
\pgfpathcurveto{\pgfqpoint{3.254430in}{2.270654in}}{\pgfqpoint{3.257702in}{2.262754in}}{\pgfqpoint{3.263526in}{2.256930in}}%
\pgfpathcurveto{\pgfqpoint{3.269350in}{2.251106in}}{\pgfqpoint{3.277250in}{2.247833in}}{\pgfqpoint{3.285486in}{2.247833in}}%
\pgfpathclose%
\pgfusepath{stroke,fill}%
\end{pgfscope}%
\begin{pgfscope}%
\pgfpathrectangle{\pgfqpoint{0.100000in}{0.212622in}}{\pgfqpoint{3.696000in}{3.696000in}}%
\pgfusepath{clip}%
\pgfsetbuttcap%
\pgfsetroundjoin%
\definecolor{currentfill}{rgb}{0.121569,0.466667,0.705882}%
\pgfsetfillcolor{currentfill}%
\pgfsetfillopacity{0.544596}%
\pgfsetlinewidth{1.003750pt}%
\definecolor{currentstroke}{rgb}{0.121569,0.466667,0.705882}%
\pgfsetstrokecolor{currentstroke}%
\pgfsetstrokeopacity{0.544596}%
\pgfsetdash{}{0pt}%
\pgfpathmoveto{\pgfqpoint{3.285575in}{2.247591in}}%
\pgfpathcurveto{\pgfqpoint{3.293811in}{2.247591in}}{\pgfqpoint{3.301711in}{2.250863in}}{\pgfqpoint{3.307535in}{2.256687in}}%
\pgfpathcurveto{\pgfqpoint{3.313359in}{2.262511in}}{\pgfqpoint{3.316631in}{2.270411in}}{\pgfqpoint{3.316631in}{2.278647in}}%
\pgfpathcurveto{\pgfqpoint{3.316631in}{2.286884in}}{\pgfqpoint{3.313359in}{2.294784in}}{\pgfqpoint{3.307535in}{2.300608in}}%
\pgfpathcurveto{\pgfqpoint{3.301711in}{2.306432in}}{\pgfqpoint{3.293811in}{2.309704in}}{\pgfqpoint{3.285575in}{2.309704in}}%
\pgfpathcurveto{\pgfqpoint{3.277339in}{2.309704in}}{\pgfqpoint{3.269439in}{2.306432in}}{\pgfqpoint{3.263615in}{2.300608in}}%
\pgfpathcurveto{\pgfqpoint{3.257791in}{2.294784in}}{\pgfqpoint{3.254518in}{2.286884in}}{\pgfqpoint{3.254518in}{2.278647in}}%
\pgfpathcurveto{\pgfqpoint{3.254518in}{2.270411in}}{\pgfqpoint{3.257791in}{2.262511in}}{\pgfqpoint{3.263615in}{2.256687in}}%
\pgfpathcurveto{\pgfqpoint{3.269439in}{2.250863in}}{\pgfqpoint{3.277339in}{2.247591in}}{\pgfqpoint{3.285575in}{2.247591in}}%
\pgfpathclose%
\pgfusepath{stroke,fill}%
\end{pgfscope}%
\begin{pgfscope}%
\pgfpathrectangle{\pgfqpoint{0.100000in}{0.212622in}}{\pgfqpoint{3.696000in}{3.696000in}}%
\pgfusepath{clip}%
\pgfsetbuttcap%
\pgfsetroundjoin%
\definecolor{currentfill}{rgb}{0.121569,0.466667,0.705882}%
\pgfsetfillcolor{currentfill}%
\pgfsetfillopacity{0.544654}%
\pgfsetlinewidth{1.003750pt}%
\definecolor{currentstroke}{rgb}{0.121569,0.466667,0.705882}%
\pgfsetstrokecolor{currentstroke}%
\pgfsetstrokeopacity{0.544654}%
\pgfsetdash{}{0pt}%
\pgfpathmoveto{\pgfqpoint{3.285608in}{2.247462in}}%
\pgfpathcurveto{\pgfqpoint{3.293844in}{2.247462in}}{\pgfqpoint{3.301744in}{2.250734in}}{\pgfqpoint{3.307568in}{2.256558in}}%
\pgfpathcurveto{\pgfqpoint{3.313392in}{2.262382in}}{\pgfqpoint{3.316664in}{2.270282in}}{\pgfqpoint{3.316664in}{2.278518in}}%
\pgfpathcurveto{\pgfqpoint{3.316664in}{2.286754in}}{\pgfqpoint{3.313392in}{2.294654in}}{\pgfqpoint{3.307568in}{2.300478in}}%
\pgfpathcurveto{\pgfqpoint{3.301744in}{2.306302in}}{\pgfqpoint{3.293844in}{2.309575in}}{\pgfqpoint{3.285608in}{2.309575in}}%
\pgfpathcurveto{\pgfqpoint{3.277372in}{2.309575in}}{\pgfqpoint{3.269472in}{2.306302in}}{\pgfqpoint{3.263648in}{2.300478in}}%
\pgfpathcurveto{\pgfqpoint{3.257824in}{2.294654in}}{\pgfqpoint{3.254551in}{2.286754in}}{\pgfqpoint{3.254551in}{2.278518in}}%
\pgfpathcurveto{\pgfqpoint{3.254551in}{2.270282in}}{\pgfqpoint{3.257824in}{2.262382in}}{\pgfqpoint{3.263648in}{2.256558in}}%
\pgfpathcurveto{\pgfqpoint{3.269472in}{2.250734in}}{\pgfqpoint{3.277372in}{2.247462in}}{\pgfqpoint{3.285608in}{2.247462in}}%
\pgfpathclose%
\pgfusepath{stroke,fill}%
\end{pgfscope}%
\begin{pgfscope}%
\pgfpathrectangle{\pgfqpoint{0.100000in}{0.212622in}}{\pgfqpoint{3.696000in}{3.696000in}}%
\pgfusepath{clip}%
\pgfsetbuttcap%
\pgfsetroundjoin%
\definecolor{currentfill}{rgb}{0.121569,0.466667,0.705882}%
\pgfsetfillcolor{currentfill}%
\pgfsetfillopacity{0.544686}%
\pgfsetlinewidth{1.003750pt}%
\definecolor{currentstroke}{rgb}{0.121569,0.466667,0.705882}%
\pgfsetstrokecolor{currentstroke}%
\pgfsetstrokeopacity{0.544686}%
\pgfsetdash{}{0pt}%
\pgfpathmoveto{\pgfqpoint{3.285618in}{2.247393in}}%
\pgfpathcurveto{\pgfqpoint{3.293854in}{2.247393in}}{\pgfqpoint{3.301754in}{2.250666in}}{\pgfqpoint{3.307578in}{2.256489in}}%
\pgfpathcurveto{\pgfqpoint{3.313402in}{2.262313in}}{\pgfqpoint{3.316674in}{2.270213in}}{\pgfqpoint{3.316674in}{2.278450in}}%
\pgfpathcurveto{\pgfqpoint{3.316674in}{2.286686in}}{\pgfqpoint{3.313402in}{2.294586in}}{\pgfqpoint{3.307578in}{2.300410in}}%
\pgfpathcurveto{\pgfqpoint{3.301754in}{2.306234in}}{\pgfqpoint{3.293854in}{2.309506in}}{\pgfqpoint{3.285618in}{2.309506in}}%
\pgfpathcurveto{\pgfqpoint{3.277382in}{2.309506in}}{\pgfqpoint{3.269482in}{2.306234in}}{\pgfqpoint{3.263658in}{2.300410in}}%
\pgfpathcurveto{\pgfqpoint{3.257834in}{2.294586in}}{\pgfqpoint{3.254561in}{2.286686in}}{\pgfqpoint{3.254561in}{2.278450in}}%
\pgfpathcurveto{\pgfqpoint{3.254561in}{2.270213in}}{\pgfqpoint{3.257834in}{2.262313in}}{\pgfqpoint{3.263658in}{2.256489in}}%
\pgfpathcurveto{\pgfqpoint{3.269482in}{2.250666in}}{\pgfqpoint{3.277382in}{2.247393in}}{\pgfqpoint{3.285618in}{2.247393in}}%
\pgfpathclose%
\pgfusepath{stroke,fill}%
\end{pgfscope}%
\begin{pgfscope}%
\pgfpathrectangle{\pgfqpoint{0.100000in}{0.212622in}}{\pgfqpoint{3.696000in}{3.696000in}}%
\pgfusepath{clip}%
\pgfsetbuttcap%
\pgfsetroundjoin%
\definecolor{currentfill}{rgb}{0.121569,0.466667,0.705882}%
\pgfsetfillcolor{currentfill}%
\pgfsetfillopacity{0.544704}%
\pgfsetlinewidth{1.003750pt}%
\definecolor{currentstroke}{rgb}{0.121569,0.466667,0.705882}%
\pgfsetstrokecolor{currentstroke}%
\pgfsetstrokeopacity{0.544704}%
\pgfsetdash{}{0pt}%
\pgfpathmoveto{\pgfqpoint{3.285619in}{2.247357in}}%
\pgfpathcurveto{\pgfqpoint{3.293855in}{2.247357in}}{\pgfqpoint{3.301755in}{2.250630in}}{\pgfqpoint{3.307579in}{2.256453in}}%
\pgfpathcurveto{\pgfqpoint{3.313403in}{2.262277in}}{\pgfqpoint{3.316676in}{2.270177in}}{\pgfqpoint{3.316676in}{2.278414in}}%
\pgfpathcurveto{\pgfqpoint{3.316676in}{2.286650in}}{\pgfqpoint{3.313403in}{2.294550in}}{\pgfqpoint{3.307579in}{2.300374in}}%
\pgfpathcurveto{\pgfqpoint{3.301755in}{2.306198in}}{\pgfqpoint{3.293855in}{2.309470in}}{\pgfqpoint{3.285619in}{2.309470in}}%
\pgfpathcurveto{\pgfqpoint{3.277383in}{2.309470in}}{\pgfqpoint{3.269483in}{2.306198in}}{\pgfqpoint{3.263659in}{2.300374in}}%
\pgfpathcurveto{\pgfqpoint{3.257835in}{2.294550in}}{\pgfqpoint{3.254563in}{2.286650in}}{\pgfqpoint{3.254563in}{2.278414in}}%
\pgfpathcurveto{\pgfqpoint{3.254563in}{2.270177in}}{\pgfqpoint{3.257835in}{2.262277in}}{\pgfqpoint{3.263659in}{2.256453in}}%
\pgfpathcurveto{\pgfqpoint{3.269483in}{2.250630in}}{\pgfqpoint{3.277383in}{2.247357in}}{\pgfqpoint{3.285619in}{2.247357in}}%
\pgfpathclose%
\pgfusepath{stroke,fill}%
\end{pgfscope}%
\begin{pgfscope}%
\pgfpathrectangle{\pgfqpoint{0.100000in}{0.212622in}}{\pgfqpoint{3.696000in}{3.696000in}}%
\pgfusepath{clip}%
\pgfsetbuttcap%
\pgfsetroundjoin%
\definecolor{currentfill}{rgb}{0.121569,0.466667,0.705882}%
\pgfsetfillcolor{currentfill}%
\pgfsetfillopacity{0.544870}%
\pgfsetlinewidth{1.003750pt}%
\definecolor{currentstroke}{rgb}{0.121569,0.466667,0.705882}%
\pgfsetstrokecolor{currentstroke}%
\pgfsetstrokeopacity{0.544870}%
\pgfsetdash{}{0pt}%
\pgfpathmoveto{\pgfqpoint{3.285584in}{2.247056in}}%
\pgfpathcurveto{\pgfqpoint{3.293820in}{2.247056in}}{\pgfqpoint{3.301720in}{2.250329in}}{\pgfqpoint{3.307544in}{2.256153in}}%
\pgfpathcurveto{\pgfqpoint{3.313368in}{2.261976in}}{\pgfqpoint{3.316641in}{2.269877in}}{\pgfqpoint{3.316641in}{2.278113in}}%
\pgfpathcurveto{\pgfqpoint{3.316641in}{2.286349in}}{\pgfqpoint{3.313368in}{2.294249in}}{\pgfqpoint{3.307544in}{2.300073in}}%
\pgfpathcurveto{\pgfqpoint{3.301720in}{2.305897in}}{\pgfqpoint{3.293820in}{2.309169in}}{\pgfqpoint{3.285584in}{2.309169in}}%
\pgfpathcurveto{\pgfqpoint{3.277348in}{2.309169in}}{\pgfqpoint{3.269448in}{2.305897in}}{\pgfqpoint{3.263624in}{2.300073in}}%
\pgfpathcurveto{\pgfqpoint{3.257800in}{2.294249in}}{\pgfqpoint{3.254528in}{2.286349in}}{\pgfqpoint{3.254528in}{2.278113in}}%
\pgfpathcurveto{\pgfqpoint{3.254528in}{2.269877in}}{\pgfqpoint{3.257800in}{2.261976in}}{\pgfqpoint{3.263624in}{2.256153in}}%
\pgfpathcurveto{\pgfqpoint{3.269448in}{2.250329in}}{\pgfqpoint{3.277348in}{2.247056in}}{\pgfqpoint{3.285584in}{2.247056in}}%
\pgfpathclose%
\pgfusepath{stroke,fill}%
\end{pgfscope}%
\begin{pgfscope}%
\pgfpathrectangle{\pgfqpoint{0.100000in}{0.212622in}}{\pgfqpoint{3.696000in}{3.696000in}}%
\pgfusepath{clip}%
\pgfsetbuttcap%
\pgfsetroundjoin%
\definecolor{currentfill}{rgb}{0.121569,0.466667,0.705882}%
\pgfsetfillcolor{currentfill}%
\pgfsetfillopacity{0.545394}%
\pgfsetlinewidth{1.003750pt}%
\definecolor{currentstroke}{rgb}{0.121569,0.466667,0.705882}%
\pgfsetstrokecolor{currentstroke}%
\pgfsetstrokeopacity{0.545394}%
\pgfsetdash{}{0pt}%
\pgfpathmoveto{\pgfqpoint{3.285342in}{2.246098in}}%
\pgfpathcurveto{\pgfqpoint{3.293578in}{2.246098in}}{\pgfqpoint{3.301478in}{2.249370in}}{\pgfqpoint{3.307302in}{2.255194in}}%
\pgfpathcurveto{\pgfqpoint{3.313126in}{2.261018in}}{\pgfqpoint{3.316398in}{2.268918in}}{\pgfqpoint{3.316398in}{2.277154in}}%
\pgfpathcurveto{\pgfqpoint{3.316398in}{2.285390in}}{\pgfqpoint{3.313126in}{2.293290in}}{\pgfqpoint{3.307302in}{2.299114in}}%
\pgfpathcurveto{\pgfqpoint{3.301478in}{2.304938in}}{\pgfqpoint{3.293578in}{2.308211in}}{\pgfqpoint{3.285342in}{2.308211in}}%
\pgfpathcurveto{\pgfqpoint{3.277106in}{2.308211in}}{\pgfqpoint{3.269206in}{2.304938in}}{\pgfqpoint{3.263382in}{2.299114in}}%
\pgfpathcurveto{\pgfqpoint{3.257558in}{2.293290in}}{\pgfqpoint{3.254285in}{2.285390in}}{\pgfqpoint{3.254285in}{2.277154in}}%
\pgfpathcurveto{\pgfqpoint{3.254285in}{2.268918in}}{\pgfqpoint{3.257558in}{2.261018in}}{\pgfqpoint{3.263382in}{2.255194in}}%
\pgfpathcurveto{\pgfqpoint{3.269206in}{2.249370in}}{\pgfqpoint{3.277106in}{2.246098in}}{\pgfqpoint{3.285342in}{2.246098in}}%
\pgfpathclose%
\pgfusepath{stroke,fill}%
\end{pgfscope}%
\begin{pgfscope}%
\pgfpathrectangle{\pgfqpoint{0.100000in}{0.212622in}}{\pgfqpoint{3.696000in}{3.696000in}}%
\pgfusepath{clip}%
\pgfsetbuttcap%
\pgfsetroundjoin%
\definecolor{currentfill}{rgb}{0.121569,0.466667,0.705882}%
\pgfsetfillcolor{currentfill}%
\pgfsetfillopacity{0.546037}%
\pgfsetlinewidth{1.003750pt}%
\definecolor{currentstroke}{rgb}{0.121569,0.466667,0.705882}%
\pgfsetstrokecolor{currentstroke}%
\pgfsetstrokeopacity{0.546037}%
\pgfsetdash{}{0pt}%
\pgfpathmoveto{\pgfqpoint{0.942501in}{1.642327in}}%
\pgfpathcurveto{\pgfqpoint{0.950738in}{1.642327in}}{\pgfqpoint{0.958638in}{1.645599in}}{\pgfqpoint{0.964462in}{1.651423in}}%
\pgfpathcurveto{\pgfqpoint{0.970286in}{1.657247in}}{\pgfqpoint{0.973558in}{1.665147in}}{\pgfqpoint{0.973558in}{1.673383in}}%
\pgfpathcurveto{\pgfqpoint{0.973558in}{1.681620in}}{\pgfqpoint{0.970286in}{1.689520in}}{\pgfqpoint{0.964462in}{1.695344in}}%
\pgfpathcurveto{\pgfqpoint{0.958638in}{1.701168in}}{\pgfqpoint{0.950738in}{1.704440in}}{\pgfqpoint{0.942501in}{1.704440in}}%
\pgfpathcurveto{\pgfqpoint{0.934265in}{1.704440in}}{\pgfqpoint{0.926365in}{1.701168in}}{\pgfqpoint{0.920541in}{1.695344in}}%
\pgfpathcurveto{\pgfqpoint{0.914717in}{1.689520in}}{\pgfqpoint{0.911445in}{1.681620in}}{\pgfqpoint{0.911445in}{1.673383in}}%
\pgfpathcurveto{\pgfqpoint{0.911445in}{1.665147in}}{\pgfqpoint{0.914717in}{1.657247in}}{\pgfqpoint{0.920541in}{1.651423in}}%
\pgfpathcurveto{\pgfqpoint{0.926365in}{1.645599in}}{\pgfqpoint{0.934265in}{1.642327in}}{\pgfqpoint{0.942501in}{1.642327in}}%
\pgfpathclose%
\pgfusepath{stroke,fill}%
\end{pgfscope}%
\begin{pgfscope}%
\pgfpathrectangle{\pgfqpoint{0.100000in}{0.212622in}}{\pgfqpoint{3.696000in}{3.696000in}}%
\pgfusepath{clip}%
\pgfsetbuttcap%
\pgfsetroundjoin%
\definecolor{currentfill}{rgb}{0.121569,0.466667,0.705882}%
\pgfsetfillcolor{currentfill}%
\pgfsetfillopacity{0.546040}%
\pgfsetlinewidth{1.003750pt}%
\definecolor{currentstroke}{rgb}{0.121569,0.466667,0.705882}%
\pgfsetstrokecolor{currentstroke}%
\pgfsetstrokeopacity{0.546040}%
\pgfsetdash{}{0pt}%
\pgfpathmoveto{\pgfqpoint{3.284888in}{2.244997in}}%
\pgfpathcurveto{\pgfqpoint{3.293124in}{2.244997in}}{\pgfqpoint{3.301024in}{2.248269in}}{\pgfqpoint{3.306848in}{2.254093in}}%
\pgfpathcurveto{\pgfqpoint{3.312672in}{2.259917in}}{\pgfqpoint{3.315944in}{2.267817in}}{\pgfqpoint{3.315944in}{2.276053in}}%
\pgfpathcurveto{\pgfqpoint{3.315944in}{2.284289in}}{\pgfqpoint{3.312672in}{2.292189in}}{\pgfqpoint{3.306848in}{2.298013in}}%
\pgfpathcurveto{\pgfqpoint{3.301024in}{2.303837in}}{\pgfqpoint{3.293124in}{2.307110in}}{\pgfqpoint{3.284888in}{2.307110in}}%
\pgfpathcurveto{\pgfqpoint{3.276652in}{2.307110in}}{\pgfqpoint{3.268752in}{2.303837in}}{\pgfqpoint{3.262928in}{2.298013in}}%
\pgfpathcurveto{\pgfqpoint{3.257104in}{2.292189in}}{\pgfqpoint{3.253831in}{2.284289in}}{\pgfqpoint{3.253831in}{2.276053in}}%
\pgfpathcurveto{\pgfqpoint{3.253831in}{2.267817in}}{\pgfqpoint{3.257104in}{2.259917in}}{\pgfqpoint{3.262928in}{2.254093in}}%
\pgfpathcurveto{\pgfqpoint{3.268752in}{2.248269in}}{\pgfqpoint{3.276652in}{2.244997in}}{\pgfqpoint{3.284888in}{2.244997in}}%
\pgfpathclose%
\pgfusepath{stroke,fill}%
\end{pgfscope}%
\begin{pgfscope}%
\pgfpathrectangle{\pgfqpoint{0.100000in}{0.212622in}}{\pgfqpoint{3.696000in}{3.696000in}}%
\pgfusepath{clip}%
\pgfsetbuttcap%
\pgfsetroundjoin%
\definecolor{currentfill}{rgb}{0.121569,0.466667,0.705882}%
\pgfsetfillcolor{currentfill}%
\pgfsetfillopacity{0.546396}%
\pgfsetlinewidth{1.003750pt}%
\definecolor{currentstroke}{rgb}{0.121569,0.466667,0.705882}%
\pgfsetstrokecolor{currentstroke}%
\pgfsetstrokeopacity{0.546396}%
\pgfsetdash{}{0pt}%
\pgfpathmoveto{\pgfqpoint{3.284560in}{2.244440in}}%
\pgfpathcurveto{\pgfqpoint{3.292796in}{2.244440in}}{\pgfqpoint{3.300696in}{2.247712in}}{\pgfqpoint{3.306520in}{2.253536in}}%
\pgfpathcurveto{\pgfqpoint{3.312344in}{2.259360in}}{\pgfqpoint{3.315617in}{2.267260in}}{\pgfqpoint{3.315617in}{2.275497in}}%
\pgfpathcurveto{\pgfqpoint{3.315617in}{2.283733in}}{\pgfqpoint{3.312344in}{2.291633in}}{\pgfqpoint{3.306520in}{2.297457in}}%
\pgfpathcurveto{\pgfqpoint{3.300696in}{2.303281in}}{\pgfqpoint{3.292796in}{2.306553in}}{\pgfqpoint{3.284560in}{2.306553in}}%
\pgfpathcurveto{\pgfqpoint{3.276324in}{2.306553in}}{\pgfqpoint{3.268424in}{2.303281in}}{\pgfqpoint{3.262600in}{2.297457in}}%
\pgfpathcurveto{\pgfqpoint{3.256776in}{2.291633in}}{\pgfqpoint{3.253504in}{2.283733in}}{\pgfqpoint{3.253504in}{2.275497in}}%
\pgfpathcurveto{\pgfqpoint{3.253504in}{2.267260in}}{\pgfqpoint{3.256776in}{2.259360in}}{\pgfqpoint{3.262600in}{2.253536in}}%
\pgfpathcurveto{\pgfqpoint{3.268424in}{2.247712in}}{\pgfqpoint{3.276324in}{2.244440in}}{\pgfqpoint{3.284560in}{2.244440in}}%
\pgfpathclose%
\pgfusepath{stroke,fill}%
\end{pgfscope}%
\begin{pgfscope}%
\pgfpathrectangle{\pgfqpoint{0.100000in}{0.212622in}}{\pgfqpoint{3.696000in}{3.696000in}}%
\pgfusepath{clip}%
\pgfsetbuttcap%
\pgfsetroundjoin%
\definecolor{currentfill}{rgb}{0.121569,0.466667,0.705882}%
\pgfsetfillcolor{currentfill}%
\pgfsetfillopacity{0.546593}%
\pgfsetlinewidth{1.003750pt}%
\definecolor{currentstroke}{rgb}{0.121569,0.466667,0.705882}%
\pgfsetstrokecolor{currentstroke}%
\pgfsetstrokeopacity{0.546593}%
\pgfsetdash{}{0pt}%
\pgfpathmoveto{\pgfqpoint{3.284364in}{2.244148in}}%
\pgfpathcurveto{\pgfqpoint{3.292601in}{2.244148in}}{\pgfqpoint{3.300501in}{2.247420in}}{\pgfqpoint{3.306325in}{2.253244in}}%
\pgfpathcurveto{\pgfqpoint{3.312149in}{2.259068in}}{\pgfqpoint{3.315421in}{2.266968in}}{\pgfqpoint{3.315421in}{2.275204in}}%
\pgfpathcurveto{\pgfqpoint{3.315421in}{2.283441in}}{\pgfqpoint{3.312149in}{2.291341in}}{\pgfqpoint{3.306325in}{2.297165in}}%
\pgfpathcurveto{\pgfqpoint{3.300501in}{2.302989in}}{\pgfqpoint{3.292601in}{2.306261in}}{\pgfqpoint{3.284364in}{2.306261in}}%
\pgfpathcurveto{\pgfqpoint{3.276128in}{2.306261in}}{\pgfqpoint{3.268228in}{2.302989in}}{\pgfqpoint{3.262404in}{2.297165in}}%
\pgfpathcurveto{\pgfqpoint{3.256580in}{2.291341in}}{\pgfqpoint{3.253308in}{2.283441in}}{\pgfqpoint{3.253308in}{2.275204in}}%
\pgfpathcurveto{\pgfqpoint{3.253308in}{2.266968in}}{\pgfqpoint{3.256580in}{2.259068in}}{\pgfqpoint{3.262404in}{2.253244in}}%
\pgfpathcurveto{\pgfqpoint{3.268228in}{2.247420in}}{\pgfqpoint{3.276128in}{2.244148in}}{\pgfqpoint{3.284364in}{2.244148in}}%
\pgfpathclose%
\pgfusepath{stroke,fill}%
\end{pgfscope}%
\begin{pgfscope}%
\pgfpathrectangle{\pgfqpoint{0.100000in}{0.212622in}}{\pgfqpoint{3.696000in}{3.696000in}}%
\pgfusepath{clip}%
\pgfsetbuttcap%
\pgfsetroundjoin%
\definecolor{currentfill}{rgb}{0.121569,0.466667,0.705882}%
\pgfsetfillcolor{currentfill}%
\pgfsetfillopacity{0.547139}%
\pgfsetlinewidth{1.003750pt}%
\definecolor{currentstroke}{rgb}{0.121569,0.466667,0.705882}%
\pgfsetstrokecolor{currentstroke}%
\pgfsetstrokeopacity{0.547139}%
\pgfsetdash{}{0pt}%
\pgfpathmoveto{\pgfqpoint{3.283696in}{2.243516in}}%
\pgfpathcurveto{\pgfqpoint{3.291932in}{2.243516in}}{\pgfqpoint{3.299832in}{2.246788in}}{\pgfqpoint{3.305656in}{2.252612in}}%
\pgfpathcurveto{\pgfqpoint{3.311480in}{2.258436in}}{\pgfqpoint{3.314753in}{2.266336in}}{\pgfqpoint{3.314753in}{2.274572in}}%
\pgfpathcurveto{\pgfqpoint{3.314753in}{2.282809in}}{\pgfqpoint{3.311480in}{2.290709in}}{\pgfqpoint{3.305656in}{2.296533in}}%
\pgfpathcurveto{\pgfqpoint{3.299832in}{2.302357in}}{\pgfqpoint{3.291932in}{2.305629in}}{\pgfqpoint{3.283696in}{2.305629in}}%
\pgfpathcurveto{\pgfqpoint{3.275460in}{2.305629in}}{\pgfqpoint{3.267560in}{2.302357in}}{\pgfqpoint{3.261736in}{2.296533in}}%
\pgfpathcurveto{\pgfqpoint{3.255912in}{2.290709in}}{\pgfqpoint{3.252640in}{2.282809in}}{\pgfqpoint{3.252640in}{2.274572in}}%
\pgfpathcurveto{\pgfqpoint{3.252640in}{2.266336in}}{\pgfqpoint{3.255912in}{2.258436in}}{\pgfqpoint{3.261736in}{2.252612in}}%
\pgfpathcurveto{\pgfqpoint{3.267560in}{2.246788in}}{\pgfqpoint{3.275460in}{2.243516in}}{\pgfqpoint{3.283696in}{2.243516in}}%
\pgfpathclose%
\pgfusepath{stroke,fill}%
\end{pgfscope}%
\begin{pgfscope}%
\pgfpathrectangle{\pgfqpoint{0.100000in}{0.212622in}}{\pgfqpoint{3.696000in}{3.696000in}}%
\pgfusepath{clip}%
\pgfsetbuttcap%
\pgfsetroundjoin%
\definecolor{currentfill}{rgb}{0.121569,0.466667,0.705882}%
\pgfsetfillcolor{currentfill}%
\pgfsetfillopacity{0.547724}%
\pgfsetlinewidth{1.003750pt}%
\definecolor{currentstroke}{rgb}{0.121569,0.466667,0.705882}%
\pgfsetstrokecolor{currentstroke}%
\pgfsetstrokeopacity{0.547724}%
\pgfsetdash{}{0pt}%
\pgfpathmoveto{\pgfqpoint{0.937769in}{1.635279in}}%
\pgfpathcurveto{\pgfqpoint{0.946005in}{1.635279in}}{\pgfqpoint{0.953905in}{1.638551in}}{\pgfqpoint{0.959729in}{1.644375in}}%
\pgfpathcurveto{\pgfqpoint{0.965553in}{1.650199in}}{\pgfqpoint{0.968825in}{1.658099in}}{\pgfqpoint{0.968825in}{1.666335in}}%
\pgfpathcurveto{\pgfqpoint{0.968825in}{1.674571in}}{\pgfqpoint{0.965553in}{1.682471in}}{\pgfqpoint{0.959729in}{1.688295in}}%
\pgfpathcurveto{\pgfqpoint{0.953905in}{1.694119in}}{\pgfqpoint{0.946005in}{1.697392in}}{\pgfqpoint{0.937769in}{1.697392in}}%
\pgfpathcurveto{\pgfqpoint{0.929533in}{1.697392in}}{\pgfqpoint{0.921632in}{1.694119in}}{\pgfqpoint{0.915809in}{1.688295in}}%
\pgfpathcurveto{\pgfqpoint{0.909985in}{1.682471in}}{\pgfqpoint{0.906712in}{1.674571in}}{\pgfqpoint{0.906712in}{1.666335in}}%
\pgfpathcurveto{\pgfqpoint{0.906712in}{1.658099in}}{\pgfqpoint{0.909985in}{1.650199in}}{\pgfqpoint{0.915809in}{1.644375in}}%
\pgfpathcurveto{\pgfqpoint{0.921632in}{1.638551in}}{\pgfqpoint{0.929533in}{1.635279in}}{\pgfqpoint{0.937769in}{1.635279in}}%
\pgfpathclose%
\pgfusepath{stroke,fill}%
\end{pgfscope}%
\begin{pgfscope}%
\pgfpathrectangle{\pgfqpoint{0.100000in}{0.212622in}}{\pgfqpoint{3.696000in}{3.696000in}}%
\pgfusepath{clip}%
\pgfsetbuttcap%
\pgfsetroundjoin%
\definecolor{currentfill}{rgb}{0.121569,0.466667,0.705882}%
\pgfsetfillcolor{currentfill}%
\pgfsetfillopacity{0.547948}%
\pgfsetlinewidth{1.003750pt}%
\definecolor{currentstroke}{rgb}{0.121569,0.466667,0.705882}%
\pgfsetstrokecolor{currentstroke}%
\pgfsetstrokeopacity{0.547948}%
\pgfsetdash{}{0pt}%
\pgfpathmoveto{\pgfqpoint{3.282597in}{2.242858in}}%
\pgfpathcurveto{\pgfqpoint{3.290834in}{2.242858in}}{\pgfqpoint{3.298734in}{2.246130in}}{\pgfqpoint{3.304558in}{2.251954in}}%
\pgfpathcurveto{\pgfqpoint{3.310381in}{2.257778in}}{\pgfqpoint{3.313654in}{2.265678in}}{\pgfqpoint{3.313654in}{2.273915in}}%
\pgfpathcurveto{\pgfqpoint{3.313654in}{2.282151in}}{\pgfqpoint{3.310381in}{2.290051in}}{\pgfqpoint{3.304558in}{2.295875in}}%
\pgfpathcurveto{\pgfqpoint{3.298734in}{2.301699in}}{\pgfqpoint{3.290834in}{2.304971in}}{\pgfqpoint{3.282597in}{2.304971in}}%
\pgfpathcurveto{\pgfqpoint{3.274361in}{2.304971in}}{\pgfqpoint{3.266461in}{2.301699in}}{\pgfqpoint{3.260637in}{2.295875in}}%
\pgfpathcurveto{\pgfqpoint{3.254813in}{2.290051in}}{\pgfqpoint{3.251541in}{2.282151in}}{\pgfqpoint{3.251541in}{2.273915in}}%
\pgfpathcurveto{\pgfqpoint{3.251541in}{2.265678in}}{\pgfqpoint{3.254813in}{2.257778in}}{\pgfqpoint{3.260637in}{2.251954in}}%
\pgfpathcurveto{\pgfqpoint{3.266461in}{2.246130in}}{\pgfqpoint{3.274361in}{2.242858in}}{\pgfqpoint{3.282597in}{2.242858in}}%
\pgfpathclose%
\pgfusepath{stroke,fill}%
\end{pgfscope}%
\begin{pgfscope}%
\pgfpathrectangle{\pgfqpoint{0.100000in}{0.212622in}}{\pgfqpoint{3.696000in}{3.696000in}}%
\pgfusepath{clip}%
\pgfsetbuttcap%
\pgfsetroundjoin%
\definecolor{currentfill}{rgb}{0.121569,0.466667,0.705882}%
\pgfsetfillcolor{currentfill}%
\pgfsetfillopacity{0.548950}%
\pgfsetlinewidth{1.003750pt}%
\definecolor{currentstroke}{rgb}{0.121569,0.466667,0.705882}%
\pgfsetstrokecolor{currentstroke}%
\pgfsetstrokeopacity{0.548950}%
\pgfsetdash{}{0pt}%
\pgfpathmoveto{\pgfqpoint{3.281250in}{2.242230in}}%
\pgfpathcurveto{\pgfqpoint{3.289486in}{2.242230in}}{\pgfqpoint{3.297386in}{2.245503in}}{\pgfqpoint{3.303210in}{2.251327in}}%
\pgfpathcurveto{\pgfqpoint{3.309034in}{2.257150in}}{\pgfqpoint{3.312306in}{2.265051in}}{\pgfqpoint{3.312306in}{2.273287in}}%
\pgfpathcurveto{\pgfqpoint{3.312306in}{2.281523in}}{\pgfqpoint{3.309034in}{2.289423in}}{\pgfqpoint{3.303210in}{2.295247in}}%
\pgfpathcurveto{\pgfqpoint{3.297386in}{2.301071in}}{\pgfqpoint{3.289486in}{2.304343in}}{\pgfqpoint{3.281250in}{2.304343in}}%
\pgfpathcurveto{\pgfqpoint{3.273014in}{2.304343in}}{\pgfqpoint{3.265114in}{2.301071in}}{\pgfqpoint{3.259290in}{2.295247in}}%
\pgfpathcurveto{\pgfqpoint{3.253466in}{2.289423in}}{\pgfqpoint{3.250193in}{2.281523in}}{\pgfqpoint{3.250193in}{2.273287in}}%
\pgfpathcurveto{\pgfqpoint{3.250193in}{2.265051in}}{\pgfqpoint{3.253466in}{2.257150in}}{\pgfqpoint{3.259290in}{2.251327in}}%
\pgfpathcurveto{\pgfqpoint{3.265114in}{2.245503in}}{\pgfqpoint{3.273014in}{2.242230in}}{\pgfqpoint{3.281250in}{2.242230in}}%
\pgfpathclose%
\pgfusepath{stroke,fill}%
\end{pgfscope}%
\begin{pgfscope}%
\pgfpathrectangle{\pgfqpoint{0.100000in}{0.212622in}}{\pgfqpoint{3.696000in}{3.696000in}}%
\pgfusepath{clip}%
\pgfsetbuttcap%
\pgfsetroundjoin%
\definecolor{currentfill}{rgb}{0.121569,0.466667,0.705882}%
\pgfsetfillcolor{currentfill}%
\pgfsetfillopacity{0.548971}%
\pgfsetlinewidth{1.003750pt}%
\definecolor{currentstroke}{rgb}{0.121569,0.466667,0.705882}%
\pgfsetstrokecolor{currentstroke}%
\pgfsetstrokeopacity{0.548971}%
\pgfsetdash{}{0pt}%
\pgfpathmoveto{\pgfqpoint{0.934045in}{1.629803in}}%
\pgfpathcurveto{\pgfqpoint{0.942281in}{1.629803in}}{\pgfqpoint{0.950181in}{1.633075in}}{\pgfqpoint{0.956005in}{1.638899in}}%
\pgfpathcurveto{\pgfqpoint{0.961829in}{1.644723in}}{\pgfqpoint{0.965102in}{1.652623in}}{\pgfqpoint{0.965102in}{1.660859in}}%
\pgfpathcurveto{\pgfqpoint{0.965102in}{1.669095in}}{\pgfqpoint{0.961829in}{1.676995in}}{\pgfqpoint{0.956005in}{1.682819in}}%
\pgfpathcurveto{\pgfqpoint{0.950181in}{1.688643in}}{\pgfqpoint{0.942281in}{1.691916in}}{\pgfqpoint{0.934045in}{1.691916in}}%
\pgfpathcurveto{\pgfqpoint{0.925809in}{1.691916in}}{\pgfqpoint{0.917909in}{1.688643in}}{\pgfqpoint{0.912085in}{1.682819in}}%
\pgfpathcurveto{\pgfqpoint{0.906261in}{1.676995in}}{\pgfqpoint{0.902989in}{1.669095in}}{\pgfqpoint{0.902989in}{1.660859in}}%
\pgfpathcurveto{\pgfqpoint{0.902989in}{1.652623in}}{\pgfqpoint{0.906261in}{1.644723in}}{\pgfqpoint{0.912085in}{1.638899in}}%
\pgfpathcurveto{\pgfqpoint{0.917909in}{1.633075in}}{\pgfqpoint{0.925809in}{1.629803in}}{\pgfqpoint{0.934045in}{1.629803in}}%
\pgfpathclose%
\pgfusepath{stroke,fill}%
\end{pgfscope}%
\begin{pgfscope}%
\pgfpathrectangle{\pgfqpoint{0.100000in}{0.212622in}}{\pgfqpoint{3.696000in}{3.696000in}}%
\pgfusepath{clip}%
\pgfsetbuttcap%
\pgfsetroundjoin%
\definecolor{currentfill}{rgb}{0.121569,0.466667,0.705882}%
\pgfsetfillcolor{currentfill}%
\pgfsetfillopacity{0.549822}%
\pgfsetlinewidth{1.003750pt}%
\definecolor{currentstroke}{rgb}{0.121569,0.466667,0.705882}%
\pgfsetstrokecolor{currentstroke}%
\pgfsetstrokeopacity{0.549822}%
\pgfsetdash{}{0pt}%
\pgfpathmoveto{\pgfqpoint{0.931629in}{1.626315in}}%
\pgfpathcurveto{\pgfqpoint{0.939865in}{1.626315in}}{\pgfqpoint{0.947765in}{1.629587in}}{\pgfqpoint{0.953589in}{1.635411in}}%
\pgfpathcurveto{\pgfqpoint{0.959413in}{1.641235in}}{\pgfqpoint{0.962685in}{1.649135in}}{\pgfqpoint{0.962685in}{1.657372in}}%
\pgfpathcurveto{\pgfqpoint{0.962685in}{1.665608in}}{\pgfqpoint{0.959413in}{1.673508in}}{\pgfqpoint{0.953589in}{1.679332in}}%
\pgfpathcurveto{\pgfqpoint{0.947765in}{1.685156in}}{\pgfqpoint{0.939865in}{1.688428in}}{\pgfqpoint{0.931629in}{1.688428in}}%
\pgfpathcurveto{\pgfqpoint{0.923392in}{1.688428in}}{\pgfqpoint{0.915492in}{1.685156in}}{\pgfqpoint{0.909668in}{1.679332in}}%
\pgfpathcurveto{\pgfqpoint{0.903844in}{1.673508in}}{\pgfqpoint{0.900572in}{1.665608in}}{\pgfqpoint{0.900572in}{1.657372in}}%
\pgfpathcurveto{\pgfqpoint{0.900572in}{1.649135in}}{\pgfqpoint{0.903844in}{1.641235in}}{\pgfqpoint{0.909668in}{1.635411in}}%
\pgfpathcurveto{\pgfqpoint{0.915492in}{1.629587in}}{\pgfqpoint{0.923392in}{1.626315in}}{\pgfqpoint{0.931629in}{1.626315in}}%
\pgfpathclose%
\pgfusepath{stroke,fill}%
\end{pgfscope}%
\begin{pgfscope}%
\pgfpathrectangle{\pgfqpoint{0.100000in}{0.212622in}}{\pgfqpoint{3.696000in}{3.696000in}}%
\pgfusepath{clip}%
\pgfsetbuttcap%
\pgfsetroundjoin%
\definecolor{currentfill}{rgb}{0.121569,0.466667,0.705882}%
\pgfsetfillcolor{currentfill}%
\pgfsetfillopacity{0.550072}%
\pgfsetlinewidth{1.003750pt}%
\definecolor{currentstroke}{rgb}{0.121569,0.466667,0.705882}%
\pgfsetstrokecolor{currentstroke}%
\pgfsetstrokeopacity{0.550072}%
\pgfsetdash{}{0pt}%
\pgfpathmoveto{\pgfqpoint{3.279474in}{2.241423in}}%
\pgfpathcurveto{\pgfqpoint{3.287710in}{2.241423in}}{\pgfqpoint{3.295610in}{2.244695in}}{\pgfqpoint{3.301434in}{2.250519in}}%
\pgfpathcurveto{\pgfqpoint{3.307258in}{2.256343in}}{\pgfqpoint{3.310530in}{2.264243in}}{\pgfqpoint{3.310530in}{2.272479in}}%
\pgfpathcurveto{\pgfqpoint{3.310530in}{2.280716in}}{\pgfqpoint{3.307258in}{2.288616in}}{\pgfqpoint{3.301434in}{2.294440in}}%
\pgfpathcurveto{\pgfqpoint{3.295610in}{2.300263in}}{\pgfqpoint{3.287710in}{2.303536in}}{\pgfqpoint{3.279474in}{2.303536in}}%
\pgfpathcurveto{\pgfqpoint{3.271238in}{2.303536in}}{\pgfqpoint{3.263338in}{2.300263in}}{\pgfqpoint{3.257514in}{2.294440in}}%
\pgfpathcurveto{\pgfqpoint{3.251690in}{2.288616in}}{\pgfqpoint{3.248417in}{2.280716in}}{\pgfqpoint{3.248417in}{2.272479in}}%
\pgfpathcurveto{\pgfqpoint{3.248417in}{2.264243in}}{\pgfqpoint{3.251690in}{2.256343in}}{\pgfqpoint{3.257514in}{2.250519in}}%
\pgfpathcurveto{\pgfqpoint{3.263338in}{2.244695in}}{\pgfqpoint{3.271238in}{2.241423in}}{\pgfqpoint{3.279474in}{2.241423in}}%
\pgfpathclose%
\pgfusepath{stroke,fill}%
\end{pgfscope}%
\begin{pgfscope}%
\pgfpathrectangle{\pgfqpoint{0.100000in}{0.212622in}}{\pgfqpoint{3.696000in}{3.696000in}}%
\pgfusepath{clip}%
\pgfsetbuttcap%
\pgfsetroundjoin%
\definecolor{currentfill}{rgb}{0.121569,0.466667,0.705882}%
\pgfsetfillcolor{currentfill}%
\pgfsetfillopacity{0.550406}%
\pgfsetlinewidth{1.003750pt}%
\definecolor{currentstroke}{rgb}{0.121569,0.466667,0.705882}%
\pgfsetstrokecolor{currentstroke}%
\pgfsetstrokeopacity{0.550406}%
\pgfsetdash{}{0pt}%
\pgfpathmoveto{\pgfqpoint{0.929865in}{1.623814in}}%
\pgfpathcurveto{\pgfqpoint{0.938102in}{1.623814in}}{\pgfqpoint{0.946002in}{1.627087in}}{\pgfqpoint{0.951826in}{1.632911in}}%
\pgfpathcurveto{\pgfqpoint{0.957650in}{1.638735in}}{\pgfqpoint{0.960922in}{1.646635in}}{\pgfqpoint{0.960922in}{1.654871in}}%
\pgfpathcurveto{\pgfqpoint{0.960922in}{1.663107in}}{\pgfqpoint{0.957650in}{1.671007in}}{\pgfqpoint{0.951826in}{1.676831in}}%
\pgfpathcurveto{\pgfqpoint{0.946002in}{1.682655in}}{\pgfqpoint{0.938102in}{1.685927in}}{\pgfqpoint{0.929865in}{1.685927in}}%
\pgfpathcurveto{\pgfqpoint{0.921629in}{1.685927in}}{\pgfqpoint{0.913729in}{1.682655in}}{\pgfqpoint{0.907905in}{1.676831in}}%
\pgfpathcurveto{\pgfqpoint{0.902081in}{1.671007in}}{\pgfqpoint{0.898809in}{1.663107in}}{\pgfqpoint{0.898809in}{1.654871in}}%
\pgfpathcurveto{\pgfqpoint{0.898809in}{1.646635in}}{\pgfqpoint{0.902081in}{1.638735in}}{\pgfqpoint{0.907905in}{1.632911in}}%
\pgfpathcurveto{\pgfqpoint{0.913729in}{1.627087in}}{\pgfqpoint{0.921629in}{1.623814in}}{\pgfqpoint{0.929865in}{1.623814in}}%
\pgfpathclose%
\pgfusepath{stroke,fill}%
\end{pgfscope}%
\begin{pgfscope}%
\pgfpathrectangle{\pgfqpoint{0.100000in}{0.212622in}}{\pgfqpoint{3.696000in}{3.696000in}}%
\pgfusepath{clip}%
\pgfsetbuttcap%
\pgfsetroundjoin%
\definecolor{currentfill}{rgb}{0.121569,0.466667,0.705882}%
\pgfsetfillcolor{currentfill}%
\pgfsetfillopacity{0.550877}%
\pgfsetlinewidth{1.003750pt}%
\definecolor{currentstroke}{rgb}{0.121569,0.466667,0.705882}%
\pgfsetstrokecolor{currentstroke}%
\pgfsetstrokeopacity{0.550877}%
\pgfsetdash{}{0pt}%
\pgfpathmoveto{\pgfqpoint{0.928492in}{1.621926in}}%
\pgfpathcurveto{\pgfqpoint{0.936728in}{1.621926in}}{\pgfqpoint{0.944629in}{1.625198in}}{\pgfqpoint{0.950452in}{1.631022in}}%
\pgfpathcurveto{\pgfqpoint{0.956276in}{1.636846in}}{\pgfqpoint{0.959549in}{1.644746in}}{\pgfqpoint{0.959549in}{1.652982in}}%
\pgfpathcurveto{\pgfqpoint{0.959549in}{1.661219in}}{\pgfqpoint{0.956276in}{1.669119in}}{\pgfqpoint{0.950452in}{1.674943in}}%
\pgfpathcurveto{\pgfqpoint{0.944629in}{1.680766in}}{\pgfqpoint{0.936728in}{1.684039in}}{\pgfqpoint{0.928492in}{1.684039in}}%
\pgfpathcurveto{\pgfqpoint{0.920256in}{1.684039in}}{\pgfqpoint{0.912356in}{1.680766in}}{\pgfqpoint{0.906532in}{1.674943in}}%
\pgfpathcurveto{\pgfqpoint{0.900708in}{1.669119in}}{\pgfqpoint{0.897436in}{1.661219in}}{\pgfqpoint{0.897436in}{1.652982in}}%
\pgfpathcurveto{\pgfqpoint{0.897436in}{1.644746in}}{\pgfqpoint{0.900708in}{1.636846in}}{\pgfqpoint{0.906532in}{1.631022in}}%
\pgfpathcurveto{\pgfqpoint{0.912356in}{1.625198in}}{\pgfqpoint{0.920256in}{1.621926in}}{\pgfqpoint{0.928492in}{1.621926in}}%
\pgfpathclose%
\pgfusepath{stroke,fill}%
\end{pgfscope}%
\begin{pgfscope}%
\pgfpathrectangle{\pgfqpoint{0.100000in}{0.212622in}}{\pgfqpoint{3.696000in}{3.696000in}}%
\pgfusepath{clip}%
\pgfsetbuttcap%
\pgfsetroundjoin%
\definecolor{currentfill}{rgb}{0.121569,0.466667,0.705882}%
\pgfsetfillcolor{currentfill}%
\pgfsetfillopacity{0.551247}%
\pgfsetlinewidth{1.003750pt}%
\definecolor{currentstroke}{rgb}{0.121569,0.466667,0.705882}%
\pgfsetstrokecolor{currentstroke}%
\pgfsetstrokeopacity{0.551247}%
\pgfsetdash{}{0pt}%
\pgfpathmoveto{\pgfqpoint{0.927372in}{1.620370in}}%
\pgfpathcurveto{\pgfqpoint{0.935608in}{1.620370in}}{\pgfqpoint{0.943508in}{1.623643in}}{\pgfqpoint{0.949332in}{1.629467in}}%
\pgfpathcurveto{\pgfqpoint{0.955156in}{1.635291in}}{\pgfqpoint{0.958428in}{1.643191in}}{\pgfqpoint{0.958428in}{1.651427in}}%
\pgfpathcurveto{\pgfqpoint{0.958428in}{1.659663in}}{\pgfqpoint{0.955156in}{1.667563in}}{\pgfqpoint{0.949332in}{1.673387in}}%
\pgfpathcurveto{\pgfqpoint{0.943508in}{1.679211in}}{\pgfqpoint{0.935608in}{1.682483in}}{\pgfqpoint{0.927372in}{1.682483in}}%
\pgfpathcurveto{\pgfqpoint{0.919135in}{1.682483in}}{\pgfqpoint{0.911235in}{1.679211in}}{\pgfqpoint{0.905411in}{1.673387in}}%
\pgfpathcurveto{\pgfqpoint{0.899587in}{1.667563in}}{\pgfqpoint{0.896315in}{1.659663in}}{\pgfqpoint{0.896315in}{1.651427in}}%
\pgfpathcurveto{\pgfqpoint{0.896315in}{1.643191in}}{\pgfqpoint{0.899587in}{1.635291in}}{\pgfqpoint{0.905411in}{1.629467in}}%
\pgfpathcurveto{\pgfqpoint{0.911235in}{1.623643in}}{\pgfqpoint{0.919135in}{1.620370in}}{\pgfqpoint{0.927372in}{1.620370in}}%
\pgfpathclose%
\pgfusepath{stroke,fill}%
\end{pgfscope}%
\begin{pgfscope}%
\pgfpathrectangle{\pgfqpoint{0.100000in}{0.212622in}}{\pgfqpoint{3.696000in}{3.696000in}}%
\pgfusepath{clip}%
\pgfsetbuttcap%
\pgfsetroundjoin%
\definecolor{currentfill}{rgb}{0.121569,0.466667,0.705882}%
\pgfsetfillcolor{currentfill}%
\pgfsetfillopacity{0.551692}%
\pgfsetlinewidth{1.003750pt}%
\definecolor{currentstroke}{rgb}{0.121569,0.466667,0.705882}%
\pgfsetstrokecolor{currentstroke}%
\pgfsetstrokeopacity{0.551692}%
\pgfsetdash{}{0pt}%
\pgfpathmoveto{\pgfqpoint{3.276942in}{2.240060in}}%
\pgfpathcurveto{\pgfqpoint{3.285179in}{2.240060in}}{\pgfqpoint{3.293079in}{2.243332in}}{\pgfqpoint{3.298903in}{2.249156in}}%
\pgfpathcurveto{\pgfqpoint{3.304726in}{2.254980in}}{\pgfqpoint{3.307999in}{2.262880in}}{\pgfqpoint{3.307999in}{2.271116in}}%
\pgfpathcurveto{\pgfqpoint{3.307999in}{2.279352in}}{\pgfqpoint{3.304726in}{2.287252in}}{\pgfqpoint{3.298903in}{2.293076in}}%
\pgfpathcurveto{\pgfqpoint{3.293079in}{2.298900in}}{\pgfqpoint{3.285179in}{2.302173in}}{\pgfqpoint{3.276942in}{2.302173in}}%
\pgfpathcurveto{\pgfqpoint{3.268706in}{2.302173in}}{\pgfqpoint{3.260806in}{2.298900in}}{\pgfqpoint{3.254982in}{2.293076in}}%
\pgfpathcurveto{\pgfqpoint{3.249158in}{2.287252in}}{\pgfqpoint{3.245886in}{2.279352in}}{\pgfqpoint{3.245886in}{2.271116in}}%
\pgfpathcurveto{\pgfqpoint{3.245886in}{2.262880in}}{\pgfqpoint{3.249158in}{2.254980in}}{\pgfqpoint{3.254982in}{2.249156in}}%
\pgfpathcurveto{\pgfqpoint{3.260806in}{2.243332in}}{\pgfqpoint{3.268706in}{2.240060in}}{\pgfqpoint{3.276942in}{2.240060in}}%
\pgfpathclose%
\pgfusepath{stroke,fill}%
\end{pgfscope}%
\begin{pgfscope}%
\pgfpathrectangle{\pgfqpoint{0.100000in}{0.212622in}}{\pgfqpoint{3.696000in}{3.696000in}}%
\pgfusepath{clip}%
\pgfsetbuttcap%
\pgfsetroundjoin%
\definecolor{currentfill}{rgb}{0.121569,0.466667,0.705882}%
\pgfsetfillcolor{currentfill}%
\pgfsetfillopacity{0.551945}%
\pgfsetlinewidth{1.003750pt}%
\definecolor{currentstroke}{rgb}{0.121569,0.466667,0.705882}%
\pgfsetstrokecolor{currentstroke}%
\pgfsetstrokeopacity{0.551945}%
\pgfsetdash{}{0pt}%
\pgfpathmoveto{\pgfqpoint{0.925302in}{1.617729in}}%
\pgfpathcurveto{\pgfqpoint{0.933538in}{1.617729in}}{\pgfqpoint{0.941438in}{1.621002in}}{\pgfqpoint{0.947262in}{1.626826in}}%
\pgfpathcurveto{\pgfqpoint{0.953086in}{1.632650in}}{\pgfqpoint{0.956358in}{1.640550in}}{\pgfqpoint{0.956358in}{1.648786in}}%
\pgfpathcurveto{\pgfqpoint{0.956358in}{1.657022in}}{\pgfqpoint{0.953086in}{1.664922in}}{\pgfqpoint{0.947262in}{1.670746in}}%
\pgfpathcurveto{\pgfqpoint{0.941438in}{1.676570in}}{\pgfqpoint{0.933538in}{1.679842in}}{\pgfqpoint{0.925302in}{1.679842in}}%
\pgfpathcurveto{\pgfqpoint{0.917066in}{1.679842in}}{\pgfqpoint{0.909166in}{1.676570in}}{\pgfqpoint{0.903342in}{1.670746in}}%
\pgfpathcurveto{\pgfqpoint{0.897518in}{1.664922in}}{\pgfqpoint{0.894245in}{1.657022in}}{\pgfqpoint{0.894245in}{1.648786in}}%
\pgfpathcurveto{\pgfqpoint{0.894245in}{1.640550in}}{\pgfqpoint{0.897518in}{1.632650in}}{\pgfqpoint{0.903342in}{1.626826in}}%
\pgfpathcurveto{\pgfqpoint{0.909166in}{1.621002in}}{\pgfqpoint{0.917066in}{1.617729in}}{\pgfqpoint{0.925302in}{1.617729in}}%
\pgfpathclose%
\pgfusepath{stroke,fill}%
\end{pgfscope}%
\begin{pgfscope}%
\pgfpathrectangle{\pgfqpoint{0.100000in}{0.212622in}}{\pgfqpoint{3.696000in}{3.696000in}}%
\pgfusepath{clip}%
\pgfsetbuttcap%
\pgfsetroundjoin%
\definecolor{currentfill}{rgb}{0.121569,0.466667,0.705882}%
\pgfsetfillcolor{currentfill}%
\pgfsetfillopacity{0.552583}%
\pgfsetlinewidth{1.003750pt}%
\definecolor{currentstroke}{rgb}{0.121569,0.466667,0.705882}%
\pgfsetstrokecolor{currentstroke}%
\pgfsetstrokeopacity{0.552583}%
\pgfsetdash{}{0pt}%
\pgfpathmoveto{\pgfqpoint{3.275355in}{2.239514in}}%
\pgfpathcurveto{\pgfqpoint{3.283591in}{2.239514in}}{\pgfqpoint{3.291491in}{2.242787in}}{\pgfqpoint{3.297315in}{2.248611in}}%
\pgfpathcurveto{\pgfqpoint{3.303139in}{2.254434in}}{\pgfqpoint{3.306411in}{2.262335in}}{\pgfqpoint{3.306411in}{2.270571in}}%
\pgfpathcurveto{\pgfqpoint{3.306411in}{2.278807in}}{\pgfqpoint{3.303139in}{2.286707in}}{\pgfqpoint{3.297315in}{2.292531in}}%
\pgfpathcurveto{\pgfqpoint{3.291491in}{2.298355in}}{\pgfqpoint{3.283591in}{2.301627in}}{\pgfqpoint{3.275355in}{2.301627in}}%
\pgfpathcurveto{\pgfqpoint{3.267119in}{2.301627in}}{\pgfqpoint{3.259219in}{2.298355in}}{\pgfqpoint{3.253395in}{2.292531in}}%
\pgfpathcurveto{\pgfqpoint{3.247571in}{2.286707in}}{\pgfqpoint{3.244298in}{2.278807in}}{\pgfqpoint{3.244298in}{2.270571in}}%
\pgfpathcurveto{\pgfqpoint{3.244298in}{2.262335in}}{\pgfqpoint{3.247571in}{2.254434in}}{\pgfqpoint{3.253395in}{2.248611in}}%
\pgfpathcurveto{\pgfqpoint{3.259219in}{2.242787in}}{\pgfqpoint{3.267119in}{2.239514in}}{\pgfqpoint{3.275355in}{2.239514in}}%
\pgfpathclose%
\pgfusepath{stroke,fill}%
\end{pgfscope}%
\begin{pgfscope}%
\pgfpathrectangle{\pgfqpoint{0.100000in}{0.212622in}}{\pgfqpoint{3.696000in}{3.696000in}}%
\pgfusepath{clip}%
\pgfsetbuttcap%
\pgfsetroundjoin%
\definecolor{currentfill}{rgb}{0.121569,0.466667,0.705882}%
\pgfsetfillcolor{currentfill}%
\pgfsetfillopacity{0.553173}%
\pgfsetlinewidth{1.003750pt}%
\definecolor{currentstroke}{rgb}{0.121569,0.466667,0.705882}%
\pgfsetstrokecolor{currentstroke}%
\pgfsetstrokeopacity{0.553173}%
\pgfsetdash{}{0pt}%
\pgfpathmoveto{\pgfqpoint{0.921545in}{1.612659in}}%
\pgfpathcurveto{\pgfqpoint{0.929781in}{1.612659in}}{\pgfqpoint{0.937682in}{1.615931in}}{\pgfqpoint{0.943505in}{1.621755in}}%
\pgfpathcurveto{\pgfqpoint{0.949329in}{1.627579in}}{\pgfqpoint{0.952602in}{1.635479in}}{\pgfqpoint{0.952602in}{1.643715in}}%
\pgfpathcurveto{\pgfqpoint{0.952602in}{1.651951in}}{\pgfqpoint{0.949329in}{1.659851in}}{\pgfqpoint{0.943505in}{1.665675in}}%
\pgfpathcurveto{\pgfqpoint{0.937682in}{1.671499in}}{\pgfqpoint{0.929781in}{1.674772in}}{\pgfqpoint{0.921545in}{1.674772in}}%
\pgfpathcurveto{\pgfqpoint{0.913309in}{1.674772in}}{\pgfqpoint{0.905409in}{1.671499in}}{\pgfqpoint{0.899585in}{1.665675in}}%
\pgfpathcurveto{\pgfqpoint{0.893761in}{1.659851in}}{\pgfqpoint{0.890489in}{1.651951in}}{\pgfqpoint{0.890489in}{1.643715in}}%
\pgfpathcurveto{\pgfqpoint{0.890489in}{1.635479in}}{\pgfqpoint{0.893761in}{1.627579in}}{\pgfqpoint{0.899585in}{1.621755in}}%
\pgfpathcurveto{\pgfqpoint{0.905409in}{1.615931in}}{\pgfqpoint{0.913309in}{1.612659in}}{\pgfqpoint{0.921545in}{1.612659in}}%
\pgfpathclose%
\pgfusepath{stroke,fill}%
\end{pgfscope}%
\begin{pgfscope}%
\pgfpathrectangle{\pgfqpoint{0.100000in}{0.212622in}}{\pgfqpoint{3.696000in}{3.696000in}}%
\pgfusepath{clip}%
\pgfsetbuttcap%
\pgfsetroundjoin%
\definecolor{currentfill}{rgb}{0.121569,0.466667,0.705882}%
\pgfsetfillcolor{currentfill}%
\pgfsetfillopacity{0.553639}%
\pgfsetlinewidth{1.003750pt}%
\definecolor{currentstroke}{rgb}{0.121569,0.466667,0.705882}%
\pgfsetstrokecolor{currentstroke}%
\pgfsetstrokeopacity{0.553639}%
\pgfsetdash{}{0pt}%
\pgfpathmoveto{\pgfqpoint{3.273339in}{2.238675in}}%
\pgfpathcurveto{\pgfqpoint{3.281575in}{2.238675in}}{\pgfqpoint{3.289475in}{2.241948in}}{\pgfqpoint{3.295299in}{2.247771in}}%
\pgfpathcurveto{\pgfqpoint{3.301123in}{2.253595in}}{\pgfqpoint{3.304396in}{2.261495in}}{\pgfqpoint{3.304396in}{2.269732in}}%
\pgfpathcurveto{\pgfqpoint{3.304396in}{2.277968in}}{\pgfqpoint{3.301123in}{2.285868in}}{\pgfqpoint{3.295299in}{2.291692in}}%
\pgfpathcurveto{\pgfqpoint{3.289475in}{2.297516in}}{\pgfqpoint{3.281575in}{2.300788in}}{\pgfqpoint{3.273339in}{2.300788in}}%
\pgfpathcurveto{\pgfqpoint{3.265103in}{2.300788in}}{\pgfqpoint{3.257203in}{2.297516in}}{\pgfqpoint{3.251379in}{2.291692in}}%
\pgfpathcurveto{\pgfqpoint{3.245555in}{2.285868in}}{\pgfqpoint{3.242283in}{2.277968in}}{\pgfqpoint{3.242283in}{2.269732in}}%
\pgfpathcurveto{\pgfqpoint{3.242283in}{2.261495in}}{\pgfqpoint{3.245555in}{2.253595in}}{\pgfqpoint{3.251379in}{2.247771in}}%
\pgfpathcurveto{\pgfqpoint{3.257203in}{2.241948in}}{\pgfqpoint{3.265103in}{2.238675in}}{\pgfqpoint{3.273339in}{2.238675in}}%
\pgfpathclose%
\pgfusepath{stroke,fill}%
\end{pgfscope}%
\begin{pgfscope}%
\pgfpathrectangle{\pgfqpoint{0.100000in}{0.212622in}}{\pgfqpoint{3.696000in}{3.696000in}}%
\pgfusepath{clip}%
\pgfsetbuttcap%
\pgfsetroundjoin%
\definecolor{currentfill}{rgb}{0.121569,0.466667,0.705882}%
\pgfsetfillcolor{currentfill}%
\pgfsetfillopacity{0.554320}%
\pgfsetlinewidth{1.003750pt}%
\definecolor{currentstroke}{rgb}{0.121569,0.466667,0.705882}%
\pgfsetstrokecolor{currentstroke}%
\pgfsetstrokeopacity{0.554320}%
\pgfsetdash{}{0pt}%
\pgfpathmoveto{\pgfqpoint{0.918050in}{1.608036in}}%
\pgfpathcurveto{\pgfqpoint{0.926286in}{1.608036in}}{\pgfqpoint{0.934187in}{1.611309in}}{\pgfqpoint{0.940010in}{1.617133in}}%
\pgfpathcurveto{\pgfqpoint{0.945834in}{1.622957in}}{\pgfqpoint{0.949107in}{1.630857in}}{\pgfqpoint{0.949107in}{1.639093in}}%
\pgfpathcurveto{\pgfqpoint{0.949107in}{1.647329in}}{\pgfqpoint{0.945834in}{1.655229in}}{\pgfqpoint{0.940010in}{1.661053in}}%
\pgfpathcurveto{\pgfqpoint{0.934187in}{1.666877in}}{\pgfqpoint{0.926286in}{1.670149in}}{\pgfqpoint{0.918050in}{1.670149in}}%
\pgfpathcurveto{\pgfqpoint{0.909814in}{1.670149in}}{\pgfqpoint{0.901914in}{1.666877in}}{\pgfqpoint{0.896090in}{1.661053in}}%
\pgfpathcurveto{\pgfqpoint{0.890266in}{1.655229in}}{\pgfqpoint{0.886994in}{1.647329in}}{\pgfqpoint{0.886994in}{1.639093in}}%
\pgfpathcurveto{\pgfqpoint{0.886994in}{1.630857in}}{\pgfqpoint{0.890266in}{1.622957in}}{\pgfqpoint{0.896090in}{1.617133in}}%
\pgfpathcurveto{\pgfqpoint{0.901914in}{1.611309in}}{\pgfqpoint{0.909814in}{1.608036in}}{\pgfqpoint{0.918050in}{1.608036in}}%
\pgfpathclose%
\pgfusepath{stroke,fill}%
\end{pgfscope}%
\begin{pgfscope}%
\pgfpathrectangle{\pgfqpoint{0.100000in}{0.212622in}}{\pgfqpoint{3.696000in}{3.696000in}}%
\pgfusepath{clip}%
\pgfsetbuttcap%
\pgfsetroundjoin%
\definecolor{currentfill}{rgb}{0.121569,0.466667,0.705882}%
\pgfsetfillcolor{currentfill}%
\pgfsetfillopacity{0.554832}%
\pgfsetlinewidth{1.003750pt}%
\definecolor{currentstroke}{rgb}{0.121569,0.466667,0.705882}%
\pgfsetstrokecolor{currentstroke}%
\pgfsetstrokeopacity{0.554832}%
\pgfsetdash{}{0pt}%
\pgfpathmoveto{\pgfqpoint{3.271155in}{2.237720in}}%
\pgfpathcurveto{\pgfqpoint{3.279391in}{2.237720in}}{\pgfqpoint{3.287291in}{2.240993in}}{\pgfqpoint{3.293115in}{2.246817in}}%
\pgfpathcurveto{\pgfqpoint{3.298939in}{2.252641in}}{\pgfqpoint{3.302211in}{2.260541in}}{\pgfqpoint{3.302211in}{2.268777in}}%
\pgfpathcurveto{\pgfqpoint{3.302211in}{2.277013in}}{\pgfqpoint{3.298939in}{2.284913in}}{\pgfqpoint{3.293115in}{2.290737in}}%
\pgfpathcurveto{\pgfqpoint{3.287291in}{2.296561in}}{\pgfqpoint{3.279391in}{2.299833in}}{\pgfqpoint{3.271155in}{2.299833in}}%
\pgfpathcurveto{\pgfqpoint{3.262919in}{2.299833in}}{\pgfqpoint{3.255019in}{2.296561in}}{\pgfqpoint{3.249195in}{2.290737in}}%
\pgfpathcurveto{\pgfqpoint{3.243371in}{2.284913in}}{\pgfqpoint{3.240098in}{2.277013in}}{\pgfqpoint{3.240098in}{2.268777in}}%
\pgfpathcurveto{\pgfqpoint{3.240098in}{2.260541in}}{\pgfqpoint{3.243371in}{2.252641in}}{\pgfqpoint{3.249195in}{2.246817in}}%
\pgfpathcurveto{\pgfqpoint{3.255019in}{2.240993in}}{\pgfqpoint{3.262919in}{2.237720in}}{\pgfqpoint{3.271155in}{2.237720in}}%
\pgfpathclose%
\pgfusepath{stroke,fill}%
\end{pgfscope}%
\begin{pgfscope}%
\pgfpathrectangle{\pgfqpoint{0.100000in}{0.212622in}}{\pgfqpoint{3.696000in}{3.696000in}}%
\pgfusepath{clip}%
\pgfsetbuttcap%
\pgfsetroundjoin%
\definecolor{currentfill}{rgb}{0.121569,0.466667,0.705882}%
\pgfsetfillcolor{currentfill}%
\pgfsetfillopacity{0.555320}%
\pgfsetlinewidth{1.003750pt}%
\definecolor{currentstroke}{rgb}{0.121569,0.466667,0.705882}%
\pgfsetstrokecolor{currentstroke}%
\pgfsetstrokeopacity{0.555320}%
\pgfsetdash{}{0pt}%
\pgfpathmoveto{\pgfqpoint{0.914903in}{1.603527in}}%
\pgfpathcurveto{\pgfqpoint{0.923139in}{1.603527in}}{\pgfqpoint{0.931039in}{1.606799in}}{\pgfqpoint{0.936863in}{1.612623in}}%
\pgfpathcurveto{\pgfqpoint{0.942687in}{1.618447in}}{\pgfqpoint{0.945959in}{1.626347in}}{\pgfqpoint{0.945959in}{1.634583in}}%
\pgfpathcurveto{\pgfqpoint{0.945959in}{1.642819in}}{\pgfqpoint{0.942687in}{1.650719in}}{\pgfqpoint{0.936863in}{1.656543in}}%
\pgfpathcurveto{\pgfqpoint{0.931039in}{1.662367in}}{\pgfqpoint{0.923139in}{1.665640in}}{\pgfqpoint{0.914903in}{1.665640in}}%
\pgfpathcurveto{\pgfqpoint{0.906666in}{1.665640in}}{\pgfqpoint{0.898766in}{1.662367in}}{\pgfqpoint{0.892942in}{1.656543in}}%
\pgfpathcurveto{\pgfqpoint{0.887118in}{1.650719in}}{\pgfqpoint{0.883846in}{1.642819in}}{\pgfqpoint{0.883846in}{1.634583in}}%
\pgfpathcurveto{\pgfqpoint{0.883846in}{1.626347in}}{\pgfqpoint{0.887118in}{1.618447in}}{\pgfqpoint{0.892942in}{1.612623in}}%
\pgfpathcurveto{\pgfqpoint{0.898766in}{1.606799in}}{\pgfqpoint{0.906666in}{1.603527in}}{\pgfqpoint{0.914903in}{1.603527in}}%
\pgfpathclose%
\pgfusepath{stroke,fill}%
\end{pgfscope}%
\begin{pgfscope}%
\pgfpathrectangle{\pgfqpoint{0.100000in}{0.212622in}}{\pgfqpoint{3.696000in}{3.696000in}}%
\pgfusepath{clip}%
\pgfsetbuttcap%
\pgfsetroundjoin%
\definecolor{currentfill}{rgb}{0.121569,0.466667,0.705882}%
\pgfsetfillcolor{currentfill}%
\pgfsetfillopacity{0.556212}%
\pgfsetlinewidth{1.003750pt}%
\definecolor{currentstroke}{rgb}{0.121569,0.466667,0.705882}%
\pgfsetstrokecolor{currentstroke}%
\pgfsetstrokeopacity{0.556212}%
\pgfsetdash{}{0pt}%
\pgfpathmoveto{\pgfqpoint{0.912144in}{1.599771in}}%
\pgfpathcurveto{\pgfqpoint{0.920380in}{1.599771in}}{\pgfqpoint{0.928280in}{1.603043in}}{\pgfqpoint{0.934104in}{1.608867in}}%
\pgfpathcurveto{\pgfqpoint{0.939928in}{1.614691in}}{\pgfqpoint{0.943200in}{1.622591in}}{\pgfqpoint{0.943200in}{1.630827in}}%
\pgfpathcurveto{\pgfqpoint{0.943200in}{1.639064in}}{\pgfqpoint{0.939928in}{1.646964in}}{\pgfqpoint{0.934104in}{1.652788in}}%
\pgfpathcurveto{\pgfqpoint{0.928280in}{1.658612in}}{\pgfqpoint{0.920380in}{1.661884in}}{\pgfqpoint{0.912144in}{1.661884in}}%
\pgfpathcurveto{\pgfqpoint{0.903907in}{1.661884in}}{\pgfqpoint{0.896007in}{1.658612in}}{\pgfqpoint{0.890183in}{1.652788in}}%
\pgfpathcurveto{\pgfqpoint{0.884359in}{1.646964in}}{\pgfqpoint{0.881087in}{1.639064in}}{\pgfqpoint{0.881087in}{1.630827in}}%
\pgfpathcurveto{\pgfqpoint{0.881087in}{1.622591in}}{\pgfqpoint{0.884359in}{1.614691in}}{\pgfqpoint{0.890183in}{1.608867in}}%
\pgfpathcurveto{\pgfqpoint{0.896007in}{1.603043in}}{\pgfqpoint{0.903907in}{1.599771in}}{\pgfqpoint{0.912144in}{1.599771in}}%
\pgfpathclose%
\pgfusepath{stroke,fill}%
\end{pgfscope}%
\begin{pgfscope}%
\pgfpathrectangle{\pgfqpoint{0.100000in}{0.212622in}}{\pgfqpoint{3.696000in}{3.696000in}}%
\pgfusepath{clip}%
\pgfsetbuttcap%
\pgfsetroundjoin%
\definecolor{currentfill}{rgb}{0.121569,0.466667,0.705882}%
\pgfsetfillcolor{currentfill}%
\pgfsetfillopacity{0.556389}%
\pgfsetlinewidth{1.003750pt}%
\definecolor{currentstroke}{rgb}{0.121569,0.466667,0.705882}%
\pgfsetstrokecolor{currentstroke}%
\pgfsetstrokeopacity{0.556389}%
\pgfsetdash{}{0pt}%
\pgfpathmoveto{\pgfqpoint{3.268364in}{2.236725in}}%
\pgfpathcurveto{\pgfqpoint{3.276600in}{2.236725in}}{\pgfqpoint{3.284500in}{2.239997in}}{\pgfqpoint{3.290324in}{2.245821in}}%
\pgfpathcurveto{\pgfqpoint{3.296148in}{2.251645in}}{\pgfqpoint{3.299420in}{2.259545in}}{\pgfqpoint{3.299420in}{2.267781in}}%
\pgfpathcurveto{\pgfqpoint{3.299420in}{2.276017in}}{\pgfqpoint{3.296148in}{2.283917in}}{\pgfqpoint{3.290324in}{2.289741in}}%
\pgfpathcurveto{\pgfqpoint{3.284500in}{2.295565in}}{\pgfqpoint{3.276600in}{2.298838in}}{\pgfqpoint{3.268364in}{2.298838in}}%
\pgfpathcurveto{\pgfqpoint{3.260127in}{2.298838in}}{\pgfqpoint{3.252227in}{2.295565in}}{\pgfqpoint{3.246403in}{2.289741in}}%
\pgfpathcurveto{\pgfqpoint{3.240579in}{2.283917in}}{\pgfqpoint{3.237307in}{2.276017in}}{\pgfqpoint{3.237307in}{2.267781in}}%
\pgfpathcurveto{\pgfqpoint{3.237307in}{2.259545in}}{\pgfqpoint{3.240579in}{2.251645in}}{\pgfqpoint{3.246403in}{2.245821in}}%
\pgfpathcurveto{\pgfqpoint{3.252227in}{2.239997in}}{\pgfqpoint{3.260127in}{2.236725in}}{\pgfqpoint{3.268364in}{2.236725in}}%
\pgfpathclose%
\pgfusepath{stroke,fill}%
\end{pgfscope}%
\begin{pgfscope}%
\pgfpathrectangle{\pgfqpoint{0.100000in}{0.212622in}}{\pgfqpoint{3.696000in}{3.696000in}}%
\pgfusepath{clip}%
\pgfsetbuttcap%
\pgfsetroundjoin%
\definecolor{currentfill}{rgb}{0.121569,0.466667,0.705882}%
\pgfsetfillcolor{currentfill}%
\pgfsetfillopacity{0.556930}%
\pgfsetlinewidth{1.003750pt}%
\definecolor{currentstroke}{rgb}{0.121569,0.466667,0.705882}%
\pgfsetstrokecolor{currentstroke}%
\pgfsetstrokeopacity{0.556930}%
\pgfsetdash{}{0pt}%
\pgfpathmoveto{\pgfqpoint{0.909798in}{1.596385in}}%
\pgfpathcurveto{\pgfqpoint{0.918034in}{1.596385in}}{\pgfqpoint{0.925934in}{1.599658in}}{\pgfqpoint{0.931758in}{1.605482in}}%
\pgfpathcurveto{\pgfqpoint{0.937582in}{1.611306in}}{\pgfqpoint{0.940855in}{1.619206in}}{\pgfqpoint{0.940855in}{1.627442in}}%
\pgfpathcurveto{\pgfqpoint{0.940855in}{1.635678in}}{\pgfqpoint{0.937582in}{1.643578in}}{\pgfqpoint{0.931758in}{1.649402in}}%
\pgfpathcurveto{\pgfqpoint{0.925934in}{1.655226in}}{\pgfqpoint{0.918034in}{1.658498in}}{\pgfqpoint{0.909798in}{1.658498in}}%
\pgfpathcurveto{\pgfqpoint{0.901562in}{1.658498in}}{\pgfqpoint{0.893662in}{1.655226in}}{\pgfqpoint{0.887838in}{1.649402in}}%
\pgfpathcurveto{\pgfqpoint{0.882014in}{1.643578in}}{\pgfqpoint{0.878742in}{1.635678in}}{\pgfqpoint{0.878742in}{1.627442in}}%
\pgfpathcurveto{\pgfqpoint{0.878742in}{1.619206in}}{\pgfqpoint{0.882014in}{1.611306in}}{\pgfqpoint{0.887838in}{1.605482in}}%
\pgfpathcurveto{\pgfqpoint{0.893662in}{1.599658in}}{\pgfqpoint{0.901562in}{1.596385in}}{\pgfqpoint{0.909798in}{1.596385in}}%
\pgfpathclose%
\pgfusepath{stroke,fill}%
\end{pgfscope}%
\begin{pgfscope}%
\pgfpathrectangle{\pgfqpoint{0.100000in}{0.212622in}}{\pgfqpoint{3.696000in}{3.696000in}}%
\pgfusepath{clip}%
\pgfsetbuttcap%
\pgfsetroundjoin%
\definecolor{currentfill}{rgb}{0.121569,0.466667,0.705882}%
\pgfsetfillcolor{currentfill}%
\pgfsetfillopacity{0.558114}%
\pgfsetlinewidth{1.003750pt}%
\definecolor{currentstroke}{rgb}{0.121569,0.466667,0.705882}%
\pgfsetstrokecolor{currentstroke}%
\pgfsetstrokeopacity{0.558114}%
\pgfsetdash{}{0pt}%
\pgfpathmoveto{\pgfqpoint{0.905750in}{1.589221in}}%
\pgfpathcurveto{\pgfqpoint{0.913987in}{1.589221in}}{\pgfqpoint{0.921887in}{1.592493in}}{\pgfqpoint{0.927711in}{1.598317in}}%
\pgfpathcurveto{\pgfqpoint{0.933535in}{1.604141in}}{\pgfqpoint{0.936807in}{1.612041in}}{\pgfqpoint{0.936807in}{1.620277in}}%
\pgfpathcurveto{\pgfqpoint{0.936807in}{1.628513in}}{\pgfqpoint{0.933535in}{1.636413in}}{\pgfqpoint{0.927711in}{1.642237in}}%
\pgfpathcurveto{\pgfqpoint{0.921887in}{1.648061in}}{\pgfqpoint{0.913987in}{1.651334in}}{\pgfqpoint{0.905750in}{1.651334in}}%
\pgfpathcurveto{\pgfqpoint{0.897514in}{1.651334in}}{\pgfqpoint{0.889614in}{1.648061in}}{\pgfqpoint{0.883790in}{1.642237in}}%
\pgfpathcurveto{\pgfqpoint{0.877966in}{1.636413in}}{\pgfqpoint{0.874694in}{1.628513in}}{\pgfqpoint{0.874694in}{1.620277in}}%
\pgfpathcurveto{\pgfqpoint{0.874694in}{1.612041in}}{\pgfqpoint{0.877966in}{1.604141in}}{\pgfqpoint{0.883790in}{1.598317in}}%
\pgfpathcurveto{\pgfqpoint{0.889614in}{1.592493in}}{\pgfqpoint{0.897514in}{1.589221in}}{\pgfqpoint{0.905750in}{1.589221in}}%
\pgfpathclose%
\pgfusepath{stroke,fill}%
\end{pgfscope}%
\begin{pgfscope}%
\pgfpathrectangle{\pgfqpoint{0.100000in}{0.212622in}}{\pgfqpoint{3.696000in}{3.696000in}}%
\pgfusepath{clip}%
\pgfsetbuttcap%
\pgfsetroundjoin%
\definecolor{currentfill}{rgb}{0.121569,0.466667,0.705882}%
\pgfsetfillcolor{currentfill}%
\pgfsetfillopacity{0.558210}%
\pgfsetlinewidth{1.003750pt}%
\definecolor{currentstroke}{rgb}{0.121569,0.466667,0.705882}%
\pgfsetstrokecolor{currentstroke}%
\pgfsetstrokeopacity{0.558210}%
\pgfsetdash{}{0pt}%
\pgfpathmoveto{\pgfqpoint{3.265231in}{2.236647in}}%
\pgfpathcurveto{\pgfqpoint{3.273467in}{2.236647in}}{\pgfqpoint{3.281367in}{2.239919in}}{\pgfqpoint{3.287191in}{2.245743in}}%
\pgfpathcurveto{\pgfqpoint{3.293015in}{2.251567in}}{\pgfqpoint{3.296287in}{2.259467in}}{\pgfqpoint{3.296287in}{2.267704in}}%
\pgfpathcurveto{\pgfqpoint{3.296287in}{2.275940in}}{\pgfqpoint{3.293015in}{2.283840in}}{\pgfqpoint{3.287191in}{2.289664in}}%
\pgfpathcurveto{\pgfqpoint{3.281367in}{2.295488in}}{\pgfqpoint{3.273467in}{2.298760in}}{\pgfqpoint{3.265231in}{2.298760in}}%
\pgfpathcurveto{\pgfqpoint{3.256994in}{2.298760in}}{\pgfqpoint{3.249094in}{2.295488in}}{\pgfqpoint{3.243270in}{2.289664in}}%
\pgfpathcurveto{\pgfqpoint{3.237447in}{2.283840in}}{\pgfqpoint{3.234174in}{2.275940in}}{\pgfqpoint{3.234174in}{2.267704in}}%
\pgfpathcurveto{\pgfqpoint{3.234174in}{2.259467in}}{\pgfqpoint{3.237447in}{2.251567in}}{\pgfqpoint{3.243270in}{2.245743in}}%
\pgfpathcurveto{\pgfqpoint{3.249094in}{2.239919in}}{\pgfqpoint{3.256994in}{2.236647in}}{\pgfqpoint{3.265231in}{2.236647in}}%
\pgfpathclose%
\pgfusepath{stroke,fill}%
\end{pgfscope}%
\begin{pgfscope}%
\pgfpathrectangle{\pgfqpoint{0.100000in}{0.212622in}}{\pgfqpoint{3.696000in}{3.696000in}}%
\pgfusepath{clip}%
\pgfsetbuttcap%
\pgfsetroundjoin%
\definecolor{currentfill}{rgb}{0.121569,0.466667,0.705882}%
\pgfsetfillcolor{currentfill}%
\pgfsetfillopacity{0.560204}%
\pgfsetlinewidth{1.003750pt}%
\definecolor{currentstroke}{rgb}{0.121569,0.466667,0.705882}%
\pgfsetstrokecolor{currentstroke}%
\pgfsetstrokeopacity{0.560204}%
\pgfsetdash{}{0pt}%
\pgfpathmoveto{\pgfqpoint{3.261897in}{2.236487in}}%
\pgfpathcurveto{\pgfqpoint{3.270134in}{2.236487in}}{\pgfqpoint{3.278034in}{2.239760in}}{\pgfqpoint{3.283858in}{2.245584in}}%
\pgfpathcurveto{\pgfqpoint{3.289682in}{2.251407in}}{\pgfqpoint{3.292954in}{2.259308in}}{\pgfqpoint{3.292954in}{2.267544in}}%
\pgfpathcurveto{\pgfqpoint{3.292954in}{2.275780in}}{\pgfqpoint{3.289682in}{2.283680in}}{\pgfqpoint{3.283858in}{2.289504in}}%
\pgfpathcurveto{\pgfqpoint{3.278034in}{2.295328in}}{\pgfqpoint{3.270134in}{2.298600in}}{\pgfqpoint{3.261897in}{2.298600in}}%
\pgfpathcurveto{\pgfqpoint{3.253661in}{2.298600in}}{\pgfqpoint{3.245761in}{2.295328in}}{\pgfqpoint{3.239937in}{2.289504in}}%
\pgfpathcurveto{\pgfqpoint{3.234113in}{2.283680in}}{\pgfqpoint{3.230841in}{2.275780in}}{\pgfqpoint{3.230841in}{2.267544in}}%
\pgfpathcurveto{\pgfqpoint{3.230841in}{2.259308in}}{\pgfqpoint{3.234113in}{2.251407in}}{\pgfqpoint{3.239937in}{2.245584in}}%
\pgfpathcurveto{\pgfqpoint{3.245761in}{2.239760in}}{\pgfqpoint{3.253661in}{2.236487in}}{\pgfqpoint{3.261897in}{2.236487in}}%
\pgfpathclose%
\pgfusepath{stroke,fill}%
\end{pgfscope}%
\begin{pgfscope}%
\pgfpathrectangle{\pgfqpoint{0.100000in}{0.212622in}}{\pgfqpoint{3.696000in}{3.696000in}}%
\pgfusepath{clip}%
\pgfsetbuttcap%
\pgfsetroundjoin%
\definecolor{currentfill}{rgb}{0.121569,0.466667,0.705882}%
\pgfsetfillcolor{currentfill}%
\pgfsetfillopacity{0.560681}%
\pgfsetlinewidth{1.003750pt}%
\definecolor{currentstroke}{rgb}{0.121569,0.466667,0.705882}%
\pgfsetstrokecolor{currentstroke}%
\pgfsetstrokeopacity{0.560681}%
\pgfsetdash{}{0pt}%
\pgfpathmoveto{\pgfqpoint{0.898606in}{1.578573in}}%
\pgfpathcurveto{\pgfqpoint{0.906842in}{1.578573in}}{\pgfqpoint{0.914742in}{1.581845in}}{\pgfqpoint{0.920566in}{1.587669in}}%
\pgfpathcurveto{\pgfqpoint{0.926390in}{1.593493in}}{\pgfqpoint{0.929662in}{1.601393in}}{\pgfqpoint{0.929662in}{1.609629in}}%
\pgfpathcurveto{\pgfqpoint{0.929662in}{1.617865in}}{\pgfqpoint{0.926390in}{1.625765in}}{\pgfqpoint{0.920566in}{1.631589in}}%
\pgfpathcurveto{\pgfqpoint{0.914742in}{1.637413in}}{\pgfqpoint{0.906842in}{1.640686in}}{\pgfqpoint{0.898606in}{1.640686in}}%
\pgfpathcurveto{\pgfqpoint{0.890369in}{1.640686in}}{\pgfqpoint{0.882469in}{1.637413in}}{\pgfqpoint{0.876645in}{1.631589in}}%
\pgfpathcurveto{\pgfqpoint{0.870821in}{1.625765in}}{\pgfqpoint{0.867549in}{1.617865in}}{\pgfqpoint{0.867549in}{1.609629in}}%
\pgfpathcurveto{\pgfqpoint{0.867549in}{1.601393in}}{\pgfqpoint{0.870821in}{1.593493in}}{\pgfqpoint{0.876645in}{1.587669in}}%
\pgfpathcurveto{\pgfqpoint{0.882469in}{1.581845in}}{\pgfqpoint{0.890369in}{1.578573in}}{\pgfqpoint{0.898606in}{1.578573in}}%
\pgfpathclose%
\pgfusepath{stroke,fill}%
\end{pgfscope}%
\begin{pgfscope}%
\pgfpathrectangle{\pgfqpoint{0.100000in}{0.212622in}}{\pgfqpoint{3.696000in}{3.696000in}}%
\pgfusepath{clip}%
\pgfsetbuttcap%
\pgfsetroundjoin%
\definecolor{currentfill}{rgb}{0.121569,0.466667,0.705882}%
\pgfsetfillcolor{currentfill}%
\pgfsetfillopacity{0.562229}%
\pgfsetlinewidth{1.003750pt}%
\definecolor{currentstroke}{rgb}{0.121569,0.466667,0.705882}%
\pgfsetstrokecolor{currentstroke}%
\pgfsetstrokeopacity{0.562229}%
\pgfsetdash{}{0pt}%
\pgfpathmoveto{\pgfqpoint{3.258040in}{2.235675in}}%
\pgfpathcurveto{\pgfqpoint{3.266276in}{2.235675in}}{\pgfqpoint{3.274176in}{2.238948in}}{\pgfqpoint{3.280000in}{2.244771in}}%
\pgfpathcurveto{\pgfqpoint{3.285824in}{2.250595in}}{\pgfqpoint{3.289096in}{2.258495in}}{\pgfqpoint{3.289096in}{2.266732in}}%
\pgfpathcurveto{\pgfqpoint{3.289096in}{2.274968in}}{\pgfqpoint{3.285824in}{2.282868in}}{\pgfqpoint{3.280000in}{2.288692in}}%
\pgfpathcurveto{\pgfqpoint{3.274176in}{2.294516in}}{\pgfqpoint{3.266276in}{2.297788in}}{\pgfqpoint{3.258040in}{2.297788in}}%
\pgfpathcurveto{\pgfqpoint{3.249803in}{2.297788in}}{\pgfqpoint{3.241903in}{2.294516in}}{\pgfqpoint{3.236079in}{2.288692in}}%
\pgfpathcurveto{\pgfqpoint{3.230255in}{2.282868in}}{\pgfqpoint{3.226983in}{2.274968in}}{\pgfqpoint{3.226983in}{2.266732in}}%
\pgfpathcurveto{\pgfqpoint{3.226983in}{2.258495in}}{\pgfqpoint{3.230255in}{2.250595in}}{\pgfqpoint{3.236079in}{2.244771in}}%
\pgfpathcurveto{\pgfqpoint{3.241903in}{2.238948in}}{\pgfqpoint{3.249803in}{2.235675in}}{\pgfqpoint{3.258040in}{2.235675in}}%
\pgfpathclose%
\pgfusepath{stroke,fill}%
\end{pgfscope}%
\begin{pgfscope}%
\pgfpathrectangle{\pgfqpoint{0.100000in}{0.212622in}}{\pgfqpoint{3.696000in}{3.696000in}}%
\pgfusepath{clip}%
\pgfsetbuttcap%
\pgfsetroundjoin%
\definecolor{currentfill}{rgb}{0.121569,0.466667,0.705882}%
\pgfsetfillcolor{currentfill}%
\pgfsetfillopacity{0.562846}%
\pgfsetlinewidth{1.003750pt}%
\definecolor{currentstroke}{rgb}{0.121569,0.466667,0.705882}%
\pgfsetstrokecolor{currentstroke}%
\pgfsetstrokeopacity{0.562846}%
\pgfsetdash{}{0pt}%
\pgfpathmoveto{\pgfqpoint{0.892161in}{1.569000in}}%
\pgfpathcurveto{\pgfqpoint{0.900398in}{1.569000in}}{\pgfqpoint{0.908298in}{1.572272in}}{\pgfqpoint{0.914122in}{1.578096in}}%
\pgfpathcurveto{\pgfqpoint{0.919945in}{1.583920in}}{\pgfqpoint{0.923218in}{1.591820in}}{\pgfqpoint{0.923218in}{1.600056in}}%
\pgfpathcurveto{\pgfqpoint{0.923218in}{1.608292in}}{\pgfqpoint{0.919945in}{1.616193in}}{\pgfqpoint{0.914122in}{1.622016in}}%
\pgfpathcurveto{\pgfqpoint{0.908298in}{1.627840in}}{\pgfqpoint{0.900398in}{1.631113in}}{\pgfqpoint{0.892161in}{1.631113in}}%
\pgfpathcurveto{\pgfqpoint{0.883925in}{1.631113in}}{\pgfqpoint{0.876025in}{1.627840in}}{\pgfqpoint{0.870201in}{1.622016in}}%
\pgfpathcurveto{\pgfqpoint{0.864377in}{1.616193in}}{\pgfqpoint{0.861105in}{1.608292in}}{\pgfqpoint{0.861105in}{1.600056in}}%
\pgfpathcurveto{\pgfqpoint{0.861105in}{1.591820in}}{\pgfqpoint{0.864377in}{1.583920in}}{\pgfqpoint{0.870201in}{1.578096in}}%
\pgfpathcurveto{\pgfqpoint{0.876025in}{1.572272in}}{\pgfqpoint{0.883925in}{1.569000in}}{\pgfqpoint{0.892161in}{1.569000in}}%
\pgfpathclose%
\pgfusepath{stroke,fill}%
\end{pgfscope}%
\begin{pgfscope}%
\pgfpathrectangle{\pgfqpoint{0.100000in}{0.212622in}}{\pgfqpoint{3.696000in}{3.696000in}}%
\pgfusepath{clip}%
\pgfsetbuttcap%
\pgfsetroundjoin%
\definecolor{currentfill}{rgb}{0.121569,0.466667,0.705882}%
\pgfsetfillcolor{currentfill}%
\pgfsetfillopacity{0.564635}%
\pgfsetlinewidth{1.003750pt}%
\definecolor{currentstroke}{rgb}{0.121569,0.466667,0.705882}%
\pgfsetstrokecolor{currentstroke}%
\pgfsetstrokeopacity{0.564635}%
\pgfsetdash{}{0pt}%
\pgfpathmoveto{\pgfqpoint{3.253524in}{2.234255in}}%
\pgfpathcurveto{\pgfqpoint{3.261760in}{2.234255in}}{\pgfqpoint{3.269660in}{2.237527in}}{\pgfqpoint{3.275484in}{2.243351in}}%
\pgfpathcurveto{\pgfqpoint{3.281308in}{2.249175in}}{\pgfqpoint{3.284580in}{2.257075in}}{\pgfqpoint{3.284580in}{2.265312in}}%
\pgfpathcurveto{\pgfqpoint{3.284580in}{2.273548in}}{\pgfqpoint{3.281308in}{2.281448in}}{\pgfqpoint{3.275484in}{2.287272in}}%
\pgfpathcurveto{\pgfqpoint{3.269660in}{2.293096in}}{\pgfqpoint{3.261760in}{2.296368in}}{\pgfqpoint{3.253524in}{2.296368in}}%
\pgfpathcurveto{\pgfqpoint{3.245288in}{2.296368in}}{\pgfqpoint{3.237388in}{2.293096in}}{\pgfqpoint{3.231564in}{2.287272in}}%
\pgfpathcurveto{\pgfqpoint{3.225740in}{2.281448in}}{\pgfqpoint{3.222467in}{2.273548in}}{\pgfqpoint{3.222467in}{2.265312in}}%
\pgfpathcurveto{\pgfqpoint{3.222467in}{2.257075in}}{\pgfqpoint{3.225740in}{2.249175in}}{\pgfqpoint{3.231564in}{2.243351in}}%
\pgfpathcurveto{\pgfqpoint{3.237388in}{2.237527in}}{\pgfqpoint{3.245288in}{2.234255in}}{\pgfqpoint{3.253524in}{2.234255in}}%
\pgfpathclose%
\pgfusepath{stroke,fill}%
\end{pgfscope}%
\begin{pgfscope}%
\pgfpathrectangle{\pgfqpoint{0.100000in}{0.212622in}}{\pgfqpoint{3.696000in}{3.696000in}}%
\pgfusepath{clip}%
\pgfsetbuttcap%
\pgfsetroundjoin%
\definecolor{currentfill}{rgb}{0.121569,0.466667,0.705882}%
\pgfsetfillcolor{currentfill}%
\pgfsetfillopacity{0.564937}%
\pgfsetlinewidth{1.003750pt}%
\definecolor{currentstroke}{rgb}{0.121569,0.466667,0.705882}%
\pgfsetstrokecolor{currentstroke}%
\pgfsetstrokeopacity{0.564937}%
\pgfsetdash{}{0pt}%
\pgfpathmoveto{\pgfqpoint{0.886590in}{1.560635in}}%
\pgfpathcurveto{\pgfqpoint{0.894827in}{1.560635in}}{\pgfqpoint{0.902727in}{1.563907in}}{\pgfqpoint{0.908551in}{1.569731in}}%
\pgfpathcurveto{\pgfqpoint{0.914375in}{1.575555in}}{\pgfqpoint{0.917647in}{1.583455in}}{\pgfqpoint{0.917647in}{1.591691in}}%
\pgfpathcurveto{\pgfqpoint{0.917647in}{1.599927in}}{\pgfqpoint{0.914375in}{1.607827in}}{\pgfqpoint{0.908551in}{1.613651in}}%
\pgfpathcurveto{\pgfqpoint{0.902727in}{1.619475in}}{\pgfqpoint{0.894827in}{1.622748in}}{\pgfqpoint{0.886590in}{1.622748in}}%
\pgfpathcurveto{\pgfqpoint{0.878354in}{1.622748in}}{\pgfqpoint{0.870454in}{1.619475in}}{\pgfqpoint{0.864630in}{1.613651in}}%
\pgfpathcurveto{\pgfqpoint{0.858806in}{1.607827in}}{\pgfqpoint{0.855534in}{1.599927in}}{\pgfqpoint{0.855534in}{1.591691in}}%
\pgfpathcurveto{\pgfqpoint{0.855534in}{1.583455in}}{\pgfqpoint{0.858806in}{1.575555in}}{\pgfqpoint{0.864630in}{1.569731in}}%
\pgfpathcurveto{\pgfqpoint{0.870454in}{1.563907in}}{\pgfqpoint{0.878354in}{1.560635in}}{\pgfqpoint{0.886590in}{1.560635in}}%
\pgfpathclose%
\pgfusepath{stroke,fill}%
\end{pgfscope}%
\begin{pgfscope}%
\pgfpathrectangle{\pgfqpoint{0.100000in}{0.212622in}}{\pgfqpoint{3.696000in}{3.696000in}}%
\pgfusepath{clip}%
\pgfsetbuttcap%
\pgfsetroundjoin%
\definecolor{currentfill}{rgb}{0.121569,0.466667,0.705882}%
\pgfsetfillcolor{currentfill}%
\pgfsetfillopacity{0.565934}%
\pgfsetlinewidth{1.003750pt}%
\definecolor{currentstroke}{rgb}{0.121569,0.466667,0.705882}%
\pgfsetstrokecolor{currentstroke}%
\pgfsetstrokeopacity{0.565934}%
\pgfsetdash{}{0pt}%
\pgfpathmoveto{\pgfqpoint{3.250995in}{2.233358in}}%
\pgfpathcurveto{\pgfqpoint{3.259231in}{2.233358in}}{\pgfqpoint{3.267131in}{2.236631in}}{\pgfqpoint{3.272955in}{2.242454in}}%
\pgfpathcurveto{\pgfqpoint{3.278779in}{2.248278in}}{\pgfqpoint{3.282051in}{2.256178in}}{\pgfqpoint{3.282051in}{2.264415in}}%
\pgfpathcurveto{\pgfqpoint{3.282051in}{2.272651in}}{\pgfqpoint{3.278779in}{2.280551in}}{\pgfqpoint{3.272955in}{2.286375in}}%
\pgfpathcurveto{\pgfqpoint{3.267131in}{2.292199in}}{\pgfqpoint{3.259231in}{2.295471in}}{\pgfqpoint{3.250995in}{2.295471in}}%
\pgfpathcurveto{\pgfqpoint{3.242758in}{2.295471in}}{\pgfqpoint{3.234858in}{2.292199in}}{\pgfqpoint{3.229035in}{2.286375in}}%
\pgfpathcurveto{\pgfqpoint{3.223211in}{2.280551in}}{\pgfqpoint{3.219938in}{2.272651in}}{\pgfqpoint{3.219938in}{2.264415in}}%
\pgfpathcurveto{\pgfqpoint{3.219938in}{2.256178in}}{\pgfqpoint{3.223211in}{2.248278in}}{\pgfqpoint{3.229035in}{2.242454in}}%
\pgfpathcurveto{\pgfqpoint{3.234858in}{2.236631in}}{\pgfqpoint{3.242758in}{2.233358in}}{\pgfqpoint{3.250995in}{2.233358in}}%
\pgfpathclose%
\pgfusepath{stroke,fill}%
\end{pgfscope}%
\begin{pgfscope}%
\pgfpathrectangle{\pgfqpoint{0.100000in}{0.212622in}}{\pgfqpoint{3.696000in}{3.696000in}}%
\pgfusepath{clip}%
\pgfsetbuttcap%
\pgfsetroundjoin%
\definecolor{currentfill}{rgb}{0.121569,0.466667,0.705882}%
\pgfsetfillcolor{currentfill}%
\pgfsetfillopacity{0.566606}%
\pgfsetlinewidth{1.003750pt}%
\definecolor{currentstroke}{rgb}{0.121569,0.466667,0.705882}%
\pgfsetstrokecolor{currentstroke}%
\pgfsetstrokeopacity{0.566606}%
\pgfsetdash{}{0pt}%
\pgfpathmoveto{\pgfqpoint{0.881865in}{1.553740in}}%
\pgfpathcurveto{\pgfqpoint{0.890101in}{1.553740in}}{\pgfqpoint{0.898001in}{1.557012in}}{\pgfqpoint{0.903825in}{1.562836in}}%
\pgfpathcurveto{\pgfqpoint{0.909649in}{1.568660in}}{\pgfqpoint{0.912922in}{1.576560in}}{\pgfqpoint{0.912922in}{1.584796in}}%
\pgfpathcurveto{\pgfqpoint{0.912922in}{1.593033in}}{\pgfqpoint{0.909649in}{1.600933in}}{\pgfqpoint{0.903825in}{1.606757in}}%
\pgfpathcurveto{\pgfqpoint{0.898001in}{1.612581in}}{\pgfqpoint{0.890101in}{1.615853in}}{\pgfqpoint{0.881865in}{1.615853in}}%
\pgfpathcurveto{\pgfqpoint{0.873629in}{1.615853in}}{\pgfqpoint{0.865729in}{1.612581in}}{\pgfqpoint{0.859905in}{1.606757in}}%
\pgfpathcurveto{\pgfqpoint{0.854081in}{1.600933in}}{\pgfqpoint{0.850809in}{1.593033in}}{\pgfqpoint{0.850809in}{1.584796in}}%
\pgfpathcurveto{\pgfqpoint{0.850809in}{1.576560in}}{\pgfqpoint{0.854081in}{1.568660in}}{\pgfqpoint{0.859905in}{1.562836in}}%
\pgfpathcurveto{\pgfqpoint{0.865729in}{1.557012in}}{\pgfqpoint{0.873629in}{1.553740in}}{\pgfqpoint{0.881865in}{1.553740in}}%
\pgfpathclose%
\pgfusepath{stroke,fill}%
\end{pgfscope}%
\begin{pgfscope}%
\pgfpathrectangle{\pgfqpoint{0.100000in}{0.212622in}}{\pgfqpoint{3.696000in}{3.696000in}}%
\pgfusepath{clip}%
\pgfsetbuttcap%
\pgfsetroundjoin%
\definecolor{currentfill}{rgb}{0.121569,0.466667,0.705882}%
\pgfsetfillcolor{currentfill}%
\pgfsetfillopacity{0.566609}%
\pgfsetlinewidth{1.003750pt}%
\definecolor{currentstroke}{rgb}{0.121569,0.466667,0.705882}%
\pgfsetstrokecolor{currentstroke}%
\pgfsetstrokeopacity{0.566609}%
\pgfsetdash{}{0pt}%
\pgfpathmoveto{\pgfqpoint{3.249535in}{2.232671in}}%
\pgfpathcurveto{\pgfqpoint{3.257771in}{2.232671in}}{\pgfqpoint{3.265671in}{2.235944in}}{\pgfqpoint{3.271495in}{2.241768in}}%
\pgfpathcurveto{\pgfqpoint{3.277319in}{2.247591in}}{\pgfqpoint{3.280592in}{2.255492in}}{\pgfqpoint{3.280592in}{2.263728in}}%
\pgfpathcurveto{\pgfqpoint{3.280592in}{2.271964in}}{\pgfqpoint{3.277319in}{2.279864in}}{\pgfqpoint{3.271495in}{2.285688in}}%
\pgfpathcurveto{\pgfqpoint{3.265671in}{2.291512in}}{\pgfqpoint{3.257771in}{2.294784in}}{\pgfqpoint{3.249535in}{2.294784in}}%
\pgfpathcurveto{\pgfqpoint{3.241299in}{2.294784in}}{\pgfqpoint{3.233399in}{2.291512in}}{\pgfqpoint{3.227575in}{2.285688in}}%
\pgfpathcurveto{\pgfqpoint{3.221751in}{2.279864in}}{\pgfqpoint{3.218479in}{2.271964in}}{\pgfqpoint{3.218479in}{2.263728in}}%
\pgfpathcurveto{\pgfqpoint{3.218479in}{2.255492in}}{\pgfqpoint{3.221751in}{2.247591in}}{\pgfqpoint{3.227575in}{2.241768in}}%
\pgfpathcurveto{\pgfqpoint{3.233399in}{2.235944in}}{\pgfqpoint{3.241299in}{2.232671in}}{\pgfqpoint{3.249535in}{2.232671in}}%
\pgfpathclose%
\pgfusepath{stroke,fill}%
\end{pgfscope}%
\begin{pgfscope}%
\pgfpathrectangle{\pgfqpoint{0.100000in}{0.212622in}}{\pgfqpoint{3.696000in}{3.696000in}}%
\pgfusepath{clip}%
\pgfsetbuttcap%
\pgfsetroundjoin%
\definecolor{currentfill}{rgb}{0.121569,0.466667,0.705882}%
\pgfsetfillcolor{currentfill}%
\pgfsetfillopacity{0.566992}%
\pgfsetlinewidth{1.003750pt}%
\definecolor{currentstroke}{rgb}{0.121569,0.466667,0.705882}%
\pgfsetstrokecolor{currentstroke}%
\pgfsetstrokeopacity{0.566992}%
\pgfsetdash{}{0pt}%
\pgfpathmoveto{\pgfqpoint{3.248765in}{2.232332in}}%
\pgfpathcurveto{\pgfqpoint{3.257001in}{2.232332in}}{\pgfqpoint{3.264901in}{2.235604in}}{\pgfqpoint{3.270725in}{2.241428in}}%
\pgfpathcurveto{\pgfqpoint{3.276549in}{2.247252in}}{\pgfqpoint{3.279821in}{2.255152in}}{\pgfqpoint{3.279821in}{2.263388in}}%
\pgfpathcurveto{\pgfqpoint{3.279821in}{2.271624in}}{\pgfqpoint{3.276549in}{2.279525in}}{\pgfqpoint{3.270725in}{2.285348in}}%
\pgfpathcurveto{\pgfqpoint{3.264901in}{2.291172in}}{\pgfqpoint{3.257001in}{2.294445in}}{\pgfqpoint{3.248765in}{2.294445in}}%
\pgfpathcurveto{\pgfqpoint{3.240528in}{2.294445in}}{\pgfqpoint{3.232628in}{2.291172in}}{\pgfqpoint{3.226804in}{2.285348in}}%
\pgfpathcurveto{\pgfqpoint{3.220980in}{2.279525in}}{\pgfqpoint{3.217708in}{2.271624in}}{\pgfqpoint{3.217708in}{2.263388in}}%
\pgfpathcurveto{\pgfqpoint{3.217708in}{2.255152in}}{\pgfqpoint{3.220980in}{2.247252in}}{\pgfqpoint{3.226804in}{2.241428in}}%
\pgfpathcurveto{\pgfqpoint{3.232628in}{2.235604in}}{\pgfqpoint{3.240528in}{2.232332in}}{\pgfqpoint{3.248765in}{2.232332in}}%
\pgfpathclose%
\pgfusepath{stroke,fill}%
\end{pgfscope}%
\begin{pgfscope}%
\pgfpathrectangle{\pgfqpoint{0.100000in}{0.212622in}}{\pgfqpoint{3.696000in}{3.696000in}}%
\pgfusepath{clip}%
\pgfsetbuttcap%
\pgfsetroundjoin%
\definecolor{currentfill}{rgb}{0.121569,0.466667,0.705882}%
\pgfsetfillcolor{currentfill}%
\pgfsetfillopacity{0.567902}%
\pgfsetlinewidth{1.003750pt}%
\definecolor{currentstroke}{rgb}{0.121569,0.466667,0.705882}%
\pgfsetstrokecolor{currentstroke}%
\pgfsetstrokeopacity{0.567902}%
\pgfsetdash{}{0pt}%
\pgfpathmoveto{\pgfqpoint{3.246927in}{2.231359in}}%
\pgfpathcurveto{\pgfqpoint{3.255164in}{2.231359in}}{\pgfqpoint{3.263064in}{2.234632in}}{\pgfqpoint{3.268888in}{2.240456in}}%
\pgfpathcurveto{\pgfqpoint{3.274711in}{2.246280in}}{\pgfqpoint{3.277984in}{2.254180in}}{\pgfqpoint{3.277984in}{2.262416in}}%
\pgfpathcurveto{\pgfqpoint{3.277984in}{2.270652in}}{\pgfqpoint{3.274711in}{2.278552in}}{\pgfqpoint{3.268888in}{2.284376in}}%
\pgfpathcurveto{\pgfqpoint{3.263064in}{2.290200in}}{\pgfqpoint{3.255164in}{2.293472in}}{\pgfqpoint{3.246927in}{2.293472in}}%
\pgfpathcurveto{\pgfqpoint{3.238691in}{2.293472in}}{\pgfqpoint{3.230791in}{2.290200in}}{\pgfqpoint{3.224967in}{2.284376in}}%
\pgfpathcurveto{\pgfqpoint{3.219143in}{2.278552in}}{\pgfqpoint{3.215871in}{2.270652in}}{\pgfqpoint{3.215871in}{2.262416in}}%
\pgfpathcurveto{\pgfqpoint{3.215871in}{2.254180in}}{\pgfqpoint{3.219143in}{2.246280in}}{\pgfqpoint{3.224967in}{2.240456in}}%
\pgfpathcurveto{\pgfqpoint{3.230791in}{2.234632in}}{\pgfqpoint{3.238691in}{2.231359in}}{\pgfqpoint{3.246927in}{2.231359in}}%
\pgfpathclose%
\pgfusepath{stroke,fill}%
\end{pgfscope}%
\begin{pgfscope}%
\pgfpathrectangle{\pgfqpoint{0.100000in}{0.212622in}}{\pgfqpoint{3.696000in}{3.696000in}}%
\pgfusepath{clip}%
\pgfsetbuttcap%
\pgfsetroundjoin%
\definecolor{currentfill}{rgb}{0.121569,0.466667,0.705882}%
\pgfsetfillcolor{currentfill}%
\pgfsetfillopacity{0.568003}%
\pgfsetlinewidth{1.003750pt}%
\definecolor{currentstroke}{rgb}{0.121569,0.466667,0.705882}%
\pgfsetstrokecolor{currentstroke}%
\pgfsetstrokeopacity{0.568003}%
\pgfsetdash{}{0pt}%
\pgfpathmoveto{\pgfqpoint{0.878210in}{1.547922in}}%
\pgfpathcurveto{\pgfqpoint{0.886446in}{1.547922in}}{\pgfqpoint{0.894346in}{1.551194in}}{\pgfqpoint{0.900170in}{1.557018in}}%
\pgfpathcurveto{\pgfqpoint{0.905994in}{1.562842in}}{\pgfqpoint{0.909267in}{1.570742in}}{\pgfqpoint{0.909267in}{1.578978in}}%
\pgfpathcurveto{\pgfqpoint{0.909267in}{1.587214in}}{\pgfqpoint{0.905994in}{1.595114in}}{\pgfqpoint{0.900170in}{1.600938in}}%
\pgfpathcurveto{\pgfqpoint{0.894346in}{1.606762in}}{\pgfqpoint{0.886446in}{1.610035in}}{\pgfqpoint{0.878210in}{1.610035in}}%
\pgfpathcurveto{\pgfqpoint{0.869974in}{1.610035in}}{\pgfqpoint{0.862074in}{1.606762in}}{\pgfqpoint{0.856250in}{1.600938in}}%
\pgfpathcurveto{\pgfqpoint{0.850426in}{1.595114in}}{\pgfqpoint{0.847154in}{1.587214in}}{\pgfqpoint{0.847154in}{1.578978in}}%
\pgfpathcurveto{\pgfqpoint{0.847154in}{1.570742in}}{\pgfqpoint{0.850426in}{1.562842in}}{\pgfqpoint{0.856250in}{1.557018in}}%
\pgfpathcurveto{\pgfqpoint{0.862074in}{1.551194in}}{\pgfqpoint{0.869974in}{1.547922in}}{\pgfqpoint{0.878210in}{1.547922in}}%
\pgfpathclose%
\pgfusepath{stroke,fill}%
\end{pgfscope}%
\begin{pgfscope}%
\pgfpathrectangle{\pgfqpoint{0.100000in}{0.212622in}}{\pgfqpoint{3.696000in}{3.696000in}}%
\pgfusepath{clip}%
\pgfsetbuttcap%
\pgfsetroundjoin%
\definecolor{currentfill}{rgb}{0.121569,0.466667,0.705882}%
\pgfsetfillcolor{currentfill}%
\pgfsetfillopacity{0.568410}%
\pgfsetlinewidth{1.003750pt}%
\definecolor{currentstroke}{rgb}{0.121569,0.466667,0.705882}%
\pgfsetstrokecolor{currentstroke}%
\pgfsetstrokeopacity{0.568410}%
\pgfsetdash{}{0pt}%
\pgfpathmoveto{\pgfqpoint{3.245902in}{2.230891in}}%
\pgfpathcurveto{\pgfqpoint{3.254138in}{2.230891in}}{\pgfqpoint{3.262038in}{2.234163in}}{\pgfqpoint{3.267862in}{2.239987in}}%
\pgfpathcurveto{\pgfqpoint{3.273686in}{2.245811in}}{\pgfqpoint{3.276958in}{2.253711in}}{\pgfqpoint{3.276958in}{2.261947in}}%
\pgfpathcurveto{\pgfqpoint{3.276958in}{2.270184in}}{\pgfqpoint{3.273686in}{2.278084in}}{\pgfqpoint{3.267862in}{2.283908in}}%
\pgfpathcurveto{\pgfqpoint{3.262038in}{2.289732in}}{\pgfqpoint{3.254138in}{2.293004in}}{\pgfqpoint{3.245902in}{2.293004in}}%
\pgfpathcurveto{\pgfqpoint{3.237665in}{2.293004in}}{\pgfqpoint{3.229765in}{2.289732in}}{\pgfqpoint{3.223941in}{2.283908in}}%
\pgfpathcurveto{\pgfqpoint{3.218118in}{2.278084in}}{\pgfqpoint{3.214845in}{2.270184in}}{\pgfqpoint{3.214845in}{2.261947in}}%
\pgfpathcurveto{\pgfqpoint{3.214845in}{2.253711in}}{\pgfqpoint{3.218118in}{2.245811in}}{\pgfqpoint{3.223941in}{2.239987in}}%
\pgfpathcurveto{\pgfqpoint{3.229765in}{2.234163in}}{\pgfqpoint{3.237665in}{2.230891in}}{\pgfqpoint{3.245902in}{2.230891in}}%
\pgfpathclose%
\pgfusepath{stroke,fill}%
\end{pgfscope}%
\begin{pgfscope}%
\pgfpathrectangle{\pgfqpoint{0.100000in}{0.212622in}}{\pgfqpoint{3.696000in}{3.696000in}}%
\pgfusepath{clip}%
\pgfsetbuttcap%
\pgfsetroundjoin%
\definecolor{currentfill}{rgb}{0.121569,0.466667,0.705882}%
\pgfsetfillcolor{currentfill}%
\pgfsetfillopacity{0.568696}%
\pgfsetlinewidth{1.003750pt}%
\definecolor{currentstroke}{rgb}{0.121569,0.466667,0.705882}%
\pgfsetstrokecolor{currentstroke}%
\pgfsetstrokeopacity{0.568696}%
\pgfsetdash{}{0pt}%
\pgfpathmoveto{\pgfqpoint{3.245357in}{2.230651in}}%
\pgfpathcurveto{\pgfqpoint{3.253593in}{2.230651in}}{\pgfqpoint{3.261494in}{2.233923in}}{\pgfqpoint{3.267317in}{2.239747in}}%
\pgfpathcurveto{\pgfqpoint{3.273141in}{2.245571in}}{\pgfqpoint{3.276414in}{2.253471in}}{\pgfqpoint{3.276414in}{2.261707in}}%
\pgfpathcurveto{\pgfqpoint{3.276414in}{2.269944in}}{\pgfqpoint{3.273141in}{2.277844in}}{\pgfqpoint{3.267317in}{2.283668in}}%
\pgfpathcurveto{\pgfqpoint{3.261494in}{2.289492in}}{\pgfqpoint{3.253593in}{2.292764in}}{\pgfqpoint{3.245357in}{2.292764in}}%
\pgfpathcurveto{\pgfqpoint{3.237121in}{2.292764in}}{\pgfqpoint{3.229221in}{2.289492in}}{\pgfqpoint{3.223397in}{2.283668in}}%
\pgfpathcurveto{\pgfqpoint{3.217573in}{2.277844in}}{\pgfqpoint{3.214301in}{2.269944in}}{\pgfqpoint{3.214301in}{2.261707in}}%
\pgfpathcurveto{\pgfqpoint{3.214301in}{2.253471in}}{\pgfqpoint{3.217573in}{2.245571in}}{\pgfqpoint{3.223397in}{2.239747in}}%
\pgfpathcurveto{\pgfqpoint{3.229221in}{2.233923in}}{\pgfqpoint{3.237121in}{2.230651in}}{\pgfqpoint{3.245357in}{2.230651in}}%
\pgfpathclose%
\pgfusepath{stroke,fill}%
\end{pgfscope}%
\begin{pgfscope}%
\pgfpathrectangle{\pgfqpoint{0.100000in}{0.212622in}}{\pgfqpoint{3.696000in}{3.696000in}}%
\pgfusepath{clip}%
\pgfsetbuttcap%
\pgfsetroundjoin%
\definecolor{currentfill}{rgb}{0.121569,0.466667,0.705882}%
\pgfsetfillcolor{currentfill}%
\pgfsetfillopacity{0.568841}%
\pgfsetlinewidth{1.003750pt}%
\definecolor{currentstroke}{rgb}{0.121569,0.466667,0.705882}%
\pgfsetstrokecolor{currentstroke}%
\pgfsetstrokeopacity{0.568841}%
\pgfsetdash{}{0pt}%
\pgfpathmoveto{\pgfqpoint{0.875873in}{1.544270in}}%
\pgfpathcurveto{\pgfqpoint{0.884109in}{1.544270in}}{\pgfqpoint{0.892009in}{1.547542in}}{\pgfqpoint{0.897833in}{1.553366in}}%
\pgfpathcurveto{\pgfqpoint{0.903657in}{1.559190in}}{\pgfqpoint{0.906930in}{1.567090in}}{\pgfqpoint{0.906930in}{1.575326in}}%
\pgfpathcurveto{\pgfqpoint{0.906930in}{1.583562in}}{\pgfqpoint{0.903657in}{1.591462in}}{\pgfqpoint{0.897833in}{1.597286in}}%
\pgfpathcurveto{\pgfqpoint{0.892009in}{1.603110in}}{\pgfqpoint{0.884109in}{1.606383in}}{\pgfqpoint{0.875873in}{1.606383in}}%
\pgfpathcurveto{\pgfqpoint{0.867637in}{1.606383in}}{\pgfqpoint{0.859737in}{1.603110in}}{\pgfqpoint{0.853913in}{1.597286in}}%
\pgfpathcurveto{\pgfqpoint{0.848089in}{1.591462in}}{\pgfqpoint{0.844817in}{1.583562in}}{\pgfqpoint{0.844817in}{1.575326in}}%
\pgfpathcurveto{\pgfqpoint{0.844817in}{1.567090in}}{\pgfqpoint{0.848089in}{1.559190in}}{\pgfqpoint{0.853913in}{1.553366in}}%
\pgfpathcurveto{\pgfqpoint{0.859737in}{1.547542in}}{\pgfqpoint{0.867637in}{1.544270in}}{\pgfqpoint{0.875873in}{1.544270in}}%
\pgfpathclose%
\pgfusepath{stroke,fill}%
\end{pgfscope}%
\begin{pgfscope}%
\pgfpathrectangle{\pgfqpoint{0.100000in}{0.212622in}}{\pgfqpoint{3.696000in}{3.696000in}}%
\pgfusepath{clip}%
\pgfsetbuttcap%
\pgfsetroundjoin%
\definecolor{currentfill}{rgb}{0.121569,0.466667,0.705882}%
\pgfsetfillcolor{currentfill}%
\pgfsetfillopacity{0.568854}%
\pgfsetlinewidth{1.003750pt}%
\definecolor{currentstroke}{rgb}{0.121569,0.466667,0.705882}%
\pgfsetstrokecolor{currentstroke}%
\pgfsetstrokeopacity{0.568854}%
\pgfsetdash{}{0pt}%
\pgfpathmoveto{\pgfqpoint{3.245046in}{2.230541in}}%
\pgfpathcurveto{\pgfqpoint{3.253282in}{2.230541in}}{\pgfqpoint{3.261182in}{2.233813in}}{\pgfqpoint{3.267006in}{2.239637in}}%
\pgfpathcurveto{\pgfqpoint{3.272830in}{2.245461in}}{\pgfqpoint{3.276102in}{2.253361in}}{\pgfqpoint{3.276102in}{2.261597in}}%
\pgfpathcurveto{\pgfqpoint{3.276102in}{2.269834in}}{\pgfqpoint{3.272830in}{2.277734in}}{\pgfqpoint{3.267006in}{2.283558in}}%
\pgfpathcurveto{\pgfqpoint{3.261182in}{2.289382in}}{\pgfqpoint{3.253282in}{2.292654in}}{\pgfqpoint{3.245046in}{2.292654in}}%
\pgfpathcurveto{\pgfqpoint{3.236810in}{2.292654in}}{\pgfqpoint{3.228910in}{2.289382in}}{\pgfqpoint{3.223086in}{2.283558in}}%
\pgfpathcurveto{\pgfqpoint{3.217262in}{2.277734in}}{\pgfqpoint{3.213989in}{2.269834in}}{\pgfqpoint{3.213989in}{2.261597in}}%
\pgfpathcurveto{\pgfqpoint{3.213989in}{2.253361in}}{\pgfqpoint{3.217262in}{2.245461in}}{\pgfqpoint{3.223086in}{2.239637in}}%
\pgfpathcurveto{\pgfqpoint{3.228910in}{2.233813in}}{\pgfqpoint{3.236810in}{2.230541in}}{\pgfqpoint{3.245046in}{2.230541in}}%
\pgfpathclose%
\pgfusepath{stroke,fill}%
\end{pgfscope}%
\begin{pgfscope}%
\pgfpathrectangle{\pgfqpoint{0.100000in}{0.212622in}}{\pgfqpoint{3.696000in}{3.696000in}}%
\pgfusepath{clip}%
\pgfsetbuttcap%
\pgfsetroundjoin%
\definecolor{currentfill}{rgb}{0.121569,0.466667,0.705882}%
\pgfsetfillcolor{currentfill}%
\pgfsetfillopacity{0.569375}%
\pgfsetlinewidth{1.003750pt}%
\definecolor{currentstroke}{rgb}{0.121569,0.466667,0.705882}%
\pgfsetstrokecolor{currentstroke}%
\pgfsetstrokeopacity{0.569375}%
\pgfsetdash{}{0pt}%
\pgfpathmoveto{\pgfqpoint{0.874461in}{1.542107in}}%
\pgfpathcurveto{\pgfqpoint{0.882698in}{1.542107in}}{\pgfqpoint{0.890598in}{1.545380in}}{\pgfqpoint{0.896422in}{1.551203in}}%
\pgfpathcurveto{\pgfqpoint{0.902246in}{1.557027in}}{\pgfqpoint{0.905518in}{1.564927in}}{\pgfqpoint{0.905518in}{1.573164in}}%
\pgfpathcurveto{\pgfqpoint{0.905518in}{1.581400in}}{\pgfqpoint{0.902246in}{1.589300in}}{\pgfqpoint{0.896422in}{1.595124in}}%
\pgfpathcurveto{\pgfqpoint{0.890598in}{1.600948in}}{\pgfqpoint{0.882698in}{1.604220in}}{\pgfqpoint{0.874461in}{1.604220in}}%
\pgfpathcurveto{\pgfqpoint{0.866225in}{1.604220in}}{\pgfqpoint{0.858325in}{1.600948in}}{\pgfqpoint{0.852501in}{1.595124in}}%
\pgfpathcurveto{\pgfqpoint{0.846677in}{1.589300in}}{\pgfqpoint{0.843405in}{1.581400in}}{\pgfqpoint{0.843405in}{1.573164in}}%
\pgfpathcurveto{\pgfqpoint{0.843405in}{1.564927in}}{\pgfqpoint{0.846677in}{1.557027in}}{\pgfqpoint{0.852501in}{1.551203in}}%
\pgfpathcurveto{\pgfqpoint{0.858325in}{1.545380in}}{\pgfqpoint{0.866225in}{1.542107in}}{\pgfqpoint{0.874461in}{1.542107in}}%
\pgfpathclose%
\pgfusepath{stroke,fill}%
\end{pgfscope}%
\begin{pgfscope}%
\pgfpathrectangle{\pgfqpoint{0.100000in}{0.212622in}}{\pgfqpoint{3.696000in}{3.696000in}}%
\pgfusepath{clip}%
\pgfsetbuttcap%
\pgfsetroundjoin%
\definecolor{currentfill}{rgb}{0.121569,0.466667,0.705882}%
\pgfsetfillcolor{currentfill}%
\pgfsetfillopacity{0.569443}%
\pgfsetlinewidth{1.003750pt}%
\definecolor{currentstroke}{rgb}{0.121569,0.466667,0.705882}%
\pgfsetstrokecolor{currentstroke}%
\pgfsetstrokeopacity{0.569443}%
\pgfsetdash{}{0pt}%
\pgfpathmoveto{\pgfqpoint{3.243978in}{2.230166in}}%
\pgfpathcurveto{\pgfqpoint{3.252214in}{2.230166in}}{\pgfqpoint{3.260114in}{2.233439in}}{\pgfqpoint{3.265938in}{2.239263in}}%
\pgfpathcurveto{\pgfqpoint{3.271762in}{2.245087in}}{\pgfqpoint{3.275034in}{2.252987in}}{\pgfqpoint{3.275034in}{2.261223in}}%
\pgfpathcurveto{\pgfqpoint{3.275034in}{2.269459in}}{\pgfqpoint{3.271762in}{2.277359in}}{\pgfqpoint{3.265938in}{2.283183in}}%
\pgfpathcurveto{\pgfqpoint{3.260114in}{2.289007in}}{\pgfqpoint{3.252214in}{2.292279in}}{\pgfqpoint{3.243978in}{2.292279in}}%
\pgfpathcurveto{\pgfqpoint{3.235742in}{2.292279in}}{\pgfqpoint{3.227841in}{2.289007in}}{\pgfqpoint{3.222018in}{2.283183in}}%
\pgfpathcurveto{\pgfqpoint{3.216194in}{2.277359in}}{\pgfqpoint{3.212921in}{2.269459in}}{\pgfqpoint{3.212921in}{2.261223in}}%
\pgfpathcurveto{\pgfqpoint{3.212921in}{2.252987in}}{\pgfqpoint{3.216194in}{2.245087in}}{\pgfqpoint{3.222018in}{2.239263in}}%
\pgfpathcurveto{\pgfqpoint{3.227841in}{2.233439in}}{\pgfqpoint{3.235742in}{2.230166in}}{\pgfqpoint{3.243978in}{2.230166in}}%
\pgfpathclose%
\pgfusepath{stroke,fill}%
\end{pgfscope}%
\begin{pgfscope}%
\pgfpathrectangle{\pgfqpoint{0.100000in}{0.212622in}}{\pgfqpoint{3.696000in}{3.696000in}}%
\pgfusepath{clip}%
\pgfsetbuttcap%
\pgfsetroundjoin%
\definecolor{currentfill}{rgb}{0.121569,0.466667,0.705882}%
\pgfsetfillcolor{currentfill}%
\pgfsetfillopacity{0.569503}%
\pgfsetlinewidth{1.003750pt}%
\definecolor{currentstroke}{rgb}{0.121569,0.466667,0.705882}%
\pgfsetstrokecolor{currentstroke}%
\pgfsetstrokeopacity{0.569503}%
\pgfsetdash{}{0pt}%
\pgfpathmoveto{\pgfqpoint{0.874106in}{1.541565in}}%
\pgfpathcurveto{\pgfqpoint{0.882342in}{1.541565in}}{\pgfqpoint{0.890242in}{1.544837in}}{\pgfqpoint{0.896066in}{1.550661in}}%
\pgfpathcurveto{\pgfqpoint{0.901890in}{1.556485in}}{\pgfqpoint{0.905162in}{1.564385in}}{\pgfqpoint{0.905162in}{1.572621in}}%
\pgfpathcurveto{\pgfqpoint{0.905162in}{1.580857in}}{\pgfqpoint{0.901890in}{1.588757in}}{\pgfqpoint{0.896066in}{1.594581in}}%
\pgfpathcurveto{\pgfqpoint{0.890242in}{1.600405in}}{\pgfqpoint{0.882342in}{1.603678in}}{\pgfqpoint{0.874106in}{1.603678in}}%
\pgfpathcurveto{\pgfqpoint{0.865869in}{1.603678in}}{\pgfqpoint{0.857969in}{1.600405in}}{\pgfqpoint{0.852146in}{1.594581in}}%
\pgfpathcurveto{\pgfqpoint{0.846322in}{1.588757in}}{\pgfqpoint{0.843049in}{1.580857in}}{\pgfqpoint{0.843049in}{1.572621in}}%
\pgfpathcurveto{\pgfqpoint{0.843049in}{1.564385in}}{\pgfqpoint{0.846322in}{1.556485in}}{\pgfqpoint{0.852146in}{1.550661in}}%
\pgfpathcurveto{\pgfqpoint{0.857969in}{1.544837in}}{\pgfqpoint{0.865869in}{1.541565in}}{\pgfqpoint{0.874106in}{1.541565in}}%
\pgfpathclose%
\pgfusepath{stroke,fill}%
\end{pgfscope}%
\begin{pgfscope}%
\pgfpathrectangle{\pgfqpoint{0.100000in}{0.212622in}}{\pgfqpoint{3.696000in}{3.696000in}}%
\pgfusepath{clip}%
\pgfsetbuttcap%
\pgfsetroundjoin%
\definecolor{currentfill}{rgb}{0.121569,0.466667,0.705882}%
\pgfsetfillcolor{currentfill}%
\pgfsetfillopacity{0.569741}%
\pgfsetlinewidth{1.003750pt}%
\definecolor{currentstroke}{rgb}{0.121569,0.466667,0.705882}%
\pgfsetstrokecolor{currentstroke}%
\pgfsetstrokeopacity{0.569741}%
\pgfsetdash{}{0pt}%
\pgfpathmoveto{\pgfqpoint{0.873464in}{1.540617in}}%
\pgfpathcurveto{\pgfqpoint{0.881700in}{1.540617in}}{\pgfqpoint{0.889600in}{1.543889in}}{\pgfqpoint{0.895424in}{1.549713in}}%
\pgfpathcurveto{\pgfqpoint{0.901248in}{1.555537in}}{\pgfqpoint{0.904520in}{1.563437in}}{\pgfqpoint{0.904520in}{1.571673in}}%
\pgfpathcurveto{\pgfqpoint{0.904520in}{1.579910in}}{\pgfqpoint{0.901248in}{1.587810in}}{\pgfqpoint{0.895424in}{1.593634in}}%
\pgfpathcurveto{\pgfqpoint{0.889600in}{1.599458in}}{\pgfqpoint{0.881700in}{1.602730in}}{\pgfqpoint{0.873464in}{1.602730in}}%
\pgfpathcurveto{\pgfqpoint{0.865227in}{1.602730in}}{\pgfqpoint{0.857327in}{1.599458in}}{\pgfqpoint{0.851503in}{1.593634in}}%
\pgfpathcurveto{\pgfqpoint{0.845679in}{1.587810in}}{\pgfqpoint{0.842407in}{1.579910in}}{\pgfqpoint{0.842407in}{1.571673in}}%
\pgfpathcurveto{\pgfqpoint{0.842407in}{1.563437in}}{\pgfqpoint{0.845679in}{1.555537in}}{\pgfqpoint{0.851503in}{1.549713in}}%
\pgfpathcurveto{\pgfqpoint{0.857327in}{1.543889in}}{\pgfqpoint{0.865227in}{1.540617in}}{\pgfqpoint{0.873464in}{1.540617in}}%
\pgfpathclose%
\pgfusepath{stroke,fill}%
\end{pgfscope}%
\begin{pgfscope}%
\pgfpathrectangle{\pgfqpoint{0.100000in}{0.212622in}}{\pgfqpoint{3.696000in}{3.696000in}}%
\pgfusepath{clip}%
\pgfsetbuttcap%
\pgfsetroundjoin%
\definecolor{currentfill}{rgb}{0.121569,0.466667,0.705882}%
\pgfsetfillcolor{currentfill}%
\pgfsetfillopacity{0.570130}%
\pgfsetlinewidth{1.003750pt}%
\definecolor{currentstroke}{rgb}{0.121569,0.466667,0.705882}%
\pgfsetstrokecolor{currentstroke}%
\pgfsetstrokeopacity{0.570130}%
\pgfsetdash{}{0pt}%
\pgfpathmoveto{\pgfqpoint{3.242685in}{2.229540in}}%
\pgfpathcurveto{\pgfqpoint{3.250922in}{2.229540in}}{\pgfqpoint{3.258822in}{2.232813in}}{\pgfqpoint{3.264646in}{2.238637in}}%
\pgfpathcurveto{\pgfqpoint{3.270470in}{2.244461in}}{\pgfqpoint{3.273742in}{2.252361in}}{\pgfqpoint{3.273742in}{2.260597in}}%
\pgfpathcurveto{\pgfqpoint{3.273742in}{2.268833in}}{\pgfqpoint{3.270470in}{2.276733in}}{\pgfqpoint{3.264646in}{2.282557in}}%
\pgfpathcurveto{\pgfqpoint{3.258822in}{2.288381in}}{\pgfqpoint{3.250922in}{2.291653in}}{\pgfqpoint{3.242685in}{2.291653in}}%
\pgfpathcurveto{\pgfqpoint{3.234449in}{2.291653in}}{\pgfqpoint{3.226549in}{2.288381in}}{\pgfqpoint{3.220725in}{2.282557in}}%
\pgfpathcurveto{\pgfqpoint{3.214901in}{2.276733in}}{\pgfqpoint{3.211629in}{2.268833in}}{\pgfqpoint{3.211629in}{2.260597in}}%
\pgfpathcurveto{\pgfqpoint{3.211629in}{2.252361in}}{\pgfqpoint{3.214901in}{2.244461in}}{\pgfqpoint{3.220725in}{2.238637in}}%
\pgfpathcurveto{\pgfqpoint{3.226549in}{2.232813in}}{\pgfqpoint{3.234449in}{2.229540in}}{\pgfqpoint{3.242685in}{2.229540in}}%
\pgfpathclose%
\pgfusepath{stroke,fill}%
\end{pgfscope}%
\begin{pgfscope}%
\pgfpathrectangle{\pgfqpoint{0.100000in}{0.212622in}}{\pgfqpoint{3.696000in}{3.696000in}}%
\pgfusepath{clip}%
\pgfsetbuttcap%
\pgfsetroundjoin%
\definecolor{currentfill}{rgb}{0.121569,0.466667,0.705882}%
\pgfsetfillcolor{currentfill}%
\pgfsetfillopacity{0.570163}%
\pgfsetlinewidth{1.003750pt}%
\definecolor{currentstroke}{rgb}{0.121569,0.466667,0.705882}%
\pgfsetstrokecolor{currentstroke}%
\pgfsetstrokeopacity{0.570163}%
\pgfsetdash{}{0pt}%
\pgfpathmoveto{\pgfqpoint{0.872282in}{1.538836in}}%
\pgfpathcurveto{\pgfqpoint{0.880519in}{1.538836in}}{\pgfqpoint{0.888419in}{1.542108in}}{\pgfqpoint{0.894243in}{1.547932in}}%
\pgfpathcurveto{\pgfqpoint{0.900066in}{1.553756in}}{\pgfqpoint{0.903339in}{1.561656in}}{\pgfqpoint{0.903339in}{1.569893in}}%
\pgfpathcurveto{\pgfqpoint{0.903339in}{1.578129in}}{\pgfqpoint{0.900066in}{1.586029in}}{\pgfqpoint{0.894243in}{1.591853in}}%
\pgfpathcurveto{\pgfqpoint{0.888419in}{1.597677in}}{\pgfqpoint{0.880519in}{1.600949in}}{\pgfqpoint{0.872282in}{1.600949in}}%
\pgfpathcurveto{\pgfqpoint{0.864046in}{1.600949in}}{\pgfqpoint{0.856146in}{1.597677in}}{\pgfqpoint{0.850322in}{1.591853in}}%
\pgfpathcurveto{\pgfqpoint{0.844498in}{1.586029in}}{\pgfqpoint{0.841226in}{1.578129in}}{\pgfqpoint{0.841226in}{1.569893in}}%
\pgfpathcurveto{\pgfqpoint{0.841226in}{1.561656in}}{\pgfqpoint{0.844498in}{1.553756in}}{\pgfqpoint{0.850322in}{1.547932in}}%
\pgfpathcurveto{\pgfqpoint{0.856146in}{1.542108in}}{\pgfqpoint{0.864046in}{1.538836in}}{\pgfqpoint{0.872282in}{1.538836in}}%
\pgfpathclose%
\pgfusepath{stroke,fill}%
\end{pgfscope}%
\begin{pgfscope}%
\pgfpathrectangle{\pgfqpoint{0.100000in}{0.212622in}}{\pgfqpoint{3.696000in}{3.696000in}}%
\pgfusepath{clip}%
\pgfsetbuttcap%
\pgfsetroundjoin%
\definecolor{currentfill}{rgb}{0.121569,0.466667,0.705882}%
\pgfsetfillcolor{currentfill}%
\pgfsetfillopacity{0.570381}%
\pgfsetlinewidth{1.003750pt}%
\definecolor{currentstroke}{rgb}{0.121569,0.466667,0.705882}%
\pgfsetstrokecolor{currentstroke}%
\pgfsetstrokeopacity{0.570381}%
\pgfsetdash{}{0pt}%
\pgfpathmoveto{\pgfqpoint{0.871699in}{1.537991in}}%
\pgfpathcurveto{\pgfqpoint{0.879935in}{1.537991in}}{\pgfqpoint{0.887835in}{1.541263in}}{\pgfqpoint{0.893659in}{1.547087in}}%
\pgfpathcurveto{\pgfqpoint{0.899483in}{1.552911in}}{\pgfqpoint{0.902755in}{1.560811in}}{\pgfqpoint{0.902755in}{1.569048in}}%
\pgfpathcurveto{\pgfqpoint{0.902755in}{1.577284in}}{\pgfqpoint{0.899483in}{1.585184in}}{\pgfqpoint{0.893659in}{1.591008in}}%
\pgfpathcurveto{\pgfqpoint{0.887835in}{1.596832in}}{\pgfqpoint{0.879935in}{1.600104in}}{\pgfqpoint{0.871699in}{1.600104in}}%
\pgfpathcurveto{\pgfqpoint{0.863463in}{1.600104in}}{\pgfqpoint{0.855563in}{1.596832in}}{\pgfqpoint{0.849739in}{1.591008in}}%
\pgfpathcurveto{\pgfqpoint{0.843915in}{1.585184in}}{\pgfqpoint{0.840642in}{1.577284in}}{\pgfqpoint{0.840642in}{1.569048in}}%
\pgfpathcurveto{\pgfqpoint{0.840642in}{1.560811in}}{\pgfqpoint{0.843915in}{1.552911in}}{\pgfqpoint{0.849739in}{1.547087in}}%
\pgfpathcurveto{\pgfqpoint{0.855563in}{1.541263in}}{\pgfqpoint{0.863463in}{1.537991in}}{\pgfqpoint{0.871699in}{1.537991in}}%
\pgfpathclose%
\pgfusepath{stroke,fill}%
\end{pgfscope}%
\begin{pgfscope}%
\pgfpathrectangle{\pgfqpoint{0.100000in}{0.212622in}}{\pgfqpoint{3.696000in}{3.696000in}}%
\pgfusepath{clip}%
\pgfsetbuttcap%
\pgfsetroundjoin%
\definecolor{currentfill}{rgb}{0.121569,0.466667,0.705882}%
\pgfsetfillcolor{currentfill}%
\pgfsetfillopacity{0.570479}%
\pgfsetlinewidth{1.003750pt}%
\definecolor{currentstroke}{rgb}{0.121569,0.466667,0.705882}%
\pgfsetstrokecolor{currentstroke}%
\pgfsetstrokeopacity{0.570479}%
\pgfsetdash{}{0pt}%
\pgfpathmoveto{\pgfqpoint{0.871428in}{1.537573in}}%
\pgfpathcurveto{\pgfqpoint{0.879664in}{1.537573in}}{\pgfqpoint{0.887564in}{1.540845in}}{\pgfqpoint{0.893388in}{1.546669in}}%
\pgfpathcurveto{\pgfqpoint{0.899212in}{1.552493in}}{\pgfqpoint{0.902484in}{1.560393in}}{\pgfqpoint{0.902484in}{1.568629in}}%
\pgfpathcurveto{\pgfqpoint{0.902484in}{1.576866in}}{\pgfqpoint{0.899212in}{1.584766in}}{\pgfqpoint{0.893388in}{1.590590in}}%
\pgfpathcurveto{\pgfqpoint{0.887564in}{1.596413in}}{\pgfqpoint{0.879664in}{1.599686in}}{\pgfqpoint{0.871428in}{1.599686in}}%
\pgfpathcurveto{\pgfqpoint{0.863192in}{1.599686in}}{\pgfqpoint{0.855292in}{1.596413in}}{\pgfqpoint{0.849468in}{1.590590in}}%
\pgfpathcurveto{\pgfqpoint{0.843644in}{1.584766in}}{\pgfqpoint{0.840371in}{1.576866in}}{\pgfqpoint{0.840371in}{1.568629in}}%
\pgfpathcurveto{\pgfqpoint{0.840371in}{1.560393in}}{\pgfqpoint{0.843644in}{1.552493in}}{\pgfqpoint{0.849468in}{1.546669in}}%
\pgfpathcurveto{\pgfqpoint{0.855292in}{1.540845in}}{\pgfqpoint{0.863192in}{1.537573in}}{\pgfqpoint{0.871428in}{1.537573in}}%
\pgfpathclose%
\pgfusepath{stroke,fill}%
\end{pgfscope}%
\begin{pgfscope}%
\pgfpathrectangle{\pgfqpoint{0.100000in}{0.212622in}}{\pgfqpoint{3.696000in}{3.696000in}}%
\pgfusepath{clip}%
\pgfsetbuttcap%
\pgfsetroundjoin%
\definecolor{currentfill}{rgb}{0.121569,0.466667,0.705882}%
\pgfsetfillcolor{currentfill}%
\pgfsetfillopacity{0.570497}%
\pgfsetlinewidth{1.003750pt}%
\definecolor{currentstroke}{rgb}{0.121569,0.466667,0.705882}%
\pgfsetstrokecolor{currentstroke}%
\pgfsetstrokeopacity{0.570497}%
\pgfsetdash{}{0pt}%
\pgfpathmoveto{\pgfqpoint{3.241941in}{2.229156in}}%
\pgfpathcurveto{\pgfqpoint{3.250178in}{2.229156in}}{\pgfqpoint{3.258078in}{2.232428in}}{\pgfqpoint{3.263902in}{2.238252in}}%
\pgfpathcurveto{\pgfqpoint{3.269725in}{2.244076in}}{\pgfqpoint{3.272998in}{2.251976in}}{\pgfqpoint{3.272998in}{2.260212in}}%
\pgfpathcurveto{\pgfqpoint{3.272998in}{2.268449in}}{\pgfqpoint{3.269725in}{2.276349in}}{\pgfqpoint{3.263902in}{2.282173in}}%
\pgfpathcurveto{\pgfqpoint{3.258078in}{2.287996in}}{\pgfqpoint{3.250178in}{2.291269in}}{\pgfqpoint{3.241941in}{2.291269in}}%
\pgfpathcurveto{\pgfqpoint{3.233705in}{2.291269in}}{\pgfqpoint{3.225805in}{2.287996in}}{\pgfqpoint{3.219981in}{2.282173in}}%
\pgfpathcurveto{\pgfqpoint{3.214157in}{2.276349in}}{\pgfqpoint{3.210885in}{2.268449in}}{\pgfqpoint{3.210885in}{2.260212in}}%
\pgfpathcurveto{\pgfqpoint{3.210885in}{2.251976in}}{\pgfqpoint{3.214157in}{2.244076in}}{\pgfqpoint{3.219981in}{2.238252in}}%
\pgfpathcurveto{\pgfqpoint{3.225805in}{2.232428in}}{\pgfqpoint{3.233705in}{2.229156in}}{\pgfqpoint{3.241941in}{2.229156in}}%
\pgfpathclose%
\pgfusepath{stroke,fill}%
\end{pgfscope}%
\begin{pgfscope}%
\pgfpathrectangle{\pgfqpoint{0.100000in}{0.212622in}}{\pgfqpoint{3.696000in}{3.696000in}}%
\pgfusepath{clip}%
\pgfsetbuttcap%
\pgfsetroundjoin%
\definecolor{currentfill}{rgb}{0.121569,0.466667,0.705882}%
\pgfsetfillcolor{currentfill}%
\pgfsetfillopacity{0.570670}%
\pgfsetlinewidth{1.003750pt}%
\definecolor{currentstroke}{rgb}{0.121569,0.466667,0.705882}%
\pgfsetstrokecolor{currentstroke}%
\pgfsetstrokeopacity{0.570670}%
\pgfsetdash{}{0pt}%
\pgfpathmoveto{\pgfqpoint{0.870947in}{1.536876in}}%
\pgfpathcurveto{\pgfqpoint{0.879183in}{1.536876in}}{\pgfqpoint{0.887083in}{1.540148in}}{\pgfqpoint{0.892907in}{1.545972in}}%
\pgfpathcurveto{\pgfqpoint{0.898731in}{1.551796in}}{\pgfqpoint{0.902003in}{1.559696in}}{\pgfqpoint{0.902003in}{1.567932in}}%
\pgfpathcurveto{\pgfqpoint{0.902003in}{1.576168in}}{\pgfqpoint{0.898731in}{1.584068in}}{\pgfqpoint{0.892907in}{1.589892in}}%
\pgfpathcurveto{\pgfqpoint{0.887083in}{1.595716in}}{\pgfqpoint{0.879183in}{1.598989in}}{\pgfqpoint{0.870947in}{1.598989in}}%
\pgfpathcurveto{\pgfqpoint{0.862710in}{1.598989in}}{\pgfqpoint{0.854810in}{1.595716in}}{\pgfqpoint{0.848986in}{1.589892in}}%
\pgfpathcurveto{\pgfqpoint{0.843163in}{1.584068in}}{\pgfqpoint{0.839890in}{1.576168in}}{\pgfqpoint{0.839890in}{1.567932in}}%
\pgfpathcurveto{\pgfqpoint{0.839890in}{1.559696in}}{\pgfqpoint{0.843163in}{1.551796in}}{\pgfqpoint{0.848986in}{1.545972in}}%
\pgfpathcurveto{\pgfqpoint{0.854810in}{1.540148in}}{\pgfqpoint{0.862710in}{1.536876in}}{\pgfqpoint{0.870947in}{1.536876in}}%
\pgfpathclose%
\pgfusepath{stroke,fill}%
\end{pgfscope}%
\begin{pgfscope}%
\pgfpathrectangle{\pgfqpoint{0.100000in}{0.212622in}}{\pgfqpoint{3.696000in}{3.696000in}}%
\pgfusepath{clip}%
\pgfsetbuttcap%
\pgfsetroundjoin%
\definecolor{currentfill}{rgb}{0.121569,0.466667,0.705882}%
\pgfsetfillcolor{currentfill}%
\pgfsetfillopacity{0.571010}%
\pgfsetlinewidth{1.003750pt}%
\definecolor{currentstroke}{rgb}{0.121569,0.466667,0.705882}%
\pgfsetstrokecolor{currentstroke}%
\pgfsetstrokeopacity{0.571010}%
\pgfsetdash{}{0pt}%
\pgfpathmoveto{\pgfqpoint{0.870192in}{1.535472in}}%
\pgfpathcurveto{\pgfqpoint{0.878429in}{1.535472in}}{\pgfqpoint{0.886329in}{1.538744in}}{\pgfqpoint{0.892153in}{1.544568in}}%
\pgfpathcurveto{\pgfqpoint{0.897977in}{1.550392in}}{\pgfqpoint{0.901249in}{1.558292in}}{\pgfqpoint{0.901249in}{1.566529in}}%
\pgfpathcurveto{\pgfqpoint{0.901249in}{1.574765in}}{\pgfqpoint{0.897977in}{1.582665in}}{\pgfqpoint{0.892153in}{1.588489in}}%
\pgfpathcurveto{\pgfqpoint{0.886329in}{1.594313in}}{\pgfqpoint{0.878429in}{1.597585in}}{\pgfqpoint{0.870192in}{1.597585in}}%
\pgfpathcurveto{\pgfqpoint{0.861956in}{1.597585in}}{\pgfqpoint{0.854056in}{1.594313in}}{\pgfqpoint{0.848232in}{1.588489in}}%
\pgfpathcurveto{\pgfqpoint{0.842408in}{1.582665in}}{\pgfqpoint{0.839136in}{1.574765in}}{\pgfqpoint{0.839136in}{1.566529in}}%
\pgfpathcurveto{\pgfqpoint{0.839136in}{1.558292in}}{\pgfqpoint{0.842408in}{1.550392in}}{\pgfqpoint{0.848232in}{1.544568in}}%
\pgfpathcurveto{\pgfqpoint{0.854056in}{1.538744in}}{\pgfqpoint{0.861956in}{1.535472in}}{\pgfqpoint{0.870192in}{1.535472in}}%
\pgfpathclose%
\pgfusepath{stroke,fill}%
\end{pgfscope}%
\begin{pgfscope}%
\pgfpathrectangle{\pgfqpoint{0.100000in}{0.212622in}}{\pgfqpoint{3.696000in}{3.696000in}}%
\pgfusepath{clip}%
\pgfsetbuttcap%
\pgfsetroundjoin%
\definecolor{currentfill}{rgb}{0.121569,0.466667,0.705882}%
\pgfsetfillcolor{currentfill}%
\pgfsetfillopacity{0.571075}%
\pgfsetlinewidth{1.003750pt}%
\definecolor{currentstroke}{rgb}{0.121569,0.466667,0.705882}%
\pgfsetstrokecolor{currentstroke}%
\pgfsetstrokeopacity{0.571075}%
\pgfsetdash{}{0pt}%
\pgfpathmoveto{\pgfqpoint{3.240810in}{2.228570in}}%
\pgfpathcurveto{\pgfqpoint{3.249046in}{2.228570in}}{\pgfqpoint{3.256946in}{2.231842in}}{\pgfqpoint{3.262770in}{2.237666in}}%
\pgfpathcurveto{\pgfqpoint{3.268594in}{2.243490in}}{\pgfqpoint{3.271866in}{2.251390in}}{\pgfqpoint{3.271866in}{2.259627in}}%
\pgfpathcurveto{\pgfqpoint{3.271866in}{2.267863in}}{\pgfqpoint{3.268594in}{2.275763in}}{\pgfqpoint{3.262770in}{2.281587in}}%
\pgfpathcurveto{\pgfqpoint{3.256946in}{2.287411in}}{\pgfqpoint{3.249046in}{2.290683in}}{\pgfqpoint{3.240810in}{2.290683in}}%
\pgfpathcurveto{\pgfqpoint{3.232574in}{2.290683in}}{\pgfqpoint{3.224674in}{2.287411in}}{\pgfqpoint{3.218850in}{2.281587in}}%
\pgfpathcurveto{\pgfqpoint{3.213026in}{2.275763in}}{\pgfqpoint{3.209753in}{2.267863in}}{\pgfqpoint{3.209753in}{2.259627in}}%
\pgfpathcurveto{\pgfqpoint{3.209753in}{2.251390in}}{\pgfqpoint{3.213026in}{2.243490in}}{\pgfqpoint{3.218850in}{2.237666in}}%
\pgfpathcurveto{\pgfqpoint{3.224674in}{2.231842in}}{\pgfqpoint{3.232574in}{2.228570in}}{\pgfqpoint{3.240810in}{2.228570in}}%
\pgfpathclose%
\pgfusepath{stroke,fill}%
\end{pgfscope}%
\begin{pgfscope}%
\pgfpathrectangle{\pgfqpoint{0.100000in}{0.212622in}}{\pgfqpoint{3.696000in}{3.696000in}}%
\pgfusepath{clip}%
\pgfsetbuttcap%
\pgfsetroundjoin%
\definecolor{currentfill}{rgb}{0.121569,0.466667,0.705882}%
\pgfsetfillcolor{currentfill}%
\pgfsetfillopacity{0.571628}%
\pgfsetlinewidth{1.003750pt}%
\definecolor{currentstroke}{rgb}{0.121569,0.466667,0.705882}%
\pgfsetstrokecolor{currentstroke}%
\pgfsetstrokeopacity{0.571628}%
\pgfsetdash{}{0pt}%
\pgfpathmoveto{\pgfqpoint{0.868753in}{1.532967in}}%
\pgfpathcurveto{\pgfqpoint{0.876989in}{1.532967in}}{\pgfqpoint{0.884889in}{1.536239in}}{\pgfqpoint{0.890713in}{1.542063in}}%
\pgfpathcurveto{\pgfqpoint{0.896537in}{1.547887in}}{\pgfqpoint{0.899810in}{1.555787in}}{\pgfqpoint{0.899810in}{1.564024in}}%
\pgfpathcurveto{\pgfqpoint{0.899810in}{1.572260in}}{\pgfqpoint{0.896537in}{1.580160in}}{\pgfqpoint{0.890713in}{1.585984in}}%
\pgfpathcurveto{\pgfqpoint{0.884889in}{1.591808in}}{\pgfqpoint{0.876989in}{1.595080in}}{\pgfqpoint{0.868753in}{1.595080in}}%
\pgfpathcurveto{\pgfqpoint{0.860517in}{1.595080in}}{\pgfqpoint{0.852617in}{1.591808in}}{\pgfqpoint{0.846793in}{1.585984in}}%
\pgfpathcurveto{\pgfqpoint{0.840969in}{1.580160in}}{\pgfqpoint{0.837697in}{1.572260in}}{\pgfqpoint{0.837697in}{1.564024in}}%
\pgfpathcurveto{\pgfqpoint{0.837697in}{1.555787in}}{\pgfqpoint{0.840969in}{1.547887in}}{\pgfqpoint{0.846793in}{1.542063in}}%
\pgfpathcurveto{\pgfqpoint{0.852617in}{1.536239in}}{\pgfqpoint{0.860517in}{1.532967in}}{\pgfqpoint{0.868753in}{1.532967in}}%
\pgfpathclose%
\pgfusepath{stroke,fill}%
\end{pgfscope}%
\begin{pgfscope}%
\pgfpathrectangle{\pgfqpoint{0.100000in}{0.212622in}}{\pgfqpoint{3.696000in}{3.696000in}}%
\pgfusepath{clip}%
\pgfsetbuttcap%
\pgfsetroundjoin%
\definecolor{currentfill}{rgb}{0.121569,0.466667,0.705882}%
\pgfsetfillcolor{currentfill}%
\pgfsetfillopacity{0.572139}%
\pgfsetlinewidth{1.003750pt}%
\definecolor{currentstroke}{rgb}{0.121569,0.466667,0.705882}%
\pgfsetstrokecolor{currentstroke}%
\pgfsetstrokeopacity{0.572139}%
\pgfsetdash{}{0pt}%
\pgfpathmoveto{\pgfqpoint{3.238724in}{2.227970in}}%
\pgfpathcurveto{\pgfqpoint{3.246960in}{2.227970in}}{\pgfqpoint{3.254861in}{2.231242in}}{\pgfqpoint{3.260684in}{2.237066in}}%
\pgfpathcurveto{\pgfqpoint{3.266508in}{2.242890in}}{\pgfqpoint{3.269781in}{2.250790in}}{\pgfqpoint{3.269781in}{2.259026in}}%
\pgfpathcurveto{\pgfqpoint{3.269781in}{2.267262in}}{\pgfqpoint{3.266508in}{2.275162in}}{\pgfqpoint{3.260684in}{2.280986in}}%
\pgfpathcurveto{\pgfqpoint{3.254861in}{2.286810in}}{\pgfqpoint{3.246960in}{2.290083in}}{\pgfqpoint{3.238724in}{2.290083in}}%
\pgfpathcurveto{\pgfqpoint{3.230488in}{2.290083in}}{\pgfqpoint{3.222588in}{2.286810in}}{\pgfqpoint{3.216764in}{2.280986in}}%
\pgfpathcurveto{\pgfqpoint{3.210940in}{2.275162in}}{\pgfqpoint{3.207668in}{2.267262in}}{\pgfqpoint{3.207668in}{2.259026in}}%
\pgfpathcurveto{\pgfqpoint{3.207668in}{2.250790in}}{\pgfqpoint{3.210940in}{2.242890in}}{\pgfqpoint{3.216764in}{2.237066in}}%
\pgfpathcurveto{\pgfqpoint{3.222588in}{2.231242in}}{\pgfqpoint{3.230488in}{2.227970in}}{\pgfqpoint{3.238724in}{2.227970in}}%
\pgfpathclose%
\pgfusepath{stroke,fill}%
\end{pgfscope}%
\begin{pgfscope}%
\pgfpathrectangle{\pgfqpoint{0.100000in}{0.212622in}}{\pgfqpoint{3.696000in}{3.696000in}}%
\pgfusepath{clip}%
\pgfsetbuttcap%
\pgfsetroundjoin%
\definecolor{currentfill}{rgb}{0.121569,0.466667,0.705882}%
\pgfsetfillcolor{currentfill}%
\pgfsetfillopacity{0.572740}%
\pgfsetlinewidth{1.003750pt}%
\definecolor{currentstroke}{rgb}{0.121569,0.466667,0.705882}%
\pgfsetstrokecolor{currentstroke}%
\pgfsetstrokeopacity{0.572740}%
\pgfsetdash{}{0pt}%
\pgfpathmoveto{\pgfqpoint{3.237618in}{2.227699in}}%
\pgfpathcurveto{\pgfqpoint{3.245855in}{2.227699in}}{\pgfqpoint{3.253755in}{2.230971in}}{\pgfqpoint{3.259579in}{2.236795in}}%
\pgfpathcurveto{\pgfqpoint{3.265403in}{2.242619in}}{\pgfqpoint{3.268675in}{2.250519in}}{\pgfqpoint{3.268675in}{2.258755in}}%
\pgfpathcurveto{\pgfqpoint{3.268675in}{2.266991in}}{\pgfqpoint{3.265403in}{2.274891in}}{\pgfqpoint{3.259579in}{2.280715in}}%
\pgfpathcurveto{\pgfqpoint{3.253755in}{2.286539in}}{\pgfqpoint{3.245855in}{2.289812in}}{\pgfqpoint{3.237618in}{2.289812in}}%
\pgfpathcurveto{\pgfqpoint{3.229382in}{2.289812in}}{\pgfqpoint{3.221482in}{2.286539in}}{\pgfqpoint{3.215658in}{2.280715in}}%
\pgfpathcurveto{\pgfqpoint{3.209834in}{2.274891in}}{\pgfqpoint{3.206562in}{2.266991in}}{\pgfqpoint{3.206562in}{2.258755in}}%
\pgfpathcurveto{\pgfqpoint{3.206562in}{2.250519in}}{\pgfqpoint{3.209834in}{2.242619in}}{\pgfqpoint{3.215658in}{2.236795in}}%
\pgfpathcurveto{\pgfqpoint{3.221482in}{2.230971in}}{\pgfqpoint{3.229382in}{2.227699in}}{\pgfqpoint{3.237618in}{2.227699in}}%
\pgfpathclose%
\pgfusepath{stroke,fill}%
\end{pgfscope}%
\begin{pgfscope}%
\pgfpathrectangle{\pgfqpoint{0.100000in}{0.212622in}}{\pgfqpoint{3.696000in}{3.696000in}}%
\pgfusepath{clip}%
\pgfsetbuttcap%
\pgfsetroundjoin%
\definecolor{currentfill}{rgb}{0.121569,0.466667,0.705882}%
\pgfsetfillcolor{currentfill}%
\pgfsetfillopacity{0.572760}%
\pgfsetlinewidth{1.003750pt}%
\definecolor{currentstroke}{rgb}{0.121569,0.466667,0.705882}%
\pgfsetstrokecolor{currentstroke}%
\pgfsetstrokeopacity{0.572760}%
\pgfsetdash{}{0pt}%
\pgfpathmoveto{\pgfqpoint{0.866548in}{1.528160in}}%
\pgfpathcurveto{\pgfqpoint{0.874784in}{1.528160in}}{\pgfqpoint{0.882684in}{1.531432in}}{\pgfqpoint{0.888508in}{1.537256in}}%
\pgfpathcurveto{\pgfqpoint{0.894332in}{1.543080in}}{\pgfqpoint{0.897604in}{1.550980in}}{\pgfqpoint{0.897604in}{1.559217in}}%
\pgfpathcurveto{\pgfqpoint{0.897604in}{1.567453in}}{\pgfqpoint{0.894332in}{1.575353in}}{\pgfqpoint{0.888508in}{1.581177in}}%
\pgfpathcurveto{\pgfqpoint{0.882684in}{1.587001in}}{\pgfqpoint{0.874784in}{1.590273in}}{\pgfqpoint{0.866548in}{1.590273in}}%
\pgfpathcurveto{\pgfqpoint{0.858311in}{1.590273in}}{\pgfqpoint{0.850411in}{1.587001in}}{\pgfqpoint{0.844587in}{1.581177in}}%
\pgfpathcurveto{\pgfqpoint{0.838763in}{1.575353in}}{\pgfqpoint{0.835491in}{1.567453in}}{\pgfqpoint{0.835491in}{1.559217in}}%
\pgfpathcurveto{\pgfqpoint{0.835491in}{1.550980in}}{\pgfqpoint{0.838763in}{1.543080in}}{\pgfqpoint{0.844587in}{1.537256in}}%
\pgfpathcurveto{\pgfqpoint{0.850411in}{1.531432in}}{\pgfqpoint{0.858311in}{1.528160in}}{\pgfqpoint{0.866548in}{1.528160in}}%
\pgfpathclose%
\pgfusepath{stroke,fill}%
\end{pgfscope}%
\begin{pgfscope}%
\pgfpathrectangle{\pgfqpoint{0.100000in}{0.212622in}}{\pgfqpoint{3.696000in}{3.696000in}}%
\pgfusepath{clip}%
\pgfsetbuttcap%
\pgfsetroundjoin%
\definecolor{currentfill}{rgb}{0.121569,0.466667,0.705882}%
\pgfsetfillcolor{currentfill}%
\pgfsetfillopacity{0.573078}%
\pgfsetlinewidth{1.003750pt}%
\definecolor{currentstroke}{rgb}{0.121569,0.466667,0.705882}%
\pgfsetstrokecolor{currentstroke}%
\pgfsetstrokeopacity{0.573078}%
\pgfsetdash{}{0pt}%
\pgfpathmoveto{\pgfqpoint{3.237036in}{2.227564in}}%
\pgfpathcurveto{\pgfqpoint{3.245272in}{2.227564in}}{\pgfqpoint{3.253172in}{2.230836in}}{\pgfqpoint{3.258996in}{2.236660in}}%
\pgfpathcurveto{\pgfqpoint{3.264820in}{2.242484in}}{\pgfqpoint{3.268093in}{2.250384in}}{\pgfqpoint{3.268093in}{2.258621in}}%
\pgfpathcurveto{\pgfqpoint{3.268093in}{2.266857in}}{\pgfqpoint{3.264820in}{2.274757in}}{\pgfqpoint{3.258996in}{2.280581in}}%
\pgfpathcurveto{\pgfqpoint{3.253172in}{2.286405in}}{\pgfqpoint{3.245272in}{2.289677in}}{\pgfqpoint{3.237036in}{2.289677in}}%
\pgfpathcurveto{\pgfqpoint{3.228800in}{2.289677in}}{\pgfqpoint{3.220900in}{2.286405in}}{\pgfqpoint{3.215076in}{2.280581in}}%
\pgfpathcurveto{\pgfqpoint{3.209252in}{2.274757in}}{\pgfqpoint{3.205980in}{2.266857in}}{\pgfqpoint{3.205980in}{2.258621in}}%
\pgfpathcurveto{\pgfqpoint{3.205980in}{2.250384in}}{\pgfqpoint{3.209252in}{2.242484in}}{\pgfqpoint{3.215076in}{2.236660in}}%
\pgfpathcurveto{\pgfqpoint{3.220900in}{2.230836in}}{\pgfqpoint{3.228800in}{2.227564in}}{\pgfqpoint{3.237036in}{2.227564in}}%
\pgfpathclose%
\pgfusepath{stroke,fill}%
\end{pgfscope}%
\begin{pgfscope}%
\pgfpathrectangle{\pgfqpoint{0.100000in}{0.212622in}}{\pgfqpoint{3.696000in}{3.696000in}}%
\pgfusepath{clip}%
\pgfsetbuttcap%
\pgfsetroundjoin%
\definecolor{currentfill}{rgb}{0.121569,0.466667,0.705882}%
\pgfsetfillcolor{currentfill}%
\pgfsetfillopacity{0.573261}%
\pgfsetlinewidth{1.003750pt}%
\definecolor{currentstroke}{rgb}{0.121569,0.466667,0.705882}%
\pgfsetstrokecolor{currentstroke}%
\pgfsetstrokeopacity{0.573261}%
\pgfsetdash{}{0pt}%
\pgfpathmoveto{\pgfqpoint{3.236692in}{2.227503in}}%
\pgfpathcurveto{\pgfqpoint{3.244929in}{2.227503in}}{\pgfqpoint{3.252829in}{2.230775in}}{\pgfqpoint{3.258653in}{2.236599in}}%
\pgfpathcurveto{\pgfqpoint{3.264477in}{2.242423in}}{\pgfqpoint{3.267749in}{2.250323in}}{\pgfqpoint{3.267749in}{2.258560in}}%
\pgfpathcurveto{\pgfqpoint{3.267749in}{2.266796in}}{\pgfqpoint{3.264477in}{2.274696in}}{\pgfqpoint{3.258653in}{2.280520in}}%
\pgfpathcurveto{\pgfqpoint{3.252829in}{2.286344in}}{\pgfqpoint{3.244929in}{2.289616in}}{\pgfqpoint{3.236692in}{2.289616in}}%
\pgfpathcurveto{\pgfqpoint{3.228456in}{2.289616in}}{\pgfqpoint{3.220556in}{2.286344in}}{\pgfqpoint{3.214732in}{2.280520in}}%
\pgfpathcurveto{\pgfqpoint{3.208908in}{2.274696in}}{\pgfqpoint{3.205636in}{2.266796in}}{\pgfqpoint{3.205636in}{2.258560in}}%
\pgfpathcurveto{\pgfqpoint{3.205636in}{2.250323in}}{\pgfqpoint{3.208908in}{2.242423in}}{\pgfqpoint{3.214732in}{2.236599in}}%
\pgfpathcurveto{\pgfqpoint{3.220556in}{2.230775in}}{\pgfqpoint{3.228456in}{2.227503in}}{\pgfqpoint{3.236692in}{2.227503in}}%
\pgfpathclose%
\pgfusepath{stroke,fill}%
\end{pgfscope}%
\begin{pgfscope}%
\pgfpathrectangle{\pgfqpoint{0.100000in}{0.212622in}}{\pgfqpoint{3.696000in}{3.696000in}}%
\pgfusepath{clip}%
\pgfsetbuttcap%
\pgfsetroundjoin%
\definecolor{currentfill}{rgb}{0.121569,0.466667,0.705882}%
\pgfsetfillcolor{currentfill}%
\pgfsetfillopacity{0.573519}%
\pgfsetlinewidth{1.003750pt}%
\definecolor{currentstroke}{rgb}{0.121569,0.466667,0.705882}%
\pgfsetstrokecolor{currentstroke}%
\pgfsetstrokeopacity{0.573519}%
\pgfsetdash{}{0pt}%
\pgfpathmoveto{\pgfqpoint{0.865007in}{1.524892in}}%
\pgfpathcurveto{\pgfqpoint{0.873243in}{1.524892in}}{\pgfqpoint{0.881143in}{1.528164in}}{\pgfqpoint{0.886967in}{1.533988in}}%
\pgfpathcurveto{\pgfqpoint{0.892791in}{1.539812in}}{\pgfqpoint{0.896064in}{1.547712in}}{\pgfqpoint{0.896064in}{1.555949in}}%
\pgfpathcurveto{\pgfqpoint{0.896064in}{1.564185in}}{\pgfqpoint{0.892791in}{1.572085in}}{\pgfqpoint{0.886967in}{1.577909in}}%
\pgfpathcurveto{\pgfqpoint{0.881143in}{1.583733in}}{\pgfqpoint{0.873243in}{1.587005in}}{\pgfqpoint{0.865007in}{1.587005in}}%
\pgfpathcurveto{\pgfqpoint{0.856771in}{1.587005in}}{\pgfqpoint{0.848871in}{1.583733in}}{\pgfqpoint{0.843047in}{1.577909in}}%
\pgfpathcurveto{\pgfqpoint{0.837223in}{1.572085in}}{\pgfqpoint{0.833951in}{1.564185in}}{\pgfqpoint{0.833951in}{1.555949in}}%
\pgfpathcurveto{\pgfqpoint{0.833951in}{1.547712in}}{\pgfqpoint{0.837223in}{1.539812in}}{\pgfqpoint{0.843047in}{1.533988in}}%
\pgfpathcurveto{\pgfqpoint{0.848871in}{1.528164in}}{\pgfqpoint{0.856771in}{1.524892in}}{\pgfqpoint{0.865007in}{1.524892in}}%
\pgfpathclose%
\pgfusepath{stroke,fill}%
\end{pgfscope}%
\begin{pgfscope}%
\pgfpathrectangle{\pgfqpoint{0.100000in}{0.212622in}}{\pgfqpoint{3.696000in}{3.696000in}}%
\pgfusepath{clip}%
\pgfsetbuttcap%
\pgfsetroundjoin%
\definecolor{currentfill}{rgb}{0.121569,0.466667,0.705882}%
\pgfsetfillcolor{currentfill}%
\pgfsetfillopacity{0.573853}%
\pgfsetlinewidth{1.003750pt}%
\definecolor{currentstroke}{rgb}{0.121569,0.466667,0.705882}%
\pgfsetstrokecolor{currentstroke}%
\pgfsetstrokeopacity{0.573853}%
\pgfsetdash{}{0pt}%
\pgfpathmoveto{\pgfqpoint{3.235712in}{2.227285in}}%
\pgfpathcurveto{\pgfqpoint{3.243948in}{2.227285in}}{\pgfqpoint{3.251848in}{2.230557in}}{\pgfqpoint{3.257672in}{2.236381in}}%
\pgfpathcurveto{\pgfqpoint{3.263496in}{2.242205in}}{\pgfqpoint{3.266769in}{2.250105in}}{\pgfqpoint{3.266769in}{2.258341in}}%
\pgfpathcurveto{\pgfqpoint{3.266769in}{2.266578in}}{\pgfqpoint{3.263496in}{2.274478in}}{\pgfqpoint{3.257672in}{2.280302in}}%
\pgfpathcurveto{\pgfqpoint{3.251848in}{2.286126in}}{\pgfqpoint{3.243948in}{2.289398in}}{\pgfqpoint{3.235712in}{2.289398in}}%
\pgfpathcurveto{\pgfqpoint{3.227476in}{2.289398in}}{\pgfqpoint{3.219576in}{2.286126in}}{\pgfqpoint{3.213752in}{2.280302in}}%
\pgfpathcurveto{\pgfqpoint{3.207928in}{2.274478in}}{\pgfqpoint{3.204656in}{2.266578in}}{\pgfqpoint{3.204656in}{2.258341in}}%
\pgfpathcurveto{\pgfqpoint{3.204656in}{2.250105in}}{\pgfqpoint{3.207928in}{2.242205in}}{\pgfqpoint{3.213752in}{2.236381in}}%
\pgfpathcurveto{\pgfqpoint{3.219576in}{2.230557in}}{\pgfqpoint{3.227476in}{2.227285in}}{\pgfqpoint{3.235712in}{2.227285in}}%
\pgfpathclose%
\pgfusepath{stroke,fill}%
\end{pgfscope}%
\begin{pgfscope}%
\pgfpathrectangle{\pgfqpoint{0.100000in}{0.212622in}}{\pgfqpoint{3.696000in}{3.696000in}}%
\pgfusepath{clip}%
\pgfsetbuttcap%
\pgfsetroundjoin%
\definecolor{currentfill}{rgb}{0.121569,0.466667,0.705882}%
\pgfsetfillcolor{currentfill}%
\pgfsetfillopacity{0.574186}%
\pgfsetlinewidth{1.003750pt}%
\definecolor{currentstroke}{rgb}{0.121569,0.466667,0.705882}%
\pgfsetstrokecolor{currentstroke}%
\pgfsetstrokeopacity{0.574186}%
\pgfsetdash{}{0pt}%
\pgfpathmoveto{\pgfqpoint{3.235147in}{2.227244in}}%
\pgfpathcurveto{\pgfqpoint{3.243383in}{2.227244in}}{\pgfqpoint{3.251283in}{2.230516in}}{\pgfqpoint{3.257107in}{2.236340in}}%
\pgfpathcurveto{\pgfqpoint{3.262931in}{2.242164in}}{\pgfqpoint{3.266204in}{2.250064in}}{\pgfqpoint{3.266204in}{2.258300in}}%
\pgfpathcurveto{\pgfqpoint{3.266204in}{2.266536in}}{\pgfqpoint{3.262931in}{2.274436in}}{\pgfqpoint{3.257107in}{2.280260in}}%
\pgfpathcurveto{\pgfqpoint{3.251283in}{2.286084in}}{\pgfqpoint{3.243383in}{2.289357in}}{\pgfqpoint{3.235147in}{2.289357in}}%
\pgfpathcurveto{\pgfqpoint{3.226911in}{2.289357in}}{\pgfqpoint{3.219011in}{2.286084in}}{\pgfqpoint{3.213187in}{2.280260in}}%
\pgfpathcurveto{\pgfqpoint{3.207363in}{2.274436in}}{\pgfqpoint{3.204091in}{2.266536in}}{\pgfqpoint{3.204091in}{2.258300in}}%
\pgfpathcurveto{\pgfqpoint{3.204091in}{2.250064in}}{\pgfqpoint{3.207363in}{2.242164in}}{\pgfqpoint{3.213187in}{2.236340in}}%
\pgfpathcurveto{\pgfqpoint{3.219011in}{2.230516in}}{\pgfqpoint{3.226911in}{2.227244in}}{\pgfqpoint{3.235147in}{2.227244in}}%
\pgfpathclose%
\pgfusepath{stroke,fill}%
\end{pgfscope}%
\begin{pgfscope}%
\pgfpathrectangle{\pgfqpoint{0.100000in}{0.212622in}}{\pgfqpoint{3.696000in}{3.696000in}}%
\pgfusepath{clip}%
\pgfsetbuttcap%
\pgfsetroundjoin%
\definecolor{currentfill}{rgb}{0.121569,0.466667,0.705882}%
\pgfsetfillcolor{currentfill}%
\pgfsetfillopacity{0.574361}%
\pgfsetlinewidth{1.003750pt}%
\definecolor{currentstroke}{rgb}{0.121569,0.466667,0.705882}%
\pgfsetstrokecolor{currentstroke}%
\pgfsetstrokeopacity{0.574361}%
\pgfsetdash{}{0pt}%
\pgfpathmoveto{\pgfqpoint{3.234826in}{2.227176in}}%
\pgfpathcurveto{\pgfqpoint{3.243062in}{2.227176in}}{\pgfqpoint{3.250962in}{2.230448in}}{\pgfqpoint{3.256786in}{2.236272in}}%
\pgfpathcurveto{\pgfqpoint{3.262610in}{2.242096in}}{\pgfqpoint{3.265882in}{2.249996in}}{\pgfqpoint{3.265882in}{2.258232in}}%
\pgfpathcurveto{\pgfqpoint{3.265882in}{2.266469in}}{\pgfqpoint{3.262610in}{2.274369in}}{\pgfqpoint{3.256786in}{2.280193in}}%
\pgfpathcurveto{\pgfqpoint{3.250962in}{2.286017in}}{\pgfqpoint{3.243062in}{2.289289in}}{\pgfqpoint{3.234826in}{2.289289in}}%
\pgfpathcurveto{\pgfqpoint{3.226590in}{2.289289in}}{\pgfqpoint{3.218690in}{2.286017in}}{\pgfqpoint{3.212866in}{2.280193in}}%
\pgfpathcurveto{\pgfqpoint{3.207042in}{2.274369in}}{\pgfqpoint{3.203769in}{2.266469in}}{\pgfqpoint{3.203769in}{2.258232in}}%
\pgfpathcurveto{\pgfqpoint{3.203769in}{2.249996in}}{\pgfqpoint{3.207042in}{2.242096in}}{\pgfqpoint{3.212866in}{2.236272in}}%
\pgfpathcurveto{\pgfqpoint{3.218690in}{2.230448in}}{\pgfqpoint{3.226590in}{2.227176in}}{\pgfqpoint{3.234826in}{2.227176in}}%
\pgfpathclose%
\pgfusepath{stroke,fill}%
\end{pgfscope}%
\begin{pgfscope}%
\pgfpathrectangle{\pgfqpoint{0.100000in}{0.212622in}}{\pgfqpoint{3.696000in}{3.696000in}}%
\pgfusepath{clip}%
\pgfsetbuttcap%
\pgfsetroundjoin%
\definecolor{currentfill}{rgb}{0.121569,0.466667,0.705882}%
\pgfsetfillcolor{currentfill}%
\pgfsetfillopacity{0.574458}%
\pgfsetlinewidth{1.003750pt}%
\definecolor{currentstroke}{rgb}{0.121569,0.466667,0.705882}%
\pgfsetstrokecolor{currentstroke}%
\pgfsetstrokeopacity{0.574458}%
\pgfsetdash{}{0pt}%
\pgfpathmoveto{\pgfqpoint{3.234658in}{2.227139in}}%
\pgfpathcurveto{\pgfqpoint{3.242894in}{2.227139in}}{\pgfqpoint{3.250794in}{2.230411in}}{\pgfqpoint{3.256618in}{2.236235in}}%
\pgfpathcurveto{\pgfqpoint{3.262442in}{2.242059in}}{\pgfqpoint{3.265714in}{2.249959in}}{\pgfqpoint{3.265714in}{2.258195in}}%
\pgfpathcurveto{\pgfqpoint{3.265714in}{2.266432in}}{\pgfqpoint{3.262442in}{2.274332in}}{\pgfqpoint{3.256618in}{2.280156in}}%
\pgfpathcurveto{\pgfqpoint{3.250794in}{2.285979in}}{\pgfqpoint{3.242894in}{2.289252in}}{\pgfqpoint{3.234658in}{2.289252in}}%
\pgfpathcurveto{\pgfqpoint{3.226422in}{2.289252in}}{\pgfqpoint{3.218522in}{2.285979in}}{\pgfqpoint{3.212698in}{2.280156in}}%
\pgfpathcurveto{\pgfqpoint{3.206874in}{2.274332in}}{\pgfqpoint{3.203601in}{2.266432in}}{\pgfqpoint{3.203601in}{2.258195in}}%
\pgfpathcurveto{\pgfqpoint{3.203601in}{2.249959in}}{\pgfqpoint{3.206874in}{2.242059in}}{\pgfqpoint{3.212698in}{2.236235in}}%
\pgfpathcurveto{\pgfqpoint{3.218522in}{2.230411in}}{\pgfqpoint{3.226422in}{2.227139in}}{\pgfqpoint{3.234658in}{2.227139in}}%
\pgfpathclose%
\pgfusepath{stroke,fill}%
\end{pgfscope}%
\begin{pgfscope}%
\pgfpathrectangle{\pgfqpoint{0.100000in}{0.212622in}}{\pgfqpoint{3.696000in}{3.696000in}}%
\pgfusepath{clip}%
\pgfsetbuttcap%
\pgfsetroundjoin%
\definecolor{currentfill}{rgb}{0.121569,0.466667,0.705882}%
\pgfsetfillcolor{currentfill}%
\pgfsetfillopacity{0.574917}%
\pgfsetlinewidth{1.003750pt}%
\definecolor{currentstroke}{rgb}{0.121569,0.466667,0.705882}%
\pgfsetstrokecolor{currentstroke}%
\pgfsetstrokeopacity{0.574917}%
\pgfsetdash{}{0pt}%
\pgfpathmoveto{\pgfqpoint{0.862544in}{1.518849in}}%
\pgfpathcurveto{\pgfqpoint{0.870780in}{1.518849in}}{\pgfqpoint{0.878680in}{1.522122in}}{\pgfqpoint{0.884504in}{1.527946in}}%
\pgfpathcurveto{\pgfqpoint{0.890328in}{1.533770in}}{\pgfqpoint{0.893601in}{1.541670in}}{\pgfqpoint{0.893601in}{1.549906in}}%
\pgfpathcurveto{\pgfqpoint{0.893601in}{1.558142in}}{\pgfqpoint{0.890328in}{1.566042in}}{\pgfqpoint{0.884504in}{1.571866in}}%
\pgfpathcurveto{\pgfqpoint{0.878680in}{1.577690in}}{\pgfqpoint{0.870780in}{1.580962in}}{\pgfqpoint{0.862544in}{1.580962in}}%
\pgfpathcurveto{\pgfqpoint{0.854308in}{1.580962in}}{\pgfqpoint{0.846408in}{1.577690in}}{\pgfqpoint{0.840584in}{1.571866in}}%
\pgfpathcurveto{\pgfqpoint{0.834760in}{1.566042in}}{\pgfqpoint{0.831488in}{1.558142in}}{\pgfqpoint{0.831488in}{1.549906in}}%
\pgfpathcurveto{\pgfqpoint{0.831488in}{1.541670in}}{\pgfqpoint{0.834760in}{1.533770in}}{\pgfqpoint{0.840584in}{1.527946in}}%
\pgfpathcurveto{\pgfqpoint{0.846408in}{1.522122in}}{\pgfqpoint{0.854308in}{1.518849in}}{\pgfqpoint{0.862544in}{1.518849in}}%
\pgfpathclose%
\pgfusepath{stroke,fill}%
\end{pgfscope}%
\begin{pgfscope}%
\pgfpathrectangle{\pgfqpoint{0.100000in}{0.212622in}}{\pgfqpoint{3.696000in}{3.696000in}}%
\pgfusepath{clip}%
\pgfsetbuttcap%
\pgfsetroundjoin%
\definecolor{currentfill}{rgb}{0.121569,0.466667,0.705882}%
\pgfsetfillcolor{currentfill}%
\pgfsetfillopacity{0.575082}%
\pgfsetlinewidth{1.003750pt}%
\definecolor{currentstroke}{rgb}{0.121569,0.466667,0.705882}%
\pgfsetstrokecolor{currentstroke}%
\pgfsetstrokeopacity{0.575082}%
\pgfsetdash{}{0pt}%
\pgfpathmoveto{\pgfqpoint{3.233425in}{2.226832in}}%
\pgfpathcurveto{\pgfqpoint{3.241662in}{2.226832in}}{\pgfqpoint{3.249562in}{2.230104in}}{\pgfqpoint{3.255386in}{2.235928in}}%
\pgfpathcurveto{\pgfqpoint{3.261210in}{2.241752in}}{\pgfqpoint{3.264482in}{2.249652in}}{\pgfqpoint{3.264482in}{2.257889in}}%
\pgfpathcurveto{\pgfqpoint{3.264482in}{2.266125in}}{\pgfqpoint{3.261210in}{2.274025in}}{\pgfqpoint{3.255386in}{2.279849in}}%
\pgfpathcurveto{\pgfqpoint{3.249562in}{2.285673in}}{\pgfqpoint{3.241662in}{2.288945in}}{\pgfqpoint{3.233425in}{2.288945in}}%
\pgfpathcurveto{\pgfqpoint{3.225189in}{2.288945in}}{\pgfqpoint{3.217289in}{2.285673in}}{\pgfqpoint{3.211465in}{2.279849in}}%
\pgfpathcurveto{\pgfqpoint{3.205641in}{2.274025in}}{\pgfqpoint{3.202369in}{2.266125in}}{\pgfqpoint{3.202369in}{2.257889in}}%
\pgfpathcurveto{\pgfqpoint{3.202369in}{2.249652in}}{\pgfqpoint{3.205641in}{2.241752in}}{\pgfqpoint{3.211465in}{2.235928in}}%
\pgfpathcurveto{\pgfqpoint{3.217289in}{2.230104in}}{\pgfqpoint{3.225189in}{2.226832in}}{\pgfqpoint{3.233425in}{2.226832in}}%
\pgfpathclose%
\pgfusepath{stroke,fill}%
\end{pgfscope}%
\begin{pgfscope}%
\pgfpathrectangle{\pgfqpoint{0.100000in}{0.212622in}}{\pgfqpoint{3.696000in}{3.696000in}}%
\pgfusepath{clip}%
\pgfsetbuttcap%
\pgfsetroundjoin%
\definecolor{currentfill}{rgb}{0.121569,0.466667,0.705882}%
\pgfsetfillcolor{currentfill}%
\pgfsetfillopacity{0.575410}%
\pgfsetlinewidth{1.003750pt}%
\definecolor{currentstroke}{rgb}{0.121569,0.466667,0.705882}%
\pgfsetstrokecolor{currentstroke}%
\pgfsetstrokeopacity{0.575410}%
\pgfsetdash{}{0pt}%
\pgfpathmoveto{\pgfqpoint{3.232741in}{2.226576in}}%
\pgfpathcurveto{\pgfqpoint{3.240977in}{2.226576in}}{\pgfqpoint{3.248877in}{2.229848in}}{\pgfqpoint{3.254701in}{2.235672in}}%
\pgfpathcurveto{\pgfqpoint{3.260525in}{2.241496in}}{\pgfqpoint{3.263797in}{2.249396in}}{\pgfqpoint{3.263797in}{2.257632in}}%
\pgfpathcurveto{\pgfqpoint{3.263797in}{2.265868in}}{\pgfqpoint{3.260525in}{2.273768in}}{\pgfqpoint{3.254701in}{2.279592in}}%
\pgfpathcurveto{\pgfqpoint{3.248877in}{2.285416in}}{\pgfqpoint{3.240977in}{2.288689in}}{\pgfqpoint{3.232741in}{2.288689in}}%
\pgfpathcurveto{\pgfqpoint{3.224505in}{2.288689in}}{\pgfqpoint{3.216605in}{2.285416in}}{\pgfqpoint{3.210781in}{2.279592in}}%
\pgfpathcurveto{\pgfqpoint{3.204957in}{2.273768in}}{\pgfqpoint{3.201684in}{2.265868in}}{\pgfqpoint{3.201684in}{2.257632in}}%
\pgfpathcurveto{\pgfqpoint{3.201684in}{2.249396in}}{\pgfqpoint{3.204957in}{2.241496in}}{\pgfqpoint{3.210781in}{2.235672in}}%
\pgfpathcurveto{\pgfqpoint{3.216605in}{2.229848in}}{\pgfqpoint{3.224505in}{2.226576in}}{\pgfqpoint{3.232741in}{2.226576in}}%
\pgfpathclose%
\pgfusepath{stroke,fill}%
\end{pgfscope}%
\begin{pgfscope}%
\pgfpathrectangle{\pgfqpoint{0.100000in}{0.212622in}}{\pgfqpoint{3.696000in}{3.696000in}}%
\pgfusepath{clip}%
\pgfsetbuttcap%
\pgfsetroundjoin%
\definecolor{currentfill}{rgb}{0.121569,0.466667,0.705882}%
\pgfsetfillcolor{currentfill}%
\pgfsetfillopacity{0.575885}%
\pgfsetlinewidth{1.003750pt}%
\definecolor{currentstroke}{rgb}{0.121569,0.466667,0.705882}%
\pgfsetstrokecolor{currentstroke}%
\pgfsetstrokeopacity{0.575885}%
\pgfsetdash{}{0pt}%
\pgfpathmoveto{\pgfqpoint{3.231821in}{2.226324in}}%
\pgfpathcurveto{\pgfqpoint{3.240057in}{2.226324in}}{\pgfqpoint{3.247957in}{2.229596in}}{\pgfqpoint{3.253781in}{2.235420in}}%
\pgfpathcurveto{\pgfqpoint{3.259605in}{2.241244in}}{\pgfqpoint{3.262878in}{2.249144in}}{\pgfqpoint{3.262878in}{2.257380in}}%
\pgfpathcurveto{\pgfqpoint{3.262878in}{2.265616in}}{\pgfqpoint{3.259605in}{2.273516in}}{\pgfqpoint{3.253781in}{2.279340in}}%
\pgfpathcurveto{\pgfqpoint{3.247957in}{2.285164in}}{\pgfqpoint{3.240057in}{2.288437in}}{\pgfqpoint{3.231821in}{2.288437in}}%
\pgfpathcurveto{\pgfqpoint{3.223585in}{2.288437in}}{\pgfqpoint{3.215685in}{2.285164in}}{\pgfqpoint{3.209861in}{2.279340in}}%
\pgfpathcurveto{\pgfqpoint{3.204037in}{2.273516in}}{\pgfqpoint{3.200765in}{2.265616in}}{\pgfqpoint{3.200765in}{2.257380in}}%
\pgfpathcurveto{\pgfqpoint{3.200765in}{2.249144in}}{\pgfqpoint{3.204037in}{2.241244in}}{\pgfqpoint{3.209861in}{2.235420in}}%
\pgfpathcurveto{\pgfqpoint{3.215685in}{2.229596in}}{\pgfqpoint{3.223585in}{2.226324in}}{\pgfqpoint{3.231821in}{2.226324in}}%
\pgfpathclose%
\pgfusepath{stroke,fill}%
\end{pgfscope}%
\begin{pgfscope}%
\pgfpathrectangle{\pgfqpoint{0.100000in}{0.212622in}}{\pgfqpoint{3.696000in}{3.696000in}}%
\pgfusepath{clip}%
\pgfsetbuttcap%
\pgfsetroundjoin%
\definecolor{currentfill}{rgb}{0.121569,0.466667,0.705882}%
\pgfsetfillcolor{currentfill}%
\pgfsetfillopacity{0.575914}%
\pgfsetlinewidth{1.003750pt}%
\definecolor{currentstroke}{rgb}{0.121569,0.466667,0.705882}%
\pgfsetstrokecolor{currentstroke}%
\pgfsetstrokeopacity{0.575914}%
\pgfsetdash{}{0pt}%
\pgfpathmoveto{\pgfqpoint{0.860725in}{1.514348in}}%
\pgfpathcurveto{\pgfqpoint{0.868961in}{1.514348in}}{\pgfqpoint{0.876861in}{1.517620in}}{\pgfqpoint{0.882685in}{1.523444in}}%
\pgfpathcurveto{\pgfqpoint{0.888509in}{1.529268in}}{\pgfqpoint{0.891782in}{1.537168in}}{\pgfqpoint{0.891782in}{1.545404in}}%
\pgfpathcurveto{\pgfqpoint{0.891782in}{1.553641in}}{\pgfqpoint{0.888509in}{1.561541in}}{\pgfqpoint{0.882685in}{1.567365in}}%
\pgfpathcurveto{\pgfqpoint{0.876861in}{1.573189in}}{\pgfqpoint{0.868961in}{1.576461in}}{\pgfqpoint{0.860725in}{1.576461in}}%
\pgfpathcurveto{\pgfqpoint{0.852489in}{1.576461in}}{\pgfqpoint{0.844589in}{1.573189in}}{\pgfqpoint{0.838765in}{1.567365in}}%
\pgfpathcurveto{\pgfqpoint{0.832941in}{1.561541in}}{\pgfqpoint{0.829669in}{1.553641in}}{\pgfqpoint{0.829669in}{1.545404in}}%
\pgfpathcurveto{\pgfqpoint{0.829669in}{1.537168in}}{\pgfqpoint{0.832941in}{1.529268in}}{\pgfqpoint{0.838765in}{1.523444in}}%
\pgfpathcurveto{\pgfqpoint{0.844589in}{1.517620in}}{\pgfqpoint{0.852489in}{1.514348in}}{\pgfqpoint{0.860725in}{1.514348in}}%
\pgfpathclose%
\pgfusepath{stroke,fill}%
\end{pgfscope}%
\begin{pgfscope}%
\pgfpathrectangle{\pgfqpoint{0.100000in}{0.212622in}}{\pgfqpoint{3.696000in}{3.696000in}}%
\pgfusepath{clip}%
\pgfsetbuttcap%
\pgfsetroundjoin%
\definecolor{currentfill}{rgb}{0.121569,0.466667,0.705882}%
\pgfsetfillcolor{currentfill}%
\pgfsetfillopacity{0.576135}%
\pgfsetlinewidth{1.003750pt}%
\definecolor{currentstroke}{rgb}{0.121569,0.466667,0.705882}%
\pgfsetstrokecolor{currentstroke}%
\pgfsetstrokeopacity{0.576135}%
\pgfsetdash{}{0pt}%
\pgfpathmoveto{\pgfqpoint{3.231291in}{2.226140in}}%
\pgfpathcurveto{\pgfqpoint{3.239527in}{2.226140in}}{\pgfqpoint{3.247427in}{2.229412in}}{\pgfqpoint{3.253251in}{2.235236in}}%
\pgfpathcurveto{\pgfqpoint{3.259075in}{2.241060in}}{\pgfqpoint{3.262347in}{2.248960in}}{\pgfqpoint{3.262347in}{2.257196in}}%
\pgfpathcurveto{\pgfqpoint{3.262347in}{2.265432in}}{\pgfqpoint{3.259075in}{2.273333in}}{\pgfqpoint{3.253251in}{2.279156in}}%
\pgfpathcurveto{\pgfqpoint{3.247427in}{2.284980in}}{\pgfqpoint{3.239527in}{2.288253in}}{\pgfqpoint{3.231291in}{2.288253in}}%
\pgfpathcurveto{\pgfqpoint{3.223054in}{2.288253in}}{\pgfqpoint{3.215154in}{2.284980in}}{\pgfqpoint{3.209330in}{2.279156in}}%
\pgfpathcurveto{\pgfqpoint{3.203506in}{2.273333in}}{\pgfqpoint{3.200234in}{2.265432in}}{\pgfqpoint{3.200234in}{2.257196in}}%
\pgfpathcurveto{\pgfqpoint{3.200234in}{2.248960in}}{\pgfqpoint{3.203506in}{2.241060in}}{\pgfqpoint{3.209330in}{2.235236in}}%
\pgfpathcurveto{\pgfqpoint{3.215154in}{2.229412in}}{\pgfqpoint{3.223054in}{2.226140in}}{\pgfqpoint{3.231291in}{2.226140in}}%
\pgfpathclose%
\pgfusepath{stroke,fill}%
\end{pgfscope}%
\begin{pgfscope}%
\pgfpathrectangle{\pgfqpoint{0.100000in}{0.212622in}}{\pgfqpoint{3.696000in}{3.696000in}}%
\pgfusepath{clip}%
\pgfsetbuttcap%
\pgfsetroundjoin%
\definecolor{currentfill}{rgb}{0.121569,0.466667,0.705882}%
\pgfsetfillcolor{currentfill}%
\pgfsetfillopacity{0.576806}%
\pgfsetlinewidth{1.003750pt}%
\definecolor{currentstroke}{rgb}{0.121569,0.466667,0.705882}%
\pgfsetstrokecolor{currentstroke}%
\pgfsetstrokeopacity{0.576806}%
\pgfsetdash{}{0pt}%
\pgfpathmoveto{\pgfqpoint{3.230033in}{2.225739in}}%
\pgfpathcurveto{\pgfqpoint{3.238270in}{2.225739in}}{\pgfqpoint{3.246170in}{2.229011in}}{\pgfqpoint{3.251994in}{2.234835in}}%
\pgfpathcurveto{\pgfqpoint{3.257818in}{2.240659in}}{\pgfqpoint{3.261090in}{2.248559in}}{\pgfqpoint{3.261090in}{2.256795in}}%
\pgfpathcurveto{\pgfqpoint{3.261090in}{2.265031in}}{\pgfqpoint{3.257818in}{2.272931in}}{\pgfqpoint{3.251994in}{2.278755in}}%
\pgfpathcurveto{\pgfqpoint{3.246170in}{2.284579in}}{\pgfqpoint{3.238270in}{2.287852in}}{\pgfqpoint{3.230033in}{2.287852in}}%
\pgfpathcurveto{\pgfqpoint{3.221797in}{2.287852in}}{\pgfqpoint{3.213897in}{2.284579in}}{\pgfqpoint{3.208073in}{2.278755in}}%
\pgfpathcurveto{\pgfqpoint{3.202249in}{2.272931in}}{\pgfqpoint{3.198977in}{2.265031in}}{\pgfqpoint{3.198977in}{2.256795in}}%
\pgfpathcurveto{\pgfqpoint{3.198977in}{2.248559in}}{\pgfqpoint{3.202249in}{2.240659in}}{\pgfqpoint{3.208073in}{2.234835in}}%
\pgfpathcurveto{\pgfqpoint{3.213897in}{2.229011in}}{\pgfqpoint{3.221797in}{2.225739in}}{\pgfqpoint{3.230033in}{2.225739in}}%
\pgfpathclose%
\pgfusepath{stroke,fill}%
\end{pgfscope}%
\begin{pgfscope}%
\pgfpathrectangle{\pgfqpoint{0.100000in}{0.212622in}}{\pgfqpoint{3.696000in}{3.696000in}}%
\pgfusepath{clip}%
\pgfsetbuttcap%
\pgfsetroundjoin%
\definecolor{currentfill}{rgb}{0.121569,0.466667,0.705882}%
\pgfsetfillcolor{currentfill}%
\pgfsetfillopacity{0.577197}%
\pgfsetlinewidth{1.003750pt}%
\definecolor{currentstroke}{rgb}{0.121569,0.466667,0.705882}%
\pgfsetstrokecolor{currentstroke}%
\pgfsetstrokeopacity{0.577197}%
\pgfsetdash{}{0pt}%
\pgfpathmoveto{\pgfqpoint{3.229364in}{2.225648in}}%
\pgfpathcurveto{\pgfqpoint{3.237600in}{2.225648in}}{\pgfqpoint{3.245500in}{2.228920in}}{\pgfqpoint{3.251324in}{2.234744in}}%
\pgfpathcurveto{\pgfqpoint{3.257148in}{2.240568in}}{\pgfqpoint{3.260421in}{2.248468in}}{\pgfqpoint{3.260421in}{2.256704in}}%
\pgfpathcurveto{\pgfqpoint{3.260421in}{2.264941in}}{\pgfqpoint{3.257148in}{2.272841in}}{\pgfqpoint{3.251324in}{2.278665in}}%
\pgfpathcurveto{\pgfqpoint{3.245500in}{2.284488in}}{\pgfqpoint{3.237600in}{2.287761in}}{\pgfqpoint{3.229364in}{2.287761in}}%
\pgfpathcurveto{\pgfqpoint{3.221128in}{2.287761in}}{\pgfqpoint{3.213228in}{2.284488in}}{\pgfqpoint{3.207404in}{2.278665in}}%
\pgfpathcurveto{\pgfqpoint{3.201580in}{2.272841in}}{\pgfqpoint{3.198308in}{2.264941in}}{\pgfqpoint{3.198308in}{2.256704in}}%
\pgfpathcurveto{\pgfqpoint{3.198308in}{2.248468in}}{\pgfqpoint{3.201580in}{2.240568in}}{\pgfqpoint{3.207404in}{2.234744in}}%
\pgfpathcurveto{\pgfqpoint{3.213228in}{2.228920in}}{\pgfqpoint{3.221128in}{2.225648in}}{\pgfqpoint{3.229364in}{2.225648in}}%
\pgfpathclose%
\pgfusepath{stroke,fill}%
\end{pgfscope}%
\begin{pgfscope}%
\pgfpathrectangle{\pgfqpoint{0.100000in}{0.212622in}}{\pgfqpoint{3.696000in}{3.696000in}}%
\pgfusepath{clip}%
\pgfsetbuttcap%
\pgfsetroundjoin%
\definecolor{currentfill}{rgb}{0.121569,0.466667,0.705882}%
\pgfsetfillcolor{currentfill}%
\pgfsetfillopacity{0.577398}%
\pgfsetlinewidth{1.003750pt}%
\definecolor{currentstroke}{rgb}{0.121569,0.466667,0.705882}%
\pgfsetstrokecolor{currentstroke}%
\pgfsetstrokeopacity{0.577398}%
\pgfsetdash{}{0pt}%
\pgfpathmoveto{\pgfqpoint{3.228967in}{2.225529in}}%
\pgfpathcurveto{\pgfqpoint{3.237203in}{2.225529in}}{\pgfqpoint{3.245103in}{2.228802in}}{\pgfqpoint{3.250927in}{2.234626in}}%
\pgfpathcurveto{\pgfqpoint{3.256751in}{2.240449in}}{\pgfqpoint{3.260023in}{2.248350in}}{\pgfqpoint{3.260023in}{2.256586in}}%
\pgfpathcurveto{\pgfqpoint{3.260023in}{2.264822in}}{\pgfqpoint{3.256751in}{2.272722in}}{\pgfqpoint{3.250927in}{2.278546in}}%
\pgfpathcurveto{\pgfqpoint{3.245103in}{2.284370in}}{\pgfqpoint{3.237203in}{2.287642in}}{\pgfqpoint{3.228967in}{2.287642in}}%
\pgfpathcurveto{\pgfqpoint{3.220730in}{2.287642in}}{\pgfqpoint{3.212830in}{2.284370in}}{\pgfqpoint{3.207006in}{2.278546in}}%
\pgfpathcurveto{\pgfqpoint{3.201183in}{2.272722in}}{\pgfqpoint{3.197910in}{2.264822in}}{\pgfqpoint{3.197910in}{2.256586in}}%
\pgfpathcurveto{\pgfqpoint{3.197910in}{2.248350in}}{\pgfqpoint{3.201183in}{2.240449in}}{\pgfqpoint{3.207006in}{2.234626in}}%
\pgfpathcurveto{\pgfqpoint{3.212830in}{2.228802in}}{\pgfqpoint{3.220730in}{2.225529in}}{\pgfqpoint{3.228967in}{2.225529in}}%
\pgfpathclose%
\pgfusepath{stroke,fill}%
\end{pgfscope}%
\begin{pgfscope}%
\pgfpathrectangle{\pgfqpoint{0.100000in}{0.212622in}}{\pgfqpoint{3.696000in}{3.696000in}}%
\pgfusepath{clip}%
\pgfsetbuttcap%
\pgfsetroundjoin%
\definecolor{currentfill}{rgb}{0.121569,0.466667,0.705882}%
\pgfsetfillcolor{currentfill}%
\pgfsetfillopacity{0.577758}%
\pgfsetlinewidth{1.003750pt}%
\definecolor{currentstroke}{rgb}{0.121569,0.466667,0.705882}%
\pgfsetstrokecolor{currentstroke}%
\pgfsetstrokeopacity{0.577758}%
\pgfsetdash{}{0pt}%
\pgfpathmoveto{\pgfqpoint{0.857884in}{1.506093in}}%
\pgfpathcurveto{\pgfqpoint{0.866120in}{1.506093in}}{\pgfqpoint{0.874020in}{1.509365in}}{\pgfqpoint{0.879844in}{1.515189in}}%
\pgfpathcurveto{\pgfqpoint{0.885668in}{1.521013in}}{\pgfqpoint{0.888940in}{1.528913in}}{\pgfqpoint{0.888940in}{1.537150in}}%
\pgfpathcurveto{\pgfqpoint{0.888940in}{1.545386in}}{\pgfqpoint{0.885668in}{1.553286in}}{\pgfqpoint{0.879844in}{1.559110in}}%
\pgfpathcurveto{\pgfqpoint{0.874020in}{1.564934in}}{\pgfqpoint{0.866120in}{1.568206in}}{\pgfqpoint{0.857884in}{1.568206in}}%
\pgfpathcurveto{\pgfqpoint{0.849647in}{1.568206in}}{\pgfqpoint{0.841747in}{1.564934in}}{\pgfqpoint{0.835923in}{1.559110in}}%
\pgfpathcurveto{\pgfqpoint{0.830100in}{1.553286in}}{\pgfqpoint{0.826827in}{1.545386in}}{\pgfqpoint{0.826827in}{1.537150in}}%
\pgfpathcurveto{\pgfqpoint{0.826827in}{1.528913in}}{\pgfqpoint{0.830100in}{1.521013in}}{\pgfqpoint{0.835923in}{1.515189in}}%
\pgfpathcurveto{\pgfqpoint{0.841747in}{1.509365in}}{\pgfqpoint{0.849647in}{1.506093in}}{\pgfqpoint{0.857884in}{1.506093in}}%
\pgfpathclose%
\pgfusepath{stroke,fill}%
\end{pgfscope}%
\begin{pgfscope}%
\pgfpathrectangle{\pgfqpoint{0.100000in}{0.212622in}}{\pgfqpoint{3.696000in}{3.696000in}}%
\pgfusepath{clip}%
\pgfsetbuttcap%
\pgfsetroundjoin%
\definecolor{currentfill}{rgb}{0.121569,0.466667,0.705882}%
\pgfsetfillcolor{currentfill}%
\pgfsetfillopacity{0.577835}%
\pgfsetlinewidth{1.003750pt}%
\definecolor{currentstroke}{rgb}{0.121569,0.466667,0.705882}%
\pgfsetstrokecolor{currentstroke}%
\pgfsetstrokeopacity{0.577835}%
\pgfsetdash{}{0pt}%
\pgfpathmoveto{\pgfqpoint{3.228156in}{2.225242in}}%
\pgfpathcurveto{\pgfqpoint{3.236393in}{2.225242in}}{\pgfqpoint{3.244293in}{2.228514in}}{\pgfqpoint{3.250117in}{2.234338in}}%
\pgfpathcurveto{\pgfqpoint{3.255941in}{2.240162in}}{\pgfqpoint{3.259213in}{2.248062in}}{\pgfqpoint{3.259213in}{2.256298in}}%
\pgfpathcurveto{\pgfqpoint{3.259213in}{2.264535in}}{\pgfqpoint{3.255941in}{2.272435in}}{\pgfqpoint{3.250117in}{2.278259in}}%
\pgfpathcurveto{\pgfqpoint{3.244293in}{2.284082in}}{\pgfqpoint{3.236393in}{2.287355in}}{\pgfqpoint{3.228156in}{2.287355in}}%
\pgfpathcurveto{\pgfqpoint{3.219920in}{2.287355in}}{\pgfqpoint{3.212020in}{2.284082in}}{\pgfqpoint{3.206196in}{2.278259in}}%
\pgfpathcurveto{\pgfqpoint{3.200372in}{2.272435in}}{\pgfqpoint{3.197100in}{2.264535in}}{\pgfqpoint{3.197100in}{2.256298in}}%
\pgfpathcurveto{\pgfqpoint{3.197100in}{2.248062in}}{\pgfqpoint{3.200372in}{2.240162in}}{\pgfqpoint{3.206196in}{2.234338in}}%
\pgfpathcurveto{\pgfqpoint{3.212020in}{2.228514in}}{\pgfqpoint{3.219920in}{2.225242in}}{\pgfqpoint{3.228156in}{2.225242in}}%
\pgfpathclose%
\pgfusepath{stroke,fill}%
\end{pgfscope}%
\begin{pgfscope}%
\pgfpathrectangle{\pgfqpoint{0.100000in}{0.212622in}}{\pgfqpoint{3.696000in}{3.696000in}}%
\pgfusepath{clip}%
\pgfsetbuttcap%
\pgfsetroundjoin%
\definecolor{currentfill}{rgb}{0.121569,0.466667,0.705882}%
\pgfsetfillcolor{currentfill}%
\pgfsetfillopacity{0.578561}%
\pgfsetlinewidth{1.003750pt}%
\definecolor{currentstroke}{rgb}{0.121569,0.466667,0.705882}%
\pgfsetstrokecolor{currentstroke}%
\pgfsetstrokeopacity{0.578561}%
\pgfsetdash{}{0pt}%
\pgfpathmoveto{\pgfqpoint{3.226671in}{2.224966in}}%
\pgfpathcurveto{\pgfqpoint{3.234908in}{2.224966in}}{\pgfqpoint{3.242808in}{2.228239in}}{\pgfqpoint{3.248632in}{2.234062in}}%
\pgfpathcurveto{\pgfqpoint{3.254456in}{2.239886in}}{\pgfqpoint{3.257728in}{2.247786in}}{\pgfqpoint{3.257728in}{2.256023in}}%
\pgfpathcurveto{\pgfqpoint{3.257728in}{2.264259in}}{\pgfqpoint{3.254456in}{2.272159in}}{\pgfqpoint{3.248632in}{2.277983in}}%
\pgfpathcurveto{\pgfqpoint{3.242808in}{2.283807in}}{\pgfqpoint{3.234908in}{2.287079in}}{\pgfqpoint{3.226671in}{2.287079in}}%
\pgfpathcurveto{\pgfqpoint{3.218435in}{2.287079in}}{\pgfqpoint{3.210535in}{2.283807in}}{\pgfqpoint{3.204711in}{2.277983in}}%
\pgfpathcurveto{\pgfqpoint{3.198887in}{2.272159in}}{\pgfqpoint{3.195615in}{2.264259in}}{\pgfqpoint{3.195615in}{2.256023in}}%
\pgfpathcurveto{\pgfqpoint{3.195615in}{2.247786in}}{\pgfqpoint{3.198887in}{2.239886in}}{\pgfqpoint{3.204711in}{2.234062in}}%
\pgfpathcurveto{\pgfqpoint{3.210535in}{2.228239in}}{\pgfqpoint{3.218435in}{2.224966in}}{\pgfqpoint{3.226671in}{2.224966in}}%
\pgfpathclose%
\pgfusepath{stroke,fill}%
\end{pgfscope}%
\begin{pgfscope}%
\pgfpathrectangle{\pgfqpoint{0.100000in}{0.212622in}}{\pgfqpoint{3.696000in}{3.696000in}}%
\pgfusepath{clip}%
\pgfsetbuttcap%
\pgfsetroundjoin%
\definecolor{currentfill}{rgb}{0.121569,0.466667,0.705882}%
\pgfsetfillcolor{currentfill}%
\pgfsetfillopacity{0.578943}%
\pgfsetlinewidth{1.003750pt}%
\definecolor{currentstroke}{rgb}{0.121569,0.466667,0.705882}%
\pgfsetstrokecolor{currentstroke}%
\pgfsetstrokeopacity{0.578943}%
\pgfsetdash{}{0pt}%
\pgfpathmoveto{\pgfqpoint{3.225846in}{2.224699in}}%
\pgfpathcurveto{\pgfqpoint{3.234082in}{2.224699in}}{\pgfqpoint{3.241982in}{2.227972in}}{\pgfqpoint{3.247806in}{2.233795in}}%
\pgfpathcurveto{\pgfqpoint{3.253630in}{2.239619in}}{\pgfqpoint{3.256902in}{2.247519in}}{\pgfqpoint{3.256902in}{2.255756in}}%
\pgfpathcurveto{\pgfqpoint{3.256902in}{2.263992in}}{\pgfqpoint{3.253630in}{2.271892in}}{\pgfqpoint{3.247806in}{2.277716in}}%
\pgfpathcurveto{\pgfqpoint{3.241982in}{2.283540in}}{\pgfqpoint{3.234082in}{2.286812in}}{\pgfqpoint{3.225846in}{2.286812in}}%
\pgfpathcurveto{\pgfqpoint{3.217609in}{2.286812in}}{\pgfqpoint{3.209709in}{2.283540in}}{\pgfqpoint{3.203885in}{2.277716in}}%
\pgfpathcurveto{\pgfqpoint{3.198061in}{2.271892in}}{\pgfqpoint{3.194789in}{2.263992in}}{\pgfqpoint{3.194789in}{2.255756in}}%
\pgfpathcurveto{\pgfqpoint{3.194789in}{2.247519in}}{\pgfqpoint{3.198061in}{2.239619in}}{\pgfqpoint{3.203885in}{2.233795in}}%
\pgfpathcurveto{\pgfqpoint{3.209709in}{2.227972in}}{\pgfqpoint{3.217609in}{2.224699in}}{\pgfqpoint{3.225846in}{2.224699in}}%
\pgfpathclose%
\pgfusepath{stroke,fill}%
\end{pgfscope}%
\begin{pgfscope}%
\pgfpathrectangle{\pgfqpoint{0.100000in}{0.212622in}}{\pgfqpoint{3.696000in}{3.696000in}}%
\pgfusepath{clip}%
\pgfsetbuttcap%
\pgfsetroundjoin%
\definecolor{currentfill}{rgb}{0.121569,0.466667,0.705882}%
\pgfsetfillcolor{currentfill}%
\pgfsetfillopacity{0.579159}%
\pgfsetlinewidth{1.003750pt}%
\definecolor{currentstroke}{rgb}{0.121569,0.466667,0.705882}%
\pgfsetstrokecolor{currentstroke}%
\pgfsetstrokeopacity{0.579159}%
\pgfsetdash{}{0pt}%
\pgfpathmoveto{\pgfqpoint{3.225416in}{2.224563in}}%
\pgfpathcurveto{\pgfqpoint{3.233652in}{2.224563in}}{\pgfqpoint{3.241552in}{2.227835in}}{\pgfqpoint{3.247376in}{2.233659in}}%
\pgfpathcurveto{\pgfqpoint{3.253200in}{2.239483in}}{\pgfqpoint{3.256473in}{2.247383in}}{\pgfqpoint{3.256473in}{2.255620in}}%
\pgfpathcurveto{\pgfqpoint{3.256473in}{2.263856in}}{\pgfqpoint{3.253200in}{2.271756in}}{\pgfqpoint{3.247376in}{2.277580in}}%
\pgfpathcurveto{\pgfqpoint{3.241552in}{2.283404in}}{\pgfqpoint{3.233652in}{2.286676in}}{\pgfqpoint{3.225416in}{2.286676in}}%
\pgfpathcurveto{\pgfqpoint{3.217180in}{2.286676in}}{\pgfqpoint{3.209280in}{2.283404in}}{\pgfqpoint{3.203456in}{2.277580in}}%
\pgfpathcurveto{\pgfqpoint{3.197632in}{2.271756in}}{\pgfqpoint{3.194360in}{2.263856in}}{\pgfqpoint{3.194360in}{2.255620in}}%
\pgfpathcurveto{\pgfqpoint{3.194360in}{2.247383in}}{\pgfqpoint{3.197632in}{2.239483in}}{\pgfqpoint{3.203456in}{2.233659in}}%
\pgfpathcurveto{\pgfqpoint{3.209280in}{2.227835in}}{\pgfqpoint{3.217180in}{2.224563in}}{\pgfqpoint{3.225416in}{2.224563in}}%
\pgfpathclose%
\pgfusepath{stroke,fill}%
\end{pgfscope}%
\begin{pgfscope}%
\pgfpathrectangle{\pgfqpoint{0.100000in}{0.212622in}}{\pgfqpoint{3.696000in}{3.696000in}}%
\pgfusepath{clip}%
\pgfsetbuttcap%
\pgfsetroundjoin%
\definecolor{currentfill}{rgb}{0.121569,0.466667,0.705882}%
\pgfsetfillcolor{currentfill}%
\pgfsetfillopacity{0.579276}%
\pgfsetlinewidth{1.003750pt}%
\definecolor{currentstroke}{rgb}{0.121569,0.466667,0.705882}%
\pgfsetstrokecolor{currentstroke}%
\pgfsetstrokeopacity{0.579276}%
\pgfsetdash{}{0pt}%
\pgfpathmoveto{\pgfqpoint{0.855625in}{1.499173in}}%
\pgfpathcurveto{\pgfqpoint{0.863862in}{1.499173in}}{\pgfqpoint{0.871762in}{1.502445in}}{\pgfqpoint{0.877586in}{1.508269in}}%
\pgfpathcurveto{\pgfqpoint{0.883410in}{1.514093in}}{\pgfqpoint{0.886682in}{1.521993in}}{\pgfqpoint{0.886682in}{1.530229in}}%
\pgfpathcurveto{\pgfqpoint{0.886682in}{1.538466in}}{\pgfqpoint{0.883410in}{1.546366in}}{\pgfqpoint{0.877586in}{1.552190in}}%
\pgfpathcurveto{\pgfqpoint{0.871762in}{1.558014in}}{\pgfqpoint{0.863862in}{1.561286in}}{\pgfqpoint{0.855625in}{1.561286in}}%
\pgfpathcurveto{\pgfqpoint{0.847389in}{1.561286in}}{\pgfqpoint{0.839489in}{1.558014in}}{\pgfqpoint{0.833665in}{1.552190in}}%
\pgfpathcurveto{\pgfqpoint{0.827841in}{1.546366in}}{\pgfqpoint{0.824569in}{1.538466in}}{\pgfqpoint{0.824569in}{1.530229in}}%
\pgfpathcurveto{\pgfqpoint{0.824569in}{1.521993in}}{\pgfqpoint{0.827841in}{1.514093in}}{\pgfqpoint{0.833665in}{1.508269in}}%
\pgfpathcurveto{\pgfqpoint{0.839489in}{1.502445in}}{\pgfqpoint{0.847389in}{1.499173in}}{\pgfqpoint{0.855625in}{1.499173in}}%
\pgfpathclose%
\pgfusepath{stroke,fill}%
\end{pgfscope}%
\begin{pgfscope}%
\pgfpathrectangle{\pgfqpoint{0.100000in}{0.212622in}}{\pgfqpoint{3.696000in}{3.696000in}}%
\pgfusepath{clip}%
\pgfsetbuttcap%
\pgfsetroundjoin%
\definecolor{currentfill}{rgb}{0.121569,0.466667,0.705882}%
\pgfsetfillcolor{currentfill}%
\pgfsetfillopacity{0.579600}%
\pgfsetlinewidth{1.003750pt}%
\definecolor{currentstroke}{rgb}{0.121569,0.466667,0.705882}%
\pgfsetstrokecolor{currentstroke}%
\pgfsetstrokeopacity{0.579600}%
\pgfsetdash{}{0pt}%
\pgfpathmoveto{\pgfqpoint{3.224462in}{2.224206in}}%
\pgfpathcurveto{\pgfqpoint{3.232698in}{2.224206in}}{\pgfqpoint{3.240598in}{2.227478in}}{\pgfqpoint{3.246422in}{2.233302in}}%
\pgfpathcurveto{\pgfqpoint{3.252246in}{2.239126in}}{\pgfqpoint{3.255518in}{2.247026in}}{\pgfqpoint{3.255518in}{2.255263in}}%
\pgfpathcurveto{\pgfqpoint{3.255518in}{2.263499in}}{\pgfqpoint{3.252246in}{2.271399in}}{\pgfqpoint{3.246422in}{2.277223in}}%
\pgfpathcurveto{\pgfqpoint{3.240598in}{2.283047in}}{\pgfqpoint{3.232698in}{2.286319in}}{\pgfqpoint{3.224462in}{2.286319in}}%
\pgfpathcurveto{\pgfqpoint{3.216226in}{2.286319in}}{\pgfqpoint{3.208325in}{2.283047in}}{\pgfqpoint{3.202502in}{2.277223in}}%
\pgfpathcurveto{\pgfqpoint{3.196678in}{2.271399in}}{\pgfqpoint{3.193405in}{2.263499in}}{\pgfqpoint{3.193405in}{2.255263in}}%
\pgfpathcurveto{\pgfqpoint{3.193405in}{2.247026in}}{\pgfqpoint{3.196678in}{2.239126in}}{\pgfqpoint{3.202502in}{2.233302in}}%
\pgfpathcurveto{\pgfqpoint{3.208325in}{2.227478in}}{\pgfqpoint{3.216226in}{2.224206in}}{\pgfqpoint{3.224462in}{2.224206in}}%
\pgfpathclose%
\pgfusepath{stroke,fill}%
\end{pgfscope}%
\begin{pgfscope}%
\pgfpathrectangle{\pgfqpoint{0.100000in}{0.212622in}}{\pgfqpoint{3.696000in}{3.696000in}}%
\pgfusepath{clip}%
\pgfsetbuttcap%
\pgfsetroundjoin%
\definecolor{currentfill}{rgb}{0.121569,0.466667,0.705882}%
\pgfsetfillcolor{currentfill}%
\pgfsetfillopacity{0.580408}%
\pgfsetlinewidth{1.003750pt}%
\definecolor{currentstroke}{rgb}{0.121569,0.466667,0.705882}%
\pgfsetstrokecolor{currentstroke}%
\pgfsetstrokeopacity{0.580408}%
\pgfsetdash{}{0pt}%
\pgfpathmoveto{\pgfqpoint{3.222855in}{2.223531in}}%
\pgfpathcurveto{\pgfqpoint{3.231091in}{2.223531in}}{\pgfqpoint{3.238991in}{2.226803in}}{\pgfqpoint{3.244815in}{2.232627in}}%
\pgfpathcurveto{\pgfqpoint{3.250639in}{2.238451in}}{\pgfqpoint{3.253912in}{2.246351in}}{\pgfqpoint{3.253912in}{2.254588in}}%
\pgfpathcurveto{\pgfqpoint{3.253912in}{2.262824in}}{\pgfqpoint{3.250639in}{2.270724in}}{\pgfqpoint{3.244815in}{2.276548in}}%
\pgfpathcurveto{\pgfqpoint{3.238991in}{2.282372in}}{\pgfqpoint{3.231091in}{2.285644in}}{\pgfqpoint{3.222855in}{2.285644in}}%
\pgfpathcurveto{\pgfqpoint{3.214619in}{2.285644in}}{\pgfqpoint{3.206719in}{2.282372in}}{\pgfqpoint{3.200895in}{2.276548in}}%
\pgfpathcurveto{\pgfqpoint{3.195071in}{2.270724in}}{\pgfqpoint{3.191799in}{2.262824in}}{\pgfqpoint{3.191799in}{2.254588in}}%
\pgfpathcurveto{\pgfqpoint{3.191799in}{2.246351in}}{\pgfqpoint{3.195071in}{2.238451in}}{\pgfqpoint{3.200895in}{2.232627in}}%
\pgfpathcurveto{\pgfqpoint{3.206719in}{2.226803in}}{\pgfqpoint{3.214619in}{2.223531in}}{\pgfqpoint{3.222855in}{2.223531in}}%
\pgfpathclose%
\pgfusepath{stroke,fill}%
\end{pgfscope}%
\begin{pgfscope}%
\pgfpathrectangle{\pgfqpoint{0.100000in}{0.212622in}}{\pgfqpoint{3.696000in}{3.696000in}}%
\pgfusepath{clip}%
\pgfsetbuttcap%
\pgfsetroundjoin%
\definecolor{currentfill}{rgb}{0.121569,0.466667,0.705882}%
\pgfsetfillcolor{currentfill}%
\pgfsetfillopacity{0.580666}%
\pgfsetlinewidth{1.003750pt}%
\definecolor{currentstroke}{rgb}{0.121569,0.466667,0.705882}%
\pgfsetstrokecolor{currentstroke}%
\pgfsetstrokeopacity{0.580666}%
\pgfsetdash{}{0pt}%
\pgfpathmoveto{\pgfqpoint{0.854010in}{1.492698in}}%
\pgfpathcurveto{\pgfqpoint{0.862246in}{1.492698in}}{\pgfqpoint{0.870146in}{1.495970in}}{\pgfqpoint{0.875970in}{1.501794in}}%
\pgfpathcurveto{\pgfqpoint{0.881794in}{1.507618in}}{\pgfqpoint{0.885066in}{1.515518in}}{\pgfqpoint{0.885066in}{1.523755in}}%
\pgfpathcurveto{\pgfqpoint{0.885066in}{1.531991in}}{\pgfqpoint{0.881794in}{1.539891in}}{\pgfqpoint{0.875970in}{1.545715in}}%
\pgfpathcurveto{\pgfqpoint{0.870146in}{1.551539in}}{\pgfqpoint{0.862246in}{1.554811in}}{\pgfqpoint{0.854010in}{1.554811in}}%
\pgfpathcurveto{\pgfqpoint{0.845774in}{1.554811in}}{\pgfqpoint{0.837874in}{1.551539in}}{\pgfqpoint{0.832050in}{1.545715in}}%
\pgfpathcurveto{\pgfqpoint{0.826226in}{1.539891in}}{\pgfqpoint{0.822953in}{1.531991in}}{\pgfqpoint{0.822953in}{1.523755in}}%
\pgfpathcurveto{\pgfqpoint{0.822953in}{1.515518in}}{\pgfqpoint{0.826226in}{1.507618in}}{\pgfqpoint{0.832050in}{1.501794in}}%
\pgfpathcurveto{\pgfqpoint{0.837874in}{1.495970in}}{\pgfqpoint{0.845774in}{1.492698in}}{\pgfqpoint{0.854010in}{1.492698in}}%
\pgfpathclose%
\pgfusepath{stroke,fill}%
\end{pgfscope}%
\begin{pgfscope}%
\pgfpathrectangle{\pgfqpoint{0.100000in}{0.212622in}}{\pgfqpoint{3.696000in}{3.696000in}}%
\pgfusepath{clip}%
\pgfsetbuttcap%
\pgfsetroundjoin%
\definecolor{currentfill}{rgb}{0.121569,0.466667,0.705882}%
\pgfsetfillcolor{currentfill}%
\pgfsetfillopacity{0.581443}%
\pgfsetlinewidth{1.003750pt}%
\definecolor{currentstroke}{rgb}{0.121569,0.466667,0.705882}%
\pgfsetstrokecolor{currentstroke}%
\pgfsetstrokeopacity{0.581443}%
\pgfsetdash{}{0pt}%
\pgfpathmoveto{\pgfqpoint{3.220887in}{2.222886in}}%
\pgfpathcurveto{\pgfqpoint{3.229124in}{2.222886in}}{\pgfqpoint{3.237024in}{2.226159in}}{\pgfqpoint{3.242848in}{2.231983in}}%
\pgfpathcurveto{\pgfqpoint{3.248672in}{2.237807in}}{\pgfqpoint{3.251944in}{2.245707in}}{\pgfqpoint{3.251944in}{2.253943in}}%
\pgfpathcurveto{\pgfqpoint{3.251944in}{2.262179in}}{\pgfqpoint{3.248672in}{2.270079in}}{\pgfqpoint{3.242848in}{2.275903in}}%
\pgfpathcurveto{\pgfqpoint{3.237024in}{2.281727in}}{\pgfqpoint{3.229124in}{2.284999in}}{\pgfqpoint{3.220887in}{2.284999in}}%
\pgfpathcurveto{\pgfqpoint{3.212651in}{2.284999in}}{\pgfqpoint{3.204751in}{2.281727in}}{\pgfqpoint{3.198927in}{2.275903in}}%
\pgfpathcurveto{\pgfqpoint{3.193103in}{2.270079in}}{\pgfqpoint{3.189831in}{2.262179in}}{\pgfqpoint{3.189831in}{2.253943in}}%
\pgfpathcurveto{\pgfqpoint{3.189831in}{2.245707in}}{\pgfqpoint{3.193103in}{2.237807in}}{\pgfqpoint{3.198927in}{2.231983in}}%
\pgfpathcurveto{\pgfqpoint{3.204751in}{2.226159in}}{\pgfqpoint{3.212651in}{2.222886in}}{\pgfqpoint{3.220887in}{2.222886in}}%
\pgfpathclose%
\pgfusepath{stroke,fill}%
\end{pgfscope}%
\begin{pgfscope}%
\pgfpathrectangle{\pgfqpoint{0.100000in}{0.212622in}}{\pgfqpoint{3.696000in}{3.696000in}}%
\pgfusepath{clip}%
\pgfsetbuttcap%
\pgfsetroundjoin%
\definecolor{currentfill}{rgb}{0.121569,0.466667,0.705882}%
\pgfsetfillcolor{currentfill}%
\pgfsetfillopacity{0.581710}%
\pgfsetlinewidth{1.003750pt}%
\definecolor{currentstroke}{rgb}{0.121569,0.466667,0.705882}%
\pgfsetstrokecolor{currentstroke}%
\pgfsetstrokeopacity{0.581710}%
\pgfsetdash{}{0pt}%
\pgfpathmoveto{\pgfqpoint{0.852901in}{1.487644in}}%
\pgfpathcurveto{\pgfqpoint{0.861137in}{1.487644in}}{\pgfqpoint{0.869038in}{1.490916in}}{\pgfqpoint{0.874861in}{1.496740in}}%
\pgfpathcurveto{\pgfqpoint{0.880685in}{1.502564in}}{\pgfqpoint{0.883958in}{1.510464in}}{\pgfqpoint{0.883958in}{1.518700in}}%
\pgfpathcurveto{\pgfqpoint{0.883958in}{1.526936in}}{\pgfqpoint{0.880685in}{1.534836in}}{\pgfqpoint{0.874861in}{1.540660in}}%
\pgfpathcurveto{\pgfqpoint{0.869038in}{1.546484in}}{\pgfqpoint{0.861137in}{1.549757in}}{\pgfqpoint{0.852901in}{1.549757in}}%
\pgfpathcurveto{\pgfqpoint{0.844665in}{1.549757in}}{\pgfqpoint{0.836765in}{1.546484in}}{\pgfqpoint{0.830941in}{1.540660in}}%
\pgfpathcurveto{\pgfqpoint{0.825117in}{1.534836in}}{\pgfqpoint{0.821845in}{1.526936in}}{\pgfqpoint{0.821845in}{1.518700in}}%
\pgfpathcurveto{\pgfqpoint{0.821845in}{1.510464in}}{\pgfqpoint{0.825117in}{1.502564in}}{\pgfqpoint{0.830941in}{1.496740in}}%
\pgfpathcurveto{\pgfqpoint{0.836765in}{1.490916in}}{\pgfqpoint{0.844665in}{1.487644in}}{\pgfqpoint{0.852901in}{1.487644in}}%
\pgfpathclose%
\pgfusepath{stroke,fill}%
\end{pgfscope}%
\begin{pgfscope}%
\pgfpathrectangle{\pgfqpoint{0.100000in}{0.212622in}}{\pgfqpoint{3.696000in}{3.696000in}}%
\pgfusepath{clip}%
\pgfsetbuttcap%
\pgfsetroundjoin%
\definecolor{currentfill}{rgb}{0.121569,0.466667,0.705882}%
\pgfsetfillcolor{currentfill}%
\pgfsetfillopacity{0.581996}%
\pgfsetlinewidth{1.003750pt}%
\definecolor{currentstroke}{rgb}{0.121569,0.466667,0.705882}%
\pgfsetstrokecolor{currentstroke}%
\pgfsetstrokeopacity{0.581996}%
\pgfsetdash{}{0pt}%
\pgfpathmoveto{\pgfqpoint{3.219745in}{2.222492in}}%
\pgfpathcurveto{\pgfqpoint{3.227981in}{2.222492in}}{\pgfqpoint{3.235881in}{2.225764in}}{\pgfqpoint{3.241705in}{2.231588in}}%
\pgfpathcurveto{\pgfqpoint{3.247529in}{2.237412in}}{\pgfqpoint{3.250801in}{2.245312in}}{\pgfqpoint{3.250801in}{2.253548in}}%
\pgfpathcurveto{\pgfqpoint{3.250801in}{2.261785in}}{\pgfqpoint{3.247529in}{2.269685in}}{\pgfqpoint{3.241705in}{2.275509in}}%
\pgfpathcurveto{\pgfqpoint{3.235881in}{2.281332in}}{\pgfqpoint{3.227981in}{2.284605in}}{\pgfqpoint{3.219745in}{2.284605in}}%
\pgfpathcurveto{\pgfqpoint{3.211509in}{2.284605in}}{\pgfqpoint{3.203609in}{2.281332in}}{\pgfqpoint{3.197785in}{2.275509in}}%
\pgfpathcurveto{\pgfqpoint{3.191961in}{2.269685in}}{\pgfqpoint{3.188688in}{2.261785in}}{\pgfqpoint{3.188688in}{2.253548in}}%
\pgfpathcurveto{\pgfqpoint{3.188688in}{2.245312in}}{\pgfqpoint{3.191961in}{2.237412in}}{\pgfqpoint{3.197785in}{2.231588in}}%
\pgfpathcurveto{\pgfqpoint{3.203609in}{2.225764in}}{\pgfqpoint{3.211509in}{2.222492in}}{\pgfqpoint{3.219745in}{2.222492in}}%
\pgfpathclose%
\pgfusepath{stroke,fill}%
\end{pgfscope}%
\begin{pgfscope}%
\pgfpathrectangle{\pgfqpoint{0.100000in}{0.212622in}}{\pgfqpoint{3.696000in}{3.696000in}}%
\pgfusepath{clip}%
\pgfsetbuttcap%
\pgfsetroundjoin%
\definecolor{currentfill}{rgb}{0.121569,0.466667,0.705882}%
\pgfsetfillcolor{currentfill}%
\pgfsetfillopacity{0.582305}%
\pgfsetlinewidth{1.003750pt}%
\definecolor{currentstroke}{rgb}{0.121569,0.466667,0.705882}%
\pgfsetstrokecolor{currentstroke}%
\pgfsetstrokeopacity{0.582305}%
\pgfsetdash{}{0pt}%
\pgfpathmoveto{\pgfqpoint{3.219153in}{2.222263in}}%
\pgfpathcurveto{\pgfqpoint{3.227390in}{2.222263in}}{\pgfqpoint{3.235290in}{2.225535in}}{\pgfqpoint{3.241114in}{2.231359in}}%
\pgfpathcurveto{\pgfqpoint{3.246938in}{2.237183in}}{\pgfqpoint{3.250210in}{2.245083in}}{\pgfqpoint{3.250210in}{2.253319in}}%
\pgfpathcurveto{\pgfqpoint{3.250210in}{2.261555in}}{\pgfqpoint{3.246938in}{2.269456in}}{\pgfqpoint{3.241114in}{2.275279in}}%
\pgfpathcurveto{\pgfqpoint{3.235290in}{2.281103in}}{\pgfqpoint{3.227390in}{2.284376in}}{\pgfqpoint{3.219153in}{2.284376in}}%
\pgfpathcurveto{\pgfqpoint{3.210917in}{2.284376in}}{\pgfqpoint{3.203017in}{2.281103in}}{\pgfqpoint{3.197193in}{2.275279in}}%
\pgfpathcurveto{\pgfqpoint{3.191369in}{2.269456in}}{\pgfqpoint{3.188097in}{2.261555in}}{\pgfqpoint{3.188097in}{2.253319in}}%
\pgfpathcurveto{\pgfqpoint{3.188097in}{2.245083in}}{\pgfqpoint{3.191369in}{2.237183in}}{\pgfqpoint{3.197193in}{2.231359in}}%
\pgfpathcurveto{\pgfqpoint{3.203017in}{2.225535in}}{\pgfqpoint{3.210917in}{2.222263in}}{\pgfqpoint{3.219153in}{2.222263in}}%
\pgfpathclose%
\pgfusepath{stroke,fill}%
\end{pgfscope}%
\begin{pgfscope}%
\pgfpathrectangle{\pgfqpoint{0.100000in}{0.212622in}}{\pgfqpoint{3.696000in}{3.696000in}}%
\pgfusepath{clip}%
\pgfsetbuttcap%
\pgfsetroundjoin%
\definecolor{currentfill}{rgb}{0.121569,0.466667,0.705882}%
\pgfsetfillcolor{currentfill}%
\pgfsetfillopacity{0.582975}%
\pgfsetlinewidth{1.003750pt}%
\definecolor{currentstroke}{rgb}{0.121569,0.466667,0.705882}%
\pgfsetstrokecolor{currentstroke}%
\pgfsetstrokeopacity{0.582975}%
\pgfsetdash{}{0pt}%
\pgfpathmoveto{\pgfqpoint{3.217707in}{2.221626in}}%
\pgfpathcurveto{\pgfqpoint{3.225943in}{2.221626in}}{\pgfqpoint{3.233843in}{2.224899in}}{\pgfqpoint{3.239667in}{2.230723in}}%
\pgfpathcurveto{\pgfqpoint{3.245491in}{2.236547in}}{\pgfqpoint{3.248763in}{2.244447in}}{\pgfqpoint{3.248763in}{2.252683in}}%
\pgfpathcurveto{\pgfqpoint{3.248763in}{2.260919in}}{\pgfqpoint{3.245491in}{2.268819in}}{\pgfqpoint{3.239667in}{2.274643in}}%
\pgfpathcurveto{\pgfqpoint{3.233843in}{2.280467in}}{\pgfqpoint{3.225943in}{2.283739in}}{\pgfqpoint{3.217707in}{2.283739in}}%
\pgfpathcurveto{\pgfqpoint{3.209471in}{2.283739in}}{\pgfqpoint{3.201571in}{2.280467in}}{\pgfqpoint{3.195747in}{2.274643in}}%
\pgfpathcurveto{\pgfqpoint{3.189923in}{2.268819in}}{\pgfqpoint{3.186650in}{2.260919in}}{\pgfqpoint{3.186650in}{2.252683in}}%
\pgfpathcurveto{\pgfqpoint{3.186650in}{2.244447in}}{\pgfqpoint{3.189923in}{2.236547in}}{\pgfqpoint{3.195747in}{2.230723in}}%
\pgfpathcurveto{\pgfqpoint{3.201571in}{2.224899in}}{\pgfqpoint{3.209471in}{2.221626in}}{\pgfqpoint{3.217707in}{2.221626in}}%
\pgfpathclose%
\pgfusepath{stroke,fill}%
\end{pgfscope}%
\begin{pgfscope}%
\pgfpathrectangle{\pgfqpoint{0.100000in}{0.212622in}}{\pgfqpoint{3.696000in}{3.696000in}}%
\pgfusepath{clip}%
\pgfsetbuttcap%
\pgfsetroundjoin%
\definecolor{currentfill}{rgb}{0.121569,0.466667,0.705882}%
\pgfsetfillcolor{currentfill}%
\pgfsetfillopacity{0.583353}%
\pgfsetlinewidth{1.003750pt}%
\definecolor{currentstroke}{rgb}{0.121569,0.466667,0.705882}%
\pgfsetstrokecolor{currentstroke}%
\pgfsetstrokeopacity{0.583353}%
\pgfsetdash{}{0pt}%
\pgfpathmoveto{\pgfqpoint{3.216925in}{2.221325in}}%
\pgfpathcurveto{\pgfqpoint{3.225161in}{2.221325in}}{\pgfqpoint{3.233061in}{2.224597in}}{\pgfqpoint{3.238885in}{2.230421in}}%
\pgfpathcurveto{\pgfqpoint{3.244709in}{2.236245in}}{\pgfqpoint{3.247981in}{2.244145in}}{\pgfqpoint{3.247981in}{2.252381in}}%
\pgfpathcurveto{\pgfqpoint{3.247981in}{2.260618in}}{\pgfqpoint{3.244709in}{2.268518in}}{\pgfqpoint{3.238885in}{2.274342in}}%
\pgfpathcurveto{\pgfqpoint{3.233061in}{2.280165in}}{\pgfqpoint{3.225161in}{2.283438in}}{\pgfqpoint{3.216925in}{2.283438in}}%
\pgfpathcurveto{\pgfqpoint{3.208688in}{2.283438in}}{\pgfqpoint{3.200788in}{2.280165in}}{\pgfqpoint{3.194964in}{2.274342in}}%
\pgfpathcurveto{\pgfqpoint{3.189140in}{2.268518in}}{\pgfqpoint{3.185868in}{2.260618in}}{\pgfqpoint{3.185868in}{2.252381in}}%
\pgfpathcurveto{\pgfqpoint{3.185868in}{2.244145in}}{\pgfqpoint{3.189140in}{2.236245in}}{\pgfqpoint{3.194964in}{2.230421in}}%
\pgfpathcurveto{\pgfqpoint{3.200788in}{2.224597in}}{\pgfqpoint{3.208688in}{2.221325in}}{\pgfqpoint{3.216925in}{2.221325in}}%
\pgfpathclose%
\pgfusepath{stroke,fill}%
\end{pgfscope}%
\begin{pgfscope}%
\pgfpathrectangle{\pgfqpoint{0.100000in}{0.212622in}}{\pgfqpoint{3.696000in}{3.696000in}}%
\pgfusepath{clip}%
\pgfsetbuttcap%
\pgfsetroundjoin%
\definecolor{currentfill}{rgb}{0.121569,0.466667,0.705882}%
\pgfsetfillcolor{currentfill}%
\pgfsetfillopacity{0.583548}%
\pgfsetlinewidth{1.003750pt}%
\definecolor{currentstroke}{rgb}{0.121569,0.466667,0.705882}%
\pgfsetstrokecolor{currentstroke}%
\pgfsetstrokeopacity{0.583548}%
\pgfsetdash{}{0pt}%
\pgfpathmoveto{\pgfqpoint{0.851675in}{1.477880in}}%
\pgfpathcurveto{\pgfqpoint{0.859911in}{1.477880in}}{\pgfqpoint{0.867811in}{1.481152in}}{\pgfqpoint{0.873635in}{1.486976in}}%
\pgfpathcurveto{\pgfqpoint{0.879459in}{1.492800in}}{\pgfqpoint{0.882732in}{1.500700in}}{\pgfqpoint{0.882732in}{1.508936in}}%
\pgfpathcurveto{\pgfqpoint{0.882732in}{1.517172in}}{\pgfqpoint{0.879459in}{1.525073in}}{\pgfqpoint{0.873635in}{1.530896in}}%
\pgfpathcurveto{\pgfqpoint{0.867811in}{1.536720in}}{\pgfqpoint{0.859911in}{1.539993in}}{\pgfqpoint{0.851675in}{1.539993in}}%
\pgfpathcurveto{\pgfqpoint{0.843439in}{1.539993in}}{\pgfqpoint{0.835539in}{1.536720in}}{\pgfqpoint{0.829715in}{1.530896in}}%
\pgfpathcurveto{\pgfqpoint{0.823891in}{1.525073in}}{\pgfqpoint{0.820619in}{1.517172in}}{\pgfqpoint{0.820619in}{1.508936in}}%
\pgfpathcurveto{\pgfqpoint{0.820619in}{1.500700in}}{\pgfqpoint{0.823891in}{1.492800in}}{\pgfqpoint{0.829715in}{1.486976in}}%
\pgfpathcurveto{\pgfqpoint{0.835539in}{1.481152in}}{\pgfqpoint{0.843439in}{1.477880in}}{\pgfqpoint{0.851675in}{1.477880in}}%
\pgfpathclose%
\pgfusepath{stroke,fill}%
\end{pgfscope}%
\begin{pgfscope}%
\pgfpathrectangle{\pgfqpoint{0.100000in}{0.212622in}}{\pgfqpoint{3.696000in}{3.696000in}}%
\pgfusepath{clip}%
\pgfsetbuttcap%
\pgfsetroundjoin%
\definecolor{currentfill}{rgb}{0.121569,0.466667,0.705882}%
\pgfsetfillcolor{currentfill}%
\pgfsetfillopacity{0.583566}%
\pgfsetlinewidth{1.003750pt}%
\definecolor{currentstroke}{rgb}{0.121569,0.466667,0.705882}%
\pgfsetstrokecolor{currentstroke}%
\pgfsetstrokeopacity{0.583566}%
\pgfsetdash{}{0pt}%
\pgfpathmoveto{\pgfqpoint{3.216520in}{2.221169in}}%
\pgfpathcurveto{\pgfqpoint{3.224757in}{2.221169in}}{\pgfqpoint{3.232657in}{2.224441in}}{\pgfqpoint{3.238481in}{2.230265in}}%
\pgfpathcurveto{\pgfqpoint{3.244304in}{2.236089in}}{\pgfqpoint{3.247577in}{2.243989in}}{\pgfqpoint{3.247577in}{2.252225in}}%
\pgfpathcurveto{\pgfqpoint{3.247577in}{2.260461in}}{\pgfqpoint{3.244304in}{2.268361in}}{\pgfqpoint{3.238481in}{2.274185in}}%
\pgfpathcurveto{\pgfqpoint{3.232657in}{2.280009in}}{\pgfqpoint{3.224757in}{2.283282in}}{\pgfqpoint{3.216520in}{2.283282in}}%
\pgfpathcurveto{\pgfqpoint{3.208284in}{2.283282in}}{\pgfqpoint{3.200384in}{2.280009in}}{\pgfqpoint{3.194560in}{2.274185in}}%
\pgfpathcurveto{\pgfqpoint{3.188736in}{2.268361in}}{\pgfqpoint{3.185464in}{2.260461in}}{\pgfqpoint{3.185464in}{2.252225in}}%
\pgfpathcurveto{\pgfqpoint{3.185464in}{2.243989in}}{\pgfqpoint{3.188736in}{2.236089in}}{\pgfqpoint{3.194560in}{2.230265in}}%
\pgfpathcurveto{\pgfqpoint{3.200384in}{2.224441in}}{\pgfqpoint{3.208284in}{2.221169in}}{\pgfqpoint{3.216520in}{2.221169in}}%
\pgfpathclose%
\pgfusepath{stroke,fill}%
\end{pgfscope}%
\begin{pgfscope}%
\pgfpathrectangle{\pgfqpoint{0.100000in}{0.212622in}}{\pgfqpoint{3.696000in}{3.696000in}}%
\pgfusepath{clip}%
\pgfsetbuttcap%
\pgfsetroundjoin%
\definecolor{currentfill}{rgb}{0.121569,0.466667,0.705882}%
\pgfsetfillcolor{currentfill}%
\pgfsetfillopacity{0.583677}%
\pgfsetlinewidth{1.003750pt}%
\definecolor{currentstroke}{rgb}{0.121569,0.466667,0.705882}%
\pgfsetstrokecolor{currentstroke}%
\pgfsetstrokeopacity{0.583677}%
\pgfsetdash{}{0pt}%
\pgfpathmoveto{\pgfqpoint{3.216283in}{2.221056in}}%
\pgfpathcurveto{\pgfqpoint{3.224519in}{2.221056in}}{\pgfqpoint{3.232419in}{2.224329in}}{\pgfqpoint{3.238243in}{2.230153in}}%
\pgfpathcurveto{\pgfqpoint{3.244067in}{2.235976in}}{\pgfqpoint{3.247340in}{2.243877in}}{\pgfqpoint{3.247340in}{2.252113in}}%
\pgfpathcurveto{\pgfqpoint{3.247340in}{2.260349in}}{\pgfqpoint{3.244067in}{2.268249in}}{\pgfqpoint{3.238243in}{2.274073in}}%
\pgfpathcurveto{\pgfqpoint{3.232419in}{2.279897in}}{\pgfqpoint{3.224519in}{2.283169in}}{\pgfqpoint{3.216283in}{2.283169in}}%
\pgfpathcurveto{\pgfqpoint{3.208047in}{2.283169in}}{\pgfqpoint{3.200147in}{2.279897in}}{\pgfqpoint{3.194323in}{2.274073in}}%
\pgfpathcurveto{\pgfqpoint{3.188499in}{2.268249in}}{\pgfqpoint{3.185227in}{2.260349in}}{\pgfqpoint{3.185227in}{2.252113in}}%
\pgfpathcurveto{\pgfqpoint{3.185227in}{2.243877in}}{\pgfqpoint{3.188499in}{2.235976in}}{\pgfqpoint{3.194323in}{2.230153in}}%
\pgfpathcurveto{\pgfqpoint{3.200147in}{2.224329in}}{\pgfqpoint{3.208047in}{2.221056in}}{\pgfqpoint{3.216283in}{2.221056in}}%
\pgfpathclose%
\pgfusepath{stroke,fill}%
\end{pgfscope}%
\begin{pgfscope}%
\pgfpathrectangle{\pgfqpoint{0.100000in}{0.212622in}}{\pgfqpoint{3.696000in}{3.696000in}}%
\pgfusepath{clip}%
\pgfsetbuttcap%
\pgfsetroundjoin%
\definecolor{currentfill}{rgb}{0.121569,0.466667,0.705882}%
\pgfsetfillcolor{currentfill}%
\pgfsetfillopacity{0.584362}%
\pgfsetlinewidth{1.003750pt}%
\definecolor{currentstroke}{rgb}{0.121569,0.466667,0.705882}%
\pgfsetstrokecolor{currentstroke}%
\pgfsetstrokeopacity{0.584362}%
\pgfsetdash{}{0pt}%
\pgfpathmoveto{\pgfqpoint{3.214963in}{2.220529in}}%
\pgfpathcurveto{\pgfqpoint{3.223200in}{2.220529in}}{\pgfqpoint{3.231100in}{2.223801in}}{\pgfqpoint{3.236924in}{2.229625in}}%
\pgfpathcurveto{\pgfqpoint{3.242748in}{2.235449in}}{\pgfqpoint{3.246020in}{2.243349in}}{\pgfqpoint{3.246020in}{2.251585in}}%
\pgfpathcurveto{\pgfqpoint{3.246020in}{2.259821in}}{\pgfqpoint{3.242748in}{2.267721in}}{\pgfqpoint{3.236924in}{2.273545in}}%
\pgfpathcurveto{\pgfqpoint{3.231100in}{2.279369in}}{\pgfqpoint{3.223200in}{2.282642in}}{\pgfqpoint{3.214963in}{2.282642in}}%
\pgfpathcurveto{\pgfqpoint{3.206727in}{2.282642in}}{\pgfqpoint{3.198827in}{2.279369in}}{\pgfqpoint{3.193003in}{2.273545in}}%
\pgfpathcurveto{\pgfqpoint{3.187179in}{2.267721in}}{\pgfqpoint{3.183907in}{2.259821in}}{\pgfqpoint{3.183907in}{2.251585in}}%
\pgfpathcurveto{\pgfqpoint{3.183907in}{2.243349in}}{\pgfqpoint{3.187179in}{2.235449in}}{\pgfqpoint{3.193003in}{2.229625in}}%
\pgfpathcurveto{\pgfqpoint{3.198827in}{2.223801in}}{\pgfqpoint{3.206727in}{2.220529in}}{\pgfqpoint{3.214963in}{2.220529in}}%
\pgfpathclose%
\pgfusepath{stroke,fill}%
\end{pgfscope}%
\begin{pgfscope}%
\pgfpathrectangle{\pgfqpoint{0.100000in}{0.212622in}}{\pgfqpoint{3.696000in}{3.696000in}}%
\pgfusepath{clip}%
\pgfsetbuttcap%
\pgfsetroundjoin%
\definecolor{currentfill}{rgb}{0.121569,0.466667,0.705882}%
\pgfsetfillcolor{currentfill}%
\pgfsetfillopacity{0.584729}%
\pgfsetlinewidth{1.003750pt}%
\definecolor{currentstroke}{rgb}{0.121569,0.466667,0.705882}%
\pgfsetstrokecolor{currentstroke}%
\pgfsetstrokeopacity{0.584729}%
\pgfsetdash{}{0pt}%
\pgfpathmoveto{\pgfqpoint{3.214219in}{2.220200in}}%
\pgfpathcurveto{\pgfqpoint{3.222455in}{2.220200in}}{\pgfqpoint{3.230355in}{2.223472in}}{\pgfqpoint{3.236179in}{2.229296in}}%
\pgfpathcurveto{\pgfqpoint{3.242003in}{2.235120in}}{\pgfqpoint{3.245276in}{2.243020in}}{\pgfqpoint{3.245276in}{2.251256in}}%
\pgfpathcurveto{\pgfqpoint{3.245276in}{2.259492in}}{\pgfqpoint{3.242003in}{2.267392in}}{\pgfqpoint{3.236179in}{2.273216in}}%
\pgfpathcurveto{\pgfqpoint{3.230355in}{2.279040in}}{\pgfqpoint{3.222455in}{2.282313in}}{\pgfqpoint{3.214219in}{2.282313in}}%
\pgfpathcurveto{\pgfqpoint{3.205983in}{2.282313in}}{\pgfqpoint{3.198083in}{2.279040in}}{\pgfqpoint{3.192259in}{2.273216in}}%
\pgfpathcurveto{\pgfqpoint{3.186435in}{2.267392in}}{\pgfqpoint{3.183163in}{2.259492in}}{\pgfqpoint{3.183163in}{2.251256in}}%
\pgfpathcurveto{\pgfqpoint{3.183163in}{2.243020in}}{\pgfqpoint{3.186435in}{2.235120in}}{\pgfqpoint{3.192259in}{2.229296in}}%
\pgfpathcurveto{\pgfqpoint{3.198083in}{2.223472in}}{\pgfqpoint{3.205983in}{2.220200in}}{\pgfqpoint{3.214219in}{2.220200in}}%
\pgfpathclose%
\pgfusepath{stroke,fill}%
\end{pgfscope}%
\begin{pgfscope}%
\pgfpathrectangle{\pgfqpoint{0.100000in}{0.212622in}}{\pgfqpoint{3.696000in}{3.696000in}}%
\pgfusepath{clip}%
\pgfsetbuttcap%
\pgfsetroundjoin%
\definecolor{currentfill}{rgb}{0.121569,0.466667,0.705882}%
\pgfsetfillcolor{currentfill}%
\pgfsetfillopacity{0.584926}%
\pgfsetlinewidth{1.003750pt}%
\definecolor{currentstroke}{rgb}{0.121569,0.466667,0.705882}%
\pgfsetstrokecolor{currentstroke}%
\pgfsetstrokeopacity{0.584926}%
\pgfsetdash{}{0pt}%
\pgfpathmoveto{\pgfqpoint{3.213790in}{2.220006in}}%
\pgfpathcurveto{\pgfqpoint{3.222026in}{2.220006in}}{\pgfqpoint{3.229926in}{2.223279in}}{\pgfqpoint{3.235750in}{2.229103in}}%
\pgfpathcurveto{\pgfqpoint{3.241574in}{2.234927in}}{\pgfqpoint{3.244846in}{2.242827in}}{\pgfqpoint{3.244846in}{2.251063in}}%
\pgfpathcurveto{\pgfqpoint{3.244846in}{2.259299in}}{\pgfqpoint{3.241574in}{2.267199in}}{\pgfqpoint{3.235750in}{2.273023in}}%
\pgfpathcurveto{\pgfqpoint{3.229926in}{2.278847in}}{\pgfqpoint{3.222026in}{2.282119in}}{\pgfqpoint{3.213790in}{2.282119in}}%
\pgfpathcurveto{\pgfqpoint{3.205554in}{2.282119in}}{\pgfqpoint{3.197654in}{2.278847in}}{\pgfqpoint{3.191830in}{2.273023in}}%
\pgfpathcurveto{\pgfqpoint{3.186006in}{2.267199in}}{\pgfqpoint{3.182733in}{2.259299in}}{\pgfqpoint{3.182733in}{2.251063in}}%
\pgfpathcurveto{\pgfqpoint{3.182733in}{2.242827in}}{\pgfqpoint{3.186006in}{2.234927in}}{\pgfqpoint{3.191830in}{2.229103in}}%
\pgfpathcurveto{\pgfqpoint{3.197654in}{2.223279in}}{\pgfqpoint{3.205554in}{2.220006in}}{\pgfqpoint{3.213790in}{2.220006in}}%
\pgfpathclose%
\pgfusepath{stroke,fill}%
\end{pgfscope}%
\begin{pgfscope}%
\pgfpathrectangle{\pgfqpoint{0.100000in}{0.212622in}}{\pgfqpoint{3.696000in}{3.696000in}}%
\pgfusepath{clip}%
\pgfsetbuttcap%
\pgfsetroundjoin%
\definecolor{currentfill}{rgb}{0.121569,0.466667,0.705882}%
\pgfsetfillcolor{currentfill}%
\pgfsetfillopacity{0.585036}%
\pgfsetlinewidth{1.003750pt}%
\definecolor{currentstroke}{rgb}{0.121569,0.466667,0.705882}%
\pgfsetstrokecolor{currentstroke}%
\pgfsetstrokeopacity{0.585036}%
\pgfsetdash{}{0pt}%
\pgfpathmoveto{\pgfqpoint{3.213562in}{2.219900in}}%
\pgfpathcurveto{\pgfqpoint{3.221798in}{2.219900in}}{\pgfqpoint{3.229698in}{2.223173in}}{\pgfqpoint{3.235522in}{2.228997in}}%
\pgfpathcurveto{\pgfqpoint{3.241346in}{2.234821in}}{\pgfqpoint{3.244619in}{2.242721in}}{\pgfqpoint{3.244619in}{2.250957in}}%
\pgfpathcurveto{\pgfqpoint{3.244619in}{2.259193in}}{\pgfqpoint{3.241346in}{2.267093in}}{\pgfqpoint{3.235522in}{2.272917in}}%
\pgfpathcurveto{\pgfqpoint{3.229698in}{2.278741in}}{\pgfqpoint{3.221798in}{2.282013in}}{\pgfqpoint{3.213562in}{2.282013in}}%
\pgfpathcurveto{\pgfqpoint{3.205326in}{2.282013in}}{\pgfqpoint{3.197426in}{2.278741in}}{\pgfqpoint{3.191602in}{2.272917in}}%
\pgfpathcurveto{\pgfqpoint{3.185778in}{2.267093in}}{\pgfqpoint{3.182506in}{2.259193in}}{\pgfqpoint{3.182506in}{2.250957in}}%
\pgfpathcurveto{\pgfqpoint{3.182506in}{2.242721in}}{\pgfqpoint{3.185778in}{2.234821in}}{\pgfqpoint{3.191602in}{2.228997in}}%
\pgfpathcurveto{\pgfqpoint{3.197426in}{2.223173in}}{\pgfqpoint{3.205326in}{2.219900in}}{\pgfqpoint{3.213562in}{2.219900in}}%
\pgfpathclose%
\pgfusepath{stroke,fill}%
\end{pgfscope}%
\begin{pgfscope}%
\pgfpathrectangle{\pgfqpoint{0.100000in}{0.212622in}}{\pgfqpoint{3.696000in}{3.696000in}}%
\pgfusepath{clip}%
\pgfsetbuttcap%
\pgfsetroundjoin%
\definecolor{currentfill}{rgb}{0.121569,0.466667,0.705882}%
\pgfsetfillcolor{currentfill}%
\pgfsetfillopacity{0.585127}%
\pgfsetlinewidth{1.003750pt}%
\definecolor{currentstroke}{rgb}{0.121569,0.466667,0.705882}%
\pgfsetstrokecolor{currentstroke}%
\pgfsetstrokeopacity{0.585127}%
\pgfsetdash{}{0pt}%
\pgfpathmoveto{\pgfqpoint{0.850787in}{1.469409in}}%
\pgfpathcurveto{\pgfqpoint{0.859023in}{1.469409in}}{\pgfqpoint{0.866924in}{1.472681in}}{\pgfqpoint{0.872747in}{1.478505in}}%
\pgfpathcurveto{\pgfqpoint{0.878571in}{1.484329in}}{\pgfqpoint{0.881844in}{1.492229in}}{\pgfqpoint{0.881844in}{1.500465in}}%
\pgfpathcurveto{\pgfqpoint{0.881844in}{1.508702in}}{\pgfqpoint{0.878571in}{1.516602in}}{\pgfqpoint{0.872747in}{1.522426in}}%
\pgfpathcurveto{\pgfqpoint{0.866924in}{1.528250in}}{\pgfqpoint{0.859023in}{1.531522in}}{\pgfqpoint{0.850787in}{1.531522in}}%
\pgfpathcurveto{\pgfqpoint{0.842551in}{1.531522in}}{\pgfqpoint{0.834651in}{1.528250in}}{\pgfqpoint{0.828827in}{1.522426in}}%
\pgfpathcurveto{\pgfqpoint{0.823003in}{1.516602in}}{\pgfqpoint{0.819731in}{1.508702in}}{\pgfqpoint{0.819731in}{1.500465in}}%
\pgfpathcurveto{\pgfqpoint{0.819731in}{1.492229in}}{\pgfqpoint{0.823003in}{1.484329in}}{\pgfqpoint{0.828827in}{1.478505in}}%
\pgfpathcurveto{\pgfqpoint{0.834651in}{1.472681in}}{\pgfqpoint{0.842551in}{1.469409in}}{\pgfqpoint{0.850787in}{1.469409in}}%
\pgfpathclose%
\pgfusepath{stroke,fill}%
\end{pgfscope}%
\begin{pgfscope}%
\pgfpathrectangle{\pgfqpoint{0.100000in}{0.212622in}}{\pgfqpoint{3.696000in}{3.696000in}}%
\pgfusepath{clip}%
\pgfsetbuttcap%
\pgfsetroundjoin%
\definecolor{currentfill}{rgb}{0.121569,0.466667,0.705882}%
\pgfsetfillcolor{currentfill}%
\pgfsetfillopacity{0.585543}%
\pgfsetlinewidth{1.003750pt}%
\definecolor{currentstroke}{rgb}{0.121569,0.466667,0.705882}%
\pgfsetstrokecolor{currentstroke}%
\pgfsetstrokeopacity{0.585543}%
\pgfsetdash{}{0pt}%
\pgfpathmoveto{\pgfqpoint{3.212529in}{2.219533in}}%
\pgfpathcurveto{\pgfqpoint{3.220765in}{2.219533in}}{\pgfqpoint{3.228665in}{2.222806in}}{\pgfqpoint{3.234489in}{2.228630in}}%
\pgfpathcurveto{\pgfqpoint{3.240313in}{2.234454in}}{\pgfqpoint{3.243585in}{2.242354in}}{\pgfqpoint{3.243585in}{2.250590in}}%
\pgfpathcurveto{\pgfqpoint{3.243585in}{2.258826in}}{\pgfqpoint{3.240313in}{2.266726in}}{\pgfqpoint{3.234489in}{2.272550in}}%
\pgfpathcurveto{\pgfqpoint{3.228665in}{2.278374in}}{\pgfqpoint{3.220765in}{2.281646in}}{\pgfqpoint{3.212529in}{2.281646in}}%
\pgfpathcurveto{\pgfqpoint{3.204293in}{2.281646in}}{\pgfqpoint{3.196393in}{2.278374in}}{\pgfqpoint{3.190569in}{2.272550in}}%
\pgfpathcurveto{\pgfqpoint{3.184745in}{2.266726in}}{\pgfqpoint{3.181472in}{2.258826in}}{\pgfqpoint{3.181472in}{2.250590in}}%
\pgfpathcurveto{\pgfqpoint{3.181472in}{2.242354in}}{\pgfqpoint{3.184745in}{2.234454in}}{\pgfqpoint{3.190569in}{2.228630in}}%
\pgfpathcurveto{\pgfqpoint{3.196393in}{2.222806in}}{\pgfqpoint{3.204293in}{2.219533in}}{\pgfqpoint{3.212529in}{2.219533in}}%
\pgfpathclose%
\pgfusepath{stroke,fill}%
\end{pgfscope}%
\begin{pgfscope}%
\pgfpathrectangle{\pgfqpoint{0.100000in}{0.212622in}}{\pgfqpoint{3.696000in}{3.696000in}}%
\pgfusepath{clip}%
\pgfsetbuttcap%
\pgfsetroundjoin%
\definecolor{currentfill}{rgb}{0.121569,0.466667,0.705882}%
\pgfsetfillcolor{currentfill}%
\pgfsetfillopacity{0.585820}%
\pgfsetlinewidth{1.003750pt}%
\definecolor{currentstroke}{rgb}{0.121569,0.466667,0.705882}%
\pgfsetstrokecolor{currentstroke}%
\pgfsetstrokeopacity{0.585820}%
\pgfsetdash{}{0pt}%
\pgfpathmoveto{\pgfqpoint{3.211951in}{2.219330in}}%
\pgfpathcurveto{\pgfqpoint{3.220188in}{2.219330in}}{\pgfqpoint{3.228088in}{2.222602in}}{\pgfqpoint{3.233912in}{2.228426in}}%
\pgfpathcurveto{\pgfqpoint{3.239735in}{2.234250in}}{\pgfqpoint{3.243008in}{2.242150in}}{\pgfqpoint{3.243008in}{2.250386in}}%
\pgfpathcurveto{\pgfqpoint{3.243008in}{2.258622in}}{\pgfqpoint{3.239735in}{2.266522in}}{\pgfqpoint{3.233912in}{2.272346in}}%
\pgfpathcurveto{\pgfqpoint{3.228088in}{2.278170in}}{\pgfqpoint{3.220188in}{2.281443in}}{\pgfqpoint{3.211951in}{2.281443in}}%
\pgfpathcurveto{\pgfqpoint{3.203715in}{2.281443in}}{\pgfqpoint{3.195815in}{2.278170in}}{\pgfqpoint{3.189991in}{2.272346in}}%
\pgfpathcurveto{\pgfqpoint{3.184167in}{2.266522in}}{\pgfqpoint{3.180895in}{2.258622in}}{\pgfqpoint{3.180895in}{2.250386in}}%
\pgfpathcurveto{\pgfqpoint{3.180895in}{2.242150in}}{\pgfqpoint{3.184167in}{2.234250in}}{\pgfqpoint{3.189991in}{2.228426in}}%
\pgfpathcurveto{\pgfqpoint{3.195815in}{2.222602in}}{\pgfqpoint{3.203715in}{2.219330in}}{\pgfqpoint{3.211951in}{2.219330in}}%
\pgfpathclose%
\pgfusepath{stroke,fill}%
\end{pgfscope}%
\begin{pgfscope}%
\pgfpathrectangle{\pgfqpoint{0.100000in}{0.212622in}}{\pgfqpoint{3.696000in}{3.696000in}}%
\pgfusepath{clip}%
\pgfsetbuttcap%
\pgfsetroundjoin%
\definecolor{currentfill}{rgb}{0.121569,0.466667,0.705882}%
\pgfsetfillcolor{currentfill}%
\pgfsetfillopacity{0.585978}%
\pgfsetlinewidth{1.003750pt}%
\definecolor{currentstroke}{rgb}{0.121569,0.466667,0.705882}%
\pgfsetstrokecolor{currentstroke}%
\pgfsetstrokeopacity{0.585978}%
\pgfsetdash{}{0pt}%
\pgfpathmoveto{\pgfqpoint{3.211652in}{2.219232in}}%
\pgfpathcurveto{\pgfqpoint{3.219888in}{2.219232in}}{\pgfqpoint{3.227788in}{2.222504in}}{\pgfqpoint{3.233612in}{2.228328in}}%
\pgfpathcurveto{\pgfqpoint{3.239436in}{2.234152in}}{\pgfqpoint{3.242708in}{2.242052in}}{\pgfqpoint{3.242708in}{2.250288in}}%
\pgfpathcurveto{\pgfqpoint{3.242708in}{2.258524in}}{\pgfqpoint{3.239436in}{2.266424in}}{\pgfqpoint{3.233612in}{2.272248in}}%
\pgfpathcurveto{\pgfqpoint{3.227788in}{2.278072in}}{\pgfqpoint{3.219888in}{2.281345in}}{\pgfqpoint{3.211652in}{2.281345in}}%
\pgfpathcurveto{\pgfqpoint{3.203416in}{2.281345in}}{\pgfqpoint{3.195516in}{2.278072in}}{\pgfqpoint{3.189692in}{2.272248in}}%
\pgfpathcurveto{\pgfqpoint{3.183868in}{2.266424in}}{\pgfqpoint{3.180595in}{2.258524in}}{\pgfqpoint{3.180595in}{2.250288in}}%
\pgfpathcurveto{\pgfqpoint{3.180595in}{2.242052in}}{\pgfqpoint{3.183868in}{2.234152in}}{\pgfqpoint{3.189692in}{2.228328in}}%
\pgfpathcurveto{\pgfqpoint{3.195516in}{2.222504in}}{\pgfqpoint{3.203416in}{2.219232in}}{\pgfqpoint{3.211652in}{2.219232in}}%
\pgfpathclose%
\pgfusepath{stroke,fill}%
\end{pgfscope}%
\begin{pgfscope}%
\pgfpathrectangle{\pgfqpoint{0.100000in}{0.212622in}}{\pgfqpoint{3.696000in}{3.696000in}}%
\pgfusepath{clip}%
\pgfsetbuttcap%
\pgfsetroundjoin%
\definecolor{currentfill}{rgb}{0.121569,0.466667,0.705882}%
\pgfsetfillcolor{currentfill}%
\pgfsetfillopacity{0.586352}%
\pgfsetlinewidth{1.003750pt}%
\definecolor{currentstroke}{rgb}{0.121569,0.466667,0.705882}%
\pgfsetstrokecolor{currentstroke}%
\pgfsetstrokeopacity{0.586352}%
\pgfsetdash{}{0pt}%
\pgfpathmoveto{\pgfqpoint{3.210921in}{2.219150in}}%
\pgfpathcurveto{\pgfqpoint{3.219157in}{2.219150in}}{\pgfqpoint{3.227057in}{2.222423in}}{\pgfqpoint{3.232881in}{2.228247in}}%
\pgfpathcurveto{\pgfqpoint{3.238705in}{2.234071in}}{\pgfqpoint{3.241977in}{2.241971in}}{\pgfqpoint{3.241977in}{2.250207in}}%
\pgfpathcurveto{\pgfqpoint{3.241977in}{2.258443in}}{\pgfqpoint{3.238705in}{2.266343in}}{\pgfqpoint{3.232881in}{2.272167in}}%
\pgfpathcurveto{\pgfqpoint{3.227057in}{2.277991in}}{\pgfqpoint{3.219157in}{2.281263in}}{\pgfqpoint{3.210921in}{2.281263in}}%
\pgfpathcurveto{\pgfqpoint{3.202685in}{2.281263in}}{\pgfqpoint{3.194785in}{2.277991in}}{\pgfqpoint{3.188961in}{2.272167in}}%
\pgfpathcurveto{\pgfqpoint{3.183137in}{2.266343in}}{\pgfqpoint{3.179864in}{2.258443in}}{\pgfqpoint{3.179864in}{2.250207in}}%
\pgfpathcurveto{\pgfqpoint{3.179864in}{2.241971in}}{\pgfqpoint{3.183137in}{2.234071in}}{\pgfqpoint{3.188961in}{2.228247in}}%
\pgfpathcurveto{\pgfqpoint{3.194785in}{2.222423in}}{\pgfqpoint{3.202685in}{2.219150in}}{\pgfqpoint{3.210921in}{2.219150in}}%
\pgfpathclose%
\pgfusepath{stroke,fill}%
\end{pgfscope}%
\begin{pgfscope}%
\pgfpathrectangle{\pgfqpoint{0.100000in}{0.212622in}}{\pgfqpoint{3.696000in}{3.696000in}}%
\pgfusepath{clip}%
\pgfsetbuttcap%
\pgfsetroundjoin%
\definecolor{currentfill}{rgb}{0.121569,0.466667,0.705882}%
\pgfsetfillcolor{currentfill}%
\pgfsetfillopacity{0.586952}%
\pgfsetlinewidth{1.003750pt}%
\definecolor{currentstroke}{rgb}{0.121569,0.466667,0.705882}%
\pgfsetstrokecolor{currentstroke}%
\pgfsetstrokeopacity{0.586952}%
\pgfsetdash{}{0pt}%
\pgfpathmoveto{\pgfqpoint{3.209717in}{2.218926in}}%
\pgfpathcurveto{\pgfqpoint{3.217954in}{2.218926in}}{\pgfqpoint{3.225854in}{2.222198in}}{\pgfqpoint{3.231678in}{2.228022in}}%
\pgfpathcurveto{\pgfqpoint{3.237501in}{2.233846in}}{\pgfqpoint{3.240774in}{2.241746in}}{\pgfqpoint{3.240774in}{2.249982in}}%
\pgfpathcurveto{\pgfqpoint{3.240774in}{2.258219in}}{\pgfqpoint{3.237501in}{2.266119in}}{\pgfqpoint{3.231678in}{2.271943in}}%
\pgfpathcurveto{\pgfqpoint{3.225854in}{2.277766in}}{\pgfqpoint{3.217954in}{2.281039in}}{\pgfqpoint{3.209717in}{2.281039in}}%
\pgfpathcurveto{\pgfqpoint{3.201481in}{2.281039in}}{\pgfqpoint{3.193581in}{2.277766in}}{\pgfqpoint{3.187757in}{2.271943in}}%
\pgfpathcurveto{\pgfqpoint{3.181933in}{2.266119in}}{\pgfqpoint{3.178661in}{2.258219in}}{\pgfqpoint{3.178661in}{2.249982in}}%
\pgfpathcurveto{\pgfqpoint{3.178661in}{2.241746in}}{\pgfqpoint{3.181933in}{2.233846in}}{\pgfqpoint{3.187757in}{2.228022in}}%
\pgfpathcurveto{\pgfqpoint{3.193581in}{2.222198in}}{\pgfqpoint{3.201481in}{2.218926in}}{\pgfqpoint{3.209717in}{2.218926in}}%
\pgfpathclose%
\pgfusepath{stroke,fill}%
\end{pgfscope}%
\begin{pgfscope}%
\pgfpathrectangle{\pgfqpoint{0.100000in}{0.212622in}}{\pgfqpoint{3.696000in}{3.696000in}}%
\pgfusepath{clip}%
\pgfsetbuttcap%
\pgfsetroundjoin%
\definecolor{currentfill}{rgb}{0.121569,0.466667,0.705882}%
\pgfsetfillcolor{currentfill}%
\pgfsetfillopacity{0.587282}%
\pgfsetlinewidth{1.003750pt}%
\definecolor{currentstroke}{rgb}{0.121569,0.466667,0.705882}%
\pgfsetstrokecolor{currentstroke}%
\pgfsetstrokeopacity{0.587282}%
\pgfsetdash{}{0pt}%
\pgfpathmoveto{\pgfqpoint{3.209093in}{2.218755in}}%
\pgfpathcurveto{\pgfqpoint{3.217329in}{2.218755in}}{\pgfqpoint{3.225229in}{2.222027in}}{\pgfqpoint{3.231053in}{2.227851in}}%
\pgfpathcurveto{\pgfqpoint{3.236877in}{2.233675in}}{\pgfqpoint{3.240149in}{2.241575in}}{\pgfqpoint{3.240149in}{2.249811in}}%
\pgfpathcurveto{\pgfqpoint{3.240149in}{2.258047in}}{\pgfqpoint{3.236877in}{2.265947in}}{\pgfqpoint{3.231053in}{2.271771in}}%
\pgfpathcurveto{\pgfqpoint{3.225229in}{2.277595in}}{\pgfqpoint{3.217329in}{2.280868in}}{\pgfqpoint{3.209093in}{2.280868in}}%
\pgfpathcurveto{\pgfqpoint{3.200857in}{2.280868in}}{\pgfqpoint{3.192957in}{2.277595in}}{\pgfqpoint{3.187133in}{2.271771in}}%
\pgfpathcurveto{\pgfqpoint{3.181309in}{2.265947in}}{\pgfqpoint{3.178036in}{2.258047in}}{\pgfqpoint{3.178036in}{2.249811in}}%
\pgfpathcurveto{\pgfqpoint{3.178036in}{2.241575in}}{\pgfqpoint{3.181309in}{2.233675in}}{\pgfqpoint{3.187133in}{2.227851in}}%
\pgfpathcurveto{\pgfqpoint{3.192957in}{2.222027in}}{\pgfqpoint{3.200857in}{2.218755in}}{\pgfqpoint{3.209093in}{2.218755in}}%
\pgfpathclose%
\pgfusepath{stroke,fill}%
\end{pgfscope}%
\begin{pgfscope}%
\pgfpathrectangle{\pgfqpoint{0.100000in}{0.212622in}}{\pgfqpoint{3.696000in}{3.696000in}}%
\pgfusepath{clip}%
\pgfsetbuttcap%
\pgfsetroundjoin%
\definecolor{currentfill}{rgb}{0.121569,0.466667,0.705882}%
\pgfsetfillcolor{currentfill}%
\pgfsetfillopacity{0.587880}%
\pgfsetlinewidth{1.003750pt}%
\definecolor{currentstroke}{rgb}{0.121569,0.466667,0.705882}%
\pgfsetstrokecolor{currentstroke}%
\pgfsetstrokeopacity{0.587880}%
\pgfsetdash{}{0pt}%
\pgfpathmoveto{\pgfqpoint{0.850435in}{1.453157in}}%
\pgfpathcurveto{\pgfqpoint{0.858672in}{1.453157in}}{\pgfqpoint{0.866572in}{1.456430in}}{\pgfqpoint{0.872396in}{1.462254in}}%
\pgfpathcurveto{\pgfqpoint{0.878220in}{1.468077in}}{\pgfqpoint{0.881492in}{1.475978in}}{\pgfqpoint{0.881492in}{1.484214in}}%
\pgfpathcurveto{\pgfqpoint{0.881492in}{1.492450in}}{\pgfqpoint{0.878220in}{1.500350in}}{\pgfqpoint{0.872396in}{1.506174in}}%
\pgfpathcurveto{\pgfqpoint{0.866572in}{1.511998in}}{\pgfqpoint{0.858672in}{1.515270in}}{\pgfqpoint{0.850435in}{1.515270in}}%
\pgfpathcurveto{\pgfqpoint{0.842199in}{1.515270in}}{\pgfqpoint{0.834299in}{1.511998in}}{\pgfqpoint{0.828475in}{1.506174in}}%
\pgfpathcurveto{\pgfqpoint{0.822651in}{1.500350in}}{\pgfqpoint{0.819379in}{1.492450in}}{\pgfqpoint{0.819379in}{1.484214in}}%
\pgfpathcurveto{\pgfqpoint{0.819379in}{1.475978in}}{\pgfqpoint{0.822651in}{1.468077in}}{\pgfqpoint{0.828475in}{1.462254in}}%
\pgfpathcurveto{\pgfqpoint{0.834299in}{1.456430in}}{\pgfqpoint{0.842199in}{1.453157in}}{\pgfqpoint{0.850435in}{1.453157in}}%
\pgfpathclose%
\pgfusepath{stroke,fill}%
\end{pgfscope}%
\begin{pgfscope}%
\pgfpathrectangle{\pgfqpoint{0.100000in}{0.212622in}}{\pgfqpoint{3.696000in}{3.696000in}}%
\pgfusepath{clip}%
\pgfsetbuttcap%
\pgfsetroundjoin%
\definecolor{currentfill}{rgb}{0.121569,0.466667,0.705882}%
\pgfsetfillcolor{currentfill}%
\pgfsetfillopacity{0.587997}%
\pgfsetlinewidth{1.003750pt}%
\definecolor{currentstroke}{rgb}{0.121569,0.466667,0.705882}%
\pgfsetstrokecolor{currentstroke}%
\pgfsetstrokeopacity{0.587997}%
\pgfsetdash{}{0pt}%
\pgfpathmoveto{\pgfqpoint{3.207598in}{2.218378in}}%
\pgfpathcurveto{\pgfqpoint{3.215834in}{2.218378in}}{\pgfqpoint{3.223734in}{2.221651in}}{\pgfqpoint{3.229558in}{2.227474in}}%
\pgfpathcurveto{\pgfqpoint{3.235382in}{2.233298in}}{\pgfqpoint{3.238655in}{2.241198in}}{\pgfqpoint{3.238655in}{2.249435in}}%
\pgfpathcurveto{\pgfqpoint{3.238655in}{2.257671in}}{\pgfqpoint{3.235382in}{2.265571in}}{\pgfqpoint{3.229558in}{2.271395in}}%
\pgfpathcurveto{\pgfqpoint{3.223734in}{2.277219in}}{\pgfqpoint{3.215834in}{2.280491in}}{\pgfqpoint{3.207598in}{2.280491in}}%
\pgfpathcurveto{\pgfqpoint{3.199362in}{2.280491in}}{\pgfqpoint{3.191462in}{2.277219in}}{\pgfqpoint{3.185638in}{2.271395in}}%
\pgfpathcurveto{\pgfqpoint{3.179814in}{2.265571in}}{\pgfqpoint{3.176542in}{2.257671in}}{\pgfqpoint{3.176542in}{2.249435in}}%
\pgfpathcurveto{\pgfqpoint{3.176542in}{2.241198in}}{\pgfqpoint{3.179814in}{2.233298in}}{\pgfqpoint{3.185638in}{2.227474in}}%
\pgfpathcurveto{\pgfqpoint{3.191462in}{2.221651in}}{\pgfqpoint{3.199362in}{2.218378in}}{\pgfqpoint{3.207598in}{2.218378in}}%
\pgfpathclose%
\pgfusepath{stroke,fill}%
\end{pgfscope}%
\begin{pgfscope}%
\pgfpathrectangle{\pgfqpoint{0.100000in}{0.212622in}}{\pgfqpoint{3.696000in}{3.696000in}}%
\pgfusepath{clip}%
\pgfsetbuttcap%
\pgfsetroundjoin%
\definecolor{currentfill}{rgb}{0.121569,0.466667,0.705882}%
\pgfsetfillcolor{currentfill}%
\pgfsetfillopacity{0.588777}%
\pgfsetlinewidth{1.003750pt}%
\definecolor{currentstroke}{rgb}{0.121569,0.466667,0.705882}%
\pgfsetstrokecolor{currentstroke}%
\pgfsetstrokeopacity{0.588777}%
\pgfsetdash{}{0pt}%
\pgfpathmoveto{\pgfqpoint{0.948252in}{1.273905in}}%
\pgfpathcurveto{\pgfqpoint{0.956488in}{1.273905in}}{\pgfqpoint{0.964388in}{1.277177in}}{\pgfqpoint{0.970212in}{1.283001in}}%
\pgfpathcurveto{\pgfqpoint{0.976036in}{1.288825in}}{\pgfqpoint{0.979308in}{1.296725in}}{\pgfqpoint{0.979308in}{1.304961in}}%
\pgfpathcurveto{\pgfqpoint{0.979308in}{1.313197in}}{\pgfqpoint{0.976036in}{1.321098in}}{\pgfqpoint{0.970212in}{1.326921in}}%
\pgfpathcurveto{\pgfqpoint{0.964388in}{1.332745in}}{\pgfqpoint{0.956488in}{1.336018in}}{\pgfqpoint{0.948252in}{1.336018in}}%
\pgfpathcurveto{\pgfqpoint{0.940015in}{1.336018in}}{\pgfqpoint{0.932115in}{1.332745in}}{\pgfqpoint{0.926291in}{1.326921in}}%
\pgfpathcurveto{\pgfqpoint{0.920467in}{1.321098in}}{\pgfqpoint{0.917195in}{1.313197in}}{\pgfqpoint{0.917195in}{1.304961in}}%
\pgfpathcurveto{\pgfqpoint{0.917195in}{1.296725in}}{\pgfqpoint{0.920467in}{1.288825in}}{\pgfqpoint{0.926291in}{1.283001in}}%
\pgfpathcurveto{\pgfqpoint{0.932115in}{1.277177in}}{\pgfqpoint{0.940015in}{1.273905in}}{\pgfqpoint{0.948252in}{1.273905in}}%
\pgfpathclose%
\pgfusepath{stroke,fill}%
\end{pgfscope}%
\begin{pgfscope}%
\pgfpathrectangle{\pgfqpoint{0.100000in}{0.212622in}}{\pgfqpoint{3.696000in}{3.696000in}}%
\pgfusepath{clip}%
\pgfsetbuttcap%
\pgfsetroundjoin%
\definecolor{currentfill}{rgb}{0.121569,0.466667,0.705882}%
\pgfsetfillcolor{currentfill}%
\pgfsetfillopacity{0.588784}%
\pgfsetlinewidth{1.003750pt}%
\definecolor{currentstroke}{rgb}{0.121569,0.466667,0.705882}%
\pgfsetstrokecolor{currentstroke}%
\pgfsetstrokeopacity{0.588784}%
\pgfsetdash{}{0pt}%
\pgfpathmoveto{\pgfqpoint{0.944313in}{1.277907in}}%
\pgfpathcurveto{\pgfqpoint{0.952549in}{1.277907in}}{\pgfqpoint{0.960449in}{1.281179in}}{\pgfqpoint{0.966273in}{1.287003in}}%
\pgfpathcurveto{\pgfqpoint{0.972097in}{1.292827in}}{\pgfqpoint{0.975369in}{1.300727in}}{\pgfqpoint{0.975369in}{1.308964in}}%
\pgfpathcurveto{\pgfqpoint{0.975369in}{1.317200in}}{\pgfqpoint{0.972097in}{1.325100in}}{\pgfqpoint{0.966273in}{1.330924in}}%
\pgfpathcurveto{\pgfqpoint{0.960449in}{1.336748in}}{\pgfqpoint{0.952549in}{1.340020in}}{\pgfqpoint{0.944313in}{1.340020in}}%
\pgfpathcurveto{\pgfqpoint{0.936076in}{1.340020in}}{\pgfqpoint{0.928176in}{1.336748in}}{\pgfqpoint{0.922352in}{1.330924in}}%
\pgfpathcurveto{\pgfqpoint{0.916529in}{1.325100in}}{\pgfqpoint{0.913256in}{1.317200in}}{\pgfqpoint{0.913256in}{1.308964in}}%
\pgfpathcurveto{\pgfqpoint{0.913256in}{1.300727in}}{\pgfqpoint{0.916529in}{1.292827in}}{\pgfqpoint{0.922352in}{1.287003in}}%
\pgfpathcurveto{\pgfqpoint{0.928176in}{1.281179in}}{\pgfqpoint{0.936076in}{1.277907in}}{\pgfqpoint{0.944313in}{1.277907in}}%
\pgfpathclose%
\pgfusepath{stroke,fill}%
\end{pgfscope}%
\begin{pgfscope}%
\pgfpathrectangle{\pgfqpoint{0.100000in}{0.212622in}}{\pgfqpoint{3.696000in}{3.696000in}}%
\pgfusepath{clip}%
\pgfsetbuttcap%
\pgfsetroundjoin%
\definecolor{currentfill}{rgb}{0.121569,0.466667,0.705882}%
\pgfsetfillcolor{currentfill}%
\pgfsetfillopacity{0.588791}%
\pgfsetlinewidth{1.003750pt}%
\definecolor{currentstroke}{rgb}{0.121569,0.466667,0.705882}%
\pgfsetstrokecolor{currentstroke}%
\pgfsetstrokeopacity{0.588791}%
\pgfsetdash{}{0pt}%
\pgfpathmoveto{\pgfqpoint{0.950950in}{1.271011in}}%
\pgfpathcurveto{\pgfqpoint{0.959187in}{1.271011in}}{\pgfqpoint{0.967087in}{1.274283in}}{\pgfqpoint{0.972911in}{1.280107in}}%
\pgfpathcurveto{\pgfqpoint{0.978735in}{1.285931in}}{\pgfqpoint{0.982007in}{1.293831in}}{\pgfqpoint{0.982007in}{1.302067in}}%
\pgfpathcurveto{\pgfqpoint{0.982007in}{1.310303in}}{\pgfqpoint{0.978735in}{1.318203in}}{\pgfqpoint{0.972911in}{1.324027in}}%
\pgfpathcurveto{\pgfqpoint{0.967087in}{1.329851in}}{\pgfqpoint{0.959187in}{1.333124in}}{\pgfqpoint{0.950950in}{1.333124in}}%
\pgfpathcurveto{\pgfqpoint{0.942714in}{1.333124in}}{\pgfqpoint{0.934814in}{1.329851in}}{\pgfqpoint{0.928990in}{1.324027in}}%
\pgfpathcurveto{\pgfqpoint{0.923166in}{1.318203in}}{\pgfqpoint{0.919894in}{1.310303in}}{\pgfqpoint{0.919894in}{1.302067in}}%
\pgfpathcurveto{\pgfqpoint{0.919894in}{1.293831in}}{\pgfqpoint{0.923166in}{1.285931in}}{\pgfqpoint{0.928990in}{1.280107in}}%
\pgfpathcurveto{\pgfqpoint{0.934814in}{1.274283in}}{\pgfqpoint{0.942714in}{1.271011in}}{\pgfqpoint{0.950950in}{1.271011in}}%
\pgfpathclose%
\pgfusepath{stroke,fill}%
\end{pgfscope}%
\begin{pgfscope}%
\pgfpathrectangle{\pgfqpoint{0.100000in}{0.212622in}}{\pgfqpoint{3.696000in}{3.696000in}}%
\pgfusepath{clip}%
\pgfsetbuttcap%
\pgfsetroundjoin%
\definecolor{currentfill}{rgb}{0.121569,0.466667,0.705882}%
\pgfsetfillcolor{currentfill}%
\pgfsetfillopacity{0.588810}%
\pgfsetlinewidth{1.003750pt}%
\definecolor{currentstroke}{rgb}{0.121569,0.466667,0.705882}%
\pgfsetstrokecolor{currentstroke}%
\pgfsetstrokeopacity{0.588810}%
\pgfsetdash{}{0pt}%
\pgfpathmoveto{\pgfqpoint{0.952296in}{1.269491in}}%
\pgfpathcurveto{\pgfqpoint{0.960532in}{1.269491in}}{\pgfqpoint{0.968433in}{1.272763in}}{\pgfqpoint{0.974256in}{1.278587in}}%
\pgfpathcurveto{\pgfqpoint{0.980080in}{1.284411in}}{\pgfqpoint{0.983353in}{1.292311in}}{\pgfqpoint{0.983353in}{1.300547in}}%
\pgfpathcurveto{\pgfqpoint{0.983353in}{1.308783in}}{\pgfqpoint{0.980080in}{1.316683in}}{\pgfqpoint{0.974256in}{1.322507in}}%
\pgfpathcurveto{\pgfqpoint{0.968433in}{1.328331in}}{\pgfqpoint{0.960532in}{1.331604in}}{\pgfqpoint{0.952296in}{1.331604in}}%
\pgfpathcurveto{\pgfqpoint{0.944060in}{1.331604in}}{\pgfqpoint{0.936160in}{1.328331in}}{\pgfqpoint{0.930336in}{1.322507in}}%
\pgfpathcurveto{\pgfqpoint{0.924512in}{1.316683in}}{\pgfqpoint{0.921240in}{1.308783in}}{\pgfqpoint{0.921240in}{1.300547in}}%
\pgfpathcurveto{\pgfqpoint{0.921240in}{1.292311in}}{\pgfqpoint{0.924512in}{1.284411in}}{\pgfqpoint{0.930336in}{1.278587in}}%
\pgfpathcurveto{\pgfqpoint{0.936160in}{1.272763in}}{\pgfqpoint{0.944060in}{1.269491in}}{\pgfqpoint{0.952296in}{1.269491in}}%
\pgfpathclose%
\pgfusepath{stroke,fill}%
\end{pgfscope}%
\begin{pgfscope}%
\pgfpathrectangle{\pgfqpoint{0.100000in}{0.212622in}}{\pgfqpoint{3.696000in}{3.696000in}}%
\pgfusepath{clip}%
\pgfsetbuttcap%
\pgfsetroundjoin%
\definecolor{currentfill}{rgb}{0.121569,0.466667,0.705882}%
\pgfsetfillcolor{currentfill}%
\pgfsetfillopacity{0.588826}%
\pgfsetlinewidth{1.003750pt}%
\definecolor{currentstroke}{rgb}{0.121569,0.466667,0.705882}%
\pgfsetstrokecolor{currentstroke}%
\pgfsetstrokeopacity{0.588826}%
\pgfsetdash{}{0pt}%
\pgfpathmoveto{\pgfqpoint{0.939412in}{1.282627in}}%
\pgfpathcurveto{\pgfqpoint{0.947648in}{1.282627in}}{\pgfqpoint{0.955548in}{1.285900in}}{\pgfqpoint{0.961372in}{1.291723in}}%
\pgfpathcurveto{\pgfqpoint{0.967196in}{1.297547in}}{\pgfqpoint{0.970468in}{1.305447in}}{\pgfqpoint{0.970468in}{1.313684in}}%
\pgfpathcurveto{\pgfqpoint{0.970468in}{1.321920in}}{\pgfqpoint{0.967196in}{1.329820in}}{\pgfqpoint{0.961372in}{1.335644in}}%
\pgfpathcurveto{\pgfqpoint{0.955548in}{1.341468in}}{\pgfqpoint{0.947648in}{1.344740in}}{\pgfqpoint{0.939412in}{1.344740in}}%
\pgfpathcurveto{\pgfqpoint{0.931176in}{1.344740in}}{\pgfqpoint{0.923276in}{1.341468in}}{\pgfqpoint{0.917452in}{1.335644in}}%
\pgfpathcurveto{\pgfqpoint{0.911628in}{1.329820in}}{\pgfqpoint{0.908355in}{1.321920in}}{\pgfqpoint{0.908355in}{1.313684in}}%
\pgfpathcurveto{\pgfqpoint{0.908355in}{1.305447in}}{\pgfqpoint{0.911628in}{1.297547in}}{\pgfqpoint{0.917452in}{1.291723in}}%
\pgfpathcurveto{\pgfqpoint{0.923276in}{1.285900in}}{\pgfqpoint{0.931176in}{1.282627in}}{\pgfqpoint{0.939412in}{1.282627in}}%
\pgfpathclose%
\pgfusepath{stroke,fill}%
\end{pgfscope}%
\begin{pgfscope}%
\pgfpathrectangle{\pgfqpoint{0.100000in}{0.212622in}}{\pgfqpoint{3.696000in}{3.696000in}}%
\pgfusepath{clip}%
\pgfsetbuttcap%
\pgfsetroundjoin%
\definecolor{currentfill}{rgb}{0.121569,0.466667,0.705882}%
\pgfsetfillcolor{currentfill}%
\pgfsetfillopacity{0.588868}%
\pgfsetlinewidth{1.003750pt}%
\definecolor{currentstroke}{rgb}{0.121569,0.466667,0.705882}%
\pgfsetstrokecolor{currentstroke}%
\pgfsetstrokeopacity{0.588868}%
\pgfsetdash{}{0pt}%
\pgfpathmoveto{\pgfqpoint{0.936666in}{1.285127in}}%
\pgfpathcurveto{\pgfqpoint{0.944902in}{1.285127in}}{\pgfqpoint{0.952802in}{1.288400in}}{\pgfqpoint{0.958626in}{1.294224in}}%
\pgfpathcurveto{\pgfqpoint{0.964450in}{1.300047in}}{\pgfqpoint{0.967722in}{1.307948in}}{\pgfqpoint{0.967722in}{1.316184in}}%
\pgfpathcurveto{\pgfqpoint{0.967722in}{1.324420in}}{\pgfqpoint{0.964450in}{1.332320in}}{\pgfqpoint{0.958626in}{1.338144in}}%
\pgfpathcurveto{\pgfqpoint{0.952802in}{1.343968in}}{\pgfqpoint{0.944902in}{1.347240in}}{\pgfqpoint{0.936666in}{1.347240in}}%
\pgfpathcurveto{\pgfqpoint{0.928429in}{1.347240in}}{\pgfqpoint{0.920529in}{1.343968in}}{\pgfqpoint{0.914705in}{1.338144in}}%
\pgfpathcurveto{\pgfqpoint{0.908881in}{1.332320in}}{\pgfqpoint{0.905609in}{1.324420in}}{\pgfqpoint{0.905609in}{1.316184in}}%
\pgfpathcurveto{\pgfqpoint{0.905609in}{1.307948in}}{\pgfqpoint{0.908881in}{1.300047in}}{\pgfqpoint{0.914705in}{1.294224in}}%
\pgfpathcurveto{\pgfqpoint{0.920529in}{1.288400in}}{\pgfqpoint{0.928429in}{1.285127in}}{\pgfqpoint{0.936666in}{1.285127in}}%
\pgfpathclose%
\pgfusepath{stroke,fill}%
\end{pgfscope}%
\begin{pgfscope}%
\pgfpathrectangle{\pgfqpoint{0.100000in}{0.212622in}}{\pgfqpoint{3.696000in}{3.696000in}}%
\pgfusepath{clip}%
\pgfsetbuttcap%
\pgfsetroundjoin%
\definecolor{currentfill}{rgb}{0.121569,0.466667,0.705882}%
\pgfsetfillcolor{currentfill}%
\pgfsetfillopacity{0.588901}%
\pgfsetlinewidth{1.003750pt}%
\definecolor{currentstroke}{rgb}{0.121569,0.466667,0.705882}%
\pgfsetstrokecolor{currentstroke}%
\pgfsetstrokeopacity{0.588901}%
\pgfsetdash{}{0pt}%
\pgfpathmoveto{\pgfqpoint{0.935130in}{1.286446in}}%
\pgfpathcurveto{\pgfqpoint{0.943366in}{1.286446in}}{\pgfqpoint{0.951266in}{1.289718in}}{\pgfqpoint{0.957090in}{1.295542in}}%
\pgfpathcurveto{\pgfqpoint{0.962914in}{1.301366in}}{\pgfqpoint{0.966186in}{1.309266in}}{\pgfqpoint{0.966186in}{1.317502in}}%
\pgfpathcurveto{\pgfqpoint{0.966186in}{1.325738in}}{\pgfqpoint{0.962914in}{1.333638in}}{\pgfqpoint{0.957090in}{1.339462in}}%
\pgfpathcurveto{\pgfqpoint{0.951266in}{1.345286in}}{\pgfqpoint{0.943366in}{1.348559in}}{\pgfqpoint{0.935130in}{1.348559in}}%
\pgfpathcurveto{\pgfqpoint{0.926893in}{1.348559in}}{\pgfqpoint{0.918993in}{1.345286in}}{\pgfqpoint{0.913170in}{1.339462in}}%
\pgfpathcurveto{\pgfqpoint{0.907346in}{1.333638in}}{\pgfqpoint{0.904073in}{1.325738in}}{\pgfqpoint{0.904073in}{1.317502in}}%
\pgfpathcurveto{\pgfqpoint{0.904073in}{1.309266in}}{\pgfqpoint{0.907346in}{1.301366in}}{\pgfqpoint{0.913170in}{1.295542in}}%
\pgfpathcurveto{\pgfqpoint{0.918993in}{1.289718in}}{\pgfqpoint{0.926893in}{1.286446in}}{\pgfqpoint{0.935130in}{1.286446in}}%
\pgfpathclose%
\pgfusepath{stroke,fill}%
\end{pgfscope}%
\begin{pgfscope}%
\pgfpathrectangle{\pgfqpoint{0.100000in}{0.212622in}}{\pgfqpoint{3.696000in}{3.696000in}}%
\pgfusepath{clip}%
\pgfsetbuttcap%
\pgfsetroundjoin%
\definecolor{currentfill}{rgb}{0.121569,0.466667,0.705882}%
\pgfsetfillcolor{currentfill}%
\pgfsetfillopacity{0.588914}%
\pgfsetlinewidth{1.003750pt}%
\definecolor{currentstroke}{rgb}{0.121569,0.466667,0.705882}%
\pgfsetstrokecolor{currentstroke}%
\pgfsetstrokeopacity{0.588914}%
\pgfsetdash{}{0pt}%
\pgfpathmoveto{\pgfqpoint{3.205869in}{2.217873in}}%
\pgfpathcurveto{\pgfqpoint{3.214106in}{2.217873in}}{\pgfqpoint{3.222006in}{2.221145in}}{\pgfqpoint{3.227830in}{2.226969in}}%
\pgfpathcurveto{\pgfqpoint{3.233654in}{2.232793in}}{\pgfqpoint{3.236926in}{2.240693in}}{\pgfqpoint{3.236926in}{2.248929in}}%
\pgfpathcurveto{\pgfqpoint{3.236926in}{2.257165in}}{\pgfqpoint{3.233654in}{2.265065in}}{\pgfqpoint{3.227830in}{2.270889in}}%
\pgfpathcurveto{\pgfqpoint{3.222006in}{2.276713in}}{\pgfqpoint{3.214106in}{2.279986in}}{\pgfqpoint{3.205869in}{2.279986in}}%
\pgfpathcurveto{\pgfqpoint{3.197633in}{2.279986in}}{\pgfqpoint{3.189733in}{2.276713in}}{\pgfqpoint{3.183909in}{2.270889in}}%
\pgfpathcurveto{\pgfqpoint{3.178085in}{2.265065in}}{\pgfqpoint{3.174813in}{2.257165in}}{\pgfqpoint{3.174813in}{2.248929in}}%
\pgfpathcurveto{\pgfqpoint{3.174813in}{2.240693in}}{\pgfqpoint{3.178085in}{2.232793in}}{\pgfqpoint{3.183909in}{2.226969in}}%
\pgfpathcurveto{\pgfqpoint{3.189733in}{2.221145in}}{\pgfqpoint{3.197633in}{2.217873in}}{\pgfqpoint{3.205869in}{2.217873in}}%
\pgfpathclose%
\pgfusepath{stroke,fill}%
\end{pgfscope}%
\begin{pgfscope}%
\pgfpathrectangle{\pgfqpoint{0.100000in}{0.212622in}}{\pgfqpoint{3.696000in}{3.696000in}}%
\pgfusepath{clip}%
\pgfsetbuttcap%
\pgfsetroundjoin%
\definecolor{currentfill}{rgb}{0.121569,0.466667,0.705882}%
\pgfsetfillcolor{currentfill}%
\pgfsetfillopacity{0.588925}%
\pgfsetlinewidth{1.003750pt}%
\definecolor{currentstroke}{rgb}{0.121569,0.466667,0.705882}%
\pgfsetstrokecolor{currentstroke}%
\pgfsetstrokeopacity{0.588925}%
\pgfsetdash{}{0pt}%
\pgfpathmoveto{\pgfqpoint{0.934272in}{1.287137in}}%
\pgfpathcurveto{\pgfqpoint{0.942508in}{1.287137in}}{\pgfqpoint{0.950408in}{1.290410in}}{\pgfqpoint{0.956232in}{1.296234in}}%
\pgfpathcurveto{\pgfqpoint{0.962056in}{1.302058in}}{\pgfqpoint{0.965328in}{1.309958in}}{\pgfqpoint{0.965328in}{1.318194in}}%
\pgfpathcurveto{\pgfqpoint{0.965328in}{1.326430in}}{\pgfqpoint{0.962056in}{1.334330in}}{\pgfqpoint{0.956232in}{1.340154in}}%
\pgfpathcurveto{\pgfqpoint{0.950408in}{1.345978in}}{\pgfqpoint{0.942508in}{1.349250in}}{\pgfqpoint{0.934272in}{1.349250in}}%
\pgfpathcurveto{\pgfqpoint{0.926035in}{1.349250in}}{\pgfqpoint{0.918135in}{1.345978in}}{\pgfqpoint{0.912311in}{1.340154in}}%
\pgfpathcurveto{\pgfqpoint{0.906488in}{1.334330in}}{\pgfqpoint{0.903215in}{1.326430in}}{\pgfqpoint{0.903215in}{1.318194in}}%
\pgfpathcurveto{\pgfqpoint{0.903215in}{1.309958in}}{\pgfqpoint{0.906488in}{1.302058in}}{\pgfqpoint{0.912311in}{1.296234in}}%
\pgfpathcurveto{\pgfqpoint{0.918135in}{1.290410in}}{\pgfqpoint{0.926035in}{1.287137in}}{\pgfqpoint{0.934272in}{1.287137in}}%
\pgfpathclose%
\pgfusepath{stroke,fill}%
\end{pgfscope}%
\begin{pgfscope}%
\pgfpathrectangle{\pgfqpoint{0.100000in}{0.212622in}}{\pgfqpoint{3.696000in}{3.696000in}}%
\pgfusepath{clip}%
\pgfsetbuttcap%
\pgfsetroundjoin%
\definecolor{currentfill}{rgb}{0.121569,0.466667,0.705882}%
\pgfsetfillcolor{currentfill}%
\pgfsetfillopacity{0.588942}%
\pgfsetlinewidth{1.003750pt}%
\definecolor{currentstroke}{rgb}{0.121569,0.466667,0.705882}%
\pgfsetstrokecolor{currentstroke}%
\pgfsetstrokeopacity{0.588942}%
\pgfsetdash{}{0pt}%
\pgfpathmoveto{\pgfqpoint{0.933793in}{1.287501in}}%
\pgfpathcurveto{\pgfqpoint{0.942029in}{1.287501in}}{\pgfqpoint{0.949930in}{1.290773in}}{\pgfqpoint{0.955753in}{1.296597in}}%
\pgfpathcurveto{\pgfqpoint{0.961577in}{1.302421in}}{\pgfqpoint{0.964850in}{1.310321in}}{\pgfqpoint{0.964850in}{1.318558in}}%
\pgfpathcurveto{\pgfqpoint{0.964850in}{1.326794in}}{\pgfqpoint{0.961577in}{1.334694in}}{\pgfqpoint{0.955753in}{1.340518in}}%
\pgfpathcurveto{\pgfqpoint{0.949930in}{1.346342in}}{\pgfqpoint{0.942029in}{1.349614in}}{\pgfqpoint{0.933793in}{1.349614in}}%
\pgfpathcurveto{\pgfqpoint{0.925557in}{1.349614in}}{\pgfqpoint{0.917657in}{1.346342in}}{\pgfqpoint{0.911833in}{1.340518in}}%
\pgfpathcurveto{\pgfqpoint{0.906009in}{1.334694in}}{\pgfqpoint{0.902737in}{1.326794in}}{\pgfqpoint{0.902737in}{1.318558in}}%
\pgfpathcurveto{\pgfqpoint{0.902737in}{1.310321in}}{\pgfqpoint{0.906009in}{1.302421in}}{\pgfqpoint{0.911833in}{1.296597in}}%
\pgfpathcurveto{\pgfqpoint{0.917657in}{1.290773in}}{\pgfqpoint{0.925557in}{1.287501in}}{\pgfqpoint{0.933793in}{1.287501in}}%
\pgfpathclose%
\pgfusepath{stroke,fill}%
\end{pgfscope}%
\begin{pgfscope}%
\pgfpathrectangle{\pgfqpoint{0.100000in}{0.212622in}}{\pgfqpoint{3.696000in}{3.696000in}}%
\pgfusepath{clip}%
\pgfsetbuttcap%
\pgfsetroundjoin%
\definecolor{currentfill}{rgb}{0.121569,0.466667,0.705882}%
\pgfsetfillcolor{currentfill}%
\pgfsetfillopacity{0.588953}%
\pgfsetlinewidth{1.003750pt}%
\definecolor{currentstroke}{rgb}{0.121569,0.466667,0.705882}%
\pgfsetstrokecolor{currentstroke}%
\pgfsetstrokeopacity{0.588953}%
\pgfsetdash{}{0pt}%
\pgfpathmoveto{\pgfqpoint{0.933527in}{1.287691in}}%
\pgfpathcurveto{\pgfqpoint{0.941763in}{1.287691in}}{\pgfqpoint{0.949663in}{1.290963in}}{\pgfqpoint{0.955487in}{1.296787in}}%
\pgfpathcurveto{\pgfqpoint{0.961311in}{1.302611in}}{\pgfqpoint{0.964583in}{1.310511in}}{\pgfqpoint{0.964583in}{1.318747in}}%
\pgfpathcurveto{\pgfqpoint{0.964583in}{1.326984in}}{\pgfqpoint{0.961311in}{1.334884in}}{\pgfqpoint{0.955487in}{1.340708in}}%
\pgfpathcurveto{\pgfqpoint{0.949663in}{1.346531in}}{\pgfqpoint{0.941763in}{1.349804in}}{\pgfqpoint{0.933527in}{1.349804in}}%
\pgfpathcurveto{\pgfqpoint{0.925291in}{1.349804in}}{\pgfqpoint{0.917391in}{1.346531in}}{\pgfqpoint{0.911567in}{1.340708in}}%
\pgfpathcurveto{\pgfqpoint{0.905743in}{1.334884in}}{\pgfqpoint{0.902470in}{1.326984in}}{\pgfqpoint{0.902470in}{1.318747in}}%
\pgfpathcurveto{\pgfqpoint{0.902470in}{1.310511in}}{\pgfqpoint{0.905743in}{1.302611in}}{\pgfqpoint{0.911567in}{1.296787in}}%
\pgfpathcurveto{\pgfqpoint{0.917391in}{1.290963in}}{\pgfqpoint{0.925291in}{1.287691in}}{\pgfqpoint{0.933527in}{1.287691in}}%
\pgfpathclose%
\pgfusepath{stroke,fill}%
\end{pgfscope}%
\begin{pgfscope}%
\pgfpathrectangle{\pgfqpoint{0.100000in}{0.212622in}}{\pgfqpoint{3.696000in}{3.696000in}}%
\pgfusepath{clip}%
\pgfsetbuttcap%
\pgfsetroundjoin%
\definecolor{currentfill}{rgb}{0.121569,0.466667,0.705882}%
\pgfsetfillcolor{currentfill}%
\pgfsetfillopacity{0.588961}%
\pgfsetlinewidth{1.003750pt}%
\definecolor{currentstroke}{rgb}{0.121569,0.466667,0.705882}%
\pgfsetstrokecolor{currentstroke}%
\pgfsetstrokeopacity{0.588961}%
\pgfsetdash{}{0pt}%
\pgfpathmoveto{\pgfqpoint{0.933379in}{1.287789in}}%
\pgfpathcurveto{\pgfqpoint{0.941615in}{1.287789in}}{\pgfqpoint{0.949515in}{1.291061in}}{\pgfqpoint{0.955339in}{1.296885in}}%
\pgfpathcurveto{\pgfqpoint{0.961163in}{1.302709in}}{\pgfqpoint{0.964435in}{1.310609in}}{\pgfqpoint{0.964435in}{1.318845in}}%
\pgfpathcurveto{\pgfqpoint{0.964435in}{1.327082in}}{\pgfqpoint{0.961163in}{1.334982in}}{\pgfqpoint{0.955339in}{1.340806in}}%
\pgfpathcurveto{\pgfqpoint{0.949515in}{1.346630in}}{\pgfqpoint{0.941615in}{1.349902in}}{\pgfqpoint{0.933379in}{1.349902in}}%
\pgfpathcurveto{\pgfqpoint{0.925143in}{1.349902in}}{\pgfqpoint{0.917243in}{1.346630in}}{\pgfqpoint{0.911419in}{1.340806in}}%
\pgfpathcurveto{\pgfqpoint{0.905595in}{1.334982in}}{\pgfqpoint{0.902322in}{1.327082in}}{\pgfqpoint{0.902322in}{1.318845in}}%
\pgfpathcurveto{\pgfqpoint{0.902322in}{1.310609in}}{\pgfqpoint{0.905595in}{1.302709in}}{\pgfqpoint{0.911419in}{1.296885in}}%
\pgfpathcurveto{\pgfqpoint{0.917243in}{1.291061in}}{\pgfqpoint{0.925143in}{1.287789in}}{\pgfqpoint{0.933379in}{1.287789in}}%
\pgfpathclose%
\pgfusepath{stroke,fill}%
\end{pgfscope}%
\begin{pgfscope}%
\pgfpathrectangle{\pgfqpoint{0.100000in}{0.212622in}}{\pgfqpoint{3.696000in}{3.696000in}}%
\pgfusepath{clip}%
\pgfsetbuttcap%
\pgfsetroundjoin%
\definecolor{currentfill}{rgb}{0.121569,0.466667,0.705882}%
\pgfsetfillcolor{currentfill}%
\pgfsetfillopacity{0.588965}%
\pgfsetlinewidth{1.003750pt}%
\definecolor{currentstroke}{rgb}{0.121569,0.466667,0.705882}%
\pgfsetstrokecolor{currentstroke}%
\pgfsetstrokeopacity{0.588965}%
\pgfsetdash{}{0pt}%
\pgfpathmoveto{\pgfqpoint{0.933297in}{1.287839in}}%
\pgfpathcurveto{\pgfqpoint{0.941533in}{1.287839in}}{\pgfqpoint{0.949433in}{1.291112in}}{\pgfqpoint{0.955257in}{1.296936in}}%
\pgfpathcurveto{\pgfqpoint{0.961081in}{1.302759in}}{\pgfqpoint{0.964353in}{1.310660in}}{\pgfqpoint{0.964353in}{1.318896in}}%
\pgfpathcurveto{\pgfqpoint{0.964353in}{1.327132in}}{\pgfqpoint{0.961081in}{1.335032in}}{\pgfqpoint{0.955257in}{1.340856in}}%
\pgfpathcurveto{\pgfqpoint{0.949433in}{1.346680in}}{\pgfqpoint{0.941533in}{1.349952in}}{\pgfqpoint{0.933297in}{1.349952in}}%
\pgfpathcurveto{\pgfqpoint{0.925061in}{1.349952in}}{\pgfqpoint{0.917161in}{1.346680in}}{\pgfqpoint{0.911337in}{1.340856in}}%
\pgfpathcurveto{\pgfqpoint{0.905513in}{1.335032in}}{\pgfqpoint{0.902240in}{1.327132in}}{\pgfqpoint{0.902240in}{1.318896in}}%
\pgfpathcurveto{\pgfqpoint{0.902240in}{1.310660in}}{\pgfqpoint{0.905513in}{1.302759in}}{\pgfqpoint{0.911337in}{1.296936in}}%
\pgfpathcurveto{\pgfqpoint{0.917161in}{1.291112in}}{\pgfqpoint{0.925061in}{1.287839in}}{\pgfqpoint{0.933297in}{1.287839in}}%
\pgfpathclose%
\pgfusepath{stroke,fill}%
\end{pgfscope}%
\begin{pgfscope}%
\pgfpathrectangle{\pgfqpoint{0.100000in}{0.212622in}}{\pgfqpoint{3.696000in}{3.696000in}}%
\pgfusepath{clip}%
\pgfsetbuttcap%
\pgfsetroundjoin%
\definecolor{currentfill}{rgb}{0.121569,0.466667,0.705882}%
\pgfsetfillcolor{currentfill}%
\pgfsetfillopacity{0.588968}%
\pgfsetlinewidth{1.003750pt}%
\definecolor{currentstroke}{rgb}{0.121569,0.466667,0.705882}%
\pgfsetstrokecolor{currentstroke}%
\pgfsetstrokeopacity{0.588968}%
\pgfsetdash{}{0pt}%
\pgfpathmoveto{\pgfqpoint{0.933252in}{1.287865in}}%
\pgfpathcurveto{\pgfqpoint{0.941488in}{1.287865in}}{\pgfqpoint{0.949388in}{1.291138in}}{\pgfqpoint{0.955212in}{1.296961in}}%
\pgfpathcurveto{\pgfqpoint{0.961036in}{1.302785in}}{\pgfqpoint{0.964308in}{1.310685in}}{\pgfqpoint{0.964308in}{1.318922in}}%
\pgfpathcurveto{\pgfqpoint{0.964308in}{1.327158in}}{\pgfqpoint{0.961036in}{1.335058in}}{\pgfqpoint{0.955212in}{1.340882in}}%
\pgfpathcurveto{\pgfqpoint{0.949388in}{1.346706in}}{\pgfqpoint{0.941488in}{1.349978in}}{\pgfqpoint{0.933252in}{1.349978in}}%
\pgfpathcurveto{\pgfqpoint{0.925015in}{1.349978in}}{\pgfqpoint{0.917115in}{1.346706in}}{\pgfqpoint{0.911291in}{1.340882in}}%
\pgfpathcurveto{\pgfqpoint{0.905467in}{1.335058in}}{\pgfqpoint{0.902195in}{1.327158in}}{\pgfqpoint{0.902195in}{1.318922in}}%
\pgfpathcurveto{\pgfqpoint{0.902195in}{1.310685in}}{\pgfqpoint{0.905467in}{1.302785in}}{\pgfqpoint{0.911291in}{1.296961in}}%
\pgfpathcurveto{\pgfqpoint{0.917115in}{1.291138in}}{\pgfqpoint{0.925015in}{1.287865in}}{\pgfqpoint{0.933252in}{1.287865in}}%
\pgfpathclose%
\pgfusepath{stroke,fill}%
\end{pgfscope}%
\begin{pgfscope}%
\pgfpathrectangle{\pgfqpoint{0.100000in}{0.212622in}}{\pgfqpoint{3.696000in}{3.696000in}}%
\pgfusepath{clip}%
\pgfsetbuttcap%
\pgfsetroundjoin%
\definecolor{currentfill}{rgb}{0.121569,0.466667,0.705882}%
\pgfsetfillcolor{currentfill}%
\pgfsetfillopacity{0.588969}%
\pgfsetlinewidth{1.003750pt}%
\definecolor{currentstroke}{rgb}{0.121569,0.466667,0.705882}%
\pgfsetstrokecolor{currentstroke}%
\pgfsetstrokeopacity{0.588969}%
\pgfsetdash{}{0pt}%
\pgfpathmoveto{\pgfqpoint{0.933226in}{1.287879in}}%
\pgfpathcurveto{\pgfqpoint{0.941463in}{1.287879in}}{\pgfqpoint{0.949363in}{1.291151in}}{\pgfqpoint{0.955187in}{1.296975in}}%
\pgfpathcurveto{\pgfqpoint{0.961011in}{1.302799in}}{\pgfqpoint{0.964283in}{1.310699in}}{\pgfqpoint{0.964283in}{1.318935in}}%
\pgfpathcurveto{\pgfqpoint{0.964283in}{1.327171in}}{\pgfqpoint{0.961011in}{1.335071in}}{\pgfqpoint{0.955187in}{1.340895in}}%
\pgfpathcurveto{\pgfqpoint{0.949363in}{1.346719in}}{\pgfqpoint{0.941463in}{1.349992in}}{\pgfqpoint{0.933226in}{1.349992in}}%
\pgfpathcurveto{\pgfqpoint{0.924990in}{1.349992in}}{\pgfqpoint{0.917090in}{1.346719in}}{\pgfqpoint{0.911266in}{1.340895in}}%
\pgfpathcurveto{\pgfqpoint{0.905442in}{1.335071in}}{\pgfqpoint{0.902170in}{1.327171in}}{\pgfqpoint{0.902170in}{1.318935in}}%
\pgfpathcurveto{\pgfqpoint{0.902170in}{1.310699in}}{\pgfqpoint{0.905442in}{1.302799in}}{\pgfqpoint{0.911266in}{1.296975in}}%
\pgfpathcurveto{\pgfqpoint{0.917090in}{1.291151in}}{\pgfqpoint{0.924990in}{1.287879in}}{\pgfqpoint{0.933226in}{1.287879in}}%
\pgfpathclose%
\pgfusepath{stroke,fill}%
\end{pgfscope}%
\begin{pgfscope}%
\pgfpathrectangle{\pgfqpoint{0.100000in}{0.212622in}}{\pgfqpoint{3.696000in}{3.696000in}}%
\pgfusepath{clip}%
\pgfsetbuttcap%
\pgfsetroundjoin%
\definecolor{currentfill}{rgb}{0.121569,0.466667,0.705882}%
\pgfsetfillcolor{currentfill}%
\pgfsetfillopacity{0.588970}%
\pgfsetlinewidth{1.003750pt}%
\definecolor{currentstroke}{rgb}{0.121569,0.466667,0.705882}%
\pgfsetstrokecolor{currentstroke}%
\pgfsetstrokeopacity{0.588970}%
\pgfsetdash{}{0pt}%
\pgfpathmoveto{\pgfqpoint{0.933213in}{1.287885in}}%
\pgfpathcurveto{\pgfqpoint{0.941449in}{1.287885in}}{\pgfqpoint{0.949349in}{1.291158in}}{\pgfqpoint{0.955173in}{1.296981in}}%
\pgfpathcurveto{\pgfqpoint{0.960997in}{1.302805in}}{\pgfqpoint{0.964269in}{1.310705in}}{\pgfqpoint{0.964269in}{1.318942in}}%
\pgfpathcurveto{\pgfqpoint{0.964269in}{1.327178in}}{\pgfqpoint{0.960997in}{1.335078in}}{\pgfqpoint{0.955173in}{1.340902in}}%
\pgfpathcurveto{\pgfqpoint{0.949349in}{1.346726in}}{\pgfqpoint{0.941449in}{1.349998in}}{\pgfqpoint{0.933213in}{1.349998in}}%
\pgfpathcurveto{\pgfqpoint{0.924976in}{1.349998in}}{\pgfqpoint{0.917076in}{1.346726in}}{\pgfqpoint{0.911252in}{1.340902in}}%
\pgfpathcurveto{\pgfqpoint{0.905428in}{1.335078in}}{\pgfqpoint{0.902156in}{1.327178in}}{\pgfqpoint{0.902156in}{1.318942in}}%
\pgfpathcurveto{\pgfqpoint{0.902156in}{1.310705in}}{\pgfqpoint{0.905428in}{1.302805in}}{\pgfqpoint{0.911252in}{1.296981in}}%
\pgfpathcurveto{\pgfqpoint{0.917076in}{1.291158in}}{\pgfqpoint{0.924976in}{1.287885in}}{\pgfqpoint{0.933213in}{1.287885in}}%
\pgfpathclose%
\pgfusepath{stroke,fill}%
\end{pgfscope}%
\begin{pgfscope}%
\pgfpathrectangle{\pgfqpoint{0.100000in}{0.212622in}}{\pgfqpoint{3.696000in}{3.696000in}}%
\pgfusepath{clip}%
\pgfsetbuttcap%
\pgfsetroundjoin%
\definecolor{currentfill}{rgb}{0.121569,0.466667,0.705882}%
\pgfsetfillcolor{currentfill}%
\pgfsetfillopacity{0.588971}%
\pgfsetlinewidth{1.003750pt}%
\definecolor{currentstroke}{rgb}{0.121569,0.466667,0.705882}%
\pgfsetstrokecolor{currentstroke}%
\pgfsetstrokeopacity{0.588971}%
\pgfsetdash{}{0pt}%
\pgfpathmoveto{\pgfqpoint{0.933205in}{1.287889in}}%
\pgfpathcurveto{\pgfqpoint{0.941441in}{1.287889in}}{\pgfqpoint{0.949341in}{1.291161in}}{\pgfqpoint{0.955165in}{1.296985in}}%
\pgfpathcurveto{\pgfqpoint{0.960989in}{1.302809in}}{\pgfqpoint{0.964262in}{1.310709in}}{\pgfqpoint{0.964262in}{1.318945in}}%
\pgfpathcurveto{\pgfqpoint{0.964262in}{1.327181in}}{\pgfqpoint{0.960989in}{1.335081in}}{\pgfqpoint{0.955165in}{1.340905in}}%
\pgfpathcurveto{\pgfqpoint{0.949341in}{1.346729in}}{\pgfqpoint{0.941441in}{1.350002in}}{\pgfqpoint{0.933205in}{1.350002in}}%
\pgfpathcurveto{\pgfqpoint{0.924969in}{1.350002in}}{\pgfqpoint{0.917069in}{1.346729in}}{\pgfqpoint{0.911245in}{1.340905in}}%
\pgfpathcurveto{\pgfqpoint{0.905421in}{1.335081in}}{\pgfqpoint{0.902149in}{1.327181in}}{\pgfqpoint{0.902149in}{1.318945in}}%
\pgfpathcurveto{\pgfqpoint{0.902149in}{1.310709in}}{\pgfqpoint{0.905421in}{1.302809in}}{\pgfqpoint{0.911245in}{1.296985in}}%
\pgfpathcurveto{\pgfqpoint{0.917069in}{1.291161in}}{\pgfqpoint{0.924969in}{1.287889in}}{\pgfqpoint{0.933205in}{1.287889in}}%
\pgfpathclose%
\pgfusepath{stroke,fill}%
\end{pgfscope}%
\begin{pgfscope}%
\pgfpathrectangle{\pgfqpoint{0.100000in}{0.212622in}}{\pgfqpoint{3.696000in}{3.696000in}}%
\pgfusepath{clip}%
\pgfsetbuttcap%
\pgfsetroundjoin%
\definecolor{currentfill}{rgb}{0.121569,0.466667,0.705882}%
\pgfsetfillcolor{currentfill}%
\pgfsetfillopacity{0.588971}%
\pgfsetlinewidth{1.003750pt}%
\definecolor{currentstroke}{rgb}{0.121569,0.466667,0.705882}%
\pgfsetstrokecolor{currentstroke}%
\pgfsetstrokeopacity{0.588971}%
\pgfsetdash{}{0pt}%
\pgfpathmoveto{\pgfqpoint{0.933201in}{1.287890in}}%
\pgfpathcurveto{\pgfqpoint{0.941437in}{1.287890in}}{\pgfqpoint{0.949337in}{1.291163in}}{\pgfqpoint{0.955161in}{1.296987in}}%
\pgfpathcurveto{\pgfqpoint{0.960985in}{1.302811in}}{\pgfqpoint{0.964257in}{1.310711in}}{\pgfqpoint{0.964257in}{1.318947in}}%
\pgfpathcurveto{\pgfqpoint{0.964257in}{1.327183in}}{\pgfqpoint{0.960985in}{1.335083in}}{\pgfqpoint{0.955161in}{1.340907in}}%
\pgfpathcurveto{\pgfqpoint{0.949337in}{1.346731in}}{\pgfqpoint{0.941437in}{1.350003in}}{\pgfqpoint{0.933201in}{1.350003in}}%
\pgfpathcurveto{\pgfqpoint{0.924965in}{1.350003in}}{\pgfqpoint{0.917065in}{1.346731in}}{\pgfqpoint{0.911241in}{1.340907in}}%
\pgfpathcurveto{\pgfqpoint{0.905417in}{1.335083in}}{\pgfqpoint{0.902144in}{1.327183in}}{\pgfqpoint{0.902144in}{1.318947in}}%
\pgfpathcurveto{\pgfqpoint{0.902144in}{1.310711in}}{\pgfqpoint{0.905417in}{1.302811in}}{\pgfqpoint{0.911241in}{1.296987in}}%
\pgfpathcurveto{\pgfqpoint{0.917065in}{1.291163in}}{\pgfqpoint{0.924965in}{1.287890in}}{\pgfqpoint{0.933201in}{1.287890in}}%
\pgfpathclose%
\pgfusepath{stroke,fill}%
\end{pgfscope}%
\begin{pgfscope}%
\pgfpathrectangle{\pgfqpoint{0.100000in}{0.212622in}}{\pgfqpoint{3.696000in}{3.696000in}}%
\pgfusepath{clip}%
\pgfsetbuttcap%
\pgfsetroundjoin%
\definecolor{currentfill}{rgb}{0.121569,0.466667,0.705882}%
\pgfsetfillcolor{currentfill}%
\pgfsetfillopacity{0.588972}%
\pgfsetlinewidth{1.003750pt}%
\definecolor{currentstroke}{rgb}{0.121569,0.466667,0.705882}%
\pgfsetstrokecolor{currentstroke}%
\pgfsetstrokeopacity{0.588972}%
\pgfsetdash{}{0pt}%
\pgfpathmoveto{\pgfqpoint{0.933199in}{1.287891in}}%
\pgfpathcurveto{\pgfqpoint{0.941435in}{1.287891in}}{\pgfqpoint{0.949335in}{1.291164in}}{\pgfqpoint{0.955159in}{1.296987in}}%
\pgfpathcurveto{\pgfqpoint{0.960983in}{1.302811in}}{\pgfqpoint{0.964255in}{1.310711in}}{\pgfqpoint{0.964255in}{1.318948in}}%
\pgfpathcurveto{\pgfqpoint{0.964255in}{1.327184in}}{\pgfqpoint{0.960983in}{1.335084in}}{\pgfqpoint{0.955159in}{1.340908in}}%
\pgfpathcurveto{\pgfqpoint{0.949335in}{1.346732in}}{\pgfqpoint{0.941435in}{1.350004in}}{\pgfqpoint{0.933199in}{1.350004in}}%
\pgfpathcurveto{\pgfqpoint{0.924962in}{1.350004in}}{\pgfqpoint{0.917062in}{1.346732in}}{\pgfqpoint{0.911238in}{1.340908in}}%
\pgfpathcurveto{\pgfqpoint{0.905414in}{1.335084in}}{\pgfqpoint{0.902142in}{1.327184in}}{\pgfqpoint{0.902142in}{1.318948in}}%
\pgfpathcurveto{\pgfqpoint{0.902142in}{1.310711in}}{\pgfqpoint{0.905414in}{1.302811in}}{\pgfqpoint{0.911238in}{1.296987in}}%
\pgfpathcurveto{\pgfqpoint{0.917062in}{1.291164in}}{\pgfqpoint{0.924962in}{1.287891in}}{\pgfqpoint{0.933199in}{1.287891in}}%
\pgfpathclose%
\pgfusepath{stroke,fill}%
\end{pgfscope}%
\begin{pgfscope}%
\pgfpathrectangle{\pgfqpoint{0.100000in}{0.212622in}}{\pgfqpoint{3.696000in}{3.696000in}}%
\pgfusepath{clip}%
\pgfsetbuttcap%
\pgfsetroundjoin%
\definecolor{currentfill}{rgb}{0.121569,0.466667,0.705882}%
\pgfsetfillcolor{currentfill}%
\pgfsetfillopacity{0.588972}%
\pgfsetlinewidth{1.003750pt}%
\definecolor{currentstroke}{rgb}{0.121569,0.466667,0.705882}%
\pgfsetstrokecolor{currentstroke}%
\pgfsetstrokeopacity{0.588972}%
\pgfsetdash{}{0pt}%
\pgfpathmoveto{\pgfqpoint{0.933197in}{1.287892in}}%
\pgfpathcurveto{\pgfqpoint{0.941434in}{1.287892in}}{\pgfqpoint{0.949334in}{1.291164in}}{\pgfqpoint{0.955158in}{1.296988in}}%
\pgfpathcurveto{\pgfqpoint{0.960982in}{1.302812in}}{\pgfqpoint{0.964254in}{1.310712in}}{\pgfqpoint{0.964254in}{1.318948in}}%
\pgfpathcurveto{\pgfqpoint{0.964254in}{1.327184in}}{\pgfqpoint{0.960982in}{1.335085in}}{\pgfqpoint{0.955158in}{1.340908in}}%
\pgfpathcurveto{\pgfqpoint{0.949334in}{1.346732in}}{\pgfqpoint{0.941434in}{1.350005in}}{\pgfqpoint{0.933197in}{1.350005in}}%
\pgfpathcurveto{\pgfqpoint{0.924961in}{1.350005in}}{\pgfqpoint{0.917061in}{1.346732in}}{\pgfqpoint{0.911237in}{1.340908in}}%
\pgfpathcurveto{\pgfqpoint{0.905413in}{1.335085in}}{\pgfqpoint{0.902141in}{1.327184in}}{\pgfqpoint{0.902141in}{1.318948in}}%
\pgfpathcurveto{\pgfqpoint{0.902141in}{1.310712in}}{\pgfqpoint{0.905413in}{1.302812in}}{\pgfqpoint{0.911237in}{1.296988in}}%
\pgfpathcurveto{\pgfqpoint{0.917061in}{1.291164in}}{\pgfqpoint{0.924961in}{1.287892in}}{\pgfqpoint{0.933197in}{1.287892in}}%
\pgfpathclose%
\pgfusepath{stroke,fill}%
\end{pgfscope}%
\begin{pgfscope}%
\pgfpathrectangle{\pgfqpoint{0.100000in}{0.212622in}}{\pgfqpoint{3.696000in}{3.696000in}}%
\pgfusepath{clip}%
\pgfsetbuttcap%
\pgfsetroundjoin%
\definecolor{currentfill}{rgb}{0.121569,0.466667,0.705882}%
\pgfsetfillcolor{currentfill}%
\pgfsetfillopacity{0.588972}%
\pgfsetlinewidth{1.003750pt}%
\definecolor{currentstroke}{rgb}{0.121569,0.466667,0.705882}%
\pgfsetstrokecolor{currentstroke}%
\pgfsetstrokeopacity{0.588972}%
\pgfsetdash{}{0pt}%
\pgfpathmoveto{\pgfqpoint{0.933197in}{1.287892in}}%
\pgfpathcurveto{\pgfqpoint{0.941433in}{1.287892in}}{\pgfqpoint{0.949333in}{1.291164in}}{\pgfqpoint{0.955157in}{1.296988in}}%
\pgfpathcurveto{\pgfqpoint{0.960981in}{1.302812in}}{\pgfqpoint{0.964253in}{1.310712in}}{\pgfqpoint{0.964253in}{1.318948in}}%
\pgfpathcurveto{\pgfqpoint{0.964253in}{1.327185in}}{\pgfqpoint{0.960981in}{1.335085in}}{\pgfqpoint{0.955157in}{1.340909in}}%
\pgfpathcurveto{\pgfqpoint{0.949333in}{1.346733in}}{\pgfqpoint{0.941433in}{1.350005in}}{\pgfqpoint{0.933197in}{1.350005in}}%
\pgfpathcurveto{\pgfqpoint{0.924960in}{1.350005in}}{\pgfqpoint{0.917060in}{1.346733in}}{\pgfqpoint{0.911236in}{1.340909in}}%
\pgfpathcurveto{\pgfqpoint{0.905412in}{1.335085in}}{\pgfqpoint{0.902140in}{1.327185in}}{\pgfqpoint{0.902140in}{1.318948in}}%
\pgfpathcurveto{\pgfqpoint{0.902140in}{1.310712in}}{\pgfqpoint{0.905412in}{1.302812in}}{\pgfqpoint{0.911236in}{1.296988in}}%
\pgfpathcurveto{\pgfqpoint{0.917060in}{1.291164in}}{\pgfqpoint{0.924960in}{1.287892in}}{\pgfqpoint{0.933197in}{1.287892in}}%
\pgfpathclose%
\pgfusepath{stroke,fill}%
\end{pgfscope}%
\begin{pgfscope}%
\pgfpathrectangle{\pgfqpoint{0.100000in}{0.212622in}}{\pgfqpoint{3.696000in}{3.696000in}}%
\pgfusepath{clip}%
\pgfsetbuttcap%
\pgfsetroundjoin%
\definecolor{currentfill}{rgb}{0.121569,0.466667,0.705882}%
\pgfsetfillcolor{currentfill}%
\pgfsetfillopacity{0.588972}%
\pgfsetlinewidth{1.003750pt}%
\definecolor{currentstroke}{rgb}{0.121569,0.466667,0.705882}%
\pgfsetstrokecolor{currentstroke}%
\pgfsetstrokeopacity{0.588972}%
\pgfsetdash{}{0pt}%
\pgfpathmoveto{\pgfqpoint{0.933196in}{1.287892in}}%
\pgfpathcurveto{\pgfqpoint{0.941433in}{1.287892in}}{\pgfqpoint{0.949333in}{1.291164in}}{\pgfqpoint{0.955157in}{1.296988in}}%
\pgfpathcurveto{\pgfqpoint{0.960980in}{1.302812in}}{\pgfqpoint{0.964253in}{1.310712in}}{\pgfqpoint{0.964253in}{1.318948in}}%
\pgfpathcurveto{\pgfqpoint{0.964253in}{1.327185in}}{\pgfqpoint{0.960980in}{1.335085in}}{\pgfqpoint{0.955157in}{1.340909in}}%
\pgfpathcurveto{\pgfqpoint{0.949333in}{1.346733in}}{\pgfqpoint{0.941433in}{1.350005in}}{\pgfqpoint{0.933196in}{1.350005in}}%
\pgfpathcurveto{\pgfqpoint{0.924960in}{1.350005in}}{\pgfqpoint{0.917060in}{1.346733in}}{\pgfqpoint{0.911236in}{1.340909in}}%
\pgfpathcurveto{\pgfqpoint{0.905412in}{1.335085in}}{\pgfqpoint{0.902140in}{1.327185in}}{\pgfqpoint{0.902140in}{1.318948in}}%
\pgfpathcurveto{\pgfqpoint{0.902140in}{1.310712in}}{\pgfqpoint{0.905412in}{1.302812in}}{\pgfqpoint{0.911236in}{1.296988in}}%
\pgfpathcurveto{\pgfqpoint{0.917060in}{1.291164in}}{\pgfqpoint{0.924960in}{1.287892in}}{\pgfqpoint{0.933196in}{1.287892in}}%
\pgfpathclose%
\pgfusepath{stroke,fill}%
\end{pgfscope}%
\begin{pgfscope}%
\pgfpathrectangle{\pgfqpoint{0.100000in}{0.212622in}}{\pgfqpoint{3.696000in}{3.696000in}}%
\pgfusepath{clip}%
\pgfsetbuttcap%
\pgfsetroundjoin%
\definecolor{currentfill}{rgb}{0.121569,0.466667,0.705882}%
\pgfsetfillcolor{currentfill}%
\pgfsetfillopacity{0.588972}%
\pgfsetlinewidth{1.003750pt}%
\definecolor{currentstroke}{rgb}{0.121569,0.466667,0.705882}%
\pgfsetstrokecolor{currentstroke}%
\pgfsetstrokeopacity{0.588972}%
\pgfsetdash{}{0pt}%
\pgfpathmoveto{\pgfqpoint{0.933196in}{1.287892in}}%
\pgfpathcurveto{\pgfqpoint{0.941432in}{1.287892in}}{\pgfqpoint{0.949332in}{1.291164in}}{\pgfqpoint{0.955156in}{1.296988in}}%
\pgfpathcurveto{\pgfqpoint{0.960980in}{1.302812in}}{\pgfqpoint{0.964253in}{1.310712in}}{\pgfqpoint{0.964253in}{1.318949in}}%
\pgfpathcurveto{\pgfqpoint{0.964253in}{1.327185in}}{\pgfqpoint{0.960980in}{1.335085in}}{\pgfqpoint{0.955156in}{1.340909in}}%
\pgfpathcurveto{\pgfqpoint{0.949332in}{1.346733in}}{\pgfqpoint{0.941432in}{1.350005in}}{\pgfqpoint{0.933196in}{1.350005in}}%
\pgfpathcurveto{\pgfqpoint{0.924960in}{1.350005in}}{\pgfqpoint{0.917060in}{1.346733in}}{\pgfqpoint{0.911236in}{1.340909in}}%
\pgfpathcurveto{\pgfqpoint{0.905412in}{1.335085in}}{\pgfqpoint{0.902140in}{1.327185in}}{\pgfqpoint{0.902140in}{1.318949in}}%
\pgfpathcurveto{\pgfqpoint{0.902140in}{1.310712in}}{\pgfqpoint{0.905412in}{1.302812in}}{\pgfqpoint{0.911236in}{1.296988in}}%
\pgfpathcurveto{\pgfqpoint{0.917060in}{1.291164in}}{\pgfqpoint{0.924960in}{1.287892in}}{\pgfqpoint{0.933196in}{1.287892in}}%
\pgfpathclose%
\pgfusepath{stroke,fill}%
\end{pgfscope}%
\begin{pgfscope}%
\pgfpathrectangle{\pgfqpoint{0.100000in}{0.212622in}}{\pgfqpoint{3.696000in}{3.696000in}}%
\pgfusepath{clip}%
\pgfsetbuttcap%
\pgfsetroundjoin%
\definecolor{currentfill}{rgb}{0.121569,0.466667,0.705882}%
\pgfsetfillcolor{currentfill}%
\pgfsetfillopacity{0.588972}%
\pgfsetlinewidth{1.003750pt}%
\definecolor{currentstroke}{rgb}{0.121569,0.466667,0.705882}%
\pgfsetstrokecolor{currentstroke}%
\pgfsetstrokeopacity{0.588972}%
\pgfsetdash{}{0pt}%
\pgfpathmoveto{\pgfqpoint{0.933196in}{1.287892in}}%
\pgfpathcurveto{\pgfqpoint{0.941432in}{1.287892in}}{\pgfqpoint{0.949332in}{1.291164in}}{\pgfqpoint{0.955156in}{1.296988in}}%
\pgfpathcurveto{\pgfqpoint{0.960980in}{1.302812in}}{\pgfqpoint{0.964252in}{1.310712in}}{\pgfqpoint{0.964252in}{1.318949in}}%
\pgfpathcurveto{\pgfqpoint{0.964252in}{1.327185in}}{\pgfqpoint{0.960980in}{1.335085in}}{\pgfqpoint{0.955156in}{1.340909in}}%
\pgfpathcurveto{\pgfqpoint{0.949332in}{1.346733in}}{\pgfqpoint{0.941432in}{1.350005in}}{\pgfqpoint{0.933196in}{1.350005in}}%
\pgfpathcurveto{\pgfqpoint{0.924960in}{1.350005in}}{\pgfqpoint{0.917060in}{1.346733in}}{\pgfqpoint{0.911236in}{1.340909in}}%
\pgfpathcurveto{\pgfqpoint{0.905412in}{1.335085in}}{\pgfqpoint{0.902139in}{1.327185in}}{\pgfqpoint{0.902139in}{1.318949in}}%
\pgfpathcurveto{\pgfqpoint{0.902139in}{1.310712in}}{\pgfqpoint{0.905412in}{1.302812in}}{\pgfqpoint{0.911236in}{1.296988in}}%
\pgfpathcurveto{\pgfqpoint{0.917060in}{1.291164in}}{\pgfqpoint{0.924960in}{1.287892in}}{\pgfqpoint{0.933196in}{1.287892in}}%
\pgfpathclose%
\pgfusepath{stroke,fill}%
\end{pgfscope}%
\begin{pgfscope}%
\pgfpathrectangle{\pgfqpoint{0.100000in}{0.212622in}}{\pgfqpoint{3.696000in}{3.696000in}}%
\pgfusepath{clip}%
\pgfsetbuttcap%
\pgfsetroundjoin%
\definecolor{currentfill}{rgb}{0.121569,0.466667,0.705882}%
\pgfsetfillcolor{currentfill}%
\pgfsetfillopacity{0.588972}%
\pgfsetlinewidth{1.003750pt}%
\definecolor{currentstroke}{rgb}{0.121569,0.466667,0.705882}%
\pgfsetstrokecolor{currentstroke}%
\pgfsetstrokeopacity{0.588972}%
\pgfsetdash{}{0pt}%
\pgfpathmoveto{\pgfqpoint{0.933196in}{1.287892in}}%
\pgfpathcurveto{\pgfqpoint{0.941432in}{1.287892in}}{\pgfqpoint{0.949332in}{1.291164in}}{\pgfqpoint{0.955156in}{1.296988in}}%
\pgfpathcurveto{\pgfqpoint{0.960980in}{1.302812in}}{\pgfqpoint{0.964252in}{1.310712in}}{\pgfqpoint{0.964252in}{1.318949in}}%
\pgfpathcurveto{\pgfqpoint{0.964252in}{1.327185in}}{\pgfqpoint{0.960980in}{1.335085in}}{\pgfqpoint{0.955156in}{1.340909in}}%
\pgfpathcurveto{\pgfqpoint{0.949332in}{1.346733in}}{\pgfqpoint{0.941432in}{1.350005in}}{\pgfqpoint{0.933196in}{1.350005in}}%
\pgfpathcurveto{\pgfqpoint{0.924960in}{1.350005in}}{\pgfqpoint{0.917060in}{1.346733in}}{\pgfqpoint{0.911236in}{1.340909in}}%
\pgfpathcurveto{\pgfqpoint{0.905412in}{1.335085in}}{\pgfqpoint{0.902139in}{1.327185in}}{\pgfqpoint{0.902139in}{1.318949in}}%
\pgfpathcurveto{\pgfqpoint{0.902139in}{1.310712in}}{\pgfqpoint{0.905412in}{1.302812in}}{\pgfqpoint{0.911236in}{1.296988in}}%
\pgfpathcurveto{\pgfqpoint{0.917060in}{1.291164in}}{\pgfqpoint{0.924960in}{1.287892in}}{\pgfqpoint{0.933196in}{1.287892in}}%
\pgfpathclose%
\pgfusepath{stroke,fill}%
\end{pgfscope}%
\begin{pgfscope}%
\pgfpathrectangle{\pgfqpoint{0.100000in}{0.212622in}}{\pgfqpoint{3.696000in}{3.696000in}}%
\pgfusepath{clip}%
\pgfsetbuttcap%
\pgfsetroundjoin%
\definecolor{currentfill}{rgb}{0.121569,0.466667,0.705882}%
\pgfsetfillcolor{currentfill}%
\pgfsetfillopacity{0.588972}%
\pgfsetlinewidth{1.003750pt}%
\definecolor{currentstroke}{rgb}{0.121569,0.466667,0.705882}%
\pgfsetstrokecolor{currentstroke}%
\pgfsetstrokeopacity{0.588972}%
\pgfsetdash{}{0pt}%
\pgfpathmoveto{\pgfqpoint{0.933196in}{1.287892in}}%
\pgfpathcurveto{\pgfqpoint{0.941432in}{1.287892in}}{\pgfqpoint{0.949332in}{1.291164in}}{\pgfqpoint{0.955156in}{1.296988in}}%
\pgfpathcurveto{\pgfqpoint{0.960980in}{1.302812in}}{\pgfqpoint{0.964252in}{1.310712in}}{\pgfqpoint{0.964252in}{1.318949in}}%
\pgfpathcurveto{\pgfqpoint{0.964252in}{1.327185in}}{\pgfqpoint{0.960980in}{1.335085in}}{\pgfqpoint{0.955156in}{1.340909in}}%
\pgfpathcurveto{\pgfqpoint{0.949332in}{1.346733in}}{\pgfqpoint{0.941432in}{1.350005in}}{\pgfqpoint{0.933196in}{1.350005in}}%
\pgfpathcurveto{\pgfqpoint{0.924960in}{1.350005in}}{\pgfqpoint{0.917060in}{1.346733in}}{\pgfqpoint{0.911236in}{1.340909in}}%
\pgfpathcurveto{\pgfqpoint{0.905412in}{1.335085in}}{\pgfqpoint{0.902139in}{1.327185in}}{\pgfqpoint{0.902139in}{1.318949in}}%
\pgfpathcurveto{\pgfqpoint{0.902139in}{1.310712in}}{\pgfqpoint{0.905412in}{1.302812in}}{\pgfqpoint{0.911236in}{1.296988in}}%
\pgfpathcurveto{\pgfqpoint{0.917060in}{1.291164in}}{\pgfqpoint{0.924960in}{1.287892in}}{\pgfqpoint{0.933196in}{1.287892in}}%
\pgfpathclose%
\pgfusepath{stroke,fill}%
\end{pgfscope}%
\begin{pgfscope}%
\pgfpathrectangle{\pgfqpoint{0.100000in}{0.212622in}}{\pgfqpoint{3.696000in}{3.696000in}}%
\pgfusepath{clip}%
\pgfsetbuttcap%
\pgfsetroundjoin%
\definecolor{currentfill}{rgb}{0.121569,0.466667,0.705882}%
\pgfsetfillcolor{currentfill}%
\pgfsetfillopacity{0.588972}%
\pgfsetlinewidth{1.003750pt}%
\definecolor{currentstroke}{rgb}{0.121569,0.466667,0.705882}%
\pgfsetstrokecolor{currentstroke}%
\pgfsetstrokeopacity{0.588972}%
\pgfsetdash{}{0pt}%
\pgfpathmoveto{\pgfqpoint{0.933196in}{1.287892in}}%
\pgfpathcurveto{\pgfqpoint{0.941432in}{1.287892in}}{\pgfqpoint{0.949332in}{1.291164in}}{\pgfqpoint{0.955156in}{1.296988in}}%
\pgfpathcurveto{\pgfqpoint{0.960980in}{1.302812in}}{\pgfqpoint{0.964252in}{1.310712in}}{\pgfqpoint{0.964252in}{1.318949in}}%
\pgfpathcurveto{\pgfqpoint{0.964252in}{1.327185in}}{\pgfqpoint{0.960980in}{1.335085in}}{\pgfqpoint{0.955156in}{1.340909in}}%
\pgfpathcurveto{\pgfqpoint{0.949332in}{1.346733in}}{\pgfqpoint{0.941432in}{1.350005in}}{\pgfqpoint{0.933196in}{1.350005in}}%
\pgfpathcurveto{\pgfqpoint{0.924960in}{1.350005in}}{\pgfqpoint{0.917060in}{1.346733in}}{\pgfqpoint{0.911236in}{1.340909in}}%
\pgfpathcurveto{\pgfqpoint{0.905412in}{1.335085in}}{\pgfqpoint{0.902139in}{1.327185in}}{\pgfqpoint{0.902139in}{1.318949in}}%
\pgfpathcurveto{\pgfqpoint{0.902139in}{1.310712in}}{\pgfqpoint{0.905412in}{1.302812in}}{\pgfqpoint{0.911236in}{1.296988in}}%
\pgfpathcurveto{\pgfqpoint{0.917060in}{1.291164in}}{\pgfqpoint{0.924960in}{1.287892in}}{\pgfqpoint{0.933196in}{1.287892in}}%
\pgfpathclose%
\pgfusepath{stroke,fill}%
\end{pgfscope}%
\begin{pgfscope}%
\pgfpathrectangle{\pgfqpoint{0.100000in}{0.212622in}}{\pgfqpoint{3.696000in}{3.696000in}}%
\pgfusepath{clip}%
\pgfsetbuttcap%
\pgfsetroundjoin%
\definecolor{currentfill}{rgb}{0.121569,0.466667,0.705882}%
\pgfsetfillcolor{currentfill}%
\pgfsetfillopacity{0.588972}%
\pgfsetlinewidth{1.003750pt}%
\definecolor{currentstroke}{rgb}{0.121569,0.466667,0.705882}%
\pgfsetstrokecolor{currentstroke}%
\pgfsetstrokeopacity{0.588972}%
\pgfsetdash{}{0pt}%
\pgfpathmoveto{\pgfqpoint{0.933196in}{1.287892in}}%
\pgfpathcurveto{\pgfqpoint{0.941432in}{1.287892in}}{\pgfqpoint{0.949332in}{1.291164in}}{\pgfqpoint{0.955156in}{1.296988in}}%
\pgfpathcurveto{\pgfqpoint{0.960980in}{1.302812in}}{\pgfqpoint{0.964252in}{1.310712in}}{\pgfqpoint{0.964252in}{1.318949in}}%
\pgfpathcurveto{\pgfqpoint{0.964252in}{1.327185in}}{\pgfqpoint{0.960980in}{1.335085in}}{\pgfqpoint{0.955156in}{1.340909in}}%
\pgfpathcurveto{\pgfqpoint{0.949332in}{1.346733in}}{\pgfqpoint{0.941432in}{1.350005in}}{\pgfqpoint{0.933196in}{1.350005in}}%
\pgfpathcurveto{\pgfqpoint{0.924960in}{1.350005in}}{\pgfqpoint{0.917060in}{1.346733in}}{\pgfqpoint{0.911236in}{1.340909in}}%
\pgfpathcurveto{\pgfqpoint{0.905412in}{1.335085in}}{\pgfqpoint{0.902139in}{1.327185in}}{\pgfqpoint{0.902139in}{1.318949in}}%
\pgfpathcurveto{\pgfqpoint{0.902139in}{1.310712in}}{\pgfqpoint{0.905412in}{1.302812in}}{\pgfqpoint{0.911236in}{1.296988in}}%
\pgfpathcurveto{\pgfqpoint{0.917060in}{1.291164in}}{\pgfqpoint{0.924960in}{1.287892in}}{\pgfqpoint{0.933196in}{1.287892in}}%
\pgfpathclose%
\pgfusepath{stroke,fill}%
\end{pgfscope}%
\begin{pgfscope}%
\pgfpathrectangle{\pgfqpoint{0.100000in}{0.212622in}}{\pgfqpoint{3.696000in}{3.696000in}}%
\pgfusepath{clip}%
\pgfsetbuttcap%
\pgfsetroundjoin%
\definecolor{currentfill}{rgb}{0.121569,0.466667,0.705882}%
\pgfsetfillcolor{currentfill}%
\pgfsetfillopacity{0.588972}%
\pgfsetlinewidth{1.003750pt}%
\definecolor{currentstroke}{rgb}{0.121569,0.466667,0.705882}%
\pgfsetstrokecolor{currentstroke}%
\pgfsetstrokeopacity{0.588972}%
\pgfsetdash{}{0pt}%
\pgfpathmoveto{\pgfqpoint{0.933196in}{1.287892in}}%
\pgfpathcurveto{\pgfqpoint{0.941432in}{1.287892in}}{\pgfqpoint{0.949332in}{1.291164in}}{\pgfqpoint{0.955156in}{1.296988in}}%
\pgfpathcurveto{\pgfqpoint{0.960980in}{1.302812in}}{\pgfqpoint{0.964252in}{1.310712in}}{\pgfqpoint{0.964252in}{1.318949in}}%
\pgfpathcurveto{\pgfqpoint{0.964252in}{1.327185in}}{\pgfqpoint{0.960980in}{1.335085in}}{\pgfqpoint{0.955156in}{1.340909in}}%
\pgfpathcurveto{\pgfqpoint{0.949332in}{1.346733in}}{\pgfqpoint{0.941432in}{1.350005in}}{\pgfqpoint{0.933196in}{1.350005in}}%
\pgfpathcurveto{\pgfqpoint{0.924960in}{1.350005in}}{\pgfqpoint{0.917060in}{1.346733in}}{\pgfqpoint{0.911236in}{1.340909in}}%
\pgfpathcurveto{\pgfqpoint{0.905412in}{1.335085in}}{\pgfqpoint{0.902139in}{1.327185in}}{\pgfqpoint{0.902139in}{1.318949in}}%
\pgfpathcurveto{\pgfqpoint{0.902139in}{1.310712in}}{\pgfqpoint{0.905412in}{1.302812in}}{\pgfqpoint{0.911236in}{1.296988in}}%
\pgfpathcurveto{\pgfqpoint{0.917060in}{1.291164in}}{\pgfqpoint{0.924960in}{1.287892in}}{\pgfqpoint{0.933196in}{1.287892in}}%
\pgfpathclose%
\pgfusepath{stroke,fill}%
\end{pgfscope}%
\begin{pgfscope}%
\pgfpathrectangle{\pgfqpoint{0.100000in}{0.212622in}}{\pgfqpoint{3.696000in}{3.696000in}}%
\pgfusepath{clip}%
\pgfsetbuttcap%
\pgfsetroundjoin%
\definecolor{currentfill}{rgb}{0.121569,0.466667,0.705882}%
\pgfsetfillcolor{currentfill}%
\pgfsetfillopacity{0.588972}%
\pgfsetlinewidth{1.003750pt}%
\definecolor{currentstroke}{rgb}{0.121569,0.466667,0.705882}%
\pgfsetstrokecolor{currentstroke}%
\pgfsetstrokeopacity{0.588972}%
\pgfsetdash{}{0pt}%
\pgfpathmoveto{\pgfqpoint{0.933196in}{1.287892in}}%
\pgfpathcurveto{\pgfqpoint{0.941432in}{1.287892in}}{\pgfqpoint{0.949332in}{1.291164in}}{\pgfqpoint{0.955156in}{1.296988in}}%
\pgfpathcurveto{\pgfqpoint{0.960980in}{1.302812in}}{\pgfqpoint{0.964252in}{1.310712in}}{\pgfqpoint{0.964252in}{1.318949in}}%
\pgfpathcurveto{\pgfqpoint{0.964252in}{1.327185in}}{\pgfqpoint{0.960980in}{1.335085in}}{\pgfqpoint{0.955156in}{1.340909in}}%
\pgfpathcurveto{\pgfqpoint{0.949332in}{1.346733in}}{\pgfqpoint{0.941432in}{1.350005in}}{\pgfqpoint{0.933196in}{1.350005in}}%
\pgfpathcurveto{\pgfqpoint{0.924960in}{1.350005in}}{\pgfqpoint{0.917060in}{1.346733in}}{\pgfqpoint{0.911236in}{1.340909in}}%
\pgfpathcurveto{\pgfqpoint{0.905412in}{1.335085in}}{\pgfqpoint{0.902139in}{1.327185in}}{\pgfqpoint{0.902139in}{1.318949in}}%
\pgfpathcurveto{\pgfqpoint{0.902139in}{1.310712in}}{\pgfqpoint{0.905412in}{1.302812in}}{\pgfqpoint{0.911236in}{1.296988in}}%
\pgfpathcurveto{\pgfqpoint{0.917060in}{1.291164in}}{\pgfqpoint{0.924960in}{1.287892in}}{\pgfqpoint{0.933196in}{1.287892in}}%
\pgfpathclose%
\pgfusepath{stroke,fill}%
\end{pgfscope}%
\begin{pgfscope}%
\pgfpathrectangle{\pgfqpoint{0.100000in}{0.212622in}}{\pgfqpoint{3.696000in}{3.696000in}}%
\pgfusepath{clip}%
\pgfsetbuttcap%
\pgfsetroundjoin%
\definecolor{currentfill}{rgb}{0.121569,0.466667,0.705882}%
\pgfsetfillcolor{currentfill}%
\pgfsetfillopacity{0.588972}%
\pgfsetlinewidth{1.003750pt}%
\definecolor{currentstroke}{rgb}{0.121569,0.466667,0.705882}%
\pgfsetstrokecolor{currentstroke}%
\pgfsetstrokeopacity{0.588972}%
\pgfsetdash{}{0pt}%
\pgfpathmoveto{\pgfqpoint{0.933196in}{1.287892in}}%
\pgfpathcurveto{\pgfqpoint{0.941432in}{1.287892in}}{\pgfqpoint{0.949332in}{1.291164in}}{\pgfqpoint{0.955156in}{1.296988in}}%
\pgfpathcurveto{\pgfqpoint{0.960980in}{1.302812in}}{\pgfqpoint{0.964252in}{1.310712in}}{\pgfqpoint{0.964252in}{1.318949in}}%
\pgfpathcurveto{\pgfqpoint{0.964252in}{1.327185in}}{\pgfqpoint{0.960980in}{1.335085in}}{\pgfqpoint{0.955156in}{1.340909in}}%
\pgfpathcurveto{\pgfqpoint{0.949332in}{1.346733in}}{\pgfqpoint{0.941432in}{1.350005in}}{\pgfqpoint{0.933196in}{1.350005in}}%
\pgfpathcurveto{\pgfqpoint{0.924960in}{1.350005in}}{\pgfqpoint{0.917060in}{1.346733in}}{\pgfqpoint{0.911236in}{1.340909in}}%
\pgfpathcurveto{\pgfqpoint{0.905412in}{1.335085in}}{\pgfqpoint{0.902139in}{1.327185in}}{\pgfqpoint{0.902139in}{1.318949in}}%
\pgfpathcurveto{\pgfqpoint{0.902139in}{1.310712in}}{\pgfqpoint{0.905412in}{1.302812in}}{\pgfqpoint{0.911236in}{1.296988in}}%
\pgfpathcurveto{\pgfqpoint{0.917060in}{1.291164in}}{\pgfqpoint{0.924960in}{1.287892in}}{\pgfqpoint{0.933196in}{1.287892in}}%
\pgfpathclose%
\pgfusepath{stroke,fill}%
\end{pgfscope}%
\begin{pgfscope}%
\pgfpathrectangle{\pgfqpoint{0.100000in}{0.212622in}}{\pgfqpoint{3.696000in}{3.696000in}}%
\pgfusepath{clip}%
\pgfsetbuttcap%
\pgfsetroundjoin%
\definecolor{currentfill}{rgb}{0.121569,0.466667,0.705882}%
\pgfsetfillcolor{currentfill}%
\pgfsetfillopacity{0.588972}%
\pgfsetlinewidth{1.003750pt}%
\definecolor{currentstroke}{rgb}{0.121569,0.466667,0.705882}%
\pgfsetstrokecolor{currentstroke}%
\pgfsetstrokeopacity{0.588972}%
\pgfsetdash{}{0pt}%
\pgfpathmoveto{\pgfqpoint{0.933196in}{1.287892in}}%
\pgfpathcurveto{\pgfqpoint{0.941432in}{1.287892in}}{\pgfqpoint{0.949332in}{1.291164in}}{\pgfqpoint{0.955156in}{1.296988in}}%
\pgfpathcurveto{\pgfqpoint{0.960980in}{1.302812in}}{\pgfqpoint{0.964252in}{1.310712in}}{\pgfqpoint{0.964252in}{1.318949in}}%
\pgfpathcurveto{\pgfqpoint{0.964252in}{1.327185in}}{\pgfqpoint{0.960980in}{1.335085in}}{\pgfqpoint{0.955156in}{1.340909in}}%
\pgfpathcurveto{\pgfqpoint{0.949332in}{1.346733in}}{\pgfqpoint{0.941432in}{1.350005in}}{\pgfqpoint{0.933196in}{1.350005in}}%
\pgfpathcurveto{\pgfqpoint{0.924960in}{1.350005in}}{\pgfqpoint{0.917060in}{1.346733in}}{\pgfqpoint{0.911236in}{1.340909in}}%
\pgfpathcurveto{\pgfqpoint{0.905412in}{1.335085in}}{\pgfqpoint{0.902139in}{1.327185in}}{\pgfqpoint{0.902139in}{1.318949in}}%
\pgfpathcurveto{\pgfqpoint{0.902139in}{1.310712in}}{\pgfqpoint{0.905412in}{1.302812in}}{\pgfqpoint{0.911236in}{1.296988in}}%
\pgfpathcurveto{\pgfqpoint{0.917060in}{1.291164in}}{\pgfqpoint{0.924960in}{1.287892in}}{\pgfqpoint{0.933196in}{1.287892in}}%
\pgfpathclose%
\pgfusepath{stroke,fill}%
\end{pgfscope}%
\begin{pgfscope}%
\pgfpathrectangle{\pgfqpoint{0.100000in}{0.212622in}}{\pgfqpoint{3.696000in}{3.696000in}}%
\pgfusepath{clip}%
\pgfsetbuttcap%
\pgfsetroundjoin%
\definecolor{currentfill}{rgb}{0.121569,0.466667,0.705882}%
\pgfsetfillcolor{currentfill}%
\pgfsetfillopacity{0.588972}%
\pgfsetlinewidth{1.003750pt}%
\definecolor{currentstroke}{rgb}{0.121569,0.466667,0.705882}%
\pgfsetstrokecolor{currentstroke}%
\pgfsetstrokeopacity{0.588972}%
\pgfsetdash{}{0pt}%
\pgfpathmoveto{\pgfqpoint{0.933196in}{1.287892in}}%
\pgfpathcurveto{\pgfqpoint{0.941432in}{1.287892in}}{\pgfqpoint{0.949332in}{1.291164in}}{\pgfqpoint{0.955156in}{1.296988in}}%
\pgfpathcurveto{\pgfqpoint{0.960980in}{1.302812in}}{\pgfqpoint{0.964252in}{1.310712in}}{\pgfqpoint{0.964252in}{1.318949in}}%
\pgfpathcurveto{\pgfqpoint{0.964252in}{1.327185in}}{\pgfqpoint{0.960980in}{1.335085in}}{\pgfqpoint{0.955156in}{1.340909in}}%
\pgfpathcurveto{\pgfqpoint{0.949332in}{1.346733in}}{\pgfqpoint{0.941432in}{1.350005in}}{\pgfqpoint{0.933196in}{1.350005in}}%
\pgfpathcurveto{\pgfqpoint{0.924960in}{1.350005in}}{\pgfqpoint{0.917060in}{1.346733in}}{\pgfqpoint{0.911236in}{1.340909in}}%
\pgfpathcurveto{\pgfqpoint{0.905412in}{1.335085in}}{\pgfqpoint{0.902139in}{1.327185in}}{\pgfqpoint{0.902139in}{1.318949in}}%
\pgfpathcurveto{\pgfqpoint{0.902139in}{1.310712in}}{\pgfqpoint{0.905412in}{1.302812in}}{\pgfqpoint{0.911236in}{1.296988in}}%
\pgfpathcurveto{\pgfqpoint{0.917060in}{1.291164in}}{\pgfqpoint{0.924960in}{1.287892in}}{\pgfqpoint{0.933196in}{1.287892in}}%
\pgfpathclose%
\pgfusepath{stroke,fill}%
\end{pgfscope}%
\begin{pgfscope}%
\pgfpathrectangle{\pgfqpoint{0.100000in}{0.212622in}}{\pgfqpoint{3.696000in}{3.696000in}}%
\pgfusepath{clip}%
\pgfsetbuttcap%
\pgfsetroundjoin%
\definecolor{currentfill}{rgb}{0.121569,0.466667,0.705882}%
\pgfsetfillcolor{currentfill}%
\pgfsetfillopacity{0.588972}%
\pgfsetlinewidth{1.003750pt}%
\definecolor{currentstroke}{rgb}{0.121569,0.466667,0.705882}%
\pgfsetstrokecolor{currentstroke}%
\pgfsetstrokeopacity{0.588972}%
\pgfsetdash{}{0pt}%
\pgfpathmoveto{\pgfqpoint{0.933196in}{1.287892in}}%
\pgfpathcurveto{\pgfqpoint{0.941432in}{1.287892in}}{\pgfqpoint{0.949332in}{1.291164in}}{\pgfqpoint{0.955156in}{1.296988in}}%
\pgfpathcurveto{\pgfqpoint{0.960980in}{1.302812in}}{\pgfqpoint{0.964252in}{1.310712in}}{\pgfqpoint{0.964252in}{1.318949in}}%
\pgfpathcurveto{\pgfqpoint{0.964252in}{1.327185in}}{\pgfqpoint{0.960980in}{1.335085in}}{\pgfqpoint{0.955156in}{1.340909in}}%
\pgfpathcurveto{\pgfqpoint{0.949332in}{1.346733in}}{\pgfqpoint{0.941432in}{1.350005in}}{\pgfqpoint{0.933196in}{1.350005in}}%
\pgfpathcurveto{\pgfqpoint{0.924960in}{1.350005in}}{\pgfqpoint{0.917060in}{1.346733in}}{\pgfqpoint{0.911236in}{1.340909in}}%
\pgfpathcurveto{\pgfqpoint{0.905412in}{1.335085in}}{\pgfqpoint{0.902139in}{1.327185in}}{\pgfqpoint{0.902139in}{1.318949in}}%
\pgfpathcurveto{\pgfqpoint{0.902139in}{1.310712in}}{\pgfqpoint{0.905412in}{1.302812in}}{\pgfqpoint{0.911236in}{1.296988in}}%
\pgfpathcurveto{\pgfqpoint{0.917060in}{1.291164in}}{\pgfqpoint{0.924960in}{1.287892in}}{\pgfqpoint{0.933196in}{1.287892in}}%
\pgfpathclose%
\pgfusepath{stroke,fill}%
\end{pgfscope}%
\begin{pgfscope}%
\pgfpathrectangle{\pgfqpoint{0.100000in}{0.212622in}}{\pgfqpoint{3.696000in}{3.696000in}}%
\pgfusepath{clip}%
\pgfsetbuttcap%
\pgfsetroundjoin%
\definecolor{currentfill}{rgb}{0.121569,0.466667,0.705882}%
\pgfsetfillcolor{currentfill}%
\pgfsetfillopacity{0.588972}%
\pgfsetlinewidth{1.003750pt}%
\definecolor{currentstroke}{rgb}{0.121569,0.466667,0.705882}%
\pgfsetstrokecolor{currentstroke}%
\pgfsetstrokeopacity{0.588972}%
\pgfsetdash{}{0pt}%
\pgfpathmoveto{\pgfqpoint{0.933196in}{1.287892in}}%
\pgfpathcurveto{\pgfqpoint{0.941432in}{1.287892in}}{\pgfqpoint{0.949332in}{1.291164in}}{\pgfqpoint{0.955156in}{1.296988in}}%
\pgfpathcurveto{\pgfqpoint{0.960980in}{1.302812in}}{\pgfqpoint{0.964252in}{1.310712in}}{\pgfqpoint{0.964252in}{1.318949in}}%
\pgfpathcurveto{\pgfqpoint{0.964252in}{1.327185in}}{\pgfqpoint{0.960980in}{1.335085in}}{\pgfqpoint{0.955156in}{1.340909in}}%
\pgfpathcurveto{\pgfqpoint{0.949332in}{1.346733in}}{\pgfqpoint{0.941432in}{1.350005in}}{\pgfqpoint{0.933196in}{1.350005in}}%
\pgfpathcurveto{\pgfqpoint{0.924960in}{1.350005in}}{\pgfqpoint{0.917060in}{1.346733in}}{\pgfqpoint{0.911236in}{1.340909in}}%
\pgfpathcurveto{\pgfqpoint{0.905412in}{1.335085in}}{\pgfqpoint{0.902139in}{1.327185in}}{\pgfqpoint{0.902139in}{1.318949in}}%
\pgfpathcurveto{\pgfqpoint{0.902139in}{1.310712in}}{\pgfqpoint{0.905412in}{1.302812in}}{\pgfqpoint{0.911236in}{1.296988in}}%
\pgfpathcurveto{\pgfqpoint{0.917060in}{1.291164in}}{\pgfqpoint{0.924960in}{1.287892in}}{\pgfqpoint{0.933196in}{1.287892in}}%
\pgfpathclose%
\pgfusepath{stroke,fill}%
\end{pgfscope}%
\begin{pgfscope}%
\pgfpathrectangle{\pgfqpoint{0.100000in}{0.212622in}}{\pgfqpoint{3.696000in}{3.696000in}}%
\pgfusepath{clip}%
\pgfsetbuttcap%
\pgfsetroundjoin%
\definecolor{currentfill}{rgb}{0.121569,0.466667,0.705882}%
\pgfsetfillcolor{currentfill}%
\pgfsetfillopacity{0.588972}%
\pgfsetlinewidth{1.003750pt}%
\definecolor{currentstroke}{rgb}{0.121569,0.466667,0.705882}%
\pgfsetstrokecolor{currentstroke}%
\pgfsetstrokeopacity{0.588972}%
\pgfsetdash{}{0pt}%
\pgfpathmoveto{\pgfqpoint{0.933196in}{1.287892in}}%
\pgfpathcurveto{\pgfqpoint{0.941432in}{1.287892in}}{\pgfqpoint{0.949332in}{1.291164in}}{\pgfqpoint{0.955156in}{1.296988in}}%
\pgfpathcurveto{\pgfqpoint{0.960980in}{1.302812in}}{\pgfqpoint{0.964252in}{1.310712in}}{\pgfqpoint{0.964252in}{1.318949in}}%
\pgfpathcurveto{\pgfqpoint{0.964252in}{1.327185in}}{\pgfqpoint{0.960980in}{1.335085in}}{\pgfqpoint{0.955156in}{1.340909in}}%
\pgfpathcurveto{\pgfqpoint{0.949332in}{1.346733in}}{\pgfqpoint{0.941432in}{1.350005in}}{\pgfqpoint{0.933196in}{1.350005in}}%
\pgfpathcurveto{\pgfqpoint{0.924960in}{1.350005in}}{\pgfqpoint{0.917060in}{1.346733in}}{\pgfqpoint{0.911236in}{1.340909in}}%
\pgfpathcurveto{\pgfqpoint{0.905412in}{1.335085in}}{\pgfqpoint{0.902139in}{1.327185in}}{\pgfqpoint{0.902139in}{1.318949in}}%
\pgfpathcurveto{\pgfqpoint{0.902139in}{1.310712in}}{\pgfqpoint{0.905412in}{1.302812in}}{\pgfqpoint{0.911236in}{1.296988in}}%
\pgfpathcurveto{\pgfqpoint{0.917060in}{1.291164in}}{\pgfqpoint{0.924960in}{1.287892in}}{\pgfqpoint{0.933196in}{1.287892in}}%
\pgfpathclose%
\pgfusepath{stroke,fill}%
\end{pgfscope}%
\begin{pgfscope}%
\pgfpathrectangle{\pgfqpoint{0.100000in}{0.212622in}}{\pgfqpoint{3.696000in}{3.696000in}}%
\pgfusepath{clip}%
\pgfsetbuttcap%
\pgfsetroundjoin%
\definecolor{currentfill}{rgb}{0.121569,0.466667,0.705882}%
\pgfsetfillcolor{currentfill}%
\pgfsetfillopacity{0.588972}%
\pgfsetlinewidth{1.003750pt}%
\definecolor{currentstroke}{rgb}{0.121569,0.466667,0.705882}%
\pgfsetstrokecolor{currentstroke}%
\pgfsetstrokeopacity{0.588972}%
\pgfsetdash{}{0pt}%
\pgfpathmoveto{\pgfqpoint{0.933196in}{1.287892in}}%
\pgfpathcurveto{\pgfqpoint{0.941432in}{1.287892in}}{\pgfqpoint{0.949332in}{1.291164in}}{\pgfqpoint{0.955156in}{1.296988in}}%
\pgfpathcurveto{\pgfqpoint{0.960980in}{1.302812in}}{\pgfqpoint{0.964252in}{1.310712in}}{\pgfqpoint{0.964252in}{1.318949in}}%
\pgfpathcurveto{\pgfqpoint{0.964252in}{1.327185in}}{\pgfqpoint{0.960980in}{1.335085in}}{\pgfqpoint{0.955156in}{1.340909in}}%
\pgfpathcurveto{\pgfqpoint{0.949332in}{1.346733in}}{\pgfqpoint{0.941432in}{1.350005in}}{\pgfqpoint{0.933196in}{1.350005in}}%
\pgfpathcurveto{\pgfqpoint{0.924960in}{1.350005in}}{\pgfqpoint{0.917060in}{1.346733in}}{\pgfqpoint{0.911236in}{1.340909in}}%
\pgfpathcurveto{\pgfqpoint{0.905412in}{1.335085in}}{\pgfqpoint{0.902139in}{1.327185in}}{\pgfqpoint{0.902139in}{1.318949in}}%
\pgfpathcurveto{\pgfqpoint{0.902139in}{1.310712in}}{\pgfqpoint{0.905412in}{1.302812in}}{\pgfqpoint{0.911236in}{1.296988in}}%
\pgfpathcurveto{\pgfqpoint{0.917060in}{1.291164in}}{\pgfqpoint{0.924960in}{1.287892in}}{\pgfqpoint{0.933196in}{1.287892in}}%
\pgfpathclose%
\pgfusepath{stroke,fill}%
\end{pgfscope}%
\begin{pgfscope}%
\pgfpathrectangle{\pgfqpoint{0.100000in}{0.212622in}}{\pgfqpoint{3.696000in}{3.696000in}}%
\pgfusepath{clip}%
\pgfsetbuttcap%
\pgfsetroundjoin%
\definecolor{currentfill}{rgb}{0.121569,0.466667,0.705882}%
\pgfsetfillcolor{currentfill}%
\pgfsetfillopacity{0.588972}%
\pgfsetlinewidth{1.003750pt}%
\definecolor{currentstroke}{rgb}{0.121569,0.466667,0.705882}%
\pgfsetstrokecolor{currentstroke}%
\pgfsetstrokeopacity{0.588972}%
\pgfsetdash{}{0pt}%
\pgfpathmoveto{\pgfqpoint{0.933196in}{1.287892in}}%
\pgfpathcurveto{\pgfqpoint{0.941432in}{1.287892in}}{\pgfqpoint{0.949332in}{1.291164in}}{\pgfqpoint{0.955156in}{1.296988in}}%
\pgfpathcurveto{\pgfqpoint{0.960980in}{1.302812in}}{\pgfqpoint{0.964252in}{1.310712in}}{\pgfqpoint{0.964252in}{1.318949in}}%
\pgfpathcurveto{\pgfqpoint{0.964252in}{1.327185in}}{\pgfqpoint{0.960980in}{1.335085in}}{\pgfqpoint{0.955156in}{1.340909in}}%
\pgfpathcurveto{\pgfqpoint{0.949332in}{1.346733in}}{\pgfqpoint{0.941432in}{1.350005in}}{\pgfqpoint{0.933196in}{1.350005in}}%
\pgfpathcurveto{\pgfqpoint{0.924960in}{1.350005in}}{\pgfqpoint{0.917060in}{1.346733in}}{\pgfqpoint{0.911236in}{1.340909in}}%
\pgfpathcurveto{\pgfqpoint{0.905412in}{1.335085in}}{\pgfqpoint{0.902139in}{1.327185in}}{\pgfqpoint{0.902139in}{1.318949in}}%
\pgfpathcurveto{\pgfqpoint{0.902139in}{1.310712in}}{\pgfqpoint{0.905412in}{1.302812in}}{\pgfqpoint{0.911236in}{1.296988in}}%
\pgfpathcurveto{\pgfqpoint{0.917060in}{1.291164in}}{\pgfqpoint{0.924960in}{1.287892in}}{\pgfqpoint{0.933196in}{1.287892in}}%
\pgfpathclose%
\pgfusepath{stroke,fill}%
\end{pgfscope}%
\begin{pgfscope}%
\pgfpathrectangle{\pgfqpoint{0.100000in}{0.212622in}}{\pgfqpoint{3.696000in}{3.696000in}}%
\pgfusepath{clip}%
\pgfsetbuttcap%
\pgfsetroundjoin%
\definecolor{currentfill}{rgb}{0.121569,0.466667,0.705882}%
\pgfsetfillcolor{currentfill}%
\pgfsetfillopacity{0.588972}%
\pgfsetlinewidth{1.003750pt}%
\definecolor{currentstroke}{rgb}{0.121569,0.466667,0.705882}%
\pgfsetstrokecolor{currentstroke}%
\pgfsetstrokeopacity{0.588972}%
\pgfsetdash{}{0pt}%
\pgfpathmoveto{\pgfqpoint{0.933196in}{1.287892in}}%
\pgfpathcurveto{\pgfqpoint{0.941432in}{1.287892in}}{\pgfqpoint{0.949332in}{1.291164in}}{\pgfqpoint{0.955156in}{1.296988in}}%
\pgfpathcurveto{\pgfqpoint{0.960980in}{1.302812in}}{\pgfqpoint{0.964252in}{1.310712in}}{\pgfqpoint{0.964252in}{1.318949in}}%
\pgfpathcurveto{\pgfqpoint{0.964252in}{1.327185in}}{\pgfqpoint{0.960980in}{1.335085in}}{\pgfqpoint{0.955156in}{1.340909in}}%
\pgfpathcurveto{\pgfqpoint{0.949332in}{1.346733in}}{\pgfqpoint{0.941432in}{1.350005in}}{\pgfqpoint{0.933196in}{1.350005in}}%
\pgfpathcurveto{\pgfqpoint{0.924960in}{1.350005in}}{\pgfqpoint{0.917060in}{1.346733in}}{\pgfqpoint{0.911236in}{1.340909in}}%
\pgfpathcurveto{\pgfqpoint{0.905412in}{1.335085in}}{\pgfqpoint{0.902139in}{1.327185in}}{\pgfqpoint{0.902139in}{1.318949in}}%
\pgfpathcurveto{\pgfqpoint{0.902139in}{1.310712in}}{\pgfqpoint{0.905412in}{1.302812in}}{\pgfqpoint{0.911236in}{1.296988in}}%
\pgfpathcurveto{\pgfqpoint{0.917060in}{1.291164in}}{\pgfqpoint{0.924960in}{1.287892in}}{\pgfqpoint{0.933196in}{1.287892in}}%
\pgfpathclose%
\pgfusepath{stroke,fill}%
\end{pgfscope}%
\begin{pgfscope}%
\pgfpathrectangle{\pgfqpoint{0.100000in}{0.212622in}}{\pgfqpoint{3.696000in}{3.696000in}}%
\pgfusepath{clip}%
\pgfsetbuttcap%
\pgfsetroundjoin%
\definecolor{currentfill}{rgb}{0.121569,0.466667,0.705882}%
\pgfsetfillcolor{currentfill}%
\pgfsetfillopacity{0.588972}%
\pgfsetlinewidth{1.003750pt}%
\definecolor{currentstroke}{rgb}{0.121569,0.466667,0.705882}%
\pgfsetstrokecolor{currentstroke}%
\pgfsetstrokeopacity{0.588972}%
\pgfsetdash{}{0pt}%
\pgfpathmoveto{\pgfqpoint{0.933196in}{1.287892in}}%
\pgfpathcurveto{\pgfqpoint{0.941432in}{1.287892in}}{\pgfqpoint{0.949332in}{1.291164in}}{\pgfqpoint{0.955156in}{1.296988in}}%
\pgfpathcurveto{\pgfqpoint{0.960980in}{1.302812in}}{\pgfqpoint{0.964252in}{1.310712in}}{\pgfqpoint{0.964252in}{1.318949in}}%
\pgfpathcurveto{\pgfqpoint{0.964252in}{1.327185in}}{\pgfqpoint{0.960980in}{1.335085in}}{\pgfqpoint{0.955156in}{1.340909in}}%
\pgfpathcurveto{\pgfqpoint{0.949332in}{1.346733in}}{\pgfqpoint{0.941432in}{1.350005in}}{\pgfqpoint{0.933196in}{1.350005in}}%
\pgfpathcurveto{\pgfqpoint{0.924960in}{1.350005in}}{\pgfqpoint{0.917060in}{1.346733in}}{\pgfqpoint{0.911236in}{1.340909in}}%
\pgfpathcurveto{\pgfqpoint{0.905412in}{1.335085in}}{\pgfqpoint{0.902139in}{1.327185in}}{\pgfqpoint{0.902139in}{1.318949in}}%
\pgfpathcurveto{\pgfqpoint{0.902139in}{1.310712in}}{\pgfqpoint{0.905412in}{1.302812in}}{\pgfqpoint{0.911236in}{1.296988in}}%
\pgfpathcurveto{\pgfqpoint{0.917060in}{1.291164in}}{\pgfqpoint{0.924960in}{1.287892in}}{\pgfqpoint{0.933196in}{1.287892in}}%
\pgfpathclose%
\pgfusepath{stroke,fill}%
\end{pgfscope}%
\begin{pgfscope}%
\pgfpathrectangle{\pgfqpoint{0.100000in}{0.212622in}}{\pgfqpoint{3.696000in}{3.696000in}}%
\pgfusepath{clip}%
\pgfsetbuttcap%
\pgfsetroundjoin%
\definecolor{currentfill}{rgb}{0.121569,0.466667,0.705882}%
\pgfsetfillcolor{currentfill}%
\pgfsetfillopacity{0.588972}%
\pgfsetlinewidth{1.003750pt}%
\definecolor{currentstroke}{rgb}{0.121569,0.466667,0.705882}%
\pgfsetstrokecolor{currentstroke}%
\pgfsetstrokeopacity{0.588972}%
\pgfsetdash{}{0pt}%
\pgfpathmoveto{\pgfqpoint{0.933196in}{1.287892in}}%
\pgfpathcurveto{\pgfqpoint{0.941432in}{1.287892in}}{\pgfqpoint{0.949332in}{1.291164in}}{\pgfqpoint{0.955156in}{1.296988in}}%
\pgfpathcurveto{\pgfqpoint{0.960980in}{1.302812in}}{\pgfqpoint{0.964252in}{1.310712in}}{\pgfqpoint{0.964252in}{1.318949in}}%
\pgfpathcurveto{\pgfqpoint{0.964252in}{1.327185in}}{\pgfqpoint{0.960980in}{1.335085in}}{\pgfqpoint{0.955156in}{1.340909in}}%
\pgfpathcurveto{\pgfqpoint{0.949332in}{1.346733in}}{\pgfqpoint{0.941432in}{1.350005in}}{\pgfqpoint{0.933196in}{1.350005in}}%
\pgfpathcurveto{\pgfqpoint{0.924960in}{1.350005in}}{\pgfqpoint{0.917060in}{1.346733in}}{\pgfqpoint{0.911236in}{1.340909in}}%
\pgfpathcurveto{\pgfqpoint{0.905412in}{1.335085in}}{\pgfqpoint{0.902139in}{1.327185in}}{\pgfqpoint{0.902139in}{1.318949in}}%
\pgfpathcurveto{\pgfqpoint{0.902139in}{1.310712in}}{\pgfqpoint{0.905412in}{1.302812in}}{\pgfqpoint{0.911236in}{1.296988in}}%
\pgfpathcurveto{\pgfqpoint{0.917060in}{1.291164in}}{\pgfqpoint{0.924960in}{1.287892in}}{\pgfqpoint{0.933196in}{1.287892in}}%
\pgfpathclose%
\pgfusepath{stroke,fill}%
\end{pgfscope}%
\begin{pgfscope}%
\pgfpathrectangle{\pgfqpoint{0.100000in}{0.212622in}}{\pgfqpoint{3.696000in}{3.696000in}}%
\pgfusepath{clip}%
\pgfsetbuttcap%
\pgfsetroundjoin%
\definecolor{currentfill}{rgb}{0.121569,0.466667,0.705882}%
\pgfsetfillcolor{currentfill}%
\pgfsetfillopacity{0.588972}%
\pgfsetlinewidth{1.003750pt}%
\definecolor{currentstroke}{rgb}{0.121569,0.466667,0.705882}%
\pgfsetstrokecolor{currentstroke}%
\pgfsetstrokeopacity{0.588972}%
\pgfsetdash{}{0pt}%
\pgfpathmoveto{\pgfqpoint{0.933196in}{1.287892in}}%
\pgfpathcurveto{\pgfqpoint{0.941432in}{1.287892in}}{\pgfqpoint{0.949332in}{1.291164in}}{\pgfqpoint{0.955156in}{1.296988in}}%
\pgfpathcurveto{\pgfqpoint{0.960980in}{1.302812in}}{\pgfqpoint{0.964252in}{1.310712in}}{\pgfqpoint{0.964252in}{1.318949in}}%
\pgfpathcurveto{\pgfqpoint{0.964252in}{1.327185in}}{\pgfqpoint{0.960980in}{1.335085in}}{\pgfqpoint{0.955156in}{1.340909in}}%
\pgfpathcurveto{\pgfqpoint{0.949332in}{1.346733in}}{\pgfqpoint{0.941432in}{1.350005in}}{\pgfqpoint{0.933196in}{1.350005in}}%
\pgfpathcurveto{\pgfqpoint{0.924960in}{1.350005in}}{\pgfqpoint{0.917060in}{1.346733in}}{\pgfqpoint{0.911236in}{1.340909in}}%
\pgfpathcurveto{\pgfqpoint{0.905412in}{1.335085in}}{\pgfqpoint{0.902139in}{1.327185in}}{\pgfqpoint{0.902139in}{1.318949in}}%
\pgfpathcurveto{\pgfqpoint{0.902139in}{1.310712in}}{\pgfqpoint{0.905412in}{1.302812in}}{\pgfqpoint{0.911236in}{1.296988in}}%
\pgfpathcurveto{\pgfqpoint{0.917060in}{1.291164in}}{\pgfqpoint{0.924960in}{1.287892in}}{\pgfqpoint{0.933196in}{1.287892in}}%
\pgfpathclose%
\pgfusepath{stroke,fill}%
\end{pgfscope}%
\begin{pgfscope}%
\pgfpathrectangle{\pgfqpoint{0.100000in}{0.212622in}}{\pgfqpoint{3.696000in}{3.696000in}}%
\pgfusepath{clip}%
\pgfsetbuttcap%
\pgfsetroundjoin%
\definecolor{currentfill}{rgb}{0.121569,0.466667,0.705882}%
\pgfsetfillcolor{currentfill}%
\pgfsetfillopacity{0.588972}%
\pgfsetlinewidth{1.003750pt}%
\definecolor{currentstroke}{rgb}{0.121569,0.466667,0.705882}%
\pgfsetstrokecolor{currentstroke}%
\pgfsetstrokeopacity{0.588972}%
\pgfsetdash{}{0pt}%
\pgfpathmoveto{\pgfqpoint{0.933196in}{1.287892in}}%
\pgfpathcurveto{\pgfqpoint{0.941432in}{1.287892in}}{\pgfqpoint{0.949332in}{1.291164in}}{\pgfqpoint{0.955156in}{1.296988in}}%
\pgfpathcurveto{\pgfqpoint{0.960980in}{1.302812in}}{\pgfqpoint{0.964252in}{1.310712in}}{\pgfqpoint{0.964252in}{1.318949in}}%
\pgfpathcurveto{\pgfqpoint{0.964252in}{1.327185in}}{\pgfqpoint{0.960980in}{1.335085in}}{\pgfqpoint{0.955156in}{1.340909in}}%
\pgfpathcurveto{\pgfqpoint{0.949332in}{1.346733in}}{\pgfqpoint{0.941432in}{1.350005in}}{\pgfqpoint{0.933196in}{1.350005in}}%
\pgfpathcurveto{\pgfqpoint{0.924960in}{1.350005in}}{\pgfqpoint{0.917060in}{1.346733in}}{\pgfqpoint{0.911236in}{1.340909in}}%
\pgfpathcurveto{\pgfqpoint{0.905412in}{1.335085in}}{\pgfqpoint{0.902139in}{1.327185in}}{\pgfqpoint{0.902139in}{1.318949in}}%
\pgfpathcurveto{\pgfqpoint{0.902139in}{1.310712in}}{\pgfqpoint{0.905412in}{1.302812in}}{\pgfqpoint{0.911236in}{1.296988in}}%
\pgfpathcurveto{\pgfqpoint{0.917060in}{1.291164in}}{\pgfqpoint{0.924960in}{1.287892in}}{\pgfqpoint{0.933196in}{1.287892in}}%
\pgfpathclose%
\pgfusepath{stroke,fill}%
\end{pgfscope}%
\begin{pgfscope}%
\pgfpathrectangle{\pgfqpoint{0.100000in}{0.212622in}}{\pgfqpoint{3.696000in}{3.696000in}}%
\pgfusepath{clip}%
\pgfsetbuttcap%
\pgfsetroundjoin%
\definecolor{currentfill}{rgb}{0.121569,0.466667,0.705882}%
\pgfsetfillcolor{currentfill}%
\pgfsetfillopacity{0.588972}%
\pgfsetlinewidth{1.003750pt}%
\definecolor{currentstroke}{rgb}{0.121569,0.466667,0.705882}%
\pgfsetstrokecolor{currentstroke}%
\pgfsetstrokeopacity{0.588972}%
\pgfsetdash{}{0pt}%
\pgfpathmoveto{\pgfqpoint{0.933196in}{1.287892in}}%
\pgfpathcurveto{\pgfqpoint{0.941432in}{1.287892in}}{\pgfqpoint{0.949332in}{1.291164in}}{\pgfqpoint{0.955156in}{1.296988in}}%
\pgfpathcurveto{\pgfqpoint{0.960980in}{1.302812in}}{\pgfqpoint{0.964252in}{1.310712in}}{\pgfqpoint{0.964252in}{1.318949in}}%
\pgfpathcurveto{\pgfqpoint{0.964252in}{1.327185in}}{\pgfqpoint{0.960980in}{1.335085in}}{\pgfqpoint{0.955156in}{1.340909in}}%
\pgfpathcurveto{\pgfqpoint{0.949332in}{1.346733in}}{\pgfqpoint{0.941432in}{1.350005in}}{\pgfqpoint{0.933196in}{1.350005in}}%
\pgfpathcurveto{\pgfqpoint{0.924960in}{1.350005in}}{\pgfqpoint{0.917060in}{1.346733in}}{\pgfqpoint{0.911236in}{1.340909in}}%
\pgfpathcurveto{\pgfqpoint{0.905412in}{1.335085in}}{\pgfqpoint{0.902139in}{1.327185in}}{\pgfqpoint{0.902139in}{1.318949in}}%
\pgfpathcurveto{\pgfqpoint{0.902139in}{1.310712in}}{\pgfqpoint{0.905412in}{1.302812in}}{\pgfqpoint{0.911236in}{1.296988in}}%
\pgfpathcurveto{\pgfqpoint{0.917060in}{1.291164in}}{\pgfqpoint{0.924960in}{1.287892in}}{\pgfqpoint{0.933196in}{1.287892in}}%
\pgfpathclose%
\pgfusepath{stroke,fill}%
\end{pgfscope}%
\begin{pgfscope}%
\pgfpathrectangle{\pgfqpoint{0.100000in}{0.212622in}}{\pgfqpoint{3.696000in}{3.696000in}}%
\pgfusepath{clip}%
\pgfsetbuttcap%
\pgfsetroundjoin%
\definecolor{currentfill}{rgb}{0.121569,0.466667,0.705882}%
\pgfsetfillcolor{currentfill}%
\pgfsetfillopacity{0.588972}%
\pgfsetlinewidth{1.003750pt}%
\definecolor{currentstroke}{rgb}{0.121569,0.466667,0.705882}%
\pgfsetstrokecolor{currentstroke}%
\pgfsetstrokeopacity{0.588972}%
\pgfsetdash{}{0pt}%
\pgfpathmoveto{\pgfqpoint{0.933196in}{1.287892in}}%
\pgfpathcurveto{\pgfqpoint{0.941432in}{1.287892in}}{\pgfqpoint{0.949332in}{1.291164in}}{\pgfqpoint{0.955156in}{1.296988in}}%
\pgfpathcurveto{\pgfqpoint{0.960980in}{1.302812in}}{\pgfqpoint{0.964252in}{1.310712in}}{\pgfqpoint{0.964252in}{1.318949in}}%
\pgfpathcurveto{\pgfqpoint{0.964252in}{1.327185in}}{\pgfqpoint{0.960980in}{1.335085in}}{\pgfqpoint{0.955156in}{1.340909in}}%
\pgfpathcurveto{\pgfqpoint{0.949332in}{1.346733in}}{\pgfqpoint{0.941432in}{1.350005in}}{\pgfqpoint{0.933196in}{1.350005in}}%
\pgfpathcurveto{\pgfqpoint{0.924960in}{1.350005in}}{\pgfqpoint{0.917060in}{1.346733in}}{\pgfqpoint{0.911236in}{1.340909in}}%
\pgfpathcurveto{\pgfqpoint{0.905412in}{1.335085in}}{\pgfqpoint{0.902139in}{1.327185in}}{\pgfqpoint{0.902139in}{1.318949in}}%
\pgfpathcurveto{\pgfqpoint{0.902139in}{1.310712in}}{\pgfqpoint{0.905412in}{1.302812in}}{\pgfqpoint{0.911236in}{1.296988in}}%
\pgfpathcurveto{\pgfqpoint{0.917060in}{1.291164in}}{\pgfqpoint{0.924960in}{1.287892in}}{\pgfqpoint{0.933196in}{1.287892in}}%
\pgfpathclose%
\pgfusepath{stroke,fill}%
\end{pgfscope}%
\begin{pgfscope}%
\pgfpathrectangle{\pgfqpoint{0.100000in}{0.212622in}}{\pgfqpoint{3.696000in}{3.696000in}}%
\pgfusepath{clip}%
\pgfsetbuttcap%
\pgfsetroundjoin%
\definecolor{currentfill}{rgb}{0.121569,0.466667,0.705882}%
\pgfsetfillcolor{currentfill}%
\pgfsetfillopacity{0.588972}%
\pgfsetlinewidth{1.003750pt}%
\definecolor{currentstroke}{rgb}{0.121569,0.466667,0.705882}%
\pgfsetstrokecolor{currentstroke}%
\pgfsetstrokeopacity{0.588972}%
\pgfsetdash{}{0pt}%
\pgfpathmoveto{\pgfqpoint{0.933196in}{1.287892in}}%
\pgfpathcurveto{\pgfqpoint{0.941432in}{1.287892in}}{\pgfqpoint{0.949332in}{1.291164in}}{\pgfqpoint{0.955156in}{1.296988in}}%
\pgfpathcurveto{\pgfqpoint{0.960980in}{1.302812in}}{\pgfqpoint{0.964252in}{1.310712in}}{\pgfqpoint{0.964252in}{1.318949in}}%
\pgfpathcurveto{\pgfqpoint{0.964252in}{1.327185in}}{\pgfqpoint{0.960980in}{1.335085in}}{\pgfqpoint{0.955156in}{1.340909in}}%
\pgfpathcurveto{\pgfqpoint{0.949332in}{1.346733in}}{\pgfqpoint{0.941432in}{1.350005in}}{\pgfqpoint{0.933196in}{1.350005in}}%
\pgfpathcurveto{\pgfqpoint{0.924960in}{1.350005in}}{\pgfqpoint{0.917060in}{1.346733in}}{\pgfqpoint{0.911236in}{1.340909in}}%
\pgfpathcurveto{\pgfqpoint{0.905412in}{1.335085in}}{\pgfqpoint{0.902139in}{1.327185in}}{\pgfqpoint{0.902139in}{1.318949in}}%
\pgfpathcurveto{\pgfqpoint{0.902139in}{1.310712in}}{\pgfqpoint{0.905412in}{1.302812in}}{\pgfqpoint{0.911236in}{1.296988in}}%
\pgfpathcurveto{\pgfqpoint{0.917060in}{1.291164in}}{\pgfqpoint{0.924960in}{1.287892in}}{\pgfqpoint{0.933196in}{1.287892in}}%
\pgfpathclose%
\pgfusepath{stroke,fill}%
\end{pgfscope}%
\begin{pgfscope}%
\pgfpathrectangle{\pgfqpoint{0.100000in}{0.212622in}}{\pgfqpoint{3.696000in}{3.696000in}}%
\pgfusepath{clip}%
\pgfsetbuttcap%
\pgfsetroundjoin%
\definecolor{currentfill}{rgb}{0.121569,0.466667,0.705882}%
\pgfsetfillcolor{currentfill}%
\pgfsetfillopacity{0.588972}%
\pgfsetlinewidth{1.003750pt}%
\definecolor{currentstroke}{rgb}{0.121569,0.466667,0.705882}%
\pgfsetstrokecolor{currentstroke}%
\pgfsetstrokeopacity{0.588972}%
\pgfsetdash{}{0pt}%
\pgfpathmoveto{\pgfqpoint{0.933196in}{1.287892in}}%
\pgfpathcurveto{\pgfqpoint{0.941432in}{1.287892in}}{\pgfqpoint{0.949332in}{1.291164in}}{\pgfqpoint{0.955156in}{1.296988in}}%
\pgfpathcurveto{\pgfqpoint{0.960980in}{1.302812in}}{\pgfqpoint{0.964252in}{1.310712in}}{\pgfqpoint{0.964252in}{1.318949in}}%
\pgfpathcurveto{\pgfqpoint{0.964252in}{1.327185in}}{\pgfqpoint{0.960980in}{1.335085in}}{\pgfqpoint{0.955156in}{1.340909in}}%
\pgfpathcurveto{\pgfqpoint{0.949332in}{1.346733in}}{\pgfqpoint{0.941432in}{1.350005in}}{\pgfqpoint{0.933196in}{1.350005in}}%
\pgfpathcurveto{\pgfqpoint{0.924960in}{1.350005in}}{\pgfqpoint{0.917060in}{1.346733in}}{\pgfqpoint{0.911236in}{1.340909in}}%
\pgfpathcurveto{\pgfqpoint{0.905412in}{1.335085in}}{\pgfqpoint{0.902139in}{1.327185in}}{\pgfqpoint{0.902139in}{1.318949in}}%
\pgfpathcurveto{\pgfqpoint{0.902139in}{1.310712in}}{\pgfqpoint{0.905412in}{1.302812in}}{\pgfqpoint{0.911236in}{1.296988in}}%
\pgfpathcurveto{\pgfqpoint{0.917060in}{1.291164in}}{\pgfqpoint{0.924960in}{1.287892in}}{\pgfqpoint{0.933196in}{1.287892in}}%
\pgfpathclose%
\pgfusepath{stroke,fill}%
\end{pgfscope}%
\begin{pgfscope}%
\pgfpathrectangle{\pgfqpoint{0.100000in}{0.212622in}}{\pgfqpoint{3.696000in}{3.696000in}}%
\pgfusepath{clip}%
\pgfsetbuttcap%
\pgfsetroundjoin%
\definecolor{currentfill}{rgb}{0.121569,0.466667,0.705882}%
\pgfsetfillcolor{currentfill}%
\pgfsetfillopacity{0.588972}%
\pgfsetlinewidth{1.003750pt}%
\definecolor{currentstroke}{rgb}{0.121569,0.466667,0.705882}%
\pgfsetstrokecolor{currentstroke}%
\pgfsetstrokeopacity{0.588972}%
\pgfsetdash{}{0pt}%
\pgfpathmoveto{\pgfqpoint{0.933196in}{1.287892in}}%
\pgfpathcurveto{\pgfqpoint{0.941432in}{1.287892in}}{\pgfqpoint{0.949332in}{1.291164in}}{\pgfqpoint{0.955156in}{1.296988in}}%
\pgfpathcurveto{\pgfqpoint{0.960980in}{1.302812in}}{\pgfqpoint{0.964252in}{1.310712in}}{\pgfqpoint{0.964252in}{1.318949in}}%
\pgfpathcurveto{\pgfqpoint{0.964252in}{1.327185in}}{\pgfqpoint{0.960980in}{1.335085in}}{\pgfqpoint{0.955156in}{1.340909in}}%
\pgfpathcurveto{\pgfqpoint{0.949332in}{1.346733in}}{\pgfqpoint{0.941432in}{1.350005in}}{\pgfqpoint{0.933196in}{1.350005in}}%
\pgfpathcurveto{\pgfqpoint{0.924960in}{1.350005in}}{\pgfqpoint{0.917060in}{1.346733in}}{\pgfqpoint{0.911236in}{1.340909in}}%
\pgfpathcurveto{\pgfqpoint{0.905412in}{1.335085in}}{\pgfqpoint{0.902139in}{1.327185in}}{\pgfqpoint{0.902139in}{1.318949in}}%
\pgfpathcurveto{\pgfqpoint{0.902139in}{1.310712in}}{\pgfqpoint{0.905412in}{1.302812in}}{\pgfqpoint{0.911236in}{1.296988in}}%
\pgfpathcurveto{\pgfqpoint{0.917060in}{1.291164in}}{\pgfqpoint{0.924960in}{1.287892in}}{\pgfqpoint{0.933196in}{1.287892in}}%
\pgfpathclose%
\pgfusepath{stroke,fill}%
\end{pgfscope}%
\begin{pgfscope}%
\pgfpathrectangle{\pgfqpoint{0.100000in}{0.212622in}}{\pgfqpoint{3.696000in}{3.696000in}}%
\pgfusepath{clip}%
\pgfsetbuttcap%
\pgfsetroundjoin%
\definecolor{currentfill}{rgb}{0.121569,0.466667,0.705882}%
\pgfsetfillcolor{currentfill}%
\pgfsetfillopacity{0.588972}%
\pgfsetlinewidth{1.003750pt}%
\definecolor{currentstroke}{rgb}{0.121569,0.466667,0.705882}%
\pgfsetstrokecolor{currentstroke}%
\pgfsetstrokeopacity{0.588972}%
\pgfsetdash{}{0pt}%
\pgfpathmoveto{\pgfqpoint{0.933196in}{1.287892in}}%
\pgfpathcurveto{\pgfqpoint{0.941432in}{1.287892in}}{\pgfqpoint{0.949332in}{1.291164in}}{\pgfqpoint{0.955156in}{1.296988in}}%
\pgfpathcurveto{\pgfqpoint{0.960980in}{1.302812in}}{\pgfqpoint{0.964252in}{1.310712in}}{\pgfqpoint{0.964252in}{1.318949in}}%
\pgfpathcurveto{\pgfqpoint{0.964252in}{1.327185in}}{\pgfqpoint{0.960980in}{1.335085in}}{\pgfqpoint{0.955156in}{1.340909in}}%
\pgfpathcurveto{\pgfqpoint{0.949332in}{1.346733in}}{\pgfqpoint{0.941432in}{1.350005in}}{\pgfqpoint{0.933196in}{1.350005in}}%
\pgfpathcurveto{\pgfqpoint{0.924960in}{1.350005in}}{\pgfqpoint{0.917060in}{1.346733in}}{\pgfqpoint{0.911236in}{1.340909in}}%
\pgfpathcurveto{\pgfqpoint{0.905412in}{1.335085in}}{\pgfqpoint{0.902139in}{1.327185in}}{\pgfqpoint{0.902139in}{1.318949in}}%
\pgfpathcurveto{\pgfqpoint{0.902139in}{1.310712in}}{\pgfqpoint{0.905412in}{1.302812in}}{\pgfqpoint{0.911236in}{1.296988in}}%
\pgfpathcurveto{\pgfqpoint{0.917060in}{1.291164in}}{\pgfqpoint{0.924960in}{1.287892in}}{\pgfqpoint{0.933196in}{1.287892in}}%
\pgfpathclose%
\pgfusepath{stroke,fill}%
\end{pgfscope}%
\begin{pgfscope}%
\pgfpathrectangle{\pgfqpoint{0.100000in}{0.212622in}}{\pgfqpoint{3.696000in}{3.696000in}}%
\pgfusepath{clip}%
\pgfsetbuttcap%
\pgfsetroundjoin%
\definecolor{currentfill}{rgb}{0.121569,0.466667,0.705882}%
\pgfsetfillcolor{currentfill}%
\pgfsetfillopacity{0.588972}%
\pgfsetlinewidth{1.003750pt}%
\definecolor{currentstroke}{rgb}{0.121569,0.466667,0.705882}%
\pgfsetstrokecolor{currentstroke}%
\pgfsetstrokeopacity{0.588972}%
\pgfsetdash{}{0pt}%
\pgfpathmoveto{\pgfqpoint{0.933196in}{1.287892in}}%
\pgfpathcurveto{\pgfqpoint{0.941432in}{1.287892in}}{\pgfqpoint{0.949332in}{1.291164in}}{\pgfqpoint{0.955156in}{1.296988in}}%
\pgfpathcurveto{\pgfqpoint{0.960980in}{1.302812in}}{\pgfqpoint{0.964252in}{1.310712in}}{\pgfqpoint{0.964252in}{1.318949in}}%
\pgfpathcurveto{\pgfqpoint{0.964252in}{1.327185in}}{\pgfqpoint{0.960980in}{1.335085in}}{\pgfqpoint{0.955156in}{1.340909in}}%
\pgfpathcurveto{\pgfqpoint{0.949332in}{1.346733in}}{\pgfqpoint{0.941432in}{1.350005in}}{\pgfqpoint{0.933196in}{1.350005in}}%
\pgfpathcurveto{\pgfqpoint{0.924960in}{1.350005in}}{\pgfqpoint{0.917060in}{1.346733in}}{\pgfqpoint{0.911236in}{1.340909in}}%
\pgfpathcurveto{\pgfqpoint{0.905412in}{1.335085in}}{\pgfqpoint{0.902139in}{1.327185in}}{\pgfqpoint{0.902139in}{1.318949in}}%
\pgfpathcurveto{\pgfqpoint{0.902139in}{1.310712in}}{\pgfqpoint{0.905412in}{1.302812in}}{\pgfqpoint{0.911236in}{1.296988in}}%
\pgfpathcurveto{\pgfqpoint{0.917060in}{1.291164in}}{\pgfqpoint{0.924960in}{1.287892in}}{\pgfqpoint{0.933196in}{1.287892in}}%
\pgfpathclose%
\pgfusepath{stroke,fill}%
\end{pgfscope}%
\begin{pgfscope}%
\pgfpathrectangle{\pgfqpoint{0.100000in}{0.212622in}}{\pgfqpoint{3.696000in}{3.696000in}}%
\pgfusepath{clip}%
\pgfsetbuttcap%
\pgfsetroundjoin%
\definecolor{currentfill}{rgb}{0.121569,0.466667,0.705882}%
\pgfsetfillcolor{currentfill}%
\pgfsetfillopacity{0.588972}%
\pgfsetlinewidth{1.003750pt}%
\definecolor{currentstroke}{rgb}{0.121569,0.466667,0.705882}%
\pgfsetstrokecolor{currentstroke}%
\pgfsetstrokeopacity{0.588972}%
\pgfsetdash{}{0pt}%
\pgfpathmoveto{\pgfqpoint{0.933196in}{1.287892in}}%
\pgfpathcurveto{\pgfqpoint{0.941432in}{1.287892in}}{\pgfqpoint{0.949332in}{1.291164in}}{\pgfqpoint{0.955156in}{1.296988in}}%
\pgfpathcurveto{\pgfqpoint{0.960980in}{1.302812in}}{\pgfqpoint{0.964252in}{1.310712in}}{\pgfqpoint{0.964252in}{1.318949in}}%
\pgfpathcurveto{\pgfqpoint{0.964252in}{1.327185in}}{\pgfqpoint{0.960980in}{1.335085in}}{\pgfqpoint{0.955156in}{1.340909in}}%
\pgfpathcurveto{\pgfqpoint{0.949332in}{1.346733in}}{\pgfqpoint{0.941432in}{1.350005in}}{\pgfqpoint{0.933196in}{1.350005in}}%
\pgfpathcurveto{\pgfqpoint{0.924960in}{1.350005in}}{\pgfqpoint{0.917060in}{1.346733in}}{\pgfqpoint{0.911236in}{1.340909in}}%
\pgfpathcurveto{\pgfqpoint{0.905412in}{1.335085in}}{\pgfqpoint{0.902139in}{1.327185in}}{\pgfqpoint{0.902139in}{1.318949in}}%
\pgfpathcurveto{\pgfqpoint{0.902139in}{1.310712in}}{\pgfqpoint{0.905412in}{1.302812in}}{\pgfqpoint{0.911236in}{1.296988in}}%
\pgfpathcurveto{\pgfqpoint{0.917060in}{1.291164in}}{\pgfqpoint{0.924960in}{1.287892in}}{\pgfqpoint{0.933196in}{1.287892in}}%
\pgfpathclose%
\pgfusepath{stroke,fill}%
\end{pgfscope}%
\begin{pgfscope}%
\pgfpathrectangle{\pgfqpoint{0.100000in}{0.212622in}}{\pgfqpoint{3.696000in}{3.696000in}}%
\pgfusepath{clip}%
\pgfsetbuttcap%
\pgfsetroundjoin%
\definecolor{currentfill}{rgb}{0.121569,0.466667,0.705882}%
\pgfsetfillcolor{currentfill}%
\pgfsetfillopacity{0.588972}%
\pgfsetlinewidth{1.003750pt}%
\definecolor{currentstroke}{rgb}{0.121569,0.466667,0.705882}%
\pgfsetstrokecolor{currentstroke}%
\pgfsetstrokeopacity{0.588972}%
\pgfsetdash{}{0pt}%
\pgfpathmoveto{\pgfqpoint{0.933196in}{1.287892in}}%
\pgfpathcurveto{\pgfqpoint{0.941432in}{1.287892in}}{\pgfqpoint{0.949332in}{1.291164in}}{\pgfqpoint{0.955156in}{1.296988in}}%
\pgfpathcurveto{\pgfqpoint{0.960980in}{1.302812in}}{\pgfqpoint{0.964252in}{1.310712in}}{\pgfqpoint{0.964252in}{1.318949in}}%
\pgfpathcurveto{\pgfqpoint{0.964252in}{1.327185in}}{\pgfqpoint{0.960980in}{1.335085in}}{\pgfqpoint{0.955156in}{1.340909in}}%
\pgfpathcurveto{\pgfqpoint{0.949332in}{1.346733in}}{\pgfqpoint{0.941432in}{1.350005in}}{\pgfqpoint{0.933196in}{1.350005in}}%
\pgfpathcurveto{\pgfqpoint{0.924960in}{1.350005in}}{\pgfqpoint{0.917060in}{1.346733in}}{\pgfqpoint{0.911236in}{1.340909in}}%
\pgfpathcurveto{\pgfqpoint{0.905412in}{1.335085in}}{\pgfqpoint{0.902139in}{1.327185in}}{\pgfqpoint{0.902139in}{1.318949in}}%
\pgfpathcurveto{\pgfqpoint{0.902139in}{1.310712in}}{\pgfqpoint{0.905412in}{1.302812in}}{\pgfqpoint{0.911236in}{1.296988in}}%
\pgfpathcurveto{\pgfqpoint{0.917060in}{1.291164in}}{\pgfqpoint{0.924960in}{1.287892in}}{\pgfqpoint{0.933196in}{1.287892in}}%
\pgfpathclose%
\pgfusepath{stroke,fill}%
\end{pgfscope}%
\begin{pgfscope}%
\pgfpathrectangle{\pgfqpoint{0.100000in}{0.212622in}}{\pgfqpoint{3.696000in}{3.696000in}}%
\pgfusepath{clip}%
\pgfsetbuttcap%
\pgfsetroundjoin%
\definecolor{currentfill}{rgb}{0.121569,0.466667,0.705882}%
\pgfsetfillcolor{currentfill}%
\pgfsetfillopacity{0.588972}%
\pgfsetlinewidth{1.003750pt}%
\definecolor{currentstroke}{rgb}{0.121569,0.466667,0.705882}%
\pgfsetstrokecolor{currentstroke}%
\pgfsetstrokeopacity{0.588972}%
\pgfsetdash{}{0pt}%
\pgfpathmoveto{\pgfqpoint{0.933196in}{1.287892in}}%
\pgfpathcurveto{\pgfqpoint{0.941432in}{1.287892in}}{\pgfqpoint{0.949332in}{1.291164in}}{\pgfqpoint{0.955156in}{1.296988in}}%
\pgfpathcurveto{\pgfqpoint{0.960980in}{1.302812in}}{\pgfqpoint{0.964252in}{1.310712in}}{\pgfqpoint{0.964252in}{1.318949in}}%
\pgfpathcurveto{\pgfqpoint{0.964252in}{1.327185in}}{\pgfqpoint{0.960980in}{1.335085in}}{\pgfqpoint{0.955156in}{1.340909in}}%
\pgfpathcurveto{\pgfqpoint{0.949332in}{1.346733in}}{\pgfqpoint{0.941432in}{1.350005in}}{\pgfqpoint{0.933196in}{1.350005in}}%
\pgfpathcurveto{\pgfqpoint{0.924960in}{1.350005in}}{\pgfqpoint{0.917060in}{1.346733in}}{\pgfqpoint{0.911236in}{1.340909in}}%
\pgfpathcurveto{\pgfqpoint{0.905412in}{1.335085in}}{\pgfqpoint{0.902139in}{1.327185in}}{\pgfqpoint{0.902139in}{1.318949in}}%
\pgfpathcurveto{\pgfqpoint{0.902139in}{1.310712in}}{\pgfqpoint{0.905412in}{1.302812in}}{\pgfqpoint{0.911236in}{1.296988in}}%
\pgfpathcurveto{\pgfqpoint{0.917060in}{1.291164in}}{\pgfqpoint{0.924960in}{1.287892in}}{\pgfqpoint{0.933196in}{1.287892in}}%
\pgfpathclose%
\pgfusepath{stroke,fill}%
\end{pgfscope}%
\begin{pgfscope}%
\pgfpathrectangle{\pgfqpoint{0.100000in}{0.212622in}}{\pgfqpoint{3.696000in}{3.696000in}}%
\pgfusepath{clip}%
\pgfsetbuttcap%
\pgfsetroundjoin%
\definecolor{currentfill}{rgb}{0.121569,0.466667,0.705882}%
\pgfsetfillcolor{currentfill}%
\pgfsetfillopacity{0.588972}%
\pgfsetlinewidth{1.003750pt}%
\definecolor{currentstroke}{rgb}{0.121569,0.466667,0.705882}%
\pgfsetstrokecolor{currentstroke}%
\pgfsetstrokeopacity{0.588972}%
\pgfsetdash{}{0pt}%
\pgfpathmoveto{\pgfqpoint{0.933196in}{1.287892in}}%
\pgfpathcurveto{\pgfqpoint{0.941432in}{1.287892in}}{\pgfqpoint{0.949332in}{1.291164in}}{\pgfqpoint{0.955156in}{1.296988in}}%
\pgfpathcurveto{\pgfqpoint{0.960980in}{1.302812in}}{\pgfqpoint{0.964252in}{1.310712in}}{\pgfqpoint{0.964252in}{1.318949in}}%
\pgfpathcurveto{\pgfqpoint{0.964252in}{1.327185in}}{\pgfqpoint{0.960980in}{1.335085in}}{\pgfqpoint{0.955156in}{1.340909in}}%
\pgfpathcurveto{\pgfqpoint{0.949332in}{1.346733in}}{\pgfqpoint{0.941432in}{1.350005in}}{\pgfqpoint{0.933196in}{1.350005in}}%
\pgfpathcurveto{\pgfqpoint{0.924960in}{1.350005in}}{\pgfqpoint{0.917060in}{1.346733in}}{\pgfqpoint{0.911236in}{1.340909in}}%
\pgfpathcurveto{\pgfqpoint{0.905412in}{1.335085in}}{\pgfqpoint{0.902139in}{1.327185in}}{\pgfqpoint{0.902139in}{1.318949in}}%
\pgfpathcurveto{\pgfqpoint{0.902139in}{1.310712in}}{\pgfqpoint{0.905412in}{1.302812in}}{\pgfqpoint{0.911236in}{1.296988in}}%
\pgfpathcurveto{\pgfqpoint{0.917060in}{1.291164in}}{\pgfqpoint{0.924960in}{1.287892in}}{\pgfqpoint{0.933196in}{1.287892in}}%
\pgfpathclose%
\pgfusepath{stroke,fill}%
\end{pgfscope}%
\begin{pgfscope}%
\pgfpathrectangle{\pgfqpoint{0.100000in}{0.212622in}}{\pgfqpoint{3.696000in}{3.696000in}}%
\pgfusepath{clip}%
\pgfsetbuttcap%
\pgfsetroundjoin%
\definecolor{currentfill}{rgb}{0.121569,0.466667,0.705882}%
\pgfsetfillcolor{currentfill}%
\pgfsetfillopacity{0.588972}%
\pgfsetlinewidth{1.003750pt}%
\definecolor{currentstroke}{rgb}{0.121569,0.466667,0.705882}%
\pgfsetstrokecolor{currentstroke}%
\pgfsetstrokeopacity{0.588972}%
\pgfsetdash{}{0pt}%
\pgfpathmoveto{\pgfqpoint{0.933196in}{1.287892in}}%
\pgfpathcurveto{\pgfqpoint{0.941432in}{1.287892in}}{\pgfqpoint{0.949332in}{1.291164in}}{\pgfqpoint{0.955156in}{1.296988in}}%
\pgfpathcurveto{\pgfqpoint{0.960980in}{1.302812in}}{\pgfqpoint{0.964252in}{1.310712in}}{\pgfqpoint{0.964252in}{1.318949in}}%
\pgfpathcurveto{\pgfqpoint{0.964252in}{1.327185in}}{\pgfqpoint{0.960980in}{1.335085in}}{\pgfqpoint{0.955156in}{1.340909in}}%
\pgfpathcurveto{\pgfqpoint{0.949332in}{1.346733in}}{\pgfqpoint{0.941432in}{1.350005in}}{\pgfqpoint{0.933196in}{1.350005in}}%
\pgfpathcurveto{\pgfqpoint{0.924960in}{1.350005in}}{\pgfqpoint{0.917060in}{1.346733in}}{\pgfqpoint{0.911236in}{1.340909in}}%
\pgfpathcurveto{\pgfqpoint{0.905412in}{1.335085in}}{\pgfqpoint{0.902139in}{1.327185in}}{\pgfqpoint{0.902139in}{1.318949in}}%
\pgfpathcurveto{\pgfqpoint{0.902139in}{1.310712in}}{\pgfqpoint{0.905412in}{1.302812in}}{\pgfqpoint{0.911236in}{1.296988in}}%
\pgfpathcurveto{\pgfqpoint{0.917060in}{1.291164in}}{\pgfqpoint{0.924960in}{1.287892in}}{\pgfqpoint{0.933196in}{1.287892in}}%
\pgfpathclose%
\pgfusepath{stroke,fill}%
\end{pgfscope}%
\begin{pgfscope}%
\pgfpathrectangle{\pgfqpoint{0.100000in}{0.212622in}}{\pgfqpoint{3.696000in}{3.696000in}}%
\pgfusepath{clip}%
\pgfsetbuttcap%
\pgfsetroundjoin%
\definecolor{currentfill}{rgb}{0.121569,0.466667,0.705882}%
\pgfsetfillcolor{currentfill}%
\pgfsetfillopacity{0.588972}%
\pgfsetlinewidth{1.003750pt}%
\definecolor{currentstroke}{rgb}{0.121569,0.466667,0.705882}%
\pgfsetstrokecolor{currentstroke}%
\pgfsetstrokeopacity{0.588972}%
\pgfsetdash{}{0pt}%
\pgfpathmoveto{\pgfqpoint{0.933196in}{1.287892in}}%
\pgfpathcurveto{\pgfqpoint{0.941432in}{1.287892in}}{\pgfqpoint{0.949332in}{1.291164in}}{\pgfqpoint{0.955156in}{1.296988in}}%
\pgfpathcurveto{\pgfqpoint{0.960980in}{1.302812in}}{\pgfqpoint{0.964252in}{1.310712in}}{\pgfqpoint{0.964252in}{1.318949in}}%
\pgfpathcurveto{\pgfqpoint{0.964252in}{1.327185in}}{\pgfqpoint{0.960980in}{1.335085in}}{\pgfqpoint{0.955156in}{1.340909in}}%
\pgfpathcurveto{\pgfqpoint{0.949332in}{1.346733in}}{\pgfqpoint{0.941432in}{1.350005in}}{\pgfqpoint{0.933196in}{1.350005in}}%
\pgfpathcurveto{\pgfqpoint{0.924960in}{1.350005in}}{\pgfqpoint{0.917060in}{1.346733in}}{\pgfqpoint{0.911236in}{1.340909in}}%
\pgfpathcurveto{\pgfqpoint{0.905412in}{1.335085in}}{\pgfqpoint{0.902139in}{1.327185in}}{\pgfqpoint{0.902139in}{1.318949in}}%
\pgfpathcurveto{\pgfqpoint{0.902139in}{1.310712in}}{\pgfqpoint{0.905412in}{1.302812in}}{\pgfqpoint{0.911236in}{1.296988in}}%
\pgfpathcurveto{\pgfqpoint{0.917060in}{1.291164in}}{\pgfqpoint{0.924960in}{1.287892in}}{\pgfqpoint{0.933196in}{1.287892in}}%
\pgfpathclose%
\pgfusepath{stroke,fill}%
\end{pgfscope}%
\begin{pgfscope}%
\pgfpathrectangle{\pgfqpoint{0.100000in}{0.212622in}}{\pgfqpoint{3.696000in}{3.696000in}}%
\pgfusepath{clip}%
\pgfsetbuttcap%
\pgfsetroundjoin%
\definecolor{currentfill}{rgb}{0.121569,0.466667,0.705882}%
\pgfsetfillcolor{currentfill}%
\pgfsetfillopacity{0.588972}%
\pgfsetlinewidth{1.003750pt}%
\definecolor{currentstroke}{rgb}{0.121569,0.466667,0.705882}%
\pgfsetstrokecolor{currentstroke}%
\pgfsetstrokeopacity{0.588972}%
\pgfsetdash{}{0pt}%
\pgfpathmoveto{\pgfqpoint{0.933196in}{1.287892in}}%
\pgfpathcurveto{\pgfqpoint{0.941432in}{1.287892in}}{\pgfqpoint{0.949332in}{1.291164in}}{\pgfqpoint{0.955156in}{1.296988in}}%
\pgfpathcurveto{\pgfqpoint{0.960980in}{1.302812in}}{\pgfqpoint{0.964252in}{1.310712in}}{\pgfqpoint{0.964252in}{1.318949in}}%
\pgfpathcurveto{\pgfqpoint{0.964252in}{1.327185in}}{\pgfqpoint{0.960980in}{1.335085in}}{\pgfqpoint{0.955156in}{1.340909in}}%
\pgfpathcurveto{\pgfqpoint{0.949332in}{1.346733in}}{\pgfqpoint{0.941432in}{1.350005in}}{\pgfqpoint{0.933196in}{1.350005in}}%
\pgfpathcurveto{\pgfqpoint{0.924960in}{1.350005in}}{\pgfqpoint{0.917060in}{1.346733in}}{\pgfqpoint{0.911236in}{1.340909in}}%
\pgfpathcurveto{\pgfqpoint{0.905412in}{1.335085in}}{\pgfqpoint{0.902139in}{1.327185in}}{\pgfqpoint{0.902139in}{1.318949in}}%
\pgfpathcurveto{\pgfqpoint{0.902139in}{1.310712in}}{\pgfqpoint{0.905412in}{1.302812in}}{\pgfqpoint{0.911236in}{1.296988in}}%
\pgfpathcurveto{\pgfqpoint{0.917060in}{1.291164in}}{\pgfqpoint{0.924960in}{1.287892in}}{\pgfqpoint{0.933196in}{1.287892in}}%
\pgfpathclose%
\pgfusepath{stroke,fill}%
\end{pgfscope}%
\begin{pgfscope}%
\pgfpathrectangle{\pgfqpoint{0.100000in}{0.212622in}}{\pgfqpoint{3.696000in}{3.696000in}}%
\pgfusepath{clip}%
\pgfsetbuttcap%
\pgfsetroundjoin%
\definecolor{currentfill}{rgb}{0.121569,0.466667,0.705882}%
\pgfsetfillcolor{currentfill}%
\pgfsetfillopacity{0.588972}%
\pgfsetlinewidth{1.003750pt}%
\definecolor{currentstroke}{rgb}{0.121569,0.466667,0.705882}%
\pgfsetstrokecolor{currentstroke}%
\pgfsetstrokeopacity{0.588972}%
\pgfsetdash{}{0pt}%
\pgfpathmoveto{\pgfqpoint{0.933196in}{1.287892in}}%
\pgfpathcurveto{\pgfqpoint{0.941432in}{1.287892in}}{\pgfqpoint{0.949332in}{1.291164in}}{\pgfqpoint{0.955156in}{1.296988in}}%
\pgfpathcurveto{\pgfqpoint{0.960980in}{1.302812in}}{\pgfqpoint{0.964252in}{1.310712in}}{\pgfqpoint{0.964252in}{1.318949in}}%
\pgfpathcurveto{\pgfqpoint{0.964252in}{1.327185in}}{\pgfqpoint{0.960980in}{1.335085in}}{\pgfqpoint{0.955156in}{1.340909in}}%
\pgfpathcurveto{\pgfqpoint{0.949332in}{1.346733in}}{\pgfqpoint{0.941432in}{1.350005in}}{\pgfqpoint{0.933196in}{1.350005in}}%
\pgfpathcurveto{\pgfqpoint{0.924960in}{1.350005in}}{\pgfqpoint{0.917060in}{1.346733in}}{\pgfqpoint{0.911236in}{1.340909in}}%
\pgfpathcurveto{\pgfqpoint{0.905412in}{1.335085in}}{\pgfqpoint{0.902139in}{1.327185in}}{\pgfqpoint{0.902139in}{1.318949in}}%
\pgfpathcurveto{\pgfqpoint{0.902139in}{1.310712in}}{\pgfqpoint{0.905412in}{1.302812in}}{\pgfqpoint{0.911236in}{1.296988in}}%
\pgfpathcurveto{\pgfqpoint{0.917060in}{1.291164in}}{\pgfqpoint{0.924960in}{1.287892in}}{\pgfqpoint{0.933196in}{1.287892in}}%
\pgfpathclose%
\pgfusepath{stroke,fill}%
\end{pgfscope}%
\begin{pgfscope}%
\pgfpathrectangle{\pgfqpoint{0.100000in}{0.212622in}}{\pgfqpoint{3.696000in}{3.696000in}}%
\pgfusepath{clip}%
\pgfsetbuttcap%
\pgfsetroundjoin%
\definecolor{currentfill}{rgb}{0.121569,0.466667,0.705882}%
\pgfsetfillcolor{currentfill}%
\pgfsetfillopacity{0.588972}%
\pgfsetlinewidth{1.003750pt}%
\definecolor{currentstroke}{rgb}{0.121569,0.466667,0.705882}%
\pgfsetstrokecolor{currentstroke}%
\pgfsetstrokeopacity{0.588972}%
\pgfsetdash{}{0pt}%
\pgfpathmoveto{\pgfqpoint{0.933196in}{1.287892in}}%
\pgfpathcurveto{\pgfqpoint{0.941432in}{1.287892in}}{\pgfqpoint{0.949332in}{1.291164in}}{\pgfqpoint{0.955156in}{1.296988in}}%
\pgfpathcurveto{\pgfqpoint{0.960980in}{1.302812in}}{\pgfqpoint{0.964252in}{1.310712in}}{\pgfqpoint{0.964252in}{1.318949in}}%
\pgfpathcurveto{\pgfqpoint{0.964252in}{1.327185in}}{\pgfqpoint{0.960980in}{1.335085in}}{\pgfqpoint{0.955156in}{1.340909in}}%
\pgfpathcurveto{\pgfqpoint{0.949332in}{1.346733in}}{\pgfqpoint{0.941432in}{1.350005in}}{\pgfqpoint{0.933196in}{1.350005in}}%
\pgfpathcurveto{\pgfqpoint{0.924960in}{1.350005in}}{\pgfqpoint{0.917060in}{1.346733in}}{\pgfqpoint{0.911236in}{1.340909in}}%
\pgfpathcurveto{\pgfqpoint{0.905412in}{1.335085in}}{\pgfqpoint{0.902139in}{1.327185in}}{\pgfqpoint{0.902139in}{1.318949in}}%
\pgfpathcurveto{\pgfqpoint{0.902139in}{1.310712in}}{\pgfqpoint{0.905412in}{1.302812in}}{\pgfqpoint{0.911236in}{1.296988in}}%
\pgfpathcurveto{\pgfqpoint{0.917060in}{1.291164in}}{\pgfqpoint{0.924960in}{1.287892in}}{\pgfqpoint{0.933196in}{1.287892in}}%
\pgfpathclose%
\pgfusepath{stroke,fill}%
\end{pgfscope}%
\begin{pgfscope}%
\pgfpathrectangle{\pgfqpoint{0.100000in}{0.212622in}}{\pgfqpoint{3.696000in}{3.696000in}}%
\pgfusepath{clip}%
\pgfsetbuttcap%
\pgfsetroundjoin%
\definecolor{currentfill}{rgb}{0.121569,0.466667,0.705882}%
\pgfsetfillcolor{currentfill}%
\pgfsetfillopacity{0.588972}%
\pgfsetlinewidth{1.003750pt}%
\definecolor{currentstroke}{rgb}{0.121569,0.466667,0.705882}%
\pgfsetstrokecolor{currentstroke}%
\pgfsetstrokeopacity{0.588972}%
\pgfsetdash{}{0pt}%
\pgfpathmoveto{\pgfqpoint{0.933196in}{1.287892in}}%
\pgfpathcurveto{\pgfqpoint{0.941432in}{1.287892in}}{\pgfqpoint{0.949332in}{1.291164in}}{\pgfqpoint{0.955156in}{1.296988in}}%
\pgfpathcurveto{\pgfqpoint{0.960980in}{1.302812in}}{\pgfqpoint{0.964252in}{1.310712in}}{\pgfqpoint{0.964252in}{1.318949in}}%
\pgfpathcurveto{\pgfqpoint{0.964252in}{1.327185in}}{\pgfqpoint{0.960980in}{1.335085in}}{\pgfqpoint{0.955156in}{1.340909in}}%
\pgfpathcurveto{\pgfqpoint{0.949332in}{1.346733in}}{\pgfqpoint{0.941432in}{1.350005in}}{\pgfqpoint{0.933196in}{1.350005in}}%
\pgfpathcurveto{\pgfqpoint{0.924960in}{1.350005in}}{\pgfqpoint{0.917060in}{1.346733in}}{\pgfqpoint{0.911236in}{1.340909in}}%
\pgfpathcurveto{\pgfqpoint{0.905412in}{1.335085in}}{\pgfqpoint{0.902139in}{1.327185in}}{\pgfqpoint{0.902139in}{1.318949in}}%
\pgfpathcurveto{\pgfqpoint{0.902139in}{1.310712in}}{\pgfqpoint{0.905412in}{1.302812in}}{\pgfqpoint{0.911236in}{1.296988in}}%
\pgfpathcurveto{\pgfqpoint{0.917060in}{1.291164in}}{\pgfqpoint{0.924960in}{1.287892in}}{\pgfqpoint{0.933196in}{1.287892in}}%
\pgfpathclose%
\pgfusepath{stroke,fill}%
\end{pgfscope}%
\begin{pgfscope}%
\pgfpathrectangle{\pgfqpoint{0.100000in}{0.212622in}}{\pgfqpoint{3.696000in}{3.696000in}}%
\pgfusepath{clip}%
\pgfsetbuttcap%
\pgfsetroundjoin%
\definecolor{currentfill}{rgb}{0.121569,0.466667,0.705882}%
\pgfsetfillcolor{currentfill}%
\pgfsetfillopacity{0.588972}%
\pgfsetlinewidth{1.003750pt}%
\definecolor{currentstroke}{rgb}{0.121569,0.466667,0.705882}%
\pgfsetstrokecolor{currentstroke}%
\pgfsetstrokeopacity{0.588972}%
\pgfsetdash{}{0pt}%
\pgfpathmoveto{\pgfqpoint{0.933196in}{1.287892in}}%
\pgfpathcurveto{\pgfqpoint{0.941432in}{1.287892in}}{\pgfqpoint{0.949332in}{1.291164in}}{\pgfqpoint{0.955156in}{1.296988in}}%
\pgfpathcurveto{\pgfqpoint{0.960980in}{1.302812in}}{\pgfqpoint{0.964252in}{1.310712in}}{\pgfqpoint{0.964252in}{1.318949in}}%
\pgfpathcurveto{\pgfqpoint{0.964252in}{1.327185in}}{\pgfqpoint{0.960980in}{1.335085in}}{\pgfqpoint{0.955156in}{1.340909in}}%
\pgfpathcurveto{\pgfqpoint{0.949332in}{1.346733in}}{\pgfqpoint{0.941432in}{1.350005in}}{\pgfqpoint{0.933196in}{1.350005in}}%
\pgfpathcurveto{\pgfqpoint{0.924960in}{1.350005in}}{\pgfqpoint{0.917060in}{1.346733in}}{\pgfqpoint{0.911236in}{1.340909in}}%
\pgfpathcurveto{\pgfqpoint{0.905412in}{1.335085in}}{\pgfqpoint{0.902139in}{1.327185in}}{\pgfqpoint{0.902139in}{1.318949in}}%
\pgfpathcurveto{\pgfqpoint{0.902139in}{1.310712in}}{\pgfqpoint{0.905412in}{1.302812in}}{\pgfqpoint{0.911236in}{1.296988in}}%
\pgfpathcurveto{\pgfqpoint{0.917060in}{1.291164in}}{\pgfqpoint{0.924960in}{1.287892in}}{\pgfqpoint{0.933196in}{1.287892in}}%
\pgfpathclose%
\pgfusepath{stroke,fill}%
\end{pgfscope}%
\begin{pgfscope}%
\pgfpathrectangle{\pgfqpoint{0.100000in}{0.212622in}}{\pgfqpoint{3.696000in}{3.696000in}}%
\pgfusepath{clip}%
\pgfsetbuttcap%
\pgfsetroundjoin%
\definecolor{currentfill}{rgb}{0.121569,0.466667,0.705882}%
\pgfsetfillcolor{currentfill}%
\pgfsetfillopacity{0.588972}%
\pgfsetlinewidth{1.003750pt}%
\definecolor{currentstroke}{rgb}{0.121569,0.466667,0.705882}%
\pgfsetstrokecolor{currentstroke}%
\pgfsetstrokeopacity{0.588972}%
\pgfsetdash{}{0pt}%
\pgfpathmoveto{\pgfqpoint{0.933196in}{1.287892in}}%
\pgfpathcurveto{\pgfqpoint{0.941432in}{1.287892in}}{\pgfqpoint{0.949332in}{1.291164in}}{\pgfqpoint{0.955156in}{1.296988in}}%
\pgfpathcurveto{\pgfqpoint{0.960980in}{1.302812in}}{\pgfqpoint{0.964252in}{1.310712in}}{\pgfqpoint{0.964252in}{1.318949in}}%
\pgfpathcurveto{\pgfqpoint{0.964252in}{1.327185in}}{\pgfqpoint{0.960980in}{1.335085in}}{\pgfqpoint{0.955156in}{1.340909in}}%
\pgfpathcurveto{\pgfqpoint{0.949332in}{1.346733in}}{\pgfqpoint{0.941432in}{1.350005in}}{\pgfqpoint{0.933196in}{1.350005in}}%
\pgfpathcurveto{\pgfqpoint{0.924960in}{1.350005in}}{\pgfqpoint{0.917060in}{1.346733in}}{\pgfqpoint{0.911236in}{1.340909in}}%
\pgfpathcurveto{\pgfqpoint{0.905412in}{1.335085in}}{\pgfqpoint{0.902139in}{1.327185in}}{\pgfqpoint{0.902139in}{1.318949in}}%
\pgfpathcurveto{\pgfqpoint{0.902139in}{1.310712in}}{\pgfqpoint{0.905412in}{1.302812in}}{\pgfqpoint{0.911236in}{1.296988in}}%
\pgfpathcurveto{\pgfqpoint{0.917060in}{1.291164in}}{\pgfqpoint{0.924960in}{1.287892in}}{\pgfqpoint{0.933196in}{1.287892in}}%
\pgfpathclose%
\pgfusepath{stroke,fill}%
\end{pgfscope}%
\begin{pgfscope}%
\pgfpathrectangle{\pgfqpoint{0.100000in}{0.212622in}}{\pgfqpoint{3.696000in}{3.696000in}}%
\pgfusepath{clip}%
\pgfsetbuttcap%
\pgfsetroundjoin%
\definecolor{currentfill}{rgb}{0.121569,0.466667,0.705882}%
\pgfsetfillcolor{currentfill}%
\pgfsetfillopacity{0.588972}%
\pgfsetlinewidth{1.003750pt}%
\definecolor{currentstroke}{rgb}{0.121569,0.466667,0.705882}%
\pgfsetstrokecolor{currentstroke}%
\pgfsetstrokeopacity{0.588972}%
\pgfsetdash{}{0pt}%
\pgfpathmoveto{\pgfqpoint{0.933196in}{1.287892in}}%
\pgfpathcurveto{\pgfqpoint{0.941432in}{1.287892in}}{\pgfqpoint{0.949332in}{1.291164in}}{\pgfqpoint{0.955156in}{1.296988in}}%
\pgfpathcurveto{\pgfqpoint{0.960980in}{1.302812in}}{\pgfqpoint{0.964252in}{1.310712in}}{\pgfqpoint{0.964252in}{1.318949in}}%
\pgfpathcurveto{\pgfqpoint{0.964252in}{1.327185in}}{\pgfqpoint{0.960980in}{1.335085in}}{\pgfqpoint{0.955156in}{1.340909in}}%
\pgfpathcurveto{\pgfqpoint{0.949332in}{1.346733in}}{\pgfqpoint{0.941432in}{1.350005in}}{\pgfqpoint{0.933196in}{1.350005in}}%
\pgfpathcurveto{\pgfqpoint{0.924960in}{1.350005in}}{\pgfqpoint{0.917060in}{1.346733in}}{\pgfqpoint{0.911236in}{1.340909in}}%
\pgfpathcurveto{\pgfqpoint{0.905412in}{1.335085in}}{\pgfqpoint{0.902139in}{1.327185in}}{\pgfqpoint{0.902139in}{1.318949in}}%
\pgfpathcurveto{\pgfqpoint{0.902139in}{1.310712in}}{\pgfqpoint{0.905412in}{1.302812in}}{\pgfqpoint{0.911236in}{1.296988in}}%
\pgfpathcurveto{\pgfqpoint{0.917060in}{1.291164in}}{\pgfqpoint{0.924960in}{1.287892in}}{\pgfqpoint{0.933196in}{1.287892in}}%
\pgfpathclose%
\pgfusepath{stroke,fill}%
\end{pgfscope}%
\begin{pgfscope}%
\pgfpathrectangle{\pgfqpoint{0.100000in}{0.212622in}}{\pgfqpoint{3.696000in}{3.696000in}}%
\pgfusepath{clip}%
\pgfsetbuttcap%
\pgfsetroundjoin%
\definecolor{currentfill}{rgb}{0.121569,0.466667,0.705882}%
\pgfsetfillcolor{currentfill}%
\pgfsetfillopacity{0.588972}%
\pgfsetlinewidth{1.003750pt}%
\definecolor{currentstroke}{rgb}{0.121569,0.466667,0.705882}%
\pgfsetstrokecolor{currentstroke}%
\pgfsetstrokeopacity{0.588972}%
\pgfsetdash{}{0pt}%
\pgfpathmoveto{\pgfqpoint{0.933196in}{1.287892in}}%
\pgfpathcurveto{\pgfqpoint{0.941432in}{1.287892in}}{\pgfqpoint{0.949332in}{1.291164in}}{\pgfqpoint{0.955156in}{1.296988in}}%
\pgfpathcurveto{\pgfqpoint{0.960980in}{1.302812in}}{\pgfqpoint{0.964252in}{1.310712in}}{\pgfqpoint{0.964252in}{1.318949in}}%
\pgfpathcurveto{\pgfqpoint{0.964252in}{1.327185in}}{\pgfqpoint{0.960980in}{1.335085in}}{\pgfqpoint{0.955156in}{1.340909in}}%
\pgfpathcurveto{\pgfqpoint{0.949332in}{1.346733in}}{\pgfqpoint{0.941432in}{1.350005in}}{\pgfqpoint{0.933196in}{1.350005in}}%
\pgfpathcurveto{\pgfqpoint{0.924960in}{1.350005in}}{\pgfqpoint{0.917060in}{1.346733in}}{\pgfqpoint{0.911236in}{1.340909in}}%
\pgfpathcurveto{\pgfqpoint{0.905412in}{1.335085in}}{\pgfqpoint{0.902139in}{1.327185in}}{\pgfqpoint{0.902139in}{1.318949in}}%
\pgfpathcurveto{\pgfqpoint{0.902139in}{1.310712in}}{\pgfqpoint{0.905412in}{1.302812in}}{\pgfqpoint{0.911236in}{1.296988in}}%
\pgfpathcurveto{\pgfqpoint{0.917060in}{1.291164in}}{\pgfqpoint{0.924960in}{1.287892in}}{\pgfqpoint{0.933196in}{1.287892in}}%
\pgfpathclose%
\pgfusepath{stroke,fill}%
\end{pgfscope}%
\begin{pgfscope}%
\pgfpathrectangle{\pgfqpoint{0.100000in}{0.212622in}}{\pgfqpoint{3.696000in}{3.696000in}}%
\pgfusepath{clip}%
\pgfsetbuttcap%
\pgfsetroundjoin%
\definecolor{currentfill}{rgb}{0.121569,0.466667,0.705882}%
\pgfsetfillcolor{currentfill}%
\pgfsetfillopacity{0.588972}%
\pgfsetlinewidth{1.003750pt}%
\definecolor{currentstroke}{rgb}{0.121569,0.466667,0.705882}%
\pgfsetstrokecolor{currentstroke}%
\pgfsetstrokeopacity{0.588972}%
\pgfsetdash{}{0pt}%
\pgfpathmoveto{\pgfqpoint{0.933196in}{1.287892in}}%
\pgfpathcurveto{\pgfqpoint{0.941432in}{1.287892in}}{\pgfqpoint{0.949332in}{1.291164in}}{\pgfqpoint{0.955156in}{1.296988in}}%
\pgfpathcurveto{\pgfqpoint{0.960980in}{1.302812in}}{\pgfqpoint{0.964252in}{1.310712in}}{\pgfqpoint{0.964252in}{1.318949in}}%
\pgfpathcurveto{\pgfqpoint{0.964252in}{1.327185in}}{\pgfqpoint{0.960980in}{1.335085in}}{\pgfqpoint{0.955156in}{1.340909in}}%
\pgfpathcurveto{\pgfqpoint{0.949332in}{1.346733in}}{\pgfqpoint{0.941432in}{1.350005in}}{\pgfqpoint{0.933196in}{1.350005in}}%
\pgfpathcurveto{\pgfqpoint{0.924960in}{1.350005in}}{\pgfqpoint{0.917060in}{1.346733in}}{\pgfqpoint{0.911236in}{1.340909in}}%
\pgfpathcurveto{\pgfqpoint{0.905412in}{1.335085in}}{\pgfqpoint{0.902139in}{1.327185in}}{\pgfqpoint{0.902139in}{1.318949in}}%
\pgfpathcurveto{\pgfqpoint{0.902139in}{1.310712in}}{\pgfqpoint{0.905412in}{1.302812in}}{\pgfqpoint{0.911236in}{1.296988in}}%
\pgfpathcurveto{\pgfqpoint{0.917060in}{1.291164in}}{\pgfqpoint{0.924960in}{1.287892in}}{\pgfqpoint{0.933196in}{1.287892in}}%
\pgfpathclose%
\pgfusepath{stroke,fill}%
\end{pgfscope}%
\begin{pgfscope}%
\pgfpathrectangle{\pgfqpoint{0.100000in}{0.212622in}}{\pgfqpoint{3.696000in}{3.696000in}}%
\pgfusepath{clip}%
\pgfsetbuttcap%
\pgfsetroundjoin%
\definecolor{currentfill}{rgb}{0.121569,0.466667,0.705882}%
\pgfsetfillcolor{currentfill}%
\pgfsetfillopacity{0.588972}%
\pgfsetlinewidth{1.003750pt}%
\definecolor{currentstroke}{rgb}{0.121569,0.466667,0.705882}%
\pgfsetstrokecolor{currentstroke}%
\pgfsetstrokeopacity{0.588972}%
\pgfsetdash{}{0pt}%
\pgfpathmoveto{\pgfqpoint{0.933196in}{1.287892in}}%
\pgfpathcurveto{\pgfqpoint{0.941432in}{1.287892in}}{\pgfqpoint{0.949332in}{1.291164in}}{\pgfqpoint{0.955156in}{1.296988in}}%
\pgfpathcurveto{\pgfqpoint{0.960980in}{1.302812in}}{\pgfqpoint{0.964252in}{1.310712in}}{\pgfqpoint{0.964252in}{1.318949in}}%
\pgfpathcurveto{\pgfqpoint{0.964252in}{1.327185in}}{\pgfqpoint{0.960980in}{1.335085in}}{\pgfqpoint{0.955156in}{1.340909in}}%
\pgfpathcurveto{\pgfqpoint{0.949332in}{1.346733in}}{\pgfqpoint{0.941432in}{1.350005in}}{\pgfqpoint{0.933196in}{1.350005in}}%
\pgfpathcurveto{\pgfqpoint{0.924960in}{1.350005in}}{\pgfqpoint{0.917060in}{1.346733in}}{\pgfqpoint{0.911236in}{1.340909in}}%
\pgfpathcurveto{\pgfqpoint{0.905412in}{1.335085in}}{\pgfqpoint{0.902139in}{1.327185in}}{\pgfqpoint{0.902139in}{1.318949in}}%
\pgfpathcurveto{\pgfqpoint{0.902139in}{1.310712in}}{\pgfqpoint{0.905412in}{1.302812in}}{\pgfqpoint{0.911236in}{1.296988in}}%
\pgfpathcurveto{\pgfqpoint{0.917060in}{1.291164in}}{\pgfqpoint{0.924960in}{1.287892in}}{\pgfqpoint{0.933196in}{1.287892in}}%
\pgfpathclose%
\pgfusepath{stroke,fill}%
\end{pgfscope}%
\begin{pgfscope}%
\pgfpathrectangle{\pgfqpoint{0.100000in}{0.212622in}}{\pgfqpoint{3.696000in}{3.696000in}}%
\pgfusepath{clip}%
\pgfsetbuttcap%
\pgfsetroundjoin%
\definecolor{currentfill}{rgb}{0.121569,0.466667,0.705882}%
\pgfsetfillcolor{currentfill}%
\pgfsetfillopacity{0.588972}%
\pgfsetlinewidth{1.003750pt}%
\definecolor{currentstroke}{rgb}{0.121569,0.466667,0.705882}%
\pgfsetstrokecolor{currentstroke}%
\pgfsetstrokeopacity{0.588972}%
\pgfsetdash{}{0pt}%
\pgfpathmoveto{\pgfqpoint{0.933196in}{1.287892in}}%
\pgfpathcurveto{\pgfqpoint{0.941432in}{1.287892in}}{\pgfqpoint{0.949332in}{1.291164in}}{\pgfqpoint{0.955156in}{1.296988in}}%
\pgfpathcurveto{\pgfqpoint{0.960980in}{1.302812in}}{\pgfqpoint{0.964252in}{1.310712in}}{\pgfqpoint{0.964252in}{1.318949in}}%
\pgfpathcurveto{\pgfqpoint{0.964252in}{1.327185in}}{\pgfqpoint{0.960980in}{1.335085in}}{\pgfqpoint{0.955156in}{1.340909in}}%
\pgfpathcurveto{\pgfqpoint{0.949332in}{1.346733in}}{\pgfqpoint{0.941432in}{1.350005in}}{\pgfqpoint{0.933196in}{1.350005in}}%
\pgfpathcurveto{\pgfqpoint{0.924960in}{1.350005in}}{\pgfqpoint{0.917060in}{1.346733in}}{\pgfqpoint{0.911236in}{1.340909in}}%
\pgfpathcurveto{\pgfqpoint{0.905412in}{1.335085in}}{\pgfqpoint{0.902139in}{1.327185in}}{\pgfqpoint{0.902139in}{1.318949in}}%
\pgfpathcurveto{\pgfqpoint{0.902139in}{1.310712in}}{\pgfqpoint{0.905412in}{1.302812in}}{\pgfqpoint{0.911236in}{1.296988in}}%
\pgfpathcurveto{\pgfqpoint{0.917060in}{1.291164in}}{\pgfqpoint{0.924960in}{1.287892in}}{\pgfqpoint{0.933196in}{1.287892in}}%
\pgfpathclose%
\pgfusepath{stroke,fill}%
\end{pgfscope}%
\begin{pgfscope}%
\pgfpathrectangle{\pgfqpoint{0.100000in}{0.212622in}}{\pgfqpoint{3.696000in}{3.696000in}}%
\pgfusepath{clip}%
\pgfsetbuttcap%
\pgfsetroundjoin%
\definecolor{currentfill}{rgb}{0.121569,0.466667,0.705882}%
\pgfsetfillcolor{currentfill}%
\pgfsetfillopacity{0.588972}%
\pgfsetlinewidth{1.003750pt}%
\definecolor{currentstroke}{rgb}{0.121569,0.466667,0.705882}%
\pgfsetstrokecolor{currentstroke}%
\pgfsetstrokeopacity{0.588972}%
\pgfsetdash{}{0pt}%
\pgfpathmoveto{\pgfqpoint{0.933196in}{1.287892in}}%
\pgfpathcurveto{\pgfqpoint{0.941432in}{1.287892in}}{\pgfqpoint{0.949332in}{1.291164in}}{\pgfqpoint{0.955156in}{1.296988in}}%
\pgfpathcurveto{\pgfqpoint{0.960980in}{1.302812in}}{\pgfqpoint{0.964252in}{1.310712in}}{\pgfqpoint{0.964252in}{1.318949in}}%
\pgfpathcurveto{\pgfqpoint{0.964252in}{1.327185in}}{\pgfqpoint{0.960980in}{1.335085in}}{\pgfqpoint{0.955156in}{1.340909in}}%
\pgfpathcurveto{\pgfqpoint{0.949332in}{1.346733in}}{\pgfqpoint{0.941432in}{1.350005in}}{\pgfqpoint{0.933196in}{1.350005in}}%
\pgfpathcurveto{\pgfqpoint{0.924960in}{1.350005in}}{\pgfqpoint{0.917060in}{1.346733in}}{\pgfqpoint{0.911236in}{1.340909in}}%
\pgfpathcurveto{\pgfqpoint{0.905412in}{1.335085in}}{\pgfqpoint{0.902139in}{1.327185in}}{\pgfqpoint{0.902139in}{1.318949in}}%
\pgfpathcurveto{\pgfqpoint{0.902139in}{1.310712in}}{\pgfqpoint{0.905412in}{1.302812in}}{\pgfqpoint{0.911236in}{1.296988in}}%
\pgfpathcurveto{\pgfqpoint{0.917060in}{1.291164in}}{\pgfqpoint{0.924960in}{1.287892in}}{\pgfqpoint{0.933196in}{1.287892in}}%
\pgfpathclose%
\pgfusepath{stroke,fill}%
\end{pgfscope}%
\begin{pgfscope}%
\pgfpathrectangle{\pgfqpoint{0.100000in}{0.212622in}}{\pgfqpoint{3.696000in}{3.696000in}}%
\pgfusepath{clip}%
\pgfsetbuttcap%
\pgfsetroundjoin%
\definecolor{currentfill}{rgb}{0.121569,0.466667,0.705882}%
\pgfsetfillcolor{currentfill}%
\pgfsetfillopacity{0.588972}%
\pgfsetlinewidth{1.003750pt}%
\definecolor{currentstroke}{rgb}{0.121569,0.466667,0.705882}%
\pgfsetstrokecolor{currentstroke}%
\pgfsetstrokeopacity{0.588972}%
\pgfsetdash{}{0pt}%
\pgfpathmoveto{\pgfqpoint{0.933196in}{1.287892in}}%
\pgfpathcurveto{\pgfqpoint{0.941432in}{1.287892in}}{\pgfqpoint{0.949332in}{1.291164in}}{\pgfqpoint{0.955156in}{1.296988in}}%
\pgfpathcurveto{\pgfqpoint{0.960980in}{1.302812in}}{\pgfqpoint{0.964252in}{1.310712in}}{\pgfqpoint{0.964252in}{1.318949in}}%
\pgfpathcurveto{\pgfqpoint{0.964252in}{1.327185in}}{\pgfqpoint{0.960980in}{1.335085in}}{\pgfqpoint{0.955156in}{1.340909in}}%
\pgfpathcurveto{\pgfqpoint{0.949332in}{1.346733in}}{\pgfqpoint{0.941432in}{1.350005in}}{\pgfqpoint{0.933196in}{1.350005in}}%
\pgfpathcurveto{\pgfqpoint{0.924960in}{1.350005in}}{\pgfqpoint{0.917060in}{1.346733in}}{\pgfqpoint{0.911236in}{1.340909in}}%
\pgfpathcurveto{\pgfqpoint{0.905412in}{1.335085in}}{\pgfqpoint{0.902139in}{1.327185in}}{\pgfqpoint{0.902139in}{1.318949in}}%
\pgfpathcurveto{\pgfqpoint{0.902139in}{1.310712in}}{\pgfqpoint{0.905412in}{1.302812in}}{\pgfqpoint{0.911236in}{1.296988in}}%
\pgfpathcurveto{\pgfqpoint{0.917060in}{1.291164in}}{\pgfqpoint{0.924960in}{1.287892in}}{\pgfqpoint{0.933196in}{1.287892in}}%
\pgfpathclose%
\pgfusepath{stroke,fill}%
\end{pgfscope}%
\begin{pgfscope}%
\pgfpathrectangle{\pgfqpoint{0.100000in}{0.212622in}}{\pgfqpoint{3.696000in}{3.696000in}}%
\pgfusepath{clip}%
\pgfsetbuttcap%
\pgfsetroundjoin%
\definecolor{currentfill}{rgb}{0.121569,0.466667,0.705882}%
\pgfsetfillcolor{currentfill}%
\pgfsetfillopacity{0.588972}%
\pgfsetlinewidth{1.003750pt}%
\definecolor{currentstroke}{rgb}{0.121569,0.466667,0.705882}%
\pgfsetstrokecolor{currentstroke}%
\pgfsetstrokeopacity{0.588972}%
\pgfsetdash{}{0pt}%
\pgfpathmoveto{\pgfqpoint{0.933196in}{1.287892in}}%
\pgfpathcurveto{\pgfqpoint{0.941432in}{1.287892in}}{\pgfqpoint{0.949332in}{1.291164in}}{\pgfqpoint{0.955156in}{1.296988in}}%
\pgfpathcurveto{\pgfqpoint{0.960980in}{1.302812in}}{\pgfqpoint{0.964252in}{1.310712in}}{\pgfqpoint{0.964252in}{1.318949in}}%
\pgfpathcurveto{\pgfqpoint{0.964252in}{1.327185in}}{\pgfqpoint{0.960980in}{1.335085in}}{\pgfqpoint{0.955156in}{1.340909in}}%
\pgfpathcurveto{\pgfqpoint{0.949332in}{1.346733in}}{\pgfqpoint{0.941432in}{1.350005in}}{\pgfqpoint{0.933196in}{1.350005in}}%
\pgfpathcurveto{\pgfqpoint{0.924960in}{1.350005in}}{\pgfqpoint{0.917060in}{1.346733in}}{\pgfqpoint{0.911236in}{1.340909in}}%
\pgfpathcurveto{\pgfqpoint{0.905412in}{1.335085in}}{\pgfqpoint{0.902139in}{1.327185in}}{\pgfqpoint{0.902139in}{1.318949in}}%
\pgfpathcurveto{\pgfqpoint{0.902139in}{1.310712in}}{\pgfqpoint{0.905412in}{1.302812in}}{\pgfqpoint{0.911236in}{1.296988in}}%
\pgfpathcurveto{\pgfqpoint{0.917060in}{1.291164in}}{\pgfqpoint{0.924960in}{1.287892in}}{\pgfqpoint{0.933196in}{1.287892in}}%
\pgfpathclose%
\pgfusepath{stroke,fill}%
\end{pgfscope}%
\begin{pgfscope}%
\pgfpathrectangle{\pgfqpoint{0.100000in}{0.212622in}}{\pgfqpoint{3.696000in}{3.696000in}}%
\pgfusepath{clip}%
\pgfsetbuttcap%
\pgfsetroundjoin%
\definecolor{currentfill}{rgb}{0.121569,0.466667,0.705882}%
\pgfsetfillcolor{currentfill}%
\pgfsetfillopacity{0.588972}%
\pgfsetlinewidth{1.003750pt}%
\definecolor{currentstroke}{rgb}{0.121569,0.466667,0.705882}%
\pgfsetstrokecolor{currentstroke}%
\pgfsetstrokeopacity{0.588972}%
\pgfsetdash{}{0pt}%
\pgfpathmoveto{\pgfqpoint{0.933196in}{1.287892in}}%
\pgfpathcurveto{\pgfqpoint{0.941432in}{1.287892in}}{\pgfqpoint{0.949332in}{1.291164in}}{\pgfqpoint{0.955156in}{1.296988in}}%
\pgfpathcurveto{\pgfqpoint{0.960980in}{1.302812in}}{\pgfqpoint{0.964252in}{1.310712in}}{\pgfqpoint{0.964252in}{1.318949in}}%
\pgfpathcurveto{\pgfqpoint{0.964252in}{1.327185in}}{\pgfqpoint{0.960980in}{1.335085in}}{\pgfqpoint{0.955156in}{1.340909in}}%
\pgfpathcurveto{\pgfqpoint{0.949332in}{1.346733in}}{\pgfqpoint{0.941432in}{1.350005in}}{\pgfqpoint{0.933196in}{1.350005in}}%
\pgfpathcurveto{\pgfqpoint{0.924960in}{1.350005in}}{\pgfqpoint{0.917060in}{1.346733in}}{\pgfqpoint{0.911236in}{1.340909in}}%
\pgfpathcurveto{\pgfqpoint{0.905412in}{1.335085in}}{\pgfqpoint{0.902139in}{1.327185in}}{\pgfqpoint{0.902139in}{1.318949in}}%
\pgfpathcurveto{\pgfqpoint{0.902139in}{1.310712in}}{\pgfqpoint{0.905412in}{1.302812in}}{\pgfqpoint{0.911236in}{1.296988in}}%
\pgfpathcurveto{\pgfqpoint{0.917060in}{1.291164in}}{\pgfqpoint{0.924960in}{1.287892in}}{\pgfqpoint{0.933196in}{1.287892in}}%
\pgfpathclose%
\pgfusepath{stroke,fill}%
\end{pgfscope}%
\begin{pgfscope}%
\pgfpathrectangle{\pgfqpoint{0.100000in}{0.212622in}}{\pgfqpoint{3.696000in}{3.696000in}}%
\pgfusepath{clip}%
\pgfsetbuttcap%
\pgfsetroundjoin%
\definecolor{currentfill}{rgb}{0.121569,0.466667,0.705882}%
\pgfsetfillcolor{currentfill}%
\pgfsetfillopacity{0.588972}%
\pgfsetlinewidth{1.003750pt}%
\definecolor{currentstroke}{rgb}{0.121569,0.466667,0.705882}%
\pgfsetstrokecolor{currentstroke}%
\pgfsetstrokeopacity{0.588972}%
\pgfsetdash{}{0pt}%
\pgfpathmoveto{\pgfqpoint{0.933196in}{1.287892in}}%
\pgfpathcurveto{\pgfqpoint{0.941432in}{1.287892in}}{\pgfqpoint{0.949332in}{1.291164in}}{\pgfqpoint{0.955156in}{1.296988in}}%
\pgfpathcurveto{\pgfqpoint{0.960980in}{1.302812in}}{\pgfqpoint{0.964252in}{1.310712in}}{\pgfqpoint{0.964252in}{1.318949in}}%
\pgfpathcurveto{\pgfqpoint{0.964252in}{1.327185in}}{\pgfqpoint{0.960980in}{1.335085in}}{\pgfqpoint{0.955156in}{1.340909in}}%
\pgfpathcurveto{\pgfqpoint{0.949332in}{1.346733in}}{\pgfqpoint{0.941432in}{1.350005in}}{\pgfqpoint{0.933196in}{1.350005in}}%
\pgfpathcurveto{\pgfqpoint{0.924960in}{1.350005in}}{\pgfqpoint{0.917060in}{1.346733in}}{\pgfqpoint{0.911236in}{1.340909in}}%
\pgfpathcurveto{\pgfqpoint{0.905412in}{1.335085in}}{\pgfqpoint{0.902139in}{1.327185in}}{\pgfqpoint{0.902139in}{1.318949in}}%
\pgfpathcurveto{\pgfqpoint{0.902139in}{1.310712in}}{\pgfqpoint{0.905412in}{1.302812in}}{\pgfqpoint{0.911236in}{1.296988in}}%
\pgfpathcurveto{\pgfqpoint{0.917060in}{1.291164in}}{\pgfqpoint{0.924960in}{1.287892in}}{\pgfqpoint{0.933196in}{1.287892in}}%
\pgfpathclose%
\pgfusepath{stroke,fill}%
\end{pgfscope}%
\begin{pgfscope}%
\pgfpathrectangle{\pgfqpoint{0.100000in}{0.212622in}}{\pgfqpoint{3.696000in}{3.696000in}}%
\pgfusepath{clip}%
\pgfsetbuttcap%
\pgfsetroundjoin%
\definecolor{currentfill}{rgb}{0.121569,0.466667,0.705882}%
\pgfsetfillcolor{currentfill}%
\pgfsetfillopacity{0.588972}%
\pgfsetlinewidth{1.003750pt}%
\definecolor{currentstroke}{rgb}{0.121569,0.466667,0.705882}%
\pgfsetstrokecolor{currentstroke}%
\pgfsetstrokeopacity{0.588972}%
\pgfsetdash{}{0pt}%
\pgfpathmoveto{\pgfqpoint{0.933196in}{1.287892in}}%
\pgfpathcurveto{\pgfqpoint{0.941432in}{1.287892in}}{\pgfqpoint{0.949332in}{1.291164in}}{\pgfqpoint{0.955156in}{1.296988in}}%
\pgfpathcurveto{\pgfqpoint{0.960980in}{1.302812in}}{\pgfqpoint{0.964252in}{1.310712in}}{\pgfqpoint{0.964252in}{1.318949in}}%
\pgfpathcurveto{\pgfqpoint{0.964252in}{1.327185in}}{\pgfqpoint{0.960980in}{1.335085in}}{\pgfqpoint{0.955156in}{1.340909in}}%
\pgfpathcurveto{\pgfqpoint{0.949332in}{1.346733in}}{\pgfqpoint{0.941432in}{1.350005in}}{\pgfqpoint{0.933196in}{1.350005in}}%
\pgfpathcurveto{\pgfqpoint{0.924960in}{1.350005in}}{\pgfqpoint{0.917060in}{1.346733in}}{\pgfqpoint{0.911236in}{1.340909in}}%
\pgfpathcurveto{\pgfqpoint{0.905412in}{1.335085in}}{\pgfqpoint{0.902139in}{1.327185in}}{\pgfqpoint{0.902139in}{1.318949in}}%
\pgfpathcurveto{\pgfqpoint{0.902139in}{1.310712in}}{\pgfqpoint{0.905412in}{1.302812in}}{\pgfqpoint{0.911236in}{1.296988in}}%
\pgfpathcurveto{\pgfqpoint{0.917060in}{1.291164in}}{\pgfqpoint{0.924960in}{1.287892in}}{\pgfqpoint{0.933196in}{1.287892in}}%
\pgfpathclose%
\pgfusepath{stroke,fill}%
\end{pgfscope}%
\begin{pgfscope}%
\pgfpathrectangle{\pgfqpoint{0.100000in}{0.212622in}}{\pgfqpoint{3.696000in}{3.696000in}}%
\pgfusepath{clip}%
\pgfsetbuttcap%
\pgfsetroundjoin%
\definecolor{currentfill}{rgb}{0.121569,0.466667,0.705882}%
\pgfsetfillcolor{currentfill}%
\pgfsetfillopacity{0.588972}%
\pgfsetlinewidth{1.003750pt}%
\definecolor{currentstroke}{rgb}{0.121569,0.466667,0.705882}%
\pgfsetstrokecolor{currentstroke}%
\pgfsetstrokeopacity{0.588972}%
\pgfsetdash{}{0pt}%
\pgfpathmoveto{\pgfqpoint{0.933196in}{1.287892in}}%
\pgfpathcurveto{\pgfqpoint{0.941432in}{1.287892in}}{\pgfqpoint{0.949332in}{1.291164in}}{\pgfqpoint{0.955156in}{1.296988in}}%
\pgfpathcurveto{\pgfqpoint{0.960980in}{1.302812in}}{\pgfqpoint{0.964252in}{1.310712in}}{\pgfqpoint{0.964252in}{1.318949in}}%
\pgfpathcurveto{\pgfqpoint{0.964252in}{1.327185in}}{\pgfqpoint{0.960980in}{1.335085in}}{\pgfqpoint{0.955156in}{1.340909in}}%
\pgfpathcurveto{\pgfqpoint{0.949332in}{1.346733in}}{\pgfqpoint{0.941432in}{1.350005in}}{\pgfqpoint{0.933196in}{1.350005in}}%
\pgfpathcurveto{\pgfqpoint{0.924960in}{1.350005in}}{\pgfqpoint{0.917060in}{1.346733in}}{\pgfqpoint{0.911236in}{1.340909in}}%
\pgfpathcurveto{\pgfqpoint{0.905412in}{1.335085in}}{\pgfqpoint{0.902139in}{1.327185in}}{\pgfqpoint{0.902139in}{1.318949in}}%
\pgfpathcurveto{\pgfqpoint{0.902139in}{1.310712in}}{\pgfqpoint{0.905412in}{1.302812in}}{\pgfqpoint{0.911236in}{1.296988in}}%
\pgfpathcurveto{\pgfqpoint{0.917060in}{1.291164in}}{\pgfqpoint{0.924960in}{1.287892in}}{\pgfqpoint{0.933196in}{1.287892in}}%
\pgfpathclose%
\pgfusepath{stroke,fill}%
\end{pgfscope}%
\begin{pgfscope}%
\pgfpathrectangle{\pgfqpoint{0.100000in}{0.212622in}}{\pgfqpoint{3.696000in}{3.696000in}}%
\pgfusepath{clip}%
\pgfsetbuttcap%
\pgfsetroundjoin%
\definecolor{currentfill}{rgb}{0.121569,0.466667,0.705882}%
\pgfsetfillcolor{currentfill}%
\pgfsetfillopacity{0.588972}%
\pgfsetlinewidth{1.003750pt}%
\definecolor{currentstroke}{rgb}{0.121569,0.466667,0.705882}%
\pgfsetstrokecolor{currentstroke}%
\pgfsetstrokeopacity{0.588972}%
\pgfsetdash{}{0pt}%
\pgfpathmoveto{\pgfqpoint{0.933196in}{1.287892in}}%
\pgfpathcurveto{\pgfqpoint{0.941432in}{1.287892in}}{\pgfqpoint{0.949332in}{1.291164in}}{\pgfqpoint{0.955156in}{1.296988in}}%
\pgfpathcurveto{\pgfqpoint{0.960980in}{1.302812in}}{\pgfqpoint{0.964252in}{1.310712in}}{\pgfqpoint{0.964252in}{1.318949in}}%
\pgfpathcurveto{\pgfqpoint{0.964252in}{1.327185in}}{\pgfqpoint{0.960980in}{1.335085in}}{\pgfqpoint{0.955156in}{1.340909in}}%
\pgfpathcurveto{\pgfqpoint{0.949332in}{1.346733in}}{\pgfqpoint{0.941432in}{1.350005in}}{\pgfqpoint{0.933196in}{1.350005in}}%
\pgfpathcurveto{\pgfqpoint{0.924960in}{1.350005in}}{\pgfqpoint{0.917060in}{1.346733in}}{\pgfqpoint{0.911236in}{1.340909in}}%
\pgfpathcurveto{\pgfqpoint{0.905412in}{1.335085in}}{\pgfqpoint{0.902139in}{1.327185in}}{\pgfqpoint{0.902139in}{1.318949in}}%
\pgfpathcurveto{\pgfqpoint{0.902139in}{1.310712in}}{\pgfqpoint{0.905412in}{1.302812in}}{\pgfqpoint{0.911236in}{1.296988in}}%
\pgfpathcurveto{\pgfqpoint{0.917060in}{1.291164in}}{\pgfqpoint{0.924960in}{1.287892in}}{\pgfqpoint{0.933196in}{1.287892in}}%
\pgfpathclose%
\pgfusepath{stroke,fill}%
\end{pgfscope}%
\begin{pgfscope}%
\pgfpathrectangle{\pgfqpoint{0.100000in}{0.212622in}}{\pgfqpoint{3.696000in}{3.696000in}}%
\pgfusepath{clip}%
\pgfsetbuttcap%
\pgfsetroundjoin%
\definecolor{currentfill}{rgb}{0.121569,0.466667,0.705882}%
\pgfsetfillcolor{currentfill}%
\pgfsetfillopacity{0.588972}%
\pgfsetlinewidth{1.003750pt}%
\definecolor{currentstroke}{rgb}{0.121569,0.466667,0.705882}%
\pgfsetstrokecolor{currentstroke}%
\pgfsetstrokeopacity{0.588972}%
\pgfsetdash{}{0pt}%
\pgfpathmoveto{\pgfqpoint{0.933196in}{1.287892in}}%
\pgfpathcurveto{\pgfqpoint{0.941432in}{1.287892in}}{\pgfqpoint{0.949332in}{1.291164in}}{\pgfqpoint{0.955156in}{1.296988in}}%
\pgfpathcurveto{\pgfqpoint{0.960980in}{1.302812in}}{\pgfqpoint{0.964252in}{1.310712in}}{\pgfqpoint{0.964252in}{1.318949in}}%
\pgfpathcurveto{\pgfqpoint{0.964252in}{1.327185in}}{\pgfqpoint{0.960980in}{1.335085in}}{\pgfqpoint{0.955156in}{1.340909in}}%
\pgfpathcurveto{\pgfqpoint{0.949332in}{1.346733in}}{\pgfqpoint{0.941432in}{1.350005in}}{\pgfqpoint{0.933196in}{1.350005in}}%
\pgfpathcurveto{\pgfqpoint{0.924960in}{1.350005in}}{\pgfqpoint{0.917060in}{1.346733in}}{\pgfqpoint{0.911236in}{1.340909in}}%
\pgfpathcurveto{\pgfqpoint{0.905412in}{1.335085in}}{\pgfqpoint{0.902139in}{1.327185in}}{\pgfqpoint{0.902139in}{1.318949in}}%
\pgfpathcurveto{\pgfqpoint{0.902139in}{1.310712in}}{\pgfqpoint{0.905412in}{1.302812in}}{\pgfqpoint{0.911236in}{1.296988in}}%
\pgfpathcurveto{\pgfqpoint{0.917060in}{1.291164in}}{\pgfqpoint{0.924960in}{1.287892in}}{\pgfqpoint{0.933196in}{1.287892in}}%
\pgfpathclose%
\pgfusepath{stroke,fill}%
\end{pgfscope}%
\begin{pgfscope}%
\pgfpathrectangle{\pgfqpoint{0.100000in}{0.212622in}}{\pgfqpoint{3.696000in}{3.696000in}}%
\pgfusepath{clip}%
\pgfsetbuttcap%
\pgfsetroundjoin%
\definecolor{currentfill}{rgb}{0.121569,0.466667,0.705882}%
\pgfsetfillcolor{currentfill}%
\pgfsetfillopacity{0.588972}%
\pgfsetlinewidth{1.003750pt}%
\definecolor{currentstroke}{rgb}{0.121569,0.466667,0.705882}%
\pgfsetstrokecolor{currentstroke}%
\pgfsetstrokeopacity{0.588972}%
\pgfsetdash{}{0pt}%
\pgfpathmoveto{\pgfqpoint{0.933196in}{1.287892in}}%
\pgfpathcurveto{\pgfqpoint{0.941432in}{1.287892in}}{\pgfqpoint{0.949332in}{1.291164in}}{\pgfqpoint{0.955156in}{1.296988in}}%
\pgfpathcurveto{\pgfqpoint{0.960980in}{1.302812in}}{\pgfqpoint{0.964252in}{1.310712in}}{\pgfqpoint{0.964252in}{1.318949in}}%
\pgfpathcurveto{\pgfqpoint{0.964252in}{1.327185in}}{\pgfqpoint{0.960980in}{1.335085in}}{\pgfqpoint{0.955156in}{1.340909in}}%
\pgfpathcurveto{\pgfqpoint{0.949332in}{1.346733in}}{\pgfqpoint{0.941432in}{1.350005in}}{\pgfqpoint{0.933196in}{1.350005in}}%
\pgfpathcurveto{\pgfqpoint{0.924960in}{1.350005in}}{\pgfqpoint{0.917060in}{1.346733in}}{\pgfqpoint{0.911236in}{1.340909in}}%
\pgfpathcurveto{\pgfqpoint{0.905412in}{1.335085in}}{\pgfqpoint{0.902139in}{1.327185in}}{\pgfqpoint{0.902139in}{1.318949in}}%
\pgfpathcurveto{\pgfqpoint{0.902139in}{1.310712in}}{\pgfqpoint{0.905412in}{1.302812in}}{\pgfqpoint{0.911236in}{1.296988in}}%
\pgfpathcurveto{\pgfqpoint{0.917060in}{1.291164in}}{\pgfqpoint{0.924960in}{1.287892in}}{\pgfqpoint{0.933196in}{1.287892in}}%
\pgfpathclose%
\pgfusepath{stroke,fill}%
\end{pgfscope}%
\begin{pgfscope}%
\pgfpathrectangle{\pgfqpoint{0.100000in}{0.212622in}}{\pgfqpoint{3.696000in}{3.696000in}}%
\pgfusepath{clip}%
\pgfsetbuttcap%
\pgfsetroundjoin%
\definecolor{currentfill}{rgb}{0.121569,0.466667,0.705882}%
\pgfsetfillcolor{currentfill}%
\pgfsetfillopacity{0.588972}%
\pgfsetlinewidth{1.003750pt}%
\definecolor{currentstroke}{rgb}{0.121569,0.466667,0.705882}%
\pgfsetstrokecolor{currentstroke}%
\pgfsetstrokeopacity{0.588972}%
\pgfsetdash{}{0pt}%
\pgfpathmoveto{\pgfqpoint{0.933196in}{1.287892in}}%
\pgfpathcurveto{\pgfqpoint{0.941432in}{1.287892in}}{\pgfqpoint{0.949332in}{1.291164in}}{\pgfqpoint{0.955156in}{1.296988in}}%
\pgfpathcurveto{\pgfqpoint{0.960980in}{1.302812in}}{\pgfqpoint{0.964252in}{1.310712in}}{\pgfqpoint{0.964252in}{1.318949in}}%
\pgfpathcurveto{\pgfqpoint{0.964252in}{1.327185in}}{\pgfqpoint{0.960980in}{1.335085in}}{\pgfqpoint{0.955156in}{1.340909in}}%
\pgfpathcurveto{\pgfqpoint{0.949332in}{1.346733in}}{\pgfqpoint{0.941432in}{1.350005in}}{\pgfqpoint{0.933196in}{1.350005in}}%
\pgfpathcurveto{\pgfqpoint{0.924960in}{1.350005in}}{\pgfqpoint{0.917060in}{1.346733in}}{\pgfqpoint{0.911236in}{1.340909in}}%
\pgfpathcurveto{\pgfqpoint{0.905412in}{1.335085in}}{\pgfqpoint{0.902139in}{1.327185in}}{\pgfqpoint{0.902139in}{1.318949in}}%
\pgfpathcurveto{\pgfqpoint{0.902139in}{1.310712in}}{\pgfqpoint{0.905412in}{1.302812in}}{\pgfqpoint{0.911236in}{1.296988in}}%
\pgfpathcurveto{\pgfqpoint{0.917060in}{1.291164in}}{\pgfqpoint{0.924960in}{1.287892in}}{\pgfqpoint{0.933196in}{1.287892in}}%
\pgfpathclose%
\pgfusepath{stroke,fill}%
\end{pgfscope}%
\begin{pgfscope}%
\pgfpathrectangle{\pgfqpoint{0.100000in}{0.212622in}}{\pgfqpoint{3.696000in}{3.696000in}}%
\pgfusepath{clip}%
\pgfsetbuttcap%
\pgfsetroundjoin%
\definecolor{currentfill}{rgb}{0.121569,0.466667,0.705882}%
\pgfsetfillcolor{currentfill}%
\pgfsetfillopacity{0.588972}%
\pgfsetlinewidth{1.003750pt}%
\definecolor{currentstroke}{rgb}{0.121569,0.466667,0.705882}%
\pgfsetstrokecolor{currentstroke}%
\pgfsetstrokeopacity{0.588972}%
\pgfsetdash{}{0pt}%
\pgfpathmoveto{\pgfqpoint{0.933196in}{1.287892in}}%
\pgfpathcurveto{\pgfqpoint{0.941432in}{1.287892in}}{\pgfqpoint{0.949332in}{1.291164in}}{\pgfqpoint{0.955156in}{1.296988in}}%
\pgfpathcurveto{\pgfqpoint{0.960980in}{1.302812in}}{\pgfqpoint{0.964252in}{1.310712in}}{\pgfqpoint{0.964252in}{1.318949in}}%
\pgfpathcurveto{\pgfqpoint{0.964252in}{1.327185in}}{\pgfqpoint{0.960980in}{1.335085in}}{\pgfqpoint{0.955156in}{1.340909in}}%
\pgfpathcurveto{\pgfqpoint{0.949332in}{1.346733in}}{\pgfqpoint{0.941432in}{1.350005in}}{\pgfqpoint{0.933196in}{1.350005in}}%
\pgfpathcurveto{\pgfqpoint{0.924960in}{1.350005in}}{\pgfqpoint{0.917060in}{1.346733in}}{\pgfqpoint{0.911236in}{1.340909in}}%
\pgfpathcurveto{\pgfqpoint{0.905412in}{1.335085in}}{\pgfqpoint{0.902139in}{1.327185in}}{\pgfqpoint{0.902139in}{1.318949in}}%
\pgfpathcurveto{\pgfqpoint{0.902139in}{1.310712in}}{\pgfqpoint{0.905412in}{1.302812in}}{\pgfqpoint{0.911236in}{1.296988in}}%
\pgfpathcurveto{\pgfqpoint{0.917060in}{1.291164in}}{\pgfqpoint{0.924960in}{1.287892in}}{\pgfqpoint{0.933196in}{1.287892in}}%
\pgfpathclose%
\pgfusepath{stroke,fill}%
\end{pgfscope}%
\begin{pgfscope}%
\pgfpathrectangle{\pgfqpoint{0.100000in}{0.212622in}}{\pgfqpoint{3.696000in}{3.696000in}}%
\pgfusepath{clip}%
\pgfsetbuttcap%
\pgfsetroundjoin%
\definecolor{currentfill}{rgb}{0.121569,0.466667,0.705882}%
\pgfsetfillcolor{currentfill}%
\pgfsetfillopacity{0.588972}%
\pgfsetlinewidth{1.003750pt}%
\definecolor{currentstroke}{rgb}{0.121569,0.466667,0.705882}%
\pgfsetstrokecolor{currentstroke}%
\pgfsetstrokeopacity{0.588972}%
\pgfsetdash{}{0pt}%
\pgfpathmoveto{\pgfqpoint{0.933196in}{1.287892in}}%
\pgfpathcurveto{\pgfqpoint{0.941432in}{1.287892in}}{\pgfqpoint{0.949332in}{1.291164in}}{\pgfqpoint{0.955156in}{1.296988in}}%
\pgfpathcurveto{\pgfqpoint{0.960980in}{1.302812in}}{\pgfqpoint{0.964252in}{1.310712in}}{\pgfqpoint{0.964252in}{1.318949in}}%
\pgfpathcurveto{\pgfqpoint{0.964252in}{1.327185in}}{\pgfqpoint{0.960980in}{1.335085in}}{\pgfqpoint{0.955156in}{1.340909in}}%
\pgfpathcurveto{\pgfqpoint{0.949332in}{1.346733in}}{\pgfqpoint{0.941432in}{1.350005in}}{\pgfqpoint{0.933196in}{1.350005in}}%
\pgfpathcurveto{\pgfqpoint{0.924960in}{1.350005in}}{\pgfqpoint{0.917060in}{1.346733in}}{\pgfqpoint{0.911236in}{1.340909in}}%
\pgfpathcurveto{\pgfqpoint{0.905412in}{1.335085in}}{\pgfqpoint{0.902139in}{1.327185in}}{\pgfqpoint{0.902139in}{1.318949in}}%
\pgfpathcurveto{\pgfqpoint{0.902139in}{1.310712in}}{\pgfqpoint{0.905412in}{1.302812in}}{\pgfqpoint{0.911236in}{1.296988in}}%
\pgfpathcurveto{\pgfqpoint{0.917060in}{1.291164in}}{\pgfqpoint{0.924960in}{1.287892in}}{\pgfqpoint{0.933196in}{1.287892in}}%
\pgfpathclose%
\pgfusepath{stroke,fill}%
\end{pgfscope}%
\begin{pgfscope}%
\pgfpathrectangle{\pgfqpoint{0.100000in}{0.212622in}}{\pgfqpoint{3.696000in}{3.696000in}}%
\pgfusepath{clip}%
\pgfsetbuttcap%
\pgfsetroundjoin%
\definecolor{currentfill}{rgb}{0.121569,0.466667,0.705882}%
\pgfsetfillcolor{currentfill}%
\pgfsetfillopacity{0.588972}%
\pgfsetlinewidth{1.003750pt}%
\definecolor{currentstroke}{rgb}{0.121569,0.466667,0.705882}%
\pgfsetstrokecolor{currentstroke}%
\pgfsetstrokeopacity{0.588972}%
\pgfsetdash{}{0pt}%
\pgfpathmoveto{\pgfqpoint{0.933196in}{1.287892in}}%
\pgfpathcurveto{\pgfqpoint{0.941432in}{1.287892in}}{\pgfqpoint{0.949332in}{1.291164in}}{\pgfqpoint{0.955156in}{1.296988in}}%
\pgfpathcurveto{\pgfqpoint{0.960980in}{1.302812in}}{\pgfqpoint{0.964252in}{1.310712in}}{\pgfqpoint{0.964252in}{1.318949in}}%
\pgfpathcurveto{\pgfqpoint{0.964252in}{1.327185in}}{\pgfqpoint{0.960980in}{1.335085in}}{\pgfqpoint{0.955156in}{1.340909in}}%
\pgfpathcurveto{\pgfqpoint{0.949332in}{1.346733in}}{\pgfqpoint{0.941432in}{1.350005in}}{\pgfqpoint{0.933196in}{1.350005in}}%
\pgfpathcurveto{\pgfqpoint{0.924960in}{1.350005in}}{\pgfqpoint{0.917060in}{1.346733in}}{\pgfqpoint{0.911236in}{1.340909in}}%
\pgfpathcurveto{\pgfqpoint{0.905412in}{1.335085in}}{\pgfqpoint{0.902139in}{1.327185in}}{\pgfqpoint{0.902139in}{1.318949in}}%
\pgfpathcurveto{\pgfqpoint{0.902139in}{1.310712in}}{\pgfqpoint{0.905412in}{1.302812in}}{\pgfqpoint{0.911236in}{1.296988in}}%
\pgfpathcurveto{\pgfqpoint{0.917060in}{1.291164in}}{\pgfqpoint{0.924960in}{1.287892in}}{\pgfqpoint{0.933196in}{1.287892in}}%
\pgfpathclose%
\pgfusepath{stroke,fill}%
\end{pgfscope}%
\begin{pgfscope}%
\pgfpathrectangle{\pgfqpoint{0.100000in}{0.212622in}}{\pgfqpoint{3.696000in}{3.696000in}}%
\pgfusepath{clip}%
\pgfsetbuttcap%
\pgfsetroundjoin%
\definecolor{currentfill}{rgb}{0.121569,0.466667,0.705882}%
\pgfsetfillcolor{currentfill}%
\pgfsetfillopacity{0.588972}%
\pgfsetlinewidth{1.003750pt}%
\definecolor{currentstroke}{rgb}{0.121569,0.466667,0.705882}%
\pgfsetstrokecolor{currentstroke}%
\pgfsetstrokeopacity{0.588972}%
\pgfsetdash{}{0pt}%
\pgfpathmoveto{\pgfqpoint{0.933196in}{1.287892in}}%
\pgfpathcurveto{\pgfqpoint{0.941432in}{1.287892in}}{\pgfqpoint{0.949332in}{1.291164in}}{\pgfqpoint{0.955156in}{1.296988in}}%
\pgfpathcurveto{\pgfqpoint{0.960980in}{1.302812in}}{\pgfqpoint{0.964252in}{1.310712in}}{\pgfqpoint{0.964252in}{1.318949in}}%
\pgfpathcurveto{\pgfqpoint{0.964252in}{1.327185in}}{\pgfqpoint{0.960980in}{1.335085in}}{\pgfqpoint{0.955156in}{1.340909in}}%
\pgfpathcurveto{\pgfqpoint{0.949332in}{1.346733in}}{\pgfqpoint{0.941432in}{1.350005in}}{\pgfqpoint{0.933196in}{1.350005in}}%
\pgfpathcurveto{\pgfqpoint{0.924960in}{1.350005in}}{\pgfqpoint{0.917060in}{1.346733in}}{\pgfqpoint{0.911236in}{1.340909in}}%
\pgfpathcurveto{\pgfqpoint{0.905412in}{1.335085in}}{\pgfqpoint{0.902139in}{1.327185in}}{\pgfqpoint{0.902139in}{1.318949in}}%
\pgfpathcurveto{\pgfqpoint{0.902139in}{1.310712in}}{\pgfqpoint{0.905412in}{1.302812in}}{\pgfqpoint{0.911236in}{1.296988in}}%
\pgfpathcurveto{\pgfqpoint{0.917060in}{1.291164in}}{\pgfqpoint{0.924960in}{1.287892in}}{\pgfqpoint{0.933196in}{1.287892in}}%
\pgfpathclose%
\pgfusepath{stroke,fill}%
\end{pgfscope}%
\begin{pgfscope}%
\pgfpathrectangle{\pgfqpoint{0.100000in}{0.212622in}}{\pgfqpoint{3.696000in}{3.696000in}}%
\pgfusepath{clip}%
\pgfsetbuttcap%
\pgfsetroundjoin%
\definecolor{currentfill}{rgb}{0.121569,0.466667,0.705882}%
\pgfsetfillcolor{currentfill}%
\pgfsetfillopacity{0.588972}%
\pgfsetlinewidth{1.003750pt}%
\definecolor{currentstroke}{rgb}{0.121569,0.466667,0.705882}%
\pgfsetstrokecolor{currentstroke}%
\pgfsetstrokeopacity{0.588972}%
\pgfsetdash{}{0pt}%
\pgfpathmoveto{\pgfqpoint{0.933196in}{1.287892in}}%
\pgfpathcurveto{\pgfqpoint{0.941432in}{1.287892in}}{\pgfqpoint{0.949332in}{1.291164in}}{\pgfqpoint{0.955156in}{1.296988in}}%
\pgfpathcurveto{\pgfqpoint{0.960980in}{1.302812in}}{\pgfqpoint{0.964252in}{1.310712in}}{\pgfqpoint{0.964252in}{1.318949in}}%
\pgfpathcurveto{\pgfqpoint{0.964252in}{1.327185in}}{\pgfqpoint{0.960980in}{1.335085in}}{\pgfqpoint{0.955156in}{1.340909in}}%
\pgfpathcurveto{\pgfqpoint{0.949332in}{1.346733in}}{\pgfqpoint{0.941432in}{1.350005in}}{\pgfqpoint{0.933196in}{1.350005in}}%
\pgfpathcurveto{\pgfqpoint{0.924960in}{1.350005in}}{\pgfqpoint{0.917060in}{1.346733in}}{\pgfqpoint{0.911236in}{1.340909in}}%
\pgfpathcurveto{\pgfqpoint{0.905412in}{1.335085in}}{\pgfqpoint{0.902139in}{1.327185in}}{\pgfqpoint{0.902139in}{1.318949in}}%
\pgfpathcurveto{\pgfqpoint{0.902139in}{1.310712in}}{\pgfqpoint{0.905412in}{1.302812in}}{\pgfqpoint{0.911236in}{1.296988in}}%
\pgfpathcurveto{\pgfqpoint{0.917060in}{1.291164in}}{\pgfqpoint{0.924960in}{1.287892in}}{\pgfqpoint{0.933196in}{1.287892in}}%
\pgfpathclose%
\pgfusepath{stroke,fill}%
\end{pgfscope}%
\begin{pgfscope}%
\pgfpathrectangle{\pgfqpoint{0.100000in}{0.212622in}}{\pgfqpoint{3.696000in}{3.696000in}}%
\pgfusepath{clip}%
\pgfsetbuttcap%
\pgfsetroundjoin%
\definecolor{currentfill}{rgb}{0.121569,0.466667,0.705882}%
\pgfsetfillcolor{currentfill}%
\pgfsetfillopacity{0.588972}%
\pgfsetlinewidth{1.003750pt}%
\definecolor{currentstroke}{rgb}{0.121569,0.466667,0.705882}%
\pgfsetstrokecolor{currentstroke}%
\pgfsetstrokeopacity{0.588972}%
\pgfsetdash{}{0pt}%
\pgfpathmoveto{\pgfqpoint{0.933196in}{1.287892in}}%
\pgfpathcurveto{\pgfqpoint{0.941432in}{1.287892in}}{\pgfqpoint{0.949332in}{1.291164in}}{\pgfqpoint{0.955156in}{1.296988in}}%
\pgfpathcurveto{\pgfqpoint{0.960980in}{1.302812in}}{\pgfqpoint{0.964252in}{1.310712in}}{\pgfqpoint{0.964252in}{1.318949in}}%
\pgfpathcurveto{\pgfqpoint{0.964252in}{1.327185in}}{\pgfqpoint{0.960980in}{1.335085in}}{\pgfqpoint{0.955156in}{1.340909in}}%
\pgfpathcurveto{\pgfqpoint{0.949332in}{1.346733in}}{\pgfqpoint{0.941432in}{1.350005in}}{\pgfqpoint{0.933196in}{1.350005in}}%
\pgfpathcurveto{\pgfqpoint{0.924960in}{1.350005in}}{\pgfqpoint{0.917060in}{1.346733in}}{\pgfqpoint{0.911236in}{1.340909in}}%
\pgfpathcurveto{\pgfqpoint{0.905412in}{1.335085in}}{\pgfqpoint{0.902139in}{1.327185in}}{\pgfqpoint{0.902139in}{1.318949in}}%
\pgfpathcurveto{\pgfqpoint{0.902139in}{1.310712in}}{\pgfqpoint{0.905412in}{1.302812in}}{\pgfqpoint{0.911236in}{1.296988in}}%
\pgfpathcurveto{\pgfqpoint{0.917060in}{1.291164in}}{\pgfqpoint{0.924960in}{1.287892in}}{\pgfqpoint{0.933196in}{1.287892in}}%
\pgfpathclose%
\pgfusepath{stroke,fill}%
\end{pgfscope}%
\begin{pgfscope}%
\pgfpathrectangle{\pgfqpoint{0.100000in}{0.212622in}}{\pgfqpoint{3.696000in}{3.696000in}}%
\pgfusepath{clip}%
\pgfsetbuttcap%
\pgfsetroundjoin%
\definecolor{currentfill}{rgb}{0.121569,0.466667,0.705882}%
\pgfsetfillcolor{currentfill}%
\pgfsetfillopacity{0.588972}%
\pgfsetlinewidth{1.003750pt}%
\definecolor{currentstroke}{rgb}{0.121569,0.466667,0.705882}%
\pgfsetstrokecolor{currentstroke}%
\pgfsetstrokeopacity{0.588972}%
\pgfsetdash{}{0pt}%
\pgfpathmoveto{\pgfqpoint{0.933196in}{1.287892in}}%
\pgfpathcurveto{\pgfqpoint{0.941432in}{1.287892in}}{\pgfqpoint{0.949332in}{1.291164in}}{\pgfqpoint{0.955156in}{1.296988in}}%
\pgfpathcurveto{\pgfqpoint{0.960980in}{1.302812in}}{\pgfqpoint{0.964252in}{1.310712in}}{\pgfqpoint{0.964252in}{1.318949in}}%
\pgfpathcurveto{\pgfqpoint{0.964252in}{1.327185in}}{\pgfqpoint{0.960980in}{1.335085in}}{\pgfqpoint{0.955156in}{1.340909in}}%
\pgfpathcurveto{\pgfqpoint{0.949332in}{1.346733in}}{\pgfqpoint{0.941432in}{1.350005in}}{\pgfqpoint{0.933196in}{1.350005in}}%
\pgfpathcurveto{\pgfqpoint{0.924960in}{1.350005in}}{\pgfqpoint{0.917060in}{1.346733in}}{\pgfqpoint{0.911236in}{1.340909in}}%
\pgfpathcurveto{\pgfqpoint{0.905412in}{1.335085in}}{\pgfqpoint{0.902139in}{1.327185in}}{\pgfqpoint{0.902139in}{1.318949in}}%
\pgfpathcurveto{\pgfqpoint{0.902139in}{1.310712in}}{\pgfqpoint{0.905412in}{1.302812in}}{\pgfqpoint{0.911236in}{1.296988in}}%
\pgfpathcurveto{\pgfqpoint{0.917060in}{1.291164in}}{\pgfqpoint{0.924960in}{1.287892in}}{\pgfqpoint{0.933196in}{1.287892in}}%
\pgfpathclose%
\pgfusepath{stroke,fill}%
\end{pgfscope}%
\begin{pgfscope}%
\pgfpathrectangle{\pgfqpoint{0.100000in}{0.212622in}}{\pgfqpoint{3.696000in}{3.696000in}}%
\pgfusepath{clip}%
\pgfsetbuttcap%
\pgfsetroundjoin%
\definecolor{currentfill}{rgb}{0.121569,0.466667,0.705882}%
\pgfsetfillcolor{currentfill}%
\pgfsetfillopacity{0.588972}%
\pgfsetlinewidth{1.003750pt}%
\definecolor{currentstroke}{rgb}{0.121569,0.466667,0.705882}%
\pgfsetstrokecolor{currentstroke}%
\pgfsetstrokeopacity{0.588972}%
\pgfsetdash{}{0pt}%
\pgfpathmoveto{\pgfqpoint{0.933196in}{1.287892in}}%
\pgfpathcurveto{\pgfqpoint{0.941432in}{1.287892in}}{\pgfqpoint{0.949332in}{1.291164in}}{\pgfqpoint{0.955156in}{1.296988in}}%
\pgfpathcurveto{\pgfqpoint{0.960980in}{1.302812in}}{\pgfqpoint{0.964252in}{1.310712in}}{\pgfqpoint{0.964252in}{1.318949in}}%
\pgfpathcurveto{\pgfqpoint{0.964252in}{1.327185in}}{\pgfqpoint{0.960980in}{1.335085in}}{\pgfqpoint{0.955156in}{1.340909in}}%
\pgfpathcurveto{\pgfqpoint{0.949332in}{1.346733in}}{\pgfqpoint{0.941432in}{1.350005in}}{\pgfqpoint{0.933196in}{1.350005in}}%
\pgfpathcurveto{\pgfqpoint{0.924960in}{1.350005in}}{\pgfqpoint{0.917060in}{1.346733in}}{\pgfqpoint{0.911236in}{1.340909in}}%
\pgfpathcurveto{\pgfqpoint{0.905412in}{1.335085in}}{\pgfqpoint{0.902139in}{1.327185in}}{\pgfqpoint{0.902139in}{1.318949in}}%
\pgfpathcurveto{\pgfqpoint{0.902139in}{1.310712in}}{\pgfqpoint{0.905412in}{1.302812in}}{\pgfqpoint{0.911236in}{1.296988in}}%
\pgfpathcurveto{\pgfqpoint{0.917060in}{1.291164in}}{\pgfqpoint{0.924960in}{1.287892in}}{\pgfqpoint{0.933196in}{1.287892in}}%
\pgfpathclose%
\pgfusepath{stroke,fill}%
\end{pgfscope}%
\begin{pgfscope}%
\pgfpathrectangle{\pgfqpoint{0.100000in}{0.212622in}}{\pgfqpoint{3.696000in}{3.696000in}}%
\pgfusepath{clip}%
\pgfsetbuttcap%
\pgfsetroundjoin%
\definecolor{currentfill}{rgb}{0.121569,0.466667,0.705882}%
\pgfsetfillcolor{currentfill}%
\pgfsetfillopacity{0.588972}%
\pgfsetlinewidth{1.003750pt}%
\definecolor{currentstroke}{rgb}{0.121569,0.466667,0.705882}%
\pgfsetstrokecolor{currentstroke}%
\pgfsetstrokeopacity{0.588972}%
\pgfsetdash{}{0pt}%
\pgfpathmoveto{\pgfqpoint{0.933196in}{1.287892in}}%
\pgfpathcurveto{\pgfqpoint{0.941432in}{1.287892in}}{\pgfqpoint{0.949332in}{1.291164in}}{\pgfqpoint{0.955156in}{1.296988in}}%
\pgfpathcurveto{\pgfqpoint{0.960980in}{1.302812in}}{\pgfqpoint{0.964252in}{1.310712in}}{\pgfqpoint{0.964252in}{1.318949in}}%
\pgfpathcurveto{\pgfqpoint{0.964252in}{1.327185in}}{\pgfqpoint{0.960980in}{1.335085in}}{\pgfqpoint{0.955156in}{1.340909in}}%
\pgfpathcurveto{\pgfqpoint{0.949332in}{1.346733in}}{\pgfqpoint{0.941432in}{1.350005in}}{\pgfqpoint{0.933196in}{1.350005in}}%
\pgfpathcurveto{\pgfqpoint{0.924960in}{1.350005in}}{\pgfqpoint{0.917060in}{1.346733in}}{\pgfqpoint{0.911236in}{1.340909in}}%
\pgfpathcurveto{\pgfqpoint{0.905412in}{1.335085in}}{\pgfqpoint{0.902139in}{1.327185in}}{\pgfqpoint{0.902139in}{1.318949in}}%
\pgfpathcurveto{\pgfqpoint{0.902139in}{1.310712in}}{\pgfqpoint{0.905412in}{1.302812in}}{\pgfqpoint{0.911236in}{1.296988in}}%
\pgfpathcurveto{\pgfqpoint{0.917060in}{1.291164in}}{\pgfqpoint{0.924960in}{1.287892in}}{\pgfqpoint{0.933196in}{1.287892in}}%
\pgfpathclose%
\pgfusepath{stroke,fill}%
\end{pgfscope}%
\begin{pgfscope}%
\pgfpathrectangle{\pgfqpoint{0.100000in}{0.212622in}}{\pgfqpoint{3.696000in}{3.696000in}}%
\pgfusepath{clip}%
\pgfsetbuttcap%
\pgfsetroundjoin%
\definecolor{currentfill}{rgb}{0.121569,0.466667,0.705882}%
\pgfsetfillcolor{currentfill}%
\pgfsetfillopacity{0.588972}%
\pgfsetlinewidth{1.003750pt}%
\definecolor{currentstroke}{rgb}{0.121569,0.466667,0.705882}%
\pgfsetstrokecolor{currentstroke}%
\pgfsetstrokeopacity{0.588972}%
\pgfsetdash{}{0pt}%
\pgfpathmoveto{\pgfqpoint{0.933196in}{1.287892in}}%
\pgfpathcurveto{\pgfqpoint{0.941432in}{1.287892in}}{\pgfqpoint{0.949332in}{1.291164in}}{\pgfqpoint{0.955156in}{1.296988in}}%
\pgfpathcurveto{\pgfqpoint{0.960980in}{1.302812in}}{\pgfqpoint{0.964252in}{1.310712in}}{\pgfqpoint{0.964252in}{1.318949in}}%
\pgfpathcurveto{\pgfqpoint{0.964252in}{1.327185in}}{\pgfqpoint{0.960980in}{1.335085in}}{\pgfqpoint{0.955156in}{1.340909in}}%
\pgfpathcurveto{\pgfqpoint{0.949332in}{1.346733in}}{\pgfqpoint{0.941432in}{1.350005in}}{\pgfqpoint{0.933196in}{1.350005in}}%
\pgfpathcurveto{\pgfqpoint{0.924960in}{1.350005in}}{\pgfqpoint{0.917060in}{1.346733in}}{\pgfqpoint{0.911236in}{1.340909in}}%
\pgfpathcurveto{\pgfqpoint{0.905412in}{1.335085in}}{\pgfqpoint{0.902139in}{1.327185in}}{\pgfqpoint{0.902139in}{1.318949in}}%
\pgfpathcurveto{\pgfqpoint{0.902139in}{1.310712in}}{\pgfqpoint{0.905412in}{1.302812in}}{\pgfqpoint{0.911236in}{1.296988in}}%
\pgfpathcurveto{\pgfqpoint{0.917060in}{1.291164in}}{\pgfqpoint{0.924960in}{1.287892in}}{\pgfqpoint{0.933196in}{1.287892in}}%
\pgfpathclose%
\pgfusepath{stroke,fill}%
\end{pgfscope}%
\begin{pgfscope}%
\pgfpathrectangle{\pgfqpoint{0.100000in}{0.212622in}}{\pgfqpoint{3.696000in}{3.696000in}}%
\pgfusepath{clip}%
\pgfsetbuttcap%
\pgfsetroundjoin%
\definecolor{currentfill}{rgb}{0.121569,0.466667,0.705882}%
\pgfsetfillcolor{currentfill}%
\pgfsetfillopacity{0.588972}%
\pgfsetlinewidth{1.003750pt}%
\definecolor{currentstroke}{rgb}{0.121569,0.466667,0.705882}%
\pgfsetstrokecolor{currentstroke}%
\pgfsetstrokeopacity{0.588972}%
\pgfsetdash{}{0pt}%
\pgfpathmoveto{\pgfqpoint{0.933196in}{1.287892in}}%
\pgfpathcurveto{\pgfqpoint{0.941432in}{1.287892in}}{\pgfqpoint{0.949332in}{1.291164in}}{\pgfqpoint{0.955156in}{1.296988in}}%
\pgfpathcurveto{\pgfqpoint{0.960980in}{1.302812in}}{\pgfqpoint{0.964252in}{1.310712in}}{\pgfqpoint{0.964252in}{1.318949in}}%
\pgfpathcurveto{\pgfqpoint{0.964252in}{1.327185in}}{\pgfqpoint{0.960980in}{1.335085in}}{\pgfqpoint{0.955156in}{1.340909in}}%
\pgfpathcurveto{\pgfqpoint{0.949332in}{1.346733in}}{\pgfqpoint{0.941432in}{1.350005in}}{\pgfqpoint{0.933196in}{1.350005in}}%
\pgfpathcurveto{\pgfqpoint{0.924960in}{1.350005in}}{\pgfqpoint{0.917060in}{1.346733in}}{\pgfqpoint{0.911236in}{1.340909in}}%
\pgfpathcurveto{\pgfqpoint{0.905412in}{1.335085in}}{\pgfqpoint{0.902139in}{1.327185in}}{\pgfqpoint{0.902139in}{1.318949in}}%
\pgfpathcurveto{\pgfqpoint{0.902139in}{1.310712in}}{\pgfqpoint{0.905412in}{1.302812in}}{\pgfqpoint{0.911236in}{1.296988in}}%
\pgfpathcurveto{\pgfqpoint{0.917060in}{1.291164in}}{\pgfqpoint{0.924960in}{1.287892in}}{\pgfqpoint{0.933196in}{1.287892in}}%
\pgfpathclose%
\pgfusepath{stroke,fill}%
\end{pgfscope}%
\begin{pgfscope}%
\pgfpathrectangle{\pgfqpoint{0.100000in}{0.212622in}}{\pgfqpoint{3.696000in}{3.696000in}}%
\pgfusepath{clip}%
\pgfsetbuttcap%
\pgfsetroundjoin%
\definecolor{currentfill}{rgb}{0.121569,0.466667,0.705882}%
\pgfsetfillcolor{currentfill}%
\pgfsetfillopacity{0.588972}%
\pgfsetlinewidth{1.003750pt}%
\definecolor{currentstroke}{rgb}{0.121569,0.466667,0.705882}%
\pgfsetstrokecolor{currentstroke}%
\pgfsetstrokeopacity{0.588972}%
\pgfsetdash{}{0pt}%
\pgfpathmoveto{\pgfqpoint{0.933196in}{1.287892in}}%
\pgfpathcurveto{\pgfqpoint{0.941432in}{1.287892in}}{\pgfqpoint{0.949332in}{1.291164in}}{\pgfqpoint{0.955156in}{1.296988in}}%
\pgfpathcurveto{\pgfqpoint{0.960980in}{1.302812in}}{\pgfqpoint{0.964252in}{1.310712in}}{\pgfqpoint{0.964252in}{1.318949in}}%
\pgfpathcurveto{\pgfqpoint{0.964252in}{1.327185in}}{\pgfqpoint{0.960980in}{1.335085in}}{\pgfqpoint{0.955156in}{1.340909in}}%
\pgfpathcurveto{\pgfqpoint{0.949332in}{1.346733in}}{\pgfqpoint{0.941432in}{1.350005in}}{\pgfqpoint{0.933196in}{1.350005in}}%
\pgfpathcurveto{\pgfqpoint{0.924960in}{1.350005in}}{\pgfqpoint{0.917060in}{1.346733in}}{\pgfqpoint{0.911236in}{1.340909in}}%
\pgfpathcurveto{\pgfqpoint{0.905412in}{1.335085in}}{\pgfqpoint{0.902139in}{1.327185in}}{\pgfqpoint{0.902139in}{1.318949in}}%
\pgfpathcurveto{\pgfqpoint{0.902139in}{1.310712in}}{\pgfqpoint{0.905412in}{1.302812in}}{\pgfqpoint{0.911236in}{1.296988in}}%
\pgfpathcurveto{\pgfqpoint{0.917060in}{1.291164in}}{\pgfqpoint{0.924960in}{1.287892in}}{\pgfqpoint{0.933196in}{1.287892in}}%
\pgfpathclose%
\pgfusepath{stroke,fill}%
\end{pgfscope}%
\begin{pgfscope}%
\pgfpathrectangle{\pgfqpoint{0.100000in}{0.212622in}}{\pgfqpoint{3.696000in}{3.696000in}}%
\pgfusepath{clip}%
\pgfsetbuttcap%
\pgfsetroundjoin%
\definecolor{currentfill}{rgb}{0.121569,0.466667,0.705882}%
\pgfsetfillcolor{currentfill}%
\pgfsetfillopacity{0.588972}%
\pgfsetlinewidth{1.003750pt}%
\definecolor{currentstroke}{rgb}{0.121569,0.466667,0.705882}%
\pgfsetstrokecolor{currentstroke}%
\pgfsetstrokeopacity{0.588972}%
\pgfsetdash{}{0pt}%
\pgfpathmoveto{\pgfqpoint{0.933196in}{1.287892in}}%
\pgfpathcurveto{\pgfqpoint{0.941432in}{1.287892in}}{\pgfqpoint{0.949332in}{1.291164in}}{\pgfqpoint{0.955156in}{1.296988in}}%
\pgfpathcurveto{\pgfqpoint{0.960980in}{1.302812in}}{\pgfqpoint{0.964252in}{1.310712in}}{\pgfqpoint{0.964252in}{1.318949in}}%
\pgfpathcurveto{\pgfqpoint{0.964252in}{1.327185in}}{\pgfqpoint{0.960980in}{1.335085in}}{\pgfqpoint{0.955156in}{1.340909in}}%
\pgfpathcurveto{\pgfqpoint{0.949332in}{1.346733in}}{\pgfqpoint{0.941432in}{1.350005in}}{\pgfqpoint{0.933196in}{1.350005in}}%
\pgfpathcurveto{\pgfqpoint{0.924960in}{1.350005in}}{\pgfqpoint{0.917060in}{1.346733in}}{\pgfqpoint{0.911236in}{1.340909in}}%
\pgfpathcurveto{\pgfqpoint{0.905412in}{1.335085in}}{\pgfqpoint{0.902139in}{1.327185in}}{\pgfqpoint{0.902139in}{1.318949in}}%
\pgfpathcurveto{\pgfqpoint{0.902139in}{1.310712in}}{\pgfqpoint{0.905412in}{1.302812in}}{\pgfqpoint{0.911236in}{1.296988in}}%
\pgfpathcurveto{\pgfqpoint{0.917060in}{1.291164in}}{\pgfqpoint{0.924960in}{1.287892in}}{\pgfqpoint{0.933196in}{1.287892in}}%
\pgfpathclose%
\pgfusepath{stroke,fill}%
\end{pgfscope}%
\begin{pgfscope}%
\pgfpathrectangle{\pgfqpoint{0.100000in}{0.212622in}}{\pgfqpoint{3.696000in}{3.696000in}}%
\pgfusepath{clip}%
\pgfsetbuttcap%
\pgfsetroundjoin%
\definecolor{currentfill}{rgb}{0.121569,0.466667,0.705882}%
\pgfsetfillcolor{currentfill}%
\pgfsetfillopacity{0.588972}%
\pgfsetlinewidth{1.003750pt}%
\definecolor{currentstroke}{rgb}{0.121569,0.466667,0.705882}%
\pgfsetstrokecolor{currentstroke}%
\pgfsetstrokeopacity{0.588972}%
\pgfsetdash{}{0pt}%
\pgfpathmoveto{\pgfqpoint{0.933196in}{1.287892in}}%
\pgfpathcurveto{\pgfqpoint{0.941432in}{1.287892in}}{\pgfqpoint{0.949332in}{1.291164in}}{\pgfqpoint{0.955156in}{1.296988in}}%
\pgfpathcurveto{\pgfqpoint{0.960980in}{1.302812in}}{\pgfqpoint{0.964252in}{1.310712in}}{\pgfqpoint{0.964252in}{1.318949in}}%
\pgfpathcurveto{\pgfqpoint{0.964252in}{1.327185in}}{\pgfqpoint{0.960980in}{1.335085in}}{\pgfqpoint{0.955156in}{1.340909in}}%
\pgfpathcurveto{\pgfqpoint{0.949332in}{1.346733in}}{\pgfqpoint{0.941432in}{1.350005in}}{\pgfqpoint{0.933196in}{1.350005in}}%
\pgfpathcurveto{\pgfqpoint{0.924960in}{1.350005in}}{\pgfqpoint{0.917060in}{1.346733in}}{\pgfqpoint{0.911236in}{1.340909in}}%
\pgfpathcurveto{\pgfqpoint{0.905412in}{1.335085in}}{\pgfqpoint{0.902139in}{1.327185in}}{\pgfqpoint{0.902139in}{1.318949in}}%
\pgfpathcurveto{\pgfqpoint{0.902139in}{1.310712in}}{\pgfqpoint{0.905412in}{1.302812in}}{\pgfqpoint{0.911236in}{1.296988in}}%
\pgfpathcurveto{\pgfqpoint{0.917060in}{1.291164in}}{\pgfqpoint{0.924960in}{1.287892in}}{\pgfqpoint{0.933196in}{1.287892in}}%
\pgfpathclose%
\pgfusepath{stroke,fill}%
\end{pgfscope}%
\begin{pgfscope}%
\pgfpathrectangle{\pgfqpoint{0.100000in}{0.212622in}}{\pgfqpoint{3.696000in}{3.696000in}}%
\pgfusepath{clip}%
\pgfsetbuttcap%
\pgfsetroundjoin%
\definecolor{currentfill}{rgb}{0.121569,0.466667,0.705882}%
\pgfsetfillcolor{currentfill}%
\pgfsetfillopacity{0.588972}%
\pgfsetlinewidth{1.003750pt}%
\definecolor{currentstroke}{rgb}{0.121569,0.466667,0.705882}%
\pgfsetstrokecolor{currentstroke}%
\pgfsetstrokeopacity{0.588972}%
\pgfsetdash{}{0pt}%
\pgfpathmoveto{\pgfqpoint{0.933196in}{1.287892in}}%
\pgfpathcurveto{\pgfqpoint{0.941432in}{1.287892in}}{\pgfqpoint{0.949332in}{1.291164in}}{\pgfqpoint{0.955156in}{1.296988in}}%
\pgfpathcurveto{\pgfqpoint{0.960980in}{1.302812in}}{\pgfqpoint{0.964252in}{1.310712in}}{\pgfqpoint{0.964252in}{1.318949in}}%
\pgfpathcurveto{\pgfqpoint{0.964252in}{1.327185in}}{\pgfqpoint{0.960980in}{1.335085in}}{\pgfqpoint{0.955156in}{1.340909in}}%
\pgfpathcurveto{\pgfqpoint{0.949332in}{1.346733in}}{\pgfqpoint{0.941432in}{1.350005in}}{\pgfqpoint{0.933196in}{1.350005in}}%
\pgfpathcurveto{\pgfqpoint{0.924960in}{1.350005in}}{\pgfqpoint{0.917060in}{1.346733in}}{\pgfqpoint{0.911236in}{1.340909in}}%
\pgfpathcurveto{\pgfqpoint{0.905412in}{1.335085in}}{\pgfqpoint{0.902139in}{1.327185in}}{\pgfqpoint{0.902139in}{1.318949in}}%
\pgfpathcurveto{\pgfqpoint{0.902139in}{1.310712in}}{\pgfqpoint{0.905412in}{1.302812in}}{\pgfqpoint{0.911236in}{1.296988in}}%
\pgfpathcurveto{\pgfqpoint{0.917060in}{1.291164in}}{\pgfqpoint{0.924960in}{1.287892in}}{\pgfqpoint{0.933196in}{1.287892in}}%
\pgfpathclose%
\pgfusepath{stroke,fill}%
\end{pgfscope}%
\begin{pgfscope}%
\pgfpathrectangle{\pgfqpoint{0.100000in}{0.212622in}}{\pgfqpoint{3.696000in}{3.696000in}}%
\pgfusepath{clip}%
\pgfsetbuttcap%
\pgfsetroundjoin%
\definecolor{currentfill}{rgb}{0.121569,0.466667,0.705882}%
\pgfsetfillcolor{currentfill}%
\pgfsetfillopacity{0.588972}%
\pgfsetlinewidth{1.003750pt}%
\definecolor{currentstroke}{rgb}{0.121569,0.466667,0.705882}%
\pgfsetstrokecolor{currentstroke}%
\pgfsetstrokeopacity{0.588972}%
\pgfsetdash{}{0pt}%
\pgfpathmoveto{\pgfqpoint{0.933196in}{1.287892in}}%
\pgfpathcurveto{\pgfqpoint{0.941432in}{1.287892in}}{\pgfqpoint{0.949332in}{1.291164in}}{\pgfqpoint{0.955156in}{1.296988in}}%
\pgfpathcurveto{\pgfqpoint{0.960980in}{1.302812in}}{\pgfqpoint{0.964252in}{1.310712in}}{\pgfqpoint{0.964252in}{1.318949in}}%
\pgfpathcurveto{\pgfqpoint{0.964252in}{1.327185in}}{\pgfqpoint{0.960980in}{1.335085in}}{\pgfqpoint{0.955156in}{1.340909in}}%
\pgfpathcurveto{\pgfqpoint{0.949332in}{1.346733in}}{\pgfqpoint{0.941432in}{1.350005in}}{\pgfqpoint{0.933196in}{1.350005in}}%
\pgfpathcurveto{\pgfqpoint{0.924960in}{1.350005in}}{\pgfqpoint{0.917060in}{1.346733in}}{\pgfqpoint{0.911236in}{1.340909in}}%
\pgfpathcurveto{\pgfqpoint{0.905412in}{1.335085in}}{\pgfqpoint{0.902139in}{1.327185in}}{\pgfqpoint{0.902139in}{1.318949in}}%
\pgfpathcurveto{\pgfqpoint{0.902139in}{1.310712in}}{\pgfqpoint{0.905412in}{1.302812in}}{\pgfqpoint{0.911236in}{1.296988in}}%
\pgfpathcurveto{\pgfqpoint{0.917060in}{1.291164in}}{\pgfqpoint{0.924960in}{1.287892in}}{\pgfqpoint{0.933196in}{1.287892in}}%
\pgfpathclose%
\pgfusepath{stroke,fill}%
\end{pgfscope}%
\begin{pgfscope}%
\pgfpathrectangle{\pgfqpoint{0.100000in}{0.212622in}}{\pgfqpoint{3.696000in}{3.696000in}}%
\pgfusepath{clip}%
\pgfsetbuttcap%
\pgfsetroundjoin%
\definecolor{currentfill}{rgb}{0.121569,0.466667,0.705882}%
\pgfsetfillcolor{currentfill}%
\pgfsetfillopacity{0.588972}%
\pgfsetlinewidth{1.003750pt}%
\definecolor{currentstroke}{rgb}{0.121569,0.466667,0.705882}%
\pgfsetstrokecolor{currentstroke}%
\pgfsetstrokeopacity{0.588972}%
\pgfsetdash{}{0pt}%
\pgfpathmoveto{\pgfqpoint{0.933196in}{1.287892in}}%
\pgfpathcurveto{\pgfqpoint{0.941432in}{1.287892in}}{\pgfqpoint{0.949332in}{1.291164in}}{\pgfqpoint{0.955156in}{1.296988in}}%
\pgfpathcurveto{\pgfqpoint{0.960980in}{1.302812in}}{\pgfqpoint{0.964252in}{1.310712in}}{\pgfqpoint{0.964252in}{1.318949in}}%
\pgfpathcurveto{\pgfqpoint{0.964252in}{1.327185in}}{\pgfqpoint{0.960980in}{1.335085in}}{\pgfqpoint{0.955156in}{1.340909in}}%
\pgfpathcurveto{\pgfqpoint{0.949332in}{1.346733in}}{\pgfqpoint{0.941432in}{1.350005in}}{\pgfqpoint{0.933196in}{1.350005in}}%
\pgfpathcurveto{\pgfqpoint{0.924960in}{1.350005in}}{\pgfqpoint{0.917060in}{1.346733in}}{\pgfqpoint{0.911236in}{1.340909in}}%
\pgfpathcurveto{\pgfqpoint{0.905412in}{1.335085in}}{\pgfqpoint{0.902139in}{1.327185in}}{\pgfqpoint{0.902139in}{1.318949in}}%
\pgfpathcurveto{\pgfqpoint{0.902139in}{1.310712in}}{\pgfqpoint{0.905412in}{1.302812in}}{\pgfqpoint{0.911236in}{1.296988in}}%
\pgfpathcurveto{\pgfqpoint{0.917060in}{1.291164in}}{\pgfqpoint{0.924960in}{1.287892in}}{\pgfqpoint{0.933196in}{1.287892in}}%
\pgfpathclose%
\pgfusepath{stroke,fill}%
\end{pgfscope}%
\begin{pgfscope}%
\pgfpathrectangle{\pgfqpoint{0.100000in}{0.212622in}}{\pgfqpoint{3.696000in}{3.696000in}}%
\pgfusepath{clip}%
\pgfsetbuttcap%
\pgfsetroundjoin%
\definecolor{currentfill}{rgb}{0.121569,0.466667,0.705882}%
\pgfsetfillcolor{currentfill}%
\pgfsetfillopacity{0.588972}%
\pgfsetlinewidth{1.003750pt}%
\definecolor{currentstroke}{rgb}{0.121569,0.466667,0.705882}%
\pgfsetstrokecolor{currentstroke}%
\pgfsetstrokeopacity{0.588972}%
\pgfsetdash{}{0pt}%
\pgfpathmoveto{\pgfqpoint{0.933196in}{1.287892in}}%
\pgfpathcurveto{\pgfqpoint{0.941432in}{1.287892in}}{\pgfqpoint{0.949332in}{1.291164in}}{\pgfqpoint{0.955156in}{1.296988in}}%
\pgfpathcurveto{\pgfqpoint{0.960980in}{1.302812in}}{\pgfqpoint{0.964252in}{1.310712in}}{\pgfqpoint{0.964252in}{1.318949in}}%
\pgfpathcurveto{\pgfqpoint{0.964252in}{1.327185in}}{\pgfqpoint{0.960980in}{1.335085in}}{\pgfqpoint{0.955156in}{1.340909in}}%
\pgfpathcurveto{\pgfqpoint{0.949332in}{1.346733in}}{\pgfqpoint{0.941432in}{1.350005in}}{\pgfqpoint{0.933196in}{1.350005in}}%
\pgfpathcurveto{\pgfqpoint{0.924960in}{1.350005in}}{\pgfqpoint{0.917060in}{1.346733in}}{\pgfqpoint{0.911236in}{1.340909in}}%
\pgfpathcurveto{\pgfqpoint{0.905412in}{1.335085in}}{\pgfqpoint{0.902139in}{1.327185in}}{\pgfqpoint{0.902139in}{1.318949in}}%
\pgfpathcurveto{\pgfqpoint{0.902139in}{1.310712in}}{\pgfqpoint{0.905412in}{1.302812in}}{\pgfqpoint{0.911236in}{1.296988in}}%
\pgfpathcurveto{\pgfqpoint{0.917060in}{1.291164in}}{\pgfqpoint{0.924960in}{1.287892in}}{\pgfqpoint{0.933196in}{1.287892in}}%
\pgfpathclose%
\pgfusepath{stroke,fill}%
\end{pgfscope}%
\begin{pgfscope}%
\pgfpathrectangle{\pgfqpoint{0.100000in}{0.212622in}}{\pgfqpoint{3.696000in}{3.696000in}}%
\pgfusepath{clip}%
\pgfsetbuttcap%
\pgfsetroundjoin%
\definecolor{currentfill}{rgb}{0.121569,0.466667,0.705882}%
\pgfsetfillcolor{currentfill}%
\pgfsetfillopacity{0.588972}%
\pgfsetlinewidth{1.003750pt}%
\definecolor{currentstroke}{rgb}{0.121569,0.466667,0.705882}%
\pgfsetstrokecolor{currentstroke}%
\pgfsetstrokeopacity{0.588972}%
\pgfsetdash{}{0pt}%
\pgfpathmoveto{\pgfqpoint{0.933196in}{1.287892in}}%
\pgfpathcurveto{\pgfqpoint{0.941432in}{1.287892in}}{\pgfqpoint{0.949332in}{1.291164in}}{\pgfqpoint{0.955156in}{1.296988in}}%
\pgfpathcurveto{\pgfqpoint{0.960980in}{1.302812in}}{\pgfqpoint{0.964252in}{1.310712in}}{\pgfqpoint{0.964252in}{1.318949in}}%
\pgfpathcurveto{\pgfqpoint{0.964252in}{1.327185in}}{\pgfqpoint{0.960980in}{1.335085in}}{\pgfqpoint{0.955156in}{1.340909in}}%
\pgfpathcurveto{\pgfqpoint{0.949332in}{1.346733in}}{\pgfqpoint{0.941432in}{1.350005in}}{\pgfqpoint{0.933196in}{1.350005in}}%
\pgfpathcurveto{\pgfqpoint{0.924960in}{1.350005in}}{\pgfqpoint{0.917060in}{1.346733in}}{\pgfqpoint{0.911236in}{1.340909in}}%
\pgfpathcurveto{\pgfqpoint{0.905412in}{1.335085in}}{\pgfqpoint{0.902139in}{1.327185in}}{\pgfqpoint{0.902139in}{1.318949in}}%
\pgfpathcurveto{\pgfqpoint{0.902139in}{1.310712in}}{\pgfqpoint{0.905412in}{1.302812in}}{\pgfqpoint{0.911236in}{1.296988in}}%
\pgfpathcurveto{\pgfqpoint{0.917060in}{1.291164in}}{\pgfqpoint{0.924960in}{1.287892in}}{\pgfqpoint{0.933196in}{1.287892in}}%
\pgfpathclose%
\pgfusepath{stroke,fill}%
\end{pgfscope}%
\begin{pgfscope}%
\pgfpathrectangle{\pgfqpoint{0.100000in}{0.212622in}}{\pgfqpoint{3.696000in}{3.696000in}}%
\pgfusepath{clip}%
\pgfsetbuttcap%
\pgfsetroundjoin%
\definecolor{currentfill}{rgb}{0.121569,0.466667,0.705882}%
\pgfsetfillcolor{currentfill}%
\pgfsetfillopacity{0.588972}%
\pgfsetlinewidth{1.003750pt}%
\definecolor{currentstroke}{rgb}{0.121569,0.466667,0.705882}%
\pgfsetstrokecolor{currentstroke}%
\pgfsetstrokeopacity{0.588972}%
\pgfsetdash{}{0pt}%
\pgfpathmoveto{\pgfqpoint{0.933196in}{1.287892in}}%
\pgfpathcurveto{\pgfqpoint{0.941432in}{1.287892in}}{\pgfqpoint{0.949332in}{1.291164in}}{\pgfqpoint{0.955156in}{1.296988in}}%
\pgfpathcurveto{\pgfqpoint{0.960980in}{1.302812in}}{\pgfqpoint{0.964252in}{1.310712in}}{\pgfqpoint{0.964252in}{1.318949in}}%
\pgfpathcurveto{\pgfqpoint{0.964252in}{1.327185in}}{\pgfqpoint{0.960980in}{1.335085in}}{\pgfqpoint{0.955156in}{1.340909in}}%
\pgfpathcurveto{\pgfqpoint{0.949332in}{1.346733in}}{\pgfqpoint{0.941432in}{1.350005in}}{\pgfqpoint{0.933196in}{1.350005in}}%
\pgfpathcurveto{\pgfqpoint{0.924960in}{1.350005in}}{\pgfqpoint{0.917060in}{1.346733in}}{\pgfqpoint{0.911236in}{1.340909in}}%
\pgfpathcurveto{\pgfqpoint{0.905412in}{1.335085in}}{\pgfqpoint{0.902139in}{1.327185in}}{\pgfqpoint{0.902139in}{1.318949in}}%
\pgfpathcurveto{\pgfqpoint{0.902139in}{1.310712in}}{\pgfqpoint{0.905412in}{1.302812in}}{\pgfqpoint{0.911236in}{1.296988in}}%
\pgfpathcurveto{\pgfqpoint{0.917060in}{1.291164in}}{\pgfqpoint{0.924960in}{1.287892in}}{\pgfqpoint{0.933196in}{1.287892in}}%
\pgfpathclose%
\pgfusepath{stroke,fill}%
\end{pgfscope}%
\begin{pgfscope}%
\pgfpathrectangle{\pgfqpoint{0.100000in}{0.212622in}}{\pgfqpoint{3.696000in}{3.696000in}}%
\pgfusepath{clip}%
\pgfsetbuttcap%
\pgfsetroundjoin%
\definecolor{currentfill}{rgb}{0.121569,0.466667,0.705882}%
\pgfsetfillcolor{currentfill}%
\pgfsetfillopacity{0.588972}%
\pgfsetlinewidth{1.003750pt}%
\definecolor{currentstroke}{rgb}{0.121569,0.466667,0.705882}%
\pgfsetstrokecolor{currentstroke}%
\pgfsetstrokeopacity{0.588972}%
\pgfsetdash{}{0pt}%
\pgfpathmoveto{\pgfqpoint{0.933196in}{1.287892in}}%
\pgfpathcurveto{\pgfqpoint{0.941432in}{1.287892in}}{\pgfqpoint{0.949332in}{1.291164in}}{\pgfqpoint{0.955156in}{1.296988in}}%
\pgfpathcurveto{\pgfqpoint{0.960980in}{1.302812in}}{\pgfqpoint{0.964252in}{1.310712in}}{\pgfqpoint{0.964252in}{1.318949in}}%
\pgfpathcurveto{\pgfqpoint{0.964252in}{1.327185in}}{\pgfqpoint{0.960980in}{1.335085in}}{\pgfqpoint{0.955156in}{1.340909in}}%
\pgfpathcurveto{\pgfqpoint{0.949332in}{1.346733in}}{\pgfqpoint{0.941432in}{1.350005in}}{\pgfqpoint{0.933196in}{1.350005in}}%
\pgfpathcurveto{\pgfqpoint{0.924960in}{1.350005in}}{\pgfqpoint{0.917060in}{1.346733in}}{\pgfqpoint{0.911236in}{1.340909in}}%
\pgfpathcurveto{\pgfqpoint{0.905412in}{1.335085in}}{\pgfqpoint{0.902139in}{1.327185in}}{\pgfqpoint{0.902139in}{1.318949in}}%
\pgfpathcurveto{\pgfqpoint{0.902139in}{1.310712in}}{\pgfqpoint{0.905412in}{1.302812in}}{\pgfqpoint{0.911236in}{1.296988in}}%
\pgfpathcurveto{\pgfqpoint{0.917060in}{1.291164in}}{\pgfqpoint{0.924960in}{1.287892in}}{\pgfqpoint{0.933196in}{1.287892in}}%
\pgfpathclose%
\pgfusepath{stroke,fill}%
\end{pgfscope}%
\begin{pgfscope}%
\pgfpathrectangle{\pgfqpoint{0.100000in}{0.212622in}}{\pgfqpoint{3.696000in}{3.696000in}}%
\pgfusepath{clip}%
\pgfsetbuttcap%
\pgfsetroundjoin%
\definecolor{currentfill}{rgb}{0.121569,0.466667,0.705882}%
\pgfsetfillcolor{currentfill}%
\pgfsetfillopacity{0.588972}%
\pgfsetlinewidth{1.003750pt}%
\definecolor{currentstroke}{rgb}{0.121569,0.466667,0.705882}%
\pgfsetstrokecolor{currentstroke}%
\pgfsetstrokeopacity{0.588972}%
\pgfsetdash{}{0pt}%
\pgfpathmoveto{\pgfqpoint{0.933196in}{1.287892in}}%
\pgfpathcurveto{\pgfqpoint{0.941432in}{1.287892in}}{\pgfqpoint{0.949332in}{1.291164in}}{\pgfqpoint{0.955156in}{1.296988in}}%
\pgfpathcurveto{\pgfqpoint{0.960980in}{1.302812in}}{\pgfqpoint{0.964252in}{1.310712in}}{\pgfqpoint{0.964252in}{1.318949in}}%
\pgfpathcurveto{\pgfqpoint{0.964252in}{1.327185in}}{\pgfqpoint{0.960980in}{1.335085in}}{\pgfqpoint{0.955156in}{1.340909in}}%
\pgfpathcurveto{\pgfqpoint{0.949332in}{1.346733in}}{\pgfqpoint{0.941432in}{1.350005in}}{\pgfqpoint{0.933196in}{1.350005in}}%
\pgfpathcurveto{\pgfqpoint{0.924960in}{1.350005in}}{\pgfqpoint{0.917060in}{1.346733in}}{\pgfqpoint{0.911236in}{1.340909in}}%
\pgfpathcurveto{\pgfqpoint{0.905412in}{1.335085in}}{\pgfqpoint{0.902139in}{1.327185in}}{\pgfqpoint{0.902139in}{1.318949in}}%
\pgfpathcurveto{\pgfqpoint{0.902139in}{1.310712in}}{\pgfqpoint{0.905412in}{1.302812in}}{\pgfqpoint{0.911236in}{1.296988in}}%
\pgfpathcurveto{\pgfqpoint{0.917060in}{1.291164in}}{\pgfqpoint{0.924960in}{1.287892in}}{\pgfqpoint{0.933196in}{1.287892in}}%
\pgfpathclose%
\pgfusepath{stroke,fill}%
\end{pgfscope}%
\begin{pgfscope}%
\pgfpathrectangle{\pgfqpoint{0.100000in}{0.212622in}}{\pgfqpoint{3.696000in}{3.696000in}}%
\pgfusepath{clip}%
\pgfsetbuttcap%
\pgfsetroundjoin%
\definecolor{currentfill}{rgb}{0.121569,0.466667,0.705882}%
\pgfsetfillcolor{currentfill}%
\pgfsetfillopacity{0.588972}%
\pgfsetlinewidth{1.003750pt}%
\definecolor{currentstroke}{rgb}{0.121569,0.466667,0.705882}%
\pgfsetstrokecolor{currentstroke}%
\pgfsetstrokeopacity{0.588972}%
\pgfsetdash{}{0pt}%
\pgfpathmoveto{\pgfqpoint{0.933196in}{1.287892in}}%
\pgfpathcurveto{\pgfqpoint{0.941432in}{1.287892in}}{\pgfqpoint{0.949332in}{1.291164in}}{\pgfqpoint{0.955156in}{1.296988in}}%
\pgfpathcurveto{\pgfqpoint{0.960980in}{1.302812in}}{\pgfqpoint{0.964252in}{1.310712in}}{\pgfqpoint{0.964252in}{1.318949in}}%
\pgfpathcurveto{\pgfqpoint{0.964252in}{1.327185in}}{\pgfqpoint{0.960980in}{1.335085in}}{\pgfqpoint{0.955156in}{1.340909in}}%
\pgfpathcurveto{\pgfqpoint{0.949332in}{1.346733in}}{\pgfqpoint{0.941432in}{1.350005in}}{\pgfqpoint{0.933196in}{1.350005in}}%
\pgfpathcurveto{\pgfqpoint{0.924960in}{1.350005in}}{\pgfqpoint{0.917060in}{1.346733in}}{\pgfqpoint{0.911236in}{1.340909in}}%
\pgfpathcurveto{\pgfqpoint{0.905412in}{1.335085in}}{\pgfqpoint{0.902139in}{1.327185in}}{\pgfqpoint{0.902139in}{1.318949in}}%
\pgfpathcurveto{\pgfqpoint{0.902139in}{1.310712in}}{\pgfqpoint{0.905412in}{1.302812in}}{\pgfqpoint{0.911236in}{1.296988in}}%
\pgfpathcurveto{\pgfqpoint{0.917060in}{1.291164in}}{\pgfqpoint{0.924960in}{1.287892in}}{\pgfqpoint{0.933196in}{1.287892in}}%
\pgfpathclose%
\pgfusepath{stroke,fill}%
\end{pgfscope}%
\begin{pgfscope}%
\pgfpathrectangle{\pgfqpoint{0.100000in}{0.212622in}}{\pgfqpoint{3.696000in}{3.696000in}}%
\pgfusepath{clip}%
\pgfsetbuttcap%
\pgfsetroundjoin%
\definecolor{currentfill}{rgb}{0.121569,0.466667,0.705882}%
\pgfsetfillcolor{currentfill}%
\pgfsetfillopacity{0.588972}%
\pgfsetlinewidth{1.003750pt}%
\definecolor{currentstroke}{rgb}{0.121569,0.466667,0.705882}%
\pgfsetstrokecolor{currentstroke}%
\pgfsetstrokeopacity{0.588972}%
\pgfsetdash{}{0pt}%
\pgfpathmoveto{\pgfqpoint{0.933196in}{1.287892in}}%
\pgfpathcurveto{\pgfqpoint{0.941432in}{1.287892in}}{\pgfqpoint{0.949332in}{1.291164in}}{\pgfqpoint{0.955156in}{1.296988in}}%
\pgfpathcurveto{\pgfqpoint{0.960980in}{1.302812in}}{\pgfqpoint{0.964252in}{1.310712in}}{\pgfqpoint{0.964252in}{1.318949in}}%
\pgfpathcurveto{\pgfqpoint{0.964252in}{1.327185in}}{\pgfqpoint{0.960980in}{1.335085in}}{\pgfqpoint{0.955156in}{1.340909in}}%
\pgfpathcurveto{\pgfqpoint{0.949332in}{1.346733in}}{\pgfqpoint{0.941432in}{1.350005in}}{\pgfqpoint{0.933196in}{1.350005in}}%
\pgfpathcurveto{\pgfqpoint{0.924960in}{1.350005in}}{\pgfqpoint{0.917060in}{1.346733in}}{\pgfqpoint{0.911236in}{1.340909in}}%
\pgfpathcurveto{\pgfqpoint{0.905412in}{1.335085in}}{\pgfqpoint{0.902139in}{1.327185in}}{\pgfqpoint{0.902139in}{1.318949in}}%
\pgfpathcurveto{\pgfqpoint{0.902139in}{1.310712in}}{\pgfqpoint{0.905412in}{1.302812in}}{\pgfqpoint{0.911236in}{1.296988in}}%
\pgfpathcurveto{\pgfqpoint{0.917060in}{1.291164in}}{\pgfqpoint{0.924960in}{1.287892in}}{\pgfqpoint{0.933196in}{1.287892in}}%
\pgfpathclose%
\pgfusepath{stroke,fill}%
\end{pgfscope}%
\begin{pgfscope}%
\pgfpathrectangle{\pgfqpoint{0.100000in}{0.212622in}}{\pgfqpoint{3.696000in}{3.696000in}}%
\pgfusepath{clip}%
\pgfsetbuttcap%
\pgfsetroundjoin%
\definecolor{currentfill}{rgb}{0.121569,0.466667,0.705882}%
\pgfsetfillcolor{currentfill}%
\pgfsetfillopacity{0.588972}%
\pgfsetlinewidth{1.003750pt}%
\definecolor{currentstroke}{rgb}{0.121569,0.466667,0.705882}%
\pgfsetstrokecolor{currentstroke}%
\pgfsetstrokeopacity{0.588972}%
\pgfsetdash{}{0pt}%
\pgfpathmoveto{\pgfqpoint{0.933196in}{1.287892in}}%
\pgfpathcurveto{\pgfqpoint{0.941432in}{1.287892in}}{\pgfqpoint{0.949332in}{1.291164in}}{\pgfqpoint{0.955156in}{1.296988in}}%
\pgfpathcurveto{\pgfqpoint{0.960980in}{1.302812in}}{\pgfqpoint{0.964252in}{1.310712in}}{\pgfqpoint{0.964252in}{1.318949in}}%
\pgfpathcurveto{\pgfqpoint{0.964252in}{1.327185in}}{\pgfqpoint{0.960980in}{1.335085in}}{\pgfqpoint{0.955156in}{1.340909in}}%
\pgfpathcurveto{\pgfqpoint{0.949332in}{1.346733in}}{\pgfqpoint{0.941432in}{1.350005in}}{\pgfqpoint{0.933196in}{1.350005in}}%
\pgfpathcurveto{\pgfqpoint{0.924960in}{1.350005in}}{\pgfqpoint{0.917060in}{1.346733in}}{\pgfqpoint{0.911236in}{1.340909in}}%
\pgfpathcurveto{\pgfqpoint{0.905412in}{1.335085in}}{\pgfqpoint{0.902139in}{1.327185in}}{\pgfqpoint{0.902139in}{1.318949in}}%
\pgfpathcurveto{\pgfqpoint{0.902139in}{1.310712in}}{\pgfqpoint{0.905412in}{1.302812in}}{\pgfqpoint{0.911236in}{1.296988in}}%
\pgfpathcurveto{\pgfqpoint{0.917060in}{1.291164in}}{\pgfqpoint{0.924960in}{1.287892in}}{\pgfqpoint{0.933196in}{1.287892in}}%
\pgfpathclose%
\pgfusepath{stroke,fill}%
\end{pgfscope}%
\begin{pgfscope}%
\pgfpathrectangle{\pgfqpoint{0.100000in}{0.212622in}}{\pgfqpoint{3.696000in}{3.696000in}}%
\pgfusepath{clip}%
\pgfsetbuttcap%
\pgfsetroundjoin%
\definecolor{currentfill}{rgb}{0.121569,0.466667,0.705882}%
\pgfsetfillcolor{currentfill}%
\pgfsetfillopacity{0.588972}%
\pgfsetlinewidth{1.003750pt}%
\definecolor{currentstroke}{rgb}{0.121569,0.466667,0.705882}%
\pgfsetstrokecolor{currentstroke}%
\pgfsetstrokeopacity{0.588972}%
\pgfsetdash{}{0pt}%
\pgfpathmoveto{\pgfqpoint{0.933196in}{1.287892in}}%
\pgfpathcurveto{\pgfqpoint{0.941432in}{1.287892in}}{\pgfqpoint{0.949332in}{1.291164in}}{\pgfqpoint{0.955156in}{1.296988in}}%
\pgfpathcurveto{\pgfqpoint{0.960980in}{1.302812in}}{\pgfqpoint{0.964252in}{1.310712in}}{\pgfqpoint{0.964252in}{1.318949in}}%
\pgfpathcurveto{\pgfqpoint{0.964252in}{1.327185in}}{\pgfqpoint{0.960980in}{1.335085in}}{\pgfqpoint{0.955156in}{1.340909in}}%
\pgfpathcurveto{\pgfqpoint{0.949332in}{1.346733in}}{\pgfqpoint{0.941432in}{1.350005in}}{\pgfqpoint{0.933196in}{1.350005in}}%
\pgfpathcurveto{\pgfqpoint{0.924960in}{1.350005in}}{\pgfqpoint{0.917060in}{1.346733in}}{\pgfqpoint{0.911236in}{1.340909in}}%
\pgfpathcurveto{\pgfqpoint{0.905412in}{1.335085in}}{\pgfqpoint{0.902139in}{1.327185in}}{\pgfqpoint{0.902139in}{1.318949in}}%
\pgfpathcurveto{\pgfqpoint{0.902139in}{1.310712in}}{\pgfqpoint{0.905412in}{1.302812in}}{\pgfqpoint{0.911236in}{1.296988in}}%
\pgfpathcurveto{\pgfqpoint{0.917060in}{1.291164in}}{\pgfqpoint{0.924960in}{1.287892in}}{\pgfqpoint{0.933196in}{1.287892in}}%
\pgfpathclose%
\pgfusepath{stroke,fill}%
\end{pgfscope}%
\begin{pgfscope}%
\pgfpathrectangle{\pgfqpoint{0.100000in}{0.212622in}}{\pgfqpoint{3.696000in}{3.696000in}}%
\pgfusepath{clip}%
\pgfsetbuttcap%
\pgfsetroundjoin%
\definecolor{currentfill}{rgb}{0.121569,0.466667,0.705882}%
\pgfsetfillcolor{currentfill}%
\pgfsetfillopacity{0.588972}%
\pgfsetlinewidth{1.003750pt}%
\definecolor{currentstroke}{rgb}{0.121569,0.466667,0.705882}%
\pgfsetstrokecolor{currentstroke}%
\pgfsetstrokeopacity{0.588972}%
\pgfsetdash{}{0pt}%
\pgfpathmoveto{\pgfqpoint{0.933196in}{1.287892in}}%
\pgfpathcurveto{\pgfqpoint{0.941432in}{1.287892in}}{\pgfqpoint{0.949332in}{1.291164in}}{\pgfqpoint{0.955156in}{1.296988in}}%
\pgfpathcurveto{\pgfqpoint{0.960980in}{1.302812in}}{\pgfqpoint{0.964252in}{1.310712in}}{\pgfqpoint{0.964252in}{1.318949in}}%
\pgfpathcurveto{\pgfqpoint{0.964252in}{1.327185in}}{\pgfqpoint{0.960980in}{1.335085in}}{\pgfqpoint{0.955156in}{1.340909in}}%
\pgfpathcurveto{\pgfqpoint{0.949332in}{1.346733in}}{\pgfqpoint{0.941432in}{1.350005in}}{\pgfqpoint{0.933196in}{1.350005in}}%
\pgfpathcurveto{\pgfqpoint{0.924960in}{1.350005in}}{\pgfqpoint{0.917060in}{1.346733in}}{\pgfqpoint{0.911236in}{1.340909in}}%
\pgfpathcurveto{\pgfqpoint{0.905412in}{1.335085in}}{\pgfqpoint{0.902139in}{1.327185in}}{\pgfqpoint{0.902139in}{1.318949in}}%
\pgfpathcurveto{\pgfqpoint{0.902139in}{1.310712in}}{\pgfqpoint{0.905412in}{1.302812in}}{\pgfqpoint{0.911236in}{1.296988in}}%
\pgfpathcurveto{\pgfqpoint{0.917060in}{1.291164in}}{\pgfqpoint{0.924960in}{1.287892in}}{\pgfqpoint{0.933196in}{1.287892in}}%
\pgfpathclose%
\pgfusepath{stroke,fill}%
\end{pgfscope}%
\begin{pgfscope}%
\pgfpathrectangle{\pgfqpoint{0.100000in}{0.212622in}}{\pgfqpoint{3.696000in}{3.696000in}}%
\pgfusepath{clip}%
\pgfsetbuttcap%
\pgfsetroundjoin%
\definecolor{currentfill}{rgb}{0.121569,0.466667,0.705882}%
\pgfsetfillcolor{currentfill}%
\pgfsetfillopacity{0.588972}%
\pgfsetlinewidth{1.003750pt}%
\definecolor{currentstroke}{rgb}{0.121569,0.466667,0.705882}%
\pgfsetstrokecolor{currentstroke}%
\pgfsetstrokeopacity{0.588972}%
\pgfsetdash{}{0pt}%
\pgfpathmoveto{\pgfqpoint{0.933196in}{1.287892in}}%
\pgfpathcurveto{\pgfqpoint{0.941432in}{1.287892in}}{\pgfqpoint{0.949332in}{1.291164in}}{\pgfqpoint{0.955156in}{1.296988in}}%
\pgfpathcurveto{\pgfqpoint{0.960980in}{1.302812in}}{\pgfqpoint{0.964252in}{1.310712in}}{\pgfqpoint{0.964252in}{1.318949in}}%
\pgfpathcurveto{\pgfqpoint{0.964252in}{1.327185in}}{\pgfqpoint{0.960980in}{1.335085in}}{\pgfqpoint{0.955156in}{1.340909in}}%
\pgfpathcurveto{\pgfqpoint{0.949332in}{1.346733in}}{\pgfqpoint{0.941432in}{1.350005in}}{\pgfqpoint{0.933196in}{1.350005in}}%
\pgfpathcurveto{\pgfqpoint{0.924960in}{1.350005in}}{\pgfqpoint{0.917060in}{1.346733in}}{\pgfqpoint{0.911236in}{1.340909in}}%
\pgfpathcurveto{\pgfqpoint{0.905412in}{1.335085in}}{\pgfqpoint{0.902139in}{1.327185in}}{\pgfqpoint{0.902139in}{1.318949in}}%
\pgfpathcurveto{\pgfqpoint{0.902139in}{1.310712in}}{\pgfqpoint{0.905412in}{1.302812in}}{\pgfqpoint{0.911236in}{1.296988in}}%
\pgfpathcurveto{\pgfqpoint{0.917060in}{1.291164in}}{\pgfqpoint{0.924960in}{1.287892in}}{\pgfqpoint{0.933196in}{1.287892in}}%
\pgfpathclose%
\pgfusepath{stroke,fill}%
\end{pgfscope}%
\begin{pgfscope}%
\pgfpathrectangle{\pgfqpoint{0.100000in}{0.212622in}}{\pgfqpoint{3.696000in}{3.696000in}}%
\pgfusepath{clip}%
\pgfsetbuttcap%
\pgfsetroundjoin%
\definecolor{currentfill}{rgb}{0.121569,0.466667,0.705882}%
\pgfsetfillcolor{currentfill}%
\pgfsetfillopacity{0.588972}%
\pgfsetlinewidth{1.003750pt}%
\definecolor{currentstroke}{rgb}{0.121569,0.466667,0.705882}%
\pgfsetstrokecolor{currentstroke}%
\pgfsetstrokeopacity{0.588972}%
\pgfsetdash{}{0pt}%
\pgfpathmoveto{\pgfqpoint{0.933196in}{1.287892in}}%
\pgfpathcurveto{\pgfqpoint{0.941432in}{1.287892in}}{\pgfqpoint{0.949332in}{1.291164in}}{\pgfqpoint{0.955156in}{1.296988in}}%
\pgfpathcurveto{\pgfqpoint{0.960980in}{1.302812in}}{\pgfqpoint{0.964252in}{1.310712in}}{\pgfqpoint{0.964252in}{1.318949in}}%
\pgfpathcurveto{\pgfqpoint{0.964252in}{1.327185in}}{\pgfqpoint{0.960980in}{1.335085in}}{\pgfqpoint{0.955156in}{1.340909in}}%
\pgfpathcurveto{\pgfqpoint{0.949332in}{1.346733in}}{\pgfqpoint{0.941432in}{1.350005in}}{\pgfqpoint{0.933196in}{1.350005in}}%
\pgfpathcurveto{\pgfqpoint{0.924960in}{1.350005in}}{\pgfqpoint{0.917060in}{1.346733in}}{\pgfqpoint{0.911236in}{1.340909in}}%
\pgfpathcurveto{\pgfqpoint{0.905412in}{1.335085in}}{\pgfqpoint{0.902139in}{1.327185in}}{\pgfqpoint{0.902139in}{1.318949in}}%
\pgfpathcurveto{\pgfqpoint{0.902139in}{1.310712in}}{\pgfqpoint{0.905412in}{1.302812in}}{\pgfqpoint{0.911236in}{1.296988in}}%
\pgfpathcurveto{\pgfqpoint{0.917060in}{1.291164in}}{\pgfqpoint{0.924960in}{1.287892in}}{\pgfqpoint{0.933196in}{1.287892in}}%
\pgfpathclose%
\pgfusepath{stroke,fill}%
\end{pgfscope}%
\begin{pgfscope}%
\pgfpathrectangle{\pgfqpoint{0.100000in}{0.212622in}}{\pgfqpoint{3.696000in}{3.696000in}}%
\pgfusepath{clip}%
\pgfsetbuttcap%
\pgfsetroundjoin%
\definecolor{currentfill}{rgb}{0.121569,0.466667,0.705882}%
\pgfsetfillcolor{currentfill}%
\pgfsetfillopacity{0.588972}%
\pgfsetlinewidth{1.003750pt}%
\definecolor{currentstroke}{rgb}{0.121569,0.466667,0.705882}%
\pgfsetstrokecolor{currentstroke}%
\pgfsetstrokeopacity{0.588972}%
\pgfsetdash{}{0pt}%
\pgfpathmoveto{\pgfqpoint{0.933196in}{1.287892in}}%
\pgfpathcurveto{\pgfqpoint{0.941432in}{1.287892in}}{\pgfqpoint{0.949332in}{1.291164in}}{\pgfqpoint{0.955156in}{1.296988in}}%
\pgfpathcurveto{\pgfqpoint{0.960980in}{1.302812in}}{\pgfqpoint{0.964252in}{1.310712in}}{\pgfqpoint{0.964252in}{1.318949in}}%
\pgfpathcurveto{\pgfqpoint{0.964252in}{1.327185in}}{\pgfqpoint{0.960980in}{1.335085in}}{\pgfqpoint{0.955156in}{1.340909in}}%
\pgfpathcurveto{\pgfqpoint{0.949332in}{1.346733in}}{\pgfqpoint{0.941432in}{1.350005in}}{\pgfqpoint{0.933196in}{1.350005in}}%
\pgfpathcurveto{\pgfqpoint{0.924960in}{1.350005in}}{\pgfqpoint{0.917060in}{1.346733in}}{\pgfqpoint{0.911236in}{1.340909in}}%
\pgfpathcurveto{\pgfqpoint{0.905412in}{1.335085in}}{\pgfqpoint{0.902139in}{1.327185in}}{\pgfqpoint{0.902139in}{1.318949in}}%
\pgfpathcurveto{\pgfqpoint{0.902139in}{1.310712in}}{\pgfqpoint{0.905412in}{1.302812in}}{\pgfqpoint{0.911236in}{1.296988in}}%
\pgfpathcurveto{\pgfqpoint{0.917060in}{1.291164in}}{\pgfqpoint{0.924960in}{1.287892in}}{\pgfqpoint{0.933196in}{1.287892in}}%
\pgfpathclose%
\pgfusepath{stroke,fill}%
\end{pgfscope}%
\begin{pgfscope}%
\pgfpathrectangle{\pgfqpoint{0.100000in}{0.212622in}}{\pgfqpoint{3.696000in}{3.696000in}}%
\pgfusepath{clip}%
\pgfsetbuttcap%
\pgfsetroundjoin%
\definecolor{currentfill}{rgb}{0.121569,0.466667,0.705882}%
\pgfsetfillcolor{currentfill}%
\pgfsetfillopacity{0.589081}%
\pgfsetlinewidth{1.003750pt}%
\definecolor{currentstroke}{rgb}{0.121569,0.466667,0.705882}%
\pgfsetstrokecolor{currentstroke}%
\pgfsetstrokeopacity{0.589081}%
\pgfsetdash{}{0pt}%
\pgfpathmoveto{\pgfqpoint{0.932684in}{1.287997in}}%
\pgfpathcurveto{\pgfqpoint{0.940920in}{1.287997in}}{\pgfqpoint{0.948820in}{1.291269in}}{\pgfqpoint{0.954644in}{1.297093in}}%
\pgfpathcurveto{\pgfqpoint{0.960468in}{1.302917in}}{\pgfqpoint{0.963740in}{1.310817in}}{\pgfqpoint{0.963740in}{1.319053in}}%
\pgfpathcurveto{\pgfqpoint{0.963740in}{1.327290in}}{\pgfqpoint{0.960468in}{1.335190in}}{\pgfqpoint{0.954644in}{1.341014in}}%
\pgfpathcurveto{\pgfqpoint{0.948820in}{1.346837in}}{\pgfqpoint{0.940920in}{1.350110in}}{\pgfqpoint{0.932684in}{1.350110in}}%
\pgfpathcurveto{\pgfqpoint{0.924447in}{1.350110in}}{\pgfqpoint{0.916547in}{1.346837in}}{\pgfqpoint{0.910723in}{1.341014in}}%
\pgfpathcurveto{\pgfqpoint{0.904899in}{1.335190in}}{\pgfqpoint{0.901627in}{1.327290in}}{\pgfqpoint{0.901627in}{1.319053in}}%
\pgfpathcurveto{\pgfqpoint{0.901627in}{1.310817in}}{\pgfqpoint{0.904899in}{1.302917in}}{\pgfqpoint{0.910723in}{1.297093in}}%
\pgfpathcurveto{\pgfqpoint{0.916547in}{1.291269in}}{\pgfqpoint{0.924447in}{1.287997in}}{\pgfqpoint{0.932684in}{1.287997in}}%
\pgfpathclose%
\pgfusepath{stroke,fill}%
\end{pgfscope}%
\begin{pgfscope}%
\pgfpathrectangle{\pgfqpoint{0.100000in}{0.212622in}}{\pgfqpoint{3.696000in}{3.696000in}}%
\pgfusepath{clip}%
\pgfsetbuttcap%
\pgfsetroundjoin%
\definecolor{currentfill}{rgb}{0.121569,0.466667,0.705882}%
\pgfsetfillcolor{currentfill}%
\pgfsetfillopacity{0.589396}%
\pgfsetlinewidth{1.003750pt}%
\definecolor{currentstroke}{rgb}{0.121569,0.466667,0.705882}%
\pgfsetstrokecolor{currentstroke}%
\pgfsetstrokeopacity{0.589396}%
\pgfsetdash{}{0pt}%
\pgfpathmoveto{\pgfqpoint{0.931049in}{1.288333in}}%
\pgfpathcurveto{\pgfqpoint{0.939285in}{1.288333in}}{\pgfqpoint{0.947185in}{1.291605in}}{\pgfqpoint{0.953009in}{1.297429in}}%
\pgfpathcurveto{\pgfqpoint{0.958833in}{1.303253in}}{\pgfqpoint{0.962105in}{1.311153in}}{\pgfqpoint{0.962105in}{1.319389in}}%
\pgfpathcurveto{\pgfqpoint{0.962105in}{1.327626in}}{\pgfqpoint{0.958833in}{1.335526in}}{\pgfqpoint{0.953009in}{1.341350in}}%
\pgfpathcurveto{\pgfqpoint{0.947185in}{1.347173in}}{\pgfqpoint{0.939285in}{1.350446in}}{\pgfqpoint{0.931049in}{1.350446in}}%
\pgfpathcurveto{\pgfqpoint{0.922812in}{1.350446in}}{\pgfqpoint{0.914912in}{1.347173in}}{\pgfqpoint{0.909088in}{1.341350in}}%
\pgfpathcurveto{\pgfqpoint{0.903265in}{1.335526in}}{\pgfqpoint{0.899992in}{1.327626in}}{\pgfqpoint{0.899992in}{1.319389in}}%
\pgfpathcurveto{\pgfqpoint{0.899992in}{1.311153in}}{\pgfqpoint{0.903265in}{1.303253in}}{\pgfqpoint{0.909088in}{1.297429in}}%
\pgfpathcurveto{\pgfqpoint{0.914912in}{1.291605in}}{\pgfqpoint{0.922812in}{1.288333in}}{\pgfqpoint{0.931049in}{1.288333in}}%
\pgfpathclose%
\pgfusepath{stroke,fill}%
\end{pgfscope}%
\begin{pgfscope}%
\pgfpathrectangle{\pgfqpoint{0.100000in}{0.212622in}}{\pgfqpoint{3.696000in}{3.696000in}}%
\pgfusepath{clip}%
\pgfsetbuttcap%
\pgfsetroundjoin%
\definecolor{currentfill}{rgb}{0.121569,0.466667,0.705882}%
\pgfsetfillcolor{currentfill}%
\pgfsetfillopacity{0.589443}%
\pgfsetlinewidth{1.003750pt}%
\definecolor{currentstroke}{rgb}{0.121569,0.466667,0.705882}%
\pgfsetstrokecolor{currentstroke}%
\pgfsetstrokeopacity{0.589443}%
\pgfsetdash{}{0pt}%
\pgfpathmoveto{\pgfqpoint{3.204949in}{2.217727in}}%
\pgfpathcurveto{\pgfqpoint{3.213185in}{2.217727in}}{\pgfqpoint{3.221085in}{2.221000in}}{\pgfqpoint{3.226909in}{2.226824in}}%
\pgfpathcurveto{\pgfqpoint{3.232733in}{2.232648in}}{\pgfqpoint{3.236005in}{2.240548in}}{\pgfqpoint{3.236005in}{2.248784in}}%
\pgfpathcurveto{\pgfqpoint{3.236005in}{2.257020in}}{\pgfqpoint{3.232733in}{2.264920in}}{\pgfqpoint{3.226909in}{2.270744in}}%
\pgfpathcurveto{\pgfqpoint{3.221085in}{2.276568in}}{\pgfqpoint{3.213185in}{2.279840in}}{\pgfqpoint{3.204949in}{2.279840in}}%
\pgfpathcurveto{\pgfqpoint{3.196713in}{2.279840in}}{\pgfqpoint{3.188812in}{2.276568in}}{\pgfqpoint{3.182989in}{2.270744in}}%
\pgfpathcurveto{\pgfqpoint{3.177165in}{2.264920in}}{\pgfqpoint{3.173892in}{2.257020in}}{\pgfqpoint{3.173892in}{2.248784in}}%
\pgfpathcurveto{\pgfqpoint{3.173892in}{2.240548in}}{\pgfqpoint{3.177165in}{2.232648in}}{\pgfqpoint{3.182989in}{2.226824in}}%
\pgfpathcurveto{\pgfqpoint{3.188812in}{2.221000in}}{\pgfqpoint{3.196713in}{2.217727in}}{\pgfqpoint{3.204949in}{2.217727in}}%
\pgfpathclose%
\pgfusepath{stroke,fill}%
\end{pgfscope}%
\begin{pgfscope}%
\pgfpathrectangle{\pgfqpoint{0.100000in}{0.212622in}}{\pgfqpoint{3.696000in}{3.696000in}}%
\pgfusepath{clip}%
\pgfsetbuttcap%
\pgfsetroundjoin%
\definecolor{currentfill}{rgb}{0.121569,0.466667,0.705882}%
\pgfsetfillcolor{currentfill}%
\pgfsetfillopacity{0.589555}%
\pgfsetlinewidth{1.003750pt}%
\definecolor{currentstroke}{rgb}{0.121569,0.466667,0.705882}%
\pgfsetstrokecolor{currentstroke}%
\pgfsetstrokeopacity{0.589555}%
\pgfsetdash{}{0pt}%
\pgfpathmoveto{\pgfqpoint{0.930124in}{1.288546in}}%
\pgfpathcurveto{\pgfqpoint{0.938360in}{1.288546in}}{\pgfqpoint{0.946260in}{1.291819in}}{\pgfqpoint{0.952084in}{1.297642in}}%
\pgfpathcurveto{\pgfqpoint{0.957908in}{1.303466in}}{\pgfqpoint{0.961180in}{1.311366in}}{\pgfqpoint{0.961180in}{1.319603in}}%
\pgfpathcurveto{\pgfqpoint{0.961180in}{1.327839in}}{\pgfqpoint{0.957908in}{1.335739in}}{\pgfqpoint{0.952084in}{1.341563in}}%
\pgfpathcurveto{\pgfqpoint{0.946260in}{1.347387in}}{\pgfqpoint{0.938360in}{1.350659in}}{\pgfqpoint{0.930124in}{1.350659in}}%
\pgfpathcurveto{\pgfqpoint{0.921887in}{1.350659in}}{\pgfqpoint{0.913987in}{1.347387in}}{\pgfqpoint{0.908163in}{1.341563in}}%
\pgfpathcurveto{\pgfqpoint{0.902339in}{1.335739in}}{\pgfqpoint{0.899067in}{1.327839in}}{\pgfqpoint{0.899067in}{1.319603in}}%
\pgfpathcurveto{\pgfqpoint{0.899067in}{1.311366in}}{\pgfqpoint{0.902339in}{1.303466in}}{\pgfqpoint{0.908163in}{1.297642in}}%
\pgfpathcurveto{\pgfqpoint{0.913987in}{1.291819in}}{\pgfqpoint{0.921887in}{1.288546in}}{\pgfqpoint{0.930124in}{1.288546in}}%
\pgfpathclose%
\pgfusepath{stroke,fill}%
\end{pgfscope}%
\begin{pgfscope}%
\pgfpathrectangle{\pgfqpoint{0.100000in}{0.212622in}}{\pgfqpoint{3.696000in}{3.696000in}}%
\pgfusepath{clip}%
\pgfsetbuttcap%
\pgfsetroundjoin%
\definecolor{currentfill}{rgb}{0.121569,0.466667,0.705882}%
\pgfsetfillcolor{currentfill}%
\pgfsetfillopacity{0.589713}%
\pgfsetlinewidth{1.003750pt}%
\definecolor{currentstroke}{rgb}{0.121569,0.466667,0.705882}%
\pgfsetstrokecolor{currentstroke}%
\pgfsetstrokeopacity{0.589713}%
\pgfsetdash{}{0pt}%
\pgfpathmoveto{\pgfqpoint{3.204414in}{2.217533in}}%
\pgfpathcurveto{\pgfqpoint{3.212651in}{2.217533in}}{\pgfqpoint{3.220551in}{2.220805in}}{\pgfqpoint{3.226375in}{2.226629in}}%
\pgfpathcurveto{\pgfqpoint{3.232198in}{2.232453in}}{\pgfqpoint{3.235471in}{2.240353in}}{\pgfqpoint{3.235471in}{2.248589in}}%
\pgfpathcurveto{\pgfqpoint{3.235471in}{2.256826in}}{\pgfqpoint{3.232198in}{2.264726in}}{\pgfqpoint{3.226375in}{2.270550in}}%
\pgfpathcurveto{\pgfqpoint{3.220551in}{2.276373in}}{\pgfqpoint{3.212651in}{2.279646in}}{\pgfqpoint{3.204414in}{2.279646in}}%
\pgfpathcurveto{\pgfqpoint{3.196178in}{2.279646in}}{\pgfqpoint{3.188278in}{2.276373in}}{\pgfqpoint{3.182454in}{2.270550in}}%
\pgfpathcurveto{\pgfqpoint{3.176630in}{2.264726in}}{\pgfqpoint{3.173358in}{2.256826in}}{\pgfqpoint{3.173358in}{2.248589in}}%
\pgfpathcurveto{\pgfqpoint{3.173358in}{2.240353in}}{\pgfqpoint{3.176630in}{2.232453in}}{\pgfqpoint{3.182454in}{2.226629in}}%
\pgfpathcurveto{\pgfqpoint{3.188278in}{2.220805in}}{\pgfqpoint{3.196178in}{2.217533in}}{\pgfqpoint{3.204414in}{2.217533in}}%
\pgfpathclose%
\pgfusepath{stroke,fill}%
\end{pgfscope}%
\begin{pgfscope}%
\pgfpathrectangle{\pgfqpoint{0.100000in}{0.212622in}}{\pgfqpoint{3.696000in}{3.696000in}}%
\pgfusepath{clip}%
\pgfsetbuttcap%
\pgfsetroundjoin%
\definecolor{currentfill}{rgb}{0.121569,0.466667,0.705882}%
\pgfsetfillcolor{currentfill}%
\pgfsetfillopacity{0.589795}%
\pgfsetlinewidth{1.003750pt}%
\definecolor{currentstroke}{rgb}{0.121569,0.466667,0.705882}%
\pgfsetstrokecolor{currentstroke}%
\pgfsetstrokeopacity{0.589795}%
\pgfsetdash{}{0pt}%
\pgfpathmoveto{\pgfqpoint{0.928628in}{1.288993in}}%
\pgfpathcurveto{\pgfqpoint{0.936864in}{1.288993in}}{\pgfqpoint{0.944764in}{1.292266in}}{\pgfqpoint{0.950588in}{1.298090in}}%
\pgfpathcurveto{\pgfqpoint{0.956412in}{1.303913in}}{\pgfqpoint{0.959685in}{1.311814in}}{\pgfqpoint{0.959685in}{1.320050in}}%
\pgfpathcurveto{\pgfqpoint{0.959685in}{1.328286in}}{\pgfqpoint{0.956412in}{1.336186in}}{\pgfqpoint{0.950588in}{1.342010in}}%
\pgfpathcurveto{\pgfqpoint{0.944764in}{1.347834in}}{\pgfqpoint{0.936864in}{1.351106in}}{\pgfqpoint{0.928628in}{1.351106in}}%
\pgfpathcurveto{\pgfqpoint{0.920392in}{1.351106in}}{\pgfqpoint{0.912492in}{1.347834in}}{\pgfqpoint{0.906668in}{1.342010in}}%
\pgfpathcurveto{\pgfqpoint{0.900844in}{1.336186in}}{\pgfqpoint{0.897572in}{1.328286in}}{\pgfqpoint{0.897572in}{1.320050in}}%
\pgfpathcurveto{\pgfqpoint{0.897572in}{1.311814in}}{\pgfqpoint{0.900844in}{1.303913in}}{\pgfqpoint{0.906668in}{1.298090in}}%
\pgfpathcurveto{\pgfqpoint{0.912492in}{1.292266in}}{\pgfqpoint{0.920392in}{1.288993in}}{\pgfqpoint{0.928628in}{1.288993in}}%
\pgfpathclose%
\pgfusepath{stroke,fill}%
\end{pgfscope}%
\begin{pgfscope}%
\pgfpathrectangle{\pgfqpoint{0.100000in}{0.212622in}}{\pgfqpoint{3.696000in}{3.696000in}}%
\pgfusepath{clip}%
\pgfsetbuttcap%
\pgfsetroundjoin%
\definecolor{currentfill}{rgb}{0.121569,0.466667,0.705882}%
\pgfsetfillcolor{currentfill}%
\pgfsetfillopacity{0.590224}%
\pgfsetlinewidth{1.003750pt}%
\definecolor{currentstroke}{rgb}{0.121569,0.466667,0.705882}%
\pgfsetstrokecolor{currentstroke}%
\pgfsetstrokeopacity{0.590224}%
\pgfsetdash{}{0pt}%
\pgfpathmoveto{\pgfqpoint{0.850262in}{1.438587in}}%
\pgfpathcurveto{\pgfqpoint{0.858498in}{1.438587in}}{\pgfqpoint{0.866398in}{1.441859in}}{\pgfqpoint{0.872222in}{1.447683in}}%
\pgfpathcurveto{\pgfqpoint{0.878046in}{1.453507in}}{\pgfqpoint{0.881319in}{1.461407in}}{\pgfqpoint{0.881319in}{1.469643in}}%
\pgfpathcurveto{\pgfqpoint{0.881319in}{1.477880in}}{\pgfqpoint{0.878046in}{1.485780in}}{\pgfqpoint{0.872222in}{1.491604in}}%
\pgfpathcurveto{\pgfqpoint{0.866398in}{1.497428in}}{\pgfqpoint{0.858498in}{1.500700in}}{\pgfqpoint{0.850262in}{1.500700in}}%
\pgfpathcurveto{\pgfqpoint{0.842026in}{1.500700in}}{\pgfqpoint{0.834126in}{1.497428in}}{\pgfqpoint{0.828302in}{1.491604in}}%
\pgfpathcurveto{\pgfqpoint{0.822478in}{1.485780in}}{\pgfqpoint{0.819206in}{1.477880in}}{\pgfqpoint{0.819206in}{1.469643in}}%
\pgfpathcurveto{\pgfqpoint{0.819206in}{1.461407in}}{\pgfqpoint{0.822478in}{1.453507in}}{\pgfqpoint{0.828302in}{1.447683in}}%
\pgfpathcurveto{\pgfqpoint{0.834126in}{1.441859in}}{\pgfqpoint{0.842026in}{1.438587in}}{\pgfqpoint{0.850262in}{1.438587in}}%
\pgfpathclose%
\pgfusepath{stroke,fill}%
\end{pgfscope}%
\begin{pgfscope}%
\pgfpathrectangle{\pgfqpoint{0.100000in}{0.212622in}}{\pgfqpoint{3.696000in}{3.696000in}}%
\pgfusepath{clip}%
\pgfsetbuttcap%
\pgfsetroundjoin%
\definecolor{currentfill}{rgb}{0.121569,0.466667,0.705882}%
\pgfsetfillcolor{currentfill}%
\pgfsetfillopacity{0.590246}%
\pgfsetlinewidth{1.003750pt}%
\definecolor{currentstroke}{rgb}{0.121569,0.466667,0.705882}%
\pgfsetstrokecolor{currentstroke}%
\pgfsetstrokeopacity{0.590246}%
\pgfsetdash{}{0pt}%
\pgfpathmoveto{\pgfqpoint{0.925728in}{1.290089in}}%
\pgfpathcurveto{\pgfqpoint{0.933964in}{1.290089in}}{\pgfqpoint{0.941864in}{1.293362in}}{\pgfqpoint{0.947688in}{1.299186in}}%
\pgfpathcurveto{\pgfqpoint{0.953512in}{1.305010in}}{\pgfqpoint{0.956784in}{1.312910in}}{\pgfqpoint{0.956784in}{1.321146in}}%
\pgfpathcurveto{\pgfqpoint{0.956784in}{1.329382in}}{\pgfqpoint{0.953512in}{1.337282in}}{\pgfqpoint{0.947688in}{1.343106in}}%
\pgfpathcurveto{\pgfqpoint{0.941864in}{1.348930in}}{\pgfqpoint{0.933964in}{1.352202in}}{\pgfqpoint{0.925728in}{1.352202in}}%
\pgfpathcurveto{\pgfqpoint{0.917491in}{1.352202in}}{\pgfqpoint{0.909591in}{1.348930in}}{\pgfqpoint{0.903767in}{1.343106in}}%
\pgfpathcurveto{\pgfqpoint{0.897943in}{1.337282in}}{\pgfqpoint{0.894671in}{1.329382in}}{\pgfqpoint{0.894671in}{1.321146in}}%
\pgfpathcurveto{\pgfqpoint{0.894671in}{1.312910in}}{\pgfqpoint{0.897943in}{1.305010in}}{\pgfqpoint{0.903767in}{1.299186in}}%
\pgfpathcurveto{\pgfqpoint{0.909591in}{1.293362in}}{\pgfqpoint{0.917491in}{1.290089in}}{\pgfqpoint{0.925728in}{1.290089in}}%
\pgfpathclose%
\pgfusepath{stroke,fill}%
\end{pgfscope}%
\begin{pgfscope}%
\pgfpathrectangle{\pgfqpoint{0.100000in}{0.212622in}}{\pgfqpoint{3.696000in}{3.696000in}}%
\pgfusepath{clip}%
\pgfsetbuttcap%
\pgfsetroundjoin%
\definecolor{currentfill}{rgb}{0.121569,0.466667,0.705882}%
\pgfsetfillcolor{currentfill}%
\pgfsetfillopacity{0.590497}%
\pgfsetlinewidth{1.003750pt}%
\definecolor{currentstroke}{rgb}{0.121569,0.466667,0.705882}%
\pgfsetstrokecolor{currentstroke}%
\pgfsetstrokeopacity{0.590497}%
\pgfsetdash{}{0pt}%
\pgfpathmoveto{\pgfqpoint{0.924114in}{1.290827in}}%
\pgfpathcurveto{\pgfqpoint{0.932350in}{1.290827in}}{\pgfqpoint{0.940250in}{1.294099in}}{\pgfqpoint{0.946074in}{1.299923in}}%
\pgfpathcurveto{\pgfqpoint{0.951898in}{1.305747in}}{\pgfqpoint{0.955170in}{1.313647in}}{\pgfqpoint{0.955170in}{1.321883in}}%
\pgfpathcurveto{\pgfqpoint{0.955170in}{1.330119in}}{\pgfqpoint{0.951898in}{1.338019in}}{\pgfqpoint{0.946074in}{1.343843in}}%
\pgfpathcurveto{\pgfqpoint{0.940250in}{1.349667in}}{\pgfqpoint{0.932350in}{1.352940in}}{\pgfqpoint{0.924114in}{1.352940in}}%
\pgfpathcurveto{\pgfqpoint{0.915877in}{1.352940in}}{\pgfqpoint{0.907977in}{1.349667in}}{\pgfqpoint{0.902153in}{1.343843in}}%
\pgfpathcurveto{\pgfqpoint{0.896330in}{1.338019in}}{\pgfqpoint{0.893057in}{1.330119in}}{\pgfqpoint{0.893057in}{1.321883in}}%
\pgfpathcurveto{\pgfqpoint{0.893057in}{1.313647in}}{\pgfqpoint{0.896330in}{1.305747in}}{\pgfqpoint{0.902153in}{1.299923in}}%
\pgfpathcurveto{\pgfqpoint{0.907977in}{1.294099in}}{\pgfqpoint{0.915877in}{1.290827in}}{\pgfqpoint{0.924114in}{1.290827in}}%
\pgfpathclose%
\pgfusepath{stroke,fill}%
\end{pgfscope}%
\begin{pgfscope}%
\pgfpathrectangle{\pgfqpoint{0.100000in}{0.212622in}}{\pgfqpoint{3.696000in}{3.696000in}}%
\pgfusepath{clip}%
\pgfsetbuttcap%
\pgfsetroundjoin%
\definecolor{currentfill}{rgb}{0.121569,0.466667,0.705882}%
\pgfsetfillcolor{currentfill}%
\pgfsetfillopacity{0.590573}%
\pgfsetlinewidth{1.003750pt}%
\definecolor{currentstroke}{rgb}{0.121569,0.466667,0.705882}%
\pgfsetstrokecolor{currentstroke}%
\pgfsetstrokeopacity{0.590573}%
\pgfsetdash{}{0pt}%
\pgfpathmoveto{\pgfqpoint{3.202746in}{2.216618in}}%
\pgfpathcurveto{\pgfqpoint{3.210983in}{2.216618in}}{\pgfqpoint{3.218883in}{2.219890in}}{\pgfqpoint{3.224707in}{2.225714in}}%
\pgfpathcurveto{\pgfqpoint{3.230531in}{2.231538in}}{\pgfqpoint{3.233803in}{2.239438in}}{\pgfqpoint{3.233803in}{2.247674in}}%
\pgfpathcurveto{\pgfqpoint{3.233803in}{2.255911in}}{\pgfqpoint{3.230531in}{2.263811in}}{\pgfqpoint{3.224707in}{2.269635in}}%
\pgfpathcurveto{\pgfqpoint{3.218883in}{2.275459in}}{\pgfqpoint{3.210983in}{2.278731in}}{\pgfqpoint{3.202746in}{2.278731in}}%
\pgfpathcurveto{\pgfqpoint{3.194510in}{2.278731in}}{\pgfqpoint{3.186610in}{2.275459in}}{\pgfqpoint{3.180786in}{2.269635in}}%
\pgfpathcurveto{\pgfqpoint{3.174962in}{2.263811in}}{\pgfqpoint{3.171690in}{2.255911in}}{\pgfqpoint{3.171690in}{2.247674in}}%
\pgfpathcurveto{\pgfqpoint{3.171690in}{2.239438in}}{\pgfqpoint{3.174962in}{2.231538in}}{\pgfqpoint{3.180786in}{2.225714in}}%
\pgfpathcurveto{\pgfqpoint{3.186610in}{2.219890in}}{\pgfqpoint{3.194510in}{2.216618in}}{\pgfqpoint{3.202746in}{2.216618in}}%
\pgfpathclose%
\pgfusepath{stroke,fill}%
\end{pgfscope}%
\begin{pgfscope}%
\pgfpathrectangle{\pgfqpoint{0.100000in}{0.212622in}}{\pgfqpoint{3.696000in}{3.696000in}}%
\pgfusepath{clip}%
\pgfsetbuttcap%
\pgfsetroundjoin%
\definecolor{currentfill}{rgb}{0.121569,0.466667,0.705882}%
\pgfsetfillcolor{currentfill}%
\pgfsetfillopacity{0.590868}%
\pgfsetlinewidth{1.003750pt}%
\definecolor{currentstroke}{rgb}{0.121569,0.466667,0.705882}%
\pgfsetstrokecolor{currentstroke}%
\pgfsetstrokeopacity{0.590868}%
\pgfsetdash{}{0pt}%
\pgfpathmoveto{\pgfqpoint{0.921766in}{1.292103in}}%
\pgfpathcurveto{\pgfqpoint{0.930003in}{1.292103in}}{\pgfqpoint{0.937903in}{1.295375in}}{\pgfqpoint{0.943727in}{1.301199in}}%
\pgfpathcurveto{\pgfqpoint{0.949551in}{1.307023in}}{\pgfqpoint{0.952823in}{1.314923in}}{\pgfqpoint{0.952823in}{1.323160in}}%
\pgfpathcurveto{\pgfqpoint{0.952823in}{1.331396in}}{\pgfqpoint{0.949551in}{1.339296in}}{\pgfqpoint{0.943727in}{1.345120in}}%
\pgfpathcurveto{\pgfqpoint{0.937903in}{1.350944in}}{\pgfqpoint{0.930003in}{1.354216in}}{\pgfqpoint{0.921766in}{1.354216in}}%
\pgfpathcurveto{\pgfqpoint{0.913530in}{1.354216in}}{\pgfqpoint{0.905630in}{1.350944in}}{\pgfqpoint{0.899806in}{1.345120in}}%
\pgfpathcurveto{\pgfqpoint{0.893982in}{1.339296in}}{\pgfqpoint{0.890710in}{1.331396in}}{\pgfqpoint{0.890710in}{1.323160in}}%
\pgfpathcurveto{\pgfqpoint{0.890710in}{1.314923in}}{\pgfqpoint{0.893982in}{1.307023in}}{\pgfqpoint{0.899806in}{1.301199in}}%
\pgfpathcurveto{\pgfqpoint{0.905630in}{1.295375in}}{\pgfqpoint{0.913530in}{1.292103in}}{\pgfqpoint{0.921766in}{1.292103in}}%
\pgfpathclose%
\pgfusepath{stroke,fill}%
\end{pgfscope}%
\begin{pgfscope}%
\pgfpathrectangle{\pgfqpoint{0.100000in}{0.212622in}}{\pgfqpoint{3.696000in}{3.696000in}}%
\pgfusepath{clip}%
\pgfsetbuttcap%
\pgfsetroundjoin%
\definecolor{currentfill}{rgb}{0.121569,0.466667,0.705882}%
\pgfsetfillcolor{currentfill}%
\pgfsetfillopacity{0.591104}%
\pgfsetlinewidth{1.003750pt}%
\definecolor{currentstroke}{rgb}{0.121569,0.466667,0.705882}%
\pgfsetstrokecolor{currentstroke}%
\pgfsetstrokeopacity{0.591104}%
\pgfsetdash{}{0pt}%
\pgfpathmoveto{\pgfqpoint{3.201895in}{2.216433in}}%
\pgfpathcurveto{\pgfqpoint{3.210132in}{2.216433in}}{\pgfqpoint{3.218032in}{2.219706in}}{\pgfqpoint{3.223856in}{2.225529in}}%
\pgfpathcurveto{\pgfqpoint{3.229680in}{2.231353in}}{\pgfqpoint{3.232952in}{2.239253in}}{\pgfqpoint{3.232952in}{2.247490in}}%
\pgfpathcurveto{\pgfqpoint{3.232952in}{2.255726in}}{\pgfqpoint{3.229680in}{2.263626in}}{\pgfqpoint{3.223856in}{2.269450in}}%
\pgfpathcurveto{\pgfqpoint{3.218032in}{2.275274in}}{\pgfqpoint{3.210132in}{2.278546in}}{\pgfqpoint{3.201895in}{2.278546in}}%
\pgfpathcurveto{\pgfqpoint{3.193659in}{2.278546in}}{\pgfqpoint{3.185759in}{2.275274in}}{\pgfqpoint{3.179935in}{2.269450in}}%
\pgfpathcurveto{\pgfqpoint{3.174111in}{2.263626in}}{\pgfqpoint{3.170839in}{2.255726in}}{\pgfqpoint{3.170839in}{2.247490in}}%
\pgfpathcurveto{\pgfqpoint{3.170839in}{2.239253in}}{\pgfqpoint{3.174111in}{2.231353in}}{\pgfqpoint{3.179935in}{2.225529in}}%
\pgfpathcurveto{\pgfqpoint{3.185759in}{2.219706in}}{\pgfqpoint{3.193659in}{2.216433in}}{\pgfqpoint{3.201895in}{2.216433in}}%
\pgfpathclose%
\pgfusepath{stroke,fill}%
\end{pgfscope}%
\begin{pgfscope}%
\pgfpathrectangle{\pgfqpoint{0.100000in}{0.212622in}}{\pgfqpoint{3.696000in}{3.696000in}}%
\pgfusepath{clip}%
\pgfsetbuttcap%
\pgfsetroundjoin%
\definecolor{currentfill}{rgb}{0.121569,0.466667,0.705882}%
\pgfsetfillcolor{currentfill}%
\pgfsetfillopacity{0.591346}%
\pgfsetlinewidth{1.003750pt}%
\definecolor{currentstroke}{rgb}{0.121569,0.466667,0.705882}%
\pgfsetstrokecolor{currentstroke}%
\pgfsetstrokeopacity{0.591346}%
\pgfsetdash{}{0pt}%
\pgfpathmoveto{\pgfqpoint{0.918828in}{1.293981in}}%
\pgfpathcurveto{\pgfqpoint{0.927064in}{1.293981in}}{\pgfqpoint{0.934964in}{1.297253in}}{\pgfqpoint{0.940788in}{1.303077in}}%
\pgfpathcurveto{\pgfqpoint{0.946612in}{1.308901in}}{\pgfqpoint{0.949885in}{1.316801in}}{\pgfqpoint{0.949885in}{1.325037in}}%
\pgfpathcurveto{\pgfqpoint{0.949885in}{1.333274in}}{\pgfqpoint{0.946612in}{1.341174in}}{\pgfqpoint{0.940788in}{1.346998in}}%
\pgfpathcurveto{\pgfqpoint{0.934964in}{1.352821in}}{\pgfqpoint{0.927064in}{1.356094in}}{\pgfqpoint{0.918828in}{1.356094in}}%
\pgfpathcurveto{\pgfqpoint{0.910592in}{1.356094in}}{\pgfqpoint{0.902692in}{1.352821in}}{\pgfqpoint{0.896868in}{1.346998in}}%
\pgfpathcurveto{\pgfqpoint{0.891044in}{1.341174in}}{\pgfqpoint{0.887772in}{1.333274in}}{\pgfqpoint{0.887772in}{1.325037in}}%
\pgfpathcurveto{\pgfqpoint{0.887772in}{1.316801in}}{\pgfqpoint{0.891044in}{1.308901in}}{\pgfqpoint{0.896868in}{1.303077in}}%
\pgfpathcurveto{\pgfqpoint{0.902692in}{1.297253in}}{\pgfqpoint{0.910592in}{1.293981in}}{\pgfqpoint{0.918828in}{1.293981in}}%
\pgfpathclose%
\pgfusepath{stroke,fill}%
\end{pgfscope}%
\begin{pgfscope}%
\pgfpathrectangle{\pgfqpoint{0.100000in}{0.212622in}}{\pgfqpoint{3.696000in}{3.696000in}}%
\pgfusepath{clip}%
\pgfsetbuttcap%
\pgfsetroundjoin%
\definecolor{currentfill}{rgb}{0.121569,0.466667,0.705882}%
\pgfsetfillcolor{currentfill}%
\pgfsetfillopacity{0.591772}%
\pgfsetlinewidth{1.003750pt}%
\definecolor{currentstroke}{rgb}{0.121569,0.466667,0.705882}%
\pgfsetstrokecolor{currentstroke}%
\pgfsetstrokeopacity{0.591772}%
\pgfsetdash{}{0pt}%
\pgfpathmoveto{\pgfqpoint{3.200721in}{2.215994in}}%
\pgfpathcurveto{\pgfqpoint{3.208957in}{2.215994in}}{\pgfqpoint{3.216857in}{2.219266in}}{\pgfqpoint{3.222681in}{2.225090in}}%
\pgfpathcurveto{\pgfqpoint{3.228505in}{2.230914in}}{\pgfqpoint{3.231777in}{2.238814in}}{\pgfqpoint{3.231777in}{2.247051in}}%
\pgfpathcurveto{\pgfqpoint{3.231777in}{2.255287in}}{\pgfqpoint{3.228505in}{2.263187in}}{\pgfqpoint{3.222681in}{2.269011in}}%
\pgfpathcurveto{\pgfqpoint{3.216857in}{2.274835in}}{\pgfqpoint{3.208957in}{2.278107in}}{\pgfqpoint{3.200721in}{2.278107in}}%
\pgfpathcurveto{\pgfqpoint{3.192485in}{2.278107in}}{\pgfqpoint{3.184584in}{2.274835in}}{\pgfqpoint{3.178761in}{2.269011in}}%
\pgfpathcurveto{\pgfqpoint{3.172937in}{2.263187in}}{\pgfqpoint{3.169664in}{2.255287in}}{\pgfqpoint{3.169664in}{2.247051in}}%
\pgfpathcurveto{\pgfqpoint{3.169664in}{2.238814in}}{\pgfqpoint{3.172937in}{2.230914in}}{\pgfqpoint{3.178761in}{2.225090in}}%
\pgfpathcurveto{\pgfqpoint{3.184584in}{2.219266in}}{\pgfqpoint{3.192485in}{2.215994in}}{\pgfqpoint{3.200721in}{2.215994in}}%
\pgfpathclose%
\pgfusepath{stroke,fill}%
\end{pgfscope}%
\begin{pgfscope}%
\pgfpathrectangle{\pgfqpoint{0.100000in}{0.212622in}}{\pgfqpoint{3.696000in}{3.696000in}}%
\pgfusepath{clip}%
\pgfsetbuttcap%
\pgfsetroundjoin%
\definecolor{currentfill}{rgb}{0.121569,0.466667,0.705882}%
\pgfsetfillcolor{currentfill}%
\pgfsetfillopacity{0.592026}%
\pgfsetlinewidth{1.003750pt}%
\definecolor{currentstroke}{rgb}{0.121569,0.466667,0.705882}%
\pgfsetstrokecolor{currentstroke}%
\pgfsetstrokeopacity{0.592026}%
\pgfsetdash{}{0pt}%
\pgfpathmoveto{\pgfqpoint{0.914788in}{1.296918in}}%
\pgfpathcurveto{\pgfqpoint{0.923024in}{1.296918in}}{\pgfqpoint{0.930924in}{1.300191in}}{\pgfqpoint{0.936748in}{1.306015in}}%
\pgfpathcurveto{\pgfqpoint{0.942572in}{1.311839in}}{\pgfqpoint{0.945844in}{1.319739in}}{\pgfqpoint{0.945844in}{1.327975in}}%
\pgfpathcurveto{\pgfqpoint{0.945844in}{1.336211in}}{\pgfqpoint{0.942572in}{1.344111in}}{\pgfqpoint{0.936748in}{1.349935in}}%
\pgfpathcurveto{\pgfqpoint{0.930924in}{1.355759in}}{\pgfqpoint{0.923024in}{1.359031in}}{\pgfqpoint{0.914788in}{1.359031in}}%
\pgfpathcurveto{\pgfqpoint{0.906552in}{1.359031in}}{\pgfqpoint{0.898652in}{1.355759in}}{\pgfqpoint{0.892828in}{1.349935in}}%
\pgfpathcurveto{\pgfqpoint{0.887004in}{1.344111in}}{\pgfqpoint{0.883731in}{1.336211in}}{\pgfqpoint{0.883731in}{1.327975in}}%
\pgfpathcurveto{\pgfqpoint{0.883731in}{1.319739in}}{\pgfqpoint{0.887004in}{1.311839in}}{\pgfqpoint{0.892828in}{1.306015in}}%
\pgfpathcurveto{\pgfqpoint{0.898652in}{1.300191in}}{\pgfqpoint{0.906552in}{1.296918in}}{\pgfqpoint{0.914788in}{1.296918in}}%
\pgfpathclose%
\pgfusepath{stroke,fill}%
\end{pgfscope}%
\begin{pgfscope}%
\pgfpathrectangle{\pgfqpoint{0.100000in}{0.212622in}}{\pgfqpoint{3.696000in}{3.696000in}}%
\pgfusepath{clip}%
\pgfsetbuttcap%
\pgfsetroundjoin%
\definecolor{currentfill}{rgb}{0.121569,0.466667,0.705882}%
\pgfsetfillcolor{currentfill}%
\pgfsetfillopacity{0.592137}%
\pgfsetlinewidth{1.003750pt}%
\definecolor{currentstroke}{rgb}{0.121569,0.466667,0.705882}%
\pgfsetstrokecolor{currentstroke}%
\pgfsetstrokeopacity{0.592137}%
\pgfsetdash{}{0pt}%
\pgfpathmoveto{\pgfqpoint{3.200080in}{2.215731in}}%
\pgfpathcurveto{\pgfqpoint{3.208316in}{2.215731in}}{\pgfqpoint{3.216216in}{2.219003in}}{\pgfqpoint{3.222040in}{2.224827in}}%
\pgfpathcurveto{\pgfqpoint{3.227864in}{2.230651in}}{\pgfqpoint{3.231136in}{2.238551in}}{\pgfqpoint{3.231136in}{2.246788in}}%
\pgfpathcurveto{\pgfqpoint{3.231136in}{2.255024in}}{\pgfqpoint{3.227864in}{2.262924in}}{\pgfqpoint{3.222040in}{2.268748in}}%
\pgfpathcurveto{\pgfqpoint{3.216216in}{2.274572in}}{\pgfqpoint{3.208316in}{2.277844in}}{\pgfqpoint{3.200080in}{2.277844in}}%
\pgfpathcurveto{\pgfqpoint{3.191843in}{2.277844in}}{\pgfqpoint{3.183943in}{2.274572in}}{\pgfqpoint{3.178120in}{2.268748in}}%
\pgfpathcurveto{\pgfqpoint{3.172296in}{2.262924in}}{\pgfqpoint{3.169023in}{2.255024in}}{\pgfqpoint{3.169023in}{2.246788in}}%
\pgfpathcurveto{\pgfqpoint{3.169023in}{2.238551in}}{\pgfqpoint{3.172296in}{2.230651in}}{\pgfqpoint{3.178120in}{2.224827in}}%
\pgfpathcurveto{\pgfqpoint{3.183943in}{2.219003in}}{\pgfqpoint{3.191843in}{2.215731in}}{\pgfqpoint{3.200080in}{2.215731in}}%
\pgfpathclose%
\pgfusepath{stroke,fill}%
\end{pgfscope}%
\begin{pgfscope}%
\pgfpathrectangle{\pgfqpoint{0.100000in}{0.212622in}}{\pgfqpoint{3.696000in}{3.696000in}}%
\pgfusepath{clip}%
\pgfsetbuttcap%
\pgfsetroundjoin%
\definecolor{currentfill}{rgb}{0.121569,0.466667,0.705882}%
\pgfsetfillcolor{currentfill}%
\pgfsetfillopacity{0.592224}%
\pgfsetlinewidth{1.003750pt}%
\definecolor{currentstroke}{rgb}{0.121569,0.466667,0.705882}%
\pgfsetstrokecolor{currentstroke}%
\pgfsetstrokeopacity{0.592224}%
\pgfsetdash{}{0pt}%
\pgfpathmoveto{\pgfqpoint{0.850875in}{1.424873in}}%
\pgfpathcurveto{\pgfqpoint{0.859111in}{1.424873in}}{\pgfqpoint{0.867011in}{1.428145in}}{\pgfqpoint{0.872835in}{1.433969in}}%
\pgfpathcurveto{\pgfqpoint{0.878659in}{1.439793in}}{\pgfqpoint{0.881931in}{1.447693in}}{\pgfqpoint{0.881931in}{1.455929in}}%
\pgfpathcurveto{\pgfqpoint{0.881931in}{1.464165in}}{\pgfqpoint{0.878659in}{1.472065in}}{\pgfqpoint{0.872835in}{1.477889in}}%
\pgfpathcurveto{\pgfqpoint{0.867011in}{1.483713in}}{\pgfqpoint{0.859111in}{1.486986in}}{\pgfqpoint{0.850875in}{1.486986in}}%
\pgfpathcurveto{\pgfqpoint{0.842639in}{1.486986in}}{\pgfqpoint{0.834739in}{1.483713in}}{\pgfqpoint{0.828915in}{1.477889in}}%
\pgfpathcurveto{\pgfqpoint{0.823091in}{1.472065in}}{\pgfqpoint{0.819818in}{1.464165in}}{\pgfqpoint{0.819818in}{1.455929in}}%
\pgfpathcurveto{\pgfqpoint{0.819818in}{1.447693in}}{\pgfqpoint{0.823091in}{1.439793in}}{\pgfqpoint{0.828915in}{1.433969in}}%
\pgfpathcurveto{\pgfqpoint{0.834739in}{1.428145in}}{\pgfqpoint{0.842639in}{1.424873in}}{\pgfqpoint{0.850875in}{1.424873in}}%
\pgfpathclose%
\pgfusepath{stroke,fill}%
\end{pgfscope}%
\begin{pgfscope}%
\pgfpathrectangle{\pgfqpoint{0.100000in}{0.212622in}}{\pgfqpoint{3.696000in}{3.696000in}}%
\pgfusepath{clip}%
\pgfsetbuttcap%
\pgfsetroundjoin%
\definecolor{currentfill}{rgb}{0.121569,0.466667,0.705882}%
\pgfsetfillcolor{currentfill}%
\pgfsetfillopacity{0.592627}%
\pgfsetlinewidth{1.003750pt}%
\definecolor{currentstroke}{rgb}{0.121569,0.466667,0.705882}%
\pgfsetstrokecolor{currentstroke}%
\pgfsetstrokeopacity{0.592627}%
\pgfsetdash{}{0pt}%
\pgfpathmoveto{\pgfqpoint{3.199126in}{2.215233in}}%
\pgfpathcurveto{\pgfqpoint{3.207363in}{2.215233in}}{\pgfqpoint{3.215263in}{2.218505in}}{\pgfqpoint{3.221087in}{2.224329in}}%
\pgfpathcurveto{\pgfqpoint{3.226911in}{2.230153in}}{\pgfqpoint{3.230183in}{2.238053in}}{\pgfqpoint{3.230183in}{2.246289in}}%
\pgfpathcurveto{\pgfqpoint{3.230183in}{2.254525in}}{\pgfqpoint{3.226911in}{2.262425in}}{\pgfqpoint{3.221087in}{2.268249in}}%
\pgfpathcurveto{\pgfqpoint{3.215263in}{2.274073in}}{\pgfqpoint{3.207363in}{2.277346in}}{\pgfqpoint{3.199126in}{2.277346in}}%
\pgfpathcurveto{\pgfqpoint{3.190890in}{2.277346in}}{\pgfqpoint{3.182990in}{2.274073in}}{\pgfqpoint{3.177166in}{2.268249in}}%
\pgfpathcurveto{\pgfqpoint{3.171342in}{2.262425in}}{\pgfqpoint{3.168070in}{2.254525in}}{\pgfqpoint{3.168070in}{2.246289in}}%
\pgfpathcurveto{\pgfqpoint{3.168070in}{2.238053in}}{\pgfqpoint{3.171342in}{2.230153in}}{\pgfqpoint{3.177166in}{2.224329in}}%
\pgfpathcurveto{\pgfqpoint{3.182990in}{2.218505in}}{\pgfqpoint{3.190890in}{2.215233in}}{\pgfqpoint{3.199126in}{2.215233in}}%
\pgfpathclose%
\pgfusepath{stroke,fill}%
\end{pgfscope}%
\begin{pgfscope}%
\pgfpathrectangle{\pgfqpoint{0.100000in}{0.212622in}}{\pgfqpoint{3.696000in}{3.696000in}}%
\pgfusepath{clip}%
\pgfsetbuttcap%
\pgfsetroundjoin%
\definecolor{currentfill}{rgb}{0.121569,0.466667,0.705882}%
\pgfsetfillcolor{currentfill}%
\pgfsetfillopacity{0.593042}%
\pgfsetlinewidth{1.003750pt}%
\definecolor{currentstroke}{rgb}{0.121569,0.466667,0.705882}%
\pgfsetstrokecolor{currentstroke}%
\pgfsetstrokeopacity{0.593042}%
\pgfsetdash{}{0pt}%
\pgfpathmoveto{\pgfqpoint{0.908969in}{1.301744in}}%
\pgfpathcurveto{\pgfqpoint{0.917205in}{1.301744in}}{\pgfqpoint{0.925105in}{1.305016in}}{\pgfqpoint{0.930929in}{1.310840in}}%
\pgfpathcurveto{\pgfqpoint{0.936753in}{1.316664in}}{\pgfqpoint{0.940026in}{1.324564in}}{\pgfqpoint{0.940026in}{1.332801in}}%
\pgfpathcurveto{\pgfqpoint{0.940026in}{1.341037in}}{\pgfqpoint{0.936753in}{1.348937in}}{\pgfqpoint{0.930929in}{1.354761in}}%
\pgfpathcurveto{\pgfqpoint{0.925105in}{1.360585in}}{\pgfqpoint{0.917205in}{1.363857in}}{\pgfqpoint{0.908969in}{1.363857in}}%
\pgfpathcurveto{\pgfqpoint{0.900733in}{1.363857in}}{\pgfqpoint{0.892833in}{1.360585in}}{\pgfqpoint{0.887009in}{1.354761in}}%
\pgfpathcurveto{\pgfqpoint{0.881185in}{1.348937in}}{\pgfqpoint{0.877913in}{1.341037in}}{\pgfqpoint{0.877913in}{1.332801in}}%
\pgfpathcurveto{\pgfqpoint{0.877913in}{1.324564in}}{\pgfqpoint{0.881185in}{1.316664in}}{\pgfqpoint{0.887009in}{1.310840in}}%
\pgfpathcurveto{\pgfqpoint{0.892833in}{1.305016in}}{\pgfqpoint{0.900733in}{1.301744in}}{\pgfqpoint{0.908969in}{1.301744in}}%
\pgfpathclose%
\pgfusepath{stroke,fill}%
\end{pgfscope}%
\begin{pgfscope}%
\pgfpathrectangle{\pgfqpoint{0.100000in}{0.212622in}}{\pgfqpoint{3.696000in}{3.696000in}}%
\pgfusepath{clip}%
\pgfsetbuttcap%
\pgfsetroundjoin%
\definecolor{currentfill}{rgb}{0.121569,0.466667,0.705882}%
\pgfsetfillcolor{currentfill}%
\pgfsetfillopacity{0.593572}%
\pgfsetlinewidth{1.003750pt}%
\definecolor{currentstroke}{rgb}{0.121569,0.466667,0.705882}%
\pgfsetstrokecolor{currentstroke}%
\pgfsetstrokeopacity{0.593572}%
\pgfsetdash{}{0pt}%
\pgfpathmoveto{\pgfqpoint{3.197465in}{2.214928in}}%
\pgfpathcurveto{\pgfqpoint{3.205701in}{2.214928in}}{\pgfqpoint{3.213601in}{2.218200in}}{\pgfqpoint{3.219425in}{2.224024in}}%
\pgfpathcurveto{\pgfqpoint{3.225249in}{2.229848in}}{\pgfqpoint{3.228521in}{2.237748in}}{\pgfqpoint{3.228521in}{2.245984in}}%
\pgfpathcurveto{\pgfqpoint{3.228521in}{2.254220in}}{\pgfqpoint{3.225249in}{2.262121in}}{\pgfqpoint{3.219425in}{2.267944in}}%
\pgfpathcurveto{\pgfqpoint{3.213601in}{2.273768in}}{\pgfqpoint{3.205701in}{2.277041in}}{\pgfqpoint{3.197465in}{2.277041in}}%
\pgfpathcurveto{\pgfqpoint{3.189228in}{2.277041in}}{\pgfqpoint{3.181328in}{2.273768in}}{\pgfqpoint{3.175504in}{2.267944in}}%
\pgfpathcurveto{\pgfqpoint{3.169681in}{2.262121in}}{\pgfqpoint{3.166408in}{2.254220in}}{\pgfqpoint{3.166408in}{2.245984in}}%
\pgfpathcurveto{\pgfqpoint{3.166408in}{2.237748in}}{\pgfqpoint{3.169681in}{2.229848in}}{\pgfqpoint{3.175504in}{2.224024in}}%
\pgfpathcurveto{\pgfqpoint{3.181328in}{2.218200in}}{\pgfqpoint{3.189228in}{2.214928in}}{\pgfqpoint{3.197465in}{2.214928in}}%
\pgfpathclose%
\pgfusepath{stroke,fill}%
\end{pgfscope}%
\begin{pgfscope}%
\pgfpathrectangle{\pgfqpoint{0.100000in}{0.212622in}}{\pgfqpoint{3.696000in}{3.696000in}}%
\pgfusepath{clip}%
\pgfsetbuttcap%
\pgfsetroundjoin%
\definecolor{currentfill}{rgb}{0.121569,0.466667,0.705882}%
\pgfsetfillcolor{currentfill}%
\pgfsetfillopacity{0.593831}%
\pgfsetlinewidth{1.003750pt}%
\definecolor{currentstroke}{rgb}{0.121569,0.466667,0.705882}%
\pgfsetstrokecolor{currentstroke}%
\pgfsetstrokeopacity{0.593831}%
\pgfsetdash{}{0pt}%
\pgfpathmoveto{\pgfqpoint{0.851545in}{1.413131in}}%
\pgfpathcurveto{\pgfqpoint{0.859782in}{1.413131in}}{\pgfqpoint{0.867682in}{1.416404in}}{\pgfqpoint{0.873506in}{1.422228in}}%
\pgfpathcurveto{\pgfqpoint{0.879329in}{1.428052in}}{\pgfqpoint{0.882602in}{1.435952in}}{\pgfqpoint{0.882602in}{1.444188in}}%
\pgfpathcurveto{\pgfqpoint{0.882602in}{1.452424in}}{\pgfqpoint{0.879329in}{1.460324in}}{\pgfqpoint{0.873506in}{1.466148in}}%
\pgfpathcurveto{\pgfqpoint{0.867682in}{1.471972in}}{\pgfqpoint{0.859782in}{1.475244in}}{\pgfqpoint{0.851545in}{1.475244in}}%
\pgfpathcurveto{\pgfqpoint{0.843309in}{1.475244in}}{\pgfqpoint{0.835409in}{1.471972in}}{\pgfqpoint{0.829585in}{1.466148in}}%
\pgfpathcurveto{\pgfqpoint{0.823761in}{1.460324in}}{\pgfqpoint{0.820489in}{1.452424in}}{\pgfqpoint{0.820489in}{1.444188in}}%
\pgfpathcurveto{\pgfqpoint{0.820489in}{1.435952in}}{\pgfqpoint{0.823761in}{1.428052in}}{\pgfqpoint{0.829585in}{1.422228in}}%
\pgfpathcurveto{\pgfqpoint{0.835409in}{1.416404in}}{\pgfqpoint{0.843309in}{1.413131in}}{\pgfqpoint{0.851545in}{1.413131in}}%
\pgfpathclose%
\pgfusepath{stroke,fill}%
\end{pgfscope}%
\begin{pgfscope}%
\pgfpathrectangle{\pgfqpoint{0.100000in}{0.212622in}}{\pgfqpoint{3.696000in}{3.696000in}}%
\pgfusepath{clip}%
\pgfsetbuttcap%
\pgfsetroundjoin%
\definecolor{currentfill}{rgb}{0.121569,0.466667,0.705882}%
\pgfsetfillcolor{currentfill}%
\pgfsetfillopacity{0.594123}%
\pgfsetlinewidth{1.003750pt}%
\definecolor{currentstroke}{rgb}{0.121569,0.466667,0.705882}%
\pgfsetstrokecolor{currentstroke}%
\pgfsetstrokeopacity{0.594123}%
\pgfsetdash{}{0pt}%
\pgfpathmoveto{\pgfqpoint{3.196594in}{2.214919in}}%
\pgfpathcurveto{\pgfqpoint{3.204831in}{2.214919in}}{\pgfqpoint{3.212731in}{2.218192in}}{\pgfqpoint{3.218555in}{2.224016in}}%
\pgfpathcurveto{\pgfqpoint{3.224379in}{2.229840in}}{\pgfqpoint{3.227651in}{2.237740in}}{\pgfqpoint{3.227651in}{2.245976in}}%
\pgfpathcurveto{\pgfqpoint{3.227651in}{2.254212in}}{\pgfqpoint{3.224379in}{2.262112in}}{\pgfqpoint{3.218555in}{2.267936in}}%
\pgfpathcurveto{\pgfqpoint{3.212731in}{2.273760in}}{\pgfqpoint{3.204831in}{2.277032in}}{\pgfqpoint{3.196594in}{2.277032in}}%
\pgfpathcurveto{\pgfqpoint{3.188358in}{2.277032in}}{\pgfqpoint{3.180458in}{2.273760in}}{\pgfqpoint{3.174634in}{2.267936in}}%
\pgfpathcurveto{\pgfqpoint{3.168810in}{2.262112in}}{\pgfqpoint{3.165538in}{2.254212in}}{\pgfqpoint{3.165538in}{2.245976in}}%
\pgfpathcurveto{\pgfqpoint{3.165538in}{2.237740in}}{\pgfqpoint{3.168810in}{2.229840in}}{\pgfqpoint{3.174634in}{2.224016in}}%
\pgfpathcurveto{\pgfqpoint{3.180458in}{2.218192in}}{\pgfqpoint{3.188358in}{2.214919in}}{\pgfqpoint{3.196594in}{2.214919in}}%
\pgfpathclose%
\pgfusepath{stroke,fill}%
\end{pgfscope}%
\begin{pgfscope}%
\pgfpathrectangle{\pgfqpoint{0.100000in}{0.212622in}}{\pgfqpoint{3.696000in}{3.696000in}}%
\pgfusepath{clip}%
\pgfsetbuttcap%
\pgfsetroundjoin%
\definecolor{currentfill}{rgb}{0.121569,0.466667,0.705882}%
\pgfsetfillcolor{currentfill}%
\pgfsetfillopacity{0.594310}%
\pgfsetlinewidth{1.003750pt}%
\definecolor{currentstroke}{rgb}{0.121569,0.466667,0.705882}%
\pgfsetstrokecolor{currentstroke}%
\pgfsetstrokeopacity{0.594310}%
\pgfsetdash{}{0pt}%
\pgfpathmoveto{\pgfqpoint{0.901830in}{1.308265in}}%
\pgfpathcurveto{\pgfqpoint{0.910066in}{1.308265in}}{\pgfqpoint{0.917966in}{1.311537in}}{\pgfqpoint{0.923790in}{1.317361in}}%
\pgfpathcurveto{\pgfqpoint{0.929614in}{1.323185in}}{\pgfqpoint{0.932886in}{1.331085in}}{\pgfqpoint{0.932886in}{1.339322in}}%
\pgfpathcurveto{\pgfqpoint{0.932886in}{1.347558in}}{\pgfqpoint{0.929614in}{1.355458in}}{\pgfqpoint{0.923790in}{1.361282in}}%
\pgfpathcurveto{\pgfqpoint{0.917966in}{1.367106in}}{\pgfqpoint{0.910066in}{1.370378in}}{\pgfqpoint{0.901830in}{1.370378in}}%
\pgfpathcurveto{\pgfqpoint{0.893594in}{1.370378in}}{\pgfqpoint{0.885694in}{1.367106in}}{\pgfqpoint{0.879870in}{1.361282in}}%
\pgfpathcurveto{\pgfqpoint{0.874046in}{1.355458in}}{\pgfqpoint{0.870773in}{1.347558in}}{\pgfqpoint{0.870773in}{1.339322in}}%
\pgfpathcurveto{\pgfqpoint{0.870773in}{1.331085in}}{\pgfqpoint{0.874046in}{1.323185in}}{\pgfqpoint{0.879870in}{1.317361in}}%
\pgfpathcurveto{\pgfqpoint{0.885694in}{1.311537in}}{\pgfqpoint{0.893594in}{1.308265in}}{\pgfqpoint{0.901830in}{1.308265in}}%
\pgfpathclose%
\pgfusepath{stroke,fill}%
\end{pgfscope}%
\begin{pgfscope}%
\pgfpathrectangle{\pgfqpoint{0.100000in}{0.212622in}}{\pgfqpoint{3.696000in}{3.696000in}}%
\pgfusepath{clip}%
\pgfsetbuttcap%
\pgfsetroundjoin%
\definecolor{currentfill}{rgb}{0.121569,0.466667,0.705882}%
\pgfsetfillcolor{currentfill}%
\pgfsetfillopacity{0.594411}%
\pgfsetlinewidth{1.003750pt}%
\definecolor{currentstroke}{rgb}{0.121569,0.466667,0.705882}%
\pgfsetstrokecolor{currentstroke}%
\pgfsetstrokeopacity{0.594411}%
\pgfsetdash{}{0pt}%
\pgfpathmoveto{\pgfqpoint{3.196090in}{2.214842in}}%
\pgfpathcurveto{\pgfqpoint{3.204327in}{2.214842in}}{\pgfqpoint{3.212227in}{2.218114in}}{\pgfqpoint{3.218051in}{2.223938in}}%
\pgfpathcurveto{\pgfqpoint{3.223874in}{2.229762in}}{\pgfqpoint{3.227147in}{2.237662in}}{\pgfqpoint{3.227147in}{2.245898in}}%
\pgfpathcurveto{\pgfqpoint{3.227147in}{2.254135in}}{\pgfqpoint{3.223874in}{2.262035in}}{\pgfqpoint{3.218051in}{2.267859in}}%
\pgfpathcurveto{\pgfqpoint{3.212227in}{2.273683in}}{\pgfqpoint{3.204327in}{2.276955in}}{\pgfqpoint{3.196090in}{2.276955in}}%
\pgfpathcurveto{\pgfqpoint{3.187854in}{2.276955in}}{\pgfqpoint{3.179954in}{2.273683in}}{\pgfqpoint{3.174130in}{2.267859in}}%
\pgfpathcurveto{\pgfqpoint{3.168306in}{2.262035in}}{\pgfqpoint{3.165034in}{2.254135in}}{\pgfqpoint{3.165034in}{2.245898in}}%
\pgfpathcurveto{\pgfqpoint{3.165034in}{2.237662in}}{\pgfqpoint{3.168306in}{2.229762in}}{\pgfqpoint{3.174130in}{2.223938in}}%
\pgfpathcurveto{\pgfqpoint{3.179954in}{2.218114in}}{\pgfqpoint{3.187854in}{2.214842in}}{\pgfqpoint{3.196090in}{2.214842in}}%
\pgfpathclose%
\pgfusepath{stroke,fill}%
\end{pgfscope}%
\begin{pgfscope}%
\pgfpathrectangle{\pgfqpoint{0.100000in}{0.212622in}}{\pgfqpoint{3.696000in}{3.696000in}}%
\pgfusepath{clip}%
\pgfsetbuttcap%
\pgfsetroundjoin%
\definecolor{currentfill}{rgb}{0.121569,0.466667,0.705882}%
\pgfsetfillcolor{currentfill}%
\pgfsetfillopacity{0.595048}%
\pgfsetlinewidth{1.003750pt}%
\definecolor{currentstroke}{rgb}{0.121569,0.466667,0.705882}%
\pgfsetstrokecolor{currentstroke}%
\pgfsetstrokeopacity{0.595048}%
\pgfsetdash{}{0pt}%
\pgfpathmoveto{\pgfqpoint{3.194995in}{2.214585in}}%
\pgfpathcurveto{\pgfqpoint{3.203231in}{2.214585in}}{\pgfqpoint{3.211131in}{2.217857in}}{\pgfqpoint{3.216955in}{2.223681in}}%
\pgfpathcurveto{\pgfqpoint{3.222779in}{2.229505in}}{\pgfqpoint{3.226051in}{2.237405in}}{\pgfqpoint{3.226051in}{2.245641in}}%
\pgfpathcurveto{\pgfqpoint{3.226051in}{2.253877in}}{\pgfqpoint{3.222779in}{2.261778in}}{\pgfqpoint{3.216955in}{2.267601in}}%
\pgfpathcurveto{\pgfqpoint{3.211131in}{2.273425in}}{\pgfqpoint{3.203231in}{2.276698in}}{\pgfqpoint{3.194995in}{2.276698in}}%
\pgfpathcurveto{\pgfqpoint{3.186758in}{2.276698in}}{\pgfqpoint{3.178858in}{2.273425in}}{\pgfqpoint{3.173034in}{2.267601in}}%
\pgfpathcurveto{\pgfqpoint{3.167210in}{2.261778in}}{\pgfqpoint{3.163938in}{2.253877in}}{\pgfqpoint{3.163938in}{2.245641in}}%
\pgfpathcurveto{\pgfqpoint{3.163938in}{2.237405in}}{\pgfqpoint{3.167210in}{2.229505in}}{\pgfqpoint{3.173034in}{2.223681in}}%
\pgfpathcurveto{\pgfqpoint{3.178858in}{2.217857in}}{\pgfqpoint{3.186758in}{2.214585in}}{\pgfqpoint{3.194995in}{2.214585in}}%
\pgfpathclose%
\pgfusepath{stroke,fill}%
\end{pgfscope}%
\begin{pgfscope}%
\pgfpathrectangle{\pgfqpoint{0.100000in}{0.212622in}}{\pgfqpoint{3.696000in}{3.696000in}}%
\pgfusepath{clip}%
\pgfsetbuttcap%
\pgfsetroundjoin%
\definecolor{currentfill}{rgb}{0.121569,0.466667,0.705882}%
\pgfsetfillcolor{currentfill}%
\pgfsetfillopacity{0.595069}%
\pgfsetlinewidth{1.003750pt}%
\definecolor{currentstroke}{rgb}{0.121569,0.466667,0.705882}%
\pgfsetstrokecolor{currentstroke}%
\pgfsetstrokeopacity{0.595069}%
\pgfsetdash{}{0pt}%
\pgfpathmoveto{\pgfqpoint{0.852506in}{1.403969in}}%
\pgfpathcurveto{\pgfqpoint{0.860742in}{1.403969in}}{\pgfqpoint{0.868642in}{1.407241in}}{\pgfqpoint{0.874466in}{1.413065in}}%
\pgfpathcurveto{\pgfqpoint{0.880290in}{1.418889in}}{\pgfqpoint{0.883562in}{1.426789in}}{\pgfqpoint{0.883562in}{1.435025in}}%
\pgfpathcurveto{\pgfqpoint{0.883562in}{1.443261in}}{\pgfqpoint{0.880290in}{1.451161in}}{\pgfqpoint{0.874466in}{1.456985in}}%
\pgfpathcurveto{\pgfqpoint{0.868642in}{1.462809in}}{\pgfqpoint{0.860742in}{1.466082in}}{\pgfqpoint{0.852506in}{1.466082in}}%
\pgfpathcurveto{\pgfqpoint{0.844269in}{1.466082in}}{\pgfqpoint{0.836369in}{1.462809in}}{\pgfqpoint{0.830545in}{1.456985in}}%
\pgfpathcurveto{\pgfqpoint{0.824721in}{1.451161in}}{\pgfqpoint{0.821449in}{1.443261in}}{\pgfqpoint{0.821449in}{1.435025in}}%
\pgfpathcurveto{\pgfqpoint{0.821449in}{1.426789in}}{\pgfqpoint{0.824721in}{1.418889in}}{\pgfqpoint{0.830545in}{1.413065in}}%
\pgfpathcurveto{\pgfqpoint{0.836369in}{1.407241in}}{\pgfqpoint{0.844269in}{1.403969in}}{\pgfqpoint{0.852506in}{1.403969in}}%
\pgfpathclose%
\pgfusepath{stroke,fill}%
\end{pgfscope}%
\begin{pgfscope}%
\pgfpathrectangle{\pgfqpoint{0.100000in}{0.212622in}}{\pgfqpoint{3.696000in}{3.696000in}}%
\pgfusepath{clip}%
\pgfsetbuttcap%
\pgfsetroundjoin%
\definecolor{currentfill}{rgb}{0.121569,0.466667,0.705882}%
\pgfsetfillcolor{currentfill}%
\pgfsetfillopacity{0.595859}%
\pgfsetlinewidth{1.003750pt}%
\definecolor{currentstroke}{rgb}{0.121569,0.466667,0.705882}%
\pgfsetstrokecolor{currentstroke}%
\pgfsetstrokeopacity{0.595859}%
\pgfsetdash{}{0pt}%
\pgfpathmoveto{\pgfqpoint{0.893374in}{1.316704in}}%
\pgfpathcurveto{\pgfqpoint{0.901611in}{1.316704in}}{\pgfqpoint{0.909511in}{1.319976in}}{\pgfqpoint{0.915335in}{1.325800in}}%
\pgfpathcurveto{\pgfqpoint{0.921159in}{1.331624in}}{\pgfqpoint{0.924431in}{1.339524in}}{\pgfqpoint{0.924431in}{1.347760in}}%
\pgfpathcurveto{\pgfqpoint{0.924431in}{1.355997in}}{\pgfqpoint{0.921159in}{1.363897in}}{\pgfqpoint{0.915335in}{1.369721in}}%
\pgfpathcurveto{\pgfqpoint{0.909511in}{1.375545in}}{\pgfqpoint{0.901611in}{1.378817in}}{\pgfqpoint{0.893374in}{1.378817in}}%
\pgfpathcurveto{\pgfqpoint{0.885138in}{1.378817in}}{\pgfqpoint{0.877238in}{1.375545in}}{\pgfqpoint{0.871414in}{1.369721in}}%
\pgfpathcurveto{\pgfqpoint{0.865590in}{1.363897in}}{\pgfqpoint{0.862318in}{1.355997in}}{\pgfqpoint{0.862318in}{1.347760in}}%
\pgfpathcurveto{\pgfqpoint{0.862318in}{1.339524in}}{\pgfqpoint{0.865590in}{1.331624in}}{\pgfqpoint{0.871414in}{1.325800in}}%
\pgfpathcurveto{\pgfqpoint{0.877238in}{1.319976in}}{\pgfqpoint{0.885138in}{1.316704in}}{\pgfqpoint{0.893374in}{1.316704in}}%
\pgfpathclose%
\pgfusepath{stroke,fill}%
\end{pgfscope}%
\begin{pgfscope}%
\pgfpathrectangle{\pgfqpoint{0.100000in}{0.212622in}}{\pgfqpoint{3.696000in}{3.696000in}}%
\pgfusepath{clip}%
\pgfsetbuttcap%
\pgfsetroundjoin%
\definecolor{currentfill}{rgb}{0.121569,0.466667,0.705882}%
\pgfsetfillcolor{currentfill}%
\pgfsetfillopacity{0.595985}%
\pgfsetlinewidth{1.003750pt}%
\definecolor{currentstroke}{rgb}{0.121569,0.466667,0.705882}%
\pgfsetstrokecolor{currentstroke}%
\pgfsetstrokeopacity{0.595985}%
\pgfsetdash{}{0pt}%
\pgfpathmoveto{\pgfqpoint{0.853354in}{1.396745in}}%
\pgfpathcurveto{\pgfqpoint{0.861590in}{1.396745in}}{\pgfqpoint{0.869490in}{1.400017in}}{\pgfqpoint{0.875314in}{1.405841in}}%
\pgfpathcurveto{\pgfqpoint{0.881138in}{1.411665in}}{\pgfqpoint{0.884410in}{1.419565in}}{\pgfqpoint{0.884410in}{1.427801in}}%
\pgfpathcurveto{\pgfqpoint{0.884410in}{1.436038in}}{\pgfqpoint{0.881138in}{1.443938in}}{\pgfqpoint{0.875314in}{1.449762in}}%
\pgfpathcurveto{\pgfqpoint{0.869490in}{1.455586in}}{\pgfqpoint{0.861590in}{1.458858in}}{\pgfqpoint{0.853354in}{1.458858in}}%
\pgfpathcurveto{\pgfqpoint{0.845117in}{1.458858in}}{\pgfqpoint{0.837217in}{1.455586in}}{\pgfqpoint{0.831393in}{1.449762in}}%
\pgfpathcurveto{\pgfqpoint{0.825569in}{1.443938in}}{\pgfqpoint{0.822297in}{1.436038in}}{\pgfqpoint{0.822297in}{1.427801in}}%
\pgfpathcurveto{\pgfqpoint{0.822297in}{1.419565in}}{\pgfqpoint{0.825569in}{1.411665in}}{\pgfqpoint{0.831393in}{1.405841in}}%
\pgfpathcurveto{\pgfqpoint{0.837217in}{1.400017in}}{\pgfqpoint{0.845117in}{1.396745in}}{\pgfqpoint{0.853354in}{1.396745in}}%
\pgfpathclose%
\pgfusepath{stroke,fill}%
\end{pgfscope}%
\begin{pgfscope}%
\pgfpathrectangle{\pgfqpoint{0.100000in}{0.212622in}}{\pgfqpoint{3.696000in}{3.696000in}}%
\pgfusepath{clip}%
\pgfsetbuttcap%
\pgfsetroundjoin%
\definecolor{currentfill}{rgb}{0.121569,0.466667,0.705882}%
\pgfsetfillcolor{currentfill}%
\pgfsetfillopacity{0.596087}%
\pgfsetlinewidth{1.003750pt}%
\definecolor{currentstroke}{rgb}{0.121569,0.466667,0.705882}%
\pgfsetstrokecolor{currentstroke}%
\pgfsetstrokeopacity{0.596087}%
\pgfsetdash{}{0pt}%
\pgfpathmoveto{\pgfqpoint{3.193385in}{2.214818in}}%
\pgfpathcurveto{\pgfqpoint{3.201622in}{2.214818in}}{\pgfqpoint{3.209522in}{2.218090in}}{\pgfqpoint{3.215346in}{2.223914in}}%
\pgfpathcurveto{\pgfqpoint{3.221170in}{2.229738in}}{\pgfqpoint{3.224442in}{2.237638in}}{\pgfqpoint{3.224442in}{2.245874in}}%
\pgfpathcurveto{\pgfqpoint{3.224442in}{2.254110in}}{\pgfqpoint{3.221170in}{2.262010in}}{\pgfqpoint{3.215346in}{2.267834in}}%
\pgfpathcurveto{\pgfqpoint{3.209522in}{2.273658in}}{\pgfqpoint{3.201622in}{2.276931in}}{\pgfqpoint{3.193385in}{2.276931in}}%
\pgfpathcurveto{\pgfqpoint{3.185149in}{2.276931in}}{\pgfqpoint{3.177249in}{2.273658in}}{\pgfqpoint{3.171425in}{2.267834in}}%
\pgfpathcurveto{\pgfqpoint{3.165601in}{2.262010in}}{\pgfqpoint{3.162329in}{2.254110in}}{\pgfqpoint{3.162329in}{2.245874in}}%
\pgfpathcurveto{\pgfqpoint{3.162329in}{2.237638in}}{\pgfqpoint{3.165601in}{2.229738in}}{\pgfqpoint{3.171425in}{2.223914in}}%
\pgfpathcurveto{\pgfqpoint{3.177249in}{2.218090in}}{\pgfqpoint{3.185149in}{2.214818in}}{\pgfqpoint{3.193385in}{2.214818in}}%
\pgfpathclose%
\pgfusepath{stroke,fill}%
\end{pgfscope}%
\begin{pgfscope}%
\pgfpathrectangle{\pgfqpoint{0.100000in}{0.212622in}}{\pgfqpoint{3.696000in}{3.696000in}}%
\pgfusepath{clip}%
\pgfsetbuttcap%
\pgfsetroundjoin%
\definecolor{currentfill}{rgb}{0.121569,0.466667,0.705882}%
\pgfsetfillcolor{currentfill}%
\pgfsetfillopacity{0.596638}%
\pgfsetlinewidth{1.003750pt}%
\definecolor{currentstroke}{rgb}{0.121569,0.466667,0.705882}%
\pgfsetstrokecolor{currentstroke}%
\pgfsetstrokeopacity{0.596638}%
\pgfsetdash{}{0pt}%
\pgfpathmoveto{\pgfqpoint{3.192518in}{2.214781in}}%
\pgfpathcurveto{\pgfqpoint{3.200754in}{2.214781in}}{\pgfqpoint{3.208654in}{2.218053in}}{\pgfqpoint{3.214478in}{2.223877in}}%
\pgfpathcurveto{\pgfqpoint{3.220302in}{2.229701in}}{\pgfqpoint{3.223574in}{2.237601in}}{\pgfqpoint{3.223574in}{2.245837in}}%
\pgfpathcurveto{\pgfqpoint{3.223574in}{2.254074in}}{\pgfqpoint{3.220302in}{2.261974in}}{\pgfqpoint{3.214478in}{2.267798in}}%
\pgfpathcurveto{\pgfqpoint{3.208654in}{2.273622in}}{\pgfqpoint{3.200754in}{2.276894in}}{\pgfqpoint{3.192518in}{2.276894in}}%
\pgfpathcurveto{\pgfqpoint{3.184282in}{2.276894in}}{\pgfqpoint{3.176382in}{2.273622in}}{\pgfqpoint{3.170558in}{2.267798in}}%
\pgfpathcurveto{\pgfqpoint{3.164734in}{2.261974in}}{\pgfqpoint{3.161461in}{2.254074in}}{\pgfqpoint{3.161461in}{2.245837in}}%
\pgfpathcurveto{\pgfqpoint{3.161461in}{2.237601in}}{\pgfqpoint{3.164734in}{2.229701in}}{\pgfqpoint{3.170558in}{2.223877in}}%
\pgfpathcurveto{\pgfqpoint{3.176382in}{2.218053in}}{\pgfqpoint{3.184282in}{2.214781in}}{\pgfqpoint{3.192518in}{2.214781in}}%
\pgfpathclose%
\pgfusepath{stroke,fill}%
\end{pgfscope}%
\begin{pgfscope}%
\pgfpathrectangle{\pgfqpoint{0.100000in}{0.212622in}}{\pgfqpoint{3.696000in}{3.696000in}}%
\pgfusepath{clip}%
\pgfsetbuttcap%
\pgfsetroundjoin%
\definecolor{currentfill}{rgb}{0.121569,0.466667,0.705882}%
\pgfsetfillcolor{currentfill}%
\pgfsetfillopacity{0.596675}%
\pgfsetlinewidth{1.003750pt}%
\definecolor{currentstroke}{rgb}{0.121569,0.466667,0.705882}%
\pgfsetstrokecolor{currentstroke}%
\pgfsetstrokeopacity{0.596675}%
\pgfsetdash{}{0pt}%
\pgfpathmoveto{\pgfqpoint{0.854219in}{1.391212in}}%
\pgfpathcurveto{\pgfqpoint{0.862455in}{1.391212in}}{\pgfqpoint{0.870355in}{1.394484in}}{\pgfqpoint{0.876179in}{1.400308in}}%
\pgfpathcurveto{\pgfqpoint{0.882003in}{1.406132in}}{\pgfqpoint{0.885275in}{1.414032in}}{\pgfqpoint{0.885275in}{1.422269in}}%
\pgfpathcurveto{\pgfqpoint{0.885275in}{1.430505in}}{\pgfqpoint{0.882003in}{1.438405in}}{\pgfqpoint{0.876179in}{1.444229in}}%
\pgfpathcurveto{\pgfqpoint{0.870355in}{1.450053in}}{\pgfqpoint{0.862455in}{1.453325in}}{\pgfqpoint{0.854219in}{1.453325in}}%
\pgfpathcurveto{\pgfqpoint{0.845982in}{1.453325in}}{\pgfqpoint{0.838082in}{1.450053in}}{\pgfqpoint{0.832258in}{1.444229in}}%
\pgfpathcurveto{\pgfqpoint{0.826434in}{1.438405in}}{\pgfqpoint{0.823162in}{1.430505in}}{\pgfqpoint{0.823162in}{1.422269in}}%
\pgfpathcurveto{\pgfqpoint{0.823162in}{1.414032in}}{\pgfqpoint{0.826434in}{1.406132in}}{\pgfqpoint{0.832258in}{1.400308in}}%
\pgfpathcurveto{\pgfqpoint{0.838082in}{1.394484in}}{\pgfqpoint{0.845982in}{1.391212in}}{\pgfqpoint{0.854219in}{1.391212in}}%
\pgfpathclose%
\pgfusepath{stroke,fill}%
\end{pgfscope}%
\begin{pgfscope}%
\pgfpathrectangle{\pgfqpoint{0.100000in}{0.212622in}}{\pgfqpoint{3.696000in}{3.696000in}}%
\pgfusepath{clip}%
\pgfsetbuttcap%
\pgfsetroundjoin%
\definecolor{currentfill}{rgb}{0.121569,0.466667,0.705882}%
\pgfsetfillcolor{currentfill}%
\pgfsetfillopacity{0.596936}%
\pgfsetlinewidth{1.003750pt}%
\definecolor{currentstroke}{rgb}{0.121569,0.466667,0.705882}%
\pgfsetstrokecolor{currentstroke}%
\pgfsetstrokeopacity{0.596936}%
\pgfsetdash{}{0pt}%
\pgfpathmoveto{\pgfqpoint{3.192069in}{2.214704in}}%
\pgfpathcurveto{\pgfqpoint{3.200306in}{2.214704in}}{\pgfqpoint{3.208206in}{2.217976in}}{\pgfqpoint{3.214030in}{2.223800in}}%
\pgfpathcurveto{\pgfqpoint{3.219854in}{2.229624in}}{\pgfqpoint{3.223126in}{2.237524in}}{\pgfqpoint{3.223126in}{2.245760in}}%
\pgfpathcurveto{\pgfqpoint{3.223126in}{2.253997in}}{\pgfqpoint{3.219854in}{2.261897in}}{\pgfqpoint{3.214030in}{2.267720in}}%
\pgfpathcurveto{\pgfqpoint{3.208206in}{2.273544in}}{\pgfqpoint{3.200306in}{2.276817in}}{\pgfqpoint{3.192069in}{2.276817in}}%
\pgfpathcurveto{\pgfqpoint{3.183833in}{2.276817in}}{\pgfqpoint{3.175933in}{2.273544in}}{\pgfqpoint{3.170109in}{2.267720in}}%
\pgfpathcurveto{\pgfqpoint{3.164285in}{2.261897in}}{\pgfqpoint{3.161013in}{2.253997in}}{\pgfqpoint{3.161013in}{2.245760in}}%
\pgfpathcurveto{\pgfqpoint{3.161013in}{2.237524in}}{\pgfqpoint{3.164285in}{2.229624in}}{\pgfqpoint{3.170109in}{2.223800in}}%
\pgfpathcurveto{\pgfqpoint{3.175933in}{2.217976in}}{\pgfqpoint{3.183833in}{2.214704in}}{\pgfqpoint{3.192069in}{2.214704in}}%
\pgfpathclose%
\pgfusepath{stroke,fill}%
\end{pgfscope}%
\begin{pgfscope}%
\pgfpathrectangle{\pgfqpoint{0.100000in}{0.212622in}}{\pgfqpoint{3.696000in}{3.696000in}}%
\pgfusepath{clip}%
\pgfsetbuttcap%
\pgfsetroundjoin%
\definecolor{currentfill}{rgb}{0.121569,0.466667,0.705882}%
\pgfsetfillcolor{currentfill}%
\pgfsetfillopacity{0.597082}%
\pgfsetlinewidth{1.003750pt}%
\definecolor{currentstroke}{rgb}{0.121569,0.466667,0.705882}%
\pgfsetstrokecolor{currentstroke}%
\pgfsetstrokeopacity{0.597082}%
\pgfsetdash{}{0pt}%
\pgfpathmoveto{\pgfqpoint{0.854825in}{1.387696in}}%
\pgfpathcurveto{\pgfqpoint{0.863061in}{1.387696in}}{\pgfqpoint{0.870961in}{1.390968in}}{\pgfqpoint{0.876785in}{1.396792in}}%
\pgfpathcurveto{\pgfqpoint{0.882609in}{1.402616in}}{\pgfqpoint{0.885881in}{1.410516in}}{\pgfqpoint{0.885881in}{1.418753in}}%
\pgfpathcurveto{\pgfqpoint{0.885881in}{1.426989in}}{\pgfqpoint{0.882609in}{1.434889in}}{\pgfqpoint{0.876785in}{1.440713in}}%
\pgfpathcurveto{\pgfqpoint{0.870961in}{1.446537in}}{\pgfqpoint{0.863061in}{1.449809in}}{\pgfqpoint{0.854825in}{1.449809in}}%
\pgfpathcurveto{\pgfqpoint{0.846589in}{1.449809in}}{\pgfqpoint{0.838689in}{1.446537in}}{\pgfqpoint{0.832865in}{1.440713in}}%
\pgfpathcurveto{\pgfqpoint{0.827041in}{1.434889in}}{\pgfqpoint{0.823768in}{1.426989in}}{\pgfqpoint{0.823768in}{1.418753in}}%
\pgfpathcurveto{\pgfqpoint{0.823768in}{1.410516in}}{\pgfqpoint{0.827041in}{1.402616in}}{\pgfqpoint{0.832865in}{1.396792in}}%
\pgfpathcurveto{\pgfqpoint{0.838689in}{1.390968in}}{\pgfqpoint{0.846589in}{1.387696in}}{\pgfqpoint{0.854825in}{1.387696in}}%
\pgfpathclose%
\pgfusepath{stroke,fill}%
\end{pgfscope}%
\begin{pgfscope}%
\pgfpathrectangle{\pgfqpoint{0.100000in}{0.212622in}}{\pgfqpoint{3.696000in}{3.696000in}}%
\pgfusepath{clip}%
\pgfsetbuttcap%
\pgfsetroundjoin%
\definecolor{currentfill}{rgb}{0.121569,0.466667,0.705882}%
\pgfsetfillcolor{currentfill}%
\pgfsetfillopacity{0.597410}%
\pgfsetlinewidth{1.003750pt}%
\definecolor{currentstroke}{rgb}{0.121569,0.466667,0.705882}%
\pgfsetstrokecolor{currentstroke}%
\pgfsetstrokeopacity{0.597410}%
\pgfsetdash{}{0pt}%
\pgfpathmoveto{\pgfqpoint{3.191290in}{2.214534in}}%
\pgfpathcurveto{\pgfqpoint{3.199526in}{2.214534in}}{\pgfqpoint{3.207426in}{2.217806in}}{\pgfqpoint{3.213250in}{2.223630in}}%
\pgfpathcurveto{\pgfqpoint{3.219074in}{2.229454in}}{\pgfqpoint{3.222346in}{2.237354in}}{\pgfqpoint{3.222346in}{2.245590in}}%
\pgfpathcurveto{\pgfqpoint{3.222346in}{2.253827in}}{\pgfqpoint{3.219074in}{2.261727in}}{\pgfqpoint{3.213250in}{2.267551in}}%
\pgfpathcurveto{\pgfqpoint{3.207426in}{2.273375in}}{\pgfqpoint{3.199526in}{2.276647in}}{\pgfqpoint{3.191290in}{2.276647in}}%
\pgfpathcurveto{\pgfqpoint{3.183054in}{2.276647in}}{\pgfqpoint{3.175154in}{2.273375in}}{\pgfqpoint{3.169330in}{2.267551in}}%
\pgfpathcurveto{\pgfqpoint{3.163506in}{2.261727in}}{\pgfqpoint{3.160233in}{2.253827in}}{\pgfqpoint{3.160233in}{2.245590in}}%
\pgfpathcurveto{\pgfqpoint{3.160233in}{2.237354in}}{\pgfqpoint{3.163506in}{2.229454in}}{\pgfqpoint{3.169330in}{2.223630in}}%
\pgfpathcurveto{\pgfqpoint{3.175154in}{2.217806in}}{\pgfqpoint{3.183054in}{2.214534in}}{\pgfqpoint{3.191290in}{2.214534in}}%
\pgfpathclose%
\pgfusepath{stroke,fill}%
\end{pgfscope}%
\begin{pgfscope}%
\pgfpathrectangle{\pgfqpoint{0.100000in}{0.212622in}}{\pgfqpoint{3.696000in}{3.696000in}}%
\pgfusepath{clip}%
\pgfsetbuttcap%
\pgfsetroundjoin%
\definecolor{currentfill}{rgb}{0.121569,0.466667,0.705882}%
\pgfsetfillcolor{currentfill}%
\pgfsetfillopacity{0.597579}%
\pgfsetlinewidth{1.003750pt}%
\definecolor{currentstroke}{rgb}{0.121569,0.466667,0.705882}%
\pgfsetstrokecolor{currentstroke}%
\pgfsetstrokeopacity{0.597579}%
\pgfsetdash{}{0pt}%
\pgfpathmoveto{\pgfqpoint{0.883794in}{1.326606in}}%
\pgfpathcurveto{\pgfqpoint{0.892030in}{1.326606in}}{\pgfqpoint{0.899930in}{1.329879in}}{\pgfqpoint{0.905754in}{1.335703in}}%
\pgfpathcurveto{\pgfqpoint{0.911578in}{1.341527in}}{\pgfqpoint{0.914850in}{1.349427in}}{\pgfqpoint{0.914850in}{1.357663in}}%
\pgfpathcurveto{\pgfqpoint{0.914850in}{1.365899in}}{\pgfqpoint{0.911578in}{1.373799in}}{\pgfqpoint{0.905754in}{1.379623in}}%
\pgfpathcurveto{\pgfqpoint{0.899930in}{1.385447in}}{\pgfqpoint{0.892030in}{1.388719in}}{\pgfqpoint{0.883794in}{1.388719in}}%
\pgfpathcurveto{\pgfqpoint{0.875557in}{1.388719in}}{\pgfqpoint{0.867657in}{1.385447in}}{\pgfqpoint{0.861833in}{1.379623in}}%
\pgfpathcurveto{\pgfqpoint{0.856009in}{1.373799in}}{\pgfqpoint{0.852737in}{1.365899in}}{\pgfqpoint{0.852737in}{1.357663in}}%
\pgfpathcurveto{\pgfqpoint{0.852737in}{1.349427in}}{\pgfqpoint{0.856009in}{1.341527in}}{\pgfqpoint{0.861833in}{1.335703in}}%
\pgfpathcurveto{\pgfqpoint{0.867657in}{1.329879in}}{\pgfqpoint{0.875557in}{1.326606in}}{\pgfqpoint{0.883794in}{1.326606in}}%
\pgfpathclose%
\pgfusepath{stroke,fill}%
\end{pgfscope}%
\begin{pgfscope}%
\pgfpathrectangle{\pgfqpoint{0.100000in}{0.212622in}}{\pgfqpoint{3.696000in}{3.696000in}}%
\pgfusepath{clip}%
\pgfsetbuttcap%
\pgfsetroundjoin%
\definecolor{currentfill}{rgb}{0.121569,0.466667,0.705882}%
\pgfsetfillcolor{currentfill}%
\pgfsetfillopacity{0.597766}%
\pgfsetlinewidth{1.003750pt}%
\definecolor{currentstroke}{rgb}{0.121569,0.466667,0.705882}%
\pgfsetstrokecolor{currentstroke}%
\pgfsetstrokeopacity{0.597766}%
\pgfsetdash{}{0pt}%
\pgfpathmoveto{\pgfqpoint{0.856250in}{1.381094in}}%
\pgfpathcurveto{\pgfqpoint{0.864486in}{1.381094in}}{\pgfqpoint{0.872386in}{1.384366in}}{\pgfqpoint{0.878210in}{1.390190in}}%
\pgfpathcurveto{\pgfqpoint{0.884034in}{1.396014in}}{\pgfqpoint{0.887306in}{1.403914in}}{\pgfqpoint{0.887306in}{1.412150in}}%
\pgfpathcurveto{\pgfqpoint{0.887306in}{1.420386in}}{\pgfqpoint{0.884034in}{1.428286in}}{\pgfqpoint{0.878210in}{1.434110in}}%
\pgfpathcurveto{\pgfqpoint{0.872386in}{1.439934in}}{\pgfqpoint{0.864486in}{1.443207in}}{\pgfqpoint{0.856250in}{1.443207in}}%
\pgfpathcurveto{\pgfqpoint{0.848013in}{1.443207in}}{\pgfqpoint{0.840113in}{1.439934in}}{\pgfqpoint{0.834289in}{1.434110in}}%
\pgfpathcurveto{\pgfqpoint{0.828465in}{1.428286in}}{\pgfqpoint{0.825193in}{1.420386in}}{\pgfqpoint{0.825193in}{1.412150in}}%
\pgfpathcurveto{\pgfqpoint{0.825193in}{1.403914in}}{\pgfqpoint{0.828465in}{1.396014in}}{\pgfqpoint{0.834289in}{1.390190in}}%
\pgfpathcurveto{\pgfqpoint{0.840113in}{1.384366in}}{\pgfqpoint{0.848013in}{1.381094in}}{\pgfqpoint{0.856250in}{1.381094in}}%
\pgfpathclose%
\pgfusepath{stroke,fill}%
\end{pgfscope}%
\begin{pgfscope}%
\pgfpathrectangle{\pgfqpoint{0.100000in}{0.212622in}}{\pgfqpoint{3.696000in}{3.696000in}}%
\pgfusepath{clip}%
\pgfsetbuttcap%
\pgfsetroundjoin%
\definecolor{currentfill}{rgb}{0.121569,0.466667,0.705882}%
\pgfsetfillcolor{currentfill}%
\pgfsetfillopacity{0.598277}%
\pgfsetlinewidth{1.003750pt}%
\definecolor{currentstroke}{rgb}{0.121569,0.466667,0.705882}%
\pgfsetstrokecolor{currentstroke}%
\pgfsetstrokeopacity{0.598277}%
\pgfsetdash{}{0pt}%
\pgfpathmoveto{\pgfqpoint{3.189655in}{2.213512in}}%
\pgfpathcurveto{\pgfqpoint{3.197891in}{2.213512in}}{\pgfqpoint{3.205791in}{2.216785in}}{\pgfqpoint{3.211615in}{2.222609in}}%
\pgfpathcurveto{\pgfqpoint{3.217439in}{2.228432in}}{\pgfqpoint{3.220711in}{2.236333in}}{\pgfqpoint{3.220711in}{2.244569in}}%
\pgfpathcurveto{\pgfqpoint{3.220711in}{2.252805in}}{\pgfqpoint{3.217439in}{2.260705in}}{\pgfqpoint{3.211615in}{2.266529in}}%
\pgfpathcurveto{\pgfqpoint{3.205791in}{2.272353in}}{\pgfqpoint{3.197891in}{2.275625in}}{\pgfqpoint{3.189655in}{2.275625in}}%
\pgfpathcurveto{\pgfqpoint{3.181418in}{2.275625in}}{\pgfqpoint{3.173518in}{2.272353in}}{\pgfqpoint{3.167695in}{2.266529in}}%
\pgfpathcurveto{\pgfqpoint{3.161871in}{2.260705in}}{\pgfqpoint{3.158598in}{2.252805in}}{\pgfqpoint{3.158598in}{2.244569in}}%
\pgfpathcurveto{\pgfqpoint{3.158598in}{2.236333in}}{\pgfqpoint{3.161871in}{2.228432in}}{\pgfqpoint{3.167695in}{2.222609in}}%
\pgfpathcurveto{\pgfqpoint{3.173518in}{2.216785in}}{\pgfqpoint{3.181418in}{2.213512in}}{\pgfqpoint{3.189655in}{2.213512in}}%
\pgfpathclose%
\pgfusepath{stroke,fill}%
\end{pgfscope}%
\begin{pgfscope}%
\pgfpathrectangle{\pgfqpoint{0.100000in}{0.212622in}}{\pgfqpoint{3.696000in}{3.696000in}}%
\pgfusepath{clip}%
\pgfsetbuttcap%
\pgfsetroundjoin%
\definecolor{currentfill}{rgb}{0.121569,0.466667,0.705882}%
\pgfsetfillcolor{currentfill}%
\pgfsetfillopacity{0.598279}%
\pgfsetlinewidth{1.003750pt}%
\definecolor{currentstroke}{rgb}{0.121569,0.466667,0.705882}%
\pgfsetstrokecolor{currentstroke}%
\pgfsetstrokeopacity{0.598279}%
\pgfsetdash{}{0pt}%
\pgfpathmoveto{\pgfqpoint{0.857549in}{1.375669in}}%
\pgfpathcurveto{\pgfqpoint{0.865785in}{1.375669in}}{\pgfqpoint{0.873685in}{1.378942in}}{\pgfqpoint{0.879509in}{1.384765in}}%
\pgfpathcurveto{\pgfqpoint{0.885333in}{1.390589in}}{\pgfqpoint{0.888605in}{1.398489in}}{\pgfqpoint{0.888605in}{1.406726in}}%
\pgfpathcurveto{\pgfqpoint{0.888605in}{1.414962in}}{\pgfqpoint{0.885333in}{1.422862in}}{\pgfqpoint{0.879509in}{1.428686in}}%
\pgfpathcurveto{\pgfqpoint{0.873685in}{1.434510in}}{\pgfqpoint{0.865785in}{1.437782in}}{\pgfqpoint{0.857549in}{1.437782in}}%
\pgfpathcurveto{\pgfqpoint{0.849312in}{1.437782in}}{\pgfqpoint{0.841412in}{1.434510in}}{\pgfqpoint{0.835589in}{1.428686in}}%
\pgfpathcurveto{\pgfqpoint{0.829765in}{1.422862in}}{\pgfqpoint{0.826492in}{1.414962in}}{\pgfqpoint{0.826492in}{1.406726in}}%
\pgfpathcurveto{\pgfqpoint{0.826492in}{1.398489in}}{\pgfqpoint{0.829765in}{1.390589in}}{\pgfqpoint{0.835589in}{1.384765in}}%
\pgfpathcurveto{\pgfqpoint{0.841412in}{1.378942in}}{\pgfqpoint{0.849312in}{1.375669in}}{\pgfqpoint{0.857549in}{1.375669in}}%
\pgfpathclose%
\pgfusepath{stroke,fill}%
\end{pgfscope}%
\begin{pgfscope}%
\pgfpathrectangle{\pgfqpoint{0.100000in}{0.212622in}}{\pgfqpoint{3.696000in}{3.696000in}}%
\pgfusepath{clip}%
\pgfsetbuttcap%
\pgfsetroundjoin%
\definecolor{currentfill}{rgb}{0.121569,0.466667,0.705882}%
\pgfsetfillcolor{currentfill}%
\pgfsetfillopacity{0.598464}%
\pgfsetlinewidth{1.003750pt}%
\definecolor{currentstroke}{rgb}{0.121569,0.466667,0.705882}%
\pgfsetstrokecolor{currentstroke}%
\pgfsetstrokeopacity{0.598464}%
\pgfsetdash{}{0pt}%
\pgfpathmoveto{\pgfqpoint{0.878525in}{1.332152in}}%
\pgfpathcurveto{\pgfqpoint{0.886762in}{1.332152in}}{\pgfqpoint{0.894662in}{1.335424in}}{\pgfqpoint{0.900486in}{1.341248in}}%
\pgfpathcurveto{\pgfqpoint{0.906310in}{1.347072in}}{\pgfqpoint{0.909582in}{1.354972in}}{\pgfqpoint{0.909582in}{1.363208in}}%
\pgfpathcurveto{\pgfqpoint{0.909582in}{1.371445in}}{\pgfqpoint{0.906310in}{1.379345in}}{\pgfqpoint{0.900486in}{1.385169in}}%
\pgfpathcurveto{\pgfqpoint{0.894662in}{1.390993in}}{\pgfqpoint{0.886762in}{1.394265in}}{\pgfqpoint{0.878525in}{1.394265in}}%
\pgfpathcurveto{\pgfqpoint{0.870289in}{1.394265in}}{\pgfqpoint{0.862389in}{1.390993in}}{\pgfqpoint{0.856565in}{1.385169in}}%
\pgfpathcurveto{\pgfqpoint{0.850741in}{1.379345in}}{\pgfqpoint{0.847469in}{1.371445in}}{\pgfqpoint{0.847469in}{1.363208in}}%
\pgfpathcurveto{\pgfqpoint{0.847469in}{1.354972in}}{\pgfqpoint{0.850741in}{1.347072in}}{\pgfqpoint{0.856565in}{1.341248in}}%
\pgfpathcurveto{\pgfqpoint{0.862389in}{1.335424in}}{\pgfqpoint{0.870289in}{1.332152in}}{\pgfqpoint{0.878525in}{1.332152in}}%
\pgfpathclose%
\pgfusepath{stroke,fill}%
\end{pgfscope}%
\begin{pgfscope}%
\pgfpathrectangle{\pgfqpoint{0.100000in}{0.212622in}}{\pgfqpoint{3.696000in}{3.696000in}}%
\pgfusepath{clip}%
\pgfsetbuttcap%
\pgfsetroundjoin%
\definecolor{currentfill}{rgb}{0.121569,0.466667,0.705882}%
\pgfsetfillcolor{currentfill}%
\pgfsetfillopacity{0.598564}%
\pgfsetlinewidth{1.003750pt}%
\definecolor{currentstroke}{rgb}{0.121569,0.466667,0.705882}%
\pgfsetstrokecolor{currentstroke}%
\pgfsetstrokeopacity{0.598564}%
\pgfsetdash{}{0pt}%
\pgfpathmoveto{\pgfqpoint{0.858488in}{1.372272in}}%
\pgfpathcurveto{\pgfqpoint{0.866724in}{1.372272in}}{\pgfqpoint{0.874624in}{1.375545in}}{\pgfqpoint{0.880448in}{1.381369in}}%
\pgfpathcurveto{\pgfqpoint{0.886272in}{1.387192in}}{\pgfqpoint{0.889544in}{1.395093in}}{\pgfqpoint{0.889544in}{1.403329in}}%
\pgfpathcurveto{\pgfqpoint{0.889544in}{1.411565in}}{\pgfqpoint{0.886272in}{1.419465in}}{\pgfqpoint{0.880448in}{1.425289in}}%
\pgfpathcurveto{\pgfqpoint{0.874624in}{1.431113in}}{\pgfqpoint{0.866724in}{1.434385in}}{\pgfqpoint{0.858488in}{1.434385in}}%
\pgfpathcurveto{\pgfqpoint{0.850251in}{1.434385in}}{\pgfqpoint{0.842351in}{1.431113in}}{\pgfqpoint{0.836527in}{1.425289in}}%
\pgfpathcurveto{\pgfqpoint{0.830703in}{1.419465in}}{\pgfqpoint{0.827431in}{1.411565in}}{\pgfqpoint{0.827431in}{1.403329in}}%
\pgfpathcurveto{\pgfqpoint{0.827431in}{1.395093in}}{\pgfqpoint{0.830703in}{1.387192in}}{\pgfqpoint{0.836527in}{1.381369in}}%
\pgfpathcurveto{\pgfqpoint{0.842351in}{1.375545in}}{\pgfqpoint{0.850251in}{1.372272in}}{\pgfqpoint{0.858488in}{1.372272in}}%
\pgfpathclose%
\pgfusepath{stroke,fill}%
\end{pgfscope}%
\begin{pgfscope}%
\pgfpathrectangle{\pgfqpoint{0.100000in}{0.212622in}}{\pgfqpoint{3.696000in}{3.696000in}}%
\pgfusepath{clip}%
\pgfsetbuttcap%
\pgfsetroundjoin%
\definecolor{currentfill}{rgb}{0.121569,0.466667,0.705882}%
\pgfsetfillcolor{currentfill}%
\pgfsetfillopacity{0.598710}%
\pgfsetlinewidth{1.003750pt}%
\definecolor{currentstroke}{rgb}{0.121569,0.466667,0.705882}%
\pgfsetstrokecolor{currentstroke}%
\pgfsetstrokeopacity{0.598710}%
\pgfsetdash{}{0pt}%
\pgfpathmoveto{\pgfqpoint{0.859079in}{1.370324in}}%
\pgfpathcurveto{\pgfqpoint{0.867316in}{1.370324in}}{\pgfqpoint{0.875216in}{1.373596in}}{\pgfqpoint{0.881040in}{1.379420in}}%
\pgfpathcurveto{\pgfqpoint{0.886863in}{1.385244in}}{\pgfqpoint{0.890136in}{1.393144in}}{\pgfqpoint{0.890136in}{1.401381in}}%
\pgfpathcurveto{\pgfqpoint{0.890136in}{1.409617in}}{\pgfqpoint{0.886863in}{1.417517in}}{\pgfqpoint{0.881040in}{1.423341in}}%
\pgfpathcurveto{\pgfqpoint{0.875216in}{1.429165in}}{\pgfqpoint{0.867316in}{1.432437in}}{\pgfqpoint{0.859079in}{1.432437in}}%
\pgfpathcurveto{\pgfqpoint{0.850843in}{1.432437in}}{\pgfqpoint{0.842943in}{1.429165in}}{\pgfqpoint{0.837119in}{1.423341in}}%
\pgfpathcurveto{\pgfqpoint{0.831295in}{1.417517in}}{\pgfqpoint{0.828023in}{1.409617in}}{\pgfqpoint{0.828023in}{1.401381in}}%
\pgfpathcurveto{\pgfqpoint{0.828023in}{1.393144in}}{\pgfqpoint{0.831295in}{1.385244in}}{\pgfqpoint{0.837119in}{1.379420in}}%
\pgfpathcurveto{\pgfqpoint{0.842943in}{1.373596in}}{\pgfqpoint{0.850843in}{1.370324in}}{\pgfqpoint{0.859079in}{1.370324in}}%
\pgfpathclose%
\pgfusepath{stroke,fill}%
\end{pgfscope}%
\begin{pgfscope}%
\pgfpathrectangle{\pgfqpoint{0.100000in}{0.212622in}}{\pgfqpoint{3.696000in}{3.696000in}}%
\pgfusepath{clip}%
\pgfsetbuttcap%
\pgfsetroundjoin%
\definecolor{currentfill}{rgb}{0.121569,0.466667,0.705882}%
\pgfsetfillcolor{currentfill}%
\pgfsetfillopacity{0.598942}%
\pgfsetlinewidth{1.003750pt}%
\definecolor{currentstroke}{rgb}{0.121569,0.466667,0.705882}%
\pgfsetstrokecolor{currentstroke}%
\pgfsetstrokeopacity{0.598942}%
\pgfsetdash{}{0pt}%
\pgfpathmoveto{\pgfqpoint{0.860300in}{1.366690in}}%
\pgfpathcurveto{\pgfqpoint{0.868536in}{1.366690in}}{\pgfqpoint{0.876436in}{1.369962in}}{\pgfqpoint{0.882260in}{1.375786in}}%
\pgfpathcurveto{\pgfqpoint{0.888084in}{1.381610in}}{\pgfqpoint{0.891356in}{1.389510in}}{\pgfqpoint{0.891356in}{1.397746in}}%
\pgfpathcurveto{\pgfqpoint{0.891356in}{1.405983in}}{\pgfqpoint{0.888084in}{1.413883in}}{\pgfqpoint{0.882260in}{1.419706in}}%
\pgfpathcurveto{\pgfqpoint{0.876436in}{1.425530in}}{\pgfqpoint{0.868536in}{1.428803in}}{\pgfqpoint{0.860300in}{1.428803in}}%
\pgfpathcurveto{\pgfqpoint{0.852063in}{1.428803in}}{\pgfqpoint{0.844163in}{1.425530in}}{\pgfqpoint{0.838339in}{1.419706in}}%
\pgfpathcurveto{\pgfqpoint{0.832515in}{1.413883in}}{\pgfqpoint{0.829243in}{1.405983in}}{\pgfqpoint{0.829243in}{1.397746in}}%
\pgfpathcurveto{\pgfqpoint{0.829243in}{1.389510in}}{\pgfqpoint{0.832515in}{1.381610in}}{\pgfqpoint{0.838339in}{1.375786in}}%
\pgfpathcurveto{\pgfqpoint{0.844163in}{1.369962in}}{\pgfqpoint{0.852063in}{1.366690in}}{\pgfqpoint{0.860300in}{1.366690in}}%
\pgfpathclose%
\pgfusepath{stroke,fill}%
\end{pgfscope}%
\begin{pgfscope}%
\pgfpathrectangle{\pgfqpoint{0.100000in}{0.212622in}}{\pgfqpoint{3.696000in}{3.696000in}}%
\pgfusepath{clip}%
\pgfsetbuttcap%
\pgfsetroundjoin%
\definecolor{currentfill}{rgb}{0.121569,0.466667,0.705882}%
\pgfsetfillcolor{currentfill}%
\pgfsetfillopacity{0.598949}%
\pgfsetlinewidth{1.003750pt}%
\definecolor{currentstroke}{rgb}{0.121569,0.466667,0.705882}%
\pgfsetstrokecolor{currentstroke}%
\pgfsetstrokeopacity{0.598949}%
\pgfsetdash{}{0pt}%
\pgfpathmoveto{\pgfqpoint{0.875628in}{1.335418in}}%
\pgfpathcurveto{\pgfqpoint{0.883865in}{1.335418in}}{\pgfqpoint{0.891765in}{1.338690in}}{\pgfqpoint{0.897589in}{1.344514in}}%
\pgfpathcurveto{\pgfqpoint{0.903413in}{1.350338in}}{\pgfqpoint{0.906685in}{1.358238in}}{\pgfqpoint{0.906685in}{1.366474in}}%
\pgfpathcurveto{\pgfqpoint{0.906685in}{1.374711in}}{\pgfqpoint{0.903413in}{1.382611in}}{\pgfqpoint{0.897589in}{1.388435in}}%
\pgfpathcurveto{\pgfqpoint{0.891765in}{1.394259in}}{\pgfqpoint{0.883865in}{1.397531in}}{\pgfqpoint{0.875628in}{1.397531in}}%
\pgfpathcurveto{\pgfqpoint{0.867392in}{1.397531in}}{\pgfqpoint{0.859492in}{1.394259in}}{\pgfqpoint{0.853668in}{1.388435in}}%
\pgfpathcurveto{\pgfqpoint{0.847844in}{1.382611in}}{\pgfqpoint{0.844572in}{1.374711in}}{\pgfqpoint{0.844572in}{1.366474in}}%
\pgfpathcurveto{\pgfqpoint{0.844572in}{1.358238in}}{\pgfqpoint{0.847844in}{1.350338in}}{\pgfqpoint{0.853668in}{1.344514in}}%
\pgfpathcurveto{\pgfqpoint{0.859492in}{1.338690in}}{\pgfqpoint{0.867392in}{1.335418in}}{\pgfqpoint{0.875628in}{1.335418in}}%
\pgfpathclose%
\pgfusepath{stroke,fill}%
\end{pgfscope}%
\begin{pgfscope}%
\pgfpathrectangle{\pgfqpoint{0.100000in}{0.212622in}}{\pgfqpoint{3.696000in}{3.696000in}}%
\pgfusepath{clip}%
\pgfsetbuttcap%
\pgfsetroundjoin%
\definecolor{currentfill}{rgb}{0.121569,0.466667,0.705882}%
\pgfsetfillcolor{currentfill}%
\pgfsetfillopacity{0.599060}%
\pgfsetlinewidth{1.003750pt}%
\definecolor{currentstroke}{rgb}{0.121569,0.466667,0.705882}%
\pgfsetstrokecolor{currentstroke}%
\pgfsetstrokeopacity{0.599060}%
\pgfsetdash{}{0pt}%
\pgfpathmoveto{\pgfqpoint{0.861065in}{1.364528in}}%
\pgfpathcurveto{\pgfqpoint{0.869301in}{1.364528in}}{\pgfqpoint{0.877201in}{1.367800in}}{\pgfqpoint{0.883025in}{1.373624in}}%
\pgfpathcurveto{\pgfqpoint{0.888849in}{1.379448in}}{\pgfqpoint{0.892122in}{1.387348in}}{\pgfqpoint{0.892122in}{1.395584in}}%
\pgfpathcurveto{\pgfqpoint{0.892122in}{1.403820in}}{\pgfqpoint{0.888849in}{1.411720in}}{\pgfqpoint{0.883025in}{1.417544in}}%
\pgfpathcurveto{\pgfqpoint{0.877201in}{1.423368in}}{\pgfqpoint{0.869301in}{1.426641in}}{\pgfqpoint{0.861065in}{1.426641in}}%
\pgfpathcurveto{\pgfqpoint{0.852829in}{1.426641in}}{\pgfqpoint{0.844929in}{1.423368in}}{\pgfqpoint{0.839105in}{1.417544in}}%
\pgfpathcurveto{\pgfqpoint{0.833281in}{1.411720in}}{\pgfqpoint{0.830009in}{1.403820in}}{\pgfqpoint{0.830009in}{1.395584in}}%
\pgfpathcurveto{\pgfqpoint{0.830009in}{1.387348in}}{\pgfqpoint{0.833281in}{1.379448in}}{\pgfqpoint{0.839105in}{1.373624in}}%
\pgfpathcurveto{\pgfqpoint{0.844929in}{1.367800in}}{\pgfqpoint{0.852829in}{1.364528in}}{\pgfqpoint{0.861065in}{1.364528in}}%
\pgfpathclose%
\pgfusepath{stroke,fill}%
\end{pgfscope}%
\begin{pgfscope}%
\pgfpathrectangle{\pgfqpoint{0.100000in}{0.212622in}}{\pgfqpoint{3.696000in}{3.696000in}}%
\pgfusepath{clip}%
\pgfsetbuttcap%
\pgfsetroundjoin%
\definecolor{currentfill}{rgb}{0.121569,0.466667,0.705882}%
\pgfsetfillcolor{currentfill}%
\pgfsetfillopacity{0.599213}%
\pgfsetlinewidth{1.003750pt}%
\definecolor{currentstroke}{rgb}{0.121569,0.466667,0.705882}%
\pgfsetstrokecolor{currentstroke}%
\pgfsetstrokeopacity{0.599213}%
\pgfsetdash{}{0pt}%
\pgfpathmoveto{\pgfqpoint{0.874039in}{1.337333in}}%
\pgfpathcurveto{\pgfqpoint{0.882275in}{1.337333in}}{\pgfqpoint{0.890176in}{1.340605in}}{\pgfqpoint{0.895999in}{1.346429in}}%
\pgfpathcurveto{\pgfqpoint{0.901823in}{1.352253in}}{\pgfqpoint{0.905096in}{1.360153in}}{\pgfqpoint{0.905096in}{1.368389in}}%
\pgfpathcurveto{\pgfqpoint{0.905096in}{1.376626in}}{\pgfqpoint{0.901823in}{1.384526in}}{\pgfqpoint{0.895999in}{1.390350in}}%
\pgfpathcurveto{\pgfqpoint{0.890176in}{1.396174in}}{\pgfqpoint{0.882275in}{1.399446in}}{\pgfqpoint{0.874039in}{1.399446in}}%
\pgfpathcurveto{\pgfqpoint{0.865803in}{1.399446in}}{\pgfqpoint{0.857903in}{1.396174in}}{\pgfqpoint{0.852079in}{1.390350in}}%
\pgfpathcurveto{\pgfqpoint{0.846255in}{1.384526in}}{\pgfqpoint{0.842983in}{1.376626in}}{\pgfqpoint{0.842983in}{1.368389in}}%
\pgfpathcurveto{\pgfqpoint{0.842983in}{1.360153in}}{\pgfqpoint{0.846255in}{1.352253in}}{\pgfqpoint{0.852079in}{1.346429in}}%
\pgfpathcurveto{\pgfqpoint{0.857903in}{1.340605in}}{\pgfqpoint{0.865803in}{1.337333in}}{\pgfqpoint{0.874039in}{1.337333in}}%
\pgfpathclose%
\pgfusepath{stroke,fill}%
\end{pgfscope}%
\begin{pgfscope}%
\pgfpathrectangle{\pgfqpoint{0.100000in}{0.212622in}}{\pgfqpoint{3.696000in}{3.696000in}}%
\pgfusepath{clip}%
\pgfsetbuttcap%
\pgfsetroundjoin%
\definecolor{currentfill}{rgb}{0.121569,0.466667,0.705882}%
\pgfsetfillcolor{currentfill}%
\pgfsetfillopacity{0.599240}%
\pgfsetlinewidth{1.003750pt}%
\definecolor{currentstroke}{rgb}{0.121569,0.466667,0.705882}%
\pgfsetstrokecolor{currentstroke}%
\pgfsetstrokeopacity{0.599240}%
\pgfsetdash{}{0pt}%
\pgfpathmoveto{\pgfqpoint{0.862582in}{1.360507in}}%
\pgfpathcurveto{\pgfqpoint{0.870819in}{1.360507in}}{\pgfqpoint{0.878719in}{1.363779in}}{\pgfqpoint{0.884543in}{1.369603in}}%
\pgfpathcurveto{\pgfqpoint{0.890366in}{1.375427in}}{\pgfqpoint{0.893639in}{1.383327in}}{\pgfqpoint{0.893639in}{1.391563in}}%
\pgfpathcurveto{\pgfqpoint{0.893639in}{1.399800in}}{\pgfqpoint{0.890366in}{1.407700in}}{\pgfqpoint{0.884543in}{1.413524in}}%
\pgfpathcurveto{\pgfqpoint{0.878719in}{1.419348in}}{\pgfqpoint{0.870819in}{1.422620in}}{\pgfqpoint{0.862582in}{1.422620in}}%
\pgfpathcurveto{\pgfqpoint{0.854346in}{1.422620in}}{\pgfqpoint{0.846446in}{1.419348in}}{\pgfqpoint{0.840622in}{1.413524in}}%
\pgfpathcurveto{\pgfqpoint{0.834798in}{1.407700in}}{\pgfqpoint{0.831526in}{1.399800in}}{\pgfqpoint{0.831526in}{1.391563in}}%
\pgfpathcurveto{\pgfqpoint{0.831526in}{1.383327in}}{\pgfqpoint{0.834798in}{1.375427in}}{\pgfqpoint{0.840622in}{1.369603in}}%
\pgfpathcurveto{\pgfqpoint{0.846446in}{1.363779in}}{\pgfqpoint{0.854346in}{1.360507in}}{\pgfqpoint{0.862582in}{1.360507in}}%
\pgfpathclose%
\pgfusepath{stroke,fill}%
\end{pgfscope}%
\begin{pgfscope}%
\pgfpathrectangle{\pgfqpoint{0.100000in}{0.212622in}}{\pgfqpoint{3.696000in}{3.696000in}}%
\pgfusepath{clip}%
\pgfsetbuttcap%
\pgfsetroundjoin%
\definecolor{currentfill}{rgb}{0.121569,0.466667,0.705882}%
\pgfsetfillcolor{currentfill}%
\pgfsetfillopacity{0.599320}%
\pgfsetlinewidth{1.003750pt}%
\definecolor{currentstroke}{rgb}{0.121569,0.466667,0.705882}%
\pgfsetstrokecolor{currentstroke}%
\pgfsetstrokeopacity{0.599320}%
\pgfsetdash{}{0pt}%
\pgfpathmoveto{\pgfqpoint{3.187861in}{2.212634in}}%
\pgfpathcurveto{\pgfqpoint{3.196097in}{2.212634in}}{\pgfqpoint{3.203997in}{2.215907in}}{\pgfqpoint{3.209821in}{2.221731in}}%
\pgfpathcurveto{\pgfqpoint{3.215645in}{2.227554in}}{\pgfqpoint{3.218917in}{2.235455in}}{\pgfqpoint{3.218917in}{2.243691in}}%
\pgfpathcurveto{\pgfqpoint{3.218917in}{2.251927in}}{\pgfqpoint{3.215645in}{2.259827in}}{\pgfqpoint{3.209821in}{2.265651in}}%
\pgfpathcurveto{\pgfqpoint{3.203997in}{2.271475in}}{\pgfqpoint{3.196097in}{2.274747in}}{\pgfqpoint{3.187861in}{2.274747in}}%
\pgfpathcurveto{\pgfqpoint{3.179625in}{2.274747in}}{\pgfqpoint{3.171725in}{2.271475in}}{\pgfqpoint{3.165901in}{2.265651in}}%
\pgfpathcurveto{\pgfqpoint{3.160077in}{2.259827in}}{\pgfqpoint{3.156804in}{2.251927in}}{\pgfqpoint{3.156804in}{2.243691in}}%
\pgfpathcurveto{\pgfqpoint{3.156804in}{2.235455in}}{\pgfqpoint{3.160077in}{2.227554in}}{\pgfqpoint{3.165901in}{2.221731in}}%
\pgfpathcurveto{\pgfqpoint{3.171725in}{2.215907in}}{\pgfqpoint{3.179625in}{2.212634in}}{\pgfqpoint{3.187861in}{2.212634in}}%
\pgfpathclose%
\pgfusepath{stroke,fill}%
\end{pgfscope}%
\begin{pgfscope}%
\pgfpathrectangle{\pgfqpoint{0.100000in}{0.212622in}}{\pgfqpoint{3.696000in}{3.696000in}}%
\pgfusepath{clip}%
\pgfsetbuttcap%
\pgfsetroundjoin%
\definecolor{currentfill}{rgb}{0.121569,0.466667,0.705882}%
\pgfsetfillcolor{currentfill}%
\pgfsetfillopacity{0.599353}%
\pgfsetlinewidth{1.003750pt}%
\definecolor{currentstroke}{rgb}{0.121569,0.466667,0.705882}%
\pgfsetstrokecolor{currentstroke}%
\pgfsetstrokeopacity{0.599353}%
\pgfsetdash{}{0pt}%
\pgfpathmoveto{\pgfqpoint{0.873170in}{1.338438in}}%
\pgfpathcurveto{\pgfqpoint{0.881406in}{1.338438in}}{\pgfqpoint{0.889306in}{1.341710in}}{\pgfqpoint{0.895130in}{1.347534in}}%
\pgfpathcurveto{\pgfqpoint{0.900954in}{1.353358in}}{\pgfqpoint{0.904227in}{1.361258in}}{\pgfqpoint{0.904227in}{1.369494in}}%
\pgfpathcurveto{\pgfqpoint{0.904227in}{1.377731in}}{\pgfqpoint{0.900954in}{1.385631in}}{\pgfqpoint{0.895130in}{1.391455in}}%
\pgfpathcurveto{\pgfqpoint{0.889306in}{1.397279in}}{\pgfqpoint{0.881406in}{1.400551in}}{\pgfqpoint{0.873170in}{1.400551in}}%
\pgfpathcurveto{\pgfqpoint{0.864934in}{1.400551in}}{\pgfqpoint{0.857034in}{1.397279in}}{\pgfqpoint{0.851210in}{1.391455in}}%
\pgfpathcurveto{\pgfqpoint{0.845386in}{1.385631in}}{\pgfqpoint{0.842114in}{1.377731in}}{\pgfqpoint{0.842114in}{1.369494in}}%
\pgfpathcurveto{\pgfqpoint{0.842114in}{1.361258in}}{\pgfqpoint{0.845386in}{1.353358in}}{\pgfqpoint{0.851210in}{1.347534in}}%
\pgfpathcurveto{\pgfqpoint{0.857034in}{1.341710in}}{\pgfqpoint{0.864934in}{1.338438in}}{\pgfqpoint{0.873170in}{1.338438in}}%
\pgfpathclose%
\pgfusepath{stroke,fill}%
\end{pgfscope}%
\begin{pgfscope}%
\pgfpathrectangle{\pgfqpoint{0.100000in}{0.212622in}}{\pgfqpoint{3.696000in}{3.696000in}}%
\pgfusepath{clip}%
\pgfsetbuttcap%
\pgfsetroundjoin%
\definecolor{currentfill}{rgb}{0.121569,0.466667,0.705882}%
\pgfsetfillcolor{currentfill}%
\pgfsetfillopacity{0.599426}%
\pgfsetlinewidth{1.003750pt}%
\definecolor{currentstroke}{rgb}{0.121569,0.466667,0.705882}%
\pgfsetstrokecolor{currentstroke}%
\pgfsetstrokeopacity{0.599426}%
\pgfsetdash{}{0pt}%
\pgfpathmoveto{\pgfqpoint{0.872697in}{1.339071in}}%
\pgfpathcurveto{\pgfqpoint{0.880933in}{1.339071in}}{\pgfqpoint{0.888833in}{1.342343in}}{\pgfqpoint{0.894657in}{1.348167in}}%
\pgfpathcurveto{\pgfqpoint{0.900481in}{1.353991in}}{\pgfqpoint{0.903753in}{1.361891in}}{\pgfqpoint{0.903753in}{1.370128in}}%
\pgfpathcurveto{\pgfqpoint{0.903753in}{1.378364in}}{\pgfqpoint{0.900481in}{1.386264in}}{\pgfqpoint{0.894657in}{1.392088in}}%
\pgfpathcurveto{\pgfqpoint{0.888833in}{1.397912in}}{\pgfqpoint{0.880933in}{1.401184in}}{\pgfqpoint{0.872697in}{1.401184in}}%
\pgfpathcurveto{\pgfqpoint{0.864460in}{1.401184in}}{\pgfqpoint{0.856560in}{1.397912in}}{\pgfqpoint{0.850736in}{1.392088in}}%
\pgfpathcurveto{\pgfqpoint{0.844912in}{1.386264in}}{\pgfqpoint{0.841640in}{1.378364in}}{\pgfqpoint{0.841640in}{1.370128in}}%
\pgfpathcurveto{\pgfqpoint{0.841640in}{1.361891in}}{\pgfqpoint{0.844912in}{1.353991in}}{\pgfqpoint{0.850736in}{1.348167in}}%
\pgfpathcurveto{\pgfqpoint{0.856560in}{1.342343in}}{\pgfqpoint{0.864460in}{1.339071in}}{\pgfqpoint{0.872697in}{1.339071in}}%
\pgfpathclose%
\pgfusepath{stroke,fill}%
\end{pgfscope}%
\begin{pgfscope}%
\pgfpathrectangle{\pgfqpoint{0.100000in}{0.212622in}}{\pgfqpoint{3.696000in}{3.696000in}}%
\pgfusepath{clip}%
\pgfsetbuttcap%
\pgfsetroundjoin%
\definecolor{currentfill}{rgb}{0.121569,0.466667,0.705882}%
\pgfsetfillcolor{currentfill}%
\pgfsetfillopacity{0.599464}%
\pgfsetlinewidth{1.003750pt}%
\definecolor{currentstroke}{rgb}{0.121569,0.466667,0.705882}%
\pgfsetstrokecolor{currentstroke}%
\pgfsetstrokeopacity{0.599464}%
\pgfsetdash{}{0pt}%
\pgfpathmoveto{\pgfqpoint{0.872439in}{1.339430in}}%
\pgfpathcurveto{\pgfqpoint{0.880675in}{1.339430in}}{\pgfqpoint{0.888576in}{1.342702in}}{\pgfqpoint{0.894399in}{1.348526in}}%
\pgfpathcurveto{\pgfqpoint{0.900223in}{1.354350in}}{\pgfqpoint{0.903496in}{1.362250in}}{\pgfqpoint{0.903496in}{1.370487in}}%
\pgfpathcurveto{\pgfqpoint{0.903496in}{1.378723in}}{\pgfqpoint{0.900223in}{1.386623in}}{\pgfqpoint{0.894399in}{1.392447in}}%
\pgfpathcurveto{\pgfqpoint{0.888576in}{1.398271in}}{\pgfqpoint{0.880675in}{1.401543in}}{\pgfqpoint{0.872439in}{1.401543in}}%
\pgfpathcurveto{\pgfqpoint{0.864203in}{1.401543in}}{\pgfqpoint{0.856303in}{1.398271in}}{\pgfqpoint{0.850479in}{1.392447in}}%
\pgfpathcurveto{\pgfqpoint{0.844655in}{1.386623in}}{\pgfqpoint{0.841383in}{1.378723in}}{\pgfqpoint{0.841383in}{1.370487in}}%
\pgfpathcurveto{\pgfqpoint{0.841383in}{1.362250in}}{\pgfqpoint{0.844655in}{1.354350in}}{\pgfqpoint{0.850479in}{1.348526in}}%
\pgfpathcurveto{\pgfqpoint{0.856303in}{1.342702in}}{\pgfqpoint{0.864203in}{1.339430in}}{\pgfqpoint{0.872439in}{1.339430in}}%
\pgfpathclose%
\pgfusepath{stroke,fill}%
\end{pgfscope}%
\begin{pgfscope}%
\pgfpathrectangle{\pgfqpoint{0.100000in}{0.212622in}}{\pgfqpoint{3.696000in}{3.696000in}}%
\pgfusepath{clip}%
\pgfsetbuttcap%
\pgfsetroundjoin%
\definecolor{currentfill}{rgb}{0.121569,0.466667,0.705882}%
\pgfsetfillcolor{currentfill}%
\pgfsetfillopacity{0.599483}%
\pgfsetlinewidth{1.003750pt}%
\definecolor{currentstroke}{rgb}{0.121569,0.466667,0.705882}%
\pgfsetstrokecolor{currentstroke}%
\pgfsetstrokeopacity{0.599483}%
\pgfsetdash{}{0pt}%
\pgfpathmoveto{\pgfqpoint{0.872300in}{1.339631in}}%
\pgfpathcurveto{\pgfqpoint{0.880536in}{1.339631in}}{\pgfqpoint{0.888436in}{1.342904in}}{\pgfqpoint{0.894260in}{1.348728in}}%
\pgfpathcurveto{\pgfqpoint{0.900084in}{1.354552in}}{\pgfqpoint{0.903356in}{1.362452in}}{\pgfqpoint{0.903356in}{1.370688in}}%
\pgfpathcurveto{\pgfqpoint{0.903356in}{1.378924in}}{\pgfqpoint{0.900084in}{1.386824in}}{\pgfqpoint{0.894260in}{1.392648in}}%
\pgfpathcurveto{\pgfqpoint{0.888436in}{1.398472in}}{\pgfqpoint{0.880536in}{1.401744in}}{\pgfqpoint{0.872300in}{1.401744in}}%
\pgfpathcurveto{\pgfqpoint{0.864064in}{1.401744in}}{\pgfqpoint{0.856164in}{1.398472in}}{\pgfqpoint{0.850340in}{1.392648in}}%
\pgfpathcurveto{\pgfqpoint{0.844516in}{1.386824in}}{\pgfqpoint{0.841243in}{1.378924in}}{\pgfqpoint{0.841243in}{1.370688in}}%
\pgfpathcurveto{\pgfqpoint{0.841243in}{1.362452in}}{\pgfqpoint{0.844516in}{1.354552in}}{\pgfqpoint{0.850340in}{1.348728in}}%
\pgfpathcurveto{\pgfqpoint{0.856164in}{1.342904in}}{\pgfqpoint{0.864064in}{1.339631in}}{\pgfqpoint{0.872300in}{1.339631in}}%
\pgfpathclose%
\pgfusepath{stroke,fill}%
\end{pgfscope}%
\begin{pgfscope}%
\pgfpathrectangle{\pgfqpoint{0.100000in}{0.212622in}}{\pgfqpoint{3.696000in}{3.696000in}}%
\pgfusepath{clip}%
\pgfsetbuttcap%
\pgfsetroundjoin%
\definecolor{currentfill}{rgb}{0.121569,0.466667,0.705882}%
\pgfsetfillcolor{currentfill}%
\pgfsetfillopacity{0.599493}%
\pgfsetlinewidth{1.003750pt}%
\definecolor{currentstroke}{rgb}{0.121569,0.466667,0.705882}%
\pgfsetstrokecolor{currentstroke}%
\pgfsetstrokeopacity{0.599493}%
\pgfsetdash{}{0pt}%
\pgfpathmoveto{\pgfqpoint{0.872225in}{1.339744in}}%
\pgfpathcurveto{\pgfqpoint{0.880461in}{1.339744in}}{\pgfqpoint{0.888361in}{1.343016in}}{\pgfqpoint{0.894185in}{1.348840in}}%
\pgfpathcurveto{\pgfqpoint{0.900009in}{1.354664in}}{\pgfqpoint{0.903281in}{1.362564in}}{\pgfqpoint{0.903281in}{1.370800in}}%
\pgfpathcurveto{\pgfqpoint{0.903281in}{1.379036in}}{\pgfqpoint{0.900009in}{1.386936in}}{\pgfqpoint{0.894185in}{1.392760in}}%
\pgfpathcurveto{\pgfqpoint{0.888361in}{1.398584in}}{\pgfqpoint{0.880461in}{1.401857in}}{\pgfqpoint{0.872225in}{1.401857in}}%
\pgfpathcurveto{\pgfqpoint{0.863988in}{1.401857in}}{\pgfqpoint{0.856088in}{1.398584in}}{\pgfqpoint{0.850264in}{1.392760in}}%
\pgfpathcurveto{\pgfqpoint{0.844441in}{1.386936in}}{\pgfqpoint{0.841168in}{1.379036in}}{\pgfqpoint{0.841168in}{1.370800in}}%
\pgfpathcurveto{\pgfqpoint{0.841168in}{1.362564in}}{\pgfqpoint{0.844441in}{1.354664in}}{\pgfqpoint{0.850264in}{1.348840in}}%
\pgfpathcurveto{\pgfqpoint{0.856088in}{1.343016in}}{\pgfqpoint{0.863988in}{1.339744in}}{\pgfqpoint{0.872225in}{1.339744in}}%
\pgfpathclose%
\pgfusepath{stroke,fill}%
\end{pgfscope}%
\begin{pgfscope}%
\pgfpathrectangle{\pgfqpoint{0.100000in}{0.212622in}}{\pgfqpoint{3.696000in}{3.696000in}}%
\pgfusepath{clip}%
\pgfsetbuttcap%
\pgfsetroundjoin%
\definecolor{currentfill}{rgb}{0.121569,0.466667,0.705882}%
\pgfsetfillcolor{currentfill}%
\pgfsetfillopacity{0.599498}%
\pgfsetlinewidth{1.003750pt}%
\definecolor{currentstroke}{rgb}{0.121569,0.466667,0.705882}%
\pgfsetstrokecolor{currentstroke}%
\pgfsetstrokeopacity{0.599498}%
\pgfsetdash{}{0pt}%
\pgfpathmoveto{\pgfqpoint{0.872184in}{1.339806in}}%
\pgfpathcurveto{\pgfqpoint{0.880421in}{1.339806in}}{\pgfqpoint{0.888321in}{1.343078in}}{\pgfqpoint{0.894145in}{1.348902in}}%
\pgfpathcurveto{\pgfqpoint{0.899968in}{1.354726in}}{\pgfqpoint{0.903241in}{1.362626in}}{\pgfqpoint{0.903241in}{1.370862in}}%
\pgfpathcurveto{\pgfqpoint{0.903241in}{1.379099in}}{\pgfqpoint{0.899968in}{1.386999in}}{\pgfqpoint{0.894145in}{1.392823in}}%
\pgfpathcurveto{\pgfqpoint{0.888321in}{1.398647in}}{\pgfqpoint{0.880421in}{1.401919in}}{\pgfqpoint{0.872184in}{1.401919in}}%
\pgfpathcurveto{\pgfqpoint{0.863948in}{1.401919in}}{\pgfqpoint{0.856048in}{1.398647in}}{\pgfqpoint{0.850224in}{1.392823in}}%
\pgfpathcurveto{\pgfqpoint{0.844400in}{1.386999in}}{\pgfqpoint{0.841128in}{1.379099in}}{\pgfqpoint{0.841128in}{1.370862in}}%
\pgfpathcurveto{\pgfqpoint{0.841128in}{1.362626in}}{\pgfqpoint{0.844400in}{1.354726in}}{\pgfqpoint{0.850224in}{1.348902in}}%
\pgfpathcurveto{\pgfqpoint{0.856048in}{1.343078in}}{\pgfqpoint{0.863948in}{1.339806in}}{\pgfqpoint{0.872184in}{1.339806in}}%
\pgfpathclose%
\pgfusepath{stroke,fill}%
\end{pgfscope}%
\begin{pgfscope}%
\pgfpathrectangle{\pgfqpoint{0.100000in}{0.212622in}}{\pgfqpoint{3.696000in}{3.696000in}}%
\pgfusepath{clip}%
\pgfsetbuttcap%
\pgfsetroundjoin%
\definecolor{currentfill}{rgb}{0.121569,0.466667,0.705882}%
\pgfsetfillcolor{currentfill}%
\pgfsetfillopacity{0.599507}%
\pgfsetlinewidth{1.003750pt}%
\definecolor{currentstroke}{rgb}{0.121569,0.466667,0.705882}%
\pgfsetstrokecolor{currentstroke}%
\pgfsetstrokeopacity{0.599507}%
\pgfsetdash{}{0pt}%
\pgfpathmoveto{\pgfqpoint{0.865527in}{1.353036in}}%
\pgfpathcurveto{\pgfqpoint{0.873763in}{1.353036in}}{\pgfqpoint{0.881663in}{1.356308in}}{\pgfqpoint{0.887487in}{1.362132in}}%
\pgfpathcurveto{\pgfqpoint{0.893311in}{1.367956in}}{\pgfqpoint{0.896583in}{1.375856in}}{\pgfqpoint{0.896583in}{1.384093in}}%
\pgfpathcurveto{\pgfqpoint{0.896583in}{1.392329in}}{\pgfqpoint{0.893311in}{1.400229in}}{\pgfqpoint{0.887487in}{1.406053in}}%
\pgfpathcurveto{\pgfqpoint{0.881663in}{1.411877in}}{\pgfqpoint{0.873763in}{1.415149in}}{\pgfqpoint{0.865527in}{1.415149in}}%
\pgfpathcurveto{\pgfqpoint{0.857291in}{1.415149in}}{\pgfqpoint{0.849391in}{1.411877in}}{\pgfqpoint{0.843567in}{1.406053in}}%
\pgfpathcurveto{\pgfqpoint{0.837743in}{1.400229in}}{\pgfqpoint{0.834470in}{1.392329in}}{\pgfqpoint{0.834470in}{1.384093in}}%
\pgfpathcurveto{\pgfqpoint{0.834470in}{1.375856in}}{\pgfqpoint{0.837743in}{1.367956in}}{\pgfqpoint{0.843567in}{1.362132in}}%
\pgfpathcurveto{\pgfqpoint{0.849391in}{1.356308in}}{\pgfqpoint{0.857291in}{1.353036in}}{\pgfqpoint{0.865527in}{1.353036in}}%
\pgfpathclose%
\pgfusepath{stroke,fill}%
\end{pgfscope}%
\begin{pgfscope}%
\pgfpathrectangle{\pgfqpoint{0.100000in}{0.212622in}}{\pgfqpoint{3.696000in}{3.696000in}}%
\pgfusepath{clip}%
\pgfsetbuttcap%
\pgfsetroundjoin%
\definecolor{currentfill}{rgb}{0.121569,0.466667,0.705882}%
\pgfsetfillcolor{currentfill}%
\pgfsetfillopacity{0.599557}%
\pgfsetlinewidth{1.003750pt}%
\definecolor{currentstroke}{rgb}{0.121569,0.466667,0.705882}%
\pgfsetstrokecolor{currentstroke}%
\pgfsetstrokeopacity{0.599557}%
\pgfsetdash{}{0pt}%
\pgfpathmoveto{\pgfqpoint{0.871618in}{1.340699in}}%
\pgfpathcurveto{\pgfqpoint{0.879854in}{1.340699in}}{\pgfqpoint{0.887754in}{1.343971in}}{\pgfqpoint{0.893578in}{1.349795in}}%
\pgfpathcurveto{\pgfqpoint{0.899402in}{1.355619in}}{\pgfqpoint{0.902674in}{1.363519in}}{\pgfqpoint{0.902674in}{1.371755in}}%
\pgfpathcurveto{\pgfqpoint{0.902674in}{1.379991in}}{\pgfqpoint{0.899402in}{1.387891in}}{\pgfqpoint{0.893578in}{1.393715in}}%
\pgfpathcurveto{\pgfqpoint{0.887754in}{1.399539in}}{\pgfqpoint{0.879854in}{1.402812in}}{\pgfqpoint{0.871618in}{1.402812in}}%
\pgfpathcurveto{\pgfqpoint{0.863381in}{1.402812in}}{\pgfqpoint{0.855481in}{1.399539in}}{\pgfqpoint{0.849657in}{1.393715in}}%
\pgfpathcurveto{\pgfqpoint{0.843833in}{1.387891in}}{\pgfqpoint{0.840561in}{1.379991in}}{\pgfqpoint{0.840561in}{1.371755in}}%
\pgfpathcurveto{\pgfqpoint{0.840561in}{1.363519in}}{\pgfqpoint{0.843833in}{1.355619in}}{\pgfqpoint{0.849657in}{1.349795in}}%
\pgfpathcurveto{\pgfqpoint{0.855481in}{1.343971in}}{\pgfqpoint{0.863381in}{1.340699in}}{\pgfqpoint{0.871618in}{1.340699in}}%
\pgfpathclose%
\pgfusepath{stroke,fill}%
\end{pgfscope}%
\begin{pgfscope}%
\pgfpathrectangle{\pgfqpoint{0.100000in}{0.212622in}}{\pgfqpoint{3.696000in}{3.696000in}}%
\pgfusepath{clip}%
\pgfsetbuttcap%
\pgfsetroundjoin%
\definecolor{currentfill}{rgb}{0.121569,0.466667,0.705882}%
\pgfsetfillcolor{currentfill}%
\pgfsetfillopacity{0.599584}%
\pgfsetlinewidth{1.003750pt}%
\definecolor{currentstroke}{rgb}{0.121569,0.466667,0.705882}%
\pgfsetstrokecolor{currentstroke}%
\pgfsetstrokeopacity{0.599584}%
\pgfsetdash{}{0pt}%
\pgfpathmoveto{\pgfqpoint{0.871314in}{1.341186in}}%
\pgfpathcurveto{\pgfqpoint{0.879551in}{1.341186in}}{\pgfqpoint{0.887451in}{1.344458in}}{\pgfqpoint{0.893275in}{1.350282in}}%
\pgfpathcurveto{\pgfqpoint{0.899098in}{1.356106in}}{\pgfqpoint{0.902371in}{1.364006in}}{\pgfqpoint{0.902371in}{1.372243in}}%
\pgfpathcurveto{\pgfqpoint{0.902371in}{1.380479in}}{\pgfqpoint{0.899098in}{1.388379in}}{\pgfqpoint{0.893275in}{1.394203in}}%
\pgfpathcurveto{\pgfqpoint{0.887451in}{1.400027in}}{\pgfqpoint{0.879551in}{1.403299in}}{\pgfqpoint{0.871314in}{1.403299in}}%
\pgfpathcurveto{\pgfqpoint{0.863078in}{1.403299in}}{\pgfqpoint{0.855178in}{1.400027in}}{\pgfqpoint{0.849354in}{1.394203in}}%
\pgfpathcurveto{\pgfqpoint{0.843530in}{1.388379in}}{\pgfqpoint{0.840258in}{1.380479in}}{\pgfqpoint{0.840258in}{1.372243in}}%
\pgfpathcurveto{\pgfqpoint{0.840258in}{1.364006in}}{\pgfqpoint{0.843530in}{1.356106in}}{\pgfqpoint{0.849354in}{1.350282in}}%
\pgfpathcurveto{\pgfqpoint{0.855178in}{1.344458in}}{\pgfqpoint{0.863078in}{1.341186in}}{\pgfqpoint{0.871314in}{1.341186in}}%
\pgfpathclose%
\pgfusepath{stroke,fill}%
\end{pgfscope}%
\begin{pgfscope}%
\pgfpathrectangle{\pgfqpoint{0.100000in}{0.212622in}}{\pgfqpoint{3.696000in}{3.696000in}}%
\pgfusepath{clip}%
\pgfsetbuttcap%
\pgfsetroundjoin%
\definecolor{currentfill}{rgb}{0.121569,0.466667,0.705882}%
\pgfsetfillcolor{currentfill}%
\pgfsetfillopacity{0.599627}%
\pgfsetlinewidth{1.003750pt}%
\definecolor{currentstroke}{rgb}{0.121569,0.466667,0.705882}%
\pgfsetstrokecolor{currentstroke}%
\pgfsetstrokeopacity{0.599627}%
\pgfsetdash{}{0pt}%
\pgfpathmoveto{\pgfqpoint{0.867264in}{1.348936in}}%
\pgfpathcurveto{\pgfqpoint{0.875500in}{1.348936in}}{\pgfqpoint{0.883400in}{1.352209in}}{\pgfqpoint{0.889224in}{1.358033in}}%
\pgfpathcurveto{\pgfqpoint{0.895048in}{1.363857in}}{\pgfqpoint{0.898320in}{1.371757in}}{\pgfqpoint{0.898320in}{1.379993in}}%
\pgfpathcurveto{\pgfqpoint{0.898320in}{1.388229in}}{\pgfqpoint{0.895048in}{1.396129in}}{\pgfqpoint{0.889224in}{1.401953in}}%
\pgfpathcurveto{\pgfqpoint{0.883400in}{1.407777in}}{\pgfqpoint{0.875500in}{1.411049in}}{\pgfqpoint{0.867264in}{1.411049in}}%
\pgfpathcurveto{\pgfqpoint{0.859028in}{1.411049in}}{\pgfqpoint{0.851127in}{1.407777in}}{\pgfqpoint{0.845304in}{1.401953in}}%
\pgfpathcurveto{\pgfqpoint{0.839480in}{1.396129in}}{\pgfqpoint{0.836207in}{1.388229in}}{\pgfqpoint{0.836207in}{1.379993in}}%
\pgfpathcurveto{\pgfqpoint{0.836207in}{1.371757in}}{\pgfqpoint{0.839480in}{1.363857in}}{\pgfqpoint{0.845304in}{1.358033in}}%
\pgfpathcurveto{\pgfqpoint{0.851127in}{1.352209in}}{\pgfqpoint{0.859028in}{1.348936in}}{\pgfqpoint{0.867264in}{1.348936in}}%
\pgfpathclose%
\pgfusepath{stroke,fill}%
\end{pgfscope}%
\begin{pgfscope}%
\pgfpathrectangle{\pgfqpoint{0.100000in}{0.212622in}}{\pgfqpoint{3.696000in}{3.696000in}}%
\pgfusepath{clip}%
\pgfsetbuttcap%
\pgfsetroundjoin%
\definecolor{currentfill}{rgb}{0.121569,0.466667,0.705882}%
\pgfsetfillcolor{currentfill}%
\pgfsetfillopacity{0.599646}%
\pgfsetlinewidth{1.003750pt}%
\definecolor{currentstroke}{rgb}{0.121569,0.466667,0.705882}%
\pgfsetstrokecolor{currentstroke}%
\pgfsetstrokeopacity{0.599646}%
\pgfsetdash{}{0pt}%
\pgfpathmoveto{\pgfqpoint{0.870477in}{1.342575in}}%
\pgfpathcurveto{\pgfqpoint{0.878714in}{1.342575in}}{\pgfqpoint{0.886614in}{1.345847in}}{\pgfqpoint{0.892438in}{1.351671in}}%
\pgfpathcurveto{\pgfqpoint{0.898262in}{1.357495in}}{\pgfqpoint{0.901534in}{1.365395in}}{\pgfqpoint{0.901534in}{1.373631in}}%
\pgfpathcurveto{\pgfqpoint{0.901534in}{1.381867in}}{\pgfqpoint{0.898262in}{1.389767in}}{\pgfqpoint{0.892438in}{1.395591in}}%
\pgfpathcurveto{\pgfqpoint{0.886614in}{1.401415in}}{\pgfqpoint{0.878714in}{1.404688in}}{\pgfqpoint{0.870477in}{1.404688in}}%
\pgfpathcurveto{\pgfqpoint{0.862241in}{1.404688in}}{\pgfqpoint{0.854341in}{1.401415in}}{\pgfqpoint{0.848517in}{1.395591in}}%
\pgfpathcurveto{\pgfqpoint{0.842693in}{1.389767in}}{\pgfqpoint{0.839421in}{1.381867in}}{\pgfqpoint{0.839421in}{1.373631in}}%
\pgfpathcurveto{\pgfqpoint{0.839421in}{1.365395in}}{\pgfqpoint{0.842693in}{1.357495in}}{\pgfqpoint{0.848517in}{1.351671in}}%
\pgfpathcurveto{\pgfqpoint{0.854341in}{1.345847in}}{\pgfqpoint{0.862241in}{1.342575in}}{\pgfqpoint{0.870477in}{1.342575in}}%
\pgfpathclose%
\pgfusepath{stroke,fill}%
\end{pgfscope}%
\begin{pgfscope}%
\pgfpathrectangle{\pgfqpoint{0.100000in}{0.212622in}}{\pgfqpoint{3.696000in}{3.696000in}}%
\pgfusepath{clip}%
\pgfsetbuttcap%
\pgfsetroundjoin%
\definecolor{currentfill}{rgb}{0.121569,0.466667,0.705882}%
\pgfsetfillcolor{currentfill}%
\pgfsetfillopacity{0.599651}%
\pgfsetlinewidth{1.003750pt}%
\definecolor{currentstroke}{rgb}{0.121569,0.466667,0.705882}%
\pgfsetstrokecolor{currentstroke}%
\pgfsetstrokeopacity{0.599651}%
\pgfsetdash{}{0pt}%
\pgfpathmoveto{\pgfqpoint{0.867740in}{1.347864in}}%
\pgfpathcurveto{\pgfqpoint{0.875977in}{1.347864in}}{\pgfqpoint{0.883877in}{1.351136in}}{\pgfqpoint{0.889701in}{1.356960in}}%
\pgfpathcurveto{\pgfqpoint{0.895525in}{1.362784in}}{\pgfqpoint{0.898797in}{1.370684in}}{\pgfqpoint{0.898797in}{1.378920in}}%
\pgfpathcurveto{\pgfqpoint{0.898797in}{1.387156in}}{\pgfqpoint{0.895525in}{1.395056in}}{\pgfqpoint{0.889701in}{1.400880in}}%
\pgfpathcurveto{\pgfqpoint{0.883877in}{1.406704in}}{\pgfqpoint{0.875977in}{1.409977in}}{\pgfqpoint{0.867740in}{1.409977in}}%
\pgfpathcurveto{\pgfqpoint{0.859504in}{1.409977in}}{\pgfqpoint{0.851604in}{1.406704in}}{\pgfqpoint{0.845780in}{1.400880in}}%
\pgfpathcurveto{\pgfqpoint{0.839956in}{1.395056in}}{\pgfqpoint{0.836684in}{1.387156in}}{\pgfqpoint{0.836684in}{1.378920in}}%
\pgfpathcurveto{\pgfqpoint{0.836684in}{1.370684in}}{\pgfqpoint{0.839956in}{1.362784in}}{\pgfqpoint{0.845780in}{1.356960in}}%
\pgfpathcurveto{\pgfqpoint{0.851604in}{1.351136in}}{\pgfqpoint{0.859504in}{1.347864in}}{\pgfqpoint{0.867740in}{1.347864in}}%
\pgfpathclose%
\pgfusepath{stroke,fill}%
\end{pgfscope}%
\begin{pgfscope}%
\pgfpathrectangle{\pgfqpoint{0.100000in}{0.212622in}}{\pgfqpoint{3.696000in}{3.696000in}}%
\pgfusepath{clip}%
\pgfsetbuttcap%
\pgfsetroundjoin%
\definecolor{currentfill}{rgb}{0.121569,0.466667,0.705882}%
\pgfsetfillcolor{currentfill}%
\pgfsetfillopacity{0.599672}%
\pgfsetlinewidth{1.003750pt}%
\definecolor{currentstroke}{rgb}{0.121569,0.466667,0.705882}%
\pgfsetstrokecolor{currentstroke}%
\pgfsetstrokeopacity{0.599672}%
\pgfsetdash{}{0pt}%
\pgfpathmoveto{\pgfqpoint{0.870033in}{1.343330in}}%
\pgfpathcurveto{\pgfqpoint{0.878269in}{1.343330in}}{\pgfqpoint{0.886169in}{1.346602in}}{\pgfqpoint{0.891993in}{1.352426in}}%
\pgfpathcurveto{\pgfqpoint{0.897817in}{1.358250in}}{\pgfqpoint{0.901090in}{1.366150in}}{\pgfqpoint{0.901090in}{1.374387in}}%
\pgfpathcurveto{\pgfqpoint{0.901090in}{1.382623in}}{\pgfqpoint{0.897817in}{1.390523in}}{\pgfqpoint{0.891993in}{1.396347in}}%
\pgfpathcurveto{\pgfqpoint{0.886169in}{1.402171in}}{\pgfqpoint{0.878269in}{1.405443in}}{\pgfqpoint{0.870033in}{1.405443in}}%
\pgfpathcurveto{\pgfqpoint{0.861797in}{1.405443in}}{\pgfqpoint{0.853897in}{1.402171in}}{\pgfqpoint{0.848073in}{1.396347in}}%
\pgfpathcurveto{\pgfqpoint{0.842249in}{1.390523in}}{\pgfqpoint{0.838977in}{1.382623in}}{\pgfqpoint{0.838977in}{1.374387in}}%
\pgfpathcurveto{\pgfqpoint{0.838977in}{1.366150in}}{\pgfqpoint{0.842249in}{1.358250in}}{\pgfqpoint{0.848073in}{1.352426in}}%
\pgfpathcurveto{\pgfqpoint{0.853897in}{1.346602in}}{\pgfqpoint{0.861797in}{1.343330in}}{\pgfqpoint{0.870033in}{1.343330in}}%
\pgfpathclose%
\pgfusepath{stroke,fill}%
\end{pgfscope}%
\begin{pgfscope}%
\pgfpathrectangle{\pgfqpoint{0.100000in}{0.212622in}}{\pgfqpoint{3.696000in}{3.696000in}}%
\pgfusepath{clip}%
\pgfsetbuttcap%
\pgfsetroundjoin%
\definecolor{currentfill}{rgb}{0.121569,0.466667,0.705882}%
\pgfsetfillcolor{currentfill}%
\pgfsetfillopacity{0.599682}%
\pgfsetlinewidth{1.003750pt}%
\definecolor{currentstroke}{rgb}{0.121569,0.466667,0.705882}%
\pgfsetstrokecolor{currentstroke}%
\pgfsetstrokeopacity{0.599682}%
\pgfsetdash{}{0pt}%
\pgfpathmoveto{\pgfqpoint{0.869798in}{1.343744in}}%
\pgfpathcurveto{\pgfqpoint{0.878034in}{1.343744in}}{\pgfqpoint{0.885934in}{1.347016in}}{\pgfqpoint{0.891758in}{1.352840in}}%
\pgfpathcurveto{\pgfqpoint{0.897582in}{1.358664in}}{\pgfqpoint{0.900855in}{1.366564in}}{\pgfqpoint{0.900855in}{1.374800in}}%
\pgfpathcurveto{\pgfqpoint{0.900855in}{1.383037in}}{\pgfqpoint{0.897582in}{1.390937in}}{\pgfqpoint{0.891758in}{1.396761in}}%
\pgfpathcurveto{\pgfqpoint{0.885934in}{1.402585in}}{\pgfqpoint{0.878034in}{1.405857in}}{\pgfqpoint{0.869798in}{1.405857in}}%
\pgfpathcurveto{\pgfqpoint{0.861562in}{1.405857in}}{\pgfqpoint{0.853662in}{1.402585in}}{\pgfqpoint{0.847838in}{1.396761in}}%
\pgfpathcurveto{\pgfqpoint{0.842014in}{1.390937in}}{\pgfqpoint{0.838742in}{1.383037in}}{\pgfqpoint{0.838742in}{1.374800in}}%
\pgfpathcurveto{\pgfqpoint{0.838742in}{1.366564in}}{\pgfqpoint{0.842014in}{1.358664in}}{\pgfqpoint{0.847838in}{1.352840in}}%
\pgfpathcurveto{\pgfqpoint{0.853662in}{1.347016in}}{\pgfqpoint{0.861562in}{1.343744in}}{\pgfqpoint{0.869798in}{1.343744in}}%
\pgfpathclose%
\pgfusepath{stroke,fill}%
\end{pgfscope}%
\begin{pgfscope}%
\pgfpathrectangle{\pgfqpoint{0.100000in}{0.212622in}}{\pgfqpoint{3.696000in}{3.696000in}}%
\pgfusepath{clip}%
\pgfsetbuttcap%
\pgfsetroundjoin%
\definecolor{currentfill}{rgb}{0.121569,0.466667,0.705882}%
\pgfsetfillcolor{currentfill}%
\pgfsetfillopacity{0.599684}%
\pgfsetlinewidth{1.003750pt}%
\definecolor{currentstroke}{rgb}{0.121569,0.466667,0.705882}%
\pgfsetstrokecolor{currentstroke}%
\pgfsetstrokeopacity{0.599684}%
\pgfsetdash{}{0pt}%
\pgfpathmoveto{\pgfqpoint{0.868663in}{1.345929in}}%
\pgfpathcurveto{\pgfqpoint{0.876899in}{1.345929in}}{\pgfqpoint{0.884799in}{1.349202in}}{\pgfqpoint{0.890623in}{1.355026in}}%
\pgfpathcurveto{\pgfqpoint{0.896447in}{1.360850in}}{\pgfqpoint{0.899720in}{1.368750in}}{\pgfqpoint{0.899720in}{1.376986in}}%
\pgfpathcurveto{\pgfqpoint{0.899720in}{1.385222in}}{\pgfqpoint{0.896447in}{1.393122in}}{\pgfqpoint{0.890623in}{1.398946in}}%
\pgfpathcurveto{\pgfqpoint{0.884799in}{1.404770in}}{\pgfqpoint{0.876899in}{1.408042in}}{\pgfqpoint{0.868663in}{1.408042in}}%
\pgfpathcurveto{\pgfqpoint{0.860427in}{1.408042in}}{\pgfqpoint{0.852527in}{1.404770in}}{\pgfqpoint{0.846703in}{1.398946in}}%
\pgfpathcurveto{\pgfqpoint{0.840879in}{1.393122in}}{\pgfqpoint{0.837607in}{1.385222in}}{\pgfqpoint{0.837607in}{1.376986in}}%
\pgfpathcurveto{\pgfqpoint{0.837607in}{1.368750in}}{\pgfqpoint{0.840879in}{1.360850in}}{\pgfqpoint{0.846703in}{1.355026in}}%
\pgfpathcurveto{\pgfqpoint{0.852527in}{1.349202in}}{\pgfqpoint{0.860427in}{1.345929in}}{\pgfqpoint{0.868663in}{1.345929in}}%
\pgfpathclose%
\pgfusepath{stroke,fill}%
\end{pgfscope}%
\begin{pgfscope}%
\pgfpathrectangle{\pgfqpoint{0.100000in}{0.212622in}}{\pgfqpoint{3.696000in}{3.696000in}}%
\pgfusepath{clip}%
\pgfsetbuttcap%
\pgfsetroundjoin%
\definecolor{currentfill}{rgb}{0.121569,0.466667,0.705882}%
\pgfsetfillcolor{currentfill}%
\pgfsetfillopacity{0.599685}%
\pgfsetlinewidth{1.003750pt}%
\definecolor{currentstroke}{rgb}{0.121569,0.466667,0.705882}%
\pgfsetstrokecolor{currentstroke}%
\pgfsetstrokeopacity{0.599685}%
\pgfsetdash{}{0pt}%
\pgfpathmoveto{\pgfqpoint{0.869675in}{1.343970in}}%
\pgfpathcurveto{\pgfqpoint{0.877911in}{1.343970in}}{\pgfqpoint{0.885811in}{1.347242in}}{\pgfqpoint{0.891635in}{1.353066in}}%
\pgfpathcurveto{\pgfqpoint{0.897459in}{1.358890in}}{\pgfqpoint{0.900732in}{1.366790in}}{\pgfqpoint{0.900732in}{1.375026in}}%
\pgfpathcurveto{\pgfqpoint{0.900732in}{1.383262in}}{\pgfqpoint{0.897459in}{1.391162in}}{\pgfqpoint{0.891635in}{1.396986in}}%
\pgfpathcurveto{\pgfqpoint{0.885811in}{1.402810in}}{\pgfqpoint{0.877911in}{1.406083in}}{\pgfqpoint{0.869675in}{1.406083in}}%
\pgfpathcurveto{\pgfqpoint{0.861439in}{1.406083in}}{\pgfqpoint{0.853539in}{1.402810in}}{\pgfqpoint{0.847715in}{1.396986in}}%
\pgfpathcurveto{\pgfqpoint{0.841891in}{1.391162in}}{\pgfqpoint{0.838619in}{1.383262in}}{\pgfqpoint{0.838619in}{1.375026in}}%
\pgfpathcurveto{\pgfqpoint{0.838619in}{1.366790in}}{\pgfqpoint{0.841891in}{1.358890in}}{\pgfqpoint{0.847715in}{1.353066in}}%
\pgfpathcurveto{\pgfqpoint{0.853539in}{1.347242in}}{\pgfqpoint{0.861439in}{1.343970in}}{\pgfqpoint{0.869675in}{1.343970in}}%
\pgfpathclose%
\pgfusepath{stroke,fill}%
\end{pgfscope}%
\begin{pgfscope}%
\pgfpathrectangle{\pgfqpoint{0.100000in}{0.212622in}}{\pgfqpoint{3.696000in}{3.696000in}}%
\pgfusepath{clip}%
\pgfsetbuttcap%
\pgfsetroundjoin%
\definecolor{currentfill}{rgb}{0.121569,0.466667,0.705882}%
\pgfsetfillcolor{currentfill}%
\pgfsetfillopacity{0.599690}%
\pgfsetlinewidth{1.003750pt}%
\definecolor{currentstroke}{rgb}{0.121569,0.466667,0.705882}%
\pgfsetstrokecolor{currentstroke}%
\pgfsetstrokeopacity{0.599690}%
\pgfsetdash{}{0pt}%
\pgfpathmoveto{\pgfqpoint{0.869009in}{1.345245in}}%
\pgfpathcurveto{\pgfqpoint{0.877245in}{1.345245in}}{\pgfqpoint{0.885145in}{1.348517in}}{\pgfqpoint{0.890969in}{1.354341in}}%
\pgfpathcurveto{\pgfqpoint{0.896793in}{1.360165in}}{\pgfqpoint{0.900066in}{1.368065in}}{\pgfqpoint{0.900066in}{1.376302in}}%
\pgfpathcurveto{\pgfqpoint{0.900066in}{1.384538in}}{\pgfqpoint{0.896793in}{1.392438in}}{\pgfqpoint{0.890969in}{1.398262in}}%
\pgfpathcurveto{\pgfqpoint{0.885145in}{1.404086in}}{\pgfqpoint{0.877245in}{1.407358in}}{\pgfqpoint{0.869009in}{1.407358in}}%
\pgfpathcurveto{\pgfqpoint{0.860773in}{1.407358in}}{\pgfqpoint{0.852873in}{1.404086in}}{\pgfqpoint{0.847049in}{1.398262in}}%
\pgfpathcurveto{\pgfqpoint{0.841225in}{1.392438in}}{\pgfqpoint{0.837953in}{1.384538in}}{\pgfqpoint{0.837953in}{1.376302in}}%
\pgfpathcurveto{\pgfqpoint{0.837953in}{1.368065in}}{\pgfqpoint{0.841225in}{1.360165in}}{\pgfqpoint{0.847049in}{1.354341in}}%
\pgfpathcurveto{\pgfqpoint{0.852873in}{1.348517in}}{\pgfqpoint{0.860773in}{1.345245in}}{\pgfqpoint{0.869009in}{1.345245in}}%
\pgfpathclose%
\pgfusepath{stroke,fill}%
\end{pgfscope}%
\begin{pgfscope}%
\pgfpathrectangle{\pgfqpoint{0.100000in}{0.212622in}}{\pgfqpoint{3.696000in}{3.696000in}}%
\pgfusepath{clip}%
\pgfsetbuttcap%
\pgfsetroundjoin%
\definecolor{currentfill}{rgb}{0.121569,0.466667,0.705882}%
\pgfsetfillcolor{currentfill}%
\pgfsetfillopacity{0.599865}%
\pgfsetlinewidth{1.003750pt}%
\definecolor{currentstroke}{rgb}{0.121569,0.466667,0.705882}%
\pgfsetstrokecolor{currentstroke}%
\pgfsetstrokeopacity{0.599865}%
\pgfsetdash{}{0pt}%
\pgfpathmoveto{\pgfqpoint{3.186774in}{2.212065in}}%
\pgfpathcurveto{\pgfqpoint{3.195010in}{2.212065in}}{\pgfqpoint{3.202910in}{2.215337in}}{\pgfqpoint{3.208734in}{2.221161in}}%
\pgfpathcurveto{\pgfqpoint{3.214558in}{2.226985in}}{\pgfqpoint{3.217830in}{2.234885in}}{\pgfqpoint{3.217830in}{2.243121in}}%
\pgfpathcurveto{\pgfqpoint{3.217830in}{2.251357in}}{\pgfqpoint{3.214558in}{2.259258in}}{\pgfqpoint{3.208734in}{2.265081in}}%
\pgfpathcurveto{\pgfqpoint{3.202910in}{2.270905in}}{\pgfqpoint{3.195010in}{2.274178in}}{\pgfqpoint{3.186774in}{2.274178in}}%
\pgfpathcurveto{\pgfqpoint{3.178537in}{2.274178in}}{\pgfqpoint{3.170637in}{2.270905in}}{\pgfqpoint{3.164813in}{2.265081in}}%
\pgfpathcurveto{\pgfqpoint{3.158989in}{2.259258in}}{\pgfqpoint{3.155717in}{2.251357in}}{\pgfqpoint{3.155717in}{2.243121in}}%
\pgfpathcurveto{\pgfqpoint{3.155717in}{2.234885in}}{\pgfqpoint{3.158989in}{2.226985in}}{\pgfqpoint{3.164813in}{2.221161in}}%
\pgfpathcurveto{\pgfqpoint{3.170637in}{2.215337in}}{\pgfqpoint{3.178537in}{2.212065in}}{\pgfqpoint{3.186774in}{2.212065in}}%
\pgfpathclose%
\pgfusepath{stroke,fill}%
\end{pgfscope}%
\begin{pgfscope}%
\pgfpathrectangle{\pgfqpoint{0.100000in}{0.212622in}}{\pgfqpoint{3.696000in}{3.696000in}}%
\pgfusepath{clip}%
\pgfsetbuttcap%
\pgfsetroundjoin%
\definecolor{currentfill}{rgb}{0.121569,0.466667,0.705882}%
\pgfsetfillcolor{currentfill}%
\pgfsetfillopacity{0.600850}%
\pgfsetlinewidth{1.003750pt}%
\definecolor{currentstroke}{rgb}{0.121569,0.466667,0.705882}%
\pgfsetstrokecolor{currentstroke}%
\pgfsetstrokeopacity{0.600850}%
\pgfsetdash{}{0pt}%
\pgfpathmoveto{\pgfqpoint{3.185001in}{2.211236in}}%
\pgfpathcurveto{\pgfqpoint{3.193237in}{2.211236in}}{\pgfqpoint{3.201137in}{2.214508in}}{\pgfqpoint{3.206961in}{2.220332in}}%
\pgfpathcurveto{\pgfqpoint{3.212785in}{2.226156in}}{\pgfqpoint{3.216058in}{2.234056in}}{\pgfqpoint{3.216058in}{2.242293in}}%
\pgfpathcurveto{\pgfqpoint{3.216058in}{2.250529in}}{\pgfqpoint{3.212785in}{2.258429in}}{\pgfqpoint{3.206961in}{2.264253in}}%
\pgfpathcurveto{\pgfqpoint{3.201137in}{2.270077in}}{\pgfqpoint{3.193237in}{2.273349in}}{\pgfqpoint{3.185001in}{2.273349in}}%
\pgfpathcurveto{\pgfqpoint{3.176765in}{2.273349in}}{\pgfqpoint{3.168865in}{2.270077in}}{\pgfqpoint{3.163041in}{2.264253in}}%
\pgfpathcurveto{\pgfqpoint{3.157217in}{2.258429in}}{\pgfqpoint{3.153945in}{2.250529in}}{\pgfqpoint{3.153945in}{2.242293in}}%
\pgfpathcurveto{\pgfqpoint{3.153945in}{2.234056in}}{\pgfqpoint{3.157217in}{2.226156in}}{\pgfqpoint{3.163041in}{2.220332in}}%
\pgfpathcurveto{\pgfqpoint{3.168865in}{2.214508in}}{\pgfqpoint{3.176765in}{2.211236in}}{\pgfqpoint{3.185001in}{2.211236in}}%
\pgfpathclose%
\pgfusepath{stroke,fill}%
\end{pgfscope}%
\begin{pgfscope}%
\pgfpathrectangle{\pgfqpoint{0.100000in}{0.212622in}}{\pgfqpoint{3.696000in}{3.696000in}}%
\pgfusepath{clip}%
\pgfsetbuttcap%
\pgfsetroundjoin%
\definecolor{currentfill}{rgb}{0.121569,0.466667,0.705882}%
\pgfsetfillcolor{currentfill}%
\pgfsetfillopacity{0.601428}%
\pgfsetlinewidth{1.003750pt}%
\definecolor{currentstroke}{rgb}{0.121569,0.466667,0.705882}%
\pgfsetstrokecolor{currentstroke}%
\pgfsetstrokeopacity{0.601428}%
\pgfsetdash{}{0pt}%
\pgfpathmoveto{\pgfqpoint{3.184011in}{2.211040in}}%
\pgfpathcurveto{\pgfqpoint{3.192248in}{2.211040in}}{\pgfqpoint{3.200148in}{2.214312in}}{\pgfqpoint{3.205972in}{2.220136in}}%
\pgfpathcurveto{\pgfqpoint{3.211796in}{2.225960in}}{\pgfqpoint{3.215068in}{2.233860in}}{\pgfqpoint{3.215068in}{2.242096in}}%
\pgfpathcurveto{\pgfqpoint{3.215068in}{2.250333in}}{\pgfqpoint{3.211796in}{2.258233in}}{\pgfqpoint{3.205972in}{2.264057in}}%
\pgfpathcurveto{\pgfqpoint{3.200148in}{2.269881in}}{\pgfqpoint{3.192248in}{2.273153in}}{\pgfqpoint{3.184011in}{2.273153in}}%
\pgfpathcurveto{\pgfqpoint{3.175775in}{2.273153in}}{\pgfqpoint{3.167875in}{2.269881in}}{\pgfqpoint{3.162051in}{2.264057in}}%
\pgfpathcurveto{\pgfqpoint{3.156227in}{2.258233in}}{\pgfqpoint{3.152955in}{2.250333in}}{\pgfqpoint{3.152955in}{2.242096in}}%
\pgfpathcurveto{\pgfqpoint{3.152955in}{2.233860in}}{\pgfqpoint{3.156227in}{2.225960in}}{\pgfqpoint{3.162051in}{2.220136in}}%
\pgfpathcurveto{\pgfqpoint{3.167875in}{2.214312in}}{\pgfqpoint{3.175775in}{2.211040in}}{\pgfqpoint{3.184011in}{2.211040in}}%
\pgfpathclose%
\pgfusepath{stroke,fill}%
\end{pgfscope}%
\begin{pgfscope}%
\pgfpathrectangle{\pgfqpoint{0.100000in}{0.212622in}}{\pgfqpoint{3.696000in}{3.696000in}}%
\pgfusepath{clip}%
\pgfsetbuttcap%
\pgfsetroundjoin%
\definecolor{currentfill}{rgb}{0.121569,0.466667,0.705882}%
\pgfsetfillcolor{currentfill}%
\pgfsetfillopacity{0.601729}%
\pgfsetlinewidth{1.003750pt}%
\definecolor{currentstroke}{rgb}{0.121569,0.466667,0.705882}%
\pgfsetstrokecolor{currentstroke}%
\pgfsetstrokeopacity{0.601729}%
\pgfsetdash{}{0pt}%
\pgfpathmoveto{\pgfqpoint{3.183432in}{2.210861in}}%
\pgfpathcurveto{\pgfqpoint{3.191668in}{2.210861in}}{\pgfqpoint{3.199568in}{2.214133in}}{\pgfqpoint{3.205392in}{2.219957in}}%
\pgfpathcurveto{\pgfqpoint{3.211216in}{2.225781in}}{\pgfqpoint{3.214488in}{2.233681in}}{\pgfqpoint{3.214488in}{2.241917in}}%
\pgfpathcurveto{\pgfqpoint{3.214488in}{2.250153in}}{\pgfqpoint{3.211216in}{2.258053in}}{\pgfqpoint{3.205392in}{2.263877in}}%
\pgfpathcurveto{\pgfqpoint{3.199568in}{2.269701in}}{\pgfqpoint{3.191668in}{2.272974in}}{\pgfqpoint{3.183432in}{2.272974in}}%
\pgfpathcurveto{\pgfqpoint{3.175195in}{2.272974in}}{\pgfqpoint{3.167295in}{2.269701in}}{\pgfqpoint{3.161471in}{2.263877in}}%
\pgfpathcurveto{\pgfqpoint{3.155647in}{2.258053in}}{\pgfqpoint{3.152375in}{2.250153in}}{\pgfqpoint{3.152375in}{2.241917in}}%
\pgfpathcurveto{\pgfqpoint{3.152375in}{2.233681in}}{\pgfqpoint{3.155647in}{2.225781in}}{\pgfqpoint{3.161471in}{2.219957in}}%
\pgfpathcurveto{\pgfqpoint{3.167295in}{2.214133in}}{\pgfqpoint{3.175195in}{2.210861in}}{\pgfqpoint{3.183432in}{2.210861in}}%
\pgfpathclose%
\pgfusepath{stroke,fill}%
\end{pgfscope}%
\begin{pgfscope}%
\pgfpathrectangle{\pgfqpoint{0.100000in}{0.212622in}}{\pgfqpoint{3.696000in}{3.696000in}}%
\pgfusepath{clip}%
\pgfsetbuttcap%
\pgfsetroundjoin%
\definecolor{currentfill}{rgb}{0.121569,0.466667,0.705882}%
\pgfsetfillcolor{currentfill}%
\pgfsetfillopacity{0.601895}%
\pgfsetlinewidth{1.003750pt}%
\definecolor{currentstroke}{rgb}{0.121569,0.466667,0.705882}%
\pgfsetstrokecolor{currentstroke}%
\pgfsetstrokeopacity{0.601895}%
\pgfsetdash{}{0pt}%
\pgfpathmoveto{\pgfqpoint{3.183119in}{2.210758in}}%
\pgfpathcurveto{\pgfqpoint{3.191355in}{2.210758in}}{\pgfqpoint{3.199255in}{2.214030in}}{\pgfqpoint{3.205079in}{2.219854in}}%
\pgfpathcurveto{\pgfqpoint{3.210903in}{2.225678in}}{\pgfqpoint{3.214175in}{2.233578in}}{\pgfqpoint{3.214175in}{2.241814in}}%
\pgfpathcurveto{\pgfqpoint{3.214175in}{2.250050in}}{\pgfqpoint{3.210903in}{2.257950in}}{\pgfqpoint{3.205079in}{2.263774in}}%
\pgfpathcurveto{\pgfqpoint{3.199255in}{2.269598in}}{\pgfqpoint{3.191355in}{2.272871in}}{\pgfqpoint{3.183119in}{2.272871in}}%
\pgfpathcurveto{\pgfqpoint{3.174882in}{2.272871in}}{\pgfqpoint{3.166982in}{2.269598in}}{\pgfqpoint{3.161158in}{2.263774in}}%
\pgfpathcurveto{\pgfqpoint{3.155334in}{2.257950in}}{\pgfqpoint{3.152062in}{2.250050in}}{\pgfqpoint{3.152062in}{2.241814in}}%
\pgfpathcurveto{\pgfqpoint{3.152062in}{2.233578in}}{\pgfqpoint{3.155334in}{2.225678in}}{\pgfqpoint{3.161158in}{2.219854in}}%
\pgfpathcurveto{\pgfqpoint{3.166982in}{2.214030in}}{\pgfqpoint{3.174882in}{2.210758in}}{\pgfqpoint{3.183119in}{2.210758in}}%
\pgfpathclose%
\pgfusepath{stroke,fill}%
\end{pgfscope}%
\begin{pgfscope}%
\pgfpathrectangle{\pgfqpoint{0.100000in}{0.212622in}}{\pgfqpoint{3.696000in}{3.696000in}}%
\pgfusepath{clip}%
\pgfsetbuttcap%
\pgfsetroundjoin%
\definecolor{currentfill}{rgb}{0.121569,0.466667,0.705882}%
\pgfsetfillcolor{currentfill}%
\pgfsetfillopacity{0.602614}%
\pgfsetlinewidth{1.003750pt}%
\definecolor{currentstroke}{rgb}{0.121569,0.466667,0.705882}%
\pgfsetstrokecolor{currentstroke}%
\pgfsetstrokeopacity{0.602614}%
\pgfsetdash{}{0pt}%
\pgfpathmoveto{\pgfqpoint{3.181693in}{2.210316in}}%
\pgfpathcurveto{\pgfqpoint{3.189930in}{2.210316in}}{\pgfqpoint{3.197830in}{2.213589in}}{\pgfqpoint{3.203654in}{2.219413in}}%
\pgfpathcurveto{\pgfqpoint{3.209478in}{2.225237in}}{\pgfqpoint{3.212750in}{2.233137in}}{\pgfqpoint{3.212750in}{2.241373in}}%
\pgfpathcurveto{\pgfqpoint{3.212750in}{2.249609in}}{\pgfqpoint{3.209478in}{2.257509in}}{\pgfqpoint{3.203654in}{2.263333in}}%
\pgfpathcurveto{\pgfqpoint{3.197830in}{2.269157in}}{\pgfqpoint{3.189930in}{2.272429in}}{\pgfqpoint{3.181693in}{2.272429in}}%
\pgfpathcurveto{\pgfqpoint{3.173457in}{2.272429in}}{\pgfqpoint{3.165557in}{2.269157in}}{\pgfqpoint{3.159733in}{2.263333in}}%
\pgfpathcurveto{\pgfqpoint{3.153909in}{2.257509in}}{\pgfqpoint{3.150637in}{2.249609in}}{\pgfqpoint{3.150637in}{2.241373in}}%
\pgfpathcurveto{\pgfqpoint{3.150637in}{2.233137in}}{\pgfqpoint{3.153909in}{2.225237in}}{\pgfqpoint{3.159733in}{2.219413in}}%
\pgfpathcurveto{\pgfqpoint{3.165557in}{2.213589in}}{\pgfqpoint{3.173457in}{2.210316in}}{\pgfqpoint{3.181693in}{2.210316in}}%
\pgfpathclose%
\pgfusepath{stroke,fill}%
\end{pgfscope}%
\begin{pgfscope}%
\pgfpathrectangle{\pgfqpoint{0.100000in}{0.212622in}}{\pgfqpoint{3.696000in}{3.696000in}}%
\pgfusepath{clip}%
\pgfsetbuttcap%
\pgfsetroundjoin%
\definecolor{currentfill}{rgb}{0.121569,0.466667,0.705882}%
\pgfsetfillcolor{currentfill}%
\pgfsetfillopacity{0.602982}%
\pgfsetlinewidth{1.003750pt}%
\definecolor{currentstroke}{rgb}{0.121569,0.466667,0.705882}%
\pgfsetstrokecolor{currentstroke}%
\pgfsetstrokeopacity{0.602982}%
\pgfsetdash{}{0pt}%
\pgfpathmoveto{\pgfqpoint{3.180901in}{2.209902in}}%
\pgfpathcurveto{\pgfqpoint{3.189137in}{2.209902in}}{\pgfqpoint{3.197037in}{2.213174in}}{\pgfqpoint{3.202861in}{2.218998in}}%
\pgfpathcurveto{\pgfqpoint{3.208685in}{2.224822in}}{\pgfqpoint{3.211957in}{2.232722in}}{\pgfqpoint{3.211957in}{2.240958in}}%
\pgfpathcurveto{\pgfqpoint{3.211957in}{2.249194in}}{\pgfqpoint{3.208685in}{2.257095in}}{\pgfqpoint{3.202861in}{2.262918in}}%
\pgfpathcurveto{\pgfqpoint{3.197037in}{2.268742in}}{\pgfqpoint{3.189137in}{2.272015in}}{\pgfqpoint{3.180901in}{2.272015in}}%
\pgfpathcurveto{\pgfqpoint{3.172665in}{2.272015in}}{\pgfqpoint{3.164765in}{2.268742in}}{\pgfqpoint{3.158941in}{2.262918in}}%
\pgfpathcurveto{\pgfqpoint{3.153117in}{2.257095in}}{\pgfqpoint{3.149844in}{2.249194in}}{\pgfqpoint{3.149844in}{2.240958in}}%
\pgfpathcurveto{\pgfqpoint{3.149844in}{2.232722in}}{\pgfqpoint{3.153117in}{2.224822in}}{\pgfqpoint{3.158941in}{2.218998in}}%
\pgfpathcurveto{\pgfqpoint{3.164765in}{2.213174in}}{\pgfqpoint{3.172665in}{2.209902in}}{\pgfqpoint{3.180901in}{2.209902in}}%
\pgfpathclose%
\pgfusepath{stroke,fill}%
\end{pgfscope}%
\begin{pgfscope}%
\pgfpathrectangle{\pgfqpoint{0.100000in}{0.212622in}}{\pgfqpoint{3.696000in}{3.696000in}}%
\pgfusepath{clip}%
\pgfsetbuttcap%
\pgfsetroundjoin%
\definecolor{currentfill}{rgb}{0.121569,0.466667,0.705882}%
\pgfsetfillcolor{currentfill}%
\pgfsetfillopacity{0.603196}%
\pgfsetlinewidth{1.003750pt}%
\definecolor{currentstroke}{rgb}{0.121569,0.466667,0.705882}%
\pgfsetstrokecolor{currentstroke}%
\pgfsetstrokeopacity{0.603196}%
\pgfsetdash{}{0pt}%
\pgfpathmoveto{\pgfqpoint{3.180487in}{2.209722in}}%
\pgfpathcurveto{\pgfqpoint{3.188723in}{2.209722in}}{\pgfqpoint{3.196623in}{2.212994in}}{\pgfqpoint{3.202447in}{2.218818in}}%
\pgfpathcurveto{\pgfqpoint{3.208271in}{2.224642in}}{\pgfqpoint{3.211544in}{2.232542in}}{\pgfqpoint{3.211544in}{2.240778in}}%
\pgfpathcurveto{\pgfqpoint{3.211544in}{2.249015in}}{\pgfqpoint{3.208271in}{2.256915in}}{\pgfqpoint{3.202447in}{2.262739in}}%
\pgfpathcurveto{\pgfqpoint{3.196623in}{2.268563in}}{\pgfqpoint{3.188723in}{2.271835in}}{\pgfqpoint{3.180487in}{2.271835in}}%
\pgfpathcurveto{\pgfqpoint{3.172251in}{2.271835in}}{\pgfqpoint{3.164351in}{2.268563in}}{\pgfqpoint{3.158527in}{2.262739in}}%
\pgfpathcurveto{\pgfqpoint{3.152703in}{2.256915in}}{\pgfqpoint{3.149431in}{2.249015in}}{\pgfqpoint{3.149431in}{2.240778in}}%
\pgfpathcurveto{\pgfqpoint{3.149431in}{2.232542in}}{\pgfqpoint{3.152703in}{2.224642in}}{\pgfqpoint{3.158527in}{2.218818in}}%
\pgfpathcurveto{\pgfqpoint{3.164351in}{2.212994in}}{\pgfqpoint{3.172251in}{2.209722in}}{\pgfqpoint{3.180487in}{2.209722in}}%
\pgfpathclose%
\pgfusepath{stroke,fill}%
\end{pgfscope}%
\begin{pgfscope}%
\pgfpathrectangle{\pgfqpoint{0.100000in}{0.212622in}}{\pgfqpoint{3.696000in}{3.696000in}}%
\pgfusepath{clip}%
\pgfsetbuttcap%
\pgfsetroundjoin%
\definecolor{currentfill}{rgb}{0.121569,0.466667,0.705882}%
\pgfsetfillcolor{currentfill}%
\pgfsetfillopacity{0.603585}%
\pgfsetlinewidth{1.003750pt}%
\definecolor{currentstroke}{rgb}{0.121569,0.466667,0.705882}%
\pgfsetstrokecolor{currentstroke}%
\pgfsetstrokeopacity{0.603585}%
\pgfsetdash{}{0pt}%
\pgfpathmoveto{\pgfqpoint{3.179785in}{2.209706in}}%
\pgfpathcurveto{\pgfqpoint{3.188021in}{2.209706in}}{\pgfqpoint{3.195921in}{2.212979in}}{\pgfqpoint{3.201745in}{2.218803in}}%
\pgfpathcurveto{\pgfqpoint{3.207569in}{2.224627in}}{\pgfqpoint{3.210841in}{2.232527in}}{\pgfqpoint{3.210841in}{2.240763in}}%
\pgfpathcurveto{\pgfqpoint{3.210841in}{2.248999in}}{\pgfqpoint{3.207569in}{2.256899in}}{\pgfqpoint{3.201745in}{2.262723in}}%
\pgfpathcurveto{\pgfqpoint{3.195921in}{2.268547in}}{\pgfqpoint{3.188021in}{2.271819in}}{\pgfqpoint{3.179785in}{2.271819in}}%
\pgfpathcurveto{\pgfqpoint{3.171548in}{2.271819in}}{\pgfqpoint{3.163648in}{2.268547in}}{\pgfqpoint{3.157824in}{2.262723in}}%
\pgfpathcurveto{\pgfqpoint{3.152000in}{2.256899in}}{\pgfqpoint{3.148728in}{2.248999in}}{\pgfqpoint{3.148728in}{2.240763in}}%
\pgfpathcurveto{\pgfqpoint{3.148728in}{2.232527in}}{\pgfqpoint{3.152000in}{2.224627in}}{\pgfqpoint{3.157824in}{2.218803in}}%
\pgfpathcurveto{\pgfqpoint{3.163648in}{2.212979in}}{\pgfqpoint{3.171548in}{2.209706in}}{\pgfqpoint{3.179785in}{2.209706in}}%
\pgfpathclose%
\pgfusepath{stroke,fill}%
\end{pgfscope}%
\begin{pgfscope}%
\pgfpathrectangle{\pgfqpoint{0.100000in}{0.212622in}}{\pgfqpoint{3.696000in}{3.696000in}}%
\pgfusepath{clip}%
\pgfsetbuttcap%
\pgfsetroundjoin%
\definecolor{currentfill}{rgb}{0.121569,0.466667,0.705882}%
\pgfsetfillcolor{currentfill}%
\pgfsetfillopacity{0.604177}%
\pgfsetlinewidth{1.003750pt}%
\definecolor{currentstroke}{rgb}{0.121569,0.466667,0.705882}%
\pgfsetstrokecolor{currentstroke}%
\pgfsetstrokeopacity{0.604177}%
\pgfsetdash{}{0pt}%
\pgfpathmoveto{\pgfqpoint{3.178671in}{2.209432in}}%
\pgfpathcurveto{\pgfqpoint{3.186907in}{2.209432in}}{\pgfqpoint{3.194807in}{2.212704in}}{\pgfqpoint{3.200631in}{2.218528in}}%
\pgfpathcurveto{\pgfqpoint{3.206455in}{2.224352in}}{\pgfqpoint{3.209728in}{2.232252in}}{\pgfqpoint{3.209728in}{2.240488in}}%
\pgfpathcurveto{\pgfqpoint{3.209728in}{2.248724in}}{\pgfqpoint{3.206455in}{2.256624in}}{\pgfqpoint{3.200631in}{2.262448in}}%
\pgfpathcurveto{\pgfqpoint{3.194807in}{2.268272in}}{\pgfqpoint{3.186907in}{2.271545in}}{\pgfqpoint{3.178671in}{2.271545in}}%
\pgfpathcurveto{\pgfqpoint{3.170435in}{2.271545in}}{\pgfqpoint{3.162535in}{2.268272in}}{\pgfqpoint{3.156711in}{2.262448in}}%
\pgfpathcurveto{\pgfqpoint{3.150887in}{2.256624in}}{\pgfqpoint{3.147615in}{2.248724in}}{\pgfqpoint{3.147615in}{2.240488in}}%
\pgfpathcurveto{\pgfqpoint{3.147615in}{2.232252in}}{\pgfqpoint{3.150887in}{2.224352in}}{\pgfqpoint{3.156711in}{2.218528in}}%
\pgfpathcurveto{\pgfqpoint{3.162535in}{2.212704in}}{\pgfqpoint{3.170435in}{2.209432in}}{\pgfqpoint{3.178671in}{2.209432in}}%
\pgfpathclose%
\pgfusepath{stroke,fill}%
\end{pgfscope}%
\begin{pgfscope}%
\pgfpathrectangle{\pgfqpoint{0.100000in}{0.212622in}}{\pgfqpoint{3.696000in}{3.696000in}}%
\pgfusepath{clip}%
\pgfsetbuttcap%
\pgfsetroundjoin%
\definecolor{currentfill}{rgb}{0.121569,0.466667,0.705882}%
\pgfsetfillcolor{currentfill}%
\pgfsetfillopacity{0.604923}%
\pgfsetlinewidth{1.003750pt}%
\definecolor{currentstroke}{rgb}{0.121569,0.466667,0.705882}%
\pgfsetstrokecolor{currentstroke}%
\pgfsetstrokeopacity{0.604923}%
\pgfsetdash{}{0pt}%
\pgfpathmoveto{\pgfqpoint{3.177346in}{2.209020in}}%
\pgfpathcurveto{\pgfqpoint{3.185582in}{2.209020in}}{\pgfqpoint{3.193482in}{2.212292in}}{\pgfqpoint{3.199306in}{2.218116in}}%
\pgfpathcurveto{\pgfqpoint{3.205130in}{2.223940in}}{\pgfqpoint{3.208403in}{2.231840in}}{\pgfqpoint{3.208403in}{2.240076in}}%
\pgfpathcurveto{\pgfqpoint{3.208403in}{2.248313in}}{\pgfqpoint{3.205130in}{2.256213in}}{\pgfqpoint{3.199306in}{2.262037in}}%
\pgfpathcurveto{\pgfqpoint{3.193482in}{2.267861in}}{\pgfqpoint{3.185582in}{2.271133in}}{\pgfqpoint{3.177346in}{2.271133in}}%
\pgfpathcurveto{\pgfqpoint{3.169110in}{2.271133in}}{\pgfqpoint{3.161210in}{2.267861in}}{\pgfqpoint{3.155386in}{2.262037in}}%
\pgfpathcurveto{\pgfqpoint{3.149562in}{2.256213in}}{\pgfqpoint{3.146290in}{2.248313in}}{\pgfqpoint{3.146290in}{2.240076in}}%
\pgfpathcurveto{\pgfqpoint{3.146290in}{2.231840in}}{\pgfqpoint{3.149562in}{2.223940in}}{\pgfqpoint{3.155386in}{2.218116in}}%
\pgfpathcurveto{\pgfqpoint{3.161210in}{2.212292in}}{\pgfqpoint{3.169110in}{2.209020in}}{\pgfqpoint{3.177346in}{2.209020in}}%
\pgfpathclose%
\pgfusepath{stroke,fill}%
\end{pgfscope}%
\begin{pgfscope}%
\pgfpathrectangle{\pgfqpoint{0.100000in}{0.212622in}}{\pgfqpoint{3.696000in}{3.696000in}}%
\pgfusepath{clip}%
\pgfsetbuttcap%
\pgfsetroundjoin%
\definecolor{currentfill}{rgb}{0.121569,0.466667,0.705882}%
\pgfsetfillcolor{currentfill}%
\pgfsetfillopacity{0.606156}%
\pgfsetlinewidth{1.003750pt}%
\definecolor{currentstroke}{rgb}{0.121569,0.466667,0.705882}%
\pgfsetstrokecolor{currentstroke}%
\pgfsetstrokeopacity{0.606156}%
\pgfsetdash{}{0pt}%
\pgfpathmoveto{\pgfqpoint{3.175362in}{2.209012in}}%
\pgfpathcurveto{\pgfqpoint{3.183598in}{2.209012in}}{\pgfqpoint{3.191498in}{2.212285in}}{\pgfqpoint{3.197322in}{2.218109in}}%
\pgfpathcurveto{\pgfqpoint{3.203146in}{2.223933in}}{\pgfqpoint{3.206418in}{2.231833in}}{\pgfqpoint{3.206418in}{2.240069in}}%
\pgfpathcurveto{\pgfqpoint{3.206418in}{2.248305in}}{\pgfqpoint{3.203146in}{2.256205in}}{\pgfqpoint{3.197322in}{2.262029in}}%
\pgfpathcurveto{\pgfqpoint{3.191498in}{2.267853in}}{\pgfqpoint{3.183598in}{2.271125in}}{\pgfqpoint{3.175362in}{2.271125in}}%
\pgfpathcurveto{\pgfqpoint{3.167125in}{2.271125in}}{\pgfqpoint{3.159225in}{2.267853in}}{\pgfqpoint{3.153401in}{2.262029in}}%
\pgfpathcurveto{\pgfqpoint{3.147577in}{2.256205in}}{\pgfqpoint{3.144305in}{2.248305in}}{\pgfqpoint{3.144305in}{2.240069in}}%
\pgfpathcurveto{\pgfqpoint{3.144305in}{2.231833in}}{\pgfqpoint{3.147577in}{2.223933in}}{\pgfqpoint{3.153401in}{2.218109in}}%
\pgfpathcurveto{\pgfqpoint{3.159225in}{2.212285in}}{\pgfqpoint{3.167125in}{2.209012in}}{\pgfqpoint{3.175362in}{2.209012in}}%
\pgfpathclose%
\pgfusepath{stroke,fill}%
\end{pgfscope}%
\begin{pgfscope}%
\pgfpathrectangle{\pgfqpoint{0.100000in}{0.212622in}}{\pgfqpoint{3.696000in}{3.696000in}}%
\pgfusepath{clip}%
\pgfsetbuttcap%
\pgfsetroundjoin%
\definecolor{currentfill}{rgb}{0.121569,0.466667,0.705882}%
\pgfsetfillcolor{currentfill}%
\pgfsetfillopacity{0.606792}%
\pgfsetlinewidth{1.003750pt}%
\definecolor{currentstroke}{rgb}{0.121569,0.466667,0.705882}%
\pgfsetstrokecolor{currentstroke}%
\pgfsetstrokeopacity{0.606792}%
\pgfsetdash{}{0pt}%
\pgfpathmoveto{\pgfqpoint{3.174180in}{2.208824in}}%
\pgfpathcurveto{\pgfqpoint{3.182416in}{2.208824in}}{\pgfqpoint{3.190316in}{2.212096in}}{\pgfqpoint{3.196140in}{2.217920in}}%
\pgfpathcurveto{\pgfqpoint{3.201964in}{2.223744in}}{\pgfqpoint{3.205237in}{2.231644in}}{\pgfqpoint{3.205237in}{2.239880in}}%
\pgfpathcurveto{\pgfqpoint{3.205237in}{2.248117in}}{\pgfqpoint{3.201964in}{2.256017in}}{\pgfqpoint{3.196140in}{2.261841in}}%
\pgfpathcurveto{\pgfqpoint{3.190316in}{2.267664in}}{\pgfqpoint{3.182416in}{2.270937in}}{\pgfqpoint{3.174180in}{2.270937in}}%
\pgfpathcurveto{\pgfqpoint{3.165944in}{2.270937in}}{\pgfqpoint{3.158044in}{2.267664in}}{\pgfqpoint{3.152220in}{2.261841in}}%
\pgfpathcurveto{\pgfqpoint{3.146396in}{2.256017in}}{\pgfqpoint{3.143124in}{2.248117in}}{\pgfqpoint{3.143124in}{2.239880in}}%
\pgfpathcurveto{\pgfqpoint{3.143124in}{2.231644in}}{\pgfqpoint{3.146396in}{2.223744in}}{\pgfqpoint{3.152220in}{2.217920in}}%
\pgfpathcurveto{\pgfqpoint{3.158044in}{2.212096in}}{\pgfqpoint{3.165944in}{2.208824in}}{\pgfqpoint{3.174180in}{2.208824in}}%
\pgfpathclose%
\pgfusepath{stroke,fill}%
\end{pgfscope}%
\begin{pgfscope}%
\pgfpathrectangle{\pgfqpoint{0.100000in}{0.212622in}}{\pgfqpoint{3.696000in}{3.696000in}}%
\pgfusepath{clip}%
\pgfsetbuttcap%
\pgfsetroundjoin%
\definecolor{currentfill}{rgb}{0.121569,0.466667,0.705882}%
\pgfsetfillcolor{currentfill}%
\pgfsetfillopacity{0.607581}%
\pgfsetlinewidth{1.003750pt}%
\definecolor{currentstroke}{rgb}{0.121569,0.466667,0.705882}%
\pgfsetstrokecolor{currentstroke}%
\pgfsetstrokeopacity{0.607581}%
\pgfsetdash{}{0pt}%
\pgfpathmoveto{\pgfqpoint{3.172755in}{2.208562in}}%
\pgfpathcurveto{\pgfqpoint{3.180991in}{2.208562in}}{\pgfqpoint{3.188891in}{2.211834in}}{\pgfqpoint{3.194715in}{2.217658in}}%
\pgfpathcurveto{\pgfqpoint{3.200539in}{2.223482in}}{\pgfqpoint{3.203811in}{2.231382in}}{\pgfqpoint{3.203811in}{2.239618in}}%
\pgfpathcurveto{\pgfqpoint{3.203811in}{2.247855in}}{\pgfqpoint{3.200539in}{2.255755in}}{\pgfqpoint{3.194715in}{2.261578in}}%
\pgfpathcurveto{\pgfqpoint{3.188891in}{2.267402in}}{\pgfqpoint{3.180991in}{2.270675in}}{\pgfqpoint{3.172755in}{2.270675in}}%
\pgfpathcurveto{\pgfqpoint{3.164518in}{2.270675in}}{\pgfqpoint{3.156618in}{2.267402in}}{\pgfqpoint{3.150794in}{2.261578in}}%
\pgfpathcurveto{\pgfqpoint{3.144970in}{2.255755in}}{\pgfqpoint{3.141698in}{2.247855in}}{\pgfqpoint{3.141698in}{2.239618in}}%
\pgfpathcurveto{\pgfqpoint{3.141698in}{2.231382in}}{\pgfqpoint{3.144970in}{2.223482in}}{\pgfqpoint{3.150794in}{2.217658in}}%
\pgfpathcurveto{\pgfqpoint{3.156618in}{2.211834in}}{\pgfqpoint{3.164518in}{2.208562in}}{\pgfqpoint{3.172755in}{2.208562in}}%
\pgfpathclose%
\pgfusepath{stroke,fill}%
\end{pgfscope}%
\begin{pgfscope}%
\pgfpathrectangle{\pgfqpoint{0.100000in}{0.212622in}}{\pgfqpoint{3.696000in}{3.696000in}}%
\pgfusepath{clip}%
\pgfsetbuttcap%
\pgfsetroundjoin%
\definecolor{currentfill}{rgb}{0.121569,0.466667,0.705882}%
\pgfsetfillcolor{currentfill}%
\pgfsetfillopacity{0.609107}%
\pgfsetlinewidth{1.003750pt}%
\definecolor{currentstroke}{rgb}{0.121569,0.466667,0.705882}%
\pgfsetstrokecolor{currentstroke}%
\pgfsetstrokeopacity{0.609107}%
\pgfsetdash{}{0pt}%
\pgfpathmoveto{\pgfqpoint{3.170213in}{2.208918in}}%
\pgfpathcurveto{\pgfqpoint{3.178449in}{2.208918in}}{\pgfqpoint{3.186349in}{2.212190in}}{\pgfqpoint{3.192173in}{2.218014in}}%
\pgfpathcurveto{\pgfqpoint{3.197997in}{2.223838in}}{\pgfqpoint{3.201270in}{2.231738in}}{\pgfqpoint{3.201270in}{2.239974in}}%
\pgfpathcurveto{\pgfqpoint{3.201270in}{2.248211in}}{\pgfqpoint{3.197997in}{2.256111in}}{\pgfqpoint{3.192173in}{2.261935in}}%
\pgfpathcurveto{\pgfqpoint{3.186349in}{2.267758in}}{\pgfqpoint{3.178449in}{2.271031in}}{\pgfqpoint{3.170213in}{2.271031in}}%
\pgfpathcurveto{\pgfqpoint{3.161977in}{2.271031in}}{\pgfqpoint{3.154077in}{2.267758in}}{\pgfqpoint{3.148253in}{2.261935in}}%
\pgfpathcurveto{\pgfqpoint{3.142429in}{2.256111in}}{\pgfqpoint{3.139157in}{2.248211in}}{\pgfqpoint{3.139157in}{2.239974in}}%
\pgfpathcurveto{\pgfqpoint{3.139157in}{2.231738in}}{\pgfqpoint{3.142429in}{2.223838in}}{\pgfqpoint{3.148253in}{2.218014in}}%
\pgfpathcurveto{\pgfqpoint{3.154077in}{2.212190in}}{\pgfqpoint{3.161977in}{2.208918in}}{\pgfqpoint{3.170213in}{2.208918in}}%
\pgfpathclose%
\pgfusepath{stroke,fill}%
\end{pgfscope}%
\begin{pgfscope}%
\pgfpathrectangle{\pgfqpoint{0.100000in}{0.212622in}}{\pgfqpoint{3.696000in}{3.696000in}}%
\pgfusepath{clip}%
\pgfsetbuttcap%
\pgfsetroundjoin%
\definecolor{currentfill}{rgb}{0.121569,0.466667,0.705882}%
\pgfsetfillcolor{currentfill}%
\pgfsetfillopacity{0.609892}%
\pgfsetlinewidth{1.003750pt}%
\definecolor{currentstroke}{rgb}{0.121569,0.466667,0.705882}%
\pgfsetstrokecolor{currentstroke}%
\pgfsetstrokeopacity{0.609892}%
\pgfsetdash{}{0pt}%
\pgfpathmoveto{\pgfqpoint{3.168699in}{2.208887in}}%
\pgfpathcurveto{\pgfqpoint{3.176935in}{2.208887in}}{\pgfqpoint{3.184835in}{2.212159in}}{\pgfqpoint{3.190659in}{2.217983in}}%
\pgfpathcurveto{\pgfqpoint{3.196483in}{2.223807in}}{\pgfqpoint{3.199756in}{2.231707in}}{\pgfqpoint{3.199756in}{2.239943in}}%
\pgfpathcurveto{\pgfqpoint{3.199756in}{2.248180in}}{\pgfqpoint{3.196483in}{2.256080in}}{\pgfqpoint{3.190659in}{2.261904in}}%
\pgfpathcurveto{\pgfqpoint{3.184835in}{2.267728in}}{\pgfqpoint{3.176935in}{2.271000in}}{\pgfqpoint{3.168699in}{2.271000in}}%
\pgfpathcurveto{\pgfqpoint{3.160463in}{2.271000in}}{\pgfqpoint{3.152563in}{2.267728in}}{\pgfqpoint{3.146739in}{2.261904in}}%
\pgfpathcurveto{\pgfqpoint{3.140915in}{2.256080in}}{\pgfqpoint{3.137643in}{2.248180in}}{\pgfqpoint{3.137643in}{2.239943in}}%
\pgfpathcurveto{\pgfqpoint{3.137643in}{2.231707in}}{\pgfqpoint{3.140915in}{2.223807in}}{\pgfqpoint{3.146739in}{2.217983in}}%
\pgfpathcurveto{\pgfqpoint{3.152563in}{2.212159in}}{\pgfqpoint{3.160463in}{2.208887in}}{\pgfqpoint{3.168699in}{2.208887in}}%
\pgfpathclose%
\pgfusepath{stroke,fill}%
\end{pgfscope}%
\begin{pgfscope}%
\pgfpathrectangle{\pgfqpoint{0.100000in}{0.212622in}}{\pgfqpoint{3.696000in}{3.696000in}}%
\pgfusepath{clip}%
\pgfsetbuttcap%
\pgfsetroundjoin%
\definecolor{currentfill}{rgb}{0.121569,0.466667,0.705882}%
\pgfsetfillcolor{currentfill}%
\pgfsetfillopacity{0.610323}%
\pgfsetlinewidth{1.003750pt}%
\definecolor{currentstroke}{rgb}{0.121569,0.466667,0.705882}%
\pgfsetstrokecolor{currentstroke}%
\pgfsetstrokeopacity{0.610323}%
\pgfsetdash{}{0pt}%
\pgfpathmoveto{\pgfqpoint{3.167877in}{2.208850in}}%
\pgfpathcurveto{\pgfqpoint{3.176113in}{2.208850in}}{\pgfqpoint{3.184013in}{2.212122in}}{\pgfqpoint{3.189837in}{2.217946in}}%
\pgfpathcurveto{\pgfqpoint{3.195661in}{2.223770in}}{\pgfqpoint{3.198934in}{2.231670in}}{\pgfqpoint{3.198934in}{2.239906in}}%
\pgfpathcurveto{\pgfqpoint{3.198934in}{2.248143in}}{\pgfqpoint{3.195661in}{2.256043in}}{\pgfqpoint{3.189837in}{2.261867in}}%
\pgfpathcurveto{\pgfqpoint{3.184013in}{2.267690in}}{\pgfqpoint{3.176113in}{2.270963in}}{\pgfqpoint{3.167877in}{2.270963in}}%
\pgfpathcurveto{\pgfqpoint{3.159641in}{2.270963in}}{\pgfqpoint{3.151741in}{2.267690in}}{\pgfqpoint{3.145917in}{2.261867in}}%
\pgfpathcurveto{\pgfqpoint{3.140093in}{2.256043in}}{\pgfqpoint{3.136821in}{2.248143in}}{\pgfqpoint{3.136821in}{2.239906in}}%
\pgfpathcurveto{\pgfqpoint{3.136821in}{2.231670in}}{\pgfqpoint{3.140093in}{2.223770in}}{\pgfqpoint{3.145917in}{2.217946in}}%
\pgfpathcurveto{\pgfqpoint{3.151741in}{2.212122in}}{\pgfqpoint{3.159641in}{2.208850in}}{\pgfqpoint{3.167877in}{2.208850in}}%
\pgfpathclose%
\pgfusepath{stroke,fill}%
\end{pgfscope}%
\begin{pgfscope}%
\pgfpathrectangle{\pgfqpoint{0.100000in}{0.212622in}}{\pgfqpoint{3.696000in}{3.696000in}}%
\pgfusepath{clip}%
\pgfsetbuttcap%
\pgfsetroundjoin%
\definecolor{currentfill}{rgb}{0.121569,0.466667,0.705882}%
\pgfsetfillcolor{currentfill}%
\pgfsetfillopacity{0.611448}%
\pgfsetlinewidth{1.003750pt}%
\definecolor{currentstroke}{rgb}{0.121569,0.466667,0.705882}%
\pgfsetstrokecolor{currentstroke}%
\pgfsetstrokeopacity{0.611448}%
\pgfsetdash{}{0pt}%
\pgfpathmoveto{\pgfqpoint{3.165512in}{2.208625in}}%
\pgfpathcurveto{\pgfqpoint{3.173749in}{2.208625in}}{\pgfqpoint{3.181649in}{2.211897in}}{\pgfqpoint{3.187473in}{2.217721in}}%
\pgfpathcurveto{\pgfqpoint{3.193297in}{2.223545in}}{\pgfqpoint{3.196569in}{2.231445in}}{\pgfqpoint{3.196569in}{2.239681in}}%
\pgfpathcurveto{\pgfqpoint{3.196569in}{2.247918in}}{\pgfqpoint{3.193297in}{2.255818in}}{\pgfqpoint{3.187473in}{2.261642in}}%
\pgfpathcurveto{\pgfqpoint{3.181649in}{2.267466in}}{\pgfqpoint{3.173749in}{2.270738in}}{\pgfqpoint{3.165512in}{2.270738in}}%
\pgfpathcurveto{\pgfqpoint{3.157276in}{2.270738in}}{\pgfqpoint{3.149376in}{2.267466in}}{\pgfqpoint{3.143552in}{2.261642in}}%
\pgfpathcurveto{\pgfqpoint{3.137728in}{2.255818in}}{\pgfqpoint{3.134456in}{2.247918in}}{\pgfqpoint{3.134456in}{2.239681in}}%
\pgfpathcurveto{\pgfqpoint{3.134456in}{2.231445in}}{\pgfqpoint{3.137728in}{2.223545in}}{\pgfqpoint{3.143552in}{2.217721in}}%
\pgfpathcurveto{\pgfqpoint{3.149376in}{2.211897in}}{\pgfqpoint{3.157276in}{2.208625in}}{\pgfqpoint{3.165512in}{2.208625in}}%
\pgfpathclose%
\pgfusepath{stroke,fill}%
\end{pgfscope}%
\begin{pgfscope}%
\pgfpathrectangle{\pgfqpoint{0.100000in}{0.212622in}}{\pgfqpoint{3.696000in}{3.696000in}}%
\pgfusepath{clip}%
\pgfsetbuttcap%
\pgfsetroundjoin%
\definecolor{currentfill}{rgb}{0.121569,0.466667,0.705882}%
\pgfsetfillcolor{currentfill}%
\pgfsetfillopacity{0.612717}%
\pgfsetlinewidth{1.003750pt}%
\definecolor{currentstroke}{rgb}{0.121569,0.466667,0.705882}%
\pgfsetstrokecolor{currentstroke}%
\pgfsetstrokeopacity{0.612717}%
\pgfsetdash{}{0pt}%
\pgfpathmoveto{\pgfqpoint{3.162696in}{2.208068in}}%
\pgfpathcurveto{\pgfqpoint{3.170932in}{2.208068in}}{\pgfqpoint{3.178832in}{2.211340in}}{\pgfqpoint{3.184656in}{2.217164in}}%
\pgfpathcurveto{\pgfqpoint{3.190480in}{2.222988in}}{\pgfqpoint{3.193752in}{2.230888in}}{\pgfqpoint{3.193752in}{2.239124in}}%
\pgfpathcurveto{\pgfqpoint{3.193752in}{2.247361in}}{\pgfqpoint{3.190480in}{2.255261in}}{\pgfqpoint{3.184656in}{2.261085in}}%
\pgfpathcurveto{\pgfqpoint{3.178832in}{2.266909in}}{\pgfqpoint{3.170932in}{2.270181in}}{\pgfqpoint{3.162696in}{2.270181in}}%
\pgfpathcurveto{\pgfqpoint{3.154460in}{2.270181in}}{\pgfqpoint{3.146559in}{2.266909in}}{\pgfqpoint{3.140736in}{2.261085in}}%
\pgfpathcurveto{\pgfqpoint{3.134912in}{2.255261in}}{\pgfqpoint{3.131639in}{2.247361in}}{\pgfqpoint{3.131639in}{2.239124in}}%
\pgfpathcurveto{\pgfqpoint{3.131639in}{2.230888in}}{\pgfqpoint{3.134912in}{2.222988in}}{\pgfqpoint{3.140736in}{2.217164in}}%
\pgfpathcurveto{\pgfqpoint{3.146559in}{2.211340in}}{\pgfqpoint{3.154460in}{2.208068in}}{\pgfqpoint{3.162696in}{2.208068in}}%
\pgfpathclose%
\pgfusepath{stroke,fill}%
\end{pgfscope}%
\begin{pgfscope}%
\pgfpathrectangle{\pgfqpoint{0.100000in}{0.212622in}}{\pgfqpoint{3.696000in}{3.696000in}}%
\pgfusepath{clip}%
\pgfsetbuttcap%
\pgfsetroundjoin%
\definecolor{currentfill}{rgb}{0.121569,0.466667,0.705882}%
\pgfsetfillcolor{currentfill}%
\pgfsetfillopacity{0.613452}%
\pgfsetlinewidth{1.003750pt}%
\definecolor{currentstroke}{rgb}{0.121569,0.466667,0.705882}%
\pgfsetstrokecolor{currentstroke}%
\pgfsetstrokeopacity{0.613452}%
\pgfsetdash{}{0pt}%
\pgfpathmoveto{\pgfqpoint{3.161232in}{2.207893in}}%
\pgfpathcurveto{\pgfqpoint{3.169468in}{2.207893in}}{\pgfqpoint{3.177368in}{2.211165in}}{\pgfqpoint{3.183192in}{2.216989in}}%
\pgfpathcurveto{\pgfqpoint{3.189016in}{2.222813in}}{\pgfqpoint{3.192289in}{2.230713in}}{\pgfqpoint{3.192289in}{2.238949in}}%
\pgfpathcurveto{\pgfqpoint{3.192289in}{2.247185in}}{\pgfqpoint{3.189016in}{2.255085in}}{\pgfqpoint{3.183192in}{2.260909in}}%
\pgfpathcurveto{\pgfqpoint{3.177368in}{2.266733in}}{\pgfqpoint{3.169468in}{2.270006in}}{\pgfqpoint{3.161232in}{2.270006in}}%
\pgfpathcurveto{\pgfqpoint{3.152996in}{2.270006in}}{\pgfqpoint{3.145096in}{2.266733in}}{\pgfqpoint{3.139272in}{2.260909in}}%
\pgfpathcurveto{\pgfqpoint{3.133448in}{2.255085in}}{\pgfqpoint{3.130176in}{2.247185in}}{\pgfqpoint{3.130176in}{2.238949in}}%
\pgfpathcurveto{\pgfqpoint{3.130176in}{2.230713in}}{\pgfqpoint{3.133448in}{2.222813in}}{\pgfqpoint{3.139272in}{2.216989in}}%
\pgfpathcurveto{\pgfqpoint{3.145096in}{2.211165in}}{\pgfqpoint{3.152996in}{2.207893in}}{\pgfqpoint{3.161232in}{2.207893in}}%
\pgfpathclose%
\pgfusepath{stroke,fill}%
\end{pgfscope}%
\begin{pgfscope}%
\pgfpathrectangle{\pgfqpoint{0.100000in}{0.212622in}}{\pgfqpoint{3.696000in}{3.696000in}}%
\pgfusepath{clip}%
\pgfsetbuttcap%
\pgfsetroundjoin%
\definecolor{currentfill}{rgb}{0.121569,0.466667,0.705882}%
\pgfsetfillcolor{currentfill}%
\pgfsetfillopacity{0.614378}%
\pgfsetlinewidth{1.003750pt}%
\definecolor{currentstroke}{rgb}{0.121569,0.466667,0.705882}%
\pgfsetstrokecolor{currentstroke}%
\pgfsetstrokeopacity{0.614378}%
\pgfsetdash{}{0pt}%
\pgfpathmoveto{\pgfqpoint{3.159062in}{2.207310in}}%
\pgfpathcurveto{\pgfqpoint{3.167299in}{2.207310in}}{\pgfqpoint{3.175199in}{2.210583in}}{\pgfqpoint{3.181023in}{2.216407in}}%
\pgfpathcurveto{\pgfqpoint{3.186847in}{2.222230in}}{\pgfqpoint{3.190119in}{2.230131in}}{\pgfqpoint{3.190119in}{2.238367in}}%
\pgfpathcurveto{\pgfqpoint{3.190119in}{2.246603in}}{\pgfqpoint{3.186847in}{2.254503in}}{\pgfqpoint{3.181023in}{2.260327in}}%
\pgfpathcurveto{\pgfqpoint{3.175199in}{2.266151in}}{\pgfqpoint{3.167299in}{2.269423in}}{\pgfqpoint{3.159062in}{2.269423in}}%
\pgfpathcurveto{\pgfqpoint{3.150826in}{2.269423in}}{\pgfqpoint{3.142926in}{2.266151in}}{\pgfqpoint{3.137102in}{2.260327in}}%
\pgfpathcurveto{\pgfqpoint{3.131278in}{2.254503in}}{\pgfqpoint{3.128006in}{2.246603in}}{\pgfqpoint{3.128006in}{2.238367in}}%
\pgfpathcurveto{\pgfqpoint{3.128006in}{2.230131in}}{\pgfqpoint{3.131278in}{2.222230in}}{\pgfqpoint{3.137102in}{2.216407in}}%
\pgfpathcurveto{\pgfqpoint{3.142926in}{2.210583in}}{\pgfqpoint{3.150826in}{2.207310in}}{\pgfqpoint{3.159062in}{2.207310in}}%
\pgfpathclose%
\pgfusepath{stroke,fill}%
\end{pgfscope}%
\begin{pgfscope}%
\pgfpathrectangle{\pgfqpoint{0.100000in}{0.212622in}}{\pgfqpoint{3.696000in}{3.696000in}}%
\pgfusepath{clip}%
\pgfsetbuttcap%
\pgfsetroundjoin%
\definecolor{currentfill}{rgb}{0.121569,0.466667,0.705882}%
\pgfsetfillcolor{currentfill}%
\pgfsetfillopacity{0.615862}%
\pgfsetlinewidth{1.003750pt}%
\definecolor{currentstroke}{rgb}{0.121569,0.466667,0.705882}%
\pgfsetstrokecolor{currentstroke}%
\pgfsetstrokeopacity{0.615862}%
\pgfsetdash{}{0pt}%
\pgfpathmoveto{\pgfqpoint{3.155663in}{2.206084in}}%
\pgfpathcurveto{\pgfqpoint{3.163899in}{2.206084in}}{\pgfqpoint{3.171800in}{2.209357in}}{\pgfqpoint{3.177623in}{2.215181in}}%
\pgfpathcurveto{\pgfqpoint{3.183447in}{2.221005in}}{\pgfqpoint{3.186720in}{2.228905in}}{\pgfqpoint{3.186720in}{2.237141in}}%
\pgfpathcurveto{\pgfqpoint{3.186720in}{2.245377in}}{\pgfqpoint{3.183447in}{2.253277in}}{\pgfqpoint{3.177623in}{2.259101in}}%
\pgfpathcurveto{\pgfqpoint{3.171800in}{2.264925in}}{\pgfqpoint{3.163899in}{2.268197in}}{\pgfqpoint{3.155663in}{2.268197in}}%
\pgfpathcurveto{\pgfqpoint{3.147427in}{2.268197in}}{\pgfqpoint{3.139527in}{2.264925in}}{\pgfqpoint{3.133703in}{2.259101in}}%
\pgfpathcurveto{\pgfqpoint{3.127879in}{2.253277in}}{\pgfqpoint{3.124607in}{2.245377in}}{\pgfqpoint{3.124607in}{2.237141in}}%
\pgfpathcurveto{\pgfqpoint{3.124607in}{2.228905in}}{\pgfqpoint{3.127879in}{2.221005in}}{\pgfqpoint{3.133703in}{2.215181in}}%
\pgfpathcurveto{\pgfqpoint{3.139527in}{2.209357in}}{\pgfqpoint{3.147427in}{2.206084in}}{\pgfqpoint{3.155663in}{2.206084in}}%
\pgfpathclose%
\pgfusepath{stroke,fill}%
\end{pgfscope}%
\begin{pgfscope}%
\pgfpathrectangle{\pgfqpoint{0.100000in}{0.212622in}}{\pgfqpoint{3.696000in}{3.696000in}}%
\pgfusepath{clip}%
\pgfsetbuttcap%
\pgfsetroundjoin%
\definecolor{currentfill}{rgb}{0.121569,0.466667,0.705882}%
\pgfsetfillcolor{currentfill}%
\pgfsetfillopacity{0.617818}%
\pgfsetlinewidth{1.003750pt}%
\definecolor{currentstroke}{rgb}{0.121569,0.466667,0.705882}%
\pgfsetstrokecolor{currentstroke}%
\pgfsetstrokeopacity{0.617818}%
\pgfsetdash{}{0pt}%
\pgfpathmoveto{\pgfqpoint{3.151357in}{2.205674in}}%
\pgfpathcurveto{\pgfqpoint{3.159594in}{2.205674in}}{\pgfqpoint{3.167494in}{2.208946in}}{\pgfqpoint{3.173318in}{2.214770in}}%
\pgfpathcurveto{\pgfqpoint{3.179142in}{2.220594in}}{\pgfqpoint{3.182414in}{2.228494in}}{\pgfqpoint{3.182414in}{2.236730in}}%
\pgfpathcurveto{\pgfqpoint{3.182414in}{2.244966in}}{\pgfqpoint{3.179142in}{2.252866in}}{\pgfqpoint{3.173318in}{2.258690in}}%
\pgfpathcurveto{\pgfqpoint{3.167494in}{2.264514in}}{\pgfqpoint{3.159594in}{2.267787in}}{\pgfqpoint{3.151357in}{2.267787in}}%
\pgfpathcurveto{\pgfqpoint{3.143121in}{2.267787in}}{\pgfqpoint{3.135221in}{2.264514in}}{\pgfqpoint{3.129397in}{2.258690in}}%
\pgfpathcurveto{\pgfqpoint{3.123573in}{2.252866in}}{\pgfqpoint{3.120301in}{2.244966in}}{\pgfqpoint{3.120301in}{2.236730in}}%
\pgfpathcurveto{\pgfqpoint{3.120301in}{2.228494in}}{\pgfqpoint{3.123573in}{2.220594in}}{\pgfqpoint{3.129397in}{2.214770in}}%
\pgfpathcurveto{\pgfqpoint{3.135221in}{2.208946in}}{\pgfqpoint{3.143121in}{2.205674in}}{\pgfqpoint{3.151357in}{2.205674in}}%
\pgfpathclose%
\pgfusepath{stroke,fill}%
\end{pgfscope}%
\begin{pgfscope}%
\pgfpathrectangle{\pgfqpoint{0.100000in}{0.212622in}}{\pgfqpoint{3.696000in}{3.696000in}}%
\pgfusepath{clip}%
\pgfsetbuttcap%
\pgfsetroundjoin%
\definecolor{currentfill}{rgb}{0.121569,0.466667,0.705882}%
\pgfsetfillcolor{currentfill}%
\pgfsetfillopacity{0.619857}%
\pgfsetlinewidth{1.003750pt}%
\definecolor{currentstroke}{rgb}{0.121569,0.466667,0.705882}%
\pgfsetstrokecolor{currentstroke}%
\pgfsetstrokeopacity{0.619857}%
\pgfsetdash{}{0pt}%
\pgfpathmoveto{\pgfqpoint{3.146475in}{2.204682in}}%
\pgfpathcurveto{\pgfqpoint{3.154711in}{2.204682in}}{\pgfqpoint{3.162611in}{2.207955in}}{\pgfqpoint{3.168435in}{2.213779in}}%
\pgfpathcurveto{\pgfqpoint{3.174259in}{2.219602in}}{\pgfqpoint{3.177531in}{2.227502in}}{\pgfqpoint{3.177531in}{2.235739in}}%
\pgfpathcurveto{\pgfqpoint{3.177531in}{2.243975in}}{\pgfqpoint{3.174259in}{2.251875in}}{\pgfqpoint{3.168435in}{2.257699in}}%
\pgfpathcurveto{\pgfqpoint{3.162611in}{2.263523in}}{\pgfqpoint{3.154711in}{2.266795in}}{\pgfqpoint{3.146475in}{2.266795in}}%
\pgfpathcurveto{\pgfqpoint{3.138238in}{2.266795in}}{\pgfqpoint{3.130338in}{2.263523in}}{\pgfqpoint{3.124514in}{2.257699in}}%
\pgfpathcurveto{\pgfqpoint{3.118690in}{2.251875in}}{\pgfqpoint{3.115418in}{2.243975in}}{\pgfqpoint{3.115418in}{2.235739in}}%
\pgfpathcurveto{\pgfqpoint{3.115418in}{2.227502in}}{\pgfqpoint{3.118690in}{2.219602in}}{\pgfqpoint{3.124514in}{2.213779in}}%
\pgfpathcurveto{\pgfqpoint{3.130338in}{2.207955in}}{\pgfqpoint{3.138238in}{2.204682in}}{\pgfqpoint{3.146475in}{2.204682in}}%
\pgfpathclose%
\pgfusepath{stroke,fill}%
\end{pgfscope}%
\begin{pgfscope}%
\pgfpathrectangle{\pgfqpoint{0.100000in}{0.212622in}}{\pgfqpoint{3.696000in}{3.696000in}}%
\pgfusepath{clip}%
\pgfsetbuttcap%
\pgfsetroundjoin%
\definecolor{currentfill}{rgb}{0.121569,0.466667,0.705882}%
\pgfsetfillcolor{currentfill}%
\pgfsetfillopacity{0.621011}%
\pgfsetlinewidth{1.003750pt}%
\definecolor{currentstroke}{rgb}{0.121569,0.466667,0.705882}%
\pgfsetstrokecolor{currentstroke}%
\pgfsetstrokeopacity{0.621011}%
\pgfsetdash{}{0pt}%
\pgfpathmoveto{\pgfqpoint{3.143910in}{2.204178in}}%
\pgfpathcurveto{\pgfqpoint{3.152146in}{2.204178in}}{\pgfqpoint{3.160046in}{2.207450in}}{\pgfqpoint{3.165870in}{2.213274in}}%
\pgfpathcurveto{\pgfqpoint{3.171694in}{2.219098in}}{\pgfqpoint{3.174966in}{2.226998in}}{\pgfqpoint{3.174966in}{2.235234in}}%
\pgfpathcurveto{\pgfqpoint{3.174966in}{2.243471in}}{\pgfqpoint{3.171694in}{2.251371in}}{\pgfqpoint{3.165870in}{2.257195in}}%
\pgfpathcurveto{\pgfqpoint{3.160046in}{2.263019in}}{\pgfqpoint{3.152146in}{2.266291in}}{\pgfqpoint{3.143910in}{2.266291in}}%
\pgfpathcurveto{\pgfqpoint{3.135673in}{2.266291in}}{\pgfqpoint{3.127773in}{2.263019in}}{\pgfqpoint{3.121950in}{2.257195in}}%
\pgfpathcurveto{\pgfqpoint{3.116126in}{2.251371in}}{\pgfqpoint{3.112853in}{2.243471in}}{\pgfqpoint{3.112853in}{2.235234in}}%
\pgfpathcurveto{\pgfqpoint{3.112853in}{2.226998in}}{\pgfqpoint{3.116126in}{2.219098in}}{\pgfqpoint{3.121950in}{2.213274in}}%
\pgfpathcurveto{\pgfqpoint{3.127773in}{2.207450in}}{\pgfqpoint{3.135673in}{2.204178in}}{\pgfqpoint{3.143910in}{2.204178in}}%
\pgfpathclose%
\pgfusepath{stroke,fill}%
\end{pgfscope}%
\begin{pgfscope}%
\pgfpathrectangle{\pgfqpoint{0.100000in}{0.212622in}}{\pgfqpoint{3.696000in}{3.696000in}}%
\pgfusepath{clip}%
\pgfsetbuttcap%
\pgfsetroundjoin%
\definecolor{currentfill}{rgb}{0.121569,0.466667,0.705882}%
\pgfsetfillcolor{currentfill}%
\pgfsetfillopacity{0.622823}%
\pgfsetlinewidth{1.003750pt}%
\definecolor{currentstroke}{rgb}{0.121569,0.466667,0.705882}%
\pgfsetstrokecolor{currentstroke}%
\pgfsetstrokeopacity{0.622823}%
\pgfsetdash{}{0pt}%
\pgfpathmoveto{\pgfqpoint{3.139771in}{2.203474in}}%
\pgfpathcurveto{\pgfqpoint{3.148007in}{2.203474in}}{\pgfqpoint{3.155908in}{2.206746in}}{\pgfqpoint{3.161731in}{2.212570in}}%
\pgfpathcurveto{\pgfqpoint{3.167555in}{2.218394in}}{\pgfqpoint{3.170828in}{2.226294in}}{\pgfqpoint{3.170828in}{2.234530in}}%
\pgfpathcurveto{\pgfqpoint{3.170828in}{2.242767in}}{\pgfqpoint{3.167555in}{2.250667in}}{\pgfqpoint{3.161731in}{2.256491in}}%
\pgfpathcurveto{\pgfqpoint{3.155908in}{2.262315in}}{\pgfqpoint{3.148007in}{2.265587in}}{\pgfqpoint{3.139771in}{2.265587in}}%
\pgfpathcurveto{\pgfqpoint{3.131535in}{2.265587in}}{\pgfqpoint{3.123635in}{2.262315in}}{\pgfqpoint{3.117811in}{2.256491in}}%
\pgfpathcurveto{\pgfqpoint{3.111987in}{2.250667in}}{\pgfqpoint{3.108715in}{2.242767in}}{\pgfqpoint{3.108715in}{2.234530in}}%
\pgfpathcurveto{\pgfqpoint{3.108715in}{2.226294in}}{\pgfqpoint{3.111987in}{2.218394in}}{\pgfqpoint{3.117811in}{2.212570in}}%
\pgfpathcurveto{\pgfqpoint{3.123635in}{2.206746in}}{\pgfqpoint{3.131535in}{2.203474in}}{\pgfqpoint{3.139771in}{2.203474in}}%
\pgfpathclose%
\pgfusepath{stroke,fill}%
\end{pgfscope}%
\begin{pgfscope}%
\pgfpathrectangle{\pgfqpoint{0.100000in}{0.212622in}}{\pgfqpoint{3.696000in}{3.696000in}}%
\pgfusepath{clip}%
\pgfsetbuttcap%
\pgfsetroundjoin%
\definecolor{currentfill}{rgb}{0.121569,0.466667,0.705882}%
\pgfsetfillcolor{currentfill}%
\pgfsetfillopacity{0.624984}%
\pgfsetlinewidth{1.003750pt}%
\definecolor{currentstroke}{rgb}{0.121569,0.466667,0.705882}%
\pgfsetstrokecolor{currentstroke}%
\pgfsetstrokeopacity{0.624984}%
\pgfsetdash{}{0pt}%
\pgfpathmoveto{\pgfqpoint{3.135099in}{2.201473in}}%
\pgfpathcurveto{\pgfqpoint{3.143335in}{2.201473in}}{\pgfqpoint{3.151235in}{2.204745in}}{\pgfqpoint{3.157059in}{2.210569in}}%
\pgfpathcurveto{\pgfqpoint{3.162883in}{2.216393in}}{\pgfqpoint{3.166155in}{2.224293in}}{\pgfqpoint{3.166155in}{2.232529in}}%
\pgfpathcurveto{\pgfqpoint{3.166155in}{2.240765in}}{\pgfqpoint{3.162883in}{2.248665in}}{\pgfqpoint{3.157059in}{2.254489in}}%
\pgfpathcurveto{\pgfqpoint{3.151235in}{2.260313in}}{\pgfqpoint{3.143335in}{2.263586in}}{\pgfqpoint{3.135099in}{2.263586in}}%
\pgfpathcurveto{\pgfqpoint{3.126863in}{2.263586in}}{\pgfqpoint{3.118963in}{2.260313in}}{\pgfqpoint{3.113139in}{2.254489in}}%
\pgfpathcurveto{\pgfqpoint{3.107315in}{2.248665in}}{\pgfqpoint{3.104042in}{2.240765in}}{\pgfqpoint{3.104042in}{2.232529in}}%
\pgfpathcurveto{\pgfqpoint{3.104042in}{2.224293in}}{\pgfqpoint{3.107315in}{2.216393in}}{\pgfqpoint{3.113139in}{2.210569in}}%
\pgfpathcurveto{\pgfqpoint{3.118963in}{2.204745in}}{\pgfqpoint{3.126863in}{2.201473in}}{\pgfqpoint{3.135099in}{2.201473in}}%
\pgfpathclose%
\pgfusepath{stroke,fill}%
\end{pgfscope}%
\begin{pgfscope}%
\pgfpathrectangle{\pgfqpoint{0.100000in}{0.212622in}}{\pgfqpoint{3.696000in}{3.696000in}}%
\pgfusepath{clip}%
\pgfsetbuttcap%
\pgfsetroundjoin%
\definecolor{currentfill}{rgb}{0.121569,0.466667,0.705882}%
\pgfsetfillcolor{currentfill}%
\pgfsetfillopacity{0.626249}%
\pgfsetlinewidth{1.003750pt}%
\definecolor{currentstroke}{rgb}{0.121569,0.466667,0.705882}%
\pgfsetstrokecolor{currentstroke}%
\pgfsetstrokeopacity{0.626249}%
\pgfsetdash{}{0pt}%
\pgfpathmoveto{\pgfqpoint{3.132561in}{2.200841in}}%
\pgfpathcurveto{\pgfqpoint{3.140797in}{2.200841in}}{\pgfqpoint{3.148697in}{2.204113in}}{\pgfqpoint{3.154521in}{2.209937in}}%
\pgfpathcurveto{\pgfqpoint{3.160345in}{2.215761in}}{\pgfqpoint{3.163617in}{2.223661in}}{\pgfqpoint{3.163617in}{2.231897in}}%
\pgfpathcurveto{\pgfqpoint{3.163617in}{2.240134in}}{\pgfqpoint{3.160345in}{2.248034in}}{\pgfqpoint{3.154521in}{2.253858in}}%
\pgfpathcurveto{\pgfqpoint{3.148697in}{2.259682in}}{\pgfqpoint{3.140797in}{2.262954in}}{\pgfqpoint{3.132561in}{2.262954in}}%
\pgfpathcurveto{\pgfqpoint{3.124324in}{2.262954in}}{\pgfqpoint{3.116424in}{2.259682in}}{\pgfqpoint{3.110600in}{2.253858in}}%
\pgfpathcurveto{\pgfqpoint{3.104776in}{2.248034in}}{\pgfqpoint{3.101504in}{2.240134in}}{\pgfqpoint{3.101504in}{2.231897in}}%
\pgfpathcurveto{\pgfqpoint{3.101504in}{2.223661in}}{\pgfqpoint{3.104776in}{2.215761in}}{\pgfqpoint{3.110600in}{2.209937in}}%
\pgfpathcurveto{\pgfqpoint{3.116424in}{2.204113in}}{\pgfqpoint{3.124324in}{2.200841in}}{\pgfqpoint{3.132561in}{2.200841in}}%
\pgfpathclose%
\pgfusepath{stroke,fill}%
\end{pgfscope}%
\begin{pgfscope}%
\pgfpathrectangle{\pgfqpoint{0.100000in}{0.212622in}}{\pgfqpoint{3.696000in}{3.696000in}}%
\pgfusepath{clip}%
\pgfsetbuttcap%
\pgfsetroundjoin%
\definecolor{currentfill}{rgb}{0.121569,0.466667,0.705882}%
\pgfsetfillcolor{currentfill}%
\pgfsetfillopacity{0.626898}%
\pgfsetlinewidth{1.003750pt}%
\definecolor{currentstroke}{rgb}{0.121569,0.466667,0.705882}%
\pgfsetstrokecolor{currentstroke}%
\pgfsetstrokeopacity{0.626898}%
\pgfsetdash{}{0pt}%
\pgfpathmoveto{\pgfqpoint{3.131066in}{2.200307in}}%
\pgfpathcurveto{\pgfqpoint{3.139302in}{2.200307in}}{\pgfqpoint{3.147202in}{2.203579in}}{\pgfqpoint{3.153026in}{2.209403in}}%
\pgfpathcurveto{\pgfqpoint{3.158850in}{2.215227in}}{\pgfqpoint{3.162123in}{2.223127in}}{\pgfqpoint{3.162123in}{2.231363in}}%
\pgfpathcurveto{\pgfqpoint{3.162123in}{2.239599in}}{\pgfqpoint{3.158850in}{2.247499in}}{\pgfqpoint{3.153026in}{2.253323in}}%
\pgfpathcurveto{\pgfqpoint{3.147202in}{2.259147in}}{\pgfqpoint{3.139302in}{2.262420in}}{\pgfqpoint{3.131066in}{2.262420in}}%
\pgfpathcurveto{\pgfqpoint{3.122830in}{2.262420in}}{\pgfqpoint{3.114930in}{2.259147in}}{\pgfqpoint{3.109106in}{2.253323in}}%
\pgfpathcurveto{\pgfqpoint{3.103282in}{2.247499in}}{\pgfqpoint{3.100010in}{2.239599in}}{\pgfqpoint{3.100010in}{2.231363in}}%
\pgfpathcurveto{\pgfqpoint{3.100010in}{2.223127in}}{\pgfqpoint{3.103282in}{2.215227in}}{\pgfqpoint{3.109106in}{2.209403in}}%
\pgfpathcurveto{\pgfqpoint{3.114930in}{2.203579in}}{\pgfqpoint{3.122830in}{2.200307in}}{\pgfqpoint{3.131066in}{2.200307in}}%
\pgfpathclose%
\pgfusepath{stroke,fill}%
\end{pgfscope}%
\begin{pgfscope}%
\pgfpathrectangle{\pgfqpoint{0.100000in}{0.212622in}}{\pgfqpoint{3.696000in}{3.696000in}}%
\pgfusepath{clip}%
\pgfsetbuttcap%
\pgfsetroundjoin%
\definecolor{currentfill}{rgb}{0.121569,0.466667,0.705882}%
\pgfsetfillcolor{currentfill}%
\pgfsetfillopacity{0.627981}%
\pgfsetlinewidth{1.003750pt}%
\definecolor{currentstroke}{rgb}{0.121569,0.466667,0.705882}%
\pgfsetstrokecolor{currentstroke}%
\pgfsetstrokeopacity{0.627981}%
\pgfsetdash{}{0pt}%
\pgfpathmoveto{\pgfqpoint{3.128746in}{2.199279in}}%
\pgfpathcurveto{\pgfqpoint{3.136982in}{2.199279in}}{\pgfqpoint{3.144882in}{2.202551in}}{\pgfqpoint{3.150706in}{2.208375in}}%
\pgfpathcurveto{\pgfqpoint{3.156530in}{2.214199in}}{\pgfqpoint{3.159802in}{2.222099in}}{\pgfqpoint{3.159802in}{2.230336in}}%
\pgfpathcurveto{\pgfqpoint{3.159802in}{2.238572in}}{\pgfqpoint{3.156530in}{2.246472in}}{\pgfqpoint{3.150706in}{2.252296in}}%
\pgfpathcurveto{\pgfqpoint{3.144882in}{2.258120in}}{\pgfqpoint{3.136982in}{2.261392in}}{\pgfqpoint{3.128746in}{2.261392in}}%
\pgfpathcurveto{\pgfqpoint{3.120510in}{2.261392in}}{\pgfqpoint{3.112609in}{2.258120in}}{\pgfqpoint{3.106786in}{2.252296in}}%
\pgfpathcurveto{\pgfqpoint{3.100962in}{2.246472in}}{\pgfqpoint{3.097689in}{2.238572in}}{\pgfqpoint{3.097689in}{2.230336in}}%
\pgfpathcurveto{\pgfqpoint{3.097689in}{2.222099in}}{\pgfqpoint{3.100962in}{2.214199in}}{\pgfqpoint{3.106786in}{2.208375in}}%
\pgfpathcurveto{\pgfqpoint{3.112609in}{2.202551in}}{\pgfqpoint{3.120510in}{2.199279in}}{\pgfqpoint{3.128746in}{2.199279in}}%
\pgfpathclose%
\pgfusepath{stroke,fill}%
\end{pgfscope}%
\begin{pgfscope}%
\pgfpathrectangle{\pgfqpoint{0.100000in}{0.212622in}}{\pgfqpoint{3.696000in}{3.696000in}}%
\pgfusepath{clip}%
\pgfsetbuttcap%
\pgfsetroundjoin%
\definecolor{currentfill}{rgb}{0.121569,0.466667,0.705882}%
\pgfsetfillcolor{currentfill}%
\pgfsetfillopacity{0.628628}%
\pgfsetlinewidth{1.003750pt}%
\definecolor{currentstroke}{rgb}{0.121569,0.466667,0.705882}%
\pgfsetstrokecolor{currentstroke}%
\pgfsetstrokeopacity{0.628628}%
\pgfsetdash{}{0pt}%
\pgfpathmoveto{\pgfqpoint{3.127408in}{2.199145in}}%
\pgfpathcurveto{\pgfqpoint{3.135645in}{2.199145in}}{\pgfqpoint{3.143545in}{2.202417in}}{\pgfqpoint{3.149369in}{2.208241in}}%
\pgfpathcurveto{\pgfqpoint{3.155193in}{2.214065in}}{\pgfqpoint{3.158465in}{2.221965in}}{\pgfqpoint{3.158465in}{2.230201in}}%
\pgfpathcurveto{\pgfqpoint{3.158465in}{2.238437in}}{\pgfqpoint{3.155193in}{2.246337in}}{\pgfqpoint{3.149369in}{2.252161in}}%
\pgfpathcurveto{\pgfqpoint{3.143545in}{2.257985in}}{\pgfqpoint{3.135645in}{2.261258in}}{\pgfqpoint{3.127408in}{2.261258in}}%
\pgfpathcurveto{\pgfqpoint{3.119172in}{2.261258in}}{\pgfqpoint{3.111272in}{2.257985in}}{\pgfqpoint{3.105448in}{2.252161in}}%
\pgfpathcurveto{\pgfqpoint{3.099624in}{2.246337in}}{\pgfqpoint{3.096352in}{2.238437in}}{\pgfqpoint{3.096352in}{2.230201in}}%
\pgfpathcurveto{\pgfqpoint{3.096352in}{2.221965in}}{\pgfqpoint{3.099624in}{2.214065in}}{\pgfqpoint{3.105448in}{2.208241in}}%
\pgfpathcurveto{\pgfqpoint{3.111272in}{2.202417in}}{\pgfqpoint{3.119172in}{2.199145in}}{\pgfqpoint{3.127408in}{2.199145in}}%
\pgfpathclose%
\pgfusepath{stroke,fill}%
\end{pgfscope}%
\begin{pgfscope}%
\pgfpathrectangle{\pgfqpoint{0.100000in}{0.212622in}}{\pgfqpoint{3.696000in}{3.696000in}}%
\pgfusepath{clip}%
\pgfsetbuttcap%
\pgfsetroundjoin%
\definecolor{currentfill}{rgb}{0.121569,0.466667,0.705882}%
\pgfsetfillcolor{currentfill}%
\pgfsetfillopacity{0.629437}%
\pgfsetlinewidth{1.003750pt}%
\definecolor{currentstroke}{rgb}{0.121569,0.466667,0.705882}%
\pgfsetstrokecolor{currentstroke}%
\pgfsetstrokeopacity{0.629437}%
\pgfsetdash{}{0pt}%
\pgfpathmoveto{\pgfqpoint{3.125626in}{2.198762in}}%
\pgfpathcurveto{\pgfqpoint{3.133862in}{2.198762in}}{\pgfqpoint{3.141762in}{2.202034in}}{\pgfqpoint{3.147586in}{2.207858in}}%
\pgfpathcurveto{\pgfqpoint{3.153410in}{2.213682in}}{\pgfqpoint{3.156682in}{2.221582in}}{\pgfqpoint{3.156682in}{2.229818in}}%
\pgfpathcurveto{\pgfqpoint{3.156682in}{2.238055in}}{\pgfqpoint{3.153410in}{2.245955in}}{\pgfqpoint{3.147586in}{2.251779in}}%
\pgfpathcurveto{\pgfqpoint{3.141762in}{2.257603in}}{\pgfqpoint{3.133862in}{2.260875in}}{\pgfqpoint{3.125626in}{2.260875in}}%
\pgfpathcurveto{\pgfqpoint{3.117389in}{2.260875in}}{\pgfqpoint{3.109489in}{2.257603in}}{\pgfqpoint{3.103665in}{2.251779in}}%
\pgfpathcurveto{\pgfqpoint{3.097841in}{2.245955in}}{\pgfqpoint{3.094569in}{2.238055in}}{\pgfqpoint{3.094569in}{2.229818in}}%
\pgfpathcurveto{\pgfqpoint{3.094569in}{2.221582in}}{\pgfqpoint{3.097841in}{2.213682in}}{\pgfqpoint{3.103665in}{2.207858in}}%
\pgfpathcurveto{\pgfqpoint{3.109489in}{2.202034in}}{\pgfqpoint{3.117389in}{2.198762in}}{\pgfqpoint{3.125626in}{2.198762in}}%
\pgfpathclose%
\pgfusepath{stroke,fill}%
\end{pgfscope}%
\begin{pgfscope}%
\pgfpathrectangle{\pgfqpoint{0.100000in}{0.212622in}}{\pgfqpoint{3.696000in}{3.696000in}}%
\pgfusepath{clip}%
\pgfsetbuttcap%
\pgfsetroundjoin%
\definecolor{currentfill}{rgb}{0.121569,0.466667,0.705882}%
\pgfsetfillcolor{currentfill}%
\pgfsetfillopacity{0.629901}%
\pgfsetlinewidth{1.003750pt}%
\definecolor{currentstroke}{rgb}{0.121569,0.466667,0.705882}%
\pgfsetstrokecolor{currentstroke}%
\pgfsetstrokeopacity{0.629901}%
\pgfsetdash{}{0pt}%
\pgfpathmoveto{\pgfqpoint{3.124698in}{2.198608in}}%
\pgfpathcurveto{\pgfqpoint{3.132935in}{2.198608in}}{\pgfqpoint{3.140835in}{2.201880in}}{\pgfqpoint{3.146659in}{2.207704in}}%
\pgfpathcurveto{\pgfqpoint{3.152483in}{2.213528in}}{\pgfqpoint{3.155755in}{2.221428in}}{\pgfqpoint{3.155755in}{2.229664in}}%
\pgfpathcurveto{\pgfqpoint{3.155755in}{2.237901in}}{\pgfqpoint{3.152483in}{2.245801in}}{\pgfqpoint{3.146659in}{2.251624in}}%
\pgfpathcurveto{\pgfqpoint{3.140835in}{2.257448in}}{\pgfqpoint{3.132935in}{2.260721in}}{\pgfqpoint{3.124698in}{2.260721in}}%
\pgfpathcurveto{\pgfqpoint{3.116462in}{2.260721in}}{\pgfqpoint{3.108562in}{2.257448in}}{\pgfqpoint{3.102738in}{2.251624in}}%
\pgfpathcurveto{\pgfqpoint{3.096914in}{2.245801in}}{\pgfqpoint{3.093642in}{2.237901in}}{\pgfqpoint{3.093642in}{2.229664in}}%
\pgfpathcurveto{\pgfqpoint{3.093642in}{2.221428in}}{\pgfqpoint{3.096914in}{2.213528in}}{\pgfqpoint{3.102738in}{2.207704in}}%
\pgfpathcurveto{\pgfqpoint{3.108562in}{2.201880in}}{\pgfqpoint{3.116462in}{2.198608in}}{\pgfqpoint{3.124698in}{2.198608in}}%
\pgfpathclose%
\pgfusepath{stroke,fill}%
\end{pgfscope}%
\begin{pgfscope}%
\pgfpathrectangle{\pgfqpoint{0.100000in}{0.212622in}}{\pgfqpoint{3.696000in}{3.696000in}}%
\pgfusepath{clip}%
\pgfsetbuttcap%
\pgfsetroundjoin%
\definecolor{currentfill}{rgb}{0.121569,0.466667,0.705882}%
\pgfsetfillcolor{currentfill}%
\pgfsetfillopacity{0.630608}%
\pgfsetlinewidth{1.003750pt}%
\definecolor{currentstroke}{rgb}{0.121569,0.466667,0.705882}%
\pgfsetstrokecolor{currentstroke}%
\pgfsetstrokeopacity{0.630608}%
\pgfsetdash{}{0pt}%
\pgfpathmoveto{\pgfqpoint{3.123098in}{2.198105in}}%
\pgfpathcurveto{\pgfqpoint{3.131334in}{2.198105in}}{\pgfqpoint{3.139234in}{2.201378in}}{\pgfqpoint{3.145058in}{2.207202in}}%
\pgfpathcurveto{\pgfqpoint{3.150882in}{2.213026in}}{\pgfqpoint{3.154155in}{2.220926in}}{\pgfqpoint{3.154155in}{2.229162in}}%
\pgfpathcurveto{\pgfqpoint{3.154155in}{2.237398in}}{\pgfqpoint{3.150882in}{2.245298in}}{\pgfqpoint{3.145058in}{2.251122in}}%
\pgfpathcurveto{\pgfqpoint{3.139234in}{2.256946in}}{\pgfqpoint{3.131334in}{2.260218in}}{\pgfqpoint{3.123098in}{2.260218in}}%
\pgfpathcurveto{\pgfqpoint{3.114862in}{2.260218in}}{\pgfqpoint{3.106962in}{2.256946in}}{\pgfqpoint{3.101138in}{2.251122in}}%
\pgfpathcurveto{\pgfqpoint{3.095314in}{2.245298in}}{\pgfqpoint{3.092042in}{2.237398in}}{\pgfqpoint{3.092042in}{2.229162in}}%
\pgfpathcurveto{\pgfqpoint{3.092042in}{2.220926in}}{\pgfqpoint{3.095314in}{2.213026in}}{\pgfqpoint{3.101138in}{2.207202in}}%
\pgfpathcurveto{\pgfqpoint{3.106962in}{2.201378in}}{\pgfqpoint{3.114862in}{2.198105in}}{\pgfqpoint{3.123098in}{2.198105in}}%
\pgfpathclose%
\pgfusepath{stroke,fill}%
\end{pgfscope}%
\begin{pgfscope}%
\pgfpathrectangle{\pgfqpoint{0.100000in}{0.212622in}}{\pgfqpoint{3.696000in}{3.696000in}}%
\pgfusepath{clip}%
\pgfsetbuttcap%
\pgfsetroundjoin%
\definecolor{currentfill}{rgb}{0.121569,0.466667,0.705882}%
\pgfsetfillcolor{currentfill}%
\pgfsetfillopacity{0.631779}%
\pgfsetlinewidth{1.003750pt}%
\definecolor{currentstroke}{rgb}{0.121569,0.466667,0.705882}%
\pgfsetstrokecolor{currentstroke}%
\pgfsetstrokeopacity{0.631779}%
\pgfsetdash{}{0pt}%
\pgfpathmoveto{\pgfqpoint{3.120653in}{2.196929in}}%
\pgfpathcurveto{\pgfqpoint{3.128889in}{2.196929in}}{\pgfqpoint{3.136789in}{2.200201in}}{\pgfqpoint{3.142613in}{2.206025in}}%
\pgfpathcurveto{\pgfqpoint{3.148437in}{2.211849in}}{\pgfqpoint{3.151710in}{2.219749in}}{\pgfqpoint{3.151710in}{2.227986in}}%
\pgfpathcurveto{\pgfqpoint{3.151710in}{2.236222in}}{\pgfqpoint{3.148437in}{2.244122in}}{\pgfqpoint{3.142613in}{2.249946in}}%
\pgfpathcurveto{\pgfqpoint{3.136789in}{2.255770in}}{\pgfqpoint{3.128889in}{2.259042in}}{\pgfqpoint{3.120653in}{2.259042in}}%
\pgfpathcurveto{\pgfqpoint{3.112417in}{2.259042in}}{\pgfqpoint{3.104517in}{2.255770in}}{\pgfqpoint{3.098693in}{2.249946in}}%
\pgfpathcurveto{\pgfqpoint{3.092869in}{2.244122in}}{\pgfqpoint{3.089597in}{2.236222in}}{\pgfqpoint{3.089597in}{2.227986in}}%
\pgfpathcurveto{\pgfqpoint{3.089597in}{2.219749in}}{\pgfqpoint{3.092869in}{2.211849in}}{\pgfqpoint{3.098693in}{2.206025in}}%
\pgfpathcurveto{\pgfqpoint{3.104517in}{2.200201in}}{\pgfqpoint{3.112417in}{2.196929in}}{\pgfqpoint{3.120653in}{2.196929in}}%
\pgfpathclose%
\pgfusepath{stroke,fill}%
\end{pgfscope}%
\begin{pgfscope}%
\pgfpathrectangle{\pgfqpoint{0.100000in}{0.212622in}}{\pgfqpoint{3.696000in}{3.696000in}}%
\pgfusepath{clip}%
\pgfsetbuttcap%
\pgfsetroundjoin%
\definecolor{currentfill}{rgb}{0.121569,0.466667,0.705882}%
\pgfsetfillcolor{currentfill}%
\pgfsetfillopacity{0.632479}%
\pgfsetlinewidth{1.003750pt}%
\definecolor{currentstroke}{rgb}{0.121569,0.466667,0.705882}%
\pgfsetstrokecolor{currentstroke}%
\pgfsetstrokeopacity{0.632479}%
\pgfsetdash{}{0pt}%
\pgfpathmoveto{\pgfqpoint{3.119274in}{2.196698in}}%
\pgfpathcurveto{\pgfqpoint{3.127510in}{2.196698in}}{\pgfqpoint{3.135410in}{2.199971in}}{\pgfqpoint{3.141234in}{2.205795in}}%
\pgfpathcurveto{\pgfqpoint{3.147058in}{2.211619in}}{\pgfqpoint{3.150331in}{2.219519in}}{\pgfqpoint{3.150331in}{2.227755in}}%
\pgfpathcurveto{\pgfqpoint{3.150331in}{2.235991in}}{\pgfqpoint{3.147058in}{2.243891in}}{\pgfqpoint{3.141234in}{2.249715in}}%
\pgfpathcurveto{\pgfqpoint{3.135410in}{2.255539in}}{\pgfqpoint{3.127510in}{2.258811in}}{\pgfqpoint{3.119274in}{2.258811in}}%
\pgfpathcurveto{\pgfqpoint{3.111038in}{2.258811in}}{\pgfqpoint{3.103138in}{2.255539in}}{\pgfqpoint{3.097314in}{2.249715in}}%
\pgfpathcurveto{\pgfqpoint{3.091490in}{2.243891in}}{\pgfqpoint{3.088218in}{2.235991in}}{\pgfqpoint{3.088218in}{2.227755in}}%
\pgfpathcurveto{\pgfqpoint{3.088218in}{2.219519in}}{\pgfqpoint{3.091490in}{2.211619in}}{\pgfqpoint{3.097314in}{2.205795in}}%
\pgfpathcurveto{\pgfqpoint{3.103138in}{2.199971in}}{\pgfqpoint{3.111038in}{2.196698in}}{\pgfqpoint{3.119274in}{2.196698in}}%
\pgfpathclose%
\pgfusepath{stroke,fill}%
\end{pgfscope}%
\begin{pgfscope}%
\pgfpathrectangle{\pgfqpoint{0.100000in}{0.212622in}}{\pgfqpoint{3.696000in}{3.696000in}}%
\pgfusepath{clip}%
\pgfsetbuttcap%
\pgfsetroundjoin%
\definecolor{currentfill}{rgb}{0.121569,0.466667,0.705882}%
\pgfsetfillcolor{currentfill}%
\pgfsetfillopacity{0.632844}%
\pgfsetlinewidth{1.003750pt}%
\definecolor{currentstroke}{rgb}{0.121569,0.466667,0.705882}%
\pgfsetstrokecolor{currentstroke}%
\pgfsetstrokeopacity{0.632844}%
\pgfsetdash{}{0pt}%
\pgfpathmoveto{\pgfqpoint{3.118483in}{2.196483in}}%
\pgfpathcurveto{\pgfqpoint{3.126720in}{2.196483in}}{\pgfqpoint{3.134620in}{2.199756in}}{\pgfqpoint{3.140444in}{2.205580in}}%
\pgfpathcurveto{\pgfqpoint{3.146268in}{2.211404in}}{\pgfqpoint{3.149540in}{2.219304in}}{\pgfqpoint{3.149540in}{2.227540in}}%
\pgfpathcurveto{\pgfqpoint{3.149540in}{2.235776in}}{\pgfqpoint{3.146268in}{2.243676in}}{\pgfqpoint{3.140444in}{2.249500in}}%
\pgfpathcurveto{\pgfqpoint{3.134620in}{2.255324in}}{\pgfqpoint{3.126720in}{2.258596in}}{\pgfqpoint{3.118483in}{2.258596in}}%
\pgfpathcurveto{\pgfqpoint{3.110247in}{2.258596in}}{\pgfqpoint{3.102347in}{2.255324in}}{\pgfqpoint{3.096523in}{2.249500in}}%
\pgfpathcurveto{\pgfqpoint{3.090699in}{2.243676in}}{\pgfqpoint{3.087427in}{2.235776in}}{\pgfqpoint{3.087427in}{2.227540in}}%
\pgfpathcurveto{\pgfqpoint{3.087427in}{2.219304in}}{\pgfqpoint{3.090699in}{2.211404in}}{\pgfqpoint{3.096523in}{2.205580in}}%
\pgfpathcurveto{\pgfqpoint{3.102347in}{2.199756in}}{\pgfqpoint{3.110247in}{2.196483in}}{\pgfqpoint{3.118483in}{2.196483in}}%
\pgfpathclose%
\pgfusepath{stroke,fill}%
\end{pgfscope}%
\begin{pgfscope}%
\pgfpathrectangle{\pgfqpoint{0.100000in}{0.212622in}}{\pgfqpoint{3.696000in}{3.696000in}}%
\pgfusepath{clip}%
\pgfsetbuttcap%
\pgfsetroundjoin%
\definecolor{currentfill}{rgb}{0.121569,0.466667,0.705882}%
\pgfsetfillcolor{currentfill}%
\pgfsetfillopacity{0.633062}%
\pgfsetlinewidth{1.003750pt}%
\definecolor{currentstroke}{rgb}{0.121569,0.466667,0.705882}%
\pgfsetstrokecolor{currentstroke}%
\pgfsetstrokeopacity{0.633062}%
\pgfsetdash{}{0pt}%
\pgfpathmoveto{\pgfqpoint{3.118074in}{2.196443in}}%
\pgfpathcurveto{\pgfqpoint{3.126311in}{2.196443in}}{\pgfqpoint{3.134211in}{2.199715in}}{\pgfqpoint{3.140035in}{2.205539in}}%
\pgfpathcurveto{\pgfqpoint{3.145859in}{2.211363in}}{\pgfqpoint{3.149131in}{2.219263in}}{\pgfqpoint{3.149131in}{2.227499in}}%
\pgfpathcurveto{\pgfqpoint{3.149131in}{2.235735in}}{\pgfqpoint{3.145859in}{2.243635in}}{\pgfqpoint{3.140035in}{2.249459in}}%
\pgfpathcurveto{\pgfqpoint{3.134211in}{2.255283in}}{\pgfqpoint{3.126311in}{2.258556in}}{\pgfqpoint{3.118074in}{2.258556in}}%
\pgfpathcurveto{\pgfqpoint{3.109838in}{2.258556in}}{\pgfqpoint{3.101938in}{2.255283in}}{\pgfqpoint{3.096114in}{2.249459in}}%
\pgfpathcurveto{\pgfqpoint{3.090290in}{2.243635in}}{\pgfqpoint{3.087018in}{2.235735in}}{\pgfqpoint{3.087018in}{2.227499in}}%
\pgfpathcurveto{\pgfqpoint{3.087018in}{2.219263in}}{\pgfqpoint{3.090290in}{2.211363in}}{\pgfqpoint{3.096114in}{2.205539in}}%
\pgfpathcurveto{\pgfqpoint{3.101938in}{2.199715in}}{\pgfqpoint{3.109838in}{2.196443in}}{\pgfqpoint{3.118074in}{2.196443in}}%
\pgfpathclose%
\pgfusepath{stroke,fill}%
\end{pgfscope}%
\begin{pgfscope}%
\pgfpathrectangle{\pgfqpoint{0.100000in}{0.212622in}}{\pgfqpoint{3.696000in}{3.696000in}}%
\pgfusepath{clip}%
\pgfsetbuttcap%
\pgfsetroundjoin%
\definecolor{currentfill}{rgb}{0.121569,0.466667,0.705882}%
\pgfsetfillcolor{currentfill}%
\pgfsetfillopacity{0.633521}%
\pgfsetlinewidth{1.003750pt}%
\definecolor{currentstroke}{rgb}{0.121569,0.466667,0.705882}%
\pgfsetstrokecolor{currentstroke}%
\pgfsetstrokeopacity{0.633521}%
\pgfsetdash{}{0pt}%
\pgfpathmoveto{\pgfqpoint{3.117057in}{2.196083in}}%
\pgfpathcurveto{\pgfqpoint{3.125293in}{2.196083in}}{\pgfqpoint{3.133193in}{2.199356in}}{\pgfqpoint{3.139017in}{2.205179in}}%
\pgfpathcurveto{\pgfqpoint{3.144841in}{2.211003in}}{\pgfqpoint{3.148113in}{2.218903in}}{\pgfqpoint{3.148113in}{2.227140in}}%
\pgfpathcurveto{\pgfqpoint{3.148113in}{2.235376in}}{\pgfqpoint{3.144841in}{2.243276in}}{\pgfqpoint{3.139017in}{2.249100in}}%
\pgfpathcurveto{\pgfqpoint{3.133193in}{2.254924in}}{\pgfqpoint{3.125293in}{2.258196in}}{\pgfqpoint{3.117057in}{2.258196in}}%
\pgfpathcurveto{\pgfqpoint{3.108821in}{2.258196in}}{\pgfqpoint{3.100921in}{2.254924in}}{\pgfqpoint{3.095097in}{2.249100in}}%
\pgfpathcurveto{\pgfqpoint{3.089273in}{2.243276in}}{\pgfqpoint{3.086000in}{2.235376in}}{\pgfqpoint{3.086000in}{2.227140in}}%
\pgfpathcurveto{\pgfqpoint{3.086000in}{2.218903in}}{\pgfqpoint{3.089273in}{2.211003in}}{\pgfqpoint{3.095097in}{2.205179in}}%
\pgfpathcurveto{\pgfqpoint{3.100921in}{2.199356in}}{\pgfqpoint{3.108821in}{2.196083in}}{\pgfqpoint{3.117057in}{2.196083in}}%
\pgfpathclose%
\pgfusepath{stroke,fill}%
\end{pgfscope}%
\begin{pgfscope}%
\pgfpathrectangle{\pgfqpoint{0.100000in}{0.212622in}}{\pgfqpoint{3.696000in}{3.696000in}}%
\pgfusepath{clip}%
\pgfsetbuttcap%
\pgfsetroundjoin%
\definecolor{currentfill}{rgb}{0.121569,0.466667,0.705882}%
\pgfsetfillcolor{currentfill}%
\pgfsetfillopacity{0.634257}%
\pgfsetlinewidth{1.003750pt}%
\definecolor{currentstroke}{rgb}{0.121569,0.466667,0.705882}%
\pgfsetstrokecolor{currentstroke}%
\pgfsetstrokeopacity{0.634257}%
\pgfsetdash{}{0pt}%
\pgfpathmoveto{\pgfqpoint{3.115546in}{2.195707in}}%
\pgfpathcurveto{\pgfqpoint{3.123783in}{2.195707in}}{\pgfqpoint{3.131683in}{2.198980in}}{\pgfqpoint{3.137507in}{2.204803in}}%
\pgfpathcurveto{\pgfqpoint{3.143331in}{2.210627in}}{\pgfqpoint{3.146603in}{2.218527in}}{\pgfqpoint{3.146603in}{2.226764in}}%
\pgfpathcurveto{\pgfqpoint{3.146603in}{2.235000in}}{\pgfqpoint{3.143331in}{2.242900in}}{\pgfqpoint{3.137507in}{2.248724in}}%
\pgfpathcurveto{\pgfqpoint{3.131683in}{2.254548in}}{\pgfqpoint{3.123783in}{2.257820in}}{\pgfqpoint{3.115546in}{2.257820in}}%
\pgfpathcurveto{\pgfqpoint{3.107310in}{2.257820in}}{\pgfqpoint{3.099410in}{2.254548in}}{\pgfqpoint{3.093586in}{2.248724in}}%
\pgfpathcurveto{\pgfqpoint{3.087762in}{2.242900in}}{\pgfqpoint{3.084490in}{2.235000in}}{\pgfqpoint{3.084490in}{2.226764in}}%
\pgfpathcurveto{\pgfqpoint{3.084490in}{2.218527in}}{\pgfqpoint{3.087762in}{2.210627in}}{\pgfqpoint{3.093586in}{2.204803in}}%
\pgfpathcurveto{\pgfqpoint{3.099410in}{2.198980in}}{\pgfqpoint{3.107310in}{2.195707in}}{\pgfqpoint{3.115546in}{2.195707in}}%
\pgfpathclose%
\pgfusepath{stroke,fill}%
\end{pgfscope}%
\begin{pgfscope}%
\pgfpathrectangle{\pgfqpoint{0.100000in}{0.212622in}}{\pgfqpoint{3.696000in}{3.696000in}}%
\pgfusepath{clip}%
\pgfsetbuttcap%
\pgfsetroundjoin%
\definecolor{currentfill}{rgb}{0.121569,0.466667,0.705882}%
\pgfsetfillcolor{currentfill}%
\pgfsetfillopacity{0.635403}%
\pgfsetlinewidth{1.003750pt}%
\definecolor{currentstroke}{rgb}{0.121569,0.466667,0.705882}%
\pgfsetstrokecolor{currentstroke}%
\pgfsetstrokeopacity{0.635403}%
\pgfsetdash{}{0pt}%
\pgfpathmoveto{\pgfqpoint{3.113439in}{2.195668in}}%
\pgfpathcurveto{\pgfqpoint{3.121675in}{2.195668in}}{\pgfqpoint{3.129575in}{2.198940in}}{\pgfqpoint{3.135399in}{2.204764in}}%
\pgfpathcurveto{\pgfqpoint{3.141223in}{2.210588in}}{\pgfqpoint{3.144495in}{2.218488in}}{\pgfqpoint{3.144495in}{2.226725in}}%
\pgfpathcurveto{\pgfqpoint{3.144495in}{2.234961in}}{\pgfqpoint{3.141223in}{2.242861in}}{\pgfqpoint{3.135399in}{2.248685in}}%
\pgfpathcurveto{\pgfqpoint{3.129575in}{2.254509in}}{\pgfqpoint{3.121675in}{2.257781in}}{\pgfqpoint{3.113439in}{2.257781in}}%
\pgfpathcurveto{\pgfqpoint{3.105203in}{2.257781in}}{\pgfqpoint{3.097303in}{2.254509in}}{\pgfqpoint{3.091479in}{2.248685in}}%
\pgfpathcurveto{\pgfqpoint{3.085655in}{2.242861in}}{\pgfqpoint{3.082382in}{2.234961in}}{\pgfqpoint{3.082382in}{2.226725in}}%
\pgfpathcurveto{\pgfqpoint{3.082382in}{2.218488in}}{\pgfqpoint{3.085655in}{2.210588in}}{\pgfqpoint{3.091479in}{2.204764in}}%
\pgfpathcurveto{\pgfqpoint{3.097303in}{2.198940in}}{\pgfqpoint{3.105203in}{2.195668in}}{\pgfqpoint{3.113439in}{2.195668in}}%
\pgfpathclose%
\pgfusepath{stroke,fill}%
\end{pgfscope}%
\begin{pgfscope}%
\pgfpathrectangle{\pgfqpoint{0.100000in}{0.212622in}}{\pgfqpoint{3.696000in}{3.696000in}}%
\pgfusepath{clip}%
\pgfsetbuttcap%
\pgfsetroundjoin%
\definecolor{currentfill}{rgb}{0.121569,0.466667,0.705882}%
\pgfsetfillcolor{currentfill}%
\pgfsetfillopacity{0.636654}%
\pgfsetlinewidth{1.003750pt}%
\definecolor{currentstroke}{rgb}{0.121569,0.466667,0.705882}%
\pgfsetstrokecolor{currentstroke}%
\pgfsetstrokeopacity{0.636654}%
\pgfsetdash{}{0pt}%
\pgfpathmoveto{\pgfqpoint{3.111016in}{2.195044in}}%
\pgfpathcurveto{\pgfqpoint{3.119253in}{2.195044in}}{\pgfqpoint{3.127153in}{2.198317in}}{\pgfqpoint{3.132977in}{2.204141in}}%
\pgfpathcurveto{\pgfqpoint{3.138801in}{2.209965in}}{\pgfqpoint{3.142073in}{2.217865in}}{\pgfqpoint{3.142073in}{2.226101in}}%
\pgfpathcurveto{\pgfqpoint{3.142073in}{2.234337in}}{\pgfqpoint{3.138801in}{2.242237in}}{\pgfqpoint{3.132977in}{2.248061in}}%
\pgfpathcurveto{\pgfqpoint{3.127153in}{2.253885in}}{\pgfqpoint{3.119253in}{2.257157in}}{\pgfqpoint{3.111016in}{2.257157in}}%
\pgfpathcurveto{\pgfqpoint{3.102780in}{2.257157in}}{\pgfqpoint{3.094880in}{2.253885in}}{\pgfqpoint{3.089056in}{2.248061in}}%
\pgfpathcurveto{\pgfqpoint{3.083232in}{2.242237in}}{\pgfqpoint{3.079960in}{2.234337in}}{\pgfqpoint{3.079960in}{2.226101in}}%
\pgfpathcurveto{\pgfqpoint{3.079960in}{2.217865in}}{\pgfqpoint{3.083232in}{2.209965in}}{\pgfqpoint{3.089056in}{2.204141in}}%
\pgfpathcurveto{\pgfqpoint{3.094880in}{2.198317in}}{\pgfqpoint{3.102780in}{2.195044in}}{\pgfqpoint{3.111016in}{2.195044in}}%
\pgfpathclose%
\pgfusepath{stroke,fill}%
\end{pgfscope}%
\begin{pgfscope}%
\pgfpathrectangle{\pgfqpoint{0.100000in}{0.212622in}}{\pgfqpoint{3.696000in}{3.696000in}}%
\pgfusepath{clip}%
\pgfsetbuttcap%
\pgfsetroundjoin%
\definecolor{currentfill}{rgb}{0.121569,0.466667,0.705882}%
\pgfsetfillcolor{currentfill}%
\pgfsetfillopacity{0.637359}%
\pgfsetlinewidth{1.003750pt}%
\definecolor{currentstroke}{rgb}{0.121569,0.466667,0.705882}%
\pgfsetstrokecolor{currentstroke}%
\pgfsetstrokeopacity{0.637359}%
\pgfsetdash{}{0pt}%
\pgfpathmoveto{\pgfqpoint{3.109719in}{2.194774in}}%
\pgfpathcurveto{\pgfqpoint{3.117956in}{2.194774in}}{\pgfqpoint{3.125856in}{2.198047in}}{\pgfqpoint{3.131679in}{2.203871in}}%
\pgfpathcurveto{\pgfqpoint{3.137503in}{2.209695in}}{\pgfqpoint{3.140776in}{2.217595in}}{\pgfqpoint{3.140776in}{2.225831in}}%
\pgfpathcurveto{\pgfqpoint{3.140776in}{2.234067in}}{\pgfqpoint{3.137503in}{2.241967in}}{\pgfqpoint{3.131679in}{2.247791in}}%
\pgfpathcurveto{\pgfqpoint{3.125856in}{2.253615in}}{\pgfqpoint{3.117956in}{2.256887in}}{\pgfqpoint{3.109719in}{2.256887in}}%
\pgfpathcurveto{\pgfqpoint{3.101483in}{2.256887in}}{\pgfqpoint{3.093583in}{2.253615in}}{\pgfqpoint{3.087759in}{2.247791in}}%
\pgfpathcurveto{\pgfqpoint{3.081935in}{2.241967in}}{\pgfqpoint{3.078663in}{2.234067in}}{\pgfqpoint{3.078663in}{2.225831in}}%
\pgfpathcurveto{\pgfqpoint{3.078663in}{2.217595in}}{\pgfqpoint{3.081935in}{2.209695in}}{\pgfqpoint{3.087759in}{2.203871in}}%
\pgfpathcurveto{\pgfqpoint{3.093583in}{2.198047in}}{\pgfqpoint{3.101483in}{2.194774in}}{\pgfqpoint{3.109719in}{2.194774in}}%
\pgfpathclose%
\pgfusepath{stroke,fill}%
\end{pgfscope}%
\begin{pgfscope}%
\pgfpathrectangle{\pgfqpoint{0.100000in}{0.212622in}}{\pgfqpoint{3.696000in}{3.696000in}}%
\pgfusepath{clip}%
\pgfsetbuttcap%
\pgfsetroundjoin%
\definecolor{currentfill}{rgb}{0.121569,0.466667,0.705882}%
\pgfsetfillcolor{currentfill}%
\pgfsetfillopacity{0.638182}%
\pgfsetlinewidth{1.003750pt}%
\definecolor{currentstroke}{rgb}{0.121569,0.466667,0.705882}%
\pgfsetstrokecolor{currentstroke}%
\pgfsetstrokeopacity{0.638182}%
\pgfsetdash{}{0pt}%
\pgfpathmoveto{\pgfqpoint{3.108034in}{2.194205in}}%
\pgfpathcurveto{\pgfqpoint{3.116271in}{2.194205in}}{\pgfqpoint{3.124171in}{2.197477in}}{\pgfqpoint{3.129995in}{2.203301in}}%
\pgfpathcurveto{\pgfqpoint{3.135819in}{2.209125in}}{\pgfqpoint{3.139091in}{2.217025in}}{\pgfqpoint{3.139091in}{2.225262in}}%
\pgfpathcurveto{\pgfqpoint{3.139091in}{2.233498in}}{\pgfqpoint{3.135819in}{2.241398in}}{\pgfqpoint{3.129995in}{2.247222in}}%
\pgfpathcurveto{\pgfqpoint{3.124171in}{2.253046in}}{\pgfqpoint{3.116271in}{2.256318in}}{\pgfqpoint{3.108034in}{2.256318in}}%
\pgfpathcurveto{\pgfqpoint{3.099798in}{2.256318in}}{\pgfqpoint{3.091898in}{2.253046in}}{\pgfqpoint{3.086074in}{2.247222in}}%
\pgfpathcurveto{\pgfqpoint{3.080250in}{2.241398in}}{\pgfqpoint{3.076978in}{2.233498in}}{\pgfqpoint{3.076978in}{2.225262in}}%
\pgfpathcurveto{\pgfqpoint{3.076978in}{2.217025in}}{\pgfqpoint{3.080250in}{2.209125in}}{\pgfqpoint{3.086074in}{2.203301in}}%
\pgfpathcurveto{\pgfqpoint{3.091898in}{2.197477in}}{\pgfqpoint{3.099798in}{2.194205in}}{\pgfqpoint{3.108034in}{2.194205in}}%
\pgfpathclose%
\pgfusepath{stroke,fill}%
\end{pgfscope}%
\begin{pgfscope}%
\pgfpathrectangle{\pgfqpoint{0.100000in}{0.212622in}}{\pgfqpoint{3.696000in}{3.696000in}}%
\pgfusepath{clip}%
\pgfsetbuttcap%
\pgfsetroundjoin%
\definecolor{currentfill}{rgb}{0.121569,0.466667,0.705882}%
\pgfsetfillcolor{currentfill}%
\pgfsetfillopacity{0.639393}%
\pgfsetlinewidth{1.003750pt}%
\definecolor{currentstroke}{rgb}{0.121569,0.466667,0.705882}%
\pgfsetstrokecolor{currentstroke}%
\pgfsetstrokeopacity{0.639393}%
\pgfsetdash{}{0pt}%
\pgfpathmoveto{\pgfqpoint{3.105749in}{2.193237in}}%
\pgfpathcurveto{\pgfqpoint{3.113985in}{2.193237in}}{\pgfqpoint{3.121885in}{2.196509in}}{\pgfqpoint{3.127709in}{2.202333in}}%
\pgfpathcurveto{\pgfqpoint{3.133533in}{2.208157in}}{\pgfqpoint{3.136805in}{2.216057in}}{\pgfqpoint{3.136805in}{2.224293in}}%
\pgfpathcurveto{\pgfqpoint{3.136805in}{2.232529in}}{\pgfqpoint{3.133533in}{2.240430in}}{\pgfqpoint{3.127709in}{2.246253in}}%
\pgfpathcurveto{\pgfqpoint{3.121885in}{2.252077in}}{\pgfqpoint{3.113985in}{2.255350in}}{\pgfqpoint{3.105749in}{2.255350in}}%
\pgfpathcurveto{\pgfqpoint{3.097512in}{2.255350in}}{\pgfqpoint{3.089612in}{2.252077in}}{\pgfqpoint{3.083788in}{2.246253in}}%
\pgfpathcurveto{\pgfqpoint{3.077964in}{2.240430in}}{\pgfqpoint{3.074692in}{2.232529in}}{\pgfqpoint{3.074692in}{2.224293in}}%
\pgfpathcurveto{\pgfqpoint{3.074692in}{2.216057in}}{\pgfqpoint{3.077964in}{2.208157in}}{\pgfqpoint{3.083788in}{2.202333in}}%
\pgfpathcurveto{\pgfqpoint{3.089612in}{2.196509in}}{\pgfqpoint{3.097512in}{2.193237in}}{\pgfqpoint{3.105749in}{2.193237in}}%
\pgfpathclose%
\pgfusepath{stroke,fill}%
\end{pgfscope}%
\begin{pgfscope}%
\pgfpathrectangle{\pgfqpoint{0.100000in}{0.212622in}}{\pgfqpoint{3.696000in}{3.696000in}}%
\pgfusepath{clip}%
\pgfsetbuttcap%
\pgfsetroundjoin%
\definecolor{currentfill}{rgb}{0.121569,0.466667,0.705882}%
\pgfsetfillcolor{currentfill}%
\pgfsetfillopacity{0.640091}%
\pgfsetlinewidth{1.003750pt}%
\definecolor{currentstroke}{rgb}{0.121569,0.466667,0.705882}%
\pgfsetstrokecolor{currentstroke}%
\pgfsetstrokeopacity{0.640091}%
\pgfsetdash{}{0pt}%
\pgfpathmoveto{\pgfqpoint{3.104511in}{2.192895in}}%
\pgfpathcurveto{\pgfqpoint{3.112748in}{2.192895in}}{\pgfqpoint{3.120648in}{2.196168in}}{\pgfqpoint{3.126472in}{2.201992in}}%
\pgfpathcurveto{\pgfqpoint{3.132296in}{2.207816in}}{\pgfqpoint{3.135568in}{2.215716in}}{\pgfqpoint{3.135568in}{2.223952in}}%
\pgfpathcurveto{\pgfqpoint{3.135568in}{2.232188in}}{\pgfqpoint{3.132296in}{2.240088in}}{\pgfqpoint{3.126472in}{2.245912in}}%
\pgfpathcurveto{\pgfqpoint{3.120648in}{2.251736in}}{\pgfqpoint{3.112748in}{2.255008in}}{\pgfqpoint{3.104511in}{2.255008in}}%
\pgfpathcurveto{\pgfqpoint{3.096275in}{2.255008in}}{\pgfqpoint{3.088375in}{2.251736in}}{\pgfqpoint{3.082551in}{2.245912in}}%
\pgfpathcurveto{\pgfqpoint{3.076727in}{2.240088in}}{\pgfqpoint{3.073455in}{2.232188in}}{\pgfqpoint{3.073455in}{2.223952in}}%
\pgfpathcurveto{\pgfqpoint{3.073455in}{2.215716in}}{\pgfqpoint{3.076727in}{2.207816in}}{\pgfqpoint{3.082551in}{2.201992in}}%
\pgfpathcurveto{\pgfqpoint{3.088375in}{2.196168in}}{\pgfqpoint{3.096275in}{2.192895in}}{\pgfqpoint{3.104511in}{2.192895in}}%
\pgfpathclose%
\pgfusepath{stroke,fill}%
\end{pgfscope}%
\begin{pgfscope}%
\pgfpathrectangle{\pgfqpoint{0.100000in}{0.212622in}}{\pgfqpoint{3.696000in}{3.696000in}}%
\pgfusepath{clip}%
\pgfsetbuttcap%
\pgfsetroundjoin%
\definecolor{currentfill}{rgb}{0.121569,0.466667,0.705882}%
\pgfsetfillcolor{currentfill}%
\pgfsetfillopacity{0.640458}%
\pgfsetlinewidth{1.003750pt}%
\definecolor{currentstroke}{rgb}{0.121569,0.466667,0.705882}%
\pgfsetstrokecolor{currentstroke}%
\pgfsetstrokeopacity{0.640458}%
\pgfsetdash{}{0pt}%
\pgfpathmoveto{\pgfqpoint{3.103802in}{2.192633in}}%
\pgfpathcurveto{\pgfqpoint{3.112038in}{2.192633in}}{\pgfqpoint{3.119938in}{2.195905in}}{\pgfqpoint{3.125762in}{2.201729in}}%
\pgfpathcurveto{\pgfqpoint{3.131586in}{2.207553in}}{\pgfqpoint{3.134858in}{2.215453in}}{\pgfqpoint{3.134858in}{2.223689in}}%
\pgfpathcurveto{\pgfqpoint{3.134858in}{2.231926in}}{\pgfqpoint{3.131586in}{2.239826in}}{\pgfqpoint{3.125762in}{2.245650in}}%
\pgfpathcurveto{\pgfqpoint{3.119938in}{2.251474in}}{\pgfqpoint{3.112038in}{2.254746in}}{\pgfqpoint{3.103802in}{2.254746in}}%
\pgfpathcurveto{\pgfqpoint{3.095565in}{2.254746in}}{\pgfqpoint{3.087665in}{2.251474in}}{\pgfqpoint{3.081841in}{2.245650in}}%
\pgfpathcurveto{\pgfqpoint{3.076017in}{2.239826in}}{\pgfqpoint{3.072745in}{2.231926in}}{\pgfqpoint{3.072745in}{2.223689in}}%
\pgfpathcurveto{\pgfqpoint{3.072745in}{2.215453in}}{\pgfqpoint{3.076017in}{2.207553in}}{\pgfqpoint{3.081841in}{2.201729in}}%
\pgfpathcurveto{\pgfqpoint{3.087665in}{2.195905in}}{\pgfqpoint{3.095565in}{2.192633in}}{\pgfqpoint{3.103802in}{2.192633in}}%
\pgfpathclose%
\pgfusepath{stroke,fill}%
\end{pgfscope}%
\begin{pgfscope}%
\pgfpathrectangle{\pgfqpoint{0.100000in}{0.212622in}}{\pgfqpoint{3.696000in}{3.696000in}}%
\pgfusepath{clip}%
\pgfsetbuttcap%
\pgfsetroundjoin%
\definecolor{currentfill}{rgb}{0.121569,0.466667,0.705882}%
\pgfsetfillcolor{currentfill}%
\pgfsetfillopacity{0.640669}%
\pgfsetlinewidth{1.003750pt}%
\definecolor{currentstroke}{rgb}{0.121569,0.466667,0.705882}%
\pgfsetstrokecolor{currentstroke}%
\pgfsetstrokeopacity{0.640669}%
\pgfsetdash{}{0pt}%
\pgfpathmoveto{\pgfqpoint{3.103445in}{2.192509in}}%
\pgfpathcurveto{\pgfqpoint{3.111681in}{2.192509in}}{\pgfqpoint{3.119581in}{2.195781in}}{\pgfqpoint{3.125405in}{2.201605in}}%
\pgfpathcurveto{\pgfqpoint{3.131229in}{2.207429in}}{\pgfqpoint{3.134501in}{2.215329in}}{\pgfqpoint{3.134501in}{2.223565in}}%
\pgfpathcurveto{\pgfqpoint{3.134501in}{2.231801in}}{\pgfqpoint{3.131229in}{2.239701in}}{\pgfqpoint{3.125405in}{2.245525in}}%
\pgfpathcurveto{\pgfqpoint{3.119581in}{2.251349in}}{\pgfqpoint{3.111681in}{2.254622in}}{\pgfqpoint{3.103445in}{2.254622in}}%
\pgfpathcurveto{\pgfqpoint{3.095208in}{2.254622in}}{\pgfqpoint{3.087308in}{2.251349in}}{\pgfqpoint{3.081484in}{2.245525in}}%
\pgfpathcurveto{\pgfqpoint{3.075660in}{2.239701in}}{\pgfqpoint{3.072388in}{2.231801in}}{\pgfqpoint{3.072388in}{2.223565in}}%
\pgfpathcurveto{\pgfqpoint{3.072388in}{2.215329in}}{\pgfqpoint{3.075660in}{2.207429in}}{\pgfqpoint{3.081484in}{2.201605in}}%
\pgfpathcurveto{\pgfqpoint{3.087308in}{2.195781in}}{\pgfqpoint{3.095208in}{2.192509in}}{\pgfqpoint{3.103445in}{2.192509in}}%
\pgfpathclose%
\pgfusepath{stroke,fill}%
\end{pgfscope}%
\begin{pgfscope}%
\pgfpathrectangle{\pgfqpoint{0.100000in}{0.212622in}}{\pgfqpoint{3.696000in}{3.696000in}}%
\pgfusepath{clip}%
\pgfsetbuttcap%
\pgfsetroundjoin%
\definecolor{currentfill}{rgb}{0.121569,0.466667,0.705882}%
\pgfsetfillcolor{currentfill}%
\pgfsetfillopacity{0.641400}%
\pgfsetlinewidth{1.003750pt}%
\definecolor{currentstroke}{rgb}{0.121569,0.466667,0.705882}%
\pgfsetstrokecolor{currentstroke}%
\pgfsetstrokeopacity{0.641400}%
\pgfsetdash{}{0pt}%
\pgfpathmoveto{\pgfqpoint{3.102049in}{2.192093in}}%
\pgfpathcurveto{\pgfqpoint{3.110285in}{2.192093in}}{\pgfqpoint{3.118185in}{2.195365in}}{\pgfqpoint{3.124009in}{2.201189in}}%
\pgfpathcurveto{\pgfqpoint{3.129833in}{2.207013in}}{\pgfqpoint{3.133106in}{2.214913in}}{\pgfqpoint{3.133106in}{2.223149in}}%
\pgfpathcurveto{\pgfqpoint{3.133106in}{2.231385in}}{\pgfqpoint{3.129833in}{2.239286in}}{\pgfqpoint{3.124009in}{2.245109in}}%
\pgfpathcurveto{\pgfqpoint{3.118185in}{2.250933in}}{\pgfqpoint{3.110285in}{2.254206in}}{\pgfqpoint{3.102049in}{2.254206in}}%
\pgfpathcurveto{\pgfqpoint{3.093813in}{2.254206in}}{\pgfqpoint{3.085913in}{2.250933in}}{\pgfqpoint{3.080089in}{2.245109in}}%
\pgfpathcurveto{\pgfqpoint{3.074265in}{2.239286in}}{\pgfqpoint{3.070993in}{2.231385in}}{\pgfqpoint{3.070993in}{2.223149in}}%
\pgfpathcurveto{\pgfqpoint{3.070993in}{2.214913in}}{\pgfqpoint{3.074265in}{2.207013in}}{\pgfqpoint{3.080089in}{2.201189in}}%
\pgfpathcurveto{\pgfqpoint{3.085913in}{2.195365in}}{\pgfqpoint{3.093813in}{2.192093in}}{\pgfqpoint{3.102049in}{2.192093in}}%
\pgfpathclose%
\pgfusepath{stroke,fill}%
\end{pgfscope}%
\begin{pgfscope}%
\pgfpathrectangle{\pgfqpoint{0.100000in}{0.212622in}}{\pgfqpoint{3.696000in}{3.696000in}}%
\pgfusepath{clip}%
\pgfsetbuttcap%
\pgfsetroundjoin%
\definecolor{currentfill}{rgb}{0.121569,0.466667,0.705882}%
\pgfsetfillcolor{currentfill}%
\pgfsetfillopacity{0.642385}%
\pgfsetlinewidth{1.003750pt}%
\definecolor{currentstroke}{rgb}{0.121569,0.466667,0.705882}%
\pgfsetstrokecolor{currentstroke}%
\pgfsetstrokeopacity{0.642385}%
\pgfsetdash{}{0pt}%
\pgfpathmoveto{\pgfqpoint{3.100212in}{2.191629in}}%
\pgfpathcurveto{\pgfqpoint{3.108449in}{2.191629in}}{\pgfqpoint{3.116349in}{2.194902in}}{\pgfqpoint{3.122173in}{2.200726in}}%
\pgfpathcurveto{\pgfqpoint{3.127997in}{2.206550in}}{\pgfqpoint{3.131269in}{2.214450in}}{\pgfqpoint{3.131269in}{2.222686in}}%
\pgfpathcurveto{\pgfqpoint{3.131269in}{2.230922in}}{\pgfqpoint{3.127997in}{2.238822in}}{\pgfqpoint{3.122173in}{2.244646in}}%
\pgfpathcurveto{\pgfqpoint{3.116349in}{2.250470in}}{\pgfqpoint{3.108449in}{2.253742in}}{\pgfqpoint{3.100212in}{2.253742in}}%
\pgfpathcurveto{\pgfqpoint{3.091976in}{2.253742in}}{\pgfqpoint{3.084076in}{2.250470in}}{\pgfqpoint{3.078252in}{2.244646in}}%
\pgfpathcurveto{\pgfqpoint{3.072428in}{2.238822in}}{\pgfqpoint{3.069156in}{2.230922in}}{\pgfqpoint{3.069156in}{2.222686in}}%
\pgfpathcurveto{\pgfqpoint{3.069156in}{2.214450in}}{\pgfqpoint{3.072428in}{2.206550in}}{\pgfqpoint{3.078252in}{2.200726in}}%
\pgfpathcurveto{\pgfqpoint{3.084076in}{2.194902in}}{\pgfqpoint{3.091976in}{2.191629in}}{\pgfqpoint{3.100212in}{2.191629in}}%
\pgfpathclose%
\pgfusepath{stroke,fill}%
\end{pgfscope}%
\begin{pgfscope}%
\pgfpathrectangle{\pgfqpoint{0.100000in}{0.212622in}}{\pgfqpoint{3.696000in}{3.696000in}}%
\pgfusepath{clip}%
\pgfsetbuttcap%
\pgfsetroundjoin%
\definecolor{currentfill}{rgb}{0.121569,0.466667,0.705882}%
\pgfsetfillcolor{currentfill}%
\pgfsetfillopacity{0.642963}%
\pgfsetlinewidth{1.003750pt}%
\definecolor{currentstroke}{rgb}{0.121569,0.466667,0.705882}%
\pgfsetstrokecolor{currentstroke}%
\pgfsetstrokeopacity{0.642963}%
\pgfsetdash{}{0pt}%
\pgfpathmoveto{\pgfqpoint{3.099257in}{2.191556in}}%
\pgfpathcurveto{\pgfqpoint{3.107493in}{2.191556in}}{\pgfqpoint{3.115393in}{2.194828in}}{\pgfqpoint{3.121217in}{2.200652in}}%
\pgfpathcurveto{\pgfqpoint{3.127041in}{2.206476in}}{\pgfqpoint{3.130313in}{2.214376in}}{\pgfqpoint{3.130313in}{2.222612in}}%
\pgfpathcurveto{\pgfqpoint{3.130313in}{2.230849in}}{\pgfqpoint{3.127041in}{2.238749in}}{\pgfqpoint{3.121217in}{2.244573in}}%
\pgfpathcurveto{\pgfqpoint{3.115393in}{2.250397in}}{\pgfqpoint{3.107493in}{2.253669in}}{\pgfqpoint{3.099257in}{2.253669in}}%
\pgfpathcurveto{\pgfqpoint{3.091020in}{2.253669in}}{\pgfqpoint{3.083120in}{2.250397in}}{\pgfqpoint{3.077296in}{2.244573in}}%
\pgfpathcurveto{\pgfqpoint{3.071473in}{2.238749in}}{\pgfqpoint{3.068200in}{2.230849in}}{\pgfqpoint{3.068200in}{2.222612in}}%
\pgfpathcurveto{\pgfqpoint{3.068200in}{2.214376in}}{\pgfqpoint{3.071473in}{2.206476in}}{\pgfqpoint{3.077296in}{2.200652in}}%
\pgfpathcurveto{\pgfqpoint{3.083120in}{2.194828in}}{\pgfqpoint{3.091020in}{2.191556in}}{\pgfqpoint{3.099257in}{2.191556in}}%
\pgfpathclose%
\pgfusepath{stroke,fill}%
\end{pgfscope}%
\begin{pgfscope}%
\pgfpathrectangle{\pgfqpoint{0.100000in}{0.212622in}}{\pgfqpoint{3.696000in}{3.696000in}}%
\pgfusepath{clip}%
\pgfsetbuttcap%
\pgfsetroundjoin%
\definecolor{currentfill}{rgb}{0.121569,0.466667,0.705882}%
\pgfsetfillcolor{currentfill}%
\pgfsetfillopacity{0.643261}%
\pgfsetlinewidth{1.003750pt}%
\definecolor{currentstroke}{rgb}{0.121569,0.466667,0.705882}%
\pgfsetstrokecolor{currentstroke}%
\pgfsetstrokeopacity{0.643261}%
\pgfsetdash{}{0pt}%
\pgfpathmoveto{\pgfqpoint{3.098698in}{2.191421in}}%
\pgfpathcurveto{\pgfqpoint{3.106934in}{2.191421in}}{\pgfqpoint{3.114834in}{2.194693in}}{\pgfqpoint{3.120658in}{2.200517in}}%
\pgfpathcurveto{\pgfqpoint{3.126482in}{2.206341in}}{\pgfqpoint{3.129754in}{2.214241in}}{\pgfqpoint{3.129754in}{2.222477in}}%
\pgfpathcurveto{\pgfqpoint{3.129754in}{2.230713in}}{\pgfqpoint{3.126482in}{2.238613in}}{\pgfqpoint{3.120658in}{2.244437in}}%
\pgfpathcurveto{\pgfqpoint{3.114834in}{2.250261in}}{\pgfqpoint{3.106934in}{2.253534in}}{\pgfqpoint{3.098698in}{2.253534in}}%
\pgfpathcurveto{\pgfqpoint{3.090461in}{2.253534in}}{\pgfqpoint{3.082561in}{2.250261in}}{\pgfqpoint{3.076737in}{2.244437in}}%
\pgfpathcurveto{\pgfqpoint{3.070913in}{2.238613in}}{\pgfqpoint{3.067641in}{2.230713in}}{\pgfqpoint{3.067641in}{2.222477in}}%
\pgfpathcurveto{\pgfqpoint{3.067641in}{2.214241in}}{\pgfqpoint{3.070913in}{2.206341in}}{\pgfqpoint{3.076737in}{2.200517in}}%
\pgfpathcurveto{\pgfqpoint{3.082561in}{2.194693in}}{\pgfqpoint{3.090461in}{2.191421in}}{\pgfqpoint{3.098698in}{2.191421in}}%
\pgfpathclose%
\pgfusepath{stroke,fill}%
\end{pgfscope}%
\begin{pgfscope}%
\pgfpathrectangle{\pgfqpoint{0.100000in}{0.212622in}}{\pgfqpoint{3.696000in}{3.696000in}}%
\pgfusepath{clip}%
\pgfsetbuttcap%
\pgfsetroundjoin%
\definecolor{currentfill}{rgb}{0.121569,0.466667,0.705882}%
\pgfsetfillcolor{currentfill}%
\pgfsetfillopacity{0.644078}%
\pgfsetlinewidth{1.003750pt}%
\definecolor{currentstroke}{rgb}{0.121569,0.466667,0.705882}%
\pgfsetstrokecolor{currentstroke}%
\pgfsetstrokeopacity{0.644078}%
\pgfsetdash{}{0pt}%
\pgfpathmoveto{\pgfqpoint{3.097316in}{2.190974in}}%
\pgfpathcurveto{\pgfqpoint{3.105553in}{2.190974in}}{\pgfqpoint{3.113453in}{2.194247in}}{\pgfqpoint{3.119277in}{2.200071in}}%
\pgfpathcurveto{\pgfqpoint{3.125101in}{2.205894in}}{\pgfqpoint{3.128373in}{2.213794in}}{\pgfqpoint{3.128373in}{2.222031in}}%
\pgfpathcurveto{\pgfqpoint{3.128373in}{2.230267in}}{\pgfqpoint{3.125101in}{2.238167in}}{\pgfqpoint{3.119277in}{2.243991in}}%
\pgfpathcurveto{\pgfqpoint{3.113453in}{2.249815in}}{\pgfqpoint{3.105553in}{2.253087in}}{\pgfqpoint{3.097316in}{2.253087in}}%
\pgfpathcurveto{\pgfqpoint{3.089080in}{2.253087in}}{\pgfqpoint{3.081180in}{2.249815in}}{\pgfqpoint{3.075356in}{2.243991in}}%
\pgfpathcurveto{\pgfqpoint{3.069532in}{2.238167in}}{\pgfqpoint{3.066260in}{2.230267in}}{\pgfqpoint{3.066260in}{2.222031in}}%
\pgfpathcurveto{\pgfqpoint{3.066260in}{2.213794in}}{\pgfqpoint{3.069532in}{2.205894in}}{\pgfqpoint{3.075356in}{2.200071in}}%
\pgfpathcurveto{\pgfqpoint{3.081180in}{2.194247in}}{\pgfqpoint{3.089080in}{2.190974in}}{\pgfqpoint{3.097316in}{2.190974in}}%
\pgfpathclose%
\pgfusepath{stroke,fill}%
\end{pgfscope}%
\begin{pgfscope}%
\pgfpathrectangle{\pgfqpoint{0.100000in}{0.212622in}}{\pgfqpoint{3.696000in}{3.696000in}}%
\pgfusepath{clip}%
\pgfsetbuttcap%
\pgfsetroundjoin%
\definecolor{currentfill}{rgb}{0.121569,0.466667,0.705882}%
\pgfsetfillcolor{currentfill}%
\pgfsetfillopacity{0.644544}%
\pgfsetlinewidth{1.003750pt}%
\definecolor{currentstroke}{rgb}{0.121569,0.466667,0.705882}%
\pgfsetstrokecolor{currentstroke}%
\pgfsetstrokeopacity{0.644544}%
\pgfsetdash{}{0pt}%
\pgfpathmoveto{\pgfqpoint{3.096518in}{2.190890in}}%
\pgfpathcurveto{\pgfqpoint{3.104755in}{2.190890in}}{\pgfqpoint{3.112655in}{2.194163in}}{\pgfqpoint{3.118478in}{2.199987in}}%
\pgfpathcurveto{\pgfqpoint{3.124302in}{2.205811in}}{\pgfqpoint{3.127575in}{2.213711in}}{\pgfqpoint{3.127575in}{2.221947in}}%
\pgfpathcurveto{\pgfqpoint{3.127575in}{2.230183in}}{\pgfqpoint{3.124302in}{2.238083in}}{\pgfqpoint{3.118478in}{2.243907in}}%
\pgfpathcurveto{\pgfqpoint{3.112655in}{2.249731in}}{\pgfqpoint{3.104755in}{2.253003in}}{\pgfqpoint{3.096518in}{2.253003in}}%
\pgfpathcurveto{\pgfqpoint{3.088282in}{2.253003in}}{\pgfqpoint{3.080382in}{2.249731in}}{\pgfqpoint{3.074558in}{2.243907in}}%
\pgfpathcurveto{\pgfqpoint{3.068734in}{2.238083in}}{\pgfqpoint{3.065462in}{2.230183in}}{\pgfqpoint{3.065462in}{2.221947in}}%
\pgfpathcurveto{\pgfqpoint{3.065462in}{2.213711in}}{\pgfqpoint{3.068734in}{2.205811in}}{\pgfqpoint{3.074558in}{2.199987in}}%
\pgfpathcurveto{\pgfqpoint{3.080382in}{2.194163in}}{\pgfqpoint{3.088282in}{2.190890in}}{\pgfqpoint{3.096518in}{2.190890in}}%
\pgfpathclose%
\pgfusepath{stroke,fill}%
\end{pgfscope}%
\begin{pgfscope}%
\pgfpathrectangle{\pgfqpoint{0.100000in}{0.212622in}}{\pgfqpoint{3.696000in}{3.696000in}}%
\pgfusepath{clip}%
\pgfsetbuttcap%
\pgfsetroundjoin%
\definecolor{currentfill}{rgb}{0.121569,0.466667,0.705882}%
\pgfsetfillcolor{currentfill}%
\pgfsetfillopacity{0.644791}%
\pgfsetlinewidth{1.003750pt}%
\definecolor{currentstroke}{rgb}{0.121569,0.466667,0.705882}%
\pgfsetstrokecolor{currentstroke}%
\pgfsetstrokeopacity{0.644791}%
\pgfsetdash{}{0pt}%
\pgfpathmoveto{\pgfqpoint{3.096063in}{2.190793in}}%
\pgfpathcurveto{\pgfqpoint{3.104300in}{2.190793in}}{\pgfqpoint{3.112200in}{2.194065in}}{\pgfqpoint{3.118024in}{2.199889in}}%
\pgfpathcurveto{\pgfqpoint{3.123848in}{2.205713in}}{\pgfqpoint{3.127120in}{2.213613in}}{\pgfqpoint{3.127120in}{2.221849in}}%
\pgfpathcurveto{\pgfqpoint{3.127120in}{2.230085in}}{\pgfqpoint{3.123848in}{2.237986in}}{\pgfqpoint{3.118024in}{2.243809in}}%
\pgfpathcurveto{\pgfqpoint{3.112200in}{2.249633in}}{\pgfqpoint{3.104300in}{2.252906in}}{\pgfqpoint{3.096063in}{2.252906in}}%
\pgfpathcurveto{\pgfqpoint{3.087827in}{2.252906in}}{\pgfqpoint{3.079927in}{2.249633in}}{\pgfqpoint{3.074103in}{2.243809in}}%
\pgfpathcurveto{\pgfqpoint{3.068279in}{2.237986in}}{\pgfqpoint{3.065007in}{2.230085in}}{\pgfqpoint{3.065007in}{2.221849in}}%
\pgfpathcurveto{\pgfqpoint{3.065007in}{2.213613in}}{\pgfqpoint{3.068279in}{2.205713in}}{\pgfqpoint{3.074103in}{2.199889in}}%
\pgfpathcurveto{\pgfqpoint{3.079927in}{2.194065in}}{\pgfqpoint{3.087827in}{2.190793in}}{\pgfqpoint{3.096063in}{2.190793in}}%
\pgfpathclose%
\pgfusepath{stroke,fill}%
\end{pgfscope}%
\begin{pgfscope}%
\pgfpathrectangle{\pgfqpoint{0.100000in}{0.212622in}}{\pgfqpoint{3.696000in}{3.696000in}}%
\pgfusepath{clip}%
\pgfsetbuttcap%
\pgfsetroundjoin%
\definecolor{currentfill}{rgb}{0.121569,0.466667,0.705882}%
\pgfsetfillcolor{currentfill}%
\pgfsetfillopacity{0.645178}%
\pgfsetlinewidth{1.003750pt}%
\definecolor{currentstroke}{rgb}{0.121569,0.466667,0.705882}%
\pgfsetstrokecolor{currentstroke}%
\pgfsetstrokeopacity{0.645178}%
\pgfsetdash{}{0pt}%
\pgfpathmoveto{\pgfqpoint{3.095396in}{2.190629in}}%
\pgfpathcurveto{\pgfqpoint{3.103632in}{2.190629in}}{\pgfqpoint{3.111532in}{2.193901in}}{\pgfqpoint{3.117356in}{2.199725in}}%
\pgfpathcurveto{\pgfqpoint{3.123180in}{2.205549in}}{\pgfqpoint{3.126452in}{2.213449in}}{\pgfqpoint{3.126452in}{2.221685in}}%
\pgfpathcurveto{\pgfqpoint{3.126452in}{2.229922in}}{\pgfqpoint{3.123180in}{2.237822in}}{\pgfqpoint{3.117356in}{2.243646in}}%
\pgfpathcurveto{\pgfqpoint{3.111532in}{2.249470in}}{\pgfqpoint{3.103632in}{2.252742in}}{\pgfqpoint{3.095396in}{2.252742in}}%
\pgfpathcurveto{\pgfqpoint{3.087159in}{2.252742in}}{\pgfqpoint{3.079259in}{2.249470in}}{\pgfqpoint{3.073435in}{2.243646in}}%
\pgfpathcurveto{\pgfqpoint{3.067611in}{2.237822in}}{\pgfqpoint{3.064339in}{2.229922in}}{\pgfqpoint{3.064339in}{2.221685in}}%
\pgfpathcurveto{\pgfqpoint{3.064339in}{2.213449in}}{\pgfqpoint{3.067611in}{2.205549in}}{\pgfqpoint{3.073435in}{2.199725in}}%
\pgfpathcurveto{\pgfqpoint{3.079259in}{2.193901in}}{\pgfqpoint{3.087159in}{2.190629in}}{\pgfqpoint{3.095396in}{2.190629in}}%
\pgfpathclose%
\pgfusepath{stroke,fill}%
\end{pgfscope}%
\begin{pgfscope}%
\pgfpathrectangle{\pgfqpoint{0.100000in}{0.212622in}}{\pgfqpoint{3.696000in}{3.696000in}}%
\pgfusepath{clip}%
\pgfsetbuttcap%
\pgfsetroundjoin%
\definecolor{currentfill}{rgb}{0.121569,0.466667,0.705882}%
\pgfsetfillcolor{currentfill}%
\pgfsetfillopacity{0.645928}%
\pgfsetlinewidth{1.003750pt}%
\definecolor{currentstroke}{rgb}{0.121569,0.466667,0.705882}%
\pgfsetstrokecolor{currentstroke}%
\pgfsetstrokeopacity{0.645928}%
\pgfsetdash{}{0pt}%
\pgfpathmoveto{\pgfqpoint{3.094010in}{2.190235in}}%
\pgfpathcurveto{\pgfqpoint{3.102247in}{2.190235in}}{\pgfqpoint{3.110147in}{2.193507in}}{\pgfqpoint{3.115971in}{2.199331in}}%
\pgfpathcurveto{\pgfqpoint{3.121795in}{2.205155in}}{\pgfqpoint{3.125067in}{2.213055in}}{\pgfqpoint{3.125067in}{2.221291in}}%
\pgfpathcurveto{\pgfqpoint{3.125067in}{2.229528in}}{\pgfqpoint{3.121795in}{2.237428in}}{\pgfqpoint{3.115971in}{2.243252in}}%
\pgfpathcurveto{\pgfqpoint{3.110147in}{2.249076in}}{\pgfqpoint{3.102247in}{2.252348in}}{\pgfqpoint{3.094010in}{2.252348in}}%
\pgfpathcurveto{\pgfqpoint{3.085774in}{2.252348in}}{\pgfqpoint{3.077874in}{2.249076in}}{\pgfqpoint{3.072050in}{2.243252in}}%
\pgfpathcurveto{\pgfqpoint{3.066226in}{2.237428in}}{\pgfqpoint{3.062954in}{2.229528in}}{\pgfqpoint{3.062954in}{2.221291in}}%
\pgfpathcurveto{\pgfqpoint{3.062954in}{2.213055in}}{\pgfqpoint{3.066226in}{2.205155in}}{\pgfqpoint{3.072050in}{2.199331in}}%
\pgfpathcurveto{\pgfqpoint{3.077874in}{2.193507in}}{\pgfqpoint{3.085774in}{2.190235in}}{\pgfqpoint{3.094010in}{2.190235in}}%
\pgfpathclose%
\pgfusepath{stroke,fill}%
\end{pgfscope}%
\begin{pgfscope}%
\pgfpathrectangle{\pgfqpoint{0.100000in}{0.212622in}}{\pgfqpoint{3.696000in}{3.696000in}}%
\pgfusepath{clip}%
\pgfsetbuttcap%
\pgfsetroundjoin%
\definecolor{currentfill}{rgb}{0.121569,0.466667,0.705882}%
\pgfsetfillcolor{currentfill}%
\pgfsetfillopacity{0.646966}%
\pgfsetlinewidth{1.003750pt}%
\definecolor{currentstroke}{rgb}{0.121569,0.466667,0.705882}%
\pgfsetstrokecolor{currentstroke}%
\pgfsetstrokeopacity{0.646966}%
\pgfsetdash{}{0pt}%
\pgfpathmoveto{\pgfqpoint{3.092416in}{2.189738in}}%
\pgfpathcurveto{\pgfqpoint{3.100652in}{2.189738in}}{\pgfqpoint{3.108552in}{2.193010in}}{\pgfqpoint{3.114376in}{2.198834in}}%
\pgfpathcurveto{\pgfqpoint{3.120200in}{2.204658in}}{\pgfqpoint{3.123473in}{2.212558in}}{\pgfqpoint{3.123473in}{2.220795in}}%
\pgfpathcurveto{\pgfqpoint{3.123473in}{2.229031in}}{\pgfqpoint{3.120200in}{2.236931in}}{\pgfqpoint{3.114376in}{2.242755in}}%
\pgfpathcurveto{\pgfqpoint{3.108552in}{2.248579in}}{\pgfqpoint{3.100652in}{2.251851in}}{\pgfqpoint{3.092416in}{2.251851in}}%
\pgfpathcurveto{\pgfqpoint{3.084180in}{2.251851in}}{\pgfqpoint{3.076280in}{2.248579in}}{\pgfqpoint{3.070456in}{2.242755in}}%
\pgfpathcurveto{\pgfqpoint{3.064632in}{2.236931in}}{\pgfqpoint{3.061360in}{2.229031in}}{\pgfqpoint{3.061360in}{2.220795in}}%
\pgfpathcurveto{\pgfqpoint{3.061360in}{2.212558in}}{\pgfqpoint{3.064632in}{2.204658in}}{\pgfqpoint{3.070456in}{2.198834in}}%
\pgfpathcurveto{\pgfqpoint{3.076280in}{2.193010in}}{\pgfqpoint{3.084180in}{2.189738in}}{\pgfqpoint{3.092416in}{2.189738in}}%
\pgfpathclose%
\pgfusepath{stroke,fill}%
\end{pgfscope}%
\begin{pgfscope}%
\pgfpathrectangle{\pgfqpoint{0.100000in}{0.212622in}}{\pgfqpoint{3.696000in}{3.696000in}}%
\pgfusepath{clip}%
\pgfsetbuttcap%
\pgfsetroundjoin%
\definecolor{currentfill}{rgb}{0.121569,0.466667,0.705882}%
\pgfsetfillcolor{currentfill}%
\pgfsetfillopacity{0.647548}%
\pgfsetlinewidth{1.003750pt}%
\definecolor{currentstroke}{rgb}{0.121569,0.466667,0.705882}%
\pgfsetstrokecolor{currentstroke}%
\pgfsetstrokeopacity{0.647548}%
\pgfsetdash{}{0pt}%
\pgfpathmoveto{\pgfqpoint{3.091521in}{2.189562in}}%
\pgfpathcurveto{\pgfqpoint{3.099757in}{2.189562in}}{\pgfqpoint{3.107657in}{2.192834in}}{\pgfqpoint{3.113481in}{2.198658in}}%
\pgfpathcurveto{\pgfqpoint{3.119305in}{2.204482in}}{\pgfqpoint{3.122578in}{2.212382in}}{\pgfqpoint{3.122578in}{2.220618in}}%
\pgfpathcurveto{\pgfqpoint{3.122578in}{2.228855in}}{\pgfqpoint{3.119305in}{2.236755in}}{\pgfqpoint{3.113481in}{2.242579in}}%
\pgfpathcurveto{\pgfqpoint{3.107657in}{2.248403in}}{\pgfqpoint{3.099757in}{2.251675in}}{\pgfqpoint{3.091521in}{2.251675in}}%
\pgfpathcurveto{\pgfqpoint{3.083285in}{2.251675in}}{\pgfqpoint{3.075385in}{2.248403in}}{\pgfqpoint{3.069561in}{2.242579in}}%
\pgfpathcurveto{\pgfqpoint{3.063737in}{2.236755in}}{\pgfqpoint{3.060465in}{2.228855in}}{\pgfqpoint{3.060465in}{2.220618in}}%
\pgfpathcurveto{\pgfqpoint{3.060465in}{2.212382in}}{\pgfqpoint{3.063737in}{2.204482in}}{\pgfqpoint{3.069561in}{2.198658in}}%
\pgfpathcurveto{\pgfqpoint{3.075385in}{2.192834in}}{\pgfqpoint{3.083285in}{2.189562in}}{\pgfqpoint{3.091521in}{2.189562in}}%
\pgfpathclose%
\pgfusepath{stroke,fill}%
\end{pgfscope}%
\begin{pgfscope}%
\pgfpathrectangle{\pgfqpoint{0.100000in}{0.212622in}}{\pgfqpoint{3.696000in}{3.696000in}}%
\pgfusepath{clip}%
\pgfsetbuttcap%
\pgfsetroundjoin%
\definecolor{currentfill}{rgb}{0.121569,0.466667,0.705882}%
\pgfsetfillcolor{currentfill}%
\pgfsetfillopacity{0.647857}%
\pgfsetlinewidth{1.003750pt}%
\definecolor{currentstroke}{rgb}{0.121569,0.466667,0.705882}%
\pgfsetstrokecolor{currentstroke}%
\pgfsetstrokeopacity{0.647857}%
\pgfsetdash{}{0pt}%
\pgfpathmoveto{\pgfqpoint{3.090988in}{2.189427in}}%
\pgfpathcurveto{\pgfqpoint{3.099224in}{2.189427in}}{\pgfqpoint{3.107124in}{2.192699in}}{\pgfqpoint{3.112948in}{2.198523in}}%
\pgfpathcurveto{\pgfqpoint{3.118772in}{2.204347in}}{\pgfqpoint{3.122044in}{2.212247in}}{\pgfqpoint{3.122044in}{2.220483in}}%
\pgfpathcurveto{\pgfqpoint{3.122044in}{2.228720in}}{\pgfqpoint{3.118772in}{2.236620in}}{\pgfqpoint{3.112948in}{2.242444in}}%
\pgfpathcurveto{\pgfqpoint{3.107124in}{2.248267in}}{\pgfqpoint{3.099224in}{2.251540in}}{\pgfqpoint{3.090988in}{2.251540in}}%
\pgfpathcurveto{\pgfqpoint{3.082751in}{2.251540in}}{\pgfqpoint{3.074851in}{2.248267in}}{\pgfqpoint{3.069027in}{2.242444in}}%
\pgfpathcurveto{\pgfqpoint{3.063203in}{2.236620in}}{\pgfqpoint{3.059931in}{2.228720in}}{\pgfqpoint{3.059931in}{2.220483in}}%
\pgfpathcurveto{\pgfqpoint{3.059931in}{2.212247in}}{\pgfqpoint{3.063203in}{2.204347in}}{\pgfqpoint{3.069027in}{2.198523in}}%
\pgfpathcurveto{\pgfqpoint{3.074851in}{2.192699in}}{\pgfqpoint{3.082751in}{2.189427in}}{\pgfqpoint{3.090988in}{2.189427in}}%
\pgfpathclose%
\pgfusepath{stroke,fill}%
\end{pgfscope}%
\begin{pgfscope}%
\pgfpathrectangle{\pgfqpoint{0.100000in}{0.212622in}}{\pgfqpoint{3.696000in}{3.696000in}}%
\pgfusepath{clip}%
\pgfsetbuttcap%
\pgfsetroundjoin%
\definecolor{currentfill}{rgb}{0.121569,0.466667,0.705882}%
\pgfsetfillcolor{currentfill}%
\pgfsetfillopacity{0.648024}%
\pgfsetlinewidth{1.003750pt}%
\definecolor{currentstroke}{rgb}{0.121569,0.466667,0.705882}%
\pgfsetstrokecolor{currentstroke}%
\pgfsetstrokeopacity{0.648024}%
\pgfsetdash{}{0pt}%
\pgfpathmoveto{\pgfqpoint{3.090690in}{2.189340in}}%
\pgfpathcurveto{\pgfqpoint{3.098927in}{2.189340in}}{\pgfqpoint{3.106827in}{2.192612in}}{\pgfqpoint{3.112651in}{2.198436in}}%
\pgfpathcurveto{\pgfqpoint{3.118475in}{2.204260in}}{\pgfqpoint{3.121747in}{2.212160in}}{\pgfqpoint{3.121747in}{2.220397in}}%
\pgfpathcurveto{\pgfqpoint{3.121747in}{2.228633in}}{\pgfqpoint{3.118475in}{2.236533in}}{\pgfqpoint{3.112651in}{2.242357in}}%
\pgfpathcurveto{\pgfqpoint{3.106827in}{2.248181in}}{\pgfqpoint{3.098927in}{2.251453in}}{\pgfqpoint{3.090690in}{2.251453in}}%
\pgfpathcurveto{\pgfqpoint{3.082454in}{2.251453in}}{\pgfqpoint{3.074554in}{2.248181in}}{\pgfqpoint{3.068730in}{2.242357in}}%
\pgfpathcurveto{\pgfqpoint{3.062906in}{2.236533in}}{\pgfqpoint{3.059634in}{2.228633in}}{\pgfqpoint{3.059634in}{2.220397in}}%
\pgfpathcurveto{\pgfqpoint{3.059634in}{2.212160in}}{\pgfqpoint{3.062906in}{2.204260in}}{\pgfqpoint{3.068730in}{2.198436in}}%
\pgfpathcurveto{\pgfqpoint{3.074554in}{2.192612in}}{\pgfqpoint{3.082454in}{2.189340in}}{\pgfqpoint{3.090690in}{2.189340in}}%
\pgfpathclose%
\pgfusepath{stroke,fill}%
\end{pgfscope}%
\begin{pgfscope}%
\pgfpathrectangle{\pgfqpoint{0.100000in}{0.212622in}}{\pgfqpoint{3.696000in}{3.696000in}}%
\pgfusepath{clip}%
\pgfsetbuttcap%
\pgfsetroundjoin%
\definecolor{currentfill}{rgb}{0.121569,0.466667,0.705882}%
\pgfsetfillcolor{currentfill}%
\pgfsetfillopacity{0.648371}%
\pgfsetlinewidth{1.003750pt}%
\definecolor{currentstroke}{rgb}{0.121569,0.466667,0.705882}%
\pgfsetstrokecolor{currentstroke}%
\pgfsetstrokeopacity{0.648371}%
\pgfsetdash{}{0pt}%
\pgfpathmoveto{\pgfqpoint{3.089988in}{2.189008in}}%
\pgfpathcurveto{\pgfqpoint{3.098225in}{2.189008in}}{\pgfqpoint{3.106125in}{2.192280in}}{\pgfqpoint{3.111949in}{2.198104in}}%
\pgfpathcurveto{\pgfqpoint{3.117773in}{2.203928in}}{\pgfqpoint{3.121045in}{2.211828in}}{\pgfqpoint{3.121045in}{2.220064in}}%
\pgfpathcurveto{\pgfqpoint{3.121045in}{2.228301in}}{\pgfqpoint{3.117773in}{2.236201in}}{\pgfqpoint{3.111949in}{2.242025in}}%
\pgfpathcurveto{\pgfqpoint{3.106125in}{2.247849in}}{\pgfqpoint{3.098225in}{2.251121in}}{\pgfqpoint{3.089988in}{2.251121in}}%
\pgfpathcurveto{\pgfqpoint{3.081752in}{2.251121in}}{\pgfqpoint{3.073852in}{2.247849in}}{\pgfqpoint{3.068028in}{2.242025in}}%
\pgfpathcurveto{\pgfqpoint{3.062204in}{2.236201in}}{\pgfqpoint{3.058932in}{2.228301in}}{\pgfqpoint{3.058932in}{2.220064in}}%
\pgfpathcurveto{\pgfqpoint{3.058932in}{2.211828in}}{\pgfqpoint{3.062204in}{2.203928in}}{\pgfqpoint{3.068028in}{2.198104in}}%
\pgfpathcurveto{\pgfqpoint{3.073852in}{2.192280in}}{\pgfqpoint{3.081752in}{2.189008in}}{\pgfqpoint{3.089988in}{2.189008in}}%
\pgfpathclose%
\pgfusepath{stroke,fill}%
\end{pgfscope}%
\begin{pgfscope}%
\pgfpathrectangle{\pgfqpoint{0.100000in}{0.212622in}}{\pgfqpoint{3.696000in}{3.696000in}}%
\pgfusepath{clip}%
\pgfsetbuttcap%
\pgfsetroundjoin%
\definecolor{currentfill}{rgb}{0.121569,0.466667,0.705882}%
\pgfsetfillcolor{currentfill}%
\pgfsetfillopacity{0.648568}%
\pgfsetlinewidth{1.003750pt}%
\definecolor{currentstroke}{rgb}{0.121569,0.466667,0.705882}%
\pgfsetstrokecolor{currentstroke}%
\pgfsetstrokeopacity{0.648568}%
\pgfsetdash{}{0pt}%
\pgfpathmoveto{\pgfqpoint{3.089584in}{2.188890in}}%
\pgfpathcurveto{\pgfqpoint{3.097820in}{2.188890in}}{\pgfqpoint{3.105720in}{2.192163in}}{\pgfqpoint{3.111544in}{2.197987in}}%
\pgfpathcurveto{\pgfqpoint{3.117368in}{2.203811in}}{\pgfqpoint{3.120640in}{2.211711in}}{\pgfqpoint{3.120640in}{2.219947in}}%
\pgfpathcurveto{\pgfqpoint{3.120640in}{2.228183in}}{\pgfqpoint{3.117368in}{2.236083in}}{\pgfqpoint{3.111544in}{2.241907in}}%
\pgfpathcurveto{\pgfqpoint{3.105720in}{2.247731in}}{\pgfqpoint{3.097820in}{2.251003in}}{\pgfqpoint{3.089584in}{2.251003in}}%
\pgfpathcurveto{\pgfqpoint{3.081348in}{2.251003in}}{\pgfqpoint{3.073447in}{2.247731in}}{\pgfqpoint{3.067624in}{2.241907in}}%
\pgfpathcurveto{\pgfqpoint{3.061800in}{2.236083in}}{\pgfqpoint{3.058527in}{2.228183in}}{\pgfqpoint{3.058527in}{2.219947in}}%
\pgfpathcurveto{\pgfqpoint{3.058527in}{2.211711in}}{\pgfqpoint{3.061800in}{2.203811in}}{\pgfqpoint{3.067624in}{2.197987in}}%
\pgfpathcurveto{\pgfqpoint{3.073447in}{2.192163in}}{\pgfqpoint{3.081348in}{2.188890in}}{\pgfqpoint{3.089584in}{2.188890in}}%
\pgfpathclose%
\pgfusepath{stroke,fill}%
\end{pgfscope}%
\begin{pgfscope}%
\pgfpathrectangle{\pgfqpoint{0.100000in}{0.212622in}}{\pgfqpoint{3.696000in}{3.696000in}}%
\pgfusepath{clip}%
\pgfsetbuttcap%
\pgfsetroundjoin%
\definecolor{currentfill}{rgb}{0.121569,0.466667,0.705882}%
\pgfsetfillcolor{currentfill}%
\pgfsetfillopacity{0.648683}%
\pgfsetlinewidth{1.003750pt}%
\definecolor{currentstroke}{rgb}{0.121569,0.466667,0.705882}%
\pgfsetstrokecolor{currentstroke}%
\pgfsetstrokeopacity{0.648683}%
\pgfsetdash{}{0pt}%
\pgfpathmoveto{\pgfqpoint{3.089371in}{2.188856in}}%
\pgfpathcurveto{\pgfqpoint{3.097608in}{2.188856in}}{\pgfqpoint{3.105508in}{2.192128in}}{\pgfqpoint{3.111332in}{2.197952in}}%
\pgfpathcurveto{\pgfqpoint{3.117156in}{2.203776in}}{\pgfqpoint{3.120428in}{2.211676in}}{\pgfqpoint{3.120428in}{2.219912in}}%
\pgfpathcurveto{\pgfqpoint{3.120428in}{2.228149in}}{\pgfqpoint{3.117156in}{2.236049in}}{\pgfqpoint{3.111332in}{2.241873in}}%
\pgfpathcurveto{\pgfqpoint{3.105508in}{2.247697in}}{\pgfqpoint{3.097608in}{2.250969in}}{\pgfqpoint{3.089371in}{2.250969in}}%
\pgfpathcurveto{\pgfqpoint{3.081135in}{2.250969in}}{\pgfqpoint{3.073235in}{2.247697in}}{\pgfqpoint{3.067411in}{2.241873in}}%
\pgfpathcurveto{\pgfqpoint{3.061587in}{2.236049in}}{\pgfqpoint{3.058315in}{2.228149in}}{\pgfqpoint{3.058315in}{2.219912in}}%
\pgfpathcurveto{\pgfqpoint{3.058315in}{2.211676in}}{\pgfqpoint{3.061587in}{2.203776in}}{\pgfqpoint{3.067411in}{2.197952in}}%
\pgfpathcurveto{\pgfqpoint{3.073235in}{2.192128in}}{\pgfqpoint{3.081135in}{2.188856in}}{\pgfqpoint{3.089371in}{2.188856in}}%
\pgfpathclose%
\pgfusepath{stroke,fill}%
\end{pgfscope}%
\begin{pgfscope}%
\pgfpathrectangle{\pgfqpoint{0.100000in}{0.212622in}}{\pgfqpoint{3.696000in}{3.696000in}}%
\pgfusepath{clip}%
\pgfsetbuttcap%
\pgfsetroundjoin%
\definecolor{currentfill}{rgb}{0.121569,0.466667,0.705882}%
\pgfsetfillcolor{currentfill}%
\pgfsetfillopacity{0.648745}%
\pgfsetlinewidth{1.003750pt}%
\definecolor{currentstroke}{rgb}{0.121569,0.466667,0.705882}%
\pgfsetstrokecolor{currentstroke}%
\pgfsetstrokeopacity{0.648745}%
\pgfsetdash{}{0pt}%
\pgfpathmoveto{\pgfqpoint{3.089253in}{2.188835in}}%
\pgfpathcurveto{\pgfqpoint{3.097490in}{2.188835in}}{\pgfqpoint{3.105390in}{2.192107in}}{\pgfqpoint{3.111213in}{2.197931in}}%
\pgfpathcurveto{\pgfqpoint{3.117037in}{2.203755in}}{\pgfqpoint{3.120310in}{2.211655in}}{\pgfqpoint{3.120310in}{2.219891in}}%
\pgfpathcurveto{\pgfqpoint{3.120310in}{2.228128in}}{\pgfqpoint{3.117037in}{2.236028in}}{\pgfqpoint{3.111213in}{2.241851in}}%
\pgfpathcurveto{\pgfqpoint{3.105390in}{2.247675in}}{\pgfqpoint{3.097490in}{2.250948in}}{\pgfqpoint{3.089253in}{2.250948in}}%
\pgfpathcurveto{\pgfqpoint{3.081017in}{2.250948in}}{\pgfqpoint{3.073117in}{2.247675in}}{\pgfqpoint{3.067293in}{2.241851in}}%
\pgfpathcurveto{\pgfqpoint{3.061469in}{2.236028in}}{\pgfqpoint{3.058197in}{2.228128in}}{\pgfqpoint{3.058197in}{2.219891in}}%
\pgfpathcurveto{\pgfqpoint{3.058197in}{2.211655in}}{\pgfqpoint{3.061469in}{2.203755in}}{\pgfqpoint{3.067293in}{2.197931in}}%
\pgfpathcurveto{\pgfqpoint{3.073117in}{2.192107in}}{\pgfqpoint{3.081017in}{2.188835in}}{\pgfqpoint{3.089253in}{2.188835in}}%
\pgfpathclose%
\pgfusepath{stroke,fill}%
\end{pgfscope}%
\begin{pgfscope}%
\pgfpathrectangle{\pgfqpoint{0.100000in}{0.212622in}}{\pgfqpoint{3.696000in}{3.696000in}}%
\pgfusepath{clip}%
\pgfsetbuttcap%
\pgfsetroundjoin%
\definecolor{currentfill}{rgb}{0.121569,0.466667,0.705882}%
\pgfsetfillcolor{currentfill}%
\pgfsetfillopacity{0.648779}%
\pgfsetlinewidth{1.003750pt}%
\definecolor{currentstroke}{rgb}{0.121569,0.466667,0.705882}%
\pgfsetstrokecolor{currentstroke}%
\pgfsetstrokeopacity{0.648779}%
\pgfsetdash{}{0pt}%
\pgfpathmoveto{\pgfqpoint{3.089188in}{2.188818in}}%
\pgfpathcurveto{\pgfqpoint{3.097424in}{2.188818in}}{\pgfqpoint{3.105324in}{2.192091in}}{\pgfqpoint{3.111148in}{2.197915in}}%
\pgfpathcurveto{\pgfqpoint{3.116972in}{2.203739in}}{\pgfqpoint{3.120245in}{2.211639in}}{\pgfqpoint{3.120245in}{2.219875in}}%
\pgfpathcurveto{\pgfqpoint{3.120245in}{2.228111in}}{\pgfqpoint{3.116972in}{2.236011in}}{\pgfqpoint{3.111148in}{2.241835in}}%
\pgfpathcurveto{\pgfqpoint{3.105324in}{2.247659in}}{\pgfqpoint{3.097424in}{2.250931in}}{\pgfqpoint{3.089188in}{2.250931in}}%
\pgfpathcurveto{\pgfqpoint{3.080952in}{2.250931in}}{\pgfqpoint{3.073052in}{2.247659in}}{\pgfqpoint{3.067228in}{2.241835in}}%
\pgfpathcurveto{\pgfqpoint{3.061404in}{2.236011in}}{\pgfqpoint{3.058132in}{2.228111in}}{\pgfqpoint{3.058132in}{2.219875in}}%
\pgfpathcurveto{\pgfqpoint{3.058132in}{2.211639in}}{\pgfqpoint{3.061404in}{2.203739in}}{\pgfqpoint{3.067228in}{2.197915in}}%
\pgfpathcurveto{\pgfqpoint{3.073052in}{2.192091in}}{\pgfqpoint{3.080952in}{2.188818in}}{\pgfqpoint{3.089188in}{2.188818in}}%
\pgfpathclose%
\pgfusepath{stroke,fill}%
\end{pgfscope}%
\begin{pgfscope}%
\pgfpathrectangle{\pgfqpoint{0.100000in}{0.212622in}}{\pgfqpoint{3.696000in}{3.696000in}}%
\pgfusepath{clip}%
\pgfsetbuttcap%
\pgfsetroundjoin%
\definecolor{currentfill}{rgb}{0.121569,0.466667,0.705882}%
\pgfsetfillcolor{currentfill}%
\pgfsetfillopacity{0.648798}%
\pgfsetlinewidth{1.003750pt}%
\definecolor{currentstroke}{rgb}{0.121569,0.466667,0.705882}%
\pgfsetstrokecolor{currentstroke}%
\pgfsetstrokeopacity{0.648798}%
\pgfsetdash{}{0pt}%
\pgfpathmoveto{\pgfqpoint{3.089150in}{2.188814in}}%
\pgfpathcurveto{\pgfqpoint{3.097386in}{2.188814in}}{\pgfqpoint{3.105286in}{2.192086in}}{\pgfqpoint{3.111110in}{2.197910in}}%
\pgfpathcurveto{\pgfqpoint{3.116934in}{2.203734in}}{\pgfqpoint{3.120206in}{2.211634in}}{\pgfqpoint{3.120206in}{2.219870in}}%
\pgfpathcurveto{\pgfqpoint{3.120206in}{2.228106in}}{\pgfqpoint{3.116934in}{2.236007in}}{\pgfqpoint{3.111110in}{2.241830in}}%
\pgfpathcurveto{\pgfqpoint{3.105286in}{2.247654in}}{\pgfqpoint{3.097386in}{2.250927in}}{\pgfqpoint{3.089150in}{2.250927in}}%
\pgfpathcurveto{\pgfqpoint{3.080913in}{2.250927in}}{\pgfqpoint{3.073013in}{2.247654in}}{\pgfqpoint{3.067189in}{2.241830in}}%
\pgfpathcurveto{\pgfqpoint{3.061365in}{2.236007in}}{\pgfqpoint{3.058093in}{2.228106in}}{\pgfqpoint{3.058093in}{2.219870in}}%
\pgfpathcurveto{\pgfqpoint{3.058093in}{2.211634in}}{\pgfqpoint{3.061365in}{2.203734in}}{\pgfqpoint{3.067189in}{2.197910in}}%
\pgfpathcurveto{\pgfqpoint{3.073013in}{2.192086in}}{\pgfqpoint{3.080913in}{2.188814in}}{\pgfqpoint{3.089150in}{2.188814in}}%
\pgfpathclose%
\pgfusepath{stroke,fill}%
\end{pgfscope}%
\begin{pgfscope}%
\pgfpathrectangle{\pgfqpoint{0.100000in}{0.212622in}}{\pgfqpoint{3.696000in}{3.696000in}}%
\pgfusepath{clip}%
\pgfsetbuttcap%
\pgfsetroundjoin%
\definecolor{currentfill}{rgb}{0.121569,0.466667,0.705882}%
\pgfsetfillcolor{currentfill}%
\pgfsetfillopacity{0.648808}%
\pgfsetlinewidth{1.003750pt}%
\definecolor{currentstroke}{rgb}{0.121569,0.466667,0.705882}%
\pgfsetstrokecolor{currentstroke}%
\pgfsetstrokeopacity{0.648808}%
\pgfsetdash{}{0pt}%
\pgfpathmoveto{\pgfqpoint{3.089130in}{2.188807in}}%
\pgfpathcurveto{\pgfqpoint{3.097366in}{2.188807in}}{\pgfqpoint{3.105267in}{2.192079in}}{\pgfqpoint{3.111090in}{2.197903in}}%
\pgfpathcurveto{\pgfqpoint{3.116914in}{2.203727in}}{\pgfqpoint{3.120187in}{2.211627in}}{\pgfqpoint{3.120187in}{2.219863in}}%
\pgfpathcurveto{\pgfqpoint{3.120187in}{2.228100in}}{\pgfqpoint{3.116914in}{2.236000in}}{\pgfqpoint{3.111090in}{2.241824in}}%
\pgfpathcurveto{\pgfqpoint{3.105267in}{2.247648in}}{\pgfqpoint{3.097366in}{2.250920in}}{\pgfqpoint{3.089130in}{2.250920in}}%
\pgfpathcurveto{\pgfqpoint{3.080894in}{2.250920in}}{\pgfqpoint{3.072994in}{2.247648in}}{\pgfqpoint{3.067170in}{2.241824in}}%
\pgfpathcurveto{\pgfqpoint{3.061346in}{2.236000in}}{\pgfqpoint{3.058074in}{2.228100in}}{\pgfqpoint{3.058074in}{2.219863in}}%
\pgfpathcurveto{\pgfqpoint{3.058074in}{2.211627in}}{\pgfqpoint{3.061346in}{2.203727in}}{\pgfqpoint{3.067170in}{2.197903in}}%
\pgfpathcurveto{\pgfqpoint{3.072994in}{2.192079in}}{\pgfqpoint{3.080894in}{2.188807in}}{\pgfqpoint{3.089130in}{2.188807in}}%
\pgfpathclose%
\pgfusepath{stroke,fill}%
\end{pgfscope}%
\begin{pgfscope}%
\pgfpathrectangle{\pgfqpoint{0.100000in}{0.212622in}}{\pgfqpoint{3.696000in}{3.696000in}}%
\pgfusepath{clip}%
\pgfsetbuttcap%
\pgfsetroundjoin%
\definecolor{currentfill}{rgb}{0.121569,0.466667,0.705882}%
\pgfsetfillcolor{currentfill}%
\pgfsetfillopacity{0.648814}%
\pgfsetlinewidth{1.003750pt}%
\definecolor{currentstroke}{rgb}{0.121569,0.466667,0.705882}%
\pgfsetstrokecolor{currentstroke}%
\pgfsetstrokeopacity{0.648814}%
\pgfsetdash{}{0pt}%
\pgfpathmoveto{\pgfqpoint{3.089120in}{2.188805in}}%
\pgfpathcurveto{\pgfqpoint{3.097356in}{2.188805in}}{\pgfqpoint{3.105256in}{2.192078in}}{\pgfqpoint{3.111080in}{2.197902in}}%
\pgfpathcurveto{\pgfqpoint{3.116904in}{2.203725in}}{\pgfqpoint{3.120176in}{2.211626in}}{\pgfqpoint{3.120176in}{2.219862in}}%
\pgfpathcurveto{\pgfqpoint{3.120176in}{2.228098in}}{\pgfqpoint{3.116904in}{2.235998in}}{\pgfqpoint{3.111080in}{2.241822in}}%
\pgfpathcurveto{\pgfqpoint{3.105256in}{2.247646in}}{\pgfqpoint{3.097356in}{2.250918in}}{\pgfqpoint{3.089120in}{2.250918in}}%
\pgfpathcurveto{\pgfqpoint{3.080884in}{2.250918in}}{\pgfqpoint{3.072984in}{2.247646in}}{\pgfqpoint{3.067160in}{2.241822in}}%
\pgfpathcurveto{\pgfqpoint{3.061336in}{2.235998in}}{\pgfqpoint{3.058063in}{2.228098in}}{\pgfqpoint{3.058063in}{2.219862in}}%
\pgfpathcurveto{\pgfqpoint{3.058063in}{2.211626in}}{\pgfqpoint{3.061336in}{2.203725in}}{\pgfqpoint{3.067160in}{2.197902in}}%
\pgfpathcurveto{\pgfqpoint{3.072984in}{2.192078in}}{\pgfqpoint{3.080884in}{2.188805in}}{\pgfqpoint{3.089120in}{2.188805in}}%
\pgfpathclose%
\pgfusepath{stroke,fill}%
\end{pgfscope}%
\begin{pgfscope}%
\pgfpathrectangle{\pgfqpoint{0.100000in}{0.212622in}}{\pgfqpoint{3.696000in}{3.696000in}}%
\pgfusepath{clip}%
\pgfsetbuttcap%
\pgfsetroundjoin%
\definecolor{currentfill}{rgb}{0.121569,0.466667,0.705882}%
\pgfsetfillcolor{currentfill}%
\pgfsetfillopacity{0.648817}%
\pgfsetlinewidth{1.003750pt}%
\definecolor{currentstroke}{rgb}{0.121569,0.466667,0.705882}%
\pgfsetstrokecolor{currentstroke}%
\pgfsetstrokeopacity{0.648817}%
\pgfsetdash{}{0pt}%
\pgfpathmoveto{\pgfqpoint{3.089114in}{2.188803in}}%
\pgfpathcurveto{\pgfqpoint{3.097350in}{2.188803in}}{\pgfqpoint{3.105250in}{2.192076in}}{\pgfqpoint{3.111074in}{2.197900in}}%
\pgfpathcurveto{\pgfqpoint{3.116898in}{2.203724in}}{\pgfqpoint{3.120170in}{2.211624in}}{\pgfqpoint{3.120170in}{2.219860in}}%
\pgfpathcurveto{\pgfqpoint{3.120170in}{2.228096in}}{\pgfqpoint{3.116898in}{2.235996in}}{\pgfqpoint{3.111074in}{2.241820in}}%
\pgfpathcurveto{\pgfqpoint{3.105250in}{2.247644in}}{\pgfqpoint{3.097350in}{2.250916in}}{\pgfqpoint{3.089114in}{2.250916in}}%
\pgfpathcurveto{\pgfqpoint{3.080878in}{2.250916in}}{\pgfqpoint{3.072978in}{2.247644in}}{\pgfqpoint{3.067154in}{2.241820in}}%
\pgfpathcurveto{\pgfqpoint{3.061330in}{2.235996in}}{\pgfqpoint{3.058057in}{2.228096in}}{\pgfqpoint{3.058057in}{2.219860in}}%
\pgfpathcurveto{\pgfqpoint{3.058057in}{2.211624in}}{\pgfqpoint{3.061330in}{2.203724in}}{\pgfqpoint{3.067154in}{2.197900in}}%
\pgfpathcurveto{\pgfqpoint{3.072978in}{2.192076in}}{\pgfqpoint{3.080878in}{2.188803in}}{\pgfqpoint{3.089114in}{2.188803in}}%
\pgfpathclose%
\pgfusepath{stroke,fill}%
\end{pgfscope}%
\begin{pgfscope}%
\pgfpathrectangle{\pgfqpoint{0.100000in}{0.212622in}}{\pgfqpoint{3.696000in}{3.696000in}}%
\pgfusepath{clip}%
\pgfsetbuttcap%
\pgfsetroundjoin%
\definecolor{currentfill}{rgb}{0.121569,0.466667,0.705882}%
\pgfsetfillcolor{currentfill}%
\pgfsetfillopacity{0.649092}%
\pgfsetlinewidth{1.003750pt}%
\definecolor{currentstroke}{rgb}{0.121569,0.466667,0.705882}%
\pgfsetstrokecolor{currentstroke}%
\pgfsetstrokeopacity{0.649092}%
\pgfsetdash{}{0pt}%
\pgfpathmoveto{\pgfqpoint{3.088603in}{2.188622in}}%
\pgfpathcurveto{\pgfqpoint{3.096839in}{2.188622in}}{\pgfqpoint{3.104739in}{2.191895in}}{\pgfqpoint{3.110563in}{2.197718in}}%
\pgfpathcurveto{\pgfqpoint{3.116387in}{2.203542in}}{\pgfqpoint{3.119660in}{2.211442in}}{\pgfqpoint{3.119660in}{2.219679in}}%
\pgfpathcurveto{\pgfqpoint{3.119660in}{2.227915in}}{\pgfqpoint{3.116387in}{2.235815in}}{\pgfqpoint{3.110563in}{2.241639in}}%
\pgfpathcurveto{\pgfqpoint{3.104739in}{2.247463in}}{\pgfqpoint{3.096839in}{2.250735in}}{\pgfqpoint{3.088603in}{2.250735in}}%
\pgfpathcurveto{\pgfqpoint{3.080367in}{2.250735in}}{\pgfqpoint{3.072467in}{2.247463in}}{\pgfqpoint{3.066643in}{2.241639in}}%
\pgfpathcurveto{\pgfqpoint{3.060819in}{2.235815in}}{\pgfqpoint{3.057547in}{2.227915in}}{\pgfqpoint{3.057547in}{2.219679in}}%
\pgfpathcurveto{\pgfqpoint{3.057547in}{2.211442in}}{\pgfqpoint{3.060819in}{2.203542in}}{\pgfqpoint{3.066643in}{2.197718in}}%
\pgfpathcurveto{\pgfqpoint{3.072467in}{2.191895in}}{\pgfqpoint{3.080367in}{2.188622in}}{\pgfqpoint{3.088603in}{2.188622in}}%
\pgfpathclose%
\pgfusepath{stroke,fill}%
\end{pgfscope}%
\begin{pgfscope}%
\pgfpathrectangle{\pgfqpoint{0.100000in}{0.212622in}}{\pgfqpoint{3.696000in}{3.696000in}}%
\pgfusepath{clip}%
\pgfsetbuttcap%
\pgfsetroundjoin%
\definecolor{currentfill}{rgb}{0.121569,0.466667,0.705882}%
\pgfsetfillcolor{currentfill}%
\pgfsetfillopacity{0.649251}%
\pgfsetlinewidth{1.003750pt}%
\definecolor{currentstroke}{rgb}{0.121569,0.466667,0.705882}%
\pgfsetstrokecolor{currentstroke}%
\pgfsetstrokeopacity{0.649251}%
\pgfsetdash{}{0pt}%
\pgfpathmoveto{\pgfqpoint{3.088327in}{2.188562in}}%
\pgfpathcurveto{\pgfqpoint{3.096563in}{2.188562in}}{\pgfqpoint{3.104463in}{2.191834in}}{\pgfqpoint{3.110287in}{2.197658in}}%
\pgfpathcurveto{\pgfqpoint{3.116111in}{2.203482in}}{\pgfqpoint{3.119384in}{2.211382in}}{\pgfqpoint{3.119384in}{2.219618in}}%
\pgfpathcurveto{\pgfqpoint{3.119384in}{2.227854in}}{\pgfqpoint{3.116111in}{2.235755in}}{\pgfqpoint{3.110287in}{2.241578in}}%
\pgfpathcurveto{\pgfqpoint{3.104463in}{2.247402in}}{\pgfqpoint{3.096563in}{2.250675in}}{\pgfqpoint{3.088327in}{2.250675in}}%
\pgfpathcurveto{\pgfqpoint{3.080091in}{2.250675in}}{\pgfqpoint{3.072191in}{2.247402in}}{\pgfqpoint{3.066367in}{2.241578in}}%
\pgfpathcurveto{\pgfqpoint{3.060543in}{2.235755in}}{\pgfqpoint{3.057271in}{2.227854in}}{\pgfqpoint{3.057271in}{2.219618in}}%
\pgfpathcurveto{\pgfqpoint{3.057271in}{2.211382in}}{\pgfqpoint{3.060543in}{2.203482in}}{\pgfqpoint{3.066367in}{2.197658in}}%
\pgfpathcurveto{\pgfqpoint{3.072191in}{2.191834in}}{\pgfqpoint{3.080091in}{2.188562in}}{\pgfqpoint{3.088327in}{2.188562in}}%
\pgfpathclose%
\pgfusepath{stroke,fill}%
\end{pgfscope}%
\begin{pgfscope}%
\pgfpathrectangle{\pgfqpoint{0.100000in}{0.212622in}}{\pgfqpoint{3.696000in}{3.696000in}}%
\pgfusepath{clip}%
\pgfsetbuttcap%
\pgfsetroundjoin%
\definecolor{currentfill}{rgb}{0.121569,0.466667,0.705882}%
\pgfsetfillcolor{currentfill}%
\pgfsetfillopacity{0.649333}%
\pgfsetlinewidth{1.003750pt}%
\definecolor{currentstroke}{rgb}{0.121569,0.466667,0.705882}%
\pgfsetstrokecolor{currentstroke}%
\pgfsetstrokeopacity{0.649333}%
\pgfsetdash{}{0pt}%
\pgfpathmoveto{\pgfqpoint{3.088166in}{2.188509in}}%
\pgfpathcurveto{\pgfqpoint{3.096403in}{2.188509in}}{\pgfqpoint{3.104303in}{2.191782in}}{\pgfqpoint{3.110127in}{2.197605in}}%
\pgfpathcurveto{\pgfqpoint{3.115951in}{2.203429in}}{\pgfqpoint{3.119223in}{2.211329in}}{\pgfqpoint{3.119223in}{2.219566in}}%
\pgfpathcurveto{\pgfqpoint{3.119223in}{2.227802in}}{\pgfqpoint{3.115951in}{2.235702in}}{\pgfqpoint{3.110127in}{2.241526in}}%
\pgfpathcurveto{\pgfqpoint{3.104303in}{2.247350in}}{\pgfqpoint{3.096403in}{2.250622in}}{\pgfqpoint{3.088166in}{2.250622in}}%
\pgfpathcurveto{\pgfqpoint{3.079930in}{2.250622in}}{\pgfqpoint{3.072030in}{2.247350in}}{\pgfqpoint{3.066206in}{2.241526in}}%
\pgfpathcurveto{\pgfqpoint{3.060382in}{2.235702in}}{\pgfqpoint{3.057110in}{2.227802in}}{\pgfqpoint{3.057110in}{2.219566in}}%
\pgfpathcurveto{\pgfqpoint{3.057110in}{2.211329in}}{\pgfqpoint{3.060382in}{2.203429in}}{\pgfqpoint{3.066206in}{2.197605in}}%
\pgfpathcurveto{\pgfqpoint{3.072030in}{2.191782in}}{\pgfqpoint{3.079930in}{2.188509in}}{\pgfqpoint{3.088166in}{2.188509in}}%
\pgfpathclose%
\pgfusepath{stroke,fill}%
\end{pgfscope}%
\begin{pgfscope}%
\pgfpathrectangle{\pgfqpoint{0.100000in}{0.212622in}}{\pgfqpoint{3.696000in}{3.696000in}}%
\pgfusepath{clip}%
\pgfsetbuttcap%
\pgfsetroundjoin%
\definecolor{currentfill}{rgb}{0.121569,0.466667,0.705882}%
\pgfsetfillcolor{currentfill}%
\pgfsetfillopacity{0.649379}%
\pgfsetlinewidth{1.003750pt}%
\definecolor{currentstroke}{rgb}{0.121569,0.466667,0.705882}%
\pgfsetstrokecolor{currentstroke}%
\pgfsetstrokeopacity{0.649379}%
\pgfsetdash{}{0pt}%
\pgfpathmoveto{\pgfqpoint{3.088084in}{2.188477in}}%
\pgfpathcurveto{\pgfqpoint{3.096320in}{2.188477in}}{\pgfqpoint{3.104220in}{2.191749in}}{\pgfqpoint{3.110044in}{2.197573in}}%
\pgfpathcurveto{\pgfqpoint{3.115868in}{2.203397in}}{\pgfqpoint{3.119140in}{2.211297in}}{\pgfqpoint{3.119140in}{2.219533in}}%
\pgfpathcurveto{\pgfqpoint{3.119140in}{2.227770in}}{\pgfqpoint{3.115868in}{2.235670in}}{\pgfqpoint{3.110044in}{2.241494in}}%
\pgfpathcurveto{\pgfqpoint{3.104220in}{2.247318in}}{\pgfqpoint{3.096320in}{2.250590in}}{\pgfqpoint{3.088084in}{2.250590in}}%
\pgfpathcurveto{\pgfqpoint{3.079847in}{2.250590in}}{\pgfqpoint{3.071947in}{2.247318in}}{\pgfqpoint{3.066123in}{2.241494in}}%
\pgfpathcurveto{\pgfqpoint{3.060299in}{2.235670in}}{\pgfqpoint{3.057027in}{2.227770in}}{\pgfqpoint{3.057027in}{2.219533in}}%
\pgfpathcurveto{\pgfqpoint{3.057027in}{2.211297in}}{\pgfqpoint{3.060299in}{2.203397in}}{\pgfqpoint{3.066123in}{2.197573in}}%
\pgfpathcurveto{\pgfqpoint{3.071947in}{2.191749in}}{\pgfqpoint{3.079847in}{2.188477in}}{\pgfqpoint{3.088084in}{2.188477in}}%
\pgfpathclose%
\pgfusepath{stroke,fill}%
\end{pgfscope}%
\begin{pgfscope}%
\pgfpathrectangle{\pgfqpoint{0.100000in}{0.212622in}}{\pgfqpoint{3.696000in}{3.696000in}}%
\pgfusepath{clip}%
\pgfsetbuttcap%
\pgfsetroundjoin%
\definecolor{currentfill}{rgb}{0.121569,0.466667,0.705882}%
\pgfsetfillcolor{currentfill}%
\pgfsetfillopacity{0.649757}%
\pgfsetlinewidth{1.003750pt}%
\definecolor{currentstroke}{rgb}{0.121569,0.466667,0.705882}%
\pgfsetstrokecolor{currentstroke}%
\pgfsetstrokeopacity{0.649757}%
\pgfsetdash{}{0pt}%
\pgfpathmoveto{\pgfqpoint{3.087369in}{2.188270in}}%
\pgfpathcurveto{\pgfqpoint{3.095605in}{2.188270in}}{\pgfqpoint{3.103505in}{2.191542in}}{\pgfqpoint{3.109329in}{2.197366in}}%
\pgfpathcurveto{\pgfqpoint{3.115153in}{2.203190in}}{\pgfqpoint{3.118426in}{2.211090in}}{\pgfqpoint{3.118426in}{2.219326in}}%
\pgfpathcurveto{\pgfqpoint{3.118426in}{2.227562in}}{\pgfqpoint{3.115153in}{2.235462in}}{\pgfqpoint{3.109329in}{2.241286in}}%
\pgfpathcurveto{\pgfqpoint{3.103505in}{2.247110in}}{\pgfqpoint{3.095605in}{2.250383in}}{\pgfqpoint{3.087369in}{2.250383in}}%
\pgfpathcurveto{\pgfqpoint{3.079133in}{2.250383in}}{\pgfqpoint{3.071233in}{2.247110in}}{\pgfqpoint{3.065409in}{2.241286in}}%
\pgfpathcurveto{\pgfqpoint{3.059585in}{2.235462in}}{\pgfqpoint{3.056313in}{2.227562in}}{\pgfqpoint{3.056313in}{2.219326in}}%
\pgfpathcurveto{\pgfqpoint{3.056313in}{2.211090in}}{\pgfqpoint{3.059585in}{2.203190in}}{\pgfqpoint{3.065409in}{2.197366in}}%
\pgfpathcurveto{\pgfqpoint{3.071233in}{2.191542in}}{\pgfqpoint{3.079133in}{2.188270in}}{\pgfqpoint{3.087369in}{2.188270in}}%
\pgfpathclose%
\pgfusepath{stroke,fill}%
\end{pgfscope}%
\begin{pgfscope}%
\pgfpathrectangle{\pgfqpoint{0.100000in}{0.212622in}}{\pgfqpoint{3.696000in}{3.696000in}}%
\pgfusepath{clip}%
\pgfsetbuttcap%
\pgfsetroundjoin%
\definecolor{currentfill}{rgb}{0.121569,0.466667,0.705882}%
\pgfsetfillcolor{currentfill}%
\pgfsetfillopacity{0.649961}%
\pgfsetlinewidth{1.003750pt}%
\definecolor{currentstroke}{rgb}{0.121569,0.466667,0.705882}%
\pgfsetstrokecolor{currentstroke}%
\pgfsetstrokeopacity{0.649961}%
\pgfsetdash{}{0pt}%
\pgfpathmoveto{\pgfqpoint{3.086968in}{2.188138in}}%
\pgfpathcurveto{\pgfqpoint{3.095204in}{2.188138in}}{\pgfqpoint{3.103104in}{2.191410in}}{\pgfqpoint{3.108928in}{2.197234in}}%
\pgfpathcurveto{\pgfqpoint{3.114752in}{2.203058in}}{\pgfqpoint{3.118024in}{2.210958in}}{\pgfqpoint{3.118024in}{2.219194in}}%
\pgfpathcurveto{\pgfqpoint{3.118024in}{2.227431in}}{\pgfqpoint{3.114752in}{2.235331in}}{\pgfqpoint{3.108928in}{2.241155in}}%
\pgfpathcurveto{\pgfqpoint{3.103104in}{2.246978in}}{\pgfqpoint{3.095204in}{2.250251in}}{\pgfqpoint{3.086968in}{2.250251in}}%
\pgfpathcurveto{\pgfqpoint{3.078732in}{2.250251in}}{\pgfqpoint{3.070832in}{2.246978in}}{\pgfqpoint{3.065008in}{2.241155in}}%
\pgfpathcurveto{\pgfqpoint{3.059184in}{2.235331in}}{\pgfqpoint{3.055911in}{2.227431in}}{\pgfqpoint{3.055911in}{2.219194in}}%
\pgfpathcurveto{\pgfqpoint{3.055911in}{2.210958in}}{\pgfqpoint{3.059184in}{2.203058in}}{\pgfqpoint{3.065008in}{2.197234in}}%
\pgfpathcurveto{\pgfqpoint{3.070832in}{2.191410in}}{\pgfqpoint{3.078732in}{2.188138in}}{\pgfqpoint{3.086968in}{2.188138in}}%
\pgfpathclose%
\pgfusepath{stroke,fill}%
\end{pgfscope}%
\begin{pgfscope}%
\pgfpathrectangle{\pgfqpoint{0.100000in}{0.212622in}}{\pgfqpoint{3.696000in}{3.696000in}}%
\pgfusepath{clip}%
\pgfsetbuttcap%
\pgfsetroundjoin%
\definecolor{currentfill}{rgb}{0.121569,0.466667,0.705882}%
\pgfsetfillcolor{currentfill}%
\pgfsetfillopacity{0.650075}%
\pgfsetlinewidth{1.003750pt}%
\definecolor{currentstroke}{rgb}{0.121569,0.466667,0.705882}%
\pgfsetstrokecolor{currentstroke}%
\pgfsetstrokeopacity{0.650075}%
\pgfsetdash{}{0pt}%
\pgfpathmoveto{\pgfqpoint{3.086757in}{2.188066in}}%
\pgfpathcurveto{\pgfqpoint{3.094994in}{2.188066in}}{\pgfqpoint{3.102894in}{2.191338in}}{\pgfqpoint{3.108718in}{2.197162in}}%
\pgfpathcurveto{\pgfqpoint{3.114541in}{2.202986in}}{\pgfqpoint{3.117814in}{2.210886in}}{\pgfqpoint{3.117814in}{2.219122in}}%
\pgfpathcurveto{\pgfqpoint{3.117814in}{2.227358in}}{\pgfqpoint{3.114541in}{2.235258in}}{\pgfqpoint{3.108718in}{2.241082in}}%
\pgfpathcurveto{\pgfqpoint{3.102894in}{2.246906in}}{\pgfqpoint{3.094994in}{2.250179in}}{\pgfqpoint{3.086757in}{2.250179in}}%
\pgfpathcurveto{\pgfqpoint{3.078521in}{2.250179in}}{\pgfqpoint{3.070621in}{2.246906in}}{\pgfqpoint{3.064797in}{2.241082in}}%
\pgfpathcurveto{\pgfqpoint{3.058973in}{2.235258in}}{\pgfqpoint{3.055701in}{2.227358in}}{\pgfqpoint{3.055701in}{2.219122in}}%
\pgfpathcurveto{\pgfqpoint{3.055701in}{2.210886in}}{\pgfqpoint{3.058973in}{2.202986in}}{\pgfqpoint{3.064797in}{2.197162in}}%
\pgfpathcurveto{\pgfqpoint{3.070621in}{2.191338in}}{\pgfqpoint{3.078521in}{2.188066in}}{\pgfqpoint{3.086757in}{2.188066in}}%
\pgfpathclose%
\pgfusepath{stroke,fill}%
\end{pgfscope}%
\begin{pgfscope}%
\pgfpathrectangle{\pgfqpoint{0.100000in}{0.212622in}}{\pgfqpoint{3.696000in}{3.696000in}}%
\pgfusepath{clip}%
\pgfsetbuttcap%
\pgfsetroundjoin%
\definecolor{currentfill}{rgb}{0.121569,0.466667,0.705882}%
\pgfsetfillcolor{currentfill}%
\pgfsetfillopacity{0.650379}%
\pgfsetlinewidth{1.003750pt}%
\definecolor{currentstroke}{rgb}{0.121569,0.466667,0.705882}%
\pgfsetstrokecolor{currentstroke}%
\pgfsetstrokeopacity{0.650379}%
\pgfsetdash{}{0pt}%
\pgfpathmoveto{\pgfqpoint{3.086131in}{2.187793in}}%
\pgfpathcurveto{\pgfqpoint{3.094367in}{2.187793in}}{\pgfqpoint{3.102267in}{2.191066in}}{\pgfqpoint{3.108091in}{2.196890in}}%
\pgfpathcurveto{\pgfqpoint{3.113915in}{2.202714in}}{\pgfqpoint{3.117187in}{2.210614in}}{\pgfqpoint{3.117187in}{2.218850in}}%
\pgfpathcurveto{\pgfqpoint{3.117187in}{2.227086in}}{\pgfqpoint{3.113915in}{2.234986in}}{\pgfqpoint{3.108091in}{2.240810in}}%
\pgfpathcurveto{\pgfqpoint{3.102267in}{2.246634in}}{\pgfqpoint{3.094367in}{2.249906in}}{\pgfqpoint{3.086131in}{2.249906in}}%
\pgfpathcurveto{\pgfqpoint{3.077895in}{2.249906in}}{\pgfqpoint{3.069995in}{2.246634in}}{\pgfqpoint{3.064171in}{2.240810in}}%
\pgfpathcurveto{\pgfqpoint{3.058347in}{2.234986in}}{\pgfqpoint{3.055074in}{2.227086in}}{\pgfqpoint{3.055074in}{2.218850in}}%
\pgfpathcurveto{\pgfqpoint{3.055074in}{2.210614in}}{\pgfqpoint{3.058347in}{2.202714in}}{\pgfqpoint{3.064171in}{2.196890in}}%
\pgfpathcurveto{\pgfqpoint{3.069995in}{2.191066in}}{\pgfqpoint{3.077895in}{2.187793in}}{\pgfqpoint{3.086131in}{2.187793in}}%
\pgfpathclose%
\pgfusepath{stroke,fill}%
\end{pgfscope}%
\begin{pgfscope}%
\pgfpathrectangle{\pgfqpoint{0.100000in}{0.212622in}}{\pgfqpoint{3.696000in}{3.696000in}}%
\pgfusepath{clip}%
\pgfsetbuttcap%
\pgfsetroundjoin%
\definecolor{currentfill}{rgb}{0.121569,0.466667,0.705882}%
\pgfsetfillcolor{currentfill}%
\pgfsetfillopacity{0.650973}%
\pgfsetlinewidth{1.003750pt}%
\definecolor{currentstroke}{rgb}{0.121569,0.466667,0.705882}%
\pgfsetstrokecolor{currentstroke}%
\pgfsetstrokeopacity{0.650973}%
\pgfsetdash{}{0pt}%
\pgfpathmoveto{\pgfqpoint{3.085081in}{2.187312in}}%
\pgfpathcurveto{\pgfqpoint{3.093318in}{2.187312in}}{\pgfqpoint{3.101218in}{2.190584in}}{\pgfqpoint{3.107041in}{2.196408in}}%
\pgfpathcurveto{\pgfqpoint{3.112865in}{2.202232in}}{\pgfqpoint{3.116138in}{2.210132in}}{\pgfqpoint{3.116138in}{2.218368in}}%
\pgfpathcurveto{\pgfqpoint{3.116138in}{2.226605in}}{\pgfqpoint{3.112865in}{2.234505in}}{\pgfqpoint{3.107041in}{2.240329in}}%
\pgfpathcurveto{\pgfqpoint{3.101218in}{2.246152in}}{\pgfqpoint{3.093318in}{2.249425in}}{\pgfqpoint{3.085081in}{2.249425in}}%
\pgfpathcurveto{\pgfqpoint{3.076845in}{2.249425in}}{\pgfqpoint{3.068945in}{2.246152in}}{\pgfqpoint{3.063121in}{2.240329in}}%
\pgfpathcurveto{\pgfqpoint{3.057297in}{2.234505in}}{\pgfqpoint{3.054025in}{2.226605in}}{\pgfqpoint{3.054025in}{2.218368in}}%
\pgfpathcurveto{\pgfqpoint{3.054025in}{2.210132in}}{\pgfqpoint{3.057297in}{2.202232in}}{\pgfqpoint{3.063121in}{2.196408in}}%
\pgfpathcurveto{\pgfqpoint{3.068945in}{2.190584in}}{\pgfqpoint{3.076845in}{2.187312in}}{\pgfqpoint{3.085081in}{2.187312in}}%
\pgfpathclose%
\pgfusepath{stroke,fill}%
\end{pgfscope}%
\begin{pgfscope}%
\pgfpathrectangle{\pgfqpoint{0.100000in}{0.212622in}}{\pgfqpoint{3.696000in}{3.696000in}}%
\pgfusepath{clip}%
\pgfsetbuttcap%
\pgfsetroundjoin%
\definecolor{currentfill}{rgb}{0.121569,0.466667,0.705882}%
\pgfsetfillcolor{currentfill}%
\pgfsetfillopacity{0.651319}%
\pgfsetlinewidth{1.003750pt}%
\definecolor{currentstroke}{rgb}{0.121569,0.466667,0.705882}%
\pgfsetstrokecolor{currentstroke}%
\pgfsetstrokeopacity{0.651319}%
\pgfsetdash{}{0pt}%
\pgfpathmoveto{\pgfqpoint{3.084501in}{2.187179in}}%
\pgfpathcurveto{\pgfqpoint{3.092737in}{2.187179in}}{\pgfqpoint{3.100637in}{2.190451in}}{\pgfqpoint{3.106461in}{2.196275in}}%
\pgfpathcurveto{\pgfqpoint{3.112285in}{2.202099in}}{\pgfqpoint{3.115557in}{2.209999in}}{\pgfqpoint{3.115557in}{2.218236in}}%
\pgfpathcurveto{\pgfqpoint{3.115557in}{2.226472in}}{\pgfqpoint{3.112285in}{2.234372in}}{\pgfqpoint{3.106461in}{2.240196in}}%
\pgfpathcurveto{\pgfqpoint{3.100637in}{2.246020in}}{\pgfqpoint{3.092737in}{2.249292in}}{\pgfqpoint{3.084501in}{2.249292in}}%
\pgfpathcurveto{\pgfqpoint{3.076265in}{2.249292in}}{\pgfqpoint{3.068364in}{2.246020in}}{\pgfqpoint{3.062541in}{2.240196in}}%
\pgfpathcurveto{\pgfqpoint{3.056717in}{2.234372in}}{\pgfqpoint{3.053444in}{2.226472in}}{\pgfqpoint{3.053444in}{2.218236in}}%
\pgfpathcurveto{\pgfqpoint{3.053444in}{2.209999in}}{\pgfqpoint{3.056717in}{2.202099in}}{\pgfqpoint{3.062541in}{2.196275in}}%
\pgfpathcurveto{\pgfqpoint{3.068364in}{2.190451in}}{\pgfqpoint{3.076265in}{2.187179in}}{\pgfqpoint{3.084501in}{2.187179in}}%
\pgfpathclose%
\pgfusepath{stroke,fill}%
\end{pgfscope}%
\begin{pgfscope}%
\pgfpathrectangle{\pgfqpoint{0.100000in}{0.212622in}}{\pgfqpoint{3.696000in}{3.696000in}}%
\pgfusepath{clip}%
\pgfsetbuttcap%
\pgfsetroundjoin%
\definecolor{currentfill}{rgb}{0.121569,0.466667,0.705882}%
\pgfsetfillcolor{currentfill}%
\pgfsetfillopacity{0.651501}%
\pgfsetlinewidth{1.003750pt}%
\definecolor{currentstroke}{rgb}{0.121569,0.466667,0.705882}%
\pgfsetstrokecolor{currentstroke}%
\pgfsetstrokeopacity{0.651501}%
\pgfsetdash{}{0pt}%
\pgfpathmoveto{\pgfqpoint{3.084165in}{2.187069in}}%
\pgfpathcurveto{\pgfqpoint{3.092402in}{2.187069in}}{\pgfqpoint{3.100302in}{2.190342in}}{\pgfqpoint{3.106126in}{2.196165in}}%
\pgfpathcurveto{\pgfqpoint{3.111949in}{2.201989in}}{\pgfqpoint{3.115222in}{2.209889in}}{\pgfqpoint{3.115222in}{2.218126in}}%
\pgfpathcurveto{\pgfqpoint{3.115222in}{2.226362in}}{\pgfqpoint{3.111949in}{2.234262in}}{\pgfqpoint{3.106126in}{2.240086in}}%
\pgfpathcurveto{\pgfqpoint{3.100302in}{2.245910in}}{\pgfqpoint{3.092402in}{2.249182in}}{\pgfqpoint{3.084165in}{2.249182in}}%
\pgfpathcurveto{\pgfqpoint{3.075929in}{2.249182in}}{\pgfqpoint{3.068029in}{2.245910in}}{\pgfqpoint{3.062205in}{2.240086in}}%
\pgfpathcurveto{\pgfqpoint{3.056381in}{2.234262in}}{\pgfqpoint{3.053109in}{2.226362in}}{\pgfqpoint{3.053109in}{2.218126in}}%
\pgfpathcurveto{\pgfqpoint{3.053109in}{2.209889in}}{\pgfqpoint{3.056381in}{2.201989in}}{\pgfqpoint{3.062205in}{2.196165in}}%
\pgfpathcurveto{\pgfqpoint{3.068029in}{2.190342in}}{\pgfqpoint{3.075929in}{2.187069in}}{\pgfqpoint{3.084165in}{2.187069in}}%
\pgfpathclose%
\pgfusepath{stroke,fill}%
\end{pgfscope}%
\begin{pgfscope}%
\pgfpathrectangle{\pgfqpoint{0.100000in}{0.212622in}}{\pgfqpoint{3.696000in}{3.696000in}}%
\pgfusepath{clip}%
\pgfsetbuttcap%
\pgfsetroundjoin%
\definecolor{currentfill}{rgb}{0.121569,0.466667,0.705882}%
\pgfsetfillcolor{currentfill}%
\pgfsetfillopacity{0.651602}%
\pgfsetlinewidth{1.003750pt}%
\definecolor{currentstroke}{rgb}{0.121569,0.466667,0.705882}%
\pgfsetstrokecolor{currentstroke}%
\pgfsetstrokeopacity{0.651602}%
\pgfsetdash{}{0pt}%
\pgfpathmoveto{\pgfqpoint{3.083993in}{2.187003in}}%
\pgfpathcurveto{\pgfqpoint{3.092229in}{2.187003in}}{\pgfqpoint{3.100129in}{2.190275in}}{\pgfqpoint{3.105953in}{2.196099in}}%
\pgfpathcurveto{\pgfqpoint{3.111777in}{2.201923in}}{\pgfqpoint{3.115049in}{2.209823in}}{\pgfqpoint{3.115049in}{2.218059in}}%
\pgfpathcurveto{\pgfqpoint{3.115049in}{2.226296in}}{\pgfqpoint{3.111777in}{2.234196in}}{\pgfqpoint{3.105953in}{2.240020in}}%
\pgfpathcurveto{\pgfqpoint{3.100129in}{2.245844in}}{\pgfqpoint{3.092229in}{2.249116in}}{\pgfqpoint{3.083993in}{2.249116in}}%
\pgfpathcurveto{\pgfqpoint{3.075757in}{2.249116in}}{\pgfqpoint{3.067856in}{2.245844in}}{\pgfqpoint{3.062033in}{2.240020in}}%
\pgfpathcurveto{\pgfqpoint{3.056209in}{2.234196in}}{\pgfqpoint{3.052936in}{2.226296in}}{\pgfqpoint{3.052936in}{2.218059in}}%
\pgfpathcurveto{\pgfqpoint{3.052936in}{2.209823in}}{\pgfqpoint{3.056209in}{2.201923in}}{\pgfqpoint{3.062033in}{2.196099in}}%
\pgfpathcurveto{\pgfqpoint{3.067856in}{2.190275in}}{\pgfqpoint{3.075757in}{2.187003in}}{\pgfqpoint{3.083993in}{2.187003in}}%
\pgfpathclose%
\pgfusepath{stroke,fill}%
\end{pgfscope}%
\begin{pgfscope}%
\pgfpathrectangle{\pgfqpoint{0.100000in}{0.212622in}}{\pgfqpoint{3.696000in}{3.696000in}}%
\pgfusepath{clip}%
\pgfsetbuttcap%
\pgfsetroundjoin%
\definecolor{currentfill}{rgb}{0.121569,0.466667,0.705882}%
\pgfsetfillcolor{currentfill}%
\pgfsetfillopacity{0.652013}%
\pgfsetlinewidth{1.003750pt}%
\definecolor{currentstroke}{rgb}{0.121569,0.466667,0.705882}%
\pgfsetstrokecolor{currentstroke}%
\pgfsetstrokeopacity{0.652013}%
\pgfsetdash{}{0pt}%
\pgfpathmoveto{\pgfqpoint{3.083344in}{2.186886in}}%
\pgfpathcurveto{\pgfqpoint{3.091580in}{2.186886in}}{\pgfqpoint{3.099480in}{2.190158in}}{\pgfqpoint{3.105304in}{2.195982in}}%
\pgfpathcurveto{\pgfqpoint{3.111128in}{2.201806in}}{\pgfqpoint{3.114401in}{2.209706in}}{\pgfqpoint{3.114401in}{2.217943in}}%
\pgfpathcurveto{\pgfqpoint{3.114401in}{2.226179in}}{\pgfqpoint{3.111128in}{2.234079in}}{\pgfqpoint{3.105304in}{2.239903in}}%
\pgfpathcurveto{\pgfqpoint{3.099480in}{2.245727in}}{\pgfqpoint{3.091580in}{2.248999in}}{\pgfqpoint{3.083344in}{2.248999in}}%
\pgfpathcurveto{\pgfqpoint{3.075108in}{2.248999in}}{\pgfqpoint{3.067208in}{2.245727in}}{\pgfqpoint{3.061384in}{2.239903in}}%
\pgfpathcurveto{\pgfqpoint{3.055560in}{2.234079in}}{\pgfqpoint{3.052288in}{2.226179in}}{\pgfqpoint{3.052288in}{2.217943in}}%
\pgfpathcurveto{\pgfqpoint{3.052288in}{2.209706in}}{\pgfqpoint{3.055560in}{2.201806in}}{\pgfqpoint{3.061384in}{2.195982in}}%
\pgfpathcurveto{\pgfqpoint{3.067208in}{2.190158in}}{\pgfqpoint{3.075108in}{2.186886in}}{\pgfqpoint{3.083344in}{2.186886in}}%
\pgfpathclose%
\pgfusepath{stroke,fill}%
\end{pgfscope}%
\begin{pgfscope}%
\pgfpathrectangle{\pgfqpoint{0.100000in}{0.212622in}}{\pgfqpoint{3.696000in}{3.696000in}}%
\pgfusepath{clip}%
\pgfsetbuttcap%
\pgfsetroundjoin%
\definecolor{currentfill}{rgb}{0.121569,0.466667,0.705882}%
\pgfsetfillcolor{currentfill}%
\pgfsetfillopacity{0.652229}%
\pgfsetlinewidth{1.003750pt}%
\definecolor{currentstroke}{rgb}{0.121569,0.466667,0.705882}%
\pgfsetstrokecolor{currentstroke}%
\pgfsetstrokeopacity{0.652229}%
\pgfsetdash{}{0pt}%
\pgfpathmoveto{\pgfqpoint{3.082967in}{2.186774in}}%
\pgfpathcurveto{\pgfqpoint{3.091203in}{2.186774in}}{\pgfqpoint{3.099103in}{2.190046in}}{\pgfqpoint{3.104927in}{2.195870in}}%
\pgfpathcurveto{\pgfqpoint{3.110751in}{2.201694in}}{\pgfqpoint{3.114023in}{2.209594in}}{\pgfqpoint{3.114023in}{2.217831in}}%
\pgfpathcurveto{\pgfqpoint{3.114023in}{2.226067in}}{\pgfqpoint{3.110751in}{2.233967in}}{\pgfqpoint{3.104927in}{2.239791in}}%
\pgfpathcurveto{\pgfqpoint{3.099103in}{2.245615in}}{\pgfqpoint{3.091203in}{2.248887in}}{\pgfqpoint{3.082967in}{2.248887in}}%
\pgfpathcurveto{\pgfqpoint{3.074731in}{2.248887in}}{\pgfqpoint{3.066831in}{2.245615in}}{\pgfqpoint{3.061007in}{2.239791in}}%
\pgfpathcurveto{\pgfqpoint{3.055183in}{2.233967in}}{\pgfqpoint{3.051910in}{2.226067in}}{\pgfqpoint{3.051910in}{2.217831in}}%
\pgfpathcurveto{\pgfqpoint{3.051910in}{2.209594in}}{\pgfqpoint{3.055183in}{2.201694in}}{\pgfqpoint{3.061007in}{2.195870in}}%
\pgfpathcurveto{\pgfqpoint{3.066831in}{2.190046in}}{\pgfqpoint{3.074731in}{2.186774in}}{\pgfqpoint{3.082967in}{2.186774in}}%
\pgfpathclose%
\pgfusepath{stroke,fill}%
\end{pgfscope}%
\begin{pgfscope}%
\pgfpathrectangle{\pgfqpoint{0.100000in}{0.212622in}}{\pgfqpoint{3.696000in}{3.696000in}}%
\pgfusepath{clip}%
\pgfsetbuttcap%
\pgfsetroundjoin%
\definecolor{currentfill}{rgb}{0.121569,0.466667,0.705882}%
\pgfsetfillcolor{currentfill}%
\pgfsetfillopacity{0.652349}%
\pgfsetlinewidth{1.003750pt}%
\definecolor{currentstroke}{rgb}{0.121569,0.466667,0.705882}%
\pgfsetstrokecolor{currentstroke}%
\pgfsetstrokeopacity{0.652349}%
\pgfsetdash{}{0pt}%
\pgfpathmoveto{\pgfqpoint{3.082768in}{2.186711in}}%
\pgfpathcurveto{\pgfqpoint{3.091005in}{2.186711in}}{\pgfqpoint{3.098905in}{2.189984in}}{\pgfqpoint{3.104729in}{2.195808in}}%
\pgfpathcurveto{\pgfqpoint{3.110553in}{2.201632in}}{\pgfqpoint{3.113825in}{2.209532in}}{\pgfqpoint{3.113825in}{2.217768in}}%
\pgfpathcurveto{\pgfqpoint{3.113825in}{2.226004in}}{\pgfqpoint{3.110553in}{2.233904in}}{\pgfqpoint{3.104729in}{2.239728in}}%
\pgfpathcurveto{\pgfqpoint{3.098905in}{2.245552in}}{\pgfqpoint{3.091005in}{2.248824in}}{\pgfqpoint{3.082768in}{2.248824in}}%
\pgfpathcurveto{\pgfqpoint{3.074532in}{2.248824in}}{\pgfqpoint{3.066632in}{2.245552in}}{\pgfqpoint{3.060808in}{2.239728in}}%
\pgfpathcurveto{\pgfqpoint{3.054984in}{2.233904in}}{\pgfqpoint{3.051712in}{2.226004in}}{\pgfqpoint{3.051712in}{2.217768in}}%
\pgfpathcurveto{\pgfqpoint{3.051712in}{2.209532in}}{\pgfqpoint{3.054984in}{2.201632in}}{\pgfqpoint{3.060808in}{2.195808in}}%
\pgfpathcurveto{\pgfqpoint{3.066632in}{2.189984in}}{\pgfqpoint{3.074532in}{2.186711in}}{\pgfqpoint{3.082768in}{2.186711in}}%
\pgfpathclose%
\pgfusepath{stroke,fill}%
\end{pgfscope}%
\begin{pgfscope}%
\pgfpathrectangle{\pgfqpoint{0.100000in}{0.212622in}}{\pgfqpoint{3.696000in}{3.696000in}}%
\pgfusepath{clip}%
\pgfsetbuttcap%
\pgfsetroundjoin%
\definecolor{currentfill}{rgb}{0.121569,0.466667,0.705882}%
\pgfsetfillcolor{currentfill}%
\pgfsetfillopacity{0.652411}%
\pgfsetlinewidth{1.003750pt}%
\definecolor{currentstroke}{rgb}{0.121569,0.466667,0.705882}%
\pgfsetstrokecolor{currentstroke}%
\pgfsetstrokeopacity{0.652411}%
\pgfsetdash{}{0pt}%
\pgfpathmoveto{\pgfqpoint{3.082653in}{2.186663in}}%
\pgfpathcurveto{\pgfqpoint{3.090889in}{2.186663in}}{\pgfqpoint{3.098789in}{2.189935in}}{\pgfqpoint{3.104613in}{2.195759in}}%
\pgfpathcurveto{\pgfqpoint{3.110437in}{2.201583in}}{\pgfqpoint{3.113710in}{2.209483in}}{\pgfqpoint{3.113710in}{2.217719in}}%
\pgfpathcurveto{\pgfqpoint{3.113710in}{2.225955in}}{\pgfqpoint{3.110437in}{2.233855in}}{\pgfqpoint{3.104613in}{2.239679in}}%
\pgfpathcurveto{\pgfqpoint{3.098789in}{2.245503in}}{\pgfqpoint{3.090889in}{2.248776in}}{\pgfqpoint{3.082653in}{2.248776in}}%
\pgfpathcurveto{\pgfqpoint{3.074417in}{2.248776in}}{\pgfqpoint{3.066517in}{2.245503in}}{\pgfqpoint{3.060693in}{2.239679in}}%
\pgfpathcurveto{\pgfqpoint{3.054869in}{2.233855in}}{\pgfqpoint{3.051597in}{2.225955in}}{\pgfqpoint{3.051597in}{2.217719in}}%
\pgfpathcurveto{\pgfqpoint{3.051597in}{2.209483in}}{\pgfqpoint{3.054869in}{2.201583in}}{\pgfqpoint{3.060693in}{2.195759in}}%
\pgfpathcurveto{\pgfqpoint{3.066517in}{2.189935in}}{\pgfqpoint{3.074417in}{2.186663in}}{\pgfqpoint{3.082653in}{2.186663in}}%
\pgfpathclose%
\pgfusepath{stroke,fill}%
\end{pgfscope}%
\begin{pgfscope}%
\pgfpathrectangle{\pgfqpoint{0.100000in}{0.212622in}}{\pgfqpoint{3.696000in}{3.696000in}}%
\pgfusepath{clip}%
\pgfsetbuttcap%
\pgfsetroundjoin%
\definecolor{currentfill}{rgb}{0.121569,0.466667,0.705882}%
\pgfsetfillcolor{currentfill}%
\pgfsetfillopacity{0.652914}%
\pgfsetlinewidth{1.003750pt}%
\definecolor{currentstroke}{rgb}{0.121569,0.466667,0.705882}%
\pgfsetstrokecolor{currentstroke}%
\pgfsetstrokeopacity{0.652914}%
\pgfsetdash{}{0pt}%
\pgfpathmoveto{\pgfqpoint{3.081763in}{2.186566in}}%
\pgfpathcurveto{\pgfqpoint{3.090000in}{2.186566in}}{\pgfqpoint{3.097900in}{2.189839in}}{\pgfqpoint{3.103724in}{2.195663in}}%
\pgfpathcurveto{\pgfqpoint{3.109548in}{2.201487in}}{\pgfqpoint{3.112820in}{2.209387in}}{\pgfqpoint{3.112820in}{2.217623in}}%
\pgfpathcurveto{\pgfqpoint{3.112820in}{2.225859in}}{\pgfqpoint{3.109548in}{2.233759in}}{\pgfqpoint{3.103724in}{2.239583in}}%
\pgfpathcurveto{\pgfqpoint{3.097900in}{2.245407in}}{\pgfqpoint{3.090000in}{2.248679in}}{\pgfqpoint{3.081763in}{2.248679in}}%
\pgfpathcurveto{\pgfqpoint{3.073527in}{2.248679in}}{\pgfqpoint{3.065627in}{2.245407in}}{\pgfqpoint{3.059803in}{2.239583in}}%
\pgfpathcurveto{\pgfqpoint{3.053979in}{2.233759in}}{\pgfqpoint{3.050707in}{2.225859in}}{\pgfqpoint{3.050707in}{2.217623in}}%
\pgfpathcurveto{\pgfqpoint{3.050707in}{2.209387in}}{\pgfqpoint{3.053979in}{2.201487in}}{\pgfqpoint{3.059803in}{2.195663in}}%
\pgfpathcurveto{\pgfqpoint{3.065627in}{2.189839in}}{\pgfqpoint{3.073527in}{2.186566in}}{\pgfqpoint{3.081763in}{2.186566in}}%
\pgfpathclose%
\pgfusepath{stroke,fill}%
\end{pgfscope}%
\begin{pgfscope}%
\pgfpathrectangle{\pgfqpoint{0.100000in}{0.212622in}}{\pgfqpoint{3.696000in}{3.696000in}}%
\pgfusepath{clip}%
\pgfsetbuttcap%
\pgfsetroundjoin%
\definecolor{currentfill}{rgb}{0.121569,0.466667,0.705882}%
\pgfsetfillcolor{currentfill}%
\pgfsetfillopacity{0.653199}%
\pgfsetlinewidth{1.003750pt}%
\definecolor{currentstroke}{rgb}{0.121569,0.466667,0.705882}%
\pgfsetstrokecolor{currentstroke}%
\pgfsetstrokeopacity{0.653199}%
\pgfsetdash{}{0pt}%
\pgfpathmoveto{\pgfqpoint{3.081300in}{2.186547in}}%
\pgfpathcurveto{\pgfqpoint{3.089536in}{2.186547in}}{\pgfqpoint{3.097436in}{2.189820in}}{\pgfqpoint{3.103260in}{2.195643in}}%
\pgfpathcurveto{\pgfqpoint{3.109084in}{2.201467in}}{\pgfqpoint{3.112356in}{2.209367in}}{\pgfqpoint{3.112356in}{2.217604in}}%
\pgfpathcurveto{\pgfqpoint{3.112356in}{2.225840in}}{\pgfqpoint{3.109084in}{2.233740in}}{\pgfqpoint{3.103260in}{2.239564in}}%
\pgfpathcurveto{\pgfqpoint{3.097436in}{2.245388in}}{\pgfqpoint{3.089536in}{2.248660in}}{\pgfqpoint{3.081300in}{2.248660in}}%
\pgfpathcurveto{\pgfqpoint{3.073064in}{2.248660in}}{\pgfqpoint{3.065163in}{2.245388in}}{\pgfqpoint{3.059340in}{2.239564in}}%
\pgfpathcurveto{\pgfqpoint{3.053516in}{2.233740in}}{\pgfqpoint{3.050243in}{2.225840in}}{\pgfqpoint{3.050243in}{2.217604in}}%
\pgfpathcurveto{\pgfqpoint{3.050243in}{2.209367in}}{\pgfqpoint{3.053516in}{2.201467in}}{\pgfqpoint{3.059340in}{2.195643in}}%
\pgfpathcurveto{\pgfqpoint{3.065163in}{2.189820in}}{\pgfqpoint{3.073064in}{2.186547in}}{\pgfqpoint{3.081300in}{2.186547in}}%
\pgfpathclose%
\pgfusepath{stroke,fill}%
\end{pgfscope}%
\begin{pgfscope}%
\pgfpathrectangle{\pgfqpoint{0.100000in}{0.212622in}}{\pgfqpoint{3.696000in}{3.696000in}}%
\pgfusepath{clip}%
\pgfsetbuttcap%
\pgfsetroundjoin%
\definecolor{currentfill}{rgb}{0.121569,0.466667,0.705882}%
\pgfsetfillcolor{currentfill}%
\pgfsetfillopacity{0.653350}%
\pgfsetlinewidth{1.003750pt}%
\definecolor{currentstroke}{rgb}{0.121569,0.466667,0.705882}%
\pgfsetstrokecolor{currentstroke}%
\pgfsetstrokeopacity{0.653350}%
\pgfsetdash{}{0pt}%
\pgfpathmoveto{\pgfqpoint{3.081032in}{2.186508in}}%
\pgfpathcurveto{\pgfqpoint{3.089268in}{2.186508in}}{\pgfqpoint{3.097168in}{2.189781in}}{\pgfqpoint{3.102992in}{2.195604in}}%
\pgfpathcurveto{\pgfqpoint{3.108816in}{2.201428in}}{\pgfqpoint{3.112088in}{2.209328in}}{\pgfqpoint{3.112088in}{2.217565in}}%
\pgfpathcurveto{\pgfqpoint{3.112088in}{2.225801in}}{\pgfqpoint{3.108816in}{2.233701in}}{\pgfqpoint{3.102992in}{2.239525in}}%
\pgfpathcurveto{\pgfqpoint{3.097168in}{2.245349in}}{\pgfqpoint{3.089268in}{2.248621in}}{\pgfqpoint{3.081032in}{2.248621in}}%
\pgfpathcurveto{\pgfqpoint{3.072796in}{2.248621in}}{\pgfqpoint{3.064896in}{2.245349in}}{\pgfqpoint{3.059072in}{2.239525in}}%
\pgfpathcurveto{\pgfqpoint{3.053248in}{2.233701in}}{\pgfqpoint{3.049975in}{2.225801in}}{\pgfqpoint{3.049975in}{2.217565in}}%
\pgfpathcurveto{\pgfqpoint{3.049975in}{2.209328in}}{\pgfqpoint{3.053248in}{2.201428in}}{\pgfqpoint{3.059072in}{2.195604in}}%
\pgfpathcurveto{\pgfqpoint{3.064896in}{2.189781in}}{\pgfqpoint{3.072796in}{2.186508in}}{\pgfqpoint{3.081032in}{2.186508in}}%
\pgfpathclose%
\pgfusepath{stroke,fill}%
\end{pgfscope}%
\begin{pgfscope}%
\pgfpathrectangle{\pgfqpoint{0.100000in}{0.212622in}}{\pgfqpoint{3.696000in}{3.696000in}}%
\pgfusepath{clip}%
\pgfsetbuttcap%
\pgfsetroundjoin%
\definecolor{currentfill}{rgb}{0.121569,0.466667,0.705882}%
\pgfsetfillcolor{currentfill}%
\pgfsetfillopacity{0.653842}%
\pgfsetlinewidth{1.003750pt}%
\definecolor{currentstroke}{rgb}{0.121569,0.466667,0.705882}%
\pgfsetstrokecolor{currentstroke}%
\pgfsetstrokeopacity{0.653842}%
\pgfsetdash{}{0pt}%
\pgfpathmoveto{\pgfqpoint{3.080230in}{2.186278in}}%
\pgfpathcurveto{\pgfqpoint{3.088466in}{2.186278in}}{\pgfqpoint{3.096367in}{2.189551in}}{\pgfqpoint{3.102190in}{2.195375in}}%
\pgfpathcurveto{\pgfqpoint{3.108014in}{2.201199in}}{\pgfqpoint{3.111287in}{2.209099in}}{\pgfqpoint{3.111287in}{2.217335in}}%
\pgfpathcurveto{\pgfqpoint{3.111287in}{2.225571in}}{\pgfqpoint{3.108014in}{2.233471in}}{\pgfqpoint{3.102190in}{2.239295in}}%
\pgfpathcurveto{\pgfqpoint{3.096367in}{2.245119in}}{\pgfqpoint{3.088466in}{2.248391in}}{\pgfqpoint{3.080230in}{2.248391in}}%
\pgfpathcurveto{\pgfqpoint{3.071994in}{2.248391in}}{\pgfqpoint{3.064094in}{2.245119in}}{\pgfqpoint{3.058270in}{2.239295in}}%
\pgfpathcurveto{\pgfqpoint{3.052446in}{2.233471in}}{\pgfqpoint{3.049174in}{2.225571in}}{\pgfqpoint{3.049174in}{2.217335in}}%
\pgfpathcurveto{\pgfqpoint{3.049174in}{2.209099in}}{\pgfqpoint{3.052446in}{2.201199in}}{\pgfqpoint{3.058270in}{2.195375in}}%
\pgfpathcurveto{\pgfqpoint{3.064094in}{2.189551in}}{\pgfqpoint{3.071994in}{2.186278in}}{\pgfqpoint{3.080230in}{2.186278in}}%
\pgfpathclose%
\pgfusepath{stroke,fill}%
\end{pgfscope}%
\begin{pgfscope}%
\pgfpathrectangle{\pgfqpoint{0.100000in}{0.212622in}}{\pgfqpoint{3.696000in}{3.696000in}}%
\pgfusepath{clip}%
\pgfsetbuttcap%
\pgfsetroundjoin%
\definecolor{currentfill}{rgb}{0.121569,0.466667,0.705882}%
\pgfsetfillcolor{currentfill}%
\pgfsetfillopacity{0.654583}%
\pgfsetlinewidth{1.003750pt}%
\definecolor{currentstroke}{rgb}{0.121569,0.466667,0.705882}%
\pgfsetstrokecolor{currentstroke}%
\pgfsetstrokeopacity{0.654583}%
\pgfsetdash{}{0pt}%
\pgfpathmoveto{\pgfqpoint{3.079107in}{2.186256in}}%
\pgfpathcurveto{\pgfqpoint{3.087343in}{2.186256in}}{\pgfqpoint{3.095243in}{2.189528in}}{\pgfqpoint{3.101067in}{2.195352in}}%
\pgfpathcurveto{\pgfqpoint{3.106891in}{2.201176in}}{\pgfqpoint{3.110163in}{2.209076in}}{\pgfqpoint{3.110163in}{2.217313in}}%
\pgfpathcurveto{\pgfqpoint{3.110163in}{2.225549in}}{\pgfqpoint{3.106891in}{2.233449in}}{\pgfqpoint{3.101067in}{2.239273in}}%
\pgfpathcurveto{\pgfqpoint{3.095243in}{2.245097in}}{\pgfqpoint{3.087343in}{2.248369in}}{\pgfqpoint{3.079107in}{2.248369in}}%
\pgfpathcurveto{\pgfqpoint{3.070870in}{2.248369in}}{\pgfqpoint{3.062970in}{2.245097in}}{\pgfqpoint{3.057146in}{2.239273in}}%
\pgfpathcurveto{\pgfqpoint{3.051322in}{2.233449in}}{\pgfqpoint{3.048050in}{2.225549in}}{\pgfqpoint{3.048050in}{2.217313in}}%
\pgfpathcurveto{\pgfqpoint{3.048050in}{2.209076in}}{\pgfqpoint{3.051322in}{2.201176in}}{\pgfqpoint{3.057146in}{2.195352in}}%
\pgfpathcurveto{\pgfqpoint{3.062970in}{2.189528in}}{\pgfqpoint{3.070870in}{2.186256in}}{\pgfqpoint{3.079107in}{2.186256in}}%
\pgfpathclose%
\pgfusepath{stroke,fill}%
\end{pgfscope}%
\begin{pgfscope}%
\pgfpathrectangle{\pgfqpoint{0.100000in}{0.212622in}}{\pgfqpoint{3.696000in}{3.696000in}}%
\pgfusepath{clip}%
\pgfsetbuttcap%
\pgfsetroundjoin%
\definecolor{currentfill}{rgb}{0.121569,0.466667,0.705882}%
\pgfsetfillcolor{currentfill}%
\pgfsetfillopacity{0.654969}%
\pgfsetlinewidth{1.003750pt}%
\definecolor{currentstroke}{rgb}{0.121569,0.466667,0.705882}%
\pgfsetstrokecolor{currentstroke}%
\pgfsetstrokeopacity{0.654969}%
\pgfsetdash{}{0pt}%
\pgfpathmoveto{\pgfqpoint{3.078455in}{2.186133in}}%
\pgfpathcurveto{\pgfqpoint{3.086691in}{2.186133in}}{\pgfqpoint{3.094592in}{2.189406in}}{\pgfqpoint{3.100415in}{2.195229in}}%
\pgfpathcurveto{\pgfqpoint{3.106239in}{2.201053in}}{\pgfqpoint{3.109512in}{2.208953in}}{\pgfqpoint{3.109512in}{2.217190in}}%
\pgfpathcurveto{\pgfqpoint{3.109512in}{2.225426in}}{\pgfqpoint{3.106239in}{2.233326in}}{\pgfqpoint{3.100415in}{2.239150in}}%
\pgfpathcurveto{\pgfqpoint{3.094592in}{2.244974in}}{\pgfqpoint{3.086691in}{2.248246in}}{\pgfqpoint{3.078455in}{2.248246in}}%
\pgfpathcurveto{\pgfqpoint{3.070219in}{2.248246in}}{\pgfqpoint{3.062319in}{2.244974in}}{\pgfqpoint{3.056495in}{2.239150in}}%
\pgfpathcurveto{\pgfqpoint{3.050671in}{2.233326in}}{\pgfqpoint{3.047399in}{2.225426in}}{\pgfqpoint{3.047399in}{2.217190in}}%
\pgfpathcurveto{\pgfqpoint{3.047399in}{2.208953in}}{\pgfqpoint{3.050671in}{2.201053in}}{\pgfqpoint{3.056495in}{2.195229in}}%
\pgfpathcurveto{\pgfqpoint{3.062319in}{2.189406in}}{\pgfqpoint{3.070219in}{2.186133in}}{\pgfqpoint{3.078455in}{2.186133in}}%
\pgfpathclose%
\pgfusepath{stroke,fill}%
\end{pgfscope}%
\begin{pgfscope}%
\pgfpathrectangle{\pgfqpoint{0.100000in}{0.212622in}}{\pgfqpoint{3.696000in}{3.696000in}}%
\pgfusepath{clip}%
\pgfsetbuttcap%
\pgfsetroundjoin%
\definecolor{currentfill}{rgb}{0.121569,0.466667,0.705882}%
\pgfsetfillcolor{currentfill}%
\pgfsetfillopacity{0.655177}%
\pgfsetlinewidth{1.003750pt}%
\definecolor{currentstroke}{rgb}{0.121569,0.466667,0.705882}%
\pgfsetstrokecolor{currentstroke}%
\pgfsetstrokeopacity{0.655177}%
\pgfsetdash{}{0pt}%
\pgfpathmoveto{\pgfqpoint{3.078119in}{2.186011in}}%
\pgfpathcurveto{\pgfqpoint{3.086356in}{2.186011in}}{\pgfqpoint{3.094256in}{2.189283in}}{\pgfqpoint{3.100080in}{2.195107in}}%
\pgfpathcurveto{\pgfqpoint{3.105904in}{2.200931in}}{\pgfqpoint{3.109176in}{2.208831in}}{\pgfqpoint{3.109176in}{2.217067in}}%
\pgfpathcurveto{\pgfqpoint{3.109176in}{2.225304in}}{\pgfqpoint{3.105904in}{2.233204in}}{\pgfqpoint{3.100080in}{2.239028in}}%
\pgfpathcurveto{\pgfqpoint{3.094256in}{2.244852in}}{\pgfqpoint{3.086356in}{2.248124in}}{\pgfqpoint{3.078119in}{2.248124in}}%
\pgfpathcurveto{\pgfqpoint{3.069883in}{2.248124in}}{\pgfqpoint{3.061983in}{2.244852in}}{\pgfqpoint{3.056159in}{2.239028in}}%
\pgfpathcurveto{\pgfqpoint{3.050335in}{2.233204in}}{\pgfqpoint{3.047063in}{2.225304in}}{\pgfqpoint{3.047063in}{2.217067in}}%
\pgfpathcurveto{\pgfqpoint{3.047063in}{2.208831in}}{\pgfqpoint{3.050335in}{2.200931in}}{\pgfqpoint{3.056159in}{2.195107in}}%
\pgfpathcurveto{\pgfqpoint{3.061983in}{2.189283in}}{\pgfqpoint{3.069883in}{2.186011in}}{\pgfqpoint{3.078119in}{2.186011in}}%
\pgfpathclose%
\pgfusepath{stroke,fill}%
\end{pgfscope}%
\begin{pgfscope}%
\pgfpathrectangle{\pgfqpoint{0.100000in}{0.212622in}}{\pgfqpoint{3.696000in}{3.696000in}}%
\pgfusepath{clip}%
\pgfsetbuttcap%
\pgfsetroundjoin%
\definecolor{currentfill}{rgb}{0.121569,0.466667,0.705882}%
\pgfsetfillcolor{currentfill}%
\pgfsetfillopacity{0.655649}%
\pgfsetlinewidth{1.003750pt}%
\definecolor{currentstroke}{rgb}{0.121569,0.466667,0.705882}%
\pgfsetstrokecolor{currentstroke}%
\pgfsetstrokeopacity{0.655649}%
\pgfsetdash{}{0pt}%
\pgfpathmoveto{\pgfqpoint{3.077256in}{2.185811in}}%
\pgfpathcurveto{\pgfqpoint{3.085492in}{2.185811in}}{\pgfqpoint{3.093392in}{2.189084in}}{\pgfqpoint{3.099216in}{2.194908in}}%
\pgfpathcurveto{\pgfqpoint{3.105040in}{2.200731in}}{\pgfqpoint{3.108312in}{2.208632in}}{\pgfqpoint{3.108312in}{2.216868in}}%
\pgfpathcurveto{\pgfqpoint{3.108312in}{2.225104in}}{\pgfqpoint{3.105040in}{2.233004in}}{\pgfqpoint{3.099216in}{2.238828in}}%
\pgfpathcurveto{\pgfqpoint{3.093392in}{2.244652in}}{\pgfqpoint{3.085492in}{2.247924in}}{\pgfqpoint{3.077256in}{2.247924in}}%
\pgfpathcurveto{\pgfqpoint{3.069019in}{2.247924in}}{\pgfqpoint{3.061119in}{2.244652in}}{\pgfqpoint{3.055295in}{2.238828in}}%
\pgfpathcurveto{\pgfqpoint{3.049471in}{2.233004in}}{\pgfqpoint{3.046199in}{2.225104in}}{\pgfqpoint{3.046199in}{2.216868in}}%
\pgfpathcurveto{\pgfqpoint{3.046199in}{2.208632in}}{\pgfqpoint{3.049471in}{2.200731in}}{\pgfqpoint{3.055295in}{2.194908in}}%
\pgfpathcurveto{\pgfqpoint{3.061119in}{2.189084in}}{\pgfqpoint{3.069019in}{2.185811in}}{\pgfqpoint{3.077256in}{2.185811in}}%
\pgfpathclose%
\pgfusepath{stroke,fill}%
\end{pgfscope}%
\begin{pgfscope}%
\pgfpathrectangle{\pgfqpoint{0.100000in}{0.212622in}}{\pgfqpoint{3.696000in}{3.696000in}}%
\pgfusepath{clip}%
\pgfsetbuttcap%
\pgfsetroundjoin%
\definecolor{currentfill}{rgb}{0.121569,0.466667,0.705882}%
\pgfsetfillcolor{currentfill}%
\pgfsetfillopacity{0.655903}%
\pgfsetlinewidth{1.003750pt}%
\definecolor{currentstroke}{rgb}{0.121569,0.466667,0.705882}%
\pgfsetstrokecolor{currentstroke}%
\pgfsetstrokeopacity{0.655903}%
\pgfsetdash{}{0pt}%
\pgfpathmoveto{\pgfqpoint{3.076782in}{2.185663in}}%
\pgfpathcurveto{\pgfqpoint{3.085018in}{2.185663in}}{\pgfqpoint{3.092918in}{2.188935in}}{\pgfqpoint{3.098742in}{2.194759in}}%
\pgfpathcurveto{\pgfqpoint{3.104566in}{2.200583in}}{\pgfqpoint{3.107838in}{2.208483in}}{\pgfqpoint{3.107838in}{2.216719in}}%
\pgfpathcurveto{\pgfqpoint{3.107838in}{2.224956in}}{\pgfqpoint{3.104566in}{2.232856in}}{\pgfqpoint{3.098742in}{2.238680in}}%
\pgfpathcurveto{\pgfqpoint{3.092918in}{2.244504in}}{\pgfqpoint{3.085018in}{2.247776in}}{\pgfqpoint{3.076782in}{2.247776in}}%
\pgfpathcurveto{\pgfqpoint{3.068545in}{2.247776in}}{\pgfqpoint{3.060645in}{2.244504in}}{\pgfqpoint{3.054821in}{2.238680in}}%
\pgfpathcurveto{\pgfqpoint{3.048997in}{2.232856in}}{\pgfqpoint{3.045725in}{2.224956in}}{\pgfqpoint{3.045725in}{2.216719in}}%
\pgfpathcurveto{\pgfqpoint{3.045725in}{2.208483in}}{\pgfqpoint{3.048997in}{2.200583in}}{\pgfqpoint{3.054821in}{2.194759in}}%
\pgfpathcurveto{\pgfqpoint{3.060645in}{2.188935in}}{\pgfqpoint{3.068545in}{2.185663in}}{\pgfqpoint{3.076782in}{2.185663in}}%
\pgfpathclose%
\pgfusepath{stroke,fill}%
\end{pgfscope}%
\begin{pgfscope}%
\pgfpathrectangle{\pgfqpoint{0.100000in}{0.212622in}}{\pgfqpoint{3.696000in}{3.696000in}}%
\pgfusepath{clip}%
\pgfsetbuttcap%
\pgfsetroundjoin%
\definecolor{currentfill}{rgb}{0.121569,0.466667,0.705882}%
\pgfsetfillcolor{currentfill}%
\pgfsetfillopacity{0.656326}%
\pgfsetlinewidth{1.003750pt}%
\definecolor{currentstroke}{rgb}{0.121569,0.466667,0.705882}%
\pgfsetstrokecolor{currentstroke}%
\pgfsetstrokeopacity{0.656326}%
\pgfsetdash{}{0pt}%
\pgfpathmoveto{\pgfqpoint{3.076056in}{2.185390in}}%
\pgfpathcurveto{\pgfqpoint{3.084293in}{2.185390in}}{\pgfqpoint{3.092193in}{2.188663in}}{\pgfqpoint{3.098017in}{2.194487in}}%
\pgfpathcurveto{\pgfqpoint{3.103841in}{2.200311in}}{\pgfqpoint{3.107113in}{2.208211in}}{\pgfqpoint{3.107113in}{2.216447in}}%
\pgfpathcurveto{\pgfqpoint{3.107113in}{2.224683in}}{\pgfqpoint{3.103841in}{2.232583in}}{\pgfqpoint{3.098017in}{2.238407in}}%
\pgfpathcurveto{\pgfqpoint{3.092193in}{2.244231in}}{\pgfqpoint{3.084293in}{2.247503in}}{\pgfqpoint{3.076056in}{2.247503in}}%
\pgfpathcurveto{\pgfqpoint{3.067820in}{2.247503in}}{\pgfqpoint{3.059920in}{2.244231in}}{\pgfqpoint{3.054096in}{2.238407in}}%
\pgfpathcurveto{\pgfqpoint{3.048272in}{2.232583in}}{\pgfqpoint{3.045000in}{2.224683in}}{\pgfqpoint{3.045000in}{2.216447in}}%
\pgfpathcurveto{\pgfqpoint{3.045000in}{2.208211in}}{\pgfqpoint{3.048272in}{2.200311in}}{\pgfqpoint{3.054096in}{2.194487in}}%
\pgfpathcurveto{\pgfqpoint{3.059920in}{2.188663in}}{\pgfqpoint{3.067820in}{2.185390in}}{\pgfqpoint{3.076056in}{2.185390in}}%
\pgfpathclose%
\pgfusepath{stroke,fill}%
\end{pgfscope}%
\begin{pgfscope}%
\pgfpathrectangle{\pgfqpoint{0.100000in}{0.212622in}}{\pgfqpoint{3.696000in}{3.696000in}}%
\pgfusepath{clip}%
\pgfsetbuttcap%
\pgfsetroundjoin%
\definecolor{currentfill}{rgb}{0.121569,0.466667,0.705882}%
\pgfsetfillcolor{currentfill}%
\pgfsetfillopacity{0.657113}%
\pgfsetlinewidth{1.003750pt}%
\definecolor{currentstroke}{rgb}{0.121569,0.466667,0.705882}%
\pgfsetstrokecolor{currentstroke}%
\pgfsetstrokeopacity{0.657113}%
\pgfsetdash{}{0pt}%
\pgfpathmoveto{\pgfqpoint{3.074680in}{2.184861in}}%
\pgfpathcurveto{\pgfqpoint{3.082916in}{2.184861in}}{\pgfqpoint{3.090816in}{2.188133in}}{\pgfqpoint{3.096640in}{2.193957in}}%
\pgfpathcurveto{\pgfqpoint{3.102464in}{2.199781in}}{\pgfqpoint{3.105736in}{2.207681in}}{\pgfqpoint{3.105736in}{2.215917in}}%
\pgfpathcurveto{\pgfqpoint{3.105736in}{2.224154in}}{\pgfqpoint{3.102464in}{2.232054in}}{\pgfqpoint{3.096640in}{2.237878in}}%
\pgfpathcurveto{\pgfqpoint{3.090816in}{2.243702in}}{\pgfqpoint{3.082916in}{2.246974in}}{\pgfqpoint{3.074680in}{2.246974in}}%
\pgfpathcurveto{\pgfqpoint{3.066444in}{2.246974in}}{\pgfqpoint{3.058544in}{2.243702in}}{\pgfqpoint{3.052720in}{2.237878in}}%
\pgfpathcurveto{\pgfqpoint{3.046896in}{2.232054in}}{\pgfqpoint{3.043623in}{2.224154in}}{\pgfqpoint{3.043623in}{2.215917in}}%
\pgfpathcurveto{\pgfqpoint{3.043623in}{2.207681in}}{\pgfqpoint{3.046896in}{2.199781in}}{\pgfqpoint{3.052720in}{2.193957in}}%
\pgfpathcurveto{\pgfqpoint{3.058544in}{2.188133in}}{\pgfqpoint{3.066444in}{2.184861in}}{\pgfqpoint{3.074680in}{2.184861in}}%
\pgfpathclose%
\pgfusepath{stroke,fill}%
\end{pgfscope}%
\begin{pgfscope}%
\pgfpathrectangle{\pgfqpoint{0.100000in}{0.212622in}}{\pgfqpoint{3.696000in}{3.696000in}}%
\pgfusepath{clip}%
\pgfsetbuttcap%
\pgfsetroundjoin%
\definecolor{currentfill}{rgb}{0.121569,0.466667,0.705882}%
\pgfsetfillcolor{currentfill}%
\pgfsetfillopacity{0.657551}%
\pgfsetlinewidth{1.003750pt}%
\definecolor{currentstroke}{rgb}{0.121569,0.466667,0.705882}%
\pgfsetstrokecolor{currentstroke}%
\pgfsetstrokeopacity{0.657551}%
\pgfsetdash{}{0pt}%
\pgfpathmoveto{\pgfqpoint{3.073938in}{2.184588in}}%
\pgfpathcurveto{\pgfqpoint{3.082175in}{2.184588in}}{\pgfqpoint{3.090075in}{2.187861in}}{\pgfqpoint{3.095899in}{2.193685in}}%
\pgfpathcurveto{\pgfqpoint{3.101723in}{2.199508in}}{\pgfqpoint{3.104995in}{2.207408in}}{\pgfqpoint{3.104995in}{2.215645in}}%
\pgfpathcurveto{\pgfqpoint{3.104995in}{2.223881in}}{\pgfqpoint{3.101723in}{2.231781in}}{\pgfqpoint{3.095899in}{2.237605in}}%
\pgfpathcurveto{\pgfqpoint{3.090075in}{2.243429in}}{\pgfqpoint{3.082175in}{2.246701in}}{\pgfqpoint{3.073938in}{2.246701in}}%
\pgfpathcurveto{\pgfqpoint{3.065702in}{2.246701in}}{\pgfqpoint{3.057802in}{2.243429in}}{\pgfqpoint{3.051978in}{2.237605in}}%
\pgfpathcurveto{\pgfqpoint{3.046154in}{2.231781in}}{\pgfqpoint{3.042882in}{2.223881in}}{\pgfqpoint{3.042882in}{2.215645in}}%
\pgfpathcurveto{\pgfqpoint{3.042882in}{2.207408in}}{\pgfqpoint{3.046154in}{2.199508in}}{\pgfqpoint{3.051978in}{2.193685in}}%
\pgfpathcurveto{\pgfqpoint{3.057802in}{2.187861in}}{\pgfqpoint{3.065702in}{2.184588in}}{\pgfqpoint{3.073938in}{2.184588in}}%
\pgfpathclose%
\pgfusepath{stroke,fill}%
\end{pgfscope}%
\begin{pgfscope}%
\pgfpathrectangle{\pgfqpoint{0.100000in}{0.212622in}}{\pgfqpoint{3.696000in}{3.696000in}}%
\pgfusepath{clip}%
\pgfsetbuttcap%
\pgfsetroundjoin%
\definecolor{currentfill}{rgb}{0.121569,0.466667,0.705882}%
\pgfsetfillcolor{currentfill}%
\pgfsetfillopacity{0.657795}%
\pgfsetlinewidth{1.003750pt}%
\definecolor{currentstroke}{rgb}{0.121569,0.466667,0.705882}%
\pgfsetstrokecolor{currentstroke}%
\pgfsetstrokeopacity{0.657795}%
\pgfsetdash{}{0pt}%
\pgfpathmoveto{\pgfqpoint{3.073558in}{2.184429in}}%
\pgfpathcurveto{\pgfqpoint{3.081794in}{2.184429in}}{\pgfqpoint{3.089695in}{2.187701in}}{\pgfqpoint{3.095518in}{2.193525in}}%
\pgfpathcurveto{\pgfqpoint{3.101342in}{2.199349in}}{\pgfqpoint{3.104615in}{2.207249in}}{\pgfqpoint{3.104615in}{2.215485in}}%
\pgfpathcurveto{\pgfqpoint{3.104615in}{2.223722in}}{\pgfqpoint{3.101342in}{2.231622in}}{\pgfqpoint{3.095518in}{2.237446in}}%
\pgfpathcurveto{\pgfqpoint{3.089695in}{2.243270in}}{\pgfqpoint{3.081794in}{2.246542in}}{\pgfqpoint{3.073558in}{2.246542in}}%
\pgfpathcurveto{\pgfqpoint{3.065322in}{2.246542in}}{\pgfqpoint{3.057422in}{2.243270in}}{\pgfqpoint{3.051598in}{2.237446in}}%
\pgfpathcurveto{\pgfqpoint{3.045774in}{2.231622in}}{\pgfqpoint{3.042502in}{2.223722in}}{\pgfqpoint{3.042502in}{2.215485in}}%
\pgfpathcurveto{\pgfqpoint{3.042502in}{2.207249in}}{\pgfqpoint{3.045774in}{2.199349in}}{\pgfqpoint{3.051598in}{2.193525in}}%
\pgfpathcurveto{\pgfqpoint{3.057422in}{2.187701in}}{\pgfqpoint{3.065322in}{2.184429in}}{\pgfqpoint{3.073558in}{2.184429in}}%
\pgfpathclose%
\pgfusepath{stroke,fill}%
\end{pgfscope}%
\begin{pgfscope}%
\pgfpathrectangle{\pgfqpoint{0.100000in}{0.212622in}}{\pgfqpoint{3.696000in}{3.696000in}}%
\pgfusepath{clip}%
\pgfsetbuttcap%
\pgfsetroundjoin%
\definecolor{currentfill}{rgb}{0.121569,0.466667,0.705882}%
\pgfsetfillcolor{currentfill}%
\pgfsetfillopacity{0.657928}%
\pgfsetlinewidth{1.003750pt}%
\definecolor{currentstroke}{rgb}{0.121569,0.466667,0.705882}%
\pgfsetstrokecolor{currentstroke}%
\pgfsetstrokeopacity{0.657928}%
\pgfsetdash{}{0pt}%
\pgfpathmoveto{\pgfqpoint{3.073338in}{2.184343in}}%
\pgfpathcurveto{\pgfqpoint{3.081574in}{2.184343in}}{\pgfqpoint{3.089474in}{2.187616in}}{\pgfqpoint{3.095298in}{2.193440in}}%
\pgfpathcurveto{\pgfqpoint{3.101122in}{2.199264in}}{\pgfqpoint{3.104395in}{2.207164in}}{\pgfqpoint{3.104395in}{2.215400in}}%
\pgfpathcurveto{\pgfqpoint{3.104395in}{2.223636in}}{\pgfqpoint{3.101122in}{2.231536in}}{\pgfqpoint{3.095298in}{2.237360in}}%
\pgfpathcurveto{\pgfqpoint{3.089474in}{2.243184in}}{\pgfqpoint{3.081574in}{2.246456in}}{\pgfqpoint{3.073338in}{2.246456in}}%
\pgfpathcurveto{\pgfqpoint{3.065102in}{2.246456in}}{\pgfqpoint{3.057202in}{2.243184in}}{\pgfqpoint{3.051378in}{2.237360in}}%
\pgfpathcurveto{\pgfqpoint{3.045554in}{2.231536in}}{\pgfqpoint{3.042282in}{2.223636in}}{\pgfqpoint{3.042282in}{2.215400in}}%
\pgfpathcurveto{\pgfqpoint{3.042282in}{2.207164in}}{\pgfqpoint{3.045554in}{2.199264in}}{\pgfqpoint{3.051378in}{2.193440in}}%
\pgfpathcurveto{\pgfqpoint{3.057202in}{2.187616in}}{\pgfqpoint{3.065102in}{2.184343in}}{\pgfqpoint{3.073338in}{2.184343in}}%
\pgfpathclose%
\pgfusepath{stroke,fill}%
\end{pgfscope}%
\begin{pgfscope}%
\pgfpathrectangle{\pgfqpoint{0.100000in}{0.212622in}}{\pgfqpoint{3.696000in}{3.696000in}}%
\pgfusepath{clip}%
\pgfsetbuttcap%
\pgfsetroundjoin%
\definecolor{currentfill}{rgb}{0.121569,0.466667,0.705882}%
\pgfsetfillcolor{currentfill}%
\pgfsetfillopacity{0.658324}%
\pgfsetlinewidth{1.003750pt}%
\definecolor{currentstroke}{rgb}{0.121569,0.466667,0.705882}%
\pgfsetstrokecolor{currentstroke}%
\pgfsetstrokeopacity{0.658324}%
\pgfsetdash{}{0pt}%
\pgfpathmoveto{\pgfqpoint{3.072743in}{2.184036in}}%
\pgfpathcurveto{\pgfqpoint{3.080980in}{2.184036in}}{\pgfqpoint{3.088880in}{2.187308in}}{\pgfqpoint{3.094704in}{2.193132in}}%
\pgfpathcurveto{\pgfqpoint{3.100528in}{2.198956in}}{\pgfqpoint{3.103800in}{2.206856in}}{\pgfqpoint{3.103800in}{2.215093in}}%
\pgfpathcurveto{\pgfqpoint{3.103800in}{2.223329in}}{\pgfqpoint{3.100528in}{2.231229in}}{\pgfqpoint{3.094704in}{2.237053in}}%
\pgfpathcurveto{\pgfqpoint{3.088880in}{2.242877in}}{\pgfqpoint{3.080980in}{2.246149in}}{\pgfqpoint{3.072743in}{2.246149in}}%
\pgfpathcurveto{\pgfqpoint{3.064507in}{2.246149in}}{\pgfqpoint{3.056607in}{2.242877in}}{\pgfqpoint{3.050783in}{2.237053in}}%
\pgfpathcurveto{\pgfqpoint{3.044959in}{2.231229in}}{\pgfqpoint{3.041687in}{2.223329in}}{\pgfqpoint{3.041687in}{2.215093in}}%
\pgfpathcurveto{\pgfqpoint{3.041687in}{2.206856in}}{\pgfqpoint{3.044959in}{2.198956in}}{\pgfqpoint{3.050783in}{2.193132in}}%
\pgfpathcurveto{\pgfqpoint{3.056607in}{2.187308in}}{\pgfqpoint{3.064507in}{2.184036in}}{\pgfqpoint{3.072743in}{2.184036in}}%
\pgfpathclose%
\pgfusepath{stroke,fill}%
\end{pgfscope}%
\begin{pgfscope}%
\pgfpathrectangle{\pgfqpoint{0.100000in}{0.212622in}}{\pgfqpoint{3.696000in}{3.696000in}}%
\pgfusepath{clip}%
\pgfsetbuttcap%
\pgfsetroundjoin%
\definecolor{currentfill}{rgb}{0.121569,0.466667,0.705882}%
\pgfsetfillcolor{currentfill}%
\pgfsetfillopacity{0.658547}%
\pgfsetlinewidth{1.003750pt}%
\definecolor{currentstroke}{rgb}{0.121569,0.466667,0.705882}%
\pgfsetstrokecolor{currentstroke}%
\pgfsetstrokeopacity{0.658547}%
\pgfsetdash{}{0pt}%
\pgfpathmoveto{\pgfqpoint{3.072420in}{2.183901in}}%
\pgfpathcurveto{\pgfqpoint{3.080656in}{2.183901in}}{\pgfqpoint{3.088556in}{2.187173in}}{\pgfqpoint{3.094380in}{2.192997in}}%
\pgfpathcurveto{\pgfqpoint{3.100204in}{2.198821in}}{\pgfqpoint{3.103476in}{2.206721in}}{\pgfqpoint{3.103476in}{2.214958in}}%
\pgfpathcurveto{\pgfqpoint{3.103476in}{2.223194in}}{\pgfqpoint{3.100204in}{2.231094in}}{\pgfqpoint{3.094380in}{2.236918in}}%
\pgfpathcurveto{\pgfqpoint{3.088556in}{2.242742in}}{\pgfqpoint{3.080656in}{2.246014in}}{\pgfqpoint{3.072420in}{2.246014in}}%
\pgfpathcurveto{\pgfqpoint{3.064183in}{2.246014in}}{\pgfqpoint{3.056283in}{2.242742in}}{\pgfqpoint{3.050459in}{2.236918in}}%
\pgfpathcurveto{\pgfqpoint{3.044635in}{2.231094in}}{\pgfqpoint{3.041363in}{2.223194in}}{\pgfqpoint{3.041363in}{2.214958in}}%
\pgfpathcurveto{\pgfqpoint{3.041363in}{2.206721in}}{\pgfqpoint{3.044635in}{2.198821in}}{\pgfqpoint{3.050459in}{2.192997in}}%
\pgfpathcurveto{\pgfqpoint{3.056283in}{2.187173in}}{\pgfqpoint{3.064183in}{2.183901in}}{\pgfqpoint{3.072420in}{2.183901in}}%
\pgfpathclose%
\pgfusepath{stroke,fill}%
\end{pgfscope}%
\begin{pgfscope}%
\pgfpathrectangle{\pgfqpoint{0.100000in}{0.212622in}}{\pgfqpoint{3.696000in}{3.696000in}}%
\pgfusepath{clip}%
\pgfsetbuttcap%
\pgfsetroundjoin%
\definecolor{currentfill}{rgb}{0.121569,0.466667,0.705882}%
\pgfsetfillcolor{currentfill}%
\pgfsetfillopacity{0.658665}%
\pgfsetlinewidth{1.003750pt}%
\definecolor{currentstroke}{rgb}{0.121569,0.466667,0.705882}%
\pgfsetstrokecolor{currentstroke}%
\pgfsetstrokeopacity{0.658665}%
\pgfsetdash{}{0pt}%
\pgfpathmoveto{\pgfqpoint{3.072228in}{2.183807in}}%
\pgfpathcurveto{\pgfqpoint{3.080465in}{2.183807in}}{\pgfqpoint{3.088365in}{2.187079in}}{\pgfqpoint{3.094188in}{2.192903in}}%
\pgfpathcurveto{\pgfqpoint{3.100012in}{2.198727in}}{\pgfqpoint{3.103285in}{2.206627in}}{\pgfqpoint{3.103285in}{2.214864in}}%
\pgfpathcurveto{\pgfqpoint{3.103285in}{2.223100in}}{\pgfqpoint{3.100012in}{2.231000in}}{\pgfqpoint{3.094188in}{2.236824in}}%
\pgfpathcurveto{\pgfqpoint{3.088365in}{2.242648in}}{\pgfqpoint{3.080465in}{2.245920in}}{\pgfqpoint{3.072228in}{2.245920in}}%
\pgfpathcurveto{\pgfqpoint{3.063992in}{2.245920in}}{\pgfqpoint{3.056092in}{2.242648in}}{\pgfqpoint{3.050268in}{2.236824in}}%
\pgfpathcurveto{\pgfqpoint{3.044444in}{2.231000in}}{\pgfqpoint{3.041172in}{2.223100in}}{\pgfqpoint{3.041172in}{2.214864in}}%
\pgfpathcurveto{\pgfqpoint{3.041172in}{2.206627in}}{\pgfqpoint{3.044444in}{2.198727in}}{\pgfqpoint{3.050268in}{2.192903in}}%
\pgfpathcurveto{\pgfqpoint{3.056092in}{2.187079in}}{\pgfqpoint{3.063992in}{2.183807in}}{\pgfqpoint{3.072228in}{2.183807in}}%
\pgfpathclose%
\pgfusepath{stroke,fill}%
\end{pgfscope}%
\begin{pgfscope}%
\pgfpathrectangle{\pgfqpoint{0.100000in}{0.212622in}}{\pgfqpoint{3.696000in}{3.696000in}}%
\pgfusepath{clip}%
\pgfsetbuttcap%
\pgfsetroundjoin%
\definecolor{currentfill}{rgb}{0.121569,0.466667,0.705882}%
\pgfsetfillcolor{currentfill}%
\pgfsetfillopacity{0.659127}%
\pgfsetlinewidth{1.003750pt}%
\definecolor{currentstroke}{rgb}{0.121569,0.466667,0.705882}%
\pgfsetstrokecolor{currentstroke}%
\pgfsetstrokeopacity{0.659127}%
\pgfsetdash{}{0pt}%
\pgfpathmoveto{\pgfqpoint{3.071590in}{2.183447in}}%
\pgfpathcurveto{\pgfqpoint{3.079826in}{2.183447in}}{\pgfqpoint{3.087726in}{2.186720in}}{\pgfqpoint{3.093550in}{2.192543in}}%
\pgfpathcurveto{\pgfqpoint{3.099374in}{2.198367in}}{\pgfqpoint{3.102646in}{2.206267in}}{\pgfqpoint{3.102646in}{2.214504in}}%
\pgfpathcurveto{\pgfqpoint{3.102646in}{2.222740in}}{\pgfqpoint{3.099374in}{2.230640in}}{\pgfqpoint{3.093550in}{2.236464in}}%
\pgfpathcurveto{\pgfqpoint{3.087726in}{2.242288in}}{\pgfqpoint{3.079826in}{2.245560in}}{\pgfqpoint{3.071590in}{2.245560in}}%
\pgfpathcurveto{\pgfqpoint{3.063353in}{2.245560in}}{\pgfqpoint{3.055453in}{2.242288in}}{\pgfqpoint{3.049629in}{2.236464in}}%
\pgfpathcurveto{\pgfqpoint{3.043805in}{2.230640in}}{\pgfqpoint{3.040533in}{2.222740in}}{\pgfqpoint{3.040533in}{2.214504in}}%
\pgfpathcurveto{\pgfqpoint{3.040533in}{2.206267in}}{\pgfqpoint{3.043805in}{2.198367in}}{\pgfqpoint{3.049629in}{2.192543in}}%
\pgfpathcurveto{\pgfqpoint{3.055453in}{2.186720in}}{\pgfqpoint{3.063353in}{2.183447in}}{\pgfqpoint{3.071590in}{2.183447in}}%
\pgfpathclose%
\pgfusepath{stroke,fill}%
\end{pgfscope}%
\begin{pgfscope}%
\pgfpathrectangle{\pgfqpoint{0.100000in}{0.212622in}}{\pgfqpoint{3.696000in}{3.696000in}}%
\pgfusepath{clip}%
\pgfsetbuttcap%
\pgfsetroundjoin%
\definecolor{currentfill}{rgb}{0.121569,0.466667,0.705882}%
\pgfsetfillcolor{currentfill}%
\pgfsetfillopacity{0.659806}%
\pgfsetlinewidth{1.003750pt}%
\definecolor{currentstroke}{rgb}{0.121569,0.466667,0.705882}%
\pgfsetstrokecolor{currentstroke}%
\pgfsetstrokeopacity{0.659806}%
\pgfsetdash{}{0pt}%
\pgfpathmoveto{\pgfqpoint{3.070641in}{2.183098in}}%
\pgfpathcurveto{\pgfqpoint{3.078877in}{2.183098in}}{\pgfqpoint{3.086777in}{2.186371in}}{\pgfqpoint{3.092601in}{2.192195in}}%
\pgfpathcurveto{\pgfqpoint{3.098425in}{2.198019in}}{\pgfqpoint{3.101697in}{2.205919in}}{\pgfqpoint{3.101697in}{2.214155in}}%
\pgfpathcurveto{\pgfqpoint{3.101697in}{2.222391in}}{\pgfqpoint{3.098425in}{2.230291in}}{\pgfqpoint{3.092601in}{2.236115in}}%
\pgfpathcurveto{\pgfqpoint{3.086777in}{2.241939in}}{\pgfqpoint{3.078877in}{2.245211in}}{\pgfqpoint{3.070641in}{2.245211in}}%
\pgfpathcurveto{\pgfqpoint{3.062405in}{2.245211in}}{\pgfqpoint{3.054505in}{2.241939in}}{\pgfqpoint{3.048681in}{2.236115in}}%
\pgfpathcurveto{\pgfqpoint{3.042857in}{2.230291in}}{\pgfqpoint{3.039584in}{2.222391in}}{\pgfqpoint{3.039584in}{2.214155in}}%
\pgfpathcurveto{\pgfqpoint{3.039584in}{2.205919in}}{\pgfqpoint{3.042857in}{2.198019in}}{\pgfqpoint{3.048681in}{2.192195in}}%
\pgfpathcurveto{\pgfqpoint{3.054505in}{2.186371in}}{\pgfqpoint{3.062405in}{2.183098in}}{\pgfqpoint{3.070641in}{2.183098in}}%
\pgfpathclose%
\pgfusepath{stroke,fill}%
\end{pgfscope}%
\begin{pgfscope}%
\pgfpathrectangle{\pgfqpoint{0.100000in}{0.212622in}}{\pgfqpoint{3.696000in}{3.696000in}}%
\pgfusepath{clip}%
\pgfsetbuttcap%
\pgfsetroundjoin%
\definecolor{currentfill}{rgb}{0.121569,0.466667,0.705882}%
\pgfsetfillcolor{currentfill}%
\pgfsetfillopacity{0.660615}%
\pgfsetlinewidth{1.003750pt}%
\definecolor{currentstroke}{rgb}{0.121569,0.466667,0.705882}%
\pgfsetstrokecolor{currentstroke}%
\pgfsetstrokeopacity{0.660615}%
\pgfsetdash{}{0pt}%
\pgfpathmoveto{\pgfqpoint{3.069359in}{2.182467in}}%
\pgfpathcurveto{\pgfqpoint{3.077596in}{2.182467in}}{\pgfqpoint{3.085496in}{2.185739in}}{\pgfqpoint{3.091320in}{2.191563in}}%
\pgfpathcurveto{\pgfqpoint{3.097144in}{2.197387in}}{\pgfqpoint{3.100416in}{2.205287in}}{\pgfqpoint{3.100416in}{2.213523in}}%
\pgfpathcurveto{\pgfqpoint{3.100416in}{2.221759in}}{\pgfqpoint{3.097144in}{2.229659in}}{\pgfqpoint{3.091320in}{2.235483in}}%
\pgfpathcurveto{\pgfqpoint{3.085496in}{2.241307in}}{\pgfqpoint{3.077596in}{2.244580in}}{\pgfqpoint{3.069359in}{2.244580in}}%
\pgfpathcurveto{\pgfqpoint{3.061123in}{2.244580in}}{\pgfqpoint{3.053223in}{2.241307in}}{\pgfqpoint{3.047399in}{2.235483in}}%
\pgfpathcurveto{\pgfqpoint{3.041575in}{2.229659in}}{\pgfqpoint{3.038303in}{2.221759in}}{\pgfqpoint{3.038303in}{2.213523in}}%
\pgfpathcurveto{\pgfqpoint{3.038303in}{2.205287in}}{\pgfqpoint{3.041575in}{2.197387in}}{\pgfqpoint{3.047399in}{2.191563in}}%
\pgfpathcurveto{\pgfqpoint{3.053223in}{2.185739in}}{\pgfqpoint{3.061123in}{2.182467in}}{\pgfqpoint{3.069359in}{2.182467in}}%
\pgfpathclose%
\pgfusepath{stroke,fill}%
\end{pgfscope}%
\begin{pgfscope}%
\pgfpathrectangle{\pgfqpoint{0.100000in}{0.212622in}}{\pgfqpoint{3.696000in}{3.696000in}}%
\pgfusepath{clip}%
\pgfsetbuttcap%
\pgfsetroundjoin%
\definecolor{currentfill}{rgb}{0.121569,0.466667,0.705882}%
\pgfsetfillcolor{currentfill}%
\pgfsetfillopacity{0.661932}%
\pgfsetlinewidth{1.003750pt}%
\definecolor{currentstroke}{rgb}{0.121569,0.466667,0.705882}%
\pgfsetstrokecolor{currentstroke}%
\pgfsetstrokeopacity{0.661932}%
\pgfsetdash{}{0pt}%
\pgfpathmoveto{\pgfqpoint{3.067288in}{2.181331in}}%
\pgfpathcurveto{\pgfqpoint{3.075524in}{2.181331in}}{\pgfqpoint{3.083424in}{2.184603in}}{\pgfqpoint{3.089248in}{2.190427in}}%
\pgfpathcurveto{\pgfqpoint{3.095072in}{2.196251in}}{\pgfqpoint{3.098344in}{2.204151in}}{\pgfqpoint{3.098344in}{2.212388in}}%
\pgfpathcurveto{\pgfqpoint{3.098344in}{2.220624in}}{\pgfqpoint{3.095072in}{2.228524in}}{\pgfqpoint{3.089248in}{2.234348in}}%
\pgfpathcurveto{\pgfqpoint{3.083424in}{2.240172in}}{\pgfqpoint{3.075524in}{2.243444in}}{\pgfqpoint{3.067288in}{2.243444in}}%
\pgfpathcurveto{\pgfqpoint{3.059052in}{2.243444in}}{\pgfqpoint{3.051152in}{2.240172in}}{\pgfqpoint{3.045328in}{2.234348in}}%
\pgfpathcurveto{\pgfqpoint{3.039504in}{2.228524in}}{\pgfqpoint{3.036231in}{2.220624in}}{\pgfqpoint{3.036231in}{2.212388in}}%
\pgfpathcurveto{\pgfqpoint{3.036231in}{2.204151in}}{\pgfqpoint{3.039504in}{2.196251in}}{\pgfqpoint{3.045328in}{2.190427in}}%
\pgfpathcurveto{\pgfqpoint{3.051152in}{2.184603in}}{\pgfqpoint{3.059052in}{2.181331in}}{\pgfqpoint{3.067288in}{2.181331in}}%
\pgfpathclose%
\pgfusepath{stroke,fill}%
\end{pgfscope}%
\begin{pgfscope}%
\pgfpathrectangle{\pgfqpoint{0.100000in}{0.212622in}}{\pgfqpoint{3.696000in}{3.696000in}}%
\pgfusepath{clip}%
\pgfsetbuttcap%
\pgfsetroundjoin%
\definecolor{currentfill}{rgb}{0.121569,0.466667,0.705882}%
\pgfsetfillcolor{currentfill}%
\pgfsetfillopacity{0.662666}%
\pgfsetlinewidth{1.003750pt}%
\definecolor{currentstroke}{rgb}{0.121569,0.466667,0.705882}%
\pgfsetstrokecolor{currentstroke}%
\pgfsetstrokeopacity{0.662666}%
\pgfsetdash{}{0pt}%
\pgfpathmoveto{\pgfqpoint{3.066170in}{2.180749in}}%
\pgfpathcurveto{\pgfqpoint{3.074407in}{2.180749in}}{\pgfqpoint{3.082307in}{2.184021in}}{\pgfqpoint{3.088131in}{2.189845in}}%
\pgfpathcurveto{\pgfqpoint{3.093954in}{2.195669in}}{\pgfqpoint{3.097227in}{2.203569in}}{\pgfqpoint{3.097227in}{2.211805in}}%
\pgfpathcurveto{\pgfqpoint{3.097227in}{2.220041in}}{\pgfqpoint{3.093954in}{2.227941in}}{\pgfqpoint{3.088131in}{2.233765in}}%
\pgfpathcurveto{\pgfqpoint{3.082307in}{2.239589in}}{\pgfqpoint{3.074407in}{2.242862in}}{\pgfqpoint{3.066170in}{2.242862in}}%
\pgfpathcurveto{\pgfqpoint{3.057934in}{2.242862in}}{\pgfqpoint{3.050034in}{2.239589in}}{\pgfqpoint{3.044210in}{2.233765in}}%
\pgfpathcurveto{\pgfqpoint{3.038386in}{2.227941in}}{\pgfqpoint{3.035114in}{2.220041in}}{\pgfqpoint{3.035114in}{2.211805in}}%
\pgfpathcurveto{\pgfqpoint{3.035114in}{2.203569in}}{\pgfqpoint{3.038386in}{2.195669in}}{\pgfqpoint{3.044210in}{2.189845in}}%
\pgfpathcurveto{\pgfqpoint{3.050034in}{2.184021in}}{\pgfqpoint{3.057934in}{2.180749in}}{\pgfqpoint{3.066170in}{2.180749in}}%
\pgfpathclose%
\pgfusepath{stroke,fill}%
\end{pgfscope}%
\begin{pgfscope}%
\pgfpathrectangle{\pgfqpoint{0.100000in}{0.212622in}}{\pgfqpoint{3.696000in}{3.696000in}}%
\pgfusepath{clip}%
\pgfsetbuttcap%
\pgfsetroundjoin%
\definecolor{currentfill}{rgb}{0.121569,0.466667,0.705882}%
\pgfsetfillcolor{currentfill}%
\pgfsetfillopacity{0.663060}%
\pgfsetlinewidth{1.003750pt}%
\definecolor{currentstroke}{rgb}{0.121569,0.466667,0.705882}%
\pgfsetstrokecolor{currentstroke}%
\pgfsetstrokeopacity{0.663060}%
\pgfsetdash{}{0pt}%
\pgfpathmoveto{\pgfqpoint{3.065518in}{2.180398in}}%
\pgfpathcurveto{\pgfqpoint{3.073754in}{2.180398in}}{\pgfqpoint{3.081655in}{2.183670in}}{\pgfqpoint{3.087478in}{2.189494in}}%
\pgfpathcurveto{\pgfqpoint{3.093302in}{2.195318in}}{\pgfqpoint{3.096575in}{2.203218in}}{\pgfqpoint{3.096575in}{2.211454in}}%
\pgfpathcurveto{\pgfqpoint{3.096575in}{2.219690in}}{\pgfqpoint{3.093302in}{2.227590in}}{\pgfqpoint{3.087478in}{2.233414in}}%
\pgfpathcurveto{\pgfqpoint{3.081655in}{2.239238in}}{\pgfqpoint{3.073754in}{2.242511in}}{\pgfqpoint{3.065518in}{2.242511in}}%
\pgfpathcurveto{\pgfqpoint{3.057282in}{2.242511in}}{\pgfqpoint{3.049382in}{2.239238in}}{\pgfqpoint{3.043558in}{2.233414in}}%
\pgfpathcurveto{\pgfqpoint{3.037734in}{2.227590in}}{\pgfqpoint{3.034462in}{2.219690in}}{\pgfqpoint{3.034462in}{2.211454in}}%
\pgfpathcurveto{\pgfqpoint{3.034462in}{2.203218in}}{\pgfqpoint{3.037734in}{2.195318in}}{\pgfqpoint{3.043558in}{2.189494in}}%
\pgfpathcurveto{\pgfqpoint{3.049382in}{2.183670in}}{\pgfqpoint{3.057282in}{2.180398in}}{\pgfqpoint{3.065518in}{2.180398in}}%
\pgfpathclose%
\pgfusepath{stroke,fill}%
\end{pgfscope}%
\begin{pgfscope}%
\pgfpathrectangle{\pgfqpoint{0.100000in}{0.212622in}}{\pgfqpoint{3.696000in}{3.696000in}}%
\pgfusepath{clip}%
\pgfsetbuttcap%
\pgfsetroundjoin%
\definecolor{currentfill}{rgb}{0.121569,0.466667,0.705882}%
\pgfsetfillcolor{currentfill}%
\pgfsetfillopacity{0.663276}%
\pgfsetlinewidth{1.003750pt}%
\definecolor{currentstroke}{rgb}{0.121569,0.466667,0.705882}%
\pgfsetstrokecolor{currentstroke}%
\pgfsetstrokeopacity{0.663276}%
\pgfsetdash{}{0pt}%
\pgfpathmoveto{\pgfqpoint{3.065149in}{2.180210in}}%
\pgfpathcurveto{\pgfqpoint{3.073385in}{2.180210in}}{\pgfqpoint{3.081286in}{2.183482in}}{\pgfqpoint{3.087109in}{2.189306in}}%
\pgfpathcurveto{\pgfqpoint{3.092933in}{2.195130in}}{\pgfqpoint{3.096206in}{2.203030in}}{\pgfqpoint{3.096206in}{2.211266in}}%
\pgfpathcurveto{\pgfqpoint{3.096206in}{2.219502in}}{\pgfqpoint{3.092933in}{2.227402in}}{\pgfqpoint{3.087109in}{2.233226in}}%
\pgfpathcurveto{\pgfqpoint{3.081286in}{2.239050in}}{\pgfqpoint{3.073385in}{2.242323in}}{\pgfqpoint{3.065149in}{2.242323in}}%
\pgfpathcurveto{\pgfqpoint{3.056913in}{2.242323in}}{\pgfqpoint{3.049013in}{2.239050in}}{\pgfqpoint{3.043189in}{2.233226in}}%
\pgfpathcurveto{\pgfqpoint{3.037365in}{2.227402in}}{\pgfqpoint{3.034093in}{2.219502in}}{\pgfqpoint{3.034093in}{2.211266in}}%
\pgfpathcurveto{\pgfqpoint{3.034093in}{2.203030in}}{\pgfqpoint{3.037365in}{2.195130in}}{\pgfqpoint{3.043189in}{2.189306in}}%
\pgfpathcurveto{\pgfqpoint{3.049013in}{2.183482in}}{\pgfqpoint{3.056913in}{2.180210in}}{\pgfqpoint{3.065149in}{2.180210in}}%
\pgfpathclose%
\pgfusepath{stroke,fill}%
\end{pgfscope}%
\begin{pgfscope}%
\pgfpathrectangle{\pgfqpoint{0.100000in}{0.212622in}}{\pgfqpoint{3.696000in}{3.696000in}}%
\pgfusepath{clip}%
\pgfsetbuttcap%
\pgfsetroundjoin%
\definecolor{currentfill}{rgb}{0.121569,0.466667,0.705882}%
\pgfsetfillcolor{currentfill}%
\pgfsetfillopacity{0.663793}%
\pgfsetlinewidth{1.003750pt}%
\definecolor{currentstroke}{rgb}{0.121569,0.466667,0.705882}%
\pgfsetstrokecolor{currentstroke}%
\pgfsetstrokeopacity{0.663793}%
\pgfsetdash{}{0pt}%
\pgfpathmoveto{\pgfqpoint{3.064397in}{2.180144in}}%
\pgfpathcurveto{\pgfqpoint{3.072633in}{2.180144in}}{\pgfqpoint{3.080533in}{2.183416in}}{\pgfqpoint{3.086357in}{2.189240in}}%
\pgfpathcurveto{\pgfqpoint{3.092181in}{2.195064in}}{\pgfqpoint{3.095454in}{2.202964in}}{\pgfqpoint{3.095454in}{2.211200in}}%
\pgfpathcurveto{\pgfqpoint{3.095454in}{2.219437in}}{\pgfqpoint{3.092181in}{2.227337in}}{\pgfqpoint{3.086357in}{2.233161in}}%
\pgfpathcurveto{\pgfqpoint{3.080533in}{2.238985in}}{\pgfqpoint{3.072633in}{2.242257in}}{\pgfqpoint{3.064397in}{2.242257in}}%
\pgfpathcurveto{\pgfqpoint{3.056161in}{2.242257in}}{\pgfqpoint{3.048261in}{2.238985in}}{\pgfqpoint{3.042437in}{2.233161in}}%
\pgfpathcurveto{\pgfqpoint{3.036613in}{2.227337in}}{\pgfqpoint{3.033341in}{2.219437in}}{\pgfqpoint{3.033341in}{2.211200in}}%
\pgfpathcurveto{\pgfqpoint{3.033341in}{2.202964in}}{\pgfqpoint{3.036613in}{2.195064in}}{\pgfqpoint{3.042437in}{2.189240in}}%
\pgfpathcurveto{\pgfqpoint{3.048261in}{2.183416in}}{\pgfqpoint{3.056161in}{2.180144in}}{\pgfqpoint{3.064397in}{2.180144in}}%
\pgfpathclose%
\pgfusepath{stroke,fill}%
\end{pgfscope}%
\begin{pgfscope}%
\pgfpathrectangle{\pgfqpoint{0.100000in}{0.212622in}}{\pgfqpoint{3.696000in}{3.696000in}}%
\pgfusepath{clip}%
\pgfsetbuttcap%
\pgfsetroundjoin%
\definecolor{currentfill}{rgb}{0.121569,0.466667,0.705882}%
\pgfsetfillcolor{currentfill}%
\pgfsetfillopacity{0.664066}%
\pgfsetlinewidth{1.003750pt}%
\definecolor{currentstroke}{rgb}{0.121569,0.466667,0.705882}%
\pgfsetstrokecolor{currentstroke}%
\pgfsetstrokeopacity{0.664066}%
\pgfsetdash{}{0pt}%
\pgfpathmoveto{\pgfqpoint{3.063963in}{2.180055in}}%
\pgfpathcurveto{\pgfqpoint{3.072199in}{2.180055in}}{\pgfqpoint{3.080099in}{2.183327in}}{\pgfqpoint{3.085923in}{2.189151in}}%
\pgfpathcurveto{\pgfqpoint{3.091747in}{2.194975in}}{\pgfqpoint{3.095019in}{2.202875in}}{\pgfqpoint{3.095019in}{2.211111in}}%
\pgfpathcurveto{\pgfqpoint{3.095019in}{2.219347in}}{\pgfqpoint{3.091747in}{2.227247in}}{\pgfqpoint{3.085923in}{2.233071in}}%
\pgfpathcurveto{\pgfqpoint{3.080099in}{2.238895in}}{\pgfqpoint{3.072199in}{2.242168in}}{\pgfqpoint{3.063963in}{2.242168in}}%
\pgfpathcurveto{\pgfqpoint{3.055727in}{2.242168in}}{\pgfqpoint{3.047826in}{2.238895in}}{\pgfqpoint{3.042003in}{2.233071in}}%
\pgfpathcurveto{\pgfqpoint{3.036179in}{2.227247in}}{\pgfqpoint{3.032906in}{2.219347in}}{\pgfqpoint{3.032906in}{2.211111in}}%
\pgfpathcurveto{\pgfqpoint{3.032906in}{2.202875in}}{\pgfqpoint{3.036179in}{2.194975in}}{\pgfqpoint{3.042003in}{2.189151in}}%
\pgfpathcurveto{\pgfqpoint{3.047826in}{2.183327in}}{\pgfqpoint{3.055727in}{2.180055in}}{\pgfqpoint{3.063963in}{2.180055in}}%
\pgfpathclose%
\pgfusepath{stroke,fill}%
\end{pgfscope}%
\begin{pgfscope}%
\pgfpathrectangle{\pgfqpoint{0.100000in}{0.212622in}}{\pgfqpoint{3.696000in}{3.696000in}}%
\pgfusepath{clip}%
\pgfsetbuttcap%
\pgfsetroundjoin%
\definecolor{currentfill}{rgb}{0.121569,0.466667,0.705882}%
\pgfsetfillcolor{currentfill}%
\pgfsetfillopacity{0.664210}%
\pgfsetlinewidth{1.003750pt}%
\definecolor{currentstroke}{rgb}{0.121569,0.466667,0.705882}%
\pgfsetstrokecolor{currentstroke}%
\pgfsetstrokeopacity{0.664210}%
\pgfsetdash{}{0pt}%
\pgfpathmoveto{\pgfqpoint{3.063732in}{2.179961in}}%
\pgfpathcurveto{\pgfqpoint{3.071968in}{2.179961in}}{\pgfqpoint{3.079868in}{2.183233in}}{\pgfqpoint{3.085692in}{2.189057in}}%
\pgfpathcurveto{\pgfqpoint{3.091516in}{2.194881in}}{\pgfqpoint{3.094788in}{2.202781in}}{\pgfqpoint{3.094788in}{2.211017in}}%
\pgfpathcurveto{\pgfqpoint{3.094788in}{2.219254in}}{\pgfqpoint{3.091516in}{2.227154in}}{\pgfqpoint{3.085692in}{2.232978in}}%
\pgfpathcurveto{\pgfqpoint{3.079868in}{2.238802in}}{\pgfqpoint{3.071968in}{2.242074in}}{\pgfqpoint{3.063732in}{2.242074in}}%
\pgfpathcurveto{\pgfqpoint{3.055495in}{2.242074in}}{\pgfqpoint{3.047595in}{2.238802in}}{\pgfqpoint{3.041771in}{2.232978in}}%
\pgfpathcurveto{\pgfqpoint{3.035947in}{2.227154in}}{\pgfqpoint{3.032675in}{2.219254in}}{\pgfqpoint{3.032675in}{2.211017in}}%
\pgfpathcurveto{\pgfqpoint{3.032675in}{2.202781in}}{\pgfqpoint{3.035947in}{2.194881in}}{\pgfqpoint{3.041771in}{2.189057in}}%
\pgfpathcurveto{\pgfqpoint{3.047595in}{2.183233in}}{\pgfqpoint{3.055495in}{2.179961in}}{\pgfqpoint{3.063732in}{2.179961in}}%
\pgfpathclose%
\pgfusepath{stroke,fill}%
\end{pgfscope}%
\begin{pgfscope}%
\pgfpathrectangle{\pgfqpoint{0.100000in}{0.212622in}}{\pgfqpoint{3.696000in}{3.696000in}}%
\pgfusepath{clip}%
\pgfsetbuttcap%
\pgfsetroundjoin%
\definecolor{currentfill}{rgb}{0.121569,0.466667,0.705882}%
\pgfsetfillcolor{currentfill}%
\pgfsetfillopacity{0.664776}%
\pgfsetlinewidth{1.003750pt}%
\definecolor{currentstroke}{rgb}{0.121569,0.466667,0.705882}%
\pgfsetstrokecolor{currentstroke}%
\pgfsetstrokeopacity{0.664776}%
\pgfsetdash{}{0pt}%
\pgfpathmoveto{\pgfqpoint{3.062830in}{2.179633in}}%
\pgfpathcurveto{\pgfqpoint{3.071066in}{2.179633in}}{\pgfqpoint{3.078966in}{2.182905in}}{\pgfqpoint{3.084790in}{2.188729in}}%
\pgfpathcurveto{\pgfqpoint{3.090614in}{2.194553in}}{\pgfqpoint{3.093886in}{2.202453in}}{\pgfqpoint{3.093886in}{2.210689in}}%
\pgfpathcurveto{\pgfqpoint{3.093886in}{2.218926in}}{\pgfqpoint{3.090614in}{2.226826in}}{\pgfqpoint{3.084790in}{2.232650in}}%
\pgfpathcurveto{\pgfqpoint{3.078966in}{2.238474in}}{\pgfqpoint{3.071066in}{2.241746in}}{\pgfqpoint{3.062830in}{2.241746in}}%
\pgfpathcurveto{\pgfqpoint{3.054593in}{2.241746in}}{\pgfqpoint{3.046693in}{2.238474in}}{\pgfqpoint{3.040869in}{2.232650in}}%
\pgfpathcurveto{\pgfqpoint{3.035045in}{2.226826in}}{\pgfqpoint{3.031773in}{2.218926in}}{\pgfqpoint{3.031773in}{2.210689in}}%
\pgfpathcurveto{\pgfqpoint{3.031773in}{2.202453in}}{\pgfqpoint{3.035045in}{2.194553in}}{\pgfqpoint{3.040869in}{2.188729in}}%
\pgfpathcurveto{\pgfqpoint{3.046693in}{2.182905in}}{\pgfqpoint{3.054593in}{2.179633in}}{\pgfqpoint{3.062830in}{2.179633in}}%
\pgfpathclose%
\pgfusepath{stroke,fill}%
\end{pgfscope}%
\begin{pgfscope}%
\pgfpathrectangle{\pgfqpoint{0.100000in}{0.212622in}}{\pgfqpoint{3.696000in}{3.696000in}}%
\pgfusepath{clip}%
\pgfsetbuttcap%
\pgfsetroundjoin%
\definecolor{currentfill}{rgb}{0.121569,0.466667,0.705882}%
\pgfsetfillcolor{currentfill}%
\pgfsetfillopacity{0.665079}%
\pgfsetlinewidth{1.003750pt}%
\definecolor{currentstroke}{rgb}{0.121569,0.466667,0.705882}%
\pgfsetstrokecolor{currentstroke}%
\pgfsetstrokeopacity{0.665079}%
\pgfsetdash{}{0pt}%
\pgfpathmoveto{\pgfqpoint{3.062323in}{2.179408in}}%
\pgfpathcurveto{\pgfqpoint{3.070559in}{2.179408in}}{\pgfqpoint{3.078459in}{2.182681in}}{\pgfqpoint{3.084283in}{2.188505in}}%
\pgfpathcurveto{\pgfqpoint{3.090107in}{2.194329in}}{\pgfqpoint{3.093379in}{2.202229in}}{\pgfqpoint{3.093379in}{2.210465in}}%
\pgfpathcurveto{\pgfqpoint{3.093379in}{2.218701in}}{\pgfqpoint{3.090107in}{2.226601in}}{\pgfqpoint{3.084283in}{2.232425in}}%
\pgfpathcurveto{\pgfqpoint{3.078459in}{2.238249in}}{\pgfqpoint{3.070559in}{2.241521in}}{\pgfqpoint{3.062323in}{2.241521in}}%
\pgfpathcurveto{\pgfqpoint{3.054087in}{2.241521in}}{\pgfqpoint{3.046187in}{2.238249in}}{\pgfqpoint{3.040363in}{2.232425in}}%
\pgfpathcurveto{\pgfqpoint{3.034539in}{2.226601in}}{\pgfqpoint{3.031266in}{2.218701in}}{\pgfqpoint{3.031266in}{2.210465in}}%
\pgfpathcurveto{\pgfqpoint{3.031266in}{2.202229in}}{\pgfqpoint{3.034539in}{2.194329in}}{\pgfqpoint{3.040363in}{2.188505in}}%
\pgfpathcurveto{\pgfqpoint{3.046187in}{2.182681in}}{\pgfqpoint{3.054087in}{2.179408in}}{\pgfqpoint{3.062323in}{2.179408in}}%
\pgfpathclose%
\pgfusepath{stroke,fill}%
\end{pgfscope}%
\begin{pgfscope}%
\pgfpathrectangle{\pgfqpoint{0.100000in}{0.212622in}}{\pgfqpoint{3.696000in}{3.696000in}}%
\pgfusepath{clip}%
\pgfsetbuttcap%
\pgfsetroundjoin%
\definecolor{currentfill}{rgb}{0.121569,0.466667,0.705882}%
\pgfsetfillcolor{currentfill}%
\pgfsetfillopacity{0.665244}%
\pgfsetlinewidth{1.003750pt}%
\definecolor{currentstroke}{rgb}{0.121569,0.466667,0.705882}%
\pgfsetstrokecolor{currentstroke}%
\pgfsetstrokeopacity{0.665244}%
\pgfsetdash{}{0pt}%
\pgfpathmoveto{\pgfqpoint{3.062056in}{2.179263in}}%
\pgfpathcurveto{\pgfqpoint{3.070292in}{2.179263in}}{\pgfqpoint{3.078192in}{2.182535in}}{\pgfqpoint{3.084016in}{2.188359in}}%
\pgfpathcurveto{\pgfqpoint{3.089840in}{2.194183in}}{\pgfqpoint{3.093112in}{2.202083in}}{\pgfqpoint{3.093112in}{2.210319in}}%
\pgfpathcurveto{\pgfqpoint{3.093112in}{2.218555in}}{\pgfqpoint{3.089840in}{2.226455in}}{\pgfqpoint{3.084016in}{2.232279in}}%
\pgfpathcurveto{\pgfqpoint{3.078192in}{2.238103in}}{\pgfqpoint{3.070292in}{2.241376in}}{\pgfqpoint{3.062056in}{2.241376in}}%
\pgfpathcurveto{\pgfqpoint{3.053819in}{2.241376in}}{\pgfqpoint{3.045919in}{2.238103in}}{\pgfqpoint{3.040095in}{2.232279in}}%
\pgfpathcurveto{\pgfqpoint{3.034272in}{2.226455in}}{\pgfqpoint{3.030999in}{2.218555in}}{\pgfqpoint{3.030999in}{2.210319in}}%
\pgfpathcurveto{\pgfqpoint{3.030999in}{2.202083in}}{\pgfqpoint{3.034272in}{2.194183in}}{\pgfqpoint{3.040095in}{2.188359in}}%
\pgfpathcurveto{\pgfqpoint{3.045919in}{2.182535in}}{\pgfqpoint{3.053819in}{2.179263in}}{\pgfqpoint{3.062056in}{2.179263in}}%
\pgfpathclose%
\pgfusepath{stroke,fill}%
\end{pgfscope}%
\begin{pgfscope}%
\pgfpathrectangle{\pgfqpoint{0.100000in}{0.212622in}}{\pgfqpoint{3.696000in}{3.696000in}}%
\pgfusepath{clip}%
\pgfsetbuttcap%
\pgfsetroundjoin%
\definecolor{currentfill}{rgb}{0.121569,0.466667,0.705882}%
\pgfsetfillcolor{currentfill}%
\pgfsetfillopacity{0.665553}%
\pgfsetlinewidth{1.003750pt}%
\definecolor{currentstroke}{rgb}{0.121569,0.466667,0.705882}%
\pgfsetstrokecolor{currentstroke}%
\pgfsetstrokeopacity{0.665553}%
\pgfsetdash{}{0pt}%
\pgfpathmoveto{\pgfqpoint{3.061527in}{2.179018in}}%
\pgfpathcurveto{\pgfqpoint{3.069763in}{2.179018in}}{\pgfqpoint{3.077663in}{2.182291in}}{\pgfqpoint{3.083487in}{2.188115in}}%
\pgfpathcurveto{\pgfqpoint{3.089311in}{2.193939in}}{\pgfqpoint{3.092584in}{2.201839in}}{\pgfqpoint{3.092584in}{2.210075in}}%
\pgfpathcurveto{\pgfqpoint{3.092584in}{2.218311in}}{\pgfqpoint{3.089311in}{2.226211in}}{\pgfqpoint{3.083487in}{2.232035in}}%
\pgfpathcurveto{\pgfqpoint{3.077663in}{2.237859in}}{\pgfqpoint{3.069763in}{2.241131in}}{\pgfqpoint{3.061527in}{2.241131in}}%
\pgfpathcurveto{\pgfqpoint{3.053291in}{2.241131in}}{\pgfqpoint{3.045391in}{2.237859in}}{\pgfqpoint{3.039567in}{2.232035in}}%
\pgfpathcurveto{\pgfqpoint{3.033743in}{2.226211in}}{\pgfqpoint{3.030471in}{2.218311in}}{\pgfqpoint{3.030471in}{2.210075in}}%
\pgfpathcurveto{\pgfqpoint{3.030471in}{2.201839in}}{\pgfqpoint{3.033743in}{2.193939in}}{\pgfqpoint{3.039567in}{2.188115in}}%
\pgfpathcurveto{\pgfqpoint{3.045391in}{2.182291in}}{\pgfqpoint{3.053291in}{2.179018in}}{\pgfqpoint{3.061527in}{2.179018in}}%
\pgfpathclose%
\pgfusepath{stroke,fill}%
\end{pgfscope}%
\begin{pgfscope}%
\pgfpathrectangle{\pgfqpoint{0.100000in}{0.212622in}}{\pgfqpoint{3.696000in}{3.696000in}}%
\pgfusepath{clip}%
\pgfsetbuttcap%
\pgfsetroundjoin%
\definecolor{currentfill}{rgb}{0.121569,0.466667,0.705882}%
\pgfsetfillcolor{currentfill}%
\pgfsetfillopacity{0.666121}%
\pgfsetlinewidth{1.003750pt}%
\definecolor{currentstroke}{rgb}{0.121569,0.466667,0.705882}%
\pgfsetstrokecolor{currentstroke}%
\pgfsetstrokeopacity{0.666121}%
\pgfsetdash{}{0pt}%
\pgfpathmoveto{\pgfqpoint{3.060673in}{2.178603in}}%
\pgfpathcurveto{\pgfqpoint{3.068909in}{2.178603in}}{\pgfqpoint{3.076809in}{2.181876in}}{\pgfqpoint{3.082633in}{2.187700in}}%
\pgfpathcurveto{\pgfqpoint{3.088457in}{2.193523in}}{\pgfqpoint{3.091729in}{2.201424in}}{\pgfqpoint{3.091729in}{2.209660in}}%
\pgfpathcurveto{\pgfqpoint{3.091729in}{2.217896in}}{\pgfqpoint{3.088457in}{2.225796in}}{\pgfqpoint{3.082633in}{2.231620in}}%
\pgfpathcurveto{\pgfqpoint{3.076809in}{2.237444in}}{\pgfqpoint{3.068909in}{2.240716in}}{\pgfqpoint{3.060673in}{2.240716in}}%
\pgfpathcurveto{\pgfqpoint{3.052437in}{2.240716in}}{\pgfqpoint{3.044536in}{2.237444in}}{\pgfqpoint{3.038713in}{2.231620in}}%
\pgfpathcurveto{\pgfqpoint{3.032889in}{2.225796in}}{\pgfqpoint{3.029616in}{2.217896in}}{\pgfqpoint{3.029616in}{2.209660in}}%
\pgfpathcurveto{\pgfqpoint{3.029616in}{2.201424in}}{\pgfqpoint{3.032889in}{2.193523in}}{\pgfqpoint{3.038713in}{2.187700in}}%
\pgfpathcurveto{\pgfqpoint{3.044536in}{2.181876in}}{\pgfqpoint{3.052437in}{2.178603in}}{\pgfqpoint{3.060673in}{2.178603in}}%
\pgfpathclose%
\pgfusepath{stroke,fill}%
\end{pgfscope}%
\begin{pgfscope}%
\pgfpathrectangle{\pgfqpoint{0.100000in}{0.212622in}}{\pgfqpoint{3.696000in}{3.696000in}}%
\pgfusepath{clip}%
\pgfsetbuttcap%
\pgfsetroundjoin%
\definecolor{currentfill}{rgb}{0.121569,0.466667,0.705882}%
\pgfsetfillcolor{currentfill}%
\pgfsetfillopacity{0.666886}%
\pgfsetlinewidth{1.003750pt}%
\definecolor{currentstroke}{rgb}{0.121569,0.466667,0.705882}%
\pgfsetstrokecolor{currentstroke}%
\pgfsetstrokeopacity{0.666886}%
\pgfsetdash{}{0pt}%
\pgfpathmoveto{\pgfqpoint{3.059578in}{2.178077in}}%
\pgfpathcurveto{\pgfqpoint{3.067814in}{2.178077in}}{\pgfqpoint{3.075714in}{2.181349in}}{\pgfqpoint{3.081538in}{2.187173in}}%
\pgfpathcurveto{\pgfqpoint{3.087362in}{2.192997in}}{\pgfqpoint{3.090634in}{2.200897in}}{\pgfqpoint{3.090634in}{2.209133in}}%
\pgfpathcurveto{\pgfqpoint{3.090634in}{2.217369in}}{\pgfqpoint{3.087362in}{2.225269in}}{\pgfqpoint{3.081538in}{2.231093in}}%
\pgfpathcurveto{\pgfqpoint{3.075714in}{2.236917in}}{\pgfqpoint{3.067814in}{2.240190in}}{\pgfqpoint{3.059578in}{2.240190in}}%
\pgfpathcurveto{\pgfqpoint{3.051341in}{2.240190in}}{\pgfqpoint{3.043441in}{2.236917in}}{\pgfqpoint{3.037617in}{2.231093in}}%
\pgfpathcurveto{\pgfqpoint{3.031794in}{2.225269in}}{\pgfqpoint{3.028521in}{2.217369in}}{\pgfqpoint{3.028521in}{2.209133in}}%
\pgfpathcurveto{\pgfqpoint{3.028521in}{2.200897in}}{\pgfqpoint{3.031794in}{2.192997in}}{\pgfqpoint{3.037617in}{2.187173in}}%
\pgfpathcurveto{\pgfqpoint{3.043441in}{2.181349in}}{\pgfqpoint{3.051341in}{2.178077in}}{\pgfqpoint{3.059578in}{2.178077in}}%
\pgfpathclose%
\pgfusepath{stroke,fill}%
\end{pgfscope}%
\begin{pgfscope}%
\pgfpathrectangle{\pgfqpoint{0.100000in}{0.212622in}}{\pgfqpoint{3.696000in}{3.696000in}}%
\pgfusepath{clip}%
\pgfsetbuttcap%
\pgfsetroundjoin%
\definecolor{currentfill}{rgb}{0.121569,0.466667,0.705882}%
\pgfsetfillcolor{currentfill}%
\pgfsetfillopacity{0.667765}%
\pgfsetlinewidth{1.003750pt}%
\definecolor{currentstroke}{rgb}{0.121569,0.466667,0.705882}%
\pgfsetstrokecolor{currentstroke}%
\pgfsetstrokeopacity{0.667765}%
\pgfsetdash{}{0pt}%
\pgfpathmoveto{\pgfqpoint{3.058182in}{2.177364in}}%
\pgfpathcurveto{\pgfqpoint{3.066419in}{2.177364in}}{\pgfqpoint{3.074319in}{2.180637in}}{\pgfqpoint{3.080143in}{2.186461in}}%
\pgfpathcurveto{\pgfqpoint{3.085967in}{2.192285in}}{\pgfqpoint{3.089239in}{2.200185in}}{\pgfqpoint{3.089239in}{2.208421in}}%
\pgfpathcurveto{\pgfqpoint{3.089239in}{2.216657in}}{\pgfqpoint{3.085967in}{2.224557in}}{\pgfqpoint{3.080143in}{2.230381in}}%
\pgfpathcurveto{\pgfqpoint{3.074319in}{2.236205in}}{\pgfqpoint{3.066419in}{2.239477in}}{\pgfqpoint{3.058182in}{2.239477in}}%
\pgfpathcurveto{\pgfqpoint{3.049946in}{2.239477in}}{\pgfqpoint{3.042046in}{2.236205in}}{\pgfqpoint{3.036222in}{2.230381in}}%
\pgfpathcurveto{\pgfqpoint{3.030398in}{2.224557in}}{\pgfqpoint{3.027126in}{2.216657in}}{\pgfqpoint{3.027126in}{2.208421in}}%
\pgfpathcurveto{\pgfqpoint{3.027126in}{2.200185in}}{\pgfqpoint{3.030398in}{2.192285in}}{\pgfqpoint{3.036222in}{2.186461in}}%
\pgfpathcurveto{\pgfqpoint{3.042046in}{2.180637in}}{\pgfqpoint{3.049946in}{2.177364in}}{\pgfqpoint{3.058182in}{2.177364in}}%
\pgfpathclose%
\pgfusepath{stroke,fill}%
\end{pgfscope}%
\begin{pgfscope}%
\pgfpathrectangle{\pgfqpoint{0.100000in}{0.212622in}}{\pgfqpoint{3.696000in}{3.696000in}}%
\pgfusepath{clip}%
\pgfsetbuttcap%
\pgfsetroundjoin%
\definecolor{currentfill}{rgb}{0.121569,0.466667,0.705882}%
\pgfsetfillcolor{currentfill}%
\pgfsetfillopacity{0.669033}%
\pgfsetlinewidth{1.003750pt}%
\definecolor{currentstroke}{rgb}{0.121569,0.466667,0.705882}%
\pgfsetstrokecolor{currentstroke}%
\pgfsetstrokeopacity{0.669033}%
\pgfsetdash{}{0pt}%
\pgfpathmoveto{\pgfqpoint{3.056338in}{2.176387in}}%
\pgfpathcurveto{\pgfqpoint{3.064574in}{2.176387in}}{\pgfqpoint{3.072474in}{2.179660in}}{\pgfqpoint{3.078298in}{2.185484in}}%
\pgfpathcurveto{\pgfqpoint{3.084122in}{2.191308in}}{\pgfqpoint{3.087394in}{2.199208in}}{\pgfqpoint{3.087394in}{2.207444in}}%
\pgfpathcurveto{\pgfqpoint{3.087394in}{2.215680in}}{\pgfqpoint{3.084122in}{2.223580in}}{\pgfqpoint{3.078298in}{2.229404in}}%
\pgfpathcurveto{\pgfqpoint{3.072474in}{2.235228in}}{\pgfqpoint{3.064574in}{2.238500in}}{\pgfqpoint{3.056338in}{2.238500in}}%
\pgfpathcurveto{\pgfqpoint{3.048102in}{2.238500in}}{\pgfqpoint{3.040202in}{2.235228in}}{\pgfqpoint{3.034378in}{2.229404in}}%
\pgfpathcurveto{\pgfqpoint{3.028554in}{2.223580in}}{\pgfqpoint{3.025281in}{2.215680in}}{\pgfqpoint{3.025281in}{2.207444in}}%
\pgfpathcurveto{\pgfqpoint{3.025281in}{2.199208in}}{\pgfqpoint{3.028554in}{2.191308in}}{\pgfqpoint{3.034378in}{2.185484in}}%
\pgfpathcurveto{\pgfqpoint{3.040202in}{2.179660in}}{\pgfqpoint{3.048102in}{2.176387in}}{\pgfqpoint{3.056338in}{2.176387in}}%
\pgfpathclose%
\pgfusepath{stroke,fill}%
\end{pgfscope}%
\begin{pgfscope}%
\pgfpathrectangle{\pgfqpoint{0.100000in}{0.212622in}}{\pgfqpoint{3.696000in}{3.696000in}}%
\pgfusepath{clip}%
\pgfsetbuttcap%
\pgfsetroundjoin%
\definecolor{currentfill}{rgb}{0.121569,0.466667,0.705882}%
\pgfsetfillcolor{currentfill}%
\pgfsetfillopacity{0.670469}%
\pgfsetlinewidth{1.003750pt}%
\definecolor{currentstroke}{rgb}{0.121569,0.466667,0.705882}%
\pgfsetstrokecolor{currentstroke}%
\pgfsetstrokeopacity{0.670469}%
\pgfsetdash{}{0pt}%
\pgfpathmoveto{\pgfqpoint{3.054361in}{2.175434in}}%
\pgfpathcurveto{\pgfqpoint{3.062597in}{2.175434in}}{\pgfqpoint{3.070497in}{2.178706in}}{\pgfqpoint{3.076321in}{2.184530in}}%
\pgfpathcurveto{\pgfqpoint{3.082145in}{2.190354in}}{\pgfqpoint{3.085418in}{2.198254in}}{\pgfqpoint{3.085418in}{2.206490in}}%
\pgfpathcurveto{\pgfqpoint{3.085418in}{2.214727in}}{\pgfqpoint{3.082145in}{2.222627in}}{\pgfqpoint{3.076321in}{2.228451in}}%
\pgfpathcurveto{\pgfqpoint{3.070497in}{2.234275in}}{\pgfqpoint{3.062597in}{2.237547in}}{\pgfqpoint{3.054361in}{2.237547in}}%
\pgfpathcurveto{\pgfqpoint{3.046125in}{2.237547in}}{\pgfqpoint{3.038225in}{2.234275in}}{\pgfqpoint{3.032401in}{2.228451in}}%
\pgfpathcurveto{\pgfqpoint{3.026577in}{2.222627in}}{\pgfqpoint{3.023305in}{2.214727in}}{\pgfqpoint{3.023305in}{2.206490in}}%
\pgfpathcurveto{\pgfqpoint{3.023305in}{2.198254in}}{\pgfqpoint{3.026577in}{2.190354in}}{\pgfqpoint{3.032401in}{2.184530in}}%
\pgfpathcurveto{\pgfqpoint{3.038225in}{2.178706in}}{\pgfqpoint{3.046125in}{2.175434in}}{\pgfqpoint{3.054361in}{2.175434in}}%
\pgfpathclose%
\pgfusepath{stroke,fill}%
\end{pgfscope}%
\begin{pgfscope}%
\pgfpathrectangle{\pgfqpoint{0.100000in}{0.212622in}}{\pgfqpoint{3.696000in}{3.696000in}}%
\pgfusepath{clip}%
\pgfsetbuttcap%
\pgfsetroundjoin%
\definecolor{currentfill}{rgb}{0.121569,0.466667,0.705882}%
\pgfsetfillcolor{currentfill}%
\pgfsetfillopacity{0.672059}%
\pgfsetlinewidth{1.003750pt}%
\definecolor{currentstroke}{rgb}{0.121569,0.466667,0.705882}%
\pgfsetstrokecolor{currentstroke}%
\pgfsetstrokeopacity{0.672059}%
\pgfsetdash{}{0pt}%
\pgfpathmoveto{\pgfqpoint{3.051872in}{2.174158in}}%
\pgfpathcurveto{\pgfqpoint{3.060109in}{2.174158in}}{\pgfqpoint{3.068009in}{2.177430in}}{\pgfqpoint{3.073833in}{2.183254in}}%
\pgfpathcurveto{\pgfqpoint{3.079657in}{2.189078in}}{\pgfqpoint{3.082929in}{2.196978in}}{\pgfqpoint{3.082929in}{2.205214in}}%
\pgfpathcurveto{\pgfqpoint{3.082929in}{2.213451in}}{\pgfqpoint{3.079657in}{2.221351in}}{\pgfqpoint{3.073833in}{2.227175in}}%
\pgfpathcurveto{\pgfqpoint{3.068009in}{2.232999in}}{\pgfqpoint{3.060109in}{2.236271in}}{\pgfqpoint{3.051872in}{2.236271in}}%
\pgfpathcurveto{\pgfqpoint{3.043636in}{2.236271in}}{\pgfqpoint{3.035736in}{2.232999in}}{\pgfqpoint{3.029912in}{2.227175in}}%
\pgfpathcurveto{\pgfqpoint{3.024088in}{2.221351in}}{\pgfqpoint{3.020816in}{2.213451in}}{\pgfqpoint{3.020816in}{2.205214in}}%
\pgfpathcurveto{\pgfqpoint{3.020816in}{2.196978in}}{\pgfqpoint{3.024088in}{2.189078in}}{\pgfqpoint{3.029912in}{2.183254in}}%
\pgfpathcurveto{\pgfqpoint{3.035736in}{2.177430in}}{\pgfqpoint{3.043636in}{2.174158in}}{\pgfqpoint{3.051872in}{2.174158in}}%
\pgfpathclose%
\pgfusepath{stroke,fill}%
\end{pgfscope}%
\begin{pgfscope}%
\pgfpathrectangle{\pgfqpoint{0.100000in}{0.212622in}}{\pgfqpoint{3.696000in}{3.696000in}}%
\pgfusepath{clip}%
\pgfsetbuttcap%
\pgfsetroundjoin%
\definecolor{currentfill}{rgb}{0.121569,0.466667,0.705882}%
\pgfsetfillcolor{currentfill}%
\pgfsetfillopacity{0.674080}%
\pgfsetlinewidth{1.003750pt}%
\definecolor{currentstroke}{rgb}{0.121569,0.466667,0.705882}%
\pgfsetstrokecolor{currentstroke}%
\pgfsetstrokeopacity{0.674080}%
\pgfsetdash{}{0pt}%
\pgfpathmoveto{\pgfqpoint{3.049057in}{2.172738in}}%
\pgfpathcurveto{\pgfqpoint{3.057293in}{2.172738in}}{\pgfqpoint{3.065193in}{2.176010in}}{\pgfqpoint{3.071017in}{2.181834in}}%
\pgfpathcurveto{\pgfqpoint{3.076841in}{2.187658in}}{\pgfqpoint{3.080113in}{2.195558in}}{\pgfqpoint{3.080113in}{2.203795in}}%
\pgfpathcurveto{\pgfqpoint{3.080113in}{2.212031in}}{\pgfqpoint{3.076841in}{2.219931in}}{\pgfqpoint{3.071017in}{2.225755in}}%
\pgfpathcurveto{\pgfqpoint{3.065193in}{2.231579in}}{\pgfqpoint{3.057293in}{2.234851in}}{\pgfqpoint{3.049057in}{2.234851in}}%
\pgfpathcurveto{\pgfqpoint{3.040821in}{2.234851in}}{\pgfqpoint{3.032921in}{2.231579in}}{\pgfqpoint{3.027097in}{2.225755in}}%
\pgfpathcurveto{\pgfqpoint{3.021273in}{2.219931in}}{\pgfqpoint{3.018000in}{2.212031in}}{\pgfqpoint{3.018000in}{2.203795in}}%
\pgfpathcurveto{\pgfqpoint{3.018000in}{2.195558in}}{\pgfqpoint{3.021273in}{2.187658in}}{\pgfqpoint{3.027097in}{2.181834in}}%
\pgfpathcurveto{\pgfqpoint{3.032921in}{2.176010in}}{\pgfqpoint{3.040821in}{2.172738in}}{\pgfqpoint{3.049057in}{2.172738in}}%
\pgfpathclose%
\pgfusepath{stroke,fill}%
\end{pgfscope}%
\begin{pgfscope}%
\pgfpathrectangle{\pgfqpoint{0.100000in}{0.212622in}}{\pgfqpoint{3.696000in}{3.696000in}}%
\pgfusepath{clip}%
\pgfsetbuttcap%
\pgfsetroundjoin%
\definecolor{currentfill}{rgb}{0.121569,0.466667,0.705882}%
\pgfsetfillcolor{currentfill}%
\pgfsetfillopacity{0.675231}%
\pgfsetlinewidth{1.003750pt}%
\definecolor{currentstroke}{rgb}{0.121569,0.466667,0.705882}%
\pgfsetstrokecolor{currentstroke}%
\pgfsetstrokeopacity{0.675231}%
\pgfsetdash{}{0pt}%
\pgfpathmoveto{\pgfqpoint{3.047449in}{2.172274in}}%
\pgfpathcurveto{\pgfqpoint{3.055685in}{2.172274in}}{\pgfqpoint{3.063585in}{2.175547in}}{\pgfqpoint{3.069409in}{2.181370in}}%
\pgfpathcurveto{\pgfqpoint{3.075233in}{2.187194in}}{\pgfqpoint{3.078505in}{2.195094in}}{\pgfqpoint{3.078505in}{2.203331in}}%
\pgfpathcurveto{\pgfqpoint{3.078505in}{2.211567in}}{\pgfqpoint{3.075233in}{2.219467in}}{\pgfqpoint{3.069409in}{2.225291in}}%
\pgfpathcurveto{\pgfqpoint{3.063585in}{2.231115in}}{\pgfqpoint{3.055685in}{2.234387in}}{\pgfqpoint{3.047449in}{2.234387in}}%
\pgfpathcurveto{\pgfqpoint{3.039212in}{2.234387in}}{\pgfqpoint{3.031312in}{2.231115in}}{\pgfqpoint{3.025488in}{2.225291in}}%
\pgfpathcurveto{\pgfqpoint{3.019664in}{2.219467in}}{\pgfqpoint{3.016392in}{2.211567in}}{\pgfqpoint{3.016392in}{2.203331in}}%
\pgfpathcurveto{\pgfqpoint{3.016392in}{2.195094in}}{\pgfqpoint{3.019664in}{2.187194in}}{\pgfqpoint{3.025488in}{2.181370in}}%
\pgfpathcurveto{\pgfqpoint{3.031312in}{2.175547in}}{\pgfqpoint{3.039212in}{2.172274in}}{\pgfqpoint{3.047449in}{2.172274in}}%
\pgfpathclose%
\pgfusepath{stroke,fill}%
\end{pgfscope}%
\begin{pgfscope}%
\pgfpathrectangle{\pgfqpoint{0.100000in}{0.212622in}}{\pgfqpoint{3.696000in}{3.696000in}}%
\pgfusepath{clip}%
\pgfsetbuttcap%
\pgfsetroundjoin%
\definecolor{currentfill}{rgb}{0.121569,0.466667,0.705882}%
\pgfsetfillcolor{currentfill}%
\pgfsetfillopacity{0.675839}%
\pgfsetlinewidth{1.003750pt}%
\definecolor{currentstroke}{rgb}{0.121569,0.466667,0.705882}%
\pgfsetstrokecolor{currentstroke}%
\pgfsetstrokeopacity{0.675839}%
\pgfsetdash{}{0pt}%
\pgfpathmoveto{\pgfqpoint{3.046505in}{2.171903in}}%
\pgfpathcurveto{\pgfqpoint{3.054741in}{2.171903in}}{\pgfqpoint{3.062641in}{2.175175in}}{\pgfqpoint{3.068465in}{2.180999in}}%
\pgfpathcurveto{\pgfqpoint{3.074289in}{2.186823in}}{\pgfqpoint{3.077561in}{2.194723in}}{\pgfqpoint{3.077561in}{2.202959in}}%
\pgfpathcurveto{\pgfqpoint{3.077561in}{2.211195in}}{\pgfqpoint{3.074289in}{2.219095in}}{\pgfqpoint{3.068465in}{2.224919in}}%
\pgfpathcurveto{\pgfqpoint{3.062641in}{2.230743in}}{\pgfqpoint{3.054741in}{2.234016in}}{\pgfqpoint{3.046505in}{2.234016in}}%
\pgfpathcurveto{\pgfqpoint{3.038269in}{2.234016in}}{\pgfqpoint{3.030368in}{2.230743in}}{\pgfqpoint{3.024545in}{2.224919in}}%
\pgfpathcurveto{\pgfqpoint{3.018721in}{2.219095in}}{\pgfqpoint{3.015448in}{2.211195in}}{\pgfqpoint{3.015448in}{2.202959in}}%
\pgfpathcurveto{\pgfqpoint{3.015448in}{2.194723in}}{\pgfqpoint{3.018721in}{2.186823in}}{\pgfqpoint{3.024545in}{2.180999in}}%
\pgfpathcurveto{\pgfqpoint{3.030368in}{2.175175in}}{\pgfqpoint{3.038269in}{2.171903in}}{\pgfqpoint{3.046505in}{2.171903in}}%
\pgfpathclose%
\pgfusepath{stroke,fill}%
\end{pgfscope}%
\begin{pgfscope}%
\pgfpathrectangle{\pgfqpoint{0.100000in}{0.212622in}}{\pgfqpoint{3.696000in}{3.696000in}}%
\pgfusepath{clip}%
\pgfsetbuttcap%
\pgfsetroundjoin%
\definecolor{currentfill}{rgb}{0.121569,0.466667,0.705882}%
\pgfsetfillcolor{currentfill}%
\pgfsetfillopacity{0.676172}%
\pgfsetlinewidth{1.003750pt}%
\definecolor{currentstroke}{rgb}{0.121569,0.466667,0.705882}%
\pgfsetstrokecolor{currentstroke}%
\pgfsetstrokeopacity{0.676172}%
\pgfsetdash{}{0pt}%
\pgfpathmoveto{\pgfqpoint{3.046023in}{2.171655in}}%
\pgfpathcurveto{\pgfqpoint{3.054259in}{2.171655in}}{\pgfqpoint{3.062160in}{2.174927in}}{\pgfqpoint{3.067983in}{2.180751in}}%
\pgfpathcurveto{\pgfqpoint{3.073807in}{2.186575in}}{\pgfqpoint{3.077080in}{2.194475in}}{\pgfqpoint{3.077080in}{2.202712in}}%
\pgfpathcurveto{\pgfqpoint{3.077080in}{2.210948in}}{\pgfqpoint{3.073807in}{2.218848in}}{\pgfqpoint{3.067983in}{2.224672in}}%
\pgfpathcurveto{\pgfqpoint{3.062160in}{2.230496in}}{\pgfqpoint{3.054259in}{2.233768in}}{\pgfqpoint{3.046023in}{2.233768in}}%
\pgfpathcurveto{\pgfqpoint{3.037787in}{2.233768in}}{\pgfqpoint{3.029887in}{2.230496in}}{\pgfqpoint{3.024063in}{2.224672in}}%
\pgfpathcurveto{\pgfqpoint{3.018239in}{2.218848in}}{\pgfqpoint{3.014967in}{2.210948in}}{\pgfqpoint{3.014967in}{2.202712in}}%
\pgfpathcurveto{\pgfqpoint{3.014967in}{2.194475in}}{\pgfqpoint{3.018239in}{2.186575in}}{\pgfqpoint{3.024063in}{2.180751in}}%
\pgfpathcurveto{\pgfqpoint{3.029887in}{2.174927in}}{\pgfqpoint{3.037787in}{2.171655in}}{\pgfqpoint{3.046023in}{2.171655in}}%
\pgfpathclose%
\pgfusepath{stroke,fill}%
\end{pgfscope}%
\begin{pgfscope}%
\pgfpathrectangle{\pgfqpoint{0.100000in}{0.212622in}}{\pgfqpoint{3.696000in}{3.696000in}}%
\pgfusepath{clip}%
\pgfsetbuttcap%
\pgfsetroundjoin%
\definecolor{currentfill}{rgb}{0.121569,0.466667,0.705882}%
\pgfsetfillcolor{currentfill}%
\pgfsetfillopacity{0.676824}%
\pgfsetlinewidth{1.003750pt}%
\definecolor{currentstroke}{rgb}{0.121569,0.466667,0.705882}%
\pgfsetstrokecolor{currentstroke}%
\pgfsetstrokeopacity{0.676824}%
\pgfsetdash{}{0pt}%
\pgfpathmoveto{\pgfqpoint{3.044929in}{2.171194in}}%
\pgfpathcurveto{\pgfqpoint{3.053165in}{2.171194in}}{\pgfqpoint{3.061065in}{2.174467in}}{\pgfqpoint{3.066889in}{2.180290in}}%
\pgfpathcurveto{\pgfqpoint{3.072713in}{2.186114in}}{\pgfqpoint{3.075985in}{2.194014in}}{\pgfqpoint{3.075985in}{2.202251in}}%
\pgfpathcurveto{\pgfqpoint{3.075985in}{2.210487in}}{\pgfqpoint{3.072713in}{2.218387in}}{\pgfqpoint{3.066889in}{2.224211in}}%
\pgfpathcurveto{\pgfqpoint{3.061065in}{2.230035in}}{\pgfqpoint{3.053165in}{2.233307in}}{\pgfqpoint{3.044929in}{2.233307in}}%
\pgfpathcurveto{\pgfqpoint{3.036693in}{2.233307in}}{\pgfqpoint{3.028793in}{2.230035in}}{\pgfqpoint{3.022969in}{2.224211in}}%
\pgfpathcurveto{\pgfqpoint{3.017145in}{2.218387in}}{\pgfqpoint{3.013872in}{2.210487in}}{\pgfqpoint{3.013872in}{2.202251in}}%
\pgfpathcurveto{\pgfqpoint{3.013872in}{2.194014in}}{\pgfqpoint{3.017145in}{2.186114in}}{\pgfqpoint{3.022969in}{2.180290in}}%
\pgfpathcurveto{\pgfqpoint{3.028793in}{2.174467in}}{\pgfqpoint{3.036693in}{2.171194in}}{\pgfqpoint{3.044929in}{2.171194in}}%
\pgfpathclose%
\pgfusepath{stroke,fill}%
\end{pgfscope}%
\begin{pgfscope}%
\pgfpathrectangle{\pgfqpoint{0.100000in}{0.212622in}}{\pgfqpoint{3.696000in}{3.696000in}}%
\pgfusepath{clip}%
\pgfsetbuttcap%
\pgfsetroundjoin%
\definecolor{currentfill}{rgb}{0.121569,0.466667,0.705882}%
\pgfsetfillcolor{currentfill}%
\pgfsetfillopacity{0.677183}%
\pgfsetlinewidth{1.003750pt}%
\definecolor{currentstroke}{rgb}{0.121569,0.466667,0.705882}%
\pgfsetstrokecolor{currentstroke}%
\pgfsetstrokeopacity{0.677183}%
\pgfsetdash{}{0pt}%
\pgfpathmoveto{\pgfqpoint{3.044309in}{2.170965in}}%
\pgfpathcurveto{\pgfqpoint{3.052545in}{2.170965in}}{\pgfqpoint{3.060445in}{2.174238in}}{\pgfqpoint{3.066269in}{2.180062in}}%
\pgfpathcurveto{\pgfqpoint{3.072093in}{2.185885in}}{\pgfqpoint{3.075365in}{2.193785in}}{\pgfqpoint{3.075365in}{2.202022in}}%
\pgfpathcurveto{\pgfqpoint{3.075365in}{2.210258in}}{\pgfqpoint{3.072093in}{2.218158in}}{\pgfqpoint{3.066269in}{2.223982in}}%
\pgfpathcurveto{\pgfqpoint{3.060445in}{2.229806in}}{\pgfqpoint{3.052545in}{2.233078in}}{\pgfqpoint{3.044309in}{2.233078in}}%
\pgfpathcurveto{\pgfqpoint{3.036072in}{2.233078in}}{\pgfqpoint{3.028172in}{2.229806in}}{\pgfqpoint{3.022348in}{2.223982in}}%
\pgfpathcurveto{\pgfqpoint{3.016524in}{2.218158in}}{\pgfqpoint{3.013252in}{2.210258in}}{\pgfqpoint{3.013252in}{2.202022in}}%
\pgfpathcurveto{\pgfqpoint{3.013252in}{2.193785in}}{\pgfqpoint{3.016524in}{2.185885in}}{\pgfqpoint{3.022348in}{2.180062in}}%
\pgfpathcurveto{\pgfqpoint{3.028172in}{2.174238in}}{\pgfqpoint{3.036072in}{2.170965in}}{\pgfqpoint{3.044309in}{2.170965in}}%
\pgfpathclose%
\pgfusepath{stroke,fill}%
\end{pgfscope}%
\begin{pgfscope}%
\pgfpathrectangle{\pgfqpoint{0.100000in}{0.212622in}}{\pgfqpoint{3.696000in}{3.696000in}}%
\pgfusepath{clip}%
\pgfsetbuttcap%
\pgfsetroundjoin%
\definecolor{currentfill}{rgb}{0.121569,0.466667,0.705882}%
\pgfsetfillcolor{currentfill}%
\pgfsetfillopacity{0.677380}%
\pgfsetlinewidth{1.003750pt}%
\definecolor{currentstroke}{rgb}{0.121569,0.466667,0.705882}%
\pgfsetstrokecolor{currentstroke}%
\pgfsetstrokeopacity{0.677380}%
\pgfsetdash{}{0pt}%
\pgfpathmoveto{\pgfqpoint{3.043987in}{2.170817in}}%
\pgfpathcurveto{\pgfqpoint{3.052224in}{2.170817in}}{\pgfqpoint{3.060124in}{2.174089in}}{\pgfqpoint{3.065948in}{2.179913in}}%
\pgfpathcurveto{\pgfqpoint{3.071772in}{2.185737in}}{\pgfqpoint{3.075044in}{2.193637in}}{\pgfqpoint{3.075044in}{2.201873in}}%
\pgfpathcurveto{\pgfqpoint{3.075044in}{2.210110in}}{\pgfqpoint{3.071772in}{2.218010in}}{\pgfqpoint{3.065948in}{2.223833in}}%
\pgfpathcurveto{\pgfqpoint{3.060124in}{2.229657in}}{\pgfqpoint{3.052224in}{2.232930in}}{\pgfqpoint{3.043987in}{2.232930in}}%
\pgfpathcurveto{\pgfqpoint{3.035751in}{2.232930in}}{\pgfqpoint{3.027851in}{2.229657in}}{\pgfqpoint{3.022027in}{2.223833in}}%
\pgfpathcurveto{\pgfqpoint{3.016203in}{2.218010in}}{\pgfqpoint{3.012931in}{2.210110in}}{\pgfqpoint{3.012931in}{2.201873in}}%
\pgfpathcurveto{\pgfqpoint{3.012931in}{2.193637in}}{\pgfqpoint{3.016203in}{2.185737in}}{\pgfqpoint{3.022027in}{2.179913in}}%
\pgfpathcurveto{\pgfqpoint{3.027851in}{2.174089in}}{\pgfqpoint{3.035751in}{2.170817in}}{\pgfqpoint{3.043987in}{2.170817in}}%
\pgfpathclose%
\pgfusepath{stroke,fill}%
\end{pgfscope}%
\begin{pgfscope}%
\pgfpathrectangle{\pgfqpoint{0.100000in}{0.212622in}}{\pgfqpoint{3.696000in}{3.696000in}}%
\pgfusepath{clip}%
\pgfsetbuttcap%
\pgfsetroundjoin%
\definecolor{currentfill}{rgb}{0.121569,0.466667,0.705882}%
\pgfsetfillcolor{currentfill}%
\pgfsetfillopacity{0.677854}%
\pgfsetlinewidth{1.003750pt}%
\definecolor{currentstroke}{rgb}{0.121569,0.466667,0.705882}%
\pgfsetstrokecolor{currentstroke}%
\pgfsetstrokeopacity{0.677854}%
\pgfsetdash{}{0pt}%
\pgfpathmoveto{\pgfqpoint{3.043325in}{2.170478in}}%
\pgfpathcurveto{\pgfqpoint{3.051561in}{2.170478in}}{\pgfqpoint{3.059461in}{2.173750in}}{\pgfqpoint{3.065285in}{2.179574in}}%
\pgfpathcurveto{\pgfqpoint{3.071109in}{2.185398in}}{\pgfqpoint{3.074381in}{2.193298in}}{\pgfqpoint{3.074381in}{2.201534in}}%
\pgfpathcurveto{\pgfqpoint{3.074381in}{2.209771in}}{\pgfqpoint{3.071109in}{2.217671in}}{\pgfqpoint{3.065285in}{2.223495in}}%
\pgfpathcurveto{\pgfqpoint{3.059461in}{2.229319in}}{\pgfqpoint{3.051561in}{2.232591in}}{\pgfqpoint{3.043325in}{2.232591in}}%
\pgfpathcurveto{\pgfqpoint{3.035089in}{2.232591in}}{\pgfqpoint{3.027189in}{2.229319in}}{\pgfqpoint{3.021365in}{2.223495in}}%
\pgfpathcurveto{\pgfqpoint{3.015541in}{2.217671in}}{\pgfqpoint{3.012268in}{2.209771in}}{\pgfqpoint{3.012268in}{2.201534in}}%
\pgfpathcurveto{\pgfqpoint{3.012268in}{2.193298in}}{\pgfqpoint{3.015541in}{2.185398in}}{\pgfqpoint{3.021365in}{2.179574in}}%
\pgfpathcurveto{\pgfqpoint{3.027189in}{2.173750in}}{\pgfqpoint{3.035089in}{2.170478in}}{\pgfqpoint{3.043325in}{2.170478in}}%
\pgfpathclose%
\pgfusepath{stroke,fill}%
\end{pgfscope}%
\begin{pgfscope}%
\pgfpathrectangle{\pgfqpoint{0.100000in}{0.212622in}}{\pgfqpoint{3.696000in}{3.696000in}}%
\pgfusepath{clip}%
\pgfsetbuttcap%
\pgfsetroundjoin%
\definecolor{currentfill}{rgb}{0.121569,0.466667,0.705882}%
\pgfsetfillcolor{currentfill}%
\pgfsetfillopacity{0.678107}%
\pgfsetlinewidth{1.003750pt}%
\definecolor{currentstroke}{rgb}{0.121569,0.466667,0.705882}%
\pgfsetstrokecolor{currentstroke}%
\pgfsetstrokeopacity{0.678107}%
\pgfsetdash{}{0pt}%
\pgfpathmoveto{\pgfqpoint{3.042937in}{2.170267in}}%
\pgfpathcurveto{\pgfqpoint{3.051173in}{2.170267in}}{\pgfqpoint{3.059073in}{2.173539in}}{\pgfqpoint{3.064897in}{2.179363in}}%
\pgfpathcurveto{\pgfqpoint{3.070721in}{2.185187in}}{\pgfqpoint{3.073993in}{2.193087in}}{\pgfqpoint{3.073993in}{2.201324in}}%
\pgfpathcurveto{\pgfqpoint{3.073993in}{2.209560in}}{\pgfqpoint{3.070721in}{2.217460in}}{\pgfqpoint{3.064897in}{2.223284in}}%
\pgfpathcurveto{\pgfqpoint{3.059073in}{2.229108in}}{\pgfqpoint{3.051173in}{2.232380in}}{\pgfqpoint{3.042937in}{2.232380in}}%
\pgfpathcurveto{\pgfqpoint{3.034700in}{2.232380in}}{\pgfqpoint{3.026800in}{2.229108in}}{\pgfqpoint{3.020976in}{2.223284in}}%
\pgfpathcurveto{\pgfqpoint{3.015152in}{2.217460in}}{\pgfqpoint{3.011880in}{2.209560in}}{\pgfqpoint{3.011880in}{2.201324in}}%
\pgfpathcurveto{\pgfqpoint{3.011880in}{2.193087in}}{\pgfqpoint{3.015152in}{2.185187in}}{\pgfqpoint{3.020976in}{2.179363in}}%
\pgfpathcurveto{\pgfqpoint{3.026800in}{2.173539in}}{\pgfqpoint{3.034700in}{2.170267in}}{\pgfqpoint{3.042937in}{2.170267in}}%
\pgfpathclose%
\pgfusepath{stroke,fill}%
\end{pgfscope}%
\begin{pgfscope}%
\pgfpathrectangle{\pgfqpoint{0.100000in}{0.212622in}}{\pgfqpoint{3.696000in}{3.696000in}}%
\pgfusepath{clip}%
\pgfsetbuttcap%
\pgfsetroundjoin%
\definecolor{currentfill}{rgb}{0.121569,0.466667,0.705882}%
\pgfsetfillcolor{currentfill}%
\pgfsetfillopacity{0.678248}%
\pgfsetlinewidth{1.003750pt}%
\definecolor{currentstroke}{rgb}{0.121569,0.466667,0.705882}%
\pgfsetstrokecolor{currentstroke}%
\pgfsetstrokeopacity{0.678248}%
\pgfsetdash{}{0pt}%
\pgfpathmoveto{\pgfqpoint{3.042735in}{2.170146in}}%
\pgfpathcurveto{\pgfqpoint{3.050971in}{2.170146in}}{\pgfqpoint{3.058871in}{2.173418in}}{\pgfqpoint{3.064695in}{2.179242in}}%
\pgfpathcurveto{\pgfqpoint{3.070519in}{2.185066in}}{\pgfqpoint{3.073791in}{2.192966in}}{\pgfqpoint{3.073791in}{2.201202in}}%
\pgfpathcurveto{\pgfqpoint{3.073791in}{2.209439in}}{\pgfqpoint{3.070519in}{2.217339in}}{\pgfqpoint{3.064695in}{2.223163in}}%
\pgfpathcurveto{\pgfqpoint{3.058871in}{2.228986in}}{\pgfqpoint{3.050971in}{2.232259in}}{\pgfqpoint{3.042735in}{2.232259in}}%
\pgfpathcurveto{\pgfqpoint{3.034498in}{2.232259in}}{\pgfqpoint{3.026598in}{2.228986in}}{\pgfqpoint{3.020774in}{2.223163in}}%
\pgfpathcurveto{\pgfqpoint{3.014950in}{2.217339in}}{\pgfqpoint{3.011678in}{2.209439in}}{\pgfqpoint{3.011678in}{2.201202in}}%
\pgfpathcurveto{\pgfqpoint{3.011678in}{2.192966in}}{\pgfqpoint{3.014950in}{2.185066in}}{\pgfqpoint{3.020774in}{2.179242in}}%
\pgfpathcurveto{\pgfqpoint{3.026598in}{2.173418in}}{\pgfqpoint{3.034498in}{2.170146in}}{\pgfqpoint{3.042735in}{2.170146in}}%
\pgfpathclose%
\pgfusepath{stroke,fill}%
\end{pgfscope}%
\begin{pgfscope}%
\pgfpathrectangle{\pgfqpoint{0.100000in}{0.212622in}}{\pgfqpoint{3.696000in}{3.696000in}}%
\pgfusepath{clip}%
\pgfsetbuttcap%
\pgfsetroundjoin%
\definecolor{currentfill}{rgb}{0.121569,0.466667,0.705882}%
\pgfsetfillcolor{currentfill}%
\pgfsetfillopacity{0.678612}%
\pgfsetlinewidth{1.003750pt}%
\definecolor{currentstroke}{rgb}{0.121569,0.466667,0.705882}%
\pgfsetstrokecolor{currentstroke}%
\pgfsetstrokeopacity{0.678612}%
\pgfsetdash{}{0pt}%
\pgfpathmoveto{\pgfqpoint{3.042174in}{2.169856in}}%
\pgfpathcurveto{\pgfqpoint{3.050410in}{2.169856in}}{\pgfqpoint{3.058310in}{2.173129in}}{\pgfqpoint{3.064134in}{2.178952in}}%
\pgfpathcurveto{\pgfqpoint{3.069958in}{2.184776in}}{\pgfqpoint{3.073230in}{2.192676in}}{\pgfqpoint{3.073230in}{2.200913in}}%
\pgfpathcurveto{\pgfqpoint{3.073230in}{2.209149in}}{\pgfqpoint{3.069958in}{2.217049in}}{\pgfqpoint{3.064134in}{2.222873in}}%
\pgfpathcurveto{\pgfqpoint{3.058310in}{2.228697in}}{\pgfqpoint{3.050410in}{2.231969in}}{\pgfqpoint{3.042174in}{2.231969in}}%
\pgfpathcurveto{\pgfqpoint{3.033938in}{2.231969in}}{\pgfqpoint{3.026038in}{2.228697in}}{\pgfqpoint{3.020214in}{2.222873in}}%
\pgfpathcurveto{\pgfqpoint{3.014390in}{2.217049in}}{\pgfqpoint{3.011117in}{2.209149in}}{\pgfqpoint{3.011117in}{2.200913in}}%
\pgfpathcurveto{\pgfqpoint{3.011117in}{2.192676in}}{\pgfqpoint{3.014390in}{2.184776in}}{\pgfqpoint{3.020214in}{2.178952in}}%
\pgfpathcurveto{\pgfqpoint{3.026038in}{2.173129in}}{\pgfqpoint{3.033938in}{2.169856in}}{\pgfqpoint{3.042174in}{2.169856in}}%
\pgfpathclose%
\pgfusepath{stroke,fill}%
\end{pgfscope}%
\begin{pgfscope}%
\pgfpathrectangle{\pgfqpoint{0.100000in}{0.212622in}}{\pgfqpoint{3.696000in}{3.696000in}}%
\pgfusepath{clip}%
\pgfsetbuttcap%
\pgfsetroundjoin%
\definecolor{currentfill}{rgb}{0.121569,0.466667,0.705882}%
\pgfsetfillcolor{currentfill}%
\pgfsetfillopacity{0.678819}%
\pgfsetlinewidth{1.003750pt}%
\definecolor{currentstroke}{rgb}{0.121569,0.466667,0.705882}%
\pgfsetstrokecolor{currentstroke}%
\pgfsetstrokeopacity{0.678819}%
\pgfsetdash{}{0pt}%
\pgfpathmoveto{\pgfqpoint{3.041853in}{2.169759in}}%
\pgfpathcurveto{\pgfqpoint{3.050089in}{2.169759in}}{\pgfqpoint{3.057990in}{2.173032in}}{\pgfqpoint{3.063813in}{2.178855in}}%
\pgfpathcurveto{\pgfqpoint{3.069637in}{2.184679in}}{\pgfqpoint{3.072910in}{2.192579in}}{\pgfqpoint{3.072910in}{2.200816in}}%
\pgfpathcurveto{\pgfqpoint{3.072910in}{2.209052in}}{\pgfqpoint{3.069637in}{2.216952in}}{\pgfqpoint{3.063813in}{2.222776in}}%
\pgfpathcurveto{\pgfqpoint{3.057990in}{2.228600in}}{\pgfqpoint{3.050089in}{2.231872in}}{\pgfqpoint{3.041853in}{2.231872in}}%
\pgfpathcurveto{\pgfqpoint{3.033617in}{2.231872in}}{\pgfqpoint{3.025717in}{2.228600in}}{\pgfqpoint{3.019893in}{2.222776in}}%
\pgfpathcurveto{\pgfqpoint{3.014069in}{2.216952in}}{\pgfqpoint{3.010797in}{2.209052in}}{\pgfqpoint{3.010797in}{2.200816in}}%
\pgfpathcurveto{\pgfqpoint{3.010797in}{2.192579in}}{\pgfqpoint{3.014069in}{2.184679in}}{\pgfqpoint{3.019893in}{2.178855in}}%
\pgfpathcurveto{\pgfqpoint{3.025717in}{2.173032in}}{\pgfqpoint{3.033617in}{2.169759in}}{\pgfqpoint{3.041853in}{2.169759in}}%
\pgfpathclose%
\pgfusepath{stroke,fill}%
\end{pgfscope}%
\begin{pgfscope}%
\pgfpathrectangle{\pgfqpoint{0.100000in}{0.212622in}}{\pgfqpoint{3.696000in}{3.696000in}}%
\pgfusepath{clip}%
\pgfsetbuttcap%
\pgfsetroundjoin%
\definecolor{currentfill}{rgb}{0.121569,0.466667,0.705882}%
\pgfsetfillcolor{currentfill}%
\pgfsetfillopacity{0.678935}%
\pgfsetlinewidth{1.003750pt}%
\definecolor{currentstroke}{rgb}{0.121569,0.466667,0.705882}%
\pgfsetstrokecolor{currentstroke}%
\pgfsetstrokeopacity{0.678935}%
\pgfsetdash{}{0pt}%
\pgfpathmoveto{\pgfqpoint{3.041690in}{2.169705in}}%
\pgfpathcurveto{\pgfqpoint{3.049926in}{2.169705in}}{\pgfqpoint{3.057826in}{2.172977in}}{\pgfqpoint{3.063650in}{2.178801in}}%
\pgfpathcurveto{\pgfqpoint{3.069474in}{2.184625in}}{\pgfqpoint{3.072746in}{2.192525in}}{\pgfqpoint{3.072746in}{2.200761in}}%
\pgfpathcurveto{\pgfqpoint{3.072746in}{2.208997in}}{\pgfqpoint{3.069474in}{2.216897in}}{\pgfqpoint{3.063650in}{2.222721in}}%
\pgfpathcurveto{\pgfqpoint{3.057826in}{2.228545in}}{\pgfqpoint{3.049926in}{2.231818in}}{\pgfqpoint{3.041690in}{2.231818in}}%
\pgfpathcurveto{\pgfqpoint{3.033454in}{2.231818in}}{\pgfqpoint{3.025553in}{2.228545in}}{\pgfqpoint{3.019730in}{2.222721in}}%
\pgfpathcurveto{\pgfqpoint{3.013906in}{2.216897in}}{\pgfqpoint{3.010633in}{2.208997in}}{\pgfqpoint{3.010633in}{2.200761in}}%
\pgfpathcurveto{\pgfqpoint{3.010633in}{2.192525in}}{\pgfqpoint{3.013906in}{2.184625in}}{\pgfqpoint{3.019730in}{2.178801in}}%
\pgfpathcurveto{\pgfqpoint{3.025553in}{2.172977in}}{\pgfqpoint{3.033454in}{2.169705in}}{\pgfqpoint{3.041690in}{2.169705in}}%
\pgfpathclose%
\pgfusepath{stroke,fill}%
\end{pgfscope}%
\begin{pgfscope}%
\pgfpathrectangle{\pgfqpoint{0.100000in}{0.212622in}}{\pgfqpoint{3.696000in}{3.696000in}}%
\pgfusepath{clip}%
\pgfsetbuttcap%
\pgfsetroundjoin%
\definecolor{currentfill}{rgb}{0.121569,0.466667,0.705882}%
\pgfsetfillcolor{currentfill}%
\pgfsetfillopacity{0.678997}%
\pgfsetlinewidth{1.003750pt}%
\definecolor{currentstroke}{rgb}{0.121569,0.466667,0.705882}%
\pgfsetstrokecolor{currentstroke}%
\pgfsetstrokeopacity{0.678997}%
\pgfsetdash{}{0pt}%
\pgfpathmoveto{\pgfqpoint{3.041593in}{2.169664in}}%
\pgfpathcurveto{\pgfqpoint{3.049830in}{2.169664in}}{\pgfqpoint{3.057730in}{2.172936in}}{\pgfqpoint{3.063554in}{2.178760in}}%
\pgfpathcurveto{\pgfqpoint{3.069378in}{2.184584in}}{\pgfqpoint{3.072650in}{2.192484in}}{\pgfqpoint{3.072650in}{2.200720in}}%
\pgfpathcurveto{\pgfqpoint{3.072650in}{2.208957in}}{\pgfqpoint{3.069378in}{2.216857in}}{\pgfqpoint{3.063554in}{2.222681in}}%
\pgfpathcurveto{\pgfqpoint{3.057730in}{2.228505in}}{\pgfqpoint{3.049830in}{2.231777in}}{\pgfqpoint{3.041593in}{2.231777in}}%
\pgfpathcurveto{\pgfqpoint{3.033357in}{2.231777in}}{\pgfqpoint{3.025457in}{2.228505in}}{\pgfqpoint{3.019633in}{2.222681in}}%
\pgfpathcurveto{\pgfqpoint{3.013809in}{2.216857in}}{\pgfqpoint{3.010537in}{2.208957in}}{\pgfqpoint{3.010537in}{2.200720in}}%
\pgfpathcurveto{\pgfqpoint{3.010537in}{2.192484in}}{\pgfqpoint{3.013809in}{2.184584in}}{\pgfqpoint{3.019633in}{2.178760in}}%
\pgfpathcurveto{\pgfqpoint{3.025457in}{2.172936in}}{\pgfqpoint{3.033357in}{2.169664in}}{\pgfqpoint{3.041593in}{2.169664in}}%
\pgfpathclose%
\pgfusepath{stroke,fill}%
\end{pgfscope}%
\begin{pgfscope}%
\pgfpathrectangle{\pgfqpoint{0.100000in}{0.212622in}}{\pgfqpoint{3.696000in}{3.696000in}}%
\pgfusepath{clip}%
\pgfsetbuttcap%
\pgfsetroundjoin%
\definecolor{currentfill}{rgb}{0.121569,0.466667,0.705882}%
\pgfsetfillcolor{currentfill}%
\pgfsetfillopacity{0.679420}%
\pgfsetlinewidth{1.003750pt}%
\definecolor{currentstroke}{rgb}{0.121569,0.466667,0.705882}%
\pgfsetstrokecolor{currentstroke}%
\pgfsetstrokeopacity{0.679420}%
\pgfsetdash{}{0pt}%
\pgfpathmoveto{\pgfqpoint{3.040956in}{2.169315in}}%
\pgfpathcurveto{\pgfqpoint{3.049192in}{2.169315in}}{\pgfqpoint{3.057093in}{2.172587in}}{\pgfqpoint{3.062916in}{2.178411in}}%
\pgfpathcurveto{\pgfqpoint{3.068740in}{2.184235in}}{\pgfqpoint{3.072013in}{2.192135in}}{\pgfqpoint{3.072013in}{2.200372in}}%
\pgfpathcurveto{\pgfqpoint{3.072013in}{2.208608in}}{\pgfqpoint{3.068740in}{2.216508in}}{\pgfqpoint{3.062916in}{2.222332in}}%
\pgfpathcurveto{\pgfqpoint{3.057093in}{2.228156in}}{\pgfqpoint{3.049192in}{2.231428in}}{\pgfqpoint{3.040956in}{2.231428in}}%
\pgfpathcurveto{\pgfqpoint{3.032720in}{2.231428in}}{\pgfqpoint{3.024820in}{2.228156in}}{\pgfqpoint{3.018996in}{2.222332in}}%
\pgfpathcurveto{\pgfqpoint{3.013172in}{2.216508in}}{\pgfqpoint{3.009900in}{2.208608in}}{\pgfqpoint{3.009900in}{2.200372in}}%
\pgfpathcurveto{\pgfqpoint{3.009900in}{2.192135in}}{\pgfqpoint{3.013172in}{2.184235in}}{\pgfqpoint{3.018996in}{2.178411in}}%
\pgfpathcurveto{\pgfqpoint{3.024820in}{2.172587in}}{\pgfqpoint{3.032720in}{2.169315in}}{\pgfqpoint{3.040956in}{2.169315in}}%
\pgfpathclose%
\pgfusepath{stroke,fill}%
\end{pgfscope}%
\begin{pgfscope}%
\pgfpathrectangle{\pgfqpoint{0.100000in}{0.212622in}}{\pgfqpoint{3.696000in}{3.696000in}}%
\pgfusepath{clip}%
\pgfsetbuttcap%
\pgfsetroundjoin%
\definecolor{currentfill}{rgb}{0.121569,0.466667,0.705882}%
\pgfsetfillcolor{currentfill}%
\pgfsetfillopacity{0.680073}%
\pgfsetlinewidth{1.003750pt}%
\definecolor{currentstroke}{rgb}{0.121569,0.466667,0.705882}%
\pgfsetstrokecolor{currentstroke}%
\pgfsetstrokeopacity{0.680073}%
\pgfsetdash{}{0pt}%
\pgfpathmoveto{\pgfqpoint{3.040058in}{2.168997in}}%
\pgfpathcurveto{\pgfqpoint{3.048294in}{2.168997in}}{\pgfqpoint{3.056194in}{2.172270in}}{\pgfqpoint{3.062018in}{2.178094in}}%
\pgfpathcurveto{\pgfqpoint{3.067842in}{2.183918in}}{\pgfqpoint{3.071114in}{2.191818in}}{\pgfqpoint{3.071114in}{2.200054in}}%
\pgfpathcurveto{\pgfqpoint{3.071114in}{2.208290in}}{\pgfqpoint{3.067842in}{2.216190in}}{\pgfqpoint{3.062018in}{2.222014in}}%
\pgfpathcurveto{\pgfqpoint{3.056194in}{2.227838in}}{\pgfqpoint{3.048294in}{2.231110in}}{\pgfqpoint{3.040058in}{2.231110in}}%
\pgfpathcurveto{\pgfqpoint{3.031822in}{2.231110in}}{\pgfqpoint{3.023922in}{2.227838in}}{\pgfqpoint{3.018098in}{2.222014in}}%
\pgfpathcurveto{\pgfqpoint{3.012274in}{2.216190in}}{\pgfqpoint{3.009001in}{2.208290in}}{\pgfqpoint{3.009001in}{2.200054in}}%
\pgfpathcurveto{\pgfqpoint{3.009001in}{2.191818in}}{\pgfqpoint{3.012274in}{2.183918in}}{\pgfqpoint{3.018098in}{2.178094in}}%
\pgfpathcurveto{\pgfqpoint{3.023922in}{2.172270in}}{\pgfqpoint{3.031822in}{2.168997in}}{\pgfqpoint{3.040058in}{2.168997in}}%
\pgfpathclose%
\pgfusepath{stroke,fill}%
\end{pgfscope}%
\begin{pgfscope}%
\pgfpathrectangle{\pgfqpoint{0.100000in}{0.212622in}}{\pgfqpoint{3.696000in}{3.696000in}}%
\pgfusepath{clip}%
\pgfsetbuttcap%
\pgfsetroundjoin%
\definecolor{currentfill}{rgb}{0.121569,0.466667,0.705882}%
\pgfsetfillcolor{currentfill}%
\pgfsetfillopacity{0.680833}%
\pgfsetlinewidth{1.003750pt}%
\definecolor{currentstroke}{rgb}{0.121569,0.466667,0.705882}%
\pgfsetstrokecolor{currentstroke}%
\pgfsetstrokeopacity{0.680833}%
\pgfsetdash{}{0pt}%
\pgfpathmoveto{\pgfqpoint{3.038893in}{2.168353in}}%
\pgfpathcurveto{\pgfqpoint{3.047130in}{2.168353in}}{\pgfqpoint{3.055030in}{2.171625in}}{\pgfqpoint{3.060854in}{2.177449in}}%
\pgfpathcurveto{\pgfqpoint{3.066678in}{2.183273in}}{\pgfqpoint{3.069950in}{2.191173in}}{\pgfqpoint{3.069950in}{2.199409in}}%
\pgfpathcurveto{\pgfqpoint{3.069950in}{2.207645in}}{\pgfqpoint{3.066678in}{2.215545in}}{\pgfqpoint{3.060854in}{2.221369in}}%
\pgfpathcurveto{\pgfqpoint{3.055030in}{2.227193in}}{\pgfqpoint{3.047130in}{2.230466in}}{\pgfqpoint{3.038893in}{2.230466in}}%
\pgfpathcurveto{\pgfqpoint{3.030657in}{2.230466in}}{\pgfqpoint{3.022757in}{2.227193in}}{\pgfqpoint{3.016933in}{2.221369in}}%
\pgfpathcurveto{\pgfqpoint{3.011109in}{2.215545in}}{\pgfqpoint{3.007837in}{2.207645in}}{\pgfqpoint{3.007837in}{2.199409in}}%
\pgfpathcurveto{\pgfqpoint{3.007837in}{2.191173in}}{\pgfqpoint{3.011109in}{2.183273in}}{\pgfqpoint{3.016933in}{2.177449in}}%
\pgfpathcurveto{\pgfqpoint{3.022757in}{2.171625in}}{\pgfqpoint{3.030657in}{2.168353in}}{\pgfqpoint{3.038893in}{2.168353in}}%
\pgfpathclose%
\pgfusepath{stroke,fill}%
\end{pgfscope}%
\begin{pgfscope}%
\pgfpathrectangle{\pgfqpoint{0.100000in}{0.212622in}}{\pgfqpoint{3.696000in}{3.696000in}}%
\pgfusepath{clip}%
\pgfsetbuttcap%
\pgfsetroundjoin%
\definecolor{currentfill}{rgb}{0.121569,0.466667,0.705882}%
\pgfsetfillcolor{currentfill}%
\pgfsetfillopacity{0.682009}%
\pgfsetlinewidth{1.003750pt}%
\definecolor{currentstroke}{rgb}{0.121569,0.466667,0.705882}%
\pgfsetstrokecolor{currentstroke}%
\pgfsetstrokeopacity{0.682009}%
\pgfsetdash{}{0pt}%
\pgfpathmoveto{\pgfqpoint{3.037310in}{2.167427in}}%
\pgfpathcurveto{\pgfqpoint{3.045546in}{2.167427in}}{\pgfqpoint{3.053446in}{2.170699in}}{\pgfqpoint{3.059270in}{2.176523in}}%
\pgfpathcurveto{\pgfqpoint{3.065094in}{2.182347in}}{\pgfqpoint{3.068366in}{2.190247in}}{\pgfqpoint{3.068366in}{2.198484in}}%
\pgfpathcurveto{\pgfqpoint{3.068366in}{2.206720in}}{\pgfqpoint{3.065094in}{2.214620in}}{\pgfqpoint{3.059270in}{2.220444in}}%
\pgfpathcurveto{\pgfqpoint{3.053446in}{2.226268in}}{\pgfqpoint{3.045546in}{2.229540in}}{\pgfqpoint{3.037310in}{2.229540in}}%
\pgfpathcurveto{\pgfqpoint{3.029073in}{2.229540in}}{\pgfqpoint{3.021173in}{2.226268in}}{\pgfqpoint{3.015349in}{2.220444in}}%
\pgfpathcurveto{\pgfqpoint{3.009525in}{2.214620in}}{\pgfqpoint{3.006253in}{2.206720in}}{\pgfqpoint{3.006253in}{2.198484in}}%
\pgfpathcurveto{\pgfqpoint{3.006253in}{2.190247in}}{\pgfqpoint{3.009525in}{2.182347in}}{\pgfqpoint{3.015349in}{2.176523in}}%
\pgfpathcurveto{\pgfqpoint{3.021173in}{2.170699in}}{\pgfqpoint{3.029073in}{2.167427in}}{\pgfqpoint{3.037310in}{2.167427in}}%
\pgfpathclose%
\pgfusepath{stroke,fill}%
\end{pgfscope}%
\begin{pgfscope}%
\pgfpathrectangle{\pgfqpoint{0.100000in}{0.212622in}}{\pgfqpoint{3.696000in}{3.696000in}}%
\pgfusepath{clip}%
\pgfsetbuttcap%
\pgfsetroundjoin%
\definecolor{currentfill}{rgb}{0.121569,0.466667,0.705882}%
\pgfsetfillcolor{currentfill}%
\pgfsetfillopacity{0.683342}%
\pgfsetlinewidth{1.003750pt}%
\definecolor{currentstroke}{rgb}{0.121569,0.466667,0.705882}%
\pgfsetstrokecolor{currentstroke}%
\pgfsetstrokeopacity{0.683342}%
\pgfsetdash{}{0pt}%
\pgfpathmoveto{\pgfqpoint{3.035222in}{2.166590in}}%
\pgfpathcurveto{\pgfqpoint{3.043458in}{2.166590in}}{\pgfqpoint{3.051358in}{2.169863in}}{\pgfqpoint{3.057182in}{2.175687in}}%
\pgfpathcurveto{\pgfqpoint{3.063006in}{2.181511in}}{\pgfqpoint{3.066279in}{2.189411in}}{\pgfqpoint{3.066279in}{2.197647in}}%
\pgfpathcurveto{\pgfqpoint{3.066279in}{2.205883in}}{\pgfqpoint{3.063006in}{2.213783in}}{\pgfqpoint{3.057182in}{2.219607in}}%
\pgfpathcurveto{\pgfqpoint{3.051358in}{2.225431in}}{\pgfqpoint{3.043458in}{2.228703in}}{\pgfqpoint{3.035222in}{2.228703in}}%
\pgfpathcurveto{\pgfqpoint{3.026986in}{2.228703in}}{\pgfqpoint{3.019086in}{2.225431in}}{\pgfqpoint{3.013262in}{2.219607in}}%
\pgfpathcurveto{\pgfqpoint{3.007438in}{2.213783in}}{\pgfqpoint{3.004166in}{2.205883in}}{\pgfqpoint{3.004166in}{2.197647in}}%
\pgfpathcurveto{\pgfqpoint{3.004166in}{2.189411in}}{\pgfqpoint{3.007438in}{2.181511in}}{\pgfqpoint{3.013262in}{2.175687in}}%
\pgfpathcurveto{\pgfqpoint{3.019086in}{2.169863in}}{\pgfqpoint{3.026986in}{2.166590in}}{\pgfqpoint{3.035222in}{2.166590in}}%
\pgfpathclose%
\pgfusepath{stroke,fill}%
\end{pgfscope}%
\begin{pgfscope}%
\pgfpathrectangle{\pgfqpoint{0.100000in}{0.212622in}}{\pgfqpoint{3.696000in}{3.696000in}}%
\pgfusepath{clip}%
\pgfsetbuttcap%
\pgfsetroundjoin%
\definecolor{currentfill}{rgb}{0.121569,0.466667,0.705882}%
\pgfsetfillcolor{currentfill}%
\pgfsetfillopacity{0.684062}%
\pgfsetlinewidth{1.003750pt}%
\definecolor{currentstroke}{rgb}{0.121569,0.466667,0.705882}%
\pgfsetstrokecolor{currentstroke}%
\pgfsetstrokeopacity{0.684062}%
\pgfsetdash{}{0pt}%
\pgfpathmoveto{\pgfqpoint{3.034038in}{2.166082in}}%
\pgfpathcurveto{\pgfqpoint{3.042275in}{2.166082in}}{\pgfqpoint{3.050175in}{2.169354in}}{\pgfqpoint{3.055999in}{2.175178in}}%
\pgfpathcurveto{\pgfqpoint{3.061823in}{2.181002in}}{\pgfqpoint{3.065095in}{2.188902in}}{\pgfqpoint{3.065095in}{2.197138in}}%
\pgfpathcurveto{\pgfqpoint{3.065095in}{2.205374in}}{\pgfqpoint{3.061823in}{2.213274in}}{\pgfqpoint{3.055999in}{2.219098in}}%
\pgfpathcurveto{\pgfqpoint{3.050175in}{2.224922in}}{\pgfqpoint{3.042275in}{2.228195in}}{\pgfqpoint{3.034038in}{2.228195in}}%
\pgfpathcurveto{\pgfqpoint{3.025802in}{2.228195in}}{\pgfqpoint{3.017902in}{2.224922in}}{\pgfqpoint{3.012078in}{2.219098in}}%
\pgfpathcurveto{\pgfqpoint{3.006254in}{2.213274in}}{\pgfqpoint{3.002982in}{2.205374in}}{\pgfqpoint{3.002982in}{2.197138in}}%
\pgfpathcurveto{\pgfqpoint{3.002982in}{2.188902in}}{\pgfqpoint{3.006254in}{2.181002in}}{\pgfqpoint{3.012078in}{2.175178in}}%
\pgfpathcurveto{\pgfqpoint{3.017902in}{2.169354in}}{\pgfqpoint{3.025802in}{2.166082in}}{\pgfqpoint{3.034038in}{2.166082in}}%
\pgfpathclose%
\pgfusepath{stroke,fill}%
\end{pgfscope}%
\begin{pgfscope}%
\pgfpathrectangle{\pgfqpoint{0.100000in}{0.212622in}}{\pgfqpoint{3.696000in}{3.696000in}}%
\pgfusepath{clip}%
\pgfsetbuttcap%
\pgfsetroundjoin%
\definecolor{currentfill}{rgb}{0.121569,0.466667,0.705882}%
\pgfsetfillcolor{currentfill}%
\pgfsetfillopacity{0.684988}%
\pgfsetlinewidth{1.003750pt}%
\definecolor{currentstroke}{rgb}{0.121569,0.466667,0.705882}%
\pgfsetstrokecolor{currentstroke}%
\pgfsetstrokeopacity{0.684988}%
\pgfsetdash{}{0pt}%
\pgfpathmoveto{\pgfqpoint{3.032649in}{2.165349in}}%
\pgfpathcurveto{\pgfqpoint{3.040886in}{2.165349in}}{\pgfqpoint{3.048786in}{2.168621in}}{\pgfqpoint{3.054610in}{2.174445in}}%
\pgfpathcurveto{\pgfqpoint{3.060434in}{2.180269in}}{\pgfqpoint{3.063706in}{2.188169in}}{\pgfqpoint{3.063706in}{2.196406in}}%
\pgfpathcurveto{\pgfqpoint{3.063706in}{2.204642in}}{\pgfqpoint{3.060434in}{2.212542in}}{\pgfqpoint{3.054610in}{2.218366in}}%
\pgfpathcurveto{\pgfqpoint{3.048786in}{2.224190in}}{\pgfqpoint{3.040886in}{2.227462in}}{\pgfqpoint{3.032649in}{2.227462in}}%
\pgfpathcurveto{\pgfqpoint{3.024413in}{2.227462in}}{\pgfqpoint{3.016513in}{2.224190in}}{\pgfqpoint{3.010689in}{2.218366in}}%
\pgfpathcurveto{\pgfqpoint{3.004865in}{2.212542in}}{\pgfqpoint{3.001593in}{2.204642in}}{\pgfqpoint{3.001593in}{2.196406in}}%
\pgfpathcurveto{\pgfqpoint{3.001593in}{2.188169in}}{\pgfqpoint{3.004865in}{2.180269in}}{\pgfqpoint{3.010689in}{2.174445in}}%
\pgfpathcurveto{\pgfqpoint{3.016513in}{2.168621in}}{\pgfqpoint{3.024413in}{2.165349in}}{\pgfqpoint{3.032649in}{2.165349in}}%
\pgfpathclose%
\pgfusepath{stroke,fill}%
\end{pgfscope}%
\begin{pgfscope}%
\pgfpathrectangle{\pgfqpoint{0.100000in}{0.212622in}}{\pgfqpoint{3.696000in}{3.696000in}}%
\pgfusepath{clip}%
\pgfsetbuttcap%
\pgfsetroundjoin%
\definecolor{currentfill}{rgb}{0.121569,0.466667,0.705882}%
\pgfsetfillcolor{currentfill}%
\pgfsetfillopacity{0.686366}%
\pgfsetlinewidth{1.003750pt}%
\definecolor{currentstroke}{rgb}{0.121569,0.466667,0.705882}%
\pgfsetstrokecolor{currentstroke}%
\pgfsetstrokeopacity{0.686366}%
\pgfsetdash{}{0pt}%
\pgfpathmoveto{\pgfqpoint{3.030227in}{2.164396in}}%
\pgfpathcurveto{\pgfqpoint{3.038463in}{2.164396in}}{\pgfqpoint{3.046363in}{2.167668in}}{\pgfqpoint{3.052187in}{2.173492in}}%
\pgfpathcurveto{\pgfqpoint{3.058011in}{2.179316in}}{\pgfqpoint{3.061283in}{2.187216in}}{\pgfqpoint{3.061283in}{2.195453in}}%
\pgfpathcurveto{\pgfqpoint{3.061283in}{2.203689in}}{\pgfqpoint{3.058011in}{2.211589in}}{\pgfqpoint{3.052187in}{2.217413in}}%
\pgfpathcurveto{\pgfqpoint{3.046363in}{2.223237in}}{\pgfqpoint{3.038463in}{2.226509in}}{\pgfqpoint{3.030227in}{2.226509in}}%
\pgfpathcurveto{\pgfqpoint{3.021990in}{2.226509in}}{\pgfqpoint{3.014090in}{2.223237in}}{\pgfqpoint{3.008266in}{2.217413in}}%
\pgfpathcurveto{\pgfqpoint{3.002443in}{2.211589in}}{\pgfqpoint{2.999170in}{2.203689in}}{\pgfqpoint{2.999170in}{2.195453in}}%
\pgfpathcurveto{\pgfqpoint{2.999170in}{2.187216in}}{\pgfqpoint{3.002443in}{2.179316in}}{\pgfqpoint{3.008266in}{2.173492in}}%
\pgfpathcurveto{\pgfqpoint{3.014090in}{2.167668in}}{\pgfqpoint{3.021990in}{2.164396in}}{\pgfqpoint{3.030227in}{2.164396in}}%
\pgfpathclose%
\pgfusepath{stroke,fill}%
\end{pgfscope}%
\begin{pgfscope}%
\pgfpathrectangle{\pgfqpoint{0.100000in}{0.212622in}}{\pgfqpoint{3.696000in}{3.696000in}}%
\pgfusepath{clip}%
\pgfsetbuttcap%
\pgfsetroundjoin%
\definecolor{currentfill}{rgb}{0.121569,0.466667,0.705882}%
\pgfsetfillcolor{currentfill}%
\pgfsetfillopacity{0.687117}%
\pgfsetlinewidth{1.003750pt}%
\definecolor{currentstroke}{rgb}{0.121569,0.466667,0.705882}%
\pgfsetstrokecolor{currentstroke}%
\pgfsetstrokeopacity{0.687117}%
\pgfsetdash{}{0pt}%
\pgfpathmoveto{\pgfqpoint{3.028894in}{2.163826in}}%
\pgfpathcurveto{\pgfqpoint{3.037130in}{2.163826in}}{\pgfqpoint{3.045030in}{2.167098in}}{\pgfqpoint{3.050854in}{2.172922in}}%
\pgfpathcurveto{\pgfqpoint{3.056678in}{2.178746in}}{\pgfqpoint{3.059950in}{2.186646in}}{\pgfqpoint{3.059950in}{2.194882in}}%
\pgfpathcurveto{\pgfqpoint{3.059950in}{2.203119in}}{\pgfqpoint{3.056678in}{2.211019in}}{\pgfqpoint{3.050854in}{2.216843in}}%
\pgfpathcurveto{\pgfqpoint{3.045030in}{2.222667in}}{\pgfqpoint{3.037130in}{2.225939in}}{\pgfqpoint{3.028894in}{2.225939in}}%
\pgfpathcurveto{\pgfqpoint{3.020658in}{2.225939in}}{\pgfqpoint{3.012758in}{2.222667in}}{\pgfqpoint{3.006934in}{2.216843in}}%
\pgfpathcurveto{\pgfqpoint{3.001110in}{2.211019in}}{\pgfqpoint{2.997837in}{2.203119in}}{\pgfqpoint{2.997837in}{2.194882in}}%
\pgfpathcurveto{\pgfqpoint{2.997837in}{2.186646in}}{\pgfqpoint{3.001110in}{2.178746in}}{\pgfqpoint{3.006934in}{2.172922in}}%
\pgfpathcurveto{\pgfqpoint{3.012758in}{2.167098in}}{\pgfqpoint{3.020658in}{2.163826in}}{\pgfqpoint{3.028894in}{2.163826in}}%
\pgfpathclose%
\pgfusepath{stroke,fill}%
\end{pgfscope}%
\begin{pgfscope}%
\pgfpathrectangle{\pgfqpoint{0.100000in}{0.212622in}}{\pgfqpoint{3.696000in}{3.696000in}}%
\pgfusepath{clip}%
\pgfsetbuttcap%
\pgfsetroundjoin%
\definecolor{currentfill}{rgb}{0.121569,0.466667,0.705882}%
\pgfsetfillcolor{currentfill}%
\pgfsetfillopacity{0.687536}%
\pgfsetlinewidth{1.003750pt}%
\definecolor{currentstroke}{rgb}{0.121569,0.466667,0.705882}%
\pgfsetstrokecolor{currentstroke}%
\pgfsetstrokeopacity{0.687536}%
\pgfsetdash{}{0pt}%
\pgfpathmoveto{\pgfqpoint{3.028220in}{2.163492in}}%
\pgfpathcurveto{\pgfqpoint{3.036456in}{2.163492in}}{\pgfqpoint{3.044356in}{2.166764in}}{\pgfqpoint{3.050180in}{2.172588in}}%
\pgfpathcurveto{\pgfqpoint{3.056004in}{2.178412in}}{\pgfqpoint{3.059276in}{2.186312in}}{\pgfqpoint{3.059276in}{2.194549in}}%
\pgfpathcurveto{\pgfqpoint{3.059276in}{2.202785in}}{\pgfqpoint{3.056004in}{2.210685in}}{\pgfqpoint{3.050180in}{2.216509in}}%
\pgfpathcurveto{\pgfqpoint{3.044356in}{2.222333in}}{\pgfqpoint{3.036456in}{2.225605in}}{\pgfqpoint{3.028220in}{2.225605in}}%
\pgfpathcurveto{\pgfqpoint{3.019984in}{2.225605in}}{\pgfqpoint{3.012083in}{2.222333in}}{\pgfqpoint{3.006260in}{2.216509in}}%
\pgfpathcurveto{\pgfqpoint{3.000436in}{2.210685in}}{\pgfqpoint{2.997163in}{2.202785in}}{\pgfqpoint{2.997163in}{2.194549in}}%
\pgfpathcurveto{\pgfqpoint{2.997163in}{2.186312in}}{\pgfqpoint{3.000436in}{2.178412in}}{\pgfqpoint{3.006260in}{2.172588in}}%
\pgfpathcurveto{\pgfqpoint{3.012083in}{2.166764in}}{\pgfqpoint{3.019984in}{2.163492in}}{\pgfqpoint{3.028220in}{2.163492in}}%
\pgfpathclose%
\pgfusepath{stroke,fill}%
\end{pgfscope}%
\begin{pgfscope}%
\pgfpathrectangle{\pgfqpoint{0.100000in}{0.212622in}}{\pgfqpoint{3.696000in}{3.696000in}}%
\pgfusepath{clip}%
\pgfsetbuttcap%
\pgfsetroundjoin%
\definecolor{currentfill}{rgb}{0.121569,0.466667,0.705882}%
\pgfsetfillcolor{currentfill}%
\pgfsetfillopacity{0.688128}%
\pgfsetlinewidth{1.003750pt}%
\definecolor{currentstroke}{rgb}{0.121569,0.466667,0.705882}%
\pgfsetstrokecolor{currentstroke}%
\pgfsetstrokeopacity{0.688128}%
\pgfsetdash{}{0pt}%
\pgfpathmoveto{\pgfqpoint{3.027165in}{2.163013in}}%
\pgfpathcurveto{\pgfqpoint{3.035401in}{2.163013in}}{\pgfqpoint{3.043301in}{2.166285in}}{\pgfqpoint{3.049125in}{2.172109in}}%
\pgfpathcurveto{\pgfqpoint{3.054949in}{2.177933in}}{\pgfqpoint{3.058221in}{2.185833in}}{\pgfqpoint{3.058221in}{2.194069in}}%
\pgfpathcurveto{\pgfqpoint{3.058221in}{2.202306in}}{\pgfqpoint{3.054949in}{2.210206in}}{\pgfqpoint{3.049125in}{2.216030in}}%
\pgfpathcurveto{\pgfqpoint{3.043301in}{2.221853in}}{\pgfqpoint{3.035401in}{2.225126in}}{\pgfqpoint{3.027165in}{2.225126in}}%
\pgfpathcurveto{\pgfqpoint{3.018929in}{2.225126in}}{\pgfqpoint{3.011029in}{2.221853in}}{\pgfqpoint{3.005205in}{2.216030in}}%
\pgfpathcurveto{\pgfqpoint{2.999381in}{2.210206in}}{\pgfqpoint{2.996108in}{2.202306in}}{\pgfqpoint{2.996108in}{2.194069in}}%
\pgfpathcurveto{\pgfqpoint{2.996108in}{2.185833in}}{\pgfqpoint{2.999381in}{2.177933in}}{\pgfqpoint{3.005205in}{2.172109in}}%
\pgfpathcurveto{\pgfqpoint{3.011029in}{2.166285in}}{\pgfqpoint{3.018929in}{2.163013in}}{\pgfqpoint{3.027165in}{2.163013in}}%
\pgfpathclose%
\pgfusepath{stroke,fill}%
\end{pgfscope}%
\begin{pgfscope}%
\pgfpathrectangle{\pgfqpoint{0.100000in}{0.212622in}}{\pgfqpoint{3.696000in}{3.696000in}}%
\pgfusepath{clip}%
\pgfsetbuttcap%
\pgfsetroundjoin%
\definecolor{currentfill}{rgb}{0.121569,0.466667,0.705882}%
\pgfsetfillcolor{currentfill}%
\pgfsetfillopacity{0.689049}%
\pgfsetlinewidth{1.003750pt}%
\definecolor{currentstroke}{rgb}{0.121569,0.466667,0.705882}%
\pgfsetstrokecolor{currentstroke}%
\pgfsetstrokeopacity{0.689049}%
\pgfsetdash{}{0pt}%
\pgfpathmoveto{\pgfqpoint{3.025760in}{2.162455in}}%
\pgfpathcurveto{\pgfqpoint{3.033996in}{2.162455in}}{\pgfqpoint{3.041896in}{2.165727in}}{\pgfqpoint{3.047720in}{2.171551in}}%
\pgfpathcurveto{\pgfqpoint{3.053544in}{2.177375in}}{\pgfqpoint{3.056817in}{2.185275in}}{\pgfqpoint{3.056817in}{2.193511in}}%
\pgfpathcurveto{\pgfqpoint{3.056817in}{2.201747in}}{\pgfqpoint{3.053544in}{2.209647in}}{\pgfqpoint{3.047720in}{2.215471in}}%
\pgfpathcurveto{\pgfqpoint{3.041896in}{2.221295in}}{\pgfqpoint{3.033996in}{2.224568in}}{\pgfqpoint{3.025760in}{2.224568in}}%
\pgfpathcurveto{\pgfqpoint{3.017524in}{2.224568in}}{\pgfqpoint{3.009624in}{2.221295in}}{\pgfqpoint{3.003800in}{2.215471in}}%
\pgfpathcurveto{\pgfqpoint{2.997976in}{2.209647in}}{\pgfqpoint{2.994704in}{2.201747in}}{\pgfqpoint{2.994704in}{2.193511in}}%
\pgfpathcurveto{\pgfqpoint{2.994704in}{2.185275in}}{\pgfqpoint{2.997976in}{2.177375in}}{\pgfqpoint{3.003800in}{2.171551in}}%
\pgfpathcurveto{\pgfqpoint{3.009624in}{2.165727in}}{\pgfqpoint{3.017524in}{2.162455in}}{\pgfqpoint{3.025760in}{2.162455in}}%
\pgfpathclose%
\pgfusepath{stroke,fill}%
\end{pgfscope}%
\begin{pgfscope}%
\pgfpathrectangle{\pgfqpoint{0.100000in}{0.212622in}}{\pgfqpoint{3.696000in}{3.696000in}}%
\pgfusepath{clip}%
\pgfsetbuttcap%
\pgfsetroundjoin%
\definecolor{currentfill}{rgb}{0.121569,0.466667,0.705882}%
\pgfsetfillcolor{currentfill}%
\pgfsetfillopacity{0.689561}%
\pgfsetlinewidth{1.003750pt}%
\definecolor{currentstroke}{rgb}{0.121569,0.466667,0.705882}%
\pgfsetstrokecolor{currentstroke}%
\pgfsetstrokeopacity{0.689561}%
\pgfsetdash{}{0pt}%
\pgfpathmoveto{\pgfqpoint{3.025020in}{2.162157in}}%
\pgfpathcurveto{\pgfqpoint{3.033256in}{2.162157in}}{\pgfqpoint{3.041156in}{2.165430in}}{\pgfqpoint{3.046980in}{2.171254in}}%
\pgfpathcurveto{\pgfqpoint{3.052804in}{2.177078in}}{\pgfqpoint{3.056076in}{2.184978in}}{\pgfqpoint{3.056076in}{2.193214in}}%
\pgfpathcurveto{\pgfqpoint{3.056076in}{2.201450in}}{\pgfqpoint{3.052804in}{2.209350in}}{\pgfqpoint{3.046980in}{2.215174in}}%
\pgfpathcurveto{\pgfqpoint{3.041156in}{2.220998in}}{\pgfqpoint{3.033256in}{2.224270in}}{\pgfqpoint{3.025020in}{2.224270in}}%
\pgfpathcurveto{\pgfqpoint{3.016784in}{2.224270in}}{\pgfqpoint{3.008884in}{2.220998in}}{\pgfqpoint{3.003060in}{2.215174in}}%
\pgfpathcurveto{\pgfqpoint{2.997236in}{2.209350in}}{\pgfqpoint{2.993963in}{2.201450in}}{\pgfqpoint{2.993963in}{2.193214in}}%
\pgfpathcurveto{\pgfqpoint{2.993963in}{2.184978in}}{\pgfqpoint{2.997236in}{2.177078in}}{\pgfqpoint{3.003060in}{2.171254in}}%
\pgfpathcurveto{\pgfqpoint{3.008884in}{2.165430in}}{\pgfqpoint{3.016784in}{2.162157in}}{\pgfqpoint{3.025020in}{2.162157in}}%
\pgfpathclose%
\pgfusepath{stroke,fill}%
\end{pgfscope}%
\begin{pgfscope}%
\pgfpathrectangle{\pgfqpoint{0.100000in}{0.212622in}}{\pgfqpoint{3.696000in}{3.696000in}}%
\pgfusepath{clip}%
\pgfsetbuttcap%
\pgfsetroundjoin%
\definecolor{currentfill}{rgb}{0.121569,0.466667,0.705882}%
\pgfsetfillcolor{currentfill}%
\pgfsetfillopacity{0.689827}%
\pgfsetlinewidth{1.003750pt}%
\definecolor{currentstroke}{rgb}{0.121569,0.466667,0.705882}%
\pgfsetstrokecolor{currentstroke}%
\pgfsetstrokeopacity{0.689827}%
\pgfsetdash{}{0pt}%
\pgfpathmoveto{\pgfqpoint{3.024566in}{2.161931in}}%
\pgfpathcurveto{\pgfqpoint{3.032802in}{2.161931in}}{\pgfqpoint{3.040702in}{2.165204in}}{\pgfqpoint{3.046526in}{2.171027in}}%
\pgfpathcurveto{\pgfqpoint{3.052350in}{2.176851in}}{\pgfqpoint{3.055623in}{2.184751in}}{\pgfqpoint{3.055623in}{2.192988in}}%
\pgfpathcurveto{\pgfqpoint{3.055623in}{2.201224in}}{\pgfqpoint{3.052350in}{2.209124in}}{\pgfqpoint{3.046526in}{2.214948in}}%
\pgfpathcurveto{\pgfqpoint{3.040702in}{2.220772in}}{\pgfqpoint{3.032802in}{2.224044in}}{\pgfqpoint{3.024566in}{2.224044in}}%
\pgfpathcurveto{\pgfqpoint{3.016330in}{2.224044in}}{\pgfqpoint{3.008430in}{2.220772in}}{\pgfqpoint{3.002606in}{2.214948in}}%
\pgfpathcurveto{\pgfqpoint{2.996782in}{2.209124in}}{\pgfqpoint{2.993510in}{2.201224in}}{\pgfqpoint{2.993510in}{2.192988in}}%
\pgfpathcurveto{\pgfqpoint{2.993510in}{2.184751in}}{\pgfqpoint{2.996782in}{2.176851in}}{\pgfqpoint{3.002606in}{2.171027in}}%
\pgfpathcurveto{\pgfqpoint{3.008430in}{2.165204in}}{\pgfqpoint{3.016330in}{2.161931in}}{\pgfqpoint{3.024566in}{2.161931in}}%
\pgfpathclose%
\pgfusepath{stroke,fill}%
\end{pgfscope}%
\begin{pgfscope}%
\pgfpathrectangle{\pgfqpoint{0.100000in}{0.212622in}}{\pgfqpoint{3.696000in}{3.696000in}}%
\pgfusepath{clip}%
\pgfsetbuttcap%
\pgfsetroundjoin%
\definecolor{currentfill}{rgb}{0.121569,0.466667,0.705882}%
\pgfsetfillcolor{currentfill}%
\pgfsetfillopacity{0.690533}%
\pgfsetlinewidth{1.003750pt}%
\definecolor{currentstroke}{rgb}{0.121569,0.466667,0.705882}%
\pgfsetstrokecolor{currentstroke}%
\pgfsetstrokeopacity{0.690533}%
\pgfsetdash{}{0pt}%
\pgfpathmoveto{\pgfqpoint{3.023317in}{2.161636in}}%
\pgfpathcurveto{\pgfqpoint{3.031553in}{2.161636in}}{\pgfqpoint{3.039453in}{2.164908in}}{\pgfqpoint{3.045277in}{2.170732in}}%
\pgfpathcurveto{\pgfqpoint{3.051101in}{2.176556in}}{\pgfqpoint{3.054374in}{2.184456in}}{\pgfqpoint{3.054374in}{2.192693in}}%
\pgfpathcurveto{\pgfqpoint{3.054374in}{2.200929in}}{\pgfqpoint{3.051101in}{2.208829in}}{\pgfqpoint{3.045277in}{2.214653in}}%
\pgfpathcurveto{\pgfqpoint{3.039453in}{2.220477in}}{\pgfqpoint{3.031553in}{2.223749in}}{\pgfqpoint{3.023317in}{2.223749in}}%
\pgfpathcurveto{\pgfqpoint{3.015081in}{2.223749in}}{\pgfqpoint{3.007181in}{2.220477in}}{\pgfqpoint{3.001357in}{2.214653in}}%
\pgfpathcurveto{\pgfqpoint{2.995533in}{2.208829in}}{\pgfqpoint{2.992261in}{2.200929in}}{\pgfqpoint{2.992261in}{2.192693in}}%
\pgfpathcurveto{\pgfqpoint{2.992261in}{2.184456in}}{\pgfqpoint{2.995533in}{2.176556in}}{\pgfqpoint{3.001357in}{2.170732in}}%
\pgfpathcurveto{\pgfqpoint{3.007181in}{2.164908in}}{\pgfqpoint{3.015081in}{2.161636in}}{\pgfqpoint{3.023317in}{2.161636in}}%
\pgfpathclose%
\pgfusepath{stroke,fill}%
\end{pgfscope}%
\begin{pgfscope}%
\pgfpathrectangle{\pgfqpoint{0.100000in}{0.212622in}}{\pgfqpoint{3.696000in}{3.696000in}}%
\pgfusepath{clip}%
\pgfsetbuttcap%
\pgfsetroundjoin%
\definecolor{currentfill}{rgb}{0.121569,0.466667,0.705882}%
\pgfsetfillcolor{currentfill}%
\pgfsetfillopacity{0.690936}%
\pgfsetlinewidth{1.003750pt}%
\definecolor{currentstroke}{rgb}{0.121569,0.466667,0.705882}%
\pgfsetstrokecolor{currentstroke}%
\pgfsetstrokeopacity{0.690936}%
\pgfsetdash{}{0pt}%
\pgfpathmoveto{\pgfqpoint{3.022676in}{2.161524in}}%
\pgfpathcurveto{\pgfqpoint{3.030912in}{2.161524in}}{\pgfqpoint{3.038812in}{2.164796in}}{\pgfqpoint{3.044636in}{2.170620in}}%
\pgfpathcurveto{\pgfqpoint{3.050460in}{2.176444in}}{\pgfqpoint{3.053732in}{2.184344in}}{\pgfqpoint{3.053732in}{2.192581in}}%
\pgfpathcurveto{\pgfqpoint{3.053732in}{2.200817in}}{\pgfqpoint{3.050460in}{2.208717in}}{\pgfqpoint{3.044636in}{2.214541in}}%
\pgfpathcurveto{\pgfqpoint{3.038812in}{2.220365in}}{\pgfqpoint{3.030912in}{2.223637in}}{\pgfqpoint{3.022676in}{2.223637in}}%
\pgfpathcurveto{\pgfqpoint{3.014440in}{2.223637in}}{\pgfqpoint{3.006540in}{2.220365in}}{\pgfqpoint{3.000716in}{2.214541in}}%
\pgfpathcurveto{\pgfqpoint{2.994892in}{2.208717in}}{\pgfqpoint{2.991619in}{2.200817in}}{\pgfqpoint{2.991619in}{2.192581in}}%
\pgfpathcurveto{\pgfqpoint{2.991619in}{2.184344in}}{\pgfqpoint{2.994892in}{2.176444in}}{\pgfqpoint{3.000716in}{2.170620in}}%
\pgfpathcurveto{\pgfqpoint{3.006540in}{2.164796in}}{\pgfqpoint{3.014440in}{2.161524in}}{\pgfqpoint{3.022676in}{2.161524in}}%
\pgfpathclose%
\pgfusepath{stroke,fill}%
\end{pgfscope}%
\begin{pgfscope}%
\pgfpathrectangle{\pgfqpoint{0.100000in}{0.212622in}}{\pgfqpoint{3.696000in}{3.696000in}}%
\pgfusepath{clip}%
\pgfsetbuttcap%
\pgfsetroundjoin%
\definecolor{currentfill}{rgb}{0.121569,0.466667,0.705882}%
\pgfsetfillcolor{currentfill}%
\pgfsetfillopacity{0.691147}%
\pgfsetlinewidth{1.003750pt}%
\definecolor{currentstroke}{rgb}{0.121569,0.466667,0.705882}%
\pgfsetstrokecolor{currentstroke}%
\pgfsetstrokeopacity{0.691147}%
\pgfsetdash{}{0pt}%
\pgfpathmoveto{\pgfqpoint{3.022311in}{2.161401in}}%
\pgfpathcurveto{\pgfqpoint{3.030547in}{2.161401in}}{\pgfqpoint{3.038447in}{2.164673in}}{\pgfqpoint{3.044271in}{2.170497in}}%
\pgfpathcurveto{\pgfqpoint{3.050095in}{2.176321in}}{\pgfqpoint{3.053367in}{2.184221in}}{\pgfqpoint{3.053367in}{2.192458in}}%
\pgfpathcurveto{\pgfqpoint{3.053367in}{2.200694in}}{\pgfqpoint{3.050095in}{2.208594in}}{\pgfqpoint{3.044271in}{2.214418in}}%
\pgfpathcurveto{\pgfqpoint{3.038447in}{2.220242in}}{\pgfqpoint{3.030547in}{2.223514in}}{\pgfqpoint{3.022311in}{2.223514in}}%
\pgfpathcurveto{\pgfqpoint{3.014074in}{2.223514in}}{\pgfqpoint{3.006174in}{2.220242in}}{\pgfqpoint{3.000350in}{2.214418in}}%
\pgfpathcurveto{\pgfqpoint{2.994526in}{2.208594in}}{\pgfqpoint{2.991254in}{2.200694in}}{\pgfqpoint{2.991254in}{2.192458in}}%
\pgfpathcurveto{\pgfqpoint{2.991254in}{2.184221in}}{\pgfqpoint{2.994526in}{2.176321in}}{\pgfqpoint{3.000350in}{2.170497in}}%
\pgfpathcurveto{\pgfqpoint{3.006174in}{2.164673in}}{\pgfqpoint{3.014074in}{2.161401in}}{\pgfqpoint{3.022311in}{2.161401in}}%
\pgfpathclose%
\pgfusepath{stroke,fill}%
\end{pgfscope}%
\begin{pgfscope}%
\pgfpathrectangle{\pgfqpoint{0.100000in}{0.212622in}}{\pgfqpoint{3.696000in}{3.696000in}}%
\pgfusepath{clip}%
\pgfsetbuttcap%
\pgfsetroundjoin%
\definecolor{currentfill}{rgb}{0.121569,0.466667,0.705882}%
\pgfsetfillcolor{currentfill}%
\pgfsetfillopacity{0.691841}%
\pgfsetlinewidth{1.003750pt}%
\definecolor{currentstroke}{rgb}{0.121569,0.466667,0.705882}%
\pgfsetstrokecolor{currentstroke}%
\pgfsetstrokeopacity{0.691841}%
\pgfsetdash{}{0pt}%
\pgfpathmoveto{\pgfqpoint{3.021071in}{2.160859in}}%
\pgfpathcurveto{\pgfqpoint{3.029307in}{2.160859in}}{\pgfqpoint{3.037207in}{2.164131in}}{\pgfqpoint{3.043031in}{2.169955in}}%
\pgfpathcurveto{\pgfqpoint{3.048855in}{2.175779in}}{\pgfqpoint{3.052127in}{2.183679in}}{\pgfqpoint{3.052127in}{2.191915in}}%
\pgfpathcurveto{\pgfqpoint{3.052127in}{2.200152in}}{\pgfqpoint{3.048855in}{2.208052in}}{\pgfqpoint{3.043031in}{2.213876in}}%
\pgfpathcurveto{\pgfqpoint{3.037207in}{2.219699in}}{\pgfqpoint{3.029307in}{2.222972in}}{\pgfqpoint{3.021071in}{2.222972in}}%
\pgfpathcurveto{\pgfqpoint{3.012835in}{2.222972in}}{\pgfqpoint{3.004935in}{2.219699in}}{\pgfqpoint{2.999111in}{2.213876in}}%
\pgfpathcurveto{\pgfqpoint{2.993287in}{2.208052in}}{\pgfqpoint{2.990014in}{2.200152in}}{\pgfqpoint{2.990014in}{2.191915in}}%
\pgfpathcurveto{\pgfqpoint{2.990014in}{2.183679in}}{\pgfqpoint{2.993287in}{2.175779in}}{\pgfqpoint{2.999111in}{2.169955in}}%
\pgfpathcurveto{\pgfqpoint{3.004935in}{2.164131in}}{\pgfqpoint{3.012835in}{2.160859in}}{\pgfqpoint{3.021071in}{2.160859in}}%
\pgfpathclose%
\pgfusepath{stroke,fill}%
\end{pgfscope}%
\begin{pgfscope}%
\pgfpathrectangle{\pgfqpoint{0.100000in}{0.212622in}}{\pgfqpoint{3.696000in}{3.696000in}}%
\pgfusepath{clip}%
\pgfsetbuttcap%
\pgfsetroundjoin%
\definecolor{currentfill}{rgb}{0.121569,0.466667,0.705882}%
\pgfsetfillcolor{currentfill}%
\pgfsetfillopacity{0.692775}%
\pgfsetlinewidth{1.003750pt}%
\definecolor{currentstroke}{rgb}{0.121569,0.466667,0.705882}%
\pgfsetstrokecolor{currentstroke}%
\pgfsetstrokeopacity{0.692775}%
\pgfsetdash{}{0pt}%
\pgfpathmoveto{\pgfqpoint{3.019575in}{2.160158in}}%
\pgfpathcurveto{\pgfqpoint{3.027811in}{2.160158in}}{\pgfqpoint{3.035711in}{2.163431in}}{\pgfqpoint{3.041535in}{2.169255in}}%
\pgfpathcurveto{\pgfqpoint{3.047359in}{2.175078in}}{\pgfqpoint{3.050631in}{2.182979in}}{\pgfqpoint{3.050631in}{2.191215in}}%
\pgfpathcurveto{\pgfqpoint{3.050631in}{2.199451in}}{\pgfqpoint{3.047359in}{2.207351in}}{\pgfqpoint{3.041535in}{2.213175in}}%
\pgfpathcurveto{\pgfqpoint{3.035711in}{2.218999in}}{\pgfqpoint{3.027811in}{2.222271in}}{\pgfqpoint{3.019575in}{2.222271in}}%
\pgfpathcurveto{\pgfqpoint{3.011339in}{2.222271in}}{\pgfqpoint{3.003439in}{2.218999in}}{\pgfqpoint{2.997615in}{2.213175in}}%
\pgfpathcurveto{\pgfqpoint{2.991791in}{2.207351in}}{\pgfqpoint{2.988518in}{2.199451in}}{\pgfqpoint{2.988518in}{2.191215in}}%
\pgfpathcurveto{\pgfqpoint{2.988518in}{2.182979in}}{\pgfqpoint{2.991791in}{2.175078in}}{\pgfqpoint{2.997615in}{2.169255in}}%
\pgfpathcurveto{\pgfqpoint{3.003439in}{2.163431in}}{\pgfqpoint{3.011339in}{2.160158in}}{\pgfqpoint{3.019575in}{2.160158in}}%
\pgfpathclose%
\pgfusepath{stroke,fill}%
\end{pgfscope}%
\begin{pgfscope}%
\pgfpathrectangle{\pgfqpoint{0.100000in}{0.212622in}}{\pgfqpoint{3.696000in}{3.696000in}}%
\pgfusepath{clip}%
\pgfsetbuttcap%
\pgfsetroundjoin%
\definecolor{currentfill}{rgb}{0.121569,0.466667,0.705882}%
\pgfsetfillcolor{currentfill}%
\pgfsetfillopacity{0.694112}%
\pgfsetlinewidth{1.003750pt}%
\definecolor{currentstroke}{rgb}{0.121569,0.466667,0.705882}%
\pgfsetstrokecolor{currentstroke}%
\pgfsetstrokeopacity{0.694112}%
\pgfsetdash{}{0pt}%
\pgfpathmoveto{\pgfqpoint{3.017242in}{2.159227in}}%
\pgfpathcurveto{\pgfqpoint{3.025478in}{2.159227in}}{\pgfqpoint{3.033378in}{2.162500in}}{\pgfqpoint{3.039202in}{2.168324in}}%
\pgfpathcurveto{\pgfqpoint{3.045026in}{2.174148in}}{\pgfqpoint{3.048298in}{2.182048in}}{\pgfqpoint{3.048298in}{2.190284in}}%
\pgfpathcurveto{\pgfqpoint{3.048298in}{2.198520in}}{\pgfqpoint{3.045026in}{2.206420in}}{\pgfqpoint{3.039202in}{2.212244in}}%
\pgfpathcurveto{\pgfqpoint{3.033378in}{2.218068in}}{\pgfqpoint{3.025478in}{2.221340in}}{\pgfqpoint{3.017242in}{2.221340in}}%
\pgfpathcurveto{\pgfqpoint{3.009005in}{2.221340in}}{\pgfqpoint{3.001105in}{2.218068in}}{\pgfqpoint{2.995281in}{2.212244in}}%
\pgfpathcurveto{\pgfqpoint{2.989458in}{2.206420in}}{\pgfqpoint{2.986185in}{2.198520in}}{\pgfqpoint{2.986185in}{2.190284in}}%
\pgfpathcurveto{\pgfqpoint{2.986185in}{2.182048in}}{\pgfqpoint{2.989458in}{2.174148in}}{\pgfqpoint{2.995281in}{2.168324in}}%
\pgfpathcurveto{\pgfqpoint{3.001105in}{2.162500in}}{\pgfqpoint{3.009005in}{2.159227in}}{\pgfqpoint{3.017242in}{2.159227in}}%
\pgfpathclose%
\pgfusepath{stroke,fill}%
\end{pgfscope}%
\begin{pgfscope}%
\pgfpathrectangle{\pgfqpoint{0.100000in}{0.212622in}}{\pgfqpoint{3.696000in}{3.696000in}}%
\pgfusepath{clip}%
\pgfsetbuttcap%
\pgfsetroundjoin%
\definecolor{currentfill}{rgb}{0.121569,0.466667,0.705882}%
\pgfsetfillcolor{currentfill}%
\pgfsetfillopacity{0.695875}%
\pgfsetlinewidth{1.003750pt}%
\definecolor{currentstroke}{rgb}{0.121569,0.466667,0.705882}%
\pgfsetstrokecolor{currentstroke}%
\pgfsetstrokeopacity{0.695875}%
\pgfsetdash{}{0pt}%
\pgfpathmoveto{\pgfqpoint{3.014318in}{2.157873in}}%
\pgfpathcurveto{\pgfqpoint{3.022555in}{2.157873in}}{\pgfqpoint{3.030455in}{2.161145in}}{\pgfqpoint{3.036279in}{2.166969in}}%
\pgfpathcurveto{\pgfqpoint{3.042102in}{2.172793in}}{\pgfqpoint{3.045375in}{2.180693in}}{\pgfqpoint{3.045375in}{2.188929in}}%
\pgfpathcurveto{\pgfqpoint{3.045375in}{2.197165in}}{\pgfqpoint{3.042102in}{2.205066in}}{\pgfqpoint{3.036279in}{2.210889in}}%
\pgfpathcurveto{\pgfqpoint{3.030455in}{2.216713in}}{\pgfqpoint{3.022555in}{2.219986in}}{\pgfqpoint{3.014318in}{2.219986in}}%
\pgfpathcurveto{\pgfqpoint{3.006082in}{2.219986in}}{\pgfqpoint{2.998182in}{2.216713in}}{\pgfqpoint{2.992358in}{2.210889in}}%
\pgfpathcurveto{\pgfqpoint{2.986534in}{2.205066in}}{\pgfqpoint{2.983262in}{2.197165in}}{\pgfqpoint{2.983262in}{2.188929in}}%
\pgfpathcurveto{\pgfqpoint{2.983262in}{2.180693in}}{\pgfqpoint{2.986534in}{2.172793in}}{\pgfqpoint{2.992358in}{2.166969in}}%
\pgfpathcurveto{\pgfqpoint{2.998182in}{2.161145in}}{\pgfqpoint{3.006082in}{2.157873in}}{\pgfqpoint{3.014318in}{2.157873in}}%
\pgfpathclose%
\pgfusepath{stroke,fill}%
\end{pgfscope}%
\begin{pgfscope}%
\pgfpathrectangle{\pgfqpoint{0.100000in}{0.212622in}}{\pgfqpoint{3.696000in}{3.696000in}}%
\pgfusepath{clip}%
\pgfsetbuttcap%
\pgfsetroundjoin%
\definecolor{currentfill}{rgb}{0.121569,0.466667,0.705882}%
\pgfsetfillcolor{currentfill}%
\pgfsetfillopacity{0.696872}%
\pgfsetlinewidth{1.003750pt}%
\definecolor{currentstroke}{rgb}{0.121569,0.466667,0.705882}%
\pgfsetstrokecolor{currentstroke}%
\pgfsetstrokeopacity{0.696872}%
\pgfsetdash{}{0pt}%
\pgfpathmoveto{\pgfqpoint{3.012810in}{2.157208in}}%
\pgfpathcurveto{\pgfqpoint{3.021046in}{2.157208in}}{\pgfqpoint{3.028946in}{2.160480in}}{\pgfqpoint{3.034770in}{2.166304in}}%
\pgfpathcurveto{\pgfqpoint{3.040594in}{2.172128in}}{\pgfqpoint{3.043866in}{2.180028in}}{\pgfqpoint{3.043866in}{2.188264in}}%
\pgfpathcurveto{\pgfqpoint{3.043866in}{2.196500in}}{\pgfqpoint{3.040594in}{2.204400in}}{\pgfqpoint{3.034770in}{2.210224in}}%
\pgfpathcurveto{\pgfqpoint{3.028946in}{2.216048in}}{\pgfqpoint{3.021046in}{2.219321in}}{\pgfqpoint{3.012810in}{2.219321in}}%
\pgfpathcurveto{\pgfqpoint{3.004573in}{2.219321in}}{\pgfqpoint{2.996673in}{2.216048in}}{\pgfqpoint{2.990849in}{2.210224in}}%
\pgfpathcurveto{\pgfqpoint{2.985026in}{2.204400in}}{\pgfqpoint{2.981753in}{2.196500in}}{\pgfqpoint{2.981753in}{2.188264in}}%
\pgfpathcurveto{\pgfqpoint{2.981753in}{2.180028in}}{\pgfqpoint{2.985026in}{2.172128in}}{\pgfqpoint{2.990849in}{2.166304in}}%
\pgfpathcurveto{\pgfqpoint{2.996673in}{2.160480in}}{\pgfqpoint{3.004573in}{2.157208in}}{\pgfqpoint{3.012810in}{2.157208in}}%
\pgfpathclose%
\pgfusepath{stroke,fill}%
\end{pgfscope}%
\begin{pgfscope}%
\pgfpathrectangle{\pgfqpoint{0.100000in}{0.212622in}}{\pgfqpoint{3.696000in}{3.696000in}}%
\pgfusepath{clip}%
\pgfsetbuttcap%
\pgfsetroundjoin%
\definecolor{currentfill}{rgb}{0.121569,0.466667,0.705882}%
\pgfsetfillcolor{currentfill}%
\pgfsetfillopacity{0.698011}%
\pgfsetlinewidth{1.003750pt}%
\definecolor{currentstroke}{rgb}{0.121569,0.466667,0.705882}%
\pgfsetstrokecolor{currentstroke}%
\pgfsetstrokeopacity{0.698011}%
\pgfsetdash{}{0pt}%
\pgfpathmoveto{\pgfqpoint{3.010908in}{2.156259in}}%
\pgfpathcurveto{\pgfqpoint{3.019144in}{2.156259in}}{\pgfqpoint{3.027044in}{2.159531in}}{\pgfqpoint{3.032868in}{2.165355in}}%
\pgfpathcurveto{\pgfqpoint{3.038692in}{2.171179in}}{\pgfqpoint{3.041964in}{2.179079in}}{\pgfqpoint{3.041964in}{2.187315in}}%
\pgfpathcurveto{\pgfqpoint{3.041964in}{2.195552in}}{\pgfqpoint{3.038692in}{2.203452in}}{\pgfqpoint{3.032868in}{2.209276in}}%
\pgfpathcurveto{\pgfqpoint{3.027044in}{2.215100in}}{\pgfqpoint{3.019144in}{2.218372in}}{\pgfqpoint{3.010908in}{2.218372in}}%
\pgfpathcurveto{\pgfqpoint{3.002672in}{2.218372in}}{\pgfqpoint{2.994772in}{2.215100in}}{\pgfqpoint{2.988948in}{2.209276in}}%
\pgfpathcurveto{\pgfqpoint{2.983124in}{2.203452in}}{\pgfqpoint{2.979851in}{2.195552in}}{\pgfqpoint{2.979851in}{2.187315in}}%
\pgfpathcurveto{\pgfqpoint{2.979851in}{2.179079in}}{\pgfqpoint{2.983124in}{2.171179in}}{\pgfqpoint{2.988948in}{2.165355in}}%
\pgfpathcurveto{\pgfqpoint{2.994772in}{2.159531in}}{\pgfqpoint{3.002672in}{2.156259in}}{\pgfqpoint{3.010908in}{2.156259in}}%
\pgfpathclose%
\pgfusepath{stroke,fill}%
\end{pgfscope}%
\begin{pgfscope}%
\pgfpathrectangle{\pgfqpoint{0.100000in}{0.212622in}}{\pgfqpoint{3.696000in}{3.696000in}}%
\pgfusepath{clip}%
\pgfsetbuttcap%
\pgfsetroundjoin%
\definecolor{currentfill}{rgb}{0.121569,0.466667,0.705882}%
\pgfsetfillcolor{currentfill}%
\pgfsetfillopacity{0.699556}%
\pgfsetlinewidth{1.003750pt}%
\definecolor{currentstroke}{rgb}{0.121569,0.466667,0.705882}%
\pgfsetstrokecolor{currentstroke}%
\pgfsetstrokeopacity{0.699556}%
\pgfsetdash{}{0pt}%
\pgfpathmoveto{\pgfqpoint{3.008782in}{2.155280in}}%
\pgfpathcurveto{\pgfqpoint{3.017018in}{2.155280in}}{\pgfqpoint{3.024918in}{2.158553in}}{\pgfqpoint{3.030742in}{2.164377in}}%
\pgfpathcurveto{\pgfqpoint{3.036566in}{2.170201in}}{\pgfqpoint{3.039838in}{2.178101in}}{\pgfqpoint{3.039838in}{2.186337in}}%
\pgfpathcurveto{\pgfqpoint{3.039838in}{2.194573in}}{\pgfqpoint{3.036566in}{2.202473in}}{\pgfqpoint{3.030742in}{2.208297in}}%
\pgfpathcurveto{\pgfqpoint{3.024918in}{2.214121in}}{\pgfqpoint{3.017018in}{2.217393in}}{\pgfqpoint{3.008782in}{2.217393in}}%
\pgfpathcurveto{\pgfqpoint{3.000545in}{2.217393in}}{\pgfqpoint{2.992645in}{2.214121in}}{\pgfqpoint{2.986821in}{2.208297in}}%
\pgfpathcurveto{\pgfqpoint{2.980998in}{2.202473in}}{\pgfqpoint{2.977725in}{2.194573in}}{\pgfqpoint{2.977725in}{2.186337in}}%
\pgfpathcurveto{\pgfqpoint{2.977725in}{2.178101in}}{\pgfqpoint{2.980998in}{2.170201in}}{\pgfqpoint{2.986821in}{2.164377in}}%
\pgfpathcurveto{\pgfqpoint{2.992645in}{2.158553in}}{\pgfqpoint{3.000545in}{2.155280in}}{\pgfqpoint{3.008782in}{2.155280in}}%
\pgfpathclose%
\pgfusepath{stroke,fill}%
\end{pgfscope}%
\begin{pgfscope}%
\pgfpathrectangle{\pgfqpoint{0.100000in}{0.212622in}}{\pgfqpoint{3.696000in}{3.696000in}}%
\pgfusepath{clip}%
\pgfsetbuttcap%
\pgfsetroundjoin%
\definecolor{currentfill}{rgb}{0.121569,0.466667,0.705882}%
\pgfsetfillcolor{currentfill}%
\pgfsetfillopacity{0.701289}%
\pgfsetlinewidth{1.003750pt}%
\definecolor{currentstroke}{rgb}{0.121569,0.466667,0.705882}%
\pgfsetstrokecolor{currentstroke}%
\pgfsetstrokeopacity{0.701289}%
\pgfsetdash{}{0pt}%
\pgfpathmoveto{\pgfqpoint{3.006500in}{2.154338in}}%
\pgfpathcurveto{\pgfqpoint{3.014736in}{2.154338in}}{\pgfqpoint{3.022636in}{2.157610in}}{\pgfqpoint{3.028460in}{2.163434in}}%
\pgfpathcurveto{\pgfqpoint{3.034284in}{2.169258in}}{\pgfqpoint{3.037556in}{2.177158in}}{\pgfqpoint{3.037556in}{2.185394in}}%
\pgfpathcurveto{\pgfqpoint{3.037556in}{2.193631in}}{\pgfqpoint{3.034284in}{2.201531in}}{\pgfqpoint{3.028460in}{2.207355in}}%
\pgfpathcurveto{\pgfqpoint{3.022636in}{2.213179in}}{\pgfqpoint{3.014736in}{2.216451in}}{\pgfqpoint{3.006500in}{2.216451in}}%
\pgfpathcurveto{\pgfqpoint{2.998263in}{2.216451in}}{\pgfqpoint{2.990363in}{2.213179in}}{\pgfqpoint{2.984539in}{2.207355in}}%
\pgfpathcurveto{\pgfqpoint{2.978715in}{2.201531in}}{\pgfqpoint{2.975443in}{2.193631in}}{\pgfqpoint{2.975443in}{2.185394in}}%
\pgfpathcurveto{\pgfqpoint{2.975443in}{2.177158in}}{\pgfqpoint{2.978715in}{2.169258in}}{\pgfqpoint{2.984539in}{2.163434in}}%
\pgfpathcurveto{\pgfqpoint{2.990363in}{2.157610in}}{\pgfqpoint{2.998263in}{2.154338in}}{\pgfqpoint{3.006500in}{2.154338in}}%
\pgfpathclose%
\pgfusepath{stroke,fill}%
\end{pgfscope}%
\begin{pgfscope}%
\pgfpathrectangle{\pgfqpoint{0.100000in}{0.212622in}}{\pgfqpoint{3.696000in}{3.696000in}}%
\pgfusepath{clip}%
\pgfsetbuttcap%
\pgfsetroundjoin%
\definecolor{currentfill}{rgb}{0.121569,0.466667,0.705882}%
\pgfsetfillcolor{currentfill}%
\pgfsetfillopacity{0.702215}%
\pgfsetlinewidth{1.003750pt}%
\definecolor{currentstroke}{rgb}{0.121569,0.466667,0.705882}%
\pgfsetstrokecolor{currentstroke}%
\pgfsetstrokeopacity{0.702215}%
\pgfsetdash{}{0pt}%
\pgfpathmoveto{\pgfqpoint{3.005156in}{2.153719in}}%
\pgfpathcurveto{\pgfqpoint{3.013393in}{2.153719in}}{\pgfqpoint{3.021293in}{2.156991in}}{\pgfqpoint{3.027117in}{2.162815in}}%
\pgfpathcurveto{\pgfqpoint{3.032940in}{2.168639in}}{\pgfqpoint{3.036213in}{2.176539in}}{\pgfqpoint{3.036213in}{2.184776in}}%
\pgfpathcurveto{\pgfqpoint{3.036213in}{2.193012in}}{\pgfqpoint{3.032940in}{2.200912in}}{\pgfqpoint{3.027117in}{2.206736in}}%
\pgfpathcurveto{\pgfqpoint{3.021293in}{2.212560in}}{\pgfqpoint{3.013393in}{2.215832in}}{\pgfqpoint{3.005156in}{2.215832in}}%
\pgfpathcurveto{\pgfqpoint{2.996920in}{2.215832in}}{\pgfqpoint{2.989020in}{2.212560in}}{\pgfqpoint{2.983196in}{2.206736in}}%
\pgfpathcurveto{\pgfqpoint{2.977372in}{2.200912in}}{\pgfqpoint{2.974100in}{2.193012in}}{\pgfqpoint{2.974100in}{2.184776in}}%
\pgfpathcurveto{\pgfqpoint{2.974100in}{2.176539in}}{\pgfqpoint{2.977372in}{2.168639in}}{\pgfqpoint{2.983196in}{2.162815in}}%
\pgfpathcurveto{\pgfqpoint{2.989020in}{2.156991in}}{\pgfqpoint{2.996920in}{2.153719in}}{\pgfqpoint{3.005156in}{2.153719in}}%
\pgfpathclose%
\pgfusepath{stroke,fill}%
\end{pgfscope}%
\begin{pgfscope}%
\pgfpathrectangle{\pgfqpoint{0.100000in}{0.212622in}}{\pgfqpoint{3.696000in}{3.696000in}}%
\pgfusepath{clip}%
\pgfsetbuttcap%
\pgfsetroundjoin%
\definecolor{currentfill}{rgb}{0.121569,0.466667,0.705882}%
\pgfsetfillcolor{currentfill}%
\pgfsetfillopacity{0.703468}%
\pgfsetlinewidth{1.003750pt}%
\definecolor{currentstroke}{rgb}{0.121569,0.466667,0.705882}%
\pgfsetstrokecolor{currentstroke}%
\pgfsetstrokeopacity{0.703468}%
\pgfsetdash{}{0pt}%
\pgfpathmoveto{\pgfqpoint{3.003266in}{2.152729in}}%
\pgfpathcurveto{\pgfqpoint{3.011502in}{2.152729in}}{\pgfqpoint{3.019402in}{2.156001in}}{\pgfqpoint{3.025226in}{2.161825in}}%
\pgfpathcurveto{\pgfqpoint{3.031050in}{2.167649in}}{\pgfqpoint{3.034323in}{2.175549in}}{\pgfqpoint{3.034323in}{2.183785in}}%
\pgfpathcurveto{\pgfqpoint{3.034323in}{2.192021in}}{\pgfqpoint{3.031050in}{2.199921in}}{\pgfqpoint{3.025226in}{2.205745in}}%
\pgfpathcurveto{\pgfqpoint{3.019402in}{2.211569in}}{\pgfqpoint{3.011502in}{2.214842in}}{\pgfqpoint{3.003266in}{2.214842in}}%
\pgfpathcurveto{\pgfqpoint{2.995030in}{2.214842in}}{\pgfqpoint{2.987130in}{2.211569in}}{\pgfqpoint{2.981306in}{2.205745in}}%
\pgfpathcurveto{\pgfqpoint{2.975482in}{2.199921in}}{\pgfqpoint{2.972210in}{2.192021in}}{\pgfqpoint{2.972210in}{2.183785in}}%
\pgfpathcurveto{\pgfqpoint{2.972210in}{2.175549in}}{\pgfqpoint{2.975482in}{2.167649in}}{\pgfqpoint{2.981306in}{2.161825in}}%
\pgfpathcurveto{\pgfqpoint{2.987130in}{2.156001in}}{\pgfqpoint{2.995030in}{2.152729in}}{\pgfqpoint{3.003266in}{2.152729in}}%
\pgfpathclose%
\pgfusepath{stroke,fill}%
\end{pgfscope}%
\begin{pgfscope}%
\pgfpathrectangle{\pgfqpoint{0.100000in}{0.212622in}}{\pgfqpoint{3.696000in}{3.696000in}}%
\pgfusepath{clip}%
\pgfsetbuttcap%
\pgfsetroundjoin%
\definecolor{currentfill}{rgb}{0.121569,0.466667,0.705882}%
\pgfsetfillcolor{currentfill}%
\pgfsetfillopacity{0.704198}%
\pgfsetlinewidth{1.003750pt}%
\definecolor{currentstroke}{rgb}{0.121569,0.466667,0.705882}%
\pgfsetstrokecolor{currentstroke}%
\pgfsetstrokeopacity{0.704198}%
\pgfsetdash{}{0pt}%
\pgfpathmoveto{\pgfqpoint{3.002256in}{2.152423in}}%
\pgfpathcurveto{\pgfqpoint{3.010492in}{2.152423in}}{\pgfqpoint{3.018392in}{2.155695in}}{\pgfqpoint{3.024216in}{2.161519in}}%
\pgfpathcurveto{\pgfqpoint{3.030040in}{2.167343in}}{\pgfqpoint{3.033312in}{2.175243in}}{\pgfqpoint{3.033312in}{2.183479in}}%
\pgfpathcurveto{\pgfqpoint{3.033312in}{2.191715in}}{\pgfqpoint{3.030040in}{2.199615in}}{\pgfqpoint{3.024216in}{2.205439in}}%
\pgfpathcurveto{\pgfqpoint{3.018392in}{2.211263in}}{\pgfqpoint{3.010492in}{2.214536in}}{\pgfqpoint{3.002256in}{2.214536in}}%
\pgfpathcurveto{\pgfqpoint{2.994019in}{2.214536in}}{\pgfqpoint{2.986119in}{2.211263in}}{\pgfqpoint{2.980295in}{2.205439in}}%
\pgfpathcurveto{\pgfqpoint{2.974471in}{2.199615in}}{\pgfqpoint{2.971199in}{2.191715in}}{\pgfqpoint{2.971199in}{2.183479in}}%
\pgfpathcurveto{\pgfqpoint{2.971199in}{2.175243in}}{\pgfqpoint{2.974471in}{2.167343in}}{\pgfqpoint{2.980295in}{2.161519in}}%
\pgfpathcurveto{\pgfqpoint{2.986119in}{2.155695in}}{\pgfqpoint{2.994019in}{2.152423in}}{\pgfqpoint{3.002256in}{2.152423in}}%
\pgfpathclose%
\pgfusepath{stroke,fill}%
\end{pgfscope}%
\begin{pgfscope}%
\pgfpathrectangle{\pgfqpoint{0.100000in}{0.212622in}}{\pgfqpoint{3.696000in}{3.696000in}}%
\pgfusepath{clip}%
\pgfsetbuttcap%
\pgfsetroundjoin%
\definecolor{currentfill}{rgb}{0.121569,0.466667,0.705882}%
\pgfsetfillcolor{currentfill}%
\pgfsetfillopacity{0.705079}%
\pgfsetlinewidth{1.003750pt}%
\definecolor{currentstroke}{rgb}{0.121569,0.466667,0.705882}%
\pgfsetstrokecolor{currentstroke}%
\pgfsetstrokeopacity{0.705079}%
\pgfsetdash{}{0pt}%
\pgfpathmoveto{\pgfqpoint{3.000857in}{2.151809in}}%
\pgfpathcurveto{\pgfqpoint{3.009094in}{2.151809in}}{\pgfqpoint{3.016994in}{2.155082in}}{\pgfqpoint{3.022818in}{2.160906in}}%
\pgfpathcurveto{\pgfqpoint{3.028641in}{2.166730in}}{\pgfqpoint{3.031914in}{2.174630in}}{\pgfqpoint{3.031914in}{2.182866in}}%
\pgfpathcurveto{\pgfqpoint{3.031914in}{2.191102in}}{\pgfqpoint{3.028641in}{2.199002in}}{\pgfqpoint{3.022818in}{2.204826in}}%
\pgfpathcurveto{\pgfqpoint{3.016994in}{2.210650in}}{\pgfqpoint{3.009094in}{2.213922in}}{\pgfqpoint{3.000857in}{2.213922in}}%
\pgfpathcurveto{\pgfqpoint{2.992621in}{2.213922in}}{\pgfqpoint{2.984721in}{2.210650in}}{\pgfqpoint{2.978897in}{2.204826in}}%
\pgfpathcurveto{\pgfqpoint{2.973073in}{2.199002in}}{\pgfqpoint{2.969801in}{2.191102in}}{\pgfqpoint{2.969801in}{2.182866in}}%
\pgfpathcurveto{\pgfqpoint{2.969801in}{2.174630in}}{\pgfqpoint{2.973073in}{2.166730in}}{\pgfqpoint{2.978897in}{2.160906in}}%
\pgfpathcurveto{\pgfqpoint{2.984721in}{2.155082in}}{\pgfqpoint{2.992621in}{2.151809in}}{\pgfqpoint{3.000857in}{2.151809in}}%
\pgfpathclose%
\pgfusepath{stroke,fill}%
\end{pgfscope}%
\begin{pgfscope}%
\pgfpathrectangle{\pgfqpoint{0.100000in}{0.212622in}}{\pgfqpoint{3.696000in}{3.696000in}}%
\pgfusepath{clip}%
\pgfsetbuttcap%
\pgfsetroundjoin%
\definecolor{currentfill}{rgb}{0.121569,0.466667,0.705882}%
\pgfsetfillcolor{currentfill}%
\pgfsetfillopacity{0.706231}%
\pgfsetlinewidth{1.003750pt}%
\definecolor{currentstroke}{rgb}{0.121569,0.466667,0.705882}%
\pgfsetstrokecolor{currentstroke}%
\pgfsetstrokeopacity{0.706231}%
\pgfsetdash{}{0pt}%
\pgfpathmoveto{\pgfqpoint{2.999197in}{2.150995in}}%
\pgfpathcurveto{\pgfqpoint{3.007433in}{2.150995in}}{\pgfqpoint{3.015333in}{2.154267in}}{\pgfqpoint{3.021157in}{2.160091in}}%
\pgfpathcurveto{\pgfqpoint{3.026981in}{2.165915in}}{\pgfqpoint{3.030254in}{2.173815in}}{\pgfqpoint{3.030254in}{2.182052in}}%
\pgfpathcurveto{\pgfqpoint{3.030254in}{2.190288in}}{\pgfqpoint{3.026981in}{2.198188in}}{\pgfqpoint{3.021157in}{2.204012in}}%
\pgfpathcurveto{\pgfqpoint{3.015333in}{2.209836in}}{\pgfqpoint{3.007433in}{2.213108in}}{\pgfqpoint{2.999197in}{2.213108in}}%
\pgfpathcurveto{\pgfqpoint{2.990961in}{2.213108in}}{\pgfqpoint{2.983061in}{2.209836in}}{\pgfqpoint{2.977237in}{2.204012in}}%
\pgfpathcurveto{\pgfqpoint{2.971413in}{2.198188in}}{\pgfqpoint{2.968141in}{2.190288in}}{\pgfqpoint{2.968141in}{2.182052in}}%
\pgfpathcurveto{\pgfqpoint{2.968141in}{2.173815in}}{\pgfqpoint{2.971413in}{2.165915in}}{\pgfqpoint{2.977237in}{2.160091in}}%
\pgfpathcurveto{\pgfqpoint{2.983061in}{2.154267in}}{\pgfqpoint{2.990961in}{2.150995in}}{\pgfqpoint{2.999197in}{2.150995in}}%
\pgfpathclose%
\pgfusepath{stroke,fill}%
\end{pgfscope}%
\begin{pgfscope}%
\pgfpathrectangle{\pgfqpoint{0.100000in}{0.212622in}}{\pgfqpoint{3.696000in}{3.696000in}}%
\pgfusepath{clip}%
\pgfsetbuttcap%
\pgfsetroundjoin%
\definecolor{currentfill}{rgb}{0.121569,0.466667,0.705882}%
\pgfsetfillcolor{currentfill}%
\pgfsetfillopacity{0.706855}%
\pgfsetlinewidth{1.003750pt}%
\definecolor{currentstroke}{rgb}{0.121569,0.466667,0.705882}%
\pgfsetstrokecolor{currentstroke}%
\pgfsetstrokeopacity{0.706855}%
\pgfsetdash{}{0pt}%
\pgfpathmoveto{\pgfqpoint{2.998236in}{2.150530in}}%
\pgfpathcurveto{\pgfqpoint{3.006472in}{2.150530in}}{\pgfqpoint{3.014372in}{2.153803in}}{\pgfqpoint{3.020196in}{2.159627in}}%
\pgfpathcurveto{\pgfqpoint{3.026020in}{2.165451in}}{\pgfqpoint{3.029293in}{2.173351in}}{\pgfqpoint{3.029293in}{2.181587in}}%
\pgfpathcurveto{\pgfqpoint{3.029293in}{2.189823in}}{\pgfqpoint{3.026020in}{2.197723in}}{\pgfqpoint{3.020196in}{2.203547in}}%
\pgfpathcurveto{\pgfqpoint{3.014372in}{2.209371in}}{\pgfqpoint{3.006472in}{2.212643in}}{\pgfqpoint{2.998236in}{2.212643in}}%
\pgfpathcurveto{\pgfqpoint{2.990000in}{2.212643in}}{\pgfqpoint{2.982100in}{2.209371in}}{\pgfqpoint{2.976276in}{2.203547in}}%
\pgfpathcurveto{\pgfqpoint{2.970452in}{2.197723in}}{\pgfqpoint{2.967180in}{2.189823in}}{\pgfqpoint{2.967180in}{2.181587in}}%
\pgfpathcurveto{\pgfqpoint{2.967180in}{2.173351in}}{\pgfqpoint{2.970452in}{2.165451in}}{\pgfqpoint{2.976276in}{2.159627in}}%
\pgfpathcurveto{\pgfqpoint{2.982100in}{2.153803in}}{\pgfqpoint{2.990000in}{2.150530in}}{\pgfqpoint{2.998236in}{2.150530in}}%
\pgfpathclose%
\pgfusepath{stroke,fill}%
\end{pgfscope}%
\begin{pgfscope}%
\pgfpathrectangle{\pgfqpoint{0.100000in}{0.212622in}}{\pgfqpoint{3.696000in}{3.696000in}}%
\pgfusepath{clip}%
\pgfsetbuttcap%
\pgfsetroundjoin%
\definecolor{currentfill}{rgb}{0.121569,0.466667,0.705882}%
\pgfsetfillcolor{currentfill}%
\pgfsetfillopacity{0.707187}%
\pgfsetlinewidth{1.003750pt}%
\definecolor{currentstroke}{rgb}{0.121569,0.466667,0.705882}%
\pgfsetstrokecolor{currentstroke}%
\pgfsetstrokeopacity{0.707187}%
\pgfsetdash{}{0pt}%
\pgfpathmoveto{\pgfqpoint{2.997668in}{2.150239in}}%
\pgfpathcurveto{\pgfqpoint{3.005904in}{2.150239in}}{\pgfqpoint{3.013804in}{2.153511in}}{\pgfqpoint{3.019628in}{2.159335in}}%
\pgfpathcurveto{\pgfqpoint{3.025452in}{2.165159in}}{\pgfqpoint{3.028724in}{2.173059in}}{\pgfqpoint{3.028724in}{2.181295in}}%
\pgfpathcurveto{\pgfqpoint{3.028724in}{2.189532in}}{\pgfqpoint{3.025452in}{2.197432in}}{\pgfqpoint{3.019628in}{2.203256in}}%
\pgfpathcurveto{\pgfqpoint{3.013804in}{2.209080in}}{\pgfqpoint{3.005904in}{2.212352in}}{\pgfqpoint{2.997668in}{2.212352in}}%
\pgfpathcurveto{\pgfqpoint{2.989431in}{2.212352in}}{\pgfqpoint{2.981531in}{2.209080in}}{\pgfqpoint{2.975707in}{2.203256in}}%
\pgfpathcurveto{\pgfqpoint{2.969884in}{2.197432in}}{\pgfqpoint{2.966611in}{2.189532in}}{\pgfqpoint{2.966611in}{2.181295in}}%
\pgfpathcurveto{\pgfqpoint{2.966611in}{2.173059in}}{\pgfqpoint{2.969884in}{2.165159in}}{\pgfqpoint{2.975707in}{2.159335in}}%
\pgfpathcurveto{\pgfqpoint{2.981531in}{2.153511in}}{\pgfqpoint{2.989431in}{2.150239in}}{\pgfqpoint{2.997668in}{2.150239in}}%
\pgfpathclose%
\pgfusepath{stroke,fill}%
\end{pgfscope}%
\begin{pgfscope}%
\pgfpathrectangle{\pgfqpoint{0.100000in}{0.212622in}}{\pgfqpoint{3.696000in}{3.696000in}}%
\pgfusepath{clip}%
\pgfsetbuttcap%
\pgfsetroundjoin%
\definecolor{currentfill}{rgb}{0.121569,0.466667,0.705882}%
\pgfsetfillcolor{currentfill}%
\pgfsetfillopacity{0.707368}%
\pgfsetlinewidth{1.003750pt}%
\definecolor{currentstroke}{rgb}{0.121569,0.466667,0.705882}%
\pgfsetstrokecolor{currentstroke}%
\pgfsetstrokeopacity{0.707368}%
\pgfsetdash{}{0pt}%
\pgfpathmoveto{\pgfqpoint{2.997366in}{2.150052in}}%
\pgfpathcurveto{\pgfqpoint{3.005602in}{2.150052in}}{\pgfqpoint{3.013502in}{2.153325in}}{\pgfqpoint{3.019326in}{2.159148in}}%
\pgfpathcurveto{\pgfqpoint{3.025150in}{2.164972in}}{\pgfqpoint{3.028423in}{2.172872in}}{\pgfqpoint{3.028423in}{2.181109in}}%
\pgfpathcurveto{\pgfqpoint{3.028423in}{2.189345in}}{\pgfqpoint{3.025150in}{2.197245in}}{\pgfqpoint{3.019326in}{2.203069in}}%
\pgfpathcurveto{\pgfqpoint{3.013502in}{2.208893in}}{\pgfqpoint{3.005602in}{2.212165in}}{\pgfqpoint{2.997366in}{2.212165in}}%
\pgfpathcurveto{\pgfqpoint{2.989130in}{2.212165in}}{\pgfqpoint{2.981230in}{2.208893in}}{\pgfqpoint{2.975406in}{2.203069in}}%
\pgfpathcurveto{\pgfqpoint{2.969582in}{2.197245in}}{\pgfqpoint{2.966310in}{2.189345in}}{\pgfqpoint{2.966310in}{2.181109in}}%
\pgfpathcurveto{\pgfqpoint{2.966310in}{2.172872in}}{\pgfqpoint{2.969582in}{2.164972in}}{\pgfqpoint{2.975406in}{2.159148in}}%
\pgfpathcurveto{\pgfqpoint{2.981230in}{2.153325in}}{\pgfqpoint{2.989130in}{2.150052in}}{\pgfqpoint{2.997366in}{2.150052in}}%
\pgfpathclose%
\pgfusepath{stroke,fill}%
\end{pgfscope}%
\begin{pgfscope}%
\pgfpathrectangle{\pgfqpoint{0.100000in}{0.212622in}}{\pgfqpoint{3.696000in}{3.696000in}}%
\pgfusepath{clip}%
\pgfsetbuttcap%
\pgfsetroundjoin%
\definecolor{currentfill}{rgb}{0.121569,0.466667,0.705882}%
\pgfsetfillcolor{currentfill}%
\pgfsetfillopacity{0.707838}%
\pgfsetlinewidth{1.003750pt}%
\definecolor{currentstroke}{rgb}{0.121569,0.466667,0.705882}%
\pgfsetstrokecolor{currentstroke}%
\pgfsetstrokeopacity{0.707838}%
\pgfsetdash{}{0pt}%
\pgfpathmoveto{\pgfqpoint{2.996563in}{2.149652in}}%
\pgfpathcurveto{\pgfqpoint{3.004799in}{2.149652in}}{\pgfqpoint{3.012699in}{2.152925in}}{\pgfqpoint{3.018523in}{2.158749in}}%
\pgfpathcurveto{\pgfqpoint{3.024347in}{2.164573in}}{\pgfqpoint{3.027619in}{2.172473in}}{\pgfqpoint{3.027619in}{2.180709in}}%
\pgfpathcurveto{\pgfqpoint{3.027619in}{2.188945in}}{\pgfqpoint{3.024347in}{2.196845in}}{\pgfqpoint{3.018523in}{2.202669in}}%
\pgfpathcurveto{\pgfqpoint{3.012699in}{2.208493in}}{\pgfqpoint{3.004799in}{2.211765in}}{\pgfqpoint{2.996563in}{2.211765in}}%
\pgfpathcurveto{\pgfqpoint{2.988326in}{2.211765in}}{\pgfqpoint{2.980426in}{2.208493in}}{\pgfqpoint{2.974602in}{2.202669in}}%
\pgfpathcurveto{\pgfqpoint{2.968779in}{2.196845in}}{\pgfqpoint{2.965506in}{2.188945in}}{\pgfqpoint{2.965506in}{2.180709in}}%
\pgfpathcurveto{\pgfqpoint{2.965506in}{2.172473in}}{\pgfqpoint{2.968779in}{2.164573in}}{\pgfqpoint{2.974602in}{2.158749in}}%
\pgfpathcurveto{\pgfqpoint{2.980426in}{2.152925in}}{\pgfqpoint{2.988326in}{2.149652in}}{\pgfqpoint{2.996563in}{2.149652in}}%
\pgfpathclose%
\pgfusepath{stroke,fill}%
\end{pgfscope}%
\begin{pgfscope}%
\pgfpathrectangle{\pgfqpoint{0.100000in}{0.212622in}}{\pgfqpoint{3.696000in}{3.696000in}}%
\pgfusepath{clip}%
\pgfsetbuttcap%
\pgfsetroundjoin%
\definecolor{currentfill}{rgb}{0.121569,0.466667,0.705882}%
\pgfsetfillcolor{currentfill}%
\pgfsetfillopacity{0.708095}%
\pgfsetlinewidth{1.003750pt}%
\definecolor{currentstroke}{rgb}{0.121569,0.466667,0.705882}%
\pgfsetstrokecolor{currentstroke}%
\pgfsetstrokeopacity{0.708095}%
\pgfsetdash{}{0pt}%
\pgfpathmoveto{\pgfqpoint{2.996101in}{2.149436in}}%
\pgfpathcurveto{\pgfqpoint{3.004338in}{2.149436in}}{\pgfqpoint{3.012238in}{2.152708in}}{\pgfqpoint{3.018062in}{2.158532in}}%
\pgfpathcurveto{\pgfqpoint{3.023885in}{2.164356in}}{\pgfqpoint{3.027158in}{2.172256in}}{\pgfqpoint{3.027158in}{2.180492in}}%
\pgfpathcurveto{\pgfqpoint{3.027158in}{2.188728in}}{\pgfqpoint{3.023885in}{2.196628in}}{\pgfqpoint{3.018062in}{2.202452in}}%
\pgfpathcurveto{\pgfqpoint{3.012238in}{2.208276in}}{\pgfqpoint{3.004338in}{2.211549in}}{\pgfqpoint{2.996101in}{2.211549in}}%
\pgfpathcurveto{\pgfqpoint{2.987865in}{2.211549in}}{\pgfqpoint{2.979965in}{2.208276in}}{\pgfqpoint{2.974141in}{2.202452in}}%
\pgfpathcurveto{\pgfqpoint{2.968317in}{2.196628in}}{\pgfqpoint{2.965045in}{2.188728in}}{\pgfqpoint{2.965045in}{2.180492in}}%
\pgfpathcurveto{\pgfqpoint{2.965045in}{2.172256in}}{\pgfqpoint{2.968317in}{2.164356in}}{\pgfqpoint{2.974141in}{2.158532in}}%
\pgfpathcurveto{\pgfqpoint{2.979965in}{2.152708in}}{\pgfqpoint{2.987865in}{2.149436in}}{\pgfqpoint{2.996101in}{2.149436in}}%
\pgfpathclose%
\pgfusepath{stroke,fill}%
\end{pgfscope}%
\begin{pgfscope}%
\pgfpathrectangle{\pgfqpoint{0.100000in}{0.212622in}}{\pgfqpoint{3.696000in}{3.696000in}}%
\pgfusepath{clip}%
\pgfsetbuttcap%
\pgfsetroundjoin%
\definecolor{currentfill}{rgb}{0.121569,0.466667,0.705882}%
\pgfsetfillcolor{currentfill}%
\pgfsetfillopacity{0.708489}%
\pgfsetlinewidth{1.003750pt}%
\definecolor{currentstroke}{rgb}{0.121569,0.466667,0.705882}%
\pgfsetstrokecolor{currentstroke}%
\pgfsetstrokeopacity{0.708489}%
\pgfsetdash{}{0pt}%
\pgfpathmoveto{\pgfqpoint{2.995440in}{2.149081in}}%
\pgfpathcurveto{\pgfqpoint{3.003676in}{2.149081in}}{\pgfqpoint{3.011576in}{2.152354in}}{\pgfqpoint{3.017400in}{2.158177in}}%
\pgfpathcurveto{\pgfqpoint{3.023224in}{2.164001in}}{\pgfqpoint{3.026496in}{2.171901in}}{\pgfqpoint{3.026496in}{2.180138in}}%
\pgfpathcurveto{\pgfqpoint{3.026496in}{2.188374in}}{\pgfqpoint{3.023224in}{2.196274in}}{\pgfqpoint{3.017400in}{2.202098in}}%
\pgfpathcurveto{\pgfqpoint{3.011576in}{2.207922in}}{\pgfqpoint{3.003676in}{2.211194in}}{\pgfqpoint{2.995440in}{2.211194in}}%
\pgfpathcurveto{\pgfqpoint{2.987203in}{2.211194in}}{\pgfqpoint{2.979303in}{2.207922in}}{\pgfqpoint{2.973479in}{2.202098in}}%
\pgfpathcurveto{\pgfqpoint{2.967656in}{2.196274in}}{\pgfqpoint{2.964383in}{2.188374in}}{\pgfqpoint{2.964383in}{2.180138in}}%
\pgfpathcurveto{\pgfqpoint{2.964383in}{2.171901in}}{\pgfqpoint{2.967656in}{2.164001in}}{\pgfqpoint{2.973479in}{2.158177in}}%
\pgfpathcurveto{\pgfqpoint{2.979303in}{2.152354in}}{\pgfqpoint{2.987203in}{2.149081in}}{\pgfqpoint{2.995440in}{2.149081in}}%
\pgfpathclose%
\pgfusepath{stroke,fill}%
\end{pgfscope}%
\begin{pgfscope}%
\pgfpathrectangle{\pgfqpoint{0.100000in}{0.212622in}}{\pgfqpoint{3.696000in}{3.696000in}}%
\pgfusepath{clip}%
\pgfsetbuttcap%
\pgfsetroundjoin%
\definecolor{currentfill}{rgb}{0.121569,0.466667,0.705882}%
\pgfsetfillcolor{currentfill}%
\pgfsetfillopacity{0.709147}%
\pgfsetlinewidth{1.003750pt}%
\definecolor{currentstroke}{rgb}{0.121569,0.466667,0.705882}%
\pgfsetstrokecolor{currentstroke}%
\pgfsetstrokeopacity{0.709147}%
\pgfsetdash{}{0pt}%
\pgfpathmoveto{\pgfqpoint{2.994382in}{2.148718in}}%
\pgfpathcurveto{\pgfqpoint{3.002619in}{2.148718in}}{\pgfqpoint{3.010519in}{2.151990in}}{\pgfqpoint{3.016343in}{2.157814in}}%
\pgfpathcurveto{\pgfqpoint{3.022166in}{2.163638in}}{\pgfqpoint{3.025439in}{2.171538in}}{\pgfqpoint{3.025439in}{2.179774in}}%
\pgfpathcurveto{\pgfqpoint{3.025439in}{2.188011in}}{\pgfqpoint{3.022166in}{2.195911in}}{\pgfqpoint{3.016343in}{2.201735in}}%
\pgfpathcurveto{\pgfqpoint{3.010519in}{2.207559in}}{\pgfqpoint{3.002619in}{2.210831in}}{\pgfqpoint{2.994382in}{2.210831in}}%
\pgfpathcurveto{\pgfqpoint{2.986146in}{2.210831in}}{\pgfqpoint{2.978246in}{2.207559in}}{\pgfqpoint{2.972422in}{2.201735in}}%
\pgfpathcurveto{\pgfqpoint{2.966598in}{2.195911in}}{\pgfqpoint{2.963326in}{2.188011in}}{\pgfqpoint{2.963326in}{2.179774in}}%
\pgfpathcurveto{\pgfqpoint{2.963326in}{2.171538in}}{\pgfqpoint{2.966598in}{2.163638in}}{\pgfqpoint{2.972422in}{2.157814in}}%
\pgfpathcurveto{\pgfqpoint{2.978246in}{2.151990in}}{\pgfqpoint{2.986146in}{2.148718in}}{\pgfqpoint{2.994382in}{2.148718in}}%
\pgfpathclose%
\pgfusepath{stroke,fill}%
\end{pgfscope}%
\begin{pgfscope}%
\pgfpathrectangle{\pgfqpoint{0.100000in}{0.212622in}}{\pgfqpoint{3.696000in}{3.696000in}}%
\pgfusepath{clip}%
\pgfsetbuttcap%
\pgfsetroundjoin%
\definecolor{currentfill}{rgb}{0.121569,0.466667,0.705882}%
\pgfsetfillcolor{currentfill}%
\pgfsetfillopacity{0.709506}%
\pgfsetlinewidth{1.003750pt}%
\definecolor{currentstroke}{rgb}{0.121569,0.466667,0.705882}%
\pgfsetstrokecolor{currentstroke}%
\pgfsetstrokeopacity{0.709506}%
\pgfsetdash{}{0pt}%
\pgfpathmoveto{\pgfqpoint{2.993808in}{2.148491in}}%
\pgfpathcurveto{\pgfqpoint{3.002044in}{2.148491in}}{\pgfqpoint{3.009944in}{2.151763in}}{\pgfqpoint{3.015768in}{2.157587in}}%
\pgfpathcurveto{\pgfqpoint{3.021592in}{2.163411in}}{\pgfqpoint{3.024864in}{2.171311in}}{\pgfqpoint{3.024864in}{2.179547in}}%
\pgfpathcurveto{\pgfqpoint{3.024864in}{2.187783in}}{\pgfqpoint{3.021592in}{2.195683in}}{\pgfqpoint{3.015768in}{2.201507in}}%
\pgfpathcurveto{\pgfqpoint{3.009944in}{2.207331in}}{\pgfqpoint{3.002044in}{2.210604in}}{\pgfqpoint{2.993808in}{2.210604in}}%
\pgfpathcurveto{\pgfqpoint{2.985571in}{2.210604in}}{\pgfqpoint{2.977671in}{2.207331in}}{\pgfqpoint{2.971847in}{2.201507in}}%
\pgfpathcurveto{\pgfqpoint{2.966023in}{2.195683in}}{\pgfqpoint{2.962751in}{2.187783in}}{\pgfqpoint{2.962751in}{2.179547in}}%
\pgfpathcurveto{\pgfqpoint{2.962751in}{2.171311in}}{\pgfqpoint{2.966023in}{2.163411in}}{\pgfqpoint{2.971847in}{2.157587in}}%
\pgfpathcurveto{\pgfqpoint{2.977671in}{2.151763in}}{\pgfqpoint{2.985571in}{2.148491in}}{\pgfqpoint{2.993808in}{2.148491in}}%
\pgfpathclose%
\pgfusepath{stroke,fill}%
\end{pgfscope}%
\begin{pgfscope}%
\pgfpathrectangle{\pgfqpoint{0.100000in}{0.212622in}}{\pgfqpoint{3.696000in}{3.696000in}}%
\pgfusepath{clip}%
\pgfsetbuttcap%
\pgfsetroundjoin%
\definecolor{currentfill}{rgb}{0.121569,0.466667,0.705882}%
\pgfsetfillcolor{currentfill}%
\pgfsetfillopacity{0.709707}%
\pgfsetlinewidth{1.003750pt}%
\definecolor{currentstroke}{rgb}{0.121569,0.466667,0.705882}%
\pgfsetstrokecolor{currentstroke}%
\pgfsetstrokeopacity{0.709707}%
\pgfsetdash{}{0pt}%
\pgfpathmoveto{\pgfqpoint{2.993508in}{2.148376in}}%
\pgfpathcurveto{\pgfqpoint{3.001744in}{2.148376in}}{\pgfqpoint{3.009644in}{2.151648in}}{\pgfqpoint{3.015468in}{2.157472in}}%
\pgfpathcurveto{\pgfqpoint{3.021292in}{2.163296in}}{\pgfqpoint{3.024564in}{2.171196in}}{\pgfqpoint{3.024564in}{2.179432in}}%
\pgfpathcurveto{\pgfqpoint{3.024564in}{2.187668in}}{\pgfqpoint{3.021292in}{2.195568in}}{\pgfqpoint{3.015468in}{2.201392in}}%
\pgfpathcurveto{\pgfqpoint{3.009644in}{2.207216in}}{\pgfqpoint{3.001744in}{2.210489in}}{\pgfqpoint{2.993508in}{2.210489in}}%
\pgfpathcurveto{\pgfqpoint{2.985272in}{2.210489in}}{\pgfqpoint{2.977372in}{2.207216in}}{\pgfqpoint{2.971548in}{2.201392in}}%
\pgfpathcurveto{\pgfqpoint{2.965724in}{2.195568in}}{\pgfqpoint{2.962451in}{2.187668in}}{\pgfqpoint{2.962451in}{2.179432in}}%
\pgfpathcurveto{\pgfqpoint{2.962451in}{2.171196in}}{\pgfqpoint{2.965724in}{2.163296in}}{\pgfqpoint{2.971548in}{2.157472in}}%
\pgfpathcurveto{\pgfqpoint{2.977372in}{2.151648in}}{\pgfqpoint{2.985272in}{2.148376in}}{\pgfqpoint{2.993508in}{2.148376in}}%
\pgfpathclose%
\pgfusepath{stroke,fill}%
\end{pgfscope}%
\begin{pgfscope}%
\pgfpathrectangle{\pgfqpoint{0.100000in}{0.212622in}}{\pgfqpoint{3.696000in}{3.696000in}}%
\pgfusepath{clip}%
\pgfsetbuttcap%
\pgfsetroundjoin%
\definecolor{currentfill}{rgb}{0.121569,0.466667,0.705882}%
\pgfsetfillcolor{currentfill}%
\pgfsetfillopacity{0.709813}%
\pgfsetlinewidth{1.003750pt}%
\definecolor{currentstroke}{rgb}{0.121569,0.466667,0.705882}%
\pgfsetstrokecolor{currentstroke}%
\pgfsetstrokeopacity{0.709813}%
\pgfsetdash{}{0pt}%
\pgfpathmoveto{\pgfqpoint{2.993328in}{2.148297in}}%
\pgfpathcurveto{\pgfqpoint{3.001565in}{2.148297in}}{\pgfqpoint{3.009465in}{2.151569in}}{\pgfqpoint{3.015289in}{2.157393in}}%
\pgfpathcurveto{\pgfqpoint{3.021113in}{2.163217in}}{\pgfqpoint{3.024385in}{2.171117in}}{\pgfqpoint{3.024385in}{2.179353in}}%
\pgfpathcurveto{\pgfqpoint{3.024385in}{2.187589in}}{\pgfqpoint{3.021113in}{2.195490in}}{\pgfqpoint{3.015289in}{2.201313in}}%
\pgfpathcurveto{\pgfqpoint{3.009465in}{2.207137in}}{\pgfqpoint{3.001565in}{2.210410in}}{\pgfqpoint{2.993328in}{2.210410in}}%
\pgfpathcurveto{\pgfqpoint{2.985092in}{2.210410in}}{\pgfqpoint{2.977192in}{2.207137in}}{\pgfqpoint{2.971368in}{2.201313in}}%
\pgfpathcurveto{\pgfqpoint{2.965544in}{2.195490in}}{\pgfqpoint{2.962272in}{2.187589in}}{\pgfqpoint{2.962272in}{2.179353in}}%
\pgfpathcurveto{\pgfqpoint{2.962272in}{2.171117in}}{\pgfqpoint{2.965544in}{2.163217in}}{\pgfqpoint{2.971368in}{2.157393in}}%
\pgfpathcurveto{\pgfqpoint{2.977192in}{2.151569in}}{\pgfqpoint{2.985092in}{2.148297in}}{\pgfqpoint{2.993328in}{2.148297in}}%
\pgfpathclose%
\pgfusepath{stroke,fill}%
\end{pgfscope}%
\begin{pgfscope}%
\pgfpathrectangle{\pgfqpoint{0.100000in}{0.212622in}}{\pgfqpoint{3.696000in}{3.696000in}}%
\pgfusepath{clip}%
\pgfsetbuttcap%
\pgfsetroundjoin%
\definecolor{currentfill}{rgb}{0.121569,0.466667,0.705882}%
\pgfsetfillcolor{currentfill}%
\pgfsetfillopacity{0.710240}%
\pgfsetlinewidth{1.003750pt}%
\definecolor{currentstroke}{rgb}{0.121569,0.466667,0.705882}%
\pgfsetstrokecolor{currentstroke}%
\pgfsetstrokeopacity{0.710240}%
\pgfsetdash{}{0pt}%
\pgfpathmoveto{\pgfqpoint{2.992719in}{2.148163in}}%
\pgfpathcurveto{\pgfqpoint{3.000955in}{2.148163in}}{\pgfqpoint{3.008856in}{2.151436in}}{\pgfqpoint{3.014679in}{2.157259in}}%
\pgfpathcurveto{\pgfqpoint{3.020503in}{2.163083in}}{\pgfqpoint{3.023776in}{2.170983in}}{\pgfqpoint{3.023776in}{2.179220in}}%
\pgfpathcurveto{\pgfqpoint{3.023776in}{2.187456in}}{\pgfqpoint{3.020503in}{2.195356in}}{\pgfqpoint{3.014679in}{2.201180in}}%
\pgfpathcurveto{\pgfqpoint{3.008856in}{2.207004in}}{\pgfqpoint{3.000955in}{2.210276in}}{\pgfqpoint{2.992719in}{2.210276in}}%
\pgfpathcurveto{\pgfqpoint{2.984483in}{2.210276in}}{\pgfqpoint{2.976583in}{2.207004in}}{\pgfqpoint{2.970759in}{2.201180in}}%
\pgfpathcurveto{\pgfqpoint{2.964935in}{2.195356in}}{\pgfqpoint{2.961663in}{2.187456in}}{\pgfqpoint{2.961663in}{2.179220in}}%
\pgfpathcurveto{\pgfqpoint{2.961663in}{2.170983in}}{\pgfqpoint{2.964935in}{2.163083in}}{\pgfqpoint{2.970759in}{2.157259in}}%
\pgfpathcurveto{\pgfqpoint{2.976583in}{2.151436in}}{\pgfqpoint{2.984483in}{2.148163in}}{\pgfqpoint{2.992719in}{2.148163in}}%
\pgfpathclose%
\pgfusepath{stroke,fill}%
\end{pgfscope}%
\begin{pgfscope}%
\pgfpathrectangle{\pgfqpoint{0.100000in}{0.212622in}}{\pgfqpoint{3.696000in}{3.696000in}}%
\pgfusepath{clip}%
\pgfsetbuttcap%
\pgfsetroundjoin%
\definecolor{currentfill}{rgb}{0.121569,0.466667,0.705882}%
\pgfsetfillcolor{currentfill}%
\pgfsetfillopacity{0.710850}%
\pgfsetlinewidth{1.003750pt}%
\definecolor{currentstroke}{rgb}{0.121569,0.466667,0.705882}%
\pgfsetstrokecolor{currentstroke}%
\pgfsetstrokeopacity{0.710850}%
\pgfsetdash{}{0pt}%
\pgfpathmoveto{\pgfqpoint{2.991875in}{2.147997in}}%
\pgfpathcurveto{\pgfqpoint{3.000111in}{2.147997in}}{\pgfqpoint{3.008011in}{2.151269in}}{\pgfqpoint{3.013835in}{2.157093in}}%
\pgfpathcurveto{\pgfqpoint{3.019659in}{2.162917in}}{\pgfqpoint{3.022932in}{2.170817in}}{\pgfqpoint{3.022932in}{2.179053in}}%
\pgfpathcurveto{\pgfqpoint{3.022932in}{2.187290in}}{\pgfqpoint{3.019659in}{2.195190in}}{\pgfqpoint{3.013835in}{2.201014in}}%
\pgfpathcurveto{\pgfqpoint{3.008011in}{2.206838in}}{\pgfqpoint{3.000111in}{2.210110in}}{\pgfqpoint{2.991875in}{2.210110in}}%
\pgfpathcurveto{\pgfqpoint{2.983639in}{2.210110in}}{\pgfqpoint{2.975739in}{2.206838in}}{\pgfqpoint{2.969915in}{2.201014in}}%
\pgfpathcurveto{\pgfqpoint{2.964091in}{2.195190in}}{\pgfqpoint{2.960819in}{2.187290in}}{\pgfqpoint{2.960819in}{2.179053in}}%
\pgfpathcurveto{\pgfqpoint{2.960819in}{2.170817in}}{\pgfqpoint{2.964091in}{2.162917in}}{\pgfqpoint{2.969915in}{2.157093in}}%
\pgfpathcurveto{\pgfqpoint{2.975739in}{2.151269in}}{\pgfqpoint{2.983639in}{2.147997in}}{\pgfqpoint{2.991875in}{2.147997in}}%
\pgfpathclose%
\pgfusepath{stroke,fill}%
\end{pgfscope}%
\begin{pgfscope}%
\pgfpathrectangle{\pgfqpoint{0.100000in}{0.212622in}}{\pgfqpoint{3.696000in}{3.696000in}}%
\pgfusepath{clip}%
\pgfsetbuttcap%
\pgfsetroundjoin%
\definecolor{currentfill}{rgb}{0.121569,0.466667,0.705882}%
\pgfsetfillcolor{currentfill}%
\pgfsetfillopacity{0.711575}%
\pgfsetlinewidth{1.003750pt}%
\definecolor{currentstroke}{rgb}{0.121569,0.466667,0.705882}%
\pgfsetstrokecolor{currentstroke}%
\pgfsetstrokeopacity{0.711575}%
\pgfsetdash{}{0pt}%
\pgfpathmoveto{\pgfqpoint{2.990746in}{2.147632in}}%
\pgfpathcurveto{\pgfqpoint{2.998983in}{2.147632in}}{\pgfqpoint{3.006883in}{2.150904in}}{\pgfqpoint{3.012707in}{2.156728in}}%
\pgfpathcurveto{\pgfqpoint{3.018531in}{2.162552in}}{\pgfqpoint{3.021803in}{2.170452in}}{\pgfqpoint{3.021803in}{2.178689in}}%
\pgfpathcurveto{\pgfqpoint{3.021803in}{2.186925in}}{\pgfqpoint{3.018531in}{2.194825in}}{\pgfqpoint{3.012707in}{2.200649in}}%
\pgfpathcurveto{\pgfqpoint{3.006883in}{2.206473in}}{\pgfqpoint{2.998983in}{2.209745in}}{\pgfqpoint{2.990746in}{2.209745in}}%
\pgfpathcurveto{\pgfqpoint{2.982510in}{2.209745in}}{\pgfqpoint{2.974610in}{2.206473in}}{\pgfqpoint{2.968786in}{2.200649in}}%
\pgfpathcurveto{\pgfqpoint{2.962962in}{2.194825in}}{\pgfqpoint{2.959690in}{2.186925in}}{\pgfqpoint{2.959690in}{2.178689in}}%
\pgfpathcurveto{\pgfqpoint{2.959690in}{2.170452in}}{\pgfqpoint{2.962962in}{2.162552in}}{\pgfqpoint{2.968786in}{2.156728in}}%
\pgfpathcurveto{\pgfqpoint{2.974610in}{2.150904in}}{\pgfqpoint{2.982510in}{2.147632in}}{\pgfqpoint{2.990746in}{2.147632in}}%
\pgfpathclose%
\pgfusepath{stroke,fill}%
\end{pgfscope}%
\begin{pgfscope}%
\pgfpathrectangle{\pgfqpoint{0.100000in}{0.212622in}}{\pgfqpoint{3.696000in}{3.696000in}}%
\pgfusepath{clip}%
\pgfsetbuttcap%
\pgfsetroundjoin%
\definecolor{currentfill}{rgb}{0.121569,0.466667,0.705882}%
\pgfsetfillcolor{currentfill}%
\pgfsetfillopacity{0.712703}%
\pgfsetlinewidth{1.003750pt}%
\definecolor{currentstroke}{rgb}{0.121569,0.466667,0.705882}%
\pgfsetstrokecolor{currentstroke}%
\pgfsetstrokeopacity{0.712703}%
\pgfsetdash{}{0pt}%
\pgfpathmoveto{\pgfqpoint{2.989163in}{2.146665in}}%
\pgfpathcurveto{\pgfqpoint{2.997400in}{2.146665in}}{\pgfqpoint{3.005300in}{2.149938in}}{\pgfqpoint{3.011124in}{2.155761in}}%
\pgfpathcurveto{\pgfqpoint{3.016948in}{2.161585in}}{\pgfqpoint{3.020220in}{2.169485in}}{\pgfqpoint{3.020220in}{2.177722in}}%
\pgfpathcurveto{\pgfqpoint{3.020220in}{2.185958in}}{\pgfqpoint{3.016948in}{2.193858in}}{\pgfqpoint{3.011124in}{2.199682in}}%
\pgfpathcurveto{\pgfqpoint{3.005300in}{2.205506in}}{\pgfqpoint{2.997400in}{2.208778in}}{\pgfqpoint{2.989163in}{2.208778in}}%
\pgfpathcurveto{\pgfqpoint{2.980927in}{2.208778in}}{\pgfqpoint{2.973027in}{2.205506in}}{\pgfqpoint{2.967203in}{2.199682in}}%
\pgfpathcurveto{\pgfqpoint{2.961379in}{2.193858in}}{\pgfqpoint{2.958107in}{2.185958in}}{\pgfqpoint{2.958107in}{2.177722in}}%
\pgfpathcurveto{\pgfqpoint{2.958107in}{2.169485in}}{\pgfqpoint{2.961379in}{2.161585in}}{\pgfqpoint{2.967203in}{2.155761in}}%
\pgfpathcurveto{\pgfqpoint{2.973027in}{2.149938in}}{\pgfqpoint{2.980927in}{2.146665in}}{\pgfqpoint{2.989163in}{2.146665in}}%
\pgfpathclose%
\pgfusepath{stroke,fill}%
\end{pgfscope}%
\begin{pgfscope}%
\pgfpathrectangle{\pgfqpoint{0.100000in}{0.212622in}}{\pgfqpoint{3.696000in}{3.696000in}}%
\pgfusepath{clip}%
\pgfsetbuttcap%
\pgfsetroundjoin%
\definecolor{currentfill}{rgb}{0.121569,0.466667,0.705882}%
\pgfsetfillcolor{currentfill}%
\pgfsetfillopacity{0.714051}%
\pgfsetlinewidth{1.003750pt}%
\definecolor{currentstroke}{rgb}{0.121569,0.466667,0.705882}%
\pgfsetstrokecolor{currentstroke}%
\pgfsetstrokeopacity{0.714051}%
\pgfsetdash{}{0pt}%
\pgfpathmoveto{\pgfqpoint{2.987238in}{2.145650in}}%
\pgfpathcurveto{\pgfqpoint{2.995474in}{2.145650in}}{\pgfqpoint{3.003374in}{2.148922in}}{\pgfqpoint{3.009198in}{2.154746in}}%
\pgfpathcurveto{\pgfqpoint{3.015022in}{2.160570in}}{\pgfqpoint{3.018295in}{2.168470in}}{\pgfqpoint{3.018295in}{2.176707in}}%
\pgfpathcurveto{\pgfqpoint{3.018295in}{2.184943in}}{\pgfqpoint{3.015022in}{2.192843in}}{\pgfqpoint{3.009198in}{2.198667in}}%
\pgfpathcurveto{\pgfqpoint{3.003374in}{2.204491in}}{\pgfqpoint{2.995474in}{2.207763in}}{\pgfqpoint{2.987238in}{2.207763in}}%
\pgfpathcurveto{\pgfqpoint{2.979002in}{2.207763in}}{\pgfqpoint{2.971102in}{2.204491in}}{\pgfqpoint{2.965278in}{2.198667in}}%
\pgfpathcurveto{\pgfqpoint{2.959454in}{2.192843in}}{\pgfqpoint{2.956182in}{2.184943in}}{\pgfqpoint{2.956182in}{2.176707in}}%
\pgfpathcurveto{\pgfqpoint{2.956182in}{2.168470in}}{\pgfqpoint{2.959454in}{2.160570in}}{\pgfqpoint{2.965278in}{2.154746in}}%
\pgfpathcurveto{\pgfqpoint{2.971102in}{2.148922in}}{\pgfqpoint{2.979002in}{2.145650in}}{\pgfqpoint{2.987238in}{2.145650in}}%
\pgfpathclose%
\pgfusepath{stroke,fill}%
\end{pgfscope}%
\begin{pgfscope}%
\pgfpathrectangle{\pgfqpoint{0.100000in}{0.212622in}}{\pgfqpoint{3.696000in}{3.696000in}}%
\pgfusepath{clip}%
\pgfsetbuttcap%
\pgfsetroundjoin%
\definecolor{currentfill}{rgb}{0.121569,0.466667,0.705882}%
\pgfsetfillcolor{currentfill}%
\pgfsetfillopacity{0.715476}%
\pgfsetlinewidth{1.003750pt}%
\definecolor{currentstroke}{rgb}{0.121569,0.466667,0.705882}%
\pgfsetstrokecolor{currentstroke}%
\pgfsetstrokeopacity{0.715476}%
\pgfsetdash{}{0pt}%
\pgfpathmoveto{\pgfqpoint{2.984894in}{2.144280in}}%
\pgfpathcurveto{\pgfqpoint{2.993131in}{2.144280in}}{\pgfqpoint{3.001031in}{2.147552in}}{\pgfqpoint{3.006855in}{2.153376in}}%
\pgfpathcurveto{\pgfqpoint{3.012678in}{2.159200in}}{\pgfqpoint{3.015951in}{2.167100in}}{\pgfqpoint{3.015951in}{2.175337in}}%
\pgfpathcurveto{\pgfqpoint{3.015951in}{2.183573in}}{\pgfqpoint{3.012678in}{2.191473in}}{\pgfqpoint{3.006855in}{2.197297in}}%
\pgfpathcurveto{\pgfqpoint{3.001031in}{2.203121in}}{\pgfqpoint{2.993131in}{2.206393in}}{\pgfqpoint{2.984894in}{2.206393in}}%
\pgfpathcurveto{\pgfqpoint{2.976658in}{2.206393in}}{\pgfqpoint{2.968758in}{2.203121in}}{\pgfqpoint{2.962934in}{2.197297in}}%
\pgfpathcurveto{\pgfqpoint{2.957110in}{2.191473in}}{\pgfqpoint{2.953838in}{2.183573in}}{\pgfqpoint{2.953838in}{2.175337in}}%
\pgfpathcurveto{\pgfqpoint{2.953838in}{2.167100in}}{\pgfqpoint{2.957110in}{2.159200in}}{\pgfqpoint{2.962934in}{2.153376in}}%
\pgfpathcurveto{\pgfqpoint{2.968758in}{2.147552in}}{\pgfqpoint{2.976658in}{2.144280in}}{\pgfqpoint{2.984894in}{2.144280in}}%
\pgfpathclose%
\pgfusepath{stroke,fill}%
\end{pgfscope}%
\begin{pgfscope}%
\pgfpathrectangle{\pgfqpoint{0.100000in}{0.212622in}}{\pgfqpoint{3.696000in}{3.696000in}}%
\pgfusepath{clip}%
\pgfsetbuttcap%
\pgfsetroundjoin%
\definecolor{currentfill}{rgb}{0.121569,0.466667,0.705882}%
\pgfsetfillcolor{currentfill}%
\pgfsetfillopacity{0.717196}%
\pgfsetlinewidth{1.003750pt}%
\definecolor{currentstroke}{rgb}{0.121569,0.466667,0.705882}%
\pgfsetstrokecolor{currentstroke}%
\pgfsetstrokeopacity{0.717196}%
\pgfsetdash{}{0pt}%
\pgfpathmoveto{\pgfqpoint{2.982260in}{2.142264in}}%
\pgfpathcurveto{\pgfqpoint{2.990497in}{2.142264in}}{\pgfqpoint{2.998397in}{2.145537in}}{\pgfqpoint{3.004221in}{2.151361in}}%
\pgfpathcurveto{\pgfqpoint{3.010045in}{2.157185in}}{\pgfqpoint{3.013317in}{2.165085in}}{\pgfqpoint{3.013317in}{2.173321in}}%
\pgfpathcurveto{\pgfqpoint{3.013317in}{2.181557in}}{\pgfqpoint{3.010045in}{2.189457in}}{\pgfqpoint{3.004221in}{2.195281in}}%
\pgfpathcurveto{\pgfqpoint{2.998397in}{2.201105in}}{\pgfqpoint{2.990497in}{2.204377in}}{\pgfqpoint{2.982260in}{2.204377in}}%
\pgfpathcurveto{\pgfqpoint{2.974024in}{2.204377in}}{\pgfqpoint{2.966124in}{2.201105in}}{\pgfqpoint{2.960300in}{2.195281in}}%
\pgfpathcurveto{\pgfqpoint{2.954476in}{2.189457in}}{\pgfqpoint{2.951204in}{2.181557in}}{\pgfqpoint{2.951204in}{2.173321in}}%
\pgfpathcurveto{\pgfqpoint{2.951204in}{2.165085in}}{\pgfqpoint{2.954476in}{2.157185in}}{\pgfqpoint{2.960300in}{2.151361in}}%
\pgfpathcurveto{\pgfqpoint{2.966124in}{2.145537in}}{\pgfqpoint{2.974024in}{2.142264in}}{\pgfqpoint{2.982260in}{2.142264in}}%
\pgfpathclose%
\pgfusepath{stroke,fill}%
\end{pgfscope}%
\begin{pgfscope}%
\pgfpathrectangle{\pgfqpoint{0.100000in}{0.212622in}}{\pgfqpoint{3.696000in}{3.696000in}}%
\pgfusepath{clip}%
\pgfsetbuttcap%
\pgfsetroundjoin%
\definecolor{currentfill}{rgb}{0.121569,0.466667,0.705882}%
\pgfsetfillcolor{currentfill}%
\pgfsetfillopacity{0.718212}%
\pgfsetlinewidth{1.003750pt}%
\definecolor{currentstroke}{rgb}{0.121569,0.466667,0.705882}%
\pgfsetstrokecolor{currentstroke}%
\pgfsetstrokeopacity{0.718212}%
\pgfsetdash{}{0pt}%
\pgfpathmoveto{\pgfqpoint{2.980826in}{2.141601in}}%
\pgfpathcurveto{\pgfqpoint{2.989062in}{2.141601in}}{\pgfqpoint{2.996962in}{2.144873in}}{\pgfqpoint{3.002786in}{2.150697in}}%
\pgfpathcurveto{\pgfqpoint{3.008610in}{2.156521in}}{\pgfqpoint{3.011882in}{2.164421in}}{\pgfqpoint{3.011882in}{2.172657in}}%
\pgfpathcurveto{\pgfqpoint{3.011882in}{2.180894in}}{\pgfqpoint{3.008610in}{2.188794in}}{\pgfqpoint{3.002786in}{2.194618in}}%
\pgfpathcurveto{\pgfqpoint{2.996962in}{2.200441in}}{\pgfqpoint{2.989062in}{2.203714in}}{\pgfqpoint{2.980826in}{2.203714in}}%
\pgfpathcurveto{\pgfqpoint{2.972589in}{2.203714in}}{\pgfqpoint{2.964689in}{2.200441in}}{\pgfqpoint{2.958865in}{2.194618in}}%
\pgfpathcurveto{\pgfqpoint{2.953041in}{2.188794in}}{\pgfqpoint{2.949769in}{2.180894in}}{\pgfqpoint{2.949769in}{2.172657in}}%
\pgfpathcurveto{\pgfqpoint{2.949769in}{2.164421in}}{\pgfqpoint{2.953041in}{2.156521in}}{\pgfqpoint{2.958865in}{2.150697in}}%
\pgfpathcurveto{\pgfqpoint{2.964689in}{2.144873in}}{\pgfqpoint{2.972589in}{2.141601in}}{\pgfqpoint{2.980826in}{2.141601in}}%
\pgfpathclose%
\pgfusepath{stroke,fill}%
\end{pgfscope}%
\begin{pgfscope}%
\pgfpathrectangle{\pgfqpoint{0.100000in}{0.212622in}}{\pgfqpoint{3.696000in}{3.696000in}}%
\pgfusepath{clip}%
\pgfsetbuttcap%
\pgfsetroundjoin%
\definecolor{currentfill}{rgb}{0.121569,0.466667,0.705882}%
\pgfsetfillcolor{currentfill}%
\pgfsetfillopacity{0.718751}%
\pgfsetlinewidth{1.003750pt}%
\definecolor{currentstroke}{rgb}{0.121569,0.466667,0.705882}%
\pgfsetstrokecolor{currentstroke}%
\pgfsetstrokeopacity{0.718751}%
\pgfsetdash{}{0pt}%
\pgfpathmoveto{\pgfqpoint{2.979971in}{2.141167in}}%
\pgfpathcurveto{\pgfqpoint{2.988207in}{2.141167in}}{\pgfqpoint{2.996107in}{2.144439in}}{\pgfqpoint{3.001931in}{2.150263in}}%
\pgfpathcurveto{\pgfqpoint{3.007755in}{2.156087in}}{\pgfqpoint{3.011027in}{2.163987in}}{\pgfqpoint{3.011027in}{2.172224in}}%
\pgfpathcurveto{\pgfqpoint{3.011027in}{2.180460in}}{\pgfqpoint{3.007755in}{2.188360in}}{\pgfqpoint{3.001931in}{2.194184in}}%
\pgfpathcurveto{\pgfqpoint{2.996107in}{2.200008in}}{\pgfqpoint{2.988207in}{2.203280in}}{\pgfqpoint{2.979971in}{2.203280in}}%
\pgfpathcurveto{\pgfqpoint{2.971735in}{2.203280in}}{\pgfqpoint{2.963834in}{2.200008in}}{\pgfqpoint{2.958011in}{2.194184in}}%
\pgfpathcurveto{\pgfqpoint{2.952187in}{2.188360in}}{\pgfqpoint{2.948914in}{2.180460in}}{\pgfqpoint{2.948914in}{2.172224in}}%
\pgfpathcurveto{\pgfqpoint{2.948914in}{2.163987in}}{\pgfqpoint{2.952187in}{2.156087in}}{\pgfqpoint{2.958011in}{2.150263in}}%
\pgfpathcurveto{\pgfqpoint{2.963834in}{2.144439in}}{\pgfqpoint{2.971735in}{2.141167in}}{\pgfqpoint{2.979971in}{2.141167in}}%
\pgfpathclose%
\pgfusepath{stroke,fill}%
\end{pgfscope}%
\begin{pgfscope}%
\pgfpathrectangle{\pgfqpoint{0.100000in}{0.212622in}}{\pgfqpoint{3.696000in}{3.696000in}}%
\pgfusepath{clip}%
\pgfsetbuttcap%
\pgfsetroundjoin%
\definecolor{currentfill}{rgb}{0.121569,0.466667,0.705882}%
\pgfsetfillcolor{currentfill}%
\pgfsetfillopacity{0.719049}%
\pgfsetlinewidth{1.003750pt}%
\definecolor{currentstroke}{rgb}{0.121569,0.466667,0.705882}%
\pgfsetstrokecolor{currentstroke}%
\pgfsetstrokeopacity{0.719049}%
\pgfsetdash{}{0pt}%
\pgfpathmoveto{\pgfqpoint{2.979529in}{2.140912in}}%
\pgfpathcurveto{\pgfqpoint{2.987765in}{2.140912in}}{\pgfqpoint{2.995665in}{2.144185in}}{\pgfqpoint{3.001489in}{2.150009in}}%
\pgfpathcurveto{\pgfqpoint{3.007313in}{2.155833in}}{\pgfqpoint{3.010585in}{2.163733in}}{\pgfqpoint{3.010585in}{2.171969in}}%
\pgfpathcurveto{\pgfqpoint{3.010585in}{2.180205in}}{\pgfqpoint{3.007313in}{2.188105in}}{\pgfqpoint{3.001489in}{2.193929in}}%
\pgfpathcurveto{\pgfqpoint{2.995665in}{2.199753in}}{\pgfqpoint{2.987765in}{2.203025in}}{\pgfqpoint{2.979529in}{2.203025in}}%
\pgfpathcurveto{\pgfqpoint{2.971293in}{2.203025in}}{\pgfqpoint{2.963393in}{2.199753in}}{\pgfqpoint{2.957569in}{2.193929in}}%
\pgfpathcurveto{\pgfqpoint{2.951745in}{2.188105in}}{\pgfqpoint{2.948472in}{2.180205in}}{\pgfqpoint{2.948472in}{2.171969in}}%
\pgfpathcurveto{\pgfqpoint{2.948472in}{2.163733in}}{\pgfqpoint{2.951745in}{2.155833in}}{\pgfqpoint{2.957569in}{2.150009in}}%
\pgfpathcurveto{\pgfqpoint{2.963393in}{2.144185in}}{\pgfqpoint{2.971293in}{2.140912in}}{\pgfqpoint{2.979529in}{2.140912in}}%
\pgfpathclose%
\pgfusepath{stroke,fill}%
\end{pgfscope}%
\begin{pgfscope}%
\pgfpathrectangle{\pgfqpoint{0.100000in}{0.212622in}}{\pgfqpoint{3.696000in}{3.696000in}}%
\pgfusepath{clip}%
\pgfsetbuttcap%
\pgfsetroundjoin%
\definecolor{currentfill}{rgb}{0.121569,0.466667,0.705882}%
\pgfsetfillcolor{currentfill}%
\pgfsetfillopacity{0.719629}%
\pgfsetlinewidth{1.003750pt}%
\definecolor{currentstroke}{rgb}{0.121569,0.466667,0.705882}%
\pgfsetstrokecolor{currentstroke}%
\pgfsetstrokeopacity{0.719629}%
\pgfsetdash{}{0pt}%
\pgfpathmoveto{\pgfqpoint{2.978516in}{2.140363in}}%
\pgfpathcurveto{\pgfqpoint{2.986752in}{2.140363in}}{\pgfqpoint{2.994652in}{2.143635in}}{\pgfqpoint{3.000476in}{2.149459in}}%
\pgfpathcurveto{\pgfqpoint{3.006300in}{2.155283in}}{\pgfqpoint{3.009572in}{2.163183in}}{\pgfqpoint{3.009572in}{2.171419in}}%
\pgfpathcurveto{\pgfqpoint{3.009572in}{2.179655in}}{\pgfqpoint{3.006300in}{2.187555in}}{\pgfqpoint{3.000476in}{2.193379in}}%
\pgfpathcurveto{\pgfqpoint{2.994652in}{2.199203in}}{\pgfqpoint{2.986752in}{2.202476in}}{\pgfqpoint{2.978516in}{2.202476in}}%
\pgfpathcurveto{\pgfqpoint{2.970279in}{2.202476in}}{\pgfqpoint{2.962379in}{2.199203in}}{\pgfqpoint{2.956555in}{2.193379in}}%
\pgfpathcurveto{\pgfqpoint{2.950731in}{2.187555in}}{\pgfqpoint{2.947459in}{2.179655in}}{\pgfqpoint{2.947459in}{2.171419in}}%
\pgfpathcurveto{\pgfqpoint{2.947459in}{2.163183in}}{\pgfqpoint{2.950731in}{2.155283in}}{\pgfqpoint{2.956555in}{2.149459in}}%
\pgfpathcurveto{\pgfqpoint{2.962379in}{2.143635in}}{\pgfqpoint{2.970279in}{2.140363in}}{\pgfqpoint{2.978516in}{2.140363in}}%
\pgfpathclose%
\pgfusepath{stroke,fill}%
\end{pgfscope}%
\begin{pgfscope}%
\pgfpathrectangle{\pgfqpoint{0.100000in}{0.212622in}}{\pgfqpoint{3.696000in}{3.696000in}}%
\pgfusepath{clip}%
\pgfsetbuttcap%
\pgfsetroundjoin%
\definecolor{currentfill}{rgb}{0.121569,0.466667,0.705882}%
\pgfsetfillcolor{currentfill}%
\pgfsetfillopacity{0.719942}%
\pgfsetlinewidth{1.003750pt}%
\definecolor{currentstroke}{rgb}{0.121569,0.466667,0.705882}%
\pgfsetstrokecolor{currentstroke}%
\pgfsetstrokeopacity{0.719942}%
\pgfsetdash{}{0pt}%
\pgfpathmoveto{\pgfqpoint{2.977936in}{2.140044in}}%
\pgfpathcurveto{\pgfqpoint{2.986172in}{2.140044in}}{\pgfqpoint{2.994072in}{2.143316in}}{\pgfqpoint{2.999896in}{2.149140in}}%
\pgfpathcurveto{\pgfqpoint{3.005720in}{2.154964in}}{\pgfqpoint{3.008993in}{2.162864in}}{\pgfqpoint{3.008993in}{2.171100in}}%
\pgfpathcurveto{\pgfqpoint{3.008993in}{2.179337in}}{\pgfqpoint{3.005720in}{2.187237in}}{\pgfqpoint{2.999896in}{2.193061in}}%
\pgfpathcurveto{\pgfqpoint{2.994072in}{2.198885in}}{\pgfqpoint{2.986172in}{2.202157in}}{\pgfqpoint{2.977936in}{2.202157in}}%
\pgfpathcurveto{\pgfqpoint{2.969700in}{2.202157in}}{\pgfqpoint{2.961800in}{2.198885in}}{\pgfqpoint{2.955976in}{2.193061in}}%
\pgfpathcurveto{\pgfqpoint{2.950152in}{2.187237in}}{\pgfqpoint{2.946880in}{2.179337in}}{\pgfqpoint{2.946880in}{2.171100in}}%
\pgfpathcurveto{\pgfqpoint{2.946880in}{2.162864in}}{\pgfqpoint{2.950152in}{2.154964in}}{\pgfqpoint{2.955976in}{2.149140in}}%
\pgfpathcurveto{\pgfqpoint{2.961800in}{2.143316in}}{\pgfqpoint{2.969700in}{2.140044in}}{\pgfqpoint{2.977936in}{2.140044in}}%
\pgfpathclose%
\pgfusepath{stroke,fill}%
\end{pgfscope}%
\begin{pgfscope}%
\pgfpathrectangle{\pgfqpoint{0.100000in}{0.212622in}}{\pgfqpoint{3.696000in}{3.696000in}}%
\pgfusepath{clip}%
\pgfsetbuttcap%
\pgfsetroundjoin%
\definecolor{currentfill}{rgb}{0.121569,0.466667,0.705882}%
\pgfsetfillcolor{currentfill}%
\pgfsetfillopacity{0.720119}%
\pgfsetlinewidth{1.003750pt}%
\definecolor{currentstroke}{rgb}{0.121569,0.466667,0.705882}%
\pgfsetstrokecolor{currentstroke}%
\pgfsetstrokeopacity{0.720119}%
\pgfsetdash{}{0pt}%
\pgfpathmoveto{\pgfqpoint{2.977646in}{2.139877in}}%
\pgfpathcurveto{\pgfqpoint{2.985882in}{2.139877in}}{\pgfqpoint{2.993782in}{2.143149in}}{\pgfqpoint{2.999606in}{2.148973in}}%
\pgfpathcurveto{\pgfqpoint{3.005430in}{2.154797in}}{\pgfqpoint{3.008702in}{2.162697in}}{\pgfqpoint{3.008702in}{2.170933in}}%
\pgfpathcurveto{\pgfqpoint{3.008702in}{2.179169in}}{\pgfqpoint{3.005430in}{2.187070in}}{\pgfqpoint{2.999606in}{2.192893in}}%
\pgfpathcurveto{\pgfqpoint{2.993782in}{2.198717in}}{\pgfqpoint{2.985882in}{2.201990in}}{\pgfqpoint{2.977646in}{2.201990in}}%
\pgfpathcurveto{\pgfqpoint{2.969409in}{2.201990in}}{\pgfqpoint{2.961509in}{2.198717in}}{\pgfqpoint{2.955685in}{2.192893in}}%
\pgfpathcurveto{\pgfqpoint{2.949862in}{2.187070in}}{\pgfqpoint{2.946589in}{2.179169in}}{\pgfqpoint{2.946589in}{2.170933in}}%
\pgfpathcurveto{\pgfqpoint{2.946589in}{2.162697in}}{\pgfqpoint{2.949862in}{2.154797in}}{\pgfqpoint{2.955685in}{2.148973in}}%
\pgfpathcurveto{\pgfqpoint{2.961509in}{2.143149in}}{\pgfqpoint{2.969409in}{2.139877in}}{\pgfqpoint{2.977646in}{2.139877in}}%
\pgfpathclose%
\pgfusepath{stroke,fill}%
\end{pgfscope}%
\begin{pgfscope}%
\pgfpathrectangle{\pgfqpoint{0.100000in}{0.212622in}}{\pgfqpoint{3.696000in}{3.696000in}}%
\pgfusepath{clip}%
\pgfsetbuttcap%
\pgfsetroundjoin%
\definecolor{currentfill}{rgb}{0.121569,0.466667,0.705882}%
\pgfsetfillcolor{currentfill}%
\pgfsetfillopacity{0.720620}%
\pgfsetlinewidth{1.003750pt}%
\definecolor{currentstroke}{rgb}{0.121569,0.466667,0.705882}%
\pgfsetstrokecolor{currentstroke}%
\pgfsetstrokeopacity{0.720620}%
\pgfsetdash{}{0pt}%
\pgfpathmoveto{\pgfqpoint{2.976745in}{2.139298in}}%
\pgfpathcurveto{\pgfqpoint{2.984981in}{2.139298in}}{\pgfqpoint{2.992881in}{2.142570in}}{\pgfqpoint{2.998705in}{2.148394in}}%
\pgfpathcurveto{\pgfqpoint{3.004529in}{2.154218in}}{\pgfqpoint{3.007801in}{2.162118in}}{\pgfqpoint{3.007801in}{2.170354in}}%
\pgfpathcurveto{\pgfqpoint{3.007801in}{2.178591in}}{\pgfqpoint{3.004529in}{2.186491in}}{\pgfqpoint{2.998705in}{2.192315in}}%
\pgfpathcurveto{\pgfqpoint{2.992881in}{2.198139in}}{\pgfqpoint{2.984981in}{2.201411in}}{\pgfqpoint{2.976745in}{2.201411in}}%
\pgfpathcurveto{\pgfqpoint{2.968509in}{2.201411in}}{\pgfqpoint{2.960608in}{2.198139in}}{\pgfqpoint{2.954785in}{2.192315in}}%
\pgfpathcurveto{\pgfqpoint{2.948961in}{2.186491in}}{\pgfqpoint{2.945688in}{2.178591in}}{\pgfqpoint{2.945688in}{2.170354in}}%
\pgfpathcurveto{\pgfqpoint{2.945688in}{2.162118in}}{\pgfqpoint{2.948961in}{2.154218in}}{\pgfqpoint{2.954785in}{2.148394in}}%
\pgfpathcurveto{\pgfqpoint{2.960608in}{2.142570in}}{\pgfqpoint{2.968509in}{2.139298in}}{\pgfqpoint{2.976745in}{2.139298in}}%
\pgfpathclose%
\pgfusepath{stroke,fill}%
\end{pgfscope}%
\begin{pgfscope}%
\pgfpathrectangle{\pgfqpoint{0.100000in}{0.212622in}}{\pgfqpoint{3.696000in}{3.696000in}}%
\pgfusepath{clip}%
\pgfsetbuttcap%
\pgfsetroundjoin%
\definecolor{currentfill}{rgb}{0.121569,0.466667,0.705882}%
\pgfsetfillcolor{currentfill}%
\pgfsetfillopacity{0.721475}%
\pgfsetlinewidth{1.003750pt}%
\definecolor{currentstroke}{rgb}{0.121569,0.466667,0.705882}%
\pgfsetstrokecolor{currentstroke}%
\pgfsetstrokeopacity{0.721475}%
\pgfsetdash{}{0pt}%
\pgfpathmoveto{\pgfqpoint{2.975164in}{2.138475in}}%
\pgfpathcurveto{\pgfqpoint{2.983401in}{2.138475in}}{\pgfqpoint{2.991301in}{2.141747in}}{\pgfqpoint{2.997125in}{2.147571in}}%
\pgfpathcurveto{\pgfqpoint{3.002949in}{2.153395in}}{\pgfqpoint{3.006221in}{2.161295in}}{\pgfqpoint{3.006221in}{2.169532in}}%
\pgfpathcurveto{\pgfqpoint{3.006221in}{2.177768in}}{\pgfqpoint{3.002949in}{2.185668in}}{\pgfqpoint{2.997125in}{2.191492in}}%
\pgfpathcurveto{\pgfqpoint{2.991301in}{2.197316in}}{\pgfqpoint{2.983401in}{2.200588in}}{\pgfqpoint{2.975164in}{2.200588in}}%
\pgfpathcurveto{\pgfqpoint{2.966928in}{2.200588in}}{\pgfqpoint{2.959028in}{2.197316in}}{\pgfqpoint{2.953204in}{2.191492in}}%
\pgfpathcurveto{\pgfqpoint{2.947380in}{2.185668in}}{\pgfqpoint{2.944108in}{2.177768in}}{\pgfqpoint{2.944108in}{2.169532in}}%
\pgfpathcurveto{\pgfqpoint{2.944108in}{2.161295in}}{\pgfqpoint{2.947380in}{2.153395in}}{\pgfqpoint{2.953204in}{2.147571in}}%
\pgfpathcurveto{\pgfqpoint{2.959028in}{2.141747in}}{\pgfqpoint{2.966928in}{2.138475in}}{\pgfqpoint{2.975164in}{2.138475in}}%
\pgfpathclose%
\pgfusepath{stroke,fill}%
\end{pgfscope}%
\begin{pgfscope}%
\pgfpathrectangle{\pgfqpoint{0.100000in}{0.212622in}}{\pgfqpoint{3.696000in}{3.696000in}}%
\pgfusepath{clip}%
\pgfsetbuttcap%
\pgfsetroundjoin%
\definecolor{currentfill}{rgb}{0.121569,0.466667,0.705882}%
\pgfsetfillcolor{currentfill}%
\pgfsetfillopacity{0.722596}%
\pgfsetlinewidth{1.003750pt}%
\definecolor{currentstroke}{rgb}{0.121569,0.466667,0.705882}%
\pgfsetstrokecolor{currentstroke}%
\pgfsetstrokeopacity{0.722596}%
\pgfsetdash{}{0pt}%
\pgfpathmoveto{\pgfqpoint{2.973328in}{2.137610in}}%
\pgfpathcurveto{\pgfqpoint{2.981565in}{2.137610in}}{\pgfqpoint{2.989465in}{2.140882in}}{\pgfqpoint{2.995289in}{2.146706in}}%
\pgfpathcurveto{\pgfqpoint{3.001113in}{2.152530in}}{\pgfqpoint{3.004385in}{2.160430in}}{\pgfqpoint{3.004385in}{2.168667in}}%
\pgfpathcurveto{\pgfqpoint{3.004385in}{2.176903in}}{\pgfqpoint{3.001113in}{2.184803in}}{\pgfqpoint{2.995289in}{2.190627in}}%
\pgfpathcurveto{\pgfqpoint{2.989465in}{2.196451in}}{\pgfqpoint{2.981565in}{2.199723in}}{\pgfqpoint{2.973328in}{2.199723in}}%
\pgfpathcurveto{\pgfqpoint{2.965092in}{2.199723in}}{\pgfqpoint{2.957192in}{2.196451in}}{\pgfqpoint{2.951368in}{2.190627in}}%
\pgfpathcurveto{\pgfqpoint{2.945544in}{2.184803in}}{\pgfqpoint{2.942272in}{2.176903in}}{\pgfqpoint{2.942272in}{2.168667in}}%
\pgfpathcurveto{\pgfqpoint{2.942272in}{2.160430in}}{\pgfqpoint{2.945544in}{2.152530in}}{\pgfqpoint{2.951368in}{2.146706in}}%
\pgfpathcurveto{\pgfqpoint{2.957192in}{2.140882in}}{\pgfqpoint{2.965092in}{2.137610in}}{\pgfqpoint{2.973328in}{2.137610in}}%
\pgfpathclose%
\pgfusepath{stroke,fill}%
\end{pgfscope}%
\begin{pgfscope}%
\pgfpathrectangle{\pgfqpoint{0.100000in}{0.212622in}}{\pgfqpoint{3.696000in}{3.696000in}}%
\pgfusepath{clip}%
\pgfsetbuttcap%
\pgfsetroundjoin%
\definecolor{currentfill}{rgb}{0.121569,0.466667,0.705882}%
\pgfsetfillcolor{currentfill}%
\pgfsetfillopacity{0.723816}%
\pgfsetlinewidth{1.003750pt}%
\definecolor{currentstroke}{rgb}{0.121569,0.466667,0.705882}%
\pgfsetstrokecolor{currentstroke}%
\pgfsetstrokeopacity{0.723816}%
\pgfsetdash{}{0pt}%
\pgfpathmoveto{\pgfqpoint{2.971198in}{2.136224in}}%
\pgfpathcurveto{\pgfqpoint{2.979435in}{2.136224in}}{\pgfqpoint{2.987335in}{2.139496in}}{\pgfqpoint{2.993159in}{2.145320in}}%
\pgfpathcurveto{\pgfqpoint{2.998983in}{2.151144in}}{\pgfqpoint{3.002255in}{2.159044in}}{\pgfqpoint{3.002255in}{2.167280in}}%
\pgfpathcurveto{\pgfqpoint{3.002255in}{2.175516in}}{\pgfqpoint{2.998983in}{2.183417in}}{\pgfqpoint{2.993159in}{2.189240in}}%
\pgfpathcurveto{\pgfqpoint{2.987335in}{2.195064in}}{\pgfqpoint{2.979435in}{2.198337in}}{\pgfqpoint{2.971198in}{2.198337in}}%
\pgfpathcurveto{\pgfqpoint{2.962962in}{2.198337in}}{\pgfqpoint{2.955062in}{2.195064in}}{\pgfqpoint{2.949238in}{2.189240in}}%
\pgfpathcurveto{\pgfqpoint{2.943414in}{2.183417in}}{\pgfqpoint{2.940142in}{2.175516in}}{\pgfqpoint{2.940142in}{2.167280in}}%
\pgfpathcurveto{\pgfqpoint{2.940142in}{2.159044in}}{\pgfqpoint{2.943414in}{2.151144in}}{\pgfqpoint{2.949238in}{2.145320in}}%
\pgfpathcurveto{\pgfqpoint{2.955062in}{2.139496in}}{\pgfqpoint{2.962962in}{2.136224in}}{\pgfqpoint{2.971198in}{2.136224in}}%
\pgfpathclose%
\pgfusepath{stroke,fill}%
\end{pgfscope}%
\begin{pgfscope}%
\pgfpathrectangle{\pgfqpoint{0.100000in}{0.212622in}}{\pgfqpoint{3.696000in}{3.696000in}}%
\pgfusepath{clip}%
\pgfsetbuttcap%
\pgfsetroundjoin%
\definecolor{currentfill}{rgb}{0.121569,0.466667,0.705882}%
\pgfsetfillcolor{currentfill}%
\pgfsetfillopacity{0.725577}%
\pgfsetlinewidth{1.003750pt}%
\definecolor{currentstroke}{rgb}{0.121569,0.466667,0.705882}%
\pgfsetstrokecolor{currentstroke}%
\pgfsetstrokeopacity{0.725577}%
\pgfsetdash{}{0pt}%
\pgfpathmoveto{\pgfqpoint{2.968389in}{2.134250in}}%
\pgfpathcurveto{\pgfqpoint{2.976626in}{2.134250in}}{\pgfqpoint{2.984526in}{2.137523in}}{\pgfqpoint{2.990350in}{2.143347in}}%
\pgfpathcurveto{\pgfqpoint{2.996174in}{2.149171in}}{\pgfqpoint{2.999446in}{2.157071in}}{\pgfqpoint{2.999446in}{2.165307in}}%
\pgfpathcurveto{\pgfqpoint{2.999446in}{2.173543in}}{\pgfqpoint{2.996174in}{2.181443in}}{\pgfqpoint{2.990350in}{2.187267in}}%
\pgfpathcurveto{\pgfqpoint{2.984526in}{2.193091in}}{\pgfqpoint{2.976626in}{2.196363in}}{\pgfqpoint{2.968389in}{2.196363in}}%
\pgfpathcurveto{\pgfqpoint{2.960153in}{2.196363in}}{\pgfqpoint{2.952253in}{2.193091in}}{\pgfqpoint{2.946429in}{2.187267in}}%
\pgfpathcurveto{\pgfqpoint{2.940605in}{2.181443in}}{\pgfqpoint{2.937333in}{2.173543in}}{\pgfqpoint{2.937333in}{2.165307in}}%
\pgfpathcurveto{\pgfqpoint{2.937333in}{2.157071in}}{\pgfqpoint{2.940605in}{2.149171in}}{\pgfqpoint{2.946429in}{2.143347in}}%
\pgfpathcurveto{\pgfqpoint{2.952253in}{2.137523in}}{\pgfqpoint{2.960153in}{2.134250in}}{\pgfqpoint{2.968389in}{2.134250in}}%
\pgfpathclose%
\pgfusepath{stroke,fill}%
\end{pgfscope}%
\begin{pgfscope}%
\pgfpathrectangle{\pgfqpoint{0.100000in}{0.212622in}}{\pgfqpoint{3.696000in}{3.696000in}}%
\pgfusepath{clip}%
\pgfsetbuttcap%
\pgfsetroundjoin%
\definecolor{currentfill}{rgb}{0.121569,0.466667,0.705882}%
\pgfsetfillcolor{currentfill}%
\pgfsetfillopacity{0.726591}%
\pgfsetlinewidth{1.003750pt}%
\definecolor{currentstroke}{rgb}{0.121569,0.466667,0.705882}%
\pgfsetstrokecolor{currentstroke}%
\pgfsetstrokeopacity{0.726591}%
\pgfsetdash{}{0pt}%
\pgfpathmoveto{\pgfqpoint{2.966905in}{2.133407in}}%
\pgfpathcurveto{\pgfqpoint{2.975141in}{2.133407in}}{\pgfqpoint{2.983041in}{2.136679in}}{\pgfqpoint{2.988865in}{2.142503in}}%
\pgfpathcurveto{\pgfqpoint{2.994689in}{2.148327in}}{\pgfqpoint{2.997961in}{2.156227in}}{\pgfqpoint{2.997961in}{2.164463in}}%
\pgfpathcurveto{\pgfqpoint{2.997961in}{2.172700in}}{\pgfqpoint{2.994689in}{2.180600in}}{\pgfqpoint{2.988865in}{2.186424in}}%
\pgfpathcurveto{\pgfqpoint{2.983041in}{2.192247in}}{\pgfqpoint{2.975141in}{2.195520in}}{\pgfqpoint{2.966905in}{2.195520in}}%
\pgfpathcurveto{\pgfqpoint{2.958668in}{2.195520in}}{\pgfqpoint{2.950768in}{2.192247in}}{\pgfqpoint{2.944944in}{2.186424in}}%
\pgfpathcurveto{\pgfqpoint{2.939121in}{2.180600in}}{\pgfqpoint{2.935848in}{2.172700in}}{\pgfqpoint{2.935848in}{2.164463in}}%
\pgfpathcurveto{\pgfqpoint{2.935848in}{2.156227in}}{\pgfqpoint{2.939121in}{2.148327in}}{\pgfqpoint{2.944944in}{2.142503in}}%
\pgfpathcurveto{\pgfqpoint{2.950768in}{2.136679in}}{\pgfqpoint{2.958668in}{2.133407in}}{\pgfqpoint{2.966905in}{2.133407in}}%
\pgfpathclose%
\pgfusepath{stroke,fill}%
\end{pgfscope}%
\begin{pgfscope}%
\pgfpathrectangle{\pgfqpoint{0.100000in}{0.212622in}}{\pgfqpoint{3.696000in}{3.696000in}}%
\pgfusepath{clip}%
\pgfsetbuttcap%
\pgfsetroundjoin%
\definecolor{currentfill}{rgb}{0.121569,0.466667,0.705882}%
\pgfsetfillcolor{currentfill}%
\pgfsetfillopacity{0.727134}%
\pgfsetlinewidth{1.003750pt}%
\definecolor{currentstroke}{rgb}{0.121569,0.466667,0.705882}%
\pgfsetstrokecolor{currentstroke}%
\pgfsetstrokeopacity{0.727134}%
\pgfsetdash{}{0pt}%
\pgfpathmoveto{\pgfqpoint{2.966029in}{2.132897in}}%
\pgfpathcurveto{\pgfqpoint{2.974265in}{2.132897in}}{\pgfqpoint{2.982165in}{2.136169in}}{\pgfqpoint{2.987989in}{2.141993in}}%
\pgfpathcurveto{\pgfqpoint{2.993813in}{2.147817in}}{\pgfqpoint{2.997085in}{2.155717in}}{\pgfqpoint{2.997085in}{2.163953in}}%
\pgfpathcurveto{\pgfqpoint{2.997085in}{2.172189in}}{\pgfqpoint{2.993813in}{2.180089in}}{\pgfqpoint{2.987989in}{2.185913in}}%
\pgfpathcurveto{\pgfqpoint{2.982165in}{2.191737in}}{\pgfqpoint{2.974265in}{2.195010in}}{\pgfqpoint{2.966029in}{2.195010in}}%
\pgfpathcurveto{\pgfqpoint{2.957793in}{2.195010in}}{\pgfqpoint{2.949893in}{2.191737in}}{\pgfqpoint{2.944069in}{2.185913in}}%
\pgfpathcurveto{\pgfqpoint{2.938245in}{2.180089in}}{\pgfqpoint{2.934972in}{2.172189in}}{\pgfqpoint{2.934972in}{2.163953in}}%
\pgfpathcurveto{\pgfqpoint{2.934972in}{2.155717in}}{\pgfqpoint{2.938245in}{2.147817in}}{\pgfqpoint{2.944069in}{2.141993in}}%
\pgfpathcurveto{\pgfqpoint{2.949893in}{2.136169in}}{\pgfqpoint{2.957793in}{2.132897in}}{\pgfqpoint{2.966029in}{2.132897in}}%
\pgfpathclose%
\pgfusepath{stroke,fill}%
\end{pgfscope}%
\begin{pgfscope}%
\pgfpathrectangle{\pgfqpoint{0.100000in}{0.212622in}}{\pgfqpoint{3.696000in}{3.696000in}}%
\pgfusepath{clip}%
\pgfsetbuttcap%
\pgfsetroundjoin%
\definecolor{currentfill}{rgb}{0.121569,0.466667,0.705882}%
\pgfsetfillcolor{currentfill}%
\pgfsetfillopacity{0.727436}%
\pgfsetlinewidth{1.003750pt}%
\definecolor{currentstroke}{rgb}{0.121569,0.466667,0.705882}%
\pgfsetstrokecolor{currentstroke}%
\pgfsetstrokeopacity{0.727436}%
\pgfsetdash{}{0pt}%
\pgfpathmoveto{\pgfqpoint{2.965582in}{2.132609in}}%
\pgfpathcurveto{\pgfqpoint{2.973819in}{2.132609in}}{\pgfqpoint{2.981719in}{2.135881in}}{\pgfqpoint{2.987543in}{2.141705in}}%
\pgfpathcurveto{\pgfqpoint{2.993366in}{2.147529in}}{\pgfqpoint{2.996639in}{2.155429in}}{\pgfqpoint{2.996639in}{2.163665in}}%
\pgfpathcurveto{\pgfqpoint{2.996639in}{2.171902in}}{\pgfqpoint{2.993366in}{2.179802in}}{\pgfqpoint{2.987543in}{2.185626in}}%
\pgfpathcurveto{\pgfqpoint{2.981719in}{2.191450in}}{\pgfqpoint{2.973819in}{2.194722in}}{\pgfqpoint{2.965582in}{2.194722in}}%
\pgfpathcurveto{\pgfqpoint{2.957346in}{2.194722in}}{\pgfqpoint{2.949446in}{2.191450in}}{\pgfqpoint{2.943622in}{2.185626in}}%
\pgfpathcurveto{\pgfqpoint{2.937798in}{2.179802in}}{\pgfqpoint{2.934526in}{2.171902in}}{\pgfqpoint{2.934526in}{2.163665in}}%
\pgfpathcurveto{\pgfqpoint{2.934526in}{2.155429in}}{\pgfqpoint{2.937798in}{2.147529in}}{\pgfqpoint{2.943622in}{2.141705in}}%
\pgfpathcurveto{\pgfqpoint{2.949446in}{2.135881in}}{\pgfqpoint{2.957346in}{2.132609in}}{\pgfqpoint{2.965582in}{2.132609in}}%
\pgfpathclose%
\pgfusepath{stroke,fill}%
\end{pgfscope}%
\begin{pgfscope}%
\pgfpathrectangle{\pgfqpoint{0.100000in}{0.212622in}}{\pgfqpoint{3.696000in}{3.696000in}}%
\pgfusepath{clip}%
\pgfsetbuttcap%
\pgfsetroundjoin%
\definecolor{currentfill}{rgb}{0.121569,0.466667,0.705882}%
\pgfsetfillcolor{currentfill}%
\pgfsetfillopacity{0.728206}%
\pgfsetlinewidth{1.003750pt}%
\definecolor{currentstroke}{rgb}{0.121569,0.466667,0.705882}%
\pgfsetstrokecolor{currentstroke}%
\pgfsetstrokeopacity{0.728206}%
\pgfsetdash{}{0pt}%
\pgfpathmoveto{\pgfqpoint{2.964291in}{2.131884in}}%
\pgfpathcurveto{\pgfqpoint{2.972527in}{2.131884in}}{\pgfqpoint{2.980428in}{2.135157in}}{\pgfqpoint{2.986251in}{2.140981in}}%
\pgfpathcurveto{\pgfqpoint{2.992075in}{2.146804in}}{\pgfqpoint{2.995348in}{2.154705in}}{\pgfqpoint{2.995348in}{2.162941in}}%
\pgfpathcurveto{\pgfqpoint{2.995348in}{2.171177in}}{\pgfqpoint{2.992075in}{2.179077in}}{\pgfqpoint{2.986251in}{2.184901in}}%
\pgfpathcurveto{\pgfqpoint{2.980428in}{2.190725in}}{\pgfqpoint{2.972527in}{2.193997in}}{\pgfqpoint{2.964291in}{2.193997in}}%
\pgfpathcurveto{\pgfqpoint{2.956055in}{2.193997in}}{\pgfqpoint{2.948155in}{2.190725in}}{\pgfqpoint{2.942331in}{2.184901in}}%
\pgfpathcurveto{\pgfqpoint{2.936507in}{2.179077in}}{\pgfqpoint{2.933235in}{2.171177in}}{\pgfqpoint{2.933235in}{2.162941in}}%
\pgfpathcurveto{\pgfqpoint{2.933235in}{2.154705in}}{\pgfqpoint{2.936507in}{2.146804in}}{\pgfqpoint{2.942331in}{2.140981in}}%
\pgfpathcurveto{\pgfqpoint{2.948155in}{2.135157in}}{\pgfqpoint{2.956055in}{2.131884in}}{\pgfqpoint{2.964291in}{2.131884in}}%
\pgfpathclose%
\pgfusepath{stroke,fill}%
\end{pgfscope}%
\begin{pgfscope}%
\pgfpathrectangle{\pgfqpoint{0.100000in}{0.212622in}}{\pgfqpoint{3.696000in}{3.696000in}}%
\pgfusepath{clip}%
\pgfsetbuttcap%
\pgfsetroundjoin%
\definecolor{currentfill}{rgb}{0.121569,0.466667,0.705882}%
\pgfsetfillcolor{currentfill}%
\pgfsetfillopacity{0.728626}%
\pgfsetlinewidth{1.003750pt}%
\definecolor{currentstroke}{rgb}{0.121569,0.466667,0.705882}%
\pgfsetstrokecolor{currentstroke}%
\pgfsetstrokeopacity{0.728626}%
\pgfsetdash{}{0pt}%
\pgfpathmoveto{\pgfqpoint{2.963569in}{2.131477in}}%
\pgfpathcurveto{\pgfqpoint{2.971806in}{2.131477in}}{\pgfqpoint{2.979706in}{2.134749in}}{\pgfqpoint{2.985530in}{2.140573in}}%
\pgfpathcurveto{\pgfqpoint{2.991353in}{2.146397in}}{\pgfqpoint{2.994626in}{2.154297in}}{\pgfqpoint{2.994626in}{2.162533in}}%
\pgfpathcurveto{\pgfqpoint{2.994626in}{2.170769in}}{\pgfqpoint{2.991353in}{2.178669in}}{\pgfqpoint{2.985530in}{2.184493in}}%
\pgfpathcurveto{\pgfqpoint{2.979706in}{2.190317in}}{\pgfqpoint{2.971806in}{2.193590in}}{\pgfqpoint{2.963569in}{2.193590in}}%
\pgfpathcurveto{\pgfqpoint{2.955333in}{2.193590in}}{\pgfqpoint{2.947433in}{2.190317in}}{\pgfqpoint{2.941609in}{2.184493in}}%
\pgfpathcurveto{\pgfqpoint{2.935785in}{2.178669in}}{\pgfqpoint{2.932513in}{2.170769in}}{\pgfqpoint{2.932513in}{2.162533in}}%
\pgfpathcurveto{\pgfqpoint{2.932513in}{2.154297in}}{\pgfqpoint{2.935785in}{2.146397in}}{\pgfqpoint{2.941609in}{2.140573in}}%
\pgfpathcurveto{\pgfqpoint{2.947433in}{2.134749in}}{\pgfqpoint{2.955333in}{2.131477in}}{\pgfqpoint{2.963569in}{2.131477in}}%
\pgfpathclose%
\pgfusepath{stroke,fill}%
\end{pgfscope}%
\begin{pgfscope}%
\pgfpathrectangle{\pgfqpoint{0.100000in}{0.212622in}}{\pgfqpoint{3.696000in}{3.696000in}}%
\pgfusepath{clip}%
\pgfsetbuttcap%
\pgfsetroundjoin%
\definecolor{currentfill}{rgb}{0.121569,0.466667,0.705882}%
\pgfsetfillcolor{currentfill}%
\pgfsetfillopacity{0.729232}%
\pgfsetlinewidth{1.003750pt}%
\definecolor{currentstroke}{rgb}{0.121569,0.466667,0.705882}%
\pgfsetstrokecolor{currentstroke}%
\pgfsetstrokeopacity{0.729232}%
\pgfsetdash{}{0pt}%
\pgfpathmoveto{\pgfqpoint{2.962622in}{2.130921in}}%
\pgfpathcurveto{\pgfqpoint{2.970858in}{2.130921in}}{\pgfqpoint{2.978759in}{2.134193in}}{\pgfqpoint{2.984582in}{2.140017in}}%
\pgfpathcurveto{\pgfqpoint{2.990406in}{2.145841in}}{\pgfqpoint{2.993679in}{2.153741in}}{\pgfqpoint{2.993679in}{2.161977in}}%
\pgfpathcurveto{\pgfqpoint{2.993679in}{2.170214in}}{\pgfqpoint{2.990406in}{2.178114in}}{\pgfqpoint{2.984582in}{2.183938in}}%
\pgfpathcurveto{\pgfqpoint{2.978759in}{2.189762in}}{\pgfqpoint{2.970858in}{2.193034in}}{\pgfqpoint{2.962622in}{2.193034in}}%
\pgfpathcurveto{\pgfqpoint{2.954386in}{2.193034in}}{\pgfqpoint{2.946486in}{2.189762in}}{\pgfqpoint{2.940662in}{2.183938in}}%
\pgfpathcurveto{\pgfqpoint{2.934838in}{2.178114in}}{\pgfqpoint{2.931566in}{2.170214in}}{\pgfqpoint{2.931566in}{2.161977in}}%
\pgfpathcurveto{\pgfqpoint{2.931566in}{2.153741in}}{\pgfqpoint{2.934838in}{2.145841in}}{\pgfqpoint{2.940662in}{2.140017in}}%
\pgfpathcurveto{\pgfqpoint{2.946486in}{2.134193in}}{\pgfqpoint{2.954386in}{2.130921in}}{\pgfqpoint{2.962622in}{2.130921in}}%
\pgfpathclose%
\pgfusepath{stroke,fill}%
\end{pgfscope}%
\begin{pgfscope}%
\pgfpathrectangle{\pgfqpoint{0.100000in}{0.212622in}}{\pgfqpoint{3.696000in}{3.696000in}}%
\pgfusepath{clip}%
\pgfsetbuttcap%
\pgfsetroundjoin%
\definecolor{currentfill}{rgb}{0.121569,0.466667,0.705882}%
\pgfsetfillcolor{currentfill}%
\pgfsetfillopacity{0.729560}%
\pgfsetlinewidth{1.003750pt}%
\definecolor{currentstroke}{rgb}{0.121569,0.466667,0.705882}%
\pgfsetstrokecolor{currentstroke}%
\pgfsetstrokeopacity{0.729560}%
\pgfsetdash{}{0pt}%
\pgfpathmoveto{\pgfqpoint{2.962053in}{2.130632in}}%
\pgfpathcurveto{\pgfqpoint{2.970289in}{2.130632in}}{\pgfqpoint{2.978189in}{2.133904in}}{\pgfqpoint{2.984013in}{2.139728in}}%
\pgfpathcurveto{\pgfqpoint{2.989837in}{2.145552in}}{\pgfqpoint{2.993109in}{2.153452in}}{\pgfqpoint{2.993109in}{2.161688in}}%
\pgfpathcurveto{\pgfqpoint{2.993109in}{2.169925in}}{\pgfqpoint{2.989837in}{2.177825in}}{\pgfqpoint{2.984013in}{2.183649in}}%
\pgfpathcurveto{\pgfqpoint{2.978189in}{2.189473in}}{\pgfqpoint{2.970289in}{2.192745in}}{\pgfqpoint{2.962053in}{2.192745in}}%
\pgfpathcurveto{\pgfqpoint{2.953817in}{2.192745in}}{\pgfqpoint{2.945917in}{2.189473in}}{\pgfqpoint{2.940093in}{2.183649in}}%
\pgfpathcurveto{\pgfqpoint{2.934269in}{2.177825in}}{\pgfqpoint{2.930996in}{2.169925in}}{\pgfqpoint{2.930996in}{2.161688in}}%
\pgfpathcurveto{\pgfqpoint{2.930996in}{2.153452in}}{\pgfqpoint{2.934269in}{2.145552in}}{\pgfqpoint{2.940093in}{2.139728in}}%
\pgfpathcurveto{\pgfqpoint{2.945917in}{2.133904in}}{\pgfqpoint{2.953817in}{2.130632in}}{\pgfqpoint{2.962053in}{2.130632in}}%
\pgfpathclose%
\pgfusepath{stroke,fill}%
\end{pgfscope}%
\begin{pgfscope}%
\pgfpathrectangle{\pgfqpoint{0.100000in}{0.212622in}}{\pgfqpoint{3.696000in}{3.696000in}}%
\pgfusepath{clip}%
\pgfsetbuttcap%
\pgfsetroundjoin%
\definecolor{currentfill}{rgb}{0.121569,0.466667,0.705882}%
\pgfsetfillcolor{currentfill}%
\pgfsetfillopacity{0.730249}%
\pgfsetlinewidth{1.003750pt}%
\definecolor{currentstroke}{rgb}{0.121569,0.466667,0.705882}%
\pgfsetstrokecolor{currentstroke}%
\pgfsetstrokeopacity{0.730249}%
\pgfsetdash{}{0pt}%
\pgfpathmoveto{\pgfqpoint{2.960972in}{2.129948in}}%
\pgfpathcurveto{\pgfqpoint{2.969209in}{2.129948in}}{\pgfqpoint{2.977109in}{2.133220in}}{\pgfqpoint{2.982933in}{2.139044in}}%
\pgfpathcurveto{\pgfqpoint{2.988756in}{2.144868in}}{\pgfqpoint{2.992029in}{2.152768in}}{\pgfqpoint{2.992029in}{2.161004in}}%
\pgfpathcurveto{\pgfqpoint{2.992029in}{2.169240in}}{\pgfqpoint{2.988756in}{2.177140in}}{\pgfqpoint{2.982933in}{2.182964in}}%
\pgfpathcurveto{\pgfqpoint{2.977109in}{2.188788in}}{\pgfqpoint{2.969209in}{2.192061in}}{\pgfqpoint{2.960972in}{2.192061in}}%
\pgfpathcurveto{\pgfqpoint{2.952736in}{2.192061in}}{\pgfqpoint{2.944836in}{2.188788in}}{\pgfqpoint{2.939012in}{2.182964in}}%
\pgfpathcurveto{\pgfqpoint{2.933188in}{2.177140in}}{\pgfqpoint{2.929916in}{2.169240in}}{\pgfqpoint{2.929916in}{2.161004in}}%
\pgfpathcurveto{\pgfqpoint{2.929916in}{2.152768in}}{\pgfqpoint{2.933188in}{2.144868in}}{\pgfqpoint{2.939012in}{2.139044in}}%
\pgfpathcurveto{\pgfqpoint{2.944836in}{2.133220in}}{\pgfqpoint{2.952736in}{2.129948in}}{\pgfqpoint{2.960972in}{2.129948in}}%
\pgfpathclose%
\pgfusepath{stroke,fill}%
\end{pgfscope}%
\begin{pgfscope}%
\pgfpathrectangle{\pgfqpoint{0.100000in}{0.212622in}}{\pgfqpoint{3.696000in}{3.696000in}}%
\pgfusepath{clip}%
\pgfsetbuttcap%
\pgfsetroundjoin%
\definecolor{currentfill}{rgb}{0.121569,0.466667,0.705882}%
\pgfsetfillcolor{currentfill}%
\pgfsetfillopacity{0.731162}%
\pgfsetlinewidth{1.003750pt}%
\definecolor{currentstroke}{rgb}{0.121569,0.466667,0.705882}%
\pgfsetstrokecolor{currentstroke}%
\pgfsetstrokeopacity{0.731162}%
\pgfsetdash{}{0pt}%
\pgfpathmoveto{\pgfqpoint{2.959675in}{2.129350in}}%
\pgfpathcurveto{\pgfqpoint{2.967911in}{2.129350in}}{\pgfqpoint{2.975811in}{2.132622in}}{\pgfqpoint{2.981635in}{2.138446in}}%
\pgfpathcurveto{\pgfqpoint{2.987459in}{2.144270in}}{\pgfqpoint{2.990731in}{2.152170in}}{\pgfqpoint{2.990731in}{2.160406in}}%
\pgfpathcurveto{\pgfqpoint{2.990731in}{2.168643in}}{\pgfqpoint{2.987459in}{2.176543in}}{\pgfqpoint{2.981635in}{2.182367in}}%
\pgfpathcurveto{\pgfqpoint{2.975811in}{2.188190in}}{\pgfqpoint{2.967911in}{2.191463in}}{\pgfqpoint{2.959675in}{2.191463in}}%
\pgfpathcurveto{\pgfqpoint{2.951438in}{2.191463in}}{\pgfqpoint{2.943538in}{2.188190in}}{\pgfqpoint{2.937714in}{2.182367in}}%
\pgfpathcurveto{\pgfqpoint{2.931890in}{2.176543in}}{\pgfqpoint{2.928618in}{2.168643in}}{\pgfqpoint{2.928618in}{2.160406in}}%
\pgfpathcurveto{\pgfqpoint{2.928618in}{2.152170in}}{\pgfqpoint{2.931890in}{2.144270in}}{\pgfqpoint{2.937714in}{2.138446in}}%
\pgfpathcurveto{\pgfqpoint{2.943538in}{2.132622in}}{\pgfqpoint{2.951438in}{2.129350in}}{\pgfqpoint{2.959675in}{2.129350in}}%
\pgfpathclose%
\pgfusepath{stroke,fill}%
\end{pgfscope}%
\begin{pgfscope}%
\pgfpathrectangle{\pgfqpoint{0.100000in}{0.212622in}}{\pgfqpoint{3.696000in}{3.696000in}}%
\pgfusepath{clip}%
\pgfsetbuttcap%
\pgfsetroundjoin%
\definecolor{currentfill}{rgb}{0.121569,0.466667,0.705882}%
\pgfsetfillcolor{currentfill}%
\pgfsetfillopacity{0.732172}%
\pgfsetlinewidth{1.003750pt}%
\definecolor{currentstroke}{rgb}{0.121569,0.466667,0.705882}%
\pgfsetstrokecolor{currentstroke}%
\pgfsetstrokeopacity{0.732172}%
\pgfsetdash{}{0pt}%
\pgfpathmoveto{\pgfqpoint{2.958045in}{2.128470in}}%
\pgfpathcurveto{\pgfqpoint{2.966282in}{2.128470in}}{\pgfqpoint{2.974182in}{2.131742in}}{\pgfqpoint{2.980006in}{2.137566in}}%
\pgfpathcurveto{\pgfqpoint{2.985829in}{2.143390in}}{\pgfqpoint{2.989102in}{2.151290in}}{\pgfqpoint{2.989102in}{2.159526in}}%
\pgfpathcurveto{\pgfqpoint{2.989102in}{2.167763in}}{\pgfqpoint{2.985829in}{2.175663in}}{\pgfqpoint{2.980006in}{2.181487in}}%
\pgfpathcurveto{\pgfqpoint{2.974182in}{2.187311in}}{\pgfqpoint{2.966282in}{2.190583in}}{\pgfqpoint{2.958045in}{2.190583in}}%
\pgfpathcurveto{\pgfqpoint{2.949809in}{2.190583in}}{\pgfqpoint{2.941909in}{2.187311in}}{\pgfqpoint{2.936085in}{2.181487in}}%
\pgfpathcurveto{\pgfqpoint{2.930261in}{2.175663in}}{\pgfqpoint{2.926989in}{2.167763in}}{\pgfqpoint{2.926989in}{2.159526in}}%
\pgfpathcurveto{\pgfqpoint{2.926989in}{2.151290in}}{\pgfqpoint{2.930261in}{2.143390in}}{\pgfqpoint{2.936085in}{2.137566in}}%
\pgfpathcurveto{\pgfqpoint{2.941909in}{2.131742in}}{\pgfqpoint{2.949809in}{2.128470in}}{\pgfqpoint{2.958045in}{2.128470in}}%
\pgfpathclose%
\pgfusepath{stroke,fill}%
\end{pgfscope}%
\begin{pgfscope}%
\pgfpathrectangle{\pgfqpoint{0.100000in}{0.212622in}}{\pgfqpoint{3.696000in}{3.696000in}}%
\pgfusepath{clip}%
\pgfsetbuttcap%
\pgfsetroundjoin%
\definecolor{currentfill}{rgb}{0.121569,0.466667,0.705882}%
\pgfsetfillcolor{currentfill}%
\pgfsetfillopacity{0.733432}%
\pgfsetlinewidth{1.003750pt}%
\definecolor{currentstroke}{rgb}{0.121569,0.466667,0.705882}%
\pgfsetstrokecolor{currentstroke}%
\pgfsetstrokeopacity{0.733432}%
\pgfsetdash{}{0pt}%
\pgfpathmoveto{\pgfqpoint{2.956193in}{2.127406in}}%
\pgfpathcurveto{\pgfqpoint{2.964429in}{2.127406in}}{\pgfqpoint{2.972329in}{2.130678in}}{\pgfqpoint{2.978153in}{2.136502in}}%
\pgfpathcurveto{\pgfqpoint{2.983977in}{2.142326in}}{\pgfqpoint{2.987250in}{2.150226in}}{\pgfqpoint{2.987250in}{2.158462in}}%
\pgfpathcurveto{\pgfqpoint{2.987250in}{2.166699in}}{\pgfqpoint{2.983977in}{2.174599in}}{\pgfqpoint{2.978153in}{2.180423in}}%
\pgfpathcurveto{\pgfqpoint{2.972329in}{2.186246in}}{\pgfqpoint{2.964429in}{2.189519in}}{\pgfqpoint{2.956193in}{2.189519in}}%
\pgfpathcurveto{\pgfqpoint{2.947957in}{2.189519in}}{\pgfqpoint{2.940057in}{2.186246in}}{\pgfqpoint{2.934233in}{2.180423in}}%
\pgfpathcurveto{\pgfqpoint{2.928409in}{2.174599in}}{\pgfqpoint{2.925137in}{2.166699in}}{\pgfqpoint{2.925137in}{2.158462in}}%
\pgfpathcurveto{\pgfqpoint{2.925137in}{2.150226in}}{\pgfqpoint{2.928409in}{2.142326in}}{\pgfqpoint{2.934233in}{2.136502in}}%
\pgfpathcurveto{\pgfqpoint{2.940057in}{2.130678in}}{\pgfqpoint{2.947957in}{2.127406in}}{\pgfqpoint{2.956193in}{2.127406in}}%
\pgfpathclose%
\pgfusepath{stroke,fill}%
\end{pgfscope}%
\begin{pgfscope}%
\pgfpathrectangle{\pgfqpoint{0.100000in}{0.212622in}}{\pgfqpoint{3.696000in}{3.696000in}}%
\pgfusepath{clip}%
\pgfsetbuttcap%
\pgfsetroundjoin%
\definecolor{currentfill}{rgb}{0.121569,0.466667,0.705882}%
\pgfsetfillcolor{currentfill}%
\pgfsetfillopacity{0.734881}%
\pgfsetlinewidth{1.003750pt}%
\definecolor{currentstroke}{rgb}{0.121569,0.466667,0.705882}%
\pgfsetstrokecolor{currentstroke}%
\pgfsetstrokeopacity{0.734881}%
\pgfsetdash{}{0pt}%
\pgfpathmoveto{\pgfqpoint{2.953673in}{2.126041in}}%
\pgfpathcurveto{\pgfqpoint{2.961909in}{2.126041in}}{\pgfqpoint{2.969809in}{2.129313in}}{\pgfqpoint{2.975633in}{2.135137in}}%
\pgfpathcurveto{\pgfqpoint{2.981457in}{2.140961in}}{\pgfqpoint{2.984729in}{2.148861in}}{\pgfqpoint{2.984729in}{2.157097in}}%
\pgfpathcurveto{\pgfqpoint{2.984729in}{2.165333in}}{\pgfqpoint{2.981457in}{2.173233in}}{\pgfqpoint{2.975633in}{2.179057in}}%
\pgfpathcurveto{\pgfqpoint{2.969809in}{2.184881in}}{\pgfqpoint{2.961909in}{2.188154in}}{\pgfqpoint{2.953673in}{2.188154in}}%
\pgfpathcurveto{\pgfqpoint{2.945436in}{2.188154in}}{\pgfqpoint{2.937536in}{2.184881in}}{\pgfqpoint{2.931712in}{2.179057in}}%
\pgfpathcurveto{\pgfqpoint{2.925888in}{2.173233in}}{\pgfqpoint{2.922616in}{2.165333in}}{\pgfqpoint{2.922616in}{2.157097in}}%
\pgfpathcurveto{\pgfqpoint{2.922616in}{2.148861in}}{\pgfqpoint{2.925888in}{2.140961in}}{\pgfqpoint{2.931712in}{2.135137in}}%
\pgfpathcurveto{\pgfqpoint{2.937536in}{2.129313in}}{\pgfqpoint{2.945436in}{2.126041in}}{\pgfqpoint{2.953673in}{2.126041in}}%
\pgfpathclose%
\pgfusepath{stroke,fill}%
\end{pgfscope}%
\begin{pgfscope}%
\pgfpathrectangle{\pgfqpoint{0.100000in}{0.212622in}}{\pgfqpoint{3.696000in}{3.696000in}}%
\pgfusepath{clip}%
\pgfsetbuttcap%
\pgfsetroundjoin%
\definecolor{currentfill}{rgb}{0.121569,0.466667,0.705882}%
\pgfsetfillcolor{currentfill}%
\pgfsetfillopacity{0.736732}%
\pgfsetlinewidth{1.003750pt}%
\definecolor{currentstroke}{rgb}{0.121569,0.466667,0.705882}%
\pgfsetstrokecolor{currentstroke}%
\pgfsetstrokeopacity{0.736732}%
\pgfsetdash{}{0pt}%
\pgfpathmoveto{\pgfqpoint{2.950783in}{2.124874in}}%
\pgfpathcurveto{\pgfqpoint{2.959020in}{2.124874in}}{\pgfqpoint{2.966920in}{2.128147in}}{\pgfqpoint{2.972744in}{2.133970in}}%
\pgfpathcurveto{\pgfqpoint{2.978568in}{2.139794in}}{\pgfqpoint{2.981840in}{2.147694in}}{\pgfqpoint{2.981840in}{2.155931in}}%
\pgfpathcurveto{\pgfqpoint{2.981840in}{2.164167in}}{\pgfqpoint{2.978568in}{2.172067in}}{\pgfqpoint{2.972744in}{2.177891in}}%
\pgfpathcurveto{\pgfqpoint{2.966920in}{2.183715in}}{\pgfqpoint{2.959020in}{2.186987in}}{\pgfqpoint{2.950783in}{2.186987in}}%
\pgfpathcurveto{\pgfqpoint{2.942547in}{2.186987in}}{\pgfqpoint{2.934647in}{2.183715in}}{\pgfqpoint{2.928823in}{2.177891in}}%
\pgfpathcurveto{\pgfqpoint{2.922999in}{2.172067in}}{\pgfqpoint{2.919727in}{2.164167in}}{\pgfqpoint{2.919727in}{2.155931in}}%
\pgfpathcurveto{\pgfqpoint{2.919727in}{2.147694in}}{\pgfqpoint{2.922999in}{2.139794in}}{\pgfqpoint{2.928823in}{2.133970in}}%
\pgfpathcurveto{\pgfqpoint{2.934647in}{2.128147in}}{\pgfqpoint{2.942547in}{2.124874in}}{\pgfqpoint{2.950783in}{2.124874in}}%
\pgfpathclose%
\pgfusepath{stroke,fill}%
\end{pgfscope}%
\begin{pgfscope}%
\pgfpathrectangle{\pgfqpoint{0.100000in}{0.212622in}}{\pgfqpoint{3.696000in}{3.696000in}}%
\pgfusepath{clip}%
\pgfsetbuttcap%
\pgfsetroundjoin%
\definecolor{currentfill}{rgb}{0.121569,0.466667,0.705882}%
\pgfsetfillcolor{currentfill}%
\pgfsetfillopacity{0.738751}%
\pgfsetlinewidth{1.003750pt}%
\definecolor{currentstroke}{rgb}{0.121569,0.466667,0.705882}%
\pgfsetstrokecolor{currentstroke}%
\pgfsetstrokeopacity{0.738751}%
\pgfsetdash{}{0pt}%
\pgfpathmoveto{\pgfqpoint{2.947802in}{2.123599in}}%
\pgfpathcurveto{\pgfqpoint{2.956038in}{2.123599in}}{\pgfqpoint{2.963938in}{2.126871in}}{\pgfqpoint{2.969762in}{2.132695in}}%
\pgfpathcurveto{\pgfqpoint{2.975586in}{2.138519in}}{\pgfqpoint{2.978858in}{2.146419in}}{\pgfqpoint{2.978858in}{2.154655in}}%
\pgfpathcurveto{\pgfqpoint{2.978858in}{2.162892in}}{\pgfqpoint{2.975586in}{2.170792in}}{\pgfqpoint{2.969762in}{2.176616in}}%
\pgfpathcurveto{\pgfqpoint{2.963938in}{2.182440in}}{\pgfqpoint{2.956038in}{2.185712in}}{\pgfqpoint{2.947802in}{2.185712in}}%
\pgfpathcurveto{\pgfqpoint{2.939566in}{2.185712in}}{\pgfqpoint{2.931666in}{2.182440in}}{\pgfqpoint{2.925842in}{2.176616in}}%
\pgfpathcurveto{\pgfqpoint{2.920018in}{2.170792in}}{\pgfqpoint{2.916745in}{2.162892in}}{\pgfqpoint{2.916745in}{2.154655in}}%
\pgfpathcurveto{\pgfqpoint{2.916745in}{2.146419in}}{\pgfqpoint{2.920018in}{2.138519in}}{\pgfqpoint{2.925842in}{2.132695in}}%
\pgfpathcurveto{\pgfqpoint{2.931666in}{2.126871in}}{\pgfqpoint{2.939566in}{2.123599in}}{\pgfqpoint{2.947802in}{2.123599in}}%
\pgfpathclose%
\pgfusepath{stroke,fill}%
\end{pgfscope}%
\begin{pgfscope}%
\pgfpathrectangle{\pgfqpoint{0.100000in}{0.212622in}}{\pgfqpoint{3.696000in}{3.696000in}}%
\pgfusepath{clip}%
\pgfsetbuttcap%
\pgfsetroundjoin%
\definecolor{currentfill}{rgb}{0.121569,0.466667,0.705882}%
\pgfsetfillcolor{currentfill}%
\pgfsetfillopacity{0.739807}%
\pgfsetlinewidth{1.003750pt}%
\definecolor{currentstroke}{rgb}{0.121569,0.466667,0.705882}%
\pgfsetstrokecolor{currentstroke}%
\pgfsetstrokeopacity{0.739807}%
\pgfsetdash{}{0pt}%
\pgfpathmoveto{\pgfqpoint{2.946070in}{2.122621in}}%
\pgfpathcurveto{\pgfqpoint{2.954306in}{2.122621in}}{\pgfqpoint{2.962207in}{2.125893in}}{\pgfqpoint{2.968030in}{2.131717in}}%
\pgfpathcurveto{\pgfqpoint{2.973854in}{2.137541in}}{\pgfqpoint{2.977127in}{2.145441in}}{\pgfqpoint{2.977127in}{2.153678in}}%
\pgfpathcurveto{\pgfqpoint{2.977127in}{2.161914in}}{\pgfqpoint{2.973854in}{2.169814in}}{\pgfqpoint{2.968030in}{2.175638in}}%
\pgfpathcurveto{\pgfqpoint{2.962207in}{2.181462in}}{\pgfqpoint{2.954306in}{2.184734in}}{\pgfqpoint{2.946070in}{2.184734in}}%
\pgfpathcurveto{\pgfqpoint{2.937834in}{2.184734in}}{\pgfqpoint{2.929934in}{2.181462in}}{\pgfqpoint{2.924110in}{2.175638in}}%
\pgfpathcurveto{\pgfqpoint{2.918286in}{2.169814in}}{\pgfqpoint{2.915014in}{2.161914in}}{\pgfqpoint{2.915014in}{2.153678in}}%
\pgfpathcurveto{\pgfqpoint{2.915014in}{2.145441in}}{\pgfqpoint{2.918286in}{2.137541in}}{\pgfqpoint{2.924110in}{2.131717in}}%
\pgfpathcurveto{\pgfqpoint{2.929934in}{2.125893in}}{\pgfqpoint{2.937834in}{2.122621in}}{\pgfqpoint{2.946070in}{2.122621in}}%
\pgfpathclose%
\pgfusepath{stroke,fill}%
\end{pgfscope}%
\begin{pgfscope}%
\pgfpathrectangle{\pgfqpoint{0.100000in}{0.212622in}}{\pgfqpoint{3.696000in}{3.696000in}}%
\pgfusepath{clip}%
\pgfsetbuttcap%
\pgfsetroundjoin%
\definecolor{currentfill}{rgb}{0.121569,0.466667,0.705882}%
\pgfsetfillcolor{currentfill}%
\pgfsetfillopacity{0.741457}%
\pgfsetlinewidth{1.003750pt}%
\definecolor{currentstroke}{rgb}{0.121569,0.466667,0.705882}%
\pgfsetstrokecolor{currentstroke}%
\pgfsetstrokeopacity{0.741457}%
\pgfsetdash{}{0pt}%
\pgfpathmoveto{\pgfqpoint{2.943673in}{2.121060in}}%
\pgfpathcurveto{\pgfqpoint{2.951909in}{2.121060in}}{\pgfqpoint{2.959809in}{2.124333in}}{\pgfqpoint{2.965633in}{2.130156in}}%
\pgfpathcurveto{\pgfqpoint{2.971457in}{2.135980in}}{\pgfqpoint{2.974729in}{2.143880in}}{\pgfqpoint{2.974729in}{2.152117in}}%
\pgfpathcurveto{\pgfqpoint{2.974729in}{2.160353in}}{\pgfqpoint{2.971457in}{2.168253in}}{\pgfqpoint{2.965633in}{2.174077in}}%
\pgfpathcurveto{\pgfqpoint{2.959809in}{2.179901in}}{\pgfqpoint{2.951909in}{2.183173in}}{\pgfqpoint{2.943673in}{2.183173in}}%
\pgfpathcurveto{\pgfqpoint{2.935436in}{2.183173in}}{\pgfqpoint{2.927536in}{2.179901in}}{\pgfqpoint{2.921712in}{2.174077in}}%
\pgfpathcurveto{\pgfqpoint{2.915889in}{2.168253in}}{\pgfqpoint{2.912616in}{2.160353in}}{\pgfqpoint{2.912616in}{2.152117in}}%
\pgfpathcurveto{\pgfqpoint{2.912616in}{2.143880in}}{\pgfqpoint{2.915889in}{2.135980in}}{\pgfqpoint{2.921712in}{2.130156in}}%
\pgfpathcurveto{\pgfqpoint{2.927536in}{2.124333in}}{\pgfqpoint{2.935436in}{2.121060in}}{\pgfqpoint{2.943673in}{2.121060in}}%
\pgfpathclose%
\pgfusepath{stroke,fill}%
\end{pgfscope}%
\begin{pgfscope}%
\pgfpathrectangle{\pgfqpoint{0.100000in}{0.212622in}}{\pgfqpoint{3.696000in}{3.696000in}}%
\pgfusepath{clip}%
\pgfsetbuttcap%
\pgfsetroundjoin%
\definecolor{currentfill}{rgb}{0.121569,0.466667,0.705882}%
\pgfsetfillcolor{currentfill}%
\pgfsetfillopacity{0.742392}%
\pgfsetlinewidth{1.003750pt}%
\definecolor{currentstroke}{rgb}{0.121569,0.466667,0.705882}%
\pgfsetstrokecolor{currentstroke}%
\pgfsetstrokeopacity{0.742392}%
\pgfsetdash{}{0pt}%
\pgfpathmoveto{\pgfqpoint{2.942361in}{2.120374in}}%
\pgfpathcurveto{\pgfqpoint{2.950597in}{2.120374in}}{\pgfqpoint{2.958497in}{2.123647in}}{\pgfqpoint{2.964321in}{2.129470in}}%
\pgfpathcurveto{\pgfqpoint{2.970145in}{2.135294in}}{\pgfqpoint{2.973417in}{2.143194in}}{\pgfqpoint{2.973417in}{2.151431in}}%
\pgfpathcurveto{\pgfqpoint{2.973417in}{2.159667in}}{\pgfqpoint{2.970145in}{2.167567in}}{\pgfqpoint{2.964321in}{2.173391in}}%
\pgfpathcurveto{\pgfqpoint{2.958497in}{2.179215in}}{\pgfqpoint{2.950597in}{2.182487in}}{\pgfqpoint{2.942361in}{2.182487in}}%
\pgfpathcurveto{\pgfqpoint{2.934125in}{2.182487in}}{\pgfqpoint{2.926225in}{2.179215in}}{\pgfqpoint{2.920401in}{2.173391in}}%
\pgfpathcurveto{\pgfqpoint{2.914577in}{2.167567in}}{\pgfqpoint{2.911304in}{2.159667in}}{\pgfqpoint{2.911304in}{2.151431in}}%
\pgfpathcurveto{\pgfqpoint{2.911304in}{2.143194in}}{\pgfqpoint{2.914577in}{2.135294in}}{\pgfqpoint{2.920401in}{2.129470in}}%
\pgfpathcurveto{\pgfqpoint{2.926225in}{2.123647in}}{\pgfqpoint{2.934125in}{2.120374in}}{\pgfqpoint{2.942361in}{2.120374in}}%
\pgfpathclose%
\pgfusepath{stroke,fill}%
\end{pgfscope}%
\begin{pgfscope}%
\pgfpathrectangle{\pgfqpoint{0.100000in}{0.212622in}}{\pgfqpoint{3.696000in}{3.696000in}}%
\pgfusepath{clip}%
\pgfsetbuttcap%
\pgfsetroundjoin%
\definecolor{currentfill}{rgb}{0.121569,0.466667,0.705882}%
\pgfsetfillcolor{currentfill}%
\pgfsetfillopacity{0.742889}%
\pgfsetlinewidth{1.003750pt}%
\definecolor{currentstroke}{rgb}{0.121569,0.466667,0.705882}%
\pgfsetstrokecolor{currentstroke}%
\pgfsetstrokeopacity{0.742889}%
\pgfsetdash{}{0pt}%
\pgfpathmoveto{\pgfqpoint{2.941578in}{2.119940in}}%
\pgfpathcurveto{\pgfqpoint{2.949815in}{2.119940in}}{\pgfqpoint{2.957715in}{2.123212in}}{\pgfqpoint{2.963539in}{2.129036in}}%
\pgfpathcurveto{\pgfqpoint{2.969363in}{2.134860in}}{\pgfqpoint{2.972635in}{2.142760in}}{\pgfqpoint{2.972635in}{2.150997in}}%
\pgfpathcurveto{\pgfqpoint{2.972635in}{2.159233in}}{\pgfqpoint{2.969363in}{2.167133in}}{\pgfqpoint{2.963539in}{2.172957in}}%
\pgfpathcurveto{\pgfqpoint{2.957715in}{2.178781in}}{\pgfqpoint{2.949815in}{2.182053in}}{\pgfqpoint{2.941578in}{2.182053in}}%
\pgfpathcurveto{\pgfqpoint{2.933342in}{2.182053in}}{\pgfqpoint{2.925442in}{2.178781in}}{\pgfqpoint{2.919618in}{2.172957in}}%
\pgfpathcurveto{\pgfqpoint{2.913794in}{2.167133in}}{\pgfqpoint{2.910522in}{2.159233in}}{\pgfqpoint{2.910522in}{2.150997in}}%
\pgfpathcurveto{\pgfqpoint{2.910522in}{2.142760in}}{\pgfqpoint{2.913794in}{2.134860in}}{\pgfqpoint{2.919618in}{2.129036in}}%
\pgfpathcurveto{\pgfqpoint{2.925442in}{2.123212in}}{\pgfqpoint{2.933342in}{2.119940in}}{\pgfqpoint{2.941578in}{2.119940in}}%
\pgfpathclose%
\pgfusepath{stroke,fill}%
\end{pgfscope}%
\begin{pgfscope}%
\pgfpathrectangle{\pgfqpoint{0.100000in}{0.212622in}}{\pgfqpoint{3.696000in}{3.696000in}}%
\pgfusepath{clip}%
\pgfsetbuttcap%
\pgfsetroundjoin%
\definecolor{currentfill}{rgb}{0.121569,0.466667,0.705882}%
\pgfsetfillcolor{currentfill}%
\pgfsetfillopacity{0.743166}%
\pgfsetlinewidth{1.003750pt}%
\definecolor{currentstroke}{rgb}{0.121569,0.466667,0.705882}%
\pgfsetstrokecolor{currentstroke}%
\pgfsetstrokeopacity{0.743166}%
\pgfsetdash{}{0pt}%
\pgfpathmoveto{\pgfqpoint{2.941174in}{2.119702in}}%
\pgfpathcurveto{\pgfqpoint{2.949410in}{2.119702in}}{\pgfqpoint{2.957310in}{2.122974in}}{\pgfqpoint{2.963134in}{2.128798in}}%
\pgfpathcurveto{\pgfqpoint{2.968958in}{2.134622in}}{\pgfqpoint{2.972230in}{2.142522in}}{\pgfqpoint{2.972230in}{2.150758in}}%
\pgfpathcurveto{\pgfqpoint{2.972230in}{2.158994in}}{\pgfqpoint{2.968958in}{2.166894in}}{\pgfqpoint{2.963134in}{2.172718in}}%
\pgfpathcurveto{\pgfqpoint{2.957310in}{2.178542in}}{\pgfqpoint{2.949410in}{2.181815in}}{\pgfqpoint{2.941174in}{2.181815in}}%
\pgfpathcurveto{\pgfqpoint{2.932937in}{2.181815in}}{\pgfqpoint{2.925037in}{2.178542in}}{\pgfqpoint{2.919214in}{2.172718in}}%
\pgfpathcurveto{\pgfqpoint{2.913390in}{2.166894in}}{\pgfqpoint{2.910117in}{2.158994in}}{\pgfqpoint{2.910117in}{2.150758in}}%
\pgfpathcurveto{\pgfqpoint{2.910117in}{2.142522in}}{\pgfqpoint{2.913390in}{2.134622in}}{\pgfqpoint{2.919214in}{2.128798in}}%
\pgfpathcurveto{\pgfqpoint{2.925037in}{2.122974in}}{\pgfqpoint{2.932937in}{2.119702in}}{\pgfqpoint{2.941174in}{2.119702in}}%
\pgfpathclose%
\pgfusepath{stroke,fill}%
\end{pgfscope}%
\begin{pgfscope}%
\pgfpathrectangle{\pgfqpoint{0.100000in}{0.212622in}}{\pgfqpoint{3.696000in}{3.696000in}}%
\pgfusepath{clip}%
\pgfsetbuttcap%
\pgfsetroundjoin%
\definecolor{currentfill}{rgb}{0.121569,0.466667,0.705882}%
\pgfsetfillcolor{currentfill}%
\pgfsetfillopacity{0.743785}%
\pgfsetlinewidth{1.003750pt}%
\definecolor{currentstroke}{rgb}{0.121569,0.466667,0.705882}%
\pgfsetstrokecolor{currentstroke}%
\pgfsetstrokeopacity{0.743785}%
\pgfsetdash{}{0pt}%
\pgfpathmoveto{\pgfqpoint{2.940194in}{2.119121in}}%
\pgfpathcurveto{\pgfqpoint{2.948430in}{2.119121in}}{\pgfqpoint{2.956330in}{2.122393in}}{\pgfqpoint{2.962154in}{2.128217in}}%
\pgfpathcurveto{\pgfqpoint{2.967978in}{2.134041in}}{\pgfqpoint{2.971250in}{2.141941in}}{\pgfqpoint{2.971250in}{2.150177in}}%
\pgfpathcurveto{\pgfqpoint{2.971250in}{2.158414in}}{\pgfqpoint{2.967978in}{2.166314in}}{\pgfqpoint{2.962154in}{2.172138in}}%
\pgfpathcurveto{\pgfqpoint{2.956330in}{2.177961in}}{\pgfqpoint{2.948430in}{2.181234in}}{\pgfqpoint{2.940194in}{2.181234in}}%
\pgfpathcurveto{\pgfqpoint{2.931957in}{2.181234in}}{\pgfqpoint{2.924057in}{2.177961in}}{\pgfqpoint{2.918233in}{2.172138in}}%
\pgfpathcurveto{\pgfqpoint{2.912410in}{2.166314in}}{\pgfqpoint{2.909137in}{2.158414in}}{\pgfqpoint{2.909137in}{2.150177in}}%
\pgfpathcurveto{\pgfqpoint{2.909137in}{2.141941in}}{\pgfqpoint{2.912410in}{2.134041in}}{\pgfqpoint{2.918233in}{2.128217in}}%
\pgfpathcurveto{\pgfqpoint{2.924057in}{2.122393in}}{\pgfqpoint{2.931957in}{2.119121in}}{\pgfqpoint{2.940194in}{2.119121in}}%
\pgfpathclose%
\pgfusepath{stroke,fill}%
\end{pgfscope}%
\begin{pgfscope}%
\pgfpathrectangle{\pgfqpoint{0.100000in}{0.212622in}}{\pgfqpoint{3.696000in}{3.696000in}}%
\pgfusepath{clip}%
\pgfsetbuttcap%
\pgfsetroundjoin%
\definecolor{currentfill}{rgb}{0.121569,0.466667,0.705882}%
\pgfsetfillcolor{currentfill}%
\pgfsetfillopacity{0.744646}%
\pgfsetlinewidth{1.003750pt}%
\definecolor{currentstroke}{rgb}{0.121569,0.466667,0.705882}%
\pgfsetstrokecolor{currentstroke}%
\pgfsetstrokeopacity{0.744646}%
\pgfsetdash{}{0pt}%
\pgfpathmoveto{\pgfqpoint{2.938739in}{2.118226in}}%
\pgfpathcurveto{\pgfqpoint{2.946975in}{2.118226in}}{\pgfqpoint{2.954875in}{2.121498in}}{\pgfqpoint{2.960699in}{2.127322in}}%
\pgfpathcurveto{\pgfqpoint{2.966523in}{2.133146in}}{\pgfqpoint{2.969795in}{2.141046in}}{\pgfqpoint{2.969795in}{2.149282in}}%
\pgfpathcurveto{\pgfqpoint{2.969795in}{2.157519in}}{\pgfqpoint{2.966523in}{2.165419in}}{\pgfqpoint{2.960699in}{2.171243in}}%
\pgfpathcurveto{\pgfqpoint{2.954875in}{2.177067in}}{\pgfqpoint{2.946975in}{2.180339in}}{\pgfqpoint{2.938739in}{2.180339in}}%
\pgfpathcurveto{\pgfqpoint{2.930503in}{2.180339in}}{\pgfqpoint{2.922603in}{2.177067in}}{\pgfqpoint{2.916779in}{2.171243in}}%
\pgfpathcurveto{\pgfqpoint{2.910955in}{2.165419in}}{\pgfqpoint{2.907682in}{2.157519in}}{\pgfqpoint{2.907682in}{2.149282in}}%
\pgfpathcurveto{\pgfqpoint{2.907682in}{2.141046in}}{\pgfqpoint{2.910955in}{2.133146in}}{\pgfqpoint{2.916779in}{2.127322in}}%
\pgfpathcurveto{\pgfqpoint{2.922603in}{2.121498in}}{\pgfqpoint{2.930503in}{2.118226in}}{\pgfqpoint{2.938739in}{2.118226in}}%
\pgfpathclose%
\pgfusepath{stroke,fill}%
\end{pgfscope}%
\begin{pgfscope}%
\pgfpathrectangle{\pgfqpoint{0.100000in}{0.212622in}}{\pgfqpoint{3.696000in}{3.696000in}}%
\pgfusepath{clip}%
\pgfsetbuttcap%
\pgfsetroundjoin%
\definecolor{currentfill}{rgb}{0.121569,0.466667,0.705882}%
\pgfsetfillcolor{currentfill}%
\pgfsetfillopacity{0.745718}%
\pgfsetlinewidth{1.003750pt}%
\definecolor{currentstroke}{rgb}{0.121569,0.466667,0.705882}%
\pgfsetstrokecolor{currentstroke}%
\pgfsetstrokeopacity{0.745718}%
\pgfsetdash{}{0pt}%
\pgfpathmoveto{\pgfqpoint{2.937140in}{2.117364in}}%
\pgfpathcurveto{\pgfqpoint{2.945376in}{2.117364in}}{\pgfqpoint{2.953276in}{2.120636in}}{\pgfqpoint{2.959100in}{2.126460in}}%
\pgfpathcurveto{\pgfqpoint{2.964924in}{2.132284in}}{\pgfqpoint{2.968197in}{2.140184in}}{\pgfqpoint{2.968197in}{2.148420in}}%
\pgfpathcurveto{\pgfqpoint{2.968197in}{2.156656in}}{\pgfqpoint{2.964924in}{2.164556in}}{\pgfqpoint{2.959100in}{2.170380in}}%
\pgfpathcurveto{\pgfqpoint{2.953276in}{2.176204in}}{\pgfqpoint{2.945376in}{2.179477in}}{\pgfqpoint{2.937140in}{2.179477in}}%
\pgfpathcurveto{\pgfqpoint{2.928904in}{2.179477in}}{\pgfqpoint{2.921004in}{2.176204in}}{\pgfqpoint{2.915180in}{2.170380in}}%
\pgfpathcurveto{\pgfqpoint{2.909356in}{2.164556in}}{\pgfqpoint{2.906084in}{2.156656in}}{\pgfqpoint{2.906084in}{2.148420in}}%
\pgfpathcurveto{\pgfqpoint{2.906084in}{2.140184in}}{\pgfqpoint{2.909356in}{2.132284in}}{\pgfqpoint{2.915180in}{2.126460in}}%
\pgfpathcurveto{\pgfqpoint{2.921004in}{2.120636in}}{\pgfqpoint{2.928904in}{2.117364in}}{\pgfqpoint{2.937140in}{2.117364in}}%
\pgfpathclose%
\pgfusepath{stroke,fill}%
\end{pgfscope}%
\begin{pgfscope}%
\pgfpathrectangle{\pgfqpoint{0.100000in}{0.212622in}}{\pgfqpoint{3.696000in}{3.696000in}}%
\pgfusepath{clip}%
\pgfsetbuttcap%
\pgfsetroundjoin%
\definecolor{currentfill}{rgb}{0.121569,0.466667,0.705882}%
\pgfsetfillcolor{currentfill}%
\pgfsetfillopacity{0.746893}%
\pgfsetlinewidth{1.003750pt}%
\definecolor{currentstroke}{rgb}{0.121569,0.466667,0.705882}%
\pgfsetstrokecolor{currentstroke}%
\pgfsetstrokeopacity{0.746893}%
\pgfsetdash{}{0pt}%
\pgfpathmoveto{\pgfqpoint{2.935219in}{2.116216in}}%
\pgfpathcurveto{\pgfqpoint{2.943455in}{2.116216in}}{\pgfqpoint{2.951355in}{2.119488in}}{\pgfqpoint{2.957179in}{2.125312in}}%
\pgfpathcurveto{\pgfqpoint{2.963003in}{2.131136in}}{\pgfqpoint{2.966275in}{2.139036in}}{\pgfqpoint{2.966275in}{2.147272in}}%
\pgfpathcurveto{\pgfqpoint{2.966275in}{2.155508in}}{\pgfqpoint{2.963003in}{2.163408in}}{\pgfqpoint{2.957179in}{2.169232in}}%
\pgfpathcurveto{\pgfqpoint{2.951355in}{2.175056in}}{\pgfqpoint{2.943455in}{2.178329in}}{\pgfqpoint{2.935219in}{2.178329in}}%
\pgfpathcurveto{\pgfqpoint{2.926983in}{2.178329in}}{\pgfqpoint{2.919083in}{2.175056in}}{\pgfqpoint{2.913259in}{2.169232in}}%
\pgfpathcurveto{\pgfqpoint{2.907435in}{2.163408in}}{\pgfqpoint{2.904162in}{2.155508in}}{\pgfqpoint{2.904162in}{2.147272in}}%
\pgfpathcurveto{\pgfqpoint{2.904162in}{2.139036in}}{\pgfqpoint{2.907435in}{2.131136in}}{\pgfqpoint{2.913259in}{2.125312in}}%
\pgfpathcurveto{\pgfqpoint{2.919083in}{2.119488in}}{\pgfqpoint{2.926983in}{2.116216in}}{\pgfqpoint{2.935219in}{2.116216in}}%
\pgfpathclose%
\pgfusepath{stroke,fill}%
\end{pgfscope}%
\begin{pgfscope}%
\pgfpathrectangle{\pgfqpoint{0.100000in}{0.212622in}}{\pgfqpoint{3.696000in}{3.696000in}}%
\pgfusepath{clip}%
\pgfsetbuttcap%
\pgfsetroundjoin%
\definecolor{currentfill}{rgb}{0.121569,0.466667,0.705882}%
\pgfsetfillcolor{currentfill}%
\pgfsetfillopacity{0.748447}%
\pgfsetlinewidth{1.003750pt}%
\definecolor{currentstroke}{rgb}{0.121569,0.466667,0.705882}%
\pgfsetstrokecolor{currentstroke}%
\pgfsetstrokeopacity{0.748447}%
\pgfsetdash{}{0pt}%
\pgfpathmoveto{\pgfqpoint{2.932771in}{2.114609in}}%
\pgfpathcurveto{\pgfqpoint{2.941007in}{2.114609in}}{\pgfqpoint{2.948907in}{2.117881in}}{\pgfqpoint{2.954731in}{2.123705in}}%
\pgfpathcurveto{\pgfqpoint{2.960555in}{2.129529in}}{\pgfqpoint{2.963827in}{2.137429in}}{\pgfqpoint{2.963827in}{2.145665in}}%
\pgfpathcurveto{\pgfqpoint{2.963827in}{2.153902in}}{\pgfqpoint{2.960555in}{2.161802in}}{\pgfqpoint{2.954731in}{2.167626in}}%
\pgfpathcurveto{\pgfqpoint{2.948907in}{2.173450in}}{\pgfqpoint{2.941007in}{2.176722in}}{\pgfqpoint{2.932771in}{2.176722in}}%
\pgfpathcurveto{\pgfqpoint{2.924535in}{2.176722in}}{\pgfqpoint{2.916634in}{2.173450in}}{\pgfqpoint{2.910811in}{2.167626in}}%
\pgfpathcurveto{\pgfqpoint{2.904987in}{2.161802in}}{\pgfqpoint{2.901714in}{2.153902in}}{\pgfqpoint{2.901714in}{2.145665in}}%
\pgfpathcurveto{\pgfqpoint{2.901714in}{2.137429in}}{\pgfqpoint{2.904987in}{2.129529in}}{\pgfqpoint{2.910811in}{2.123705in}}%
\pgfpathcurveto{\pgfqpoint{2.916634in}{2.117881in}}{\pgfqpoint{2.924535in}{2.114609in}}{\pgfqpoint{2.932771in}{2.114609in}}%
\pgfpathclose%
\pgfusepath{stroke,fill}%
\end{pgfscope}%
\begin{pgfscope}%
\pgfpathrectangle{\pgfqpoint{0.100000in}{0.212622in}}{\pgfqpoint{3.696000in}{3.696000in}}%
\pgfusepath{clip}%
\pgfsetbuttcap%
\pgfsetroundjoin%
\definecolor{currentfill}{rgb}{0.121569,0.466667,0.705882}%
\pgfsetfillcolor{currentfill}%
\pgfsetfillopacity{0.750319}%
\pgfsetlinewidth{1.003750pt}%
\definecolor{currentstroke}{rgb}{0.121569,0.466667,0.705882}%
\pgfsetstrokecolor{currentstroke}%
\pgfsetstrokeopacity{0.750319}%
\pgfsetdash{}{0pt}%
\pgfpathmoveto{\pgfqpoint{2.930099in}{2.113311in}}%
\pgfpathcurveto{\pgfqpoint{2.938336in}{2.113311in}}{\pgfqpoint{2.946236in}{2.116584in}}{\pgfqpoint{2.952060in}{2.122408in}}%
\pgfpathcurveto{\pgfqpoint{2.957883in}{2.128232in}}{\pgfqpoint{2.961156in}{2.136132in}}{\pgfqpoint{2.961156in}{2.144368in}}%
\pgfpathcurveto{\pgfqpoint{2.961156in}{2.152604in}}{\pgfqpoint{2.957883in}{2.160504in}}{\pgfqpoint{2.952060in}{2.166328in}}%
\pgfpathcurveto{\pgfqpoint{2.946236in}{2.172152in}}{\pgfqpoint{2.938336in}{2.175424in}}{\pgfqpoint{2.930099in}{2.175424in}}%
\pgfpathcurveto{\pgfqpoint{2.921863in}{2.175424in}}{\pgfqpoint{2.913963in}{2.172152in}}{\pgfqpoint{2.908139in}{2.166328in}}%
\pgfpathcurveto{\pgfqpoint{2.902315in}{2.160504in}}{\pgfqpoint{2.899043in}{2.152604in}}{\pgfqpoint{2.899043in}{2.144368in}}%
\pgfpathcurveto{\pgfqpoint{2.899043in}{2.136132in}}{\pgfqpoint{2.902315in}{2.128232in}}{\pgfqpoint{2.908139in}{2.122408in}}%
\pgfpathcurveto{\pgfqpoint{2.913963in}{2.116584in}}{\pgfqpoint{2.921863in}{2.113311in}}{\pgfqpoint{2.930099in}{2.113311in}}%
\pgfpathclose%
\pgfusepath{stroke,fill}%
\end{pgfscope}%
\begin{pgfscope}%
\pgfpathrectangle{\pgfqpoint{0.100000in}{0.212622in}}{\pgfqpoint{3.696000in}{3.696000in}}%
\pgfusepath{clip}%
\pgfsetbuttcap%
\pgfsetroundjoin%
\definecolor{currentfill}{rgb}{0.121569,0.466667,0.705882}%
\pgfsetfillcolor{currentfill}%
\pgfsetfillopacity{0.751304}%
\pgfsetlinewidth{1.003750pt}%
\definecolor{currentstroke}{rgb}{0.121569,0.466667,0.705882}%
\pgfsetstrokecolor{currentstroke}%
\pgfsetstrokeopacity{0.751304}%
\pgfsetdash{}{0pt}%
\pgfpathmoveto{\pgfqpoint{2.928518in}{2.112410in}}%
\pgfpathcurveto{\pgfqpoint{2.936754in}{2.112410in}}{\pgfqpoint{2.944654in}{2.115682in}}{\pgfqpoint{2.950478in}{2.121506in}}%
\pgfpathcurveto{\pgfqpoint{2.956302in}{2.127330in}}{\pgfqpoint{2.959574in}{2.135230in}}{\pgfqpoint{2.959574in}{2.143467in}}%
\pgfpathcurveto{\pgfqpoint{2.959574in}{2.151703in}}{\pgfqpoint{2.956302in}{2.159603in}}{\pgfqpoint{2.950478in}{2.165427in}}%
\pgfpathcurveto{\pgfqpoint{2.944654in}{2.171251in}}{\pgfqpoint{2.936754in}{2.174523in}}{\pgfqpoint{2.928518in}{2.174523in}}%
\pgfpathcurveto{\pgfqpoint{2.920282in}{2.174523in}}{\pgfqpoint{2.912382in}{2.171251in}}{\pgfqpoint{2.906558in}{2.165427in}}%
\pgfpathcurveto{\pgfqpoint{2.900734in}{2.159603in}}{\pgfqpoint{2.897461in}{2.151703in}}{\pgfqpoint{2.897461in}{2.143467in}}%
\pgfpathcurveto{\pgfqpoint{2.897461in}{2.135230in}}{\pgfqpoint{2.900734in}{2.127330in}}{\pgfqpoint{2.906558in}{2.121506in}}%
\pgfpathcurveto{\pgfqpoint{2.912382in}{2.115682in}}{\pgfqpoint{2.920282in}{2.112410in}}{\pgfqpoint{2.928518in}{2.112410in}}%
\pgfpathclose%
\pgfusepath{stroke,fill}%
\end{pgfscope}%
\begin{pgfscope}%
\pgfpathrectangle{\pgfqpoint{0.100000in}{0.212622in}}{\pgfqpoint{3.696000in}{3.696000in}}%
\pgfusepath{clip}%
\pgfsetbuttcap%
\pgfsetroundjoin%
\definecolor{currentfill}{rgb}{0.121569,0.466667,0.705882}%
\pgfsetfillcolor{currentfill}%
\pgfsetfillopacity{0.752447}%
\pgfsetlinewidth{1.003750pt}%
\definecolor{currentstroke}{rgb}{0.121569,0.466667,0.705882}%
\pgfsetstrokecolor{currentstroke}%
\pgfsetstrokeopacity{0.752447}%
\pgfsetdash{}{0pt}%
\pgfpathmoveto{\pgfqpoint{2.926851in}{2.111285in}}%
\pgfpathcurveto{\pgfqpoint{2.935087in}{2.111285in}}{\pgfqpoint{2.942987in}{2.114557in}}{\pgfqpoint{2.948811in}{2.120381in}}%
\pgfpathcurveto{\pgfqpoint{2.954635in}{2.126205in}}{\pgfqpoint{2.957907in}{2.134105in}}{\pgfqpoint{2.957907in}{2.142341in}}%
\pgfpathcurveto{\pgfqpoint{2.957907in}{2.150577in}}{\pgfqpoint{2.954635in}{2.158477in}}{\pgfqpoint{2.948811in}{2.164301in}}%
\pgfpathcurveto{\pgfqpoint{2.942987in}{2.170125in}}{\pgfqpoint{2.935087in}{2.173398in}}{\pgfqpoint{2.926851in}{2.173398in}}%
\pgfpathcurveto{\pgfqpoint{2.918614in}{2.173398in}}{\pgfqpoint{2.910714in}{2.170125in}}{\pgfqpoint{2.904890in}{2.164301in}}%
\pgfpathcurveto{\pgfqpoint{2.899066in}{2.158477in}}{\pgfqpoint{2.895794in}{2.150577in}}{\pgfqpoint{2.895794in}{2.142341in}}%
\pgfpathcurveto{\pgfqpoint{2.895794in}{2.134105in}}{\pgfqpoint{2.899066in}{2.126205in}}{\pgfqpoint{2.904890in}{2.120381in}}%
\pgfpathcurveto{\pgfqpoint{2.910714in}{2.114557in}}{\pgfqpoint{2.918614in}{2.111285in}}{\pgfqpoint{2.926851in}{2.111285in}}%
\pgfpathclose%
\pgfusepath{stroke,fill}%
\end{pgfscope}%
\begin{pgfscope}%
\pgfpathrectangle{\pgfqpoint{0.100000in}{0.212622in}}{\pgfqpoint{3.696000in}{3.696000in}}%
\pgfusepath{clip}%
\pgfsetbuttcap%
\pgfsetroundjoin%
\definecolor{currentfill}{rgb}{0.121569,0.466667,0.705882}%
\pgfsetfillcolor{currentfill}%
\pgfsetfillopacity{0.753934}%
\pgfsetlinewidth{1.003750pt}%
\definecolor{currentstroke}{rgb}{0.121569,0.466667,0.705882}%
\pgfsetstrokecolor{currentstroke}%
\pgfsetstrokeopacity{0.753934}%
\pgfsetdash{}{0pt}%
\pgfpathmoveto{\pgfqpoint{2.924818in}{2.109588in}}%
\pgfpathcurveto{\pgfqpoint{2.933054in}{2.109588in}}{\pgfqpoint{2.940954in}{2.112860in}}{\pgfqpoint{2.946778in}{2.118684in}}%
\pgfpathcurveto{\pgfqpoint{2.952602in}{2.124508in}}{\pgfqpoint{2.955874in}{2.132408in}}{\pgfqpoint{2.955874in}{2.140644in}}%
\pgfpathcurveto{\pgfqpoint{2.955874in}{2.148881in}}{\pgfqpoint{2.952602in}{2.156781in}}{\pgfqpoint{2.946778in}{2.162605in}}%
\pgfpathcurveto{\pgfqpoint{2.940954in}{2.168428in}}{\pgfqpoint{2.933054in}{2.171701in}}{\pgfqpoint{2.924818in}{2.171701in}}%
\pgfpathcurveto{\pgfqpoint{2.916582in}{2.171701in}}{\pgfqpoint{2.908682in}{2.168428in}}{\pgfqpoint{2.902858in}{2.162605in}}%
\pgfpathcurveto{\pgfqpoint{2.897034in}{2.156781in}}{\pgfqpoint{2.893761in}{2.148881in}}{\pgfqpoint{2.893761in}{2.140644in}}%
\pgfpathcurveto{\pgfqpoint{2.893761in}{2.132408in}}{\pgfqpoint{2.897034in}{2.124508in}}{\pgfqpoint{2.902858in}{2.118684in}}%
\pgfpathcurveto{\pgfqpoint{2.908682in}{2.112860in}}{\pgfqpoint{2.916582in}{2.109588in}}{\pgfqpoint{2.924818in}{2.109588in}}%
\pgfpathclose%
\pgfusepath{stroke,fill}%
\end{pgfscope}%
\begin{pgfscope}%
\pgfpathrectangle{\pgfqpoint{0.100000in}{0.212622in}}{\pgfqpoint{3.696000in}{3.696000in}}%
\pgfusepath{clip}%
\pgfsetbuttcap%
\pgfsetroundjoin%
\definecolor{currentfill}{rgb}{0.121569,0.466667,0.705882}%
\pgfsetfillcolor{currentfill}%
\pgfsetfillopacity{0.754742}%
\pgfsetlinewidth{1.003750pt}%
\definecolor{currentstroke}{rgb}{0.121569,0.466667,0.705882}%
\pgfsetstrokecolor{currentstroke}%
\pgfsetstrokeopacity{0.754742}%
\pgfsetdash{}{0pt}%
\pgfpathmoveto{\pgfqpoint{2.923644in}{2.108631in}}%
\pgfpathcurveto{\pgfqpoint{2.931880in}{2.108631in}}{\pgfqpoint{2.939780in}{2.111904in}}{\pgfqpoint{2.945604in}{2.117727in}}%
\pgfpathcurveto{\pgfqpoint{2.951428in}{2.123551in}}{\pgfqpoint{2.954700in}{2.131451in}}{\pgfqpoint{2.954700in}{2.139688in}}%
\pgfpathcurveto{\pgfqpoint{2.954700in}{2.147924in}}{\pgfqpoint{2.951428in}{2.155824in}}{\pgfqpoint{2.945604in}{2.161648in}}%
\pgfpathcurveto{\pgfqpoint{2.939780in}{2.167472in}}{\pgfqpoint{2.931880in}{2.170744in}}{\pgfqpoint{2.923644in}{2.170744in}}%
\pgfpathcurveto{\pgfqpoint{2.915408in}{2.170744in}}{\pgfqpoint{2.907507in}{2.167472in}}{\pgfqpoint{2.901684in}{2.161648in}}%
\pgfpathcurveto{\pgfqpoint{2.895860in}{2.155824in}}{\pgfqpoint{2.892587in}{2.147924in}}{\pgfqpoint{2.892587in}{2.139688in}}%
\pgfpathcurveto{\pgfqpoint{2.892587in}{2.131451in}}{\pgfqpoint{2.895860in}{2.123551in}}{\pgfqpoint{2.901684in}{2.117727in}}%
\pgfpathcurveto{\pgfqpoint{2.907507in}{2.111904in}}{\pgfqpoint{2.915408in}{2.108631in}}{\pgfqpoint{2.923644in}{2.108631in}}%
\pgfpathclose%
\pgfusepath{stroke,fill}%
\end{pgfscope}%
\begin{pgfscope}%
\pgfpathrectangle{\pgfqpoint{0.100000in}{0.212622in}}{\pgfqpoint{3.696000in}{3.696000in}}%
\pgfusepath{clip}%
\pgfsetbuttcap%
\pgfsetroundjoin%
\definecolor{currentfill}{rgb}{0.121569,0.466667,0.705882}%
\pgfsetfillcolor{currentfill}%
\pgfsetfillopacity{0.755194}%
\pgfsetlinewidth{1.003750pt}%
\definecolor{currentstroke}{rgb}{0.121569,0.466667,0.705882}%
\pgfsetstrokecolor{currentstroke}%
\pgfsetstrokeopacity{0.755194}%
\pgfsetdash{}{0pt}%
\pgfpathmoveto{\pgfqpoint{2.923043in}{2.108113in}}%
\pgfpathcurveto{\pgfqpoint{2.931279in}{2.108113in}}{\pgfqpoint{2.939179in}{2.111385in}}{\pgfqpoint{2.945003in}{2.117209in}}%
\pgfpathcurveto{\pgfqpoint{2.950827in}{2.123033in}}{\pgfqpoint{2.954099in}{2.130933in}}{\pgfqpoint{2.954099in}{2.139170in}}%
\pgfpathcurveto{\pgfqpoint{2.954099in}{2.147406in}}{\pgfqpoint{2.950827in}{2.155306in}}{\pgfqpoint{2.945003in}{2.161130in}}%
\pgfpathcurveto{\pgfqpoint{2.939179in}{2.166954in}}{\pgfqpoint{2.931279in}{2.170226in}}{\pgfqpoint{2.923043in}{2.170226in}}%
\pgfpathcurveto{\pgfqpoint{2.914807in}{2.170226in}}{\pgfqpoint{2.906907in}{2.166954in}}{\pgfqpoint{2.901083in}{2.161130in}}%
\pgfpathcurveto{\pgfqpoint{2.895259in}{2.155306in}}{\pgfqpoint{2.891986in}{2.147406in}}{\pgfqpoint{2.891986in}{2.139170in}}%
\pgfpathcurveto{\pgfqpoint{2.891986in}{2.130933in}}{\pgfqpoint{2.895259in}{2.123033in}}{\pgfqpoint{2.901083in}{2.117209in}}%
\pgfpathcurveto{\pgfqpoint{2.906907in}{2.111385in}}{\pgfqpoint{2.914807in}{2.108113in}}{\pgfqpoint{2.923043in}{2.108113in}}%
\pgfpathclose%
\pgfusepath{stroke,fill}%
\end{pgfscope}%
\begin{pgfscope}%
\pgfpathrectangle{\pgfqpoint{0.100000in}{0.212622in}}{\pgfqpoint{3.696000in}{3.696000in}}%
\pgfusepath{clip}%
\pgfsetbuttcap%
\pgfsetroundjoin%
\definecolor{currentfill}{rgb}{0.121569,0.466667,0.705882}%
\pgfsetfillcolor{currentfill}%
\pgfsetfillopacity{0.755437}%
\pgfsetlinewidth{1.003750pt}%
\definecolor{currentstroke}{rgb}{0.121569,0.466667,0.705882}%
\pgfsetstrokecolor{currentstroke}%
\pgfsetstrokeopacity{0.755437}%
\pgfsetdash{}{0pt}%
\pgfpathmoveto{\pgfqpoint{2.922678in}{2.107821in}}%
\pgfpathcurveto{\pgfqpoint{2.930914in}{2.107821in}}{\pgfqpoint{2.938814in}{2.111093in}}{\pgfqpoint{2.944638in}{2.116917in}}%
\pgfpathcurveto{\pgfqpoint{2.950462in}{2.122741in}}{\pgfqpoint{2.953734in}{2.130641in}}{\pgfqpoint{2.953734in}{2.138877in}}%
\pgfpathcurveto{\pgfqpoint{2.953734in}{2.147113in}}{\pgfqpoint{2.950462in}{2.155013in}}{\pgfqpoint{2.944638in}{2.160837in}}%
\pgfpathcurveto{\pgfqpoint{2.938814in}{2.166661in}}{\pgfqpoint{2.930914in}{2.169934in}}{\pgfqpoint{2.922678in}{2.169934in}}%
\pgfpathcurveto{\pgfqpoint{2.914442in}{2.169934in}}{\pgfqpoint{2.906541in}{2.166661in}}{\pgfqpoint{2.900718in}{2.160837in}}%
\pgfpathcurveto{\pgfqpoint{2.894894in}{2.155013in}}{\pgfqpoint{2.891621in}{2.147113in}}{\pgfqpoint{2.891621in}{2.138877in}}%
\pgfpathcurveto{\pgfqpoint{2.891621in}{2.130641in}}{\pgfqpoint{2.894894in}{2.122741in}}{\pgfqpoint{2.900718in}{2.116917in}}%
\pgfpathcurveto{\pgfqpoint{2.906541in}{2.111093in}}{\pgfqpoint{2.914442in}{2.107821in}}{\pgfqpoint{2.922678in}{2.107821in}}%
\pgfpathclose%
\pgfusepath{stroke,fill}%
\end{pgfscope}%
\begin{pgfscope}%
\pgfpathrectangle{\pgfqpoint{0.100000in}{0.212622in}}{\pgfqpoint{3.696000in}{3.696000in}}%
\pgfusepath{clip}%
\pgfsetbuttcap%
\pgfsetroundjoin%
\definecolor{currentfill}{rgb}{0.121569,0.466667,0.705882}%
\pgfsetfillcolor{currentfill}%
\pgfsetfillopacity{0.756001}%
\pgfsetlinewidth{1.003750pt}%
\definecolor{currentstroke}{rgb}{0.121569,0.466667,0.705882}%
\pgfsetstrokecolor{currentstroke}%
\pgfsetstrokeopacity{0.756001}%
\pgfsetdash{}{0pt}%
\pgfpathmoveto{\pgfqpoint{2.921845in}{2.107051in}}%
\pgfpathcurveto{\pgfqpoint{2.930082in}{2.107051in}}{\pgfqpoint{2.937982in}{2.110323in}}{\pgfqpoint{2.943806in}{2.116147in}}%
\pgfpathcurveto{\pgfqpoint{2.949630in}{2.121971in}}{\pgfqpoint{2.952902in}{2.129871in}}{\pgfqpoint{2.952902in}{2.138108in}}%
\pgfpathcurveto{\pgfqpoint{2.952902in}{2.146344in}}{\pgfqpoint{2.949630in}{2.154244in}}{\pgfqpoint{2.943806in}{2.160068in}}%
\pgfpathcurveto{\pgfqpoint{2.937982in}{2.165892in}}{\pgfqpoint{2.930082in}{2.169164in}}{\pgfqpoint{2.921845in}{2.169164in}}%
\pgfpathcurveto{\pgfqpoint{2.913609in}{2.169164in}}{\pgfqpoint{2.905709in}{2.165892in}}{\pgfqpoint{2.899885in}{2.160068in}}%
\pgfpathcurveto{\pgfqpoint{2.894061in}{2.154244in}}{\pgfqpoint{2.890789in}{2.146344in}}{\pgfqpoint{2.890789in}{2.138108in}}%
\pgfpathcurveto{\pgfqpoint{2.890789in}{2.129871in}}{\pgfqpoint{2.894061in}{2.121971in}}{\pgfqpoint{2.899885in}{2.116147in}}%
\pgfpathcurveto{\pgfqpoint{2.905709in}{2.110323in}}{\pgfqpoint{2.913609in}{2.107051in}}{\pgfqpoint{2.921845in}{2.107051in}}%
\pgfpathclose%
\pgfusepath{stroke,fill}%
\end{pgfscope}%
\begin{pgfscope}%
\pgfpathrectangle{\pgfqpoint{0.100000in}{0.212622in}}{\pgfqpoint{3.696000in}{3.696000in}}%
\pgfusepath{clip}%
\pgfsetbuttcap%
\pgfsetroundjoin%
\definecolor{currentfill}{rgb}{0.121569,0.466667,0.705882}%
\pgfsetfillcolor{currentfill}%
\pgfsetfillopacity{0.756327}%
\pgfsetlinewidth{1.003750pt}%
\definecolor{currentstroke}{rgb}{0.121569,0.466667,0.705882}%
\pgfsetstrokecolor{currentstroke}%
\pgfsetstrokeopacity{0.756327}%
\pgfsetdash{}{0pt}%
\pgfpathmoveto{\pgfqpoint{2.921407in}{2.106715in}}%
\pgfpathcurveto{\pgfqpoint{2.929643in}{2.106715in}}{\pgfqpoint{2.937543in}{2.109988in}}{\pgfqpoint{2.943367in}{2.115811in}}%
\pgfpathcurveto{\pgfqpoint{2.949191in}{2.121635in}}{\pgfqpoint{2.952463in}{2.129535in}}{\pgfqpoint{2.952463in}{2.137772in}}%
\pgfpathcurveto{\pgfqpoint{2.952463in}{2.146008in}}{\pgfqpoint{2.949191in}{2.153908in}}{\pgfqpoint{2.943367in}{2.159732in}}%
\pgfpathcurveto{\pgfqpoint{2.937543in}{2.165556in}}{\pgfqpoint{2.929643in}{2.168828in}}{\pgfqpoint{2.921407in}{2.168828in}}%
\pgfpathcurveto{\pgfqpoint{2.913171in}{2.168828in}}{\pgfqpoint{2.905270in}{2.165556in}}{\pgfqpoint{2.899447in}{2.159732in}}%
\pgfpathcurveto{\pgfqpoint{2.893623in}{2.153908in}}{\pgfqpoint{2.890350in}{2.146008in}}{\pgfqpoint{2.890350in}{2.137772in}}%
\pgfpathcurveto{\pgfqpoint{2.890350in}{2.129535in}}{\pgfqpoint{2.893623in}{2.121635in}}{\pgfqpoint{2.899447in}{2.115811in}}%
\pgfpathcurveto{\pgfqpoint{2.905270in}{2.109988in}}{\pgfqpoint{2.913171in}{2.106715in}}{\pgfqpoint{2.921407in}{2.106715in}}%
\pgfpathclose%
\pgfusepath{stroke,fill}%
\end{pgfscope}%
\begin{pgfscope}%
\pgfpathrectangle{\pgfqpoint{0.100000in}{0.212622in}}{\pgfqpoint{3.696000in}{3.696000in}}%
\pgfusepath{clip}%
\pgfsetbuttcap%
\pgfsetroundjoin%
\definecolor{currentfill}{rgb}{0.121569,0.466667,0.705882}%
\pgfsetfillcolor{currentfill}%
\pgfsetfillopacity{0.756890}%
\pgfsetlinewidth{1.003750pt}%
\definecolor{currentstroke}{rgb}{0.121569,0.466667,0.705882}%
\pgfsetstrokecolor{currentstroke}%
\pgfsetstrokeopacity{0.756890}%
\pgfsetdash{}{0pt}%
\pgfpathmoveto{\pgfqpoint{2.920543in}{2.106033in}}%
\pgfpathcurveto{\pgfqpoint{2.928779in}{2.106033in}}{\pgfqpoint{2.936679in}{2.109305in}}{\pgfqpoint{2.942503in}{2.115129in}}%
\pgfpathcurveto{\pgfqpoint{2.948327in}{2.120953in}}{\pgfqpoint{2.951599in}{2.128853in}}{\pgfqpoint{2.951599in}{2.137090in}}%
\pgfpathcurveto{\pgfqpoint{2.951599in}{2.145326in}}{\pgfqpoint{2.948327in}{2.153226in}}{\pgfqpoint{2.942503in}{2.159050in}}%
\pgfpathcurveto{\pgfqpoint{2.936679in}{2.164874in}}{\pgfqpoint{2.928779in}{2.168146in}}{\pgfqpoint{2.920543in}{2.168146in}}%
\pgfpathcurveto{\pgfqpoint{2.912307in}{2.168146in}}{\pgfqpoint{2.904406in}{2.164874in}}{\pgfqpoint{2.898583in}{2.159050in}}%
\pgfpathcurveto{\pgfqpoint{2.892759in}{2.153226in}}{\pgfqpoint{2.889486in}{2.145326in}}{\pgfqpoint{2.889486in}{2.137090in}}%
\pgfpathcurveto{\pgfqpoint{2.889486in}{2.128853in}}{\pgfqpoint{2.892759in}{2.120953in}}{\pgfqpoint{2.898583in}{2.115129in}}%
\pgfpathcurveto{\pgfqpoint{2.904406in}{2.109305in}}{\pgfqpoint{2.912307in}{2.106033in}}{\pgfqpoint{2.920543in}{2.106033in}}%
\pgfpathclose%
\pgfusepath{stroke,fill}%
\end{pgfscope}%
\begin{pgfscope}%
\pgfpathrectangle{\pgfqpoint{0.100000in}{0.212622in}}{\pgfqpoint{3.696000in}{3.696000in}}%
\pgfusepath{clip}%
\pgfsetbuttcap%
\pgfsetroundjoin%
\definecolor{currentfill}{rgb}{0.121569,0.466667,0.705882}%
\pgfsetfillcolor{currentfill}%
\pgfsetfillopacity{0.757954}%
\pgfsetlinewidth{1.003750pt}%
\definecolor{currentstroke}{rgb}{0.121569,0.466667,0.705882}%
\pgfsetstrokecolor{currentstroke}%
\pgfsetstrokeopacity{0.757954}%
\pgfsetdash{}{0pt}%
\pgfpathmoveto{\pgfqpoint{2.919062in}{2.104821in}}%
\pgfpathcurveto{\pgfqpoint{2.927298in}{2.104821in}}{\pgfqpoint{2.935198in}{2.108093in}}{\pgfqpoint{2.941022in}{2.113917in}}%
\pgfpathcurveto{\pgfqpoint{2.946846in}{2.119741in}}{\pgfqpoint{2.950118in}{2.127641in}}{\pgfqpoint{2.950118in}{2.135877in}}%
\pgfpathcurveto{\pgfqpoint{2.950118in}{2.144114in}}{\pgfqpoint{2.946846in}{2.152014in}}{\pgfqpoint{2.941022in}{2.157838in}}%
\pgfpathcurveto{\pgfqpoint{2.935198in}{2.163662in}}{\pgfqpoint{2.927298in}{2.166934in}}{\pgfqpoint{2.919062in}{2.166934in}}%
\pgfpathcurveto{\pgfqpoint{2.910825in}{2.166934in}}{\pgfqpoint{2.902925in}{2.163662in}}{\pgfqpoint{2.897101in}{2.157838in}}%
\pgfpathcurveto{\pgfqpoint{2.891278in}{2.152014in}}{\pgfqpoint{2.888005in}{2.144114in}}{\pgfqpoint{2.888005in}{2.135877in}}%
\pgfpathcurveto{\pgfqpoint{2.888005in}{2.127641in}}{\pgfqpoint{2.891278in}{2.119741in}}{\pgfqpoint{2.897101in}{2.113917in}}%
\pgfpathcurveto{\pgfqpoint{2.902925in}{2.108093in}}{\pgfqpoint{2.910825in}{2.104821in}}{\pgfqpoint{2.919062in}{2.104821in}}%
\pgfpathclose%
\pgfusepath{stroke,fill}%
\end{pgfscope}%
\begin{pgfscope}%
\pgfpathrectangle{\pgfqpoint{0.100000in}{0.212622in}}{\pgfqpoint{3.696000in}{3.696000in}}%
\pgfusepath{clip}%
\pgfsetbuttcap%
\pgfsetroundjoin%
\definecolor{currentfill}{rgb}{0.121569,0.466667,0.705882}%
\pgfsetfillcolor{currentfill}%
\pgfsetfillopacity{0.758514}%
\pgfsetlinewidth{1.003750pt}%
\definecolor{currentstroke}{rgb}{0.121569,0.466667,0.705882}%
\pgfsetstrokecolor{currentstroke}%
\pgfsetstrokeopacity{0.758514}%
\pgfsetdash{}{0pt}%
\pgfpathmoveto{\pgfqpoint{2.918130in}{2.104099in}}%
\pgfpathcurveto{\pgfqpoint{2.926366in}{2.104099in}}{\pgfqpoint{2.934266in}{2.107371in}}{\pgfqpoint{2.940090in}{2.113195in}}%
\pgfpathcurveto{\pgfqpoint{2.945914in}{2.119019in}}{\pgfqpoint{2.949187in}{2.126919in}}{\pgfqpoint{2.949187in}{2.135155in}}%
\pgfpathcurveto{\pgfqpoint{2.949187in}{2.143392in}}{\pgfqpoint{2.945914in}{2.151292in}}{\pgfqpoint{2.940090in}{2.157115in}}%
\pgfpathcurveto{\pgfqpoint{2.934266in}{2.162939in}}{\pgfqpoint{2.926366in}{2.166212in}}{\pgfqpoint{2.918130in}{2.166212in}}%
\pgfpathcurveto{\pgfqpoint{2.909894in}{2.166212in}}{\pgfqpoint{2.901994in}{2.162939in}}{\pgfqpoint{2.896170in}{2.157115in}}%
\pgfpathcurveto{\pgfqpoint{2.890346in}{2.151292in}}{\pgfqpoint{2.887074in}{2.143392in}}{\pgfqpoint{2.887074in}{2.135155in}}%
\pgfpathcurveto{\pgfqpoint{2.887074in}{2.126919in}}{\pgfqpoint{2.890346in}{2.119019in}}{\pgfqpoint{2.896170in}{2.113195in}}%
\pgfpathcurveto{\pgfqpoint{2.901994in}{2.107371in}}{\pgfqpoint{2.909894in}{2.104099in}}{\pgfqpoint{2.918130in}{2.104099in}}%
\pgfpathclose%
\pgfusepath{stroke,fill}%
\end{pgfscope}%
\begin{pgfscope}%
\pgfpathrectangle{\pgfqpoint{0.100000in}{0.212622in}}{\pgfqpoint{3.696000in}{3.696000in}}%
\pgfusepath{clip}%
\pgfsetbuttcap%
\pgfsetroundjoin%
\definecolor{currentfill}{rgb}{0.121569,0.466667,0.705882}%
\pgfsetfillcolor{currentfill}%
\pgfsetfillopacity{0.759312}%
\pgfsetlinewidth{1.003750pt}%
\definecolor{currentstroke}{rgb}{0.121569,0.466667,0.705882}%
\pgfsetstrokecolor{currentstroke}%
\pgfsetstrokeopacity{0.759312}%
\pgfsetdash{}{0pt}%
\pgfpathmoveto{\pgfqpoint{2.916852in}{2.103131in}}%
\pgfpathcurveto{\pgfqpoint{2.925088in}{2.103131in}}{\pgfqpoint{2.932988in}{2.106403in}}{\pgfqpoint{2.938812in}{2.112227in}}%
\pgfpathcurveto{\pgfqpoint{2.944636in}{2.118051in}}{\pgfqpoint{2.947908in}{2.125951in}}{\pgfqpoint{2.947908in}{2.134187in}}%
\pgfpathcurveto{\pgfqpoint{2.947908in}{2.142424in}}{\pgfqpoint{2.944636in}{2.150324in}}{\pgfqpoint{2.938812in}{2.156148in}}%
\pgfpathcurveto{\pgfqpoint{2.932988in}{2.161972in}}{\pgfqpoint{2.925088in}{2.165244in}}{\pgfqpoint{2.916852in}{2.165244in}}%
\pgfpathcurveto{\pgfqpoint{2.908615in}{2.165244in}}{\pgfqpoint{2.900715in}{2.161972in}}{\pgfqpoint{2.894891in}{2.156148in}}%
\pgfpathcurveto{\pgfqpoint{2.889068in}{2.150324in}}{\pgfqpoint{2.885795in}{2.142424in}}{\pgfqpoint{2.885795in}{2.134187in}}%
\pgfpathcurveto{\pgfqpoint{2.885795in}{2.125951in}}{\pgfqpoint{2.889068in}{2.118051in}}{\pgfqpoint{2.894891in}{2.112227in}}%
\pgfpathcurveto{\pgfqpoint{2.900715in}{2.106403in}}{\pgfqpoint{2.908615in}{2.103131in}}{\pgfqpoint{2.916852in}{2.103131in}}%
\pgfpathclose%
\pgfusepath{stroke,fill}%
\end{pgfscope}%
\begin{pgfscope}%
\pgfpathrectangle{\pgfqpoint{0.100000in}{0.212622in}}{\pgfqpoint{3.696000in}{3.696000in}}%
\pgfusepath{clip}%
\pgfsetbuttcap%
\pgfsetroundjoin%
\definecolor{currentfill}{rgb}{0.121569,0.466667,0.705882}%
\pgfsetfillcolor{currentfill}%
\pgfsetfillopacity{0.759759}%
\pgfsetlinewidth{1.003750pt}%
\definecolor{currentstroke}{rgb}{0.121569,0.466667,0.705882}%
\pgfsetstrokecolor{currentstroke}%
\pgfsetstrokeopacity{0.759759}%
\pgfsetdash{}{0pt}%
\pgfpathmoveto{\pgfqpoint{2.916205in}{2.102604in}}%
\pgfpathcurveto{\pgfqpoint{2.924441in}{2.102604in}}{\pgfqpoint{2.932341in}{2.105876in}}{\pgfqpoint{2.938165in}{2.111700in}}%
\pgfpathcurveto{\pgfqpoint{2.943989in}{2.117524in}}{\pgfqpoint{2.947261in}{2.125424in}}{\pgfqpoint{2.947261in}{2.133661in}}%
\pgfpathcurveto{\pgfqpoint{2.947261in}{2.141897in}}{\pgfqpoint{2.943989in}{2.149797in}}{\pgfqpoint{2.938165in}{2.155621in}}%
\pgfpathcurveto{\pgfqpoint{2.932341in}{2.161445in}}{\pgfqpoint{2.924441in}{2.164717in}}{\pgfqpoint{2.916205in}{2.164717in}}%
\pgfpathcurveto{\pgfqpoint{2.907969in}{2.164717in}}{\pgfqpoint{2.900069in}{2.161445in}}{\pgfqpoint{2.894245in}{2.155621in}}%
\pgfpathcurveto{\pgfqpoint{2.888421in}{2.149797in}}{\pgfqpoint{2.885148in}{2.141897in}}{\pgfqpoint{2.885148in}{2.133661in}}%
\pgfpathcurveto{\pgfqpoint{2.885148in}{2.125424in}}{\pgfqpoint{2.888421in}{2.117524in}}{\pgfqpoint{2.894245in}{2.111700in}}%
\pgfpathcurveto{\pgfqpoint{2.900069in}{2.105876in}}{\pgfqpoint{2.907969in}{2.102604in}}{\pgfqpoint{2.916205in}{2.102604in}}%
\pgfpathclose%
\pgfusepath{stroke,fill}%
\end{pgfscope}%
\begin{pgfscope}%
\pgfpathrectangle{\pgfqpoint{0.100000in}{0.212622in}}{\pgfqpoint{3.696000in}{3.696000in}}%
\pgfusepath{clip}%
\pgfsetbuttcap%
\pgfsetroundjoin%
\definecolor{currentfill}{rgb}{0.121569,0.466667,0.705882}%
\pgfsetfillcolor{currentfill}%
\pgfsetfillopacity{0.760453}%
\pgfsetlinewidth{1.003750pt}%
\definecolor{currentstroke}{rgb}{0.121569,0.466667,0.705882}%
\pgfsetstrokecolor{currentstroke}%
\pgfsetstrokeopacity{0.760453}%
\pgfsetdash{}{0pt}%
\pgfpathmoveto{\pgfqpoint{2.915105in}{2.101828in}}%
\pgfpathcurveto{\pgfqpoint{2.923341in}{2.101828in}}{\pgfqpoint{2.931242in}{2.105101in}}{\pgfqpoint{2.937065in}{2.110925in}}%
\pgfpathcurveto{\pgfqpoint{2.942889in}{2.116749in}}{\pgfqpoint{2.946162in}{2.124649in}}{\pgfqpoint{2.946162in}{2.132885in}}%
\pgfpathcurveto{\pgfqpoint{2.946162in}{2.141121in}}{\pgfqpoint{2.942889in}{2.149021in}}{\pgfqpoint{2.937065in}{2.154845in}}%
\pgfpathcurveto{\pgfqpoint{2.931242in}{2.160669in}}{\pgfqpoint{2.923341in}{2.163941in}}{\pgfqpoint{2.915105in}{2.163941in}}%
\pgfpathcurveto{\pgfqpoint{2.906869in}{2.163941in}}{\pgfqpoint{2.898969in}{2.160669in}}{\pgfqpoint{2.893145in}{2.154845in}}%
\pgfpathcurveto{\pgfqpoint{2.887321in}{2.149021in}}{\pgfqpoint{2.884049in}{2.141121in}}{\pgfqpoint{2.884049in}{2.132885in}}%
\pgfpathcurveto{\pgfqpoint{2.884049in}{2.124649in}}{\pgfqpoint{2.887321in}{2.116749in}}{\pgfqpoint{2.893145in}{2.110925in}}%
\pgfpathcurveto{\pgfqpoint{2.898969in}{2.105101in}}{\pgfqpoint{2.906869in}{2.101828in}}{\pgfqpoint{2.915105in}{2.101828in}}%
\pgfpathclose%
\pgfusepath{stroke,fill}%
\end{pgfscope}%
\begin{pgfscope}%
\pgfpathrectangle{\pgfqpoint{0.100000in}{0.212622in}}{\pgfqpoint{3.696000in}{3.696000in}}%
\pgfusepath{clip}%
\pgfsetbuttcap%
\pgfsetroundjoin%
\definecolor{currentfill}{rgb}{0.121569,0.466667,0.705882}%
\pgfsetfillcolor{currentfill}%
\pgfsetfillopacity{0.761493}%
\pgfsetlinewidth{1.003750pt}%
\definecolor{currentstroke}{rgb}{0.121569,0.466667,0.705882}%
\pgfsetstrokecolor{currentstroke}%
\pgfsetstrokeopacity{0.761493}%
\pgfsetdash{}{0pt}%
\pgfpathmoveto{\pgfqpoint{2.913716in}{2.100749in}}%
\pgfpathcurveto{\pgfqpoint{2.921952in}{2.100749in}}{\pgfqpoint{2.929852in}{2.104021in}}{\pgfqpoint{2.935676in}{2.109845in}}%
\pgfpathcurveto{\pgfqpoint{2.941500in}{2.115669in}}{\pgfqpoint{2.944772in}{2.123569in}}{\pgfqpoint{2.944772in}{2.131805in}}%
\pgfpathcurveto{\pgfqpoint{2.944772in}{2.140041in}}{\pgfqpoint{2.941500in}{2.147941in}}{\pgfqpoint{2.935676in}{2.153765in}}%
\pgfpathcurveto{\pgfqpoint{2.929852in}{2.159589in}}{\pgfqpoint{2.921952in}{2.162862in}}{\pgfqpoint{2.913716in}{2.162862in}}%
\pgfpathcurveto{\pgfqpoint{2.905479in}{2.162862in}}{\pgfqpoint{2.897579in}{2.159589in}}{\pgfqpoint{2.891755in}{2.153765in}}%
\pgfpathcurveto{\pgfqpoint{2.885931in}{2.147941in}}{\pgfqpoint{2.882659in}{2.140041in}}{\pgfqpoint{2.882659in}{2.131805in}}%
\pgfpathcurveto{\pgfqpoint{2.882659in}{2.123569in}}{\pgfqpoint{2.885931in}{2.115669in}}{\pgfqpoint{2.891755in}{2.109845in}}%
\pgfpathcurveto{\pgfqpoint{2.897579in}{2.104021in}}{\pgfqpoint{2.905479in}{2.100749in}}{\pgfqpoint{2.913716in}{2.100749in}}%
\pgfpathclose%
\pgfusepath{stroke,fill}%
\end{pgfscope}%
\begin{pgfscope}%
\pgfpathrectangle{\pgfqpoint{0.100000in}{0.212622in}}{\pgfqpoint{3.696000in}{3.696000in}}%
\pgfusepath{clip}%
\pgfsetbuttcap%
\pgfsetroundjoin%
\definecolor{currentfill}{rgb}{0.121569,0.466667,0.705882}%
\pgfsetfillcolor{currentfill}%
\pgfsetfillopacity{0.762731}%
\pgfsetlinewidth{1.003750pt}%
\definecolor{currentstroke}{rgb}{0.121569,0.466667,0.705882}%
\pgfsetstrokecolor{currentstroke}%
\pgfsetstrokeopacity{0.762731}%
\pgfsetdash{}{0pt}%
\pgfpathmoveto{\pgfqpoint{2.912223in}{2.099741in}}%
\pgfpathcurveto{\pgfqpoint{2.920459in}{2.099741in}}{\pgfqpoint{2.928359in}{2.103014in}}{\pgfqpoint{2.934183in}{2.108837in}}%
\pgfpathcurveto{\pgfqpoint{2.940007in}{2.114661in}}{\pgfqpoint{2.943279in}{2.122561in}}{\pgfqpoint{2.943279in}{2.130798in}}%
\pgfpathcurveto{\pgfqpoint{2.943279in}{2.139034in}}{\pgfqpoint{2.940007in}{2.146934in}}{\pgfqpoint{2.934183in}{2.152758in}}%
\pgfpathcurveto{\pgfqpoint{2.928359in}{2.158582in}}{\pgfqpoint{2.920459in}{2.161854in}}{\pgfqpoint{2.912223in}{2.161854in}}%
\pgfpathcurveto{\pgfqpoint{2.903986in}{2.161854in}}{\pgfqpoint{2.896086in}{2.158582in}}{\pgfqpoint{2.890262in}{2.152758in}}%
\pgfpathcurveto{\pgfqpoint{2.884438in}{2.146934in}}{\pgfqpoint{2.881166in}{2.139034in}}{\pgfqpoint{2.881166in}{2.130798in}}%
\pgfpathcurveto{\pgfqpoint{2.881166in}{2.122561in}}{\pgfqpoint{2.884438in}{2.114661in}}{\pgfqpoint{2.890262in}{2.108837in}}%
\pgfpathcurveto{\pgfqpoint{2.896086in}{2.103014in}}{\pgfqpoint{2.903986in}{2.099741in}}{\pgfqpoint{2.912223in}{2.099741in}}%
\pgfpathclose%
\pgfusepath{stroke,fill}%
\end{pgfscope}%
\begin{pgfscope}%
\pgfpathrectangle{\pgfqpoint{0.100000in}{0.212622in}}{\pgfqpoint{3.696000in}{3.696000in}}%
\pgfusepath{clip}%
\pgfsetbuttcap%
\pgfsetroundjoin%
\definecolor{currentfill}{rgb}{0.121569,0.466667,0.705882}%
\pgfsetfillcolor{currentfill}%
\pgfsetfillopacity{0.764182}%
\pgfsetlinewidth{1.003750pt}%
\definecolor{currentstroke}{rgb}{0.121569,0.466667,0.705882}%
\pgfsetstrokecolor{currentstroke}%
\pgfsetstrokeopacity{0.764182}%
\pgfsetdash{}{0pt}%
\pgfpathmoveto{\pgfqpoint{2.910238in}{2.098444in}}%
\pgfpathcurveto{\pgfqpoint{2.918474in}{2.098444in}}{\pgfqpoint{2.926374in}{2.101717in}}{\pgfqpoint{2.932198in}{2.107541in}}%
\pgfpathcurveto{\pgfqpoint{2.938022in}{2.113365in}}{\pgfqpoint{2.941294in}{2.121265in}}{\pgfqpoint{2.941294in}{2.129501in}}%
\pgfpathcurveto{\pgfqpoint{2.941294in}{2.137737in}}{\pgfqpoint{2.938022in}{2.145637in}}{\pgfqpoint{2.932198in}{2.151461in}}%
\pgfpathcurveto{\pgfqpoint{2.926374in}{2.157285in}}{\pgfqpoint{2.918474in}{2.160557in}}{\pgfqpoint{2.910238in}{2.160557in}}%
\pgfpathcurveto{\pgfqpoint{2.902001in}{2.160557in}}{\pgfqpoint{2.894101in}{2.157285in}}{\pgfqpoint{2.888277in}{2.151461in}}%
\pgfpathcurveto{\pgfqpoint{2.882453in}{2.145637in}}{\pgfqpoint{2.879181in}{2.137737in}}{\pgfqpoint{2.879181in}{2.129501in}}%
\pgfpathcurveto{\pgfqpoint{2.879181in}{2.121265in}}{\pgfqpoint{2.882453in}{2.113365in}}{\pgfqpoint{2.888277in}{2.107541in}}%
\pgfpathcurveto{\pgfqpoint{2.894101in}{2.101717in}}{\pgfqpoint{2.902001in}{2.098444in}}{\pgfqpoint{2.910238in}{2.098444in}}%
\pgfpathclose%
\pgfusepath{stroke,fill}%
\end{pgfscope}%
\begin{pgfscope}%
\pgfpathrectangle{\pgfqpoint{0.100000in}{0.212622in}}{\pgfqpoint{3.696000in}{3.696000in}}%
\pgfusepath{clip}%
\pgfsetbuttcap%
\pgfsetroundjoin%
\definecolor{currentfill}{rgb}{0.121569,0.466667,0.705882}%
\pgfsetfillcolor{currentfill}%
\pgfsetfillopacity{0.766092}%
\pgfsetlinewidth{1.003750pt}%
\definecolor{currentstroke}{rgb}{0.121569,0.466667,0.705882}%
\pgfsetstrokecolor{currentstroke}%
\pgfsetstrokeopacity{0.766092}%
\pgfsetdash{}{0pt}%
\pgfpathmoveto{\pgfqpoint{2.907467in}{2.096090in}}%
\pgfpathcurveto{\pgfqpoint{2.915703in}{2.096090in}}{\pgfqpoint{2.923603in}{2.099362in}}{\pgfqpoint{2.929427in}{2.105186in}}%
\pgfpathcurveto{\pgfqpoint{2.935251in}{2.111010in}}{\pgfqpoint{2.938523in}{2.118910in}}{\pgfqpoint{2.938523in}{2.127146in}}%
\pgfpathcurveto{\pgfqpoint{2.938523in}{2.135383in}}{\pgfqpoint{2.935251in}{2.143283in}}{\pgfqpoint{2.929427in}{2.149107in}}%
\pgfpathcurveto{\pgfqpoint{2.923603in}{2.154931in}}{\pgfqpoint{2.915703in}{2.158203in}}{\pgfqpoint{2.907467in}{2.158203in}}%
\pgfpathcurveto{\pgfqpoint{2.899230in}{2.158203in}}{\pgfqpoint{2.891330in}{2.154931in}}{\pgfqpoint{2.885506in}{2.149107in}}%
\pgfpathcurveto{\pgfqpoint{2.879683in}{2.143283in}}{\pgfqpoint{2.876410in}{2.135383in}}{\pgfqpoint{2.876410in}{2.127146in}}%
\pgfpathcurveto{\pgfqpoint{2.876410in}{2.118910in}}{\pgfqpoint{2.879683in}{2.111010in}}{\pgfqpoint{2.885506in}{2.105186in}}%
\pgfpathcurveto{\pgfqpoint{2.891330in}{2.099362in}}{\pgfqpoint{2.899230in}{2.096090in}}{\pgfqpoint{2.907467in}{2.096090in}}%
\pgfpathclose%
\pgfusepath{stroke,fill}%
\end{pgfscope}%
\begin{pgfscope}%
\pgfpathrectangle{\pgfqpoint{0.100000in}{0.212622in}}{\pgfqpoint{3.696000in}{3.696000in}}%
\pgfusepath{clip}%
\pgfsetbuttcap%
\pgfsetroundjoin%
\definecolor{currentfill}{rgb}{0.121569,0.466667,0.705882}%
\pgfsetfillcolor{currentfill}%
\pgfsetfillopacity{0.767193}%
\pgfsetlinewidth{1.003750pt}%
\definecolor{currentstroke}{rgb}{0.121569,0.466667,0.705882}%
\pgfsetstrokecolor{currentstroke}%
\pgfsetstrokeopacity{0.767193}%
\pgfsetdash{}{0pt}%
\pgfpathmoveto{\pgfqpoint{2.906006in}{2.095062in}}%
\pgfpathcurveto{\pgfqpoint{2.914243in}{2.095062in}}{\pgfqpoint{2.922143in}{2.098334in}}{\pgfqpoint{2.927967in}{2.104158in}}%
\pgfpathcurveto{\pgfqpoint{2.933791in}{2.109982in}}{\pgfqpoint{2.937063in}{2.117882in}}{\pgfqpoint{2.937063in}{2.126119in}}%
\pgfpathcurveto{\pgfqpoint{2.937063in}{2.134355in}}{\pgfqpoint{2.933791in}{2.142255in}}{\pgfqpoint{2.927967in}{2.148079in}}%
\pgfpathcurveto{\pgfqpoint{2.922143in}{2.153903in}}{\pgfqpoint{2.914243in}{2.157175in}}{\pgfqpoint{2.906006in}{2.157175in}}%
\pgfpathcurveto{\pgfqpoint{2.897770in}{2.157175in}}{\pgfqpoint{2.889870in}{2.153903in}}{\pgfqpoint{2.884046in}{2.148079in}}%
\pgfpathcurveto{\pgfqpoint{2.878222in}{2.142255in}}{\pgfqpoint{2.874950in}{2.134355in}}{\pgfqpoint{2.874950in}{2.126119in}}%
\pgfpathcurveto{\pgfqpoint{2.874950in}{2.117882in}}{\pgfqpoint{2.878222in}{2.109982in}}{\pgfqpoint{2.884046in}{2.104158in}}%
\pgfpathcurveto{\pgfqpoint{2.889870in}{2.098334in}}{\pgfqpoint{2.897770in}{2.095062in}}{\pgfqpoint{2.906006in}{2.095062in}}%
\pgfpathclose%
\pgfusepath{stroke,fill}%
\end{pgfscope}%
\begin{pgfscope}%
\pgfpathrectangle{\pgfqpoint{0.100000in}{0.212622in}}{\pgfqpoint{3.696000in}{3.696000in}}%
\pgfusepath{clip}%
\pgfsetbuttcap%
\pgfsetroundjoin%
\definecolor{currentfill}{rgb}{0.121569,0.466667,0.705882}%
\pgfsetfillcolor{currentfill}%
\pgfsetfillopacity{0.767782}%
\pgfsetlinewidth{1.003750pt}%
\definecolor{currentstroke}{rgb}{0.121569,0.466667,0.705882}%
\pgfsetstrokecolor{currentstroke}%
\pgfsetstrokeopacity{0.767782}%
\pgfsetdash{}{0pt}%
\pgfpathmoveto{\pgfqpoint{2.905122in}{2.094456in}}%
\pgfpathcurveto{\pgfqpoint{2.913358in}{2.094456in}}{\pgfqpoint{2.921259in}{2.097728in}}{\pgfqpoint{2.927082in}{2.103552in}}%
\pgfpathcurveto{\pgfqpoint{2.932906in}{2.109376in}}{\pgfqpoint{2.936179in}{2.117276in}}{\pgfqpoint{2.936179in}{2.125512in}}%
\pgfpathcurveto{\pgfqpoint{2.936179in}{2.133748in}}{\pgfqpoint{2.932906in}{2.141648in}}{\pgfqpoint{2.927082in}{2.147472in}}%
\pgfpathcurveto{\pgfqpoint{2.921259in}{2.153296in}}{\pgfqpoint{2.913358in}{2.156569in}}{\pgfqpoint{2.905122in}{2.156569in}}%
\pgfpathcurveto{\pgfqpoint{2.896886in}{2.156569in}}{\pgfqpoint{2.888986in}{2.153296in}}{\pgfqpoint{2.883162in}{2.147472in}}%
\pgfpathcurveto{\pgfqpoint{2.877338in}{2.141648in}}{\pgfqpoint{2.874066in}{2.133748in}}{\pgfqpoint{2.874066in}{2.125512in}}%
\pgfpathcurveto{\pgfqpoint{2.874066in}{2.117276in}}{\pgfqpoint{2.877338in}{2.109376in}}{\pgfqpoint{2.883162in}{2.103552in}}%
\pgfpathcurveto{\pgfqpoint{2.888986in}{2.097728in}}{\pgfqpoint{2.896886in}{2.094456in}}{\pgfqpoint{2.905122in}{2.094456in}}%
\pgfpathclose%
\pgfusepath{stroke,fill}%
\end{pgfscope}%
\begin{pgfscope}%
\pgfpathrectangle{\pgfqpoint{0.100000in}{0.212622in}}{\pgfqpoint{3.696000in}{3.696000in}}%
\pgfusepath{clip}%
\pgfsetbuttcap%
\pgfsetroundjoin%
\definecolor{currentfill}{rgb}{0.121569,0.466667,0.705882}%
\pgfsetfillcolor{currentfill}%
\pgfsetfillopacity{0.768955}%
\pgfsetlinewidth{1.003750pt}%
\definecolor{currentstroke}{rgb}{0.121569,0.466667,0.705882}%
\pgfsetstrokecolor{currentstroke}%
\pgfsetstrokeopacity{0.768955}%
\pgfsetdash{}{0pt}%
\pgfpathmoveto{\pgfqpoint{2.903308in}{2.093251in}}%
\pgfpathcurveto{\pgfqpoint{2.911544in}{2.093251in}}{\pgfqpoint{2.919444in}{2.096523in}}{\pgfqpoint{2.925268in}{2.102347in}}%
\pgfpathcurveto{\pgfqpoint{2.931092in}{2.108171in}}{\pgfqpoint{2.934364in}{2.116071in}}{\pgfqpoint{2.934364in}{2.124307in}}%
\pgfpathcurveto{\pgfqpoint{2.934364in}{2.132543in}}{\pgfqpoint{2.931092in}{2.140443in}}{\pgfqpoint{2.925268in}{2.146267in}}%
\pgfpathcurveto{\pgfqpoint{2.919444in}{2.152091in}}{\pgfqpoint{2.911544in}{2.155364in}}{\pgfqpoint{2.903308in}{2.155364in}}%
\pgfpathcurveto{\pgfqpoint{2.895071in}{2.155364in}}{\pgfqpoint{2.887171in}{2.152091in}}{\pgfqpoint{2.881347in}{2.146267in}}%
\pgfpathcurveto{\pgfqpoint{2.875523in}{2.140443in}}{\pgfqpoint{2.872251in}{2.132543in}}{\pgfqpoint{2.872251in}{2.124307in}}%
\pgfpathcurveto{\pgfqpoint{2.872251in}{2.116071in}}{\pgfqpoint{2.875523in}{2.108171in}}{\pgfqpoint{2.881347in}{2.102347in}}%
\pgfpathcurveto{\pgfqpoint{2.887171in}{2.096523in}}{\pgfqpoint{2.895071in}{2.093251in}}{\pgfqpoint{2.903308in}{2.093251in}}%
\pgfpathclose%
\pgfusepath{stroke,fill}%
\end{pgfscope}%
\begin{pgfscope}%
\pgfpathrectangle{\pgfqpoint{0.100000in}{0.212622in}}{\pgfqpoint{3.696000in}{3.696000in}}%
\pgfusepath{clip}%
\pgfsetbuttcap%
\pgfsetroundjoin%
\definecolor{currentfill}{rgb}{0.121569,0.466667,0.705882}%
\pgfsetfillcolor{currentfill}%
\pgfsetfillopacity{0.769593}%
\pgfsetlinewidth{1.003750pt}%
\definecolor{currentstroke}{rgb}{0.121569,0.466667,0.705882}%
\pgfsetstrokecolor{currentstroke}%
\pgfsetstrokeopacity{0.769593}%
\pgfsetdash{}{0pt}%
\pgfpathmoveto{\pgfqpoint{2.902373in}{2.092479in}}%
\pgfpathcurveto{\pgfqpoint{2.910609in}{2.092479in}}{\pgfqpoint{2.918509in}{2.095751in}}{\pgfqpoint{2.924333in}{2.101575in}}%
\pgfpathcurveto{\pgfqpoint{2.930157in}{2.107399in}}{\pgfqpoint{2.933430in}{2.115299in}}{\pgfqpoint{2.933430in}{2.123535in}}%
\pgfpathcurveto{\pgfqpoint{2.933430in}{2.131771in}}{\pgfqpoint{2.930157in}{2.139672in}}{\pgfqpoint{2.924333in}{2.145495in}}%
\pgfpathcurveto{\pgfqpoint{2.918509in}{2.151319in}}{\pgfqpoint{2.910609in}{2.154592in}}{\pgfqpoint{2.902373in}{2.154592in}}%
\pgfpathcurveto{\pgfqpoint{2.894137in}{2.154592in}}{\pgfqpoint{2.886237in}{2.151319in}}{\pgfqpoint{2.880413in}{2.145495in}}%
\pgfpathcurveto{\pgfqpoint{2.874589in}{2.139672in}}{\pgfqpoint{2.871317in}{2.131771in}}{\pgfqpoint{2.871317in}{2.123535in}}%
\pgfpathcurveto{\pgfqpoint{2.871317in}{2.115299in}}{\pgfqpoint{2.874589in}{2.107399in}}{\pgfqpoint{2.880413in}{2.101575in}}%
\pgfpathcurveto{\pgfqpoint{2.886237in}{2.095751in}}{\pgfqpoint{2.894137in}{2.092479in}}{\pgfqpoint{2.902373in}{2.092479in}}%
\pgfpathclose%
\pgfusepath{stroke,fill}%
\end{pgfscope}%
\begin{pgfscope}%
\pgfpathrectangle{\pgfqpoint{0.100000in}{0.212622in}}{\pgfqpoint{3.696000in}{3.696000in}}%
\pgfusepath{clip}%
\pgfsetbuttcap%
\pgfsetroundjoin%
\definecolor{currentfill}{rgb}{0.121569,0.466667,0.705882}%
\pgfsetfillcolor{currentfill}%
\pgfsetfillopacity{0.769954}%
\pgfsetlinewidth{1.003750pt}%
\definecolor{currentstroke}{rgb}{0.121569,0.466667,0.705882}%
\pgfsetstrokecolor{currentstroke}%
\pgfsetstrokeopacity{0.769954}%
\pgfsetdash{}{0pt}%
\pgfpathmoveto{\pgfqpoint{2.901789in}{2.092182in}}%
\pgfpathcurveto{\pgfqpoint{2.910025in}{2.092182in}}{\pgfqpoint{2.917925in}{2.095454in}}{\pgfqpoint{2.923749in}{2.101278in}}%
\pgfpathcurveto{\pgfqpoint{2.929573in}{2.107102in}}{\pgfqpoint{2.932845in}{2.115002in}}{\pgfqpoint{2.932845in}{2.123239in}}%
\pgfpathcurveto{\pgfqpoint{2.932845in}{2.131475in}}{\pgfqpoint{2.929573in}{2.139375in}}{\pgfqpoint{2.923749in}{2.145199in}}%
\pgfpathcurveto{\pgfqpoint{2.917925in}{2.151023in}}{\pgfqpoint{2.910025in}{2.154295in}}{\pgfqpoint{2.901789in}{2.154295in}}%
\pgfpathcurveto{\pgfqpoint{2.893553in}{2.154295in}}{\pgfqpoint{2.885653in}{2.151023in}}{\pgfqpoint{2.879829in}{2.145199in}}%
\pgfpathcurveto{\pgfqpoint{2.874005in}{2.139375in}}{\pgfqpoint{2.870732in}{2.131475in}}{\pgfqpoint{2.870732in}{2.123239in}}%
\pgfpathcurveto{\pgfqpoint{2.870732in}{2.115002in}}{\pgfqpoint{2.874005in}{2.107102in}}{\pgfqpoint{2.879829in}{2.101278in}}%
\pgfpathcurveto{\pgfqpoint{2.885653in}{2.095454in}}{\pgfqpoint{2.893553in}{2.092182in}}{\pgfqpoint{2.901789in}{2.092182in}}%
\pgfpathclose%
\pgfusepath{stroke,fill}%
\end{pgfscope}%
\begin{pgfscope}%
\pgfpathrectangle{\pgfqpoint{0.100000in}{0.212622in}}{\pgfqpoint{3.696000in}{3.696000in}}%
\pgfusepath{clip}%
\pgfsetbuttcap%
\pgfsetroundjoin%
\definecolor{currentfill}{rgb}{0.121569,0.466667,0.705882}%
\pgfsetfillcolor{currentfill}%
\pgfsetfillopacity{0.770983}%
\pgfsetlinewidth{1.003750pt}%
\definecolor{currentstroke}{rgb}{0.121569,0.466667,0.705882}%
\pgfsetstrokecolor{currentstroke}%
\pgfsetstrokeopacity{0.770983}%
\pgfsetdash{}{0pt}%
\pgfpathmoveto{\pgfqpoint{2.899921in}{2.091150in}}%
\pgfpathcurveto{\pgfqpoint{2.908157in}{2.091150in}}{\pgfqpoint{2.916057in}{2.094423in}}{\pgfqpoint{2.921881in}{2.100247in}}%
\pgfpathcurveto{\pgfqpoint{2.927705in}{2.106071in}}{\pgfqpoint{2.930978in}{2.113971in}}{\pgfqpoint{2.930978in}{2.122207in}}%
\pgfpathcurveto{\pgfqpoint{2.930978in}{2.130443in}}{\pgfqpoint{2.927705in}{2.138343in}}{\pgfqpoint{2.921881in}{2.144167in}}%
\pgfpathcurveto{\pgfqpoint{2.916057in}{2.149991in}}{\pgfqpoint{2.908157in}{2.153263in}}{\pgfqpoint{2.899921in}{2.153263in}}%
\pgfpathcurveto{\pgfqpoint{2.891685in}{2.153263in}}{\pgfqpoint{2.883785in}{2.149991in}}{\pgfqpoint{2.877961in}{2.144167in}}%
\pgfpathcurveto{\pgfqpoint{2.872137in}{2.138343in}}{\pgfqpoint{2.868865in}{2.130443in}}{\pgfqpoint{2.868865in}{2.122207in}}%
\pgfpathcurveto{\pgfqpoint{2.868865in}{2.113971in}}{\pgfqpoint{2.872137in}{2.106071in}}{\pgfqpoint{2.877961in}{2.100247in}}%
\pgfpathcurveto{\pgfqpoint{2.883785in}{2.094423in}}{\pgfqpoint{2.891685in}{2.091150in}}{\pgfqpoint{2.899921in}{2.091150in}}%
\pgfpathclose%
\pgfusepath{stroke,fill}%
\end{pgfscope}%
\begin{pgfscope}%
\pgfpathrectangle{\pgfqpoint{0.100000in}{0.212622in}}{\pgfqpoint{3.696000in}{3.696000in}}%
\pgfusepath{clip}%
\pgfsetbuttcap%
\pgfsetroundjoin%
\definecolor{currentfill}{rgb}{0.121569,0.466667,0.705882}%
\pgfsetfillcolor{currentfill}%
\pgfsetfillopacity{0.772508}%
\pgfsetlinewidth{1.003750pt}%
\definecolor{currentstroke}{rgb}{0.121569,0.466667,0.705882}%
\pgfsetstrokecolor{currentstroke}%
\pgfsetstrokeopacity{0.772508}%
\pgfsetdash{}{0pt}%
\pgfpathmoveto{\pgfqpoint{2.897246in}{2.089159in}}%
\pgfpathcurveto{\pgfqpoint{2.905482in}{2.089159in}}{\pgfqpoint{2.913382in}{2.092431in}}{\pgfqpoint{2.919206in}{2.098255in}}%
\pgfpathcurveto{\pgfqpoint{2.925030in}{2.104079in}}{\pgfqpoint{2.928302in}{2.111979in}}{\pgfqpoint{2.928302in}{2.120215in}}%
\pgfpathcurveto{\pgfqpoint{2.928302in}{2.128451in}}{\pgfqpoint{2.925030in}{2.136352in}}{\pgfqpoint{2.919206in}{2.142175in}}%
\pgfpathcurveto{\pgfqpoint{2.913382in}{2.147999in}}{\pgfqpoint{2.905482in}{2.151272in}}{\pgfqpoint{2.897246in}{2.151272in}}%
\pgfpathcurveto{\pgfqpoint{2.889009in}{2.151272in}}{\pgfqpoint{2.881109in}{2.147999in}}{\pgfqpoint{2.875285in}{2.142175in}}%
\pgfpathcurveto{\pgfqpoint{2.869461in}{2.136352in}}{\pgfqpoint{2.866189in}{2.128451in}}{\pgfqpoint{2.866189in}{2.120215in}}%
\pgfpathcurveto{\pgfqpoint{2.866189in}{2.111979in}}{\pgfqpoint{2.869461in}{2.104079in}}{\pgfqpoint{2.875285in}{2.098255in}}%
\pgfpathcurveto{\pgfqpoint{2.881109in}{2.092431in}}{\pgfqpoint{2.889009in}{2.089159in}}{\pgfqpoint{2.897246in}{2.089159in}}%
\pgfpathclose%
\pgfusepath{stroke,fill}%
\end{pgfscope}%
\begin{pgfscope}%
\pgfpathrectangle{\pgfqpoint{0.100000in}{0.212622in}}{\pgfqpoint{3.696000in}{3.696000in}}%
\pgfusepath{clip}%
\pgfsetbuttcap%
\pgfsetroundjoin%
\definecolor{currentfill}{rgb}{0.121569,0.466667,0.705882}%
\pgfsetfillcolor{currentfill}%
\pgfsetfillopacity{0.774810}%
\pgfsetlinewidth{1.003750pt}%
\definecolor{currentstroke}{rgb}{0.121569,0.466667,0.705882}%
\pgfsetstrokecolor{currentstroke}%
\pgfsetstrokeopacity{0.774810}%
\pgfsetdash{}{0pt}%
\pgfpathmoveto{\pgfqpoint{2.893312in}{2.086165in}}%
\pgfpathcurveto{\pgfqpoint{2.901549in}{2.086165in}}{\pgfqpoint{2.909449in}{2.089437in}}{\pgfqpoint{2.915273in}{2.095261in}}%
\pgfpathcurveto{\pgfqpoint{2.921097in}{2.101085in}}{\pgfqpoint{2.924369in}{2.108985in}}{\pgfqpoint{2.924369in}{2.117221in}}%
\pgfpathcurveto{\pgfqpoint{2.924369in}{2.125457in}}{\pgfqpoint{2.921097in}{2.133357in}}{\pgfqpoint{2.915273in}{2.139181in}}%
\pgfpathcurveto{\pgfqpoint{2.909449in}{2.145005in}}{\pgfqpoint{2.901549in}{2.148278in}}{\pgfqpoint{2.893312in}{2.148278in}}%
\pgfpathcurveto{\pgfqpoint{2.885076in}{2.148278in}}{\pgfqpoint{2.877176in}{2.145005in}}{\pgfqpoint{2.871352in}{2.139181in}}%
\pgfpathcurveto{\pgfqpoint{2.865528in}{2.133357in}}{\pgfqpoint{2.862256in}{2.125457in}}{\pgfqpoint{2.862256in}{2.117221in}}%
\pgfpathcurveto{\pgfqpoint{2.862256in}{2.108985in}}{\pgfqpoint{2.865528in}{2.101085in}}{\pgfqpoint{2.871352in}{2.095261in}}%
\pgfpathcurveto{\pgfqpoint{2.877176in}{2.089437in}}{\pgfqpoint{2.885076in}{2.086165in}}{\pgfqpoint{2.893312in}{2.086165in}}%
\pgfpathclose%
\pgfusepath{stroke,fill}%
\end{pgfscope}%
\begin{pgfscope}%
\pgfpathrectangle{\pgfqpoint{0.100000in}{0.212622in}}{\pgfqpoint{3.696000in}{3.696000in}}%
\pgfusepath{clip}%
\pgfsetbuttcap%
\pgfsetroundjoin%
\definecolor{currentfill}{rgb}{0.121569,0.466667,0.705882}%
\pgfsetfillcolor{currentfill}%
\pgfsetfillopacity{0.776027}%
\pgfsetlinewidth{1.003750pt}%
\definecolor{currentstroke}{rgb}{0.121569,0.466667,0.705882}%
\pgfsetstrokecolor{currentstroke}%
\pgfsetstrokeopacity{0.776027}%
\pgfsetdash{}{0pt}%
\pgfpathmoveto{\pgfqpoint{2.890974in}{2.084369in}}%
\pgfpathcurveto{\pgfqpoint{2.899211in}{2.084369in}}{\pgfqpoint{2.907111in}{2.087641in}}{\pgfqpoint{2.912935in}{2.093465in}}%
\pgfpathcurveto{\pgfqpoint{2.918759in}{2.099289in}}{\pgfqpoint{2.922031in}{2.107189in}}{\pgfqpoint{2.922031in}{2.115426in}}%
\pgfpathcurveto{\pgfqpoint{2.922031in}{2.123662in}}{\pgfqpoint{2.918759in}{2.131562in}}{\pgfqpoint{2.912935in}{2.137386in}}%
\pgfpathcurveto{\pgfqpoint{2.907111in}{2.143210in}}{\pgfqpoint{2.899211in}{2.146482in}}{\pgfqpoint{2.890974in}{2.146482in}}%
\pgfpathcurveto{\pgfqpoint{2.882738in}{2.146482in}}{\pgfqpoint{2.874838in}{2.143210in}}{\pgfqpoint{2.869014in}{2.137386in}}%
\pgfpathcurveto{\pgfqpoint{2.863190in}{2.131562in}}{\pgfqpoint{2.859918in}{2.123662in}}{\pgfqpoint{2.859918in}{2.115426in}}%
\pgfpathcurveto{\pgfqpoint{2.859918in}{2.107189in}}{\pgfqpoint{2.863190in}{2.099289in}}{\pgfqpoint{2.869014in}{2.093465in}}%
\pgfpathcurveto{\pgfqpoint{2.874838in}{2.087641in}}{\pgfqpoint{2.882738in}{2.084369in}}{\pgfqpoint{2.890974in}{2.084369in}}%
\pgfpathclose%
\pgfusepath{stroke,fill}%
\end{pgfscope}%
\begin{pgfscope}%
\pgfpathrectangle{\pgfqpoint{0.100000in}{0.212622in}}{\pgfqpoint{3.696000in}{3.696000in}}%
\pgfusepath{clip}%
\pgfsetbuttcap%
\pgfsetroundjoin%
\definecolor{currentfill}{rgb}{0.121569,0.466667,0.705882}%
\pgfsetfillcolor{currentfill}%
\pgfsetfillopacity{0.778198}%
\pgfsetlinewidth{1.003750pt}%
\definecolor{currentstroke}{rgb}{0.121569,0.466667,0.705882}%
\pgfsetstrokecolor{currentstroke}%
\pgfsetstrokeopacity{0.778198}%
\pgfsetdash{}{0pt}%
\pgfpathmoveto{\pgfqpoint{2.887233in}{2.082087in}}%
\pgfpathcurveto{\pgfqpoint{2.895470in}{2.082087in}}{\pgfqpoint{2.903370in}{2.085360in}}{\pgfqpoint{2.909194in}{2.091184in}}%
\pgfpathcurveto{\pgfqpoint{2.915018in}{2.097008in}}{\pgfqpoint{2.918290in}{2.104908in}}{\pgfqpoint{2.918290in}{2.113144in}}%
\pgfpathcurveto{\pgfqpoint{2.918290in}{2.121380in}}{\pgfqpoint{2.915018in}{2.129280in}}{\pgfqpoint{2.909194in}{2.135104in}}%
\pgfpathcurveto{\pgfqpoint{2.903370in}{2.140928in}}{\pgfqpoint{2.895470in}{2.144200in}}{\pgfqpoint{2.887233in}{2.144200in}}%
\pgfpathcurveto{\pgfqpoint{2.878997in}{2.144200in}}{\pgfqpoint{2.871097in}{2.140928in}}{\pgfqpoint{2.865273in}{2.135104in}}%
\pgfpathcurveto{\pgfqpoint{2.859449in}{2.129280in}}{\pgfqpoint{2.856177in}{2.121380in}}{\pgfqpoint{2.856177in}{2.113144in}}%
\pgfpathcurveto{\pgfqpoint{2.856177in}{2.104908in}}{\pgfqpoint{2.859449in}{2.097008in}}{\pgfqpoint{2.865273in}{2.091184in}}%
\pgfpathcurveto{\pgfqpoint{2.871097in}{2.085360in}}{\pgfqpoint{2.878997in}{2.082087in}}{\pgfqpoint{2.887233in}{2.082087in}}%
\pgfpathclose%
\pgfusepath{stroke,fill}%
\end{pgfscope}%
\begin{pgfscope}%
\pgfpathrectangle{\pgfqpoint{0.100000in}{0.212622in}}{\pgfqpoint{3.696000in}{3.696000in}}%
\pgfusepath{clip}%
\pgfsetbuttcap%
\pgfsetroundjoin%
\definecolor{currentfill}{rgb}{0.121569,0.466667,0.705882}%
\pgfsetfillcolor{currentfill}%
\pgfsetfillopacity{0.780801}%
\pgfsetlinewidth{1.003750pt}%
\definecolor{currentstroke}{rgb}{0.121569,0.466667,0.705882}%
\pgfsetstrokecolor{currentstroke}%
\pgfsetstrokeopacity{0.780801}%
\pgfsetdash{}{0pt}%
\pgfpathmoveto{\pgfqpoint{2.883149in}{2.079941in}}%
\pgfpathcurveto{\pgfqpoint{2.891385in}{2.079941in}}{\pgfqpoint{2.899285in}{2.083214in}}{\pgfqpoint{2.905109in}{2.089037in}}%
\pgfpathcurveto{\pgfqpoint{2.910933in}{2.094861in}}{\pgfqpoint{2.914206in}{2.102761in}}{\pgfqpoint{2.914206in}{2.110998in}}%
\pgfpathcurveto{\pgfqpoint{2.914206in}{2.119234in}}{\pgfqpoint{2.910933in}{2.127134in}}{\pgfqpoint{2.905109in}{2.132958in}}%
\pgfpathcurveto{\pgfqpoint{2.899285in}{2.138782in}}{\pgfqpoint{2.891385in}{2.142054in}}{\pgfqpoint{2.883149in}{2.142054in}}%
\pgfpathcurveto{\pgfqpoint{2.874913in}{2.142054in}}{\pgfqpoint{2.867013in}{2.138782in}}{\pgfqpoint{2.861189in}{2.132958in}}%
\pgfpathcurveto{\pgfqpoint{2.855365in}{2.127134in}}{\pgfqpoint{2.852093in}{2.119234in}}{\pgfqpoint{2.852093in}{2.110998in}}%
\pgfpathcurveto{\pgfqpoint{2.852093in}{2.102761in}}{\pgfqpoint{2.855365in}{2.094861in}}{\pgfqpoint{2.861189in}{2.089037in}}%
\pgfpathcurveto{\pgfqpoint{2.867013in}{2.083214in}}{\pgfqpoint{2.874913in}{2.079941in}}{\pgfqpoint{2.883149in}{2.079941in}}%
\pgfpathclose%
\pgfusepath{stroke,fill}%
\end{pgfscope}%
\begin{pgfscope}%
\pgfpathrectangle{\pgfqpoint{0.100000in}{0.212622in}}{\pgfqpoint{3.696000in}{3.696000in}}%
\pgfusepath{clip}%
\pgfsetbuttcap%
\pgfsetroundjoin%
\definecolor{currentfill}{rgb}{0.121569,0.466667,0.705882}%
\pgfsetfillcolor{currentfill}%
\pgfsetfillopacity{0.782172}%
\pgfsetlinewidth{1.003750pt}%
\definecolor{currentstroke}{rgb}{0.121569,0.466667,0.705882}%
\pgfsetstrokecolor{currentstroke}%
\pgfsetstrokeopacity{0.782172}%
\pgfsetdash{}{0pt}%
\pgfpathmoveto{\pgfqpoint{2.880672in}{2.078593in}}%
\pgfpathcurveto{\pgfqpoint{2.888908in}{2.078593in}}{\pgfqpoint{2.896808in}{2.081865in}}{\pgfqpoint{2.902632in}{2.087689in}}%
\pgfpathcurveto{\pgfqpoint{2.908456in}{2.093513in}}{\pgfqpoint{2.911728in}{2.101413in}}{\pgfqpoint{2.911728in}{2.109650in}}%
\pgfpathcurveto{\pgfqpoint{2.911728in}{2.117886in}}{\pgfqpoint{2.908456in}{2.125786in}}{\pgfqpoint{2.902632in}{2.131610in}}%
\pgfpathcurveto{\pgfqpoint{2.896808in}{2.137434in}}{\pgfqpoint{2.888908in}{2.140706in}}{\pgfqpoint{2.880672in}{2.140706in}}%
\pgfpathcurveto{\pgfqpoint{2.872435in}{2.140706in}}{\pgfqpoint{2.864535in}{2.137434in}}{\pgfqpoint{2.858711in}{2.131610in}}%
\pgfpathcurveto{\pgfqpoint{2.852888in}{2.125786in}}{\pgfqpoint{2.849615in}{2.117886in}}{\pgfqpoint{2.849615in}{2.109650in}}%
\pgfpathcurveto{\pgfqpoint{2.849615in}{2.101413in}}{\pgfqpoint{2.852888in}{2.093513in}}{\pgfqpoint{2.858711in}{2.087689in}}%
\pgfpathcurveto{\pgfqpoint{2.864535in}{2.081865in}}{\pgfqpoint{2.872435in}{2.078593in}}{\pgfqpoint{2.880672in}{2.078593in}}%
\pgfpathclose%
\pgfusepath{stroke,fill}%
\end{pgfscope}%
\begin{pgfscope}%
\pgfpathrectangle{\pgfqpoint{0.100000in}{0.212622in}}{\pgfqpoint{3.696000in}{3.696000in}}%
\pgfusepath{clip}%
\pgfsetbuttcap%
\pgfsetroundjoin%
\definecolor{currentfill}{rgb}{0.121569,0.466667,0.705882}%
\pgfsetfillcolor{currentfill}%
\pgfsetfillopacity{0.784571}%
\pgfsetlinewidth{1.003750pt}%
\definecolor{currentstroke}{rgb}{0.121569,0.466667,0.705882}%
\pgfsetstrokecolor{currentstroke}%
\pgfsetstrokeopacity{0.784571}%
\pgfsetdash{}{0pt}%
\pgfpathmoveto{\pgfqpoint{2.875569in}{2.074331in}}%
\pgfpathcurveto{\pgfqpoint{2.883805in}{2.074331in}}{\pgfqpoint{2.891705in}{2.077603in}}{\pgfqpoint{2.897529in}{2.083427in}}%
\pgfpathcurveto{\pgfqpoint{2.903353in}{2.089251in}}{\pgfqpoint{2.906625in}{2.097151in}}{\pgfqpoint{2.906625in}{2.105388in}}%
\pgfpathcurveto{\pgfqpoint{2.906625in}{2.113624in}}{\pgfqpoint{2.903353in}{2.121524in}}{\pgfqpoint{2.897529in}{2.127348in}}%
\pgfpathcurveto{\pgfqpoint{2.891705in}{2.133172in}}{\pgfqpoint{2.883805in}{2.136444in}}{\pgfqpoint{2.875569in}{2.136444in}}%
\pgfpathcurveto{\pgfqpoint{2.867332in}{2.136444in}}{\pgfqpoint{2.859432in}{2.133172in}}{\pgfqpoint{2.853608in}{2.127348in}}%
\pgfpathcurveto{\pgfqpoint{2.847784in}{2.121524in}}{\pgfqpoint{2.844512in}{2.113624in}}{\pgfqpoint{2.844512in}{2.105388in}}%
\pgfpathcurveto{\pgfqpoint{2.844512in}{2.097151in}}{\pgfqpoint{2.847784in}{2.089251in}}{\pgfqpoint{2.853608in}{2.083427in}}%
\pgfpathcurveto{\pgfqpoint{2.859432in}{2.077603in}}{\pgfqpoint{2.867332in}{2.074331in}}{\pgfqpoint{2.875569in}{2.074331in}}%
\pgfpathclose%
\pgfusepath{stroke,fill}%
\end{pgfscope}%
\begin{pgfscope}%
\pgfpathrectangle{\pgfqpoint{0.100000in}{0.212622in}}{\pgfqpoint{3.696000in}{3.696000in}}%
\pgfusepath{clip}%
\pgfsetbuttcap%
\pgfsetroundjoin%
\definecolor{currentfill}{rgb}{0.121569,0.466667,0.705882}%
\pgfsetfillcolor{currentfill}%
\pgfsetfillopacity{0.785956}%
\pgfsetlinewidth{1.003750pt}%
\definecolor{currentstroke}{rgb}{0.121569,0.466667,0.705882}%
\pgfsetstrokecolor{currentstroke}%
\pgfsetstrokeopacity{0.785956}%
\pgfsetdash{}{0pt}%
\pgfpathmoveto{\pgfqpoint{2.872771in}{2.072389in}}%
\pgfpathcurveto{\pgfqpoint{2.881008in}{2.072389in}}{\pgfqpoint{2.888908in}{2.075661in}}{\pgfqpoint{2.894732in}{2.081485in}}%
\pgfpathcurveto{\pgfqpoint{2.900556in}{2.087309in}}{\pgfqpoint{2.903828in}{2.095209in}}{\pgfqpoint{2.903828in}{2.103445in}}%
\pgfpathcurveto{\pgfqpoint{2.903828in}{2.111682in}}{\pgfqpoint{2.900556in}{2.119582in}}{\pgfqpoint{2.894732in}{2.125406in}}%
\pgfpathcurveto{\pgfqpoint{2.888908in}{2.131229in}}{\pgfqpoint{2.881008in}{2.134502in}}{\pgfqpoint{2.872771in}{2.134502in}}%
\pgfpathcurveto{\pgfqpoint{2.864535in}{2.134502in}}{\pgfqpoint{2.856635in}{2.131229in}}{\pgfqpoint{2.850811in}{2.125406in}}%
\pgfpathcurveto{\pgfqpoint{2.844987in}{2.119582in}}{\pgfqpoint{2.841715in}{2.111682in}}{\pgfqpoint{2.841715in}{2.103445in}}%
\pgfpathcurveto{\pgfqpoint{2.841715in}{2.095209in}}{\pgfqpoint{2.844987in}{2.087309in}}{\pgfqpoint{2.850811in}{2.081485in}}%
\pgfpathcurveto{\pgfqpoint{2.856635in}{2.075661in}}{\pgfqpoint{2.864535in}{2.072389in}}{\pgfqpoint{2.872771in}{2.072389in}}%
\pgfpathclose%
\pgfusepath{stroke,fill}%
\end{pgfscope}%
\begin{pgfscope}%
\pgfpathrectangle{\pgfqpoint{0.100000in}{0.212622in}}{\pgfqpoint{3.696000in}{3.696000in}}%
\pgfusepath{clip}%
\pgfsetbuttcap%
\pgfsetroundjoin%
\definecolor{currentfill}{rgb}{0.121569,0.466667,0.705882}%
\pgfsetfillcolor{currentfill}%
\pgfsetfillopacity{0.787442}%
\pgfsetlinewidth{1.003750pt}%
\definecolor{currentstroke}{rgb}{0.121569,0.466667,0.705882}%
\pgfsetstrokecolor{currentstroke}%
\pgfsetstrokeopacity{0.787442}%
\pgfsetdash{}{0pt}%
\pgfpathmoveto{\pgfqpoint{2.869321in}{2.069876in}}%
\pgfpathcurveto{\pgfqpoint{2.877557in}{2.069876in}}{\pgfqpoint{2.885457in}{2.073148in}}{\pgfqpoint{2.891281in}{2.078972in}}%
\pgfpathcurveto{\pgfqpoint{2.897105in}{2.084796in}}{\pgfqpoint{2.900377in}{2.092696in}}{\pgfqpoint{2.900377in}{2.100933in}}%
\pgfpathcurveto{\pgfqpoint{2.900377in}{2.109169in}}{\pgfqpoint{2.897105in}{2.117069in}}{\pgfqpoint{2.891281in}{2.122893in}}%
\pgfpathcurveto{\pgfqpoint{2.885457in}{2.128717in}}{\pgfqpoint{2.877557in}{2.131989in}}{\pgfqpoint{2.869321in}{2.131989in}}%
\pgfpathcurveto{\pgfqpoint{2.861084in}{2.131989in}}{\pgfqpoint{2.853184in}{2.128717in}}{\pgfqpoint{2.847360in}{2.122893in}}%
\pgfpathcurveto{\pgfqpoint{2.841536in}{2.117069in}}{\pgfqpoint{2.838264in}{2.109169in}}{\pgfqpoint{2.838264in}{2.100933in}}%
\pgfpathcurveto{\pgfqpoint{2.838264in}{2.092696in}}{\pgfqpoint{2.841536in}{2.084796in}}{\pgfqpoint{2.847360in}{2.078972in}}%
\pgfpathcurveto{\pgfqpoint{2.853184in}{2.073148in}}{\pgfqpoint{2.861084in}{2.069876in}}{\pgfqpoint{2.869321in}{2.069876in}}%
\pgfpathclose%
\pgfusepath{stroke,fill}%
\end{pgfscope}%
\begin{pgfscope}%
\pgfpathrectangle{\pgfqpoint{0.100000in}{0.212622in}}{\pgfqpoint{3.696000in}{3.696000in}}%
\pgfusepath{clip}%
\pgfsetbuttcap%
\pgfsetroundjoin%
\definecolor{currentfill}{rgb}{0.121569,0.466667,0.705882}%
\pgfsetfillcolor{currentfill}%
\pgfsetfillopacity{0.789790}%
\pgfsetlinewidth{1.003750pt}%
\definecolor{currentstroke}{rgb}{0.121569,0.466667,0.705882}%
\pgfsetstrokecolor{currentstroke}%
\pgfsetstrokeopacity{0.789790}%
\pgfsetdash{}{0pt}%
\pgfpathmoveto{\pgfqpoint{2.864746in}{2.067112in}}%
\pgfpathcurveto{\pgfqpoint{2.872982in}{2.067112in}}{\pgfqpoint{2.880882in}{2.070385in}}{\pgfqpoint{2.886706in}{2.076209in}}%
\pgfpathcurveto{\pgfqpoint{2.892530in}{2.082033in}}{\pgfqpoint{2.895802in}{2.089933in}}{\pgfqpoint{2.895802in}{2.098169in}}%
\pgfpathcurveto{\pgfqpoint{2.895802in}{2.106405in}}{\pgfqpoint{2.892530in}{2.114305in}}{\pgfqpoint{2.886706in}{2.120129in}}%
\pgfpathcurveto{\pgfqpoint{2.880882in}{2.125953in}}{\pgfqpoint{2.872982in}{2.129225in}}{\pgfqpoint{2.864746in}{2.129225in}}%
\pgfpathcurveto{\pgfqpoint{2.856510in}{2.129225in}}{\pgfqpoint{2.848610in}{2.125953in}}{\pgfqpoint{2.842786in}{2.120129in}}%
\pgfpathcurveto{\pgfqpoint{2.836962in}{2.114305in}}{\pgfqpoint{2.833689in}{2.106405in}}{\pgfqpoint{2.833689in}{2.098169in}}%
\pgfpathcurveto{\pgfqpoint{2.833689in}{2.089933in}}{\pgfqpoint{2.836962in}{2.082033in}}{\pgfqpoint{2.842786in}{2.076209in}}%
\pgfpathcurveto{\pgfqpoint{2.848610in}{2.070385in}}{\pgfqpoint{2.856510in}{2.067112in}}{\pgfqpoint{2.864746in}{2.067112in}}%
\pgfpathclose%
\pgfusepath{stroke,fill}%
\end{pgfscope}%
\begin{pgfscope}%
\pgfpathrectangle{\pgfqpoint{0.100000in}{0.212622in}}{\pgfqpoint{3.696000in}{3.696000in}}%
\pgfusepath{clip}%
\pgfsetbuttcap%
\pgfsetroundjoin%
\definecolor{currentfill}{rgb}{0.121569,0.466667,0.705882}%
\pgfsetfillcolor{currentfill}%
\pgfsetfillopacity{0.792535}%
\pgfsetlinewidth{1.003750pt}%
\definecolor{currentstroke}{rgb}{0.121569,0.466667,0.705882}%
\pgfsetstrokecolor{currentstroke}%
\pgfsetstrokeopacity{0.792535}%
\pgfsetdash{}{0pt}%
\pgfpathmoveto{\pgfqpoint{2.858747in}{2.064397in}}%
\pgfpathcurveto{\pgfqpoint{2.866983in}{2.064397in}}{\pgfqpoint{2.874883in}{2.067670in}}{\pgfqpoint{2.880707in}{2.073494in}}%
\pgfpathcurveto{\pgfqpoint{2.886531in}{2.079318in}}{\pgfqpoint{2.889803in}{2.087218in}}{\pgfqpoint{2.889803in}{2.095454in}}%
\pgfpathcurveto{\pgfqpoint{2.889803in}{2.103690in}}{\pgfqpoint{2.886531in}{2.111590in}}{\pgfqpoint{2.880707in}{2.117414in}}%
\pgfpathcurveto{\pgfqpoint{2.874883in}{2.123238in}}{\pgfqpoint{2.866983in}{2.126510in}}{\pgfqpoint{2.858747in}{2.126510in}}%
\pgfpathcurveto{\pgfqpoint{2.850511in}{2.126510in}}{\pgfqpoint{2.842610in}{2.123238in}}{\pgfqpoint{2.836787in}{2.117414in}}%
\pgfpathcurveto{\pgfqpoint{2.830963in}{2.111590in}}{\pgfqpoint{2.827690in}{2.103690in}}{\pgfqpoint{2.827690in}{2.095454in}}%
\pgfpathcurveto{\pgfqpoint{2.827690in}{2.087218in}}{\pgfqpoint{2.830963in}{2.079318in}}{\pgfqpoint{2.836787in}{2.073494in}}%
\pgfpathcurveto{\pgfqpoint{2.842610in}{2.067670in}}{\pgfqpoint{2.850511in}{2.064397in}}{\pgfqpoint{2.858747in}{2.064397in}}%
\pgfpathclose%
\pgfusepath{stroke,fill}%
\end{pgfscope}%
\begin{pgfscope}%
\pgfpathrectangle{\pgfqpoint{0.100000in}{0.212622in}}{\pgfqpoint{3.696000in}{3.696000in}}%
\pgfusepath{clip}%
\pgfsetbuttcap%
\pgfsetroundjoin%
\definecolor{currentfill}{rgb}{0.121569,0.466667,0.705882}%
\pgfsetfillcolor{currentfill}%
\pgfsetfillopacity{0.793981}%
\pgfsetlinewidth{1.003750pt}%
\definecolor{currentstroke}{rgb}{0.121569,0.466667,0.705882}%
\pgfsetstrokecolor{currentstroke}%
\pgfsetstrokeopacity{0.793981}%
\pgfsetdash{}{0pt}%
\pgfpathmoveto{\pgfqpoint{2.855390in}{2.062547in}}%
\pgfpathcurveto{\pgfqpoint{2.863626in}{2.062547in}}{\pgfqpoint{2.871526in}{2.065820in}}{\pgfqpoint{2.877350in}{2.071643in}}%
\pgfpathcurveto{\pgfqpoint{2.883174in}{2.077467in}}{\pgfqpoint{2.886446in}{2.085367in}}{\pgfqpoint{2.886446in}{2.093604in}}%
\pgfpathcurveto{\pgfqpoint{2.886446in}{2.101840in}}{\pgfqpoint{2.883174in}{2.109740in}}{\pgfqpoint{2.877350in}{2.115564in}}%
\pgfpathcurveto{\pgfqpoint{2.871526in}{2.121388in}}{\pgfqpoint{2.863626in}{2.124660in}}{\pgfqpoint{2.855390in}{2.124660in}}%
\pgfpathcurveto{\pgfqpoint{2.847154in}{2.124660in}}{\pgfqpoint{2.839254in}{2.121388in}}{\pgfqpoint{2.833430in}{2.115564in}}%
\pgfpathcurveto{\pgfqpoint{2.827606in}{2.109740in}}{\pgfqpoint{2.824333in}{2.101840in}}{\pgfqpoint{2.824333in}{2.093604in}}%
\pgfpathcurveto{\pgfqpoint{2.824333in}{2.085367in}}{\pgfqpoint{2.827606in}{2.077467in}}{\pgfqpoint{2.833430in}{2.071643in}}%
\pgfpathcurveto{\pgfqpoint{2.839254in}{2.065820in}}{\pgfqpoint{2.847154in}{2.062547in}}{\pgfqpoint{2.855390in}{2.062547in}}%
\pgfpathclose%
\pgfusepath{stroke,fill}%
\end{pgfscope}%
\begin{pgfscope}%
\pgfpathrectangle{\pgfqpoint{0.100000in}{0.212622in}}{\pgfqpoint{3.696000in}{3.696000in}}%
\pgfusepath{clip}%
\pgfsetbuttcap%
\pgfsetroundjoin%
\definecolor{currentfill}{rgb}{0.121569,0.466667,0.705882}%
\pgfsetfillcolor{currentfill}%
\pgfsetfillopacity{0.794801}%
\pgfsetlinewidth{1.003750pt}%
\definecolor{currentstroke}{rgb}{0.121569,0.466667,0.705882}%
\pgfsetstrokecolor{currentstroke}%
\pgfsetstrokeopacity{0.794801}%
\pgfsetdash{}{0pt}%
\pgfpathmoveto{\pgfqpoint{2.853679in}{2.061519in}}%
\pgfpathcurveto{\pgfqpoint{2.861915in}{2.061519in}}{\pgfqpoint{2.869815in}{2.064791in}}{\pgfqpoint{2.875639in}{2.070615in}}%
\pgfpathcurveto{\pgfqpoint{2.881463in}{2.076439in}}{\pgfqpoint{2.884736in}{2.084339in}}{\pgfqpoint{2.884736in}{2.092575in}}%
\pgfpathcurveto{\pgfqpoint{2.884736in}{2.100811in}}{\pgfqpoint{2.881463in}{2.108711in}}{\pgfqpoint{2.875639in}{2.114535in}}%
\pgfpathcurveto{\pgfqpoint{2.869815in}{2.120359in}}{\pgfqpoint{2.861915in}{2.123632in}}{\pgfqpoint{2.853679in}{2.123632in}}%
\pgfpathcurveto{\pgfqpoint{2.845443in}{2.123632in}}{\pgfqpoint{2.837543in}{2.120359in}}{\pgfqpoint{2.831719in}{2.114535in}}%
\pgfpathcurveto{\pgfqpoint{2.825895in}{2.108711in}}{\pgfqpoint{2.822623in}{2.100811in}}{\pgfqpoint{2.822623in}{2.092575in}}%
\pgfpathcurveto{\pgfqpoint{2.822623in}{2.084339in}}{\pgfqpoint{2.825895in}{2.076439in}}{\pgfqpoint{2.831719in}{2.070615in}}%
\pgfpathcurveto{\pgfqpoint{2.837543in}{2.064791in}}{\pgfqpoint{2.845443in}{2.061519in}}{\pgfqpoint{2.853679in}{2.061519in}}%
\pgfpathclose%
\pgfusepath{stroke,fill}%
\end{pgfscope}%
\begin{pgfscope}%
\pgfpathrectangle{\pgfqpoint{0.100000in}{0.212622in}}{\pgfqpoint{3.696000in}{3.696000in}}%
\pgfusepath{clip}%
\pgfsetbuttcap%
\pgfsetroundjoin%
\definecolor{currentfill}{rgb}{0.121569,0.466667,0.705882}%
\pgfsetfillcolor{currentfill}%
\pgfsetfillopacity{0.796505}%
\pgfsetlinewidth{1.003750pt}%
\definecolor{currentstroke}{rgb}{0.121569,0.466667,0.705882}%
\pgfsetstrokecolor{currentstroke}%
\pgfsetstrokeopacity{0.796505}%
\pgfsetdash{}{0pt}%
\pgfpathmoveto{\pgfqpoint{2.849919in}{2.059527in}}%
\pgfpathcurveto{\pgfqpoint{2.858156in}{2.059527in}}{\pgfqpoint{2.866056in}{2.062800in}}{\pgfqpoint{2.871879in}{2.068624in}}%
\pgfpathcurveto{\pgfqpoint{2.877703in}{2.074448in}}{\pgfqpoint{2.880976in}{2.082348in}}{\pgfqpoint{2.880976in}{2.090584in}}%
\pgfpathcurveto{\pgfqpoint{2.880976in}{2.098820in}}{\pgfqpoint{2.877703in}{2.106720in}}{\pgfqpoint{2.871879in}{2.112544in}}%
\pgfpathcurveto{\pgfqpoint{2.866056in}{2.118368in}}{\pgfqpoint{2.858156in}{2.121640in}}{\pgfqpoint{2.849919in}{2.121640in}}%
\pgfpathcurveto{\pgfqpoint{2.841683in}{2.121640in}}{\pgfqpoint{2.833783in}{2.118368in}}{\pgfqpoint{2.827959in}{2.112544in}}%
\pgfpathcurveto{\pgfqpoint{2.822135in}{2.106720in}}{\pgfqpoint{2.818863in}{2.098820in}}{\pgfqpoint{2.818863in}{2.090584in}}%
\pgfpathcurveto{\pgfqpoint{2.818863in}{2.082348in}}{\pgfqpoint{2.822135in}{2.074448in}}{\pgfqpoint{2.827959in}{2.068624in}}%
\pgfpathcurveto{\pgfqpoint{2.833783in}{2.062800in}}{\pgfqpoint{2.841683in}{2.059527in}}{\pgfqpoint{2.849919in}{2.059527in}}%
\pgfpathclose%
\pgfusepath{stroke,fill}%
\end{pgfscope}%
\begin{pgfscope}%
\pgfpathrectangle{\pgfqpoint{0.100000in}{0.212622in}}{\pgfqpoint{3.696000in}{3.696000in}}%
\pgfusepath{clip}%
\pgfsetbuttcap%
\pgfsetroundjoin%
\definecolor{currentfill}{rgb}{0.121569,0.466667,0.705882}%
\pgfsetfillcolor{currentfill}%
\pgfsetfillopacity{0.798335}%
\pgfsetlinewidth{1.003750pt}%
\definecolor{currentstroke}{rgb}{0.121569,0.466667,0.705882}%
\pgfsetstrokecolor{currentstroke}%
\pgfsetstrokeopacity{0.798335}%
\pgfsetdash{}{0pt}%
\pgfpathmoveto{\pgfqpoint{2.845967in}{2.056548in}}%
\pgfpathcurveto{\pgfqpoint{2.854204in}{2.056548in}}{\pgfqpoint{2.862104in}{2.059820in}}{\pgfqpoint{2.867928in}{2.065644in}}%
\pgfpathcurveto{\pgfqpoint{2.873752in}{2.071468in}}{\pgfqpoint{2.877024in}{2.079368in}}{\pgfqpoint{2.877024in}{2.087604in}}%
\pgfpathcurveto{\pgfqpoint{2.877024in}{2.095841in}}{\pgfqpoint{2.873752in}{2.103741in}}{\pgfqpoint{2.867928in}{2.109565in}}%
\pgfpathcurveto{\pgfqpoint{2.862104in}{2.115388in}}{\pgfqpoint{2.854204in}{2.118661in}}{\pgfqpoint{2.845967in}{2.118661in}}%
\pgfpathcurveto{\pgfqpoint{2.837731in}{2.118661in}}{\pgfqpoint{2.829831in}{2.115388in}}{\pgfqpoint{2.824007in}{2.109565in}}%
\pgfpathcurveto{\pgfqpoint{2.818183in}{2.103741in}}{\pgfqpoint{2.814911in}{2.095841in}}{\pgfqpoint{2.814911in}{2.087604in}}%
\pgfpathcurveto{\pgfqpoint{2.814911in}{2.079368in}}{\pgfqpoint{2.818183in}{2.071468in}}{\pgfqpoint{2.824007in}{2.065644in}}%
\pgfpathcurveto{\pgfqpoint{2.829831in}{2.059820in}}{\pgfqpoint{2.837731in}{2.056548in}}{\pgfqpoint{2.845967in}{2.056548in}}%
\pgfpathclose%
\pgfusepath{stroke,fill}%
\end{pgfscope}%
\begin{pgfscope}%
\pgfpathrectangle{\pgfqpoint{0.100000in}{0.212622in}}{\pgfqpoint{3.696000in}{3.696000in}}%
\pgfusepath{clip}%
\pgfsetbuttcap%
\pgfsetroundjoin%
\definecolor{currentfill}{rgb}{0.121569,0.466667,0.705882}%
\pgfsetfillcolor{currentfill}%
\pgfsetfillopacity{0.800367}%
\pgfsetlinewidth{1.003750pt}%
\definecolor{currentstroke}{rgb}{0.121569,0.466667,0.705882}%
\pgfsetstrokecolor{currentstroke}%
\pgfsetstrokeopacity{0.800367}%
\pgfsetdash{}{0pt}%
\pgfpathmoveto{\pgfqpoint{2.841991in}{2.053461in}}%
\pgfpathcurveto{\pgfqpoint{2.850227in}{2.053461in}}{\pgfqpoint{2.858127in}{2.056733in}}{\pgfqpoint{2.863951in}{2.062557in}}%
\pgfpathcurveto{\pgfqpoint{2.869775in}{2.068381in}}{\pgfqpoint{2.873047in}{2.076281in}}{\pgfqpoint{2.873047in}{2.084517in}}%
\pgfpathcurveto{\pgfqpoint{2.873047in}{2.092754in}}{\pgfqpoint{2.869775in}{2.100654in}}{\pgfqpoint{2.863951in}{2.106478in}}%
\pgfpathcurveto{\pgfqpoint{2.858127in}{2.112302in}}{\pgfqpoint{2.850227in}{2.115574in}}{\pgfqpoint{2.841991in}{2.115574in}}%
\pgfpathcurveto{\pgfqpoint{2.833754in}{2.115574in}}{\pgfqpoint{2.825854in}{2.112302in}}{\pgfqpoint{2.820031in}{2.106478in}}%
\pgfpathcurveto{\pgfqpoint{2.814207in}{2.100654in}}{\pgfqpoint{2.810934in}{2.092754in}}{\pgfqpoint{2.810934in}{2.084517in}}%
\pgfpathcurveto{\pgfqpoint{2.810934in}{2.076281in}}{\pgfqpoint{2.814207in}{2.068381in}}{\pgfqpoint{2.820031in}{2.062557in}}%
\pgfpathcurveto{\pgfqpoint{2.825854in}{2.056733in}}{\pgfqpoint{2.833754in}{2.053461in}}{\pgfqpoint{2.841991in}{2.053461in}}%
\pgfpathclose%
\pgfusepath{stroke,fill}%
\end{pgfscope}%
\begin{pgfscope}%
\pgfpathrectangle{\pgfqpoint{0.100000in}{0.212622in}}{\pgfqpoint{3.696000in}{3.696000in}}%
\pgfusepath{clip}%
\pgfsetbuttcap%
\pgfsetroundjoin%
\definecolor{currentfill}{rgb}{0.121569,0.466667,0.705882}%
\pgfsetfillcolor{currentfill}%
\pgfsetfillopacity{0.801479}%
\pgfsetlinewidth{1.003750pt}%
\definecolor{currentstroke}{rgb}{0.121569,0.466667,0.705882}%
\pgfsetstrokecolor{currentstroke}%
\pgfsetstrokeopacity{0.801479}%
\pgfsetdash{}{0pt}%
\pgfpathmoveto{\pgfqpoint{2.839655in}{2.051884in}}%
\pgfpathcurveto{\pgfqpoint{2.847891in}{2.051884in}}{\pgfqpoint{2.855791in}{2.055156in}}{\pgfqpoint{2.861615in}{2.060980in}}%
\pgfpathcurveto{\pgfqpoint{2.867439in}{2.066804in}}{\pgfqpoint{2.870712in}{2.074704in}}{\pgfqpoint{2.870712in}{2.082940in}}%
\pgfpathcurveto{\pgfqpoint{2.870712in}{2.091177in}}{\pgfqpoint{2.867439in}{2.099077in}}{\pgfqpoint{2.861615in}{2.104901in}}%
\pgfpathcurveto{\pgfqpoint{2.855791in}{2.110725in}}{\pgfqpoint{2.847891in}{2.113997in}}{\pgfqpoint{2.839655in}{2.113997in}}%
\pgfpathcurveto{\pgfqpoint{2.831419in}{2.113997in}}{\pgfqpoint{2.823519in}{2.110725in}}{\pgfqpoint{2.817695in}{2.104901in}}%
\pgfpathcurveto{\pgfqpoint{2.811871in}{2.099077in}}{\pgfqpoint{2.808599in}{2.091177in}}{\pgfqpoint{2.808599in}{2.082940in}}%
\pgfpathcurveto{\pgfqpoint{2.808599in}{2.074704in}}{\pgfqpoint{2.811871in}{2.066804in}}{\pgfqpoint{2.817695in}{2.060980in}}%
\pgfpathcurveto{\pgfqpoint{2.823519in}{2.055156in}}{\pgfqpoint{2.831419in}{2.051884in}}{\pgfqpoint{2.839655in}{2.051884in}}%
\pgfpathclose%
\pgfusepath{stroke,fill}%
\end{pgfscope}%
\begin{pgfscope}%
\pgfpathrectangle{\pgfqpoint{0.100000in}{0.212622in}}{\pgfqpoint{3.696000in}{3.696000in}}%
\pgfusepath{clip}%
\pgfsetbuttcap%
\pgfsetroundjoin%
\definecolor{currentfill}{rgb}{0.121569,0.466667,0.705882}%
\pgfsetfillcolor{currentfill}%
\pgfsetfillopacity{0.803149}%
\pgfsetlinewidth{1.003750pt}%
\definecolor{currentstroke}{rgb}{0.121569,0.466667,0.705882}%
\pgfsetstrokecolor{currentstroke}%
\pgfsetstrokeopacity{0.803149}%
\pgfsetdash{}{0pt}%
\pgfpathmoveto{\pgfqpoint{2.835675in}{2.049002in}}%
\pgfpathcurveto{\pgfqpoint{2.843911in}{2.049002in}}{\pgfqpoint{2.851811in}{2.052275in}}{\pgfqpoint{2.857635in}{2.058099in}}%
\pgfpathcurveto{\pgfqpoint{2.863459in}{2.063923in}}{\pgfqpoint{2.866731in}{2.071823in}}{\pgfqpoint{2.866731in}{2.080059in}}%
\pgfpathcurveto{\pgfqpoint{2.866731in}{2.088295in}}{\pgfqpoint{2.863459in}{2.096195in}}{\pgfqpoint{2.857635in}{2.102019in}}%
\pgfpathcurveto{\pgfqpoint{2.851811in}{2.107843in}}{\pgfqpoint{2.843911in}{2.111115in}}{\pgfqpoint{2.835675in}{2.111115in}}%
\pgfpathcurveto{\pgfqpoint{2.827439in}{2.111115in}}{\pgfqpoint{2.819539in}{2.107843in}}{\pgfqpoint{2.813715in}{2.102019in}}%
\pgfpathcurveto{\pgfqpoint{2.807891in}{2.096195in}}{\pgfqpoint{2.804618in}{2.088295in}}{\pgfqpoint{2.804618in}{2.080059in}}%
\pgfpathcurveto{\pgfqpoint{2.804618in}{2.071823in}}{\pgfqpoint{2.807891in}{2.063923in}}{\pgfqpoint{2.813715in}{2.058099in}}%
\pgfpathcurveto{\pgfqpoint{2.819539in}{2.052275in}}{\pgfqpoint{2.827439in}{2.049002in}}{\pgfqpoint{2.835675in}{2.049002in}}%
\pgfpathclose%
\pgfusepath{stroke,fill}%
\end{pgfscope}%
\begin{pgfscope}%
\pgfpathrectangle{\pgfqpoint{0.100000in}{0.212622in}}{\pgfqpoint{3.696000in}{3.696000in}}%
\pgfusepath{clip}%
\pgfsetbuttcap%
\pgfsetroundjoin%
\definecolor{currentfill}{rgb}{0.121569,0.466667,0.705882}%
\pgfsetfillcolor{currentfill}%
\pgfsetfillopacity{0.805062}%
\pgfsetlinewidth{1.003750pt}%
\definecolor{currentstroke}{rgb}{0.121569,0.466667,0.705882}%
\pgfsetstrokecolor{currentstroke}%
\pgfsetstrokeopacity{0.805062}%
\pgfsetdash{}{0pt}%
\pgfpathmoveto{\pgfqpoint{2.831551in}{2.046281in}}%
\pgfpathcurveto{\pgfqpoint{2.839787in}{2.046281in}}{\pgfqpoint{2.847687in}{2.049553in}}{\pgfqpoint{2.853511in}{2.055377in}}%
\pgfpathcurveto{\pgfqpoint{2.859335in}{2.061201in}}{\pgfqpoint{2.862607in}{2.069101in}}{\pgfqpoint{2.862607in}{2.077337in}}%
\pgfpathcurveto{\pgfqpoint{2.862607in}{2.085574in}}{\pgfqpoint{2.859335in}{2.093474in}}{\pgfqpoint{2.853511in}{2.099298in}}%
\pgfpathcurveto{\pgfqpoint{2.847687in}{2.105122in}}{\pgfqpoint{2.839787in}{2.108394in}}{\pgfqpoint{2.831551in}{2.108394in}}%
\pgfpathcurveto{\pgfqpoint{2.823315in}{2.108394in}}{\pgfqpoint{2.815414in}{2.105122in}}{\pgfqpoint{2.809591in}{2.099298in}}%
\pgfpathcurveto{\pgfqpoint{2.803767in}{2.093474in}}{\pgfqpoint{2.800494in}{2.085574in}}{\pgfqpoint{2.800494in}{2.077337in}}%
\pgfpathcurveto{\pgfqpoint{2.800494in}{2.069101in}}{\pgfqpoint{2.803767in}{2.061201in}}{\pgfqpoint{2.809591in}{2.055377in}}%
\pgfpathcurveto{\pgfqpoint{2.815414in}{2.049553in}}{\pgfqpoint{2.823315in}{2.046281in}}{\pgfqpoint{2.831551in}{2.046281in}}%
\pgfpathclose%
\pgfusepath{stroke,fill}%
\end{pgfscope}%
\begin{pgfscope}%
\pgfpathrectangle{\pgfqpoint{0.100000in}{0.212622in}}{\pgfqpoint{3.696000in}{3.696000in}}%
\pgfusepath{clip}%
\pgfsetbuttcap%
\pgfsetroundjoin%
\definecolor{currentfill}{rgb}{0.121569,0.466667,0.705882}%
\pgfsetfillcolor{currentfill}%
\pgfsetfillopacity{0.806094}%
\pgfsetlinewidth{1.003750pt}%
\definecolor{currentstroke}{rgb}{0.121569,0.466667,0.705882}%
\pgfsetstrokecolor{currentstroke}%
\pgfsetstrokeopacity{0.806094}%
\pgfsetdash{}{0pt}%
\pgfpathmoveto{\pgfqpoint{2.829289in}{2.044642in}}%
\pgfpathcurveto{\pgfqpoint{2.837525in}{2.044642in}}{\pgfqpoint{2.845425in}{2.047914in}}{\pgfqpoint{2.851249in}{2.053738in}}%
\pgfpathcurveto{\pgfqpoint{2.857073in}{2.059562in}}{\pgfqpoint{2.860345in}{2.067462in}}{\pgfqpoint{2.860345in}{2.075698in}}%
\pgfpathcurveto{\pgfqpoint{2.860345in}{2.083935in}}{\pgfqpoint{2.857073in}{2.091835in}}{\pgfqpoint{2.851249in}{2.097659in}}%
\pgfpathcurveto{\pgfqpoint{2.845425in}{2.103483in}}{\pgfqpoint{2.837525in}{2.106755in}}{\pgfqpoint{2.829289in}{2.106755in}}%
\pgfpathcurveto{\pgfqpoint{2.821052in}{2.106755in}}{\pgfqpoint{2.813152in}{2.103483in}}{\pgfqpoint{2.807328in}{2.097659in}}%
\pgfpathcurveto{\pgfqpoint{2.801504in}{2.091835in}}{\pgfqpoint{2.798232in}{2.083935in}}{\pgfqpoint{2.798232in}{2.075698in}}%
\pgfpathcurveto{\pgfqpoint{2.798232in}{2.067462in}}{\pgfqpoint{2.801504in}{2.059562in}}{\pgfqpoint{2.807328in}{2.053738in}}%
\pgfpathcurveto{\pgfqpoint{2.813152in}{2.047914in}}{\pgfqpoint{2.821052in}{2.044642in}}{\pgfqpoint{2.829289in}{2.044642in}}%
\pgfpathclose%
\pgfusepath{stroke,fill}%
\end{pgfscope}%
\begin{pgfscope}%
\pgfpathrectangle{\pgfqpoint{0.100000in}{0.212622in}}{\pgfqpoint{3.696000in}{3.696000in}}%
\pgfusepath{clip}%
\pgfsetbuttcap%
\pgfsetroundjoin%
\definecolor{currentfill}{rgb}{0.121569,0.466667,0.705882}%
\pgfsetfillcolor{currentfill}%
\pgfsetfillopacity{0.807838}%
\pgfsetlinewidth{1.003750pt}%
\definecolor{currentstroke}{rgb}{0.121569,0.466667,0.705882}%
\pgfsetstrokecolor{currentstroke}%
\pgfsetstrokeopacity{0.807838}%
\pgfsetdash{}{0pt}%
\pgfpathmoveto{\pgfqpoint{2.825866in}{2.041454in}}%
\pgfpathcurveto{\pgfqpoint{2.834102in}{2.041454in}}{\pgfqpoint{2.842002in}{2.044726in}}{\pgfqpoint{2.847826in}{2.050550in}}%
\pgfpathcurveto{\pgfqpoint{2.853650in}{2.056374in}}{\pgfqpoint{2.856922in}{2.064274in}}{\pgfqpoint{2.856922in}{2.072510in}}%
\pgfpathcurveto{\pgfqpoint{2.856922in}{2.080746in}}{\pgfqpoint{2.853650in}{2.088646in}}{\pgfqpoint{2.847826in}{2.094470in}}%
\pgfpathcurveto{\pgfqpoint{2.842002in}{2.100294in}}{\pgfqpoint{2.834102in}{2.103567in}}{\pgfqpoint{2.825866in}{2.103567in}}%
\pgfpathcurveto{\pgfqpoint{2.817630in}{2.103567in}}{\pgfqpoint{2.809730in}{2.100294in}}{\pgfqpoint{2.803906in}{2.094470in}}%
\pgfpathcurveto{\pgfqpoint{2.798082in}{2.088646in}}{\pgfqpoint{2.794809in}{2.080746in}}{\pgfqpoint{2.794809in}{2.072510in}}%
\pgfpathcurveto{\pgfqpoint{2.794809in}{2.064274in}}{\pgfqpoint{2.798082in}{2.056374in}}{\pgfqpoint{2.803906in}{2.050550in}}%
\pgfpathcurveto{\pgfqpoint{2.809730in}{2.044726in}}{\pgfqpoint{2.817630in}{2.041454in}}{\pgfqpoint{2.825866in}{2.041454in}}%
\pgfpathclose%
\pgfusepath{stroke,fill}%
\end{pgfscope}%
\begin{pgfscope}%
\pgfpathrectangle{\pgfqpoint{0.100000in}{0.212622in}}{\pgfqpoint{3.696000in}{3.696000in}}%
\pgfusepath{clip}%
\pgfsetbuttcap%
\pgfsetroundjoin%
\definecolor{currentfill}{rgb}{0.121569,0.466667,0.705882}%
\pgfsetfillcolor{currentfill}%
\pgfsetfillopacity{0.809801}%
\pgfsetlinewidth{1.003750pt}%
\definecolor{currentstroke}{rgb}{0.121569,0.466667,0.705882}%
\pgfsetstrokecolor{currentstroke}%
\pgfsetstrokeopacity{0.809801}%
\pgfsetdash{}{0pt}%
\pgfpathmoveto{\pgfqpoint{2.822332in}{2.038365in}}%
\pgfpathcurveto{\pgfqpoint{2.830568in}{2.038365in}}{\pgfqpoint{2.838468in}{2.041637in}}{\pgfqpoint{2.844292in}{2.047461in}}%
\pgfpathcurveto{\pgfqpoint{2.850116in}{2.053285in}}{\pgfqpoint{2.853388in}{2.061185in}}{\pgfqpoint{2.853388in}{2.069422in}}%
\pgfpathcurveto{\pgfqpoint{2.853388in}{2.077658in}}{\pgfqpoint{2.850116in}{2.085558in}}{\pgfqpoint{2.844292in}{2.091382in}}%
\pgfpathcurveto{\pgfqpoint{2.838468in}{2.097206in}}{\pgfqpoint{2.830568in}{2.100478in}}{\pgfqpoint{2.822332in}{2.100478in}}%
\pgfpathcurveto{\pgfqpoint{2.814095in}{2.100478in}}{\pgfqpoint{2.806195in}{2.097206in}}{\pgfqpoint{2.800371in}{2.091382in}}%
\pgfpathcurveto{\pgfqpoint{2.794547in}{2.085558in}}{\pgfqpoint{2.791275in}{2.077658in}}{\pgfqpoint{2.791275in}{2.069422in}}%
\pgfpathcurveto{\pgfqpoint{2.791275in}{2.061185in}}{\pgfqpoint{2.794547in}{2.053285in}}{\pgfqpoint{2.800371in}{2.047461in}}%
\pgfpathcurveto{\pgfqpoint{2.806195in}{2.041637in}}{\pgfqpoint{2.814095in}{2.038365in}}{\pgfqpoint{2.822332in}{2.038365in}}%
\pgfpathclose%
\pgfusepath{stroke,fill}%
\end{pgfscope}%
\begin{pgfscope}%
\pgfpathrectangle{\pgfqpoint{0.100000in}{0.212622in}}{\pgfqpoint{3.696000in}{3.696000in}}%
\pgfusepath{clip}%
\pgfsetbuttcap%
\pgfsetroundjoin%
\definecolor{currentfill}{rgb}{0.121569,0.466667,0.705882}%
\pgfsetfillcolor{currentfill}%
\pgfsetfillopacity{0.811776}%
\pgfsetlinewidth{1.003750pt}%
\definecolor{currentstroke}{rgb}{0.121569,0.466667,0.705882}%
\pgfsetstrokecolor{currentstroke}%
\pgfsetstrokeopacity{0.811776}%
\pgfsetdash{}{0pt}%
\pgfpathmoveto{\pgfqpoint{2.818235in}{2.034630in}}%
\pgfpathcurveto{\pgfqpoint{2.826472in}{2.034630in}}{\pgfqpoint{2.834372in}{2.037903in}}{\pgfqpoint{2.840195in}{2.043727in}}%
\pgfpathcurveto{\pgfqpoint{2.846019in}{2.049551in}}{\pgfqpoint{2.849292in}{2.057451in}}{\pgfqpoint{2.849292in}{2.065687in}}%
\pgfpathcurveto{\pgfqpoint{2.849292in}{2.073923in}}{\pgfqpoint{2.846019in}{2.081823in}}{\pgfqpoint{2.840195in}{2.087647in}}%
\pgfpathcurveto{\pgfqpoint{2.834372in}{2.093471in}}{\pgfqpoint{2.826472in}{2.096743in}}{\pgfqpoint{2.818235in}{2.096743in}}%
\pgfpathcurveto{\pgfqpoint{2.809999in}{2.096743in}}{\pgfqpoint{2.802099in}{2.093471in}}{\pgfqpoint{2.796275in}{2.087647in}}%
\pgfpathcurveto{\pgfqpoint{2.790451in}{2.081823in}}{\pgfqpoint{2.787179in}{2.073923in}}{\pgfqpoint{2.787179in}{2.065687in}}%
\pgfpathcurveto{\pgfqpoint{2.787179in}{2.057451in}}{\pgfqpoint{2.790451in}{2.049551in}}{\pgfqpoint{2.796275in}{2.043727in}}%
\pgfpathcurveto{\pgfqpoint{2.802099in}{2.037903in}}{\pgfqpoint{2.809999in}{2.034630in}}{\pgfqpoint{2.818235in}{2.034630in}}%
\pgfpathclose%
\pgfusepath{stroke,fill}%
\end{pgfscope}%
\begin{pgfscope}%
\pgfpathrectangle{\pgfqpoint{0.100000in}{0.212622in}}{\pgfqpoint{3.696000in}{3.696000in}}%
\pgfusepath{clip}%
\pgfsetbuttcap%
\pgfsetroundjoin%
\definecolor{currentfill}{rgb}{0.121569,0.466667,0.705882}%
\pgfsetfillcolor{currentfill}%
\pgfsetfillopacity{0.814579}%
\pgfsetlinewidth{1.003750pt}%
\definecolor{currentstroke}{rgb}{0.121569,0.466667,0.705882}%
\pgfsetstrokecolor{currentstroke}%
\pgfsetstrokeopacity{0.814579}%
\pgfsetdash{}{0pt}%
\pgfpathmoveto{\pgfqpoint{2.812493in}{2.028850in}}%
\pgfpathcurveto{\pgfqpoint{2.820729in}{2.028850in}}{\pgfqpoint{2.828630in}{2.032123in}}{\pgfqpoint{2.834453in}{2.037946in}}%
\pgfpathcurveto{\pgfqpoint{2.840277in}{2.043770in}}{\pgfqpoint{2.843550in}{2.051670in}}{\pgfqpoint{2.843550in}{2.059907in}}%
\pgfpathcurveto{\pgfqpoint{2.843550in}{2.068143in}}{\pgfqpoint{2.840277in}{2.076043in}}{\pgfqpoint{2.834453in}{2.081867in}}%
\pgfpathcurveto{\pgfqpoint{2.828630in}{2.087691in}}{\pgfqpoint{2.820729in}{2.090963in}}{\pgfqpoint{2.812493in}{2.090963in}}%
\pgfpathcurveto{\pgfqpoint{2.804257in}{2.090963in}}{\pgfqpoint{2.796357in}{2.087691in}}{\pgfqpoint{2.790533in}{2.081867in}}%
\pgfpathcurveto{\pgfqpoint{2.784709in}{2.076043in}}{\pgfqpoint{2.781437in}{2.068143in}}{\pgfqpoint{2.781437in}{2.059907in}}%
\pgfpathcurveto{\pgfqpoint{2.781437in}{2.051670in}}{\pgfqpoint{2.784709in}{2.043770in}}{\pgfqpoint{2.790533in}{2.037946in}}%
\pgfpathcurveto{\pgfqpoint{2.796357in}{2.032123in}}{\pgfqpoint{2.804257in}{2.028850in}}{\pgfqpoint{2.812493in}{2.028850in}}%
\pgfpathclose%
\pgfusepath{stroke,fill}%
\end{pgfscope}%
\begin{pgfscope}%
\pgfpathrectangle{\pgfqpoint{0.100000in}{0.212622in}}{\pgfqpoint{3.696000in}{3.696000in}}%
\pgfusepath{clip}%
\pgfsetbuttcap%
\pgfsetroundjoin%
\definecolor{currentfill}{rgb}{0.121569,0.466667,0.705882}%
\pgfsetfillcolor{currentfill}%
\pgfsetfillopacity{0.816211}%
\pgfsetlinewidth{1.003750pt}%
\definecolor{currentstroke}{rgb}{0.121569,0.466667,0.705882}%
\pgfsetstrokecolor{currentstroke}%
\pgfsetstrokeopacity{0.816211}%
\pgfsetdash{}{0pt}%
\pgfpathmoveto{\pgfqpoint{2.809427in}{2.026142in}}%
\pgfpathcurveto{\pgfqpoint{2.817664in}{2.026142in}}{\pgfqpoint{2.825564in}{2.029414in}}{\pgfqpoint{2.831388in}{2.035238in}}%
\pgfpathcurveto{\pgfqpoint{2.837212in}{2.041062in}}{\pgfqpoint{2.840484in}{2.048962in}}{\pgfqpoint{2.840484in}{2.057198in}}%
\pgfpathcurveto{\pgfqpoint{2.840484in}{2.065435in}}{\pgfqpoint{2.837212in}{2.073335in}}{\pgfqpoint{2.831388in}{2.079159in}}%
\pgfpathcurveto{\pgfqpoint{2.825564in}{2.084983in}}{\pgfqpoint{2.817664in}{2.088255in}}{\pgfqpoint{2.809427in}{2.088255in}}%
\pgfpathcurveto{\pgfqpoint{2.801191in}{2.088255in}}{\pgfqpoint{2.793291in}{2.084983in}}{\pgfqpoint{2.787467in}{2.079159in}}%
\pgfpathcurveto{\pgfqpoint{2.781643in}{2.073335in}}{\pgfqpoint{2.778371in}{2.065435in}}{\pgfqpoint{2.778371in}{2.057198in}}%
\pgfpathcurveto{\pgfqpoint{2.778371in}{2.048962in}}{\pgfqpoint{2.781643in}{2.041062in}}{\pgfqpoint{2.787467in}{2.035238in}}%
\pgfpathcurveto{\pgfqpoint{2.793291in}{2.029414in}}{\pgfqpoint{2.801191in}{2.026142in}}{\pgfqpoint{2.809427in}{2.026142in}}%
\pgfpathclose%
\pgfusepath{stroke,fill}%
\end{pgfscope}%
\begin{pgfscope}%
\pgfpathrectangle{\pgfqpoint{0.100000in}{0.212622in}}{\pgfqpoint{3.696000in}{3.696000in}}%
\pgfusepath{clip}%
\pgfsetbuttcap%
\pgfsetroundjoin%
\definecolor{currentfill}{rgb}{0.121569,0.466667,0.705882}%
\pgfsetfillcolor{currentfill}%
\pgfsetfillopacity{0.817868}%
\pgfsetlinewidth{1.003750pt}%
\definecolor{currentstroke}{rgb}{0.121569,0.466667,0.705882}%
\pgfsetstrokecolor{currentstroke}%
\pgfsetstrokeopacity{0.817868}%
\pgfsetdash{}{0pt}%
\pgfpathmoveto{\pgfqpoint{2.805844in}{2.022823in}}%
\pgfpathcurveto{\pgfqpoint{2.814080in}{2.022823in}}{\pgfqpoint{2.821981in}{2.026096in}}{\pgfqpoint{2.827804in}{2.031920in}}%
\pgfpathcurveto{\pgfqpoint{2.833628in}{2.037744in}}{\pgfqpoint{2.836901in}{2.045644in}}{\pgfqpoint{2.836901in}{2.053880in}}%
\pgfpathcurveto{\pgfqpoint{2.836901in}{2.062116in}}{\pgfqpoint{2.833628in}{2.070016in}}{\pgfqpoint{2.827804in}{2.075840in}}%
\pgfpathcurveto{\pgfqpoint{2.821981in}{2.081664in}}{\pgfqpoint{2.814080in}{2.084936in}}{\pgfqpoint{2.805844in}{2.084936in}}%
\pgfpathcurveto{\pgfqpoint{2.797608in}{2.084936in}}{\pgfqpoint{2.789708in}{2.081664in}}{\pgfqpoint{2.783884in}{2.075840in}}%
\pgfpathcurveto{\pgfqpoint{2.778060in}{2.070016in}}{\pgfqpoint{2.774788in}{2.062116in}}{\pgfqpoint{2.774788in}{2.053880in}}%
\pgfpathcurveto{\pgfqpoint{2.774788in}{2.045644in}}{\pgfqpoint{2.778060in}{2.037744in}}{\pgfqpoint{2.783884in}{2.031920in}}%
\pgfpathcurveto{\pgfqpoint{2.789708in}{2.026096in}}{\pgfqpoint{2.797608in}{2.022823in}}{\pgfqpoint{2.805844in}{2.022823in}}%
\pgfpathclose%
\pgfusepath{stroke,fill}%
\end{pgfscope}%
\begin{pgfscope}%
\pgfpathrectangle{\pgfqpoint{0.100000in}{0.212622in}}{\pgfqpoint{3.696000in}{3.696000in}}%
\pgfusepath{clip}%
\pgfsetbuttcap%
\pgfsetroundjoin%
\definecolor{currentfill}{rgb}{0.121569,0.466667,0.705882}%
\pgfsetfillcolor{currentfill}%
\pgfsetfillopacity{0.820098}%
\pgfsetlinewidth{1.003750pt}%
\definecolor{currentstroke}{rgb}{0.121569,0.466667,0.705882}%
\pgfsetstrokecolor{currentstroke}%
\pgfsetstrokeopacity{0.820098}%
\pgfsetdash{}{0pt}%
\pgfpathmoveto{\pgfqpoint{2.801561in}{2.017960in}}%
\pgfpathcurveto{\pgfqpoint{2.809798in}{2.017960in}}{\pgfqpoint{2.817698in}{2.021233in}}{\pgfqpoint{2.823522in}{2.027057in}}%
\pgfpathcurveto{\pgfqpoint{2.829345in}{2.032881in}}{\pgfqpoint{2.832618in}{2.040781in}}{\pgfqpoint{2.832618in}{2.049017in}}%
\pgfpathcurveto{\pgfqpoint{2.832618in}{2.057253in}}{\pgfqpoint{2.829345in}{2.065153in}}{\pgfqpoint{2.823522in}{2.070977in}}%
\pgfpathcurveto{\pgfqpoint{2.817698in}{2.076801in}}{\pgfqpoint{2.809798in}{2.080073in}}{\pgfqpoint{2.801561in}{2.080073in}}%
\pgfpathcurveto{\pgfqpoint{2.793325in}{2.080073in}}{\pgfqpoint{2.785425in}{2.076801in}}{\pgfqpoint{2.779601in}{2.070977in}}%
\pgfpathcurveto{\pgfqpoint{2.773777in}{2.065153in}}{\pgfqpoint{2.770505in}{2.057253in}}{\pgfqpoint{2.770505in}{2.049017in}}%
\pgfpathcurveto{\pgfqpoint{2.770505in}{2.040781in}}{\pgfqpoint{2.773777in}{2.032881in}}{\pgfqpoint{2.779601in}{2.027057in}}%
\pgfpathcurveto{\pgfqpoint{2.785425in}{2.021233in}}{\pgfqpoint{2.793325in}{2.017960in}}{\pgfqpoint{2.801561in}{2.017960in}}%
\pgfpathclose%
\pgfusepath{stroke,fill}%
\end{pgfscope}%
\begin{pgfscope}%
\pgfpathrectangle{\pgfqpoint{0.100000in}{0.212622in}}{\pgfqpoint{3.696000in}{3.696000in}}%
\pgfusepath{clip}%
\pgfsetbuttcap%
\pgfsetroundjoin%
\definecolor{currentfill}{rgb}{0.121569,0.466667,0.705882}%
\pgfsetfillcolor{currentfill}%
\pgfsetfillopacity{0.822706}%
\pgfsetlinewidth{1.003750pt}%
\definecolor{currentstroke}{rgb}{0.121569,0.466667,0.705882}%
\pgfsetstrokecolor{currentstroke}%
\pgfsetstrokeopacity{0.822706}%
\pgfsetdash{}{0pt}%
\pgfpathmoveto{\pgfqpoint{2.796943in}{2.013980in}}%
\pgfpathcurveto{\pgfqpoint{2.805179in}{2.013980in}}{\pgfqpoint{2.813079in}{2.017252in}}{\pgfqpoint{2.818903in}{2.023076in}}%
\pgfpathcurveto{\pgfqpoint{2.824727in}{2.028900in}}{\pgfqpoint{2.827999in}{2.036800in}}{\pgfqpoint{2.827999in}{2.045036in}}%
\pgfpathcurveto{\pgfqpoint{2.827999in}{2.053272in}}{\pgfqpoint{2.824727in}{2.061172in}}{\pgfqpoint{2.818903in}{2.066996in}}%
\pgfpathcurveto{\pgfqpoint{2.813079in}{2.072820in}}{\pgfqpoint{2.805179in}{2.076093in}}{\pgfqpoint{2.796943in}{2.076093in}}%
\pgfpathcurveto{\pgfqpoint{2.788706in}{2.076093in}}{\pgfqpoint{2.780806in}{2.072820in}}{\pgfqpoint{2.774982in}{2.066996in}}%
\pgfpathcurveto{\pgfqpoint{2.769158in}{2.061172in}}{\pgfqpoint{2.765886in}{2.053272in}}{\pgfqpoint{2.765886in}{2.045036in}}%
\pgfpathcurveto{\pgfqpoint{2.765886in}{2.036800in}}{\pgfqpoint{2.769158in}{2.028900in}}{\pgfqpoint{2.774982in}{2.023076in}}%
\pgfpathcurveto{\pgfqpoint{2.780806in}{2.017252in}}{\pgfqpoint{2.788706in}{2.013980in}}{\pgfqpoint{2.796943in}{2.013980in}}%
\pgfpathclose%
\pgfusepath{stroke,fill}%
\end{pgfscope}%
\begin{pgfscope}%
\pgfpathrectangle{\pgfqpoint{0.100000in}{0.212622in}}{\pgfqpoint{3.696000in}{3.696000in}}%
\pgfusepath{clip}%
\pgfsetbuttcap%
\pgfsetroundjoin%
\definecolor{currentfill}{rgb}{0.121569,0.466667,0.705882}%
\pgfsetfillcolor{currentfill}%
\pgfsetfillopacity{0.824060}%
\pgfsetlinewidth{1.003750pt}%
\definecolor{currentstroke}{rgb}{0.121569,0.466667,0.705882}%
\pgfsetstrokecolor{currentstroke}%
\pgfsetstrokeopacity{0.824060}%
\pgfsetdash{}{0pt}%
\pgfpathmoveto{\pgfqpoint{2.794245in}{2.011425in}}%
\pgfpathcurveto{\pgfqpoint{2.802481in}{2.011425in}}{\pgfqpoint{2.810381in}{2.014697in}}{\pgfqpoint{2.816205in}{2.020521in}}%
\pgfpathcurveto{\pgfqpoint{2.822029in}{2.026345in}}{\pgfqpoint{2.825301in}{2.034245in}}{\pgfqpoint{2.825301in}{2.042481in}}%
\pgfpathcurveto{\pgfqpoint{2.825301in}{2.050717in}}{\pgfqpoint{2.822029in}{2.058617in}}{\pgfqpoint{2.816205in}{2.064441in}}%
\pgfpathcurveto{\pgfqpoint{2.810381in}{2.070265in}}{\pgfqpoint{2.802481in}{2.073538in}}{\pgfqpoint{2.794245in}{2.073538in}}%
\pgfpathcurveto{\pgfqpoint{2.786009in}{2.073538in}}{\pgfqpoint{2.778108in}{2.070265in}}{\pgfqpoint{2.772285in}{2.064441in}}%
\pgfpathcurveto{\pgfqpoint{2.766461in}{2.058617in}}{\pgfqpoint{2.763188in}{2.050717in}}{\pgfqpoint{2.763188in}{2.042481in}}%
\pgfpathcurveto{\pgfqpoint{2.763188in}{2.034245in}}{\pgfqpoint{2.766461in}{2.026345in}}{\pgfqpoint{2.772285in}{2.020521in}}%
\pgfpathcurveto{\pgfqpoint{2.778108in}{2.014697in}}{\pgfqpoint{2.786009in}{2.011425in}}{\pgfqpoint{2.794245in}{2.011425in}}%
\pgfpathclose%
\pgfusepath{stroke,fill}%
\end{pgfscope}%
\begin{pgfscope}%
\pgfpathrectangle{\pgfqpoint{0.100000in}{0.212622in}}{\pgfqpoint{3.696000in}{3.696000in}}%
\pgfusepath{clip}%
\pgfsetbuttcap%
\pgfsetroundjoin%
\definecolor{currentfill}{rgb}{0.121569,0.466667,0.705882}%
\pgfsetfillcolor{currentfill}%
\pgfsetfillopacity{0.824762}%
\pgfsetlinewidth{1.003750pt}%
\definecolor{currentstroke}{rgb}{0.121569,0.466667,0.705882}%
\pgfsetstrokecolor{currentstroke}%
\pgfsetstrokeopacity{0.824762}%
\pgfsetdash{}{0pt}%
\pgfpathmoveto{\pgfqpoint{0.527370in}{2.581723in}}%
\pgfpathcurveto{\pgfqpoint{0.535606in}{2.581723in}}{\pgfqpoint{0.543506in}{2.584995in}}{\pgfqpoint{0.549330in}{2.590819in}}%
\pgfpathcurveto{\pgfqpoint{0.555154in}{2.596643in}}{\pgfqpoint{0.558426in}{2.604543in}}{\pgfqpoint{0.558426in}{2.612780in}}%
\pgfpathcurveto{\pgfqpoint{0.558426in}{2.621016in}}{\pgfqpoint{0.555154in}{2.628916in}}{\pgfqpoint{0.549330in}{2.634740in}}%
\pgfpathcurveto{\pgfqpoint{0.543506in}{2.640564in}}{\pgfqpoint{0.535606in}{2.643836in}}{\pgfqpoint{0.527370in}{2.643836in}}%
\pgfpathcurveto{\pgfqpoint{0.519133in}{2.643836in}}{\pgfqpoint{0.511233in}{2.640564in}}{\pgfqpoint{0.505409in}{2.634740in}}%
\pgfpathcurveto{\pgfqpoint{0.499586in}{2.628916in}}{\pgfqpoint{0.496313in}{2.621016in}}{\pgfqpoint{0.496313in}{2.612780in}}%
\pgfpathcurveto{\pgfqpoint{0.496313in}{2.604543in}}{\pgfqpoint{0.499586in}{2.596643in}}{\pgfqpoint{0.505409in}{2.590819in}}%
\pgfpathcurveto{\pgfqpoint{0.511233in}{2.584995in}}{\pgfqpoint{0.519133in}{2.581723in}}{\pgfqpoint{0.527370in}{2.581723in}}%
\pgfpathclose%
\pgfusepath{stroke,fill}%
\end{pgfscope}%
\begin{pgfscope}%
\pgfpathrectangle{\pgfqpoint{0.100000in}{0.212622in}}{\pgfqpoint{3.696000in}{3.696000in}}%
\pgfusepath{clip}%
\pgfsetbuttcap%
\pgfsetroundjoin%
\definecolor{currentfill}{rgb}{0.121569,0.466667,0.705882}%
\pgfsetfillcolor{currentfill}%
\pgfsetfillopacity{0.824849}%
\pgfsetlinewidth{1.003750pt}%
\definecolor{currentstroke}{rgb}{0.121569,0.466667,0.705882}%
\pgfsetstrokecolor{currentstroke}%
\pgfsetstrokeopacity{0.824849}%
\pgfsetdash{}{0pt}%
\pgfpathmoveto{\pgfqpoint{2.792895in}{2.010168in}}%
\pgfpathcurveto{\pgfqpoint{2.801131in}{2.010168in}}{\pgfqpoint{2.809031in}{2.013441in}}{\pgfqpoint{2.814855in}{2.019265in}}%
\pgfpathcurveto{\pgfqpoint{2.820679in}{2.025089in}}{\pgfqpoint{2.823951in}{2.032989in}}{\pgfqpoint{2.823951in}{2.041225in}}%
\pgfpathcurveto{\pgfqpoint{2.823951in}{2.049461in}}{\pgfqpoint{2.820679in}{2.057361in}}{\pgfqpoint{2.814855in}{2.063185in}}%
\pgfpathcurveto{\pgfqpoint{2.809031in}{2.069009in}}{\pgfqpoint{2.801131in}{2.072281in}}{\pgfqpoint{2.792895in}{2.072281in}}%
\pgfpathcurveto{\pgfqpoint{2.784659in}{2.072281in}}{\pgfqpoint{2.776759in}{2.069009in}}{\pgfqpoint{2.770935in}{2.063185in}}%
\pgfpathcurveto{\pgfqpoint{2.765111in}{2.057361in}}{\pgfqpoint{2.761838in}{2.049461in}}{\pgfqpoint{2.761838in}{2.041225in}}%
\pgfpathcurveto{\pgfqpoint{2.761838in}{2.032989in}}{\pgfqpoint{2.765111in}{2.025089in}}{\pgfqpoint{2.770935in}{2.019265in}}%
\pgfpathcurveto{\pgfqpoint{2.776759in}{2.013441in}}{\pgfqpoint{2.784659in}{2.010168in}}{\pgfqpoint{2.792895in}{2.010168in}}%
\pgfpathclose%
\pgfusepath{stroke,fill}%
\end{pgfscope}%
\begin{pgfscope}%
\pgfpathrectangle{\pgfqpoint{0.100000in}{0.212622in}}{\pgfqpoint{3.696000in}{3.696000in}}%
\pgfusepath{clip}%
\pgfsetbuttcap%
\pgfsetroundjoin%
\definecolor{currentfill}{rgb}{0.121569,0.466667,0.705882}%
\pgfsetfillcolor{currentfill}%
\pgfsetfillopacity{0.825048}%
\pgfsetlinewidth{1.003750pt}%
\definecolor{currentstroke}{rgb}{0.121569,0.466667,0.705882}%
\pgfsetstrokecolor{currentstroke}%
\pgfsetstrokeopacity{0.825048}%
\pgfsetdash{}{0pt}%
\pgfpathmoveto{\pgfqpoint{0.530609in}{2.580949in}}%
\pgfpathcurveto{\pgfqpoint{0.538845in}{2.580949in}}{\pgfqpoint{0.546745in}{2.584221in}}{\pgfqpoint{0.552569in}{2.590045in}}%
\pgfpathcurveto{\pgfqpoint{0.558393in}{2.595869in}}{\pgfqpoint{0.561665in}{2.603769in}}{\pgfqpoint{0.561665in}{2.612005in}}%
\pgfpathcurveto{\pgfqpoint{0.561665in}{2.620241in}}{\pgfqpoint{0.558393in}{2.628141in}}{\pgfqpoint{0.552569in}{2.633965in}}%
\pgfpathcurveto{\pgfqpoint{0.546745in}{2.639789in}}{\pgfqpoint{0.538845in}{2.643062in}}{\pgfqpoint{0.530609in}{2.643062in}}%
\pgfpathcurveto{\pgfqpoint{0.522372in}{2.643062in}}{\pgfqpoint{0.514472in}{2.639789in}}{\pgfqpoint{0.508648in}{2.633965in}}%
\pgfpathcurveto{\pgfqpoint{0.502824in}{2.628141in}}{\pgfqpoint{0.499552in}{2.620241in}}{\pgfqpoint{0.499552in}{2.612005in}}%
\pgfpathcurveto{\pgfqpoint{0.499552in}{2.603769in}}{\pgfqpoint{0.502824in}{2.595869in}}{\pgfqpoint{0.508648in}{2.590045in}}%
\pgfpathcurveto{\pgfqpoint{0.514472in}{2.584221in}}{\pgfqpoint{0.522372in}{2.580949in}}{\pgfqpoint{0.530609in}{2.580949in}}%
\pgfpathclose%
\pgfusepath{stroke,fill}%
\end{pgfscope}%
\begin{pgfscope}%
\pgfpathrectangle{\pgfqpoint{0.100000in}{0.212622in}}{\pgfqpoint{3.696000in}{3.696000in}}%
\pgfusepath{clip}%
\pgfsetbuttcap%
\pgfsetroundjoin%
\definecolor{currentfill}{rgb}{0.121569,0.466667,0.705882}%
\pgfsetfillcolor{currentfill}%
\pgfsetfillopacity{0.825285}%
\pgfsetlinewidth{1.003750pt}%
\definecolor{currentstroke}{rgb}{0.121569,0.466667,0.705882}%
\pgfsetstrokecolor{currentstroke}%
\pgfsetstrokeopacity{0.825285}%
\pgfsetdash{}{0pt}%
\pgfpathmoveto{\pgfqpoint{0.533294in}{2.580306in}}%
\pgfpathcurveto{\pgfqpoint{0.541530in}{2.580306in}}{\pgfqpoint{0.549430in}{2.583579in}}{\pgfqpoint{0.555254in}{2.589403in}}%
\pgfpathcurveto{\pgfqpoint{0.561078in}{2.595226in}}{\pgfqpoint{0.564351in}{2.603127in}}{\pgfqpoint{0.564351in}{2.611363in}}%
\pgfpathcurveto{\pgfqpoint{0.564351in}{2.619599in}}{\pgfqpoint{0.561078in}{2.627499in}}{\pgfqpoint{0.555254in}{2.633323in}}%
\pgfpathcurveto{\pgfqpoint{0.549430in}{2.639147in}}{\pgfqpoint{0.541530in}{2.642419in}}{\pgfqpoint{0.533294in}{2.642419in}}%
\pgfpathcurveto{\pgfqpoint{0.525058in}{2.642419in}}{\pgfqpoint{0.517158in}{2.639147in}}{\pgfqpoint{0.511334in}{2.633323in}}%
\pgfpathcurveto{\pgfqpoint{0.505510in}{2.627499in}}{\pgfqpoint{0.502238in}{2.619599in}}{\pgfqpoint{0.502238in}{2.611363in}}%
\pgfpathcurveto{\pgfqpoint{0.502238in}{2.603127in}}{\pgfqpoint{0.505510in}{2.595226in}}{\pgfqpoint{0.511334in}{2.589403in}}%
\pgfpathcurveto{\pgfqpoint{0.517158in}{2.583579in}}{\pgfqpoint{0.525058in}{2.580306in}}{\pgfqpoint{0.533294in}{2.580306in}}%
\pgfpathclose%
\pgfusepath{stroke,fill}%
\end{pgfscope}%
\begin{pgfscope}%
\pgfpathrectangle{\pgfqpoint{0.100000in}{0.212622in}}{\pgfqpoint{3.696000in}{3.696000in}}%
\pgfusepath{clip}%
\pgfsetbuttcap%
\pgfsetroundjoin%
\definecolor{currentfill}{rgb}{0.121569,0.466667,0.705882}%
\pgfsetfillcolor{currentfill}%
\pgfsetfillopacity{0.825756}%
\pgfsetlinewidth{1.003750pt}%
\definecolor{currentstroke}{rgb}{0.121569,0.466667,0.705882}%
\pgfsetstrokecolor{currentstroke}%
\pgfsetstrokeopacity{0.825756}%
\pgfsetdash{}{0pt}%
\pgfpathmoveto{\pgfqpoint{0.538069in}{2.578925in}}%
\pgfpathcurveto{\pgfqpoint{0.546305in}{2.578925in}}{\pgfqpoint{0.554205in}{2.582197in}}{\pgfqpoint{0.560029in}{2.588021in}}%
\pgfpathcurveto{\pgfqpoint{0.565853in}{2.593845in}}{\pgfqpoint{0.569125in}{2.601745in}}{\pgfqpoint{0.569125in}{2.609981in}}%
\pgfpathcurveto{\pgfqpoint{0.569125in}{2.618218in}}{\pgfqpoint{0.565853in}{2.626118in}}{\pgfqpoint{0.560029in}{2.631942in}}%
\pgfpathcurveto{\pgfqpoint{0.554205in}{2.637765in}}{\pgfqpoint{0.546305in}{2.641038in}}{\pgfqpoint{0.538069in}{2.641038in}}%
\pgfpathcurveto{\pgfqpoint{0.529832in}{2.641038in}}{\pgfqpoint{0.521932in}{2.637765in}}{\pgfqpoint{0.516108in}{2.631942in}}%
\pgfpathcurveto{\pgfqpoint{0.510285in}{2.626118in}}{\pgfqpoint{0.507012in}{2.618218in}}{\pgfqpoint{0.507012in}{2.609981in}}%
\pgfpathcurveto{\pgfqpoint{0.507012in}{2.601745in}}{\pgfqpoint{0.510285in}{2.593845in}}{\pgfqpoint{0.516108in}{2.588021in}}%
\pgfpathcurveto{\pgfqpoint{0.521932in}{2.582197in}}{\pgfqpoint{0.529832in}{2.578925in}}{\pgfqpoint{0.538069in}{2.578925in}}%
\pgfpathclose%
\pgfusepath{stroke,fill}%
\end{pgfscope}%
\begin{pgfscope}%
\pgfpathrectangle{\pgfqpoint{0.100000in}{0.212622in}}{\pgfqpoint{3.696000in}{3.696000in}}%
\pgfusepath{clip}%
\pgfsetbuttcap%
\pgfsetroundjoin%
\definecolor{currentfill}{rgb}{0.121569,0.466667,0.705882}%
\pgfsetfillcolor{currentfill}%
\pgfsetfillopacity{0.826184}%
\pgfsetlinewidth{1.003750pt}%
\definecolor{currentstroke}{rgb}{0.121569,0.466667,0.705882}%
\pgfsetstrokecolor{currentstroke}%
\pgfsetstrokeopacity{0.826184}%
\pgfsetdash{}{0pt}%
\pgfpathmoveto{\pgfqpoint{0.542142in}{2.577296in}}%
\pgfpathcurveto{\pgfqpoint{0.550378in}{2.577296in}}{\pgfqpoint{0.558278in}{2.580568in}}{\pgfqpoint{0.564102in}{2.586392in}}%
\pgfpathcurveto{\pgfqpoint{0.569926in}{2.592216in}}{\pgfqpoint{0.573198in}{2.600116in}}{\pgfqpoint{0.573198in}{2.608352in}}%
\pgfpathcurveto{\pgfqpoint{0.573198in}{2.616588in}}{\pgfqpoint{0.569926in}{2.624488in}}{\pgfqpoint{0.564102in}{2.630312in}}%
\pgfpathcurveto{\pgfqpoint{0.558278in}{2.636136in}}{\pgfqpoint{0.550378in}{2.639409in}}{\pgfqpoint{0.542142in}{2.639409in}}%
\pgfpathcurveto{\pgfqpoint{0.533905in}{2.639409in}}{\pgfqpoint{0.526005in}{2.636136in}}{\pgfqpoint{0.520181in}{2.630312in}}%
\pgfpathcurveto{\pgfqpoint{0.514357in}{2.624488in}}{\pgfqpoint{0.511085in}{2.616588in}}{\pgfqpoint{0.511085in}{2.608352in}}%
\pgfpathcurveto{\pgfqpoint{0.511085in}{2.600116in}}{\pgfqpoint{0.514357in}{2.592216in}}{\pgfqpoint{0.520181in}{2.586392in}}%
\pgfpathcurveto{\pgfqpoint{0.526005in}{2.580568in}}{\pgfqpoint{0.533905in}{2.577296in}}{\pgfqpoint{0.542142in}{2.577296in}}%
\pgfpathclose%
\pgfusepath{stroke,fill}%
\end{pgfscope}%
\begin{pgfscope}%
\pgfpathrectangle{\pgfqpoint{0.100000in}{0.212622in}}{\pgfqpoint{3.696000in}{3.696000in}}%
\pgfusepath{clip}%
\pgfsetbuttcap%
\pgfsetroundjoin%
\definecolor{currentfill}{rgb}{0.121569,0.466667,0.705882}%
\pgfsetfillcolor{currentfill}%
\pgfsetfillopacity{0.826367}%
\pgfsetlinewidth{1.003750pt}%
\definecolor{currentstroke}{rgb}{0.121569,0.466667,0.705882}%
\pgfsetstrokecolor{currentstroke}%
\pgfsetstrokeopacity{0.826367}%
\pgfsetdash{}{0pt}%
\pgfpathmoveto{\pgfqpoint{2.789742in}{2.007053in}}%
\pgfpathcurveto{\pgfqpoint{2.797978in}{2.007053in}}{\pgfqpoint{2.805878in}{2.010325in}}{\pgfqpoint{2.811702in}{2.016149in}}%
\pgfpathcurveto{\pgfqpoint{2.817526in}{2.021973in}}{\pgfqpoint{2.820799in}{2.029873in}}{\pgfqpoint{2.820799in}{2.038109in}}%
\pgfpathcurveto{\pgfqpoint{2.820799in}{2.046346in}}{\pgfqpoint{2.817526in}{2.054246in}}{\pgfqpoint{2.811702in}{2.060070in}}%
\pgfpathcurveto{\pgfqpoint{2.805878in}{2.065894in}}{\pgfqpoint{2.797978in}{2.069166in}}{\pgfqpoint{2.789742in}{2.069166in}}%
\pgfpathcurveto{\pgfqpoint{2.781506in}{2.069166in}}{\pgfqpoint{2.773606in}{2.065894in}}{\pgfqpoint{2.767782in}{2.060070in}}%
\pgfpathcurveto{\pgfqpoint{2.761958in}{2.054246in}}{\pgfqpoint{2.758686in}{2.046346in}}{\pgfqpoint{2.758686in}{2.038109in}}%
\pgfpathcurveto{\pgfqpoint{2.758686in}{2.029873in}}{\pgfqpoint{2.761958in}{2.021973in}}{\pgfqpoint{2.767782in}{2.016149in}}%
\pgfpathcurveto{\pgfqpoint{2.773606in}{2.010325in}}{\pgfqpoint{2.781506in}{2.007053in}}{\pgfqpoint{2.789742in}{2.007053in}}%
\pgfpathclose%
\pgfusepath{stroke,fill}%
\end{pgfscope}%
\begin{pgfscope}%
\pgfpathrectangle{\pgfqpoint{0.100000in}{0.212622in}}{\pgfqpoint{3.696000in}{3.696000in}}%
\pgfusepath{clip}%
\pgfsetbuttcap%
\pgfsetroundjoin%
\definecolor{currentfill}{rgb}{0.121569,0.466667,0.705882}%
\pgfsetfillcolor{currentfill}%
\pgfsetfillopacity{0.826840}%
\pgfsetlinewidth{1.003750pt}%
\definecolor{currentstroke}{rgb}{0.121569,0.466667,0.705882}%
\pgfsetstrokecolor{currentstroke}%
\pgfsetstrokeopacity{0.826840}%
\pgfsetdash{}{0pt}%
\pgfpathmoveto{\pgfqpoint{0.549481in}{2.573181in}}%
\pgfpathcurveto{\pgfqpoint{0.557717in}{2.573181in}}{\pgfqpoint{0.565617in}{2.576454in}}{\pgfqpoint{0.571441in}{2.582277in}}%
\pgfpathcurveto{\pgfqpoint{0.577265in}{2.588101in}}{\pgfqpoint{0.580537in}{2.596001in}}{\pgfqpoint{0.580537in}{2.604238in}}%
\pgfpathcurveto{\pgfqpoint{0.580537in}{2.612474in}}{\pgfqpoint{0.577265in}{2.620374in}}{\pgfqpoint{0.571441in}{2.626198in}}%
\pgfpathcurveto{\pgfqpoint{0.565617in}{2.632022in}}{\pgfqpoint{0.557717in}{2.635294in}}{\pgfqpoint{0.549481in}{2.635294in}}%
\pgfpathcurveto{\pgfqpoint{0.541245in}{2.635294in}}{\pgfqpoint{0.533344in}{2.632022in}}{\pgfqpoint{0.527521in}{2.626198in}}%
\pgfpathcurveto{\pgfqpoint{0.521697in}{2.620374in}}{\pgfqpoint{0.518424in}{2.612474in}}{\pgfqpoint{0.518424in}{2.604238in}}%
\pgfpathcurveto{\pgfqpoint{0.518424in}{2.596001in}}{\pgfqpoint{0.521697in}{2.588101in}}{\pgfqpoint{0.527521in}{2.582277in}}%
\pgfpathcurveto{\pgfqpoint{0.533344in}{2.576454in}}{\pgfqpoint{0.541245in}{2.573181in}}{\pgfqpoint{0.549481in}{2.573181in}}%
\pgfpathclose%
\pgfusepath{stroke,fill}%
\end{pgfscope}%
\begin{pgfscope}%
\pgfpathrectangle{\pgfqpoint{0.100000in}{0.212622in}}{\pgfqpoint{3.696000in}{3.696000in}}%
\pgfusepath{clip}%
\pgfsetbuttcap%
\pgfsetroundjoin%
\definecolor{currentfill}{rgb}{0.121569,0.466667,0.705882}%
\pgfsetfillcolor{currentfill}%
\pgfsetfillopacity{0.827375}%
\pgfsetlinewidth{1.003750pt}%
\definecolor{currentstroke}{rgb}{0.121569,0.466667,0.705882}%
\pgfsetstrokecolor{currentstroke}%
\pgfsetstrokeopacity{0.827375}%
\pgfsetdash{}{0pt}%
\pgfpathmoveto{\pgfqpoint{0.555515in}{2.570091in}}%
\pgfpathcurveto{\pgfqpoint{0.563751in}{2.570091in}}{\pgfqpoint{0.571651in}{2.573363in}}{\pgfqpoint{0.577475in}{2.579187in}}%
\pgfpathcurveto{\pgfqpoint{0.583299in}{2.585011in}}{\pgfqpoint{0.586572in}{2.592911in}}{\pgfqpoint{0.586572in}{2.601147in}}%
\pgfpathcurveto{\pgfqpoint{0.586572in}{2.609384in}}{\pgfqpoint{0.583299in}{2.617284in}}{\pgfqpoint{0.577475in}{2.623108in}}%
\pgfpathcurveto{\pgfqpoint{0.571651in}{2.628932in}}{\pgfqpoint{0.563751in}{2.632204in}}{\pgfqpoint{0.555515in}{2.632204in}}%
\pgfpathcurveto{\pgfqpoint{0.547279in}{2.632204in}}{\pgfqpoint{0.539379in}{2.628932in}}{\pgfqpoint{0.533555in}{2.623108in}}%
\pgfpathcurveto{\pgfqpoint{0.527731in}{2.617284in}}{\pgfqpoint{0.524459in}{2.609384in}}{\pgfqpoint{0.524459in}{2.601147in}}%
\pgfpathcurveto{\pgfqpoint{0.524459in}{2.592911in}}{\pgfqpoint{0.527731in}{2.585011in}}{\pgfqpoint{0.533555in}{2.579187in}}%
\pgfpathcurveto{\pgfqpoint{0.539379in}{2.573363in}}{\pgfqpoint{0.547279in}{2.570091in}}{\pgfqpoint{0.555515in}{2.570091in}}%
\pgfpathclose%
\pgfusepath{stroke,fill}%
\end{pgfscope}%
\begin{pgfscope}%
\pgfpathrectangle{\pgfqpoint{0.100000in}{0.212622in}}{\pgfqpoint{3.696000in}{3.696000in}}%
\pgfusepath{clip}%
\pgfsetbuttcap%
\pgfsetroundjoin%
\definecolor{currentfill}{rgb}{0.121569,0.466667,0.705882}%
\pgfsetfillcolor{currentfill}%
\pgfsetfillopacity{0.827781}%
\pgfsetlinewidth{1.003750pt}%
\definecolor{currentstroke}{rgb}{0.121569,0.466667,0.705882}%
\pgfsetstrokecolor{currentstroke}%
\pgfsetstrokeopacity{0.827781}%
\pgfsetdash{}{0pt}%
\pgfpathmoveto{\pgfqpoint{0.560167in}{2.567155in}}%
\pgfpathcurveto{\pgfqpoint{0.568403in}{2.567155in}}{\pgfqpoint{0.576303in}{2.570427in}}{\pgfqpoint{0.582127in}{2.576251in}}%
\pgfpathcurveto{\pgfqpoint{0.587951in}{2.582075in}}{\pgfqpoint{0.591223in}{2.589975in}}{\pgfqpoint{0.591223in}{2.598211in}}%
\pgfpathcurveto{\pgfqpoint{0.591223in}{2.606447in}}{\pgfqpoint{0.587951in}{2.614348in}}{\pgfqpoint{0.582127in}{2.620171in}}%
\pgfpathcurveto{\pgfqpoint{0.576303in}{2.625995in}}{\pgfqpoint{0.568403in}{2.629268in}}{\pgfqpoint{0.560167in}{2.629268in}}%
\pgfpathcurveto{\pgfqpoint{0.551930in}{2.629268in}}{\pgfqpoint{0.544030in}{2.625995in}}{\pgfqpoint{0.538206in}{2.620171in}}%
\pgfpathcurveto{\pgfqpoint{0.532382in}{2.614348in}}{\pgfqpoint{0.529110in}{2.606447in}}{\pgfqpoint{0.529110in}{2.598211in}}%
\pgfpathcurveto{\pgfqpoint{0.529110in}{2.589975in}}{\pgfqpoint{0.532382in}{2.582075in}}{\pgfqpoint{0.538206in}{2.576251in}}%
\pgfpathcurveto{\pgfqpoint{0.544030in}{2.570427in}}{\pgfqpoint{0.551930in}{2.567155in}}{\pgfqpoint{0.560167in}{2.567155in}}%
\pgfpathclose%
\pgfusepath{stroke,fill}%
\end{pgfscope}%
\begin{pgfscope}%
\pgfpathrectangle{\pgfqpoint{0.100000in}{0.212622in}}{\pgfqpoint{3.696000in}{3.696000in}}%
\pgfusepath{clip}%
\pgfsetbuttcap%
\pgfsetroundjoin%
\definecolor{currentfill}{rgb}{0.121569,0.466667,0.705882}%
\pgfsetfillcolor{currentfill}%
\pgfsetfillopacity{0.827948}%
\pgfsetlinewidth{1.003750pt}%
\definecolor{currentstroke}{rgb}{0.121569,0.466667,0.705882}%
\pgfsetstrokecolor{currentstroke}%
\pgfsetstrokeopacity{0.827948}%
\pgfsetdash{}{0pt}%
\pgfpathmoveto{\pgfqpoint{0.561876in}{2.565890in}}%
\pgfpathcurveto{\pgfqpoint{0.570113in}{2.565890in}}{\pgfqpoint{0.578013in}{2.569162in}}{\pgfqpoint{0.583837in}{2.574986in}}%
\pgfpathcurveto{\pgfqpoint{0.589661in}{2.580810in}}{\pgfqpoint{0.592933in}{2.588710in}}{\pgfqpoint{0.592933in}{2.596946in}}%
\pgfpathcurveto{\pgfqpoint{0.592933in}{2.605182in}}{\pgfqpoint{0.589661in}{2.613083in}}{\pgfqpoint{0.583837in}{2.618906in}}%
\pgfpathcurveto{\pgfqpoint{0.578013in}{2.624730in}}{\pgfqpoint{0.570113in}{2.628003in}}{\pgfqpoint{0.561876in}{2.628003in}}%
\pgfpathcurveto{\pgfqpoint{0.553640in}{2.628003in}}{\pgfqpoint{0.545740in}{2.624730in}}{\pgfqpoint{0.539916in}{2.618906in}}%
\pgfpathcurveto{\pgfqpoint{0.534092in}{2.613083in}}{\pgfqpoint{0.530820in}{2.605182in}}{\pgfqpoint{0.530820in}{2.596946in}}%
\pgfpathcurveto{\pgfqpoint{0.530820in}{2.588710in}}{\pgfqpoint{0.534092in}{2.580810in}}{\pgfqpoint{0.539916in}{2.574986in}}%
\pgfpathcurveto{\pgfqpoint{0.545740in}{2.569162in}}{\pgfqpoint{0.553640in}{2.565890in}}{\pgfqpoint{0.561876in}{2.565890in}}%
\pgfpathclose%
\pgfusepath{stroke,fill}%
\end{pgfscope}%
\begin{pgfscope}%
\pgfpathrectangle{\pgfqpoint{0.100000in}{0.212622in}}{\pgfqpoint{3.696000in}{3.696000in}}%
\pgfusepath{clip}%
\pgfsetbuttcap%
\pgfsetroundjoin%
\definecolor{currentfill}{rgb}{0.121569,0.466667,0.705882}%
\pgfsetfillcolor{currentfill}%
\pgfsetfillopacity{0.828056}%
\pgfsetlinewidth{1.003750pt}%
\definecolor{currentstroke}{rgb}{0.121569,0.466667,0.705882}%
\pgfsetstrokecolor{currentstroke}%
\pgfsetstrokeopacity{0.828056}%
\pgfsetdash{}{0pt}%
\pgfpathmoveto{\pgfqpoint{2.786140in}{2.003522in}}%
\pgfpathcurveto{\pgfqpoint{2.794376in}{2.003522in}}{\pgfqpoint{2.802276in}{2.006794in}}{\pgfqpoint{2.808100in}{2.012618in}}%
\pgfpathcurveto{\pgfqpoint{2.813924in}{2.018442in}}{\pgfqpoint{2.817196in}{2.026342in}}{\pgfqpoint{2.817196in}{2.034578in}}%
\pgfpathcurveto{\pgfqpoint{2.817196in}{2.042815in}}{\pgfqpoint{2.813924in}{2.050715in}}{\pgfqpoint{2.808100in}{2.056539in}}%
\pgfpathcurveto{\pgfqpoint{2.802276in}{2.062362in}}{\pgfqpoint{2.794376in}{2.065635in}}{\pgfqpoint{2.786140in}{2.065635in}}%
\pgfpathcurveto{\pgfqpoint{2.777903in}{2.065635in}}{\pgfqpoint{2.770003in}{2.062362in}}{\pgfqpoint{2.764179in}{2.056539in}}%
\pgfpathcurveto{\pgfqpoint{2.758355in}{2.050715in}}{\pgfqpoint{2.755083in}{2.042815in}}{\pgfqpoint{2.755083in}{2.034578in}}%
\pgfpathcurveto{\pgfqpoint{2.755083in}{2.026342in}}{\pgfqpoint{2.758355in}{2.018442in}}{\pgfqpoint{2.764179in}{2.012618in}}%
\pgfpathcurveto{\pgfqpoint{2.770003in}{2.006794in}}{\pgfqpoint{2.777903in}{2.003522in}}{\pgfqpoint{2.786140in}{2.003522in}}%
\pgfpathclose%
\pgfusepath{stroke,fill}%
\end{pgfscope}%
\begin{pgfscope}%
\pgfpathrectangle{\pgfqpoint{0.100000in}{0.212622in}}{\pgfqpoint{3.696000in}{3.696000in}}%
\pgfusepath{clip}%
\pgfsetbuttcap%
\pgfsetroundjoin%
\definecolor{currentfill}{rgb}{0.121569,0.466667,0.705882}%
\pgfsetfillcolor{currentfill}%
\pgfsetfillopacity{0.828327}%
\pgfsetlinewidth{1.003750pt}%
\definecolor{currentstroke}{rgb}{0.121569,0.466667,0.705882}%
\pgfsetstrokecolor{currentstroke}%
\pgfsetstrokeopacity{0.828327}%
\pgfsetdash{}{0pt}%
\pgfpathmoveto{\pgfqpoint{0.565018in}{2.564236in}}%
\pgfpathcurveto{\pgfqpoint{0.573254in}{2.564236in}}{\pgfqpoint{0.581154in}{2.567508in}}{\pgfqpoint{0.586978in}{2.573332in}}%
\pgfpathcurveto{\pgfqpoint{0.592802in}{2.579156in}}{\pgfqpoint{0.596075in}{2.587056in}}{\pgfqpoint{0.596075in}{2.595292in}}%
\pgfpathcurveto{\pgfqpoint{0.596075in}{2.603528in}}{\pgfqpoint{0.592802in}{2.611428in}}{\pgfqpoint{0.586978in}{2.617252in}}%
\pgfpathcurveto{\pgfqpoint{0.581154in}{2.623076in}}{\pgfqpoint{0.573254in}{2.626349in}}{\pgfqpoint{0.565018in}{2.626349in}}%
\pgfpathcurveto{\pgfqpoint{0.556782in}{2.626349in}}{\pgfqpoint{0.548882in}{2.623076in}}{\pgfqpoint{0.543058in}{2.617252in}}%
\pgfpathcurveto{\pgfqpoint{0.537234in}{2.611428in}}{\pgfqpoint{0.533962in}{2.603528in}}{\pgfqpoint{0.533962in}{2.595292in}}%
\pgfpathcurveto{\pgfqpoint{0.533962in}{2.587056in}}{\pgfqpoint{0.537234in}{2.579156in}}{\pgfqpoint{0.543058in}{2.573332in}}%
\pgfpathcurveto{\pgfqpoint{0.548882in}{2.567508in}}{\pgfqpoint{0.556782in}{2.564236in}}{\pgfqpoint{0.565018in}{2.564236in}}%
\pgfpathclose%
\pgfusepath{stroke,fill}%
\end{pgfscope}%
\begin{pgfscope}%
\pgfpathrectangle{\pgfqpoint{0.100000in}{0.212622in}}{\pgfqpoint{3.696000in}{3.696000in}}%
\pgfusepath{clip}%
\pgfsetbuttcap%
\pgfsetroundjoin%
\definecolor{currentfill}{rgb}{0.121569,0.466667,0.705882}%
\pgfsetfillcolor{currentfill}%
\pgfsetfillopacity{0.828647}%
\pgfsetlinewidth{1.003750pt}%
\definecolor{currentstroke}{rgb}{0.121569,0.466667,0.705882}%
\pgfsetstrokecolor{currentstroke}%
\pgfsetstrokeopacity{0.828647}%
\pgfsetdash{}{0pt}%
\pgfpathmoveto{\pgfqpoint{0.567487in}{2.563430in}}%
\pgfpathcurveto{\pgfqpoint{0.575723in}{2.563430in}}{\pgfqpoint{0.583623in}{2.566702in}}{\pgfqpoint{0.589447in}{2.572526in}}%
\pgfpathcurveto{\pgfqpoint{0.595271in}{2.578350in}}{\pgfqpoint{0.598544in}{2.586250in}}{\pgfqpoint{0.598544in}{2.594486in}}%
\pgfpathcurveto{\pgfqpoint{0.598544in}{2.602722in}}{\pgfqpoint{0.595271in}{2.610622in}}{\pgfqpoint{0.589447in}{2.616446in}}%
\pgfpathcurveto{\pgfqpoint{0.583623in}{2.622270in}}{\pgfqpoint{0.575723in}{2.625543in}}{\pgfqpoint{0.567487in}{2.625543in}}%
\pgfpathcurveto{\pgfqpoint{0.559251in}{2.625543in}}{\pgfqpoint{0.551351in}{2.622270in}}{\pgfqpoint{0.545527in}{2.616446in}}%
\pgfpathcurveto{\pgfqpoint{0.539703in}{2.610622in}}{\pgfqpoint{0.536431in}{2.602722in}}{\pgfqpoint{0.536431in}{2.594486in}}%
\pgfpathcurveto{\pgfqpoint{0.536431in}{2.586250in}}{\pgfqpoint{0.539703in}{2.578350in}}{\pgfqpoint{0.545527in}{2.572526in}}%
\pgfpathcurveto{\pgfqpoint{0.551351in}{2.566702in}}{\pgfqpoint{0.559251in}{2.563430in}}{\pgfqpoint{0.567487in}{2.563430in}}%
\pgfpathclose%
\pgfusepath{stroke,fill}%
\end{pgfscope}%
\begin{pgfscope}%
\pgfpathrectangle{\pgfqpoint{0.100000in}{0.212622in}}{\pgfqpoint{3.696000in}{3.696000in}}%
\pgfusepath{clip}%
\pgfsetbuttcap%
\pgfsetroundjoin%
\definecolor{currentfill}{rgb}{0.121569,0.466667,0.705882}%
\pgfsetfillcolor{currentfill}%
\pgfsetfillopacity{0.829019}%
\pgfsetlinewidth{1.003750pt}%
\definecolor{currentstroke}{rgb}{0.121569,0.466667,0.705882}%
\pgfsetstrokecolor{currentstroke}%
\pgfsetstrokeopacity{0.829019}%
\pgfsetdash{}{0pt}%
\pgfpathmoveto{\pgfqpoint{2.784320in}{2.001632in}}%
\pgfpathcurveto{\pgfqpoint{2.792557in}{2.001632in}}{\pgfqpoint{2.800457in}{2.004904in}}{\pgfqpoint{2.806281in}{2.010728in}}%
\pgfpathcurveto{\pgfqpoint{2.812105in}{2.016552in}}{\pgfqpoint{2.815377in}{2.024452in}}{\pgfqpoint{2.815377in}{2.032688in}}%
\pgfpathcurveto{\pgfqpoint{2.815377in}{2.040924in}}{\pgfqpoint{2.812105in}{2.048825in}}{\pgfqpoint{2.806281in}{2.054648in}}%
\pgfpathcurveto{\pgfqpoint{2.800457in}{2.060472in}}{\pgfqpoint{2.792557in}{2.063745in}}{\pgfqpoint{2.784320in}{2.063745in}}%
\pgfpathcurveto{\pgfqpoint{2.776084in}{2.063745in}}{\pgfqpoint{2.768184in}{2.060472in}}{\pgfqpoint{2.762360in}{2.054648in}}%
\pgfpathcurveto{\pgfqpoint{2.756536in}{2.048825in}}{\pgfqpoint{2.753264in}{2.040924in}}{\pgfqpoint{2.753264in}{2.032688in}}%
\pgfpathcurveto{\pgfqpoint{2.753264in}{2.024452in}}{\pgfqpoint{2.756536in}{2.016552in}}{\pgfqpoint{2.762360in}{2.010728in}}%
\pgfpathcurveto{\pgfqpoint{2.768184in}{2.004904in}}{\pgfqpoint{2.776084in}{2.001632in}}{\pgfqpoint{2.784320in}{2.001632in}}%
\pgfpathclose%
\pgfusepath{stroke,fill}%
\end{pgfscope}%
\begin{pgfscope}%
\pgfpathrectangle{\pgfqpoint{0.100000in}{0.212622in}}{\pgfqpoint{3.696000in}{3.696000in}}%
\pgfusepath{clip}%
\pgfsetbuttcap%
\pgfsetroundjoin%
\definecolor{currentfill}{rgb}{0.121569,0.466667,0.705882}%
\pgfsetfillcolor{currentfill}%
\pgfsetfillopacity{0.829172}%
\pgfsetlinewidth{1.003750pt}%
\definecolor{currentstroke}{rgb}{0.121569,0.466667,0.705882}%
\pgfsetstrokecolor{currentstroke}%
\pgfsetstrokeopacity{0.829172}%
\pgfsetdash{}{0pt}%
\pgfpathmoveto{\pgfqpoint{0.571989in}{2.561601in}}%
\pgfpathcurveto{\pgfqpoint{0.580225in}{2.561601in}}{\pgfqpoint{0.588125in}{2.564873in}}{\pgfqpoint{0.593949in}{2.570697in}}%
\pgfpathcurveto{\pgfqpoint{0.599773in}{2.576521in}}{\pgfqpoint{0.603045in}{2.584421in}}{\pgfqpoint{0.603045in}{2.592658in}}%
\pgfpathcurveto{\pgfqpoint{0.603045in}{2.600894in}}{\pgfqpoint{0.599773in}{2.608794in}}{\pgfqpoint{0.593949in}{2.614618in}}%
\pgfpathcurveto{\pgfqpoint{0.588125in}{2.620442in}}{\pgfqpoint{0.580225in}{2.623714in}}{\pgfqpoint{0.571989in}{2.623714in}}%
\pgfpathcurveto{\pgfqpoint{0.563753in}{2.623714in}}{\pgfqpoint{0.555853in}{2.620442in}}{\pgfqpoint{0.550029in}{2.614618in}}%
\pgfpathcurveto{\pgfqpoint{0.544205in}{2.608794in}}{\pgfqpoint{0.540932in}{2.600894in}}{\pgfqpoint{0.540932in}{2.592658in}}%
\pgfpathcurveto{\pgfqpoint{0.540932in}{2.584421in}}{\pgfqpoint{0.544205in}{2.576521in}}{\pgfqpoint{0.550029in}{2.570697in}}%
\pgfpathcurveto{\pgfqpoint{0.555853in}{2.564873in}}{\pgfqpoint{0.563753in}{2.561601in}}{\pgfqpoint{0.571989in}{2.561601in}}%
\pgfpathclose%
\pgfusepath{stroke,fill}%
\end{pgfscope}%
\begin{pgfscope}%
\pgfpathrectangle{\pgfqpoint{0.100000in}{0.212622in}}{\pgfqpoint{3.696000in}{3.696000in}}%
\pgfusepath{clip}%
\pgfsetbuttcap%
\pgfsetroundjoin%
\definecolor{currentfill}{rgb}{0.121569,0.466667,0.705882}%
\pgfsetfillcolor{currentfill}%
\pgfsetfillopacity{0.829934}%
\pgfsetlinewidth{1.003750pt}%
\definecolor{currentstroke}{rgb}{0.121569,0.466667,0.705882}%
\pgfsetstrokecolor{currentstroke}%
\pgfsetstrokeopacity{0.829934}%
\pgfsetdash{}{0pt}%
\pgfpathmoveto{\pgfqpoint{0.580285in}{2.557253in}}%
\pgfpathcurveto{\pgfqpoint{0.588521in}{2.557253in}}{\pgfqpoint{0.596421in}{2.560525in}}{\pgfqpoint{0.602245in}{2.566349in}}%
\pgfpathcurveto{\pgfqpoint{0.608069in}{2.572173in}}{\pgfqpoint{0.611342in}{2.580073in}}{\pgfqpoint{0.611342in}{2.588310in}}%
\pgfpathcurveto{\pgfqpoint{0.611342in}{2.596546in}}{\pgfqpoint{0.608069in}{2.604446in}}{\pgfqpoint{0.602245in}{2.610270in}}%
\pgfpathcurveto{\pgfqpoint{0.596421in}{2.616094in}}{\pgfqpoint{0.588521in}{2.619366in}}{\pgfqpoint{0.580285in}{2.619366in}}%
\pgfpathcurveto{\pgfqpoint{0.572049in}{2.619366in}}{\pgfqpoint{0.564149in}{2.616094in}}{\pgfqpoint{0.558325in}{2.610270in}}%
\pgfpathcurveto{\pgfqpoint{0.552501in}{2.604446in}}{\pgfqpoint{0.549229in}{2.596546in}}{\pgfqpoint{0.549229in}{2.588310in}}%
\pgfpathcurveto{\pgfqpoint{0.549229in}{2.580073in}}{\pgfqpoint{0.552501in}{2.572173in}}{\pgfqpoint{0.558325in}{2.566349in}}%
\pgfpathcurveto{\pgfqpoint{0.564149in}{2.560525in}}{\pgfqpoint{0.572049in}{2.557253in}}{\pgfqpoint{0.580285in}{2.557253in}}%
\pgfpathclose%
\pgfusepath{stroke,fill}%
\end{pgfscope}%
\begin{pgfscope}%
\pgfpathrectangle{\pgfqpoint{0.100000in}{0.212622in}}{\pgfqpoint{3.696000in}{3.696000in}}%
\pgfusepath{clip}%
\pgfsetbuttcap%
\pgfsetroundjoin%
\definecolor{currentfill}{rgb}{0.121569,0.466667,0.705882}%
\pgfsetfillcolor{currentfill}%
\pgfsetfillopacity{0.830361}%
\pgfsetlinewidth{1.003750pt}%
\definecolor{currentstroke}{rgb}{0.121569,0.466667,0.705882}%
\pgfsetstrokecolor{currentstroke}%
\pgfsetstrokeopacity{0.830361}%
\pgfsetdash{}{0pt}%
\pgfpathmoveto{\pgfqpoint{2.781548in}{1.998933in}}%
\pgfpathcurveto{\pgfqpoint{2.789785in}{1.998933in}}{\pgfqpoint{2.797685in}{2.002205in}}{\pgfqpoint{2.803509in}{2.008029in}}%
\pgfpathcurveto{\pgfqpoint{2.809333in}{2.013853in}}{\pgfqpoint{2.812605in}{2.021753in}}{\pgfqpoint{2.812605in}{2.029990in}}%
\pgfpathcurveto{\pgfqpoint{2.812605in}{2.038226in}}{\pgfqpoint{2.809333in}{2.046126in}}{\pgfqpoint{2.803509in}{2.051950in}}%
\pgfpathcurveto{\pgfqpoint{2.797685in}{2.057774in}}{\pgfqpoint{2.789785in}{2.061046in}}{\pgfqpoint{2.781548in}{2.061046in}}%
\pgfpathcurveto{\pgfqpoint{2.773312in}{2.061046in}}{\pgfqpoint{2.765412in}{2.057774in}}{\pgfqpoint{2.759588in}{2.051950in}}%
\pgfpathcurveto{\pgfqpoint{2.753764in}{2.046126in}}{\pgfqpoint{2.750492in}{2.038226in}}{\pgfqpoint{2.750492in}{2.029990in}}%
\pgfpathcurveto{\pgfqpoint{2.750492in}{2.021753in}}{\pgfqpoint{2.753764in}{2.013853in}}{\pgfqpoint{2.759588in}{2.008029in}}%
\pgfpathcurveto{\pgfqpoint{2.765412in}{2.002205in}}{\pgfqpoint{2.773312in}{1.998933in}}{\pgfqpoint{2.781548in}{1.998933in}}%
\pgfpathclose%
\pgfusepath{stroke,fill}%
\end{pgfscope}%
\begin{pgfscope}%
\pgfpathrectangle{\pgfqpoint{0.100000in}{0.212622in}}{\pgfqpoint{3.696000in}{3.696000in}}%
\pgfusepath{clip}%
\pgfsetbuttcap%
\pgfsetroundjoin%
\definecolor{currentfill}{rgb}{0.121569,0.466667,0.705882}%
\pgfsetfillcolor{currentfill}%
\pgfsetfillopacity{0.831114}%
\pgfsetlinewidth{1.003750pt}%
\definecolor{currentstroke}{rgb}{0.121569,0.466667,0.705882}%
\pgfsetstrokecolor{currentstroke}%
\pgfsetstrokeopacity{0.831114}%
\pgfsetdash{}{0pt}%
\pgfpathmoveto{\pgfqpoint{0.595445in}{2.548116in}}%
\pgfpathcurveto{\pgfqpoint{0.603681in}{2.548116in}}{\pgfqpoint{0.611581in}{2.551388in}}{\pgfqpoint{0.617405in}{2.557212in}}%
\pgfpathcurveto{\pgfqpoint{0.623229in}{2.563036in}}{\pgfqpoint{0.626502in}{2.570936in}}{\pgfqpoint{0.626502in}{2.579172in}}%
\pgfpathcurveto{\pgfqpoint{0.626502in}{2.587408in}}{\pgfqpoint{0.623229in}{2.595308in}}{\pgfqpoint{0.617405in}{2.601132in}}%
\pgfpathcurveto{\pgfqpoint{0.611581in}{2.606956in}}{\pgfqpoint{0.603681in}{2.610229in}}{\pgfqpoint{0.595445in}{2.610229in}}%
\pgfpathcurveto{\pgfqpoint{0.587209in}{2.610229in}}{\pgfqpoint{0.579309in}{2.606956in}}{\pgfqpoint{0.573485in}{2.601132in}}%
\pgfpathcurveto{\pgfqpoint{0.567661in}{2.595308in}}{\pgfqpoint{0.564389in}{2.587408in}}{\pgfqpoint{0.564389in}{2.579172in}}%
\pgfpathcurveto{\pgfqpoint{0.564389in}{2.570936in}}{\pgfqpoint{0.567661in}{2.563036in}}{\pgfqpoint{0.573485in}{2.557212in}}%
\pgfpathcurveto{\pgfqpoint{0.579309in}{2.551388in}}{\pgfqpoint{0.587209in}{2.548116in}}{\pgfqpoint{0.595445in}{2.548116in}}%
\pgfpathclose%
\pgfusepath{stroke,fill}%
\end{pgfscope}%
\begin{pgfscope}%
\pgfpathrectangle{\pgfqpoint{0.100000in}{0.212622in}}{\pgfqpoint{3.696000in}{3.696000in}}%
\pgfusepath{clip}%
\pgfsetbuttcap%
\pgfsetroundjoin%
\definecolor{currentfill}{rgb}{0.121569,0.466667,0.705882}%
\pgfsetfillcolor{currentfill}%
\pgfsetfillopacity{0.832358}%
\pgfsetlinewidth{1.003750pt}%
\definecolor{currentstroke}{rgb}{0.121569,0.466667,0.705882}%
\pgfsetstrokecolor{currentstroke}%
\pgfsetstrokeopacity{0.832358}%
\pgfsetdash{}{0pt}%
\pgfpathmoveto{\pgfqpoint{2.778122in}{1.995423in}}%
\pgfpathcurveto{\pgfqpoint{2.786358in}{1.995423in}}{\pgfqpoint{2.794258in}{1.998695in}}{\pgfqpoint{2.800082in}{2.004519in}}%
\pgfpathcurveto{\pgfqpoint{2.805906in}{2.010343in}}{\pgfqpoint{2.809179in}{2.018243in}}{\pgfqpoint{2.809179in}{2.026479in}}%
\pgfpathcurveto{\pgfqpoint{2.809179in}{2.034716in}}{\pgfqpoint{2.805906in}{2.042616in}}{\pgfqpoint{2.800082in}{2.048440in}}%
\pgfpathcurveto{\pgfqpoint{2.794258in}{2.054264in}}{\pgfqpoint{2.786358in}{2.057536in}}{\pgfqpoint{2.778122in}{2.057536in}}%
\pgfpathcurveto{\pgfqpoint{2.769886in}{2.057536in}}{\pgfqpoint{2.761986in}{2.054264in}}{\pgfqpoint{2.756162in}{2.048440in}}%
\pgfpathcurveto{\pgfqpoint{2.750338in}{2.042616in}}{\pgfqpoint{2.747066in}{2.034716in}}{\pgfqpoint{2.747066in}{2.026479in}}%
\pgfpathcurveto{\pgfqpoint{2.747066in}{2.018243in}}{\pgfqpoint{2.750338in}{2.010343in}}{\pgfqpoint{2.756162in}{2.004519in}}%
\pgfpathcurveto{\pgfqpoint{2.761986in}{1.998695in}}{\pgfqpoint{2.769886in}{1.995423in}}{\pgfqpoint{2.778122in}{1.995423in}}%
\pgfpathclose%
\pgfusepath{stroke,fill}%
\end{pgfscope}%
\begin{pgfscope}%
\pgfpathrectangle{\pgfqpoint{0.100000in}{0.212622in}}{\pgfqpoint{3.696000in}{3.696000in}}%
\pgfusepath{clip}%
\pgfsetbuttcap%
\pgfsetroundjoin%
\definecolor{currentfill}{rgb}{0.121569,0.466667,0.705882}%
\pgfsetfillcolor{currentfill}%
\pgfsetfillopacity{0.832481}%
\pgfsetlinewidth{1.003750pt}%
\definecolor{currentstroke}{rgb}{0.121569,0.466667,0.705882}%
\pgfsetstrokecolor{currentstroke}%
\pgfsetstrokeopacity{0.832481}%
\pgfsetdash{}{0pt}%
\pgfpathmoveto{\pgfqpoint{0.610067in}{2.541206in}}%
\pgfpathcurveto{\pgfqpoint{0.618304in}{2.541206in}}{\pgfqpoint{0.626204in}{2.544479in}}{\pgfqpoint{0.632028in}{2.550302in}}%
\pgfpathcurveto{\pgfqpoint{0.637851in}{2.556126in}}{\pgfqpoint{0.641124in}{2.564026in}}{\pgfqpoint{0.641124in}{2.572263in}}%
\pgfpathcurveto{\pgfqpoint{0.641124in}{2.580499in}}{\pgfqpoint{0.637851in}{2.588399in}}{\pgfqpoint{0.632028in}{2.594223in}}%
\pgfpathcurveto{\pgfqpoint{0.626204in}{2.600047in}}{\pgfqpoint{0.618304in}{2.603319in}}{\pgfqpoint{0.610067in}{2.603319in}}%
\pgfpathcurveto{\pgfqpoint{0.601831in}{2.603319in}}{\pgfqpoint{0.593931in}{2.600047in}}{\pgfqpoint{0.588107in}{2.594223in}}%
\pgfpathcurveto{\pgfqpoint{0.582283in}{2.588399in}}{\pgfqpoint{0.579011in}{2.580499in}}{\pgfqpoint{0.579011in}{2.572263in}}%
\pgfpathcurveto{\pgfqpoint{0.579011in}{2.564026in}}{\pgfqpoint{0.582283in}{2.556126in}}{\pgfqpoint{0.588107in}{2.550302in}}%
\pgfpathcurveto{\pgfqpoint{0.593931in}{2.544479in}}{\pgfqpoint{0.601831in}{2.541206in}}{\pgfqpoint{0.610067in}{2.541206in}}%
\pgfpathclose%
\pgfusepath{stroke,fill}%
\end{pgfscope}%
\begin{pgfscope}%
\pgfpathrectangle{\pgfqpoint{0.100000in}{0.212622in}}{\pgfqpoint{3.696000in}{3.696000in}}%
\pgfusepath{clip}%
\pgfsetbuttcap%
\pgfsetroundjoin%
\definecolor{currentfill}{rgb}{0.121569,0.466667,0.705882}%
\pgfsetfillcolor{currentfill}%
\pgfsetfillopacity{0.833527}%
\pgfsetlinewidth{1.003750pt}%
\definecolor{currentstroke}{rgb}{0.121569,0.466667,0.705882}%
\pgfsetstrokecolor{currentstroke}%
\pgfsetstrokeopacity{0.833527}%
\pgfsetdash{}{0pt}%
\pgfpathmoveto{\pgfqpoint{2.776261in}{1.993915in}}%
\pgfpathcurveto{\pgfqpoint{2.784498in}{1.993915in}}{\pgfqpoint{2.792398in}{1.997188in}}{\pgfqpoint{2.798222in}{2.003012in}}%
\pgfpathcurveto{\pgfqpoint{2.804046in}{2.008835in}}{\pgfqpoint{2.807318in}{2.016735in}}{\pgfqpoint{2.807318in}{2.024972in}}%
\pgfpathcurveto{\pgfqpoint{2.807318in}{2.033208in}}{\pgfqpoint{2.804046in}{2.041108in}}{\pgfqpoint{2.798222in}{2.046932in}}%
\pgfpathcurveto{\pgfqpoint{2.792398in}{2.052756in}}{\pgfqpoint{2.784498in}{2.056028in}}{\pgfqpoint{2.776261in}{2.056028in}}%
\pgfpathcurveto{\pgfqpoint{2.768025in}{2.056028in}}{\pgfqpoint{2.760125in}{2.052756in}}{\pgfqpoint{2.754301in}{2.046932in}}%
\pgfpathcurveto{\pgfqpoint{2.748477in}{2.041108in}}{\pgfqpoint{2.745205in}{2.033208in}}{\pgfqpoint{2.745205in}{2.024972in}}%
\pgfpathcurveto{\pgfqpoint{2.745205in}{2.016735in}}{\pgfqpoint{2.748477in}{2.008835in}}{\pgfqpoint{2.754301in}{2.003012in}}%
\pgfpathcurveto{\pgfqpoint{2.760125in}{1.997188in}}{\pgfqpoint{2.768025in}{1.993915in}}{\pgfqpoint{2.776261in}{1.993915in}}%
\pgfpathclose%
\pgfusepath{stroke,fill}%
\end{pgfscope}%
\begin{pgfscope}%
\pgfpathrectangle{\pgfqpoint{0.100000in}{0.212622in}}{\pgfqpoint{3.696000in}{3.696000in}}%
\pgfusepath{clip}%
\pgfsetbuttcap%
\pgfsetroundjoin%
\definecolor{currentfill}{rgb}{0.121569,0.466667,0.705882}%
\pgfsetfillcolor{currentfill}%
\pgfsetfillopacity{0.833706}%
\pgfsetlinewidth{1.003750pt}%
\definecolor{currentstroke}{rgb}{0.121569,0.466667,0.705882}%
\pgfsetstrokecolor{currentstroke}%
\pgfsetstrokeopacity{0.833706}%
\pgfsetdash{}{0pt}%
\pgfpathmoveto{\pgfqpoint{0.621932in}{2.535903in}}%
\pgfpathcurveto{\pgfqpoint{0.630168in}{2.535903in}}{\pgfqpoint{0.638068in}{2.539175in}}{\pgfqpoint{0.643892in}{2.544999in}}%
\pgfpathcurveto{\pgfqpoint{0.649716in}{2.550823in}}{\pgfqpoint{0.652988in}{2.558723in}}{\pgfqpoint{0.652988in}{2.566959in}}%
\pgfpathcurveto{\pgfqpoint{0.652988in}{2.575196in}}{\pgfqpoint{0.649716in}{2.583096in}}{\pgfqpoint{0.643892in}{2.588920in}}%
\pgfpathcurveto{\pgfqpoint{0.638068in}{2.594744in}}{\pgfqpoint{0.630168in}{2.598016in}}{\pgfqpoint{0.621932in}{2.598016in}}%
\pgfpathcurveto{\pgfqpoint{0.613695in}{2.598016in}}{\pgfqpoint{0.605795in}{2.594744in}}{\pgfqpoint{0.599971in}{2.588920in}}%
\pgfpathcurveto{\pgfqpoint{0.594147in}{2.583096in}}{\pgfqpoint{0.590875in}{2.575196in}}{\pgfqpoint{0.590875in}{2.566959in}}%
\pgfpathcurveto{\pgfqpoint{0.590875in}{2.558723in}}{\pgfqpoint{0.594147in}{2.550823in}}{\pgfqpoint{0.599971in}{2.544999in}}%
\pgfpathcurveto{\pgfqpoint{0.605795in}{2.539175in}}{\pgfqpoint{0.613695in}{2.535903in}}{\pgfqpoint{0.621932in}{2.535903in}}%
\pgfpathclose%
\pgfusepath{stroke,fill}%
\end{pgfscope}%
\begin{pgfscope}%
\pgfpathrectangle{\pgfqpoint{0.100000in}{0.212622in}}{\pgfqpoint{3.696000in}{3.696000in}}%
\pgfusepath{clip}%
\pgfsetbuttcap%
\pgfsetroundjoin%
\definecolor{currentfill}{rgb}{0.121569,0.466667,0.705882}%
\pgfsetfillcolor{currentfill}%
\pgfsetfillopacity{0.834742}%
\pgfsetlinewidth{1.003750pt}%
\definecolor{currentstroke}{rgb}{0.121569,0.466667,0.705882}%
\pgfsetstrokecolor{currentstroke}%
\pgfsetstrokeopacity{0.834742}%
\pgfsetdash{}{0pt}%
\pgfpathmoveto{\pgfqpoint{2.773991in}{1.991867in}}%
\pgfpathcurveto{\pgfqpoint{2.782227in}{1.991867in}}{\pgfqpoint{2.790128in}{1.995139in}}{\pgfqpoint{2.795951in}{2.000963in}}%
\pgfpathcurveto{\pgfqpoint{2.801775in}{2.006787in}}{\pgfqpoint{2.805048in}{2.014687in}}{\pgfqpoint{2.805048in}{2.022923in}}%
\pgfpathcurveto{\pgfqpoint{2.805048in}{2.031160in}}{\pgfqpoint{2.801775in}{2.039060in}}{\pgfqpoint{2.795951in}{2.044884in}}%
\pgfpathcurveto{\pgfqpoint{2.790128in}{2.050707in}}{\pgfqpoint{2.782227in}{2.053980in}}{\pgfqpoint{2.773991in}{2.053980in}}%
\pgfpathcurveto{\pgfqpoint{2.765755in}{2.053980in}}{\pgfqpoint{2.757855in}{2.050707in}}{\pgfqpoint{2.752031in}{2.044884in}}%
\pgfpathcurveto{\pgfqpoint{2.746207in}{2.039060in}}{\pgfqpoint{2.742935in}{2.031160in}}{\pgfqpoint{2.742935in}{2.022923in}}%
\pgfpathcurveto{\pgfqpoint{2.742935in}{2.014687in}}{\pgfqpoint{2.746207in}{2.006787in}}{\pgfqpoint{2.752031in}{2.000963in}}%
\pgfpathcurveto{\pgfqpoint{2.757855in}{1.995139in}}{\pgfqpoint{2.765755in}{1.991867in}}{\pgfqpoint{2.773991in}{1.991867in}}%
\pgfpathclose%
\pgfusepath{stroke,fill}%
\end{pgfscope}%
\begin{pgfscope}%
\pgfpathrectangle{\pgfqpoint{0.100000in}{0.212622in}}{\pgfqpoint{3.696000in}{3.696000in}}%
\pgfusepath{clip}%
\pgfsetbuttcap%
\pgfsetroundjoin%
\definecolor{currentfill}{rgb}{0.121569,0.466667,0.705882}%
\pgfsetfillcolor{currentfill}%
\pgfsetfillopacity{0.834969}%
\pgfsetlinewidth{1.003750pt}%
\definecolor{currentstroke}{rgb}{0.121569,0.466667,0.705882}%
\pgfsetstrokecolor{currentstroke}%
\pgfsetstrokeopacity{0.834969}%
\pgfsetdash{}{0pt}%
\pgfpathmoveto{\pgfqpoint{0.632811in}{2.531682in}}%
\pgfpathcurveto{\pgfqpoint{0.641047in}{2.531682in}}{\pgfqpoint{0.648948in}{2.534955in}}{\pgfqpoint{0.654771in}{2.540779in}}%
\pgfpathcurveto{\pgfqpoint{0.660595in}{2.546602in}}{\pgfqpoint{0.663868in}{2.554503in}}{\pgfqpoint{0.663868in}{2.562739in}}%
\pgfpathcurveto{\pgfqpoint{0.663868in}{2.570975in}}{\pgfqpoint{0.660595in}{2.578875in}}{\pgfqpoint{0.654771in}{2.584699in}}%
\pgfpathcurveto{\pgfqpoint{0.648948in}{2.590523in}}{\pgfqpoint{0.641047in}{2.593795in}}{\pgfqpoint{0.632811in}{2.593795in}}%
\pgfpathcurveto{\pgfqpoint{0.624575in}{2.593795in}}{\pgfqpoint{0.616675in}{2.590523in}}{\pgfqpoint{0.610851in}{2.584699in}}%
\pgfpathcurveto{\pgfqpoint{0.605027in}{2.578875in}}{\pgfqpoint{0.601755in}{2.570975in}}{\pgfqpoint{0.601755in}{2.562739in}}%
\pgfpathcurveto{\pgfqpoint{0.601755in}{2.554503in}}{\pgfqpoint{0.605027in}{2.546602in}}{\pgfqpoint{0.610851in}{2.540779in}}%
\pgfpathcurveto{\pgfqpoint{0.616675in}{2.534955in}}{\pgfqpoint{0.624575in}{2.531682in}}{\pgfqpoint{0.632811in}{2.531682in}}%
\pgfpathclose%
\pgfusepath{stroke,fill}%
\end{pgfscope}%
\begin{pgfscope}%
\pgfpathrectangle{\pgfqpoint{0.100000in}{0.212622in}}{\pgfqpoint{3.696000in}{3.696000in}}%
\pgfusepath{clip}%
\pgfsetbuttcap%
\pgfsetroundjoin%
\definecolor{currentfill}{rgb}{0.121569,0.466667,0.705882}%
\pgfsetfillcolor{currentfill}%
\pgfsetfillopacity{0.836111}%
\pgfsetlinewidth{1.003750pt}%
\definecolor{currentstroke}{rgb}{0.121569,0.466667,0.705882}%
\pgfsetstrokecolor{currentstroke}%
\pgfsetstrokeopacity{0.836111}%
\pgfsetdash{}{0pt}%
\pgfpathmoveto{\pgfqpoint{0.641992in}{2.528271in}}%
\pgfpathcurveto{\pgfqpoint{0.650228in}{2.528271in}}{\pgfqpoint{0.658128in}{2.531544in}}{\pgfqpoint{0.663952in}{2.537368in}}%
\pgfpathcurveto{\pgfqpoint{0.669776in}{2.543191in}}{\pgfqpoint{0.673049in}{2.551092in}}{\pgfqpoint{0.673049in}{2.559328in}}%
\pgfpathcurveto{\pgfqpoint{0.673049in}{2.567564in}}{\pgfqpoint{0.669776in}{2.575464in}}{\pgfqpoint{0.663952in}{2.581288in}}%
\pgfpathcurveto{\pgfqpoint{0.658128in}{2.587112in}}{\pgfqpoint{0.650228in}{2.590384in}}{\pgfqpoint{0.641992in}{2.590384in}}%
\pgfpathcurveto{\pgfqpoint{0.633756in}{2.590384in}}{\pgfqpoint{0.625856in}{2.587112in}}{\pgfqpoint{0.620032in}{2.581288in}}%
\pgfpathcurveto{\pgfqpoint{0.614208in}{2.575464in}}{\pgfqpoint{0.610936in}{2.567564in}}{\pgfqpoint{0.610936in}{2.559328in}}%
\pgfpathcurveto{\pgfqpoint{0.610936in}{2.551092in}}{\pgfqpoint{0.614208in}{2.543191in}}{\pgfqpoint{0.620032in}{2.537368in}}%
\pgfpathcurveto{\pgfqpoint{0.625856in}{2.531544in}}{\pgfqpoint{0.633756in}{2.528271in}}{\pgfqpoint{0.641992in}{2.528271in}}%
\pgfpathclose%
\pgfusepath{stroke,fill}%
\end{pgfscope}%
\begin{pgfscope}%
\pgfpathrectangle{\pgfqpoint{0.100000in}{0.212622in}}{\pgfqpoint{3.696000in}{3.696000in}}%
\pgfusepath{clip}%
\pgfsetbuttcap%
\pgfsetroundjoin%
\definecolor{currentfill}{rgb}{0.121569,0.466667,0.705882}%
\pgfsetfillcolor{currentfill}%
\pgfsetfillopacity{0.836924}%
\pgfsetlinewidth{1.003750pt}%
\definecolor{currentstroke}{rgb}{0.121569,0.466667,0.705882}%
\pgfsetstrokecolor{currentstroke}%
\pgfsetstrokeopacity{0.836924}%
\pgfsetdash{}{0pt}%
\pgfpathmoveto{\pgfqpoint{2.770292in}{1.988394in}}%
\pgfpathcurveto{\pgfqpoint{2.778528in}{1.988394in}}{\pgfqpoint{2.786428in}{1.991667in}}{\pgfqpoint{2.792252in}{1.997490in}}%
\pgfpathcurveto{\pgfqpoint{2.798076in}{2.003314in}}{\pgfqpoint{2.801348in}{2.011214in}}{\pgfqpoint{2.801348in}{2.019451in}}%
\pgfpathcurveto{\pgfqpoint{2.801348in}{2.027687in}}{\pgfqpoint{2.798076in}{2.035587in}}{\pgfqpoint{2.792252in}{2.041411in}}%
\pgfpathcurveto{\pgfqpoint{2.786428in}{2.047235in}}{\pgfqpoint{2.778528in}{2.050507in}}{\pgfqpoint{2.770292in}{2.050507in}}%
\pgfpathcurveto{\pgfqpoint{2.762055in}{2.050507in}}{\pgfqpoint{2.754155in}{2.047235in}}{\pgfqpoint{2.748331in}{2.041411in}}%
\pgfpathcurveto{\pgfqpoint{2.742508in}{2.035587in}}{\pgfqpoint{2.739235in}{2.027687in}}{\pgfqpoint{2.739235in}{2.019451in}}%
\pgfpathcurveto{\pgfqpoint{2.739235in}{2.011214in}}{\pgfqpoint{2.742508in}{2.003314in}}{\pgfqpoint{2.748331in}{1.997490in}}%
\pgfpathcurveto{\pgfqpoint{2.754155in}{1.991667in}}{\pgfqpoint{2.762055in}{1.988394in}}{\pgfqpoint{2.770292in}{1.988394in}}%
\pgfpathclose%
\pgfusepath{stroke,fill}%
\end{pgfscope}%
\begin{pgfscope}%
\pgfpathrectangle{\pgfqpoint{0.100000in}{0.212622in}}{\pgfqpoint{3.696000in}{3.696000in}}%
\pgfusepath{clip}%
\pgfsetbuttcap%
\pgfsetroundjoin%
\definecolor{currentfill}{rgb}{0.121569,0.466667,0.705882}%
\pgfsetfillcolor{currentfill}%
\pgfsetfillopacity{0.838185}%
\pgfsetlinewidth{1.003750pt}%
\definecolor{currentstroke}{rgb}{0.121569,0.466667,0.705882}%
\pgfsetstrokecolor{currentstroke}%
\pgfsetstrokeopacity{0.838185}%
\pgfsetdash{}{0pt}%
\pgfpathmoveto{\pgfqpoint{0.658806in}{2.522452in}}%
\pgfpathcurveto{\pgfqpoint{0.667043in}{2.522452in}}{\pgfqpoint{0.674943in}{2.525724in}}{\pgfqpoint{0.680767in}{2.531548in}}%
\pgfpathcurveto{\pgfqpoint{0.686591in}{2.537372in}}{\pgfqpoint{0.689863in}{2.545272in}}{\pgfqpoint{0.689863in}{2.553508in}}%
\pgfpathcurveto{\pgfqpoint{0.689863in}{2.561745in}}{\pgfqpoint{0.686591in}{2.569645in}}{\pgfqpoint{0.680767in}{2.575469in}}%
\pgfpathcurveto{\pgfqpoint{0.674943in}{2.581292in}}{\pgfqpoint{0.667043in}{2.584565in}}{\pgfqpoint{0.658806in}{2.584565in}}%
\pgfpathcurveto{\pgfqpoint{0.650570in}{2.584565in}}{\pgfqpoint{0.642670in}{2.581292in}}{\pgfqpoint{0.636846in}{2.575469in}}%
\pgfpathcurveto{\pgfqpoint{0.631022in}{2.569645in}}{\pgfqpoint{0.627750in}{2.561745in}}{\pgfqpoint{0.627750in}{2.553508in}}%
\pgfpathcurveto{\pgfqpoint{0.627750in}{2.545272in}}{\pgfqpoint{0.631022in}{2.537372in}}{\pgfqpoint{0.636846in}{2.531548in}}%
\pgfpathcurveto{\pgfqpoint{0.642670in}{2.525724in}}{\pgfqpoint{0.650570in}{2.522452in}}{\pgfqpoint{0.658806in}{2.522452in}}%
\pgfpathclose%
\pgfusepath{stroke,fill}%
\end{pgfscope}%
\begin{pgfscope}%
\pgfpathrectangle{\pgfqpoint{0.100000in}{0.212622in}}{\pgfqpoint{3.696000in}{3.696000in}}%
\pgfusepath{clip}%
\pgfsetbuttcap%
\pgfsetroundjoin%
\definecolor{currentfill}{rgb}{0.121569,0.466667,0.705882}%
\pgfsetfillcolor{currentfill}%
\pgfsetfillopacity{0.839150}%
\pgfsetlinewidth{1.003750pt}%
\definecolor{currentstroke}{rgb}{0.121569,0.466667,0.705882}%
\pgfsetstrokecolor{currentstroke}%
\pgfsetstrokeopacity{0.839150}%
\pgfsetdash{}{0pt}%
\pgfpathmoveto{\pgfqpoint{2.766298in}{1.984032in}}%
\pgfpathcurveto{\pgfqpoint{2.774534in}{1.984032in}}{\pgfqpoint{2.782434in}{1.987304in}}{\pgfqpoint{2.788258in}{1.993128in}}%
\pgfpathcurveto{\pgfqpoint{2.794082in}{1.998952in}}{\pgfqpoint{2.797354in}{2.006852in}}{\pgfqpoint{2.797354in}{2.015088in}}%
\pgfpathcurveto{\pgfqpoint{2.797354in}{2.023324in}}{\pgfqpoint{2.794082in}{2.031224in}}{\pgfqpoint{2.788258in}{2.037048in}}%
\pgfpathcurveto{\pgfqpoint{2.782434in}{2.042872in}}{\pgfqpoint{2.774534in}{2.046145in}}{\pgfqpoint{2.766298in}{2.046145in}}%
\pgfpathcurveto{\pgfqpoint{2.758061in}{2.046145in}}{\pgfqpoint{2.750161in}{2.042872in}}{\pgfqpoint{2.744337in}{2.037048in}}%
\pgfpathcurveto{\pgfqpoint{2.738513in}{2.031224in}}{\pgfqpoint{2.735241in}{2.023324in}}{\pgfqpoint{2.735241in}{2.015088in}}%
\pgfpathcurveto{\pgfqpoint{2.735241in}{2.006852in}}{\pgfqpoint{2.738513in}{1.998952in}}{\pgfqpoint{2.744337in}{1.993128in}}%
\pgfpathcurveto{\pgfqpoint{2.750161in}{1.987304in}}{\pgfqpoint{2.758061in}{1.984032in}}{\pgfqpoint{2.766298in}{1.984032in}}%
\pgfpathclose%
\pgfusepath{stroke,fill}%
\end{pgfscope}%
\begin{pgfscope}%
\pgfpathrectangle{\pgfqpoint{0.100000in}{0.212622in}}{\pgfqpoint{3.696000in}{3.696000in}}%
\pgfusepath{clip}%
\pgfsetbuttcap%
\pgfsetroundjoin%
\definecolor{currentfill}{rgb}{0.121569,0.466667,0.705882}%
\pgfsetfillcolor{currentfill}%
\pgfsetfillopacity{0.839765}%
\pgfsetlinewidth{1.003750pt}%
\definecolor{currentstroke}{rgb}{0.121569,0.466667,0.705882}%
\pgfsetstrokecolor{currentstroke}%
\pgfsetstrokeopacity{0.839765}%
\pgfsetdash{}{0pt}%
\pgfpathmoveto{\pgfqpoint{0.673212in}{2.515713in}}%
\pgfpathcurveto{\pgfqpoint{0.681449in}{2.515713in}}{\pgfqpoint{0.689349in}{2.518985in}}{\pgfqpoint{0.695173in}{2.524809in}}%
\pgfpathcurveto{\pgfqpoint{0.700997in}{2.530633in}}{\pgfqpoint{0.704269in}{2.538533in}}{\pgfqpoint{0.704269in}{2.546769in}}%
\pgfpathcurveto{\pgfqpoint{0.704269in}{2.555005in}}{\pgfqpoint{0.700997in}{2.562906in}}{\pgfqpoint{0.695173in}{2.568729in}}%
\pgfpathcurveto{\pgfqpoint{0.689349in}{2.574553in}}{\pgfqpoint{0.681449in}{2.577826in}}{\pgfqpoint{0.673212in}{2.577826in}}%
\pgfpathcurveto{\pgfqpoint{0.664976in}{2.577826in}}{\pgfqpoint{0.657076in}{2.574553in}}{\pgfqpoint{0.651252in}{2.568729in}}%
\pgfpathcurveto{\pgfqpoint{0.645428in}{2.562906in}}{\pgfqpoint{0.642156in}{2.555005in}}{\pgfqpoint{0.642156in}{2.546769in}}%
\pgfpathcurveto{\pgfqpoint{0.642156in}{2.538533in}}{\pgfqpoint{0.645428in}{2.530633in}}{\pgfqpoint{0.651252in}{2.524809in}}%
\pgfpathcurveto{\pgfqpoint{0.657076in}{2.518985in}}{\pgfqpoint{0.664976in}{2.515713in}}{\pgfqpoint{0.673212in}{2.515713in}}%
\pgfpathclose%
\pgfusepath{stroke,fill}%
\end{pgfscope}%
\begin{pgfscope}%
\pgfpathrectangle{\pgfqpoint{0.100000in}{0.212622in}}{\pgfqpoint{3.696000in}{3.696000in}}%
\pgfusepath{clip}%
\pgfsetbuttcap%
\pgfsetroundjoin%
\definecolor{currentfill}{rgb}{0.121569,0.466667,0.705882}%
\pgfsetfillcolor{currentfill}%
\pgfsetfillopacity{0.841201}%
\pgfsetlinewidth{1.003750pt}%
\definecolor{currentstroke}{rgb}{0.121569,0.466667,0.705882}%
\pgfsetstrokecolor{currentstroke}%
\pgfsetstrokeopacity{0.841201}%
\pgfsetdash{}{0pt}%
\pgfpathmoveto{\pgfqpoint{0.686015in}{2.509284in}}%
\pgfpathcurveto{\pgfqpoint{0.694251in}{2.509284in}}{\pgfqpoint{0.702151in}{2.512556in}}{\pgfqpoint{0.707975in}{2.518380in}}%
\pgfpathcurveto{\pgfqpoint{0.713799in}{2.524204in}}{\pgfqpoint{0.717072in}{2.532104in}}{\pgfqpoint{0.717072in}{2.540340in}}%
\pgfpathcurveto{\pgfqpoint{0.717072in}{2.548576in}}{\pgfqpoint{0.713799in}{2.556476in}}{\pgfqpoint{0.707975in}{2.562300in}}%
\pgfpathcurveto{\pgfqpoint{0.702151in}{2.568124in}}{\pgfqpoint{0.694251in}{2.571397in}}{\pgfqpoint{0.686015in}{2.571397in}}%
\pgfpathcurveto{\pgfqpoint{0.677779in}{2.571397in}}{\pgfqpoint{0.669879in}{2.568124in}}{\pgfqpoint{0.664055in}{2.562300in}}%
\pgfpathcurveto{\pgfqpoint{0.658231in}{2.556476in}}{\pgfqpoint{0.654959in}{2.548576in}}{\pgfqpoint{0.654959in}{2.540340in}}%
\pgfpathcurveto{\pgfqpoint{0.654959in}{2.532104in}}{\pgfqpoint{0.658231in}{2.524204in}}{\pgfqpoint{0.664055in}{2.518380in}}%
\pgfpathcurveto{\pgfqpoint{0.669879in}{2.512556in}}{\pgfqpoint{0.677779in}{2.509284in}}{\pgfqpoint{0.686015in}{2.509284in}}%
\pgfpathclose%
\pgfusepath{stroke,fill}%
\end{pgfscope}%
\begin{pgfscope}%
\pgfpathrectangle{\pgfqpoint{0.100000in}{0.212622in}}{\pgfqpoint{3.696000in}{3.696000in}}%
\pgfusepath{clip}%
\pgfsetbuttcap%
\pgfsetroundjoin%
\definecolor{currentfill}{rgb}{0.121569,0.466667,0.705882}%
\pgfsetfillcolor{currentfill}%
\pgfsetfillopacity{0.841414}%
\pgfsetlinewidth{1.003750pt}%
\definecolor{currentstroke}{rgb}{0.121569,0.466667,0.705882}%
\pgfsetstrokecolor{currentstroke}%
\pgfsetstrokeopacity{0.841414}%
\pgfsetdash{}{0pt}%
\pgfpathmoveto{\pgfqpoint{2.761697in}{1.979070in}}%
\pgfpathcurveto{\pgfqpoint{2.769934in}{1.979070in}}{\pgfqpoint{2.777834in}{1.982342in}}{\pgfqpoint{2.783658in}{1.988166in}}%
\pgfpathcurveto{\pgfqpoint{2.789482in}{1.993990in}}{\pgfqpoint{2.792754in}{2.001890in}}{\pgfqpoint{2.792754in}{2.010127in}}%
\pgfpathcurveto{\pgfqpoint{2.792754in}{2.018363in}}{\pgfqpoint{2.789482in}{2.026263in}}{\pgfqpoint{2.783658in}{2.032087in}}%
\pgfpathcurveto{\pgfqpoint{2.777834in}{2.037911in}}{\pgfqpoint{2.769934in}{2.041183in}}{\pgfqpoint{2.761697in}{2.041183in}}%
\pgfpathcurveto{\pgfqpoint{2.753461in}{2.041183in}}{\pgfqpoint{2.745561in}{2.037911in}}{\pgfqpoint{2.739737in}{2.032087in}}%
\pgfpathcurveto{\pgfqpoint{2.733913in}{2.026263in}}{\pgfqpoint{2.730641in}{2.018363in}}{\pgfqpoint{2.730641in}{2.010127in}}%
\pgfpathcurveto{\pgfqpoint{2.730641in}{2.001890in}}{\pgfqpoint{2.733913in}{1.993990in}}{\pgfqpoint{2.739737in}{1.988166in}}%
\pgfpathcurveto{\pgfqpoint{2.745561in}{1.982342in}}{\pgfqpoint{2.753461in}{1.979070in}}{\pgfqpoint{2.761697in}{1.979070in}}%
\pgfpathclose%
\pgfusepath{stroke,fill}%
\end{pgfscope}%
\begin{pgfscope}%
\pgfpathrectangle{\pgfqpoint{0.100000in}{0.212622in}}{\pgfqpoint{3.696000in}{3.696000in}}%
\pgfusepath{clip}%
\pgfsetbuttcap%
\pgfsetroundjoin%
\definecolor{currentfill}{rgb}{0.121569,0.466667,0.705882}%
\pgfsetfillcolor{currentfill}%
\pgfsetfillopacity{0.842327}%
\pgfsetlinewidth{1.003750pt}%
\definecolor{currentstroke}{rgb}{0.121569,0.466667,0.705882}%
\pgfsetstrokecolor{currentstroke}%
\pgfsetstrokeopacity{0.842327}%
\pgfsetdash{}{0pt}%
\pgfpathmoveto{\pgfqpoint{0.696129in}{2.504734in}}%
\pgfpathcurveto{\pgfqpoint{0.704365in}{2.504734in}}{\pgfqpoint{0.712265in}{2.508006in}}{\pgfqpoint{0.718089in}{2.513830in}}%
\pgfpathcurveto{\pgfqpoint{0.723913in}{2.519654in}}{\pgfqpoint{0.727185in}{2.527554in}}{\pgfqpoint{0.727185in}{2.535790in}}%
\pgfpathcurveto{\pgfqpoint{0.727185in}{2.544027in}}{\pgfqpoint{0.723913in}{2.551927in}}{\pgfqpoint{0.718089in}{2.557751in}}%
\pgfpathcurveto{\pgfqpoint{0.712265in}{2.563574in}}{\pgfqpoint{0.704365in}{2.566847in}}{\pgfqpoint{0.696129in}{2.566847in}}%
\pgfpathcurveto{\pgfqpoint{0.687893in}{2.566847in}}{\pgfqpoint{0.679993in}{2.563574in}}{\pgfqpoint{0.674169in}{2.557751in}}%
\pgfpathcurveto{\pgfqpoint{0.668345in}{2.551927in}}{\pgfqpoint{0.665072in}{2.544027in}}{\pgfqpoint{0.665072in}{2.535790in}}%
\pgfpathcurveto{\pgfqpoint{0.665072in}{2.527554in}}{\pgfqpoint{0.668345in}{2.519654in}}{\pgfqpoint{0.674169in}{2.513830in}}%
\pgfpathcurveto{\pgfqpoint{0.679993in}{2.508006in}}{\pgfqpoint{0.687893in}{2.504734in}}{\pgfqpoint{0.696129in}{2.504734in}}%
\pgfpathclose%
\pgfusepath{stroke,fill}%
\end{pgfscope}%
\begin{pgfscope}%
\pgfpathrectangle{\pgfqpoint{0.100000in}{0.212622in}}{\pgfqpoint{3.696000in}{3.696000in}}%
\pgfusepath{clip}%
\pgfsetbuttcap%
\pgfsetroundjoin%
\definecolor{currentfill}{rgb}{0.121569,0.466667,0.705882}%
\pgfsetfillcolor{currentfill}%
\pgfsetfillopacity{0.842702}%
\pgfsetlinewidth{1.003750pt}%
\definecolor{currentstroke}{rgb}{0.121569,0.466667,0.705882}%
\pgfsetstrokecolor{currentstroke}%
\pgfsetstrokeopacity{0.842702}%
\pgfsetdash{}{0pt}%
\pgfpathmoveto{\pgfqpoint{2.759346in}{1.976432in}}%
\pgfpathcurveto{\pgfqpoint{2.767582in}{1.976432in}}{\pgfqpoint{2.775482in}{1.979704in}}{\pgfqpoint{2.781306in}{1.985528in}}%
\pgfpathcurveto{\pgfqpoint{2.787130in}{1.991352in}}{\pgfqpoint{2.790402in}{1.999252in}}{\pgfqpoint{2.790402in}{2.007488in}}%
\pgfpathcurveto{\pgfqpoint{2.790402in}{2.015725in}}{\pgfqpoint{2.787130in}{2.023625in}}{\pgfqpoint{2.781306in}{2.029449in}}%
\pgfpathcurveto{\pgfqpoint{2.775482in}{2.035273in}}{\pgfqpoint{2.767582in}{2.038545in}}{\pgfqpoint{2.759346in}{2.038545in}}%
\pgfpathcurveto{\pgfqpoint{2.751109in}{2.038545in}}{\pgfqpoint{2.743209in}{2.035273in}}{\pgfqpoint{2.737385in}{2.029449in}}%
\pgfpathcurveto{\pgfqpoint{2.731562in}{2.023625in}}{\pgfqpoint{2.728289in}{2.015725in}}{\pgfqpoint{2.728289in}{2.007488in}}%
\pgfpathcurveto{\pgfqpoint{2.728289in}{1.999252in}}{\pgfqpoint{2.731562in}{1.991352in}}{\pgfqpoint{2.737385in}{1.985528in}}%
\pgfpathcurveto{\pgfqpoint{2.743209in}{1.979704in}}{\pgfqpoint{2.751109in}{1.976432in}}{\pgfqpoint{2.759346in}{1.976432in}}%
\pgfpathclose%
\pgfusepath{stroke,fill}%
\end{pgfscope}%
\begin{pgfscope}%
\pgfpathrectangle{\pgfqpoint{0.100000in}{0.212622in}}{\pgfqpoint{3.696000in}{3.696000in}}%
\pgfusepath{clip}%
\pgfsetbuttcap%
\pgfsetroundjoin%
\definecolor{currentfill}{rgb}{0.121569,0.466667,0.705882}%
\pgfsetfillcolor{currentfill}%
\pgfsetfillopacity{0.843379}%
\pgfsetlinewidth{1.003750pt}%
\definecolor{currentstroke}{rgb}{0.121569,0.466667,0.705882}%
\pgfsetstrokecolor{currentstroke}%
\pgfsetstrokeopacity{0.843379}%
\pgfsetdash{}{0pt}%
\pgfpathmoveto{\pgfqpoint{0.705341in}{2.500408in}}%
\pgfpathcurveto{\pgfqpoint{0.713577in}{2.500408in}}{\pgfqpoint{0.721477in}{2.503680in}}{\pgfqpoint{0.727301in}{2.509504in}}%
\pgfpathcurveto{\pgfqpoint{0.733125in}{2.515328in}}{\pgfqpoint{0.736397in}{2.523228in}}{\pgfqpoint{0.736397in}{2.531464in}}%
\pgfpathcurveto{\pgfqpoint{0.736397in}{2.539700in}}{\pgfqpoint{0.733125in}{2.547601in}}{\pgfqpoint{0.727301in}{2.553424in}}%
\pgfpathcurveto{\pgfqpoint{0.721477in}{2.559248in}}{\pgfqpoint{0.713577in}{2.562521in}}{\pgfqpoint{0.705341in}{2.562521in}}%
\pgfpathcurveto{\pgfqpoint{0.697104in}{2.562521in}}{\pgfqpoint{0.689204in}{2.559248in}}{\pgfqpoint{0.683380in}{2.553424in}}%
\pgfpathcurveto{\pgfqpoint{0.677557in}{2.547601in}}{\pgfqpoint{0.674284in}{2.539700in}}{\pgfqpoint{0.674284in}{2.531464in}}%
\pgfpathcurveto{\pgfqpoint{0.674284in}{2.523228in}}{\pgfqpoint{0.677557in}{2.515328in}}{\pgfqpoint{0.683380in}{2.509504in}}%
\pgfpathcurveto{\pgfqpoint{0.689204in}{2.503680in}}{\pgfqpoint{0.697104in}{2.500408in}}{\pgfqpoint{0.705341in}{2.500408in}}%
\pgfpathclose%
\pgfusepath{stroke,fill}%
\end{pgfscope}%
\begin{pgfscope}%
\pgfpathrectangle{\pgfqpoint{0.100000in}{0.212622in}}{\pgfqpoint{3.696000in}{3.696000in}}%
\pgfusepath{clip}%
\pgfsetbuttcap%
\pgfsetroundjoin%
\definecolor{currentfill}{rgb}{0.121569,0.466667,0.705882}%
\pgfsetfillcolor{currentfill}%
\pgfsetfillopacity{0.844784}%
\pgfsetlinewidth{1.003750pt}%
\definecolor{currentstroke}{rgb}{0.121569,0.466667,0.705882}%
\pgfsetstrokecolor{currentstroke}%
\pgfsetstrokeopacity{0.844784}%
\pgfsetdash{}{0pt}%
\pgfpathmoveto{\pgfqpoint{2.755073in}{1.971692in}}%
\pgfpathcurveto{\pgfqpoint{2.763310in}{1.971692in}}{\pgfqpoint{2.771210in}{1.974965in}}{\pgfqpoint{2.777034in}{1.980789in}}%
\pgfpathcurveto{\pgfqpoint{2.782857in}{1.986613in}}{\pgfqpoint{2.786130in}{1.994513in}}{\pgfqpoint{2.786130in}{2.002749in}}%
\pgfpathcurveto{\pgfqpoint{2.786130in}{2.010985in}}{\pgfqpoint{2.782857in}{2.018885in}}{\pgfqpoint{2.777034in}{2.024709in}}%
\pgfpathcurveto{\pgfqpoint{2.771210in}{2.030533in}}{\pgfqpoint{2.763310in}{2.033805in}}{\pgfqpoint{2.755073in}{2.033805in}}%
\pgfpathcurveto{\pgfqpoint{2.746837in}{2.033805in}}{\pgfqpoint{2.738937in}{2.030533in}}{\pgfqpoint{2.733113in}{2.024709in}}%
\pgfpathcurveto{\pgfqpoint{2.727289in}{2.018885in}}{\pgfqpoint{2.724017in}{2.010985in}}{\pgfqpoint{2.724017in}{2.002749in}}%
\pgfpathcurveto{\pgfqpoint{2.724017in}{1.994513in}}{\pgfqpoint{2.727289in}{1.986613in}}{\pgfqpoint{2.733113in}{1.980789in}}%
\pgfpathcurveto{\pgfqpoint{2.738937in}{1.974965in}}{\pgfqpoint{2.746837in}{1.971692in}}{\pgfqpoint{2.755073in}{1.971692in}}%
\pgfpathclose%
\pgfusepath{stroke,fill}%
\end{pgfscope}%
\begin{pgfscope}%
\pgfpathrectangle{\pgfqpoint{0.100000in}{0.212622in}}{\pgfqpoint{3.696000in}{3.696000in}}%
\pgfusepath{clip}%
\pgfsetbuttcap%
\pgfsetroundjoin%
\definecolor{currentfill}{rgb}{0.121569,0.466667,0.705882}%
\pgfsetfillcolor{currentfill}%
\pgfsetfillopacity{0.845399}%
\pgfsetlinewidth{1.003750pt}%
\definecolor{currentstroke}{rgb}{0.121569,0.466667,0.705882}%
\pgfsetstrokecolor{currentstroke}%
\pgfsetstrokeopacity{0.845399}%
\pgfsetdash{}{0pt}%
\pgfpathmoveto{\pgfqpoint{0.722157in}{2.493496in}}%
\pgfpathcurveto{\pgfqpoint{0.730394in}{2.493496in}}{\pgfqpoint{0.738294in}{2.496769in}}{\pgfqpoint{0.744118in}{2.502593in}}%
\pgfpathcurveto{\pgfqpoint{0.749942in}{2.508416in}}{\pgfqpoint{0.753214in}{2.516317in}}{\pgfqpoint{0.753214in}{2.524553in}}%
\pgfpathcurveto{\pgfqpoint{0.753214in}{2.532789in}}{\pgfqpoint{0.749942in}{2.540689in}}{\pgfqpoint{0.744118in}{2.546513in}}%
\pgfpathcurveto{\pgfqpoint{0.738294in}{2.552337in}}{\pgfqpoint{0.730394in}{2.555609in}}{\pgfqpoint{0.722157in}{2.555609in}}%
\pgfpathcurveto{\pgfqpoint{0.713921in}{2.555609in}}{\pgfqpoint{0.706021in}{2.552337in}}{\pgfqpoint{0.700197in}{2.546513in}}%
\pgfpathcurveto{\pgfqpoint{0.694373in}{2.540689in}}{\pgfqpoint{0.691101in}{2.532789in}}{\pgfqpoint{0.691101in}{2.524553in}}%
\pgfpathcurveto{\pgfqpoint{0.691101in}{2.516317in}}{\pgfqpoint{0.694373in}{2.508416in}}{\pgfqpoint{0.700197in}{2.502593in}}%
\pgfpathcurveto{\pgfqpoint{0.706021in}{2.496769in}}{\pgfqpoint{0.713921in}{2.493496in}}{\pgfqpoint{0.722157in}{2.493496in}}%
\pgfpathclose%
\pgfusepath{stroke,fill}%
\end{pgfscope}%
\begin{pgfscope}%
\pgfpathrectangle{\pgfqpoint{0.100000in}{0.212622in}}{\pgfqpoint{3.696000in}{3.696000in}}%
\pgfusepath{clip}%
\pgfsetbuttcap%
\pgfsetroundjoin%
\definecolor{currentfill}{rgb}{0.121569,0.466667,0.705882}%
\pgfsetfillcolor{currentfill}%
\pgfsetfillopacity{0.847038}%
\pgfsetlinewidth{1.003750pt}%
\definecolor{currentstroke}{rgb}{0.121569,0.466667,0.705882}%
\pgfsetstrokecolor{currentstroke}%
\pgfsetstrokeopacity{0.847038}%
\pgfsetdash{}{0pt}%
\pgfpathmoveto{\pgfqpoint{0.736665in}{2.487381in}}%
\pgfpathcurveto{\pgfqpoint{0.744902in}{2.487381in}}{\pgfqpoint{0.752802in}{2.490653in}}{\pgfqpoint{0.758626in}{2.496477in}}%
\pgfpathcurveto{\pgfqpoint{0.764449in}{2.502301in}}{\pgfqpoint{0.767722in}{2.510201in}}{\pgfqpoint{0.767722in}{2.518437in}}%
\pgfpathcurveto{\pgfqpoint{0.767722in}{2.526674in}}{\pgfqpoint{0.764449in}{2.534574in}}{\pgfqpoint{0.758626in}{2.540397in}}%
\pgfpathcurveto{\pgfqpoint{0.752802in}{2.546221in}}{\pgfqpoint{0.744902in}{2.549494in}}{\pgfqpoint{0.736665in}{2.549494in}}%
\pgfpathcurveto{\pgfqpoint{0.728429in}{2.549494in}}{\pgfqpoint{0.720529in}{2.546221in}}{\pgfqpoint{0.714705in}{2.540397in}}%
\pgfpathcurveto{\pgfqpoint{0.708881in}{2.534574in}}{\pgfqpoint{0.705609in}{2.526674in}}{\pgfqpoint{0.705609in}{2.518437in}}%
\pgfpathcurveto{\pgfqpoint{0.705609in}{2.510201in}}{\pgfqpoint{0.708881in}{2.502301in}}{\pgfqpoint{0.714705in}{2.496477in}}%
\pgfpathcurveto{\pgfqpoint{0.720529in}{2.490653in}}{\pgfqpoint{0.728429in}{2.487381in}}{\pgfqpoint{0.736665in}{2.487381in}}%
\pgfpathclose%
\pgfusepath{stroke,fill}%
\end{pgfscope}%
\begin{pgfscope}%
\pgfpathrectangle{\pgfqpoint{0.100000in}{0.212622in}}{\pgfqpoint{3.696000in}{3.696000in}}%
\pgfusepath{clip}%
\pgfsetbuttcap%
\pgfsetroundjoin%
\definecolor{currentfill}{rgb}{0.121569,0.466667,0.705882}%
\pgfsetfillcolor{currentfill}%
\pgfsetfillopacity{0.847095}%
\pgfsetlinewidth{1.003750pt}%
\definecolor{currentstroke}{rgb}{0.121569,0.466667,0.705882}%
\pgfsetstrokecolor{currentstroke}%
\pgfsetstrokeopacity{0.847095}%
\pgfsetdash{}{0pt}%
\pgfpathmoveto{\pgfqpoint{2.750496in}{1.967088in}}%
\pgfpathcurveto{\pgfqpoint{2.758733in}{1.967088in}}{\pgfqpoint{2.766633in}{1.970360in}}{\pgfqpoint{2.772457in}{1.976184in}}%
\pgfpathcurveto{\pgfqpoint{2.778281in}{1.982008in}}{\pgfqpoint{2.781553in}{1.989908in}}{\pgfqpoint{2.781553in}{1.998145in}}%
\pgfpathcurveto{\pgfqpoint{2.781553in}{2.006381in}}{\pgfqpoint{2.778281in}{2.014281in}}{\pgfqpoint{2.772457in}{2.020105in}}%
\pgfpathcurveto{\pgfqpoint{2.766633in}{2.025929in}}{\pgfqpoint{2.758733in}{2.029201in}}{\pgfqpoint{2.750496in}{2.029201in}}%
\pgfpathcurveto{\pgfqpoint{2.742260in}{2.029201in}}{\pgfqpoint{2.734360in}{2.025929in}}{\pgfqpoint{2.728536in}{2.020105in}}%
\pgfpathcurveto{\pgfqpoint{2.722712in}{2.014281in}}{\pgfqpoint{2.719440in}{2.006381in}}{\pgfqpoint{2.719440in}{1.998145in}}%
\pgfpathcurveto{\pgfqpoint{2.719440in}{1.989908in}}{\pgfqpoint{2.722712in}{1.982008in}}{\pgfqpoint{2.728536in}{1.976184in}}%
\pgfpathcurveto{\pgfqpoint{2.734360in}{1.970360in}}{\pgfqpoint{2.742260in}{1.967088in}}{\pgfqpoint{2.750496in}{1.967088in}}%
\pgfpathclose%
\pgfusepath{stroke,fill}%
\end{pgfscope}%
\begin{pgfscope}%
\pgfpathrectangle{\pgfqpoint{0.100000in}{0.212622in}}{\pgfqpoint{3.696000in}{3.696000in}}%
\pgfusepath{clip}%
\pgfsetbuttcap%
\pgfsetroundjoin%
\definecolor{currentfill}{rgb}{0.121569,0.466667,0.705882}%
\pgfsetfillcolor{currentfill}%
\pgfsetfillopacity{0.848130}%
\pgfsetlinewidth{1.003750pt}%
\definecolor{currentstroke}{rgb}{0.121569,0.466667,0.705882}%
\pgfsetstrokecolor{currentstroke}%
\pgfsetstrokeopacity{0.848130}%
\pgfsetdash{}{0pt}%
\pgfpathmoveto{\pgfqpoint{0.747838in}{2.481752in}}%
\pgfpathcurveto{\pgfqpoint{0.756074in}{2.481752in}}{\pgfqpoint{0.763974in}{2.485025in}}{\pgfqpoint{0.769798in}{2.490848in}}%
\pgfpathcurveto{\pgfqpoint{0.775622in}{2.496672in}}{\pgfqpoint{0.778894in}{2.504572in}}{\pgfqpoint{0.778894in}{2.512809in}}%
\pgfpathcurveto{\pgfqpoint{0.778894in}{2.521045in}}{\pgfqpoint{0.775622in}{2.528945in}}{\pgfqpoint{0.769798in}{2.534769in}}%
\pgfpathcurveto{\pgfqpoint{0.763974in}{2.540593in}}{\pgfqpoint{0.756074in}{2.543865in}}{\pgfqpoint{0.747838in}{2.543865in}}%
\pgfpathcurveto{\pgfqpoint{0.739602in}{2.543865in}}{\pgfqpoint{0.731702in}{2.540593in}}{\pgfqpoint{0.725878in}{2.534769in}}%
\pgfpathcurveto{\pgfqpoint{0.720054in}{2.528945in}}{\pgfqpoint{0.716781in}{2.521045in}}{\pgfqpoint{0.716781in}{2.512809in}}%
\pgfpathcurveto{\pgfqpoint{0.716781in}{2.504572in}}{\pgfqpoint{0.720054in}{2.496672in}}{\pgfqpoint{0.725878in}{2.490848in}}%
\pgfpathcurveto{\pgfqpoint{0.731702in}{2.485025in}}{\pgfqpoint{0.739602in}{2.481752in}}{\pgfqpoint{0.747838in}{2.481752in}}%
\pgfpathclose%
\pgfusepath{stroke,fill}%
\end{pgfscope}%
\begin{pgfscope}%
\pgfpathrectangle{\pgfqpoint{0.100000in}{0.212622in}}{\pgfqpoint{3.696000in}{3.696000in}}%
\pgfusepath{clip}%
\pgfsetbuttcap%
\pgfsetroundjoin%
\definecolor{currentfill}{rgb}{0.121569,0.466667,0.705882}%
\pgfsetfillcolor{currentfill}%
\pgfsetfillopacity{0.848416}%
\pgfsetlinewidth{1.003750pt}%
\definecolor{currentstroke}{rgb}{0.121569,0.466667,0.705882}%
\pgfsetstrokecolor{currentstroke}%
\pgfsetstrokeopacity{0.848416}%
\pgfsetdash{}{0pt}%
\pgfpathmoveto{\pgfqpoint{2.748098in}{1.964749in}}%
\pgfpathcurveto{\pgfqpoint{2.756334in}{1.964749in}}{\pgfqpoint{2.764234in}{1.968021in}}{\pgfqpoint{2.770058in}{1.973845in}}%
\pgfpathcurveto{\pgfqpoint{2.775882in}{1.979669in}}{\pgfqpoint{2.779154in}{1.987569in}}{\pgfqpoint{2.779154in}{1.995805in}}%
\pgfpathcurveto{\pgfqpoint{2.779154in}{2.004041in}}{\pgfqpoint{2.775882in}{2.011941in}}{\pgfqpoint{2.770058in}{2.017765in}}%
\pgfpathcurveto{\pgfqpoint{2.764234in}{2.023589in}}{\pgfqpoint{2.756334in}{2.026862in}}{\pgfqpoint{2.748098in}{2.026862in}}%
\pgfpathcurveto{\pgfqpoint{2.739862in}{2.026862in}}{\pgfqpoint{2.731961in}{2.023589in}}{\pgfqpoint{2.726138in}{2.017765in}}%
\pgfpathcurveto{\pgfqpoint{2.720314in}{2.011941in}}{\pgfqpoint{2.717041in}{2.004041in}}{\pgfqpoint{2.717041in}{1.995805in}}%
\pgfpathcurveto{\pgfqpoint{2.717041in}{1.987569in}}{\pgfqpoint{2.720314in}{1.979669in}}{\pgfqpoint{2.726138in}{1.973845in}}%
\pgfpathcurveto{\pgfqpoint{2.731961in}{1.968021in}}{\pgfqpoint{2.739862in}{1.964749in}}{\pgfqpoint{2.748098in}{1.964749in}}%
\pgfpathclose%
\pgfusepath{stroke,fill}%
\end{pgfscope}%
\begin{pgfscope}%
\pgfpathrectangle{\pgfqpoint{0.100000in}{0.212622in}}{\pgfqpoint{3.696000in}{3.696000in}}%
\pgfusepath{clip}%
\pgfsetbuttcap%
\pgfsetroundjoin%
\definecolor{currentfill}{rgb}{0.121569,0.466667,0.705882}%
\pgfsetfillcolor{currentfill}%
\pgfsetfillopacity{0.848905}%
\pgfsetlinewidth{1.003750pt}%
\definecolor{currentstroke}{rgb}{0.121569,0.466667,0.705882}%
\pgfsetstrokecolor{currentstroke}%
\pgfsetstrokeopacity{0.848905}%
\pgfsetdash{}{0pt}%
\pgfpathmoveto{\pgfqpoint{0.756497in}{2.477538in}}%
\pgfpathcurveto{\pgfqpoint{0.764733in}{2.477538in}}{\pgfqpoint{0.772633in}{2.480811in}}{\pgfqpoint{0.778457in}{2.486635in}}%
\pgfpathcurveto{\pgfqpoint{0.784281in}{2.492458in}}{\pgfqpoint{0.787553in}{2.500358in}}{\pgfqpoint{0.787553in}{2.508595in}}%
\pgfpathcurveto{\pgfqpoint{0.787553in}{2.516831in}}{\pgfqpoint{0.784281in}{2.524731in}}{\pgfqpoint{0.778457in}{2.530555in}}%
\pgfpathcurveto{\pgfqpoint{0.772633in}{2.536379in}}{\pgfqpoint{0.764733in}{2.539651in}}{\pgfqpoint{0.756497in}{2.539651in}}%
\pgfpathcurveto{\pgfqpoint{0.748261in}{2.539651in}}{\pgfqpoint{0.740361in}{2.536379in}}{\pgfqpoint{0.734537in}{2.530555in}}%
\pgfpathcurveto{\pgfqpoint{0.728713in}{2.524731in}}{\pgfqpoint{0.725440in}{2.516831in}}{\pgfqpoint{0.725440in}{2.508595in}}%
\pgfpathcurveto{\pgfqpoint{0.725440in}{2.500358in}}{\pgfqpoint{0.728713in}{2.492458in}}{\pgfqpoint{0.734537in}{2.486635in}}%
\pgfpathcurveto{\pgfqpoint{0.740361in}{2.480811in}}{\pgfqpoint{0.748261in}{2.477538in}}{\pgfqpoint{0.756497in}{2.477538in}}%
\pgfpathclose%
\pgfusepath{stroke,fill}%
\end{pgfscope}%
\begin{pgfscope}%
\pgfpathrectangle{\pgfqpoint{0.100000in}{0.212622in}}{\pgfqpoint{3.696000in}{3.696000in}}%
\pgfusepath{clip}%
\pgfsetbuttcap%
\pgfsetroundjoin%
\definecolor{currentfill}{rgb}{0.121569,0.466667,0.705882}%
\pgfsetfillcolor{currentfill}%
\pgfsetfillopacity{0.849121}%
\pgfsetlinewidth{1.003750pt}%
\definecolor{currentstroke}{rgb}{0.121569,0.466667,0.705882}%
\pgfsetstrokecolor{currentstroke}%
\pgfsetstrokeopacity{0.849121}%
\pgfsetdash{}{0pt}%
\pgfpathmoveto{\pgfqpoint{2.746694in}{1.963403in}}%
\pgfpathcurveto{\pgfqpoint{2.754931in}{1.963403in}}{\pgfqpoint{2.762831in}{1.966676in}}{\pgfqpoint{2.768655in}{1.972500in}}%
\pgfpathcurveto{\pgfqpoint{2.774479in}{1.978323in}}{\pgfqpoint{2.777751in}{1.986224in}}{\pgfqpoint{2.777751in}{1.994460in}}%
\pgfpathcurveto{\pgfqpoint{2.777751in}{2.002696in}}{\pgfqpoint{2.774479in}{2.010596in}}{\pgfqpoint{2.768655in}{2.016420in}}%
\pgfpathcurveto{\pgfqpoint{2.762831in}{2.022244in}}{\pgfqpoint{2.754931in}{2.025516in}}{\pgfqpoint{2.746694in}{2.025516in}}%
\pgfpathcurveto{\pgfqpoint{2.738458in}{2.025516in}}{\pgfqpoint{2.730558in}{2.022244in}}{\pgfqpoint{2.724734in}{2.016420in}}%
\pgfpathcurveto{\pgfqpoint{2.718910in}{2.010596in}}{\pgfqpoint{2.715638in}{2.002696in}}{\pgfqpoint{2.715638in}{1.994460in}}%
\pgfpathcurveto{\pgfqpoint{2.715638in}{1.986224in}}{\pgfqpoint{2.718910in}{1.978323in}}{\pgfqpoint{2.724734in}{1.972500in}}%
\pgfpathcurveto{\pgfqpoint{2.730558in}{1.966676in}}{\pgfqpoint{2.738458in}{1.963403in}}{\pgfqpoint{2.746694in}{1.963403in}}%
\pgfpathclose%
\pgfusepath{stroke,fill}%
\end{pgfscope}%
\begin{pgfscope}%
\pgfpathrectangle{\pgfqpoint{0.100000in}{0.212622in}}{\pgfqpoint{3.696000in}{3.696000in}}%
\pgfusepath{clip}%
\pgfsetbuttcap%
\pgfsetroundjoin%
\definecolor{currentfill}{rgb}{0.121569,0.466667,0.705882}%
\pgfsetfillcolor{currentfill}%
\pgfsetfillopacity{0.850248}%
\pgfsetlinewidth{1.003750pt}%
\definecolor{currentstroke}{rgb}{0.121569,0.466667,0.705882}%
\pgfsetstrokecolor{currentstroke}%
\pgfsetstrokeopacity{0.850248}%
\pgfsetdash{}{0pt}%
\pgfpathmoveto{\pgfqpoint{0.772256in}{2.469424in}}%
\pgfpathcurveto{\pgfqpoint{0.780492in}{2.469424in}}{\pgfqpoint{0.788392in}{2.472696in}}{\pgfqpoint{0.794216in}{2.478520in}}%
\pgfpathcurveto{\pgfqpoint{0.800040in}{2.484344in}}{\pgfqpoint{0.803313in}{2.492244in}}{\pgfqpoint{0.803313in}{2.500480in}}%
\pgfpathcurveto{\pgfqpoint{0.803313in}{2.508717in}}{\pgfqpoint{0.800040in}{2.516617in}}{\pgfqpoint{0.794216in}{2.522441in}}%
\pgfpathcurveto{\pgfqpoint{0.788392in}{2.528264in}}{\pgfqpoint{0.780492in}{2.531537in}}{\pgfqpoint{0.772256in}{2.531537in}}%
\pgfpathcurveto{\pgfqpoint{0.764020in}{2.531537in}}{\pgfqpoint{0.756120in}{2.528264in}}{\pgfqpoint{0.750296in}{2.522441in}}%
\pgfpathcurveto{\pgfqpoint{0.744472in}{2.516617in}}{\pgfqpoint{0.741200in}{2.508717in}}{\pgfqpoint{0.741200in}{2.500480in}}%
\pgfpathcurveto{\pgfqpoint{0.741200in}{2.492244in}}{\pgfqpoint{0.744472in}{2.484344in}}{\pgfqpoint{0.750296in}{2.478520in}}%
\pgfpathcurveto{\pgfqpoint{0.756120in}{2.472696in}}{\pgfqpoint{0.764020in}{2.469424in}}{\pgfqpoint{0.772256in}{2.469424in}}%
\pgfpathclose%
\pgfusepath{stroke,fill}%
\end{pgfscope}%
\begin{pgfscope}%
\pgfpathrectangle{\pgfqpoint{0.100000in}{0.212622in}}{\pgfqpoint{3.696000in}{3.696000in}}%
\pgfusepath{clip}%
\pgfsetbuttcap%
\pgfsetroundjoin%
\definecolor{currentfill}{rgb}{0.121569,0.466667,0.705882}%
\pgfsetfillcolor{currentfill}%
\pgfsetfillopacity{0.850552}%
\pgfsetlinewidth{1.003750pt}%
\definecolor{currentstroke}{rgb}{0.121569,0.466667,0.705882}%
\pgfsetstrokecolor{currentstroke}%
\pgfsetstrokeopacity{0.850552}%
\pgfsetdash{}{0pt}%
\pgfpathmoveto{\pgfqpoint{2.744033in}{1.960910in}}%
\pgfpathcurveto{\pgfqpoint{2.752269in}{1.960910in}}{\pgfqpoint{2.760169in}{1.964182in}}{\pgfqpoint{2.765993in}{1.970006in}}%
\pgfpathcurveto{\pgfqpoint{2.771817in}{1.975830in}}{\pgfqpoint{2.775089in}{1.983730in}}{\pgfqpoint{2.775089in}{1.991966in}}%
\pgfpathcurveto{\pgfqpoint{2.775089in}{2.000203in}}{\pgfqpoint{2.771817in}{2.008103in}}{\pgfqpoint{2.765993in}{2.013927in}}%
\pgfpathcurveto{\pgfqpoint{2.760169in}{2.019751in}}{\pgfqpoint{2.752269in}{2.023023in}}{\pgfqpoint{2.744033in}{2.023023in}}%
\pgfpathcurveto{\pgfqpoint{2.735796in}{2.023023in}}{\pgfqpoint{2.727896in}{2.019751in}}{\pgfqpoint{2.722072in}{2.013927in}}%
\pgfpathcurveto{\pgfqpoint{2.716248in}{2.008103in}}{\pgfqpoint{2.712976in}{2.000203in}}{\pgfqpoint{2.712976in}{1.991966in}}%
\pgfpathcurveto{\pgfqpoint{2.712976in}{1.983730in}}{\pgfqpoint{2.716248in}{1.975830in}}{\pgfqpoint{2.722072in}{1.970006in}}%
\pgfpathcurveto{\pgfqpoint{2.727896in}{1.964182in}}{\pgfqpoint{2.735796in}{1.960910in}}{\pgfqpoint{2.744033in}{1.960910in}}%
\pgfpathclose%
\pgfusepath{stroke,fill}%
\end{pgfscope}%
\begin{pgfscope}%
\pgfpathrectangle{\pgfqpoint{0.100000in}{0.212622in}}{\pgfqpoint{3.696000in}{3.696000in}}%
\pgfusepath{clip}%
\pgfsetbuttcap%
\pgfsetroundjoin%
\definecolor{currentfill}{rgb}{0.121569,0.466667,0.705882}%
\pgfsetfillcolor{currentfill}%
\pgfsetfillopacity{0.851606}%
\pgfsetlinewidth{1.003750pt}%
\definecolor{currentstroke}{rgb}{0.121569,0.466667,0.705882}%
\pgfsetstrokecolor{currentstroke}%
\pgfsetstrokeopacity{0.851606}%
\pgfsetdash{}{0pt}%
\pgfpathmoveto{\pgfqpoint{0.785409in}{2.462955in}}%
\pgfpathcurveto{\pgfqpoint{0.793645in}{2.462955in}}{\pgfqpoint{0.801545in}{2.466227in}}{\pgfqpoint{0.807369in}{2.472051in}}%
\pgfpathcurveto{\pgfqpoint{0.813193in}{2.477875in}}{\pgfqpoint{0.816465in}{2.485775in}}{\pgfqpoint{0.816465in}{2.494011in}}%
\pgfpathcurveto{\pgfqpoint{0.816465in}{2.502248in}}{\pgfqpoint{0.813193in}{2.510148in}}{\pgfqpoint{0.807369in}{2.515972in}}%
\pgfpathcurveto{\pgfqpoint{0.801545in}{2.521796in}}{\pgfqpoint{0.793645in}{2.525068in}}{\pgfqpoint{0.785409in}{2.525068in}}%
\pgfpathcurveto{\pgfqpoint{0.777173in}{2.525068in}}{\pgfqpoint{0.769273in}{2.521796in}}{\pgfqpoint{0.763449in}{2.515972in}}%
\pgfpathcurveto{\pgfqpoint{0.757625in}{2.510148in}}{\pgfqpoint{0.754352in}{2.502248in}}{\pgfqpoint{0.754352in}{2.494011in}}%
\pgfpathcurveto{\pgfqpoint{0.754352in}{2.485775in}}{\pgfqpoint{0.757625in}{2.477875in}}{\pgfqpoint{0.763449in}{2.472051in}}%
\pgfpathcurveto{\pgfqpoint{0.769273in}{2.466227in}}{\pgfqpoint{0.777173in}{2.462955in}}{\pgfqpoint{0.785409in}{2.462955in}}%
\pgfpathclose%
\pgfusepath{stroke,fill}%
\end{pgfscope}%
\begin{pgfscope}%
\pgfpathrectangle{\pgfqpoint{0.100000in}{0.212622in}}{\pgfqpoint{3.696000in}{3.696000in}}%
\pgfusepath{clip}%
\pgfsetbuttcap%
\pgfsetroundjoin%
\definecolor{currentfill}{rgb}{0.121569,0.466667,0.705882}%
\pgfsetfillcolor{currentfill}%
\pgfsetfillopacity{0.852163}%
\pgfsetlinewidth{1.003750pt}%
\definecolor{currentstroke}{rgb}{0.121569,0.466667,0.705882}%
\pgfsetstrokecolor{currentstroke}%
\pgfsetstrokeopacity{0.852163}%
\pgfsetdash{}{0pt}%
\pgfpathmoveto{\pgfqpoint{2.741068in}{1.957942in}}%
\pgfpathcurveto{\pgfqpoint{2.749304in}{1.957942in}}{\pgfqpoint{2.757204in}{1.961214in}}{\pgfqpoint{2.763028in}{1.967038in}}%
\pgfpathcurveto{\pgfqpoint{2.768852in}{1.972862in}}{\pgfqpoint{2.772124in}{1.980762in}}{\pgfqpoint{2.772124in}{1.988998in}}%
\pgfpathcurveto{\pgfqpoint{2.772124in}{1.997235in}}{\pgfqpoint{2.768852in}{2.005135in}}{\pgfqpoint{2.763028in}{2.010959in}}%
\pgfpathcurveto{\pgfqpoint{2.757204in}{2.016783in}}{\pgfqpoint{2.749304in}{2.020055in}}{\pgfqpoint{2.741068in}{2.020055in}}%
\pgfpathcurveto{\pgfqpoint{2.732831in}{2.020055in}}{\pgfqpoint{2.724931in}{2.016783in}}{\pgfqpoint{2.719107in}{2.010959in}}%
\pgfpathcurveto{\pgfqpoint{2.713283in}{2.005135in}}{\pgfqpoint{2.710011in}{1.997235in}}{\pgfqpoint{2.710011in}{1.988998in}}%
\pgfpathcurveto{\pgfqpoint{2.710011in}{1.980762in}}{\pgfqpoint{2.713283in}{1.972862in}}{\pgfqpoint{2.719107in}{1.967038in}}%
\pgfpathcurveto{\pgfqpoint{2.724931in}{1.961214in}}{\pgfqpoint{2.732831in}{1.957942in}}{\pgfqpoint{2.741068in}{1.957942in}}%
\pgfpathclose%
\pgfusepath{stroke,fill}%
\end{pgfscope}%
\begin{pgfscope}%
\pgfpathrectangle{\pgfqpoint{0.100000in}{0.212622in}}{\pgfqpoint{3.696000in}{3.696000in}}%
\pgfusepath{clip}%
\pgfsetbuttcap%
\pgfsetroundjoin%
\definecolor{currentfill}{rgb}{0.121569,0.466667,0.705882}%
\pgfsetfillcolor{currentfill}%
\pgfsetfillopacity{0.852888}%
\pgfsetlinewidth{1.003750pt}%
\definecolor{currentstroke}{rgb}{0.121569,0.466667,0.705882}%
\pgfsetstrokecolor{currentstroke}%
\pgfsetstrokeopacity{0.852888}%
\pgfsetdash{}{0pt}%
\pgfpathmoveto{\pgfqpoint{0.796329in}{2.458259in}}%
\pgfpathcurveto{\pgfqpoint{0.804565in}{2.458259in}}{\pgfqpoint{0.812465in}{2.461531in}}{\pgfqpoint{0.818289in}{2.467355in}}%
\pgfpathcurveto{\pgfqpoint{0.824113in}{2.473179in}}{\pgfqpoint{0.827385in}{2.481079in}}{\pgfqpoint{0.827385in}{2.489315in}}%
\pgfpathcurveto{\pgfqpoint{0.827385in}{2.497552in}}{\pgfqpoint{0.824113in}{2.505452in}}{\pgfqpoint{0.818289in}{2.511276in}}%
\pgfpathcurveto{\pgfqpoint{0.812465in}{2.517100in}}{\pgfqpoint{0.804565in}{2.520372in}}{\pgfqpoint{0.796329in}{2.520372in}}%
\pgfpathcurveto{\pgfqpoint{0.788092in}{2.520372in}}{\pgfqpoint{0.780192in}{2.517100in}}{\pgfqpoint{0.774368in}{2.511276in}}%
\pgfpathcurveto{\pgfqpoint{0.768544in}{2.505452in}}{\pgfqpoint{0.765272in}{2.497552in}}{\pgfqpoint{0.765272in}{2.489315in}}%
\pgfpathcurveto{\pgfqpoint{0.765272in}{2.481079in}}{\pgfqpoint{0.768544in}{2.473179in}}{\pgfqpoint{0.774368in}{2.467355in}}%
\pgfpathcurveto{\pgfqpoint{0.780192in}{2.461531in}}{\pgfqpoint{0.788092in}{2.458259in}}{\pgfqpoint{0.796329in}{2.458259in}}%
\pgfpathclose%
\pgfusepath{stroke,fill}%
\end{pgfscope}%
\begin{pgfscope}%
\pgfpathrectangle{\pgfqpoint{0.100000in}{0.212622in}}{\pgfqpoint{3.696000in}{3.696000in}}%
\pgfusepath{clip}%
\pgfsetbuttcap%
\pgfsetroundjoin%
\definecolor{currentfill}{rgb}{0.121569,0.466667,0.705882}%
\pgfsetfillcolor{currentfill}%
\pgfsetfillopacity{0.853955}%
\pgfsetlinewidth{1.003750pt}%
\definecolor{currentstroke}{rgb}{0.121569,0.466667,0.705882}%
\pgfsetstrokecolor{currentstroke}%
\pgfsetstrokeopacity{0.853955}%
\pgfsetdash{}{0pt}%
\pgfpathmoveto{\pgfqpoint{2.737367in}{1.954030in}}%
\pgfpathcurveto{\pgfqpoint{2.745603in}{1.954030in}}{\pgfqpoint{2.753503in}{1.957303in}}{\pgfqpoint{2.759327in}{1.963127in}}%
\pgfpathcurveto{\pgfqpoint{2.765151in}{1.968951in}}{\pgfqpoint{2.768424in}{1.976851in}}{\pgfqpoint{2.768424in}{1.985087in}}%
\pgfpathcurveto{\pgfqpoint{2.768424in}{1.993323in}}{\pgfqpoint{2.765151in}{2.001223in}}{\pgfqpoint{2.759327in}{2.007047in}}%
\pgfpathcurveto{\pgfqpoint{2.753503in}{2.012871in}}{\pgfqpoint{2.745603in}{2.016143in}}{\pgfqpoint{2.737367in}{2.016143in}}%
\pgfpathcurveto{\pgfqpoint{2.729131in}{2.016143in}}{\pgfqpoint{2.721231in}{2.012871in}}{\pgfqpoint{2.715407in}{2.007047in}}%
\pgfpathcurveto{\pgfqpoint{2.709583in}{2.001223in}}{\pgfqpoint{2.706311in}{1.993323in}}{\pgfqpoint{2.706311in}{1.985087in}}%
\pgfpathcurveto{\pgfqpoint{2.706311in}{1.976851in}}{\pgfqpoint{2.709583in}{1.968951in}}{\pgfqpoint{2.715407in}{1.963127in}}%
\pgfpathcurveto{\pgfqpoint{2.721231in}{1.957303in}}{\pgfqpoint{2.729131in}{1.954030in}}{\pgfqpoint{2.737367in}{1.954030in}}%
\pgfpathclose%
\pgfusepath{stroke,fill}%
\end{pgfscope}%
\begin{pgfscope}%
\pgfpathrectangle{\pgfqpoint{0.100000in}{0.212622in}}{\pgfqpoint{3.696000in}{3.696000in}}%
\pgfusepath{clip}%
\pgfsetbuttcap%
\pgfsetroundjoin%
\definecolor{currentfill}{rgb}{0.121569,0.466667,0.705882}%
\pgfsetfillcolor{currentfill}%
\pgfsetfillopacity{0.854100}%
\pgfsetlinewidth{1.003750pt}%
\definecolor{currentstroke}{rgb}{0.121569,0.466667,0.705882}%
\pgfsetstrokecolor{currentstroke}%
\pgfsetstrokeopacity{0.854100}%
\pgfsetdash{}{0pt}%
\pgfpathmoveto{\pgfqpoint{0.806513in}{2.453872in}}%
\pgfpathcurveto{\pgfqpoint{0.814750in}{2.453872in}}{\pgfqpoint{0.822650in}{2.457145in}}{\pgfqpoint{0.828474in}{2.462969in}}%
\pgfpathcurveto{\pgfqpoint{0.834298in}{2.468793in}}{\pgfqpoint{0.837570in}{2.476693in}}{\pgfqpoint{0.837570in}{2.484929in}}%
\pgfpathcurveto{\pgfqpoint{0.837570in}{2.493165in}}{\pgfqpoint{0.834298in}{2.501065in}}{\pgfqpoint{0.828474in}{2.506889in}}%
\pgfpathcurveto{\pgfqpoint{0.822650in}{2.512713in}}{\pgfqpoint{0.814750in}{2.515985in}}{\pgfqpoint{0.806513in}{2.515985in}}%
\pgfpathcurveto{\pgfqpoint{0.798277in}{2.515985in}}{\pgfqpoint{0.790377in}{2.512713in}}{\pgfqpoint{0.784553in}{2.506889in}}%
\pgfpathcurveto{\pgfqpoint{0.778729in}{2.501065in}}{\pgfqpoint{0.775457in}{2.493165in}}{\pgfqpoint{0.775457in}{2.484929in}}%
\pgfpathcurveto{\pgfqpoint{0.775457in}{2.476693in}}{\pgfqpoint{0.778729in}{2.468793in}}{\pgfqpoint{0.784553in}{2.462969in}}%
\pgfpathcurveto{\pgfqpoint{0.790377in}{2.457145in}}{\pgfqpoint{0.798277in}{2.453872in}}{\pgfqpoint{0.806513in}{2.453872in}}%
\pgfpathclose%
\pgfusepath{stroke,fill}%
\end{pgfscope}%
\begin{pgfscope}%
\pgfpathrectangle{\pgfqpoint{0.100000in}{0.212622in}}{\pgfqpoint{3.696000in}{3.696000in}}%
\pgfusepath{clip}%
\pgfsetbuttcap%
\pgfsetroundjoin%
\definecolor{currentfill}{rgb}{0.121569,0.466667,0.705882}%
\pgfsetfillcolor{currentfill}%
\pgfsetfillopacity{0.855020}%
\pgfsetlinewidth{1.003750pt}%
\definecolor{currentstroke}{rgb}{0.121569,0.466667,0.705882}%
\pgfsetstrokecolor{currentstroke}%
\pgfsetstrokeopacity{0.855020}%
\pgfsetdash{}{0pt}%
\pgfpathmoveto{\pgfqpoint{2.735558in}{1.952154in}}%
\pgfpathcurveto{\pgfqpoint{2.743794in}{1.952154in}}{\pgfqpoint{2.751694in}{1.955427in}}{\pgfqpoint{2.757518in}{1.961250in}}%
\pgfpathcurveto{\pgfqpoint{2.763342in}{1.967074in}}{\pgfqpoint{2.766614in}{1.974974in}}{\pgfqpoint{2.766614in}{1.983211in}}%
\pgfpathcurveto{\pgfqpoint{2.766614in}{1.991447in}}{\pgfqpoint{2.763342in}{1.999347in}}{\pgfqpoint{2.757518in}{2.005171in}}%
\pgfpathcurveto{\pgfqpoint{2.751694in}{2.010995in}}{\pgfqpoint{2.743794in}{2.014267in}}{\pgfqpoint{2.735558in}{2.014267in}}%
\pgfpathcurveto{\pgfqpoint{2.727321in}{2.014267in}}{\pgfqpoint{2.719421in}{2.010995in}}{\pgfqpoint{2.713597in}{2.005171in}}%
\pgfpathcurveto{\pgfqpoint{2.707773in}{1.999347in}}{\pgfqpoint{2.704501in}{1.991447in}}{\pgfqpoint{2.704501in}{1.983211in}}%
\pgfpathcurveto{\pgfqpoint{2.704501in}{1.974974in}}{\pgfqpoint{2.707773in}{1.967074in}}{\pgfqpoint{2.713597in}{1.961250in}}%
\pgfpathcurveto{\pgfqpoint{2.719421in}{1.955427in}}{\pgfqpoint{2.727321in}{1.952154in}}{\pgfqpoint{2.735558in}{1.952154in}}%
\pgfpathclose%
\pgfusepath{stroke,fill}%
\end{pgfscope}%
\begin{pgfscope}%
\pgfpathrectangle{\pgfqpoint{0.100000in}{0.212622in}}{\pgfqpoint{3.696000in}{3.696000in}}%
\pgfusepath{clip}%
\pgfsetbuttcap%
\pgfsetroundjoin%
\definecolor{currentfill}{rgb}{0.121569,0.466667,0.705882}%
\pgfsetfillcolor{currentfill}%
\pgfsetfillopacity{0.856268}%
\pgfsetlinewidth{1.003750pt}%
\definecolor{currentstroke}{rgb}{0.121569,0.466667,0.705882}%
\pgfsetstrokecolor{currentstroke}%
\pgfsetstrokeopacity{0.856268}%
\pgfsetdash{}{0pt}%
\pgfpathmoveto{\pgfqpoint{0.825143in}{2.445973in}}%
\pgfpathcurveto{\pgfqpoint{0.833379in}{2.445973in}}{\pgfqpoint{0.841280in}{2.449245in}}{\pgfqpoint{0.847103in}{2.455069in}}%
\pgfpathcurveto{\pgfqpoint{0.852927in}{2.460893in}}{\pgfqpoint{0.856200in}{2.468793in}}{\pgfqpoint{0.856200in}{2.477030in}}%
\pgfpathcurveto{\pgfqpoint{0.856200in}{2.485266in}}{\pgfqpoint{0.852927in}{2.493166in}}{\pgfqpoint{0.847103in}{2.498990in}}%
\pgfpathcurveto{\pgfqpoint{0.841280in}{2.504814in}}{\pgfqpoint{0.833379in}{2.508086in}}{\pgfqpoint{0.825143in}{2.508086in}}%
\pgfpathcurveto{\pgfqpoint{0.816907in}{2.508086in}}{\pgfqpoint{0.809007in}{2.504814in}}{\pgfqpoint{0.803183in}{2.498990in}}%
\pgfpathcurveto{\pgfqpoint{0.797359in}{2.493166in}}{\pgfqpoint{0.794087in}{2.485266in}}{\pgfqpoint{0.794087in}{2.477030in}}%
\pgfpathcurveto{\pgfqpoint{0.794087in}{2.468793in}}{\pgfqpoint{0.797359in}{2.460893in}}{\pgfqpoint{0.803183in}{2.455069in}}%
\pgfpathcurveto{\pgfqpoint{0.809007in}{2.449245in}}{\pgfqpoint{0.816907in}{2.445973in}}{\pgfqpoint{0.825143in}{2.445973in}}%
\pgfpathclose%
\pgfusepath{stroke,fill}%
\end{pgfscope}%
\begin{pgfscope}%
\pgfpathrectangle{\pgfqpoint{0.100000in}{0.212622in}}{\pgfqpoint{3.696000in}{3.696000in}}%
\pgfusepath{clip}%
\pgfsetbuttcap%
\pgfsetroundjoin%
\definecolor{currentfill}{rgb}{0.121569,0.466667,0.705882}%
\pgfsetfillcolor{currentfill}%
\pgfsetfillopacity{0.856929}%
\pgfsetlinewidth{1.003750pt}%
\definecolor{currentstroke}{rgb}{0.121569,0.466667,0.705882}%
\pgfsetstrokecolor{currentstroke}%
\pgfsetstrokeopacity{0.856929}%
\pgfsetdash{}{0pt}%
\pgfpathmoveto{\pgfqpoint{2.731884in}{1.948806in}}%
\pgfpathcurveto{\pgfqpoint{2.740120in}{1.948806in}}{\pgfqpoint{2.748020in}{1.952078in}}{\pgfqpoint{2.753844in}{1.957902in}}%
\pgfpathcurveto{\pgfqpoint{2.759668in}{1.963726in}}{\pgfqpoint{2.762940in}{1.971626in}}{\pgfqpoint{2.762940in}{1.979862in}}%
\pgfpathcurveto{\pgfqpoint{2.762940in}{1.988098in}}{\pgfqpoint{2.759668in}{1.995999in}}{\pgfqpoint{2.753844in}{2.001822in}}%
\pgfpathcurveto{\pgfqpoint{2.748020in}{2.007646in}}{\pgfqpoint{2.740120in}{2.010919in}}{\pgfqpoint{2.731884in}{2.010919in}}%
\pgfpathcurveto{\pgfqpoint{2.723647in}{2.010919in}}{\pgfqpoint{2.715747in}{2.007646in}}{\pgfqpoint{2.709923in}{2.001822in}}%
\pgfpathcurveto{\pgfqpoint{2.704099in}{1.995999in}}{\pgfqpoint{2.700827in}{1.988098in}}{\pgfqpoint{2.700827in}{1.979862in}}%
\pgfpathcurveto{\pgfqpoint{2.700827in}{1.971626in}}{\pgfqpoint{2.704099in}{1.963726in}}{\pgfqpoint{2.709923in}{1.957902in}}%
\pgfpathcurveto{\pgfqpoint{2.715747in}{1.952078in}}{\pgfqpoint{2.723647in}{1.948806in}}{\pgfqpoint{2.731884in}{1.948806in}}%
\pgfpathclose%
\pgfusepath{stroke,fill}%
\end{pgfscope}%
\begin{pgfscope}%
\pgfpathrectangle{\pgfqpoint{0.100000in}{0.212622in}}{\pgfqpoint{3.696000in}{3.696000in}}%
\pgfusepath{clip}%
\pgfsetbuttcap%
\pgfsetroundjoin%
\definecolor{currentfill}{rgb}{0.121569,0.466667,0.705882}%
\pgfsetfillcolor{currentfill}%
\pgfsetfillopacity{0.857925}%
\pgfsetlinewidth{1.003750pt}%
\definecolor{currentstroke}{rgb}{0.121569,0.466667,0.705882}%
\pgfsetstrokecolor{currentstroke}%
\pgfsetstrokeopacity{0.857925}%
\pgfsetdash{}{0pt}%
\pgfpathmoveto{\pgfqpoint{2.729818in}{1.946664in}}%
\pgfpathcurveto{\pgfqpoint{2.738054in}{1.946664in}}{\pgfqpoint{2.745954in}{1.949936in}}{\pgfqpoint{2.751778in}{1.955760in}}%
\pgfpathcurveto{\pgfqpoint{2.757602in}{1.961584in}}{\pgfqpoint{2.760874in}{1.969484in}}{\pgfqpoint{2.760874in}{1.977721in}}%
\pgfpathcurveto{\pgfqpoint{2.760874in}{1.985957in}}{\pgfqpoint{2.757602in}{1.993857in}}{\pgfqpoint{2.751778in}{1.999681in}}%
\pgfpathcurveto{\pgfqpoint{2.745954in}{2.005505in}}{\pgfqpoint{2.738054in}{2.008777in}}{\pgfqpoint{2.729818in}{2.008777in}}%
\pgfpathcurveto{\pgfqpoint{2.721582in}{2.008777in}}{\pgfqpoint{2.713681in}{2.005505in}}{\pgfqpoint{2.707858in}{1.999681in}}%
\pgfpathcurveto{\pgfqpoint{2.702034in}{1.993857in}}{\pgfqpoint{2.698761in}{1.985957in}}{\pgfqpoint{2.698761in}{1.977721in}}%
\pgfpathcurveto{\pgfqpoint{2.698761in}{1.969484in}}{\pgfqpoint{2.702034in}{1.961584in}}{\pgfqpoint{2.707858in}{1.955760in}}%
\pgfpathcurveto{\pgfqpoint{2.713681in}{1.949936in}}{\pgfqpoint{2.721582in}{1.946664in}}{\pgfqpoint{2.729818in}{1.946664in}}%
\pgfpathclose%
\pgfusepath{stroke,fill}%
\end{pgfscope}%
\begin{pgfscope}%
\pgfpathrectangle{\pgfqpoint{0.100000in}{0.212622in}}{\pgfqpoint{3.696000in}{3.696000in}}%
\pgfusepath{clip}%
\pgfsetbuttcap%
\pgfsetroundjoin%
\definecolor{currentfill}{rgb}{0.121569,0.466667,0.705882}%
\pgfsetfillcolor{currentfill}%
\pgfsetfillopacity{0.858216}%
\pgfsetlinewidth{1.003750pt}%
\definecolor{currentstroke}{rgb}{0.121569,0.466667,0.705882}%
\pgfsetstrokecolor{currentstroke}%
\pgfsetstrokeopacity{0.858216}%
\pgfsetdash{}{0pt}%
\pgfpathmoveto{\pgfqpoint{0.842887in}{2.437450in}}%
\pgfpathcurveto{\pgfqpoint{0.851123in}{2.437450in}}{\pgfqpoint{0.859023in}{2.440723in}}{\pgfqpoint{0.864847in}{2.446547in}}%
\pgfpathcurveto{\pgfqpoint{0.870671in}{2.452371in}}{\pgfqpoint{0.873943in}{2.460271in}}{\pgfqpoint{0.873943in}{2.468507in}}%
\pgfpathcurveto{\pgfqpoint{0.873943in}{2.476743in}}{\pgfqpoint{0.870671in}{2.484643in}}{\pgfqpoint{0.864847in}{2.490467in}}%
\pgfpathcurveto{\pgfqpoint{0.859023in}{2.496291in}}{\pgfqpoint{0.851123in}{2.499563in}}{\pgfqpoint{0.842887in}{2.499563in}}%
\pgfpathcurveto{\pgfqpoint{0.834650in}{2.499563in}}{\pgfqpoint{0.826750in}{2.496291in}}{\pgfqpoint{0.820926in}{2.490467in}}%
\pgfpathcurveto{\pgfqpoint{0.815102in}{2.484643in}}{\pgfqpoint{0.811830in}{2.476743in}}{\pgfqpoint{0.811830in}{2.468507in}}%
\pgfpathcurveto{\pgfqpoint{0.811830in}{2.460271in}}{\pgfqpoint{0.815102in}{2.452371in}}{\pgfqpoint{0.820926in}{2.446547in}}%
\pgfpathcurveto{\pgfqpoint{0.826750in}{2.440723in}}{\pgfqpoint{0.834650in}{2.437450in}}{\pgfqpoint{0.842887in}{2.437450in}}%
\pgfpathclose%
\pgfusepath{stroke,fill}%
\end{pgfscope}%
\begin{pgfscope}%
\pgfpathrectangle{\pgfqpoint{0.100000in}{0.212622in}}{\pgfqpoint{3.696000in}{3.696000in}}%
\pgfusepath{clip}%
\pgfsetbuttcap%
\pgfsetroundjoin%
\definecolor{currentfill}{rgb}{0.121569,0.466667,0.705882}%
\pgfsetfillcolor{currentfill}%
\pgfsetfillopacity{0.858506}%
\pgfsetlinewidth{1.003750pt}%
\definecolor{currentstroke}{rgb}{0.121569,0.466667,0.705882}%
\pgfsetstrokecolor{currentstroke}%
\pgfsetstrokeopacity{0.858506}%
\pgfsetdash{}{0pt}%
\pgfpathmoveto{\pgfqpoint{2.728799in}{1.945580in}}%
\pgfpathcurveto{\pgfqpoint{2.737035in}{1.945580in}}{\pgfqpoint{2.744935in}{1.948852in}}{\pgfqpoint{2.750759in}{1.954676in}}%
\pgfpathcurveto{\pgfqpoint{2.756583in}{1.960500in}}{\pgfqpoint{2.759856in}{1.968400in}}{\pgfqpoint{2.759856in}{1.976636in}}%
\pgfpathcurveto{\pgfqpoint{2.759856in}{1.984872in}}{\pgfqpoint{2.756583in}{1.992772in}}{\pgfqpoint{2.750759in}{1.998596in}}%
\pgfpathcurveto{\pgfqpoint{2.744935in}{2.004420in}}{\pgfqpoint{2.737035in}{2.007693in}}{\pgfqpoint{2.728799in}{2.007693in}}%
\pgfpathcurveto{\pgfqpoint{2.720563in}{2.007693in}}{\pgfqpoint{2.712663in}{2.004420in}}{\pgfqpoint{2.706839in}{1.998596in}}%
\pgfpathcurveto{\pgfqpoint{2.701015in}{1.992772in}}{\pgfqpoint{2.697743in}{1.984872in}}{\pgfqpoint{2.697743in}{1.976636in}}%
\pgfpathcurveto{\pgfqpoint{2.697743in}{1.968400in}}{\pgfqpoint{2.701015in}{1.960500in}}{\pgfqpoint{2.706839in}{1.954676in}}%
\pgfpathcurveto{\pgfqpoint{2.712663in}{1.948852in}}{\pgfqpoint{2.720563in}{1.945580in}}{\pgfqpoint{2.728799in}{1.945580in}}%
\pgfpathclose%
\pgfusepath{stroke,fill}%
\end{pgfscope}%
\begin{pgfscope}%
\pgfpathrectangle{\pgfqpoint{0.100000in}{0.212622in}}{\pgfqpoint{3.696000in}{3.696000in}}%
\pgfusepath{clip}%
\pgfsetbuttcap%
\pgfsetroundjoin%
\definecolor{currentfill}{rgb}{0.121569,0.466667,0.705882}%
\pgfsetfillcolor{currentfill}%
\pgfsetfillopacity{0.859571}%
\pgfsetlinewidth{1.003750pt}%
\definecolor{currentstroke}{rgb}{0.121569,0.466667,0.705882}%
\pgfsetstrokecolor{currentstroke}%
\pgfsetstrokeopacity{0.859571}%
\pgfsetdash{}{0pt}%
\pgfpathmoveto{\pgfqpoint{0.857779in}{2.428664in}}%
\pgfpathcurveto{\pgfqpoint{0.866016in}{2.428664in}}{\pgfqpoint{0.873916in}{2.431936in}}{\pgfqpoint{0.879740in}{2.437760in}}%
\pgfpathcurveto{\pgfqpoint{0.885563in}{2.443584in}}{\pgfqpoint{0.888836in}{2.451484in}}{\pgfqpoint{0.888836in}{2.459721in}}%
\pgfpathcurveto{\pgfqpoint{0.888836in}{2.467957in}}{\pgfqpoint{0.885563in}{2.475857in}}{\pgfqpoint{0.879740in}{2.481681in}}%
\pgfpathcurveto{\pgfqpoint{0.873916in}{2.487505in}}{\pgfqpoint{0.866016in}{2.490777in}}{\pgfqpoint{0.857779in}{2.490777in}}%
\pgfpathcurveto{\pgfqpoint{0.849543in}{2.490777in}}{\pgfqpoint{0.841643in}{2.487505in}}{\pgfqpoint{0.835819in}{2.481681in}}%
\pgfpathcurveto{\pgfqpoint{0.829995in}{2.475857in}}{\pgfqpoint{0.826723in}{2.467957in}}{\pgfqpoint{0.826723in}{2.459721in}}%
\pgfpathcurveto{\pgfqpoint{0.826723in}{2.451484in}}{\pgfqpoint{0.829995in}{2.443584in}}{\pgfqpoint{0.835819in}{2.437760in}}%
\pgfpathcurveto{\pgfqpoint{0.841643in}{2.431936in}}{\pgfqpoint{0.849543in}{2.428664in}}{\pgfqpoint{0.857779in}{2.428664in}}%
\pgfpathclose%
\pgfusepath{stroke,fill}%
\end{pgfscope}%
\begin{pgfscope}%
\pgfpathrectangle{\pgfqpoint{0.100000in}{0.212622in}}{\pgfqpoint{3.696000in}{3.696000in}}%
\pgfusepath{clip}%
\pgfsetbuttcap%
\pgfsetroundjoin%
\definecolor{currentfill}{rgb}{0.121569,0.466667,0.705882}%
\pgfsetfillcolor{currentfill}%
\pgfsetfillopacity{0.859661}%
\pgfsetlinewidth{1.003750pt}%
\definecolor{currentstroke}{rgb}{0.121569,0.466667,0.705882}%
\pgfsetstrokecolor{currentstroke}%
\pgfsetstrokeopacity{0.859661}%
\pgfsetdash{}{0pt}%
\pgfpathmoveto{\pgfqpoint{2.726534in}{1.942709in}}%
\pgfpathcurveto{\pgfqpoint{2.734770in}{1.942709in}}{\pgfqpoint{2.742670in}{1.945981in}}{\pgfqpoint{2.748494in}{1.951805in}}%
\pgfpathcurveto{\pgfqpoint{2.754318in}{1.957629in}}{\pgfqpoint{2.757590in}{1.965529in}}{\pgfqpoint{2.757590in}{1.973765in}}%
\pgfpathcurveto{\pgfqpoint{2.757590in}{1.982001in}}{\pgfqpoint{2.754318in}{1.989901in}}{\pgfqpoint{2.748494in}{1.995725in}}%
\pgfpathcurveto{\pgfqpoint{2.742670in}{2.001549in}}{\pgfqpoint{2.734770in}{2.004822in}}{\pgfqpoint{2.726534in}{2.004822in}}%
\pgfpathcurveto{\pgfqpoint{2.718297in}{2.004822in}}{\pgfqpoint{2.710397in}{2.001549in}}{\pgfqpoint{2.704573in}{1.995725in}}%
\pgfpathcurveto{\pgfqpoint{2.698750in}{1.989901in}}{\pgfqpoint{2.695477in}{1.982001in}}{\pgfqpoint{2.695477in}{1.973765in}}%
\pgfpathcurveto{\pgfqpoint{2.695477in}{1.965529in}}{\pgfqpoint{2.698750in}{1.957629in}}{\pgfqpoint{2.704573in}{1.951805in}}%
\pgfpathcurveto{\pgfqpoint{2.710397in}{1.945981in}}{\pgfqpoint{2.718297in}{1.942709in}}{\pgfqpoint{2.726534in}{1.942709in}}%
\pgfpathclose%
\pgfusepath{stroke,fill}%
\end{pgfscope}%
\begin{pgfscope}%
\pgfpathrectangle{\pgfqpoint{0.100000in}{0.212622in}}{\pgfqpoint{3.696000in}{3.696000in}}%
\pgfusepath{clip}%
\pgfsetbuttcap%
\pgfsetroundjoin%
\definecolor{currentfill}{rgb}{0.121569,0.466667,0.705882}%
\pgfsetfillcolor{currentfill}%
\pgfsetfillopacity{0.860640}%
\pgfsetlinewidth{1.003750pt}%
\definecolor{currentstroke}{rgb}{0.121569,0.466667,0.705882}%
\pgfsetstrokecolor{currentstroke}%
\pgfsetstrokeopacity{0.860640}%
\pgfsetdash{}{0pt}%
\pgfpathmoveto{\pgfqpoint{0.870005in}{2.421431in}}%
\pgfpathcurveto{\pgfqpoint{0.878241in}{2.421431in}}{\pgfqpoint{0.886141in}{2.424703in}}{\pgfqpoint{0.891965in}{2.430527in}}%
\pgfpathcurveto{\pgfqpoint{0.897789in}{2.436351in}}{\pgfqpoint{0.901062in}{2.444251in}}{\pgfqpoint{0.901062in}{2.452487in}}%
\pgfpathcurveto{\pgfqpoint{0.901062in}{2.460724in}}{\pgfqpoint{0.897789in}{2.468624in}}{\pgfqpoint{0.891965in}{2.474447in}}%
\pgfpathcurveto{\pgfqpoint{0.886141in}{2.480271in}}{\pgfqpoint{0.878241in}{2.483544in}}{\pgfqpoint{0.870005in}{2.483544in}}%
\pgfpathcurveto{\pgfqpoint{0.861769in}{2.483544in}}{\pgfqpoint{0.853869in}{2.480271in}}{\pgfqpoint{0.848045in}{2.474447in}}%
\pgfpathcurveto{\pgfqpoint{0.842221in}{2.468624in}}{\pgfqpoint{0.838949in}{2.460724in}}{\pgfqpoint{0.838949in}{2.452487in}}%
\pgfpathcurveto{\pgfqpoint{0.838949in}{2.444251in}}{\pgfqpoint{0.842221in}{2.436351in}}{\pgfqpoint{0.848045in}{2.430527in}}%
\pgfpathcurveto{\pgfqpoint{0.853869in}{2.424703in}}{\pgfqpoint{0.861769in}{2.421431in}}{\pgfqpoint{0.870005in}{2.421431in}}%
\pgfpathclose%
\pgfusepath{stroke,fill}%
\end{pgfscope}%
\begin{pgfscope}%
\pgfpathrectangle{\pgfqpoint{0.100000in}{0.212622in}}{\pgfqpoint{3.696000in}{3.696000in}}%
\pgfusepath{clip}%
\pgfsetbuttcap%
\pgfsetroundjoin%
\definecolor{currentfill}{rgb}{0.121569,0.466667,0.705882}%
\pgfsetfillcolor{currentfill}%
\pgfsetfillopacity{0.861033}%
\pgfsetlinewidth{1.003750pt}%
\definecolor{currentstroke}{rgb}{0.121569,0.466667,0.705882}%
\pgfsetstrokecolor{currentstroke}%
\pgfsetstrokeopacity{0.861033}%
\pgfsetdash{}{0pt}%
\pgfpathmoveto{\pgfqpoint{2.723897in}{1.939384in}}%
\pgfpathcurveto{\pgfqpoint{2.732134in}{1.939384in}}{\pgfqpoint{2.740034in}{1.942656in}}{\pgfqpoint{2.745858in}{1.948480in}}%
\pgfpathcurveto{\pgfqpoint{2.751682in}{1.954304in}}{\pgfqpoint{2.754954in}{1.962204in}}{\pgfqpoint{2.754954in}{1.970440in}}%
\pgfpathcurveto{\pgfqpoint{2.754954in}{1.978677in}}{\pgfqpoint{2.751682in}{1.986577in}}{\pgfqpoint{2.745858in}{1.992401in}}%
\pgfpathcurveto{\pgfqpoint{2.740034in}{1.998225in}}{\pgfqpoint{2.732134in}{2.001497in}}{\pgfqpoint{2.723897in}{2.001497in}}%
\pgfpathcurveto{\pgfqpoint{2.715661in}{2.001497in}}{\pgfqpoint{2.707761in}{1.998225in}}{\pgfqpoint{2.701937in}{1.992401in}}%
\pgfpathcurveto{\pgfqpoint{2.696113in}{1.986577in}}{\pgfqpoint{2.692841in}{1.978677in}}{\pgfqpoint{2.692841in}{1.970440in}}%
\pgfpathcurveto{\pgfqpoint{2.692841in}{1.962204in}}{\pgfqpoint{2.696113in}{1.954304in}}{\pgfqpoint{2.701937in}{1.948480in}}%
\pgfpathcurveto{\pgfqpoint{2.707761in}{1.942656in}}{\pgfqpoint{2.715661in}{1.939384in}}{\pgfqpoint{2.723897in}{1.939384in}}%
\pgfpathclose%
\pgfusepath{stroke,fill}%
\end{pgfscope}%
\begin{pgfscope}%
\pgfpathrectangle{\pgfqpoint{0.100000in}{0.212622in}}{\pgfqpoint{3.696000in}{3.696000in}}%
\pgfusepath{clip}%
\pgfsetbuttcap%
\pgfsetroundjoin%
\definecolor{currentfill}{rgb}{0.121569,0.466667,0.705882}%
\pgfsetfillcolor{currentfill}%
\pgfsetfillopacity{0.861847}%
\pgfsetlinewidth{1.003750pt}%
\definecolor{currentstroke}{rgb}{0.121569,0.466667,0.705882}%
\pgfsetstrokecolor{currentstroke}%
\pgfsetstrokeopacity{0.861847}%
\pgfsetdash{}{0pt}%
\pgfpathmoveto{\pgfqpoint{2.722538in}{1.937840in}}%
\pgfpathcurveto{\pgfqpoint{2.730774in}{1.937840in}}{\pgfqpoint{2.738674in}{1.941113in}}{\pgfqpoint{2.744498in}{1.946937in}}%
\pgfpathcurveto{\pgfqpoint{2.750322in}{1.952761in}}{\pgfqpoint{2.753594in}{1.960661in}}{\pgfqpoint{2.753594in}{1.968897in}}%
\pgfpathcurveto{\pgfqpoint{2.753594in}{1.977133in}}{\pgfqpoint{2.750322in}{1.985033in}}{\pgfqpoint{2.744498in}{1.990857in}}%
\pgfpathcurveto{\pgfqpoint{2.738674in}{1.996681in}}{\pgfqpoint{2.730774in}{1.999953in}}{\pgfqpoint{2.722538in}{1.999953in}}%
\pgfpathcurveto{\pgfqpoint{2.714301in}{1.999953in}}{\pgfqpoint{2.706401in}{1.996681in}}{\pgfqpoint{2.700577in}{1.990857in}}%
\pgfpathcurveto{\pgfqpoint{2.694754in}{1.985033in}}{\pgfqpoint{2.691481in}{1.977133in}}{\pgfqpoint{2.691481in}{1.968897in}}%
\pgfpathcurveto{\pgfqpoint{2.691481in}{1.960661in}}{\pgfqpoint{2.694754in}{1.952761in}}{\pgfqpoint{2.700577in}{1.946937in}}%
\pgfpathcurveto{\pgfqpoint{2.706401in}{1.941113in}}{\pgfqpoint{2.714301in}{1.937840in}}{\pgfqpoint{2.722538in}{1.937840in}}%
\pgfpathclose%
\pgfusepath{stroke,fill}%
\end{pgfscope}%
\begin{pgfscope}%
\pgfpathrectangle{\pgfqpoint{0.100000in}{0.212622in}}{\pgfqpoint{3.696000in}{3.696000in}}%
\pgfusepath{clip}%
\pgfsetbuttcap%
\pgfsetroundjoin%
\definecolor{currentfill}{rgb}{0.121569,0.466667,0.705882}%
\pgfsetfillcolor{currentfill}%
\pgfsetfillopacity{0.862504}%
\pgfsetlinewidth{1.003750pt}%
\definecolor{currentstroke}{rgb}{0.121569,0.466667,0.705882}%
\pgfsetstrokecolor{currentstroke}%
\pgfsetstrokeopacity{0.862504}%
\pgfsetdash{}{0pt}%
\pgfpathmoveto{\pgfqpoint{0.892145in}{2.407366in}}%
\pgfpathcurveto{\pgfqpoint{0.900382in}{2.407366in}}{\pgfqpoint{0.908282in}{2.410639in}}{\pgfqpoint{0.914106in}{2.416463in}}%
\pgfpathcurveto{\pgfqpoint{0.919930in}{2.422287in}}{\pgfqpoint{0.923202in}{2.430187in}}{\pgfqpoint{0.923202in}{2.438423in}}%
\pgfpathcurveto{\pgfqpoint{0.923202in}{2.446659in}}{\pgfqpoint{0.919930in}{2.454559in}}{\pgfqpoint{0.914106in}{2.460383in}}%
\pgfpathcurveto{\pgfqpoint{0.908282in}{2.466207in}}{\pgfqpoint{0.900382in}{2.469479in}}{\pgfqpoint{0.892145in}{2.469479in}}%
\pgfpathcurveto{\pgfqpoint{0.883909in}{2.469479in}}{\pgfqpoint{0.876009in}{2.466207in}}{\pgfqpoint{0.870185in}{2.460383in}}%
\pgfpathcurveto{\pgfqpoint{0.864361in}{2.454559in}}{\pgfqpoint{0.861089in}{2.446659in}}{\pgfqpoint{0.861089in}{2.438423in}}%
\pgfpathcurveto{\pgfqpoint{0.861089in}{2.430187in}}{\pgfqpoint{0.864361in}{2.422287in}}{\pgfqpoint{0.870185in}{2.416463in}}%
\pgfpathcurveto{\pgfqpoint{0.876009in}{2.410639in}}{\pgfqpoint{0.883909in}{2.407366in}}{\pgfqpoint{0.892145in}{2.407366in}}%
\pgfpathclose%
\pgfusepath{stroke,fill}%
\end{pgfscope}%
\begin{pgfscope}%
\pgfpathrectangle{\pgfqpoint{0.100000in}{0.212622in}}{\pgfqpoint{3.696000in}{3.696000in}}%
\pgfusepath{clip}%
\pgfsetbuttcap%
\pgfsetroundjoin%
\definecolor{currentfill}{rgb}{0.121569,0.466667,0.705882}%
\pgfsetfillcolor{currentfill}%
\pgfsetfillopacity{0.863034}%
\pgfsetlinewidth{1.003750pt}%
\definecolor{currentstroke}{rgb}{0.121569,0.466667,0.705882}%
\pgfsetstrokecolor{currentstroke}%
\pgfsetstrokeopacity{0.863034}%
\pgfsetdash{}{0pt}%
\pgfpathmoveto{\pgfqpoint{2.720253in}{1.935210in}}%
\pgfpathcurveto{\pgfqpoint{2.728489in}{1.935210in}}{\pgfqpoint{2.736389in}{1.938482in}}{\pgfqpoint{2.742213in}{1.944306in}}%
\pgfpathcurveto{\pgfqpoint{2.748037in}{1.950130in}}{\pgfqpoint{2.751310in}{1.958030in}}{\pgfqpoint{2.751310in}{1.966266in}}%
\pgfpathcurveto{\pgfqpoint{2.751310in}{1.974503in}}{\pgfqpoint{2.748037in}{1.982403in}}{\pgfqpoint{2.742213in}{1.988227in}}%
\pgfpathcurveto{\pgfqpoint{2.736389in}{1.994050in}}{\pgfqpoint{2.728489in}{1.997323in}}{\pgfqpoint{2.720253in}{1.997323in}}%
\pgfpathcurveto{\pgfqpoint{2.712017in}{1.997323in}}{\pgfqpoint{2.704117in}{1.994050in}}{\pgfqpoint{2.698293in}{1.988227in}}%
\pgfpathcurveto{\pgfqpoint{2.692469in}{1.982403in}}{\pgfqpoint{2.689197in}{1.974503in}}{\pgfqpoint{2.689197in}{1.966266in}}%
\pgfpathcurveto{\pgfqpoint{2.689197in}{1.958030in}}{\pgfqpoint{2.692469in}{1.950130in}}{\pgfqpoint{2.698293in}{1.944306in}}%
\pgfpathcurveto{\pgfqpoint{2.704117in}{1.938482in}}{\pgfqpoint{2.712017in}{1.935210in}}{\pgfqpoint{2.720253in}{1.935210in}}%
\pgfpathclose%
\pgfusepath{stroke,fill}%
\end{pgfscope}%
\begin{pgfscope}%
\pgfpathrectangle{\pgfqpoint{0.100000in}{0.212622in}}{\pgfqpoint{3.696000in}{3.696000in}}%
\pgfusepath{clip}%
\pgfsetbuttcap%
\pgfsetroundjoin%
\definecolor{currentfill}{rgb}{0.121569,0.466667,0.705882}%
\pgfsetfillcolor{currentfill}%
\pgfsetfillopacity{0.864425}%
\pgfsetlinewidth{1.003750pt}%
\definecolor{currentstroke}{rgb}{0.121569,0.466667,0.705882}%
\pgfsetstrokecolor{currentstroke}%
\pgfsetstrokeopacity{0.864425}%
\pgfsetdash{}{0pt}%
\pgfpathmoveto{\pgfqpoint{0.913688in}{2.395013in}}%
\pgfpathcurveto{\pgfqpoint{0.921924in}{2.395013in}}{\pgfqpoint{0.929824in}{2.398285in}}{\pgfqpoint{0.935648in}{2.404109in}}%
\pgfpathcurveto{\pgfqpoint{0.941472in}{2.409933in}}{\pgfqpoint{0.944744in}{2.417833in}}{\pgfqpoint{0.944744in}{2.426069in}}%
\pgfpathcurveto{\pgfqpoint{0.944744in}{2.434306in}}{\pgfqpoint{0.941472in}{2.442206in}}{\pgfqpoint{0.935648in}{2.448030in}}%
\pgfpathcurveto{\pgfqpoint{0.929824in}{2.453853in}}{\pgfqpoint{0.921924in}{2.457126in}}{\pgfqpoint{0.913688in}{2.457126in}}%
\pgfpathcurveto{\pgfqpoint{0.905452in}{2.457126in}}{\pgfqpoint{0.897552in}{2.453853in}}{\pgfqpoint{0.891728in}{2.448030in}}%
\pgfpathcurveto{\pgfqpoint{0.885904in}{2.442206in}}{\pgfqpoint{0.882631in}{2.434306in}}{\pgfqpoint{0.882631in}{2.426069in}}%
\pgfpathcurveto{\pgfqpoint{0.882631in}{2.417833in}}{\pgfqpoint{0.885904in}{2.409933in}}{\pgfqpoint{0.891728in}{2.404109in}}%
\pgfpathcurveto{\pgfqpoint{0.897552in}{2.398285in}}{\pgfqpoint{0.905452in}{2.395013in}}{\pgfqpoint{0.913688in}{2.395013in}}%
\pgfpathclose%
\pgfusepath{stroke,fill}%
\end{pgfscope}%
\begin{pgfscope}%
\pgfpathrectangle{\pgfqpoint{0.100000in}{0.212622in}}{\pgfqpoint{3.696000in}{3.696000in}}%
\pgfusepath{clip}%
\pgfsetbuttcap%
\pgfsetroundjoin%
\definecolor{currentfill}{rgb}{0.121569,0.466667,0.705882}%
\pgfsetfillcolor{currentfill}%
\pgfsetfillopacity{0.864689}%
\pgfsetlinewidth{1.003750pt}%
\definecolor{currentstroke}{rgb}{0.121569,0.466667,0.705882}%
\pgfsetstrokecolor{currentstroke}%
\pgfsetstrokeopacity{0.864689}%
\pgfsetdash{}{0pt}%
\pgfpathmoveto{\pgfqpoint{2.717499in}{1.931670in}}%
\pgfpathcurveto{\pgfqpoint{2.725735in}{1.931670in}}{\pgfqpoint{2.733635in}{1.934942in}}{\pgfqpoint{2.739459in}{1.940766in}}%
\pgfpathcurveto{\pgfqpoint{2.745283in}{1.946590in}}{\pgfqpoint{2.748556in}{1.954490in}}{\pgfqpoint{2.748556in}{1.962726in}}%
\pgfpathcurveto{\pgfqpoint{2.748556in}{1.970962in}}{\pgfqpoint{2.745283in}{1.978862in}}{\pgfqpoint{2.739459in}{1.984686in}}%
\pgfpathcurveto{\pgfqpoint{2.733635in}{1.990510in}}{\pgfqpoint{2.725735in}{1.993783in}}{\pgfqpoint{2.717499in}{1.993783in}}%
\pgfpathcurveto{\pgfqpoint{2.709263in}{1.993783in}}{\pgfqpoint{2.701363in}{1.990510in}}{\pgfqpoint{2.695539in}{1.984686in}}%
\pgfpathcurveto{\pgfqpoint{2.689715in}{1.978862in}}{\pgfqpoint{2.686443in}{1.970962in}}{\pgfqpoint{2.686443in}{1.962726in}}%
\pgfpathcurveto{\pgfqpoint{2.686443in}{1.954490in}}{\pgfqpoint{2.689715in}{1.946590in}}{\pgfqpoint{2.695539in}{1.940766in}}%
\pgfpathcurveto{\pgfqpoint{2.701363in}{1.934942in}}{\pgfqpoint{2.709263in}{1.931670in}}{\pgfqpoint{2.717499in}{1.931670in}}%
\pgfpathclose%
\pgfusepath{stroke,fill}%
\end{pgfscope}%
\begin{pgfscope}%
\pgfpathrectangle{\pgfqpoint{0.100000in}{0.212622in}}{\pgfqpoint{3.696000in}{3.696000in}}%
\pgfusepath{clip}%
\pgfsetbuttcap%
\pgfsetroundjoin%
\definecolor{currentfill}{rgb}{0.121569,0.466667,0.705882}%
\pgfsetfillcolor{currentfill}%
\pgfsetfillopacity{0.865680}%
\pgfsetlinewidth{1.003750pt}%
\definecolor{currentstroke}{rgb}{0.121569,0.466667,0.705882}%
\pgfsetstrokecolor{currentstroke}%
\pgfsetstrokeopacity{0.865680}%
\pgfsetdash{}{0pt}%
\pgfpathmoveto{\pgfqpoint{2.715987in}{1.930229in}}%
\pgfpathcurveto{\pgfqpoint{2.724223in}{1.930229in}}{\pgfqpoint{2.732123in}{1.933502in}}{\pgfqpoint{2.737947in}{1.939326in}}%
\pgfpathcurveto{\pgfqpoint{2.743771in}{1.945150in}}{\pgfqpoint{2.747043in}{1.953050in}}{\pgfqpoint{2.747043in}{1.961286in}}%
\pgfpathcurveto{\pgfqpoint{2.747043in}{1.969522in}}{\pgfqpoint{2.743771in}{1.977422in}}{\pgfqpoint{2.737947in}{1.983246in}}%
\pgfpathcurveto{\pgfqpoint{2.732123in}{1.989070in}}{\pgfqpoint{2.724223in}{1.992342in}}{\pgfqpoint{2.715987in}{1.992342in}}%
\pgfpathcurveto{\pgfqpoint{2.707751in}{1.992342in}}{\pgfqpoint{2.699850in}{1.989070in}}{\pgfqpoint{2.694027in}{1.983246in}}%
\pgfpathcurveto{\pgfqpoint{2.688203in}{1.977422in}}{\pgfqpoint{2.684930in}{1.969522in}}{\pgfqpoint{2.684930in}{1.961286in}}%
\pgfpathcurveto{\pgfqpoint{2.684930in}{1.953050in}}{\pgfqpoint{2.688203in}{1.945150in}}{\pgfqpoint{2.694027in}{1.939326in}}%
\pgfpathcurveto{\pgfqpoint{2.699850in}{1.933502in}}{\pgfqpoint{2.707751in}{1.930229in}}{\pgfqpoint{2.715987in}{1.930229in}}%
\pgfpathclose%
\pgfusepath{stroke,fill}%
\end{pgfscope}%
\begin{pgfscope}%
\pgfpathrectangle{\pgfqpoint{0.100000in}{0.212622in}}{\pgfqpoint{3.696000in}{3.696000in}}%
\pgfusepath{clip}%
\pgfsetbuttcap%
\pgfsetroundjoin%
\definecolor{currentfill}{rgb}{0.121569,0.466667,0.705882}%
\pgfsetfillcolor{currentfill}%
\pgfsetfillopacity{0.866194}%
\pgfsetlinewidth{1.003750pt}%
\definecolor{currentstroke}{rgb}{0.121569,0.466667,0.705882}%
\pgfsetstrokecolor{currentstroke}%
\pgfsetstrokeopacity{0.866194}%
\pgfsetdash{}{0pt}%
\pgfpathmoveto{\pgfqpoint{2.715084in}{1.929304in}}%
\pgfpathcurveto{\pgfqpoint{2.723320in}{1.929304in}}{\pgfqpoint{2.731220in}{1.932576in}}{\pgfqpoint{2.737044in}{1.938400in}}%
\pgfpathcurveto{\pgfqpoint{2.742868in}{1.944224in}}{\pgfqpoint{2.746141in}{1.952124in}}{\pgfqpoint{2.746141in}{1.960360in}}%
\pgfpathcurveto{\pgfqpoint{2.746141in}{1.968597in}}{\pgfqpoint{2.742868in}{1.976497in}}{\pgfqpoint{2.737044in}{1.982321in}}%
\pgfpathcurveto{\pgfqpoint{2.731220in}{1.988145in}}{\pgfqpoint{2.723320in}{1.991417in}}{\pgfqpoint{2.715084in}{1.991417in}}%
\pgfpathcurveto{\pgfqpoint{2.706848in}{1.991417in}}{\pgfqpoint{2.698948in}{1.988145in}}{\pgfqpoint{2.693124in}{1.982321in}}%
\pgfpathcurveto{\pgfqpoint{2.687300in}{1.976497in}}{\pgfqpoint{2.684028in}{1.968597in}}{\pgfqpoint{2.684028in}{1.960360in}}%
\pgfpathcurveto{\pgfqpoint{2.684028in}{1.952124in}}{\pgfqpoint{2.687300in}{1.944224in}}{\pgfqpoint{2.693124in}{1.938400in}}%
\pgfpathcurveto{\pgfqpoint{2.698948in}{1.932576in}}{\pgfqpoint{2.706848in}{1.929304in}}{\pgfqpoint{2.715084in}{1.929304in}}%
\pgfpathclose%
\pgfusepath{stroke,fill}%
\end{pgfscope}%
\begin{pgfscope}%
\pgfpathrectangle{\pgfqpoint{0.100000in}{0.212622in}}{\pgfqpoint{3.696000in}{3.696000in}}%
\pgfusepath{clip}%
\pgfsetbuttcap%
\pgfsetroundjoin%
\definecolor{currentfill}{rgb}{0.121569,0.466667,0.705882}%
\pgfsetfillcolor{currentfill}%
\pgfsetfillopacity{0.866680}%
\pgfsetlinewidth{1.003750pt}%
\definecolor{currentstroke}{rgb}{0.121569,0.466667,0.705882}%
\pgfsetstrokecolor{currentstroke}%
\pgfsetstrokeopacity{0.866680}%
\pgfsetdash{}{0pt}%
\pgfpathmoveto{\pgfqpoint{0.934343in}{2.385838in}}%
\pgfpathcurveto{\pgfqpoint{0.942579in}{2.385838in}}{\pgfqpoint{0.950479in}{2.389110in}}{\pgfqpoint{0.956303in}{2.394934in}}%
\pgfpathcurveto{\pgfqpoint{0.962127in}{2.400758in}}{\pgfqpoint{0.965399in}{2.408658in}}{\pgfqpoint{0.965399in}{2.416894in}}%
\pgfpathcurveto{\pgfqpoint{0.965399in}{2.425131in}}{\pgfqpoint{0.962127in}{2.433031in}}{\pgfqpoint{0.956303in}{2.438855in}}%
\pgfpathcurveto{\pgfqpoint{0.950479in}{2.444679in}}{\pgfqpoint{0.942579in}{2.447951in}}{\pgfqpoint{0.934343in}{2.447951in}}%
\pgfpathcurveto{\pgfqpoint{0.926106in}{2.447951in}}{\pgfqpoint{0.918206in}{2.444679in}}{\pgfqpoint{0.912382in}{2.438855in}}%
\pgfpathcurveto{\pgfqpoint{0.906558in}{2.433031in}}{\pgfqpoint{0.903286in}{2.425131in}}{\pgfqpoint{0.903286in}{2.416894in}}%
\pgfpathcurveto{\pgfqpoint{0.903286in}{2.408658in}}{\pgfqpoint{0.906558in}{2.400758in}}{\pgfqpoint{0.912382in}{2.394934in}}%
\pgfpathcurveto{\pgfqpoint{0.918206in}{2.389110in}}{\pgfqpoint{0.926106in}{2.385838in}}{\pgfqpoint{0.934343in}{2.385838in}}%
\pgfpathclose%
\pgfusepath{stroke,fill}%
\end{pgfscope}%
\begin{pgfscope}%
\pgfpathrectangle{\pgfqpoint{0.100000in}{0.212622in}}{\pgfqpoint{3.696000in}{3.696000in}}%
\pgfusepath{clip}%
\pgfsetbuttcap%
\pgfsetroundjoin%
\definecolor{currentfill}{rgb}{0.121569,0.466667,0.705882}%
\pgfsetfillcolor{currentfill}%
\pgfsetfillopacity{0.867112}%
\pgfsetlinewidth{1.003750pt}%
\definecolor{currentstroke}{rgb}{0.121569,0.466667,0.705882}%
\pgfsetstrokecolor{currentstroke}%
\pgfsetstrokeopacity{0.867112}%
\pgfsetdash{}{0pt}%
\pgfpathmoveto{\pgfqpoint{2.713602in}{1.927663in}}%
\pgfpathcurveto{\pgfqpoint{2.721838in}{1.927663in}}{\pgfqpoint{2.729738in}{1.930935in}}{\pgfqpoint{2.735562in}{1.936759in}}%
\pgfpathcurveto{\pgfqpoint{2.741386in}{1.942583in}}{\pgfqpoint{2.744658in}{1.950483in}}{\pgfqpoint{2.744658in}{1.958719in}}%
\pgfpathcurveto{\pgfqpoint{2.744658in}{1.966956in}}{\pgfqpoint{2.741386in}{1.974856in}}{\pgfqpoint{2.735562in}{1.980680in}}%
\pgfpathcurveto{\pgfqpoint{2.729738in}{1.986504in}}{\pgfqpoint{2.721838in}{1.989776in}}{\pgfqpoint{2.713602in}{1.989776in}}%
\pgfpathcurveto{\pgfqpoint{2.705365in}{1.989776in}}{\pgfqpoint{2.697465in}{1.986504in}}{\pgfqpoint{2.691641in}{1.980680in}}%
\pgfpathcurveto{\pgfqpoint{2.685817in}{1.974856in}}{\pgfqpoint{2.682545in}{1.966956in}}{\pgfqpoint{2.682545in}{1.958719in}}%
\pgfpathcurveto{\pgfqpoint{2.682545in}{1.950483in}}{\pgfqpoint{2.685817in}{1.942583in}}{\pgfqpoint{2.691641in}{1.936759in}}%
\pgfpathcurveto{\pgfqpoint{2.697465in}{1.930935in}}{\pgfqpoint{2.705365in}{1.927663in}}{\pgfqpoint{2.713602in}{1.927663in}}%
\pgfpathclose%
\pgfusepath{stroke,fill}%
\end{pgfscope}%
\begin{pgfscope}%
\pgfpathrectangle{\pgfqpoint{0.100000in}{0.212622in}}{\pgfqpoint{3.696000in}{3.696000in}}%
\pgfusepath{clip}%
\pgfsetbuttcap%
\pgfsetroundjoin%
\definecolor{currentfill}{rgb}{0.121569,0.466667,0.705882}%
\pgfsetfillcolor{currentfill}%
\pgfsetfillopacity{0.868304}%
\pgfsetlinewidth{1.003750pt}%
\definecolor{currentstroke}{rgb}{0.121569,0.466667,0.705882}%
\pgfsetstrokecolor{currentstroke}%
\pgfsetstrokeopacity{0.868304}%
\pgfsetdash{}{0pt}%
\pgfpathmoveto{\pgfqpoint{0.950394in}{2.377909in}}%
\pgfpathcurveto{\pgfqpoint{0.958630in}{2.377909in}}{\pgfqpoint{0.966530in}{2.381182in}}{\pgfqpoint{0.972354in}{2.387006in}}%
\pgfpathcurveto{\pgfqpoint{0.978178in}{2.392829in}}{\pgfqpoint{0.981451in}{2.400730in}}{\pgfqpoint{0.981451in}{2.408966in}}%
\pgfpathcurveto{\pgfqpoint{0.981451in}{2.417202in}}{\pgfqpoint{0.978178in}{2.425102in}}{\pgfqpoint{0.972354in}{2.430926in}}%
\pgfpathcurveto{\pgfqpoint{0.966530in}{2.436750in}}{\pgfqpoint{0.958630in}{2.440022in}}{\pgfqpoint{0.950394in}{2.440022in}}%
\pgfpathcurveto{\pgfqpoint{0.942158in}{2.440022in}}{\pgfqpoint{0.934258in}{2.436750in}}{\pgfqpoint{0.928434in}{2.430926in}}%
\pgfpathcurveto{\pgfqpoint{0.922610in}{2.425102in}}{\pgfqpoint{0.919338in}{2.417202in}}{\pgfqpoint{0.919338in}{2.408966in}}%
\pgfpathcurveto{\pgfqpoint{0.919338in}{2.400730in}}{\pgfqpoint{0.922610in}{2.392829in}}{\pgfqpoint{0.928434in}{2.387006in}}%
\pgfpathcurveto{\pgfqpoint{0.934258in}{2.381182in}}{\pgfqpoint{0.942158in}{2.377909in}}{\pgfqpoint{0.950394in}{2.377909in}}%
\pgfpathclose%
\pgfusepath{stroke,fill}%
\end{pgfscope}%
\begin{pgfscope}%
\pgfpathrectangle{\pgfqpoint{0.100000in}{0.212622in}}{\pgfqpoint{3.696000in}{3.696000in}}%
\pgfusepath{clip}%
\pgfsetbuttcap%
\pgfsetroundjoin%
\definecolor{currentfill}{rgb}{0.121569,0.466667,0.705882}%
\pgfsetfillcolor{currentfill}%
\pgfsetfillopacity{0.868348}%
\pgfsetlinewidth{1.003750pt}%
\definecolor{currentstroke}{rgb}{0.121569,0.466667,0.705882}%
\pgfsetstrokecolor{currentstroke}%
\pgfsetstrokeopacity{0.868348}%
\pgfsetdash{}{0pt}%
\pgfpathmoveto{\pgfqpoint{2.711072in}{1.924943in}}%
\pgfpathcurveto{\pgfqpoint{2.719308in}{1.924943in}}{\pgfqpoint{2.727208in}{1.928215in}}{\pgfqpoint{2.733032in}{1.934039in}}%
\pgfpathcurveto{\pgfqpoint{2.738856in}{1.939863in}}{\pgfqpoint{2.742128in}{1.947763in}}{\pgfqpoint{2.742128in}{1.956000in}}%
\pgfpathcurveto{\pgfqpoint{2.742128in}{1.964236in}}{\pgfqpoint{2.738856in}{1.972136in}}{\pgfqpoint{2.733032in}{1.977960in}}%
\pgfpathcurveto{\pgfqpoint{2.727208in}{1.983784in}}{\pgfqpoint{2.719308in}{1.987056in}}{\pgfqpoint{2.711072in}{1.987056in}}%
\pgfpathcurveto{\pgfqpoint{2.702835in}{1.987056in}}{\pgfqpoint{2.694935in}{1.983784in}}{\pgfqpoint{2.689111in}{1.977960in}}%
\pgfpathcurveto{\pgfqpoint{2.683288in}{1.972136in}}{\pgfqpoint{2.680015in}{1.964236in}}{\pgfqpoint{2.680015in}{1.956000in}}%
\pgfpathcurveto{\pgfqpoint{2.680015in}{1.947763in}}{\pgfqpoint{2.683288in}{1.939863in}}{\pgfqpoint{2.689111in}{1.934039in}}%
\pgfpathcurveto{\pgfqpoint{2.694935in}{1.928215in}}{\pgfqpoint{2.702835in}{1.924943in}}{\pgfqpoint{2.711072in}{1.924943in}}%
\pgfpathclose%
\pgfusepath{stroke,fill}%
\end{pgfscope}%
\begin{pgfscope}%
\pgfpathrectangle{\pgfqpoint{0.100000in}{0.212622in}}{\pgfqpoint{3.696000in}{3.696000in}}%
\pgfusepath{clip}%
\pgfsetbuttcap%
\pgfsetroundjoin%
\definecolor{currentfill}{rgb}{0.121569,0.466667,0.705882}%
\pgfsetfillcolor{currentfill}%
\pgfsetfillopacity{0.869752}%
\pgfsetlinewidth{1.003750pt}%
\definecolor{currentstroke}{rgb}{0.121569,0.466667,0.705882}%
\pgfsetstrokecolor{currentstroke}%
\pgfsetstrokeopacity{0.869752}%
\pgfsetdash{}{0pt}%
\pgfpathmoveto{\pgfqpoint{0.965471in}{2.370548in}}%
\pgfpathcurveto{\pgfqpoint{0.973707in}{2.370548in}}{\pgfqpoint{0.981607in}{2.373821in}}{\pgfqpoint{0.987431in}{2.379645in}}%
\pgfpathcurveto{\pgfqpoint{0.993255in}{2.385468in}}{\pgfqpoint{0.996527in}{2.393369in}}{\pgfqpoint{0.996527in}{2.401605in}}%
\pgfpathcurveto{\pgfqpoint{0.996527in}{2.409841in}}{\pgfqpoint{0.993255in}{2.417741in}}{\pgfqpoint{0.987431in}{2.423565in}}%
\pgfpathcurveto{\pgfqpoint{0.981607in}{2.429389in}}{\pgfqpoint{0.973707in}{2.432661in}}{\pgfqpoint{0.965471in}{2.432661in}}%
\pgfpathcurveto{\pgfqpoint{0.957235in}{2.432661in}}{\pgfqpoint{0.949335in}{2.429389in}}{\pgfqpoint{0.943511in}{2.423565in}}%
\pgfpathcurveto{\pgfqpoint{0.937687in}{2.417741in}}{\pgfqpoint{0.934414in}{2.409841in}}{\pgfqpoint{0.934414in}{2.401605in}}%
\pgfpathcurveto{\pgfqpoint{0.934414in}{2.393369in}}{\pgfqpoint{0.937687in}{2.385468in}}{\pgfqpoint{0.943511in}{2.379645in}}%
\pgfpathcurveto{\pgfqpoint{0.949335in}{2.373821in}}{\pgfqpoint{0.957235in}{2.370548in}}{\pgfqpoint{0.965471in}{2.370548in}}%
\pgfpathclose%
\pgfusepath{stroke,fill}%
\end{pgfscope}%
\begin{pgfscope}%
\pgfpathrectangle{\pgfqpoint{0.100000in}{0.212622in}}{\pgfqpoint{3.696000in}{3.696000in}}%
\pgfusepath{clip}%
\pgfsetbuttcap%
\pgfsetroundjoin%
\definecolor{currentfill}{rgb}{0.121569,0.466667,0.705882}%
\pgfsetfillcolor{currentfill}%
\pgfsetfillopacity{0.869923}%
\pgfsetlinewidth{1.003750pt}%
\definecolor{currentstroke}{rgb}{0.121569,0.466667,0.705882}%
\pgfsetstrokecolor{currentstroke}%
\pgfsetstrokeopacity{0.869923}%
\pgfsetdash{}{0pt}%
\pgfpathmoveto{\pgfqpoint{2.707789in}{1.921695in}}%
\pgfpathcurveto{\pgfqpoint{2.716026in}{1.921695in}}{\pgfqpoint{2.723926in}{1.924968in}}{\pgfqpoint{2.729750in}{1.930792in}}%
\pgfpathcurveto{\pgfqpoint{2.735574in}{1.936615in}}{\pgfqpoint{2.738846in}{1.944516in}}{\pgfqpoint{2.738846in}{1.952752in}}%
\pgfpathcurveto{\pgfqpoint{2.738846in}{1.960988in}}{\pgfqpoint{2.735574in}{1.968888in}}{\pgfqpoint{2.729750in}{1.974712in}}%
\pgfpathcurveto{\pgfqpoint{2.723926in}{1.980536in}}{\pgfqpoint{2.716026in}{1.983808in}}{\pgfqpoint{2.707789in}{1.983808in}}%
\pgfpathcurveto{\pgfqpoint{2.699553in}{1.983808in}}{\pgfqpoint{2.691653in}{1.980536in}}{\pgfqpoint{2.685829in}{1.974712in}}%
\pgfpathcurveto{\pgfqpoint{2.680005in}{1.968888in}}{\pgfqpoint{2.676733in}{1.960988in}}{\pgfqpoint{2.676733in}{1.952752in}}%
\pgfpathcurveto{\pgfqpoint{2.676733in}{1.944516in}}{\pgfqpoint{2.680005in}{1.936615in}}{\pgfqpoint{2.685829in}{1.930792in}}%
\pgfpathcurveto{\pgfqpoint{2.691653in}{1.924968in}}{\pgfqpoint{2.699553in}{1.921695in}}{\pgfqpoint{2.707789in}{1.921695in}}%
\pgfpathclose%
\pgfusepath{stroke,fill}%
\end{pgfscope}%
\begin{pgfscope}%
\pgfpathrectangle{\pgfqpoint{0.100000in}{0.212622in}}{\pgfqpoint{3.696000in}{3.696000in}}%
\pgfusepath{clip}%
\pgfsetbuttcap%
\pgfsetroundjoin%
\definecolor{currentfill}{rgb}{0.121569,0.466667,0.705882}%
\pgfsetfillcolor{currentfill}%
\pgfsetfillopacity{0.871714}%
\pgfsetlinewidth{1.003750pt}%
\definecolor{currentstroke}{rgb}{0.121569,0.466667,0.705882}%
\pgfsetstrokecolor{currentstroke}%
\pgfsetstrokeopacity{0.871714}%
\pgfsetdash{}{0pt}%
\pgfpathmoveto{\pgfqpoint{2.704443in}{1.918420in}}%
\pgfpathcurveto{\pgfqpoint{2.712679in}{1.918420in}}{\pgfqpoint{2.720579in}{1.921692in}}{\pgfqpoint{2.726403in}{1.927516in}}%
\pgfpathcurveto{\pgfqpoint{2.732227in}{1.933340in}}{\pgfqpoint{2.735500in}{1.941240in}}{\pgfqpoint{2.735500in}{1.949476in}}%
\pgfpathcurveto{\pgfqpoint{2.735500in}{1.957713in}}{\pgfqpoint{2.732227in}{1.965613in}}{\pgfqpoint{2.726403in}{1.971437in}}%
\pgfpathcurveto{\pgfqpoint{2.720579in}{1.977261in}}{\pgfqpoint{2.712679in}{1.980533in}}{\pgfqpoint{2.704443in}{1.980533in}}%
\pgfpathcurveto{\pgfqpoint{2.696207in}{1.980533in}}{\pgfqpoint{2.688307in}{1.977261in}}{\pgfqpoint{2.682483in}{1.971437in}}%
\pgfpathcurveto{\pgfqpoint{2.676659in}{1.965613in}}{\pgfqpoint{2.673387in}{1.957713in}}{\pgfqpoint{2.673387in}{1.949476in}}%
\pgfpathcurveto{\pgfqpoint{2.673387in}{1.941240in}}{\pgfqpoint{2.676659in}{1.933340in}}{\pgfqpoint{2.682483in}{1.927516in}}%
\pgfpathcurveto{\pgfqpoint{2.688307in}{1.921692in}}{\pgfqpoint{2.696207in}{1.918420in}}{\pgfqpoint{2.704443in}{1.918420in}}%
\pgfpathclose%
\pgfusepath{stroke,fill}%
\end{pgfscope}%
\begin{pgfscope}%
\pgfpathrectangle{\pgfqpoint{0.100000in}{0.212622in}}{\pgfqpoint{3.696000in}{3.696000in}}%
\pgfusepath{clip}%
\pgfsetbuttcap%
\pgfsetroundjoin%
\definecolor{currentfill}{rgb}{0.121569,0.466667,0.705882}%
\pgfsetfillcolor{currentfill}%
\pgfsetfillopacity{0.872356}%
\pgfsetlinewidth{1.003750pt}%
\definecolor{currentstroke}{rgb}{0.121569,0.466667,0.705882}%
\pgfsetstrokecolor{currentstroke}%
\pgfsetstrokeopacity{0.872356}%
\pgfsetdash{}{0pt}%
\pgfpathmoveto{\pgfqpoint{0.992780in}{2.356482in}}%
\pgfpathcurveto{\pgfqpoint{1.001017in}{2.356482in}}{\pgfqpoint{1.008917in}{2.359755in}}{\pgfqpoint{1.014741in}{2.365579in}}%
\pgfpathcurveto{\pgfqpoint{1.020565in}{2.371403in}}{\pgfqpoint{1.023837in}{2.379303in}}{\pgfqpoint{1.023837in}{2.387539in}}%
\pgfpathcurveto{\pgfqpoint{1.023837in}{2.395775in}}{\pgfqpoint{1.020565in}{2.403675in}}{\pgfqpoint{1.014741in}{2.409499in}}%
\pgfpathcurveto{\pgfqpoint{1.008917in}{2.415323in}}{\pgfqpoint{1.001017in}{2.418595in}}{\pgfqpoint{0.992780in}{2.418595in}}%
\pgfpathcurveto{\pgfqpoint{0.984544in}{2.418595in}}{\pgfqpoint{0.976644in}{2.415323in}}{\pgfqpoint{0.970820in}{2.409499in}}%
\pgfpathcurveto{\pgfqpoint{0.964996in}{2.403675in}}{\pgfqpoint{0.961724in}{2.395775in}}{\pgfqpoint{0.961724in}{2.387539in}}%
\pgfpathcurveto{\pgfqpoint{0.961724in}{2.379303in}}{\pgfqpoint{0.964996in}{2.371403in}}{\pgfqpoint{0.970820in}{2.365579in}}%
\pgfpathcurveto{\pgfqpoint{0.976644in}{2.359755in}}{\pgfqpoint{0.984544in}{2.356482in}}{\pgfqpoint{0.992780in}{2.356482in}}%
\pgfpathclose%
\pgfusepath{stroke,fill}%
\end{pgfscope}%
\begin{pgfscope}%
\pgfpathrectangle{\pgfqpoint{0.100000in}{0.212622in}}{\pgfqpoint{3.696000in}{3.696000in}}%
\pgfusepath{clip}%
\pgfsetbuttcap%
\pgfsetroundjoin%
\definecolor{currentfill}{rgb}{0.121569,0.466667,0.705882}%
\pgfsetfillcolor{currentfill}%
\pgfsetfillopacity{0.874412}%
\pgfsetlinewidth{1.003750pt}%
\definecolor{currentstroke}{rgb}{0.121569,0.466667,0.705882}%
\pgfsetstrokecolor{currentstroke}%
\pgfsetstrokeopacity{0.874412}%
\pgfsetdash{}{0pt}%
\pgfpathmoveto{\pgfqpoint{2.698984in}{1.913229in}}%
\pgfpathcurveto{\pgfqpoint{2.707220in}{1.913229in}}{\pgfqpoint{2.715120in}{1.916502in}}{\pgfqpoint{2.720944in}{1.922326in}}%
\pgfpathcurveto{\pgfqpoint{2.726768in}{1.928150in}}{\pgfqpoint{2.730040in}{1.936050in}}{\pgfqpoint{2.730040in}{1.944286in}}%
\pgfpathcurveto{\pgfqpoint{2.730040in}{1.952522in}}{\pgfqpoint{2.726768in}{1.960422in}}{\pgfqpoint{2.720944in}{1.966246in}}%
\pgfpathcurveto{\pgfqpoint{2.715120in}{1.972070in}}{\pgfqpoint{2.707220in}{1.975342in}}{\pgfqpoint{2.698984in}{1.975342in}}%
\pgfpathcurveto{\pgfqpoint{2.690748in}{1.975342in}}{\pgfqpoint{2.682848in}{1.972070in}}{\pgfqpoint{2.677024in}{1.966246in}}%
\pgfpathcurveto{\pgfqpoint{2.671200in}{1.960422in}}{\pgfqpoint{2.667927in}{1.952522in}}{\pgfqpoint{2.667927in}{1.944286in}}%
\pgfpathcurveto{\pgfqpoint{2.667927in}{1.936050in}}{\pgfqpoint{2.671200in}{1.928150in}}{\pgfqpoint{2.677024in}{1.922326in}}%
\pgfpathcurveto{\pgfqpoint{2.682848in}{1.916502in}}{\pgfqpoint{2.690748in}{1.913229in}}{\pgfqpoint{2.698984in}{1.913229in}}%
\pgfpathclose%
\pgfusepath{stroke,fill}%
\end{pgfscope}%
\begin{pgfscope}%
\pgfpathrectangle{\pgfqpoint{0.100000in}{0.212622in}}{\pgfqpoint{3.696000in}{3.696000in}}%
\pgfusepath{clip}%
\pgfsetbuttcap%
\pgfsetroundjoin%
\definecolor{currentfill}{rgb}{0.121569,0.466667,0.705882}%
\pgfsetfillcolor{currentfill}%
\pgfsetfillopacity{0.874747}%
\pgfsetlinewidth{1.003750pt}%
\definecolor{currentstroke}{rgb}{0.121569,0.466667,0.705882}%
\pgfsetstrokecolor{currentstroke}%
\pgfsetstrokeopacity{0.874747}%
\pgfsetdash{}{0pt}%
\pgfpathmoveto{\pgfqpoint{1.019041in}{2.343078in}}%
\pgfpathcurveto{\pgfqpoint{1.027277in}{2.343078in}}{\pgfqpoint{1.035177in}{2.346351in}}{\pgfqpoint{1.041001in}{2.352175in}}%
\pgfpathcurveto{\pgfqpoint{1.046825in}{2.357999in}}{\pgfqpoint{1.050098in}{2.365899in}}{\pgfqpoint{1.050098in}{2.374135in}}%
\pgfpathcurveto{\pgfqpoint{1.050098in}{2.382371in}}{\pgfqpoint{1.046825in}{2.390271in}}{\pgfqpoint{1.041001in}{2.396095in}}%
\pgfpathcurveto{\pgfqpoint{1.035177in}{2.401919in}}{\pgfqpoint{1.027277in}{2.405191in}}{\pgfqpoint{1.019041in}{2.405191in}}%
\pgfpathcurveto{\pgfqpoint{1.010805in}{2.405191in}}{\pgfqpoint{1.002905in}{2.401919in}}{\pgfqpoint{0.997081in}{2.396095in}}%
\pgfpathcurveto{\pgfqpoint{0.991257in}{2.390271in}}{\pgfqpoint{0.987985in}{2.382371in}}{\pgfqpoint{0.987985in}{2.374135in}}%
\pgfpathcurveto{\pgfqpoint{0.987985in}{2.365899in}}{\pgfqpoint{0.991257in}{2.357999in}}{\pgfqpoint{0.997081in}{2.352175in}}%
\pgfpathcurveto{\pgfqpoint{1.002905in}{2.346351in}}{\pgfqpoint{1.010805in}{2.343078in}}{\pgfqpoint{1.019041in}{2.343078in}}%
\pgfpathclose%
\pgfusepath{stroke,fill}%
\end{pgfscope}%
\begin{pgfscope}%
\pgfpathrectangle{\pgfqpoint{0.100000in}{0.212622in}}{\pgfqpoint{3.696000in}{3.696000in}}%
\pgfusepath{clip}%
\pgfsetbuttcap%
\pgfsetroundjoin%
\definecolor{currentfill}{rgb}{0.121569,0.466667,0.705882}%
\pgfsetfillcolor{currentfill}%
\pgfsetfillopacity{0.876569}%
\pgfsetlinewidth{1.003750pt}%
\definecolor{currentstroke}{rgb}{0.121569,0.466667,0.705882}%
\pgfsetstrokecolor{currentstroke}%
\pgfsetstrokeopacity{0.876569}%
\pgfsetdash{}{0pt}%
\pgfpathmoveto{\pgfqpoint{1.042000in}{2.330565in}}%
\pgfpathcurveto{\pgfqpoint{1.050236in}{2.330565in}}{\pgfqpoint{1.058136in}{2.333837in}}{\pgfqpoint{1.063960in}{2.339661in}}%
\pgfpathcurveto{\pgfqpoint{1.069784in}{2.345485in}}{\pgfqpoint{1.073057in}{2.353385in}}{\pgfqpoint{1.073057in}{2.361621in}}%
\pgfpathcurveto{\pgfqpoint{1.073057in}{2.369857in}}{\pgfqpoint{1.069784in}{2.377757in}}{\pgfqpoint{1.063960in}{2.383581in}}%
\pgfpathcurveto{\pgfqpoint{1.058136in}{2.389405in}}{\pgfqpoint{1.050236in}{2.392678in}}{\pgfqpoint{1.042000in}{2.392678in}}%
\pgfpathcurveto{\pgfqpoint{1.033764in}{2.392678in}}{\pgfqpoint{1.025864in}{2.389405in}}{\pgfqpoint{1.020040in}{2.383581in}}%
\pgfpathcurveto{\pgfqpoint{1.014216in}{2.377757in}}{\pgfqpoint{1.010944in}{2.369857in}}{\pgfqpoint{1.010944in}{2.361621in}}%
\pgfpathcurveto{\pgfqpoint{1.010944in}{2.353385in}}{\pgfqpoint{1.014216in}{2.345485in}}{\pgfqpoint{1.020040in}{2.339661in}}%
\pgfpathcurveto{\pgfqpoint{1.025864in}{2.333837in}}{\pgfqpoint{1.033764in}{2.330565in}}{\pgfqpoint{1.042000in}{2.330565in}}%
\pgfpathclose%
\pgfusepath{stroke,fill}%
\end{pgfscope}%
\begin{pgfscope}%
\pgfpathrectangle{\pgfqpoint{0.100000in}{0.212622in}}{\pgfqpoint{3.696000in}{3.696000in}}%
\pgfusepath{clip}%
\pgfsetbuttcap%
\pgfsetroundjoin%
\definecolor{currentfill}{rgb}{0.121569,0.466667,0.705882}%
\pgfsetfillcolor{currentfill}%
\pgfsetfillopacity{0.877503}%
\pgfsetlinewidth{1.003750pt}%
\definecolor{currentstroke}{rgb}{0.121569,0.466667,0.705882}%
\pgfsetstrokecolor{currentstroke}%
\pgfsetstrokeopacity{0.877503}%
\pgfsetdash{}{0pt}%
\pgfpathmoveto{\pgfqpoint{2.692778in}{1.907133in}}%
\pgfpathcurveto{\pgfqpoint{2.701015in}{1.907133in}}{\pgfqpoint{2.708915in}{1.910405in}}{\pgfqpoint{2.714739in}{1.916229in}}%
\pgfpathcurveto{\pgfqpoint{2.720563in}{1.922053in}}{\pgfqpoint{2.723835in}{1.929953in}}{\pgfqpoint{2.723835in}{1.938190in}}%
\pgfpathcurveto{\pgfqpoint{2.723835in}{1.946426in}}{\pgfqpoint{2.720563in}{1.954326in}}{\pgfqpoint{2.714739in}{1.960150in}}%
\pgfpathcurveto{\pgfqpoint{2.708915in}{1.965974in}}{\pgfqpoint{2.701015in}{1.969246in}}{\pgfqpoint{2.692778in}{1.969246in}}%
\pgfpathcurveto{\pgfqpoint{2.684542in}{1.969246in}}{\pgfqpoint{2.676642in}{1.965974in}}{\pgfqpoint{2.670818in}{1.960150in}}%
\pgfpathcurveto{\pgfqpoint{2.664994in}{1.954326in}}{\pgfqpoint{2.661722in}{1.946426in}}{\pgfqpoint{2.661722in}{1.938190in}}%
\pgfpathcurveto{\pgfqpoint{2.661722in}{1.929953in}}{\pgfqpoint{2.664994in}{1.922053in}}{\pgfqpoint{2.670818in}{1.916229in}}%
\pgfpathcurveto{\pgfqpoint{2.676642in}{1.910405in}}{\pgfqpoint{2.684542in}{1.907133in}}{\pgfqpoint{2.692778in}{1.907133in}}%
\pgfpathclose%
\pgfusepath{stroke,fill}%
\end{pgfscope}%
\begin{pgfscope}%
\pgfpathrectangle{\pgfqpoint{0.100000in}{0.212622in}}{\pgfqpoint{3.696000in}{3.696000in}}%
\pgfusepath{clip}%
\pgfsetbuttcap%
\pgfsetroundjoin%
\definecolor{currentfill}{rgb}{0.121569,0.466667,0.705882}%
\pgfsetfillcolor{currentfill}%
\pgfsetfillopacity{0.878222}%
\pgfsetlinewidth{1.003750pt}%
\definecolor{currentstroke}{rgb}{0.121569,0.466667,0.705882}%
\pgfsetstrokecolor{currentstroke}%
\pgfsetstrokeopacity{0.878222}%
\pgfsetdash{}{0pt}%
\pgfpathmoveto{\pgfqpoint{1.063166in}{2.317874in}}%
\pgfpathcurveto{\pgfqpoint{1.071403in}{2.317874in}}{\pgfqpoint{1.079303in}{2.321146in}}{\pgfqpoint{1.085127in}{2.326970in}}%
\pgfpathcurveto{\pgfqpoint{1.090951in}{2.332794in}}{\pgfqpoint{1.094223in}{2.340694in}}{\pgfqpoint{1.094223in}{2.348931in}}%
\pgfpathcurveto{\pgfqpoint{1.094223in}{2.357167in}}{\pgfqpoint{1.090951in}{2.365067in}}{\pgfqpoint{1.085127in}{2.370891in}}%
\pgfpathcurveto{\pgfqpoint{1.079303in}{2.376715in}}{\pgfqpoint{1.071403in}{2.379987in}}{\pgfqpoint{1.063166in}{2.379987in}}%
\pgfpathcurveto{\pgfqpoint{1.054930in}{2.379987in}}{\pgfqpoint{1.047030in}{2.376715in}}{\pgfqpoint{1.041206in}{2.370891in}}%
\pgfpathcurveto{\pgfqpoint{1.035382in}{2.365067in}}{\pgfqpoint{1.032110in}{2.357167in}}{\pgfqpoint{1.032110in}{2.348931in}}%
\pgfpathcurveto{\pgfqpoint{1.032110in}{2.340694in}}{\pgfqpoint{1.035382in}{2.332794in}}{\pgfqpoint{1.041206in}{2.326970in}}%
\pgfpathcurveto{\pgfqpoint{1.047030in}{2.321146in}}{\pgfqpoint{1.054930in}{2.317874in}}{\pgfqpoint{1.063166in}{2.317874in}}%
\pgfpathclose%
\pgfusepath{stroke,fill}%
\end{pgfscope}%
\begin{pgfscope}%
\pgfpathrectangle{\pgfqpoint{0.100000in}{0.212622in}}{\pgfqpoint{3.696000in}{3.696000in}}%
\pgfusepath{clip}%
\pgfsetbuttcap%
\pgfsetroundjoin%
\definecolor{currentfill}{rgb}{0.121569,0.466667,0.705882}%
\pgfsetfillcolor{currentfill}%
\pgfsetfillopacity{0.879666}%
\pgfsetlinewidth{1.003750pt}%
\definecolor{currentstroke}{rgb}{0.121569,0.466667,0.705882}%
\pgfsetstrokecolor{currentstroke}%
\pgfsetstrokeopacity{0.879666}%
\pgfsetdash{}{0pt}%
\pgfpathmoveto{\pgfqpoint{1.083305in}{2.305380in}}%
\pgfpathcurveto{\pgfqpoint{1.091541in}{2.305380in}}{\pgfqpoint{1.099441in}{2.308652in}}{\pgfqpoint{1.105265in}{2.314476in}}%
\pgfpathcurveto{\pgfqpoint{1.111089in}{2.320300in}}{\pgfqpoint{1.114361in}{2.328200in}}{\pgfqpoint{1.114361in}{2.336437in}}%
\pgfpathcurveto{\pgfqpoint{1.114361in}{2.344673in}}{\pgfqpoint{1.111089in}{2.352573in}}{\pgfqpoint{1.105265in}{2.358397in}}%
\pgfpathcurveto{\pgfqpoint{1.099441in}{2.364221in}}{\pgfqpoint{1.091541in}{2.367493in}}{\pgfqpoint{1.083305in}{2.367493in}}%
\pgfpathcurveto{\pgfqpoint{1.075068in}{2.367493in}}{\pgfqpoint{1.067168in}{2.364221in}}{\pgfqpoint{1.061344in}{2.358397in}}%
\pgfpathcurveto{\pgfqpoint{1.055520in}{2.352573in}}{\pgfqpoint{1.052248in}{2.344673in}}{\pgfqpoint{1.052248in}{2.336437in}}%
\pgfpathcurveto{\pgfqpoint{1.052248in}{2.328200in}}{\pgfqpoint{1.055520in}{2.320300in}}{\pgfqpoint{1.061344in}{2.314476in}}%
\pgfpathcurveto{\pgfqpoint{1.067168in}{2.308652in}}{\pgfqpoint{1.075068in}{2.305380in}}{\pgfqpoint{1.083305in}{2.305380in}}%
\pgfpathclose%
\pgfusepath{stroke,fill}%
\end{pgfscope}%
\begin{pgfscope}%
\pgfpathrectangle{\pgfqpoint{0.100000in}{0.212622in}}{\pgfqpoint{3.696000in}{3.696000in}}%
\pgfusepath{clip}%
\pgfsetbuttcap%
\pgfsetroundjoin%
\definecolor{currentfill}{rgb}{0.121569,0.466667,0.705882}%
\pgfsetfillcolor{currentfill}%
\pgfsetfillopacity{0.880840}%
\pgfsetlinewidth{1.003750pt}%
\definecolor{currentstroke}{rgb}{0.121569,0.466667,0.705882}%
\pgfsetstrokecolor{currentstroke}%
\pgfsetstrokeopacity{0.880840}%
\pgfsetdash{}{0pt}%
\pgfpathmoveto{\pgfqpoint{1.102768in}{2.291627in}}%
\pgfpathcurveto{\pgfqpoint{1.111004in}{2.291627in}}{\pgfqpoint{1.118904in}{2.294899in}}{\pgfqpoint{1.124728in}{2.300723in}}%
\pgfpathcurveto{\pgfqpoint{1.130552in}{2.306547in}}{\pgfqpoint{1.133824in}{2.314447in}}{\pgfqpoint{1.133824in}{2.322683in}}%
\pgfpathcurveto{\pgfqpoint{1.133824in}{2.330919in}}{\pgfqpoint{1.130552in}{2.338819in}}{\pgfqpoint{1.124728in}{2.344643in}}%
\pgfpathcurveto{\pgfqpoint{1.118904in}{2.350467in}}{\pgfqpoint{1.111004in}{2.353740in}}{\pgfqpoint{1.102768in}{2.353740in}}%
\pgfpathcurveto{\pgfqpoint{1.094532in}{2.353740in}}{\pgfqpoint{1.086632in}{2.350467in}}{\pgfqpoint{1.080808in}{2.344643in}}%
\pgfpathcurveto{\pgfqpoint{1.074984in}{2.338819in}}{\pgfqpoint{1.071711in}{2.330919in}}{\pgfqpoint{1.071711in}{2.322683in}}%
\pgfpathcurveto{\pgfqpoint{1.071711in}{2.314447in}}{\pgfqpoint{1.074984in}{2.306547in}}{\pgfqpoint{1.080808in}{2.300723in}}%
\pgfpathcurveto{\pgfqpoint{1.086632in}{2.294899in}}{\pgfqpoint{1.094532in}{2.291627in}}{\pgfqpoint{1.102768in}{2.291627in}}%
\pgfpathclose%
\pgfusepath{stroke,fill}%
\end{pgfscope}%
\begin{pgfscope}%
\pgfpathrectangle{\pgfqpoint{0.100000in}{0.212622in}}{\pgfqpoint{3.696000in}{3.696000in}}%
\pgfusepath{clip}%
\pgfsetbuttcap%
\pgfsetroundjoin%
\definecolor{currentfill}{rgb}{0.121569,0.466667,0.705882}%
\pgfsetfillcolor{currentfill}%
\pgfsetfillopacity{0.881116}%
\pgfsetlinewidth{1.003750pt}%
\definecolor{currentstroke}{rgb}{0.121569,0.466667,0.705882}%
\pgfsetstrokecolor{currentstroke}%
\pgfsetstrokeopacity{0.881116}%
\pgfsetdash{}{0pt}%
\pgfpathmoveto{\pgfqpoint{2.686233in}{1.901369in}}%
\pgfpathcurveto{\pgfqpoint{2.694469in}{1.901369in}}{\pgfqpoint{2.702369in}{1.904641in}}{\pgfqpoint{2.708193in}{1.910465in}}%
\pgfpathcurveto{\pgfqpoint{2.714017in}{1.916289in}}{\pgfqpoint{2.717289in}{1.924189in}}{\pgfqpoint{2.717289in}{1.932425in}}%
\pgfpathcurveto{\pgfqpoint{2.717289in}{1.940661in}}{\pgfqpoint{2.714017in}{1.948561in}}{\pgfqpoint{2.708193in}{1.954385in}}%
\pgfpathcurveto{\pgfqpoint{2.702369in}{1.960209in}}{\pgfqpoint{2.694469in}{1.963482in}}{\pgfqpoint{2.686233in}{1.963482in}}%
\pgfpathcurveto{\pgfqpoint{2.677997in}{1.963482in}}{\pgfqpoint{2.670096in}{1.960209in}}{\pgfqpoint{2.664273in}{1.954385in}}%
\pgfpathcurveto{\pgfqpoint{2.658449in}{1.948561in}}{\pgfqpoint{2.655176in}{1.940661in}}{\pgfqpoint{2.655176in}{1.932425in}}%
\pgfpathcurveto{\pgfqpoint{2.655176in}{1.924189in}}{\pgfqpoint{2.658449in}{1.916289in}}{\pgfqpoint{2.664273in}{1.910465in}}%
\pgfpathcurveto{\pgfqpoint{2.670096in}{1.904641in}}{\pgfqpoint{2.677997in}{1.901369in}}{\pgfqpoint{2.686233in}{1.901369in}}%
\pgfpathclose%
\pgfusepath{stroke,fill}%
\end{pgfscope}%
\begin{pgfscope}%
\pgfpathrectangle{\pgfqpoint{0.100000in}{0.212622in}}{\pgfqpoint{3.696000in}{3.696000in}}%
\pgfusepath{clip}%
\pgfsetbuttcap%
\pgfsetroundjoin%
\definecolor{currentfill}{rgb}{0.121569,0.466667,0.705882}%
\pgfsetfillcolor{currentfill}%
\pgfsetfillopacity{0.882333}%
\pgfsetlinewidth{1.003750pt}%
\definecolor{currentstroke}{rgb}{0.121569,0.466667,0.705882}%
\pgfsetstrokecolor{currentstroke}%
\pgfsetstrokeopacity{0.882333}%
\pgfsetdash{}{0pt}%
\pgfpathmoveto{\pgfqpoint{1.120048in}{2.282988in}}%
\pgfpathcurveto{\pgfqpoint{1.128284in}{2.282988in}}{\pgfqpoint{1.136184in}{2.286260in}}{\pgfqpoint{1.142008in}{2.292084in}}%
\pgfpathcurveto{\pgfqpoint{1.147832in}{2.297908in}}{\pgfqpoint{1.151104in}{2.305808in}}{\pgfqpoint{1.151104in}{2.314044in}}%
\pgfpathcurveto{\pgfqpoint{1.151104in}{2.322281in}}{\pgfqpoint{1.147832in}{2.330181in}}{\pgfqpoint{1.142008in}{2.336005in}}%
\pgfpathcurveto{\pgfqpoint{1.136184in}{2.341829in}}{\pgfqpoint{1.128284in}{2.345101in}}{\pgfqpoint{1.120048in}{2.345101in}}%
\pgfpathcurveto{\pgfqpoint{1.111812in}{2.345101in}}{\pgfqpoint{1.103912in}{2.341829in}}{\pgfqpoint{1.098088in}{2.336005in}}%
\pgfpathcurveto{\pgfqpoint{1.092264in}{2.330181in}}{\pgfqpoint{1.088991in}{2.322281in}}{\pgfqpoint{1.088991in}{2.314044in}}%
\pgfpathcurveto{\pgfqpoint{1.088991in}{2.305808in}}{\pgfqpoint{1.092264in}{2.297908in}}{\pgfqpoint{1.098088in}{2.292084in}}%
\pgfpathcurveto{\pgfqpoint{1.103912in}{2.286260in}}{\pgfqpoint{1.111812in}{2.282988in}}{\pgfqpoint{1.120048in}{2.282988in}}%
\pgfpathclose%
\pgfusepath{stroke,fill}%
\end{pgfscope}%
\begin{pgfscope}%
\pgfpathrectangle{\pgfqpoint{0.100000in}{0.212622in}}{\pgfqpoint{3.696000in}{3.696000in}}%
\pgfusepath{clip}%
\pgfsetbuttcap%
\pgfsetroundjoin%
\definecolor{currentfill}{rgb}{0.121569,0.466667,0.705882}%
\pgfsetfillcolor{currentfill}%
\pgfsetfillopacity{0.882990}%
\pgfsetlinewidth{1.003750pt}%
\definecolor{currentstroke}{rgb}{0.121569,0.466667,0.705882}%
\pgfsetstrokecolor{currentstroke}%
\pgfsetstrokeopacity{0.882990}%
\pgfsetdash{}{0pt}%
\pgfpathmoveto{\pgfqpoint{2.682364in}{1.897750in}}%
\pgfpathcurveto{\pgfqpoint{2.690600in}{1.897750in}}{\pgfqpoint{2.698501in}{1.901022in}}{\pgfqpoint{2.704324in}{1.906846in}}%
\pgfpathcurveto{\pgfqpoint{2.710148in}{1.912670in}}{\pgfqpoint{2.713421in}{1.920570in}}{\pgfqpoint{2.713421in}{1.928806in}}%
\pgfpathcurveto{\pgfqpoint{2.713421in}{1.937043in}}{\pgfqpoint{2.710148in}{1.944943in}}{\pgfqpoint{2.704324in}{1.950767in}}%
\pgfpathcurveto{\pgfqpoint{2.698501in}{1.956591in}}{\pgfqpoint{2.690600in}{1.959863in}}{\pgfqpoint{2.682364in}{1.959863in}}%
\pgfpathcurveto{\pgfqpoint{2.674128in}{1.959863in}}{\pgfqpoint{2.666228in}{1.956591in}}{\pgfqpoint{2.660404in}{1.950767in}}%
\pgfpathcurveto{\pgfqpoint{2.654580in}{1.944943in}}{\pgfqpoint{2.651308in}{1.937043in}}{\pgfqpoint{2.651308in}{1.928806in}}%
\pgfpathcurveto{\pgfqpoint{2.651308in}{1.920570in}}{\pgfqpoint{2.654580in}{1.912670in}}{\pgfqpoint{2.660404in}{1.906846in}}%
\pgfpathcurveto{\pgfqpoint{2.666228in}{1.901022in}}{\pgfqpoint{2.674128in}{1.897750in}}{\pgfqpoint{2.682364in}{1.897750in}}%
\pgfpathclose%
\pgfusepath{stroke,fill}%
\end{pgfscope}%
\begin{pgfscope}%
\pgfpathrectangle{\pgfqpoint{0.100000in}{0.212622in}}{\pgfqpoint{3.696000in}{3.696000in}}%
\pgfusepath{clip}%
\pgfsetbuttcap%
\pgfsetroundjoin%
\definecolor{currentfill}{rgb}{0.121569,0.466667,0.705882}%
\pgfsetfillcolor{currentfill}%
\pgfsetfillopacity{0.883793}%
\pgfsetlinewidth{1.003750pt}%
\definecolor{currentstroke}{rgb}{0.121569,0.466667,0.705882}%
\pgfsetstrokecolor{currentstroke}%
\pgfsetstrokeopacity{0.883793}%
\pgfsetdash{}{0pt}%
\pgfpathmoveto{\pgfqpoint{1.134192in}{2.277288in}}%
\pgfpathcurveto{\pgfqpoint{1.142428in}{2.277288in}}{\pgfqpoint{1.150328in}{2.280561in}}{\pgfqpoint{1.156152in}{2.286384in}}%
\pgfpathcurveto{\pgfqpoint{1.161976in}{2.292208in}}{\pgfqpoint{1.165248in}{2.300108in}}{\pgfqpoint{1.165248in}{2.308345in}}%
\pgfpathcurveto{\pgfqpoint{1.165248in}{2.316581in}}{\pgfqpoint{1.161976in}{2.324481in}}{\pgfqpoint{1.156152in}{2.330305in}}%
\pgfpathcurveto{\pgfqpoint{1.150328in}{2.336129in}}{\pgfqpoint{1.142428in}{2.339401in}}{\pgfqpoint{1.134192in}{2.339401in}}%
\pgfpathcurveto{\pgfqpoint{1.125956in}{2.339401in}}{\pgfqpoint{1.118055in}{2.336129in}}{\pgfqpoint{1.112232in}{2.330305in}}%
\pgfpathcurveto{\pgfqpoint{1.106408in}{2.324481in}}{\pgfqpoint{1.103135in}{2.316581in}}{\pgfqpoint{1.103135in}{2.308345in}}%
\pgfpathcurveto{\pgfqpoint{1.103135in}{2.300108in}}{\pgfqpoint{1.106408in}{2.292208in}}{\pgfqpoint{1.112232in}{2.286384in}}%
\pgfpathcurveto{\pgfqpoint{1.118055in}{2.280561in}}{\pgfqpoint{1.125956in}{2.277288in}}{\pgfqpoint{1.134192in}{2.277288in}}%
\pgfpathclose%
\pgfusepath{stroke,fill}%
\end{pgfscope}%
\begin{pgfscope}%
\pgfpathrectangle{\pgfqpoint{0.100000in}{0.212622in}}{\pgfqpoint{3.696000in}{3.696000in}}%
\pgfusepath{clip}%
\pgfsetbuttcap%
\pgfsetroundjoin%
\definecolor{currentfill}{rgb}{0.121569,0.466667,0.705882}%
\pgfsetfillcolor{currentfill}%
\pgfsetfillopacity{0.885172}%
\pgfsetlinewidth{1.003750pt}%
\definecolor{currentstroke}{rgb}{0.121569,0.466667,0.705882}%
\pgfsetstrokecolor{currentstroke}%
\pgfsetstrokeopacity{0.885172}%
\pgfsetdash{}{0pt}%
\pgfpathmoveto{\pgfqpoint{1.147904in}{2.271772in}}%
\pgfpathcurveto{\pgfqpoint{1.156140in}{2.271772in}}{\pgfqpoint{1.164040in}{2.275044in}}{\pgfqpoint{1.169864in}{2.280868in}}%
\pgfpathcurveto{\pgfqpoint{1.175688in}{2.286692in}}{\pgfqpoint{1.178961in}{2.294592in}}{\pgfqpoint{1.178961in}{2.302828in}}%
\pgfpathcurveto{\pgfqpoint{1.178961in}{2.311065in}}{\pgfqpoint{1.175688in}{2.318965in}}{\pgfqpoint{1.169864in}{2.324789in}}%
\pgfpathcurveto{\pgfqpoint{1.164040in}{2.330613in}}{\pgfqpoint{1.156140in}{2.333885in}}{\pgfqpoint{1.147904in}{2.333885in}}%
\pgfpathcurveto{\pgfqpoint{1.139668in}{2.333885in}}{\pgfqpoint{1.131768in}{2.330613in}}{\pgfqpoint{1.125944in}{2.324789in}}%
\pgfpathcurveto{\pgfqpoint{1.120120in}{2.318965in}}{\pgfqpoint{1.116848in}{2.311065in}}{\pgfqpoint{1.116848in}{2.302828in}}%
\pgfpathcurveto{\pgfqpoint{1.116848in}{2.294592in}}{\pgfqpoint{1.120120in}{2.286692in}}{\pgfqpoint{1.125944in}{2.280868in}}%
\pgfpathcurveto{\pgfqpoint{1.131768in}{2.275044in}}{\pgfqpoint{1.139668in}{2.271772in}}{\pgfqpoint{1.147904in}{2.271772in}}%
\pgfpathclose%
\pgfusepath{stroke,fill}%
\end{pgfscope}%
\begin{pgfscope}%
\pgfpathrectangle{\pgfqpoint{0.100000in}{0.212622in}}{\pgfqpoint{3.696000in}{3.696000in}}%
\pgfusepath{clip}%
\pgfsetbuttcap%
\pgfsetroundjoin%
\definecolor{currentfill}{rgb}{0.121569,0.466667,0.705882}%
\pgfsetfillcolor{currentfill}%
\pgfsetfillopacity{0.885254}%
\pgfsetlinewidth{1.003750pt}%
\definecolor{currentstroke}{rgb}{0.121569,0.466667,0.705882}%
\pgfsetstrokecolor{currentstroke}%
\pgfsetstrokeopacity{0.885254}%
\pgfsetdash{}{0pt}%
\pgfpathmoveto{\pgfqpoint{2.678326in}{1.894312in}}%
\pgfpathcurveto{\pgfqpoint{2.686562in}{1.894312in}}{\pgfqpoint{2.694462in}{1.897585in}}{\pgfqpoint{2.700286in}{1.903409in}}%
\pgfpathcurveto{\pgfqpoint{2.706110in}{1.909233in}}{\pgfqpoint{2.709383in}{1.917133in}}{\pgfqpoint{2.709383in}{1.925369in}}%
\pgfpathcurveto{\pgfqpoint{2.709383in}{1.933605in}}{\pgfqpoint{2.706110in}{1.941505in}}{\pgfqpoint{2.700286in}{1.947329in}}%
\pgfpathcurveto{\pgfqpoint{2.694462in}{1.953153in}}{\pgfqpoint{2.686562in}{1.956425in}}{\pgfqpoint{2.678326in}{1.956425in}}%
\pgfpathcurveto{\pgfqpoint{2.670090in}{1.956425in}}{\pgfqpoint{2.662190in}{1.953153in}}{\pgfqpoint{2.656366in}{1.947329in}}%
\pgfpathcurveto{\pgfqpoint{2.650542in}{1.941505in}}{\pgfqpoint{2.647270in}{1.933605in}}{\pgfqpoint{2.647270in}{1.925369in}}%
\pgfpathcurveto{\pgfqpoint{2.647270in}{1.917133in}}{\pgfqpoint{2.650542in}{1.909233in}}{\pgfqpoint{2.656366in}{1.903409in}}%
\pgfpathcurveto{\pgfqpoint{2.662190in}{1.897585in}}{\pgfqpoint{2.670090in}{1.894312in}}{\pgfqpoint{2.678326in}{1.894312in}}%
\pgfpathclose%
\pgfusepath{stroke,fill}%
\end{pgfscope}%
\begin{pgfscope}%
\pgfpathrectangle{\pgfqpoint{0.100000in}{0.212622in}}{\pgfqpoint{3.696000in}{3.696000in}}%
\pgfusepath{clip}%
\pgfsetbuttcap%
\pgfsetroundjoin%
\definecolor{currentfill}{rgb}{0.121569,0.466667,0.705882}%
\pgfsetfillcolor{currentfill}%
\pgfsetfillopacity{0.887700}%
\pgfsetlinewidth{1.003750pt}%
\definecolor{currentstroke}{rgb}{0.121569,0.466667,0.705882}%
\pgfsetstrokecolor{currentstroke}%
\pgfsetstrokeopacity{0.887700}%
\pgfsetdash{}{0pt}%
\pgfpathmoveto{\pgfqpoint{1.172913in}{2.262095in}}%
\pgfpathcurveto{\pgfqpoint{1.181149in}{2.262095in}}{\pgfqpoint{1.189049in}{2.265368in}}{\pgfqpoint{1.194873in}{2.271191in}}%
\pgfpathcurveto{\pgfqpoint{1.200697in}{2.277015in}}{\pgfqpoint{1.203969in}{2.284915in}}{\pgfqpoint{1.203969in}{2.293152in}}%
\pgfpathcurveto{\pgfqpoint{1.203969in}{2.301388in}}{\pgfqpoint{1.200697in}{2.309288in}}{\pgfqpoint{1.194873in}{2.315112in}}%
\pgfpathcurveto{\pgfqpoint{1.189049in}{2.320936in}}{\pgfqpoint{1.181149in}{2.324208in}}{\pgfqpoint{1.172913in}{2.324208in}}%
\pgfpathcurveto{\pgfqpoint{1.164676in}{2.324208in}}{\pgfqpoint{1.156776in}{2.320936in}}{\pgfqpoint{1.150952in}{2.315112in}}%
\pgfpathcurveto{\pgfqpoint{1.145128in}{2.309288in}}{\pgfqpoint{1.141856in}{2.301388in}}{\pgfqpoint{1.141856in}{2.293152in}}%
\pgfpathcurveto{\pgfqpoint{1.141856in}{2.284915in}}{\pgfqpoint{1.145128in}{2.277015in}}{\pgfqpoint{1.150952in}{2.271191in}}%
\pgfpathcurveto{\pgfqpoint{1.156776in}{2.265368in}}{\pgfqpoint{1.164676in}{2.262095in}}{\pgfqpoint{1.172913in}{2.262095in}}%
\pgfpathclose%
\pgfusepath{stroke,fill}%
\end{pgfscope}%
\begin{pgfscope}%
\pgfpathrectangle{\pgfqpoint{0.100000in}{0.212622in}}{\pgfqpoint{3.696000in}{3.696000in}}%
\pgfusepath{clip}%
\pgfsetbuttcap%
\pgfsetroundjoin%
\definecolor{currentfill}{rgb}{0.121569,0.466667,0.705882}%
\pgfsetfillcolor{currentfill}%
\pgfsetfillopacity{0.888026}%
\pgfsetlinewidth{1.003750pt}%
\definecolor{currentstroke}{rgb}{0.121569,0.466667,0.705882}%
\pgfsetstrokecolor{currentstroke}%
\pgfsetstrokeopacity{0.888026}%
\pgfsetdash{}{0pt}%
\pgfpathmoveto{\pgfqpoint{2.672614in}{1.888968in}}%
\pgfpathcurveto{\pgfqpoint{2.680850in}{1.888968in}}{\pgfqpoint{2.688750in}{1.892240in}}{\pgfqpoint{2.694574in}{1.898064in}}%
\pgfpathcurveto{\pgfqpoint{2.700398in}{1.903888in}}{\pgfqpoint{2.703670in}{1.911788in}}{\pgfqpoint{2.703670in}{1.920024in}}%
\pgfpathcurveto{\pgfqpoint{2.703670in}{1.928260in}}{\pgfqpoint{2.700398in}{1.936160in}}{\pgfqpoint{2.694574in}{1.941984in}}%
\pgfpathcurveto{\pgfqpoint{2.688750in}{1.947808in}}{\pgfqpoint{2.680850in}{1.951081in}}{\pgfqpoint{2.672614in}{1.951081in}}%
\pgfpathcurveto{\pgfqpoint{2.664377in}{1.951081in}}{\pgfqpoint{2.656477in}{1.947808in}}{\pgfqpoint{2.650653in}{1.941984in}}%
\pgfpathcurveto{\pgfqpoint{2.644829in}{1.936160in}}{\pgfqpoint{2.641557in}{1.928260in}}{\pgfqpoint{2.641557in}{1.920024in}}%
\pgfpathcurveto{\pgfqpoint{2.641557in}{1.911788in}}{\pgfqpoint{2.644829in}{1.903888in}}{\pgfqpoint{2.650653in}{1.898064in}}%
\pgfpathcurveto{\pgfqpoint{2.656477in}{1.892240in}}{\pgfqpoint{2.664377in}{1.888968in}}{\pgfqpoint{2.672614in}{1.888968in}}%
\pgfpathclose%
\pgfusepath{stroke,fill}%
\end{pgfscope}%
\begin{pgfscope}%
\pgfpathrectangle{\pgfqpoint{0.100000in}{0.212622in}}{\pgfqpoint{3.696000in}{3.696000in}}%
\pgfusepath{clip}%
\pgfsetbuttcap%
\pgfsetroundjoin%
\definecolor{currentfill}{rgb}{0.121569,0.466667,0.705882}%
\pgfsetfillcolor{currentfill}%
\pgfsetfillopacity{0.889574}%
\pgfsetlinewidth{1.003750pt}%
\definecolor{currentstroke}{rgb}{0.121569,0.466667,0.705882}%
\pgfsetstrokecolor{currentstroke}%
\pgfsetstrokeopacity{0.889574}%
\pgfsetdash{}{0pt}%
\pgfpathmoveto{\pgfqpoint{1.195471in}{2.251330in}}%
\pgfpathcurveto{\pgfqpoint{1.203707in}{2.251330in}}{\pgfqpoint{1.211607in}{2.254602in}}{\pgfqpoint{1.217431in}{2.260426in}}%
\pgfpathcurveto{\pgfqpoint{1.223255in}{2.266250in}}{\pgfqpoint{1.226527in}{2.274150in}}{\pgfqpoint{1.226527in}{2.282387in}}%
\pgfpathcurveto{\pgfqpoint{1.226527in}{2.290623in}}{\pgfqpoint{1.223255in}{2.298523in}}{\pgfqpoint{1.217431in}{2.304347in}}%
\pgfpathcurveto{\pgfqpoint{1.211607in}{2.310171in}}{\pgfqpoint{1.203707in}{2.313443in}}{\pgfqpoint{1.195471in}{2.313443in}}%
\pgfpathcurveto{\pgfqpoint{1.187235in}{2.313443in}}{\pgfqpoint{1.179335in}{2.310171in}}{\pgfqpoint{1.173511in}{2.304347in}}%
\pgfpathcurveto{\pgfqpoint{1.167687in}{2.298523in}}{\pgfqpoint{1.164414in}{2.290623in}}{\pgfqpoint{1.164414in}{2.282387in}}%
\pgfpathcurveto{\pgfqpoint{1.164414in}{2.274150in}}{\pgfqpoint{1.167687in}{2.266250in}}{\pgfqpoint{1.173511in}{2.260426in}}%
\pgfpathcurveto{\pgfqpoint{1.179335in}{2.254602in}}{\pgfqpoint{1.187235in}{2.251330in}}{\pgfqpoint{1.195471in}{2.251330in}}%
\pgfpathclose%
\pgfusepath{stroke,fill}%
\end{pgfscope}%
\begin{pgfscope}%
\pgfpathrectangle{\pgfqpoint{0.100000in}{0.212622in}}{\pgfqpoint{3.696000in}{3.696000in}}%
\pgfusepath{clip}%
\pgfsetbuttcap%
\pgfsetroundjoin%
\definecolor{currentfill}{rgb}{0.121569,0.466667,0.705882}%
\pgfsetfillcolor{currentfill}%
\pgfsetfillopacity{0.890877}%
\pgfsetlinewidth{1.003750pt}%
\definecolor{currentstroke}{rgb}{0.121569,0.466667,0.705882}%
\pgfsetstrokecolor{currentstroke}%
\pgfsetstrokeopacity{0.890877}%
\pgfsetdash{}{0pt}%
\pgfpathmoveto{\pgfqpoint{1.214336in}{2.240724in}}%
\pgfpathcurveto{\pgfqpoint{1.222572in}{2.240724in}}{\pgfqpoint{1.230472in}{2.243996in}}{\pgfqpoint{1.236296in}{2.249820in}}%
\pgfpathcurveto{\pgfqpoint{1.242120in}{2.255644in}}{\pgfqpoint{1.245393in}{2.263544in}}{\pgfqpoint{1.245393in}{2.271780in}}%
\pgfpathcurveto{\pgfqpoint{1.245393in}{2.280016in}}{\pgfqpoint{1.242120in}{2.287916in}}{\pgfqpoint{1.236296in}{2.293740in}}%
\pgfpathcurveto{\pgfqpoint{1.230472in}{2.299564in}}{\pgfqpoint{1.222572in}{2.302837in}}{\pgfqpoint{1.214336in}{2.302837in}}%
\pgfpathcurveto{\pgfqpoint{1.206100in}{2.302837in}}{\pgfqpoint{1.198200in}{2.299564in}}{\pgfqpoint{1.192376in}{2.293740in}}%
\pgfpathcurveto{\pgfqpoint{1.186552in}{2.287916in}}{\pgfqpoint{1.183280in}{2.280016in}}{\pgfqpoint{1.183280in}{2.271780in}}%
\pgfpathcurveto{\pgfqpoint{1.183280in}{2.263544in}}{\pgfqpoint{1.186552in}{2.255644in}}{\pgfqpoint{1.192376in}{2.249820in}}%
\pgfpathcurveto{\pgfqpoint{1.198200in}{2.243996in}}{\pgfqpoint{1.206100in}{2.240724in}}{\pgfqpoint{1.214336in}{2.240724in}}%
\pgfpathclose%
\pgfusepath{stroke,fill}%
\end{pgfscope}%
\begin{pgfscope}%
\pgfpathrectangle{\pgfqpoint{0.100000in}{0.212622in}}{\pgfqpoint{3.696000in}{3.696000in}}%
\pgfusepath{clip}%
\pgfsetbuttcap%
\pgfsetroundjoin%
\definecolor{currentfill}{rgb}{0.121569,0.466667,0.705882}%
\pgfsetfillcolor{currentfill}%
\pgfsetfillopacity{0.891065}%
\pgfsetlinewidth{1.003750pt}%
\definecolor{currentstroke}{rgb}{0.121569,0.466667,0.705882}%
\pgfsetstrokecolor{currentstroke}%
\pgfsetstrokeopacity{0.891065}%
\pgfsetdash{}{0pt}%
\pgfpathmoveto{\pgfqpoint{2.666192in}{1.883385in}}%
\pgfpathcurveto{\pgfqpoint{2.674429in}{1.883385in}}{\pgfqpoint{2.682329in}{1.886657in}}{\pgfqpoint{2.688152in}{1.892481in}}%
\pgfpathcurveto{\pgfqpoint{2.693976in}{1.898305in}}{\pgfqpoint{2.697249in}{1.906205in}}{\pgfqpoint{2.697249in}{1.914441in}}%
\pgfpathcurveto{\pgfqpoint{2.697249in}{1.922678in}}{\pgfqpoint{2.693976in}{1.930578in}}{\pgfqpoint{2.688152in}{1.936402in}}%
\pgfpathcurveto{\pgfqpoint{2.682329in}{1.942226in}}{\pgfqpoint{2.674429in}{1.945498in}}{\pgfqpoint{2.666192in}{1.945498in}}%
\pgfpathcurveto{\pgfqpoint{2.657956in}{1.945498in}}{\pgfqpoint{2.650056in}{1.942226in}}{\pgfqpoint{2.644232in}{1.936402in}}%
\pgfpathcurveto{\pgfqpoint{2.638408in}{1.930578in}}{\pgfqpoint{2.635136in}{1.922678in}}{\pgfqpoint{2.635136in}{1.914441in}}%
\pgfpathcurveto{\pgfqpoint{2.635136in}{1.906205in}}{\pgfqpoint{2.638408in}{1.898305in}}{\pgfqpoint{2.644232in}{1.892481in}}%
\pgfpathcurveto{\pgfqpoint{2.650056in}{1.886657in}}{\pgfqpoint{2.657956in}{1.883385in}}{\pgfqpoint{2.666192in}{1.883385in}}%
\pgfpathclose%
\pgfusepath{stroke,fill}%
\end{pgfscope}%
\begin{pgfscope}%
\pgfpathrectangle{\pgfqpoint{0.100000in}{0.212622in}}{\pgfqpoint{3.696000in}{3.696000in}}%
\pgfusepath{clip}%
\pgfsetbuttcap%
\pgfsetroundjoin%
\definecolor{currentfill}{rgb}{0.121569,0.466667,0.705882}%
\pgfsetfillcolor{currentfill}%
\pgfsetfillopacity{0.891878}%
\pgfsetlinewidth{1.003750pt}%
\definecolor{currentstroke}{rgb}{0.121569,0.466667,0.705882}%
\pgfsetstrokecolor{currentstroke}%
\pgfsetstrokeopacity{0.891878}%
\pgfsetdash{}{0pt}%
\pgfpathmoveto{\pgfqpoint{1.229140in}{2.232422in}}%
\pgfpathcurveto{\pgfqpoint{1.237376in}{2.232422in}}{\pgfqpoint{1.245276in}{2.235694in}}{\pgfqpoint{1.251100in}{2.241518in}}%
\pgfpathcurveto{\pgfqpoint{1.256924in}{2.247342in}}{\pgfqpoint{1.260197in}{2.255242in}}{\pgfqpoint{1.260197in}{2.263478in}}%
\pgfpathcurveto{\pgfqpoint{1.260197in}{2.271714in}}{\pgfqpoint{1.256924in}{2.279614in}}{\pgfqpoint{1.251100in}{2.285438in}}%
\pgfpathcurveto{\pgfqpoint{1.245276in}{2.291262in}}{\pgfqpoint{1.237376in}{2.294535in}}{\pgfqpoint{1.229140in}{2.294535in}}%
\pgfpathcurveto{\pgfqpoint{1.220904in}{2.294535in}}{\pgfqpoint{1.213004in}{2.291262in}}{\pgfqpoint{1.207180in}{2.285438in}}%
\pgfpathcurveto{\pgfqpoint{1.201356in}{2.279614in}}{\pgfqpoint{1.198084in}{2.271714in}}{\pgfqpoint{1.198084in}{2.263478in}}%
\pgfpathcurveto{\pgfqpoint{1.198084in}{2.255242in}}{\pgfqpoint{1.201356in}{2.247342in}}{\pgfqpoint{1.207180in}{2.241518in}}%
\pgfpathcurveto{\pgfqpoint{1.213004in}{2.235694in}}{\pgfqpoint{1.220904in}{2.232422in}}{\pgfqpoint{1.229140in}{2.232422in}}%
\pgfpathclose%
\pgfusepath{stroke,fill}%
\end{pgfscope}%
\begin{pgfscope}%
\pgfpathrectangle{\pgfqpoint{0.100000in}{0.212622in}}{\pgfqpoint{3.696000in}{3.696000in}}%
\pgfusepath{clip}%
\pgfsetbuttcap%
\pgfsetroundjoin%
\definecolor{currentfill}{rgb}{0.121569,0.466667,0.705882}%
\pgfsetfillcolor{currentfill}%
\pgfsetfillopacity{0.893543}%
\pgfsetlinewidth{1.003750pt}%
\definecolor{currentstroke}{rgb}{0.121569,0.466667,0.705882}%
\pgfsetstrokecolor{currentstroke}%
\pgfsetstrokeopacity{0.893543}%
\pgfsetdash{}{0pt}%
\pgfpathmoveto{\pgfqpoint{1.255966in}{2.215845in}}%
\pgfpathcurveto{\pgfqpoint{1.264203in}{2.215845in}}{\pgfqpoint{1.272103in}{2.219117in}}{\pgfqpoint{1.277927in}{2.224941in}}%
\pgfpathcurveto{\pgfqpoint{1.283751in}{2.230765in}}{\pgfqpoint{1.287023in}{2.238665in}}{\pgfqpoint{1.287023in}{2.246901in}}%
\pgfpathcurveto{\pgfqpoint{1.287023in}{2.255137in}}{\pgfqpoint{1.283751in}{2.263037in}}{\pgfqpoint{1.277927in}{2.268861in}}%
\pgfpathcurveto{\pgfqpoint{1.272103in}{2.274685in}}{\pgfqpoint{1.264203in}{2.277958in}}{\pgfqpoint{1.255966in}{2.277958in}}%
\pgfpathcurveto{\pgfqpoint{1.247730in}{2.277958in}}{\pgfqpoint{1.239830in}{2.274685in}}{\pgfqpoint{1.234006in}{2.268861in}}%
\pgfpathcurveto{\pgfqpoint{1.228182in}{2.263037in}}{\pgfqpoint{1.224910in}{2.255137in}}{\pgfqpoint{1.224910in}{2.246901in}}%
\pgfpathcurveto{\pgfqpoint{1.224910in}{2.238665in}}{\pgfqpoint{1.228182in}{2.230765in}}{\pgfqpoint{1.234006in}{2.224941in}}%
\pgfpathcurveto{\pgfqpoint{1.239830in}{2.219117in}}{\pgfqpoint{1.247730in}{2.215845in}}{\pgfqpoint{1.255966in}{2.215845in}}%
\pgfpathclose%
\pgfusepath{stroke,fill}%
\end{pgfscope}%
\begin{pgfscope}%
\pgfpathrectangle{\pgfqpoint{0.100000in}{0.212622in}}{\pgfqpoint{3.696000in}{3.696000in}}%
\pgfusepath{clip}%
\pgfsetbuttcap%
\pgfsetroundjoin%
\definecolor{currentfill}{rgb}{0.121569,0.466667,0.705882}%
\pgfsetfillcolor{currentfill}%
\pgfsetfillopacity{0.894424}%
\pgfsetlinewidth{1.003750pt}%
\definecolor{currentstroke}{rgb}{0.121569,0.466667,0.705882}%
\pgfsetstrokecolor{currentstroke}%
\pgfsetstrokeopacity{0.894424}%
\pgfsetdash{}{0pt}%
\pgfpathmoveto{\pgfqpoint{2.659758in}{1.877617in}}%
\pgfpathcurveto{\pgfqpoint{2.667994in}{1.877617in}}{\pgfqpoint{2.675894in}{1.880889in}}{\pgfqpoint{2.681718in}{1.886713in}}%
\pgfpathcurveto{\pgfqpoint{2.687542in}{1.892537in}}{\pgfqpoint{2.690814in}{1.900437in}}{\pgfqpoint{2.690814in}{1.908673in}}%
\pgfpathcurveto{\pgfqpoint{2.690814in}{1.916910in}}{\pgfqpoint{2.687542in}{1.924810in}}{\pgfqpoint{2.681718in}{1.930634in}}%
\pgfpathcurveto{\pgfqpoint{2.675894in}{1.936458in}}{\pgfqpoint{2.667994in}{1.939730in}}{\pgfqpoint{2.659758in}{1.939730in}}%
\pgfpathcurveto{\pgfqpoint{2.651521in}{1.939730in}}{\pgfqpoint{2.643621in}{1.936458in}}{\pgfqpoint{2.637797in}{1.930634in}}%
\pgfpathcurveto{\pgfqpoint{2.631973in}{1.924810in}}{\pgfqpoint{2.628701in}{1.916910in}}{\pgfqpoint{2.628701in}{1.908673in}}%
\pgfpathcurveto{\pgfqpoint{2.628701in}{1.900437in}}{\pgfqpoint{2.631973in}{1.892537in}}{\pgfqpoint{2.637797in}{1.886713in}}%
\pgfpathcurveto{\pgfqpoint{2.643621in}{1.880889in}}{\pgfqpoint{2.651521in}{1.877617in}}{\pgfqpoint{2.659758in}{1.877617in}}%
\pgfpathclose%
\pgfusepath{stroke,fill}%
\end{pgfscope}%
\begin{pgfscope}%
\pgfpathrectangle{\pgfqpoint{0.100000in}{0.212622in}}{\pgfqpoint{3.696000in}{3.696000in}}%
\pgfusepath{clip}%
\pgfsetbuttcap%
\pgfsetroundjoin%
\definecolor{currentfill}{rgb}{0.121569,0.466667,0.705882}%
\pgfsetfillcolor{currentfill}%
\pgfsetfillopacity{0.895384}%
\pgfsetlinewidth{1.003750pt}%
\definecolor{currentstroke}{rgb}{0.121569,0.466667,0.705882}%
\pgfsetstrokecolor{currentstroke}%
\pgfsetstrokeopacity{0.895384}%
\pgfsetdash{}{0pt}%
\pgfpathmoveto{\pgfqpoint{1.281053in}{2.201772in}}%
\pgfpathcurveto{\pgfqpoint{1.289289in}{2.201772in}}{\pgfqpoint{1.297190in}{2.205044in}}{\pgfqpoint{1.303013in}{2.210868in}}%
\pgfpathcurveto{\pgfqpoint{1.308837in}{2.216692in}}{\pgfqpoint{1.312110in}{2.224592in}}{\pgfqpoint{1.312110in}{2.232828in}}%
\pgfpathcurveto{\pgfqpoint{1.312110in}{2.241064in}}{\pgfqpoint{1.308837in}{2.248964in}}{\pgfqpoint{1.303013in}{2.254788in}}%
\pgfpathcurveto{\pgfqpoint{1.297190in}{2.260612in}}{\pgfqpoint{1.289289in}{2.263885in}}{\pgfqpoint{1.281053in}{2.263885in}}%
\pgfpathcurveto{\pgfqpoint{1.272817in}{2.263885in}}{\pgfqpoint{1.264917in}{2.260612in}}{\pgfqpoint{1.259093in}{2.254788in}}%
\pgfpathcurveto{\pgfqpoint{1.253269in}{2.248964in}}{\pgfqpoint{1.249997in}{2.241064in}}{\pgfqpoint{1.249997in}{2.232828in}}%
\pgfpathcurveto{\pgfqpoint{1.249997in}{2.224592in}}{\pgfqpoint{1.253269in}{2.216692in}}{\pgfqpoint{1.259093in}{2.210868in}}%
\pgfpathcurveto{\pgfqpoint{1.264917in}{2.205044in}}{\pgfqpoint{1.272817in}{2.201772in}}{\pgfqpoint{1.281053in}{2.201772in}}%
\pgfpathclose%
\pgfusepath{stroke,fill}%
\end{pgfscope}%
\begin{pgfscope}%
\pgfpathrectangle{\pgfqpoint{0.100000in}{0.212622in}}{\pgfqpoint{3.696000in}{3.696000in}}%
\pgfusepath{clip}%
\pgfsetbuttcap%
\pgfsetroundjoin%
\definecolor{currentfill}{rgb}{0.121569,0.466667,0.705882}%
\pgfsetfillcolor{currentfill}%
\pgfsetfillopacity{0.896195}%
\pgfsetlinewidth{1.003750pt}%
\definecolor{currentstroke}{rgb}{0.121569,0.466667,0.705882}%
\pgfsetstrokecolor{currentstroke}%
\pgfsetstrokeopacity{0.896195}%
\pgfsetdash{}{0pt}%
\pgfpathmoveto{\pgfqpoint{2.656162in}{1.873999in}}%
\pgfpathcurveto{\pgfqpoint{2.664399in}{1.873999in}}{\pgfqpoint{2.672299in}{1.877271in}}{\pgfqpoint{2.678122in}{1.883095in}}%
\pgfpathcurveto{\pgfqpoint{2.683946in}{1.888919in}}{\pgfqpoint{2.687219in}{1.896819in}}{\pgfqpoint{2.687219in}{1.905056in}}%
\pgfpathcurveto{\pgfqpoint{2.687219in}{1.913292in}}{\pgfqpoint{2.683946in}{1.921192in}}{\pgfqpoint{2.678122in}{1.927016in}}%
\pgfpathcurveto{\pgfqpoint{2.672299in}{1.932840in}}{\pgfqpoint{2.664399in}{1.936112in}}{\pgfqpoint{2.656162in}{1.936112in}}%
\pgfpathcurveto{\pgfqpoint{2.647926in}{1.936112in}}{\pgfqpoint{2.640026in}{1.932840in}}{\pgfqpoint{2.634202in}{1.927016in}}%
\pgfpathcurveto{\pgfqpoint{2.628378in}{1.921192in}}{\pgfqpoint{2.625106in}{1.913292in}}{\pgfqpoint{2.625106in}{1.905056in}}%
\pgfpathcurveto{\pgfqpoint{2.625106in}{1.896819in}}{\pgfqpoint{2.628378in}{1.888919in}}{\pgfqpoint{2.634202in}{1.883095in}}%
\pgfpathcurveto{\pgfqpoint{2.640026in}{1.877271in}}{\pgfqpoint{2.647926in}{1.873999in}}{\pgfqpoint{2.656162in}{1.873999in}}%
\pgfpathclose%
\pgfusepath{stroke,fill}%
\end{pgfscope}%
\begin{pgfscope}%
\pgfpathrectangle{\pgfqpoint{0.100000in}{0.212622in}}{\pgfqpoint{3.696000in}{3.696000in}}%
\pgfusepath{clip}%
\pgfsetbuttcap%
\pgfsetroundjoin%
\definecolor{currentfill}{rgb}{0.121569,0.466667,0.705882}%
\pgfsetfillcolor{currentfill}%
\pgfsetfillopacity{0.897745}%
\pgfsetlinewidth{1.003750pt}%
\definecolor{currentstroke}{rgb}{0.121569,0.466667,0.705882}%
\pgfsetstrokecolor{currentstroke}%
\pgfsetstrokeopacity{0.897745}%
\pgfsetdash{}{0pt}%
\pgfpathmoveto{\pgfqpoint{1.303558in}{2.193261in}}%
\pgfpathcurveto{\pgfqpoint{1.311794in}{2.193261in}}{\pgfqpoint{1.319694in}{2.196533in}}{\pgfqpoint{1.325518in}{2.202357in}}%
\pgfpathcurveto{\pgfqpoint{1.331342in}{2.208181in}}{\pgfqpoint{1.334614in}{2.216081in}}{\pgfqpoint{1.334614in}{2.224317in}}%
\pgfpathcurveto{\pgfqpoint{1.334614in}{2.232553in}}{\pgfqpoint{1.331342in}{2.240453in}}{\pgfqpoint{1.325518in}{2.246277in}}%
\pgfpathcurveto{\pgfqpoint{1.319694in}{2.252101in}}{\pgfqpoint{1.311794in}{2.255374in}}{\pgfqpoint{1.303558in}{2.255374in}}%
\pgfpathcurveto{\pgfqpoint{1.295321in}{2.255374in}}{\pgfqpoint{1.287421in}{2.252101in}}{\pgfqpoint{1.281597in}{2.246277in}}%
\pgfpathcurveto{\pgfqpoint{1.275773in}{2.240453in}}{\pgfqpoint{1.272501in}{2.232553in}}{\pgfqpoint{1.272501in}{2.224317in}}%
\pgfpathcurveto{\pgfqpoint{1.272501in}{2.216081in}}{\pgfqpoint{1.275773in}{2.208181in}}{\pgfqpoint{1.281597in}{2.202357in}}%
\pgfpathcurveto{\pgfqpoint{1.287421in}{2.196533in}}{\pgfqpoint{1.295321in}{2.193261in}}{\pgfqpoint{1.303558in}{2.193261in}}%
\pgfpathclose%
\pgfusepath{stroke,fill}%
\end{pgfscope}%
\begin{pgfscope}%
\pgfpathrectangle{\pgfqpoint{0.100000in}{0.212622in}}{\pgfqpoint{3.696000in}{3.696000in}}%
\pgfusepath{clip}%
\pgfsetbuttcap%
\pgfsetroundjoin%
\definecolor{currentfill}{rgb}{0.121569,0.466667,0.705882}%
\pgfsetfillcolor{currentfill}%
\pgfsetfillopacity{0.898700}%
\pgfsetlinewidth{1.003750pt}%
\definecolor{currentstroke}{rgb}{0.121569,0.466667,0.705882}%
\pgfsetstrokecolor{currentstroke}%
\pgfsetstrokeopacity{0.898700}%
\pgfsetdash{}{0pt}%
\pgfpathmoveto{\pgfqpoint{2.651227in}{1.868404in}}%
\pgfpathcurveto{\pgfqpoint{2.659464in}{1.868404in}}{\pgfqpoint{2.667364in}{1.871676in}}{\pgfqpoint{2.673188in}{1.877500in}}%
\pgfpathcurveto{\pgfqpoint{2.679012in}{1.883324in}}{\pgfqpoint{2.682284in}{1.891224in}}{\pgfqpoint{2.682284in}{1.899460in}}%
\pgfpathcurveto{\pgfqpoint{2.682284in}{1.907697in}}{\pgfqpoint{2.679012in}{1.915597in}}{\pgfqpoint{2.673188in}{1.921420in}}%
\pgfpathcurveto{\pgfqpoint{2.667364in}{1.927244in}}{\pgfqpoint{2.659464in}{1.930517in}}{\pgfqpoint{2.651227in}{1.930517in}}%
\pgfpathcurveto{\pgfqpoint{2.642991in}{1.930517in}}{\pgfqpoint{2.635091in}{1.927244in}}{\pgfqpoint{2.629267in}{1.921420in}}%
\pgfpathcurveto{\pgfqpoint{2.623443in}{1.915597in}}{\pgfqpoint{2.620171in}{1.907697in}}{\pgfqpoint{2.620171in}{1.899460in}}%
\pgfpathcurveto{\pgfqpoint{2.620171in}{1.891224in}}{\pgfqpoint{2.623443in}{1.883324in}}{\pgfqpoint{2.629267in}{1.877500in}}%
\pgfpathcurveto{\pgfqpoint{2.635091in}{1.871676in}}{\pgfqpoint{2.642991in}{1.868404in}}{\pgfqpoint{2.651227in}{1.868404in}}%
\pgfpathclose%
\pgfusepath{stroke,fill}%
\end{pgfscope}%
\begin{pgfscope}%
\pgfpathrectangle{\pgfqpoint{0.100000in}{0.212622in}}{\pgfqpoint{3.696000in}{3.696000in}}%
\pgfusepath{clip}%
\pgfsetbuttcap%
\pgfsetroundjoin%
\definecolor{currentfill}{rgb}{0.121569,0.466667,0.705882}%
\pgfsetfillcolor{currentfill}%
\pgfsetfillopacity{0.899788}%
\pgfsetlinewidth{1.003750pt}%
\definecolor{currentstroke}{rgb}{0.121569,0.466667,0.705882}%
\pgfsetstrokecolor{currentstroke}%
\pgfsetstrokeopacity{0.899788}%
\pgfsetdash{}{0pt}%
\pgfpathmoveto{\pgfqpoint{1.323906in}{2.185393in}}%
\pgfpathcurveto{\pgfqpoint{1.332143in}{2.185393in}}{\pgfqpoint{1.340043in}{2.188665in}}{\pgfqpoint{1.345867in}{2.194489in}}%
\pgfpathcurveto{\pgfqpoint{1.351691in}{2.200313in}}{\pgfqpoint{1.354963in}{2.208213in}}{\pgfqpoint{1.354963in}{2.216450in}}%
\pgfpathcurveto{\pgfqpoint{1.354963in}{2.224686in}}{\pgfqpoint{1.351691in}{2.232586in}}{\pgfqpoint{1.345867in}{2.238410in}}%
\pgfpathcurveto{\pgfqpoint{1.340043in}{2.244234in}}{\pgfqpoint{1.332143in}{2.247506in}}{\pgfqpoint{1.323906in}{2.247506in}}%
\pgfpathcurveto{\pgfqpoint{1.315670in}{2.247506in}}{\pgfqpoint{1.307770in}{2.244234in}}{\pgfqpoint{1.301946in}{2.238410in}}%
\pgfpathcurveto{\pgfqpoint{1.296122in}{2.232586in}}{\pgfqpoint{1.292850in}{2.224686in}}{\pgfqpoint{1.292850in}{2.216450in}}%
\pgfpathcurveto{\pgfqpoint{1.292850in}{2.208213in}}{\pgfqpoint{1.296122in}{2.200313in}}{\pgfqpoint{1.301946in}{2.194489in}}%
\pgfpathcurveto{\pgfqpoint{1.307770in}{2.188665in}}{\pgfqpoint{1.315670in}{2.185393in}}{\pgfqpoint{1.323906in}{2.185393in}}%
\pgfpathclose%
\pgfusepath{stroke,fill}%
\end{pgfscope}%
\begin{pgfscope}%
\pgfpathrectangle{\pgfqpoint{0.100000in}{0.212622in}}{\pgfqpoint{3.696000in}{3.696000in}}%
\pgfusepath{clip}%
\pgfsetbuttcap%
\pgfsetroundjoin%
\definecolor{currentfill}{rgb}{0.121569,0.466667,0.705882}%
\pgfsetfillcolor{currentfill}%
\pgfsetfillopacity{0.900161}%
\pgfsetlinewidth{1.003750pt}%
\definecolor{currentstroke}{rgb}{0.121569,0.466667,0.705882}%
\pgfsetstrokecolor{currentstroke}%
\pgfsetstrokeopacity{0.900161}%
\pgfsetdash{}{0pt}%
\pgfpathmoveto{\pgfqpoint{2.648613in}{1.865742in}}%
\pgfpathcurveto{\pgfqpoint{2.656849in}{1.865742in}}{\pgfqpoint{2.664749in}{1.869014in}}{\pgfqpoint{2.670573in}{1.874838in}}%
\pgfpathcurveto{\pgfqpoint{2.676397in}{1.880662in}}{\pgfqpoint{2.679670in}{1.888562in}}{\pgfqpoint{2.679670in}{1.896799in}}%
\pgfpathcurveto{\pgfqpoint{2.679670in}{1.905035in}}{\pgfqpoint{2.676397in}{1.912935in}}{\pgfqpoint{2.670573in}{1.918759in}}%
\pgfpathcurveto{\pgfqpoint{2.664749in}{1.924583in}}{\pgfqpoint{2.656849in}{1.927855in}}{\pgfqpoint{2.648613in}{1.927855in}}%
\pgfpathcurveto{\pgfqpoint{2.640377in}{1.927855in}}{\pgfqpoint{2.632477in}{1.924583in}}{\pgfqpoint{2.626653in}{1.918759in}}%
\pgfpathcurveto{\pgfqpoint{2.620829in}{1.912935in}}{\pgfqpoint{2.617557in}{1.905035in}}{\pgfqpoint{2.617557in}{1.896799in}}%
\pgfpathcurveto{\pgfqpoint{2.617557in}{1.888562in}}{\pgfqpoint{2.620829in}{1.880662in}}{\pgfqpoint{2.626653in}{1.874838in}}%
\pgfpathcurveto{\pgfqpoint{2.632477in}{1.869014in}}{\pgfqpoint{2.640377in}{1.865742in}}{\pgfqpoint{2.648613in}{1.865742in}}%
\pgfpathclose%
\pgfusepath{stroke,fill}%
\end{pgfscope}%
\begin{pgfscope}%
\pgfpathrectangle{\pgfqpoint{0.100000in}{0.212622in}}{\pgfqpoint{3.696000in}{3.696000in}}%
\pgfusepath{clip}%
\pgfsetbuttcap%
\pgfsetroundjoin%
\definecolor{currentfill}{rgb}{0.121569,0.466667,0.705882}%
\pgfsetfillcolor{currentfill}%
\pgfsetfillopacity{0.900913}%
\pgfsetlinewidth{1.003750pt}%
\definecolor{currentstroke}{rgb}{0.121569,0.466667,0.705882}%
\pgfsetstrokecolor{currentstroke}%
\pgfsetstrokeopacity{0.900913}%
\pgfsetdash{}{0pt}%
\pgfpathmoveto{\pgfqpoint{2.647079in}{1.864046in}}%
\pgfpathcurveto{\pgfqpoint{2.655316in}{1.864046in}}{\pgfqpoint{2.663216in}{1.867318in}}{\pgfqpoint{2.669040in}{1.873142in}}%
\pgfpathcurveto{\pgfqpoint{2.674864in}{1.878966in}}{\pgfqpoint{2.678136in}{1.886866in}}{\pgfqpoint{2.678136in}{1.895102in}}%
\pgfpathcurveto{\pgfqpoint{2.678136in}{1.903338in}}{\pgfqpoint{2.674864in}{1.911238in}}{\pgfqpoint{2.669040in}{1.917062in}}%
\pgfpathcurveto{\pgfqpoint{2.663216in}{1.922886in}}{\pgfqpoint{2.655316in}{1.926159in}}{\pgfqpoint{2.647079in}{1.926159in}}%
\pgfpathcurveto{\pgfqpoint{2.638843in}{1.926159in}}{\pgfqpoint{2.630943in}{1.922886in}}{\pgfqpoint{2.625119in}{1.917062in}}%
\pgfpathcurveto{\pgfqpoint{2.619295in}{1.911238in}}{\pgfqpoint{2.616023in}{1.903338in}}{\pgfqpoint{2.616023in}{1.895102in}}%
\pgfpathcurveto{\pgfqpoint{2.616023in}{1.886866in}}{\pgfqpoint{2.619295in}{1.878966in}}{\pgfqpoint{2.625119in}{1.873142in}}%
\pgfpathcurveto{\pgfqpoint{2.630943in}{1.867318in}}{\pgfqpoint{2.638843in}{1.864046in}}{\pgfqpoint{2.647079in}{1.864046in}}%
\pgfpathclose%
\pgfusepath{stroke,fill}%
\end{pgfscope}%
\begin{pgfscope}%
\pgfpathrectangle{\pgfqpoint{0.100000in}{0.212622in}}{\pgfqpoint{3.696000in}{3.696000in}}%
\pgfusepath{clip}%
\pgfsetbuttcap%
\pgfsetroundjoin%
\definecolor{currentfill}{rgb}{0.121569,0.466667,0.705882}%
\pgfsetfillcolor{currentfill}%
\pgfsetfillopacity{0.901705}%
\pgfsetlinewidth{1.003750pt}%
\definecolor{currentstroke}{rgb}{0.121569,0.466667,0.705882}%
\pgfsetstrokecolor{currentstroke}%
\pgfsetstrokeopacity{0.901705}%
\pgfsetdash{}{0pt}%
\pgfpathmoveto{\pgfqpoint{1.343664in}{2.177887in}}%
\pgfpathcurveto{\pgfqpoint{1.351900in}{2.177887in}}{\pgfqpoint{1.359800in}{2.181159in}}{\pgfqpoint{1.365624in}{2.186983in}}%
\pgfpathcurveto{\pgfqpoint{1.371448in}{2.192807in}}{\pgfqpoint{1.374720in}{2.200707in}}{\pgfqpoint{1.374720in}{2.208943in}}%
\pgfpathcurveto{\pgfqpoint{1.374720in}{2.217179in}}{\pgfqpoint{1.371448in}{2.225079in}}{\pgfqpoint{1.365624in}{2.230903in}}%
\pgfpathcurveto{\pgfqpoint{1.359800in}{2.236727in}}{\pgfqpoint{1.351900in}{2.240000in}}{\pgfqpoint{1.343664in}{2.240000in}}%
\pgfpathcurveto{\pgfqpoint{1.335427in}{2.240000in}}{\pgfqpoint{1.327527in}{2.236727in}}{\pgfqpoint{1.321703in}{2.230903in}}%
\pgfpathcurveto{\pgfqpoint{1.315880in}{2.225079in}}{\pgfqpoint{1.312607in}{2.217179in}}{\pgfqpoint{1.312607in}{2.208943in}}%
\pgfpathcurveto{\pgfqpoint{1.312607in}{2.200707in}}{\pgfqpoint{1.315880in}{2.192807in}}{\pgfqpoint{1.321703in}{2.186983in}}%
\pgfpathcurveto{\pgfqpoint{1.327527in}{2.181159in}}{\pgfqpoint{1.335427in}{2.177887in}}{\pgfqpoint{1.343664in}{2.177887in}}%
\pgfpathclose%
\pgfusepath{stroke,fill}%
\end{pgfscope}%
\begin{pgfscope}%
\pgfpathrectangle{\pgfqpoint{0.100000in}{0.212622in}}{\pgfqpoint{3.696000in}{3.696000in}}%
\pgfusepath{clip}%
\pgfsetbuttcap%
\pgfsetroundjoin%
\definecolor{currentfill}{rgb}{0.121569,0.466667,0.705882}%
\pgfsetfillcolor{currentfill}%
\pgfsetfillopacity{0.901966}%
\pgfsetlinewidth{1.003750pt}%
\definecolor{currentstroke}{rgb}{0.121569,0.466667,0.705882}%
\pgfsetstrokecolor{currentstroke}%
\pgfsetstrokeopacity{0.901966}%
\pgfsetdash{}{0pt}%
\pgfpathmoveto{\pgfqpoint{2.645196in}{1.861779in}}%
\pgfpathcurveto{\pgfqpoint{2.653432in}{1.861779in}}{\pgfqpoint{2.661332in}{1.865051in}}{\pgfqpoint{2.667156in}{1.870875in}}%
\pgfpathcurveto{\pgfqpoint{2.672980in}{1.876699in}}{\pgfqpoint{2.676253in}{1.884599in}}{\pgfqpoint{2.676253in}{1.892835in}}%
\pgfpathcurveto{\pgfqpoint{2.676253in}{1.901071in}}{\pgfqpoint{2.672980in}{1.908971in}}{\pgfqpoint{2.667156in}{1.914795in}}%
\pgfpathcurveto{\pgfqpoint{2.661332in}{1.920619in}}{\pgfqpoint{2.653432in}{1.923892in}}{\pgfqpoint{2.645196in}{1.923892in}}%
\pgfpathcurveto{\pgfqpoint{2.636960in}{1.923892in}}{\pgfqpoint{2.629060in}{1.920619in}}{\pgfqpoint{2.623236in}{1.914795in}}%
\pgfpathcurveto{\pgfqpoint{2.617412in}{1.908971in}}{\pgfqpoint{2.614140in}{1.901071in}}{\pgfqpoint{2.614140in}{1.892835in}}%
\pgfpathcurveto{\pgfqpoint{2.614140in}{1.884599in}}{\pgfqpoint{2.617412in}{1.876699in}}{\pgfqpoint{2.623236in}{1.870875in}}%
\pgfpathcurveto{\pgfqpoint{2.629060in}{1.865051in}}{\pgfqpoint{2.636960in}{1.861779in}}{\pgfqpoint{2.645196in}{1.861779in}}%
\pgfpathclose%
\pgfusepath{stroke,fill}%
\end{pgfscope}%
\begin{pgfscope}%
\pgfpathrectangle{\pgfqpoint{0.100000in}{0.212622in}}{\pgfqpoint{3.696000in}{3.696000in}}%
\pgfusepath{clip}%
\pgfsetbuttcap%
\pgfsetroundjoin%
\definecolor{currentfill}{rgb}{0.121569,0.466667,0.705882}%
\pgfsetfillcolor{currentfill}%
\pgfsetfillopacity{0.903425}%
\pgfsetlinewidth{1.003750pt}%
\definecolor{currentstroke}{rgb}{0.121569,0.466667,0.705882}%
\pgfsetstrokecolor{currentstroke}%
\pgfsetstrokeopacity{0.903425}%
\pgfsetdash{}{0pt}%
\pgfpathmoveto{\pgfqpoint{1.361908in}{2.170487in}}%
\pgfpathcurveto{\pgfqpoint{1.370144in}{2.170487in}}{\pgfqpoint{1.378044in}{2.173760in}}{\pgfqpoint{1.383868in}{2.179583in}}%
\pgfpathcurveto{\pgfqpoint{1.389692in}{2.185407in}}{\pgfqpoint{1.392964in}{2.193307in}}{\pgfqpoint{1.392964in}{2.201544in}}%
\pgfpathcurveto{\pgfqpoint{1.392964in}{2.209780in}}{\pgfqpoint{1.389692in}{2.217680in}}{\pgfqpoint{1.383868in}{2.223504in}}%
\pgfpathcurveto{\pgfqpoint{1.378044in}{2.229328in}}{\pgfqpoint{1.370144in}{2.232600in}}{\pgfqpoint{1.361908in}{2.232600in}}%
\pgfpathcurveto{\pgfqpoint{1.353672in}{2.232600in}}{\pgfqpoint{1.345771in}{2.229328in}}{\pgfqpoint{1.339948in}{2.223504in}}%
\pgfpathcurveto{\pgfqpoint{1.334124in}{2.217680in}}{\pgfqpoint{1.330851in}{2.209780in}}{\pgfqpoint{1.330851in}{2.201544in}}%
\pgfpathcurveto{\pgfqpoint{1.330851in}{2.193307in}}{\pgfqpoint{1.334124in}{2.185407in}}{\pgfqpoint{1.339948in}{2.179583in}}%
\pgfpathcurveto{\pgfqpoint{1.345771in}{2.173760in}}{\pgfqpoint{1.353672in}{2.170487in}}{\pgfqpoint{1.361908in}{2.170487in}}%
\pgfpathclose%
\pgfusepath{stroke,fill}%
\end{pgfscope}%
\begin{pgfscope}%
\pgfpathrectangle{\pgfqpoint{0.100000in}{0.212622in}}{\pgfqpoint{3.696000in}{3.696000in}}%
\pgfusepath{clip}%
\pgfsetbuttcap%
\pgfsetroundjoin%
\definecolor{currentfill}{rgb}{0.121569,0.466667,0.705882}%
\pgfsetfillcolor{currentfill}%
\pgfsetfillopacity{0.903438}%
\pgfsetlinewidth{1.003750pt}%
\definecolor{currentstroke}{rgb}{0.121569,0.466667,0.705882}%
\pgfsetstrokecolor{currentstroke}%
\pgfsetstrokeopacity{0.903438}%
\pgfsetdash{}{0pt}%
\pgfpathmoveto{\pgfqpoint{2.642317in}{1.859045in}}%
\pgfpathcurveto{\pgfqpoint{2.650553in}{1.859045in}}{\pgfqpoint{2.658453in}{1.862317in}}{\pgfqpoint{2.664277in}{1.868141in}}%
\pgfpathcurveto{\pgfqpoint{2.670101in}{1.873965in}}{\pgfqpoint{2.673373in}{1.881865in}}{\pgfqpoint{2.673373in}{1.890102in}}%
\pgfpathcurveto{\pgfqpoint{2.673373in}{1.898338in}}{\pgfqpoint{2.670101in}{1.906238in}}{\pgfqpoint{2.664277in}{1.912062in}}%
\pgfpathcurveto{\pgfqpoint{2.658453in}{1.917886in}}{\pgfqpoint{2.650553in}{1.921158in}}{\pgfqpoint{2.642317in}{1.921158in}}%
\pgfpathcurveto{\pgfqpoint{2.634080in}{1.921158in}}{\pgfqpoint{2.626180in}{1.917886in}}{\pgfqpoint{2.620356in}{1.912062in}}%
\pgfpathcurveto{\pgfqpoint{2.614533in}{1.906238in}}{\pgfqpoint{2.611260in}{1.898338in}}{\pgfqpoint{2.611260in}{1.890102in}}%
\pgfpathcurveto{\pgfqpoint{2.611260in}{1.881865in}}{\pgfqpoint{2.614533in}{1.873965in}}{\pgfqpoint{2.620356in}{1.868141in}}%
\pgfpathcurveto{\pgfqpoint{2.626180in}{1.862317in}}{\pgfqpoint{2.634080in}{1.859045in}}{\pgfqpoint{2.642317in}{1.859045in}}%
\pgfpathclose%
\pgfusepath{stroke,fill}%
\end{pgfscope}%
\begin{pgfscope}%
\pgfpathrectangle{\pgfqpoint{0.100000in}{0.212622in}}{\pgfqpoint{3.696000in}{3.696000in}}%
\pgfusepath{clip}%
\pgfsetbuttcap%
\pgfsetroundjoin%
\definecolor{currentfill}{rgb}{0.121569,0.466667,0.705882}%
\pgfsetfillcolor{currentfill}%
\pgfsetfillopacity{0.904210}%
\pgfsetlinewidth{1.003750pt}%
\definecolor{currentstroke}{rgb}{0.121569,0.466667,0.705882}%
\pgfsetstrokecolor{currentstroke}%
\pgfsetstrokeopacity{0.904210}%
\pgfsetdash{}{0pt}%
\pgfpathmoveto{\pgfqpoint{2.640675in}{1.857361in}}%
\pgfpathcurveto{\pgfqpoint{2.648911in}{1.857361in}}{\pgfqpoint{2.656811in}{1.860633in}}{\pgfqpoint{2.662635in}{1.866457in}}%
\pgfpathcurveto{\pgfqpoint{2.668459in}{1.872281in}}{\pgfqpoint{2.671732in}{1.880181in}}{\pgfqpoint{2.671732in}{1.888417in}}%
\pgfpathcurveto{\pgfqpoint{2.671732in}{1.896654in}}{\pgfqpoint{2.668459in}{1.904554in}}{\pgfqpoint{2.662635in}{1.910378in}}%
\pgfpathcurveto{\pgfqpoint{2.656811in}{1.916202in}}{\pgfqpoint{2.648911in}{1.919474in}}{\pgfqpoint{2.640675in}{1.919474in}}%
\pgfpathcurveto{\pgfqpoint{2.632439in}{1.919474in}}{\pgfqpoint{2.624539in}{1.916202in}}{\pgfqpoint{2.618715in}{1.910378in}}%
\pgfpathcurveto{\pgfqpoint{2.612891in}{1.904554in}}{\pgfqpoint{2.609619in}{1.896654in}}{\pgfqpoint{2.609619in}{1.888417in}}%
\pgfpathcurveto{\pgfqpoint{2.609619in}{1.880181in}}{\pgfqpoint{2.612891in}{1.872281in}}{\pgfqpoint{2.618715in}{1.866457in}}%
\pgfpathcurveto{\pgfqpoint{2.624539in}{1.860633in}}{\pgfqpoint{2.632439in}{1.857361in}}{\pgfqpoint{2.640675in}{1.857361in}}%
\pgfpathclose%
\pgfusepath{stroke,fill}%
\end{pgfscope}%
\begin{pgfscope}%
\pgfpathrectangle{\pgfqpoint{0.100000in}{0.212622in}}{\pgfqpoint{3.696000in}{3.696000in}}%
\pgfusepath{clip}%
\pgfsetbuttcap%
\pgfsetroundjoin%
\definecolor{currentfill}{rgb}{0.121569,0.466667,0.705882}%
\pgfsetfillcolor{currentfill}%
\pgfsetfillopacity{0.904751}%
\pgfsetlinewidth{1.003750pt}%
\definecolor{currentstroke}{rgb}{0.121569,0.466667,0.705882}%
\pgfsetstrokecolor{currentstroke}%
\pgfsetstrokeopacity{0.904751}%
\pgfsetdash{}{0pt}%
\pgfpathmoveto{\pgfqpoint{1.377699in}{2.163496in}}%
\pgfpathcurveto{\pgfqpoint{1.385935in}{2.163496in}}{\pgfqpoint{1.393835in}{2.166768in}}{\pgfqpoint{1.399659in}{2.172592in}}%
\pgfpathcurveto{\pgfqpoint{1.405483in}{2.178416in}}{\pgfqpoint{1.408756in}{2.186316in}}{\pgfqpoint{1.408756in}{2.194553in}}%
\pgfpathcurveto{\pgfqpoint{1.408756in}{2.202789in}}{\pgfqpoint{1.405483in}{2.210689in}}{\pgfqpoint{1.399659in}{2.216513in}}%
\pgfpathcurveto{\pgfqpoint{1.393835in}{2.222337in}}{\pgfqpoint{1.385935in}{2.225609in}}{\pgfqpoint{1.377699in}{2.225609in}}%
\pgfpathcurveto{\pgfqpoint{1.369463in}{2.225609in}}{\pgfqpoint{1.361563in}{2.222337in}}{\pgfqpoint{1.355739in}{2.216513in}}%
\pgfpathcurveto{\pgfqpoint{1.349915in}{2.210689in}}{\pgfqpoint{1.346643in}{2.202789in}}{\pgfqpoint{1.346643in}{2.194553in}}%
\pgfpathcurveto{\pgfqpoint{1.346643in}{2.186316in}}{\pgfqpoint{1.349915in}{2.178416in}}{\pgfqpoint{1.355739in}{2.172592in}}%
\pgfpathcurveto{\pgfqpoint{1.361563in}{2.166768in}}{\pgfqpoint{1.369463in}{2.163496in}}{\pgfqpoint{1.377699in}{2.163496in}}%
\pgfpathclose%
\pgfusepath{stroke,fill}%
\end{pgfscope}%
\begin{pgfscope}%
\pgfpathrectangle{\pgfqpoint{0.100000in}{0.212622in}}{\pgfqpoint{3.696000in}{3.696000in}}%
\pgfusepath{clip}%
\pgfsetbuttcap%
\pgfsetroundjoin%
\definecolor{currentfill}{rgb}{0.121569,0.466667,0.705882}%
\pgfsetfillcolor{currentfill}%
\pgfsetfillopacity{0.905188}%
\pgfsetlinewidth{1.003750pt}%
\definecolor{currentstroke}{rgb}{0.121569,0.466667,0.705882}%
\pgfsetstrokecolor{currentstroke}%
\pgfsetstrokeopacity{0.905188}%
\pgfsetdash{}{0pt}%
\pgfpathmoveto{\pgfqpoint{2.638857in}{1.855487in}}%
\pgfpathcurveto{\pgfqpoint{2.647093in}{1.855487in}}{\pgfqpoint{2.654993in}{1.858759in}}{\pgfqpoint{2.660817in}{1.864583in}}%
\pgfpathcurveto{\pgfqpoint{2.666641in}{1.870407in}}{\pgfqpoint{2.669913in}{1.878307in}}{\pgfqpoint{2.669913in}{1.886544in}}%
\pgfpathcurveto{\pgfqpoint{2.669913in}{1.894780in}}{\pgfqpoint{2.666641in}{1.902680in}}{\pgfqpoint{2.660817in}{1.908504in}}%
\pgfpathcurveto{\pgfqpoint{2.654993in}{1.914328in}}{\pgfqpoint{2.647093in}{1.917600in}}{\pgfqpoint{2.638857in}{1.917600in}}%
\pgfpathcurveto{\pgfqpoint{2.630620in}{1.917600in}}{\pgfqpoint{2.622720in}{1.914328in}}{\pgfqpoint{2.616896in}{1.908504in}}%
\pgfpathcurveto{\pgfqpoint{2.611072in}{1.902680in}}{\pgfqpoint{2.607800in}{1.894780in}}{\pgfqpoint{2.607800in}{1.886544in}}%
\pgfpathcurveto{\pgfqpoint{2.607800in}{1.878307in}}{\pgfqpoint{2.611072in}{1.870407in}}{\pgfqpoint{2.616896in}{1.864583in}}%
\pgfpathcurveto{\pgfqpoint{2.622720in}{1.858759in}}{\pgfqpoint{2.630620in}{1.855487in}}{\pgfqpoint{2.638857in}{1.855487in}}%
\pgfpathclose%
\pgfusepath{stroke,fill}%
\end{pgfscope}%
\begin{pgfscope}%
\pgfpathrectangle{\pgfqpoint{0.100000in}{0.212622in}}{\pgfqpoint{3.696000in}{3.696000in}}%
\pgfusepath{clip}%
\pgfsetbuttcap%
\pgfsetroundjoin%
\definecolor{currentfill}{rgb}{0.121569,0.466667,0.705882}%
\pgfsetfillcolor{currentfill}%
\pgfsetfillopacity{0.905900}%
\pgfsetlinewidth{1.003750pt}%
\definecolor{currentstroke}{rgb}{0.121569,0.466667,0.705882}%
\pgfsetstrokecolor{currentstroke}%
\pgfsetstrokeopacity{0.905900}%
\pgfsetdash{}{0pt}%
\pgfpathmoveto{\pgfqpoint{1.391021in}{2.158249in}}%
\pgfpathcurveto{\pgfqpoint{1.399258in}{2.158249in}}{\pgfqpoint{1.407158in}{2.161521in}}{\pgfqpoint{1.412982in}{2.167345in}}%
\pgfpathcurveto{\pgfqpoint{1.418806in}{2.173169in}}{\pgfqpoint{1.422078in}{2.181069in}}{\pgfqpoint{1.422078in}{2.189305in}}%
\pgfpathcurveto{\pgfqpoint{1.422078in}{2.197541in}}{\pgfqpoint{1.418806in}{2.205441in}}{\pgfqpoint{1.412982in}{2.211265in}}%
\pgfpathcurveto{\pgfqpoint{1.407158in}{2.217089in}}{\pgfqpoint{1.399258in}{2.220362in}}{\pgfqpoint{1.391021in}{2.220362in}}%
\pgfpathcurveto{\pgfqpoint{1.382785in}{2.220362in}}{\pgfqpoint{1.374885in}{2.217089in}}{\pgfqpoint{1.369061in}{2.211265in}}%
\pgfpathcurveto{\pgfqpoint{1.363237in}{2.205441in}}{\pgfqpoint{1.359965in}{2.197541in}}{\pgfqpoint{1.359965in}{2.189305in}}%
\pgfpathcurveto{\pgfqpoint{1.359965in}{2.181069in}}{\pgfqpoint{1.363237in}{2.173169in}}{\pgfqpoint{1.369061in}{2.167345in}}%
\pgfpathcurveto{\pgfqpoint{1.374885in}{2.161521in}}{\pgfqpoint{1.382785in}{2.158249in}}{\pgfqpoint{1.391021in}{2.158249in}}%
\pgfpathclose%
\pgfusepath{stroke,fill}%
\end{pgfscope}%
\begin{pgfscope}%
\pgfpathrectangle{\pgfqpoint{0.100000in}{0.212622in}}{\pgfqpoint{3.696000in}{3.696000in}}%
\pgfusepath{clip}%
\pgfsetbuttcap%
\pgfsetroundjoin%
\definecolor{currentfill}{rgb}{0.121569,0.466667,0.705882}%
\pgfsetfillcolor{currentfill}%
\pgfsetfillopacity{0.906906}%
\pgfsetlinewidth{1.003750pt}%
\definecolor{currentstroke}{rgb}{0.121569,0.466667,0.705882}%
\pgfsetstrokecolor{currentstroke}%
\pgfsetstrokeopacity{0.906906}%
\pgfsetdash{}{0pt}%
\pgfpathmoveto{\pgfqpoint{1.403311in}{2.152857in}}%
\pgfpathcurveto{\pgfqpoint{1.411547in}{2.152857in}}{\pgfqpoint{1.419447in}{2.156130in}}{\pgfqpoint{1.425271in}{2.161953in}}%
\pgfpathcurveto{\pgfqpoint{1.431095in}{2.167777in}}{\pgfqpoint{1.434367in}{2.175677in}}{\pgfqpoint{1.434367in}{2.183914in}}%
\pgfpathcurveto{\pgfqpoint{1.434367in}{2.192150in}}{\pgfqpoint{1.431095in}{2.200050in}}{\pgfqpoint{1.425271in}{2.205874in}}%
\pgfpathcurveto{\pgfqpoint{1.419447in}{2.211698in}}{\pgfqpoint{1.411547in}{2.214970in}}{\pgfqpoint{1.403311in}{2.214970in}}%
\pgfpathcurveto{\pgfqpoint{1.395075in}{2.214970in}}{\pgfqpoint{1.387175in}{2.211698in}}{\pgfqpoint{1.381351in}{2.205874in}}%
\pgfpathcurveto{\pgfqpoint{1.375527in}{2.200050in}}{\pgfqpoint{1.372254in}{2.192150in}}{\pgfqpoint{1.372254in}{2.183914in}}%
\pgfpathcurveto{\pgfqpoint{1.372254in}{2.175677in}}{\pgfqpoint{1.375527in}{2.167777in}}{\pgfqpoint{1.381351in}{2.161953in}}%
\pgfpathcurveto{\pgfqpoint{1.387175in}{2.156130in}}{\pgfqpoint{1.395075in}{2.152857in}}{\pgfqpoint{1.403311in}{2.152857in}}%
\pgfpathclose%
\pgfusepath{stroke,fill}%
\end{pgfscope}%
\begin{pgfscope}%
\pgfpathrectangle{\pgfqpoint{0.100000in}{0.212622in}}{\pgfqpoint{3.696000in}{3.696000in}}%
\pgfusepath{clip}%
\pgfsetbuttcap%
\pgfsetroundjoin%
\definecolor{currentfill}{rgb}{0.121569,0.466667,0.705882}%
\pgfsetfillcolor{currentfill}%
\pgfsetfillopacity{0.907026}%
\pgfsetlinewidth{1.003750pt}%
\definecolor{currentstroke}{rgb}{0.121569,0.466667,0.705882}%
\pgfsetstrokecolor{currentstroke}%
\pgfsetstrokeopacity{0.907026}%
\pgfsetdash{}{0pt}%
\pgfpathmoveto{\pgfqpoint{2.635094in}{1.851546in}}%
\pgfpathcurveto{\pgfqpoint{2.643330in}{1.851546in}}{\pgfqpoint{2.651230in}{1.854818in}}{\pgfqpoint{2.657054in}{1.860642in}}%
\pgfpathcurveto{\pgfqpoint{2.662878in}{1.866466in}}{\pgfqpoint{2.666150in}{1.874366in}}{\pgfqpoint{2.666150in}{1.882602in}}%
\pgfpathcurveto{\pgfqpoint{2.666150in}{1.890839in}}{\pgfqpoint{2.662878in}{1.898739in}}{\pgfqpoint{2.657054in}{1.904563in}}%
\pgfpathcurveto{\pgfqpoint{2.651230in}{1.910387in}}{\pgfqpoint{2.643330in}{1.913659in}}{\pgfqpoint{2.635094in}{1.913659in}}%
\pgfpathcurveto{\pgfqpoint{2.626857in}{1.913659in}}{\pgfqpoint{2.618957in}{1.910387in}}{\pgfqpoint{2.613133in}{1.904563in}}%
\pgfpathcurveto{\pgfqpoint{2.607309in}{1.898739in}}{\pgfqpoint{2.604037in}{1.890839in}}{\pgfqpoint{2.604037in}{1.882602in}}%
\pgfpathcurveto{\pgfqpoint{2.604037in}{1.874366in}}{\pgfqpoint{2.607309in}{1.866466in}}{\pgfqpoint{2.613133in}{1.860642in}}%
\pgfpathcurveto{\pgfqpoint{2.618957in}{1.854818in}}{\pgfqpoint{2.626857in}{1.851546in}}{\pgfqpoint{2.635094in}{1.851546in}}%
\pgfpathclose%
\pgfusepath{stroke,fill}%
\end{pgfscope}%
\begin{pgfscope}%
\pgfpathrectangle{\pgfqpoint{0.100000in}{0.212622in}}{\pgfqpoint{3.696000in}{3.696000in}}%
\pgfusepath{clip}%
\pgfsetbuttcap%
\pgfsetroundjoin%
\definecolor{currentfill}{rgb}{0.121569,0.466667,0.705882}%
\pgfsetfillcolor{currentfill}%
\pgfsetfillopacity{0.907954}%
\pgfsetlinewidth{1.003750pt}%
\definecolor{currentstroke}{rgb}{0.121569,0.466667,0.705882}%
\pgfsetstrokecolor{currentstroke}%
\pgfsetstrokeopacity{0.907954}%
\pgfsetdash{}{0pt}%
\pgfpathmoveto{\pgfqpoint{1.415117in}{2.148308in}}%
\pgfpathcurveto{\pgfqpoint{1.423353in}{2.148308in}}{\pgfqpoint{1.431253in}{2.151581in}}{\pgfqpoint{1.437077in}{2.157405in}}%
\pgfpathcurveto{\pgfqpoint{1.442901in}{2.163229in}}{\pgfqpoint{1.446173in}{2.171129in}}{\pgfqpoint{1.446173in}{2.179365in}}%
\pgfpathcurveto{\pgfqpoint{1.446173in}{2.187601in}}{\pgfqpoint{1.442901in}{2.195501in}}{\pgfqpoint{1.437077in}{2.201325in}}%
\pgfpathcurveto{\pgfqpoint{1.431253in}{2.207149in}}{\pgfqpoint{1.423353in}{2.210421in}}{\pgfqpoint{1.415117in}{2.210421in}}%
\pgfpathcurveto{\pgfqpoint{1.406881in}{2.210421in}}{\pgfqpoint{1.398980in}{2.207149in}}{\pgfqpoint{1.393157in}{2.201325in}}%
\pgfpathcurveto{\pgfqpoint{1.387333in}{2.195501in}}{\pgfqpoint{1.384060in}{2.187601in}}{\pgfqpoint{1.384060in}{2.179365in}}%
\pgfpathcurveto{\pgfqpoint{1.384060in}{2.171129in}}{\pgfqpoint{1.387333in}{2.163229in}}{\pgfqpoint{1.393157in}{2.157405in}}%
\pgfpathcurveto{\pgfqpoint{1.398980in}{2.151581in}}{\pgfqpoint{1.406881in}{2.148308in}}{\pgfqpoint{1.415117in}{2.148308in}}%
\pgfpathclose%
\pgfusepath{stroke,fill}%
\end{pgfscope}%
\begin{pgfscope}%
\pgfpathrectangle{\pgfqpoint{0.100000in}{0.212622in}}{\pgfqpoint{3.696000in}{3.696000in}}%
\pgfusepath{clip}%
\pgfsetbuttcap%
\pgfsetroundjoin%
\definecolor{currentfill}{rgb}{0.121569,0.466667,0.705882}%
\pgfsetfillcolor{currentfill}%
\pgfsetfillopacity{0.909078}%
\pgfsetlinewidth{1.003750pt}%
\definecolor{currentstroke}{rgb}{0.121569,0.466667,0.705882}%
\pgfsetstrokecolor{currentstroke}%
\pgfsetstrokeopacity{0.909078}%
\pgfsetdash{}{0pt}%
\pgfpathmoveto{\pgfqpoint{1.424704in}{2.144554in}}%
\pgfpathcurveto{\pgfqpoint{1.432940in}{2.144554in}}{\pgfqpoint{1.440840in}{2.147826in}}{\pgfqpoint{1.446664in}{2.153650in}}%
\pgfpathcurveto{\pgfqpoint{1.452488in}{2.159474in}}{\pgfqpoint{1.455760in}{2.167374in}}{\pgfqpoint{1.455760in}{2.175610in}}%
\pgfpathcurveto{\pgfqpoint{1.455760in}{2.183846in}}{\pgfqpoint{1.452488in}{2.191746in}}{\pgfqpoint{1.446664in}{2.197570in}}%
\pgfpathcurveto{\pgfqpoint{1.440840in}{2.203394in}}{\pgfqpoint{1.432940in}{2.206667in}}{\pgfqpoint{1.424704in}{2.206667in}}%
\pgfpathcurveto{\pgfqpoint{1.416468in}{2.206667in}}{\pgfqpoint{1.408567in}{2.203394in}}{\pgfqpoint{1.402744in}{2.197570in}}%
\pgfpathcurveto{\pgfqpoint{1.396920in}{2.191746in}}{\pgfqpoint{1.393647in}{2.183846in}}{\pgfqpoint{1.393647in}{2.175610in}}%
\pgfpathcurveto{\pgfqpoint{1.393647in}{2.167374in}}{\pgfqpoint{1.396920in}{2.159474in}}{\pgfqpoint{1.402744in}{2.153650in}}%
\pgfpathcurveto{\pgfqpoint{1.408567in}{2.147826in}}{\pgfqpoint{1.416468in}{2.144554in}}{\pgfqpoint{1.424704in}{2.144554in}}%
\pgfpathclose%
\pgfusepath{stroke,fill}%
\end{pgfscope}%
\begin{pgfscope}%
\pgfpathrectangle{\pgfqpoint{0.100000in}{0.212622in}}{\pgfqpoint{3.696000in}{3.696000in}}%
\pgfusepath{clip}%
\pgfsetbuttcap%
\pgfsetroundjoin%
\definecolor{currentfill}{rgb}{0.121569,0.466667,0.705882}%
\pgfsetfillcolor{currentfill}%
\pgfsetfillopacity{0.909143}%
\pgfsetlinewidth{1.003750pt}%
\definecolor{currentstroke}{rgb}{0.121569,0.466667,0.705882}%
\pgfsetstrokecolor{currentstroke}%
\pgfsetstrokeopacity{0.909143}%
\pgfsetdash{}{0pt}%
\pgfpathmoveto{\pgfqpoint{2.631252in}{1.847748in}}%
\pgfpathcurveto{\pgfqpoint{2.639488in}{1.847748in}}{\pgfqpoint{2.647389in}{1.851020in}}{\pgfqpoint{2.653212in}{1.856844in}}%
\pgfpathcurveto{\pgfqpoint{2.659036in}{1.862668in}}{\pgfqpoint{2.662309in}{1.870568in}}{\pgfqpoint{2.662309in}{1.878804in}}%
\pgfpathcurveto{\pgfqpoint{2.662309in}{1.887041in}}{\pgfqpoint{2.659036in}{1.894941in}}{\pgfqpoint{2.653212in}{1.900765in}}%
\pgfpathcurveto{\pgfqpoint{2.647389in}{1.906589in}}{\pgfqpoint{2.639488in}{1.909861in}}{\pgfqpoint{2.631252in}{1.909861in}}%
\pgfpathcurveto{\pgfqpoint{2.623016in}{1.909861in}}{\pgfqpoint{2.615116in}{1.906589in}}{\pgfqpoint{2.609292in}{1.900765in}}%
\pgfpathcurveto{\pgfqpoint{2.603468in}{1.894941in}}{\pgfqpoint{2.600196in}{1.887041in}}{\pgfqpoint{2.600196in}{1.878804in}}%
\pgfpathcurveto{\pgfqpoint{2.600196in}{1.870568in}}{\pgfqpoint{2.603468in}{1.862668in}}{\pgfqpoint{2.609292in}{1.856844in}}%
\pgfpathcurveto{\pgfqpoint{2.615116in}{1.851020in}}{\pgfqpoint{2.623016in}{1.847748in}}{\pgfqpoint{2.631252in}{1.847748in}}%
\pgfpathclose%
\pgfusepath{stroke,fill}%
\end{pgfscope}%
\begin{pgfscope}%
\pgfpathrectangle{\pgfqpoint{0.100000in}{0.212622in}}{\pgfqpoint{3.696000in}{3.696000in}}%
\pgfusepath{clip}%
\pgfsetbuttcap%
\pgfsetroundjoin%
\definecolor{currentfill}{rgb}{0.121569,0.466667,0.705882}%
\pgfsetfillcolor{currentfill}%
\pgfsetfillopacity{0.909907}%
\pgfsetlinewidth{1.003750pt}%
\definecolor{currentstroke}{rgb}{0.121569,0.466667,0.705882}%
\pgfsetstrokecolor{currentstroke}%
\pgfsetstrokeopacity{0.909907}%
\pgfsetdash{}{0pt}%
\pgfpathmoveto{\pgfqpoint{1.431870in}{2.141507in}}%
\pgfpathcurveto{\pgfqpoint{1.440106in}{2.141507in}}{\pgfqpoint{1.448006in}{2.144780in}}{\pgfqpoint{1.453830in}{2.150604in}}%
\pgfpathcurveto{\pgfqpoint{1.459654in}{2.156428in}}{\pgfqpoint{1.462926in}{2.164328in}}{\pgfqpoint{1.462926in}{2.172564in}}%
\pgfpathcurveto{\pgfqpoint{1.462926in}{2.180800in}}{\pgfqpoint{1.459654in}{2.188700in}}{\pgfqpoint{1.453830in}{2.194524in}}%
\pgfpathcurveto{\pgfqpoint{1.448006in}{2.200348in}}{\pgfqpoint{1.440106in}{2.203620in}}{\pgfqpoint{1.431870in}{2.203620in}}%
\pgfpathcurveto{\pgfqpoint{1.423634in}{2.203620in}}{\pgfqpoint{1.415734in}{2.200348in}}{\pgfqpoint{1.409910in}{2.194524in}}%
\pgfpathcurveto{\pgfqpoint{1.404086in}{2.188700in}}{\pgfqpoint{1.400813in}{2.180800in}}{\pgfqpoint{1.400813in}{2.172564in}}%
\pgfpathcurveto{\pgfqpoint{1.400813in}{2.164328in}}{\pgfqpoint{1.404086in}{2.156428in}}{\pgfqpoint{1.409910in}{2.150604in}}%
\pgfpathcurveto{\pgfqpoint{1.415734in}{2.144780in}}{\pgfqpoint{1.423634in}{2.141507in}}{\pgfqpoint{1.431870in}{2.141507in}}%
\pgfpathclose%
\pgfusepath{stroke,fill}%
\end{pgfscope}%
\begin{pgfscope}%
\pgfpathrectangle{\pgfqpoint{0.100000in}{0.212622in}}{\pgfqpoint{3.696000in}{3.696000in}}%
\pgfusepath{clip}%
\pgfsetbuttcap%
\pgfsetroundjoin%
\definecolor{currentfill}{rgb}{0.121569,0.466667,0.705882}%
\pgfsetfillcolor{currentfill}%
\pgfsetfillopacity{0.910340}%
\pgfsetlinewidth{1.003750pt}%
\definecolor{currentstroke}{rgb}{0.121569,0.466667,0.705882}%
\pgfsetstrokecolor{currentstroke}%
\pgfsetstrokeopacity{0.910340}%
\pgfsetdash{}{0pt}%
\pgfpathmoveto{\pgfqpoint{2.629258in}{1.845746in}}%
\pgfpathcurveto{\pgfqpoint{2.637494in}{1.845746in}}{\pgfqpoint{2.645394in}{1.849018in}}{\pgfqpoint{2.651218in}{1.854842in}}%
\pgfpathcurveto{\pgfqpoint{2.657042in}{1.860666in}}{\pgfqpoint{2.660314in}{1.868566in}}{\pgfqpoint{2.660314in}{1.876802in}}%
\pgfpathcurveto{\pgfqpoint{2.660314in}{1.885038in}}{\pgfqpoint{2.657042in}{1.892938in}}{\pgfqpoint{2.651218in}{1.898762in}}%
\pgfpathcurveto{\pgfqpoint{2.645394in}{1.904586in}}{\pgfqpoint{2.637494in}{1.907859in}}{\pgfqpoint{2.629258in}{1.907859in}}%
\pgfpathcurveto{\pgfqpoint{2.621021in}{1.907859in}}{\pgfqpoint{2.613121in}{1.904586in}}{\pgfqpoint{2.607297in}{1.898762in}}%
\pgfpathcurveto{\pgfqpoint{2.601473in}{1.892938in}}{\pgfqpoint{2.598201in}{1.885038in}}{\pgfqpoint{2.598201in}{1.876802in}}%
\pgfpathcurveto{\pgfqpoint{2.598201in}{1.868566in}}{\pgfqpoint{2.601473in}{1.860666in}}{\pgfqpoint{2.607297in}{1.854842in}}%
\pgfpathcurveto{\pgfqpoint{2.613121in}{1.849018in}}{\pgfqpoint{2.621021in}{1.845746in}}{\pgfqpoint{2.629258in}{1.845746in}}%
\pgfpathclose%
\pgfusepath{stroke,fill}%
\end{pgfscope}%
\begin{pgfscope}%
\pgfpathrectangle{\pgfqpoint{0.100000in}{0.212622in}}{\pgfqpoint{3.696000in}{3.696000in}}%
\pgfusepath{clip}%
\pgfsetbuttcap%
\pgfsetroundjoin%
\definecolor{currentfill}{rgb}{0.121569,0.466667,0.705882}%
\pgfsetfillcolor{currentfill}%
\pgfsetfillopacity{0.910595}%
\pgfsetlinewidth{1.003750pt}%
\definecolor{currentstroke}{rgb}{0.121569,0.466667,0.705882}%
\pgfsetstrokecolor{currentstroke}%
\pgfsetstrokeopacity{0.910595}%
\pgfsetdash{}{0pt}%
\pgfpathmoveto{\pgfqpoint{1.437897in}{2.138887in}}%
\pgfpathcurveto{\pgfqpoint{1.446134in}{2.138887in}}{\pgfqpoint{1.454034in}{2.142160in}}{\pgfqpoint{1.459858in}{2.147984in}}%
\pgfpathcurveto{\pgfqpoint{1.465681in}{2.153807in}}{\pgfqpoint{1.468954in}{2.161707in}}{\pgfqpoint{1.468954in}{2.169944in}}%
\pgfpathcurveto{\pgfqpoint{1.468954in}{2.178180in}}{\pgfqpoint{1.465681in}{2.186080in}}{\pgfqpoint{1.459858in}{2.191904in}}%
\pgfpathcurveto{\pgfqpoint{1.454034in}{2.197728in}}{\pgfqpoint{1.446134in}{2.201000in}}{\pgfqpoint{1.437897in}{2.201000in}}%
\pgfpathcurveto{\pgfqpoint{1.429661in}{2.201000in}}{\pgfqpoint{1.421761in}{2.197728in}}{\pgfqpoint{1.415937in}{2.191904in}}%
\pgfpathcurveto{\pgfqpoint{1.410113in}{2.186080in}}{\pgfqpoint{1.406841in}{2.178180in}}{\pgfqpoint{1.406841in}{2.169944in}}%
\pgfpathcurveto{\pgfqpoint{1.406841in}{2.161707in}}{\pgfqpoint{1.410113in}{2.153807in}}{\pgfqpoint{1.415937in}{2.147984in}}%
\pgfpathcurveto{\pgfqpoint{1.421761in}{2.142160in}}{\pgfqpoint{1.429661in}{2.138887in}}{\pgfqpoint{1.437897in}{2.138887in}}%
\pgfpathclose%
\pgfusepath{stroke,fill}%
\end{pgfscope}%
\begin{pgfscope}%
\pgfpathrectangle{\pgfqpoint{0.100000in}{0.212622in}}{\pgfqpoint{3.696000in}{3.696000in}}%
\pgfusepath{clip}%
\pgfsetbuttcap%
\pgfsetroundjoin%
\definecolor{currentfill}{rgb}{0.121569,0.466667,0.705882}%
\pgfsetfillcolor{currentfill}%
\pgfsetfillopacity{0.911675}%
\pgfsetlinewidth{1.003750pt}%
\definecolor{currentstroke}{rgb}{0.121569,0.466667,0.705882}%
\pgfsetstrokecolor{currentstroke}%
\pgfsetstrokeopacity{0.911675}%
\pgfsetdash{}{0pt}%
\pgfpathmoveto{\pgfqpoint{2.626811in}{1.843342in}}%
\pgfpathcurveto{\pgfqpoint{2.635047in}{1.843342in}}{\pgfqpoint{2.642947in}{1.846614in}}{\pgfqpoint{2.648771in}{1.852438in}}%
\pgfpathcurveto{\pgfqpoint{2.654595in}{1.858262in}}{\pgfqpoint{2.657868in}{1.866162in}}{\pgfqpoint{2.657868in}{1.874398in}}%
\pgfpathcurveto{\pgfqpoint{2.657868in}{1.882634in}}{\pgfqpoint{2.654595in}{1.890534in}}{\pgfqpoint{2.648771in}{1.896358in}}%
\pgfpathcurveto{\pgfqpoint{2.642947in}{1.902182in}}{\pgfqpoint{2.635047in}{1.905455in}}{\pgfqpoint{2.626811in}{1.905455in}}%
\pgfpathcurveto{\pgfqpoint{2.618575in}{1.905455in}}{\pgfqpoint{2.610675in}{1.902182in}}{\pgfqpoint{2.604851in}{1.896358in}}%
\pgfpathcurveto{\pgfqpoint{2.599027in}{1.890534in}}{\pgfqpoint{2.595755in}{1.882634in}}{\pgfqpoint{2.595755in}{1.874398in}}%
\pgfpathcurveto{\pgfqpoint{2.595755in}{1.866162in}}{\pgfqpoint{2.599027in}{1.858262in}}{\pgfqpoint{2.604851in}{1.852438in}}%
\pgfpathcurveto{\pgfqpoint{2.610675in}{1.846614in}}{\pgfqpoint{2.618575in}{1.843342in}}{\pgfqpoint{2.626811in}{1.843342in}}%
\pgfpathclose%
\pgfusepath{stroke,fill}%
\end{pgfscope}%
\begin{pgfscope}%
\pgfpathrectangle{\pgfqpoint{0.100000in}{0.212622in}}{\pgfqpoint{3.696000in}{3.696000in}}%
\pgfusepath{clip}%
\pgfsetbuttcap%
\pgfsetroundjoin%
\definecolor{currentfill}{rgb}{0.121569,0.466667,0.705882}%
\pgfsetfillcolor{currentfill}%
\pgfsetfillopacity{0.911784}%
\pgfsetlinewidth{1.003750pt}%
\definecolor{currentstroke}{rgb}{0.121569,0.466667,0.705882}%
\pgfsetstrokecolor{currentstroke}%
\pgfsetstrokeopacity{0.911784}%
\pgfsetdash{}{0pt}%
\pgfpathmoveto{\pgfqpoint{1.448800in}{2.133510in}}%
\pgfpathcurveto{\pgfqpoint{1.457037in}{2.133510in}}{\pgfqpoint{1.464937in}{2.136782in}}{\pgfqpoint{1.470761in}{2.142606in}}%
\pgfpathcurveto{\pgfqpoint{1.476584in}{2.148430in}}{\pgfqpoint{1.479857in}{2.156330in}}{\pgfqpoint{1.479857in}{2.164566in}}%
\pgfpathcurveto{\pgfqpoint{1.479857in}{2.172802in}}{\pgfqpoint{1.476584in}{2.180702in}}{\pgfqpoint{1.470761in}{2.186526in}}%
\pgfpathcurveto{\pgfqpoint{1.464937in}{2.192350in}}{\pgfqpoint{1.457037in}{2.195623in}}{\pgfqpoint{1.448800in}{2.195623in}}%
\pgfpathcurveto{\pgfqpoint{1.440564in}{2.195623in}}{\pgfqpoint{1.432664in}{2.192350in}}{\pgfqpoint{1.426840in}{2.186526in}}%
\pgfpathcurveto{\pgfqpoint{1.421016in}{2.180702in}}{\pgfqpoint{1.417744in}{2.172802in}}{\pgfqpoint{1.417744in}{2.164566in}}%
\pgfpathcurveto{\pgfqpoint{1.417744in}{2.156330in}}{\pgfqpoint{1.421016in}{2.148430in}}{\pgfqpoint{1.426840in}{2.142606in}}%
\pgfpathcurveto{\pgfqpoint{1.432664in}{2.136782in}}{\pgfqpoint{1.440564in}{2.133510in}}{\pgfqpoint{1.448800in}{2.133510in}}%
\pgfpathclose%
\pgfusepath{stroke,fill}%
\end{pgfscope}%
\begin{pgfscope}%
\pgfpathrectangle{\pgfqpoint{0.100000in}{0.212622in}}{\pgfqpoint{3.696000in}{3.696000in}}%
\pgfusepath{clip}%
\pgfsetbuttcap%
\pgfsetroundjoin%
\definecolor{currentfill}{rgb}{0.121569,0.466667,0.705882}%
\pgfsetfillcolor{currentfill}%
\pgfsetfillopacity{0.912916}%
\pgfsetlinewidth{1.003750pt}%
\definecolor{currentstroke}{rgb}{0.121569,0.466667,0.705882}%
\pgfsetstrokecolor{currentstroke}%
\pgfsetstrokeopacity{0.912916}%
\pgfsetdash{}{0pt}%
\pgfpathmoveto{\pgfqpoint{1.458667in}{2.128907in}}%
\pgfpathcurveto{\pgfqpoint{1.466903in}{2.128907in}}{\pgfqpoint{1.474803in}{2.132180in}}{\pgfqpoint{1.480627in}{2.138004in}}%
\pgfpathcurveto{\pgfqpoint{1.486451in}{2.143827in}}{\pgfqpoint{1.489723in}{2.151727in}}{\pgfqpoint{1.489723in}{2.159964in}}%
\pgfpathcurveto{\pgfqpoint{1.489723in}{2.168200in}}{\pgfqpoint{1.486451in}{2.176100in}}{\pgfqpoint{1.480627in}{2.181924in}}%
\pgfpathcurveto{\pgfqpoint{1.474803in}{2.187748in}}{\pgfqpoint{1.466903in}{2.191020in}}{\pgfqpoint{1.458667in}{2.191020in}}%
\pgfpathcurveto{\pgfqpoint{1.450431in}{2.191020in}}{\pgfqpoint{1.442530in}{2.187748in}}{\pgfqpoint{1.436707in}{2.181924in}}%
\pgfpathcurveto{\pgfqpoint{1.430883in}{2.176100in}}{\pgfqpoint{1.427610in}{2.168200in}}{\pgfqpoint{1.427610in}{2.159964in}}%
\pgfpathcurveto{\pgfqpoint{1.427610in}{2.151727in}}{\pgfqpoint{1.430883in}{2.143827in}}{\pgfqpoint{1.436707in}{2.138004in}}%
\pgfpathcurveto{\pgfqpoint{1.442530in}{2.132180in}}{\pgfqpoint{1.450431in}{2.128907in}}{\pgfqpoint{1.458667in}{2.128907in}}%
\pgfpathclose%
\pgfusepath{stroke,fill}%
\end{pgfscope}%
\begin{pgfscope}%
\pgfpathrectangle{\pgfqpoint{0.100000in}{0.212622in}}{\pgfqpoint{3.696000in}{3.696000in}}%
\pgfusepath{clip}%
\pgfsetbuttcap%
\pgfsetroundjoin%
\definecolor{currentfill}{rgb}{0.121569,0.466667,0.705882}%
\pgfsetfillcolor{currentfill}%
\pgfsetfillopacity{0.913754}%
\pgfsetlinewidth{1.003750pt}%
\definecolor{currentstroke}{rgb}{0.121569,0.466667,0.705882}%
\pgfsetstrokecolor{currentstroke}%
\pgfsetstrokeopacity{0.913754}%
\pgfsetdash{}{0pt}%
\pgfpathmoveto{\pgfqpoint{2.623532in}{1.839576in}}%
\pgfpathcurveto{\pgfqpoint{2.631769in}{1.839576in}}{\pgfqpoint{2.639669in}{1.842849in}}{\pgfqpoint{2.645492in}{1.848672in}}%
\pgfpathcurveto{\pgfqpoint{2.651316in}{1.854496in}}{\pgfqpoint{2.654589in}{1.862396in}}{\pgfqpoint{2.654589in}{1.870633in}}%
\pgfpathcurveto{\pgfqpoint{2.654589in}{1.878869in}}{\pgfqpoint{2.651316in}{1.886769in}}{\pgfqpoint{2.645492in}{1.892593in}}%
\pgfpathcurveto{\pgfqpoint{2.639669in}{1.898417in}}{\pgfqpoint{2.631769in}{1.901689in}}{\pgfqpoint{2.623532in}{1.901689in}}%
\pgfpathcurveto{\pgfqpoint{2.615296in}{1.901689in}}{\pgfqpoint{2.607396in}{1.898417in}}{\pgfqpoint{2.601572in}{1.892593in}}%
\pgfpathcurveto{\pgfqpoint{2.595748in}{1.886769in}}{\pgfqpoint{2.592476in}{1.878869in}}{\pgfqpoint{2.592476in}{1.870633in}}%
\pgfpathcurveto{\pgfqpoint{2.592476in}{1.862396in}}{\pgfqpoint{2.595748in}{1.854496in}}{\pgfqpoint{2.601572in}{1.848672in}}%
\pgfpathcurveto{\pgfqpoint{2.607396in}{1.842849in}}{\pgfqpoint{2.615296in}{1.839576in}}{\pgfqpoint{2.623532in}{1.839576in}}%
\pgfpathclose%
\pgfusepath{stroke,fill}%
\end{pgfscope}%
\begin{pgfscope}%
\pgfpathrectangle{\pgfqpoint{0.100000in}{0.212622in}}{\pgfqpoint{3.696000in}{3.696000in}}%
\pgfusepath{clip}%
\pgfsetbuttcap%
\pgfsetroundjoin%
\definecolor{currentfill}{rgb}{0.121569,0.466667,0.705882}%
\pgfsetfillcolor{currentfill}%
\pgfsetfillopacity{0.913839}%
\pgfsetlinewidth{1.003750pt}%
\definecolor{currentstroke}{rgb}{0.121569,0.466667,0.705882}%
\pgfsetstrokecolor{currentstroke}%
\pgfsetstrokeopacity{0.913839}%
\pgfsetdash{}{0pt}%
\pgfpathmoveto{\pgfqpoint{1.467352in}{2.125660in}}%
\pgfpathcurveto{\pgfqpoint{1.475588in}{2.125660in}}{\pgfqpoint{1.483488in}{2.128932in}}{\pgfqpoint{1.489312in}{2.134756in}}%
\pgfpathcurveto{\pgfqpoint{1.495136in}{2.140580in}}{\pgfqpoint{1.498409in}{2.148480in}}{\pgfqpoint{1.498409in}{2.156716in}}%
\pgfpathcurveto{\pgfqpoint{1.498409in}{2.164953in}}{\pgfqpoint{1.495136in}{2.172853in}}{\pgfqpoint{1.489312in}{2.178677in}}%
\pgfpathcurveto{\pgfqpoint{1.483488in}{2.184501in}}{\pgfqpoint{1.475588in}{2.187773in}}{\pgfqpoint{1.467352in}{2.187773in}}%
\pgfpathcurveto{\pgfqpoint{1.459116in}{2.187773in}}{\pgfqpoint{1.451216in}{2.184501in}}{\pgfqpoint{1.445392in}{2.178677in}}%
\pgfpathcurveto{\pgfqpoint{1.439568in}{2.172853in}}{\pgfqpoint{1.436296in}{2.164953in}}{\pgfqpoint{1.436296in}{2.156716in}}%
\pgfpathcurveto{\pgfqpoint{1.436296in}{2.148480in}}{\pgfqpoint{1.439568in}{2.140580in}}{\pgfqpoint{1.445392in}{2.134756in}}%
\pgfpathcurveto{\pgfqpoint{1.451216in}{2.128932in}}{\pgfqpoint{1.459116in}{2.125660in}}{\pgfqpoint{1.467352in}{2.125660in}}%
\pgfpathclose%
\pgfusepath{stroke,fill}%
\end{pgfscope}%
\begin{pgfscope}%
\pgfpathrectangle{\pgfqpoint{0.100000in}{0.212622in}}{\pgfqpoint{3.696000in}{3.696000in}}%
\pgfusepath{clip}%
\pgfsetbuttcap%
\pgfsetroundjoin%
\definecolor{currentfill}{rgb}{0.121569,0.466667,0.705882}%
\pgfsetfillcolor{currentfill}%
\pgfsetfillopacity{0.914521}%
\pgfsetlinewidth{1.003750pt}%
\definecolor{currentstroke}{rgb}{0.121569,0.466667,0.705882}%
\pgfsetstrokecolor{currentstroke}%
\pgfsetstrokeopacity{0.914521}%
\pgfsetdash{}{0pt}%
\pgfpathmoveto{\pgfqpoint{1.474747in}{2.121955in}}%
\pgfpathcurveto{\pgfqpoint{1.482983in}{2.121955in}}{\pgfqpoint{1.490883in}{2.125227in}}{\pgfqpoint{1.496707in}{2.131051in}}%
\pgfpathcurveto{\pgfqpoint{1.502531in}{2.136875in}}{\pgfqpoint{1.505804in}{2.144775in}}{\pgfqpoint{1.505804in}{2.153011in}}%
\pgfpathcurveto{\pgfqpoint{1.505804in}{2.161248in}}{\pgfqpoint{1.502531in}{2.169148in}}{\pgfqpoint{1.496707in}{2.174972in}}%
\pgfpathcurveto{\pgfqpoint{1.490883in}{2.180795in}}{\pgfqpoint{1.482983in}{2.184068in}}{\pgfqpoint{1.474747in}{2.184068in}}%
\pgfpathcurveto{\pgfqpoint{1.466511in}{2.184068in}}{\pgfqpoint{1.458611in}{2.180795in}}{\pgfqpoint{1.452787in}{2.174972in}}%
\pgfpathcurveto{\pgfqpoint{1.446963in}{2.169148in}}{\pgfqpoint{1.443691in}{2.161248in}}{\pgfqpoint{1.443691in}{2.153011in}}%
\pgfpathcurveto{\pgfqpoint{1.443691in}{2.144775in}}{\pgfqpoint{1.446963in}{2.136875in}}{\pgfqpoint{1.452787in}{2.131051in}}%
\pgfpathcurveto{\pgfqpoint{1.458611in}{2.125227in}}{\pgfqpoint{1.466511in}{2.121955in}}{\pgfqpoint{1.474747in}{2.121955in}}%
\pgfpathclose%
\pgfusepath{stroke,fill}%
\end{pgfscope}%
\begin{pgfscope}%
\pgfpathrectangle{\pgfqpoint{0.100000in}{0.212622in}}{\pgfqpoint{3.696000in}{3.696000in}}%
\pgfusepath{clip}%
\pgfsetbuttcap%
\pgfsetroundjoin%
\definecolor{currentfill}{rgb}{0.121569,0.466667,0.705882}%
\pgfsetfillcolor{currentfill}%
\pgfsetfillopacity{0.914921}%
\pgfsetlinewidth{1.003750pt}%
\definecolor{currentstroke}{rgb}{0.121569,0.466667,0.705882}%
\pgfsetstrokecolor{currentstroke}%
\pgfsetstrokeopacity{0.914921}%
\pgfsetdash{}{0pt}%
\pgfpathmoveto{\pgfqpoint{2.621679in}{1.837683in}}%
\pgfpathcurveto{\pgfqpoint{2.629916in}{1.837683in}}{\pgfqpoint{2.637816in}{1.840956in}}{\pgfqpoint{2.643640in}{1.846779in}}%
\pgfpathcurveto{\pgfqpoint{2.649463in}{1.852603in}}{\pgfqpoint{2.652736in}{1.860503in}}{\pgfqpoint{2.652736in}{1.868740in}}%
\pgfpathcurveto{\pgfqpoint{2.652736in}{1.876976in}}{\pgfqpoint{2.649463in}{1.884876in}}{\pgfqpoint{2.643640in}{1.890700in}}%
\pgfpathcurveto{\pgfqpoint{2.637816in}{1.896524in}}{\pgfqpoint{2.629916in}{1.899796in}}{\pgfqpoint{2.621679in}{1.899796in}}%
\pgfpathcurveto{\pgfqpoint{2.613443in}{1.899796in}}{\pgfqpoint{2.605543in}{1.896524in}}{\pgfqpoint{2.599719in}{1.890700in}}%
\pgfpathcurveto{\pgfqpoint{2.593895in}{1.884876in}}{\pgfqpoint{2.590623in}{1.876976in}}{\pgfqpoint{2.590623in}{1.868740in}}%
\pgfpathcurveto{\pgfqpoint{2.590623in}{1.860503in}}{\pgfqpoint{2.593895in}{1.852603in}}{\pgfqpoint{2.599719in}{1.846779in}}%
\pgfpathcurveto{\pgfqpoint{2.605543in}{1.840956in}}{\pgfqpoint{2.613443in}{1.837683in}}{\pgfqpoint{2.621679in}{1.837683in}}%
\pgfpathclose%
\pgfusepath{stroke,fill}%
\end{pgfscope}%
\begin{pgfscope}%
\pgfpathrectangle{\pgfqpoint{0.100000in}{0.212622in}}{\pgfqpoint{3.696000in}{3.696000in}}%
\pgfusepath{clip}%
\pgfsetbuttcap%
\pgfsetroundjoin%
\definecolor{currentfill}{rgb}{0.121569,0.466667,0.705882}%
\pgfsetfillcolor{currentfill}%
\pgfsetfillopacity{0.915087}%
\pgfsetlinewidth{1.003750pt}%
\definecolor{currentstroke}{rgb}{0.121569,0.466667,0.705882}%
\pgfsetstrokecolor{currentstroke}%
\pgfsetstrokeopacity{0.915087}%
\pgfsetdash{}{0pt}%
\pgfpathmoveto{\pgfqpoint{1.481243in}{2.118651in}}%
\pgfpathcurveto{\pgfqpoint{1.489479in}{2.118651in}}{\pgfqpoint{1.497379in}{2.121924in}}{\pgfqpoint{1.503203in}{2.127748in}}%
\pgfpathcurveto{\pgfqpoint{1.509027in}{2.133571in}}{\pgfqpoint{1.512299in}{2.141472in}}{\pgfqpoint{1.512299in}{2.149708in}}%
\pgfpathcurveto{\pgfqpoint{1.512299in}{2.157944in}}{\pgfqpoint{1.509027in}{2.165844in}}{\pgfqpoint{1.503203in}{2.171668in}}%
\pgfpathcurveto{\pgfqpoint{1.497379in}{2.177492in}}{\pgfqpoint{1.489479in}{2.180764in}}{\pgfqpoint{1.481243in}{2.180764in}}%
\pgfpathcurveto{\pgfqpoint{1.473006in}{2.180764in}}{\pgfqpoint{1.465106in}{2.177492in}}{\pgfqpoint{1.459282in}{2.171668in}}%
\pgfpathcurveto{\pgfqpoint{1.453459in}{2.165844in}}{\pgfqpoint{1.450186in}{2.157944in}}{\pgfqpoint{1.450186in}{2.149708in}}%
\pgfpathcurveto{\pgfqpoint{1.450186in}{2.141472in}}{\pgfqpoint{1.453459in}{2.133571in}}{\pgfqpoint{1.459282in}{2.127748in}}%
\pgfpathcurveto{\pgfqpoint{1.465106in}{2.121924in}}{\pgfqpoint{1.473006in}{2.118651in}}{\pgfqpoint{1.481243in}{2.118651in}}%
\pgfpathclose%
\pgfusepath{stroke,fill}%
\end{pgfscope}%
\begin{pgfscope}%
\pgfpathrectangle{\pgfqpoint{0.100000in}{0.212622in}}{\pgfqpoint{3.696000in}{3.696000in}}%
\pgfusepath{clip}%
\pgfsetbuttcap%
\pgfsetroundjoin%
\definecolor{currentfill}{rgb}{0.121569,0.466667,0.705882}%
\pgfsetfillcolor{currentfill}%
\pgfsetfillopacity{0.915540}%
\pgfsetlinewidth{1.003750pt}%
\definecolor{currentstroke}{rgb}{0.121569,0.466667,0.705882}%
\pgfsetstrokecolor{currentstroke}%
\pgfsetstrokeopacity{0.915540}%
\pgfsetdash{}{0pt}%
\pgfpathmoveto{\pgfqpoint{2.620600in}{1.836554in}}%
\pgfpathcurveto{\pgfqpoint{2.628836in}{1.836554in}}{\pgfqpoint{2.636736in}{1.839827in}}{\pgfqpoint{2.642560in}{1.845650in}}%
\pgfpathcurveto{\pgfqpoint{2.648384in}{1.851474in}}{\pgfqpoint{2.651656in}{1.859374in}}{\pgfqpoint{2.651656in}{1.867611in}}%
\pgfpathcurveto{\pgfqpoint{2.651656in}{1.875847in}}{\pgfqpoint{2.648384in}{1.883747in}}{\pgfqpoint{2.642560in}{1.889571in}}%
\pgfpathcurveto{\pgfqpoint{2.636736in}{1.895395in}}{\pgfqpoint{2.628836in}{1.898667in}}{\pgfqpoint{2.620600in}{1.898667in}}%
\pgfpathcurveto{\pgfqpoint{2.612364in}{1.898667in}}{\pgfqpoint{2.604464in}{1.895395in}}{\pgfqpoint{2.598640in}{1.889571in}}%
\pgfpathcurveto{\pgfqpoint{2.592816in}{1.883747in}}{\pgfqpoint{2.589543in}{1.875847in}}{\pgfqpoint{2.589543in}{1.867611in}}%
\pgfpathcurveto{\pgfqpoint{2.589543in}{1.859374in}}{\pgfqpoint{2.592816in}{1.851474in}}{\pgfqpoint{2.598640in}{1.845650in}}%
\pgfpathcurveto{\pgfqpoint{2.604464in}{1.839827in}}{\pgfqpoint{2.612364in}{1.836554in}}{\pgfqpoint{2.620600in}{1.836554in}}%
\pgfpathclose%
\pgfusepath{stroke,fill}%
\end{pgfscope}%
\begin{pgfscope}%
\pgfpathrectangle{\pgfqpoint{0.100000in}{0.212622in}}{\pgfqpoint{3.696000in}{3.696000in}}%
\pgfusepath{clip}%
\pgfsetbuttcap%
\pgfsetroundjoin%
\definecolor{currentfill}{rgb}{0.121569,0.466667,0.705882}%
\pgfsetfillcolor{currentfill}%
\pgfsetfillopacity{0.915905}%
\pgfsetlinewidth{1.003750pt}%
\definecolor{currentstroke}{rgb}{0.121569,0.466667,0.705882}%
\pgfsetstrokecolor{currentstroke}%
\pgfsetstrokeopacity{0.915905}%
\pgfsetdash{}{0pt}%
\pgfpathmoveto{\pgfqpoint{1.493004in}{2.111058in}}%
\pgfpathcurveto{\pgfqpoint{1.501240in}{2.111058in}}{\pgfqpoint{1.509140in}{2.114330in}}{\pgfqpoint{1.514964in}{2.120154in}}%
\pgfpathcurveto{\pgfqpoint{1.520788in}{2.125978in}}{\pgfqpoint{1.524060in}{2.133878in}}{\pgfqpoint{1.524060in}{2.142114in}}%
\pgfpathcurveto{\pgfqpoint{1.524060in}{2.150351in}}{\pgfqpoint{1.520788in}{2.158251in}}{\pgfqpoint{1.514964in}{2.164075in}}%
\pgfpathcurveto{\pgfqpoint{1.509140in}{2.169899in}}{\pgfqpoint{1.501240in}{2.173171in}}{\pgfqpoint{1.493004in}{2.173171in}}%
\pgfpathcurveto{\pgfqpoint{1.484767in}{2.173171in}}{\pgfqpoint{1.476867in}{2.169899in}}{\pgfqpoint{1.471043in}{2.164075in}}%
\pgfpathcurveto{\pgfqpoint{1.465219in}{2.158251in}}{\pgfqpoint{1.461947in}{2.150351in}}{\pgfqpoint{1.461947in}{2.142114in}}%
\pgfpathcurveto{\pgfqpoint{1.461947in}{2.133878in}}{\pgfqpoint{1.465219in}{2.125978in}}{\pgfqpoint{1.471043in}{2.120154in}}%
\pgfpathcurveto{\pgfqpoint{1.476867in}{2.114330in}}{\pgfqpoint{1.484767in}{2.111058in}}{\pgfqpoint{1.493004in}{2.111058in}}%
\pgfpathclose%
\pgfusepath{stroke,fill}%
\end{pgfscope}%
\begin{pgfscope}%
\pgfpathrectangle{\pgfqpoint{0.100000in}{0.212622in}}{\pgfqpoint{3.696000in}{3.696000in}}%
\pgfusepath{clip}%
\pgfsetbuttcap%
\pgfsetroundjoin%
\definecolor{currentfill}{rgb}{0.121569,0.466667,0.705882}%
\pgfsetfillcolor{currentfill}%
\pgfsetfillopacity{0.916466}%
\pgfsetlinewidth{1.003750pt}%
\definecolor{currentstroke}{rgb}{0.121569,0.466667,0.705882}%
\pgfsetstrokecolor{currentstroke}%
\pgfsetstrokeopacity{0.916466}%
\pgfsetdash{}{0pt}%
\pgfpathmoveto{\pgfqpoint{2.619029in}{1.834813in}}%
\pgfpathcurveto{\pgfqpoint{2.627265in}{1.834813in}}{\pgfqpoint{2.635165in}{1.838085in}}{\pgfqpoint{2.640989in}{1.843909in}}%
\pgfpathcurveto{\pgfqpoint{2.646813in}{1.849733in}}{\pgfqpoint{2.650085in}{1.857633in}}{\pgfqpoint{2.650085in}{1.865869in}}%
\pgfpathcurveto{\pgfqpoint{2.650085in}{1.874106in}}{\pgfqpoint{2.646813in}{1.882006in}}{\pgfqpoint{2.640989in}{1.887830in}}%
\pgfpathcurveto{\pgfqpoint{2.635165in}{1.893654in}}{\pgfqpoint{2.627265in}{1.896926in}}{\pgfqpoint{2.619029in}{1.896926in}}%
\pgfpathcurveto{\pgfqpoint{2.610793in}{1.896926in}}{\pgfqpoint{2.602892in}{1.893654in}}{\pgfqpoint{2.597069in}{1.887830in}}%
\pgfpathcurveto{\pgfqpoint{2.591245in}{1.882006in}}{\pgfqpoint{2.587972in}{1.874106in}}{\pgfqpoint{2.587972in}{1.865869in}}%
\pgfpathcurveto{\pgfqpoint{2.587972in}{1.857633in}}{\pgfqpoint{2.591245in}{1.849733in}}{\pgfqpoint{2.597069in}{1.843909in}}%
\pgfpathcurveto{\pgfqpoint{2.602892in}{1.838085in}}{\pgfqpoint{2.610793in}{1.834813in}}{\pgfqpoint{2.619029in}{1.834813in}}%
\pgfpathclose%
\pgfusepath{stroke,fill}%
\end{pgfscope}%
\begin{pgfscope}%
\pgfpathrectangle{\pgfqpoint{0.100000in}{0.212622in}}{\pgfqpoint{3.696000in}{3.696000in}}%
\pgfusepath{clip}%
\pgfsetbuttcap%
\pgfsetroundjoin%
\definecolor{currentfill}{rgb}{0.121569,0.466667,0.705882}%
\pgfsetfillcolor{currentfill}%
\pgfsetfillopacity{0.916731}%
\pgfsetlinewidth{1.003750pt}%
\definecolor{currentstroke}{rgb}{0.121569,0.466667,0.705882}%
\pgfsetstrokecolor{currentstroke}%
\pgfsetstrokeopacity{0.916731}%
\pgfsetdash{}{0pt}%
\pgfpathmoveto{\pgfqpoint{1.502174in}{2.106596in}}%
\pgfpathcurveto{\pgfqpoint{1.510410in}{2.106596in}}{\pgfqpoint{1.518310in}{2.109868in}}{\pgfqpoint{1.524134in}{2.115692in}}%
\pgfpathcurveto{\pgfqpoint{1.529958in}{2.121516in}}{\pgfqpoint{1.533231in}{2.129416in}}{\pgfqpoint{1.533231in}{2.137653in}}%
\pgfpathcurveto{\pgfqpoint{1.533231in}{2.145889in}}{\pgfqpoint{1.529958in}{2.153789in}}{\pgfqpoint{1.524134in}{2.159613in}}%
\pgfpathcurveto{\pgfqpoint{1.518310in}{2.165437in}}{\pgfqpoint{1.510410in}{2.168709in}}{\pgfqpoint{1.502174in}{2.168709in}}%
\pgfpathcurveto{\pgfqpoint{1.493938in}{2.168709in}}{\pgfqpoint{1.486038in}{2.165437in}}{\pgfqpoint{1.480214in}{2.159613in}}%
\pgfpathcurveto{\pgfqpoint{1.474390in}{2.153789in}}{\pgfqpoint{1.471118in}{2.145889in}}{\pgfqpoint{1.471118in}{2.137653in}}%
\pgfpathcurveto{\pgfqpoint{1.471118in}{2.129416in}}{\pgfqpoint{1.474390in}{2.121516in}}{\pgfqpoint{1.480214in}{2.115692in}}%
\pgfpathcurveto{\pgfqpoint{1.486038in}{2.109868in}}{\pgfqpoint{1.493938in}{2.106596in}}{\pgfqpoint{1.502174in}{2.106596in}}%
\pgfpathclose%
\pgfusepath{stroke,fill}%
\end{pgfscope}%
\begin{pgfscope}%
\pgfpathrectangle{\pgfqpoint{0.100000in}{0.212622in}}{\pgfqpoint{3.696000in}{3.696000in}}%
\pgfusepath{clip}%
\pgfsetbuttcap%
\pgfsetroundjoin%
\definecolor{currentfill}{rgb}{0.121569,0.466667,0.705882}%
\pgfsetfillcolor{currentfill}%
\pgfsetfillopacity{0.917461}%
\pgfsetlinewidth{1.003750pt}%
\definecolor{currentstroke}{rgb}{0.121569,0.466667,0.705882}%
\pgfsetstrokecolor{currentstroke}%
\pgfsetstrokeopacity{0.917461}%
\pgfsetdash{}{0pt}%
\pgfpathmoveto{\pgfqpoint{1.509267in}{2.103795in}}%
\pgfpathcurveto{\pgfqpoint{1.517503in}{2.103795in}}{\pgfqpoint{1.525403in}{2.107067in}}{\pgfqpoint{1.531227in}{2.112891in}}%
\pgfpathcurveto{\pgfqpoint{1.537051in}{2.118715in}}{\pgfqpoint{1.540323in}{2.126615in}}{\pgfqpoint{1.540323in}{2.134851in}}%
\pgfpathcurveto{\pgfqpoint{1.540323in}{2.143087in}}{\pgfqpoint{1.537051in}{2.150987in}}{\pgfqpoint{1.531227in}{2.156811in}}%
\pgfpathcurveto{\pgfqpoint{1.525403in}{2.162635in}}{\pgfqpoint{1.517503in}{2.165908in}}{\pgfqpoint{1.509267in}{2.165908in}}%
\pgfpathcurveto{\pgfqpoint{1.501030in}{2.165908in}}{\pgfqpoint{1.493130in}{2.162635in}}{\pgfqpoint{1.487306in}{2.156811in}}%
\pgfpathcurveto{\pgfqpoint{1.481482in}{2.150987in}}{\pgfqpoint{1.478210in}{2.143087in}}{\pgfqpoint{1.478210in}{2.134851in}}%
\pgfpathcurveto{\pgfqpoint{1.478210in}{2.126615in}}{\pgfqpoint{1.481482in}{2.118715in}}{\pgfqpoint{1.487306in}{2.112891in}}%
\pgfpathcurveto{\pgfqpoint{1.493130in}{2.107067in}}{\pgfqpoint{1.501030in}{2.103795in}}{\pgfqpoint{1.509267in}{2.103795in}}%
\pgfpathclose%
\pgfusepath{stroke,fill}%
\end{pgfscope}%
\begin{pgfscope}%
\pgfpathrectangle{\pgfqpoint{0.100000in}{0.212622in}}{\pgfqpoint{3.696000in}{3.696000in}}%
\pgfusepath{clip}%
\pgfsetbuttcap%
\pgfsetroundjoin%
\definecolor{currentfill}{rgb}{0.121569,0.466667,0.705882}%
\pgfsetfillcolor{currentfill}%
\pgfsetfillopacity{0.917923}%
\pgfsetlinewidth{1.003750pt}%
\definecolor{currentstroke}{rgb}{0.121569,0.466667,0.705882}%
\pgfsetstrokecolor{currentstroke}%
\pgfsetstrokeopacity{0.917923}%
\pgfsetdash{}{0pt}%
\pgfpathmoveto{\pgfqpoint{2.616755in}{1.832181in}}%
\pgfpathcurveto{\pgfqpoint{2.624991in}{1.832181in}}{\pgfqpoint{2.632891in}{1.835453in}}{\pgfqpoint{2.638715in}{1.841277in}}%
\pgfpathcurveto{\pgfqpoint{2.644539in}{1.847101in}}{\pgfqpoint{2.647812in}{1.855001in}}{\pgfqpoint{2.647812in}{1.863237in}}%
\pgfpathcurveto{\pgfqpoint{2.647812in}{1.871473in}}{\pgfqpoint{2.644539in}{1.879373in}}{\pgfqpoint{2.638715in}{1.885197in}}%
\pgfpathcurveto{\pgfqpoint{2.632891in}{1.891021in}}{\pgfqpoint{2.624991in}{1.894294in}}{\pgfqpoint{2.616755in}{1.894294in}}%
\pgfpathcurveto{\pgfqpoint{2.608519in}{1.894294in}}{\pgfqpoint{2.600619in}{1.891021in}}{\pgfqpoint{2.594795in}{1.885197in}}%
\pgfpathcurveto{\pgfqpoint{2.588971in}{1.879373in}}{\pgfqpoint{2.585699in}{1.871473in}}{\pgfqpoint{2.585699in}{1.863237in}}%
\pgfpathcurveto{\pgfqpoint{2.585699in}{1.855001in}}{\pgfqpoint{2.588971in}{1.847101in}}{\pgfqpoint{2.594795in}{1.841277in}}%
\pgfpathcurveto{\pgfqpoint{2.600619in}{1.835453in}}{\pgfqpoint{2.608519in}{1.832181in}}{\pgfqpoint{2.616755in}{1.832181in}}%
\pgfpathclose%
\pgfusepath{stroke,fill}%
\end{pgfscope}%
\begin{pgfscope}%
\pgfpathrectangle{\pgfqpoint{0.100000in}{0.212622in}}{\pgfqpoint{3.696000in}{3.696000in}}%
\pgfusepath{clip}%
\pgfsetbuttcap%
\pgfsetroundjoin%
\definecolor{currentfill}{rgb}{0.121569,0.466667,0.705882}%
\pgfsetfillcolor{currentfill}%
\pgfsetfillopacity{0.918847}%
\pgfsetlinewidth{1.003750pt}%
\definecolor{currentstroke}{rgb}{0.121569,0.466667,0.705882}%
\pgfsetstrokecolor{currentstroke}%
\pgfsetstrokeopacity{0.918847}%
\pgfsetdash{}{0pt}%
\pgfpathmoveto{\pgfqpoint{1.522136in}{2.098954in}}%
\pgfpathcurveto{\pgfqpoint{1.530373in}{2.098954in}}{\pgfqpoint{1.538273in}{2.102227in}}{\pgfqpoint{1.544097in}{2.108051in}}%
\pgfpathcurveto{\pgfqpoint{1.549920in}{2.113874in}}{\pgfqpoint{1.553193in}{2.121775in}}{\pgfqpoint{1.553193in}{2.130011in}}%
\pgfpathcurveto{\pgfqpoint{1.553193in}{2.138247in}}{\pgfqpoint{1.549920in}{2.146147in}}{\pgfqpoint{1.544097in}{2.151971in}}%
\pgfpathcurveto{\pgfqpoint{1.538273in}{2.157795in}}{\pgfqpoint{1.530373in}{2.161067in}}{\pgfqpoint{1.522136in}{2.161067in}}%
\pgfpathcurveto{\pgfqpoint{1.513900in}{2.161067in}}{\pgfqpoint{1.506000in}{2.157795in}}{\pgfqpoint{1.500176in}{2.151971in}}%
\pgfpathcurveto{\pgfqpoint{1.494352in}{2.146147in}}{\pgfqpoint{1.491080in}{2.138247in}}{\pgfqpoint{1.491080in}{2.130011in}}%
\pgfpathcurveto{\pgfqpoint{1.491080in}{2.121775in}}{\pgfqpoint{1.494352in}{2.113874in}}{\pgfqpoint{1.500176in}{2.108051in}}%
\pgfpathcurveto{\pgfqpoint{1.506000in}{2.102227in}}{\pgfqpoint{1.513900in}{2.098954in}}{\pgfqpoint{1.522136in}{2.098954in}}%
\pgfpathclose%
\pgfusepath{stroke,fill}%
\end{pgfscope}%
\begin{pgfscope}%
\pgfpathrectangle{\pgfqpoint{0.100000in}{0.212622in}}{\pgfqpoint{3.696000in}{3.696000in}}%
\pgfusepath{clip}%
\pgfsetbuttcap%
\pgfsetroundjoin%
\definecolor{currentfill}{rgb}{0.121569,0.466667,0.705882}%
\pgfsetfillcolor{currentfill}%
\pgfsetfillopacity{0.919543}%
\pgfsetlinewidth{1.003750pt}%
\definecolor{currentstroke}{rgb}{0.121569,0.466667,0.705882}%
\pgfsetstrokecolor{currentstroke}%
\pgfsetstrokeopacity{0.919543}%
\pgfsetdash{}{0pt}%
\pgfpathmoveto{\pgfqpoint{2.613897in}{1.829144in}}%
\pgfpathcurveto{\pgfqpoint{2.622133in}{1.829144in}}{\pgfqpoint{2.630033in}{1.832416in}}{\pgfqpoint{2.635857in}{1.838240in}}%
\pgfpathcurveto{\pgfqpoint{2.641681in}{1.844064in}}{\pgfqpoint{2.644953in}{1.851964in}}{\pgfqpoint{2.644953in}{1.860201in}}%
\pgfpathcurveto{\pgfqpoint{2.644953in}{1.868437in}}{\pgfqpoint{2.641681in}{1.876337in}}{\pgfqpoint{2.635857in}{1.882161in}}%
\pgfpathcurveto{\pgfqpoint{2.630033in}{1.887985in}}{\pgfqpoint{2.622133in}{1.891257in}}{\pgfqpoint{2.613897in}{1.891257in}}%
\pgfpathcurveto{\pgfqpoint{2.605660in}{1.891257in}}{\pgfqpoint{2.597760in}{1.887985in}}{\pgfqpoint{2.591936in}{1.882161in}}%
\pgfpathcurveto{\pgfqpoint{2.586113in}{1.876337in}}{\pgfqpoint{2.582840in}{1.868437in}}{\pgfqpoint{2.582840in}{1.860201in}}%
\pgfpathcurveto{\pgfqpoint{2.582840in}{1.851964in}}{\pgfqpoint{2.586113in}{1.844064in}}{\pgfqpoint{2.591936in}{1.838240in}}%
\pgfpathcurveto{\pgfqpoint{2.597760in}{1.832416in}}{\pgfqpoint{2.605660in}{1.829144in}}{\pgfqpoint{2.613897in}{1.829144in}}%
\pgfpathclose%
\pgfusepath{stroke,fill}%
\end{pgfscope}%
\begin{pgfscope}%
\pgfpathrectangle{\pgfqpoint{0.100000in}{0.212622in}}{\pgfqpoint{3.696000in}{3.696000in}}%
\pgfusepath{clip}%
\pgfsetbuttcap%
\pgfsetroundjoin%
\definecolor{currentfill}{rgb}{0.121569,0.466667,0.705882}%
\pgfsetfillcolor{currentfill}%
\pgfsetfillopacity{0.921175}%
\pgfsetlinewidth{1.003750pt}%
\definecolor{currentstroke}{rgb}{0.121569,0.466667,0.705882}%
\pgfsetstrokecolor{currentstroke}%
\pgfsetstrokeopacity{0.921175}%
\pgfsetdash{}{0pt}%
\pgfpathmoveto{\pgfqpoint{2.611032in}{1.824933in}}%
\pgfpathcurveto{\pgfqpoint{2.619268in}{1.824933in}}{\pgfqpoint{2.627168in}{1.828205in}}{\pgfqpoint{2.632992in}{1.834029in}}%
\pgfpathcurveto{\pgfqpoint{2.638816in}{1.839853in}}{\pgfqpoint{2.642088in}{1.847753in}}{\pgfqpoint{2.642088in}{1.855989in}}%
\pgfpathcurveto{\pgfqpoint{2.642088in}{1.864226in}}{\pgfqpoint{2.638816in}{1.872126in}}{\pgfqpoint{2.632992in}{1.877949in}}%
\pgfpathcurveto{\pgfqpoint{2.627168in}{1.883773in}}{\pgfqpoint{2.619268in}{1.887046in}}{\pgfqpoint{2.611032in}{1.887046in}}%
\pgfpathcurveto{\pgfqpoint{2.602795in}{1.887046in}}{\pgfqpoint{2.594895in}{1.883773in}}{\pgfqpoint{2.589071in}{1.877949in}}%
\pgfpathcurveto{\pgfqpoint{2.583247in}{1.872126in}}{\pgfqpoint{2.579975in}{1.864226in}}{\pgfqpoint{2.579975in}{1.855989in}}%
\pgfpathcurveto{\pgfqpoint{2.579975in}{1.847753in}}{\pgfqpoint{2.583247in}{1.839853in}}{\pgfqpoint{2.589071in}{1.834029in}}%
\pgfpathcurveto{\pgfqpoint{2.594895in}{1.828205in}}{\pgfqpoint{2.602795in}{1.824933in}}{\pgfqpoint{2.611032in}{1.824933in}}%
\pgfpathclose%
\pgfusepath{stroke,fill}%
\end{pgfscope}%
\begin{pgfscope}%
\pgfpathrectangle{\pgfqpoint{0.100000in}{0.212622in}}{\pgfqpoint{3.696000in}{3.696000in}}%
\pgfusepath{clip}%
\pgfsetbuttcap%
\pgfsetroundjoin%
\definecolor{currentfill}{rgb}{0.121569,0.466667,0.705882}%
\pgfsetfillcolor{currentfill}%
\pgfsetfillopacity{0.921367}%
\pgfsetlinewidth{1.003750pt}%
\definecolor{currentstroke}{rgb}{0.121569,0.466667,0.705882}%
\pgfsetstrokecolor{currentstroke}%
\pgfsetstrokeopacity{0.921367}%
\pgfsetdash{}{0pt}%
\pgfpathmoveto{\pgfqpoint{1.545687in}{2.090680in}}%
\pgfpathcurveto{\pgfqpoint{1.553924in}{2.090680in}}{\pgfqpoint{1.561824in}{2.093953in}}{\pgfqpoint{1.567648in}{2.099777in}}%
\pgfpathcurveto{\pgfqpoint{1.573472in}{2.105601in}}{\pgfqpoint{1.576744in}{2.113501in}}{\pgfqpoint{1.576744in}{2.121737in}}%
\pgfpathcurveto{\pgfqpoint{1.576744in}{2.129973in}}{\pgfqpoint{1.573472in}{2.137873in}}{\pgfqpoint{1.567648in}{2.143697in}}%
\pgfpathcurveto{\pgfqpoint{1.561824in}{2.149521in}}{\pgfqpoint{1.553924in}{2.152793in}}{\pgfqpoint{1.545687in}{2.152793in}}%
\pgfpathcurveto{\pgfqpoint{1.537451in}{2.152793in}}{\pgfqpoint{1.529551in}{2.149521in}}{\pgfqpoint{1.523727in}{2.143697in}}%
\pgfpathcurveto{\pgfqpoint{1.517903in}{2.137873in}}{\pgfqpoint{1.514631in}{2.129973in}}{\pgfqpoint{1.514631in}{2.121737in}}%
\pgfpathcurveto{\pgfqpoint{1.514631in}{2.113501in}}{\pgfqpoint{1.517903in}{2.105601in}}{\pgfqpoint{1.523727in}{2.099777in}}%
\pgfpathcurveto{\pgfqpoint{1.529551in}{2.093953in}}{\pgfqpoint{1.537451in}{2.090680in}}{\pgfqpoint{1.545687in}{2.090680in}}%
\pgfpathclose%
\pgfusepath{stroke,fill}%
\end{pgfscope}%
\begin{pgfscope}%
\pgfpathrectangle{\pgfqpoint{0.100000in}{0.212622in}}{\pgfqpoint{3.696000in}{3.696000in}}%
\pgfusepath{clip}%
\pgfsetbuttcap%
\pgfsetroundjoin%
\definecolor{currentfill}{rgb}{0.121569,0.466667,0.705882}%
\pgfsetfillcolor{currentfill}%
\pgfsetfillopacity{0.923237}%
\pgfsetlinewidth{1.003750pt}%
\definecolor{currentstroke}{rgb}{0.121569,0.466667,0.705882}%
\pgfsetstrokecolor{currentstroke}%
\pgfsetstrokeopacity{0.923237}%
\pgfsetdash{}{0pt}%
\pgfpathmoveto{\pgfqpoint{2.607654in}{1.820494in}}%
\pgfpathcurveto{\pgfqpoint{2.615890in}{1.820494in}}{\pgfqpoint{2.623790in}{1.823767in}}{\pgfqpoint{2.629614in}{1.829591in}}%
\pgfpathcurveto{\pgfqpoint{2.635438in}{1.835415in}}{\pgfqpoint{2.638710in}{1.843315in}}{\pgfqpoint{2.638710in}{1.851551in}}%
\pgfpathcurveto{\pgfqpoint{2.638710in}{1.859787in}}{\pgfqpoint{2.635438in}{1.867687in}}{\pgfqpoint{2.629614in}{1.873511in}}%
\pgfpathcurveto{\pgfqpoint{2.623790in}{1.879335in}}{\pgfqpoint{2.615890in}{1.882607in}}{\pgfqpoint{2.607654in}{1.882607in}}%
\pgfpathcurveto{\pgfqpoint{2.599417in}{1.882607in}}{\pgfqpoint{2.591517in}{1.879335in}}{\pgfqpoint{2.585693in}{1.873511in}}%
\pgfpathcurveto{\pgfqpoint{2.579869in}{1.867687in}}{\pgfqpoint{2.576597in}{1.859787in}}{\pgfqpoint{2.576597in}{1.851551in}}%
\pgfpathcurveto{\pgfqpoint{2.576597in}{1.843315in}}{\pgfqpoint{2.579869in}{1.835415in}}{\pgfqpoint{2.585693in}{1.829591in}}%
\pgfpathcurveto{\pgfqpoint{2.591517in}{1.823767in}}{\pgfqpoint{2.599417in}{1.820494in}}{\pgfqpoint{2.607654in}{1.820494in}}%
\pgfpathclose%
\pgfusepath{stroke,fill}%
\end{pgfscope}%
\begin{pgfscope}%
\pgfpathrectangle{\pgfqpoint{0.100000in}{0.212622in}}{\pgfqpoint{3.696000in}{3.696000in}}%
\pgfusepath{clip}%
\pgfsetbuttcap%
\pgfsetroundjoin%
\definecolor{currentfill}{rgb}{0.121569,0.466667,0.705882}%
\pgfsetfillcolor{currentfill}%
\pgfsetfillopacity{0.923519}%
\pgfsetlinewidth{1.003750pt}%
\definecolor{currentstroke}{rgb}{0.121569,0.466667,0.705882}%
\pgfsetstrokecolor{currentstroke}%
\pgfsetstrokeopacity{0.923519}%
\pgfsetdash{}{0pt}%
\pgfpathmoveto{\pgfqpoint{1.567866in}{2.081495in}}%
\pgfpathcurveto{\pgfqpoint{1.576102in}{2.081495in}}{\pgfqpoint{1.584002in}{2.084768in}}{\pgfqpoint{1.589826in}{2.090592in}}%
\pgfpathcurveto{\pgfqpoint{1.595650in}{2.096415in}}{\pgfqpoint{1.598922in}{2.104315in}}{\pgfqpoint{1.598922in}{2.112552in}}%
\pgfpathcurveto{\pgfqpoint{1.598922in}{2.120788in}}{\pgfqpoint{1.595650in}{2.128688in}}{\pgfqpoint{1.589826in}{2.134512in}}%
\pgfpathcurveto{\pgfqpoint{1.584002in}{2.140336in}}{\pgfqpoint{1.576102in}{2.143608in}}{\pgfqpoint{1.567866in}{2.143608in}}%
\pgfpathcurveto{\pgfqpoint{1.559630in}{2.143608in}}{\pgfqpoint{1.551730in}{2.140336in}}{\pgfqpoint{1.545906in}{2.134512in}}%
\pgfpathcurveto{\pgfqpoint{1.540082in}{2.128688in}}{\pgfqpoint{1.536809in}{2.120788in}}{\pgfqpoint{1.536809in}{2.112552in}}%
\pgfpathcurveto{\pgfqpoint{1.536809in}{2.104315in}}{\pgfqpoint{1.540082in}{2.096415in}}{\pgfqpoint{1.545906in}{2.090592in}}%
\pgfpathcurveto{\pgfqpoint{1.551730in}{2.084768in}}{\pgfqpoint{1.559630in}{2.081495in}}{\pgfqpoint{1.567866in}{2.081495in}}%
\pgfpathclose%
\pgfusepath{stroke,fill}%
\end{pgfscope}%
\begin{pgfscope}%
\pgfpathrectangle{\pgfqpoint{0.100000in}{0.212622in}}{\pgfqpoint{3.696000in}{3.696000in}}%
\pgfusepath{clip}%
\pgfsetbuttcap%
\pgfsetroundjoin%
\definecolor{currentfill}{rgb}{0.121569,0.466667,0.705882}%
\pgfsetfillcolor{currentfill}%
\pgfsetfillopacity{0.924727}%
\pgfsetlinewidth{1.003750pt}%
\definecolor{currentstroke}{rgb}{0.121569,0.466667,0.705882}%
\pgfsetstrokecolor{currentstroke}%
\pgfsetstrokeopacity{0.924727}%
\pgfsetdash{}{0pt}%
\pgfpathmoveto{\pgfqpoint{1.585477in}{2.070685in}}%
\pgfpathcurveto{\pgfqpoint{1.593713in}{2.070685in}}{\pgfqpoint{1.601613in}{2.073958in}}{\pgfqpoint{1.607437in}{2.079782in}}%
\pgfpathcurveto{\pgfqpoint{1.613261in}{2.085606in}}{\pgfqpoint{1.616533in}{2.093506in}}{\pgfqpoint{1.616533in}{2.101742in}}%
\pgfpathcurveto{\pgfqpoint{1.616533in}{2.109978in}}{\pgfqpoint{1.613261in}{2.117878in}}{\pgfqpoint{1.607437in}{2.123702in}}%
\pgfpathcurveto{\pgfqpoint{1.601613in}{2.129526in}}{\pgfqpoint{1.593713in}{2.132798in}}{\pgfqpoint{1.585477in}{2.132798in}}%
\pgfpathcurveto{\pgfqpoint{1.577240in}{2.132798in}}{\pgfqpoint{1.569340in}{2.129526in}}{\pgfqpoint{1.563516in}{2.123702in}}%
\pgfpathcurveto{\pgfqpoint{1.557692in}{2.117878in}}{\pgfqpoint{1.554420in}{2.109978in}}{\pgfqpoint{1.554420in}{2.101742in}}%
\pgfpathcurveto{\pgfqpoint{1.554420in}{2.093506in}}{\pgfqpoint{1.557692in}{2.085606in}}{\pgfqpoint{1.563516in}{2.079782in}}%
\pgfpathcurveto{\pgfqpoint{1.569340in}{2.073958in}}{\pgfqpoint{1.577240in}{2.070685in}}{\pgfqpoint{1.585477in}{2.070685in}}%
\pgfpathclose%
\pgfusepath{stroke,fill}%
\end{pgfscope}%
\begin{pgfscope}%
\pgfpathrectangle{\pgfqpoint{0.100000in}{0.212622in}}{\pgfqpoint{3.696000in}{3.696000in}}%
\pgfusepath{clip}%
\pgfsetbuttcap%
\pgfsetroundjoin%
\definecolor{currentfill}{rgb}{0.121569,0.466667,0.705882}%
\pgfsetfillcolor{currentfill}%
\pgfsetfillopacity{0.925445}%
\pgfsetlinewidth{1.003750pt}%
\definecolor{currentstroke}{rgb}{0.121569,0.466667,0.705882}%
\pgfsetstrokecolor{currentstroke}%
\pgfsetstrokeopacity{0.925445}%
\pgfsetdash{}{0pt}%
\pgfpathmoveto{\pgfqpoint{2.603735in}{1.815864in}}%
\pgfpathcurveto{\pgfqpoint{2.611971in}{1.815864in}}{\pgfqpoint{2.619871in}{1.819137in}}{\pgfqpoint{2.625695in}{1.824961in}}%
\pgfpathcurveto{\pgfqpoint{2.631519in}{1.830785in}}{\pgfqpoint{2.634792in}{1.838685in}}{\pgfqpoint{2.634792in}{1.846921in}}%
\pgfpathcurveto{\pgfqpoint{2.634792in}{1.855157in}}{\pgfqpoint{2.631519in}{1.863057in}}{\pgfqpoint{2.625695in}{1.868881in}}%
\pgfpathcurveto{\pgfqpoint{2.619871in}{1.874705in}}{\pgfqpoint{2.611971in}{1.877977in}}{\pgfqpoint{2.603735in}{1.877977in}}%
\pgfpathcurveto{\pgfqpoint{2.595499in}{1.877977in}}{\pgfqpoint{2.587599in}{1.874705in}}{\pgfqpoint{2.581775in}{1.868881in}}%
\pgfpathcurveto{\pgfqpoint{2.575951in}{1.863057in}}{\pgfqpoint{2.572679in}{1.855157in}}{\pgfqpoint{2.572679in}{1.846921in}}%
\pgfpathcurveto{\pgfqpoint{2.572679in}{1.838685in}}{\pgfqpoint{2.575951in}{1.830785in}}{\pgfqpoint{2.581775in}{1.824961in}}%
\pgfpathcurveto{\pgfqpoint{2.587599in}{1.819137in}}{\pgfqpoint{2.595499in}{1.815864in}}{\pgfqpoint{2.603735in}{1.815864in}}%
\pgfpathclose%
\pgfusepath{stroke,fill}%
\end{pgfscope}%
\begin{pgfscope}%
\pgfpathrectangle{\pgfqpoint{0.100000in}{0.212622in}}{\pgfqpoint{3.696000in}{3.696000in}}%
\pgfusepath{clip}%
\pgfsetbuttcap%
\pgfsetroundjoin%
\definecolor{currentfill}{rgb}{0.121569,0.466667,0.705882}%
\pgfsetfillcolor{currentfill}%
\pgfsetfillopacity{0.925812}%
\pgfsetlinewidth{1.003750pt}%
\definecolor{currentstroke}{rgb}{0.121569,0.466667,0.705882}%
\pgfsetstrokecolor{currentstroke}%
\pgfsetstrokeopacity{0.925812}%
\pgfsetdash{}{0pt}%
\pgfpathmoveto{\pgfqpoint{1.601707in}{2.060710in}}%
\pgfpathcurveto{\pgfqpoint{1.609944in}{2.060710in}}{\pgfqpoint{1.617844in}{2.063982in}}{\pgfqpoint{1.623668in}{2.069806in}}%
\pgfpathcurveto{\pgfqpoint{1.629492in}{2.075630in}}{\pgfqpoint{1.632764in}{2.083530in}}{\pgfqpoint{1.632764in}{2.091767in}}%
\pgfpathcurveto{\pgfqpoint{1.632764in}{2.100003in}}{\pgfqpoint{1.629492in}{2.107903in}}{\pgfqpoint{1.623668in}{2.113727in}}%
\pgfpathcurveto{\pgfqpoint{1.617844in}{2.119551in}}{\pgfqpoint{1.609944in}{2.122823in}}{\pgfqpoint{1.601707in}{2.122823in}}%
\pgfpathcurveto{\pgfqpoint{1.593471in}{2.122823in}}{\pgfqpoint{1.585571in}{2.119551in}}{\pgfqpoint{1.579747in}{2.113727in}}%
\pgfpathcurveto{\pgfqpoint{1.573923in}{2.107903in}}{\pgfqpoint{1.570651in}{2.100003in}}{\pgfqpoint{1.570651in}{2.091767in}}%
\pgfpathcurveto{\pgfqpoint{1.570651in}{2.083530in}}{\pgfqpoint{1.573923in}{2.075630in}}{\pgfqpoint{1.579747in}{2.069806in}}%
\pgfpathcurveto{\pgfqpoint{1.585571in}{2.063982in}}{\pgfqpoint{1.593471in}{2.060710in}}{\pgfqpoint{1.601707in}{2.060710in}}%
\pgfpathclose%
\pgfusepath{stroke,fill}%
\end{pgfscope}%
\begin{pgfscope}%
\pgfpathrectangle{\pgfqpoint{0.100000in}{0.212622in}}{\pgfqpoint{3.696000in}{3.696000in}}%
\pgfusepath{clip}%
\pgfsetbuttcap%
\pgfsetroundjoin%
\definecolor{currentfill}{rgb}{0.121569,0.466667,0.705882}%
\pgfsetfillcolor{currentfill}%
\pgfsetfillopacity{0.927585}%
\pgfsetlinewidth{1.003750pt}%
\definecolor{currentstroke}{rgb}{0.121569,0.466667,0.705882}%
\pgfsetstrokecolor{currentstroke}%
\pgfsetstrokeopacity{0.927585}%
\pgfsetdash{}{0pt}%
\pgfpathmoveto{\pgfqpoint{1.631091in}{2.040717in}}%
\pgfpathcurveto{\pgfqpoint{1.639327in}{2.040717in}}{\pgfqpoint{1.647227in}{2.043989in}}{\pgfqpoint{1.653051in}{2.049813in}}%
\pgfpathcurveto{\pgfqpoint{1.658875in}{2.055637in}}{\pgfqpoint{1.662148in}{2.063537in}}{\pgfqpoint{1.662148in}{2.071774in}}%
\pgfpathcurveto{\pgfqpoint{1.662148in}{2.080010in}}{\pgfqpoint{1.658875in}{2.087910in}}{\pgfqpoint{1.653051in}{2.093734in}}%
\pgfpathcurveto{\pgfqpoint{1.647227in}{2.099558in}}{\pgfqpoint{1.639327in}{2.102830in}}{\pgfqpoint{1.631091in}{2.102830in}}%
\pgfpathcurveto{\pgfqpoint{1.622855in}{2.102830in}}{\pgfqpoint{1.614955in}{2.099558in}}{\pgfqpoint{1.609131in}{2.093734in}}%
\pgfpathcurveto{\pgfqpoint{1.603307in}{2.087910in}}{\pgfqpoint{1.600035in}{2.080010in}}{\pgfqpoint{1.600035in}{2.071774in}}%
\pgfpathcurveto{\pgfqpoint{1.600035in}{2.063537in}}{\pgfqpoint{1.603307in}{2.055637in}}{\pgfqpoint{1.609131in}{2.049813in}}%
\pgfpathcurveto{\pgfqpoint{1.614955in}{2.043989in}}{\pgfqpoint{1.622855in}{2.040717in}}{\pgfqpoint{1.631091in}{2.040717in}}%
\pgfpathclose%
\pgfusepath{stroke,fill}%
\end{pgfscope}%
\begin{pgfscope}%
\pgfpathrectangle{\pgfqpoint{0.100000in}{0.212622in}}{\pgfqpoint{3.696000in}{3.696000in}}%
\pgfusepath{clip}%
\pgfsetbuttcap%
\pgfsetroundjoin%
\definecolor{currentfill}{rgb}{0.121569,0.466667,0.705882}%
\pgfsetfillcolor{currentfill}%
\pgfsetfillopacity{0.928592}%
\pgfsetlinewidth{1.003750pt}%
\definecolor{currentstroke}{rgb}{0.121569,0.466667,0.705882}%
\pgfsetstrokecolor{currentstroke}%
\pgfsetstrokeopacity{0.928592}%
\pgfsetdash{}{0pt}%
\pgfpathmoveto{\pgfqpoint{2.598693in}{1.810036in}}%
\pgfpathcurveto{\pgfqpoint{2.606929in}{1.810036in}}{\pgfqpoint{2.614829in}{1.813309in}}{\pgfqpoint{2.620653in}{1.819133in}}%
\pgfpathcurveto{\pgfqpoint{2.626477in}{1.824957in}}{\pgfqpoint{2.629749in}{1.832857in}}{\pgfqpoint{2.629749in}{1.841093in}}%
\pgfpathcurveto{\pgfqpoint{2.629749in}{1.849329in}}{\pgfqpoint{2.626477in}{1.857229in}}{\pgfqpoint{2.620653in}{1.863053in}}%
\pgfpathcurveto{\pgfqpoint{2.614829in}{1.868877in}}{\pgfqpoint{2.606929in}{1.872149in}}{\pgfqpoint{2.598693in}{1.872149in}}%
\pgfpathcurveto{\pgfqpoint{2.590457in}{1.872149in}}{\pgfqpoint{2.582557in}{1.868877in}}{\pgfqpoint{2.576733in}{1.863053in}}%
\pgfpathcurveto{\pgfqpoint{2.570909in}{1.857229in}}{\pgfqpoint{2.567636in}{1.849329in}}{\pgfqpoint{2.567636in}{1.841093in}}%
\pgfpathcurveto{\pgfqpoint{2.567636in}{1.832857in}}{\pgfqpoint{2.570909in}{1.824957in}}{\pgfqpoint{2.576733in}{1.819133in}}%
\pgfpathcurveto{\pgfqpoint{2.582557in}{1.813309in}}{\pgfqpoint{2.590457in}{1.810036in}}{\pgfqpoint{2.598693in}{1.810036in}}%
\pgfpathclose%
\pgfusepath{stroke,fill}%
\end{pgfscope}%
\begin{pgfscope}%
\pgfpathrectangle{\pgfqpoint{0.100000in}{0.212622in}}{\pgfqpoint{3.696000in}{3.696000in}}%
\pgfusepath{clip}%
\pgfsetbuttcap%
\pgfsetroundjoin%
\definecolor{currentfill}{rgb}{0.121569,0.466667,0.705882}%
\pgfsetfillcolor{currentfill}%
\pgfsetfillopacity{0.930078}%
\pgfsetlinewidth{1.003750pt}%
\definecolor{currentstroke}{rgb}{0.121569,0.466667,0.705882}%
\pgfsetstrokecolor{currentstroke}%
\pgfsetstrokeopacity{0.930078}%
\pgfsetdash{}{0pt}%
\pgfpathmoveto{\pgfqpoint{1.659326in}{2.025728in}}%
\pgfpathcurveto{\pgfqpoint{1.667562in}{2.025728in}}{\pgfqpoint{1.675462in}{2.029001in}}{\pgfqpoint{1.681286in}{2.034825in}}%
\pgfpathcurveto{\pgfqpoint{1.687110in}{2.040649in}}{\pgfqpoint{1.690383in}{2.048549in}}{\pgfqpoint{1.690383in}{2.056785in}}%
\pgfpathcurveto{\pgfqpoint{1.690383in}{2.065021in}}{\pgfqpoint{1.687110in}{2.072921in}}{\pgfqpoint{1.681286in}{2.078745in}}%
\pgfpathcurveto{\pgfqpoint{1.675462in}{2.084569in}}{\pgfqpoint{1.667562in}{2.087841in}}{\pgfqpoint{1.659326in}{2.087841in}}%
\pgfpathcurveto{\pgfqpoint{1.651090in}{2.087841in}}{\pgfqpoint{1.643190in}{2.084569in}}{\pgfqpoint{1.637366in}{2.078745in}}%
\pgfpathcurveto{\pgfqpoint{1.631542in}{2.072921in}}{\pgfqpoint{1.628270in}{2.065021in}}{\pgfqpoint{1.628270in}{2.056785in}}%
\pgfpathcurveto{\pgfqpoint{1.628270in}{2.048549in}}{\pgfqpoint{1.631542in}{2.040649in}}{\pgfqpoint{1.637366in}{2.034825in}}%
\pgfpathcurveto{\pgfqpoint{1.643190in}{2.029001in}}{\pgfqpoint{1.651090in}{2.025728in}}{\pgfqpoint{1.659326in}{2.025728in}}%
\pgfpathclose%
\pgfusepath{stroke,fill}%
\end{pgfscope}%
\begin{pgfscope}%
\pgfpathrectangle{\pgfqpoint{0.100000in}{0.212622in}}{\pgfqpoint{3.696000in}{3.696000in}}%
\pgfusepath{clip}%
\pgfsetbuttcap%
\pgfsetroundjoin%
\definecolor{currentfill}{rgb}{0.121569,0.466667,0.705882}%
\pgfsetfillcolor{currentfill}%
\pgfsetfillopacity{0.932319}%
\pgfsetlinewidth{1.003750pt}%
\definecolor{currentstroke}{rgb}{0.121569,0.466667,0.705882}%
\pgfsetstrokecolor{currentstroke}%
\pgfsetstrokeopacity{0.932319}%
\pgfsetdash{}{0pt}%
\pgfpathmoveto{\pgfqpoint{2.591568in}{1.802664in}}%
\pgfpathcurveto{\pgfqpoint{2.599804in}{1.802664in}}{\pgfqpoint{2.607704in}{1.805936in}}{\pgfqpoint{2.613528in}{1.811760in}}%
\pgfpathcurveto{\pgfqpoint{2.619352in}{1.817584in}}{\pgfqpoint{2.622625in}{1.825484in}}{\pgfqpoint{2.622625in}{1.833721in}}%
\pgfpathcurveto{\pgfqpoint{2.622625in}{1.841957in}}{\pgfqpoint{2.619352in}{1.849857in}}{\pgfqpoint{2.613528in}{1.855681in}}%
\pgfpathcurveto{\pgfqpoint{2.607704in}{1.861505in}}{\pgfqpoint{2.599804in}{1.864777in}}{\pgfqpoint{2.591568in}{1.864777in}}%
\pgfpathcurveto{\pgfqpoint{2.583332in}{1.864777in}}{\pgfqpoint{2.575432in}{1.861505in}}{\pgfqpoint{2.569608in}{1.855681in}}%
\pgfpathcurveto{\pgfqpoint{2.563784in}{1.849857in}}{\pgfqpoint{2.560512in}{1.841957in}}{\pgfqpoint{2.560512in}{1.833721in}}%
\pgfpathcurveto{\pgfqpoint{2.560512in}{1.825484in}}{\pgfqpoint{2.563784in}{1.817584in}}{\pgfqpoint{2.569608in}{1.811760in}}%
\pgfpathcurveto{\pgfqpoint{2.575432in}{1.805936in}}{\pgfqpoint{2.583332in}{1.802664in}}{\pgfqpoint{2.591568in}{1.802664in}}%
\pgfpathclose%
\pgfusepath{stroke,fill}%
\end{pgfscope}%
\begin{pgfscope}%
\pgfpathrectangle{\pgfqpoint{0.100000in}{0.212622in}}{\pgfqpoint{3.696000in}{3.696000in}}%
\pgfusepath{clip}%
\pgfsetbuttcap%
\pgfsetroundjoin%
\definecolor{currentfill}{rgb}{0.121569,0.466667,0.705882}%
\pgfsetfillcolor{currentfill}%
\pgfsetfillopacity{0.933246}%
\pgfsetlinewidth{1.003750pt}%
\definecolor{currentstroke}{rgb}{0.121569,0.466667,0.705882}%
\pgfsetstrokecolor{currentstroke}%
\pgfsetstrokeopacity{0.933246}%
\pgfsetdash{}{0pt}%
\pgfpathmoveto{\pgfqpoint{1.685644in}{2.016907in}}%
\pgfpathcurveto{\pgfqpoint{1.693880in}{2.016907in}}{\pgfqpoint{1.701780in}{2.020179in}}{\pgfqpoint{1.707604in}{2.026003in}}%
\pgfpathcurveto{\pgfqpoint{1.713428in}{2.031827in}}{\pgfqpoint{1.716700in}{2.039727in}}{\pgfqpoint{1.716700in}{2.047963in}}%
\pgfpathcurveto{\pgfqpoint{1.716700in}{2.056200in}}{\pgfqpoint{1.713428in}{2.064100in}}{\pgfqpoint{1.707604in}{2.069924in}}%
\pgfpathcurveto{\pgfqpoint{1.701780in}{2.075748in}}{\pgfqpoint{1.693880in}{2.079020in}}{\pgfqpoint{1.685644in}{2.079020in}}%
\pgfpathcurveto{\pgfqpoint{1.677408in}{2.079020in}}{\pgfqpoint{1.669508in}{2.075748in}}{\pgfqpoint{1.663684in}{2.069924in}}%
\pgfpathcurveto{\pgfqpoint{1.657860in}{2.064100in}}{\pgfqpoint{1.654587in}{2.056200in}}{\pgfqpoint{1.654587in}{2.047963in}}%
\pgfpathcurveto{\pgfqpoint{1.654587in}{2.039727in}}{\pgfqpoint{1.657860in}{2.031827in}}{\pgfqpoint{1.663684in}{2.026003in}}%
\pgfpathcurveto{\pgfqpoint{1.669508in}{2.020179in}}{\pgfqpoint{1.677408in}{2.016907in}}{\pgfqpoint{1.685644in}{2.016907in}}%
\pgfpathclose%
\pgfusepath{stroke,fill}%
\end{pgfscope}%
\begin{pgfscope}%
\pgfpathrectangle{\pgfqpoint{0.100000in}{0.212622in}}{\pgfqpoint{3.696000in}{3.696000in}}%
\pgfusepath{clip}%
\pgfsetbuttcap%
\pgfsetroundjoin%
\definecolor{currentfill}{rgb}{0.121569,0.466667,0.705882}%
\pgfsetfillcolor{currentfill}%
\pgfsetfillopacity{0.935989}%
\pgfsetlinewidth{1.003750pt}%
\definecolor{currentstroke}{rgb}{0.121569,0.466667,0.705882}%
\pgfsetstrokecolor{currentstroke}%
\pgfsetstrokeopacity{0.935989}%
\pgfsetdash{}{0pt}%
\pgfpathmoveto{\pgfqpoint{1.708497in}{2.009026in}}%
\pgfpathcurveto{\pgfqpoint{1.716733in}{2.009026in}}{\pgfqpoint{1.724633in}{2.012298in}}{\pgfqpoint{1.730457in}{2.018122in}}%
\pgfpathcurveto{\pgfqpoint{1.736281in}{2.023946in}}{\pgfqpoint{1.739553in}{2.031846in}}{\pgfqpoint{1.739553in}{2.040082in}}%
\pgfpathcurveto{\pgfqpoint{1.739553in}{2.048318in}}{\pgfqpoint{1.736281in}{2.056219in}}{\pgfqpoint{1.730457in}{2.062042in}}%
\pgfpathcurveto{\pgfqpoint{1.724633in}{2.067866in}}{\pgfqpoint{1.716733in}{2.071139in}}{\pgfqpoint{1.708497in}{2.071139in}}%
\pgfpathcurveto{\pgfqpoint{1.700260in}{2.071139in}}{\pgfqpoint{1.692360in}{2.067866in}}{\pgfqpoint{1.686536in}{2.062042in}}%
\pgfpathcurveto{\pgfqpoint{1.680712in}{2.056219in}}{\pgfqpoint{1.677440in}{2.048318in}}{\pgfqpoint{1.677440in}{2.040082in}}%
\pgfpathcurveto{\pgfqpoint{1.677440in}{2.031846in}}{\pgfqpoint{1.680712in}{2.023946in}}{\pgfqpoint{1.686536in}{2.018122in}}%
\pgfpathcurveto{\pgfqpoint{1.692360in}{2.012298in}}{\pgfqpoint{1.700260in}{2.009026in}}{\pgfqpoint{1.708497in}{2.009026in}}%
\pgfpathclose%
\pgfusepath{stroke,fill}%
\end{pgfscope}%
\begin{pgfscope}%
\pgfpathrectangle{\pgfqpoint{0.100000in}{0.212622in}}{\pgfqpoint{3.696000in}{3.696000in}}%
\pgfusepath{clip}%
\pgfsetbuttcap%
\pgfsetroundjoin%
\definecolor{currentfill}{rgb}{0.121569,0.466667,0.705882}%
\pgfsetfillcolor{currentfill}%
\pgfsetfillopacity{0.936898}%
\pgfsetlinewidth{1.003750pt}%
\definecolor{currentstroke}{rgb}{0.121569,0.466667,0.705882}%
\pgfsetstrokecolor{currentstroke}%
\pgfsetstrokeopacity{0.936898}%
\pgfsetdash{}{0pt}%
\pgfpathmoveto{\pgfqpoint{2.583095in}{1.792763in}}%
\pgfpathcurveto{\pgfqpoint{2.591331in}{1.792763in}}{\pgfqpoint{2.599231in}{1.796035in}}{\pgfqpoint{2.605055in}{1.801859in}}%
\pgfpathcurveto{\pgfqpoint{2.610879in}{1.807683in}}{\pgfqpoint{2.614151in}{1.815583in}}{\pgfqpoint{2.614151in}{1.823820in}}%
\pgfpathcurveto{\pgfqpoint{2.614151in}{1.832056in}}{\pgfqpoint{2.610879in}{1.839956in}}{\pgfqpoint{2.605055in}{1.845780in}}%
\pgfpathcurveto{\pgfqpoint{2.599231in}{1.851604in}}{\pgfqpoint{2.591331in}{1.854876in}}{\pgfqpoint{2.583095in}{1.854876in}}%
\pgfpathcurveto{\pgfqpoint{2.574858in}{1.854876in}}{\pgfqpoint{2.566958in}{1.851604in}}{\pgfqpoint{2.561134in}{1.845780in}}%
\pgfpathcurveto{\pgfqpoint{2.555310in}{1.839956in}}{\pgfqpoint{2.552038in}{1.832056in}}{\pgfqpoint{2.552038in}{1.823820in}}%
\pgfpathcurveto{\pgfqpoint{2.552038in}{1.815583in}}{\pgfqpoint{2.555310in}{1.807683in}}{\pgfqpoint{2.561134in}{1.801859in}}%
\pgfpathcurveto{\pgfqpoint{2.566958in}{1.796035in}}{\pgfqpoint{2.574858in}{1.792763in}}{\pgfqpoint{2.583095in}{1.792763in}}%
\pgfpathclose%
\pgfusepath{stroke,fill}%
\end{pgfscope}%
\begin{pgfscope}%
\pgfpathrectangle{\pgfqpoint{0.100000in}{0.212622in}}{\pgfqpoint{3.696000in}{3.696000in}}%
\pgfusepath{clip}%
\pgfsetbuttcap%
\pgfsetroundjoin%
\definecolor{currentfill}{rgb}{0.121569,0.466667,0.705882}%
\pgfsetfillcolor{currentfill}%
\pgfsetfillopacity{0.938692}%
\pgfsetlinewidth{1.003750pt}%
\definecolor{currentstroke}{rgb}{0.121569,0.466667,0.705882}%
\pgfsetstrokecolor{currentstroke}%
\pgfsetstrokeopacity{0.938692}%
\pgfsetdash{}{0pt}%
\pgfpathmoveto{\pgfqpoint{1.730407in}{2.002295in}}%
\pgfpathcurveto{\pgfqpoint{1.738644in}{2.002295in}}{\pgfqpoint{1.746544in}{2.005567in}}{\pgfqpoint{1.752368in}{2.011391in}}%
\pgfpathcurveto{\pgfqpoint{1.758192in}{2.017215in}}{\pgfqpoint{1.761464in}{2.025115in}}{\pgfqpoint{1.761464in}{2.033351in}}%
\pgfpathcurveto{\pgfqpoint{1.761464in}{2.041587in}}{\pgfqpoint{1.758192in}{2.049487in}}{\pgfqpoint{1.752368in}{2.055311in}}%
\pgfpathcurveto{\pgfqpoint{1.746544in}{2.061135in}}{\pgfqpoint{1.738644in}{2.064408in}}{\pgfqpoint{1.730407in}{2.064408in}}%
\pgfpathcurveto{\pgfqpoint{1.722171in}{2.064408in}}{\pgfqpoint{1.714271in}{2.061135in}}{\pgfqpoint{1.708447in}{2.055311in}}%
\pgfpathcurveto{\pgfqpoint{1.702623in}{2.049487in}}{\pgfqpoint{1.699351in}{2.041587in}}{\pgfqpoint{1.699351in}{2.033351in}}%
\pgfpathcurveto{\pgfqpoint{1.699351in}{2.025115in}}{\pgfqpoint{1.702623in}{2.017215in}}{\pgfqpoint{1.708447in}{2.011391in}}%
\pgfpathcurveto{\pgfqpoint{1.714271in}{2.005567in}}{\pgfqpoint{1.722171in}{2.002295in}}{\pgfqpoint{1.730407in}{2.002295in}}%
\pgfpathclose%
\pgfusepath{stroke,fill}%
\end{pgfscope}%
\begin{pgfscope}%
\pgfpathrectangle{\pgfqpoint{0.100000in}{0.212622in}}{\pgfqpoint{3.696000in}{3.696000in}}%
\pgfusepath{clip}%
\pgfsetbuttcap%
\pgfsetroundjoin%
\definecolor{currentfill}{rgb}{0.121569,0.466667,0.705882}%
\pgfsetfillcolor{currentfill}%
\pgfsetfillopacity{0.940977}%
\pgfsetlinewidth{1.003750pt}%
\definecolor{currentstroke}{rgb}{0.121569,0.466667,0.705882}%
\pgfsetstrokecolor{currentstroke}%
\pgfsetstrokeopacity{0.940977}%
\pgfsetdash{}{0pt}%
\pgfpathmoveto{\pgfqpoint{1.749579in}{1.995501in}}%
\pgfpathcurveto{\pgfqpoint{1.757816in}{1.995501in}}{\pgfqpoint{1.765716in}{1.998774in}}{\pgfqpoint{1.771540in}{2.004598in}}%
\pgfpathcurveto{\pgfqpoint{1.777363in}{2.010422in}}{\pgfqpoint{1.780636in}{2.018322in}}{\pgfqpoint{1.780636in}{2.026558in}}%
\pgfpathcurveto{\pgfqpoint{1.780636in}{2.034794in}}{\pgfqpoint{1.777363in}{2.042694in}}{\pgfqpoint{1.771540in}{2.048518in}}%
\pgfpathcurveto{\pgfqpoint{1.765716in}{2.054342in}}{\pgfqpoint{1.757816in}{2.057614in}}{\pgfqpoint{1.749579in}{2.057614in}}%
\pgfpathcurveto{\pgfqpoint{1.741343in}{2.057614in}}{\pgfqpoint{1.733443in}{2.054342in}}{\pgfqpoint{1.727619in}{2.048518in}}%
\pgfpathcurveto{\pgfqpoint{1.721795in}{2.042694in}}{\pgfqpoint{1.718523in}{2.034794in}}{\pgfqpoint{1.718523in}{2.026558in}}%
\pgfpathcurveto{\pgfqpoint{1.718523in}{2.018322in}}{\pgfqpoint{1.721795in}{2.010422in}}{\pgfqpoint{1.727619in}{2.004598in}}%
\pgfpathcurveto{\pgfqpoint{1.733443in}{1.998774in}}{\pgfqpoint{1.741343in}{1.995501in}}{\pgfqpoint{1.749579in}{1.995501in}}%
\pgfpathclose%
\pgfusepath{stroke,fill}%
\end{pgfscope}%
\begin{pgfscope}%
\pgfpathrectangle{\pgfqpoint{0.100000in}{0.212622in}}{\pgfqpoint{3.696000in}{3.696000in}}%
\pgfusepath{clip}%
\pgfsetbuttcap%
\pgfsetroundjoin%
\definecolor{currentfill}{rgb}{0.121569,0.466667,0.705882}%
\pgfsetfillcolor{currentfill}%
\pgfsetfillopacity{0.941697}%
\pgfsetlinewidth{1.003750pt}%
\definecolor{currentstroke}{rgb}{0.121569,0.466667,0.705882}%
\pgfsetstrokecolor{currentstroke}%
\pgfsetstrokeopacity{0.941697}%
\pgfsetdash{}{0pt}%
\pgfpathmoveto{\pgfqpoint{2.573153in}{1.780900in}}%
\pgfpathcurveto{\pgfqpoint{2.581389in}{1.780900in}}{\pgfqpoint{2.589289in}{1.784172in}}{\pgfqpoint{2.595113in}{1.789996in}}%
\pgfpathcurveto{\pgfqpoint{2.600937in}{1.795820in}}{\pgfqpoint{2.604209in}{1.803720in}}{\pgfqpoint{2.604209in}{1.811956in}}%
\pgfpathcurveto{\pgfqpoint{2.604209in}{1.820192in}}{\pgfqpoint{2.600937in}{1.828092in}}{\pgfqpoint{2.595113in}{1.833916in}}%
\pgfpathcurveto{\pgfqpoint{2.589289in}{1.839740in}}{\pgfqpoint{2.581389in}{1.843013in}}{\pgfqpoint{2.573153in}{1.843013in}}%
\pgfpathcurveto{\pgfqpoint{2.564916in}{1.843013in}}{\pgfqpoint{2.557016in}{1.839740in}}{\pgfqpoint{2.551192in}{1.833916in}}%
\pgfpathcurveto{\pgfqpoint{2.545368in}{1.828092in}}{\pgfqpoint{2.542096in}{1.820192in}}{\pgfqpoint{2.542096in}{1.811956in}}%
\pgfpathcurveto{\pgfqpoint{2.542096in}{1.803720in}}{\pgfqpoint{2.545368in}{1.795820in}}{\pgfqpoint{2.551192in}{1.789996in}}%
\pgfpathcurveto{\pgfqpoint{2.557016in}{1.784172in}}{\pgfqpoint{2.564916in}{1.780900in}}{\pgfqpoint{2.573153in}{1.780900in}}%
\pgfpathclose%
\pgfusepath{stroke,fill}%
\end{pgfscope}%
\begin{pgfscope}%
\pgfpathrectangle{\pgfqpoint{0.100000in}{0.212622in}}{\pgfqpoint{3.696000in}{3.696000in}}%
\pgfusepath{clip}%
\pgfsetbuttcap%
\pgfsetroundjoin%
\definecolor{currentfill}{rgb}{0.121569,0.466667,0.705882}%
\pgfsetfillcolor{currentfill}%
\pgfsetfillopacity{0.942793}%
\pgfsetlinewidth{1.003750pt}%
\definecolor{currentstroke}{rgb}{0.121569,0.466667,0.705882}%
\pgfsetstrokecolor{currentstroke}%
\pgfsetstrokeopacity{0.942793}%
\pgfsetdash{}{0pt}%
\pgfpathmoveto{\pgfqpoint{1.766184in}{1.988597in}}%
\pgfpathcurveto{\pgfqpoint{1.774420in}{1.988597in}}{\pgfqpoint{1.782320in}{1.991869in}}{\pgfqpoint{1.788144in}{1.997693in}}%
\pgfpathcurveto{\pgfqpoint{1.793968in}{2.003517in}}{\pgfqpoint{1.797240in}{2.011417in}}{\pgfqpoint{1.797240in}{2.019654in}}%
\pgfpathcurveto{\pgfqpoint{1.797240in}{2.027890in}}{\pgfqpoint{1.793968in}{2.035790in}}{\pgfqpoint{1.788144in}{2.041614in}}%
\pgfpathcurveto{\pgfqpoint{1.782320in}{2.047438in}}{\pgfqpoint{1.774420in}{2.050710in}}{\pgfqpoint{1.766184in}{2.050710in}}%
\pgfpathcurveto{\pgfqpoint{1.757947in}{2.050710in}}{\pgfqpoint{1.750047in}{2.047438in}}{\pgfqpoint{1.744223in}{2.041614in}}%
\pgfpathcurveto{\pgfqpoint{1.738399in}{2.035790in}}{\pgfqpoint{1.735127in}{2.027890in}}{\pgfqpoint{1.735127in}{2.019654in}}%
\pgfpathcurveto{\pgfqpoint{1.735127in}{2.011417in}}{\pgfqpoint{1.738399in}{2.003517in}}{\pgfqpoint{1.744223in}{1.997693in}}%
\pgfpathcurveto{\pgfqpoint{1.750047in}{1.991869in}}{\pgfqpoint{1.757947in}{1.988597in}}{\pgfqpoint{1.766184in}{1.988597in}}%
\pgfpathclose%
\pgfusepath{stroke,fill}%
\end{pgfscope}%
\begin{pgfscope}%
\pgfpathrectangle{\pgfqpoint{0.100000in}{0.212622in}}{\pgfqpoint{3.696000in}{3.696000in}}%
\pgfusepath{clip}%
\pgfsetbuttcap%
\pgfsetroundjoin%
\definecolor{currentfill}{rgb}{0.121569,0.466667,0.705882}%
\pgfsetfillcolor{currentfill}%
\pgfsetfillopacity{0.944337}%
\pgfsetlinewidth{1.003750pt}%
\definecolor{currentstroke}{rgb}{0.121569,0.466667,0.705882}%
\pgfsetstrokecolor{currentstroke}%
\pgfsetstrokeopacity{0.944337}%
\pgfsetdash{}{0pt}%
\pgfpathmoveto{\pgfqpoint{1.780986in}{1.982335in}}%
\pgfpathcurveto{\pgfqpoint{1.789222in}{1.982335in}}{\pgfqpoint{1.797123in}{1.985607in}}{\pgfqpoint{1.802946in}{1.991431in}}%
\pgfpathcurveto{\pgfqpoint{1.808770in}{1.997255in}}{\pgfqpoint{1.812043in}{2.005155in}}{\pgfqpoint{1.812043in}{2.013392in}}%
\pgfpathcurveto{\pgfqpoint{1.812043in}{2.021628in}}{\pgfqpoint{1.808770in}{2.029528in}}{\pgfqpoint{1.802946in}{2.035352in}}%
\pgfpathcurveto{\pgfqpoint{1.797123in}{2.041176in}}{\pgfqpoint{1.789222in}{2.044448in}}{\pgfqpoint{1.780986in}{2.044448in}}%
\pgfpathcurveto{\pgfqpoint{1.772750in}{2.044448in}}{\pgfqpoint{1.764850in}{2.041176in}}{\pgfqpoint{1.759026in}{2.035352in}}%
\pgfpathcurveto{\pgfqpoint{1.753202in}{2.029528in}}{\pgfqpoint{1.749930in}{2.021628in}}{\pgfqpoint{1.749930in}{2.013392in}}%
\pgfpathcurveto{\pgfqpoint{1.749930in}{2.005155in}}{\pgfqpoint{1.753202in}{1.997255in}}{\pgfqpoint{1.759026in}{1.991431in}}%
\pgfpathcurveto{\pgfqpoint{1.764850in}{1.985607in}}{\pgfqpoint{1.772750in}{1.982335in}}{\pgfqpoint{1.780986in}{1.982335in}}%
\pgfpathclose%
\pgfusepath{stroke,fill}%
\end{pgfscope}%
\begin{pgfscope}%
\pgfpathrectangle{\pgfqpoint{0.100000in}{0.212622in}}{\pgfqpoint{3.696000in}{3.696000in}}%
\pgfusepath{clip}%
\pgfsetbuttcap%
\pgfsetroundjoin%
\definecolor{currentfill}{rgb}{0.121569,0.466667,0.705882}%
\pgfsetfillcolor{currentfill}%
\pgfsetfillopacity{0.946862}%
\pgfsetlinewidth{1.003750pt}%
\definecolor{currentstroke}{rgb}{0.121569,0.466667,0.705882}%
\pgfsetstrokecolor{currentstroke}%
\pgfsetstrokeopacity{0.946862}%
\pgfsetdash{}{0pt}%
\pgfpathmoveto{\pgfqpoint{2.561453in}{1.766537in}}%
\pgfpathcurveto{\pgfqpoint{2.569689in}{1.766537in}}{\pgfqpoint{2.577589in}{1.769809in}}{\pgfqpoint{2.583413in}{1.775633in}}%
\pgfpathcurveto{\pgfqpoint{2.589237in}{1.781457in}}{\pgfqpoint{2.592509in}{1.789357in}}{\pgfqpoint{2.592509in}{1.797594in}}%
\pgfpathcurveto{\pgfqpoint{2.592509in}{1.805830in}}{\pgfqpoint{2.589237in}{1.813730in}}{\pgfqpoint{2.583413in}{1.819554in}}%
\pgfpathcurveto{\pgfqpoint{2.577589in}{1.825378in}}{\pgfqpoint{2.569689in}{1.828650in}}{\pgfqpoint{2.561453in}{1.828650in}}%
\pgfpathcurveto{\pgfqpoint{2.553217in}{1.828650in}}{\pgfqpoint{2.545316in}{1.825378in}}{\pgfqpoint{2.539493in}{1.819554in}}%
\pgfpathcurveto{\pgfqpoint{2.533669in}{1.813730in}}{\pgfqpoint{2.530396in}{1.805830in}}{\pgfqpoint{2.530396in}{1.797594in}}%
\pgfpathcurveto{\pgfqpoint{2.530396in}{1.789357in}}{\pgfqpoint{2.533669in}{1.781457in}}{\pgfqpoint{2.539493in}{1.775633in}}%
\pgfpathcurveto{\pgfqpoint{2.545316in}{1.769809in}}{\pgfqpoint{2.553217in}{1.766537in}}{\pgfqpoint{2.561453in}{1.766537in}}%
\pgfpathclose%
\pgfusepath{stroke,fill}%
\end{pgfscope}%
\begin{pgfscope}%
\pgfpathrectangle{\pgfqpoint{0.100000in}{0.212622in}}{\pgfqpoint{3.696000in}{3.696000in}}%
\pgfusepath{clip}%
\pgfsetbuttcap%
\pgfsetroundjoin%
\definecolor{currentfill}{rgb}{0.121569,0.466667,0.705882}%
\pgfsetfillcolor{currentfill}%
\pgfsetfillopacity{0.947004}%
\pgfsetlinewidth{1.003750pt}%
\definecolor{currentstroke}{rgb}{0.121569,0.466667,0.705882}%
\pgfsetstrokecolor{currentstroke}%
\pgfsetstrokeopacity{0.947004}%
\pgfsetdash{}{0pt}%
\pgfpathmoveto{\pgfqpoint{1.807702in}{1.969247in}}%
\pgfpathcurveto{\pgfqpoint{1.815938in}{1.969247in}}{\pgfqpoint{1.823839in}{1.972520in}}{\pgfqpoint{1.829662in}{1.978344in}}%
\pgfpathcurveto{\pgfqpoint{1.835486in}{1.984167in}}{\pgfqpoint{1.838759in}{1.992068in}}{\pgfqpoint{1.838759in}{2.000304in}}%
\pgfpathcurveto{\pgfqpoint{1.838759in}{2.008540in}}{\pgfqpoint{1.835486in}{2.016440in}}{\pgfqpoint{1.829662in}{2.022264in}}%
\pgfpathcurveto{\pgfqpoint{1.823839in}{2.028088in}}{\pgfqpoint{1.815938in}{2.031360in}}{\pgfqpoint{1.807702in}{2.031360in}}%
\pgfpathcurveto{\pgfqpoint{1.799466in}{2.031360in}}{\pgfqpoint{1.791566in}{2.028088in}}{\pgfqpoint{1.785742in}{2.022264in}}%
\pgfpathcurveto{\pgfqpoint{1.779918in}{2.016440in}}{\pgfqpoint{1.776646in}{2.008540in}}{\pgfqpoint{1.776646in}{2.000304in}}%
\pgfpathcurveto{\pgfqpoint{1.776646in}{1.992068in}}{\pgfqpoint{1.779918in}{1.984167in}}{\pgfqpoint{1.785742in}{1.978344in}}%
\pgfpathcurveto{\pgfqpoint{1.791566in}{1.972520in}}{\pgfqpoint{1.799466in}{1.969247in}}{\pgfqpoint{1.807702in}{1.969247in}}%
\pgfpathclose%
\pgfusepath{stroke,fill}%
\end{pgfscope}%
\begin{pgfscope}%
\pgfpathrectangle{\pgfqpoint{0.100000in}{0.212622in}}{\pgfqpoint{3.696000in}{3.696000in}}%
\pgfusepath{clip}%
\pgfsetbuttcap%
\pgfsetroundjoin%
\definecolor{currentfill}{rgb}{0.121569,0.466667,0.705882}%
\pgfsetfillcolor{currentfill}%
\pgfsetfillopacity{0.949705}%
\pgfsetlinewidth{1.003750pt}%
\definecolor{currentstroke}{rgb}{0.121569,0.466667,0.705882}%
\pgfsetstrokecolor{currentstroke}%
\pgfsetstrokeopacity{0.949705}%
\pgfsetdash{}{0pt}%
\pgfpathmoveto{\pgfqpoint{1.832003in}{1.958328in}}%
\pgfpathcurveto{\pgfqpoint{1.840239in}{1.958328in}}{\pgfqpoint{1.848139in}{1.961600in}}{\pgfqpoint{1.853963in}{1.967424in}}%
\pgfpathcurveto{\pgfqpoint{1.859787in}{1.973248in}}{\pgfqpoint{1.863059in}{1.981148in}}{\pgfqpoint{1.863059in}{1.989384in}}%
\pgfpathcurveto{\pgfqpoint{1.863059in}{1.997620in}}{\pgfqpoint{1.859787in}{2.005520in}}{\pgfqpoint{1.853963in}{2.011344in}}%
\pgfpathcurveto{\pgfqpoint{1.848139in}{2.017168in}}{\pgfqpoint{1.840239in}{2.020441in}}{\pgfqpoint{1.832003in}{2.020441in}}%
\pgfpathcurveto{\pgfqpoint{1.823766in}{2.020441in}}{\pgfqpoint{1.815866in}{2.017168in}}{\pgfqpoint{1.810042in}{2.011344in}}%
\pgfpathcurveto{\pgfqpoint{1.804219in}{2.005520in}}{\pgfqpoint{1.800946in}{1.997620in}}{\pgfqpoint{1.800946in}{1.989384in}}%
\pgfpathcurveto{\pgfqpoint{1.800946in}{1.981148in}}{\pgfqpoint{1.804219in}{1.973248in}}{\pgfqpoint{1.810042in}{1.967424in}}%
\pgfpathcurveto{\pgfqpoint{1.815866in}{1.961600in}}{\pgfqpoint{1.823766in}{1.958328in}}{\pgfqpoint{1.832003in}{1.958328in}}%
\pgfpathclose%
\pgfusepath{stroke,fill}%
\end{pgfscope}%
\begin{pgfscope}%
\pgfpathrectangle{\pgfqpoint{0.100000in}{0.212622in}}{\pgfqpoint{3.696000in}{3.696000in}}%
\pgfusepath{clip}%
\pgfsetbuttcap%
\pgfsetroundjoin%
\definecolor{currentfill}{rgb}{0.121569,0.466667,0.705882}%
\pgfsetfillcolor{currentfill}%
\pgfsetfillopacity{0.949828}%
\pgfsetlinewidth{1.003750pt}%
\definecolor{currentstroke}{rgb}{0.121569,0.466667,0.705882}%
\pgfsetstrokecolor{currentstroke}%
\pgfsetstrokeopacity{0.949828}%
\pgfsetdash{}{0pt}%
\pgfpathmoveto{\pgfqpoint{2.555206in}{1.759167in}}%
\pgfpathcurveto{\pgfqpoint{2.563442in}{1.759167in}}{\pgfqpoint{2.571342in}{1.762439in}}{\pgfqpoint{2.577166in}{1.768263in}}%
\pgfpathcurveto{\pgfqpoint{2.582990in}{1.774087in}}{\pgfqpoint{2.586263in}{1.781987in}}{\pgfqpoint{2.586263in}{1.790224in}}%
\pgfpathcurveto{\pgfqpoint{2.586263in}{1.798460in}}{\pgfqpoint{2.582990in}{1.806360in}}{\pgfqpoint{2.577166in}{1.812184in}}%
\pgfpathcurveto{\pgfqpoint{2.571342in}{1.818008in}}{\pgfqpoint{2.563442in}{1.821280in}}{\pgfqpoint{2.555206in}{1.821280in}}%
\pgfpathcurveto{\pgfqpoint{2.546970in}{1.821280in}}{\pgfqpoint{2.539070in}{1.818008in}}{\pgfqpoint{2.533246in}{1.812184in}}%
\pgfpathcurveto{\pgfqpoint{2.527422in}{1.806360in}}{\pgfqpoint{2.524150in}{1.798460in}}{\pgfqpoint{2.524150in}{1.790224in}}%
\pgfpathcurveto{\pgfqpoint{2.524150in}{1.781987in}}{\pgfqpoint{2.527422in}{1.774087in}}{\pgfqpoint{2.533246in}{1.768263in}}%
\pgfpathcurveto{\pgfqpoint{2.539070in}{1.762439in}}{\pgfqpoint{2.546970in}{1.759167in}}{\pgfqpoint{2.555206in}{1.759167in}}%
\pgfpathclose%
\pgfusepath{stroke,fill}%
\end{pgfscope}%
\begin{pgfscope}%
\pgfpathrectangle{\pgfqpoint{0.100000in}{0.212622in}}{\pgfqpoint{3.696000in}{3.696000in}}%
\pgfusepath{clip}%
\pgfsetbuttcap%
\pgfsetroundjoin%
\definecolor{currentfill}{rgb}{0.121569,0.466667,0.705882}%
\pgfsetfillcolor{currentfill}%
\pgfsetfillopacity{0.951424}%
\pgfsetlinewidth{1.003750pt}%
\definecolor{currentstroke}{rgb}{0.121569,0.466667,0.705882}%
\pgfsetstrokecolor{currentstroke}%
\pgfsetstrokeopacity{0.951424}%
\pgfsetdash{}{0pt}%
\pgfpathmoveto{\pgfqpoint{2.551585in}{1.755077in}}%
\pgfpathcurveto{\pgfqpoint{2.559822in}{1.755077in}}{\pgfqpoint{2.567722in}{1.758350in}}{\pgfqpoint{2.573546in}{1.764174in}}%
\pgfpathcurveto{\pgfqpoint{2.579370in}{1.769998in}}{\pgfqpoint{2.582642in}{1.777898in}}{\pgfqpoint{2.582642in}{1.786134in}}%
\pgfpathcurveto{\pgfqpoint{2.582642in}{1.794370in}}{\pgfqpoint{2.579370in}{1.802270in}}{\pgfqpoint{2.573546in}{1.808094in}}%
\pgfpathcurveto{\pgfqpoint{2.567722in}{1.813918in}}{\pgfqpoint{2.559822in}{1.817190in}}{\pgfqpoint{2.551585in}{1.817190in}}%
\pgfpathcurveto{\pgfqpoint{2.543349in}{1.817190in}}{\pgfqpoint{2.535449in}{1.813918in}}{\pgfqpoint{2.529625in}{1.808094in}}%
\pgfpathcurveto{\pgfqpoint{2.523801in}{1.802270in}}{\pgfqpoint{2.520529in}{1.794370in}}{\pgfqpoint{2.520529in}{1.786134in}}%
\pgfpathcurveto{\pgfqpoint{2.520529in}{1.777898in}}{\pgfqpoint{2.523801in}{1.769998in}}{\pgfqpoint{2.529625in}{1.764174in}}%
\pgfpathcurveto{\pgfqpoint{2.535449in}{1.758350in}}{\pgfqpoint{2.543349in}{1.755077in}}{\pgfqpoint{2.551585in}{1.755077in}}%
\pgfpathclose%
\pgfusepath{stroke,fill}%
\end{pgfscope}%
\begin{pgfscope}%
\pgfpathrectangle{\pgfqpoint{0.100000in}{0.212622in}}{\pgfqpoint{3.696000in}{3.696000in}}%
\pgfusepath{clip}%
\pgfsetbuttcap%
\pgfsetroundjoin%
\definecolor{currentfill}{rgb}{0.121569,0.466667,0.705882}%
\pgfsetfillcolor{currentfill}%
\pgfsetfillopacity{0.951973}%
\pgfsetlinewidth{1.003750pt}%
\definecolor{currentstroke}{rgb}{0.121569,0.466667,0.705882}%
\pgfsetstrokecolor{currentstroke}%
\pgfsetstrokeopacity{0.951973}%
\pgfsetdash{}{0pt}%
\pgfpathmoveto{\pgfqpoint{1.852274in}{1.949915in}}%
\pgfpathcurveto{\pgfqpoint{1.860510in}{1.949915in}}{\pgfqpoint{1.868411in}{1.953187in}}{\pgfqpoint{1.874234in}{1.959011in}}%
\pgfpathcurveto{\pgfqpoint{1.880058in}{1.964835in}}{\pgfqpoint{1.883331in}{1.972735in}}{\pgfqpoint{1.883331in}{1.980972in}}%
\pgfpathcurveto{\pgfqpoint{1.883331in}{1.989208in}}{\pgfqpoint{1.880058in}{1.997108in}}{\pgfqpoint{1.874234in}{2.002932in}}%
\pgfpathcurveto{\pgfqpoint{1.868411in}{2.008756in}}{\pgfqpoint{1.860510in}{2.012028in}}{\pgfqpoint{1.852274in}{2.012028in}}%
\pgfpathcurveto{\pgfqpoint{1.844038in}{2.012028in}}{\pgfqpoint{1.836138in}{2.008756in}}{\pgfqpoint{1.830314in}{2.002932in}}%
\pgfpathcurveto{\pgfqpoint{1.824490in}{1.997108in}}{\pgfqpoint{1.821218in}{1.989208in}}{\pgfqpoint{1.821218in}{1.980972in}}%
\pgfpathcurveto{\pgfqpoint{1.821218in}{1.972735in}}{\pgfqpoint{1.824490in}{1.964835in}}{\pgfqpoint{1.830314in}{1.959011in}}%
\pgfpathcurveto{\pgfqpoint{1.836138in}{1.953187in}}{\pgfqpoint{1.844038in}{1.949915in}}{\pgfqpoint{1.852274in}{1.949915in}}%
\pgfpathclose%
\pgfusepath{stroke,fill}%
\end{pgfscope}%
\begin{pgfscope}%
\pgfpathrectangle{\pgfqpoint{0.100000in}{0.212622in}}{\pgfqpoint{3.696000in}{3.696000in}}%
\pgfusepath{clip}%
\pgfsetbuttcap%
\pgfsetroundjoin%
\definecolor{currentfill}{rgb}{0.121569,0.466667,0.705882}%
\pgfsetfillcolor{currentfill}%
\pgfsetfillopacity{0.953245}%
\pgfsetlinewidth{1.003750pt}%
\definecolor{currentstroke}{rgb}{0.121569,0.466667,0.705882}%
\pgfsetstrokecolor{currentstroke}%
\pgfsetstrokeopacity{0.953245}%
\pgfsetdash{}{0pt}%
\pgfpathmoveto{\pgfqpoint{2.547958in}{1.751115in}}%
\pgfpathcurveto{\pgfqpoint{2.556195in}{1.751115in}}{\pgfqpoint{2.564095in}{1.754388in}}{\pgfqpoint{2.569919in}{1.760212in}}%
\pgfpathcurveto{\pgfqpoint{2.575743in}{1.766036in}}{\pgfqpoint{2.579015in}{1.773936in}}{\pgfqpoint{2.579015in}{1.782172in}}%
\pgfpathcurveto{\pgfqpoint{2.579015in}{1.790408in}}{\pgfqpoint{2.575743in}{1.798308in}}{\pgfqpoint{2.569919in}{1.804132in}}%
\pgfpathcurveto{\pgfqpoint{2.564095in}{1.809956in}}{\pgfqpoint{2.556195in}{1.813228in}}{\pgfqpoint{2.547958in}{1.813228in}}%
\pgfpathcurveto{\pgfqpoint{2.539722in}{1.813228in}}{\pgfqpoint{2.531822in}{1.809956in}}{\pgfqpoint{2.525998in}{1.804132in}}%
\pgfpathcurveto{\pgfqpoint{2.520174in}{1.798308in}}{\pgfqpoint{2.516902in}{1.790408in}}{\pgfqpoint{2.516902in}{1.782172in}}%
\pgfpathcurveto{\pgfqpoint{2.516902in}{1.773936in}}{\pgfqpoint{2.520174in}{1.766036in}}{\pgfqpoint{2.525998in}{1.760212in}}%
\pgfpathcurveto{\pgfqpoint{2.531822in}{1.754388in}}{\pgfqpoint{2.539722in}{1.751115in}}{\pgfqpoint{2.547958in}{1.751115in}}%
\pgfpathclose%
\pgfusepath{stroke,fill}%
\end{pgfscope}%
\begin{pgfscope}%
\pgfpathrectangle{\pgfqpoint{0.100000in}{0.212622in}}{\pgfqpoint{3.696000in}{3.696000in}}%
\pgfusepath{clip}%
\pgfsetbuttcap%
\pgfsetroundjoin%
\definecolor{currentfill}{rgb}{0.121569,0.466667,0.705882}%
\pgfsetfillcolor{currentfill}%
\pgfsetfillopacity{0.954197}%
\pgfsetlinewidth{1.003750pt}%
\definecolor{currentstroke}{rgb}{0.121569,0.466667,0.705882}%
\pgfsetstrokecolor{currentstroke}%
\pgfsetstrokeopacity{0.954197}%
\pgfsetdash{}{0pt}%
\pgfpathmoveto{\pgfqpoint{1.871449in}{1.941687in}}%
\pgfpathcurveto{\pgfqpoint{1.879685in}{1.941687in}}{\pgfqpoint{1.887586in}{1.944959in}}{\pgfqpoint{1.893409in}{1.950783in}}%
\pgfpathcurveto{\pgfqpoint{1.899233in}{1.956607in}}{\pgfqpoint{1.902506in}{1.964507in}}{\pgfqpoint{1.902506in}{1.972743in}}%
\pgfpathcurveto{\pgfqpoint{1.902506in}{1.980980in}}{\pgfqpoint{1.899233in}{1.988880in}}{\pgfqpoint{1.893409in}{1.994703in}}%
\pgfpathcurveto{\pgfqpoint{1.887586in}{2.000527in}}{\pgfqpoint{1.879685in}{2.003800in}}{\pgfqpoint{1.871449in}{2.003800in}}%
\pgfpathcurveto{\pgfqpoint{1.863213in}{2.003800in}}{\pgfqpoint{1.855313in}{2.000527in}}{\pgfqpoint{1.849489in}{1.994703in}}%
\pgfpathcurveto{\pgfqpoint{1.843665in}{1.988880in}}{\pgfqpoint{1.840393in}{1.980980in}}{\pgfqpoint{1.840393in}{1.972743in}}%
\pgfpathcurveto{\pgfqpoint{1.840393in}{1.964507in}}{\pgfqpoint{1.843665in}{1.956607in}}{\pgfqpoint{1.849489in}{1.950783in}}%
\pgfpathcurveto{\pgfqpoint{1.855313in}{1.944959in}}{\pgfqpoint{1.863213in}{1.941687in}}{\pgfqpoint{1.871449in}{1.941687in}}%
\pgfpathclose%
\pgfusepath{stroke,fill}%
\end{pgfscope}%
\begin{pgfscope}%
\pgfpathrectangle{\pgfqpoint{0.100000in}{0.212622in}}{\pgfqpoint{3.696000in}{3.696000in}}%
\pgfusepath{clip}%
\pgfsetbuttcap%
\pgfsetroundjoin%
\definecolor{currentfill}{rgb}{0.121569,0.466667,0.705882}%
\pgfsetfillcolor{currentfill}%
\pgfsetfillopacity{0.956197}%
\pgfsetlinewidth{1.003750pt}%
\definecolor{currentstroke}{rgb}{0.121569,0.466667,0.705882}%
\pgfsetstrokecolor{currentstroke}%
\pgfsetstrokeopacity{0.956197}%
\pgfsetdash{}{0pt}%
\pgfpathmoveto{\pgfqpoint{2.541333in}{1.744204in}}%
\pgfpathcurveto{\pgfqpoint{2.549569in}{1.744204in}}{\pgfqpoint{2.557469in}{1.747476in}}{\pgfqpoint{2.563293in}{1.753300in}}%
\pgfpathcurveto{\pgfqpoint{2.569117in}{1.759124in}}{\pgfqpoint{2.572389in}{1.767024in}}{\pgfqpoint{2.572389in}{1.775261in}}%
\pgfpathcurveto{\pgfqpoint{2.572389in}{1.783497in}}{\pgfqpoint{2.569117in}{1.791397in}}{\pgfqpoint{2.563293in}{1.797221in}}%
\pgfpathcurveto{\pgfqpoint{2.557469in}{1.803045in}}{\pgfqpoint{2.549569in}{1.806317in}}{\pgfqpoint{2.541333in}{1.806317in}}%
\pgfpathcurveto{\pgfqpoint{2.533097in}{1.806317in}}{\pgfqpoint{2.525197in}{1.803045in}}{\pgfqpoint{2.519373in}{1.797221in}}%
\pgfpathcurveto{\pgfqpoint{2.513549in}{1.791397in}}{\pgfqpoint{2.510276in}{1.783497in}}{\pgfqpoint{2.510276in}{1.775261in}}%
\pgfpathcurveto{\pgfqpoint{2.510276in}{1.767024in}}{\pgfqpoint{2.513549in}{1.759124in}}{\pgfqpoint{2.519373in}{1.753300in}}%
\pgfpathcurveto{\pgfqpoint{2.525197in}{1.747476in}}{\pgfqpoint{2.533097in}{1.744204in}}{\pgfqpoint{2.541333in}{1.744204in}}%
\pgfpathclose%
\pgfusepath{stroke,fill}%
\end{pgfscope}%
\begin{pgfscope}%
\pgfpathrectangle{\pgfqpoint{0.100000in}{0.212622in}}{\pgfqpoint{3.696000in}{3.696000in}}%
\pgfusepath{clip}%
\pgfsetbuttcap%
\pgfsetroundjoin%
\definecolor{currentfill}{rgb}{0.121569,0.466667,0.705882}%
\pgfsetfillcolor{currentfill}%
\pgfsetfillopacity{0.956591}%
\pgfsetlinewidth{1.003750pt}%
\definecolor{currentstroke}{rgb}{0.121569,0.466667,0.705882}%
\pgfsetstrokecolor{currentstroke}%
\pgfsetstrokeopacity{0.956591}%
\pgfsetdash{}{0pt}%
\pgfpathmoveto{\pgfqpoint{1.890184in}{1.935332in}}%
\pgfpathcurveto{\pgfqpoint{1.898421in}{1.935332in}}{\pgfqpoint{1.906321in}{1.938604in}}{\pgfqpoint{1.912145in}{1.944428in}}%
\pgfpathcurveto{\pgfqpoint{1.917969in}{1.950252in}}{\pgfqpoint{1.921241in}{1.958152in}}{\pgfqpoint{1.921241in}{1.966388in}}%
\pgfpathcurveto{\pgfqpoint{1.921241in}{1.974624in}}{\pgfqpoint{1.917969in}{1.982524in}}{\pgfqpoint{1.912145in}{1.988348in}}%
\pgfpathcurveto{\pgfqpoint{1.906321in}{1.994172in}}{\pgfqpoint{1.898421in}{1.997445in}}{\pgfqpoint{1.890184in}{1.997445in}}%
\pgfpathcurveto{\pgfqpoint{1.881948in}{1.997445in}}{\pgfqpoint{1.874048in}{1.994172in}}{\pgfqpoint{1.868224in}{1.988348in}}%
\pgfpathcurveto{\pgfqpoint{1.862400in}{1.982524in}}{\pgfqpoint{1.859128in}{1.974624in}}{\pgfqpoint{1.859128in}{1.966388in}}%
\pgfpathcurveto{\pgfqpoint{1.859128in}{1.958152in}}{\pgfqpoint{1.862400in}{1.950252in}}{\pgfqpoint{1.868224in}{1.944428in}}%
\pgfpathcurveto{\pgfqpoint{1.874048in}{1.938604in}}{\pgfqpoint{1.881948in}{1.935332in}}{\pgfqpoint{1.890184in}{1.935332in}}%
\pgfpathclose%
\pgfusepath{stroke,fill}%
\end{pgfscope}%
\begin{pgfscope}%
\pgfpathrectangle{\pgfqpoint{0.100000in}{0.212622in}}{\pgfqpoint{3.696000in}{3.696000in}}%
\pgfusepath{clip}%
\pgfsetbuttcap%
\pgfsetroundjoin%
\definecolor{currentfill}{rgb}{0.121569,0.466667,0.705882}%
\pgfsetfillcolor{currentfill}%
\pgfsetfillopacity{0.958904}%
\pgfsetlinewidth{1.003750pt}%
\definecolor{currentstroke}{rgb}{0.121569,0.466667,0.705882}%
\pgfsetstrokecolor{currentstroke}%
\pgfsetstrokeopacity{0.958904}%
\pgfsetdash{}{0pt}%
\pgfpathmoveto{\pgfqpoint{1.907255in}{1.930321in}}%
\pgfpathcurveto{\pgfqpoint{1.915491in}{1.930321in}}{\pgfqpoint{1.923391in}{1.933593in}}{\pgfqpoint{1.929215in}{1.939417in}}%
\pgfpathcurveto{\pgfqpoint{1.935039in}{1.945241in}}{\pgfqpoint{1.938311in}{1.953141in}}{\pgfqpoint{1.938311in}{1.961377in}}%
\pgfpathcurveto{\pgfqpoint{1.938311in}{1.969613in}}{\pgfqpoint{1.935039in}{1.977513in}}{\pgfqpoint{1.929215in}{1.983337in}}%
\pgfpathcurveto{\pgfqpoint{1.923391in}{1.989161in}}{\pgfqpoint{1.915491in}{1.992434in}}{\pgfqpoint{1.907255in}{1.992434in}}%
\pgfpathcurveto{\pgfqpoint{1.899018in}{1.992434in}}{\pgfqpoint{1.891118in}{1.989161in}}{\pgfqpoint{1.885294in}{1.983337in}}%
\pgfpathcurveto{\pgfqpoint{1.879471in}{1.977513in}}{\pgfqpoint{1.876198in}{1.969613in}}{\pgfqpoint{1.876198in}{1.961377in}}%
\pgfpathcurveto{\pgfqpoint{1.876198in}{1.953141in}}{\pgfqpoint{1.879471in}{1.945241in}}{\pgfqpoint{1.885294in}{1.939417in}}%
\pgfpathcurveto{\pgfqpoint{1.891118in}{1.933593in}}{\pgfqpoint{1.899018in}{1.930321in}}{\pgfqpoint{1.907255in}{1.930321in}}%
\pgfpathclose%
\pgfusepath{stroke,fill}%
\end{pgfscope}%
\begin{pgfscope}%
\pgfpathrectangle{\pgfqpoint{0.100000in}{0.212622in}}{\pgfqpoint{3.696000in}{3.696000in}}%
\pgfusepath{clip}%
\pgfsetbuttcap%
\pgfsetroundjoin%
\definecolor{currentfill}{rgb}{0.121569,0.466667,0.705882}%
\pgfsetfillcolor{currentfill}%
\pgfsetfillopacity{0.959603}%
\pgfsetlinewidth{1.003750pt}%
\definecolor{currentstroke}{rgb}{0.121569,0.466667,0.705882}%
\pgfsetstrokecolor{currentstroke}%
\pgfsetstrokeopacity{0.959603}%
\pgfsetdash{}{0pt}%
\pgfpathmoveto{\pgfqpoint{2.534271in}{1.736669in}}%
\pgfpathcurveto{\pgfqpoint{2.542507in}{1.736669in}}{\pgfqpoint{2.550407in}{1.739941in}}{\pgfqpoint{2.556231in}{1.745765in}}%
\pgfpathcurveto{\pgfqpoint{2.562055in}{1.751589in}}{\pgfqpoint{2.565328in}{1.759489in}}{\pgfqpoint{2.565328in}{1.767726in}}%
\pgfpathcurveto{\pgfqpoint{2.565328in}{1.775962in}}{\pgfqpoint{2.562055in}{1.783862in}}{\pgfqpoint{2.556231in}{1.789686in}}%
\pgfpathcurveto{\pgfqpoint{2.550407in}{1.795510in}}{\pgfqpoint{2.542507in}{1.798782in}}{\pgfqpoint{2.534271in}{1.798782in}}%
\pgfpathcurveto{\pgfqpoint{2.526035in}{1.798782in}}{\pgfqpoint{2.518135in}{1.795510in}}{\pgfqpoint{2.512311in}{1.789686in}}%
\pgfpathcurveto{\pgfqpoint{2.506487in}{1.783862in}}{\pgfqpoint{2.503215in}{1.775962in}}{\pgfqpoint{2.503215in}{1.767726in}}%
\pgfpathcurveto{\pgfqpoint{2.503215in}{1.759489in}}{\pgfqpoint{2.506487in}{1.751589in}}{\pgfqpoint{2.512311in}{1.745765in}}%
\pgfpathcurveto{\pgfqpoint{2.518135in}{1.739941in}}{\pgfqpoint{2.526035in}{1.736669in}}{\pgfqpoint{2.534271in}{1.736669in}}%
\pgfpathclose%
\pgfusepath{stroke,fill}%
\end{pgfscope}%
\begin{pgfscope}%
\pgfpathrectangle{\pgfqpoint{0.100000in}{0.212622in}}{\pgfqpoint{3.696000in}{3.696000in}}%
\pgfusepath{clip}%
\pgfsetbuttcap%
\pgfsetroundjoin%
\definecolor{currentfill}{rgb}{0.121569,0.466667,0.705882}%
\pgfsetfillcolor{currentfill}%
\pgfsetfillopacity{0.960965}%
\pgfsetlinewidth{1.003750pt}%
\definecolor{currentstroke}{rgb}{0.121569,0.466667,0.705882}%
\pgfsetstrokecolor{currentstroke}%
\pgfsetstrokeopacity{0.960965}%
\pgfsetdash{}{0pt}%
\pgfpathmoveto{\pgfqpoint{1.923233in}{1.925506in}}%
\pgfpathcurveto{\pgfqpoint{1.931470in}{1.925506in}}{\pgfqpoint{1.939370in}{1.928779in}}{\pgfqpoint{1.945194in}{1.934603in}}%
\pgfpathcurveto{\pgfqpoint{1.951017in}{1.940427in}}{\pgfqpoint{1.954290in}{1.948327in}}{\pgfqpoint{1.954290in}{1.956563in}}%
\pgfpathcurveto{\pgfqpoint{1.954290in}{1.964799in}}{\pgfqpoint{1.951017in}{1.972699in}}{\pgfqpoint{1.945194in}{1.978523in}}%
\pgfpathcurveto{\pgfqpoint{1.939370in}{1.984347in}}{\pgfqpoint{1.931470in}{1.987619in}}{\pgfqpoint{1.923233in}{1.987619in}}%
\pgfpathcurveto{\pgfqpoint{1.914997in}{1.987619in}}{\pgfqpoint{1.907097in}{1.984347in}}{\pgfqpoint{1.901273in}{1.978523in}}%
\pgfpathcurveto{\pgfqpoint{1.895449in}{1.972699in}}{\pgfqpoint{1.892177in}{1.964799in}}{\pgfqpoint{1.892177in}{1.956563in}}%
\pgfpathcurveto{\pgfqpoint{1.892177in}{1.948327in}}{\pgfqpoint{1.895449in}{1.940427in}}{\pgfqpoint{1.901273in}{1.934603in}}%
\pgfpathcurveto{\pgfqpoint{1.907097in}{1.928779in}}{\pgfqpoint{1.914997in}{1.925506in}}{\pgfqpoint{1.923233in}{1.925506in}}%
\pgfpathclose%
\pgfusepath{stroke,fill}%
\end{pgfscope}%
\begin{pgfscope}%
\pgfpathrectangle{\pgfqpoint{0.100000in}{0.212622in}}{\pgfqpoint{3.696000in}{3.696000in}}%
\pgfusepath{clip}%
\pgfsetbuttcap%
\pgfsetroundjoin%
\definecolor{currentfill}{rgb}{0.121569,0.466667,0.705882}%
\pgfsetfillcolor{currentfill}%
\pgfsetfillopacity{0.962850}%
\pgfsetlinewidth{1.003750pt}%
\definecolor{currentstroke}{rgb}{0.121569,0.466667,0.705882}%
\pgfsetstrokecolor{currentstroke}%
\pgfsetstrokeopacity{0.962850}%
\pgfsetdash{}{0pt}%
\pgfpathmoveto{\pgfqpoint{1.938029in}{1.921132in}}%
\pgfpathcurveto{\pgfqpoint{1.946265in}{1.921132in}}{\pgfqpoint{1.954165in}{1.924404in}}{\pgfqpoint{1.959989in}{1.930228in}}%
\pgfpathcurveto{\pgfqpoint{1.965813in}{1.936052in}}{\pgfqpoint{1.969085in}{1.943952in}}{\pgfqpoint{1.969085in}{1.952188in}}%
\pgfpathcurveto{\pgfqpoint{1.969085in}{1.960425in}}{\pgfqpoint{1.965813in}{1.968325in}}{\pgfqpoint{1.959989in}{1.974149in}}%
\pgfpathcurveto{\pgfqpoint{1.954165in}{1.979972in}}{\pgfqpoint{1.946265in}{1.983245in}}{\pgfqpoint{1.938029in}{1.983245in}}%
\pgfpathcurveto{\pgfqpoint{1.929793in}{1.983245in}}{\pgfqpoint{1.921892in}{1.979972in}}{\pgfqpoint{1.916069in}{1.974149in}}%
\pgfpathcurveto{\pgfqpoint{1.910245in}{1.968325in}}{\pgfqpoint{1.906972in}{1.960425in}}{\pgfqpoint{1.906972in}{1.952188in}}%
\pgfpathcurveto{\pgfqpoint{1.906972in}{1.943952in}}{\pgfqpoint{1.910245in}{1.936052in}}{\pgfqpoint{1.916069in}{1.930228in}}%
\pgfpathcurveto{\pgfqpoint{1.921892in}{1.924404in}}{\pgfqpoint{1.929793in}{1.921132in}}{\pgfqpoint{1.938029in}{1.921132in}}%
\pgfpathclose%
\pgfusepath{stroke,fill}%
\end{pgfscope}%
\begin{pgfscope}%
\pgfpathrectangle{\pgfqpoint{0.100000in}{0.212622in}}{\pgfqpoint{3.696000in}{3.696000in}}%
\pgfusepath{clip}%
\pgfsetbuttcap%
\pgfsetroundjoin%
\definecolor{currentfill}{rgb}{0.121569,0.466667,0.705882}%
\pgfsetfillcolor{currentfill}%
\pgfsetfillopacity{0.963978}%
\pgfsetlinewidth{1.003750pt}%
\definecolor{currentstroke}{rgb}{0.121569,0.466667,0.705882}%
\pgfsetstrokecolor{currentstroke}%
\pgfsetstrokeopacity{0.963978}%
\pgfsetdash{}{0pt}%
\pgfpathmoveto{\pgfqpoint{2.525489in}{1.729010in}}%
\pgfpathcurveto{\pgfqpoint{2.533725in}{1.729010in}}{\pgfqpoint{2.541626in}{1.732282in}}{\pgfqpoint{2.547449in}{1.738106in}}%
\pgfpathcurveto{\pgfqpoint{2.553273in}{1.743930in}}{\pgfqpoint{2.556546in}{1.751830in}}{\pgfqpoint{2.556546in}{1.760066in}}%
\pgfpathcurveto{\pgfqpoint{2.556546in}{1.768302in}}{\pgfqpoint{2.553273in}{1.776202in}}{\pgfqpoint{2.547449in}{1.782026in}}%
\pgfpathcurveto{\pgfqpoint{2.541626in}{1.787850in}}{\pgfqpoint{2.533725in}{1.791123in}}{\pgfqpoint{2.525489in}{1.791123in}}%
\pgfpathcurveto{\pgfqpoint{2.517253in}{1.791123in}}{\pgfqpoint{2.509353in}{1.787850in}}{\pgfqpoint{2.503529in}{1.782026in}}%
\pgfpathcurveto{\pgfqpoint{2.497705in}{1.776202in}}{\pgfqpoint{2.494433in}{1.768302in}}{\pgfqpoint{2.494433in}{1.760066in}}%
\pgfpathcurveto{\pgfqpoint{2.494433in}{1.751830in}}{\pgfqpoint{2.497705in}{1.743930in}}{\pgfqpoint{2.503529in}{1.738106in}}%
\pgfpathcurveto{\pgfqpoint{2.509353in}{1.732282in}}{\pgfqpoint{2.517253in}{1.729010in}}{\pgfqpoint{2.525489in}{1.729010in}}%
\pgfpathclose%
\pgfusepath{stroke,fill}%
\end{pgfscope}%
\begin{pgfscope}%
\pgfpathrectangle{\pgfqpoint{0.100000in}{0.212622in}}{\pgfqpoint{3.696000in}{3.696000in}}%
\pgfusepath{clip}%
\pgfsetbuttcap%
\pgfsetroundjoin%
\definecolor{currentfill}{rgb}{0.121569,0.466667,0.705882}%
\pgfsetfillcolor{currentfill}%
\pgfsetfillopacity{0.964413}%
\pgfsetlinewidth{1.003750pt}%
\definecolor{currentstroke}{rgb}{0.121569,0.466667,0.705882}%
\pgfsetstrokecolor{currentstroke}%
\pgfsetstrokeopacity{0.964413}%
\pgfsetdash{}{0pt}%
\pgfpathmoveto{\pgfqpoint{1.952080in}{1.916361in}}%
\pgfpathcurveto{\pgfqpoint{1.960317in}{1.916361in}}{\pgfqpoint{1.968217in}{1.919634in}}{\pgfqpoint{1.974041in}{1.925458in}}%
\pgfpathcurveto{\pgfqpoint{1.979865in}{1.931281in}}{\pgfqpoint{1.983137in}{1.939182in}}{\pgfqpoint{1.983137in}{1.947418in}}%
\pgfpathcurveto{\pgfqpoint{1.983137in}{1.955654in}}{\pgfqpoint{1.979865in}{1.963554in}}{\pgfqpoint{1.974041in}{1.969378in}}%
\pgfpathcurveto{\pgfqpoint{1.968217in}{1.975202in}}{\pgfqpoint{1.960317in}{1.978474in}}{\pgfqpoint{1.952080in}{1.978474in}}%
\pgfpathcurveto{\pgfqpoint{1.943844in}{1.978474in}}{\pgfqpoint{1.935944in}{1.975202in}}{\pgfqpoint{1.930120in}{1.969378in}}%
\pgfpathcurveto{\pgfqpoint{1.924296in}{1.963554in}}{\pgfqpoint{1.921024in}{1.955654in}}{\pgfqpoint{1.921024in}{1.947418in}}%
\pgfpathcurveto{\pgfqpoint{1.921024in}{1.939182in}}{\pgfqpoint{1.924296in}{1.931281in}}{\pgfqpoint{1.930120in}{1.925458in}}%
\pgfpathcurveto{\pgfqpoint{1.935944in}{1.919634in}}{\pgfqpoint{1.943844in}{1.916361in}}{\pgfqpoint{1.952080in}{1.916361in}}%
\pgfpathclose%
\pgfusepath{stroke,fill}%
\end{pgfscope}%
\begin{pgfscope}%
\pgfpathrectangle{\pgfqpoint{0.100000in}{0.212622in}}{\pgfqpoint{3.696000in}{3.696000in}}%
\pgfusepath{clip}%
\pgfsetbuttcap%
\pgfsetroundjoin%
\definecolor{currentfill}{rgb}{0.121569,0.466667,0.705882}%
\pgfsetfillcolor{currentfill}%
\pgfsetfillopacity{0.965555}%
\pgfsetlinewidth{1.003750pt}%
\definecolor{currentstroke}{rgb}{0.121569,0.466667,0.705882}%
\pgfsetstrokecolor{currentstroke}%
\pgfsetstrokeopacity{0.965555}%
\pgfsetdash{}{0pt}%
\pgfpathmoveto{\pgfqpoint{1.963699in}{1.911726in}}%
\pgfpathcurveto{\pgfqpoint{1.971935in}{1.911726in}}{\pgfqpoint{1.979835in}{1.914998in}}{\pgfqpoint{1.985659in}{1.920822in}}%
\pgfpathcurveto{\pgfqpoint{1.991483in}{1.926646in}}{\pgfqpoint{1.994755in}{1.934546in}}{\pgfqpoint{1.994755in}{1.942783in}}%
\pgfpathcurveto{\pgfqpoint{1.994755in}{1.951019in}}{\pgfqpoint{1.991483in}{1.958919in}}{\pgfqpoint{1.985659in}{1.964743in}}%
\pgfpathcurveto{\pgfqpoint{1.979835in}{1.970567in}}{\pgfqpoint{1.971935in}{1.973839in}}{\pgfqpoint{1.963699in}{1.973839in}}%
\pgfpathcurveto{\pgfqpoint{1.955462in}{1.973839in}}{\pgfqpoint{1.947562in}{1.970567in}}{\pgfqpoint{1.941738in}{1.964743in}}%
\pgfpathcurveto{\pgfqpoint{1.935914in}{1.958919in}}{\pgfqpoint{1.932642in}{1.951019in}}{\pgfqpoint{1.932642in}{1.942783in}}%
\pgfpathcurveto{\pgfqpoint{1.932642in}{1.934546in}}{\pgfqpoint{1.935914in}{1.926646in}}{\pgfqpoint{1.941738in}{1.920822in}}%
\pgfpathcurveto{\pgfqpoint{1.947562in}{1.914998in}}{\pgfqpoint{1.955462in}{1.911726in}}{\pgfqpoint{1.963699in}{1.911726in}}%
\pgfpathclose%
\pgfusepath{stroke,fill}%
\end{pgfscope}%
\begin{pgfscope}%
\pgfpathrectangle{\pgfqpoint{0.100000in}{0.212622in}}{\pgfqpoint{3.696000in}{3.696000in}}%
\pgfusepath{clip}%
\pgfsetbuttcap%
\pgfsetroundjoin%
\definecolor{currentfill}{rgb}{0.121569,0.466667,0.705882}%
\pgfsetfillcolor{currentfill}%
\pgfsetfillopacity{0.966548}%
\pgfsetlinewidth{1.003750pt}%
\definecolor{currentstroke}{rgb}{0.121569,0.466667,0.705882}%
\pgfsetstrokecolor{currentstroke}%
\pgfsetstrokeopacity{0.966548}%
\pgfsetdash{}{0pt}%
\pgfpathmoveto{\pgfqpoint{1.973118in}{1.908467in}}%
\pgfpathcurveto{\pgfqpoint{1.981354in}{1.908467in}}{\pgfqpoint{1.989254in}{1.911739in}}{\pgfqpoint{1.995078in}{1.917563in}}%
\pgfpathcurveto{\pgfqpoint{2.000902in}{1.923387in}}{\pgfqpoint{2.004174in}{1.931287in}}{\pgfqpoint{2.004174in}{1.939523in}}%
\pgfpathcurveto{\pgfqpoint{2.004174in}{1.947759in}}{\pgfqpoint{2.000902in}{1.955659in}}{\pgfqpoint{1.995078in}{1.961483in}}%
\pgfpathcurveto{\pgfqpoint{1.989254in}{1.967307in}}{\pgfqpoint{1.981354in}{1.970580in}}{\pgfqpoint{1.973118in}{1.970580in}}%
\pgfpathcurveto{\pgfqpoint{1.964882in}{1.970580in}}{\pgfqpoint{1.956982in}{1.967307in}}{\pgfqpoint{1.951158in}{1.961483in}}%
\pgfpathcurveto{\pgfqpoint{1.945334in}{1.955659in}}{\pgfqpoint{1.942061in}{1.947759in}}{\pgfqpoint{1.942061in}{1.939523in}}%
\pgfpathcurveto{\pgfqpoint{1.942061in}{1.931287in}}{\pgfqpoint{1.945334in}{1.923387in}}{\pgfqpoint{1.951158in}{1.917563in}}%
\pgfpathcurveto{\pgfqpoint{1.956982in}{1.911739in}}{\pgfqpoint{1.964882in}{1.908467in}}{\pgfqpoint{1.973118in}{1.908467in}}%
\pgfpathclose%
\pgfusepath{stroke,fill}%
\end{pgfscope}%
\begin{pgfscope}%
\pgfpathrectangle{\pgfqpoint{0.100000in}{0.212622in}}{\pgfqpoint{3.696000in}{3.696000in}}%
\pgfusepath{clip}%
\pgfsetbuttcap%
\pgfsetroundjoin%
\definecolor{currentfill}{rgb}{0.121569,0.466667,0.705882}%
\pgfsetfillcolor{currentfill}%
\pgfsetfillopacity{0.967442}%
\pgfsetlinewidth{1.003750pt}%
\definecolor{currentstroke}{rgb}{0.121569,0.466667,0.705882}%
\pgfsetstrokecolor{currentstroke}%
\pgfsetstrokeopacity{0.967442}%
\pgfsetdash{}{0pt}%
\pgfpathmoveto{\pgfqpoint{1.981852in}{1.904965in}}%
\pgfpathcurveto{\pgfqpoint{1.990088in}{1.904965in}}{\pgfqpoint{1.997988in}{1.908237in}}{\pgfqpoint{2.003812in}{1.914061in}}%
\pgfpathcurveto{\pgfqpoint{2.009636in}{1.919885in}}{\pgfqpoint{2.012908in}{1.927785in}}{\pgfqpoint{2.012908in}{1.936022in}}%
\pgfpathcurveto{\pgfqpoint{2.012908in}{1.944258in}}{\pgfqpoint{2.009636in}{1.952158in}}{\pgfqpoint{2.003812in}{1.957982in}}%
\pgfpathcurveto{\pgfqpoint{1.997988in}{1.963806in}}{\pgfqpoint{1.990088in}{1.967078in}}{\pgfqpoint{1.981852in}{1.967078in}}%
\pgfpathcurveto{\pgfqpoint{1.973616in}{1.967078in}}{\pgfqpoint{1.965715in}{1.963806in}}{\pgfqpoint{1.959892in}{1.957982in}}%
\pgfpathcurveto{\pgfqpoint{1.954068in}{1.952158in}}{\pgfqpoint{1.950795in}{1.944258in}}{\pgfqpoint{1.950795in}{1.936022in}}%
\pgfpathcurveto{\pgfqpoint{1.950795in}{1.927785in}}{\pgfqpoint{1.954068in}{1.919885in}}{\pgfqpoint{1.959892in}{1.914061in}}%
\pgfpathcurveto{\pgfqpoint{1.965715in}{1.908237in}}{\pgfqpoint{1.973616in}{1.904965in}}{\pgfqpoint{1.981852in}{1.904965in}}%
\pgfpathclose%
\pgfusepath{stroke,fill}%
\end{pgfscope}%
\begin{pgfscope}%
\pgfpathrectangle{\pgfqpoint{0.100000in}{0.212622in}}{\pgfqpoint{3.696000in}{3.696000in}}%
\pgfusepath{clip}%
\pgfsetbuttcap%
\pgfsetroundjoin%
\definecolor{currentfill}{rgb}{0.121569,0.466667,0.705882}%
\pgfsetfillcolor{currentfill}%
\pgfsetfillopacity{0.969065}%
\pgfsetlinewidth{1.003750pt}%
\definecolor{currentstroke}{rgb}{0.121569,0.466667,0.705882}%
\pgfsetstrokecolor{currentstroke}%
\pgfsetstrokeopacity{0.969065}%
\pgfsetdash{}{0pt}%
\pgfpathmoveto{\pgfqpoint{2.516429in}{1.723265in}}%
\pgfpathcurveto{\pgfqpoint{2.524666in}{1.723265in}}{\pgfqpoint{2.532566in}{1.726538in}}{\pgfqpoint{2.538390in}{1.732362in}}%
\pgfpathcurveto{\pgfqpoint{2.544214in}{1.738186in}}{\pgfqpoint{2.547486in}{1.746086in}}{\pgfqpoint{2.547486in}{1.754322in}}%
\pgfpathcurveto{\pgfqpoint{2.547486in}{1.762558in}}{\pgfqpoint{2.544214in}{1.770458in}}{\pgfqpoint{2.538390in}{1.776282in}}%
\pgfpathcurveto{\pgfqpoint{2.532566in}{1.782106in}}{\pgfqpoint{2.524666in}{1.785378in}}{\pgfqpoint{2.516429in}{1.785378in}}%
\pgfpathcurveto{\pgfqpoint{2.508193in}{1.785378in}}{\pgfqpoint{2.500293in}{1.782106in}}{\pgfqpoint{2.494469in}{1.776282in}}%
\pgfpathcurveto{\pgfqpoint{2.488645in}{1.770458in}}{\pgfqpoint{2.485373in}{1.762558in}}{\pgfqpoint{2.485373in}{1.754322in}}%
\pgfpathcurveto{\pgfqpoint{2.485373in}{1.746086in}}{\pgfqpoint{2.488645in}{1.738186in}}{\pgfqpoint{2.494469in}{1.732362in}}%
\pgfpathcurveto{\pgfqpoint{2.500293in}{1.726538in}}{\pgfqpoint{2.508193in}{1.723265in}}{\pgfqpoint{2.516429in}{1.723265in}}%
\pgfpathclose%
\pgfusepath{stroke,fill}%
\end{pgfscope}%
\begin{pgfscope}%
\pgfpathrectangle{\pgfqpoint{0.100000in}{0.212622in}}{\pgfqpoint{3.696000in}{3.696000in}}%
\pgfusepath{clip}%
\pgfsetbuttcap%
\pgfsetroundjoin%
\definecolor{currentfill}{rgb}{0.121569,0.466667,0.705882}%
\pgfsetfillcolor{currentfill}%
\pgfsetfillopacity{0.969218}%
\pgfsetlinewidth{1.003750pt}%
\definecolor{currentstroke}{rgb}{0.121569,0.466667,0.705882}%
\pgfsetstrokecolor{currentstroke}%
\pgfsetstrokeopacity{0.969218}%
\pgfsetdash{}{0pt}%
\pgfpathmoveto{\pgfqpoint{1.997861in}{1.900001in}}%
\pgfpathcurveto{\pgfqpoint{2.006097in}{1.900001in}}{\pgfqpoint{2.013997in}{1.903273in}}{\pgfqpoint{2.019821in}{1.909097in}}%
\pgfpathcurveto{\pgfqpoint{2.025645in}{1.914921in}}{\pgfqpoint{2.028917in}{1.922821in}}{\pgfqpoint{2.028917in}{1.931058in}}%
\pgfpathcurveto{\pgfqpoint{2.028917in}{1.939294in}}{\pgfqpoint{2.025645in}{1.947194in}}{\pgfqpoint{2.019821in}{1.953018in}}%
\pgfpathcurveto{\pgfqpoint{2.013997in}{1.958842in}}{\pgfqpoint{2.006097in}{1.962114in}}{\pgfqpoint{1.997861in}{1.962114in}}%
\pgfpathcurveto{\pgfqpoint{1.989624in}{1.962114in}}{\pgfqpoint{1.981724in}{1.958842in}}{\pgfqpoint{1.975900in}{1.953018in}}%
\pgfpathcurveto{\pgfqpoint{1.970076in}{1.947194in}}{\pgfqpoint{1.966804in}{1.939294in}}{\pgfqpoint{1.966804in}{1.931058in}}%
\pgfpathcurveto{\pgfqpoint{1.966804in}{1.922821in}}{\pgfqpoint{1.970076in}{1.914921in}}{\pgfqpoint{1.975900in}{1.909097in}}%
\pgfpathcurveto{\pgfqpoint{1.981724in}{1.903273in}}{\pgfqpoint{1.989624in}{1.900001in}}{\pgfqpoint{1.997861in}{1.900001in}}%
\pgfpathclose%
\pgfusepath{stroke,fill}%
\end{pgfscope}%
\begin{pgfscope}%
\pgfpathrectangle{\pgfqpoint{0.100000in}{0.212622in}}{\pgfqpoint{3.696000in}{3.696000in}}%
\pgfusepath{clip}%
\pgfsetbuttcap%
\pgfsetroundjoin%
\definecolor{currentfill}{rgb}{0.121569,0.466667,0.705882}%
\pgfsetfillcolor{currentfill}%
\pgfsetfillopacity{0.970496}%
\pgfsetlinewidth{1.003750pt}%
\definecolor{currentstroke}{rgb}{0.121569,0.466667,0.705882}%
\pgfsetstrokecolor{currentstroke}%
\pgfsetstrokeopacity{0.970496}%
\pgfsetdash{}{0pt}%
\pgfpathmoveto{\pgfqpoint{2.011117in}{1.893547in}}%
\pgfpathcurveto{\pgfqpoint{2.019353in}{1.893547in}}{\pgfqpoint{2.027253in}{1.896819in}}{\pgfqpoint{2.033077in}{1.902643in}}%
\pgfpathcurveto{\pgfqpoint{2.038901in}{1.908467in}}{\pgfqpoint{2.042173in}{1.916367in}}{\pgfqpoint{2.042173in}{1.924603in}}%
\pgfpathcurveto{\pgfqpoint{2.042173in}{1.932840in}}{\pgfqpoint{2.038901in}{1.940740in}}{\pgfqpoint{2.033077in}{1.946564in}}%
\pgfpathcurveto{\pgfqpoint{2.027253in}{1.952388in}}{\pgfqpoint{2.019353in}{1.955660in}}{\pgfqpoint{2.011117in}{1.955660in}}%
\pgfpathcurveto{\pgfqpoint{2.002881in}{1.955660in}}{\pgfqpoint{1.994981in}{1.952388in}}{\pgfqpoint{1.989157in}{1.946564in}}%
\pgfpathcurveto{\pgfqpoint{1.983333in}{1.940740in}}{\pgfqpoint{1.980060in}{1.932840in}}{\pgfqpoint{1.980060in}{1.924603in}}%
\pgfpathcurveto{\pgfqpoint{1.980060in}{1.916367in}}{\pgfqpoint{1.983333in}{1.908467in}}{\pgfqpoint{1.989157in}{1.902643in}}%
\pgfpathcurveto{\pgfqpoint{1.994981in}{1.896819in}}{\pgfqpoint{2.002881in}{1.893547in}}{\pgfqpoint{2.011117in}{1.893547in}}%
\pgfpathclose%
\pgfusepath{stroke,fill}%
\end{pgfscope}%
\begin{pgfscope}%
\pgfpathrectangle{\pgfqpoint{0.100000in}{0.212622in}}{\pgfqpoint{3.696000in}{3.696000in}}%
\pgfusepath{clip}%
\pgfsetbuttcap%
\pgfsetroundjoin%
\definecolor{currentfill}{rgb}{0.121569,0.466667,0.705882}%
\pgfsetfillcolor{currentfill}%
\pgfsetfillopacity{0.971470}%
\pgfsetlinewidth{1.003750pt}%
\definecolor{currentstroke}{rgb}{0.121569,0.466667,0.705882}%
\pgfsetstrokecolor{currentstroke}%
\pgfsetstrokeopacity{0.971470}%
\pgfsetdash{}{0pt}%
\pgfpathmoveto{\pgfqpoint{2.021488in}{1.888421in}}%
\pgfpathcurveto{\pgfqpoint{2.029724in}{1.888421in}}{\pgfqpoint{2.037624in}{1.891693in}}{\pgfqpoint{2.043448in}{1.897517in}}%
\pgfpathcurveto{\pgfqpoint{2.049272in}{1.903341in}}{\pgfqpoint{2.052544in}{1.911241in}}{\pgfqpoint{2.052544in}{1.919477in}}%
\pgfpathcurveto{\pgfqpoint{2.052544in}{1.927714in}}{\pgfqpoint{2.049272in}{1.935614in}}{\pgfqpoint{2.043448in}{1.941438in}}%
\pgfpathcurveto{\pgfqpoint{2.037624in}{1.947262in}}{\pgfqpoint{2.029724in}{1.950534in}}{\pgfqpoint{2.021488in}{1.950534in}}%
\pgfpathcurveto{\pgfqpoint{2.013252in}{1.950534in}}{\pgfqpoint{2.005352in}{1.947262in}}{\pgfqpoint{1.999528in}{1.941438in}}%
\pgfpathcurveto{\pgfqpoint{1.993704in}{1.935614in}}{\pgfqpoint{1.990431in}{1.927714in}}{\pgfqpoint{1.990431in}{1.919477in}}%
\pgfpathcurveto{\pgfqpoint{1.990431in}{1.911241in}}{\pgfqpoint{1.993704in}{1.903341in}}{\pgfqpoint{1.999528in}{1.897517in}}%
\pgfpathcurveto{\pgfqpoint{2.005352in}{1.891693in}}{\pgfqpoint{2.013252in}{1.888421in}}{\pgfqpoint{2.021488in}{1.888421in}}%
\pgfpathclose%
\pgfusepath{stroke,fill}%
\end{pgfscope}%
\begin{pgfscope}%
\pgfpathrectangle{\pgfqpoint{0.100000in}{0.212622in}}{\pgfqpoint{3.696000in}{3.696000in}}%
\pgfusepath{clip}%
\pgfsetbuttcap%
\pgfsetroundjoin%
\definecolor{currentfill}{rgb}{0.121569,0.466667,0.705882}%
\pgfsetfillcolor{currentfill}%
\pgfsetfillopacity{0.971798}%
\pgfsetlinewidth{1.003750pt}%
\definecolor{currentstroke}{rgb}{0.121569,0.466667,0.705882}%
\pgfsetstrokecolor{currentstroke}%
\pgfsetstrokeopacity{0.971798}%
\pgfsetdash{}{0pt}%
\pgfpathmoveto{\pgfqpoint{2.511490in}{1.719624in}}%
\pgfpathcurveto{\pgfqpoint{2.519726in}{1.719624in}}{\pgfqpoint{2.527626in}{1.722897in}}{\pgfqpoint{2.533450in}{1.728720in}}%
\pgfpathcurveto{\pgfqpoint{2.539274in}{1.734544in}}{\pgfqpoint{2.542547in}{1.742444in}}{\pgfqpoint{2.542547in}{1.750681in}}%
\pgfpathcurveto{\pgfqpoint{2.542547in}{1.758917in}}{\pgfqpoint{2.539274in}{1.766817in}}{\pgfqpoint{2.533450in}{1.772641in}}%
\pgfpathcurveto{\pgfqpoint{2.527626in}{1.778465in}}{\pgfqpoint{2.519726in}{1.781737in}}{\pgfqpoint{2.511490in}{1.781737in}}%
\pgfpathcurveto{\pgfqpoint{2.503254in}{1.781737in}}{\pgfqpoint{2.495354in}{1.778465in}}{\pgfqpoint{2.489530in}{1.772641in}}%
\pgfpathcurveto{\pgfqpoint{2.483706in}{1.766817in}}{\pgfqpoint{2.480434in}{1.758917in}}{\pgfqpoint{2.480434in}{1.750681in}}%
\pgfpathcurveto{\pgfqpoint{2.480434in}{1.742444in}}{\pgfqpoint{2.483706in}{1.734544in}}{\pgfqpoint{2.489530in}{1.728720in}}%
\pgfpathcurveto{\pgfqpoint{2.495354in}{1.722897in}}{\pgfqpoint{2.503254in}{1.719624in}}{\pgfqpoint{2.511490in}{1.719624in}}%
\pgfpathclose%
\pgfusepath{stroke,fill}%
\end{pgfscope}%
\begin{pgfscope}%
\pgfpathrectangle{\pgfqpoint{0.100000in}{0.212622in}}{\pgfqpoint{3.696000in}{3.696000in}}%
\pgfusepath{clip}%
\pgfsetbuttcap%
\pgfsetroundjoin%
\definecolor{currentfill}{rgb}{0.121569,0.466667,0.705882}%
\pgfsetfillcolor{currentfill}%
\pgfsetfillopacity{0.972321}%
\pgfsetlinewidth{1.003750pt}%
\definecolor{currentstroke}{rgb}{0.121569,0.466667,0.705882}%
\pgfsetstrokecolor{currentstroke}%
\pgfsetstrokeopacity{0.972321}%
\pgfsetdash{}{0pt}%
\pgfpathmoveto{\pgfqpoint{2.030648in}{1.884078in}}%
\pgfpathcurveto{\pgfqpoint{2.038885in}{1.884078in}}{\pgfqpoint{2.046785in}{1.887350in}}{\pgfqpoint{2.052609in}{1.893174in}}%
\pgfpathcurveto{\pgfqpoint{2.058433in}{1.898998in}}{\pgfqpoint{2.061705in}{1.906898in}}{\pgfqpoint{2.061705in}{1.915135in}}%
\pgfpathcurveto{\pgfqpoint{2.061705in}{1.923371in}}{\pgfqpoint{2.058433in}{1.931271in}}{\pgfqpoint{2.052609in}{1.937095in}}%
\pgfpathcurveto{\pgfqpoint{2.046785in}{1.942919in}}{\pgfqpoint{2.038885in}{1.946191in}}{\pgfqpoint{2.030648in}{1.946191in}}%
\pgfpathcurveto{\pgfqpoint{2.022412in}{1.946191in}}{\pgfqpoint{2.014512in}{1.942919in}}{\pgfqpoint{2.008688in}{1.937095in}}%
\pgfpathcurveto{\pgfqpoint{2.002864in}{1.931271in}}{\pgfqpoint{1.999592in}{1.923371in}}{\pgfqpoint{1.999592in}{1.915135in}}%
\pgfpathcurveto{\pgfqpoint{1.999592in}{1.906898in}}{\pgfqpoint{2.002864in}{1.898998in}}{\pgfqpoint{2.008688in}{1.893174in}}%
\pgfpathcurveto{\pgfqpoint{2.014512in}{1.887350in}}{\pgfqpoint{2.022412in}{1.884078in}}{\pgfqpoint{2.030648in}{1.884078in}}%
\pgfpathclose%
\pgfusepath{stroke,fill}%
\end{pgfscope}%
\begin{pgfscope}%
\pgfpathrectangle{\pgfqpoint{0.100000in}{0.212622in}}{\pgfqpoint{3.696000in}{3.696000in}}%
\pgfusepath{clip}%
\pgfsetbuttcap%
\pgfsetroundjoin%
\definecolor{currentfill}{rgb}{0.121569,0.466667,0.705882}%
\pgfsetfillcolor{currentfill}%
\pgfsetfillopacity{0.973021}%
\pgfsetlinewidth{1.003750pt}%
\definecolor{currentstroke}{rgb}{0.121569,0.466667,0.705882}%
\pgfsetstrokecolor{currentstroke}%
\pgfsetstrokeopacity{0.973021}%
\pgfsetdash{}{0pt}%
\pgfpathmoveto{\pgfqpoint{2.039237in}{1.879215in}}%
\pgfpathcurveto{\pgfqpoint{2.047473in}{1.879215in}}{\pgfqpoint{2.055373in}{1.882487in}}{\pgfqpoint{2.061197in}{1.888311in}}%
\pgfpathcurveto{\pgfqpoint{2.067021in}{1.894135in}}{\pgfqpoint{2.070293in}{1.902035in}}{\pgfqpoint{2.070293in}{1.910272in}}%
\pgfpathcurveto{\pgfqpoint{2.070293in}{1.918508in}}{\pgfqpoint{2.067021in}{1.926408in}}{\pgfqpoint{2.061197in}{1.932232in}}%
\pgfpathcurveto{\pgfqpoint{2.055373in}{1.938056in}}{\pgfqpoint{2.047473in}{1.941328in}}{\pgfqpoint{2.039237in}{1.941328in}}%
\pgfpathcurveto{\pgfqpoint{2.031000in}{1.941328in}}{\pgfqpoint{2.023100in}{1.938056in}}{\pgfqpoint{2.017276in}{1.932232in}}%
\pgfpathcurveto{\pgfqpoint{2.011452in}{1.926408in}}{\pgfqpoint{2.008180in}{1.918508in}}{\pgfqpoint{2.008180in}{1.910272in}}%
\pgfpathcurveto{\pgfqpoint{2.008180in}{1.902035in}}{\pgfqpoint{2.011452in}{1.894135in}}{\pgfqpoint{2.017276in}{1.888311in}}%
\pgfpathcurveto{\pgfqpoint{2.023100in}{1.882487in}}{\pgfqpoint{2.031000in}{1.879215in}}{\pgfqpoint{2.039237in}{1.879215in}}%
\pgfpathclose%
\pgfusepath{stroke,fill}%
\end{pgfscope}%
\begin{pgfscope}%
\pgfpathrectangle{\pgfqpoint{0.100000in}{0.212622in}}{\pgfqpoint{3.696000in}{3.696000in}}%
\pgfusepath{clip}%
\pgfsetbuttcap%
\pgfsetroundjoin%
\definecolor{currentfill}{rgb}{0.121569,0.466667,0.705882}%
\pgfsetfillcolor{currentfill}%
\pgfsetfillopacity{0.973236}%
\pgfsetlinewidth{1.003750pt}%
\definecolor{currentstroke}{rgb}{0.121569,0.466667,0.705882}%
\pgfsetstrokecolor{currentstroke}%
\pgfsetstrokeopacity{0.973236}%
\pgfsetdash{}{0pt}%
\pgfpathmoveto{\pgfqpoint{2.508649in}{1.717344in}}%
\pgfpathcurveto{\pgfqpoint{2.516885in}{1.717344in}}{\pgfqpoint{2.524785in}{1.720616in}}{\pgfqpoint{2.530609in}{1.726440in}}%
\pgfpathcurveto{\pgfqpoint{2.536433in}{1.732264in}}{\pgfqpoint{2.539706in}{1.740164in}}{\pgfqpoint{2.539706in}{1.748401in}}%
\pgfpathcurveto{\pgfqpoint{2.539706in}{1.756637in}}{\pgfqpoint{2.536433in}{1.764537in}}{\pgfqpoint{2.530609in}{1.770361in}}%
\pgfpathcurveto{\pgfqpoint{2.524785in}{1.776185in}}{\pgfqpoint{2.516885in}{1.779457in}}{\pgfqpoint{2.508649in}{1.779457in}}%
\pgfpathcurveto{\pgfqpoint{2.500413in}{1.779457in}}{\pgfqpoint{2.492513in}{1.776185in}}{\pgfqpoint{2.486689in}{1.770361in}}%
\pgfpathcurveto{\pgfqpoint{2.480865in}{1.764537in}}{\pgfqpoint{2.477593in}{1.756637in}}{\pgfqpoint{2.477593in}{1.748401in}}%
\pgfpathcurveto{\pgfqpoint{2.477593in}{1.740164in}}{\pgfqpoint{2.480865in}{1.732264in}}{\pgfqpoint{2.486689in}{1.726440in}}%
\pgfpathcurveto{\pgfqpoint{2.492513in}{1.720616in}}{\pgfqpoint{2.500413in}{1.717344in}}{\pgfqpoint{2.508649in}{1.717344in}}%
\pgfpathclose%
\pgfusepath{stroke,fill}%
\end{pgfscope}%
\begin{pgfscope}%
\pgfpathrectangle{\pgfqpoint{0.100000in}{0.212622in}}{\pgfqpoint{3.696000in}{3.696000in}}%
\pgfusepath{clip}%
\pgfsetbuttcap%
\pgfsetroundjoin%
\definecolor{currentfill}{rgb}{0.121569,0.466667,0.705882}%
\pgfsetfillcolor{currentfill}%
\pgfsetfillopacity{0.973711}%
\pgfsetlinewidth{1.003750pt}%
\definecolor{currentstroke}{rgb}{0.121569,0.466667,0.705882}%
\pgfsetstrokecolor{currentstroke}%
\pgfsetstrokeopacity{0.973711}%
\pgfsetdash{}{0pt}%
\pgfpathmoveto{\pgfqpoint{2.046843in}{1.875087in}}%
\pgfpathcurveto{\pgfqpoint{2.055079in}{1.875087in}}{\pgfqpoint{2.062979in}{1.878359in}}{\pgfqpoint{2.068803in}{1.884183in}}%
\pgfpathcurveto{\pgfqpoint{2.074627in}{1.890007in}}{\pgfqpoint{2.077899in}{1.897907in}}{\pgfqpoint{2.077899in}{1.906144in}}%
\pgfpathcurveto{\pgfqpoint{2.077899in}{1.914380in}}{\pgfqpoint{2.074627in}{1.922280in}}{\pgfqpoint{2.068803in}{1.928104in}}%
\pgfpathcurveto{\pgfqpoint{2.062979in}{1.933928in}}{\pgfqpoint{2.055079in}{1.937200in}}{\pgfqpoint{2.046843in}{1.937200in}}%
\pgfpathcurveto{\pgfqpoint{2.038606in}{1.937200in}}{\pgfqpoint{2.030706in}{1.933928in}}{\pgfqpoint{2.024882in}{1.928104in}}%
\pgfpathcurveto{\pgfqpoint{2.019058in}{1.922280in}}{\pgfqpoint{2.015786in}{1.914380in}}{\pgfqpoint{2.015786in}{1.906144in}}%
\pgfpathcurveto{\pgfqpoint{2.015786in}{1.897907in}}{\pgfqpoint{2.019058in}{1.890007in}}{\pgfqpoint{2.024882in}{1.884183in}}%
\pgfpathcurveto{\pgfqpoint{2.030706in}{1.878359in}}{\pgfqpoint{2.038606in}{1.875087in}}{\pgfqpoint{2.046843in}{1.875087in}}%
\pgfpathclose%
\pgfusepath{stroke,fill}%
\end{pgfscope}%
\begin{pgfscope}%
\pgfpathrectangle{\pgfqpoint{0.100000in}{0.212622in}}{\pgfqpoint{3.696000in}{3.696000in}}%
\pgfusepath{clip}%
\pgfsetbuttcap%
\pgfsetroundjoin%
\definecolor{currentfill}{rgb}{0.121569,0.466667,0.705882}%
\pgfsetfillcolor{currentfill}%
\pgfsetfillopacity{0.974042}%
\pgfsetlinewidth{1.003750pt}%
\definecolor{currentstroke}{rgb}{0.121569,0.466667,0.705882}%
\pgfsetstrokecolor{currentstroke}%
\pgfsetstrokeopacity{0.974042}%
\pgfsetdash{}{0pt}%
\pgfpathmoveto{\pgfqpoint{2.507180in}{1.716084in}}%
\pgfpathcurveto{\pgfqpoint{2.515416in}{1.716084in}}{\pgfqpoint{2.523316in}{1.719356in}}{\pgfqpoint{2.529140in}{1.725180in}}%
\pgfpathcurveto{\pgfqpoint{2.534964in}{1.731004in}}{\pgfqpoint{2.538236in}{1.738904in}}{\pgfqpoint{2.538236in}{1.747140in}}%
\pgfpathcurveto{\pgfqpoint{2.538236in}{1.755376in}}{\pgfqpoint{2.534964in}{1.763277in}}{\pgfqpoint{2.529140in}{1.769100in}}%
\pgfpathcurveto{\pgfqpoint{2.523316in}{1.774924in}}{\pgfqpoint{2.515416in}{1.778197in}}{\pgfqpoint{2.507180in}{1.778197in}}%
\pgfpathcurveto{\pgfqpoint{2.498943in}{1.778197in}}{\pgfqpoint{2.491043in}{1.774924in}}{\pgfqpoint{2.485219in}{1.769100in}}%
\pgfpathcurveto{\pgfqpoint{2.479395in}{1.763277in}}{\pgfqpoint{2.476123in}{1.755376in}}{\pgfqpoint{2.476123in}{1.747140in}}%
\pgfpathcurveto{\pgfqpoint{2.476123in}{1.738904in}}{\pgfqpoint{2.479395in}{1.731004in}}{\pgfqpoint{2.485219in}{1.725180in}}%
\pgfpathcurveto{\pgfqpoint{2.491043in}{1.719356in}}{\pgfqpoint{2.498943in}{1.716084in}}{\pgfqpoint{2.507180in}{1.716084in}}%
\pgfpathclose%
\pgfusepath{stroke,fill}%
\end{pgfscope}%
\begin{pgfscope}%
\pgfpathrectangle{\pgfqpoint{0.100000in}{0.212622in}}{\pgfqpoint{3.696000in}{3.696000in}}%
\pgfusepath{clip}%
\pgfsetbuttcap%
\pgfsetroundjoin%
\definecolor{currentfill}{rgb}{0.121569,0.466667,0.705882}%
\pgfsetfillcolor{currentfill}%
\pgfsetfillopacity{0.974379}%
\pgfsetlinewidth{1.003750pt}%
\definecolor{currentstroke}{rgb}{0.121569,0.466667,0.705882}%
\pgfsetstrokecolor{currentstroke}%
\pgfsetstrokeopacity{0.974379}%
\pgfsetdash{}{0pt}%
\pgfpathmoveto{\pgfqpoint{2.052685in}{1.872915in}}%
\pgfpathcurveto{\pgfqpoint{2.060921in}{1.872915in}}{\pgfqpoint{2.068821in}{1.876188in}}{\pgfqpoint{2.074645in}{1.882012in}}%
\pgfpathcurveto{\pgfqpoint{2.080469in}{1.887836in}}{\pgfqpoint{2.083742in}{1.895736in}}{\pgfqpoint{2.083742in}{1.903972in}}%
\pgfpathcurveto{\pgfqpoint{2.083742in}{1.912208in}}{\pgfqpoint{2.080469in}{1.920108in}}{\pgfqpoint{2.074645in}{1.925932in}}%
\pgfpathcurveto{\pgfqpoint{2.068821in}{1.931756in}}{\pgfqpoint{2.060921in}{1.935028in}}{\pgfqpoint{2.052685in}{1.935028in}}%
\pgfpathcurveto{\pgfqpoint{2.044449in}{1.935028in}}{\pgfqpoint{2.036549in}{1.931756in}}{\pgfqpoint{2.030725in}{1.925932in}}%
\pgfpathcurveto{\pgfqpoint{2.024901in}{1.920108in}}{\pgfqpoint{2.021629in}{1.912208in}}{\pgfqpoint{2.021629in}{1.903972in}}%
\pgfpathcurveto{\pgfqpoint{2.021629in}{1.895736in}}{\pgfqpoint{2.024901in}{1.887836in}}{\pgfqpoint{2.030725in}{1.882012in}}%
\pgfpathcurveto{\pgfqpoint{2.036549in}{1.876188in}}{\pgfqpoint{2.044449in}{1.872915in}}{\pgfqpoint{2.052685in}{1.872915in}}%
\pgfpathclose%
\pgfusepath{stroke,fill}%
\end{pgfscope}%
\begin{pgfscope}%
\pgfpathrectangle{\pgfqpoint{0.100000in}{0.212622in}}{\pgfqpoint{3.696000in}{3.696000in}}%
\pgfusepath{clip}%
\pgfsetbuttcap%
\pgfsetroundjoin%
\definecolor{currentfill}{rgb}{0.121569,0.466667,0.705882}%
\pgfsetfillcolor{currentfill}%
\pgfsetfillopacity{0.975625}%
\pgfsetlinewidth{1.003750pt}%
\definecolor{currentstroke}{rgb}{0.121569,0.466667,0.705882}%
\pgfsetstrokecolor{currentstroke}%
\pgfsetstrokeopacity{0.975625}%
\pgfsetdash{}{0pt}%
\pgfpathmoveto{\pgfqpoint{2.063336in}{1.869246in}}%
\pgfpathcurveto{\pgfqpoint{2.071572in}{1.869246in}}{\pgfqpoint{2.079473in}{1.872518in}}{\pgfqpoint{2.085296in}{1.878342in}}%
\pgfpathcurveto{\pgfqpoint{2.091120in}{1.884166in}}{\pgfqpoint{2.094393in}{1.892066in}}{\pgfqpoint{2.094393in}{1.900303in}}%
\pgfpathcurveto{\pgfqpoint{2.094393in}{1.908539in}}{\pgfqpoint{2.091120in}{1.916439in}}{\pgfqpoint{2.085296in}{1.922263in}}%
\pgfpathcurveto{\pgfqpoint{2.079473in}{1.928087in}}{\pgfqpoint{2.071572in}{1.931359in}}{\pgfqpoint{2.063336in}{1.931359in}}%
\pgfpathcurveto{\pgfqpoint{2.055100in}{1.931359in}}{\pgfqpoint{2.047200in}{1.928087in}}{\pgfqpoint{2.041376in}{1.922263in}}%
\pgfpathcurveto{\pgfqpoint{2.035552in}{1.916439in}}{\pgfqpoint{2.032280in}{1.908539in}}{\pgfqpoint{2.032280in}{1.900303in}}%
\pgfpathcurveto{\pgfqpoint{2.032280in}{1.892066in}}{\pgfqpoint{2.035552in}{1.884166in}}{\pgfqpoint{2.041376in}{1.878342in}}%
\pgfpathcurveto{\pgfqpoint{2.047200in}{1.872518in}}{\pgfqpoint{2.055100in}{1.869246in}}{\pgfqpoint{2.063336in}{1.869246in}}%
\pgfpathclose%
\pgfusepath{stroke,fill}%
\end{pgfscope}%
\begin{pgfscope}%
\pgfpathrectangle{\pgfqpoint{0.100000in}{0.212622in}}{\pgfqpoint{3.696000in}{3.696000in}}%
\pgfusepath{clip}%
\pgfsetbuttcap%
\pgfsetroundjoin%
\definecolor{currentfill}{rgb}{0.121569,0.466667,0.705882}%
\pgfsetfillcolor{currentfill}%
\pgfsetfillopacity{0.976022}%
\pgfsetlinewidth{1.003750pt}%
\definecolor{currentstroke}{rgb}{0.121569,0.466667,0.705882}%
\pgfsetstrokecolor{currentstroke}%
\pgfsetstrokeopacity{0.976022}%
\pgfsetdash{}{0pt}%
\pgfpathmoveto{\pgfqpoint{2.503182in}{1.712506in}}%
\pgfpathcurveto{\pgfqpoint{2.511418in}{1.712506in}}{\pgfqpoint{2.519318in}{1.715778in}}{\pgfqpoint{2.525142in}{1.721602in}}%
\pgfpathcurveto{\pgfqpoint{2.530966in}{1.727426in}}{\pgfqpoint{2.534239in}{1.735326in}}{\pgfqpoint{2.534239in}{1.743562in}}%
\pgfpathcurveto{\pgfqpoint{2.534239in}{1.751799in}}{\pgfqpoint{2.530966in}{1.759699in}}{\pgfqpoint{2.525142in}{1.765523in}}%
\pgfpathcurveto{\pgfqpoint{2.519318in}{1.771347in}}{\pgfqpoint{2.511418in}{1.774619in}}{\pgfqpoint{2.503182in}{1.774619in}}%
\pgfpathcurveto{\pgfqpoint{2.494946in}{1.774619in}}{\pgfqpoint{2.487046in}{1.771347in}}{\pgfqpoint{2.481222in}{1.765523in}}%
\pgfpathcurveto{\pgfqpoint{2.475398in}{1.759699in}}{\pgfqpoint{2.472126in}{1.751799in}}{\pgfqpoint{2.472126in}{1.743562in}}%
\pgfpathcurveto{\pgfqpoint{2.472126in}{1.735326in}}{\pgfqpoint{2.475398in}{1.727426in}}{\pgfqpoint{2.481222in}{1.721602in}}%
\pgfpathcurveto{\pgfqpoint{2.487046in}{1.715778in}}{\pgfqpoint{2.494946in}{1.712506in}}{\pgfqpoint{2.503182in}{1.712506in}}%
\pgfpathclose%
\pgfusepath{stroke,fill}%
\end{pgfscope}%
\begin{pgfscope}%
\pgfpathrectangle{\pgfqpoint{0.100000in}{0.212622in}}{\pgfqpoint{3.696000in}{3.696000in}}%
\pgfusepath{clip}%
\pgfsetbuttcap%
\pgfsetroundjoin%
\definecolor{currentfill}{rgb}{0.121569,0.466667,0.705882}%
\pgfsetfillcolor{currentfill}%
\pgfsetfillopacity{0.977979}%
\pgfsetlinewidth{1.003750pt}%
\definecolor{currentstroke}{rgb}{0.121569,0.466667,0.705882}%
\pgfsetstrokecolor{currentstroke}%
\pgfsetstrokeopacity{0.977979}%
\pgfsetdash{}{0pt}%
\pgfpathmoveto{\pgfqpoint{2.082884in}{1.863756in}}%
\pgfpathcurveto{\pgfqpoint{2.091120in}{1.863756in}}{\pgfqpoint{2.099020in}{1.867028in}}{\pgfqpoint{2.104844in}{1.872852in}}%
\pgfpathcurveto{\pgfqpoint{2.110668in}{1.878676in}}{\pgfqpoint{2.113940in}{1.886576in}}{\pgfqpoint{2.113940in}{1.894812in}}%
\pgfpathcurveto{\pgfqpoint{2.113940in}{1.903049in}}{\pgfqpoint{2.110668in}{1.910949in}}{\pgfqpoint{2.104844in}{1.916773in}}%
\pgfpathcurveto{\pgfqpoint{2.099020in}{1.922597in}}{\pgfqpoint{2.091120in}{1.925869in}}{\pgfqpoint{2.082884in}{1.925869in}}%
\pgfpathcurveto{\pgfqpoint{2.074648in}{1.925869in}}{\pgfqpoint{2.066748in}{1.922597in}}{\pgfqpoint{2.060924in}{1.916773in}}%
\pgfpathcurveto{\pgfqpoint{2.055100in}{1.910949in}}{\pgfqpoint{2.051827in}{1.903049in}}{\pgfqpoint{2.051827in}{1.894812in}}%
\pgfpathcurveto{\pgfqpoint{2.051827in}{1.886576in}}{\pgfqpoint{2.055100in}{1.878676in}}{\pgfqpoint{2.060924in}{1.872852in}}%
\pgfpathcurveto{\pgfqpoint{2.066748in}{1.867028in}}{\pgfqpoint{2.074648in}{1.863756in}}{\pgfqpoint{2.082884in}{1.863756in}}%
\pgfpathclose%
\pgfusepath{stroke,fill}%
\end{pgfscope}%
\begin{pgfscope}%
\pgfpathrectangle{\pgfqpoint{0.100000in}{0.212622in}}{\pgfqpoint{3.696000in}{3.696000in}}%
\pgfusepath{clip}%
\pgfsetbuttcap%
\pgfsetroundjoin%
\definecolor{currentfill}{rgb}{0.121569,0.466667,0.705882}%
\pgfsetfillcolor{currentfill}%
\pgfsetfillopacity{0.978376}%
\pgfsetlinewidth{1.003750pt}%
\definecolor{currentstroke}{rgb}{0.121569,0.466667,0.705882}%
\pgfsetstrokecolor{currentstroke}%
\pgfsetstrokeopacity{0.978376}%
\pgfsetdash{}{0pt}%
\pgfpathmoveto{\pgfqpoint{2.498862in}{1.707927in}}%
\pgfpathcurveto{\pgfqpoint{2.507098in}{1.707927in}}{\pgfqpoint{2.514998in}{1.711199in}}{\pgfqpoint{2.520822in}{1.717023in}}%
\pgfpathcurveto{\pgfqpoint{2.526646in}{1.722847in}}{\pgfqpoint{2.529918in}{1.730747in}}{\pgfqpoint{2.529918in}{1.738983in}}%
\pgfpathcurveto{\pgfqpoint{2.529918in}{1.747219in}}{\pgfqpoint{2.526646in}{1.755120in}}{\pgfqpoint{2.520822in}{1.760943in}}%
\pgfpathcurveto{\pgfqpoint{2.514998in}{1.766767in}}{\pgfqpoint{2.507098in}{1.770040in}}{\pgfqpoint{2.498862in}{1.770040in}}%
\pgfpathcurveto{\pgfqpoint{2.490626in}{1.770040in}}{\pgfqpoint{2.482725in}{1.766767in}}{\pgfqpoint{2.476902in}{1.760943in}}%
\pgfpathcurveto{\pgfqpoint{2.471078in}{1.755120in}}{\pgfqpoint{2.467805in}{1.747219in}}{\pgfqpoint{2.467805in}{1.738983in}}%
\pgfpathcurveto{\pgfqpoint{2.467805in}{1.730747in}}{\pgfqpoint{2.471078in}{1.722847in}}{\pgfqpoint{2.476902in}{1.717023in}}%
\pgfpathcurveto{\pgfqpoint{2.482725in}{1.711199in}}{\pgfqpoint{2.490626in}{1.707927in}}{\pgfqpoint{2.498862in}{1.707927in}}%
\pgfpathclose%
\pgfusepath{stroke,fill}%
\end{pgfscope}%
\begin{pgfscope}%
\pgfpathrectangle{\pgfqpoint{0.100000in}{0.212622in}}{\pgfqpoint{3.696000in}{3.696000in}}%
\pgfusepath{clip}%
\pgfsetbuttcap%
\pgfsetroundjoin%
\definecolor{currentfill}{rgb}{0.121569,0.466667,0.705882}%
\pgfsetfillcolor{currentfill}%
\pgfsetfillopacity{0.979843}%
\pgfsetlinewidth{1.003750pt}%
\definecolor{currentstroke}{rgb}{0.121569,0.466667,0.705882}%
\pgfsetstrokecolor{currentstroke}%
\pgfsetstrokeopacity{0.979843}%
\pgfsetdash{}{0pt}%
\pgfpathmoveto{\pgfqpoint{2.100945in}{1.856285in}}%
\pgfpathcurveto{\pgfqpoint{2.109181in}{1.856285in}}{\pgfqpoint{2.117081in}{1.859557in}}{\pgfqpoint{2.122905in}{1.865381in}}%
\pgfpathcurveto{\pgfqpoint{2.128729in}{1.871205in}}{\pgfqpoint{2.132001in}{1.879105in}}{\pgfqpoint{2.132001in}{1.887341in}}%
\pgfpathcurveto{\pgfqpoint{2.132001in}{1.895577in}}{\pgfqpoint{2.128729in}{1.903478in}}{\pgfqpoint{2.122905in}{1.909301in}}%
\pgfpathcurveto{\pgfqpoint{2.117081in}{1.915125in}}{\pgfqpoint{2.109181in}{1.918398in}}{\pgfqpoint{2.100945in}{1.918398in}}%
\pgfpathcurveto{\pgfqpoint{2.092709in}{1.918398in}}{\pgfqpoint{2.084809in}{1.915125in}}{\pgfqpoint{2.078985in}{1.909301in}}%
\pgfpathcurveto{\pgfqpoint{2.073161in}{1.903478in}}{\pgfqpoint{2.069888in}{1.895577in}}{\pgfqpoint{2.069888in}{1.887341in}}%
\pgfpathcurveto{\pgfqpoint{2.069888in}{1.879105in}}{\pgfqpoint{2.073161in}{1.871205in}}{\pgfqpoint{2.078985in}{1.865381in}}%
\pgfpathcurveto{\pgfqpoint{2.084809in}{1.859557in}}{\pgfqpoint{2.092709in}{1.856285in}}{\pgfqpoint{2.100945in}{1.856285in}}%
\pgfpathclose%
\pgfusepath{stroke,fill}%
\end{pgfscope}%
\begin{pgfscope}%
\pgfpathrectangle{\pgfqpoint{0.100000in}{0.212622in}}{\pgfqpoint{3.696000in}{3.696000in}}%
\pgfusepath{clip}%
\pgfsetbuttcap%
\pgfsetroundjoin%
\definecolor{currentfill}{rgb}{0.121569,0.466667,0.705882}%
\pgfsetfillcolor{currentfill}%
\pgfsetfillopacity{0.981264}%
\pgfsetlinewidth{1.003750pt}%
\definecolor{currentstroke}{rgb}{0.121569,0.466667,0.705882}%
\pgfsetstrokecolor{currentstroke}%
\pgfsetstrokeopacity{0.981264}%
\pgfsetdash{}{0pt}%
\pgfpathmoveto{\pgfqpoint{2.493066in}{1.701502in}}%
\pgfpathcurveto{\pgfqpoint{2.501302in}{1.701502in}}{\pgfqpoint{2.509202in}{1.704774in}}{\pgfqpoint{2.515026in}{1.710598in}}%
\pgfpathcurveto{\pgfqpoint{2.520850in}{1.716422in}}{\pgfqpoint{2.524122in}{1.724322in}}{\pgfqpoint{2.524122in}{1.732558in}}%
\pgfpathcurveto{\pgfqpoint{2.524122in}{1.740795in}}{\pgfqpoint{2.520850in}{1.748695in}}{\pgfqpoint{2.515026in}{1.754519in}}%
\pgfpathcurveto{\pgfqpoint{2.509202in}{1.760343in}}{\pgfqpoint{2.501302in}{1.763615in}}{\pgfqpoint{2.493066in}{1.763615in}}%
\pgfpathcurveto{\pgfqpoint{2.484829in}{1.763615in}}{\pgfqpoint{2.476929in}{1.760343in}}{\pgfqpoint{2.471105in}{1.754519in}}%
\pgfpathcurveto{\pgfqpoint{2.465281in}{1.748695in}}{\pgfqpoint{2.462009in}{1.740795in}}{\pgfqpoint{2.462009in}{1.732558in}}%
\pgfpathcurveto{\pgfqpoint{2.462009in}{1.724322in}}{\pgfqpoint{2.465281in}{1.716422in}}{\pgfqpoint{2.471105in}{1.710598in}}%
\pgfpathcurveto{\pgfqpoint{2.476929in}{1.704774in}}{\pgfqpoint{2.484829in}{1.701502in}}{\pgfqpoint{2.493066in}{1.701502in}}%
\pgfpathclose%
\pgfusepath{stroke,fill}%
\end{pgfscope}%
\begin{pgfscope}%
\pgfpathrectangle{\pgfqpoint{0.100000in}{0.212622in}}{\pgfqpoint{3.696000in}{3.696000in}}%
\pgfusepath{clip}%
\pgfsetbuttcap%
\pgfsetroundjoin%
\definecolor{currentfill}{rgb}{0.121569,0.466667,0.705882}%
\pgfsetfillcolor{currentfill}%
\pgfsetfillopacity{0.981402}%
\pgfsetlinewidth{1.003750pt}%
\definecolor{currentstroke}{rgb}{0.121569,0.466667,0.705882}%
\pgfsetstrokecolor{currentstroke}%
\pgfsetstrokeopacity{0.981402}%
\pgfsetdash{}{0pt}%
\pgfpathmoveto{\pgfqpoint{2.115659in}{1.849914in}}%
\pgfpathcurveto{\pgfqpoint{2.123895in}{1.849914in}}{\pgfqpoint{2.131795in}{1.853186in}}{\pgfqpoint{2.137619in}{1.859010in}}%
\pgfpathcurveto{\pgfqpoint{2.143443in}{1.864834in}}{\pgfqpoint{2.146716in}{1.872734in}}{\pgfqpoint{2.146716in}{1.880970in}}%
\pgfpathcurveto{\pgfqpoint{2.146716in}{1.889207in}}{\pgfqpoint{2.143443in}{1.897107in}}{\pgfqpoint{2.137619in}{1.902931in}}%
\pgfpathcurveto{\pgfqpoint{2.131795in}{1.908755in}}{\pgfqpoint{2.123895in}{1.912027in}}{\pgfqpoint{2.115659in}{1.912027in}}%
\pgfpathcurveto{\pgfqpoint{2.107423in}{1.912027in}}{\pgfqpoint{2.099523in}{1.908755in}}{\pgfqpoint{2.093699in}{1.902931in}}%
\pgfpathcurveto{\pgfqpoint{2.087875in}{1.897107in}}{\pgfqpoint{2.084603in}{1.889207in}}{\pgfqpoint{2.084603in}{1.880970in}}%
\pgfpathcurveto{\pgfqpoint{2.084603in}{1.872734in}}{\pgfqpoint{2.087875in}{1.864834in}}{\pgfqpoint{2.093699in}{1.859010in}}%
\pgfpathcurveto{\pgfqpoint{2.099523in}{1.853186in}}{\pgfqpoint{2.107423in}{1.849914in}}{\pgfqpoint{2.115659in}{1.849914in}}%
\pgfpathclose%
\pgfusepath{stroke,fill}%
\end{pgfscope}%
\begin{pgfscope}%
\pgfpathrectangle{\pgfqpoint{0.100000in}{0.212622in}}{\pgfqpoint{3.696000in}{3.696000in}}%
\pgfusepath{clip}%
\pgfsetbuttcap%
\pgfsetroundjoin%
\definecolor{currentfill}{rgb}{0.121569,0.466667,0.705882}%
\pgfsetfillcolor{currentfill}%
\pgfsetfillopacity{0.982826}%
\pgfsetlinewidth{1.003750pt}%
\definecolor{currentstroke}{rgb}{0.121569,0.466667,0.705882}%
\pgfsetstrokecolor{currentstroke}%
\pgfsetstrokeopacity{0.982826}%
\pgfsetdash{}{0pt}%
\pgfpathmoveto{\pgfqpoint{2.128741in}{1.844621in}}%
\pgfpathcurveto{\pgfqpoint{2.136977in}{1.844621in}}{\pgfqpoint{2.144877in}{1.847893in}}{\pgfqpoint{2.150701in}{1.853717in}}%
\pgfpathcurveto{\pgfqpoint{2.156525in}{1.859541in}}{\pgfqpoint{2.159797in}{1.867441in}}{\pgfqpoint{2.159797in}{1.875677in}}%
\pgfpathcurveto{\pgfqpoint{2.159797in}{1.883914in}}{\pgfqpoint{2.156525in}{1.891814in}}{\pgfqpoint{2.150701in}{1.897638in}}%
\pgfpathcurveto{\pgfqpoint{2.144877in}{1.903462in}}{\pgfqpoint{2.136977in}{1.906734in}}{\pgfqpoint{2.128741in}{1.906734in}}%
\pgfpathcurveto{\pgfqpoint{2.120505in}{1.906734in}}{\pgfqpoint{2.112605in}{1.903462in}}{\pgfqpoint{2.106781in}{1.897638in}}%
\pgfpathcurveto{\pgfqpoint{2.100957in}{1.891814in}}{\pgfqpoint{2.097684in}{1.883914in}}{\pgfqpoint{2.097684in}{1.875677in}}%
\pgfpathcurveto{\pgfqpoint{2.097684in}{1.867441in}}{\pgfqpoint{2.100957in}{1.859541in}}{\pgfqpoint{2.106781in}{1.853717in}}%
\pgfpathcurveto{\pgfqpoint{2.112605in}{1.847893in}}{\pgfqpoint{2.120505in}{1.844621in}}{\pgfqpoint{2.128741in}{1.844621in}}%
\pgfpathclose%
\pgfusepath{stroke,fill}%
\end{pgfscope}%
\begin{pgfscope}%
\pgfpathrectangle{\pgfqpoint{0.100000in}{0.212622in}}{\pgfqpoint{3.696000in}{3.696000in}}%
\pgfusepath{clip}%
\pgfsetbuttcap%
\pgfsetroundjoin%
\definecolor{currentfill}{rgb}{0.121569,0.466667,0.705882}%
\pgfsetfillcolor{currentfill}%
\pgfsetfillopacity{0.984098}%
\pgfsetlinewidth{1.003750pt}%
\definecolor{currentstroke}{rgb}{0.121569,0.466667,0.705882}%
\pgfsetstrokecolor{currentstroke}%
\pgfsetstrokeopacity{0.984098}%
\pgfsetdash{}{0pt}%
\pgfpathmoveto{\pgfqpoint{2.141290in}{1.838854in}}%
\pgfpathcurveto{\pgfqpoint{2.149526in}{1.838854in}}{\pgfqpoint{2.157426in}{1.842126in}}{\pgfqpoint{2.163250in}{1.847950in}}%
\pgfpathcurveto{\pgfqpoint{2.169074in}{1.853774in}}{\pgfqpoint{2.172347in}{1.861674in}}{\pgfqpoint{2.172347in}{1.869911in}}%
\pgfpathcurveto{\pgfqpoint{2.172347in}{1.878147in}}{\pgfqpoint{2.169074in}{1.886047in}}{\pgfqpoint{2.163250in}{1.891871in}}%
\pgfpathcurveto{\pgfqpoint{2.157426in}{1.897695in}}{\pgfqpoint{2.149526in}{1.900967in}}{\pgfqpoint{2.141290in}{1.900967in}}%
\pgfpathcurveto{\pgfqpoint{2.133054in}{1.900967in}}{\pgfqpoint{2.125154in}{1.897695in}}{\pgfqpoint{2.119330in}{1.891871in}}%
\pgfpathcurveto{\pgfqpoint{2.113506in}{1.886047in}}{\pgfqpoint{2.110234in}{1.878147in}}{\pgfqpoint{2.110234in}{1.869911in}}%
\pgfpathcurveto{\pgfqpoint{2.110234in}{1.861674in}}{\pgfqpoint{2.113506in}{1.853774in}}{\pgfqpoint{2.119330in}{1.847950in}}%
\pgfpathcurveto{\pgfqpoint{2.125154in}{1.842126in}}{\pgfqpoint{2.133054in}{1.838854in}}{\pgfqpoint{2.141290in}{1.838854in}}%
\pgfpathclose%
\pgfusepath{stroke,fill}%
\end{pgfscope}%
\begin{pgfscope}%
\pgfpathrectangle{\pgfqpoint{0.100000in}{0.212622in}}{\pgfqpoint{3.696000in}{3.696000in}}%
\pgfusepath{clip}%
\pgfsetbuttcap%
\pgfsetroundjoin%
\definecolor{currentfill}{rgb}{0.121569,0.466667,0.705882}%
\pgfsetfillcolor{currentfill}%
\pgfsetfillopacity{0.984460}%
\pgfsetlinewidth{1.003750pt}%
\definecolor{currentstroke}{rgb}{0.121569,0.466667,0.705882}%
\pgfsetstrokecolor{currentstroke}%
\pgfsetstrokeopacity{0.984460}%
\pgfsetdash{}{0pt}%
\pgfpathmoveto{\pgfqpoint{2.487897in}{1.695048in}}%
\pgfpathcurveto{\pgfqpoint{2.496133in}{1.695048in}}{\pgfqpoint{2.504033in}{1.698321in}}{\pgfqpoint{2.509857in}{1.704144in}}%
\pgfpathcurveto{\pgfqpoint{2.515681in}{1.709968in}}{\pgfqpoint{2.518953in}{1.717868in}}{\pgfqpoint{2.518953in}{1.726105in}}%
\pgfpathcurveto{\pgfqpoint{2.518953in}{1.734341in}}{\pgfqpoint{2.515681in}{1.742241in}}{\pgfqpoint{2.509857in}{1.748065in}}%
\pgfpathcurveto{\pgfqpoint{2.504033in}{1.753889in}}{\pgfqpoint{2.496133in}{1.757161in}}{\pgfqpoint{2.487897in}{1.757161in}}%
\pgfpathcurveto{\pgfqpoint{2.479661in}{1.757161in}}{\pgfqpoint{2.471761in}{1.753889in}}{\pgfqpoint{2.465937in}{1.748065in}}%
\pgfpathcurveto{\pgfqpoint{2.460113in}{1.742241in}}{\pgfqpoint{2.456840in}{1.734341in}}{\pgfqpoint{2.456840in}{1.726105in}}%
\pgfpathcurveto{\pgfqpoint{2.456840in}{1.717868in}}{\pgfqpoint{2.460113in}{1.709968in}}{\pgfqpoint{2.465937in}{1.704144in}}%
\pgfpathcurveto{\pgfqpoint{2.471761in}{1.698321in}}{\pgfqpoint{2.479661in}{1.695048in}}{\pgfqpoint{2.487897in}{1.695048in}}%
\pgfpathclose%
\pgfusepath{stroke,fill}%
\end{pgfscope}%
\begin{pgfscope}%
\pgfpathrectangle{\pgfqpoint{0.100000in}{0.212622in}}{\pgfqpoint{3.696000in}{3.696000in}}%
\pgfusepath{clip}%
\pgfsetbuttcap%
\pgfsetroundjoin%
\definecolor{currentfill}{rgb}{0.121569,0.466667,0.705882}%
\pgfsetfillcolor{currentfill}%
\pgfsetfillopacity{0.985348}%
\pgfsetlinewidth{1.003750pt}%
\definecolor{currentstroke}{rgb}{0.121569,0.466667,0.705882}%
\pgfsetstrokecolor{currentstroke}%
\pgfsetstrokeopacity{0.985348}%
\pgfsetdash{}{0pt}%
\pgfpathmoveto{\pgfqpoint{2.152954in}{1.833306in}}%
\pgfpathcurveto{\pgfqpoint{2.161190in}{1.833306in}}{\pgfqpoint{2.169091in}{1.836578in}}{\pgfqpoint{2.174914in}{1.842402in}}%
\pgfpathcurveto{\pgfqpoint{2.180738in}{1.848226in}}{\pgfqpoint{2.184011in}{1.856126in}}{\pgfqpoint{2.184011in}{1.864362in}}%
\pgfpathcurveto{\pgfqpoint{2.184011in}{1.872599in}}{\pgfqpoint{2.180738in}{1.880499in}}{\pgfqpoint{2.174914in}{1.886323in}}%
\pgfpathcurveto{\pgfqpoint{2.169091in}{1.892147in}}{\pgfqpoint{2.161190in}{1.895419in}}{\pgfqpoint{2.152954in}{1.895419in}}%
\pgfpathcurveto{\pgfqpoint{2.144718in}{1.895419in}}{\pgfqpoint{2.136818in}{1.892147in}}{\pgfqpoint{2.130994in}{1.886323in}}%
\pgfpathcurveto{\pgfqpoint{2.125170in}{1.880499in}}{\pgfqpoint{2.121898in}{1.872599in}}{\pgfqpoint{2.121898in}{1.864362in}}%
\pgfpathcurveto{\pgfqpoint{2.121898in}{1.856126in}}{\pgfqpoint{2.125170in}{1.848226in}}{\pgfqpoint{2.130994in}{1.842402in}}%
\pgfpathcurveto{\pgfqpoint{2.136818in}{1.836578in}}{\pgfqpoint{2.144718in}{1.833306in}}{\pgfqpoint{2.152954in}{1.833306in}}%
\pgfpathclose%
\pgfusepath{stroke,fill}%
\end{pgfscope}%
\begin{pgfscope}%
\pgfpathrectangle{\pgfqpoint{0.100000in}{0.212622in}}{\pgfqpoint{3.696000in}{3.696000in}}%
\pgfusepath{clip}%
\pgfsetbuttcap%
\pgfsetroundjoin%
\definecolor{currentfill}{rgb}{0.121569,0.466667,0.705882}%
\pgfsetfillcolor{currentfill}%
\pgfsetfillopacity{0.986366}%
\pgfsetlinewidth{1.003750pt}%
\definecolor{currentstroke}{rgb}{0.121569,0.466667,0.705882}%
\pgfsetstrokecolor{currentstroke}%
\pgfsetstrokeopacity{0.986366}%
\pgfsetdash{}{0pt}%
\pgfpathmoveto{\pgfqpoint{2.162612in}{1.829186in}}%
\pgfpathcurveto{\pgfqpoint{2.170848in}{1.829186in}}{\pgfqpoint{2.178748in}{1.832458in}}{\pgfqpoint{2.184572in}{1.838282in}}%
\pgfpathcurveto{\pgfqpoint{2.190396in}{1.844106in}}{\pgfqpoint{2.193669in}{1.852006in}}{\pgfqpoint{2.193669in}{1.860242in}}%
\pgfpathcurveto{\pgfqpoint{2.193669in}{1.868479in}}{\pgfqpoint{2.190396in}{1.876379in}}{\pgfqpoint{2.184572in}{1.882203in}}%
\pgfpathcurveto{\pgfqpoint{2.178748in}{1.888027in}}{\pgfqpoint{2.170848in}{1.891299in}}{\pgfqpoint{2.162612in}{1.891299in}}%
\pgfpathcurveto{\pgfqpoint{2.154376in}{1.891299in}}{\pgfqpoint{2.146476in}{1.888027in}}{\pgfqpoint{2.140652in}{1.882203in}}%
\pgfpathcurveto{\pgfqpoint{2.134828in}{1.876379in}}{\pgfqpoint{2.131556in}{1.868479in}}{\pgfqpoint{2.131556in}{1.860242in}}%
\pgfpathcurveto{\pgfqpoint{2.131556in}{1.852006in}}{\pgfqpoint{2.134828in}{1.844106in}}{\pgfqpoint{2.140652in}{1.838282in}}%
\pgfpathcurveto{\pgfqpoint{2.146476in}{1.832458in}}{\pgfqpoint{2.154376in}{1.829186in}}{\pgfqpoint{2.162612in}{1.829186in}}%
\pgfpathclose%
\pgfusepath{stroke,fill}%
\end{pgfscope}%
\begin{pgfscope}%
\pgfpathrectangle{\pgfqpoint{0.100000in}{0.212622in}}{\pgfqpoint{3.696000in}{3.696000in}}%
\pgfusepath{clip}%
\pgfsetbuttcap%
\pgfsetroundjoin%
\definecolor{currentfill}{rgb}{0.121569,0.466667,0.705882}%
\pgfsetfillcolor{currentfill}%
\pgfsetfillopacity{0.987057}%
\pgfsetlinewidth{1.003750pt}%
\definecolor{currentstroke}{rgb}{0.121569,0.466667,0.705882}%
\pgfsetstrokecolor{currentstroke}%
\pgfsetstrokeopacity{0.987057}%
\pgfsetdash{}{0pt}%
\pgfpathmoveto{\pgfqpoint{2.169352in}{1.826088in}}%
\pgfpathcurveto{\pgfqpoint{2.177589in}{1.826088in}}{\pgfqpoint{2.185489in}{1.829360in}}{\pgfqpoint{2.191313in}{1.835184in}}%
\pgfpathcurveto{\pgfqpoint{2.197137in}{1.841008in}}{\pgfqpoint{2.200409in}{1.848908in}}{\pgfqpoint{2.200409in}{1.857145in}}%
\pgfpathcurveto{\pgfqpoint{2.200409in}{1.865381in}}{\pgfqpoint{2.197137in}{1.873281in}}{\pgfqpoint{2.191313in}{1.879105in}}%
\pgfpathcurveto{\pgfqpoint{2.185489in}{1.884929in}}{\pgfqpoint{2.177589in}{1.888201in}}{\pgfqpoint{2.169352in}{1.888201in}}%
\pgfpathcurveto{\pgfqpoint{2.161116in}{1.888201in}}{\pgfqpoint{2.153216in}{1.884929in}}{\pgfqpoint{2.147392in}{1.879105in}}%
\pgfpathcurveto{\pgfqpoint{2.141568in}{1.873281in}}{\pgfqpoint{2.138296in}{1.865381in}}{\pgfqpoint{2.138296in}{1.857145in}}%
\pgfpathcurveto{\pgfqpoint{2.138296in}{1.848908in}}{\pgfqpoint{2.141568in}{1.841008in}}{\pgfqpoint{2.147392in}{1.835184in}}%
\pgfpathcurveto{\pgfqpoint{2.153216in}{1.829360in}}{\pgfqpoint{2.161116in}{1.826088in}}{\pgfqpoint{2.169352in}{1.826088in}}%
\pgfpathclose%
\pgfusepath{stroke,fill}%
\end{pgfscope}%
\begin{pgfscope}%
\pgfpathrectangle{\pgfqpoint{0.100000in}{0.212622in}}{\pgfqpoint{3.696000in}{3.696000in}}%
\pgfusepath{clip}%
\pgfsetbuttcap%
\pgfsetroundjoin%
\definecolor{currentfill}{rgb}{0.121569,0.466667,0.705882}%
\pgfsetfillcolor{currentfill}%
\pgfsetfillopacity{0.987995}%
\pgfsetlinewidth{1.003750pt}%
\definecolor{currentstroke}{rgb}{0.121569,0.466667,0.705882}%
\pgfsetstrokecolor{currentstroke}%
\pgfsetstrokeopacity{0.987995}%
\pgfsetdash{}{0pt}%
\pgfpathmoveto{\pgfqpoint{2.482385in}{1.689339in}}%
\pgfpathcurveto{\pgfqpoint{2.490622in}{1.689339in}}{\pgfqpoint{2.498522in}{1.692611in}}{\pgfqpoint{2.504346in}{1.698435in}}%
\pgfpathcurveto{\pgfqpoint{2.510170in}{1.704259in}}{\pgfqpoint{2.513442in}{1.712159in}}{\pgfqpoint{2.513442in}{1.720395in}}%
\pgfpathcurveto{\pgfqpoint{2.513442in}{1.728632in}}{\pgfqpoint{2.510170in}{1.736532in}}{\pgfqpoint{2.504346in}{1.742356in}}%
\pgfpathcurveto{\pgfqpoint{2.498522in}{1.748180in}}{\pgfqpoint{2.490622in}{1.751452in}}{\pgfqpoint{2.482385in}{1.751452in}}%
\pgfpathcurveto{\pgfqpoint{2.474149in}{1.751452in}}{\pgfqpoint{2.466249in}{1.748180in}}{\pgfqpoint{2.460425in}{1.742356in}}%
\pgfpathcurveto{\pgfqpoint{2.454601in}{1.736532in}}{\pgfqpoint{2.451329in}{1.728632in}}{\pgfqpoint{2.451329in}{1.720395in}}%
\pgfpathcurveto{\pgfqpoint{2.451329in}{1.712159in}}{\pgfqpoint{2.454601in}{1.704259in}}{\pgfqpoint{2.460425in}{1.698435in}}%
\pgfpathcurveto{\pgfqpoint{2.466249in}{1.692611in}}{\pgfqpoint{2.474149in}{1.689339in}}{\pgfqpoint{2.482385in}{1.689339in}}%
\pgfpathclose%
\pgfusepath{stroke,fill}%
\end{pgfscope}%
\begin{pgfscope}%
\pgfpathrectangle{\pgfqpoint{0.100000in}{0.212622in}}{\pgfqpoint{3.696000in}{3.696000in}}%
\pgfusepath{clip}%
\pgfsetbuttcap%
\pgfsetroundjoin%
\definecolor{currentfill}{rgb}{0.121569,0.466667,0.705882}%
\pgfsetfillcolor{currentfill}%
\pgfsetfillopacity{0.988299}%
\pgfsetlinewidth{1.003750pt}%
\definecolor{currentstroke}{rgb}{0.121569,0.466667,0.705882}%
\pgfsetstrokecolor{currentstroke}%
\pgfsetstrokeopacity{0.988299}%
\pgfsetdash{}{0pt}%
\pgfpathmoveto{\pgfqpoint{2.181689in}{1.820663in}}%
\pgfpathcurveto{\pgfqpoint{2.189925in}{1.820663in}}{\pgfqpoint{2.197825in}{1.823935in}}{\pgfqpoint{2.203649in}{1.829759in}}%
\pgfpathcurveto{\pgfqpoint{2.209473in}{1.835583in}}{\pgfqpoint{2.212746in}{1.843483in}}{\pgfqpoint{2.212746in}{1.851720in}}%
\pgfpathcurveto{\pgfqpoint{2.212746in}{1.859956in}}{\pgfqpoint{2.209473in}{1.867856in}}{\pgfqpoint{2.203649in}{1.873680in}}%
\pgfpathcurveto{\pgfqpoint{2.197825in}{1.879504in}}{\pgfqpoint{2.189925in}{1.882776in}}{\pgfqpoint{2.181689in}{1.882776in}}%
\pgfpathcurveto{\pgfqpoint{2.173453in}{1.882776in}}{\pgfqpoint{2.165553in}{1.879504in}}{\pgfqpoint{2.159729in}{1.873680in}}%
\pgfpathcurveto{\pgfqpoint{2.153905in}{1.867856in}}{\pgfqpoint{2.150633in}{1.859956in}}{\pgfqpoint{2.150633in}{1.851720in}}%
\pgfpathcurveto{\pgfqpoint{2.150633in}{1.843483in}}{\pgfqpoint{2.153905in}{1.835583in}}{\pgfqpoint{2.159729in}{1.829759in}}%
\pgfpathcurveto{\pgfqpoint{2.165553in}{1.823935in}}{\pgfqpoint{2.173453in}{1.820663in}}{\pgfqpoint{2.181689in}{1.820663in}}%
\pgfpathclose%
\pgfusepath{stroke,fill}%
\end{pgfscope}%
\begin{pgfscope}%
\pgfpathrectangle{\pgfqpoint{0.100000in}{0.212622in}}{\pgfqpoint{3.696000in}{3.696000in}}%
\pgfusepath{clip}%
\pgfsetbuttcap%
\pgfsetroundjoin%
\definecolor{currentfill}{rgb}{0.121569,0.466667,0.705882}%
\pgfsetfillcolor{currentfill}%
\pgfsetfillopacity{0.989276}%
\pgfsetlinewidth{1.003750pt}%
\definecolor{currentstroke}{rgb}{0.121569,0.466667,0.705882}%
\pgfsetstrokecolor{currentstroke}%
\pgfsetstrokeopacity{0.989276}%
\pgfsetdash{}{0pt}%
\pgfpathmoveto{\pgfqpoint{2.192945in}{1.814921in}}%
\pgfpathcurveto{\pgfqpoint{2.201181in}{1.814921in}}{\pgfqpoint{2.209081in}{1.818193in}}{\pgfqpoint{2.214905in}{1.824017in}}%
\pgfpathcurveto{\pgfqpoint{2.220729in}{1.829841in}}{\pgfqpoint{2.224001in}{1.837741in}}{\pgfqpoint{2.224001in}{1.845977in}}%
\pgfpathcurveto{\pgfqpoint{2.224001in}{1.854213in}}{\pgfqpoint{2.220729in}{1.862114in}}{\pgfqpoint{2.214905in}{1.867937in}}%
\pgfpathcurveto{\pgfqpoint{2.209081in}{1.873761in}}{\pgfqpoint{2.201181in}{1.877034in}}{\pgfqpoint{2.192945in}{1.877034in}}%
\pgfpathcurveto{\pgfqpoint{2.184709in}{1.877034in}}{\pgfqpoint{2.176809in}{1.873761in}}{\pgfqpoint{2.170985in}{1.867937in}}%
\pgfpathcurveto{\pgfqpoint{2.165161in}{1.862114in}}{\pgfqpoint{2.161888in}{1.854213in}}{\pgfqpoint{2.161888in}{1.845977in}}%
\pgfpathcurveto{\pgfqpoint{2.161888in}{1.837741in}}{\pgfqpoint{2.165161in}{1.829841in}}{\pgfqpoint{2.170985in}{1.824017in}}%
\pgfpathcurveto{\pgfqpoint{2.176809in}{1.818193in}}{\pgfqpoint{2.184709in}{1.814921in}}{\pgfqpoint{2.192945in}{1.814921in}}%
\pgfpathclose%
\pgfusepath{stroke,fill}%
\end{pgfscope}%
\begin{pgfscope}%
\pgfpathrectangle{\pgfqpoint{0.100000in}{0.212622in}}{\pgfqpoint{3.696000in}{3.696000in}}%
\pgfusepath{clip}%
\pgfsetbuttcap%
\pgfsetroundjoin%
\definecolor{currentfill}{rgb}{0.121569,0.466667,0.705882}%
\pgfsetfillcolor{currentfill}%
\pgfsetfillopacity{0.989775}%
\pgfsetlinewidth{1.003750pt}%
\definecolor{currentstroke}{rgb}{0.121569,0.466667,0.705882}%
\pgfsetstrokecolor{currentstroke}%
\pgfsetstrokeopacity{0.989775}%
\pgfsetdash{}{0pt}%
\pgfpathmoveto{\pgfqpoint{2.479447in}{1.685101in}}%
\pgfpathcurveto{\pgfqpoint{2.487683in}{1.685101in}}{\pgfqpoint{2.495583in}{1.688374in}}{\pgfqpoint{2.501407in}{1.694198in}}%
\pgfpathcurveto{\pgfqpoint{2.507231in}{1.700022in}}{\pgfqpoint{2.510503in}{1.707922in}}{\pgfqpoint{2.510503in}{1.716158in}}%
\pgfpathcurveto{\pgfqpoint{2.510503in}{1.724394in}}{\pgfqpoint{2.507231in}{1.732294in}}{\pgfqpoint{2.501407in}{1.738118in}}%
\pgfpathcurveto{\pgfqpoint{2.495583in}{1.743942in}}{\pgfqpoint{2.487683in}{1.747214in}}{\pgfqpoint{2.479447in}{1.747214in}}%
\pgfpathcurveto{\pgfqpoint{2.471210in}{1.747214in}}{\pgfqpoint{2.463310in}{1.743942in}}{\pgfqpoint{2.457486in}{1.738118in}}%
\pgfpathcurveto{\pgfqpoint{2.451662in}{1.732294in}}{\pgfqpoint{2.448390in}{1.724394in}}{\pgfqpoint{2.448390in}{1.716158in}}%
\pgfpathcurveto{\pgfqpoint{2.448390in}{1.707922in}}{\pgfqpoint{2.451662in}{1.700022in}}{\pgfqpoint{2.457486in}{1.694198in}}%
\pgfpathcurveto{\pgfqpoint{2.463310in}{1.688374in}}{\pgfqpoint{2.471210in}{1.685101in}}{\pgfqpoint{2.479447in}{1.685101in}}%
\pgfpathclose%
\pgfusepath{stroke,fill}%
\end{pgfscope}%
\begin{pgfscope}%
\pgfpathrectangle{\pgfqpoint{0.100000in}{0.212622in}}{\pgfqpoint{3.696000in}{3.696000in}}%
\pgfusepath{clip}%
\pgfsetbuttcap%
\pgfsetroundjoin%
\definecolor{currentfill}{rgb}{0.121569,0.466667,0.705882}%
\pgfsetfillcolor{currentfill}%
\pgfsetfillopacity{0.990449}%
\pgfsetlinewidth{1.003750pt}%
\definecolor{currentstroke}{rgb}{0.121569,0.466667,0.705882}%
\pgfsetstrokecolor{currentstroke}%
\pgfsetstrokeopacity{0.990449}%
\pgfsetdash{}{0pt}%
\pgfpathmoveto{\pgfqpoint{2.203110in}{1.810927in}}%
\pgfpathcurveto{\pgfqpoint{2.211346in}{1.810927in}}{\pgfqpoint{2.219246in}{1.814199in}}{\pgfqpoint{2.225070in}{1.820023in}}%
\pgfpathcurveto{\pgfqpoint{2.230894in}{1.825847in}}{\pgfqpoint{2.234166in}{1.833747in}}{\pgfqpoint{2.234166in}{1.841983in}}%
\pgfpathcurveto{\pgfqpoint{2.234166in}{1.850219in}}{\pgfqpoint{2.230894in}{1.858119in}}{\pgfqpoint{2.225070in}{1.863943in}}%
\pgfpathcurveto{\pgfqpoint{2.219246in}{1.869767in}}{\pgfqpoint{2.211346in}{1.873040in}}{\pgfqpoint{2.203110in}{1.873040in}}%
\pgfpathcurveto{\pgfqpoint{2.194873in}{1.873040in}}{\pgfqpoint{2.186973in}{1.869767in}}{\pgfqpoint{2.181149in}{1.863943in}}%
\pgfpathcurveto{\pgfqpoint{2.175325in}{1.858119in}}{\pgfqpoint{2.172053in}{1.850219in}}{\pgfqpoint{2.172053in}{1.841983in}}%
\pgfpathcurveto{\pgfqpoint{2.172053in}{1.833747in}}{\pgfqpoint{2.175325in}{1.825847in}}{\pgfqpoint{2.181149in}{1.820023in}}%
\pgfpathcurveto{\pgfqpoint{2.186973in}{1.814199in}}{\pgfqpoint{2.194873in}{1.810927in}}{\pgfqpoint{2.203110in}{1.810927in}}%
\pgfpathclose%
\pgfusepath{stroke,fill}%
\end{pgfscope}%
\begin{pgfscope}%
\pgfpathrectangle{\pgfqpoint{0.100000in}{0.212622in}}{\pgfqpoint{3.696000in}{3.696000in}}%
\pgfusepath{clip}%
\pgfsetbuttcap%
\pgfsetroundjoin%
\definecolor{currentfill}{rgb}{0.121569,0.466667,0.705882}%
\pgfsetfillcolor{currentfill}%
\pgfsetfillopacity{0.990799}%
\pgfsetlinewidth{1.003750pt}%
\definecolor{currentstroke}{rgb}{0.121569,0.466667,0.705882}%
\pgfsetstrokecolor{currentstroke}%
\pgfsetstrokeopacity{0.990799}%
\pgfsetdash{}{0pt}%
\pgfpathmoveto{\pgfqpoint{2.477771in}{1.683086in}}%
\pgfpathcurveto{\pgfqpoint{2.486007in}{1.683086in}}{\pgfqpoint{2.493907in}{1.686358in}}{\pgfqpoint{2.499731in}{1.692182in}}%
\pgfpathcurveto{\pgfqpoint{2.505555in}{1.698006in}}{\pgfqpoint{2.508827in}{1.705906in}}{\pgfqpoint{2.508827in}{1.714142in}}%
\pgfpathcurveto{\pgfqpoint{2.508827in}{1.722378in}}{\pgfqpoint{2.505555in}{1.730279in}}{\pgfqpoint{2.499731in}{1.736102in}}%
\pgfpathcurveto{\pgfqpoint{2.493907in}{1.741926in}}{\pgfqpoint{2.486007in}{1.745199in}}{\pgfqpoint{2.477771in}{1.745199in}}%
\pgfpathcurveto{\pgfqpoint{2.469534in}{1.745199in}}{\pgfqpoint{2.461634in}{1.741926in}}{\pgfqpoint{2.455810in}{1.736102in}}%
\pgfpathcurveto{\pgfqpoint{2.449987in}{1.730279in}}{\pgfqpoint{2.446714in}{1.722378in}}{\pgfqpoint{2.446714in}{1.714142in}}%
\pgfpathcurveto{\pgfqpoint{2.446714in}{1.705906in}}{\pgfqpoint{2.449987in}{1.698006in}}{\pgfqpoint{2.455810in}{1.692182in}}%
\pgfpathcurveto{\pgfqpoint{2.461634in}{1.686358in}}{\pgfqpoint{2.469534in}{1.683086in}}{\pgfqpoint{2.477771in}{1.683086in}}%
\pgfpathclose%
\pgfusepath{stroke,fill}%
\end{pgfscope}%
\begin{pgfscope}%
\pgfpathrectangle{\pgfqpoint{0.100000in}{0.212622in}}{\pgfqpoint{3.696000in}{3.696000in}}%
\pgfusepath{clip}%
\pgfsetbuttcap%
\pgfsetroundjoin%
\definecolor{currentfill}{rgb}{0.121569,0.466667,0.705882}%
\pgfsetfillcolor{currentfill}%
\pgfsetfillopacity{0.991427}%
\pgfsetlinewidth{1.003750pt}%
\definecolor{currentstroke}{rgb}{0.121569,0.466667,0.705882}%
\pgfsetstrokecolor{currentstroke}%
\pgfsetstrokeopacity{0.991427}%
\pgfsetdash{}{0pt}%
\pgfpathmoveto{\pgfqpoint{2.211541in}{1.808103in}}%
\pgfpathcurveto{\pgfqpoint{2.219778in}{1.808103in}}{\pgfqpoint{2.227678in}{1.811375in}}{\pgfqpoint{2.233502in}{1.817199in}}%
\pgfpathcurveto{\pgfqpoint{2.239326in}{1.823023in}}{\pgfqpoint{2.242598in}{1.830923in}}{\pgfqpoint{2.242598in}{1.839159in}}%
\pgfpathcurveto{\pgfqpoint{2.242598in}{1.847396in}}{\pgfqpoint{2.239326in}{1.855296in}}{\pgfqpoint{2.233502in}{1.861120in}}%
\pgfpathcurveto{\pgfqpoint{2.227678in}{1.866943in}}{\pgfqpoint{2.219778in}{1.870216in}}{\pgfqpoint{2.211541in}{1.870216in}}%
\pgfpathcurveto{\pgfqpoint{2.203305in}{1.870216in}}{\pgfqpoint{2.195405in}{1.866943in}}{\pgfqpoint{2.189581in}{1.861120in}}%
\pgfpathcurveto{\pgfqpoint{2.183757in}{1.855296in}}{\pgfqpoint{2.180485in}{1.847396in}}{\pgfqpoint{2.180485in}{1.839159in}}%
\pgfpathcurveto{\pgfqpoint{2.180485in}{1.830923in}}{\pgfqpoint{2.183757in}{1.823023in}}{\pgfqpoint{2.189581in}{1.817199in}}%
\pgfpathcurveto{\pgfqpoint{2.195405in}{1.811375in}}{\pgfqpoint{2.203305in}{1.808103in}}{\pgfqpoint{2.211541in}{1.808103in}}%
\pgfpathclose%
\pgfusepath{stroke,fill}%
\end{pgfscope}%
\begin{pgfscope}%
\pgfpathrectangle{\pgfqpoint{0.100000in}{0.212622in}}{\pgfqpoint{3.696000in}{3.696000in}}%
\pgfusepath{clip}%
\pgfsetbuttcap%
\pgfsetroundjoin%
\definecolor{currentfill}{rgb}{0.121569,0.466667,0.705882}%
\pgfsetfillcolor{currentfill}%
\pgfsetfillopacity{0.991937}%
\pgfsetlinewidth{1.003750pt}%
\definecolor{currentstroke}{rgb}{0.121569,0.466667,0.705882}%
\pgfsetstrokecolor{currentstroke}%
\pgfsetstrokeopacity{0.991937}%
\pgfsetdash{}{0pt}%
\pgfpathmoveto{\pgfqpoint{2.475900in}{1.680445in}}%
\pgfpathcurveto{\pgfqpoint{2.484136in}{1.680445in}}{\pgfqpoint{2.492036in}{1.683717in}}{\pgfqpoint{2.497860in}{1.689541in}}%
\pgfpathcurveto{\pgfqpoint{2.503684in}{1.695365in}}{\pgfqpoint{2.506956in}{1.703265in}}{\pgfqpoint{2.506956in}{1.711501in}}%
\pgfpathcurveto{\pgfqpoint{2.506956in}{1.719738in}}{\pgfqpoint{2.503684in}{1.727638in}}{\pgfqpoint{2.497860in}{1.733462in}}%
\pgfpathcurveto{\pgfqpoint{2.492036in}{1.739285in}}{\pgfqpoint{2.484136in}{1.742558in}}{\pgfqpoint{2.475900in}{1.742558in}}%
\pgfpathcurveto{\pgfqpoint{2.467663in}{1.742558in}}{\pgfqpoint{2.459763in}{1.739285in}}{\pgfqpoint{2.453939in}{1.733462in}}%
\pgfpathcurveto{\pgfqpoint{2.448115in}{1.727638in}}{\pgfqpoint{2.444843in}{1.719738in}}{\pgfqpoint{2.444843in}{1.711501in}}%
\pgfpathcurveto{\pgfqpoint{2.444843in}{1.703265in}}{\pgfqpoint{2.448115in}{1.695365in}}{\pgfqpoint{2.453939in}{1.689541in}}%
\pgfpathcurveto{\pgfqpoint{2.459763in}{1.683717in}}{\pgfqpoint{2.467663in}{1.680445in}}{\pgfqpoint{2.475900in}{1.680445in}}%
\pgfpathclose%
\pgfusepath{stroke,fill}%
\end{pgfscope}%
\begin{pgfscope}%
\pgfpathrectangle{\pgfqpoint{0.100000in}{0.212622in}}{\pgfqpoint{3.696000in}{3.696000in}}%
\pgfusepath{clip}%
\pgfsetbuttcap%
\pgfsetroundjoin%
\definecolor{currentfill}{rgb}{0.121569,0.466667,0.705882}%
\pgfsetfillcolor{currentfill}%
\pgfsetfillopacity{0.992285}%
\pgfsetlinewidth{1.003750pt}%
\definecolor{currentstroke}{rgb}{0.121569,0.466667,0.705882}%
\pgfsetstrokecolor{currentstroke}%
\pgfsetstrokeopacity{0.992285}%
\pgfsetdash{}{0pt}%
\pgfpathmoveto{\pgfqpoint{2.218876in}{1.805680in}}%
\pgfpathcurveto{\pgfqpoint{2.227112in}{1.805680in}}{\pgfqpoint{2.235012in}{1.808952in}}{\pgfqpoint{2.240836in}{1.814776in}}%
\pgfpathcurveto{\pgfqpoint{2.246660in}{1.820600in}}{\pgfqpoint{2.249932in}{1.828500in}}{\pgfqpoint{2.249932in}{1.836736in}}%
\pgfpathcurveto{\pgfqpoint{2.249932in}{1.844972in}}{\pgfqpoint{2.246660in}{1.852872in}}{\pgfqpoint{2.240836in}{1.858696in}}%
\pgfpathcurveto{\pgfqpoint{2.235012in}{1.864520in}}{\pgfqpoint{2.227112in}{1.867793in}}{\pgfqpoint{2.218876in}{1.867793in}}%
\pgfpathcurveto{\pgfqpoint{2.210639in}{1.867793in}}{\pgfqpoint{2.202739in}{1.864520in}}{\pgfqpoint{2.196915in}{1.858696in}}%
\pgfpathcurveto{\pgfqpoint{2.191092in}{1.852872in}}{\pgfqpoint{2.187819in}{1.844972in}}{\pgfqpoint{2.187819in}{1.836736in}}%
\pgfpathcurveto{\pgfqpoint{2.187819in}{1.828500in}}{\pgfqpoint{2.191092in}{1.820600in}}{\pgfqpoint{2.196915in}{1.814776in}}%
\pgfpathcurveto{\pgfqpoint{2.202739in}{1.808952in}}{\pgfqpoint{2.210639in}{1.805680in}}{\pgfqpoint{2.218876in}{1.805680in}}%
\pgfpathclose%
\pgfusepath{stroke,fill}%
\end{pgfscope}%
\begin{pgfscope}%
\pgfpathrectangle{\pgfqpoint{0.100000in}{0.212622in}}{\pgfqpoint{3.696000in}{3.696000in}}%
\pgfusepath{clip}%
\pgfsetbuttcap%
\pgfsetroundjoin%
\definecolor{currentfill}{rgb}{0.121569,0.466667,0.705882}%
\pgfsetfillcolor{currentfill}%
\pgfsetfillopacity{0.992578}%
\pgfsetlinewidth{1.003750pt}%
\definecolor{currentstroke}{rgb}{0.121569,0.466667,0.705882}%
\pgfsetstrokecolor{currentstroke}%
\pgfsetstrokeopacity{0.992578}%
\pgfsetdash{}{0pt}%
\pgfpathmoveto{\pgfqpoint{2.474837in}{1.679107in}}%
\pgfpathcurveto{\pgfqpoint{2.483073in}{1.679107in}}{\pgfqpoint{2.490973in}{1.682379in}}{\pgfqpoint{2.496797in}{1.688203in}}%
\pgfpathcurveto{\pgfqpoint{2.502621in}{1.694027in}}{\pgfqpoint{2.505893in}{1.701927in}}{\pgfqpoint{2.505893in}{1.710163in}}%
\pgfpathcurveto{\pgfqpoint{2.505893in}{1.718400in}}{\pgfqpoint{2.502621in}{1.726300in}}{\pgfqpoint{2.496797in}{1.732124in}}%
\pgfpathcurveto{\pgfqpoint{2.490973in}{1.737948in}}{\pgfqpoint{2.483073in}{1.741220in}}{\pgfqpoint{2.474837in}{1.741220in}}%
\pgfpathcurveto{\pgfqpoint{2.466600in}{1.741220in}}{\pgfqpoint{2.458700in}{1.737948in}}{\pgfqpoint{2.452876in}{1.732124in}}%
\pgfpathcurveto{\pgfqpoint{2.447052in}{1.726300in}}{\pgfqpoint{2.443780in}{1.718400in}}{\pgfqpoint{2.443780in}{1.710163in}}%
\pgfpathcurveto{\pgfqpoint{2.443780in}{1.701927in}}{\pgfqpoint{2.447052in}{1.694027in}}{\pgfqpoint{2.452876in}{1.688203in}}%
\pgfpathcurveto{\pgfqpoint{2.458700in}{1.682379in}}{\pgfqpoint{2.466600in}{1.679107in}}{\pgfqpoint{2.474837in}{1.679107in}}%
\pgfpathclose%
\pgfusepath{stroke,fill}%
\end{pgfscope}%
\begin{pgfscope}%
\pgfpathrectangle{\pgfqpoint{0.100000in}{0.212622in}}{\pgfqpoint{3.696000in}{3.696000in}}%
\pgfusepath{clip}%
\pgfsetbuttcap%
\pgfsetroundjoin%
\definecolor{currentfill}{rgb}{0.121569,0.466667,0.705882}%
\pgfsetfillcolor{currentfill}%
\pgfsetfillopacity{0.992912}%
\pgfsetlinewidth{1.003750pt}%
\definecolor{currentstroke}{rgb}{0.121569,0.466667,0.705882}%
\pgfsetstrokecolor{currentstroke}%
\pgfsetstrokeopacity{0.992912}%
\pgfsetdash{}{0pt}%
\pgfpathmoveto{\pgfqpoint{2.474216in}{1.678290in}}%
\pgfpathcurveto{\pgfqpoint{2.482452in}{1.678290in}}{\pgfqpoint{2.490352in}{1.681562in}}{\pgfqpoint{2.496176in}{1.687386in}}%
\pgfpathcurveto{\pgfqpoint{2.502000in}{1.693210in}}{\pgfqpoint{2.505273in}{1.701110in}}{\pgfqpoint{2.505273in}{1.709346in}}%
\pgfpathcurveto{\pgfqpoint{2.505273in}{1.717583in}}{\pgfqpoint{2.502000in}{1.725483in}}{\pgfqpoint{2.496176in}{1.731307in}}%
\pgfpathcurveto{\pgfqpoint{2.490352in}{1.737131in}}{\pgfqpoint{2.482452in}{1.740403in}}{\pgfqpoint{2.474216in}{1.740403in}}%
\pgfpathcurveto{\pgfqpoint{2.465980in}{1.740403in}}{\pgfqpoint{2.458080in}{1.737131in}}{\pgfqpoint{2.452256in}{1.731307in}}%
\pgfpathcurveto{\pgfqpoint{2.446432in}{1.725483in}}{\pgfqpoint{2.443160in}{1.717583in}}{\pgfqpoint{2.443160in}{1.709346in}}%
\pgfpathcurveto{\pgfqpoint{2.443160in}{1.701110in}}{\pgfqpoint{2.446432in}{1.693210in}}{\pgfqpoint{2.452256in}{1.687386in}}%
\pgfpathcurveto{\pgfqpoint{2.458080in}{1.681562in}}{\pgfqpoint{2.465980in}{1.678290in}}{\pgfqpoint{2.474216in}{1.678290in}}%
\pgfpathclose%
\pgfusepath{stroke,fill}%
\end{pgfscope}%
\begin{pgfscope}%
\pgfpathrectangle{\pgfqpoint{0.100000in}{0.212622in}}{\pgfqpoint{3.696000in}{3.696000in}}%
\pgfusepath{clip}%
\pgfsetbuttcap%
\pgfsetroundjoin%
\definecolor{currentfill}{rgb}{0.121569,0.466667,0.705882}%
\pgfsetfillcolor{currentfill}%
\pgfsetfillopacity{0.993108}%
\pgfsetlinewidth{1.003750pt}%
\definecolor{currentstroke}{rgb}{0.121569,0.466667,0.705882}%
\pgfsetstrokecolor{currentstroke}%
\pgfsetstrokeopacity{0.993108}%
\pgfsetdash{}{0pt}%
\pgfpathmoveto{\pgfqpoint{2.473869in}{1.677918in}}%
\pgfpathcurveto{\pgfqpoint{2.482105in}{1.677918in}}{\pgfqpoint{2.490005in}{1.681191in}}{\pgfqpoint{2.495829in}{1.687015in}}%
\pgfpathcurveto{\pgfqpoint{2.501653in}{1.692838in}}{\pgfqpoint{2.504926in}{1.700739in}}{\pgfqpoint{2.504926in}{1.708975in}}%
\pgfpathcurveto{\pgfqpoint{2.504926in}{1.717211in}}{\pgfqpoint{2.501653in}{1.725111in}}{\pgfqpoint{2.495829in}{1.730935in}}%
\pgfpathcurveto{\pgfqpoint{2.490005in}{1.736759in}}{\pgfqpoint{2.482105in}{1.740031in}}{\pgfqpoint{2.473869in}{1.740031in}}%
\pgfpathcurveto{\pgfqpoint{2.465633in}{1.740031in}}{\pgfqpoint{2.457733in}{1.736759in}}{\pgfqpoint{2.451909in}{1.730935in}}%
\pgfpathcurveto{\pgfqpoint{2.446085in}{1.725111in}}{\pgfqpoint{2.442813in}{1.717211in}}{\pgfqpoint{2.442813in}{1.708975in}}%
\pgfpathcurveto{\pgfqpoint{2.442813in}{1.700739in}}{\pgfqpoint{2.446085in}{1.692838in}}{\pgfqpoint{2.451909in}{1.687015in}}%
\pgfpathcurveto{\pgfqpoint{2.457733in}{1.681191in}}{\pgfqpoint{2.465633in}{1.677918in}}{\pgfqpoint{2.473869in}{1.677918in}}%
\pgfpathclose%
\pgfusepath{stroke,fill}%
\end{pgfscope}%
\begin{pgfscope}%
\pgfpathrectangle{\pgfqpoint{0.100000in}{0.212622in}}{\pgfqpoint{3.696000in}{3.696000in}}%
\pgfusepath{clip}%
\pgfsetbuttcap%
\pgfsetroundjoin%
\definecolor{currentfill}{rgb}{0.121569,0.466667,0.705882}%
\pgfsetfillcolor{currentfill}%
\pgfsetfillopacity{0.993219}%
\pgfsetlinewidth{1.003750pt}%
\definecolor{currentstroke}{rgb}{0.121569,0.466667,0.705882}%
\pgfsetstrokecolor{currentstroke}%
\pgfsetstrokeopacity{0.993219}%
\pgfsetdash{}{0pt}%
\pgfpathmoveto{\pgfqpoint{2.473690in}{1.677724in}}%
\pgfpathcurveto{\pgfqpoint{2.481926in}{1.677724in}}{\pgfqpoint{2.489826in}{1.680996in}}{\pgfqpoint{2.495650in}{1.686820in}}%
\pgfpathcurveto{\pgfqpoint{2.501474in}{1.692644in}}{\pgfqpoint{2.504746in}{1.700544in}}{\pgfqpoint{2.504746in}{1.708780in}}%
\pgfpathcurveto{\pgfqpoint{2.504746in}{1.717017in}}{\pgfqpoint{2.501474in}{1.724917in}}{\pgfqpoint{2.495650in}{1.730741in}}%
\pgfpathcurveto{\pgfqpoint{2.489826in}{1.736565in}}{\pgfqpoint{2.481926in}{1.739837in}}{\pgfqpoint{2.473690in}{1.739837in}}%
\pgfpathcurveto{\pgfqpoint{2.465454in}{1.739837in}}{\pgfqpoint{2.457554in}{1.736565in}}{\pgfqpoint{2.451730in}{1.730741in}}%
\pgfpathcurveto{\pgfqpoint{2.445906in}{1.724917in}}{\pgfqpoint{2.442633in}{1.717017in}}{\pgfqpoint{2.442633in}{1.708780in}}%
\pgfpathcurveto{\pgfqpoint{2.442633in}{1.700544in}}{\pgfqpoint{2.445906in}{1.692644in}}{\pgfqpoint{2.451730in}{1.686820in}}%
\pgfpathcurveto{\pgfqpoint{2.457554in}{1.680996in}}{\pgfqpoint{2.465454in}{1.677724in}}{\pgfqpoint{2.473690in}{1.677724in}}%
\pgfpathclose%
\pgfusepath{stroke,fill}%
\end{pgfscope}%
\begin{pgfscope}%
\pgfpathrectangle{\pgfqpoint{0.100000in}{0.212622in}}{\pgfqpoint{3.696000in}{3.696000in}}%
\pgfusepath{clip}%
\pgfsetbuttcap%
\pgfsetroundjoin%
\definecolor{currentfill}{rgb}{0.121569,0.466667,0.705882}%
\pgfsetfillcolor{currentfill}%
\pgfsetfillopacity{0.993279}%
\pgfsetlinewidth{1.003750pt}%
\definecolor{currentstroke}{rgb}{0.121569,0.466667,0.705882}%
\pgfsetstrokecolor{currentstroke}%
\pgfsetstrokeopacity{0.993279}%
\pgfsetdash{}{0pt}%
\pgfpathmoveto{\pgfqpoint{2.473584in}{1.677621in}}%
\pgfpathcurveto{\pgfqpoint{2.481820in}{1.677621in}}{\pgfqpoint{2.489720in}{1.680893in}}{\pgfqpoint{2.495544in}{1.686717in}}%
\pgfpathcurveto{\pgfqpoint{2.501368in}{1.692541in}}{\pgfqpoint{2.504640in}{1.700441in}}{\pgfqpoint{2.504640in}{1.708677in}}%
\pgfpathcurveto{\pgfqpoint{2.504640in}{1.716913in}}{\pgfqpoint{2.501368in}{1.724813in}}{\pgfqpoint{2.495544in}{1.730637in}}%
\pgfpathcurveto{\pgfqpoint{2.489720in}{1.736461in}}{\pgfqpoint{2.481820in}{1.739734in}}{\pgfqpoint{2.473584in}{1.739734in}}%
\pgfpathcurveto{\pgfqpoint{2.465348in}{1.739734in}}{\pgfqpoint{2.457448in}{1.736461in}}{\pgfqpoint{2.451624in}{1.730637in}}%
\pgfpathcurveto{\pgfqpoint{2.445800in}{1.724813in}}{\pgfqpoint{2.442527in}{1.716913in}}{\pgfqpoint{2.442527in}{1.708677in}}%
\pgfpathcurveto{\pgfqpoint{2.442527in}{1.700441in}}{\pgfqpoint{2.445800in}{1.692541in}}{\pgfqpoint{2.451624in}{1.686717in}}%
\pgfpathcurveto{\pgfqpoint{2.457448in}{1.680893in}}{\pgfqpoint{2.465348in}{1.677621in}}{\pgfqpoint{2.473584in}{1.677621in}}%
\pgfpathclose%
\pgfusepath{stroke,fill}%
\end{pgfscope}%
\begin{pgfscope}%
\pgfpathrectangle{\pgfqpoint{0.100000in}{0.212622in}}{\pgfqpoint{3.696000in}{3.696000in}}%
\pgfusepath{clip}%
\pgfsetbuttcap%
\pgfsetroundjoin%
\definecolor{currentfill}{rgb}{0.121569,0.466667,0.705882}%
\pgfsetfillcolor{currentfill}%
\pgfsetfillopacity{0.993313}%
\pgfsetlinewidth{1.003750pt}%
\definecolor{currentstroke}{rgb}{0.121569,0.466667,0.705882}%
\pgfsetstrokecolor{currentstroke}%
\pgfsetstrokeopacity{0.993313}%
\pgfsetdash{}{0pt}%
\pgfpathmoveto{\pgfqpoint{2.473524in}{1.677570in}}%
\pgfpathcurveto{\pgfqpoint{2.481760in}{1.677570in}}{\pgfqpoint{2.489660in}{1.680842in}}{\pgfqpoint{2.495484in}{1.686666in}}%
\pgfpathcurveto{\pgfqpoint{2.501308in}{1.692490in}}{\pgfqpoint{2.504580in}{1.700390in}}{\pgfqpoint{2.504580in}{1.708627in}}%
\pgfpathcurveto{\pgfqpoint{2.504580in}{1.716863in}}{\pgfqpoint{2.501308in}{1.724763in}}{\pgfqpoint{2.495484in}{1.730587in}}%
\pgfpathcurveto{\pgfqpoint{2.489660in}{1.736411in}}{\pgfqpoint{2.481760in}{1.739683in}}{\pgfqpoint{2.473524in}{1.739683in}}%
\pgfpathcurveto{\pgfqpoint{2.465288in}{1.739683in}}{\pgfqpoint{2.457388in}{1.736411in}}{\pgfqpoint{2.451564in}{1.730587in}}%
\pgfpathcurveto{\pgfqpoint{2.445740in}{1.724763in}}{\pgfqpoint{2.442467in}{1.716863in}}{\pgfqpoint{2.442467in}{1.708627in}}%
\pgfpathcurveto{\pgfqpoint{2.442467in}{1.700390in}}{\pgfqpoint{2.445740in}{1.692490in}}{\pgfqpoint{2.451564in}{1.686666in}}%
\pgfpathcurveto{\pgfqpoint{2.457388in}{1.680842in}}{\pgfqpoint{2.465288in}{1.677570in}}{\pgfqpoint{2.473524in}{1.677570in}}%
\pgfpathclose%
\pgfusepath{stroke,fill}%
\end{pgfscope}%
\begin{pgfscope}%
\pgfpathrectangle{\pgfqpoint{0.100000in}{0.212622in}}{\pgfqpoint{3.696000in}{3.696000in}}%
\pgfusepath{clip}%
\pgfsetbuttcap%
\pgfsetroundjoin%
\definecolor{currentfill}{rgb}{0.121569,0.466667,0.705882}%
\pgfsetfillcolor{currentfill}%
\pgfsetfillopacity{0.993333}%
\pgfsetlinewidth{1.003750pt}%
\definecolor{currentstroke}{rgb}{0.121569,0.466667,0.705882}%
\pgfsetstrokecolor{currentstroke}%
\pgfsetstrokeopacity{0.993333}%
\pgfsetdash{}{0pt}%
\pgfpathmoveto{\pgfqpoint{2.473488in}{1.677549in}}%
\pgfpathcurveto{\pgfqpoint{2.481724in}{1.677549in}}{\pgfqpoint{2.489624in}{1.680821in}}{\pgfqpoint{2.495448in}{1.686645in}}%
\pgfpathcurveto{\pgfqpoint{2.501272in}{1.692469in}}{\pgfqpoint{2.504544in}{1.700369in}}{\pgfqpoint{2.504544in}{1.708605in}}%
\pgfpathcurveto{\pgfqpoint{2.504544in}{1.716841in}}{\pgfqpoint{2.501272in}{1.724741in}}{\pgfqpoint{2.495448in}{1.730565in}}%
\pgfpathcurveto{\pgfqpoint{2.489624in}{1.736389in}}{\pgfqpoint{2.481724in}{1.739662in}}{\pgfqpoint{2.473488in}{1.739662in}}%
\pgfpathcurveto{\pgfqpoint{2.465251in}{1.739662in}}{\pgfqpoint{2.457351in}{1.736389in}}{\pgfqpoint{2.451527in}{1.730565in}}%
\pgfpathcurveto{\pgfqpoint{2.445703in}{1.724741in}}{\pgfqpoint{2.442431in}{1.716841in}}{\pgfqpoint{2.442431in}{1.708605in}}%
\pgfpathcurveto{\pgfqpoint{2.442431in}{1.700369in}}{\pgfqpoint{2.445703in}{1.692469in}}{\pgfqpoint{2.451527in}{1.686645in}}%
\pgfpathcurveto{\pgfqpoint{2.457351in}{1.680821in}}{\pgfqpoint{2.465251in}{1.677549in}}{\pgfqpoint{2.473488in}{1.677549in}}%
\pgfpathclose%
\pgfusepath{stroke,fill}%
\end{pgfscope}%
\begin{pgfscope}%
\pgfpathrectangle{\pgfqpoint{0.100000in}{0.212622in}}{\pgfqpoint{3.696000in}{3.696000in}}%
\pgfusepath{clip}%
\pgfsetbuttcap%
\pgfsetroundjoin%
\definecolor{currentfill}{rgb}{0.121569,0.466667,0.705882}%
\pgfsetfillcolor{currentfill}%
\pgfsetfillopacity{0.993343}%
\pgfsetlinewidth{1.003750pt}%
\definecolor{currentstroke}{rgb}{0.121569,0.466667,0.705882}%
\pgfsetstrokecolor{currentstroke}%
\pgfsetstrokeopacity{0.993343}%
\pgfsetdash{}{0pt}%
\pgfpathmoveto{\pgfqpoint{2.473466in}{1.677535in}}%
\pgfpathcurveto{\pgfqpoint{2.481703in}{1.677535in}}{\pgfqpoint{2.489603in}{1.680808in}}{\pgfqpoint{2.495427in}{1.686632in}}%
\pgfpathcurveto{\pgfqpoint{2.501250in}{1.692456in}}{\pgfqpoint{2.504523in}{1.700356in}}{\pgfqpoint{2.504523in}{1.708592in}}%
\pgfpathcurveto{\pgfqpoint{2.504523in}{1.716828in}}{\pgfqpoint{2.501250in}{1.724728in}}{\pgfqpoint{2.495427in}{1.730552in}}%
\pgfpathcurveto{\pgfqpoint{2.489603in}{1.736376in}}{\pgfqpoint{2.481703in}{1.739648in}}{\pgfqpoint{2.473466in}{1.739648in}}%
\pgfpathcurveto{\pgfqpoint{2.465230in}{1.739648in}}{\pgfqpoint{2.457330in}{1.736376in}}{\pgfqpoint{2.451506in}{1.730552in}}%
\pgfpathcurveto{\pgfqpoint{2.445682in}{1.724728in}}{\pgfqpoint{2.442410in}{1.716828in}}{\pgfqpoint{2.442410in}{1.708592in}}%
\pgfpathcurveto{\pgfqpoint{2.442410in}{1.700356in}}{\pgfqpoint{2.445682in}{1.692456in}}{\pgfqpoint{2.451506in}{1.686632in}}%
\pgfpathcurveto{\pgfqpoint{2.457330in}{1.680808in}}{\pgfqpoint{2.465230in}{1.677535in}}{\pgfqpoint{2.473466in}{1.677535in}}%
\pgfpathclose%
\pgfusepath{stroke,fill}%
\end{pgfscope}%
\begin{pgfscope}%
\pgfpathrectangle{\pgfqpoint{0.100000in}{0.212622in}}{\pgfqpoint{3.696000in}{3.696000in}}%
\pgfusepath{clip}%
\pgfsetbuttcap%
\pgfsetroundjoin%
\definecolor{currentfill}{rgb}{0.121569,0.466667,0.705882}%
\pgfsetfillcolor{currentfill}%
\pgfsetfillopacity{0.993348}%
\pgfsetlinewidth{1.003750pt}%
\definecolor{currentstroke}{rgb}{0.121569,0.466667,0.705882}%
\pgfsetstrokecolor{currentstroke}%
\pgfsetstrokeopacity{0.993348}%
\pgfsetdash{}{0pt}%
\pgfpathmoveto{\pgfqpoint{2.473453in}{1.677529in}}%
\pgfpathcurveto{\pgfqpoint{2.481690in}{1.677529in}}{\pgfqpoint{2.489590in}{1.680802in}}{\pgfqpoint{2.495414in}{1.686625in}}%
\pgfpathcurveto{\pgfqpoint{2.501238in}{1.692449in}}{\pgfqpoint{2.504510in}{1.700349in}}{\pgfqpoint{2.504510in}{1.708586in}}%
\pgfpathcurveto{\pgfqpoint{2.504510in}{1.716822in}}{\pgfqpoint{2.501238in}{1.724722in}}{\pgfqpoint{2.495414in}{1.730546in}}%
\pgfpathcurveto{\pgfqpoint{2.489590in}{1.736370in}}{\pgfqpoint{2.481690in}{1.739642in}}{\pgfqpoint{2.473453in}{1.739642in}}%
\pgfpathcurveto{\pgfqpoint{2.465217in}{1.739642in}}{\pgfqpoint{2.457317in}{1.736370in}}{\pgfqpoint{2.451493in}{1.730546in}}%
\pgfpathcurveto{\pgfqpoint{2.445669in}{1.724722in}}{\pgfqpoint{2.442397in}{1.716822in}}{\pgfqpoint{2.442397in}{1.708586in}}%
\pgfpathcurveto{\pgfqpoint{2.442397in}{1.700349in}}{\pgfqpoint{2.445669in}{1.692449in}}{\pgfqpoint{2.451493in}{1.686625in}}%
\pgfpathcurveto{\pgfqpoint{2.457317in}{1.680802in}}{\pgfqpoint{2.465217in}{1.677529in}}{\pgfqpoint{2.473453in}{1.677529in}}%
\pgfpathclose%
\pgfusepath{stroke,fill}%
\end{pgfscope}%
\begin{pgfscope}%
\pgfpathrectangle{\pgfqpoint{0.100000in}{0.212622in}}{\pgfqpoint{3.696000in}{3.696000in}}%
\pgfusepath{clip}%
\pgfsetbuttcap%
\pgfsetroundjoin%
\definecolor{currentfill}{rgb}{0.121569,0.466667,0.705882}%
\pgfsetfillcolor{currentfill}%
\pgfsetfillopacity{0.993852}%
\pgfsetlinewidth{1.003750pt}%
\definecolor{currentstroke}{rgb}{0.121569,0.466667,0.705882}%
\pgfsetstrokecolor{currentstroke}%
\pgfsetstrokeopacity{0.993852}%
\pgfsetdash{}{0pt}%
\pgfpathmoveto{\pgfqpoint{2.232287in}{1.801579in}}%
\pgfpathcurveto{\pgfqpoint{2.240524in}{1.801579in}}{\pgfqpoint{2.248424in}{1.804852in}}{\pgfqpoint{2.254248in}{1.810676in}}%
\pgfpathcurveto{\pgfqpoint{2.260071in}{1.816499in}}{\pgfqpoint{2.263344in}{1.824400in}}{\pgfqpoint{2.263344in}{1.832636in}}%
\pgfpathcurveto{\pgfqpoint{2.263344in}{1.840872in}}{\pgfqpoint{2.260071in}{1.848772in}}{\pgfqpoint{2.254248in}{1.854596in}}%
\pgfpathcurveto{\pgfqpoint{2.248424in}{1.860420in}}{\pgfqpoint{2.240524in}{1.863692in}}{\pgfqpoint{2.232287in}{1.863692in}}%
\pgfpathcurveto{\pgfqpoint{2.224051in}{1.863692in}}{\pgfqpoint{2.216151in}{1.860420in}}{\pgfqpoint{2.210327in}{1.854596in}}%
\pgfpathcurveto{\pgfqpoint{2.204503in}{1.848772in}}{\pgfqpoint{2.201231in}{1.840872in}}{\pgfqpoint{2.201231in}{1.832636in}}%
\pgfpathcurveto{\pgfqpoint{2.201231in}{1.824400in}}{\pgfqpoint{2.204503in}{1.816499in}}{\pgfqpoint{2.210327in}{1.810676in}}%
\pgfpathcurveto{\pgfqpoint{2.216151in}{1.804852in}}{\pgfqpoint{2.224051in}{1.801579in}}{\pgfqpoint{2.232287in}{1.801579in}}%
\pgfpathclose%
\pgfusepath{stroke,fill}%
\end{pgfscope}%
\begin{pgfscope}%
\pgfpathrectangle{\pgfqpoint{0.100000in}{0.212622in}}{\pgfqpoint{3.696000in}{3.696000in}}%
\pgfusepath{clip}%
\pgfsetbuttcap%
\pgfsetroundjoin%
\definecolor{currentfill}{rgb}{0.121569,0.466667,0.705882}%
\pgfsetfillcolor{currentfill}%
\pgfsetfillopacity{0.994062}%
\pgfsetlinewidth{1.003750pt}%
\definecolor{currentstroke}{rgb}{0.121569,0.466667,0.705882}%
\pgfsetstrokecolor{currentstroke}%
\pgfsetstrokeopacity{0.994062}%
\pgfsetdash{}{0pt}%
\pgfpathmoveto{\pgfqpoint{2.471617in}{1.676714in}}%
\pgfpathcurveto{\pgfqpoint{2.479853in}{1.676714in}}{\pgfqpoint{2.487753in}{1.679987in}}{\pgfqpoint{2.493577in}{1.685810in}}%
\pgfpathcurveto{\pgfqpoint{2.499401in}{1.691634in}}{\pgfqpoint{2.502673in}{1.699534in}}{\pgfqpoint{2.502673in}{1.707771in}}%
\pgfpathcurveto{\pgfqpoint{2.502673in}{1.716007in}}{\pgfqpoint{2.499401in}{1.723907in}}{\pgfqpoint{2.493577in}{1.729731in}}%
\pgfpathcurveto{\pgfqpoint{2.487753in}{1.735555in}}{\pgfqpoint{2.479853in}{1.738827in}}{\pgfqpoint{2.471617in}{1.738827in}}%
\pgfpathcurveto{\pgfqpoint{2.463380in}{1.738827in}}{\pgfqpoint{2.455480in}{1.735555in}}{\pgfqpoint{2.449656in}{1.729731in}}%
\pgfpathcurveto{\pgfqpoint{2.443832in}{1.723907in}}{\pgfqpoint{2.440560in}{1.716007in}}{\pgfqpoint{2.440560in}{1.707771in}}%
\pgfpathcurveto{\pgfqpoint{2.440560in}{1.699534in}}{\pgfqpoint{2.443832in}{1.691634in}}{\pgfqpoint{2.449656in}{1.685810in}}%
\pgfpathcurveto{\pgfqpoint{2.455480in}{1.679987in}}{\pgfqpoint{2.463380in}{1.676714in}}{\pgfqpoint{2.471617in}{1.676714in}}%
\pgfpathclose%
\pgfusepath{stroke,fill}%
\end{pgfscope}%
\begin{pgfscope}%
\pgfpathrectangle{\pgfqpoint{0.100000in}{0.212622in}}{\pgfqpoint{3.696000in}{3.696000in}}%
\pgfusepath{clip}%
\pgfsetbuttcap%
\pgfsetroundjoin%
\definecolor{currentfill}{rgb}{0.121569,0.466667,0.705882}%
\pgfsetfillcolor{currentfill}%
\pgfsetfillopacity{0.994452}%
\pgfsetlinewidth{1.003750pt}%
\definecolor{currentstroke}{rgb}{0.121569,0.466667,0.705882}%
\pgfsetstrokecolor{currentstroke}%
\pgfsetstrokeopacity{0.994452}%
\pgfsetdash{}{0pt}%
\pgfpathmoveto{\pgfqpoint{2.470525in}{1.676375in}}%
\pgfpathcurveto{\pgfqpoint{2.478761in}{1.676375in}}{\pgfqpoint{2.486661in}{1.679647in}}{\pgfqpoint{2.492485in}{1.685471in}}%
\pgfpathcurveto{\pgfqpoint{2.498309in}{1.691295in}}{\pgfqpoint{2.501582in}{1.699195in}}{\pgfqpoint{2.501582in}{1.707432in}}%
\pgfpathcurveto{\pgfqpoint{2.501582in}{1.715668in}}{\pgfqpoint{2.498309in}{1.723568in}}{\pgfqpoint{2.492485in}{1.729392in}}%
\pgfpathcurveto{\pgfqpoint{2.486661in}{1.735216in}}{\pgfqpoint{2.478761in}{1.738488in}}{\pgfqpoint{2.470525in}{1.738488in}}%
\pgfpathcurveto{\pgfqpoint{2.462289in}{1.738488in}}{\pgfqpoint{2.454389in}{1.735216in}}{\pgfqpoint{2.448565in}{1.729392in}}%
\pgfpathcurveto{\pgfqpoint{2.442741in}{1.723568in}}{\pgfqpoint{2.439469in}{1.715668in}}{\pgfqpoint{2.439469in}{1.707432in}}%
\pgfpathcurveto{\pgfqpoint{2.439469in}{1.699195in}}{\pgfqpoint{2.442741in}{1.691295in}}{\pgfqpoint{2.448565in}{1.685471in}}%
\pgfpathcurveto{\pgfqpoint{2.454389in}{1.679647in}}{\pgfqpoint{2.462289in}{1.676375in}}{\pgfqpoint{2.470525in}{1.676375in}}%
\pgfpathclose%
\pgfusepath{stroke,fill}%
\end{pgfscope}%
\begin{pgfscope}%
\pgfpathrectangle{\pgfqpoint{0.100000in}{0.212622in}}{\pgfqpoint{3.696000in}{3.696000in}}%
\pgfusepath{clip}%
\pgfsetbuttcap%
\pgfsetroundjoin%
\definecolor{currentfill}{rgb}{0.121569,0.466667,0.705882}%
\pgfsetfillcolor{currentfill}%
\pgfsetfillopacity{0.994831}%
\pgfsetlinewidth{1.003750pt}%
\definecolor{currentstroke}{rgb}{0.121569,0.466667,0.705882}%
\pgfsetstrokecolor{currentstroke}%
\pgfsetstrokeopacity{0.994831}%
\pgfsetdash{}{0pt}%
\pgfpathmoveto{\pgfqpoint{2.243908in}{1.796076in}}%
\pgfpathcurveto{\pgfqpoint{2.252144in}{1.796076in}}{\pgfqpoint{2.260044in}{1.799348in}}{\pgfqpoint{2.265868in}{1.805172in}}%
\pgfpathcurveto{\pgfqpoint{2.271692in}{1.810996in}}{\pgfqpoint{2.274964in}{1.818896in}}{\pgfqpoint{2.274964in}{1.827132in}}%
\pgfpathcurveto{\pgfqpoint{2.274964in}{1.835368in}}{\pgfqpoint{2.271692in}{1.843268in}}{\pgfqpoint{2.265868in}{1.849092in}}%
\pgfpathcurveto{\pgfqpoint{2.260044in}{1.854916in}}{\pgfqpoint{2.252144in}{1.858189in}}{\pgfqpoint{2.243908in}{1.858189in}}%
\pgfpathcurveto{\pgfqpoint{2.235672in}{1.858189in}}{\pgfqpoint{2.227772in}{1.854916in}}{\pgfqpoint{2.221948in}{1.849092in}}%
\pgfpathcurveto{\pgfqpoint{2.216124in}{1.843268in}}{\pgfqpoint{2.212851in}{1.835368in}}{\pgfqpoint{2.212851in}{1.827132in}}%
\pgfpathcurveto{\pgfqpoint{2.212851in}{1.818896in}}{\pgfqpoint{2.216124in}{1.810996in}}{\pgfqpoint{2.221948in}{1.805172in}}%
\pgfpathcurveto{\pgfqpoint{2.227772in}{1.799348in}}{\pgfqpoint{2.235672in}{1.796076in}}{\pgfqpoint{2.243908in}{1.796076in}}%
\pgfpathclose%
\pgfusepath{stroke,fill}%
\end{pgfscope}%
\begin{pgfscope}%
\pgfpathrectangle{\pgfqpoint{0.100000in}{0.212622in}}{\pgfqpoint{3.696000in}{3.696000in}}%
\pgfusepath{clip}%
\pgfsetbuttcap%
\pgfsetroundjoin%
\definecolor{currentfill}{rgb}{0.121569,0.466667,0.705882}%
\pgfsetfillcolor{currentfill}%
\pgfsetfillopacity{0.994964}%
\pgfsetlinewidth{1.003750pt}%
\definecolor{currentstroke}{rgb}{0.121569,0.466667,0.705882}%
\pgfsetstrokecolor{currentstroke}%
\pgfsetstrokeopacity{0.994964}%
\pgfsetdash{}{0pt}%
\pgfpathmoveto{\pgfqpoint{2.468934in}{1.676082in}}%
\pgfpathcurveto{\pgfqpoint{2.477171in}{1.676082in}}{\pgfqpoint{2.485071in}{1.679354in}}{\pgfqpoint{2.490895in}{1.685178in}}%
\pgfpathcurveto{\pgfqpoint{2.496719in}{1.691002in}}{\pgfqpoint{2.499991in}{1.698902in}}{\pgfqpoint{2.499991in}{1.707139in}}%
\pgfpathcurveto{\pgfqpoint{2.499991in}{1.715375in}}{\pgfqpoint{2.496719in}{1.723275in}}{\pgfqpoint{2.490895in}{1.729099in}}%
\pgfpathcurveto{\pgfqpoint{2.485071in}{1.734923in}}{\pgfqpoint{2.477171in}{1.738195in}}{\pgfqpoint{2.468934in}{1.738195in}}%
\pgfpathcurveto{\pgfqpoint{2.460698in}{1.738195in}}{\pgfqpoint{2.452798in}{1.734923in}}{\pgfqpoint{2.446974in}{1.729099in}}%
\pgfpathcurveto{\pgfqpoint{2.441150in}{1.723275in}}{\pgfqpoint{2.437878in}{1.715375in}}{\pgfqpoint{2.437878in}{1.707139in}}%
\pgfpathcurveto{\pgfqpoint{2.437878in}{1.698902in}}{\pgfqpoint{2.441150in}{1.691002in}}{\pgfqpoint{2.446974in}{1.685178in}}%
\pgfpathcurveto{\pgfqpoint{2.452798in}{1.679354in}}{\pgfqpoint{2.460698in}{1.676082in}}{\pgfqpoint{2.468934in}{1.676082in}}%
\pgfpathclose%
\pgfusepath{stroke,fill}%
\end{pgfscope}%
\begin{pgfscope}%
\pgfpathrectangle{\pgfqpoint{0.100000in}{0.212622in}}{\pgfqpoint{3.696000in}{3.696000in}}%
\pgfusepath{clip}%
\pgfsetbuttcap%
\pgfsetroundjoin%
\definecolor{currentfill}{rgb}{0.121569,0.466667,0.705882}%
\pgfsetfillcolor{currentfill}%
\pgfsetfillopacity{0.995368}%
\pgfsetlinewidth{1.003750pt}%
\definecolor{currentstroke}{rgb}{0.121569,0.466667,0.705882}%
\pgfsetstrokecolor{currentstroke}%
\pgfsetstrokeopacity{0.995368}%
\pgfsetdash{}{0pt}%
\pgfpathmoveto{\pgfqpoint{2.252137in}{1.791530in}}%
\pgfpathcurveto{\pgfqpoint{2.260373in}{1.791530in}}{\pgfqpoint{2.268273in}{1.794802in}}{\pgfqpoint{2.274097in}{1.800626in}}%
\pgfpathcurveto{\pgfqpoint{2.279921in}{1.806450in}}{\pgfqpoint{2.283193in}{1.814350in}}{\pgfqpoint{2.283193in}{1.822586in}}%
\pgfpathcurveto{\pgfqpoint{2.283193in}{1.830823in}}{\pgfqpoint{2.279921in}{1.838723in}}{\pgfqpoint{2.274097in}{1.844547in}}%
\pgfpathcurveto{\pgfqpoint{2.268273in}{1.850371in}}{\pgfqpoint{2.260373in}{1.853643in}}{\pgfqpoint{2.252137in}{1.853643in}}%
\pgfpathcurveto{\pgfqpoint{2.243901in}{1.853643in}}{\pgfqpoint{2.236001in}{1.850371in}}{\pgfqpoint{2.230177in}{1.844547in}}%
\pgfpathcurveto{\pgfqpoint{2.224353in}{1.838723in}}{\pgfqpoint{2.221080in}{1.830823in}}{\pgfqpoint{2.221080in}{1.822586in}}%
\pgfpathcurveto{\pgfqpoint{2.221080in}{1.814350in}}{\pgfqpoint{2.224353in}{1.806450in}}{\pgfqpoint{2.230177in}{1.800626in}}%
\pgfpathcurveto{\pgfqpoint{2.236001in}{1.794802in}}{\pgfqpoint{2.243901in}{1.791530in}}{\pgfqpoint{2.252137in}{1.791530in}}%
\pgfpathclose%
\pgfusepath{stroke,fill}%
\end{pgfscope}%
\begin{pgfscope}%
\pgfpathrectangle{\pgfqpoint{0.100000in}{0.212622in}}{\pgfqpoint{3.696000in}{3.696000in}}%
\pgfusepath{clip}%
\pgfsetbuttcap%
\pgfsetroundjoin%
\definecolor{currentfill}{rgb}{0.121569,0.466667,0.705882}%
\pgfsetfillcolor{currentfill}%
\pgfsetfillopacity{0.995755}%
\pgfsetlinewidth{1.003750pt}%
\definecolor{currentstroke}{rgb}{0.121569,0.466667,0.705882}%
\pgfsetstrokecolor{currentstroke}%
\pgfsetstrokeopacity{0.995755}%
\pgfsetdash{}{0pt}%
\pgfpathmoveto{\pgfqpoint{2.258858in}{1.787682in}}%
\pgfpathcurveto{\pgfqpoint{2.267095in}{1.787682in}}{\pgfqpoint{2.274995in}{1.790954in}}{\pgfqpoint{2.280818in}{1.796778in}}%
\pgfpathcurveto{\pgfqpoint{2.286642in}{1.802602in}}{\pgfqpoint{2.289915in}{1.810502in}}{\pgfqpoint{2.289915in}{1.818738in}}%
\pgfpathcurveto{\pgfqpoint{2.289915in}{1.826975in}}{\pgfqpoint{2.286642in}{1.834875in}}{\pgfqpoint{2.280818in}{1.840699in}}%
\pgfpathcurveto{\pgfqpoint{2.274995in}{1.846522in}}{\pgfqpoint{2.267095in}{1.849795in}}{\pgfqpoint{2.258858in}{1.849795in}}%
\pgfpathcurveto{\pgfqpoint{2.250622in}{1.849795in}}{\pgfqpoint{2.242722in}{1.846522in}}{\pgfqpoint{2.236898in}{1.840699in}}%
\pgfpathcurveto{\pgfqpoint{2.231074in}{1.834875in}}{\pgfqpoint{2.227802in}{1.826975in}}{\pgfqpoint{2.227802in}{1.818738in}}%
\pgfpathcurveto{\pgfqpoint{2.227802in}{1.810502in}}{\pgfqpoint{2.231074in}{1.802602in}}{\pgfqpoint{2.236898in}{1.796778in}}%
\pgfpathcurveto{\pgfqpoint{2.242722in}{1.790954in}}{\pgfqpoint{2.250622in}{1.787682in}}{\pgfqpoint{2.258858in}{1.787682in}}%
\pgfpathclose%
\pgfusepath{stroke,fill}%
\end{pgfscope}%
\begin{pgfscope}%
\pgfpathrectangle{\pgfqpoint{0.100000in}{0.212622in}}{\pgfqpoint{3.696000in}{3.696000in}}%
\pgfusepath{clip}%
\pgfsetbuttcap%
\pgfsetroundjoin%
\definecolor{currentfill}{rgb}{0.121569,0.466667,0.705882}%
\pgfsetfillcolor{currentfill}%
\pgfsetfillopacity{0.995759}%
\pgfsetlinewidth{1.003750pt}%
\definecolor{currentstroke}{rgb}{0.121569,0.466667,0.705882}%
\pgfsetstrokecolor{currentstroke}%
\pgfsetstrokeopacity{0.995759}%
\pgfsetdash{}{0pt}%
\pgfpathmoveto{\pgfqpoint{2.466216in}{1.675871in}}%
\pgfpathcurveto{\pgfqpoint{2.474453in}{1.675871in}}{\pgfqpoint{2.482353in}{1.679143in}}{\pgfqpoint{2.488177in}{1.684967in}}%
\pgfpathcurveto{\pgfqpoint{2.494001in}{1.690791in}}{\pgfqpoint{2.497273in}{1.698691in}}{\pgfqpoint{2.497273in}{1.706927in}}%
\pgfpathcurveto{\pgfqpoint{2.497273in}{1.715163in}}{\pgfqpoint{2.494001in}{1.723063in}}{\pgfqpoint{2.488177in}{1.728887in}}%
\pgfpathcurveto{\pgfqpoint{2.482353in}{1.734711in}}{\pgfqpoint{2.474453in}{1.737984in}}{\pgfqpoint{2.466216in}{1.737984in}}%
\pgfpathcurveto{\pgfqpoint{2.457980in}{1.737984in}}{\pgfqpoint{2.450080in}{1.734711in}}{\pgfqpoint{2.444256in}{1.728887in}}%
\pgfpathcurveto{\pgfqpoint{2.438432in}{1.723063in}}{\pgfqpoint{2.435160in}{1.715163in}}{\pgfqpoint{2.435160in}{1.706927in}}%
\pgfpathcurveto{\pgfqpoint{2.435160in}{1.698691in}}{\pgfqpoint{2.438432in}{1.690791in}}{\pgfqpoint{2.444256in}{1.684967in}}%
\pgfpathcurveto{\pgfqpoint{2.450080in}{1.679143in}}{\pgfqpoint{2.457980in}{1.675871in}}{\pgfqpoint{2.466216in}{1.675871in}}%
\pgfpathclose%
\pgfusepath{stroke,fill}%
\end{pgfscope}%
\begin{pgfscope}%
\pgfpathrectangle{\pgfqpoint{0.100000in}{0.212622in}}{\pgfqpoint{3.696000in}{3.696000in}}%
\pgfusepath{clip}%
\pgfsetbuttcap%
\pgfsetroundjoin%
\definecolor{currentfill}{rgb}{0.121569,0.466667,0.705882}%
\pgfsetfillcolor{currentfill}%
\pgfsetfillopacity{0.996191}%
\pgfsetlinewidth{1.003750pt}%
\definecolor{currentstroke}{rgb}{0.121569,0.466667,0.705882}%
\pgfsetstrokecolor{currentstroke}%
\pgfsetstrokeopacity{0.996191}%
\pgfsetdash{}{0pt}%
\pgfpathmoveto{\pgfqpoint{2.464634in}{1.675927in}}%
\pgfpathcurveto{\pgfqpoint{2.472870in}{1.675927in}}{\pgfqpoint{2.480770in}{1.679199in}}{\pgfqpoint{2.486594in}{1.685023in}}%
\pgfpathcurveto{\pgfqpoint{2.492418in}{1.690847in}}{\pgfqpoint{2.495691in}{1.698747in}}{\pgfqpoint{2.495691in}{1.706984in}}%
\pgfpathcurveto{\pgfqpoint{2.495691in}{1.715220in}}{\pgfqpoint{2.492418in}{1.723120in}}{\pgfqpoint{2.486594in}{1.728944in}}%
\pgfpathcurveto{\pgfqpoint{2.480770in}{1.734768in}}{\pgfqpoint{2.472870in}{1.738040in}}{\pgfqpoint{2.464634in}{1.738040in}}%
\pgfpathcurveto{\pgfqpoint{2.456398in}{1.738040in}}{\pgfqpoint{2.448498in}{1.734768in}}{\pgfqpoint{2.442674in}{1.728944in}}%
\pgfpathcurveto{\pgfqpoint{2.436850in}{1.723120in}}{\pgfqpoint{2.433578in}{1.715220in}}{\pgfqpoint{2.433578in}{1.706984in}}%
\pgfpathcurveto{\pgfqpoint{2.433578in}{1.698747in}}{\pgfqpoint{2.436850in}{1.690847in}}{\pgfqpoint{2.442674in}{1.685023in}}%
\pgfpathcurveto{\pgfqpoint{2.448498in}{1.679199in}}{\pgfqpoint{2.456398in}{1.675927in}}{\pgfqpoint{2.464634in}{1.675927in}}%
\pgfpathclose%
\pgfusepath{stroke,fill}%
\end{pgfscope}%
\begin{pgfscope}%
\pgfpathrectangle{\pgfqpoint{0.100000in}{0.212622in}}{\pgfqpoint{3.696000in}{3.696000in}}%
\pgfusepath{clip}%
\pgfsetbuttcap%
\pgfsetroundjoin%
\definecolor{currentfill}{rgb}{0.121569,0.466667,0.705882}%
\pgfsetfillcolor{currentfill}%
\pgfsetfillopacity{0.996305}%
\pgfsetlinewidth{1.003750pt}%
\definecolor{currentstroke}{rgb}{0.121569,0.466667,0.705882}%
\pgfsetstrokecolor{currentstroke}%
\pgfsetstrokeopacity{0.996305}%
\pgfsetdash{}{0pt}%
\pgfpathmoveto{\pgfqpoint{2.271061in}{1.779695in}}%
\pgfpathcurveto{\pgfqpoint{2.279297in}{1.779695in}}{\pgfqpoint{2.287197in}{1.782968in}}{\pgfqpoint{2.293021in}{1.788792in}}%
\pgfpathcurveto{\pgfqpoint{2.298845in}{1.794616in}}{\pgfqpoint{2.302117in}{1.802516in}}{\pgfqpoint{2.302117in}{1.810752in}}%
\pgfpathcurveto{\pgfqpoint{2.302117in}{1.818988in}}{\pgfqpoint{2.298845in}{1.826888in}}{\pgfqpoint{2.293021in}{1.832712in}}%
\pgfpathcurveto{\pgfqpoint{2.287197in}{1.838536in}}{\pgfqpoint{2.279297in}{1.841808in}}{\pgfqpoint{2.271061in}{1.841808in}}%
\pgfpathcurveto{\pgfqpoint{2.262825in}{1.841808in}}{\pgfqpoint{2.254924in}{1.838536in}}{\pgfqpoint{2.249101in}{1.832712in}}%
\pgfpathcurveto{\pgfqpoint{2.243277in}{1.826888in}}{\pgfqpoint{2.240004in}{1.818988in}}{\pgfqpoint{2.240004in}{1.810752in}}%
\pgfpathcurveto{\pgfqpoint{2.240004in}{1.802516in}}{\pgfqpoint{2.243277in}{1.794616in}}{\pgfqpoint{2.249101in}{1.788792in}}%
\pgfpathcurveto{\pgfqpoint{2.254924in}{1.782968in}}{\pgfqpoint{2.262825in}{1.779695in}}{\pgfqpoint{2.271061in}{1.779695in}}%
\pgfpathclose%
\pgfusepath{stroke,fill}%
\end{pgfscope}%
\begin{pgfscope}%
\pgfpathrectangle{\pgfqpoint{0.100000in}{0.212622in}}{\pgfqpoint{3.696000in}{3.696000in}}%
\pgfusepath{clip}%
\pgfsetbuttcap%
\pgfsetroundjoin%
\definecolor{currentfill}{rgb}{0.121569,0.466667,0.705882}%
\pgfsetfillcolor{currentfill}%
\pgfsetfillopacity{0.996731}%
\pgfsetlinewidth{1.003750pt}%
\definecolor{currentstroke}{rgb}{0.121569,0.466667,0.705882}%
\pgfsetstrokecolor{currentstroke}%
\pgfsetstrokeopacity{0.996731}%
\pgfsetdash{}{0pt}%
\pgfpathmoveto{\pgfqpoint{2.462481in}{1.676253in}}%
\pgfpathcurveto{\pgfqpoint{2.470717in}{1.676253in}}{\pgfqpoint{2.478617in}{1.679526in}}{\pgfqpoint{2.484441in}{1.685350in}}%
\pgfpathcurveto{\pgfqpoint{2.490265in}{1.691174in}}{\pgfqpoint{2.493537in}{1.699074in}}{\pgfqpoint{2.493537in}{1.707310in}}%
\pgfpathcurveto{\pgfqpoint{2.493537in}{1.715546in}}{\pgfqpoint{2.490265in}{1.723446in}}{\pgfqpoint{2.484441in}{1.729270in}}%
\pgfpathcurveto{\pgfqpoint{2.478617in}{1.735094in}}{\pgfqpoint{2.470717in}{1.738366in}}{\pgfqpoint{2.462481in}{1.738366in}}%
\pgfpathcurveto{\pgfqpoint{2.454244in}{1.738366in}}{\pgfqpoint{2.446344in}{1.735094in}}{\pgfqpoint{2.440520in}{1.729270in}}%
\pgfpathcurveto{\pgfqpoint{2.434696in}{1.723446in}}{\pgfqpoint{2.431424in}{1.715546in}}{\pgfqpoint{2.431424in}{1.707310in}}%
\pgfpathcurveto{\pgfqpoint{2.431424in}{1.699074in}}{\pgfqpoint{2.434696in}{1.691174in}}{\pgfqpoint{2.440520in}{1.685350in}}%
\pgfpathcurveto{\pgfqpoint{2.446344in}{1.679526in}}{\pgfqpoint{2.454244in}{1.676253in}}{\pgfqpoint{2.462481in}{1.676253in}}%
\pgfpathclose%
\pgfusepath{stroke,fill}%
\end{pgfscope}%
\begin{pgfscope}%
\pgfpathrectangle{\pgfqpoint{0.100000in}{0.212622in}}{\pgfqpoint{3.696000in}{3.696000in}}%
\pgfusepath{clip}%
\pgfsetbuttcap%
\pgfsetroundjoin%
\definecolor{currentfill}{rgb}{0.121569,0.466667,0.705882}%
\pgfsetfillcolor{currentfill}%
\pgfsetfillopacity{0.997015}%
\pgfsetlinewidth{1.003750pt}%
\definecolor{currentstroke}{rgb}{0.121569,0.466667,0.705882}%
\pgfsetstrokecolor{currentstroke}%
\pgfsetstrokeopacity{0.997015}%
\pgfsetdash{}{0pt}%
\pgfpathmoveto{\pgfqpoint{2.461235in}{1.676553in}}%
\pgfpathcurveto{\pgfqpoint{2.469471in}{1.676553in}}{\pgfqpoint{2.477372in}{1.679825in}}{\pgfqpoint{2.483195in}{1.685649in}}%
\pgfpathcurveto{\pgfqpoint{2.489019in}{1.691473in}}{\pgfqpoint{2.492292in}{1.699373in}}{\pgfqpoint{2.492292in}{1.707609in}}%
\pgfpathcurveto{\pgfqpoint{2.492292in}{1.715845in}}{\pgfqpoint{2.489019in}{1.723745in}}{\pgfqpoint{2.483195in}{1.729569in}}%
\pgfpathcurveto{\pgfqpoint{2.477372in}{1.735393in}}{\pgfqpoint{2.469471in}{1.738666in}}{\pgfqpoint{2.461235in}{1.738666in}}%
\pgfpathcurveto{\pgfqpoint{2.452999in}{1.738666in}}{\pgfqpoint{2.445099in}{1.735393in}}{\pgfqpoint{2.439275in}{1.729569in}}%
\pgfpathcurveto{\pgfqpoint{2.433451in}{1.723745in}}{\pgfqpoint{2.430179in}{1.715845in}}{\pgfqpoint{2.430179in}{1.707609in}}%
\pgfpathcurveto{\pgfqpoint{2.430179in}{1.699373in}}{\pgfqpoint{2.433451in}{1.691473in}}{\pgfqpoint{2.439275in}{1.685649in}}%
\pgfpathcurveto{\pgfqpoint{2.445099in}{1.679825in}}{\pgfqpoint{2.452999in}{1.676553in}}{\pgfqpoint{2.461235in}{1.676553in}}%
\pgfpathclose%
\pgfusepath{stroke,fill}%
\end{pgfscope}%
\begin{pgfscope}%
\pgfpathrectangle{\pgfqpoint{0.100000in}{0.212622in}}{\pgfqpoint{3.696000in}{3.696000in}}%
\pgfusepath{clip}%
\pgfsetbuttcap%
\pgfsetroundjoin%
\definecolor{currentfill}{rgb}{0.121569,0.466667,0.705882}%
\pgfsetfillcolor{currentfill}%
\pgfsetfillopacity{0.997161}%
\pgfsetlinewidth{1.003750pt}%
\definecolor{currentstroke}{rgb}{0.121569,0.466667,0.705882}%
\pgfsetstrokecolor{currentstroke}%
\pgfsetstrokeopacity{0.997161}%
\pgfsetdash{}{0pt}%
\pgfpathmoveto{\pgfqpoint{2.460520in}{1.676773in}}%
\pgfpathcurveto{\pgfqpoint{2.468756in}{1.676773in}}{\pgfqpoint{2.476656in}{1.680045in}}{\pgfqpoint{2.482480in}{1.685869in}}%
\pgfpathcurveto{\pgfqpoint{2.488304in}{1.691693in}}{\pgfqpoint{2.491576in}{1.699593in}}{\pgfqpoint{2.491576in}{1.707830in}}%
\pgfpathcurveto{\pgfqpoint{2.491576in}{1.716066in}}{\pgfqpoint{2.488304in}{1.723966in}}{\pgfqpoint{2.482480in}{1.729790in}}%
\pgfpathcurveto{\pgfqpoint{2.476656in}{1.735614in}}{\pgfqpoint{2.468756in}{1.738886in}}{\pgfqpoint{2.460520in}{1.738886in}}%
\pgfpathcurveto{\pgfqpoint{2.452283in}{1.738886in}}{\pgfqpoint{2.444383in}{1.735614in}}{\pgfqpoint{2.438559in}{1.729790in}}%
\pgfpathcurveto{\pgfqpoint{2.432735in}{1.723966in}}{\pgfqpoint{2.429463in}{1.716066in}}{\pgfqpoint{2.429463in}{1.707830in}}%
\pgfpathcurveto{\pgfqpoint{2.429463in}{1.699593in}}{\pgfqpoint{2.432735in}{1.691693in}}{\pgfqpoint{2.438559in}{1.685869in}}%
\pgfpathcurveto{\pgfqpoint{2.444383in}{1.680045in}}{\pgfqpoint{2.452283in}{1.676773in}}{\pgfqpoint{2.460520in}{1.676773in}}%
\pgfpathclose%
\pgfusepath{stroke,fill}%
\end{pgfscope}%
\begin{pgfscope}%
\pgfpathrectangle{\pgfqpoint{0.100000in}{0.212622in}}{\pgfqpoint{3.696000in}{3.696000in}}%
\pgfusepath{clip}%
\pgfsetbuttcap%
\pgfsetroundjoin%
\definecolor{currentfill}{rgb}{0.121569,0.466667,0.705882}%
\pgfsetfillcolor{currentfill}%
\pgfsetfillopacity{0.997236}%
\pgfsetlinewidth{1.003750pt}%
\definecolor{currentstroke}{rgb}{0.121569,0.466667,0.705882}%
\pgfsetstrokecolor{currentstroke}%
\pgfsetstrokeopacity{0.997236}%
\pgfsetdash{}{0pt}%
\pgfpathmoveto{\pgfqpoint{2.460112in}{1.676928in}}%
\pgfpathcurveto{\pgfqpoint{2.468349in}{1.676928in}}{\pgfqpoint{2.476249in}{1.680201in}}{\pgfqpoint{2.482073in}{1.686024in}}%
\pgfpathcurveto{\pgfqpoint{2.487897in}{1.691848in}}{\pgfqpoint{2.491169in}{1.699748in}}{\pgfqpoint{2.491169in}{1.707985in}}%
\pgfpathcurveto{\pgfqpoint{2.491169in}{1.716221in}}{\pgfqpoint{2.487897in}{1.724121in}}{\pgfqpoint{2.482073in}{1.729945in}}%
\pgfpathcurveto{\pgfqpoint{2.476249in}{1.735769in}}{\pgfqpoint{2.468349in}{1.739041in}}{\pgfqpoint{2.460112in}{1.739041in}}%
\pgfpathcurveto{\pgfqpoint{2.451876in}{1.739041in}}{\pgfqpoint{2.443976in}{1.735769in}}{\pgfqpoint{2.438152in}{1.729945in}}%
\pgfpathcurveto{\pgfqpoint{2.432328in}{1.724121in}}{\pgfqpoint{2.429056in}{1.716221in}}{\pgfqpoint{2.429056in}{1.707985in}}%
\pgfpathcurveto{\pgfqpoint{2.429056in}{1.699748in}}{\pgfqpoint{2.432328in}{1.691848in}}{\pgfqpoint{2.438152in}{1.686024in}}%
\pgfpathcurveto{\pgfqpoint{2.443976in}{1.680201in}}{\pgfqpoint{2.451876in}{1.676928in}}{\pgfqpoint{2.460112in}{1.676928in}}%
\pgfpathclose%
\pgfusepath{stroke,fill}%
\end{pgfscope}%
\begin{pgfscope}%
\pgfpathrectangle{\pgfqpoint{0.100000in}{0.212622in}}{\pgfqpoint{3.696000in}{3.696000in}}%
\pgfusepath{clip}%
\pgfsetbuttcap%
\pgfsetroundjoin%
\definecolor{currentfill}{rgb}{0.121569,0.466667,0.705882}%
\pgfsetfillcolor{currentfill}%
\pgfsetfillopacity{0.997274}%
\pgfsetlinewidth{1.003750pt}%
\definecolor{currentstroke}{rgb}{0.121569,0.466667,0.705882}%
\pgfsetstrokecolor{currentstroke}%
\pgfsetstrokeopacity{0.997274}%
\pgfsetdash{}{0pt}%
\pgfpathmoveto{\pgfqpoint{2.459882in}{1.677033in}}%
\pgfpathcurveto{\pgfqpoint{2.468119in}{1.677033in}}{\pgfqpoint{2.476019in}{1.680305in}}{\pgfqpoint{2.481843in}{1.686129in}}%
\pgfpathcurveto{\pgfqpoint{2.487667in}{1.691953in}}{\pgfqpoint{2.490939in}{1.699853in}}{\pgfqpoint{2.490939in}{1.708089in}}%
\pgfpathcurveto{\pgfqpoint{2.490939in}{1.716326in}}{\pgfqpoint{2.487667in}{1.724226in}}{\pgfqpoint{2.481843in}{1.730050in}}%
\pgfpathcurveto{\pgfqpoint{2.476019in}{1.735873in}}{\pgfqpoint{2.468119in}{1.739146in}}{\pgfqpoint{2.459882in}{1.739146in}}%
\pgfpathcurveto{\pgfqpoint{2.451646in}{1.739146in}}{\pgfqpoint{2.443746in}{1.735873in}}{\pgfqpoint{2.437922in}{1.730050in}}%
\pgfpathcurveto{\pgfqpoint{2.432098in}{1.724226in}}{\pgfqpoint{2.428826in}{1.716326in}}{\pgfqpoint{2.428826in}{1.708089in}}%
\pgfpathcurveto{\pgfqpoint{2.428826in}{1.699853in}}{\pgfqpoint{2.432098in}{1.691953in}}{\pgfqpoint{2.437922in}{1.686129in}}%
\pgfpathcurveto{\pgfqpoint{2.443746in}{1.680305in}}{\pgfqpoint{2.451646in}{1.677033in}}{\pgfqpoint{2.459882in}{1.677033in}}%
\pgfpathclose%
\pgfusepath{stroke,fill}%
\end{pgfscope}%
\begin{pgfscope}%
\pgfpathrectangle{\pgfqpoint{0.100000in}{0.212622in}}{\pgfqpoint{3.696000in}{3.696000in}}%
\pgfusepath{clip}%
\pgfsetbuttcap%
\pgfsetroundjoin%
\definecolor{currentfill}{rgb}{0.121569,0.466667,0.705882}%
\pgfsetfillcolor{currentfill}%
\pgfsetfillopacity{0.997293}%
\pgfsetlinewidth{1.003750pt}%
\definecolor{currentstroke}{rgb}{0.121569,0.466667,0.705882}%
\pgfsetstrokecolor{currentstroke}%
\pgfsetstrokeopacity{0.997293}%
\pgfsetdash{}{0pt}%
\pgfpathmoveto{\pgfqpoint{2.459753in}{1.677098in}}%
\pgfpathcurveto{\pgfqpoint{2.467989in}{1.677098in}}{\pgfqpoint{2.475889in}{1.680370in}}{\pgfqpoint{2.481713in}{1.686194in}}%
\pgfpathcurveto{\pgfqpoint{2.487537in}{1.692018in}}{\pgfqpoint{2.490810in}{1.699918in}}{\pgfqpoint{2.490810in}{1.708155in}}%
\pgfpathcurveto{\pgfqpoint{2.490810in}{1.716391in}}{\pgfqpoint{2.487537in}{1.724291in}}{\pgfqpoint{2.481713in}{1.730115in}}%
\pgfpathcurveto{\pgfqpoint{2.475889in}{1.735939in}}{\pgfqpoint{2.467989in}{1.739211in}}{\pgfqpoint{2.459753in}{1.739211in}}%
\pgfpathcurveto{\pgfqpoint{2.451517in}{1.739211in}}{\pgfqpoint{2.443617in}{1.735939in}}{\pgfqpoint{2.437793in}{1.730115in}}%
\pgfpathcurveto{\pgfqpoint{2.431969in}{1.724291in}}{\pgfqpoint{2.428697in}{1.716391in}}{\pgfqpoint{2.428697in}{1.708155in}}%
\pgfpathcurveto{\pgfqpoint{2.428697in}{1.699918in}}{\pgfqpoint{2.431969in}{1.692018in}}{\pgfqpoint{2.437793in}{1.686194in}}%
\pgfpathcurveto{\pgfqpoint{2.443617in}{1.680370in}}{\pgfqpoint{2.451517in}{1.677098in}}{\pgfqpoint{2.459753in}{1.677098in}}%
\pgfpathclose%
\pgfusepath{stroke,fill}%
\end{pgfscope}%
\begin{pgfscope}%
\pgfpathrectangle{\pgfqpoint{0.100000in}{0.212622in}}{\pgfqpoint{3.696000in}{3.696000in}}%
\pgfusepath{clip}%
\pgfsetbuttcap%
\pgfsetroundjoin%
\definecolor{currentfill}{rgb}{0.121569,0.466667,0.705882}%
\pgfsetfillcolor{currentfill}%
\pgfsetfillopacity{0.997302}%
\pgfsetlinewidth{1.003750pt}%
\definecolor{currentstroke}{rgb}{0.121569,0.466667,0.705882}%
\pgfsetstrokecolor{currentstroke}%
\pgfsetstrokeopacity{0.997302}%
\pgfsetdash{}{0pt}%
\pgfpathmoveto{\pgfqpoint{2.459681in}{1.677136in}}%
\pgfpathcurveto{\pgfqpoint{2.467918in}{1.677136in}}{\pgfqpoint{2.475818in}{1.680409in}}{\pgfqpoint{2.481642in}{1.686232in}}%
\pgfpathcurveto{\pgfqpoint{2.487466in}{1.692056in}}{\pgfqpoint{2.490738in}{1.699956in}}{\pgfqpoint{2.490738in}{1.708193in}}%
\pgfpathcurveto{\pgfqpoint{2.490738in}{1.716429in}}{\pgfqpoint{2.487466in}{1.724329in}}{\pgfqpoint{2.481642in}{1.730153in}}%
\pgfpathcurveto{\pgfqpoint{2.475818in}{1.735977in}}{\pgfqpoint{2.467918in}{1.739249in}}{\pgfqpoint{2.459681in}{1.739249in}}%
\pgfpathcurveto{\pgfqpoint{2.451445in}{1.739249in}}{\pgfqpoint{2.443545in}{1.735977in}}{\pgfqpoint{2.437721in}{1.730153in}}%
\pgfpathcurveto{\pgfqpoint{2.431897in}{1.724329in}}{\pgfqpoint{2.428625in}{1.716429in}}{\pgfqpoint{2.428625in}{1.708193in}}%
\pgfpathcurveto{\pgfqpoint{2.428625in}{1.699956in}}{\pgfqpoint{2.431897in}{1.692056in}}{\pgfqpoint{2.437721in}{1.686232in}}%
\pgfpathcurveto{\pgfqpoint{2.443545in}{1.680409in}}{\pgfqpoint{2.451445in}{1.677136in}}{\pgfqpoint{2.459681in}{1.677136in}}%
\pgfpathclose%
\pgfusepath{stroke,fill}%
\end{pgfscope}%
\begin{pgfscope}%
\pgfpathrectangle{\pgfqpoint{0.100000in}{0.212622in}}{\pgfqpoint{3.696000in}{3.696000in}}%
\pgfusepath{clip}%
\pgfsetbuttcap%
\pgfsetroundjoin%
\definecolor{currentfill}{rgb}{0.121569,0.466667,0.705882}%
\pgfsetfillcolor{currentfill}%
\pgfsetfillopacity{0.997621}%
\pgfsetlinewidth{1.003750pt}%
\definecolor{currentstroke}{rgb}{0.121569,0.466667,0.705882}%
\pgfsetstrokecolor{currentstroke}%
\pgfsetstrokeopacity{0.997621}%
\pgfsetdash{}{0pt}%
\pgfpathmoveto{\pgfqpoint{2.456804in}{1.678798in}}%
\pgfpathcurveto{\pgfqpoint{2.465041in}{1.678798in}}{\pgfqpoint{2.472941in}{1.682070in}}{\pgfqpoint{2.478765in}{1.687894in}}%
\pgfpathcurveto{\pgfqpoint{2.484589in}{1.693718in}}{\pgfqpoint{2.487861in}{1.701618in}}{\pgfqpoint{2.487861in}{1.709854in}}%
\pgfpathcurveto{\pgfqpoint{2.487861in}{1.718090in}}{\pgfqpoint{2.484589in}{1.725990in}}{\pgfqpoint{2.478765in}{1.731814in}}%
\pgfpathcurveto{\pgfqpoint{2.472941in}{1.737638in}}{\pgfqpoint{2.465041in}{1.740911in}}{\pgfqpoint{2.456804in}{1.740911in}}%
\pgfpathcurveto{\pgfqpoint{2.448568in}{1.740911in}}{\pgfqpoint{2.440668in}{1.737638in}}{\pgfqpoint{2.434844in}{1.731814in}}%
\pgfpathcurveto{\pgfqpoint{2.429020in}{1.725990in}}{\pgfqpoint{2.425748in}{1.718090in}}{\pgfqpoint{2.425748in}{1.709854in}}%
\pgfpathcurveto{\pgfqpoint{2.425748in}{1.701618in}}{\pgfqpoint{2.429020in}{1.693718in}}{\pgfqpoint{2.434844in}{1.687894in}}%
\pgfpathcurveto{\pgfqpoint{2.440668in}{1.682070in}}{\pgfqpoint{2.448568in}{1.678798in}}{\pgfqpoint{2.456804in}{1.678798in}}%
\pgfpathclose%
\pgfusepath{stroke,fill}%
\end{pgfscope}%
\begin{pgfscope}%
\pgfpathrectangle{\pgfqpoint{0.100000in}{0.212622in}}{\pgfqpoint{3.696000in}{3.696000in}}%
\pgfusepath{clip}%
\pgfsetbuttcap%
\pgfsetroundjoin%
\definecolor{currentfill}{rgb}{0.121569,0.466667,0.705882}%
\pgfsetfillcolor{currentfill}%
\pgfsetfillopacity{0.997637}%
\pgfsetlinewidth{1.003750pt}%
\definecolor{currentstroke}{rgb}{0.121569,0.466667,0.705882}%
\pgfsetstrokecolor{currentstroke}%
\pgfsetstrokeopacity{0.997637}%
\pgfsetdash{}{0pt}%
\pgfpathmoveto{\pgfqpoint{2.293481in}{1.768105in}}%
\pgfpathcurveto{\pgfqpoint{2.301718in}{1.768105in}}{\pgfqpoint{2.309618in}{1.771377in}}{\pgfqpoint{2.315442in}{1.777201in}}%
\pgfpathcurveto{\pgfqpoint{2.321265in}{1.783025in}}{\pgfqpoint{2.324538in}{1.790925in}}{\pgfqpoint{2.324538in}{1.799162in}}%
\pgfpathcurveto{\pgfqpoint{2.324538in}{1.807398in}}{\pgfqpoint{2.321265in}{1.815298in}}{\pgfqpoint{2.315442in}{1.821122in}}%
\pgfpathcurveto{\pgfqpoint{2.309618in}{1.826946in}}{\pgfqpoint{2.301718in}{1.830218in}}{\pgfqpoint{2.293481in}{1.830218in}}%
\pgfpathcurveto{\pgfqpoint{2.285245in}{1.830218in}}{\pgfqpoint{2.277345in}{1.826946in}}{\pgfqpoint{2.271521in}{1.821122in}}%
\pgfpathcurveto{\pgfqpoint{2.265697in}{1.815298in}}{\pgfqpoint{2.262425in}{1.807398in}}{\pgfqpoint{2.262425in}{1.799162in}}%
\pgfpathcurveto{\pgfqpoint{2.262425in}{1.790925in}}{\pgfqpoint{2.265697in}{1.783025in}}{\pgfqpoint{2.271521in}{1.777201in}}%
\pgfpathcurveto{\pgfqpoint{2.277345in}{1.771377in}}{\pgfqpoint{2.285245in}{1.768105in}}{\pgfqpoint{2.293481in}{1.768105in}}%
\pgfpathclose%
\pgfusepath{stroke,fill}%
\end{pgfscope}%
\begin{pgfscope}%
\pgfpathrectangle{\pgfqpoint{0.100000in}{0.212622in}}{\pgfqpoint{3.696000in}{3.696000in}}%
\pgfusepath{clip}%
\pgfsetbuttcap%
\pgfsetroundjoin%
\definecolor{currentfill}{rgb}{0.121569,0.466667,0.705882}%
\pgfsetfillcolor{currentfill}%
\pgfsetfillopacity{0.998135}%
\pgfsetlinewidth{1.003750pt}%
\definecolor{currentstroke}{rgb}{0.121569,0.466667,0.705882}%
\pgfsetstrokecolor{currentstroke}%
\pgfsetstrokeopacity{0.998135}%
\pgfsetdash{}{0pt}%
\pgfpathmoveto{\pgfqpoint{2.451218in}{1.682238in}}%
\pgfpathcurveto{\pgfqpoint{2.459454in}{1.682238in}}{\pgfqpoint{2.467354in}{1.685510in}}{\pgfqpoint{2.473178in}{1.691334in}}%
\pgfpathcurveto{\pgfqpoint{2.479002in}{1.697158in}}{\pgfqpoint{2.482274in}{1.705058in}}{\pgfqpoint{2.482274in}{1.713295in}}%
\pgfpathcurveto{\pgfqpoint{2.482274in}{1.721531in}}{\pgfqpoint{2.479002in}{1.729431in}}{\pgfqpoint{2.473178in}{1.735255in}}%
\pgfpathcurveto{\pgfqpoint{2.467354in}{1.741079in}}{\pgfqpoint{2.459454in}{1.744351in}}{\pgfqpoint{2.451218in}{1.744351in}}%
\pgfpathcurveto{\pgfqpoint{2.442981in}{1.744351in}}{\pgfqpoint{2.435081in}{1.741079in}}{\pgfqpoint{2.429257in}{1.735255in}}%
\pgfpathcurveto{\pgfqpoint{2.423433in}{1.729431in}}{\pgfqpoint{2.420161in}{1.721531in}}{\pgfqpoint{2.420161in}{1.713295in}}%
\pgfpathcurveto{\pgfqpoint{2.420161in}{1.705058in}}{\pgfqpoint{2.423433in}{1.697158in}}{\pgfqpoint{2.429257in}{1.691334in}}%
\pgfpathcurveto{\pgfqpoint{2.435081in}{1.685510in}}{\pgfqpoint{2.442981in}{1.682238in}}{\pgfqpoint{2.451218in}{1.682238in}}%
\pgfpathclose%
\pgfusepath{stroke,fill}%
\end{pgfscope}%
\begin{pgfscope}%
\pgfpathrectangle{\pgfqpoint{0.100000in}{0.212622in}}{\pgfqpoint{3.696000in}{3.696000in}}%
\pgfusepath{clip}%
\pgfsetbuttcap%
\pgfsetroundjoin%
\definecolor{currentfill}{rgb}{0.121569,0.466667,0.705882}%
\pgfsetfillcolor{currentfill}%
\pgfsetfillopacity{0.998369}%
\pgfsetlinewidth{1.003750pt}%
\definecolor{currentstroke}{rgb}{0.121569,0.466667,0.705882}%
\pgfsetstrokecolor{currentstroke}%
\pgfsetstrokeopacity{0.998369}%
\pgfsetdash{}{0pt}%
\pgfpathmoveto{\pgfqpoint{2.313989in}{1.758148in}}%
\pgfpathcurveto{\pgfqpoint{2.322225in}{1.758148in}}{\pgfqpoint{2.330125in}{1.761420in}}{\pgfqpoint{2.335949in}{1.767244in}}%
\pgfpathcurveto{\pgfqpoint{2.341773in}{1.773068in}}{\pgfqpoint{2.345045in}{1.780968in}}{\pgfqpoint{2.345045in}{1.789205in}}%
\pgfpathcurveto{\pgfqpoint{2.345045in}{1.797441in}}{\pgfqpoint{2.341773in}{1.805341in}}{\pgfqpoint{2.335949in}{1.811165in}}%
\pgfpathcurveto{\pgfqpoint{2.330125in}{1.816989in}}{\pgfqpoint{2.322225in}{1.820261in}}{\pgfqpoint{2.313989in}{1.820261in}}%
\pgfpathcurveto{\pgfqpoint{2.305753in}{1.820261in}}{\pgfqpoint{2.297853in}{1.816989in}}{\pgfqpoint{2.292029in}{1.811165in}}%
\pgfpathcurveto{\pgfqpoint{2.286205in}{1.805341in}}{\pgfqpoint{2.282932in}{1.797441in}}{\pgfqpoint{2.282932in}{1.789205in}}%
\pgfpathcurveto{\pgfqpoint{2.282932in}{1.780968in}}{\pgfqpoint{2.286205in}{1.773068in}}{\pgfqpoint{2.292029in}{1.767244in}}%
\pgfpathcurveto{\pgfqpoint{2.297853in}{1.761420in}}{\pgfqpoint{2.305753in}{1.758148in}}{\pgfqpoint{2.313989in}{1.758148in}}%
\pgfpathclose%
\pgfusepath{stroke,fill}%
\end{pgfscope}%
\begin{pgfscope}%
\pgfpathrectangle{\pgfqpoint{0.100000in}{0.212622in}}{\pgfqpoint{3.696000in}{3.696000in}}%
\pgfusepath{clip}%
\pgfsetbuttcap%
\pgfsetroundjoin%
\definecolor{currentfill}{rgb}{0.121569,0.466667,0.705882}%
\pgfsetfillcolor{currentfill}%
\pgfsetfillopacity{0.998644}%
\pgfsetlinewidth{1.003750pt}%
\definecolor{currentstroke}{rgb}{0.121569,0.466667,0.705882}%
\pgfsetstrokecolor{currentstroke}%
\pgfsetstrokeopacity{0.998644}%
\pgfsetdash{}{0pt}%
\pgfpathmoveto{\pgfqpoint{2.443972in}{1.686635in}}%
\pgfpathcurveto{\pgfqpoint{2.452208in}{1.686635in}}{\pgfqpoint{2.460108in}{1.689907in}}{\pgfqpoint{2.465932in}{1.695731in}}%
\pgfpathcurveto{\pgfqpoint{2.471756in}{1.701555in}}{\pgfqpoint{2.475028in}{1.709455in}}{\pgfqpoint{2.475028in}{1.717691in}}%
\pgfpathcurveto{\pgfqpoint{2.475028in}{1.725928in}}{\pgfqpoint{2.471756in}{1.733828in}}{\pgfqpoint{2.465932in}{1.739651in}}%
\pgfpathcurveto{\pgfqpoint{2.460108in}{1.745475in}}{\pgfqpoint{2.452208in}{1.748748in}}{\pgfqpoint{2.443972in}{1.748748in}}%
\pgfpathcurveto{\pgfqpoint{2.435735in}{1.748748in}}{\pgfqpoint{2.427835in}{1.745475in}}{\pgfqpoint{2.422011in}{1.739651in}}%
\pgfpathcurveto{\pgfqpoint{2.416187in}{1.733828in}}{\pgfqpoint{2.412915in}{1.725928in}}{\pgfqpoint{2.412915in}{1.717691in}}%
\pgfpathcurveto{\pgfqpoint{2.412915in}{1.709455in}}{\pgfqpoint{2.416187in}{1.701555in}}{\pgfqpoint{2.422011in}{1.695731in}}%
\pgfpathcurveto{\pgfqpoint{2.427835in}{1.689907in}}{\pgfqpoint{2.435735in}{1.686635in}}{\pgfqpoint{2.443972in}{1.686635in}}%
\pgfpathclose%
\pgfusepath{stroke,fill}%
\end{pgfscope}%
\begin{pgfscope}%
\pgfpathrectangle{\pgfqpoint{0.100000in}{0.212622in}}{\pgfqpoint{3.696000in}{3.696000in}}%
\pgfusepath{clip}%
\pgfsetbuttcap%
\pgfsetroundjoin%
\definecolor{currentfill}{rgb}{0.121569,0.466667,0.705882}%
\pgfsetfillcolor{currentfill}%
\pgfsetfillopacity{0.999017}%
\pgfsetlinewidth{1.003750pt}%
\definecolor{currentstroke}{rgb}{0.121569,0.466667,0.705882}%
\pgfsetstrokecolor{currentstroke}%
\pgfsetstrokeopacity{0.999017}%
\pgfsetdash{}{0pt}%
\pgfpathmoveto{\pgfqpoint{2.330917in}{1.749835in}}%
\pgfpathcurveto{\pgfqpoint{2.339154in}{1.749835in}}{\pgfqpoint{2.347054in}{1.753107in}}{\pgfqpoint{2.352878in}{1.758931in}}%
\pgfpathcurveto{\pgfqpoint{2.358702in}{1.764755in}}{\pgfqpoint{2.361974in}{1.772655in}}{\pgfqpoint{2.361974in}{1.780891in}}%
\pgfpathcurveto{\pgfqpoint{2.361974in}{1.789128in}}{\pgfqpoint{2.358702in}{1.797028in}}{\pgfqpoint{2.352878in}{1.802852in}}%
\pgfpathcurveto{\pgfqpoint{2.347054in}{1.808676in}}{\pgfqpoint{2.339154in}{1.811948in}}{\pgfqpoint{2.330917in}{1.811948in}}%
\pgfpathcurveto{\pgfqpoint{2.322681in}{1.811948in}}{\pgfqpoint{2.314781in}{1.808676in}}{\pgfqpoint{2.308957in}{1.802852in}}%
\pgfpathcurveto{\pgfqpoint{2.303133in}{1.797028in}}{\pgfqpoint{2.299861in}{1.789128in}}{\pgfqpoint{2.299861in}{1.780891in}}%
\pgfpathcurveto{\pgfqpoint{2.299861in}{1.772655in}}{\pgfqpoint{2.303133in}{1.764755in}}{\pgfqpoint{2.308957in}{1.758931in}}%
\pgfpathcurveto{\pgfqpoint{2.314781in}{1.753107in}}{\pgfqpoint{2.322681in}{1.749835in}}{\pgfqpoint{2.330917in}{1.749835in}}%
\pgfpathclose%
\pgfusepath{stroke,fill}%
\end{pgfscope}%
\begin{pgfscope}%
\pgfpathrectangle{\pgfqpoint{0.100000in}{0.212622in}}{\pgfqpoint{3.696000in}{3.696000in}}%
\pgfusepath{clip}%
\pgfsetbuttcap%
\pgfsetroundjoin%
\definecolor{currentfill}{rgb}{0.121569,0.466667,0.705882}%
\pgfsetfillcolor{currentfill}%
\pgfsetfillopacity{0.999049}%
\pgfsetlinewidth{1.003750pt}%
\definecolor{currentstroke}{rgb}{0.121569,0.466667,0.705882}%
\pgfsetstrokecolor{currentstroke}%
\pgfsetstrokeopacity{0.999049}%
\pgfsetdash{}{0pt}%
\pgfpathmoveto{\pgfqpoint{2.435786in}{1.691304in}}%
\pgfpathcurveto{\pgfqpoint{2.444022in}{1.691304in}}{\pgfqpoint{2.451922in}{1.694576in}}{\pgfqpoint{2.457746in}{1.700400in}}%
\pgfpathcurveto{\pgfqpoint{2.463570in}{1.706224in}}{\pgfqpoint{2.466842in}{1.714124in}}{\pgfqpoint{2.466842in}{1.722360in}}%
\pgfpathcurveto{\pgfqpoint{2.466842in}{1.730596in}}{\pgfqpoint{2.463570in}{1.738496in}}{\pgfqpoint{2.457746in}{1.744320in}}%
\pgfpathcurveto{\pgfqpoint{2.451922in}{1.750144in}}{\pgfqpoint{2.444022in}{1.753417in}}{\pgfqpoint{2.435786in}{1.753417in}}%
\pgfpathcurveto{\pgfqpoint{2.427549in}{1.753417in}}{\pgfqpoint{2.419649in}{1.750144in}}{\pgfqpoint{2.413825in}{1.744320in}}%
\pgfpathcurveto{\pgfqpoint{2.408001in}{1.738496in}}{\pgfqpoint{2.404729in}{1.730596in}}{\pgfqpoint{2.404729in}{1.722360in}}%
\pgfpathcurveto{\pgfqpoint{2.404729in}{1.714124in}}{\pgfqpoint{2.408001in}{1.706224in}}{\pgfqpoint{2.413825in}{1.700400in}}%
\pgfpathcurveto{\pgfqpoint{2.419649in}{1.694576in}}{\pgfqpoint{2.427549in}{1.691304in}}{\pgfqpoint{2.435786in}{1.691304in}}%
\pgfpathclose%
\pgfusepath{stroke,fill}%
\end{pgfscope}%
\begin{pgfscope}%
\pgfpathrectangle{\pgfqpoint{0.100000in}{0.212622in}}{\pgfqpoint{3.696000in}{3.696000in}}%
\pgfusepath{clip}%
\pgfsetbuttcap%
\pgfsetroundjoin%
\definecolor{currentfill}{rgb}{0.121569,0.466667,0.705882}%
\pgfsetfillcolor{currentfill}%
\pgfsetfillopacity{0.999422}%
\pgfsetlinewidth{1.003750pt}%
\definecolor{currentstroke}{rgb}{0.121569,0.466667,0.705882}%
\pgfsetstrokecolor{currentstroke}%
\pgfsetstrokeopacity{0.999422}%
\pgfsetdash{}{0pt}%
\pgfpathmoveto{\pgfqpoint{2.426744in}{1.696526in}}%
\pgfpathcurveto{\pgfqpoint{2.434980in}{1.696526in}}{\pgfqpoint{2.442880in}{1.699799in}}{\pgfqpoint{2.448704in}{1.705622in}}%
\pgfpathcurveto{\pgfqpoint{2.454528in}{1.711446in}}{\pgfqpoint{2.457800in}{1.719346in}}{\pgfqpoint{2.457800in}{1.727583in}}%
\pgfpathcurveto{\pgfqpoint{2.457800in}{1.735819in}}{\pgfqpoint{2.454528in}{1.743719in}}{\pgfqpoint{2.448704in}{1.749543in}}%
\pgfpathcurveto{\pgfqpoint{2.442880in}{1.755367in}}{\pgfqpoint{2.434980in}{1.758639in}}{\pgfqpoint{2.426744in}{1.758639in}}%
\pgfpathcurveto{\pgfqpoint{2.418507in}{1.758639in}}{\pgfqpoint{2.410607in}{1.755367in}}{\pgfqpoint{2.404784in}{1.749543in}}%
\pgfpathcurveto{\pgfqpoint{2.398960in}{1.743719in}}{\pgfqpoint{2.395687in}{1.735819in}}{\pgfqpoint{2.395687in}{1.727583in}}%
\pgfpathcurveto{\pgfqpoint{2.395687in}{1.719346in}}{\pgfqpoint{2.398960in}{1.711446in}}{\pgfqpoint{2.404784in}{1.705622in}}%
\pgfpathcurveto{\pgfqpoint{2.410607in}{1.699799in}}{\pgfqpoint{2.418507in}{1.696526in}}{\pgfqpoint{2.426744in}{1.696526in}}%
\pgfpathclose%
\pgfusepath{stroke,fill}%
\end{pgfscope}%
\begin{pgfscope}%
\pgfpathrectangle{\pgfqpoint{0.100000in}{0.212622in}}{\pgfqpoint{3.696000in}{3.696000in}}%
\pgfusepath{clip}%
\pgfsetbuttcap%
\pgfsetroundjoin%
\definecolor{currentfill}{rgb}{0.121569,0.466667,0.705882}%
\pgfsetfillcolor{currentfill}%
\pgfsetfillopacity{0.999553}%
\pgfsetlinewidth{1.003750pt}%
\definecolor{currentstroke}{rgb}{0.121569,0.466667,0.705882}%
\pgfsetstrokecolor{currentstroke}%
\pgfsetstrokeopacity{0.999553}%
\pgfsetdash{}{0pt}%
\pgfpathmoveto{\pgfqpoint{2.347146in}{1.741653in}}%
\pgfpathcurveto{\pgfqpoint{2.355382in}{1.741653in}}{\pgfqpoint{2.363282in}{1.744926in}}{\pgfqpoint{2.369106in}{1.750750in}}%
\pgfpathcurveto{\pgfqpoint{2.374930in}{1.756573in}}{\pgfqpoint{2.378202in}{1.764474in}}{\pgfqpoint{2.378202in}{1.772710in}}%
\pgfpathcurveto{\pgfqpoint{2.378202in}{1.780946in}}{\pgfqpoint{2.374930in}{1.788846in}}{\pgfqpoint{2.369106in}{1.794670in}}%
\pgfpathcurveto{\pgfqpoint{2.363282in}{1.800494in}}{\pgfqpoint{2.355382in}{1.803766in}}{\pgfqpoint{2.347146in}{1.803766in}}%
\pgfpathcurveto{\pgfqpoint{2.338909in}{1.803766in}}{\pgfqpoint{2.331009in}{1.800494in}}{\pgfqpoint{2.325185in}{1.794670in}}%
\pgfpathcurveto{\pgfqpoint{2.319362in}{1.788846in}}{\pgfqpoint{2.316089in}{1.780946in}}{\pgfqpoint{2.316089in}{1.772710in}}%
\pgfpathcurveto{\pgfqpoint{2.316089in}{1.764474in}}{\pgfqpoint{2.319362in}{1.756573in}}{\pgfqpoint{2.325185in}{1.750750in}}%
\pgfpathcurveto{\pgfqpoint{2.331009in}{1.744926in}}{\pgfqpoint{2.338909in}{1.741653in}}{\pgfqpoint{2.347146in}{1.741653in}}%
\pgfpathclose%
\pgfusepath{stroke,fill}%
\end{pgfscope}%
\begin{pgfscope}%
\pgfpathrectangle{\pgfqpoint{0.100000in}{0.212622in}}{\pgfqpoint{3.696000in}{3.696000in}}%
\pgfusepath{clip}%
\pgfsetbuttcap%
\pgfsetroundjoin%
\definecolor{currentfill}{rgb}{0.121569,0.466667,0.705882}%
\pgfsetfillcolor{currentfill}%
\pgfsetfillopacity{0.999781}%
\pgfsetlinewidth{1.003750pt}%
\definecolor{currentstroke}{rgb}{0.121569,0.466667,0.705882}%
\pgfsetstrokecolor{currentstroke}%
\pgfsetstrokeopacity{0.999781}%
\pgfsetdash{}{0pt}%
\pgfpathmoveto{\pgfqpoint{2.362493in}{1.732201in}}%
\pgfpathcurveto{\pgfqpoint{2.370730in}{1.732201in}}{\pgfqpoint{2.378630in}{1.735473in}}{\pgfqpoint{2.384454in}{1.741297in}}%
\pgfpathcurveto{\pgfqpoint{2.390277in}{1.747121in}}{\pgfqpoint{2.393550in}{1.755021in}}{\pgfqpoint{2.393550in}{1.763258in}}%
\pgfpathcurveto{\pgfqpoint{2.393550in}{1.771494in}}{\pgfqpoint{2.390277in}{1.779394in}}{\pgfqpoint{2.384454in}{1.785218in}}%
\pgfpathcurveto{\pgfqpoint{2.378630in}{1.791042in}}{\pgfqpoint{2.370730in}{1.794314in}}{\pgfqpoint{2.362493in}{1.794314in}}%
\pgfpathcurveto{\pgfqpoint{2.354257in}{1.794314in}}{\pgfqpoint{2.346357in}{1.791042in}}{\pgfqpoint{2.340533in}{1.785218in}}%
\pgfpathcurveto{\pgfqpoint{2.334709in}{1.779394in}}{\pgfqpoint{2.331437in}{1.771494in}}{\pgfqpoint{2.331437in}{1.763258in}}%
\pgfpathcurveto{\pgfqpoint{2.331437in}{1.755021in}}{\pgfqpoint{2.334709in}{1.747121in}}{\pgfqpoint{2.340533in}{1.741297in}}%
\pgfpathcurveto{\pgfqpoint{2.346357in}{1.735473in}}{\pgfqpoint{2.354257in}{1.732201in}}{\pgfqpoint{2.362493in}{1.732201in}}%
\pgfpathclose%
\pgfusepath{stroke,fill}%
\end{pgfscope}%
\begin{pgfscope}%
\pgfpathrectangle{\pgfqpoint{0.100000in}{0.212622in}}{\pgfqpoint{3.696000in}{3.696000in}}%
\pgfusepath{clip}%
\pgfsetbuttcap%
\pgfsetroundjoin%
\definecolor{currentfill}{rgb}{0.121569,0.466667,0.705882}%
\pgfsetfillcolor{currentfill}%
\pgfsetfillopacity{0.999788}%
\pgfsetlinewidth{1.003750pt}%
\definecolor{currentstroke}{rgb}{0.121569,0.466667,0.705882}%
\pgfsetstrokecolor{currentstroke}%
\pgfsetstrokeopacity{0.999788}%
\pgfsetdash{}{0pt}%
\pgfpathmoveto{\pgfqpoint{2.414922in}{1.703179in}}%
\pgfpathcurveto{\pgfqpoint{2.423158in}{1.703179in}}{\pgfqpoint{2.431058in}{1.706451in}}{\pgfqpoint{2.436882in}{1.712275in}}%
\pgfpathcurveto{\pgfqpoint{2.442706in}{1.718099in}}{\pgfqpoint{2.445978in}{1.725999in}}{\pgfqpoint{2.445978in}{1.734235in}}%
\pgfpathcurveto{\pgfqpoint{2.445978in}{1.742472in}}{\pgfqpoint{2.442706in}{1.750372in}}{\pgfqpoint{2.436882in}{1.756196in}}%
\pgfpathcurveto{\pgfqpoint{2.431058in}{1.762019in}}{\pgfqpoint{2.423158in}{1.765292in}}{\pgfqpoint{2.414922in}{1.765292in}}%
\pgfpathcurveto{\pgfqpoint{2.406686in}{1.765292in}}{\pgfqpoint{2.398786in}{1.762019in}}{\pgfqpoint{2.392962in}{1.756196in}}%
\pgfpathcurveto{\pgfqpoint{2.387138in}{1.750372in}}{\pgfqpoint{2.383865in}{1.742472in}}{\pgfqpoint{2.383865in}{1.734235in}}%
\pgfpathcurveto{\pgfqpoint{2.383865in}{1.725999in}}{\pgfqpoint{2.387138in}{1.718099in}}{\pgfqpoint{2.392962in}{1.712275in}}%
\pgfpathcurveto{\pgfqpoint{2.398786in}{1.706451in}}{\pgfqpoint{2.406686in}{1.703179in}}{\pgfqpoint{2.414922in}{1.703179in}}%
\pgfpathclose%
\pgfusepath{stroke,fill}%
\end{pgfscope}%
\begin{pgfscope}%
\pgfpathrectangle{\pgfqpoint{0.100000in}{0.212622in}}{\pgfqpoint{3.696000in}{3.696000in}}%
\pgfusepath{clip}%
\pgfsetbuttcap%
\pgfsetroundjoin%
\definecolor{currentfill}{rgb}{0.121569,0.466667,0.705882}%
\pgfsetfillcolor{currentfill}%
\pgfsetfillopacity{0.999876}%
\pgfsetlinewidth{1.003750pt}%
\definecolor{currentstroke}{rgb}{0.121569,0.466667,0.705882}%
\pgfsetstrokecolor{currentstroke}%
\pgfsetstrokeopacity{0.999876}%
\pgfsetdash{}{0pt}%
\pgfpathmoveto{\pgfqpoint{2.399567in}{1.710932in}}%
\pgfpathcurveto{\pgfqpoint{2.407803in}{1.710932in}}{\pgfqpoint{2.415703in}{1.714204in}}{\pgfqpoint{2.421527in}{1.720028in}}%
\pgfpathcurveto{\pgfqpoint{2.427351in}{1.725852in}}{\pgfqpoint{2.430624in}{1.733752in}}{\pgfqpoint{2.430624in}{1.741988in}}%
\pgfpathcurveto{\pgfqpoint{2.430624in}{1.750225in}}{\pgfqpoint{2.427351in}{1.758125in}}{\pgfqpoint{2.421527in}{1.763949in}}%
\pgfpathcurveto{\pgfqpoint{2.415703in}{1.769773in}}{\pgfqpoint{2.407803in}{1.773045in}}{\pgfqpoint{2.399567in}{1.773045in}}%
\pgfpathcurveto{\pgfqpoint{2.391331in}{1.773045in}}{\pgfqpoint{2.383431in}{1.769773in}}{\pgfqpoint{2.377607in}{1.763949in}}%
\pgfpathcurveto{\pgfqpoint{2.371783in}{1.758125in}}{\pgfqpoint{2.368511in}{1.750225in}}{\pgfqpoint{2.368511in}{1.741988in}}%
\pgfpathcurveto{\pgfqpoint{2.368511in}{1.733752in}}{\pgfqpoint{2.371783in}{1.725852in}}{\pgfqpoint{2.377607in}{1.720028in}}%
\pgfpathcurveto{\pgfqpoint{2.383431in}{1.714204in}}{\pgfqpoint{2.391331in}{1.710932in}}{\pgfqpoint{2.399567in}{1.710932in}}%
\pgfpathclose%
\pgfusepath{stroke,fill}%
\end{pgfscope}%
\begin{pgfscope}%
\pgfpathrectangle{\pgfqpoint{0.100000in}{0.212622in}}{\pgfqpoint{3.696000in}{3.696000in}}%
\pgfusepath{clip}%
\pgfsetbuttcap%
\pgfsetroundjoin%
\definecolor{currentfill}{rgb}{0.121569,0.466667,0.705882}%
\pgfsetfillcolor{currentfill}%
\pgfsetfillopacity{0.999936}%
\pgfsetlinewidth{1.003750pt}%
\definecolor{currentstroke}{rgb}{0.121569,0.466667,0.705882}%
\pgfsetstrokecolor{currentstroke}%
\pgfsetstrokeopacity{0.999936}%
\pgfsetdash{}{0pt}%
\pgfpathmoveto{\pgfqpoint{2.374086in}{1.725881in}}%
\pgfpathcurveto{\pgfqpoint{2.382322in}{1.725881in}}{\pgfqpoint{2.390222in}{1.729153in}}{\pgfqpoint{2.396046in}{1.734977in}}%
\pgfpathcurveto{\pgfqpoint{2.401870in}{1.740801in}}{\pgfqpoint{2.405142in}{1.748701in}}{\pgfqpoint{2.405142in}{1.756937in}}%
\pgfpathcurveto{\pgfqpoint{2.405142in}{1.765174in}}{\pgfqpoint{2.401870in}{1.773074in}}{\pgfqpoint{2.396046in}{1.778898in}}%
\pgfpathcurveto{\pgfqpoint{2.390222in}{1.784722in}}{\pgfqpoint{2.382322in}{1.787994in}}{\pgfqpoint{2.374086in}{1.787994in}}%
\pgfpathcurveto{\pgfqpoint{2.365850in}{1.787994in}}{\pgfqpoint{2.357949in}{1.784722in}}{\pgfqpoint{2.352126in}{1.778898in}}%
\pgfpathcurveto{\pgfqpoint{2.346302in}{1.773074in}}{\pgfqpoint{2.343029in}{1.765174in}}{\pgfqpoint{2.343029in}{1.756937in}}%
\pgfpathcurveto{\pgfqpoint{2.343029in}{1.748701in}}{\pgfqpoint{2.346302in}{1.740801in}}{\pgfqpoint{2.352126in}{1.734977in}}%
\pgfpathcurveto{\pgfqpoint{2.357949in}{1.729153in}}{\pgfqpoint{2.365850in}{1.725881in}}{\pgfqpoint{2.374086in}{1.725881in}}%
\pgfpathclose%
\pgfusepath{stroke,fill}%
\end{pgfscope}%
\begin{pgfscope}%
\pgfpathrectangle{\pgfqpoint{0.100000in}{0.212622in}}{\pgfqpoint{3.696000in}{3.696000in}}%
\pgfusepath{clip}%
\pgfsetbuttcap%
\pgfsetroundjoin%
\definecolor{currentfill}{rgb}{0.121569,0.466667,0.705882}%
\pgfsetfillcolor{currentfill}%
\pgfsetlinewidth{1.003750pt}%
\definecolor{currentstroke}{rgb}{0.121569,0.466667,0.705882}%
\pgfsetstrokecolor{currentstroke}%
\pgfsetdash{}{0pt}%
\pgfpathmoveto{\pgfqpoint{2.383112in}{1.720943in}}%
\pgfpathcurveto{\pgfqpoint{2.391349in}{1.720943in}}{\pgfqpoint{2.399249in}{1.724216in}}{\pgfqpoint{2.405073in}{1.730040in}}%
\pgfpathcurveto{\pgfqpoint{2.410896in}{1.735863in}}{\pgfqpoint{2.414169in}{1.743764in}}{\pgfqpoint{2.414169in}{1.752000in}}%
\pgfpathcurveto{\pgfqpoint{2.414169in}{1.760236in}}{\pgfqpoint{2.410896in}{1.768136in}}{\pgfqpoint{2.405073in}{1.773960in}}%
\pgfpathcurveto{\pgfqpoint{2.399249in}{1.779784in}}{\pgfqpoint{2.391349in}{1.783056in}}{\pgfqpoint{2.383112in}{1.783056in}}%
\pgfpathcurveto{\pgfqpoint{2.374876in}{1.783056in}}{\pgfqpoint{2.366976in}{1.779784in}}{\pgfqpoint{2.361152in}{1.773960in}}%
\pgfpathcurveto{\pgfqpoint{2.355328in}{1.768136in}}{\pgfqpoint{2.352056in}{1.760236in}}{\pgfqpoint{2.352056in}{1.752000in}}%
\pgfpathcurveto{\pgfqpoint{2.352056in}{1.743764in}}{\pgfqpoint{2.355328in}{1.735863in}}{\pgfqpoint{2.361152in}{1.730040in}}%
\pgfpathcurveto{\pgfqpoint{2.366976in}{1.724216in}}{\pgfqpoint{2.374876in}{1.720943in}}{\pgfqpoint{2.383112in}{1.720943in}}%
\pgfpathclose%
\pgfusepath{stroke,fill}%
\end{pgfscope}%
\begin{pgfscope}%
\definecolor{textcolor}{rgb}{0.000000,0.000000,0.000000}%
\pgfsetstrokecolor{textcolor}%
\pgfsetfillcolor{textcolor}%
\pgftext[x=1.948000in,y=3.991956in,,base]{\color{textcolor}\rmfamily\fontsize{12.000000}{14.400000}\selectfont Madgwick}%
\end{pgfscope}%
\begin{pgfscope}%
\pgfsetbuttcap%
\pgfsetmiterjoin%
\definecolor{currentfill}{rgb}{1.000000,1.000000,1.000000}%
\pgfsetfillcolor{currentfill}%
\pgfsetfillopacity{0.800000}%
\pgfsetlinewidth{1.003750pt}%
\definecolor{currentstroke}{rgb}{0.800000,0.800000,0.800000}%
\pgfsetstrokecolor{currentstroke}%
\pgfsetstrokeopacity{0.800000}%
\pgfsetdash{}{0pt}%
\pgfpathmoveto{\pgfqpoint{2.104889in}{3.410289in}}%
\pgfpathlineto{\pgfqpoint{3.698778in}{3.410289in}}%
\pgfpathquadraticcurveto{\pgfqpoint{3.726556in}{3.410289in}}{\pgfqpoint{3.726556in}{3.438067in}}%
\pgfpathlineto{\pgfqpoint{3.726556in}{3.811400in}}%
\pgfpathquadraticcurveto{\pgfqpoint{3.726556in}{3.839178in}}{\pgfqpoint{3.698778in}{3.839178in}}%
\pgfpathlineto{\pgfqpoint{2.104889in}{3.839178in}}%
\pgfpathquadraticcurveto{\pgfqpoint{2.077111in}{3.839178in}}{\pgfqpoint{2.077111in}{3.811400in}}%
\pgfpathlineto{\pgfqpoint{2.077111in}{3.438067in}}%
\pgfpathquadraticcurveto{\pgfqpoint{2.077111in}{3.410289in}}{\pgfqpoint{2.104889in}{3.410289in}}%
\pgfpathclose%
\pgfusepath{stroke,fill}%
\end{pgfscope}%
\begin{pgfscope}%
\pgfsetrectcap%
\pgfsetroundjoin%
\pgfsetlinewidth{1.505625pt}%
\definecolor{currentstroke}{rgb}{0.121569,0.466667,0.705882}%
\pgfsetstrokecolor{currentstroke}%
\pgfsetdash{}{0pt}%
\pgfpathmoveto{\pgfqpoint{2.132667in}{3.735011in}}%
\pgfpathlineto{\pgfqpoint{2.410444in}{3.735011in}}%
\pgfusepath{stroke}%
\end{pgfscope}%
\begin{pgfscope}%
\definecolor{textcolor}{rgb}{0.000000,0.000000,0.000000}%
\pgfsetstrokecolor{textcolor}%
\pgfsetfillcolor{textcolor}%
\pgftext[x=2.521555in,y=3.686400in,left,base]{\color{textcolor}\rmfamily\fontsize{10.000000}{12.000000}\selectfont Ground truth}%
\end{pgfscope}%
\begin{pgfscope}%
\pgfsetbuttcap%
\pgfsetroundjoin%
\definecolor{currentfill}{rgb}{0.121569,0.466667,0.705882}%
\pgfsetfillcolor{currentfill}%
\pgfsetlinewidth{1.003750pt}%
\definecolor{currentstroke}{rgb}{0.121569,0.466667,0.705882}%
\pgfsetstrokecolor{currentstroke}%
\pgfsetdash{}{0pt}%
\pgfsys@defobject{currentmarker}{\pgfqpoint{-0.031056in}{-0.031056in}}{\pgfqpoint{0.031056in}{0.031056in}}{%
\pgfpathmoveto{\pgfqpoint{0.000000in}{-0.031056in}}%
\pgfpathcurveto{\pgfqpoint{0.008236in}{-0.031056in}}{\pgfqpoint{0.016136in}{-0.027784in}}{\pgfqpoint{0.021960in}{-0.021960in}}%
\pgfpathcurveto{\pgfqpoint{0.027784in}{-0.016136in}}{\pgfqpoint{0.031056in}{-0.008236in}}{\pgfqpoint{0.031056in}{0.000000in}}%
\pgfpathcurveto{\pgfqpoint{0.031056in}{0.008236in}}{\pgfqpoint{0.027784in}{0.016136in}}{\pgfqpoint{0.021960in}{0.021960in}}%
\pgfpathcurveto{\pgfqpoint{0.016136in}{0.027784in}}{\pgfqpoint{0.008236in}{0.031056in}}{\pgfqpoint{0.000000in}{0.031056in}}%
\pgfpathcurveto{\pgfqpoint{-0.008236in}{0.031056in}}{\pgfqpoint{-0.016136in}{0.027784in}}{\pgfqpoint{-0.021960in}{0.021960in}}%
\pgfpathcurveto{\pgfqpoint{-0.027784in}{0.016136in}}{\pgfqpoint{-0.031056in}{0.008236in}}{\pgfqpoint{-0.031056in}{0.000000in}}%
\pgfpathcurveto{\pgfqpoint{-0.031056in}{-0.008236in}}{\pgfqpoint{-0.027784in}{-0.016136in}}{\pgfqpoint{-0.021960in}{-0.021960in}}%
\pgfpathcurveto{\pgfqpoint{-0.016136in}{-0.027784in}}{\pgfqpoint{-0.008236in}{-0.031056in}}{\pgfqpoint{0.000000in}{-0.031056in}}%
\pgfpathclose%
\pgfusepath{stroke,fill}%
}%
\begin{pgfscope}%
\pgfsys@transformshift{2.271555in}{3.529248in}%
\pgfsys@useobject{currentmarker}{}%
\end{pgfscope}%
\end{pgfscope}%
\begin{pgfscope}%
\definecolor{textcolor}{rgb}{0.000000,0.000000,0.000000}%
\pgfsetstrokecolor{textcolor}%
\pgfsetfillcolor{textcolor}%
\pgftext[x=2.521555in,y=3.492789in,left,base]{\color{textcolor}\rmfamily\fontsize{10.000000}{12.000000}\selectfont Estimated position}%
\end{pgfscope}%
\end{pgfpicture}%
\makeatother%
\endgroup%
}
%         \caption{INS Hardware}
%         \label{fig:square162D}
%     \end{subfigure}
%     \begin{subfigure}{0.49\textwidth}
%         \centering
%         \resizebox{1\linewidth}{!}{%% Creator: Matplotlib, PGF backend
%%
%% To include the figure in your LaTeX document, write
%%   \input{<filename>.pgf}
%%
%% Make sure the required packages are loaded in your preamble
%%   \usepackage{pgf}
%%
%% and, on pdftex
%%   \usepackage[utf8]{inputenc}\DeclareUnicodeCharacter{2212}{-}
%%
%% or, on luatex and xetex
%%   \usepackage{unicode-math}
%%
%% Figures using additional raster images can only be included by \input if
%% they are in the same directory as the main LaTeX file. For loading figures
%% from other directories you can use the `import` package
%%   \usepackage{import}
%%
%% and then include the figures with
%%   \import{<path to file>}{<filename>.pgf}
%%
%% Matplotlib used the following preamble
%%   \usepackage{fontspec}
%%   \setmainfont{DejaVuSerif.ttf}[Path=C:/Users/Claudio/AppData/Local/Programs/Python/Python39/Lib/site-packages/matplotlib/mpl-data/fonts/ttf/]
%%   \setsansfont{DejaVuSans.ttf}[Path=C:/Users/Claudio/AppData/Local/Programs/Python/Python39/Lib/site-packages/matplotlib/mpl-data/fonts/ttf/]
%%   \setmonofont{DejaVuSansMono.ttf}[Path=C:/Users/Claudio/AppData/Local/Programs/Python/Python39/Lib/site-packages/matplotlib/mpl-data/fonts/ttf/]
%%
\begingroup%
\makeatletter%
\begin{pgfpicture}%
\pgfpathrectangle{\pgfpointorigin}{\pgfqpoint{4.342069in}{4.226689in}}%
\pgfusepath{use as bounding box, clip}%
\begin{pgfscope}%
\pgfsetbuttcap%
\pgfsetmiterjoin%
\definecolor{currentfill}{rgb}{1.000000,1.000000,1.000000}%
\pgfsetfillcolor{currentfill}%
\pgfsetlinewidth{0.000000pt}%
\definecolor{currentstroke}{rgb}{1.000000,1.000000,1.000000}%
\pgfsetstrokecolor{currentstroke}%
\pgfsetdash{}{0pt}%
\pgfpathmoveto{\pgfqpoint{0.000000in}{0.000000in}}%
\pgfpathlineto{\pgfqpoint{4.342069in}{0.000000in}}%
\pgfpathlineto{\pgfqpoint{4.342069in}{4.226689in}}%
\pgfpathlineto{\pgfqpoint{0.000000in}{4.226689in}}%
\pgfpathclose%
\pgfusepath{fill}%
\end{pgfscope}%
\begin{pgfscope}%
\pgfsetbuttcap%
\pgfsetmiterjoin%
\definecolor{currentfill}{rgb}{1.000000,1.000000,1.000000}%
\pgfsetfillcolor{currentfill}%
\pgfsetlinewidth{0.000000pt}%
\definecolor{currentstroke}{rgb}{0.000000,0.000000,0.000000}%
\pgfsetstrokecolor{currentstroke}%
\pgfsetstrokeopacity{0.000000}%
\pgfsetdash{}{0pt}%
\pgfpathmoveto{\pgfqpoint{0.100000in}{0.220728in}}%
\pgfpathlineto{\pgfqpoint{3.796000in}{0.220728in}}%
\pgfpathlineto{\pgfqpoint{3.796000in}{3.916728in}}%
\pgfpathlineto{\pgfqpoint{0.100000in}{3.916728in}}%
\pgfpathclose%
\pgfusepath{fill}%
\end{pgfscope}%
\begin{pgfscope}%
\pgfsetbuttcap%
\pgfsetmiterjoin%
\definecolor{currentfill}{rgb}{0.950000,0.950000,0.950000}%
\pgfsetfillcolor{currentfill}%
\pgfsetfillopacity{0.500000}%
\pgfsetlinewidth{1.003750pt}%
\definecolor{currentstroke}{rgb}{0.950000,0.950000,0.950000}%
\pgfsetstrokecolor{currentstroke}%
\pgfsetstrokeopacity{0.500000}%
\pgfsetdash{}{0pt}%
\pgfpathmoveto{\pgfqpoint{0.379073in}{1.132043in}}%
\pgfpathlineto{\pgfqpoint{1.599613in}{2.155124in}}%
\pgfpathlineto{\pgfqpoint{1.582647in}{3.630589in}}%
\pgfpathlineto{\pgfqpoint{0.303698in}{2.697271in}}%
\pgfusepath{stroke,fill}%
\end{pgfscope}%
\begin{pgfscope}%
\pgfsetbuttcap%
\pgfsetmiterjoin%
\definecolor{currentfill}{rgb}{0.900000,0.900000,0.900000}%
\pgfsetfillcolor{currentfill}%
\pgfsetfillopacity{0.500000}%
\pgfsetlinewidth{1.003750pt}%
\definecolor{currentstroke}{rgb}{0.900000,0.900000,0.900000}%
\pgfsetstrokecolor{currentstroke}%
\pgfsetstrokeopacity{0.500000}%
\pgfsetdash{}{0pt}%
\pgfpathmoveto{\pgfqpoint{1.599613in}{2.155124in}}%
\pgfpathlineto{\pgfqpoint{3.558144in}{1.585856in}}%
\pgfpathlineto{\pgfqpoint{3.628038in}{3.112142in}}%
\pgfpathlineto{\pgfqpoint{1.582647in}{3.630589in}}%
\pgfusepath{stroke,fill}%
\end{pgfscope}%
\begin{pgfscope}%
\pgfsetbuttcap%
\pgfsetmiterjoin%
\definecolor{currentfill}{rgb}{0.925000,0.925000,0.925000}%
\pgfsetfillcolor{currentfill}%
\pgfsetfillopacity{0.500000}%
\pgfsetlinewidth{1.003750pt}%
\definecolor{currentstroke}{rgb}{0.925000,0.925000,0.925000}%
\pgfsetstrokecolor{currentstroke}%
\pgfsetstrokeopacity{0.500000}%
\pgfsetdash{}{0pt}%
\pgfpathmoveto{\pgfqpoint{0.379073in}{1.132043in}}%
\pgfpathlineto{\pgfqpoint{2.455212in}{0.453976in}}%
\pgfpathlineto{\pgfqpoint{3.558144in}{1.585856in}}%
\pgfpathlineto{\pgfqpoint{1.599613in}{2.155124in}}%
\pgfusepath{stroke,fill}%
\end{pgfscope}%
\begin{pgfscope}%
\pgfsetrectcap%
\pgfsetroundjoin%
\pgfsetlinewidth{0.803000pt}%
\definecolor{currentstroke}{rgb}{0.000000,0.000000,0.000000}%
\pgfsetstrokecolor{currentstroke}%
\pgfsetdash{}{0pt}%
\pgfpathmoveto{\pgfqpoint{0.379073in}{1.132043in}}%
\pgfpathlineto{\pgfqpoint{2.455212in}{0.453976in}}%
\pgfusepath{stroke}%
\end{pgfscope}%
\begin{pgfscope}%
\definecolor{textcolor}{rgb}{0.000000,0.000000,0.000000}%
\pgfsetstrokecolor{textcolor}%
\pgfsetfillcolor{textcolor}%
\pgftext[x=0.697927in, y=0.423808in, left, base,rotate=341.912962]{\color{textcolor}\sffamily\fontsize{10.000000}{12.000000}\selectfont Position X [\(\displaystyle m\)]}%
\end{pgfscope}%
\begin{pgfscope}%
\pgfsetbuttcap%
\pgfsetroundjoin%
\pgfsetlinewidth{0.803000pt}%
\definecolor{currentstroke}{rgb}{0.690196,0.690196,0.690196}%
\pgfsetstrokecolor{currentstroke}%
\pgfsetdash{}{0pt}%
\pgfpathmoveto{\pgfqpoint{0.697629in}{1.028003in}}%
\pgfpathlineto{\pgfqpoint{1.901248in}{2.067451in}}%
\pgfpathlineto{\pgfqpoint{1.897097in}{3.550885in}}%
\pgfusepath{stroke}%
\end{pgfscope}%
\begin{pgfscope}%
\pgfsetbuttcap%
\pgfsetroundjoin%
\pgfsetlinewidth{0.803000pt}%
\definecolor{currentstroke}{rgb}{0.690196,0.690196,0.690196}%
\pgfsetstrokecolor{currentstroke}%
\pgfsetdash{}{0pt}%
\pgfpathmoveto{\pgfqpoint{1.125744in}{0.888181in}}%
\pgfpathlineto{\pgfqpoint{2.305979in}{1.949811in}}%
\pgfpathlineto{\pgfqpoint{2.319344in}{3.443858in}}%
\pgfusepath{stroke}%
\end{pgfscope}%
\begin{pgfscope}%
\pgfsetbuttcap%
\pgfsetroundjoin%
\pgfsetlinewidth{0.803000pt}%
\definecolor{currentstroke}{rgb}{0.690196,0.690196,0.690196}%
\pgfsetstrokecolor{currentstroke}%
\pgfsetdash{}{0pt}%
\pgfpathmoveto{\pgfqpoint{1.563447in}{0.745227in}}%
\pgfpathlineto{\pgfqpoint{2.719014in}{1.829758in}}%
\pgfpathlineto{\pgfqpoint{2.750634in}{3.334538in}}%
\pgfusepath{stroke}%
\end{pgfscope}%
\begin{pgfscope}%
\pgfsetbuttcap%
\pgfsetroundjoin%
\pgfsetlinewidth{0.803000pt}%
\definecolor{currentstroke}{rgb}{0.690196,0.690196,0.690196}%
\pgfsetstrokecolor{currentstroke}%
\pgfsetdash{}{0pt}%
\pgfpathmoveto{\pgfqpoint{2.011065in}{0.599035in}}%
\pgfpathlineto{\pgfqpoint{3.140610in}{1.707217in}}%
\pgfpathlineto{\pgfqpoint{3.191260in}{3.222853in}}%
\pgfusepath{stroke}%
\end{pgfscope}%
\begin{pgfscope}%
\pgfsetrectcap%
\pgfsetroundjoin%
\pgfsetlinewidth{0.803000pt}%
\definecolor{currentstroke}{rgb}{0.000000,0.000000,0.000000}%
\pgfsetstrokecolor{currentstroke}%
\pgfsetdash{}{0pt}%
\pgfpathmoveto{\pgfqpoint{0.708114in}{1.037058in}}%
\pgfpathlineto{\pgfqpoint{0.676613in}{1.009854in}}%
\pgfusepath{stroke}%
\end{pgfscope}%
\begin{pgfscope}%
\definecolor{textcolor}{rgb}{0.000000,0.000000,0.000000}%
\pgfsetstrokecolor{textcolor}%
\pgfsetfillcolor{textcolor}%
\pgftext[x=0.593253in,y=0.808465in,,top]{\color{textcolor}\sffamily\fontsize{10.000000}{12.000000}\selectfont 0}%
\end{pgfscope}%
\begin{pgfscope}%
\pgfsetrectcap%
\pgfsetroundjoin%
\pgfsetlinewidth{0.803000pt}%
\definecolor{currentstroke}{rgb}{0.000000,0.000000,0.000000}%
\pgfsetstrokecolor{currentstroke}%
\pgfsetdash{}{0pt}%
\pgfpathmoveto{\pgfqpoint{1.136034in}{0.897437in}}%
\pgfpathlineto{\pgfqpoint{1.105117in}{0.869627in}}%
\pgfusepath{stroke}%
\end{pgfscope}%
\begin{pgfscope}%
\definecolor{textcolor}{rgb}{0.000000,0.000000,0.000000}%
\pgfsetstrokecolor{textcolor}%
\pgfsetfillcolor{textcolor}%
\pgftext[x=1.021825in,y=0.665675in,,top]{\color{textcolor}\sffamily\fontsize{10.000000}{12.000000}\selectfont 5}%
\end{pgfscope}%
\begin{pgfscope}%
\pgfsetrectcap%
\pgfsetroundjoin%
\pgfsetlinewidth{0.803000pt}%
\definecolor{currentstroke}{rgb}{0.000000,0.000000,0.000000}%
\pgfsetstrokecolor{currentstroke}%
\pgfsetdash{}{0pt}%
\pgfpathmoveto{\pgfqpoint{1.573532in}{0.754692in}}%
\pgfpathlineto{\pgfqpoint{1.543232in}{0.726255in}}%
\pgfusepath{stroke}%
\end{pgfscope}%
\begin{pgfscope}%
\definecolor{textcolor}{rgb}{0.000000,0.000000,0.000000}%
\pgfsetstrokecolor{textcolor}%
\pgfsetfillcolor{textcolor}%
\pgftext[x=1.460032in,y=0.519673in,,top]{\color{textcolor}\sffamily\fontsize{10.000000}{12.000000}\selectfont 10}%
\end{pgfscope}%
\begin{pgfscope}%
\pgfsetrectcap%
\pgfsetroundjoin%
\pgfsetlinewidth{0.803000pt}%
\definecolor{currentstroke}{rgb}{0.000000,0.000000,0.000000}%
\pgfsetstrokecolor{currentstroke}%
\pgfsetdash{}{0pt}%
\pgfpathmoveto{\pgfqpoint{2.020933in}{0.608715in}}%
\pgfpathlineto{\pgfqpoint{1.991287in}{0.579630in}}%
\pgfusepath{stroke}%
\end{pgfscope}%
\begin{pgfscope}%
\definecolor{textcolor}{rgb}{0.000000,0.000000,0.000000}%
\pgfsetstrokecolor{textcolor}%
\pgfsetfillcolor{textcolor}%
\pgftext[x=1.908203in,y=0.370352in,,top]{\color{textcolor}\sffamily\fontsize{10.000000}{12.000000}\selectfont 15}%
\end{pgfscope}%
\begin{pgfscope}%
\pgfsetrectcap%
\pgfsetroundjoin%
\pgfsetlinewidth{0.803000pt}%
\definecolor{currentstroke}{rgb}{0.000000,0.000000,0.000000}%
\pgfsetstrokecolor{currentstroke}%
\pgfsetdash{}{0pt}%
\pgfpathmoveto{\pgfqpoint{3.558144in}{1.585856in}}%
\pgfpathlineto{\pgfqpoint{2.455212in}{0.453976in}}%
\pgfusepath{stroke}%
\end{pgfscope}%
\begin{pgfscope}%
\definecolor{textcolor}{rgb}{0.000000,0.000000,0.000000}%
\pgfsetstrokecolor{textcolor}%
\pgfsetfillcolor{textcolor}%
\pgftext[x=3.103916in, y=0.291339in, left, base,rotate=45.742112]{\color{textcolor}\sffamily\fontsize{10.000000}{12.000000}\selectfont Position Y [\(\displaystyle m\)]}%
\end{pgfscope}%
\begin{pgfscope}%
\pgfsetbuttcap%
\pgfsetroundjoin%
\pgfsetlinewidth{0.803000pt}%
\definecolor{currentstroke}{rgb}{0.690196,0.690196,0.690196}%
\pgfsetstrokecolor{currentstroke}%
\pgfsetdash{}{0pt}%
\pgfpathmoveto{\pgfqpoint{0.464545in}{2.814650in}}%
\pgfpathlineto{\pgfqpoint{0.532066in}{1.260285in}}%
\pgfpathlineto{\pgfqpoint{2.593998in}{0.596405in}}%
\pgfusepath{stroke}%
\end{pgfscope}%
\begin{pgfscope}%
\pgfsetbuttcap%
\pgfsetroundjoin%
\pgfsetlinewidth{0.803000pt}%
\definecolor{currentstroke}{rgb}{0.690196,0.690196,0.690196}%
\pgfsetstrokecolor{currentstroke}%
\pgfsetdash{}{0pt}%
\pgfpathmoveto{\pgfqpoint{0.775141in}{3.041309in}}%
\pgfpathlineto{\pgfqpoint{0.827908in}{1.508266in}}%
\pgfpathlineto{\pgfqpoint{2.861931in}{0.871370in}}%
\pgfusepath{stroke}%
\end{pgfscope}%
\begin{pgfscope}%
\pgfsetbuttcap%
\pgfsetroundjoin%
\pgfsetlinewidth{0.803000pt}%
\definecolor{currentstroke}{rgb}{0.690196,0.690196,0.690196}%
\pgfsetstrokecolor{currentstroke}%
\pgfsetdash{}{0pt}%
\pgfpathmoveto{\pgfqpoint{1.073013in}{3.258682in}}%
\pgfpathlineto{\pgfqpoint{1.112143in}{1.746517in}}%
\pgfpathlineto{\pgfqpoint{3.118813in}{1.134994in}}%
\pgfusepath{stroke}%
\end{pgfscope}%
\begin{pgfscope}%
\pgfsetbuttcap%
\pgfsetroundjoin%
\pgfsetlinewidth{0.803000pt}%
\definecolor{currentstroke}{rgb}{0.690196,0.690196,0.690196}%
\pgfsetstrokecolor{currentstroke}%
\pgfsetdash{}{0pt}%
\pgfpathmoveto{\pgfqpoint{1.358927in}{3.467328in}}%
\pgfpathlineto{\pgfqpoint{1.385441in}{1.975600in}}%
\pgfpathlineto{\pgfqpoint{3.365312in}{1.387963in}}%
\pgfusepath{stroke}%
\end{pgfscope}%
\begin{pgfscope}%
\pgfsetrectcap%
\pgfsetroundjoin%
\pgfsetlinewidth{0.803000pt}%
\definecolor{currentstroke}{rgb}{0.000000,0.000000,0.000000}%
\pgfsetstrokecolor{currentstroke}%
\pgfsetdash{}{0pt}%
\pgfpathmoveto{\pgfqpoint{2.576626in}{0.601998in}}%
\pgfpathlineto{\pgfqpoint{2.628786in}{0.585204in}}%
\pgfusepath{stroke}%
\end{pgfscope}%
\begin{pgfscope}%
\definecolor{textcolor}{rgb}{0.000000,0.000000,0.000000}%
\pgfsetstrokecolor{textcolor}%
\pgfsetfillcolor{textcolor}%
\pgftext[x=2.772058in,y=0.410920in,,top]{\color{textcolor}\sffamily\fontsize{10.000000}{12.000000}\selectfont 0}%
\end{pgfscope}%
\begin{pgfscope}%
\pgfsetrectcap%
\pgfsetroundjoin%
\pgfsetlinewidth{0.803000pt}%
\definecolor{currentstroke}{rgb}{0.000000,0.000000,0.000000}%
\pgfsetstrokecolor{currentstroke}%
\pgfsetdash{}{0pt}%
\pgfpathmoveto{\pgfqpoint{2.844813in}{0.876731in}}%
\pgfpathlineto{\pgfqpoint{2.896211in}{0.860637in}}%
\pgfusepath{stroke}%
\end{pgfscope}%
\begin{pgfscope}%
\definecolor{textcolor}{rgb}{0.000000,0.000000,0.000000}%
\pgfsetstrokecolor{textcolor}%
\pgfsetfillcolor{textcolor}%
\pgftext[x=3.036396in,y=0.689954in,,top]{\color{textcolor}\sffamily\fontsize{10.000000}{12.000000}\selectfont 5}%
\end{pgfscope}%
\begin{pgfscope}%
\pgfsetrectcap%
\pgfsetroundjoin%
\pgfsetlinewidth{0.803000pt}%
\definecolor{currentstroke}{rgb}{0.000000,0.000000,0.000000}%
\pgfsetstrokecolor{currentstroke}%
\pgfsetdash{}{0pt}%
\pgfpathmoveto{\pgfqpoint{3.101942in}{1.140135in}}%
\pgfpathlineto{\pgfqpoint{3.152596in}{1.124699in}}%
\pgfusepath{stroke}%
\end{pgfscope}%
\begin{pgfscope}%
\definecolor{textcolor}{rgb}{0.000000,0.000000,0.000000}%
\pgfsetstrokecolor{textcolor}%
\pgfsetfillcolor{textcolor}%
\pgftext[x=3.289824in,y=0.957472in,,top]{\color{textcolor}\sffamily\fontsize{10.000000}{12.000000}\selectfont 10}%
\end{pgfscope}%
\begin{pgfscope}%
\pgfsetrectcap%
\pgfsetroundjoin%
\pgfsetlinewidth{0.803000pt}%
\definecolor{currentstroke}{rgb}{0.000000,0.000000,0.000000}%
\pgfsetstrokecolor{currentstroke}%
\pgfsetdash{}{0pt}%
\pgfpathmoveto{\pgfqpoint{3.348683in}{1.392899in}}%
\pgfpathlineto{\pgfqpoint{3.398611in}{1.378080in}}%
\pgfusepath{stroke}%
\end{pgfscope}%
\begin{pgfscope}%
\definecolor{textcolor}{rgb}{0.000000,0.000000,0.000000}%
\pgfsetstrokecolor{textcolor}%
\pgfsetfillcolor{textcolor}%
\pgftext[x=3.533003in,y=1.214172in,,top]{\color{textcolor}\sffamily\fontsize{10.000000}{12.000000}\selectfont 15}%
\end{pgfscope}%
\begin{pgfscope}%
\pgfsetrectcap%
\pgfsetroundjoin%
\pgfsetlinewidth{0.803000pt}%
\definecolor{currentstroke}{rgb}{0.000000,0.000000,0.000000}%
\pgfsetstrokecolor{currentstroke}%
\pgfsetdash{}{0pt}%
\pgfpathmoveto{\pgfqpoint{3.558144in}{1.585856in}}%
\pgfpathlineto{\pgfqpoint{3.628038in}{3.112142in}}%
\pgfusepath{stroke}%
\end{pgfscope}%
\begin{pgfscope}%
\definecolor{textcolor}{rgb}{0.000000,0.000000,0.000000}%
\pgfsetstrokecolor{textcolor}%
\pgfsetfillcolor{textcolor}%
\pgftext[x=4.169544in, y=1.928890in, left, base,rotate=87.378092]{\color{textcolor}\sffamily\fontsize{10.000000}{12.000000}\selectfont Position Z [\(\displaystyle m\)]}%
\end{pgfscope}%
\begin{pgfscope}%
\pgfsetbuttcap%
\pgfsetroundjoin%
\pgfsetlinewidth{0.803000pt}%
\definecolor{currentstroke}{rgb}{0.690196,0.690196,0.690196}%
\pgfsetstrokecolor{currentstroke}%
\pgfsetdash{}{0pt}%
\pgfpathmoveto{\pgfqpoint{3.562758in}{1.686598in}}%
\pgfpathlineto{\pgfqpoint{1.598491in}{2.252704in}}%
\pgfpathlineto{\pgfqpoint{0.374106in}{1.235193in}}%
\pgfusepath{stroke}%
\end{pgfscope}%
\begin{pgfscope}%
\pgfsetbuttcap%
\pgfsetroundjoin%
\pgfsetlinewidth{0.803000pt}%
\definecolor{currentstroke}{rgb}{0.690196,0.690196,0.690196}%
\pgfsetstrokecolor{currentstroke}%
\pgfsetdash{}{0pt}%
\pgfpathmoveto{\pgfqpoint{3.571660in}{1.881000in}}%
\pgfpathlineto{\pgfqpoint{1.596327in}{2.440926in}}%
\pgfpathlineto{\pgfqpoint{0.364518in}{1.434305in}}%
\pgfusepath{stroke}%
\end{pgfscope}%
\begin{pgfscope}%
\pgfsetbuttcap%
\pgfsetroundjoin%
\pgfsetlinewidth{0.803000pt}%
\definecolor{currentstroke}{rgb}{0.690196,0.690196,0.690196}%
\pgfsetstrokecolor{currentstroke}%
\pgfsetdash{}{0pt}%
\pgfpathmoveto{\pgfqpoint{3.580664in}{2.077626in}}%
\pgfpathlineto{\pgfqpoint{1.594139in}{2.631199in}}%
\pgfpathlineto{\pgfqpoint{0.354815in}{1.635783in}}%
\pgfusepath{stroke}%
\end{pgfscope}%
\begin{pgfscope}%
\pgfsetbuttcap%
\pgfsetroundjoin%
\pgfsetlinewidth{0.803000pt}%
\definecolor{currentstroke}{rgb}{0.690196,0.690196,0.690196}%
\pgfsetstrokecolor{currentstroke}%
\pgfsetdash{}{0pt}%
\pgfpathmoveto{\pgfqpoint{3.589772in}{2.276515in}}%
\pgfpathlineto{\pgfqpoint{1.591927in}{2.823557in}}%
\pgfpathlineto{\pgfqpoint{0.344997in}{1.839669in}}%
\pgfusepath{stroke}%
\end{pgfscope}%
\begin{pgfscope}%
\pgfsetbuttcap%
\pgfsetroundjoin%
\pgfsetlinewidth{0.803000pt}%
\definecolor{currentstroke}{rgb}{0.690196,0.690196,0.690196}%
\pgfsetstrokecolor{currentstroke}%
\pgfsetdash{}{0pt}%
\pgfpathmoveto{\pgfqpoint{3.598985in}{2.477706in}}%
\pgfpathlineto{\pgfqpoint{1.589691in}{3.018034in}}%
\pgfpathlineto{\pgfqpoint{0.335060in}{2.046006in}}%
\pgfusepath{stroke}%
\end{pgfscope}%
\begin{pgfscope}%
\pgfsetbuttcap%
\pgfsetroundjoin%
\pgfsetlinewidth{0.803000pt}%
\definecolor{currentstroke}{rgb}{0.690196,0.690196,0.690196}%
\pgfsetstrokecolor{currentstroke}%
\pgfsetdash{}{0pt}%
\pgfpathmoveto{\pgfqpoint{3.608305in}{2.681240in}}%
\pgfpathlineto{\pgfqpoint{1.587430in}{3.214665in}}%
\pgfpathlineto{\pgfqpoint{0.325004in}{2.254839in}}%
\pgfusepath{stroke}%
\end{pgfscope}%
\begin{pgfscope}%
\pgfsetbuttcap%
\pgfsetroundjoin%
\pgfsetlinewidth{0.803000pt}%
\definecolor{currentstroke}{rgb}{0.690196,0.690196,0.690196}%
\pgfsetstrokecolor{currentstroke}%
\pgfsetdash{}{0pt}%
\pgfpathmoveto{\pgfqpoint{3.617735in}{2.887157in}}%
\pgfpathlineto{\pgfqpoint{1.585143in}{3.413486in}}%
\pgfpathlineto{\pgfqpoint{0.314825in}{2.466213in}}%
\pgfusepath{stroke}%
\end{pgfscope}%
\begin{pgfscope}%
\pgfsetrectcap%
\pgfsetroundjoin%
\pgfsetlinewidth{0.803000pt}%
\definecolor{currentstroke}{rgb}{0.000000,0.000000,0.000000}%
\pgfsetstrokecolor{currentstroke}%
\pgfsetdash{}{0pt}%
\pgfpathmoveto{\pgfqpoint{3.546270in}{1.691350in}}%
\pgfpathlineto{\pgfqpoint{3.595773in}{1.677083in}}%
\pgfusepath{stroke}%
\end{pgfscope}%
\begin{pgfscope}%
\definecolor{textcolor}{rgb}{0.000000,0.000000,0.000000}%
\pgfsetstrokecolor{textcolor}%
\pgfsetfillcolor{textcolor}%
\pgftext[x=3.816944in,y=1.722582in,,top]{\color{textcolor}\sffamily\fontsize{10.000000}{12.000000}\selectfont 0.0}%
\end{pgfscope}%
\begin{pgfscope}%
\pgfsetrectcap%
\pgfsetroundjoin%
\pgfsetlinewidth{0.803000pt}%
\definecolor{currentstroke}{rgb}{0.000000,0.000000,0.000000}%
\pgfsetstrokecolor{currentstroke}%
\pgfsetdash{}{0pt}%
\pgfpathmoveto{\pgfqpoint{3.555075in}{1.885701in}}%
\pgfpathlineto{\pgfqpoint{3.604870in}{1.871586in}}%
\pgfusepath{stroke}%
\end{pgfscope}%
\begin{pgfscope}%
\definecolor{textcolor}{rgb}{0.000000,0.000000,0.000000}%
\pgfsetstrokecolor{textcolor}%
\pgfsetfillcolor{textcolor}%
\pgftext[x=3.827261in,y=1.916600in,,top]{\color{textcolor}\sffamily\fontsize{10.000000}{12.000000}\selectfont 0.1}%
\end{pgfscope}%
\begin{pgfscope}%
\pgfsetrectcap%
\pgfsetroundjoin%
\pgfsetlinewidth{0.803000pt}%
\definecolor{currentstroke}{rgb}{0.000000,0.000000,0.000000}%
\pgfsetstrokecolor{currentstroke}%
\pgfsetdash{}{0pt}%
\pgfpathmoveto{\pgfqpoint{3.563980in}{2.082275in}}%
\pgfpathlineto{\pgfqpoint{3.614072in}{2.068316in}}%
\pgfusepath{stroke}%
\end{pgfscope}%
\begin{pgfscope}%
\definecolor{textcolor}{rgb}{0.000000,0.000000,0.000000}%
\pgfsetstrokecolor{textcolor}%
\pgfsetfillcolor{textcolor}%
\pgftext[x=3.837696in,y=2.112831in,,top]{\color{textcolor}\sffamily\fontsize{10.000000}{12.000000}\selectfont 0.2}%
\end{pgfscope}%
\begin{pgfscope}%
\pgfsetrectcap%
\pgfsetroundjoin%
\pgfsetlinewidth{0.803000pt}%
\definecolor{currentstroke}{rgb}{0.000000,0.000000,0.000000}%
\pgfsetstrokecolor{currentstroke}%
\pgfsetdash{}{0pt}%
\pgfpathmoveto{\pgfqpoint{3.572988in}{2.281110in}}%
\pgfpathlineto{\pgfqpoint{3.623379in}{2.267312in}}%
\pgfusepath{stroke}%
\end{pgfscope}%
\begin{pgfscope}%
\definecolor{textcolor}{rgb}{0.000000,0.000000,0.000000}%
\pgfsetstrokecolor{textcolor}%
\pgfsetfillcolor{textcolor}%
\pgftext[x=3.848250in,y=2.311314in,,top]{\color{textcolor}\sffamily\fontsize{10.000000}{12.000000}\selectfont 0.3}%
\end{pgfscope}%
\begin{pgfscope}%
\pgfsetrectcap%
\pgfsetroundjoin%
\pgfsetlinewidth{0.803000pt}%
\definecolor{currentstroke}{rgb}{0.000000,0.000000,0.000000}%
\pgfsetstrokecolor{currentstroke}%
\pgfsetdash{}{0pt}%
\pgfpathmoveto{\pgfqpoint{3.582101in}{2.482246in}}%
\pgfpathlineto{\pgfqpoint{3.632794in}{2.468614in}}%
\pgfusepath{stroke}%
\end{pgfscope}%
\begin{pgfscope}%
\definecolor{textcolor}{rgb}{0.000000,0.000000,0.000000}%
\pgfsetstrokecolor{textcolor}%
\pgfsetfillcolor{textcolor}%
\pgftext[x=3.858927in,y=2.512087in,,top]{\color{textcolor}\sffamily\fontsize{10.000000}{12.000000}\selectfont 0.4}%
\end{pgfscope}%
\begin{pgfscope}%
\pgfsetrectcap%
\pgfsetroundjoin%
\pgfsetlinewidth{0.803000pt}%
\definecolor{currentstroke}{rgb}{0.000000,0.000000,0.000000}%
\pgfsetstrokecolor{currentstroke}%
\pgfsetdash{}{0pt}%
\pgfpathmoveto{\pgfqpoint{3.591319in}{2.685723in}}%
\pgfpathlineto{\pgfqpoint{3.642319in}{2.672261in}}%
\pgfusepath{stroke}%
\end{pgfscope}%
\begin{pgfscope}%
\definecolor{textcolor}{rgb}{0.000000,0.000000,0.000000}%
\pgfsetstrokecolor{textcolor}%
\pgfsetfillcolor{textcolor}%
\pgftext[x=3.869727in,y=2.715190in,,top]{\color{textcolor}\sffamily\fontsize{10.000000}{12.000000}\selectfont 0.5}%
\end{pgfscope}%
\begin{pgfscope}%
\pgfsetrectcap%
\pgfsetroundjoin%
\pgfsetlinewidth{0.803000pt}%
\definecolor{currentstroke}{rgb}{0.000000,0.000000,0.000000}%
\pgfsetstrokecolor{currentstroke}%
\pgfsetdash{}{0pt}%
\pgfpathmoveto{\pgfqpoint{3.600645in}{2.891582in}}%
\pgfpathlineto{\pgfqpoint{3.651956in}{2.878295in}}%
\pgfusepath{stroke}%
\end{pgfscope}%
\begin{pgfscope}%
\definecolor{textcolor}{rgb}{0.000000,0.000000,0.000000}%
\pgfsetstrokecolor{textcolor}%
\pgfsetfillcolor{textcolor}%
\pgftext[x=3.880653in,y=2.920665in,,top]{\color{textcolor}\sffamily\fontsize{10.000000}{12.000000}\selectfont 0.6}%
\end{pgfscope}%
\begin{pgfscope}%
\pgfpathrectangle{\pgfqpoint{0.100000in}{0.220728in}}{\pgfqpoint{3.696000in}{3.696000in}}%
\pgfusepath{clip}%
\pgfsetrectcap%
\pgfsetroundjoin%
\pgfsetlinewidth{1.505625pt}%
\definecolor{currentstroke}{rgb}{1.000000,0.000000,0.000000}%
\pgfsetstrokecolor{currentstroke}%
\pgfsetdash{}{0pt}%
\pgfpathmoveto{\pgfqpoint{0.845071in}{1.261386in}}%
\pgfpathlineto{\pgfqpoint{0.845071in}{1.261386in}}%
\pgfusepath{stroke}%
\end{pgfscope}%
\begin{pgfscope}%
\pgfpathrectangle{\pgfqpoint{0.100000in}{0.220728in}}{\pgfqpoint{3.696000in}{3.696000in}}%
\pgfusepath{clip}%
\pgfsetrectcap%
\pgfsetroundjoin%
\pgfsetlinewidth{1.505625pt}%
\definecolor{currentstroke}{rgb}{1.000000,0.000000,0.000000}%
\pgfsetstrokecolor{currentstroke}%
\pgfsetdash{}{0pt}%
\pgfpathmoveto{\pgfqpoint{1.776593in}{3.293739in}}%
\pgfpathlineto{\pgfqpoint{1.742067in}{2.029460in}}%
\pgfusepath{stroke}%
\end{pgfscope}%
\begin{pgfscope}%
\pgfpathrectangle{\pgfqpoint{0.100000in}{0.220728in}}{\pgfqpoint{3.696000in}{3.696000in}}%
\pgfusepath{clip}%
\pgfsetrectcap%
\pgfsetroundjoin%
\pgfsetlinewidth{1.505625pt}%
\definecolor{currentstroke}{rgb}{1.000000,0.000000,0.000000}%
\pgfsetstrokecolor{currentstroke}%
\pgfsetdash{}{0pt}%
\pgfpathmoveto{\pgfqpoint{3.401463in}{2.969095in}}%
\pgfpathlineto{\pgfqpoint{3.081581in}{1.637273in}}%
\pgfusepath{stroke}%
\end{pgfscope}%
\begin{pgfscope}%
\pgfpathrectangle{\pgfqpoint{0.100000in}{0.220728in}}{\pgfqpoint{3.696000in}{3.696000in}}%
\pgfusepath{clip}%
\pgfsetrectcap%
\pgfsetroundjoin%
\pgfsetlinewidth{1.505625pt}%
\definecolor{currentstroke}{rgb}{1.000000,0.000000,0.000000}%
\pgfsetstrokecolor{currentstroke}%
\pgfsetdash{}{0pt}%
\pgfpathmoveto{\pgfqpoint{2.384544in}{0.749136in}}%
\pgfpathlineto{\pgfqpoint{2.243952in}{0.814661in}}%
\pgfusepath{stroke}%
\end{pgfscope}%
\begin{pgfscope}%
\pgfpathrectangle{\pgfqpoint{0.100000in}{0.220728in}}{\pgfqpoint{3.696000in}{3.696000in}}%
\pgfusepath{clip}%
\pgfsetrectcap%
\pgfsetroundjoin%
\pgfsetlinewidth{1.505625pt}%
\definecolor{currentstroke}{rgb}{1.000000,0.000000,0.000000}%
\pgfsetstrokecolor{currentstroke}%
\pgfsetdash{}{0pt}%
\pgfpathmoveto{\pgfqpoint{0.592749in}{1.489290in}}%
\pgfpathlineto{\pgfqpoint{0.845071in}{1.261386in}}%
\pgfusepath{stroke}%
\end{pgfscope}%
\begin{pgfscope}%
\pgfpathrectangle{\pgfqpoint{0.100000in}{0.220728in}}{\pgfqpoint{3.696000in}{3.696000in}}%
\pgfusepath{clip}%
\pgfsetrectcap%
\pgfsetroundjoin%
\pgfsetlinewidth{1.505625pt}%
\definecolor{currentstroke}{rgb}{0.121569,0.466667,0.705882}%
\pgfsetstrokecolor{currentstroke}%
\pgfsetdash{}{0pt}%
\pgfpathmoveto{\pgfqpoint{0.845071in}{1.261386in}}%
\pgfpathlineto{\pgfqpoint{1.742067in}{2.029460in}}%
\pgfpathlineto{\pgfqpoint{3.081581in}{1.637273in}}%
\pgfpathlineto{\pgfqpoint{2.243952in}{0.814661in}}%
\pgfpathlineto{\pgfqpoint{0.845071in}{1.261386in}}%
\pgfusepath{stroke}%
\end{pgfscope}%
\begin{pgfscope}%
\pgfpathrectangle{\pgfqpoint{0.100000in}{0.220728in}}{\pgfqpoint{3.696000in}{3.696000in}}%
\pgfusepath{clip}%
\pgfsetbuttcap%
\pgfsetroundjoin%
\definecolor{currentfill}{rgb}{1.000000,0.498039,0.054902}%
\pgfsetfillcolor{currentfill}%
\pgfsetfillopacity{0.300000}%
\pgfsetlinewidth{1.003750pt}%
\definecolor{currentstroke}{rgb}{1.000000,0.498039,0.054902}%
\pgfsetstrokecolor{currentstroke}%
\pgfsetstrokeopacity{0.300000}%
\pgfsetdash{}{0pt}%
\pgfpathmoveto{\pgfqpoint{1.776593in}{3.262683in}}%
\pgfpathcurveto{\pgfqpoint{1.784830in}{3.262683in}}{\pgfqpoint{1.792730in}{3.265955in}}{\pgfqpoint{1.798554in}{3.271779in}}%
\pgfpathcurveto{\pgfqpoint{1.804378in}{3.277603in}}{\pgfqpoint{1.807650in}{3.285503in}}{\pgfqpoint{1.807650in}{3.293739in}}%
\pgfpathcurveto{\pgfqpoint{1.807650in}{3.301976in}}{\pgfqpoint{1.804378in}{3.309876in}}{\pgfqpoint{1.798554in}{3.315700in}}%
\pgfpathcurveto{\pgfqpoint{1.792730in}{3.321523in}}{\pgfqpoint{1.784830in}{3.324796in}}{\pgfqpoint{1.776593in}{3.324796in}}%
\pgfpathcurveto{\pgfqpoint{1.768357in}{3.324796in}}{\pgfqpoint{1.760457in}{3.321523in}}{\pgfqpoint{1.754633in}{3.315700in}}%
\pgfpathcurveto{\pgfqpoint{1.748809in}{3.309876in}}{\pgfqpoint{1.745537in}{3.301976in}}{\pgfqpoint{1.745537in}{3.293739in}}%
\pgfpathcurveto{\pgfqpoint{1.745537in}{3.285503in}}{\pgfqpoint{1.748809in}{3.277603in}}{\pgfqpoint{1.754633in}{3.271779in}}%
\pgfpathcurveto{\pgfqpoint{1.760457in}{3.265955in}}{\pgfqpoint{1.768357in}{3.262683in}}{\pgfqpoint{1.776593in}{3.262683in}}%
\pgfpathclose%
\pgfusepath{stroke,fill}%
\end{pgfscope}%
\begin{pgfscope}%
\pgfpathrectangle{\pgfqpoint{0.100000in}{0.220728in}}{\pgfqpoint{3.696000in}{3.696000in}}%
\pgfusepath{clip}%
\pgfsetbuttcap%
\pgfsetroundjoin%
\definecolor{currentfill}{rgb}{1.000000,0.498039,0.054902}%
\pgfsetfillcolor{currentfill}%
\pgfsetfillopacity{0.611573}%
\pgfsetlinewidth{1.003750pt}%
\definecolor{currentstroke}{rgb}{1.000000,0.498039,0.054902}%
\pgfsetstrokecolor{currentstroke}%
\pgfsetstrokeopacity{0.611573}%
\pgfsetdash{}{0pt}%
\pgfpathmoveto{\pgfqpoint{0.845071in}{1.230330in}}%
\pgfpathcurveto{\pgfqpoint{0.853307in}{1.230330in}}{\pgfqpoint{0.861207in}{1.233602in}}{\pgfqpoint{0.867031in}{1.239426in}}%
\pgfpathcurveto{\pgfqpoint{0.872855in}{1.245250in}}{\pgfqpoint{0.876128in}{1.253150in}}{\pgfqpoint{0.876128in}{1.261386in}}%
\pgfpathcurveto{\pgfqpoint{0.876128in}{1.269622in}}{\pgfqpoint{0.872855in}{1.277522in}}{\pgfqpoint{0.867031in}{1.283346in}}%
\pgfpathcurveto{\pgfqpoint{0.861207in}{1.289170in}}{\pgfqpoint{0.853307in}{1.292443in}}{\pgfqpoint{0.845071in}{1.292443in}}%
\pgfpathcurveto{\pgfqpoint{0.836835in}{1.292443in}}{\pgfqpoint{0.828935in}{1.289170in}}{\pgfqpoint{0.823111in}{1.283346in}}%
\pgfpathcurveto{\pgfqpoint{0.817287in}{1.277522in}}{\pgfqpoint{0.814015in}{1.269622in}}{\pgfqpoint{0.814015in}{1.261386in}}%
\pgfpathcurveto{\pgfqpoint{0.814015in}{1.253150in}}{\pgfqpoint{0.817287in}{1.245250in}}{\pgfqpoint{0.823111in}{1.239426in}}%
\pgfpathcurveto{\pgfqpoint{0.828935in}{1.233602in}}{\pgfqpoint{0.836835in}{1.230330in}}{\pgfqpoint{0.845071in}{1.230330in}}%
\pgfpathclose%
\pgfusepath{stroke,fill}%
\end{pgfscope}%
\begin{pgfscope}%
\pgfpathrectangle{\pgfqpoint{0.100000in}{0.220728in}}{\pgfqpoint{3.696000in}{3.696000in}}%
\pgfusepath{clip}%
\pgfsetbuttcap%
\pgfsetroundjoin%
\definecolor{currentfill}{rgb}{1.000000,0.498039,0.054902}%
\pgfsetfillcolor{currentfill}%
\pgfsetfillopacity{0.651420}%
\pgfsetlinewidth{1.003750pt}%
\definecolor{currentstroke}{rgb}{1.000000,0.498039,0.054902}%
\pgfsetstrokecolor{currentstroke}%
\pgfsetstrokeopacity{0.651420}%
\pgfsetdash{}{0pt}%
\pgfpathmoveto{\pgfqpoint{0.592749in}{1.458234in}}%
\pgfpathcurveto{\pgfqpoint{0.600985in}{1.458234in}}{\pgfqpoint{0.608886in}{1.461506in}}{\pgfqpoint{0.614709in}{1.467330in}}%
\pgfpathcurveto{\pgfqpoint{0.620533in}{1.473154in}}{\pgfqpoint{0.623806in}{1.481054in}}{\pgfqpoint{0.623806in}{1.489290in}}%
\pgfpathcurveto{\pgfqpoint{0.623806in}{1.497526in}}{\pgfqpoint{0.620533in}{1.505426in}}{\pgfqpoint{0.614709in}{1.511250in}}%
\pgfpathcurveto{\pgfqpoint{0.608886in}{1.517074in}}{\pgfqpoint{0.600985in}{1.520347in}}{\pgfqpoint{0.592749in}{1.520347in}}%
\pgfpathcurveto{\pgfqpoint{0.584513in}{1.520347in}}{\pgfqpoint{0.576613in}{1.517074in}}{\pgfqpoint{0.570789in}{1.511250in}}%
\pgfpathcurveto{\pgfqpoint{0.564965in}{1.505426in}}{\pgfqpoint{0.561693in}{1.497526in}}{\pgfqpoint{0.561693in}{1.489290in}}%
\pgfpathcurveto{\pgfqpoint{0.561693in}{1.481054in}}{\pgfqpoint{0.564965in}{1.473154in}}{\pgfqpoint{0.570789in}{1.467330in}}%
\pgfpathcurveto{\pgfqpoint{0.576613in}{1.461506in}}{\pgfqpoint{0.584513in}{1.458234in}}{\pgfqpoint{0.592749in}{1.458234in}}%
\pgfpathclose%
\pgfusepath{stroke,fill}%
\end{pgfscope}%
\begin{pgfscope}%
\pgfpathrectangle{\pgfqpoint{0.100000in}{0.220728in}}{\pgfqpoint{3.696000in}{3.696000in}}%
\pgfusepath{clip}%
\pgfsetbuttcap%
\pgfsetroundjoin%
\definecolor{currentfill}{rgb}{1.000000,0.498039,0.054902}%
\pgfsetfillcolor{currentfill}%
\pgfsetfillopacity{0.669975}%
\pgfsetlinewidth{1.003750pt}%
\definecolor{currentstroke}{rgb}{1.000000,0.498039,0.054902}%
\pgfsetstrokecolor{currentstroke}%
\pgfsetstrokeopacity{0.669975}%
\pgfsetdash{}{0pt}%
\pgfpathmoveto{\pgfqpoint{3.401463in}{2.938039in}}%
\pgfpathcurveto{\pgfqpoint{3.409700in}{2.938039in}}{\pgfqpoint{3.417600in}{2.941311in}}{\pgfqpoint{3.423424in}{2.947135in}}%
\pgfpathcurveto{\pgfqpoint{3.429248in}{2.952959in}}{\pgfqpoint{3.432520in}{2.960859in}}{\pgfqpoint{3.432520in}{2.969095in}}%
\pgfpathcurveto{\pgfqpoint{3.432520in}{2.977332in}}{\pgfqpoint{3.429248in}{2.985232in}}{\pgfqpoint{3.423424in}{2.991056in}}%
\pgfpathcurveto{\pgfqpoint{3.417600in}{2.996880in}}{\pgfqpoint{3.409700in}{3.000152in}}{\pgfqpoint{3.401463in}{3.000152in}}%
\pgfpathcurveto{\pgfqpoint{3.393227in}{3.000152in}}{\pgfqpoint{3.385327in}{2.996880in}}{\pgfqpoint{3.379503in}{2.991056in}}%
\pgfpathcurveto{\pgfqpoint{3.373679in}{2.985232in}}{\pgfqpoint{3.370407in}{2.977332in}}{\pgfqpoint{3.370407in}{2.969095in}}%
\pgfpathcurveto{\pgfqpoint{3.370407in}{2.960859in}}{\pgfqpoint{3.373679in}{2.952959in}}{\pgfqpoint{3.379503in}{2.947135in}}%
\pgfpathcurveto{\pgfqpoint{3.385327in}{2.941311in}}{\pgfqpoint{3.393227in}{2.938039in}}{\pgfqpoint{3.401463in}{2.938039in}}%
\pgfpathclose%
\pgfusepath{stroke,fill}%
\end{pgfscope}%
\begin{pgfscope}%
\pgfpathrectangle{\pgfqpoint{0.100000in}{0.220728in}}{\pgfqpoint{3.696000in}{3.696000in}}%
\pgfusepath{clip}%
\pgfsetbuttcap%
\pgfsetroundjoin%
\definecolor{currentfill}{rgb}{1.000000,0.498039,0.054902}%
\pgfsetfillcolor{currentfill}%
\pgfsetlinewidth{1.003750pt}%
\definecolor{currentstroke}{rgb}{1.000000,0.498039,0.054902}%
\pgfsetstrokecolor{currentstroke}%
\pgfsetdash{}{0pt}%
\pgfpathmoveto{\pgfqpoint{2.384544in}{0.718080in}}%
\pgfpathcurveto{\pgfqpoint{2.392781in}{0.718080in}}{\pgfqpoint{2.400681in}{0.721352in}}{\pgfqpoint{2.406505in}{0.727176in}}%
\pgfpathcurveto{\pgfqpoint{2.412329in}{0.733000in}}{\pgfqpoint{2.415601in}{0.740900in}}{\pgfqpoint{2.415601in}{0.749136in}}%
\pgfpathcurveto{\pgfqpoint{2.415601in}{0.757372in}}{\pgfqpoint{2.412329in}{0.765272in}}{\pgfqpoint{2.406505in}{0.771096in}}%
\pgfpathcurveto{\pgfqpoint{2.400681in}{0.776920in}}{\pgfqpoint{2.392781in}{0.780193in}}{\pgfqpoint{2.384544in}{0.780193in}}%
\pgfpathcurveto{\pgfqpoint{2.376308in}{0.780193in}}{\pgfqpoint{2.368408in}{0.776920in}}{\pgfqpoint{2.362584in}{0.771096in}}%
\pgfpathcurveto{\pgfqpoint{2.356760in}{0.765272in}}{\pgfqpoint{2.353488in}{0.757372in}}{\pgfqpoint{2.353488in}{0.749136in}}%
\pgfpathcurveto{\pgfqpoint{2.353488in}{0.740900in}}{\pgfqpoint{2.356760in}{0.733000in}}{\pgfqpoint{2.362584in}{0.727176in}}%
\pgfpathcurveto{\pgfqpoint{2.368408in}{0.721352in}}{\pgfqpoint{2.376308in}{0.718080in}}{\pgfqpoint{2.384544in}{0.718080in}}%
\pgfpathclose%
\pgfusepath{stroke,fill}%
\end{pgfscope}%
\begin{pgfscope}%
\definecolor{textcolor}{rgb}{0.000000,0.000000,0.000000}%
\pgfsetstrokecolor{textcolor}%
\pgfsetfillcolor{textcolor}%
\pgftext[x=1.948000in,y=4.000061in,,base]{\color{textcolor}\sffamily\fontsize{12.000000}{14.400000}\selectfont FLAE}%
\end{pgfscope}%
\begin{pgfscope}%
\pgfpathrectangle{\pgfqpoint{0.100000in}{0.220728in}}{\pgfqpoint{3.696000in}{3.696000in}}%
\pgfusepath{clip}%
\pgfsetbuttcap%
\pgfsetroundjoin%
\definecolor{currentfill}{rgb}{0.121569,0.466667,0.705882}%
\pgfsetfillcolor{currentfill}%
\pgfsetfillopacity{0.300000}%
\pgfsetlinewidth{1.003750pt}%
\definecolor{currentstroke}{rgb}{0.121569,0.466667,0.705882}%
\pgfsetstrokecolor{currentstroke}%
\pgfsetstrokeopacity{0.300000}%
\pgfsetdash{}{0pt}%
\pgfpathmoveto{\pgfqpoint{1.773800in}{3.261306in}}%
\pgfpathcurveto{\pgfqpoint{1.782036in}{3.261306in}}{\pgfqpoint{1.789936in}{3.264579in}}{\pgfqpoint{1.795760in}{3.270402in}}%
\pgfpathcurveto{\pgfqpoint{1.801584in}{3.276226in}}{\pgfqpoint{1.804856in}{3.284126in}}{\pgfqpoint{1.804856in}{3.292363in}}%
\pgfpathcurveto{\pgfqpoint{1.804856in}{3.300599in}}{\pgfqpoint{1.801584in}{3.308499in}}{\pgfqpoint{1.795760in}{3.314323in}}%
\pgfpathcurveto{\pgfqpoint{1.789936in}{3.320147in}}{\pgfqpoint{1.782036in}{3.323419in}}{\pgfqpoint{1.773800in}{3.323419in}}%
\pgfpathcurveto{\pgfqpoint{1.765563in}{3.323419in}}{\pgfqpoint{1.757663in}{3.320147in}}{\pgfqpoint{1.751839in}{3.314323in}}%
\pgfpathcurveto{\pgfqpoint{1.746015in}{3.308499in}}{\pgfqpoint{1.742743in}{3.300599in}}{\pgfqpoint{1.742743in}{3.292363in}}%
\pgfpathcurveto{\pgfqpoint{1.742743in}{3.284126in}}{\pgfqpoint{1.746015in}{3.276226in}}{\pgfqpoint{1.751839in}{3.270402in}}%
\pgfpathcurveto{\pgfqpoint{1.757663in}{3.264579in}}{\pgfqpoint{1.765563in}{3.261306in}}{\pgfqpoint{1.773800in}{3.261306in}}%
\pgfpathclose%
\pgfusepath{stroke,fill}%
\end{pgfscope}%
\begin{pgfscope}%
\pgfpathrectangle{\pgfqpoint{0.100000in}{0.220728in}}{\pgfqpoint{3.696000in}{3.696000in}}%
\pgfusepath{clip}%
\pgfsetbuttcap%
\pgfsetroundjoin%
\definecolor{currentfill}{rgb}{0.121569,0.466667,0.705882}%
\pgfsetfillcolor{currentfill}%
\pgfsetfillopacity{0.300139}%
\pgfsetlinewidth{1.003750pt}%
\definecolor{currentstroke}{rgb}{0.121569,0.466667,0.705882}%
\pgfsetstrokecolor{currentstroke}%
\pgfsetstrokeopacity{0.300139}%
\pgfsetdash{}{0pt}%
\pgfpathmoveto{\pgfqpoint{1.771942in}{3.260125in}}%
\pgfpathcurveto{\pgfqpoint{1.780178in}{3.260125in}}{\pgfqpoint{1.788079in}{3.263397in}}{\pgfqpoint{1.793902in}{3.269221in}}%
\pgfpathcurveto{\pgfqpoint{1.799726in}{3.275045in}}{\pgfqpoint{1.802999in}{3.282945in}}{\pgfqpoint{1.802999in}{3.291181in}}%
\pgfpathcurveto{\pgfqpoint{1.802999in}{3.299418in}}{\pgfqpoint{1.799726in}{3.307318in}}{\pgfqpoint{1.793902in}{3.313142in}}%
\pgfpathcurveto{\pgfqpoint{1.788079in}{3.318966in}}{\pgfqpoint{1.780178in}{3.322238in}}{\pgfqpoint{1.771942in}{3.322238in}}%
\pgfpathcurveto{\pgfqpoint{1.763706in}{3.322238in}}{\pgfqpoint{1.755806in}{3.318966in}}{\pgfqpoint{1.749982in}{3.313142in}}%
\pgfpathcurveto{\pgfqpoint{1.744158in}{3.307318in}}{\pgfqpoint{1.740886in}{3.299418in}}{\pgfqpoint{1.740886in}{3.291181in}}%
\pgfpathcurveto{\pgfqpoint{1.740886in}{3.282945in}}{\pgfqpoint{1.744158in}{3.275045in}}{\pgfqpoint{1.749982in}{3.269221in}}%
\pgfpathcurveto{\pgfqpoint{1.755806in}{3.263397in}}{\pgfqpoint{1.763706in}{3.260125in}}{\pgfqpoint{1.771942in}{3.260125in}}%
\pgfpathclose%
\pgfusepath{stroke,fill}%
\end{pgfscope}%
\begin{pgfscope}%
\pgfpathrectangle{\pgfqpoint{0.100000in}{0.220728in}}{\pgfqpoint{3.696000in}{3.696000in}}%
\pgfusepath{clip}%
\pgfsetbuttcap%
\pgfsetroundjoin%
\definecolor{currentfill}{rgb}{0.121569,0.466667,0.705882}%
\pgfsetfillcolor{currentfill}%
\pgfsetfillopacity{0.300195}%
\pgfsetlinewidth{1.003750pt}%
\definecolor{currentstroke}{rgb}{0.121569,0.466667,0.705882}%
\pgfsetstrokecolor{currentstroke}%
\pgfsetstrokeopacity{0.300195}%
\pgfsetdash{}{0pt}%
\pgfpathmoveto{\pgfqpoint{1.776593in}{3.262683in}}%
\pgfpathcurveto{\pgfqpoint{1.784830in}{3.262683in}}{\pgfqpoint{1.792730in}{3.265955in}}{\pgfqpoint{1.798554in}{3.271779in}}%
\pgfpathcurveto{\pgfqpoint{1.804378in}{3.277603in}}{\pgfqpoint{1.807650in}{3.285503in}}{\pgfqpoint{1.807650in}{3.293739in}}%
\pgfpathcurveto{\pgfqpoint{1.807650in}{3.301976in}}{\pgfqpoint{1.804378in}{3.309876in}}{\pgfqpoint{1.798554in}{3.315700in}}%
\pgfpathcurveto{\pgfqpoint{1.792730in}{3.321523in}}{\pgfqpoint{1.784830in}{3.324796in}}{\pgfqpoint{1.776593in}{3.324796in}}%
\pgfpathcurveto{\pgfqpoint{1.768357in}{3.324796in}}{\pgfqpoint{1.760457in}{3.321523in}}{\pgfqpoint{1.754633in}{3.315700in}}%
\pgfpathcurveto{\pgfqpoint{1.748809in}{3.309876in}}{\pgfqpoint{1.745537in}{3.301976in}}{\pgfqpoint{1.745537in}{3.293739in}}%
\pgfpathcurveto{\pgfqpoint{1.745537in}{3.285503in}}{\pgfqpoint{1.748809in}{3.277603in}}{\pgfqpoint{1.754633in}{3.271779in}}%
\pgfpathcurveto{\pgfqpoint{1.760457in}{3.265955in}}{\pgfqpoint{1.768357in}{3.262683in}}{\pgfqpoint{1.776593in}{3.262683in}}%
\pgfpathclose%
\pgfusepath{stroke,fill}%
\end{pgfscope}%
\begin{pgfscope}%
\pgfpathrectangle{\pgfqpoint{0.100000in}{0.220728in}}{\pgfqpoint{3.696000in}{3.696000in}}%
\pgfusepath{clip}%
\pgfsetbuttcap%
\pgfsetroundjoin%
\definecolor{currentfill}{rgb}{0.121569,0.466667,0.705882}%
\pgfsetfillcolor{currentfill}%
\pgfsetfillopacity{0.300231}%
\pgfsetlinewidth{1.003750pt}%
\definecolor{currentstroke}{rgb}{0.121569,0.466667,0.705882}%
\pgfsetstrokecolor{currentstroke}%
\pgfsetstrokeopacity{0.300231}%
\pgfsetdash{}{0pt}%
\pgfpathmoveto{\pgfqpoint{1.768640in}{3.256927in}}%
\pgfpathcurveto{\pgfqpoint{1.776876in}{3.256927in}}{\pgfqpoint{1.784776in}{3.260200in}}{\pgfqpoint{1.790600in}{3.266024in}}%
\pgfpathcurveto{\pgfqpoint{1.796424in}{3.271848in}}{\pgfqpoint{1.799696in}{3.279748in}}{\pgfqpoint{1.799696in}{3.287984in}}%
\pgfpathcurveto{\pgfqpoint{1.799696in}{3.296220in}}{\pgfqpoint{1.796424in}{3.304120in}}{\pgfqpoint{1.790600in}{3.309944in}}%
\pgfpathcurveto{\pgfqpoint{1.784776in}{3.315768in}}{\pgfqpoint{1.776876in}{3.319040in}}{\pgfqpoint{1.768640in}{3.319040in}}%
\pgfpathcurveto{\pgfqpoint{1.760403in}{3.319040in}}{\pgfqpoint{1.752503in}{3.315768in}}{\pgfqpoint{1.746679in}{3.309944in}}%
\pgfpathcurveto{\pgfqpoint{1.740855in}{3.304120in}}{\pgfqpoint{1.737583in}{3.296220in}}{\pgfqpoint{1.737583in}{3.287984in}}%
\pgfpathcurveto{\pgfqpoint{1.737583in}{3.279748in}}{\pgfqpoint{1.740855in}{3.271848in}}{\pgfqpoint{1.746679in}{3.266024in}}%
\pgfpathcurveto{\pgfqpoint{1.752503in}{3.260200in}}{\pgfqpoint{1.760403in}{3.256927in}}{\pgfqpoint{1.768640in}{3.256927in}}%
\pgfpathclose%
\pgfusepath{stroke,fill}%
\end{pgfscope}%
\begin{pgfscope}%
\pgfpathrectangle{\pgfqpoint{0.100000in}{0.220728in}}{\pgfqpoint{3.696000in}{3.696000in}}%
\pgfusepath{clip}%
\pgfsetbuttcap%
\pgfsetroundjoin%
\definecolor{currentfill}{rgb}{0.121569,0.466667,0.705882}%
\pgfsetfillcolor{currentfill}%
\pgfsetfillopacity{0.300519}%
\pgfsetlinewidth{1.003750pt}%
\definecolor{currentstroke}{rgb}{0.121569,0.466667,0.705882}%
\pgfsetstrokecolor{currentstroke}%
\pgfsetstrokeopacity{0.300519}%
\pgfsetdash{}{0pt}%
\pgfpathmoveto{\pgfqpoint{1.766478in}{3.254266in}}%
\pgfpathcurveto{\pgfqpoint{1.774714in}{3.254266in}}{\pgfqpoint{1.782614in}{3.257538in}}{\pgfqpoint{1.788438in}{3.263362in}}%
\pgfpathcurveto{\pgfqpoint{1.794262in}{3.269186in}}{\pgfqpoint{1.797534in}{3.277086in}}{\pgfqpoint{1.797534in}{3.285322in}}%
\pgfpathcurveto{\pgfqpoint{1.797534in}{3.293559in}}{\pgfqpoint{1.794262in}{3.301459in}}{\pgfqpoint{1.788438in}{3.307283in}}%
\pgfpathcurveto{\pgfqpoint{1.782614in}{3.313107in}}{\pgfqpoint{1.774714in}{3.316379in}}{\pgfqpoint{1.766478in}{3.316379in}}%
\pgfpathcurveto{\pgfqpoint{1.758242in}{3.316379in}}{\pgfqpoint{1.750342in}{3.313107in}}{\pgfqpoint{1.744518in}{3.307283in}}%
\pgfpathcurveto{\pgfqpoint{1.738694in}{3.301459in}}{\pgfqpoint{1.735421in}{3.293559in}}{\pgfqpoint{1.735421in}{3.285322in}}%
\pgfpathcurveto{\pgfqpoint{1.735421in}{3.277086in}}{\pgfqpoint{1.738694in}{3.269186in}}{\pgfqpoint{1.744518in}{3.263362in}}%
\pgfpathcurveto{\pgfqpoint{1.750342in}{3.257538in}}{\pgfqpoint{1.758242in}{3.254266in}}{\pgfqpoint{1.766478in}{3.254266in}}%
\pgfpathclose%
\pgfusepath{stroke,fill}%
\end{pgfscope}%
\begin{pgfscope}%
\pgfpathrectangle{\pgfqpoint{0.100000in}{0.220728in}}{\pgfqpoint{3.696000in}{3.696000in}}%
\pgfusepath{clip}%
\pgfsetbuttcap%
\pgfsetroundjoin%
\definecolor{currentfill}{rgb}{0.121569,0.466667,0.705882}%
\pgfsetfillcolor{currentfill}%
\pgfsetfillopacity{0.300656}%
\pgfsetlinewidth{1.003750pt}%
\definecolor{currentstroke}{rgb}{0.121569,0.466667,0.705882}%
\pgfsetstrokecolor{currentstroke}%
\pgfsetstrokeopacity{0.300656}%
\pgfsetdash{}{0pt}%
\pgfpathmoveto{\pgfqpoint{1.765828in}{3.253008in}}%
\pgfpathcurveto{\pgfqpoint{1.774064in}{3.253008in}}{\pgfqpoint{1.781964in}{3.256281in}}{\pgfqpoint{1.787788in}{3.262104in}}%
\pgfpathcurveto{\pgfqpoint{1.793612in}{3.267928in}}{\pgfqpoint{1.796884in}{3.275828in}}{\pgfqpoint{1.796884in}{3.284065in}}%
\pgfpathcurveto{\pgfqpoint{1.796884in}{3.292301in}}{\pgfqpoint{1.793612in}{3.300201in}}{\pgfqpoint{1.787788in}{3.306025in}}%
\pgfpathcurveto{\pgfqpoint{1.781964in}{3.311849in}}{\pgfqpoint{1.774064in}{3.315121in}}{\pgfqpoint{1.765828in}{3.315121in}}%
\pgfpathcurveto{\pgfqpoint{1.757592in}{3.315121in}}{\pgfqpoint{1.749692in}{3.311849in}}{\pgfqpoint{1.743868in}{3.306025in}}%
\pgfpathcurveto{\pgfqpoint{1.738044in}{3.300201in}}{\pgfqpoint{1.734771in}{3.292301in}}{\pgfqpoint{1.734771in}{3.284065in}}%
\pgfpathcurveto{\pgfqpoint{1.734771in}{3.275828in}}{\pgfqpoint{1.738044in}{3.267928in}}{\pgfqpoint{1.743868in}{3.262104in}}%
\pgfpathcurveto{\pgfqpoint{1.749692in}{3.256281in}}{\pgfqpoint{1.757592in}{3.253008in}}{\pgfqpoint{1.765828in}{3.253008in}}%
\pgfpathclose%
\pgfusepath{stroke,fill}%
\end{pgfscope}%
\begin{pgfscope}%
\pgfpathrectangle{\pgfqpoint{0.100000in}{0.220728in}}{\pgfqpoint{3.696000in}{3.696000in}}%
\pgfusepath{clip}%
\pgfsetbuttcap%
\pgfsetroundjoin%
\definecolor{currentfill}{rgb}{0.121569,0.466667,0.705882}%
\pgfsetfillcolor{currentfill}%
\pgfsetfillopacity{0.300717}%
\pgfsetlinewidth{1.003750pt}%
\definecolor{currentstroke}{rgb}{0.121569,0.466667,0.705882}%
\pgfsetstrokecolor{currentstroke}%
\pgfsetstrokeopacity{0.300717}%
\pgfsetdash{}{0pt}%
\pgfpathmoveto{\pgfqpoint{1.765642in}{3.252571in}}%
\pgfpathcurveto{\pgfqpoint{1.773878in}{3.252571in}}{\pgfqpoint{1.781778in}{3.255843in}}{\pgfqpoint{1.787602in}{3.261667in}}%
\pgfpathcurveto{\pgfqpoint{1.793426in}{3.267491in}}{\pgfqpoint{1.796698in}{3.275391in}}{\pgfqpoint{1.796698in}{3.283627in}}%
\pgfpathcurveto{\pgfqpoint{1.796698in}{3.291864in}}{\pgfqpoint{1.793426in}{3.299764in}}{\pgfqpoint{1.787602in}{3.305588in}}%
\pgfpathcurveto{\pgfqpoint{1.781778in}{3.311412in}}{\pgfqpoint{1.773878in}{3.314684in}}{\pgfqpoint{1.765642in}{3.314684in}}%
\pgfpathcurveto{\pgfqpoint{1.757405in}{3.314684in}}{\pgfqpoint{1.749505in}{3.311412in}}{\pgfqpoint{1.743681in}{3.305588in}}%
\pgfpathcurveto{\pgfqpoint{1.737857in}{3.299764in}}{\pgfqpoint{1.734585in}{3.291864in}}{\pgfqpoint{1.734585in}{3.283627in}}%
\pgfpathcurveto{\pgfqpoint{1.734585in}{3.275391in}}{\pgfqpoint{1.737857in}{3.267491in}}{\pgfqpoint{1.743681in}{3.261667in}}%
\pgfpathcurveto{\pgfqpoint{1.749505in}{3.255843in}}{\pgfqpoint{1.757405in}{3.252571in}}{\pgfqpoint{1.765642in}{3.252571in}}%
\pgfpathclose%
\pgfusepath{stroke,fill}%
\end{pgfscope}%
\begin{pgfscope}%
\pgfpathrectangle{\pgfqpoint{0.100000in}{0.220728in}}{\pgfqpoint{3.696000in}{3.696000in}}%
\pgfusepath{clip}%
\pgfsetbuttcap%
\pgfsetroundjoin%
\definecolor{currentfill}{rgb}{0.121569,0.466667,0.705882}%
\pgfsetfillcolor{currentfill}%
\pgfsetfillopacity{0.300727}%
\pgfsetlinewidth{1.003750pt}%
\definecolor{currentstroke}{rgb}{0.121569,0.466667,0.705882}%
\pgfsetstrokecolor{currentstroke}%
\pgfsetstrokeopacity{0.300727}%
\pgfsetdash{}{0pt}%
\pgfpathmoveto{\pgfqpoint{1.780057in}{3.263735in}}%
\pgfpathcurveto{\pgfqpoint{1.788293in}{3.263735in}}{\pgfqpoint{1.796193in}{3.267008in}}{\pgfqpoint{1.802017in}{3.272832in}}%
\pgfpathcurveto{\pgfqpoint{1.807841in}{3.278655in}}{\pgfqpoint{1.811113in}{3.286556in}}{\pgfqpoint{1.811113in}{3.294792in}}%
\pgfpathcurveto{\pgfqpoint{1.811113in}{3.303028in}}{\pgfqpoint{1.807841in}{3.310928in}}{\pgfqpoint{1.802017in}{3.316752in}}%
\pgfpathcurveto{\pgfqpoint{1.796193in}{3.322576in}}{\pgfqpoint{1.788293in}{3.325848in}}{\pgfqpoint{1.780057in}{3.325848in}}%
\pgfpathcurveto{\pgfqpoint{1.771821in}{3.325848in}}{\pgfqpoint{1.763921in}{3.322576in}}{\pgfqpoint{1.758097in}{3.316752in}}%
\pgfpathcurveto{\pgfqpoint{1.752273in}{3.310928in}}{\pgfqpoint{1.749000in}{3.303028in}}{\pgfqpoint{1.749000in}{3.294792in}}%
\pgfpathcurveto{\pgfqpoint{1.749000in}{3.286556in}}{\pgfqpoint{1.752273in}{3.278655in}}{\pgfqpoint{1.758097in}{3.272832in}}%
\pgfpathcurveto{\pgfqpoint{1.763921in}{3.267008in}}{\pgfqpoint{1.771821in}{3.263735in}}{\pgfqpoint{1.780057in}{3.263735in}}%
\pgfpathclose%
\pgfusepath{stroke,fill}%
\end{pgfscope}%
\begin{pgfscope}%
\pgfpathrectangle{\pgfqpoint{0.100000in}{0.220728in}}{\pgfqpoint{3.696000in}{3.696000in}}%
\pgfusepath{clip}%
\pgfsetbuttcap%
\pgfsetroundjoin%
\definecolor{currentfill}{rgb}{0.121569,0.466667,0.705882}%
\pgfsetfillcolor{currentfill}%
\pgfsetfillopacity{0.300839}%
\pgfsetlinewidth{1.003750pt}%
\definecolor{currentstroke}{rgb}{0.121569,0.466667,0.705882}%
\pgfsetstrokecolor{currentstroke}%
\pgfsetstrokeopacity{0.300839}%
\pgfsetdash{}{0pt}%
\pgfpathmoveto{\pgfqpoint{1.765314in}{3.251811in}}%
\pgfpathcurveto{\pgfqpoint{1.773551in}{3.251811in}}{\pgfqpoint{1.781451in}{3.255083in}}{\pgfqpoint{1.787275in}{3.260907in}}%
\pgfpathcurveto{\pgfqpoint{1.793098in}{3.266731in}}{\pgfqpoint{1.796371in}{3.274631in}}{\pgfqpoint{1.796371in}{3.282867in}}%
\pgfpathcurveto{\pgfqpoint{1.796371in}{3.291103in}}{\pgfqpoint{1.793098in}{3.299004in}}{\pgfqpoint{1.787275in}{3.304827in}}%
\pgfpathcurveto{\pgfqpoint{1.781451in}{3.310651in}}{\pgfqpoint{1.773551in}{3.313924in}}{\pgfqpoint{1.765314in}{3.313924in}}%
\pgfpathcurveto{\pgfqpoint{1.757078in}{3.313924in}}{\pgfqpoint{1.749178in}{3.310651in}}{\pgfqpoint{1.743354in}{3.304827in}}%
\pgfpathcurveto{\pgfqpoint{1.737530in}{3.299004in}}{\pgfqpoint{1.734258in}{3.291103in}}{\pgfqpoint{1.734258in}{3.282867in}}%
\pgfpathcurveto{\pgfqpoint{1.734258in}{3.274631in}}{\pgfqpoint{1.737530in}{3.266731in}}{\pgfqpoint{1.743354in}{3.260907in}}%
\pgfpathcurveto{\pgfqpoint{1.749178in}{3.255083in}}{\pgfqpoint{1.757078in}{3.251811in}}{\pgfqpoint{1.765314in}{3.251811in}}%
\pgfpathclose%
\pgfusepath{stroke,fill}%
\end{pgfscope}%
\begin{pgfscope}%
\pgfpathrectangle{\pgfqpoint{0.100000in}{0.220728in}}{\pgfqpoint{3.696000in}{3.696000in}}%
\pgfusepath{clip}%
\pgfsetbuttcap%
\pgfsetroundjoin%
\definecolor{currentfill}{rgb}{0.121569,0.466667,0.705882}%
\pgfsetfillcolor{currentfill}%
\pgfsetfillopacity{0.301043}%
\pgfsetlinewidth{1.003750pt}%
\definecolor{currentstroke}{rgb}{0.121569,0.466667,0.705882}%
\pgfsetstrokecolor{currentstroke}%
\pgfsetstrokeopacity{0.301043}%
\pgfsetdash{}{0pt}%
\pgfpathmoveto{\pgfqpoint{1.781863in}{3.263805in}}%
\pgfpathcurveto{\pgfqpoint{1.790099in}{3.263805in}}{\pgfqpoint{1.797999in}{3.267077in}}{\pgfqpoint{1.803823in}{3.272901in}}%
\pgfpathcurveto{\pgfqpoint{1.809647in}{3.278725in}}{\pgfqpoint{1.812919in}{3.286625in}}{\pgfqpoint{1.812919in}{3.294861in}}%
\pgfpathcurveto{\pgfqpoint{1.812919in}{3.303098in}}{\pgfqpoint{1.809647in}{3.310998in}}{\pgfqpoint{1.803823in}{3.316822in}}%
\pgfpathcurveto{\pgfqpoint{1.797999in}{3.322645in}}{\pgfqpoint{1.790099in}{3.325918in}}{\pgfqpoint{1.781863in}{3.325918in}}%
\pgfpathcurveto{\pgfqpoint{1.773627in}{3.325918in}}{\pgfqpoint{1.765727in}{3.322645in}}{\pgfqpoint{1.759903in}{3.316822in}}%
\pgfpathcurveto{\pgfqpoint{1.754079in}{3.310998in}}{\pgfqpoint{1.750806in}{3.303098in}}{\pgfqpoint{1.750806in}{3.294861in}}%
\pgfpathcurveto{\pgfqpoint{1.750806in}{3.286625in}}{\pgfqpoint{1.754079in}{3.278725in}}{\pgfqpoint{1.759903in}{3.272901in}}%
\pgfpathcurveto{\pgfqpoint{1.765727in}{3.267077in}}{\pgfqpoint{1.773627in}{3.263805in}}{\pgfqpoint{1.781863in}{3.263805in}}%
\pgfpathclose%
\pgfusepath{stroke,fill}%
\end{pgfscope}%
\begin{pgfscope}%
\pgfpathrectangle{\pgfqpoint{0.100000in}{0.220728in}}{\pgfqpoint{3.696000in}{3.696000in}}%
\pgfusepath{clip}%
\pgfsetbuttcap%
\pgfsetroundjoin%
\definecolor{currentfill}{rgb}{0.121569,0.466667,0.705882}%
\pgfsetfillcolor{currentfill}%
\pgfsetfillopacity{0.301076}%
\pgfsetlinewidth{1.003750pt}%
\definecolor{currentstroke}{rgb}{0.121569,0.466667,0.705882}%
\pgfsetstrokecolor{currentstroke}%
\pgfsetstrokeopacity{0.301076}%
\pgfsetdash{}{0pt}%
\pgfpathmoveto{\pgfqpoint{1.764731in}{3.250488in}}%
\pgfpathcurveto{\pgfqpoint{1.772967in}{3.250488in}}{\pgfqpoint{1.780867in}{3.253761in}}{\pgfqpoint{1.786691in}{3.259585in}}%
\pgfpathcurveto{\pgfqpoint{1.792515in}{3.265409in}}{\pgfqpoint{1.795787in}{3.273309in}}{\pgfqpoint{1.795787in}{3.281545in}}%
\pgfpathcurveto{\pgfqpoint{1.795787in}{3.289781in}}{\pgfqpoint{1.792515in}{3.297681in}}{\pgfqpoint{1.786691in}{3.303505in}}%
\pgfpathcurveto{\pgfqpoint{1.780867in}{3.309329in}}{\pgfqpoint{1.772967in}{3.312601in}}{\pgfqpoint{1.764731in}{3.312601in}}%
\pgfpathcurveto{\pgfqpoint{1.756494in}{3.312601in}}{\pgfqpoint{1.748594in}{3.309329in}}{\pgfqpoint{1.742770in}{3.303505in}}%
\pgfpathcurveto{\pgfqpoint{1.736946in}{3.297681in}}{\pgfqpoint{1.733674in}{3.289781in}}{\pgfqpoint{1.733674in}{3.281545in}}%
\pgfpathcurveto{\pgfqpoint{1.733674in}{3.273309in}}{\pgfqpoint{1.736946in}{3.265409in}}{\pgfqpoint{1.742770in}{3.259585in}}%
\pgfpathcurveto{\pgfqpoint{1.748594in}{3.253761in}}{\pgfqpoint{1.756494in}{3.250488in}}{\pgfqpoint{1.764731in}{3.250488in}}%
\pgfpathclose%
\pgfusepath{stroke,fill}%
\end{pgfscope}%
\begin{pgfscope}%
\pgfpathrectangle{\pgfqpoint{0.100000in}{0.220728in}}{\pgfqpoint{3.696000in}{3.696000in}}%
\pgfusepath{clip}%
\pgfsetbuttcap%
\pgfsetroundjoin%
\definecolor{currentfill}{rgb}{0.121569,0.466667,0.705882}%
\pgfsetfillcolor{currentfill}%
\pgfsetfillopacity{0.301119}%
\pgfsetlinewidth{1.003750pt}%
\definecolor{currentstroke}{rgb}{0.121569,0.466667,0.705882}%
\pgfsetstrokecolor{currentstroke}%
\pgfsetstrokeopacity{0.301119}%
\pgfsetdash{}{0pt}%
\pgfpathmoveto{\pgfqpoint{1.782915in}{3.263759in}}%
\pgfpathcurveto{\pgfqpoint{1.791151in}{3.263759in}}{\pgfqpoint{1.799051in}{3.267031in}}{\pgfqpoint{1.804875in}{3.272855in}}%
\pgfpathcurveto{\pgfqpoint{1.810699in}{3.278679in}}{\pgfqpoint{1.813971in}{3.286579in}}{\pgfqpoint{1.813971in}{3.294815in}}%
\pgfpathcurveto{\pgfqpoint{1.813971in}{3.303052in}}{\pgfqpoint{1.810699in}{3.310952in}}{\pgfqpoint{1.804875in}{3.316776in}}%
\pgfpathcurveto{\pgfqpoint{1.799051in}{3.322600in}}{\pgfqpoint{1.791151in}{3.325872in}}{\pgfqpoint{1.782915in}{3.325872in}}%
\pgfpathcurveto{\pgfqpoint{1.774679in}{3.325872in}}{\pgfqpoint{1.766779in}{3.322600in}}{\pgfqpoint{1.760955in}{3.316776in}}%
\pgfpathcurveto{\pgfqpoint{1.755131in}{3.310952in}}{\pgfqpoint{1.751858in}{3.303052in}}{\pgfqpoint{1.751858in}{3.294815in}}%
\pgfpathcurveto{\pgfqpoint{1.751858in}{3.286579in}}{\pgfqpoint{1.755131in}{3.278679in}}{\pgfqpoint{1.760955in}{3.272855in}}%
\pgfpathcurveto{\pgfqpoint{1.766779in}{3.267031in}}{\pgfqpoint{1.774679in}{3.263759in}}{\pgfqpoint{1.782915in}{3.263759in}}%
\pgfpathclose%
\pgfusepath{stroke,fill}%
\end{pgfscope}%
\begin{pgfscope}%
\pgfpathrectangle{\pgfqpoint{0.100000in}{0.220728in}}{\pgfqpoint{3.696000in}{3.696000in}}%
\pgfusepath{clip}%
\pgfsetbuttcap%
\pgfsetroundjoin%
\definecolor{currentfill}{rgb}{0.121569,0.466667,0.705882}%
\pgfsetfillcolor{currentfill}%
\pgfsetfillopacity{0.301205}%
\pgfsetlinewidth{1.003750pt}%
\definecolor{currentstroke}{rgb}{0.121569,0.466667,0.705882}%
\pgfsetstrokecolor{currentstroke}%
\pgfsetstrokeopacity{0.301205}%
\pgfsetdash{}{0pt}%
\pgfpathmoveto{\pgfqpoint{1.764447in}{3.249746in}}%
\pgfpathcurveto{\pgfqpoint{1.772684in}{3.249746in}}{\pgfqpoint{1.780584in}{3.253018in}}{\pgfqpoint{1.786408in}{3.258842in}}%
\pgfpathcurveto{\pgfqpoint{1.792231in}{3.264666in}}{\pgfqpoint{1.795504in}{3.272566in}}{\pgfqpoint{1.795504in}{3.280803in}}%
\pgfpathcurveto{\pgfqpoint{1.795504in}{3.289039in}}{\pgfqpoint{1.792231in}{3.296939in}}{\pgfqpoint{1.786408in}{3.302763in}}%
\pgfpathcurveto{\pgfqpoint{1.780584in}{3.308587in}}{\pgfqpoint{1.772684in}{3.311859in}}{\pgfqpoint{1.764447in}{3.311859in}}%
\pgfpathcurveto{\pgfqpoint{1.756211in}{3.311859in}}{\pgfqpoint{1.748311in}{3.308587in}}{\pgfqpoint{1.742487in}{3.302763in}}%
\pgfpathcurveto{\pgfqpoint{1.736663in}{3.296939in}}{\pgfqpoint{1.733391in}{3.289039in}}{\pgfqpoint{1.733391in}{3.280803in}}%
\pgfpathcurveto{\pgfqpoint{1.733391in}{3.272566in}}{\pgfqpoint{1.736663in}{3.264666in}}{\pgfqpoint{1.742487in}{3.258842in}}%
\pgfpathcurveto{\pgfqpoint{1.748311in}{3.253018in}}{\pgfqpoint{1.756211in}{3.249746in}}{\pgfqpoint{1.764447in}{3.249746in}}%
\pgfpathclose%
\pgfusepath{stroke,fill}%
\end{pgfscope}%
\begin{pgfscope}%
\pgfpathrectangle{\pgfqpoint{0.100000in}{0.220728in}}{\pgfqpoint{3.696000in}{3.696000in}}%
\pgfusepath{clip}%
\pgfsetbuttcap%
\pgfsetroundjoin%
\definecolor{currentfill}{rgb}{0.121569,0.466667,0.705882}%
\pgfsetfillcolor{currentfill}%
\pgfsetfillopacity{0.301219}%
\pgfsetlinewidth{1.003750pt}%
\definecolor{currentstroke}{rgb}{0.121569,0.466667,0.705882}%
\pgfsetstrokecolor{currentstroke}%
\pgfsetstrokeopacity{0.301219}%
\pgfsetdash{}{0pt}%
\pgfpathmoveto{\pgfqpoint{1.783439in}{3.263702in}}%
\pgfpathcurveto{\pgfqpoint{1.791675in}{3.263702in}}{\pgfqpoint{1.799575in}{3.266974in}}{\pgfqpoint{1.805399in}{3.272798in}}%
\pgfpathcurveto{\pgfqpoint{1.811223in}{3.278622in}}{\pgfqpoint{1.814495in}{3.286522in}}{\pgfqpoint{1.814495in}{3.294758in}}%
\pgfpathcurveto{\pgfqpoint{1.814495in}{3.302994in}}{\pgfqpoint{1.811223in}{3.310894in}}{\pgfqpoint{1.805399in}{3.316718in}}%
\pgfpathcurveto{\pgfqpoint{1.799575in}{3.322542in}}{\pgfqpoint{1.791675in}{3.325815in}}{\pgfqpoint{1.783439in}{3.325815in}}%
\pgfpathcurveto{\pgfqpoint{1.775203in}{3.325815in}}{\pgfqpoint{1.767303in}{3.322542in}}{\pgfqpoint{1.761479in}{3.316718in}}%
\pgfpathcurveto{\pgfqpoint{1.755655in}{3.310894in}}{\pgfqpoint{1.752382in}{3.302994in}}{\pgfqpoint{1.752382in}{3.294758in}}%
\pgfpathcurveto{\pgfqpoint{1.752382in}{3.286522in}}{\pgfqpoint{1.755655in}{3.278622in}}{\pgfqpoint{1.761479in}{3.272798in}}%
\pgfpathcurveto{\pgfqpoint{1.767303in}{3.266974in}}{\pgfqpoint{1.775203in}{3.263702in}}{\pgfqpoint{1.783439in}{3.263702in}}%
\pgfpathclose%
\pgfusepath{stroke,fill}%
\end{pgfscope}%
\begin{pgfscope}%
\pgfpathrectangle{\pgfqpoint{0.100000in}{0.220728in}}{\pgfqpoint{3.696000in}{3.696000in}}%
\pgfusepath{clip}%
\pgfsetbuttcap%
\pgfsetroundjoin%
\definecolor{currentfill}{rgb}{0.121569,0.466667,0.705882}%
\pgfsetfillcolor{currentfill}%
\pgfsetfillopacity{0.301424}%
\pgfsetlinewidth{1.003750pt}%
\definecolor{currentstroke}{rgb}{0.121569,0.466667,0.705882}%
\pgfsetstrokecolor{currentstroke}%
\pgfsetstrokeopacity{0.301424}%
\pgfsetdash{}{0pt}%
\pgfpathmoveto{\pgfqpoint{1.763793in}{3.248463in}}%
\pgfpathcurveto{\pgfqpoint{1.772030in}{3.248463in}}{\pgfqpoint{1.779930in}{3.251735in}}{\pgfqpoint{1.785754in}{3.257559in}}%
\pgfpathcurveto{\pgfqpoint{1.791578in}{3.263383in}}{\pgfqpoint{1.794850in}{3.271283in}}{\pgfqpoint{1.794850in}{3.279519in}}%
\pgfpathcurveto{\pgfqpoint{1.794850in}{3.287756in}}{\pgfqpoint{1.791578in}{3.295656in}}{\pgfqpoint{1.785754in}{3.301480in}}%
\pgfpathcurveto{\pgfqpoint{1.779930in}{3.307303in}}{\pgfqpoint{1.772030in}{3.310576in}}{\pgfqpoint{1.763793in}{3.310576in}}%
\pgfpathcurveto{\pgfqpoint{1.755557in}{3.310576in}}{\pgfqpoint{1.747657in}{3.307303in}}{\pgfqpoint{1.741833in}{3.301480in}}%
\pgfpathcurveto{\pgfqpoint{1.736009in}{3.295656in}}{\pgfqpoint{1.732737in}{3.287756in}}{\pgfqpoint{1.732737in}{3.279519in}}%
\pgfpathcurveto{\pgfqpoint{1.732737in}{3.271283in}}{\pgfqpoint{1.736009in}{3.263383in}}{\pgfqpoint{1.741833in}{3.257559in}}%
\pgfpathcurveto{\pgfqpoint{1.747657in}{3.251735in}}{\pgfqpoint{1.755557in}{3.248463in}}{\pgfqpoint{1.763793in}{3.248463in}}%
\pgfpathclose%
\pgfusepath{stroke,fill}%
\end{pgfscope}%
\begin{pgfscope}%
\pgfpathrectangle{\pgfqpoint{0.100000in}{0.220728in}}{\pgfqpoint{3.696000in}{3.696000in}}%
\pgfusepath{clip}%
\pgfsetbuttcap%
\pgfsetroundjoin%
\definecolor{currentfill}{rgb}{0.121569,0.466667,0.705882}%
\pgfsetfillcolor{currentfill}%
\pgfsetfillopacity{0.301468}%
\pgfsetlinewidth{1.003750pt}%
\definecolor{currentstroke}{rgb}{0.121569,0.466667,0.705882}%
\pgfsetstrokecolor{currentstroke}%
\pgfsetstrokeopacity{0.301468}%
\pgfsetdash{}{0pt}%
\pgfpathmoveto{\pgfqpoint{1.784323in}{3.263562in}}%
\pgfpathcurveto{\pgfqpoint{1.792560in}{3.263562in}}{\pgfqpoint{1.800460in}{3.266834in}}{\pgfqpoint{1.806283in}{3.272658in}}%
\pgfpathcurveto{\pgfqpoint{1.812107in}{3.278482in}}{\pgfqpoint{1.815380in}{3.286382in}}{\pgfqpoint{1.815380in}{3.294618in}}%
\pgfpathcurveto{\pgfqpoint{1.815380in}{3.302854in}}{\pgfqpoint{1.812107in}{3.310754in}}{\pgfqpoint{1.806283in}{3.316578in}}%
\pgfpathcurveto{\pgfqpoint{1.800460in}{3.322402in}}{\pgfqpoint{1.792560in}{3.325675in}}{\pgfqpoint{1.784323in}{3.325675in}}%
\pgfpathcurveto{\pgfqpoint{1.776087in}{3.325675in}}{\pgfqpoint{1.768187in}{3.322402in}}{\pgfqpoint{1.762363in}{3.316578in}}%
\pgfpathcurveto{\pgfqpoint{1.756539in}{3.310754in}}{\pgfqpoint{1.753267in}{3.302854in}}{\pgfqpoint{1.753267in}{3.294618in}}%
\pgfpathcurveto{\pgfqpoint{1.753267in}{3.286382in}}{\pgfqpoint{1.756539in}{3.278482in}}{\pgfqpoint{1.762363in}{3.272658in}}%
\pgfpathcurveto{\pgfqpoint{1.768187in}{3.266834in}}{\pgfqpoint{1.776087in}{3.263562in}}{\pgfqpoint{1.784323in}{3.263562in}}%
\pgfpathclose%
\pgfusepath{stroke,fill}%
\end{pgfscope}%
\begin{pgfscope}%
\pgfpathrectangle{\pgfqpoint{0.100000in}{0.220728in}}{\pgfqpoint{3.696000in}{3.696000in}}%
\pgfusepath{clip}%
\pgfsetbuttcap%
\pgfsetroundjoin%
\definecolor{currentfill}{rgb}{0.121569,0.466667,0.705882}%
\pgfsetfillcolor{currentfill}%
\pgfsetfillopacity{0.301536}%
\pgfsetlinewidth{1.003750pt}%
\definecolor{currentstroke}{rgb}{0.121569,0.466667,0.705882}%
\pgfsetstrokecolor{currentstroke}%
\pgfsetstrokeopacity{0.301536}%
\pgfsetdash{}{0pt}%
\pgfpathmoveto{\pgfqpoint{1.784869in}{3.263383in}}%
\pgfpathcurveto{\pgfqpoint{1.793105in}{3.263383in}}{\pgfqpoint{1.801005in}{3.266655in}}{\pgfqpoint{1.806829in}{3.272479in}}%
\pgfpathcurveto{\pgfqpoint{1.812653in}{3.278303in}}{\pgfqpoint{1.815926in}{3.286203in}}{\pgfqpoint{1.815926in}{3.294440in}}%
\pgfpathcurveto{\pgfqpoint{1.815926in}{3.302676in}}{\pgfqpoint{1.812653in}{3.310576in}}{\pgfqpoint{1.806829in}{3.316400in}}%
\pgfpathcurveto{\pgfqpoint{1.801005in}{3.322224in}}{\pgfqpoint{1.793105in}{3.325496in}}{\pgfqpoint{1.784869in}{3.325496in}}%
\pgfpathcurveto{\pgfqpoint{1.776633in}{3.325496in}}{\pgfqpoint{1.768733in}{3.322224in}}{\pgfqpoint{1.762909in}{3.316400in}}%
\pgfpathcurveto{\pgfqpoint{1.757085in}{3.310576in}}{\pgfqpoint{1.753813in}{3.302676in}}{\pgfqpoint{1.753813in}{3.294440in}}%
\pgfpathcurveto{\pgfqpoint{1.753813in}{3.286203in}}{\pgfqpoint{1.757085in}{3.278303in}}{\pgfqpoint{1.762909in}{3.272479in}}%
\pgfpathcurveto{\pgfqpoint{1.768733in}{3.266655in}}{\pgfqpoint{1.776633in}{3.263383in}}{\pgfqpoint{1.784869in}{3.263383in}}%
\pgfpathclose%
\pgfusepath{stroke,fill}%
\end{pgfscope}%
\begin{pgfscope}%
\pgfpathrectangle{\pgfqpoint{0.100000in}{0.220728in}}{\pgfqpoint{3.696000in}{3.696000in}}%
\pgfusepath{clip}%
\pgfsetbuttcap%
\pgfsetroundjoin%
\definecolor{currentfill}{rgb}{0.121569,0.466667,0.705882}%
\pgfsetfillcolor{currentfill}%
\pgfsetfillopacity{0.301608}%
\pgfsetlinewidth{1.003750pt}%
\definecolor{currentstroke}{rgb}{0.121569,0.466667,0.705882}%
\pgfsetstrokecolor{currentstroke}%
\pgfsetstrokeopacity{0.301608}%
\pgfsetdash{}{0pt}%
\pgfpathmoveto{\pgfqpoint{1.785142in}{3.263343in}}%
\pgfpathcurveto{\pgfqpoint{1.793378in}{3.263343in}}{\pgfqpoint{1.801278in}{3.266616in}}{\pgfqpoint{1.807102in}{3.272440in}}%
\pgfpathcurveto{\pgfqpoint{1.812926in}{3.278264in}}{\pgfqpoint{1.816198in}{3.286164in}}{\pgfqpoint{1.816198in}{3.294400in}}%
\pgfpathcurveto{\pgfqpoint{1.816198in}{3.302636in}}{\pgfqpoint{1.812926in}{3.310536in}}{\pgfqpoint{1.807102in}{3.316360in}}%
\pgfpathcurveto{\pgfqpoint{1.801278in}{3.322184in}}{\pgfqpoint{1.793378in}{3.325456in}}{\pgfqpoint{1.785142in}{3.325456in}}%
\pgfpathcurveto{\pgfqpoint{1.776905in}{3.325456in}}{\pgfqpoint{1.769005in}{3.322184in}}{\pgfqpoint{1.763181in}{3.316360in}}%
\pgfpathcurveto{\pgfqpoint{1.757357in}{3.310536in}}{\pgfqpoint{1.754085in}{3.302636in}}{\pgfqpoint{1.754085in}{3.294400in}}%
\pgfpathcurveto{\pgfqpoint{1.754085in}{3.286164in}}{\pgfqpoint{1.757357in}{3.278264in}}{\pgfqpoint{1.763181in}{3.272440in}}%
\pgfpathcurveto{\pgfqpoint{1.769005in}{3.266616in}}{\pgfqpoint{1.776905in}{3.263343in}}{\pgfqpoint{1.785142in}{3.263343in}}%
\pgfpathclose%
\pgfusepath{stroke,fill}%
\end{pgfscope}%
\begin{pgfscope}%
\pgfpathrectangle{\pgfqpoint{0.100000in}{0.220728in}}{\pgfqpoint{3.696000in}{3.696000in}}%
\pgfusepath{clip}%
\pgfsetbuttcap%
\pgfsetroundjoin%
\definecolor{currentfill}{rgb}{0.121569,0.466667,0.705882}%
\pgfsetfillcolor{currentfill}%
\pgfsetfillopacity{0.301636}%
\pgfsetlinewidth{1.003750pt}%
\definecolor{currentstroke}{rgb}{0.121569,0.466667,0.705882}%
\pgfsetstrokecolor{currentstroke}%
\pgfsetstrokeopacity{0.301636}%
\pgfsetdash{}{0pt}%
\pgfpathmoveto{\pgfqpoint{1.785306in}{3.263317in}}%
\pgfpathcurveto{\pgfqpoint{1.793542in}{3.263317in}}{\pgfqpoint{1.801442in}{3.266590in}}{\pgfqpoint{1.807266in}{3.272414in}}%
\pgfpathcurveto{\pgfqpoint{1.813090in}{3.278237in}}{\pgfqpoint{1.816362in}{3.286138in}}{\pgfqpoint{1.816362in}{3.294374in}}%
\pgfpathcurveto{\pgfqpoint{1.816362in}{3.302610in}}{\pgfqpoint{1.813090in}{3.310510in}}{\pgfqpoint{1.807266in}{3.316334in}}%
\pgfpathcurveto{\pgfqpoint{1.801442in}{3.322158in}}{\pgfqpoint{1.793542in}{3.325430in}}{\pgfqpoint{1.785306in}{3.325430in}}%
\pgfpathcurveto{\pgfqpoint{1.777069in}{3.325430in}}{\pgfqpoint{1.769169in}{3.322158in}}{\pgfqpoint{1.763345in}{3.316334in}}%
\pgfpathcurveto{\pgfqpoint{1.757521in}{3.310510in}}{\pgfqpoint{1.754249in}{3.302610in}}{\pgfqpoint{1.754249in}{3.294374in}}%
\pgfpathcurveto{\pgfqpoint{1.754249in}{3.286138in}}{\pgfqpoint{1.757521in}{3.278237in}}{\pgfqpoint{1.763345in}{3.272414in}}%
\pgfpathcurveto{\pgfqpoint{1.769169in}{3.266590in}}{\pgfqpoint{1.777069in}{3.263317in}}{\pgfqpoint{1.785306in}{3.263317in}}%
\pgfpathclose%
\pgfusepath{stroke,fill}%
\end{pgfscope}%
\begin{pgfscope}%
\pgfpathrectangle{\pgfqpoint{0.100000in}{0.220728in}}{\pgfqpoint{3.696000in}{3.696000in}}%
\pgfusepath{clip}%
\pgfsetbuttcap%
\pgfsetroundjoin%
\definecolor{currentfill}{rgb}{0.121569,0.466667,0.705882}%
\pgfsetfillcolor{currentfill}%
\pgfsetfillopacity{0.301781}%
\pgfsetlinewidth{1.003750pt}%
\definecolor{currentstroke}{rgb}{0.121569,0.466667,0.705882}%
\pgfsetstrokecolor{currentstroke}%
\pgfsetstrokeopacity{0.301781}%
\pgfsetdash{}{0pt}%
\pgfpathmoveto{\pgfqpoint{1.785945in}{3.263214in}}%
\pgfpathcurveto{\pgfqpoint{1.794182in}{3.263214in}}{\pgfqpoint{1.802082in}{3.266486in}}{\pgfqpoint{1.807906in}{3.272310in}}%
\pgfpathcurveto{\pgfqpoint{1.813730in}{3.278134in}}{\pgfqpoint{1.817002in}{3.286034in}}{\pgfqpoint{1.817002in}{3.294270in}}%
\pgfpathcurveto{\pgfqpoint{1.817002in}{3.302507in}}{\pgfqpoint{1.813730in}{3.310407in}}{\pgfqpoint{1.807906in}{3.316231in}}%
\pgfpathcurveto{\pgfqpoint{1.802082in}{3.322055in}}{\pgfqpoint{1.794182in}{3.325327in}}{\pgfqpoint{1.785945in}{3.325327in}}%
\pgfpathcurveto{\pgfqpoint{1.777709in}{3.325327in}}{\pgfqpoint{1.769809in}{3.322055in}}{\pgfqpoint{1.763985in}{3.316231in}}%
\pgfpathcurveto{\pgfqpoint{1.758161in}{3.310407in}}{\pgfqpoint{1.754889in}{3.302507in}}{\pgfqpoint{1.754889in}{3.294270in}}%
\pgfpathcurveto{\pgfqpoint{1.754889in}{3.286034in}}{\pgfqpoint{1.758161in}{3.278134in}}{\pgfqpoint{1.763985in}{3.272310in}}%
\pgfpathcurveto{\pgfqpoint{1.769809in}{3.266486in}}{\pgfqpoint{1.777709in}{3.263214in}}{\pgfqpoint{1.785945in}{3.263214in}}%
\pgfpathclose%
\pgfusepath{stroke,fill}%
\end{pgfscope}%
\begin{pgfscope}%
\pgfpathrectangle{\pgfqpoint{0.100000in}{0.220728in}}{\pgfqpoint{3.696000in}{3.696000in}}%
\pgfusepath{clip}%
\pgfsetbuttcap%
\pgfsetroundjoin%
\definecolor{currentfill}{rgb}{0.121569,0.466667,0.705882}%
\pgfsetfillcolor{currentfill}%
\pgfsetfillopacity{0.301842}%
\pgfsetlinewidth{1.003750pt}%
\definecolor{currentstroke}{rgb}{0.121569,0.466667,0.705882}%
\pgfsetstrokecolor{currentstroke}%
\pgfsetstrokeopacity{0.301842}%
\pgfsetdash{}{0pt}%
\pgfpathmoveto{\pgfqpoint{1.762747in}{3.246064in}}%
\pgfpathcurveto{\pgfqpoint{1.770983in}{3.246064in}}{\pgfqpoint{1.778883in}{3.249336in}}{\pgfqpoint{1.784707in}{3.255160in}}%
\pgfpathcurveto{\pgfqpoint{1.790531in}{3.260984in}}{\pgfqpoint{1.793804in}{3.268884in}}{\pgfqpoint{1.793804in}{3.277120in}}%
\pgfpathcurveto{\pgfqpoint{1.793804in}{3.285356in}}{\pgfqpoint{1.790531in}{3.293256in}}{\pgfqpoint{1.784707in}{3.299080in}}%
\pgfpathcurveto{\pgfqpoint{1.778883in}{3.304904in}}{\pgfqpoint{1.770983in}{3.308177in}}{\pgfqpoint{1.762747in}{3.308177in}}%
\pgfpathcurveto{\pgfqpoint{1.754511in}{3.308177in}}{\pgfqpoint{1.746611in}{3.304904in}}{\pgfqpoint{1.740787in}{3.299080in}}%
\pgfpathcurveto{\pgfqpoint{1.734963in}{3.293256in}}{\pgfqpoint{1.731691in}{3.285356in}}{\pgfqpoint{1.731691in}{3.277120in}}%
\pgfpathcurveto{\pgfqpoint{1.731691in}{3.268884in}}{\pgfqpoint{1.734963in}{3.260984in}}{\pgfqpoint{1.740787in}{3.255160in}}%
\pgfpathcurveto{\pgfqpoint{1.746611in}{3.249336in}}{\pgfqpoint{1.754511in}{3.246064in}}{\pgfqpoint{1.762747in}{3.246064in}}%
\pgfpathclose%
\pgfusepath{stroke,fill}%
\end{pgfscope}%
\begin{pgfscope}%
\pgfpathrectangle{\pgfqpoint{0.100000in}{0.220728in}}{\pgfqpoint{3.696000in}{3.696000in}}%
\pgfusepath{clip}%
\pgfsetbuttcap%
\pgfsetroundjoin%
\definecolor{currentfill}{rgb}{0.121569,0.466667,0.705882}%
\pgfsetfillcolor{currentfill}%
\pgfsetfillopacity{0.302372}%
\pgfsetlinewidth{1.003750pt}%
\definecolor{currentstroke}{rgb}{0.121569,0.466667,0.705882}%
\pgfsetstrokecolor{currentstroke}%
\pgfsetstrokeopacity{0.302372}%
\pgfsetdash{}{0pt}%
\pgfpathmoveto{\pgfqpoint{1.797600in}{3.262603in}}%
\pgfpathcurveto{\pgfqpoint{1.805836in}{3.262603in}}{\pgfqpoint{1.813736in}{3.265875in}}{\pgfqpoint{1.819560in}{3.271699in}}%
\pgfpathcurveto{\pgfqpoint{1.825384in}{3.277523in}}{\pgfqpoint{1.828657in}{3.285423in}}{\pgfqpoint{1.828657in}{3.293659in}}%
\pgfpathcurveto{\pgfqpoint{1.828657in}{3.301896in}}{\pgfqpoint{1.825384in}{3.309796in}}{\pgfqpoint{1.819560in}{3.315620in}}%
\pgfpathcurveto{\pgfqpoint{1.813736in}{3.321443in}}{\pgfqpoint{1.805836in}{3.324716in}}{\pgfqpoint{1.797600in}{3.324716in}}%
\pgfpathcurveto{\pgfqpoint{1.789364in}{3.324716in}}{\pgfqpoint{1.781464in}{3.321443in}}{\pgfqpoint{1.775640in}{3.315620in}}%
\pgfpathcurveto{\pgfqpoint{1.769816in}{3.309796in}}{\pgfqpoint{1.766544in}{3.301896in}}{\pgfqpoint{1.766544in}{3.293659in}}%
\pgfpathcurveto{\pgfqpoint{1.766544in}{3.285423in}}{\pgfqpoint{1.769816in}{3.277523in}}{\pgfqpoint{1.775640in}{3.271699in}}%
\pgfpathcurveto{\pgfqpoint{1.781464in}{3.265875in}}{\pgfqpoint{1.789364in}{3.262603in}}{\pgfqpoint{1.797600in}{3.262603in}}%
\pgfpathclose%
\pgfusepath{stroke,fill}%
\end{pgfscope}%
\begin{pgfscope}%
\pgfpathrectangle{\pgfqpoint{0.100000in}{0.220728in}}{\pgfqpoint{3.696000in}{3.696000in}}%
\pgfusepath{clip}%
\pgfsetbuttcap%
\pgfsetroundjoin%
\definecolor{currentfill}{rgb}{0.121569,0.466667,0.705882}%
\pgfsetfillcolor{currentfill}%
\pgfsetfillopacity{0.302396}%
\pgfsetlinewidth{1.003750pt}%
\definecolor{currentstroke}{rgb}{0.121569,0.466667,0.705882}%
\pgfsetstrokecolor{currentstroke}%
\pgfsetstrokeopacity{0.302396}%
\pgfsetdash{}{0pt}%
\pgfpathmoveto{\pgfqpoint{1.788728in}{3.263623in}}%
\pgfpathcurveto{\pgfqpoint{1.796965in}{3.263623in}}{\pgfqpoint{1.804865in}{3.266895in}}{\pgfqpoint{1.810689in}{3.272719in}}%
\pgfpathcurveto{\pgfqpoint{1.816513in}{3.278543in}}{\pgfqpoint{1.819785in}{3.286443in}}{\pgfqpoint{1.819785in}{3.294679in}}%
\pgfpathcurveto{\pgfqpoint{1.819785in}{3.302915in}}{\pgfqpoint{1.816513in}{3.310815in}}{\pgfqpoint{1.810689in}{3.316639in}}%
\pgfpathcurveto{\pgfqpoint{1.804865in}{3.322463in}}{\pgfqpoint{1.796965in}{3.325736in}}{\pgfqpoint{1.788728in}{3.325736in}}%
\pgfpathcurveto{\pgfqpoint{1.780492in}{3.325736in}}{\pgfqpoint{1.772592in}{3.322463in}}{\pgfqpoint{1.766768in}{3.316639in}}%
\pgfpathcurveto{\pgfqpoint{1.760944in}{3.310815in}}{\pgfqpoint{1.757672in}{3.302915in}}{\pgfqpoint{1.757672in}{3.294679in}}%
\pgfpathcurveto{\pgfqpoint{1.757672in}{3.286443in}}{\pgfqpoint{1.760944in}{3.278543in}}{\pgfqpoint{1.766768in}{3.272719in}}%
\pgfpathcurveto{\pgfqpoint{1.772592in}{3.266895in}}{\pgfqpoint{1.780492in}{3.263623in}}{\pgfqpoint{1.788728in}{3.263623in}}%
\pgfpathclose%
\pgfusepath{stroke,fill}%
\end{pgfscope}%
\begin{pgfscope}%
\pgfpathrectangle{\pgfqpoint{0.100000in}{0.220728in}}{\pgfqpoint{3.696000in}{3.696000in}}%
\pgfusepath{clip}%
\pgfsetbuttcap%
\pgfsetroundjoin%
\definecolor{currentfill}{rgb}{0.121569,0.466667,0.705882}%
\pgfsetfillcolor{currentfill}%
\pgfsetfillopacity{0.302573}%
\pgfsetlinewidth{1.003750pt}%
\definecolor{currentstroke}{rgb}{0.121569,0.466667,0.705882}%
\pgfsetstrokecolor{currentstroke}%
\pgfsetstrokeopacity{0.302573}%
\pgfsetdash{}{0pt}%
\pgfpathmoveto{\pgfqpoint{1.760652in}{3.241776in}}%
\pgfpathcurveto{\pgfqpoint{1.768888in}{3.241776in}}{\pgfqpoint{1.776788in}{3.245048in}}{\pgfqpoint{1.782612in}{3.250872in}}%
\pgfpathcurveto{\pgfqpoint{1.788436in}{3.256696in}}{\pgfqpoint{1.791708in}{3.264596in}}{\pgfqpoint{1.791708in}{3.272832in}}%
\pgfpathcurveto{\pgfqpoint{1.791708in}{3.281068in}}{\pgfqpoint{1.788436in}{3.288969in}}{\pgfqpoint{1.782612in}{3.294792in}}%
\pgfpathcurveto{\pgfqpoint{1.776788in}{3.300616in}}{\pgfqpoint{1.768888in}{3.303889in}}{\pgfqpoint{1.760652in}{3.303889in}}%
\pgfpathcurveto{\pgfqpoint{1.752416in}{3.303889in}}{\pgfqpoint{1.744516in}{3.300616in}}{\pgfqpoint{1.738692in}{3.294792in}}%
\pgfpathcurveto{\pgfqpoint{1.732868in}{3.288969in}}{\pgfqpoint{1.729595in}{3.281068in}}{\pgfqpoint{1.729595in}{3.272832in}}%
\pgfpathcurveto{\pgfqpoint{1.729595in}{3.264596in}}{\pgfqpoint{1.732868in}{3.256696in}}{\pgfqpoint{1.738692in}{3.250872in}}%
\pgfpathcurveto{\pgfqpoint{1.744516in}{3.245048in}}{\pgfqpoint{1.752416in}{3.241776in}}{\pgfqpoint{1.760652in}{3.241776in}}%
\pgfpathclose%
\pgfusepath{stroke,fill}%
\end{pgfscope}%
\begin{pgfscope}%
\pgfpathrectangle{\pgfqpoint{0.100000in}{0.220728in}}{\pgfqpoint{3.696000in}{3.696000in}}%
\pgfusepath{clip}%
\pgfsetbuttcap%
\pgfsetroundjoin%
\definecolor{currentfill}{rgb}{0.121569,0.466667,0.705882}%
\pgfsetfillcolor{currentfill}%
\pgfsetfillopacity{0.303018}%
\pgfsetlinewidth{1.003750pt}%
\definecolor{currentstroke}{rgb}{0.121569,0.466667,0.705882}%
\pgfsetstrokecolor{currentstroke}%
\pgfsetstrokeopacity{0.303018}%
\pgfsetdash{}{0pt}%
\pgfpathmoveto{\pgfqpoint{1.758933in}{3.238590in}}%
\pgfpathcurveto{\pgfqpoint{1.767169in}{3.238590in}}{\pgfqpoint{1.775069in}{3.241862in}}{\pgfqpoint{1.780893in}{3.247686in}}%
\pgfpathcurveto{\pgfqpoint{1.786717in}{3.253510in}}{\pgfqpoint{1.789990in}{3.261410in}}{\pgfqpoint{1.789990in}{3.269646in}}%
\pgfpathcurveto{\pgfqpoint{1.789990in}{3.277883in}}{\pgfqpoint{1.786717in}{3.285783in}}{\pgfqpoint{1.780893in}{3.291607in}}%
\pgfpathcurveto{\pgfqpoint{1.775069in}{3.297431in}}{\pgfqpoint{1.767169in}{3.300703in}}{\pgfqpoint{1.758933in}{3.300703in}}%
\pgfpathcurveto{\pgfqpoint{1.750697in}{3.300703in}}{\pgfqpoint{1.742797in}{3.297431in}}{\pgfqpoint{1.736973in}{3.291607in}}%
\pgfpathcurveto{\pgfqpoint{1.731149in}{3.285783in}}{\pgfqpoint{1.727877in}{3.277883in}}{\pgfqpoint{1.727877in}{3.269646in}}%
\pgfpathcurveto{\pgfqpoint{1.727877in}{3.261410in}}{\pgfqpoint{1.731149in}{3.253510in}}{\pgfqpoint{1.736973in}{3.247686in}}%
\pgfpathcurveto{\pgfqpoint{1.742797in}{3.241862in}}{\pgfqpoint{1.750697in}{3.238590in}}{\pgfqpoint{1.758933in}{3.238590in}}%
\pgfpathclose%
\pgfusepath{stroke,fill}%
\end{pgfscope}%
\begin{pgfscope}%
\pgfpathrectangle{\pgfqpoint{0.100000in}{0.220728in}}{\pgfqpoint{3.696000in}{3.696000in}}%
\pgfusepath{clip}%
\pgfsetbuttcap%
\pgfsetroundjoin%
\definecolor{currentfill}{rgb}{0.121569,0.466667,0.705882}%
\pgfsetfillcolor{currentfill}%
\pgfsetfillopacity{0.303162}%
\pgfsetlinewidth{1.003750pt}%
\definecolor{currentstroke}{rgb}{0.121569,0.466667,0.705882}%
\pgfsetstrokecolor{currentstroke}%
\pgfsetstrokeopacity{0.303162}%
\pgfsetdash{}{0pt}%
\pgfpathmoveto{\pgfqpoint{1.792210in}{3.262587in}}%
\pgfpathcurveto{\pgfqpoint{1.800446in}{3.262587in}}{\pgfqpoint{1.808346in}{3.265859in}}{\pgfqpoint{1.814170in}{3.271683in}}%
\pgfpathcurveto{\pgfqpoint{1.819994in}{3.277507in}}{\pgfqpoint{1.823266in}{3.285407in}}{\pgfqpoint{1.823266in}{3.293643in}}%
\pgfpathcurveto{\pgfqpoint{1.823266in}{3.301880in}}{\pgfqpoint{1.819994in}{3.309780in}}{\pgfqpoint{1.814170in}{3.315604in}}%
\pgfpathcurveto{\pgfqpoint{1.808346in}{3.321428in}}{\pgfqpoint{1.800446in}{3.324700in}}{\pgfqpoint{1.792210in}{3.324700in}}%
\pgfpathcurveto{\pgfqpoint{1.783973in}{3.324700in}}{\pgfqpoint{1.776073in}{3.321428in}}{\pgfqpoint{1.770249in}{3.315604in}}%
\pgfpathcurveto{\pgfqpoint{1.764426in}{3.309780in}}{\pgfqpoint{1.761153in}{3.301880in}}{\pgfqpoint{1.761153in}{3.293643in}}%
\pgfpathcurveto{\pgfqpoint{1.761153in}{3.285407in}}{\pgfqpoint{1.764426in}{3.277507in}}{\pgfqpoint{1.770249in}{3.271683in}}%
\pgfpathcurveto{\pgfqpoint{1.776073in}{3.265859in}}{\pgfqpoint{1.783973in}{3.262587in}}{\pgfqpoint{1.792210in}{3.262587in}}%
\pgfpathclose%
\pgfusepath{stroke,fill}%
\end{pgfscope}%
\begin{pgfscope}%
\pgfpathrectangle{\pgfqpoint{0.100000in}{0.220728in}}{\pgfqpoint{3.696000in}{3.696000in}}%
\pgfusepath{clip}%
\pgfsetbuttcap%
\pgfsetroundjoin%
\definecolor{currentfill}{rgb}{0.121569,0.466667,0.705882}%
\pgfsetfillcolor{currentfill}%
\pgfsetfillopacity{0.303297}%
\pgfsetlinewidth{1.003750pt}%
\definecolor{currentstroke}{rgb}{0.121569,0.466667,0.705882}%
\pgfsetstrokecolor{currentstroke}%
\pgfsetstrokeopacity{0.303297}%
\pgfsetdash{}{0pt}%
\pgfpathmoveto{\pgfqpoint{1.758650in}{3.236745in}}%
\pgfpathcurveto{\pgfqpoint{1.766886in}{3.236745in}}{\pgfqpoint{1.774787in}{3.240018in}}{\pgfqpoint{1.780610in}{3.245842in}}%
\pgfpathcurveto{\pgfqpoint{1.786434in}{3.251666in}}{\pgfqpoint{1.789707in}{3.259566in}}{\pgfqpoint{1.789707in}{3.267802in}}%
\pgfpathcurveto{\pgfqpoint{1.789707in}{3.276038in}}{\pgfqpoint{1.786434in}{3.283938in}}{\pgfqpoint{1.780610in}{3.289762in}}%
\pgfpathcurveto{\pgfqpoint{1.774787in}{3.295586in}}{\pgfqpoint{1.766886in}{3.298858in}}{\pgfqpoint{1.758650in}{3.298858in}}%
\pgfpathcurveto{\pgfqpoint{1.750414in}{3.298858in}}{\pgfqpoint{1.742514in}{3.295586in}}{\pgfqpoint{1.736690in}{3.289762in}}%
\pgfpathcurveto{\pgfqpoint{1.730866in}{3.283938in}}{\pgfqpoint{1.727594in}{3.276038in}}{\pgfqpoint{1.727594in}{3.267802in}}%
\pgfpathcurveto{\pgfqpoint{1.727594in}{3.259566in}}{\pgfqpoint{1.730866in}{3.251666in}}{\pgfqpoint{1.736690in}{3.245842in}}%
\pgfpathcurveto{\pgfqpoint{1.742514in}{3.240018in}}{\pgfqpoint{1.750414in}{3.236745in}}{\pgfqpoint{1.758650in}{3.236745in}}%
\pgfpathclose%
\pgfusepath{stroke,fill}%
\end{pgfscope}%
\begin{pgfscope}%
\pgfpathrectangle{\pgfqpoint{0.100000in}{0.220728in}}{\pgfqpoint{3.696000in}{3.696000in}}%
\pgfusepath{clip}%
\pgfsetbuttcap%
\pgfsetroundjoin%
\definecolor{currentfill}{rgb}{0.121569,0.466667,0.705882}%
\pgfsetfillcolor{currentfill}%
\pgfsetfillopacity{0.303699}%
\pgfsetlinewidth{1.003750pt}%
\definecolor{currentstroke}{rgb}{0.121569,0.466667,0.705882}%
\pgfsetstrokecolor{currentstroke}%
\pgfsetstrokeopacity{0.303699}%
\pgfsetdash{}{0pt}%
\pgfpathmoveto{\pgfqpoint{1.756899in}{3.233950in}}%
\pgfpathcurveto{\pgfqpoint{1.765135in}{3.233950in}}{\pgfqpoint{1.773035in}{3.237222in}}{\pgfqpoint{1.778859in}{3.243046in}}%
\pgfpathcurveto{\pgfqpoint{1.784683in}{3.248870in}}{\pgfqpoint{1.787955in}{3.256770in}}{\pgfqpoint{1.787955in}{3.265006in}}%
\pgfpathcurveto{\pgfqpoint{1.787955in}{3.273243in}}{\pgfqpoint{1.784683in}{3.281143in}}{\pgfqpoint{1.778859in}{3.286967in}}%
\pgfpathcurveto{\pgfqpoint{1.773035in}{3.292790in}}{\pgfqpoint{1.765135in}{3.296063in}}{\pgfqpoint{1.756899in}{3.296063in}}%
\pgfpathcurveto{\pgfqpoint{1.748662in}{3.296063in}}{\pgfqpoint{1.740762in}{3.292790in}}{\pgfqpoint{1.734938in}{3.286967in}}%
\pgfpathcurveto{\pgfqpoint{1.729115in}{3.281143in}}{\pgfqpoint{1.725842in}{3.273243in}}{\pgfqpoint{1.725842in}{3.265006in}}%
\pgfpathcurveto{\pgfqpoint{1.725842in}{3.256770in}}{\pgfqpoint{1.729115in}{3.248870in}}{\pgfqpoint{1.734938in}{3.243046in}}%
\pgfpathcurveto{\pgfqpoint{1.740762in}{3.237222in}}{\pgfqpoint{1.748662in}{3.233950in}}{\pgfqpoint{1.756899in}{3.233950in}}%
\pgfpathclose%
\pgfusepath{stroke,fill}%
\end{pgfscope}%
\begin{pgfscope}%
\pgfpathrectangle{\pgfqpoint{0.100000in}{0.220728in}}{\pgfqpoint{3.696000in}{3.696000in}}%
\pgfusepath{clip}%
\pgfsetbuttcap%
\pgfsetroundjoin%
\definecolor{currentfill}{rgb}{0.121569,0.466667,0.705882}%
\pgfsetfillcolor{currentfill}%
\pgfsetfillopacity{0.304040}%
\pgfsetlinewidth{1.003750pt}%
\definecolor{currentstroke}{rgb}{0.121569,0.466667,0.705882}%
\pgfsetstrokecolor{currentstroke}%
\pgfsetstrokeopacity{0.304040}%
\pgfsetdash{}{0pt}%
\pgfpathmoveto{\pgfqpoint{1.756041in}{3.232044in}}%
\pgfpathcurveto{\pgfqpoint{1.764277in}{3.232044in}}{\pgfqpoint{1.772177in}{3.235316in}}{\pgfqpoint{1.778001in}{3.241140in}}%
\pgfpathcurveto{\pgfqpoint{1.783825in}{3.246964in}}{\pgfqpoint{1.787097in}{3.254864in}}{\pgfqpoint{1.787097in}{3.263100in}}%
\pgfpathcurveto{\pgfqpoint{1.787097in}{3.271336in}}{\pgfqpoint{1.783825in}{3.279236in}}{\pgfqpoint{1.778001in}{3.285060in}}%
\pgfpathcurveto{\pgfqpoint{1.772177in}{3.290884in}}{\pgfqpoint{1.764277in}{3.294157in}}{\pgfqpoint{1.756041in}{3.294157in}}%
\pgfpathcurveto{\pgfqpoint{1.747804in}{3.294157in}}{\pgfqpoint{1.739904in}{3.290884in}}{\pgfqpoint{1.734080in}{3.285060in}}%
\pgfpathcurveto{\pgfqpoint{1.728256in}{3.279236in}}{\pgfqpoint{1.724984in}{3.271336in}}{\pgfqpoint{1.724984in}{3.263100in}}%
\pgfpathcurveto{\pgfqpoint{1.724984in}{3.254864in}}{\pgfqpoint{1.728256in}{3.246964in}}{\pgfqpoint{1.734080in}{3.241140in}}%
\pgfpathcurveto{\pgfqpoint{1.739904in}{3.235316in}}{\pgfqpoint{1.747804in}{3.232044in}}{\pgfqpoint{1.756041in}{3.232044in}}%
\pgfpathclose%
\pgfusepath{stroke,fill}%
\end{pgfscope}%
\begin{pgfscope}%
\pgfpathrectangle{\pgfqpoint{0.100000in}{0.220728in}}{\pgfqpoint{3.696000in}{3.696000in}}%
\pgfusepath{clip}%
\pgfsetbuttcap%
\pgfsetroundjoin%
\definecolor{currentfill}{rgb}{0.121569,0.466667,0.705882}%
\pgfsetfillcolor{currentfill}%
\pgfsetfillopacity{0.304289}%
\pgfsetlinewidth{1.003750pt}%
\definecolor{currentstroke}{rgb}{0.121569,0.466667,0.705882}%
\pgfsetstrokecolor{currentstroke}%
\pgfsetstrokeopacity{0.304289}%
\pgfsetdash{}{0pt}%
\pgfpathmoveto{\pgfqpoint{1.802354in}{3.263482in}}%
\pgfpathcurveto{\pgfqpoint{1.810590in}{3.263482in}}{\pgfqpoint{1.818490in}{3.266754in}}{\pgfqpoint{1.824314in}{3.272578in}}%
\pgfpathcurveto{\pgfqpoint{1.830138in}{3.278402in}}{\pgfqpoint{1.833410in}{3.286302in}}{\pgfqpoint{1.833410in}{3.294538in}}%
\pgfpathcurveto{\pgfqpoint{1.833410in}{3.302775in}}{\pgfqpoint{1.830138in}{3.310675in}}{\pgfqpoint{1.824314in}{3.316499in}}%
\pgfpathcurveto{\pgfqpoint{1.818490in}{3.322323in}}{\pgfqpoint{1.810590in}{3.325595in}}{\pgfqpoint{1.802354in}{3.325595in}}%
\pgfpathcurveto{\pgfqpoint{1.794117in}{3.325595in}}{\pgfqpoint{1.786217in}{3.322323in}}{\pgfqpoint{1.780393in}{3.316499in}}%
\pgfpathcurveto{\pgfqpoint{1.774569in}{3.310675in}}{\pgfqpoint{1.771297in}{3.302775in}}{\pgfqpoint{1.771297in}{3.294538in}}%
\pgfpathcurveto{\pgfqpoint{1.771297in}{3.286302in}}{\pgfqpoint{1.774569in}{3.278402in}}{\pgfqpoint{1.780393in}{3.272578in}}%
\pgfpathcurveto{\pgfqpoint{1.786217in}{3.266754in}}{\pgfqpoint{1.794117in}{3.263482in}}{\pgfqpoint{1.802354in}{3.263482in}}%
\pgfpathclose%
\pgfusepath{stroke,fill}%
\end{pgfscope}%
\begin{pgfscope}%
\pgfpathrectangle{\pgfqpoint{0.100000in}{0.220728in}}{\pgfqpoint{3.696000in}{3.696000in}}%
\pgfusepath{clip}%
\pgfsetbuttcap%
\pgfsetroundjoin%
\definecolor{currentfill}{rgb}{0.121569,0.466667,0.705882}%
\pgfsetfillcolor{currentfill}%
\pgfsetfillopacity{0.304707}%
\pgfsetlinewidth{1.003750pt}%
\definecolor{currentstroke}{rgb}{0.121569,0.466667,0.705882}%
\pgfsetstrokecolor{currentstroke}%
\pgfsetstrokeopacity{0.304707}%
\pgfsetdash{}{0pt}%
\pgfpathmoveto{\pgfqpoint{1.755326in}{3.228275in}}%
\pgfpathcurveto{\pgfqpoint{1.763563in}{3.228275in}}{\pgfqpoint{1.771463in}{3.231547in}}{\pgfqpoint{1.777287in}{3.237371in}}%
\pgfpathcurveto{\pgfqpoint{1.783110in}{3.243195in}}{\pgfqpoint{1.786383in}{3.251095in}}{\pgfqpoint{1.786383in}{3.259331in}}%
\pgfpathcurveto{\pgfqpoint{1.786383in}{3.267568in}}{\pgfqpoint{1.783110in}{3.275468in}}{\pgfqpoint{1.777287in}{3.281292in}}%
\pgfpathcurveto{\pgfqpoint{1.771463in}{3.287116in}}{\pgfqpoint{1.763563in}{3.290388in}}{\pgfqpoint{1.755326in}{3.290388in}}%
\pgfpathcurveto{\pgfqpoint{1.747090in}{3.290388in}}{\pgfqpoint{1.739190in}{3.287116in}}{\pgfqpoint{1.733366in}{3.281292in}}%
\pgfpathcurveto{\pgfqpoint{1.727542in}{3.275468in}}{\pgfqpoint{1.724270in}{3.267568in}}{\pgfqpoint{1.724270in}{3.259331in}}%
\pgfpathcurveto{\pgfqpoint{1.724270in}{3.251095in}}{\pgfqpoint{1.727542in}{3.243195in}}{\pgfqpoint{1.733366in}{3.237371in}}%
\pgfpathcurveto{\pgfqpoint{1.739190in}{3.231547in}}{\pgfqpoint{1.747090in}{3.228275in}}{\pgfqpoint{1.755326in}{3.228275in}}%
\pgfpathclose%
\pgfusepath{stroke,fill}%
\end{pgfscope}%
\begin{pgfscope}%
\pgfpathrectangle{\pgfqpoint{0.100000in}{0.220728in}}{\pgfqpoint{3.696000in}{3.696000in}}%
\pgfusepath{clip}%
\pgfsetbuttcap%
\pgfsetroundjoin%
\definecolor{currentfill}{rgb}{0.121569,0.466667,0.705882}%
\pgfsetfillcolor{currentfill}%
\pgfsetfillopacity{0.304875}%
\pgfsetlinewidth{1.003750pt}%
\definecolor{currentstroke}{rgb}{0.121569,0.466667,0.705882}%
\pgfsetstrokecolor{currentstroke}%
\pgfsetstrokeopacity{0.304875}%
\pgfsetdash{}{0pt}%
\pgfpathmoveto{\pgfqpoint{1.754496in}{3.227094in}}%
\pgfpathcurveto{\pgfqpoint{1.762732in}{3.227094in}}{\pgfqpoint{1.770632in}{3.230366in}}{\pgfqpoint{1.776456in}{3.236190in}}%
\pgfpathcurveto{\pgfqpoint{1.782280in}{3.242014in}}{\pgfqpoint{1.785552in}{3.249914in}}{\pgfqpoint{1.785552in}{3.258151in}}%
\pgfpathcurveto{\pgfqpoint{1.785552in}{3.266387in}}{\pgfqpoint{1.782280in}{3.274287in}}{\pgfqpoint{1.776456in}{3.280111in}}%
\pgfpathcurveto{\pgfqpoint{1.770632in}{3.285935in}}{\pgfqpoint{1.762732in}{3.289207in}}{\pgfqpoint{1.754496in}{3.289207in}}%
\pgfpathcurveto{\pgfqpoint{1.746259in}{3.289207in}}{\pgfqpoint{1.738359in}{3.285935in}}{\pgfqpoint{1.732535in}{3.280111in}}%
\pgfpathcurveto{\pgfqpoint{1.726711in}{3.274287in}}{\pgfqpoint{1.723439in}{3.266387in}}{\pgfqpoint{1.723439in}{3.258151in}}%
\pgfpathcurveto{\pgfqpoint{1.723439in}{3.249914in}}{\pgfqpoint{1.726711in}{3.242014in}}{\pgfqpoint{1.732535in}{3.236190in}}%
\pgfpathcurveto{\pgfqpoint{1.738359in}{3.230366in}}{\pgfqpoint{1.746259in}{3.227094in}}{\pgfqpoint{1.754496in}{3.227094in}}%
\pgfpathclose%
\pgfusepath{stroke,fill}%
\end{pgfscope}%
\begin{pgfscope}%
\pgfpathrectangle{\pgfqpoint{0.100000in}{0.220728in}}{\pgfqpoint{3.696000in}{3.696000in}}%
\pgfusepath{clip}%
\pgfsetbuttcap%
\pgfsetroundjoin%
\definecolor{currentfill}{rgb}{0.121569,0.466667,0.705882}%
\pgfsetfillcolor{currentfill}%
\pgfsetfillopacity{0.305292}%
\pgfsetlinewidth{1.003750pt}%
\definecolor{currentstroke}{rgb}{0.121569,0.466667,0.705882}%
\pgfsetstrokecolor{currentstroke}%
\pgfsetstrokeopacity{0.305292}%
\pgfsetdash{}{0pt}%
\pgfpathmoveto{\pgfqpoint{1.804919in}{3.263602in}}%
\pgfpathcurveto{\pgfqpoint{1.813155in}{3.263602in}}{\pgfqpoint{1.821055in}{3.266874in}}{\pgfqpoint{1.826879in}{3.272698in}}%
\pgfpathcurveto{\pgfqpoint{1.832703in}{3.278522in}}{\pgfqpoint{1.835975in}{3.286422in}}{\pgfqpoint{1.835975in}{3.294658in}}%
\pgfpathcurveto{\pgfqpoint{1.835975in}{3.302895in}}{\pgfqpoint{1.832703in}{3.310795in}}{\pgfqpoint{1.826879in}{3.316619in}}%
\pgfpathcurveto{\pgfqpoint{1.821055in}{3.322442in}}{\pgfqpoint{1.813155in}{3.325715in}}{\pgfqpoint{1.804919in}{3.325715in}}%
\pgfpathcurveto{\pgfqpoint{1.796683in}{3.325715in}}{\pgfqpoint{1.788783in}{3.322442in}}{\pgfqpoint{1.782959in}{3.316619in}}%
\pgfpathcurveto{\pgfqpoint{1.777135in}{3.310795in}}{\pgfqpoint{1.773862in}{3.302895in}}{\pgfqpoint{1.773862in}{3.294658in}}%
\pgfpathcurveto{\pgfqpoint{1.773862in}{3.286422in}}{\pgfqpoint{1.777135in}{3.278522in}}{\pgfqpoint{1.782959in}{3.272698in}}%
\pgfpathcurveto{\pgfqpoint{1.788783in}{3.266874in}}{\pgfqpoint{1.796683in}{3.263602in}}{\pgfqpoint{1.804919in}{3.263602in}}%
\pgfpathclose%
\pgfusepath{stroke,fill}%
\end{pgfscope}%
\begin{pgfscope}%
\pgfpathrectangle{\pgfqpoint{0.100000in}{0.220728in}}{\pgfqpoint{3.696000in}{3.696000in}}%
\pgfusepath{clip}%
\pgfsetbuttcap%
\pgfsetroundjoin%
\definecolor{currentfill}{rgb}{0.121569,0.466667,0.705882}%
\pgfsetfillcolor{currentfill}%
\pgfsetfillopacity{0.305343}%
\pgfsetlinewidth{1.003750pt}%
\definecolor{currentstroke}{rgb}{0.121569,0.466667,0.705882}%
\pgfsetstrokecolor{currentstroke}%
\pgfsetstrokeopacity{0.305343}%
\pgfsetdash{}{0pt}%
\pgfpathmoveto{\pgfqpoint{1.754013in}{3.224672in}}%
\pgfpathcurveto{\pgfqpoint{1.762249in}{3.224672in}}{\pgfqpoint{1.770149in}{3.227944in}}{\pgfqpoint{1.775973in}{3.233768in}}%
\pgfpathcurveto{\pgfqpoint{1.781797in}{3.239592in}}{\pgfqpoint{1.785069in}{3.247492in}}{\pgfqpoint{1.785069in}{3.255728in}}%
\pgfpathcurveto{\pgfqpoint{1.785069in}{3.263965in}}{\pgfqpoint{1.781797in}{3.271865in}}{\pgfqpoint{1.775973in}{3.277689in}}%
\pgfpathcurveto{\pgfqpoint{1.770149in}{3.283513in}}{\pgfqpoint{1.762249in}{3.286785in}}{\pgfqpoint{1.754013in}{3.286785in}}%
\pgfpathcurveto{\pgfqpoint{1.745776in}{3.286785in}}{\pgfqpoint{1.737876in}{3.283513in}}{\pgfqpoint{1.732052in}{3.277689in}}%
\pgfpathcurveto{\pgfqpoint{1.726228in}{3.271865in}}{\pgfqpoint{1.722956in}{3.263965in}}{\pgfqpoint{1.722956in}{3.255728in}}%
\pgfpathcurveto{\pgfqpoint{1.722956in}{3.247492in}}{\pgfqpoint{1.726228in}{3.239592in}}{\pgfqpoint{1.732052in}{3.233768in}}%
\pgfpathcurveto{\pgfqpoint{1.737876in}{3.227944in}}{\pgfqpoint{1.745776in}{3.224672in}}{\pgfqpoint{1.754013in}{3.224672in}}%
\pgfpathclose%
\pgfusepath{stroke,fill}%
\end{pgfscope}%
\begin{pgfscope}%
\pgfpathrectangle{\pgfqpoint{0.100000in}{0.220728in}}{\pgfqpoint{3.696000in}{3.696000in}}%
\pgfusepath{clip}%
\pgfsetbuttcap%
\pgfsetroundjoin%
\definecolor{currentfill}{rgb}{0.121569,0.466667,0.705882}%
\pgfsetfillcolor{currentfill}%
\pgfsetfillopacity{0.305664}%
\pgfsetlinewidth{1.003750pt}%
\definecolor{currentstroke}{rgb}{0.121569,0.466667,0.705882}%
\pgfsetstrokecolor{currentstroke}%
\pgfsetstrokeopacity{0.305664}%
\pgfsetdash{}{0pt}%
\pgfpathmoveto{\pgfqpoint{1.753399in}{3.222806in}}%
\pgfpathcurveto{\pgfqpoint{1.761635in}{3.222806in}}{\pgfqpoint{1.769535in}{3.226078in}}{\pgfqpoint{1.775359in}{3.231902in}}%
\pgfpathcurveto{\pgfqpoint{1.781183in}{3.237726in}}{\pgfqpoint{1.784455in}{3.245626in}}{\pgfqpoint{1.784455in}{3.253862in}}%
\pgfpathcurveto{\pgfqpoint{1.784455in}{3.262099in}}{\pgfqpoint{1.781183in}{3.269999in}}{\pgfqpoint{1.775359in}{3.275823in}}%
\pgfpathcurveto{\pgfqpoint{1.769535in}{3.281646in}}{\pgfqpoint{1.761635in}{3.284919in}}{\pgfqpoint{1.753399in}{3.284919in}}%
\pgfpathcurveto{\pgfqpoint{1.745163in}{3.284919in}}{\pgfqpoint{1.737263in}{3.281646in}}{\pgfqpoint{1.731439in}{3.275823in}}%
\pgfpathcurveto{\pgfqpoint{1.725615in}{3.269999in}}{\pgfqpoint{1.722342in}{3.262099in}}{\pgfqpoint{1.722342in}{3.253862in}}%
\pgfpathcurveto{\pgfqpoint{1.722342in}{3.245626in}}{\pgfqpoint{1.725615in}{3.237726in}}{\pgfqpoint{1.731439in}{3.231902in}}%
\pgfpathcurveto{\pgfqpoint{1.737263in}{3.226078in}}{\pgfqpoint{1.745163in}{3.222806in}}{\pgfqpoint{1.753399in}{3.222806in}}%
\pgfpathclose%
\pgfusepath{stroke,fill}%
\end{pgfscope}%
\begin{pgfscope}%
\pgfpathrectangle{\pgfqpoint{0.100000in}{0.220728in}}{\pgfqpoint{3.696000in}{3.696000in}}%
\pgfusepath{clip}%
\pgfsetbuttcap%
\pgfsetroundjoin%
\definecolor{currentfill}{rgb}{0.121569,0.466667,0.705882}%
\pgfsetfillcolor{currentfill}%
\pgfsetfillopacity{0.305720}%
\pgfsetlinewidth{1.003750pt}%
\definecolor{currentstroke}{rgb}{0.121569,0.466667,0.705882}%
\pgfsetstrokecolor{currentstroke}%
\pgfsetstrokeopacity{0.305720}%
\pgfsetdash{}{0pt}%
\pgfpathmoveto{\pgfqpoint{1.806495in}{3.263591in}}%
\pgfpathcurveto{\pgfqpoint{1.814732in}{3.263591in}}{\pgfqpoint{1.822632in}{3.266863in}}{\pgfqpoint{1.828456in}{3.272687in}}%
\pgfpathcurveto{\pgfqpoint{1.834279in}{3.278511in}}{\pgfqpoint{1.837552in}{3.286411in}}{\pgfqpoint{1.837552in}{3.294648in}}%
\pgfpathcurveto{\pgfqpoint{1.837552in}{3.302884in}}{\pgfqpoint{1.834279in}{3.310784in}}{\pgfqpoint{1.828456in}{3.316608in}}%
\pgfpathcurveto{\pgfqpoint{1.822632in}{3.322432in}}{\pgfqpoint{1.814732in}{3.325704in}}{\pgfqpoint{1.806495in}{3.325704in}}%
\pgfpathcurveto{\pgfqpoint{1.798259in}{3.325704in}}{\pgfqpoint{1.790359in}{3.322432in}}{\pgfqpoint{1.784535in}{3.316608in}}%
\pgfpathcurveto{\pgfqpoint{1.778711in}{3.310784in}}{\pgfqpoint{1.775439in}{3.302884in}}{\pgfqpoint{1.775439in}{3.294648in}}%
\pgfpathcurveto{\pgfqpoint{1.775439in}{3.286411in}}{\pgfqpoint{1.778711in}{3.278511in}}{\pgfqpoint{1.784535in}{3.272687in}}%
\pgfpathcurveto{\pgfqpoint{1.790359in}{3.266863in}}{\pgfqpoint{1.798259in}{3.263591in}}{\pgfqpoint{1.806495in}{3.263591in}}%
\pgfpathclose%
\pgfusepath{stroke,fill}%
\end{pgfscope}%
\begin{pgfscope}%
\pgfpathrectangle{\pgfqpoint{0.100000in}{0.220728in}}{\pgfqpoint{3.696000in}{3.696000in}}%
\pgfusepath{clip}%
\pgfsetbuttcap%
\pgfsetroundjoin%
\definecolor{currentfill}{rgb}{0.121569,0.466667,0.705882}%
\pgfsetfillcolor{currentfill}%
\pgfsetfillopacity{0.306146}%
\pgfsetlinewidth{1.003750pt}%
\definecolor{currentstroke}{rgb}{0.121569,0.466667,0.705882}%
\pgfsetstrokecolor{currentstroke}%
\pgfsetstrokeopacity{0.306146}%
\pgfsetdash{}{0pt}%
\pgfpathmoveto{\pgfqpoint{1.751617in}{3.219618in}}%
\pgfpathcurveto{\pgfqpoint{1.759853in}{3.219618in}}{\pgfqpoint{1.767753in}{3.222891in}}{\pgfqpoint{1.773577in}{3.228715in}}%
\pgfpathcurveto{\pgfqpoint{1.779401in}{3.234539in}}{\pgfqpoint{1.782674in}{3.242439in}}{\pgfqpoint{1.782674in}{3.250675in}}%
\pgfpathcurveto{\pgfqpoint{1.782674in}{3.258911in}}{\pgfqpoint{1.779401in}{3.266811in}}{\pgfqpoint{1.773577in}{3.272635in}}%
\pgfpathcurveto{\pgfqpoint{1.767753in}{3.278459in}}{\pgfqpoint{1.759853in}{3.281731in}}{\pgfqpoint{1.751617in}{3.281731in}}%
\pgfpathcurveto{\pgfqpoint{1.743381in}{3.281731in}}{\pgfqpoint{1.735481in}{3.278459in}}{\pgfqpoint{1.729657in}{3.272635in}}%
\pgfpathcurveto{\pgfqpoint{1.723833in}{3.266811in}}{\pgfqpoint{1.720561in}{3.258911in}}{\pgfqpoint{1.720561in}{3.250675in}}%
\pgfpathcurveto{\pgfqpoint{1.720561in}{3.242439in}}{\pgfqpoint{1.723833in}{3.234539in}}{\pgfqpoint{1.729657in}{3.228715in}}%
\pgfpathcurveto{\pgfqpoint{1.735481in}{3.222891in}}{\pgfqpoint{1.743381in}{3.219618in}}{\pgfqpoint{1.751617in}{3.219618in}}%
\pgfpathclose%
\pgfusepath{stroke,fill}%
\end{pgfscope}%
\begin{pgfscope}%
\pgfpathrectangle{\pgfqpoint{0.100000in}{0.220728in}}{\pgfqpoint{3.696000in}{3.696000in}}%
\pgfusepath{clip}%
\pgfsetbuttcap%
\pgfsetroundjoin%
\definecolor{currentfill}{rgb}{0.121569,0.466667,0.705882}%
\pgfsetfillcolor{currentfill}%
\pgfsetfillopacity{0.306402}%
\pgfsetlinewidth{1.003750pt}%
\definecolor{currentstroke}{rgb}{0.121569,0.466667,0.705882}%
\pgfsetstrokecolor{currentstroke}%
\pgfsetstrokeopacity{0.306402}%
\pgfsetdash{}{0pt}%
\pgfpathmoveto{\pgfqpoint{1.751333in}{3.218129in}}%
\pgfpathcurveto{\pgfqpoint{1.759570in}{3.218129in}}{\pgfqpoint{1.767470in}{3.221402in}}{\pgfqpoint{1.773294in}{3.227226in}}%
\pgfpathcurveto{\pgfqpoint{1.779118in}{3.233049in}}{\pgfqpoint{1.782390in}{3.240950in}}{\pgfqpoint{1.782390in}{3.249186in}}%
\pgfpathcurveto{\pgfqpoint{1.782390in}{3.257422in}}{\pgfqpoint{1.779118in}{3.265322in}}{\pgfqpoint{1.773294in}{3.271146in}}%
\pgfpathcurveto{\pgfqpoint{1.767470in}{3.276970in}}{\pgfqpoint{1.759570in}{3.280242in}}{\pgfqpoint{1.751333in}{3.280242in}}%
\pgfpathcurveto{\pgfqpoint{1.743097in}{3.280242in}}{\pgfqpoint{1.735197in}{3.276970in}}{\pgfqpoint{1.729373in}{3.271146in}}%
\pgfpathcurveto{\pgfqpoint{1.723549in}{3.265322in}}{\pgfqpoint{1.720277in}{3.257422in}}{\pgfqpoint{1.720277in}{3.249186in}}%
\pgfpathcurveto{\pgfqpoint{1.720277in}{3.240950in}}{\pgfqpoint{1.723549in}{3.233049in}}{\pgfqpoint{1.729373in}{3.227226in}}%
\pgfpathcurveto{\pgfqpoint{1.735197in}{3.221402in}}{\pgfqpoint{1.743097in}{3.218129in}}{\pgfqpoint{1.751333in}{3.218129in}}%
\pgfpathclose%
\pgfusepath{stroke,fill}%
\end{pgfscope}%
\begin{pgfscope}%
\pgfpathrectangle{\pgfqpoint{0.100000in}{0.220728in}}{\pgfqpoint{3.696000in}{3.696000in}}%
\pgfusepath{clip}%
\pgfsetbuttcap%
\pgfsetroundjoin%
\definecolor{currentfill}{rgb}{0.121569,0.466667,0.705882}%
\pgfsetfillcolor{currentfill}%
\pgfsetfillopacity{0.306503}%
\pgfsetlinewidth{1.003750pt}%
\definecolor{currentstroke}{rgb}{0.121569,0.466667,0.705882}%
\pgfsetstrokecolor{currentstroke}%
\pgfsetstrokeopacity{0.306503}%
\pgfsetdash{}{0pt}%
\pgfpathmoveto{\pgfqpoint{1.808201in}{3.263653in}}%
\pgfpathcurveto{\pgfqpoint{1.816437in}{3.263653in}}{\pgfqpoint{1.824337in}{3.266925in}}{\pgfqpoint{1.830161in}{3.272749in}}%
\pgfpathcurveto{\pgfqpoint{1.835985in}{3.278573in}}{\pgfqpoint{1.839257in}{3.286473in}}{\pgfqpoint{1.839257in}{3.294709in}}%
\pgfpathcurveto{\pgfqpoint{1.839257in}{3.302945in}}{\pgfqpoint{1.835985in}{3.310845in}}{\pgfqpoint{1.830161in}{3.316669in}}%
\pgfpathcurveto{\pgfqpoint{1.824337in}{3.322493in}}{\pgfqpoint{1.816437in}{3.325766in}}{\pgfqpoint{1.808201in}{3.325766in}}%
\pgfpathcurveto{\pgfqpoint{1.799965in}{3.325766in}}{\pgfqpoint{1.792065in}{3.322493in}}{\pgfqpoint{1.786241in}{3.316669in}}%
\pgfpathcurveto{\pgfqpoint{1.780417in}{3.310845in}}{\pgfqpoint{1.777144in}{3.302945in}}{\pgfqpoint{1.777144in}{3.294709in}}%
\pgfpathcurveto{\pgfqpoint{1.777144in}{3.286473in}}{\pgfqpoint{1.780417in}{3.278573in}}{\pgfqpoint{1.786241in}{3.272749in}}%
\pgfpathcurveto{\pgfqpoint{1.792065in}{3.266925in}}{\pgfqpoint{1.799965in}{3.263653in}}{\pgfqpoint{1.808201in}{3.263653in}}%
\pgfpathclose%
\pgfusepath{stroke,fill}%
\end{pgfscope}%
\begin{pgfscope}%
\pgfpathrectangle{\pgfqpoint{0.100000in}{0.220728in}}{\pgfqpoint{3.696000in}{3.696000in}}%
\pgfusepath{clip}%
\pgfsetbuttcap%
\pgfsetroundjoin%
\definecolor{currentfill}{rgb}{0.121569,0.466667,0.705882}%
\pgfsetfillcolor{currentfill}%
\pgfsetfillopacity{0.306788}%
\pgfsetlinewidth{1.003750pt}%
\definecolor{currentstroke}{rgb}{0.121569,0.466667,0.705882}%
\pgfsetstrokecolor{currentstroke}%
\pgfsetstrokeopacity{0.306788}%
\pgfsetdash{}{0pt}%
\pgfpathmoveto{\pgfqpoint{1.809343in}{3.263590in}}%
\pgfpathcurveto{\pgfqpoint{1.817579in}{3.263590in}}{\pgfqpoint{1.825479in}{3.266862in}}{\pgfqpoint{1.831303in}{3.272686in}}%
\pgfpathcurveto{\pgfqpoint{1.837127in}{3.278510in}}{\pgfqpoint{1.840399in}{3.286410in}}{\pgfqpoint{1.840399in}{3.294647in}}%
\pgfpathcurveto{\pgfqpoint{1.840399in}{3.302883in}}{\pgfqpoint{1.837127in}{3.310783in}}{\pgfqpoint{1.831303in}{3.316607in}}%
\pgfpathcurveto{\pgfqpoint{1.825479in}{3.322431in}}{\pgfqpoint{1.817579in}{3.325703in}}{\pgfqpoint{1.809343in}{3.325703in}}%
\pgfpathcurveto{\pgfqpoint{1.801106in}{3.325703in}}{\pgfqpoint{1.793206in}{3.322431in}}{\pgfqpoint{1.787382in}{3.316607in}}%
\pgfpathcurveto{\pgfqpoint{1.781558in}{3.310783in}}{\pgfqpoint{1.778286in}{3.302883in}}{\pgfqpoint{1.778286in}{3.294647in}}%
\pgfpathcurveto{\pgfqpoint{1.778286in}{3.286410in}}{\pgfqpoint{1.781558in}{3.278510in}}{\pgfqpoint{1.787382in}{3.272686in}}%
\pgfpathcurveto{\pgfqpoint{1.793206in}{3.266862in}}{\pgfqpoint{1.801106in}{3.263590in}}{\pgfqpoint{1.809343in}{3.263590in}}%
\pgfpathclose%
\pgfusepath{stroke,fill}%
\end{pgfscope}%
\begin{pgfscope}%
\pgfpathrectangle{\pgfqpoint{0.100000in}{0.220728in}}{\pgfqpoint{3.696000in}{3.696000in}}%
\pgfusepath{clip}%
\pgfsetbuttcap%
\pgfsetroundjoin%
\definecolor{currentfill}{rgb}{0.121569,0.466667,0.705882}%
\pgfsetfillcolor{currentfill}%
\pgfsetfillopacity{0.306795}%
\pgfsetlinewidth{1.003750pt}%
\definecolor{currentstroke}{rgb}{0.121569,0.466667,0.705882}%
\pgfsetstrokecolor{currentstroke}%
\pgfsetstrokeopacity{0.306795}%
\pgfsetdash{}{0pt}%
\pgfpathmoveto{\pgfqpoint{1.749924in}{3.215824in}}%
\pgfpathcurveto{\pgfqpoint{1.758160in}{3.215824in}}{\pgfqpoint{1.766060in}{3.219096in}}{\pgfqpoint{1.771884in}{3.224920in}}%
\pgfpathcurveto{\pgfqpoint{1.777708in}{3.230744in}}{\pgfqpoint{1.780981in}{3.238644in}}{\pgfqpoint{1.780981in}{3.246880in}}%
\pgfpathcurveto{\pgfqpoint{1.780981in}{3.255117in}}{\pgfqpoint{1.777708in}{3.263017in}}{\pgfqpoint{1.771884in}{3.268841in}}%
\pgfpathcurveto{\pgfqpoint{1.766060in}{3.274665in}}{\pgfqpoint{1.758160in}{3.277937in}}{\pgfqpoint{1.749924in}{3.277937in}}%
\pgfpathcurveto{\pgfqpoint{1.741688in}{3.277937in}}{\pgfqpoint{1.733788in}{3.274665in}}{\pgfqpoint{1.727964in}{3.268841in}}%
\pgfpathcurveto{\pgfqpoint{1.722140in}{3.263017in}}{\pgfqpoint{1.718868in}{3.255117in}}{\pgfqpoint{1.718868in}{3.246880in}}%
\pgfpathcurveto{\pgfqpoint{1.718868in}{3.238644in}}{\pgfqpoint{1.722140in}{3.230744in}}{\pgfqpoint{1.727964in}{3.224920in}}%
\pgfpathcurveto{\pgfqpoint{1.733788in}{3.219096in}}{\pgfqpoint{1.741688in}{3.215824in}}{\pgfqpoint{1.749924in}{3.215824in}}%
\pgfpathclose%
\pgfusepath{stroke,fill}%
\end{pgfscope}%
\begin{pgfscope}%
\pgfpathrectangle{\pgfqpoint{0.100000in}{0.220728in}}{\pgfqpoint{3.696000in}{3.696000in}}%
\pgfusepath{clip}%
\pgfsetbuttcap%
\pgfsetroundjoin%
\definecolor{currentfill}{rgb}{0.121569,0.466667,0.705882}%
\pgfsetfillcolor{currentfill}%
\pgfsetfillopacity{0.306981}%
\pgfsetlinewidth{1.003750pt}%
\definecolor{currentstroke}{rgb}{0.121569,0.466667,0.705882}%
\pgfsetstrokecolor{currentstroke}%
\pgfsetstrokeopacity{0.306981}%
\pgfsetdash{}{0pt}%
\pgfpathmoveto{\pgfqpoint{1.809913in}{3.263546in}}%
\pgfpathcurveto{\pgfqpoint{1.818150in}{3.263546in}}{\pgfqpoint{1.826050in}{3.266819in}}{\pgfqpoint{1.831873in}{3.272643in}}%
\pgfpathcurveto{\pgfqpoint{1.837697in}{3.278466in}}{\pgfqpoint{1.840970in}{3.286367in}}{\pgfqpoint{1.840970in}{3.294603in}}%
\pgfpathcurveto{\pgfqpoint{1.840970in}{3.302839in}}{\pgfqpoint{1.837697in}{3.310739in}}{\pgfqpoint{1.831873in}{3.316563in}}%
\pgfpathcurveto{\pgfqpoint{1.826050in}{3.322387in}}{\pgfqpoint{1.818150in}{3.325659in}}{\pgfqpoint{1.809913in}{3.325659in}}%
\pgfpathcurveto{\pgfqpoint{1.801677in}{3.325659in}}{\pgfqpoint{1.793777in}{3.322387in}}{\pgfqpoint{1.787953in}{3.316563in}}%
\pgfpathcurveto{\pgfqpoint{1.782129in}{3.310739in}}{\pgfqpoint{1.778857in}{3.302839in}}{\pgfqpoint{1.778857in}{3.294603in}}%
\pgfpathcurveto{\pgfqpoint{1.778857in}{3.286367in}}{\pgfqpoint{1.782129in}{3.278466in}}{\pgfqpoint{1.787953in}{3.272643in}}%
\pgfpathcurveto{\pgfqpoint{1.793777in}{3.266819in}}{\pgfqpoint{1.801677in}{3.263546in}}{\pgfqpoint{1.809913in}{3.263546in}}%
\pgfpathclose%
\pgfusepath{stroke,fill}%
\end{pgfscope}%
\begin{pgfscope}%
\pgfpathrectangle{\pgfqpoint{0.100000in}{0.220728in}}{\pgfqpoint{3.696000in}{3.696000in}}%
\pgfusepath{clip}%
\pgfsetbuttcap%
\pgfsetroundjoin%
\definecolor{currentfill}{rgb}{0.121569,0.466667,0.705882}%
\pgfsetfillcolor{currentfill}%
\pgfsetfillopacity{0.307500}%
\pgfsetlinewidth{1.003750pt}%
\definecolor{currentstroke}{rgb}{0.121569,0.466667,0.705882}%
\pgfsetstrokecolor{currentstroke}%
\pgfsetstrokeopacity{0.307500}%
\pgfsetdash{}{0pt}%
\pgfpathmoveto{\pgfqpoint{1.747736in}{3.211115in}}%
\pgfpathcurveto{\pgfqpoint{1.755973in}{3.211115in}}{\pgfqpoint{1.763873in}{3.214387in}}{\pgfqpoint{1.769697in}{3.220211in}}%
\pgfpathcurveto{\pgfqpoint{1.775521in}{3.226035in}}{\pgfqpoint{1.778793in}{3.233935in}}{\pgfqpoint{1.778793in}{3.242172in}}%
\pgfpathcurveto{\pgfqpoint{1.778793in}{3.250408in}}{\pgfqpoint{1.775521in}{3.258308in}}{\pgfqpoint{1.769697in}{3.264132in}}%
\pgfpathcurveto{\pgfqpoint{1.763873in}{3.269956in}}{\pgfqpoint{1.755973in}{3.273228in}}{\pgfqpoint{1.747736in}{3.273228in}}%
\pgfpathcurveto{\pgfqpoint{1.739500in}{3.273228in}}{\pgfqpoint{1.731600in}{3.269956in}}{\pgfqpoint{1.725776in}{3.264132in}}%
\pgfpathcurveto{\pgfqpoint{1.719952in}{3.258308in}}{\pgfqpoint{1.716680in}{3.250408in}}{\pgfqpoint{1.716680in}{3.242172in}}%
\pgfpathcurveto{\pgfqpoint{1.716680in}{3.233935in}}{\pgfqpoint{1.719952in}{3.226035in}}{\pgfqpoint{1.725776in}{3.220211in}}%
\pgfpathcurveto{\pgfqpoint{1.731600in}{3.214387in}}{\pgfqpoint{1.739500in}{3.211115in}}{\pgfqpoint{1.747736in}{3.211115in}}%
\pgfpathclose%
\pgfusepath{stroke,fill}%
\end{pgfscope}%
\begin{pgfscope}%
\pgfpathrectangle{\pgfqpoint{0.100000in}{0.220728in}}{\pgfqpoint{3.696000in}{3.696000in}}%
\pgfusepath{clip}%
\pgfsetbuttcap%
\pgfsetroundjoin%
\definecolor{currentfill}{rgb}{0.121569,0.466667,0.705882}%
\pgfsetfillcolor{currentfill}%
\pgfsetfillopacity{0.307561}%
\pgfsetlinewidth{1.003750pt}%
\definecolor{currentstroke}{rgb}{0.121569,0.466667,0.705882}%
\pgfsetstrokecolor{currentstroke}%
\pgfsetstrokeopacity{0.307561}%
\pgfsetdash{}{0pt}%
\pgfpathmoveto{\pgfqpoint{1.811436in}{3.263574in}}%
\pgfpathcurveto{\pgfqpoint{1.819672in}{3.263574in}}{\pgfqpoint{1.827573in}{3.266846in}}{\pgfqpoint{1.833396in}{3.272670in}}%
\pgfpathcurveto{\pgfqpoint{1.839220in}{3.278494in}}{\pgfqpoint{1.842493in}{3.286394in}}{\pgfqpoint{1.842493in}{3.294630in}}%
\pgfpathcurveto{\pgfqpoint{1.842493in}{3.302866in}}{\pgfqpoint{1.839220in}{3.310767in}}{\pgfqpoint{1.833396in}{3.316590in}}%
\pgfpathcurveto{\pgfqpoint{1.827573in}{3.322414in}}{\pgfqpoint{1.819672in}{3.325687in}}{\pgfqpoint{1.811436in}{3.325687in}}%
\pgfpathcurveto{\pgfqpoint{1.803200in}{3.325687in}}{\pgfqpoint{1.795300in}{3.322414in}}{\pgfqpoint{1.789476in}{3.316590in}}%
\pgfpathcurveto{\pgfqpoint{1.783652in}{3.310767in}}{\pgfqpoint{1.780380in}{3.302866in}}{\pgfqpoint{1.780380in}{3.294630in}}%
\pgfpathcurveto{\pgfqpoint{1.780380in}{3.286394in}}{\pgfqpoint{1.783652in}{3.278494in}}{\pgfqpoint{1.789476in}{3.272670in}}%
\pgfpathcurveto{\pgfqpoint{1.795300in}{3.266846in}}{\pgfqpoint{1.803200in}{3.263574in}}{\pgfqpoint{1.811436in}{3.263574in}}%
\pgfpathclose%
\pgfusepath{stroke,fill}%
\end{pgfscope}%
\begin{pgfscope}%
\pgfpathrectangle{\pgfqpoint{0.100000in}{0.220728in}}{\pgfqpoint{3.696000in}{3.696000in}}%
\pgfusepath{clip}%
\pgfsetbuttcap%
\pgfsetroundjoin%
\definecolor{currentfill}{rgb}{0.121569,0.466667,0.705882}%
\pgfsetfillcolor{currentfill}%
\pgfsetfillopacity{0.307843}%
\pgfsetlinewidth{1.003750pt}%
\definecolor{currentstroke}{rgb}{0.121569,0.466667,0.705882}%
\pgfsetstrokecolor{currentstroke}%
\pgfsetstrokeopacity{0.307843}%
\pgfsetdash{}{0pt}%
\pgfpathmoveto{\pgfqpoint{1.812378in}{3.263734in}}%
\pgfpathcurveto{\pgfqpoint{1.820614in}{3.263734in}}{\pgfqpoint{1.828514in}{3.267006in}}{\pgfqpoint{1.834338in}{3.272830in}}%
\pgfpathcurveto{\pgfqpoint{1.840162in}{3.278654in}}{\pgfqpoint{1.843434in}{3.286554in}}{\pgfqpoint{1.843434in}{3.294790in}}%
\pgfpathcurveto{\pgfqpoint{1.843434in}{3.303026in}}{\pgfqpoint{1.840162in}{3.310927in}}{\pgfqpoint{1.834338in}{3.316750in}}%
\pgfpathcurveto{\pgfqpoint{1.828514in}{3.322574in}}{\pgfqpoint{1.820614in}{3.325847in}}{\pgfqpoint{1.812378in}{3.325847in}}%
\pgfpathcurveto{\pgfqpoint{1.804141in}{3.325847in}}{\pgfqpoint{1.796241in}{3.322574in}}{\pgfqpoint{1.790417in}{3.316750in}}%
\pgfpathcurveto{\pgfqpoint{1.784593in}{3.310927in}}{\pgfqpoint{1.781321in}{3.303026in}}{\pgfqpoint{1.781321in}{3.294790in}}%
\pgfpathcurveto{\pgfqpoint{1.781321in}{3.286554in}}{\pgfqpoint{1.784593in}{3.278654in}}{\pgfqpoint{1.790417in}{3.272830in}}%
\pgfpathcurveto{\pgfqpoint{1.796241in}{3.267006in}}{\pgfqpoint{1.804141in}{3.263734in}}{\pgfqpoint{1.812378in}{3.263734in}}%
\pgfpathclose%
\pgfusepath{stroke,fill}%
\end{pgfscope}%
\begin{pgfscope}%
\pgfpathrectangle{\pgfqpoint{0.100000in}{0.220728in}}{\pgfqpoint{3.696000in}{3.696000in}}%
\pgfusepath{clip}%
\pgfsetbuttcap%
\pgfsetroundjoin%
\definecolor{currentfill}{rgb}{0.121569,0.466667,0.705882}%
\pgfsetfillcolor{currentfill}%
\pgfsetfillopacity{0.308039}%
\pgfsetlinewidth{1.003750pt}%
\definecolor{currentstroke}{rgb}{0.121569,0.466667,0.705882}%
\pgfsetstrokecolor{currentstroke}%
\pgfsetstrokeopacity{0.308039}%
\pgfsetdash{}{0pt}%
\pgfpathmoveto{\pgfqpoint{1.747356in}{3.207541in}}%
\pgfpathcurveto{\pgfqpoint{1.755593in}{3.207541in}}{\pgfqpoint{1.763493in}{3.210813in}}{\pgfqpoint{1.769317in}{3.216637in}}%
\pgfpathcurveto{\pgfqpoint{1.775141in}{3.222461in}}{\pgfqpoint{1.778413in}{3.230361in}}{\pgfqpoint{1.778413in}{3.238597in}}%
\pgfpathcurveto{\pgfqpoint{1.778413in}{3.246834in}}{\pgfqpoint{1.775141in}{3.254734in}}{\pgfqpoint{1.769317in}{3.260557in}}%
\pgfpathcurveto{\pgfqpoint{1.763493in}{3.266381in}}{\pgfqpoint{1.755593in}{3.269654in}}{\pgfqpoint{1.747356in}{3.269654in}}%
\pgfpathcurveto{\pgfqpoint{1.739120in}{3.269654in}}{\pgfqpoint{1.731220in}{3.266381in}}{\pgfqpoint{1.725396in}{3.260557in}}%
\pgfpathcurveto{\pgfqpoint{1.719572in}{3.254734in}}{\pgfqpoint{1.716300in}{3.246834in}}{\pgfqpoint{1.716300in}{3.238597in}}%
\pgfpathcurveto{\pgfqpoint{1.716300in}{3.230361in}}{\pgfqpoint{1.719572in}{3.222461in}}{\pgfqpoint{1.725396in}{3.216637in}}%
\pgfpathcurveto{\pgfqpoint{1.731220in}{3.210813in}}{\pgfqpoint{1.739120in}{3.207541in}}{\pgfqpoint{1.747356in}{3.207541in}}%
\pgfpathclose%
\pgfusepath{stroke,fill}%
\end{pgfscope}%
\begin{pgfscope}%
\pgfpathrectangle{\pgfqpoint{0.100000in}{0.220728in}}{\pgfqpoint{3.696000in}{3.696000in}}%
\pgfusepath{clip}%
\pgfsetbuttcap%
\pgfsetroundjoin%
\definecolor{currentfill}{rgb}{0.121569,0.466667,0.705882}%
\pgfsetfillcolor{currentfill}%
\pgfsetfillopacity{0.308723}%
\pgfsetlinewidth{1.003750pt}%
\definecolor{currentstroke}{rgb}{0.121569,0.466667,0.705882}%
\pgfsetstrokecolor{currentstroke}%
\pgfsetstrokeopacity{0.308723}%
\pgfsetdash{}{0pt}%
\pgfpathmoveto{\pgfqpoint{1.743674in}{3.202316in}}%
\pgfpathcurveto{\pgfqpoint{1.751910in}{3.202316in}}{\pgfqpoint{1.759810in}{3.205588in}}{\pgfqpoint{1.765634in}{3.211412in}}%
\pgfpathcurveto{\pgfqpoint{1.771458in}{3.217236in}}{\pgfqpoint{1.774730in}{3.225136in}}{\pgfqpoint{1.774730in}{3.233372in}}%
\pgfpathcurveto{\pgfqpoint{1.774730in}{3.241609in}}{\pgfqpoint{1.771458in}{3.249509in}}{\pgfqpoint{1.765634in}{3.255333in}}%
\pgfpathcurveto{\pgfqpoint{1.759810in}{3.261157in}}{\pgfqpoint{1.751910in}{3.264429in}}{\pgfqpoint{1.743674in}{3.264429in}}%
\pgfpathcurveto{\pgfqpoint{1.735438in}{3.264429in}}{\pgfqpoint{1.727538in}{3.261157in}}{\pgfqpoint{1.721714in}{3.255333in}}%
\pgfpathcurveto{\pgfqpoint{1.715890in}{3.249509in}}{\pgfqpoint{1.712617in}{3.241609in}}{\pgfqpoint{1.712617in}{3.233372in}}%
\pgfpathcurveto{\pgfqpoint{1.712617in}{3.225136in}}{\pgfqpoint{1.715890in}{3.217236in}}{\pgfqpoint{1.721714in}{3.211412in}}%
\pgfpathcurveto{\pgfqpoint{1.727538in}{3.205588in}}{\pgfqpoint{1.735438in}{3.202316in}}{\pgfqpoint{1.743674in}{3.202316in}}%
\pgfpathclose%
\pgfusepath{stroke,fill}%
\end{pgfscope}%
\begin{pgfscope}%
\pgfpathrectangle{\pgfqpoint{0.100000in}{0.220728in}}{\pgfqpoint{3.696000in}{3.696000in}}%
\pgfusepath{clip}%
\pgfsetbuttcap%
\pgfsetroundjoin%
\definecolor{currentfill}{rgb}{0.121569,0.466667,0.705882}%
\pgfsetfillcolor{currentfill}%
\pgfsetfillopacity{0.308736}%
\pgfsetlinewidth{1.003750pt}%
\definecolor{currentstroke}{rgb}{0.121569,0.466667,0.705882}%
\pgfsetstrokecolor{currentstroke}%
\pgfsetstrokeopacity{0.308736}%
\pgfsetdash{}{0pt}%
\pgfpathmoveto{\pgfqpoint{1.816178in}{3.263044in}}%
\pgfpathcurveto{\pgfqpoint{1.824414in}{3.263044in}}{\pgfqpoint{1.832314in}{3.266317in}}{\pgfqpoint{1.838138in}{3.272141in}}%
\pgfpathcurveto{\pgfqpoint{1.843962in}{3.277965in}}{\pgfqpoint{1.847234in}{3.285865in}}{\pgfqpoint{1.847234in}{3.294101in}}%
\pgfpathcurveto{\pgfqpoint{1.847234in}{3.302337in}}{\pgfqpoint{1.843962in}{3.310237in}}{\pgfqpoint{1.838138in}{3.316061in}}%
\pgfpathcurveto{\pgfqpoint{1.832314in}{3.321885in}}{\pgfqpoint{1.824414in}{3.325157in}}{\pgfqpoint{1.816178in}{3.325157in}}%
\pgfpathcurveto{\pgfqpoint{1.807941in}{3.325157in}}{\pgfqpoint{1.800041in}{3.321885in}}{\pgfqpoint{1.794217in}{3.316061in}}%
\pgfpathcurveto{\pgfqpoint{1.788393in}{3.310237in}}{\pgfqpoint{1.785121in}{3.302337in}}{\pgfqpoint{1.785121in}{3.294101in}}%
\pgfpathcurveto{\pgfqpoint{1.785121in}{3.285865in}}{\pgfqpoint{1.788393in}{3.277965in}}{\pgfqpoint{1.794217in}{3.272141in}}%
\pgfpathcurveto{\pgfqpoint{1.800041in}{3.266317in}}{\pgfqpoint{1.807941in}{3.263044in}}{\pgfqpoint{1.816178in}{3.263044in}}%
\pgfpathclose%
\pgfusepath{stroke,fill}%
\end{pgfscope}%
\begin{pgfscope}%
\pgfpathrectangle{\pgfqpoint{0.100000in}{0.220728in}}{\pgfqpoint{3.696000in}{3.696000in}}%
\pgfusepath{clip}%
\pgfsetbuttcap%
\pgfsetroundjoin%
\definecolor{currentfill}{rgb}{0.121569,0.466667,0.705882}%
\pgfsetfillcolor{currentfill}%
\pgfsetfillopacity{0.309564}%
\pgfsetlinewidth{1.003750pt}%
\definecolor{currentstroke}{rgb}{0.121569,0.466667,0.705882}%
\pgfsetstrokecolor{currentstroke}%
\pgfsetstrokeopacity{0.309564}%
\pgfsetdash{}{0pt}%
\pgfpathmoveto{\pgfqpoint{1.741729in}{3.197436in}}%
\pgfpathcurveto{\pgfqpoint{1.749966in}{3.197436in}}{\pgfqpoint{1.757866in}{3.200708in}}{\pgfqpoint{1.763690in}{3.206532in}}%
\pgfpathcurveto{\pgfqpoint{1.769513in}{3.212356in}}{\pgfqpoint{1.772786in}{3.220256in}}{\pgfqpoint{1.772786in}{3.228492in}}%
\pgfpathcurveto{\pgfqpoint{1.772786in}{3.236728in}}{\pgfqpoint{1.769513in}{3.244628in}}{\pgfqpoint{1.763690in}{3.250452in}}%
\pgfpathcurveto{\pgfqpoint{1.757866in}{3.256276in}}{\pgfqpoint{1.749966in}{3.259549in}}{\pgfqpoint{1.741729in}{3.259549in}}%
\pgfpathcurveto{\pgfqpoint{1.733493in}{3.259549in}}{\pgfqpoint{1.725593in}{3.256276in}}{\pgfqpoint{1.719769in}{3.250452in}}%
\pgfpathcurveto{\pgfqpoint{1.713945in}{3.244628in}}{\pgfqpoint{1.710673in}{3.236728in}}{\pgfqpoint{1.710673in}{3.228492in}}%
\pgfpathcurveto{\pgfqpoint{1.710673in}{3.220256in}}{\pgfqpoint{1.713945in}{3.212356in}}{\pgfqpoint{1.719769in}{3.206532in}}%
\pgfpathcurveto{\pgfqpoint{1.725593in}{3.200708in}}{\pgfqpoint{1.733493in}{3.197436in}}{\pgfqpoint{1.741729in}{3.197436in}}%
\pgfpathclose%
\pgfusepath{stroke,fill}%
\end{pgfscope}%
\begin{pgfscope}%
\pgfpathrectangle{\pgfqpoint{0.100000in}{0.220728in}}{\pgfqpoint{3.696000in}{3.696000in}}%
\pgfusepath{clip}%
\pgfsetbuttcap%
\pgfsetroundjoin%
\definecolor{currentfill}{rgb}{0.121569,0.466667,0.705882}%
\pgfsetfillcolor{currentfill}%
\pgfsetfillopacity{0.309904}%
\pgfsetlinewidth{1.003750pt}%
\definecolor{currentstroke}{rgb}{0.121569,0.466667,0.705882}%
\pgfsetstrokecolor{currentstroke}%
\pgfsetstrokeopacity{0.309904}%
\pgfsetdash{}{0pt}%
\pgfpathmoveto{\pgfqpoint{1.820442in}{3.262289in}}%
\pgfpathcurveto{\pgfqpoint{1.828678in}{3.262289in}}{\pgfqpoint{1.836578in}{3.265562in}}{\pgfqpoint{1.842402in}{3.271386in}}%
\pgfpathcurveto{\pgfqpoint{1.848226in}{3.277209in}}{\pgfqpoint{1.851499in}{3.285110in}}{\pgfqpoint{1.851499in}{3.293346in}}%
\pgfpathcurveto{\pgfqpoint{1.851499in}{3.301582in}}{\pgfqpoint{1.848226in}{3.309482in}}{\pgfqpoint{1.842402in}{3.315306in}}%
\pgfpathcurveto{\pgfqpoint{1.836578in}{3.321130in}}{\pgfqpoint{1.828678in}{3.324402in}}{\pgfqpoint{1.820442in}{3.324402in}}%
\pgfpathcurveto{\pgfqpoint{1.812206in}{3.324402in}}{\pgfqpoint{1.804306in}{3.321130in}}{\pgfqpoint{1.798482in}{3.315306in}}%
\pgfpathcurveto{\pgfqpoint{1.792658in}{3.309482in}}{\pgfqpoint{1.789386in}{3.301582in}}{\pgfqpoint{1.789386in}{3.293346in}}%
\pgfpathcurveto{\pgfqpoint{1.789386in}{3.285110in}}{\pgfqpoint{1.792658in}{3.277209in}}{\pgfqpoint{1.798482in}{3.271386in}}%
\pgfpathcurveto{\pgfqpoint{1.804306in}{3.265562in}}{\pgfqpoint{1.812206in}{3.262289in}}{\pgfqpoint{1.820442in}{3.262289in}}%
\pgfpathclose%
\pgfusepath{stroke,fill}%
\end{pgfscope}%
\begin{pgfscope}%
\pgfpathrectangle{\pgfqpoint{0.100000in}{0.220728in}}{\pgfqpoint{3.696000in}{3.696000in}}%
\pgfusepath{clip}%
\pgfsetbuttcap%
\pgfsetroundjoin%
\definecolor{currentfill}{rgb}{0.121569,0.466667,0.705882}%
\pgfsetfillcolor{currentfill}%
\pgfsetfillopacity{0.310321}%
\pgfsetlinewidth{1.003750pt}%
\definecolor{currentstroke}{rgb}{0.121569,0.466667,0.705882}%
\pgfsetstrokecolor{currentstroke}%
\pgfsetstrokeopacity{0.310321}%
\pgfsetdash{}{0pt}%
\pgfpathmoveto{\pgfqpoint{1.822806in}{3.260916in}}%
\pgfpathcurveto{\pgfqpoint{1.831043in}{3.260916in}}{\pgfqpoint{1.838943in}{3.264188in}}{\pgfqpoint{1.844767in}{3.270012in}}%
\pgfpathcurveto{\pgfqpoint{1.850591in}{3.275836in}}{\pgfqpoint{1.853863in}{3.283736in}}{\pgfqpoint{1.853863in}{3.291972in}}%
\pgfpathcurveto{\pgfqpoint{1.853863in}{3.300208in}}{\pgfqpoint{1.850591in}{3.308108in}}{\pgfqpoint{1.844767in}{3.313932in}}%
\pgfpathcurveto{\pgfqpoint{1.838943in}{3.319756in}}{\pgfqpoint{1.831043in}{3.323029in}}{\pgfqpoint{1.822806in}{3.323029in}}%
\pgfpathcurveto{\pgfqpoint{1.814570in}{3.323029in}}{\pgfqpoint{1.806670in}{3.319756in}}{\pgfqpoint{1.800846in}{3.313932in}}%
\pgfpathcurveto{\pgfqpoint{1.795022in}{3.308108in}}{\pgfqpoint{1.791750in}{3.300208in}}{\pgfqpoint{1.791750in}{3.291972in}}%
\pgfpathcurveto{\pgfqpoint{1.791750in}{3.283736in}}{\pgfqpoint{1.795022in}{3.275836in}}{\pgfqpoint{1.800846in}{3.270012in}}%
\pgfpathcurveto{\pgfqpoint{1.806670in}{3.264188in}}{\pgfqpoint{1.814570in}{3.260916in}}{\pgfqpoint{1.822806in}{3.260916in}}%
\pgfpathclose%
\pgfusepath{stroke,fill}%
\end{pgfscope}%
\begin{pgfscope}%
\pgfpathrectangle{\pgfqpoint{0.100000in}{0.220728in}}{\pgfqpoint{3.696000in}{3.696000in}}%
\pgfusepath{clip}%
\pgfsetbuttcap%
\pgfsetroundjoin%
\definecolor{currentfill}{rgb}{0.121569,0.466667,0.705882}%
\pgfsetfillcolor{currentfill}%
\pgfsetfillopacity{0.311140}%
\pgfsetlinewidth{1.003750pt}%
\definecolor{currentstroke}{rgb}{0.121569,0.466667,0.705882}%
\pgfsetstrokecolor{currentstroke}%
\pgfsetstrokeopacity{0.311140}%
\pgfsetdash{}{0pt}%
\pgfpathmoveto{\pgfqpoint{1.739613in}{3.187898in}}%
\pgfpathcurveto{\pgfqpoint{1.747849in}{3.187898in}}{\pgfqpoint{1.755749in}{3.191171in}}{\pgfqpoint{1.761573in}{3.196995in}}%
\pgfpathcurveto{\pgfqpoint{1.767397in}{3.202819in}}{\pgfqpoint{1.770669in}{3.210719in}}{\pgfqpoint{1.770669in}{3.218955in}}%
\pgfpathcurveto{\pgfqpoint{1.770669in}{3.227191in}}{\pgfqpoint{1.767397in}{3.235091in}}{\pgfqpoint{1.761573in}{3.240915in}}%
\pgfpathcurveto{\pgfqpoint{1.755749in}{3.246739in}}{\pgfqpoint{1.747849in}{3.250011in}}{\pgfqpoint{1.739613in}{3.250011in}}%
\pgfpathcurveto{\pgfqpoint{1.731376in}{3.250011in}}{\pgfqpoint{1.723476in}{3.246739in}}{\pgfqpoint{1.717652in}{3.240915in}}%
\pgfpathcurveto{\pgfqpoint{1.711829in}{3.235091in}}{\pgfqpoint{1.708556in}{3.227191in}}{\pgfqpoint{1.708556in}{3.218955in}}%
\pgfpathcurveto{\pgfqpoint{1.708556in}{3.210719in}}{\pgfqpoint{1.711829in}{3.202819in}}{\pgfqpoint{1.717652in}{3.196995in}}%
\pgfpathcurveto{\pgfqpoint{1.723476in}{3.191171in}}{\pgfqpoint{1.731376in}{3.187898in}}{\pgfqpoint{1.739613in}{3.187898in}}%
\pgfpathclose%
\pgfusepath{stroke,fill}%
\end{pgfscope}%
\begin{pgfscope}%
\pgfpathrectangle{\pgfqpoint{0.100000in}{0.220728in}}{\pgfqpoint{3.696000in}{3.696000in}}%
\pgfusepath{clip}%
\pgfsetbuttcap%
\pgfsetroundjoin%
\definecolor{currentfill}{rgb}{0.121569,0.466667,0.705882}%
\pgfsetfillcolor{currentfill}%
\pgfsetfillopacity{0.311524}%
\pgfsetlinewidth{1.003750pt}%
\definecolor{currentstroke}{rgb}{0.121569,0.466667,0.705882}%
\pgfsetstrokecolor{currentstroke}%
\pgfsetstrokeopacity{0.311524}%
\pgfsetdash{}{0pt}%
\pgfpathmoveto{\pgfqpoint{1.825829in}{3.261117in}}%
\pgfpathcurveto{\pgfqpoint{1.834065in}{3.261117in}}{\pgfqpoint{1.841966in}{3.264389in}}{\pgfqpoint{1.847789in}{3.270213in}}%
\pgfpathcurveto{\pgfqpoint{1.853613in}{3.276037in}}{\pgfqpoint{1.856886in}{3.283937in}}{\pgfqpoint{1.856886in}{3.292173in}}%
\pgfpathcurveto{\pgfqpoint{1.856886in}{3.300409in}}{\pgfqpoint{1.853613in}{3.308309in}}{\pgfqpoint{1.847789in}{3.314133in}}%
\pgfpathcurveto{\pgfqpoint{1.841966in}{3.319957in}}{\pgfqpoint{1.834065in}{3.323230in}}{\pgfqpoint{1.825829in}{3.323230in}}%
\pgfpathcurveto{\pgfqpoint{1.817593in}{3.323230in}}{\pgfqpoint{1.809693in}{3.319957in}}{\pgfqpoint{1.803869in}{3.314133in}}%
\pgfpathcurveto{\pgfqpoint{1.798045in}{3.308309in}}{\pgfqpoint{1.794773in}{3.300409in}}{\pgfqpoint{1.794773in}{3.292173in}}%
\pgfpathcurveto{\pgfqpoint{1.794773in}{3.283937in}}{\pgfqpoint{1.798045in}{3.276037in}}{\pgfqpoint{1.803869in}{3.270213in}}%
\pgfpathcurveto{\pgfqpoint{1.809693in}{3.264389in}}{\pgfqpoint{1.817593in}{3.261117in}}{\pgfqpoint{1.825829in}{3.261117in}}%
\pgfpathclose%
\pgfusepath{stroke,fill}%
\end{pgfscope}%
\begin{pgfscope}%
\pgfpathrectangle{\pgfqpoint{0.100000in}{0.220728in}}{\pgfqpoint{3.696000in}{3.696000in}}%
\pgfusepath{clip}%
\pgfsetbuttcap%
\pgfsetroundjoin%
\definecolor{currentfill}{rgb}{0.121569,0.466667,0.705882}%
\pgfsetfillcolor{currentfill}%
\pgfsetfillopacity{0.311978}%
\pgfsetlinewidth{1.003750pt}%
\definecolor{currentstroke}{rgb}{0.121569,0.466667,0.705882}%
\pgfsetstrokecolor{currentstroke}%
\pgfsetstrokeopacity{0.311978}%
\pgfsetdash{}{0pt}%
\pgfpathmoveto{\pgfqpoint{1.735365in}{3.181543in}}%
\pgfpathcurveto{\pgfqpoint{1.743601in}{3.181543in}}{\pgfqpoint{1.751501in}{3.184815in}}{\pgfqpoint{1.757325in}{3.190639in}}%
\pgfpathcurveto{\pgfqpoint{1.763149in}{3.196463in}}{\pgfqpoint{1.766422in}{3.204363in}}{\pgfqpoint{1.766422in}{3.212599in}}%
\pgfpathcurveto{\pgfqpoint{1.766422in}{3.220836in}}{\pgfqpoint{1.763149in}{3.228736in}}{\pgfqpoint{1.757325in}{3.234560in}}%
\pgfpathcurveto{\pgfqpoint{1.751501in}{3.240383in}}{\pgfqpoint{1.743601in}{3.243656in}}{\pgfqpoint{1.735365in}{3.243656in}}%
\pgfpathcurveto{\pgfqpoint{1.727129in}{3.243656in}}{\pgfqpoint{1.719229in}{3.240383in}}{\pgfqpoint{1.713405in}{3.234560in}}%
\pgfpathcurveto{\pgfqpoint{1.707581in}{3.228736in}}{\pgfqpoint{1.704309in}{3.220836in}}{\pgfqpoint{1.704309in}{3.212599in}}%
\pgfpathcurveto{\pgfqpoint{1.704309in}{3.204363in}}{\pgfqpoint{1.707581in}{3.196463in}}{\pgfqpoint{1.713405in}{3.190639in}}%
\pgfpathcurveto{\pgfqpoint{1.719229in}{3.184815in}}{\pgfqpoint{1.727129in}{3.181543in}}{\pgfqpoint{1.735365in}{3.181543in}}%
\pgfpathclose%
\pgfusepath{stroke,fill}%
\end{pgfscope}%
\begin{pgfscope}%
\pgfpathrectangle{\pgfqpoint{0.100000in}{0.220728in}}{\pgfqpoint{3.696000in}{3.696000in}}%
\pgfusepath{clip}%
\pgfsetbuttcap%
\pgfsetroundjoin%
\definecolor{currentfill}{rgb}{0.121569,0.466667,0.705882}%
\pgfsetfillcolor{currentfill}%
\pgfsetfillopacity{0.312794}%
\pgfsetlinewidth{1.003750pt}%
\definecolor{currentstroke}{rgb}{0.121569,0.466667,0.705882}%
\pgfsetstrokecolor{currentstroke}%
\pgfsetstrokeopacity{0.312794}%
\pgfsetdash{}{0pt}%
\pgfpathmoveto{\pgfqpoint{1.734205in}{3.176729in}}%
\pgfpathcurveto{\pgfqpoint{1.742441in}{3.176729in}}{\pgfqpoint{1.750341in}{3.180001in}}{\pgfqpoint{1.756165in}{3.185825in}}%
\pgfpathcurveto{\pgfqpoint{1.761989in}{3.191649in}}{\pgfqpoint{1.765261in}{3.199549in}}{\pgfqpoint{1.765261in}{3.207785in}}%
\pgfpathcurveto{\pgfqpoint{1.765261in}{3.216022in}}{\pgfqpoint{1.761989in}{3.223922in}}{\pgfqpoint{1.756165in}{3.229746in}}%
\pgfpathcurveto{\pgfqpoint{1.750341in}{3.235570in}}{\pgfqpoint{1.742441in}{3.238842in}}{\pgfqpoint{1.734205in}{3.238842in}}%
\pgfpathcurveto{\pgfqpoint{1.725969in}{3.238842in}}{\pgfqpoint{1.718069in}{3.235570in}}{\pgfqpoint{1.712245in}{3.229746in}}%
\pgfpathcurveto{\pgfqpoint{1.706421in}{3.223922in}}{\pgfqpoint{1.703148in}{3.216022in}}{\pgfqpoint{1.703148in}{3.207785in}}%
\pgfpathcurveto{\pgfqpoint{1.703148in}{3.199549in}}{\pgfqpoint{1.706421in}{3.191649in}}{\pgfqpoint{1.712245in}{3.185825in}}%
\pgfpathcurveto{\pgfqpoint{1.718069in}{3.180001in}}{\pgfqpoint{1.725969in}{3.176729in}}{\pgfqpoint{1.734205in}{3.176729in}}%
\pgfpathclose%
\pgfusepath{stroke,fill}%
\end{pgfscope}%
\begin{pgfscope}%
\pgfpathrectangle{\pgfqpoint{0.100000in}{0.220728in}}{\pgfqpoint{3.696000in}{3.696000in}}%
\pgfusepath{clip}%
\pgfsetbuttcap%
\pgfsetroundjoin%
\definecolor{currentfill}{rgb}{0.121569,0.466667,0.705882}%
\pgfsetfillcolor{currentfill}%
\pgfsetfillopacity{0.313498}%
\pgfsetlinewidth{1.003750pt}%
\definecolor{currentstroke}{rgb}{0.121569,0.466667,0.705882}%
\pgfsetstrokecolor{currentstroke}%
\pgfsetstrokeopacity{0.313498}%
\pgfsetdash{}{0pt}%
\pgfpathmoveto{\pgfqpoint{1.732448in}{3.173054in}}%
\pgfpathcurveto{\pgfqpoint{1.740684in}{3.173054in}}{\pgfqpoint{1.748584in}{3.176326in}}{\pgfqpoint{1.754408in}{3.182150in}}%
\pgfpathcurveto{\pgfqpoint{1.760232in}{3.187974in}}{\pgfqpoint{1.763504in}{3.195874in}}{\pgfqpoint{1.763504in}{3.204111in}}%
\pgfpathcurveto{\pgfqpoint{1.763504in}{3.212347in}}{\pgfqpoint{1.760232in}{3.220247in}}{\pgfqpoint{1.754408in}{3.226071in}}%
\pgfpathcurveto{\pgfqpoint{1.748584in}{3.231895in}}{\pgfqpoint{1.740684in}{3.235167in}}{\pgfqpoint{1.732448in}{3.235167in}}%
\pgfpathcurveto{\pgfqpoint{1.724211in}{3.235167in}}{\pgfqpoint{1.716311in}{3.231895in}}{\pgfqpoint{1.710487in}{3.226071in}}%
\pgfpathcurveto{\pgfqpoint{1.704663in}{3.220247in}}{\pgfqpoint{1.701391in}{3.212347in}}{\pgfqpoint{1.701391in}{3.204111in}}%
\pgfpathcurveto{\pgfqpoint{1.701391in}{3.195874in}}{\pgfqpoint{1.704663in}{3.187974in}}{\pgfqpoint{1.710487in}{3.182150in}}%
\pgfpathcurveto{\pgfqpoint{1.716311in}{3.176326in}}{\pgfqpoint{1.724211in}{3.173054in}}{\pgfqpoint{1.732448in}{3.173054in}}%
\pgfpathclose%
\pgfusepath{stroke,fill}%
\end{pgfscope}%
\begin{pgfscope}%
\pgfpathrectangle{\pgfqpoint{0.100000in}{0.220728in}}{\pgfqpoint{3.696000in}{3.696000in}}%
\pgfusepath{clip}%
\pgfsetbuttcap%
\pgfsetroundjoin%
\definecolor{currentfill}{rgb}{0.121569,0.466667,0.705882}%
\pgfsetfillcolor{currentfill}%
\pgfsetfillopacity{0.314504}%
\pgfsetlinewidth{1.003750pt}%
\definecolor{currentstroke}{rgb}{0.121569,0.466667,0.705882}%
\pgfsetstrokecolor{currentstroke}%
\pgfsetstrokeopacity{0.314504}%
\pgfsetdash{}{0pt}%
\pgfpathmoveto{\pgfqpoint{1.826017in}{3.260078in}}%
\pgfpathcurveto{\pgfqpoint{1.834253in}{3.260078in}}{\pgfqpoint{1.842153in}{3.263351in}}{\pgfqpoint{1.847977in}{3.269175in}}%
\pgfpathcurveto{\pgfqpoint{1.853801in}{3.274998in}}{\pgfqpoint{1.857073in}{3.282899in}}{\pgfqpoint{1.857073in}{3.291135in}}%
\pgfpathcurveto{\pgfqpoint{1.857073in}{3.299371in}}{\pgfqpoint{1.853801in}{3.307271in}}{\pgfqpoint{1.847977in}{3.313095in}}%
\pgfpathcurveto{\pgfqpoint{1.842153in}{3.318919in}}{\pgfqpoint{1.834253in}{3.322191in}}{\pgfqpoint{1.826017in}{3.322191in}}%
\pgfpathcurveto{\pgfqpoint{1.817780in}{3.322191in}}{\pgfqpoint{1.809880in}{3.318919in}}{\pgfqpoint{1.804056in}{3.313095in}}%
\pgfpathcurveto{\pgfqpoint{1.798232in}{3.307271in}}{\pgfqpoint{1.794960in}{3.299371in}}{\pgfqpoint{1.794960in}{3.291135in}}%
\pgfpathcurveto{\pgfqpoint{1.794960in}{3.282899in}}{\pgfqpoint{1.798232in}{3.274998in}}{\pgfqpoint{1.804056in}{3.269175in}}%
\pgfpathcurveto{\pgfqpoint{1.809880in}{3.263351in}}{\pgfqpoint{1.817780in}{3.260078in}}{\pgfqpoint{1.826017in}{3.260078in}}%
\pgfpathclose%
\pgfusepath{stroke,fill}%
\end{pgfscope}%
\begin{pgfscope}%
\pgfpathrectangle{\pgfqpoint{0.100000in}{0.220728in}}{\pgfqpoint{3.696000in}{3.696000in}}%
\pgfusepath{clip}%
\pgfsetbuttcap%
\pgfsetroundjoin%
\definecolor{currentfill}{rgb}{0.121569,0.466667,0.705882}%
\pgfsetfillcolor{currentfill}%
\pgfsetfillopacity{0.314548}%
\pgfsetlinewidth{1.003750pt}%
\definecolor{currentstroke}{rgb}{0.121569,0.466667,0.705882}%
\pgfsetstrokecolor{currentstroke}%
\pgfsetstrokeopacity{0.314548}%
\pgfsetdash{}{0pt}%
\pgfpathmoveto{\pgfqpoint{1.728638in}{3.166071in}}%
\pgfpathcurveto{\pgfqpoint{1.736875in}{3.166071in}}{\pgfqpoint{1.744775in}{3.169343in}}{\pgfqpoint{1.750599in}{3.175167in}}%
\pgfpathcurveto{\pgfqpoint{1.756422in}{3.180991in}}{\pgfqpoint{1.759695in}{3.188891in}}{\pgfqpoint{1.759695in}{3.197127in}}%
\pgfpathcurveto{\pgfqpoint{1.759695in}{3.205363in}}{\pgfqpoint{1.756422in}{3.213263in}}{\pgfqpoint{1.750599in}{3.219087in}}%
\pgfpathcurveto{\pgfqpoint{1.744775in}{3.224911in}}{\pgfqpoint{1.736875in}{3.228184in}}{\pgfqpoint{1.728638in}{3.228184in}}%
\pgfpathcurveto{\pgfqpoint{1.720402in}{3.228184in}}{\pgfqpoint{1.712502in}{3.224911in}}{\pgfqpoint{1.706678in}{3.219087in}}%
\pgfpathcurveto{\pgfqpoint{1.700854in}{3.213263in}}{\pgfqpoint{1.697582in}{3.205363in}}{\pgfqpoint{1.697582in}{3.197127in}}%
\pgfpathcurveto{\pgfqpoint{1.697582in}{3.188891in}}{\pgfqpoint{1.700854in}{3.180991in}}{\pgfqpoint{1.706678in}{3.175167in}}%
\pgfpathcurveto{\pgfqpoint{1.712502in}{3.169343in}}{\pgfqpoint{1.720402in}{3.166071in}}{\pgfqpoint{1.728638in}{3.166071in}}%
\pgfpathclose%
\pgfusepath{stroke,fill}%
\end{pgfscope}%
\begin{pgfscope}%
\pgfpathrectangle{\pgfqpoint{0.100000in}{0.220728in}}{\pgfqpoint{3.696000in}{3.696000in}}%
\pgfusepath{clip}%
\pgfsetbuttcap%
\pgfsetroundjoin%
\definecolor{currentfill}{rgb}{0.121569,0.466667,0.705882}%
\pgfsetfillcolor{currentfill}%
\pgfsetfillopacity{0.315151}%
\pgfsetlinewidth{1.003750pt}%
\definecolor{currentstroke}{rgb}{0.121569,0.466667,0.705882}%
\pgfsetstrokecolor{currentstroke}%
\pgfsetstrokeopacity{0.315151}%
\pgfsetdash{}{0pt}%
\pgfpathmoveto{\pgfqpoint{1.829179in}{3.259267in}}%
\pgfpathcurveto{\pgfqpoint{1.837415in}{3.259267in}}{\pgfqpoint{1.845315in}{3.262539in}}{\pgfqpoint{1.851139in}{3.268363in}}%
\pgfpathcurveto{\pgfqpoint{1.856963in}{3.274187in}}{\pgfqpoint{1.860235in}{3.282087in}}{\pgfqpoint{1.860235in}{3.290323in}}%
\pgfpathcurveto{\pgfqpoint{1.860235in}{3.298559in}}{\pgfqpoint{1.856963in}{3.306460in}}{\pgfqpoint{1.851139in}{3.312283in}}%
\pgfpathcurveto{\pgfqpoint{1.845315in}{3.318107in}}{\pgfqpoint{1.837415in}{3.321380in}}{\pgfqpoint{1.829179in}{3.321380in}}%
\pgfpathcurveto{\pgfqpoint{1.820942in}{3.321380in}}{\pgfqpoint{1.813042in}{3.318107in}}{\pgfqpoint{1.807218in}{3.312283in}}%
\pgfpathcurveto{\pgfqpoint{1.801394in}{3.306460in}}{\pgfqpoint{1.798122in}{3.298559in}}{\pgfqpoint{1.798122in}{3.290323in}}%
\pgfpathcurveto{\pgfqpoint{1.798122in}{3.282087in}}{\pgfqpoint{1.801394in}{3.274187in}}{\pgfqpoint{1.807218in}{3.268363in}}%
\pgfpathcurveto{\pgfqpoint{1.813042in}{3.262539in}}{\pgfqpoint{1.820942in}{3.259267in}}{\pgfqpoint{1.829179in}{3.259267in}}%
\pgfpathclose%
\pgfusepath{stroke,fill}%
\end{pgfscope}%
\begin{pgfscope}%
\pgfpathrectangle{\pgfqpoint{0.100000in}{0.220728in}}{\pgfqpoint{3.696000in}{3.696000in}}%
\pgfusepath{clip}%
\pgfsetbuttcap%
\pgfsetroundjoin%
\definecolor{currentfill}{rgb}{0.121569,0.466667,0.705882}%
\pgfsetfillcolor{currentfill}%
\pgfsetfillopacity{0.315689}%
\pgfsetlinewidth{1.003750pt}%
\definecolor{currentstroke}{rgb}{0.121569,0.466667,0.705882}%
\pgfsetstrokecolor{currentstroke}%
\pgfsetstrokeopacity{0.315689}%
\pgfsetdash{}{0pt}%
\pgfpathmoveto{\pgfqpoint{1.727562in}{3.158665in}}%
\pgfpathcurveto{\pgfqpoint{1.735798in}{3.158665in}}{\pgfqpoint{1.743698in}{3.161937in}}{\pgfqpoint{1.749522in}{3.167761in}}%
\pgfpathcurveto{\pgfqpoint{1.755346in}{3.173585in}}{\pgfqpoint{1.758618in}{3.181485in}}{\pgfqpoint{1.758618in}{3.189721in}}%
\pgfpathcurveto{\pgfqpoint{1.758618in}{3.197958in}}{\pgfqpoint{1.755346in}{3.205858in}}{\pgfqpoint{1.749522in}{3.211682in}}%
\pgfpathcurveto{\pgfqpoint{1.743698in}{3.217505in}}{\pgfqpoint{1.735798in}{3.220778in}}{\pgfqpoint{1.727562in}{3.220778in}}%
\pgfpathcurveto{\pgfqpoint{1.719325in}{3.220778in}}{\pgfqpoint{1.711425in}{3.217505in}}{\pgfqpoint{1.705601in}{3.211682in}}%
\pgfpathcurveto{\pgfqpoint{1.699777in}{3.205858in}}{\pgfqpoint{1.696505in}{3.197958in}}{\pgfqpoint{1.696505in}{3.189721in}}%
\pgfpathcurveto{\pgfqpoint{1.696505in}{3.181485in}}{\pgfqpoint{1.699777in}{3.173585in}}{\pgfqpoint{1.705601in}{3.167761in}}%
\pgfpathcurveto{\pgfqpoint{1.711425in}{3.161937in}}{\pgfqpoint{1.719325in}{3.158665in}}{\pgfqpoint{1.727562in}{3.158665in}}%
\pgfpathclose%
\pgfusepath{stroke,fill}%
\end{pgfscope}%
\begin{pgfscope}%
\pgfpathrectangle{\pgfqpoint{0.100000in}{0.220728in}}{\pgfqpoint{3.696000in}{3.696000in}}%
\pgfusepath{clip}%
\pgfsetbuttcap%
\pgfsetroundjoin%
\definecolor{currentfill}{rgb}{0.121569,0.466667,0.705882}%
\pgfsetfillcolor{currentfill}%
\pgfsetfillopacity{0.316271}%
\pgfsetlinewidth{1.003750pt}%
\definecolor{currentstroke}{rgb}{0.121569,0.466667,0.705882}%
\pgfsetstrokecolor{currentstroke}%
\pgfsetstrokeopacity{0.316271}%
\pgfsetdash{}{0pt}%
\pgfpathmoveto{\pgfqpoint{1.724530in}{3.153916in}}%
\pgfpathcurveto{\pgfqpoint{1.732767in}{3.153916in}}{\pgfqpoint{1.740667in}{3.157188in}}{\pgfqpoint{1.746491in}{3.163012in}}%
\pgfpathcurveto{\pgfqpoint{1.752314in}{3.168836in}}{\pgfqpoint{1.755587in}{3.176736in}}{\pgfqpoint{1.755587in}{3.184972in}}%
\pgfpathcurveto{\pgfqpoint{1.755587in}{3.193209in}}{\pgfqpoint{1.752314in}{3.201109in}}{\pgfqpoint{1.746491in}{3.206933in}}%
\pgfpathcurveto{\pgfqpoint{1.740667in}{3.212757in}}{\pgfqpoint{1.732767in}{3.216029in}}{\pgfqpoint{1.724530in}{3.216029in}}%
\pgfpathcurveto{\pgfqpoint{1.716294in}{3.216029in}}{\pgfqpoint{1.708394in}{3.212757in}}{\pgfqpoint{1.702570in}{3.206933in}}%
\pgfpathcurveto{\pgfqpoint{1.696746in}{3.201109in}}{\pgfqpoint{1.693474in}{3.193209in}}{\pgfqpoint{1.693474in}{3.184972in}}%
\pgfpathcurveto{\pgfqpoint{1.693474in}{3.176736in}}{\pgfqpoint{1.696746in}{3.168836in}}{\pgfqpoint{1.702570in}{3.163012in}}%
\pgfpathcurveto{\pgfqpoint{1.708394in}{3.157188in}}{\pgfqpoint{1.716294in}{3.153916in}}{\pgfqpoint{1.724530in}{3.153916in}}%
\pgfpathclose%
\pgfusepath{stroke,fill}%
\end{pgfscope}%
\begin{pgfscope}%
\pgfpathrectangle{\pgfqpoint{0.100000in}{0.220728in}}{\pgfqpoint{3.696000in}{3.696000in}}%
\pgfusepath{clip}%
\pgfsetbuttcap%
\pgfsetroundjoin%
\definecolor{currentfill}{rgb}{0.121569,0.466667,0.705882}%
\pgfsetfillcolor{currentfill}%
\pgfsetfillopacity{0.316428}%
\pgfsetlinewidth{1.003750pt}%
\definecolor{currentstroke}{rgb}{0.121569,0.466667,0.705882}%
\pgfsetstrokecolor{currentstroke}%
\pgfsetstrokeopacity{0.316428}%
\pgfsetdash{}{0pt}%
\pgfpathmoveto{\pgfqpoint{1.832916in}{3.259021in}}%
\pgfpathcurveto{\pgfqpoint{1.841152in}{3.259021in}}{\pgfqpoint{1.849053in}{3.262293in}}{\pgfqpoint{1.854876in}{3.268117in}}%
\pgfpathcurveto{\pgfqpoint{1.860700in}{3.273941in}}{\pgfqpoint{1.863973in}{3.281841in}}{\pgfqpoint{1.863973in}{3.290077in}}%
\pgfpathcurveto{\pgfqpoint{1.863973in}{3.298313in}}{\pgfqpoint{1.860700in}{3.306214in}}{\pgfqpoint{1.854876in}{3.312037in}}%
\pgfpathcurveto{\pgfqpoint{1.849053in}{3.317861in}}{\pgfqpoint{1.841152in}{3.321134in}}{\pgfqpoint{1.832916in}{3.321134in}}%
\pgfpathcurveto{\pgfqpoint{1.824680in}{3.321134in}}{\pgfqpoint{1.816780in}{3.317861in}}{\pgfqpoint{1.810956in}{3.312037in}}%
\pgfpathcurveto{\pgfqpoint{1.805132in}{3.306214in}}{\pgfqpoint{1.801860in}{3.298313in}}{\pgfqpoint{1.801860in}{3.290077in}}%
\pgfpathcurveto{\pgfqpoint{1.801860in}{3.281841in}}{\pgfqpoint{1.805132in}{3.273941in}}{\pgfqpoint{1.810956in}{3.268117in}}%
\pgfpathcurveto{\pgfqpoint{1.816780in}{3.262293in}}{\pgfqpoint{1.824680in}{3.259021in}}{\pgfqpoint{1.832916in}{3.259021in}}%
\pgfpathclose%
\pgfusepath{stroke,fill}%
\end{pgfscope}%
\begin{pgfscope}%
\pgfpathrectangle{\pgfqpoint{0.100000in}{0.220728in}}{\pgfqpoint{3.696000in}{3.696000in}}%
\pgfusepath{clip}%
\pgfsetbuttcap%
\pgfsetroundjoin%
\definecolor{currentfill}{rgb}{0.121569,0.466667,0.705882}%
\pgfsetfillcolor{currentfill}%
\pgfsetfillopacity{0.316956}%
\pgfsetlinewidth{1.003750pt}%
\definecolor{currentstroke}{rgb}{0.121569,0.466667,0.705882}%
\pgfsetstrokecolor{currentstroke}%
\pgfsetstrokeopacity{0.316956}%
\pgfsetdash{}{0pt}%
\pgfpathmoveto{\pgfqpoint{1.723144in}{3.149632in}}%
\pgfpathcurveto{\pgfqpoint{1.731380in}{3.149632in}}{\pgfqpoint{1.739281in}{3.152905in}}{\pgfqpoint{1.745104in}{3.158729in}}%
\pgfpathcurveto{\pgfqpoint{1.750928in}{3.164552in}}{\pgfqpoint{1.754201in}{3.172453in}}{\pgfqpoint{1.754201in}{3.180689in}}%
\pgfpathcurveto{\pgfqpoint{1.754201in}{3.188925in}}{\pgfqpoint{1.750928in}{3.196825in}}{\pgfqpoint{1.745104in}{3.202649in}}%
\pgfpathcurveto{\pgfqpoint{1.739281in}{3.208473in}}{\pgfqpoint{1.731380in}{3.211745in}}{\pgfqpoint{1.723144in}{3.211745in}}%
\pgfpathcurveto{\pgfqpoint{1.714908in}{3.211745in}}{\pgfqpoint{1.707008in}{3.208473in}}{\pgfqpoint{1.701184in}{3.202649in}}%
\pgfpathcurveto{\pgfqpoint{1.695360in}{3.196825in}}{\pgfqpoint{1.692088in}{3.188925in}}{\pgfqpoint{1.692088in}{3.180689in}}%
\pgfpathcurveto{\pgfqpoint{1.692088in}{3.172453in}}{\pgfqpoint{1.695360in}{3.164552in}}{\pgfqpoint{1.701184in}{3.158729in}}%
\pgfpathcurveto{\pgfqpoint{1.707008in}{3.152905in}}{\pgfqpoint{1.714908in}{3.149632in}}{\pgfqpoint{1.723144in}{3.149632in}}%
\pgfpathclose%
\pgfusepath{stroke,fill}%
\end{pgfscope}%
\begin{pgfscope}%
\pgfpathrectangle{\pgfqpoint{0.100000in}{0.220728in}}{\pgfqpoint{3.696000in}{3.696000in}}%
\pgfusepath{clip}%
\pgfsetbuttcap%
\pgfsetroundjoin%
\definecolor{currentfill}{rgb}{0.121569,0.466667,0.705882}%
\pgfsetfillcolor{currentfill}%
\pgfsetfillopacity{0.317535}%
\pgfsetlinewidth{1.003750pt}%
\definecolor{currentstroke}{rgb}{0.121569,0.466667,0.705882}%
\pgfsetstrokecolor{currentstroke}%
\pgfsetstrokeopacity{0.317535}%
\pgfsetdash{}{0pt}%
\pgfpathmoveto{\pgfqpoint{1.722237in}{3.146024in}}%
\pgfpathcurveto{\pgfqpoint{1.730473in}{3.146024in}}{\pgfqpoint{1.738373in}{3.149297in}}{\pgfqpoint{1.744197in}{3.155120in}}%
\pgfpathcurveto{\pgfqpoint{1.750021in}{3.160944in}}{\pgfqpoint{1.753293in}{3.168844in}}{\pgfqpoint{1.753293in}{3.177081in}}%
\pgfpathcurveto{\pgfqpoint{1.753293in}{3.185317in}}{\pgfqpoint{1.750021in}{3.193217in}}{\pgfqpoint{1.744197in}{3.199041in}}%
\pgfpathcurveto{\pgfqpoint{1.738373in}{3.204865in}}{\pgfqpoint{1.730473in}{3.208137in}}{\pgfqpoint{1.722237in}{3.208137in}}%
\pgfpathcurveto{\pgfqpoint{1.714000in}{3.208137in}}{\pgfqpoint{1.706100in}{3.204865in}}{\pgfqpoint{1.700276in}{3.199041in}}%
\pgfpathcurveto{\pgfqpoint{1.694452in}{3.193217in}}{\pgfqpoint{1.691180in}{3.185317in}}{\pgfqpoint{1.691180in}{3.177081in}}%
\pgfpathcurveto{\pgfqpoint{1.691180in}{3.168844in}}{\pgfqpoint{1.694452in}{3.160944in}}{\pgfqpoint{1.700276in}{3.155120in}}%
\pgfpathcurveto{\pgfqpoint{1.706100in}{3.149297in}}{\pgfqpoint{1.714000in}{3.146024in}}{\pgfqpoint{1.722237in}{3.146024in}}%
\pgfpathclose%
\pgfusepath{stroke,fill}%
\end{pgfscope}%
\begin{pgfscope}%
\pgfpathrectangle{\pgfqpoint{0.100000in}{0.220728in}}{\pgfqpoint{3.696000in}{3.696000in}}%
\pgfusepath{clip}%
\pgfsetbuttcap%
\pgfsetroundjoin%
\definecolor{currentfill}{rgb}{0.121569,0.466667,0.705882}%
\pgfsetfillcolor{currentfill}%
\pgfsetfillopacity{0.317608}%
\pgfsetlinewidth{1.003750pt}%
\definecolor{currentstroke}{rgb}{0.121569,0.466667,0.705882}%
\pgfsetstrokecolor{currentstroke}%
\pgfsetstrokeopacity{0.317608}%
\pgfsetdash{}{0pt}%
\pgfpathmoveto{\pgfqpoint{1.721782in}{3.145346in}}%
\pgfpathcurveto{\pgfqpoint{1.730018in}{3.145346in}}{\pgfqpoint{1.737918in}{3.148618in}}{\pgfqpoint{1.743742in}{3.154442in}}%
\pgfpathcurveto{\pgfqpoint{1.749566in}{3.160266in}}{\pgfqpoint{1.752838in}{3.168166in}}{\pgfqpoint{1.752838in}{3.176402in}}%
\pgfpathcurveto{\pgfqpoint{1.752838in}{3.184639in}}{\pgfqpoint{1.749566in}{3.192539in}}{\pgfqpoint{1.743742in}{3.198363in}}%
\pgfpathcurveto{\pgfqpoint{1.737918in}{3.204187in}}{\pgfqpoint{1.730018in}{3.207459in}}{\pgfqpoint{1.721782in}{3.207459in}}%
\pgfpathcurveto{\pgfqpoint{1.713546in}{3.207459in}}{\pgfqpoint{1.705645in}{3.204187in}}{\pgfqpoint{1.699822in}{3.198363in}}%
\pgfpathcurveto{\pgfqpoint{1.693998in}{3.192539in}}{\pgfqpoint{1.690725in}{3.184639in}}{\pgfqpoint{1.690725in}{3.176402in}}%
\pgfpathcurveto{\pgfqpoint{1.690725in}{3.168166in}}{\pgfqpoint{1.693998in}{3.160266in}}{\pgfqpoint{1.699822in}{3.154442in}}%
\pgfpathcurveto{\pgfqpoint{1.705645in}{3.148618in}}{\pgfqpoint{1.713546in}{3.145346in}}{\pgfqpoint{1.721782in}{3.145346in}}%
\pgfpathclose%
\pgfusepath{stroke,fill}%
\end{pgfscope}%
\begin{pgfscope}%
\pgfpathrectangle{\pgfqpoint{0.100000in}{0.220728in}}{\pgfqpoint{3.696000in}{3.696000in}}%
\pgfusepath{clip}%
\pgfsetbuttcap%
\pgfsetroundjoin%
\definecolor{currentfill}{rgb}{0.121569,0.466667,0.705882}%
\pgfsetfillcolor{currentfill}%
\pgfsetfillopacity{0.317750}%
\pgfsetlinewidth{1.003750pt}%
\definecolor{currentstroke}{rgb}{0.121569,0.466667,0.705882}%
\pgfsetstrokecolor{currentstroke}%
\pgfsetstrokeopacity{0.317750}%
\pgfsetdash{}{0pt}%
\pgfpathmoveto{\pgfqpoint{1.838595in}{3.259056in}}%
\pgfpathcurveto{\pgfqpoint{1.846832in}{3.259056in}}{\pgfqpoint{1.854732in}{3.262329in}}{\pgfqpoint{1.860556in}{3.268153in}}%
\pgfpathcurveto{\pgfqpoint{1.866380in}{3.273977in}}{\pgfqpoint{1.869652in}{3.281877in}}{\pgfqpoint{1.869652in}{3.290113in}}%
\pgfpathcurveto{\pgfqpoint{1.869652in}{3.298349in}}{\pgfqpoint{1.866380in}{3.306249in}}{\pgfqpoint{1.860556in}{3.312073in}}%
\pgfpathcurveto{\pgfqpoint{1.854732in}{3.317897in}}{\pgfqpoint{1.846832in}{3.321169in}}{\pgfqpoint{1.838595in}{3.321169in}}%
\pgfpathcurveto{\pgfqpoint{1.830359in}{3.321169in}}{\pgfqpoint{1.822459in}{3.317897in}}{\pgfqpoint{1.816635in}{3.312073in}}%
\pgfpathcurveto{\pgfqpoint{1.810811in}{3.306249in}}{\pgfqpoint{1.807539in}{3.298349in}}{\pgfqpoint{1.807539in}{3.290113in}}%
\pgfpathcurveto{\pgfqpoint{1.807539in}{3.281877in}}{\pgfqpoint{1.810811in}{3.273977in}}{\pgfqpoint{1.816635in}{3.268153in}}%
\pgfpathcurveto{\pgfqpoint{1.822459in}{3.262329in}}{\pgfqpoint{1.830359in}{3.259056in}}{\pgfqpoint{1.838595in}{3.259056in}}%
\pgfpathclose%
\pgfusepath{stroke,fill}%
\end{pgfscope}%
\begin{pgfscope}%
\pgfpathrectangle{\pgfqpoint{0.100000in}{0.220728in}}{\pgfqpoint{3.696000in}{3.696000in}}%
\pgfusepath{clip}%
\pgfsetbuttcap%
\pgfsetroundjoin%
\definecolor{currentfill}{rgb}{0.121569,0.466667,0.705882}%
\pgfsetfillcolor{currentfill}%
\pgfsetfillopacity{0.317836}%
\pgfsetlinewidth{1.003750pt}%
\definecolor{currentstroke}{rgb}{0.121569,0.466667,0.705882}%
\pgfsetstrokecolor{currentstroke}%
\pgfsetstrokeopacity{0.317836}%
\pgfsetdash{}{0pt}%
\pgfpathmoveto{\pgfqpoint{1.721352in}{3.144037in}}%
\pgfpathcurveto{\pgfqpoint{1.729588in}{3.144037in}}{\pgfqpoint{1.737488in}{3.147310in}}{\pgfqpoint{1.743312in}{3.153134in}}%
\pgfpathcurveto{\pgfqpoint{1.749136in}{3.158958in}}{\pgfqpoint{1.752408in}{3.166858in}}{\pgfqpoint{1.752408in}{3.175094in}}%
\pgfpathcurveto{\pgfqpoint{1.752408in}{3.183330in}}{\pgfqpoint{1.749136in}{3.191230in}}{\pgfqpoint{1.743312in}{3.197054in}}%
\pgfpathcurveto{\pgfqpoint{1.737488in}{3.202878in}}{\pgfqpoint{1.729588in}{3.206150in}}{\pgfqpoint{1.721352in}{3.206150in}}%
\pgfpathcurveto{\pgfqpoint{1.713116in}{3.206150in}}{\pgfqpoint{1.705216in}{3.202878in}}{\pgfqpoint{1.699392in}{3.197054in}}%
\pgfpathcurveto{\pgfqpoint{1.693568in}{3.191230in}}{\pgfqpoint{1.690295in}{3.183330in}}{\pgfqpoint{1.690295in}{3.175094in}}%
\pgfpathcurveto{\pgfqpoint{1.690295in}{3.166858in}}{\pgfqpoint{1.693568in}{3.158958in}}{\pgfqpoint{1.699392in}{3.153134in}}%
\pgfpathcurveto{\pgfqpoint{1.705216in}{3.147310in}}{\pgfqpoint{1.713116in}{3.144037in}}{\pgfqpoint{1.721352in}{3.144037in}}%
\pgfpathclose%
\pgfusepath{stroke,fill}%
\end{pgfscope}%
\begin{pgfscope}%
\pgfpathrectangle{\pgfqpoint{0.100000in}{0.220728in}}{\pgfqpoint{3.696000in}{3.696000in}}%
\pgfusepath{clip}%
\pgfsetbuttcap%
\pgfsetroundjoin%
\definecolor{currentfill}{rgb}{0.121569,0.466667,0.705882}%
\pgfsetfillcolor{currentfill}%
\pgfsetfillopacity{0.318251}%
\pgfsetlinewidth{1.003750pt}%
\definecolor{currentstroke}{rgb}{0.121569,0.466667,0.705882}%
\pgfsetstrokecolor{currentstroke}%
\pgfsetstrokeopacity{0.318251}%
\pgfsetdash{}{0pt}%
\pgfpathmoveto{\pgfqpoint{1.720423in}{3.141780in}}%
\pgfpathcurveto{\pgfqpoint{1.728659in}{3.141780in}}{\pgfqpoint{1.736559in}{3.145052in}}{\pgfqpoint{1.742383in}{3.150876in}}%
\pgfpathcurveto{\pgfqpoint{1.748207in}{3.156700in}}{\pgfqpoint{1.751479in}{3.164600in}}{\pgfqpoint{1.751479in}{3.172836in}}%
\pgfpathcurveto{\pgfqpoint{1.751479in}{3.181072in}}{\pgfqpoint{1.748207in}{3.188972in}}{\pgfqpoint{1.742383in}{3.194796in}}%
\pgfpathcurveto{\pgfqpoint{1.736559in}{3.200620in}}{\pgfqpoint{1.728659in}{3.203893in}}{\pgfqpoint{1.720423in}{3.203893in}}%
\pgfpathcurveto{\pgfqpoint{1.712187in}{3.203893in}}{\pgfqpoint{1.704287in}{3.200620in}}{\pgfqpoint{1.698463in}{3.194796in}}%
\pgfpathcurveto{\pgfqpoint{1.692639in}{3.188972in}}{\pgfqpoint{1.689366in}{3.181072in}}{\pgfqpoint{1.689366in}{3.172836in}}%
\pgfpathcurveto{\pgfqpoint{1.689366in}{3.164600in}}{\pgfqpoint{1.692639in}{3.156700in}}{\pgfqpoint{1.698463in}{3.150876in}}%
\pgfpathcurveto{\pgfqpoint{1.704287in}{3.145052in}}{\pgfqpoint{1.712187in}{3.141780in}}{\pgfqpoint{1.720423in}{3.141780in}}%
\pgfpathclose%
\pgfusepath{stroke,fill}%
\end{pgfscope}%
\begin{pgfscope}%
\pgfpathrectangle{\pgfqpoint{0.100000in}{0.220728in}}{\pgfqpoint{3.696000in}{3.696000in}}%
\pgfusepath{clip}%
\pgfsetbuttcap%
\pgfsetroundjoin%
\definecolor{currentfill}{rgb}{0.121569,0.466667,0.705882}%
\pgfsetfillcolor{currentfill}%
\pgfsetfillopacity{0.318824}%
\pgfsetlinewidth{1.003750pt}%
\definecolor{currentstroke}{rgb}{0.121569,0.466667,0.705882}%
\pgfsetstrokecolor{currentstroke}%
\pgfsetstrokeopacity{0.318824}%
\pgfsetdash{}{0pt}%
\pgfpathmoveto{\pgfqpoint{1.717928in}{3.137839in}}%
\pgfpathcurveto{\pgfqpoint{1.726164in}{3.137839in}}{\pgfqpoint{1.734064in}{3.141111in}}{\pgfqpoint{1.739888in}{3.146935in}}%
\pgfpathcurveto{\pgfqpoint{1.745712in}{3.152759in}}{\pgfqpoint{1.748985in}{3.160659in}}{\pgfqpoint{1.748985in}{3.168895in}}%
\pgfpathcurveto{\pgfqpoint{1.748985in}{3.177131in}}{\pgfqpoint{1.745712in}{3.185031in}}{\pgfqpoint{1.739888in}{3.190855in}}%
\pgfpathcurveto{\pgfqpoint{1.734064in}{3.196679in}}{\pgfqpoint{1.726164in}{3.199952in}}{\pgfqpoint{1.717928in}{3.199952in}}%
\pgfpathcurveto{\pgfqpoint{1.709692in}{3.199952in}}{\pgfqpoint{1.701792in}{3.196679in}}{\pgfqpoint{1.695968in}{3.190855in}}%
\pgfpathcurveto{\pgfqpoint{1.690144in}{3.185031in}}{\pgfqpoint{1.686872in}{3.177131in}}{\pgfqpoint{1.686872in}{3.168895in}}%
\pgfpathcurveto{\pgfqpoint{1.686872in}{3.160659in}}{\pgfqpoint{1.690144in}{3.152759in}}{\pgfqpoint{1.695968in}{3.146935in}}%
\pgfpathcurveto{\pgfqpoint{1.701792in}{3.141111in}}{\pgfqpoint{1.709692in}{3.137839in}}{\pgfqpoint{1.717928in}{3.137839in}}%
\pgfpathclose%
\pgfusepath{stroke,fill}%
\end{pgfscope}%
\begin{pgfscope}%
\pgfpathrectangle{\pgfqpoint{0.100000in}{0.220728in}}{\pgfqpoint{3.696000in}{3.696000in}}%
\pgfusepath{clip}%
\pgfsetbuttcap%
\pgfsetroundjoin%
\definecolor{currentfill}{rgb}{0.121569,0.466667,0.705882}%
\pgfsetfillcolor{currentfill}%
\pgfsetfillopacity{0.319234}%
\pgfsetlinewidth{1.003750pt}%
\definecolor{currentstroke}{rgb}{0.121569,0.466667,0.705882}%
\pgfsetstrokecolor{currentstroke}%
\pgfsetstrokeopacity{0.319234}%
\pgfsetdash{}{0pt}%
\pgfpathmoveto{\pgfqpoint{1.717592in}{3.134948in}}%
\pgfpathcurveto{\pgfqpoint{1.725828in}{3.134948in}}{\pgfqpoint{1.733728in}{3.138221in}}{\pgfqpoint{1.739552in}{3.144045in}}%
\pgfpathcurveto{\pgfqpoint{1.745376in}{3.149869in}}{\pgfqpoint{1.748648in}{3.157769in}}{\pgfqpoint{1.748648in}{3.166005in}}%
\pgfpathcurveto{\pgfqpoint{1.748648in}{3.174241in}}{\pgfqpoint{1.745376in}{3.182141in}}{\pgfqpoint{1.739552in}{3.187965in}}%
\pgfpathcurveto{\pgfqpoint{1.733728in}{3.193789in}}{\pgfqpoint{1.725828in}{3.197061in}}{\pgfqpoint{1.717592in}{3.197061in}}%
\pgfpathcurveto{\pgfqpoint{1.709355in}{3.197061in}}{\pgfqpoint{1.701455in}{3.193789in}}{\pgfqpoint{1.695631in}{3.187965in}}%
\pgfpathcurveto{\pgfqpoint{1.689807in}{3.182141in}}{\pgfqpoint{1.686535in}{3.174241in}}{\pgfqpoint{1.686535in}{3.166005in}}%
\pgfpathcurveto{\pgfqpoint{1.686535in}{3.157769in}}{\pgfqpoint{1.689807in}{3.149869in}}{\pgfqpoint{1.695631in}{3.144045in}}%
\pgfpathcurveto{\pgfqpoint{1.701455in}{3.138221in}}{\pgfqpoint{1.709355in}{3.134948in}}{\pgfqpoint{1.717592in}{3.134948in}}%
\pgfpathclose%
\pgfusepath{stroke,fill}%
\end{pgfscope}%
\begin{pgfscope}%
\pgfpathrectangle{\pgfqpoint{0.100000in}{0.220728in}}{\pgfqpoint{3.696000in}{3.696000in}}%
\pgfusepath{clip}%
\pgfsetbuttcap%
\pgfsetroundjoin%
\definecolor{currentfill}{rgb}{0.121569,0.466667,0.705882}%
\pgfsetfillcolor{currentfill}%
\pgfsetfillopacity{0.319420}%
\pgfsetlinewidth{1.003750pt}%
\definecolor{currentstroke}{rgb}{0.121569,0.466667,0.705882}%
\pgfsetstrokecolor{currentstroke}%
\pgfsetstrokeopacity{0.319420}%
\pgfsetdash{}{0pt}%
\pgfpathmoveto{\pgfqpoint{1.716551in}{3.133462in}}%
\pgfpathcurveto{\pgfqpoint{1.724787in}{3.133462in}}{\pgfqpoint{1.732687in}{3.136734in}}{\pgfqpoint{1.738511in}{3.142558in}}%
\pgfpathcurveto{\pgfqpoint{1.744335in}{3.148382in}}{\pgfqpoint{1.747608in}{3.156282in}}{\pgfqpoint{1.747608in}{3.164518in}}%
\pgfpathcurveto{\pgfqpoint{1.747608in}{3.172755in}}{\pgfqpoint{1.744335in}{3.180655in}}{\pgfqpoint{1.738511in}{3.186479in}}%
\pgfpathcurveto{\pgfqpoint{1.732687in}{3.192302in}}{\pgfqpoint{1.724787in}{3.195575in}}{\pgfqpoint{1.716551in}{3.195575in}}%
\pgfpathcurveto{\pgfqpoint{1.708315in}{3.195575in}}{\pgfqpoint{1.700415in}{3.192302in}}{\pgfqpoint{1.694591in}{3.186479in}}%
\pgfpathcurveto{\pgfqpoint{1.688767in}{3.180655in}}{\pgfqpoint{1.685495in}{3.172755in}}{\pgfqpoint{1.685495in}{3.164518in}}%
\pgfpathcurveto{\pgfqpoint{1.685495in}{3.156282in}}{\pgfqpoint{1.688767in}{3.148382in}}{\pgfqpoint{1.694591in}{3.142558in}}%
\pgfpathcurveto{\pgfqpoint{1.700415in}{3.136734in}}{\pgfqpoint{1.708315in}{3.133462in}}{\pgfqpoint{1.716551in}{3.133462in}}%
\pgfpathclose%
\pgfusepath{stroke,fill}%
\end{pgfscope}%
\begin{pgfscope}%
\pgfpathrectangle{\pgfqpoint{0.100000in}{0.220728in}}{\pgfqpoint{3.696000in}{3.696000in}}%
\pgfusepath{clip}%
\pgfsetbuttcap%
\pgfsetroundjoin%
\definecolor{currentfill}{rgb}{0.121569,0.466667,0.705882}%
\pgfsetfillcolor{currentfill}%
\pgfsetfillopacity{0.319920}%
\pgfsetlinewidth{1.003750pt}%
\definecolor{currentstroke}{rgb}{0.121569,0.466667,0.705882}%
\pgfsetstrokecolor{currentstroke}%
\pgfsetstrokeopacity{0.319920}%
\pgfsetdash{}{0pt}%
\pgfpathmoveto{\pgfqpoint{1.715377in}{3.130511in}}%
\pgfpathcurveto{\pgfqpoint{1.723613in}{3.130511in}}{\pgfqpoint{1.731513in}{3.133783in}}{\pgfqpoint{1.737337in}{3.139607in}}%
\pgfpathcurveto{\pgfqpoint{1.743161in}{3.145431in}}{\pgfqpoint{1.746433in}{3.153331in}}{\pgfqpoint{1.746433in}{3.161567in}}%
\pgfpathcurveto{\pgfqpoint{1.746433in}{3.169804in}}{\pgfqpoint{1.743161in}{3.177704in}}{\pgfqpoint{1.737337in}{3.183528in}}%
\pgfpathcurveto{\pgfqpoint{1.731513in}{3.189352in}}{\pgfqpoint{1.723613in}{3.192624in}}{\pgfqpoint{1.715377in}{3.192624in}}%
\pgfpathcurveto{\pgfqpoint{1.707141in}{3.192624in}}{\pgfqpoint{1.699241in}{3.189352in}}{\pgfqpoint{1.693417in}{3.183528in}}%
\pgfpathcurveto{\pgfqpoint{1.687593in}{3.177704in}}{\pgfqpoint{1.684320in}{3.169804in}}{\pgfqpoint{1.684320in}{3.161567in}}%
\pgfpathcurveto{\pgfqpoint{1.684320in}{3.153331in}}{\pgfqpoint{1.687593in}{3.145431in}}{\pgfqpoint{1.693417in}{3.139607in}}%
\pgfpathcurveto{\pgfqpoint{1.699241in}{3.133783in}}{\pgfqpoint{1.707141in}{3.130511in}}{\pgfqpoint{1.715377in}{3.130511in}}%
\pgfpathclose%
\pgfusepath{stroke,fill}%
\end{pgfscope}%
\begin{pgfscope}%
\pgfpathrectangle{\pgfqpoint{0.100000in}{0.220728in}}{\pgfqpoint{3.696000in}{3.696000in}}%
\pgfusepath{clip}%
\pgfsetbuttcap%
\pgfsetroundjoin%
\definecolor{currentfill}{rgb}{0.121569,0.466667,0.705882}%
\pgfsetfillcolor{currentfill}%
\pgfsetfillopacity{0.320754}%
\pgfsetlinewidth{1.003750pt}%
\definecolor{currentstroke}{rgb}{0.121569,0.466667,0.705882}%
\pgfsetstrokecolor{currentstroke}%
\pgfsetstrokeopacity{0.320754}%
\pgfsetdash{}{0pt}%
\pgfpathmoveto{\pgfqpoint{1.846310in}{3.258468in}}%
\pgfpathcurveto{\pgfqpoint{1.854546in}{3.258468in}}{\pgfqpoint{1.862446in}{3.261740in}}{\pgfqpoint{1.868270in}{3.267564in}}%
\pgfpathcurveto{\pgfqpoint{1.874094in}{3.273388in}}{\pgfqpoint{1.877367in}{3.281288in}}{\pgfqpoint{1.877367in}{3.289524in}}%
\pgfpathcurveto{\pgfqpoint{1.877367in}{3.297760in}}{\pgfqpoint{1.874094in}{3.305660in}}{\pgfqpoint{1.868270in}{3.311484in}}%
\pgfpathcurveto{\pgfqpoint{1.862446in}{3.317308in}}{\pgfqpoint{1.854546in}{3.320581in}}{\pgfqpoint{1.846310in}{3.320581in}}%
\pgfpathcurveto{\pgfqpoint{1.838074in}{3.320581in}}{\pgfqpoint{1.830174in}{3.317308in}}{\pgfqpoint{1.824350in}{3.311484in}}%
\pgfpathcurveto{\pgfqpoint{1.818526in}{3.305660in}}{\pgfqpoint{1.815254in}{3.297760in}}{\pgfqpoint{1.815254in}{3.289524in}}%
\pgfpathcurveto{\pgfqpoint{1.815254in}{3.281288in}}{\pgfqpoint{1.818526in}{3.273388in}}{\pgfqpoint{1.824350in}{3.267564in}}%
\pgfpathcurveto{\pgfqpoint{1.830174in}{3.261740in}}{\pgfqpoint{1.838074in}{3.258468in}}{\pgfqpoint{1.846310in}{3.258468in}}%
\pgfpathclose%
\pgfusepath{stroke,fill}%
\end{pgfscope}%
\begin{pgfscope}%
\pgfpathrectangle{\pgfqpoint{0.100000in}{0.220728in}}{\pgfqpoint{3.696000in}{3.696000in}}%
\pgfusepath{clip}%
\pgfsetbuttcap%
\pgfsetroundjoin%
\definecolor{currentfill}{rgb}{0.121569,0.466667,0.705882}%
\pgfsetfillcolor{currentfill}%
\pgfsetfillopacity{0.320836}%
\pgfsetlinewidth{1.003750pt}%
\definecolor{currentstroke}{rgb}{0.121569,0.466667,0.705882}%
\pgfsetstrokecolor{currentstroke}%
\pgfsetstrokeopacity{0.320836}%
\pgfsetdash{}{0pt}%
\pgfpathmoveto{\pgfqpoint{1.713875in}{3.124748in}}%
\pgfpathcurveto{\pgfqpoint{1.722111in}{3.124748in}}{\pgfqpoint{1.730011in}{3.128021in}}{\pgfqpoint{1.735835in}{3.133845in}}%
\pgfpathcurveto{\pgfqpoint{1.741659in}{3.139669in}}{\pgfqpoint{1.744931in}{3.147569in}}{\pgfqpoint{1.744931in}{3.155805in}}%
\pgfpathcurveto{\pgfqpoint{1.744931in}{3.164041in}}{\pgfqpoint{1.741659in}{3.171941in}}{\pgfqpoint{1.735835in}{3.177765in}}%
\pgfpathcurveto{\pgfqpoint{1.730011in}{3.183589in}}{\pgfqpoint{1.722111in}{3.186861in}}{\pgfqpoint{1.713875in}{3.186861in}}%
\pgfpathcurveto{\pgfqpoint{1.705639in}{3.186861in}}{\pgfqpoint{1.697739in}{3.183589in}}{\pgfqpoint{1.691915in}{3.177765in}}%
\pgfpathcurveto{\pgfqpoint{1.686091in}{3.171941in}}{\pgfqpoint{1.682818in}{3.164041in}}{\pgfqpoint{1.682818in}{3.155805in}}%
\pgfpathcurveto{\pgfqpoint{1.682818in}{3.147569in}}{\pgfqpoint{1.686091in}{3.139669in}}{\pgfqpoint{1.691915in}{3.133845in}}%
\pgfpathcurveto{\pgfqpoint{1.697739in}{3.128021in}}{\pgfqpoint{1.705639in}{3.124748in}}{\pgfqpoint{1.713875in}{3.124748in}}%
\pgfpathclose%
\pgfusepath{stroke,fill}%
\end{pgfscope}%
\begin{pgfscope}%
\pgfpathrectangle{\pgfqpoint{0.100000in}{0.220728in}}{\pgfqpoint{3.696000in}{3.696000in}}%
\pgfusepath{clip}%
\pgfsetbuttcap%
\pgfsetroundjoin%
\definecolor{currentfill}{rgb}{0.121569,0.466667,0.705882}%
\pgfsetfillcolor{currentfill}%
\pgfsetfillopacity{0.321412}%
\pgfsetlinewidth{1.003750pt}%
\definecolor{currentstroke}{rgb}{0.121569,0.466667,0.705882}%
\pgfsetstrokecolor{currentstroke}%
\pgfsetstrokeopacity{0.321412}%
\pgfsetdash{}{0pt}%
\pgfpathmoveto{\pgfqpoint{1.711421in}{3.120919in}}%
\pgfpathcurveto{\pgfqpoint{1.719657in}{3.120919in}}{\pgfqpoint{1.727557in}{3.124191in}}{\pgfqpoint{1.733381in}{3.130015in}}%
\pgfpathcurveto{\pgfqpoint{1.739205in}{3.135839in}}{\pgfqpoint{1.742477in}{3.143739in}}{\pgfqpoint{1.742477in}{3.151975in}}%
\pgfpathcurveto{\pgfqpoint{1.742477in}{3.160212in}}{\pgfqpoint{1.739205in}{3.168112in}}{\pgfqpoint{1.733381in}{3.173936in}}%
\pgfpathcurveto{\pgfqpoint{1.727557in}{3.179760in}}{\pgfqpoint{1.719657in}{3.183032in}}{\pgfqpoint{1.711421in}{3.183032in}}%
\pgfpathcurveto{\pgfqpoint{1.703184in}{3.183032in}}{\pgfqpoint{1.695284in}{3.179760in}}{\pgfqpoint{1.689460in}{3.173936in}}%
\pgfpathcurveto{\pgfqpoint{1.683637in}{3.168112in}}{\pgfqpoint{1.680364in}{3.160212in}}{\pgfqpoint{1.680364in}{3.151975in}}%
\pgfpathcurveto{\pgfqpoint{1.680364in}{3.143739in}}{\pgfqpoint{1.683637in}{3.135839in}}{\pgfqpoint{1.689460in}{3.130015in}}%
\pgfpathcurveto{\pgfqpoint{1.695284in}{3.124191in}}{\pgfqpoint{1.703184in}{3.120919in}}{\pgfqpoint{1.711421in}{3.120919in}}%
\pgfpathclose%
\pgfusepath{stroke,fill}%
\end{pgfscope}%
\begin{pgfscope}%
\pgfpathrectangle{\pgfqpoint{0.100000in}{0.220728in}}{\pgfqpoint{3.696000in}{3.696000in}}%
\pgfusepath{clip}%
\pgfsetbuttcap%
\pgfsetroundjoin%
\definecolor{currentfill}{rgb}{0.121569,0.466667,0.705882}%
\pgfsetfillcolor{currentfill}%
\pgfsetfillopacity{0.322022}%
\pgfsetlinewidth{1.003750pt}%
\definecolor{currentstroke}{rgb}{0.121569,0.466667,0.705882}%
\pgfsetstrokecolor{currentstroke}%
\pgfsetstrokeopacity{0.322022}%
\pgfsetdash{}{0pt}%
\pgfpathmoveto{\pgfqpoint{1.710692in}{3.116952in}}%
\pgfpathcurveto{\pgfqpoint{1.718928in}{3.116952in}}{\pgfqpoint{1.726828in}{3.120224in}}{\pgfqpoint{1.732652in}{3.126048in}}%
\pgfpathcurveto{\pgfqpoint{1.738476in}{3.131872in}}{\pgfqpoint{1.741749in}{3.139772in}}{\pgfqpoint{1.741749in}{3.148009in}}%
\pgfpathcurveto{\pgfqpoint{1.741749in}{3.156245in}}{\pgfqpoint{1.738476in}{3.164145in}}{\pgfqpoint{1.732652in}{3.169969in}}%
\pgfpathcurveto{\pgfqpoint{1.726828in}{3.175793in}}{\pgfqpoint{1.718928in}{3.179065in}}{\pgfqpoint{1.710692in}{3.179065in}}%
\pgfpathcurveto{\pgfqpoint{1.702456in}{3.179065in}}{\pgfqpoint{1.694556in}{3.175793in}}{\pgfqpoint{1.688732in}{3.169969in}}%
\pgfpathcurveto{\pgfqpoint{1.682908in}{3.164145in}}{\pgfqpoint{1.679636in}{3.156245in}}{\pgfqpoint{1.679636in}{3.148009in}}%
\pgfpathcurveto{\pgfqpoint{1.679636in}{3.139772in}}{\pgfqpoint{1.682908in}{3.131872in}}{\pgfqpoint{1.688732in}{3.126048in}}%
\pgfpathcurveto{\pgfqpoint{1.694556in}{3.120224in}}{\pgfqpoint{1.702456in}{3.116952in}}{\pgfqpoint{1.710692in}{3.116952in}}%
\pgfpathclose%
\pgfusepath{stroke,fill}%
\end{pgfscope}%
\begin{pgfscope}%
\pgfpathrectangle{\pgfqpoint{0.100000in}{0.220728in}}{\pgfqpoint{3.696000in}{3.696000in}}%
\pgfusepath{clip}%
\pgfsetbuttcap%
\pgfsetroundjoin%
\definecolor{currentfill}{rgb}{0.121569,0.466667,0.705882}%
\pgfsetfillcolor{currentfill}%
\pgfsetfillopacity{0.322209}%
\pgfsetlinewidth{1.003750pt}%
\definecolor{currentstroke}{rgb}{0.121569,0.466667,0.705882}%
\pgfsetstrokecolor{currentstroke}%
\pgfsetstrokeopacity{0.322209}%
\pgfsetdash{}{0pt}%
\pgfpathmoveto{\pgfqpoint{1.709796in}{3.115647in}}%
\pgfpathcurveto{\pgfqpoint{1.718033in}{3.115647in}}{\pgfqpoint{1.725933in}{3.118919in}}{\pgfqpoint{1.731757in}{3.124743in}}%
\pgfpathcurveto{\pgfqpoint{1.737581in}{3.130567in}}{\pgfqpoint{1.740853in}{3.138467in}}{\pgfqpoint{1.740853in}{3.146703in}}%
\pgfpathcurveto{\pgfqpoint{1.740853in}{3.154940in}}{\pgfqpoint{1.737581in}{3.162840in}}{\pgfqpoint{1.731757in}{3.168664in}}%
\pgfpathcurveto{\pgfqpoint{1.725933in}{3.174488in}}{\pgfqpoint{1.718033in}{3.177760in}}{\pgfqpoint{1.709796in}{3.177760in}}%
\pgfpathcurveto{\pgfqpoint{1.701560in}{3.177760in}}{\pgfqpoint{1.693660in}{3.174488in}}{\pgfqpoint{1.687836in}{3.168664in}}%
\pgfpathcurveto{\pgfqpoint{1.682012in}{3.162840in}}{\pgfqpoint{1.678740in}{3.154940in}}{\pgfqpoint{1.678740in}{3.146703in}}%
\pgfpathcurveto{\pgfqpoint{1.678740in}{3.138467in}}{\pgfqpoint{1.682012in}{3.130567in}}{\pgfqpoint{1.687836in}{3.124743in}}%
\pgfpathcurveto{\pgfqpoint{1.693660in}{3.118919in}}{\pgfqpoint{1.701560in}{3.115647in}}{\pgfqpoint{1.709796in}{3.115647in}}%
\pgfpathclose%
\pgfusepath{stroke,fill}%
\end{pgfscope}%
\begin{pgfscope}%
\pgfpathrectangle{\pgfqpoint{0.100000in}{0.220728in}}{\pgfqpoint{3.696000in}{3.696000in}}%
\pgfusepath{clip}%
\pgfsetbuttcap%
\pgfsetroundjoin%
\definecolor{currentfill}{rgb}{0.121569,0.466667,0.705882}%
\pgfsetfillcolor{currentfill}%
\pgfsetfillopacity{0.322362}%
\pgfsetlinewidth{1.003750pt}%
\definecolor{currentstroke}{rgb}{0.121569,0.466667,0.705882}%
\pgfsetstrokecolor{currentstroke}%
\pgfsetstrokeopacity{0.322362}%
\pgfsetdash{}{0pt}%
\pgfpathmoveto{\pgfqpoint{1.709486in}{3.114808in}}%
\pgfpathcurveto{\pgfqpoint{1.717723in}{3.114808in}}{\pgfqpoint{1.725623in}{3.118080in}}{\pgfqpoint{1.731447in}{3.123904in}}%
\pgfpathcurveto{\pgfqpoint{1.737271in}{3.129728in}}{\pgfqpoint{1.740543in}{3.137628in}}{\pgfqpoint{1.740543in}{3.145865in}}%
\pgfpathcurveto{\pgfqpoint{1.740543in}{3.154101in}}{\pgfqpoint{1.737271in}{3.162001in}}{\pgfqpoint{1.731447in}{3.167825in}}%
\pgfpathcurveto{\pgfqpoint{1.725623in}{3.173649in}}{\pgfqpoint{1.717723in}{3.176921in}}{\pgfqpoint{1.709486in}{3.176921in}}%
\pgfpathcurveto{\pgfqpoint{1.701250in}{3.176921in}}{\pgfqpoint{1.693350in}{3.173649in}}{\pgfqpoint{1.687526in}{3.167825in}}%
\pgfpathcurveto{\pgfqpoint{1.681702in}{3.162001in}}{\pgfqpoint{1.678430in}{3.154101in}}{\pgfqpoint{1.678430in}{3.145865in}}%
\pgfpathcurveto{\pgfqpoint{1.678430in}{3.137628in}}{\pgfqpoint{1.681702in}{3.129728in}}{\pgfqpoint{1.687526in}{3.123904in}}%
\pgfpathcurveto{\pgfqpoint{1.693350in}{3.118080in}}{\pgfqpoint{1.701250in}{3.114808in}}{\pgfqpoint{1.709486in}{3.114808in}}%
\pgfpathclose%
\pgfusepath{stroke,fill}%
\end{pgfscope}%
\begin{pgfscope}%
\pgfpathrectangle{\pgfqpoint{0.100000in}{0.220728in}}{\pgfqpoint{3.696000in}{3.696000in}}%
\pgfusepath{clip}%
\pgfsetbuttcap%
\pgfsetroundjoin%
\definecolor{currentfill}{rgb}{0.121569,0.466667,0.705882}%
\pgfsetfillcolor{currentfill}%
\pgfsetfillopacity{0.322388}%
\pgfsetlinewidth{1.003750pt}%
\definecolor{currentstroke}{rgb}{0.121569,0.466667,0.705882}%
\pgfsetstrokecolor{currentstroke}%
\pgfsetstrokeopacity{0.322388}%
\pgfsetdash{}{0pt}%
\pgfpathmoveto{\pgfqpoint{1.709435in}{3.114654in}}%
\pgfpathcurveto{\pgfqpoint{1.717671in}{3.114654in}}{\pgfqpoint{1.725572in}{3.117927in}}{\pgfqpoint{1.731395in}{3.123751in}}%
\pgfpathcurveto{\pgfqpoint{1.737219in}{3.129575in}}{\pgfqpoint{1.740492in}{3.137475in}}{\pgfqpoint{1.740492in}{3.145711in}}%
\pgfpathcurveto{\pgfqpoint{1.740492in}{3.153947in}}{\pgfqpoint{1.737219in}{3.161847in}}{\pgfqpoint{1.731395in}{3.167671in}}%
\pgfpathcurveto{\pgfqpoint{1.725572in}{3.173495in}}{\pgfqpoint{1.717671in}{3.176767in}}{\pgfqpoint{1.709435in}{3.176767in}}%
\pgfpathcurveto{\pgfqpoint{1.701199in}{3.176767in}}{\pgfqpoint{1.693299in}{3.173495in}}{\pgfqpoint{1.687475in}{3.167671in}}%
\pgfpathcurveto{\pgfqpoint{1.681651in}{3.161847in}}{\pgfqpoint{1.678379in}{3.153947in}}{\pgfqpoint{1.678379in}{3.145711in}}%
\pgfpathcurveto{\pgfqpoint{1.678379in}{3.137475in}}{\pgfqpoint{1.681651in}{3.129575in}}{\pgfqpoint{1.687475in}{3.123751in}}%
\pgfpathcurveto{\pgfqpoint{1.693299in}{3.117927in}}{\pgfqpoint{1.701199in}{3.114654in}}{\pgfqpoint{1.709435in}{3.114654in}}%
\pgfpathclose%
\pgfusepath{stroke,fill}%
\end{pgfscope}%
\begin{pgfscope}%
\pgfpathrectangle{\pgfqpoint{0.100000in}{0.220728in}}{\pgfqpoint{3.696000in}{3.696000in}}%
\pgfusepath{clip}%
\pgfsetbuttcap%
\pgfsetroundjoin%
\definecolor{currentfill}{rgb}{0.121569,0.466667,0.705882}%
\pgfsetfillcolor{currentfill}%
\pgfsetfillopacity{0.322433}%
\pgfsetlinewidth{1.003750pt}%
\definecolor{currentstroke}{rgb}{0.121569,0.466667,0.705882}%
\pgfsetstrokecolor{currentstroke}%
\pgfsetstrokeopacity{0.322433}%
\pgfsetdash{}{0pt}%
\pgfpathmoveto{\pgfqpoint{1.709319in}{3.114391in}}%
\pgfpathcurveto{\pgfqpoint{1.717555in}{3.114391in}}{\pgfqpoint{1.725455in}{3.117664in}}{\pgfqpoint{1.731279in}{3.123488in}}%
\pgfpathcurveto{\pgfqpoint{1.737103in}{3.129311in}}{\pgfqpoint{1.740375in}{3.137212in}}{\pgfqpoint{1.740375in}{3.145448in}}%
\pgfpathcurveto{\pgfqpoint{1.740375in}{3.153684in}}{\pgfqpoint{1.737103in}{3.161584in}}{\pgfqpoint{1.731279in}{3.167408in}}%
\pgfpathcurveto{\pgfqpoint{1.725455in}{3.173232in}}{\pgfqpoint{1.717555in}{3.176504in}}{\pgfqpoint{1.709319in}{3.176504in}}%
\pgfpathcurveto{\pgfqpoint{1.701082in}{3.176504in}}{\pgfqpoint{1.693182in}{3.173232in}}{\pgfqpoint{1.687359in}{3.167408in}}%
\pgfpathcurveto{\pgfqpoint{1.681535in}{3.161584in}}{\pgfqpoint{1.678262in}{3.153684in}}{\pgfqpoint{1.678262in}{3.145448in}}%
\pgfpathcurveto{\pgfqpoint{1.678262in}{3.137212in}}{\pgfqpoint{1.681535in}{3.129311in}}{\pgfqpoint{1.687359in}{3.123488in}}%
\pgfpathcurveto{\pgfqpoint{1.693182in}{3.117664in}}{\pgfqpoint{1.701082in}{3.114391in}}{\pgfqpoint{1.709319in}{3.114391in}}%
\pgfpathclose%
\pgfusepath{stroke,fill}%
\end{pgfscope}%
\begin{pgfscope}%
\pgfpathrectangle{\pgfqpoint{0.100000in}{0.220728in}}{\pgfqpoint{3.696000in}{3.696000in}}%
\pgfusepath{clip}%
\pgfsetbuttcap%
\pgfsetroundjoin%
\definecolor{currentfill}{rgb}{0.121569,0.466667,0.705882}%
\pgfsetfillcolor{currentfill}%
\pgfsetfillopacity{0.322512}%
\pgfsetlinewidth{1.003750pt}%
\definecolor{currentstroke}{rgb}{0.121569,0.466667,0.705882}%
\pgfsetstrokecolor{currentstroke}%
\pgfsetstrokeopacity{0.322512}%
\pgfsetdash{}{0pt}%
\pgfpathmoveto{\pgfqpoint{1.709120in}{3.113891in}}%
\pgfpathcurveto{\pgfqpoint{1.717356in}{3.113891in}}{\pgfqpoint{1.725256in}{3.117163in}}{\pgfqpoint{1.731080in}{3.122987in}}%
\pgfpathcurveto{\pgfqpoint{1.736904in}{3.128811in}}{\pgfqpoint{1.740176in}{3.136711in}}{\pgfqpoint{1.740176in}{3.144947in}}%
\pgfpathcurveto{\pgfqpoint{1.740176in}{3.153184in}}{\pgfqpoint{1.736904in}{3.161084in}}{\pgfqpoint{1.731080in}{3.166908in}}%
\pgfpathcurveto{\pgfqpoint{1.725256in}{3.172732in}}{\pgfqpoint{1.717356in}{3.176004in}}{\pgfqpoint{1.709120in}{3.176004in}}%
\pgfpathcurveto{\pgfqpoint{1.700883in}{3.176004in}}{\pgfqpoint{1.692983in}{3.172732in}}{\pgfqpoint{1.687159in}{3.166908in}}%
\pgfpathcurveto{\pgfqpoint{1.681335in}{3.161084in}}{\pgfqpoint{1.678063in}{3.153184in}}{\pgfqpoint{1.678063in}{3.144947in}}%
\pgfpathcurveto{\pgfqpoint{1.678063in}{3.136711in}}{\pgfqpoint{1.681335in}{3.128811in}}{\pgfqpoint{1.687159in}{3.122987in}}%
\pgfpathcurveto{\pgfqpoint{1.692983in}{3.117163in}}{\pgfqpoint{1.700883in}{3.113891in}}{\pgfqpoint{1.709120in}{3.113891in}}%
\pgfpathclose%
\pgfusepath{stroke,fill}%
\end{pgfscope}%
\begin{pgfscope}%
\pgfpathrectangle{\pgfqpoint{0.100000in}{0.220728in}}{\pgfqpoint{3.696000in}{3.696000in}}%
\pgfusepath{clip}%
\pgfsetbuttcap%
\pgfsetroundjoin%
\definecolor{currentfill}{rgb}{0.121569,0.466667,0.705882}%
\pgfsetfillcolor{currentfill}%
\pgfsetfillopacity{0.322611}%
\pgfsetlinewidth{1.003750pt}%
\definecolor{currentstroke}{rgb}{0.121569,0.466667,0.705882}%
\pgfsetstrokecolor{currentstroke}%
\pgfsetstrokeopacity{0.322611}%
\pgfsetdash{}{0pt}%
\pgfpathmoveto{\pgfqpoint{1.856269in}{3.256420in}}%
\pgfpathcurveto{\pgfqpoint{1.864505in}{3.256420in}}{\pgfqpoint{1.872405in}{3.259692in}}{\pgfqpoint{1.878229in}{3.265516in}}%
\pgfpathcurveto{\pgfqpoint{1.884053in}{3.271340in}}{\pgfqpoint{1.887325in}{3.279240in}}{\pgfqpoint{1.887325in}{3.287476in}}%
\pgfpathcurveto{\pgfqpoint{1.887325in}{3.295713in}}{\pgfqpoint{1.884053in}{3.303613in}}{\pgfqpoint{1.878229in}{3.309437in}}%
\pgfpathcurveto{\pgfqpoint{1.872405in}{3.315261in}}{\pgfqpoint{1.864505in}{3.318533in}}{\pgfqpoint{1.856269in}{3.318533in}}%
\pgfpathcurveto{\pgfqpoint{1.848032in}{3.318533in}}{\pgfqpoint{1.840132in}{3.315261in}}{\pgfqpoint{1.834308in}{3.309437in}}%
\pgfpathcurveto{\pgfqpoint{1.828484in}{3.303613in}}{\pgfqpoint{1.825212in}{3.295713in}}{\pgfqpoint{1.825212in}{3.287476in}}%
\pgfpathcurveto{\pgfqpoint{1.825212in}{3.279240in}}{\pgfqpoint{1.828484in}{3.271340in}}{\pgfqpoint{1.834308in}{3.265516in}}%
\pgfpathcurveto{\pgfqpoint{1.840132in}{3.259692in}}{\pgfqpoint{1.848032in}{3.256420in}}{\pgfqpoint{1.856269in}{3.256420in}}%
\pgfpathclose%
\pgfusepath{stroke,fill}%
\end{pgfscope}%
\begin{pgfscope}%
\pgfpathrectangle{\pgfqpoint{0.100000in}{0.220728in}}{\pgfqpoint{3.696000in}{3.696000in}}%
\pgfusepath{clip}%
\pgfsetbuttcap%
\pgfsetroundjoin%
\definecolor{currentfill}{rgb}{0.121569,0.466667,0.705882}%
\pgfsetfillcolor{currentfill}%
\pgfsetfillopacity{0.322668}%
\pgfsetlinewidth{1.003750pt}%
\definecolor{currentstroke}{rgb}{0.121569,0.466667,0.705882}%
\pgfsetstrokecolor{currentstroke}%
\pgfsetstrokeopacity{0.322668}%
\pgfsetdash{}{0pt}%
\pgfpathmoveto{\pgfqpoint{1.708751in}{3.113036in}}%
\pgfpathcurveto{\pgfqpoint{1.716987in}{3.113036in}}{\pgfqpoint{1.724887in}{3.116308in}}{\pgfqpoint{1.730711in}{3.122132in}}%
\pgfpathcurveto{\pgfqpoint{1.736535in}{3.127956in}}{\pgfqpoint{1.739807in}{3.135856in}}{\pgfqpoint{1.739807in}{3.144092in}}%
\pgfpathcurveto{\pgfqpoint{1.739807in}{3.152328in}}{\pgfqpoint{1.736535in}{3.160228in}}{\pgfqpoint{1.730711in}{3.166052in}}%
\pgfpathcurveto{\pgfqpoint{1.724887in}{3.171876in}}{\pgfqpoint{1.716987in}{3.175149in}}{\pgfqpoint{1.708751in}{3.175149in}}%
\pgfpathcurveto{\pgfqpoint{1.700514in}{3.175149in}}{\pgfqpoint{1.692614in}{3.171876in}}{\pgfqpoint{1.686790in}{3.166052in}}%
\pgfpathcurveto{\pgfqpoint{1.680966in}{3.160228in}}{\pgfqpoint{1.677694in}{3.152328in}}{\pgfqpoint{1.677694in}{3.144092in}}%
\pgfpathcurveto{\pgfqpoint{1.677694in}{3.135856in}}{\pgfqpoint{1.680966in}{3.127956in}}{\pgfqpoint{1.686790in}{3.122132in}}%
\pgfpathcurveto{\pgfqpoint{1.692614in}{3.116308in}}{\pgfqpoint{1.700514in}{3.113036in}}{\pgfqpoint{1.708751in}{3.113036in}}%
\pgfpathclose%
\pgfusepath{stroke,fill}%
\end{pgfscope}%
\begin{pgfscope}%
\pgfpathrectangle{\pgfqpoint{0.100000in}{0.220728in}}{\pgfqpoint{3.696000in}{3.696000in}}%
\pgfusepath{clip}%
\pgfsetbuttcap%
\pgfsetroundjoin%
\definecolor{currentfill}{rgb}{0.121569,0.466667,0.705882}%
\pgfsetfillcolor{currentfill}%
\pgfsetfillopacity{0.322907}%
\pgfsetlinewidth{1.003750pt}%
\definecolor{currentstroke}{rgb}{0.121569,0.466667,0.705882}%
\pgfsetstrokecolor{currentstroke}%
\pgfsetstrokeopacity{0.322907}%
\pgfsetdash{}{0pt}%
\pgfpathmoveto{\pgfqpoint{1.707988in}{3.111380in}}%
\pgfpathcurveto{\pgfqpoint{1.716224in}{3.111380in}}{\pgfqpoint{1.724124in}{3.114653in}}{\pgfqpoint{1.729948in}{3.120477in}}%
\pgfpathcurveto{\pgfqpoint{1.735772in}{3.126300in}}{\pgfqpoint{1.739044in}{3.134201in}}{\pgfqpoint{1.739044in}{3.142437in}}%
\pgfpathcurveto{\pgfqpoint{1.739044in}{3.150673in}}{\pgfqpoint{1.735772in}{3.158573in}}{\pgfqpoint{1.729948in}{3.164397in}}%
\pgfpathcurveto{\pgfqpoint{1.724124in}{3.170221in}}{\pgfqpoint{1.716224in}{3.173493in}}{\pgfqpoint{1.707988in}{3.173493in}}%
\pgfpathcurveto{\pgfqpoint{1.699752in}{3.173493in}}{\pgfqpoint{1.691852in}{3.170221in}}{\pgfqpoint{1.686028in}{3.164397in}}%
\pgfpathcurveto{\pgfqpoint{1.680204in}{3.158573in}}{\pgfqpoint{1.676931in}{3.150673in}}{\pgfqpoint{1.676931in}{3.142437in}}%
\pgfpathcurveto{\pgfqpoint{1.676931in}{3.134201in}}{\pgfqpoint{1.680204in}{3.126300in}}{\pgfqpoint{1.686028in}{3.120477in}}%
\pgfpathcurveto{\pgfqpoint{1.691852in}{3.114653in}}{\pgfqpoint{1.699752in}{3.111380in}}{\pgfqpoint{1.707988in}{3.111380in}}%
\pgfpathclose%
\pgfusepath{stroke,fill}%
\end{pgfscope}%
\begin{pgfscope}%
\pgfpathrectangle{\pgfqpoint{0.100000in}{0.220728in}}{\pgfqpoint{3.696000in}{3.696000in}}%
\pgfusepath{clip}%
\pgfsetbuttcap%
\pgfsetroundjoin%
\definecolor{currentfill}{rgb}{0.121569,0.466667,0.705882}%
\pgfsetfillcolor{currentfill}%
\pgfsetfillopacity{0.323404}%
\pgfsetlinewidth{1.003750pt}%
\definecolor{currentstroke}{rgb}{0.121569,0.466667,0.705882}%
\pgfsetstrokecolor{currentstroke}%
\pgfsetstrokeopacity{0.323404}%
\pgfsetdash{}{0pt}%
\pgfpathmoveto{\pgfqpoint{1.707579in}{3.108068in}}%
\pgfpathcurveto{\pgfqpoint{1.715816in}{3.108068in}}{\pgfqpoint{1.723716in}{3.111340in}}{\pgfqpoint{1.729540in}{3.117164in}}%
\pgfpathcurveto{\pgfqpoint{1.735363in}{3.122988in}}{\pgfqpoint{1.738636in}{3.130888in}}{\pgfqpoint{1.738636in}{3.139124in}}%
\pgfpathcurveto{\pgfqpoint{1.738636in}{3.147361in}}{\pgfqpoint{1.735363in}{3.155261in}}{\pgfqpoint{1.729540in}{3.161085in}}%
\pgfpathcurveto{\pgfqpoint{1.723716in}{3.166909in}}{\pgfqpoint{1.715816in}{3.170181in}}{\pgfqpoint{1.707579in}{3.170181in}}%
\pgfpathcurveto{\pgfqpoint{1.699343in}{3.170181in}}{\pgfqpoint{1.691443in}{3.166909in}}{\pgfqpoint{1.685619in}{3.161085in}}%
\pgfpathcurveto{\pgfqpoint{1.679795in}{3.155261in}}{\pgfqpoint{1.676523in}{3.147361in}}{\pgfqpoint{1.676523in}{3.139124in}}%
\pgfpathcurveto{\pgfqpoint{1.676523in}{3.130888in}}{\pgfqpoint{1.679795in}{3.122988in}}{\pgfqpoint{1.685619in}{3.117164in}}%
\pgfpathcurveto{\pgfqpoint{1.691443in}{3.111340in}}{\pgfqpoint{1.699343in}{3.108068in}}{\pgfqpoint{1.707579in}{3.108068in}}%
\pgfpathclose%
\pgfusepath{stroke,fill}%
\end{pgfscope}%
\begin{pgfscope}%
\pgfpathrectangle{\pgfqpoint{0.100000in}{0.220728in}}{\pgfqpoint{3.696000in}{3.696000in}}%
\pgfusepath{clip}%
\pgfsetbuttcap%
\pgfsetroundjoin%
\definecolor{currentfill}{rgb}{0.121569,0.466667,0.705882}%
\pgfsetfillcolor{currentfill}%
\pgfsetfillopacity{0.323615}%
\pgfsetlinewidth{1.003750pt}%
\definecolor{currentstroke}{rgb}{0.121569,0.466667,0.705882}%
\pgfsetstrokecolor{currentstroke}%
\pgfsetstrokeopacity{0.323615}%
\pgfsetdash{}{0pt}%
\pgfpathmoveto{\pgfqpoint{1.706551in}{3.106467in}}%
\pgfpathcurveto{\pgfqpoint{1.714787in}{3.106467in}}{\pgfqpoint{1.722687in}{3.109739in}}{\pgfqpoint{1.728511in}{3.115563in}}%
\pgfpathcurveto{\pgfqpoint{1.734335in}{3.121387in}}{\pgfqpoint{1.737607in}{3.129287in}}{\pgfqpoint{1.737607in}{3.137523in}}%
\pgfpathcurveto{\pgfqpoint{1.737607in}{3.145759in}}{\pgfqpoint{1.734335in}{3.153659in}}{\pgfqpoint{1.728511in}{3.159483in}}%
\pgfpathcurveto{\pgfqpoint{1.722687in}{3.165307in}}{\pgfqpoint{1.714787in}{3.168580in}}{\pgfqpoint{1.706551in}{3.168580in}}%
\pgfpathcurveto{\pgfqpoint{1.698315in}{3.168580in}}{\pgfqpoint{1.690414in}{3.165307in}}{\pgfqpoint{1.684591in}{3.159483in}}%
\pgfpathcurveto{\pgfqpoint{1.678767in}{3.153659in}}{\pgfqpoint{1.675494in}{3.145759in}}{\pgfqpoint{1.675494in}{3.137523in}}%
\pgfpathcurveto{\pgfqpoint{1.675494in}{3.129287in}}{\pgfqpoint{1.678767in}{3.121387in}}{\pgfqpoint{1.684591in}{3.115563in}}%
\pgfpathcurveto{\pgfqpoint{1.690414in}{3.109739in}}{\pgfqpoint{1.698315in}{3.106467in}}{\pgfqpoint{1.706551in}{3.106467in}}%
\pgfpathclose%
\pgfusepath{stroke,fill}%
\end{pgfscope}%
\begin{pgfscope}%
\pgfpathrectangle{\pgfqpoint{0.100000in}{0.220728in}}{\pgfqpoint{3.696000in}{3.696000in}}%
\pgfusepath{clip}%
\pgfsetbuttcap%
\pgfsetroundjoin%
\definecolor{currentfill}{rgb}{0.121569,0.466667,0.705882}%
\pgfsetfillcolor{currentfill}%
\pgfsetfillopacity{0.323775}%
\pgfsetlinewidth{1.003750pt}%
\definecolor{currentstroke}{rgb}{0.121569,0.466667,0.705882}%
\pgfsetstrokecolor{currentstroke}%
\pgfsetstrokeopacity{0.323775}%
\pgfsetdash{}{0pt}%
\pgfpathmoveto{\pgfqpoint{1.706292in}{3.105610in}}%
\pgfpathcurveto{\pgfqpoint{1.714529in}{3.105610in}}{\pgfqpoint{1.722429in}{3.108883in}}{\pgfqpoint{1.728252in}{3.114707in}}%
\pgfpathcurveto{\pgfqpoint{1.734076in}{3.120531in}}{\pgfqpoint{1.737349in}{3.128431in}}{\pgfqpoint{1.737349in}{3.136667in}}%
\pgfpathcurveto{\pgfqpoint{1.737349in}{3.144903in}}{\pgfqpoint{1.734076in}{3.152803in}}{\pgfqpoint{1.728252in}{3.158627in}}%
\pgfpathcurveto{\pgfqpoint{1.722429in}{3.164451in}}{\pgfqpoint{1.714529in}{3.167723in}}{\pgfqpoint{1.706292in}{3.167723in}}%
\pgfpathcurveto{\pgfqpoint{1.698056in}{3.167723in}}{\pgfqpoint{1.690156in}{3.164451in}}{\pgfqpoint{1.684332in}{3.158627in}}%
\pgfpathcurveto{\pgfqpoint{1.678508in}{3.152803in}}{\pgfqpoint{1.675236in}{3.144903in}}{\pgfqpoint{1.675236in}{3.136667in}}%
\pgfpathcurveto{\pgfqpoint{1.675236in}{3.128431in}}{\pgfqpoint{1.678508in}{3.120531in}}{\pgfqpoint{1.684332in}{3.114707in}}%
\pgfpathcurveto{\pgfqpoint{1.690156in}{3.108883in}}{\pgfqpoint{1.698056in}{3.105610in}}{\pgfqpoint{1.706292in}{3.105610in}}%
\pgfpathclose%
\pgfusepath{stroke,fill}%
\end{pgfscope}%
\begin{pgfscope}%
\pgfpathrectangle{\pgfqpoint{0.100000in}{0.220728in}}{\pgfqpoint{3.696000in}{3.696000in}}%
\pgfusepath{clip}%
\pgfsetbuttcap%
\pgfsetroundjoin%
\definecolor{currentfill}{rgb}{0.121569,0.466667,0.705882}%
\pgfsetfillcolor{currentfill}%
\pgfsetfillopacity{0.323982}%
\pgfsetlinewidth{1.003750pt}%
\definecolor{currentstroke}{rgb}{0.121569,0.466667,0.705882}%
\pgfsetstrokecolor{currentstroke}%
\pgfsetstrokeopacity{0.323982}%
\pgfsetdash{}{0pt}%
\pgfpathmoveto{\pgfqpoint{1.705375in}{3.104150in}}%
\pgfpathcurveto{\pgfqpoint{1.713612in}{3.104150in}}{\pgfqpoint{1.721512in}{3.107422in}}{\pgfqpoint{1.727336in}{3.113246in}}%
\pgfpathcurveto{\pgfqpoint{1.733160in}{3.119070in}}{\pgfqpoint{1.736432in}{3.126970in}}{\pgfqpoint{1.736432in}{3.135206in}}%
\pgfpathcurveto{\pgfqpoint{1.736432in}{3.143442in}}{\pgfqpoint{1.733160in}{3.151342in}}{\pgfqpoint{1.727336in}{3.157166in}}%
\pgfpathcurveto{\pgfqpoint{1.721512in}{3.162990in}}{\pgfqpoint{1.713612in}{3.166263in}}{\pgfqpoint{1.705375in}{3.166263in}}%
\pgfpathcurveto{\pgfqpoint{1.697139in}{3.166263in}}{\pgfqpoint{1.689239in}{3.162990in}}{\pgfqpoint{1.683415in}{3.157166in}}%
\pgfpathcurveto{\pgfqpoint{1.677591in}{3.151342in}}{\pgfqpoint{1.674319in}{3.143442in}}{\pgfqpoint{1.674319in}{3.135206in}}%
\pgfpathcurveto{\pgfqpoint{1.674319in}{3.126970in}}{\pgfqpoint{1.677591in}{3.119070in}}{\pgfqpoint{1.683415in}{3.113246in}}%
\pgfpathcurveto{\pgfqpoint{1.689239in}{3.107422in}}{\pgfqpoint{1.697139in}{3.104150in}}{\pgfqpoint{1.705375in}{3.104150in}}%
\pgfpathclose%
\pgfusepath{stroke,fill}%
\end{pgfscope}%
\begin{pgfscope}%
\pgfpathrectangle{\pgfqpoint{0.100000in}{0.220728in}}{\pgfqpoint{3.696000in}{3.696000in}}%
\pgfusepath{clip}%
\pgfsetbuttcap%
\pgfsetroundjoin%
\definecolor{currentfill}{rgb}{0.121569,0.466667,0.705882}%
\pgfsetfillcolor{currentfill}%
\pgfsetfillopacity{0.324410}%
\pgfsetlinewidth{1.003750pt}%
\definecolor{currentstroke}{rgb}{0.121569,0.466667,0.705882}%
\pgfsetstrokecolor{currentstroke}%
\pgfsetstrokeopacity{0.324410}%
\pgfsetdash{}{0pt}%
\pgfpathmoveto{\pgfqpoint{1.867144in}{3.253060in}}%
\pgfpathcurveto{\pgfqpoint{1.875381in}{3.253060in}}{\pgfqpoint{1.883281in}{3.256333in}}{\pgfqpoint{1.889105in}{3.262157in}}%
\pgfpathcurveto{\pgfqpoint{1.894928in}{3.267981in}}{\pgfqpoint{1.898201in}{3.275881in}}{\pgfqpoint{1.898201in}{3.284117in}}%
\pgfpathcurveto{\pgfqpoint{1.898201in}{3.292353in}}{\pgfqpoint{1.894928in}{3.300253in}}{\pgfqpoint{1.889105in}{3.306077in}}%
\pgfpathcurveto{\pgfqpoint{1.883281in}{3.311901in}}{\pgfqpoint{1.875381in}{3.315173in}}{\pgfqpoint{1.867144in}{3.315173in}}%
\pgfpathcurveto{\pgfqpoint{1.858908in}{3.315173in}}{\pgfqpoint{1.851008in}{3.311901in}}{\pgfqpoint{1.845184in}{3.306077in}}%
\pgfpathcurveto{\pgfqpoint{1.839360in}{3.300253in}}{\pgfqpoint{1.836088in}{3.292353in}}{\pgfqpoint{1.836088in}{3.284117in}}%
\pgfpathcurveto{\pgfqpoint{1.836088in}{3.275881in}}{\pgfqpoint{1.839360in}{3.267981in}}{\pgfqpoint{1.845184in}{3.262157in}}%
\pgfpathcurveto{\pgfqpoint{1.851008in}{3.256333in}}{\pgfqpoint{1.858908in}{3.253060in}}{\pgfqpoint{1.867144in}{3.253060in}}%
\pgfpathclose%
\pgfusepath{stroke,fill}%
\end{pgfscope}%
\begin{pgfscope}%
\pgfpathrectangle{\pgfqpoint{0.100000in}{0.220728in}}{\pgfqpoint{3.696000in}{3.696000in}}%
\pgfusepath{clip}%
\pgfsetbuttcap%
\pgfsetroundjoin%
\definecolor{currentfill}{rgb}{0.121569,0.466667,0.705882}%
\pgfsetfillcolor{currentfill}%
\pgfsetfillopacity{0.324507}%
\pgfsetlinewidth{1.003750pt}%
\definecolor{currentstroke}{rgb}{0.121569,0.466667,0.705882}%
\pgfsetstrokecolor{currentstroke}%
\pgfsetstrokeopacity{0.324507}%
\pgfsetdash{}{0pt}%
\pgfpathmoveto{\pgfqpoint{1.704445in}{3.101346in}}%
\pgfpathcurveto{\pgfqpoint{1.712681in}{3.101346in}}{\pgfqpoint{1.720581in}{3.104619in}}{\pgfqpoint{1.726405in}{3.110443in}}%
\pgfpathcurveto{\pgfqpoint{1.732229in}{3.116266in}}{\pgfqpoint{1.735501in}{3.124166in}}{\pgfqpoint{1.735501in}{3.132403in}}%
\pgfpathcurveto{\pgfqpoint{1.735501in}{3.140639in}}{\pgfqpoint{1.732229in}{3.148539in}}{\pgfqpoint{1.726405in}{3.154363in}}%
\pgfpathcurveto{\pgfqpoint{1.720581in}{3.160187in}}{\pgfqpoint{1.712681in}{3.163459in}}{\pgfqpoint{1.704445in}{3.163459in}}%
\pgfpathcurveto{\pgfqpoint{1.696208in}{3.163459in}}{\pgfqpoint{1.688308in}{3.160187in}}{\pgfqpoint{1.682484in}{3.154363in}}%
\pgfpathcurveto{\pgfqpoint{1.676660in}{3.148539in}}{\pgfqpoint{1.673388in}{3.140639in}}{\pgfqpoint{1.673388in}{3.132403in}}%
\pgfpathcurveto{\pgfqpoint{1.673388in}{3.124166in}}{\pgfqpoint{1.676660in}{3.116266in}}{\pgfqpoint{1.682484in}{3.110443in}}%
\pgfpathcurveto{\pgfqpoint{1.688308in}{3.104619in}}{\pgfqpoint{1.696208in}{3.101346in}}{\pgfqpoint{1.704445in}{3.101346in}}%
\pgfpathclose%
\pgfusepath{stroke,fill}%
\end{pgfscope}%
\begin{pgfscope}%
\pgfpathrectangle{\pgfqpoint{0.100000in}{0.220728in}}{\pgfqpoint{3.696000in}{3.696000in}}%
\pgfusepath{clip}%
\pgfsetbuttcap%
\pgfsetroundjoin%
\definecolor{currentfill}{rgb}{0.121569,0.466667,0.705882}%
\pgfsetfillcolor{currentfill}%
\pgfsetfillopacity{0.324789}%
\pgfsetlinewidth{1.003750pt}%
\definecolor{currentstroke}{rgb}{0.121569,0.466667,0.705882}%
\pgfsetstrokecolor{currentstroke}%
\pgfsetstrokeopacity{0.324789}%
\pgfsetdash{}{0pt}%
\pgfpathmoveto{\pgfqpoint{1.703282in}{3.099511in}}%
\pgfpathcurveto{\pgfqpoint{1.711518in}{3.099511in}}{\pgfqpoint{1.719418in}{3.102783in}}{\pgfqpoint{1.725242in}{3.108607in}}%
\pgfpathcurveto{\pgfqpoint{1.731066in}{3.114431in}}{\pgfqpoint{1.734339in}{3.122331in}}{\pgfqpoint{1.734339in}{3.130567in}}%
\pgfpathcurveto{\pgfqpoint{1.734339in}{3.138804in}}{\pgfqpoint{1.731066in}{3.146704in}}{\pgfqpoint{1.725242in}{3.152528in}}%
\pgfpathcurveto{\pgfqpoint{1.719418in}{3.158352in}}{\pgfqpoint{1.711518in}{3.161624in}}{\pgfqpoint{1.703282in}{3.161624in}}%
\pgfpathcurveto{\pgfqpoint{1.695046in}{3.161624in}}{\pgfqpoint{1.687146in}{3.158352in}}{\pgfqpoint{1.681322in}{3.152528in}}%
\pgfpathcurveto{\pgfqpoint{1.675498in}{3.146704in}}{\pgfqpoint{1.672226in}{3.138804in}}{\pgfqpoint{1.672226in}{3.130567in}}%
\pgfpathcurveto{\pgfqpoint{1.672226in}{3.122331in}}{\pgfqpoint{1.675498in}{3.114431in}}{\pgfqpoint{1.681322in}{3.108607in}}%
\pgfpathcurveto{\pgfqpoint{1.687146in}{3.102783in}}{\pgfqpoint{1.695046in}{3.099511in}}{\pgfqpoint{1.703282in}{3.099511in}}%
\pgfpathclose%
\pgfusepath{stroke,fill}%
\end{pgfscope}%
\begin{pgfscope}%
\pgfpathrectangle{\pgfqpoint{0.100000in}{0.220728in}}{\pgfqpoint{3.696000in}{3.696000in}}%
\pgfusepath{clip}%
\pgfsetbuttcap%
\pgfsetroundjoin%
\definecolor{currentfill}{rgb}{0.121569,0.466667,0.705882}%
\pgfsetfillcolor{currentfill}%
\pgfsetfillopacity{0.324881}%
\pgfsetlinewidth{1.003750pt}%
\definecolor{currentstroke}{rgb}{0.121569,0.466667,0.705882}%
\pgfsetstrokecolor{currentstroke}%
\pgfsetstrokeopacity{0.324881}%
\pgfsetdash{}{0pt}%
\pgfpathmoveto{\pgfqpoint{1.703000in}{3.098914in}}%
\pgfpathcurveto{\pgfqpoint{1.711237in}{3.098914in}}{\pgfqpoint{1.719137in}{3.102186in}}{\pgfqpoint{1.724961in}{3.108010in}}%
\pgfpathcurveto{\pgfqpoint{1.730784in}{3.113834in}}{\pgfqpoint{1.734057in}{3.121734in}}{\pgfqpoint{1.734057in}{3.129970in}}%
\pgfpathcurveto{\pgfqpoint{1.734057in}{3.138207in}}{\pgfqpoint{1.730784in}{3.146107in}}{\pgfqpoint{1.724961in}{3.151931in}}%
\pgfpathcurveto{\pgfqpoint{1.719137in}{3.157755in}}{\pgfqpoint{1.711237in}{3.161027in}}{\pgfqpoint{1.703000in}{3.161027in}}%
\pgfpathcurveto{\pgfqpoint{1.694764in}{3.161027in}}{\pgfqpoint{1.686864in}{3.157755in}}{\pgfqpoint{1.681040in}{3.151931in}}%
\pgfpathcurveto{\pgfqpoint{1.675216in}{3.146107in}}{\pgfqpoint{1.671944in}{3.138207in}}{\pgfqpoint{1.671944in}{3.129970in}}%
\pgfpathcurveto{\pgfqpoint{1.671944in}{3.121734in}}{\pgfqpoint{1.675216in}{3.113834in}}{\pgfqpoint{1.681040in}{3.108010in}}%
\pgfpathcurveto{\pgfqpoint{1.686864in}{3.102186in}}{\pgfqpoint{1.694764in}{3.098914in}}{\pgfqpoint{1.703000in}{3.098914in}}%
\pgfpathclose%
\pgfusepath{stroke,fill}%
\end{pgfscope}%
\begin{pgfscope}%
\pgfpathrectangle{\pgfqpoint{0.100000in}{0.220728in}}{\pgfqpoint{3.696000in}{3.696000in}}%
\pgfusepath{clip}%
\pgfsetbuttcap%
\pgfsetroundjoin%
\definecolor{currentfill}{rgb}{0.121569,0.466667,0.705882}%
\pgfsetfillcolor{currentfill}%
\pgfsetfillopacity{0.325065}%
\pgfsetlinewidth{1.003750pt}%
\definecolor{currentstroke}{rgb}{0.121569,0.466667,0.705882}%
\pgfsetstrokecolor{currentstroke}%
\pgfsetstrokeopacity{0.325065}%
\pgfsetdash{}{0pt}%
\pgfpathmoveto{\pgfqpoint{1.702606in}{3.097800in}}%
\pgfpathcurveto{\pgfqpoint{1.710842in}{3.097800in}}{\pgfqpoint{1.718742in}{3.101073in}}{\pgfqpoint{1.724566in}{3.106897in}}%
\pgfpathcurveto{\pgfqpoint{1.730390in}{3.112720in}}{\pgfqpoint{1.733662in}{3.120621in}}{\pgfqpoint{1.733662in}{3.128857in}}%
\pgfpathcurveto{\pgfqpoint{1.733662in}{3.137093in}}{\pgfqpoint{1.730390in}{3.144993in}}{\pgfqpoint{1.724566in}{3.150817in}}%
\pgfpathcurveto{\pgfqpoint{1.718742in}{3.156641in}}{\pgfqpoint{1.710842in}{3.159913in}}{\pgfqpoint{1.702606in}{3.159913in}}%
\pgfpathcurveto{\pgfqpoint{1.694369in}{3.159913in}}{\pgfqpoint{1.686469in}{3.156641in}}{\pgfqpoint{1.680645in}{3.150817in}}%
\pgfpathcurveto{\pgfqpoint{1.674821in}{3.144993in}}{\pgfqpoint{1.671549in}{3.137093in}}{\pgfqpoint{1.671549in}{3.128857in}}%
\pgfpathcurveto{\pgfqpoint{1.671549in}{3.120621in}}{\pgfqpoint{1.674821in}{3.112720in}}{\pgfqpoint{1.680645in}{3.106897in}}%
\pgfpathcurveto{\pgfqpoint{1.686469in}{3.101073in}}{\pgfqpoint{1.694369in}{3.097800in}}{\pgfqpoint{1.702606in}{3.097800in}}%
\pgfpathclose%
\pgfusepath{stroke,fill}%
\end{pgfscope}%
\begin{pgfscope}%
\pgfpathrectangle{\pgfqpoint{0.100000in}{0.220728in}}{\pgfqpoint{3.696000in}{3.696000in}}%
\pgfusepath{clip}%
\pgfsetbuttcap%
\pgfsetroundjoin%
\definecolor{currentfill}{rgb}{0.121569,0.466667,0.705882}%
\pgfsetfillcolor{currentfill}%
\pgfsetfillopacity{0.325119}%
\pgfsetlinewidth{1.003750pt}%
\definecolor{currentstroke}{rgb}{0.121569,0.466667,0.705882}%
\pgfsetstrokecolor{currentstroke}%
\pgfsetstrokeopacity{0.325119}%
\pgfsetdash{}{0pt}%
\pgfpathmoveto{\pgfqpoint{1.702451in}{3.097452in}}%
\pgfpathcurveto{\pgfqpoint{1.710687in}{3.097452in}}{\pgfqpoint{1.718587in}{3.100725in}}{\pgfqpoint{1.724411in}{3.106549in}}%
\pgfpathcurveto{\pgfqpoint{1.730235in}{3.112373in}}{\pgfqpoint{1.733507in}{3.120273in}}{\pgfqpoint{1.733507in}{3.128509in}}%
\pgfpathcurveto{\pgfqpoint{1.733507in}{3.136745in}}{\pgfqpoint{1.730235in}{3.144645in}}{\pgfqpoint{1.724411in}{3.150469in}}%
\pgfpathcurveto{\pgfqpoint{1.718587in}{3.156293in}}{\pgfqpoint{1.710687in}{3.159565in}}{\pgfqpoint{1.702451in}{3.159565in}}%
\pgfpathcurveto{\pgfqpoint{1.694215in}{3.159565in}}{\pgfqpoint{1.686315in}{3.156293in}}{\pgfqpoint{1.680491in}{3.150469in}}%
\pgfpathcurveto{\pgfqpoint{1.674667in}{3.144645in}}{\pgfqpoint{1.671394in}{3.136745in}}{\pgfqpoint{1.671394in}{3.128509in}}%
\pgfpathcurveto{\pgfqpoint{1.671394in}{3.120273in}}{\pgfqpoint{1.674667in}{3.112373in}}{\pgfqpoint{1.680491in}{3.106549in}}%
\pgfpathcurveto{\pgfqpoint{1.686315in}{3.100725in}}{\pgfqpoint{1.694215in}{3.097452in}}{\pgfqpoint{1.702451in}{3.097452in}}%
\pgfpathclose%
\pgfusepath{stroke,fill}%
\end{pgfscope}%
\begin{pgfscope}%
\pgfpathrectangle{\pgfqpoint{0.100000in}{0.220728in}}{\pgfqpoint{3.696000in}{3.696000in}}%
\pgfusepath{clip}%
\pgfsetbuttcap%
\pgfsetroundjoin%
\definecolor{currentfill}{rgb}{0.121569,0.466667,0.705882}%
\pgfsetfillcolor{currentfill}%
\pgfsetfillopacity{0.325228}%
\pgfsetlinewidth{1.003750pt}%
\definecolor{currentstroke}{rgb}{0.121569,0.466667,0.705882}%
\pgfsetstrokecolor{currentstroke}%
\pgfsetstrokeopacity{0.325228}%
\pgfsetdash{}{0pt}%
\pgfpathmoveto{\pgfqpoint{1.702215in}{3.096830in}}%
\pgfpathcurveto{\pgfqpoint{1.710452in}{3.096830in}}{\pgfqpoint{1.718352in}{3.100102in}}{\pgfqpoint{1.724176in}{3.105926in}}%
\pgfpathcurveto{\pgfqpoint{1.729999in}{3.111750in}}{\pgfqpoint{1.733272in}{3.119650in}}{\pgfqpoint{1.733272in}{3.127886in}}%
\pgfpathcurveto{\pgfqpoint{1.733272in}{3.136123in}}{\pgfqpoint{1.729999in}{3.144023in}}{\pgfqpoint{1.724176in}{3.149847in}}%
\pgfpathcurveto{\pgfqpoint{1.718352in}{3.155670in}}{\pgfqpoint{1.710452in}{3.158943in}}{\pgfqpoint{1.702215in}{3.158943in}}%
\pgfpathcurveto{\pgfqpoint{1.693979in}{3.158943in}}{\pgfqpoint{1.686079in}{3.155670in}}{\pgfqpoint{1.680255in}{3.149847in}}%
\pgfpathcurveto{\pgfqpoint{1.674431in}{3.144023in}}{\pgfqpoint{1.671159in}{3.136123in}}{\pgfqpoint{1.671159in}{3.127886in}}%
\pgfpathcurveto{\pgfqpoint{1.671159in}{3.119650in}}{\pgfqpoint{1.674431in}{3.111750in}}{\pgfqpoint{1.680255in}{3.105926in}}%
\pgfpathcurveto{\pgfqpoint{1.686079in}{3.100102in}}{\pgfqpoint{1.693979in}{3.096830in}}{\pgfqpoint{1.702215in}{3.096830in}}%
\pgfpathclose%
\pgfusepath{stroke,fill}%
\end{pgfscope}%
\begin{pgfscope}%
\pgfpathrectangle{\pgfqpoint{0.100000in}{0.220728in}}{\pgfqpoint{3.696000in}{3.696000in}}%
\pgfusepath{clip}%
\pgfsetbuttcap%
\pgfsetroundjoin%
\definecolor{currentfill}{rgb}{0.121569,0.466667,0.705882}%
\pgfsetfillcolor{currentfill}%
\pgfsetfillopacity{0.325428}%
\pgfsetlinewidth{1.003750pt}%
\definecolor{currentstroke}{rgb}{0.121569,0.466667,0.705882}%
\pgfsetstrokecolor{currentstroke}%
\pgfsetstrokeopacity{0.325428}%
\pgfsetdash{}{0pt}%
\pgfpathmoveto{\pgfqpoint{1.701848in}{3.095663in}}%
\pgfpathcurveto{\pgfqpoint{1.710085in}{3.095663in}}{\pgfqpoint{1.717985in}{3.098935in}}{\pgfqpoint{1.723809in}{3.104759in}}%
\pgfpathcurveto{\pgfqpoint{1.729633in}{3.110583in}}{\pgfqpoint{1.732905in}{3.118483in}}{\pgfqpoint{1.732905in}{3.126720in}}%
\pgfpathcurveto{\pgfqpoint{1.732905in}{3.134956in}}{\pgfqpoint{1.729633in}{3.142856in}}{\pgfqpoint{1.723809in}{3.148680in}}%
\pgfpathcurveto{\pgfqpoint{1.717985in}{3.154504in}}{\pgfqpoint{1.710085in}{3.157776in}}{\pgfqpoint{1.701848in}{3.157776in}}%
\pgfpathcurveto{\pgfqpoint{1.693612in}{3.157776in}}{\pgfqpoint{1.685712in}{3.154504in}}{\pgfqpoint{1.679888in}{3.148680in}}%
\pgfpathcurveto{\pgfqpoint{1.674064in}{3.142856in}}{\pgfqpoint{1.670792in}{3.134956in}}{\pgfqpoint{1.670792in}{3.126720in}}%
\pgfpathcurveto{\pgfqpoint{1.670792in}{3.118483in}}{\pgfqpoint{1.674064in}{3.110583in}}{\pgfqpoint{1.679888in}{3.104759in}}%
\pgfpathcurveto{\pgfqpoint{1.685712in}{3.098935in}}{\pgfqpoint{1.693612in}{3.095663in}}{\pgfqpoint{1.701848in}{3.095663in}}%
\pgfpathclose%
\pgfusepath{stroke,fill}%
\end{pgfscope}%
\begin{pgfscope}%
\pgfpathrectangle{\pgfqpoint{0.100000in}{0.220728in}}{\pgfqpoint{3.696000in}{3.696000in}}%
\pgfusepath{clip}%
\pgfsetbuttcap%
\pgfsetroundjoin%
\definecolor{currentfill}{rgb}{0.121569,0.466667,0.705882}%
\pgfsetfillcolor{currentfill}%
\pgfsetfillopacity{0.325764}%
\pgfsetlinewidth{1.003750pt}%
\definecolor{currentstroke}{rgb}{0.121569,0.466667,0.705882}%
\pgfsetstrokecolor{currentstroke}%
\pgfsetstrokeopacity{0.325764}%
\pgfsetdash{}{0pt}%
\pgfpathmoveto{\pgfqpoint{1.700833in}{3.093724in}}%
\pgfpathcurveto{\pgfqpoint{1.709069in}{3.093724in}}{\pgfqpoint{1.716970in}{3.096997in}}{\pgfqpoint{1.722793in}{3.102821in}}%
\pgfpathcurveto{\pgfqpoint{1.728617in}{3.108644in}}{\pgfqpoint{1.731890in}{3.116544in}}{\pgfqpoint{1.731890in}{3.124781in}}%
\pgfpathcurveto{\pgfqpoint{1.731890in}{3.133017in}}{\pgfqpoint{1.728617in}{3.140917in}}{\pgfqpoint{1.722793in}{3.146741in}}%
\pgfpathcurveto{\pgfqpoint{1.716970in}{3.152565in}}{\pgfqpoint{1.709069in}{3.155837in}}{\pgfqpoint{1.700833in}{3.155837in}}%
\pgfpathcurveto{\pgfqpoint{1.692597in}{3.155837in}}{\pgfqpoint{1.684697in}{3.152565in}}{\pgfqpoint{1.678873in}{3.146741in}}%
\pgfpathcurveto{\pgfqpoint{1.673049in}{3.140917in}}{\pgfqpoint{1.669777in}{3.133017in}}{\pgfqpoint{1.669777in}{3.124781in}}%
\pgfpathcurveto{\pgfqpoint{1.669777in}{3.116544in}}{\pgfqpoint{1.673049in}{3.108644in}}{\pgfqpoint{1.678873in}{3.102821in}}%
\pgfpathcurveto{\pgfqpoint{1.684697in}{3.096997in}}{\pgfqpoint{1.692597in}{3.093724in}}{\pgfqpoint{1.700833in}{3.093724in}}%
\pgfpathclose%
\pgfusepath{stroke,fill}%
\end{pgfscope}%
\begin{pgfscope}%
\pgfpathrectangle{\pgfqpoint{0.100000in}{0.220728in}}{\pgfqpoint{3.696000in}{3.696000in}}%
\pgfusepath{clip}%
\pgfsetbuttcap%
\pgfsetroundjoin%
\definecolor{currentfill}{rgb}{0.121569,0.466667,0.705882}%
\pgfsetfillcolor{currentfill}%
\pgfsetfillopacity{0.326405}%
\pgfsetlinewidth{1.003750pt}%
\definecolor{currentstroke}{rgb}{0.121569,0.466667,0.705882}%
\pgfsetstrokecolor{currentstroke}%
\pgfsetstrokeopacity{0.326405}%
\pgfsetdash{}{0pt}%
\pgfpathmoveto{\pgfqpoint{1.700578in}{3.089498in}}%
\pgfpathcurveto{\pgfqpoint{1.708815in}{3.089498in}}{\pgfqpoint{1.716715in}{3.092770in}}{\pgfqpoint{1.722539in}{3.098594in}}%
\pgfpathcurveto{\pgfqpoint{1.728363in}{3.104418in}}{\pgfqpoint{1.731635in}{3.112318in}}{\pgfqpoint{1.731635in}{3.120555in}}%
\pgfpathcurveto{\pgfqpoint{1.731635in}{3.128791in}}{\pgfqpoint{1.728363in}{3.136691in}}{\pgfqpoint{1.722539in}{3.142515in}}%
\pgfpathcurveto{\pgfqpoint{1.716715in}{3.148339in}}{\pgfqpoint{1.708815in}{3.151611in}}{\pgfqpoint{1.700578in}{3.151611in}}%
\pgfpathcurveto{\pgfqpoint{1.692342in}{3.151611in}}{\pgfqpoint{1.684442in}{3.148339in}}{\pgfqpoint{1.678618in}{3.142515in}}%
\pgfpathcurveto{\pgfqpoint{1.672794in}{3.136691in}}{\pgfqpoint{1.669522in}{3.128791in}}{\pgfqpoint{1.669522in}{3.120555in}}%
\pgfpathcurveto{\pgfqpoint{1.669522in}{3.112318in}}{\pgfqpoint{1.672794in}{3.104418in}}{\pgfqpoint{1.678618in}{3.098594in}}%
\pgfpathcurveto{\pgfqpoint{1.684442in}{3.092770in}}{\pgfqpoint{1.692342in}{3.089498in}}{\pgfqpoint{1.700578in}{3.089498in}}%
\pgfpathclose%
\pgfusepath{stroke,fill}%
\end{pgfscope}%
\begin{pgfscope}%
\pgfpathrectangle{\pgfqpoint{0.100000in}{0.220728in}}{\pgfqpoint{3.696000in}{3.696000in}}%
\pgfusepath{clip}%
\pgfsetbuttcap%
\pgfsetroundjoin%
\definecolor{currentfill}{rgb}{0.121569,0.466667,0.705882}%
\pgfsetfillcolor{currentfill}%
\pgfsetfillopacity{0.326588}%
\pgfsetlinewidth{1.003750pt}%
\definecolor{currentstroke}{rgb}{0.121569,0.466667,0.705882}%
\pgfsetstrokecolor{currentstroke}%
\pgfsetstrokeopacity{0.326588}%
\pgfsetdash{}{0pt}%
\pgfpathmoveto{\pgfqpoint{1.699755in}{3.088453in}}%
\pgfpathcurveto{\pgfqpoint{1.707991in}{3.088453in}}{\pgfqpoint{1.715892in}{3.091726in}}{\pgfqpoint{1.721715in}{3.097550in}}%
\pgfpathcurveto{\pgfqpoint{1.727539in}{3.103373in}}{\pgfqpoint{1.730812in}{3.111274in}}{\pgfqpoint{1.730812in}{3.119510in}}%
\pgfpathcurveto{\pgfqpoint{1.730812in}{3.127746in}}{\pgfqpoint{1.727539in}{3.135646in}}{\pgfqpoint{1.721715in}{3.141470in}}%
\pgfpathcurveto{\pgfqpoint{1.715892in}{3.147294in}}{\pgfqpoint{1.707991in}{3.150566in}}{\pgfqpoint{1.699755in}{3.150566in}}%
\pgfpathcurveto{\pgfqpoint{1.691519in}{3.150566in}}{\pgfqpoint{1.683619in}{3.147294in}}{\pgfqpoint{1.677795in}{3.141470in}}%
\pgfpathcurveto{\pgfqpoint{1.671971in}{3.135646in}}{\pgfqpoint{1.668699in}{3.127746in}}{\pgfqpoint{1.668699in}{3.119510in}}%
\pgfpathcurveto{\pgfqpoint{1.668699in}{3.111274in}}{\pgfqpoint{1.671971in}{3.103373in}}{\pgfqpoint{1.677795in}{3.097550in}}%
\pgfpathcurveto{\pgfqpoint{1.683619in}{3.091726in}}{\pgfqpoint{1.691519in}{3.088453in}}{\pgfqpoint{1.699755in}{3.088453in}}%
\pgfpathclose%
\pgfusepath{stroke,fill}%
\end{pgfscope}%
\begin{pgfscope}%
\pgfpathrectangle{\pgfqpoint{0.100000in}{0.220728in}}{\pgfqpoint{3.696000in}{3.696000in}}%
\pgfusepath{clip}%
\pgfsetbuttcap%
\pgfsetroundjoin%
\definecolor{currentfill}{rgb}{0.121569,0.466667,0.705882}%
\pgfsetfillcolor{currentfill}%
\pgfsetfillopacity{0.327018}%
\pgfsetlinewidth{1.003750pt}%
\definecolor{currentstroke}{rgb}{0.121569,0.466667,0.705882}%
\pgfsetstrokecolor{currentstroke}%
\pgfsetstrokeopacity{0.327018}%
\pgfsetdash{}{0pt}%
\pgfpathmoveto{\pgfqpoint{1.699175in}{3.086019in}}%
\pgfpathcurveto{\pgfqpoint{1.707411in}{3.086019in}}{\pgfqpoint{1.715311in}{3.089291in}}{\pgfqpoint{1.721135in}{3.095115in}}%
\pgfpathcurveto{\pgfqpoint{1.726959in}{3.100939in}}{\pgfqpoint{1.730231in}{3.108839in}}{\pgfqpoint{1.730231in}{3.117075in}}%
\pgfpathcurveto{\pgfqpoint{1.730231in}{3.125312in}}{\pgfqpoint{1.726959in}{3.133212in}}{\pgfqpoint{1.721135in}{3.139036in}}%
\pgfpathcurveto{\pgfqpoint{1.715311in}{3.144860in}}{\pgfqpoint{1.707411in}{3.148132in}}{\pgfqpoint{1.699175in}{3.148132in}}%
\pgfpathcurveto{\pgfqpoint{1.690938in}{3.148132in}}{\pgfqpoint{1.683038in}{3.144860in}}{\pgfqpoint{1.677214in}{3.139036in}}%
\pgfpathcurveto{\pgfqpoint{1.671390in}{3.133212in}}{\pgfqpoint{1.668118in}{3.125312in}}{\pgfqpoint{1.668118in}{3.117075in}}%
\pgfpathcurveto{\pgfqpoint{1.668118in}{3.108839in}}{\pgfqpoint{1.671390in}{3.100939in}}{\pgfqpoint{1.677214in}{3.095115in}}%
\pgfpathcurveto{\pgfqpoint{1.683038in}{3.089291in}}{\pgfqpoint{1.690938in}{3.086019in}}{\pgfqpoint{1.699175in}{3.086019in}}%
\pgfpathclose%
\pgfusepath{stroke,fill}%
\end{pgfscope}%
\begin{pgfscope}%
\pgfpathrectangle{\pgfqpoint{0.100000in}{0.220728in}}{\pgfqpoint{3.696000in}{3.696000in}}%
\pgfusepath{clip}%
\pgfsetbuttcap%
\pgfsetroundjoin%
\definecolor{currentfill}{rgb}{0.121569,0.466667,0.705882}%
\pgfsetfillcolor{currentfill}%
\pgfsetfillopacity{0.327492}%
\pgfsetlinewidth{1.003750pt}%
\definecolor{currentstroke}{rgb}{0.121569,0.466667,0.705882}%
\pgfsetstrokecolor{currentstroke}%
\pgfsetstrokeopacity{0.327492}%
\pgfsetdash{}{0pt}%
\pgfpathmoveto{\pgfqpoint{1.877554in}{3.251656in}}%
\pgfpathcurveto{\pgfqpoint{1.885790in}{3.251656in}}{\pgfqpoint{1.893690in}{3.254928in}}{\pgfqpoint{1.899514in}{3.260752in}}%
\pgfpathcurveto{\pgfqpoint{1.905338in}{3.266576in}}{\pgfqpoint{1.908610in}{3.274476in}}{\pgfqpoint{1.908610in}{3.282712in}}%
\pgfpathcurveto{\pgfqpoint{1.908610in}{3.290949in}}{\pgfqpoint{1.905338in}{3.298849in}}{\pgfqpoint{1.899514in}{3.304672in}}%
\pgfpathcurveto{\pgfqpoint{1.893690in}{3.310496in}}{\pgfqpoint{1.885790in}{3.313769in}}{\pgfqpoint{1.877554in}{3.313769in}}%
\pgfpathcurveto{\pgfqpoint{1.869317in}{3.313769in}}{\pgfqpoint{1.861417in}{3.310496in}}{\pgfqpoint{1.855594in}{3.304672in}}%
\pgfpathcurveto{\pgfqpoint{1.849770in}{3.298849in}}{\pgfqpoint{1.846497in}{3.290949in}}{\pgfqpoint{1.846497in}{3.282712in}}%
\pgfpathcurveto{\pgfqpoint{1.846497in}{3.274476in}}{\pgfqpoint{1.849770in}{3.266576in}}{\pgfqpoint{1.855594in}{3.260752in}}%
\pgfpathcurveto{\pgfqpoint{1.861417in}{3.254928in}}{\pgfqpoint{1.869317in}{3.251656in}}{\pgfqpoint{1.877554in}{3.251656in}}%
\pgfpathclose%
\pgfusepath{stroke,fill}%
\end{pgfscope}%
\begin{pgfscope}%
\pgfpathrectangle{\pgfqpoint{0.100000in}{0.220728in}}{\pgfqpoint{3.696000in}{3.696000in}}%
\pgfusepath{clip}%
\pgfsetbuttcap%
\pgfsetroundjoin%
\definecolor{currentfill}{rgb}{0.121569,0.466667,0.705882}%
\pgfsetfillcolor{currentfill}%
\pgfsetfillopacity{0.327774}%
\pgfsetlinewidth{1.003750pt}%
\definecolor{currentstroke}{rgb}{0.121569,0.466667,0.705882}%
\pgfsetstrokecolor{currentstroke}%
\pgfsetstrokeopacity{0.327774}%
\pgfsetdash{}{0pt}%
\pgfpathmoveto{\pgfqpoint{1.697729in}{3.081692in}}%
\pgfpathcurveto{\pgfqpoint{1.705965in}{3.081692in}}{\pgfqpoint{1.713865in}{3.084964in}}{\pgfqpoint{1.719689in}{3.090788in}}%
\pgfpathcurveto{\pgfqpoint{1.725513in}{3.096612in}}{\pgfqpoint{1.728785in}{3.104512in}}{\pgfqpoint{1.728785in}{3.112748in}}%
\pgfpathcurveto{\pgfqpoint{1.728785in}{3.120984in}}{\pgfqpoint{1.725513in}{3.128884in}}{\pgfqpoint{1.719689in}{3.134708in}}%
\pgfpathcurveto{\pgfqpoint{1.713865in}{3.140532in}}{\pgfqpoint{1.705965in}{3.143805in}}{\pgfqpoint{1.697729in}{3.143805in}}%
\pgfpathcurveto{\pgfqpoint{1.689493in}{3.143805in}}{\pgfqpoint{1.681593in}{3.140532in}}{\pgfqpoint{1.675769in}{3.134708in}}%
\pgfpathcurveto{\pgfqpoint{1.669945in}{3.128884in}}{\pgfqpoint{1.666672in}{3.120984in}}{\pgfqpoint{1.666672in}{3.112748in}}%
\pgfpathcurveto{\pgfqpoint{1.666672in}{3.104512in}}{\pgfqpoint{1.669945in}{3.096612in}}{\pgfqpoint{1.675769in}{3.090788in}}%
\pgfpathcurveto{\pgfqpoint{1.681593in}{3.084964in}}{\pgfqpoint{1.689493in}{3.081692in}}{\pgfqpoint{1.697729in}{3.081692in}}%
\pgfpathclose%
\pgfusepath{stroke,fill}%
\end{pgfscope}%
\begin{pgfscope}%
\pgfpathrectangle{\pgfqpoint{0.100000in}{0.220728in}}{\pgfqpoint{3.696000in}{3.696000in}}%
\pgfusepath{clip}%
\pgfsetbuttcap%
\pgfsetroundjoin%
\definecolor{currentfill}{rgb}{0.121569,0.466667,0.705882}%
\pgfsetfillcolor{currentfill}%
\pgfsetfillopacity{0.329047}%
\pgfsetlinewidth{1.003750pt}%
\definecolor{currentstroke}{rgb}{0.121569,0.466667,0.705882}%
\pgfsetstrokecolor{currentstroke}%
\pgfsetstrokeopacity{0.329047}%
\pgfsetdash{}{0pt}%
\pgfpathmoveto{\pgfqpoint{1.694212in}{3.074123in}}%
\pgfpathcurveto{\pgfqpoint{1.702448in}{3.074123in}}{\pgfqpoint{1.710348in}{3.077395in}}{\pgfqpoint{1.716172in}{3.083219in}}%
\pgfpathcurveto{\pgfqpoint{1.721996in}{3.089043in}}{\pgfqpoint{1.725268in}{3.096943in}}{\pgfqpoint{1.725268in}{3.105179in}}%
\pgfpathcurveto{\pgfqpoint{1.725268in}{3.113416in}}{\pgfqpoint{1.721996in}{3.121316in}}{\pgfqpoint{1.716172in}{3.127140in}}%
\pgfpathcurveto{\pgfqpoint{1.710348in}{3.132964in}}{\pgfqpoint{1.702448in}{3.136236in}}{\pgfqpoint{1.694212in}{3.136236in}}%
\pgfpathcurveto{\pgfqpoint{1.685975in}{3.136236in}}{\pgfqpoint{1.678075in}{3.132964in}}{\pgfqpoint{1.672251in}{3.127140in}}%
\pgfpathcurveto{\pgfqpoint{1.666427in}{3.121316in}}{\pgfqpoint{1.663155in}{3.113416in}}{\pgfqpoint{1.663155in}{3.105179in}}%
\pgfpathcurveto{\pgfqpoint{1.663155in}{3.096943in}}{\pgfqpoint{1.666427in}{3.089043in}}{\pgfqpoint{1.672251in}{3.083219in}}%
\pgfpathcurveto{\pgfqpoint{1.678075in}{3.077395in}}{\pgfqpoint{1.685975in}{3.074123in}}{\pgfqpoint{1.694212in}{3.074123in}}%
\pgfpathclose%
\pgfusepath{stroke,fill}%
\end{pgfscope}%
\begin{pgfscope}%
\pgfpathrectangle{\pgfqpoint{0.100000in}{0.220728in}}{\pgfqpoint{3.696000in}{3.696000in}}%
\pgfusepath{clip}%
\pgfsetbuttcap%
\pgfsetroundjoin%
\definecolor{currentfill}{rgb}{0.121569,0.466667,0.705882}%
\pgfsetfillcolor{currentfill}%
\pgfsetfillopacity{0.329814}%
\pgfsetlinewidth{1.003750pt}%
\definecolor{currentstroke}{rgb}{0.121569,0.466667,0.705882}%
\pgfsetstrokecolor{currentstroke}%
\pgfsetstrokeopacity{0.329814}%
\pgfsetdash{}{0pt}%
\pgfpathmoveto{\pgfqpoint{1.891653in}{3.249230in}}%
\pgfpathcurveto{\pgfqpoint{1.899889in}{3.249230in}}{\pgfqpoint{1.907789in}{3.252502in}}{\pgfqpoint{1.913613in}{3.258326in}}%
\pgfpathcurveto{\pgfqpoint{1.919437in}{3.264150in}}{\pgfqpoint{1.922709in}{3.272050in}}{\pgfqpoint{1.922709in}{3.280286in}}%
\pgfpathcurveto{\pgfqpoint{1.922709in}{3.288522in}}{\pgfqpoint{1.919437in}{3.296422in}}{\pgfqpoint{1.913613in}{3.302246in}}%
\pgfpathcurveto{\pgfqpoint{1.907789in}{3.308070in}}{\pgfqpoint{1.899889in}{3.311343in}}{\pgfqpoint{1.891653in}{3.311343in}}%
\pgfpathcurveto{\pgfqpoint{1.883417in}{3.311343in}}{\pgfqpoint{1.875517in}{3.308070in}}{\pgfqpoint{1.869693in}{3.302246in}}%
\pgfpathcurveto{\pgfqpoint{1.863869in}{3.296422in}}{\pgfqpoint{1.860596in}{3.288522in}}{\pgfqpoint{1.860596in}{3.280286in}}%
\pgfpathcurveto{\pgfqpoint{1.860596in}{3.272050in}}{\pgfqpoint{1.863869in}{3.264150in}}{\pgfqpoint{1.869693in}{3.258326in}}%
\pgfpathcurveto{\pgfqpoint{1.875517in}{3.252502in}}{\pgfqpoint{1.883417in}{3.249230in}}{\pgfqpoint{1.891653in}{3.249230in}}%
\pgfpathclose%
\pgfusepath{stroke,fill}%
\end{pgfscope}%
\begin{pgfscope}%
\pgfpathrectangle{\pgfqpoint{0.100000in}{0.220728in}}{\pgfqpoint{3.696000in}{3.696000in}}%
\pgfusepath{clip}%
\pgfsetbuttcap%
\pgfsetroundjoin%
\definecolor{currentfill}{rgb}{0.121569,0.466667,0.705882}%
\pgfsetfillcolor{currentfill}%
\pgfsetfillopacity{0.330296}%
\pgfsetlinewidth{1.003750pt}%
\definecolor{currentstroke}{rgb}{0.121569,0.466667,0.705882}%
\pgfsetstrokecolor{currentstroke}%
\pgfsetstrokeopacity{0.330296}%
\pgfsetdash{}{0pt}%
\pgfpathmoveto{\pgfqpoint{1.692149in}{3.067392in}}%
\pgfpathcurveto{\pgfqpoint{1.700385in}{3.067392in}}{\pgfqpoint{1.708285in}{3.070665in}}{\pgfqpoint{1.714109in}{3.076489in}}%
\pgfpathcurveto{\pgfqpoint{1.719933in}{3.082312in}}{\pgfqpoint{1.723206in}{3.090213in}}{\pgfqpoint{1.723206in}{3.098449in}}%
\pgfpathcurveto{\pgfqpoint{1.723206in}{3.106685in}}{\pgfqpoint{1.719933in}{3.114585in}}{\pgfqpoint{1.714109in}{3.120409in}}%
\pgfpathcurveto{\pgfqpoint{1.708285in}{3.126233in}}{\pgfqpoint{1.700385in}{3.129505in}}{\pgfqpoint{1.692149in}{3.129505in}}%
\pgfpathcurveto{\pgfqpoint{1.683913in}{3.129505in}}{\pgfqpoint{1.676013in}{3.126233in}}{\pgfqpoint{1.670189in}{3.120409in}}%
\pgfpathcurveto{\pgfqpoint{1.664365in}{3.114585in}}{\pgfqpoint{1.661093in}{3.106685in}}{\pgfqpoint{1.661093in}{3.098449in}}%
\pgfpathcurveto{\pgfqpoint{1.661093in}{3.090213in}}{\pgfqpoint{1.664365in}{3.082312in}}{\pgfqpoint{1.670189in}{3.076489in}}%
\pgfpathcurveto{\pgfqpoint{1.676013in}{3.070665in}}{\pgfqpoint{1.683913in}{3.067392in}}{\pgfqpoint{1.692149in}{3.067392in}}%
\pgfpathclose%
\pgfusepath{stroke,fill}%
\end{pgfscope}%
\begin{pgfscope}%
\pgfpathrectangle{\pgfqpoint{0.100000in}{0.220728in}}{\pgfqpoint{3.696000in}{3.696000in}}%
\pgfusepath{clip}%
\pgfsetbuttcap%
\pgfsetroundjoin%
\definecolor{currentfill}{rgb}{0.121569,0.466667,0.705882}%
\pgfsetfillcolor{currentfill}%
\pgfsetfillopacity{0.331009}%
\pgfsetlinewidth{1.003750pt}%
\definecolor{currentstroke}{rgb}{0.121569,0.466667,0.705882}%
\pgfsetstrokecolor{currentstroke}%
\pgfsetstrokeopacity{0.331009}%
\pgfsetdash{}{0pt}%
\pgfpathmoveto{\pgfqpoint{1.689274in}{3.063419in}}%
\pgfpathcurveto{\pgfqpoint{1.697510in}{3.063419in}}{\pgfqpoint{1.705410in}{3.066691in}}{\pgfqpoint{1.711234in}{3.072515in}}%
\pgfpathcurveto{\pgfqpoint{1.717058in}{3.078339in}}{\pgfqpoint{1.720331in}{3.086239in}}{\pgfqpoint{1.720331in}{3.094475in}}%
\pgfpathcurveto{\pgfqpoint{1.720331in}{3.102711in}}{\pgfqpoint{1.717058in}{3.110611in}}{\pgfqpoint{1.711234in}{3.116435in}}%
\pgfpathcurveto{\pgfqpoint{1.705410in}{3.122259in}}{\pgfqpoint{1.697510in}{3.125532in}}{\pgfqpoint{1.689274in}{3.125532in}}%
\pgfpathcurveto{\pgfqpoint{1.681038in}{3.125532in}}{\pgfqpoint{1.673138in}{3.122259in}}{\pgfqpoint{1.667314in}{3.116435in}}%
\pgfpathcurveto{\pgfqpoint{1.661490in}{3.110611in}}{\pgfqpoint{1.658218in}{3.102711in}}{\pgfqpoint{1.658218in}{3.094475in}}%
\pgfpathcurveto{\pgfqpoint{1.658218in}{3.086239in}}{\pgfqpoint{1.661490in}{3.078339in}}{\pgfqpoint{1.667314in}{3.072515in}}%
\pgfpathcurveto{\pgfqpoint{1.673138in}{3.066691in}}{\pgfqpoint{1.681038in}{3.063419in}}{\pgfqpoint{1.689274in}{3.063419in}}%
\pgfpathclose%
\pgfusepath{stroke,fill}%
\end{pgfscope}%
\begin{pgfscope}%
\pgfpathrectangle{\pgfqpoint{0.100000in}{0.220728in}}{\pgfqpoint{3.696000in}{3.696000in}}%
\pgfusepath{clip}%
\pgfsetbuttcap%
\pgfsetroundjoin%
\definecolor{currentfill}{rgb}{0.121569,0.466667,0.705882}%
\pgfsetfillcolor{currentfill}%
\pgfsetfillopacity{0.331570}%
\pgfsetlinewidth{1.003750pt}%
\definecolor{currentstroke}{rgb}{0.121569,0.466667,0.705882}%
\pgfsetstrokecolor{currentstroke}%
\pgfsetstrokeopacity{0.331570}%
\pgfsetdash{}{0pt}%
\pgfpathmoveto{\pgfqpoint{1.688913in}{3.059516in}}%
\pgfpathcurveto{\pgfqpoint{1.697150in}{3.059516in}}{\pgfqpoint{1.705050in}{3.062789in}}{\pgfqpoint{1.710874in}{3.068613in}}%
\pgfpathcurveto{\pgfqpoint{1.716697in}{3.074436in}}{\pgfqpoint{1.719970in}{3.082336in}}{\pgfqpoint{1.719970in}{3.090573in}}%
\pgfpathcurveto{\pgfqpoint{1.719970in}{3.098809in}}{\pgfqpoint{1.716697in}{3.106709in}}{\pgfqpoint{1.710874in}{3.112533in}}%
\pgfpathcurveto{\pgfqpoint{1.705050in}{3.118357in}}{\pgfqpoint{1.697150in}{3.121629in}}{\pgfqpoint{1.688913in}{3.121629in}}%
\pgfpathcurveto{\pgfqpoint{1.680677in}{3.121629in}}{\pgfqpoint{1.672777in}{3.118357in}}{\pgfqpoint{1.666953in}{3.112533in}}%
\pgfpathcurveto{\pgfqpoint{1.661129in}{3.106709in}}{\pgfqpoint{1.657857in}{3.098809in}}{\pgfqpoint{1.657857in}{3.090573in}}%
\pgfpathcurveto{\pgfqpoint{1.657857in}{3.082336in}}{\pgfqpoint{1.661129in}{3.074436in}}{\pgfqpoint{1.666953in}{3.068613in}}%
\pgfpathcurveto{\pgfqpoint{1.672777in}{3.062789in}}{\pgfqpoint{1.680677in}{3.059516in}}{\pgfqpoint{1.688913in}{3.059516in}}%
\pgfpathclose%
\pgfusepath{stroke,fill}%
\end{pgfscope}%
\begin{pgfscope}%
\pgfpathrectangle{\pgfqpoint{0.100000in}{0.220728in}}{\pgfqpoint{3.696000in}{3.696000in}}%
\pgfusepath{clip}%
\pgfsetbuttcap%
\pgfsetroundjoin%
\definecolor{currentfill}{rgb}{0.121569,0.466667,0.705882}%
\pgfsetfillcolor{currentfill}%
\pgfsetfillopacity{0.331703}%
\pgfsetlinewidth{1.003750pt}%
\definecolor{currentstroke}{rgb}{0.121569,0.466667,0.705882}%
\pgfsetstrokecolor{currentstroke}%
\pgfsetstrokeopacity{0.331703}%
\pgfsetdash{}{0pt}%
\pgfpathmoveto{\pgfqpoint{1.688076in}{3.058558in}}%
\pgfpathcurveto{\pgfqpoint{1.696312in}{3.058558in}}{\pgfqpoint{1.704212in}{3.061830in}}{\pgfqpoint{1.710036in}{3.067654in}}%
\pgfpathcurveto{\pgfqpoint{1.715860in}{3.073478in}}{\pgfqpoint{1.719133in}{3.081378in}}{\pgfqpoint{1.719133in}{3.089614in}}%
\pgfpathcurveto{\pgfqpoint{1.719133in}{3.097851in}}{\pgfqpoint{1.715860in}{3.105751in}}{\pgfqpoint{1.710036in}{3.111575in}}%
\pgfpathcurveto{\pgfqpoint{1.704212in}{3.117399in}}{\pgfqpoint{1.696312in}{3.120671in}}{\pgfqpoint{1.688076in}{3.120671in}}%
\pgfpathcurveto{\pgfqpoint{1.679840in}{3.120671in}}{\pgfqpoint{1.671940in}{3.117399in}}{\pgfqpoint{1.666116in}{3.111575in}}%
\pgfpathcurveto{\pgfqpoint{1.660292in}{3.105751in}}{\pgfqpoint{1.657020in}{3.097851in}}{\pgfqpoint{1.657020in}{3.089614in}}%
\pgfpathcurveto{\pgfqpoint{1.657020in}{3.081378in}}{\pgfqpoint{1.660292in}{3.073478in}}{\pgfqpoint{1.666116in}{3.067654in}}%
\pgfpathcurveto{\pgfqpoint{1.671940in}{3.061830in}}{\pgfqpoint{1.679840in}{3.058558in}}{\pgfqpoint{1.688076in}{3.058558in}}%
\pgfpathclose%
\pgfusepath{stroke,fill}%
\end{pgfscope}%
\begin{pgfscope}%
\pgfpathrectangle{\pgfqpoint{0.100000in}{0.220728in}}{\pgfqpoint{3.696000in}{3.696000in}}%
\pgfusepath{clip}%
\pgfsetbuttcap%
\pgfsetroundjoin%
\definecolor{currentfill}{rgb}{0.121569,0.466667,0.705882}%
\pgfsetfillcolor{currentfill}%
\pgfsetfillopacity{0.331718}%
\pgfsetlinewidth{1.003750pt}%
\definecolor{currentstroke}{rgb}{0.121569,0.466667,0.705882}%
\pgfsetstrokecolor{currentstroke}%
\pgfsetstrokeopacity{0.331718}%
\pgfsetdash{}{0pt}%
\pgfpathmoveto{\pgfqpoint{1.898925in}{3.248858in}}%
\pgfpathcurveto{\pgfqpoint{1.907161in}{3.248858in}}{\pgfqpoint{1.915061in}{3.252130in}}{\pgfqpoint{1.920885in}{3.257954in}}%
\pgfpathcurveto{\pgfqpoint{1.926709in}{3.263778in}}{\pgfqpoint{1.929982in}{3.271678in}}{\pgfqpoint{1.929982in}{3.279915in}}%
\pgfpathcurveto{\pgfqpoint{1.929982in}{3.288151in}}{\pgfqpoint{1.926709in}{3.296051in}}{\pgfqpoint{1.920885in}{3.301875in}}%
\pgfpathcurveto{\pgfqpoint{1.915061in}{3.307699in}}{\pgfqpoint{1.907161in}{3.310971in}}{\pgfqpoint{1.898925in}{3.310971in}}%
\pgfpathcurveto{\pgfqpoint{1.890689in}{3.310971in}}{\pgfqpoint{1.882789in}{3.307699in}}{\pgfqpoint{1.876965in}{3.301875in}}%
\pgfpathcurveto{\pgfqpoint{1.871141in}{3.296051in}}{\pgfqpoint{1.867869in}{3.288151in}}{\pgfqpoint{1.867869in}{3.279915in}}%
\pgfpathcurveto{\pgfqpoint{1.867869in}{3.271678in}}{\pgfqpoint{1.871141in}{3.263778in}}{\pgfqpoint{1.876965in}{3.257954in}}%
\pgfpathcurveto{\pgfqpoint{1.882789in}{3.252130in}}{\pgfqpoint{1.890689in}{3.248858in}}{\pgfqpoint{1.898925in}{3.248858in}}%
\pgfpathclose%
\pgfusepath{stroke,fill}%
\end{pgfscope}%
\begin{pgfscope}%
\pgfpathrectangle{\pgfqpoint{0.100000in}{0.220728in}}{\pgfqpoint{3.696000in}{3.696000in}}%
\pgfusepath{clip}%
\pgfsetbuttcap%
\pgfsetroundjoin%
\definecolor{currentfill}{rgb}{0.121569,0.466667,0.705882}%
\pgfsetfillcolor{currentfill}%
\pgfsetfillopacity{0.332095}%
\pgfsetlinewidth{1.003750pt}%
\definecolor{currentstroke}{rgb}{0.121569,0.466667,0.705882}%
\pgfsetstrokecolor{currentstroke}%
\pgfsetstrokeopacity{0.332095}%
\pgfsetdash{}{0pt}%
\pgfpathmoveto{\pgfqpoint{1.687536in}{3.056282in}}%
\pgfpathcurveto{\pgfqpoint{1.695772in}{3.056282in}}{\pgfqpoint{1.703672in}{3.059554in}}{\pgfqpoint{1.709496in}{3.065378in}}%
\pgfpathcurveto{\pgfqpoint{1.715320in}{3.071202in}}{\pgfqpoint{1.718592in}{3.079102in}}{\pgfqpoint{1.718592in}{3.087338in}}%
\pgfpathcurveto{\pgfqpoint{1.718592in}{3.095574in}}{\pgfqpoint{1.715320in}{3.103474in}}{\pgfqpoint{1.709496in}{3.109298in}}%
\pgfpathcurveto{\pgfqpoint{1.703672in}{3.115122in}}{\pgfqpoint{1.695772in}{3.118395in}}{\pgfqpoint{1.687536in}{3.118395in}}%
\pgfpathcurveto{\pgfqpoint{1.679299in}{3.118395in}}{\pgfqpoint{1.671399in}{3.115122in}}{\pgfqpoint{1.665575in}{3.109298in}}%
\pgfpathcurveto{\pgfqpoint{1.659752in}{3.103474in}}{\pgfqpoint{1.656479in}{3.095574in}}{\pgfqpoint{1.656479in}{3.087338in}}%
\pgfpathcurveto{\pgfqpoint{1.656479in}{3.079102in}}{\pgfqpoint{1.659752in}{3.071202in}}{\pgfqpoint{1.665575in}{3.065378in}}%
\pgfpathcurveto{\pgfqpoint{1.671399in}{3.059554in}}{\pgfqpoint{1.679299in}{3.056282in}}{\pgfqpoint{1.687536in}{3.056282in}}%
\pgfpathclose%
\pgfusepath{stroke,fill}%
\end{pgfscope}%
\begin{pgfscope}%
\pgfpathrectangle{\pgfqpoint{0.100000in}{0.220728in}}{\pgfqpoint{3.696000in}{3.696000in}}%
\pgfusepath{clip}%
\pgfsetbuttcap%
\pgfsetroundjoin%
\definecolor{currentfill}{rgb}{0.121569,0.466667,0.705882}%
\pgfsetfillcolor{currentfill}%
\pgfsetfillopacity{0.332806}%
\pgfsetlinewidth{1.003750pt}%
\definecolor{currentstroke}{rgb}{0.121569,0.466667,0.705882}%
\pgfsetstrokecolor{currentstroke}%
\pgfsetstrokeopacity{0.332806}%
\pgfsetdash{}{0pt}%
\pgfpathmoveto{\pgfqpoint{1.686221in}{3.052311in}}%
\pgfpathcurveto{\pgfqpoint{1.694457in}{3.052311in}}{\pgfqpoint{1.702357in}{3.055584in}}{\pgfqpoint{1.708181in}{3.061408in}}%
\pgfpathcurveto{\pgfqpoint{1.714005in}{3.067232in}}{\pgfqpoint{1.717278in}{3.075132in}}{\pgfqpoint{1.717278in}{3.083368in}}%
\pgfpathcurveto{\pgfqpoint{1.717278in}{3.091604in}}{\pgfqpoint{1.714005in}{3.099504in}}{\pgfqpoint{1.708181in}{3.105328in}}%
\pgfpathcurveto{\pgfqpoint{1.702357in}{3.111152in}}{\pgfqpoint{1.694457in}{3.114424in}}{\pgfqpoint{1.686221in}{3.114424in}}%
\pgfpathcurveto{\pgfqpoint{1.677985in}{3.114424in}}{\pgfqpoint{1.670085in}{3.111152in}}{\pgfqpoint{1.664261in}{3.105328in}}%
\pgfpathcurveto{\pgfqpoint{1.658437in}{3.099504in}}{\pgfqpoint{1.655165in}{3.091604in}}{\pgfqpoint{1.655165in}{3.083368in}}%
\pgfpathcurveto{\pgfqpoint{1.655165in}{3.075132in}}{\pgfqpoint{1.658437in}{3.067232in}}{\pgfqpoint{1.664261in}{3.061408in}}%
\pgfpathcurveto{\pgfqpoint{1.670085in}{3.055584in}}{\pgfqpoint{1.677985in}{3.052311in}}{\pgfqpoint{1.686221in}{3.052311in}}%
\pgfpathclose%
\pgfusepath{stroke,fill}%
\end{pgfscope}%
\begin{pgfscope}%
\pgfpathrectangle{\pgfqpoint{0.100000in}{0.220728in}}{\pgfqpoint{3.696000in}{3.696000in}}%
\pgfusepath{clip}%
\pgfsetbuttcap%
\pgfsetroundjoin%
\definecolor{currentfill}{rgb}{0.121569,0.466667,0.705882}%
\pgfsetfillcolor{currentfill}%
\pgfsetfillopacity{0.333861}%
\pgfsetlinewidth{1.003750pt}%
\definecolor{currentstroke}{rgb}{0.121569,0.466667,0.705882}%
\pgfsetstrokecolor{currentstroke}%
\pgfsetstrokeopacity{0.333861}%
\pgfsetdash{}{0pt}%
\pgfpathmoveto{\pgfqpoint{1.682133in}{3.045835in}}%
\pgfpathcurveto{\pgfqpoint{1.690369in}{3.045835in}}{\pgfqpoint{1.698269in}{3.049107in}}{\pgfqpoint{1.704093in}{3.054931in}}%
\pgfpathcurveto{\pgfqpoint{1.709917in}{3.060755in}}{\pgfqpoint{1.713189in}{3.068655in}}{\pgfqpoint{1.713189in}{3.076891in}}%
\pgfpathcurveto{\pgfqpoint{1.713189in}{3.085128in}}{\pgfqpoint{1.709917in}{3.093028in}}{\pgfqpoint{1.704093in}{3.098852in}}%
\pgfpathcurveto{\pgfqpoint{1.698269in}{3.104676in}}{\pgfqpoint{1.690369in}{3.107948in}}{\pgfqpoint{1.682133in}{3.107948in}}%
\pgfpathcurveto{\pgfqpoint{1.673897in}{3.107948in}}{\pgfqpoint{1.665996in}{3.104676in}}{\pgfqpoint{1.660173in}{3.098852in}}%
\pgfpathcurveto{\pgfqpoint{1.654349in}{3.093028in}}{\pgfqpoint{1.651076in}{3.085128in}}{\pgfqpoint{1.651076in}{3.076891in}}%
\pgfpathcurveto{\pgfqpoint{1.651076in}{3.068655in}}{\pgfqpoint{1.654349in}{3.060755in}}{\pgfqpoint{1.660173in}{3.054931in}}%
\pgfpathcurveto{\pgfqpoint{1.665996in}{3.049107in}}{\pgfqpoint{1.673897in}{3.045835in}}{\pgfqpoint{1.682133in}{3.045835in}}%
\pgfpathclose%
\pgfusepath{stroke,fill}%
\end{pgfscope}%
\begin{pgfscope}%
\pgfpathrectangle{\pgfqpoint{0.100000in}{0.220728in}}{\pgfqpoint{3.696000in}{3.696000in}}%
\pgfusepath{clip}%
\pgfsetbuttcap%
\pgfsetroundjoin%
\definecolor{currentfill}{rgb}{0.121569,0.466667,0.705882}%
\pgfsetfillcolor{currentfill}%
\pgfsetfillopacity{0.334331}%
\pgfsetlinewidth{1.003750pt}%
\definecolor{currentstroke}{rgb}{0.121569,0.466667,0.705882}%
\pgfsetstrokecolor{currentstroke}%
\pgfsetstrokeopacity{0.334331}%
\pgfsetdash{}{0pt}%
\pgfpathmoveto{\pgfqpoint{1.906692in}{3.248944in}}%
\pgfpathcurveto{\pgfqpoint{1.914929in}{3.248944in}}{\pgfqpoint{1.922829in}{3.252217in}}{\pgfqpoint{1.928653in}{3.258040in}}%
\pgfpathcurveto{\pgfqpoint{1.934477in}{3.263864in}}{\pgfqpoint{1.937749in}{3.271764in}}{\pgfqpoint{1.937749in}{3.280001in}}%
\pgfpathcurveto{\pgfqpoint{1.937749in}{3.288237in}}{\pgfqpoint{1.934477in}{3.296137in}}{\pgfqpoint{1.928653in}{3.301961in}}%
\pgfpathcurveto{\pgfqpoint{1.922829in}{3.307785in}}{\pgfqpoint{1.914929in}{3.311057in}}{\pgfqpoint{1.906692in}{3.311057in}}%
\pgfpathcurveto{\pgfqpoint{1.898456in}{3.311057in}}{\pgfqpoint{1.890556in}{3.307785in}}{\pgfqpoint{1.884732in}{3.301961in}}%
\pgfpathcurveto{\pgfqpoint{1.878908in}{3.296137in}}{\pgfqpoint{1.875636in}{3.288237in}}{\pgfqpoint{1.875636in}{3.280001in}}%
\pgfpathcurveto{\pgfqpoint{1.875636in}{3.271764in}}{\pgfqpoint{1.878908in}{3.263864in}}{\pgfqpoint{1.884732in}{3.258040in}}%
\pgfpathcurveto{\pgfqpoint{1.890556in}{3.252217in}}{\pgfqpoint{1.898456in}{3.248944in}}{\pgfqpoint{1.906692in}{3.248944in}}%
\pgfpathclose%
\pgfusepath{stroke,fill}%
\end{pgfscope}%
\begin{pgfscope}%
\pgfpathrectangle{\pgfqpoint{0.100000in}{0.220728in}}{\pgfqpoint{3.696000in}{3.696000in}}%
\pgfusepath{clip}%
\pgfsetbuttcap%
\pgfsetroundjoin%
\definecolor{currentfill}{rgb}{0.121569,0.466667,0.705882}%
\pgfsetfillcolor{currentfill}%
\pgfsetfillopacity{0.334836}%
\pgfsetlinewidth{1.003750pt}%
\definecolor{currentstroke}{rgb}{0.121569,0.466667,0.705882}%
\pgfsetstrokecolor{currentstroke}%
\pgfsetstrokeopacity{0.334836}%
\pgfsetdash{}{0pt}%
\pgfpathmoveto{\pgfqpoint{1.681508in}{3.039923in}}%
\pgfpathcurveto{\pgfqpoint{1.689744in}{3.039923in}}{\pgfqpoint{1.697644in}{3.043196in}}{\pgfqpoint{1.703468in}{3.049019in}}%
\pgfpathcurveto{\pgfqpoint{1.709292in}{3.054843in}}{\pgfqpoint{1.712564in}{3.062743in}}{\pgfqpoint{1.712564in}{3.070980in}}%
\pgfpathcurveto{\pgfqpoint{1.712564in}{3.079216in}}{\pgfqpoint{1.709292in}{3.087116in}}{\pgfqpoint{1.703468in}{3.092940in}}%
\pgfpathcurveto{\pgfqpoint{1.697644in}{3.098764in}}{\pgfqpoint{1.689744in}{3.102036in}}{\pgfqpoint{1.681508in}{3.102036in}}%
\pgfpathcurveto{\pgfqpoint{1.673272in}{3.102036in}}{\pgfqpoint{1.665371in}{3.098764in}}{\pgfqpoint{1.659548in}{3.092940in}}%
\pgfpathcurveto{\pgfqpoint{1.653724in}{3.087116in}}{\pgfqpoint{1.650451in}{3.079216in}}{\pgfqpoint{1.650451in}{3.070980in}}%
\pgfpathcurveto{\pgfqpoint{1.650451in}{3.062743in}}{\pgfqpoint{1.653724in}{3.054843in}}{\pgfqpoint{1.659548in}{3.049019in}}%
\pgfpathcurveto{\pgfqpoint{1.665371in}{3.043196in}}{\pgfqpoint{1.673272in}{3.039923in}}{\pgfqpoint{1.681508in}{3.039923in}}%
\pgfpathclose%
\pgfusepath{stroke,fill}%
\end{pgfscope}%
\begin{pgfscope}%
\pgfpathrectangle{\pgfqpoint{0.100000in}{0.220728in}}{\pgfqpoint{3.696000in}{3.696000in}}%
\pgfusepath{clip}%
\pgfsetbuttcap%
\pgfsetroundjoin%
\definecolor{currentfill}{rgb}{0.121569,0.466667,0.705882}%
\pgfsetfillcolor{currentfill}%
\pgfsetfillopacity{0.335235}%
\pgfsetlinewidth{1.003750pt}%
\definecolor{currentstroke}{rgb}{0.121569,0.466667,0.705882}%
\pgfsetstrokecolor{currentstroke}%
\pgfsetstrokeopacity{0.335235}%
\pgfsetdash{}{0pt}%
\pgfpathmoveto{\pgfqpoint{1.679590in}{3.037247in}}%
\pgfpathcurveto{\pgfqpoint{1.687827in}{3.037247in}}{\pgfqpoint{1.695727in}{3.040520in}}{\pgfqpoint{1.701550in}{3.046344in}}%
\pgfpathcurveto{\pgfqpoint{1.707374in}{3.052168in}}{\pgfqpoint{1.710647in}{3.060068in}}{\pgfqpoint{1.710647in}{3.068304in}}%
\pgfpathcurveto{\pgfqpoint{1.710647in}{3.076540in}}{\pgfqpoint{1.707374in}{3.084440in}}{\pgfqpoint{1.701550in}{3.090264in}}%
\pgfpathcurveto{\pgfqpoint{1.695727in}{3.096088in}}{\pgfqpoint{1.687827in}{3.099360in}}{\pgfqpoint{1.679590in}{3.099360in}}%
\pgfpathcurveto{\pgfqpoint{1.671354in}{3.099360in}}{\pgfqpoint{1.663454in}{3.096088in}}{\pgfqpoint{1.657630in}{3.090264in}}%
\pgfpathcurveto{\pgfqpoint{1.651806in}{3.084440in}}{\pgfqpoint{1.648534in}{3.076540in}}{\pgfqpoint{1.648534in}{3.068304in}}%
\pgfpathcurveto{\pgfqpoint{1.648534in}{3.060068in}}{\pgfqpoint{1.651806in}{3.052168in}}{\pgfqpoint{1.657630in}{3.046344in}}%
\pgfpathcurveto{\pgfqpoint{1.663454in}{3.040520in}}{\pgfqpoint{1.671354in}{3.037247in}}{\pgfqpoint{1.679590in}{3.037247in}}%
\pgfpathclose%
\pgfusepath{stroke,fill}%
\end{pgfscope}%
\begin{pgfscope}%
\pgfpathrectangle{\pgfqpoint{0.100000in}{0.220728in}}{\pgfqpoint{3.696000in}{3.696000in}}%
\pgfusepath{clip}%
\pgfsetbuttcap%
\pgfsetroundjoin%
\definecolor{currentfill}{rgb}{0.121569,0.466667,0.705882}%
\pgfsetfillcolor{currentfill}%
\pgfsetfillopacity{0.335448}%
\pgfsetlinewidth{1.003750pt}%
\definecolor{currentstroke}{rgb}{0.121569,0.466667,0.705882}%
\pgfsetstrokecolor{currentstroke}%
\pgfsetstrokeopacity{0.335448}%
\pgfsetdash{}{0pt}%
\pgfpathmoveto{\pgfqpoint{1.679341in}{3.036079in}}%
\pgfpathcurveto{\pgfqpoint{1.687578in}{3.036079in}}{\pgfqpoint{1.695478in}{3.039351in}}{\pgfqpoint{1.701302in}{3.045175in}}%
\pgfpathcurveto{\pgfqpoint{1.707126in}{3.050999in}}{\pgfqpoint{1.710398in}{3.058899in}}{\pgfqpoint{1.710398in}{3.067135in}}%
\pgfpathcurveto{\pgfqpoint{1.710398in}{3.075372in}}{\pgfqpoint{1.707126in}{3.083272in}}{\pgfqpoint{1.701302in}{3.089096in}}%
\pgfpathcurveto{\pgfqpoint{1.695478in}{3.094920in}}{\pgfqpoint{1.687578in}{3.098192in}}{\pgfqpoint{1.679341in}{3.098192in}}%
\pgfpathcurveto{\pgfqpoint{1.671105in}{3.098192in}}{\pgfqpoint{1.663205in}{3.094920in}}{\pgfqpoint{1.657381in}{3.089096in}}%
\pgfpathcurveto{\pgfqpoint{1.651557in}{3.083272in}}{\pgfqpoint{1.648285in}{3.075372in}}{\pgfqpoint{1.648285in}{3.067135in}}%
\pgfpathcurveto{\pgfqpoint{1.648285in}{3.058899in}}{\pgfqpoint{1.651557in}{3.050999in}}{\pgfqpoint{1.657381in}{3.045175in}}%
\pgfpathcurveto{\pgfqpoint{1.663205in}{3.039351in}}{\pgfqpoint{1.671105in}{3.036079in}}{\pgfqpoint{1.679341in}{3.036079in}}%
\pgfpathclose%
\pgfusepath{stroke,fill}%
\end{pgfscope}%
\begin{pgfscope}%
\pgfpathrectangle{\pgfqpoint{0.100000in}{0.220728in}}{\pgfqpoint{3.696000in}{3.696000in}}%
\pgfusepath{clip}%
\pgfsetbuttcap%
\pgfsetroundjoin%
\definecolor{currentfill}{rgb}{0.121569,0.466667,0.705882}%
\pgfsetfillcolor{currentfill}%
\pgfsetfillopacity{0.335697}%
\pgfsetlinewidth{1.003750pt}%
\definecolor{currentstroke}{rgb}{0.121569,0.466667,0.705882}%
\pgfsetstrokecolor{currentstroke}%
\pgfsetstrokeopacity{0.335697}%
\pgfsetdash{}{0pt}%
\pgfpathmoveto{\pgfqpoint{1.678008in}{3.034259in}}%
\pgfpathcurveto{\pgfqpoint{1.686244in}{3.034259in}}{\pgfqpoint{1.694144in}{3.037531in}}{\pgfqpoint{1.699968in}{3.043355in}}%
\pgfpathcurveto{\pgfqpoint{1.705792in}{3.049179in}}{\pgfqpoint{1.709065in}{3.057079in}}{\pgfqpoint{1.709065in}{3.065316in}}%
\pgfpathcurveto{\pgfqpoint{1.709065in}{3.073552in}}{\pgfqpoint{1.705792in}{3.081452in}}{\pgfqpoint{1.699968in}{3.087276in}}%
\pgfpathcurveto{\pgfqpoint{1.694144in}{3.093100in}}{\pgfqpoint{1.686244in}{3.096372in}}{\pgfqpoint{1.678008in}{3.096372in}}%
\pgfpathcurveto{\pgfqpoint{1.669772in}{3.096372in}}{\pgfqpoint{1.661872in}{3.093100in}}{\pgfqpoint{1.656048in}{3.087276in}}%
\pgfpathcurveto{\pgfqpoint{1.650224in}{3.081452in}}{\pgfqpoint{1.646952in}{3.073552in}}{\pgfqpoint{1.646952in}{3.065316in}}%
\pgfpathcurveto{\pgfqpoint{1.646952in}{3.057079in}}{\pgfqpoint{1.650224in}{3.049179in}}{\pgfqpoint{1.656048in}{3.043355in}}%
\pgfpathcurveto{\pgfqpoint{1.661872in}{3.037531in}}{\pgfqpoint{1.669772in}{3.034259in}}{\pgfqpoint{1.678008in}{3.034259in}}%
\pgfpathclose%
\pgfusepath{stroke,fill}%
\end{pgfscope}%
\begin{pgfscope}%
\pgfpathrectangle{\pgfqpoint{0.100000in}{0.220728in}}{\pgfqpoint{3.696000in}{3.696000in}}%
\pgfusepath{clip}%
\pgfsetbuttcap%
\pgfsetroundjoin%
\definecolor{currentfill}{rgb}{0.121569,0.466667,0.705882}%
\pgfsetfillcolor{currentfill}%
\pgfsetfillopacity{0.336198}%
\pgfsetlinewidth{1.003750pt}%
\definecolor{currentstroke}{rgb}{0.121569,0.466667,0.705882}%
\pgfsetstrokecolor{currentstroke}%
\pgfsetstrokeopacity{0.336198}%
\pgfsetdash{}{0pt}%
\pgfpathmoveto{\pgfqpoint{1.916328in}{3.248093in}}%
\pgfpathcurveto{\pgfqpoint{1.924565in}{3.248093in}}{\pgfqpoint{1.932465in}{3.251366in}}{\pgfqpoint{1.938289in}{3.257189in}}%
\pgfpathcurveto{\pgfqpoint{1.944112in}{3.263013in}}{\pgfqpoint{1.947385in}{3.270913in}}{\pgfqpoint{1.947385in}{3.279150in}}%
\pgfpathcurveto{\pgfqpoint{1.947385in}{3.287386in}}{\pgfqpoint{1.944112in}{3.295286in}}{\pgfqpoint{1.938289in}{3.301110in}}%
\pgfpathcurveto{\pgfqpoint{1.932465in}{3.306934in}}{\pgfqpoint{1.924565in}{3.310206in}}{\pgfqpoint{1.916328in}{3.310206in}}%
\pgfpathcurveto{\pgfqpoint{1.908092in}{3.310206in}}{\pgfqpoint{1.900192in}{3.306934in}}{\pgfqpoint{1.894368in}{3.301110in}}%
\pgfpathcurveto{\pgfqpoint{1.888544in}{3.295286in}}{\pgfqpoint{1.885272in}{3.287386in}}{\pgfqpoint{1.885272in}{3.279150in}}%
\pgfpathcurveto{\pgfqpoint{1.885272in}{3.270913in}}{\pgfqpoint{1.888544in}{3.263013in}}{\pgfqpoint{1.894368in}{3.257189in}}%
\pgfpathcurveto{\pgfqpoint{1.900192in}{3.251366in}}{\pgfqpoint{1.908092in}{3.248093in}}{\pgfqpoint{1.916328in}{3.248093in}}%
\pgfpathclose%
\pgfusepath{stroke,fill}%
\end{pgfscope}%
\begin{pgfscope}%
\pgfpathrectangle{\pgfqpoint{0.100000in}{0.220728in}}{\pgfqpoint{3.696000in}{3.696000in}}%
\pgfusepath{clip}%
\pgfsetbuttcap%
\pgfsetroundjoin%
\definecolor{currentfill}{rgb}{0.121569,0.466667,0.705882}%
\pgfsetfillcolor{currentfill}%
\pgfsetfillopacity{0.336394}%
\pgfsetlinewidth{1.003750pt}%
\definecolor{currentstroke}{rgb}{0.121569,0.466667,0.705882}%
\pgfsetstrokecolor{currentstroke}%
\pgfsetstrokeopacity{0.336394}%
\pgfsetdash{}{0pt}%
\pgfpathmoveto{\pgfqpoint{1.676618in}{3.030676in}}%
\pgfpathcurveto{\pgfqpoint{1.684854in}{3.030676in}}{\pgfqpoint{1.692755in}{3.033949in}}{\pgfqpoint{1.698578in}{3.039773in}}%
\pgfpathcurveto{\pgfqpoint{1.704402in}{3.045597in}}{\pgfqpoint{1.707675in}{3.053497in}}{\pgfqpoint{1.707675in}{3.061733in}}%
\pgfpathcurveto{\pgfqpoint{1.707675in}{3.069969in}}{\pgfqpoint{1.704402in}{3.077869in}}{\pgfqpoint{1.698578in}{3.083693in}}%
\pgfpathcurveto{\pgfqpoint{1.692755in}{3.089517in}}{\pgfqpoint{1.684854in}{3.092789in}}{\pgfqpoint{1.676618in}{3.092789in}}%
\pgfpathcurveto{\pgfqpoint{1.668382in}{3.092789in}}{\pgfqpoint{1.660482in}{3.089517in}}{\pgfqpoint{1.654658in}{3.083693in}}%
\pgfpathcurveto{\pgfqpoint{1.648834in}{3.077869in}}{\pgfqpoint{1.645562in}{3.069969in}}{\pgfqpoint{1.645562in}{3.061733in}}%
\pgfpathcurveto{\pgfqpoint{1.645562in}{3.053497in}}{\pgfqpoint{1.648834in}{3.045597in}}{\pgfqpoint{1.654658in}{3.039773in}}%
\pgfpathcurveto{\pgfqpoint{1.660482in}{3.033949in}}{\pgfqpoint{1.668382in}{3.030676in}}{\pgfqpoint{1.676618in}{3.030676in}}%
\pgfpathclose%
\pgfusepath{stroke,fill}%
\end{pgfscope}%
\begin{pgfscope}%
\pgfpathrectangle{\pgfqpoint{0.100000in}{0.220728in}}{\pgfqpoint{3.696000in}{3.696000in}}%
\pgfusepath{clip}%
\pgfsetbuttcap%
\pgfsetroundjoin%
\definecolor{currentfill}{rgb}{0.121569,0.466667,0.705882}%
\pgfsetfillcolor{currentfill}%
\pgfsetfillopacity{0.337661}%
\pgfsetlinewidth{1.003750pt}%
\definecolor{currentstroke}{rgb}{0.121569,0.466667,0.705882}%
\pgfsetstrokecolor{currentstroke}%
\pgfsetstrokeopacity{0.337661}%
\pgfsetdash{}{0pt}%
\pgfpathmoveto{\pgfqpoint{1.674006in}{3.024224in}}%
\pgfpathcurveto{\pgfqpoint{1.682243in}{3.024224in}}{\pgfqpoint{1.690143in}{3.027496in}}{\pgfqpoint{1.695967in}{3.033320in}}%
\pgfpathcurveto{\pgfqpoint{1.701791in}{3.039144in}}{\pgfqpoint{1.705063in}{3.047044in}}{\pgfqpoint{1.705063in}{3.055281in}}%
\pgfpathcurveto{\pgfqpoint{1.705063in}{3.063517in}}{\pgfqpoint{1.701791in}{3.071417in}}{\pgfqpoint{1.695967in}{3.077241in}}%
\pgfpathcurveto{\pgfqpoint{1.690143in}{3.083065in}}{\pgfqpoint{1.682243in}{3.086337in}}{\pgfqpoint{1.674006in}{3.086337in}}%
\pgfpathcurveto{\pgfqpoint{1.665770in}{3.086337in}}{\pgfqpoint{1.657870in}{3.083065in}}{\pgfqpoint{1.652046in}{3.077241in}}%
\pgfpathcurveto{\pgfqpoint{1.646222in}{3.071417in}}{\pgfqpoint{1.642950in}{3.063517in}}{\pgfqpoint{1.642950in}{3.055281in}}%
\pgfpathcurveto{\pgfqpoint{1.642950in}{3.047044in}}{\pgfqpoint{1.646222in}{3.039144in}}{\pgfqpoint{1.652046in}{3.033320in}}%
\pgfpathcurveto{\pgfqpoint{1.657870in}{3.027496in}}{\pgfqpoint{1.665770in}{3.024224in}}{\pgfqpoint{1.674006in}{3.024224in}}%
\pgfpathclose%
\pgfusepath{stroke,fill}%
\end{pgfscope}%
\begin{pgfscope}%
\pgfpathrectangle{\pgfqpoint{0.100000in}{0.220728in}}{\pgfqpoint{3.696000in}{3.696000in}}%
\pgfusepath{clip}%
\pgfsetbuttcap%
\pgfsetroundjoin%
\definecolor{currentfill}{rgb}{0.121569,0.466667,0.705882}%
\pgfsetfillcolor{currentfill}%
\pgfsetfillopacity{0.339383}%
\pgfsetlinewidth{1.003750pt}%
\definecolor{currentstroke}{rgb}{0.121569,0.466667,0.705882}%
\pgfsetstrokecolor{currentstroke}%
\pgfsetstrokeopacity{0.339383}%
\pgfsetdash{}{0pt}%
\pgfpathmoveto{\pgfqpoint{1.666725in}{3.012926in}}%
\pgfpathcurveto{\pgfqpoint{1.674961in}{3.012926in}}{\pgfqpoint{1.682861in}{3.016198in}}{\pgfqpoint{1.688685in}{3.022022in}}%
\pgfpathcurveto{\pgfqpoint{1.694509in}{3.027846in}}{\pgfqpoint{1.697781in}{3.035746in}}{\pgfqpoint{1.697781in}{3.043982in}}%
\pgfpathcurveto{\pgfqpoint{1.697781in}{3.052219in}}{\pgfqpoint{1.694509in}{3.060119in}}{\pgfqpoint{1.688685in}{3.065943in}}%
\pgfpathcurveto{\pgfqpoint{1.682861in}{3.071766in}}{\pgfqpoint{1.674961in}{3.075039in}}{\pgfqpoint{1.666725in}{3.075039in}}%
\pgfpathcurveto{\pgfqpoint{1.658488in}{3.075039in}}{\pgfqpoint{1.650588in}{3.071766in}}{\pgfqpoint{1.644764in}{3.065943in}}%
\pgfpathcurveto{\pgfqpoint{1.638940in}{3.060119in}}{\pgfqpoint{1.635668in}{3.052219in}}{\pgfqpoint{1.635668in}{3.043982in}}%
\pgfpathcurveto{\pgfqpoint{1.635668in}{3.035746in}}{\pgfqpoint{1.638940in}{3.027846in}}{\pgfqpoint{1.644764in}{3.022022in}}%
\pgfpathcurveto{\pgfqpoint{1.650588in}{3.016198in}}{\pgfqpoint{1.658488in}{3.012926in}}{\pgfqpoint{1.666725in}{3.012926in}}%
\pgfpathclose%
\pgfusepath{stroke,fill}%
\end{pgfscope}%
\begin{pgfscope}%
\pgfpathrectangle{\pgfqpoint{0.100000in}{0.220728in}}{\pgfqpoint{3.696000in}{3.696000in}}%
\pgfusepath{clip}%
\pgfsetbuttcap%
\pgfsetroundjoin%
\definecolor{currentfill}{rgb}{0.121569,0.466667,0.705882}%
\pgfsetfillcolor{currentfill}%
\pgfsetfillopacity{0.340070}%
\pgfsetlinewidth{1.003750pt}%
\definecolor{currentstroke}{rgb}{0.121569,0.466667,0.705882}%
\pgfsetstrokecolor{currentstroke}%
\pgfsetstrokeopacity{0.340070}%
\pgfsetdash{}{0pt}%
\pgfpathmoveto{\pgfqpoint{1.928731in}{3.247499in}}%
\pgfpathcurveto{\pgfqpoint{1.936968in}{3.247499in}}{\pgfqpoint{1.944868in}{3.250771in}}{\pgfqpoint{1.950692in}{3.256595in}}%
\pgfpathcurveto{\pgfqpoint{1.956516in}{3.262419in}}{\pgfqpoint{1.959788in}{3.270319in}}{\pgfqpoint{1.959788in}{3.278555in}}%
\pgfpathcurveto{\pgfqpoint{1.959788in}{3.286792in}}{\pgfqpoint{1.956516in}{3.294692in}}{\pgfqpoint{1.950692in}{3.300516in}}%
\pgfpathcurveto{\pgfqpoint{1.944868in}{3.306340in}}{\pgfqpoint{1.936968in}{3.309612in}}{\pgfqpoint{1.928731in}{3.309612in}}%
\pgfpathcurveto{\pgfqpoint{1.920495in}{3.309612in}}{\pgfqpoint{1.912595in}{3.306340in}}{\pgfqpoint{1.906771in}{3.300516in}}%
\pgfpathcurveto{\pgfqpoint{1.900947in}{3.294692in}}{\pgfqpoint{1.897675in}{3.286792in}}{\pgfqpoint{1.897675in}{3.278555in}}%
\pgfpathcurveto{\pgfqpoint{1.897675in}{3.270319in}}{\pgfqpoint{1.900947in}{3.262419in}}{\pgfqpoint{1.906771in}{3.256595in}}%
\pgfpathcurveto{\pgfqpoint{1.912595in}{3.250771in}}{\pgfqpoint{1.920495in}{3.247499in}}{\pgfqpoint{1.928731in}{3.247499in}}%
\pgfpathclose%
\pgfusepath{stroke,fill}%
\end{pgfscope}%
\begin{pgfscope}%
\pgfpathrectangle{\pgfqpoint{0.100000in}{0.220728in}}{\pgfqpoint{3.696000in}{3.696000in}}%
\pgfusepath{clip}%
\pgfsetbuttcap%
\pgfsetroundjoin%
\definecolor{currentfill}{rgb}{0.121569,0.466667,0.705882}%
\pgfsetfillcolor{currentfill}%
\pgfsetfillopacity{0.341130}%
\pgfsetlinewidth{1.003750pt}%
\definecolor{currentstroke}{rgb}{0.121569,0.466667,0.705882}%
\pgfsetstrokecolor{currentstroke}%
\pgfsetstrokeopacity{0.341130}%
\pgfsetdash{}{0pt}%
\pgfpathmoveto{\pgfqpoint{1.665544in}{3.000778in}}%
\pgfpathcurveto{\pgfqpoint{1.673780in}{3.000778in}}{\pgfqpoint{1.681680in}{3.004050in}}{\pgfqpoint{1.687504in}{3.009874in}}%
\pgfpathcurveto{\pgfqpoint{1.693328in}{3.015698in}}{\pgfqpoint{1.696601in}{3.023598in}}{\pgfqpoint{1.696601in}{3.031835in}}%
\pgfpathcurveto{\pgfqpoint{1.696601in}{3.040071in}}{\pgfqpoint{1.693328in}{3.047971in}}{\pgfqpoint{1.687504in}{3.053795in}}%
\pgfpathcurveto{\pgfqpoint{1.681680in}{3.059619in}}{\pgfqpoint{1.673780in}{3.062891in}}{\pgfqpoint{1.665544in}{3.062891in}}%
\pgfpathcurveto{\pgfqpoint{1.657308in}{3.062891in}}{\pgfqpoint{1.649408in}{3.059619in}}{\pgfqpoint{1.643584in}{3.053795in}}%
\pgfpathcurveto{\pgfqpoint{1.637760in}{3.047971in}}{\pgfqpoint{1.634488in}{3.040071in}}{\pgfqpoint{1.634488in}{3.031835in}}%
\pgfpathcurveto{\pgfqpoint{1.634488in}{3.023598in}}{\pgfqpoint{1.637760in}{3.015698in}}{\pgfqpoint{1.643584in}{3.009874in}}%
\pgfpathcurveto{\pgfqpoint{1.649408in}{3.004050in}}{\pgfqpoint{1.657308in}{3.000778in}}{\pgfqpoint{1.665544in}{3.000778in}}%
\pgfpathclose%
\pgfusepath{stroke,fill}%
\end{pgfscope}%
\begin{pgfscope}%
\pgfpathrectangle{\pgfqpoint{0.100000in}{0.220728in}}{\pgfqpoint{3.696000in}{3.696000in}}%
\pgfusepath{clip}%
\pgfsetbuttcap%
\pgfsetroundjoin%
\definecolor{currentfill}{rgb}{0.121569,0.466667,0.705882}%
\pgfsetfillcolor{currentfill}%
\pgfsetfillopacity{0.342084}%
\pgfsetlinewidth{1.003750pt}%
\definecolor{currentstroke}{rgb}{0.121569,0.466667,0.705882}%
\pgfsetstrokecolor{currentstroke}%
\pgfsetstrokeopacity{0.342084}%
\pgfsetdash{}{0pt}%
\pgfpathmoveto{\pgfqpoint{1.659573in}{2.993006in}}%
\pgfpathcurveto{\pgfqpoint{1.667809in}{2.993006in}}{\pgfqpoint{1.675709in}{2.996279in}}{\pgfqpoint{1.681533in}{3.002103in}}%
\pgfpathcurveto{\pgfqpoint{1.687357in}{3.007927in}}{\pgfqpoint{1.690629in}{3.015827in}}{\pgfqpoint{1.690629in}{3.024063in}}%
\pgfpathcurveto{\pgfqpoint{1.690629in}{3.032299in}}{\pgfqpoint{1.687357in}{3.040199in}}{\pgfqpoint{1.681533in}{3.046023in}}%
\pgfpathcurveto{\pgfqpoint{1.675709in}{3.051847in}}{\pgfqpoint{1.667809in}{3.055119in}}{\pgfqpoint{1.659573in}{3.055119in}}%
\pgfpathcurveto{\pgfqpoint{1.651337in}{3.055119in}}{\pgfqpoint{1.643437in}{3.051847in}}{\pgfqpoint{1.637613in}{3.046023in}}%
\pgfpathcurveto{\pgfqpoint{1.631789in}{3.040199in}}{\pgfqpoint{1.628516in}{3.032299in}}{\pgfqpoint{1.628516in}{3.024063in}}%
\pgfpathcurveto{\pgfqpoint{1.628516in}{3.015827in}}{\pgfqpoint{1.631789in}{3.007927in}}{\pgfqpoint{1.637613in}{3.002103in}}%
\pgfpathcurveto{\pgfqpoint{1.643437in}{2.996279in}}{\pgfqpoint{1.651337in}{2.993006in}}{\pgfqpoint{1.659573in}{2.993006in}}%
\pgfpathclose%
\pgfusepath{stroke,fill}%
\end{pgfscope}%
\begin{pgfscope}%
\pgfpathrectangle{\pgfqpoint{0.100000in}{0.220728in}}{\pgfqpoint{3.696000in}{3.696000in}}%
\pgfusepath{clip}%
\pgfsetbuttcap%
\pgfsetroundjoin%
\definecolor{currentfill}{rgb}{0.121569,0.466667,0.705882}%
\pgfsetfillcolor{currentfill}%
\pgfsetfillopacity{0.343196}%
\pgfsetlinewidth{1.003750pt}%
\definecolor{currentstroke}{rgb}{0.121569,0.466667,0.705882}%
\pgfsetstrokecolor{currentstroke}%
\pgfsetstrokeopacity{0.343196}%
\pgfsetdash{}{0pt}%
\pgfpathmoveto{\pgfqpoint{1.657646in}{2.986640in}}%
\pgfpathcurveto{\pgfqpoint{1.665882in}{2.986640in}}{\pgfqpoint{1.673782in}{2.989912in}}{\pgfqpoint{1.679606in}{2.995736in}}%
\pgfpathcurveto{\pgfqpoint{1.685430in}{3.001560in}}{\pgfqpoint{1.688702in}{3.009460in}}{\pgfqpoint{1.688702in}{3.017696in}}%
\pgfpathcurveto{\pgfqpoint{1.688702in}{3.025932in}}{\pgfqpoint{1.685430in}{3.033832in}}{\pgfqpoint{1.679606in}{3.039656in}}%
\pgfpathcurveto{\pgfqpoint{1.673782in}{3.045480in}}{\pgfqpoint{1.665882in}{3.048753in}}{\pgfqpoint{1.657646in}{3.048753in}}%
\pgfpathcurveto{\pgfqpoint{1.649409in}{3.048753in}}{\pgfqpoint{1.641509in}{3.045480in}}{\pgfqpoint{1.635685in}{3.039656in}}%
\pgfpathcurveto{\pgfqpoint{1.629861in}{3.033832in}}{\pgfqpoint{1.626589in}{3.025932in}}{\pgfqpoint{1.626589in}{3.017696in}}%
\pgfpathcurveto{\pgfqpoint{1.626589in}{3.009460in}}{\pgfqpoint{1.629861in}{3.001560in}}{\pgfqpoint{1.635685in}{2.995736in}}%
\pgfpathcurveto{\pgfqpoint{1.641509in}{2.989912in}}{\pgfqpoint{1.649409in}{2.986640in}}{\pgfqpoint{1.657646in}{2.986640in}}%
\pgfpathclose%
\pgfusepath{stroke,fill}%
\end{pgfscope}%
\begin{pgfscope}%
\pgfpathrectangle{\pgfqpoint{0.100000in}{0.220728in}}{\pgfqpoint{3.696000in}{3.696000in}}%
\pgfusepath{clip}%
\pgfsetbuttcap%
\pgfsetroundjoin%
\definecolor{currentfill}{rgb}{0.121569,0.466667,0.705882}%
\pgfsetfillcolor{currentfill}%
\pgfsetfillopacity{0.343379}%
\pgfsetlinewidth{1.003750pt}%
\definecolor{currentstroke}{rgb}{0.121569,0.466667,0.705882}%
\pgfsetstrokecolor{currentstroke}%
\pgfsetstrokeopacity{0.343379}%
\pgfsetdash{}{0pt}%
\pgfpathmoveto{\pgfqpoint{1.943685in}{3.245758in}}%
\pgfpathcurveto{\pgfqpoint{1.951921in}{3.245758in}}{\pgfqpoint{1.959821in}{3.249031in}}{\pgfqpoint{1.965645in}{3.254855in}}%
\pgfpathcurveto{\pgfqpoint{1.971469in}{3.260679in}}{\pgfqpoint{1.974741in}{3.268579in}}{\pgfqpoint{1.974741in}{3.276815in}}%
\pgfpathcurveto{\pgfqpoint{1.974741in}{3.285051in}}{\pgfqpoint{1.971469in}{3.292951in}}{\pgfqpoint{1.965645in}{3.298775in}}%
\pgfpathcurveto{\pgfqpoint{1.959821in}{3.304599in}}{\pgfqpoint{1.951921in}{3.307871in}}{\pgfqpoint{1.943685in}{3.307871in}}%
\pgfpathcurveto{\pgfqpoint{1.935448in}{3.307871in}}{\pgfqpoint{1.927548in}{3.304599in}}{\pgfqpoint{1.921724in}{3.298775in}}%
\pgfpathcurveto{\pgfqpoint{1.915900in}{3.292951in}}{\pgfqpoint{1.912628in}{3.285051in}}{\pgfqpoint{1.912628in}{3.276815in}}%
\pgfpathcurveto{\pgfqpoint{1.912628in}{3.268579in}}{\pgfqpoint{1.915900in}{3.260679in}}{\pgfqpoint{1.921724in}{3.254855in}}%
\pgfpathcurveto{\pgfqpoint{1.927548in}{3.249031in}}{\pgfqpoint{1.935448in}{3.245758in}}{\pgfqpoint{1.943685in}{3.245758in}}%
\pgfpathclose%
\pgfusepath{stroke,fill}%
\end{pgfscope}%
\begin{pgfscope}%
\pgfpathrectangle{\pgfqpoint{0.100000in}{0.220728in}}{\pgfqpoint{3.696000in}{3.696000in}}%
\pgfusepath{clip}%
\pgfsetbuttcap%
\pgfsetroundjoin%
\definecolor{currentfill}{rgb}{0.121569,0.466667,0.705882}%
\pgfsetfillcolor{currentfill}%
\pgfsetfillopacity{0.343978}%
\pgfsetlinewidth{1.003750pt}%
\definecolor{currentstroke}{rgb}{0.121569,0.466667,0.705882}%
\pgfsetstrokecolor{currentstroke}%
\pgfsetstrokeopacity{0.343978}%
\pgfsetdash{}{0pt}%
\pgfpathmoveto{\pgfqpoint{1.655386in}{2.982054in}}%
\pgfpathcurveto{\pgfqpoint{1.663623in}{2.982054in}}{\pgfqpoint{1.671523in}{2.985327in}}{\pgfqpoint{1.677347in}{2.991151in}}%
\pgfpathcurveto{\pgfqpoint{1.683171in}{2.996975in}}{\pgfqpoint{1.686443in}{3.004875in}}{\pgfqpoint{1.686443in}{3.013111in}}%
\pgfpathcurveto{\pgfqpoint{1.686443in}{3.021347in}}{\pgfqpoint{1.683171in}{3.029247in}}{\pgfqpoint{1.677347in}{3.035071in}}%
\pgfpathcurveto{\pgfqpoint{1.671523in}{3.040895in}}{\pgfqpoint{1.663623in}{3.044167in}}{\pgfqpoint{1.655386in}{3.044167in}}%
\pgfpathcurveto{\pgfqpoint{1.647150in}{3.044167in}}{\pgfqpoint{1.639250in}{3.040895in}}{\pgfqpoint{1.633426in}{3.035071in}}%
\pgfpathcurveto{\pgfqpoint{1.627602in}{3.029247in}}{\pgfqpoint{1.624330in}{3.021347in}}{\pgfqpoint{1.624330in}{3.013111in}}%
\pgfpathcurveto{\pgfqpoint{1.624330in}{3.004875in}}{\pgfqpoint{1.627602in}{2.996975in}}{\pgfqpoint{1.633426in}{2.991151in}}%
\pgfpathcurveto{\pgfqpoint{1.639250in}{2.985327in}}{\pgfqpoint{1.647150in}{2.982054in}}{\pgfqpoint{1.655386in}{2.982054in}}%
\pgfpathclose%
\pgfusepath{stroke,fill}%
\end{pgfscope}%
\begin{pgfscope}%
\pgfpathrectangle{\pgfqpoint{0.100000in}{0.220728in}}{\pgfqpoint{3.696000in}{3.696000in}}%
\pgfusepath{clip}%
\pgfsetbuttcap%
\pgfsetroundjoin%
\definecolor{currentfill}{rgb}{0.121569,0.466667,0.705882}%
\pgfsetfillcolor{currentfill}%
\pgfsetfillopacity{0.344139}%
\pgfsetlinewidth{1.003750pt}%
\definecolor{currentstroke}{rgb}{0.121569,0.466667,0.705882}%
\pgfsetstrokecolor{currentstroke}%
\pgfsetstrokeopacity{0.344139}%
\pgfsetdash{}{0pt}%
\pgfpathmoveto{\pgfqpoint{1.961842in}{3.244541in}}%
\pgfpathcurveto{\pgfqpoint{1.970078in}{3.244541in}}{\pgfqpoint{1.977978in}{3.247814in}}{\pgfqpoint{1.983802in}{3.253638in}}%
\pgfpathcurveto{\pgfqpoint{1.989626in}{3.259462in}}{\pgfqpoint{1.992898in}{3.267362in}}{\pgfqpoint{1.992898in}{3.275598in}}%
\pgfpathcurveto{\pgfqpoint{1.992898in}{3.283834in}}{\pgfqpoint{1.989626in}{3.291734in}}{\pgfqpoint{1.983802in}{3.297558in}}%
\pgfpathcurveto{\pgfqpoint{1.977978in}{3.303382in}}{\pgfqpoint{1.970078in}{3.306654in}}{\pgfqpoint{1.961842in}{3.306654in}}%
\pgfpathcurveto{\pgfqpoint{1.953605in}{3.306654in}}{\pgfqpoint{1.945705in}{3.303382in}}{\pgfqpoint{1.939881in}{3.297558in}}%
\pgfpathcurveto{\pgfqpoint{1.934057in}{3.291734in}}{\pgfqpoint{1.930785in}{3.283834in}}{\pgfqpoint{1.930785in}{3.275598in}}%
\pgfpathcurveto{\pgfqpoint{1.930785in}{3.267362in}}{\pgfqpoint{1.934057in}{3.259462in}}{\pgfqpoint{1.939881in}{3.253638in}}%
\pgfpathcurveto{\pgfqpoint{1.945705in}{3.247814in}}{\pgfqpoint{1.953605in}{3.244541in}}{\pgfqpoint{1.961842in}{3.244541in}}%
\pgfpathclose%
\pgfusepath{stroke,fill}%
\end{pgfscope}%
\begin{pgfscope}%
\pgfpathrectangle{\pgfqpoint{0.100000in}{0.220728in}}{\pgfqpoint{3.696000in}{3.696000in}}%
\pgfusepath{clip}%
\pgfsetbuttcap%
\pgfsetroundjoin%
\definecolor{currentfill}{rgb}{0.121569,0.466667,0.705882}%
\pgfsetfillcolor{currentfill}%
\pgfsetfillopacity{0.344614}%
\pgfsetlinewidth{1.003750pt}%
\definecolor{currentstroke}{rgb}{0.121569,0.466667,0.705882}%
\pgfsetstrokecolor{currentstroke}%
\pgfsetstrokeopacity{0.344614}%
\pgfsetdash{}{0pt}%
\pgfpathmoveto{\pgfqpoint{1.654163in}{2.978439in}}%
\pgfpathcurveto{\pgfqpoint{1.662400in}{2.978439in}}{\pgfqpoint{1.670300in}{2.981711in}}{\pgfqpoint{1.676124in}{2.987535in}}%
\pgfpathcurveto{\pgfqpoint{1.681948in}{2.993359in}}{\pgfqpoint{1.685220in}{3.001259in}}{\pgfqpoint{1.685220in}{3.009495in}}%
\pgfpathcurveto{\pgfqpoint{1.685220in}{3.017731in}}{\pgfqpoint{1.681948in}{3.025631in}}{\pgfqpoint{1.676124in}{3.031455in}}%
\pgfpathcurveto{\pgfqpoint{1.670300in}{3.037279in}}{\pgfqpoint{1.662400in}{3.040552in}}{\pgfqpoint{1.654163in}{3.040552in}}%
\pgfpathcurveto{\pgfqpoint{1.645927in}{3.040552in}}{\pgfqpoint{1.638027in}{3.037279in}}{\pgfqpoint{1.632203in}{3.031455in}}%
\pgfpathcurveto{\pgfqpoint{1.626379in}{3.025631in}}{\pgfqpoint{1.623107in}{3.017731in}}{\pgfqpoint{1.623107in}{3.009495in}}%
\pgfpathcurveto{\pgfqpoint{1.623107in}{3.001259in}}{\pgfqpoint{1.626379in}{2.993359in}}{\pgfqpoint{1.632203in}{2.987535in}}%
\pgfpathcurveto{\pgfqpoint{1.638027in}{2.981711in}}{\pgfqpoint{1.645927in}{2.978439in}}{\pgfqpoint{1.654163in}{2.978439in}}%
\pgfpathclose%
\pgfusepath{stroke,fill}%
\end{pgfscope}%
\begin{pgfscope}%
\pgfpathrectangle{\pgfqpoint{0.100000in}{0.220728in}}{\pgfqpoint{3.696000in}{3.696000in}}%
\pgfusepath{clip}%
\pgfsetbuttcap%
\pgfsetroundjoin%
\definecolor{currentfill}{rgb}{0.121569,0.466667,0.705882}%
\pgfsetfillcolor{currentfill}%
\pgfsetfillopacity{0.345827}%
\pgfsetlinewidth{1.003750pt}%
\definecolor{currentstroke}{rgb}{0.121569,0.466667,0.705882}%
\pgfsetstrokecolor{currentstroke}%
\pgfsetstrokeopacity{0.345827}%
\pgfsetdash{}{0pt}%
\pgfpathmoveto{\pgfqpoint{1.651875in}{2.972151in}}%
\pgfpathcurveto{\pgfqpoint{1.660111in}{2.972151in}}{\pgfqpoint{1.668011in}{2.975423in}}{\pgfqpoint{1.673835in}{2.981247in}}%
\pgfpathcurveto{\pgfqpoint{1.679659in}{2.987071in}}{\pgfqpoint{1.682931in}{2.994971in}}{\pgfqpoint{1.682931in}{3.003207in}}%
\pgfpathcurveto{\pgfqpoint{1.682931in}{3.011444in}}{\pgfqpoint{1.679659in}{3.019344in}}{\pgfqpoint{1.673835in}{3.025168in}}%
\pgfpathcurveto{\pgfqpoint{1.668011in}{3.030992in}}{\pgfqpoint{1.660111in}{3.034264in}}{\pgfqpoint{1.651875in}{3.034264in}}%
\pgfpathcurveto{\pgfqpoint{1.643638in}{3.034264in}}{\pgfqpoint{1.635738in}{3.030992in}}{\pgfqpoint{1.629914in}{3.025168in}}%
\pgfpathcurveto{\pgfqpoint{1.624090in}{3.019344in}}{\pgfqpoint{1.620818in}{3.011444in}}{\pgfqpoint{1.620818in}{3.003207in}}%
\pgfpathcurveto{\pgfqpoint{1.620818in}{2.994971in}}{\pgfqpoint{1.624090in}{2.987071in}}{\pgfqpoint{1.629914in}{2.981247in}}%
\pgfpathcurveto{\pgfqpoint{1.635738in}{2.975423in}}{\pgfqpoint{1.643638in}{2.972151in}}{\pgfqpoint{1.651875in}{2.972151in}}%
\pgfpathclose%
\pgfusepath{stroke,fill}%
\end{pgfscope}%
\begin{pgfscope}%
\pgfpathrectangle{\pgfqpoint{0.100000in}{0.220728in}}{\pgfqpoint{3.696000in}{3.696000in}}%
\pgfusepath{clip}%
\pgfsetbuttcap%
\pgfsetroundjoin%
\definecolor{currentfill}{rgb}{0.121569,0.466667,0.705882}%
\pgfsetfillcolor{currentfill}%
\pgfsetfillopacity{0.347585}%
\pgfsetlinewidth{1.003750pt}%
\definecolor{currentstroke}{rgb}{0.121569,0.466667,0.705882}%
\pgfsetstrokecolor{currentstroke}%
\pgfsetstrokeopacity{0.347585}%
\pgfsetdash{}{0pt}%
\pgfpathmoveto{\pgfqpoint{1.645314in}{2.961294in}}%
\pgfpathcurveto{\pgfqpoint{1.653550in}{2.961294in}}{\pgfqpoint{1.661450in}{2.964566in}}{\pgfqpoint{1.667274in}{2.970390in}}%
\pgfpathcurveto{\pgfqpoint{1.673098in}{2.976214in}}{\pgfqpoint{1.676371in}{2.984114in}}{\pgfqpoint{1.676371in}{2.992350in}}%
\pgfpathcurveto{\pgfqpoint{1.676371in}{3.000586in}}{\pgfqpoint{1.673098in}{3.008486in}}{\pgfqpoint{1.667274in}{3.014310in}}%
\pgfpathcurveto{\pgfqpoint{1.661450in}{3.020134in}}{\pgfqpoint{1.653550in}{3.023407in}}{\pgfqpoint{1.645314in}{3.023407in}}%
\pgfpathcurveto{\pgfqpoint{1.637078in}{3.023407in}}{\pgfqpoint{1.629178in}{3.020134in}}{\pgfqpoint{1.623354in}{3.014310in}}%
\pgfpathcurveto{\pgfqpoint{1.617530in}{3.008486in}}{\pgfqpoint{1.614258in}{3.000586in}}{\pgfqpoint{1.614258in}{2.992350in}}%
\pgfpathcurveto{\pgfqpoint{1.614258in}{2.984114in}}{\pgfqpoint{1.617530in}{2.976214in}}{\pgfqpoint{1.623354in}{2.970390in}}%
\pgfpathcurveto{\pgfqpoint{1.629178in}{2.964566in}}{\pgfqpoint{1.637078in}{2.961294in}}{\pgfqpoint{1.645314in}{2.961294in}}%
\pgfpathclose%
\pgfusepath{stroke,fill}%
\end{pgfscope}%
\begin{pgfscope}%
\pgfpathrectangle{\pgfqpoint{0.100000in}{0.220728in}}{\pgfqpoint{3.696000in}{3.696000in}}%
\pgfusepath{clip}%
\pgfsetbuttcap%
\pgfsetroundjoin%
\definecolor{currentfill}{rgb}{0.121569,0.466667,0.705882}%
\pgfsetfillcolor{currentfill}%
\pgfsetfillopacity{0.349077}%
\pgfsetlinewidth{1.003750pt}%
\definecolor{currentstroke}{rgb}{0.121569,0.466667,0.705882}%
\pgfsetstrokecolor{currentstroke}%
\pgfsetstrokeopacity{0.349077}%
\pgfsetdash{}{0pt}%
\pgfpathmoveto{\pgfqpoint{1.644025in}{2.950570in}}%
\pgfpathcurveto{\pgfqpoint{1.652261in}{2.950570in}}{\pgfqpoint{1.660161in}{2.953843in}}{\pgfqpoint{1.665985in}{2.959666in}}%
\pgfpathcurveto{\pgfqpoint{1.671809in}{2.965490in}}{\pgfqpoint{1.675082in}{2.973390in}}{\pgfqpoint{1.675082in}{2.981627in}}%
\pgfpathcurveto{\pgfqpoint{1.675082in}{2.989863in}}{\pgfqpoint{1.671809in}{2.997763in}}{\pgfqpoint{1.665985in}{3.003587in}}%
\pgfpathcurveto{\pgfqpoint{1.660161in}{3.009411in}}{\pgfqpoint{1.652261in}{3.012683in}}{\pgfqpoint{1.644025in}{3.012683in}}%
\pgfpathcurveto{\pgfqpoint{1.635789in}{3.012683in}}{\pgfqpoint{1.627889in}{3.009411in}}{\pgfqpoint{1.622065in}{3.003587in}}%
\pgfpathcurveto{\pgfqpoint{1.616241in}{2.997763in}}{\pgfqpoint{1.612969in}{2.989863in}}{\pgfqpoint{1.612969in}{2.981627in}}%
\pgfpathcurveto{\pgfqpoint{1.612969in}{2.973390in}}{\pgfqpoint{1.616241in}{2.965490in}}{\pgfqpoint{1.622065in}{2.959666in}}%
\pgfpathcurveto{\pgfqpoint{1.627889in}{2.953843in}}{\pgfqpoint{1.635789in}{2.950570in}}{\pgfqpoint{1.644025in}{2.950570in}}%
\pgfpathclose%
\pgfusepath{stroke,fill}%
\end{pgfscope}%
\begin{pgfscope}%
\pgfpathrectangle{\pgfqpoint{0.100000in}{0.220728in}}{\pgfqpoint{3.696000in}{3.696000in}}%
\pgfusepath{clip}%
\pgfsetbuttcap%
\pgfsetroundjoin%
\definecolor{currentfill}{rgb}{0.121569,0.466667,0.705882}%
\pgfsetfillcolor{currentfill}%
\pgfsetfillopacity{0.349958}%
\pgfsetlinewidth{1.003750pt}%
\definecolor{currentstroke}{rgb}{0.121569,0.466667,0.705882}%
\pgfsetstrokecolor{currentstroke}%
\pgfsetstrokeopacity{0.349958}%
\pgfsetdash{}{0pt}%
\pgfpathmoveto{\pgfqpoint{1.638861in}{2.943882in}}%
\pgfpathcurveto{\pgfqpoint{1.647097in}{2.943882in}}{\pgfqpoint{1.654998in}{2.947154in}}{\pgfqpoint{1.660821in}{2.952978in}}%
\pgfpathcurveto{\pgfqpoint{1.666645in}{2.958802in}}{\pgfqpoint{1.669918in}{2.966702in}}{\pgfqpoint{1.669918in}{2.974938in}}%
\pgfpathcurveto{\pgfqpoint{1.669918in}{2.983175in}}{\pgfqpoint{1.666645in}{2.991075in}}{\pgfqpoint{1.660821in}{2.996898in}}%
\pgfpathcurveto{\pgfqpoint{1.654998in}{3.002722in}}{\pgfqpoint{1.647097in}{3.005995in}}{\pgfqpoint{1.638861in}{3.005995in}}%
\pgfpathcurveto{\pgfqpoint{1.630625in}{3.005995in}}{\pgfqpoint{1.622725in}{3.002722in}}{\pgfqpoint{1.616901in}{2.996898in}}%
\pgfpathcurveto{\pgfqpoint{1.611077in}{2.991075in}}{\pgfqpoint{1.607805in}{2.983175in}}{\pgfqpoint{1.607805in}{2.974938in}}%
\pgfpathcurveto{\pgfqpoint{1.607805in}{2.966702in}}{\pgfqpoint{1.611077in}{2.958802in}}{\pgfqpoint{1.616901in}{2.952978in}}%
\pgfpathcurveto{\pgfqpoint{1.622725in}{2.947154in}}{\pgfqpoint{1.630625in}{2.943882in}}{\pgfqpoint{1.638861in}{2.943882in}}%
\pgfpathclose%
\pgfusepath{stroke,fill}%
\end{pgfscope}%
\begin{pgfscope}%
\pgfpathrectangle{\pgfqpoint{0.100000in}{0.220728in}}{\pgfqpoint{3.696000in}{3.696000in}}%
\pgfusepath{clip}%
\pgfsetbuttcap%
\pgfsetroundjoin%
\definecolor{currentfill}{rgb}{0.121569,0.466667,0.705882}%
\pgfsetfillcolor{currentfill}%
\pgfsetfillopacity{0.349992}%
\pgfsetlinewidth{1.003750pt}%
\definecolor{currentstroke}{rgb}{0.121569,0.466667,0.705882}%
\pgfsetstrokecolor{currentstroke}%
\pgfsetstrokeopacity{0.349992}%
\pgfsetdash{}{0pt}%
\pgfpathmoveto{\pgfqpoint{1.975723in}{3.245126in}}%
\pgfpathcurveto{\pgfqpoint{1.983960in}{3.245126in}}{\pgfqpoint{1.991860in}{3.248398in}}{\pgfqpoint{1.997684in}{3.254222in}}%
\pgfpathcurveto{\pgfqpoint{2.003507in}{3.260046in}}{\pgfqpoint{2.006780in}{3.267946in}}{\pgfqpoint{2.006780in}{3.276182in}}%
\pgfpathcurveto{\pgfqpoint{2.006780in}{3.284418in}}{\pgfqpoint{2.003507in}{3.292319in}}{\pgfqpoint{1.997684in}{3.298142in}}%
\pgfpathcurveto{\pgfqpoint{1.991860in}{3.303966in}}{\pgfqpoint{1.983960in}{3.307239in}}{\pgfqpoint{1.975723in}{3.307239in}}%
\pgfpathcurveto{\pgfqpoint{1.967487in}{3.307239in}}{\pgfqpoint{1.959587in}{3.303966in}}{\pgfqpoint{1.953763in}{3.298142in}}%
\pgfpathcurveto{\pgfqpoint{1.947939in}{3.292319in}}{\pgfqpoint{1.944667in}{3.284418in}}{\pgfqpoint{1.944667in}{3.276182in}}%
\pgfpathcurveto{\pgfqpoint{1.944667in}{3.267946in}}{\pgfqpoint{1.947939in}{3.260046in}}{\pgfqpoint{1.953763in}{3.254222in}}%
\pgfpathcurveto{\pgfqpoint{1.959587in}{3.248398in}}{\pgfqpoint{1.967487in}{3.245126in}}{\pgfqpoint{1.975723in}{3.245126in}}%
\pgfpathclose%
\pgfusepath{stroke,fill}%
\end{pgfscope}%
\begin{pgfscope}%
\pgfpathrectangle{\pgfqpoint{0.100000in}{0.220728in}}{\pgfqpoint{3.696000in}{3.696000in}}%
\pgfusepath{clip}%
\pgfsetbuttcap%
\pgfsetroundjoin%
\definecolor{currentfill}{rgb}{0.121569,0.466667,0.705882}%
\pgfsetfillcolor{currentfill}%
\pgfsetfillopacity{0.350775}%
\pgfsetlinewidth{1.003750pt}%
\definecolor{currentstroke}{rgb}{0.121569,0.466667,0.705882}%
\pgfsetstrokecolor{currentstroke}%
\pgfsetstrokeopacity{0.350775}%
\pgfsetdash{}{0pt}%
\pgfpathmoveto{\pgfqpoint{1.637940in}{2.939536in}}%
\pgfpathcurveto{\pgfqpoint{1.646176in}{2.939536in}}{\pgfqpoint{1.654076in}{2.942808in}}{\pgfqpoint{1.659900in}{2.948632in}}%
\pgfpathcurveto{\pgfqpoint{1.665724in}{2.954456in}}{\pgfqpoint{1.668996in}{2.962356in}}{\pgfqpoint{1.668996in}{2.970592in}}%
\pgfpathcurveto{\pgfqpoint{1.668996in}{2.978829in}}{\pgfqpoint{1.665724in}{2.986729in}}{\pgfqpoint{1.659900in}{2.992553in}}%
\pgfpathcurveto{\pgfqpoint{1.654076in}{2.998377in}}{\pgfqpoint{1.646176in}{3.001649in}}{\pgfqpoint{1.637940in}{3.001649in}}%
\pgfpathcurveto{\pgfqpoint{1.629704in}{3.001649in}}{\pgfqpoint{1.621804in}{2.998377in}}{\pgfqpoint{1.615980in}{2.992553in}}%
\pgfpathcurveto{\pgfqpoint{1.610156in}{2.986729in}}{\pgfqpoint{1.606883in}{2.978829in}}{\pgfqpoint{1.606883in}{2.970592in}}%
\pgfpathcurveto{\pgfqpoint{1.606883in}{2.962356in}}{\pgfqpoint{1.610156in}{2.954456in}}{\pgfqpoint{1.615980in}{2.948632in}}%
\pgfpathcurveto{\pgfqpoint{1.621804in}{2.942808in}}{\pgfqpoint{1.629704in}{2.939536in}}{\pgfqpoint{1.637940in}{2.939536in}}%
\pgfpathclose%
\pgfusepath{stroke,fill}%
\end{pgfscope}%
\begin{pgfscope}%
\pgfpathrectangle{\pgfqpoint{0.100000in}{0.220728in}}{\pgfqpoint{3.696000in}{3.696000in}}%
\pgfusepath{clip}%
\pgfsetbuttcap%
\pgfsetroundjoin%
\definecolor{currentfill}{rgb}{0.121569,0.466667,0.705882}%
\pgfsetfillcolor{currentfill}%
\pgfsetfillopacity{0.352018}%
\pgfsetlinewidth{1.003750pt}%
\definecolor{currentstroke}{rgb}{0.121569,0.466667,0.705882}%
\pgfsetstrokecolor{currentstroke}%
\pgfsetstrokeopacity{0.352018}%
\pgfsetdash{}{0pt}%
\pgfpathmoveto{\pgfqpoint{1.633573in}{2.932862in}}%
\pgfpathcurveto{\pgfqpoint{1.641809in}{2.932862in}}{\pgfqpoint{1.649709in}{2.936134in}}{\pgfqpoint{1.655533in}{2.941958in}}%
\pgfpathcurveto{\pgfqpoint{1.661357in}{2.947782in}}{\pgfqpoint{1.664629in}{2.955682in}}{\pgfqpoint{1.664629in}{2.963918in}}%
\pgfpathcurveto{\pgfqpoint{1.664629in}{2.972155in}}{\pgfqpoint{1.661357in}{2.980055in}}{\pgfqpoint{1.655533in}{2.985879in}}%
\pgfpathcurveto{\pgfqpoint{1.649709in}{2.991703in}}{\pgfqpoint{1.641809in}{2.994975in}}{\pgfqpoint{1.633573in}{2.994975in}}%
\pgfpathcurveto{\pgfqpoint{1.625337in}{2.994975in}}{\pgfqpoint{1.617437in}{2.991703in}}{\pgfqpoint{1.611613in}{2.985879in}}%
\pgfpathcurveto{\pgfqpoint{1.605789in}{2.980055in}}{\pgfqpoint{1.602516in}{2.972155in}}{\pgfqpoint{1.602516in}{2.963918in}}%
\pgfpathcurveto{\pgfqpoint{1.602516in}{2.955682in}}{\pgfqpoint{1.605789in}{2.947782in}}{\pgfqpoint{1.611613in}{2.941958in}}%
\pgfpathcurveto{\pgfqpoint{1.617437in}{2.936134in}}{\pgfqpoint{1.625337in}{2.932862in}}{\pgfqpoint{1.633573in}{2.932862in}}%
\pgfpathclose%
\pgfusepath{stroke,fill}%
\end{pgfscope}%
\begin{pgfscope}%
\pgfpathrectangle{\pgfqpoint{0.100000in}{0.220728in}}{\pgfqpoint{3.696000in}{3.696000in}}%
\pgfusepath{clip}%
\pgfsetbuttcap%
\pgfsetroundjoin%
\definecolor{currentfill}{rgb}{0.121569,0.466667,0.705882}%
\pgfsetfillcolor{currentfill}%
\pgfsetfillopacity{0.352025}%
\pgfsetlinewidth{1.003750pt}%
\definecolor{currentstroke}{rgb}{0.121569,0.466667,0.705882}%
\pgfsetstrokecolor{currentstroke}%
\pgfsetstrokeopacity{0.352025}%
\pgfsetdash{}{0pt}%
\pgfpathmoveto{\pgfqpoint{1.996067in}{3.240046in}}%
\pgfpathcurveto{\pgfqpoint{2.004303in}{3.240046in}}{\pgfqpoint{2.012203in}{3.243318in}}{\pgfqpoint{2.018027in}{3.249142in}}%
\pgfpathcurveto{\pgfqpoint{2.023851in}{3.254966in}}{\pgfqpoint{2.027123in}{3.262866in}}{\pgfqpoint{2.027123in}{3.271102in}}%
\pgfpathcurveto{\pgfqpoint{2.027123in}{3.279339in}}{\pgfqpoint{2.023851in}{3.287239in}}{\pgfqpoint{2.018027in}{3.293063in}}%
\pgfpathcurveto{\pgfqpoint{2.012203in}{3.298887in}}{\pgfqpoint{2.004303in}{3.302159in}}{\pgfqpoint{1.996067in}{3.302159in}}%
\pgfpathcurveto{\pgfqpoint{1.987830in}{3.302159in}}{\pgfqpoint{1.979930in}{3.298887in}}{\pgfqpoint{1.974106in}{3.293063in}}%
\pgfpathcurveto{\pgfqpoint{1.968283in}{3.287239in}}{\pgfqpoint{1.965010in}{3.279339in}}{\pgfqpoint{1.965010in}{3.271102in}}%
\pgfpathcurveto{\pgfqpoint{1.965010in}{3.262866in}}{\pgfqpoint{1.968283in}{3.254966in}}{\pgfqpoint{1.974106in}{3.249142in}}%
\pgfpathcurveto{\pgfqpoint{1.979930in}{3.243318in}}{\pgfqpoint{1.987830in}{3.240046in}}{\pgfqpoint{1.996067in}{3.240046in}}%
\pgfpathclose%
\pgfusepath{stroke,fill}%
\end{pgfscope}%
\begin{pgfscope}%
\pgfpathrectangle{\pgfqpoint{0.100000in}{0.220728in}}{\pgfqpoint{3.696000in}{3.696000in}}%
\pgfusepath{clip}%
\pgfsetbuttcap%
\pgfsetroundjoin%
\definecolor{currentfill}{rgb}{0.121569,0.466667,0.705882}%
\pgfsetfillcolor{currentfill}%
\pgfsetfillopacity{0.354598}%
\pgfsetlinewidth{1.003750pt}%
\definecolor{currentstroke}{rgb}{0.121569,0.466667,0.705882}%
\pgfsetstrokecolor{currentstroke}%
\pgfsetstrokeopacity{0.354598}%
\pgfsetdash{}{0pt}%
\pgfpathmoveto{\pgfqpoint{1.627944in}{2.919443in}}%
\pgfpathcurveto{\pgfqpoint{1.636180in}{2.919443in}}{\pgfqpoint{1.644080in}{2.922716in}}{\pgfqpoint{1.649904in}{2.928540in}}%
\pgfpathcurveto{\pgfqpoint{1.655728in}{2.934364in}}{\pgfqpoint{1.659000in}{2.942264in}}{\pgfqpoint{1.659000in}{2.950500in}}%
\pgfpathcurveto{\pgfqpoint{1.659000in}{2.958736in}}{\pgfqpoint{1.655728in}{2.966636in}}{\pgfqpoint{1.649904in}{2.972460in}}%
\pgfpathcurveto{\pgfqpoint{1.644080in}{2.978284in}}{\pgfqpoint{1.636180in}{2.981556in}}{\pgfqpoint{1.627944in}{2.981556in}}%
\pgfpathcurveto{\pgfqpoint{1.619707in}{2.981556in}}{\pgfqpoint{1.611807in}{2.978284in}}{\pgfqpoint{1.605984in}{2.972460in}}%
\pgfpathcurveto{\pgfqpoint{1.600160in}{2.966636in}}{\pgfqpoint{1.596887in}{2.958736in}}{\pgfqpoint{1.596887in}{2.950500in}}%
\pgfpathcurveto{\pgfqpoint{1.596887in}{2.942264in}}{\pgfqpoint{1.600160in}{2.934364in}}{\pgfqpoint{1.605984in}{2.928540in}}%
\pgfpathcurveto{\pgfqpoint{1.611807in}{2.922716in}}{\pgfqpoint{1.619707in}{2.919443in}}{\pgfqpoint{1.627944in}{2.919443in}}%
\pgfpathclose%
\pgfusepath{stroke,fill}%
\end{pgfscope}%
\begin{pgfscope}%
\pgfpathrectangle{\pgfqpoint{0.100000in}{0.220728in}}{\pgfqpoint{3.696000in}{3.696000in}}%
\pgfusepath{clip}%
\pgfsetbuttcap%
\pgfsetroundjoin%
\definecolor{currentfill}{rgb}{0.121569,0.466667,0.705882}%
\pgfsetfillcolor{currentfill}%
\pgfsetfillopacity{0.354869}%
\pgfsetlinewidth{1.003750pt}%
\definecolor{currentstroke}{rgb}{0.121569,0.466667,0.705882}%
\pgfsetstrokecolor{currentstroke}%
\pgfsetstrokeopacity{0.354869}%
\pgfsetdash{}{0pt}%
\pgfpathmoveto{\pgfqpoint{2.005360in}{3.237935in}}%
\pgfpathcurveto{\pgfqpoint{2.013596in}{3.237935in}}{\pgfqpoint{2.021496in}{3.241207in}}{\pgfqpoint{2.027320in}{3.247031in}}%
\pgfpathcurveto{\pgfqpoint{2.033144in}{3.252855in}}{\pgfqpoint{2.036417in}{3.260755in}}{\pgfqpoint{2.036417in}{3.268992in}}%
\pgfpathcurveto{\pgfqpoint{2.036417in}{3.277228in}}{\pgfqpoint{2.033144in}{3.285128in}}{\pgfqpoint{2.027320in}{3.290952in}}%
\pgfpathcurveto{\pgfqpoint{2.021496in}{3.296776in}}{\pgfqpoint{2.013596in}{3.300048in}}{\pgfqpoint{2.005360in}{3.300048in}}%
\pgfpathcurveto{\pgfqpoint{1.997124in}{3.300048in}}{\pgfqpoint{1.989224in}{3.296776in}}{\pgfqpoint{1.983400in}{3.290952in}}%
\pgfpathcurveto{\pgfqpoint{1.977576in}{3.285128in}}{\pgfqpoint{1.974304in}{3.277228in}}{\pgfqpoint{1.974304in}{3.268992in}}%
\pgfpathcurveto{\pgfqpoint{1.974304in}{3.260755in}}{\pgfqpoint{1.977576in}{3.252855in}}{\pgfqpoint{1.983400in}{3.247031in}}%
\pgfpathcurveto{\pgfqpoint{1.989224in}{3.241207in}}{\pgfqpoint{1.997124in}{3.237935in}}{\pgfqpoint{2.005360in}{3.237935in}}%
\pgfpathclose%
\pgfusepath{stroke,fill}%
\end{pgfscope}%
\begin{pgfscope}%
\pgfpathrectangle{\pgfqpoint{0.100000in}{0.220728in}}{\pgfqpoint{3.696000in}{3.696000in}}%
\pgfusepath{clip}%
\pgfsetbuttcap%
\pgfsetroundjoin%
\definecolor{currentfill}{rgb}{0.121569,0.466667,0.705882}%
\pgfsetfillcolor{currentfill}%
\pgfsetfillopacity{0.357555}%
\pgfsetlinewidth{1.003750pt}%
\definecolor{currentstroke}{rgb}{0.121569,0.466667,0.705882}%
\pgfsetstrokecolor{currentstroke}%
\pgfsetstrokeopacity{0.357555}%
\pgfsetdash{}{0pt}%
\pgfpathmoveto{\pgfqpoint{2.016257in}{3.236382in}}%
\pgfpathcurveto{\pgfqpoint{2.024493in}{3.236382in}}{\pgfqpoint{2.032393in}{3.239654in}}{\pgfqpoint{2.038217in}{3.245478in}}%
\pgfpathcurveto{\pgfqpoint{2.044041in}{3.251302in}}{\pgfqpoint{2.047313in}{3.259202in}}{\pgfqpoint{2.047313in}{3.267438in}}%
\pgfpathcurveto{\pgfqpoint{2.047313in}{3.275674in}}{\pgfqpoint{2.044041in}{3.283575in}}{\pgfqpoint{2.038217in}{3.289398in}}%
\pgfpathcurveto{\pgfqpoint{2.032393in}{3.295222in}}{\pgfqpoint{2.024493in}{3.298495in}}{\pgfqpoint{2.016257in}{3.298495in}}%
\pgfpathcurveto{\pgfqpoint{2.008020in}{3.298495in}}{\pgfqpoint{2.000120in}{3.295222in}}{\pgfqpoint{1.994296in}{3.289398in}}%
\pgfpathcurveto{\pgfqpoint{1.988472in}{3.283575in}}{\pgfqpoint{1.985200in}{3.275674in}}{\pgfqpoint{1.985200in}{3.267438in}}%
\pgfpathcurveto{\pgfqpoint{1.985200in}{3.259202in}}{\pgfqpoint{1.988472in}{3.251302in}}{\pgfqpoint{1.994296in}{3.245478in}}%
\pgfpathcurveto{\pgfqpoint{2.000120in}{3.239654in}}{\pgfqpoint{2.008020in}{3.236382in}}{\pgfqpoint{2.016257in}{3.236382in}}%
\pgfpathclose%
\pgfusepath{stroke,fill}%
\end{pgfscope}%
\begin{pgfscope}%
\pgfpathrectangle{\pgfqpoint{0.100000in}{0.220728in}}{\pgfqpoint{3.696000in}{3.696000in}}%
\pgfusepath{clip}%
\pgfsetbuttcap%
\pgfsetroundjoin%
\definecolor{currentfill}{rgb}{0.121569,0.466667,0.705882}%
\pgfsetfillcolor{currentfill}%
\pgfsetfillopacity{0.359233}%
\pgfsetlinewidth{1.003750pt}%
\definecolor{currentstroke}{rgb}{0.121569,0.466667,0.705882}%
\pgfsetstrokecolor{currentstroke}%
\pgfsetstrokeopacity{0.359233}%
\pgfsetdash{}{0pt}%
\pgfpathmoveto{\pgfqpoint{2.022034in}{3.235726in}}%
\pgfpathcurveto{\pgfqpoint{2.030270in}{3.235726in}}{\pgfqpoint{2.038171in}{3.238998in}}{\pgfqpoint{2.043994in}{3.244822in}}%
\pgfpathcurveto{\pgfqpoint{2.049818in}{3.250646in}}{\pgfqpoint{2.053091in}{3.258546in}}{\pgfqpoint{2.053091in}{3.266782in}}%
\pgfpathcurveto{\pgfqpoint{2.053091in}{3.275019in}}{\pgfqpoint{2.049818in}{3.282919in}}{\pgfqpoint{2.043994in}{3.288743in}}%
\pgfpathcurveto{\pgfqpoint{2.038171in}{3.294567in}}{\pgfqpoint{2.030270in}{3.297839in}}{\pgfqpoint{2.022034in}{3.297839in}}%
\pgfpathcurveto{\pgfqpoint{2.013798in}{3.297839in}}{\pgfqpoint{2.005898in}{3.294567in}}{\pgfqpoint{2.000074in}{3.288743in}}%
\pgfpathcurveto{\pgfqpoint{1.994250in}{3.282919in}}{\pgfqpoint{1.990978in}{3.275019in}}{\pgfqpoint{1.990978in}{3.266782in}}%
\pgfpathcurveto{\pgfqpoint{1.990978in}{3.258546in}}{\pgfqpoint{1.994250in}{3.250646in}}{\pgfqpoint{2.000074in}{3.244822in}}%
\pgfpathcurveto{\pgfqpoint{2.005898in}{3.238998in}}{\pgfqpoint{2.013798in}{3.235726in}}{\pgfqpoint{2.022034in}{3.235726in}}%
\pgfpathclose%
\pgfusepath{stroke,fill}%
\end{pgfscope}%
\begin{pgfscope}%
\pgfpathrectangle{\pgfqpoint{0.100000in}{0.220728in}}{\pgfqpoint{3.696000in}{3.696000in}}%
\pgfusepath{clip}%
\pgfsetbuttcap%
\pgfsetroundjoin%
\definecolor{currentfill}{rgb}{0.121569,0.466667,0.705882}%
\pgfsetfillcolor{currentfill}%
\pgfsetfillopacity{0.359281}%
\pgfsetlinewidth{1.003750pt}%
\definecolor{currentstroke}{rgb}{0.121569,0.466667,0.705882}%
\pgfsetstrokecolor{currentstroke}%
\pgfsetstrokeopacity{0.359281}%
\pgfsetdash{}{0pt}%
\pgfpathmoveto{\pgfqpoint{1.618357in}{2.894448in}}%
\pgfpathcurveto{\pgfqpoint{1.626593in}{2.894448in}}{\pgfqpoint{1.634493in}{2.897720in}}{\pgfqpoint{1.640317in}{2.903544in}}%
\pgfpathcurveto{\pgfqpoint{1.646141in}{2.909368in}}{\pgfqpoint{1.649414in}{2.917268in}}{\pgfqpoint{1.649414in}{2.925504in}}%
\pgfpathcurveto{\pgfqpoint{1.649414in}{2.933740in}}{\pgfqpoint{1.646141in}{2.941640in}}{\pgfqpoint{1.640317in}{2.947464in}}%
\pgfpathcurveto{\pgfqpoint{1.634493in}{2.953288in}}{\pgfqpoint{1.626593in}{2.956561in}}{\pgfqpoint{1.618357in}{2.956561in}}%
\pgfpathcurveto{\pgfqpoint{1.610121in}{2.956561in}}{\pgfqpoint{1.602221in}{2.953288in}}{\pgfqpoint{1.596397in}{2.947464in}}%
\pgfpathcurveto{\pgfqpoint{1.590573in}{2.941640in}}{\pgfqpoint{1.587301in}{2.933740in}}{\pgfqpoint{1.587301in}{2.925504in}}%
\pgfpathcurveto{\pgfqpoint{1.587301in}{2.917268in}}{\pgfqpoint{1.590573in}{2.909368in}}{\pgfqpoint{1.596397in}{2.903544in}}%
\pgfpathcurveto{\pgfqpoint{1.602221in}{2.897720in}}{\pgfqpoint{1.610121in}{2.894448in}}{\pgfqpoint{1.618357in}{2.894448in}}%
\pgfpathclose%
\pgfusepath{stroke,fill}%
\end{pgfscope}%
\begin{pgfscope}%
\pgfpathrectangle{\pgfqpoint{0.100000in}{0.220728in}}{\pgfqpoint{3.696000in}{3.696000in}}%
\pgfusepath{clip}%
\pgfsetbuttcap%
\pgfsetroundjoin%
\definecolor{currentfill}{rgb}{0.121569,0.466667,0.705882}%
\pgfsetfillcolor{currentfill}%
\pgfsetfillopacity{0.361961}%
\pgfsetlinewidth{1.003750pt}%
\definecolor{currentstroke}{rgb}{0.121569,0.466667,0.705882}%
\pgfsetstrokecolor{currentstroke}%
\pgfsetstrokeopacity{0.361961}%
\pgfsetdash{}{0pt}%
\pgfpathmoveto{\pgfqpoint{2.029510in}{3.234128in}}%
\pgfpathcurveto{\pgfqpoint{2.037746in}{3.234128in}}{\pgfqpoint{2.045646in}{3.237400in}}{\pgfqpoint{2.051470in}{3.243224in}}%
\pgfpathcurveto{\pgfqpoint{2.057294in}{3.249048in}}{\pgfqpoint{2.060566in}{3.256948in}}{\pgfqpoint{2.060566in}{3.265184in}}%
\pgfpathcurveto{\pgfqpoint{2.060566in}{3.273421in}}{\pgfqpoint{2.057294in}{3.281321in}}{\pgfqpoint{2.051470in}{3.287145in}}%
\pgfpathcurveto{\pgfqpoint{2.045646in}{3.292969in}}{\pgfqpoint{2.037746in}{3.296241in}}{\pgfqpoint{2.029510in}{3.296241in}}%
\pgfpathcurveto{\pgfqpoint{2.021274in}{3.296241in}}{\pgfqpoint{2.013373in}{3.292969in}}{\pgfqpoint{2.007550in}{3.287145in}}%
\pgfpathcurveto{\pgfqpoint{2.001726in}{3.281321in}}{\pgfqpoint{1.998453in}{3.273421in}}{\pgfqpoint{1.998453in}{3.265184in}}%
\pgfpathcurveto{\pgfqpoint{1.998453in}{3.256948in}}{\pgfqpoint{2.001726in}{3.249048in}}{\pgfqpoint{2.007550in}{3.243224in}}%
\pgfpathcurveto{\pgfqpoint{2.013373in}{3.237400in}}{\pgfqpoint{2.021274in}{3.234128in}}{\pgfqpoint{2.029510in}{3.234128in}}%
\pgfpathclose%
\pgfusepath{stroke,fill}%
\end{pgfscope}%
\begin{pgfscope}%
\pgfpathrectangle{\pgfqpoint{0.100000in}{0.220728in}}{\pgfqpoint{3.696000in}{3.696000in}}%
\pgfusepath{clip}%
\pgfsetbuttcap%
\pgfsetroundjoin%
\definecolor{currentfill}{rgb}{0.121569,0.466667,0.705882}%
\pgfsetfillcolor{currentfill}%
\pgfsetfillopacity{0.362990}%
\pgfsetlinewidth{1.003750pt}%
\definecolor{currentstroke}{rgb}{0.121569,0.466667,0.705882}%
\pgfsetstrokecolor{currentstroke}%
\pgfsetstrokeopacity{0.362990}%
\pgfsetdash{}{0pt}%
\pgfpathmoveto{\pgfqpoint{1.604059in}{2.871895in}}%
\pgfpathcurveto{\pgfqpoint{1.612295in}{2.871895in}}{\pgfqpoint{1.620195in}{2.875167in}}{\pgfqpoint{1.626019in}{2.880991in}}%
\pgfpathcurveto{\pgfqpoint{1.631843in}{2.886815in}}{\pgfqpoint{1.635115in}{2.894715in}}{\pgfqpoint{1.635115in}{2.902952in}}%
\pgfpathcurveto{\pgfqpoint{1.635115in}{2.911188in}}{\pgfqpoint{1.631843in}{2.919088in}}{\pgfqpoint{1.626019in}{2.924912in}}%
\pgfpathcurveto{\pgfqpoint{1.620195in}{2.930736in}}{\pgfqpoint{1.612295in}{2.934008in}}{\pgfqpoint{1.604059in}{2.934008in}}%
\pgfpathcurveto{\pgfqpoint{1.595822in}{2.934008in}}{\pgfqpoint{1.587922in}{2.930736in}}{\pgfqpoint{1.582098in}{2.924912in}}%
\pgfpathcurveto{\pgfqpoint{1.576275in}{2.919088in}}{\pgfqpoint{1.573002in}{2.911188in}}{\pgfqpoint{1.573002in}{2.902952in}}%
\pgfpathcurveto{\pgfqpoint{1.573002in}{2.894715in}}{\pgfqpoint{1.576275in}{2.886815in}}{\pgfqpoint{1.582098in}{2.880991in}}%
\pgfpathcurveto{\pgfqpoint{1.587922in}{2.875167in}}{\pgfqpoint{1.595822in}{2.871895in}}{\pgfqpoint{1.604059in}{2.871895in}}%
\pgfpathclose%
\pgfusepath{stroke,fill}%
\end{pgfscope}%
\begin{pgfscope}%
\pgfpathrectangle{\pgfqpoint{0.100000in}{0.220728in}}{\pgfqpoint{3.696000in}{3.696000in}}%
\pgfusepath{clip}%
\pgfsetbuttcap%
\pgfsetroundjoin%
\definecolor{currentfill}{rgb}{0.121569,0.466667,0.705882}%
\pgfsetfillcolor{currentfill}%
\pgfsetfillopacity{0.363685}%
\pgfsetlinewidth{1.003750pt}%
\definecolor{currentstroke}{rgb}{0.121569,0.466667,0.705882}%
\pgfsetstrokecolor{currentstroke}%
\pgfsetstrokeopacity{0.363685}%
\pgfsetdash{}{0pt}%
\pgfpathmoveto{\pgfqpoint{2.039783in}{3.231508in}}%
\pgfpathcurveto{\pgfqpoint{2.048019in}{3.231508in}}{\pgfqpoint{2.055919in}{3.234781in}}{\pgfqpoint{2.061743in}{3.240605in}}%
\pgfpathcurveto{\pgfqpoint{2.067567in}{3.246429in}}{\pgfqpoint{2.070839in}{3.254329in}}{\pgfqpoint{2.070839in}{3.262565in}}%
\pgfpathcurveto{\pgfqpoint{2.070839in}{3.270801in}}{\pgfqpoint{2.067567in}{3.278701in}}{\pgfqpoint{2.061743in}{3.284525in}}%
\pgfpathcurveto{\pgfqpoint{2.055919in}{3.290349in}}{\pgfqpoint{2.048019in}{3.293621in}}{\pgfqpoint{2.039783in}{3.293621in}}%
\pgfpathcurveto{\pgfqpoint{2.031546in}{3.293621in}}{\pgfqpoint{2.023646in}{3.290349in}}{\pgfqpoint{2.017822in}{3.284525in}}%
\pgfpathcurveto{\pgfqpoint{2.011998in}{3.278701in}}{\pgfqpoint{2.008726in}{3.270801in}}{\pgfqpoint{2.008726in}{3.262565in}}%
\pgfpathcurveto{\pgfqpoint{2.008726in}{3.254329in}}{\pgfqpoint{2.011998in}{3.246429in}}{\pgfqpoint{2.017822in}{3.240605in}}%
\pgfpathcurveto{\pgfqpoint{2.023646in}{3.234781in}}{\pgfqpoint{2.031546in}{3.231508in}}{\pgfqpoint{2.039783in}{3.231508in}}%
\pgfpathclose%
\pgfusepath{stroke,fill}%
\end{pgfscope}%
\begin{pgfscope}%
\pgfpathrectangle{\pgfqpoint{0.100000in}{0.220728in}}{\pgfqpoint{3.696000in}{3.696000in}}%
\pgfusepath{clip}%
\pgfsetbuttcap%
\pgfsetroundjoin%
\definecolor{currentfill}{rgb}{0.121569,0.466667,0.705882}%
\pgfsetfillcolor{currentfill}%
\pgfsetfillopacity{0.366241}%
\pgfsetlinewidth{1.003750pt}%
\definecolor{currentstroke}{rgb}{0.121569,0.466667,0.705882}%
\pgfsetstrokecolor{currentstroke}%
\pgfsetstrokeopacity{0.366241}%
\pgfsetdash{}{0pt}%
\pgfpathmoveto{\pgfqpoint{2.051497in}{3.229094in}}%
\pgfpathcurveto{\pgfqpoint{2.059733in}{3.229094in}}{\pgfqpoint{2.067633in}{3.232366in}}{\pgfqpoint{2.073457in}{3.238190in}}%
\pgfpathcurveto{\pgfqpoint{2.079281in}{3.244014in}}{\pgfqpoint{2.082553in}{3.251914in}}{\pgfqpoint{2.082553in}{3.260150in}}%
\pgfpathcurveto{\pgfqpoint{2.082553in}{3.268386in}}{\pgfqpoint{2.079281in}{3.276286in}}{\pgfqpoint{2.073457in}{3.282110in}}%
\pgfpathcurveto{\pgfqpoint{2.067633in}{3.287934in}}{\pgfqpoint{2.059733in}{3.291206in}}{\pgfqpoint{2.051497in}{3.291206in}}%
\pgfpathcurveto{\pgfqpoint{2.043261in}{3.291206in}}{\pgfqpoint{2.035361in}{3.287934in}}{\pgfqpoint{2.029537in}{3.282110in}}%
\pgfpathcurveto{\pgfqpoint{2.023713in}{3.276286in}}{\pgfqpoint{2.020440in}{3.268386in}}{\pgfqpoint{2.020440in}{3.260150in}}%
\pgfpathcurveto{\pgfqpoint{2.020440in}{3.251914in}}{\pgfqpoint{2.023713in}{3.244014in}}{\pgfqpoint{2.029537in}{3.238190in}}%
\pgfpathcurveto{\pgfqpoint{2.035361in}{3.232366in}}{\pgfqpoint{2.043261in}{3.229094in}}{\pgfqpoint{2.051497in}{3.229094in}}%
\pgfpathclose%
\pgfusepath{stroke,fill}%
\end{pgfscope}%
\begin{pgfscope}%
\pgfpathrectangle{\pgfqpoint{0.100000in}{0.220728in}}{\pgfqpoint{3.696000in}{3.696000in}}%
\pgfusepath{clip}%
\pgfsetbuttcap%
\pgfsetroundjoin%
\definecolor{currentfill}{rgb}{0.121569,0.466667,0.705882}%
\pgfsetfillcolor{currentfill}%
\pgfsetfillopacity{0.367385}%
\pgfsetlinewidth{1.003750pt}%
\definecolor{currentstroke}{rgb}{0.121569,0.466667,0.705882}%
\pgfsetstrokecolor{currentstroke}%
\pgfsetstrokeopacity{0.367385}%
\pgfsetdash{}{0pt}%
\pgfpathmoveto{\pgfqpoint{1.598985in}{2.847392in}}%
\pgfpathcurveto{\pgfqpoint{1.607222in}{2.847392in}}{\pgfqpoint{1.615122in}{2.850665in}}{\pgfqpoint{1.620946in}{2.856489in}}%
\pgfpathcurveto{\pgfqpoint{1.626770in}{2.862313in}}{\pgfqpoint{1.630042in}{2.870213in}}{\pgfqpoint{1.630042in}{2.878449in}}%
\pgfpathcurveto{\pgfqpoint{1.630042in}{2.886685in}}{\pgfqpoint{1.626770in}{2.894585in}}{\pgfqpoint{1.620946in}{2.900409in}}%
\pgfpathcurveto{\pgfqpoint{1.615122in}{2.906233in}}{\pgfqpoint{1.607222in}{2.909505in}}{\pgfqpoint{1.598985in}{2.909505in}}%
\pgfpathcurveto{\pgfqpoint{1.590749in}{2.909505in}}{\pgfqpoint{1.582849in}{2.906233in}}{\pgfqpoint{1.577025in}{2.900409in}}%
\pgfpathcurveto{\pgfqpoint{1.571201in}{2.894585in}}{\pgfqpoint{1.567929in}{2.886685in}}{\pgfqpoint{1.567929in}{2.878449in}}%
\pgfpathcurveto{\pgfqpoint{1.567929in}{2.870213in}}{\pgfqpoint{1.571201in}{2.862313in}}{\pgfqpoint{1.577025in}{2.856489in}}%
\pgfpathcurveto{\pgfqpoint{1.582849in}{2.850665in}}{\pgfqpoint{1.590749in}{2.847392in}}{\pgfqpoint{1.598985in}{2.847392in}}%
\pgfpathclose%
\pgfusepath{stroke,fill}%
\end{pgfscope}%
\begin{pgfscope}%
\pgfpathrectangle{\pgfqpoint{0.100000in}{0.220728in}}{\pgfqpoint{3.696000in}{3.696000in}}%
\pgfusepath{clip}%
\pgfsetbuttcap%
\pgfsetroundjoin%
\definecolor{currentfill}{rgb}{0.121569,0.466667,0.705882}%
\pgfsetfillcolor{currentfill}%
\pgfsetfillopacity{0.370152}%
\pgfsetlinewidth{1.003750pt}%
\definecolor{currentstroke}{rgb}{0.121569,0.466667,0.705882}%
\pgfsetstrokecolor{currentstroke}%
\pgfsetstrokeopacity{0.370152}%
\pgfsetdash{}{0pt}%
\pgfpathmoveto{\pgfqpoint{1.584961in}{2.827644in}}%
\pgfpathcurveto{\pgfqpoint{1.593197in}{2.827644in}}{\pgfqpoint{1.601097in}{2.830916in}}{\pgfqpoint{1.606921in}{2.836740in}}%
\pgfpathcurveto{\pgfqpoint{1.612745in}{2.842564in}}{\pgfqpoint{1.616017in}{2.850464in}}{\pgfqpoint{1.616017in}{2.858700in}}%
\pgfpathcurveto{\pgfqpoint{1.616017in}{2.866937in}}{\pgfqpoint{1.612745in}{2.874837in}}{\pgfqpoint{1.606921in}{2.880661in}}%
\pgfpathcurveto{\pgfqpoint{1.601097in}{2.886485in}}{\pgfqpoint{1.593197in}{2.889757in}}{\pgfqpoint{1.584961in}{2.889757in}}%
\pgfpathcurveto{\pgfqpoint{1.576724in}{2.889757in}}{\pgfqpoint{1.568824in}{2.886485in}}{\pgfqpoint{1.563000in}{2.880661in}}%
\pgfpathcurveto{\pgfqpoint{1.557176in}{2.874837in}}{\pgfqpoint{1.553904in}{2.866937in}}{\pgfqpoint{1.553904in}{2.858700in}}%
\pgfpathcurveto{\pgfqpoint{1.553904in}{2.850464in}}{\pgfqpoint{1.557176in}{2.842564in}}{\pgfqpoint{1.563000in}{2.836740in}}%
\pgfpathcurveto{\pgfqpoint{1.568824in}{2.830916in}}{\pgfqpoint{1.576724in}{2.827644in}}{\pgfqpoint{1.584961in}{2.827644in}}%
\pgfpathclose%
\pgfusepath{stroke,fill}%
\end{pgfscope}%
\begin{pgfscope}%
\pgfpathrectangle{\pgfqpoint{0.100000in}{0.220728in}}{\pgfqpoint{3.696000in}{3.696000in}}%
\pgfusepath{clip}%
\pgfsetbuttcap%
\pgfsetroundjoin%
\definecolor{currentfill}{rgb}{0.121569,0.466667,0.705882}%
\pgfsetfillcolor{currentfill}%
\pgfsetfillopacity{0.370205}%
\pgfsetlinewidth{1.003750pt}%
\definecolor{currentstroke}{rgb}{0.121569,0.466667,0.705882}%
\pgfsetstrokecolor{currentstroke}%
\pgfsetstrokeopacity{0.370205}%
\pgfsetdash{}{0pt}%
\pgfpathmoveto{\pgfqpoint{2.062911in}{3.228542in}}%
\pgfpathcurveto{\pgfqpoint{2.071147in}{3.228542in}}{\pgfqpoint{2.079047in}{3.231815in}}{\pgfqpoint{2.084871in}{3.237639in}}%
\pgfpathcurveto{\pgfqpoint{2.090695in}{3.243463in}}{\pgfqpoint{2.093967in}{3.251363in}}{\pgfqpoint{2.093967in}{3.259599in}}%
\pgfpathcurveto{\pgfqpoint{2.093967in}{3.267835in}}{\pgfqpoint{2.090695in}{3.275735in}}{\pgfqpoint{2.084871in}{3.281559in}}%
\pgfpathcurveto{\pgfqpoint{2.079047in}{3.287383in}}{\pgfqpoint{2.071147in}{3.290655in}}{\pgfqpoint{2.062911in}{3.290655in}}%
\pgfpathcurveto{\pgfqpoint{2.054675in}{3.290655in}}{\pgfqpoint{2.046774in}{3.287383in}}{\pgfqpoint{2.040951in}{3.281559in}}%
\pgfpathcurveto{\pgfqpoint{2.035127in}{3.275735in}}{\pgfqpoint{2.031854in}{3.267835in}}{\pgfqpoint{2.031854in}{3.259599in}}%
\pgfpathcurveto{\pgfqpoint{2.031854in}{3.251363in}}{\pgfqpoint{2.035127in}{3.243463in}}{\pgfqpoint{2.040951in}{3.237639in}}%
\pgfpathcurveto{\pgfqpoint{2.046774in}{3.231815in}}{\pgfqpoint{2.054675in}{3.228542in}}{\pgfqpoint{2.062911in}{3.228542in}}%
\pgfpathclose%
\pgfusepath{stroke,fill}%
\end{pgfscope}%
\begin{pgfscope}%
\pgfpathrectangle{\pgfqpoint{0.100000in}{0.220728in}}{\pgfqpoint{3.696000in}{3.696000in}}%
\pgfusepath{clip}%
\pgfsetbuttcap%
\pgfsetroundjoin%
\definecolor{currentfill}{rgb}{0.121569,0.466667,0.705882}%
\pgfsetfillcolor{currentfill}%
\pgfsetfillopacity{0.373283}%
\pgfsetlinewidth{1.003750pt}%
\definecolor{currentstroke}{rgb}{0.121569,0.466667,0.705882}%
\pgfsetstrokecolor{currentstroke}%
\pgfsetstrokeopacity{0.373283}%
\pgfsetdash{}{0pt}%
\pgfpathmoveto{\pgfqpoint{1.582709in}{2.805114in}}%
\pgfpathcurveto{\pgfqpoint{1.590945in}{2.805114in}}{\pgfqpoint{1.598846in}{2.808386in}}{\pgfqpoint{1.604669in}{2.814210in}}%
\pgfpathcurveto{\pgfqpoint{1.610493in}{2.820034in}}{\pgfqpoint{1.613766in}{2.827934in}}{\pgfqpoint{1.613766in}{2.836170in}}%
\pgfpathcurveto{\pgfqpoint{1.613766in}{2.844406in}}{\pgfqpoint{1.610493in}{2.852307in}}{\pgfqpoint{1.604669in}{2.858130in}}%
\pgfpathcurveto{\pgfqpoint{1.598846in}{2.863954in}}{\pgfqpoint{1.590945in}{2.867227in}}{\pgfqpoint{1.582709in}{2.867227in}}%
\pgfpathcurveto{\pgfqpoint{1.574473in}{2.867227in}}{\pgfqpoint{1.566573in}{2.863954in}}{\pgfqpoint{1.560749in}{2.858130in}}%
\pgfpathcurveto{\pgfqpoint{1.554925in}{2.852307in}}{\pgfqpoint{1.551653in}{2.844406in}}{\pgfqpoint{1.551653in}{2.836170in}}%
\pgfpathcurveto{\pgfqpoint{1.551653in}{2.827934in}}{\pgfqpoint{1.554925in}{2.820034in}}{\pgfqpoint{1.560749in}{2.814210in}}%
\pgfpathcurveto{\pgfqpoint{1.566573in}{2.808386in}}{\pgfqpoint{1.574473in}{2.805114in}}{\pgfqpoint{1.582709in}{2.805114in}}%
\pgfpathclose%
\pgfusepath{stroke,fill}%
\end{pgfscope}%
\begin{pgfscope}%
\pgfpathrectangle{\pgfqpoint{0.100000in}{0.220728in}}{\pgfqpoint{3.696000in}{3.696000in}}%
\pgfusepath{clip}%
\pgfsetbuttcap%
\pgfsetroundjoin%
\definecolor{currentfill}{rgb}{0.121569,0.466667,0.705882}%
\pgfsetfillcolor{currentfill}%
\pgfsetfillopacity{0.373923}%
\pgfsetlinewidth{1.003750pt}%
\definecolor{currentstroke}{rgb}{0.121569,0.466667,0.705882}%
\pgfsetstrokecolor{currentstroke}%
\pgfsetstrokeopacity{0.373923}%
\pgfsetdash{}{0pt}%
\pgfpathmoveto{\pgfqpoint{2.077734in}{3.227875in}}%
\pgfpathcurveto{\pgfqpoint{2.085971in}{3.227875in}}{\pgfqpoint{2.093871in}{3.231147in}}{\pgfqpoint{2.099695in}{3.236971in}}%
\pgfpathcurveto{\pgfqpoint{2.105519in}{3.242795in}}{\pgfqpoint{2.108791in}{3.250695in}}{\pgfqpoint{2.108791in}{3.258932in}}%
\pgfpathcurveto{\pgfqpoint{2.108791in}{3.267168in}}{\pgfqpoint{2.105519in}{3.275068in}}{\pgfqpoint{2.099695in}{3.280892in}}%
\pgfpathcurveto{\pgfqpoint{2.093871in}{3.286716in}}{\pgfqpoint{2.085971in}{3.289988in}}{\pgfqpoint{2.077734in}{3.289988in}}%
\pgfpathcurveto{\pgfqpoint{2.069498in}{3.289988in}}{\pgfqpoint{2.061598in}{3.286716in}}{\pgfqpoint{2.055774in}{3.280892in}}%
\pgfpathcurveto{\pgfqpoint{2.049950in}{3.275068in}}{\pgfqpoint{2.046678in}{3.267168in}}{\pgfqpoint{2.046678in}{3.258932in}}%
\pgfpathcurveto{\pgfqpoint{2.046678in}{3.250695in}}{\pgfqpoint{2.049950in}{3.242795in}}{\pgfqpoint{2.055774in}{3.236971in}}%
\pgfpathcurveto{\pgfqpoint{2.061598in}{3.231147in}}{\pgfqpoint{2.069498in}{3.227875in}}{\pgfqpoint{2.077734in}{3.227875in}}%
\pgfpathclose%
\pgfusepath{stroke,fill}%
\end{pgfscope}%
\begin{pgfscope}%
\pgfpathrectangle{\pgfqpoint{0.100000in}{0.220728in}}{\pgfqpoint{3.696000in}{3.696000in}}%
\pgfusepath{clip}%
\pgfsetbuttcap%
\pgfsetroundjoin%
\definecolor{currentfill}{rgb}{0.121569,0.466667,0.705882}%
\pgfsetfillcolor{currentfill}%
\pgfsetfillopacity{0.375026}%
\pgfsetlinewidth{1.003750pt}%
\definecolor{currentstroke}{rgb}{0.121569,0.466667,0.705882}%
\pgfsetstrokecolor{currentstroke}%
\pgfsetstrokeopacity{0.375026}%
\pgfsetdash{}{0pt}%
\pgfpathmoveto{\pgfqpoint{1.569784in}{2.789959in}}%
\pgfpathcurveto{\pgfqpoint{1.578020in}{2.789959in}}{\pgfqpoint{1.585920in}{2.793231in}}{\pgfqpoint{1.591744in}{2.799055in}}%
\pgfpathcurveto{\pgfqpoint{1.597568in}{2.804879in}}{\pgfqpoint{1.600840in}{2.812779in}}{\pgfqpoint{1.600840in}{2.821016in}}%
\pgfpathcurveto{\pgfqpoint{1.600840in}{2.829252in}}{\pgfqpoint{1.597568in}{2.837152in}}{\pgfqpoint{1.591744in}{2.842976in}}%
\pgfpathcurveto{\pgfqpoint{1.585920in}{2.848800in}}{\pgfqpoint{1.578020in}{2.852072in}}{\pgfqpoint{1.569784in}{2.852072in}}%
\pgfpathcurveto{\pgfqpoint{1.561547in}{2.852072in}}{\pgfqpoint{1.553647in}{2.848800in}}{\pgfqpoint{1.547823in}{2.842976in}}%
\pgfpathcurveto{\pgfqpoint{1.541999in}{2.837152in}}{\pgfqpoint{1.538727in}{2.829252in}}{\pgfqpoint{1.538727in}{2.821016in}}%
\pgfpathcurveto{\pgfqpoint{1.538727in}{2.812779in}}{\pgfqpoint{1.541999in}{2.804879in}}{\pgfqpoint{1.547823in}{2.799055in}}%
\pgfpathcurveto{\pgfqpoint{1.553647in}{2.793231in}}{\pgfqpoint{1.561547in}{2.789959in}}{\pgfqpoint{1.569784in}{2.789959in}}%
\pgfpathclose%
\pgfusepath{stroke,fill}%
\end{pgfscope}%
\begin{pgfscope}%
\pgfpathrectangle{\pgfqpoint{0.100000in}{0.220728in}}{\pgfqpoint{3.696000in}{3.696000in}}%
\pgfusepath{clip}%
\pgfsetbuttcap%
\pgfsetroundjoin%
\definecolor{currentfill}{rgb}{0.121569,0.466667,0.705882}%
\pgfsetfillcolor{currentfill}%
\pgfsetfillopacity{0.377738}%
\pgfsetlinewidth{1.003750pt}%
\definecolor{currentstroke}{rgb}{0.121569,0.466667,0.705882}%
\pgfsetstrokecolor{currentstroke}%
\pgfsetstrokeopacity{0.377738}%
\pgfsetdash{}{0pt}%
\pgfpathmoveto{\pgfqpoint{1.565440in}{2.774171in}}%
\pgfpathcurveto{\pgfqpoint{1.573676in}{2.774171in}}{\pgfqpoint{1.581576in}{2.777444in}}{\pgfqpoint{1.587400in}{2.783268in}}%
\pgfpathcurveto{\pgfqpoint{1.593224in}{2.789091in}}{\pgfqpoint{1.596497in}{2.796992in}}{\pgfqpoint{1.596497in}{2.805228in}}%
\pgfpathcurveto{\pgfqpoint{1.596497in}{2.813464in}}{\pgfqpoint{1.593224in}{2.821364in}}{\pgfqpoint{1.587400in}{2.827188in}}%
\pgfpathcurveto{\pgfqpoint{1.581576in}{2.833012in}}{\pgfqpoint{1.573676in}{2.836284in}}{\pgfqpoint{1.565440in}{2.836284in}}%
\pgfpathcurveto{\pgfqpoint{1.557204in}{2.836284in}}{\pgfqpoint{1.549304in}{2.833012in}}{\pgfqpoint{1.543480in}{2.827188in}}%
\pgfpathcurveto{\pgfqpoint{1.537656in}{2.821364in}}{\pgfqpoint{1.534384in}{2.813464in}}{\pgfqpoint{1.534384in}{2.805228in}}%
\pgfpathcurveto{\pgfqpoint{1.534384in}{2.796992in}}{\pgfqpoint{1.537656in}{2.789091in}}{\pgfqpoint{1.543480in}{2.783268in}}%
\pgfpathcurveto{\pgfqpoint{1.549304in}{2.777444in}}{\pgfqpoint{1.557204in}{2.774171in}}{\pgfqpoint{1.565440in}{2.774171in}}%
\pgfpathclose%
\pgfusepath{stroke,fill}%
\end{pgfscope}%
\begin{pgfscope}%
\pgfpathrectangle{\pgfqpoint{0.100000in}{0.220728in}}{\pgfqpoint{3.696000in}{3.696000in}}%
\pgfusepath{clip}%
\pgfsetbuttcap%
\pgfsetroundjoin%
\definecolor{currentfill}{rgb}{0.121569,0.466667,0.705882}%
\pgfsetfillcolor{currentfill}%
\pgfsetfillopacity{0.377743}%
\pgfsetlinewidth{1.003750pt}%
\definecolor{currentstroke}{rgb}{0.121569,0.466667,0.705882}%
\pgfsetstrokecolor{currentstroke}%
\pgfsetstrokeopacity{0.377743}%
\pgfsetdash{}{0pt}%
\pgfpathmoveto{\pgfqpoint{2.092427in}{3.225157in}}%
\pgfpathcurveto{\pgfqpoint{2.100663in}{3.225157in}}{\pgfqpoint{2.108563in}{3.228429in}}{\pgfqpoint{2.114387in}{3.234253in}}%
\pgfpathcurveto{\pgfqpoint{2.120211in}{3.240077in}}{\pgfqpoint{2.123483in}{3.247977in}}{\pgfqpoint{2.123483in}{3.256213in}}%
\pgfpathcurveto{\pgfqpoint{2.123483in}{3.264449in}}{\pgfqpoint{2.120211in}{3.272349in}}{\pgfqpoint{2.114387in}{3.278173in}}%
\pgfpathcurveto{\pgfqpoint{2.108563in}{3.283997in}}{\pgfqpoint{2.100663in}{3.287270in}}{\pgfqpoint{2.092427in}{3.287270in}}%
\pgfpathcurveto{\pgfqpoint{2.084190in}{3.287270in}}{\pgfqpoint{2.076290in}{3.283997in}}{\pgfqpoint{2.070466in}{3.278173in}}%
\pgfpathcurveto{\pgfqpoint{2.064642in}{3.272349in}}{\pgfqpoint{2.061370in}{3.264449in}}{\pgfqpoint{2.061370in}{3.256213in}}%
\pgfpathcurveto{\pgfqpoint{2.061370in}{3.247977in}}{\pgfqpoint{2.064642in}{3.240077in}}{\pgfqpoint{2.070466in}{3.234253in}}%
\pgfpathcurveto{\pgfqpoint{2.076290in}{3.228429in}}{\pgfqpoint{2.084190in}{3.225157in}}{\pgfqpoint{2.092427in}{3.225157in}}%
\pgfpathclose%
\pgfusepath{stroke,fill}%
\end{pgfscope}%
\begin{pgfscope}%
\pgfpathrectangle{\pgfqpoint{0.100000in}{0.220728in}}{\pgfqpoint{3.696000in}{3.696000in}}%
\pgfusepath{clip}%
\pgfsetbuttcap%
\pgfsetroundjoin%
\definecolor{currentfill}{rgb}{0.121569,0.466667,0.705882}%
\pgfsetfillcolor{currentfill}%
\pgfsetfillopacity{0.379876}%
\pgfsetlinewidth{1.003750pt}%
\definecolor{currentstroke}{rgb}{0.121569,0.466667,0.705882}%
\pgfsetstrokecolor{currentstroke}%
\pgfsetstrokeopacity{0.379876}%
\pgfsetdash{}{0pt}%
\pgfpathmoveto{\pgfqpoint{1.558296in}{2.760355in}}%
\pgfpathcurveto{\pgfqpoint{1.566532in}{2.760355in}}{\pgfqpoint{1.574432in}{2.763627in}}{\pgfqpoint{1.580256in}{2.769451in}}%
\pgfpathcurveto{\pgfqpoint{1.586080in}{2.775275in}}{\pgfqpoint{1.589352in}{2.783175in}}{\pgfqpoint{1.589352in}{2.791411in}}%
\pgfpathcurveto{\pgfqpoint{1.589352in}{2.799648in}}{\pgfqpoint{1.586080in}{2.807548in}}{\pgfqpoint{1.580256in}{2.813372in}}%
\pgfpathcurveto{\pgfqpoint{1.574432in}{2.819196in}}{\pgfqpoint{1.566532in}{2.822468in}}{\pgfqpoint{1.558296in}{2.822468in}}%
\pgfpathcurveto{\pgfqpoint{1.550060in}{2.822468in}}{\pgfqpoint{1.542160in}{2.819196in}}{\pgfqpoint{1.536336in}{2.813372in}}%
\pgfpathcurveto{\pgfqpoint{1.530512in}{2.807548in}}{\pgfqpoint{1.527239in}{2.799648in}}{\pgfqpoint{1.527239in}{2.791411in}}%
\pgfpathcurveto{\pgfqpoint{1.527239in}{2.783175in}}{\pgfqpoint{1.530512in}{2.775275in}}{\pgfqpoint{1.536336in}{2.769451in}}%
\pgfpathcurveto{\pgfqpoint{1.542160in}{2.763627in}}{\pgfqpoint{1.550060in}{2.760355in}}{\pgfqpoint{1.558296in}{2.760355in}}%
\pgfpathclose%
\pgfusepath{stroke,fill}%
\end{pgfscope}%
\begin{pgfscope}%
\pgfpathrectangle{\pgfqpoint{0.100000in}{0.220728in}}{\pgfqpoint{3.696000in}{3.696000in}}%
\pgfusepath{clip}%
\pgfsetbuttcap%
\pgfsetroundjoin%
\definecolor{currentfill}{rgb}{0.121569,0.466667,0.705882}%
\pgfsetfillcolor{currentfill}%
\pgfsetfillopacity{0.380092}%
\pgfsetlinewidth{1.003750pt}%
\definecolor{currentstroke}{rgb}{0.121569,0.466667,0.705882}%
\pgfsetstrokecolor{currentstroke}%
\pgfsetstrokeopacity{0.380092}%
\pgfsetdash{}{0pt}%
\pgfpathmoveto{\pgfqpoint{2.100427in}{3.224472in}}%
\pgfpathcurveto{\pgfqpoint{2.108664in}{3.224472in}}{\pgfqpoint{2.116564in}{3.227744in}}{\pgfqpoint{2.122387in}{3.233568in}}%
\pgfpathcurveto{\pgfqpoint{2.128211in}{3.239392in}}{\pgfqpoint{2.131484in}{3.247292in}}{\pgfqpoint{2.131484in}{3.255528in}}%
\pgfpathcurveto{\pgfqpoint{2.131484in}{3.263764in}}{\pgfqpoint{2.128211in}{3.271664in}}{\pgfqpoint{2.122387in}{3.277488in}}%
\pgfpathcurveto{\pgfqpoint{2.116564in}{3.283312in}}{\pgfqpoint{2.108664in}{3.286585in}}{\pgfqpoint{2.100427in}{3.286585in}}%
\pgfpathcurveto{\pgfqpoint{2.092191in}{3.286585in}}{\pgfqpoint{2.084291in}{3.283312in}}{\pgfqpoint{2.078467in}{3.277488in}}%
\pgfpathcurveto{\pgfqpoint{2.072643in}{3.271664in}}{\pgfqpoint{2.069371in}{3.263764in}}{\pgfqpoint{2.069371in}{3.255528in}}%
\pgfpathcurveto{\pgfqpoint{2.069371in}{3.247292in}}{\pgfqpoint{2.072643in}{3.239392in}}{\pgfqpoint{2.078467in}{3.233568in}}%
\pgfpathcurveto{\pgfqpoint{2.084291in}{3.227744in}}{\pgfqpoint{2.092191in}{3.224472in}}{\pgfqpoint{2.100427in}{3.224472in}}%
\pgfpathclose%
\pgfusepath{stroke,fill}%
\end{pgfscope}%
\begin{pgfscope}%
\pgfpathrectangle{\pgfqpoint{0.100000in}{0.220728in}}{\pgfqpoint{3.696000in}{3.696000in}}%
\pgfusepath{clip}%
\pgfsetbuttcap%
\pgfsetroundjoin%
\definecolor{currentfill}{rgb}{0.121569,0.466667,0.705882}%
\pgfsetfillcolor{currentfill}%
\pgfsetfillopacity{0.381333}%
\pgfsetlinewidth{1.003750pt}%
\definecolor{currentstroke}{rgb}{0.121569,0.466667,0.705882}%
\pgfsetstrokecolor{currentstroke}%
\pgfsetstrokeopacity{0.381333}%
\pgfsetdash{}{0pt}%
\pgfpathmoveto{\pgfqpoint{2.109878in}{3.221927in}}%
\pgfpathcurveto{\pgfqpoint{2.118114in}{3.221927in}}{\pgfqpoint{2.126014in}{3.225199in}}{\pgfqpoint{2.131838in}{3.231023in}}%
\pgfpathcurveto{\pgfqpoint{2.137662in}{3.236847in}}{\pgfqpoint{2.140934in}{3.244747in}}{\pgfqpoint{2.140934in}{3.252983in}}%
\pgfpathcurveto{\pgfqpoint{2.140934in}{3.261219in}}{\pgfqpoint{2.137662in}{3.269120in}}{\pgfqpoint{2.131838in}{3.274943in}}%
\pgfpathcurveto{\pgfqpoint{2.126014in}{3.280767in}}{\pgfqpoint{2.118114in}{3.284040in}}{\pgfqpoint{2.109878in}{3.284040in}}%
\pgfpathcurveto{\pgfqpoint{2.101641in}{3.284040in}}{\pgfqpoint{2.093741in}{3.280767in}}{\pgfqpoint{2.087917in}{3.274943in}}%
\pgfpathcurveto{\pgfqpoint{2.082093in}{3.269120in}}{\pgfqpoint{2.078821in}{3.261219in}}{\pgfqpoint{2.078821in}{3.252983in}}%
\pgfpathcurveto{\pgfqpoint{2.078821in}{3.244747in}}{\pgfqpoint{2.082093in}{3.236847in}}{\pgfqpoint{2.087917in}{3.231023in}}%
\pgfpathcurveto{\pgfqpoint{2.093741in}{3.225199in}}{\pgfqpoint{2.101641in}{3.221927in}}{\pgfqpoint{2.109878in}{3.221927in}}%
\pgfpathclose%
\pgfusepath{stroke,fill}%
\end{pgfscope}%
\begin{pgfscope}%
\pgfpathrectangle{\pgfqpoint{0.100000in}{0.220728in}}{\pgfqpoint{3.696000in}{3.696000in}}%
\pgfusepath{clip}%
\pgfsetbuttcap%
\pgfsetroundjoin%
\definecolor{currentfill}{rgb}{0.121569,0.466667,0.705882}%
\pgfsetfillcolor{currentfill}%
\pgfsetfillopacity{0.382228}%
\pgfsetlinewidth{1.003750pt}%
\definecolor{currentstroke}{rgb}{0.121569,0.466667,0.705882}%
\pgfsetstrokecolor{currentstroke}%
\pgfsetstrokeopacity{0.382228}%
\pgfsetdash{}{0pt}%
\pgfpathmoveto{\pgfqpoint{1.553662in}{2.747494in}}%
\pgfpathcurveto{\pgfqpoint{1.561899in}{2.747494in}}{\pgfqpoint{1.569799in}{2.750766in}}{\pgfqpoint{1.575623in}{2.756590in}}%
\pgfpathcurveto{\pgfqpoint{1.581447in}{2.762414in}}{\pgfqpoint{1.584719in}{2.770314in}}{\pgfqpoint{1.584719in}{2.778550in}}%
\pgfpathcurveto{\pgfqpoint{1.584719in}{2.786787in}}{\pgfqpoint{1.581447in}{2.794687in}}{\pgfqpoint{1.575623in}{2.800510in}}%
\pgfpathcurveto{\pgfqpoint{1.569799in}{2.806334in}}{\pgfqpoint{1.561899in}{2.809607in}}{\pgfqpoint{1.553662in}{2.809607in}}%
\pgfpathcurveto{\pgfqpoint{1.545426in}{2.809607in}}{\pgfqpoint{1.537526in}{2.806334in}}{\pgfqpoint{1.531702in}{2.800510in}}%
\pgfpathcurveto{\pgfqpoint{1.525878in}{2.794687in}}{\pgfqpoint{1.522606in}{2.786787in}}{\pgfqpoint{1.522606in}{2.778550in}}%
\pgfpathcurveto{\pgfqpoint{1.522606in}{2.770314in}}{\pgfqpoint{1.525878in}{2.762414in}}{\pgfqpoint{1.531702in}{2.756590in}}%
\pgfpathcurveto{\pgfqpoint{1.537526in}{2.750766in}}{\pgfqpoint{1.545426in}{2.747494in}}{\pgfqpoint{1.553662in}{2.747494in}}%
\pgfpathclose%
\pgfusepath{stroke,fill}%
\end{pgfscope}%
\begin{pgfscope}%
\pgfpathrectangle{\pgfqpoint{0.100000in}{0.220728in}}{\pgfqpoint{3.696000in}{3.696000in}}%
\pgfusepath{clip}%
\pgfsetbuttcap%
\pgfsetroundjoin%
\definecolor{currentfill}{rgb}{0.121569,0.466667,0.705882}%
\pgfsetfillcolor{currentfill}%
\pgfsetfillopacity{0.383045}%
\pgfsetlinewidth{1.003750pt}%
\definecolor{currentstroke}{rgb}{0.121569,0.466667,0.705882}%
\pgfsetstrokecolor{currentstroke}%
\pgfsetstrokeopacity{0.383045}%
\pgfsetdash{}{0pt}%
\pgfpathmoveto{\pgfqpoint{2.114131in}{3.221942in}}%
\pgfpathcurveto{\pgfqpoint{2.122367in}{3.221942in}}{\pgfqpoint{2.130267in}{3.225214in}}{\pgfqpoint{2.136091in}{3.231038in}}%
\pgfpathcurveto{\pgfqpoint{2.141915in}{3.236862in}}{\pgfqpoint{2.145187in}{3.244762in}}{\pgfqpoint{2.145187in}{3.252999in}}%
\pgfpathcurveto{\pgfqpoint{2.145187in}{3.261235in}}{\pgfqpoint{2.141915in}{3.269135in}}{\pgfqpoint{2.136091in}{3.274959in}}%
\pgfpathcurveto{\pgfqpoint{2.130267in}{3.280783in}}{\pgfqpoint{2.122367in}{3.284055in}}{\pgfqpoint{2.114131in}{3.284055in}}%
\pgfpathcurveto{\pgfqpoint{2.105895in}{3.284055in}}{\pgfqpoint{2.097995in}{3.280783in}}{\pgfqpoint{2.092171in}{3.274959in}}%
\pgfpathcurveto{\pgfqpoint{2.086347in}{3.269135in}}{\pgfqpoint{2.083074in}{3.261235in}}{\pgfqpoint{2.083074in}{3.252999in}}%
\pgfpathcurveto{\pgfqpoint{2.083074in}{3.244762in}}{\pgfqpoint{2.086347in}{3.236862in}}{\pgfqpoint{2.092171in}{3.231038in}}%
\pgfpathcurveto{\pgfqpoint{2.097995in}{3.225214in}}{\pgfqpoint{2.105895in}{3.221942in}}{\pgfqpoint{2.114131in}{3.221942in}}%
\pgfpathclose%
\pgfusepath{stroke,fill}%
\end{pgfscope}%
\begin{pgfscope}%
\pgfpathrectangle{\pgfqpoint{0.100000in}{0.220728in}}{\pgfqpoint{3.696000in}{3.696000in}}%
\pgfusepath{clip}%
\pgfsetbuttcap%
\pgfsetroundjoin%
\definecolor{currentfill}{rgb}{0.121569,0.466667,0.705882}%
\pgfsetfillcolor{currentfill}%
\pgfsetfillopacity{0.384396}%
\pgfsetlinewidth{1.003750pt}%
\definecolor{currentstroke}{rgb}{0.121569,0.466667,0.705882}%
\pgfsetstrokecolor{currentstroke}%
\pgfsetstrokeopacity{0.384396}%
\pgfsetdash{}{0pt}%
\pgfpathmoveto{\pgfqpoint{2.121897in}{3.220441in}}%
\pgfpathcurveto{\pgfqpoint{2.130134in}{3.220441in}}{\pgfqpoint{2.138034in}{3.223714in}}{\pgfqpoint{2.143857in}{3.229538in}}%
\pgfpathcurveto{\pgfqpoint{2.149681in}{3.235362in}}{\pgfqpoint{2.152954in}{3.243262in}}{\pgfqpoint{2.152954in}{3.251498in}}%
\pgfpathcurveto{\pgfqpoint{2.152954in}{3.259734in}}{\pgfqpoint{2.149681in}{3.267634in}}{\pgfqpoint{2.143857in}{3.273458in}}%
\pgfpathcurveto{\pgfqpoint{2.138034in}{3.279282in}}{\pgfqpoint{2.130134in}{3.282554in}}{\pgfqpoint{2.121897in}{3.282554in}}%
\pgfpathcurveto{\pgfqpoint{2.113661in}{3.282554in}}{\pgfqpoint{2.105761in}{3.279282in}}{\pgfqpoint{2.099937in}{3.273458in}}%
\pgfpathcurveto{\pgfqpoint{2.094113in}{3.267634in}}{\pgfqpoint{2.090841in}{3.259734in}}{\pgfqpoint{2.090841in}{3.251498in}}%
\pgfpathcurveto{\pgfqpoint{2.090841in}{3.243262in}}{\pgfqpoint{2.094113in}{3.235362in}}{\pgfqpoint{2.099937in}{3.229538in}}%
\pgfpathcurveto{\pgfqpoint{2.105761in}{3.223714in}}{\pgfqpoint{2.113661in}{3.220441in}}{\pgfqpoint{2.121897in}{3.220441in}}%
\pgfpathclose%
\pgfusepath{stroke,fill}%
\end{pgfscope}%
\begin{pgfscope}%
\pgfpathrectangle{\pgfqpoint{0.100000in}{0.220728in}}{\pgfqpoint{3.696000in}{3.696000in}}%
\pgfusepath{clip}%
\pgfsetbuttcap%
\pgfsetroundjoin%
\definecolor{currentfill}{rgb}{0.121569,0.466667,0.705882}%
\pgfsetfillcolor{currentfill}%
\pgfsetfillopacity{0.386261}%
\pgfsetlinewidth{1.003750pt}%
\definecolor{currentstroke}{rgb}{0.121569,0.466667,0.705882}%
\pgfsetstrokecolor{currentstroke}%
\pgfsetstrokeopacity{0.386261}%
\pgfsetdash{}{0pt}%
\pgfpathmoveto{\pgfqpoint{1.543920in}{2.724082in}}%
\pgfpathcurveto{\pgfqpoint{1.552157in}{2.724082in}}{\pgfqpoint{1.560057in}{2.727354in}}{\pgfqpoint{1.565881in}{2.733178in}}%
\pgfpathcurveto{\pgfqpoint{1.571705in}{2.739002in}}{\pgfqpoint{1.574977in}{2.746902in}}{\pgfqpoint{1.574977in}{2.755139in}}%
\pgfpathcurveto{\pgfqpoint{1.574977in}{2.763375in}}{\pgfqpoint{1.571705in}{2.771275in}}{\pgfqpoint{1.565881in}{2.777099in}}%
\pgfpathcurveto{\pgfqpoint{1.560057in}{2.782923in}}{\pgfqpoint{1.552157in}{2.786195in}}{\pgfqpoint{1.543920in}{2.786195in}}%
\pgfpathcurveto{\pgfqpoint{1.535684in}{2.786195in}}{\pgfqpoint{1.527784in}{2.782923in}}{\pgfqpoint{1.521960in}{2.777099in}}%
\pgfpathcurveto{\pgfqpoint{1.516136in}{2.771275in}}{\pgfqpoint{1.512864in}{2.763375in}}{\pgfqpoint{1.512864in}{2.755139in}}%
\pgfpathcurveto{\pgfqpoint{1.512864in}{2.746902in}}{\pgfqpoint{1.516136in}{2.739002in}}{\pgfqpoint{1.521960in}{2.733178in}}%
\pgfpathcurveto{\pgfqpoint{1.527784in}{2.727354in}}{\pgfqpoint{1.535684in}{2.724082in}}{\pgfqpoint{1.543920in}{2.724082in}}%
\pgfpathclose%
\pgfusepath{stroke,fill}%
\end{pgfscope}%
\begin{pgfscope}%
\pgfpathrectangle{\pgfqpoint{0.100000in}{0.220728in}}{\pgfqpoint{3.696000in}{3.696000in}}%
\pgfusepath{clip}%
\pgfsetbuttcap%
\pgfsetroundjoin%
\definecolor{currentfill}{rgb}{0.121569,0.466667,0.705882}%
\pgfsetfillcolor{currentfill}%
\pgfsetfillopacity{0.386437}%
\pgfsetlinewidth{1.003750pt}%
\definecolor{currentstroke}{rgb}{0.121569,0.466667,0.705882}%
\pgfsetstrokecolor{currentstroke}%
\pgfsetstrokeopacity{0.386437}%
\pgfsetdash{}{0pt}%
\pgfpathmoveto{\pgfqpoint{2.129772in}{3.219465in}}%
\pgfpathcurveto{\pgfqpoint{2.138008in}{3.219465in}}{\pgfqpoint{2.145908in}{3.222737in}}{\pgfqpoint{2.151732in}{3.228561in}}%
\pgfpathcurveto{\pgfqpoint{2.157556in}{3.234385in}}{\pgfqpoint{2.160828in}{3.242285in}}{\pgfqpoint{2.160828in}{3.250522in}}%
\pgfpathcurveto{\pgfqpoint{2.160828in}{3.258758in}}{\pgfqpoint{2.157556in}{3.266658in}}{\pgfqpoint{2.151732in}{3.272482in}}%
\pgfpathcurveto{\pgfqpoint{2.145908in}{3.278306in}}{\pgfqpoint{2.138008in}{3.281578in}}{\pgfqpoint{2.129772in}{3.281578in}}%
\pgfpathcurveto{\pgfqpoint{2.121536in}{3.281578in}}{\pgfqpoint{2.113636in}{3.278306in}}{\pgfqpoint{2.107812in}{3.272482in}}%
\pgfpathcurveto{\pgfqpoint{2.101988in}{3.266658in}}{\pgfqpoint{2.098715in}{3.258758in}}{\pgfqpoint{2.098715in}{3.250522in}}%
\pgfpathcurveto{\pgfqpoint{2.098715in}{3.242285in}}{\pgfqpoint{2.101988in}{3.234385in}}{\pgfqpoint{2.107812in}{3.228561in}}%
\pgfpathcurveto{\pgfqpoint{2.113636in}{3.222737in}}{\pgfqpoint{2.121536in}{3.219465in}}{\pgfqpoint{2.129772in}{3.219465in}}%
\pgfpathclose%
\pgfusepath{stroke,fill}%
\end{pgfscope}%
\begin{pgfscope}%
\pgfpathrectangle{\pgfqpoint{0.100000in}{0.220728in}}{\pgfqpoint{3.696000in}{3.696000in}}%
\pgfusepath{clip}%
\pgfsetbuttcap%
\pgfsetroundjoin%
\definecolor{currentfill}{rgb}{0.121569,0.466667,0.705882}%
\pgfsetfillcolor{currentfill}%
\pgfsetfillopacity{0.387898}%
\pgfsetlinewidth{1.003750pt}%
\definecolor{currentstroke}{rgb}{0.121569,0.466667,0.705882}%
\pgfsetstrokecolor{currentstroke}%
\pgfsetstrokeopacity{0.387898}%
\pgfsetdash{}{0pt}%
\pgfpathmoveto{\pgfqpoint{2.142876in}{3.218199in}}%
\pgfpathcurveto{\pgfqpoint{2.151112in}{3.218199in}}{\pgfqpoint{2.159012in}{3.221471in}}{\pgfqpoint{2.164836in}{3.227295in}}%
\pgfpathcurveto{\pgfqpoint{2.170660in}{3.233119in}}{\pgfqpoint{2.173932in}{3.241019in}}{\pgfqpoint{2.173932in}{3.249255in}}%
\pgfpathcurveto{\pgfqpoint{2.173932in}{3.257492in}}{\pgfqpoint{2.170660in}{3.265392in}}{\pgfqpoint{2.164836in}{3.271216in}}%
\pgfpathcurveto{\pgfqpoint{2.159012in}{3.277039in}}{\pgfqpoint{2.151112in}{3.280312in}}{\pgfqpoint{2.142876in}{3.280312in}}%
\pgfpathcurveto{\pgfqpoint{2.134639in}{3.280312in}}{\pgfqpoint{2.126739in}{3.277039in}}{\pgfqpoint{2.120915in}{3.271216in}}%
\pgfpathcurveto{\pgfqpoint{2.115091in}{3.265392in}}{\pgfqpoint{2.111819in}{3.257492in}}{\pgfqpoint{2.111819in}{3.249255in}}%
\pgfpathcurveto{\pgfqpoint{2.111819in}{3.241019in}}{\pgfqpoint{2.115091in}{3.233119in}}{\pgfqpoint{2.120915in}{3.227295in}}%
\pgfpathcurveto{\pgfqpoint{2.126739in}{3.221471in}}{\pgfqpoint{2.134639in}{3.218199in}}{\pgfqpoint{2.142876in}{3.218199in}}%
\pgfpathclose%
\pgfusepath{stroke,fill}%
\end{pgfscope}%
\begin{pgfscope}%
\pgfpathrectangle{\pgfqpoint{0.100000in}{0.220728in}}{\pgfqpoint{3.696000in}{3.696000in}}%
\pgfusepath{clip}%
\pgfsetbuttcap%
\pgfsetroundjoin%
\definecolor{currentfill}{rgb}{0.121569,0.466667,0.705882}%
\pgfsetfillcolor{currentfill}%
\pgfsetfillopacity{0.389370}%
\pgfsetlinewidth{1.003750pt}%
\definecolor{currentstroke}{rgb}{0.121569,0.466667,0.705882}%
\pgfsetstrokecolor{currentstroke}%
\pgfsetstrokeopacity{0.389370}%
\pgfsetdash{}{0pt}%
\pgfpathmoveto{\pgfqpoint{2.137613in}{3.218901in}}%
\pgfpathcurveto{\pgfqpoint{2.145850in}{3.218901in}}{\pgfqpoint{2.153750in}{3.222173in}}{\pgfqpoint{2.159574in}{3.227997in}}%
\pgfpathcurveto{\pgfqpoint{2.165398in}{3.233821in}}{\pgfqpoint{2.168670in}{3.241721in}}{\pgfqpoint{2.168670in}{3.249957in}}%
\pgfpathcurveto{\pgfqpoint{2.168670in}{3.258194in}}{\pgfqpoint{2.165398in}{3.266094in}}{\pgfqpoint{2.159574in}{3.271918in}}%
\pgfpathcurveto{\pgfqpoint{2.153750in}{3.277742in}}{\pgfqpoint{2.145850in}{3.281014in}}{\pgfqpoint{2.137613in}{3.281014in}}%
\pgfpathcurveto{\pgfqpoint{2.129377in}{3.281014in}}{\pgfqpoint{2.121477in}{3.277742in}}{\pgfqpoint{2.115653in}{3.271918in}}%
\pgfpathcurveto{\pgfqpoint{2.109829in}{3.266094in}}{\pgfqpoint{2.106557in}{3.258194in}}{\pgfqpoint{2.106557in}{3.249957in}}%
\pgfpathcurveto{\pgfqpoint{2.106557in}{3.241721in}}{\pgfqpoint{2.109829in}{3.233821in}}{\pgfqpoint{2.115653in}{3.227997in}}%
\pgfpathcurveto{\pgfqpoint{2.121477in}{3.222173in}}{\pgfqpoint{2.129377in}{3.218901in}}{\pgfqpoint{2.137613in}{3.218901in}}%
\pgfpathclose%
\pgfusepath{stroke,fill}%
\end{pgfscope}%
\begin{pgfscope}%
\pgfpathrectangle{\pgfqpoint{0.100000in}{0.220728in}}{\pgfqpoint{3.696000in}{3.696000in}}%
\pgfusepath{clip}%
\pgfsetbuttcap%
\pgfsetroundjoin%
\definecolor{currentfill}{rgb}{0.121569,0.466667,0.705882}%
\pgfsetfillcolor{currentfill}%
\pgfsetfillopacity{0.389706}%
\pgfsetlinewidth{1.003750pt}%
\definecolor{currentstroke}{rgb}{0.121569,0.466667,0.705882}%
\pgfsetstrokecolor{currentstroke}%
\pgfsetstrokeopacity{0.389706}%
\pgfsetdash{}{0pt}%
\pgfpathmoveto{\pgfqpoint{1.532457in}{2.702509in}}%
\pgfpathcurveto{\pgfqpoint{1.540694in}{2.702509in}}{\pgfqpoint{1.548594in}{2.705782in}}{\pgfqpoint{1.554418in}{2.711606in}}%
\pgfpathcurveto{\pgfqpoint{1.560242in}{2.717430in}}{\pgfqpoint{1.563514in}{2.725330in}}{\pgfqpoint{1.563514in}{2.733566in}}%
\pgfpathcurveto{\pgfqpoint{1.563514in}{2.741802in}}{\pgfqpoint{1.560242in}{2.749702in}}{\pgfqpoint{1.554418in}{2.755526in}}%
\pgfpathcurveto{\pgfqpoint{1.548594in}{2.761350in}}{\pgfqpoint{1.540694in}{2.764622in}}{\pgfqpoint{1.532457in}{2.764622in}}%
\pgfpathcurveto{\pgfqpoint{1.524221in}{2.764622in}}{\pgfqpoint{1.516321in}{2.761350in}}{\pgfqpoint{1.510497in}{2.755526in}}%
\pgfpathcurveto{\pgfqpoint{1.504673in}{2.749702in}}{\pgfqpoint{1.501401in}{2.741802in}}{\pgfqpoint{1.501401in}{2.733566in}}%
\pgfpathcurveto{\pgfqpoint{1.501401in}{2.725330in}}{\pgfqpoint{1.504673in}{2.717430in}}{\pgfqpoint{1.510497in}{2.711606in}}%
\pgfpathcurveto{\pgfqpoint{1.516321in}{2.705782in}}{\pgfqpoint{1.524221in}{2.702509in}}{\pgfqpoint{1.532457in}{2.702509in}}%
\pgfpathclose%
\pgfusepath{stroke,fill}%
\end{pgfscope}%
\begin{pgfscope}%
\pgfpathrectangle{\pgfqpoint{0.100000in}{0.220728in}}{\pgfqpoint{3.696000in}{3.696000in}}%
\pgfusepath{clip}%
\pgfsetbuttcap%
\pgfsetroundjoin%
\definecolor{currentfill}{rgb}{0.121569,0.466667,0.705882}%
\pgfsetfillcolor{currentfill}%
\pgfsetfillopacity{0.390363}%
\pgfsetlinewidth{1.003750pt}%
\definecolor{currentstroke}{rgb}{0.121569,0.466667,0.705882}%
\pgfsetstrokecolor{currentstroke}%
\pgfsetstrokeopacity{0.390363}%
\pgfsetdash{}{0pt}%
\pgfpathmoveto{\pgfqpoint{2.149088in}{3.217146in}}%
\pgfpathcurveto{\pgfqpoint{2.157325in}{3.217146in}}{\pgfqpoint{2.165225in}{3.220418in}}{\pgfqpoint{2.171049in}{3.226242in}}%
\pgfpathcurveto{\pgfqpoint{2.176872in}{3.232066in}}{\pgfqpoint{2.180145in}{3.239966in}}{\pgfqpoint{2.180145in}{3.248202in}}%
\pgfpathcurveto{\pgfqpoint{2.180145in}{3.256438in}}{\pgfqpoint{2.176872in}{3.264339in}}{\pgfqpoint{2.171049in}{3.270162in}}%
\pgfpathcurveto{\pgfqpoint{2.165225in}{3.275986in}}{\pgfqpoint{2.157325in}{3.279259in}}{\pgfqpoint{2.149088in}{3.279259in}}%
\pgfpathcurveto{\pgfqpoint{2.140852in}{3.279259in}}{\pgfqpoint{2.132952in}{3.275986in}}{\pgfqpoint{2.127128in}{3.270162in}}%
\pgfpathcurveto{\pgfqpoint{2.121304in}{3.264339in}}{\pgfqpoint{2.118032in}{3.256438in}}{\pgfqpoint{2.118032in}{3.248202in}}%
\pgfpathcurveto{\pgfqpoint{2.118032in}{3.239966in}}{\pgfqpoint{2.121304in}{3.232066in}}{\pgfqpoint{2.127128in}{3.226242in}}%
\pgfpathcurveto{\pgfqpoint{2.132952in}{3.220418in}}{\pgfqpoint{2.140852in}{3.217146in}}{\pgfqpoint{2.149088in}{3.217146in}}%
\pgfpathclose%
\pgfusepath{stroke,fill}%
\end{pgfscope}%
\begin{pgfscope}%
\pgfpathrectangle{\pgfqpoint{0.100000in}{0.220728in}}{\pgfqpoint{3.696000in}{3.696000in}}%
\pgfusepath{clip}%
\pgfsetbuttcap%
\pgfsetroundjoin%
\definecolor{currentfill}{rgb}{0.121569,0.466667,0.705882}%
\pgfsetfillcolor{currentfill}%
\pgfsetfillopacity{0.393286}%
\pgfsetlinewidth{1.003750pt}%
\definecolor{currentstroke}{rgb}{0.121569,0.466667,0.705882}%
\pgfsetstrokecolor{currentstroke}%
\pgfsetstrokeopacity{0.393286}%
\pgfsetdash{}{0pt}%
\pgfpathmoveto{\pgfqpoint{1.525996in}{2.678306in}}%
\pgfpathcurveto{\pgfqpoint{1.534232in}{2.678306in}}{\pgfqpoint{1.542132in}{2.681579in}}{\pgfqpoint{1.547956in}{2.687403in}}%
\pgfpathcurveto{\pgfqpoint{1.553780in}{2.693227in}}{\pgfqpoint{1.557053in}{2.701127in}}{\pgfqpoint{1.557053in}{2.709363in}}%
\pgfpathcurveto{\pgfqpoint{1.557053in}{2.717599in}}{\pgfqpoint{1.553780in}{2.725499in}}{\pgfqpoint{1.547956in}{2.731323in}}%
\pgfpathcurveto{\pgfqpoint{1.542132in}{2.737147in}}{\pgfqpoint{1.534232in}{2.740419in}}{\pgfqpoint{1.525996in}{2.740419in}}%
\pgfpathcurveto{\pgfqpoint{1.517760in}{2.740419in}}{\pgfqpoint{1.509860in}{2.737147in}}{\pgfqpoint{1.504036in}{2.731323in}}%
\pgfpathcurveto{\pgfqpoint{1.498212in}{2.725499in}}{\pgfqpoint{1.494940in}{2.717599in}}{\pgfqpoint{1.494940in}{2.709363in}}%
\pgfpathcurveto{\pgfqpoint{1.494940in}{2.701127in}}{\pgfqpoint{1.498212in}{2.693227in}}{\pgfqpoint{1.504036in}{2.687403in}}%
\pgfpathcurveto{\pgfqpoint{1.509860in}{2.681579in}}{\pgfqpoint{1.517760in}{2.678306in}}{\pgfqpoint{1.525996in}{2.678306in}}%
\pgfpathclose%
\pgfusepath{stroke,fill}%
\end{pgfscope}%
\begin{pgfscope}%
\pgfpathrectangle{\pgfqpoint{0.100000in}{0.220728in}}{\pgfqpoint{3.696000in}{3.696000in}}%
\pgfusepath{clip}%
\pgfsetbuttcap%
\pgfsetroundjoin%
\definecolor{currentfill}{rgb}{0.121569,0.466667,0.705882}%
\pgfsetfillcolor{currentfill}%
\pgfsetfillopacity{0.393672}%
\pgfsetlinewidth{1.003750pt}%
\definecolor{currentstroke}{rgb}{0.121569,0.466667,0.705882}%
\pgfsetstrokecolor{currentstroke}%
\pgfsetstrokeopacity{0.393672}%
\pgfsetdash{}{0pt}%
\pgfpathmoveto{\pgfqpoint{2.153295in}{3.213958in}}%
\pgfpathcurveto{\pgfqpoint{2.161531in}{3.213958in}}{\pgfqpoint{2.169431in}{3.217231in}}{\pgfqpoint{2.175255in}{3.223055in}}%
\pgfpathcurveto{\pgfqpoint{2.181079in}{3.228878in}}{\pgfqpoint{2.184351in}{3.236779in}}{\pgfqpoint{2.184351in}{3.245015in}}%
\pgfpathcurveto{\pgfqpoint{2.184351in}{3.253251in}}{\pgfqpoint{2.181079in}{3.261151in}}{\pgfqpoint{2.175255in}{3.266975in}}%
\pgfpathcurveto{\pgfqpoint{2.169431in}{3.272799in}}{\pgfqpoint{2.161531in}{3.276071in}}{\pgfqpoint{2.153295in}{3.276071in}}%
\pgfpathcurveto{\pgfqpoint{2.145059in}{3.276071in}}{\pgfqpoint{2.137159in}{3.272799in}}{\pgfqpoint{2.131335in}{3.266975in}}%
\pgfpathcurveto{\pgfqpoint{2.125511in}{3.261151in}}{\pgfqpoint{2.122238in}{3.253251in}}{\pgfqpoint{2.122238in}{3.245015in}}%
\pgfpathcurveto{\pgfqpoint{2.122238in}{3.236779in}}{\pgfqpoint{2.125511in}{3.228878in}}{\pgfqpoint{2.131335in}{3.223055in}}%
\pgfpathcurveto{\pgfqpoint{2.137159in}{3.217231in}}{\pgfqpoint{2.145059in}{3.213958in}}{\pgfqpoint{2.153295in}{3.213958in}}%
\pgfpathclose%
\pgfusepath{stroke,fill}%
\end{pgfscope}%
\begin{pgfscope}%
\pgfpathrectangle{\pgfqpoint{0.100000in}{0.220728in}}{\pgfqpoint{3.696000in}{3.696000in}}%
\pgfusepath{clip}%
\pgfsetbuttcap%
\pgfsetroundjoin%
\definecolor{currentfill}{rgb}{0.121569,0.466667,0.705882}%
\pgfsetfillcolor{currentfill}%
\pgfsetfillopacity{0.396063}%
\pgfsetlinewidth{1.003750pt}%
\definecolor{currentstroke}{rgb}{0.121569,0.466667,0.705882}%
\pgfsetstrokecolor{currentstroke}%
\pgfsetstrokeopacity{0.396063}%
\pgfsetdash{}{0pt}%
\pgfpathmoveto{\pgfqpoint{1.514494in}{2.659003in}}%
\pgfpathcurveto{\pgfqpoint{1.522731in}{2.659003in}}{\pgfqpoint{1.530631in}{2.662275in}}{\pgfqpoint{1.536455in}{2.668099in}}%
\pgfpathcurveto{\pgfqpoint{1.542279in}{2.673923in}}{\pgfqpoint{1.545551in}{2.681823in}}{\pgfqpoint{1.545551in}{2.690059in}}%
\pgfpathcurveto{\pgfqpoint{1.545551in}{2.698296in}}{\pgfqpoint{1.542279in}{2.706196in}}{\pgfqpoint{1.536455in}{2.712020in}}%
\pgfpathcurveto{\pgfqpoint{1.530631in}{2.717844in}}{\pgfqpoint{1.522731in}{2.721116in}}{\pgfqpoint{1.514494in}{2.721116in}}%
\pgfpathcurveto{\pgfqpoint{1.506258in}{2.721116in}}{\pgfqpoint{1.498358in}{2.717844in}}{\pgfqpoint{1.492534in}{2.712020in}}%
\pgfpathcurveto{\pgfqpoint{1.486710in}{2.706196in}}{\pgfqpoint{1.483438in}{2.698296in}}{\pgfqpoint{1.483438in}{2.690059in}}%
\pgfpathcurveto{\pgfqpoint{1.483438in}{2.681823in}}{\pgfqpoint{1.486710in}{2.673923in}}{\pgfqpoint{1.492534in}{2.668099in}}%
\pgfpathcurveto{\pgfqpoint{1.498358in}{2.662275in}}{\pgfqpoint{1.506258in}{2.659003in}}{\pgfqpoint{1.514494in}{2.659003in}}%
\pgfpathclose%
\pgfusepath{stroke,fill}%
\end{pgfscope}%
\begin{pgfscope}%
\pgfpathrectangle{\pgfqpoint{0.100000in}{0.220728in}}{\pgfqpoint{3.696000in}{3.696000in}}%
\pgfusepath{clip}%
\pgfsetbuttcap%
\pgfsetroundjoin%
\definecolor{currentfill}{rgb}{0.121569,0.466667,0.705882}%
\pgfsetfillcolor{currentfill}%
\pgfsetfillopacity{0.396284}%
\pgfsetlinewidth{1.003750pt}%
\definecolor{currentstroke}{rgb}{0.121569,0.466667,0.705882}%
\pgfsetstrokecolor{currentstroke}%
\pgfsetstrokeopacity{0.396284}%
\pgfsetdash{}{0pt}%
\pgfpathmoveto{\pgfqpoint{2.161269in}{3.212413in}}%
\pgfpathcurveto{\pgfqpoint{2.169506in}{3.212413in}}{\pgfqpoint{2.177406in}{3.215685in}}{\pgfqpoint{2.183230in}{3.221509in}}%
\pgfpathcurveto{\pgfqpoint{2.189054in}{3.227333in}}{\pgfqpoint{2.192326in}{3.235233in}}{\pgfqpoint{2.192326in}{3.243469in}}%
\pgfpathcurveto{\pgfqpoint{2.192326in}{3.251706in}}{\pgfqpoint{2.189054in}{3.259606in}}{\pgfqpoint{2.183230in}{3.265429in}}%
\pgfpathcurveto{\pgfqpoint{2.177406in}{3.271253in}}{\pgfqpoint{2.169506in}{3.274526in}}{\pgfqpoint{2.161269in}{3.274526in}}%
\pgfpathcurveto{\pgfqpoint{2.153033in}{3.274526in}}{\pgfqpoint{2.145133in}{3.271253in}}{\pgfqpoint{2.139309in}{3.265429in}}%
\pgfpathcurveto{\pgfqpoint{2.133485in}{3.259606in}}{\pgfqpoint{2.130213in}{3.251706in}}{\pgfqpoint{2.130213in}{3.243469in}}%
\pgfpathcurveto{\pgfqpoint{2.130213in}{3.235233in}}{\pgfqpoint{2.133485in}{3.227333in}}{\pgfqpoint{2.139309in}{3.221509in}}%
\pgfpathcurveto{\pgfqpoint{2.145133in}{3.215685in}}{\pgfqpoint{2.153033in}{3.212413in}}{\pgfqpoint{2.161269in}{3.212413in}}%
\pgfpathclose%
\pgfusepath{stroke,fill}%
\end{pgfscope}%
\begin{pgfscope}%
\pgfpathrectangle{\pgfqpoint{0.100000in}{0.220728in}}{\pgfqpoint{3.696000in}{3.696000in}}%
\pgfusepath{clip}%
\pgfsetbuttcap%
\pgfsetroundjoin%
\definecolor{currentfill}{rgb}{0.121569,0.466667,0.705882}%
\pgfsetfillcolor{currentfill}%
\pgfsetfillopacity{0.399041}%
\pgfsetlinewidth{1.003750pt}%
\definecolor{currentstroke}{rgb}{0.121569,0.466667,0.705882}%
\pgfsetstrokecolor{currentstroke}%
\pgfsetstrokeopacity{0.399041}%
\pgfsetdash{}{0pt}%
\pgfpathmoveto{\pgfqpoint{1.507883in}{2.637934in}}%
\pgfpathcurveto{\pgfqpoint{1.516120in}{2.637934in}}{\pgfqpoint{1.524020in}{2.641207in}}{\pgfqpoint{1.529844in}{2.647031in}}%
\pgfpathcurveto{\pgfqpoint{1.535668in}{2.652855in}}{\pgfqpoint{1.538940in}{2.660755in}}{\pgfqpoint{1.538940in}{2.668991in}}%
\pgfpathcurveto{\pgfqpoint{1.538940in}{2.677227in}}{\pgfqpoint{1.535668in}{2.685127in}}{\pgfqpoint{1.529844in}{2.690951in}}%
\pgfpathcurveto{\pgfqpoint{1.524020in}{2.696775in}}{\pgfqpoint{1.516120in}{2.700047in}}{\pgfqpoint{1.507883in}{2.700047in}}%
\pgfpathcurveto{\pgfqpoint{1.499647in}{2.700047in}}{\pgfqpoint{1.491747in}{2.696775in}}{\pgfqpoint{1.485923in}{2.690951in}}%
\pgfpathcurveto{\pgfqpoint{1.480099in}{2.685127in}}{\pgfqpoint{1.476827in}{2.677227in}}{\pgfqpoint{1.476827in}{2.668991in}}%
\pgfpathcurveto{\pgfqpoint{1.476827in}{2.660755in}}{\pgfqpoint{1.480099in}{2.652855in}}{\pgfqpoint{1.485923in}{2.647031in}}%
\pgfpathcurveto{\pgfqpoint{1.491747in}{2.641207in}}{\pgfqpoint{1.499647in}{2.637934in}}{\pgfqpoint{1.507883in}{2.637934in}}%
\pgfpathclose%
\pgfusepath{stroke,fill}%
\end{pgfscope}%
\begin{pgfscope}%
\pgfpathrectangle{\pgfqpoint{0.100000in}{0.220728in}}{\pgfqpoint{3.696000in}{3.696000in}}%
\pgfusepath{clip}%
\pgfsetbuttcap%
\pgfsetroundjoin%
\definecolor{currentfill}{rgb}{0.121569,0.466667,0.705882}%
\pgfsetfillcolor{currentfill}%
\pgfsetfillopacity{0.399127}%
\pgfsetlinewidth{1.003750pt}%
\definecolor{currentstroke}{rgb}{0.121569,0.466667,0.705882}%
\pgfsetstrokecolor{currentstroke}%
\pgfsetstrokeopacity{0.399127}%
\pgfsetdash{}{0pt}%
\pgfpathmoveto{\pgfqpoint{2.169989in}{3.211297in}}%
\pgfpathcurveto{\pgfqpoint{2.178225in}{3.211297in}}{\pgfqpoint{2.186125in}{3.214569in}}{\pgfqpoint{2.191949in}{3.220393in}}%
\pgfpathcurveto{\pgfqpoint{2.197773in}{3.226217in}}{\pgfqpoint{2.201045in}{3.234117in}}{\pgfqpoint{2.201045in}{3.242353in}}%
\pgfpathcurveto{\pgfqpoint{2.201045in}{3.250590in}}{\pgfqpoint{2.197773in}{3.258490in}}{\pgfqpoint{2.191949in}{3.264314in}}%
\pgfpathcurveto{\pgfqpoint{2.186125in}{3.270138in}}{\pgfqpoint{2.178225in}{3.273410in}}{\pgfqpoint{2.169989in}{3.273410in}}%
\pgfpathcurveto{\pgfqpoint{2.161753in}{3.273410in}}{\pgfqpoint{2.153853in}{3.270138in}}{\pgfqpoint{2.148029in}{3.264314in}}%
\pgfpathcurveto{\pgfqpoint{2.142205in}{3.258490in}}{\pgfqpoint{2.138932in}{3.250590in}}{\pgfqpoint{2.138932in}{3.242353in}}%
\pgfpathcurveto{\pgfqpoint{2.138932in}{3.234117in}}{\pgfqpoint{2.142205in}{3.226217in}}{\pgfqpoint{2.148029in}{3.220393in}}%
\pgfpathcurveto{\pgfqpoint{2.153853in}{3.214569in}}{\pgfqpoint{2.161753in}{3.211297in}}{\pgfqpoint{2.169989in}{3.211297in}}%
\pgfpathclose%
\pgfusepath{stroke,fill}%
\end{pgfscope}%
\begin{pgfscope}%
\pgfpathrectangle{\pgfqpoint{0.100000in}{0.220728in}}{\pgfqpoint{3.696000in}{3.696000in}}%
\pgfusepath{clip}%
\pgfsetbuttcap%
\pgfsetroundjoin%
\definecolor{currentfill}{rgb}{0.121569,0.466667,0.705882}%
\pgfsetfillcolor{currentfill}%
\pgfsetfillopacity{0.401200}%
\pgfsetlinewidth{1.003750pt}%
\definecolor{currentstroke}{rgb}{0.121569,0.466667,0.705882}%
\pgfsetstrokecolor{currentstroke}%
\pgfsetstrokeopacity{0.401200}%
\pgfsetdash{}{0pt}%
\pgfpathmoveto{\pgfqpoint{1.496059in}{2.621300in}}%
\pgfpathcurveto{\pgfqpoint{1.504295in}{2.621300in}}{\pgfqpoint{1.512195in}{2.624572in}}{\pgfqpoint{1.518019in}{2.630396in}}%
\pgfpathcurveto{\pgfqpoint{1.523843in}{2.636220in}}{\pgfqpoint{1.527115in}{2.644120in}}{\pgfqpoint{1.527115in}{2.652356in}}%
\pgfpathcurveto{\pgfqpoint{1.527115in}{2.660593in}}{\pgfqpoint{1.523843in}{2.668493in}}{\pgfqpoint{1.518019in}{2.674317in}}%
\pgfpathcurveto{\pgfqpoint{1.512195in}{2.680141in}}{\pgfqpoint{1.504295in}{2.683413in}}{\pgfqpoint{1.496059in}{2.683413in}}%
\pgfpathcurveto{\pgfqpoint{1.487823in}{2.683413in}}{\pgfqpoint{1.479922in}{2.680141in}}{\pgfqpoint{1.474099in}{2.674317in}}%
\pgfpathcurveto{\pgfqpoint{1.468275in}{2.668493in}}{\pgfqpoint{1.465002in}{2.660593in}}{\pgfqpoint{1.465002in}{2.652356in}}%
\pgfpathcurveto{\pgfqpoint{1.465002in}{2.644120in}}{\pgfqpoint{1.468275in}{2.636220in}}{\pgfqpoint{1.474099in}{2.630396in}}%
\pgfpathcurveto{\pgfqpoint{1.479922in}{2.624572in}}{\pgfqpoint{1.487823in}{2.621300in}}{\pgfqpoint{1.496059in}{2.621300in}}%
\pgfpathclose%
\pgfusepath{stroke,fill}%
\end{pgfscope}%
\begin{pgfscope}%
\pgfpathrectangle{\pgfqpoint{0.100000in}{0.220728in}}{\pgfqpoint{3.696000in}{3.696000in}}%
\pgfusepath{clip}%
\pgfsetbuttcap%
\pgfsetroundjoin%
\definecolor{currentfill}{rgb}{0.121569,0.466667,0.705882}%
\pgfsetfillcolor{currentfill}%
\pgfsetfillopacity{0.401278}%
\pgfsetlinewidth{1.003750pt}%
\definecolor{currentstroke}{rgb}{0.121569,0.466667,0.705882}%
\pgfsetstrokecolor{currentstroke}%
\pgfsetstrokeopacity{0.401278}%
\pgfsetdash{}{0pt}%
\pgfpathmoveto{\pgfqpoint{2.182484in}{3.208757in}}%
\pgfpathcurveto{\pgfqpoint{2.190720in}{3.208757in}}{\pgfqpoint{2.198620in}{3.212029in}}{\pgfqpoint{2.204444in}{3.217853in}}%
\pgfpathcurveto{\pgfqpoint{2.210268in}{3.223677in}}{\pgfqpoint{2.213541in}{3.231577in}}{\pgfqpoint{2.213541in}{3.239813in}}%
\pgfpathcurveto{\pgfqpoint{2.213541in}{3.248049in}}{\pgfqpoint{2.210268in}{3.255949in}}{\pgfqpoint{2.204444in}{3.261773in}}%
\pgfpathcurveto{\pgfqpoint{2.198620in}{3.267597in}}{\pgfqpoint{2.190720in}{3.270870in}}{\pgfqpoint{2.182484in}{3.270870in}}%
\pgfpathcurveto{\pgfqpoint{2.174248in}{3.270870in}}{\pgfqpoint{2.166348in}{3.267597in}}{\pgfqpoint{2.160524in}{3.261773in}}%
\pgfpathcurveto{\pgfqpoint{2.154700in}{3.255949in}}{\pgfqpoint{2.151428in}{3.248049in}}{\pgfqpoint{2.151428in}{3.239813in}}%
\pgfpathcurveto{\pgfqpoint{2.151428in}{3.231577in}}{\pgfqpoint{2.154700in}{3.223677in}}{\pgfqpoint{2.160524in}{3.217853in}}%
\pgfpathcurveto{\pgfqpoint{2.166348in}{3.212029in}}{\pgfqpoint{2.174248in}{3.208757in}}{\pgfqpoint{2.182484in}{3.208757in}}%
\pgfpathclose%
\pgfusepath{stroke,fill}%
\end{pgfscope}%
\begin{pgfscope}%
\pgfpathrectangle{\pgfqpoint{0.100000in}{0.220728in}}{\pgfqpoint{3.696000in}{3.696000in}}%
\pgfusepath{clip}%
\pgfsetbuttcap%
\pgfsetroundjoin%
\definecolor{currentfill}{rgb}{0.121569,0.466667,0.705882}%
\pgfsetfillcolor{currentfill}%
\pgfsetfillopacity{0.401941}%
\pgfsetlinewidth{1.003750pt}%
\definecolor{currentstroke}{rgb}{0.121569,0.466667,0.705882}%
\pgfsetstrokecolor{currentstroke}%
\pgfsetstrokeopacity{0.401941}%
\pgfsetdash{}{0pt}%
\pgfpathmoveto{\pgfqpoint{2.189304in}{3.204913in}}%
\pgfpathcurveto{\pgfqpoint{2.197540in}{3.204913in}}{\pgfqpoint{2.205440in}{3.208185in}}{\pgfqpoint{2.211264in}{3.214009in}}%
\pgfpathcurveto{\pgfqpoint{2.217088in}{3.219833in}}{\pgfqpoint{2.220360in}{3.227733in}}{\pgfqpoint{2.220360in}{3.235969in}}%
\pgfpathcurveto{\pgfqpoint{2.220360in}{3.244205in}}{\pgfqpoint{2.217088in}{3.252105in}}{\pgfqpoint{2.211264in}{3.257929in}}%
\pgfpathcurveto{\pgfqpoint{2.205440in}{3.263753in}}{\pgfqpoint{2.197540in}{3.267026in}}{\pgfqpoint{2.189304in}{3.267026in}}%
\pgfpathcurveto{\pgfqpoint{2.181067in}{3.267026in}}{\pgfqpoint{2.173167in}{3.263753in}}{\pgfqpoint{2.167343in}{3.257929in}}%
\pgfpathcurveto{\pgfqpoint{2.161520in}{3.252105in}}{\pgfqpoint{2.158247in}{3.244205in}}{\pgfqpoint{2.158247in}{3.235969in}}%
\pgfpathcurveto{\pgfqpoint{2.158247in}{3.227733in}}{\pgfqpoint{2.161520in}{3.219833in}}{\pgfqpoint{2.167343in}{3.214009in}}%
\pgfpathcurveto{\pgfqpoint{2.173167in}{3.208185in}}{\pgfqpoint{2.181067in}{3.204913in}}{\pgfqpoint{2.189304in}{3.204913in}}%
\pgfpathclose%
\pgfusepath{stroke,fill}%
\end{pgfscope}%
\begin{pgfscope}%
\pgfpathrectangle{\pgfqpoint{0.100000in}{0.220728in}}{\pgfqpoint{3.696000in}{3.696000in}}%
\pgfusepath{clip}%
\pgfsetbuttcap%
\pgfsetroundjoin%
\definecolor{currentfill}{rgb}{0.121569,0.466667,0.705882}%
\pgfsetfillcolor{currentfill}%
\pgfsetfillopacity{0.403248}%
\pgfsetlinewidth{1.003750pt}%
\definecolor{currentstroke}{rgb}{0.121569,0.466667,0.705882}%
\pgfsetstrokecolor{currentstroke}%
\pgfsetstrokeopacity{0.403248}%
\pgfsetdash{}{0pt}%
\pgfpathmoveto{\pgfqpoint{2.197114in}{3.202661in}}%
\pgfpathcurveto{\pgfqpoint{2.205351in}{3.202661in}}{\pgfqpoint{2.213251in}{3.205933in}}{\pgfqpoint{2.219075in}{3.211757in}}%
\pgfpathcurveto{\pgfqpoint{2.224898in}{3.217581in}}{\pgfqpoint{2.228171in}{3.225481in}}{\pgfqpoint{2.228171in}{3.233717in}}%
\pgfpathcurveto{\pgfqpoint{2.228171in}{3.241953in}}{\pgfqpoint{2.224898in}{3.249853in}}{\pgfqpoint{2.219075in}{3.255677in}}%
\pgfpathcurveto{\pgfqpoint{2.213251in}{3.261501in}}{\pgfqpoint{2.205351in}{3.264774in}}{\pgfqpoint{2.197114in}{3.264774in}}%
\pgfpathcurveto{\pgfqpoint{2.188878in}{3.264774in}}{\pgfqpoint{2.180978in}{3.261501in}}{\pgfqpoint{2.175154in}{3.255677in}}%
\pgfpathcurveto{\pgfqpoint{2.169330in}{3.249853in}}{\pgfqpoint{2.166058in}{3.241953in}}{\pgfqpoint{2.166058in}{3.233717in}}%
\pgfpathcurveto{\pgfqpoint{2.166058in}{3.225481in}}{\pgfqpoint{2.169330in}{3.217581in}}{\pgfqpoint{2.175154in}{3.211757in}}%
\pgfpathcurveto{\pgfqpoint{2.180978in}{3.205933in}}{\pgfqpoint{2.188878in}{3.202661in}}{\pgfqpoint{2.197114in}{3.202661in}}%
\pgfpathclose%
\pgfusepath{stroke,fill}%
\end{pgfscope}%
\begin{pgfscope}%
\pgfpathrectangle{\pgfqpoint{0.100000in}{0.220728in}}{\pgfqpoint{3.696000in}{3.696000in}}%
\pgfusepath{clip}%
\pgfsetbuttcap%
\pgfsetroundjoin%
\definecolor{currentfill}{rgb}{0.121569,0.466667,0.705882}%
\pgfsetfillcolor{currentfill}%
\pgfsetfillopacity{0.403885}%
\pgfsetlinewidth{1.003750pt}%
\definecolor{currentstroke}{rgb}{0.121569,0.466667,0.705882}%
\pgfsetstrokecolor{currentstroke}%
\pgfsetstrokeopacity{0.403885}%
\pgfsetdash{}{0pt}%
\pgfpathmoveto{\pgfqpoint{1.493641in}{2.602513in}}%
\pgfpathcurveto{\pgfqpoint{1.501877in}{2.602513in}}{\pgfqpoint{1.509777in}{2.605785in}}{\pgfqpoint{1.515601in}{2.611609in}}%
\pgfpathcurveto{\pgfqpoint{1.521425in}{2.617433in}}{\pgfqpoint{1.524698in}{2.625333in}}{\pgfqpoint{1.524698in}{2.633569in}}%
\pgfpathcurveto{\pgfqpoint{1.524698in}{2.641805in}}{\pgfqpoint{1.521425in}{2.649705in}}{\pgfqpoint{1.515601in}{2.655529in}}%
\pgfpathcurveto{\pgfqpoint{1.509777in}{2.661353in}}{\pgfqpoint{1.501877in}{2.664626in}}{\pgfqpoint{1.493641in}{2.664626in}}%
\pgfpathcurveto{\pgfqpoint{1.485405in}{2.664626in}}{\pgfqpoint{1.477505in}{2.661353in}}{\pgfqpoint{1.471681in}{2.655529in}}%
\pgfpathcurveto{\pgfqpoint{1.465857in}{2.649705in}}{\pgfqpoint{1.462585in}{2.641805in}}{\pgfqpoint{1.462585in}{2.633569in}}%
\pgfpathcurveto{\pgfqpoint{1.462585in}{2.625333in}}{\pgfqpoint{1.465857in}{2.617433in}}{\pgfqpoint{1.471681in}{2.611609in}}%
\pgfpathcurveto{\pgfqpoint{1.477505in}{2.605785in}}{\pgfqpoint{1.485405in}{2.602513in}}{\pgfqpoint{1.493641in}{2.602513in}}%
\pgfpathclose%
\pgfusepath{stroke,fill}%
\end{pgfscope}%
\begin{pgfscope}%
\pgfpathrectangle{\pgfqpoint{0.100000in}{0.220728in}}{\pgfqpoint{3.696000in}{3.696000in}}%
\pgfusepath{clip}%
\pgfsetbuttcap%
\pgfsetroundjoin%
\definecolor{currentfill}{rgb}{0.121569,0.466667,0.705882}%
\pgfsetfillcolor{currentfill}%
\pgfsetfillopacity{0.404958}%
\pgfsetlinewidth{1.003750pt}%
\definecolor{currentstroke}{rgb}{0.121569,0.466667,0.705882}%
\pgfsetstrokecolor{currentstroke}%
\pgfsetstrokeopacity{0.404958}%
\pgfsetdash{}{0pt}%
\pgfpathmoveto{\pgfqpoint{1.483139in}{2.590327in}}%
\pgfpathcurveto{\pgfqpoint{1.491376in}{2.590327in}}{\pgfqpoint{1.499276in}{2.593600in}}{\pgfqpoint{1.505100in}{2.599424in}}%
\pgfpathcurveto{\pgfqpoint{1.510923in}{2.605248in}}{\pgfqpoint{1.514196in}{2.613148in}}{\pgfqpoint{1.514196in}{2.621384in}}%
\pgfpathcurveto{\pgfqpoint{1.514196in}{2.629620in}}{\pgfqpoint{1.510923in}{2.637520in}}{\pgfqpoint{1.505100in}{2.643344in}}%
\pgfpathcurveto{\pgfqpoint{1.499276in}{2.649168in}}{\pgfqpoint{1.491376in}{2.652440in}}{\pgfqpoint{1.483139in}{2.652440in}}%
\pgfpathcurveto{\pgfqpoint{1.474903in}{2.652440in}}{\pgfqpoint{1.467003in}{2.649168in}}{\pgfqpoint{1.461179in}{2.643344in}}%
\pgfpathcurveto{\pgfqpoint{1.455355in}{2.637520in}}{\pgfqpoint{1.452083in}{2.629620in}}{\pgfqpoint{1.452083in}{2.621384in}}%
\pgfpathcurveto{\pgfqpoint{1.452083in}{2.613148in}}{\pgfqpoint{1.455355in}{2.605248in}}{\pgfqpoint{1.461179in}{2.599424in}}%
\pgfpathcurveto{\pgfqpoint{1.467003in}{2.593600in}}{\pgfqpoint{1.474903in}{2.590327in}}{\pgfqpoint{1.483139in}{2.590327in}}%
\pgfpathclose%
\pgfusepath{stroke,fill}%
\end{pgfscope}%
\begin{pgfscope}%
\pgfpathrectangle{\pgfqpoint{0.100000in}{0.220728in}}{\pgfqpoint{3.696000in}{3.696000in}}%
\pgfusepath{clip}%
\pgfsetbuttcap%
\pgfsetroundjoin%
\definecolor{currentfill}{rgb}{0.121569,0.466667,0.705882}%
\pgfsetfillcolor{currentfill}%
\pgfsetfillopacity{0.405095}%
\pgfsetlinewidth{1.003750pt}%
\definecolor{currentstroke}{rgb}{0.121569,0.466667,0.705882}%
\pgfsetstrokecolor{currentstroke}%
\pgfsetstrokeopacity{0.405095}%
\pgfsetdash{}{0pt}%
\pgfpathmoveto{\pgfqpoint{2.206427in}{3.200938in}}%
\pgfpathcurveto{\pgfqpoint{2.214663in}{3.200938in}}{\pgfqpoint{2.222563in}{3.204210in}}{\pgfqpoint{2.228387in}{3.210034in}}%
\pgfpathcurveto{\pgfqpoint{2.234211in}{3.215858in}}{\pgfqpoint{2.237483in}{3.223758in}}{\pgfqpoint{2.237483in}{3.231994in}}%
\pgfpathcurveto{\pgfqpoint{2.237483in}{3.240230in}}{\pgfqpoint{2.234211in}{3.248130in}}{\pgfqpoint{2.228387in}{3.253954in}}%
\pgfpathcurveto{\pgfqpoint{2.222563in}{3.259778in}}{\pgfqpoint{2.214663in}{3.263051in}}{\pgfqpoint{2.206427in}{3.263051in}}%
\pgfpathcurveto{\pgfqpoint{2.198190in}{3.263051in}}{\pgfqpoint{2.190290in}{3.259778in}}{\pgfqpoint{2.184466in}{3.253954in}}%
\pgfpathcurveto{\pgfqpoint{2.178642in}{3.248130in}}{\pgfqpoint{2.175370in}{3.240230in}}{\pgfqpoint{2.175370in}{3.231994in}}%
\pgfpathcurveto{\pgfqpoint{2.175370in}{3.223758in}}{\pgfqpoint{2.178642in}{3.215858in}}{\pgfqpoint{2.184466in}{3.210034in}}%
\pgfpathcurveto{\pgfqpoint{2.190290in}{3.204210in}}{\pgfqpoint{2.198190in}{3.200938in}}{\pgfqpoint{2.206427in}{3.200938in}}%
\pgfpathclose%
\pgfusepath{stroke,fill}%
\end{pgfscope}%
\begin{pgfscope}%
\pgfpathrectangle{\pgfqpoint{0.100000in}{0.220728in}}{\pgfqpoint{3.696000in}{3.696000in}}%
\pgfusepath{clip}%
\pgfsetbuttcap%
\pgfsetroundjoin%
\definecolor{currentfill}{rgb}{0.121569,0.466667,0.705882}%
\pgfsetfillcolor{currentfill}%
\pgfsetfillopacity{0.407029}%
\pgfsetlinewidth{1.003750pt}%
\definecolor{currentstroke}{rgb}{0.121569,0.466667,0.705882}%
\pgfsetstrokecolor{currentstroke}%
\pgfsetstrokeopacity{0.407029}%
\pgfsetdash{}{0pt}%
\pgfpathmoveto{\pgfqpoint{1.480991in}{2.577269in}}%
\pgfpathcurveto{\pgfqpoint{1.489228in}{2.577269in}}{\pgfqpoint{1.497128in}{2.580542in}}{\pgfqpoint{1.502952in}{2.586365in}}%
\pgfpathcurveto{\pgfqpoint{1.508776in}{2.592189in}}{\pgfqpoint{1.512048in}{2.600089in}}{\pgfqpoint{1.512048in}{2.608326in}}%
\pgfpathcurveto{\pgfqpoint{1.512048in}{2.616562in}}{\pgfqpoint{1.508776in}{2.624462in}}{\pgfqpoint{1.502952in}{2.630286in}}%
\pgfpathcurveto{\pgfqpoint{1.497128in}{2.636110in}}{\pgfqpoint{1.489228in}{2.639382in}}{\pgfqpoint{1.480991in}{2.639382in}}%
\pgfpathcurveto{\pgfqpoint{1.472755in}{2.639382in}}{\pgfqpoint{1.464855in}{2.636110in}}{\pgfqpoint{1.459031in}{2.630286in}}%
\pgfpathcurveto{\pgfqpoint{1.453207in}{2.624462in}}{\pgfqpoint{1.449935in}{2.616562in}}{\pgfqpoint{1.449935in}{2.608326in}}%
\pgfpathcurveto{\pgfqpoint{1.449935in}{2.600089in}}{\pgfqpoint{1.453207in}{2.592189in}}{\pgfqpoint{1.459031in}{2.586365in}}%
\pgfpathcurveto{\pgfqpoint{1.464855in}{2.580542in}}{\pgfqpoint{1.472755in}{2.577269in}}{\pgfqpoint{1.480991in}{2.577269in}}%
\pgfpathclose%
\pgfusepath{stroke,fill}%
\end{pgfscope}%
\begin{pgfscope}%
\pgfpathrectangle{\pgfqpoint{0.100000in}{0.220728in}}{\pgfqpoint{3.696000in}{3.696000in}}%
\pgfusepath{clip}%
\pgfsetbuttcap%
\pgfsetroundjoin%
\definecolor{currentfill}{rgb}{0.121569,0.466667,0.705882}%
\pgfsetfillcolor{currentfill}%
\pgfsetfillopacity{0.408262}%
\pgfsetlinewidth{1.003750pt}%
\definecolor{currentstroke}{rgb}{0.121569,0.466667,0.705882}%
\pgfsetstrokecolor{currentstroke}%
\pgfsetstrokeopacity{0.408262}%
\pgfsetdash{}{0pt}%
\pgfpathmoveto{\pgfqpoint{2.233668in}{3.192694in}}%
\pgfpathcurveto{\pgfqpoint{2.241904in}{3.192694in}}{\pgfqpoint{2.249804in}{3.195967in}}{\pgfqpoint{2.255628in}{3.201791in}}%
\pgfpathcurveto{\pgfqpoint{2.261452in}{3.207615in}}{\pgfqpoint{2.264724in}{3.215515in}}{\pgfqpoint{2.264724in}{3.223751in}}%
\pgfpathcurveto{\pgfqpoint{2.264724in}{3.231987in}}{\pgfqpoint{2.261452in}{3.239887in}}{\pgfqpoint{2.255628in}{3.245711in}}%
\pgfpathcurveto{\pgfqpoint{2.249804in}{3.251535in}}{\pgfqpoint{2.241904in}{3.254807in}}{\pgfqpoint{2.233668in}{3.254807in}}%
\pgfpathcurveto{\pgfqpoint{2.225431in}{3.254807in}}{\pgfqpoint{2.217531in}{3.251535in}}{\pgfqpoint{2.211707in}{3.245711in}}%
\pgfpathcurveto{\pgfqpoint{2.205883in}{3.239887in}}{\pgfqpoint{2.202611in}{3.231987in}}{\pgfqpoint{2.202611in}{3.223751in}}%
\pgfpathcurveto{\pgfqpoint{2.202611in}{3.215515in}}{\pgfqpoint{2.205883in}{3.207615in}}{\pgfqpoint{2.211707in}{3.201791in}}%
\pgfpathcurveto{\pgfqpoint{2.217531in}{3.195967in}}{\pgfqpoint{2.225431in}{3.192694in}}{\pgfqpoint{2.233668in}{3.192694in}}%
\pgfpathclose%
\pgfusepath{stroke,fill}%
\end{pgfscope}%
\begin{pgfscope}%
\pgfpathrectangle{\pgfqpoint{0.100000in}{0.220728in}}{\pgfqpoint{3.696000in}{3.696000in}}%
\pgfusepath{clip}%
\pgfsetbuttcap%
\pgfsetroundjoin%
\definecolor{currentfill}{rgb}{0.121569,0.466667,0.705882}%
\pgfsetfillcolor{currentfill}%
\pgfsetfillopacity{0.408909}%
\pgfsetlinewidth{1.003750pt}%
\definecolor{currentstroke}{rgb}{0.121569,0.466667,0.705882}%
\pgfsetstrokecolor{currentstroke}%
\pgfsetstrokeopacity{0.408909}%
\pgfsetdash{}{0pt}%
\pgfpathmoveto{\pgfqpoint{2.217584in}{3.199138in}}%
\pgfpathcurveto{\pgfqpoint{2.225820in}{3.199138in}}{\pgfqpoint{2.233720in}{3.202411in}}{\pgfqpoint{2.239544in}{3.208235in}}%
\pgfpathcurveto{\pgfqpoint{2.245368in}{3.214059in}}{\pgfqpoint{2.248641in}{3.221959in}}{\pgfqpoint{2.248641in}{3.230195in}}%
\pgfpathcurveto{\pgfqpoint{2.248641in}{3.238431in}}{\pgfqpoint{2.245368in}{3.246331in}}{\pgfqpoint{2.239544in}{3.252155in}}%
\pgfpathcurveto{\pgfqpoint{2.233720in}{3.257979in}}{\pgfqpoint{2.225820in}{3.261251in}}{\pgfqpoint{2.217584in}{3.261251in}}%
\pgfpathcurveto{\pgfqpoint{2.209348in}{3.261251in}}{\pgfqpoint{2.201448in}{3.257979in}}{\pgfqpoint{2.195624in}{3.252155in}}%
\pgfpathcurveto{\pgfqpoint{2.189800in}{3.246331in}}{\pgfqpoint{2.186528in}{3.238431in}}{\pgfqpoint{2.186528in}{3.230195in}}%
\pgfpathcurveto{\pgfqpoint{2.186528in}{3.221959in}}{\pgfqpoint{2.189800in}{3.214059in}}{\pgfqpoint{2.195624in}{3.208235in}}%
\pgfpathcurveto{\pgfqpoint{2.201448in}{3.202411in}}{\pgfqpoint{2.209348in}{3.199138in}}{\pgfqpoint{2.217584in}{3.199138in}}%
\pgfpathclose%
\pgfusepath{stroke,fill}%
\end{pgfscope}%
\begin{pgfscope}%
\pgfpathrectangle{\pgfqpoint{0.100000in}{0.220728in}}{\pgfqpoint{3.696000in}{3.696000in}}%
\pgfusepath{clip}%
\pgfsetbuttcap%
\pgfsetroundjoin%
\definecolor{currentfill}{rgb}{0.121569,0.466667,0.705882}%
\pgfsetfillcolor{currentfill}%
\pgfsetfillopacity{0.409276}%
\pgfsetlinewidth{1.003750pt}%
\definecolor{currentstroke}{rgb}{0.121569,0.466667,0.705882}%
\pgfsetstrokecolor{currentstroke}%
\pgfsetstrokeopacity{0.409276}%
\pgfsetdash{}{0pt}%
\pgfpathmoveto{\pgfqpoint{1.465754in}{2.559684in}}%
\pgfpathcurveto{\pgfqpoint{1.473990in}{2.559684in}}{\pgfqpoint{1.481890in}{2.562957in}}{\pgfqpoint{1.487714in}{2.568781in}}%
\pgfpathcurveto{\pgfqpoint{1.493538in}{2.574605in}}{\pgfqpoint{1.496810in}{2.582505in}}{\pgfqpoint{1.496810in}{2.590741in}}%
\pgfpathcurveto{\pgfqpoint{1.496810in}{2.598977in}}{\pgfqpoint{1.493538in}{2.606877in}}{\pgfqpoint{1.487714in}{2.612701in}}%
\pgfpathcurveto{\pgfqpoint{1.481890in}{2.618525in}}{\pgfqpoint{1.473990in}{2.621797in}}{\pgfqpoint{1.465754in}{2.621797in}}%
\pgfpathcurveto{\pgfqpoint{1.457518in}{2.621797in}}{\pgfqpoint{1.449617in}{2.618525in}}{\pgfqpoint{1.443794in}{2.612701in}}%
\pgfpathcurveto{\pgfqpoint{1.437970in}{2.606877in}}{\pgfqpoint{1.434697in}{2.598977in}}{\pgfqpoint{1.434697in}{2.590741in}}%
\pgfpathcurveto{\pgfqpoint{1.434697in}{2.582505in}}{\pgfqpoint{1.437970in}{2.574605in}}{\pgfqpoint{1.443794in}{2.568781in}}%
\pgfpathcurveto{\pgfqpoint{1.449617in}{2.562957in}}{\pgfqpoint{1.457518in}{2.559684in}}{\pgfqpoint{1.465754in}{2.559684in}}%
\pgfpathclose%
\pgfusepath{stroke,fill}%
\end{pgfscope}%
\begin{pgfscope}%
\pgfpathrectangle{\pgfqpoint{0.100000in}{0.220728in}}{\pgfqpoint{3.696000in}{3.696000in}}%
\pgfusepath{clip}%
\pgfsetbuttcap%
\pgfsetroundjoin%
\definecolor{currentfill}{rgb}{0.121569,0.466667,0.705882}%
\pgfsetfillcolor{currentfill}%
\pgfsetfillopacity{0.410538}%
\pgfsetlinewidth{1.003750pt}%
\definecolor{currentstroke}{rgb}{0.121569,0.466667,0.705882}%
\pgfsetstrokecolor{currentstroke}%
\pgfsetstrokeopacity{0.410538}%
\pgfsetdash{}{0pt}%
\pgfpathmoveto{\pgfqpoint{2.248931in}{3.187014in}}%
\pgfpathcurveto{\pgfqpoint{2.257167in}{3.187014in}}{\pgfqpoint{2.265067in}{3.190286in}}{\pgfqpoint{2.270891in}{3.196110in}}%
\pgfpathcurveto{\pgfqpoint{2.276715in}{3.201934in}}{\pgfqpoint{2.279988in}{3.209834in}}{\pgfqpoint{2.279988in}{3.218070in}}%
\pgfpathcurveto{\pgfqpoint{2.279988in}{3.226306in}}{\pgfqpoint{2.276715in}{3.234206in}}{\pgfqpoint{2.270891in}{3.240030in}}%
\pgfpathcurveto{\pgfqpoint{2.265067in}{3.245854in}}{\pgfqpoint{2.257167in}{3.249127in}}{\pgfqpoint{2.248931in}{3.249127in}}%
\pgfpathcurveto{\pgfqpoint{2.240695in}{3.249127in}}{\pgfqpoint{2.232795in}{3.245854in}}{\pgfqpoint{2.226971in}{3.240030in}}%
\pgfpathcurveto{\pgfqpoint{2.221147in}{3.234206in}}{\pgfqpoint{2.217875in}{3.226306in}}{\pgfqpoint{2.217875in}{3.218070in}}%
\pgfpathcurveto{\pgfqpoint{2.217875in}{3.209834in}}{\pgfqpoint{2.221147in}{3.201934in}}{\pgfqpoint{2.226971in}{3.196110in}}%
\pgfpathcurveto{\pgfqpoint{2.232795in}{3.190286in}}{\pgfqpoint{2.240695in}{3.187014in}}{\pgfqpoint{2.248931in}{3.187014in}}%
\pgfpathclose%
\pgfusepath{stroke,fill}%
\end{pgfscope}%
\begin{pgfscope}%
\pgfpathrectangle{\pgfqpoint{0.100000in}{0.220728in}}{\pgfqpoint{3.696000in}{3.696000in}}%
\pgfusepath{clip}%
\pgfsetbuttcap%
\pgfsetroundjoin%
\definecolor{currentfill}{rgb}{0.121569,0.466667,0.705882}%
\pgfsetfillcolor{currentfill}%
\pgfsetfillopacity{0.412794}%
\pgfsetlinewidth{1.003750pt}%
\definecolor{currentstroke}{rgb}{0.121569,0.466667,0.705882}%
\pgfsetstrokecolor{currentstroke}%
\pgfsetstrokeopacity{0.412794}%
\pgfsetdash{}{0pt}%
\pgfpathmoveto{\pgfqpoint{1.457774in}{2.539848in}}%
\pgfpathcurveto{\pgfqpoint{1.466010in}{2.539848in}}{\pgfqpoint{1.473910in}{2.543120in}}{\pgfqpoint{1.479734in}{2.548944in}}%
\pgfpathcurveto{\pgfqpoint{1.485558in}{2.554768in}}{\pgfqpoint{1.488831in}{2.562668in}}{\pgfqpoint{1.488831in}{2.570904in}}%
\pgfpathcurveto{\pgfqpoint{1.488831in}{2.579140in}}{\pgfqpoint{1.485558in}{2.587040in}}{\pgfqpoint{1.479734in}{2.592864in}}%
\pgfpathcurveto{\pgfqpoint{1.473910in}{2.598688in}}{\pgfqpoint{1.466010in}{2.601961in}}{\pgfqpoint{1.457774in}{2.601961in}}%
\pgfpathcurveto{\pgfqpoint{1.449538in}{2.601961in}}{\pgfqpoint{1.441638in}{2.598688in}}{\pgfqpoint{1.435814in}{2.592864in}}%
\pgfpathcurveto{\pgfqpoint{1.429990in}{2.587040in}}{\pgfqpoint{1.426718in}{2.579140in}}{\pgfqpoint{1.426718in}{2.570904in}}%
\pgfpathcurveto{\pgfqpoint{1.426718in}{2.562668in}}{\pgfqpoint{1.429990in}{2.554768in}}{\pgfqpoint{1.435814in}{2.548944in}}%
\pgfpathcurveto{\pgfqpoint{1.441638in}{2.543120in}}{\pgfqpoint{1.449538in}{2.539848in}}{\pgfqpoint{1.457774in}{2.539848in}}%
\pgfpathclose%
\pgfusepath{stroke,fill}%
\end{pgfscope}%
\begin{pgfscope}%
\pgfpathrectangle{\pgfqpoint{0.100000in}{0.220728in}}{\pgfqpoint{3.696000in}{3.696000in}}%
\pgfusepath{clip}%
\pgfsetbuttcap%
\pgfsetroundjoin%
\definecolor{currentfill}{rgb}{0.121569,0.466667,0.705882}%
\pgfsetfillcolor{currentfill}%
\pgfsetfillopacity{0.415710}%
\pgfsetlinewidth{1.003750pt}%
\definecolor{currentstroke}{rgb}{0.121569,0.466667,0.705882}%
\pgfsetstrokecolor{currentstroke}%
\pgfsetstrokeopacity{0.415710}%
\pgfsetdash{}{0pt}%
\pgfpathmoveto{\pgfqpoint{1.447444in}{2.522597in}}%
\pgfpathcurveto{\pgfqpoint{1.455680in}{2.522597in}}{\pgfqpoint{1.463580in}{2.525870in}}{\pgfqpoint{1.469404in}{2.531694in}}%
\pgfpathcurveto{\pgfqpoint{1.475228in}{2.537517in}}{\pgfqpoint{1.478500in}{2.545418in}}{\pgfqpoint{1.478500in}{2.553654in}}%
\pgfpathcurveto{\pgfqpoint{1.478500in}{2.561890in}}{\pgfqpoint{1.475228in}{2.569790in}}{\pgfqpoint{1.469404in}{2.575614in}}%
\pgfpathcurveto{\pgfqpoint{1.463580in}{2.581438in}}{\pgfqpoint{1.455680in}{2.584710in}}{\pgfqpoint{1.447444in}{2.584710in}}%
\pgfpathcurveto{\pgfqpoint{1.439208in}{2.584710in}}{\pgfqpoint{1.431308in}{2.581438in}}{\pgfqpoint{1.425484in}{2.575614in}}%
\pgfpathcurveto{\pgfqpoint{1.419660in}{2.569790in}}{\pgfqpoint{1.416388in}{2.561890in}}{\pgfqpoint{1.416388in}{2.553654in}}%
\pgfpathcurveto{\pgfqpoint{1.416388in}{2.545418in}}{\pgfqpoint{1.419660in}{2.537517in}}{\pgfqpoint{1.425484in}{2.531694in}}%
\pgfpathcurveto{\pgfqpoint{1.431308in}{2.525870in}}{\pgfqpoint{1.439208in}{2.522597in}}{\pgfqpoint{1.447444in}{2.522597in}}%
\pgfpathclose%
\pgfusepath{stroke,fill}%
\end{pgfscope}%
\begin{pgfscope}%
\pgfpathrectangle{\pgfqpoint{0.100000in}{0.220728in}}{\pgfqpoint{3.696000in}{3.696000in}}%
\pgfusepath{clip}%
\pgfsetbuttcap%
\pgfsetroundjoin%
\definecolor{currentfill}{rgb}{0.121569,0.466667,0.705882}%
\pgfsetfillcolor{currentfill}%
\pgfsetfillopacity{0.415925}%
\pgfsetlinewidth{1.003750pt}%
\definecolor{currentstroke}{rgb}{0.121569,0.466667,0.705882}%
\pgfsetstrokecolor{currentstroke}%
\pgfsetstrokeopacity{0.415925}%
\pgfsetdash{}{0pt}%
\pgfpathmoveto{\pgfqpoint{2.262169in}{3.186658in}}%
\pgfpathcurveto{\pgfqpoint{2.270406in}{3.186658in}}{\pgfqpoint{2.278306in}{3.189930in}}{\pgfqpoint{2.284130in}{3.195754in}}%
\pgfpathcurveto{\pgfqpoint{2.289954in}{3.201578in}}{\pgfqpoint{2.293226in}{3.209478in}}{\pgfqpoint{2.293226in}{3.217714in}}%
\pgfpathcurveto{\pgfqpoint{2.293226in}{3.225950in}}{\pgfqpoint{2.289954in}{3.233851in}}{\pgfqpoint{2.284130in}{3.239674in}}%
\pgfpathcurveto{\pgfqpoint{2.278306in}{3.245498in}}{\pgfqpoint{2.270406in}{3.248771in}}{\pgfqpoint{2.262169in}{3.248771in}}%
\pgfpathcurveto{\pgfqpoint{2.253933in}{3.248771in}}{\pgfqpoint{2.246033in}{3.245498in}}{\pgfqpoint{2.240209in}{3.239674in}}%
\pgfpathcurveto{\pgfqpoint{2.234385in}{3.233851in}}{\pgfqpoint{2.231113in}{3.225950in}}{\pgfqpoint{2.231113in}{3.217714in}}%
\pgfpathcurveto{\pgfqpoint{2.231113in}{3.209478in}}{\pgfqpoint{2.234385in}{3.201578in}}{\pgfqpoint{2.240209in}{3.195754in}}%
\pgfpathcurveto{\pgfqpoint{2.246033in}{3.189930in}}{\pgfqpoint{2.253933in}{3.186658in}}{\pgfqpoint{2.262169in}{3.186658in}}%
\pgfpathclose%
\pgfusepath{stroke,fill}%
\end{pgfscope}%
\begin{pgfscope}%
\pgfpathrectangle{\pgfqpoint{0.100000in}{0.220728in}}{\pgfqpoint{3.696000in}{3.696000in}}%
\pgfusepath{clip}%
\pgfsetbuttcap%
\pgfsetroundjoin%
\definecolor{currentfill}{rgb}{0.121569,0.466667,0.705882}%
\pgfsetfillcolor{currentfill}%
\pgfsetfillopacity{0.418154}%
\pgfsetlinewidth{1.003750pt}%
\definecolor{currentstroke}{rgb}{0.121569,0.466667,0.705882}%
\pgfsetstrokecolor{currentstroke}%
\pgfsetstrokeopacity{0.418154}%
\pgfsetdash{}{0pt}%
\pgfpathmoveto{\pgfqpoint{2.280608in}{3.182052in}}%
\pgfpathcurveto{\pgfqpoint{2.288844in}{3.182052in}}{\pgfqpoint{2.296744in}{3.185325in}}{\pgfqpoint{2.302568in}{3.191149in}}%
\pgfpathcurveto{\pgfqpoint{2.308392in}{3.196973in}}{\pgfqpoint{2.311664in}{3.204873in}}{\pgfqpoint{2.311664in}{3.213109in}}%
\pgfpathcurveto{\pgfqpoint{2.311664in}{3.221345in}}{\pgfqpoint{2.308392in}{3.229245in}}{\pgfqpoint{2.302568in}{3.235069in}}%
\pgfpathcurveto{\pgfqpoint{2.296744in}{3.240893in}}{\pgfqpoint{2.288844in}{3.244165in}}{\pgfqpoint{2.280608in}{3.244165in}}%
\pgfpathcurveto{\pgfqpoint{2.272372in}{3.244165in}}{\pgfqpoint{2.264472in}{3.240893in}}{\pgfqpoint{2.258648in}{3.235069in}}%
\pgfpathcurveto{\pgfqpoint{2.252824in}{3.229245in}}{\pgfqpoint{2.249551in}{3.221345in}}{\pgfqpoint{2.249551in}{3.213109in}}%
\pgfpathcurveto{\pgfqpoint{2.249551in}{3.204873in}}{\pgfqpoint{2.252824in}{3.196973in}}{\pgfqpoint{2.258648in}{3.191149in}}%
\pgfpathcurveto{\pgfqpoint{2.264472in}{3.185325in}}{\pgfqpoint{2.272372in}{3.182052in}}{\pgfqpoint{2.280608in}{3.182052in}}%
\pgfpathclose%
\pgfusepath{stroke,fill}%
\end{pgfscope}%
\begin{pgfscope}%
\pgfpathrectangle{\pgfqpoint{0.100000in}{0.220728in}}{\pgfqpoint{3.696000in}{3.696000in}}%
\pgfusepath{clip}%
\pgfsetbuttcap%
\pgfsetroundjoin%
\definecolor{currentfill}{rgb}{0.121569,0.466667,0.705882}%
\pgfsetfillcolor{currentfill}%
\pgfsetfillopacity{0.418782}%
\pgfsetlinewidth{1.003750pt}%
\definecolor{currentstroke}{rgb}{0.121569,0.466667,0.705882}%
\pgfsetstrokecolor{currentstroke}%
\pgfsetstrokeopacity{0.418782}%
\pgfsetdash{}{0pt}%
\pgfpathmoveto{\pgfqpoint{1.439858in}{2.505394in}}%
\pgfpathcurveto{\pgfqpoint{1.448094in}{2.505394in}}{\pgfqpoint{1.455994in}{2.508666in}}{\pgfqpoint{1.461818in}{2.514490in}}%
\pgfpathcurveto{\pgfqpoint{1.467642in}{2.520314in}}{\pgfqpoint{1.470914in}{2.528214in}}{\pgfqpoint{1.470914in}{2.536450in}}%
\pgfpathcurveto{\pgfqpoint{1.470914in}{2.544686in}}{\pgfqpoint{1.467642in}{2.552587in}}{\pgfqpoint{1.461818in}{2.558410in}}%
\pgfpathcurveto{\pgfqpoint{1.455994in}{2.564234in}}{\pgfqpoint{1.448094in}{2.567507in}}{\pgfqpoint{1.439858in}{2.567507in}}%
\pgfpathcurveto{\pgfqpoint{1.431622in}{2.567507in}}{\pgfqpoint{1.423722in}{2.564234in}}{\pgfqpoint{1.417898in}{2.558410in}}%
\pgfpathcurveto{\pgfqpoint{1.412074in}{2.552587in}}{\pgfqpoint{1.408801in}{2.544686in}}{\pgfqpoint{1.408801in}{2.536450in}}%
\pgfpathcurveto{\pgfqpoint{1.408801in}{2.528214in}}{\pgfqpoint{1.412074in}{2.520314in}}{\pgfqpoint{1.417898in}{2.514490in}}%
\pgfpathcurveto{\pgfqpoint{1.423722in}{2.508666in}}{\pgfqpoint{1.431622in}{2.505394in}}{\pgfqpoint{1.439858in}{2.505394in}}%
\pgfpathclose%
\pgfusepath{stroke,fill}%
\end{pgfscope}%
\begin{pgfscope}%
\pgfpathrectangle{\pgfqpoint{0.100000in}{0.220728in}}{\pgfqpoint{3.696000in}{3.696000in}}%
\pgfusepath{clip}%
\pgfsetbuttcap%
\pgfsetroundjoin%
\definecolor{currentfill}{rgb}{0.121569,0.466667,0.705882}%
\pgfsetfillcolor{currentfill}%
\pgfsetfillopacity{0.421805}%
\pgfsetlinewidth{1.003750pt}%
\definecolor{currentstroke}{rgb}{0.121569,0.466667,0.705882}%
\pgfsetstrokecolor{currentstroke}%
\pgfsetstrokeopacity{0.421805}%
\pgfsetdash{}{0pt}%
\pgfpathmoveto{\pgfqpoint{1.433786in}{2.488124in}}%
\pgfpathcurveto{\pgfqpoint{1.442022in}{2.488124in}}{\pgfqpoint{1.449922in}{2.491396in}}{\pgfqpoint{1.455746in}{2.497220in}}%
\pgfpathcurveto{\pgfqpoint{1.461570in}{2.503044in}}{\pgfqpoint{1.464843in}{2.510944in}}{\pgfqpoint{1.464843in}{2.519180in}}%
\pgfpathcurveto{\pgfqpoint{1.464843in}{2.527417in}}{\pgfqpoint{1.461570in}{2.535317in}}{\pgfqpoint{1.455746in}{2.541141in}}%
\pgfpathcurveto{\pgfqpoint{1.449922in}{2.546965in}}{\pgfqpoint{1.442022in}{2.550237in}}{\pgfqpoint{1.433786in}{2.550237in}}%
\pgfpathcurveto{\pgfqpoint{1.425550in}{2.550237in}}{\pgfqpoint{1.417650in}{2.546965in}}{\pgfqpoint{1.411826in}{2.541141in}}%
\pgfpathcurveto{\pgfqpoint{1.406002in}{2.535317in}}{\pgfqpoint{1.402730in}{2.527417in}}{\pgfqpoint{1.402730in}{2.519180in}}%
\pgfpathcurveto{\pgfqpoint{1.402730in}{2.510944in}}{\pgfqpoint{1.406002in}{2.503044in}}{\pgfqpoint{1.411826in}{2.497220in}}%
\pgfpathcurveto{\pgfqpoint{1.417650in}{2.491396in}}{\pgfqpoint{1.425550in}{2.488124in}}{\pgfqpoint{1.433786in}{2.488124in}}%
\pgfpathclose%
\pgfusepath{stroke,fill}%
\end{pgfscope}%
\begin{pgfscope}%
\pgfpathrectangle{\pgfqpoint{0.100000in}{0.220728in}}{\pgfqpoint{3.696000in}{3.696000in}}%
\pgfusepath{clip}%
\pgfsetbuttcap%
\pgfsetroundjoin%
\definecolor{currentfill}{rgb}{0.121569,0.466667,0.705882}%
\pgfsetfillcolor{currentfill}%
\pgfsetfillopacity{0.422588}%
\pgfsetlinewidth{1.003750pt}%
\definecolor{currentstroke}{rgb}{0.121569,0.466667,0.705882}%
\pgfsetstrokecolor{currentstroke}%
\pgfsetstrokeopacity{0.422588}%
\pgfsetdash{}{0pt}%
\pgfpathmoveto{\pgfqpoint{2.296318in}{3.174820in}}%
\pgfpathcurveto{\pgfqpoint{2.304554in}{3.174820in}}{\pgfqpoint{2.312454in}{3.178092in}}{\pgfqpoint{2.318278in}{3.183916in}}%
\pgfpathcurveto{\pgfqpoint{2.324102in}{3.189740in}}{\pgfqpoint{2.327374in}{3.197640in}}{\pgfqpoint{2.327374in}{3.205876in}}%
\pgfpathcurveto{\pgfqpoint{2.327374in}{3.214112in}}{\pgfqpoint{2.324102in}{3.222012in}}{\pgfqpoint{2.318278in}{3.227836in}}%
\pgfpathcurveto{\pgfqpoint{2.312454in}{3.233660in}}{\pgfqpoint{2.304554in}{3.236933in}}{\pgfqpoint{2.296318in}{3.236933in}}%
\pgfpathcurveto{\pgfqpoint{2.288082in}{3.236933in}}{\pgfqpoint{2.280182in}{3.233660in}}{\pgfqpoint{2.274358in}{3.227836in}}%
\pgfpathcurveto{\pgfqpoint{2.268534in}{3.222012in}}{\pgfqpoint{2.265261in}{3.214112in}}{\pgfqpoint{2.265261in}{3.205876in}}%
\pgfpathcurveto{\pgfqpoint{2.265261in}{3.197640in}}{\pgfqpoint{2.268534in}{3.189740in}}{\pgfqpoint{2.274358in}{3.183916in}}%
\pgfpathcurveto{\pgfqpoint{2.280182in}{3.178092in}}{\pgfqpoint{2.288082in}{3.174820in}}{\pgfqpoint{2.296318in}{3.174820in}}%
\pgfpathclose%
\pgfusepath{stroke,fill}%
\end{pgfscope}%
\begin{pgfscope}%
\pgfpathrectangle{\pgfqpoint{0.100000in}{0.220728in}}{\pgfqpoint{3.696000in}{3.696000in}}%
\pgfusepath{clip}%
\pgfsetbuttcap%
\pgfsetroundjoin%
\definecolor{currentfill}{rgb}{0.121569,0.466667,0.705882}%
\pgfsetfillcolor{currentfill}%
\pgfsetfillopacity{0.423778}%
\pgfsetlinewidth{1.003750pt}%
\definecolor{currentstroke}{rgb}{0.121569,0.466667,0.705882}%
\pgfsetstrokecolor{currentstroke}%
\pgfsetstrokeopacity{0.423778}%
\pgfsetdash{}{0pt}%
\pgfpathmoveto{\pgfqpoint{1.423890in}{2.474353in}}%
\pgfpathcurveto{\pgfqpoint{1.432126in}{2.474353in}}{\pgfqpoint{1.440026in}{2.477625in}}{\pgfqpoint{1.445850in}{2.483449in}}%
\pgfpathcurveto{\pgfqpoint{1.451674in}{2.489273in}}{\pgfqpoint{1.454947in}{2.497173in}}{\pgfqpoint{1.454947in}{2.505409in}}%
\pgfpathcurveto{\pgfqpoint{1.454947in}{2.513646in}}{\pgfqpoint{1.451674in}{2.521546in}}{\pgfqpoint{1.445850in}{2.527370in}}%
\pgfpathcurveto{\pgfqpoint{1.440026in}{2.533194in}}{\pgfqpoint{1.432126in}{2.536466in}}{\pgfqpoint{1.423890in}{2.536466in}}%
\pgfpathcurveto{\pgfqpoint{1.415654in}{2.536466in}}{\pgfqpoint{1.407754in}{2.533194in}}{\pgfqpoint{1.401930in}{2.527370in}}%
\pgfpathcurveto{\pgfqpoint{1.396106in}{2.521546in}}{\pgfqpoint{1.392834in}{2.513646in}}{\pgfqpoint{1.392834in}{2.505409in}}%
\pgfpathcurveto{\pgfqpoint{1.392834in}{2.497173in}}{\pgfqpoint{1.396106in}{2.489273in}}{\pgfqpoint{1.401930in}{2.483449in}}%
\pgfpathcurveto{\pgfqpoint{1.407754in}{2.477625in}}{\pgfqpoint{1.415654in}{2.474353in}}{\pgfqpoint{1.423890in}{2.474353in}}%
\pgfpathclose%
\pgfusepath{stroke,fill}%
\end{pgfscope}%
\begin{pgfscope}%
\pgfpathrectangle{\pgfqpoint{0.100000in}{0.220728in}}{\pgfqpoint{3.696000in}{3.696000in}}%
\pgfusepath{clip}%
\pgfsetbuttcap%
\pgfsetroundjoin%
\definecolor{currentfill}{rgb}{0.121569,0.466667,0.705882}%
\pgfsetfillcolor{currentfill}%
\pgfsetfillopacity{0.424030}%
\pgfsetlinewidth{1.003750pt}%
\definecolor{currentstroke}{rgb}{0.121569,0.466667,0.705882}%
\pgfsetstrokecolor{currentstroke}%
\pgfsetstrokeopacity{0.424030}%
\pgfsetdash{}{0pt}%
\pgfpathmoveto{\pgfqpoint{2.311928in}{3.169467in}}%
\pgfpathcurveto{\pgfqpoint{2.320165in}{3.169467in}}{\pgfqpoint{2.328065in}{3.172739in}}{\pgfqpoint{2.333889in}{3.178563in}}%
\pgfpathcurveto{\pgfqpoint{2.339713in}{3.184387in}}{\pgfqpoint{2.342985in}{3.192287in}}{\pgfqpoint{2.342985in}{3.200523in}}%
\pgfpathcurveto{\pgfqpoint{2.342985in}{3.208759in}}{\pgfqpoint{2.339713in}{3.216660in}}{\pgfqpoint{2.333889in}{3.222483in}}%
\pgfpathcurveto{\pgfqpoint{2.328065in}{3.228307in}}{\pgfqpoint{2.320165in}{3.231580in}}{\pgfqpoint{2.311928in}{3.231580in}}%
\pgfpathcurveto{\pgfqpoint{2.303692in}{3.231580in}}{\pgfqpoint{2.295792in}{3.228307in}}{\pgfqpoint{2.289968in}{3.222483in}}%
\pgfpathcurveto{\pgfqpoint{2.284144in}{3.216660in}}{\pgfqpoint{2.280872in}{3.208759in}}{\pgfqpoint{2.280872in}{3.200523in}}%
\pgfpathcurveto{\pgfqpoint{2.280872in}{3.192287in}}{\pgfqpoint{2.284144in}{3.184387in}}{\pgfqpoint{2.289968in}{3.178563in}}%
\pgfpathcurveto{\pgfqpoint{2.295792in}{3.172739in}}{\pgfqpoint{2.303692in}{3.169467in}}{\pgfqpoint{2.311928in}{3.169467in}}%
\pgfpathclose%
\pgfusepath{stroke,fill}%
\end{pgfscope}%
\begin{pgfscope}%
\pgfpathrectangle{\pgfqpoint{0.100000in}{0.220728in}}{\pgfqpoint{3.696000in}{3.696000in}}%
\pgfusepath{clip}%
\pgfsetbuttcap%
\pgfsetroundjoin%
\definecolor{currentfill}{rgb}{0.121569,0.466667,0.705882}%
\pgfsetfillcolor{currentfill}%
\pgfsetfillopacity{0.424666}%
\pgfsetlinewidth{1.003750pt}%
\definecolor{currentstroke}{rgb}{0.121569,0.466667,0.705882}%
\pgfsetstrokecolor{currentstroke}%
\pgfsetstrokeopacity{0.424666}%
\pgfsetdash{}{0pt}%
\pgfpathmoveto{\pgfqpoint{2.305628in}{3.171097in}}%
\pgfpathcurveto{\pgfqpoint{2.313864in}{3.171097in}}{\pgfqpoint{2.321764in}{3.174369in}}{\pgfqpoint{2.327588in}{3.180193in}}%
\pgfpathcurveto{\pgfqpoint{2.333412in}{3.186017in}}{\pgfqpoint{2.336684in}{3.193917in}}{\pgfqpoint{2.336684in}{3.202153in}}%
\pgfpathcurveto{\pgfqpoint{2.336684in}{3.210390in}}{\pgfqpoint{2.333412in}{3.218290in}}{\pgfqpoint{2.327588in}{3.224114in}}%
\pgfpathcurveto{\pgfqpoint{2.321764in}{3.229937in}}{\pgfqpoint{2.313864in}{3.233210in}}{\pgfqpoint{2.305628in}{3.233210in}}%
\pgfpathcurveto{\pgfqpoint{2.297392in}{3.233210in}}{\pgfqpoint{2.289492in}{3.229937in}}{\pgfqpoint{2.283668in}{3.224114in}}%
\pgfpathcurveto{\pgfqpoint{2.277844in}{3.218290in}}{\pgfqpoint{2.274571in}{3.210390in}}{\pgfqpoint{2.274571in}{3.202153in}}%
\pgfpathcurveto{\pgfqpoint{2.274571in}{3.193917in}}{\pgfqpoint{2.277844in}{3.186017in}}{\pgfqpoint{2.283668in}{3.180193in}}%
\pgfpathcurveto{\pgfqpoint{2.289492in}{3.174369in}}{\pgfqpoint{2.297392in}{3.171097in}}{\pgfqpoint{2.305628in}{3.171097in}}%
\pgfpathclose%
\pgfusepath{stroke,fill}%
\end{pgfscope}%
\begin{pgfscope}%
\pgfpathrectangle{\pgfqpoint{0.100000in}{0.220728in}}{\pgfqpoint{3.696000in}{3.696000in}}%
\pgfusepath{clip}%
\pgfsetbuttcap%
\pgfsetroundjoin%
\definecolor{currentfill}{rgb}{0.121569,0.466667,0.705882}%
\pgfsetfillcolor{currentfill}%
\pgfsetfillopacity{0.425745}%
\pgfsetlinewidth{1.003750pt}%
\definecolor{currentstroke}{rgb}{0.121569,0.466667,0.705882}%
\pgfsetstrokecolor{currentstroke}%
\pgfsetstrokeopacity{0.425745}%
\pgfsetdash{}{0pt}%
\pgfpathmoveto{\pgfqpoint{2.317784in}{3.168443in}}%
\pgfpathcurveto{\pgfqpoint{2.326020in}{3.168443in}}{\pgfqpoint{2.333920in}{3.171715in}}{\pgfqpoint{2.339744in}{3.177539in}}%
\pgfpathcurveto{\pgfqpoint{2.345568in}{3.183363in}}{\pgfqpoint{2.348840in}{3.191263in}}{\pgfqpoint{2.348840in}{3.199499in}}%
\pgfpathcurveto{\pgfqpoint{2.348840in}{3.207735in}}{\pgfqpoint{2.345568in}{3.215635in}}{\pgfqpoint{2.339744in}{3.221459in}}%
\pgfpathcurveto{\pgfqpoint{2.333920in}{3.227283in}}{\pgfqpoint{2.326020in}{3.230555in}}{\pgfqpoint{2.317784in}{3.230555in}}%
\pgfpathcurveto{\pgfqpoint{2.309548in}{3.230555in}}{\pgfqpoint{2.301648in}{3.227283in}}{\pgfqpoint{2.295824in}{3.221459in}}%
\pgfpathcurveto{\pgfqpoint{2.290000in}{3.215635in}}{\pgfqpoint{2.286727in}{3.207735in}}{\pgfqpoint{2.286727in}{3.199499in}}%
\pgfpathcurveto{\pgfqpoint{2.286727in}{3.191263in}}{\pgfqpoint{2.290000in}{3.183363in}}{\pgfqpoint{2.295824in}{3.177539in}}%
\pgfpathcurveto{\pgfqpoint{2.301648in}{3.171715in}}{\pgfqpoint{2.309548in}{3.168443in}}{\pgfqpoint{2.317784in}{3.168443in}}%
\pgfpathclose%
\pgfusepath{stroke,fill}%
\end{pgfscope}%
\begin{pgfscope}%
\pgfpathrectangle{\pgfqpoint{0.100000in}{0.220728in}}{\pgfqpoint{3.696000in}{3.696000in}}%
\pgfusepath{clip}%
\pgfsetbuttcap%
\pgfsetroundjoin%
\definecolor{currentfill}{rgb}{0.121569,0.466667,0.705882}%
\pgfsetfillcolor{currentfill}%
\pgfsetfillopacity{0.426502}%
\pgfsetlinewidth{1.003750pt}%
\definecolor{currentstroke}{rgb}{0.121569,0.466667,0.705882}%
\pgfsetstrokecolor{currentstroke}%
\pgfsetstrokeopacity{0.426502}%
\pgfsetdash{}{0pt}%
\pgfpathmoveto{\pgfqpoint{1.421638in}{2.458326in}}%
\pgfpathcurveto{\pgfqpoint{1.429874in}{2.458326in}}{\pgfqpoint{1.437774in}{2.461598in}}{\pgfqpoint{1.443598in}{2.467422in}}%
\pgfpathcurveto{\pgfqpoint{1.449422in}{2.473246in}}{\pgfqpoint{1.452694in}{2.481146in}}{\pgfqpoint{1.452694in}{2.489383in}}%
\pgfpathcurveto{\pgfqpoint{1.452694in}{2.497619in}}{\pgfqpoint{1.449422in}{2.505519in}}{\pgfqpoint{1.443598in}{2.511343in}}%
\pgfpathcurveto{\pgfqpoint{1.437774in}{2.517167in}}{\pgfqpoint{1.429874in}{2.520439in}}{\pgfqpoint{1.421638in}{2.520439in}}%
\pgfpathcurveto{\pgfqpoint{1.413401in}{2.520439in}}{\pgfqpoint{1.405501in}{2.517167in}}{\pgfqpoint{1.399677in}{2.511343in}}%
\pgfpathcurveto{\pgfqpoint{1.393853in}{2.505519in}}{\pgfqpoint{1.390581in}{2.497619in}}{\pgfqpoint{1.390581in}{2.489383in}}%
\pgfpathcurveto{\pgfqpoint{1.390581in}{2.481146in}}{\pgfqpoint{1.393853in}{2.473246in}}{\pgfqpoint{1.399677in}{2.467422in}}%
\pgfpathcurveto{\pgfqpoint{1.405501in}{2.461598in}}{\pgfqpoint{1.413401in}{2.458326in}}{\pgfqpoint{1.421638in}{2.458326in}}%
\pgfpathclose%
\pgfusepath{stroke,fill}%
\end{pgfscope}%
\begin{pgfscope}%
\pgfpathrectangle{\pgfqpoint{0.100000in}{0.220728in}}{\pgfqpoint{3.696000in}{3.696000in}}%
\pgfusepath{clip}%
\pgfsetbuttcap%
\pgfsetroundjoin%
\definecolor{currentfill}{rgb}{0.121569,0.466667,0.705882}%
\pgfsetfillcolor{currentfill}%
\pgfsetfillopacity{0.427108}%
\pgfsetlinewidth{1.003750pt}%
\definecolor{currentstroke}{rgb}{0.121569,0.466667,0.705882}%
\pgfsetstrokecolor{currentstroke}%
\pgfsetstrokeopacity{0.427108}%
\pgfsetdash{}{0pt}%
\pgfpathmoveto{\pgfqpoint{2.325079in}{3.166234in}}%
\pgfpathcurveto{\pgfqpoint{2.333315in}{3.166234in}}{\pgfqpoint{2.341215in}{3.169506in}}{\pgfqpoint{2.347039in}{3.175330in}}%
\pgfpathcurveto{\pgfqpoint{2.352863in}{3.181154in}}{\pgfqpoint{2.356135in}{3.189054in}}{\pgfqpoint{2.356135in}{3.197290in}}%
\pgfpathcurveto{\pgfqpoint{2.356135in}{3.205526in}}{\pgfqpoint{2.352863in}{3.213426in}}{\pgfqpoint{2.347039in}{3.219250in}}%
\pgfpathcurveto{\pgfqpoint{2.341215in}{3.225074in}}{\pgfqpoint{2.333315in}{3.228347in}}{\pgfqpoint{2.325079in}{3.228347in}}%
\pgfpathcurveto{\pgfqpoint{2.316842in}{3.228347in}}{\pgfqpoint{2.308942in}{3.225074in}}{\pgfqpoint{2.303118in}{3.219250in}}%
\pgfpathcurveto{\pgfqpoint{2.297295in}{3.213426in}}{\pgfqpoint{2.294022in}{3.205526in}}{\pgfqpoint{2.294022in}{3.197290in}}%
\pgfpathcurveto{\pgfqpoint{2.294022in}{3.189054in}}{\pgfqpoint{2.297295in}{3.181154in}}{\pgfqpoint{2.303118in}{3.175330in}}%
\pgfpathcurveto{\pgfqpoint{2.308942in}{3.169506in}}{\pgfqpoint{2.316842in}{3.166234in}}{\pgfqpoint{2.325079in}{3.166234in}}%
\pgfpathclose%
\pgfusepath{stroke,fill}%
\end{pgfscope}%
\begin{pgfscope}%
\pgfpathrectangle{\pgfqpoint{0.100000in}{0.220728in}}{\pgfqpoint{3.696000in}{3.696000in}}%
\pgfusepath{clip}%
\pgfsetbuttcap%
\pgfsetroundjoin%
\definecolor{currentfill}{rgb}{0.121569,0.466667,0.705882}%
\pgfsetfillcolor{currentfill}%
\pgfsetfillopacity{0.427698}%
\pgfsetlinewidth{1.003750pt}%
\definecolor{currentstroke}{rgb}{0.121569,0.466667,0.705882}%
\pgfsetstrokecolor{currentstroke}%
\pgfsetstrokeopacity{0.427698}%
\pgfsetdash{}{0pt}%
\pgfpathmoveto{\pgfqpoint{1.413759in}{2.448061in}}%
\pgfpathcurveto{\pgfqpoint{1.421995in}{2.448061in}}{\pgfqpoint{1.429896in}{2.451334in}}{\pgfqpoint{1.435719in}{2.457158in}}%
\pgfpathcurveto{\pgfqpoint{1.441543in}{2.462982in}}{\pgfqpoint{1.444816in}{2.470882in}}{\pgfqpoint{1.444816in}{2.479118in}}%
\pgfpathcurveto{\pgfqpoint{1.444816in}{2.487354in}}{\pgfqpoint{1.441543in}{2.495254in}}{\pgfqpoint{1.435719in}{2.501078in}}%
\pgfpathcurveto{\pgfqpoint{1.429896in}{2.506902in}}{\pgfqpoint{1.421995in}{2.510174in}}{\pgfqpoint{1.413759in}{2.510174in}}%
\pgfpathcurveto{\pgfqpoint{1.405523in}{2.510174in}}{\pgfqpoint{1.397623in}{2.506902in}}{\pgfqpoint{1.391799in}{2.501078in}}%
\pgfpathcurveto{\pgfqpoint{1.385975in}{2.495254in}}{\pgfqpoint{1.382703in}{2.487354in}}{\pgfqpoint{1.382703in}{2.479118in}}%
\pgfpathcurveto{\pgfqpoint{1.382703in}{2.470882in}}{\pgfqpoint{1.385975in}{2.462982in}}{\pgfqpoint{1.391799in}{2.457158in}}%
\pgfpathcurveto{\pgfqpoint{1.397623in}{2.451334in}}{\pgfqpoint{1.405523in}{2.448061in}}{\pgfqpoint{1.413759in}{2.448061in}}%
\pgfpathclose%
\pgfusepath{stroke,fill}%
\end{pgfscope}%
\begin{pgfscope}%
\pgfpathrectangle{\pgfqpoint{0.100000in}{0.220728in}}{\pgfqpoint{3.696000in}{3.696000in}}%
\pgfusepath{clip}%
\pgfsetbuttcap%
\pgfsetroundjoin%
\definecolor{currentfill}{rgb}{0.121569,0.466667,0.705882}%
\pgfsetfillcolor{currentfill}%
\pgfsetfillopacity{0.427859}%
\pgfsetlinewidth{1.003750pt}%
\definecolor{currentstroke}{rgb}{0.121569,0.466667,0.705882}%
\pgfsetstrokecolor{currentstroke}%
\pgfsetstrokeopacity{0.427859}%
\pgfsetdash{}{0pt}%
\pgfpathmoveto{\pgfqpoint{2.328996in}{3.164695in}}%
\pgfpathcurveto{\pgfqpoint{2.337232in}{3.164695in}}{\pgfqpoint{2.345133in}{3.167967in}}{\pgfqpoint{2.350956in}{3.173791in}}%
\pgfpathcurveto{\pgfqpoint{2.356780in}{3.179615in}}{\pgfqpoint{2.360053in}{3.187515in}}{\pgfqpoint{2.360053in}{3.195751in}}%
\pgfpathcurveto{\pgfqpoint{2.360053in}{3.203988in}}{\pgfqpoint{2.356780in}{3.211888in}}{\pgfqpoint{2.350956in}{3.217712in}}%
\pgfpathcurveto{\pgfqpoint{2.345133in}{3.223536in}}{\pgfqpoint{2.337232in}{3.226808in}}{\pgfqpoint{2.328996in}{3.226808in}}%
\pgfpathcurveto{\pgfqpoint{2.320760in}{3.226808in}}{\pgfqpoint{2.312860in}{3.223536in}}{\pgfqpoint{2.307036in}{3.217712in}}%
\pgfpathcurveto{\pgfqpoint{2.301212in}{3.211888in}}{\pgfqpoint{2.297940in}{3.203988in}}{\pgfqpoint{2.297940in}{3.195751in}}%
\pgfpathcurveto{\pgfqpoint{2.297940in}{3.187515in}}{\pgfqpoint{2.301212in}{3.179615in}}{\pgfqpoint{2.307036in}{3.173791in}}%
\pgfpathcurveto{\pgfqpoint{2.312860in}{3.167967in}}{\pgfqpoint{2.320760in}{3.164695in}}{\pgfqpoint{2.328996in}{3.164695in}}%
\pgfpathclose%
\pgfusepath{stroke,fill}%
\end{pgfscope}%
\begin{pgfscope}%
\pgfpathrectangle{\pgfqpoint{0.100000in}{0.220728in}}{\pgfqpoint{3.696000in}{3.696000in}}%
\pgfusepath{clip}%
\pgfsetbuttcap%
\pgfsetroundjoin%
\definecolor{currentfill}{rgb}{0.121569,0.466667,0.705882}%
\pgfsetfillcolor{currentfill}%
\pgfsetfillopacity{0.428314}%
\pgfsetlinewidth{1.003750pt}%
\definecolor{currentstroke}{rgb}{0.121569,0.466667,0.705882}%
\pgfsetstrokecolor{currentstroke}%
\pgfsetstrokeopacity{0.428314}%
\pgfsetdash{}{0pt}%
\pgfpathmoveto{\pgfqpoint{2.331213in}{3.164240in}}%
\pgfpathcurveto{\pgfqpoint{2.339449in}{3.164240in}}{\pgfqpoint{2.347349in}{3.167512in}}{\pgfqpoint{2.353173in}{3.173336in}}%
\pgfpathcurveto{\pgfqpoint{2.358997in}{3.179160in}}{\pgfqpoint{2.362269in}{3.187060in}}{\pgfqpoint{2.362269in}{3.195296in}}%
\pgfpathcurveto{\pgfqpoint{2.362269in}{3.203533in}}{\pgfqpoint{2.358997in}{3.211433in}}{\pgfqpoint{2.353173in}{3.217257in}}%
\pgfpathcurveto{\pgfqpoint{2.347349in}{3.223080in}}{\pgfqpoint{2.339449in}{3.226353in}}{\pgfqpoint{2.331213in}{3.226353in}}%
\pgfpathcurveto{\pgfqpoint{2.322977in}{3.226353in}}{\pgfqpoint{2.315077in}{3.223080in}}{\pgfqpoint{2.309253in}{3.217257in}}%
\pgfpathcurveto{\pgfqpoint{2.303429in}{3.211433in}}{\pgfqpoint{2.300156in}{3.203533in}}{\pgfqpoint{2.300156in}{3.195296in}}%
\pgfpathcurveto{\pgfqpoint{2.300156in}{3.187060in}}{\pgfqpoint{2.303429in}{3.179160in}}{\pgfqpoint{2.309253in}{3.173336in}}%
\pgfpathcurveto{\pgfqpoint{2.315077in}{3.167512in}}{\pgfqpoint{2.322977in}{3.164240in}}{\pgfqpoint{2.331213in}{3.164240in}}%
\pgfpathclose%
\pgfusepath{stroke,fill}%
\end{pgfscope}%
\begin{pgfscope}%
\pgfpathrectangle{\pgfqpoint{0.100000in}{0.220728in}}{\pgfqpoint{3.696000in}{3.696000in}}%
\pgfusepath{clip}%
\pgfsetbuttcap%
\pgfsetroundjoin%
\definecolor{currentfill}{rgb}{0.121569,0.466667,0.705882}%
\pgfsetfillcolor{currentfill}%
\pgfsetfillopacity{0.428696}%
\pgfsetlinewidth{1.003750pt}%
\definecolor{currentstroke}{rgb}{0.121569,0.466667,0.705882}%
\pgfsetstrokecolor{currentstroke}%
\pgfsetstrokeopacity{0.428696}%
\pgfsetdash{}{0pt}%
\pgfpathmoveto{\pgfqpoint{2.334163in}{3.162714in}}%
\pgfpathcurveto{\pgfqpoint{2.342399in}{3.162714in}}{\pgfqpoint{2.350299in}{3.165986in}}{\pgfqpoint{2.356123in}{3.171810in}}%
\pgfpathcurveto{\pgfqpoint{2.361947in}{3.177634in}}{\pgfqpoint{2.365220in}{3.185534in}}{\pgfqpoint{2.365220in}{3.193770in}}%
\pgfpathcurveto{\pgfqpoint{2.365220in}{3.202006in}}{\pgfqpoint{2.361947in}{3.209906in}}{\pgfqpoint{2.356123in}{3.215730in}}%
\pgfpathcurveto{\pgfqpoint{2.350299in}{3.221554in}}{\pgfqpoint{2.342399in}{3.224827in}}{\pgfqpoint{2.334163in}{3.224827in}}%
\pgfpathcurveto{\pgfqpoint{2.325927in}{3.224827in}}{\pgfqpoint{2.318027in}{3.221554in}}{\pgfqpoint{2.312203in}{3.215730in}}%
\pgfpathcurveto{\pgfqpoint{2.306379in}{3.209906in}}{\pgfqpoint{2.303107in}{3.202006in}}{\pgfqpoint{2.303107in}{3.193770in}}%
\pgfpathcurveto{\pgfqpoint{2.303107in}{3.185534in}}{\pgfqpoint{2.306379in}{3.177634in}}{\pgfqpoint{2.312203in}{3.171810in}}%
\pgfpathcurveto{\pgfqpoint{2.318027in}{3.165986in}}{\pgfqpoint{2.325927in}{3.162714in}}{\pgfqpoint{2.334163in}{3.162714in}}%
\pgfpathclose%
\pgfusepath{stroke,fill}%
\end{pgfscope}%
\begin{pgfscope}%
\pgfpathrectangle{\pgfqpoint{0.100000in}{0.220728in}}{\pgfqpoint{3.696000in}{3.696000in}}%
\pgfusepath{clip}%
\pgfsetbuttcap%
\pgfsetroundjoin%
\definecolor{currentfill}{rgb}{0.121569,0.466667,0.705882}%
\pgfsetfillcolor{currentfill}%
\pgfsetfillopacity{0.429170}%
\pgfsetlinewidth{1.003750pt}%
\definecolor{currentstroke}{rgb}{0.121569,0.466667,0.705882}%
\pgfsetstrokecolor{currentstroke}%
\pgfsetstrokeopacity{0.429170}%
\pgfsetdash{}{0pt}%
\pgfpathmoveto{\pgfqpoint{2.335601in}{3.162427in}}%
\pgfpathcurveto{\pgfqpoint{2.343837in}{3.162427in}}{\pgfqpoint{2.351737in}{3.165699in}}{\pgfqpoint{2.357561in}{3.171523in}}%
\pgfpathcurveto{\pgfqpoint{2.363385in}{3.177347in}}{\pgfqpoint{2.366657in}{3.185247in}}{\pgfqpoint{2.366657in}{3.193483in}}%
\pgfpathcurveto{\pgfqpoint{2.366657in}{3.201719in}}{\pgfqpoint{2.363385in}{3.209620in}}{\pgfqpoint{2.357561in}{3.215443in}}%
\pgfpathcurveto{\pgfqpoint{2.351737in}{3.221267in}}{\pgfqpoint{2.343837in}{3.224540in}}{\pgfqpoint{2.335601in}{3.224540in}}%
\pgfpathcurveto{\pgfqpoint{2.327364in}{3.224540in}}{\pgfqpoint{2.319464in}{3.221267in}}{\pgfqpoint{2.313640in}{3.215443in}}%
\pgfpathcurveto{\pgfqpoint{2.307816in}{3.209620in}}{\pgfqpoint{2.304544in}{3.201719in}}{\pgfqpoint{2.304544in}{3.193483in}}%
\pgfpathcurveto{\pgfqpoint{2.304544in}{3.185247in}}{\pgfqpoint{2.307816in}{3.177347in}}{\pgfqpoint{2.313640in}{3.171523in}}%
\pgfpathcurveto{\pgfqpoint{2.319464in}{3.165699in}}{\pgfqpoint{2.327364in}{3.162427in}}{\pgfqpoint{2.335601in}{3.162427in}}%
\pgfpathclose%
\pgfusepath{stroke,fill}%
\end{pgfscope}%
\begin{pgfscope}%
\pgfpathrectangle{\pgfqpoint{0.100000in}{0.220728in}}{\pgfqpoint{3.696000in}{3.696000in}}%
\pgfusepath{clip}%
\pgfsetbuttcap%
\pgfsetroundjoin%
\definecolor{currentfill}{rgb}{0.121569,0.466667,0.705882}%
\pgfsetfillcolor{currentfill}%
\pgfsetfillopacity{0.429461}%
\pgfsetlinewidth{1.003750pt}%
\definecolor{currentstroke}{rgb}{0.121569,0.466667,0.705882}%
\pgfsetstrokecolor{currentstroke}%
\pgfsetstrokeopacity{0.429461}%
\pgfsetdash{}{0pt}%
\pgfpathmoveto{\pgfqpoint{1.411287in}{2.437893in}}%
\pgfpathcurveto{\pgfqpoint{1.419524in}{2.437893in}}{\pgfqpoint{1.427424in}{2.441165in}}{\pgfqpoint{1.433248in}{2.446989in}}%
\pgfpathcurveto{\pgfqpoint{1.439072in}{2.452813in}}{\pgfqpoint{1.442344in}{2.460713in}}{\pgfqpoint{1.442344in}{2.468949in}}%
\pgfpathcurveto{\pgfqpoint{1.442344in}{2.477185in}}{\pgfqpoint{1.439072in}{2.485086in}}{\pgfqpoint{1.433248in}{2.490909in}}%
\pgfpathcurveto{\pgfqpoint{1.427424in}{2.496733in}}{\pgfqpoint{1.419524in}{2.500006in}}{\pgfqpoint{1.411287in}{2.500006in}}%
\pgfpathcurveto{\pgfqpoint{1.403051in}{2.500006in}}{\pgfqpoint{1.395151in}{2.496733in}}{\pgfqpoint{1.389327in}{2.490909in}}%
\pgfpathcurveto{\pgfqpoint{1.383503in}{2.485086in}}{\pgfqpoint{1.380231in}{2.477185in}}{\pgfqpoint{1.380231in}{2.468949in}}%
\pgfpathcurveto{\pgfqpoint{1.380231in}{2.460713in}}{\pgfqpoint{1.383503in}{2.452813in}}{\pgfqpoint{1.389327in}{2.446989in}}%
\pgfpathcurveto{\pgfqpoint{1.395151in}{2.441165in}}{\pgfqpoint{1.403051in}{2.437893in}}{\pgfqpoint{1.411287in}{2.437893in}}%
\pgfpathclose%
\pgfusepath{stroke,fill}%
\end{pgfscope}%
\begin{pgfscope}%
\pgfpathrectangle{\pgfqpoint{0.100000in}{0.220728in}}{\pgfqpoint{3.696000in}{3.696000in}}%
\pgfusepath{clip}%
\pgfsetbuttcap%
\pgfsetroundjoin%
\definecolor{currentfill}{rgb}{0.121569,0.466667,0.705882}%
\pgfsetfillcolor{currentfill}%
\pgfsetfillopacity{0.429884}%
\pgfsetlinewidth{1.003750pt}%
\definecolor{currentstroke}{rgb}{0.121569,0.466667,0.705882}%
\pgfsetstrokecolor{currentstroke}%
\pgfsetstrokeopacity{0.429884}%
\pgfsetdash{}{0pt}%
\pgfpathmoveto{\pgfqpoint{2.339520in}{3.161151in}}%
\pgfpathcurveto{\pgfqpoint{2.347756in}{3.161151in}}{\pgfqpoint{2.355656in}{3.164424in}}{\pgfqpoint{2.361480in}{3.170248in}}%
\pgfpathcurveto{\pgfqpoint{2.367304in}{3.176072in}}{\pgfqpoint{2.370577in}{3.183972in}}{\pgfqpoint{2.370577in}{3.192208in}}%
\pgfpathcurveto{\pgfqpoint{2.370577in}{3.200444in}}{\pgfqpoint{2.367304in}{3.208344in}}{\pgfqpoint{2.361480in}{3.214168in}}%
\pgfpathcurveto{\pgfqpoint{2.355656in}{3.219992in}}{\pgfqpoint{2.347756in}{3.223264in}}{\pgfqpoint{2.339520in}{3.223264in}}%
\pgfpathcurveto{\pgfqpoint{2.331284in}{3.223264in}}{\pgfqpoint{2.323384in}{3.219992in}}{\pgfqpoint{2.317560in}{3.214168in}}%
\pgfpathcurveto{\pgfqpoint{2.311736in}{3.208344in}}{\pgfqpoint{2.308464in}{3.200444in}}{\pgfqpoint{2.308464in}{3.192208in}}%
\pgfpathcurveto{\pgfqpoint{2.308464in}{3.183972in}}{\pgfqpoint{2.311736in}{3.176072in}}{\pgfqpoint{2.317560in}{3.170248in}}%
\pgfpathcurveto{\pgfqpoint{2.323384in}{3.164424in}}{\pgfqpoint{2.331284in}{3.161151in}}{\pgfqpoint{2.339520in}{3.161151in}}%
\pgfpathclose%
\pgfusepath{stroke,fill}%
\end{pgfscope}%
\begin{pgfscope}%
\pgfpathrectangle{\pgfqpoint{0.100000in}{0.220728in}}{\pgfqpoint{3.696000in}{3.696000in}}%
\pgfusepath{clip}%
\pgfsetbuttcap%
\pgfsetroundjoin%
\definecolor{currentfill}{rgb}{0.121569,0.466667,0.705882}%
\pgfsetfillcolor{currentfill}%
\pgfsetfillopacity{0.430340}%
\pgfsetlinewidth{1.003750pt}%
\definecolor{currentstroke}{rgb}{0.121569,0.466667,0.705882}%
\pgfsetstrokecolor{currentstroke}%
\pgfsetstrokeopacity{0.430340}%
\pgfsetdash{}{0pt}%
\pgfpathmoveto{\pgfqpoint{1.406301in}{2.430940in}}%
\pgfpathcurveto{\pgfqpoint{1.414538in}{2.430940in}}{\pgfqpoint{1.422438in}{2.434212in}}{\pgfqpoint{1.428262in}{2.440036in}}%
\pgfpathcurveto{\pgfqpoint{1.434086in}{2.445860in}}{\pgfqpoint{1.437358in}{2.453760in}}{\pgfqpoint{1.437358in}{2.461996in}}%
\pgfpathcurveto{\pgfqpoint{1.437358in}{2.470232in}}{\pgfqpoint{1.434086in}{2.478132in}}{\pgfqpoint{1.428262in}{2.483956in}}%
\pgfpathcurveto{\pgfqpoint{1.422438in}{2.489780in}}{\pgfqpoint{1.414538in}{2.493053in}}{\pgfqpoint{1.406301in}{2.493053in}}%
\pgfpathcurveto{\pgfqpoint{1.398065in}{2.493053in}}{\pgfqpoint{1.390165in}{2.489780in}}{\pgfqpoint{1.384341in}{2.483956in}}%
\pgfpathcurveto{\pgfqpoint{1.378517in}{2.478132in}}{\pgfqpoint{1.375245in}{2.470232in}}{\pgfqpoint{1.375245in}{2.461996in}}%
\pgfpathcurveto{\pgfqpoint{1.375245in}{2.453760in}}{\pgfqpoint{1.378517in}{2.445860in}}{\pgfqpoint{1.384341in}{2.440036in}}%
\pgfpathcurveto{\pgfqpoint{1.390165in}{2.434212in}}{\pgfqpoint{1.398065in}{2.430940in}}{\pgfqpoint{1.406301in}{2.430940in}}%
\pgfpathclose%
\pgfusepath{stroke,fill}%
\end{pgfscope}%
\begin{pgfscope}%
\pgfpathrectangle{\pgfqpoint{0.100000in}{0.220728in}}{\pgfqpoint{3.696000in}{3.696000in}}%
\pgfusepath{clip}%
\pgfsetbuttcap%
\pgfsetroundjoin%
\definecolor{currentfill}{rgb}{0.121569,0.466667,0.705882}%
\pgfsetfillcolor{currentfill}%
\pgfsetfillopacity{0.430430}%
\pgfsetlinewidth{1.003750pt}%
\definecolor{currentstroke}{rgb}{0.121569,0.466667,0.705882}%
\pgfsetstrokecolor{currentstroke}%
\pgfsetstrokeopacity{0.430430}%
\pgfsetdash{}{0pt}%
\pgfpathmoveto{\pgfqpoint{2.341463in}{3.160447in}}%
\pgfpathcurveto{\pgfqpoint{2.349700in}{3.160447in}}{\pgfqpoint{2.357600in}{3.163720in}}{\pgfqpoint{2.363424in}{3.169544in}}%
\pgfpathcurveto{\pgfqpoint{2.369247in}{3.175368in}}{\pgfqpoint{2.372520in}{3.183268in}}{\pgfqpoint{2.372520in}{3.191504in}}%
\pgfpathcurveto{\pgfqpoint{2.372520in}{3.199740in}}{\pgfqpoint{2.369247in}{3.207640in}}{\pgfqpoint{2.363424in}{3.213464in}}%
\pgfpathcurveto{\pgfqpoint{2.357600in}{3.219288in}}{\pgfqpoint{2.349700in}{3.222560in}}{\pgfqpoint{2.341463in}{3.222560in}}%
\pgfpathcurveto{\pgfqpoint{2.333227in}{3.222560in}}{\pgfqpoint{2.325327in}{3.219288in}}{\pgfqpoint{2.319503in}{3.213464in}}%
\pgfpathcurveto{\pgfqpoint{2.313679in}{3.207640in}}{\pgfqpoint{2.310407in}{3.199740in}}{\pgfqpoint{2.310407in}{3.191504in}}%
\pgfpathcurveto{\pgfqpoint{2.310407in}{3.183268in}}{\pgfqpoint{2.313679in}{3.175368in}}{\pgfqpoint{2.319503in}{3.169544in}}%
\pgfpathcurveto{\pgfqpoint{2.325327in}{3.163720in}}{\pgfqpoint{2.333227in}{3.160447in}}{\pgfqpoint{2.341463in}{3.160447in}}%
\pgfpathclose%
\pgfusepath{stroke,fill}%
\end{pgfscope}%
\begin{pgfscope}%
\pgfpathrectangle{\pgfqpoint{0.100000in}{0.220728in}}{\pgfqpoint{3.696000in}{3.696000in}}%
\pgfusepath{clip}%
\pgfsetbuttcap%
\pgfsetroundjoin%
\definecolor{currentfill}{rgb}{0.121569,0.466667,0.705882}%
\pgfsetfillcolor{currentfill}%
\pgfsetfillopacity{0.430725}%
\pgfsetlinewidth{1.003750pt}%
\definecolor{currentstroke}{rgb}{0.121569,0.466667,0.705882}%
\pgfsetstrokecolor{currentstroke}%
\pgfsetstrokeopacity{0.430725}%
\pgfsetdash{}{0pt}%
\pgfpathmoveto{\pgfqpoint{2.342592in}{3.160210in}}%
\pgfpathcurveto{\pgfqpoint{2.350828in}{3.160210in}}{\pgfqpoint{2.358728in}{3.163483in}}{\pgfqpoint{2.364552in}{3.169307in}}%
\pgfpathcurveto{\pgfqpoint{2.370376in}{3.175130in}}{\pgfqpoint{2.373649in}{3.183031in}}{\pgfqpoint{2.373649in}{3.191267in}}%
\pgfpathcurveto{\pgfqpoint{2.373649in}{3.199503in}}{\pgfqpoint{2.370376in}{3.207403in}}{\pgfqpoint{2.364552in}{3.213227in}}%
\pgfpathcurveto{\pgfqpoint{2.358728in}{3.219051in}}{\pgfqpoint{2.350828in}{3.222323in}}{\pgfqpoint{2.342592in}{3.222323in}}%
\pgfpathcurveto{\pgfqpoint{2.334356in}{3.222323in}}{\pgfqpoint{2.326456in}{3.219051in}}{\pgfqpoint{2.320632in}{3.213227in}}%
\pgfpathcurveto{\pgfqpoint{2.314808in}{3.207403in}}{\pgfqpoint{2.311536in}{3.199503in}}{\pgfqpoint{2.311536in}{3.191267in}}%
\pgfpathcurveto{\pgfqpoint{2.311536in}{3.183031in}}{\pgfqpoint{2.314808in}{3.175130in}}{\pgfqpoint{2.320632in}{3.169307in}}%
\pgfpathcurveto{\pgfqpoint{2.326456in}{3.163483in}}{\pgfqpoint{2.334356in}{3.160210in}}{\pgfqpoint{2.342592in}{3.160210in}}%
\pgfpathclose%
\pgfusepath{stroke,fill}%
\end{pgfscope}%
\begin{pgfscope}%
\pgfpathrectangle{\pgfqpoint{0.100000in}{0.220728in}}{\pgfqpoint{3.696000in}{3.696000in}}%
\pgfusepath{clip}%
\pgfsetbuttcap%
\pgfsetroundjoin%
\definecolor{currentfill}{rgb}{0.121569,0.466667,0.705882}%
\pgfsetfillcolor{currentfill}%
\pgfsetfillopacity{0.431506}%
\pgfsetlinewidth{1.003750pt}%
\definecolor{currentstroke}{rgb}{0.121569,0.466667,0.705882}%
\pgfsetstrokecolor{currentstroke}%
\pgfsetstrokeopacity{0.431506}%
\pgfsetdash{}{0pt}%
\pgfpathmoveto{\pgfqpoint{2.345450in}{3.159685in}}%
\pgfpathcurveto{\pgfqpoint{2.353686in}{3.159685in}}{\pgfqpoint{2.361586in}{3.162957in}}{\pgfqpoint{2.367410in}{3.168781in}}%
\pgfpathcurveto{\pgfqpoint{2.373234in}{3.174605in}}{\pgfqpoint{2.376506in}{3.182505in}}{\pgfqpoint{2.376506in}{3.190741in}}%
\pgfpathcurveto{\pgfqpoint{2.376506in}{3.198978in}}{\pgfqpoint{2.373234in}{3.206878in}}{\pgfqpoint{2.367410in}{3.212702in}}%
\pgfpathcurveto{\pgfqpoint{2.361586in}{3.218526in}}{\pgfqpoint{2.353686in}{3.221798in}}{\pgfqpoint{2.345450in}{3.221798in}}%
\pgfpathcurveto{\pgfqpoint{2.337214in}{3.221798in}}{\pgfqpoint{2.329314in}{3.218526in}}{\pgfqpoint{2.323490in}{3.212702in}}%
\pgfpathcurveto{\pgfqpoint{2.317666in}{3.206878in}}{\pgfqpoint{2.314393in}{3.198978in}}{\pgfqpoint{2.314393in}{3.190741in}}%
\pgfpathcurveto{\pgfqpoint{2.314393in}{3.182505in}}{\pgfqpoint{2.317666in}{3.174605in}}{\pgfqpoint{2.323490in}{3.168781in}}%
\pgfpathcurveto{\pgfqpoint{2.329314in}{3.162957in}}{\pgfqpoint{2.337214in}{3.159685in}}{\pgfqpoint{2.345450in}{3.159685in}}%
\pgfpathclose%
\pgfusepath{stroke,fill}%
\end{pgfscope}%
\begin{pgfscope}%
\pgfpathrectangle{\pgfqpoint{0.100000in}{0.220728in}}{\pgfqpoint{3.696000in}{3.696000in}}%
\pgfusepath{clip}%
\pgfsetbuttcap%
\pgfsetroundjoin%
\definecolor{currentfill}{rgb}{0.121569,0.466667,0.705882}%
\pgfsetfillcolor{currentfill}%
\pgfsetfillopacity{0.432820}%
\pgfsetlinewidth{1.003750pt}%
\definecolor{currentstroke}{rgb}{0.121569,0.466667,0.705882}%
\pgfsetstrokecolor{currentstroke}%
\pgfsetstrokeopacity{0.432820}%
\pgfsetdash{}{0pt}%
\pgfpathmoveto{\pgfqpoint{1.401766in}{2.416501in}}%
\pgfpathcurveto{\pgfqpoint{1.410002in}{2.416501in}}{\pgfqpoint{1.417903in}{2.419773in}}{\pgfqpoint{1.423726in}{2.425597in}}%
\pgfpathcurveto{\pgfqpoint{1.429550in}{2.431421in}}{\pgfqpoint{1.432823in}{2.439321in}}{\pgfqpoint{1.432823in}{2.447557in}}%
\pgfpathcurveto{\pgfqpoint{1.432823in}{2.455794in}}{\pgfqpoint{1.429550in}{2.463694in}}{\pgfqpoint{1.423726in}{2.469518in}}%
\pgfpathcurveto{\pgfqpoint{1.417903in}{2.475342in}}{\pgfqpoint{1.410002in}{2.478614in}}{\pgfqpoint{1.401766in}{2.478614in}}%
\pgfpathcurveto{\pgfqpoint{1.393530in}{2.478614in}}{\pgfqpoint{1.385630in}{2.475342in}}{\pgfqpoint{1.379806in}{2.469518in}}%
\pgfpathcurveto{\pgfqpoint{1.373982in}{2.463694in}}{\pgfqpoint{1.370710in}{2.455794in}}{\pgfqpoint{1.370710in}{2.447557in}}%
\pgfpathcurveto{\pgfqpoint{1.370710in}{2.439321in}}{\pgfqpoint{1.373982in}{2.431421in}}{\pgfqpoint{1.379806in}{2.425597in}}%
\pgfpathcurveto{\pgfqpoint{1.385630in}{2.419773in}}{\pgfqpoint{1.393530in}{2.416501in}}{\pgfqpoint{1.401766in}{2.416501in}}%
\pgfpathclose%
\pgfusepath{stroke,fill}%
\end{pgfscope}%
\begin{pgfscope}%
\pgfpathrectangle{\pgfqpoint{0.100000in}{0.220728in}}{\pgfqpoint{3.696000in}{3.696000in}}%
\pgfusepath{clip}%
\pgfsetbuttcap%
\pgfsetroundjoin%
\definecolor{currentfill}{rgb}{0.121569,0.466667,0.705882}%
\pgfsetfillcolor{currentfill}%
\pgfsetfillopacity{0.433170}%
\pgfsetlinewidth{1.003750pt}%
\definecolor{currentstroke}{rgb}{0.121569,0.466667,0.705882}%
\pgfsetstrokecolor{currentstroke}%
\pgfsetstrokeopacity{0.433170}%
\pgfsetdash{}{0pt}%
\pgfpathmoveto{\pgfqpoint{2.349402in}{3.159020in}}%
\pgfpathcurveto{\pgfqpoint{2.357639in}{3.159020in}}{\pgfqpoint{2.365539in}{3.162292in}}{\pgfqpoint{2.371363in}{3.168116in}}%
\pgfpathcurveto{\pgfqpoint{2.377187in}{3.173940in}}{\pgfqpoint{2.380459in}{3.181840in}}{\pgfqpoint{2.380459in}{3.190077in}}%
\pgfpathcurveto{\pgfqpoint{2.380459in}{3.198313in}}{\pgfqpoint{2.377187in}{3.206213in}}{\pgfqpoint{2.371363in}{3.212037in}}%
\pgfpathcurveto{\pgfqpoint{2.365539in}{3.217861in}}{\pgfqpoint{2.357639in}{3.221133in}}{\pgfqpoint{2.349402in}{3.221133in}}%
\pgfpathcurveto{\pgfqpoint{2.341166in}{3.221133in}}{\pgfqpoint{2.333266in}{3.217861in}}{\pgfqpoint{2.327442in}{3.212037in}}%
\pgfpathcurveto{\pgfqpoint{2.321618in}{3.206213in}}{\pgfqpoint{2.318346in}{3.198313in}}{\pgfqpoint{2.318346in}{3.190077in}}%
\pgfpathcurveto{\pgfqpoint{2.318346in}{3.181840in}}{\pgfqpoint{2.321618in}{3.173940in}}{\pgfqpoint{2.327442in}{3.168116in}}%
\pgfpathcurveto{\pgfqpoint{2.333266in}{3.162292in}}{\pgfqpoint{2.341166in}{3.159020in}}{\pgfqpoint{2.349402in}{3.159020in}}%
\pgfpathclose%
\pgfusepath{stroke,fill}%
\end{pgfscope}%
\begin{pgfscope}%
\pgfpathrectangle{\pgfqpoint{0.100000in}{0.220728in}}{\pgfqpoint{3.696000in}{3.696000in}}%
\pgfusepath{clip}%
\pgfsetbuttcap%
\pgfsetroundjoin%
\definecolor{currentfill}{rgb}{0.121569,0.466667,0.705882}%
\pgfsetfillcolor{currentfill}%
\pgfsetfillopacity{0.433617}%
\pgfsetlinewidth{1.003750pt}%
\definecolor{currentstroke}{rgb}{0.121569,0.466667,0.705882}%
\pgfsetstrokecolor{currentstroke}%
\pgfsetstrokeopacity{0.433617}%
\pgfsetdash{}{0pt}%
\pgfpathmoveto{\pgfqpoint{2.352163in}{3.158385in}}%
\pgfpathcurveto{\pgfqpoint{2.360399in}{3.158385in}}{\pgfqpoint{2.368299in}{3.161658in}}{\pgfqpoint{2.374123in}{3.167481in}}%
\pgfpathcurveto{\pgfqpoint{2.379947in}{3.173305in}}{\pgfqpoint{2.383219in}{3.181205in}}{\pgfqpoint{2.383219in}{3.189442in}}%
\pgfpathcurveto{\pgfqpoint{2.383219in}{3.197678in}}{\pgfqpoint{2.379947in}{3.205578in}}{\pgfqpoint{2.374123in}{3.211402in}}%
\pgfpathcurveto{\pgfqpoint{2.368299in}{3.217226in}}{\pgfqpoint{2.360399in}{3.220498in}}{\pgfqpoint{2.352163in}{3.220498in}}%
\pgfpathcurveto{\pgfqpoint{2.343926in}{3.220498in}}{\pgfqpoint{2.336026in}{3.217226in}}{\pgfqpoint{2.330202in}{3.211402in}}%
\pgfpathcurveto{\pgfqpoint{2.324378in}{3.205578in}}{\pgfqpoint{2.321106in}{3.197678in}}{\pgfqpoint{2.321106in}{3.189442in}}%
\pgfpathcurveto{\pgfqpoint{2.321106in}{3.181205in}}{\pgfqpoint{2.324378in}{3.173305in}}{\pgfqpoint{2.330202in}{3.167481in}}%
\pgfpathcurveto{\pgfqpoint{2.336026in}{3.161658in}}{\pgfqpoint{2.343926in}{3.158385in}}{\pgfqpoint{2.352163in}{3.158385in}}%
\pgfpathclose%
\pgfusepath{stroke,fill}%
\end{pgfscope}%
\begin{pgfscope}%
\pgfpathrectangle{\pgfqpoint{0.100000in}{0.220728in}}{\pgfqpoint{3.696000in}{3.696000in}}%
\pgfusepath{clip}%
\pgfsetbuttcap%
\pgfsetroundjoin%
\definecolor{currentfill}{rgb}{0.121569,0.466667,0.705882}%
\pgfsetfillcolor{currentfill}%
\pgfsetfillopacity{0.434656}%
\pgfsetlinewidth{1.003750pt}%
\definecolor{currentstroke}{rgb}{0.121569,0.466667,0.705882}%
\pgfsetstrokecolor{currentstroke}%
\pgfsetstrokeopacity{0.434656}%
\pgfsetdash{}{0pt}%
\pgfpathmoveto{\pgfqpoint{2.356436in}{3.157328in}}%
\pgfpathcurveto{\pgfqpoint{2.364672in}{3.157328in}}{\pgfqpoint{2.372572in}{3.160600in}}{\pgfqpoint{2.378396in}{3.166424in}}%
\pgfpathcurveto{\pgfqpoint{2.384220in}{3.172248in}}{\pgfqpoint{2.387492in}{3.180148in}}{\pgfqpoint{2.387492in}{3.188385in}}%
\pgfpathcurveto{\pgfqpoint{2.387492in}{3.196621in}}{\pgfqpoint{2.384220in}{3.204521in}}{\pgfqpoint{2.378396in}{3.210345in}}%
\pgfpathcurveto{\pgfqpoint{2.372572in}{3.216169in}}{\pgfqpoint{2.364672in}{3.219441in}}{\pgfqpoint{2.356436in}{3.219441in}}%
\pgfpathcurveto{\pgfqpoint{2.348199in}{3.219441in}}{\pgfqpoint{2.340299in}{3.216169in}}{\pgfqpoint{2.334475in}{3.210345in}}%
\pgfpathcurveto{\pgfqpoint{2.328652in}{3.204521in}}{\pgfqpoint{2.325379in}{3.196621in}}{\pgfqpoint{2.325379in}{3.188385in}}%
\pgfpathcurveto{\pgfqpoint{2.325379in}{3.180148in}}{\pgfqpoint{2.328652in}{3.172248in}}{\pgfqpoint{2.334475in}{3.166424in}}%
\pgfpathcurveto{\pgfqpoint{2.340299in}{3.160600in}}{\pgfqpoint{2.348199in}{3.157328in}}{\pgfqpoint{2.356436in}{3.157328in}}%
\pgfpathclose%
\pgfusepath{stroke,fill}%
\end{pgfscope}%
\begin{pgfscope}%
\pgfpathrectangle{\pgfqpoint{0.100000in}{0.220728in}}{\pgfqpoint{3.696000in}{3.696000in}}%
\pgfusepath{clip}%
\pgfsetbuttcap%
\pgfsetroundjoin%
\definecolor{currentfill}{rgb}{0.121569,0.466667,0.705882}%
\pgfsetfillcolor{currentfill}%
\pgfsetfillopacity{0.436239}%
\pgfsetlinewidth{1.003750pt}%
\definecolor{currentstroke}{rgb}{0.121569,0.466667,0.705882}%
\pgfsetstrokecolor{currentstroke}%
\pgfsetstrokeopacity{0.436239}%
\pgfsetdash{}{0pt}%
\pgfpathmoveto{\pgfqpoint{2.362173in}{3.155511in}}%
\pgfpathcurveto{\pgfqpoint{2.370409in}{3.155511in}}{\pgfqpoint{2.378309in}{3.158783in}}{\pgfqpoint{2.384133in}{3.164607in}}%
\pgfpathcurveto{\pgfqpoint{2.389957in}{3.170431in}}{\pgfqpoint{2.393229in}{3.178331in}}{\pgfqpoint{2.393229in}{3.186568in}}%
\pgfpathcurveto{\pgfqpoint{2.393229in}{3.194804in}}{\pgfqpoint{2.389957in}{3.202704in}}{\pgfqpoint{2.384133in}{3.208528in}}%
\pgfpathcurveto{\pgfqpoint{2.378309in}{3.214352in}}{\pgfqpoint{2.370409in}{3.217624in}}{\pgfqpoint{2.362173in}{3.217624in}}%
\pgfpathcurveto{\pgfqpoint{2.353936in}{3.217624in}}{\pgfqpoint{2.346036in}{3.214352in}}{\pgfqpoint{2.340212in}{3.208528in}}%
\pgfpathcurveto{\pgfqpoint{2.334388in}{3.202704in}}{\pgfqpoint{2.331116in}{3.194804in}}{\pgfqpoint{2.331116in}{3.186568in}}%
\pgfpathcurveto{\pgfqpoint{2.331116in}{3.178331in}}{\pgfqpoint{2.334388in}{3.170431in}}{\pgfqpoint{2.340212in}{3.164607in}}%
\pgfpathcurveto{\pgfqpoint{2.346036in}{3.158783in}}{\pgfqpoint{2.353936in}{3.155511in}}{\pgfqpoint{2.362173in}{3.155511in}}%
\pgfpathclose%
\pgfusepath{stroke,fill}%
\end{pgfscope}%
\begin{pgfscope}%
\pgfpathrectangle{\pgfqpoint{0.100000in}{0.220728in}}{\pgfqpoint{3.696000in}{3.696000in}}%
\pgfusepath{clip}%
\pgfsetbuttcap%
\pgfsetroundjoin%
\definecolor{currentfill}{rgb}{0.121569,0.466667,0.705882}%
\pgfsetfillcolor{currentfill}%
\pgfsetfillopacity{0.436602}%
\pgfsetlinewidth{1.003750pt}%
\definecolor{currentstroke}{rgb}{0.121569,0.466667,0.705882}%
\pgfsetstrokecolor{currentstroke}%
\pgfsetstrokeopacity{0.436602}%
\pgfsetdash{}{0pt}%
\pgfpathmoveto{\pgfqpoint{1.387479in}{2.393276in}}%
\pgfpathcurveto{\pgfqpoint{1.395715in}{2.393276in}}{\pgfqpoint{1.403615in}{2.396548in}}{\pgfqpoint{1.409439in}{2.402372in}}%
\pgfpathcurveto{\pgfqpoint{1.415263in}{2.408196in}}{\pgfqpoint{1.418535in}{2.416096in}}{\pgfqpoint{1.418535in}{2.424332in}}%
\pgfpathcurveto{\pgfqpoint{1.418535in}{2.432568in}}{\pgfqpoint{1.415263in}{2.440468in}}{\pgfqpoint{1.409439in}{2.446292in}}%
\pgfpathcurveto{\pgfqpoint{1.403615in}{2.452116in}}{\pgfqpoint{1.395715in}{2.455389in}}{\pgfqpoint{1.387479in}{2.455389in}}%
\pgfpathcurveto{\pgfqpoint{1.379243in}{2.455389in}}{\pgfqpoint{1.371343in}{2.452116in}}{\pgfqpoint{1.365519in}{2.446292in}}%
\pgfpathcurveto{\pgfqpoint{1.359695in}{2.440468in}}{\pgfqpoint{1.356422in}{2.432568in}}{\pgfqpoint{1.356422in}{2.424332in}}%
\pgfpathcurveto{\pgfqpoint{1.356422in}{2.416096in}}{\pgfqpoint{1.359695in}{2.408196in}}{\pgfqpoint{1.365519in}{2.402372in}}%
\pgfpathcurveto{\pgfqpoint{1.371343in}{2.396548in}}{\pgfqpoint{1.379243in}{2.393276in}}{\pgfqpoint{1.387479in}{2.393276in}}%
\pgfpathclose%
\pgfusepath{stroke,fill}%
\end{pgfscope}%
\begin{pgfscope}%
\pgfpathrectangle{\pgfqpoint{0.100000in}{0.220728in}}{\pgfqpoint{3.696000in}{3.696000in}}%
\pgfusepath{clip}%
\pgfsetbuttcap%
\pgfsetroundjoin%
\definecolor{currentfill}{rgb}{0.121569,0.466667,0.705882}%
\pgfsetfillcolor{currentfill}%
\pgfsetfillopacity{0.437649}%
\pgfsetlinewidth{1.003750pt}%
\definecolor{currentstroke}{rgb}{0.121569,0.466667,0.705882}%
\pgfsetstrokecolor{currentstroke}%
\pgfsetstrokeopacity{0.437649}%
\pgfsetdash{}{0pt}%
\pgfpathmoveto{\pgfqpoint{2.369242in}{3.153247in}}%
\pgfpathcurveto{\pgfqpoint{2.377478in}{3.153247in}}{\pgfqpoint{2.385378in}{3.156519in}}{\pgfqpoint{2.391202in}{3.162343in}}%
\pgfpathcurveto{\pgfqpoint{2.397026in}{3.168167in}}{\pgfqpoint{2.400299in}{3.176067in}}{\pgfqpoint{2.400299in}{3.184304in}}%
\pgfpathcurveto{\pgfqpoint{2.400299in}{3.192540in}}{\pgfqpoint{2.397026in}{3.200440in}}{\pgfqpoint{2.391202in}{3.206264in}}%
\pgfpathcurveto{\pgfqpoint{2.385378in}{3.212088in}}{\pgfqpoint{2.377478in}{3.215360in}}{\pgfqpoint{2.369242in}{3.215360in}}%
\pgfpathcurveto{\pgfqpoint{2.361006in}{3.215360in}}{\pgfqpoint{2.353106in}{3.212088in}}{\pgfqpoint{2.347282in}{3.206264in}}%
\pgfpathcurveto{\pgfqpoint{2.341458in}{3.200440in}}{\pgfqpoint{2.338186in}{3.192540in}}{\pgfqpoint{2.338186in}{3.184304in}}%
\pgfpathcurveto{\pgfqpoint{2.338186in}{3.176067in}}{\pgfqpoint{2.341458in}{3.168167in}}{\pgfqpoint{2.347282in}{3.162343in}}%
\pgfpathcurveto{\pgfqpoint{2.353106in}{3.156519in}}{\pgfqpoint{2.361006in}{3.153247in}}{\pgfqpoint{2.369242in}{3.153247in}}%
\pgfpathclose%
\pgfusepath{stroke,fill}%
\end{pgfscope}%
\begin{pgfscope}%
\pgfpathrectangle{\pgfqpoint{0.100000in}{0.220728in}}{\pgfqpoint{3.696000in}{3.696000in}}%
\pgfusepath{clip}%
\pgfsetbuttcap%
\pgfsetroundjoin%
\definecolor{currentfill}{rgb}{0.121569,0.466667,0.705882}%
\pgfsetfillcolor{currentfill}%
\pgfsetfillopacity{0.440022}%
\pgfsetlinewidth{1.003750pt}%
\definecolor{currentstroke}{rgb}{0.121569,0.466667,0.705882}%
\pgfsetstrokecolor{currentstroke}%
\pgfsetstrokeopacity{0.440022}%
\pgfsetdash{}{0pt}%
\pgfpathmoveto{\pgfqpoint{2.376158in}{3.152088in}}%
\pgfpathcurveto{\pgfqpoint{2.384394in}{3.152088in}}{\pgfqpoint{2.392294in}{3.155360in}}{\pgfqpoint{2.398118in}{3.161184in}}%
\pgfpathcurveto{\pgfqpoint{2.403942in}{3.167008in}}{\pgfqpoint{2.407215in}{3.174908in}}{\pgfqpoint{2.407215in}{3.183144in}}%
\pgfpathcurveto{\pgfqpoint{2.407215in}{3.191380in}}{\pgfqpoint{2.403942in}{3.199280in}}{\pgfqpoint{2.398118in}{3.205104in}}%
\pgfpathcurveto{\pgfqpoint{2.392294in}{3.210928in}}{\pgfqpoint{2.384394in}{3.214201in}}{\pgfqpoint{2.376158in}{3.214201in}}%
\pgfpathcurveto{\pgfqpoint{2.367922in}{3.214201in}}{\pgfqpoint{2.360022in}{3.210928in}}{\pgfqpoint{2.354198in}{3.205104in}}%
\pgfpathcurveto{\pgfqpoint{2.348374in}{3.199280in}}{\pgfqpoint{2.345102in}{3.191380in}}{\pgfqpoint{2.345102in}{3.183144in}}%
\pgfpathcurveto{\pgfqpoint{2.345102in}{3.174908in}}{\pgfqpoint{2.348374in}{3.167008in}}{\pgfqpoint{2.354198in}{3.161184in}}%
\pgfpathcurveto{\pgfqpoint{2.360022in}{3.155360in}}{\pgfqpoint{2.367922in}{3.152088in}}{\pgfqpoint{2.376158in}{3.152088in}}%
\pgfpathclose%
\pgfusepath{stroke,fill}%
\end{pgfscope}%
\begin{pgfscope}%
\pgfpathrectangle{\pgfqpoint{0.100000in}{0.220728in}}{\pgfqpoint{3.696000in}{3.696000in}}%
\pgfusepath{clip}%
\pgfsetbuttcap%
\pgfsetroundjoin%
\definecolor{currentfill}{rgb}{0.121569,0.466667,0.705882}%
\pgfsetfillcolor{currentfill}%
\pgfsetfillopacity{0.440623}%
\pgfsetlinewidth{1.003750pt}%
\definecolor{currentstroke}{rgb}{0.121569,0.466667,0.705882}%
\pgfsetstrokecolor{currentstroke}%
\pgfsetstrokeopacity{0.440623}%
\pgfsetdash{}{0pt}%
\pgfpathmoveto{\pgfqpoint{1.376013in}{2.370119in}}%
\pgfpathcurveto{\pgfqpoint{1.384249in}{2.370119in}}{\pgfqpoint{1.392149in}{2.373391in}}{\pgfqpoint{1.397973in}{2.379215in}}%
\pgfpathcurveto{\pgfqpoint{1.403797in}{2.385039in}}{\pgfqpoint{1.407069in}{2.392939in}}{\pgfqpoint{1.407069in}{2.401175in}}%
\pgfpathcurveto{\pgfqpoint{1.407069in}{2.409411in}}{\pgfqpoint{1.403797in}{2.417311in}}{\pgfqpoint{1.397973in}{2.423135in}}%
\pgfpathcurveto{\pgfqpoint{1.392149in}{2.428959in}}{\pgfqpoint{1.384249in}{2.432232in}}{\pgfqpoint{1.376013in}{2.432232in}}%
\pgfpathcurveto{\pgfqpoint{1.367776in}{2.432232in}}{\pgfqpoint{1.359876in}{2.428959in}}{\pgfqpoint{1.354052in}{2.423135in}}%
\pgfpathcurveto{\pgfqpoint{1.348229in}{2.417311in}}{\pgfqpoint{1.344956in}{2.409411in}}{\pgfqpoint{1.344956in}{2.401175in}}%
\pgfpathcurveto{\pgfqpoint{1.344956in}{2.392939in}}{\pgfqpoint{1.348229in}{2.385039in}}{\pgfqpoint{1.354052in}{2.379215in}}%
\pgfpathcurveto{\pgfqpoint{1.359876in}{2.373391in}}{\pgfqpoint{1.367776in}{2.370119in}}{\pgfqpoint{1.376013in}{2.370119in}}%
\pgfpathclose%
\pgfusepath{stroke,fill}%
\end{pgfscope}%
\begin{pgfscope}%
\pgfpathrectangle{\pgfqpoint{0.100000in}{0.220728in}}{\pgfqpoint{3.696000in}{3.696000in}}%
\pgfusepath{clip}%
\pgfsetbuttcap%
\pgfsetroundjoin%
\definecolor{currentfill}{rgb}{0.121569,0.466667,0.705882}%
\pgfsetfillcolor{currentfill}%
\pgfsetfillopacity{0.441437}%
\pgfsetlinewidth{1.003750pt}%
\definecolor{currentstroke}{rgb}{0.121569,0.466667,0.705882}%
\pgfsetstrokecolor{currentstroke}%
\pgfsetstrokeopacity{0.441437}%
\pgfsetdash{}{0pt}%
\pgfpathmoveto{\pgfqpoint{2.385028in}{3.148604in}}%
\pgfpathcurveto{\pgfqpoint{2.393265in}{3.148604in}}{\pgfqpoint{2.401165in}{3.151876in}}{\pgfqpoint{2.406989in}{3.157700in}}%
\pgfpathcurveto{\pgfqpoint{2.412813in}{3.163524in}}{\pgfqpoint{2.416085in}{3.171424in}}{\pgfqpoint{2.416085in}{3.179660in}}%
\pgfpathcurveto{\pgfqpoint{2.416085in}{3.187897in}}{\pgfqpoint{2.412813in}{3.195797in}}{\pgfqpoint{2.406989in}{3.201621in}}%
\pgfpathcurveto{\pgfqpoint{2.401165in}{3.207445in}}{\pgfqpoint{2.393265in}{3.210717in}}{\pgfqpoint{2.385028in}{3.210717in}}%
\pgfpathcurveto{\pgfqpoint{2.376792in}{3.210717in}}{\pgfqpoint{2.368892in}{3.207445in}}{\pgfqpoint{2.363068in}{3.201621in}}%
\pgfpathcurveto{\pgfqpoint{2.357244in}{3.195797in}}{\pgfqpoint{2.353972in}{3.187897in}}{\pgfqpoint{2.353972in}{3.179660in}}%
\pgfpathcurveto{\pgfqpoint{2.353972in}{3.171424in}}{\pgfqpoint{2.357244in}{3.163524in}}{\pgfqpoint{2.363068in}{3.157700in}}%
\pgfpathcurveto{\pgfqpoint{2.368892in}{3.151876in}}{\pgfqpoint{2.376792in}{3.148604in}}{\pgfqpoint{2.385028in}{3.148604in}}%
\pgfpathclose%
\pgfusepath{stroke,fill}%
\end{pgfscope}%
\begin{pgfscope}%
\pgfpathrectangle{\pgfqpoint{0.100000in}{0.220728in}}{\pgfqpoint{3.696000in}{3.696000in}}%
\pgfusepath{clip}%
\pgfsetbuttcap%
\pgfsetroundjoin%
\definecolor{currentfill}{rgb}{0.121569,0.466667,0.705882}%
\pgfsetfillcolor{currentfill}%
\pgfsetfillopacity{0.443535}%
\pgfsetlinewidth{1.003750pt}%
\definecolor{currentstroke}{rgb}{0.121569,0.466667,0.705882}%
\pgfsetstrokecolor{currentstroke}%
\pgfsetstrokeopacity{0.443535}%
\pgfsetdash{}{0pt}%
\pgfpathmoveto{\pgfqpoint{2.393823in}{3.145788in}}%
\pgfpathcurveto{\pgfqpoint{2.402059in}{3.145788in}}{\pgfqpoint{2.409959in}{3.149060in}}{\pgfqpoint{2.415783in}{3.154884in}}%
\pgfpathcurveto{\pgfqpoint{2.421607in}{3.160708in}}{\pgfqpoint{2.424879in}{3.168608in}}{\pgfqpoint{2.424879in}{3.176844in}}%
\pgfpathcurveto{\pgfqpoint{2.424879in}{3.185080in}}{\pgfqpoint{2.421607in}{3.192981in}}{\pgfqpoint{2.415783in}{3.198804in}}%
\pgfpathcurveto{\pgfqpoint{2.409959in}{3.204628in}}{\pgfqpoint{2.402059in}{3.207901in}}{\pgfqpoint{2.393823in}{3.207901in}}%
\pgfpathcurveto{\pgfqpoint{2.385586in}{3.207901in}}{\pgfqpoint{2.377686in}{3.204628in}}{\pgfqpoint{2.371862in}{3.198804in}}%
\pgfpathcurveto{\pgfqpoint{2.366038in}{3.192981in}}{\pgfqpoint{2.362766in}{3.185080in}}{\pgfqpoint{2.362766in}{3.176844in}}%
\pgfpathcurveto{\pgfqpoint{2.362766in}{3.168608in}}{\pgfqpoint{2.366038in}{3.160708in}}{\pgfqpoint{2.371862in}{3.154884in}}%
\pgfpathcurveto{\pgfqpoint{2.377686in}{3.149060in}}{\pgfqpoint{2.385586in}{3.145788in}}{\pgfqpoint{2.393823in}{3.145788in}}%
\pgfpathclose%
\pgfusepath{stroke,fill}%
\end{pgfscope}%
\begin{pgfscope}%
\pgfpathrectangle{\pgfqpoint{0.100000in}{0.220728in}}{\pgfqpoint{3.696000in}{3.696000in}}%
\pgfusepath{clip}%
\pgfsetbuttcap%
\pgfsetroundjoin%
\definecolor{currentfill}{rgb}{0.121569,0.466667,0.705882}%
\pgfsetfillcolor{currentfill}%
\pgfsetfillopacity{0.444759}%
\pgfsetlinewidth{1.003750pt}%
\definecolor{currentstroke}{rgb}{0.121569,0.466667,0.705882}%
\pgfsetstrokecolor{currentstroke}%
\pgfsetstrokeopacity{0.444759}%
\pgfsetdash{}{0pt}%
\pgfpathmoveto{\pgfqpoint{1.367825in}{2.345931in}}%
\pgfpathcurveto{\pgfqpoint{1.376062in}{2.345931in}}{\pgfqpoint{1.383962in}{2.349203in}}{\pgfqpoint{1.389786in}{2.355027in}}%
\pgfpathcurveto{\pgfqpoint{1.395610in}{2.360851in}}{\pgfqpoint{1.398882in}{2.368751in}}{\pgfqpoint{1.398882in}{2.376987in}}%
\pgfpathcurveto{\pgfqpoint{1.398882in}{2.385223in}}{\pgfqpoint{1.395610in}{2.393123in}}{\pgfqpoint{1.389786in}{2.398947in}}%
\pgfpathcurveto{\pgfqpoint{1.383962in}{2.404771in}}{\pgfqpoint{1.376062in}{2.408044in}}{\pgfqpoint{1.367825in}{2.408044in}}%
\pgfpathcurveto{\pgfqpoint{1.359589in}{2.408044in}}{\pgfqpoint{1.351689in}{2.404771in}}{\pgfqpoint{1.345865in}{2.398947in}}%
\pgfpathcurveto{\pgfqpoint{1.340041in}{2.393123in}}{\pgfqpoint{1.336769in}{2.385223in}}{\pgfqpoint{1.336769in}{2.376987in}}%
\pgfpathcurveto{\pgfqpoint{1.336769in}{2.368751in}}{\pgfqpoint{1.340041in}{2.360851in}}{\pgfqpoint{1.345865in}{2.355027in}}%
\pgfpathcurveto{\pgfqpoint{1.351689in}{2.349203in}}{\pgfqpoint{1.359589in}{2.345931in}}{\pgfqpoint{1.367825in}{2.345931in}}%
\pgfpathclose%
\pgfusepath{stroke,fill}%
\end{pgfscope}%
\begin{pgfscope}%
\pgfpathrectangle{\pgfqpoint{0.100000in}{0.220728in}}{\pgfqpoint{3.696000in}{3.696000in}}%
\pgfusepath{clip}%
\pgfsetbuttcap%
\pgfsetroundjoin%
\definecolor{currentfill}{rgb}{0.121569,0.466667,0.705882}%
\pgfsetfillcolor{currentfill}%
\pgfsetfillopacity{0.446278}%
\pgfsetlinewidth{1.003750pt}%
\definecolor{currentstroke}{rgb}{0.121569,0.466667,0.705882}%
\pgfsetstrokecolor{currentstroke}%
\pgfsetstrokeopacity{0.446278}%
\pgfsetdash{}{0pt}%
\pgfpathmoveto{\pgfqpoint{2.404820in}{3.144115in}}%
\pgfpathcurveto{\pgfqpoint{2.413057in}{3.144115in}}{\pgfqpoint{2.420957in}{3.147388in}}{\pgfqpoint{2.426781in}{3.153212in}}%
\pgfpathcurveto{\pgfqpoint{2.432604in}{3.159036in}}{\pgfqpoint{2.435877in}{3.166936in}}{\pgfqpoint{2.435877in}{3.175172in}}%
\pgfpathcurveto{\pgfqpoint{2.435877in}{3.183408in}}{\pgfqpoint{2.432604in}{3.191308in}}{\pgfqpoint{2.426781in}{3.197132in}}%
\pgfpathcurveto{\pgfqpoint{2.420957in}{3.202956in}}{\pgfqpoint{2.413057in}{3.206228in}}{\pgfqpoint{2.404820in}{3.206228in}}%
\pgfpathcurveto{\pgfqpoint{2.396584in}{3.206228in}}{\pgfqpoint{2.388684in}{3.202956in}}{\pgfqpoint{2.382860in}{3.197132in}}%
\pgfpathcurveto{\pgfqpoint{2.377036in}{3.191308in}}{\pgfqpoint{2.373764in}{3.183408in}}{\pgfqpoint{2.373764in}{3.175172in}}%
\pgfpathcurveto{\pgfqpoint{2.373764in}{3.166936in}}{\pgfqpoint{2.377036in}{3.159036in}}{\pgfqpoint{2.382860in}{3.153212in}}%
\pgfpathcurveto{\pgfqpoint{2.388684in}{3.147388in}}{\pgfqpoint{2.396584in}{3.144115in}}{\pgfqpoint{2.404820in}{3.144115in}}%
\pgfpathclose%
\pgfusepath{stroke,fill}%
\end{pgfscope}%
\begin{pgfscope}%
\pgfpathrectangle{\pgfqpoint{0.100000in}{0.220728in}}{\pgfqpoint{3.696000in}{3.696000in}}%
\pgfusepath{clip}%
\pgfsetbuttcap%
\pgfsetroundjoin%
\definecolor{currentfill}{rgb}{0.121569,0.466667,0.705882}%
\pgfsetfillcolor{currentfill}%
\pgfsetfillopacity{0.447232}%
\pgfsetlinewidth{1.003750pt}%
\definecolor{currentstroke}{rgb}{0.121569,0.466667,0.705882}%
\pgfsetstrokecolor{currentstroke}%
\pgfsetstrokeopacity{0.447232}%
\pgfsetdash{}{0pt}%
\pgfpathmoveto{\pgfqpoint{1.353544in}{2.326967in}}%
\pgfpathcurveto{\pgfqpoint{1.361780in}{2.326967in}}{\pgfqpoint{1.369680in}{2.330240in}}{\pgfqpoint{1.375504in}{2.336063in}}%
\pgfpathcurveto{\pgfqpoint{1.381328in}{2.341887in}}{\pgfqpoint{1.384600in}{2.349787in}}{\pgfqpoint{1.384600in}{2.358024in}}%
\pgfpathcurveto{\pgfqpoint{1.384600in}{2.366260in}}{\pgfqpoint{1.381328in}{2.374160in}}{\pgfqpoint{1.375504in}{2.379984in}}%
\pgfpathcurveto{\pgfqpoint{1.369680in}{2.385808in}}{\pgfqpoint{1.361780in}{2.389080in}}{\pgfqpoint{1.353544in}{2.389080in}}%
\pgfpathcurveto{\pgfqpoint{1.345307in}{2.389080in}}{\pgfqpoint{1.337407in}{2.385808in}}{\pgfqpoint{1.331583in}{2.379984in}}%
\pgfpathcurveto{\pgfqpoint{1.325760in}{2.374160in}}{\pgfqpoint{1.322487in}{2.366260in}}{\pgfqpoint{1.322487in}{2.358024in}}%
\pgfpathcurveto{\pgfqpoint{1.322487in}{2.349787in}}{\pgfqpoint{1.325760in}{2.341887in}}{\pgfqpoint{1.331583in}{2.336063in}}%
\pgfpathcurveto{\pgfqpoint{1.337407in}{2.330240in}}{\pgfqpoint{1.345307in}{2.326967in}}{\pgfqpoint{1.353544in}{2.326967in}}%
\pgfpathclose%
\pgfusepath{stroke,fill}%
\end{pgfscope}%
\begin{pgfscope}%
\pgfpathrectangle{\pgfqpoint{0.100000in}{0.220728in}}{\pgfqpoint{3.696000in}{3.696000in}}%
\pgfusepath{clip}%
\pgfsetbuttcap%
\pgfsetroundjoin%
\definecolor{currentfill}{rgb}{0.121569,0.466667,0.705882}%
\pgfsetfillcolor{currentfill}%
\pgfsetfillopacity{0.448547}%
\pgfsetlinewidth{1.003750pt}%
\definecolor{currentstroke}{rgb}{0.121569,0.466667,0.705882}%
\pgfsetstrokecolor{currentstroke}%
\pgfsetstrokeopacity{0.448547}%
\pgfsetdash{}{0pt}%
\pgfpathmoveto{\pgfqpoint{2.416605in}{3.140308in}}%
\pgfpathcurveto{\pgfqpoint{2.424842in}{3.140308in}}{\pgfqpoint{2.432742in}{3.143580in}}{\pgfqpoint{2.438566in}{3.149404in}}%
\pgfpathcurveto{\pgfqpoint{2.444390in}{3.155228in}}{\pgfqpoint{2.447662in}{3.163128in}}{\pgfqpoint{2.447662in}{3.171364in}}%
\pgfpathcurveto{\pgfqpoint{2.447662in}{3.179600in}}{\pgfqpoint{2.444390in}{3.187500in}}{\pgfqpoint{2.438566in}{3.193324in}}%
\pgfpathcurveto{\pgfqpoint{2.432742in}{3.199148in}}{\pgfqpoint{2.424842in}{3.202421in}}{\pgfqpoint{2.416605in}{3.202421in}}%
\pgfpathcurveto{\pgfqpoint{2.408369in}{3.202421in}}{\pgfqpoint{2.400469in}{3.199148in}}{\pgfqpoint{2.394645in}{3.193324in}}%
\pgfpathcurveto{\pgfqpoint{2.388821in}{3.187500in}}{\pgfqpoint{2.385549in}{3.179600in}}{\pgfqpoint{2.385549in}{3.171364in}}%
\pgfpathcurveto{\pgfqpoint{2.385549in}{3.163128in}}{\pgfqpoint{2.388821in}{3.155228in}}{\pgfqpoint{2.394645in}{3.149404in}}%
\pgfpathcurveto{\pgfqpoint{2.400469in}{3.143580in}}{\pgfqpoint{2.408369in}{3.140308in}}{\pgfqpoint{2.416605in}{3.140308in}}%
\pgfpathclose%
\pgfusepath{stroke,fill}%
\end{pgfscope}%
\begin{pgfscope}%
\pgfpathrectangle{\pgfqpoint{0.100000in}{0.220728in}}{\pgfqpoint{3.696000in}{3.696000in}}%
\pgfusepath{clip}%
\pgfsetbuttcap%
\pgfsetroundjoin%
\definecolor{currentfill}{rgb}{0.121569,0.466667,0.705882}%
\pgfsetfillcolor{currentfill}%
\pgfsetfillopacity{0.450714}%
\pgfsetlinewidth{1.003750pt}%
\definecolor{currentstroke}{rgb}{0.121569,0.466667,0.705882}%
\pgfsetstrokecolor{currentstroke}%
\pgfsetstrokeopacity{0.450714}%
\pgfsetdash{}{0pt}%
\pgfpathmoveto{\pgfqpoint{1.349960in}{2.304890in}}%
\pgfpathcurveto{\pgfqpoint{1.358196in}{2.304890in}}{\pgfqpoint{1.366096in}{2.308162in}}{\pgfqpoint{1.371920in}{2.313986in}}%
\pgfpathcurveto{\pgfqpoint{1.377744in}{2.319810in}}{\pgfqpoint{1.381017in}{2.327710in}}{\pgfqpoint{1.381017in}{2.335946in}}%
\pgfpathcurveto{\pgfqpoint{1.381017in}{2.344182in}}{\pgfqpoint{1.377744in}{2.352082in}}{\pgfqpoint{1.371920in}{2.357906in}}%
\pgfpathcurveto{\pgfqpoint{1.366096in}{2.363730in}}{\pgfqpoint{1.358196in}{2.367003in}}{\pgfqpoint{1.349960in}{2.367003in}}%
\pgfpathcurveto{\pgfqpoint{1.341724in}{2.367003in}}{\pgfqpoint{1.333824in}{2.363730in}}{\pgfqpoint{1.328000in}{2.357906in}}%
\pgfpathcurveto{\pgfqpoint{1.322176in}{2.352082in}}{\pgfqpoint{1.318904in}{2.344182in}}{\pgfqpoint{1.318904in}{2.335946in}}%
\pgfpathcurveto{\pgfqpoint{1.318904in}{2.327710in}}{\pgfqpoint{1.322176in}{2.319810in}}{\pgfqpoint{1.328000in}{2.313986in}}%
\pgfpathcurveto{\pgfqpoint{1.333824in}{2.308162in}}{\pgfqpoint{1.341724in}{2.304890in}}{\pgfqpoint{1.349960in}{2.304890in}}%
\pgfpathclose%
\pgfusepath{stroke,fill}%
\end{pgfscope}%
\begin{pgfscope}%
\pgfpathrectangle{\pgfqpoint{0.100000in}{0.220728in}}{\pgfqpoint{3.696000in}{3.696000in}}%
\pgfusepath{clip}%
\pgfsetbuttcap%
\pgfsetroundjoin%
\definecolor{currentfill}{rgb}{0.121569,0.466667,0.705882}%
\pgfsetfillcolor{currentfill}%
\pgfsetfillopacity{0.452164}%
\pgfsetlinewidth{1.003750pt}%
\definecolor{currentstroke}{rgb}{0.121569,0.466667,0.705882}%
\pgfsetstrokecolor{currentstroke}%
\pgfsetstrokeopacity{0.452164}%
\pgfsetdash{}{0pt}%
\pgfpathmoveto{\pgfqpoint{1.336983in}{2.290592in}}%
\pgfpathcurveto{\pgfqpoint{1.345220in}{2.290592in}}{\pgfqpoint{1.353120in}{2.293864in}}{\pgfqpoint{1.358944in}{2.299688in}}%
\pgfpathcurveto{\pgfqpoint{1.364768in}{2.305512in}}{\pgfqpoint{1.368040in}{2.313412in}}{\pgfqpoint{1.368040in}{2.321648in}}%
\pgfpathcurveto{\pgfqpoint{1.368040in}{2.329885in}}{\pgfqpoint{1.364768in}{2.337785in}}{\pgfqpoint{1.358944in}{2.343609in}}%
\pgfpathcurveto{\pgfqpoint{1.353120in}{2.349433in}}{\pgfqpoint{1.345220in}{2.352705in}}{\pgfqpoint{1.336983in}{2.352705in}}%
\pgfpathcurveto{\pgfqpoint{1.328747in}{2.352705in}}{\pgfqpoint{1.320847in}{2.349433in}}{\pgfqpoint{1.315023in}{2.343609in}}%
\pgfpathcurveto{\pgfqpoint{1.309199in}{2.337785in}}{\pgfqpoint{1.305927in}{2.329885in}}{\pgfqpoint{1.305927in}{2.321648in}}%
\pgfpathcurveto{\pgfqpoint{1.305927in}{2.313412in}}{\pgfqpoint{1.309199in}{2.305512in}}{\pgfqpoint{1.315023in}{2.299688in}}%
\pgfpathcurveto{\pgfqpoint{1.320847in}{2.293864in}}{\pgfqpoint{1.328747in}{2.290592in}}{\pgfqpoint{1.336983in}{2.290592in}}%
\pgfpathclose%
\pgfusepath{stroke,fill}%
\end{pgfscope}%
\begin{pgfscope}%
\pgfpathrectangle{\pgfqpoint{0.100000in}{0.220728in}}{\pgfqpoint{3.696000in}{3.696000in}}%
\pgfusepath{clip}%
\pgfsetbuttcap%
\pgfsetroundjoin%
\definecolor{currentfill}{rgb}{0.121569,0.466667,0.705882}%
\pgfsetfillcolor{currentfill}%
\pgfsetfillopacity{0.452193}%
\pgfsetlinewidth{1.003750pt}%
\definecolor{currentstroke}{rgb}{0.121569,0.466667,0.705882}%
\pgfsetstrokecolor{currentstroke}%
\pgfsetstrokeopacity{0.452193}%
\pgfsetdash{}{0pt}%
\pgfpathmoveto{\pgfqpoint{2.429387in}{3.135773in}}%
\pgfpathcurveto{\pgfqpoint{2.437623in}{3.135773in}}{\pgfqpoint{2.445523in}{3.139045in}}{\pgfqpoint{2.451347in}{3.144869in}}%
\pgfpathcurveto{\pgfqpoint{2.457171in}{3.150693in}}{\pgfqpoint{2.460443in}{3.158593in}}{\pgfqpoint{2.460443in}{3.166830in}}%
\pgfpathcurveto{\pgfqpoint{2.460443in}{3.175066in}}{\pgfqpoint{2.457171in}{3.182966in}}{\pgfqpoint{2.451347in}{3.188790in}}%
\pgfpathcurveto{\pgfqpoint{2.445523in}{3.194614in}}{\pgfqpoint{2.437623in}{3.197886in}}{\pgfqpoint{2.429387in}{3.197886in}}%
\pgfpathcurveto{\pgfqpoint{2.421151in}{3.197886in}}{\pgfqpoint{2.413251in}{3.194614in}}{\pgfqpoint{2.407427in}{3.188790in}}%
\pgfpathcurveto{\pgfqpoint{2.401603in}{3.182966in}}{\pgfqpoint{2.398330in}{3.175066in}}{\pgfqpoint{2.398330in}{3.166830in}}%
\pgfpathcurveto{\pgfqpoint{2.398330in}{3.158593in}}{\pgfqpoint{2.401603in}{3.150693in}}{\pgfqpoint{2.407427in}{3.144869in}}%
\pgfpathcurveto{\pgfqpoint{2.413251in}{3.139045in}}{\pgfqpoint{2.421151in}{3.135773in}}{\pgfqpoint{2.429387in}{3.135773in}}%
\pgfpathclose%
\pgfusepath{stroke,fill}%
\end{pgfscope}%
\begin{pgfscope}%
\pgfpathrectangle{\pgfqpoint{0.100000in}{0.220728in}}{\pgfqpoint{3.696000in}{3.696000in}}%
\pgfusepath{clip}%
\pgfsetbuttcap%
\pgfsetroundjoin%
\definecolor{currentfill}{rgb}{0.121569,0.466667,0.705882}%
\pgfsetfillcolor{currentfill}%
\pgfsetfillopacity{0.454237}%
\pgfsetlinewidth{1.003750pt}%
\definecolor{currentstroke}{rgb}{0.121569,0.466667,0.705882}%
\pgfsetstrokecolor{currentstroke}%
\pgfsetstrokeopacity{0.454237}%
\pgfsetdash{}{0pt}%
\pgfpathmoveto{\pgfqpoint{2.436605in}{3.133930in}}%
\pgfpathcurveto{\pgfqpoint{2.444842in}{3.133930in}}{\pgfqpoint{2.452742in}{3.137202in}}{\pgfqpoint{2.458566in}{3.143026in}}%
\pgfpathcurveto{\pgfqpoint{2.464390in}{3.148850in}}{\pgfqpoint{2.467662in}{3.156750in}}{\pgfqpoint{2.467662in}{3.164987in}}%
\pgfpathcurveto{\pgfqpoint{2.467662in}{3.173223in}}{\pgfqpoint{2.464390in}{3.181123in}}{\pgfqpoint{2.458566in}{3.186947in}}%
\pgfpathcurveto{\pgfqpoint{2.452742in}{3.192771in}}{\pgfqpoint{2.444842in}{3.196043in}}{\pgfqpoint{2.436605in}{3.196043in}}%
\pgfpathcurveto{\pgfqpoint{2.428369in}{3.196043in}}{\pgfqpoint{2.420469in}{3.192771in}}{\pgfqpoint{2.414645in}{3.186947in}}%
\pgfpathcurveto{\pgfqpoint{2.408821in}{3.181123in}}{\pgfqpoint{2.405549in}{3.173223in}}{\pgfqpoint{2.405549in}{3.164987in}}%
\pgfpathcurveto{\pgfqpoint{2.405549in}{3.156750in}}{\pgfqpoint{2.408821in}{3.148850in}}{\pgfqpoint{2.414645in}{3.143026in}}%
\pgfpathcurveto{\pgfqpoint{2.420469in}{3.137202in}}{\pgfqpoint{2.428369in}{3.133930in}}{\pgfqpoint{2.436605in}{3.133930in}}%
\pgfpathclose%
\pgfusepath{stroke,fill}%
\end{pgfscope}%
\begin{pgfscope}%
\pgfpathrectangle{\pgfqpoint{0.100000in}{0.220728in}}{\pgfqpoint{3.696000in}{3.696000in}}%
\pgfusepath{clip}%
\pgfsetbuttcap%
\pgfsetroundjoin%
\definecolor{currentfill}{rgb}{0.121569,0.466667,0.705882}%
\pgfsetfillcolor{currentfill}%
\pgfsetfillopacity{0.454670}%
\pgfsetlinewidth{1.003750pt}%
\definecolor{currentstroke}{rgb}{0.121569,0.466667,0.705882}%
\pgfsetstrokecolor{currentstroke}%
\pgfsetstrokeopacity{0.454670}%
\pgfsetdash{}{0pt}%
\pgfpathmoveto{\pgfqpoint{1.334143in}{2.273745in}}%
\pgfpathcurveto{\pgfqpoint{1.342379in}{2.273745in}}{\pgfqpoint{1.350279in}{2.277017in}}{\pgfqpoint{1.356103in}{2.282841in}}%
\pgfpathcurveto{\pgfqpoint{1.361927in}{2.288665in}}{\pgfqpoint{1.365199in}{2.296565in}}{\pgfqpoint{1.365199in}{2.304802in}}%
\pgfpathcurveto{\pgfqpoint{1.365199in}{2.313038in}}{\pgfqpoint{1.361927in}{2.320938in}}{\pgfqpoint{1.356103in}{2.326762in}}%
\pgfpathcurveto{\pgfqpoint{1.350279in}{2.332586in}}{\pgfqpoint{1.342379in}{2.335858in}}{\pgfqpoint{1.334143in}{2.335858in}}%
\pgfpathcurveto{\pgfqpoint{1.325907in}{2.335858in}}{\pgfqpoint{1.318007in}{2.332586in}}{\pgfqpoint{1.312183in}{2.326762in}}%
\pgfpathcurveto{\pgfqpoint{1.306359in}{2.320938in}}{\pgfqpoint{1.303086in}{2.313038in}}{\pgfqpoint{1.303086in}{2.304802in}}%
\pgfpathcurveto{\pgfqpoint{1.303086in}{2.296565in}}{\pgfqpoint{1.306359in}{2.288665in}}{\pgfqpoint{1.312183in}{2.282841in}}%
\pgfpathcurveto{\pgfqpoint{1.318007in}{2.277017in}}{\pgfqpoint{1.325907in}{2.273745in}}{\pgfqpoint{1.334143in}{2.273745in}}%
\pgfpathclose%
\pgfusepath{stroke,fill}%
\end{pgfscope}%
\begin{pgfscope}%
\pgfpathrectangle{\pgfqpoint{0.100000in}{0.220728in}}{\pgfqpoint{3.696000in}{3.696000in}}%
\pgfusepath{clip}%
\pgfsetbuttcap%
\pgfsetroundjoin%
\definecolor{currentfill}{rgb}{0.121569,0.466667,0.705882}%
\pgfsetfillcolor{currentfill}%
\pgfsetfillopacity{0.455009}%
\pgfsetlinewidth{1.003750pt}%
\definecolor{currentstroke}{rgb}{0.121569,0.466667,0.705882}%
\pgfsetstrokecolor{currentstroke}%
\pgfsetstrokeopacity{0.455009}%
\pgfsetdash{}{0pt}%
\pgfpathmoveto{\pgfqpoint{2.440924in}{3.132534in}}%
\pgfpathcurveto{\pgfqpoint{2.449160in}{3.132534in}}{\pgfqpoint{2.457060in}{3.135807in}}{\pgfqpoint{2.462884in}{3.141631in}}%
\pgfpathcurveto{\pgfqpoint{2.468708in}{3.147455in}}{\pgfqpoint{2.471980in}{3.155355in}}{\pgfqpoint{2.471980in}{3.163591in}}%
\pgfpathcurveto{\pgfqpoint{2.471980in}{3.171827in}}{\pgfqpoint{2.468708in}{3.179727in}}{\pgfqpoint{2.462884in}{3.185551in}}%
\pgfpathcurveto{\pgfqpoint{2.457060in}{3.191375in}}{\pgfqpoint{2.449160in}{3.194647in}}{\pgfqpoint{2.440924in}{3.194647in}}%
\pgfpathcurveto{\pgfqpoint{2.432687in}{3.194647in}}{\pgfqpoint{2.424787in}{3.191375in}}{\pgfqpoint{2.418963in}{3.185551in}}%
\pgfpathcurveto{\pgfqpoint{2.413139in}{3.179727in}}{\pgfqpoint{2.409867in}{3.171827in}}{\pgfqpoint{2.409867in}{3.163591in}}%
\pgfpathcurveto{\pgfqpoint{2.409867in}{3.155355in}}{\pgfqpoint{2.413139in}{3.147455in}}{\pgfqpoint{2.418963in}{3.141631in}}%
\pgfpathcurveto{\pgfqpoint{2.424787in}{3.135807in}}{\pgfqpoint{2.432687in}{3.132534in}}{\pgfqpoint{2.440924in}{3.132534in}}%
\pgfpathclose%
\pgfusepath{stroke,fill}%
\end{pgfscope}%
\begin{pgfscope}%
\pgfpathrectangle{\pgfqpoint{0.100000in}{0.220728in}}{\pgfqpoint{3.696000in}{3.696000in}}%
\pgfusepath{clip}%
\pgfsetbuttcap%
\pgfsetroundjoin%
\definecolor{currentfill}{rgb}{0.121569,0.466667,0.705882}%
\pgfsetfillcolor{currentfill}%
\pgfsetfillopacity{0.456410}%
\pgfsetlinewidth{1.003750pt}%
\definecolor{currentstroke}{rgb}{0.121569,0.466667,0.705882}%
\pgfsetstrokecolor{currentstroke}%
\pgfsetstrokeopacity{0.456410}%
\pgfsetdash{}{0pt}%
\pgfpathmoveto{\pgfqpoint{1.325486in}{2.261797in}}%
\pgfpathcurveto{\pgfqpoint{1.333722in}{2.261797in}}{\pgfqpoint{1.341622in}{2.265070in}}{\pgfqpoint{1.347446in}{2.270894in}}%
\pgfpathcurveto{\pgfqpoint{1.353270in}{2.276718in}}{\pgfqpoint{1.356543in}{2.284618in}}{\pgfqpoint{1.356543in}{2.292854in}}%
\pgfpathcurveto{\pgfqpoint{1.356543in}{2.301090in}}{\pgfqpoint{1.353270in}{2.308990in}}{\pgfqpoint{1.347446in}{2.314814in}}%
\pgfpathcurveto{\pgfqpoint{1.341622in}{2.320638in}}{\pgfqpoint{1.333722in}{2.323910in}}{\pgfqpoint{1.325486in}{2.323910in}}%
\pgfpathcurveto{\pgfqpoint{1.317250in}{2.323910in}}{\pgfqpoint{1.309350in}{2.320638in}}{\pgfqpoint{1.303526in}{2.314814in}}%
\pgfpathcurveto{\pgfqpoint{1.297702in}{2.308990in}}{\pgfqpoint{1.294430in}{2.301090in}}{\pgfqpoint{1.294430in}{2.292854in}}%
\pgfpathcurveto{\pgfqpoint{1.294430in}{2.284618in}}{\pgfqpoint{1.297702in}{2.276718in}}{\pgfqpoint{1.303526in}{2.270894in}}%
\pgfpathcurveto{\pgfqpoint{1.309350in}{2.265070in}}{\pgfqpoint{1.317250in}{2.261797in}}{\pgfqpoint{1.325486in}{2.261797in}}%
\pgfpathclose%
\pgfusepath{stroke,fill}%
\end{pgfscope}%
\begin{pgfscope}%
\pgfpathrectangle{\pgfqpoint{0.100000in}{0.220728in}}{\pgfqpoint{3.696000in}{3.696000in}}%
\pgfusepath{clip}%
\pgfsetbuttcap%
\pgfsetroundjoin%
\definecolor{currentfill}{rgb}{0.121569,0.466667,0.705882}%
\pgfsetfillcolor{currentfill}%
\pgfsetfillopacity{0.456715}%
\pgfsetlinewidth{1.003750pt}%
\definecolor{currentstroke}{rgb}{0.121569,0.466667,0.705882}%
\pgfsetstrokecolor{currentstroke}%
\pgfsetstrokeopacity{0.456715}%
\pgfsetdash{}{0pt}%
\pgfpathmoveto{\pgfqpoint{2.445808in}{3.131834in}}%
\pgfpathcurveto{\pgfqpoint{2.454044in}{3.131834in}}{\pgfqpoint{2.461944in}{3.135107in}}{\pgfqpoint{2.467768in}{3.140930in}}%
\pgfpathcurveto{\pgfqpoint{2.473592in}{3.146754in}}{\pgfqpoint{2.476864in}{3.154654in}}{\pgfqpoint{2.476864in}{3.162891in}}%
\pgfpathcurveto{\pgfqpoint{2.476864in}{3.171127in}}{\pgfqpoint{2.473592in}{3.179027in}}{\pgfqpoint{2.467768in}{3.184851in}}%
\pgfpathcurveto{\pgfqpoint{2.461944in}{3.190675in}}{\pgfqpoint{2.454044in}{3.193947in}}{\pgfqpoint{2.445808in}{3.193947in}}%
\pgfpathcurveto{\pgfqpoint{2.437571in}{3.193947in}}{\pgfqpoint{2.429671in}{3.190675in}}{\pgfqpoint{2.423847in}{3.184851in}}%
\pgfpathcurveto{\pgfqpoint{2.418023in}{3.179027in}}{\pgfqpoint{2.414751in}{3.171127in}}{\pgfqpoint{2.414751in}{3.162891in}}%
\pgfpathcurveto{\pgfqpoint{2.414751in}{3.154654in}}{\pgfqpoint{2.418023in}{3.146754in}}{\pgfqpoint{2.423847in}{3.140930in}}%
\pgfpathcurveto{\pgfqpoint{2.429671in}{3.135107in}}{\pgfqpoint{2.437571in}{3.131834in}}{\pgfqpoint{2.445808in}{3.131834in}}%
\pgfpathclose%
\pgfusepath{stroke,fill}%
\end{pgfscope}%
\begin{pgfscope}%
\pgfpathrectangle{\pgfqpoint{0.100000in}{0.220728in}}{\pgfqpoint{3.696000in}{3.696000in}}%
\pgfusepath{clip}%
\pgfsetbuttcap%
\pgfsetroundjoin%
\definecolor{currentfill}{rgb}{0.121569,0.466667,0.705882}%
\pgfsetfillcolor{currentfill}%
\pgfsetfillopacity{0.457278}%
\pgfsetlinewidth{1.003750pt}%
\definecolor{currentstroke}{rgb}{0.121569,0.466667,0.705882}%
\pgfsetstrokecolor{currentstroke}%
\pgfsetstrokeopacity{0.457278}%
\pgfsetdash{}{0pt}%
\pgfpathmoveto{\pgfqpoint{2.448801in}{3.130775in}}%
\pgfpathcurveto{\pgfqpoint{2.457037in}{3.130775in}}{\pgfqpoint{2.464937in}{3.134048in}}{\pgfqpoint{2.470761in}{3.139872in}}%
\pgfpathcurveto{\pgfqpoint{2.476585in}{3.145696in}}{\pgfqpoint{2.479857in}{3.153596in}}{\pgfqpoint{2.479857in}{3.161832in}}%
\pgfpathcurveto{\pgfqpoint{2.479857in}{3.170068in}}{\pgfqpoint{2.476585in}{3.177968in}}{\pgfqpoint{2.470761in}{3.183792in}}%
\pgfpathcurveto{\pgfqpoint{2.464937in}{3.189616in}}{\pgfqpoint{2.457037in}{3.192888in}}{\pgfqpoint{2.448801in}{3.192888in}}%
\pgfpathcurveto{\pgfqpoint{2.440565in}{3.192888in}}{\pgfqpoint{2.432665in}{3.189616in}}{\pgfqpoint{2.426841in}{3.183792in}}%
\pgfpathcurveto{\pgfqpoint{2.421017in}{3.177968in}}{\pgfqpoint{2.417744in}{3.170068in}}{\pgfqpoint{2.417744in}{3.161832in}}%
\pgfpathcurveto{\pgfqpoint{2.417744in}{3.153596in}}{\pgfqpoint{2.421017in}{3.145696in}}{\pgfqpoint{2.426841in}{3.139872in}}%
\pgfpathcurveto{\pgfqpoint{2.432665in}{3.134048in}}{\pgfqpoint{2.440565in}{3.130775in}}{\pgfqpoint{2.448801in}{3.130775in}}%
\pgfpathclose%
\pgfusepath{stroke,fill}%
\end{pgfscope}%
\begin{pgfscope}%
\pgfpathrectangle{\pgfqpoint{0.100000in}{0.220728in}}{\pgfqpoint{3.696000in}{3.696000in}}%
\pgfusepath{clip}%
\pgfsetbuttcap%
\pgfsetroundjoin%
\definecolor{currentfill}{rgb}{0.121569,0.466667,0.705882}%
\pgfsetfillcolor{currentfill}%
\pgfsetfillopacity{0.457714}%
\pgfsetlinewidth{1.003750pt}%
\definecolor{currentstroke}{rgb}{0.121569,0.466667,0.705882}%
\pgfsetstrokecolor{currentstroke}%
\pgfsetstrokeopacity{0.457714}%
\pgfsetdash{}{0pt}%
\pgfpathmoveto{\pgfqpoint{2.450350in}{3.130417in}}%
\pgfpathcurveto{\pgfqpoint{2.458586in}{3.130417in}}{\pgfqpoint{2.466486in}{3.133689in}}{\pgfqpoint{2.472310in}{3.139513in}}%
\pgfpathcurveto{\pgfqpoint{2.478134in}{3.145337in}}{\pgfqpoint{2.481406in}{3.153237in}}{\pgfqpoint{2.481406in}{3.161473in}}%
\pgfpathcurveto{\pgfqpoint{2.481406in}{3.169710in}}{\pgfqpoint{2.478134in}{3.177610in}}{\pgfqpoint{2.472310in}{3.183434in}}%
\pgfpathcurveto{\pgfqpoint{2.466486in}{3.189258in}}{\pgfqpoint{2.458586in}{3.192530in}}{\pgfqpoint{2.450350in}{3.192530in}}%
\pgfpathcurveto{\pgfqpoint{2.442114in}{3.192530in}}{\pgfqpoint{2.434213in}{3.189258in}}{\pgfqpoint{2.428390in}{3.183434in}}%
\pgfpathcurveto{\pgfqpoint{2.422566in}{3.177610in}}{\pgfqpoint{2.419293in}{3.169710in}}{\pgfqpoint{2.419293in}{3.161473in}}%
\pgfpathcurveto{\pgfqpoint{2.419293in}{3.153237in}}{\pgfqpoint{2.422566in}{3.145337in}}{\pgfqpoint{2.428390in}{3.139513in}}%
\pgfpathcurveto{\pgfqpoint{2.434213in}{3.133689in}}{\pgfqpoint{2.442114in}{3.130417in}}{\pgfqpoint{2.450350in}{3.130417in}}%
\pgfpathclose%
\pgfusepath{stroke,fill}%
\end{pgfscope}%
\begin{pgfscope}%
\pgfpathrectangle{\pgfqpoint{0.100000in}{0.220728in}}{\pgfqpoint{3.696000in}{3.696000in}}%
\pgfusepath{clip}%
\pgfsetbuttcap%
\pgfsetroundjoin%
\definecolor{currentfill}{rgb}{0.121569,0.466667,0.705882}%
\pgfsetfillcolor{currentfill}%
\pgfsetfillopacity{0.458574}%
\pgfsetlinewidth{1.003750pt}%
\definecolor{currentstroke}{rgb}{0.121569,0.466667,0.705882}%
\pgfsetstrokecolor{currentstroke}%
\pgfsetstrokeopacity{0.458574}%
\pgfsetdash{}{0pt}%
\pgfpathmoveto{\pgfqpoint{1.320295in}{2.249359in}}%
\pgfpathcurveto{\pgfqpoint{1.328531in}{2.249359in}}{\pgfqpoint{1.336431in}{2.252631in}}{\pgfqpoint{1.342255in}{2.258455in}}%
\pgfpathcurveto{\pgfqpoint{1.348079in}{2.264279in}}{\pgfqpoint{1.351351in}{2.272179in}}{\pgfqpoint{1.351351in}{2.280415in}}%
\pgfpathcurveto{\pgfqpoint{1.351351in}{2.288651in}}{\pgfqpoint{1.348079in}{2.296551in}}{\pgfqpoint{1.342255in}{2.302375in}}%
\pgfpathcurveto{\pgfqpoint{1.336431in}{2.308199in}}{\pgfqpoint{1.328531in}{2.311472in}}{\pgfqpoint{1.320295in}{2.311472in}}%
\pgfpathcurveto{\pgfqpoint{1.312058in}{2.311472in}}{\pgfqpoint{1.304158in}{2.308199in}}{\pgfqpoint{1.298334in}{2.302375in}}%
\pgfpathcurveto{\pgfqpoint{1.292511in}{2.296551in}}{\pgfqpoint{1.289238in}{2.288651in}}{\pgfqpoint{1.289238in}{2.280415in}}%
\pgfpathcurveto{\pgfqpoint{1.289238in}{2.272179in}}{\pgfqpoint{1.292511in}{2.264279in}}{\pgfqpoint{1.298334in}{2.258455in}}%
\pgfpathcurveto{\pgfqpoint{1.304158in}{2.252631in}}{\pgfqpoint{1.312058in}{2.249359in}}{\pgfqpoint{1.320295in}{2.249359in}}%
\pgfpathclose%
\pgfusepath{stroke,fill}%
\end{pgfscope}%
\begin{pgfscope}%
\pgfpathrectangle{\pgfqpoint{0.100000in}{0.220728in}}{\pgfqpoint{3.696000in}{3.696000in}}%
\pgfusepath{clip}%
\pgfsetbuttcap%
\pgfsetroundjoin%
\definecolor{currentfill}{rgb}{0.121569,0.466667,0.705882}%
\pgfsetfillcolor{currentfill}%
\pgfsetfillopacity{0.458784}%
\pgfsetlinewidth{1.003750pt}%
\definecolor{currentstroke}{rgb}{0.121569,0.466667,0.705882}%
\pgfsetstrokecolor{currentstroke}%
\pgfsetstrokeopacity{0.458784}%
\pgfsetdash{}{0pt}%
\pgfpathmoveto{\pgfqpoint{2.453294in}{3.129896in}}%
\pgfpathcurveto{\pgfqpoint{2.461530in}{3.129896in}}{\pgfqpoint{2.469430in}{3.133168in}}{\pgfqpoint{2.475254in}{3.138992in}}%
\pgfpathcurveto{\pgfqpoint{2.481078in}{3.144816in}}{\pgfqpoint{2.484351in}{3.152716in}}{\pgfqpoint{2.484351in}{3.160952in}}%
\pgfpathcurveto{\pgfqpoint{2.484351in}{3.169188in}}{\pgfqpoint{2.481078in}{3.177088in}}{\pgfqpoint{2.475254in}{3.182912in}}%
\pgfpathcurveto{\pgfqpoint{2.469430in}{3.188736in}}{\pgfqpoint{2.461530in}{3.192009in}}{\pgfqpoint{2.453294in}{3.192009in}}%
\pgfpathcurveto{\pgfqpoint{2.445058in}{3.192009in}}{\pgfqpoint{2.437158in}{3.188736in}}{\pgfqpoint{2.431334in}{3.182912in}}%
\pgfpathcurveto{\pgfqpoint{2.425510in}{3.177088in}}{\pgfqpoint{2.422238in}{3.169188in}}{\pgfqpoint{2.422238in}{3.160952in}}%
\pgfpathcurveto{\pgfqpoint{2.422238in}{3.152716in}}{\pgfqpoint{2.425510in}{3.144816in}}{\pgfqpoint{2.431334in}{3.138992in}}%
\pgfpathcurveto{\pgfqpoint{2.437158in}{3.133168in}}{\pgfqpoint{2.445058in}{3.129896in}}{\pgfqpoint{2.453294in}{3.129896in}}%
\pgfpathclose%
\pgfusepath{stroke,fill}%
\end{pgfscope}%
\begin{pgfscope}%
\pgfpathrectangle{\pgfqpoint{0.100000in}{0.220728in}}{\pgfqpoint{3.696000in}{3.696000in}}%
\pgfusepath{clip}%
\pgfsetbuttcap%
\pgfsetroundjoin%
\definecolor{currentfill}{rgb}{0.121569,0.466667,0.705882}%
\pgfsetfillcolor{currentfill}%
\pgfsetfillopacity{0.459173}%
\pgfsetlinewidth{1.003750pt}%
\definecolor{currentstroke}{rgb}{0.121569,0.466667,0.705882}%
\pgfsetstrokecolor{currentstroke}%
\pgfsetstrokeopacity{0.459173}%
\pgfsetdash{}{0pt}%
\pgfpathmoveto{\pgfqpoint{2.455110in}{3.129315in}}%
\pgfpathcurveto{\pgfqpoint{2.463347in}{3.129315in}}{\pgfqpoint{2.471247in}{3.132587in}}{\pgfqpoint{2.477071in}{3.138411in}}%
\pgfpathcurveto{\pgfqpoint{2.482895in}{3.144235in}}{\pgfqpoint{2.486167in}{3.152135in}}{\pgfqpoint{2.486167in}{3.160372in}}%
\pgfpathcurveto{\pgfqpoint{2.486167in}{3.168608in}}{\pgfqpoint{2.482895in}{3.176508in}}{\pgfqpoint{2.477071in}{3.182332in}}%
\pgfpathcurveto{\pgfqpoint{2.471247in}{3.188156in}}{\pgfqpoint{2.463347in}{3.191428in}}{\pgfqpoint{2.455110in}{3.191428in}}%
\pgfpathcurveto{\pgfqpoint{2.446874in}{3.191428in}}{\pgfqpoint{2.438974in}{3.188156in}}{\pgfqpoint{2.433150in}{3.182332in}}%
\pgfpathcurveto{\pgfqpoint{2.427326in}{3.176508in}}{\pgfqpoint{2.424054in}{3.168608in}}{\pgfqpoint{2.424054in}{3.160372in}}%
\pgfpathcurveto{\pgfqpoint{2.424054in}{3.152135in}}{\pgfqpoint{2.427326in}{3.144235in}}{\pgfqpoint{2.433150in}{3.138411in}}%
\pgfpathcurveto{\pgfqpoint{2.438974in}{3.132587in}}{\pgfqpoint{2.446874in}{3.129315in}}{\pgfqpoint{2.455110in}{3.129315in}}%
\pgfpathclose%
\pgfusepath{stroke,fill}%
\end{pgfscope}%
\begin{pgfscope}%
\pgfpathrectangle{\pgfqpoint{0.100000in}{0.220728in}}{\pgfqpoint{3.696000in}{3.696000in}}%
\pgfusepath{clip}%
\pgfsetbuttcap%
\pgfsetroundjoin%
\definecolor{currentfill}{rgb}{0.121569,0.466667,0.705882}%
\pgfsetfillcolor{currentfill}%
\pgfsetfillopacity{0.459871}%
\pgfsetlinewidth{1.003750pt}%
\definecolor{currentstroke}{rgb}{0.121569,0.466667,0.705882}%
\pgfsetstrokecolor{currentstroke}%
\pgfsetstrokeopacity{0.459871}%
\pgfsetdash{}{0pt}%
\pgfpathmoveto{\pgfqpoint{2.457587in}{3.128478in}}%
\pgfpathcurveto{\pgfqpoint{2.465823in}{3.128478in}}{\pgfqpoint{2.473723in}{3.131750in}}{\pgfqpoint{2.479547in}{3.137574in}}%
\pgfpathcurveto{\pgfqpoint{2.485371in}{3.143398in}}{\pgfqpoint{2.488643in}{3.151298in}}{\pgfqpoint{2.488643in}{3.159534in}}%
\pgfpathcurveto{\pgfqpoint{2.488643in}{3.167770in}}{\pgfqpoint{2.485371in}{3.175670in}}{\pgfqpoint{2.479547in}{3.181494in}}%
\pgfpathcurveto{\pgfqpoint{2.473723in}{3.187318in}}{\pgfqpoint{2.465823in}{3.190591in}}{\pgfqpoint{2.457587in}{3.190591in}}%
\pgfpathcurveto{\pgfqpoint{2.449350in}{3.190591in}}{\pgfqpoint{2.441450in}{3.187318in}}{\pgfqpoint{2.435626in}{3.181494in}}%
\pgfpathcurveto{\pgfqpoint{2.429802in}{3.175670in}}{\pgfqpoint{2.426530in}{3.167770in}}{\pgfqpoint{2.426530in}{3.159534in}}%
\pgfpathcurveto{\pgfqpoint{2.426530in}{3.151298in}}{\pgfqpoint{2.429802in}{3.143398in}}{\pgfqpoint{2.435626in}{3.137574in}}%
\pgfpathcurveto{\pgfqpoint{2.441450in}{3.131750in}}{\pgfqpoint{2.449350in}{3.128478in}}{\pgfqpoint{2.457587in}{3.128478in}}%
\pgfpathclose%
\pgfusepath{stroke,fill}%
\end{pgfscope}%
\begin{pgfscope}%
\pgfpathrectangle{\pgfqpoint{0.100000in}{0.220728in}}{\pgfqpoint{3.696000in}{3.696000in}}%
\pgfusepath{clip}%
\pgfsetbuttcap%
\pgfsetroundjoin%
\definecolor{currentfill}{rgb}{0.121569,0.466667,0.705882}%
\pgfsetfillcolor{currentfill}%
\pgfsetfillopacity{0.459969}%
\pgfsetlinewidth{1.003750pt}%
\definecolor{currentstroke}{rgb}{0.121569,0.466667,0.705882}%
\pgfsetstrokecolor{currentstroke}%
\pgfsetstrokeopacity{0.459969}%
\pgfsetdash{}{0pt}%
\pgfpathmoveto{\pgfqpoint{2.459202in}{3.127740in}}%
\pgfpathcurveto{\pgfqpoint{2.467439in}{3.127740in}}{\pgfqpoint{2.475339in}{3.131012in}}{\pgfqpoint{2.481162in}{3.136836in}}%
\pgfpathcurveto{\pgfqpoint{2.486986in}{3.142660in}}{\pgfqpoint{2.490259in}{3.150560in}}{\pgfqpoint{2.490259in}{3.158796in}}%
\pgfpathcurveto{\pgfqpoint{2.490259in}{3.167032in}}{\pgfqpoint{2.486986in}{3.174933in}}{\pgfqpoint{2.481162in}{3.180756in}}%
\pgfpathcurveto{\pgfqpoint{2.475339in}{3.186580in}}{\pgfqpoint{2.467439in}{3.189853in}}{\pgfqpoint{2.459202in}{3.189853in}}%
\pgfpathcurveto{\pgfqpoint{2.450966in}{3.189853in}}{\pgfqpoint{2.443066in}{3.186580in}}{\pgfqpoint{2.437242in}{3.180756in}}%
\pgfpathcurveto{\pgfqpoint{2.431418in}{3.174933in}}{\pgfqpoint{2.428146in}{3.167032in}}{\pgfqpoint{2.428146in}{3.158796in}}%
\pgfpathcurveto{\pgfqpoint{2.428146in}{3.150560in}}{\pgfqpoint{2.431418in}{3.142660in}}{\pgfqpoint{2.437242in}{3.136836in}}%
\pgfpathcurveto{\pgfqpoint{2.443066in}{3.131012in}}{\pgfqpoint{2.450966in}{3.127740in}}{\pgfqpoint{2.459202in}{3.127740in}}%
\pgfpathclose%
\pgfusepath{stroke,fill}%
\end{pgfscope}%
\begin{pgfscope}%
\pgfpathrectangle{\pgfqpoint{0.100000in}{0.220728in}}{\pgfqpoint{3.696000in}{3.696000in}}%
\pgfusepath{clip}%
\pgfsetbuttcap%
\pgfsetroundjoin%
\definecolor{currentfill}{rgb}{0.121569,0.466667,0.705882}%
\pgfsetfillcolor{currentfill}%
\pgfsetfillopacity{0.460109}%
\pgfsetlinewidth{1.003750pt}%
\definecolor{currentstroke}{rgb}{0.121569,0.466667,0.705882}%
\pgfsetstrokecolor{currentstroke}%
\pgfsetstrokeopacity{0.460109}%
\pgfsetdash{}{0pt}%
\pgfpathmoveto{\pgfqpoint{2.460031in}{3.127436in}}%
\pgfpathcurveto{\pgfqpoint{2.468268in}{3.127436in}}{\pgfqpoint{2.476168in}{3.130709in}}{\pgfqpoint{2.481992in}{3.136532in}}%
\pgfpathcurveto{\pgfqpoint{2.487816in}{3.142356in}}{\pgfqpoint{2.491088in}{3.150256in}}{\pgfqpoint{2.491088in}{3.158493in}}%
\pgfpathcurveto{\pgfqpoint{2.491088in}{3.166729in}}{\pgfqpoint{2.487816in}{3.174629in}}{\pgfqpoint{2.481992in}{3.180453in}}%
\pgfpathcurveto{\pgfqpoint{2.476168in}{3.186277in}}{\pgfqpoint{2.468268in}{3.189549in}}{\pgfqpoint{2.460031in}{3.189549in}}%
\pgfpathcurveto{\pgfqpoint{2.451795in}{3.189549in}}{\pgfqpoint{2.443895in}{3.186277in}}{\pgfqpoint{2.438071in}{3.180453in}}%
\pgfpathcurveto{\pgfqpoint{2.432247in}{3.174629in}}{\pgfqpoint{2.428975in}{3.166729in}}{\pgfqpoint{2.428975in}{3.158493in}}%
\pgfpathcurveto{\pgfqpoint{2.428975in}{3.150256in}}{\pgfqpoint{2.432247in}{3.142356in}}{\pgfqpoint{2.438071in}{3.136532in}}%
\pgfpathcurveto{\pgfqpoint{2.443895in}{3.130709in}}{\pgfqpoint{2.451795in}{3.127436in}}{\pgfqpoint{2.460031in}{3.127436in}}%
\pgfpathclose%
\pgfusepath{stroke,fill}%
\end{pgfscope}%
\begin{pgfscope}%
\pgfpathrectangle{\pgfqpoint{0.100000in}{0.220728in}}{\pgfqpoint{3.696000in}{3.696000in}}%
\pgfusepath{clip}%
\pgfsetbuttcap%
\pgfsetroundjoin%
\definecolor{currentfill}{rgb}{0.121569,0.466667,0.705882}%
\pgfsetfillcolor{currentfill}%
\pgfsetfillopacity{0.460786}%
\pgfsetlinewidth{1.003750pt}%
\definecolor{currentstroke}{rgb}{0.121569,0.466667,0.705882}%
\pgfsetstrokecolor{currentstroke}%
\pgfsetstrokeopacity{0.460786}%
\pgfsetdash{}{0pt}%
\pgfpathmoveto{\pgfqpoint{2.462410in}{3.127402in}}%
\pgfpathcurveto{\pgfqpoint{2.470647in}{3.127402in}}{\pgfqpoint{2.478547in}{3.130674in}}{\pgfqpoint{2.484371in}{3.136498in}}%
\pgfpathcurveto{\pgfqpoint{2.490195in}{3.142322in}}{\pgfqpoint{2.493467in}{3.150222in}}{\pgfqpoint{2.493467in}{3.158458in}}%
\pgfpathcurveto{\pgfqpoint{2.493467in}{3.166695in}}{\pgfqpoint{2.490195in}{3.174595in}}{\pgfqpoint{2.484371in}{3.180419in}}%
\pgfpathcurveto{\pgfqpoint{2.478547in}{3.186243in}}{\pgfqpoint{2.470647in}{3.189515in}}{\pgfqpoint{2.462410in}{3.189515in}}%
\pgfpathcurveto{\pgfqpoint{2.454174in}{3.189515in}}{\pgfqpoint{2.446274in}{3.186243in}}{\pgfqpoint{2.440450in}{3.180419in}}%
\pgfpathcurveto{\pgfqpoint{2.434626in}{3.174595in}}{\pgfqpoint{2.431354in}{3.166695in}}{\pgfqpoint{2.431354in}{3.158458in}}%
\pgfpathcurveto{\pgfqpoint{2.431354in}{3.150222in}}{\pgfqpoint{2.434626in}{3.142322in}}{\pgfqpoint{2.440450in}{3.136498in}}%
\pgfpathcurveto{\pgfqpoint{2.446274in}{3.130674in}}{\pgfqpoint{2.454174in}{3.127402in}}{\pgfqpoint{2.462410in}{3.127402in}}%
\pgfpathclose%
\pgfusepath{stroke,fill}%
\end{pgfscope}%
\begin{pgfscope}%
\pgfpathrectangle{\pgfqpoint{0.100000in}{0.220728in}}{\pgfqpoint{3.696000in}{3.696000in}}%
\pgfusepath{clip}%
\pgfsetbuttcap%
\pgfsetroundjoin%
\definecolor{currentfill}{rgb}{0.121569,0.466667,0.705882}%
\pgfsetfillcolor{currentfill}%
\pgfsetfillopacity{0.461670}%
\pgfsetlinewidth{1.003750pt}%
\definecolor{currentstroke}{rgb}{0.121569,0.466667,0.705882}%
\pgfsetstrokecolor{currentstroke}%
\pgfsetstrokeopacity{0.461670}%
\pgfsetdash{}{0pt}%
\pgfpathmoveto{\pgfqpoint{2.467098in}{3.126622in}}%
\pgfpathcurveto{\pgfqpoint{2.475334in}{3.126622in}}{\pgfqpoint{2.483234in}{3.129895in}}{\pgfqpoint{2.489058in}{3.135718in}}%
\pgfpathcurveto{\pgfqpoint{2.494882in}{3.141542in}}{\pgfqpoint{2.498154in}{3.149442in}}{\pgfqpoint{2.498154in}{3.157679in}}%
\pgfpathcurveto{\pgfqpoint{2.498154in}{3.165915in}}{\pgfqpoint{2.494882in}{3.173815in}}{\pgfqpoint{2.489058in}{3.179639in}}%
\pgfpathcurveto{\pgfqpoint{2.483234in}{3.185463in}}{\pgfqpoint{2.475334in}{3.188735in}}{\pgfqpoint{2.467098in}{3.188735in}}%
\pgfpathcurveto{\pgfqpoint{2.458862in}{3.188735in}}{\pgfqpoint{2.450962in}{3.185463in}}{\pgfqpoint{2.445138in}{3.179639in}}%
\pgfpathcurveto{\pgfqpoint{2.439314in}{3.173815in}}{\pgfqpoint{2.436041in}{3.165915in}}{\pgfqpoint{2.436041in}{3.157679in}}%
\pgfpathcurveto{\pgfqpoint{2.436041in}{3.149442in}}{\pgfqpoint{2.439314in}{3.141542in}}{\pgfqpoint{2.445138in}{3.135718in}}%
\pgfpathcurveto{\pgfqpoint{2.450962in}{3.129895in}}{\pgfqpoint{2.458862in}{3.126622in}}{\pgfqpoint{2.467098in}{3.126622in}}%
\pgfpathclose%
\pgfusepath{stroke,fill}%
\end{pgfscope}%
\begin{pgfscope}%
\pgfpathrectangle{\pgfqpoint{0.100000in}{0.220728in}}{\pgfqpoint{3.696000in}{3.696000in}}%
\pgfusepath{clip}%
\pgfsetbuttcap%
\pgfsetroundjoin%
\definecolor{currentfill}{rgb}{0.121569,0.466667,0.705882}%
\pgfsetfillcolor{currentfill}%
\pgfsetfillopacity{0.462089}%
\pgfsetlinewidth{1.003750pt}%
\definecolor{currentstroke}{rgb}{0.121569,0.466667,0.705882}%
\pgfsetstrokecolor{currentstroke}%
\pgfsetstrokeopacity{0.462089}%
\pgfsetdash{}{0pt}%
\pgfpathmoveto{\pgfqpoint{1.308233in}{2.227850in}}%
\pgfpathcurveto{\pgfqpoint{1.316469in}{2.227850in}}{\pgfqpoint{1.324369in}{2.231122in}}{\pgfqpoint{1.330193in}{2.236946in}}%
\pgfpathcurveto{\pgfqpoint{1.336017in}{2.242770in}}{\pgfqpoint{1.339290in}{2.250670in}}{\pgfqpoint{1.339290in}{2.258906in}}%
\pgfpathcurveto{\pgfqpoint{1.339290in}{2.267142in}}{\pgfqpoint{1.336017in}{2.275043in}}{\pgfqpoint{1.330193in}{2.280866in}}%
\pgfpathcurveto{\pgfqpoint{1.324369in}{2.286690in}}{\pgfqpoint{1.316469in}{2.289963in}}{\pgfqpoint{1.308233in}{2.289963in}}%
\pgfpathcurveto{\pgfqpoint{1.299997in}{2.289963in}}{\pgfqpoint{1.292097in}{2.286690in}}{\pgfqpoint{1.286273in}{2.280866in}}%
\pgfpathcurveto{\pgfqpoint{1.280449in}{2.275043in}}{\pgfqpoint{1.277177in}{2.267142in}}{\pgfqpoint{1.277177in}{2.258906in}}%
\pgfpathcurveto{\pgfqpoint{1.277177in}{2.250670in}}{\pgfqpoint{1.280449in}{2.242770in}}{\pgfqpoint{1.286273in}{2.236946in}}%
\pgfpathcurveto{\pgfqpoint{1.292097in}{2.231122in}}{\pgfqpoint{1.299997in}{2.227850in}}{\pgfqpoint{1.308233in}{2.227850in}}%
\pgfpathclose%
\pgfusepath{stroke,fill}%
\end{pgfscope}%
\begin{pgfscope}%
\pgfpathrectangle{\pgfqpoint{0.100000in}{0.220728in}}{\pgfqpoint{3.696000in}{3.696000in}}%
\pgfusepath{clip}%
\pgfsetbuttcap%
\pgfsetroundjoin%
\definecolor{currentfill}{rgb}{0.121569,0.466667,0.705882}%
\pgfsetfillcolor{currentfill}%
\pgfsetfillopacity{0.463079}%
\pgfsetlinewidth{1.003750pt}%
\definecolor{currentstroke}{rgb}{0.121569,0.466667,0.705882}%
\pgfsetstrokecolor{currentstroke}%
\pgfsetstrokeopacity{0.463079}%
\pgfsetdash{}{0pt}%
\pgfpathmoveto{\pgfqpoint{2.471777in}{3.124873in}}%
\pgfpathcurveto{\pgfqpoint{2.480014in}{3.124873in}}{\pgfqpoint{2.487914in}{3.128145in}}{\pgfqpoint{2.493738in}{3.133969in}}%
\pgfpathcurveto{\pgfqpoint{2.499561in}{3.139793in}}{\pgfqpoint{2.502834in}{3.147693in}}{\pgfqpoint{2.502834in}{3.155930in}}%
\pgfpathcurveto{\pgfqpoint{2.502834in}{3.164166in}}{\pgfqpoint{2.499561in}{3.172066in}}{\pgfqpoint{2.493738in}{3.177890in}}%
\pgfpathcurveto{\pgfqpoint{2.487914in}{3.183714in}}{\pgfqpoint{2.480014in}{3.186986in}}{\pgfqpoint{2.471777in}{3.186986in}}%
\pgfpathcurveto{\pgfqpoint{2.463541in}{3.186986in}}{\pgfqpoint{2.455641in}{3.183714in}}{\pgfqpoint{2.449817in}{3.177890in}}%
\pgfpathcurveto{\pgfqpoint{2.443993in}{3.172066in}}{\pgfqpoint{2.440721in}{3.164166in}}{\pgfqpoint{2.440721in}{3.155930in}}%
\pgfpathcurveto{\pgfqpoint{2.440721in}{3.147693in}}{\pgfqpoint{2.443993in}{3.139793in}}{\pgfqpoint{2.449817in}{3.133969in}}%
\pgfpathcurveto{\pgfqpoint{2.455641in}{3.128145in}}{\pgfqpoint{2.463541in}{3.124873in}}{\pgfqpoint{2.471777in}{3.124873in}}%
\pgfpathclose%
\pgfusepath{stroke,fill}%
\end{pgfscope}%
\begin{pgfscope}%
\pgfpathrectangle{\pgfqpoint{0.100000in}{0.220728in}}{\pgfqpoint{3.696000in}{3.696000in}}%
\pgfusepath{clip}%
\pgfsetbuttcap%
\pgfsetroundjoin%
\definecolor{currentfill}{rgb}{0.121569,0.466667,0.705882}%
\pgfsetfillcolor{currentfill}%
\pgfsetfillopacity{0.464306}%
\pgfsetlinewidth{1.003750pt}%
\definecolor{currentstroke}{rgb}{0.121569,0.466667,0.705882}%
\pgfsetstrokecolor{currentstroke}%
\pgfsetstrokeopacity{0.464306}%
\pgfsetdash{}{0pt}%
\pgfpathmoveto{\pgfqpoint{2.477389in}{3.122694in}}%
\pgfpathcurveto{\pgfqpoint{2.485625in}{3.122694in}}{\pgfqpoint{2.493525in}{3.125966in}}{\pgfqpoint{2.499349in}{3.131790in}}%
\pgfpathcurveto{\pgfqpoint{2.505173in}{3.137614in}}{\pgfqpoint{2.508445in}{3.145514in}}{\pgfqpoint{2.508445in}{3.153750in}}%
\pgfpathcurveto{\pgfqpoint{2.508445in}{3.161986in}}{\pgfqpoint{2.505173in}{3.169886in}}{\pgfqpoint{2.499349in}{3.175710in}}%
\pgfpathcurveto{\pgfqpoint{2.493525in}{3.181534in}}{\pgfqpoint{2.485625in}{3.184807in}}{\pgfqpoint{2.477389in}{3.184807in}}%
\pgfpathcurveto{\pgfqpoint{2.469152in}{3.184807in}}{\pgfqpoint{2.461252in}{3.181534in}}{\pgfqpoint{2.455428in}{3.175710in}}%
\pgfpathcurveto{\pgfqpoint{2.449604in}{3.169886in}}{\pgfqpoint{2.446332in}{3.161986in}}{\pgfqpoint{2.446332in}{3.153750in}}%
\pgfpathcurveto{\pgfqpoint{2.446332in}{3.145514in}}{\pgfqpoint{2.449604in}{3.137614in}}{\pgfqpoint{2.455428in}{3.131790in}}%
\pgfpathcurveto{\pgfqpoint{2.461252in}{3.125966in}}{\pgfqpoint{2.469152in}{3.122694in}}{\pgfqpoint{2.477389in}{3.122694in}}%
\pgfpathclose%
\pgfusepath{stroke,fill}%
\end{pgfscope}%
\begin{pgfscope}%
\pgfpathrectangle{\pgfqpoint{0.100000in}{0.220728in}}{\pgfqpoint{3.696000in}{3.696000in}}%
\pgfusepath{clip}%
\pgfsetbuttcap%
\pgfsetroundjoin%
\definecolor{currentfill}{rgb}{0.121569,0.466667,0.705882}%
\pgfsetfillcolor{currentfill}%
\pgfsetfillopacity{0.465188}%
\pgfsetlinewidth{1.003750pt}%
\definecolor{currentstroke}{rgb}{0.121569,0.466667,0.705882}%
\pgfsetstrokecolor{currentstroke}%
\pgfsetstrokeopacity{0.465188}%
\pgfsetdash{}{0pt}%
\pgfpathmoveto{\pgfqpoint{1.295830in}{2.208006in}}%
\pgfpathcurveto{\pgfqpoint{1.304067in}{2.208006in}}{\pgfqpoint{1.311967in}{2.211279in}}{\pgfqpoint{1.317791in}{2.217103in}}%
\pgfpathcurveto{\pgfqpoint{1.323615in}{2.222926in}}{\pgfqpoint{1.326887in}{2.230827in}}{\pgfqpoint{1.326887in}{2.239063in}}%
\pgfpathcurveto{\pgfqpoint{1.326887in}{2.247299in}}{\pgfqpoint{1.323615in}{2.255199in}}{\pgfqpoint{1.317791in}{2.261023in}}%
\pgfpathcurveto{\pgfqpoint{1.311967in}{2.266847in}}{\pgfqpoint{1.304067in}{2.270119in}}{\pgfqpoint{1.295830in}{2.270119in}}%
\pgfpathcurveto{\pgfqpoint{1.287594in}{2.270119in}}{\pgfqpoint{1.279694in}{2.266847in}}{\pgfqpoint{1.273870in}{2.261023in}}%
\pgfpathcurveto{\pgfqpoint{1.268046in}{2.255199in}}{\pgfqpoint{1.264774in}{2.247299in}}{\pgfqpoint{1.264774in}{2.239063in}}%
\pgfpathcurveto{\pgfqpoint{1.264774in}{2.230827in}}{\pgfqpoint{1.268046in}{2.222926in}}{\pgfqpoint{1.273870in}{2.217103in}}%
\pgfpathcurveto{\pgfqpoint{1.279694in}{2.211279in}}{\pgfqpoint{1.287594in}{2.208006in}}{\pgfqpoint{1.295830in}{2.208006in}}%
\pgfpathclose%
\pgfusepath{stroke,fill}%
\end{pgfscope}%
\begin{pgfscope}%
\pgfpathrectangle{\pgfqpoint{0.100000in}{0.220728in}}{\pgfqpoint{3.696000in}{3.696000in}}%
\pgfusepath{clip}%
\pgfsetbuttcap%
\pgfsetroundjoin%
\definecolor{currentfill}{rgb}{0.121569,0.466667,0.705882}%
\pgfsetfillcolor{currentfill}%
\pgfsetfillopacity{0.465349}%
\pgfsetlinewidth{1.003750pt}%
\definecolor{currentstroke}{rgb}{0.121569,0.466667,0.705882}%
\pgfsetstrokecolor{currentstroke}%
\pgfsetstrokeopacity{0.465349}%
\pgfsetdash{}{0pt}%
\pgfpathmoveto{\pgfqpoint{2.483870in}{3.118879in}}%
\pgfpathcurveto{\pgfqpoint{2.492106in}{3.118879in}}{\pgfqpoint{2.500006in}{3.122151in}}{\pgfqpoint{2.505830in}{3.127975in}}%
\pgfpathcurveto{\pgfqpoint{2.511654in}{3.133799in}}{\pgfqpoint{2.514926in}{3.141699in}}{\pgfqpoint{2.514926in}{3.149935in}}%
\pgfpathcurveto{\pgfqpoint{2.514926in}{3.158171in}}{\pgfqpoint{2.511654in}{3.166071in}}{\pgfqpoint{2.505830in}{3.171895in}}%
\pgfpathcurveto{\pgfqpoint{2.500006in}{3.177719in}}{\pgfqpoint{2.492106in}{3.180992in}}{\pgfqpoint{2.483870in}{3.180992in}}%
\pgfpathcurveto{\pgfqpoint{2.475633in}{3.180992in}}{\pgfqpoint{2.467733in}{3.177719in}}{\pgfqpoint{2.461909in}{3.171895in}}%
\pgfpathcurveto{\pgfqpoint{2.456085in}{3.166071in}}{\pgfqpoint{2.452813in}{3.158171in}}{\pgfqpoint{2.452813in}{3.149935in}}%
\pgfpathcurveto{\pgfqpoint{2.452813in}{3.141699in}}{\pgfqpoint{2.456085in}{3.133799in}}{\pgfqpoint{2.461909in}{3.127975in}}%
\pgfpathcurveto{\pgfqpoint{2.467733in}{3.122151in}}{\pgfqpoint{2.475633in}{3.118879in}}{\pgfqpoint{2.483870in}{3.118879in}}%
\pgfpathclose%
\pgfusepath{stroke,fill}%
\end{pgfscope}%
\begin{pgfscope}%
\pgfpathrectangle{\pgfqpoint{0.100000in}{0.220728in}}{\pgfqpoint{3.696000in}{3.696000in}}%
\pgfusepath{clip}%
\pgfsetbuttcap%
\pgfsetroundjoin%
\definecolor{currentfill}{rgb}{0.121569,0.466667,0.705882}%
\pgfsetfillcolor{currentfill}%
\pgfsetfillopacity{0.466558}%
\pgfsetlinewidth{1.003750pt}%
\definecolor{currentstroke}{rgb}{0.121569,0.466667,0.705882}%
\pgfsetstrokecolor{currentstroke}%
\pgfsetstrokeopacity{0.466558}%
\pgfsetdash{}{0pt}%
\pgfpathmoveto{\pgfqpoint{2.487084in}{3.118468in}}%
\pgfpathcurveto{\pgfqpoint{2.495321in}{3.118468in}}{\pgfqpoint{2.503221in}{3.121740in}}{\pgfqpoint{2.509045in}{3.127564in}}%
\pgfpathcurveto{\pgfqpoint{2.514869in}{3.133388in}}{\pgfqpoint{2.518141in}{3.141288in}}{\pgfqpoint{2.518141in}{3.149524in}}%
\pgfpathcurveto{\pgfqpoint{2.518141in}{3.157760in}}{\pgfqpoint{2.514869in}{3.165661in}}{\pgfqpoint{2.509045in}{3.171484in}}%
\pgfpathcurveto{\pgfqpoint{2.503221in}{3.177308in}}{\pgfqpoint{2.495321in}{3.180581in}}{\pgfqpoint{2.487084in}{3.180581in}}%
\pgfpathcurveto{\pgfqpoint{2.478848in}{3.180581in}}{\pgfqpoint{2.470948in}{3.177308in}}{\pgfqpoint{2.465124in}{3.171484in}}%
\pgfpathcurveto{\pgfqpoint{2.459300in}{3.165661in}}{\pgfqpoint{2.456028in}{3.157760in}}{\pgfqpoint{2.456028in}{3.149524in}}%
\pgfpathcurveto{\pgfqpoint{2.456028in}{3.141288in}}{\pgfqpoint{2.459300in}{3.133388in}}{\pgfqpoint{2.465124in}{3.127564in}}%
\pgfpathcurveto{\pgfqpoint{2.470948in}{3.121740in}}{\pgfqpoint{2.478848in}{3.118468in}}{\pgfqpoint{2.487084in}{3.118468in}}%
\pgfpathclose%
\pgfusepath{stroke,fill}%
\end{pgfscope}%
\begin{pgfscope}%
\pgfpathrectangle{\pgfqpoint{0.100000in}{0.220728in}}{\pgfqpoint{3.696000in}{3.696000in}}%
\pgfusepath{clip}%
\pgfsetbuttcap%
\pgfsetroundjoin%
\definecolor{currentfill}{rgb}{0.121569,0.466667,0.705882}%
\pgfsetfillcolor{currentfill}%
\pgfsetfillopacity{0.467639}%
\pgfsetlinewidth{1.003750pt}%
\definecolor{currentstroke}{rgb}{0.121569,0.466667,0.705882}%
\pgfsetstrokecolor{currentstroke}%
\pgfsetstrokeopacity{0.467639}%
\pgfsetdash{}{0pt}%
\pgfpathmoveto{\pgfqpoint{2.491515in}{3.117957in}}%
\pgfpathcurveto{\pgfqpoint{2.499752in}{3.117957in}}{\pgfqpoint{2.507652in}{3.121229in}}{\pgfqpoint{2.513476in}{3.127053in}}%
\pgfpathcurveto{\pgfqpoint{2.519299in}{3.132877in}}{\pgfqpoint{2.522572in}{3.140777in}}{\pgfqpoint{2.522572in}{3.149014in}}%
\pgfpathcurveto{\pgfqpoint{2.522572in}{3.157250in}}{\pgfqpoint{2.519299in}{3.165150in}}{\pgfqpoint{2.513476in}{3.170974in}}%
\pgfpathcurveto{\pgfqpoint{2.507652in}{3.176798in}}{\pgfqpoint{2.499752in}{3.180070in}}{\pgfqpoint{2.491515in}{3.180070in}}%
\pgfpathcurveto{\pgfqpoint{2.483279in}{3.180070in}}{\pgfqpoint{2.475379in}{3.176798in}}{\pgfqpoint{2.469555in}{3.170974in}}%
\pgfpathcurveto{\pgfqpoint{2.463731in}{3.165150in}}{\pgfqpoint{2.460459in}{3.157250in}}{\pgfqpoint{2.460459in}{3.149014in}}%
\pgfpathcurveto{\pgfqpoint{2.460459in}{3.140777in}}{\pgfqpoint{2.463731in}{3.132877in}}{\pgfqpoint{2.469555in}{3.127053in}}%
\pgfpathcurveto{\pgfqpoint{2.475379in}{3.121229in}}{\pgfqpoint{2.483279in}{3.117957in}}{\pgfqpoint{2.491515in}{3.117957in}}%
\pgfpathclose%
\pgfusepath{stroke,fill}%
\end{pgfscope}%
\begin{pgfscope}%
\pgfpathrectangle{\pgfqpoint{0.100000in}{0.220728in}}{\pgfqpoint{3.696000in}{3.696000in}}%
\pgfusepath{clip}%
\pgfsetbuttcap%
\pgfsetroundjoin%
\definecolor{currentfill}{rgb}{0.121569,0.466667,0.705882}%
\pgfsetfillcolor{currentfill}%
\pgfsetfillopacity{0.469837}%
\pgfsetlinewidth{1.003750pt}%
\definecolor{currentstroke}{rgb}{0.121569,0.466667,0.705882}%
\pgfsetstrokecolor{currentstroke}%
\pgfsetstrokeopacity{0.469837}%
\pgfsetdash{}{0pt}%
\pgfpathmoveto{\pgfqpoint{2.495843in}{3.113426in}}%
\pgfpathcurveto{\pgfqpoint{2.504080in}{3.113426in}}{\pgfqpoint{2.511980in}{3.116699in}}{\pgfqpoint{2.517803in}{3.122523in}}%
\pgfpathcurveto{\pgfqpoint{2.523627in}{3.128346in}}{\pgfqpoint{2.526900in}{3.136247in}}{\pgfqpoint{2.526900in}{3.144483in}}%
\pgfpathcurveto{\pgfqpoint{2.526900in}{3.152719in}}{\pgfqpoint{2.523627in}{3.160619in}}{\pgfqpoint{2.517803in}{3.166443in}}%
\pgfpathcurveto{\pgfqpoint{2.511980in}{3.172267in}}{\pgfqpoint{2.504080in}{3.175539in}}{\pgfqpoint{2.495843in}{3.175539in}}%
\pgfpathcurveto{\pgfqpoint{2.487607in}{3.175539in}}{\pgfqpoint{2.479707in}{3.172267in}}{\pgfqpoint{2.473883in}{3.166443in}}%
\pgfpathcurveto{\pgfqpoint{2.468059in}{3.160619in}}{\pgfqpoint{2.464787in}{3.152719in}}{\pgfqpoint{2.464787in}{3.144483in}}%
\pgfpathcurveto{\pgfqpoint{2.464787in}{3.136247in}}{\pgfqpoint{2.468059in}{3.128346in}}{\pgfqpoint{2.473883in}{3.122523in}}%
\pgfpathcurveto{\pgfqpoint{2.479707in}{3.116699in}}{\pgfqpoint{2.487607in}{3.113426in}}{\pgfqpoint{2.495843in}{3.113426in}}%
\pgfpathclose%
\pgfusepath{stroke,fill}%
\end{pgfscope}%
\begin{pgfscope}%
\pgfpathrectangle{\pgfqpoint{0.100000in}{0.220728in}}{\pgfqpoint{3.696000in}{3.696000in}}%
\pgfusepath{clip}%
\pgfsetbuttcap%
\pgfsetroundjoin%
\definecolor{currentfill}{rgb}{0.121569,0.466667,0.705882}%
\pgfsetfillcolor{currentfill}%
\pgfsetfillopacity{0.471806}%
\pgfsetlinewidth{1.003750pt}%
\definecolor{currentstroke}{rgb}{0.121569,0.466667,0.705882}%
\pgfsetstrokecolor{currentstroke}%
\pgfsetstrokeopacity{0.471806}%
\pgfsetdash{}{0pt}%
\pgfpathmoveto{\pgfqpoint{2.502892in}{3.111327in}}%
\pgfpathcurveto{\pgfqpoint{2.511128in}{3.111327in}}{\pgfqpoint{2.519028in}{3.114599in}}{\pgfqpoint{2.524852in}{3.120423in}}%
\pgfpathcurveto{\pgfqpoint{2.530676in}{3.126247in}}{\pgfqpoint{2.533948in}{3.134147in}}{\pgfqpoint{2.533948in}{3.142383in}}%
\pgfpathcurveto{\pgfqpoint{2.533948in}{3.150619in}}{\pgfqpoint{2.530676in}{3.158519in}}{\pgfqpoint{2.524852in}{3.164343in}}%
\pgfpathcurveto{\pgfqpoint{2.519028in}{3.170167in}}{\pgfqpoint{2.511128in}{3.173440in}}{\pgfqpoint{2.502892in}{3.173440in}}%
\pgfpathcurveto{\pgfqpoint{2.494655in}{3.173440in}}{\pgfqpoint{2.486755in}{3.170167in}}{\pgfqpoint{2.480931in}{3.164343in}}%
\pgfpathcurveto{\pgfqpoint{2.475108in}{3.158519in}}{\pgfqpoint{2.471835in}{3.150619in}}{\pgfqpoint{2.471835in}{3.142383in}}%
\pgfpathcurveto{\pgfqpoint{2.471835in}{3.134147in}}{\pgfqpoint{2.475108in}{3.126247in}}{\pgfqpoint{2.480931in}{3.120423in}}%
\pgfpathcurveto{\pgfqpoint{2.486755in}{3.114599in}}{\pgfqpoint{2.494655in}{3.111327in}}{\pgfqpoint{2.502892in}{3.111327in}}%
\pgfpathclose%
\pgfusepath{stroke,fill}%
\end{pgfscope}%
\begin{pgfscope}%
\pgfpathrectangle{\pgfqpoint{0.100000in}{0.220728in}}{\pgfqpoint{3.696000in}{3.696000in}}%
\pgfusepath{clip}%
\pgfsetbuttcap%
\pgfsetroundjoin%
\definecolor{currentfill}{rgb}{0.121569,0.466667,0.705882}%
\pgfsetfillcolor{currentfill}%
\pgfsetfillopacity{0.472001}%
\pgfsetlinewidth{1.003750pt}%
\definecolor{currentstroke}{rgb}{0.121569,0.466667,0.705882}%
\pgfsetstrokecolor{currentstroke}%
\pgfsetstrokeopacity{0.472001}%
\pgfsetdash{}{0pt}%
\pgfpathmoveto{\pgfqpoint{1.282704in}{2.166788in}}%
\pgfpathcurveto{\pgfqpoint{1.290940in}{2.166788in}}{\pgfqpoint{1.298841in}{2.170060in}}{\pgfqpoint{1.304664in}{2.175884in}}%
\pgfpathcurveto{\pgfqpoint{1.310488in}{2.181708in}}{\pgfqpoint{1.313761in}{2.189608in}}{\pgfqpoint{1.313761in}{2.197844in}}%
\pgfpathcurveto{\pgfqpoint{1.313761in}{2.206080in}}{\pgfqpoint{1.310488in}{2.213980in}}{\pgfqpoint{1.304664in}{2.219804in}}%
\pgfpathcurveto{\pgfqpoint{1.298841in}{2.225628in}}{\pgfqpoint{1.290940in}{2.228901in}}{\pgfqpoint{1.282704in}{2.228901in}}%
\pgfpathcurveto{\pgfqpoint{1.274468in}{2.228901in}}{\pgfqpoint{1.266568in}{2.225628in}}{\pgfqpoint{1.260744in}{2.219804in}}%
\pgfpathcurveto{\pgfqpoint{1.254920in}{2.213980in}}{\pgfqpoint{1.251648in}{2.206080in}}{\pgfqpoint{1.251648in}{2.197844in}}%
\pgfpathcurveto{\pgfqpoint{1.251648in}{2.189608in}}{\pgfqpoint{1.254920in}{2.181708in}}{\pgfqpoint{1.260744in}{2.175884in}}%
\pgfpathcurveto{\pgfqpoint{1.266568in}{2.170060in}}{\pgfqpoint{1.274468in}{2.166788in}}{\pgfqpoint{1.282704in}{2.166788in}}%
\pgfpathclose%
\pgfusepath{stroke,fill}%
\end{pgfscope}%
\begin{pgfscope}%
\pgfpathrectangle{\pgfqpoint{0.100000in}{0.220728in}}{\pgfqpoint{3.696000in}{3.696000in}}%
\pgfusepath{clip}%
\pgfsetbuttcap%
\pgfsetroundjoin%
\definecolor{currentfill}{rgb}{0.121569,0.466667,0.705882}%
\pgfsetfillcolor{currentfill}%
\pgfsetfillopacity{0.472588}%
\pgfsetlinewidth{1.003750pt}%
\definecolor{currentstroke}{rgb}{0.121569,0.466667,0.705882}%
\pgfsetstrokecolor{currentstroke}%
\pgfsetstrokeopacity{0.472588}%
\pgfsetdash{}{0pt}%
\pgfpathmoveto{\pgfqpoint{2.507119in}{3.110013in}}%
\pgfpathcurveto{\pgfqpoint{2.515356in}{3.110013in}}{\pgfqpoint{2.523256in}{3.113285in}}{\pgfqpoint{2.529080in}{3.119109in}}%
\pgfpathcurveto{\pgfqpoint{2.534904in}{3.124933in}}{\pgfqpoint{2.538176in}{3.132833in}}{\pgfqpoint{2.538176in}{3.141069in}}%
\pgfpathcurveto{\pgfqpoint{2.538176in}{3.149306in}}{\pgfqpoint{2.534904in}{3.157206in}}{\pgfqpoint{2.529080in}{3.163030in}}%
\pgfpathcurveto{\pgfqpoint{2.523256in}{3.168854in}}{\pgfqpoint{2.515356in}{3.172126in}}{\pgfqpoint{2.507119in}{3.172126in}}%
\pgfpathcurveto{\pgfqpoint{2.498883in}{3.172126in}}{\pgfqpoint{2.490983in}{3.168854in}}{\pgfqpoint{2.485159in}{3.163030in}}%
\pgfpathcurveto{\pgfqpoint{2.479335in}{3.157206in}}{\pgfqpoint{2.476063in}{3.149306in}}{\pgfqpoint{2.476063in}{3.141069in}}%
\pgfpathcurveto{\pgfqpoint{2.476063in}{3.132833in}}{\pgfqpoint{2.479335in}{3.124933in}}{\pgfqpoint{2.485159in}{3.119109in}}%
\pgfpathcurveto{\pgfqpoint{2.490983in}{3.113285in}}{\pgfqpoint{2.498883in}{3.110013in}}{\pgfqpoint{2.507119in}{3.110013in}}%
\pgfpathclose%
\pgfusepath{stroke,fill}%
\end{pgfscope}%
\begin{pgfscope}%
\pgfpathrectangle{\pgfqpoint{0.100000in}{0.220728in}}{\pgfqpoint{3.696000in}{3.696000in}}%
\pgfusepath{clip}%
\pgfsetbuttcap%
\pgfsetroundjoin%
\definecolor{currentfill}{rgb}{0.121569,0.466667,0.705882}%
\pgfsetfillcolor{currentfill}%
\pgfsetfillopacity{0.474222}%
\pgfsetlinewidth{1.003750pt}%
\definecolor{currentstroke}{rgb}{0.121569,0.466667,0.705882}%
\pgfsetstrokecolor{currentstroke}%
\pgfsetstrokeopacity{0.474222}%
\pgfsetdash{}{0pt}%
\pgfpathmoveto{\pgfqpoint{2.511651in}{3.109051in}}%
\pgfpathcurveto{\pgfqpoint{2.519887in}{3.109051in}}{\pgfqpoint{2.527787in}{3.112323in}}{\pgfqpoint{2.533611in}{3.118147in}}%
\pgfpathcurveto{\pgfqpoint{2.539435in}{3.123971in}}{\pgfqpoint{2.542707in}{3.131871in}}{\pgfqpoint{2.542707in}{3.140107in}}%
\pgfpathcurveto{\pgfqpoint{2.542707in}{3.148344in}}{\pgfqpoint{2.539435in}{3.156244in}}{\pgfqpoint{2.533611in}{3.162067in}}%
\pgfpathcurveto{\pgfqpoint{2.527787in}{3.167891in}}{\pgfqpoint{2.519887in}{3.171164in}}{\pgfqpoint{2.511651in}{3.171164in}}%
\pgfpathcurveto{\pgfqpoint{2.503414in}{3.171164in}}{\pgfqpoint{2.495514in}{3.167891in}}{\pgfqpoint{2.489690in}{3.162067in}}%
\pgfpathcurveto{\pgfqpoint{2.483866in}{3.156244in}}{\pgfqpoint{2.480594in}{3.148344in}}{\pgfqpoint{2.480594in}{3.140107in}}%
\pgfpathcurveto{\pgfqpoint{2.480594in}{3.131871in}}{\pgfqpoint{2.483866in}{3.123971in}}{\pgfqpoint{2.489690in}{3.118147in}}%
\pgfpathcurveto{\pgfqpoint{2.495514in}{3.112323in}}{\pgfqpoint{2.503414in}{3.109051in}}{\pgfqpoint{2.511651in}{3.109051in}}%
\pgfpathclose%
\pgfusepath{stroke,fill}%
\end{pgfscope}%
\begin{pgfscope}%
\pgfpathrectangle{\pgfqpoint{0.100000in}{0.220728in}}{\pgfqpoint{3.696000in}{3.696000in}}%
\pgfusepath{clip}%
\pgfsetbuttcap%
\pgfsetroundjoin%
\definecolor{currentfill}{rgb}{0.121569,0.466667,0.705882}%
\pgfsetfillcolor{currentfill}%
\pgfsetfillopacity{0.474655}%
\pgfsetlinewidth{1.003750pt}%
\definecolor{currentstroke}{rgb}{0.121569,0.466667,0.705882}%
\pgfsetstrokecolor{currentstroke}%
\pgfsetstrokeopacity{0.474655}%
\pgfsetdash{}{0pt}%
\pgfpathmoveto{\pgfqpoint{2.514571in}{3.107857in}}%
\pgfpathcurveto{\pgfqpoint{2.522807in}{3.107857in}}{\pgfqpoint{2.530707in}{3.111129in}}{\pgfqpoint{2.536531in}{3.116953in}}%
\pgfpathcurveto{\pgfqpoint{2.542355in}{3.122777in}}{\pgfqpoint{2.545627in}{3.130677in}}{\pgfqpoint{2.545627in}{3.138914in}}%
\pgfpathcurveto{\pgfqpoint{2.545627in}{3.147150in}}{\pgfqpoint{2.542355in}{3.155050in}}{\pgfqpoint{2.536531in}{3.160874in}}%
\pgfpathcurveto{\pgfqpoint{2.530707in}{3.166698in}}{\pgfqpoint{2.522807in}{3.169970in}}{\pgfqpoint{2.514571in}{3.169970in}}%
\pgfpathcurveto{\pgfqpoint{2.506334in}{3.169970in}}{\pgfqpoint{2.498434in}{3.166698in}}{\pgfqpoint{2.492610in}{3.160874in}}%
\pgfpathcurveto{\pgfqpoint{2.486787in}{3.155050in}}{\pgfqpoint{2.483514in}{3.147150in}}{\pgfqpoint{2.483514in}{3.138914in}}%
\pgfpathcurveto{\pgfqpoint{2.483514in}{3.130677in}}{\pgfqpoint{2.486787in}{3.122777in}}{\pgfqpoint{2.492610in}{3.116953in}}%
\pgfpathcurveto{\pgfqpoint{2.498434in}{3.111129in}}{\pgfqpoint{2.506334in}{3.107857in}}{\pgfqpoint{2.514571in}{3.107857in}}%
\pgfpathclose%
\pgfusepath{stroke,fill}%
\end{pgfscope}%
\begin{pgfscope}%
\pgfpathrectangle{\pgfqpoint{0.100000in}{0.220728in}}{\pgfqpoint{3.696000in}{3.696000in}}%
\pgfusepath{clip}%
\pgfsetbuttcap%
\pgfsetroundjoin%
\definecolor{currentfill}{rgb}{0.121569,0.466667,0.705882}%
\pgfsetfillcolor{currentfill}%
\pgfsetfillopacity{0.475629}%
\pgfsetlinewidth{1.003750pt}%
\definecolor{currentstroke}{rgb}{0.121569,0.466667,0.705882}%
\pgfsetstrokecolor{currentstroke}%
\pgfsetstrokeopacity{0.475629}%
\pgfsetdash{}{0pt}%
\pgfpathmoveto{\pgfqpoint{2.517916in}{3.107076in}}%
\pgfpathcurveto{\pgfqpoint{2.526153in}{3.107076in}}{\pgfqpoint{2.534053in}{3.110349in}}{\pgfqpoint{2.539876in}{3.116173in}}%
\pgfpathcurveto{\pgfqpoint{2.545700in}{3.121997in}}{\pgfqpoint{2.548973in}{3.129897in}}{\pgfqpoint{2.548973in}{3.138133in}}%
\pgfpathcurveto{\pgfqpoint{2.548973in}{3.146369in}}{\pgfqpoint{2.545700in}{3.154269in}}{\pgfqpoint{2.539876in}{3.160093in}}%
\pgfpathcurveto{\pgfqpoint{2.534053in}{3.165917in}}{\pgfqpoint{2.526153in}{3.169189in}}{\pgfqpoint{2.517916in}{3.169189in}}%
\pgfpathcurveto{\pgfqpoint{2.509680in}{3.169189in}}{\pgfqpoint{2.501780in}{3.165917in}}{\pgfqpoint{2.495956in}{3.160093in}}%
\pgfpathcurveto{\pgfqpoint{2.490132in}{3.154269in}}{\pgfqpoint{2.486860in}{3.146369in}}{\pgfqpoint{2.486860in}{3.138133in}}%
\pgfpathcurveto{\pgfqpoint{2.486860in}{3.129897in}}{\pgfqpoint{2.490132in}{3.121997in}}{\pgfqpoint{2.495956in}{3.116173in}}%
\pgfpathcurveto{\pgfqpoint{2.501780in}{3.110349in}}{\pgfqpoint{2.509680in}{3.107076in}}{\pgfqpoint{2.517916in}{3.107076in}}%
\pgfpathclose%
\pgfusepath{stroke,fill}%
\end{pgfscope}%
\begin{pgfscope}%
\pgfpathrectangle{\pgfqpoint{0.100000in}{0.220728in}}{\pgfqpoint{3.696000in}{3.696000in}}%
\pgfusepath{clip}%
\pgfsetbuttcap%
\pgfsetroundjoin%
\definecolor{currentfill}{rgb}{0.121569,0.466667,0.705882}%
\pgfsetfillcolor{currentfill}%
\pgfsetfillopacity{0.476340}%
\pgfsetlinewidth{1.003750pt}%
\definecolor{currentstroke}{rgb}{0.121569,0.466667,0.705882}%
\pgfsetstrokecolor{currentstroke}%
\pgfsetstrokeopacity{0.476340}%
\pgfsetdash{}{0pt}%
\pgfpathmoveto{\pgfqpoint{2.519609in}{3.107007in}}%
\pgfpathcurveto{\pgfqpoint{2.527846in}{3.107007in}}{\pgfqpoint{2.535746in}{3.110279in}}{\pgfqpoint{2.541570in}{3.116103in}}%
\pgfpathcurveto{\pgfqpoint{2.547394in}{3.121927in}}{\pgfqpoint{2.550666in}{3.129827in}}{\pgfqpoint{2.550666in}{3.138063in}}%
\pgfpathcurveto{\pgfqpoint{2.550666in}{3.146300in}}{\pgfqpoint{2.547394in}{3.154200in}}{\pgfqpoint{2.541570in}{3.160023in}}%
\pgfpathcurveto{\pgfqpoint{2.535746in}{3.165847in}}{\pgfqpoint{2.527846in}{3.169120in}}{\pgfqpoint{2.519609in}{3.169120in}}%
\pgfpathcurveto{\pgfqpoint{2.511373in}{3.169120in}}{\pgfqpoint{2.503473in}{3.165847in}}{\pgfqpoint{2.497649in}{3.160023in}}%
\pgfpathcurveto{\pgfqpoint{2.491825in}{3.154200in}}{\pgfqpoint{2.488553in}{3.146300in}}{\pgfqpoint{2.488553in}{3.138063in}}%
\pgfpathcurveto{\pgfqpoint{2.488553in}{3.129827in}}{\pgfqpoint{2.491825in}{3.121927in}}{\pgfqpoint{2.497649in}{3.116103in}}%
\pgfpathcurveto{\pgfqpoint{2.503473in}{3.110279in}}{\pgfqpoint{2.511373in}{3.107007in}}{\pgfqpoint{2.519609in}{3.107007in}}%
\pgfpathclose%
\pgfusepath{stroke,fill}%
\end{pgfscope}%
\begin{pgfscope}%
\pgfpathrectangle{\pgfqpoint{0.100000in}{0.220728in}}{\pgfqpoint{3.696000in}{3.696000in}}%
\pgfusepath{clip}%
\pgfsetbuttcap%
\pgfsetroundjoin%
\definecolor{currentfill}{rgb}{0.121569,0.466667,0.705882}%
\pgfsetfillcolor{currentfill}%
\pgfsetfillopacity{0.476764}%
\pgfsetlinewidth{1.003750pt}%
\definecolor{currentstroke}{rgb}{0.121569,0.466667,0.705882}%
\pgfsetstrokecolor{currentstroke}%
\pgfsetstrokeopacity{0.476764}%
\pgfsetdash{}{0pt}%
\pgfpathmoveto{\pgfqpoint{2.522462in}{3.105604in}}%
\pgfpathcurveto{\pgfqpoint{2.530699in}{3.105604in}}{\pgfqpoint{2.538599in}{3.108876in}}{\pgfqpoint{2.544423in}{3.114700in}}%
\pgfpathcurveto{\pgfqpoint{2.550246in}{3.120524in}}{\pgfqpoint{2.553519in}{3.128424in}}{\pgfqpoint{2.553519in}{3.136661in}}%
\pgfpathcurveto{\pgfqpoint{2.553519in}{3.144897in}}{\pgfqpoint{2.550246in}{3.152797in}}{\pgfqpoint{2.544423in}{3.158621in}}%
\pgfpathcurveto{\pgfqpoint{2.538599in}{3.164445in}}{\pgfqpoint{2.530699in}{3.167717in}}{\pgfqpoint{2.522462in}{3.167717in}}%
\pgfpathcurveto{\pgfqpoint{2.514226in}{3.167717in}}{\pgfqpoint{2.506326in}{3.164445in}}{\pgfqpoint{2.500502in}{3.158621in}}%
\pgfpathcurveto{\pgfqpoint{2.494678in}{3.152797in}}{\pgfqpoint{2.491406in}{3.144897in}}{\pgfqpoint{2.491406in}{3.136661in}}%
\pgfpathcurveto{\pgfqpoint{2.491406in}{3.128424in}}{\pgfqpoint{2.494678in}{3.120524in}}{\pgfqpoint{2.500502in}{3.114700in}}%
\pgfpathcurveto{\pgfqpoint{2.506326in}{3.108876in}}{\pgfqpoint{2.514226in}{3.105604in}}{\pgfqpoint{2.522462in}{3.105604in}}%
\pgfpathclose%
\pgfusepath{stroke,fill}%
\end{pgfscope}%
\begin{pgfscope}%
\pgfpathrectangle{\pgfqpoint{0.100000in}{0.220728in}}{\pgfqpoint{3.696000in}{3.696000in}}%
\pgfusepath{clip}%
\pgfsetbuttcap%
\pgfsetroundjoin%
\definecolor{currentfill}{rgb}{0.121569,0.466667,0.705882}%
\pgfsetfillcolor{currentfill}%
\pgfsetfillopacity{0.476800}%
\pgfsetlinewidth{1.003750pt}%
\definecolor{currentstroke}{rgb}{0.121569,0.466667,0.705882}%
\pgfsetstrokecolor{currentstroke}%
\pgfsetstrokeopacity{0.476800}%
\pgfsetdash{}{0pt}%
\pgfpathmoveto{\pgfqpoint{1.259171in}{2.133767in}}%
\pgfpathcurveto{\pgfqpoint{1.267408in}{2.133767in}}{\pgfqpoint{1.275308in}{2.137039in}}{\pgfqpoint{1.281132in}{2.142863in}}%
\pgfpathcurveto{\pgfqpoint{1.286955in}{2.148687in}}{\pgfqpoint{1.290228in}{2.156587in}}{\pgfqpoint{1.290228in}{2.164824in}}%
\pgfpathcurveto{\pgfqpoint{1.290228in}{2.173060in}}{\pgfqpoint{1.286955in}{2.180960in}}{\pgfqpoint{1.281132in}{2.186784in}}%
\pgfpathcurveto{\pgfqpoint{1.275308in}{2.192608in}}{\pgfqpoint{1.267408in}{2.195880in}}{\pgfqpoint{1.259171in}{2.195880in}}%
\pgfpathcurveto{\pgfqpoint{1.250935in}{2.195880in}}{\pgfqpoint{1.243035in}{2.192608in}}{\pgfqpoint{1.237211in}{2.186784in}}%
\pgfpathcurveto{\pgfqpoint{1.231387in}{2.180960in}}{\pgfqpoint{1.228115in}{2.173060in}}{\pgfqpoint{1.228115in}{2.164824in}}%
\pgfpathcurveto{\pgfqpoint{1.228115in}{2.156587in}}{\pgfqpoint{1.231387in}{2.148687in}}{\pgfqpoint{1.237211in}{2.142863in}}%
\pgfpathcurveto{\pgfqpoint{1.243035in}{2.137039in}}{\pgfqpoint{1.250935in}{2.133767in}}{\pgfqpoint{1.259171in}{2.133767in}}%
\pgfpathclose%
\pgfusepath{stroke,fill}%
\end{pgfscope}%
\begin{pgfscope}%
\pgfpathrectangle{\pgfqpoint{0.100000in}{0.220728in}}{\pgfqpoint{3.696000in}{3.696000in}}%
\pgfusepath{clip}%
\pgfsetbuttcap%
\pgfsetroundjoin%
\definecolor{currentfill}{rgb}{0.121569,0.466667,0.705882}%
\pgfsetfillcolor{currentfill}%
\pgfsetfillopacity{0.477789}%
\pgfsetlinewidth{1.003750pt}%
\definecolor{currentstroke}{rgb}{0.121569,0.466667,0.705882}%
\pgfsetstrokecolor{currentstroke}%
\pgfsetstrokeopacity{0.477789}%
\pgfsetdash{}{0pt}%
\pgfpathmoveto{\pgfqpoint{2.525590in}{3.104223in}}%
\pgfpathcurveto{\pgfqpoint{2.533826in}{3.104223in}}{\pgfqpoint{2.541726in}{3.107496in}}{\pgfqpoint{2.547550in}{3.113320in}}%
\pgfpathcurveto{\pgfqpoint{2.553374in}{3.119144in}}{\pgfqpoint{2.556647in}{3.127044in}}{\pgfqpoint{2.556647in}{3.135280in}}%
\pgfpathcurveto{\pgfqpoint{2.556647in}{3.143516in}}{\pgfqpoint{2.553374in}{3.151416in}}{\pgfqpoint{2.547550in}{3.157240in}}%
\pgfpathcurveto{\pgfqpoint{2.541726in}{3.163064in}}{\pgfqpoint{2.533826in}{3.166336in}}{\pgfqpoint{2.525590in}{3.166336in}}%
\pgfpathcurveto{\pgfqpoint{2.517354in}{3.166336in}}{\pgfqpoint{2.509454in}{3.163064in}}{\pgfqpoint{2.503630in}{3.157240in}}%
\pgfpathcurveto{\pgfqpoint{2.497806in}{3.151416in}}{\pgfqpoint{2.494534in}{3.143516in}}{\pgfqpoint{2.494534in}{3.135280in}}%
\pgfpathcurveto{\pgfqpoint{2.494534in}{3.127044in}}{\pgfqpoint{2.497806in}{3.119144in}}{\pgfqpoint{2.503630in}{3.113320in}}%
\pgfpathcurveto{\pgfqpoint{2.509454in}{3.107496in}}{\pgfqpoint{2.517354in}{3.104223in}}{\pgfqpoint{2.525590in}{3.104223in}}%
\pgfpathclose%
\pgfusepath{stroke,fill}%
\end{pgfscope}%
\begin{pgfscope}%
\pgfpathrectangle{\pgfqpoint{0.100000in}{0.220728in}}{\pgfqpoint{3.696000in}{3.696000in}}%
\pgfusepath{clip}%
\pgfsetbuttcap%
\pgfsetroundjoin%
\definecolor{currentfill}{rgb}{0.121569,0.466667,0.705882}%
\pgfsetfillcolor{currentfill}%
\pgfsetfillopacity{0.477921}%
\pgfsetlinewidth{1.003750pt}%
\definecolor{currentstroke}{rgb}{0.121569,0.466667,0.705882}%
\pgfsetstrokecolor{currentstroke}%
\pgfsetstrokeopacity{0.477921}%
\pgfsetdash{}{0pt}%
\pgfpathmoveto{\pgfqpoint{2.527846in}{3.103603in}}%
\pgfpathcurveto{\pgfqpoint{2.536082in}{3.103603in}}{\pgfqpoint{2.543982in}{3.106876in}}{\pgfqpoint{2.549806in}{3.112700in}}%
\pgfpathcurveto{\pgfqpoint{2.555630in}{3.118524in}}{\pgfqpoint{2.558902in}{3.126424in}}{\pgfqpoint{2.558902in}{3.134660in}}%
\pgfpathcurveto{\pgfqpoint{2.558902in}{3.142896in}}{\pgfqpoint{2.555630in}{3.150796in}}{\pgfqpoint{2.549806in}{3.156620in}}%
\pgfpathcurveto{\pgfqpoint{2.543982in}{3.162444in}}{\pgfqpoint{2.536082in}{3.165716in}}{\pgfqpoint{2.527846in}{3.165716in}}%
\pgfpathcurveto{\pgfqpoint{2.519609in}{3.165716in}}{\pgfqpoint{2.511709in}{3.162444in}}{\pgfqpoint{2.505885in}{3.156620in}}%
\pgfpathcurveto{\pgfqpoint{2.500061in}{3.150796in}}{\pgfqpoint{2.496789in}{3.142896in}}{\pgfqpoint{2.496789in}{3.134660in}}%
\pgfpathcurveto{\pgfqpoint{2.496789in}{3.126424in}}{\pgfqpoint{2.500061in}{3.118524in}}{\pgfqpoint{2.505885in}{3.112700in}}%
\pgfpathcurveto{\pgfqpoint{2.511709in}{3.106876in}}{\pgfqpoint{2.519609in}{3.103603in}}{\pgfqpoint{2.527846in}{3.103603in}}%
\pgfpathclose%
\pgfusepath{stroke,fill}%
\end{pgfscope}%
\begin{pgfscope}%
\pgfpathrectangle{\pgfqpoint{0.100000in}{0.220728in}}{\pgfqpoint{3.696000in}{3.696000in}}%
\pgfusepath{clip}%
\pgfsetbuttcap%
\pgfsetroundjoin%
\definecolor{currentfill}{rgb}{0.121569,0.466667,0.705882}%
\pgfsetfillcolor{currentfill}%
\pgfsetfillopacity{0.478537}%
\pgfsetlinewidth{1.003750pt}%
\definecolor{currentstroke}{rgb}{0.121569,0.466667,0.705882}%
\pgfsetstrokecolor{currentstroke}%
\pgfsetstrokeopacity{0.478537}%
\pgfsetdash{}{0pt}%
\pgfpathmoveto{\pgfqpoint{2.530666in}{3.103083in}}%
\pgfpathcurveto{\pgfqpoint{2.538903in}{3.103083in}}{\pgfqpoint{2.546803in}{3.106355in}}{\pgfqpoint{2.552627in}{3.112179in}}%
\pgfpathcurveto{\pgfqpoint{2.558451in}{3.118003in}}{\pgfqpoint{2.561723in}{3.125903in}}{\pgfqpoint{2.561723in}{3.134139in}}%
\pgfpathcurveto{\pgfqpoint{2.561723in}{3.142376in}}{\pgfqpoint{2.558451in}{3.150276in}}{\pgfqpoint{2.552627in}{3.156099in}}%
\pgfpathcurveto{\pgfqpoint{2.546803in}{3.161923in}}{\pgfqpoint{2.538903in}{3.165196in}}{\pgfqpoint{2.530666in}{3.165196in}}%
\pgfpathcurveto{\pgfqpoint{2.522430in}{3.165196in}}{\pgfqpoint{2.514530in}{3.161923in}}{\pgfqpoint{2.508706in}{3.156099in}}%
\pgfpathcurveto{\pgfqpoint{2.502882in}{3.150276in}}{\pgfqpoint{2.499610in}{3.142376in}}{\pgfqpoint{2.499610in}{3.134139in}}%
\pgfpathcurveto{\pgfqpoint{2.499610in}{3.125903in}}{\pgfqpoint{2.502882in}{3.118003in}}{\pgfqpoint{2.508706in}{3.112179in}}%
\pgfpathcurveto{\pgfqpoint{2.514530in}{3.106355in}}{\pgfqpoint{2.522430in}{3.103083in}}{\pgfqpoint{2.530666in}{3.103083in}}%
\pgfpathclose%
\pgfusepath{stroke,fill}%
\end{pgfscope}%
\begin{pgfscope}%
\pgfpathrectangle{\pgfqpoint{0.100000in}{0.220728in}}{\pgfqpoint{3.696000in}{3.696000in}}%
\pgfusepath{clip}%
\pgfsetbuttcap%
\pgfsetroundjoin%
\definecolor{currentfill}{rgb}{0.121569,0.466667,0.705882}%
\pgfsetfillcolor{currentfill}%
\pgfsetfillopacity{0.479763}%
\pgfsetlinewidth{1.003750pt}%
\definecolor{currentstroke}{rgb}{0.121569,0.466667,0.705882}%
\pgfsetstrokecolor{currentstroke}%
\pgfsetstrokeopacity{0.479763}%
\pgfsetdash{}{0pt}%
\pgfpathmoveto{\pgfqpoint{2.534573in}{3.102613in}}%
\pgfpathcurveto{\pgfqpoint{2.542809in}{3.102613in}}{\pgfqpoint{2.550709in}{3.105885in}}{\pgfqpoint{2.556533in}{3.111709in}}%
\pgfpathcurveto{\pgfqpoint{2.562357in}{3.117533in}}{\pgfqpoint{2.565630in}{3.125433in}}{\pgfqpoint{2.565630in}{3.133669in}}%
\pgfpathcurveto{\pgfqpoint{2.565630in}{3.141906in}}{\pgfqpoint{2.562357in}{3.149806in}}{\pgfqpoint{2.556533in}{3.155630in}}%
\pgfpathcurveto{\pgfqpoint{2.550709in}{3.161454in}}{\pgfqpoint{2.542809in}{3.164726in}}{\pgfqpoint{2.534573in}{3.164726in}}%
\pgfpathcurveto{\pgfqpoint{2.526337in}{3.164726in}}{\pgfqpoint{2.518437in}{3.161454in}}{\pgfqpoint{2.512613in}{3.155630in}}%
\pgfpathcurveto{\pgfqpoint{2.506789in}{3.149806in}}{\pgfqpoint{2.503517in}{3.141906in}}{\pgfqpoint{2.503517in}{3.133669in}}%
\pgfpathcurveto{\pgfqpoint{2.503517in}{3.125433in}}{\pgfqpoint{2.506789in}{3.117533in}}{\pgfqpoint{2.512613in}{3.111709in}}%
\pgfpathcurveto{\pgfqpoint{2.518437in}{3.105885in}}{\pgfqpoint{2.526337in}{3.102613in}}{\pgfqpoint{2.534573in}{3.102613in}}%
\pgfpathclose%
\pgfusepath{stroke,fill}%
\end{pgfscope}%
\begin{pgfscope}%
\pgfpathrectangle{\pgfqpoint{0.100000in}{0.220728in}}{\pgfqpoint{3.696000in}{3.696000in}}%
\pgfusepath{clip}%
\pgfsetbuttcap%
\pgfsetroundjoin%
\definecolor{currentfill}{rgb}{0.121569,0.466667,0.705882}%
\pgfsetfillcolor{currentfill}%
\pgfsetfillopacity{0.480461}%
\pgfsetlinewidth{1.003750pt}%
\definecolor{currentstroke}{rgb}{0.121569,0.466667,0.705882}%
\pgfsetstrokecolor{currentstroke}%
\pgfsetstrokeopacity{0.480461}%
\pgfsetdash{}{0pt}%
\pgfpathmoveto{\pgfqpoint{2.542063in}{3.100834in}}%
\pgfpathcurveto{\pgfqpoint{2.550300in}{3.100834in}}{\pgfqpoint{2.558200in}{3.104106in}}{\pgfqpoint{2.564024in}{3.109930in}}%
\pgfpathcurveto{\pgfqpoint{2.569848in}{3.115754in}}{\pgfqpoint{2.573120in}{3.123654in}}{\pgfqpoint{2.573120in}{3.131890in}}%
\pgfpathcurveto{\pgfqpoint{2.573120in}{3.140126in}}{\pgfqpoint{2.569848in}{3.148026in}}{\pgfqpoint{2.564024in}{3.153850in}}%
\pgfpathcurveto{\pgfqpoint{2.558200in}{3.159674in}}{\pgfqpoint{2.550300in}{3.162947in}}{\pgfqpoint{2.542063in}{3.162947in}}%
\pgfpathcurveto{\pgfqpoint{2.533827in}{3.162947in}}{\pgfqpoint{2.525927in}{3.159674in}}{\pgfqpoint{2.520103in}{3.153850in}}%
\pgfpathcurveto{\pgfqpoint{2.514279in}{3.148026in}}{\pgfqpoint{2.511007in}{3.140126in}}{\pgfqpoint{2.511007in}{3.131890in}}%
\pgfpathcurveto{\pgfqpoint{2.511007in}{3.123654in}}{\pgfqpoint{2.514279in}{3.115754in}}{\pgfqpoint{2.520103in}{3.109930in}}%
\pgfpathcurveto{\pgfqpoint{2.525927in}{3.104106in}}{\pgfqpoint{2.533827in}{3.100834in}}{\pgfqpoint{2.542063in}{3.100834in}}%
\pgfpathclose%
\pgfusepath{stroke,fill}%
\end{pgfscope}%
\begin{pgfscope}%
\pgfpathrectangle{\pgfqpoint{0.100000in}{0.220728in}}{\pgfqpoint{3.696000in}{3.696000in}}%
\pgfusepath{clip}%
\pgfsetbuttcap%
\pgfsetroundjoin%
\definecolor{currentfill}{rgb}{0.121569,0.466667,0.705882}%
\pgfsetfillcolor{currentfill}%
\pgfsetfillopacity{0.482546}%
\pgfsetlinewidth{1.003750pt}%
\definecolor{currentstroke}{rgb}{0.121569,0.466667,0.705882}%
\pgfsetstrokecolor{currentstroke}%
\pgfsetstrokeopacity{0.482546}%
\pgfsetdash{}{0pt}%
\pgfpathmoveto{\pgfqpoint{2.549383in}{3.098958in}}%
\pgfpathcurveto{\pgfqpoint{2.557619in}{3.098958in}}{\pgfqpoint{2.565519in}{3.102230in}}{\pgfqpoint{2.571343in}{3.108054in}}%
\pgfpathcurveto{\pgfqpoint{2.577167in}{3.113878in}}{\pgfqpoint{2.580439in}{3.121778in}}{\pgfqpoint{2.580439in}{3.130014in}}%
\pgfpathcurveto{\pgfqpoint{2.580439in}{3.138251in}}{\pgfqpoint{2.577167in}{3.146151in}}{\pgfqpoint{2.571343in}{3.151975in}}%
\pgfpathcurveto{\pgfqpoint{2.565519in}{3.157799in}}{\pgfqpoint{2.557619in}{3.161071in}}{\pgfqpoint{2.549383in}{3.161071in}}%
\pgfpathcurveto{\pgfqpoint{2.541146in}{3.161071in}}{\pgfqpoint{2.533246in}{3.157799in}}{\pgfqpoint{2.527422in}{3.151975in}}%
\pgfpathcurveto{\pgfqpoint{2.521598in}{3.146151in}}{\pgfqpoint{2.518326in}{3.138251in}}{\pgfqpoint{2.518326in}{3.130014in}}%
\pgfpathcurveto{\pgfqpoint{2.518326in}{3.121778in}}{\pgfqpoint{2.521598in}{3.113878in}}{\pgfqpoint{2.527422in}{3.108054in}}%
\pgfpathcurveto{\pgfqpoint{2.533246in}{3.102230in}}{\pgfqpoint{2.541146in}{3.098958in}}{\pgfqpoint{2.549383in}{3.098958in}}%
\pgfpathclose%
\pgfusepath{stroke,fill}%
\end{pgfscope}%
\begin{pgfscope}%
\pgfpathrectangle{\pgfqpoint{0.100000in}{0.220728in}}{\pgfqpoint{3.696000in}{3.696000in}}%
\pgfusepath{clip}%
\pgfsetbuttcap%
\pgfsetroundjoin%
\definecolor{currentfill}{rgb}{0.121569,0.466667,0.705882}%
\pgfsetfillcolor{currentfill}%
\pgfsetfillopacity{0.482783}%
\pgfsetlinewidth{1.003750pt}%
\definecolor{currentstroke}{rgb}{0.121569,0.466667,0.705882}%
\pgfsetstrokecolor{currentstroke}%
\pgfsetstrokeopacity{0.482783}%
\pgfsetdash{}{0pt}%
\pgfpathmoveto{\pgfqpoint{1.253838in}{2.094947in}}%
\pgfpathcurveto{\pgfqpoint{1.262075in}{2.094947in}}{\pgfqpoint{1.269975in}{2.098219in}}{\pgfqpoint{1.275799in}{2.104043in}}%
\pgfpathcurveto{\pgfqpoint{1.281622in}{2.109867in}}{\pgfqpoint{1.284895in}{2.117767in}}{\pgfqpoint{1.284895in}{2.126003in}}%
\pgfpathcurveto{\pgfqpoint{1.284895in}{2.134239in}}{\pgfqpoint{1.281622in}{2.142140in}}{\pgfqpoint{1.275799in}{2.147963in}}%
\pgfpathcurveto{\pgfqpoint{1.269975in}{2.153787in}}{\pgfqpoint{1.262075in}{2.157060in}}{\pgfqpoint{1.253838in}{2.157060in}}%
\pgfpathcurveto{\pgfqpoint{1.245602in}{2.157060in}}{\pgfqpoint{1.237702in}{2.153787in}}{\pgfqpoint{1.231878in}{2.147963in}}%
\pgfpathcurveto{\pgfqpoint{1.226054in}{2.142140in}}{\pgfqpoint{1.222782in}{2.134239in}}{\pgfqpoint{1.222782in}{2.126003in}}%
\pgfpathcurveto{\pgfqpoint{1.222782in}{2.117767in}}{\pgfqpoint{1.226054in}{2.109867in}}{\pgfqpoint{1.231878in}{2.104043in}}%
\pgfpathcurveto{\pgfqpoint{1.237702in}{2.098219in}}{\pgfqpoint{1.245602in}{2.094947in}}{\pgfqpoint{1.253838in}{2.094947in}}%
\pgfpathclose%
\pgfusepath{stroke,fill}%
\end{pgfscope}%
\begin{pgfscope}%
\pgfpathrectangle{\pgfqpoint{0.100000in}{0.220728in}}{\pgfqpoint{3.696000in}{3.696000in}}%
\pgfusepath{clip}%
\pgfsetbuttcap%
\pgfsetroundjoin%
\definecolor{currentfill}{rgb}{0.121569,0.466667,0.705882}%
\pgfsetfillcolor{currentfill}%
\pgfsetfillopacity{0.484695}%
\pgfsetlinewidth{1.003750pt}%
\definecolor{currentstroke}{rgb}{0.121569,0.466667,0.705882}%
\pgfsetstrokecolor{currentstroke}%
\pgfsetstrokeopacity{0.484695}%
\pgfsetdash{}{0pt}%
\pgfpathmoveto{\pgfqpoint{2.557895in}{3.097213in}}%
\pgfpathcurveto{\pgfqpoint{2.566131in}{3.097213in}}{\pgfqpoint{2.574031in}{3.100486in}}{\pgfqpoint{2.579855in}{3.106310in}}%
\pgfpathcurveto{\pgfqpoint{2.585679in}{3.112133in}}{\pgfqpoint{2.588951in}{3.120033in}}{\pgfqpoint{2.588951in}{3.128270in}}%
\pgfpathcurveto{\pgfqpoint{2.588951in}{3.136506in}}{\pgfqpoint{2.585679in}{3.144406in}}{\pgfqpoint{2.579855in}{3.150230in}}%
\pgfpathcurveto{\pgfqpoint{2.574031in}{3.156054in}}{\pgfqpoint{2.566131in}{3.159326in}}{\pgfqpoint{2.557895in}{3.159326in}}%
\pgfpathcurveto{\pgfqpoint{2.549658in}{3.159326in}}{\pgfqpoint{2.541758in}{3.156054in}}{\pgfqpoint{2.535934in}{3.150230in}}%
\pgfpathcurveto{\pgfqpoint{2.530110in}{3.144406in}}{\pgfqpoint{2.526838in}{3.136506in}}{\pgfqpoint{2.526838in}{3.128270in}}%
\pgfpathcurveto{\pgfqpoint{2.526838in}{3.120033in}}{\pgfqpoint{2.530110in}{3.112133in}}{\pgfqpoint{2.535934in}{3.106310in}}%
\pgfpathcurveto{\pgfqpoint{2.541758in}{3.100486in}}{\pgfqpoint{2.549658in}{3.097213in}}{\pgfqpoint{2.557895in}{3.097213in}}%
\pgfpathclose%
\pgfusepath{stroke,fill}%
\end{pgfscope}%
\begin{pgfscope}%
\pgfpathrectangle{\pgfqpoint{0.100000in}{0.220728in}}{\pgfqpoint{3.696000in}{3.696000in}}%
\pgfusepath{clip}%
\pgfsetbuttcap%
\pgfsetroundjoin%
\definecolor{currentfill}{rgb}{0.121569,0.466667,0.705882}%
\pgfsetfillcolor{currentfill}%
\pgfsetfillopacity{0.486225}%
\pgfsetlinewidth{1.003750pt}%
\definecolor{currentstroke}{rgb}{0.121569,0.466667,0.705882}%
\pgfsetstrokecolor{currentstroke}%
\pgfsetstrokeopacity{0.486225}%
\pgfsetdash{}{0pt}%
\pgfpathmoveto{\pgfqpoint{1.231820in}{2.069578in}}%
\pgfpathcurveto{\pgfqpoint{1.240056in}{2.069578in}}{\pgfqpoint{1.247956in}{2.072851in}}{\pgfqpoint{1.253780in}{2.078675in}}%
\pgfpathcurveto{\pgfqpoint{1.259604in}{2.084499in}}{\pgfqpoint{1.262876in}{2.092399in}}{\pgfqpoint{1.262876in}{2.100635in}}%
\pgfpathcurveto{\pgfqpoint{1.262876in}{2.108871in}}{\pgfqpoint{1.259604in}{2.116771in}}{\pgfqpoint{1.253780in}{2.122595in}}%
\pgfpathcurveto{\pgfqpoint{1.247956in}{2.128419in}}{\pgfqpoint{1.240056in}{2.131691in}}{\pgfqpoint{1.231820in}{2.131691in}}%
\pgfpathcurveto{\pgfqpoint{1.223583in}{2.131691in}}{\pgfqpoint{1.215683in}{2.128419in}}{\pgfqpoint{1.209859in}{2.122595in}}%
\pgfpathcurveto{\pgfqpoint{1.204035in}{2.116771in}}{\pgfqpoint{1.200763in}{2.108871in}}{\pgfqpoint{1.200763in}{2.100635in}}%
\pgfpathcurveto{\pgfqpoint{1.200763in}{2.092399in}}{\pgfqpoint{1.204035in}{2.084499in}}{\pgfqpoint{1.209859in}{2.078675in}}%
\pgfpathcurveto{\pgfqpoint{1.215683in}{2.072851in}}{\pgfqpoint{1.223583in}{2.069578in}}{\pgfqpoint{1.231820in}{2.069578in}}%
\pgfpathclose%
\pgfusepath{stroke,fill}%
\end{pgfscope}%
\begin{pgfscope}%
\pgfpathrectangle{\pgfqpoint{0.100000in}{0.220728in}}{\pgfqpoint{3.696000in}{3.696000in}}%
\pgfusepath{clip}%
\pgfsetbuttcap%
\pgfsetroundjoin%
\definecolor{currentfill}{rgb}{0.121569,0.466667,0.705882}%
\pgfsetfillcolor{currentfill}%
\pgfsetfillopacity{0.486453}%
\pgfsetlinewidth{1.003750pt}%
\definecolor{currentstroke}{rgb}{0.121569,0.466667,0.705882}%
\pgfsetstrokecolor{currentstroke}%
\pgfsetstrokeopacity{0.486453}%
\pgfsetdash{}{0pt}%
\pgfpathmoveto{\pgfqpoint{2.568452in}{3.093935in}}%
\pgfpathcurveto{\pgfqpoint{2.576688in}{3.093935in}}{\pgfqpoint{2.584589in}{3.097207in}}{\pgfqpoint{2.590412in}{3.103031in}}%
\pgfpathcurveto{\pgfqpoint{2.596236in}{3.108855in}}{\pgfqpoint{2.599509in}{3.116755in}}{\pgfqpoint{2.599509in}{3.124991in}}%
\pgfpathcurveto{\pgfqpoint{2.599509in}{3.133227in}}{\pgfqpoint{2.596236in}{3.141127in}}{\pgfqpoint{2.590412in}{3.146951in}}%
\pgfpathcurveto{\pgfqpoint{2.584589in}{3.152775in}}{\pgfqpoint{2.576688in}{3.156048in}}{\pgfqpoint{2.568452in}{3.156048in}}%
\pgfpathcurveto{\pgfqpoint{2.560216in}{3.156048in}}{\pgfqpoint{2.552316in}{3.152775in}}{\pgfqpoint{2.546492in}{3.146951in}}%
\pgfpathcurveto{\pgfqpoint{2.540668in}{3.141127in}}{\pgfqpoint{2.537396in}{3.133227in}}{\pgfqpoint{2.537396in}{3.124991in}}%
\pgfpathcurveto{\pgfqpoint{2.537396in}{3.116755in}}{\pgfqpoint{2.540668in}{3.108855in}}{\pgfqpoint{2.546492in}{3.103031in}}%
\pgfpathcurveto{\pgfqpoint{2.552316in}{3.097207in}}{\pgfqpoint{2.560216in}{3.093935in}}{\pgfqpoint{2.568452in}{3.093935in}}%
\pgfpathclose%
\pgfusepath{stroke,fill}%
\end{pgfscope}%
\begin{pgfscope}%
\pgfpathrectangle{\pgfqpoint{0.100000in}{0.220728in}}{\pgfqpoint{3.696000in}{3.696000in}}%
\pgfusepath{clip}%
\pgfsetbuttcap%
\pgfsetroundjoin%
\definecolor{currentfill}{rgb}{0.121569,0.466667,0.705882}%
\pgfsetfillcolor{currentfill}%
\pgfsetfillopacity{0.490561}%
\pgfsetlinewidth{1.003750pt}%
\definecolor{currentstroke}{rgb}{0.121569,0.466667,0.705882}%
\pgfsetstrokecolor{currentstroke}%
\pgfsetstrokeopacity{0.490561}%
\pgfsetdash{}{0pt}%
\pgfpathmoveto{\pgfqpoint{1.227804in}{2.035156in}}%
\pgfpathcurveto{\pgfqpoint{1.236040in}{2.035156in}}{\pgfqpoint{1.243940in}{2.038429in}}{\pgfqpoint{1.249764in}{2.044253in}}%
\pgfpathcurveto{\pgfqpoint{1.255588in}{2.050077in}}{\pgfqpoint{1.258860in}{2.057977in}}{\pgfqpoint{1.258860in}{2.066213in}}%
\pgfpathcurveto{\pgfqpoint{1.258860in}{2.074449in}}{\pgfqpoint{1.255588in}{2.082349in}}{\pgfqpoint{1.249764in}{2.088173in}}%
\pgfpathcurveto{\pgfqpoint{1.243940in}{2.093997in}}{\pgfqpoint{1.236040in}{2.097269in}}{\pgfqpoint{1.227804in}{2.097269in}}%
\pgfpathcurveto{\pgfqpoint{1.219567in}{2.097269in}}{\pgfqpoint{1.211667in}{2.093997in}}{\pgfqpoint{1.205843in}{2.088173in}}%
\pgfpathcurveto{\pgfqpoint{1.200019in}{2.082349in}}{\pgfqpoint{1.196747in}{2.074449in}}{\pgfqpoint{1.196747in}{2.066213in}}%
\pgfpathcurveto{\pgfqpoint{1.196747in}{2.057977in}}{\pgfqpoint{1.200019in}{2.050077in}}{\pgfqpoint{1.205843in}{2.044253in}}%
\pgfpathcurveto{\pgfqpoint{1.211667in}{2.038429in}}{\pgfqpoint{1.219567in}{2.035156in}}{\pgfqpoint{1.227804in}{2.035156in}}%
\pgfpathclose%
\pgfusepath{stroke,fill}%
\end{pgfscope}%
\begin{pgfscope}%
\pgfpathrectangle{\pgfqpoint{0.100000in}{0.220728in}}{\pgfqpoint{3.696000in}{3.696000in}}%
\pgfusepath{clip}%
\pgfsetbuttcap%
\pgfsetroundjoin%
\definecolor{currentfill}{rgb}{0.121569,0.466667,0.705882}%
\pgfsetfillcolor{currentfill}%
\pgfsetfillopacity{0.490593}%
\pgfsetlinewidth{1.003750pt}%
\definecolor{currentstroke}{rgb}{0.121569,0.466667,0.705882}%
\pgfsetstrokecolor{currentstroke}%
\pgfsetstrokeopacity{0.490593}%
\pgfsetdash{}{0pt}%
\pgfpathmoveto{\pgfqpoint{2.578594in}{3.094132in}}%
\pgfpathcurveto{\pgfqpoint{2.586830in}{3.094132in}}{\pgfqpoint{2.594730in}{3.097404in}}{\pgfqpoint{2.600554in}{3.103228in}}%
\pgfpathcurveto{\pgfqpoint{2.606378in}{3.109052in}}{\pgfqpoint{2.609650in}{3.116952in}}{\pgfqpoint{2.609650in}{3.125188in}}%
\pgfpathcurveto{\pgfqpoint{2.609650in}{3.133425in}}{\pgfqpoint{2.606378in}{3.141325in}}{\pgfqpoint{2.600554in}{3.147149in}}%
\pgfpathcurveto{\pgfqpoint{2.594730in}{3.152973in}}{\pgfqpoint{2.586830in}{3.156245in}}{\pgfqpoint{2.578594in}{3.156245in}}%
\pgfpathcurveto{\pgfqpoint{2.570357in}{3.156245in}}{\pgfqpoint{2.562457in}{3.152973in}}{\pgfqpoint{2.556633in}{3.147149in}}%
\pgfpathcurveto{\pgfqpoint{2.550809in}{3.141325in}}{\pgfqpoint{2.547537in}{3.133425in}}{\pgfqpoint{2.547537in}{3.125188in}}%
\pgfpathcurveto{\pgfqpoint{2.547537in}{3.116952in}}{\pgfqpoint{2.550809in}{3.109052in}}{\pgfqpoint{2.556633in}{3.103228in}}%
\pgfpathcurveto{\pgfqpoint{2.562457in}{3.097404in}}{\pgfqpoint{2.570357in}{3.094132in}}{\pgfqpoint{2.578594in}{3.094132in}}%
\pgfpathclose%
\pgfusepath{stroke,fill}%
\end{pgfscope}%
\begin{pgfscope}%
\pgfpathrectangle{\pgfqpoint{0.100000in}{0.220728in}}{\pgfqpoint{3.696000in}{3.696000in}}%
\pgfusepath{clip}%
\pgfsetbuttcap%
\pgfsetroundjoin%
\definecolor{currentfill}{rgb}{0.121569,0.466667,0.705882}%
\pgfsetfillcolor{currentfill}%
\pgfsetfillopacity{0.492088}%
\pgfsetlinewidth{1.003750pt}%
\definecolor{currentstroke}{rgb}{0.121569,0.466667,0.705882}%
\pgfsetstrokecolor{currentstroke}%
\pgfsetstrokeopacity{0.492088}%
\pgfsetdash{}{0pt}%
\pgfpathmoveto{\pgfqpoint{2.584887in}{3.093012in}}%
\pgfpathcurveto{\pgfqpoint{2.593123in}{3.093012in}}{\pgfqpoint{2.601023in}{3.096284in}}{\pgfqpoint{2.606847in}{3.102108in}}%
\pgfpathcurveto{\pgfqpoint{2.612671in}{3.107932in}}{\pgfqpoint{2.615943in}{3.115832in}}{\pgfqpoint{2.615943in}{3.124068in}}%
\pgfpathcurveto{\pgfqpoint{2.615943in}{3.132304in}}{\pgfqpoint{2.612671in}{3.140204in}}{\pgfqpoint{2.606847in}{3.146028in}}%
\pgfpathcurveto{\pgfqpoint{2.601023in}{3.151852in}}{\pgfqpoint{2.593123in}{3.155125in}}{\pgfqpoint{2.584887in}{3.155125in}}%
\pgfpathcurveto{\pgfqpoint{2.576650in}{3.155125in}}{\pgfqpoint{2.568750in}{3.151852in}}{\pgfqpoint{2.562926in}{3.146028in}}%
\pgfpathcurveto{\pgfqpoint{2.557103in}{3.140204in}}{\pgfqpoint{2.553830in}{3.132304in}}{\pgfqpoint{2.553830in}{3.124068in}}%
\pgfpathcurveto{\pgfqpoint{2.553830in}{3.115832in}}{\pgfqpoint{2.557103in}{3.107932in}}{\pgfqpoint{2.562926in}{3.102108in}}%
\pgfpathcurveto{\pgfqpoint{2.568750in}{3.096284in}}{\pgfqpoint{2.576650in}{3.093012in}}{\pgfqpoint{2.584887in}{3.093012in}}%
\pgfpathclose%
\pgfusepath{stroke,fill}%
\end{pgfscope}%
\begin{pgfscope}%
\pgfpathrectangle{\pgfqpoint{0.100000in}{0.220728in}}{\pgfqpoint{3.696000in}{3.696000in}}%
\pgfusepath{clip}%
\pgfsetbuttcap%
\pgfsetroundjoin%
\definecolor{currentfill}{rgb}{0.121569,0.466667,0.705882}%
\pgfsetfillcolor{currentfill}%
\pgfsetfillopacity{0.493103}%
\pgfsetlinewidth{1.003750pt}%
\definecolor{currentstroke}{rgb}{0.121569,0.466667,0.705882}%
\pgfsetstrokecolor{currentstroke}%
\pgfsetstrokeopacity{0.493103}%
\pgfsetdash{}{0pt}%
\pgfpathmoveto{\pgfqpoint{1.208654in}{2.012272in}}%
\pgfpathcurveto{\pgfqpoint{1.216890in}{2.012272in}}{\pgfqpoint{1.224791in}{2.015544in}}{\pgfqpoint{1.230614in}{2.021368in}}%
\pgfpathcurveto{\pgfqpoint{1.236438in}{2.027192in}}{\pgfqpoint{1.239711in}{2.035092in}}{\pgfqpoint{1.239711in}{2.043329in}}%
\pgfpathcurveto{\pgfqpoint{1.239711in}{2.051565in}}{\pgfqpoint{1.236438in}{2.059465in}}{\pgfqpoint{1.230614in}{2.065289in}}%
\pgfpathcurveto{\pgfqpoint{1.224791in}{2.071113in}}{\pgfqpoint{1.216890in}{2.074385in}}{\pgfqpoint{1.208654in}{2.074385in}}%
\pgfpathcurveto{\pgfqpoint{1.200418in}{2.074385in}}{\pgfqpoint{1.192518in}{2.071113in}}{\pgfqpoint{1.186694in}{2.065289in}}%
\pgfpathcurveto{\pgfqpoint{1.180870in}{2.059465in}}{\pgfqpoint{1.177598in}{2.051565in}}{\pgfqpoint{1.177598in}{2.043329in}}%
\pgfpathcurveto{\pgfqpoint{1.177598in}{2.035092in}}{\pgfqpoint{1.180870in}{2.027192in}}{\pgfqpoint{1.186694in}{2.021368in}}%
\pgfpathcurveto{\pgfqpoint{1.192518in}{2.015544in}}{\pgfqpoint{1.200418in}{2.012272in}}{\pgfqpoint{1.208654in}{2.012272in}}%
\pgfpathclose%
\pgfusepath{stroke,fill}%
\end{pgfscope}%
\begin{pgfscope}%
\pgfpathrectangle{\pgfqpoint{0.100000in}{0.220728in}}{\pgfqpoint{3.696000in}{3.696000in}}%
\pgfusepath{clip}%
\pgfsetbuttcap%
\pgfsetroundjoin%
\definecolor{currentfill}{rgb}{0.121569,0.466667,0.705882}%
\pgfsetfillcolor{currentfill}%
\pgfsetfillopacity{0.494209}%
\pgfsetlinewidth{1.003750pt}%
\definecolor{currentstroke}{rgb}{0.121569,0.466667,0.705882}%
\pgfsetstrokecolor{currentstroke}%
\pgfsetstrokeopacity{0.494209}%
\pgfsetdash{}{0pt}%
\pgfpathmoveto{\pgfqpoint{2.591314in}{3.091462in}}%
\pgfpathcurveto{\pgfqpoint{2.599551in}{3.091462in}}{\pgfqpoint{2.607451in}{3.094734in}}{\pgfqpoint{2.613275in}{3.100558in}}%
\pgfpathcurveto{\pgfqpoint{2.619099in}{3.106382in}}{\pgfqpoint{2.622371in}{3.114282in}}{\pgfqpoint{2.622371in}{3.122518in}}%
\pgfpathcurveto{\pgfqpoint{2.622371in}{3.130754in}}{\pgfqpoint{2.619099in}{3.138654in}}{\pgfqpoint{2.613275in}{3.144478in}}%
\pgfpathcurveto{\pgfqpoint{2.607451in}{3.150302in}}{\pgfqpoint{2.599551in}{3.153575in}}{\pgfqpoint{2.591314in}{3.153575in}}%
\pgfpathcurveto{\pgfqpoint{2.583078in}{3.153575in}}{\pgfqpoint{2.575178in}{3.150302in}}{\pgfqpoint{2.569354in}{3.144478in}}%
\pgfpathcurveto{\pgfqpoint{2.563530in}{3.138654in}}{\pgfqpoint{2.560258in}{3.130754in}}{\pgfqpoint{2.560258in}{3.122518in}}%
\pgfpathcurveto{\pgfqpoint{2.560258in}{3.114282in}}{\pgfqpoint{2.563530in}{3.106382in}}{\pgfqpoint{2.569354in}{3.100558in}}%
\pgfpathcurveto{\pgfqpoint{2.575178in}{3.094734in}}{\pgfqpoint{2.583078in}{3.091462in}}{\pgfqpoint{2.591314in}{3.091462in}}%
\pgfpathclose%
\pgfusepath{stroke,fill}%
\end{pgfscope}%
\begin{pgfscope}%
\pgfpathrectangle{\pgfqpoint{0.100000in}{0.220728in}}{\pgfqpoint{3.696000in}{3.696000in}}%
\pgfusepath{clip}%
\pgfsetbuttcap%
\pgfsetroundjoin%
\definecolor{currentfill}{rgb}{0.121569,0.466667,0.705882}%
\pgfsetfillcolor{currentfill}%
\pgfsetfillopacity{0.496378}%
\pgfsetlinewidth{1.003750pt}%
\definecolor{currentstroke}{rgb}{0.121569,0.466667,0.705882}%
\pgfsetstrokecolor{currentstroke}%
\pgfsetstrokeopacity{0.496378}%
\pgfsetdash{}{0pt}%
\pgfpathmoveto{\pgfqpoint{2.599399in}{3.089681in}}%
\pgfpathcurveto{\pgfqpoint{2.607635in}{3.089681in}}{\pgfqpoint{2.615535in}{3.092954in}}{\pgfqpoint{2.621359in}{3.098777in}}%
\pgfpathcurveto{\pgfqpoint{2.627183in}{3.104601in}}{\pgfqpoint{2.630456in}{3.112501in}}{\pgfqpoint{2.630456in}{3.120738in}}%
\pgfpathcurveto{\pgfqpoint{2.630456in}{3.128974in}}{\pgfqpoint{2.627183in}{3.136874in}}{\pgfqpoint{2.621359in}{3.142698in}}%
\pgfpathcurveto{\pgfqpoint{2.615535in}{3.148522in}}{\pgfqpoint{2.607635in}{3.151794in}}{\pgfqpoint{2.599399in}{3.151794in}}%
\pgfpathcurveto{\pgfqpoint{2.591163in}{3.151794in}}{\pgfqpoint{2.583263in}{3.148522in}}{\pgfqpoint{2.577439in}{3.142698in}}%
\pgfpathcurveto{\pgfqpoint{2.571615in}{3.136874in}}{\pgfqpoint{2.568343in}{3.128974in}}{\pgfqpoint{2.568343in}{3.120738in}}%
\pgfpathcurveto{\pgfqpoint{2.568343in}{3.112501in}}{\pgfqpoint{2.571615in}{3.104601in}}{\pgfqpoint{2.577439in}{3.098777in}}%
\pgfpathcurveto{\pgfqpoint{2.583263in}{3.092954in}}{\pgfqpoint{2.591163in}{3.089681in}}{\pgfqpoint{2.599399in}{3.089681in}}%
\pgfpathclose%
\pgfusepath{stroke,fill}%
\end{pgfscope}%
\begin{pgfscope}%
\pgfpathrectangle{\pgfqpoint{0.100000in}{0.220728in}}{\pgfqpoint{3.696000in}{3.696000in}}%
\pgfusepath{clip}%
\pgfsetbuttcap%
\pgfsetroundjoin%
\definecolor{currentfill}{rgb}{0.121569,0.466667,0.705882}%
\pgfsetfillcolor{currentfill}%
\pgfsetfillopacity{0.497409}%
\pgfsetlinewidth{1.003750pt}%
\definecolor{currentstroke}{rgb}{0.121569,0.466667,0.705882}%
\pgfsetstrokecolor{currentstroke}%
\pgfsetstrokeopacity{0.497409}%
\pgfsetdash{}{0pt}%
\pgfpathmoveto{\pgfqpoint{1.201961in}{1.986763in}}%
\pgfpathcurveto{\pgfqpoint{1.210198in}{1.986763in}}{\pgfqpoint{1.218098in}{1.990035in}}{\pgfqpoint{1.223922in}{1.995859in}}%
\pgfpathcurveto{\pgfqpoint{1.229746in}{2.001683in}}{\pgfqpoint{1.233018in}{2.009583in}}{\pgfqpoint{1.233018in}{2.017819in}}%
\pgfpathcurveto{\pgfqpoint{1.233018in}{2.026056in}}{\pgfqpoint{1.229746in}{2.033956in}}{\pgfqpoint{1.223922in}{2.039780in}}%
\pgfpathcurveto{\pgfqpoint{1.218098in}{2.045604in}}{\pgfqpoint{1.210198in}{2.048876in}}{\pgfqpoint{1.201961in}{2.048876in}}%
\pgfpathcurveto{\pgfqpoint{1.193725in}{2.048876in}}{\pgfqpoint{1.185825in}{2.045604in}}{\pgfqpoint{1.180001in}{2.039780in}}%
\pgfpathcurveto{\pgfqpoint{1.174177in}{2.033956in}}{\pgfqpoint{1.170905in}{2.026056in}}{\pgfqpoint{1.170905in}{2.017819in}}%
\pgfpathcurveto{\pgfqpoint{1.170905in}{2.009583in}}{\pgfqpoint{1.174177in}{2.001683in}}{\pgfqpoint{1.180001in}{1.995859in}}%
\pgfpathcurveto{\pgfqpoint{1.185825in}{1.990035in}}{\pgfqpoint{1.193725in}{1.986763in}}{\pgfqpoint{1.201961in}{1.986763in}}%
\pgfpathclose%
\pgfusepath{stroke,fill}%
\end{pgfscope}%
\begin{pgfscope}%
\pgfpathrectangle{\pgfqpoint{0.100000in}{0.220728in}}{\pgfqpoint{3.696000in}{3.696000in}}%
\pgfusepath{clip}%
\pgfsetbuttcap%
\pgfsetroundjoin%
\definecolor{currentfill}{rgb}{0.121569,0.466667,0.705882}%
\pgfsetfillcolor{currentfill}%
\pgfsetfillopacity{0.498265}%
\pgfsetlinewidth{1.003750pt}%
\definecolor{currentstroke}{rgb}{0.121569,0.466667,0.705882}%
\pgfsetstrokecolor{currentstroke}%
\pgfsetstrokeopacity{0.498265}%
\pgfsetdash{}{0pt}%
\pgfpathmoveto{\pgfqpoint{2.609169in}{3.086791in}}%
\pgfpathcurveto{\pgfqpoint{2.617405in}{3.086791in}}{\pgfqpoint{2.625305in}{3.090063in}}{\pgfqpoint{2.631129in}{3.095887in}}%
\pgfpathcurveto{\pgfqpoint{2.636953in}{3.101711in}}{\pgfqpoint{2.640225in}{3.109611in}}{\pgfqpoint{2.640225in}{3.117848in}}%
\pgfpathcurveto{\pgfqpoint{2.640225in}{3.126084in}}{\pgfqpoint{2.636953in}{3.133984in}}{\pgfqpoint{2.631129in}{3.139808in}}%
\pgfpathcurveto{\pgfqpoint{2.625305in}{3.145632in}}{\pgfqpoint{2.617405in}{3.148904in}}{\pgfqpoint{2.609169in}{3.148904in}}%
\pgfpathcurveto{\pgfqpoint{2.600933in}{3.148904in}}{\pgfqpoint{2.593033in}{3.145632in}}{\pgfqpoint{2.587209in}{3.139808in}}%
\pgfpathcurveto{\pgfqpoint{2.581385in}{3.133984in}}{\pgfqpoint{2.578112in}{3.126084in}}{\pgfqpoint{2.578112in}{3.117848in}}%
\pgfpathcurveto{\pgfqpoint{2.578112in}{3.109611in}}{\pgfqpoint{2.581385in}{3.101711in}}{\pgfqpoint{2.587209in}{3.095887in}}%
\pgfpathcurveto{\pgfqpoint{2.593033in}{3.090063in}}{\pgfqpoint{2.600933in}{3.086791in}}{\pgfqpoint{2.609169in}{3.086791in}}%
\pgfpathclose%
\pgfusepath{stroke,fill}%
\end{pgfscope}%
\begin{pgfscope}%
\pgfpathrectangle{\pgfqpoint{0.100000in}{0.220728in}}{\pgfqpoint{3.696000in}{3.696000in}}%
\pgfusepath{clip}%
\pgfsetbuttcap%
\pgfsetroundjoin%
\definecolor{currentfill}{rgb}{0.121569,0.466667,0.705882}%
\pgfsetfillcolor{currentfill}%
\pgfsetfillopacity{0.500748}%
\pgfsetlinewidth{1.003750pt}%
\definecolor{currentstroke}{rgb}{0.121569,0.466667,0.705882}%
\pgfsetstrokecolor{currentstroke}%
\pgfsetstrokeopacity{0.500748}%
\pgfsetdash{}{0pt}%
\pgfpathmoveto{\pgfqpoint{1.189086in}{1.964140in}}%
\pgfpathcurveto{\pgfqpoint{1.197322in}{1.964140in}}{\pgfqpoint{1.205222in}{1.967412in}}{\pgfqpoint{1.211046in}{1.973236in}}%
\pgfpathcurveto{\pgfqpoint{1.216870in}{1.979060in}}{\pgfqpoint{1.220142in}{1.986960in}}{\pgfqpoint{1.220142in}{1.995197in}}%
\pgfpathcurveto{\pgfqpoint{1.220142in}{2.003433in}}{\pgfqpoint{1.216870in}{2.011333in}}{\pgfqpoint{1.211046in}{2.017157in}}%
\pgfpathcurveto{\pgfqpoint{1.205222in}{2.022981in}}{\pgfqpoint{1.197322in}{2.026253in}}{\pgfqpoint{1.189086in}{2.026253in}}%
\pgfpathcurveto{\pgfqpoint{1.180849in}{2.026253in}}{\pgfqpoint{1.172949in}{2.022981in}}{\pgfqpoint{1.167125in}{2.017157in}}%
\pgfpathcurveto{\pgfqpoint{1.161301in}{2.011333in}}{\pgfqpoint{1.158029in}{2.003433in}}{\pgfqpoint{1.158029in}{1.995197in}}%
\pgfpathcurveto{\pgfqpoint{1.158029in}{1.986960in}}{\pgfqpoint{1.161301in}{1.979060in}}{\pgfqpoint{1.167125in}{1.973236in}}%
\pgfpathcurveto{\pgfqpoint{1.172949in}{1.967412in}}{\pgfqpoint{1.180849in}{1.964140in}}{\pgfqpoint{1.189086in}{1.964140in}}%
\pgfpathclose%
\pgfusepath{stroke,fill}%
\end{pgfscope}%
\begin{pgfscope}%
\pgfpathrectangle{\pgfqpoint{0.100000in}{0.220728in}}{\pgfqpoint{3.696000in}{3.696000in}}%
\pgfusepath{clip}%
\pgfsetbuttcap%
\pgfsetroundjoin%
\definecolor{currentfill}{rgb}{0.121569,0.466667,0.705882}%
\pgfsetfillcolor{currentfill}%
\pgfsetfillopacity{0.501274}%
\pgfsetlinewidth{1.003750pt}%
\definecolor{currentstroke}{rgb}{0.121569,0.466667,0.705882}%
\pgfsetstrokecolor{currentstroke}%
\pgfsetstrokeopacity{0.501274}%
\pgfsetdash{}{0pt}%
\pgfpathmoveto{\pgfqpoint{2.619262in}{3.085464in}}%
\pgfpathcurveto{\pgfqpoint{2.627498in}{3.085464in}}{\pgfqpoint{2.635398in}{3.088737in}}{\pgfqpoint{2.641222in}{3.094561in}}%
\pgfpathcurveto{\pgfqpoint{2.647046in}{3.100385in}}{\pgfqpoint{2.650318in}{3.108285in}}{\pgfqpoint{2.650318in}{3.116521in}}%
\pgfpathcurveto{\pgfqpoint{2.650318in}{3.124757in}}{\pgfqpoint{2.647046in}{3.132657in}}{\pgfqpoint{2.641222in}{3.138481in}}%
\pgfpathcurveto{\pgfqpoint{2.635398in}{3.144305in}}{\pgfqpoint{2.627498in}{3.147577in}}{\pgfqpoint{2.619262in}{3.147577in}}%
\pgfpathcurveto{\pgfqpoint{2.611025in}{3.147577in}}{\pgfqpoint{2.603125in}{3.144305in}}{\pgfqpoint{2.597302in}{3.138481in}}%
\pgfpathcurveto{\pgfqpoint{2.591478in}{3.132657in}}{\pgfqpoint{2.588205in}{3.124757in}}{\pgfqpoint{2.588205in}{3.116521in}}%
\pgfpathcurveto{\pgfqpoint{2.588205in}{3.108285in}}{\pgfqpoint{2.591478in}{3.100385in}}{\pgfqpoint{2.597302in}{3.094561in}}%
\pgfpathcurveto{\pgfqpoint{2.603125in}{3.088737in}}{\pgfqpoint{2.611025in}{3.085464in}}{\pgfqpoint{2.619262in}{3.085464in}}%
\pgfpathclose%
\pgfusepath{stroke,fill}%
\end{pgfscope}%
\begin{pgfscope}%
\pgfpathrectangle{\pgfqpoint{0.100000in}{0.220728in}}{\pgfqpoint{3.696000in}{3.696000in}}%
\pgfusepath{clip}%
\pgfsetbuttcap%
\pgfsetroundjoin%
\definecolor{currentfill}{rgb}{0.121569,0.466667,0.705882}%
\pgfsetfillcolor{currentfill}%
\pgfsetfillopacity{0.502762}%
\pgfsetlinewidth{1.003750pt}%
\definecolor{currentstroke}{rgb}{0.121569,0.466667,0.705882}%
\pgfsetstrokecolor{currentstroke}%
\pgfsetstrokeopacity{0.502762}%
\pgfsetdash{}{0pt}%
\pgfpathmoveto{\pgfqpoint{2.632764in}{3.079789in}}%
\pgfpathcurveto{\pgfqpoint{2.641001in}{3.079789in}}{\pgfqpoint{2.648901in}{3.083062in}}{\pgfqpoint{2.654725in}{3.088886in}}%
\pgfpathcurveto{\pgfqpoint{2.660549in}{3.094710in}}{\pgfqpoint{2.663821in}{3.102610in}}{\pgfqpoint{2.663821in}{3.110846in}}%
\pgfpathcurveto{\pgfqpoint{2.663821in}{3.119082in}}{\pgfqpoint{2.660549in}{3.126982in}}{\pgfqpoint{2.654725in}{3.132806in}}%
\pgfpathcurveto{\pgfqpoint{2.648901in}{3.138630in}}{\pgfqpoint{2.641001in}{3.141902in}}{\pgfqpoint{2.632764in}{3.141902in}}%
\pgfpathcurveto{\pgfqpoint{2.624528in}{3.141902in}}{\pgfqpoint{2.616628in}{3.138630in}}{\pgfqpoint{2.610804in}{3.132806in}}%
\pgfpathcurveto{\pgfqpoint{2.604980in}{3.126982in}}{\pgfqpoint{2.601708in}{3.119082in}}{\pgfqpoint{2.601708in}{3.110846in}}%
\pgfpathcurveto{\pgfqpoint{2.601708in}{3.102610in}}{\pgfqpoint{2.604980in}{3.094710in}}{\pgfqpoint{2.610804in}{3.088886in}}%
\pgfpathcurveto{\pgfqpoint{2.616628in}{3.083062in}}{\pgfqpoint{2.624528in}{3.079789in}}{\pgfqpoint{2.632764in}{3.079789in}}%
\pgfpathclose%
\pgfusepath{stroke,fill}%
\end{pgfscope}%
\begin{pgfscope}%
\pgfpathrectangle{\pgfqpoint{0.100000in}{0.220728in}}{\pgfqpoint{3.696000in}{3.696000in}}%
\pgfusepath{clip}%
\pgfsetbuttcap%
\pgfsetroundjoin%
\definecolor{currentfill}{rgb}{0.121569,0.466667,0.705882}%
\pgfsetfillcolor{currentfill}%
\pgfsetfillopacity{0.504483}%
\pgfsetlinewidth{1.003750pt}%
\definecolor{currentstroke}{rgb}{0.121569,0.466667,0.705882}%
\pgfsetstrokecolor{currentstroke}%
\pgfsetstrokeopacity{0.504483}%
\pgfsetdash{}{0pt}%
\pgfpathmoveto{\pgfqpoint{1.178241in}{1.943272in}}%
\pgfpathcurveto{\pgfqpoint{1.186478in}{1.943272in}}{\pgfqpoint{1.194378in}{1.946545in}}{\pgfqpoint{1.200202in}{1.952369in}}%
\pgfpathcurveto{\pgfqpoint{1.206026in}{1.958193in}}{\pgfqpoint{1.209298in}{1.966093in}}{\pgfqpoint{1.209298in}{1.974329in}}%
\pgfpathcurveto{\pgfqpoint{1.209298in}{1.982565in}}{\pgfqpoint{1.206026in}{1.990465in}}{\pgfqpoint{1.200202in}{1.996289in}}%
\pgfpathcurveto{\pgfqpoint{1.194378in}{2.002113in}}{\pgfqpoint{1.186478in}{2.005385in}}{\pgfqpoint{1.178241in}{2.005385in}}%
\pgfpathcurveto{\pgfqpoint{1.170005in}{2.005385in}}{\pgfqpoint{1.162105in}{2.002113in}}{\pgfqpoint{1.156281in}{1.996289in}}%
\pgfpathcurveto{\pgfqpoint{1.150457in}{1.990465in}}{\pgfqpoint{1.147185in}{1.982565in}}{\pgfqpoint{1.147185in}{1.974329in}}%
\pgfpathcurveto{\pgfqpoint{1.147185in}{1.966093in}}{\pgfqpoint{1.150457in}{1.958193in}}{\pgfqpoint{1.156281in}{1.952369in}}%
\pgfpathcurveto{\pgfqpoint{1.162105in}{1.946545in}}{\pgfqpoint{1.170005in}{1.943272in}}{\pgfqpoint{1.178241in}{1.943272in}}%
\pgfpathclose%
\pgfusepath{stroke,fill}%
\end{pgfscope}%
\begin{pgfscope}%
\pgfpathrectangle{\pgfqpoint{0.100000in}{0.220728in}}{\pgfqpoint{3.696000in}{3.696000in}}%
\pgfusepath{clip}%
\pgfsetbuttcap%
\pgfsetroundjoin%
\definecolor{currentfill}{rgb}{0.121569,0.466667,0.705882}%
\pgfsetfillcolor{currentfill}%
\pgfsetfillopacity{0.506158}%
\pgfsetlinewidth{1.003750pt}%
\definecolor{currentstroke}{rgb}{0.121569,0.466667,0.705882}%
\pgfsetstrokecolor{currentstroke}%
\pgfsetstrokeopacity{0.506158}%
\pgfsetdash{}{0pt}%
\pgfpathmoveto{\pgfqpoint{2.646837in}{3.077063in}}%
\pgfpathcurveto{\pgfqpoint{2.655074in}{3.077063in}}{\pgfqpoint{2.662974in}{3.080335in}}{\pgfqpoint{2.668798in}{3.086159in}}%
\pgfpathcurveto{\pgfqpoint{2.674622in}{3.091983in}}{\pgfqpoint{2.677894in}{3.099883in}}{\pgfqpoint{2.677894in}{3.108120in}}%
\pgfpathcurveto{\pgfqpoint{2.677894in}{3.116356in}}{\pgfqpoint{2.674622in}{3.124256in}}{\pgfqpoint{2.668798in}{3.130080in}}%
\pgfpathcurveto{\pgfqpoint{2.662974in}{3.135904in}}{\pgfqpoint{2.655074in}{3.139176in}}{\pgfqpoint{2.646837in}{3.139176in}}%
\pgfpathcurveto{\pgfqpoint{2.638601in}{3.139176in}}{\pgfqpoint{2.630701in}{3.135904in}}{\pgfqpoint{2.624877in}{3.130080in}}%
\pgfpathcurveto{\pgfqpoint{2.619053in}{3.124256in}}{\pgfqpoint{2.615781in}{3.116356in}}{\pgfqpoint{2.615781in}{3.108120in}}%
\pgfpathcurveto{\pgfqpoint{2.615781in}{3.099883in}}{\pgfqpoint{2.619053in}{3.091983in}}{\pgfqpoint{2.624877in}{3.086159in}}%
\pgfpathcurveto{\pgfqpoint{2.630701in}{3.080335in}}{\pgfqpoint{2.638601in}{3.077063in}}{\pgfqpoint{2.646837in}{3.077063in}}%
\pgfpathclose%
\pgfusepath{stroke,fill}%
\end{pgfscope}%
\begin{pgfscope}%
\pgfpathrectangle{\pgfqpoint{0.100000in}{0.220728in}}{\pgfqpoint{3.696000in}{3.696000in}}%
\pgfusepath{clip}%
\pgfsetbuttcap%
\pgfsetroundjoin%
\definecolor{currentfill}{rgb}{0.121569,0.466667,0.705882}%
\pgfsetfillcolor{currentfill}%
\pgfsetfillopacity{0.508070}%
\pgfsetlinewidth{1.003750pt}%
\definecolor{currentstroke}{rgb}{0.121569,0.466667,0.705882}%
\pgfsetstrokecolor{currentstroke}%
\pgfsetstrokeopacity{0.508070}%
\pgfsetdash{}{0pt}%
\pgfpathmoveto{\pgfqpoint{1.169077in}{1.921794in}}%
\pgfpathcurveto{\pgfqpoint{1.177313in}{1.921794in}}{\pgfqpoint{1.185213in}{1.925067in}}{\pgfqpoint{1.191037in}{1.930891in}}%
\pgfpathcurveto{\pgfqpoint{1.196861in}{1.936715in}}{\pgfqpoint{1.200133in}{1.944615in}}{\pgfqpoint{1.200133in}{1.952851in}}%
\pgfpathcurveto{\pgfqpoint{1.200133in}{1.961087in}}{\pgfqpoint{1.196861in}{1.968987in}}{\pgfqpoint{1.191037in}{1.974811in}}%
\pgfpathcurveto{\pgfqpoint{1.185213in}{1.980635in}}{\pgfqpoint{1.177313in}{1.983907in}}{\pgfqpoint{1.169077in}{1.983907in}}%
\pgfpathcurveto{\pgfqpoint{1.160841in}{1.983907in}}{\pgfqpoint{1.152941in}{1.980635in}}{\pgfqpoint{1.147117in}{1.974811in}}%
\pgfpathcurveto{\pgfqpoint{1.141293in}{1.968987in}}{\pgfqpoint{1.138020in}{1.961087in}}{\pgfqpoint{1.138020in}{1.952851in}}%
\pgfpathcurveto{\pgfqpoint{1.138020in}{1.944615in}}{\pgfqpoint{1.141293in}{1.936715in}}{\pgfqpoint{1.147117in}{1.930891in}}%
\pgfpathcurveto{\pgfqpoint{1.152941in}{1.925067in}}{\pgfqpoint{1.160841in}{1.921794in}}{\pgfqpoint{1.169077in}{1.921794in}}%
\pgfpathclose%
\pgfusepath{stroke,fill}%
\end{pgfscope}%
\begin{pgfscope}%
\pgfpathrectangle{\pgfqpoint{0.100000in}{0.220728in}}{\pgfqpoint{3.696000in}{3.696000in}}%
\pgfusepath{clip}%
\pgfsetbuttcap%
\pgfsetroundjoin%
\definecolor{currentfill}{rgb}{0.121569,0.466667,0.705882}%
\pgfsetfillcolor{currentfill}%
\pgfsetfillopacity{0.508161}%
\pgfsetlinewidth{1.003750pt}%
\definecolor{currentstroke}{rgb}{0.121569,0.466667,0.705882}%
\pgfsetstrokecolor{currentstroke}%
\pgfsetstrokeopacity{0.508161}%
\pgfsetdash{}{0pt}%
\pgfpathmoveto{\pgfqpoint{2.654252in}{3.075105in}}%
\pgfpathcurveto{\pgfqpoint{2.662488in}{3.075105in}}{\pgfqpoint{2.670388in}{3.078378in}}{\pgfqpoint{2.676212in}{3.084202in}}%
\pgfpathcurveto{\pgfqpoint{2.682036in}{3.090026in}}{\pgfqpoint{2.685309in}{3.097926in}}{\pgfqpoint{2.685309in}{3.106162in}}%
\pgfpathcurveto{\pgfqpoint{2.685309in}{3.114398in}}{\pgfqpoint{2.682036in}{3.122298in}}{\pgfqpoint{2.676212in}{3.128122in}}%
\pgfpathcurveto{\pgfqpoint{2.670388in}{3.133946in}}{\pgfqpoint{2.662488in}{3.137218in}}{\pgfqpoint{2.654252in}{3.137218in}}%
\pgfpathcurveto{\pgfqpoint{2.646016in}{3.137218in}}{\pgfqpoint{2.638116in}{3.133946in}}{\pgfqpoint{2.632292in}{3.128122in}}%
\pgfpathcurveto{\pgfqpoint{2.626468in}{3.122298in}}{\pgfqpoint{2.623196in}{3.114398in}}{\pgfqpoint{2.623196in}{3.106162in}}%
\pgfpathcurveto{\pgfqpoint{2.623196in}{3.097926in}}{\pgfqpoint{2.626468in}{3.090026in}}{\pgfqpoint{2.632292in}{3.084202in}}%
\pgfpathcurveto{\pgfqpoint{2.638116in}{3.078378in}}{\pgfqpoint{2.646016in}{3.075105in}}{\pgfqpoint{2.654252in}{3.075105in}}%
\pgfpathclose%
\pgfusepath{stroke,fill}%
\end{pgfscope}%
\begin{pgfscope}%
\pgfpathrectangle{\pgfqpoint{0.100000in}{0.220728in}}{\pgfqpoint{3.696000in}{3.696000in}}%
\pgfusepath{clip}%
\pgfsetbuttcap%
\pgfsetroundjoin%
\definecolor{currentfill}{rgb}{0.121569,0.466667,0.705882}%
\pgfsetfillcolor{currentfill}%
\pgfsetfillopacity{0.509848}%
\pgfsetlinewidth{1.003750pt}%
\definecolor{currentstroke}{rgb}{0.121569,0.466667,0.705882}%
\pgfsetstrokecolor{currentstroke}%
\pgfsetstrokeopacity{0.509848}%
\pgfsetdash{}{0pt}%
\pgfpathmoveto{\pgfqpoint{2.663042in}{3.072732in}}%
\pgfpathcurveto{\pgfqpoint{2.671278in}{3.072732in}}{\pgfqpoint{2.679178in}{3.076004in}}{\pgfqpoint{2.685002in}{3.081828in}}%
\pgfpathcurveto{\pgfqpoint{2.690826in}{3.087652in}}{\pgfqpoint{2.694098in}{3.095552in}}{\pgfqpoint{2.694098in}{3.103788in}}%
\pgfpathcurveto{\pgfqpoint{2.694098in}{3.112024in}}{\pgfqpoint{2.690826in}{3.119924in}}{\pgfqpoint{2.685002in}{3.125748in}}%
\pgfpathcurveto{\pgfqpoint{2.679178in}{3.131572in}}{\pgfqpoint{2.671278in}{3.134845in}}{\pgfqpoint{2.663042in}{3.134845in}}%
\pgfpathcurveto{\pgfqpoint{2.654806in}{3.134845in}}{\pgfqpoint{2.646906in}{3.131572in}}{\pgfqpoint{2.641082in}{3.125748in}}%
\pgfpathcurveto{\pgfqpoint{2.635258in}{3.119924in}}{\pgfqpoint{2.631985in}{3.112024in}}{\pgfqpoint{2.631985in}{3.103788in}}%
\pgfpathcurveto{\pgfqpoint{2.631985in}{3.095552in}}{\pgfqpoint{2.635258in}{3.087652in}}{\pgfqpoint{2.641082in}{3.081828in}}%
\pgfpathcurveto{\pgfqpoint{2.646906in}{3.076004in}}{\pgfqpoint{2.654806in}{3.072732in}}{\pgfqpoint{2.663042in}{3.072732in}}%
\pgfpathclose%
\pgfusepath{stroke,fill}%
\end{pgfscope}%
\begin{pgfscope}%
\pgfpathrectangle{\pgfqpoint{0.100000in}{0.220728in}}{\pgfqpoint{3.696000in}{3.696000in}}%
\pgfusepath{clip}%
\pgfsetbuttcap%
\pgfsetroundjoin%
\definecolor{currentfill}{rgb}{0.121569,0.466667,0.705882}%
\pgfsetfillcolor{currentfill}%
\pgfsetfillopacity{0.510816}%
\pgfsetlinewidth{1.003750pt}%
\definecolor{currentstroke}{rgb}{0.121569,0.466667,0.705882}%
\pgfsetstrokecolor{currentstroke}%
\pgfsetstrokeopacity{0.510816}%
\pgfsetdash{}{0pt}%
\pgfpathmoveto{\pgfqpoint{1.157138in}{1.905085in}}%
\pgfpathcurveto{\pgfqpoint{1.165374in}{1.905085in}}{\pgfqpoint{1.173274in}{1.908357in}}{\pgfqpoint{1.179098in}{1.914181in}}%
\pgfpathcurveto{\pgfqpoint{1.184922in}{1.920005in}}{\pgfqpoint{1.188194in}{1.927905in}}{\pgfqpoint{1.188194in}{1.936141in}}%
\pgfpathcurveto{\pgfqpoint{1.188194in}{1.944378in}}{\pgfqpoint{1.184922in}{1.952278in}}{\pgfqpoint{1.179098in}{1.958102in}}%
\pgfpathcurveto{\pgfqpoint{1.173274in}{1.963926in}}{\pgfqpoint{1.165374in}{1.967198in}}{\pgfqpoint{1.157138in}{1.967198in}}%
\pgfpathcurveto{\pgfqpoint{1.148902in}{1.967198in}}{\pgfqpoint{1.141001in}{1.963926in}}{\pgfqpoint{1.135178in}{1.958102in}}%
\pgfpathcurveto{\pgfqpoint{1.129354in}{1.952278in}}{\pgfqpoint{1.126081in}{1.944378in}}{\pgfqpoint{1.126081in}{1.936141in}}%
\pgfpathcurveto{\pgfqpoint{1.126081in}{1.927905in}}{\pgfqpoint{1.129354in}{1.920005in}}{\pgfqpoint{1.135178in}{1.914181in}}%
\pgfpathcurveto{\pgfqpoint{1.141001in}{1.908357in}}{\pgfqpoint{1.148902in}{1.905085in}}{\pgfqpoint{1.157138in}{1.905085in}}%
\pgfpathclose%
\pgfusepath{stroke,fill}%
\end{pgfscope}%
\begin{pgfscope}%
\pgfpathrectangle{\pgfqpoint{0.100000in}{0.220728in}}{\pgfqpoint{3.696000in}{3.696000in}}%
\pgfusepath{clip}%
\pgfsetbuttcap%
\pgfsetroundjoin%
\definecolor{currentfill}{rgb}{0.121569,0.466667,0.705882}%
\pgfsetfillcolor{currentfill}%
\pgfsetfillopacity{0.512776}%
\pgfsetlinewidth{1.003750pt}%
\definecolor{currentstroke}{rgb}{0.121569,0.466667,0.705882}%
\pgfsetstrokecolor{currentstroke}%
\pgfsetstrokeopacity{0.512776}%
\pgfsetdash{}{0pt}%
\pgfpathmoveto{\pgfqpoint{2.672914in}{3.070910in}}%
\pgfpathcurveto{\pgfqpoint{2.681150in}{3.070910in}}{\pgfqpoint{2.689050in}{3.074182in}}{\pgfqpoint{2.694874in}{3.080006in}}%
\pgfpathcurveto{\pgfqpoint{2.700698in}{3.085830in}}{\pgfqpoint{2.703970in}{3.093730in}}{\pgfqpoint{2.703970in}{3.101966in}}%
\pgfpathcurveto{\pgfqpoint{2.703970in}{3.110203in}}{\pgfqpoint{2.700698in}{3.118103in}}{\pgfqpoint{2.694874in}{3.123927in}}%
\pgfpathcurveto{\pgfqpoint{2.689050in}{3.129751in}}{\pgfqpoint{2.681150in}{3.133023in}}{\pgfqpoint{2.672914in}{3.133023in}}%
\pgfpathcurveto{\pgfqpoint{2.664678in}{3.133023in}}{\pgfqpoint{2.656777in}{3.129751in}}{\pgfqpoint{2.650954in}{3.123927in}}%
\pgfpathcurveto{\pgfqpoint{2.645130in}{3.118103in}}{\pgfqpoint{2.641857in}{3.110203in}}{\pgfqpoint{2.641857in}{3.101966in}}%
\pgfpathcurveto{\pgfqpoint{2.641857in}{3.093730in}}{\pgfqpoint{2.645130in}{3.085830in}}{\pgfqpoint{2.650954in}{3.080006in}}%
\pgfpathcurveto{\pgfqpoint{2.656777in}{3.074182in}}{\pgfqpoint{2.664678in}{3.070910in}}{\pgfqpoint{2.672914in}{3.070910in}}%
\pgfpathclose%
\pgfusepath{stroke,fill}%
\end{pgfscope}%
\begin{pgfscope}%
\pgfpathrectangle{\pgfqpoint{0.100000in}{0.220728in}}{\pgfqpoint{3.696000in}{3.696000in}}%
\pgfusepath{clip}%
\pgfsetbuttcap%
\pgfsetroundjoin%
\definecolor{currentfill}{rgb}{0.121569,0.466667,0.705882}%
\pgfsetfillcolor{currentfill}%
\pgfsetfillopacity{0.513892}%
\pgfsetlinewidth{1.003750pt}%
\definecolor{currentstroke}{rgb}{0.121569,0.466667,0.705882}%
\pgfsetstrokecolor{currentstroke}%
\pgfsetstrokeopacity{0.513892}%
\pgfsetdash{}{0pt}%
\pgfpathmoveto{\pgfqpoint{2.678746in}{3.069153in}}%
\pgfpathcurveto{\pgfqpoint{2.686982in}{3.069153in}}{\pgfqpoint{2.694882in}{3.072425in}}{\pgfqpoint{2.700706in}{3.078249in}}%
\pgfpathcurveto{\pgfqpoint{2.706530in}{3.084073in}}{\pgfqpoint{2.709802in}{3.091973in}}{\pgfqpoint{2.709802in}{3.100209in}}%
\pgfpathcurveto{\pgfqpoint{2.709802in}{3.108446in}}{\pgfqpoint{2.706530in}{3.116346in}}{\pgfqpoint{2.700706in}{3.122170in}}%
\pgfpathcurveto{\pgfqpoint{2.694882in}{3.127994in}}{\pgfqpoint{2.686982in}{3.131266in}}{\pgfqpoint{2.678746in}{3.131266in}}%
\pgfpathcurveto{\pgfqpoint{2.670510in}{3.131266in}}{\pgfqpoint{2.662610in}{3.127994in}}{\pgfqpoint{2.656786in}{3.122170in}}%
\pgfpathcurveto{\pgfqpoint{2.650962in}{3.116346in}}{\pgfqpoint{2.647689in}{3.108446in}}{\pgfqpoint{2.647689in}{3.100209in}}%
\pgfpathcurveto{\pgfqpoint{2.647689in}{3.091973in}}{\pgfqpoint{2.650962in}{3.084073in}}{\pgfqpoint{2.656786in}{3.078249in}}%
\pgfpathcurveto{\pgfqpoint{2.662610in}{3.072425in}}{\pgfqpoint{2.670510in}{3.069153in}}{\pgfqpoint{2.678746in}{3.069153in}}%
\pgfpathclose%
\pgfusepath{stroke,fill}%
\end{pgfscope}%
\begin{pgfscope}%
\pgfpathrectangle{\pgfqpoint{0.100000in}{0.220728in}}{\pgfqpoint{3.696000in}{3.696000in}}%
\pgfusepath{clip}%
\pgfsetbuttcap%
\pgfsetroundjoin%
\definecolor{currentfill}{rgb}{0.121569,0.466667,0.705882}%
\pgfsetfillcolor{currentfill}%
\pgfsetfillopacity{0.513953}%
\pgfsetlinewidth{1.003750pt}%
\definecolor{currentstroke}{rgb}{0.121569,0.466667,0.705882}%
\pgfsetstrokecolor{currentstroke}%
\pgfsetstrokeopacity{0.513953}%
\pgfsetdash{}{0pt}%
\pgfpathmoveto{\pgfqpoint{1.152084in}{1.885574in}}%
\pgfpathcurveto{\pgfqpoint{1.160320in}{1.885574in}}{\pgfqpoint{1.168220in}{1.888847in}}{\pgfqpoint{1.174044in}{1.894670in}}%
\pgfpathcurveto{\pgfqpoint{1.179868in}{1.900494in}}{\pgfqpoint{1.183141in}{1.908394in}}{\pgfqpoint{1.183141in}{1.916631in}}%
\pgfpathcurveto{\pgfqpoint{1.183141in}{1.924867in}}{\pgfqpoint{1.179868in}{1.932767in}}{\pgfqpoint{1.174044in}{1.938591in}}%
\pgfpathcurveto{\pgfqpoint{1.168220in}{1.944415in}}{\pgfqpoint{1.160320in}{1.947687in}}{\pgfqpoint{1.152084in}{1.947687in}}%
\pgfpathcurveto{\pgfqpoint{1.143848in}{1.947687in}}{\pgfqpoint{1.135948in}{1.944415in}}{\pgfqpoint{1.130124in}{1.938591in}}%
\pgfpathcurveto{\pgfqpoint{1.124300in}{1.932767in}}{\pgfqpoint{1.121028in}{1.924867in}}{\pgfqpoint{1.121028in}{1.916631in}}%
\pgfpathcurveto{\pgfqpoint{1.121028in}{1.908394in}}{\pgfqpoint{1.124300in}{1.900494in}}{\pgfqpoint{1.130124in}{1.894670in}}%
\pgfpathcurveto{\pgfqpoint{1.135948in}{1.888847in}}{\pgfqpoint{1.143848in}{1.885574in}}{\pgfqpoint{1.152084in}{1.885574in}}%
\pgfpathclose%
\pgfusepath{stroke,fill}%
\end{pgfscope}%
\begin{pgfscope}%
\pgfpathrectangle{\pgfqpoint{0.100000in}{0.220728in}}{\pgfqpoint{3.696000in}{3.696000in}}%
\pgfusepath{clip}%
\pgfsetbuttcap%
\pgfsetroundjoin%
\definecolor{currentfill}{rgb}{0.121569,0.466667,0.705882}%
\pgfsetfillcolor{currentfill}%
\pgfsetfillopacity{0.514335}%
\pgfsetlinewidth{1.003750pt}%
\definecolor{currentstroke}{rgb}{0.121569,0.466667,0.705882}%
\pgfsetstrokecolor{currentstroke}%
\pgfsetstrokeopacity{0.514335}%
\pgfsetdash{}{0pt}%
\pgfpathmoveto{\pgfqpoint{2.686224in}{3.066284in}}%
\pgfpathcurveto{\pgfqpoint{2.694461in}{3.066284in}}{\pgfqpoint{2.702361in}{3.069557in}}{\pgfqpoint{2.708185in}{3.075381in}}%
\pgfpathcurveto{\pgfqpoint{2.714009in}{3.081204in}}{\pgfqpoint{2.717281in}{3.089104in}}{\pgfqpoint{2.717281in}{3.097341in}}%
\pgfpathcurveto{\pgfqpoint{2.717281in}{3.105577in}}{\pgfqpoint{2.714009in}{3.113477in}}{\pgfqpoint{2.708185in}{3.119301in}}%
\pgfpathcurveto{\pgfqpoint{2.702361in}{3.125125in}}{\pgfqpoint{2.694461in}{3.128397in}}{\pgfqpoint{2.686224in}{3.128397in}}%
\pgfpathcurveto{\pgfqpoint{2.677988in}{3.128397in}}{\pgfqpoint{2.670088in}{3.125125in}}{\pgfqpoint{2.664264in}{3.119301in}}%
\pgfpathcurveto{\pgfqpoint{2.658440in}{3.113477in}}{\pgfqpoint{2.655168in}{3.105577in}}{\pgfqpoint{2.655168in}{3.097341in}}%
\pgfpathcurveto{\pgfqpoint{2.655168in}{3.089104in}}{\pgfqpoint{2.658440in}{3.081204in}}{\pgfqpoint{2.664264in}{3.075381in}}%
\pgfpathcurveto{\pgfqpoint{2.670088in}{3.069557in}}{\pgfqpoint{2.677988in}{3.066284in}}{\pgfqpoint{2.686224in}{3.066284in}}%
\pgfpathclose%
\pgfusepath{stroke,fill}%
\end{pgfscope}%
\begin{pgfscope}%
\pgfpathrectangle{\pgfqpoint{0.100000in}{0.220728in}}{\pgfqpoint{3.696000in}{3.696000in}}%
\pgfusepath{clip}%
\pgfsetbuttcap%
\pgfsetroundjoin%
\definecolor{currentfill}{rgb}{0.121569,0.466667,0.705882}%
\pgfsetfillcolor{currentfill}%
\pgfsetfillopacity{0.515507}%
\pgfsetlinewidth{1.003750pt}%
\definecolor{currentstroke}{rgb}{0.121569,0.466667,0.705882}%
\pgfsetstrokecolor{currentstroke}%
\pgfsetstrokeopacity{0.515507}%
\pgfsetdash{}{0pt}%
\pgfpathmoveto{\pgfqpoint{1.141991in}{1.873876in}}%
\pgfpathcurveto{\pgfqpoint{1.150227in}{1.873876in}}{\pgfqpoint{1.158127in}{1.877148in}}{\pgfqpoint{1.163951in}{1.882972in}}%
\pgfpathcurveto{\pgfqpoint{1.169775in}{1.888796in}}{\pgfqpoint{1.173048in}{1.896696in}}{\pgfqpoint{1.173048in}{1.904932in}}%
\pgfpathcurveto{\pgfqpoint{1.173048in}{1.913169in}}{\pgfqpoint{1.169775in}{1.921069in}}{\pgfqpoint{1.163951in}{1.926893in}}%
\pgfpathcurveto{\pgfqpoint{1.158127in}{1.932716in}}{\pgfqpoint{1.150227in}{1.935989in}}{\pgfqpoint{1.141991in}{1.935989in}}%
\pgfpathcurveto{\pgfqpoint{1.133755in}{1.935989in}}{\pgfqpoint{1.125855in}{1.932716in}}{\pgfqpoint{1.120031in}{1.926893in}}%
\pgfpathcurveto{\pgfqpoint{1.114207in}{1.921069in}}{\pgfqpoint{1.110935in}{1.913169in}}{\pgfqpoint{1.110935in}{1.904932in}}%
\pgfpathcurveto{\pgfqpoint{1.110935in}{1.896696in}}{\pgfqpoint{1.114207in}{1.888796in}}{\pgfqpoint{1.120031in}{1.882972in}}%
\pgfpathcurveto{\pgfqpoint{1.125855in}{1.877148in}}{\pgfqpoint{1.133755in}{1.873876in}}{\pgfqpoint{1.141991in}{1.873876in}}%
\pgfpathclose%
\pgfusepath{stroke,fill}%
\end{pgfscope}%
\begin{pgfscope}%
\pgfpathrectangle{\pgfqpoint{0.100000in}{0.220728in}}{\pgfqpoint{3.696000in}{3.696000in}}%
\pgfusepath{clip}%
\pgfsetbuttcap%
\pgfsetroundjoin%
\definecolor{currentfill}{rgb}{0.121569,0.466667,0.705882}%
\pgfsetfillcolor{currentfill}%
\pgfsetfillopacity{0.517129}%
\pgfsetlinewidth{1.003750pt}%
\definecolor{currentstroke}{rgb}{0.121569,0.466667,0.705882}%
\pgfsetstrokecolor{currentstroke}%
\pgfsetstrokeopacity{0.517129}%
\pgfsetdash{}{0pt}%
\pgfpathmoveto{\pgfqpoint{2.694141in}{3.065441in}}%
\pgfpathcurveto{\pgfqpoint{2.702377in}{3.065441in}}{\pgfqpoint{2.710277in}{3.068714in}}{\pgfqpoint{2.716101in}{3.074538in}}%
\pgfpathcurveto{\pgfqpoint{2.721925in}{3.080361in}}{\pgfqpoint{2.725197in}{3.088262in}}{\pgfqpoint{2.725197in}{3.096498in}}%
\pgfpathcurveto{\pgfqpoint{2.725197in}{3.104734in}}{\pgfqpoint{2.721925in}{3.112634in}}{\pgfqpoint{2.716101in}{3.118458in}}%
\pgfpathcurveto{\pgfqpoint{2.710277in}{3.124282in}}{\pgfqpoint{2.702377in}{3.127554in}}{\pgfqpoint{2.694141in}{3.127554in}}%
\pgfpathcurveto{\pgfqpoint{2.685904in}{3.127554in}}{\pgfqpoint{2.678004in}{3.124282in}}{\pgfqpoint{2.672180in}{3.118458in}}%
\pgfpathcurveto{\pgfqpoint{2.666356in}{3.112634in}}{\pgfqpoint{2.663084in}{3.104734in}}{\pgfqpoint{2.663084in}{3.096498in}}%
\pgfpathcurveto{\pgfqpoint{2.663084in}{3.088262in}}{\pgfqpoint{2.666356in}{3.080361in}}{\pgfqpoint{2.672180in}{3.074538in}}%
\pgfpathcurveto{\pgfqpoint{2.678004in}{3.068714in}}{\pgfqpoint{2.685904in}{3.065441in}}{\pgfqpoint{2.694141in}{3.065441in}}%
\pgfpathclose%
\pgfusepath{stroke,fill}%
\end{pgfscope}%
\begin{pgfscope}%
\pgfpathrectangle{\pgfqpoint{0.100000in}{0.220728in}}{\pgfqpoint{3.696000in}{3.696000in}}%
\pgfusepath{clip}%
\pgfsetbuttcap%
\pgfsetroundjoin%
\definecolor{currentfill}{rgb}{0.121569,0.466667,0.705882}%
\pgfsetfillcolor{currentfill}%
\pgfsetfillopacity{0.517280}%
\pgfsetlinewidth{1.003750pt}%
\definecolor{currentstroke}{rgb}{0.121569,0.466667,0.705882}%
\pgfsetstrokecolor{currentstroke}%
\pgfsetstrokeopacity{0.517280}%
\pgfsetdash{}{0pt}%
\pgfpathmoveto{\pgfqpoint{1.139972in}{1.861851in}}%
\pgfpathcurveto{\pgfqpoint{1.148209in}{1.861851in}}{\pgfqpoint{1.156109in}{1.865124in}}{\pgfqpoint{1.161933in}{1.870948in}}%
\pgfpathcurveto{\pgfqpoint{1.167757in}{1.876772in}}{\pgfqpoint{1.171029in}{1.884672in}}{\pgfqpoint{1.171029in}{1.892908in}}%
\pgfpathcurveto{\pgfqpoint{1.171029in}{1.901144in}}{\pgfqpoint{1.167757in}{1.909044in}}{\pgfqpoint{1.161933in}{1.914868in}}%
\pgfpathcurveto{\pgfqpoint{1.156109in}{1.920692in}}{\pgfqpoint{1.148209in}{1.923964in}}{\pgfqpoint{1.139972in}{1.923964in}}%
\pgfpathcurveto{\pgfqpoint{1.131736in}{1.923964in}}{\pgfqpoint{1.123836in}{1.920692in}}{\pgfqpoint{1.118012in}{1.914868in}}%
\pgfpathcurveto{\pgfqpoint{1.112188in}{1.909044in}}{\pgfqpoint{1.108916in}{1.901144in}}{\pgfqpoint{1.108916in}{1.892908in}}%
\pgfpathcurveto{\pgfqpoint{1.108916in}{1.884672in}}{\pgfqpoint{1.112188in}{1.876772in}}{\pgfqpoint{1.118012in}{1.870948in}}%
\pgfpathcurveto{\pgfqpoint{1.123836in}{1.865124in}}{\pgfqpoint{1.131736in}{1.861851in}}{\pgfqpoint{1.139972in}{1.861851in}}%
\pgfpathclose%
\pgfusepath{stroke,fill}%
\end{pgfscope}%
\begin{pgfscope}%
\pgfpathrectangle{\pgfqpoint{0.100000in}{0.220728in}}{\pgfqpoint{3.696000in}{3.696000in}}%
\pgfusepath{clip}%
\pgfsetbuttcap%
\pgfsetroundjoin%
\definecolor{currentfill}{rgb}{0.121569,0.466667,0.705882}%
\pgfsetfillcolor{currentfill}%
\pgfsetfillopacity{0.518036}%
\pgfsetlinewidth{1.003750pt}%
\definecolor{currentstroke}{rgb}{0.121569,0.466667,0.705882}%
\pgfsetstrokecolor{currentstroke}%
\pgfsetstrokeopacity{0.518036}%
\pgfsetdash{}{0pt}%
\pgfpathmoveto{\pgfqpoint{1.134141in}{1.855773in}}%
\pgfpathcurveto{\pgfqpoint{1.142377in}{1.855773in}}{\pgfqpoint{1.150277in}{1.859045in}}{\pgfqpoint{1.156101in}{1.864869in}}%
\pgfpathcurveto{\pgfqpoint{1.161925in}{1.870693in}}{\pgfqpoint{1.165198in}{1.878593in}}{\pgfqpoint{1.165198in}{1.886830in}}%
\pgfpathcurveto{\pgfqpoint{1.165198in}{1.895066in}}{\pgfqpoint{1.161925in}{1.902966in}}{\pgfqpoint{1.156101in}{1.908790in}}%
\pgfpathcurveto{\pgfqpoint{1.150277in}{1.914614in}}{\pgfqpoint{1.142377in}{1.917886in}}{\pgfqpoint{1.134141in}{1.917886in}}%
\pgfpathcurveto{\pgfqpoint{1.125905in}{1.917886in}}{\pgfqpoint{1.118005in}{1.914614in}}{\pgfqpoint{1.112181in}{1.908790in}}%
\pgfpathcurveto{\pgfqpoint{1.106357in}{1.902966in}}{\pgfqpoint{1.103085in}{1.895066in}}{\pgfqpoint{1.103085in}{1.886830in}}%
\pgfpathcurveto{\pgfqpoint{1.103085in}{1.878593in}}{\pgfqpoint{1.106357in}{1.870693in}}{\pgfqpoint{1.112181in}{1.864869in}}%
\pgfpathcurveto{\pgfqpoint{1.118005in}{1.859045in}}{\pgfqpoint{1.125905in}{1.855773in}}{\pgfqpoint{1.134141in}{1.855773in}}%
\pgfpathclose%
\pgfusepath{stroke,fill}%
\end{pgfscope}%
\begin{pgfscope}%
\pgfpathrectangle{\pgfqpoint{0.100000in}{0.220728in}}{\pgfqpoint{3.696000in}{3.696000in}}%
\pgfusepath{clip}%
\pgfsetbuttcap%
\pgfsetroundjoin%
\definecolor{currentfill}{rgb}{0.121569,0.466667,0.705882}%
\pgfsetfillcolor{currentfill}%
\pgfsetfillopacity{0.518833}%
\pgfsetlinewidth{1.003750pt}%
\definecolor{currentstroke}{rgb}{0.121569,0.466667,0.705882}%
\pgfsetstrokecolor{currentstroke}%
\pgfsetstrokeopacity{0.518833}%
\pgfsetdash{}{0pt}%
\pgfpathmoveto{\pgfqpoint{2.705700in}{3.061996in}}%
\pgfpathcurveto{\pgfqpoint{2.713936in}{3.061996in}}{\pgfqpoint{2.721836in}{3.065268in}}{\pgfqpoint{2.727660in}{3.071092in}}%
\pgfpathcurveto{\pgfqpoint{2.733484in}{3.076916in}}{\pgfqpoint{2.736757in}{3.084816in}}{\pgfqpoint{2.736757in}{3.093053in}}%
\pgfpathcurveto{\pgfqpoint{2.736757in}{3.101289in}}{\pgfqpoint{2.733484in}{3.109189in}}{\pgfqpoint{2.727660in}{3.115013in}}%
\pgfpathcurveto{\pgfqpoint{2.721836in}{3.120837in}}{\pgfqpoint{2.713936in}{3.124109in}}{\pgfqpoint{2.705700in}{3.124109in}}%
\pgfpathcurveto{\pgfqpoint{2.697464in}{3.124109in}}{\pgfqpoint{2.689564in}{3.120837in}}{\pgfqpoint{2.683740in}{3.115013in}}%
\pgfpathcurveto{\pgfqpoint{2.677916in}{3.109189in}}{\pgfqpoint{2.674644in}{3.101289in}}{\pgfqpoint{2.674644in}{3.093053in}}%
\pgfpathcurveto{\pgfqpoint{2.674644in}{3.084816in}}{\pgfqpoint{2.677916in}{3.076916in}}{\pgfqpoint{2.683740in}{3.071092in}}%
\pgfpathcurveto{\pgfqpoint{2.689564in}{3.065268in}}{\pgfqpoint{2.697464in}{3.061996in}}{\pgfqpoint{2.705700in}{3.061996in}}%
\pgfpathclose%
\pgfusepath{stroke,fill}%
\end{pgfscope}%
\begin{pgfscope}%
\pgfpathrectangle{\pgfqpoint{0.100000in}{0.220728in}}{\pgfqpoint{3.696000in}{3.696000in}}%
\pgfusepath{clip}%
\pgfsetbuttcap%
\pgfsetroundjoin%
\definecolor{currentfill}{rgb}{0.121569,0.466667,0.705882}%
\pgfsetfillcolor{currentfill}%
\pgfsetfillopacity{0.519114}%
\pgfsetlinewidth{1.003750pt}%
\definecolor{currentstroke}{rgb}{0.121569,0.466667,0.705882}%
\pgfsetstrokecolor{currentstroke}%
\pgfsetstrokeopacity{0.519114}%
\pgfsetdash{}{0pt}%
\pgfpathmoveto{\pgfqpoint{1.132793in}{1.848680in}}%
\pgfpathcurveto{\pgfqpoint{1.141029in}{1.848680in}}{\pgfqpoint{1.148929in}{1.851953in}}{\pgfqpoint{1.154753in}{1.857777in}}%
\pgfpathcurveto{\pgfqpoint{1.160577in}{1.863601in}}{\pgfqpoint{1.163849in}{1.871501in}}{\pgfqpoint{1.163849in}{1.879737in}}%
\pgfpathcurveto{\pgfqpoint{1.163849in}{1.887973in}}{\pgfqpoint{1.160577in}{1.895873in}}{\pgfqpoint{1.154753in}{1.901697in}}%
\pgfpathcurveto{\pgfqpoint{1.148929in}{1.907521in}}{\pgfqpoint{1.141029in}{1.910793in}}{\pgfqpoint{1.132793in}{1.910793in}}%
\pgfpathcurveto{\pgfqpoint{1.124557in}{1.910793in}}{\pgfqpoint{1.116657in}{1.907521in}}{\pgfqpoint{1.110833in}{1.901697in}}%
\pgfpathcurveto{\pgfqpoint{1.105009in}{1.895873in}}{\pgfqpoint{1.101736in}{1.887973in}}{\pgfqpoint{1.101736in}{1.879737in}}%
\pgfpathcurveto{\pgfqpoint{1.101736in}{1.871501in}}{\pgfqpoint{1.105009in}{1.863601in}}{\pgfqpoint{1.110833in}{1.857777in}}%
\pgfpathcurveto{\pgfqpoint{1.116657in}{1.851953in}}{\pgfqpoint{1.124557in}{1.848680in}}{\pgfqpoint{1.132793in}{1.848680in}}%
\pgfpathclose%
\pgfusepath{stroke,fill}%
\end{pgfscope}%
\begin{pgfscope}%
\pgfpathrectangle{\pgfqpoint{0.100000in}{0.220728in}}{\pgfqpoint{3.696000in}{3.696000in}}%
\pgfusepath{clip}%
\pgfsetbuttcap%
\pgfsetroundjoin%
\definecolor{currentfill}{rgb}{0.121569,0.466667,0.705882}%
\pgfsetfillcolor{currentfill}%
\pgfsetfillopacity{0.519524}%
\pgfsetlinewidth{1.003750pt}%
\definecolor{currentstroke}{rgb}{0.121569,0.466667,0.705882}%
\pgfsetstrokecolor{currentstroke}%
\pgfsetstrokeopacity{0.519524}%
\pgfsetdash{}{0pt}%
\pgfpathmoveto{\pgfqpoint{1.129225in}{1.844774in}}%
\pgfpathcurveto{\pgfqpoint{1.137461in}{1.844774in}}{\pgfqpoint{1.145361in}{1.848046in}}{\pgfqpoint{1.151185in}{1.853870in}}%
\pgfpathcurveto{\pgfqpoint{1.157009in}{1.859694in}}{\pgfqpoint{1.160281in}{1.867594in}}{\pgfqpoint{1.160281in}{1.875830in}}%
\pgfpathcurveto{\pgfqpoint{1.160281in}{1.884067in}}{\pgfqpoint{1.157009in}{1.891967in}}{\pgfqpoint{1.151185in}{1.897791in}}%
\pgfpathcurveto{\pgfqpoint{1.145361in}{1.903615in}}{\pgfqpoint{1.137461in}{1.906887in}}{\pgfqpoint{1.129225in}{1.906887in}}%
\pgfpathcurveto{\pgfqpoint{1.120988in}{1.906887in}}{\pgfqpoint{1.113088in}{1.903615in}}{\pgfqpoint{1.107264in}{1.897791in}}%
\pgfpathcurveto{\pgfqpoint{1.101441in}{1.891967in}}{\pgfqpoint{1.098168in}{1.884067in}}{\pgfqpoint{1.098168in}{1.875830in}}%
\pgfpathcurveto{\pgfqpoint{1.098168in}{1.867594in}}{\pgfqpoint{1.101441in}{1.859694in}}{\pgfqpoint{1.107264in}{1.853870in}}%
\pgfpathcurveto{\pgfqpoint{1.113088in}{1.848046in}}{\pgfqpoint{1.120988in}{1.844774in}}{\pgfqpoint{1.129225in}{1.844774in}}%
\pgfpathclose%
\pgfusepath{stroke,fill}%
\end{pgfscope}%
\begin{pgfscope}%
\pgfpathrectangle{\pgfqpoint{0.100000in}{0.220728in}}{\pgfqpoint{3.696000in}{3.696000in}}%
\pgfusepath{clip}%
\pgfsetbuttcap%
\pgfsetroundjoin%
\definecolor{currentfill}{rgb}{0.121569,0.466667,0.705882}%
\pgfsetfillcolor{currentfill}%
\pgfsetfillopacity{0.520092}%
\pgfsetlinewidth{1.003750pt}%
\definecolor{currentstroke}{rgb}{0.121569,0.466667,0.705882}%
\pgfsetstrokecolor{currentstroke}%
\pgfsetstrokeopacity{0.520092}%
\pgfsetdash{}{0pt}%
\pgfpathmoveto{\pgfqpoint{1.128219in}{1.841327in}}%
\pgfpathcurveto{\pgfqpoint{1.136455in}{1.841327in}}{\pgfqpoint{1.144355in}{1.844599in}}{\pgfqpoint{1.150179in}{1.850423in}}%
\pgfpathcurveto{\pgfqpoint{1.156003in}{1.856247in}}{\pgfqpoint{1.159276in}{1.864147in}}{\pgfqpoint{1.159276in}{1.872384in}}%
\pgfpathcurveto{\pgfqpoint{1.159276in}{1.880620in}}{\pgfqpoint{1.156003in}{1.888520in}}{\pgfqpoint{1.150179in}{1.894344in}}%
\pgfpathcurveto{\pgfqpoint{1.144355in}{1.900168in}}{\pgfqpoint{1.136455in}{1.903440in}}{\pgfqpoint{1.128219in}{1.903440in}}%
\pgfpathcurveto{\pgfqpoint{1.119983in}{1.903440in}}{\pgfqpoint{1.112083in}{1.900168in}}{\pgfqpoint{1.106259in}{1.894344in}}%
\pgfpathcurveto{\pgfqpoint{1.100435in}{1.888520in}}{\pgfqpoint{1.097163in}{1.880620in}}{\pgfqpoint{1.097163in}{1.872384in}}%
\pgfpathcurveto{\pgfqpoint{1.097163in}{1.864147in}}{\pgfqpoint{1.100435in}{1.856247in}}{\pgfqpoint{1.106259in}{1.850423in}}%
\pgfpathcurveto{\pgfqpoint{1.112083in}{1.844599in}}{\pgfqpoint{1.119983in}{1.841327in}}{\pgfqpoint{1.128219in}{1.841327in}}%
\pgfpathclose%
\pgfusepath{stroke,fill}%
\end{pgfscope}%
\begin{pgfscope}%
\pgfpathrectangle{\pgfqpoint{0.100000in}{0.220728in}}{\pgfqpoint{3.696000in}{3.696000in}}%
\pgfusepath{clip}%
\pgfsetbuttcap%
\pgfsetroundjoin%
\definecolor{currentfill}{rgb}{0.121569,0.466667,0.705882}%
\pgfsetfillcolor{currentfill}%
\pgfsetfillopacity{0.520733}%
\pgfsetlinewidth{1.003750pt}%
\definecolor{currentstroke}{rgb}{0.121569,0.466667,0.705882}%
\pgfsetstrokecolor{currentstroke}%
\pgfsetstrokeopacity{0.520733}%
\pgfsetdash{}{0pt}%
\pgfpathmoveto{\pgfqpoint{1.123985in}{1.836770in}}%
\pgfpathcurveto{\pgfqpoint{1.132221in}{1.836770in}}{\pgfqpoint{1.140121in}{1.840042in}}{\pgfqpoint{1.145945in}{1.845866in}}%
\pgfpathcurveto{\pgfqpoint{1.151769in}{1.851690in}}{\pgfqpoint{1.155042in}{1.859590in}}{\pgfqpoint{1.155042in}{1.867826in}}%
\pgfpathcurveto{\pgfqpoint{1.155042in}{1.876062in}}{\pgfqpoint{1.151769in}{1.883962in}}{\pgfqpoint{1.145945in}{1.889786in}}%
\pgfpathcurveto{\pgfqpoint{1.140121in}{1.895610in}}{\pgfqpoint{1.132221in}{1.898883in}}{\pgfqpoint{1.123985in}{1.898883in}}%
\pgfpathcurveto{\pgfqpoint{1.115749in}{1.898883in}}{\pgfqpoint{1.107849in}{1.895610in}}{\pgfqpoint{1.102025in}{1.889786in}}%
\pgfpathcurveto{\pgfqpoint{1.096201in}{1.883962in}}{\pgfqpoint{1.092929in}{1.876062in}}{\pgfqpoint{1.092929in}{1.867826in}}%
\pgfpathcurveto{\pgfqpoint{1.092929in}{1.859590in}}{\pgfqpoint{1.096201in}{1.851690in}}{\pgfqpoint{1.102025in}{1.845866in}}%
\pgfpathcurveto{\pgfqpoint{1.107849in}{1.840042in}}{\pgfqpoint{1.115749in}{1.836770in}}{\pgfqpoint{1.123985in}{1.836770in}}%
\pgfpathclose%
\pgfusepath{stroke,fill}%
\end{pgfscope}%
\begin{pgfscope}%
\pgfpathrectangle{\pgfqpoint{0.100000in}{0.220728in}}{\pgfqpoint{3.696000in}{3.696000in}}%
\pgfusepath{clip}%
\pgfsetbuttcap%
\pgfsetroundjoin%
\definecolor{currentfill}{rgb}{0.121569,0.466667,0.705882}%
\pgfsetfillcolor{currentfill}%
\pgfsetfillopacity{0.521502}%
\pgfsetlinewidth{1.003750pt}%
\definecolor{currentstroke}{rgb}{0.121569,0.466667,0.705882}%
\pgfsetstrokecolor{currentstroke}%
\pgfsetstrokeopacity{0.521502}%
\pgfsetdash{}{0pt}%
\pgfpathmoveto{\pgfqpoint{1.121761in}{1.832245in}}%
\pgfpathcurveto{\pgfqpoint{1.129997in}{1.832245in}}{\pgfqpoint{1.137897in}{1.835517in}}{\pgfqpoint{1.143721in}{1.841341in}}%
\pgfpathcurveto{\pgfqpoint{1.149545in}{1.847165in}}{\pgfqpoint{1.152817in}{1.855065in}}{\pgfqpoint{1.152817in}{1.863301in}}%
\pgfpathcurveto{\pgfqpoint{1.152817in}{1.871538in}}{\pgfqpoint{1.149545in}{1.879438in}}{\pgfqpoint{1.143721in}{1.885262in}}%
\pgfpathcurveto{\pgfqpoint{1.137897in}{1.891086in}}{\pgfqpoint{1.129997in}{1.894358in}}{\pgfqpoint{1.121761in}{1.894358in}}%
\pgfpathcurveto{\pgfqpoint{1.113524in}{1.894358in}}{\pgfqpoint{1.105624in}{1.891086in}}{\pgfqpoint{1.099800in}{1.885262in}}%
\pgfpathcurveto{\pgfqpoint{1.093977in}{1.879438in}}{\pgfqpoint{1.090704in}{1.871538in}}{\pgfqpoint{1.090704in}{1.863301in}}%
\pgfpathcurveto{\pgfqpoint{1.090704in}{1.855065in}}{\pgfqpoint{1.093977in}{1.847165in}}{\pgfqpoint{1.099800in}{1.841341in}}%
\pgfpathcurveto{\pgfqpoint{1.105624in}{1.835517in}}{\pgfqpoint{1.113524in}{1.832245in}}{\pgfqpoint{1.121761in}{1.832245in}}%
\pgfpathclose%
\pgfusepath{stroke,fill}%
\end{pgfscope}%
\begin{pgfscope}%
\pgfpathrectangle{\pgfqpoint{0.100000in}{0.220728in}}{\pgfqpoint{3.696000in}{3.696000in}}%
\pgfusepath{clip}%
\pgfsetbuttcap%
\pgfsetroundjoin%
\definecolor{currentfill}{rgb}{0.121569,0.466667,0.705882}%
\pgfsetfillcolor{currentfill}%
\pgfsetfillopacity{0.522091}%
\pgfsetlinewidth{1.003750pt}%
\definecolor{currentstroke}{rgb}{0.121569,0.466667,0.705882}%
\pgfsetstrokecolor{currentstroke}%
\pgfsetstrokeopacity{0.522091}%
\pgfsetdash{}{0pt}%
\pgfpathmoveto{\pgfqpoint{1.119797in}{1.828749in}}%
\pgfpathcurveto{\pgfqpoint{1.128033in}{1.828749in}}{\pgfqpoint{1.135933in}{1.832022in}}{\pgfqpoint{1.141757in}{1.837846in}}%
\pgfpathcurveto{\pgfqpoint{1.147581in}{1.843670in}}{\pgfqpoint{1.150853in}{1.851570in}}{\pgfqpoint{1.150853in}{1.859806in}}%
\pgfpathcurveto{\pgfqpoint{1.150853in}{1.868042in}}{\pgfqpoint{1.147581in}{1.875942in}}{\pgfqpoint{1.141757in}{1.881766in}}%
\pgfpathcurveto{\pgfqpoint{1.135933in}{1.887590in}}{\pgfqpoint{1.128033in}{1.890862in}}{\pgfqpoint{1.119797in}{1.890862in}}%
\pgfpathcurveto{\pgfqpoint{1.111560in}{1.890862in}}{\pgfqpoint{1.103660in}{1.887590in}}{\pgfqpoint{1.097836in}{1.881766in}}%
\pgfpathcurveto{\pgfqpoint{1.092013in}{1.875942in}}{\pgfqpoint{1.088740in}{1.868042in}}{\pgfqpoint{1.088740in}{1.859806in}}%
\pgfpathcurveto{\pgfqpoint{1.088740in}{1.851570in}}{\pgfqpoint{1.092013in}{1.843670in}}{\pgfqpoint{1.097836in}{1.837846in}}%
\pgfpathcurveto{\pgfqpoint{1.103660in}{1.832022in}}{\pgfqpoint{1.111560in}{1.828749in}}{\pgfqpoint{1.119797in}{1.828749in}}%
\pgfpathclose%
\pgfusepath{stroke,fill}%
\end{pgfscope}%
\begin{pgfscope}%
\pgfpathrectangle{\pgfqpoint{0.100000in}{0.220728in}}{\pgfqpoint{3.696000in}{3.696000in}}%
\pgfusepath{clip}%
\pgfsetbuttcap%
\pgfsetroundjoin%
\definecolor{currentfill}{rgb}{0.121569,0.466667,0.705882}%
\pgfsetfillcolor{currentfill}%
\pgfsetfillopacity{0.522585}%
\pgfsetlinewidth{1.003750pt}%
\definecolor{currentstroke}{rgb}{0.121569,0.466667,0.705882}%
\pgfsetstrokecolor{currentstroke}%
\pgfsetstrokeopacity{0.522585}%
\pgfsetdash{}{0pt}%
\pgfpathmoveto{\pgfqpoint{2.719126in}{3.060236in}}%
\pgfpathcurveto{\pgfqpoint{2.727363in}{3.060236in}}{\pgfqpoint{2.735263in}{3.063509in}}{\pgfqpoint{2.741087in}{3.069333in}}%
\pgfpathcurveto{\pgfqpoint{2.746911in}{3.075157in}}{\pgfqpoint{2.750183in}{3.083057in}}{\pgfqpoint{2.750183in}{3.091293in}}%
\pgfpathcurveto{\pgfqpoint{2.750183in}{3.099529in}}{\pgfqpoint{2.746911in}{3.107429in}}{\pgfqpoint{2.741087in}{3.113253in}}%
\pgfpathcurveto{\pgfqpoint{2.735263in}{3.119077in}}{\pgfqpoint{2.727363in}{3.122349in}}{\pgfqpoint{2.719126in}{3.122349in}}%
\pgfpathcurveto{\pgfqpoint{2.710890in}{3.122349in}}{\pgfqpoint{2.702990in}{3.119077in}}{\pgfqpoint{2.697166in}{3.113253in}}%
\pgfpathcurveto{\pgfqpoint{2.691342in}{3.107429in}}{\pgfqpoint{2.688070in}{3.099529in}}{\pgfqpoint{2.688070in}{3.091293in}}%
\pgfpathcurveto{\pgfqpoint{2.688070in}{3.083057in}}{\pgfqpoint{2.691342in}{3.075157in}}{\pgfqpoint{2.697166in}{3.069333in}}%
\pgfpathcurveto{\pgfqpoint{2.702990in}{3.063509in}}{\pgfqpoint{2.710890in}{3.060236in}}{\pgfqpoint{2.719126in}{3.060236in}}%
\pgfpathclose%
\pgfusepath{stroke,fill}%
\end{pgfscope}%
\begin{pgfscope}%
\pgfpathrectangle{\pgfqpoint{0.100000in}{0.220728in}}{\pgfqpoint{3.696000in}{3.696000in}}%
\pgfusepath{clip}%
\pgfsetbuttcap%
\pgfsetroundjoin%
\definecolor{currentfill}{rgb}{0.121569,0.466667,0.705882}%
\pgfsetfillcolor{currentfill}%
\pgfsetfillopacity{0.523056}%
\pgfsetlinewidth{1.003750pt}%
\definecolor{currentstroke}{rgb}{0.121569,0.466667,0.705882}%
\pgfsetstrokecolor{currentstroke}%
\pgfsetstrokeopacity{0.523056}%
\pgfsetdash{}{0pt}%
\pgfpathmoveto{\pgfqpoint{1.115834in}{1.822537in}}%
\pgfpathcurveto{\pgfqpoint{1.124070in}{1.822537in}}{\pgfqpoint{1.131970in}{1.825810in}}{\pgfqpoint{1.137794in}{1.831633in}}%
\pgfpathcurveto{\pgfqpoint{1.143618in}{1.837457in}}{\pgfqpoint{1.146890in}{1.845357in}}{\pgfqpoint{1.146890in}{1.853594in}}%
\pgfpathcurveto{\pgfqpoint{1.146890in}{1.861830in}}{\pgfqpoint{1.143618in}{1.869730in}}{\pgfqpoint{1.137794in}{1.875554in}}%
\pgfpathcurveto{\pgfqpoint{1.131970in}{1.881378in}}{\pgfqpoint{1.124070in}{1.884650in}}{\pgfqpoint{1.115834in}{1.884650in}}%
\pgfpathcurveto{\pgfqpoint{1.107597in}{1.884650in}}{\pgfqpoint{1.099697in}{1.881378in}}{\pgfqpoint{1.093873in}{1.875554in}}%
\pgfpathcurveto{\pgfqpoint{1.088049in}{1.869730in}}{\pgfqpoint{1.084777in}{1.861830in}}{\pgfqpoint{1.084777in}{1.853594in}}%
\pgfpathcurveto{\pgfqpoint{1.084777in}{1.845357in}}{\pgfqpoint{1.088049in}{1.837457in}}{\pgfqpoint{1.093873in}{1.831633in}}%
\pgfpathcurveto{\pgfqpoint{1.099697in}{1.825810in}}{\pgfqpoint{1.107597in}{1.822537in}}{\pgfqpoint{1.115834in}{1.822537in}}%
\pgfpathclose%
\pgfusepath{stroke,fill}%
\end{pgfscope}%
\begin{pgfscope}%
\pgfpathrectangle{\pgfqpoint{0.100000in}{0.220728in}}{\pgfqpoint{3.696000in}{3.696000in}}%
\pgfusepath{clip}%
\pgfsetbuttcap%
\pgfsetroundjoin%
\definecolor{currentfill}{rgb}{0.121569,0.466667,0.705882}%
\pgfsetfillcolor{currentfill}%
\pgfsetfillopacity{0.525178}%
\pgfsetlinewidth{1.003750pt}%
\definecolor{currentstroke}{rgb}{0.121569,0.466667,0.705882}%
\pgfsetstrokecolor{currentstroke}%
\pgfsetstrokeopacity{0.525178}%
\pgfsetdash{}{0pt}%
\pgfpathmoveto{\pgfqpoint{1.109421in}{1.811563in}}%
\pgfpathcurveto{\pgfqpoint{1.117658in}{1.811563in}}{\pgfqpoint{1.125558in}{1.814836in}}{\pgfqpoint{1.131382in}{1.820659in}}%
\pgfpathcurveto{\pgfqpoint{1.137205in}{1.826483in}}{\pgfqpoint{1.140478in}{1.834383in}}{\pgfqpoint{1.140478in}{1.842620in}}%
\pgfpathcurveto{\pgfqpoint{1.140478in}{1.850856in}}{\pgfqpoint{1.137205in}{1.858756in}}{\pgfqpoint{1.131382in}{1.864580in}}%
\pgfpathcurveto{\pgfqpoint{1.125558in}{1.870404in}}{\pgfqpoint{1.117658in}{1.873676in}}{\pgfqpoint{1.109421in}{1.873676in}}%
\pgfpathcurveto{\pgfqpoint{1.101185in}{1.873676in}}{\pgfqpoint{1.093285in}{1.870404in}}{\pgfqpoint{1.087461in}{1.864580in}}%
\pgfpathcurveto{\pgfqpoint{1.081637in}{1.858756in}}{\pgfqpoint{1.078365in}{1.850856in}}{\pgfqpoint{1.078365in}{1.842620in}}%
\pgfpathcurveto{\pgfqpoint{1.078365in}{1.834383in}}{\pgfqpoint{1.081637in}{1.826483in}}{\pgfqpoint{1.087461in}{1.820659in}}%
\pgfpathcurveto{\pgfqpoint{1.093285in}{1.814836in}}{\pgfqpoint{1.101185in}{1.811563in}}{\pgfqpoint{1.109421in}{1.811563in}}%
\pgfpathclose%
\pgfusepath{stroke,fill}%
\end{pgfscope}%
\begin{pgfscope}%
\pgfpathrectangle{\pgfqpoint{0.100000in}{0.220728in}}{\pgfqpoint{3.696000in}{3.696000in}}%
\pgfusepath{clip}%
\pgfsetbuttcap%
\pgfsetroundjoin%
\definecolor{currentfill}{rgb}{0.121569,0.466667,0.705882}%
\pgfsetfillcolor{currentfill}%
\pgfsetfillopacity{0.526430}%
\pgfsetlinewidth{1.003750pt}%
\definecolor{currentstroke}{rgb}{0.121569,0.466667,0.705882}%
\pgfsetstrokecolor{currentstroke}%
\pgfsetstrokeopacity{0.526430}%
\pgfsetdash{}{0pt}%
\pgfpathmoveto{\pgfqpoint{1.102606in}{1.803069in}}%
\pgfpathcurveto{\pgfqpoint{1.110842in}{1.803069in}}{\pgfqpoint{1.118742in}{1.806341in}}{\pgfqpoint{1.124566in}{1.812165in}}%
\pgfpathcurveto{\pgfqpoint{1.130390in}{1.817989in}}{\pgfqpoint{1.133662in}{1.825889in}}{\pgfqpoint{1.133662in}{1.834125in}}%
\pgfpathcurveto{\pgfqpoint{1.133662in}{1.842362in}}{\pgfqpoint{1.130390in}{1.850262in}}{\pgfqpoint{1.124566in}{1.856086in}}%
\pgfpathcurveto{\pgfqpoint{1.118742in}{1.861909in}}{\pgfqpoint{1.110842in}{1.865182in}}{\pgfqpoint{1.102606in}{1.865182in}}%
\pgfpathcurveto{\pgfqpoint{1.094370in}{1.865182in}}{\pgfqpoint{1.086469in}{1.861909in}}{\pgfqpoint{1.080646in}{1.856086in}}%
\pgfpathcurveto{\pgfqpoint{1.074822in}{1.850262in}}{\pgfqpoint{1.071549in}{1.842362in}}{\pgfqpoint{1.071549in}{1.834125in}}%
\pgfpathcurveto{\pgfqpoint{1.071549in}{1.825889in}}{\pgfqpoint{1.074822in}{1.817989in}}{\pgfqpoint{1.080646in}{1.812165in}}%
\pgfpathcurveto{\pgfqpoint{1.086469in}{1.806341in}}{\pgfqpoint{1.094370in}{1.803069in}}{\pgfqpoint{1.102606in}{1.803069in}}%
\pgfpathclose%
\pgfusepath{stroke,fill}%
\end{pgfscope}%
\begin{pgfscope}%
\pgfpathrectangle{\pgfqpoint{0.100000in}{0.220728in}}{\pgfqpoint{3.696000in}{3.696000in}}%
\pgfusepath{clip}%
\pgfsetbuttcap%
\pgfsetroundjoin%
\definecolor{currentfill}{rgb}{0.121569,0.466667,0.705882}%
\pgfsetfillcolor{currentfill}%
\pgfsetfillopacity{0.526682}%
\pgfsetlinewidth{1.003750pt}%
\definecolor{currentstroke}{rgb}{0.121569,0.466667,0.705882}%
\pgfsetstrokecolor{currentstroke}%
\pgfsetstrokeopacity{0.526682}%
\pgfsetdash{}{0pt}%
\pgfpathmoveto{\pgfqpoint{2.732589in}{3.057269in}}%
\pgfpathcurveto{\pgfqpoint{2.740825in}{3.057269in}}{\pgfqpoint{2.748725in}{3.060542in}}{\pgfqpoint{2.754549in}{3.066366in}}%
\pgfpathcurveto{\pgfqpoint{2.760373in}{3.072190in}}{\pgfqpoint{2.763645in}{3.080090in}}{\pgfqpoint{2.763645in}{3.088326in}}%
\pgfpathcurveto{\pgfqpoint{2.763645in}{3.096562in}}{\pgfqpoint{2.760373in}{3.104462in}}{\pgfqpoint{2.754549in}{3.110286in}}%
\pgfpathcurveto{\pgfqpoint{2.748725in}{3.116110in}}{\pgfqpoint{2.740825in}{3.119382in}}{\pgfqpoint{2.732589in}{3.119382in}}%
\pgfpathcurveto{\pgfqpoint{2.724353in}{3.119382in}}{\pgfqpoint{2.716453in}{3.116110in}}{\pgfqpoint{2.710629in}{3.110286in}}%
\pgfpathcurveto{\pgfqpoint{2.704805in}{3.104462in}}{\pgfqpoint{2.701532in}{3.096562in}}{\pgfqpoint{2.701532in}{3.088326in}}%
\pgfpathcurveto{\pgfqpoint{2.701532in}{3.080090in}}{\pgfqpoint{2.704805in}{3.072190in}}{\pgfqpoint{2.710629in}{3.066366in}}%
\pgfpathcurveto{\pgfqpoint{2.716453in}{3.060542in}}{\pgfqpoint{2.724353in}{3.057269in}}{\pgfqpoint{2.732589in}{3.057269in}}%
\pgfpathclose%
\pgfusepath{stroke,fill}%
\end{pgfscope}%
\begin{pgfscope}%
\pgfpathrectangle{\pgfqpoint{0.100000in}{0.220728in}}{\pgfqpoint{3.696000in}{3.696000in}}%
\pgfusepath{clip}%
\pgfsetbuttcap%
\pgfsetroundjoin%
\definecolor{currentfill}{rgb}{0.121569,0.466667,0.705882}%
\pgfsetfillcolor{currentfill}%
\pgfsetfillopacity{0.528778}%
\pgfsetlinewidth{1.003750pt}%
\definecolor{currentstroke}{rgb}{0.121569,0.466667,0.705882}%
\pgfsetstrokecolor{currentstroke}%
\pgfsetstrokeopacity{0.528778}%
\pgfsetdash{}{0pt}%
\pgfpathmoveto{\pgfqpoint{2.740224in}{3.055622in}}%
\pgfpathcurveto{\pgfqpoint{2.748460in}{3.055622in}}{\pgfqpoint{2.756360in}{3.058895in}}{\pgfqpoint{2.762184in}{3.064719in}}%
\pgfpathcurveto{\pgfqpoint{2.768008in}{3.070543in}}{\pgfqpoint{2.771280in}{3.078443in}}{\pgfqpoint{2.771280in}{3.086679in}}%
\pgfpathcurveto{\pgfqpoint{2.771280in}{3.094915in}}{\pgfqpoint{2.768008in}{3.102815in}}{\pgfqpoint{2.762184in}{3.108639in}}%
\pgfpathcurveto{\pgfqpoint{2.756360in}{3.114463in}}{\pgfqpoint{2.748460in}{3.117735in}}{\pgfqpoint{2.740224in}{3.117735in}}%
\pgfpathcurveto{\pgfqpoint{2.731987in}{3.117735in}}{\pgfqpoint{2.724087in}{3.114463in}}{\pgfqpoint{2.718263in}{3.108639in}}%
\pgfpathcurveto{\pgfqpoint{2.712439in}{3.102815in}}{\pgfqpoint{2.709167in}{3.094915in}}{\pgfqpoint{2.709167in}{3.086679in}}%
\pgfpathcurveto{\pgfqpoint{2.709167in}{3.078443in}}{\pgfqpoint{2.712439in}{3.070543in}}{\pgfqpoint{2.718263in}{3.064719in}}%
\pgfpathcurveto{\pgfqpoint{2.724087in}{3.058895in}}{\pgfqpoint{2.731987in}{3.055622in}}{\pgfqpoint{2.740224in}{3.055622in}}%
\pgfpathclose%
\pgfusepath{stroke,fill}%
\end{pgfscope}%
\begin{pgfscope}%
\pgfpathrectangle{\pgfqpoint{0.100000in}{0.220728in}}{\pgfqpoint{3.696000in}{3.696000in}}%
\pgfusepath{clip}%
\pgfsetbuttcap%
\pgfsetroundjoin%
\definecolor{currentfill}{rgb}{0.121569,0.466667,0.705882}%
\pgfsetfillcolor{currentfill}%
\pgfsetfillopacity{0.529756}%
\pgfsetlinewidth{1.003750pt}%
\definecolor{currentstroke}{rgb}{0.121569,0.466667,0.705882}%
\pgfsetstrokecolor{currentstroke}%
\pgfsetstrokeopacity{0.529756}%
\pgfsetdash{}{0pt}%
\pgfpathmoveto{\pgfqpoint{1.094678in}{1.785292in}}%
\pgfpathcurveto{\pgfqpoint{1.102914in}{1.785292in}}{\pgfqpoint{1.110814in}{1.788565in}}{\pgfqpoint{1.116638in}{1.794388in}}%
\pgfpathcurveto{\pgfqpoint{1.122462in}{1.800212in}}{\pgfqpoint{1.125735in}{1.808112in}}{\pgfqpoint{1.125735in}{1.816349in}}%
\pgfpathcurveto{\pgfqpoint{1.125735in}{1.824585in}}{\pgfqpoint{1.122462in}{1.832485in}}{\pgfqpoint{1.116638in}{1.838309in}}%
\pgfpathcurveto{\pgfqpoint{1.110814in}{1.844133in}}{\pgfqpoint{1.102914in}{1.847405in}}{\pgfqpoint{1.094678in}{1.847405in}}%
\pgfpathcurveto{\pgfqpoint{1.086442in}{1.847405in}}{\pgfqpoint{1.078542in}{1.844133in}}{\pgfqpoint{1.072718in}{1.838309in}}%
\pgfpathcurveto{\pgfqpoint{1.066894in}{1.832485in}}{\pgfqpoint{1.063622in}{1.824585in}}{\pgfqpoint{1.063622in}{1.816349in}}%
\pgfpathcurveto{\pgfqpoint{1.063622in}{1.808112in}}{\pgfqpoint{1.066894in}{1.800212in}}{\pgfqpoint{1.072718in}{1.794388in}}%
\pgfpathcurveto{\pgfqpoint{1.078542in}{1.788565in}}{\pgfqpoint{1.086442in}{1.785292in}}{\pgfqpoint{1.094678in}{1.785292in}}%
\pgfpathclose%
\pgfusepath{stroke,fill}%
\end{pgfscope}%
\begin{pgfscope}%
\pgfpathrectangle{\pgfqpoint{0.100000in}{0.220728in}}{\pgfqpoint{3.696000in}{3.696000in}}%
\pgfusepath{clip}%
\pgfsetbuttcap%
\pgfsetroundjoin%
\definecolor{currentfill}{rgb}{0.121569,0.466667,0.705882}%
\pgfsetfillcolor{currentfill}%
\pgfsetfillopacity{0.530918}%
\pgfsetlinewidth{1.003750pt}%
\definecolor{currentstroke}{rgb}{0.121569,0.466667,0.705882}%
\pgfsetstrokecolor{currentstroke}%
\pgfsetstrokeopacity{0.530918}%
\pgfsetdash{}{0pt}%
\pgfpathmoveto{\pgfqpoint{2.748643in}{3.053477in}}%
\pgfpathcurveto{\pgfqpoint{2.756880in}{3.053477in}}{\pgfqpoint{2.764780in}{3.056750in}}{\pgfqpoint{2.770604in}{3.062574in}}%
\pgfpathcurveto{\pgfqpoint{2.776428in}{3.068397in}}{\pgfqpoint{2.779700in}{3.076298in}}{\pgfqpoint{2.779700in}{3.084534in}}%
\pgfpathcurveto{\pgfqpoint{2.779700in}{3.092770in}}{\pgfqpoint{2.776428in}{3.100670in}}{\pgfqpoint{2.770604in}{3.106494in}}%
\pgfpathcurveto{\pgfqpoint{2.764780in}{3.112318in}}{\pgfqpoint{2.756880in}{3.115590in}}{\pgfqpoint{2.748643in}{3.115590in}}%
\pgfpathcurveto{\pgfqpoint{2.740407in}{3.115590in}}{\pgfqpoint{2.732507in}{3.112318in}}{\pgfqpoint{2.726683in}{3.106494in}}%
\pgfpathcurveto{\pgfqpoint{2.720859in}{3.100670in}}{\pgfqpoint{2.717587in}{3.092770in}}{\pgfqpoint{2.717587in}{3.084534in}}%
\pgfpathcurveto{\pgfqpoint{2.717587in}{3.076298in}}{\pgfqpoint{2.720859in}{3.068397in}}{\pgfqpoint{2.726683in}{3.062574in}}%
\pgfpathcurveto{\pgfqpoint{2.732507in}{3.056750in}}{\pgfqpoint{2.740407in}{3.053477in}}{\pgfqpoint{2.748643in}{3.053477in}}%
\pgfpathclose%
\pgfusepath{stroke,fill}%
\end{pgfscope}%
\begin{pgfscope}%
\pgfpathrectangle{\pgfqpoint{0.100000in}{0.220728in}}{\pgfqpoint{3.696000in}{3.696000in}}%
\pgfusepath{clip}%
\pgfsetbuttcap%
\pgfsetroundjoin%
\definecolor{currentfill}{rgb}{0.121569,0.466667,0.705882}%
\pgfsetfillcolor{currentfill}%
\pgfsetfillopacity{0.531269}%
\pgfsetlinewidth{1.003750pt}%
\definecolor{currentstroke}{rgb}{0.121569,0.466667,0.705882}%
\pgfsetstrokecolor{currentstroke}%
\pgfsetstrokeopacity{0.531269}%
\pgfsetdash{}{0pt}%
\pgfpathmoveto{\pgfqpoint{1.084636in}{1.771801in}}%
\pgfpathcurveto{\pgfqpoint{1.092872in}{1.771801in}}{\pgfqpoint{1.100772in}{1.775073in}}{\pgfqpoint{1.106596in}{1.780897in}}%
\pgfpathcurveto{\pgfqpoint{1.112420in}{1.786721in}}{\pgfqpoint{1.115692in}{1.794621in}}{\pgfqpoint{1.115692in}{1.802857in}}%
\pgfpathcurveto{\pgfqpoint{1.115692in}{1.811093in}}{\pgfqpoint{1.112420in}{1.818994in}}{\pgfqpoint{1.106596in}{1.824817in}}%
\pgfpathcurveto{\pgfqpoint{1.100772in}{1.830641in}}{\pgfqpoint{1.092872in}{1.833914in}}{\pgfqpoint{1.084636in}{1.833914in}}%
\pgfpathcurveto{\pgfqpoint{1.076399in}{1.833914in}}{\pgfqpoint{1.068499in}{1.830641in}}{\pgfqpoint{1.062675in}{1.824817in}}%
\pgfpathcurveto{\pgfqpoint{1.056851in}{1.818994in}}{\pgfqpoint{1.053579in}{1.811093in}}{\pgfqpoint{1.053579in}{1.802857in}}%
\pgfpathcurveto{\pgfqpoint{1.053579in}{1.794621in}}{\pgfqpoint{1.056851in}{1.786721in}}{\pgfqpoint{1.062675in}{1.780897in}}%
\pgfpathcurveto{\pgfqpoint{1.068499in}{1.775073in}}{\pgfqpoint{1.076399in}{1.771801in}}{\pgfqpoint{1.084636in}{1.771801in}}%
\pgfpathclose%
\pgfusepath{stroke,fill}%
\end{pgfscope}%
\begin{pgfscope}%
\pgfpathrectangle{\pgfqpoint{0.100000in}{0.220728in}}{\pgfqpoint{3.696000in}{3.696000in}}%
\pgfusepath{clip}%
\pgfsetbuttcap%
\pgfsetroundjoin%
\definecolor{currentfill}{rgb}{0.121569,0.466667,0.705882}%
\pgfsetfillcolor{currentfill}%
\pgfsetfillopacity{0.533379}%
\pgfsetlinewidth{1.003750pt}%
\definecolor{currentstroke}{rgb}{0.121569,0.466667,0.705882}%
\pgfsetstrokecolor{currentstroke}%
\pgfsetstrokeopacity{0.533379}%
\pgfsetdash{}{0pt}%
\pgfpathmoveto{\pgfqpoint{2.757479in}{3.050269in}}%
\pgfpathcurveto{\pgfqpoint{2.765715in}{3.050269in}}{\pgfqpoint{2.773615in}{3.053542in}}{\pgfqpoint{2.779439in}{3.059366in}}%
\pgfpathcurveto{\pgfqpoint{2.785263in}{3.065189in}}{\pgfqpoint{2.788535in}{3.073090in}}{\pgfqpoint{2.788535in}{3.081326in}}%
\pgfpathcurveto{\pgfqpoint{2.788535in}{3.089562in}}{\pgfqpoint{2.785263in}{3.097462in}}{\pgfqpoint{2.779439in}{3.103286in}}%
\pgfpathcurveto{\pgfqpoint{2.773615in}{3.109110in}}{\pgfqpoint{2.765715in}{3.112382in}}{\pgfqpoint{2.757479in}{3.112382in}}%
\pgfpathcurveto{\pgfqpoint{2.749242in}{3.112382in}}{\pgfqpoint{2.741342in}{3.109110in}}{\pgfqpoint{2.735518in}{3.103286in}}%
\pgfpathcurveto{\pgfqpoint{2.729695in}{3.097462in}}{\pgfqpoint{2.726422in}{3.089562in}}{\pgfqpoint{2.726422in}{3.081326in}}%
\pgfpathcurveto{\pgfqpoint{2.726422in}{3.073090in}}{\pgfqpoint{2.729695in}{3.065189in}}{\pgfqpoint{2.735518in}{3.059366in}}%
\pgfpathcurveto{\pgfqpoint{2.741342in}{3.053542in}}{\pgfqpoint{2.749242in}{3.050269in}}{\pgfqpoint{2.757479in}{3.050269in}}%
\pgfpathclose%
\pgfusepath{stroke,fill}%
\end{pgfscope}%
\begin{pgfscope}%
\pgfpathrectangle{\pgfqpoint{0.100000in}{0.220728in}}{\pgfqpoint{3.696000in}{3.696000in}}%
\pgfusepath{clip}%
\pgfsetbuttcap%
\pgfsetroundjoin%
\definecolor{currentfill}{rgb}{0.121569,0.466667,0.705882}%
\pgfsetfillcolor{currentfill}%
\pgfsetfillopacity{0.533594}%
\pgfsetlinewidth{1.003750pt}%
\definecolor{currentstroke}{rgb}{0.121569,0.466667,0.705882}%
\pgfsetstrokecolor{currentstroke}%
\pgfsetstrokeopacity{0.533594}%
\pgfsetdash{}{0pt}%
\pgfpathmoveto{\pgfqpoint{1.082354in}{1.755697in}}%
\pgfpathcurveto{\pgfqpoint{1.090590in}{1.755697in}}{\pgfqpoint{1.098490in}{1.758969in}}{\pgfqpoint{1.104314in}{1.764793in}}%
\pgfpathcurveto{\pgfqpoint{1.110138in}{1.770617in}}{\pgfqpoint{1.113411in}{1.778517in}}{\pgfqpoint{1.113411in}{1.786753in}}%
\pgfpathcurveto{\pgfqpoint{1.113411in}{1.794990in}}{\pgfqpoint{1.110138in}{1.802890in}}{\pgfqpoint{1.104314in}{1.808714in}}%
\pgfpathcurveto{\pgfqpoint{1.098490in}{1.814538in}}{\pgfqpoint{1.090590in}{1.817810in}}{\pgfqpoint{1.082354in}{1.817810in}}%
\pgfpathcurveto{\pgfqpoint{1.074118in}{1.817810in}}{\pgfqpoint{1.066218in}{1.814538in}}{\pgfqpoint{1.060394in}{1.808714in}}%
\pgfpathcurveto{\pgfqpoint{1.054570in}{1.802890in}}{\pgfqpoint{1.051298in}{1.794990in}}{\pgfqpoint{1.051298in}{1.786753in}}%
\pgfpathcurveto{\pgfqpoint{1.051298in}{1.778517in}}{\pgfqpoint{1.054570in}{1.770617in}}{\pgfqpoint{1.060394in}{1.764793in}}%
\pgfpathcurveto{\pgfqpoint{1.066218in}{1.758969in}}{\pgfqpoint{1.074118in}{1.755697in}}{\pgfqpoint{1.082354in}{1.755697in}}%
\pgfpathclose%
\pgfusepath{stroke,fill}%
\end{pgfscope}%
\begin{pgfscope}%
\pgfpathrectangle{\pgfqpoint{0.100000in}{0.220728in}}{\pgfqpoint{3.696000in}{3.696000in}}%
\pgfusepath{clip}%
\pgfsetbuttcap%
\pgfsetroundjoin%
\definecolor{currentfill}{rgb}{0.121569,0.466667,0.705882}%
\pgfsetfillcolor{currentfill}%
\pgfsetfillopacity{0.534134}%
\pgfsetlinewidth{1.003750pt}%
\definecolor{currentstroke}{rgb}{0.121569,0.466667,0.705882}%
\pgfsetstrokecolor{currentstroke}%
\pgfsetstrokeopacity{0.534134}%
\pgfsetdash{}{0pt}%
\pgfpathmoveto{\pgfqpoint{1.074721in}{1.748034in}}%
\pgfpathcurveto{\pgfqpoint{1.082958in}{1.748034in}}{\pgfqpoint{1.090858in}{1.751306in}}{\pgfqpoint{1.096681in}{1.757130in}}%
\pgfpathcurveto{\pgfqpoint{1.102505in}{1.762954in}}{\pgfqpoint{1.105778in}{1.770854in}}{\pgfqpoint{1.105778in}{1.779091in}}%
\pgfpathcurveto{\pgfqpoint{1.105778in}{1.787327in}}{\pgfqpoint{1.102505in}{1.795227in}}{\pgfqpoint{1.096681in}{1.801051in}}%
\pgfpathcurveto{\pgfqpoint{1.090858in}{1.806875in}}{\pgfqpoint{1.082958in}{1.810147in}}{\pgfqpoint{1.074721in}{1.810147in}}%
\pgfpathcurveto{\pgfqpoint{1.066485in}{1.810147in}}{\pgfqpoint{1.058585in}{1.806875in}}{\pgfqpoint{1.052761in}{1.801051in}}%
\pgfpathcurveto{\pgfqpoint{1.046937in}{1.795227in}}{\pgfqpoint{1.043665in}{1.787327in}}{\pgfqpoint{1.043665in}{1.779091in}}%
\pgfpathcurveto{\pgfqpoint{1.043665in}{1.770854in}}{\pgfqpoint{1.046937in}{1.762954in}}{\pgfqpoint{1.052761in}{1.757130in}}%
\pgfpathcurveto{\pgfqpoint{1.058585in}{1.751306in}}{\pgfqpoint{1.066485in}{1.748034in}}{\pgfqpoint{1.074721in}{1.748034in}}%
\pgfpathclose%
\pgfusepath{stroke,fill}%
\end{pgfscope}%
\begin{pgfscope}%
\pgfpathrectangle{\pgfqpoint{0.100000in}{0.220728in}}{\pgfqpoint{3.696000in}{3.696000in}}%
\pgfusepath{clip}%
\pgfsetbuttcap%
\pgfsetroundjoin%
\definecolor{currentfill}{rgb}{0.121569,0.466667,0.705882}%
\pgfsetfillcolor{currentfill}%
\pgfsetfillopacity{0.534671}%
\pgfsetlinewidth{1.003750pt}%
\definecolor{currentstroke}{rgb}{0.121569,0.466667,0.705882}%
\pgfsetstrokecolor{currentstroke}%
\pgfsetstrokeopacity{0.534671}%
\pgfsetdash{}{0pt}%
\pgfpathmoveto{\pgfqpoint{2.751697in}{3.050836in}}%
\pgfpathcurveto{\pgfqpoint{2.759933in}{3.050836in}}{\pgfqpoint{2.767833in}{3.054109in}}{\pgfqpoint{2.773657in}{3.059933in}}%
\pgfpathcurveto{\pgfqpoint{2.779481in}{3.065757in}}{\pgfqpoint{2.782753in}{3.073657in}}{\pgfqpoint{2.782753in}{3.081893in}}%
\pgfpathcurveto{\pgfqpoint{2.782753in}{3.090129in}}{\pgfqpoint{2.779481in}{3.098029in}}{\pgfqpoint{2.773657in}{3.103853in}}%
\pgfpathcurveto{\pgfqpoint{2.767833in}{3.109677in}}{\pgfqpoint{2.759933in}{3.112949in}}{\pgfqpoint{2.751697in}{3.112949in}}%
\pgfpathcurveto{\pgfqpoint{2.743461in}{3.112949in}}{\pgfqpoint{2.735561in}{3.109677in}}{\pgfqpoint{2.729737in}{3.103853in}}%
\pgfpathcurveto{\pgfqpoint{2.723913in}{3.098029in}}{\pgfqpoint{2.720640in}{3.090129in}}{\pgfqpoint{2.720640in}{3.081893in}}%
\pgfpathcurveto{\pgfqpoint{2.720640in}{3.073657in}}{\pgfqpoint{2.723913in}{3.065757in}}{\pgfqpoint{2.729737in}{3.059933in}}%
\pgfpathcurveto{\pgfqpoint{2.735561in}{3.054109in}}{\pgfqpoint{2.743461in}{3.050836in}}{\pgfqpoint{2.751697in}{3.050836in}}%
\pgfpathclose%
\pgfusepath{stroke,fill}%
\end{pgfscope}%
\begin{pgfscope}%
\pgfpathrectangle{\pgfqpoint{0.100000in}{0.220728in}}{\pgfqpoint{3.696000in}{3.696000in}}%
\pgfusepath{clip}%
\pgfsetbuttcap%
\pgfsetroundjoin%
\definecolor{currentfill}{rgb}{0.121569,0.466667,0.705882}%
\pgfsetfillcolor{currentfill}%
\pgfsetfillopacity{0.535372}%
\pgfsetlinewidth{1.003750pt}%
\definecolor{currentstroke}{rgb}{0.121569,0.466667,0.705882}%
\pgfsetstrokecolor{currentstroke}%
\pgfsetstrokeopacity{0.535372}%
\pgfsetdash{}{0pt}%
\pgfpathmoveto{\pgfqpoint{1.072904in}{1.739418in}}%
\pgfpathcurveto{\pgfqpoint{1.081140in}{1.739418in}}{\pgfqpoint{1.089040in}{1.742690in}}{\pgfqpoint{1.094864in}{1.748514in}}%
\pgfpathcurveto{\pgfqpoint{1.100688in}{1.754338in}}{\pgfqpoint{1.103960in}{1.762238in}}{\pgfqpoint{1.103960in}{1.770474in}}%
\pgfpathcurveto{\pgfqpoint{1.103960in}{1.778711in}}{\pgfqpoint{1.100688in}{1.786611in}}{\pgfqpoint{1.094864in}{1.792435in}}%
\pgfpathcurveto{\pgfqpoint{1.089040in}{1.798258in}}{\pgfqpoint{1.081140in}{1.801531in}}{\pgfqpoint{1.072904in}{1.801531in}}%
\pgfpathcurveto{\pgfqpoint{1.064667in}{1.801531in}}{\pgfqpoint{1.056767in}{1.798258in}}{\pgfqpoint{1.050943in}{1.792435in}}%
\pgfpathcurveto{\pgfqpoint{1.045119in}{1.786611in}}{\pgfqpoint{1.041847in}{1.778711in}}{\pgfqpoint{1.041847in}{1.770474in}}%
\pgfpathcurveto{\pgfqpoint{1.041847in}{1.762238in}}{\pgfqpoint{1.045119in}{1.754338in}}{\pgfqpoint{1.050943in}{1.748514in}}%
\pgfpathcurveto{\pgfqpoint{1.056767in}{1.742690in}}{\pgfqpoint{1.064667in}{1.739418in}}{\pgfqpoint{1.072904in}{1.739418in}}%
\pgfpathclose%
\pgfusepath{stroke,fill}%
\end{pgfscope}%
\begin{pgfscope}%
\pgfpathrectangle{\pgfqpoint{0.100000in}{0.220728in}}{\pgfqpoint{3.696000in}{3.696000in}}%
\pgfusepath{clip}%
\pgfsetbuttcap%
\pgfsetroundjoin%
\definecolor{currentfill}{rgb}{0.121569,0.466667,0.705882}%
\pgfsetfillcolor{currentfill}%
\pgfsetfillopacity{0.535722}%
\pgfsetlinewidth{1.003750pt}%
\definecolor{currentstroke}{rgb}{0.121569,0.466667,0.705882}%
\pgfsetstrokecolor{currentstroke}%
\pgfsetstrokeopacity{0.535722}%
\pgfsetdash{}{0pt}%
\pgfpathmoveto{\pgfqpoint{1.070057in}{1.736123in}}%
\pgfpathcurveto{\pgfqpoint{1.078293in}{1.736123in}}{\pgfqpoint{1.086193in}{1.739396in}}{\pgfqpoint{1.092017in}{1.745220in}}%
\pgfpathcurveto{\pgfqpoint{1.097841in}{1.751043in}}{\pgfqpoint{1.101113in}{1.758943in}}{\pgfqpoint{1.101113in}{1.767180in}}%
\pgfpathcurveto{\pgfqpoint{1.101113in}{1.775416in}}{\pgfqpoint{1.097841in}{1.783316in}}{\pgfqpoint{1.092017in}{1.789140in}}%
\pgfpathcurveto{\pgfqpoint{1.086193in}{1.794964in}}{\pgfqpoint{1.078293in}{1.798236in}}{\pgfqpoint{1.070057in}{1.798236in}}%
\pgfpathcurveto{\pgfqpoint{1.061820in}{1.798236in}}{\pgfqpoint{1.053920in}{1.794964in}}{\pgfqpoint{1.048096in}{1.789140in}}%
\pgfpathcurveto{\pgfqpoint{1.042272in}{1.783316in}}{\pgfqpoint{1.039000in}{1.775416in}}{\pgfqpoint{1.039000in}{1.767180in}}%
\pgfpathcurveto{\pgfqpoint{1.039000in}{1.758943in}}{\pgfqpoint{1.042272in}{1.751043in}}{\pgfqpoint{1.048096in}{1.745220in}}%
\pgfpathcurveto{\pgfqpoint{1.053920in}{1.739396in}}{\pgfqpoint{1.061820in}{1.736123in}}{\pgfqpoint{1.070057in}{1.736123in}}%
\pgfpathclose%
\pgfusepath{stroke,fill}%
\end{pgfscope}%
\begin{pgfscope}%
\pgfpathrectangle{\pgfqpoint{0.100000in}{0.220728in}}{\pgfqpoint{3.696000in}{3.696000in}}%
\pgfusepath{clip}%
\pgfsetbuttcap%
\pgfsetroundjoin%
\definecolor{currentfill}{rgb}{0.121569,0.466667,0.705882}%
\pgfsetfillcolor{currentfill}%
\pgfsetfillopacity{0.536749}%
\pgfsetlinewidth{1.003750pt}%
\definecolor{currentstroke}{rgb}{0.121569,0.466667,0.705882}%
\pgfsetstrokecolor{currentstroke}%
\pgfsetstrokeopacity{0.536749}%
\pgfsetdash{}{0pt}%
\pgfpathmoveto{\pgfqpoint{2.758148in}{3.050563in}}%
\pgfpathcurveto{\pgfqpoint{2.766385in}{3.050563in}}{\pgfqpoint{2.774285in}{3.053835in}}{\pgfqpoint{2.780109in}{3.059659in}}%
\pgfpathcurveto{\pgfqpoint{2.785933in}{3.065483in}}{\pgfqpoint{2.789205in}{3.073383in}}{\pgfqpoint{2.789205in}{3.081619in}}%
\pgfpathcurveto{\pgfqpoint{2.789205in}{3.089856in}}{\pgfqpoint{2.785933in}{3.097756in}}{\pgfqpoint{2.780109in}{3.103580in}}%
\pgfpathcurveto{\pgfqpoint{2.774285in}{3.109404in}}{\pgfqpoint{2.766385in}{3.112676in}}{\pgfqpoint{2.758148in}{3.112676in}}%
\pgfpathcurveto{\pgfqpoint{2.749912in}{3.112676in}}{\pgfqpoint{2.742012in}{3.109404in}}{\pgfqpoint{2.736188in}{3.103580in}}%
\pgfpathcurveto{\pgfqpoint{2.730364in}{3.097756in}}{\pgfqpoint{2.727092in}{3.089856in}}{\pgfqpoint{2.727092in}{3.081619in}}%
\pgfpathcurveto{\pgfqpoint{2.727092in}{3.073383in}}{\pgfqpoint{2.730364in}{3.065483in}}{\pgfqpoint{2.736188in}{3.059659in}}%
\pgfpathcurveto{\pgfqpoint{2.742012in}{3.053835in}}{\pgfqpoint{2.749912in}{3.050563in}}{\pgfqpoint{2.758148in}{3.050563in}}%
\pgfpathclose%
\pgfusepath{stroke,fill}%
\end{pgfscope}%
\begin{pgfscope}%
\pgfpathrectangle{\pgfqpoint{0.100000in}{0.220728in}}{\pgfqpoint{3.696000in}{3.696000in}}%
\pgfusepath{clip}%
\pgfsetbuttcap%
\pgfsetroundjoin%
\definecolor{currentfill}{rgb}{0.121569,0.466667,0.705882}%
\pgfsetfillcolor{currentfill}%
\pgfsetfillopacity{0.536962}%
\pgfsetlinewidth{1.003750pt}%
\definecolor{currentstroke}{rgb}{0.121569,0.466667,0.705882}%
\pgfsetstrokecolor{currentstroke}%
\pgfsetstrokeopacity{0.536962}%
\pgfsetdash{}{0pt}%
\pgfpathmoveto{\pgfqpoint{1.067092in}{1.728969in}}%
\pgfpathcurveto{\pgfqpoint{1.075328in}{1.728969in}}{\pgfqpoint{1.083228in}{1.732241in}}{\pgfqpoint{1.089052in}{1.738065in}}%
\pgfpathcurveto{\pgfqpoint{1.094876in}{1.743889in}}{\pgfqpoint{1.098149in}{1.751789in}}{\pgfqpoint{1.098149in}{1.760026in}}%
\pgfpathcurveto{\pgfqpoint{1.098149in}{1.768262in}}{\pgfqpoint{1.094876in}{1.776162in}}{\pgfqpoint{1.089052in}{1.781986in}}%
\pgfpathcurveto{\pgfqpoint{1.083228in}{1.787810in}}{\pgfqpoint{1.075328in}{1.791082in}}{\pgfqpoint{1.067092in}{1.791082in}}%
\pgfpathcurveto{\pgfqpoint{1.058856in}{1.791082in}}{\pgfqpoint{1.050956in}{1.787810in}}{\pgfqpoint{1.045132in}{1.781986in}}%
\pgfpathcurveto{\pgfqpoint{1.039308in}{1.776162in}}{\pgfqpoint{1.036036in}{1.768262in}}{\pgfqpoint{1.036036in}{1.760026in}}%
\pgfpathcurveto{\pgfqpoint{1.036036in}{1.751789in}}{\pgfqpoint{1.039308in}{1.743889in}}{\pgfqpoint{1.045132in}{1.738065in}}%
\pgfpathcurveto{\pgfqpoint{1.050956in}{1.732241in}}{\pgfqpoint{1.058856in}{1.728969in}}{\pgfqpoint{1.067092in}{1.728969in}}%
\pgfpathclose%
\pgfusepath{stroke,fill}%
\end{pgfscope}%
\begin{pgfscope}%
\pgfpathrectangle{\pgfqpoint{0.100000in}{0.220728in}}{\pgfqpoint{3.696000in}{3.696000in}}%
\pgfusepath{clip}%
\pgfsetbuttcap%
\pgfsetroundjoin%
\definecolor{currentfill}{rgb}{0.121569,0.466667,0.705882}%
\pgfsetfillcolor{currentfill}%
\pgfsetfillopacity{0.537870}%
\pgfsetlinewidth{1.003750pt}%
\definecolor{currentstroke}{rgb}{0.121569,0.466667,0.705882}%
\pgfsetstrokecolor{currentstroke}%
\pgfsetstrokeopacity{0.537870}%
\pgfsetdash{}{0pt}%
\pgfpathmoveto{\pgfqpoint{1.063071in}{1.723592in}}%
\pgfpathcurveto{\pgfqpoint{1.071307in}{1.723592in}}{\pgfqpoint{1.079207in}{1.726865in}}{\pgfqpoint{1.085031in}{1.732688in}}%
\pgfpathcurveto{\pgfqpoint{1.090855in}{1.738512in}}{\pgfqpoint{1.094127in}{1.746412in}}{\pgfqpoint{1.094127in}{1.754649in}}%
\pgfpathcurveto{\pgfqpoint{1.094127in}{1.762885in}}{\pgfqpoint{1.090855in}{1.770785in}}{\pgfqpoint{1.085031in}{1.776609in}}%
\pgfpathcurveto{\pgfqpoint{1.079207in}{1.782433in}}{\pgfqpoint{1.071307in}{1.785705in}}{\pgfqpoint{1.063071in}{1.785705in}}%
\pgfpathcurveto{\pgfqpoint{1.054834in}{1.785705in}}{\pgfqpoint{1.046934in}{1.782433in}}{\pgfqpoint{1.041110in}{1.776609in}}%
\pgfpathcurveto{\pgfqpoint{1.035286in}{1.770785in}}{\pgfqpoint{1.032014in}{1.762885in}}{\pgfqpoint{1.032014in}{1.754649in}}%
\pgfpathcurveto{\pgfqpoint{1.032014in}{1.746412in}}{\pgfqpoint{1.035286in}{1.738512in}}{\pgfqpoint{1.041110in}{1.732688in}}%
\pgfpathcurveto{\pgfqpoint{1.046934in}{1.726865in}}{\pgfqpoint{1.054834in}{1.723592in}}{\pgfqpoint{1.063071in}{1.723592in}}%
\pgfpathclose%
\pgfusepath{stroke,fill}%
\end{pgfscope}%
\begin{pgfscope}%
\pgfpathrectangle{\pgfqpoint{0.100000in}{0.220728in}}{\pgfqpoint{3.696000in}{3.696000in}}%
\pgfusepath{clip}%
\pgfsetbuttcap%
\pgfsetroundjoin%
\definecolor{currentfill}{rgb}{0.121569,0.466667,0.705882}%
\pgfsetfillcolor{currentfill}%
\pgfsetfillopacity{0.538651}%
\pgfsetlinewidth{1.003750pt}%
\definecolor{currentstroke}{rgb}{0.121569,0.466667,0.705882}%
\pgfsetstrokecolor{currentstroke}%
\pgfsetstrokeopacity{0.538651}%
\pgfsetdash{}{0pt}%
\pgfpathmoveto{\pgfqpoint{1.060308in}{1.718572in}}%
\pgfpathcurveto{\pgfqpoint{1.068545in}{1.718572in}}{\pgfqpoint{1.076445in}{1.721844in}}{\pgfqpoint{1.082269in}{1.727668in}}%
\pgfpathcurveto{\pgfqpoint{1.088093in}{1.733492in}}{\pgfqpoint{1.091365in}{1.741392in}}{\pgfqpoint{1.091365in}{1.749628in}}%
\pgfpathcurveto{\pgfqpoint{1.091365in}{1.757864in}}{\pgfqpoint{1.088093in}{1.765764in}}{\pgfqpoint{1.082269in}{1.771588in}}%
\pgfpathcurveto{\pgfqpoint{1.076445in}{1.777412in}}{\pgfqpoint{1.068545in}{1.780685in}}{\pgfqpoint{1.060308in}{1.780685in}}%
\pgfpathcurveto{\pgfqpoint{1.052072in}{1.780685in}}{\pgfqpoint{1.044172in}{1.777412in}}{\pgfqpoint{1.038348in}{1.771588in}}%
\pgfpathcurveto{\pgfqpoint{1.032524in}{1.765764in}}{\pgfqpoint{1.029252in}{1.757864in}}{\pgfqpoint{1.029252in}{1.749628in}}%
\pgfpathcurveto{\pgfqpoint{1.029252in}{1.741392in}}{\pgfqpoint{1.032524in}{1.733492in}}{\pgfqpoint{1.038348in}{1.727668in}}%
\pgfpathcurveto{\pgfqpoint{1.044172in}{1.721844in}}{\pgfqpoint{1.052072in}{1.718572in}}{\pgfqpoint{1.060308in}{1.718572in}}%
\pgfpathclose%
\pgfusepath{stroke,fill}%
\end{pgfscope}%
\begin{pgfscope}%
\pgfpathrectangle{\pgfqpoint{0.100000in}{0.220728in}}{\pgfqpoint{3.696000in}{3.696000in}}%
\pgfusepath{clip}%
\pgfsetbuttcap%
\pgfsetroundjoin%
\definecolor{currentfill}{rgb}{0.121569,0.466667,0.705882}%
\pgfsetfillcolor{currentfill}%
\pgfsetfillopacity{0.538948}%
\pgfsetlinewidth{1.003750pt}%
\definecolor{currentstroke}{rgb}{0.121569,0.466667,0.705882}%
\pgfsetstrokecolor{currentstroke}%
\pgfsetstrokeopacity{0.538948}%
\pgfsetdash{}{0pt}%
\pgfpathmoveto{\pgfqpoint{2.766932in}{3.048495in}}%
\pgfpathcurveto{\pgfqpoint{2.775168in}{3.048495in}}{\pgfqpoint{2.783068in}{3.051767in}}{\pgfqpoint{2.788892in}{3.057591in}}%
\pgfpathcurveto{\pgfqpoint{2.794716in}{3.063415in}}{\pgfqpoint{2.797988in}{3.071315in}}{\pgfqpoint{2.797988in}{3.079551in}}%
\pgfpathcurveto{\pgfqpoint{2.797988in}{3.087788in}}{\pgfqpoint{2.794716in}{3.095688in}}{\pgfqpoint{2.788892in}{3.101512in}}%
\pgfpathcurveto{\pgfqpoint{2.783068in}{3.107335in}}{\pgfqpoint{2.775168in}{3.110608in}}{\pgfqpoint{2.766932in}{3.110608in}}%
\pgfpathcurveto{\pgfqpoint{2.758696in}{3.110608in}}{\pgfqpoint{2.750795in}{3.107335in}}{\pgfqpoint{2.744972in}{3.101512in}}%
\pgfpathcurveto{\pgfqpoint{2.739148in}{3.095688in}}{\pgfqpoint{2.735875in}{3.087788in}}{\pgfqpoint{2.735875in}{3.079551in}}%
\pgfpathcurveto{\pgfqpoint{2.735875in}{3.071315in}}{\pgfqpoint{2.739148in}{3.063415in}}{\pgfqpoint{2.744972in}{3.057591in}}%
\pgfpathcurveto{\pgfqpoint{2.750795in}{3.051767in}}{\pgfqpoint{2.758696in}{3.048495in}}{\pgfqpoint{2.766932in}{3.048495in}}%
\pgfpathclose%
\pgfusepath{stroke,fill}%
\end{pgfscope}%
\begin{pgfscope}%
\pgfpathrectangle{\pgfqpoint{0.100000in}{0.220728in}}{\pgfqpoint{3.696000in}{3.696000in}}%
\pgfusepath{clip}%
\pgfsetbuttcap%
\pgfsetroundjoin%
\definecolor{currentfill}{rgb}{0.121569,0.466667,0.705882}%
\pgfsetfillcolor{currentfill}%
\pgfsetfillopacity{0.540255}%
\pgfsetlinewidth{1.003750pt}%
\definecolor{currentstroke}{rgb}{0.121569,0.466667,0.705882}%
\pgfsetstrokecolor{currentstroke}%
\pgfsetstrokeopacity{0.540255}%
\pgfsetdash{}{0pt}%
\pgfpathmoveto{\pgfqpoint{1.055997in}{1.709345in}}%
\pgfpathcurveto{\pgfqpoint{1.064233in}{1.709345in}}{\pgfqpoint{1.072133in}{1.712618in}}{\pgfqpoint{1.077957in}{1.718441in}}%
\pgfpathcurveto{\pgfqpoint{1.083781in}{1.724265in}}{\pgfqpoint{1.087054in}{1.732165in}}{\pgfqpoint{1.087054in}{1.740402in}}%
\pgfpathcurveto{\pgfqpoint{1.087054in}{1.748638in}}{\pgfqpoint{1.083781in}{1.756538in}}{\pgfqpoint{1.077957in}{1.762362in}}%
\pgfpathcurveto{\pgfqpoint{1.072133in}{1.768186in}}{\pgfqpoint{1.064233in}{1.771458in}}{\pgfqpoint{1.055997in}{1.771458in}}%
\pgfpathcurveto{\pgfqpoint{1.047761in}{1.771458in}}{\pgfqpoint{1.039861in}{1.768186in}}{\pgfqpoint{1.034037in}{1.762362in}}%
\pgfpathcurveto{\pgfqpoint{1.028213in}{1.756538in}}{\pgfqpoint{1.024941in}{1.748638in}}{\pgfqpoint{1.024941in}{1.740402in}}%
\pgfpathcurveto{\pgfqpoint{1.024941in}{1.732165in}}{\pgfqpoint{1.028213in}{1.724265in}}{\pgfqpoint{1.034037in}{1.718441in}}%
\pgfpathcurveto{\pgfqpoint{1.039861in}{1.712618in}}{\pgfqpoint{1.047761in}{1.709345in}}{\pgfqpoint{1.055997in}{1.709345in}}%
\pgfpathclose%
\pgfusepath{stroke,fill}%
\end{pgfscope}%
\begin{pgfscope}%
\pgfpathrectangle{\pgfqpoint{0.100000in}{0.220728in}}{\pgfqpoint{3.696000in}{3.696000in}}%
\pgfusepath{clip}%
\pgfsetbuttcap%
\pgfsetroundjoin%
\definecolor{currentfill}{rgb}{0.121569,0.466667,0.705882}%
\pgfsetfillcolor{currentfill}%
\pgfsetfillopacity{0.541501}%
\pgfsetlinewidth{1.003750pt}%
\definecolor{currentstroke}{rgb}{0.121569,0.466667,0.705882}%
\pgfsetstrokecolor{currentstroke}%
\pgfsetstrokeopacity{0.541501}%
\pgfsetdash{}{0pt}%
\pgfpathmoveto{\pgfqpoint{2.776929in}{3.046282in}}%
\pgfpathcurveto{\pgfqpoint{2.785165in}{3.046282in}}{\pgfqpoint{2.793065in}{3.049554in}}{\pgfqpoint{2.798889in}{3.055378in}}%
\pgfpathcurveto{\pgfqpoint{2.804713in}{3.061202in}}{\pgfqpoint{2.807986in}{3.069102in}}{\pgfqpoint{2.807986in}{3.077339in}}%
\pgfpathcurveto{\pgfqpoint{2.807986in}{3.085575in}}{\pgfqpoint{2.804713in}{3.093475in}}{\pgfqpoint{2.798889in}{3.099299in}}%
\pgfpathcurveto{\pgfqpoint{2.793065in}{3.105123in}}{\pgfqpoint{2.785165in}{3.108395in}}{\pgfqpoint{2.776929in}{3.108395in}}%
\pgfpathcurveto{\pgfqpoint{2.768693in}{3.108395in}}{\pgfqpoint{2.760793in}{3.105123in}}{\pgfqpoint{2.754969in}{3.099299in}}%
\pgfpathcurveto{\pgfqpoint{2.749145in}{3.093475in}}{\pgfqpoint{2.745873in}{3.085575in}}{\pgfqpoint{2.745873in}{3.077339in}}%
\pgfpathcurveto{\pgfqpoint{2.745873in}{3.069102in}}{\pgfqpoint{2.749145in}{3.061202in}}{\pgfqpoint{2.754969in}{3.055378in}}%
\pgfpathcurveto{\pgfqpoint{2.760793in}{3.049554in}}{\pgfqpoint{2.768693in}{3.046282in}}{\pgfqpoint{2.776929in}{3.046282in}}%
\pgfpathclose%
\pgfusepath{stroke,fill}%
\end{pgfscope}%
\begin{pgfscope}%
\pgfpathrectangle{\pgfqpoint{0.100000in}{0.220728in}}{\pgfqpoint{3.696000in}{3.696000in}}%
\pgfusepath{clip}%
\pgfsetbuttcap%
\pgfsetroundjoin%
\definecolor{currentfill}{rgb}{0.121569,0.466667,0.705882}%
\pgfsetfillcolor{currentfill}%
\pgfsetfillopacity{0.542313}%
\pgfsetlinewidth{1.003750pt}%
\definecolor{currentstroke}{rgb}{0.121569,0.466667,0.705882}%
\pgfsetstrokecolor{currentstroke}%
\pgfsetstrokeopacity{0.542313}%
\pgfsetdash{}{0pt}%
\pgfpathmoveto{\pgfqpoint{2.791540in}{3.042352in}}%
\pgfpathcurveto{\pgfqpoint{2.799776in}{3.042352in}}{\pgfqpoint{2.807676in}{3.045624in}}{\pgfqpoint{2.813500in}{3.051448in}}%
\pgfpathcurveto{\pgfqpoint{2.819324in}{3.057272in}}{\pgfqpoint{2.822596in}{3.065172in}}{\pgfqpoint{2.822596in}{3.073408in}}%
\pgfpathcurveto{\pgfqpoint{2.822596in}{3.081644in}}{\pgfqpoint{2.819324in}{3.089544in}}{\pgfqpoint{2.813500in}{3.095368in}}%
\pgfpathcurveto{\pgfqpoint{2.807676in}{3.101192in}}{\pgfqpoint{2.799776in}{3.104465in}}{\pgfqpoint{2.791540in}{3.104465in}}%
\pgfpathcurveto{\pgfqpoint{2.783303in}{3.104465in}}{\pgfqpoint{2.775403in}{3.101192in}}{\pgfqpoint{2.769579in}{3.095368in}}%
\pgfpathcurveto{\pgfqpoint{2.763755in}{3.089544in}}{\pgfqpoint{2.760483in}{3.081644in}}{\pgfqpoint{2.760483in}{3.073408in}}%
\pgfpathcurveto{\pgfqpoint{2.760483in}{3.065172in}}{\pgfqpoint{2.763755in}{3.057272in}}{\pgfqpoint{2.769579in}{3.051448in}}%
\pgfpathcurveto{\pgfqpoint{2.775403in}{3.045624in}}{\pgfqpoint{2.783303in}{3.042352in}}{\pgfqpoint{2.791540in}{3.042352in}}%
\pgfpathclose%
\pgfusepath{stroke,fill}%
\end{pgfscope}%
\begin{pgfscope}%
\pgfpathrectangle{\pgfqpoint{0.100000in}{0.220728in}}{\pgfqpoint{3.696000in}{3.696000in}}%
\pgfusepath{clip}%
\pgfsetbuttcap%
\pgfsetroundjoin%
\definecolor{currentfill}{rgb}{0.121569,0.466667,0.705882}%
\pgfsetfillcolor{currentfill}%
\pgfsetfillopacity{0.542485}%
\pgfsetlinewidth{1.003750pt}%
\definecolor{currentstroke}{rgb}{0.121569,0.466667,0.705882}%
\pgfsetstrokecolor{currentstroke}%
\pgfsetstrokeopacity{0.542485}%
\pgfsetdash{}{0pt}%
\pgfpathmoveto{\pgfqpoint{2.782866in}{3.044932in}}%
\pgfpathcurveto{\pgfqpoint{2.791102in}{3.044932in}}{\pgfqpoint{2.799002in}{3.048204in}}{\pgfqpoint{2.804826in}{3.054028in}}%
\pgfpathcurveto{\pgfqpoint{2.810650in}{3.059852in}}{\pgfqpoint{2.813922in}{3.067752in}}{\pgfqpoint{2.813922in}{3.075989in}}%
\pgfpathcurveto{\pgfqpoint{2.813922in}{3.084225in}}{\pgfqpoint{2.810650in}{3.092125in}}{\pgfqpoint{2.804826in}{3.097949in}}%
\pgfpathcurveto{\pgfqpoint{2.799002in}{3.103773in}}{\pgfqpoint{2.791102in}{3.107045in}}{\pgfqpoint{2.782866in}{3.107045in}}%
\pgfpathcurveto{\pgfqpoint{2.774630in}{3.107045in}}{\pgfqpoint{2.766730in}{3.103773in}}{\pgfqpoint{2.760906in}{3.097949in}}%
\pgfpathcurveto{\pgfqpoint{2.755082in}{3.092125in}}{\pgfqpoint{2.751809in}{3.084225in}}{\pgfqpoint{2.751809in}{3.075989in}}%
\pgfpathcurveto{\pgfqpoint{2.751809in}{3.067752in}}{\pgfqpoint{2.755082in}{3.059852in}}{\pgfqpoint{2.760906in}{3.054028in}}%
\pgfpathcurveto{\pgfqpoint{2.766730in}{3.048204in}}{\pgfqpoint{2.774630in}{3.044932in}}{\pgfqpoint{2.782866in}{3.044932in}}%
\pgfpathclose%
\pgfusepath{stroke,fill}%
\end{pgfscope}%
\begin{pgfscope}%
\pgfpathrectangle{\pgfqpoint{0.100000in}{0.220728in}}{\pgfqpoint{3.696000in}{3.696000in}}%
\pgfusepath{clip}%
\pgfsetbuttcap%
\pgfsetroundjoin%
\definecolor{currentfill}{rgb}{0.121569,0.466667,0.705882}%
\pgfsetfillcolor{currentfill}%
\pgfsetfillopacity{0.542597}%
\pgfsetlinewidth{1.003750pt}%
\definecolor{currentstroke}{rgb}{0.121569,0.466667,0.705882}%
\pgfsetstrokecolor{currentstroke}%
\pgfsetstrokeopacity{0.542597}%
\pgfsetdash{}{0pt}%
\pgfpathmoveto{\pgfqpoint{1.046422in}{1.692392in}}%
\pgfpathcurveto{\pgfqpoint{1.054658in}{1.692392in}}{\pgfqpoint{1.062559in}{1.695665in}}{\pgfqpoint{1.068382in}{1.701489in}}%
\pgfpathcurveto{\pgfqpoint{1.074206in}{1.707313in}}{\pgfqpoint{1.077479in}{1.715213in}}{\pgfqpoint{1.077479in}{1.723449in}}%
\pgfpathcurveto{\pgfqpoint{1.077479in}{1.731685in}}{\pgfqpoint{1.074206in}{1.739585in}}{\pgfqpoint{1.068382in}{1.745409in}}%
\pgfpathcurveto{\pgfqpoint{1.062559in}{1.751233in}}{\pgfqpoint{1.054658in}{1.754505in}}{\pgfqpoint{1.046422in}{1.754505in}}%
\pgfpathcurveto{\pgfqpoint{1.038186in}{1.754505in}}{\pgfqpoint{1.030286in}{1.751233in}}{\pgfqpoint{1.024462in}{1.745409in}}%
\pgfpathcurveto{\pgfqpoint{1.018638in}{1.739585in}}{\pgfqpoint{1.015366in}{1.731685in}}{\pgfqpoint{1.015366in}{1.723449in}}%
\pgfpathcurveto{\pgfqpoint{1.015366in}{1.715213in}}{\pgfqpoint{1.018638in}{1.707313in}}{\pgfqpoint{1.024462in}{1.701489in}}%
\pgfpathcurveto{\pgfqpoint{1.030286in}{1.695665in}}{\pgfqpoint{1.038186in}{1.692392in}}{\pgfqpoint{1.046422in}{1.692392in}}%
\pgfpathclose%
\pgfusepath{stroke,fill}%
\end{pgfscope}%
\begin{pgfscope}%
\pgfpathrectangle{\pgfqpoint{0.100000in}{0.220728in}}{\pgfqpoint{3.696000in}{3.696000in}}%
\pgfusepath{clip}%
\pgfsetbuttcap%
\pgfsetroundjoin%
\definecolor{currentfill}{rgb}{0.121569,0.466667,0.705882}%
\pgfsetfillcolor{currentfill}%
\pgfsetfillopacity{0.545383}%
\pgfsetlinewidth{1.003750pt}%
\definecolor{currentstroke}{rgb}{0.121569,0.466667,0.705882}%
\pgfsetstrokecolor{currentstroke}%
\pgfsetstrokeopacity{0.545383}%
\pgfsetdash{}{0pt}%
\pgfpathmoveto{\pgfqpoint{2.800093in}{3.042550in}}%
\pgfpathcurveto{\pgfqpoint{2.808329in}{3.042550in}}{\pgfqpoint{2.816229in}{3.045823in}}{\pgfqpoint{2.822053in}{3.051646in}}%
\pgfpathcurveto{\pgfqpoint{2.827877in}{3.057470in}}{\pgfqpoint{2.831149in}{3.065370in}}{\pgfqpoint{2.831149in}{3.073607in}}%
\pgfpathcurveto{\pgfqpoint{2.831149in}{3.081843in}}{\pgfqpoint{2.827877in}{3.089743in}}{\pgfqpoint{2.822053in}{3.095567in}}%
\pgfpathcurveto{\pgfqpoint{2.816229in}{3.101391in}}{\pgfqpoint{2.808329in}{3.104663in}}{\pgfqpoint{2.800093in}{3.104663in}}%
\pgfpathcurveto{\pgfqpoint{2.791857in}{3.104663in}}{\pgfqpoint{2.783956in}{3.101391in}}{\pgfqpoint{2.778133in}{3.095567in}}%
\pgfpathcurveto{\pgfqpoint{2.772309in}{3.089743in}}{\pgfqpoint{2.769036in}{3.081843in}}{\pgfqpoint{2.769036in}{3.073607in}}%
\pgfpathcurveto{\pgfqpoint{2.769036in}{3.065370in}}{\pgfqpoint{2.772309in}{3.057470in}}{\pgfqpoint{2.778133in}{3.051646in}}%
\pgfpathcurveto{\pgfqpoint{2.783956in}{3.045823in}}{\pgfqpoint{2.791857in}{3.042550in}}{\pgfqpoint{2.800093in}{3.042550in}}%
\pgfpathclose%
\pgfusepath{stroke,fill}%
\end{pgfscope}%
\begin{pgfscope}%
\pgfpathrectangle{\pgfqpoint{0.100000in}{0.220728in}}{\pgfqpoint{3.696000in}{3.696000in}}%
\pgfusepath{clip}%
\pgfsetbuttcap%
\pgfsetroundjoin%
\definecolor{currentfill}{rgb}{0.121569,0.466667,0.705882}%
\pgfsetfillcolor{currentfill}%
\pgfsetfillopacity{0.545394}%
\pgfsetlinewidth{1.003750pt}%
\definecolor{currentstroke}{rgb}{0.121569,0.466667,0.705882}%
\pgfsetstrokecolor{currentstroke}%
\pgfsetstrokeopacity{0.545394}%
\pgfsetdash{}{0pt}%
\pgfpathmoveto{\pgfqpoint{1.042242in}{1.675168in}}%
\pgfpathcurveto{\pgfqpoint{1.050478in}{1.675168in}}{\pgfqpoint{1.058378in}{1.678441in}}{\pgfqpoint{1.064202in}{1.684265in}}%
\pgfpathcurveto{\pgfqpoint{1.070026in}{1.690088in}}{\pgfqpoint{1.073298in}{1.697989in}}{\pgfqpoint{1.073298in}{1.706225in}}%
\pgfpathcurveto{\pgfqpoint{1.073298in}{1.714461in}}{\pgfqpoint{1.070026in}{1.722361in}}{\pgfqpoint{1.064202in}{1.728185in}}%
\pgfpathcurveto{\pgfqpoint{1.058378in}{1.734009in}}{\pgfqpoint{1.050478in}{1.737281in}}{\pgfqpoint{1.042242in}{1.737281in}}%
\pgfpathcurveto{\pgfqpoint{1.034006in}{1.737281in}}{\pgfqpoint{1.026106in}{1.734009in}}{\pgfqpoint{1.020282in}{1.728185in}}%
\pgfpathcurveto{\pgfqpoint{1.014458in}{1.722361in}}{\pgfqpoint{1.011185in}{1.714461in}}{\pgfqpoint{1.011185in}{1.706225in}}%
\pgfpathcurveto{\pgfqpoint{1.011185in}{1.697989in}}{\pgfqpoint{1.014458in}{1.690088in}}{\pgfqpoint{1.020282in}{1.684265in}}%
\pgfpathcurveto{\pgfqpoint{1.026106in}{1.678441in}}{\pgfqpoint{1.034006in}{1.675168in}}{\pgfqpoint{1.042242in}{1.675168in}}%
\pgfpathclose%
\pgfusepath{stroke,fill}%
\end{pgfscope}%
\begin{pgfscope}%
\pgfpathrectangle{\pgfqpoint{0.100000in}{0.220728in}}{\pgfqpoint{3.696000in}{3.696000in}}%
\pgfusepath{clip}%
\pgfsetbuttcap%
\pgfsetroundjoin%
\definecolor{currentfill}{rgb}{0.121569,0.466667,0.705882}%
\pgfsetfillcolor{currentfill}%
\pgfsetfillopacity{0.546507}%
\pgfsetlinewidth{1.003750pt}%
\definecolor{currentstroke}{rgb}{0.121569,0.466667,0.705882}%
\pgfsetstrokecolor{currentstroke}%
\pgfsetstrokeopacity{0.546507}%
\pgfsetdash{}{0pt}%
\pgfpathmoveto{\pgfqpoint{2.805029in}{3.041064in}}%
\pgfpathcurveto{\pgfqpoint{2.813265in}{3.041064in}}{\pgfqpoint{2.821165in}{3.044336in}}{\pgfqpoint{2.826989in}{3.050160in}}%
\pgfpathcurveto{\pgfqpoint{2.832813in}{3.055984in}}{\pgfqpoint{2.836085in}{3.063884in}}{\pgfqpoint{2.836085in}{3.072120in}}%
\pgfpathcurveto{\pgfqpoint{2.836085in}{3.080357in}}{\pgfqpoint{2.832813in}{3.088257in}}{\pgfqpoint{2.826989in}{3.094081in}}%
\pgfpathcurveto{\pgfqpoint{2.821165in}{3.099905in}}{\pgfqpoint{2.813265in}{3.103177in}}{\pgfqpoint{2.805029in}{3.103177in}}%
\pgfpathcurveto{\pgfqpoint{2.796793in}{3.103177in}}{\pgfqpoint{2.788893in}{3.099905in}}{\pgfqpoint{2.783069in}{3.094081in}}%
\pgfpathcurveto{\pgfqpoint{2.777245in}{3.088257in}}{\pgfqpoint{2.773972in}{3.080357in}}{\pgfqpoint{2.773972in}{3.072120in}}%
\pgfpathcurveto{\pgfqpoint{2.773972in}{3.063884in}}{\pgfqpoint{2.777245in}{3.055984in}}{\pgfqpoint{2.783069in}{3.050160in}}%
\pgfpathcurveto{\pgfqpoint{2.788893in}{3.044336in}}{\pgfqpoint{2.796793in}{3.041064in}}{\pgfqpoint{2.805029in}{3.041064in}}%
\pgfpathclose%
\pgfusepath{stroke,fill}%
\end{pgfscope}%
\begin{pgfscope}%
\pgfpathrectangle{\pgfqpoint{0.100000in}{0.220728in}}{\pgfqpoint{3.696000in}{3.696000in}}%
\pgfusepath{clip}%
\pgfsetbuttcap%
\pgfsetroundjoin%
\definecolor{currentfill}{rgb}{0.121569,0.466667,0.705882}%
\pgfsetfillcolor{currentfill}%
\pgfsetfillopacity{0.546668}%
\pgfsetlinewidth{1.003750pt}%
\definecolor{currentstroke}{rgb}{0.121569,0.466667,0.705882}%
\pgfsetstrokecolor{currentstroke}%
\pgfsetstrokeopacity{0.546668}%
\pgfsetdash{}{0pt}%
\pgfpathmoveto{\pgfqpoint{1.033105in}{1.663046in}}%
\pgfpathcurveto{\pgfqpoint{1.041341in}{1.663046in}}{\pgfqpoint{1.049241in}{1.666318in}}{\pgfqpoint{1.055065in}{1.672142in}}%
\pgfpathcurveto{\pgfqpoint{1.060889in}{1.677966in}}{\pgfqpoint{1.064161in}{1.685866in}}{\pgfqpoint{1.064161in}{1.694102in}}%
\pgfpathcurveto{\pgfqpoint{1.064161in}{1.702339in}}{\pgfqpoint{1.060889in}{1.710239in}}{\pgfqpoint{1.055065in}{1.716063in}}%
\pgfpathcurveto{\pgfqpoint{1.049241in}{1.721887in}}{\pgfqpoint{1.041341in}{1.725159in}}{\pgfqpoint{1.033105in}{1.725159in}}%
\pgfpathcurveto{\pgfqpoint{1.024869in}{1.725159in}}{\pgfqpoint{1.016969in}{1.721887in}}{\pgfqpoint{1.011145in}{1.716063in}}%
\pgfpathcurveto{\pgfqpoint{1.005321in}{1.710239in}}{\pgfqpoint{1.002048in}{1.702339in}}{\pgfqpoint{1.002048in}{1.694102in}}%
\pgfpathcurveto{\pgfqpoint{1.002048in}{1.685866in}}{\pgfqpoint{1.005321in}{1.677966in}}{\pgfqpoint{1.011145in}{1.672142in}}%
\pgfpathcurveto{\pgfqpoint{1.016969in}{1.666318in}}{\pgfqpoint{1.024869in}{1.663046in}}{\pgfqpoint{1.033105in}{1.663046in}}%
\pgfpathclose%
\pgfusepath{stroke,fill}%
\end{pgfscope}%
\begin{pgfscope}%
\pgfpathrectangle{\pgfqpoint{0.100000in}{0.220728in}}{\pgfqpoint{3.696000in}{3.696000in}}%
\pgfusepath{clip}%
\pgfsetbuttcap%
\pgfsetroundjoin%
\definecolor{currentfill}{rgb}{0.121569,0.466667,0.705882}%
\pgfsetfillcolor{currentfill}%
\pgfsetfillopacity{0.547961}%
\pgfsetlinewidth{1.003750pt}%
\definecolor{currentstroke}{rgb}{0.121569,0.466667,0.705882}%
\pgfsetstrokecolor{currentstroke}%
\pgfsetstrokeopacity{0.547961}%
\pgfsetdash{}{0pt}%
\pgfpathmoveto{\pgfqpoint{2.811944in}{3.039212in}}%
\pgfpathcurveto{\pgfqpoint{2.820180in}{3.039212in}}{\pgfqpoint{2.828080in}{3.042484in}}{\pgfqpoint{2.833904in}{3.048308in}}%
\pgfpathcurveto{\pgfqpoint{2.839728in}{3.054132in}}{\pgfqpoint{2.843001in}{3.062032in}}{\pgfqpoint{2.843001in}{3.070268in}}%
\pgfpathcurveto{\pgfqpoint{2.843001in}{3.078505in}}{\pgfqpoint{2.839728in}{3.086405in}}{\pgfqpoint{2.833904in}{3.092229in}}%
\pgfpathcurveto{\pgfqpoint{2.828080in}{3.098052in}}{\pgfqpoint{2.820180in}{3.101325in}}{\pgfqpoint{2.811944in}{3.101325in}}%
\pgfpathcurveto{\pgfqpoint{2.803708in}{3.101325in}}{\pgfqpoint{2.795808in}{3.098052in}}{\pgfqpoint{2.789984in}{3.092229in}}%
\pgfpathcurveto{\pgfqpoint{2.784160in}{3.086405in}}{\pgfqpoint{2.780888in}{3.078505in}}{\pgfqpoint{2.780888in}{3.070268in}}%
\pgfpathcurveto{\pgfqpoint{2.780888in}{3.062032in}}{\pgfqpoint{2.784160in}{3.054132in}}{\pgfqpoint{2.789984in}{3.048308in}}%
\pgfpathcurveto{\pgfqpoint{2.795808in}{3.042484in}}{\pgfqpoint{2.803708in}{3.039212in}}{\pgfqpoint{2.811944in}{3.039212in}}%
\pgfpathclose%
\pgfusepath{stroke,fill}%
\end{pgfscope}%
\begin{pgfscope}%
\pgfpathrectangle{\pgfqpoint{0.100000in}{0.220728in}}{\pgfqpoint{3.696000in}{3.696000in}}%
\pgfusepath{clip}%
\pgfsetbuttcap%
\pgfsetroundjoin%
\definecolor{currentfill}{rgb}{0.121569,0.466667,0.705882}%
\pgfsetfillcolor{currentfill}%
\pgfsetfillopacity{0.548648}%
\pgfsetlinewidth{1.003750pt}%
\definecolor{currentstroke}{rgb}{0.121569,0.466667,0.705882}%
\pgfsetstrokecolor{currentstroke}%
\pgfsetstrokeopacity{0.548648}%
\pgfsetdash{}{0pt}%
\pgfpathmoveto{\pgfqpoint{1.031401in}{1.649968in}}%
\pgfpathcurveto{\pgfqpoint{1.039637in}{1.649968in}}{\pgfqpoint{1.047537in}{1.653240in}}{\pgfqpoint{1.053361in}{1.659064in}}%
\pgfpathcurveto{\pgfqpoint{1.059185in}{1.664888in}}{\pgfqpoint{1.062457in}{1.672788in}}{\pgfqpoint{1.062457in}{1.681025in}}%
\pgfpathcurveto{\pgfqpoint{1.062457in}{1.689261in}}{\pgfqpoint{1.059185in}{1.697161in}}{\pgfqpoint{1.053361in}{1.702985in}}%
\pgfpathcurveto{\pgfqpoint{1.047537in}{1.708809in}}{\pgfqpoint{1.039637in}{1.712081in}}{\pgfqpoint{1.031401in}{1.712081in}}%
\pgfpathcurveto{\pgfqpoint{1.023164in}{1.712081in}}{\pgfqpoint{1.015264in}{1.708809in}}{\pgfqpoint{1.009440in}{1.702985in}}%
\pgfpathcurveto{\pgfqpoint{1.003616in}{1.697161in}}{\pgfqpoint{1.000344in}{1.689261in}}{\pgfqpoint{1.000344in}{1.681025in}}%
\pgfpathcurveto{\pgfqpoint{1.000344in}{1.672788in}}{\pgfqpoint{1.003616in}{1.664888in}}{\pgfqpoint{1.009440in}{1.659064in}}%
\pgfpathcurveto{\pgfqpoint{1.015264in}{1.653240in}}{\pgfqpoint{1.023164in}{1.649968in}}{\pgfqpoint{1.031401in}{1.649968in}}%
\pgfpathclose%
\pgfusepath{stroke,fill}%
\end{pgfscope}%
\begin{pgfscope}%
\pgfpathrectangle{\pgfqpoint{0.100000in}{0.220728in}}{\pgfqpoint{3.696000in}{3.696000in}}%
\pgfusepath{clip}%
\pgfsetbuttcap%
\pgfsetroundjoin%
\definecolor{currentfill}{rgb}{0.121569,0.466667,0.705882}%
\pgfsetfillcolor{currentfill}%
\pgfsetfillopacity{0.549104}%
\pgfsetlinewidth{1.003750pt}%
\definecolor{currentstroke}{rgb}{0.121569,0.466667,0.705882}%
\pgfsetstrokecolor{currentstroke}%
\pgfsetstrokeopacity{0.549104}%
\pgfsetdash{}{0pt}%
\pgfpathmoveto{\pgfqpoint{2.815556in}{3.038957in}}%
\pgfpathcurveto{\pgfqpoint{2.823792in}{3.038957in}}{\pgfqpoint{2.831692in}{3.042230in}}{\pgfqpoint{2.837516in}{3.048053in}}%
\pgfpathcurveto{\pgfqpoint{2.843340in}{3.053877in}}{\pgfqpoint{2.846612in}{3.061777in}}{\pgfqpoint{2.846612in}{3.070014in}}%
\pgfpathcurveto{\pgfqpoint{2.846612in}{3.078250in}}{\pgfqpoint{2.843340in}{3.086150in}}{\pgfqpoint{2.837516in}{3.091974in}}%
\pgfpathcurveto{\pgfqpoint{2.831692in}{3.097798in}}{\pgfqpoint{2.823792in}{3.101070in}}{\pgfqpoint{2.815556in}{3.101070in}}%
\pgfpathcurveto{\pgfqpoint{2.807320in}{3.101070in}}{\pgfqpoint{2.799420in}{3.097798in}}{\pgfqpoint{2.793596in}{3.091974in}}%
\pgfpathcurveto{\pgfqpoint{2.787772in}{3.086150in}}{\pgfqpoint{2.784499in}{3.078250in}}{\pgfqpoint{2.784499in}{3.070014in}}%
\pgfpathcurveto{\pgfqpoint{2.784499in}{3.061777in}}{\pgfqpoint{2.787772in}{3.053877in}}{\pgfqpoint{2.793596in}{3.048053in}}%
\pgfpathcurveto{\pgfqpoint{2.799420in}{3.042230in}}{\pgfqpoint{2.807320in}{3.038957in}}{\pgfqpoint{2.815556in}{3.038957in}}%
\pgfpathclose%
\pgfusepath{stroke,fill}%
\end{pgfscope}%
\begin{pgfscope}%
\pgfpathrectangle{\pgfqpoint{0.100000in}{0.220728in}}{\pgfqpoint{3.696000in}{3.696000in}}%
\pgfusepath{clip}%
\pgfsetbuttcap%
\pgfsetroundjoin%
\definecolor{currentfill}{rgb}{0.121569,0.466667,0.705882}%
\pgfsetfillcolor{currentfill}%
\pgfsetfillopacity{0.549635}%
\pgfsetlinewidth{1.003750pt}%
\definecolor{currentstroke}{rgb}{0.121569,0.466667,0.705882}%
\pgfsetstrokecolor{currentstroke}%
\pgfsetstrokeopacity{0.549635}%
\pgfsetdash{}{0pt}%
\pgfpathmoveto{\pgfqpoint{1.025263in}{1.642931in}}%
\pgfpathcurveto{\pgfqpoint{1.033499in}{1.642931in}}{\pgfqpoint{1.041399in}{1.646203in}}{\pgfqpoint{1.047223in}{1.652027in}}%
\pgfpathcurveto{\pgfqpoint{1.053047in}{1.657851in}}{\pgfqpoint{1.056320in}{1.665751in}}{\pgfqpoint{1.056320in}{1.673987in}}%
\pgfpathcurveto{\pgfqpoint{1.056320in}{1.682224in}}{\pgfqpoint{1.053047in}{1.690124in}}{\pgfqpoint{1.047223in}{1.695948in}}%
\pgfpathcurveto{\pgfqpoint{1.041399in}{1.701772in}}{\pgfqpoint{1.033499in}{1.705044in}}{\pgfqpoint{1.025263in}{1.705044in}}%
\pgfpathcurveto{\pgfqpoint{1.017027in}{1.705044in}}{\pgfqpoint{1.009127in}{1.701772in}}{\pgfqpoint{1.003303in}{1.695948in}}%
\pgfpathcurveto{\pgfqpoint{0.997479in}{1.690124in}}{\pgfqpoint{0.994207in}{1.682224in}}{\pgfqpoint{0.994207in}{1.673987in}}%
\pgfpathcurveto{\pgfqpoint{0.994207in}{1.665751in}}{\pgfqpoint{0.997479in}{1.657851in}}{\pgfqpoint{1.003303in}{1.652027in}}%
\pgfpathcurveto{\pgfqpoint{1.009127in}{1.646203in}}{\pgfqpoint{1.017027in}{1.642931in}}{\pgfqpoint{1.025263in}{1.642931in}}%
\pgfpathclose%
\pgfusepath{stroke,fill}%
\end{pgfscope}%
\begin{pgfscope}%
\pgfpathrectangle{\pgfqpoint{0.100000in}{0.220728in}}{\pgfqpoint{3.696000in}{3.696000in}}%
\pgfusepath{clip}%
\pgfsetbuttcap%
\pgfsetroundjoin%
\definecolor{currentfill}{rgb}{0.121569,0.466667,0.705882}%
\pgfsetfillcolor{currentfill}%
\pgfsetfillopacity{0.550100}%
\pgfsetlinewidth{1.003750pt}%
\definecolor{currentstroke}{rgb}{0.121569,0.466667,0.705882}%
\pgfsetstrokecolor{currentstroke}%
\pgfsetstrokeopacity{0.550100}%
\pgfsetdash{}{0pt}%
\pgfpathmoveto{\pgfqpoint{2.820212in}{3.038081in}}%
\pgfpathcurveto{\pgfqpoint{2.828448in}{3.038081in}}{\pgfqpoint{2.836348in}{3.041353in}}{\pgfqpoint{2.842172in}{3.047177in}}%
\pgfpathcurveto{\pgfqpoint{2.847996in}{3.053001in}}{\pgfqpoint{2.851268in}{3.060901in}}{\pgfqpoint{2.851268in}{3.069138in}}%
\pgfpathcurveto{\pgfqpoint{2.851268in}{3.077374in}}{\pgfqpoint{2.847996in}{3.085274in}}{\pgfqpoint{2.842172in}{3.091098in}}%
\pgfpathcurveto{\pgfqpoint{2.836348in}{3.096922in}}{\pgfqpoint{2.828448in}{3.100194in}}{\pgfqpoint{2.820212in}{3.100194in}}%
\pgfpathcurveto{\pgfqpoint{2.811976in}{3.100194in}}{\pgfqpoint{2.804075in}{3.096922in}}{\pgfqpoint{2.798252in}{3.091098in}}%
\pgfpathcurveto{\pgfqpoint{2.792428in}{3.085274in}}{\pgfqpoint{2.789155in}{3.077374in}}{\pgfqpoint{2.789155in}{3.069138in}}%
\pgfpathcurveto{\pgfqpoint{2.789155in}{3.060901in}}{\pgfqpoint{2.792428in}{3.053001in}}{\pgfqpoint{2.798252in}{3.047177in}}%
\pgfpathcurveto{\pgfqpoint{2.804075in}{3.041353in}}{\pgfqpoint{2.811976in}{3.038081in}}{\pgfqpoint{2.820212in}{3.038081in}}%
\pgfpathclose%
\pgfusepath{stroke,fill}%
\end{pgfscope}%
\begin{pgfscope}%
\pgfpathrectangle{\pgfqpoint{0.100000in}{0.220728in}}{\pgfqpoint{3.696000in}{3.696000in}}%
\pgfusepath{clip}%
\pgfsetbuttcap%
\pgfsetroundjoin%
\definecolor{currentfill}{rgb}{0.121569,0.466667,0.705882}%
\pgfsetfillcolor{currentfill}%
\pgfsetfillopacity{0.550629}%
\pgfsetlinewidth{1.003750pt}%
\definecolor{currentstroke}{rgb}{0.121569,0.466667,0.705882}%
\pgfsetstrokecolor{currentstroke}%
\pgfsetstrokeopacity{0.550629}%
\pgfsetdash{}{0pt}%
\pgfpathmoveto{\pgfqpoint{1.024002in}{1.636451in}}%
\pgfpathcurveto{\pgfqpoint{1.032238in}{1.636451in}}{\pgfqpoint{1.040138in}{1.639723in}}{\pgfqpoint{1.045962in}{1.645547in}}%
\pgfpathcurveto{\pgfqpoint{1.051786in}{1.651371in}}{\pgfqpoint{1.055058in}{1.659271in}}{\pgfqpoint{1.055058in}{1.667508in}}%
\pgfpathcurveto{\pgfqpoint{1.055058in}{1.675744in}}{\pgfqpoint{1.051786in}{1.683644in}}{\pgfqpoint{1.045962in}{1.689468in}}%
\pgfpathcurveto{\pgfqpoint{1.040138in}{1.695292in}}{\pgfqpoint{1.032238in}{1.698564in}}{\pgfqpoint{1.024002in}{1.698564in}}%
\pgfpathcurveto{\pgfqpoint{1.015765in}{1.698564in}}{\pgfqpoint{1.007865in}{1.695292in}}{\pgfqpoint{1.002041in}{1.689468in}}%
\pgfpathcurveto{\pgfqpoint{0.996218in}{1.683644in}}{\pgfqpoint{0.992945in}{1.675744in}}{\pgfqpoint{0.992945in}{1.667508in}}%
\pgfpathcurveto{\pgfqpoint{0.992945in}{1.659271in}}{\pgfqpoint{0.996218in}{1.651371in}}{\pgfqpoint{1.002041in}{1.645547in}}%
\pgfpathcurveto{\pgfqpoint{1.007865in}{1.639723in}}{\pgfqpoint{1.015765in}{1.636451in}}{\pgfqpoint{1.024002in}{1.636451in}}%
\pgfpathclose%
\pgfusepath{stroke,fill}%
\end{pgfscope}%
\begin{pgfscope}%
\pgfpathrectangle{\pgfqpoint{0.100000in}{0.220728in}}{\pgfqpoint{3.696000in}{3.696000in}}%
\pgfusepath{clip}%
\pgfsetbuttcap%
\pgfsetroundjoin%
\definecolor{currentfill}{rgb}{0.121569,0.466667,0.705882}%
\pgfsetfillcolor{currentfill}%
\pgfsetfillopacity{0.550701}%
\pgfsetlinewidth{1.003750pt}%
\definecolor{currentstroke}{rgb}{0.121569,0.466667,0.705882}%
\pgfsetstrokecolor{currentstroke}%
\pgfsetstrokeopacity{0.550701}%
\pgfsetdash{}{0pt}%
\pgfpathmoveto{\pgfqpoint{2.822725in}{3.037647in}}%
\pgfpathcurveto{\pgfqpoint{2.830961in}{3.037647in}}{\pgfqpoint{2.838861in}{3.040919in}}{\pgfqpoint{2.844685in}{3.046743in}}%
\pgfpathcurveto{\pgfqpoint{2.850509in}{3.052567in}}{\pgfqpoint{2.853782in}{3.060467in}}{\pgfqpoint{2.853782in}{3.068703in}}%
\pgfpathcurveto{\pgfqpoint{2.853782in}{3.076940in}}{\pgfqpoint{2.850509in}{3.084840in}}{\pgfqpoint{2.844685in}{3.090664in}}%
\pgfpathcurveto{\pgfqpoint{2.838861in}{3.096488in}}{\pgfqpoint{2.830961in}{3.099760in}}{\pgfqpoint{2.822725in}{3.099760in}}%
\pgfpathcurveto{\pgfqpoint{2.814489in}{3.099760in}}{\pgfqpoint{2.806589in}{3.096488in}}{\pgfqpoint{2.800765in}{3.090664in}}%
\pgfpathcurveto{\pgfqpoint{2.794941in}{3.084840in}}{\pgfqpoint{2.791669in}{3.076940in}}{\pgfqpoint{2.791669in}{3.068703in}}%
\pgfpathcurveto{\pgfqpoint{2.791669in}{3.060467in}}{\pgfqpoint{2.794941in}{3.052567in}}{\pgfqpoint{2.800765in}{3.046743in}}%
\pgfpathcurveto{\pgfqpoint{2.806589in}{3.040919in}}{\pgfqpoint{2.814489in}{3.037647in}}{\pgfqpoint{2.822725in}{3.037647in}}%
\pgfpathclose%
\pgfusepath{stroke,fill}%
\end{pgfscope}%
\begin{pgfscope}%
\pgfpathrectangle{\pgfqpoint{0.100000in}{0.220728in}}{\pgfqpoint{3.696000in}{3.696000in}}%
\pgfusepath{clip}%
\pgfsetbuttcap%
\pgfsetroundjoin%
\definecolor{currentfill}{rgb}{0.121569,0.466667,0.705882}%
\pgfsetfillcolor{currentfill}%
\pgfsetfillopacity{0.551101}%
\pgfsetlinewidth{1.003750pt}%
\definecolor{currentstroke}{rgb}{0.121569,0.466667,0.705882}%
\pgfsetstrokecolor{currentstroke}%
\pgfsetstrokeopacity{0.551101}%
\pgfsetdash{}{0pt}%
\pgfpathmoveto{\pgfqpoint{1.021290in}{1.632893in}}%
\pgfpathcurveto{\pgfqpoint{1.029526in}{1.632893in}}{\pgfqpoint{1.037426in}{1.636165in}}{\pgfqpoint{1.043250in}{1.641989in}}%
\pgfpathcurveto{\pgfqpoint{1.049074in}{1.647813in}}{\pgfqpoint{1.052346in}{1.655713in}}{\pgfqpoint{1.052346in}{1.663949in}}%
\pgfpathcurveto{\pgfqpoint{1.052346in}{1.672186in}}{\pgfqpoint{1.049074in}{1.680086in}}{\pgfqpoint{1.043250in}{1.685910in}}%
\pgfpathcurveto{\pgfqpoint{1.037426in}{1.691734in}}{\pgfqpoint{1.029526in}{1.695006in}}{\pgfqpoint{1.021290in}{1.695006in}}%
\pgfpathcurveto{\pgfqpoint{1.013053in}{1.695006in}}{\pgfqpoint{1.005153in}{1.691734in}}{\pgfqpoint{0.999329in}{1.685910in}}%
\pgfpathcurveto{\pgfqpoint{0.993505in}{1.680086in}}{\pgfqpoint{0.990233in}{1.672186in}}{\pgfqpoint{0.990233in}{1.663949in}}%
\pgfpathcurveto{\pgfqpoint{0.990233in}{1.655713in}}{\pgfqpoint{0.993505in}{1.647813in}}{\pgfqpoint{0.999329in}{1.641989in}}%
\pgfpathcurveto{\pgfqpoint{1.005153in}{1.636165in}}{\pgfqpoint{1.013053in}{1.632893in}}{\pgfqpoint{1.021290in}{1.632893in}}%
\pgfpathclose%
\pgfusepath{stroke,fill}%
\end{pgfscope}%
\begin{pgfscope}%
\pgfpathrectangle{\pgfqpoint{0.100000in}{0.220728in}}{\pgfqpoint{3.696000in}{3.696000in}}%
\pgfusepath{clip}%
\pgfsetbuttcap%
\pgfsetroundjoin%
\definecolor{currentfill}{rgb}{0.121569,0.466667,0.705882}%
\pgfsetfillcolor{currentfill}%
\pgfsetfillopacity{0.551414}%
\pgfsetlinewidth{1.003750pt}%
\definecolor{currentstroke}{rgb}{0.121569,0.466667,0.705882}%
\pgfsetstrokecolor{currentstroke}%
\pgfsetstrokeopacity{0.551414}%
\pgfsetdash{}{0pt}%
\pgfpathmoveto{\pgfqpoint{2.826601in}{3.036413in}}%
\pgfpathcurveto{\pgfqpoint{2.834838in}{3.036413in}}{\pgfqpoint{2.842738in}{3.039686in}}{\pgfqpoint{2.848562in}{3.045510in}}%
\pgfpathcurveto{\pgfqpoint{2.854386in}{3.051333in}}{\pgfqpoint{2.857658in}{3.059234in}}{\pgfqpoint{2.857658in}{3.067470in}}%
\pgfpathcurveto{\pgfqpoint{2.857658in}{3.075706in}}{\pgfqpoint{2.854386in}{3.083606in}}{\pgfqpoint{2.848562in}{3.089430in}}%
\pgfpathcurveto{\pgfqpoint{2.842738in}{3.095254in}}{\pgfqpoint{2.834838in}{3.098526in}}{\pgfqpoint{2.826601in}{3.098526in}}%
\pgfpathcurveto{\pgfqpoint{2.818365in}{3.098526in}}{\pgfqpoint{2.810465in}{3.095254in}}{\pgfqpoint{2.804641in}{3.089430in}}%
\pgfpathcurveto{\pgfqpoint{2.798817in}{3.083606in}}{\pgfqpoint{2.795545in}{3.075706in}}{\pgfqpoint{2.795545in}{3.067470in}}%
\pgfpathcurveto{\pgfqpoint{2.795545in}{3.059234in}}{\pgfqpoint{2.798817in}{3.051333in}}{\pgfqpoint{2.804641in}{3.045510in}}%
\pgfpathcurveto{\pgfqpoint{2.810465in}{3.039686in}}{\pgfqpoint{2.818365in}{3.036413in}}{\pgfqpoint{2.826601in}{3.036413in}}%
\pgfpathclose%
\pgfusepath{stroke,fill}%
\end{pgfscope}%
\begin{pgfscope}%
\pgfpathrectangle{\pgfqpoint{0.100000in}{0.220728in}}{\pgfqpoint{3.696000in}{3.696000in}}%
\pgfusepath{clip}%
\pgfsetbuttcap%
\pgfsetroundjoin%
\definecolor{currentfill}{rgb}{0.121569,0.466667,0.705882}%
\pgfsetfillcolor{currentfill}%
\pgfsetfillopacity{0.551669}%
\pgfsetlinewidth{1.003750pt}%
\definecolor{currentstroke}{rgb}{0.121569,0.466667,0.705882}%
\pgfsetstrokecolor{currentstroke}%
\pgfsetstrokeopacity{0.551669}%
\pgfsetdash{}{0pt}%
\pgfpathmoveto{\pgfqpoint{1.020570in}{1.629391in}}%
\pgfpathcurveto{\pgfqpoint{1.028806in}{1.629391in}}{\pgfqpoint{1.036706in}{1.632663in}}{\pgfqpoint{1.042530in}{1.638487in}}%
\pgfpathcurveto{\pgfqpoint{1.048354in}{1.644311in}}{\pgfqpoint{1.051626in}{1.652211in}}{\pgfqpoint{1.051626in}{1.660447in}}%
\pgfpathcurveto{\pgfqpoint{1.051626in}{1.668683in}}{\pgfqpoint{1.048354in}{1.676583in}}{\pgfqpoint{1.042530in}{1.682407in}}%
\pgfpathcurveto{\pgfqpoint{1.036706in}{1.688231in}}{\pgfqpoint{1.028806in}{1.691504in}}{\pgfqpoint{1.020570in}{1.691504in}}%
\pgfpathcurveto{\pgfqpoint{1.012333in}{1.691504in}}{\pgfqpoint{1.004433in}{1.688231in}}{\pgfqpoint{0.998609in}{1.682407in}}%
\pgfpathcurveto{\pgfqpoint{0.992785in}{1.676583in}}{\pgfqpoint{0.989513in}{1.668683in}}{\pgfqpoint{0.989513in}{1.660447in}}%
\pgfpathcurveto{\pgfqpoint{0.989513in}{1.652211in}}{\pgfqpoint{0.992785in}{1.644311in}}{\pgfqpoint{0.998609in}{1.638487in}}%
\pgfpathcurveto{\pgfqpoint{1.004433in}{1.632663in}}{\pgfqpoint{1.012333in}{1.629391in}}{\pgfqpoint{1.020570in}{1.629391in}}%
\pgfpathclose%
\pgfusepath{stroke,fill}%
\end{pgfscope}%
\begin{pgfscope}%
\pgfpathrectangle{\pgfqpoint{0.100000in}{0.220728in}}{\pgfqpoint{3.696000in}{3.696000in}}%
\pgfusepath{clip}%
\pgfsetbuttcap%
\pgfsetroundjoin%
\definecolor{currentfill}{rgb}{0.121569,0.466667,0.705882}%
\pgfsetfillcolor{currentfill}%
\pgfsetfillopacity{0.551949}%
\pgfsetlinewidth{1.003750pt}%
\definecolor{currentstroke}{rgb}{0.121569,0.466667,0.705882}%
\pgfsetstrokecolor{currentstroke}%
\pgfsetstrokeopacity{0.551949}%
\pgfsetdash{}{0pt}%
\pgfpathmoveto{\pgfqpoint{1.018761in}{1.627303in}}%
\pgfpathcurveto{\pgfqpoint{1.026997in}{1.627303in}}{\pgfqpoint{1.034897in}{1.630575in}}{\pgfqpoint{1.040721in}{1.636399in}}%
\pgfpathcurveto{\pgfqpoint{1.046545in}{1.642223in}}{\pgfqpoint{1.049818in}{1.650123in}}{\pgfqpoint{1.049818in}{1.658359in}}%
\pgfpathcurveto{\pgfqpoint{1.049818in}{1.666596in}}{\pgfqpoint{1.046545in}{1.674496in}}{\pgfqpoint{1.040721in}{1.680320in}}%
\pgfpathcurveto{\pgfqpoint{1.034897in}{1.686144in}}{\pgfqpoint{1.026997in}{1.689416in}}{\pgfqpoint{1.018761in}{1.689416in}}%
\pgfpathcurveto{\pgfqpoint{1.010525in}{1.689416in}}{\pgfqpoint{1.002625in}{1.686144in}}{\pgfqpoint{0.996801in}{1.680320in}}%
\pgfpathcurveto{\pgfqpoint{0.990977in}{1.674496in}}{\pgfqpoint{0.987705in}{1.666596in}}{\pgfqpoint{0.987705in}{1.658359in}}%
\pgfpathcurveto{\pgfqpoint{0.987705in}{1.650123in}}{\pgfqpoint{0.990977in}{1.642223in}}{\pgfqpoint{0.996801in}{1.636399in}}%
\pgfpathcurveto{\pgfqpoint{1.002625in}{1.630575in}}{\pgfqpoint{1.010525in}{1.627303in}}{\pgfqpoint{1.018761in}{1.627303in}}%
\pgfpathclose%
\pgfusepath{stroke,fill}%
\end{pgfscope}%
\begin{pgfscope}%
\pgfpathrectangle{\pgfqpoint{0.100000in}{0.220728in}}{\pgfqpoint{3.696000in}{3.696000in}}%
\pgfusepath{clip}%
\pgfsetbuttcap%
\pgfsetroundjoin%
\definecolor{currentfill}{rgb}{0.121569,0.466667,0.705882}%
\pgfsetfillcolor{currentfill}%
\pgfsetfillopacity{0.552336}%
\pgfsetlinewidth{1.003750pt}%
\definecolor{currentstroke}{rgb}{0.121569,0.466667,0.705882}%
\pgfsetstrokecolor{currentstroke}%
\pgfsetstrokeopacity{0.552336}%
\pgfsetdash{}{0pt}%
\pgfpathmoveto{\pgfqpoint{2.831843in}{3.035722in}}%
\pgfpathcurveto{\pgfqpoint{2.840079in}{3.035722in}}{\pgfqpoint{2.847979in}{3.038994in}}{\pgfqpoint{2.853803in}{3.044818in}}%
\pgfpathcurveto{\pgfqpoint{2.859627in}{3.050642in}}{\pgfqpoint{2.862899in}{3.058542in}}{\pgfqpoint{2.862899in}{3.066779in}}%
\pgfpathcurveto{\pgfqpoint{2.862899in}{3.075015in}}{\pgfqpoint{2.859627in}{3.082915in}}{\pgfqpoint{2.853803in}{3.088739in}}%
\pgfpathcurveto{\pgfqpoint{2.847979in}{3.094563in}}{\pgfqpoint{2.840079in}{3.097835in}}{\pgfqpoint{2.831843in}{3.097835in}}%
\pgfpathcurveto{\pgfqpoint{2.823607in}{3.097835in}}{\pgfqpoint{2.815707in}{3.094563in}}{\pgfqpoint{2.809883in}{3.088739in}}%
\pgfpathcurveto{\pgfqpoint{2.804059in}{3.082915in}}{\pgfqpoint{2.800786in}{3.075015in}}{\pgfqpoint{2.800786in}{3.066779in}}%
\pgfpathcurveto{\pgfqpoint{2.800786in}{3.058542in}}{\pgfqpoint{2.804059in}{3.050642in}}{\pgfqpoint{2.809883in}{3.044818in}}%
\pgfpathcurveto{\pgfqpoint{2.815707in}{3.038994in}}{\pgfqpoint{2.823607in}{3.035722in}}{\pgfqpoint{2.831843in}{3.035722in}}%
\pgfpathclose%
\pgfusepath{stroke,fill}%
\end{pgfscope}%
\begin{pgfscope}%
\pgfpathrectangle{\pgfqpoint{0.100000in}{0.220728in}}{\pgfqpoint{3.696000in}{3.696000in}}%
\pgfusepath{clip}%
\pgfsetbuttcap%
\pgfsetroundjoin%
\definecolor{currentfill}{rgb}{0.121569,0.466667,0.705882}%
\pgfsetfillcolor{currentfill}%
\pgfsetfillopacity{0.552827}%
\pgfsetlinewidth{1.003750pt}%
\definecolor{currentstroke}{rgb}{0.121569,0.466667,0.705882}%
\pgfsetstrokecolor{currentstroke}%
\pgfsetstrokeopacity{0.552827}%
\pgfsetdash{}{0pt}%
\pgfpathmoveto{\pgfqpoint{1.017189in}{1.622622in}}%
\pgfpathcurveto{\pgfqpoint{1.025425in}{1.622622in}}{\pgfqpoint{1.033325in}{1.625894in}}{\pgfqpoint{1.039149in}{1.631718in}}%
\pgfpathcurveto{\pgfqpoint{1.044973in}{1.637542in}}{\pgfqpoint{1.048245in}{1.645442in}}{\pgfqpoint{1.048245in}{1.653678in}}%
\pgfpathcurveto{\pgfqpoint{1.048245in}{1.661914in}}{\pgfqpoint{1.044973in}{1.669814in}}{\pgfqpoint{1.039149in}{1.675638in}}%
\pgfpathcurveto{\pgfqpoint{1.033325in}{1.681462in}}{\pgfqpoint{1.025425in}{1.684735in}}{\pgfqpoint{1.017189in}{1.684735in}}%
\pgfpathcurveto{\pgfqpoint{1.008953in}{1.684735in}}{\pgfqpoint{1.001053in}{1.681462in}}{\pgfqpoint{0.995229in}{1.675638in}}%
\pgfpathcurveto{\pgfqpoint{0.989405in}{1.669814in}}{\pgfqpoint{0.986132in}{1.661914in}}{\pgfqpoint{0.986132in}{1.653678in}}%
\pgfpathcurveto{\pgfqpoint{0.986132in}{1.645442in}}{\pgfqpoint{0.989405in}{1.637542in}}{\pgfqpoint{0.995229in}{1.631718in}}%
\pgfpathcurveto{\pgfqpoint{1.001053in}{1.625894in}}{\pgfqpoint{1.008953in}{1.622622in}}{\pgfqpoint{1.017189in}{1.622622in}}%
\pgfpathclose%
\pgfusepath{stroke,fill}%
\end{pgfscope}%
\begin{pgfscope}%
\pgfpathrectangle{\pgfqpoint{0.100000in}{0.220728in}}{\pgfqpoint{3.696000in}{3.696000in}}%
\pgfusepath{clip}%
\pgfsetbuttcap%
\pgfsetroundjoin%
\definecolor{currentfill}{rgb}{0.121569,0.466667,0.705882}%
\pgfsetfillcolor{currentfill}%
\pgfsetfillopacity{0.553160}%
\pgfsetlinewidth{1.003750pt}%
\definecolor{currentstroke}{rgb}{0.121569,0.466667,0.705882}%
\pgfsetstrokecolor{currentstroke}%
\pgfsetstrokeopacity{0.553160}%
\pgfsetdash{}{0pt}%
\pgfpathmoveto{\pgfqpoint{2.834441in}{3.035595in}}%
\pgfpathcurveto{\pgfqpoint{2.842677in}{3.035595in}}{\pgfqpoint{2.850577in}{3.038867in}}{\pgfqpoint{2.856401in}{3.044691in}}%
\pgfpathcurveto{\pgfqpoint{2.862225in}{3.050515in}}{\pgfqpoint{2.865497in}{3.058415in}}{\pgfqpoint{2.865497in}{3.066651in}}%
\pgfpathcurveto{\pgfqpoint{2.865497in}{3.074888in}}{\pgfqpoint{2.862225in}{3.082788in}}{\pgfqpoint{2.856401in}{3.088612in}}%
\pgfpathcurveto{\pgfqpoint{2.850577in}{3.094436in}}{\pgfqpoint{2.842677in}{3.097708in}}{\pgfqpoint{2.834441in}{3.097708in}}%
\pgfpathcurveto{\pgfqpoint{2.826205in}{3.097708in}}{\pgfqpoint{2.818305in}{3.094436in}}{\pgfqpoint{2.812481in}{3.088612in}}%
\pgfpathcurveto{\pgfqpoint{2.806657in}{3.082788in}}{\pgfqpoint{2.803384in}{3.074888in}}{\pgfqpoint{2.803384in}{3.066651in}}%
\pgfpathcurveto{\pgfqpoint{2.803384in}{3.058415in}}{\pgfqpoint{2.806657in}{3.050515in}}{\pgfqpoint{2.812481in}{3.044691in}}%
\pgfpathcurveto{\pgfqpoint{2.818305in}{3.038867in}}{\pgfqpoint{2.826205in}{3.035595in}}{\pgfqpoint{2.834441in}{3.035595in}}%
\pgfpathclose%
\pgfusepath{stroke,fill}%
\end{pgfscope}%
\begin{pgfscope}%
\pgfpathrectangle{\pgfqpoint{0.100000in}{0.220728in}}{\pgfqpoint{3.696000in}{3.696000in}}%
\pgfusepath{clip}%
\pgfsetbuttcap%
\pgfsetroundjoin%
\definecolor{currentfill}{rgb}{0.121569,0.466667,0.705882}%
\pgfsetfillcolor{currentfill}%
\pgfsetfillopacity{0.553854}%
\pgfsetlinewidth{1.003750pt}%
\definecolor{currentstroke}{rgb}{0.121569,0.466667,0.705882}%
\pgfsetstrokecolor{currentstroke}%
\pgfsetstrokeopacity{0.553854}%
\pgfsetdash{}{0pt}%
\pgfpathmoveto{\pgfqpoint{1.011471in}{1.615500in}}%
\pgfpathcurveto{\pgfqpoint{1.019708in}{1.615500in}}{\pgfqpoint{1.027608in}{1.618773in}}{\pgfqpoint{1.033431in}{1.624596in}}%
\pgfpathcurveto{\pgfqpoint{1.039255in}{1.630420in}}{\pgfqpoint{1.042528in}{1.638320in}}{\pgfqpoint{1.042528in}{1.646557in}}%
\pgfpathcurveto{\pgfqpoint{1.042528in}{1.654793in}}{\pgfqpoint{1.039255in}{1.662693in}}{\pgfqpoint{1.033431in}{1.668517in}}%
\pgfpathcurveto{\pgfqpoint{1.027608in}{1.674341in}}{\pgfqpoint{1.019708in}{1.677613in}}{\pgfqpoint{1.011471in}{1.677613in}}%
\pgfpathcurveto{\pgfqpoint{1.003235in}{1.677613in}}{\pgfqpoint{0.995335in}{1.674341in}}{\pgfqpoint{0.989511in}{1.668517in}}%
\pgfpathcurveto{\pgfqpoint{0.983687in}{1.662693in}}{\pgfqpoint{0.980415in}{1.654793in}}{\pgfqpoint{0.980415in}{1.646557in}}%
\pgfpathcurveto{\pgfqpoint{0.980415in}{1.638320in}}{\pgfqpoint{0.983687in}{1.630420in}}{\pgfqpoint{0.989511in}{1.624596in}}%
\pgfpathcurveto{\pgfqpoint{0.995335in}{1.618773in}}{\pgfqpoint{1.003235in}{1.615500in}}{\pgfqpoint{1.011471in}{1.615500in}}%
\pgfpathclose%
\pgfusepath{stroke,fill}%
\end{pgfscope}%
\begin{pgfscope}%
\pgfpathrectangle{\pgfqpoint{0.100000in}{0.220728in}}{\pgfqpoint{3.696000in}{3.696000in}}%
\pgfusepath{clip}%
\pgfsetbuttcap%
\pgfsetroundjoin%
\definecolor{currentfill}{rgb}{0.121569,0.466667,0.705882}%
\pgfsetfillcolor{currentfill}%
\pgfsetfillopacity{0.554039}%
\pgfsetlinewidth{1.003750pt}%
\definecolor{currentstroke}{rgb}{0.121569,0.466667,0.705882}%
\pgfsetstrokecolor{currentstroke}%
\pgfsetstrokeopacity{0.554039}%
\pgfsetdash{}{0pt}%
\pgfpathmoveto{\pgfqpoint{2.838212in}{3.034944in}}%
\pgfpathcurveto{\pgfqpoint{2.846449in}{3.034944in}}{\pgfqpoint{2.854349in}{3.038216in}}{\pgfqpoint{2.860173in}{3.044040in}}%
\pgfpathcurveto{\pgfqpoint{2.865997in}{3.049864in}}{\pgfqpoint{2.869269in}{3.057764in}}{\pgfqpoint{2.869269in}{3.066000in}}%
\pgfpathcurveto{\pgfqpoint{2.869269in}{3.074237in}}{\pgfqpoint{2.865997in}{3.082137in}}{\pgfqpoint{2.860173in}{3.087961in}}%
\pgfpathcurveto{\pgfqpoint{2.854349in}{3.093784in}}{\pgfqpoint{2.846449in}{3.097057in}}{\pgfqpoint{2.838212in}{3.097057in}}%
\pgfpathcurveto{\pgfqpoint{2.829976in}{3.097057in}}{\pgfqpoint{2.822076in}{3.093784in}}{\pgfqpoint{2.816252in}{3.087961in}}%
\pgfpathcurveto{\pgfqpoint{2.810428in}{3.082137in}}{\pgfqpoint{2.807156in}{3.074237in}}{\pgfqpoint{2.807156in}{3.066000in}}%
\pgfpathcurveto{\pgfqpoint{2.807156in}{3.057764in}}{\pgfqpoint{2.810428in}{3.049864in}}{\pgfqpoint{2.816252in}{3.044040in}}%
\pgfpathcurveto{\pgfqpoint{2.822076in}{3.038216in}}{\pgfqpoint{2.829976in}{3.034944in}}{\pgfqpoint{2.838212in}{3.034944in}}%
\pgfpathclose%
\pgfusepath{stroke,fill}%
\end{pgfscope}%
\begin{pgfscope}%
\pgfpathrectangle{\pgfqpoint{0.100000in}{0.220728in}}{\pgfqpoint{3.696000in}{3.696000in}}%
\pgfusepath{clip}%
\pgfsetbuttcap%
\pgfsetroundjoin%
\definecolor{currentfill}{rgb}{0.121569,0.466667,0.705882}%
\pgfsetfillcolor{currentfill}%
\pgfsetfillopacity{0.554632}%
\pgfsetlinewidth{1.003750pt}%
\definecolor{currentstroke}{rgb}{0.121569,0.466667,0.705882}%
\pgfsetstrokecolor{currentstroke}%
\pgfsetstrokeopacity{0.554632}%
\pgfsetdash{}{0pt}%
\pgfpathmoveto{\pgfqpoint{2.840191in}{3.034718in}}%
\pgfpathcurveto{\pgfqpoint{2.848427in}{3.034718in}}{\pgfqpoint{2.856327in}{3.037990in}}{\pgfqpoint{2.862151in}{3.043814in}}%
\pgfpathcurveto{\pgfqpoint{2.867975in}{3.049638in}}{\pgfqpoint{2.871248in}{3.057538in}}{\pgfqpoint{2.871248in}{3.065774in}}%
\pgfpathcurveto{\pgfqpoint{2.871248in}{3.074011in}}{\pgfqpoint{2.867975in}{3.081911in}}{\pgfqpoint{2.862151in}{3.087735in}}%
\pgfpathcurveto{\pgfqpoint{2.856327in}{3.093558in}}{\pgfqpoint{2.848427in}{3.096831in}}{\pgfqpoint{2.840191in}{3.096831in}}%
\pgfpathcurveto{\pgfqpoint{2.831955in}{3.096831in}}{\pgfqpoint{2.824055in}{3.093558in}}{\pgfqpoint{2.818231in}{3.087735in}}%
\pgfpathcurveto{\pgfqpoint{2.812407in}{3.081911in}}{\pgfqpoint{2.809135in}{3.074011in}}{\pgfqpoint{2.809135in}{3.065774in}}%
\pgfpathcurveto{\pgfqpoint{2.809135in}{3.057538in}}{\pgfqpoint{2.812407in}{3.049638in}}{\pgfqpoint{2.818231in}{3.043814in}}%
\pgfpathcurveto{\pgfqpoint{2.824055in}{3.037990in}}{\pgfqpoint{2.831955in}{3.034718in}}{\pgfqpoint{2.840191in}{3.034718in}}%
\pgfpathclose%
\pgfusepath{stroke,fill}%
\end{pgfscope}%
\begin{pgfscope}%
\pgfpathrectangle{\pgfqpoint{0.100000in}{0.220728in}}{\pgfqpoint{3.696000in}{3.696000in}}%
\pgfusepath{clip}%
\pgfsetbuttcap%
\pgfsetroundjoin%
\definecolor{currentfill}{rgb}{0.121569,0.466667,0.705882}%
\pgfsetfillcolor{currentfill}%
\pgfsetfillopacity{0.555249}%
\pgfsetlinewidth{1.003750pt}%
\definecolor{currentstroke}{rgb}{0.121569,0.466667,0.705882}%
\pgfsetstrokecolor{currentstroke}%
\pgfsetstrokeopacity{0.555249}%
\pgfsetdash{}{0pt}%
\pgfpathmoveto{\pgfqpoint{1.008548in}{1.607463in}}%
\pgfpathcurveto{\pgfqpoint{1.016784in}{1.607463in}}{\pgfqpoint{1.024684in}{1.610736in}}{\pgfqpoint{1.030508in}{1.616560in}}%
\pgfpathcurveto{\pgfqpoint{1.036332in}{1.622383in}}{\pgfqpoint{1.039605in}{1.630284in}}{\pgfqpoint{1.039605in}{1.638520in}}%
\pgfpathcurveto{\pgfqpoint{1.039605in}{1.646756in}}{\pgfqpoint{1.036332in}{1.654656in}}{\pgfqpoint{1.030508in}{1.660480in}}%
\pgfpathcurveto{\pgfqpoint{1.024684in}{1.666304in}}{\pgfqpoint{1.016784in}{1.669576in}}{\pgfqpoint{1.008548in}{1.669576in}}%
\pgfpathcurveto{\pgfqpoint{1.000312in}{1.669576in}}{\pgfqpoint{0.992412in}{1.666304in}}{\pgfqpoint{0.986588in}{1.660480in}}%
\pgfpathcurveto{\pgfqpoint{0.980764in}{1.654656in}}{\pgfqpoint{0.977492in}{1.646756in}}{\pgfqpoint{0.977492in}{1.638520in}}%
\pgfpathcurveto{\pgfqpoint{0.977492in}{1.630284in}}{\pgfqpoint{0.980764in}{1.622383in}}{\pgfqpoint{0.986588in}{1.616560in}}%
\pgfpathcurveto{\pgfqpoint{0.992412in}{1.610736in}}{\pgfqpoint{1.000312in}{1.607463in}}{\pgfqpoint{1.008548in}{1.607463in}}%
\pgfpathclose%
\pgfusepath{stroke,fill}%
\end{pgfscope}%
\begin{pgfscope}%
\pgfpathrectangle{\pgfqpoint{0.100000in}{0.220728in}}{\pgfqpoint{3.696000in}{3.696000in}}%
\pgfusepath{clip}%
\pgfsetbuttcap%
\pgfsetroundjoin%
\definecolor{currentfill}{rgb}{0.121569,0.466667,0.705882}%
\pgfsetfillcolor{currentfill}%
\pgfsetfillopacity{0.555384}%
\pgfsetlinewidth{1.003750pt}%
\definecolor{currentstroke}{rgb}{0.121569,0.466667,0.705882}%
\pgfsetstrokecolor{currentstroke}%
\pgfsetstrokeopacity{0.555384}%
\pgfsetdash{}{0pt}%
\pgfpathmoveto{\pgfqpoint{2.843256in}{3.034046in}}%
\pgfpathcurveto{\pgfqpoint{2.851492in}{3.034046in}}{\pgfqpoint{2.859392in}{3.037318in}}{\pgfqpoint{2.865216in}{3.043142in}}%
\pgfpathcurveto{\pgfqpoint{2.871040in}{3.048966in}}{\pgfqpoint{2.874312in}{3.056866in}}{\pgfqpoint{2.874312in}{3.065102in}}%
\pgfpathcurveto{\pgfqpoint{2.874312in}{3.073338in}}{\pgfqpoint{2.871040in}{3.081238in}}{\pgfqpoint{2.865216in}{3.087062in}}%
\pgfpathcurveto{\pgfqpoint{2.859392in}{3.092886in}}{\pgfqpoint{2.851492in}{3.096159in}}{\pgfqpoint{2.843256in}{3.096159in}}%
\pgfpathcurveto{\pgfqpoint{2.835019in}{3.096159in}}{\pgfqpoint{2.827119in}{3.092886in}}{\pgfqpoint{2.821295in}{3.087062in}}%
\pgfpathcurveto{\pgfqpoint{2.815472in}{3.081238in}}{\pgfqpoint{2.812199in}{3.073338in}}{\pgfqpoint{2.812199in}{3.065102in}}%
\pgfpathcurveto{\pgfqpoint{2.812199in}{3.056866in}}{\pgfqpoint{2.815472in}{3.048966in}}{\pgfqpoint{2.821295in}{3.043142in}}%
\pgfpathcurveto{\pgfqpoint{2.827119in}{3.037318in}}{\pgfqpoint{2.835019in}{3.034046in}}{\pgfqpoint{2.843256in}{3.034046in}}%
\pgfpathclose%
\pgfusepath{stroke,fill}%
\end{pgfscope}%
\begin{pgfscope}%
\pgfpathrectangle{\pgfqpoint{0.100000in}{0.220728in}}{\pgfqpoint{3.696000in}{3.696000in}}%
\pgfusepath{clip}%
\pgfsetbuttcap%
\pgfsetroundjoin%
\definecolor{currentfill}{rgb}{0.121569,0.466667,0.705882}%
\pgfsetfillcolor{currentfill}%
\pgfsetfillopacity{0.556168}%
\pgfsetlinewidth{1.003750pt}%
\definecolor{currentstroke}{rgb}{0.121569,0.466667,0.705882}%
\pgfsetstrokecolor{currentstroke}%
\pgfsetstrokeopacity{0.556168}%
\pgfsetdash{}{0pt}%
\pgfpathmoveto{\pgfqpoint{2.846949in}{3.033168in}}%
\pgfpathcurveto{\pgfqpoint{2.855185in}{3.033168in}}{\pgfqpoint{2.863085in}{3.036440in}}{\pgfqpoint{2.868909in}{3.042264in}}%
\pgfpathcurveto{\pgfqpoint{2.874733in}{3.048088in}}{\pgfqpoint{2.878005in}{3.055988in}}{\pgfqpoint{2.878005in}{3.064224in}}%
\pgfpathcurveto{\pgfqpoint{2.878005in}{3.072460in}}{\pgfqpoint{2.874733in}{3.080360in}}{\pgfqpoint{2.868909in}{3.086184in}}%
\pgfpathcurveto{\pgfqpoint{2.863085in}{3.092008in}}{\pgfqpoint{2.855185in}{3.095281in}}{\pgfqpoint{2.846949in}{3.095281in}}%
\pgfpathcurveto{\pgfqpoint{2.838712in}{3.095281in}}{\pgfqpoint{2.830812in}{3.092008in}}{\pgfqpoint{2.824988in}{3.086184in}}%
\pgfpathcurveto{\pgfqpoint{2.819164in}{3.080360in}}{\pgfqpoint{2.815892in}{3.072460in}}{\pgfqpoint{2.815892in}{3.064224in}}%
\pgfpathcurveto{\pgfqpoint{2.815892in}{3.055988in}}{\pgfqpoint{2.819164in}{3.048088in}}{\pgfqpoint{2.824988in}{3.042264in}}%
\pgfpathcurveto{\pgfqpoint{2.830812in}{3.036440in}}{\pgfqpoint{2.838712in}{3.033168in}}{\pgfqpoint{2.846949in}{3.033168in}}%
\pgfpathclose%
\pgfusepath{stroke,fill}%
\end{pgfscope}%
\begin{pgfscope}%
\pgfpathrectangle{\pgfqpoint{0.100000in}{0.220728in}}{\pgfqpoint{3.696000in}{3.696000in}}%
\pgfusepath{clip}%
\pgfsetbuttcap%
\pgfsetroundjoin%
\definecolor{currentfill}{rgb}{0.121569,0.466667,0.705882}%
\pgfsetfillcolor{currentfill}%
\pgfsetfillopacity{0.556418}%
\pgfsetlinewidth{1.003750pt}%
\definecolor{currentstroke}{rgb}{0.121569,0.466667,0.705882}%
\pgfsetstrokecolor{currentstroke}%
\pgfsetstrokeopacity{0.556418}%
\pgfsetdash{}{0pt}%
\pgfpathmoveto{\pgfqpoint{1.004583in}{1.601144in}}%
\pgfpathcurveto{\pgfqpoint{1.012819in}{1.601144in}}{\pgfqpoint{1.020719in}{1.604417in}}{\pgfqpoint{1.026543in}{1.610241in}}%
\pgfpathcurveto{\pgfqpoint{1.032367in}{1.616064in}}{\pgfqpoint{1.035639in}{1.623965in}}{\pgfqpoint{1.035639in}{1.632201in}}%
\pgfpathcurveto{\pgfqpoint{1.035639in}{1.640437in}}{\pgfqpoint{1.032367in}{1.648337in}}{\pgfqpoint{1.026543in}{1.654161in}}%
\pgfpathcurveto{\pgfqpoint{1.020719in}{1.659985in}}{\pgfqpoint{1.012819in}{1.663257in}}{\pgfqpoint{1.004583in}{1.663257in}}%
\pgfpathcurveto{\pgfqpoint{0.996346in}{1.663257in}}{\pgfqpoint{0.988446in}{1.659985in}}{\pgfqpoint{0.982622in}{1.654161in}}%
\pgfpathcurveto{\pgfqpoint{0.976798in}{1.648337in}}{\pgfqpoint{0.973526in}{1.640437in}}{\pgfqpoint{0.973526in}{1.632201in}}%
\pgfpathcurveto{\pgfqpoint{0.973526in}{1.623965in}}{\pgfqpoint{0.976798in}{1.616064in}}{\pgfqpoint{0.982622in}{1.610241in}}%
\pgfpathcurveto{\pgfqpoint{0.988446in}{1.604417in}}{\pgfqpoint{0.996346in}{1.601144in}}{\pgfqpoint{1.004583in}{1.601144in}}%
\pgfpathclose%
\pgfusepath{stroke,fill}%
\end{pgfscope}%
\begin{pgfscope}%
\pgfpathrectangle{\pgfqpoint{0.100000in}{0.220728in}}{\pgfqpoint{3.696000in}{3.696000in}}%
\pgfusepath{clip}%
\pgfsetbuttcap%
\pgfsetroundjoin%
\definecolor{currentfill}{rgb}{0.121569,0.466667,0.705882}%
\pgfsetfillcolor{currentfill}%
\pgfsetfillopacity{0.556807}%
\pgfsetlinewidth{1.003750pt}%
\definecolor{currentstroke}{rgb}{0.121569,0.466667,0.705882}%
\pgfsetstrokecolor{currentstroke}%
\pgfsetstrokeopacity{0.556807}%
\pgfsetdash{}{0pt}%
\pgfpathmoveto{\pgfqpoint{2.848802in}{3.032979in}}%
\pgfpathcurveto{\pgfqpoint{2.857039in}{3.032979in}}{\pgfqpoint{2.864939in}{3.036252in}}{\pgfqpoint{2.870763in}{3.042076in}}%
\pgfpathcurveto{\pgfqpoint{2.876587in}{3.047900in}}{\pgfqpoint{2.879859in}{3.055800in}}{\pgfqpoint{2.879859in}{3.064036in}}%
\pgfpathcurveto{\pgfqpoint{2.879859in}{3.072272in}}{\pgfqpoint{2.876587in}{3.080172in}}{\pgfqpoint{2.870763in}{3.085996in}}%
\pgfpathcurveto{\pgfqpoint{2.864939in}{3.091820in}}{\pgfqpoint{2.857039in}{3.095092in}}{\pgfqpoint{2.848802in}{3.095092in}}%
\pgfpathcurveto{\pgfqpoint{2.840566in}{3.095092in}}{\pgfqpoint{2.832666in}{3.091820in}}{\pgfqpoint{2.826842in}{3.085996in}}%
\pgfpathcurveto{\pgfqpoint{2.821018in}{3.080172in}}{\pgfqpoint{2.817746in}{3.072272in}}{\pgfqpoint{2.817746in}{3.064036in}}%
\pgfpathcurveto{\pgfqpoint{2.817746in}{3.055800in}}{\pgfqpoint{2.821018in}{3.047900in}}{\pgfqpoint{2.826842in}{3.042076in}}%
\pgfpathcurveto{\pgfqpoint{2.832666in}{3.036252in}}{\pgfqpoint{2.840566in}{3.032979in}}{\pgfqpoint{2.848802in}{3.032979in}}%
\pgfpathclose%
\pgfusepath{stroke,fill}%
\end{pgfscope}%
\begin{pgfscope}%
\pgfpathrectangle{\pgfqpoint{0.100000in}{0.220728in}}{\pgfqpoint{3.696000in}{3.696000in}}%
\pgfusepath{clip}%
\pgfsetbuttcap%
\pgfsetroundjoin%
\definecolor{currentfill}{rgb}{0.121569,0.466667,0.705882}%
\pgfsetfillcolor{currentfill}%
\pgfsetfillopacity{0.557065}%
\pgfsetlinewidth{1.003750pt}%
\definecolor{currentstroke}{rgb}{0.121569,0.466667,0.705882}%
\pgfsetstrokecolor{currentstroke}%
\pgfsetstrokeopacity{0.557065}%
\pgfsetdash{}{0pt}%
\pgfpathmoveto{\pgfqpoint{2.849881in}{3.032669in}}%
\pgfpathcurveto{\pgfqpoint{2.858117in}{3.032669in}}{\pgfqpoint{2.866017in}{3.035941in}}{\pgfqpoint{2.871841in}{3.041765in}}%
\pgfpathcurveto{\pgfqpoint{2.877665in}{3.047589in}}{\pgfqpoint{2.880938in}{3.055489in}}{\pgfqpoint{2.880938in}{3.063726in}}%
\pgfpathcurveto{\pgfqpoint{2.880938in}{3.071962in}}{\pgfqpoint{2.877665in}{3.079862in}}{\pgfqpoint{2.871841in}{3.085686in}}%
\pgfpathcurveto{\pgfqpoint{2.866017in}{3.091510in}}{\pgfqpoint{2.858117in}{3.094782in}}{\pgfqpoint{2.849881in}{3.094782in}}%
\pgfpathcurveto{\pgfqpoint{2.841645in}{3.094782in}}{\pgfqpoint{2.833745in}{3.091510in}}{\pgfqpoint{2.827921in}{3.085686in}}%
\pgfpathcurveto{\pgfqpoint{2.822097in}{3.079862in}}{\pgfqpoint{2.818825in}{3.071962in}}{\pgfqpoint{2.818825in}{3.063726in}}%
\pgfpathcurveto{\pgfqpoint{2.818825in}{3.055489in}}{\pgfqpoint{2.822097in}{3.047589in}}{\pgfqpoint{2.827921in}{3.041765in}}%
\pgfpathcurveto{\pgfqpoint{2.833745in}{3.035941in}}{\pgfqpoint{2.841645in}{3.032669in}}{\pgfqpoint{2.849881in}{3.032669in}}%
\pgfpathclose%
\pgfusepath{stroke,fill}%
\end{pgfscope}%
\begin{pgfscope}%
\pgfpathrectangle{\pgfqpoint{0.100000in}{0.220728in}}{\pgfqpoint{3.696000in}{3.696000in}}%
\pgfusepath{clip}%
\pgfsetbuttcap%
\pgfsetroundjoin%
\definecolor{currentfill}{rgb}{0.121569,0.466667,0.705882}%
\pgfsetfillcolor{currentfill}%
\pgfsetfillopacity{0.557254}%
\pgfsetlinewidth{1.003750pt}%
\definecolor{currentstroke}{rgb}{0.121569,0.466667,0.705882}%
\pgfsetstrokecolor{currentstroke}%
\pgfsetstrokeopacity{0.557254}%
\pgfsetdash{}{0pt}%
\pgfpathmoveto{\pgfqpoint{2.852119in}{3.031882in}}%
\pgfpathcurveto{\pgfqpoint{2.860355in}{3.031882in}}{\pgfqpoint{2.868255in}{3.035155in}}{\pgfqpoint{2.874079in}{3.040979in}}%
\pgfpathcurveto{\pgfqpoint{2.879903in}{3.046803in}}{\pgfqpoint{2.883175in}{3.054703in}}{\pgfqpoint{2.883175in}{3.062939in}}%
\pgfpathcurveto{\pgfqpoint{2.883175in}{3.071175in}}{\pgfqpoint{2.879903in}{3.079075in}}{\pgfqpoint{2.874079in}{3.084899in}}%
\pgfpathcurveto{\pgfqpoint{2.868255in}{3.090723in}}{\pgfqpoint{2.860355in}{3.093995in}}{\pgfqpoint{2.852119in}{3.093995in}}%
\pgfpathcurveto{\pgfqpoint{2.843882in}{3.093995in}}{\pgfqpoint{2.835982in}{3.090723in}}{\pgfqpoint{2.830158in}{3.084899in}}%
\pgfpathcurveto{\pgfqpoint{2.824334in}{3.079075in}}{\pgfqpoint{2.821062in}{3.071175in}}{\pgfqpoint{2.821062in}{3.062939in}}%
\pgfpathcurveto{\pgfqpoint{2.821062in}{3.054703in}}{\pgfqpoint{2.824334in}{3.046803in}}{\pgfqpoint{2.830158in}{3.040979in}}%
\pgfpathcurveto{\pgfqpoint{2.835982in}{3.035155in}}{\pgfqpoint{2.843882in}{3.031882in}}{\pgfqpoint{2.852119in}{3.031882in}}%
\pgfpathclose%
\pgfusepath{stroke,fill}%
\end{pgfscope}%
\begin{pgfscope}%
\pgfpathrectangle{\pgfqpoint{0.100000in}{0.220728in}}{\pgfqpoint{3.696000in}{3.696000in}}%
\pgfusepath{clip}%
\pgfsetbuttcap%
\pgfsetroundjoin%
\definecolor{currentfill}{rgb}{0.121569,0.466667,0.705882}%
\pgfsetfillcolor{currentfill}%
\pgfsetfillopacity{0.557404}%
\pgfsetlinewidth{1.003750pt}%
\definecolor{currentstroke}{rgb}{0.121569,0.466667,0.705882}%
\pgfsetstrokecolor{currentstroke}%
\pgfsetstrokeopacity{0.557404}%
\pgfsetdash{}{0pt}%
\pgfpathmoveto{\pgfqpoint{1.001578in}{1.594961in}}%
\pgfpathcurveto{\pgfqpoint{1.009814in}{1.594961in}}{\pgfqpoint{1.017714in}{1.598233in}}{\pgfqpoint{1.023538in}{1.604057in}}%
\pgfpathcurveto{\pgfqpoint{1.029362in}{1.609881in}}{\pgfqpoint{1.032635in}{1.617781in}}{\pgfqpoint{1.032635in}{1.626017in}}%
\pgfpathcurveto{\pgfqpoint{1.032635in}{1.634254in}}{\pgfqpoint{1.029362in}{1.642154in}}{\pgfqpoint{1.023538in}{1.647978in}}%
\pgfpathcurveto{\pgfqpoint{1.017714in}{1.653802in}}{\pgfqpoint{1.009814in}{1.657074in}}{\pgfqpoint{1.001578in}{1.657074in}}%
\pgfpathcurveto{\pgfqpoint{0.993342in}{1.657074in}}{\pgfqpoint{0.985442in}{1.653802in}}{\pgfqpoint{0.979618in}{1.647978in}}%
\pgfpathcurveto{\pgfqpoint{0.973794in}{1.642154in}}{\pgfqpoint{0.970522in}{1.634254in}}{\pgfqpoint{0.970522in}{1.626017in}}%
\pgfpathcurveto{\pgfqpoint{0.970522in}{1.617781in}}{\pgfqpoint{0.973794in}{1.609881in}}{\pgfqpoint{0.979618in}{1.604057in}}%
\pgfpathcurveto{\pgfqpoint{0.985442in}{1.598233in}}{\pgfqpoint{0.993342in}{1.594961in}}{\pgfqpoint{1.001578in}{1.594961in}}%
\pgfpathclose%
\pgfusepath{stroke,fill}%
\end{pgfscope}%
\begin{pgfscope}%
\pgfpathrectangle{\pgfqpoint{0.100000in}{0.220728in}}{\pgfqpoint{3.696000in}{3.696000in}}%
\pgfusepath{clip}%
\pgfsetbuttcap%
\pgfsetroundjoin%
\definecolor{currentfill}{rgb}{0.121569,0.466667,0.705882}%
\pgfsetfillcolor{currentfill}%
\pgfsetfillopacity{0.557566}%
\pgfsetlinewidth{1.003750pt}%
\definecolor{currentstroke}{rgb}{0.121569,0.466667,0.705882}%
\pgfsetstrokecolor{currentstroke}%
\pgfsetstrokeopacity{0.557566}%
\pgfsetdash{}{0pt}%
\pgfpathmoveto{\pgfqpoint{2.853214in}{3.031740in}}%
\pgfpathcurveto{\pgfqpoint{2.861451in}{3.031740in}}{\pgfqpoint{2.869351in}{3.035012in}}{\pgfqpoint{2.875175in}{3.040836in}}%
\pgfpathcurveto{\pgfqpoint{2.880998in}{3.046660in}}{\pgfqpoint{2.884271in}{3.054560in}}{\pgfqpoint{2.884271in}{3.062796in}}%
\pgfpathcurveto{\pgfqpoint{2.884271in}{3.071033in}}{\pgfqpoint{2.880998in}{3.078933in}}{\pgfqpoint{2.875175in}{3.084757in}}%
\pgfpathcurveto{\pgfqpoint{2.869351in}{3.090580in}}{\pgfqpoint{2.861451in}{3.093853in}}{\pgfqpoint{2.853214in}{3.093853in}}%
\pgfpathcurveto{\pgfqpoint{2.844978in}{3.093853in}}{\pgfqpoint{2.837078in}{3.090580in}}{\pgfqpoint{2.831254in}{3.084757in}}%
\pgfpathcurveto{\pgfqpoint{2.825430in}{3.078933in}}{\pgfqpoint{2.822158in}{3.071033in}}{\pgfqpoint{2.822158in}{3.062796in}}%
\pgfpathcurveto{\pgfqpoint{2.822158in}{3.054560in}}{\pgfqpoint{2.825430in}{3.046660in}}{\pgfqpoint{2.831254in}{3.040836in}}%
\pgfpathcurveto{\pgfqpoint{2.837078in}{3.035012in}}{\pgfqpoint{2.844978in}{3.031740in}}{\pgfqpoint{2.853214in}{3.031740in}}%
\pgfpathclose%
\pgfusepath{stroke,fill}%
\end{pgfscope}%
\begin{pgfscope}%
\pgfpathrectangle{\pgfqpoint{0.100000in}{0.220728in}}{\pgfqpoint{3.696000in}{3.696000in}}%
\pgfusepath{clip}%
\pgfsetbuttcap%
\pgfsetroundjoin%
\definecolor{currentfill}{rgb}{0.121569,0.466667,0.705882}%
\pgfsetfillcolor{currentfill}%
\pgfsetfillopacity{0.558229}%
\pgfsetlinewidth{1.003750pt}%
\definecolor{currentstroke}{rgb}{0.121569,0.466667,0.705882}%
\pgfsetstrokecolor{currentstroke}%
\pgfsetstrokeopacity{0.558229}%
\pgfsetdash{}{0pt}%
\pgfpathmoveto{\pgfqpoint{2.855693in}{3.030141in}}%
\pgfpathcurveto{\pgfqpoint{2.863930in}{3.030141in}}{\pgfqpoint{2.871830in}{3.033413in}}{\pgfqpoint{2.877654in}{3.039237in}}%
\pgfpathcurveto{\pgfqpoint{2.883477in}{3.045061in}}{\pgfqpoint{2.886750in}{3.052961in}}{\pgfqpoint{2.886750in}{3.061198in}}%
\pgfpathcurveto{\pgfqpoint{2.886750in}{3.069434in}}{\pgfqpoint{2.883477in}{3.077334in}}{\pgfqpoint{2.877654in}{3.083158in}}%
\pgfpathcurveto{\pgfqpoint{2.871830in}{3.088982in}}{\pgfqpoint{2.863930in}{3.092254in}}{\pgfqpoint{2.855693in}{3.092254in}}%
\pgfpathcurveto{\pgfqpoint{2.847457in}{3.092254in}}{\pgfqpoint{2.839557in}{3.088982in}}{\pgfqpoint{2.833733in}{3.083158in}}%
\pgfpathcurveto{\pgfqpoint{2.827909in}{3.077334in}}{\pgfqpoint{2.824637in}{3.069434in}}{\pgfqpoint{2.824637in}{3.061198in}}%
\pgfpathcurveto{\pgfqpoint{2.824637in}{3.052961in}}{\pgfqpoint{2.827909in}{3.045061in}}{\pgfqpoint{2.833733in}{3.039237in}}%
\pgfpathcurveto{\pgfqpoint{2.839557in}{3.033413in}}{\pgfqpoint{2.847457in}{3.030141in}}{\pgfqpoint{2.855693in}{3.030141in}}%
\pgfpathclose%
\pgfusepath{stroke,fill}%
\end{pgfscope}%
\begin{pgfscope}%
\pgfpathrectangle{\pgfqpoint{0.100000in}{0.220728in}}{\pgfqpoint{3.696000in}{3.696000in}}%
\pgfusepath{clip}%
\pgfsetbuttcap%
\pgfsetroundjoin%
\definecolor{currentfill}{rgb}{0.121569,0.466667,0.705882}%
\pgfsetfillcolor{currentfill}%
\pgfsetfillopacity{0.558273}%
\pgfsetlinewidth{1.003750pt}%
\definecolor{currentstroke}{rgb}{0.121569,0.466667,0.705882}%
\pgfsetstrokecolor{currentstroke}%
\pgfsetstrokeopacity{0.558273}%
\pgfsetdash{}{0pt}%
\pgfpathmoveto{\pgfqpoint{0.999133in}{1.589776in}}%
\pgfpathcurveto{\pgfqpoint{1.007370in}{1.589776in}}{\pgfqpoint{1.015270in}{1.593048in}}{\pgfqpoint{1.021094in}{1.598872in}}%
\pgfpathcurveto{\pgfqpoint{1.026918in}{1.604696in}}{\pgfqpoint{1.030190in}{1.612596in}}{\pgfqpoint{1.030190in}{1.620832in}}%
\pgfpathcurveto{\pgfqpoint{1.030190in}{1.629068in}}{\pgfqpoint{1.026918in}{1.636968in}}{\pgfqpoint{1.021094in}{1.642792in}}%
\pgfpathcurveto{\pgfqpoint{1.015270in}{1.648616in}}{\pgfqpoint{1.007370in}{1.651889in}}{\pgfqpoint{0.999133in}{1.651889in}}%
\pgfpathcurveto{\pgfqpoint{0.990897in}{1.651889in}}{\pgfqpoint{0.982997in}{1.648616in}}{\pgfqpoint{0.977173in}{1.642792in}}%
\pgfpathcurveto{\pgfqpoint{0.971349in}{1.636968in}}{\pgfqpoint{0.968077in}{1.629068in}}{\pgfqpoint{0.968077in}{1.620832in}}%
\pgfpathcurveto{\pgfqpoint{0.968077in}{1.612596in}}{\pgfqpoint{0.971349in}{1.604696in}}{\pgfqpoint{0.977173in}{1.598872in}}%
\pgfpathcurveto{\pgfqpoint{0.982997in}{1.593048in}}{\pgfqpoint{0.990897in}{1.589776in}}{\pgfqpoint{0.999133in}{1.589776in}}%
\pgfpathclose%
\pgfusepath{stroke,fill}%
\end{pgfscope}%
\begin{pgfscope}%
\pgfpathrectangle{\pgfqpoint{0.100000in}{0.220728in}}{\pgfqpoint{3.696000in}{3.696000in}}%
\pgfusepath{clip}%
\pgfsetbuttcap%
\pgfsetroundjoin%
\definecolor{currentfill}{rgb}{0.121569,0.466667,0.705882}%
\pgfsetfillcolor{currentfill}%
\pgfsetfillopacity{0.559145}%
\pgfsetlinewidth{1.003750pt}%
\definecolor{currentstroke}{rgb}{0.121569,0.466667,0.705882}%
\pgfsetstrokecolor{currentstroke}%
\pgfsetstrokeopacity{0.559145}%
\pgfsetdash{}{0pt}%
\pgfpathmoveto{\pgfqpoint{2.858942in}{3.029670in}}%
\pgfpathcurveto{\pgfqpoint{2.867178in}{3.029670in}}{\pgfqpoint{2.875078in}{3.032942in}}{\pgfqpoint{2.880902in}{3.038766in}}%
\pgfpathcurveto{\pgfqpoint{2.886726in}{3.044590in}}{\pgfqpoint{2.889999in}{3.052490in}}{\pgfqpoint{2.889999in}{3.060727in}}%
\pgfpathcurveto{\pgfqpoint{2.889999in}{3.068963in}}{\pgfqpoint{2.886726in}{3.076863in}}{\pgfqpoint{2.880902in}{3.082687in}}%
\pgfpathcurveto{\pgfqpoint{2.875078in}{3.088511in}}{\pgfqpoint{2.867178in}{3.091783in}}{\pgfqpoint{2.858942in}{3.091783in}}%
\pgfpathcurveto{\pgfqpoint{2.850706in}{3.091783in}}{\pgfqpoint{2.842806in}{3.088511in}}{\pgfqpoint{2.836982in}{3.082687in}}%
\pgfpathcurveto{\pgfqpoint{2.831158in}{3.076863in}}{\pgfqpoint{2.827886in}{3.068963in}}{\pgfqpoint{2.827886in}{3.060727in}}%
\pgfpathcurveto{\pgfqpoint{2.827886in}{3.052490in}}{\pgfqpoint{2.831158in}{3.044590in}}{\pgfqpoint{2.836982in}{3.038766in}}%
\pgfpathcurveto{\pgfqpoint{2.842806in}{3.032942in}}{\pgfqpoint{2.850706in}{3.029670in}}{\pgfqpoint{2.858942in}{3.029670in}}%
\pgfpathclose%
\pgfusepath{stroke,fill}%
\end{pgfscope}%
\begin{pgfscope}%
\pgfpathrectangle{\pgfqpoint{0.100000in}{0.220728in}}{\pgfqpoint{3.696000in}{3.696000in}}%
\pgfusepath{clip}%
\pgfsetbuttcap%
\pgfsetroundjoin%
\definecolor{currentfill}{rgb}{0.121569,0.466667,0.705882}%
\pgfsetfillcolor{currentfill}%
\pgfsetfillopacity{0.559785}%
\pgfsetlinewidth{1.003750pt}%
\definecolor{currentstroke}{rgb}{0.121569,0.466667,0.705882}%
\pgfsetstrokecolor{currentstroke}%
\pgfsetstrokeopacity{0.559785}%
\pgfsetdash{}{0pt}%
\pgfpathmoveto{\pgfqpoint{0.993781in}{1.581198in}}%
\pgfpathcurveto{\pgfqpoint{1.002017in}{1.581198in}}{\pgfqpoint{1.009917in}{1.584470in}}{\pgfqpoint{1.015741in}{1.590294in}}%
\pgfpathcurveto{\pgfqpoint{1.021565in}{1.596118in}}{\pgfqpoint{1.024837in}{1.604018in}}{\pgfqpoint{1.024837in}{1.612254in}}%
\pgfpathcurveto{\pgfqpoint{1.024837in}{1.620490in}}{\pgfqpoint{1.021565in}{1.628390in}}{\pgfqpoint{1.015741in}{1.634214in}}%
\pgfpathcurveto{\pgfqpoint{1.009917in}{1.640038in}}{\pgfqpoint{1.002017in}{1.643311in}}{\pgfqpoint{0.993781in}{1.643311in}}%
\pgfpathcurveto{\pgfqpoint{0.985545in}{1.643311in}}{\pgfqpoint{0.977645in}{1.640038in}}{\pgfqpoint{0.971821in}{1.634214in}}%
\pgfpathcurveto{\pgfqpoint{0.965997in}{1.628390in}}{\pgfqpoint{0.962724in}{1.620490in}}{\pgfqpoint{0.962724in}{1.612254in}}%
\pgfpathcurveto{\pgfqpoint{0.962724in}{1.604018in}}{\pgfqpoint{0.965997in}{1.596118in}}{\pgfqpoint{0.971821in}{1.590294in}}%
\pgfpathcurveto{\pgfqpoint{0.977645in}{1.584470in}}{\pgfqpoint{0.985545in}{1.581198in}}{\pgfqpoint{0.993781in}{1.581198in}}%
\pgfpathclose%
\pgfusepath{stroke,fill}%
\end{pgfscope}%
\begin{pgfscope}%
\pgfpathrectangle{\pgfqpoint{0.100000in}{0.220728in}}{\pgfqpoint{3.696000in}{3.696000in}}%
\pgfusepath{clip}%
\pgfsetbuttcap%
\pgfsetroundjoin%
\definecolor{currentfill}{rgb}{0.121569,0.466667,0.705882}%
\pgfsetfillcolor{currentfill}%
\pgfsetfillopacity{0.560271}%
\pgfsetlinewidth{1.003750pt}%
\definecolor{currentstroke}{rgb}{0.121569,0.466667,0.705882}%
\pgfsetstrokecolor{currentstroke}%
\pgfsetstrokeopacity{0.560271}%
\pgfsetdash{}{0pt}%
\pgfpathmoveto{\pgfqpoint{2.863140in}{3.028753in}}%
\pgfpathcurveto{\pgfqpoint{2.871376in}{3.028753in}}{\pgfqpoint{2.879276in}{3.032025in}}{\pgfqpoint{2.885100in}{3.037849in}}%
\pgfpathcurveto{\pgfqpoint{2.890924in}{3.043673in}}{\pgfqpoint{2.894196in}{3.051573in}}{\pgfqpoint{2.894196in}{3.059809in}}%
\pgfpathcurveto{\pgfqpoint{2.894196in}{3.068045in}}{\pgfqpoint{2.890924in}{3.075945in}}{\pgfqpoint{2.885100in}{3.081769in}}%
\pgfpathcurveto{\pgfqpoint{2.879276in}{3.087593in}}{\pgfqpoint{2.871376in}{3.090866in}}{\pgfqpoint{2.863140in}{3.090866in}}%
\pgfpathcurveto{\pgfqpoint{2.854904in}{3.090866in}}{\pgfqpoint{2.847004in}{3.087593in}}{\pgfqpoint{2.841180in}{3.081769in}}%
\pgfpathcurveto{\pgfqpoint{2.835356in}{3.075945in}}{\pgfqpoint{2.832083in}{3.068045in}}{\pgfqpoint{2.832083in}{3.059809in}}%
\pgfpathcurveto{\pgfqpoint{2.832083in}{3.051573in}}{\pgfqpoint{2.835356in}{3.043673in}}{\pgfqpoint{2.841180in}{3.037849in}}%
\pgfpathcurveto{\pgfqpoint{2.847004in}{3.032025in}}{\pgfqpoint{2.854904in}{3.028753in}}{\pgfqpoint{2.863140in}{3.028753in}}%
\pgfpathclose%
\pgfusepath{stroke,fill}%
\end{pgfscope}%
\begin{pgfscope}%
\pgfpathrectangle{\pgfqpoint{0.100000in}{0.220728in}}{\pgfqpoint{3.696000in}{3.696000in}}%
\pgfusepath{clip}%
\pgfsetbuttcap%
\pgfsetroundjoin%
\definecolor{currentfill}{rgb}{0.121569,0.466667,0.705882}%
\pgfsetfillcolor{currentfill}%
\pgfsetfillopacity{0.561177}%
\pgfsetlinewidth{1.003750pt}%
\definecolor{currentstroke}{rgb}{0.121569,0.466667,0.705882}%
\pgfsetstrokecolor{currentstroke}%
\pgfsetstrokeopacity{0.561177}%
\pgfsetdash{}{0pt}%
\pgfpathmoveto{\pgfqpoint{2.868846in}{3.027375in}}%
\pgfpathcurveto{\pgfqpoint{2.877083in}{3.027375in}}{\pgfqpoint{2.884983in}{3.030647in}}{\pgfqpoint{2.890807in}{3.036471in}}%
\pgfpathcurveto{\pgfqpoint{2.896631in}{3.042295in}}{\pgfqpoint{2.899903in}{3.050195in}}{\pgfqpoint{2.899903in}{3.058431in}}%
\pgfpathcurveto{\pgfqpoint{2.899903in}{3.066668in}}{\pgfqpoint{2.896631in}{3.074568in}}{\pgfqpoint{2.890807in}{3.080392in}}%
\pgfpathcurveto{\pgfqpoint{2.884983in}{3.086216in}}{\pgfqpoint{2.877083in}{3.089488in}}{\pgfqpoint{2.868846in}{3.089488in}}%
\pgfpathcurveto{\pgfqpoint{2.860610in}{3.089488in}}{\pgfqpoint{2.852710in}{3.086216in}}{\pgfqpoint{2.846886in}{3.080392in}}%
\pgfpathcurveto{\pgfqpoint{2.841062in}{3.074568in}}{\pgfqpoint{2.837790in}{3.066668in}}{\pgfqpoint{2.837790in}{3.058431in}}%
\pgfpathcurveto{\pgfqpoint{2.837790in}{3.050195in}}{\pgfqpoint{2.841062in}{3.042295in}}{\pgfqpoint{2.846886in}{3.036471in}}%
\pgfpathcurveto{\pgfqpoint{2.852710in}{3.030647in}}{\pgfqpoint{2.860610in}{3.027375in}}{\pgfqpoint{2.868846in}{3.027375in}}%
\pgfpathclose%
\pgfusepath{stroke,fill}%
\end{pgfscope}%
\begin{pgfscope}%
\pgfpathrectangle{\pgfqpoint{0.100000in}{0.220728in}}{\pgfqpoint{3.696000in}{3.696000in}}%
\pgfusepath{clip}%
\pgfsetbuttcap%
\pgfsetroundjoin%
\definecolor{currentfill}{rgb}{0.121569,0.466667,0.705882}%
\pgfsetfillcolor{currentfill}%
\pgfsetfillopacity{0.562732}%
\pgfsetlinewidth{1.003750pt}%
\definecolor{currentstroke}{rgb}{0.121569,0.466667,0.705882}%
\pgfsetstrokecolor{currentstroke}%
\pgfsetstrokeopacity{0.562732}%
\pgfsetdash{}{0pt}%
\pgfpathmoveto{\pgfqpoint{0.986548in}{1.563496in}}%
\pgfpathcurveto{\pgfqpoint{0.994784in}{1.563496in}}{\pgfqpoint{1.002684in}{1.566768in}}{\pgfqpoint{1.008508in}{1.572592in}}%
\pgfpathcurveto{\pgfqpoint{1.014332in}{1.578416in}}{\pgfqpoint{1.017604in}{1.586316in}}{\pgfqpoint{1.017604in}{1.594553in}}%
\pgfpathcurveto{\pgfqpoint{1.017604in}{1.602789in}}{\pgfqpoint{1.014332in}{1.610689in}}{\pgfqpoint{1.008508in}{1.616513in}}%
\pgfpathcurveto{\pgfqpoint{1.002684in}{1.622337in}}{\pgfqpoint{0.994784in}{1.625609in}}{\pgfqpoint{0.986548in}{1.625609in}}%
\pgfpathcurveto{\pgfqpoint{0.978311in}{1.625609in}}{\pgfqpoint{0.970411in}{1.622337in}}{\pgfqpoint{0.964587in}{1.616513in}}%
\pgfpathcurveto{\pgfqpoint{0.958763in}{1.610689in}}{\pgfqpoint{0.955491in}{1.602789in}}{\pgfqpoint{0.955491in}{1.594553in}}%
\pgfpathcurveto{\pgfqpoint{0.955491in}{1.586316in}}{\pgfqpoint{0.958763in}{1.578416in}}{\pgfqpoint{0.964587in}{1.572592in}}%
\pgfpathcurveto{\pgfqpoint{0.970411in}{1.566768in}}{\pgfqpoint{0.978311in}{1.563496in}}{\pgfqpoint{0.986548in}{1.563496in}}%
\pgfpathclose%
\pgfusepath{stroke,fill}%
\end{pgfscope}%
\begin{pgfscope}%
\pgfpathrectangle{\pgfqpoint{0.100000in}{0.220728in}}{\pgfqpoint{3.696000in}{3.696000in}}%
\pgfusepath{clip}%
\pgfsetbuttcap%
\pgfsetroundjoin%
\definecolor{currentfill}{rgb}{0.121569,0.466667,0.705882}%
\pgfsetfillcolor{currentfill}%
\pgfsetfillopacity{0.562767}%
\pgfsetlinewidth{1.003750pt}%
\definecolor{currentstroke}{rgb}{0.121569,0.466667,0.705882}%
\pgfsetstrokecolor{currentstroke}%
\pgfsetstrokeopacity{0.562767}%
\pgfsetdash{}{0pt}%
\pgfpathmoveto{\pgfqpoint{2.876266in}{3.025828in}}%
\pgfpathcurveto{\pgfqpoint{2.884502in}{3.025828in}}{\pgfqpoint{2.892402in}{3.029100in}}{\pgfqpoint{2.898226in}{3.034924in}}%
\pgfpathcurveto{\pgfqpoint{2.904050in}{3.040748in}}{\pgfqpoint{2.907322in}{3.048648in}}{\pgfqpoint{2.907322in}{3.056885in}}%
\pgfpathcurveto{\pgfqpoint{2.907322in}{3.065121in}}{\pgfqpoint{2.904050in}{3.073021in}}{\pgfqpoint{2.898226in}{3.078845in}}%
\pgfpathcurveto{\pgfqpoint{2.892402in}{3.084669in}}{\pgfqpoint{2.884502in}{3.087941in}}{\pgfqpoint{2.876266in}{3.087941in}}%
\pgfpathcurveto{\pgfqpoint{2.868030in}{3.087941in}}{\pgfqpoint{2.860130in}{3.084669in}}{\pgfqpoint{2.854306in}{3.078845in}}%
\pgfpathcurveto{\pgfqpoint{2.848482in}{3.073021in}}{\pgfqpoint{2.845209in}{3.065121in}}{\pgfqpoint{2.845209in}{3.056885in}}%
\pgfpathcurveto{\pgfqpoint{2.845209in}{3.048648in}}{\pgfqpoint{2.848482in}{3.040748in}}{\pgfqpoint{2.854306in}{3.034924in}}%
\pgfpathcurveto{\pgfqpoint{2.860130in}{3.029100in}}{\pgfqpoint{2.868030in}{3.025828in}}{\pgfqpoint{2.876266in}{3.025828in}}%
\pgfpathclose%
\pgfusepath{stroke,fill}%
\end{pgfscope}%
\begin{pgfscope}%
\pgfpathrectangle{\pgfqpoint{0.100000in}{0.220728in}}{\pgfqpoint{3.696000in}{3.696000in}}%
\pgfusepath{clip}%
\pgfsetbuttcap%
\pgfsetroundjoin%
\definecolor{currentfill}{rgb}{0.121569,0.466667,0.705882}%
\pgfsetfillcolor{currentfill}%
\pgfsetfillopacity{0.564638}%
\pgfsetlinewidth{1.003750pt}%
\definecolor{currentstroke}{rgb}{0.121569,0.466667,0.705882}%
\pgfsetstrokecolor{currentstroke}%
\pgfsetstrokeopacity{0.564638}%
\pgfsetdash{}{0pt}%
\pgfpathmoveto{\pgfqpoint{2.884276in}{3.024301in}}%
\pgfpathcurveto{\pgfqpoint{2.892512in}{3.024301in}}{\pgfqpoint{2.900412in}{3.027573in}}{\pgfqpoint{2.906236in}{3.033397in}}%
\pgfpathcurveto{\pgfqpoint{2.912060in}{3.039221in}}{\pgfqpoint{2.915332in}{3.047121in}}{\pgfqpoint{2.915332in}{3.055357in}}%
\pgfpathcurveto{\pgfqpoint{2.915332in}{3.063593in}}{\pgfqpoint{2.912060in}{3.071494in}}{\pgfqpoint{2.906236in}{3.077317in}}%
\pgfpathcurveto{\pgfqpoint{2.900412in}{3.083141in}}{\pgfqpoint{2.892512in}{3.086414in}}{\pgfqpoint{2.884276in}{3.086414in}}%
\pgfpathcurveto{\pgfqpoint{2.876039in}{3.086414in}}{\pgfqpoint{2.868139in}{3.083141in}}{\pgfqpoint{2.862315in}{3.077317in}}%
\pgfpathcurveto{\pgfqpoint{2.856491in}{3.071494in}}{\pgfqpoint{2.853219in}{3.063593in}}{\pgfqpoint{2.853219in}{3.055357in}}%
\pgfpathcurveto{\pgfqpoint{2.853219in}{3.047121in}}{\pgfqpoint{2.856491in}{3.039221in}}{\pgfqpoint{2.862315in}{3.033397in}}%
\pgfpathcurveto{\pgfqpoint{2.868139in}{3.027573in}}{\pgfqpoint{2.876039in}{3.024301in}}{\pgfqpoint{2.884276in}{3.024301in}}%
\pgfpathclose%
\pgfusepath{stroke,fill}%
\end{pgfscope}%
\begin{pgfscope}%
\pgfpathrectangle{\pgfqpoint{0.100000in}{0.220728in}}{\pgfqpoint{3.696000in}{3.696000in}}%
\pgfusepath{clip}%
\pgfsetbuttcap%
\pgfsetroundjoin%
\definecolor{currentfill}{rgb}{0.121569,0.466667,0.705882}%
\pgfsetfillcolor{currentfill}%
\pgfsetfillopacity{0.564665}%
\pgfsetlinewidth{1.003750pt}%
\definecolor{currentstroke}{rgb}{0.121569,0.466667,0.705882}%
\pgfsetstrokecolor{currentstroke}%
\pgfsetstrokeopacity{0.564665}%
\pgfsetdash{}{0pt}%
\pgfpathmoveto{\pgfqpoint{0.977185in}{1.549449in}}%
\pgfpathcurveto{\pgfqpoint{0.985421in}{1.549449in}}{\pgfqpoint{0.993321in}{1.552721in}}{\pgfqpoint{0.999145in}{1.558545in}}%
\pgfpathcurveto{\pgfqpoint{1.004969in}{1.564369in}}{\pgfqpoint{1.008241in}{1.572269in}}{\pgfqpoint{1.008241in}{1.580505in}}%
\pgfpathcurveto{\pgfqpoint{1.008241in}{1.588741in}}{\pgfqpoint{1.004969in}{1.596641in}}{\pgfqpoint{0.999145in}{1.602465in}}%
\pgfpathcurveto{\pgfqpoint{0.993321in}{1.608289in}}{\pgfqpoint{0.985421in}{1.611562in}}{\pgfqpoint{0.977185in}{1.611562in}}%
\pgfpathcurveto{\pgfqpoint{0.968948in}{1.611562in}}{\pgfqpoint{0.961048in}{1.608289in}}{\pgfqpoint{0.955224in}{1.602465in}}%
\pgfpathcurveto{\pgfqpoint{0.949401in}{1.596641in}}{\pgfqpoint{0.946128in}{1.588741in}}{\pgfqpoint{0.946128in}{1.580505in}}%
\pgfpathcurveto{\pgfqpoint{0.946128in}{1.572269in}}{\pgfqpoint{0.949401in}{1.564369in}}{\pgfqpoint{0.955224in}{1.558545in}}%
\pgfpathcurveto{\pgfqpoint{0.961048in}{1.552721in}}{\pgfqpoint{0.968948in}{1.549449in}}{\pgfqpoint{0.977185in}{1.549449in}}%
\pgfpathclose%
\pgfusepath{stroke,fill}%
\end{pgfscope}%
\begin{pgfscope}%
\pgfpathrectangle{\pgfqpoint{0.100000in}{0.220728in}}{\pgfqpoint{3.696000in}{3.696000in}}%
\pgfusepath{clip}%
\pgfsetbuttcap%
\pgfsetroundjoin%
\definecolor{currentfill}{rgb}{0.121569,0.466667,0.705882}%
\pgfsetfillcolor{currentfill}%
\pgfsetfillopacity{0.565166}%
\pgfsetlinewidth{1.003750pt}%
\definecolor{currentstroke}{rgb}{0.121569,0.466667,0.705882}%
\pgfsetstrokecolor{currentstroke}%
\pgfsetstrokeopacity{0.565166}%
\pgfsetdash{}{0pt}%
\pgfpathmoveto{\pgfqpoint{2.889203in}{3.023847in}}%
\pgfpathcurveto{\pgfqpoint{2.897439in}{3.023847in}}{\pgfqpoint{2.905339in}{3.027119in}}{\pgfqpoint{2.911163in}{3.032943in}}%
\pgfpathcurveto{\pgfqpoint{2.916987in}{3.038767in}}{\pgfqpoint{2.920259in}{3.046667in}}{\pgfqpoint{2.920259in}{3.054903in}}%
\pgfpathcurveto{\pgfqpoint{2.920259in}{3.063139in}}{\pgfqpoint{2.916987in}{3.071039in}}{\pgfqpoint{2.911163in}{3.076863in}}%
\pgfpathcurveto{\pgfqpoint{2.905339in}{3.082687in}}{\pgfqpoint{2.897439in}{3.085960in}}{\pgfqpoint{2.889203in}{3.085960in}}%
\pgfpathcurveto{\pgfqpoint{2.880967in}{3.085960in}}{\pgfqpoint{2.873067in}{3.082687in}}{\pgfqpoint{2.867243in}{3.076863in}}%
\pgfpathcurveto{\pgfqpoint{2.861419in}{3.071039in}}{\pgfqpoint{2.858146in}{3.063139in}}{\pgfqpoint{2.858146in}{3.054903in}}%
\pgfpathcurveto{\pgfqpoint{2.858146in}{3.046667in}}{\pgfqpoint{2.861419in}{3.038767in}}{\pgfqpoint{2.867243in}{3.032943in}}%
\pgfpathcurveto{\pgfqpoint{2.873067in}{3.027119in}}{\pgfqpoint{2.880967in}{3.023847in}}{\pgfqpoint{2.889203in}{3.023847in}}%
\pgfpathclose%
\pgfusepath{stroke,fill}%
\end{pgfscope}%
\begin{pgfscope}%
\pgfpathrectangle{\pgfqpoint{0.100000in}{0.220728in}}{\pgfqpoint{3.696000in}{3.696000in}}%
\pgfusepath{clip}%
\pgfsetbuttcap%
\pgfsetroundjoin%
\definecolor{currentfill}{rgb}{0.121569,0.466667,0.705882}%
\pgfsetfillcolor{currentfill}%
\pgfsetfillopacity{0.565870}%
\pgfsetlinewidth{1.003750pt}%
\definecolor{currentstroke}{rgb}{0.121569,0.466667,0.705882}%
\pgfsetstrokecolor{currentstroke}%
\pgfsetstrokeopacity{0.565870}%
\pgfsetdash{}{0pt}%
\pgfpathmoveto{\pgfqpoint{2.891610in}{3.023867in}}%
\pgfpathcurveto{\pgfqpoint{2.899846in}{3.023867in}}{\pgfqpoint{2.907746in}{3.027140in}}{\pgfqpoint{2.913570in}{3.032964in}}%
\pgfpathcurveto{\pgfqpoint{2.919394in}{3.038787in}}{\pgfqpoint{2.922666in}{3.046688in}}{\pgfqpoint{2.922666in}{3.054924in}}%
\pgfpathcurveto{\pgfqpoint{2.922666in}{3.063160in}}{\pgfqpoint{2.919394in}{3.071060in}}{\pgfqpoint{2.913570in}{3.076884in}}%
\pgfpathcurveto{\pgfqpoint{2.907746in}{3.082708in}}{\pgfqpoint{2.899846in}{3.085980in}}{\pgfqpoint{2.891610in}{3.085980in}}%
\pgfpathcurveto{\pgfqpoint{2.883373in}{3.085980in}}{\pgfqpoint{2.875473in}{3.082708in}}{\pgfqpoint{2.869649in}{3.076884in}}%
\pgfpathcurveto{\pgfqpoint{2.863825in}{3.071060in}}{\pgfqpoint{2.860553in}{3.063160in}}{\pgfqpoint{2.860553in}{3.054924in}}%
\pgfpathcurveto{\pgfqpoint{2.860553in}{3.046688in}}{\pgfqpoint{2.863825in}{3.038787in}}{\pgfqpoint{2.869649in}{3.032964in}}%
\pgfpathcurveto{\pgfqpoint{2.875473in}{3.027140in}}{\pgfqpoint{2.883373in}{3.023867in}}{\pgfqpoint{2.891610in}{3.023867in}}%
\pgfpathclose%
\pgfusepath{stroke,fill}%
\end{pgfscope}%
\begin{pgfscope}%
\pgfpathrectangle{\pgfqpoint{0.100000in}{0.220728in}}{\pgfqpoint{3.696000in}{3.696000in}}%
\pgfusepath{clip}%
\pgfsetbuttcap%
\pgfsetroundjoin%
\definecolor{currentfill}{rgb}{0.121569,0.466667,0.705882}%
\pgfsetfillcolor{currentfill}%
\pgfsetfillopacity{0.567072}%
\pgfsetlinewidth{1.003750pt}%
\definecolor{currentstroke}{rgb}{0.121569,0.466667,0.705882}%
\pgfsetstrokecolor{currentstroke}%
\pgfsetstrokeopacity{0.567072}%
\pgfsetdash{}{0pt}%
\pgfpathmoveto{\pgfqpoint{0.971987in}{1.535077in}}%
\pgfpathcurveto{\pgfqpoint{0.980223in}{1.535077in}}{\pgfqpoint{0.988124in}{1.538349in}}{\pgfqpoint{0.993947in}{1.544173in}}%
\pgfpathcurveto{\pgfqpoint{0.999771in}{1.549997in}}{\pgfqpoint{1.003044in}{1.557897in}}{\pgfqpoint{1.003044in}{1.566133in}}%
\pgfpathcurveto{\pgfqpoint{1.003044in}{1.574370in}}{\pgfqpoint{0.999771in}{1.582270in}}{\pgfqpoint{0.993947in}{1.588094in}}%
\pgfpathcurveto{\pgfqpoint{0.988124in}{1.593918in}}{\pgfqpoint{0.980223in}{1.597190in}}{\pgfqpoint{0.971987in}{1.597190in}}%
\pgfpathcurveto{\pgfqpoint{0.963751in}{1.597190in}}{\pgfqpoint{0.955851in}{1.593918in}}{\pgfqpoint{0.950027in}{1.588094in}}%
\pgfpathcurveto{\pgfqpoint{0.944203in}{1.582270in}}{\pgfqpoint{0.940931in}{1.574370in}}{\pgfqpoint{0.940931in}{1.566133in}}%
\pgfpathcurveto{\pgfqpoint{0.940931in}{1.557897in}}{\pgfqpoint{0.944203in}{1.549997in}}{\pgfqpoint{0.950027in}{1.544173in}}%
\pgfpathcurveto{\pgfqpoint{0.955851in}{1.538349in}}{\pgfqpoint{0.963751in}{1.535077in}}{\pgfqpoint{0.971987in}{1.535077in}}%
\pgfpathclose%
\pgfusepath{stroke,fill}%
\end{pgfscope}%
\begin{pgfscope}%
\pgfpathrectangle{\pgfqpoint{0.100000in}{0.220728in}}{\pgfqpoint{3.696000in}{3.696000in}}%
\pgfusepath{clip}%
\pgfsetbuttcap%
\pgfsetroundjoin%
\definecolor{currentfill}{rgb}{0.121569,0.466667,0.705882}%
\pgfsetfillcolor{currentfill}%
\pgfsetfillopacity{0.567109}%
\pgfsetlinewidth{1.003750pt}%
\definecolor{currentstroke}{rgb}{0.121569,0.466667,0.705882}%
\pgfsetstrokecolor{currentstroke}%
\pgfsetstrokeopacity{0.567109}%
\pgfsetdash{}{0pt}%
\pgfpathmoveto{\pgfqpoint{2.899900in}{3.021189in}}%
\pgfpathcurveto{\pgfqpoint{2.908137in}{3.021189in}}{\pgfqpoint{2.916037in}{3.024461in}}{\pgfqpoint{2.921860in}{3.030285in}}%
\pgfpathcurveto{\pgfqpoint{2.927684in}{3.036109in}}{\pgfqpoint{2.930957in}{3.044009in}}{\pgfqpoint{2.930957in}{3.052245in}}%
\pgfpathcurveto{\pgfqpoint{2.930957in}{3.060482in}}{\pgfqpoint{2.927684in}{3.068382in}}{\pgfqpoint{2.921860in}{3.074205in}}%
\pgfpathcurveto{\pgfqpoint{2.916037in}{3.080029in}}{\pgfqpoint{2.908137in}{3.083302in}}{\pgfqpoint{2.899900in}{3.083302in}}%
\pgfpathcurveto{\pgfqpoint{2.891664in}{3.083302in}}{\pgfqpoint{2.883764in}{3.080029in}}{\pgfqpoint{2.877940in}{3.074205in}}%
\pgfpathcurveto{\pgfqpoint{2.872116in}{3.068382in}}{\pgfqpoint{2.868844in}{3.060482in}}{\pgfqpoint{2.868844in}{3.052245in}}%
\pgfpathcurveto{\pgfqpoint{2.868844in}{3.044009in}}{\pgfqpoint{2.872116in}{3.036109in}}{\pgfqpoint{2.877940in}{3.030285in}}%
\pgfpathcurveto{\pgfqpoint{2.883764in}{3.024461in}}{\pgfqpoint{2.891664in}{3.021189in}}{\pgfqpoint{2.899900in}{3.021189in}}%
\pgfpathclose%
\pgfusepath{stroke,fill}%
\end{pgfscope}%
\begin{pgfscope}%
\pgfpathrectangle{\pgfqpoint{0.100000in}{0.220728in}}{\pgfqpoint{3.696000in}{3.696000in}}%
\pgfusepath{clip}%
\pgfsetbuttcap%
\pgfsetroundjoin%
\definecolor{currentfill}{rgb}{0.121569,0.466667,0.705882}%
\pgfsetfillcolor{currentfill}%
\pgfsetfillopacity{0.567549}%
\pgfsetlinewidth{1.003750pt}%
\definecolor{currentstroke}{rgb}{0.121569,0.466667,0.705882}%
\pgfsetstrokecolor{currentstroke}%
\pgfsetstrokeopacity{0.567549}%
\pgfsetdash{}{0pt}%
\pgfpathmoveto{\pgfqpoint{2.893800in}{3.022716in}}%
\pgfpathcurveto{\pgfqpoint{2.902037in}{3.022716in}}{\pgfqpoint{2.909937in}{3.025988in}}{\pgfqpoint{2.915761in}{3.031812in}}%
\pgfpathcurveto{\pgfqpoint{2.921584in}{3.037636in}}{\pgfqpoint{2.924857in}{3.045536in}}{\pgfqpoint{2.924857in}{3.053772in}}%
\pgfpathcurveto{\pgfqpoint{2.924857in}{3.062009in}}{\pgfqpoint{2.921584in}{3.069909in}}{\pgfqpoint{2.915761in}{3.075733in}}%
\pgfpathcurveto{\pgfqpoint{2.909937in}{3.081556in}}{\pgfqpoint{2.902037in}{3.084829in}}{\pgfqpoint{2.893800in}{3.084829in}}%
\pgfpathcurveto{\pgfqpoint{2.885564in}{3.084829in}}{\pgfqpoint{2.877664in}{3.081556in}}{\pgfqpoint{2.871840in}{3.075733in}}%
\pgfpathcurveto{\pgfqpoint{2.866016in}{3.069909in}}{\pgfqpoint{2.862744in}{3.062009in}}{\pgfqpoint{2.862744in}{3.053772in}}%
\pgfpathcurveto{\pgfqpoint{2.862744in}{3.045536in}}{\pgfqpoint{2.866016in}{3.037636in}}{\pgfqpoint{2.871840in}{3.031812in}}%
\pgfpathcurveto{\pgfqpoint{2.877664in}{3.025988in}}{\pgfqpoint{2.885564in}{3.022716in}}{\pgfqpoint{2.893800in}{3.022716in}}%
\pgfpathclose%
\pgfusepath{stroke,fill}%
\end{pgfscope}%
\begin{pgfscope}%
\pgfpathrectangle{\pgfqpoint{0.100000in}{0.220728in}}{\pgfqpoint{3.696000in}{3.696000in}}%
\pgfusepath{clip}%
\pgfsetbuttcap%
\pgfsetroundjoin%
\definecolor{currentfill}{rgb}{0.121569,0.466667,0.705882}%
\pgfsetfillcolor{currentfill}%
\pgfsetfillopacity{0.567914}%
\pgfsetlinewidth{1.003750pt}%
\definecolor{currentstroke}{rgb}{0.121569,0.466667,0.705882}%
\pgfsetstrokecolor{currentstroke}%
\pgfsetstrokeopacity{0.567914}%
\pgfsetdash{}{0pt}%
\pgfpathmoveto{\pgfqpoint{2.902750in}{3.020993in}}%
\pgfpathcurveto{\pgfqpoint{2.910987in}{3.020993in}}{\pgfqpoint{2.918887in}{3.024265in}}{\pgfqpoint{2.924711in}{3.030089in}}%
\pgfpathcurveto{\pgfqpoint{2.930535in}{3.035913in}}{\pgfqpoint{2.933807in}{3.043813in}}{\pgfqpoint{2.933807in}{3.052049in}}%
\pgfpathcurveto{\pgfqpoint{2.933807in}{3.060285in}}{\pgfqpoint{2.930535in}{3.068185in}}{\pgfqpoint{2.924711in}{3.074009in}}%
\pgfpathcurveto{\pgfqpoint{2.918887in}{3.079833in}}{\pgfqpoint{2.910987in}{3.083106in}}{\pgfqpoint{2.902750in}{3.083106in}}%
\pgfpathcurveto{\pgfqpoint{2.894514in}{3.083106in}}{\pgfqpoint{2.886614in}{3.079833in}}{\pgfqpoint{2.880790in}{3.074009in}}%
\pgfpathcurveto{\pgfqpoint{2.874966in}{3.068185in}}{\pgfqpoint{2.871694in}{3.060285in}}{\pgfqpoint{2.871694in}{3.052049in}}%
\pgfpathcurveto{\pgfqpoint{2.871694in}{3.043813in}}{\pgfqpoint{2.874966in}{3.035913in}}{\pgfqpoint{2.880790in}{3.030089in}}%
\pgfpathcurveto{\pgfqpoint{2.886614in}{3.024265in}}{\pgfqpoint{2.894514in}{3.020993in}}{\pgfqpoint{2.902750in}{3.020993in}}%
\pgfpathclose%
\pgfusepath{stroke,fill}%
\end{pgfscope}%
\begin{pgfscope}%
\pgfpathrectangle{\pgfqpoint{0.100000in}{0.220728in}}{\pgfqpoint{3.696000in}{3.696000in}}%
\pgfusepath{clip}%
\pgfsetbuttcap%
\pgfsetroundjoin%
\definecolor{currentfill}{rgb}{0.121569,0.466667,0.705882}%
\pgfsetfillcolor{currentfill}%
\pgfsetfillopacity{0.568318}%
\pgfsetlinewidth{1.003750pt}%
\definecolor{currentstroke}{rgb}{0.121569,0.466667,0.705882}%
\pgfsetstrokecolor{currentstroke}%
\pgfsetstrokeopacity{0.568318}%
\pgfsetdash{}{0pt}%
\pgfpathmoveto{\pgfqpoint{0.964076in}{1.525781in}}%
\pgfpathcurveto{\pgfqpoint{0.972313in}{1.525781in}}{\pgfqpoint{0.980213in}{1.529054in}}{\pgfqpoint{0.986036in}{1.534878in}}%
\pgfpathcurveto{\pgfqpoint{0.991860in}{1.540702in}}{\pgfqpoint{0.995133in}{1.548602in}}{\pgfqpoint{0.995133in}{1.556838in}}%
\pgfpathcurveto{\pgfqpoint{0.995133in}{1.565074in}}{\pgfqpoint{0.991860in}{1.572974in}}{\pgfqpoint{0.986036in}{1.578798in}}%
\pgfpathcurveto{\pgfqpoint{0.980213in}{1.584622in}}{\pgfqpoint{0.972313in}{1.587894in}}{\pgfqpoint{0.964076in}{1.587894in}}%
\pgfpathcurveto{\pgfqpoint{0.955840in}{1.587894in}}{\pgfqpoint{0.947940in}{1.584622in}}{\pgfqpoint{0.942116in}{1.578798in}}%
\pgfpathcurveto{\pgfqpoint{0.936292in}{1.572974in}}{\pgfqpoint{0.933020in}{1.565074in}}{\pgfqpoint{0.933020in}{1.556838in}}%
\pgfpathcurveto{\pgfqpoint{0.933020in}{1.548602in}}{\pgfqpoint{0.936292in}{1.540702in}}{\pgfqpoint{0.942116in}{1.534878in}}%
\pgfpathcurveto{\pgfqpoint{0.947940in}{1.529054in}}{\pgfqpoint{0.955840in}{1.525781in}}{\pgfqpoint{0.964076in}{1.525781in}}%
\pgfpathclose%
\pgfusepath{stroke,fill}%
\end{pgfscope}%
\begin{pgfscope}%
\pgfpathrectangle{\pgfqpoint{0.100000in}{0.220728in}}{\pgfqpoint{3.696000in}{3.696000in}}%
\pgfusepath{clip}%
\pgfsetbuttcap%
\pgfsetroundjoin%
\definecolor{currentfill}{rgb}{0.121569,0.466667,0.705882}%
\pgfsetfillcolor{currentfill}%
\pgfsetfillopacity{0.568401}%
\pgfsetlinewidth{1.003750pt}%
\definecolor{currentstroke}{rgb}{0.121569,0.466667,0.705882}%
\pgfsetstrokecolor{currentstroke}%
\pgfsetstrokeopacity{0.568401}%
\pgfsetdash{}{0pt}%
\pgfpathmoveto{\pgfqpoint{2.907323in}{3.020449in}}%
\pgfpathcurveto{\pgfqpoint{2.915560in}{3.020449in}}{\pgfqpoint{2.923460in}{3.023721in}}{\pgfqpoint{2.929284in}{3.029545in}}%
\pgfpathcurveto{\pgfqpoint{2.935107in}{3.035369in}}{\pgfqpoint{2.938380in}{3.043269in}}{\pgfqpoint{2.938380in}{3.051505in}}%
\pgfpathcurveto{\pgfqpoint{2.938380in}{3.059742in}}{\pgfqpoint{2.935107in}{3.067642in}}{\pgfqpoint{2.929284in}{3.073466in}}%
\pgfpathcurveto{\pgfqpoint{2.923460in}{3.079290in}}{\pgfqpoint{2.915560in}{3.082562in}}{\pgfqpoint{2.907323in}{3.082562in}}%
\pgfpathcurveto{\pgfqpoint{2.899087in}{3.082562in}}{\pgfqpoint{2.891187in}{3.079290in}}{\pgfqpoint{2.885363in}{3.073466in}}%
\pgfpathcurveto{\pgfqpoint{2.879539in}{3.067642in}}{\pgfqpoint{2.876267in}{3.059742in}}{\pgfqpoint{2.876267in}{3.051505in}}%
\pgfpathcurveto{\pgfqpoint{2.876267in}{3.043269in}}{\pgfqpoint{2.879539in}{3.035369in}}{\pgfqpoint{2.885363in}{3.029545in}}%
\pgfpathcurveto{\pgfqpoint{2.891187in}{3.023721in}}{\pgfqpoint{2.899087in}{3.020449in}}{\pgfqpoint{2.907323in}{3.020449in}}%
\pgfpathclose%
\pgfusepath{stroke,fill}%
\end{pgfscope}%
\begin{pgfscope}%
\pgfpathrectangle{\pgfqpoint{0.100000in}{0.220728in}}{\pgfqpoint{3.696000in}{3.696000in}}%
\pgfusepath{clip}%
\pgfsetbuttcap%
\pgfsetroundjoin%
\definecolor{currentfill}{rgb}{0.121569,0.466667,0.705882}%
\pgfsetfillcolor{currentfill}%
\pgfsetfillopacity{0.569803}%
\pgfsetlinewidth{1.003750pt}%
\definecolor{currentstroke}{rgb}{0.121569,0.466667,0.705882}%
\pgfsetstrokecolor{currentstroke}%
\pgfsetstrokeopacity{0.569803}%
\pgfsetdash{}{0pt}%
\pgfpathmoveto{\pgfqpoint{2.911918in}{3.019926in}}%
\pgfpathcurveto{\pgfqpoint{2.920154in}{3.019926in}}{\pgfqpoint{2.928055in}{3.023198in}}{\pgfqpoint{2.933878in}{3.029022in}}%
\pgfpathcurveto{\pgfqpoint{2.939702in}{3.034846in}}{\pgfqpoint{2.942975in}{3.042746in}}{\pgfqpoint{2.942975in}{3.050983in}}%
\pgfpathcurveto{\pgfqpoint{2.942975in}{3.059219in}}{\pgfqpoint{2.939702in}{3.067119in}}{\pgfqpoint{2.933878in}{3.072943in}}%
\pgfpathcurveto{\pgfqpoint{2.928055in}{3.078767in}}{\pgfqpoint{2.920154in}{3.082039in}}{\pgfqpoint{2.911918in}{3.082039in}}%
\pgfpathcurveto{\pgfqpoint{2.903682in}{3.082039in}}{\pgfqpoint{2.895782in}{3.078767in}}{\pgfqpoint{2.889958in}{3.072943in}}%
\pgfpathcurveto{\pgfqpoint{2.884134in}{3.067119in}}{\pgfqpoint{2.880862in}{3.059219in}}{\pgfqpoint{2.880862in}{3.050983in}}%
\pgfpathcurveto{\pgfqpoint{2.880862in}{3.042746in}}{\pgfqpoint{2.884134in}{3.034846in}}{\pgfqpoint{2.889958in}{3.029022in}}%
\pgfpathcurveto{\pgfqpoint{2.895782in}{3.023198in}}{\pgfqpoint{2.903682in}{3.019926in}}{\pgfqpoint{2.911918in}{3.019926in}}%
\pgfpathclose%
\pgfusepath{stroke,fill}%
\end{pgfscope}%
\begin{pgfscope}%
\pgfpathrectangle{\pgfqpoint{0.100000in}{0.220728in}}{\pgfqpoint{3.696000in}{3.696000in}}%
\pgfusepath{clip}%
\pgfsetbuttcap%
\pgfsetroundjoin%
\definecolor{currentfill}{rgb}{0.121569,0.466667,0.705882}%
\pgfsetfillcolor{currentfill}%
\pgfsetfillopacity{0.569911}%
\pgfsetlinewidth{1.003750pt}%
\definecolor{currentstroke}{rgb}{0.121569,0.466667,0.705882}%
\pgfsetstrokecolor{currentstroke}%
\pgfsetstrokeopacity{0.569911}%
\pgfsetdash{}{0pt}%
\pgfpathmoveto{\pgfqpoint{0.962810in}{1.515048in}}%
\pgfpathcurveto{\pgfqpoint{0.971046in}{1.515048in}}{\pgfqpoint{0.978946in}{1.518321in}}{\pgfqpoint{0.984770in}{1.524145in}}%
\pgfpathcurveto{\pgfqpoint{0.990594in}{1.529968in}}{\pgfqpoint{0.993866in}{1.537869in}}{\pgfqpoint{0.993866in}{1.546105in}}%
\pgfpathcurveto{\pgfqpoint{0.993866in}{1.554341in}}{\pgfqpoint{0.990594in}{1.562241in}}{\pgfqpoint{0.984770in}{1.568065in}}%
\pgfpathcurveto{\pgfqpoint{0.978946in}{1.573889in}}{\pgfqpoint{0.971046in}{1.577161in}}{\pgfqpoint{0.962810in}{1.577161in}}%
\pgfpathcurveto{\pgfqpoint{0.954574in}{1.577161in}}{\pgfqpoint{0.946674in}{1.573889in}}{\pgfqpoint{0.940850in}{1.568065in}}%
\pgfpathcurveto{\pgfqpoint{0.935026in}{1.562241in}}{\pgfqpoint{0.931753in}{1.554341in}}{\pgfqpoint{0.931753in}{1.546105in}}%
\pgfpathcurveto{\pgfqpoint{0.931753in}{1.537869in}}{\pgfqpoint{0.935026in}{1.529968in}}{\pgfqpoint{0.940850in}{1.524145in}}%
\pgfpathcurveto{\pgfqpoint{0.946674in}{1.518321in}}{\pgfqpoint{0.954574in}{1.515048in}}{\pgfqpoint{0.962810in}{1.515048in}}%
\pgfpathclose%
\pgfusepath{stroke,fill}%
\end{pgfscope}%
\begin{pgfscope}%
\pgfpathrectangle{\pgfqpoint{0.100000in}{0.220728in}}{\pgfqpoint{3.696000in}{3.696000in}}%
\pgfusepath{clip}%
\pgfsetbuttcap%
\pgfsetroundjoin%
\definecolor{currentfill}{rgb}{0.121569,0.466667,0.705882}%
\pgfsetfillcolor{currentfill}%
\pgfsetfillopacity{0.570466}%
\pgfsetlinewidth{1.003750pt}%
\definecolor{currentstroke}{rgb}{0.121569,0.466667,0.705882}%
\pgfsetstrokecolor{currentstroke}%
\pgfsetstrokeopacity{0.570466}%
\pgfsetdash{}{0pt}%
\pgfpathmoveto{\pgfqpoint{0.958621in}{1.510442in}}%
\pgfpathcurveto{\pgfqpoint{0.966857in}{1.510442in}}{\pgfqpoint{0.974757in}{1.513714in}}{\pgfqpoint{0.980581in}{1.519538in}}%
\pgfpathcurveto{\pgfqpoint{0.986405in}{1.525362in}}{\pgfqpoint{0.989678in}{1.533262in}}{\pgfqpoint{0.989678in}{1.541498in}}%
\pgfpathcurveto{\pgfqpoint{0.989678in}{1.549735in}}{\pgfqpoint{0.986405in}{1.557635in}}{\pgfqpoint{0.980581in}{1.563458in}}%
\pgfpathcurveto{\pgfqpoint{0.974757in}{1.569282in}}{\pgfqpoint{0.966857in}{1.572555in}}{\pgfqpoint{0.958621in}{1.572555in}}%
\pgfpathcurveto{\pgfqpoint{0.950385in}{1.572555in}}{\pgfqpoint{0.942485in}{1.569282in}}{\pgfqpoint{0.936661in}{1.563458in}}%
\pgfpathcurveto{\pgfqpoint{0.930837in}{1.557635in}}{\pgfqpoint{0.927565in}{1.549735in}}{\pgfqpoint{0.927565in}{1.541498in}}%
\pgfpathcurveto{\pgfqpoint{0.927565in}{1.533262in}}{\pgfqpoint{0.930837in}{1.525362in}}{\pgfqpoint{0.936661in}{1.519538in}}%
\pgfpathcurveto{\pgfqpoint{0.942485in}{1.513714in}}{\pgfqpoint{0.950385in}{1.510442in}}{\pgfqpoint{0.958621in}{1.510442in}}%
\pgfpathclose%
\pgfusepath{stroke,fill}%
\end{pgfscope}%
\begin{pgfscope}%
\pgfpathrectangle{\pgfqpoint{0.100000in}{0.220728in}}{\pgfqpoint{3.696000in}{3.696000in}}%
\pgfusepath{clip}%
\pgfsetbuttcap%
\pgfsetroundjoin%
\definecolor{currentfill}{rgb}{0.121569,0.466667,0.705882}%
\pgfsetfillcolor{currentfill}%
\pgfsetfillopacity{0.571086}%
\pgfsetlinewidth{1.003750pt}%
\definecolor{currentstroke}{rgb}{0.121569,0.466667,0.705882}%
\pgfsetstrokecolor{currentstroke}%
\pgfsetstrokeopacity{0.571086}%
\pgfsetdash{}{0pt}%
\pgfpathmoveto{\pgfqpoint{0.957736in}{1.506528in}}%
\pgfpathcurveto{\pgfqpoint{0.965972in}{1.506528in}}{\pgfqpoint{0.973872in}{1.509800in}}{\pgfqpoint{0.979696in}{1.515624in}}%
\pgfpathcurveto{\pgfqpoint{0.985520in}{1.521448in}}{\pgfqpoint{0.988792in}{1.529348in}}{\pgfqpoint{0.988792in}{1.537584in}}%
\pgfpathcurveto{\pgfqpoint{0.988792in}{1.545821in}}{\pgfqpoint{0.985520in}{1.553721in}}{\pgfqpoint{0.979696in}{1.559545in}}%
\pgfpathcurveto{\pgfqpoint{0.973872in}{1.565369in}}{\pgfqpoint{0.965972in}{1.568641in}}{\pgfqpoint{0.957736in}{1.568641in}}%
\pgfpathcurveto{\pgfqpoint{0.949499in}{1.568641in}}{\pgfqpoint{0.941599in}{1.565369in}}{\pgfqpoint{0.935775in}{1.559545in}}%
\pgfpathcurveto{\pgfqpoint{0.929951in}{1.553721in}}{\pgfqpoint{0.926679in}{1.545821in}}{\pgfqpoint{0.926679in}{1.537584in}}%
\pgfpathcurveto{\pgfqpoint{0.926679in}{1.529348in}}{\pgfqpoint{0.929951in}{1.521448in}}{\pgfqpoint{0.935775in}{1.515624in}}%
\pgfpathcurveto{\pgfqpoint{0.941599in}{1.509800in}}{\pgfqpoint{0.949499in}{1.506528in}}{\pgfqpoint{0.957736in}{1.506528in}}%
\pgfpathclose%
\pgfusepath{stroke,fill}%
\end{pgfscope}%
\begin{pgfscope}%
\pgfpathrectangle{\pgfqpoint{0.100000in}{0.220728in}}{\pgfqpoint{3.696000in}{3.696000in}}%
\pgfusepath{clip}%
\pgfsetbuttcap%
\pgfsetroundjoin%
\definecolor{currentfill}{rgb}{0.121569,0.466667,0.705882}%
\pgfsetfillcolor{currentfill}%
\pgfsetfillopacity{0.571172}%
\pgfsetlinewidth{1.003750pt}%
\definecolor{currentstroke}{rgb}{0.121569,0.466667,0.705882}%
\pgfsetstrokecolor{currentstroke}%
\pgfsetstrokeopacity{0.571172}%
\pgfsetdash{}{0pt}%
\pgfpathmoveto{\pgfqpoint{0.957130in}{1.505775in}}%
\pgfpathcurveto{\pgfqpoint{0.965366in}{1.505775in}}{\pgfqpoint{0.973266in}{1.509047in}}{\pgfqpoint{0.979090in}{1.514871in}}%
\pgfpathcurveto{\pgfqpoint{0.984914in}{1.520695in}}{\pgfqpoint{0.988187in}{1.528595in}}{\pgfqpoint{0.988187in}{1.536831in}}%
\pgfpathcurveto{\pgfqpoint{0.988187in}{1.545067in}}{\pgfqpoint{0.984914in}{1.552967in}}{\pgfqpoint{0.979090in}{1.558791in}}%
\pgfpathcurveto{\pgfqpoint{0.973266in}{1.564615in}}{\pgfqpoint{0.965366in}{1.567888in}}{\pgfqpoint{0.957130in}{1.567888in}}%
\pgfpathcurveto{\pgfqpoint{0.948894in}{1.567888in}}{\pgfqpoint{0.940994in}{1.564615in}}{\pgfqpoint{0.935170in}{1.558791in}}%
\pgfpathcurveto{\pgfqpoint{0.929346in}{1.552967in}}{\pgfqpoint{0.926074in}{1.545067in}}{\pgfqpoint{0.926074in}{1.536831in}}%
\pgfpathcurveto{\pgfqpoint{0.926074in}{1.528595in}}{\pgfqpoint{0.929346in}{1.520695in}}{\pgfqpoint{0.935170in}{1.514871in}}%
\pgfpathcurveto{\pgfqpoint{0.940994in}{1.509047in}}{\pgfqpoint{0.948894in}{1.505775in}}{\pgfqpoint{0.957130in}{1.505775in}}%
\pgfpathclose%
\pgfusepath{stroke,fill}%
\end{pgfscope}%
\begin{pgfscope}%
\pgfpathrectangle{\pgfqpoint{0.100000in}{0.220728in}}{\pgfqpoint{3.696000in}{3.696000in}}%
\pgfusepath{clip}%
\pgfsetbuttcap%
\pgfsetroundjoin%
\definecolor{currentfill}{rgb}{0.121569,0.466667,0.705882}%
\pgfsetfillcolor{currentfill}%
\pgfsetfillopacity{0.571295}%
\pgfsetlinewidth{1.003750pt}%
\definecolor{currentstroke}{rgb}{0.121569,0.466667,0.705882}%
\pgfsetstrokecolor{currentstroke}%
\pgfsetstrokeopacity{0.571295}%
\pgfsetdash{}{0pt}%
\pgfpathmoveto{\pgfqpoint{2.917503in}{3.019544in}}%
\pgfpathcurveto{\pgfqpoint{2.925740in}{3.019544in}}{\pgfqpoint{2.933640in}{3.022816in}}{\pgfqpoint{2.939464in}{3.028640in}}%
\pgfpathcurveto{\pgfqpoint{2.945287in}{3.034464in}}{\pgfqpoint{2.948560in}{3.042364in}}{\pgfqpoint{2.948560in}{3.050601in}}%
\pgfpathcurveto{\pgfqpoint{2.948560in}{3.058837in}}{\pgfqpoint{2.945287in}{3.066737in}}{\pgfqpoint{2.939464in}{3.072561in}}%
\pgfpathcurveto{\pgfqpoint{2.933640in}{3.078385in}}{\pgfqpoint{2.925740in}{3.081657in}}{\pgfqpoint{2.917503in}{3.081657in}}%
\pgfpathcurveto{\pgfqpoint{2.909267in}{3.081657in}}{\pgfqpoint{2.901367in}{3.078385in}}{\pgfqpoint{2.895543in}{3.072561in}}%
\pgfpathcurveto{\pgfqpoint{2.889719in}{3.066737in}}{\pgfqpoint{2.886447in}{3.058837in}}{\pgfqpoint{2.886447in}{3.050601in}}%
\pgfpathcurveto{\pgfqpoint{2.886447in}{3.042364in}}{\pgfqpoint{2.889719in}{3.034464in}}{\pgfqpoint{2.895543in}{3.028640in}}%
\pgfpathcurveto{\pgfqpoint{2.901367in}{3.022816in}}{\pgfqpoint{2.909267in}{3.019544in}}{\pgfqpoint{2.917503in}{3.019544in}}%
\pgfpathclose%
\pgfusepath{stroke,fill}%
\end{pgfscope}%
\begin{pgfscope}%
\pgfpathrectangle{\pgfqpoint{0.100000in}{0.220728in}}{\pgfqpoint{3.696000in}{3.696000in}}%
\pgfusepath{clip}%
\pgfsetbuttcap%
\pgfsetroundjoin%
\definecolor{currentfill}{rgb}{0.121569,0.466667,0.705882}%
\pgfsetfillcolor{currentfill}%
\pgfsetfillopacity{0.571449}%
\pgfsetlinewidth{1.003750pt}%
\definecolor{currentstroke}{rgb}{0.121569,0.466667,0.705882}%
\pgfsetstrokecolor{currentstroke}%
\pgfsetstrokeopacity{0.571449}%
\pgfsetdash{}{0pt}%
\pgfpathmoveto{\pgfqpoint{0.956618in}{1.504096in}}%
\pgfpathcurveto{\pgfqpoint{0.964854in}{1.504096in}}{\pgfqpoint{0.972754in}{1.507369in}}{\pgfqpoint{0.978578in}{1.513193in}}%
\pgfpathcurveto{\pgfqpoint{0.984402in}{1.519017in}}{\pgfqpoint{0.987674in}{1.526917in}}{\pgfqpoint{0.987674in}{1.535153in}}%
\pgfpathcurveto{\pgfqpoint{0.987674in}{1.543389in}}{\pgfqpoint{0.984402in}{1.551289in}}{\pgfqpoint{0.978578in}{1.557113in}}%
\pgfpathcurveto{\pgfqpoint{0.972754in}{1.562937in}}{\pgfqpoint{0.964854in}{1.566209in}}{\pgfqpoint{0.956618in}{1.566209in}}%
\pgfpathcurveto{\pgfqpoint{0.948381in}{1.566209in}}{\pgfqpoint{0.940481in}{1.562937in}}{\pgfqpoint{0.934657in}{1.557113in}}%
\pgfpathcurveto{\pgfqpoint{0.928833in}{1.551289in}}{\pgfqpoint{0.925561in}{1.543389in}}{\pgfqpoint{0.925561in}{1.535153in}}%
\pgfpathcurveto{\pgfqpoint{0.925561in}{1.526917in}}{\pgfqpoint{0.928833in}{1.519017in}}{\pgfqpoint{0.934657in}{1.513193in}}%
\pgfpathcurveto{\pgfqpoint{0.940481in}{1.507369in}}{\pgfqpoint{0.948381in}{1.504096in}}{\pgfqpoint{0.956618in}{1.504096in}}%
\pgfpathclose%
\pgfusepath{stroke,fill}%
\end{pgfscope}%
\begin{pgfscope}%
\pgfpathrectangle{\pgfqpoint{0.100000in}{0.220728in}}{\pgfqpoint{3.696000in}{3.696000in}}%
\pgfusepath{clip}%
\pgfsetbuttcap%
\pgfsetroundjoin%
\definecolor{currentfill}{rgb}{0.121569,0.466667,0.705882}%
\pgfsetfillcolor{currentfill}%
\pgfsetfillopacity{0.571849}%
\pgfsetlinewidth{1.003750pt}%
\definecolor{currentstroke}{rgb}{0.121569,0.466667,0.705882}%
\pgfsetstrokecolor{currentstroke}%
\pgfsetstrokeopacity{0.571849}%
\pgfsetdash{}{0pt}%
\pgfpathmoveto{\pgfqpoint{0.954785in}{1.501678in}}%
\pgfpathcurveto{\pgfqpoint{0.963022in}{1.501678in}}{\pgfqpoint{0.970922in}{1.504950in}}{\pgfqpoint{0.976746in}{1.510774in}}%
\pgfpathcurveto{\pgfqpoint{0.982570in}{1.516598in}}{\pgfqpoint{0.985842in}{1.524498in}}{\pgfqpoint{0.985842in}{1.532734in}}%
\pgfpathcurveto{\pgfqpoint{0.985842in}{1.540970in}}{\pgfqpoint{0.982570in}{1.548871in}}{\pgfqpoint{0.976746in}{1.554694in}}%
\pgfpathcurveto{\pgfqpoint{0.970922in}{1.560518in}}{\pgfqpoint{0.963022in}{1.563791in}}{\pgfqpoint{0.954785in}{1.563791in}}%
\pgfpathcurveto{\pgfqpoint{0.946549in}{1.563791in}}{\pgfqpoint{0.938649in}{1.560518in}}{\pgfqpoint{0.932825in}{1.554694in}}%
\pgfpathcurveto{\pgfqpoint{0.927001in}{1.548871in}}{\pgfqpoint{0.923729in}{1.540970in}}{\pgfqpoint{0.923729in}{1.532734in}}%
\pgfpathcurveto{\pgfqpoint{0.923729in}{1.524498in}}{\pgfqpoint{0.927001in}{1.516598in}}{\pgfqpoint{0.932825in}{1.510774in}}%
\pgfpathcurveto{\pgfqpoint{0.938649in}{1.504950in}}{\pgfqpoint{0.946549in}{1.501678in}}{\pgfqpoint{0.954785in}{1.501678in}}%
\pgfpathclose%
\pgfusepath{stroke,fill}%
\end{pgfscope}%
\begin{pgfscope}%
\pgfpathrectangle{\pgfqpoint{0.100000in}{0.220728in}}{\pgfqpoint{3.696000in}{3.696000in}}%
\pgfusepath{clip}%
\pgfsetbuttcap%
\pgfsetroundjoin%
\definecolor{currentfill}{rgb}{0.121569,0.466667,0.705882}%
\pgfsetfillcolor{currentfill}%
\pgfsetfillopacity{0.572101}%
\pgfsetlinewidth{1.003750pt}%
\definecolor{currentstroke}{rgb}{0.121569,0.466667,0.705882}%
\pgfsetstrokecolor{currentstroke}%
\pgfsetstrokeopacity{0.572101}%
\pgfsetdash{}{0pt}%
\pgfpathmoveto{\pgfqpoint{0.954187in}{1.500257in}}%
\pgfpathcurveto{\pgfqpoint{0.962423in}{1.500257in}}{\pgfqpoint{0.970323in}{1.503529in}}{\pgfqpoint{0.976147in}{1.509353in}}%
\pgfpathcurveto{\pgfqpoint{0.981971in}{1.515177in}}{\pgfqpoint{0.985243in}{1.523077in}}{\pgfqpoint{0.985243in}{1.531314in}}%
\pgfpathcurveto{\pgfqpoint{0.985243in}{1.539550in}}{\pgfqpoint{0.981971in}{1.547450in}}{\pgfqpoint{0.976147in}{1.553274in}}%
\pgfpathcurveto{\pgfqpoint{0.970323in}{1.559098in}}{\pgfqpoint{0.962423in}{1.562370in}}{\pgfqpoint{0.954187in}{1.562370in}}%
\pgfpathcurveto{\pgfqpoint{0.945951in}{1.562370in}}{\pgfqpoint{0.938051in}{1.559098in}}{\pgfqpoint{0.932227in}{1.553274in}}%
\pgfpathcurveto{\pgfqpoint{0.926403in}{1.547450in}}{\pgfqpoint{0.923130in}{1.539550in}}{\pgfqpoint{0.923130in}{1.531314in}}%
\pgfpathcurveto{\pgfqpoint{0.923130in}{1.523077in}}{\pgfqpoint{0.926403in}{1.515177in}}{\pgfqpoint{0.932227in}{1.509353in}}%
\pgfpathcurveto{\pgfqpoint{0.938051in}{1.503529in}}{\pgfqpoint{0.945951in}{1.500257in}}{\pgfqpoint{0.954187in}{1.500257in}}%
\pgfpathclose%
\pgfusepath{stroke,fill}%
\end{pgfscope}%
\begin{pgfscope}%
\pgfpathrectangle{\pgfqpoint{0.100000in}{0.220728in}}{\pgfqpoint{3.696000in}{3.696000in}}%
\pgfusepath{clip}%
\pgfsetbuttcap%
\pgfsetroundjoin%
\definecolor{currentfill}{rgb}{0.121569,0.466667,0.705882}%
\pgfsetfillcolor{currentfill}%
\pgfsetfillopacity{0.572212}%
\pgfsetlinewidth{1.003750pt}%
\definecolor{currentstroke}{rgb}{0.121569,0.466667,0.705882}%
\pgfsetstrokecolor{currentstroke}%
\pgfsetstrokeopacity{0.572212}%
\pgfsetdash{}{0pt}%
\pgfpathmoveto{\pgfqpoint{0.953850in}{1.499630in}}%
\pgfpathcurveto{\pgfqpoint{0.962086in}{1.499630in}}{\pgfqpoint{0.969986in}{1.502903in}}{\pgfqpoint{0.975810in}{1.508726in}}%
\pgfpathcurveto{\pgfqpoint{0.981634in}{1.514550in}}{\pgfqpoint{0.984906in}{1.522450in}}{\pgfqpoint{0.984906in}{1.530687in}}%
\pgfpathcurveto{\pgfqpoint{0.984906in}{1.538923in}}{\pgfqpoint{0.981634in}{1.546823in}}{\pgfqpoint{0.975810in}{1.552647in}}%
\pgfpathcurveto{\pgfqpoint{0.969986in}{1.558471in}}{\pgfqpoint{0.962086in}{1.561743in}}{\pgfqpoint{0.953850in}{1.561743in}}%
\pgfpathcurveto{\pgfqpoint{0.945613in}{1.561743in}}{\pgfqpoint{0.937713in}{1.558471in}}{\pgfqpoint{0.931889in}{1.552647in}}%
\pgfpathcurveto{\pgfqpoint{0.926065in}{1.546823in}}{\pgfqpoint{0.922793in}{1.538923in}}{\pgfqpoint{0.922793in}{1.530687in}}%
\pgfpathcurveto{\pgfqpoint{0.922793in}{1.522450in}}{\pgfqpoint{0.926065in}{1.514550in}}{\pgfqpoint{0.931889in}{1.508726in}}%
\pgfpathcurveto{\pgfqpoint{0.937713in}{1.502903in}}{\pgfqpoint{0.945613in}{1.499630in}}{\pgfqpoint{0.953850in}{1.499630in}}%
\pgfpathclose%
\pgfusepath{stroke,fill}%
\end{pgfscope}%
\begin{pgfscope}%
\pgfpathrectangle{\pgfqpoint{0.100000in}{0.220728in}}{\pgfqpoint{3.696000in}{3.696000in}}%
\pgfusepath{clip}%
\pgfsetbuttcap%
\pgfsetroundjoin%
\definecolor{currentfill}{rgb}{0.121569,0.466667,0.705882}%
\pgfsetfillcolor{currentfill}%
\pgfsetfillopacity{0.572405}%
\pgfsetlinewidth{1.003750pt}%
\definecolor{currentstroke}{rgb}{0.121569,0.466667,0.705882}%
\pgfsetstrokecolor{currentstroke}%
\pgfsetstrokeopacity{0.572405}%
\pgfsetdash{}{0pt}%
\pgfpathmoveto{\pgfqpoint{0.953227in}{1.498466in}}%
\pgfpathcurveto{\pgfqpoint{0.961463in}{1.498466in}}{\pgfqpoint{0.969363in}{1.501738in}}{\pgfqpoint{0.975187in}{1.507562in}}%
\pgfpathcurveto{\pgfqpoint{0.981011in}{1.513386in}}{\pgfqpoint{0.984283in}{1.521286in}}{\pgfqpoint{0.984283in}{1.529523in}}%
\pgfpathcurveto{\pgfqpoint{0.984283in}{1.537759in}}{\pgfqpoint{0.981011in}{1.545659in}}{\pgfqpoint{0.975187in}{1.551483in}}%
\pgfpathcurveto{\pgfqpoint{0.969363in}{1.557307in}}{\pgfqpoint{0.961463in}{1.560579in}}{\pgfqpoint{0.953227in}{1.560579in}}%
\pgfpathcurveto{\pgfqpoint{0.944990in}{1.560579in}}{\pgfqpoint{0.937090in}{1.557307in}}{\pgfqpoint{0.931266in}{1.551483in}}%
\pgfpathcurveto{\pgfqpoint{0.925442in}{1.545659in}}{\pgfqpoint{0.922170in}{1.537759in}}{\pgfqpoint{0.922170in}{1.529523in}}%
\pgfpathcurveto{\pgfqpoint{0.922170in}{1.521286in}}{\pgfqpoint{0.925442in}{1.513386in}}{\pgfqpoint{0.931266in}{1.507562in}}%
\pgfpathcurveto{\pgfqpoint{0.937090in}{1.501738in}}{\pgfqpoint{0.944990in}{1.498466in}}{\pgfqpoint{0.953227in}{1.498466in}}%
\pgfpathclose%
\pgfusepath{stroke,fill}%
\end{pgfscope}%
\begin{pgfscope}%
\pgfpathrectangle{\pgfqpoint{0.100000in}{0.220728in}}{\pgfqpoint{3.696000in}{3.696000in}}%
\pgfusepath{clip}%
\pgfsetbuttcap%
\pgfsetroundjoin%
\definecolor{currentfill}{rgb}{0.121569,0.466667,0.705882}%
\pgfsetfillcolor{currentfill}%
\pgfsetfillopacity{0.572575}%
\pgfsetlinewidth{1.003750pt}%
\definecolor{currentstroke}{rgb}{0.121569,0.466667,0.705882}%
\pgfsetstrokecolor{currentstroke}%
\pgfsetstrokeopacity{0.572575}%
\pgfsetdash{}{0pt}%
\pgfpathmoveto{\pgfqpoint{2.924182in}{3.018207in}}%
\pgfpathcurveto{\pgfqpoint{2.932419in}{3.018207in}}{\pgfqpoint{2.940319in}{3.021479in}}{\pgfqpoint{2.946143in}{3.027303in}}%
\pgfpathcurveto{\pgfqpoint{2.951967in}{3.033127in}}{\pgfqpoint{2.955239in}{3.041027in}}{\pgfqpoint{2.955239in}{3.049263in}}%
\pgfpathcurveto{\pgfqpoint{2.955239in}{3.057500in}}{\pgfqpoint{2.951967in}{3.065400in}}{\pgfqpoint{2.946143in}{3.071224in}}%
\pgfpathcurveto{\pgfqpoint{2.940319in}{3.077047in}}{\pgfqpoint{2.932419in}{3.080320in}}{\pgfqpoint{2.924182in}{3.080320in}}%
\pgfpathcurveto{\pgfqpoint{2.915946in}{3.080320in}}{\pgfqpoint{2.908046in}{3.077047in}}{\pgfqpoint{2.902222in}{3.071224in}}%
\pgfpathcurveto{\pgfqpoint{2.896398in}{3.065400in}}{\pgfqpoint{2.893126in}{3.057500in}}{\pgfqpoint{2.893126in}{3.049263in}}%
\pgfpathcurveto{\pgfqpoint{2.893126in}{3.041027in}}{\pgfqpoint{2.896398in}{3.033127in}}{\pgfqpoint{2.902222in}{3.027303in}}%
\pgfpathcurveto{\pgfqpoint{2.908046in}{3.021479in}}{\pgfqpoint{2.915946in}{3.018207in}}{\pgfqpoint{2.924182in}{3.018207in}}%
\pgfpathclose%
\pgfusepath{stroke,fill}%
\end{pgfscope}%
\begin{pgfscope}%
\pgfpathrectangle{\pgfqpoint{0.100000in}{0.220728in}}{\pgfqpoint{3.696000in}{3.696000in}}%
\pgfusepath{clip}%
\pgfsetbuttcap%
\pgfsetroundjoin%
\definecolor{currentfill}{rgb}{0.121569,0.466667,0.705882}%
\pgfsetfillcolor{currentfill}%
\pgfsetfillopacity{0.572810}%
\pgfsetlinewidth{1.003750pt}%
\definecolor{currentstroke}{rgb}{0.121569,0.466667,0.705882}%
\pgfsetstrokecolor{currentstroke}%
\pgfsetstrokeopacity{0.572810}%
\pgfsetdash{}{0pt}%
\pgfpathmoveto{\pgfqpoint{0.952204in}{1.496431in}}%
\pgfpathcurveto{\pgfqpoint{0.960440in}{1.496431in}}{\pgfqpoint{0.968340in}{1.499703in}}{\pgfqpoint{0.974164in}{1.505527in}}%
\pgfpathcurveto{\pgfqpoint{0.979988in}{1.511351in}}{\pgfqpoint{0.983261in}{1.519251in}}{\pgfqpoint{0.983261in}{1.527487in}}%
\pgfpathcurveto{\pgfqpoint{0.983261in}{1.535723in}}{\pgfqpoint{0.979988in}{1.543623in}}{\pgfqpoint{0.974164in}{1.549447in}}%
\pgfpathcurveto{\pgfqpoint{0.968340in}{1.555271in}}{\pgfqpoint{0.960440in}{1.558544in}}{\pgfqpoint{0.952204in}{1.558544in}}%
\pgfpathcurveto{\pgfqpoint{0.943968in}{1.558544in}}{\pgfqpoint{0.936068in}{1.555271in}}{\pgfqpoint{0.930244in}{1.549447in}}%
\pgfpathcurveto{\pgfqpoint{0.924420in}{1.543623in}}{\pgfqpoint{0.921148in}{1.535723in}}{\pgfqpoint{0.921148in}{1.527487in}}%
\pgfpathcurveto{\pgfqpoint{0.921148in}{1.519251in}}{\pgfqpoint{0.924420in}{1.511351in}}{\pgfqpoint{0.930244in}{1.505527in}}%
\pgfpathcurveto{\pgfqpoint{0.936068in}{1.499703in}}{\pgfqpoint{0.943968in}{1.496431in}}{\pgfqpoint{0.952204in}{1.496431in}}%
\pgfpathclose%
\pgfusepath{stroke,fill}%
\end{pgfscope}%
\begin{pgfscope}%
\pgfpathrectangle{\pgfqpoint{0.100000in}{0.220728in}}{\pgfqpoint{3.696000in}{3.696000in}}%
\pgfusepath{clip}%
\pgfsetbuttcap%
\pgfsetroundjoin%
\definecolor{currentfill}{rgb}{0.121569,0.466667,0.705882}%
\pgfsetfillcolor{currentfill}%
\pgfsetfillopacity{0.573439}%
\pgfsetlinewidth{1.003750pt}%
\definecolor{currentstroke}{rgb}{0.121569,0.466667,0.705882}%
\pgfsetstrokecolor{currentstroke}%
\pgfsetstrokeopacity{0.573439}%
\pgfsetdash{}{0pt}%
\pgfpathmoveto{\pgfqpoint{0.949964in}{1.492784in}}%
\pgfpathcurveto{\pgfqpoint{0.958200in}{1.492784in}}{\pgfqpoint{0.966100in}{1.496056in}}{\pgfqpoint{0.971924in}{1.501880in}}%
\pgfpathcurveto{\pgfqpoint{0.977748in}{1.507704in}}{\pgfqpoint{0.981020in}{1.515604in}}{\pgfqpoint{0.981020in}{1.523841in}}%
\pgfpathcurveto{\pgfqpoint{0.981020in}{1.532077in}}{\pgfqpoint{0.977748in}{1.539977in}}{\pgfqpoint{0.971924in}{1.545801in}}%
\pgfpathcurveto{\pgfqpoint{0.966100in}{1.551625in}}{\pgfqpoint{0.958200in}{1.554897in}}{\pgfqpoint{0.949964in}{1.554897in}}%
\pgfpathcurveto{\pgfqpoint{0.941727in}{1.554897in}}{\pgfqpoint{0.933827in}{1.551625in}}{\pgfqpoint{0.928004in}{1.545801in}}%
\pgfpathcurveto{\pgfqpoint{0.922180in}{1.539977in}}{\pgfqpoint{0.918907in}{1.532077in}}{\pgfqpoint{0.918907in}{1.523841in}}%
\pgfpathcurveto{\pgfqpoint{0.918907in}{1.515604in}}{\pgfqpoint{0.922180in}{1.507704in}}{\pgfqpoint{0.928004in}{1.501880in}}%
\pgfpathcurveto{\pgfqpoint{0.933827in}{1.496056in}}{\pgfqpoint{0.941727in}{1.492784in}}{\pgfqpoint{0.949964in}{1.492784in}}%
\pgfpathclose%
\pgfusepath{stroke,fill}%
\end{pgfscope}%
\begin{pgfscope}%
\pgfpathrectangle{\pgfqpoint{0.100000in}{0.220728in}}{\pgfqpoint{3.696000in}{3.696000in}}%
\pgfusepath{clip}%
\pgfsetbuttcap%
\pgfsetroundjoin%
\definecolor{currentfill}{rgb}{0.121569,0.466667,0.705882}%
\pgfsetfillcolor{currentfill}%
\pgfsetfillopacity{0.574586}%
\pgfsetlinewidth{1.003750pt}%
\definecolor{currentstroke}{rgb}{0.121569,0.466667,0.705882}%
\pgfsetstrokecolor{currentstroke}%
\pgfsetstrokeopacity{0.574586}%
\pgfsetdash{}{0pt}%
\pgfpathmoveto{\pgfqpoint{2.933012in}{3.016288in}}%
\pgfpathcurveto{\pgfqpoint{2.941248in}{3.016288in}}{\pgfqpoint{2.949148in}{3.019560in}}{\pgfqpoint{2.954972in}{3.025384in}}%
\pgfpathcurveto{\pgfqpoint{2.960796in}{3.031208in}}{\pgfqpoint{2.964068in}{3.039108in}}{\pgfqpoint{2.964068in}{3.047344in}}%
\pgfpathcurveto{\pgfqpoint{2.964068in}{3.055580in}}{\pgfqpoint{2.960796in}{3.063480in}}{\pgfqpoint{2.954972in}{3.069304in}}%
\pgfpathcurveto{\pgfqpoint{2.949148in}{3.075128in}}{\pgfqpoint{2.941248in}{3.078401in}}{\pgfqpoint{2.933012in}{3.078401in}}%
\pgfpathcurveto{\pgfqpoint{2.924775in}{3.078401in}}{\pgfqpoint{2.916875in}{3.075128in}}{\pgfqpoint{2.911051in}{3.069304in}}%
\pgfpathcurveto{\pgfqpoint{2.905227in}{3.063480in}}{\pgfqpoint{2.901955in}{3.055580in}}{\pgfqpoint{2.901955in}{3.047344in}}%
\pgfpathcurveto{\pgfqpoint{2.901955in}{3.039108in}}{\pgfqpoint{2.905227in}{3.031208in}}{\pgfqpoint{2.911051in}{3.025384in}}%
\pgfpathcurveto{\pgfqpoint{2.916875in}{3.019560in}}{\pgfqpoint{2.924775in}{3.016288in}}{\pgfqpoint{2.933012in}{3.016288in}}%
\pgfpathclose%
\pgfusepath{stroke,fill}%
\end{pgfscope}%
\begin{pgfscope}%
\pgfpathrectangle{\pgfqpoint{0.100000in}{0.220728in}}{\pgfqpoint{3.696000in}{3.696000in}}%
\pgfusepath{clip}%
\pgfsetbuttcap%
\pgfsetroundjoin%
\definecolor{currentfill}{rgb}{0.121569,0.466667,0.705882}%
\pgfsetfillcolor{currentfill}%
\pgfsetfillopacity{0.574851}%
\pgfsetlinewidth{1.003750pt}%
\definecolor{currentstroke}{rgb}{0.121569,0.466667,0.705882}%
\pgfsetstrokecolor{currentstroke}%
\pgfsetstrokeopacity{0.574851}%
\pgfsetdash{}{0pt}%
\pgfpathmoveto{\pgfqpoint{0.947297in}{1.485730in}}%
\pgfpathcurveto{\pgfqpoint{0.955533in}{1.485730in}}{\pgfqpoint{0.963433in}{1.489003in}}{\pgfqpoint{0.969257in}{1.494826in}}%
\pgfpathcurveto{\pgfqpoint{0.975081in}{1.500650in}}{\pgfqpoint{0.978353in}{1.508550in}}{\pgfqpoint{0.978353in}{1.516787in}}%
\pgfpathcurveto{\pgfqpoint{0.978353in}{1.525023in}}{\pgfqpoint{0.975081in}{1.532923in}}{\pgfqpoint{0.969257in}{1.538747in}}%
\pgfpathcurveto{\pgfqpoint{0.963433in}{1.544571in}}{\pgfqpoint{0.955533in}{1.547843in}}{\pgfqpoint{0.947297in}{1.547843in}}%
\pgfpathcurveto{\pgfqpoint{0.939061in}{1.547843in}}{\pgfqpoint{0.931161in}{1.544571in}}{\pgfqpoint{0.925337in}{1.538747in}}%
\pgfpathcurveto{\pgfqpoint{0.919513in}{1.532923in}}{\pgfqpoint{0.916240in}{1.525023in}}{\pgfqpoint{0.916240in}{1.516787in}}%
\pgfpathcurveto{\pgfqpoint{0.916240in}{1.508550in}}{\pgfqpoint{0.919513in}{1.500650in}}{\pgfqpoint{0.925337in}{1.494826in}}%
\pgfpathcurveto{\pgfqpoint{0.931161in}{1.489003in}}{\pgfqpoint{0.939061in}{1.485730in}}{\pgfqpoint{0.947297in}{1.485730in}}%
\pgfpathclose%
\pgfusepath{stroke,fill}%
\end{pgfscope}%
\begin{pgfscope}%
\pgfpathrectangle{\pgfqpoint{0.100000in}{0.220728in}}{\pgfqpoint{3.696000in}{3.696000in}}%
\pgfusepath{clip}%
\pgfsetbuttcap%
\pgfsetroundjoin%
\definecolor{currentfill}{rgb}{0.121569,0.466667,0.705882}%
\pgfsetfillcolor{currentfill}%
\pgfsetfillopacity{0.575533}%
\pgfsetlinewidth{1.003750pt}%
\definecolor{currentstroke}{rgb}{0.121569,0.466667,0.705882}%
\pgfsetstrokecolor{currentstroke}%
\pgfsetstrokeopacity{0.575533}%
\pgfsetdash{}{0pt}%
\pgfpathmoveto{\pgfqpoint{0.944201in}{1.481438in}}%
\pgfpathcurveto{\pgfqpoint{0.952437in}{1.481438in}}{\pgfqpoint{0.960338in}{1.484710in}}{\pgfqpoint{0.966161in}{1.490534in}}%
\pgfpathcurveto{\pgfqpoint{0.971985in}{1.496358in}}{\pgfqpoint{0.975258in}{1.504258in}}{\pgfqpoint{0.975258in}{1.512495in}}%
\pgfpathcurveto{\pgfqpoint{0.975258in}{1.520731in}}{\pgfqpoint{0.971985in}{1.528631in}}{\pgfqpoint{0.966161in}{1.534455in}}%
\pgfpathcurveto{\pgfqpoint{0.960338in}{1.540279in}}{\pgfqpoint{0.952437in}{1.543551in}}{\pgfqpoint{0.944201in}{1.543551in}}%
\pgfpathcurveto{\pgfqpoint{0.935965in}{1.543551in}}{\pgfqpoint{0.928065in}{1.540279in}}{\pgfqpoint{0.922241in}{1.534455in}}%
\pgfpathcurveto{\pgfqpoint{0.916417in}{1.528631in}}{\pgfqpoint{0.913145in}{1.520731in}}{\pgfqpoint{0.913145in}{1.512495in}}%
\pgfpathcurveto{\pgfqpoint{0.913145in}{1.504258in}}{\pgfqpoint{0.916417in}{1.496358in}}{\pgfqpoint{0.922241in}{1.490534in}}%
\pgfpathcurveto{\pgfqpoint{0.928065in}{1.484710in}}{\pgfqpoint{0.935965in}{1.481438in}}{\pgfqpoint{0.944201in}{1.481438in}}%
\pgfpathclose%
\pgfusepath{stroke,fill}%
\end{pgfscope}%
\begin{pgfscope}%
\pgfpathrectangle{\pgfqpoint{0.100000in}{0.220728in}}{\pgfqpoint{3.696000in}{3.696000in}}%
\pgfusepath{clip}%
\pgfsetbuttcap%
\pgfsetroundjoin%
\definecolor{currentfill}{rgb}{0.121569,0.466667,0.705882}%
\pgfsetfillcolor{currentfill}%
\pgfsetfillopacity{0.577181}%
\pgfsetlinewidth{1.003750pt}%
\definecolor{currentstroke}{rgb}{0.121569,0.466667,0.705882}%
\pgfsetstrokecolor{currentstroke}%
\pgfsetstrokeopacity{0.577181}%
\pgfsetdash{}{0pt}%
\pgfpathmoveto{\pgfqpoint{2.943488in}{3.015413in}}%
\pgfpathcurveto{\pgfqpoint{2.951724in}{3.015413in}}{\pgfqpoint{2.959624in}{3.018686in}}{\pgfqpoint{2.965448in}{3.024510in}}%
\pgfpathcurveto{\pgfqpoint{2.971272in}{3.030334in}}{\pgfqpoint{2.974544in}{3.038234in}}{\pgfqpoint{2.974544in}{3.046470in}}%
\pgfpathcurveto{\pgfqpoint{2.974544in}{3.054706in}}{\pgfqpoint{2.971272in}{3.062606in}}{\pgfqpoint{2.965448in}{3.068430in}}%
\pgfpathcurveto{\pgfqpoint{2.959624in}{3.074254in}}{\pgfqpoint{2.951724in}{3.077526in}}{\pgfqpoint{2.943488in}{3.077526in}}%
\pgfpathcurveto{\pgfqpoint{2.935251in}{3.077526in}}{\pgfqpoint{2.927351in}{3.074254in}}{\pgfqpoint{2.921527in}{3.068430in}}%
\pgfpathcurveto{\pgfqpoint{2.915703in}{3.062606in}}{\pgfqpoint{2.912431in}{3.054706in}}{\pgfqpoint{2.912431in}{3.046470in}}%
\pgfpathcurveto{\pgfqpoint{2.912431in}{3.038234in}}{\pgfqpoint{2.915703in}{3.030334in}}{\pgfqpoint{2.921527in}{3.024510in}}%
\pgfpathcurveto{\pgfqpoint{2.927351in}{3.018686in}}{\pgfqpoint{2.935251in}{3.015413in}}{\pgfqpoint{2.943488in}{3.015413in}}%
\pgfpathclose%
\pgfusepath{stroke,fill}%
\end{pgfscope}%
\begin{pgfscope}%
\pgfpathrectangle{\pgfqpoint{0.100000in}{0.220728in}}{\pgfqpoint{3.696000in}{3.696000in}}%
\pgfusepath{clip}%
\pgfsetbuttcap%
\pgfsetroundjoin%
\definecolor{currentfill}{rgb}{0.121569,0.466667,0.705882}%
\pgfsetfillcolor{currentfill}%
\pgfsetfillopacity{0.577203}%
\pgfsetlinewidth{1.003750pt}%
\definecolor{currentstroke}{rgb}{0.121569,0.466667,0.705882}%
\pgfsetstrokecolor{currentstroke}%
\pgfsetstrokeopacity{0.577203}%
\pgfsetdash{}{0pt}%
\pgfpathmoveto{\pgfqpoint{0.941129in}{1.472349in}}%
\pgfpathcurveto{\pgfqpoint{0.949365in}{1.472349in}}{\pgfqpoint{0.957265in}{1.475622in}}{\pgfqpoint{0.963089in}{1.481446in}}%
\pgfpathcurveto{\pgfqpoint{0.968913in}{1.487269in}}{\pgfqpoint{0.972185in}{1.495170in}}{\pgfqpoint{0.972185in}{1.503406in}}%
\pgfpathcurveto{\pgfqpoint{0.972185in}{1.511642in}}{\pgfqpoint{0.968913in}{1.519542in}}{\pgfqpoint{0.963089in}{1.525366in}}%
\pgfpathcurveto{\pgfqpoint{0.957265in}{1.531190in}}{\pgfqpoint{0.949365in}{1.534462in}}{\pgfqpoint{0.941129in}{1.534462in}}%
\pgfpathcurveto{\pgfqpoint{0.932892in}{1.534462in}}{\pgfqpoint{0.924992in}{1.531190in}}{\pgfqpoint{0.919168in}{1.525366in}}%
\pgfpathcurveto{\pgfqpoint{0.913344in}{1.519542in}}{\pgfqpoint{0.910072in}{1.511642in}}{\pgfqpoint{0.910072in}{1.503406in}}%
\pgfpathcurveto{\pgfqpoint{0.910072in}{1.495170in}}{\pgfqpoint{0.913344in}{1.487269in}}{\pgfqpoint{0.919168in}{1.481446in}}%
\pgfpathcurveto{\pgfqpoint{0.924992in}{1.475622in}}{\pgfqpoint{0.932892in}{1.472349in}}{\pgfqpoint{0.941129in}{1.472349in}}%
\pgfpathclose%
\pgfusepath{stroke,fill}%
\end{pgfscope}%
\begin{pgfscope}%
\pgfpathrectangle{\pgfqpoint{0.100000in}{0.220728in}}{\pgfqpoint{3.696000in}{3.696000in}}%
\pgfusepath{clip}%
\pgfsetbuttcap%
\pgfsetroundjoin%
\definecolor{currentfill}{rgb}{0.121569,0.466667,0.705882}%
\pgfsetfillcolor{currentfill}%
\pgfsetfillopacity{0.578078}%
\pgfsetlinewidth{1.003750pt}%
\definecolor{currentstroke}{rgb}{0.121569,0.466667,0.705882}%
\pgfsetstrokecolor{currentstroke}%
\pgfsetstrokeopacity{0.578078}%
\pgfsetdash{}{0pt}%
\pgfpathmoveto{\pgfqpoint{0.937013in}{1.466610in}}%
\pgfpathcurveto{\pgfqpoint{0.945249in}{1.466610in}}{\pgfqpoint{0.953149in}{1.469882in}}{\pgfqpoint{0.958973in}{1.475706in}}%
\pgfpathcurveto{\pgfqpoint{0.964797in}{1.481530in}}{\pgfqpoint{0.968069in}{1.489430in}}{\pgfqpoint{0.968069in}{1.497667in}}%
\pgfpathcurveto{\pgfqpoint{0.968069in}{1.505903in}}{\pgfqpoint{0.964797in}{1.513803in}}{\pgfqpoint{0.958973in}{1.519627in}}%
\pgfpathcurveto{\pgfqpoint{0.953149in}{1.525451in}}{\pgfqpoint{0.945249in}{1.528723in}}{\pgfqpoint{0.937013in}{1.528723in}}%
\pgfpathcurveto{\pgfqpoint{0.928776in}{1.528723in}}{\pgfqpoint{0.920876in}{1.525451in}}{\pgfqpoint{0.915052in}{1.519627in}}%
\pgfpathcurveto{\pgfqpoint{0.909229in}{1.513803in}}{\pgfqpoint{0.905956in}{1.505903in}}{\pgfqpoint{0.905956in}{1.497667in}}%
\pgfpathcurveto{\pgfqpoint{0.905956in}{1.489430in}}{\pgfqpoint{0.909229in}{1.481530in}}{\pgfqpoint{0.915052in}{1.475706in}}%
\pgfpathcurveto{\pgfqpoint{0.920876in}{1.469882in}}{\pgfqpoint{0.928776in}{1.466610in}}{\pgfqpoint{0.937013in}{1.466610in}}%
\pgfpathclose%
\pgfusepath{stroke,fill}%
\end{pgfscope}%
\begin{pgfscope}%
\pgfpathrectangle{\pgfqpoint{0.100000in}{0.220728in}}{\pgfqpoint{3.696000in}{3.696000in}}%
\pgfusepath{clip}%
\pgfsetbuttcap%
\pgfsetroundjoin%
\definecolor{currentfill}{rgb}{0.121569,0.466667,0.705882}%
\pgfsetfillcolor{currentfill}%
\pgfsetfillopacity{0.578679}%
\pgfsetlinewidth{1.003750pt}%
\definecolor{currentstroke}{rgb}{0.121569,0.466667,0.705882}%
\pgfsetstrokecolor{currentstroke}%
\pgfsetstrokeopacity{0.578679}%
\pgfsetdash{}{0pt}%
\pgfpathmoveto{\pgfqpoint{2.955977in}{3.013598in}}%
\pgfpathcurveto{\pgfqpoint{2.964214in}{3.013598in}}{\pgfqpoint{2.972114in}{3.016870in}}{\pgfqpoint{2.977938in}{3.022694in}}%
\pgfpathcurveto{\pgfqpoint{2.983761in}{3.028518in}}{\pgfqpoint{2.987034in}{3.036418in}}{\pgfqpoint{2.987034in}{3.044655in}}%
\pgfpathcurveto{\pgfqpoint{2.987034in}{3.052891in}}{\pgfqpoint{2.983761in}{3.060791in}}{\pgfqpoint{2.977938in}{3.066615in}}%
\pgfpathcurveto{\pgfqpoint{2.972114in}{3.072439in}}{\pgfqpoint{2.964214in}{3.075711in}}{\pgfqpoint{2.955977in}{3.075711in}}%
\pgfpathcurveto{\pgfqpoint{2.947741in}{3.075711in}}{\pgfqpoint{2.939841in}{3.072439in}}{\pgfqpoint{2.934017in}{3.066615in}}%
\pgfpathcurveto{\pgfqpoint{2.928193in}{3.060791in}}{\pgfqpoint{2.924921in}{3.052891in}}{\pgfqpoint{2.924921in}{3.044655in}}%
\pgfpathcurveto{\pgfqpoint{2.924921in}{3.036418in}}{\pgfqpoint{2.928193in}{3.028518in}}{\pgfqpoint{2.934017in}{3.022694in}}%
\pgfpathcurveto{\pgfqpoint{2.939841in}{3.016870in}}{\pgfqpoint{2.947741in}{3.013598in}}{\pgfqpoint{2.955977in}{3.013598in}}%
\pgfpathclose%
\pgfusepath{stroke,fill}%
\end{pgfscope}%
\begin{pgfscope}%
\pgfpathrectangle{\pgfqpoint{0.100000in}{0.220728in}}{\pgfqpoint{3.696000in}{3.696000in}}%
\pgfusepath{clip}%
\pgfsetbuttcap%
\pgfsetroundjoin%
\definecolor{currentfill}{rgb}{0.121569,0.466667,0.705882}%
\pgfsetfillcolor{currentfill}%
\pgfsetfillopacity{0.580185}%
\pgfsetlinewidth{1.003750pt}%
\definecolor{currentstroke}{rgb}{0.121569,0.466667,0.705882}%
\pgfsetstrokecolor{currentstroke}%
\pgfsetstrokeopacity{0.580185}%
\pgfsetdash{}{0pt}%
\pgfpathmoveto{\pgfqpoint{0.932219in}{1.454716in}}%
\pgfpathcurveto{\pgfqpoint{0.940456in}{1.454716in}}{\pgfqpoint{0.948356in}{1.457988in}}{\pgfqpoint{0.954180in}{1.463812in}}%
\pgfpathcurveto{\pgfqpoint{0.960004in}{1.469636in}}{\pgfqpoint{0.963276in}{1.477536in}}{\pgfqpoint{0.963276in}{1.485772in}}%
\pgfpathcurveto{\pgfqpoint{0.963276in}{1.494008in}}{\pgfqpoint{0.960004in}{1.501908in}}{\pgfqpoint{0.954180in}{1.507732in}}%
\pgfpathcurveto{\pgfqpoint{0.948356in}{1.513556in}}{\pgfqpoint{0.940456in}{1.516829in}}{\pgfqpoint{0.932219in}{1.516829in}}%
\pgfpathcurveto{\pgfqpoint{0.923983in}{1.516829in}}{\pgfqpoint{0.916083in}{1.513556in}}{\pgfqpoint{0.910259in}{1.507732in}}%
\pgfpathcurveto{\pgfqpoint{0.904435in}{1.501908in}}{\pgfqpoint{0.901163in}{1.494008in}}{\pgfqpoint{0.901163in}{1.485772in}}%
\pgfpathcurveto{\pgfqpoint{0.901163in}{1.477536in}}{\pgfqpoint{0.904435in}{1.469636in}}{\pgfqpoint{0.910259in}{1.463812in}}%
\pgfpathcurveto{\pgfqpoint{0.916083in}{1.457988in}}{\pgfqpoint{0.923983in}{1.454716in}}{\pgfqpoint{0.932219in}{1.454716in}}%
\pgfpathclose%
\pgfusepath{stroke,fill}%
\end{pgfscope}%
\begin{pgfscope}%
\pgfpathrectangle{\pgfqpoint{0.100000in}{0.220728in}}{\pgfqpoint{3.696000in}{3.696000in}}%
\pgfusepath{clip}%
\pgfsetbuttcap%
\pgfsetroundjoin%
\definecolor{currentfill}{rgb}{0.121569,0.466667,0.705882}%
\pgfsetfillcolor{currentfill}%
\pgfsetfillopacity{0.580231}%
\pgfsetlinewidth{1.003750pt}%
\definecolor{currentstroke}{rgb}{0.121569,0.466667,0.705882}%
\pgfsetstrokecolor{currentstroke}%
\pgfsetstrokeopacity{0.580231}%
\pgfsetdash{}{0pt}%
\pgfpathmoveto{\pgfqpoint{2.961956in}{3.011931in}}%
\pgfpathcurveto{\pgfqpoint{2.970192in}{3.011931in}}{\pgfqpoint{2.978092in}{3.015204in}}{\pgfqpoint{2.983916in}{3.021028in}}%
\pgfpathcurveto{\pgfqpoint{2.989740in}{3.026851in}}{\pgfqpoint{2.993012in}{3.034752in}}{\pgfqpoint{2.993012in}{3.042988in}}%
\pgfpathcurveto{\pgfqpoint{2.993012in}{3.051224in}}{\pgfqpoint{2.989740in}{3.059124in}}{\pgfqpoint{2.983916in}{3.064948in}}%
\pgfpathcurveto{\pgfqpoint{2.978092in}{3.070772in}}{\pgfqpoint{2.970192in}{3.074044in}}{\pgfqpoint{2.961956in}{3.074044in}}%
\pgfpathcurveto{\pgfqpoint{2.953719in}{3.074044in}}{\pgfqpoint{2.945819in}{3.070772in}}{\pgfqpoint{2.939995in}{3.064948in}}%
\pgfpathcurveto{\pgfqpoint{2.934171in}{3.059124in}}{\pgfqpoint{2.930899in}{3.051224in}}{\pgfqpoint{2.930899in}{3.042988in}}%
\pgfpathcurveto{\pgfqpoint{2.930899in}{3.034752in}}{\pgfqpoint{2.934171in}{3.026851in}}{\pgfqpoint{2.939995in}{3.021028in}}%
\pgfpathcurveto{\pgfqpoint{2.945819in}{3.015204in}}{\pgfqpoint{2.953719in}{3.011931in}}{\pgfqpoint{2.961956in}{3.011931in}}%
\pgfpathclose%
\pgfusepath{stroke,fill}%
\end{pgfscope}%
\begin{pgfscope}%
\pgfpathrectangle{\pgfqpoint{0.100000in}{0.220728in}}{\pgfqpoint{3.696000in}{3.696000in}}%
\pgfusepath{clip}%
\pgfsetbuttcap%
\pgfsetroundjoin%
\definecolor{currentfill}{rgb}{0.121569,0.466667,0.705882}%
\pgfsetfillcolor{currentfill}%
\pgfsetfillopacity{0.580908}%
\pgfsetlinewidth{1.003750pt}%
\definecolor{currentstroke}{rgb}{0.121569,0.466667,0.705882}%
\pgfsetstrokecolor{currentstroke}%
\pgfsetstrokeopacity{0.580908}%
\pgfsetdash{}{0pt}%
\pgfpathmoveto{\pgfqpoint{2.970002in}{3.010648in}}%
\pgfpathcurveto{\pgfqpoint{2.978238in}{3.010648in}}{\pgfqpoint{2.986138in}{3.013921in}}{\pgfqpoint{2.991962in}{3.019745in}}%
\pgfpathcurveto{\pgfqpoint{2.997786in}{3.025569in}}{\pgfqpoint{3.001059in}{3.033469in}}{\pgfqpoint{3.001059in}{3.041705in}}%
\pgfpathcurveto{\pgfqpoint{3.001059in}{3.049941in}}{\pgfqpoint{2.997786in}{3.057841in}}{\pgfqpoint{2.991962in}{3.063665in}}%
\pgfpathcurveto{\pgfqpoint{2.986138in}{3.069489in}}{\pgfqpoint{2.978238in}{3.072761in}}{\pgfqpoint{2.970002in}{3.072761in}}%
\pgfpathcurveto{\pgfqpoint{2.961766in}{3.072761in}}{\pgfqpoint{2.953866in}{3.069489in}}{\pgfqpoint{2.948042in}{3.063665in}}%
\pgfpathcurveto{\pgfqpoint{2.942218in}{3.057841in}}{\pgfqpoint{2.938946in}{3.049941in}}{\pgfqpoint{2.938946in}{3.041705in}}%
\pgfpathcurveto{\pgfqpoint{2.938946in}{3.033469in}}{\pgfqpoint{2.942218in}{3.025569in}}{\pgfqpoint{2.948042in}{3.019745in}}%
\pgfpathcurveto{\pgfqpoint{2.953866in}{3.013921in}}{\pgfqpoint{2.961766in}{3.010648in}}{\pgfqpoint{2.970002in}{3.010648in}}%
\pgfpathclose%
\pgfusepath{stroke,fill}%
\end{pgfscope}%
\begin{pgfscope}%
\pgfpathrectangle{\pgfqpoint{0.100000in}{0.220728in}}{\pgfqpoint{3.696000in}{3.696000in}}%
\pgfusepath{clip}%
\pgfsetbuttcap%
\pgfsetroundjoin%
\definecolor{currentfill}{rgb}{0.121569,0.466667,0.705882}%
\pgfsetfillcolor{currentfill}%
\pgfsetfillopacity{0.581331}%
\pgfsetlinewidth{1.003750pt}%
\definecolor{currentstroke}{rgb}{0.121569,0.466667,0.705882}%
\pgfsetstrokecolor{currentstroke}%
\pgfsetstrokeopacity{0.581331}%
\pgfsetdash{}{0pt}%
\pgfpathmoveto{\pgfqpoint{0.926005in}{1.445467in}}%
\pgfpathcurveto{\pgfqpoint{0.934242in}{1.445467in}}{\pgfqpoint{0.942142in}{1.448740in}}{\pgfqpoint{0.947966in}{1.454563in}}%
\pgfpathcurveto{\pgfqpoint{0.953789in}{1.460387in}}{\pgfqpoint{0.957062in}{1.468287in}}{\pgfqpoint{0.957062in}{1.476524in}}%
\pgfpathcurveto{\pgfqpoint{0.957062in}{1.484760in}}{\pgfqpoint{0.953789in}{1.492660in}}{\pgfqpoint{0.947966in}{1.498484in}}%
\pgfpathcurveto{\pgfqpoint{0.942142in}{1.504308in}}{\pgfqpoint{0.934242in}{1.507580in}}{\pgfqpoint{0.926005in}{1.507580in}}%
\pgfpathcurveto{\pgfqpoint{0.917769in}{1.507580in}}{\pgfqpoint{0.909869in}{1.504308in}}{\pgfqpoint{0.904045in}{1.498484in}}%
\pgfpathcurveto{\pgfqpoint{0.898221in}{1.492660in}}{\pgfqpoint{0.894949in}{1.484760in}}{\pgfqpoint{0.894949in}{1.476524in}}%
\pgfpathcurveto{\pgfqpoint{0.894949in}{1.468287in}}{\pgfqpoint{0.898221in}{1.460387in}}{\pgfqpoint{0.904045in}{1.454563in}}%
\pgfpathcurveto{\pgfqpoint{0.909869in}{1.448740in}}{\pgfqpoint{0.917769in}{1.445467in}}{\pgfqpoint{0.926005in}{1.445467in}}%
\pgfpathclose%
\pgfusepath{stroke,fill}%
\end{pgfscope}%
\begin{pgfscope}%
\pgfpathrectangle{\pgfqpoint{0.100000in}{0.220728in}}{\pgfqpoint{3.696000in}{3.696000in}}%
\pgfusepath{clip}%
\pgfsetbuttcap%
\pgfsetroundjoin%
\definecolor{currentfill}{rgb}{0.121569,0.466667,0.705882}%
\pgfsetfillcolor{currentfill}%
\pgfsetfillopacity{0.581926}%
\pgfsetlinewidth{1.003750pt}%
\definecolor{currentstroke}{rgb}{0.121569,0.466667,0.705882}%
\pgfsetstrokecolor{currentstroke}%
\pgfsetstrokeopacity{0.581926}%
\pgfsetdash{}{0pt}%
\pgfpathmoveto{\pgfqpoint{2.979431in}{3.008459in}}%
\pgfpathcurveto{\pgfqpoint{2.987668in}{3.008459in}}{\pgfqpoint{2.995568in}{3.011731in}}{\pgfqpoint{3.001392in}{3.017555in}}%
\pgfpathcurveto{\pgfqpoint{3.007215in}{3.023379in}}{\pgfqpoint{3.010488in}{3.031279in}}{\pgfqpoint{3.010488in}{3.039515in}}%
\pgfpathcurveto{\pgfqpoint{3.010488in}{3.047752in}}{\pgfqpoint{3.007215in}{3.055652in}}{\pgfqpoint{3.001392in}{3.061476in}}%
\pgfpathcurveto{\pgfqpoint{2.995568in}{3.067300in}}{\pgfqpoint{2.987668in}{3.070572in}}{\pgfqpoint{2.979431in}{3.070572in}}%
\pgfpathcurveto{\pgfqpoint{2.971195in}{3.070572in}}{\pgfqpoint{2.963295in}{3.067300in}}{\pgfqpoint{2.957471in}{3.061476in}}%
\pgfpathcurveto{\pgfqpoint{2.951647in}{3.055652in}}{\pgfqpoint{2.948375in}{3.047752in}}{\pgfqpoint{2.948375in}{3.039515in}}%
\pgfpathcurveto{\pgfqpoint{2.948375in}{3.031279in}}{\pgfqpoint{2.951647in}{3.023379in}}{\pgfqpoint{2.957471in}{3.017555in}}%
\pgfpathcurveto{\pgfqpoint{2.963295in}{3.011731in}}{\pgfqpoint{2.971195in}{3.008459in}}{\pgfqpoint{2.979431in}{3.008459in}}%
\pgfpathclose%
\pgfusepath{stroke,fill}%
\end{pgfscope}%
\begin{pgfscope}%
\pgfpathrectangle{\pgfqpoint{0.100000in}{0.220728in}}{\pgfqpoint{3.696000in}{3.696000in}}%
\pgfusepath{clip}%
\pgfsetbuttcap%
\pgfsetroundjoin%
\definecolor{currentfill}{rgb}{0.121569,0.466667,0.705882}%
\pgfsetfillcolor{currentfill}%
\pgfsetfillopacity{0.582935}%
\pgfsetlinewidth{1.003750pt}%
\definecolor{currentstroke}{rgb}{0.121569,0.466667,0.705882}%
\pgfsetstrokecolor{currentstroke}%
\pgfsetstrokeopacity{0.582935}%
\pgfsetdash{}{0pt}%
\pgfpathmoveto{\pgfqpoint{0.921821in}{1.437060in}}%
\pgfpathcurveto{\pgfqpoint{0.930058in}{1.437060in}}{\pgfqpoint{0.937958in}{1.440332in}}{\pgfqpoint{0.943782in}{1.446156in}}%
\pgfpathcurveto{\pgfqpoint{0.949605in}{1.451980in}}{\pgfqpoint{0.952878in}{1.459880in}}{\pgfqpoint{0.952878in}{1.468116in}}%
\pgfpathcurveto{\pgfqpoint{0.952878in}{1.476352in}}{\pgfqpoint{0.949605in}{1.484252in}}{\pgfqpoint{0.943782in}{1.490076in}}%
\pgfpathcurveto{\pgfqpoint{0.937958in}{1.495900in}}{\pgfqpoint{0.930058in}{1.499173in}}{\pgfqpoint{0.921821in}{1.499173in}}%
\pgfpathcurveto{\pgfqpoint{0.913585in}{1.499173in}}{\pgfqpoint{0.905685in}{1.495900in}}{\pgfqpoint{0.899861in}{1.490076in}}%
\pgfpathcurveto{\pgfqpoint{0.894037in}{1.484252in}}{\pgfqpoint{0.890765in}{1.476352in}}{\pgfqpoint{0.890765in}{1.468116in}}%
\pgfpathcurveto{\pgfqpoint{0.890765in}{1.459880in}}{\pgfqpoint{0.894037in}{1.451980in}}{\pgfqpoint{0.899861in}{1.446156in}}%
\pgfpathcurveto{\pgfqpoint{0.905685in}{1.440332in}}{\pgfqpoint{0.913585in}{1.437060in}}{\pgfqpoint{0.921821in}{1.437060in}}%
\pgfpathclose%
\pgfusepath{stroke,fill}%
\end{pgfscope}%
\begin{pgfscope}%
\pgfpathrectangle{\pgfqpoint{0.100000in}{0.220728in}}{\pgfqpoint{3.696000in}{3.696000in}}%
\pgfusepath{clip}%
\pgfsetbuttcap%
\pgfsetroundjoin%
\definecolor{currentfill}{rgb}{0.121569,0.466667,0.705882}%
\pgfsetfillcolor{currentfill}%
\pgfsetfillopacity{0.583807}%
\pgfsetlinewidth{1.003750pt}%
\definecolor{currentstroke}{rgb}{0.121569,0.466667,0.705882}%
\pgfsetstrokecolor{currentstroke}%
\pgfsetstrokeopacity{0.583807}%
\pgfsetdash{}{0pt}%
\pgfpathmoveto{\pgfqpoint{0.917628in}{1.430705in}}%
\pgfpathcurveto{\pgfqpoint{0.925865in}{1.430705in}}{\pgfqpoint{0.933765in}{1.433977in}}{\pgfqpoint{0.939589in}{1.439801in}}%
\pgfpathcurveto{\pgfqpoint{0.945412in}{1.445625in}}{\pgfqpoint{0.948685in}{1.453525in}}{\pgfqpoint{0.948685in}{1.461761in}}%
\pgfpathcurveto{\pgfqpoint{0.948685in}{1.469998in}}{\pgfqpoint{0.945412in}{1.477898in}}{\pgfqpoint{0.939589in}{1.483722in}}%
\pgfpathcurveto{\pgfqpoint{0.933765in}{1.489546in}}{\pgfqpoint{0.925865in}{1.492818in}}{\pgfqpoint{0.917628in}{1.492818in}}%
\pgfpathcurveto{\pgfqpoint{0.909392in}{1.492818in}}{\pgfqpoint{0.901492in}{1.489546in}}{\pgfqpoint{0.895668in}{1.483722in}}%
\pgfpathcurveto{\pgfqpoint{0.889844in}{1.477898in}}{\pgfqpoint{0.886572in}{1.469998in}}{\pgfqpoint{0.886572in}{1.461761in}}%
\pgfpathcurveto{\pgfqpoint{0.886572in}{1.453525in}}{\pgfqpoint{0.889844in}{1.445625in}}{\pgfqpoint{0.895668in}{1.439801in}}%
\pgfpathcurveto{\pgfqpoint{0.901492in}{1.433977in}}{\pgfqpoint{0.909392in}{1.430705in}}{\pgfqpoint{0.917628in}{1.430705in}}%
\pgfpathclose%
\pgfusepath{stroke,fill}%
\end{pgfscope}%
\begin{pgfscope}%
\pgfpathrectangle{\pgfqpoint{0.100000in}{0.220728in}}{\pgfqpoint{3.696000in}{3.696000in}}%
\pgfusepath{clip}%
\pgfsetbuttcap%
\pgfsetroundjoin%
\definecolor{currentfill}{rgb}{0.121569,0.466667,0.705882}%
\pgfsetfillcolor{currentfill}%
\pgfsetfillopacity{0.584748}%
\pgfsetlinewidth{1.003750pt}%
\definecolor{currentstroke}{rgb}{0.121569,0.466667,0.705882}%
\pgfsetstrokecolor{currentstroke}%
\pgfsetstrokeopacity{0.584748}%
\pgfsetdash{}{0pt}%
\pgfpathmoveto{\pgfqpoint{2.988497in}{3.007771in}}%
\pgfpathcurveto{\pgfqpoint{2.996734in}{3.007771in}}{\pgfqpoint{3.004634in}{3.011043in}}{\pgfqpoint{3.010458in}{3.016867in}}%
\pgfpathcurveto{\pgfqpoint{3.016282in}{3.022691in}}{\pgfqpoint{3.019554in}{3.030591in}}{\pgfqpoint{3.019554in}{3.038828in}}%
\pgfpathcurveto{\pgfqpoint{3.019554in}{3.047064in}}{\pgfqpoint{3.016282in}{3.054964in}}{\pgfqpoint{3.010458in}{3.060788in}}%
\pgfpathcurveto{\pgfqpoint{3.004634in}{3.066612in}}{\pgfqpoint{2.996734in}{3.069884in}}{\pgfqpoint{2.988497in}{3.069884in}}%
\pgfpathcurveto{\pgfqpoint{2.980261in}{3.069884in}}{\pgfqpoint{2.972361in}{3.066612in}}{\pgfqpoint{2.966537in}{3.060788in}}%
\pgfpathcurveto{\pgfqpoint{2.960713in}{3.054964in}}{\pgfqpoint{2.957441in}{3.047064in}}{\pgfqpoint{2.957441in}{3.038828in}}%
\pgfpathcurveto{\pgfqpoint{2.957441in}{3.030591in}}{\pgfqpoint{2.960713in}{3.022691in}}{\pgfqpoint{2.966537in}{3.016867in}}%
\pgfpathcurveto{\pgfqpoint{2.972361in}{3.011043in}}{\pgfqpoint{2.980261in}{3.007771in}}{\pgfqpoint{2.988497in}{3.007771in}}%
\pgfpathclose%
\pgfusepath{stroke,fill}%
\end{pgfscope}%
\begin{pgfscope}%
\pgfpathrectangle{\pgfqpoint{0.100000in}{0.220728in}}{\pgfqpoint{3.696000in}{3.696000in}}%
\pgfusepath{clip}%
\pgfsetbuttcap%
\pgfsetroundjoin%
\definecolor{currentfill}{rgb}{0.121569,0.466667,0.705882}%
\pgfsetfillcolor{currentfill}%
\pgfsetfillopacity{0.585897}%
\pgfsetlinewidth{1.003750pt}%
\definecolor{currentstroke}{rgb}{0.121569,0.466667,0.705882}%
\pgfsetstrokecolor{currentstroke}%
\pgfsetstrokeopacity{0.585897}%
\pgfsetdash{}{0pt}%
\pgfpathmoveto{\pgfqpoint{0.911658in}{1.418765in}}%
\pgfpathcurveto{\pgfqpoint{0.919894in}{1.418765in}}{\pgfqpoint{0.927794in}{1.422038in}}{\pgfqpoint{0.933618in}{1.427861in}}%
\pgfpathcurveto{\pgfqpoint{0.939442in}{1.433685in}}{\pgfqpoint{0.942714in}{1.441585in}}{\pgfqpoint{0.942714in}{1.449822in}}%
\pgfpathcurveto{\pgfqpoint{0.942714in}{1.458058in}}{\pgfqpoint{0.939442in}{1.465958in}}{\pgfqpoint{0.933618in}{1.471782in}}%
\pgfpathcurveto{\pgfqpoint{0.927794in}{1.477606in}}{\pgfqpoint{0.919894in}{1.480878in}}{\pgfqpoint{0.911658in}{1.480878in}}%
\pgfpathcurveto{\pgfqpoint{0.903421in}{1.480878in}}{\pgfqpoint{0.895521in}{1.477606in}}{\pgfqpoint{0.889697in}{1.471782in}}%
\pgfpathcurveto{\pgfqpoint{0.883873in}{1.465958in}}{\pgfqpoint{0.880601in}{1.458058in}}{\pgfqpoint{0.880601in}{1.449822in}}%
\pgfpathcurveto{\pgfqpoint{0.880601in}{1.441585in}}{\pgfqpoint{0.883873in}{1.433685in}}{\pgfqpoint{0.889697in}{1.427861in}}%
\pgfpathcurveto{\pgfqpoint{0.895521in}{1.422038in}}{\pgfqpoint{0.903421in}{1.418765in}}{\pgfqpoint{0.911658in}{1.418765in}}%
\pgfpathclose%
\pgfusepath{stroke,fill}%
\end{pgfscope}%
\begin{pgfscope}%
\pgfpathrectangle{\pgfqpoint{0.100000in}{0.220728in}}{\pgfqpoint{3.696000in}{3.696000in}}%
\pgfusepath{clip}%
\pgfsetbuttcap%
\pgfsetroundjoin%
\definecolor{currentfill}{rgb}{0.121569,0.466667,0.705882}%
\pgfsetfillcolor{currentfill}%
\pgfsetfillopacity{0.586890}%
\pgfsetlinewidth{1.003750pt}%
\definecolor{currentstroke}{rgb}{0.121569,0.466667,0.705882}%
\pgfsetstrokecolor{currentstroke}%
\pgfsetstrokeopacity{0.586890}%
\pgfsetdash{}{0pt}%
\pgfpathmoveto{\pgfqpoint{3.000279in}{3.005710in}}%
\pgfpathcurveto{\pgfqpoint{3.008515in}{3.005710in}}{\pgfqpoint{3.016415in}{3.008982in}}{\pgfqpoint{3.022239in}{3.014806in}}%
\pgfpathcurveto{\pgfqpoint{3.028063in}{3.020630in}}{\pgfqpoint{3.031335in}{3.028530in}}{\pgfqpoint{3.031335in}{3.036766in}}%
\pgfpathcurveto{\pgfqpoint{3.031335in}{3.045002in}}{\pgfqpoint{3.028063in}{3.052903in}}{\pgfqpoint{3.022239in}{3.058726in}}%
\pgfpathcurveto{\pgfqpoint{3.016415in}{3.064550in}}{\pgfqpoint{3.008515in}{3.067823in}}{\pgfqpoint{3.000279in}{3.067823in}}%
\pgfpathcurveto{\pgfqpoint{2.992043in}{3.067823in}}{\pgfqpoint{2.984143in}{3.064550in}}{\pgfqpoint{2.978319in}{3.058726in}}%
\pgfpathcurveto{\pgfqpoint{2.972495in}{3.052903in}}{\pgfqpoint{2.969222in}{3.045002in}}{\pgfqpoint{2.969222in}{3.036766in}}%
\pgfpathcurveto{\pgfqpoint{2.969222in}{3.028530in}}{\pgfqpoint{2.972495in}{3.020630in}}{\pgfqpoint{2.978319in}{3.014806in}}%
\pgfpathcurveto{\pgfqpoint{2.984143in}{3.008982in}}{\pgfqpoint{2.992043in}{3.005710in}}{\pgfqpoint{3.000279in}{3.005710in}}%
\pgfpathclose%
\pgfusepath{stroke,fill}%
\end{pgfscope}%
\begin{pgfscope}%
\pgfpathrectangle{\pgfqpoint{0.100000in}{0.220728in}}{\pgfqpoint{3.696000in}{3.696000in}}%
\pgfusepath{clip}%
\pgfsetbuttcap%
\pgfsetroundjoin%
\definecolor{currentfill}{rgb}{0.121569,0.466667,0.705882}%
\pgfsetfillcolor{currentfill}%
\pgfsetfillopacity{0.587446}%
\pgfsetlinewidth{1.003750pt}%
\definecolor{currentstroke}{rgb}{0.121569,0.466667,0.705882}%
\pgfsetstrokecolor{currentstroke}%
\pgfsetstrokeopacity{0.587446}%
\pgfsetdash{}{0pt}%
\pgfpathmoveto{\pgfqpoint{0.905400in}{1.408976in}}%
\pgfpathcurveto{\pgfqpoint{0.913636in}{1.408976in}}{\pgfqpoint{0.921536in}{1.412248in}}{\pgfqpoint{0.927360in}{1.418072in}}%
\pgfpathcurveto{\pgfqpoint{0.933184in}{1.423896in}}{\pgfqpoint{0.936456in}{1.431796in}}{\pgfqpoint{0.936456in}{1.440032in}}%
\pgfpathcurveto{\pgfqpoint{0.936456in}{1.448268in}}{\pgfqpoint{0.933184in}{1.456168in}}{\pgfqpoint{0.927360in}{1.461992in}}%
\pgfpathcurveto{\pgfqpoint{0.921536in}{1.467816in}}{\pgfqpoint{0.913636in}{1.471089in}}{\pgfqpoint{0.905400in}{1.471089in}}%
\pgfpathcurveto{\pgfqpoint{0.897163in}{1.471089in}}{\pgfqpoint{0.889263in}{1.467816in}}{\pgfqpoint{0.883439in}{1.461992in}}%
\pgfpathcurveto{\pgfqpoint{0.877615in}{1.456168in}}{\pgfqpoint{0.874343in}{1.448268in}}{\pgfqpoint{0.874343in}{1.440032in}}%
\pgfpathcurveto{\pgfqpoint{0.874343in}{1.431796in}}{\pgfqpoint{0.877615in}{1.423896in}}{\pgfqpoint{0.883439in}{1.418072in}}%
\pgfpathcurveto{\pgfqpoint{0.889263in}{1.412248in}}{\pgfqpoint{0.897163in}{1.408976in}}{\pgfqpoint{0.905400in}{1.408976in}}%
\pgfpathclose%
\pgfusepath{stroke,fill}%
\end{pgfscope}%
\begin{pgfscope}%
\pgfpathrectangle{\pgfqpoint{0.100000in}{0.220728in}}{\pgfqpoint{3.696000in}{3.696000in}}%
\pgfusepath{clip}%
\pgfsetbuttcap%
\pgfsetroundjoin%
\definecolor{currentfill}{rgb}{0.121569,0.466667,0.705882}%
\pgfsetfillcolor{currentfill}%
\pgfsetfillopacity{0.589486}%
\pgfsetlinewidth{1.003750pt}%
\definecolor{currentstroke}{rgb}{0.121569,0.466667,0.705882}%
\pgfsetstrokecolor{currentstroke}%
\pgfsetstrokeopacity{0.589486}%
\pgfsetdash{}{0pt}%
\pgfpathmoveto{\pgfqpoint{3.011837in}{3.001935in}}%
\pgfpathcurveto{\pgfqpoint{3.020073in}{3.001935in}}{\pgfqpoint{3.027973in}{3.005207in}}{\pgfqpoint{3.033797in}{3.011031in}}%
\pgfpathcurveto{\pgfqpoint{3.039621in}{3.016855in}}{\pgfqpoint{3.042893in}{3.024755in}}{\pgfqpoint{3.042893in}{3.032991in}}%
\pgfpathcurveto{\pgfqpoint{3.042893in}{3.041227in}}{\pgfqpoint{3.039621in}{3.049127in}}{\pgfqpoint{3.033797in}{3.054951in}}%
\pgfpathcurveto{\pgfqpoint{3.027973in}{3.060775in}}{\pgfqpoint{3.020073in}{3.064048in}}{\pgfqpoint{3.011837in}{3.064048in}}%
\pgfpathcurveto{\pgfqpoint{3.003600in}{3.064048in}}{\pgfqpoint{2.995700in}{3.060775in}}{\pgfqpoint{2.989876in}{3.054951in}}%
\pgfpathcurveto{\pgfqpoint{2.984052in}{3.049127in}}{\pgfqpoint{2.980780in}{3.041227in}}{\pgfqpoint{2.980780in}{3.032991in}}%
\pgfpathcurveto{\pgfqpoint{2.980780in}{3.024755in}}{\pgfqpoint{2.984052in}{3.016855in}}{\pgfqpoint{2.989876in}{3.011031in}}%
\pgfpathcurveto{\pgfqpoint{2.995700in}{3.005207in}}{\pgfqpoint{3.003600in}{3.001935in}}{\pgfqpoint{3.011837in}{3.001935in}}%
\pgfpathclose%
\pgfusepath{stroke,fill}%
\end{pgfscope}%
\begin{pgfscope}%
\pgfpathrectangle{\pgfqpoint{0.100000in}{0.220728in}}{\pgfqpoint{3.696000in}{3.696000in}}%
\pgfusepath{clip}%
\pgfsetbuttcap%
\pgfsetroundjoin%
\definecolor{currentfill}{rgb}{0.121569,0.466667,0.705882}%
\pgfsetfillcolor{currentfill}%
\pgfsetfillopacity{0.591155}%
\pgfsetlinewidth{1.003750pt}%
\definecolor{currentstroke}{rgb}{0.121569,0.466667,0.705882}%
\pgfsetstrokecolor{currentstroke}%
\pgfsetstrokeopacity{0.591155}%
\pgfsetdash{}{0pt}%
\pgfpathmoveto{\pgfqpoint{0.900443in}{1.388740in}}%
\pgfpathcurveto{\pgfqpoint{0.908680in}{1.388740in}}{\pgfqpoint{0.916580in}{1.392012in}}{\pgfqpoint{0.922404in}{1.397836in}}%
\pgfpathcurveto{\pgfqpoint{0.928228in}{1.403660in}}{\pgfqpoint{0.931500in}{1.411560in}}{\pgfqpoint{0.931500in}{1.419797in}}%
\pgfpathcurveto{\pgfqpoint{0.931500in}{1.428033in}}{\pgfqpoint{0.928228in}{1.435933in}}{\pgfqpoint{0.922404in}{1.441757in}}%
\pgfpathcurveto{\pgfqpoint{0.916580in}{1.447581in}}{\pgfqpoint{0.908680in}{1.450853in}}{\pgfqpoint{0.900443in}{1.450853in}}%
\pgfpathcurveto{\pgfqpoint{0.892207in}{1.450853in}}{\pgfqpoint{0.884307in}{1.447581in}}{\pgfqpoint{0.878483in}{1.441757in}}%
\pgfpathcurveto{\pgfqpoint{0.872659in}{1.435933in}}{\pgfqpoint{0.869387in}{1.428033in}}{\pgfqpoint{0.869387in}{1.419797in}}%
\pgfpathcurveto{\pgfqpoint{0.869387in}{1.411560in}}{\pgfqpoint{0.872659in}{1.403660in}}{\pgfqpoint{0.878483in}{1.397836in}}%
\pgfpathcurveto{\pgfqpoint{0.884307in}{1.392012in}}{\pgfqpoint{0.892207in}{1.388740in}}{\pgfqpoint{0.900443in}{1.388740in}}%
\pgfpathclose%
\pgfusepath{stroke,fill}%
\end{pgfscope}%
\begin{pgfscope}%
\pgfpathrectangle{\pgfqpoint{0.100000in}{0.220728in}}{\pgfqpoint{3.696000in}{3.696000in}}%
\pgfusepath{clip}%
\pgfsetbuttcap%
\pgfsetroundjoin%
\definecolor{currentfill}{rgb}{0.121569,0.466667,0.705882}%
\pgfsetfillcolor{currentfill}%
\pgfsetfillopacity{0.591297}%
\pgfsetlinewidth{1.003750pt}%
\definecolor{currentstroke}{rgb}{0.121569,0.466667,0.705882}%
\pgfsetstrokecolor{currentstroke}%
\pgfsetstrokeopacity{0.591297}%
\pgfsetdash{}{0pt}%
\pgfpathmoveto{\pgfqpoint{3.018033in}{3.000862in}}%
\pgfpathcurveto{\pgfqpoint{3.026270in}{3.000862in}}{\pgfqpoint{3.034170in}{3.004134in}}{\pgfqpoint{3.039994in}{3.009958in}}%
\pgfpathcurveto{\pgfqpoint{3.045818in}{3.015782in}}{\pgfqpoint{3.049090in}{3.023682in}}{\pgfqpoint{3.049090in}{3.031918in}}%
\pgfpathcurveto{\pgfqpoint{3.049090in}{3.040155in}}{\pgfqpoint{3.045818in}{3.048055in}}{\pgfqpoint{3.039994in}{3.053879in}}%
\pgfpathcurveto{\pgfqpoint{3.034170in}{3.059702in}}{\pgfqpoint{3.026270in}{3.062975in}}{\pgfqpoint{3.018033in}{3.062975in}}%
\pgfpathcurveto{\pgfqpoint{3.009797in}{3.062975in}}{\pgfqpoint{3.001897in}{3.059702in}}{\pgfqpoint{2.996073in}{3.053879in}}%
\pgfpathcurveto{\pgfqpoint{2.990249in}{3.048055in}}{\pgfqpoint{2.986977in}{3.040155in}}{\pgfqpoint{2.986977in}{3.031918in}}%
\pgfpathcurveto{\pgfqpoint{2.986977in}{3.023682in}}{\pgfqpoint{2.990249in}{3.015782in}}{\pgfqpoint{2.996073in}{3.009958in}}%
\pgfpathcurveto{\pgfqpoint{3.001897in}{3.004134in}}{\pgfqpoint{3.009797in}{3.000862in}}{\pgfqpoint{3.018033in}{3.000862in}}%
\pgfpathclose%
\pgfusepath{stroke,fill}%
\end{pgfscope}%
\begin{pgfscope}%
\pgfpathrectangle{\pgfqpoint{0.100000in}{0.220728in}}{\pgfqpoint{3.696000in}{3.696000in}}%
\pgfusepath{clip}%
\pgfsetbuttcap%
\pgfsetroundjoin%
\definecolor{currentfill}{rgb}{0.121569,0.466667,0.705882}%
\pgfsetfillcolor{currentfill}%
\pgfsetfillopacity{0.592163}%
\pgfsetlinewidth{1.003750pt}%
\definecolor{currentstroke}{rgb}{0.121569,0.466667,0.705882}%
\pgfsetstrokecolor{currentstroke}%
\pgfsetstrokeopacity{0.592163}%
\pgfsetdash{}{0pt}%
\pgfpathmoveto{\pgfqpoint{3.027257in}{2.999696in}}%
\pgfpathcurveto{\pgfqpoint{3.035493in}{2.999696in}}{\pgfqpoint{3.043393in}{3.002969in}}{\pgfqpoint{3.049217in}{3.008793in}}%
\pgfpathcurveto{\pgfqpoint{3.055041in}{3.014617in}}{\pgfqpoint{3.058313in}{3.022517in}}{\pgfqpoint{3.058313in}{3.030753in}}%
\pgfpathcurveto{\pgfqpoint{3.058313in}{3.038989in}}{\pgfqpoint{3.055041in}{3.046889in}}{\pgfqpoint{3.049217in}{3.052713in}}%
\pgfpathcurveto{\pgfqpoint{3.043393in}{3.058537in}}{\pgfqpoint{3.035493in}{3.061809in}}{\pgfqpoint{3.027257in}{3.061809in}}%
\pgfpathcurveto{\pgfqpoint{3.019021in}{3.061809in}}{\pgfqpoint{3.011121in}{3.058537in}}{\pgfqpoint{3.005297in}{3.052713in}}%
\pgfpathcurveto{\pgfqpoint{2.999473in}{3.046889in}}{\pgfqpoint{2.996200in}{3.038989in}}{\pgfqpoint{2.996200in}{3.030753in}}%
\pgfpathcurveto{\pgfqpoint{2.996200in}{3.022517in}}{\pgfqpoint{2.999473in}{3.014617in}}{\pgfqpoint{3.005297in}{3.008793in}}%
\pgfpathcurveto{\pgfqpoint{3.011121in}{3.002969in}}{\pgfqpoint{3.019021in}{2.999696in}}{\pgfqpoint{3.027257in}{2.999696in}}%
\pgfpathclose%
\pgfusepath{stroke,fill}%
\end{pgfscope}%
\begin{pgfscope}%
\pgfpathrectangle{\pgfqpoint{0.100000in}{0.220728in}}{\pgfqpoint{3.696000in}{3.696000in}}%
\pgfusepath{clip}%
\pgfsetbuttcap%
\pgfsetroundjoin%
\definecolor{currentfill}{rgb}{0.121569,0.466667,0.705882}%
\pgfsetfillcolor{currentfill}%
\pgfsetfillopacity{0.592704}%
\pgfsetlinewidth{1.003750pt}%
\definecolor{currentstroke}{rgb}{0.121569,0.466667,0.705882}%
\pgfsetstrokecolor{currentstroke}%
\pgfsetstrokeopacity{0.592704}%
\pgfsetdash{}{0pt}%
\pgfpathmoveto{\pgfqpoint{0.887904in}{1.375319in}}%
\pgfpathcurveto{\pgfqpoint{0.896141in}{1.375319in}}{\pgfqpoint{0.904041in}{1.378591in}}{\pgfqpoint{0.909865in}{1.384415in}}%
\pgfpathcurveto{\pgfqpoint{0.915689in}{1.390239in}}{\pgfqpoint{0.918961in}{1.398139in}}{\pgfqpoint{0.918961in}{1.406376in}}%
\pgfpathcurveto{\pgfqpoint{0.918961in}{1.414612in}}{\pgfqpoint{0.915689in}{1.422512in}}{\pgfqpoint{0.909865in}{1.428336in}}%
\pgfpathcurveto{\pgfqpoint{0.904041in}{1.434160in}}{\pgfqpoint{0.896141in}{1.437432in}}{\pgfqpoint{0.887904in}{1.437432in}}%
\pgfpathcurveto{\pgfqpoint{0.879668in}{1.437432in}}{\pgfqpoint{0.871768in}{1.434160in}}{\pgfqpoint{0.865944in}{1.428336in}}%
\pgfpathcurveto{\pgfqpoint{0.860120in}{1.422512in}}{\pgfqpoint{0.856848in}{1.414612in}}{\pgfqpoint{0.856848in}{1.406376in}}%
\pgfpathcurveto{\pgfqpoint{0.856848in}{1.398139in}}{\pgfqpoint{0.860120in}{1.390239in}}{\pgfqpoint{0.865944in}{1.384415in}}%
\pgfpathcurveto{\pgfqpoint{0.871768in}{1.378591in}}{\pgfqpoint{0.879668in}{1.375319in}}{\pgfqpoint{0.887904in}{1.375319in}}%
\pgfpathclose%
\pgfusepath{stroke,fill}%
\end{pgfscope}%
\begin{pgfscope}%
\pgfpathrectangle{\pgfqpoint{0.100000in}{0.220728in}}{\pgfqpoint{3.696000in}{3.696000in}}%
\pgfusepath{clip}%
\pgfsetbuttcap%
\pgfsetroundjoin%
\definecolor{currentfill}{rgb}{0.121569,0.466667,0.705882}%
\pgfsetfillcolor{currentfill}%
\pgfsetfillopacity{0.592989}%
\pgfsetlinewidth{1.003750pt}%
\definecolor{currentstroke}{rgb}{0.121569,0.466667,0.705882}%
\pgfsetstrokecolor{currentstroke}%
\pgfsetstrokeopacity{0.592989}%
\pgfsetdash{}{0pt}%
\pgfpathmoveto{\pgfqpoint{3.031922in}{2.998385in}}%
\pgfpathcurveto{\pgfqpoint{3.040158in}{2.998385in}}{\pgfqpoint{3.048058in}{3.001657in}}{\pgfqpoint{3.053882in}{3.007481in}}%
\pgfpathcurveto{\pgfqpoint{3.059706in}{3.013305in}}{\pgfqpoint{3.062978in}{3.021205in}}{\pgfqpoint{3.062978in}{3.029441in}}%
\pgfpathcurveto{\pgfqpoint{3.062978in}{3.037678in}}{\pgfqpoint{3.059706in}{3.045578in}}{\pgfqpoint{3.053882in}{3.051402in}}%
\pgfpathcurveto{\pgfqpoint{3.048058in}{3.057225in}}{\pgfqpoint{3.040158in}{3.060498in}}{\pgfqpoint{3.031922in}{3.060498in}}%
\pgfpathcurveto{\pgfqpoint{3.023686in}{3.060498in}}{\pgfqpoint{3.015786in}{3.057225in}}{\pgfqpoint{3.009962in}{3.051402in}}%
\pgfpathcurveto{\pgfqpoint{3.004138in}{3.045578in}}{\pgfqpoint{3.000865in}{3.037678in}}{\pgfqpoint{3.000865in}{3.029441in}}%
\pgfpathcurveto{\pgfqpoint{3.000865in}{3.021205in}}{\pgfqpoint{3.004138in}{3.013305in}}{\pgfqpoint{3.009962in}{3.007481in}}%
\pgfpathcurveto{\pgfqpoint{3.015786in}{3.001657in}}{\pgfqpoint{3.023686in}{2.998385in}}{\pgfqpoint{3.031922in}{2.998385in}}%
\pgfpathclose%
\pgfusepath{stroke,fill}%
\end{pgfscope}%
\begin{pgfscope}%
\pgfpathrectangle{\pgfqpoint{0.100000in}{0.220728in}}{\pgfqpoint{3.696000in}{3.696000in}}%
\pgfusepath{clip}%
\pgfsetbuttcap%
\pgfsetroundjoin%
\definecolor{currentfill}{rgb}{0.121569,0.466667,0.705882}%
\pgfsetfillcolor{currentfill}%
\pgfsetfillopacity{0.594191}%
\pgfsetlinewidth{1.003750pt}%
\definecolor{currentstroke}{rgb}{0.121569,0.466667,0.705882}%
\pgfsetstrokecolor{currentstroke}%
\pgfsetstrokeopacity{0.594191}%
\pgfsetdash{}{0pt}%
\pgfpathmoveto{\pgfqpoint{3.037869in}{2.997729in}}%
\pgfpathcurveto{\pgfqpoint{3.046106in}{2.997729in}}{\pgfqpoint{3.054006in}{3.001001in}}{\pgfqpoint{3.059830in}{3.006825in}}%
\pgfpathcurveto{\pgfqpoint{3.065654in}{3.012649in}}{\pgfqpoint{3.068926in}{3.020549in}}{\pgfqpoint{3.068926in}{3.028785in}}%
\pgfpathcurveto{\pgfqpoint{3.068926in}{3.037022in}}{\pgfqpoint{3.065654in}{3.044922in}}{\pgfqpoint{3.059830in}{3.050746in}}%
\pgfpathcurveto{\pgfqpoint{3.054006in}{3.056570in}}{\pgfqpoint{3.046106in}{3.059842in}}{\pgfqpoint{3.037869in}{3.059842in}}%
\pgfpathcurveto{\pgfqpoint{3.029633in}{3.059842in}}{\pgfqpoint{3.021733in}{3.056570in}}{\pgfqpoint{3.015909in}{3.050746in}}%
\pgfpathcurveto{\pgfqpoint{3.010085in}{3.044922in}}{\pgfqpoint{3.006813in}{3.037022in}}{\pgfqpoint{3.006813in}{3.028785in}}%
\pgfpathcurveto{\pgfqpoint{3.006813in}{3.020549in}}{\pgfqpoint{3.010085in}{3.012649in}}{\pgfqpoint{3.015909in}{3.006825in}}%
\pgfpathcurveto{\pgfqpoint{3.021733in}{3.001001in}}{\pgfqpoint{3.029633in}{2.997729in}}{\pgfqpoint{3.037869in}{2.997729in}}%
\pgfpathclose%
\pgfusepath{stroke,fill}%
\end{pgfscope}%
\begin{pgfscope}%
\pgfpathrectangle{\pgfqpoint{0.100000in}{0.220728in}}{\pgfqpoint{3.696000in}{3.696000in}}%
\pgfusepath{clip}%
\pgfsetbuttcap%
\pgfsetroundjoin%
\definecolor{currentfill}{rgb}{0.121569,0.466667,0.705882}%
\pgfsetfillcolor{currentfill}%
\pgfsetfillopacity{0.595149}%
\pgfsetlinewidth{1.003750pt}%
\definecolor{currentstroke}{rgb}{0.121569,0.466667,0.705882}%
\pgfsetstrokecolor{currentstroke}%
\pgfsetstrokeopacity{0.595149}%
\pgfsetdash{}{0pt}%
\pgfpathmoveto{\pgfqpoint{0.884125in}{1.357543in}}%
\pgfpathcurveto{\pgfqpoint{0.892361in}{1.357543in}}{\pgfqpoint{0.900261in}{1.360816in}}{\pgfqpoint{0.906085in}{1.366640in}}%
\pgfpathcurveto{\pgfqpoint{0.911909in}{1.372463in}}{\pgfqpoint{0.915181in}{1.380364in}}{\pgfqpoint{0.915181in}{1.388600in}}%
\pgfpathcurveto{\pgfqpoint{0.915181in}{1.396836in}}{\pgfqpoint{0.911909in}{1.404736in}}{\pgfqpoint{0.906085in}{1.410560in}}%
\pgfpathcurveto{\pgfqpoint{0.900261in}{1.416384in}}{\pgfqpoint{0.892361in}{1.419656in}}{\pgfqpoint{0.884125in}{1.419656in}}%
\pgfpathcurveto{\pgfqpoint{0.875888in}{1.419656in}}{\pgfqpoint{0.867988in}{1.416384in}}{\pgfqpoint{0.862164in}{1.410560in}}%
\pgfpathcurveto{\pgfqpoint{0.856340in}{1.404736in}}{\pgfqpoint{0.853068in}{1.396836in}}{\pgfqpoint{0.853068in}{1.388600in}}%
\pgfpathcurveto{\pgfqpoint{0.853068in}{1.380364in}}{\pgfqpoint{0.856340in}{1.372463in}}{\pgfqpoint{0.862164in}{1.366640in}}%
\pgfpathcurveto{\pgfqpoint{0.867988in}{1.360816in}}{\pgfqpoint{0.875888in}{1.357543in}}{\pgfqpoint{0.884125in}{1.357543in}}%
\pgfpathclose%
\pgfusepath{stroke,fill}%
\end{pgfscope}%
\begin{pgfscope}%
\pgfpathrectangle{\pgfqpoint{0.100000in}{0.220728in}}{\pgfqpoint{3.696000in}{3.696000in}}%
\pgfusepath{clip}%
\pgfsetbuttcap%
\pgfsetroundjoin%
\definecolor{currentfill}{rgb}{0.121569,0.466667,0.705882}%
\pgfsetfillcolor{currentfill}%
\pgfsetfillopacity{0.595871}%
\pgfsetlinewidth{1.003750pt}%
\definecolor{currentstroke}{rgb}{0.121569,0.466667,0.705882}%
\pgfsetstrokecolor{currentstroke}%
\pgfsetstrokeopacity{0.595871}%
\pgfsetdash{}{0pt}%
\pgfpathmoveto{\pgfqpoint{3.046550in}{2.996714in}}%
\pgfpathcurveto{\pgfqpoint{3.054786in}{2.996714in}}{\pgfqpoint{3.062686in}{2.999986in}}{\pgfqpoint{3.068510in}{3.005810in}}%
\pgfpathcurveto{\pgfqpoint{3.074334in}{3.011634in}}{\pgfqpoint{3.077606in}{3.019534in}}{\pgfqpoint{3.077606in}{3.027770in}}%
\pgfpathcurveto{\pgfqpoint{3.077606in}{3.036007in}}{\pgfqpoint{3.074334in}{3.043907in}}{\pgfqpoint{3.068510in}{3.049730in}}%
\pgfpathcurveto{\pgfqpoint{3.062686in}{3.055554in}}{\pgfqpoint{3.054786in}{3.058827in}}{\pgfqpoint{3.046550in}{3.058827in}}%
\pgfpathcurveto{\pgfqpoint{3.038313in}{3.058827in}}{\pgfqpoint{3.030413in}{3.055554in}}{\pgfqpoint{3.024589in}{3.049730in}}%
\pgfpathcurveto{\pgfqpoint{3.018765in}{3.043907in}}{\pgfqpoint{3.015493in}{3.036007in}}{\pgfqpoint{3.015493in}{3.027770in}}%
\pgfpathcurveto{\pgfqpoint{3.015493in}{3.019534in}}{\pgfqpoint{3.018765in}{3.011634in}}{\pgfqpoint{3.024589in}{3.005810in}}%
\pgfpathcurveto{\pgfqpoint{3.030413in}{2.999986in}}{\pgfqpoint{3.038313in}{2.996714in}}{\pgfqpoint{3.046550in}{2.996714in}}%
\pgfpathclose%
\pgfusepath{stroke,fill}%
\end{pgfscope}%
\begin{pgfscope}%
\pgfpathrectangle{\pgfqpoint{0.100000in}{0.220728in}}{\pgfqpoint{3.696000in}{3.696000in}}%
\pgfusepath{clip}%
\pgfsetbuttcap%
\pgfsetroundjoin%
\definecolor{currentfill}{rgb}{0.121569,0.466667,0.705882}%
\pgfsetfillcolor{currentfill}%
\pgfsetfillopacity{0.596282}%
\pgfsetlinewidth{1.003750pt}%
\definecolor{currentstroke}{rgb}{0.121569,0.466667,0.705882}%
\pgfsetstrokecolor{currentstroke}%
\pgfsetstrokeopacity{0.596282}%
\pgfsetdash{}{0pt}%
\pgfpathmoveto{\pgfqpoint{0.874681in}{1.347650in}}%
\pgfpathcurveto{\pgfqpoint{0.882917in}{1.347650in}}{\pgfqpoint{0.890817in}{1.350922in}}{\pgfqpoint{0.896641in}{1.356746in}}%
\pgfpathcurveto{\pgfqpoint{0.902465in}{1.362570in}}{\pgfqpoint{0.905737in}{1.370470in}}{\pgfqpoint{0.905737in}{1.378706in}}%
\pgfpathcurveto{\pgfqpoint{0.905737in}{1.386942in}}{\pgfqpoint{0.902465in}{1.394842in}}{\pgfqpoint{0.896641in}{1.400666in}}%
\pgfpathcurveto{\pgfqpoint{0.890817in}{1.406490in}}{\pgfqpoint{0.882917in}{1.409763in}}{\pgfqpoint{0.874681in}{1.409763in}}%
\pgfpathcurveto{\pgfqpoint{0.866444in}{1.409763in}}{\pgfqpoint{0.858544in}{1.406490in}}{\pgfqpoint{0.852720in}{1.400666in}}%
\pgfpathcurveto{\pgfqpoint{0.846896in}{1.394842in}}{\pgfqpoint{0.843624in}{1.386942in}}{\pgfqpoint{0.843624in}{1.378706in}}%
\pgfpathcurveto{\pgfqpoint{0.843624in}{1.370470in}}{\pgfqpoint{0.846896in}{1.362570in}}{\pgfqpoint{0.852720in}{1.356746in}}%
\pgfpathcurveto{\pgfqpoint{0.858544in}{1.350922in}}{\pgfqpoint{0.866444in}{1.347650in}}{\pgfqpoint{0.874681in}{1.347650in}}%
\pgfpathclose%
\pgfusepath{stroke,fill}%
\end{pgfscope}%
\begin{pgfscope}%
\pgfpathrectangle{\pgfqpoint{0.100000in}{0.220728in}}{\pgfqpoint{3.696000in}{3.696000in}}%
\pgfusepath{clip}%
\pgfsetbuttcap%
\pgfsetroundjoin%
\definecolor{currentfill}{rgb}{0.121569,0.466667,0.705882}%
\pgfsetfillcolor{currentfill}%
\pgfsetfillopacity{0.596916}%
\pgfsetlinewidth{1.003750pt}%
\definecolor{currentstroke}{rgb}{0.121569,0.466667,0.705882}%
\pgfsetstrokecolor{currentstroke}%
\pgfsetstrokeopacity{0.596916}%
\pgfsetdash{}{0pt}%
\pgfpathmoveto{\pgfqpoint{3.050964in}{2.995359in}}%
\pgfpathcurveto{\pgfqpoint{3.059200in}{2.995359in}}{\pgfqpoint{3.067101in}{2.998631in}}{\pgfqpoint{3.072924in}{3.004455in}}%
\pgfpathcurveto{\pgfqpoint{3.078748in}{3.010279in}}{\pgfqpoint{3.082021in}{3.018179in}}{\pgfqpoint{3.082021in}{3.026415in}}%
\pgfpathcurveto{\pgfqpoint{3.082021in}{3.034652in}}{\pgfqpoint{3.078748in}{3.042552in}}{\pgfqpoint{3.072924in}{3.048376in}}%
\pgfpathcurveto{\pgfqpoint{3.067101in}{3.054199in}}{\pgfqpoint{3.059200in}{3.057472in}}{\pgfqpoint{3.050964in}{3.057472in}}%
\pgfpathcurveto{\pgfqpoint{3.042728in}{3.057472in}}{\pgfqpoint{3.034828in}{3.054199in}}{\pgfqpoint{3.029004in}{3.048376in}}%
\pgfpathcurveto{\pgfqpoint{3.023180in}{3.042552in}}{\pgfqpoint{3.019908in}{3.034652in}}{\pgfqpoint{3.019908in}{3.026415in}}%
\pgfpathcurveto{\pgfqpoint{3.019908in}{3.018179in}}{\pgfqpoint{3.023180in}{3.010279in}}{\pgfqpoint{3.029004in}{3.004455in}}%
\pgfpathcurveto{\pgfqpoint{3.034828in}{2.998631in}}{\pgfqpoint{3.042728in}{2.995359in}}{\pgfqpoint{3.050964in}{2.995359in}}%
\pgfpathclose%
\pgfusepath{stroke,fill}%
\end{pgfscope}%
\begin{pgfscope}%
\pgfpathrectangle{\pgfqpoint{0.100000in}{0.220728in}}{\pgfqpoint{3.696000in}{3.696000in}}%
\pgfusepath{clip}%
\pgfsetbuttcap%
\pgfsetroundjoin%
\definecolor{currentfill}{rgb}{0.121569,0.466667,0.705882}%
\pgfsetfillcolor{currentfill}%
\pgfsetfillopacity{0.598119}%
\pgfsetlinewidth{1.003750pt}%
\definecolor{currentstroke}{rgb}{0.121569,0.466667,0.705882}%
\pgfsetstrokecolor{currentstroke}%
\pgfsetstrokeopacity{0.598119}%
\pgfsetdash{}{0pt}%
\pgfpathmoveto{\pgfqpoint{0.872846in}{1.336943in}}%
\pgfpathcurveto{\pgfqpoint{0.881082in}{1.336943in}}{\pgfqpoint{0.888982in}{1.340216in}}{\pgfqpoint{0.894806in}{1.346040in}}%
\pgfpathcurveto{\pgfqpoint{0.900630in}{1.351864in}}{\pgfqpoint{0.903902in}{1.359764in}}{\pgfqpoint{0.903902in}{1.368000in}}%
\pgfpathcurveto{\pgfqpoint{0.903902in}{1.376236in}}{\pgfqpoint{0.900630in}{1.384136in}}{\pgfqpoint{0.894806in}{1.389960in}}%
\pgfpathcurveto{\pgfqpoint{0.888982in}{1.395784in}}{\pgfqpoint{0.881082in}{1.399056in}}{\pgfqpoint{0.872846in}{1.399056in}}%
\pgfpathcurveto{\pgfqpoint{0.864610in}{1.399056in}}{\pgfqpoint{0.856710in}{1.395784in}}{\pgfqpoint{0.850886in}{1.389960in}}%
\pgfpathcurveto{\pgfqpoint{0.845062in}{1.384136in}}{\pgfqpoint{0.841789in}{1.376236in}}{\pgfqpoint{0.841789in}{1.368000in}}%
\pgfpathcurveto{\pgfqpoint{0.841789in}{1.359764in}}{\pgfqpoint{0.845062in}{1.351864in}}{\pgfqpoint{0.850886in}{1.346040in}}%
\pgfpathcurveto{\pgfqpoint{0.856710in}{1.340216in}}{\pgfqpoint{0.864610in}{1.336943in}}{\pgfqpoint{0.872846in}{1.336943in}}%
\pgfpathclose%
\pgfusepath{stroke,fill}%
\end{pgfscope}%
\begin{pgfscope}%
\pgfpathrectangle{\pgfqpoint{0.100000in}{0.220728in}}{\pgfqpoint{3.696000in}{3.696000in}}%
\pgfusepath{clip}%
\pgfsetbuttcap%
\pgfsetroundjoin%
\definecolor{currentfill}{rgb}{0.121569,0.466667,0.705882}%
\pgfsetfillcolor{currentfill}%
\pgfsetfillopacity{0.598233}%
\pgfsetlinewidth{1.003750pt}%
\definecolor{currentstroke}{rgb}{0.121569,0.466667,0.705882}%
\pgfsetstrokecolor{currentstroke}%
\pgfsetstrokeopacity{0.598233}%
\pgfsetdash{}{0pt}%
\pgfpathmoveto{\pgfqpoint{3.056151in}{2.994422in}}%
\pgfpathcurveto{\pgfqpoint{3.064388in}{2.994422in}}{\pgfqpoint{3.072288in}{2.997695in}}{\pgfqpoint{3.078112in}{3.003519in}}%
\pgfpathcurveto{\pgfqpoint{3.083936in}{3.009343in}}{\pgfqpoint{3.087208in}{3.017243in}}{\pgfqpoint{3.087208in}{3.025479in}}%
\pgfpathcurveto{\pgfqpoint{3.087208in}{3.033715in}}{\pgfqpoint{3.083936in}{3.041615in}}{\pgfqpoint{3.078112in}{3.047439in}}%
\pgfpathcurveto{\pgfqpoint{3.072288in}{3.053263in}}{\pgfqpoint{3.064388in}{3.056535in}}{\pgfqpoint{3.056151in}{3.056535in}}%
\pgfpathcurveto{\pgfqpoint{3.047915in}{3.056535in}}{\pgfqpoint{3.040015in}{3.053263in}}{\pgfqpoint{3.034191in}{3.047439in}}%
\pgfpathcurveto{\pgfqpoint{3.028367in}{3.041615in}}{\pgfqpoint{3.025095in}{3.033715in}}{\pgfqpoint{3.025095in}{3.025479in}}%
\pgfpathcurveto{\pgfqpoint{3.025095in}{3.017243in}}{\pgfqpoint{3.028367in}{3.009343in}}{\pgfqpoint{3.034191in}{3.003519in}}%
\pgfpathcurveto{\pgfqpoint{3.040015in}{2.997695in}}{\pgfqpoint{3.047915in}{2.994422in}}{\pgfqpoint{3.056151in}{2.994422in}}%
\pgfpathclose%
\pgfusepath{stroke,fill}%
\end{pgfscope}%
\begin{pgfscope}%
\pgfpathrectangle{\pgfqpoint{0.100000in}{0.220728in}}{\pgfqpoint{3.696000in}{3.696000in}}%
\pgfusepath{clip}%
\pgfsetbuttcap%
\pgfsetroundjoin%
\definecolor{currentfill}{rgb}{0.121569,0.466667,0.705882}%
\pgfsetfillcolor{currentfill}%
\pgfsetfillopacity{0.598805}%
\pgfsetlinewidth{1.003750pt}%
\definecolor{currentstroke}{rgb}{0.121569,0.466667,0.705882}%
\pgfsetstrokecolor{currentstroke}%
\pgfsetstrokeopacity{0.598805}%
\pgfsetdash{}{0pt}%
\pgfpathmoveto{\pgfqpoint{0.867366in}{1.330559in}}%
\pgfpathcurveto{\pgfqpoint{0.875603in}{1.330559in}}{\pgfqpoint{0.883503in}{1.333831in}}{\pgfqpoint{0.889327in}{1.339655in}}%
\pgfpathcurveto{\pgfqpoint{0.895151in}{1.345479in}}{\pgfqpoint{0.898423in}{1.353379in}}{\pgfqpoint{0.898423in}{1.361615in}}%
\pgfpathcurveto{\pgfqpoint{0.898423in}{1.369851in}}{\pgfqpoint{0.895151in}{1.377751in}}{\pgfqpoint{0.889327in}{1.383575in}}%
\pgfpathcurveto{\pgfqpoint{0.883503in}{1.389399in}}{\pgfqpoint{0.875603in}{1.392672in}}{\pgfqpoint{0.867366in}{1.392672in}}%
\pgfpathcurveto{\pgfqpoint{0.859130in}{1.392672in}}{\pgfqpoint{0.851230in}{1.389399in}}{\pgfqpoint{0.845406in}{1.383575in}}%
\pgfpathcurveto{\pgfqpoint{0.839582in}{1.377751in}}{\pgfqpoint{0.836310in}{1.369851in}}{\pgfqpoint{0.836310in}{1.361615in}}%
\pgfpathcurveto{\pgfqpoint{0.836310in}{1.353379in}}{\pgfqpoint{0.839582in}{1.345479in}}{\pgfqpoint{0.845406in}{1.339655in}}%
\pgfpathcurveto{\pgfqpoint{0.851230in}{1.333831in}}{\pgfqpoint{0.859130in}{1.330559in}}{\pgfqpoint{0.867366in}{1.330559in}}%
\pgfpathclose%
\pgfusepath{stroke,fill}%
\end{pgfscope}%
\begin{pgfscope}%
\pgfpathrectangle{\pgfqpoint{0.100000in}{0.220728in}}{\pgfqpoint{3.696000in}{3.696000in}}%
\pgfusepath{clip}%
\pgfsetbuttcap%
\pgfsetroundjoin%
\definecolor{currentfill}{rgb}{0.121569,0.466667,0.705882}%
\pgfsetfillcolor{currentfill}%
\pgfsetfillopacity{0.598918}%
\pgfsetlinewidth{1.003750pt}%
\definecolor{currentstroke}{rgb}{0.121569,0.466667,0.705882}%
\pgfsetstrokecolor{currentstroke}%
\pgfsetstrokeopacity{0.598918}%
\pgfsetdash{}{0pt}%
\pgfpathmoveto{\pgfqpoint{3.059078in}{2.993992in}}%
\pgfpathcurveto{\pgfqpoint{3.067314in}{2.993992in}}{\pgfqpoint{3.075214in}{2.997265in}}{\pgfqpoint{3.081038in}{3.003089in}}%
\pgfpathcurveto{\pgfqpoint{3.086862in}{3.008912in}}{\pgfqpoint{3.090135in}{3.016813in}}{\pgfqpoint{3.090135in}{3.025049in}}%
\pgfpathcurveto{\pgfqpoint{3.090135in}{3.033285in}}{\pgfqpoint{3.086862in}{3.041185in}}{\pgfqpoint{3.081038in}{3.047009in}}%
\pgfpathcurveto{\pgfqpoint{3.075214in}{3.052833in}}{\pgfqpoint{3.067314in}{3.056105in}}{\pgfqpoint{3.059078in}{3.056105in}}%
\pgfpathcurveto{\pgfqpoint{3.050842in}{3.056105in}}{\pgfqpoint{3.042942in}{3.052833in}}{\pgfqpoint{3.037118in}{3.047009in}}%
\pgfpathcurveto{\pgfqpoint{3.031294in}{3.041185in}}{\pgfqpoint{3.028022in}{3.033285in}}{\pgfqpoint{3.028022in}{3.025049in}}%
\pgfpathcurveto{\pgfqpoint{3.028022in}{3.016813in}}{\pgfqpoint{3.031294in}{3.008912in}}{\pgfqpoint{3.037118in}{3.003089in}}%
\pgfpathcurveto{\pgfqpoint{3.042942in}{2.997265in}}{\pgfqpoint{3.050842in}{2.993992in}}{\pgfqpoint{3.059078in}{2.993992in}}%
\pgfpathclose%
\pgfusepath{stroke,fill}%
\end{pgfscope}%
\begin{pgfscope}%
\pgfpathrectangle{\pgfqpoint{0.100000in}{0.220728in}}{\pgfqpoint{3.696000in}{3.696000in}}%
\pgfusepath{clip}%
\pgfsetbuttcap%
\pgfsetroundjoin%
\definecolor{currentfill}{rgb}{0.121569,0.466667,0.705882}%
\pgfsetfillcolor{currentfill}%
\pgfsetfillopacity{0.599894}%
\pgfsetlinewidth{1.003750pt}%
\definecolor{currentstroke}{rgb}{0.121569,0.466667,0.705882}%
\pgfsetstrokecolor{currentstroke}%
\pgfsetstrokeopacity{0.599894}%
\pgfsetdash{}{0pt}%
\pgfpathmoveto{\pgfqpoint{0.865697in}{1.324499in}}%
\pgfpathcurveto{\pgfqpoint{0.873934in}{1.324499in}}{\pgfqpoint{0.881834in}{1.327771in}}{\pgfqpoint{0.887658in}{1.333595in}}%
\pgfpathcurveto{\pgfqpoint{0.893482in}{1.339419in}}{\pgfqpoint{0.896754in}{1.347319in}}{\pgfqpoint{0.896754in}{1.355555in}}%
\pgfpathcurveto{\pgfqpoint{0.896754in}{1.363791in}}{\pgfqpoint{0.893482in}{1.371691in}}{\pgfqpoint{0.887658in}{1.377515in}}%
\pgfpathcurveto{\pgfqpoint{0.881834in}{1.383339in}}{\pgfqpoint{0.873934in}{1.386612in}}{\pgfqpoint{0.865697in}{1.386612in}}%
\pgfpathcurveto{\pgfqpoint{0.857461in}{1.386612in}}{\pgfqpoint{0.849561in}{1.383339in}}{\pgfqpoint{0.843737in}{1.377515in}}%
\pgfpathcurveto{\pgfqpoint{0.837913in}{1.371691in}}{\pgfqpoint{0.834641in}{1.363791in}}{\pgfqpoint{0.834641in}{1.355555in}}%
\pgfpathcurveto{\pgfqpoint{0.834641in}{1.347319in}}{\pgfqpoint{0.837913in}{1.339419in}}{\pgfqpoint{0.843737in}{1.333595in}}%
\pgfpathcurveto{\pgfqpoint{0.849561in}{1.327771in}}{\pgfqpoint{0.857461in}{1.324499in}}{\pgfqpoint{0.865697in}{1.324499in}}%
\pgfpathclose%
\pgfusepath{stroke,fill}%
\end{pgfscope}%
\begin{pgfscope}%
\pgfpathrectangle{\pgfqpoint{0.100000in}{0.220728in}}{\pgfqpoint{3.696000in}{3.696000in}}%
\pgfusepath{clip}%
\pgfsetbuttcap%
\pgfsetroundjoin%
\definecolor{currentfill}{rgb}{0.121569,0.466667,0.705882}%
\pgfsetfillcolor{currentfill}%
\pgfsetfillopacity{0.600011}%
\pgfsetlinewidth{1.003750pt}%
\definecolor{currentstroke}{rgb}{0.121569,0.466667,0.705882}%
\pgfsetstrokecolor{currentstroke}%
\pgfsetstrokeopacity{0.600011}%
\pgfsetdash{}{0pt}%
\pgfpathmoveto{\pgfqpoint{3.064006in}{2.993176in}}%
\pgfpathcurveto{\pgfqpoint{3.072242in}{2.993176in}}{\pgfqpoint{3.080142in}{2.996448in}}{\pgfqpoint{3.085966in}{3.002272in}}%
\pgfpathcurveto{\pgfqpoint{3.091790in}{3.008096in}}{\pgfqpoint{3.095063in}{3.015996in}}{\pgfqpoint{3.095063in}{3.024233in}}%
\pgfpathcurveto{\pgfqpoint{3.095063in}{3.032469in}}{\pgfqpoint{3.091790in}{3.040369in}}{\pgfqpoint{3.085966in}{3.046193in}}%
\pgfpathcurveto{\pgfqpoint{3.080142in}{3.052017in}}{\pgfqpoint{3.072242in}{3.055289in}}{\pgfqpoint{3.064006in}{3.055289in}}%
\pgfpathcurveto{\pgfqpoint{3.055770in}{3.055289in}}{\pgfqpoint{3.047870in}{3.052017in}}{\pgfqpoint{3.042046in}{3.046193in}}%
\pgfpathcurveto{\pgfqpoint{3.036222in}{3.040369in}}{\pgfqpoint{3.032950in}{3.032469in}}{\pgfqpoint{3.032950in}{3.024233in}}%
\pgfpathcurveto{\pgfqpoint{3.032950in}{3.015996in}}{\pgfqpoint{3.036222in}{3.008096in}}{\pgfqpoint{3.042046in}{3.002272in}}%
\pgfpathcurveto{\pgfqpoint{3.047870in}{2.996448in}}{\pgfqpoint{3.055770in}{2.993176in}}{\pgfqpoint{3.064006in}{2.993176in}}%
\pgfpathclose%
\pgfusepath{stroke,fill}%
\end{pgfscope}%
\begin{pgfscope}%
\pgfpathrectangle{\pgfqpoint{0.100000in}{0.220728in}}{\pgfqpoint{3.696000in}{3.696000in}}%
\pgfusepath{clip}%
\pgfsetbuttcap%
\pgfsetroundjoin%
\definecolor{currentfill}{rgb}{0.121569,0.466667,0.705882}%
\pgfsetfillcolor{currentfill}%
\pgfsetfillopacity{0.600322}%
\pgfsetlinewidth{1.003750pt}%
\definecolor{currentstroke}{rgb}{0.121569,0.466667,0.705882}%
\pgfsetstrokecolor{currentstroke}%
\pgfsetstrokeopacity{0.600322}%
\pgfsetdash{}{0pt}%
\pgfpathmoveto{\pgfqpoint{0.863323in}{1.321333in}}%
\pgfpathcurveto{\pgfqpoint{0.871560in}{1.321333in}}{\pgfqpoint{0.879460in}{1.324605in}}{\pgfqpoint{0.885283in}{1.330429in}}%
\pgfpathcurveto{\pgfqpoint{0.891107in}{1.336253in}}{\pgfqpoint{0.894380in}{1.344153in}}{\pgfqpoint{0.894380in}{1.352389in}}%
\pgfpathcurveto{\pgfqpoint{0.894380in}{1.360626in}}{\pgfqpoint{0.891107in}{1.368526in}}{\pgfqpoint{0.885283in}{1.374350in}}%
\pgfpathcurveto{\pgfqpoint{0.879460in}{1.380173in}}{\pgfqpoint{0.871560in}{1.383446in}}{\pgfqpoint{0.863323in}{1.383446in}}%
\pgfpathcurveto{\pgfqpoint{0.855087in}{1.383446in}}{\pgfqpoint{0.847187in}{1.380173in}}{\pgfqpoint{0.841363in}{1.374350in}}%
\pgfpathcurveto{\pgfqpoint{0.835539in}{1.368526in}}{\pgfqpoint{0.832267in}{1.360626in}}{\pgfqpoint{0.832267in}{1.352389in}}%
\pgfpathcurveto{\pgfqpoint{0.832267in}{1.344153in}}{\pgfqpoint{0.835539in}{1.336253in}}{\pgfqpoint{0.841363in}{1.330429in}}%
\pgfpathcurveto{\pgfqpoint{0.847187in}{1.324605in}}{\pgfqpoint{0.855087in}{1.321333in}}{\pgfqpoint{0.863323in}{1.321333in}}%
\pgfpathclose%
\pgfusepath{stroke,fill}%
\end{pgfscope}%
\begin{pgfscope}%
\pgfpathrectangle{\pgfqpoint{0.100000in}{0.220728in}}{\pgfqpoint{3.696000in}{3.696000in}}%
\pgfusepath{clip}%
\pgfsetbuttcap%
\pgfsetroundjoin%
\definecolor{currentfill}{rgb}{0.121569,0.466667,0.705882}%
\pgfsetfillcolor{currentfill}%
\pgfsetfillopacity{0.601352}%
\pgfsetlinewidth{1.003750pt}%
\definecolor{currentstroke}{rgb}{0.121569,0.466667,0.705882}%
\pgfsetstrokecolor{currentstroke}%
\pgfsetstrokeopacity{0.601352}%
\pgfsetdash{}{0pt}%
\pgfpathmoveto{\pgfqpoint{3.070712in}{2.991878in}}%
\pgfpathcurveto{\pgfqpoint{3.078949in}{2.991878in}}{\pgfqpoint{3.086849in}{2.995151in}}{\pgfqpoint{3.092673in}{3.000975in}}%
\pgfpathcurveto{\pgfqpoint{3.098496in}{3.006799in}}{\pgfqpoint{3.101769in}{3.014699in}}{\pgfqpoint{3.101769in}{3.022935in}}%
\pgfpathcurveto{\pgfqpoint{3.101769in}{3.031171in}}{\pgfqpoint{3.098496in}{3.039071in}}{\pgfqpoint{3.092673in}{3.044895in}}%
\pgfpathcurveto{\pgfqpoint{3.086849in}{3.050719in}}{\pgfqpoint{3.078949in}{3.053991in}}{\pgfqpoint{3.070712in}{3.053991in}}%
\pgfpathcurveto{\pgfqpoint{3.062476in}{3.053991in}}{\pgfqpoint{3.054576in}{3.050719in}}{\pgfqpoint{3.048752in}{3.044895in}}%
\pgfpathcurveto{\pgfqpoint{3.042928in}{3.039071in}}{\pgfqpoint{3.039656in}{3.031171in}}{\pgfqpoint{3.039656in}{3.022935in}}%
\pgfpathcurveto{\pgfqpoint{3.039656in}{3.014699in}}{\pgfqpoint{3.042928in}{3.006799in}}{\pgfqpoint{3.048752in}{3.000975in}}%
\pgfpathcurveto{\pgfqpoint{3.054576in}{2.995151in}}{\pgfqpoint{3.062476in}{2.991878in}}{\pgfqpoint{3.070712in}{2.991878in}}%
\pgfpathclose%
\pgfusepath{stroke,fill}%
\end{pgfscope}%
\begin{pgfscope}%
\pgfpathrectangle{\pgfqpoint{0.100000in}{0.220728in}}{\pgfqpoint{3.696000in}{3.696000in}}%
\pgfusepath{clip}%
\pgfsetbuttcap%
\pgfsetroundjoin%
\definecolor{currentfill}{rgb}{0.121569,0.466667,0.705882}%
\pgfsetfillcolor{currentfill}%
\pgfsetfillopacity{0.601518}%
\pgfsetlinewidth{1.003750pt}%
\definecolor{currentstroke}{rgb}{0.121569,0.466667,0.705882}%
\pgfsetstrokecolor{currentstroke}%
\pgfsetstrokeopacity{0.601518}%
\pgfsetdash{}{0pt}%
\pgfpathmoveto{\pgfqpoint{0.860923in}{1.314718in}}%
\pgfpathcurveto{\pgfqpoint{0.869160in}{1.314718in}}{\pgfqpoint{0.877060in}{1.317990in}}{\pgfqpoint{0.882884in}{1.323814in}}%
\pgfpathcurveto{\pgfqpoint{0.888708in}{1.329638in}}{\pgfqpoint{0.891980in}{1.337538in}}{\pgfqpoint{0.891980in}{1.345774in}}%
\pgfpathcurveto{\pgfqpoint{0.891980in}{1.354011in}}{\pgfqpoint{0.888708in}{1.361911in}}{\pgfqpoint{0.882884in}{1.367735in}}%
\pgfpathcurveto{\pgfqpoint{0.877060in}{1.373558in}}{\pgfqpoint{0.869160in}{1.376831in}}{\pgfqpoint{0.860923in}{1.376831in}}%
\pgfpathcurveto{\pgfqpoint{0.852687in}{1.376831in}}{\pgfqpoint{0.844787in}{1.373558in}}{\pgfqpoint{0.838963in}{1.367735in}}%
\pgfpathcurveto{\pgfqpoint{0.833139in}{1.361911in}}{\pgfqpoint{0.829867in}{1.354011in}}{\pgfqpoint{0.829867in}{1.345774in}}%
\pgfpathcurveto{\pgfqpoint{0.829867in}{1.337538in}}{\pgfqpoint{0.833139in}{1.329638in}}{\pgfqpoint{0.838963in}{1.323814in}}%
\pgfpathcurveto{\pgfqpoint{0.844787in}{1.317990in}}{\pgfqpoint{0.852687in}{1.314718in}}{\pgfqpoint{0.860923in}{1.314718in}}%
\pgfpathclose%
\pgfusepath{stroke,fill}%
\end{pgfscope}%
\begin{pgfscope}%
\pgfpathrectangle{\pgfqpoint{0.100000in}{0.220728in}}{\pgfqpoint{3.696000in}{3.696000in}}%
\pgfusepath{clip}%
\pgfsetbuttcap%
\pgfsetroundjoin%
\definecolor{currentfill}{rgb}{0.121569,0.466667,0.705882}%
\pgfsetfillcolor{currentfill}%
\pgfsetfillopacity{0.602101}%
\pgfsetlinewidth{1.003750pt}%
\definecolor{currentstroke}{rgb}{0.121569,0.466667,0.705882}%
\pgfsetstrokecolor{currentstroke}%
\pgfsetstrokeopacity{0.602101}%
\pgfsetdash{}{0pt}%
\pgfpathmoveto{\pgfqpoint{0.857407in}{1.310084in}}%
\pgfpathcurveto{\pgfqpoint{0.865643in}{1.310084in}}{\pgfqpoint{0.873543in}{1.313357in}}{\pgfqpoint{0.879367in}{1.319181in}}%
\pgfpathcurveto{\pgfqpoint{0.885191in}{1.325005in}}{\pgfqpoint{0.888463in}{1.332905in}}{\pgfqpoint{0.888463in}{1.341141in}}%
\pgfpathcurveto{\pgfqpoint{0.888463in}{1.349377in}}{\pgfqpoint{0.885191in}{1.357277in}}{\pgfqpoint{0.879367in}{1.363101in}}%
\pgfpathcurveto{\pgfqpoint{0.873543in}{1.368925in}}{\pgfqpoint{0.865643in}{1.372197in}}{\pgfqpoint{0.857407in}{1.372197in}}%
\pgfpathcurveto{\pgfqpoint{0.849170in}{1.372197in}}{\pgfqpoint{0.841270in}{1.368925in}}{\pgfqpoint{0.835446in}{1.363101in}}%
\pgfpathcurveto{\pgfqpoint{0.829622in}{1.357277in}}{\pgfqpoint{0.826350in}{1.349377in}}{\pgfqpoint{0.826350in}{1.341141in}}%
\pgfpathcurveto{\pgfqpoint{0.826350in}{1.332905in}}{\pgfqpoint{0.829622in}{1.325005in}}{\pgfqpoint{0.835446in}{1.319181in}}%
\pgfpathcurveto{\pgfqpoint{0.841270in}{1.313357in}}{\pgfqpoint{0.849170in}{1.310084in}}{\pgfqpoint{0.857407in}{1.310084in}}%
\pgfpathclose%
\pgfusepath{stroke,fill}%
\end{pgfscope}%
\begin{pgfscope}%
\pgfpathrectangle{\pgfqpoint{0.100000in}{0.220728in}}{\pgfqpoint{3.696000in}{3.696000in}}%
\pgfusepath{clip}%
\pgfsetbuttcap%
\pgfsetroundjoin%
\definecolor{currentfill}{rgb}{0.121569,0.466667,0.705882}%
\pgfsetfillcolor{currentfill}%
\pgfsetfillopacity{0.602250}%
\pgfsetlinewidth{1.003750pt}%
\definecolor{currentstroke}{rgb}{0.121569,0.466667,0.705882}%
\pgfsetstrokecolor{currentstroke}%
\pgfsetstrokeopacity{0.602250}%
\pgfsetdash{}{0pt}%
\pgfpathmoveto{\pgfqpoint{3.079067in}{2.989890in}}%
\pgfpathcurveto{\pgfqpoint{3.087303in}{2.989890in}}{\pgfqpoint{3.095203in}{2.993162in}}{\pgfqpoint{3.101027in}{2.998986in}}%
\pgfpathcurveto{\pgfqpoint{3.106851in}{3.004810in}}{\pgfqpoint{3.110123in}{3.012710in}}{\pgfqpoint{3.110123in}{3.020947in}}%
\pgfpathcurveto{\pgfqpoint{3.110123in}{3.029183in}}{\pgfqpoint{3.106851in}{3.037083in}}{\pgfqpoint{3.101027in}{3.042907in}}%
\pgfpathcurveto{\pgfqpoint{3.095203in}{3.048731in}}{\pgfqpoint{3.087303in}{3.052003in}}{\pgfqpoint{3.079067in}{3.052003in}}%
\pgfpathcurveto{\pgfqpoint{3.070830in}{3.052003in}}{\pgfqpoint{3.062930in}{3.048731in}}{\pgfqpoint{3.057106in}{3.042907in}}%
\pgfpathcurveto{\pgfqpoint{3.051282in}{3.037083in}}{\pgfqpoint{3.048010in}{3.029183in}}{\pgfqpoint{3.048010in}{3.020947in}}%
\pgfpathcurveto{\pgfqpoint{3.048010in}{3.012710in}}{\pgfqpoint{3.051282in}{3.004810in}}{\pgfqpoint{3.057106in}{2.998986in}}%
\pgfpathcurveto{\pgfqpoint{3.062930in}{2.993162in}}{\pgfqpoint{3.070830in}{2.989890in}}{\pgfqpoint{3.079067in}{2.989890in}}%
\pgfpathclose%
\pgfusepath{stroke,fill}%
\end{pgfscope}%
\begin{pgfscope}%
\pgfpathrectangle{\pgfqpoint{0.100000in}{0.220728in}}{\pgfqpoint{3.696000in}{3.696000in}}%
\pgfusepath{clip}%
\pgfsetbuttcap%
\pgfsetroundjoin%
\definecolor{currentfill}{rgb}{0.121569,0.466667,0.705882}%
\pgfsetfillcolor{currentfill}%
\pgfsetfillopacity{0.602713}%
\pgfsetlinewidth{1.003750pt}%
\definecolor{currentstroke}{rgb}{0.121569,0.466667,0.705882}%
\pgfsetstrokecolor{currentstroke}%
\pgfsetstrokeopacity{0.602713}%
\pgfsetdash{}{0pt}%
\pgfpathmoveto{\pgfqpoint{0.856433in}{1.306826in}}%
\pgfpathcurveto{\pgfqpoint{0.864669in}{1.306826in}}{\pgfqpoint{0.872569in}{1.310098in}}{\pgfqpoint{0.878393in}{1.315922in}}%
\pgfpathcurveto{\pgfqpoint{0.884217in}{1.321746in}}{\pgfqpoint{0.887489in}{1.329646in}}{\pgfqpoint{0.887489in}{1.337882in}}%
\pgfpathcurveto{\pgfqpoint{0.887489in}{1.346118in}}{\pgfqpoint{0.884217in}{1.354018in}}{\pgfqpoint{0.878393in}{1.359842in}}%
\pgfpathcurveto{\pgfqpoint{0.872569in}{1.365666in}}{\pgfqpoint{0.864669in}{1.368939in}}{\pgfqpoint{0.856433in}{1.368939in}}%
\pgfpathcurveto{\pgfqpoint{0.848197in}{1.368939in}}{\pgfqpoint{0.840297in}{1.365666in}}{\pgfqpoint{0.834473in}{1.359842in}}%
\pgfpathcurveto{\pgfqpoint{0.828649in}{1.354018in}}{\pgfqpoint{0.825376in}{1.346118in}}{\pgfqpoint{0.825376in}{1.337882in}}%
\pgfpathcurveto{\pgfqpoint{0.825376in}{1.329646in}}{\pgfqpoint{0.828649in}{1.321746in}}{\pgfqpoint{0.834473in}{1.315922in}}%
\pgfpathcurveto{\pgfqpoint{0.840297in}{1.310098in}}{\pgfqpoint{0.848197in}{1.306826in}}{\pgfqpoint{0.856433in}{1.306826in}}%
\pgfpathclose%
\pgfusepath{stroke,fill}%
\end{pgfscope}%
\begin{pgfscope}%
\pgfpathrectangle{\pgfqpoint{0.100000in}{0.220728in}}{\pgfqpoint{3.696000in}{3.696000in}}%
\pgfusepath{clip}%
\pgfsetbuttcap%
\pgfsetroundjoin%
\definecolor{currentfill}{rgb}{0.121569,0.466667,0.705882}%
\pgfsetfillcolor{currentfill}%
\pgfsetfillopacity{0.602915}%
\pgfsetlinewidth{1.003750pt}%
\definecolor{currentstroke}{rgb}{0.121569,0.466667,0.705882}%
\pgfsetstrokecolor{currentstroke}%
\pgfsetstrokeopacity{0.602915}%
\pgfsetdash{}{0pt}%
\pgfpathmoveto{\pgfqpoint{0.855221in}{1.305253in}}%
\pgfpathcurveto{\pgfqpoint{0.863457in}{1.305253in}}{\pgfqpoint{0.871357in}{1.308525in}}{\pgfqpoint{0.877181in}{1.314349in}}%
\pgfpathcurveto{\pgfqpoint{0.883005in}{1.320173in}}{\pgfqpoint{0.886278in}{1.328073in}}{\pgfqpoint{0.886278in}{1.336310in}}%
\pgfpathcurveto{\pgfqpoint{0.886278in}{1.344546in}}{\pgfqpoint{0.883005in}{1.352446in}}{\pgfqpoint{0.877181in}{1.358270in}}%
\pgfpathcurveto{\pgfqpoint{0.871357in}{1.364094in}}{\pgfqpoint{0.863457in}{1.367366in}}{\pgfqpoint{0.855221in}{1.367366in}}%
\pgfpathcurveto{\pgfqpoint{0.846985in}{1.367366in}}{\pgfqpoint{0.839085in}{1.364094in}}{\pgfqpoint{0.833261in}{1.358270in}}%
\pgfpathcurveto{\pgfqpoint{0.827437in}{1.352446in}}{\pgfqpoint{0.824165in}{1.344546in}}{\pgfqpoint{0.824165in}{1.336310in}}%
\pgfpathcurveto{\pgfqpoint{0.824165in}{1.328073in}}{\pgfqpoint{0.827437in}{1.320173in}}{\pgfqpoint{0.833261in}{1.314349in}}%
\pgfpathcurveto{\pgfqpoint{0.839085in}{1.308525in}}{\pgfqpoint{0.846985in}{1.305253in}}{\pgfqpoint{0.855221in}{1.305253in}}%
\pgfpathclose%
\pgfusepath{stroke,fill}%
\end{pgfscope}%
\begin{pgfscope}%
\pgfpathrectangle{\pgfqpoint{0.100000in}{0.220728in}}{\pgfqpoint{3.696000in}{3.696000in}}%
\pgfusepath{clip}%
\pgfsetbuttcap%
\pgfsetroundjoin%
\definecolor{currentfill}{rgb}{0.121569,0.466667,0.705882}%
\pgfsetfillcolor{currentfill}%
\pgfsetfillopacity{0.603547}%
\pgfsetlinewidth{1.003750pt}%
\definecolor{currentstroke}{rgb}{0.121569,0.466667,0.705882}%
\pgfsetstrokecolor{currentstroke}%
\pgfsetstrokeopacity{0.603547}%
\pgfsetdash{}{0pt}%
\pgfpathmoveto{\pgfqpoint{0.854333in}{1.301896in}}%
\pgfpathcurveto{\pgfqpoint{0.862569in}{1.301896in}}{\pgfqpoint{0.870469in}{1.305168in}}{\pgfqpoint{0.876293in}{1.310992in}}%
\pgfpathcurveto{\pgfqpoint{0.882117in}{1.316816in}}{\pgfqpoint{0.885389in}{1.324716in}}{\pgfqpoint{0.885389in}{1.332952in}}%
\pgfpathcurveto{\pgfqpoint{0.885389in}{1.341188in}}{\pgfqpoint{0.882117in}{1.349088in}}{\pgfqpoint{0.876293in}{1.354912in}}%
\pgfpathcurveto{\pgfqpoint{0.870469in}{1.360736in}}{\pgfqpoint{0.862569in}{1.364009in}}{\pgfqpoint{0.854333in}{1.364009in}}%
\pgfpathcurveto{\pgfqpoint{0.846097in}{1.364009in}}{\pgfqpoint{0.838197in}{1.360736in}}{\pgfqpoint{0.832373in}{1.354912in}}%
\pgfpathcurveto{\pgfqpoint{0.826549in}{1.349088in}}{\pgfqpoint{0.823276in}{1.341188in}}{\pgfqpoint{0.823276in}{1.332952in}}%
\pgfpathcurveto{\pgfqpoint{0.823276in}{1.324716in}}{\pgfqpoint{0.826549in}{1.316816in}}{\pgfqpoint{0.832373in}{1.310992in}}%
\pgfpathcurveto{\pgfqpoint{0.838197in}{1.305168in}}{\pgfqpoint{0.846097in}{1.301896in}}{\pgfqpoint{0.854333in}{1.301896in}}%
\pgfpathclose%
\pgfusepath{stroke,fill}%
\end{pgfscope}%
\begin{pgfscope}%
\pgfpathrectangle{\pgfqpoint{0.100000in}{0.220728in}}{\pgfqpoint{3.696000in}{3.696000in}}%
\pgfusepath{clip}%
\pgfsetbuttcap%
\pgfsetroundjoin%
\definecolor{currentfill}{rgb}{0.121569,0.466667,0.705882}%
\pgfsetfillcolor{currentfill}%
\pgfsetfillopacity{0.603806}%
\pgfsetlinewidth{1.003750pt}%
\definecolor{currentstroke}{rgb}{0.121569,0.466667,0.705882}%
\pgfsetstrokecolor{currentstroke}%
\pgfsetstrokeopacity{0.603806}%
\pgfsetdash{}{0pt}%
\pgfpathmoveto{\pgfqpoint{0.853196in}{1.300135in}}%
\pgfpathcurveto{\pgfqpoint{0.861432in}{1.300135in}}{\pgfqpoint{0.869332in}{1.303407in}}{\pgfqpoint{0.875156in}{1.309231in}}%
\pgfpathcurveto{\pgfqpoint{0.880980in}{1.315055in}}{\pgfqpoint{0.884252in}{1.322955in}}{\pgfqpoint{0.884252in}{1.331191in}}%
\pgfpathcurveto{\pgfqpoint{0.884252in}{1.339428in}}{\pgfqpoint{0.880980in}{1.347328in}}{\pgfqpoint{0.875156in}{1.353152in}}%
\pgfpathcurveto{\pgfqpoint{0.869332in}{1.358976in}}{\pgfqpoint{0.861432in}{1.362248in}}{\pgfqpoint{0.853196in}{1.362248in}}%
\pgfpathcurveto{\pgfqpoint{0.844960in}{1.362248in}}{\pgfqpoint{0.837060in}{1.358976in}}{\pgfqpoint{0.831236in}{1.353152in}}%
\pgfpathcurveto{\pgfqpoint{0.825412in}{1.347328in}}{\pgfqpoint{0.822139in}{1.339428in}}{\pgfqpoint{0.822139in}{1.331191in}}%
\pgfpathcurveto{\pgfqpoint{0.822139in}{1.322955in}}{\pgfqpoint{0.825412in}{1.315055in}}{\pgfqpoint{0.831236in}{1.309231in}}%
\pgfpathcurveto{\pgfqpoint{0.837060in}{1.303407in}}{\pgfqpoint{0.844960in}{1.300135in}}{\pgfqpoint{0.853196in}{1.300135in}}%
\pgfpathclose%
\pgfusepath{stroke,fill}%
\end{pgfscope}%
\begin{pgfscope}%
\pgfpathrectangle{\pgfqpoint{0.100000in}{0.220728in}}{\pgfqpoint{3.696000in}{3.696000in}}%
\pgfusepath{clip}%
\pgfsetbuttcap%
\pgfsetroundjoin%
\definecolor{currentfill}{rgb}{0.121569,0.466667,0.705882}%
\pgfsetfillcolor{currentfill}%
\pgfsetfillopacity{0.604396}%
\pgfsetlinewidth{1.003750pt}%
\definecolor{currentstroke}{rgb}{0.121569,0.466667,0.705882}%
\pgfsetstrokecolor{currentstroke}%
\pgfsetstrokeopacity{0.604396}%
\pgfsetdash{}{0pt}%
\pgfpathmoveto{\pgfqpoint{0.852610in}{1.296175in}}%
\pgfpathcurveto{\pgfqpoint{0.860846in}{1.296175in}}{\pgfqpoint{0.868746in}{1.299448in}}{\pgfqpoint{0.874570in}{1.305271in}}%
\pgfpathcurveto{\pgfqpoint{0.880394in}{1.311095in}}{\pgfqpoint{0.883666in}{1.318995in}}{\pgfqpoint{0.883666in}{1.327232in}}%
\pgfpathcurveto{\pgfqpoint{0.883666in}{1.335468in}}{\pgfqpoint{0.880394in}{1.343368in}}{\pgfqpoint{0.874570in}{1.349192in}}%
\pgfpathcurveto{\pgfqpoint{0.868746in}{1.355016in}}{\pgfqpoint{0.860846in}{1.358288in}}{\pgfqpoint{0.852610in}{1.358288in}}%
\pgfpathcurveto{\pgfqpoint{0.844373in}{1.358288in}}{\pgfqpoint{0.836473in}{1.355016in}}{\pgfqpoint{0.830649in}{1.349192in}}%
\pgfpathcurveto{\pgfqpoint{0.824825in}{1.343368in}}{\pgfqpoint{0.821553in}{1.335468in}}{\pgfqpoint{0.821553in}{1.327232in}}%
\pgfpathcurveto{\pgfqpoint{0.821553in}{1.318995in}}{\pgfqpoint{0.824825in}{1.311095in}}{\pgfqpoint{0.830649in}{1.305271in}}%
\pgfpathcurveto{\pgfqpoint{0.836473in}{1.299448in}}{\pgfqpoint{0.844373in}{1.296175in}}{\pgfqpoint{0.852610in}{1.296175in}}%
\pgfpathclose%
\pgfusepath{stroke,fill}%
\end{pgfscope}%
\begin{pgfscope}%
\pgfpathrectangle{\pgfqpoint{0.100000in}{0.220728in}}{\pgfqpoint{3.696000in}{3.696000in}}%
\pgfusepath{clip}%
\pgfsetbuttcap%
\pgfsetroundjoin%
\definecolor{currentfill}{rgb}{0.121569,0.466667,0.705882}%
\pgfsetfillcolor{currentfill}%
\pgfsetfillopacity{0.604861}%
\pgfsetlinewidth{1.003750pt}%
\definecolor{currentstroke}{rgb}{0.121569,0.466667,0.705882}%
\pgfsetstrokecolor{currentstroke}%
\pgfsetstrokeopacity{0.604861}%
\pgfsetdash{}{0pt}%
\pgfpathmoveto{\pgfqpoint{3.087994in}{2.989761in}}%
\pgfpathcurveto{\pgfqpoint{3.096230in}{2.989761in}}{\pgfqpoint{3.104130in}{2.993033in}}{\pgfqpoint{3.109954in}{2.998857in}}%
\pgfpathcurveto{\pgfqpoint{3.115778in}{3.004681in}}{\pgfqpoint{3.119050in}{3.012581in}}{\pgfqpoint{3.119050in}{3.020817in}}%
\pgfpathcurveto{\pgfqpoint{3.119050in}{3.029054in}}{\pgfqpoint{3.115778in}{3.036954in}}{\pgfqpoint{3.109954in}{3.042778in}}%
\pgfpathcurveto{\pgfqpoint{3.104130in}{3.048602in}}{\pgfqpoint{3.096230in}{3.051874in}}{\pgfqpoint{3.087994in}{3.051874in}}%
\pgfpathcurveto{\pgfqpoint{3.079758in}{3.051874in}}{\pgfqpoint{3.071858in}{3.048602in}}{\pgfqpoint{3.066034in}{3.042778in}}%
\pgfpathcurveto{\pgfqpoint{3.060210in}{3.036954in}}{\pgfqpoint{3.056937in}{3.029054in}}{\pgfqpoint{3.056937in}{3.020817in}}%
\pgfpathcurveto{\pgfqpoint{3.056937in}{3.012581in}}{\pgfqpoint{3.060210in}{3.004681in}}{\pgfqpoint{3.066034in}{2.998857in}}%
\pgfpathcurveto{\pgfqpoint{3.071858in}{2.993033in}}{\pgfqpoint{3.079758in}{2.989761in}}{\pgfqpoint{3.087994in}{2.989761in}}%
\pgfpathclose%
\pgfusepath{stroke,fill}%
\end{pgfscope}%
\begin{pgfscope}%
\pgfpathrectangle{\pgfqpoint{0.100000in}{0.220728in}}{\pgfqpoint{3.696000in}{3.696000in}}%
\pgfusepath{clip}%
\pgfsetbuttcap%
\pgfsetroundjoin%
\definecolor{currentfill}{rgb}{0.121569,0.466667,0.705882}%
\pgfsetfillcolor{currentfill}%
\pgfsetfillopacity{0.604898}%
\pgfsetlinewidth{1.003750pt}%
\definecolor{currentstroke}{rgb}{0.121569,0.466667,0.705882}%
\pgfsetstrokecolor{currentstroke}%
\pgfsetstrokeopacity{0.604898}%
\pgfsetdash{}{0pt}%
\pgfpathmoveto{\pgfqpoint{0.848291in}{1.290123in}}%
\pgfpathcurveto{\pgfqpoint{0.856528in}{1.290123in}}{\pgfqpoint{0.864428in}{1.293396in}}{\pgfqpoint{0.870252in}{1.299220in}}%
\pgfpathcurveto{\pgfqpoint{0.876076in}{1.305044in}}{\pgfqpoint{0.879348in}{1.312944in}}{\pgfqpoint{0.879348in}{1.321180in}}%
\pgfpathcurveto{\pgfqpoint{0.879348in}{1.329416in}}{\pgfqpoint{0.876076in}{1.337316in}}{\pgfqpoint{0.870252in}{1.343140in}}%
\pgfpathcurveto{\pgfqpoint{0.864428in}{1.348964in}}{\pgfqpoint{0.856528in}{1.352236in}}{\pgfqpoint{0.848291in}{1.352236in}}%
\pgfpathcurveto{\pgfqpoint{0.840055in}{1.352236in}}{\pgfqpoint{0.832155in}{1.348964in}}{\pgfqpoint{0.826331in}{1.343140in}}%
\pgfpathcurveto{\pgfqpoint{0.820507in}{1.337316in}}{\pgfqpoint{0.817235in}{1.329416in}}{\pgfqpoint{0.817235in}{1.321180in}}%
\pgfpathcurveto{\pgfqpoint{0.817235in}{1.312944in}}{\pgfqpoint{0.820507in}{1.305044in}}{\pgfqpoint{0.826331in}{1.299220in}}%
\pgfpathcurveto{\pgfqpoint{0.832155in}{1.293396in}}{\pgfqpoint{0.840055in}{1.290123in}}{\pgfqpoint{0.848291in}{1.290123in}}%
\pgfpathclose%
\pgfusepath{stroke,fill}%
\end{pgfscope}%
\begin{pgfscope}%
\pgfpathrectangle{\pgfqpoint{0.100000in}{0.220728in}}{\pgfqpoint{3.696000in}{3.696000in}}%
\pgfusepath{clip}%
\pgfsetbuttcap%
\pgfsetroundjoin%
\definecolor{currentfill}{rgb}{0.121569,0.466667,0.705882}%
\pgfsetfillcolor{currentfill}%
\pgfsetfillopacity{0.605446}%
\pgfsetlinewidth{1.003750pt}%
\definecolor{currentstroke}{rgb}{0.121569,0.466667,0.705882}%
\pgfsetstrokecolor{currentstroke}%
\pgfsetstrokeopacity{0.605446}%
\pgfsetdash{}{0pt}%
\pgfpathmoveto{\pgfqpoint{0.847395in}{1.286116in}}%
\pgfpathcurveto{\pgfqpoint{0.855631in}{1.286116in}}{\pgfqpoint{0.863531in}{1.289389in}}{\pgfqpoint{0.869355in}{1.295213in}}%
\pgfpathcurveto{\pgfqpoint{0.875179in}{1.301036in}}{\pgfqpoint{0.878451in}{1.308937in}}{\pgfqpoint{0.878451in}{1.317173in}}%
\pgfpathcurveto{\pgfqpoint{0.878451in}{1.325409in}}{\pgfqpoint{0.875179in}{1.333309in}}{\pgfqpoint{0.869355in}{1.339133in}}%
\pgfpathcurveto{\pgfqpoint{0.863531in}{1.344957in}}{\pgfqpoint{0.855631in}{1.348229in}}{\pgfqpoint{0.847395in}{1.348229in}}%
\pgfpathcurveto{\pgfqpoint{0.839158in}{1.348229in}}{\pgfqpoint{0.831258in}{1.344957in}}{\pgfqpoint{0.825434in}{1.339133in}}%
\pgfpathcurveto{\pgfqpoint{0.819610in}{1.333309in}}{\pgfqpoint{0.816338in}{1.325409in}}{\pgfqpoint{0.816338in}{1.317173in}}%
\pgfpathcurveto{\pgfqpoint{0.816338in}{1.308937in}}{\pgfqpoint{0.819610in}{1.301036in}}{\pgfqpoint{0.825434in}{1.295213in}}%
\pgfpathcurveto{\pgfqpoint{0.831258in}{1.289389in}}{\pgfqpoint{0.839158in}{1.286116in}}{\pgfqpoint{0.847395in}{1.286116in}}%
\pgfpathclose%
\pgfusepath{stroke,fill}%
\end{pgfscope}%
\begin{pgfscope}%
\pgfpathrectangle{\pgfqpoint{0.100000in}{0.220728in}}{\pgfqpoint{3.696000in}{3.696000in}}%
\pgfusepath{clip}%
\pgfsetbuttcap%
\pgfsetroundjoin%
\definecolor{currentfill}{rgb}{0.121569,0.466667,0.705882}%
\pgfsetfillcolor{currentfill}%
\pgfsetfillopacity{0.605588}%
\pgfsetlinewidth{1.003750pt}%
\definecolor{currentstroke}{rgb}{0.121569,0.466667,0.705882}%
\pgfsetstrokecolor{currentstroke}%
\pgfsetstrokeopacity{0.605588}%
\pgfsetdash{}{0pt}%
\pgfpathmoveto{\pgfqpoint{0.846964in}{1.285221in}}%
\pgfpathcurveto{\pgfqpoint{0.855200in}{1.285221in}}{\pgfqpoint{0.863100in}{1.288493in}}{\pgfqpoint{0.868924in}{1.294317in}}%
\pgfpathcurveto{\pgfqpoint{0.874748in}{1.300141in}}{\pgfqpoint{0.878020in}{1.308041in}}{\pgfqpoint{0.878020in}{1.316277in}}%
\pgfpathcurveto{\pgfqpoint{0.878020in}{1.324514in}}{\pgfqpoint{0.874748in}{1.332414in}}{\pgfqpoint{0.868924in}{1.338238in}}%
\pgfpathcurveto{\pgfqpoint{0.863100in}{1.344062in}}{\pgfqpoint{0.855200in}{1.347334in}}{\pgfqpoint{0.846964in}{1.347334in}}%
\pgfpathcurveto{\pgfqpoint{0.838727in}{1.347334in}}{\pgfqpoint{0.830827in}{1.344062in}}{\pgfqpoint{0.825003in}{1.338238in}}%
\pgfpathcurveto{\pgfqpoint{0.819180in}{1.332414in}}{\pgfqpoint{0.815907in}{1.324514in}}{\pgfqpoint{0.815907in}{1.316277in}}%
\pgfpathcurveto{\pgfqpoint{0.815907in}{1.308041in}}{\pgfqpoint{0.819180in}{1.300141in}}{\pgfqpoint{0.825003in}{1.294317in}}%
\pgfpathcurveto{\pgfqpoint{0.830827in}{1.288493in}}{\pgfqpoint{0.838727in}{1.285221in}}{\pgfqpoint{0.846964in}{1.285221in}}%
\pgfpathclose%
\pgfusepath{stroke,fill}%
\end{pgfscope}%
\begin{pgfscope}%
\pgfpathrectangle{\pgfqpoint{0.100000in}{0.220728in}}{\pgfqpoint{3.696000in}{3.696000in}}%
\pgfusepath{clip}%
\pgfsetbuttcap%
\pgfsetroundjoin%
\definecolor{currentfill}{rgb}{0.121569,0.466667,0.705882}%
\pgfsetfillcolor{currentfill}%
\pgfsetfillopacity{0.605864}%
\pgfsetlinewidth{1.003750pt}%
\definecolor{currentstroke}{rgb}{0.121569,0.466667,0.705882}%
\pgfsetstrokecolor{currentstroke}%
\pgfsetstrokeopacity{0.605864}%
\pgfsetdash{}{0pt}%
\pgfpathmoveto{\pgfqpoint{0.846124in}{1.283730in}}%
\pgfpathcurveto{\pgfqpoint{0.854361in}{1.283730in}}{\pgfqpoint{0.862261in}{1.287002in}}{\pgfqpoint{0.868085in}{1.292826in}}%
\pgfpathcurveto{\pgfqpoint{0.873909in}{1.298650in}}{\pgfqpoint{0.877181in}{1.306550in}}{\pgfqpoint{0.877181in}{1.314787in}}%
\pgfpathcurveto{\pgfqpoint{0.877181in}{1.323023in}}{\pgfqpoint{0.873909in}{1.330923in}}{\pgfqpoint{0.868085in}{1.336747in}}%
\pgfpathcurveto{\pgfqpoint{0.862261in}{1.342571in}}{\pgfqpoint{0.854361in}{1.345843in}}{\pgfqpoint{0.846124in}{1.345843in}}%
\pgfpathcurveto{\pgfqpoint{0.837888in}{1.345843in}}{\pgfqpoint{0.829988in}{1.342571in}}{\pgfqpoint{0.824164in}{1.336747in}}%
\pgfpathcurveto{\pgfqpoint{0.818340in}{1.330923in}}{\pgfqpoint{0.815068in}{1.323023in}}{\pgfqpoint{0.815068in}{1.314787in}}%
\pgfpathcurveto{\pgfqpoint{0.815068in}{1.306550in}}{\pgfqpoint{0.818340in}{1.298650in}}{\pgfqpoint{0.824164in}{1.292826in}}%
\pgfpathcurveto{\pgfqpoint{0.829988in}{1.287002in}}{\pgfqpoint{0.837888in}{1.283730in}}{\pgfqpoint{0.846124in}{1.283730in}}%
\pgfpathclose%
\pgfusepath{stroke,fill}%
\end{pgfscope}%
\begin{pgfscope}%
\pgfpathrectangle{\pgfqpoint{0.100000in}{0.220728in}}{\pgfqpoint{3.696000in}{3.696000in}}%
\pgfusepath{clip}%
\pgfsetbuttcap%
\pgfsetroundjoin%
\definecolor{currentfill}{rgb}{0.121569,0.466667,0.705882}%
\pgfsetfillcolor{currentfill}%
\pgfsetfillopacity{0.605955}%
\pgfsetlinewidth{1.003750pt}%
\definecolor{currentstroke}{rgb}{0.121569,0.466667,0.705882}%
\pgfsetstrokecolor{currentstroke}%
\pgfsetstrokeopacity{0.605955}%
\pgfsetdash{}{0pt}%
\pgfpathmoveto{\pgfqpoint{0.845850in}{1.283154in}}%
\pgfpathcurveto{\pgfqpoint{0.854086in}{1.283154in}}{\pgfqpoint{0.861986in}{1.286426in}}{\pgfqpoint{0.867810in}{1.292250in}}%
\pgfpathcurveto{\pgfqpoint{0.873634in}{1.298074in}}{\pgfqpoint{0.876907in}{1.305974in}}{\pgfqpoint{0.876907in}{1.314210in}}%
\pgfpathcurveto{\pgfqpoint{0.876907in}{1.322446in}}{\pgfqpoint{0.873634in}{1.330346in}}{\pgfqpoint{0.867810in}{1.336170in}}%
\pgfpathcurveto{\pgfqpoint{0.861986in}{1.341994in}}{\pgfqpoint{0.854086in}{1.345267in}}{\pgfqpoint{0.845850in}{1.345267in}}%
\pgfpathcurveto{\pgfqpoint{0.837614in}{1.345267in}}{\pgfqpoint{0.829714in}{1.341994in}}{\pgfqpoint{0.823890in}{1.336170in}}%
\pgfpathcurveto{\pgfqpoint{0.818066in}{1.330346in}}{\pgfqpoint{0.814794in}{1.322446in}}{\pgfqpoint{0.814794in}{1.314210in}}%
\pgfpathcurveto{\pgfqpoint{0.814794in}{1.305974in}}{\pgfqpoint{0.818066in}{1.298074in}}{\pgfqpoint{0.823890in}{1.292250in}}%
\pgfpathcurveto{\pgfqpoint{0.829714in}{1.286426in}}{\pgfqpoint{0.837614in}{1.283154in}}{\pgfqpoint{0.845850in}{1.283154in}}%
\pgfpathclose%
\pgfusepath{stroke,fill}%
\end{pgfscope}%
\begin{pgfscope}%
\pgfpathrectangle{\pgfqpoint{0.100000in}{0.220728in}}{\pgfqpoint{3.696000in}{3.696000in}}%
\pgfusepath{clip}%
\pgfsetbuttcap%
\pgfsetroundjoin%
\definecolor{currentfill}{rgb}{0.121569,0.466667,0.705882}%
\pgfsetfillcolor{currentfill}%
\pgfsetfillopacity{0.606128}%
\pgfsetlinewidth{1.003750pt}%
\definecolor{currentstroke}{rgb}{0.121569,0.466667,0.705882}%
\pgfsetstrokecolor{currentstroke}%
\pgfsetstrokeopacity{0.606128}%
\pgfsetdash{}{0pt}%
\pgfpathmoveto{\pgfqpoint{0.845396in}{1.282088in}}%
\pgfpathcurveto{\pgfqpoint{0.853632in}{1.282088in}}{\pgfqpoint{0.861532in}{1.285360in}}{\pgfqpoint{0.867356in}{1.291184in}}%
\pgfpathcurveto{\pgfqpoint{0.873180in}{1.297008in}}{\pgfqpoint{0.876452in}{1.304908in}}{\pgfqpoint{0.876452in}{1.313144in}}%
\pgfpathcurveto{\pgfqpoint{0.876452in}{1.321381in}}{\pgfqpoint{0.873180in}{1.329281in}}{\pgfqpoint{0.867356in}{1.335105in}}%
\pgfpathcurveto{\pgfqpoint{0.861532in}{1.340929in}}{\pgfqpoint{0.853632in}{1.344201in}}{\pgfqpoint{0.845396in}{1.344201in}}%
\pgfpathcurveto{\pgfqpoint{0.837160in}{1.344201in}}{\pgfqpoint{0.829260in}{1.340929in}}{\pgfqpoint{0.823436in}{1.335105in}}%
\pgfpathcurveto{\pgfqpoint{0.817612in}{1.329281in}}{\pgfqpoint{0.814339in}{1.321381in}}{\pgfqpoint{0.814339in}{1.313144in}}%
\pgfpathcurveto{\pgfqpoint{0.814339in}{1.304908in}}{\pgfqpoint{0.817612in}{1.297008in}}{\pgfqpoint{0.823436in}{1.291184in}}%
\pgfpathcurveto{\pgfqpoint{0.829260in}{1.285360in}}{\pgfqpoint{0.837160in}{1.282088in}}{\pgfqpoint{0.845396in}{1.282088in}}%
\pgfpathclose%
\pgfusepath{stroke,fill}%
\end{pgfscope}%
\begin{pgfscope}%
\pgfpathrectangle{\pgfqpoint{0.100000in}{0.220728in}}{\pgfqpoint{3.696000in}{3.696000in}}%
\pgfusepath{clip}%
\pgfsetbuttcap%
\pgfsetroundjoin%
\definecolor{currentfill}{rgb}{0.121569,0.466667,0.705882}%
\pgfsetfillcolor{currentfill}%
\pgfsetfillopacity{0.606156}%
\pgfsetlinewidth{1.003750pt}%
\definecolor{currentstroke}{rgb}{0.121569,0.466667,0.705882}%
\pgfsetstrokecolor{currentstroke}%
\pgfsetstrokeopacity{0.606156}%
\pgfsetdash{}{0pt}%
\pgfpathmoveto{\pgfqpoint{0.845300in}{1.281908in}}%
\pgfpathcurveto{\pgfqpoint{0.853537in}{1.281908in}}{\pgfqpoint{0.861437in}{1.285180in}}{\pgfqpoint{0.867261in}{1.291004in}}%
\pgfpathcurveto{\pgfqpoint{0.873085in}{1.296828in}}{\pgfqpoint{0.876357in}{1.304728in}}{\pgfqpoint{0.876357in}{1.312964in}}%
\pgfpathcurveto{\pgfqpoint{0.876357in}{1.321201in}}{\pgfqpoint{0.873085in}{1.329101in}}{\pgfqpoint{0.867261in}{1.334925in}}%
\pgfpathcurveto{\pgfqpoint{0.861437in}{1.340748in}}{\pgfqpoint{0.853537in}{1.344021in}}{\pgfqpoint{0.845300in}{1.344021in}}%
\pgfpathcurveto{\pgfqpoint{0.837064in}{1.344021in}}{\pgfqpoint{0.829164in}{1.340748in}}{\pgfqpoint{0.823340in}{1.334925in}}%
\pgfpathcurveto{\pgfqpoint{0.817516in}{1.329101in}}{\pgfqpoint{0.814244in}{1.321201in}}{\pgfqpoint{0.814244in}{1.312964in}}%
\pgfpathcurveto{\pgfqpoint{0.814244in}{1.304728in}}{\pgfqpoint{0.817516in}{1.296828in}}{\pgfqpoint{0.823340in}{1.291004in}}%
\pgfpathcurveto{\pgfqpoint{0.829164in}{1.285180in}}{\pgfqpoint{0.837064in}{1.281908in}}{\pgfqpoint{0.845300in}{1.281908in}}%
\pgfpathclose%
\pgfusepath{stroke,fill}%
\end{pgfscope}%
\begin{pgfscope}%
\pgfpathrectangle{\pgfqpoint{0.100000in}{0.220728in}}{\pgfqpoint{3.696000in}{3.696000in}}%
\pgfusepath{clip}%
\pgfsetbuttcap%
\pgfsetroundjoin%
\definecolor{currentfill}{rgb}{0.121569,0.466667,0.705882}%
\pgfsetfillcolor{currentfill}%
\pgfsetfillopacity{0.606211}%
\pgfsetlinewidth{1.003750pt}%
\definecolor{currentstroke}{rgb}{0.121569,0.466667,0.705882}%
\pgfsetstrokecolor{currentstroke}%
\pgfsetstrokeopacity{0.606211}%
\pgfsetdash{}{0pt}%
\pgfpathmoveto{\pgfqpoint{0.845142in}{1.281580in}}%
\pgfpathcurveto{\pgfqpoint{0.853378in}{1.281580in}}{\pgfqpoint{0.861278in}{1.284852in}}{\pgfqpoint{0.867102in}{1.290676in}}%
\pgfpathcurveto{\pgfqpoint{0.872926in}{1.296500in}}{\pgfqpoint{0.876199in}{1.304400in}}{\pgfqpoint{0.876199in}{1.312637in}}%
\pgfpathcurveto{\pgfqpoint{0.876199in}{1.320873in}}{\pgfqpoint{0.872926in}{1.328773in}}{\pgfqpoint{0.867102in}{1.334597in}}%
\pgfpathcurveto{\pgfqpoint{0.861278in}{1.340421in}}{\pgfqpoint{0.853378in}{1.343693in}}{\pgfqpoint{0.845142in}{1.343693in}}%
\pgfpathcurveto{\pgfqpoint{0.836906in}{1.343693in}}{\pgfqpoint{0.829006in}{1.340421in}}{\pgfqpoint{0.823182in}{1.334597in}}%
\pgfpathcurveto{\pgfqpoint{0.817358in}{1.328773in}}{\pgfqpoint{0.814086in}{1.320873in}}{\pgfqpoint{0.814086in}{1.312637in}}%
\pgfpathcurveto{\pgfqpoint{0.814086in}{1.304400in}}{\pgfqpoint{0.817358in}{1.296500in}}{\pgfqpoint{0.823182in}{1.290676in}}%
\pgfpathcurveto{\pgfqpoint{0.829006in}{1.284852in}}{\pgfqpoint{0.836906in}{1.281580in}}{\pgfqpoint{0.845142in}{1.281580in}}%
\pgfpathclose%
\pgfusepath{stroke,fill}%
\end{pgfscope}%
\begin{pgfscope}%
\pgfpathrectangle{\pgfqpoint{0.100000in}{0.220728in}}{\pgfqpoint{3.696000in}{3.696000in}}%
\pgfusepath{clip}%
\pgfsetbuttcap%
\pgfsetroundjoin%
\definecolor{currentfill}{rgb}{0.121569,0.466667,0.705882}%
\pgfsetfillcolor{currentfill}%
\pgfsetfillopacity{0.606314}%
\pgfsetlinewidth{1.003750pt}%
\definecolor{currentstroke}{rgb}{0.121569,0.466667,0.705882}%
\pgfsetstrokecolor{currentstroke}%
\pgfsetstrokeopacity{0.606314}%
\pgfsetdash{}{0pt}%
\pgfpathmoveto{\pgfqpoint{0.844921in}{1.280939in}}%
\pgfpathcurveto{\pgfqpoint{0.853157in}{1.280939in}}{\pgfqpoint{0.861057in}{1.284212in}}{\pgfqpoint{0.866881in}{1.290036in}}%
\pgfpathcurveto{\pgfqpoint{0.872705in}{1.295859in}}{\pgfqpoint{0.875978in}{1.303760in}}{\pgfqpoint{0.875978in}{1.311996in}}%
\pgfpathcurveto{\pgfqpoint{0.875978in}{1.320232in}}{\pgfqpoint{0.872705in}{1.328132in}}{\pgfqpoint{0.866881in}{1.333956in}}%
\pgfpathcurveto{\pgfqpoint{0.861057in}{1.339780in}}{\pgfqpoint{0.853157in}{1.343052in}}{\pgfqpoint{0.844921in}{1.343052in}}%
\pgfpathcurveto{\pgfqpoint{0.836685in}{1.343052in}}{\pgfqpoint{0.828785in}{1.339780in}}{\pgfqpoint{0.822961in}{1.333956in}}%
\pgfpathcurveto{\pgfqpoint{0.817137in}{1.328132in}}{\pgfqpoint{0.813865in}{1.320232in}}{\pgfqpoint{0.813865in}{1.311996in}}%
\pgfpathcurveto{\pgfqpoint{0.813865in}{1.303760in}}{\pgfqpoint{0.817137in}{1.295859in}}{\pgfqpoint{0.822961in}{1.290036in}}%
\pgfpathcurveto{\pgfqpoint{0.828785in}{1.284212in}}{\pgfqpoint{0.836685in}{1.280939in}}{\pgfqpoint{0.844921in}{1.280939in}}%
\pgfpathclose%
\pgfusepath{stroke,fill}%
\end{pgfscope}%
\begin{pgfscope}%
\pgfpathrectangle{\pgfqpoint{0.100000in}{0.220728in}}{\pgfqpoint{3.696000in}{3.696000in}}%
\pgfusepath{clip}%
\pgfsetbuttcap%
\pgfsetroundjoin%
\definecolor{currentfill}{rgb}{0.121569,0.466667,0.705882}%
\pgfsetfillcolor{currentfill}%
\pgfsetfillopacity{0.606490}%
\pgfsetlinewidth{1.003750pt}%
\definecolor{currentstroke}{rgb}{0.121569,0.466667,0.705882}%
\pgfsetstrokecolor{currentstroke}%
\pgfsetstrokeopacity{0.606490}%
\pgfsetdash{}{0pt}%
\pgfpathmoveto{\pgfqpoint{0.844373in}{1.279864in}}%
\pgfpathcurveto{\pgfqpoint{0.852610in}{1.279864in}}{\pgfqpoint{0.860510in}{1.283136in}}{\pgfqpoint{0.866334in}{1.288960in}}%
\pgfpathcurveto{\pgfqpoint{0.872157in}{1.294784in}}{\pgfqpoint{0.875430in}{1.302684in}}{\pgfqpoint{0.875430in}{1.310920in}}%
\pgfpathcurveto{\pgfqpoint{0.875430in}{1.319157in}}{\pgfqpoint{0.872157in}{1.327057in}}{\pgfqpoint{0.866334in}{1.332881in}}%
\pgfpathcurveto{\pgfqpoint{0.860510in}{1.338705in}}{\pgfqpoint{0.852610in}{1.341977in}}{\pgfqpoint{0.844373in}{1.341977in}}%
\pgfpathcurveto{\pgfqpoint{0.836137in}{1.341977in}}{\pgfqpoint{0.828237in}{1.338705in}}{\pgfqpoint{0.822413in}{1.332881in}}%
\pgfpathcurveto{\pgfqpoint{0.816589in}{1.327057in}}{\pgfqpoint{0.813317in}{1.319157in}}{\pgfqpoint{0.813317in}{1.310920in}}%
\pgfpathcurveto{\pgfqpoint{0.813317in}{1.302684in}}{\pgfqpoint{0.816589in}{1.294784in}}{\pgfqpoint{0.822413in}{1.288960in}}%
\pgfpathcurveto{\pgfqpoint{0.828237in}{1.283136in}}{\pgfqpoint{0.836137in}{1.279864in}}{\pgfqpoint{0.844373in}{1.279864in}}%
\pgfpathclose%
\pgfusepath{stroke,fill}%
\end{pgfscope}%
\begin{pgfscope}%
\pgfpathrectangle{\pgfqpoint{0.100000in}{0.220728in}}{\pgfqpoint{3.696000in}{3.696000in}}%
\pgfusepath{clip}%
\pgfsetbuttcap%
\pgfsetroundjoin%
\definecolor{currentfill}{rgb}{0.121569,0.466667,0.705882}%
\pgfsetfillcolor{currentfill}%
\pgfsetfillopacity{0.606552}%
\pgfsetlinewidth{1.003750pt}%
\definecolor{currentstroke}{rgb}{0.121569,0.466667,0.705882}%
\pgfsetstrokecolor{currentstroke}%
\pgfsetstrokeopacity{0.606552}%
\pgfsetdash{}{0pt}%
\pgfpathmoveto{\pgfqpoint{0.844200in}{1.279467in}}%
\pgfpathcurveto{\pgfqpoint{0.852437in}{1.279467in}}{\pgfqpoint{0.860337in}{1.282740in}}{\pgfqpoint{0.866161in}{1.288563in}}%
\pgfpathcurveto{\pgfqpoint{0.871984in}{1.294387in}}{\pgfqpoint{0.875257in}{1.302287in}}{\pgfqpoint{0.875257in}{1.310524in}}%
\pgfpathcurveto{\pgfqpoint{0.875257in}{1.318760in}}{\pgfqpoint{0.871984in}{1.326660in}}{\pgfqpoint{0.866161in}{1.332484in}}%
\pgfpathcurveto{\pgfqpoint{0.860337in}{1.338308in}}{\pgfqpoint{0.852437in}{1.341580in}}{\pgfqpoint{0.844200in}{1.341580in}}%
\pgfpathcurveto{\pgfqpoint{0.835964in}{1.341580in}}{\pgfqpoint{0.828064in}{1.338308in}}{\pgfqpoint{0.822240in}{1.332484in}}%
\pgfpathcurveto{\pgfqpoint{0.816416in}{1.326660in}}{\pgfqpoint{0.813144in}{1.318760in}}{\pgfqpoint{0.813144in}{1.310524in}}%
\pgfpathcurveto{\pgfqpoint{0.813144in}{1.302287in}}{\pgfqpoint{0.816416in}{1.294387in}}{\pgfqpoint{0.822240in}{1.288563in}}%
\pgfpathcurveto{\pgfqpoint{0.828064in}{1.282740in}}{\pgfqpoint{0.835964in}{1.279467in}}{\pgfqpoint{0.844200in}{1.279467in}}%
\pgfpathclose%
\pgfusepath{stroke,fill}%
\end{pgfscope}%
\begin{pgfscope}%
\pgfpathrectangle{\pgfqpoint{0.100000in}{0.220728in}}{\pgfqpoint{3.696000in}{3.696000in}}%
\pgfusepath{clip}%
\pgfsetbuttcap%
\pgfsetroundjoin%
\definecolor{currentfill}{rgb}{0.121569,0.466667,0.705882}%
\pgfsetfillcolor{currentfill}%
\pgfsetfillopacity{0.606658}%
\pgfsetlinewidth{1.003750pt}%
\definecolor{currentstroke}{rgb}{0.121569,0.466667,0.705882}%
\pgfsetstrokecolor{currentstroke}%
\pgfsetstrokeopacity{0.606658}%
\pgfsetdash{}{0pt}%
\pgfpathmoveto{\pgfqpoint{0.843912in}{1.278698in}}%
\pgfpathcurveto{\pgfqpoint{0.852148in}{1.278698in}}{\pgfqpoint{0.860048in}{1.281970in}}{\pgfqpoint{0.865872in}{1.287794in}}%
\pgfpathcurveto{\pgfqpoint{0.871696in}{1.293618in}}{\pgfqpoint{0.874969in}{1.301518in}}{\pgfqpoint{0.874969in}{1.309754in}}%
\pgfpathcurveto{\pgfqpoint{0.874969in}{1.317991in}}{\pgfqpoint{0.871696in}{1.325891in}}{\pgfqpoint{0.865872in}{1.331715in}}%
\pgfpathcurveto{\pgfqpoint{0.860048in}{1.337538in}}{\pgfqpoint{0.852148in}{1.340811in}}{\pgfqpoint{0.843912in}{1.340811in}}%
\pgfpathcurveto{\pgfqpoint{0.835676in}{1.340811in}}{\pgfqpoint{0.827776in}{1.337538in}}{\pgfqpoint{0.821952in}{1.331715in}}%
\pgfpathcurveto{\pgfqpoint{0.816128in}{1.325891in}}{\pgfqpoint{0.812856in}{1.317991in}}{\pgfqpoint{0.812856in}{1.309754in}}%
\pgfpathcurveto{\pgfqpoint{0.812856in}{1.301518in}}{\pgfqpoint{0.816128in}{1.293618in}}{\pgfqpoint{0.821952in}{1.287794in}}%
\pgfpathcurveto{\pgfqpoint{0.827776in}{1.281970in}}{\pgfqpoint{0.835676in}{1.278698in}}{\pgfqpoint{0.843912in}{1.278698in}}%
\pgfpathclose%
\pgfusepath{stroke,fill}%
\end{pgfscope}%
\begin{pgfscope}%
\pgfpathrectangle{\pgfqpoint{0.100000in}{0.220728in}}{\pgfqpoint{3.696000in}{3.696000in}}%
\pgfusepath{clip}%
\pgfsetbuttcap%
\pgfsetroundjoin%
\definecolor{currentfill}{rgb}{0.121569,0.466667,0.705882}%
\pgfsetfillcolor{currentfill}%
\pgfsetfillopacity{0.606666}%
\pgfsetlinewidth{1.003750pt}%
\definecolor{currentstroke}{rgb}{0.121569,0.466667,0.705882}%
\pgfsetstrokecolor{currentstroke}%
\pgfsetstrokeopacity{0.606666}%
\pgfsetdash{}{0pt}%
\pgfpathmoveto{\pgfqpoint{0.843893in}{1.278646in}}%
\pgfpathcurveto{\pgfqpoint{0.852129in}{1.278646in}}{\pgfqpoint{0.860029in}{1.281918in}}{\pgfqpoint{0.865853in}{1.287742in}}%
\pgfpathcurveto{\pgfqpoint{0.871677in}{1.293566in}}{\pgfqpoint{0.874949in}{1.301466in}}{\pgfqpoint{0.874949in}{1.309702in}}%
\pgfpathcurveto{\pgfqpoint{0.874949in}{1.317938in}}{\pgfqpoint{0.871677in}{1.325838in}}{\pgfqpoint{0.865853in}{1.331662in}}%
\pgfpathcurveto{\pgfqpoint{0.860029in}{1.337486in}}{\pgfqpoint{0.852129in}{1.340759in}}{\pgfqpoint{0.843893in}{1.340759in}}%
\pgfpathcurveto{\pgfqpoint{0.835656in}{1.340759in}}{\pgfqpoint{0.827756in}{1.337486in}}{\pgfqpoint{0.821932in}{1.331662in}}%
\pgfpathcurveto{\pgfqpoint{0.816109in}{1.325838in}}{\pgfqpoint{0.812836in}{1.317938in}}{\pgfqpoint{0.812836in}{1.309702in}}%
\pgfpathcurveto{\pgfqpoint{0.812836in}{1.301466in}}{\pgfqpoint{0.816109in}{1.293566in}}{\pgfqpoint{0.821932in}{1.287742in}}%
\pgfpathcurveto{\pgfqpoint{0.827756in}{1.281918in}}{\pgfqpoint{0.835656in}{1.278646in}}{\pgfqpoint{0.843893in}{1.278646in}}%
\pgfpathclose%
\pgfusepath{stroke,fill}%
\end{pgfscope}%
\begin{pgfscope}%
\pgfpathrectangle{\pgfqpoint{0.100000in}{0.220728in}}{\pgfqpoint{3.696000in}{3.696000in}}%
\pgfusepath{clip}%
\pgfsetbuttcap%
\pgfsetroundjoin%
\definecolor{currentfill}{rgb}{0.121569,0.466667,0.705882}%
\pgfsetfillcolor{currentfill}%
\pgfsetfillopacity{0.606680}%
\pgfsetlinewidth{1.003750pt}%
\definecolor{currentstroke}{rgb}{0.121569,0.466667,0.705882}%
\pgfsetstrokecolor{currentstroke}%
\pgfsetstrokeopacity{0.606680}%
\pgfsetdash{}{0pt}%
\pgfpathmoveto{\pgfqpoint{0.843857in}{1.278550in}}%
\pgfpathcurveto{\pgfqpoint{0.852093in}{1.278550in}}{\pgfqpoint{0.859993in}{1.281823in}}{\pgfqpoint{0.865817in}{1.287647in}}%
\pgfpathcurveto{\pgfqpoint{0.871641in}{1.293470in}}{\pgfqpoint{0.874914in}{1.301370in}}{\pgfqpoint{0.874914in}{1.309607in}}%
\pgfpathcurveto{\pgfqpoint{0.874914in}{1.317843in}}{\pgfqpoint{0.871641in}{1.325743in}}{\pgfqpoint{0.865817in}{1.331567in}}%
\pgfpathcurveto{\pgfqpoint{0.859993in}{1.337391in}}{\pgfqpoint{0.852093in}{1.340663in}}{\pgfqpoint{0.843857in}{1.340663in}}%
\pgfpathcurveto{\pgfqpoint{0.835621in}{1.340663in}}{\pgfqpoint{0.827721in}{1.337391in}}{\pgfqpoint{0.821897in}{1.331567in}}%
\pgfpathcurveto{\pgfqpoint{0.816073in}{1.325743in}}{\pgfqpoint{0.812801in}{1.317843in}}{\pgfqpoint{0.812801in}{1.309607in}}%
\pgfpathcurveto{\pgfqpoint{0.812801in}{1.301370in}}{\pgfqpoint{0.816073in}{1.293470in}}{\pgfqpoint{0.821897in}{1.287647in}}%
\pgfpathcurveto{\pgfqpoint{0.827721in}{1.281823in}}{\pgfqpoint{0.835621in}{1.278550in}}{\pgfqpoint{0.843857in}{1.278550in}}%
\pgfpathclose%
\pgfusepath{stroke,fill}%
\end{pgfscope}%
\begin{pgfscope}%
\pgfpathrectangle{\pgfqpoint{0.100000in}{0.220728in}}{\pgfqpoint{3.696000in}{3.696000in}}%
\pgfusepath{clip}%
\pgfsetbuttcap%
\pgfsetroundjoin%
\definecolor{currentfill}{rgb}{0.121569,0.466667,0.705882}%
\pgfsetfillcolor{currentfill}%
\pgfsetfillopacity{0.606704}%
\pgfsetlinewidth{1.003750pt}%
\definecolor{currentstroke}{rgb}{0.121569,0.466667,0.705882}%
\pgfsetstrokecolor{currentstroke}%
\pgfsetstrokeopacity{0.606704}%
\pgfsetdash{}{0pt}%
\pgfpathmoveto{\pgfqpoint{0.843787in}{1.278377in}}%
\pgfpathcurveto{\pgfqpoint{0.852023in}{1.278377in}}{\pgfqpoint{0.859923in}{1.281650in}}{\pgfqpoint{0.865747in}{1.287474in}}%
\pgfpathcurveto{\pgfqpoint{0.871571in}{1.293298in}}{\pgfqpoint{0.874843in}{1.301198in}}{\pgfqpoint{0.874843in}{1.309434in}}%
\pgfpathcurveto{\pgfqpoint{0.874843in}{1.317670in}}{\pgfqpoint{0.871571in}{1.325570in}}{\pgfqpoint{0.865747in}{1.331394in}}%
\pgfpathcurveto{\pgfqpoint{0.859923in}{1.337218in}}{\pgfqpoint{0.852023in}{1.340490in}}{\pgfqpoint{0.843787in}{1.340490in}}%
\pgfpathcurveto{\pgfqpoint{0.835551in}{1.340490in}}{\pgfqpoint{0.827650in}{1.337218in}}{\pgfqpoint{0.821827in}{1.331394in}}%
\pgfpathcurveto{\pgfqpoint{0.816003in}{1.325570in}}{\pgfqpoint{0.812730in}{1.317670in}}{\pgfqpoint{0.812730in}{1.309434in}}%
\pgfpathcurveto{\pgfqpoint{0.812730in}{1.301198in}}{\pgfqpoint{0.816003in}{1.293298in}}{\pgfqpoint{0.821827in}{1.287474in}}%
\pgfpathcurveto{\pgfqpoint{0.827650in}{1.281650in}}{\pgfqpoint{0.835551in}{1.278377in}}{\pgfqpoint{0.843787in}{1.278377in}}%
\pgfpathclose%
\pgfusepath{stroke,fill}%
\end{pgfscope}%
\begin{pgfscope}%
\pgfpathrectangle{\pgfqpoint{0.100000in}{0.220728in}}{\pgfqpoint{3.696000in}{3.696000in}}%
\pgfusepath{clip}%
\pgfsetbuttcap%
\pgfsetroundjoin%
\definecolor{currentfill}{rgb}{0.121569,0.466667,0.705882}%
\pgfsetfillcolor{currentfill}%
\pgfsetfillopacity{0.606752}%
\pgfsetlinewidth{1.003750pt}%
\definecolor{currentstroke}{rgb}{0.121569,0.466667,0.705882}%
\pgfsetstrokecolor{currentstroke}%
\pgfsetstrokeopacity{0.606752}%
\pgfsetdash{}{0pt}%
\pgfpathmoveto{\pgfqpoint{0.843674in}{1.278060in}}%
\pgfpathcurveto{\pgfqpoint{0.851910in}{1.278060in}}{\pgfqpoint{0.859810in}{1.281333in}}{\pgfqpoint{0.865634in}{1.287156in}}%
\pgfpathcurveto{\pgfqpoint{0.871458in}{1.292980in}}{\pgfqpoint{0.874731in}{1.300880in}}{\pgfqpoint{0.874731in}{1.309117in}}%
\pgfpathcurveto{\pgfqpoint{0.874731in}{1.317353in}}{\pgfqpoint{0.871458in}{1.325253in}}{\pgfqpoint{0.865634in}{1.331077in}}%
\pgfpathcurveto{\pgfqpoint{0.859810in}{1.336901in}}{\pgfqpoint{0.851910in}{1.340173in}}{\pgfqpoint{0.843674in}{1.340173in}}%
\pgfpathcurveto{\pgfqpoint{0.835438in}{1.340173in}}{\pgfqpoint{0.827538in}{1.336901in}}{\pgfqpoint{0.821714in}{1.331077in}}%
\pgfpathcurveto{\pgfqpoint{0.815890in}{1.325253in}}{\pgfqpoint{0.812618in}{1.317353in}}{\pgfqpoint{0.812618in}{1.309117in}}%
\pgfpathcurveto{\pgfqpoint{0.812618in}{1.300880in}}{\pgfqpoint{0.815890in}{1.292980in}}{\pgfqpoint{0.821714in}{1.287156in}}%
\pgfpathcurveto{\pgfqpoint{0.827538in}{1.281333in}}{\pgfqpoint{0.835438in}{1.278060in}}{\pgfqpoint{0.843674in}{1.278060in}}%
\pgfpathclose%
\pgfusepath{stroke,fill}%
\end{pgfscope}%
\begin{pgfscope}%
\pgfpathrectangle{\pgfqpoint{0.100000in}{0.220728in}}{\pgfqpoint{3.696000in}{3.696000in}}%
\pgfusepath{clip}%
\pgfsetbuttcap%
\pgfsetroundjoin%
\definecolor{currentfill}{rgb}{0.121569,0.466667,0.705882}%
\pgfsetfillcolor{currentfill}%
\pgfsetfillopacity{0.606840}%
\pgfsetlinewidth{1.003750pt}%
\definecolor{currentstroke}{rgb}{0.121569,0.466667,0.705882}%
\pgfsetstrokecolor{currentstroke}%
\pgfsetstrokeopacity{0.606840}%
\pgfsetdash{}{0pt}%
\pgfpathmoveto{\pgfqpoint{0.843464in}{1.277498in}}%
\pgfpathcurveto{\pgfqpoint{0.851700in}{1.277498in}}{\pgfqpoint{0.859600in}{1.280771in}}{\pgfqpoint{0.865424in}{1.286594in}}%
\pgfpathcurveto{\pgfqpoint{0.871248in}{1.292418in}}{\pgfqpoint{0.874520in}{1.300318in}}{\pgfqpoint{0.874520in}{1.308555in}}%
\pgfpathcurveto{\pgfqpoint{0.874520in}{1.316791in}}{\pgfqpoint{0.871248in}{1.324691in}}{\pgfqpoint{0.865424in}{1.330515in}}%
\pgfpathcurveto{\pgfqpoint{0.859600in}{1.336339in}}{\pgfqpoint{0.851700in}{1.339611in}}{\pgfqpoint{0.843464in}{1.339611in}}%
\pgfpathcurveto{\pgfqpoint{0.835227in}{1.339611in}}{\pgfqpoint{0.827327in}{1.336339in}}{\pgfqpoint{0.821503in}{1.330515in}}%
\pgfpathcurveto{\pgfqpoint{0.815679in}{1.324691in}}{\pgfqpoint{0.812407in}{1.316791in}}{\pgfqpoint{0.812407in}{1.308555in}}%
\pgfpathcurveto{\pgfqpoint{0.812407in}{1.300318in}}{\pgfqpoint{0.815679in}{1.292418in}}{\pgfqpoint{0.821503in}{1.286594in}}%
\pgfpathcurveto{\pgfqpoint{0.827327in}{1.280771in}}{\pgfqpoint{0.835227in}{1.277498in}}{\pgfqpoint{0.843464in}{1.277498in}}%
\pgfpathclose%
\pgfusepath{stroke,fill}%
\end{pgfscope}%
\begin{pgfscope}%
\pgfpathrectangle{\pgfqpoint{0.100000in}{0.220728in}}{\pgfqpoint{3.696000in}{3.696000in}}%
\pgfusepath{clip}%
\pgfsetbuttcap%
\pgfsetroundjoin%
\definecolor{currentfill}{rgb}{0.121569,0.466667,0.705882}%
\pgfsetfillcolor{currentfill}%
\pgfsetfillopacity{0.606995}%
\pgfsetlinewidth{1.003750pt}%
\definecolor{currentstroke}{rgb}{0.121569,0.466667,0.705882}%
\pgfsetstrokecolor{currentstroke}%
\pgfsetstrokeopacity{0.606995}%
\pgfsetdash{}{0pt}%
\pgfpathmoveto{\pgfqpoint{0.843054in}{1.276475in}}%
\pgfpathcurveto{\pgfqpoint{0.851290in}{1.276475in}}{\pgfqpoint{0.859190in}{1.279748in}}{\pgfqpoint{0.865014in}{1.285572in}}%
\pgfpathcurveto{\pgfqpoint{0.870838in}{1.291395in}}{\pgfqpoint{0.874110in}{1.299296in}}{\pgfqpoint{0.874110in}{1.307532in}}%
\pgfpathcurveto{\pgfqpoint{0.874110in}{1.315768in}}{\pgfqpoint{0.870838in}{1.323668in}}{\pgfqpoint{0.865014in}{1.329492in}}%
\pgfpathcurveto{\pgfqpoint{0.859190in}{1.335316in}}{\pgfqpoint{0.851290in}{1.338588in}}{\pgfqpoint{0.843054in}{1.338588in}}%
\pgfpathcurveto{\pgfqpoint{0.834817in}{1.338588in}}{\pgfqpoint{0.826917in}{1.335316in}}{\pgfqpoint{0.821093in}{1.329492in}}%
\pgfpathcurveto{\pgfqpoint{0.815270in}{1.323668in}}{\pgfqpoint{0.811997in}{1.315768in}}{\pgfqpoint{0.811997in}{1.307532in}}%
\pgfpathcurveto{\pgfqpoint{0.811997in}{1.299296in}}{\pgfqpoint{0.815270in}{1.291395in}}{\pgfqpoint{0.821093in}{1.285572in}}%
\pgfpathcurveto{\pgfqpoint{0.826917in}{1.279748in}}{\pgfqpoint{0.834817in}{1.276475in}}{\pgfqpoint{0.843054in}{1.276475in}}%
\pgfpathclose%
\pgfusepath{stroke,fill}%
\end{pgfscope}%
\begin{pgfscope}%
\pgfpathrectangle{\pgfqpoint{0.100000in}{0.220728in}}{\pgfqpoint{3.696000in}{3.696000in}}%
\pgfusepath{clip}%
\pgfsetbuttcap%
\pgfsetroundjoin%
\definecolor{currentfill}{rgb}{0.121569,0.466667,0.705882}%
\pgfsetfillcolor{currentfill}%
\pgfsetfillopacity{0.607296}%
\pgfsetlinewidth{1.003750pt}%
\definecolor{currentstroke}{rgb}{0.121569,0.466667,0.705882}%
\pgfsetstrokecolor{currentstroke}%
\pgfsetstrokeopacity{0.607296}%
\pgfsetdash{}{0pt}%
\pgfpathmoveto{\pgfqpoint{0.842226in}{1.274771in}}%
\pgfpathcurveto{\pgfqpoint{0.850463in}{1.274771in}}{\pgfqpoint{0.858363in}{1.278043in}}{\pgfqpoint{0.864187in}{1.283867in}}%
\pgfpathcurveto{\pgfqpoint{0.870010in}{1.289691in}}{\pgfqpoint{0.873283in}{1.297591in}}{\pgfqpoint{0.873283in}{1.305827in}}%
\pgfpathcurveto{\pgfqpoint{0.873283in}{1.314064in}}{\pgfqpoint{0.870010in}{1.321964in}}{\pgfqpoint{0.864187in}{1.327788in}}%
\pgfpathcurveto{\pgfqpoint{0.858363in}{1.333612in}}{\pgfqpoint{0.850463in}{1.336884in}}{\pgfqpoint{0.842226in}{1.336884in}}%
\pgfpathcurveto{\pgfqpoint{0.833990in}{1.336884in}}{\pgfqpoint{0.826090in}{1.333612in}}{\pgfqpoint{0.820266in}{1.327788in}}%
\pgfpathcurveto{\pgfqpoint{0.814442in}{1.321964in}}{\pgfqpoint{0.811170in}{1.314064in}}{\pgfqpoint{0.811170in}{1.305827in}}%
\pgfpathcurveto{\pgfqpoint{0.811170in}{1.297591in}}{\pgfqpoint{0.814442in}{1.289691in}}{\pgfqpoint{0.820266in}{1.283867in}}%
\pgfpathcurveto{\pgfqpoint{0.826090in}{1.278043in}}{\pgfqpoint{0.833990in}{1.274771in}}{\pgfqpoint{0.842226in}{1.274771in}}%
\pgfpathclose%
\pgfusepath{stroke,fill}%
\end{pgfscope}%
\begin{pgfscope}%
\pgfpathrectangle{\pgfqpoint{0.100000in}{0.220728in}}{\pgfqpoint{3.696000in}{3.696000in}}%
\pgfusepath{clip}%
\pgfsetbuttcap%
\pgfsetroundjoin%
\definecolor{currentfill}{rgb}{0.121569,0.466667,0.705882}%
\pgfsetfillcolor{currentfill}%
\pgfsetfillopacity{0.607998}%
\pgfsetlinewidth{1.003750pt}%
\definecolor{currentstroke}{rgb}{0.121569,0.466667,0.705882}%
\pgfsetstrokecolor{currentstroke}%
\pgfsetstrokeopacity{0.607998}%
\pgfsetdash{}{0pt}%
\pgfpathmoveto{\pgfqpoint{0.840618in}{1.272392in}}%
\pgfpathcurveto{\pgfqpoint{0.848854in}{1.272392in}}{\pgfqpoint{0.856754in}{1.275665in}}{\pgfqpoint{0.862578in}{1.281489in}}%
\pgfpathcurveto{\pgfqpoint{0.868402in}{1.287313in}}{\pgfqpoint{0.871674in}{1.295213in}}{\pgfqpoint{0.871674in}{1.303449in}}%
\pgfpathcurveto{\pgfqpoint{0.871674in}{1.311685in}}{\pgfqpoint{0.868402in}{1.319585in}}{\pgfqpoint{0.862578in}{1.325409in}}%
\pgfpathcurveto{\pgfqpoint{0.856754in}{1.331233in}}{\pgfqpoint{0.848854in}{1.334505in}}{\pgfqpoint{0.840618in}{1.334505in}}%
\pgfpathcurveto{\pgfqpoint{0.832381in}{1.334505in}}{\pgfqpoint{0.824481in}{1.331233in}}{\pgfqpoint{0.818657in}{1.325409in}}%
\pgfpathcurveto{\pgfqpoint{0.812833in}{1.319585in}}{\pgfqpoint{0.809561in}{1.311685in}}{\pgfqpoint{0.809561in}{1.303449in}}%
\pgfpathcurveto{\pgfqpoint{0.809561in}{1.295213in}}{\pgfqpoint{0.812833in}{1.287313in}}{\pgfqpoint{0.818657in}{1.281489in}}%
\pgfpathcurveto{\pgfqpoint{0.824481in}{1.275665in}}{\pgfqpoint{0.832381in}{1.272392in}}{\pgfqpoint{0.840618in}{1.272392in}}%
\pgfpathclose%
\pgfusepath{stroke,fill}%
\end{pgfscope}%
\begin{pgfscope}%
\pgfpathrectangle{\pgfqpoint{0.100000in}{0.220728in}}{\pgfqpoint{3.696000in}{3.696000in}}%
\pgfusepath{clip}%
\pgfsetbuttcap%
\pgfsetroundjoin%
\definecolor{currentfill}{rgb}{0.121569,0.466667,0.705882}%
\pgfsetfillcolor{currentfill}%
\pgfsetfillopacity{0.609138}%
\pgfsetlinewidth{1.003750pt}%
\definecolor{currentstroke}{rgb}{0.121569,0.466667,0.705882}%
\pgfsetstrokecolor{currentstroke}%
\pgfsetstrokeopacity{0.609138}%
\pgfsetdash{}{0pt}%
\pgfpathmoveto{\pgfqpoint{0.838435in}{1.266830in}}%
\pgfpathcurveto{\pgfqpoint{0.846671in}{1.266830in}}{\pgfqpoint{0.854571in}{1.270102in}}{\pgfqpoint{0.860395in}{1.275926in}}%
\pgfpathcurveto{\pgfqpoint{0.866219in}{1.281750in}}{\pgfqpoint{0.869491in}{1.289650in}}{\pgfqpoint{0.869491in}{1.297887in}}%
\pgfpathcurveto{\pgfqpoint{0.869491in}{1.306123in}}{\pgfqpoint{0.866219in}{1.314023in}}{\pgfqpoint{0.860395in}{1.319847in}}%
\pgfpathcurveto{\pgfqpoint{0.854571in}{1.325671in}}{\pgfqpoint{0.846671in}{1.328943in}}{\pgfqpoint{0.838435in}{1.328943in}}%
\pgfpathcurveto{\pgfqpoint{0.830198in}{1.328943in}}{\pgfqpoint{0.822298in}{1.325671in}}{\pgfqpoint{0.816474in}{1.319847in}}%
\pgfpathcurveto{\pgfqpoint{0.810651in}{1.314023in}}{\pgfqpoint{0.807378in}{1.306123in}}{\pgfqpoint{0.807378in}{1.297887in}}%
\pgfpathcurveto{\pgfqpoint{0.807378in}{1.289650in}}{\pgfqpoint{0.810651in}{1.281750in}}{\pgfqpoint{0.816474in}{1.275926in}}%
\pgfpathcurveto{\pgfqpoint{0.822298in}{1.270102in}}{\pgfqpoint{0.830198in}{1.266830in}}{\pgfqpoint{0.838435in}{1.266830in}}%
\pgfpathclose%
\pgfusepath{stroke,fill}%
\end{pgfscope}%
\begin{pgfscope}%
\pgfpathrectangle{\pgfqpoint{0.100000in}{0.220728in}}{\pgfqpoint{3.696000in}{3.696000in}}%
\pgfusepath{clip}%
\pgfsetbuttcap%
\pgfsetroundjoin%
\definecolor{currentfill}{rgb}{0.121569,0.466667,0.705882}%
\pgfsetfillcolor{currentfill}%
\pgfsetfillopacity{0.610256}%
\pgfsetlinewidth{1.003750pt}%
\definecolor{currentstroke}{rgb}{0.121569,0.466667,0.705882}%
\pgfsetstrokecolor{currentstroke}%
\pgfsetstrokeopacity{0.610256}%
\pgfsetdash{}{0pt}%
\pgfpathmoveto{\pgfqpoint{3.092393in}{2.986958in}}%
\pgfpathcurveto{\pgfqpoint{3.100629in}{2.986958in}}{\pgfqpoint{3.108529in}{2.990231in}}{\pgfqpoint{3.114353in}{2.996055in}}%
\pgfpathcurveto{\pgfqpoint{3.120177in}{3.001879in}}{\pgfqpoint{3.123449in}{3.009779in}}{\pgfqpoint{3.123449in}{3.018015in}}%
\pgfpathcurveto{\pgfqpoint{3.123449in}{3.026251in}}{\pgfqpoint{3.120177in}{3.034151in}}{\pgfqpoint{3.114353in}{3.039975in}}%
\pgfpathcurveto{\pgfqpoint{3.108529in}{3.045799in}}{\pgfqpoint{3.100629in}{3.049071in}}{\pgfqpoint{3.092393in}{3.049071in}}%
\pgfpathcurveto{\pgfqpoint{3.084157in}{3.049071in}}{\pgfqpoint{3.076257in}{3.045799in}}{\pgfqpoint{3.070433in}{3.039975in}}%
\pgfpathcurveto{\pgfqpoint{3.064609in}{3.034151in}}{\pgfqpoint{3.061336in}{3.026251in}}{\pgfqpoint{3.061336in}{3.018015in}}%
\pgfpathcurveto{\pgfqpoint{3.061336in}{3.009779in}}{\pgfqpoint{3.064609in}{3.001879in}}{\pgfqpoint{3.070433in}{2.996055in}}%
\pgfpathcurveto{\pgfqpoint{3.076257in}{2.990231in}}{\pgfqpoint{3.084157in}{2.986958in}}{\pgfqpoint{3.092393in}{2.986958in}}%
\pgfpathclose%
\pgfusepath{stroke,fill}%
\end{pgfscope}%
\begin{pgfscope}%
\pgfpathrectangle{\pgfqpoint{0.100000in}{0.220728in}}{\pgfqpoint{3.696000in}{3.696000in}}%
\pgfusepath{clip}%
\pgfsetbuttcap%
\pgfsetroundjoin%
\definecolor{currentfill}{rgb}{0.121569,0.466667,0.705882}%
\pgfsetfillcolor{currentfill}%
\pgfsetfillopacity{0.610458}%
\pgfsetlinewidth{1.003750pt}%
\definecolor{currentstroke}{rgb}{0.121569,0.466667,0.705882}%
\pgfsetstrokecolor{currentstroke}%
\pgfsetstrokeopacity{0.610458}%
\pgfsetdash{}{0pt}%
\pgfpathmoveto{\pgfqpoint{0.832914in}{1.255331in}}%
\pgfpathcurveto{\pgfqpoint{0.841150in}{1.255331in}}{\pgfqpoint{0.849050in}{1.258604in}}{\pgfqpoint{0.854874in}{1.264428in}}%
\pgfpathcurveto{\pgfqpoint{0.860698in}{1.270251in}}{\pgfqpoint{0.863970in}{1.278151in}}{\pgfqpoint{0.863970in}{1.286388in}}%
\pgfpathcurveto{\pgfqpoint{0.863970in}{1.294624in}}{\pgfqpoint{0.860698in}{1.302524in}}{\pgfqpoint{0.854874in}{1.308348in}}%
\pgfpathcurveto{\pgfqpoint{0.849050in}{1.314172in}}{\pgfqpoint{0.841150in}{1.317444in}}{\pgfqpoint{0.832914in}{1.317444in}}%
\pgfpathcurveto{\pgfqpoint{0.824677in}{1.317444in}}{\pgfqpoint{0.816777in}{1.314172in}}{\pgfqpoint{0.810953in}{1.308348in}}%
\pgfpathcurveto{\pgfqpoint{0.805130in}{1.302524in}}{\pgfqpoint{0.801857in}{1.294624in}}{\pgfqpoint{0.801857in}{1.286388in}}%
\pgfpathcurveto{\pgfqpoint{0.801857in}{1.278151in}}{\pgfqpoint{0.805130in}{1.270251in}}{\pgfqpoint{0.810953in}{1.264428in}}%
\pgfpathcurveto{\pgfqpoint{0.816777in}{1.258604in}}{\pgfqpoint{0.824677in}{1.255331in}}{\pgfqpoint{0.832914in}{1.255331in}}%
\pgfpathclose%
\pgfusepath{stroke,fill}%
\end{pgfscope}%
\begin{pgfscope}%
\pgfpathrectangle{\pgfqpoint{0.100000in}{0.220728in}}{\pgfqpoint{3.696000in}{3.696000in}}%
\pgfusepath{clip}%
\pgfsetbuttcap%
\pgfsetroundjoin%
\definecolor{currentfill}{rgb}{0.121569,0.466667,0.705882}%
\pgfsetfillcolor{currentfill}%
\pgfsetfillopacity{0.611621}%
\pgfsetlinewidth{1.003750pt}%
\definecolor{currentstroke}{rgb}{0.121569,0.466667,0.705882}%
\pgfsetstrokecolor{currentstroke}%
\pgfsetstrokeopacity{0.611621}%
\pgfsetdash{}{0pt}%
\pgfpathmoveto{\pgfqpoint{3.105696in}{2.982641in}}%
\pgfpathcurveto{\pgfqpoint{3.113932in}{2.982641in}}{\pgfqpoint{3.121832in}{2.985914in}}{\pgfqpoint{3.127656in}{2.991738in}}%
\pgfpathcurveto{\pgfqpoint{3.133480in}{2.997562in}}{\pgfqpoint{3.136752in}{3.005462in}}{\pgfqpoint{3.136752in}{3.013698in}}%
\pgfpathcurveto{\pgfqpoint{3.136752in}{3.021934in}}{\pgfqpoint{3.133480in}{3.029834in}}{\pgfqpoint{3.127656in}{3.035658in}}%
\pgfpathcurveto{\pgfqpoint{3.121832in}{3.041482in}}{\pgfqpoint{3.113932in}{3.044754in}}{\pgfqpoint{3.105696in}{3.044754in}}%
\pgfpathcurveto{\pgfqpoint{3.097460in}{3.044754in}}{\pgfqpoint{3.089560in}{3.041482in}}{\pgfqpoint{3.083736in}{3.035658in}}%
\pgfpathcurveto{\pgfqpoint{3.077912in}{3.029834in}}{\pgfqpoint{3.074639in}{3.021934in}}{\pgfqpoint{3.074639in}{3.013698in}}%
\pgfpathcurveto{\pgfqpoint{3.074639in}{3.005462in}}{\pgfqpoint{3.077912in}{2.997562in}}{\pgfqpoint{3.083736in}{2.991738in}}%
\pgfpathcurveto{\pgfqpoint{3.089560in}{2.985914in}}{\pgfqpoint{3.097460in}{2.982641in}}{\pgfqpoint{3.105696in}{2.982641in}}%
\pgfpathclose%
\pgfusepath{stroke,fill}%
\end{pgfscope}%
\begin{pgfscope}%
\pgfpathrectangle{\pgfqpoint{0.100000in}{0.220728in}}{\pgfqpoint{3.696000in}{3.696000in}}%
\pgfusepath{clip}%
\pgfsetbuttcap%
\pgfsetroundjoin%
\definecolor{currentfill}{rgb}{0.121569,0.466667,0.705882}%
\pgfsetfillcolor{currentfill}%
\pgfsetfillopacity{0.611647}%
\pgfsetlinewidth{1.003750pt}%
\definecolor{currentstroke}{rgb}{0.121569,0.466667,0.705882}%
\pgfsetstrokecolor{currentstroke}%
\pgfsetstrokeopacity{0.611647}%
\pgfsetdash{}{0pt}%
\pgfpathmoveto{\pgfqpoint{0.845071in}{1.230330in}}%
\pgfpathcurveto{\pgfqpoint{0.853307in}{1.230330in}}{\pgfqpoint{0.861207in}{1.233602in}}{\pgfqpoint{0.867031in}{1.239426in}}%
\pgfpathcurveto{\pgfqpoint{0.872855in}{1.245250in}}{\pgfqpoint{0.876128in}{1.253150in}}{\pgfqpoint{0.876128in}{1.261386in}}%
\pgfpathcurveto{\pgfqpoint{0.876128in}{1.269622in}}{\pgfqpoint{0.872855in}{1.277522in}}{\pgfqpoint{0.867031in}{1.283346in}}%
\pgfpathcurveto{\pgfqpoint{0.861207in}{1.289170in}}{\pgfqpoint{0.853307in}{1.292443in}}{\pgfqpoint{0.845071in}{1.292443in}}%
\pgfpathcurveto{\pgfqpoint{0.836835in}{1.292443in}}{\pgfqpoint{0.828935in}{1.289170in}}{\pgfqpoint{0.823111in}{1.283346in}}%
\pgfpathcurveto{\pgfqpoint{0.817287in}{1.277522in}}{\pgfqpoint{0.814015in}{1.269622in}}{\pgfqpoint{0.814015in}{1.261386in}}%
\pgfpathcurveto{\pgfqpoint{0.814015in}{1.253150in}}{\pgfqpoint{0.817287in}{1.245250in}}{\pgfqpoint{0.823111in}{1.239426in}}%
\pgfpathcurveto{\pgfqpoint{0.828935in}{1.233602in}}{\pgfqpoint{0.836835in}{1.230330in}}{\pgfqpoint{0.845071in}{1.230330in}}%
\pgfpathclose%
\pgfusepath{stroke,fill}%
\end{pgfscope}%
\begin{pgfscope}%
\pgfpathrectangle{\pgfqpoint{0.100000in}{0.220728in}}{\pgfqpoint{3.696000in}{3.696000in}}%
\pgfusepath{clip}%
\pgfsetbuttcap%
\pgfsetroundjoin%
\definecolor{currentfill}{rgb}{0.121569,0.466667,0.705882}%
\pgfsetfillcolor{currentfill}%
\pgfsetfillopacity{0.611953}%
\pgfsetlinewidth{1.003750pt}%
\definecolor{currentstroke}{rgb}{0.121569,0.466667,0.705882}%
\pgfsetstrokecolor{currentstroke}%
\pgfsetstrokeopacity{0.611953}%
\pgfsetdash{}{0pt}%
\pgfpathmoveto{\pgfqpoint{0.829351in}{1.245086in}}%
\pgfpathcurveto{\pgfqpoint{0.837588in}{1.245086in}}{\pgfqpoint{0.845488in}{1.248358in}}{\pgfqpoint{0.851312in}{1.254182in}}%
\pgfpathcurveto{\pgfqpoint{0.857135in}{1.260006in}}{\pgfqpoint{0.860408in}{1.267906in}}{\pgfqpoint{0.860408in}{1.276142in}}%
\pgfpathcurveto{\pgfqpoint{0.860408in}{1.284378in}}{\pgfqpoint{0.857135in}{1.292278in}}{\pgfqpoint{0.851312in}{1.298102in}}%
\pgfpathcurveto{\pgfqpoint{0.845488in}{1.303926in}}{\pgfqpoint{0.837588in}{1.307199in}}{\pgfqpoint{0.829351in}{1.307199in}}%
\pgfpathcurveto{\pgfqpoint{0.821115in}{1.307199in}}{\pgfqpoint{0.813215in}{1.303926in}}{\pgfqpoint{0.807391in}{1.298102in}}%
\pgfpathcurveto{\pgfqpoint{0.801567in}{1.292278in}}{\pgfqpoint{0.798295in}{1.284378in}}{\pgfqpoint{0.798295in}{1.276142in}}%
\pgfpathcurveto{\pgfqpoint{0.798295in}{1.267906in}}{\pgfqpoint{0.801567in}{1.260006in}}{\pgfqpoint{0.807391in}{1.254182in}}%
\pgfpathcurveto{\pgfqpoint{0.813215in}{1.248358in}}{\pgfqpoint{0.821115in}{1.245086in}}{\pgfqpoint{0.829351in}{1.245086in}}%
\pgfpathclose%
\pgfusepath{stroke,fill}%
\end{pgfscope}%
\begin{pgfscope}%
\pgfpathrectangle{\pgfqpoint{0.100000in}{0.220728in}}{\pgfqpoint{3.696000in}{3.696000in}}%
\pgfusepath{clip}%
\pgfsetbuttcap%
\pgfsetroundjoin%
\definecolor{currentfill}{rgb}{0.121569,0.466667,0.705882}%
\pgfsetfillcolor{currentfill}%
\pgfsetfillopacity{0.612004}%
\pgfsetlinewidth{1.003750pt}%
\definecolor{currentstroke}{rgb}{0.121569,0.466667,0.705882}%
\pgfsetstrokecolor{currentstroke}%
\pgfsetstrokeopacity{0.612004}%
\pgfsetdash{}{0pt}%
\pgfpathmoveto{\pgfqpoint{0.843997in}{1.229043in}}%
\pgfpathcurveto{\pgfqpoint{0.852233in}{1.229043in}}{\pgfqpoint{0.860133in}{1.232315in}}{\pgfqpoint{0.865957in}{1.238139in}}%
\pgfpathcurveto{\pgfqpoint{0.871781in}{1.243963in}}{\pgfqpoint{0.875054in}{1.251863in}}{\pgfqpoint{0.875054in}{1.260100in}}%
\pgfpathcurveto{\pgfqpoint{0.875054in}{1.268336in}}{\pgfqpoint{0.871781in}{1.276236in}}{\pgfqpoint{0.865957in}{1.282060in}}%
\pgfpathcurveto{\pgfqpoint{0.860133in}{1.287884in}}{\pgfqpoint{0.852233in}{1.291156in}}{\pgfqpoint{0.843997in}{1.291156in}}%
\pgfpathcurveto{\pgfqpoint{0.835761in}{1.291156in}}{\pgfqpoint{0.827861in}{1.287884in}}{\pgfqpoint{0.822037in}{1.282060in}}%
\pgfpathcurveto{\pgfqpoint{0.816213in}{1.276236in}}{\pgfqpoint{0.812941in}{1.268336in}}{\pgfqpoint{0.812941in}{1.260100in}}%
\pgfpathcurveto{\pgfqpoint{0.812941in}{1.251863in}}{\pgfqpoint{0.816213in}{1.243963in}}{\pgfqpoint{0.822037in}{1.238139in}}%
\pgfpathcurveto{\pgfqpoint{0.827861in}{1.232315in}}{\pgfqpoint{0.835761in}{1.229043in}}{\pgfqpoint{0.843997in}{1.229043in}}%
\pgfpathclose%
\pgfusepath{stroke,fill}%
\end{pgfscope}%
\begin{pgfscope}%
\pgfpathrectangle{\pgfqpoint{0.100000in}{0.220728in}}{\pgfqpoint{3.696000in}{3.696000in}}%
\pgfusepath{clip}%
\pgfsetbuttcap%
\pgfsetroundjoin%
\definecolor{currentfill}{rgb}{0.121569,0.466667,0.705882}%
\pgfsetfillcolor{currentfill}%
\pgfsetfillopacity{0.612753}%
\pgfsetlinewidth{1.003750pt}%
\definecolor{currentstroke}{rgb}{0.121569,0.466667,0.705882}%
\pgfsetstrokecolor{currentstroke}%
\pgfsetstrokeopacity{0.612753}%
\pgfsetdash{}{0pt}%
\pgfpathmoveto{\pgfqpoint{0.842015in}{1.226557in}}%
\pgfpathcurveto{\pgfqpoint{0.850251in}{1.226557in}}{\pgfqpoint{0.858151in}{1.229829in}}{\pgfqpoint{0.863975in}{1.235653in}}%
\pgfpathcurveto{\pgfqpoint{0.869799in}{1.241477in}}{\pgfqpoint{0.873071in}{1.249377in}}{\pgfqpoint{0.873071in}{1.257613in}}%
\pgfpathcurveto{\pgfqpoint{0.873071in}{1.265850in}}{\pgfqpoint{0.869799in}{1.273750in}}{\pgfqpoint{0.863975in}{1.279574in}}%
\pgfpathcurveto{\pgfqpoint{0.858151in}{1.285398in}}{\pgfqpoint{0.850251in}{1.288670in}}{\pgfqpoint{0.842015in}{1.288670in}}%
\pgfpathcurveto{\pgfqpoint{0.833779in}{1.288670in}}{\pgfqpoint{0.825878in}{1.285398in}}{\pgfqpoint{0.820055in}{1.279574in}}%
\pgfpathcurveto{\pgfqpoint{0.814231in}{1.273750in}}{\pgfqpoint{0.810958in}{1.265850in}}{\pgfqpoint{0.810958in}{1.257613in}}%
\pgfpathcurveto{\pgfqpoint{0.810958in}{1.249377in}}{\pgfqpoint{0.814231in}{1.241477in}}{\pgfqpoint{0.820055in}{1.235653in}}%
\pgfpathcurveto{\pgfqpoint{0.825878in}{1.229829in}}{\pgfqpoint{0.833779in}{1.226557in}}{\pgfqpoint{0.842015in}{1.226557in}}%
\pgfpathclose%
\pgfusepath{stroke,fill}%
\end{pgfscope}%
\begin{pgfscope}%
\pgfpathrectangle{\pgfqpoint{0.100000in}{0.220728in}}{\pgfqpoint{3.696000in}{3.696000in}}%
\pgfusepath{clip}%
\pgfsetbuttcap%
\pgfsetroundjoin%
\definecolor{currentfill}{rgb}{0.121569,0.466667,0.705882}%
\pgfsetfillcolor{currentfill}%
\pgfsetfillopacity{0.613139}%
\pgfsetlinewidth{1.003750pt}%
\definecolor{currentstroke}{rgb}{0.121569,0.466667,0.705882}%
\pgfsetstrokecolor{currentstroke}%
\pgfsetstrokeopacity{0.613139}%
\pgfsetdash{}{0pt}%
\pgfpathmoveto{\pgfqpoint{0.826359in}{1.236053in}}%
\pgfpathcurveto{\pgfqpoint{0.834596in}{1.236053in}}{\pgfqpoint{0.842496in}{1.239325in}}{\pgfqpoint{0.848320in}{1.245149in}}%
\pgfpathcurveto{\pgfqpoint{0.854143in}{1.250973in}}{\pgfqpoint{0.857416in}{1.258873in}}{\pgfqpoint{0.857416in}{1.267109in}}%
\pgfpathcurveto{\pgfqpoint{0.857416in}{1.275345in}}{\pgfqpoint{0.854143in}{1.283246in}}{\pgfqpoint{0.848320in}{1.289069in}}%
\pgfpathcurveto{\pgfqpoint{0.842496in}{1.294893in}}{\pgfqpoint{0.834596in}{1.298166in}}{\pgfqpoint{0.826359in}{1.298166in}}%
\pgfpathcurveto{\pgfqpoint{0.818123in}{1.298166in}}{\pgfqpoint{0.810223in}{1.294893in}}{\pgfqpoint{0.804399in}{1.289069in}}%
\pgfpathcurveto{\pgfqpoint{0.798575in}{1.283246in}}{\pgfqpoint{0.795303in}{1.275345in}}{\pgfqpoint{0.795303in}{1.267109in}}%
\pgfpathcurveto{\pgfqpoint{0.795303in}{1.258873in}}{\pgfqpoint{0.798575in}{1.250973in}}{\pgfqpoint{0.804399in}{1.245149in}}%
\pgfpathcurveto{\pgfqpoint{0.810223in}{1.239325in}}{\pgfqpoint{0.818123in}{1.236053in}}{\pgfqpoint{0.826359in}{1.236053in}}%
\pgfpathclose%
\pgfusepath{stroke,fill}%
\end{pgfscope}%
\begin{pgfscope}%
\pgfpathrectangle{\pgfqpoint{0.100000in}{0.220728in}}{\pgfqpoint{3.696000in}{3.696000in}}%
\pgfusepath{clip}%
\pgfsetbuttcap%
\pgfsetroundjoin%
\definecolor{currentfill}{rgb}{0.121569,0.466667,0.705882}%
\pgfsetfillcolor{currentfill}%
\pgfsetfillopacity{0.613744}%
\pgfsetlinewidth{1.003750pt}%
\definecolor{currentstroke}{rgb}{0.121569,0.466667,0.705882}%
\pgfsetstrokecolor{currentstroke}%
\pgfsetstrokeopacity{0.613744}%
\pgfsetdash{}{0pt}%
\pgfpathmoveto{\pgfqpoint{0.838954in}{1.223048in}}%
\pgfpathcurveto{\pgfqpoint{0.847190in}{1.223048in}}{\pgfqpoint{0.855090in}{1.226320in}}{\pgfqpoint{0.860914in}{1.232144in}}%
\pgfpathcurveto{\pgfqpoint{0.866738in}{1.237968in}}{\pgfqpoint{0.870010in}{1.245868in}}{\pgfqpoint{0.870010in}{1.254105in}}%
\pgfpathcurveto{\pgfqpoint{0.870010in}{1.262341in}}{\pgfqpoint{0.866738in}{1.270241in}}{\pgfqpoint{0.860914in}{1.276065in}}%
\pgfpathcurveto{\pgfqpoint{0.855090in}{1.281889in}}{\pgfqpoint{0.847190in}{1.285161in}}{\pgfqpoint{0.838954in}{1.285161in}}%
\pgfpathcurveto{\pgfqpoint{0.830717in}{1.285161in}}{\pgfqpoint{0.822817in}{1.281889in}}{\pgfqpoint{0.816993in}{1.276065in}}%
\pgfpathcurveto{\pgfqpoint{0.811169in}{1.270241in}}{\pgfqpoint{0.807897in}{1.262341in}}{\pgfqpoint{0.807897in}{1.254105in}}%
\pgfpathcurveto{\pgfqpoint{0.807897in}{1.245868in}}{\pgfqpoint{0.811169in}{1.237968in}}{\pgfqpoint{0.816993in}{1.232144in}}%
\pgfpathcurveto{\pgfqpoint{0.822817in}{1.226320in}}{\pgfqpoint{0.830717in}{1.223048in}}{\pgfqpoint{0.838954in}{1.223048in}}%
\pgfpathclose%
\pgfusepath{stroke,fill}%
\end{pgfscope}%
\begin{pgfscope}%
\pgfpathrectangle{\pgfqpoint{0.100000in}{0.220728in}}{\pgfqpoint{3.696000in}{3.696000in}}%
\pgfusepath{clip}%
\pgfsetbuttcap%
\pgfsetroundjoin%
\definecolor{currentfill}{rgb}{0.121569,0.466667,0.705882}%
\pgfsetfillcolor{currentfill}%
\pgfsetfillopacity{0.614011}%
\pgfsetlinewidth{1.003750pt}%
\definecolor{currentstroke}{rgb}{0.121569,0.466667,0.705882}%
\pgfsetstrokecolor{currentstroke}%
\pgfsetstrokeopacity{0.614011}%
\pgfsetdash{}{0pt}%
\pgfpathmoveto{\pgfqpoint{0.823482in}{1.228622in}}%
\pgfpathcurveto{\pgfqpoint{0.831718in}{1.228622in}}{\pgfqpoint{0.839618in}{1.231894in}}{\pgfqpoint{0.845442in}{1.237718in}}%
\pgfpathcurveto{\pgfqpoint{0.851266in}{1.243542in}}{\pgfqpoint{0.854538in}{1.251442in}}{\pgfqpoint{0.854538in}{1.259679in}}%
\pgfpathcurveto{\pgfqpoint{0.854538in}{1.267915in}}{\pgfqpoint{0.851266in}{1.275815in}}{\pgfqpoint{0.845442in}{1.281639in}}%
\pgfpathcurveto{\pgfqpoint{0.839618in}{1.287463in}}{\pgfqpoint{0.831718in}{1.290735in}}{\pgfqpoint{0.823482in}{1.290735in}}%
\pgfpathcurveto{\pgfqpoint{0.815245in}{1.290735in}}{\pgfqpoint{0.807345in}{1.287463in}}{\pgfqpoint{0.801521in}{1.281639in}}%
\pgfpathcurveto{\pgfqpoint{0.795698in}{1.275815in}}{\pgfqpoint{0.792425in}{1.267915in}}{\pgfqpoint{0.792425in}{1.259679in}}%
\pgfpathcurveto{\pgfqpoint{0.792425in}{1.251442in}}{\pgfqpoint{0.795698in}{1.243542in}}{\pgfqpoint{0.801521in}{1.237718in}}%
\pgfpathcurveto{\pgfqpoint{0.807345in}{1.231894in}}{\pgfqpoint{0.815245in}{1.228622in}}{\pgfqpoint{0.823482in}{1.228622in}}%
\pgfpathclose%
\pgfusepath{stroke,fill}%
\end{pgfscope}%
\begin{pgfscope}%
\pgfpathrectangle{\pgfqpoint{0.100000in}{0.220728in}}{\pgfqpoint{3.696000in}{3.696000in}}%
\pgfusepath{clip}%
\pgfsetbuttcap%
\pgfsetroundjoin%
\definecolor{currentfill}{rgb}{0.121569,0.466667,0.705882}%
\pgfsetfillcolor{currentfill}%
\pgfsetfillopacity{0.614792}%
\pgfsetlinewidth{1.003750pt}%
\definecolor{currentstroke}{rgb}{0.121569,0.466667,0.705882}%
\pgfsetstrokecolor{currentstroke}%
\pgfsetstrokeopacity{0.614792}%
\pgfsetdash{}{0pt}%
\pgfpathmoveto{\pgfqpoint{0.822684in}{1.223416in}}%
\pgfpathcurveto{\pgfqpoint{0.830920in}{1.223416in}}{\pgfqpoint{0.838821in}{1.226688in}}{\pgfqpoint{0.844644in}{1.232512in}}%
\pgfpathcurveto{\pgfqpoint{0.850468in}{1.238336in}}{\pgfqpoint{0.853741in}{1.246236in}}{\pgfqpoint{0.853741in}{1.254472in}}%
\pgfpathcurveto{\pgfqpoint{0.853741in}{1.262709in}}{\pgfqpoint{0.850468in}{1.270609in}}{\pgfqpoint{0.844644in}{1.276433in}}%
\pgfpathcurveto{\pgfqpoint{0.838821in}{1.282257in}}{\pgfqpoint{0.830920in}{1.285529in}}{\pgfqpoint{0.822684in}{1.285529in}}%
\pgfpathcurveto{\pgfqpoint{0.814448in}{1.285529in}}{\pgfqpoint{0.806548in}{1.282257in}}{\pgfqpoint{0.800724in}{1.276433in}}%
\pgfpathcurveto{\pgfqpoint{0.794900in}{1.270609in}}{\pgfqpoint{0.791628in}{1.262709in}}{\pgfqpoint{0.791628in}{1.254472in}}%
\pgfpathcurveto{\pgfqpoint{0.791628in}{1.246236in}}{\pgfqpoint{0.794900in}{1.238336in}}{\pgfqpoint{0.800724in}{1.232512in}}%
\pgfpathcurveto{\pgfqpoint{0.806548in}{1.226688in}}{\pgfqpoint{0.814448in}{1.223416in}}{\pgfqpoint{0.822684in}{1.223416in}}%
\pgfpathclose%
\pgfusepath{stroke,fill}%
\end{pgfscope}%
\begin{pgfscope}%
\pgfpathrectangle{\pgfqpoint{0.100000in}{0.220728in}}{\pgfqpoint{3.696000in}{3.696000in}}%
\pgfusepath{clip}%
\pgfsetbuttcap%
\pgfsetroundjoin%
\definecolor{currentfill}{rgb}{0.121569,0.466667,0.705882}%
\pgfsetfillcolor{currentfill}%
\pgfsetfillopacity{0.615003}%
\pgfsetlinewidth{1.003750pt}%
\definecolor{currentstroke}{rgb}{0.121569,0.466667,0.705882}%
\pgfsetstrokecolor{currentstroke}%
\pgfsetstrokeopacity{0.615003}%
\pgfsetdash{}{0pt}%
\pgfpathmoveto{\pgfqpoint{0.835204in}{1.218986in}}%
\pgfpathcurveto{\pgfqpoint{0.843440in}{1.218986in}}{\pgfqpoint{0.851340in}{1.222258in}}{\pgfqpoint{0.857164in}{1.228082in}}%
\pgfpathcurveto{\pgfqpoint{0.862988in}{1.233906in}}{\pgfqpoint{0.866261in}{1.241806in}}{\pgfqpoint{0.866261in}{1.250042in}}%
\pgfpathcurveto{\pgfqpoint{0.866261in}{1.258279in}}{\pgfqpoint{0.862988in}{1.266179in}}{\pgfqpoint{0.857164in}{1.272003in}}%
\pgfpathcurveto{\pgfqpoint{0.851340in}{1.277827in}}{\pgfqpoint{0.843440in}{1.281099in}}{\pgfqpoint{0.835204in}{1.281099in}}%
\pgfpathcurveto{\pgfqpoint{0.826968in}{1.281099in}}{\pgfqpoint{0.819068in}{1.277827in}}{\pgfqpoint{0.813244in}{1.272003in}}%
\pgfpathcurveto{\pgfqpoint{0.807420in}{1.266179in}}{\pgfqpoint{0.804148in}{1.258279in}}{\pgfqpoint{0.804148in}{1.250042in}}%
\pgfpathcurveto{\pgfqpoint{0.804148in}{1.241806in}}{\pgfqpoint{0.807420in}{1.233906in}}{\pgfqpoint{0.813244in}{1.228082in}}%
\pgfpathcurveto{\pgfqpoint{0.819068in}{1.222258in}}{\pgfqpoint{0.826968in}{1.218986in}}{\pgfqpoint{0.835204in}{1.218986in}}%
\pgfpathclose%
\pgfusepath{stroke,fill}%
\end{pgfscope}%
\begin{pgfscope}%
\pgfpathrectangle{\pgfqpoint{0.100000in}{0.220728in}}{\pgfqpoint{3.696000in}{3.696000in}}%
\pgfusepath{clip}%
\pgfsetbuttcap%
\pgfsetroundjoin%
\definecolor{currentfill}{rgb}{0.121569,0.466667,0.705882}%
\pgfsetfillcolor{currentfill}%
\pgfsetfillopacity{0.615377}%
\pgfsetlinewidth{1.003750pt}%
\definecolor{currentstroke}{rgb}{0.121569,0.466667,0.705882}%
\pgfsetstrokecolor{currentstroke}%
\pgfsetstrokeopacity{0.615377}%
\pgfsetdash{}{0pt}%
\pgfpathmoveto{\pgfqpoint{0.822321in}{1.219635in}}%
\pgfpathcurveto{\pgfqpoint{0.830557in}{1.219635in}}{\pgfqpoint{0.838457in}{1.222908in}}{\pgfqpoint{0.844281in}{1.228732in}}%
\pgfpathcurveto{\pgfqpoint{0.850105in}{1.234556in}}{\pgfqpoint{0.853377in}{1.242456in}}{\pgfqpoint{0.853377in}{1.250692in}}%
\pgfpathcurveto{\pgfqpoint{0.853377in}{1.258928in}}{\pgfqpoint{0.850105in}{1.266828in}}{\pgfqpoint{0.844281in}{1.272652in}}%
\pgfpathcurveto{\pgfqpoint{0.838457in}{1.278476in}}{\pgfqpoint{0.830557in}{1.281748in}}{\pgfqpoint{0.822321in}{1.281748in}}%
\pgfpathcurveto{\pgfqpoint{0.814085in}{1.281748in}}{\pgfqpoint{0.806185in}{1.278476in}}{\pgfqpoint{0.800361in}{1.272652in}}%
\pgfpathcurveto{\pgfqpoint{0.794537in}{1.266828in}}{\pgfqpoint{0.791264in}{1.258928in}}{\pgfqpoint{0.791264in}{1.250692in}}%
\pgfpathcurveto{\pgfqpoint{0.791264in}{1.242456in}}{\pgfqpoint{0.794537in}{1.234556in}}{\pgfqpoint{0.800361in}{1.228732in}}%
\pgfpathcurveto{\pgfqpoint{0.806185in}{1.222908in}}{\pgfqpoint{0.814085in}{1.219635in}}{\pgfqpoint{0.822321in}{1.219635in}}%
\pgfpathclose%
\pgfusepath{stroke,fill}%
\end{pgfscope}%
\begin{pgfscope}%
\pgfpathrectangle{\pgfqpoint{0.100000in}{0.220728in}}{\pgfqpoint{3.696000in}{3.696000in}}%
\pgfusepath{clip}%
\pgfsetbuttcap%
\pgfsetroundjoin%
\definecolor{currentfill}{rgb}{0.121569,0.466667,0.705882}%
\pgfsetfillcolor{currentfill}%
\pgfsetfillopacity{0.615738}%
\pgfsetlinewidth{1.003750pt}%
\definecolor{currentstroke}{rgb}{0.121569,0.466667,0.705882}%
\pgfsetstrokecolor{currentstroke}%
\pgfsetstrokeopacity{0.615738}%
\pgfsetdash{}{0pt}%
\pgfpathmoveto{\pgfqpoint{0.833339in}{1.216617in}}%
\pgfpathcurveto{\pgfqpoint{0.841575in}{1.216617in}}{\pgfqpoint{0.849475in}{1.219889in}}{\pgfqpoint{0.855299in}{1.225713in}}%
\pgfpathcurveto{\pgfqpoint{0.861123in}{1.231537in}}{\pgfqpoint{0.864395in}{1.239437in}}{\pgfqpoint{0.864395in}{1.247674in}}%
\pgfpathcurveto{\pgfqpoint{0.864395in}{1.255910in}}{\pgfqpoint{0.861123in}{1.263810in}}{\pgfqpoint{0.855299in}{1.269634in}}%
\pgfpathcurveto{\pgfqpoint{0.849475in}{1.275458in}}{\pgfqpoint{0.841575in}{1.278730in}}{\pgfqpoint{0.833339in}{1.278730in}}%
\pgfpathcurveto{\pgfqpoint{0.825102in}{1.278730in}}{\pgfqpoint{0.817202in}{1.275458in}}{\pgfqpoint{0.811379in}{1.269634in}}%
\pgfpathcurveto{\pgfqpoint{0.805555in}{1.263810in}}{\pgfqpoint{0.802282in}{1.255910in}}{\pgfqpoint{0.802282in}{1.247674in}}%
\pgfpathcurveto{\pgfqpoint{0.802282in}{1.239437in}}{\pgfqpoint{0.805555in}{1.231537in}}{\pgfqpoint{0.811379in}{1.225713in}}%
\pgfpathcurveto{\pgfqpoint{0.817202in}{1.219889in}}{\pgfqpoint{0.825102in}{1.216617in}}{\pgfqpoint{0.833339in}{1.216617in}}%
\pgfpathclose%
\pgfusepath{stroke,fill}%
\end{pgfscope}%
\begin{pgfscope}%
\pgfpathrectangle{\pgfqpoint{0.100000in}{0.220728in}}{\pgfqpoint{3.696000in}{3.696000in}}%
\pgfusepath{clip}%
\pgfsetbuttcap%
\pgfsetroundjoin%
\definecolor{currentfill}{rgb}{0.121569,0.466667,0.705882}%
\pgfsetfillcolor{currentfill}%
\pgfsetfillopacity{0.615748}%
\pgfsetlinewidth{1.003750pt}%
\definecolor{currentstroke}{rgb}{0.121569,0.466667,0.705882}%
\pgfsetstrokecolor{currentstroke}%
\pgfsetstrokeopacity{0.615748}%
\pgfsetdash{}{0pt}%
\pgfpathmoveto{\pgfqpoint{0.823174in}{1.216724in}}%
\pgfpathcurveto{\pgfqpoint{0.831411in}{1.216724in}}{\pgfqpoint{0.839311in}{1.219996in}}{\pgfqpoint{0.845135in}{1.225820in}}%
\pgfpathcurveto{\pgfqpoint{0.850958in}{1.231644in}}{\pgfqpoint{0.854231in}{1.239544in}}{\pgfqpoint{0.854231in}{1.247780in}}%
\pgfpathcurveto{\pgfqpoint{0.854231in}{1.256016in}}{\pgfqpoint{0.850958in}{1.263917in}}{\pgfqpoint{0.845135in}{1.269740in}}%
\pgfpathcurveto{\pgfqpoint{0.839311in}{1.275564in}}{\pgfqpoint{0.831411in}{1.278837in}}{\pgfqpoint{0.823174in}{1.278837in}}%
\pgfpathcurveto{\pgfqpoint{0.814938in}{1.278837in}}{\pgfqpoint{0.807038in}{1.275564in}}{\pgfqpoint{0.801214in}{1.269740in}}%
\pgfpathcurveto{\pgfqpoint{0.795390in}{1.263917in}}{\pgfqpoint{0.792118in}{1.256016in}}{\pgfqpoint{0.792118in}{1.247780in}}%
\pgfpathcurveto{\pgfqpoint{0.792118in}{1.239544in}}{\pgfqpoint{0.795390in}{1.231644in}}{\pgfqpoint{0.801214in}{1.225820in}}%
\pgfpathcurveto{\pgfqpoint{0.807038in}{1.219996in}}{\pgfqpoint{0.814938in}{1.216724in}}{\pgfqpoint{0.823174in}{1.216724in}}%
\pgfpathclose%
\pgfusepath{stroke,fill}%
\end{pgfscope}%
\begin{pgfscope}%
\pgfpathrectangle{\pgfqpoint{0.100000in}{0.220728in}}{\pgfqpoint{3.696000in}{3.696000in}}%
\pgfusepath{clip}%
\pgfsetbuttcap%
\pgfsetroundjoin%
\definecolor{currentfill}{rgb}{0.121569,0.466667,0.705882}%
\pgfsetfillcolor{currentfill}%
\pgfsetfillopacity{0.615816}%
\pgfsetlinewidth{1.003750pt}%
\definecolor{currentstroke}{rgb}{0.121569,0.466667,0.705882}%
\pgfsetstrokecolor{currentstroke}%
\pgfsetstrokeopacity{0.615816}%
\pgfsetdash{}{0pt}%
\pgfpathmoveto{\pgfqpoint{3.119517in}{2.981589in}}%
\pgfpathcurveto{\pgfqpoint{3.127753in}{2.981589in}}{\pgfqpoint{3.135653in}{2.984861in}}{\pgfqpoint{3.141477in}{2.990685in}}%
\pgfpathcurveto{\pgfqpoint{3.147301in}{2.996509in}}{\pgfqpoint{3.150574in}{3.004409in}}{\pgfqpoint{3.150574in}{3.012645in}}%
\pgfpathcurveto{\pgfqpoint{3.150574in}{3.020882in}}{\pgfqpoint{3.147301in}{3.028782in}}{\pgfqpoint{3.141477in}{3.034606in}}%
\pgfpathcurveto{\pgfqpoint{3.135653in}{3.040429in}}{\pgfqpoint{3.127753in}{3.043702in}}{\pgfqpoint{3.119517in}{3.043702in}}%
\pgfpathcurveto{\pgfqpoint{3.111281in}{3.043702in}}{\pgfqpoint{3.103381in}{3.040429in}}{\pgfqpoint{3.097557in}{3.034606in}}%
\pgfpathcurveto{\pgfqpoint{3.091733in}{3.028782in}}{\pgfqpoint{3.088461in}{3.020882in}}{\pgfqpoint{3.088461in}{3.012645in}}%
\pgfpathcurveto{\pgfqpoint{3.088461in}{3.004409in}}{\pgfqpoint{3.091733in}{2.996509in}}{\pgfqpoint{3.097557in}{2.990685in}}%
\pgfpathcurveto{\pgfqpoint{3.103381in}{2.984861in}}{\pgfqpoint{3.111281in}{2.981589in}}{\pgfqpoint{3.119517in}{2.981589in}}%
\pgfpathclose%
\pgfusepath{stroke,fill}%
\end{pgfscope}%
\begin{pgfscope}%
\pgfpathrectangle{\pgfqpoint{0.100000in}{0.220728in}}{\pgfqpoint{3.696000in}{3.696000in}}%
\pgfusepath{clip}%
\pgfsetbuttcap%
\pgfsetroundjoin%
\definecolor{currentfill}{rgb}{0.121569,0.466667,0.705882}%
\pgfsetfillcolor{currentfill}%
\pgfsetfillopacity{0.615967}%
\pgfsetlinewidth{1.003750pt}%
\definecolor{currentstroke}{rgb}{0.121569,0.466667,0.705882}%
\pgfsetstrokecolor{currentstroke}%
\pgfsetstrokeopacity{0.615967}%
\pgfsetdash{}{0pt}%
\pgfpathmoveto{\pgfqpoint{0.824119in}{1.215019in}}%
\pgfpathcurveto{\pgfqpoint{0.832355in}{1.215019in}}{\pgfqpoint{0.840255in}{1.218291in}}{\pgfqpoint{0.846079in}{1.224115in}}%
\pgfpathcurveto{\pgfqpoint{0.851903in}{1.229939in}}{\pgfqpoint{0.855176in}{1.237839in}}{\pgfqpoint{0.855176in}{1.246076in}}%
\pgfpathcurveto{\pgfqpoint{0.855176in}{1.254312in}}{\pgfqpoint{0.851903in}{1.262212in}}{\pgfqpoint{0.846079in}{1.268036in}}%
\pgfpathcurveto{\pgfqpoint{0.840255in}{1.273860in}}{\pgfqpoint{0.832355in}{1.277132in}}{\pgfqpoint{0.824119in}{1.277132in}}%
\pgfpathcurveto{\pgfqpoint{0.815883in}{1.277132in}}{\pgfqpoint{0.807983in}{1.273860in}}{\pgfqpoint{0.802159in}{1.268036in}}%
\pgfpathcurveto{\pgfqpoint{0.796335in}{1.262212in}}{\pgfqpoint{0.793063in}{1.254312in}}{\pgfqpoint{0.793063in}{1.246076in}}%
\pgfpathcurveto{\pgfqpoint{0.793063in}{1.237839in}}{\pgfqpoint{0.796335in}{1.229939in}}{\pgfqpoint{0.802159in}{1.224115in}}%
\pgfpathcurveto{\pgfqpoint{0.807983in}{1.218291in}}{\pgfqpoint{0.815883in}{1.215019in}}{\pgfqpoint{0.824119in}{1.215019in}}%
\pgfpathclose%
\pgfusepath{stroke,fill}%
\end{pgfscope}%
\begin{pgfscope}%
\pgfpathrectangle{\pgfqpoint{0.100000in}{0.220728in}}{\pgfqpoint{3.696000in}{3.696000in}}%
\pgfusepath{clip}%
\pgfsetbuttcap%
\pgfsetroundjoin%
\definecolor{currentfill}{rgb}{0.121569,0.466667,0.705882}%
\pgfsetfillcolor{currentfill}%
\pgfsetfillopacity{0.616135}%
\pgfsetlinewidth{1.003750pt}%
\definecolor{currentstroke}{rgb}{0.121569,0.466667,0.705882}%
\pgfsetstrokecolor{currentstroke}%
\pgfsetstrokeopacity{0.616135}%
\pgfsetdash{}{0pt}%
\pgfpathmoveto{\pgfqpoint{0.832336in}{1.215254in}}%
\pgfpathcurveto{\pgfqpoint{0.840572in}{1.215254in}}{\pgfqpoint{0.848472in}{1.218526in}}{\pgfqpoint{0.854296in}{1.224350in}}%
\pgfpathcurveto{\pgfqpoint{0.860120in}{1.230174in}}{\pgfqpoint{0.863393in}{1.238074in}}{\pgfqpoint{0.863393in}{1.246310in}}%
\pgfpathcurveto{\pgfqpoint{0.863393in}{1.254547in}}{\pgfqpoint{0.860120in}{1.262447in}}{\pgfqpoint{0.854296in}{1.268271in}}%
\pgfpathcurveto{\pgfqpoint{0.848472in}{1.274094in}}{\pgfqpoint{0.840572in}{1.277367in}}{\pgfqpoint{0.832336in}{1.277367in}}%
\pgfpathcurveto{\pgfqpoint{0.824100in}{1.277367in}}{\pgfqpoint{0.816200in}{1.274094in}}{\pgfqpoint{0.810376in}{1.268271in}}%
\pgfpathcurveto{\pgfqpoint{0.804552in}{1.262447in}}{\pgfqpoint{0.801280in}{1.254547in}}{\pgfqpoint{0.801280in}{1.246310in}}%
\pgfpathcurveto{\pgfqpoint{0.801280in}{1.238074in}}{\pgfqpoint{0.804552in}{1.230174in}}{\pgfqpoint{0.810376in}{1.224350in}}%
\pgfpathcurveto{\pgfqpoint{0.816200in}{1.218526in}}{\pgfqpoint{0.824100in}{1.215254in}}{\pgfqpoint{0.832336in}{1.215254in}}%
\pgfpathclose%
\pgfusepath{stroke,fill}%
\end{pgfscope}%
\begin{pgfscope}%
\pgfpathrectangle{\pgfqpoint{0.100000in}{0.220728in}}{\pgfqpoint{3.696000in}{3.696000in}}%
\pgfusepath{clip}%
\pgfsetbuttcap%
\pgfsetroundjoin%
\definecolor{currentfill}{rgb}{0.121569,0.466667,0.705882}%
\pgfsetfillcolor{currentfill}%
\pgfsetfillopacity{0.616350}%
\pgfsetlinewidth{1.003750pt}%
\definecolor{currentstroke}{rgb}{0.121569,0.466667,0.705882}%
\pgfsetstrokecolor{currentstroke}%
\pgfsetstrokeopacity{0.616350}%
\pgfsetdash{}{0pt}%
\pgfpathmoveto{\pgfqpoint{0.831751in}{1.214535in}}%
\pgfpathcurveto{\pgfqpoint{0.839988in}{1.214535in}}{\pgfqpoint{0.847888in}{1.217808in}}{\pgfqpoint{0.853712in}{1.223631in}}%
\pgfpathcurveto{\pgfqpoint{0.859536in}{1.229455in}}{\pgfqpoint{0.862808in}{1.237355in}}{\pgfqpoint{0.862808in}{1.245592in}}%
\pgfpathcurveto{\pgfqpoint{0.862808in}{1.253828in}}{\pgfqpoint{0.859536in}{1.261728in}}{\pgfqpoint{0.853712in}{1.267552in}}%
\pgfpathcurveto{\pgfqpoint{0.847888in}{1.273376in}}{\pgfqpoint{0.839988in}{1.276648in}}{\pgfqpoint{0.831751in}{1.276648in}}%
\pgfpathcurveto{\pgfqpoint{0.823515in}{1.276648in}}{\pgfqpoint{0.815615in}{1.273376in}}{\pgfqpoint{0.809791in}{1.267552in}}%
\pgfpathcurveto{\pgfqpoint{0.803967in}{1.261728in}}{\pgfqpoint{0.800695in}{1.253828in}}{\pgfqpoint{0.800695in}{1.245592in}}%
\pgfpathcurveto{\pgfqpoint{0.800695in}{1.237355in}}{\pgfqpoint{0.803967in}{1.229455in}}{\pgfqpoint{0.809791in}{1.223631in}}%
\pgfpathcurveto{\pgfqpoint{0.815615in}{1.217808in}}{\pgfqpoint{0.823515in}{1.214535in}}{\pgfqpoint{0.831751in}{1.214535in}}%
\pgfpathclose%
\pgfusepath{stroke,fill}%
\end{pgfscope}%
\begin{pgfscope}%
\pgfpathrectangle{\pgfqpoint{0.100000in}{0.220728in}}{\pgfqpoint{3.696000in}{3.696000in}}%
\pgfusepath{clip}%
\pgfsetbuttcap%
\pgfsetroundjoin%
\definecolor{currentfill}{rgb}{0.121569,0.466667,0.705882}%
\pgfsetfillcolor{currentfill}%
\pgfsetfillopacity{0.616379}%
\pgfsetlinewidth{1.003750pt}%
\definecolor{currentstroke}{rgb}{0.121569,0.466667,0.705882}%
\pgfsetstrokecolor{currentstroke}%
\pgfsetstrokeopacity{0.616379}%
\pgfsetdash{}{0pt}%
\pgfpathmoveto{\pgfqpoint{0.826405in}{1.213225in}}%
\pgfpathcurveto{\pgfqpoint{0.834641in}{1.213225in}}{\pgfqpoint{0.842541in}{1.216497in}}{\pgfqpoint{0.848365in}{1.222321in}}%
\pgfpathcurveto{\pgfqpoint{0.854189in}{1.228145in}}{\pgfqpoint{0.857461in}{1.236045in}}{\pgfqpoint{0.857461in}{1.244282in}}%
\pgfpathcurveto{\pgfqpoint{0.857461in}{1.252518in}}{\pgfqpoint{0.854189in}{1.260418in}}{\pgfqpoint{0.848365in}{1.266242in}}%
\pgfpathcurveto{\pgfqpoint{0.842541in}{1.272066in}}{\pgfqpoint{0.834641in}{1.275338in}}{\pgfqpoint{0.826405in}{1.275338in}}%
\pgfpathcurveto{\pgfqpoint{0.818169in}{1.275338in}}{\pgfqpoint{0.810269in}{1.272066in}}{\pgfqpoint{0.804445in}{1.266242in}}%
\pgfpathcurveto{\pgfqpoint{0.798621in}{1.260418in}}{\pgfqpoint{0.795348in}{1.252518in}}{\pgfqpoint{0.795348in}{1.244282in}}%
\pgfpathcurveto{\pgfqpoint{0.795348in}{1.236045in}}{\pgfqpoint{0.798621in}{1.228145in}}{\pgfqpoint{0.804445in}{1.222321in}}%
\pgfpathcurveto{\pgfqpoint{0.810269in}{1.216497in}}{\pgfqpoint{0.818169in}{1.213225in}}{\pgfqpoint{0.826405in}{1.213225in}}%
\pgfpathclose%
\pgfusepath{stroke,fill}%
\end{pgfscope}%
\begin{pgfscope}%
\pgfpathrectangle{\pgfqpoint{0.100000in}{0.220728in}}{\pgfqpoint{3.696000in}{3.696000in}}%
\pgfusepath{clip}%
\pgfsetbuttcap%
\pgfsetroundjoin%
\definecolor{currentfill}{rgb}{0.121569,0.466667,0.705882}%
\pgfsetfillcolor{currentfill}%
\pgfsetfillopacity{0.616465}%
\pgfsetlinewidth{1.003750pt}%
\definecolor{currentstroke}{rgb}{0.121569,0.466667,0.705882}%
\pgfsetstrokecolor{currentstroke}%
\pgfsetstrokeopacity{0.616465}%
\pgfsetdash{}{0pt}%
\pgfpathmoveto{\pgfqpoint{0.831442in}{1.214114in}}%
\pgfpathcurveto{\pgfqpoint{0.839679in}{1.214114in}}{\pgfqpoint{0.847579in}{1.217386in}}{\pgfqpoint{0.853403in}{1.223210in}}%
\pgfpathcurveto{\pgfqpoint{0.859227in}{1.229034in}}{\pgfqpoint{0.862499in}{1.236934in}}{\pgfqpoint{0.862499in}{1.245171in}}%
\pgfpathcurveto{\pgfqpoint{0.862499in}{1.253407in}}{\pgfqpoint{0.859227in}{1.261307in}}{\pgfqpoint{0.853403in}{1.267131in}}%
\pgfpathcurveto{\pgfqpoint{0.847579in}{1.272955in}}{\pgfqpoint{0.839679in}{1.276227in}}{\pgfqpoint{0.831442in}{1.276227in}}%
\pgfpathcurveto{\pgfqpoint{0.823206in}{1.276227in}}{\pgfqpoint{0.815306in}{1.272955in}}{\pgfqpoint{0.809482in}{1.267131in}}%
\pgfpathcurveto{\pgfqpoint{0.803658in}{1.261307in}}{\pgfqpoint{0.800386in}{1.253407in}}{\pgfqpoint{0.800386in}{1.245171in}}%
\pgfpathcurveto{\pgfqpoint{0.800386in}{1.236934in}}{\pgfqpoint{0.803658in}{1.229034in}}{\pgfqpoint{0.809482in}{1.223210in}}%
\pgfpathcurveto{\pgfqpoint{0.815306in}{1.217386in}}{\pgfqpoint{0.823206in}{1.214114in}}{\pgfqpoint{0.831442in}{1.214114in}}%
\pgfpathclose%
\pgfusepath{stroke,fill}%
\end{pgfscope}%
\begin{pgfscope}%
\pgfpathrectangle{\pgfqpoint{0.100000in}{0.220728in}}{\pgfqpoint{3.696000in}{3.696000in}}%
\pgfusepath{clip}%
\pgfsetbuttcap%
\pgfsetroundjoin%
\definecolor{currentfill}{rgb}{0.121569,0.466667,0.705882}%
\pgfsetfillcolor{currentfill}%
\pgfsetfillopacity{0.616524}%
\pgfsetlinewidth{1.003750pt}%
\definecolor{currentstroke}{rgb}{0.121569,0.466667,0.705882}%
\pgfsetstrokecolor{currentstroke}%
\pgfsetstrokeopacity{0.616524}%
\pgfsetdash{}{0pt}%
\pgfpathmoveto{\pgfqpoint{0.831268in}{1.213870in}}%
\pgfpathcurveto{\pgfqpoint{0.839504in}{1.213870in}}{\pgfqpoint{0.847404in}{1.217143in}}{\pgfqpoint{0.853228in}{1.222967in}}%
\pgfpathcurveto{\pgfqpoint{0.859052in}{1.228790in}}{\pgfqpoint{0.862324in}{1.236691in}}{\pgfqpoint{0.862324in}{1.244927in}}%
\pgfpathcurveto{\pgfqpoint{0.862324in}{1.253163in}}{\pgfqpoint{0.859052in}{1.261063in}}{\pgfqpoint{0.853228in}{1.266887in}}%
\pgfpathcurveto{\pgfqpoint{0.847404in}{1.272711in}}{\pgfqpoint{0.839504in}{1.275983in}}{\pgfqpoint{0.831268in}{1.275983in}}%
\pgfpathcurveto{\pgfqpoint{0.823031in}{1.275983in}}{\pgfqpoint{0.815131in}{1.272711in}}{\pgfqpoint{0.809307in}{1.266887in}}%
\pgfpathcurveto{\pgfqpoint{0.803483in}{1.261063in}}{\pgfqpoint{0.800211in}{1.253163in}}{\pgfqpoint{0.800211in}{1.244927in}}%
\pgfpathcurveto{\pgfqpoint{0.800211in}{1.236691in}}{\pgfqpoint{0.803483in}{1.228790in}}{\pgfqpoint{0.809307in}{1.222967in}}%
\pgfpathcurveto{\pgfqpoint{0.815131in}{1.217143in}}{\pgfqpoint{0.823031in}{1.213870in}}{\pgfqpoint{0.831268in}{1.213870in}}%
\pgfpathclose%
\pgfusepath{stroke,fill}%
\end{pgfscope}%
\begin{pgfscope}%
\pgfpathrectangle{\pgfqpoint{0.100000in}{0.220728in}}{\pgfqpoint{3.696000in}{3.696000in}}%
\pgfusepath{clip}%
\pgfsetbuttcap%
\pgfsetroundjoin%
\definecolor{currentfill}{rgb}{0.121569,0.466667,0.705882}%
\pgfsetfillcolor{currentfill}%
\pgfsetfillopacity{0.616556}%
\pgfsetlinewidth{1.003750pt}%
\definecolor{currentstroke}{rgb}{0.121569,0.466667,0.705882}%
\pgfsetstrokecolor{currentstroke}%
\pgfsetstrokeopacity{0.616556}%
\pgfsetdash{}{0pt}%
\pgfpathmoveto{\pgfqpoint{0.831169in}{1.213738in}}%
\pgfpathcurveto{\pgfqpoint{0.839405in}{1.213738in}}{\pgfqpoint{0.847305in}{1.217011in}}{\pgfqpoint{0.853129in}{1.222834in}}%
\pgfpathcurveto{\pgfqpoint{0.858953in}{1.228658in}}{\pgfqpoint{0.862225in}{1.236558in}}{\pgfqpoint{0.862225in}{1.244795in}}%
\pgfpathcurveto{\pgfqpoint{0.862225in}{1.253031in}}{\pgfqpoint{0.858953in}{1.260931in}}{\pgfqpoint{0.853129in}{1.266755in}}%
\pgfpathcurveto{\pgfqpoint{0.847305in}{1.272579in}}{\pgfqpoint{0.839405in}{1.275851in}}{\pgfqpoint{0.831169in}{1.275851in}}%
\pgfpathcurveto{\pgfqpoint{0.822932in}{1.275851in}}{\pgfqpoint{0.815032in}{1.272579in}}{\pgfqpoint{0.809208in}{1.266755in}}%
\pgfpathcurveto{\pgfqpoint{0.803384in}{1.260931in}}{\pgfqpoint{0.800112in}{1.253031in}}{\pgfqpoint{0.800112in}{1.244795in}}%
\pgfpathcurveto{\pgfqpoint{0.800112in}{1.236558in}}{\pgfqpoint{0.803384in}{1.228658in}}{\pgfqpoint{0.809208in}{1.222834in}}%
\pgfpathcurveto{\pgfqpoint{0.815032in}{1.217011in}}{\pgfqpoint{0.822932in}{1.213738in}}{\pgfqpoint{0.831169in}{1.213738in}}%
\pgfpathclose%
\pgfusepath{stroke,fill}%
\end{pgfscope}%
\begin{pgfscope}%
\pgfpathrectangle{\pgfqpoint{0.100000in}{0.220728in}}{\pgfqpoint{3.696000in}{3.696000in}}%
\pgfusepath{clip}%
\pgfsetbuttcap%
\pgfsetroundjoin%
\definecolor{currentfill}{rgb}{0.121569,0.466667,0.705882}%
\pgfsetfillcolor{currentfill}%
\pgfsetfillopacity{0.616564}%
\pgfsetlinewidth{1.003750pt}%
\definecolor{currentstroke}{rgb}{0.121569,0.466667,0.705882}%
\pgfsetstrokecolor{currentstroke}%
\pgfsetstrokeopacity{0.616564}%
\pgfsetdash{}{0pt}%
\pgfpathmoveto{\pgfqpoint{0.827725in}{1.212535in}}%
\pgfpathcurveto{\pgfqpoint{0.835961in}{1.212535in}}{\pgfqpoint{0.843862in}{1.215808in}}{\pgfqpoint{0.849685in}{1.221632in}}%
\pgfpathcurveto{\pgfqpoint{0.855509in}{1.227456in}}{\pgfqpoint{0.858782in}{1.235356in}}{\pgfqpoint{0.858782in}{1.243592in}}%
\pgfpathcurveto{\pgfqpoint{0.858782in}{1.251828in}}{\pgfqpoint{0.855509in}{1.259728in}}{\pgfqpoint{0.849685in}{1.265552in}}%
\pgfpathcurveto{\pgfqpoint{0.843862in}{1.271376in}}{\pgfqpoint{0.835961in}{1.274648in}}{\pgfqpoint{0.827725in}{1.274648in}}%
\pgfpathcurveto{\pgfqpoint{0.819489in}{1.274648in}}{\pgfqpoint{0.811589in}{1.271376in}}{\pgfqpoint{0.805765in}{1.265552in}}%
\pgfpathcurveto{\pgfqpoint{0.799941in}{1.259728in}}{\pgfqpoint{0.796669in}{1.251828in}}{\pgfqpoint{0.796669in}{1.243592in}}%
\pgfpathcurveto{\pgfqpoint{0.796669in}{1.235356in}}{\pgfqpoint{0.799941in}{1.227456in}}{\pgfqpoint{0.805765in}{1.221632in}}%
\pgfpathcurveto{\pgfqpoint{0.811589in}{1.215808in}}{\pgfqpoint{0.819489in}{1.212535in}}{\pgfqpoint{0.827725in}{1.212535in}}%
\pgfpathclose%
\pgfusepath{stroke,fill}%
\end{pgfscope}%
\begin{pgfscope}%
\pgfpathrectangle{\pgfqpoint{0.100000in}{0.220728in}}{\pgfqpoint{3.696000in}{3.696000in}}%
\pgfusepath{clip}%
\pgfsetbuttcap%
\pgfsetroundjoin%
\definecolor{currentfill}{rgb}{0.121569,0.466667,0.705882}%
\pgfsetfillcolor{currentfill}%
\pgfsetfillopacity{0.616574}%
\pgfsetlinewidth{1.003750pt}%
\definecolor{currentstroke}{rgb}{0.121569,0.466667,0.705882}%
\pgfsetstrokecolor{currentstroke}%
\pgfsetstrokeopacity{0.616574}%
\pgfsetdash{}{0pt}%
\pgfpathmoveto{\pgfqpoint{0.831115in}{1.213668in}}%
\pgfpathcurveto{\pgfqpoint{0.839351in}{1.213668in}}{\pgfqpoint{0.847251in}{1.216941in}}{\pgfqpoint{0.853075in}{1.222765in}}%
\pgfpathcurveto{\pgfqpoint{0.858899in}{1.228589in}}{\pgfqpoint{0.862171in}{1.236489in}}{\pgfqpoint{0.862171in}{1.244725in}}%
\pgfpathcurveto{\pgfqpoint{0.862171in}{1.252961in}}{\pgfqpoint{0.858899in}{1.260861in}}{\pgfqpoint{0.853075in}{1.266685in}}%
\pgfpathcurveto{\pgfqpoint{0.847251in}{1.272509in}}{\pgfqpoint{0.839351in}{1.275781in}}{\pgfqpoint{0.831115in}{1.275781in}}%
\pgfpathcurveto{\pgfqpoint{0.822879in}{1.275781in}}{\pgfqpoint{0.814979in}{1.272509in}}{\pgfqpoint{0.809155in}{1.266685in}}%
\pgfpathcurveto{\pgfqpoint{0.803331in}{1.260861in}}{\pgfqpoint{0.800058in}{1.252961in}}{\pgfqpoint{0.800058in}{1.244725in}}%
\pgfpathcurveto{\pgfqpoint{0.800058in}{1.236489in}}{\pgfqpoint{0.803331in}{1.228589in}}{\pgfqpoint{0.809155in}{1.222765in}}%
\pgfpathcurveto{\pgfqpoint{0.814979in}{1.216941in}}{\pgfqpoint{0.822879in}{1.213668in}}{\pgfqpoint{0.831115in}{1.213668in}}%
\pgfpathclose%
\pgfusepath{stroke,fill}%
\end{pgfscope}%
\begin{pgfscope}%
\pgfpathrectangle{\pgfqpoint{0.100000in}{0.220728in}}{\pgfqpoint{3.696000in}{3.696000in}}%
\pgfusepath{clip}%
\pgfsetbuttcap%
\pgfsetroundjoin%
\definecolor{currentfill}{rgb}{0.121569,0.466667,0.705882}%
\pgfsetfillcolor{currentfill}%
\pgfsetfillopacity{0.616582}%
\pgfsetlinewidth{1.003750pt}%
\definecolor{currentstroke}{rgb}{0.121569,0.466667,0.705882}%
\pgfsetstrokecolor{currentstroke}%
\pgfsetstrokeopacity{0.616582}%
\pgfsetdash{}{0pt}%
\pgfpathmoveto{\pgfqpoint{0.830483in}{1.213220in}}%
\pgfpathcurveto{\pgfqpoint{0.838719in}{1.213220in}}{\pgfqpoint{0.846619in}{1.216492in}}{\pgfqpoint{0.852443in}{1.222316in}}%
\pgfpathcurveto{\pgfqpoint{0.858267in}{1.228140in}}{\pgfqpoint{0.861539in}{1.236040in}}{\pgfqpoint{0.861539in}{1.244276in}}%
\pgfpathcurveto{\pgfqpoint{0.861539in}{1.252512in}}{\pgfqpoint{0.858267in}{1.260413in}}{\pgfqpoint{0.852443in}{1.266236in}}%
\pgfpathcurveto{\pgfqpoint{0.846619in}{1.272060in}}{\pgfqpoint{0.838719in}{1.275333in}}{\pgfqpoint{0.830483in}{1.275333in}}%
\pgfpathcurveto{\pgfqpoint{0.822247in}{1.275333in}}{\pgfqpoint{0.814347in}{1.272060in}}{\pgfqpoint{0.808523in}{1.266236in}}%
\pgfpathcurveto{\pgfqpoint{0.802699in}{1.260413in}}{\pgfqpoint{0.799426in}{1.252512in}}{\pgfqpoint{0.799426in}{1.244276in}}%
\pgfpathcurveto{\pgfqpoint{0.799426in}{1.236040in}}{\pgfqpoint{0.802699in}{1.228140in}}{\pgfqpoint{0.808523in}{1.222316in}}%
\pgfpathcurveto{\pgfqpoint{0.814347in}{1.216492in}}{\pgfqpoint{0.822247in}{1.213220in}}{\pgfqpoint{0.830483in}{1.213220in}}%
\pgfpathclose%
\pgfusepath{stroke,fill}%
\end{pgfscope}%
\begin{pgfscope}%
\pgfpathrectangle{\pgfqpoint{0.100000in}{0.220728in}}{\pgfqpoint{3.696000in}{3.696000in}}%
\pgfusepath{clip}%
\pgfsetbuttcap%
\pgfsetroundjoin%
\definecolor{currentfill}{rgb}{0.121569,0.466667,0.705882}%
\pgfsetfillcolor{currentfill}%
\pgfsetfillopacity{0.616585}%
\pgfsetlinewidth{1.003750pt}%
\definecolor{currentstroke}{rgb}{0.121569,0.466667,0.705882}%
\pgfsetstrokecolor{currentstroke}%
\pgfsetstrokeopacity{0.616585}%
\pgfsetdash{}{0pt}%
\pgfpathmoveto{\pgfqpoint{0.831087in}{1.213628in}}%
\pgfpathcurveto{\pgfqpoint{0.839323in}{1.213628in}}{\pgfqpoint{0.847223in}{1.216901in}}{\pgfqpoint{0.853047in}{1.222725in}}%
\pgfpathcurveto{\pgfqpoint{0.858871in}{1.228549in}}{\pgfqpoint{0.862143in}{1.236449in}}{\pgfqpoint{0.862143in}{1.244685in}}%
\pgfpathcurveto{\pgfqpoint{0.862143in}{1.252921in}}{\pgfqpoint{0.858871in}{1.260821in}}{\pgfqpoint{0.853047in}{1.266645in}}%
\pgfpathcurveto{\pgfqpoint{0.847223in}{1.272469in}}{\pgfqpoint{0.839323in}{1.275741in}}{\pgfqpoint{0.831087in}{1.275741in}}%
\pgfpathcurveto{\pgfqpoint{0.822850in}{1.275741in}}{\pgfqpoint{0.814950in}{1.272469in}}{\pgfqpoint{0.809126in}{1.266645in}}%
\pgfpathcurveto{\pgfqpoint{0.803302in}{1.260821in}}{\pgfqpoint{0.800030in}{1.252921in}}{\pgfqpoint{0.800030in}{1.244685in}}%
\pgfpathcurveto{\pgfqpoint{0.800030in}{1.236449in}}{\pgfqpoint{0.803302in}{1.228549in}}{\pgfqpoint{0.809126in}{1.222725in}}%
\pgfpathcurveto{\pgfqpoint{0.814950in}{1.216901in}}{\pgfqpoint{0.822850in}{1.213628in}}{\pgfqpoint{0.831087in}{1.213628in}}%
\pgfpathclose%
\pgfusepath{stroke,fill}%
\end{pgfscope}%
\begin{pgfscope}%
\pgfpathrectangle{\pgfqpoint{0.100000in}{0.220728in}}{\pgfqpoint{3.696000in}{3.696000in}}%
\pgfusepath{clip}%
\pgfsetbuttcap%
\pgfsetroundjoin%
\definecolor{currentfill}{rgb}{0.121569,0.466667,0.705882}%
\pgfsetfillcolor{currentfill}%
\pgfsetfillopacity{0.616589}%
\pgfsetlinewidth{1.003750pt}%
\definecolor{currentstroke}{rgb}{0.121569,0.466667,0.705882}%
\pgfsetstrokecolor{currentstroke}%
\pgfsetstrokeopacity{0.616589}%
\pgfsetdash{}{0pt}%
\pgfpathmoveto{\pgfqpoint{0.831071in}{1.213603in}}%
\pgfpathcurveto{\pgfqpoint{0.839308in}{1.213603in}}{\pgfqpoint{0.847208in}{1.216875in}}{\pgfqpoint{0.853032in}{1.222699in}}%
\pgfpathcurveto{\pgfqpoint{0.858855in}{1.228523in}}{\pgfqpoint{0.862128in}{1.236423in}}{\pgfqpoint{0.862128in}{1.244660in}}%
\pgfpathcurveto{\pgfqpoint{0.862128in}{1.252896in}}{\pgfqpoint{0.858855in}{1.260796in}}{\pgfqpoint{0.853032in}{1.266620in}}%
\pgfpathcurveto{\pgfqpoint{0.847208in}{1.272444in}}{\pgfqpoint{0.839308in}{1.275716in}}{\pgfqpoint{0.831071in}{1.275716in}}%
\pgfpathcurveto{\pgfqpoint{0.822835in}{1.275716in}}{\pgfqpoint{0.814935in}{1.272444in}}{\pgfqpoint{0.809111in}{1.266620in}}%
\pgfpathcurveto{\pgfqpoint{0.803287in}{1.260796in}}{\pgfqpoint{0.800015in}{1.252896in}}{\pgfqpoint{0.800015in}{1.244660in}}%
\pgfpathcurveto{\pgfqpoint{0.800015in}{1.236423in}}{\pgfqpoint{0.803287in}{1.228523in}}{\pgfqpoint{0.809111in}{1.222699in}}%
\pgfpathcurveto{\pgfqpoint{0.814935in}{1.216875in}}{\pgfqpoint{0.822835in}{1.213603in}}{\pgfqpoint{0.831071in}{1.213603in}}%
\pgfpathclose%
\pgfusepath{stroke,fill}%
\end{pgfscope}%
\begin{pgfscope}%
\pgfpathrectangle{\pgfqpoint{0.100000in}{0.220728in}}{\pgfqpoint{3.696000in}{3.696000in}}%
\pgfusepath{clip}%
\pgfsetbuttcap%
\pgfsetroundjoin%
\definecolor{currentfill}{rgb}{0.121569,0.466667,0.705882}%
\pgfsetfillcolor{currentfill}%
\pgfsetfillopacity{0.616592}%
\pgfsetlinewidth{1.003750pt}%
\definecolor{currentstroke}{rgb}{0.121569,0.466667,0.705882}%
\pgfsetstrokecolor{currentstroke}%
\pgfsetstrokeopacity{0.616592}%
\pgfsetdash{}{0pt}%
\pgfpathmoveto{\pgfqpoint{0.831063in}{1.213590in}}%
\pgfpathcurveto{\pgfqpoint{0.839299in}{1.213590in}}{\pgfqpoint{0.847199in}{1.216862in}}{\pgfqpoint{0.853023in}{1.222686in}}%
\pgfpathcurveto{\pgfqpoint{0.858847in}{1.228510in}}{\pgfqpoint{0.862119in}{1.236410in}}{\pgfqpoint{0.862119in}{1.244646in}}%
\pgfpathcurveto{\pgfqpoint{0.862119in}{1.252883in}}{\pgfqpoint{0.858847in}{1.260783in}}{\pgfqpoint{0.853023in}{1.266607in}}%
\pgfpathcurveto{\pgfqpoint{0.847199in}{1.272431in}}{\pgfqpoint{0.839299in}{1.275703in}}{\pgfqpoint{0.831063in}{1.275703in}}%
\pgfpathcurveto{\pgfqpoint{0.822826in}{1.275703in}}{\pgfqpoint{0.814926in}{1.272431in}}{\pgfqpoint{0.809103in}{1.266607in}}%
\pgfpathcurveto{\pgfqpoint{0.803279in}{1.260783in}}{\pgfqpoint{0.800006in}{1.252883in}}{\pgfqpoint{0.800006in}{1.244646in}}%
\pgfpathcurveto{\pgfqpoint{0.800006in}{1.236410in}}{\pgfqpoint{0.803279in}{1.228510in}}{\pgfqpoint{0.809103in}{1.222686in}}%
\pgfpathcurveto{\pgfqpoint{0.814926in}{1.216862in}}{\pgfqpoint{0.822826in}{1.213590in}}{\pgfqpoint{0.831063in}{1.213590in}}%
\pgfpathclose%
\pgfusepath{stroke,fill}%
\end{pgfscope}%
\begin{pgfscope}%
\pgfpathrectangle{\pgfqpoint{0.100000in}{0.220728in}}{\pgfqpoint{3.696000in}{3.696000in}}%
\pgfusepath{clip}%
\pgfsetbuttcap%
\pgfsetroundjoin%
\definecolor{currentfill}{rgb}{0.121569,0.466667,0.705882}%
\pgfsetfillcolor{currentfill}%
\pgfsetfillopacity{0.616594}%
\pgfsetlinewidth{1.003750pt}%
\definecolor{currentstroke}{rgb}{0.121569,0.466667,0.705882}%
\pgfsetstrokecolor{currentstroke}%
\pgfsetstrokeopacity{0.616594}%
\pgfsetdash{}{0pt}%
\pgfpathmoveto{\pgfqpoint{0.831058in}{1.213583in}}%
\pgfpathcurveto{\pgfqpoint{0.839294in}{1.213583in}}{\pgfqpoint{0.847194in}{1.216855in}}{\pgfqpoint{0.853018in}{1.222679in}}%
\pgfpathcurveto{\pgfqpoint{0.858842in}{1.228503in}}{\pgfqpoint{0.862114in}{1.236403in}}{\pgfqpoint{0.862114in}{1.244639in}}%
\pgfpathcurveto{\pgfqpoint{0.862114in}{1.252876in}}{\pgfqpoint{0.858842in}{1.260776in}}{\pgfqpoint{0.853018in}{1.266600in}}%
\pgfpathcurveto{\pgfqpoint{0.847194in}{1.272424in}}{\pgfqpoint{0.839294in}{1.275696in}}{\pgfqpoint{0.831058in}{1.275696in}}%
\pgfpathcurveto{\pgfqpoint{0.822822in}{1.275696in}}{\pgfqpoint{0.814922in}{1.272424in}}{\pgfqpoint{0.809098in}{1.266600in}}%
\pgfpathcurveto{\pgfqpoint{0.803274in}{1.260776in}}{\pgfqpoint{0.800001in}{1.252876in}}{\pgfqpoint{0.800001in}{1.244639in}}%
\pgfpathcurveto{\pgfqpoint{0.800001in}{1.236403in}}{\pgfqpoint{0.803274in}{1.228503in}}{\pgfqpoint{0.809098in}{1.222679in}}%
\pgfpathcurveto{\pgfqpoint{0.814922in}{1.216855in}}{\pgfqpoint{0.822822in}{1.213583in}}{\pgfqpoint{0.831058in}{1.213583in}}%
\pgfpathclose%
\pgfusepath{stroke,fill}%
\end{pgfscope}%
\begin{pgfscope}%
\pgfpathrectangle{\pgfqpoint{0.100000in}{0.220728in}}{\pgfqpoint{3.696000in}{3.696000in}}%
\pgfusepath{clip}%
\pgfsetbuttcap%
\pgfsetroundjoin%
\definecolor{currentfill}{rgb}{0.121569,0.466667,0.705882}%
\pgfsetfillcolor{currentfill}%
\pgfsetfillopacity{0.616595}%
\pgfsetlinewidth{1.003750pt}%
\definecolor{currentstroke}{rgb}{0.121569,0.466667,0.705882}%
\pgfsetstrokecolor{currentstroke}%
\pgfsetstrokeopacity{0.616595}%
\pgfsetdash{}{0pt}%
\pgfpathmoveto{\pgfqpoint{0.831055in}{1.213579in}}%
\pgfpathcurveto{\pgfqpoint{0.839292in}{1.213579in}}{\pgfqpoint{0.847192in}{1.216851in}}{\pgfqpoint{0.853016in}{1.222675in}}%
\pgfpathcurveto{\pgfqpoint{0.858840in}{1.228499in}}{\pgfqpoint{0.862112in}{1.236399in}}{\pgfqpoint{0.862112in}{1.244636in}}%
\pgfpathcurveto{\pgfqpoint{0.862112in}{1.252872in}}{\pgfqpoint{0.858840in}{1.260772in}}{\pgfqpoint{0.853016in}{1.266596in}}%
\pgfpathcurveto{\pgfqpoint{0.847192in}{1.272420in}}{\pgfqpoint{0.839292in}{1.275692in}}{\pgfqpoint{0.831055in}{1.275692in}}%
\pgfpathcurveto{\pgfqpoint{0.822819in}{1.275692in}}{\pgfqpoint{0.814919in}{1.272420in}}{\pgfqpoint{0.809095in}{1.266596in}}%
\pgfpathcurveto{\pgfqpoint{0.803271in}{1.260772in}}{\pgfqpoint{0.799999in}{1.252872in}}{\pgfqpoint{0.799999in}{1.244636in}}%
\pgfpathcurveto{\pgfqpoint{0.799999in}{1.236399in}}{\pgfqpoint{0.803271in}{1.228499in}}{\pgfqpoint{0.809095in}{1.222675in}}%
\pgfpathcurveto{\pgfqpoint{0.814919in}{1.216851in}}{\pgfqpoint{0.822819in}{1.213579in}}{\pgfqpoint{0.831055in}{1.213579in}}%
\pgfpathclose%
\pgfusepath{stroke,fill}%
\end{pgfscope}%
\begin{pgfscope}%
\pgfpathrectangle{\pgfqpoint{0.100000in}{0.220728in}}{\pgfqpoint{3.696000in}{3.696000in}}%
\pgfusepath{clip}%
\pgfsetbuttcap%
\pgfsetroundjoin%
\definecolor{currentfill}{rgb}{0.121569,0.466667,0.705882}%
\pgfsetfillcolor{currentfill}%
\pgfsetfillopacity{0.616595}%
\pgfsetlinewidth{1.003750pt}%
\definecolor{currentstroke}{rgb}{0.121569,0.466667,0.705882}%
\pgfsetstrokecolor{currentstroke}%
\pgfsetstrokeopacity{0.616595}%
\pgfsetdash{}{0pt}%
\pgfpathmoveto{\pgfqpoint{0.831054in}{1.213577in}}%
\pgfpathcurveto{\pgfqpoint{0.839290in}{1.213577in}}{\pgfqpoint{0.847190in}{1.216849in}}{\pgfqpoint{0.853014in}{1.222673in}}%
\pgfpathcurveto{\pgfqpoint{0.858838in}{1.228497in}}{\pgfqpoint{0.862110in}{1.236397in}}{\pgfqpoint{0.862110in}{1.244633in}}%
\pgfpathcurveto{\pgfqpoint{0.862110in}{1.252870in}}{\pgfqpoint{0.858838in}{1.260770in}}{\pgfqpoint{0.853014in}{1.266594in}}%
\pgfpathcurveto{\pgfqpoint{0.847190in}{1.272418in}}{\pgfqpoint{0.839290in}{1.275690in}}{\pgfqpoint{0.831054in}{1.275690in}}%
\pgfpathcurveto{\pgfqpoint{0.822818in}{1.275690in}}{\pgfqpoint{0.814918in}{1.272418in}}{\pgfqpoint{0.809094in}{1.266594in}}%
\pgfpathcurveto{\pgfqpoint{0.803270in}{1.260770in}}{\pgfqpoint{0.799997in}{1.252870in}}{\pgfqpoint{0.799997in}{1.244633in}}%
\pgfpathcurveto{\pgfqpoint{0.799997in}{1.236397in}}{\pgfqpoint{0.803270in}{1.228497in}}{\pgfqpoint{0.809094in}{1.222673in}}%
\pgfpathcurveto{\pgfqpoint{0.814918in}{1.216849in}}{\pgfqpoint{0.822818in}{1.213577in}}{\pgfqpoint{0.831054in}{1.213577in}}%
\pgfpathclose%
\pgfusepath{stroke,fill}%
\end{pgfscope}%
\begin{pgfscope}%
\pgfpathrectangle{\pgfqpoint{0.100000in}{0.220728in}}{\pgfqpoint{3.696000in}{3.696000in}}%
\pgfusepath{clip}%
\pgfsetbuttcap%
\pgfsetroundjoin%
\definecolor{currentfill}{rgb}{0.121569,0.466667,0.705882}%
\pgfsetfillcolor{currentfill}%
\pgfsetfillopacity{0.616595}%
\pgfsetlinewidth{1.003750pt}%
\definecolor{currentstroke}{rgb}{0.121569,0.466667,0.705882}%
\pgfsetstrokecolor{currentstroke}%
\pgfsetstrokeopacity{0.616595}%
\pgfsetdash{}{0pt}%
\pgfpathmoveto{\pgfqpoint{0.831053in}{1.213576in}}%
\pgfpathcurveto{\pgfqpoint{0.839289in}{1.213576in}}{\pgfqpoint{0.847190in}{1.216848in}}{\pgfqpoint{0.853013in}{1.222672in}}%
\pgfpathcurveto{\pgfqpoint{0.858837in}{1.228496in}}{\pgfqpoint{0.862110in}{1.236396in}}{\pgfqpoint{0.862110in}{1.244632in}}%
\pgfpathcurveto{\pgfqpoint{0.862110in}{1.252868in}}{\pgfqpoint{0.858837in}{1.260768in}}{\pgfqpoint{0.853013in}{1.266592in}}%
\pgfpathcurveto{\pgfqpoint{0.847190in}{1.272416in}}{\pgfqpoint{0.839289in}{1.275689in}}{\pgfqpoint{0.831053in}{1.275689in}}%
\pgfpathcurveto{\pgfqpoint{0.822817in}{1.275689in}}{\pgfqpoint{0.814917in}{1.272416in}}{\pgfqpoint{0.809093in}{1.266592in}}%
\pgfpathcurveto{\pgfqpoint{0.803269in}{1.260768in}}{\pgfqpoint{0.799997in}{1.252868in}}{\pgfqpoint{0.799997in}{1.244632in}}%
\pgfpathcurveto{\pgfqpoint{0.799997in}{1.236396in}}{\pgfqpoint{0.803269in}{1.228496in}}{\pgfqpoint{0.809093in}{1.222672in}}%
\pgfpathcurveto{\pgfqpoint{0.814917in}{1.216848in}}{\pgfqpoint{0.822817in}{1.213576in}}{\pgfqpoint{0.831053in}{1.213576in}}%
\pgfpathclose%
\pgfusepath{stroke,fill}%
\end{pgfscope}%
\begin{pgfscope}%
\pgfpathrectangle{\pgfqpoint{0.100000in}{0.220728in}}{\pgfqpoint{3.696000in}{3.696000in}}%
\pgfusepath{clip}%
\pgfsetbuttcap%
\pgfsetroundjoin%
\definecolor{currentfill}{rgb}{0.121569,0.466667,0.705882}%
\pgfsetfillcolor{currentfill}%
\pgfsetfillopacity{0.616596}%
\pgfsetlinewidth{1.003750pt}%
\definecolor{currentstroke}{rgb}{0.121569,0.466667,0.705882}%
\pgfsetstrokecolor{currentstroke}%
\pgfsetstrokeopacity{0.616596}%
\pgfsetdash{}{0pt}%
\pgfpathmoveto{\pgfqpoint{0.831053in}{1.213575in}}%
\pgfpathcurveto{\pgfqpoint{0.839289in}{1.213575in}}{\pgfqpoint{0.847189in}{1.216847in}}{\pgfqpoint{0.853013in}{1.222671in}}%
\pgfpathcurveto{\pgfqpoint{0.858837in}{1.228495in}}{\pgfqpoint{0.862109in}{1.236395in}}{\pgfqpoint{0.862109in}{1.244631in}}%
\pgfpathcurveto{\pgfqpoint{0.862109in}{1.252868in}}{\pgfqpoint{0.858837in}{1.260768in}}{\pgfqpoint{0.853013in}{1.266592in}}%
\pgfpathcurveto{\pgfqpoint{0.847189in}{1.272416in}}{\pgfqpoint{0.839289in}{1.275688in}}{\pgfqpoint{0.831053in}{1.275688in}}%
\pgfpathcurveto{\pgfqpoint{0.822816in}{1.275688in}}{\pgfqpoint{0.814916in}{1.272416in}}{\pgfqpoint{0.809092in}{1.266592in}}%
\pgfpathcurveto{\pgfqpoint{0.803269in}{1.260768in}}{\pgfqpoint{0.799996in}{1.252868in}}{\pgfqpoint{0.799996in}{1.244631in}}%
\pgfpathcurveto{\pgfqpoint{0.799996in}{1.236395in}}{\pgfqpoint{0.803269in}{1.228495in}}{\pgfqpoint{0.809092in}{1.222671in}}%
\pgfpathcurveto{\pgfqpoint{0.814916in}{1.216847in}}{\pgfqpoint{0.822816in}{1.213575in}}{\pgfqpoint{0.831053in}{1.213575in}}%
\pgfpathclose%
\pgfusepath{stroke,fill}%
\end{pgfscope}%
\begin{pgfscope}%
\pgfpathrectangle{\pgfqpoint{0.100000in}{0.220728in}}{\pgfqpoint{3.696000in}{3.696000in}}%
\pgfusepath{clip}%
\pgfsetbuttcap%
\pgfsetroundjoin%
\definecolor{currentfill}{rgb}{0.121569,0.466667,0.705882}%
\pgfsetfillcolor{currentfill}%
\pgfsetfillopacity{0.616596}%
\pgfsetlinewidth{1.003750pt}%
\definecolor{currentstroke}{rgb}{0.121569,0.466667,0.705882}%
\pgfsetstrokecolor{currentstroke}%
\pgfsetstrokeopacity{0.616596}%
\pgfsetdash{}{0pt}%
\pgfpathmoveto{\pgfqpoint{0.831053in}{1.213575in}}%
\pgfpathcurveto{\pgfqpoint{0.839289in}{1.213575in}}{\pgfqpoint{0.847189in}{1.216847in}}{\pgfqpoint{0.853013in}{1.222671in}}%
\pgfpathcurveto{\pgfqpoint{0.858837in}{1.228495in}}{\pgfqpoint{0.862109in}{1.236395in}}{\pgfqpoint{0.862109in}{1.244631in}}%
\pgfpathcurveto{\pgfqpoint{0.862109in}{1.252867in}}{\pgfqpoint{0.858837in}{1.260767in}}{\pgfqpoint{0.853013in}{1.266591in}}%
\pgfpathcurveto{\pgfqpoint{0.847189in}{1.272415in}}{\pgfqpoint{0.839289in}{1.275688in}}{\pgfqpoint{0.831053in}{1.275688in}}%
\pgfpathcurveto{\pgfqpoint{0.822816in}{1.275688in}}{\pgfqpoint{0.814916in}{1.272415in}}{\pgfqpoint{0.809092in}{1.266591in}}%
\pgfpathcurveto{\pgfqpoint{0.803268in}{1.260767in}}{\pgfqpoint{0.799996in}{1.252867in}}{\pgfqpoint{0.799996in}{1.244631in}}%
\pgfpathcurveto{\pgfqpoint{0.799996in}{1.236395in}}{\pgfqpoint{0.803268in}{1.228495in}}{\pgfqpoint{0.809092in}{1.222671in}}%
\pgfpathcurveto{\pgfqpoint{0.814916in}{1.216847in}}{\pgfqpoint{0.822816in}{1.213575in}}{\pgfqpoint{0.831053in}{1.213575in}}%
\pgfpathclose%
\pgfusepath{stroke,fill}%
\end{pgfscope}%
\begin{pgfscope}%
\pgfpathrectangle{\pgfqpoint{0.100000in}{0.220728in}}{\pgfqpoint{3.696000in}{3.696000in}}%
\pgfusepath{clip}%
\pgfsetbuttcap%
\pgfsetroundjoin%
\definecolor{currentfill}{rgb}{0.121569,0.466667,0.705882}%
\pgfsetfillcolor{currentfill}%
\pgfsetfillopacity{0.616596}%
\pgfsetlinewidth{1.003750pt}%
\definecolor{currentstroke}{rgb}{0.121569,0.466667,0.705882}%
\pgfsetstrokecolor{currentstroke}%
\pgfsetstrokeopacity{0.616596}%
\pgfsetdash{}{0pt}%
\pgfpathmoveto{\pgfqpoint{0.831052in}{1.213574in}}%
\pgfpathcurveto{\pgfqpoint{0.839289in}{1.213574in}}{\pgfqpoint{0.847189in}{1.216847in}}{\pgfqpoint{0.853013in}{1.222671in}}%
\pgfpathcurveto{\pgfqpoint{0.858837in}{1.228495in}}{\pgfqpoint{0.862109in}{1.236395in}}{\pgfqpoint{0.862109in}{1.244631in}}%
\pgfpathcurveto{\pgfqpoint{0.862109in}{1.252867in}}{\pgfqpoint{0.858837in}{1.260767in}}{\pgfqpoint{0.853013in}{1.266591in}}%
\pgfpathcurveto{\pgfqpoint{0.847189in}{1.272415in}}{\pgfqpoint{0.839289in}{1.275687in}}{\pgfqpoint{0.831052in}{1.275687in}}%
\pgfpathcurveto{\pgfqpoint{0.822816in}{1.275687in}}{\pgfqpoint{0.814916in}{1.272415in}}{\pgfqpoint{0.809092in}{1.266591in}}%
\pgfpathcurveto{\pgfqpoint{0.803268in}{1.260767in}}{\pgfqpoint{0.799996in}{1.252867in}}{\pgfqpoint{0.799996in}{1.244631in}}%
\pgfpathcurveto{\pgfqpoint{0.799996in}{1.236395in}}{\pgfqpoint{0.803268in}{1.228495in}}{\pgfqpoint{0.809092in}{1.222671in}}%
\pgfpathcurveto{\pgfqpoint{0.814916in}{1.216847in}}{\pgfqpoint{0.822816in}{1.213574in}}{\pgfqpoint{0.831052in}{1.213574in}}%
\pgfpathclose%
\pgfusepath{stroke,fill}%
\end{pgfscope}%
\begin{pgfscope}%
\pgfpathrectangle{\pgfqpoint{0.100000in}{0.220728in}}{\pgfqpoint{3.696000in}{3.696000in}}%
\pgfusepath{clip}%
\pgfsetbuttcap%
\pgfsetroundjoin%
\definecolor{currentfill}{rgb}{0.121569,0.466667,0.705882}%
\pgfsetfillcolor{currentfill}%
\pgfsetfillopacity{0.616596}%
\pgfsetlinewidth{1.003750pt}%
\definecolor{currentstroke}{rgb}{0.121569,0.466667,0.705882}%
\pgfsetstrokecolor{currentstroke}%
\pgfsetstrokeopacity{0.616596}%
\pgfsetdash{}{0pt}%
\pgfpathmoveto{\pgfqpoint{0.831052in}{1.213574in}}%
\pgfpathcurveto{\pgfqpoint{0.839289in}{1.213574in}}{\pgfqpoint{0.847189in}{1.216847in}}{\pgfqpoint{0.853013in}{1.222671in}}%
\pgfpathcurveto{\pgfqpoint{0.858836in}{1.228494in}}{\pgfqpoint{0.862109in}{1.236394in}}{\pgfqpoint{0.862109in}{1.244631in}}%
\pgfpathcurveto{\pgfqpoint{0.862109in}{1.252867in}}{\pgfqpoint{0.858836in}{1.260767in}}{\pgfqpoint{0.853013in}{1.266591in}}%
\pgfpathcurveto{\pgfqpoint{0.847189in}{1.272415in}}{\pgfqpoint{0.839289in}{1.275687in}}{\pgfqpoint{0.831052in}{1.275687in}}%
\pgfpathcurveto{\pgfqpoint{0.822816in}{1.275687in}}{\pgfqpoint{0.814916in}{1.272415in}}{\pgfqpoint{0.809092in}{1.266591in}}%
\pgfpathcurveto{\pgfqpoint{0.803268in}{1.260767in}}{\pgfqpoint{0.799996in}{1.252867in}}{\pgfqpoint{0.799996in}{1.244631in}}%
\pgfpathcurveto{\pgfqpoint{0.799996in}{1.236394in}}{\pgfqpoint{0.803268in}{1.228494in}}{\pgfqpoint{0.809092in}{1.222671in}}%
\pgfpathcurveto{\pgfqpoint{0.814916in}{1.216847in}}{\pgfqpoint{0.822816in}{1.213574in}}{\pgfqpoint{0.831052in}{1.213574in}}%
\pgfpathclose%
\pgfusepath{stroke,fill}%
\end{pgfscope}%
\begin{pgfscope}%
\pgfpathrectangle{\pgfqpoint{0.100000in}{0.220728in}}{\pgfqpoint{3.696000in}{3.696000in}}%
\pgfusepath{clip}%
\pgfsetbuttcap%
\pgfsetroundjoin%
\definecolor{currentfill}{rgb}{0.121569,0.466667,0.705882}%
\pgfsetfillcolor{currentfill}%
\pgfsetfillopacity{0.616596}%
\pgfsetlinewidth{1.003750pt}%
\definecolor{currentstroke}{rgb}{0.121569,0.466667,0.705882}%
\pgfsetstrokecolor{currentstroke}%
\pgfsetstrokeopacity{0.616596}%
\pgfsetdash{}{0pt}%
\pgfpathmoveto{\pgfqpoint{0.831052in}{1.213574in}}%
\pgfpathcurveto{\pgfqpoint{0.839289in}{1.213574in}}{\pgfqpoint{0.847189in}{1.216847in}}{\pgfqpoint{0.853013in}{1.222670in}}%
\pgfpathcurveto{\pgfqpoint{0.858836in}{1.228494in}}{\pgfqpoint{0.862109in}{1.236394in}}{\pgfqpoint{0.862109in}{1.244631in}}%
\pgfpathcurveto{\pgfqpoint{0.862109in}{1.252867in}}{\pgfqpoint{0.858836in}{1.260767in}}{\pgfqpoint{0.853013in}{1.266591in}}%
\pgfpathcurveto{\pgfqpoint{0.847189in}{1.272415in}}{\pgfqpoint{0.839289in}{1.275687in}}{\pgfqpoint{0.831052in}{1.275687in}}%
\pgfpathcurveto{\pgfqpoint{0.822816in}{1.275687in}}{\pgfqpoint{0.814916in}{1.272415in}}{\pgfqpoint{0.809092in}{1.266591in}}%
\pgfpathcurveto{\pgfqpoint{0.803268in}{1.260767in}}{\pgfqpoint{0.799996in}{1.252867in}}{\pgfqpoint{0.799996in}{1.244631in}}%
\pgfpathcurveto{\pgfqpoint{0.799996in}{1.236394in}}{\pgfqpoint{0.803268in}{1.228494in}}{\pgfqpoint{0.809092in}{1.222670in}}%
\pgfpathcurveto{\pgfqpoint{0.814916in}{1.216847in}}{\pgfqpoint{0.822816in}{1.213574in}}{\pgfqpoint{0.831052in}{1.213574in}}%
\pgfpathclose%
\pgfusepath{stroke,fill}%
\end{pgfscope}%
\begin{pgfscope}%
\pgfpathrectangle{\pgfqpoint{0.100000in}{0.220728in}}{\pgfqpoint{3.696000in}{3.696000in}}%
\pgfusepath{clip}%
\pgfsetbuttcap%
\pgfsetroundjoin%
\definecolor{currentfill}{rgb}{0.121569,0.466667,0.705882}%
\pgfsetfillcolor{currentfill}%
\pgfsetfillopacity{0.616596}%
\pgfsetlinewidth{1.003750pt}%
\definecolor{currentstroke}{rgb}{0.121569,0.466667,0.705882}%
\pgfsetstrokecolor{currentstroke}%
\pgfsetstrokeopacity{0.616596}%
\pgfsetdash{}{0pt}%
\pgfpathmoveto{\pgfqpoint{0.831052in}{1.213574in}}%
\pgfpathcurveto{\pgfqpoint{0.839289in}{1.213574in}}{\pgfqpoint{0.847189in}{1.216846in}}{\pgfqpoint{0.853012in}{1.222670in}}%
\pgfpathcurveto{\pgfqpoint{0.858836in}{1.228494in}}{\pgfqpoint{0.862109in}{1.236394in}}{\pgfqpoint{0.862109in}{1.244631in}}%
\pgfpathcurveto{\pgfqpoint{0.862109in}{1.252867in}}{\pgfqpoint{0.858836in}{1.260767in}}{\pgfqpoint{0.853012in}{1.266591in}}%
\pgfpathcurveto{\pgfqpoint{0.847189in}{1.272415in}}{\pgfqpoint{0.839289in}{1.275687in}}{\pgfqpoint{0.831052in}{1.275687in}}%
\pgfpathcurveto{\pgfqpoint{0.822816in}{1.275687in}}{\pgfqpoint{0.814916in}{1.272415in}}{\pgfqpoint{0.809092in}{1.266591in}}%
\pgfpathcurveto{\pgfqpoint{0.803268in}{1.260767in}}{\pgfqpoint{0.799996in}{1.252867in}}{\pgfqpoint{0.799996in}{1.244631in}}%
\pgfpathcurveto{\pgfqpoint{0.799996in}{1.236394in}}{\pgfqpoint{0.803268in}{1.228494in}}{\pgfqpoint{0.809092in}{1.222670in}}%
\pgfpathcurveto{\pgfqpoint{0.814916in}{1.216846in}}{\pgfqpoint{0.822816in}{1.213574in}}{\pgfqpoint{0.831052in}{1.213574in}}%
\pgfpathclose%
\pgfusepath{stroke,fill}%
\end{pgfscope}%
\begin{pgfscope}%
\pgfpathrectangle{\pgfqpoint{0.100000in}{0.220728in}}{\pgfqpoint{3.696000in}{3.696000in}}%
\pgfusepath{clip}%
\pgfsetbuttcap%
\pgfsetroundjoin%
\definecolor{currentfill}{rgb}{0.121569,0.466667,0.705882}%
\pgfsetfillcolor{currentfill}%
\pgfsetfillopacity{0.616596}%
\pgfsetlinewidth{1.003750pt}%
\definecolor{currentstroke}{rgb}{0.121569,0.466667,0.705882}%
\pgfsetstrokecolor{currentstroke}%
\pgfsetstrokeopacity{0.616596}%
\pgfsetdash{}{0pt}%
\pgfpathmoveto{\pgfqpoint{0.831052in}{1.213574in}}%
\pgfpathcurveto{\pgfqpoint{0.839289in}{1.213574in}}{\pgfqpoint{0.847189in}{1.216846in}}{\pgfqpoint{0.853012in}{1.222670in}}%
\pgfpathcurveto{\pgfqpoint{0.858836in}{1.228494in}}{\pgfqpoint{0.862109in}{1.236394in}}{\pgfqpoint{0.862109in}{1.244631in}}%
\pgfpathcurveto{\pgfqpoint{0.862109in}{1.252867in}}{\pgfqpoint{0.858836in}{1.260767in}}{\pgfqpoint{0.853012in}{1.266591in}}%
\pgfpathcurveto{\pgfqpoint{0.847189in}{1.272415in}}{\pgfqpoint{0.839289in}{1.275687in}}{\pgfqpoint{0.831052in}{1.275687in}}%
\pgfpathcurveto{\pgfqpoint{0.822816in}{1.275687in}}{\pgfqpoint{0.814916in}{1.272415in}}{\pgfqpoint{0.809092in}{1.266591in}}%
\pgfpathcurveto{\pgfqpoint{0.803268in}{1.260767in}}{\pgfqpoint{0.799996in}{1.252867in}}{\pgfqpoint{0.799996in}{1.244631in}}%
\pgfpathcurveto{\pgfqpoint{0.799996in}{1.236394in}}{\pgfqpoint{0.803268in}{1.228494in}}{\pgfqpoint{0.809092in}{1.222670in}}%
\pgfpathcurveto{\pgfqpoint{0.814916in}{1.216846in}}{\pgfqpoint{0.822816in}{1.213574in}}{\pgfqpoint{0.831052in}{1.213574in}}%
\pgfpathclose%
\pgfusepath{stroke,fill}%
\end{pgfscope}%
\begin{pgfscope}%
\pgfpathrectangle{\pgfqpoint{0.100000in}{0.220728in}}{\pgfqpoint{3.696000in}{3.696000in}}%
\pgfusepath{clip}%
\pgfsetbuttcap%
\pgfsetroundjoin%
\definecolor{currentfill}{rgb}{0.121569,0.466667,0.705882}%
\pgfsetfillcolor{currentfill}%
\pgfsetfillopacity{0.616596}%
\pgfsetlinewidth{1.003750pt}%
\definecolor{currentstroke}{rgb}{0.121569,0.466667,0.705882}%
\pgfsetstrokecolor{currentstroke}%
\pgfsetstrokeopacity{0.616596}%
\pgfsetdash{}{0pt}%
\pgfpathmoveto{\pgfqpoint{0.831052in}{1.213574in}}%
\pgfpathcurveto{\pgfqpoint{0.839288in}{1.213574in}}{\pgfqpoint{0.847189in}{1.216846in}}{\pgfqpoint{0.853012in}{1.222670in}}%
\pgfpathcurveto{\pgfqpoint{0.858836in}{1.228494in}}{\pgfqpoint{0.862109in}{1.236394in}}{\pgfqpoint{0.862109in}{1.244631in}}%
\pgfpathcurveto{\pgfqpoint{0.862109in}{1.252867in}}{\pgfqpoint{0.858836in}{1.260767in}}{\pgfqpoint{0.853012in}{1.266591in}}%
\pgfpathcurveto{\pgfqpoint{0.847189in}{1.272415in}}{\pgfqpoint{0.839288in}{1.275687in}}{\pgfqpoint{0.831052in}{1.275687in}}%
\pgfpathcurveto{\pgfqpoint{0.822816in}{1.275687in}}{\pgfqpoint{0.814916in}{1.272415in}}{\pgfqpoint{0.809092in}{1.266591in}}%
\pgfpathcurveto{\pgfqpoint{0.803268in}{1.260767in}}{\pgfqpoint{0.799996in}{1.252867in}}{\pgfqpoint{0.799996in}{1.244631in}}%
\pgfpathcurveto{\pgfqpoint{0.799996in}{1.236394in}}{\pgfqpoint{0.803268in}{1.228494in}}{\pgfqpoint{0.809092in}{1.222670in}}%
\pgfpathcurveto{\pgfqpoint{0.814916in}{1.216846in}}{\pgfqpoint{0.822816in}{1.213574in}}{\pgfqpoint{0.831052in}{1.213574in}}%
\pgfpathclose%
\pgfusepath{stroke,fill}%
\end{pgfscope}%
\begin{pgfscope}%
\pgfpathrectangle{\pgfqpoint{0.100000in}{0.220728in}}{\pgfqpoint{3.696000in}{3.696000in}}%
\pgfusepath{clip}%
\pgfsetbuttcap%
\pgfsetroundjoin%
\definecolor{currentfill}{rgb}{0.121569,0.466667,0.705882}%
\pgfsetfillcolor{currentfill}%
\pgfsetfillopacity{0.616596}%
\pgfsetlinewidth{1.003750pt}%
\definecolor{currentstroke}{rgb}{0.121569,0.466667,0.705882}%
\pgfsetstrokecolor{currentstroke}%
\pgfsetstrokeopacity{0.616596}%
\pgfsetdash{}{0pt}%
\pgfpathmoveto{\pgfqpoint{0.831052in}{1.213574in}}%
\pgfpathcurveto{\pgfqpoint{0.839288in}{1.213574in}}{\pgfqpoint{0.847189in}{1.216846in}}{\pgfqpoint{0.853012in}{1.222670in}}%
\pgfpathcurveto{\pgfqpoint{0.858836in}{1.228494in}}{\pgfqpoint{0.862109in}{1.236394in}}{\pgfqpoint{0.862109in}{1.244631in}}%
\pgfpathcurveto{\pgfqpoint{0.862109in}{1.252867in}}{\pgfqpoint{0.858836in}{1.260767in}}{\pgfqpoint{0.853012in}{1.266591in}}%
\pgfpathcurveto{\pgfqpoint{0.847189in}{1.272415in}}{\pgfqpoint{0.839288in}{1.275687in}}{\pgfqpoint{0.831052in}{1.275687in}}%
\pgfpathcurveto{\pgfqpoint{0.822816in}{1.275687in}}{\pgfqpoint{0.814916in}{1.272415in}}{\pgfqpoint{0.809092in}{1.266591in}}%
\pgfpathcurveto{\pgfqpoint{0.803268in}{1.260767in}}{\pgfqpoint{0.799996in}{1.252867in}}{\pgfqpoint{0.799996in}{1.244631in}}%
\pgfpathcurveto{\pgfqpoint{0.799996in}{1.236394in}}{\pgfqpoint{0.803268in}{1.228494in}}{\pgfqpoint{0.809092in}{1.222670in}}%
\pgfpathcurveto{\pgfqpoint{0.814916in}{1.216846in}}{\pgfqpoint{0.822816in}{1.213574in}}{\pgfqpoint{0.831052in}{1.213574in}}%
\pgfpathclose%
\pgfusepath{stroke,fill}%
\end{pgfscope}%
\begin{pgfscope}%
\pgfpathrectangle{\pgfqpoint{0.100000in}{0.220728in}}{\pgfqpoint{3.696000in}{3.696000in}}%
\pgfusepath{clip}%
\pgfsetbuttcap%
\pgfsetroundjoin%
\definecolor{currentfill}{rgb}{0.121569,0.466667,0.705882}%
\pgfsetfillcolor{currentfill}%
\pgfsetfillopacity{0.616596}%
\pgfsetlinewidth{1.003750pt}%
\definecolor{currentstroke}{rgb}{0.121569,0.466667,0.705882}%
\pgfsetstrokecolor{currentstroke}%
\pgfsetstrokeopacity{0.616596}%
\pgfsetdash{}{0pt}%
\pgfpathmoveto{\pgfqpoint{0.831052in}{1.213574in}}%
\pgfpathcurveto{\pgfqpoint{0.839288in}{1.213574in}}{\pgfqpoint{0.847189in}{1.216846in}}{\pgfqpoint{0.853012in}{1.222670in}}%
\pgfpathcurveto{\pgfqpoint{0.858836in}{1.228494in}}{\pgfqpoint{0.862109in}{1.236394in}}{\pgfqpoint{0.862109in}{1.244631in}}%
\pgfpathcurveto{\pgfqpoint{0.862109in}{1.252867in}}{\pgfqpoint{0.858836in}{1.260767in}}{\pgfqpoint{0.853012in}{1.266591in}}%
\pgfpathcurveto{\pgfqpoint{0.847189in}{1.272415in}}{\pgfqpoint{0.839288in}{1.275687in}}{\pgfqpoint{0.831052in}{1.275687in}}%
\pgfpathcurveto{\pgfqpoint{0.822816in}{1.275687in}}{\pgfqpoint{0.814916in}{1.272415in}}{\pgfqpoint{0.809092in}{1.266591in}}%
\pgfpathcurveto{\pgfqpoint{0.803268in}{1.260767in}}{\pgfqpoint{0.799996in}{1.252867in}}{\pgfqpoint{0.799996in}{1.244631in}}%
\pgfpathcurveto{\pgfqpoint{0.799996in}{1.236394in}}{\pgfqpoint{0.803268in}{1.228494in}}{\pgfqpoint{0.809092in}{1.222670in}}%
\pgfpathcurveto{\pgfqpoint{0.814916in}{1.216846in}}{\pgfqpoint{0.822816in}{1.213574in}}{\pgfqpoint{0.831052in}{1.213574in}}%
\pgfpathclose%
\pgfusepath{stroke,fill}%
\end{pgfscope}%
\begin{pgfscope}%
\pgfpathrectangle{\pgfqpoint{0.100000in}{0.220728in}}{\pgfqpoint{3.696000in}{3.696000in}}%
\pgfusepath{clip}%
\pgfsetbuttcap%
\pgfsetroundjoin%
\definecolor{currentfill}{rgb}{0.121569,0.466667,0.705882}%
\pgfsetfillcolor{currentfill}%
\pgfsetfillopacity{0.616596}%
\pgfsetlinewidth{1.003750pt}%
\definecolor{currentstroke}{rgb}{0.121569,0.466667,0.705882}%
\pgfsetstrokecolor{currentstroke}%
\pgfsetstrokeopacity{0.616596}%
\pgfsetdash{}{0pt}%
\pgfpathmoveto{\pgfqpoint{0.831052in}{1.213574in}}%
\pgfpathcurveto{\pgfqpoint{0.839288in}{1.213574in}}{\pgfqpoint{0.847189in}{1.216846in}}{\pgfqpoint{0.853012in}{1.222670in}}%
\pgfpathcurveto{\pgfqpoint{0.858836in}{1.228494in}}{\pgfqpoint{0.862109in}{1.236394in}}{\pgfqpoint{0.862109in}{1.244631in}}%
\pgfpathcurveto{\pgfqpoint{0.862109in}{1.252867in}}{\pgfqpoint{0.858836in}{1.260767in}}{\pgfqpoint{0.853012in}{1.266591in}}%
\pgfpathcurveto{\pgfqpoint{0.847189in}{1.272415in}}{\pgfqpoint{0.839288in}{1.275687in}}{\pgfqpoint{0.831052in}{1.275687in}}%
\pgfpathcurveto{\pgfqpoint{0.822816in}{1.275687in}}{\pgfqpoint{0.814916in}{1.272415in}}{\pgfqpoint{0.809092in}{1.266591in}}%
\pgfpathcurveto{\pgfqpoint{0.803268in}{1.260767in}}{\pgfqpoint{0.799996in}{1.252867in}}{\pgfqpoint{0.799996in}{1.244631in}}%
\pgfpathcurveto{\pgfqpoint{0.799996in}{1.236394in}}{\pgfqpoint{0.803268in}{1.228494in}}{\pgfqpoint{0.809092in}{1.222670in}}%
\pgfpathcurveto{\pgfqpoint{0.814916in}{1.216846in}}{\pgfqpoint{0.822816in}{1.213574in}}{\pgfqpoint{0.831052in}{1.213574in}}%
\pgfpathclose%
\pgfusepath{stroke,fill}%
\end{pgfscope}%
\begin{pgfscope}%
\pgfpathrectangle{\pgfqpoint{0.100000in}{0.220728in}}{\pgfqpoint{3.696000in}{3.696000in}}%
\pgfusepath{clip}%
\pgfsetbuttcap%
\pgfsetroundjoin%
\definecolor{currentfill}{rgb}{0.121569,0.466667,0.705882}%
\pgfsetfillcolor{currentfill}%
\pgfsetfillopacity{0.616596}%
\pgfsetlinewidth{1.003750pt}%
\definecolor{currentstroke}{rgb}{0.121569,0.466667,0.705882}%
\pgfsetstrokecolor{currentstroke}%
\pgfsetstrokeopacity{0.616596}%
\pgfsetdash{}{0pt}%
\pgfpathmoveto{\pgfqpoint{0.831052in}{1.213574in}}%
\pgfpathcurveto{\pgfqpoint{0.839288in}{1.213574in}}{\pgfqpoint{0.847189in}{1.216846in}}{\pgfqpoint{0.853012in}{1.222670in}}%
\pgfpathcurveto{\pgfqpoint{0.858836in}{1.228494in}}{\pgfqpoint{0.862109in}{1.236394in}}{\pgfqpoint{0.862109in}{1.244631in}}%
\pgfpathcurveto{\pgfqpoint{0.862109in}{1.252867in}}{\pgfqpoint{0.858836in}{1.260767in}}{\pgfqpoint{0.853012in}{1.266591in}}%
\pgfpathcurveto{\pgfqpoint{0.847189in}{1.272415in}}{\pgfqpoint{0.839288in}{1.275687in}}{\pgfqpoint{0.831052in}{1.275687in}}%
\pgfpathcurveto{\pgfqpoint{0.822816in}{1.275687in}}{\pgfqpoint{0.814916in}{1.272415in}}{\pgfqpoint{0.809092in}{1.266591in}}%
\pgfpathcurveto{\pgfqpoint{0.803268in}{1.260767in}}{\pgfqpoint{0.799996in}{1.252867in}}{\pgfqpoint{0.799996in}{1.244631in}}%
\pgfpathcurveto{\pgfqpoint{0.799996in}{1.236394in}}{\pgfqpoint{0.803268in}{1.228494in}}{\pgfqpoint{0.809092in}{1.222670in}}%
\pgfpathcurveto{\pgfqpoint{0.814916in}{1.216846in}}{\pgfqpoint{0.822816in}{1.213574in}}{\pgfqpoint{0.831052in}{1.213574in}}%
\pgfpathclose%
\pgfusepath{stroke,fill}%
\end{pgfscope}%
\begin{pgfscope}%
\pgfpathrectangle{\pgfqpoint{0.100000in}{0.220728in}}{\pgfqpoint{3.696000in}{3.696000in}}%
\pgfusepath{clip}%
\pgfsetbuttcap%
\pgfsetroundjoin%
\definecolor{currentfill}{rgb}{0.121569,0.466667,0.705882}%
\pgfsetfillcolor{currentfill}%
\pgfsetfillopacity{0.616596}%
\pgfsetlinewidth{1.003750pt}%
\definecolor{currentstroke}{rgb}{0.121569,0.466667,0.705882}%
\pgfsetstrokecolor{currentstroke}%
\pgfsetstrokeopacity{0.616596}%
\pgfsetdash{}{0pt}%
\pgfpathmoveto{\pgfqpoint{0.831052in}{1.213574in}}%
\pgfpathcurveto{\pgfqpoint{0.839288in}{1.213574in}}{\pgfqpoint{0.847189in}{1.216846in}}{\pgfqpoint{0.853012in}{1.222670in}}%
\pgfpathcurveto{\pgfqpoint{0.858836in}{1.228494in}}{\pgfqpoint{0.862109in}{1.236394in}}{\pgfqpoint{0.862109in}{1.244631in}}%
\pgfpathcurveto{\pgfqpoint{0.862109in}{1.252867in}}{\pgfqpoint{0.858836in}{1.260767in}}{\pgfqpoint{0.853012in}{1.266591in}}%
\pgfpathcurveto{\pgfqpoint{0.847189in}{1.272415in}}{\pgfqpoint{0.839288in}{1.275687in}}{\pgfqpoint{0.831052in}{1.275687in}}%
\pgfpathcurveto{\pgfqpoint{0.822816in}{1.275687in}}{\pgfqpoint{0.814916in}{1.272415in}}{\pgfqpoint{0.809092in}{1.266591in}}%
\pgfpathcurveto{\pgfqpoint{0.803268in}{1.260767in}}{\pgfqpoint{0.799996in}{1.252867in}}{\pgfqpoint{0.799996in}{1.244631in}}%
\pgfpathcurveto{\pgfqpoint{0.799996in}{1.236394in}}{\pgfqpoint{0.803268in}{1.228494in}}{\pgfqpoint{0.809092in}{1.222670in}}%
\pgfpathcurveto{\pgfqpoint{0.814916in}{1.216846in}}{\pgfqpoint{0.822816in}{1.213574in}}{\pgfqpoint{0.831052in}{1.213574in}}%
\pgfpathclose%
\pgfusepath{stroke,fill}%
\end{pgfscope}%
\begin{pgfscope}%
\pgfpathrectangle{\pgfqpoint{0.100000in}{0.220728in}}{\pgfqpoint{3.696000in}{3.696000in}}%
\pgfusepath{clip}%
\pgfsetbuttcap%
\pgfsetroundjoin%
\definecolor{currentfill}{rgb}{0.121569,0.466667,0.705882}%
\pgfsetfillcolor{currentfill}%
\pgfsetfillopacity{0.616596}%
\pgfsetlinewidth{1.003750pt}%
\definecolor{currentstroke}{rgb}{0.121569,0.466667,0.705882}%
\pgfsetstrokecolor{currentstroke}%
\pgfsetstrokeopacity{0.616596}%
\pgfsetdash{}{0pt}%
\pgfpathmoveto{\pgfqpoint{0.831052in}{1.213574in}}%
\pgfpathcurveto{\pgfqpoint{0.839288in}{1.213574in}}{\pgfqpoint{0.847189in}{1.216846in}}{\pgfqpoint{0.853012in}{1.222670in}}%
\pgfpathcurveto{\pgfqpoint{0.858836in}{1.228494in}}{\pgfqpoint{0.862109in}{1.236394in}}{\pgfqpoint{0.862109in}{1.244631in}}%
\pgfpathcurveto{\pgfqpoint{0.862109in}{1.252867in}}{\pgfqpoint{0.858836in}{1.260767in}}{\pgfqpoint{0.853012in}{1.266591in}}%
\pgfpathcurveto{\pgfqpoint{0.847189in}{1.272415in}}{\pgfqpoint{0.839288in}{1.275687in}}{\pgfqpoint{0.831052in}{1.275687in}}%
\pgfpathcurveto{\pgfqpoint{0.822816in}{1.275687in}}{\pgfqpoint{0.814916in}{1.272415in}}{\pgfqpoint{0.809092in}{1.266591in}}%
\pgfpathcurveto{\pgfqpoint{0.803268in}{1.260767in}}{\pgfqpoint{0.799996in}{1.252867in}}{\pgfqpoint{0.799996in}{1.244631in}}%
\pgfpathcurveto{\pgfqpoint{0.799996in}{1.236394in}}{\pgfqpoint{0.803268in}{1.228494in}}{\pgfqpoint{0.809092in}{1.222670in}}%
\pgfpathcurveto{\pgfqpoint{0.814916in}{1.216846in}}{\pgfqpoint{0.822816in}{1.213574in}}{\pgfqpoint{0.831052in}{1.213574in}}%
\pgfpathclose%
\pgfusepath{stroke,fill}%
\end{pgfscope}%
\begin{pgfscope}%
\pgfpathrectangle{\pgfqpoint{0.100000in}{0.220728in}}{\pgfqpoint{3.696000in}{3.696000in}}%
\pgfusepath{clip}%
\pgfsetbuttcap%
\pgfsetroundjoin%
\definecolor{currentfill}{rgb}{0.121569,0.466667,0.705882}%
\pgfsetfillcolor{currentfill}%
\pgfsetfillopacity{0.616596}%
\pgfsetlinewidth{1.003750pt}%
\definecolor{currentstroke}{rgb}{0.121569,0.466667,0.705882}%
\pgfsetstrokecolor{currentstroke}%
\pgfsetstrokeopacity{0.616596}%
\pgfsetdash{}{0pt}%
\pgfpathmoveto{\pgfqpoint{0.831052in}{1.213574in}}%
\pgfpathcurveto{\pgfqpoint{0.839288in}{1.213574in}}{\pgfqpoint{0.847189in}{1.216846in}}{\pgfqpoint{0.853012in}{1.222670in}}%
\pgfpathcurveto{\pgfqpoint{0.858836in}{1.228494in}}{\pgfqpoint{0.862109in}{1.236394in}}{\pgfqpoint{0.862109in}{1.244631in}}%
\pgfpathcurveto{\pgfqpoint{0.862109in}{1.252867in}}{\pgfqpoint{0.858836in}{1.260767in}}{\pgfqpoint{0.853012in}{1.266591in}}%
\pgfpathcurveto{\pgfqpoint{0.847189in}{1.272415in}}{\pgfqpoint{0.839288in}{1.275687in}}{\pgfqpoint{0.831052in}{1.275687in}}%
\pgfpathcurveto{\pgfqpoint{0.822816in}{1.275687in}}{\pgfqpoint{0.814916in}{1.272415in}}{\pgfqpoint{0.809092in}{1.266591in}}%
\pgfpathcurveto{\pgfqpoint{0.803268in}{1.260767in}}{\pgfqpoint{0.799996in}{1.252867in}}{\pgfqpoint{0.799996in}{1.244631in}}%
\pgfpathcurveto{\pgfqpoint{0.799996in}{1.236394in}}{\pgfqpoint{0.803268in}{1.228494in}}{\pgfqpoint{0.809092in}{1.222670in}}%
\pgfpathcurveto{\pgfqpoint{0.814916in}{1.216846in}}{\pgfqpoint{0.822816in}{1.213574in}}{\pgfqpoint{0.831052in}{1.213574in}}%
\pgfpathclose%
\pgfusepath{stroke,fill}%
\end{pgfscope}%
\begin{pgfscope}%
\pgfpathrectangle{\pgfqpoint{0.100000in}{0.220728in}}{\pgfqpoint{3.696000in}{3.696000in}}%
\pgfusepath{clip}%
\pgfsetbuttcap%
\pgfsetroundjoin%
\definecolor{currentfill}{rgb}{0.121569,0.466667,0.705882}%
\pgfsetfillcolor{currentfill}%
\pgfsetfillopacity{0.616596}%
\pgfsetlinewidth{1.003750pt}%
\definecolor{currentstroke}{rgb}{0.121569,0.466667,0.705882}%
\pgfsetstrokecolor{currentstroke}%
\pgfsetstrokeopacity{0.616596}%
\pgfsetdash{}{0pt}%
\pgfpathmoveto{\pgfqpoint{0.831052in}{1.213574in}}%
\pgfpathcurveto{\pgfqpoint{0.839288in}{1.213574in}}{\pgfqpoint{0.847189in}{1.216846in}}{\pgfqpoint{0.853012in}{1.222670in}}%
\pgfpathcurveto{\pgfqpoint{0.858836in}{1.228494in}}{\pgfqpoint{0.862109in}{1.236394in}}{\pgfqpoint{0.862109in}{1.244631in}}%
\pgfpathcurveto{\pgfqpoint{0.862109in}{1.252867in}}{\pgfqpoint{0.858836in}{1.260767in}}{\pgfqpoint{0.853012in}{1.266591in}}%
\pgfpathcurveto{\pgfqpoint{0.847189in}{1.272415in}}{\pgfqpoint{0.839288in}{1.275687in}}{\pgfqpoint{0.831052in}{1.275687in}}%
\pgfpathcurveto{\pgfqpoint{0.822816in}{1.275687in}}{\pgfqpoint{0.814916in}{1.272415in}}{\pgfqpoint{0.809092in}{1.266591in}}%
\pgfpathcurveto{\pgfqpoint{0.803268in}{1.260767in}}{\pgfqpoint{0.799996in}{1.252867in}}{\pgfqpoint{0.799996in}{1.244631in}}%
\pgfpathcurveto{\pgfqpoint{0.799996in}{1.236394in}}{\pgfqpoint{0.803268in}{1.228494in}}{\pgfqpoint{0.809092in}{1.222670in}}%
\pgfpathcurveto{\pgfqpoint{0.814916in}{1.216846in}}{\pgfqpoint{0.822816in}{1.213574in}}{\pgfqpoint{0.831052in}{1.213574in}}%
\pgfpathclose%
\pgfusepath{stroke,fill}%
\end{pgfscope}%
\begin{pgfscope}%
\pgfpathrectangle{\pgfqpoint{0.100000in}{0.220728in}}{\pgfqpoint{3.696000in}{3.696000in}}%
\pgfusepath{clip}%
\pgfsetbuttcap%
\pgfsetroundjoin%
\definecolor{currentfill}{rgb}{0.121569,0.466667,0.705882}%
\pgfsetfillcolor{currentfill}%
\pgfsetfillopacity{0.616596}%
\pgfsetlinewidth{1.003750pt}%
\definecolor{currentstroke}{rgb}{0.121569,0.466667,0.705882}%
\pgfsetstrokecolor{currentstroke}%
\pgfsetstrokeopacity{0.616596}%
\pgfsetdash{}{0pt}%
\pgfpathmoveto{\pgfqpoint{0.831052in}{1.213574in}}%
\pgfpathcurveto{\pgfqpoint{0.839288in}{1.213574in}}{\pgfqpoint{0.847189in}{1.216846in}}{\pgfqpoint{0.853012in}{1.222670in}}%
\pgfpathcurveto{\pgfqpoint{0.858836in}{1.228494in}}{\pgfqpoint{0.862109in}{1.236394in}}{\pgfqpoint{0.862109in}{1.244631in}}%
\pgfpathcurveto{\pgfqpoint{0.862109in}{1.252867in}}{\pgfqpoint{0.858836in}{1.260767in}}{\pgfqpoint{0.853012in}{1.266591in}}%
\pgfpathcurveto{\pgfqpoint{0.847189in}{1.272415in}}{\pgfqpoint{0.839288in}{1.275687in}}{\pgfqpoint{0.831052in}{1.275687in}}%
\pgfpathcurveto{\pgfqpoint{0.822816in}{1.275687in}}{\pgfqpoint{0.814916in}{1.272415in}}{\pgfqpoint{0.809092in}{1.266591in}}%
\pgfpathcurveto{\pgfqpoint{0.803268in}{1.260767in}}{\pgfqpoint{0.799996in}{1.252867in}}{\pgfqpoint{0.799996in}{1.244631in}}%
\pgfpathcurveto{\pgfqpoint{0.799996in}{1.236394in}}{\pgfqpoint{0.803268in}{1.228494in}}{\pgfqpoint{0.809092in}{1.222670in}}%
\pgfpathcurveto{\pgfqpoint{0.814916in}{1.216846in}}{\pgfqpoint{0.822816in}{1.213574in}}{\pgfqpoint{0.831052in}{1.213574in}}%
\pgfpathclose%
\pgfusepath{stroke,fill}%
\end{pgfscope}%
\begin{pgfscope}%
\pgfpathrectangle{\pgfqpoint{0.100000in}{0.220728in}}{\pgfqpoint{3.696000in}{3.696000in}}%
\pgfusepath{clip}%
\pgfsetbuttcap%
\pgfsetroundjoin%
\definecolor{currentfill}{rgb}{0.121569,0.466667,0.705882}%
\pgfsetfillcolor{currentfill}%
\pgfsetfillopacity{0.616596}%
\pgfsetlinewidth{1.003750pt}%
\definecolor{currentstroke}{rgb}{0.121569,0.466667,0.705882}%
\pgfsetstrokecolor{currentstroke}%
\pgfsetstrokeopacity{0.616596}%
\pgfsetdash{}{0pt}%
\pgfpathmoveto{\pgfqpoint{0.831052in}{1.213574in}}%
\pgfpathcurveto{\pgfqpoint{0.839288in}{1.213574in}}{\pgfqpoint{0.847189in}{1.216846in}}{\pgfqpoint{0.853012in}{1.222670in}}%
\pgfpathcurveto{\pgfqpoint{0.858836in}{1.228494in}}{\pgfqpoint{0.862109in}{1.236394in}}{\pgfqpoint{0.862109in}{1.244631in}}%
\pgfpathcurveto{\pgfqpoint{0.862109in}{1.252867in}}{\pgfqpoint{0.858836in}{1.260767in}}{\pgfqpoint{0.853012in}{1.266591in}}%
\pgfpathcurveto{\pgfqpoint{0.847189in}{1.272415in}}{\pgfqpoint{0.839288in}{1.275687in}}{\pgfqpoint{0.831052in}{1.275687in}}%
\pgfpathcurveto{\pgfqpoint{0.822816in}{1.275687in}}{\pgfqpoint{0.814916in}{1.272415in}}{\pgfqpoint{0.809092in}{1.266591in}}%
\pgfpathcurveto{\pgfqpoint{0.803268in}{1.260767in}}{\pgfqpoint{0.799996in}{1.252867in}}{\pgfqpoint{0.799996in}{1.244631in}}%
\pgfpathcurveto{\pgfqpoint{0.799996in}{1.236394in}}{\pgfqpoint{0.803268in}{1.228494in}}{\pgfqpoint{0.809092in}{1.222670in}}%
\pgfpathcurveto{\pgfqpoint{0.814916in}{1.216846in}}{\pgfqpoint{0.822816in}{1.213574in}}{\pgfqpoint{0.831052in}{1.213574in}}%
\pgfpathclose%
\pgfusepath{stroke,fill}%
\end{pgfscope}%
\begin{pgfscope}%
\pgfpathrectangle{\pgfqpoint{0.100000in}{0.220728in}}{\pgfqpoint{3.696000in}{3.696000in}}%
\pgfusepath{clip}%
\pgfsetbuttcap%
\pgfsetroundjoin%
\definecolor{currentfill}{rgb}{0.121569,0.466667,0.705882}%
\pgfsetfillcolor{currentfill}%
\pgfsetfillopacity{0.616596}%
\pgfsetlinewidth{1.003750pt}%
\definecolor{currentstroke}{rgb}{0.121569,0.466667,0.705882}%
\pgfsetstrokecolor{currentstroke}%
\pgfsetstrokeopacity{0.616596}%
\pgfsetdash{}{0pt}%
\pgfpathmoveto{\pgfqpoint{0.831052in}{1.213574in}}%
\pgfpathcurveto{\pgfqpoint{0.839288in}{1.213574in}}{\pgfqpoint{0.847189in}{1.216846in}}{\pgfqpoint{0.853012in}{1.222670in}}%
\pgfpathcurveto{\pgfqpoint{0.858836in}{1.228494in}}{\pgfqpoint{0.862109in}{1.236394in}}{\pgfqpoint{0.862109in}{1.244631in}}%
\pgfpathcurveto{\pgfqpoint{0.862109in}{1.252867in}}{\pgfqpoint{0.858836in}{1.260767in}}{\pgfqpoint{0.853012in}{1.266591in}}%
\pgfpathcurveto{\pgfqpoint{0.847189in}{1.272415in}}{\pgfqpoint{0.839288in}{1.275687in}}{\pgfqpoint{0.831052in}{1.275687in}}%
\pgfpathcurveto{\pgfqpoint{0.822816in}{1.275687in}}{\pgfqpoint{0.814916in}{1.272415in}}{\pgfqpoint{0.809092in}{1.266591in}}%
\pgfpathcurveto{\pgfqpoint{0.803268in}{1.260767in}}{\pgfqpoint{0.799996in}{1.252867in}}{\pgfqpoint{0.799996in}{1.244631in}}%
\pgfpathcurveto{\pgfqpoint{0.799996in}{1.236394in}}{\pgfqpoint{0.803268in}{1.228494in}}{\pgfqpoint{0.809092in}{1.222670in}}%
\pgfpathcurveto{\pgfqpoint{0.814916in}{1.216846in}}{\pgfqpoint{0.822816in}{1.213574in}}{\pgfqpoint{0.831052in}{1.213574in}}%
\pgfpathclose%
\pgfusepath{stroke,fill}%
\end{pgfscope}%
\begin{pgfscope}%
\pgfpathrectangle{\pgfqpoint{0.100000in}{0.220728in}}{\pgfqpoint{3.696000in}{3.696000in}}%
\pgfusepath{clip}%
\pgfsetbuttcap%
\pgfsetroundjoin%
\definecolor{currentfill}{rgb}{0.121569,0.466667,0.705882}%
\pgfsetfillcolor{currentfill}%
\pgfsetfillopacity{0.616596}%
\pgfsetlinewidth{1.003750pt}%
\definecolor{currentstroke}{rgb}{0.121569,0.466667,0.705882}%
\pgfsetstrokecolor{currentstroke}%
\pgfsetstrokeopacity{0.616596}%
\pgfsetdash{}{0pt}%
\pgfpathmoveto{\pgfqpoint{0.831052in}{1.213574in}}%
\pgfpathcurveto{\pgfqpoint{0.839288in}{1.213574in}}{\pgfqpoint{0.847189in}{1.216846in}}{\pgfqpoint{0.853012in}{1.222670in}}%
\pgfpathcurveto{\pgfqpoint{0.858836in}{1.228494in}}{\pgfqpoint{0.862109in}{1.236394in}}{\pgfqpoint{0.862109in}{1.244631in}}%
\pgfpathcurveto{\pgfqpoint{0.862109in}{1.252867in}}{\pgfqpoint{0.858836in}{1.260767in}}{\pgfqpoint{0.853012in}{1.266591in}}%
\pgfpathcurveto{\pgfqpoint{0.847189in}{1.272415in}}{\pgfqpoint{0.839288in}{1.275687in}}{\pgfqpoint{0.831052in}{1.275687in}}%
\pgfpathcurveto{\pgfqpoint{0.822816in}{1.275687in}}{\pgfqpoint{0.814916in}{1.272415in}}{\pgfqpoint{0.809092in}{1.266591in}}%
\pgfpathcurveto{\pgfqpoint{0.803268in}{1.260767in}}{\pgfqpoint{0.799996in}{1.252867in}}{\pgfqpoint{0.799996in}{1.244631in}}%
\pgfpathcurveto{\pgfqpoint{0.799996in}{1.236394in}}{\pgfqpoint{0.803268in}{1.228494in}}{\pgfqpoint{0.809092in}{1.222670in}}%
\pgfpathcurveto{\pgfqpoint{0.814916in}{1.216846in}}{\pgfqpoint{0.822816in}{1.213574in}}{\pgfqpoint{0.831052in}{1.213574in}}%
\pgfpathclose%
\pgfusepath{stroke,fill}%
\end{pgfscope}%
\begin{pgfscope}%
\pgfpathrectangle{\pgfqpoint{0.100000in}{0.220728in}}{\pgfqpoint{3.696000in}{3.696000in}}%
\pgfusepath{clip}%
\pgfsetbuttcap%
\pgfsetroundjoin%
\definecolor{currentfill}{rgb}{0.121569,0.466667,0.705882}%
\pgfsetfillcolor{currentfill}%
\pgfsetfillopacity{0.616596}%
\pgfsetlinewidth{1.003750pt}%
\definecolor{currentstroke}{rgb}{0.121569,0.466667,0.705882}%
\pgfsetstrokecolor{currentstroke}%
\pgfsetstrokeopacity{0.616596}%
\pgfsetdash{}{0pt}%
\pgfpathmoveto{\pgfqpoint{0.831052in}{1.213574in}}%
\pgfpathcurveto{\pgfqpoint{0.839288in}{1.213574in}}{\pgfqpoint{0.847189in}{1.216846in}}{\pgfqpoint{0.853012in}{1.222670in}}%
\pgfpathcurveto{\pgfqpoint{0.858836in}{1.228494in}}{\pgfqpoint{0.862109in}{1.236394in}}{\pgfqpoint{0.862109in}{1.244631in}}%
\pgfpathcurveto{\pgfqpoint{0.862109in}{1.252867in}}{\pgfqpoint{0.858836in}{1.260767in}}{\pgfqpoint{0.853012in}{1.266591in}}%
\pgfpathcurveto{\pgfqpoint{0.847189in}{1.272415in}}{\pgfqpoint{0.839288in}{1.275687in}}{\pgfqpoint{0.831052in}{1.275687in}}%
\pgfpathcurveto{\pgfqpoint{0.822816in}{1.275687in}}{\pgfqpoint{0.814916in}{1.272415in}}{\pgfqpoint{0.809092in}{1.266591in}}%
\pgfpathcurveto{\pgfqpoint{0.803268in}{1.260767in}}{\pgfqpoint{0.799996in}{1.252867in}}{\pgfqpoint{0.799996in}{1.244631in}}%
\pgfpathcurveto{\pgfqpoint{0.799996in}{1.236394in}}{\pgfqpoint{0.803268in}{1.228494in}}{\pgfqpoint{0.809092in}{1.222670in}}%
\pgfpathcurveto{\pgfqpoint{0.814916in}{1.216846in}}{\pgfqpoint{0.822816in}{1.213574in}}{\pgfqpoint{0.831052in}{1.213574in}}%
\pgfpathclose%
\pgfusepath{stroke,fill}%
\end{pgfscope}%
\begin{pgfscope}%
\pgfpathrectangle{\pgfqpoint{0.100000in}{0.220728in}}{\pgfqpoint{3.696000in}{3.696000in}}%
\pgfusepath{clip}%
\pgfsetbuttcap%
\pgfsetroundjoin%
\definecolor{currentfill}{rgb}{0.121569,0.466667,0.705882}%
\pgfsetfillcolor{currentfill}%
\pgfsetfillopacity{0.616596}%
\pgfsetlinewidth{1.003750pt}%
\definecolor{currentstroke}{rgb}{0.121569,0.466667,0.705882}%
\pgfsetstrokecolor{currentstroke}%
\pgfsetstrokeopacity{0.616596}%
\pgfsetdash{}{0pt}%
\pgfpathmoveto{\pgfqpoint{0.831052in}{1.213574in}}%
\pgfpathcurveto{\pgfqpoint{0.839288in}{1.213574in}}{\pgfqpoint{0.847189in}{1.216846in}}{\pgfqpoint{0.853012in}{1.222670in}}%
\pgfpathcurveto{\pgfqpoint{0.858836in}{1.228494in}}{\pgfqpoint{0.862109in}{1.236394in}}{\pgfqpoint{0.862109in}{1.244631in}}%
\pgfpathcurveto{\pgfqpoint{0.862109in}{1.252867in}}{\pgfqpoint{0.858836in}{1.260767in}}{\pgfqpoint{0.853012in}{1.266591in}}%
\pgfpathcurveto{\pgfqpoint{0.847189in}{1.272415in}}{\pgfqpoint{0.839288in}{1.275687in}}{\pgfqpoint{0.831052in}{1.275687in}}%
\pgfpathcurveto{\pgfqpoint{0.822816in}{1.275687in}}{\pgfqpoint{0.814916in}{1.272415in}}{\pgfqpoint{0.809092in}{1.266591in}}%
\pgfpathcurveto{\pgfqpoint{0.803268in}{1.260767in}}{\pgfqpoint{0.799996in}{1.252867in}}{\pgfqpoint{0.799996in}{1.244631in}}%
\pgfpathcurveto{\pgfqpoint{0.799996in}{1.236394in}}{\pgfqpoint{0.803268in}{1.228494in}}{\pgfqpoint{0.809092in}{1.222670in}}%
\pgfpathcurveto{\pgfqpoint{0.814916in}{1.216846in}}{\pgfqpoint{0.822816in}{1.213574in}}{\pgfqpoint{0.831052in}{1.213574in}}%
\pgfpathclose%
\pgfusepath{stroke,fill}%
\end{pgfscope}%
\begin{pgfscope}%
\pgfpathrectangle{\pgfqpoint{0.100000in}{0.220728in}}{\pgfqpoint{3.696000in}{3.696000in}}%
\pgfusepath{clip}%
\pgfsetbuttcap%
\pgfsetroundjoin%
\definecolor{currentfill}{rgb}{0.121569,0.466667,0.705882}%
\pgfsetfillcolor{currentfill}%
\pgfsetfillopacity{0.616596}%
\pgfsetlinewidth{1.003750pt}%
\definecolor{currentstroke}{rgb}{0.121569,0.466667,0.705882}%
\pgfsetstrokecolor{currentstroke}%
\pgfsetstrokeopacity{0.616596}%
\pgfsetdash{}{0pt}%
\pgfpathmoveto{\pgfqpoint{0.831052in}{1.213574in}}%
\pgfpathcurveto{\pgfqpoint{0.839288in}{1.213574in}}{\pgfqpoint{0.847189in}{1.216846in}}{\pgfqpoint{0.853012in}{1.222670in}}%
\pgfpathcurveto{\pgfqpoint{0.858836in}{1.228494in}}{\pgfqpoint{0.862109in}{1.236394in}}{\pgfqpoint{0.862109in}{1.244631in}}%
\pgfpathcurveto{\pgfqpoint{0.862109in}{1.252867in}}{\pgfqpoint{0.858836in}{1.260767in}}{\pgfqpoint{0.853012in}{1.266591in}}%
\pgfpathcurveto{\pgfqpoint{0.847189in}{1.272415in}}{\pgfqpoint{0.839288in}{1.275687in}}{\pgfqpoint{0.831052in}{1.275687in}}%
\pgfpathcurveto{\pgfqpoint{0.822816in}{1.275687in}}{\pgfqpoint{0.814916in}{1.272415in}}{\pgfqpoint{0.809092in}{1.266591in}}%
\pgfpathcurveto{\pgfqpoint{0.803268in}{1.260767in}}{\pgfqpoint{0.799996in}{1.252867in}}{\pgfqpoint{0.799996in}{1.244631in}}%
\pgfpathcurveto{\pgfqpoint{0.799996in}{1.236394in}}{\pgfqpoint{0.803268in}{1.228494in}}{\pgfqpoint{0.809092in}{1.222670in}}%
\pgfpathcurveto{\pgfqpoint{0.814916in}{1.216846in}}{\pgfqpoint{0.822816in}{1.213574in}}{\pgfqpoint{0.831052in}{1.213574in}}%
\pgfpathclose%
\pgfusepath{stroke,fill}%
\end{pgfscope}%
\begin{pgfscope}%
\pgfpathrectangle{\pgfqpoint{0.100000in}{0.220728in}}{\pgfqpoint{3.696000in}{3.696000in}}%
\pgfusepath{clip}%
\pgfsetbuttcap%
\pgfsetroundjoin%
\definecolor{currentfill}{rgb}{0.121569,0.466667,0.705882}%
\pgfsetfillcolor{currentfill}%
\pgfsetfillopacity{0.616596}%
\pgfsetlinewidth{1.003750pt}%
\definecolor{currentstroke}{rgb}{0.121569,0.466667,0.705882}%
\pgfsetstrokecolor{currentstroke}%
\pgfsetstrokeopacity{0.616596}%
\pgfsetdash{}{0pt}%
\pgfpathmoveto{\pgfqpoint{0.831052in}{1.213574in}}%
\pgfpathcurveto{\pgfqpoint{0.839288in}{1.213574in}}{\pgfqpoint{0.847189in}{1.216846in}}{\pgfqpoint{0.853012in}{1.222670in}}%
\pgfpathcurveto{\pgfqpoint{0.858836in}{1.228494in}}{\pgfqpoint{0.862109in}{1.236394in}}{\pgfqpoint{0.862109in}{1.244631in}}%
\pgfpathcurveto{\pgfqpoint{0.862109in}{1.252867in}}{\pgfqpoint{0.858836in}{1.260767in}}{\pgfqpoint{0.853012in}{1.266591in}}%
\pgfpathcurveto{\pgfqpoint{0.847189in}{1.272415in}}{\pgfqpoint{0.839288in}{1.275687in}}{\pgfqpoint{0.831052in}{1.275687in}}%
\pgfpathcurveto{\pgfqpoint{0.822816in}{1.275687in}}{\pgfqpoint{0.814916in}{1.272415in}}{\pgfqpoint{0.809092in}{1.266591in}}%
\pgfpathcurveto{\pgfqpoint{0.803268in}{1.260767in}}{\pgfqpoint{0.799996in}{1.252867in}}{\pgfqpoint{0.799996in}{1.244631in}}%
\pgfpathcurveto{\pgfqpoint{0.799996in}{1.236394in}}{\pgfqpoint{0.803268in}{1.228494in}}{\pgfqpoint{0.809092in}{1.222670in}}%
\pgfpathcurveto{\pgfqpoint{0.814916in}{1.216846in}}{\pgfqpoint{0.822816in}{1.213574in}}{\pgfqpoint{0.831052in}{1.213574in}}%
\pgfpathclose%
\pgfusepath{stroke,fill}%
\end{pgfscope}%
\begin{pgfscope}%
\pgfpathrectangle{\pgfqpoint{0.100000in}{0.220728in}}{\pgfqpoint{3.696000in}{3.696000in}}%
\pgfusepath{clip}%
\pgfsetbuttcap%
\pgfsetroundjoin%
\definecolor{currentfill}{rgb}{0.121569,0.466667,0.705882}%
\pgfsetfillcolor{currentfill}%
\pgfsetfillopacity{0.616596}%
\pgfsetlinewidth{1.003750pt}%
\definecolor{currentstroke}{rgb}{0.121569,0.466667,0.705882}%
\pgfsetstrokecolor{currentstroke}%
\pgfsetstrokeopacity{0.616596}%
\pgfsetdash{}{0pt}%
\pgfpathmoveto{\pgfqpoint{0.831052in}{1.213574in}}%
\pgfpathcurveto{\pgfqpoint{0.839288in}{1.213574in}}{\pgfqpoint{0.847189in}{1.216846in}}{\pgfqpoint{0.853012in}{1.222670in}}%
\pgfpathcurveto{\pgfqpoint{0.858836in}{1.228494in}}{\pgfqpoint{0.862109in}{1.236394in}}{\pgfqpoint{0.862109in}{1.244631in}}%
\pgfpathcurveto{\pgfqpoint{0.862109in}{1.252867in}}{\pgfqpoint{0.858836in}{1.260767in}}{\pgfqpoint{0.853012in}{1.266591in}}%
\pgfpathcurveto{\pgfqpoint{0.847189in}{1.272415in}}{\pgfqpoint{0.839288in}{1.275687in}}{\pgfqpoint{0.831052in}{1.275687in}}%
\pgfpathcurveto{\pgfqpoint{0.822816in}{1.275687in}}{\pgfqpoint{0.814916in}{1.272415in}}{\pgfqpoint{0.809092in}{1.266591in}}%
\pgfpathcurveto{\pgfqpoint{0.803268in}{1.260767in}}{\pgfqpoint{0.799996in}{1.252867in}}{\pgfqpoint{0.799996in}{1.244631in}}%
\pgfpathcurveto{\pgfqpoint{0.799996in}{1.236394in}}{\pgfqpoint{0.803268in}{1.228494in}}{\pgfqpoint{0.809092in}{1.222670in}}%
\pgfpathcurveto{\pgfqpoint{0.814916in}{1.216846in}}{\pgfqpoint{0.822816in}{1.213574in}}{\pgfqpoint{0.831052in}{1.213574in}}%
\pgfpathclose%
\pgfusepath{stroke,fill}%
\end{pgfscope}%
\begin{pgfscope}%
\pgfpathrectangle{\pgfqpoint{0.100000in}{0.220728in}}{\pgfqpoint{3.696000in}{3.696000in}}%
\pgfusepath{clip}%
\pgfsetbuttcap%
\pgfsetroundjoin%
\definecolor{currentfill}{rgb}{0.121569,0.466667,0.705882}%
\pgfsetfillcolor{currentfill}%
\pgfsetfillopacity{0.616596}%
\pgfsetlinewidth{1.003750pt}%
\definecolor{currentstroke}{rgb}{0.121569,0.466667,0.705882}%
\pgfsetstrokecolor{currentstroke}%
\pgfsetstrokeopacity{0.616596}%
\pgfsetdash{}{0pt}%
\pgfpathmoveto{\pgfqpoint{0.831052in}{1.213574in}}%
\pgfpathcurveto{\pgfqpoint{0.839288in}{1.213574in}}{\pgfqpoint{0.847189in}{1.216846in}}{\pgfqpoint{0.853012in}{1.222670in}}%
\pgfpathcurveto{\pgfqpoint{0.858836in}{1.228494in}}{\pgfqpoint{0.862109in}{1.236394in}}{\pgfqpoint{0.862109in}{1.244631in}}%
\pgfpathcurveto{\pgfqpoint{0.862109in}{1.252867in}}{\pgfqpoint{0.858836in}{1.260767in}}{\pgfqpoint{0.853012in}{1.266591in}}%
\pgfpathcurveto{\pgfqpoint{0.847189in}{1.272415in}}{\pgfqpoint{0.839288in}{1.275687in}}{\pgfqpoint{0.831052in}{1.275687in}}%
\pgfpathcurveto{\pgfqpoint{0.822816in}{1.275687in}}{\pgfqpoint{0.814916in}{1.272415in}}{\pgfqpoint{0.809092in}{1.266591in}}%
\pgfpathcurveto{\pgfqpoint{0.803268in}{1.260767in}}{\pgfqpoint{0.799996in}{1.252867in}}{\pgfqpoint{0.799996in}{1.244631in}}%
\pgfpathcurveto{\pgfqpoint{0.799996in}{1.236394in}}{\pgfqpoint{0.803268in}{1.228494in}}{\pgfqpoint{0.809092in}{1.222670in}}%
\pgfpathcurveto{\pgfqpoint{0.814916in}{1.216846in}}{\pgfqpoint{0.822816in}{1.213574in}}{\pgfqpoint{0.831052in}{1.213574in}}%
\pgfpathclose%
\pgfusepath{stroke,fill}%
\end{pgfscope}%
\begin{pgfscope}%
\pgfpathrectangle{\pgfqpoint{0.100000in}{0.220728in}}{\pgfqpoint{3.696000in}{3.696000in}}%
\pgfusepath{clip}%
\pgfsetbuttcap%
\pgfsetroundjoin%
\definecolor{currentfill}{rgb}{0.121569,0.466667,0.705882}%
\pgfsetfillcolor{currentfill}%
\pgfsetfillopacity{0.616596}%
\pgfsetlinewidth{1.003750pt}%
\definecolor{currentstroke}{rgb}{0.121569,0.466667,0.705882}%
\pgfsetstrokecolor{currentstroke}%
\pgfsetstrokeopacity{0.616596}%
\pgfsetdash{}{0pt}%
\pgfpathmoveto{\pgfqpoint{0.831052in}{1.213574in}}%
\pgfpathcurveto{\pgfqpoint{0.839288in}{1.213574in}}{\pgfqpoint{0.847189in}{1.216846in}}{\pgfqpoint{0.853012in}{1.222670in}}%
\pgfpathcurveto{\pgfqpoint{0.858836in}{1.228494in}}{\pgfqpoint{0.862109in}{1.236394in}}{\pgfqpoint{0.862109in}{1.244631in}}%
\pgfpathcurveto{\pgfqpoint{0.862109in}{1.252867in}}{\pgfqpoint{0.858836in}{1.260767in}}{\pgfqpoint{0.853012in}{1.266591in}}%
\pgfpathcurveto{\pgfqpoint{0.847189in}{1.272415in}}{\pgfqpoint{0.839288in}{1.275687in}}{\pgfqpoint{0.831052in}{1.275687in}}%
\pgfpathcurveto{\pgfqpoint{0.822816in}{1.275687in}}{\pgfqpoint{0.814916in}{1.272415in}}{\pgfqpoint{0.809092in}{1.266591in}}%
\pgfpathcurveto{\pgfqpoint{0.803268in}{1.260767in}}{\pgfqpoint{0.799996in}{1.252867in}}{\pgfqpoint{0.799996in}{1.244631in}}%
\pgfpathcurveto{\pgfqpoint{0.799996in}{1.236394in}}{\pgfqpoint{0.803268in}{1.228494in}}{\pgfqpoint{0.809092in}{1.222670in}}%
\pgfpathcurveto{\pgfqpoint{0.814916in}{1.216846in}}{\pgfqpoint{0.822816in}{1.213574in}}{\pgfqpoint{0.831052in}{1.213574in}}%
\pgfpathclose%
\pgfusepath{stroke,fill}%
\end{pgfscope}%
\begin{pgfscope}%
\pgfpathrectangle{\pgfqpoint{0.100000in}{0.220728in}}{\pgfqpoint{3.696000in}{3.696000in}}%
\pgfusepath{clip}%
\pgfsetbuttcap%
\pgfsetroundjoin%
\definecolor{currentfill}{rgb}{0.121569,0.466667,0.705882}%
\pgfsetfillcolor{currentfill}%
\pgfsetfillopacity{0.616596}%
\pgfsetlinewidth{1.003750pt}%
\definecolor{currentstroke}{rgb}{0.121569,0.466667,0.705882}%
\pgfsetstrokecolor{currentstroke}%
\pgfsetstrokeopacity{0.616596}%
\pgfsetdash{}{0pt}%
\pgfpathmoveto{\pgfqpoint{0.831052in}{1.213574in}}%
\pgfpathcurveto{\pgfqpoint{0.839288in}{1.213574in}}{\pgfqpoint{0.847189in}{1.216846in}}{\pgfqpoint{0.853012in}{1.222670in}}%
\pgfpathcurveto{\pgfqpoint{0.858836in}{1.228494in}}{\pgfqpoint{0.862109in}{1.236394in}}{\pgfqpoint{0.862109in}{1.244631in}}%
\pgfpathcurveto{\pgfqpoint{0.862109in}{1.252867in}}{\pgfqpoint{0.858836in}{1.260767in}}{\pgfqpoint{0.853012in}{1.266591in}}%
\pgfpathcurveto{\pgfqpoint{0.847189in}{1.272415in}}{\pgfqpoint{0.839288in}{1.275687in}}{\pgfqpoint{0.831052in}{1.275687in}}%
\pgfpathcurveto{\pgfqpoint{0.822816in}{1.275687in}}{\pgfqpoint{0.814916in}{1.272415in}}{\pgfqpoint{0.809092in}{1.266591in}}%
\pgfpathcurveto{\pgfqpoint{0.803268in}{1.260767in}}{\pgfqpoint{0.799996in}{1.252867in}}{\pgfqpoint{0.799996in}{1.244631in}}%
\pgfpathcurveto{\pgfqpoint{0.799996in}{1.236394in}}{\pgfqpoint{0.803268in}{1.228494in}}{\pgfqpoint{0.809092in}{1.222670in}}%
\pgfpathcurveto{\pgfqpoint{0.814916in}{1.216846in}}{\pgfqpoint{0.822816in}{1.213574in}}{\pgfqpoint{0.831052in}{1.213574in}}%
\pgfpathclose%
\pgfusepath{stroke,fill}%
\end{pgfscope}%
\begin{pgfscope}%
\pgfpathrectangle{\pgfqpoint{0.100000in}{0.220728in}}{\pgfqpoint{3.696000in}{3.696000in}}%
\pgfusepath{clip}%
\pgfsetbuttcap%
\pgfsetroundjoin%
\definecolor{currentfill}{rgb}{0.121569,0.466667,0.705882}%
\pgfsetfillcolor{currentfill}%
\pgfsetfillopacity{0.616596}%
\pgfsetlinewidth{1.003750pt}%
\definecolor{currentstroke}{rgb}{0.121569,0.466667,0.705882}%
\pgfsetstrokecolor{currentstroke}%
\pgfsetstrokeopacity{0.616596}%
\pgfsetdash{}{0pt}%
\pgfpathmoveto{\pgfqpoint{0.831052in}{1.213574in}}%
\pgfpathcurveto{\pgfqpoint{0.839288in}{1.213574in}}{\pgfqpoint{0.847189in}{1.216846in}}{\pgfqpoint{0.853012in}{1.222670in}}%
\pgfpathcurveto{\pgfqpoint{0.858836in}{1.228494in}}{\pgfqpoint{0.862109in}{1.236394in}}{\pgfqpoint{0.862109in}{1.244631in}}%
\pgfpathcurveto{\pgfqpoint{0.862109in}{1.252867in}}{\pgfqpoint{0.858836in}{1.260767in}}{\pgfqpoint{0.853012in}{1.266591in}}%
\pgfpathcurveto{\pgfqpoint{0.847189in}{1.272415in}}{\pgfqpoint{0.839288in}{1.275687in}}{\pgfqpoint{0.831052in}{1.275687in}}%
\pgfpathcurveto{\pgfqpoint{0.822816in}{1.275687in}}{\pgfqpoint{0.814916in}{1.272415in}}{\pgfqpoint{0.809092in}{1.266591in}}%
\pgfpathcurveto{\pgfqpoint{0.803268in}{1.260767in}}{\pgfqpoint{0.799996in}{1.252867in}}{\pgfqpoint{0.799996in}{1.244631in}}%
\pgfpathcurveto{\pgfqpoint{0.799996in}{1.236394in}}{\pgfqpoint{0.803268in}{1.228494in}}{\pgfqpoint{0.809092in}{1.222670in}}%
\pgfpathcurveto{\pgfqpoint{0.814916in}{1.216846in}}{\pgfqpoint{0.822816in}{1.213574in}}{\pgfqpoint{0.831052in}{1.213574in}}%
\pgfpathclose%
\pgfusepath{stroke,fill}%
\end{pgfscope}%
\begin{pgfscope}%
\pgfpathrectangle{\pgfqpoint{0.100000in}{0.220728in}}{\pgfqpoint{3.696000in}{3.696000in}}%
\pgfusepath{clip}%
\pgfsetbuttcap%
\pgfsetroundjoin%
\definecolor{currentfill}{rgb}{0.121569,0.466667,0.705882}%
\pgfsetfillcolor{currentfill}%
\pgfsetfillopacity{0.616596}%
\pgfsetlinewidth{1.003750pt}%
\definecolor{currentstroke}{rgb}{0.121569,0.466667,0.705882}%
\pgfsetstrokecolor{currentstroke}%
\pgfsetstrokeopacity{0.616596}%
\pgfsetdash{}{0pt}%
\pgfpathmoveto{\pgfqpoint{0.831052in}{1.213574in}}%
\pgfpathcurveto{\pgfqpoint{0.839288in}{1.213574in}}{\pgfqpoint{0.847189in}{1.216846in}}{\pgfqpoint{0.853012in}{1.222670in}}%
\pgfpathcurveto{\pgfqpoint{0.858836in}{1.228494in}}{\pgfqpoint{0.862109in}{1.236394in}}{\pgfqpoint{0.862109in}{1.244631in}}%
\pgfpathcurveto{\pgfqpoint{0.862109in}{1.252867in}}{\pgfqpoint{0.858836in}{1.260767in}}{\pgfqpoint{0.853012in}{1.266591in}}%
\pgfpathcurveto{\pgfqpoint{0.847189in}{1.272415in}}{\pgfqpoint{0.839288in}{1.275687in}}{\pgfqpoint{0.831052in}{1.275687in}}%
\pgfpathcurveto{\pgfqpoint{0.822816in}{1.275687in}}{\pgfqpoint{0.814916in}{1.272415in}}{\pgfqpoint{0.809092in}{1.266591in}}%
\pgfpathcurveto{\pgfqpoint{0.803268in}{1.260767in}}{\pgfqpoint{0.799996in}{1.252867in}}{\pgfqpoint{0.799996in}{1.244631in}}%
\pgfpathcurveto{\pgfqpoint{0.799996in}{1.236394in}}{\pgfqpoint{0.803268in}{1.228494in}}{\pgfqpoint{0.809092in}{1.222670in}}%
\pgfpathcurveto{\pgfqpoint{0.814916in}{1.216846in}}{\pgfqpoint{0.822816in}{1.213574in}}{\pgfqpoint{0.831052in}{1.213574in}}%
\pgfpathclose%
\pgfusepath{stroke,fill}%
\end{pgfscope}%
\begin{pgfscope}%
\pgfpathrectangle{\pgfqpoint{0.100000in}{0.220728in}}{\pgfqpoint{3.696000in}{3.696000in}}%
\pgfusepath{clip}%
\pgfsetbuttcap%
\pgfsetroundjoin%
\definecolor{currentfill}{rgb}{0.121569,0.466667,0.705882}%
\pgfsetfillcolor{currentfill}%
\pgfsetfillopacity{0.616596}%
\pgfsetlinewidth{1.003750pt}%
\definecolor{currentstroke}{rgb}{0.121569,0.466667,0.705882}%
\pgfsetstrokecolor{currentstroke}%
\pgfsetstrokeopacity{0.616596}%
\pgfsetdash{}{0pt}%
\pgfpathmoveto{\pgfqpoint{0.831052in}{1.213574in}}%
\pgfpathcurveto{\pgfqpoint{0.839288in}{1.213574in}}{\pgfqpoint{0.847189in}{1.216846in}}{\pgfqpoint{0.853012in}{1.222670in}}%
\pgfpathcurveto{\pgfqpoint{0.858836in}{1.228494in}}{\pgfqpoint{0.862109in}{1.236394in}}{\pgfqpoint{0.862109in}{1.244631in}}%
\pgfpathcurveto{\pgfqpoint{0.862109in}{1.252867in}}{\pgfqpoint{0.858836in}{1.260767in}}{\pgfqpoint{0.853012in}{1.266591in}}%
\pgfpathcurveto{\pgfqpoint{0.847189in}{1.272415in}}{\pgfqpoint{0.839288in}{1.275687in}}{\pgfqpoint{0.831052in}{1.275687in}}%
\pgfpathcurveto{\pgfqpoint{0.822816in}{1.275687in}}{\pgfqpoint{0.814916in}{1.272415in}}{\pgfqpoint{0.809092in}{1.266591in}}%
\pgfpathcurveto{\pgfqpoint{0.803268in}{1.260767in}}{\pgfqpoint{0.799996in}{1.252867in}}{\pgfqpoint{0.799996in}{1.244631in}}%
\pgfpathcurveto{\pgfqpoint{0.799996in}{1.236394in}}{\pgfqpoint{0.803268in}{1.228494in}}{\pgfqpoint{0.809092in}{1.222670in}}%
\pgfpathcurveto{\pgfqpoint{0.814916in}{1.216846in}}{\pgfqpoint{0.822816in}{1.213574in}}{\pgfqpoint{0.831052in}{1.213574in}}%
\pgfpathclose%
\pgfusepath{stroke,fill}%
\end{pgfscope}%
\begin{pgfscope}%
\pgfpathrectangle{\pgfqpoint{0.100000in}{0.220728in}}{\pgfqpoint{3.696000in}{3.696000in}}%
\pgfusepath{clip}%
\pgfsetbuttcap%
\pgfsetroundjoin%
\definecolor{currentfill}{rgb}{0.121569,0.466667,0.705882}%
\pgfsetfillcolor{currentfill}%
\pgfsetfillopacity{0.616596}%
\pgfsetlinewidth{1.003750pt}%
\definecolor{currentstroke}{rgb}{0.121569,0.466667,0.705882}%
\pgfsetstrokecolor{currentstroke}%
\pgfsetstrokeopacity{0.616596}%
\pgfsetdash{}{0pt}%
\pgfpathmoveto{\pgfqpoint{0.831052in}{1.213574in}}%
\pgfpathcurveto{\pgfqpoint{0.839288in}{1.213574in}}{\pgfqpoint{0.847189in}{1.216846in}}{\pgfqpoint{0.853012in}{1.222670in}}%
\pgfpathcurveto{\pgfqpoint{0.858836in}{1.228494in}}{\pgfqpoint{0.862109in}{1.236394in}}{\pgfqpoint{0.862109in}{1.244631in}}%
\pgfpathcurveto{\pgfqpoint{0.862109in}{1.252867in}}{\pgfqpoint{0.858836in}{1.260767in}}{\pgfqpoint{0.853012in}{1.266591in}}%
\pgfpathcurveto{\pgfqpoint{0.847189in}{1.272415in}}{\pgfqpoint{0.839288in}{1.275687in}}{\pgfqpoint{0.831052in}{1.275687in}}%
\pgfpathcurveto{\pgfqpoint{0.822816in}{1.275687in}}{\pgfqpoint{0.814916in}{1.272415in}}{\pgfqpoint{0.809092in}{1.266591in}}%
\pgfpathcurveto{\pgfqpoint{0.803268in}{1.260767in}}{\pgfqpoint{0.799996in}{1.252867in}}{\pgfqpoint{0.799996in}{1.244631in}}%
\pgfpathcurveto{\pgfqpoint{0.799996in}{1.236394in}}{\pgfqpoint{0.803268in}{1.228494in}}{\pgfqpoint{0.809092in}{1.222670in}}%
\pgfpathcurveto{\pgfqpoint{0.814916in}{1.216846in}}{\pgfqpoint{0.822816in}{1.213574in}}{\pgfqpoint{0.831052in}{1.213574in}}%
\pgfpathclose%
\pgfusepath{stroke,fill}%
\end{pgfscope}%
\begin{pgfscope}%
\pgfpathrectangle{\pgfqpoint{0.100000in}{0.220728in}}{\pgfqpoint{3.696000in}{3.696000in}}%
\pgfusepath{clip}%
\pgfsetbuttcap%
\pgfsetroundjoin%
\definecolor{currentfill}{rgb}{0.121569,0.466667,0.705882}%
\pgfsetfillcolor{currentfill}%
\pgfsetfillopacity{0.616596}%
\pgfsetlinewidth{1.003750pt}%
\definecolor{currentstroke}{rgb}{0.121569,0.466667,0.705882}%
\pgfsetstrokecolor{currentstroke}%
\pgfsetstrokeopacity{0.616596}%
\pgfsetdash{}{0pt}%
\pgfpathmoveto{\pgfqpoint{0.831052in}{1.213574in}}%
\pgfpathcurveto{\pgfqpoint{0.839288in}{1.213574in}}{\pgfqpoint{0.847189in}{1.216846in}}{\pgfqpoint{0.853012in}{1.222670in}}%
\pgfpathcurveto{\pgfqpoint{0.858836in}{1.228494in}}{\pgfqpoint{0.862109in}{1.236394in}}{\pgfqpoint{0.862109in}{1.244631in}}%
\pgfpathcurveto{\pgfqpoint{0.862109in}{1.252867in}}{\pgfqpoint{0.858836in}{1.260767in}}{\pgfqpoint{0.853012in}{1.266591in}}%
\pgfpathcurveto{\pgfqpoint{0.847189in}{1.272415in}}{\pgfqpoint{0.839288in}{1.275687in}}{\pgfqpoint{0.831052in}{1.275687in}}%
\pgfpathcurveto{\pgfqpoint{0.822816in}{1.275687in}}{\pgfqpoint{0.814916in}{1.272415in}}{\pgfqpoint{0.809092in}{1.266591in}}%
\pgfpathcurveto{\pgfqpoint{0.803268in}{1.260767in}}{\pgfqpoint{0.799996in}{1.252867in}}{\pgfqpoint{0.799996in}{1.244631in}}%
\pgfpathcurveto{\pgfqpoint{0.799996in}{1.236394in}}{\pgfqpoint{0.803268in}{1.228494in}}{\pgfqpoint{0.809092in}{1.222670in}}%
\pgfpathcurveto{\pgfqpoint{0.814916in}{1.216846in}}{\pgfqpoint{0.822816in}{1.213574in}}{\pgfqpoint{0.831052in}{1.213574in}}%
\pgfpathclose%
\pgfusepath{stroke,fill}%
\end{pgfscope}%
\begin{pgfscope}%
\pgfpathrectangle{\pgfqpoint{0.100000in}{0.220728in}}{\pgfqpoint{3.696000in}{3.696000in}}%
\pgfusepath{clip}%
\pgfsetbuttcap%
\pgfsetroundjoin%
\definecolor{currentfill}{rgb}{0.121569,0.466667,0.705882}%
\pgfsetfillcolor{currentfill}%
\pgfsetfillopacity{0.616596}%
\pgfsetlinewidth{1.003750pt}%
\definecolor{currentstroke}{rgb}{0.121569,0.466667,0.705882}%
\pgfsetstrokecolor{currentstroke}%
\pgfsetstrokeopacity{0.616596}%
\pgfsetdash{}{0pt}%
\pgfpathmoveto{\pgfqpoint{0.831052in}{1.213574in}}%
\pgfpathcurveto{\pgfqpoint{0.839288in}{1.213574in}}{\pgfqpoint{0.847189in}{1.216846in}}{\pgfqpoint{0.853012in}{1.222670in}}%
\pgfpathcurveto{\pgfqpoint{0.858836in}{1.228494in}}{\pgfqpoint{0.862109in}{1.236394in}}{\pgfqpoint{0.862109in}{1.244631in}}%
\pgfpathcurveto{\pgfqpoint{0.862109in}{1.252867in}}{\pgfqpoint{0.858836in}{1.260767in}}{\pgfqpoint{0.853012in}{1.266591in}}%
\pgfpathcurveto{\pgfqpoint{0.847189in}{1.272415in}}{\pgfqpoint{0.839288in}{1.275687in}}{\pgfqpoint{0.831052in}{1.275687in}}%
\pgfpathcurveto{\pgfqpoint{0.822816in}{1.275687in}}{\pgfqpoint{0.814916in}{1.272415in}}{\pgfqpoint{0.809092in}{1.266591in}}%
\pgfpathcurveto{\pgfqpoint{0.803268in}{1.260767in}}{\pgfqpoint{0.799996in}{1.252867in}}{\pgfqpoint{0.799996in}{1.244631in}}%
\pgfpathcurveto{\pgfqpoint{0.799996in}{1.236394in}}{\pgfqpoint{0.803268in}{1.228494in}}{\pgfqpoint{0.809092in}{1.222670in}}%
\pgfpathcurveto{\pgfqpoint{0.814916in}{1.216846in}}{\pgfqpoint{0.822816in}{1.213574in}}{\pgfqpoint{0.831052in}{1.213574in}}%
\pgfpathclose%
\pgfusepath{stroke,fill}%
\end{pgfscope}%
\begin{pgfscope}%
\pgfpathrectangle{\pgfqpoint{0.100000in}{0.220728in}}{\pgfqpoint{3.696000in}{3.696000in}}%
\pgfusepath{clip}%
\pgfsetbuttcap%
\pgfsetroundjoin%
\definecolor{currentfill}{rgb}{0.121569,0.466667,0.705882}%
\pgfsetfillcolor{currentfill}%
\pgfsetfillopacity{0.616596}%
\pgfsetlinewidth{1.003750pt}%
\definecolor{currentstroke}{rgb}{0.121569,0.466667,0.705882}%
\pgfsetstrokecolor{currentstroke}%
\pgfsetstrokeopacity{0.616596}%
\pgfsetdash{}{0pt}%
\pgfpathmoveto{\pgfqpoint{0.831052in}{1.213574in}}%
\pgfpathcurveto{\pgfqpoint{0.839288in}{1.213574in}}{\pgfqpoint{0.847189in}{1.216846in}}{\pgfqpoint{0.853012in}{1.222670in}}%
\pgfpathcurveto{\pgfqpoint{0.858836in}{1.228494in}}{\pgfqpoint{0.862109in}{1.236394in}}{\pgfqpoint{0.862109in}{1.244631in}}%
\pgfpathcurveto{\pgfqpoint{0.862109in}{1.252867in}}{\pgfqpoint{0.858836in}{1.260767in}}{\pgfqpoint{0.853012in}{1.266591in}}%
\pgfpathcurveto{\pgfqpoint{0.847189in}{1.272415in}}{\pgfqpoint{0.839288in}{1.275687in}}{\pgfqpoint{0.831052in}{1.275687in}}%
\pgfpathcurveto{\pgfqpoint{0.822816in}{1.275687in}}{\pgfqpoint{0.814916in}{1.272415in}}{\pgfqpoint{0.809092in}{1.266591in}}%
\pgfpathcurveto{\pgfqpoint{0.803268in}{1.260767in}}{\pgfqpoint{0.799996in}{1.252867in}}{\pgfqpoint{0.799996in}{1.244631in}}%
\pgfpathcurveto{\pgfqpoint{0.799996in}{1.236394in}}{\pgfqpoint{0.803268in}{1.228494in}}{\pgfqpoint{0.809092in}{1.222670in}}%
\pgfpathcurveto{\pgfqpoint{0.814916in}{1.216846in}}{\pgfqpoint{0.822816in}{1.213574in}}{\pgfqpoint{0.831052in}{1.213574in}}%
\pgfpathclose%
\pgfusepath{stroke,fill}%
\end{pgfscope}%
\begin{pgfscope}%
\pgfpathrectangle{\pgfqpoint{0.100000in}{0.220728in}}{\pgfqpoint{3.696000in}{3.696000in}}%
\pgfusepath{clip}%
\pgfsetbuttcap%
\pgfsetroundjoin%
\definecolor{currentfill}{rgb}{0.121569,0.466667,0.705882}%
\pgfsetfillcolor{currentfill}%
\pgfsetfillopacity{0.616596}%
\pgfsetlinewidth{1.003750pt}%
\definecolor{currentstroke}{rgb}{0.121569,0.466667,0.705882}%
\pgfsetstrokecolor{currentstroke}%
\pgfsetstrokeopacity{0.616596}%
\pgfsetdash{}{0pt}%
\pgfpathmoveto{\pgfqpoint{0.831052in}{1.213574in}}%
\pgfpathcurveto{\pgfqpoint{0.839288in}{1.213574in}}{\pgfqpoint{0.847189in}{1.216846in}}{\pgfqpoint{0.853012in}{1.222670in}}%
\pgfpathcurveto{\pgfqpoint{0.858836in}{1.228494in}}{\pgfqpoint{0.862109in}{1.236394in}}{\pgfqpoint{0.862109in}{1.244631in}}%
\pgfpathcurveto{\pgfqpoint{0.862109in}{1.252867in}}{\pgfqpoint{0.858836in}{1.260767in}}{\pgfqpoint{0.853012in}{1.266591in}}%
\pgfpathcurveto{\pgfqpoint{0.847189in}{1.272415in}}{\pgfqpoint{0.839288in}{1.275687in}}{\pgfqpoint{0.831052in}{1.275687in}}%
\pgfpathcurveto{\pgfqpoint{0.822816in}{1.275687in}}{\pgfqpoint{0.814916in}{1.272415in}}{\pgfqpoint{0.809092in}{1.266591in}}%
\pgfpathcurveto{\pgfqpoint{0.803268in}{1.260767in}}{\pgfqpoint{0.799996in}{1.252867in}}{\pgfqpoint{0.799996in}{1.244631in}}%
\pgfpathcurveto{\pgfqpoint{0.799996in}{1.236394in}}{\pgfqpoint{0.803268in}{1.228494in}}{\pgfqpoint{0.809092in}{1.222670in}}%
\pgfpathcurveto{\pgfqpoint{0.814916in}{1.216846in}}{\pgfqpoint{0.822816in}{1.213574in}}{\pgfqpoint{0.831052in}{1.213574in}}%
\pgfpathclose%
\pgfusepath{stroke,fill}%
\end{pgfscope}%
\begin{pgfscope}%
\pgfpathrectangle{\pgfqpoint{0.100000in}{0.220728in}}{\pgfqpoint{3.696000in}{3.696000in}}%
\pgfusepath{clip}%
\pgfsetbuttcap%
\pgfsetroundjoin%
\definecolor{currentfill}{rgb}{0.121569,0.466667,0.705882}%
\pgfsetfillcolor{currentfill}%
\pgfsetfillopacity{0.616596}%
\pgfsetlinewidth{1.003750pt}%
\definecolor{currentstroke}{rgb}{0.121569,0.466667,0.705882}%
\pgfsetstrokecolor{currentstroke}%
\pgfsetstrokeopacity{0.616596}%
\pgfsetdash{}{0pt}%
\pgfpathmoveto{\pgfqpoint{0.831052in}{1.213574in}}%
\pgfpathcurveto{\pgfqpoint{0.839288in}{1.213574in}}{\pgfqpoint{0.847189in}{1.216846in}}{\pgfqpoint{0.853012in}{1.222670in}}%
\pgfpathcurveto{\pgfqpoint{0.858836in}{1.228494in}}{\pgfqpoint{0.862109in}{1.236394in}}{\pgfqpoint{0.862109in}{1.244631in}}%
\pgfpathcurveto{\pgfqpoint{0.862109in}{1.252867in}}{\pgfqpoint{0.858836in}{1.260767in}}{\pgfqpoint{0.853012in}{1.266591in}}%
\pgfpathcurveto{\pgfqpoint{0.847189in}{1.272415in}}{\pgfqpoint{0.839288in}{1.275687in}}{\pgfqpoint{0.831052in}{1.275687in}}%
\pgfpathcurveto{\pgfqpoint{0.822816in}{1.275687in}}{\pgfqpoint{0.814916in}{1.272415in}}{\pgfqpoint{0.809092in}{1.266591in}}%
\pgfpathcurveto{\pgfqpoint{0.803268in}{1.260767in}}{\pgfqpoint{0.799996in}{1.252867in}}{\pgfqpoint{0.799996in}{1.244631in}}%
\pgfpathcurveto{\pgfqpoint{0.799996in}{1.236394in}}{\pgfqpoint{0.803268in}{1.228494in}}{\pgfqpoint{0.809092in}{1.222670in}}%
\pgfpathcurveto{\pgfqpoint{0.814916in}{1.216846in}}{\pgfqpoint{0.822816in}{1.213574in}}{\pgfqpoint{0.831052in}{1.213574in}}%
\pgfpathclose%
\pgfusepath{stroke,fill}%
\end{pgfscope}%
\begin{pgfscope}%
\pgfpathrectangle{\pgfqpoint{0.100000in}{0.220728in}}{\pgfqpoint{3.696000in}{3.696000in}}%
\pgfusepath{clip}%
\pgfsetbuttcap%
\pgfsetroundjoin%
\definecolor{currentfill}{rgb}{0.121569,0.466667,0.705882}%
\pgfsetfillcolor{currentfill}%
\pgfsetfillopacity{0.616596}%
\pgfsetlinewidth{1.003750pt}%
\definecolor{currentstroke}{rgb}{0.121569,0.466667,0.705882}%
\pgfsetstrokecolor{currentstroke}%
\pgfsetstrokeopacity{0.616596}%
\pgfsetdash{}{0pt}%
\pgfpathmoveto{\pgfqpoint{0.831052in}{1.213574in}}%
\pgfpathcurveto{\pgfqpoint{0.839288in}{1.213574in}}{\pgfqpoint{0.847189in}{1.216846in}}{\pgfqpoint{0.853012in}{1.222670in}}%
\pgfpathcurveto{\pgfqpoint{0.858836in}{1.228494in}}{\pgfqpoint{0.862109in}{1.236394in}}{\pgfqpoint{0.862109in}{1.244631in}}%
\pgfpathcurveto{\pgfqpoint{0.862109in}{1.252867in}}{\pgfqpoint{0.858836in}{1.260767in}}{\pgfqpoint{0.853012in}{1.266591in}}%
\pgfpathcurveto{\pgfqpoint{0.847189in}{1.272415in}}{\pgfqpoint{0.839288in}{1.275687in}}{\pgfqpoint{0.831052in}{1.275687in}}%
\pgfpathcurveto{\pgfqpoint{0.822816in}{1.275687in}}{\pgfqpoint{0.814916in}{1.272415in}}{\pgfqpoint{0.809092in}{1.266591in}}%
\pgfpathcurveto{\pgfqpoint{0.803268in}{1.260767in}}{\pgfqpoint{0.799996in}{1.252867in}}{\pgfqpoint{0.799996in}{1.244631in}}%
\pgfpathcurveto{\pgfqpoint{0.799996in}{1.236394in}}{\pgfqpoint{0.803268in}{1.228494in}}{\pgfqpoint{0.809092in}{1.222670in}}%
\pgfpathcurveto{\pgfqpoint{0.814916in}{1.216846in}}{\pgfqpoint{0.822816in}{1.213574in}}{\pgfqpoint{0.831052in}{1.213574in}}%
\pgfpathclose%
\pgfusepath{stroke,fill}%
\end{pgfscope}%
\begin{pgfscope}%
\pgfpathrectangle{\pgfqpoint{0.100000in}{0.220728in}}{\pgfqpoint{3.696000in}{3.696000in}}%
\pgfusepath{clip}%
\pgfsetbuttcap%
\pgfsetroundjoin%
\definecolor{currentfill}{rgb}{0.121569,0.466667,0.705882}%
\pgfsetfillcolor{currentfill}%
\pgfsetfillopacity{0.616596}%
\pgfsetlinewidth{1.003750pt}%
\definecolor{currentstroke}{rgb}{0.121569,0.466667,0.705882}%
\pgfsetstrokecolor{currentstroke}%
\pgfsetstrokeopacity{0.616596}%
\pgfsetdash{}{0pt}%
\pgfpathmoveto{\pgfqpoint{0.831052in}{1.213574in}}%
\pgfpathcurveto{\pgfqpoint{0.839288in}{1.213574in}}{\pgfqpoint{0.847189in}{1.216846in}}{\pgfqpoint{0.853012in}{1.222670in}}%
\pgfpathcurveto{\pgfqpoint{0.858836in}{1.228494in}}{\pgfqpoint{0.862109in}{1.236394in}}{\pgfqpoint{0.862109in}{1.244631in}}%
\pgfpathcurveto{\pgfqpoint{0.862109in}{1.252867in}}{\pgfqpoint{0.858836in}{1.260767in}}{\pgfqpoint{0.853012in}{1.266591in}}%
\pgfpathcurveto{\pgfqpoint{0.847189in}{1.272415in}}{\pgfqpoint{0.839288in}{1.275687in}}{\pgfqpoint{0.831052in}{1.275687in}}%
\pgfpathcurveto{\pgfqpoint{0.822816in}{1.275687in}}{\pgfqpoint{0.814916in}{1.272415in}}{\pgfqpoint{0.809092in}{1.266591in}}%
\pgfpathcurveto{\pgfqpoint{0.803268in}{1.260767in}}{\pgfqpoint{0.799996in}{1.252867in}}{\pgfqpoint{0.799996in}{1.244631in}}%
\pgfpathcurveto{\pgfqpoint{0.799996in}{1.236394in}}{\pgfqpoint{0.803268in}{1.228494in}}{\pgfqpoint{0.809092in}{1.222670in}}%
\pgfpathcurveto{\pgfqpoint{0.814916in}{1.216846in}}{\pgfqpoint{0.822816in}{1.213574in}}{\pgfqpoint{0.831052in}{1.213574in}}%
\pgfpathclose%
\pgfusepath{stroke,fill}%
\end{pgfscope}%
\begin{pgfscope}%
\pgfpathrectangle{\pgfqpoint{0.100000in}{0.220728in}}{\pgfqpoint{3.696000in}{3.696000in}}%
\pgfusepath{clip}%
\pgfsetbuttcap%
\pgfsetroundjoin%
\definecolor{currentfill}{rgb}{0.121569,0.466667,0.705882}%
\pgfsetfillcolor{currentfill}%
\pgfsetfillopacity{0.616596}%
\pgfsetlinewidth{1.003750pt}%
\definecolor{currentstroke}{rgb}{0.121569,0.466667,0.705882}%
\pgfsetstrokecolor{currentstroke}%
\pgfsetstrokeopacity{0.616596}%
\pgfsetdash{}{0pt}%
\pgfpathmoveto{\pgfqpoint{0.831052in}{1.213574in}}%
\pgfpathcurveto{\pgfqpoint{0.839288in}{1.213574in}}{\pgfqpoint{0.847189in}{1.216846in}}{\pgfqpoint{0.853012in}{1.222670in}}%
\pgfpathcurveto{\pgfqpoint{0.858836in}{1.228494in}}{\pgfqpoint{0.862109in}{1.236394in}}{\pgfqpoint{0.862109in}{1.244631in}}%
\pgfpathcurveto{\pgfqpoint{0.862109in}{1.252867in}}{\pgfqpoint{0.858836in}{1.260767in}}{\pgfqpoint{0.853012in}{1.266591in}}%
\pgfpathcurveto{\pgfqpoint{0.847189in}{1.272415in}}{\pgfqpoint{0.839288in}{1.275687in}}{\pgfqpoint{0.831052in}{1.275687in}}%
\pgfpathcurveto{\pgfqpoint{0.822816in}{1.275687in}}{\pgfqpoint{0.814916in}{1.272415in}}{\pgfqpoint{0.809092in}{1.266591in}}%
\pgfpathcurveto{\pgfqpoint{0.803268in}{1.260767in}}{\pgfqpoint{0.799996in}{1.252867in}}{\pgfqpoint{0.799996in}{1.244631in}}%
\pgfpathcurveto{\pgfqpoint{0.799996in}{1.236394in}}{\pgfqpoint{0.803268in}{1.228494in}}{\pgfqpoint{0.809092in}{1.222670in}}%
\pgfpathcurveto{\pgfqpoint{0.814916in}{1.216846in}}{\pgfqpoint{0.822816in}{1.213574in}}{\pgfqpoint{0.831052in}{1.213574in}}%
\pgfpathclose%
\pgfusepath{stroke,fill}%
\end{pgfscope}%
\begin{pgfscope}%
\pgfpathrectangle{\pgfqpoint{0.100000in}{0.220728in}}{\pgfqpoint{3.696000in}{3.696000in}}%
\pgfusepath{clip}%
\pgfsetbuttcap%
\pgfsetroundjoin%
\definecolor{currentfill}{rgb}{0.121569,0.466667,0.705882}%
\pgfsetfillcolor{currentfill}%
\pgfsetfillopacity{0.616596}%
\pgfsetlinewidth{1.003750pt}%
\definecolor{currentstroke}{rgb}{0.121569,0.466667,0.705882}%
\pgfsetstrokecolor{currentstroke}%
\pgfsetstrokeopacity{0.616596}%
\pgfsetdash{}{0pt}%
\pgfpathmoveto{\pgfqpoint{0.831052in}{1.213574in}}%
\pgfpathcurveto{\pgfqpoint{0.839288in}{1.213574in}}{\pgfqpoint{0.847189in}{1.216846in}}{\pgfqpoint{0.853012in}{1.222670in}}%
\pgfpathcurveto{\pgfqpoint{0.858836in}{1.228494in}}{\pgfqpoint{0.862109in}{1.236394in}}{\pgfqpoint{0.862109in}{1.244631in}}%
\pgfpathcurveto{\pgfqpoint{0.862109in}{1.252867in}}{\pgfqpoint{0.858836in}{1.260767in}}{\pgfqpoint{0.853012in}{1.266591in}}%
\pgfpathcurveto{\pgfqpoint{0.847189in}{1.272415in}}{\pgfqpoint{0.839288in}{1.275687in}}{\pgfqpoint{0.831052in}{1.275687in}}%
\pgfpathcurveto{\pgfqpoint{0.822816in}{1.275687in}}{\pgfqpoint{0.814916in}{1.272415in}}{\pgfqpoint{0.809092in}{1.266591in}}%
\pgfpathcurveto{\pgfqpoint{0.803268in}{1.260767in}}{\pgfqpoint{0.799996in}{1.252867in}}{\pgfqpoint{0.799996in}{1.244631in}}%
\pgfpathcurveto{\pgfqpoint{0.799996in}{1.236394in}}{\pgfqpoint{0.803268in}{1.228494in}}{\pgfqpoint{0.809092in}{1.222670in}}%
\pgfpathcurveto{\pgfqpoint{0.814916in}{1.216846in}}{\pgfqpoint{0.822816in}{1.213574in}}{\pgfqpoint{0.831052in}{1.213574in}}%
\pgfpathclose%
\pgfusepath{stroke,fill}%
\end{pgfscope}%
\begin{pgfscope}%
\pgfpathrectangle{\pgfqpoint{0.100000in}{0.220728in}}{\pgfqpoint{3.696000in}{3.696000in}}%
\pgfusepath{clip}%
\pgfsetbuttcap%
\pgfsetroundjoin%
\definecolor{currentfill}{rgb}{0.121569,0.466667,0.705882}%
\pgfsetfillcolor{currentfill}%
\pgfsetfillopacity{0.616596}%
\pgfsetlinewidth{1.003750pt}%
\definecolor{currentstroke}{rgb}{0.121569,0.466667,0.705882}%
\pgfsetstrokecolor{currentstroke}%
\pgfsetstrokeopacity{0.616596}%
\pgfsetdash{}{0pt}%
\pgfpathmoveto{\pgfqpoint{0.831052in}{1.213574in}}%
\pgfpathcurveto{\pgfqpoint{0.839288in}{1.213574in}}{\pgfqpoint{0.847189in}{1.216846in}}{\pgfqpoint{0.853012in}{1.222670in}}%
\pgfpathcurveto{\pgfqpoint{0.858836in}{1.228494in}}{\pgfqpoint{0.862109in}{1.236394in}}{\pgfqpoint{0.862109in}{1.244631in}}%
\pgfpathcurveto{\pgfqpoint{0.862109in}{1.252867in}}{\pgfqpoint{0.858836in}{1.260767in}}{\pgfqpoint{0.853012in}{1.266591in}}%
\pgfpathcurveto{\pgfqpoint{0.847189in}{1.272415in}}{\pgfqpoint{0.839288in}{1.275687in}}{\pgfqpoint{0.831052in}{1.275687in}}%
\pgfpathcurveto{\pgfqpoint{0.822816in}{1.275687in}}{\pgfqpoint{0.814916in}{1.272415in}}{\pgfqpoint{0.809092in}{1.266591in}}%
\pgfpathcurveto{\pgfqpoint{0.803268in}{1.260767in}}{\pgfqpoint{0.799996in}{1.252867in}}{\pgfqpoint{0.799996in}{1.244631in}}%
\pgfpathcurveto{\pgfqpoint{0.799996in}{1.236394in}}{\pgfqpoint{0.803268in}{1.228494in}}{\pgfqpoint{0.809092in}{1.222670in}}%
\pgfpathcurveto{\pgfqpoint{0.814916in}{1.216846in}}{\pgfqpoint{0.822816in}{1.213574in}}{\pgfqpoint{0.831052in}{1.213574in}}%
\pgfpathclose%
\pgfusepath{stroke,fill}%
\end{pgfscope}%
\begin{pgfscope}%
\pgfpathrectangle{\pgfqpoint{0.100000in}{0.220728in}}{\pgfqpoint{3.696000in}{3.696000in}}%
\pgfusepath{clip}%
\pgfsetbuttcap%
\pgfsetroundjoin%
\definecolor{currentfill}{rgb}{0.121569,0.466667,0.705882}%
\pgfsetfillcolor{currentfill}%
\pgfsetfillopacity{0.616596}%
\pgfsetlinewidth{1.003750pt}%
\definecolor{currentstroke}{rgb}{0.121569,0.466667,0.705882}%
\pgfsetstrokecolor{currentstroke}%
\pgfsetstrokeopacity{0.616596}%
\pgfsetdash{}{0pt}%
\pgfpathmoveto{\pgfqpoint{0.831052in}{1.213574in}}%
\pgfpathcurveto{\pgfqpoint{0.839288in}{1.213574in}}{\pgfqpoint{0.847189in}{1.216846in}}{\pgfqpoint{0.853012in}{1.222670in}}%
\pgfpathcurveto{\pgfqpoint{0.858836in}{1.228494in}}{\pgfqpoint{0.862109in}{1.236394in}}{\pgfqpoint{0.862109in}{1.244631in}}%
\pgfpathcurveto{\pgfqpoint{0.862109in}{1.252867in}}{\pgfqpoint{0.858836in}{1.260767in}}{\pgfqpoint{0.853012in}{1.266591in}}%
\pgfpathcurveto{\pgfqpoint{0.847189in}{1.272415in}}{\pgfqpoint{0.839288in}{1.275687in}}{\pgfqpoint{0.831052in}{1.275687in}}%
\pgfpathcurveto{\pgfqpoint{0.822816in}{1.275687in}}{\pgfqpoint{0.814916in}{1.272415in}}{\pgfqpoint{0.809092in}{1.266591in}}%
\pgfpathcurveto{\pgfqpoint{0.803268in}{1.260767in}}{\pgfqpoint{0.799996in}{1.252867in}}{\pgfqpoint{0.799996in}{1.244631in}}%
\pgfpathcurveto{\pgfqpoint{0.799996in}{1.236394in}}{\pgfqpoint{0.803268in}{1.228494in}}{\pgfqpoint{0.809092in}{1.222670in}}%
\pgfpathcurveto{\pgfqpoint{0.814916in}{1.216846in}}{\pgfqpoint{0.822816in}{1.213574in}}{\pgfqpoint{0.831052in}{1.213574in}}%
\pgfpathclose%
\pgfusepath{stroke,fill}%
\end{pgfscope}%
\begin{pgfscope}%
\pgfpathrectangle{\pgfqpoint{0.100000in}{0.220728in}}{\pgfqpoint{3.696000in}{3.696000in}}%
\pgfusepath{clip}%
\pgfsetbuttcap%
\pgfsetroundjoin%
\definecolor{currentfill}{rgb}{0.121569,0.466667,0.705882}%
\pgfsetfillcolor{currentfill}%
\pgfsetfillopacity{0.616596}%
\pgfsetlinewidth{1.003750pt}%
\definecolor{currentstroke}{rgb}{0.121569,0.466667,0.705882}%
\pgfsetstrokecolor{currentstroke}%
\pgfsetstrokeopacity{0.616596}%
\pgfsetdash{}{0pt}%
\pgfpathmoveto{\pgfqpoint{0.831052in}{1.213574in}}%
\pgfpathcurveto{\pgfqpoint{0.839288in}{1.213574in}}{\pgfqpoint{0.847189in}{1.216846in}}{\pgfqpoint{0.853012in}{1.222670in}}%
\pgfpathcurveto{\pgfqpoint{0.858836in}{1.228494in}}{\pgfqpoint{0.862109in}{1.236394in}}{\pgfqpoint{0.862109in}{1.244631in}}%
\pgfpathcurveto{\pgfqpoint{0.862109in}{1.252867in}}{\pgfqpoint{0.858836in}{1.260767in}}{\pgfqpoint{0.853012in}{1.266591in}}%
\pgfpathcurveto{\pgfqpoint{0.847189in}{1.272415in}}{\pgfqpoint{0.839288in}{1.275687in}}{\pgfqpoint{0.831052in}{1.275687in}}%
\pgfpathcurveto{\pgfqpoint{0.822816in}{1.275687in}}{\pgfqpoint{0.814916in}{1.272415in}}{\pgfqpoint{0.809092in}{1.266591in}}%
\pgfpathcurveto{\pgfqpoint{0.803268in}{1.260767in}}{\pgfqpoint{0.799996in}{1.252867in}}{\pgfqpoint{0.799996in}{1.244631in}}%
\pgfpathcurveto{\pgfqpoint{0.799996in}{1.236394in}}{\pgfqpoint{0.803268in}{1.228494in}}{\pgfqpoint{0.809092in}{1.222670in}}%
\pgfpathcurveto{\pgfqpoint{0.814916in}{1.216846in}}{\pgfqpoint{0.822816in}{1.213574in}}{\pgfqpoint{0.831052in}{1.213574in}}%
\pgfpathclose%
\pgfusepath{stroke,fill}%
\end{pgfscope}%
\begin{pgfscope}%
\pgfpathrectangle{\pgfqpoint{0.100000in}{0.220728in}}{\pgfqpoint{3.696000in}{3.696000in}}%
\pgfusepath{clip}%
\pgfsetbuttcap%
\pgfsetroundjoin%
\definecolor{currentfill}{rgb}{0.121569,0.466667,0.705882}%
\pgfsetfillcolor{currentfill}%
\pgfsetfillopacity{0.616596}%
\pgfsetlinewidth{1.003750pt}%
\definecolor{currentstroke}{rgb}{0.121569,0.466667,0.705882}%
\pgfsetstrokecolor{currentstroke}%
\pgfsetstrokeopacity{0.616596}%
\pgfsetdash{}{0pt}%
\pgfpathmoveto{\pgfqpoint{0.831052in}{1.213574in}}%
\pgfpathcurveto{\pgfqpoint{0.839288in}{1.213574in}}{\pgfqpoint{0.847189in}{1.216846in}}{\pgfqpoint{0.853012in}{1.222670in}}%
\pgfpathcurveto{\pgfqpoint{0.858836in}{1.228494in}}{\pgfqpoint{0.862109in}{1.236394in}}{\pgfqpoint{0.862109in}{1.244631in}}%
\pgfpathcurveto{\pgfqpoint{0.862109in}{1.252867in}}{\pgfqpoint{0.858836in}{1.260767in}}{\pgfqpoint{0.853012in}{1.266591in}}%
\pgfpathcurveto{\pgfqpoint{0.847189in}{1.272415in}}{\pgfqpoint{0.839288in}{1.275687in}}{\pgfqpoint{0.831052in}{1.275687in}}%
\pgfpathcurveto{\pgfqpoint{0.822816in}{1.275687in}}{\pgfqpoint{0.814916in}{1.272415in}}{\pgfqpoint{0.809092in}{1.266591in}}%
\pgfpathcurveto{\pgfqpoint{0.803268in}{1.260767in}}{\pgfqpoint{0.799996in}{1.252867in}}{\pgfqpoint{0.799996in}{1.244631in}}%
\pgfpathcurveto{\pgfqpoint{0.799996in}{1.236394in}}{\pgfqpoint{0.803268in}{1.228494in}}{\pgfqpoint{0.809092in}{1.222670in}}%
\pgfpathcurveto{\pgfqpoint{0.814916in}{1.216846in}}{\pgfqpoint{0.822816in}{1.213574in}}{\pgfqpoint{0.831052in}{1.213574in}}%
\pgfpathclose%
\pgfusepath{stroke,fill}%
\end{pgfscope}%
\begin{pgfscope}%
\pgfpathrectangle{\pgfqpoint{0.100000in}{0.220728in}}{\pgfqpoint{3.696000in}{3.696000in}}%
\pgfusepath{clip}%
\pgfsetbuttcap%
\pgfsetroundjoin%
\definecolor{currentfill}{rgb}{0.121569,0.466667,0.705882}%
\pgfsetfillcolor{currentfill}%
\pgfsetfillopacity{0.616596}%
\pgfsetlinewidth{1.003750pt}%
\definecolor{currentstroke}{rgb}{0.121569,0.466667,0.705882}%
\pgfsetstrokecolor{currentstroke}%
\pgfsetstrokeopacity{0.616596}%
\pgfsetdash{}{0pt}%
\pgfpathmoveto{\pgfqpoint{0.831052in}{1.213574in}}%
\pgfpathcurveto{\pgfqpoint{0.839288in}{1.213574in}}{\pgfqpoint{0.847189in}{1.216846in}}{\pgfqpoint{0.853012in}{1.222670in}}%
\pgfpathcurveto{\pgfqpoint{0.858836in}{1.228494in}}{\pgfqpoint{0.862109in}{1.236394in}}{\pgfqpoint{0.862109in}{1.244631in}}%
\pgfpathcurveto{\pgfqpoint{0.862109in}{1.252867in}}{\pgfqpoint{0.858836in}{1.260767in}}{\pgfqpoint{0.853012in}{1.266591in}}%
\pgfpathcurveto{\pgfqpoint{0.847189in}{1.272415in}}{\pgfqpoint{0.839288in}{1.275687in}}{\pgfqpoint{0.831052in}{1.275687in}}%
\pgfpathcurveto{\pgfqpoint{0.822816in}{1.275687in}}{\pgfqpoint{0.814916in}{1.272415in}}{\pgfqpoint{0.809092in}{1.266591in}}%
\pgfpathcurveto{\pgfqpoint{0.803268in}{1.260767in}}{\pgfqpoint{0.799996in}{1.252867in}}{\pgfqpoint{0.799996in}{1.244631in}}%
\pgfpathcurveto{\pgfqpoint{0.799996in}{1.236394in}}{\pgfqpoint{0.803268in}{1.228494in}}{\pgfqpoint{0.809092in}{1.222670in}}%
\pgfpathcurveto{\pgfqpoint{0.814916in}{1.216846in}}{\pgfqpoint{0.822816in}{1.213574in}}{\pgfqpoint{0.831052in}{1.213574in}}%
\pgfpathclose%
\pgfusepath{stroke,fill}%
\end{pgfscope}%
\begin{pgfscope}%
\pgfpathrectangle{\pgfqpoint{0.100000in}{0.220728in}}{\pgfqpoint{3.696000in}{3.696000in}}%
\pgfusepath{clip}%
\pgfsetbuttcap%
\pgfsetroundjoin%
\definecolor{currentfill}{rgb}{0.121569,0.466667,0.705882}%
\pgfsetfillcolor{currentfill}%
\pgfsetfillopacity{0.616596}%
\pgfsetlinewidth{1.003750pt}%
\definecolor{currentstroke}{rgb}{0.121569,0.466667,0.705882}%
\pgfsetstrokecolor{currentstroke}%
\pgfsetstrokeopacity{0.616596}%
\pgfsetdash{}{0pt}%
\pgfpathmoveto{\pgfqpoint{0.831052in}{1.213574in}}%
\pgfpathcurveto{\pgfqpoint{0.839288in}{1.213574in}}{\pgfqpoint{0.847189in}{1.216846in}}{\pgfqpoint{0.853012in}{1.222670in}}%
\pgfpathcurveto{\pgfqpoint{0.858836in}{1.228494in}}{\pgfqpoint{0.862109in}{1.236394in}}{\pgfqpoint{0.862109in}{1.244631in}}%
\pgfpathcurveto{\pgfqpoint{0.862109in}{1.252867in}}{\pgfqpoint{0.858836in}{1.260767in}}{\pgfqpoint{0.853012in}{1.266591in}}%
\pgfpathcurveto{\pgfqpoint{0.847189in}{1.272415in}}{\pgfqpoint{0.839288in}{1.275687in}}{\pgfqpoint{0.831052in}{1.275687in}}%
\pgfpathcurveto{\pgfqpoint{0.822816in}{1.275687in}}{\pgfqpoint{0.814916in}{1.272415in}}{\pgfqpoint{0.809092in}{1.266591in}}%
\pgfpathcurveto{\pgfqpoint{0.803268in}{1.260767in}}{\pgfqpoint{0.799996in}{1.252867in}}{\pgfqpoint{0.799996in}{1.244631in}}%
\pgfpathcurveto{\pgfqpoint{0.799996in}{1.236394in}}{\pgfqpoint{0.803268in}{1.228494in}}{\pgfqpoint{0.809092in}{1.222670in}}%
\pgfpathcurveto{\pgfqpoint{0.814916in}{1.216846in}}{\pgfqpoint{0.822816in}{1.213574in}}{\pgfqpoint{0.831052in}{1.213574in}}%
\pgfpathclose%
\pgfusepath{stroke,fill}%
\end{pgfscope}%
\begin{pgfscope}%
\pgfpathrectangle{\pgfqpoint{0.100000in}{0.220728in}}{\pgfqpoint{3.696000in}{3.696000in}}%
\pgfusepath{clip}%
\pgfsetbuttcap%
\pgfsetroundjoin%
\definecolor{currentfill}{rgb}{0.121569,0.466667,0.705882}%
\pgfsetfillcolor{currentfill}%
\pgfsetfillopacity{0.616596}%
\pgfsetlinewidth{1.003750pt}%
\definecolor{currentstroke}{rgb}{0.121569,0.466667,0.705882}%
\pgfsetstrokecolor{currentstroke}%
\pgfsetstrokeopacity{0.616596}%
\pgfsetdash{}{0pt}%
\pgfpathmoveto{\pgfqpoint{0.831052in}{1.213574in}}%
\pgfpathcurveto{\pgfqpoint{0.839288in}{1.213574in}}{\pgfqpoint{0.847189in}{1.216846in}}{\pgfqpoint{0.853012in}{1.222670in}}%
\pgfpathcurveto{\pgfqpoint{0.858836in}{1.228494in}}{\pgfqpoint{0.862109in}{1.236394in}}{\pgfqpoint{0.862109in}{1.244631in}}%
\pgfpathcurveto{\pgfqpoint{0.862109in}{1.252867in}}{\pgfqpoint{0.858836in}{1.260767in}}{\pgfqpoint{0.853012in}{1.266591in}}%
\pgfpathcurveto{\pgfqpoint{0.847189in}{1.272415in}}{\pgfqpoint{0.839288in}{1.275687in}}{\pgfqpoint{0.831052in}{1.275687in}}%
\pgfpathcurveto{\pgfqpoint{0.822816in}{1.275687in}}{\pgfqpoint{0.814916in}{1.272415in}}{\pgfqpoint{0.809092in}{1.266591in}}%
\pgfpathcurveto{\pgfqpoint{0.803268in}{1.260767in}}{\pgfqpoint{0.799996in}{1.252867in}}{\pgfqpoint{0.799996in}{1.244631in}}%
\pgfpathcurveto{\pgfqpoint{0.799996in}{1.236394in}}{\pgfqpoint{0.803268in}{1.228494in}}{\pgfqpoint{0.809092in}{1.222670in}}%
\pgfpathcurveto{\pgfqpoint{0.814916in}{1.216846in}}{\pgfqpoint{0.822816in}{1.213574in}}{\pgfqpoint{0.831052in}{1.213574in}}%
\pgfpathclose%
\pgfusepath{stroke,fill}%
\end{pgfscope}%
\begin{pgfscope}%
\pgfpathrectangle{\pgfqpoint{0.100000in}{0.220728in}}{\pgfqpoint{3.696000in}{3.696000in}}%
\pgfusepath{clip}%
\pgfsetbuttcap%
\pgfsetroundjoin%
\definecolor{currentfill}{rgb}{0.121569,0.466667,0.705882}%
\pgfsetfillcolor{currentfill}%
\pgfsetfillopacity{0.616596}%
\pgfsetlinewidth{1.003750pt}%
\definecolor{currentstroke}{rgb}{0.121569,0.466667,0.705882}%
\pgfsetstrokecolor{currentstroke}%
\pgfsetstrokeopacity{0.616596}%
\pgfsetdash{}{0pt}%
\pgfpathmoveto{\pgfqpoint{0.831052in}{1.213574in}}%
\pgfpathcurveto{\pgfqpoint{0.839288in}{1.213574in}}{\pgfqpoint{0.847189in}{1.216846in}}{\pgfqpoint{0.853012in}{1.222670in}}%
\pgfpathcurveto{\pgfqpoint{0.858836in}{1.228494in}}{\pgfqpoint{0.862109in}{1.236394in}}{\pgfqpoint{0.862109in}{1.244631in}}%
\pgfpathcurveto{\pgfqpoint{0.862109in}{1.252867in}}{\pgfqpoint{0.858836in}{1.260767in}}{\pgfqpoint{0.853012in}{1.266591in}}%
\pgfpathcurveto{\pgfqpoint{0.847189in}{1.272415in}}{\pgfqpoint{0.839288in}{1.275687in}}{\pgfqpoint{0.831052in}{1.275687in}}%
\pgfpathcurveto{\pgfqpoint{0.822816in}{1.275687in}}{\pgfqpoint{0.814916in}{1.272415in}}{\pgfqpoint{0.809092in}{1.266591in}}%
\pgfpathcurveto{\pgfqpoint{0.803268in}{1.260767in}}{\pgfqpoint{0.799996in}{1.252867in}}{\pgfqpoint{0.799996in}{1.244631in}}%
\pgfpathcurveto{\pgfqpoint{0.799996in}{1.236394in}}{\pgfqpoint{0.803268in}{1.228494in}}{\pgfqpoint{0.809092in}{1.222670in}}%
\pgfpathcurveto{\pgfqpoint{0.814916in}{1.216846in}}{\pgfqpoint{0.822816in}{1.213574in}}{\pgfqpoint{0.831052in}{1.213574in}}%
\pgfpathclose%
\pgfusepath{stroke,fill}%
\end{pgfscope}%
\begin{pgfscope}%
\pgfpathrectangle{\pgfqpoint{0.100000in}{0.220728in}}{\pgfqpoint{3.696000in}{3.696000in}}%
\pgfusepath{clip}%
\pgfsetbuttcap%
\pgfsetroundjoin%
\definecolor{currentfill}{rgb}{0.121569,0.466667,0.705882}%
\pgfsetfillcolor{currentfill}%
\pgfsetfillopacity{0.616596}%
\pgfsetlinewidth{1.003750pt}%
\definecolor{currentstroke}{rgb}{0.121569,0.466667,0.705882}%
\pgfsetstrokecolor{currentstroke}%
\pgfsetstrokeopacity{0.616596}%
\pgfsetdash{}{0pt}%
\pgfpathmoveto{\pgfqpoint{0.831052in}{1.213574in}}%
\pgfpathcurveto{\pgfqpoint{0.839288in}{1.213574in}}{\pgfqpoint{0.847189in}{1.216846in}}{\pgfqpoint{0.853012in}{1.222670in}}%
\pgfpathcurveto{\pgfqpoint{0.858836in}{1.228494in}}{\pgfqpoint{0.862109in}{1.236394in}}{\pgfqpoint{0.862109in}{1.244631in}}%
\pgfpathcurveto{\pgfqpoint{0.862109in}{1.252867in}}{\pgfqpoint{0.858836in}{1.260767in}}{\pgfqpoint{0.853012in}{1.266591in}}%
\pgfpathcurveto{\pgfqpoint{0.847189in}{1.272415in}}{\pgfqpoint{0.839288in}{1.275687in}}{\pgfqpoint{0.831052in}{1.275687in}}%
\pgfpathcurveto{\pgfqpoint{0.822816in}{1.275687in}}{\pgfqpoint{0.814916in}{1.272415in}}{\pgfqpoint{0.809092in}{1.266591in}}%
\pgfpathcurveto{\pgfqpoint{0.803268in}{1.260767in}}{\pgfqpoint{0.799996in}{1.252867in}}{\pgfqpoint{0.799996in}{1.244631in}}%
\pgfpathcurveto{\pgfqpoint{0.799996in}{1.236394in}}{\pgfqpoint{0.803268in}{1.228494in}}{\pgfqpoint{0.809092in}{1.222670in}}%
\pgfpathcurveto{\pgfqpoint{0.814916in}{1.216846in}}{\pgfqpoint{0.822816in}{1.213574in}}{\pgfqpoint{0.831052in}{1.213574in}}%
\pgfpathclose%
\pgfusepath{stroke,fill}%
\end{pgfscope}%
\begin{pgfscope}%
\pgfpathrectangle{\pgfqpoint{0.100000in}{0.220728in}}{\pgfqpoint{3.696000in}{3.696000in}}%
\pgfusepath{clip}%
\pgfsetbuttcap%
\pgfsetroundjoin%
\definecolor{currentfill}{rgb}{0.121569,0.466667,0.705882}%
\pgfsetfillcolor{currentfill}%
\pgfsetfillopacity{0.616596}%
\pgfsetlinewidth{1.003750pt}%
\definecolor{currentstroke}{rgb}{0.121569,0.466667,0.705882}%
\pgfsetstrokecolor{currentstroke}%
\pgfsetstrokeopacity{0.616596}%
\pgfsetdash{}{0pt}%
\pgfpathmoveto{\pgfqpoint{0.831052in}{1.213574in}}%
\pgfpathcurveto{\pgfqpoint{0.839288in}{1.213574in}}{\pgfqpoint{0.847189in}{1.216846in}}{\pgfqpoint{0.853012in}{1.222670in}}%
\pgfpathcurveto{\pgfqpoint{0.858836in}{1.228494in}}{\pgfqpoint{0.862109in}{1.236394in}}{\pgfqpoint{0.862109in}{1.244631in}}%
\pgfpathcurveto{\pgfqpoint{0.862109in}{1.252867in}}{\pgfqpoint{0.858836in}{1.260767in}}{\pgfqpoint{0.853012in}{1.266591in}}%
\pgfpathcurveto{\pgfqpoint{0.847189in}{1.272415in}}{\pgfqpoint{0.839288in}{1.275687in}}{\pgfqpoint{0.831052in}{1.275687in}}%
\pgfpathcurveto{\pgfqpoint{0.822816in}{1.275687in}}{\pgfqpoint{0.814916in}{1.272415in}}{\pgfqpoint{0.809092in}{1.266591in}}%
\pgfpathcurveto{\pgfqpoint{0.803268in}{1.260767in}}{\pgfqpoint{0.799996in}{1.252867in}}{\pgfqpoint{0.799996in}{1.244631in}}%
\pgfpathcurveto{\pgfqpoint{0.799996in}{1.236394in}}{\pgfqpoint{0.803268in}{1.228494in}}{\pgfqpoint{0.809092in}{1.222670in}}%
\pgfpathcurveto{\pgfqpoint{0.814916in}{1.216846in}}{\pgfqpoint{0.822816in}{1.213574in}}{\pgfqpoint{0.831052in}{1.213574in}}%
\pgfpathclose%
\pgfusepath{stroke,fill}%
\end{pgfscope}%
\begin{pgfscope}%
\pgfpathrectangle{\pgfqpoint{0.100000in}{0.220728in}}{\pgfqpoint{3.696000in}{3.696000in}}%
\pgfusepath{clip}%
\pgfsetbuttcap%
\pgfsetroundjoin%
\definecolor{currentfill}{rgb}{0.121569,0.466667,0.705882}%
\pgfsetfillcolor{currentfill}%
\pgfsetfillopacity{0.616596}%
\pgfsetlinewidth{1.003750pt}%
\definecolor{currentstroke}{rgb}{0.121569,0.466667,0.705882}%
\pgfsetstrokecolor{currentstroke}%
\pgfsetstrokeopacity{0.616596}%
\pgfsetdash{}{0pt}%
\pgfpathmoveto{\pgfqpoint{0.831052in}{1.213574in}}%
\pgfpathcurveto{\pgfqpoint{0.839288in}{1.213574in}}{\pgfqpoint{0.847189in}{1.216846in}}{\pgfqpoint{0.853012in}{1.222670in}}%
\pgfpathcurveto{\pgfqpoint{0.858836in}{1.228494in}}{\pgfqpoint{0.862109in}{1.236394in}}{\pgfqpoint{0.862109in}{1.244631in}}%
\pgfpathcurveto{\pgfqpoint{0.862109in}{1.252867in}}{\pgfqpoint{0.858836in}{1.260767in}}{\pgfqpoint{0.853012in}{1.266591in}}%
\pgfpathcurveto{\pgfqpoint{0.847189in}{1.272415in}}{\pgfqpoint{0.839288in}{1.275687in}}{\pgfqpoint{0.831052in}{1.275687in}}%
\pgfpathcurveto{\pgfqpoint{0.822816in}{1.275687in}}{\pgfqpoint{0.814916in}{1.272415in}}{\pgfqpoint{0.809092in}{1.266591in}}%
\pgfpathcurveto{\pgfqpoint{0.803268in}{1.260767in}}{\pgfqpoint{0.799996in}{1.252867in}}{\pgfqpoint{0.799996in}{1.244631in}}%
\pgfpathcurveto{\pgfqpoint{0.799996in}{1.236394in}}{\pgfqpoint{0.803268in}{1.228494in}}{\pgfqpoint{0.809092in}{1.222670in}}%
\pgfpathcurveto{\pgfqpoint{0.814916in}{1.216846in}}{\pgfqpoint{0.822816in}{1.213574in}}{\pgfqpoint{0.831052in}{1.213574in}}%
\pgfpathclose%
\pgfusepath{stroke,fill}%
\end{pgfscope}%
\begin{pgfscope}%
\pgfpathrectangle{\pgfqpoint{0.100000in}{0.220728in}}{\pgfqpoint{3.696000in}{3.696000in}}%
\pgfusepath{clip}%
\pgfsetbuttcap%
\pgfsetroundjoin%
\definecolor{currentfill}{rgb}{0.121569,0.466667,0.705882}%
\pgfsetfillcolor{currentfill}%
\pgfsetfillopacity{0.616596}%
\pgfsetlinewidth{1.003750pt}%
\definecolor{currentstroke}{rgb}{0.121569,0.466667,0.705882}%
\pgfsetstrokecolor{currentstroke}%
\pgfsetstrokeopacity{0.616596}%
\pgfsetdash{}{0pt}%
\pgfpathmoveto{\pgfqpoint{0.831052in}{1.213574in}}%
\pgfpathcurveto{\pgfqpoint{0.839288in}{1.213574in}}{\pgfqpoint{0.847189in}{1.216846in}}{\pgfqpoint{0.853012in}{1.222670in}}%
\pgfpathcurveto{\pgfqpoint{0.858836in}{1.228494in}}{\pgfqpoint{0.862109in}{1.236394in}}{\pgfqpoint{0.862109in}{1.244631in}}%
\pgfpathcurveto{\pgfqpoint{0.862109in}{1.252867in}}{\pgfqpoint{0.858836in}{1.260767in}}{\pgfqpoint{0.853012in}{1.266591in}}%
\pgfpathcurveto{\pgfqpoint{0.847189in}{1.272415in}}{\pgfqpoint{0.839288in}{1.275687in}}{\pgfqpoint{0.831052in}{1.275687in}}%
\pgfpathcurveto{\pgfqpoint{0.822816in}{1.275687in}}{\pgfqpoint{0.814916in}{1.272415in}}{\pgfqpoint{0.809092in}{1.266591in}}%
\pgfpathcurveto{\pgfqpoint{0.803268in}{1.260767in}}{\pgfqpoint{0.799996in}{1.252867in}}{\pgfqpoint{0.799996in}{1.244631in}}%
\pgfpathcurveto{\pgfqpoint{0.799996in}{1.236394in}}{\pgfqpoint{0.803268in}{1.228494in}}{\pgfqpoint{0.809092in}{1.222670in}}%
\pgfpathcurveto{\pgfqpoint{0.814916in}{1.216846in}}{\pgfqpoint{0.822816in}{1.213574in}}{\pgfqpoint{0.831052in}{1.213574in}}%
\pgfpathclose%
\pgfusepath{stroke,fill}%
\end{pgfscope}%
\begin{pgfscope}%
\pgfpathrectangle{\pgfqpoint{0.100000in}{0.220728in}}{\pgfqpoint{3.696000in}{3.696000in}}%
\pgfusepath{clip}%
\pgfsetbuttcap%
\pgfsetroundjoin%
\definecolor{currentfill}{rgb}{0.121569,0.466667,0.705882}%
\pgfsetfillcolor{currentfill}%
\pgfsetfillopacity{0.616596}%
\pgfsetlinewidth{1.003750pt}%
\definecolor{currentstroke}{rgb}{0.121569,0.466667,0.705882}%
\pgfsetstrokecolor{currentstroke}%
\pgfsetstrokeopacity{0.616596}%
\pgfsetdash{}{0pt}%
\pgfpathmoveto{\pgfqpoint{0.831052in}{1.213574in}}%
\pgfpathcurveto{\pgfqpoint{0.839288in}{1.213574in}}{\pgfqpoint{0.847189in}{1.216846in}}{\pgfqpoint{0.853012in}{1.222670in}}%
\pgfpathcurveto{\pgfqpoint{0.858836in}{1.228494in}}{\pgfqpoint{0.862109in}{1.236394in}}{\pgfqpoint{0.862109in}{1.244631in}}%
\pgfpathcurveto{\pgfqpoint{0.862109in}{1.252867in}}{\pgfqpoint{0.858836in}{1.260767in}}{\pgfqpoint{0.853012in}{1.266591in}}%
\pgfpathcurveto{\pgfqpoint{0.847189in}{1.272415in}}{\pgfqpoint{0.839288in}{1.275687in}}{\pgfqpoint{0.831052in}{1.275687in}}%
\pgfpathcurveto{\pgfqpoint{0.822816in}{1.275687in}}{\pgfqpoint{0.814916in}{1.272415in}}{\pgfqpoint{0.809092in}{1.266591in}}%
\pgfpathcurveto{\pgfqpoint{0.803268in}{1.260767in}}{\pgfqpoint{0.799996in}{1.252867in}}{\pgfqpoint{0.799996in}{1.244631in}}%
\pgfpathcurveto{\pgfqpoint{0.799996in}{1.236394in}}{\pgfqpoint{0.803268in}{1.228494in}}{\pgfqpoint{0.809092in}{1.222670in}}%
\pgfpathcurveto{\pgfqpoint{0.814916in}{1.216846in}}{\pgfqpoint{0.822816in}{1.213574in}}{\pgfqpoint{0.831052in}{1.213574in}}%
\pgfpathclose%
\pgfusepath{stroke,fill}%
\end{pgfscope}%
\begin{pgfscope}%
\pgfpathrectangle{\pgfqpoint{0.100000in}{0.220728in}}{\pgfqpoint{3.696000in}{3.696000in}}%
\pgfusepath{clip}%
\pgfsetbuttcap%
\pgfsetroundjoin%
\definecolor{currentfill}{rgb}{0.121569,0.466667,0.705882}%
\pgfsetfillcolor{currentfill}%
\pgfsetfillopacity{0.616596}%
\pgfsetlinewidth{1.003750pt}%
\definecolor{currentstroke}{rgb}{0.121569,0.466667,0.705882}%
\pgfsetstrokecolor{currentstroke}%
\pgfsetstrokeopacity{0.616596}%
\pgfsetdash{}{0pt}%
\pgfpathmoveto{\pgfqpoint{0.831052in}{1.213574in}}%
\pgfpathcurveto{\pgfqpoint{0.839288in}{1.213574in}}{\pgfqpoint{0.847189in}{1.216846in}}{\pgfqpoint{0.853012in}{1.222670in}}%
\pgfpathcurveto{\pgfqpoint{0.858836in}{1.228494in}}{\pgfqpoint{0.862109in}{1.236394in}}{\pgfqpoint{0.862109in}{1.244631in}}%
\pgfpathcurveto{\pgfqpoint{0.862109in}{1.252867in}}{\pgfqpoint{0.858836in}{1.260767in}}{\pgfqpoint{0.853012in}{1.266591in}}%
\pgfpathcurveto{\pgfqpoint{0.847189in}{1.272415in}}{\pgfqpoint{0.839288in}{1.275687in}}{\pgfqpoint{0.831052in}{1.275687in}}%
\pgfpathcurveto{\pgfqpoint{0.822816in}{1.275687in}}{\pgfqpoint{0.814916in}{1.272415in}}{\pgfqpoint{0.809092in}{1.266591in}}%
\pgfpathcurveto{\pgfqpoint{0.803268in}{1.260767in}}{\pgfqpoint{0.799996in}{1.252867in}}{\pgfqpoint{0.799996in}{1.244631in}}%
\pgfpathcurveto{\pgfqpoint{0.799996in}{1.236394in}}{\pgfqpoint{0.803268in}{1.228494in}}{\pgfqpoint{0.809092in}{1.222670in}}%
\pgfpathcurveto{\pgfqpoint{0.814916in}{1.216846in}}{\pgfqpoint{0.822816in}{1.213574in}}{\pgfqpoint{0.831052in}{1.213574in}}%
\pgfpathclose%
\pgfusepath{stroke,fill}%
\end{pgfscope}%
\begin{pgfscope}%
\pgfpathrectangle{\pgfqpoint{0.100000in}{0.220728in}}{\pgfqpoint{3.696000in}{3.696000in}}%
\pgfusepath{clip}%
\pgfsetbuttcap%
\pgfsetroundjoin%
\definecolor{currentfill}{rgb}{0.121569,0.466667,0.705882}%
\pgfsetfillcolor{currentfill}%
\pgfsetfillopacity{0.616596}%
\pgfsetlinewidth{1.003750pt}%
\definecolor{currentstroke}{rgb}{0.121569,0.466667,0.705882}%
\pgfsetstrokecolor{currentstroke}%
\pgfsetstrokeopacity{0.616596}%
\pgfsetdash{}{0pt}%
\pgfpathmoveto{\pgfqpoint{0.831052in}{1.213574in}}%
\pgfpathcurveto{\pgfqpoint{0.839288in}{1.213574in}}{\pgfqpoint{0.847189in}{1.216846in}}{\pgfqpoint{0.853012in}{1.222670in}}%
\pgfpathcurveto{\pgfqpoint{0.858836in}{1.228494in}}{\pgfqpoint{0.862109in}{1.236394in}}{\pgfqpoint{0.862109in}{1.244631in}}%
\pgfpathcurveto{\pgfqpoint{0.862109in}{1.252867in}}{\pgfqpoint{0.858836in}{1.260767in}}{\pgfqpoint{0.853012in}{1.266591in}}%
\pgfpathcurveto{\pgfqpoint{0.847189in}{1.272415in}}{\pgfqpoint{0.839288in}{1.275687in}}{\pgfqpoint{0.831052in}{1.275687in}}%
\pgfpathcurveto{\pgfqpoint{0.822816in}{1.275687in}}{\pgfqpoint{0.814916in}{1.272415in}}{\pgfqpoint{0.809092in}{1.266591in}}%
\pgfpathcurveto{\pgfqpoint{0.803268in}{1.260767in}}{\pgfqpoint{0.799996in}{1.252867in}}{\pgfqpoint{0.799996in}{1.244631in}}%
\pgfpathcurveto{\pgfqpoint{0.799996in}{1.236394in}}{\pgfqpoint{0.803268in}{1.228494in}}{\pgfqpoint{0.809092in}{1.222670in}}%
\pgfpathcurveto{\pgfqpoint{0.814916in}{1.216846in}}{\pgfqpoint{0.822816in}{1.213574in}}{\pgfqpoint{0.831052in}{1.213574in}}%
\pgfpathclose%
\pgfusepath{stroke,fill}%
\end{pgfscope}%
\begin{pgfscope}%
\pgfpathrectangle{\pgfqpoint{0.100000in}{0.220728in}}{\pgfqpoint{3.696000in}{3.696000in}}%
\pgfusepath{clip}%
\pgfsetbuttcap%
\pgfsetroundjoin%
\definecolor{currentfill}{rgb}{0.121569,0.466667,0.705882}%
\pgfsetfillcolor{currentfill}%
\pgfsetfillopacity{0.616596}%
\pgfsetlinewidth{1.003750pt}%
\definecolor{currentstroke}{rgb}{0.121569,0.466667,0.705882}%
\pgfsetstrokecolor{currentstroke}%
\pgfsetstrokeopacity{0.616596}%
\pgfsetdash{}{0pt}%
\pgfpathmoveto{\pgfqpoint{0.831052in}{1.213574in}}%
\pgfpathcurveto{\pgfqpoint{0.839288in}{1.213574in}}{\pgfqpoint{0.847189in}{1.216846in}}{\pgfqpoint{0.853012in}{1.222670in}}%
\pgfpathcurveto{\pgfqpoint{0.858836in}{1.228494in}}{\pgfqpoint{0.862109in}{1.236394in}}{\pgfqpoint{0.862109in}{1.244631in}}%
\pgfpathcurveto{\pgfqpoint{0.862109in}{1.252867in}}{\pgfqpoint{0.858836in}{1.260767in}}{\pgfqpoint{0.853012in}{1.266591in}}%
\pgfpathcurveto{\pgfqpoint{0.847189in}{1.272415in}}{\pgfqpoint{0.839288in}{1.275687in}}{\pgfqpoint{0.831052in}{1.275687in}}%
\pgfpathcurveto{\pgfqpoint{0.822816in}{1.275687in}}{\pgfqpoint{0.814916in}{1.272415in}}{\pgfqpoint{0.809092in}{1.266591in}}%
\pgfpathcurveto{\pgfqpoint{0.803268in}{1.260767in}}{\pgfqpoint{0.799996in}{1.252867in}}{\pgfqpoint{0.799996in}{1.244631in}}%
\pgfpathcurveto{\pgfqpoint{0.799996in}{1.236394in}}{\pgfqpoint{0.803268in}{1.228494in}}{\pgfqpoint{0.809092in}{1.222670in}}%
\pgfpathcurveto{\pgfqpoint{0.814916in}{1.216846in}}{\pgfqpoint{0.822816in}{1.213574in}}{\pgfqpoint{0.831052in}{1.213574in}}%
\pgfpathclose%
\pgfusepath{stroke,fill}%
\end{pgfscope}%
\begin{pgfscope}%
\pgfpathrectangle{\pgfqpoint{0.100000in}{0.220728in}}{\pgfqpoint{3.696000in}{3.696000in}}%
\pgfusepath{clip}%
\pgfsetbuttcap%
\pgfsetroundjoin%
\definecolor{currentfill}{rgb}{0.121569,0.466667,0.705882}%
\pgfsetfillcolor{currentfill}%
\pgfsetfillopacity{0.616596}%
\pgfsetlinewidth{1.003750pt}%
\definecolor{currentstroke}{rgb}{0.121569,0.466667,0.705882}%
\pgfsetstrokecolor{currentstroke}%
\pgfsetstrokeopacity{0.616596}%
\pgfsetdash{}{0pt}%
\pgfpathmoveto{\pgfqpoint{0.831052in}{1.213574in}}%
\pgfpathcurveto{\pgfqpoint{0.839288in}{1.213574in}}{\pgfqpoint{0.847189in}{1.216846in}}{\pgfqpoint{0.853012in}{1.222670in}}%
\pgfpathcurveto{\pgfqpoint{0.858836in}{1.228494in}}{\pgfqpoint{0.862109in}{1.236394in}}{\pgfqpoint{0.862109in}{1.244631in}}%
\pgfpathcurveto{\pgfqpoint{0.862109in}{1.252867in}}{\pgfqpoint{0.858836in}{1.260767in}}{\pgfqpoint{0.853012in}{1.266591in}}%
\pgfpathcurveto{\pgfqpoint{0.847189in}{1.272415in}}{\pgfqpoint{0.839288in}{1.275687in}}{\pgfqpoint{0.831052in}{1.275687in}}%
\pgfpathcurveto{\pgfqpoint{0.822816in}{1.275687in}}{\pgfqpoint{0.814916in}{1.272415in}}{\pgfqpoint{0.809092in}{1.266591in}}%
\pgfpathcurveto{\pgfqpoint{0.803268in}{1.260767in}}{\pgfqpoint{0.799996in}{1.252867in}}{\pgfqpoint{0.799996in}{1.244631in}}%
\pgfpathcurveto{\pgfqpoint{0.799996in}{1.236394in}}{\pgfqpoint{0.803268in}{1.228494in}}{\pgfqpoint{0.809092in}{1.222670in}}%
\pgfpathcurveto{\pgfqpoint{0.814916in}{1.216846in}}{\pgfqpoint{0.822816in}{1.213574in}}{\pgfqpoint{0.831052in}{1.213574in}}%
\pgfpathclose%
\pgfusepath{stroke,fill}%
\end{pgfscope}%
\begin{pgfscope}%
\pgfpathrectangle{\pgfqpoint{0.100000in}{0.220728in}}{\pgfqpoint{3.696000in}{3.696000in}}%
\pgfusepath{clip}%
\pgfsetbuttcap%
\pgfsetroundjoin%
\definecolor{currentfill}{rgb}{0.121569,0.466667,0.705882}%
\pgfsetfillcolor{currentfill}%
\pgfsetfillopacity{0.616596}%
\pgfsetlinewidth{1.003750pt}%
\definecolor{currentstroke}{rgb}{0.121569,0.466667,0.705882}%
\pgfsetstrokecolor{currentstroke}%
\pgfsetstrokeopacity{0.616596}%
\pgfsetdash{}{0pt}%
\pgfpathmoveto{\pgfqpoint{0.831052in}{1.213574in}}%
\pgfpathcurveto{\pgfqpoint{0.839288in}{1.213574in}}{\pgfqpoint{0.847189in}{1.216846in}}{\pgfqpoint{0.853012in}{1.222670in}}%
\pgfpathcurveto{\pgfqpoint{0.858836in}{1.228494in}}{\pgfqpoint{0.862109in}{1.236394in}}{\pgfqpoint{0.862109in}{1.244631in}}%
\pgfpathcurveto{\pgfqpoint{0.862109in}{1.252867in}}{\pgfqpoint{0.858836in}{1.260767in}}{\pgfqpoint{0.853012in}{1.266591in}}%
\pgfpathcurveto{\pgfqpoint{0.847189in}{1.272415in}}{\pgfqpoint{0.839288in}{1.275687in}}{\pgfqpoint{0.831052in}{1.275687in}}%
\pgfpathcurveto{\pgfqpoint{0.822816in}{1.275687in}}{\pgfqpoint{0.814916in}{1.272415in}}{\pgfqpoint{0.809092in}{1.266591in}}%
\pgfpathcurveto{\pgfqpoint{0.803268in}{1.260767in}}{\pgfqpoint{0.799996in}{1.252867in}}{\pgfqpoint{0.799996in}{1.244631in}}%
\pgfpathcurveto{\pgfqpoint{0.799996in}{1.236394in}}{\pgfqpoint{0.803268in}{1.228494in}}{\pgfqpoint{0.809092in}{1.222670in}}%
\pgfpathcurveto{\pgfqpoint{0.814916in}{1.216846in}}{\pgfqpoint{0.822816in}{1.213574in}}{\pgfqpoint{0.831052in}{1.213574in}}%
\pgfpathclose%
\pgfusepath{stroke,fill}%
\end{pgfscope}%
\begin{pgfscope}%
\pgfpathrectangle{\pgfqpoint{0.100000in}{0.220728in}}{\pgfqpoint{3.696000in}{3.696000in}}%
\pgfusepath{clip}%
\pgfsetbuttcap%
\pgfsetroundjoin%
\definecolor{currentfill}{rgb}{0.121569,0.466667,0.705882}%
\pgfsetfillcolor{currentfill}%
\pgfsetfillopacity{0.616596}%
\pgfsetlinewidth{1.003750pt}%
\definecolor{currentstroke}{rgb}{0.121569,0.466667,0.705882}%
\pgfsetstrokecolor{currentstroke}%
\pgfsetstrokeopacity{0.616596}%
\pgfsetdash{}{0pt}%
\pgfpathmoveto{\pgfqpoint{0.831052in}{1.213574in}}%
\pgfpathcurveto{\pgfqpoint{0.839288in}{1.213574in}}{\pgfqpoint{0.847189in}{1.216846in}}{\pgfqpoint{0.853012in}{1.222670in}}%
\pgfpathcurveto{\pgfqpoint{0.858836in}{1.228494in}}{\pgfqpoint{0.862109in}{1.236394in}}{\pgfqpoint{0.862109in}{1.244631in}}%
\pgfpathcurveto{\pgfqpoint{0.862109in}{1.252867in}}{\pgfqpoint{0.858836in}{1.260767in}}{\pgfqpoint{0.853012in}{1.266591in}}%
\pgfpathcurveto{\pgfqpoint{0.847189in}{1.272415in}}{\pgfqpoint{0.839288in}{1.275687in}}{\pgfqpoint{0.831052in}{1.275687in}}%
\pgfpathcurveto{\pgfqpoint{0.822816in}{1.275687in}}{\pgfqpoint{0.814916in}{1.272415in}}{\pgfqpoint{0.809092in}{1.266591in}}%
\pgfpathcurveto{\pgfqpoint{0.803268in}{1.260767in}}{\pgfqpoint{0.799996in}{1.252867in}}{\pgfqpoint{0.799996in}{1.244631in}}%
\pgfpathcurveto{\pgfqpoint{0.799996in}{1.236394in}}{\pgfqpoint{0.803268in}{1.228494in}}{\pgfqpoint{0.809092in}{1.222670in}}%
\pgfpathcurveto{\pgfqpoint{0.814916in}{1.216846in}}{\pgfqpoint{0.822816in}{1.213574in}}{\pgfqpoint{0.831052in}{1.213574in}}%
\pgfpathclose%
\pgfusepath{stroke,fill}%
\end{pgfscope}%
\begin{pgfscope}%
\pgfpathrectangle{\pgfqpoint{0.100000in}{0.220728in}}{\pgfqpoint{3.696000in}{3.696000in}}%
\pgfusepath{clip}%
\pgfsetbuttcap%
\pgfsetroundjoin%
\definecolor{currentfill}{rgb}{0.121569,0.466667,0.705882}%
\pgfsetfillcolor{currentfill}%
\pgfsetfillopacity{0.616596}%
\pgfsetlinewidth{1.003750pt}%
\definecolor{currentstroke}{rgb}{0.121569,0.466667,0.705882}%
\pgfsetstrokecolor{currentstroke}%
\pgfsetstrokeopacity{0.616596}%
\pgfsetdash{}{0pt}%
\pgfpathmoveto{\pgfqpoint{0.831052in}{1.213574in}}%
\pgfpathcurveto{\pgfqpoint{0.839288in}{1.213574in}}{\pgfqpoint{0.847189in}{1.216846in}}{\pgfqpoint{0.853012in}{1.222670in}}%
\pgfpathcurveto{\pgfqpoint{0.858836in}{1.228494in}}{\pgfqpoint{0.862109in}{1.236394in}}{\pgfqpoint{0.862109in}{1.244631in}}%
\pgfpathcurveto{\pgfqpoint{0.862109in}{1.252867in}}{\pgfqpoint{0.858836in}{1.260767in}}{\pgfqpoint{0.853012in}{1.266591in}}%
\pgfpathcurveto{\pgfqpoint{0.847189in}{1.272415in}}{\pgfqpoint{0.839288in}{1.275687in}}{\pgfqpoint{0.831052in}{1.275687in}}%
\pgfpathcurveto{\pgfqpoint{0.822816in}{1.275687in}}{\pgfqpoint{0.814916in}{1.272415in}}{\pgfqpoint{0.809092in}{1.266591in}}%
\pgfpathcurveto{\pgfqpoint{0.803268in}{1.260767in}}{\pgfqpoint{0.799996in}{1.252867in}}{\pgfqpoint{0.799996in}{1.244631in}}%
\pgfpathcurveto{\pgfqpoint{0.799996in}{1.236394in}}{\pgfqpoint{0.803268in}{1.228494in}}{\pgfqpoint{0.809092in}{1.222670in}}%
\pgfpathcurveto{\pgfqpoint{0.814916in}{1.216846in}}{\pgfqpoint{0.822816in}{1.213574in}}{\pgfqpoint{0.831052in}{1.213574in}}%
\pgfpathclose%
\pgfusepath{stroke,fill}%
\end{pgfscope}%
\begin{pgfscope}%
\pgfpathrectangle{\pgfqpoint{0.100000in}{0.220728in}}{\pgfqpoint{3.696000in}{3.696000in}}%
\pgfusepath{clip}%
\pgfsetbuttcap%
\pgfsetroundjoin%
\definecolor{currentfill}{rgb}{0.121569,0.466667,0.705882}%
\pgfsetfillcolor{currentfill}%
\pgfsetfillopacity{0.616596}%
\pgfsetlinewidth{1.003750pt}%
\definecolor{currentstroke}{rgb}{0.121569,0.466667,0.705882}%
\pgfsetstrokecolor{currentstroke}%
\pgfsetstrokeopacity{0.616596}%
\pgfsetdash{}{0pt}%
\pgfpathmoveto{\pgfqpoint{0.831052in}{1.213574in}}%
\pgfpathcurveto{\pgfqpoint{0.839288in}{1.213574in}}{\pgfqpoint{0.847189in}{1.216846in}}{\pgfqpoint{0.853012in}{1.222670in}}%
\pgfpathcurveto{\pgfqpoint{0.858836in}{1.228494in}}{\pgfqpoint{0.862109in}{1.236394in}}{\pgfqpoint{0.862109in}{1.244631in}}%
\pgfpathcurveto{\pgfqpoint{0.862109in}{1.252867in}}{\pgfqpoint{0.858836in}{1.260767in}}{\pgfqpoint{0.853012in}{1.266591in}}%
\pgfpathcurveto{\pgfqpoint{0.847189in}{1.272415in}}{\pgfqpoint{0.839288in}{1.275687in}}{\pgfqpoint{0.831052in}{1.275687in}}%
\pgfpathcurveto{\pgfqpoint{0.822816in}{1.275687in}}{\pgfqpoint{0.814916in}{1.272415in}}{\pgfqpoint{0.809092in}{1.266591in}}%
\pgfpathcurveto{\pgfqpoint{0.803268in}{1.260767in}}{\pgfqpoint{0.799996in}{1.252867in}}{\pgfqpoint{0.799996in}{1.244631in}}%
\pgfpathcurveto{\pgfqpoint{0.799996in}{1.236394in}}{\pgfqpoint{0.803268in}{1.228494in}}{\pgfqpoint{0.809092in}{1.222670in}}%
\pgfpathcurveto{\pgfqpoint{0.814916in}{1.216846in}}{\pgfqpoint{0.822816in}{1.213574in}}{\pgfqpoint{0.831052in}{1.213574in}}%
\pgfpathclose%
\pgfusepath{stroke,fill}%
\end{pgfscope}%
\begin{pgfscope}%
\pgfpathrectangle{\pgfqpoint{0.100000in}{0.220728in}}{\pgfqpoint{3.696000in}{3.696000in}}%
\pgfusepath{clip}%
\pgfsetbuttcap%
\pgfsetroundjoin%
\definecolor{currentfill}{rgb}{0.121569,0.466667,0.705882}%
\pgfsetfillcolor{currentfill}%
\pgfsetfillopacity{0.616596}%
\pgfsetlinewidth{1.003750pt}%
\definecolor{currentstroke}{rgb}{0.121569,0.466667,0.705882}%
\pgfsetstrokecolor{currentstroke}%
\pgfsetstrokeopacity{0.616596}%
\pgfsetdash{}{0pt}%
\pgfpathmoveto{\pgfqpoint{0.831052in}{1.213574in}}%
\pgfpathcurveto{\pgfqpoint{0.839288in}{1.213574in}}{\pgfqpoint{0.847189in}{1.216846in}}{\pgfqpoint{0.853012in}{1.222670in}}%
\pgfpathcurveto{\pgfqpoint{0.858836in}{1.228494in}}{\pgfqpoint{0.862109in}{1.236394in}}{\pgfqpoint{0.862109in}{1.244631in}}%
\pgfpathcurveto{\pgfqpoint{0.862109in}{1.252867in}}{\pgfqpoint{0.858836in}{1.260767in}}{\pgfqpoint{0.853012in}{1.266591in}}%
\pgfpathcurveto{\pgfqpoint{0.847189in}{1.272415in}}{\pgfqpoint{0.839288in}{1.275687in}}{\pgfqpoint{0.831052in}{1.275687in}}%
\pgfpathcurveto{\pgfqpoint{0.822816in}{1.275687in}}{\pgfqpoint{0.814916in}{1.272415in}}{\pgfqpoint{0.809092in}{1.266591in}}%
\pgfpathcurveto{\pgfqpoint{0.803268in}{1.260767in}}{\pgfqpoint{0.799996in}{1.252867in}}{\pgfqpoint{0.799996in}{1.244631in}}%
\pgfpathcurveto{\pgfqpoint{0.799996in}{1.236394in}}{\pgfqpoint{0.803268in}{1.228494in}}{\pgfqpoint{0.809092in}{1.222670in}}%
\pgfpathcurveto{\pgfqpoint{0.814916in}{1.216846in}}{\pgfqpoint{0.822816in}{1.213574in}}{\pgfqpoint{0.831052in}{1.213574in}}%
\pgfpathclose%
\pgfusepath{stroke,fill}%
\end{pgfscope}%
\begin{pgfscope}%
\pgfpathrectangle{\pgfqpoint{0.100000in}{0.220728in}}{\pgfqpoint{3.696000in}{3.696000in}}%
\pgfusepath{clip}%
\pgfsetbuttcap%
\pgfsetroundjoin%
\definecolor{currentfill}{rgb}{0.121569,0.466667,0.705882}%
\pgfsetfillcolor{currentfill}%
\pgfsetfillopacity{0.616596}%
\pgfsetlinewidth{1.003750pt}%
\definecolor{currentstroke}{rgb}{0.121569,0.466667,0.705882}%
\pgfsetstrokecolor{currentstroke}%
\pgfsetstrokeopacity{0.616596}%
\pgfsetdash{}{0pt}%
\pgfpathmoveto{\pgfqpoint{0.831052in}{1.213574in}}%
\pgfpathcurveto{\pgfqpoint{0.839288in}{1.213574in}}{\pgfqpoint{0.847189in}{1.216846in}}{\pgfqpoint{0.853012in}{1.222670in}}%
\pgfpathcurveto{\pgfqpoint{0.858836in}{1.228494in}}{\pgfqpoint{0.862109in}{1.236394in}}{\pgfqpoint{0.862109in}{1.244631in}}%
\pgfpathcurveto{\pgfqpoint{0.862109in}{1.252867in}}{\pgfqpoint{0.858836in}{1.260767in}}{\pgfqpoint{0.853012in}{1.266591in}}%
\pgfpathcurveto{\pgfqpoint{0.847189in}{1.272415in}}{\pgfqpoint{0.839288in}{1.275687in}}{\pgfqpoint{0.831052in}{1.275687in}}%
\pgfpathcurveto{\pgfqpoint{0.822816in}{1.275687in}}{\pgfqpoint{0.814916in}{1.272415in}}{\pgfqpoint{0.809092in}{1.266591in}}%
\pgfpathcurveto{\pgfqpoint{0.803268in}{1.260767in}}{\pgfqpoint{0.799996in}{1.252867in}}{\pgfqpoint{0.799996in}{1.244631in}}%
\pgfpathcurveto{\pgfqpoint{0.799996in}{1.236394in}}{\pgfqpoint{0.803268in}{1.228494in}}{\pgfqpoint{0.809092in}{1.222670in}}%
\pgfpathcurveto{\pgfqpoint{0.814916in}{1.216846in}}{\pgfqpoint{0.822816in}{1.213574in}}{\pgfqpoint{0.831052in}{1.213574in}}%
\pgfpathclose%
\pgfusepath{stroke,fill}%
\end{pgfscope}%
\begin{pgfscope}%
\pgfpathrectangle{\pgfqpoint{0.100000in}{0.220728in}}{\pgfqpoint{3.696000in}{3.696000in}}%
\pgfusepath{clip}%
\pgfsetbuttcap%
\pgfsetroundjoin%
\definecolor{currentfill}{rgb}{0.121569,0.466667,0.705882}%
\pgfsetfillcolor{currentfill}%
\pgfsetfillopacity{0.616596}%
\pgfsetlinewidth{1.003750pt}%
\definecolor{currentstroke}{rgb}{0.121569,0.466667,0.705882}%
\pgfsetstrokecolor{currentstroke}%
\pgfsetstrokeopacity{0.616596}%
\pgfsetdash{}{0pt}%
\pgfpathmoveto{\pgfqpoint{0.831052in}{1.213574in}}%
\pgfpathcurveto{\pgfqpoint{0.839288in}{1.213574in}}{\pgfqpoint{0.847189in}{1.216846in}}{\pgfqpoint{0.853012in}{1.222670in}}%
\pgfpathcurveto{\pgfqpoint{0.858836in}{1.228494in}}{\pgfqpoint{0.862109in}{1.236394in}}{\pgfqpoint{0.862109in}{1.244631in}}%
\pgfpathcurveto{\pgfqpoint{0.862109in}{1.252867in}}{\pgfqpoint{0.858836in}{1.260767in}}{\pgfqpoint{0.853012in}{1.266591in}}%
\pgfpathcurveto{\pgfqpoint{0.847189in}{1.272415in}}{\pgfqpoint{0.839288in}{1.275687in}}{\pgfqpoint{0.831052in}{1.275687in}}%
\pgfpathcurveto{\pgfqpoint{0.822816in}{1.275687in}}{\pgfqpoint{0.814916in}{1.272415in}}{\pgfqpoint{0.809092in}{1.266591in}}%
\pgfpathcurveto{\pgfqpoint{0.803268in}{1.260767in}}{\pgfqpoint{0.799996in}{1.252867in}}{\pgfqpoint{0.799996in}{1.244631in}}%
\pgfpathcurveto{\pgfqpoint{0.799996in}{1.236394in}}{\pgfqpoint{0.803268in}{1.228494in}}{\pgfqpoint{0.809092in}{1.222670in}}%
\pgfpathcurveto{\pgfqpoint{0.814916in}{1.216846in}}{\pgfqpoint{0.822816in}{1.213574in}}{\pgfqpoint{0.831052in}{1.213574in}}%
\pgfpathclose%
\pgfusepath{stroke,fill}%
\end{pgfscope}%
\begin{pgfscope}%
\pgfpathrectangle{\pgfqpoint{0.100000in}{0.220728in}}{\pgfqpoint{3.696000in}{3.696000in}}%
\pgfusepath{clip}%
\pgfsetbuttcap%
\pgfsetroundjoin%
\definecolor{currentfill}{rgb}{0.121569,0.466667,0.705882}%
\pgfsetfillcolor{currentfill}%
\pgfsetfillopacity{0.616596}%
\pgfsetlinewidth{1.003750pt}%
\definecolor{currentstroke}{rgb}{0.121569,0.466667,0.705882}%
\pgfsetstrokecolor{currentstroke}%
\pgfsetstrokeopacity{0.616596}%
\pgfsetdash{}{0pt}%
\pgfpathmoveto{\pgfqpoint{0.831052in}{1.213574in}}%
\pgfpathcurveto{\pgfqpoint{0.839288in}{1.213574in}}{\pgfqpoint{0.847189in}{1.216846in}}{\pgfqpoint{0.853012in}{1.222670in}}%
\pgfpathcurveto{\pgfqpoint{0.858836in}{1.228494in}}{\pgfqpoint{0.862109in}{1.236394in}}{\pgfqpoint{0.862109in}{1.244631in}}%
\pgfpathcurveto{\pgfqpoint{0.862109in}{1.252867in}}{\pgfqpoint{0.858836in}{1.260767in}}{\pgfqpoint{0.853012in}{1.266591in}}%
\pgfpathcurveto{\pgfqpoint{0.847189in}{1.272415in}}{\pgfqpoint{0.839288in}{1.275687in}}{\pgfqpoint{0.831052in}{1.275687in}}%
\pgfpathcurveto{\pgfqpoint{0.822816in}{1.275687in}}{\pgfqpoint{0.814916in}{1.272415in}}{\pgfqpoint{0.809092in}{1.266591in}}%
\pgfpathcurveto{\pgfqpoint{0.803268in}{1.260767in}}{\pgfqpoint{0.799996in}{1.252867in}}{\pgfqpoint{0.799996in}{1.244631in}}%
\pgfpathcurveto{\pgfqpoint{0.799996in}{1.236394in}}{\pgfqpoint{0.803268in}{1.228494in}}{\pgfqpoint{0.809092in}{1.222670in}}%
\pgfpathcurveto{\pgfqpoint{0.814916in}{1.216846in}}{\pgfqpoint{0.822816in}{1.213574in}}{\pgfqpoint{0.831052in}{1.213574in}}%
\pgfpathclose%
\pgfusepath{stroke,fill}%
\end{pgfscope}%
\begin{pgfscope}%
\pgfpathrectangle{\pgfqpoint{0.100000in}{0.220728in}}{\pgfqpoint{3.696000in}{3.696000in}}%
\pgfusepath{clip}%
\pgfsetbuttcap%
\pgfsetroundjoin%
\definecolor{currentfill}{rgb}{0.121569,0.466667,0.705882}%
\pgfsetfillcolor{currentfill}%
\pgfsetfillopacity{0.616596}%
\pgfsetlinewidth{1.003750pt}%
\definecolor{currentstroke}{rgb}{0.121569,0.466667,0.705882}%
\pgfsetstrokecolor{currentstroke}%
\pgfsetstrokeopacity{0.616596}%
\pgfsetdash{}{0pt}%
\pgfpathmoveto{\pgfqpoint{0.831052in}{1.213574in}}%
\pgfpathcurveto{\pgfqpoint{0.839288in}{1.213574in}}{\pgfqpoint{0.847189in}{1.216846in}}{\pgfqpoint{0.853012in}{1.222670in}}%
\pgfpathcurveto{\pgfqpoint{0.858836in}{1.228494in}}{\pgfqpoint{0.862109in}{1.236394in}}{\pgfqpoint{0.862109in}{1.244631in}}%
\pgfpathcurveto{\pgfqpoint{0.862109in}{1.252867in}}{\pgfqpoint{0.858836in}{1.260767in}}{\pgfqpoint{0.853012in}{1.266591in}}%
\pgfpathcurveto{\pgfqpoint{0.847189in}{1.272415in}}{\pgfqpoint{0.839288in}{1.275687in}}{\pgfqpoint{0.831052in}{1.275687in}}%
\pgfpathcurveto{\pgfqpoint{0.822816in}{1.275687in}}{\pgfqpoint{0.814916in}{1.272415in}}{\pgfqpoint{0.809092in}{1.266591in}}%
\pgfpathcurveto{\pgfqpoint{0.803268in}{1.260767in}}{\pgfqpoint{0.799996in}{1.252867in}}{\pgfqpoint{0.799996in}{1.244631in}}%
\pgfpathcurveto{\pgfqpoint{0.799996in}{1.236394in}}{\pgfqpoint{0.803268in}{1.228494in}}{\pgfqpoint{0.809092in}{1.222670in}}%
\pgfpathcurveto{\pgfqpoint{0.814916in}{1.216846in}}{\pgfqpoint{0.822816in}{1.213574in}}{\pgfqpoint{0.831052in}{1.213574in}}%
\pgfpathclose%
\pgfusepath{stroke,fill}%
\end{pgfscope}%
\begin{pgfscope}%
\pgfpathrectangle{\pgfqpoint{0.100000in}{0.220728in}}{\pgfqpoint{3.696000in}{3.696000in}}%
\pgfusepath{clip}%
\pgfsetbuttcap%
\pgfsetroundjoin%
\definecolor{currentfill}{rgb}{0.121569,0.466667,0.705882}%
\pgfsetfillcolor{currentfill}%
\pgfsetfillopacity{0.616596}%
\pgfsetlinewidth{1.003750pt}%
\definecolor{currentstroke}{rgb}{0.121569,0.466667,0.705882}%
\pgfsetstrokecolor{currentstroke}%
\pgfsetstrokeopacity{0.616596}%
\pgfsetdash{}{0pt}%
\pgfpathmoveto{\pgfqpoint{0.831052in}{1.213574in}}%
\pgfpathcurveto{\pgfqpoint{0.839288in}{1.213574in}}{\pgfqpoint{0.847189in}{1.216846in}}{\pgfqpoint{0.853012in}{1.222670in}}%
\pgfpathcurveto{\pgfqpoint{0.858836in}{1.228494in}}{\pgfqpoint{0.862109in}{1.236394in}}{\pgfqpoint{0.862109in}{1.244631in}}%
\pgfpathcurveto{\pgfqpoint{0.862109in}{1.252867in}}{\pgfqpoint{0.858836in}{1.260767in}}{\pgfqpoint{0.853012in}{1.266591in}}%
\pgfpathcurveto{\pgfqpoint{0.847189in}{1.272415in}}{\pgfqpoint{0.839288in}{1.275687in}}{\pgfqpoint{0.831052in}{1.275687in}}%
\pgfpathcurveto{\pgfqpoint{0.822816in}{1.275687in}}{\pgfqpoint{0.814916in}{1.272415in}}{\pgfqpoint{0.809092in}{1.266591in}}%
\pgfpathcurveto{\pgfqpoint{0.803268in}{1.260767in}}{\pgfqpoint{0.799996in}{1.252867in}}{\pgfqpoint{0.799996in}{1.244631in}}%
\pgfpathcurveto{\pgfqpoint{0.799996in}{1.236394in}}{\pgfqpoint{0.803268in}{1.228494in}}{\pgfqpoint{0.809092in}{1.222670in}}%
\pgfpathcurveto{\pgfqpoint{0.814916in}{1.216846in}}{\pgfqpoint{0.822816in}{1.213574in}}{\pgfqpoint{0.831052in}{1.213574in}}%
\pgfpathclose%
\pgfusepath{stroke,fill}%
\end{pgfscope}%
\begin{pgfscope}%
\pgfpathrectangle{\pgfqpoint{0.100000in}{0.220728in}}{\pgfqpoint{3.696000in}{3.696000in}}%
\pgfusepath{clip}%
\pgfsetbuttcap%
\pgfsetroundjoin%
\definecolor{currentfill}{rgb}{0.121569,0.466667,0.705882}%
\pgfsetfillcolor{currentfill}%
\pgfsetfillopacity{0.616596}%
\pgfsetlinewidth{1.003750pt}%
\definecolor{currentstroke}{rgb}{0.121569,0.466667,0.705882}%
\pgfsetstrokecolor{currentstroke}%
\pgfsetstrokeopacity{0.616596}%
\pgfsetdash{}{0pt}%
\pgfpathmoveto{\pgfqpoint{0.831052in}{1.213574in}}%
\pgfpathcurveto{\pgfqpoint{0.839288in}{1.213574in}}{\pgfqpoint{0.847189in}{1.216846in}}{\pgfqpoint{0.853012in}{1.222670in}}%
\pgfpathcurveto{\pgfqpoint{0.858836in}{1.228494in}}{\pgfqpoint{0.862109in}{1.236394in}}{\pgfqpoint{0.862109in}{1.244631in}}%
\pgfpathcurveto{\pgfqpoint{0.862109in}{1.252867in}}{\pgfqpoint{0.858836in}{1.260767in}}{\pgfqpoint{0.853012in}{1.266591in}}%
\pgfpathcurveto{\pgfqpoint{0.847189in}{1.272415in}}{\pgfqpoint{0.839288in}{1.275687in}}{\pgfqpoint{0.831052in}{1.275687in}}%
\pgfpathcurveto{\pgfqpoint{0.822816in}{1.275687in}}{\pgfqpoint{0.814916in}{1.272415in}}{\pgfqpoint{0.809092in}{1.266591in}}%
\pgfpathcurveto{\pgfqpoint{0.803268in}{1.260767in}}{\pgfqpoint{0.799996in}{1.252867in}}{\pgfqpoint{0.799996in}{1.244631in}}%
\pgfpathcurveto{\pgfqpoint{0.799996in}{1.236394in}}{\pgfqpoint{0.803268in}{1.228494in}}{\pgfqpoint{0.809092in}{1.222670in}}%
\pgfpathcurveto{\pgfqpoint{0.814916in}{1.216846in}}{\pgfqpoint{0.822816in}{1.213574in}}{\pgfqpoint{0.831052in}{1.213574in}}%
\pgfpathclose%
\pgfusepath{stroke,fill}%
\end{pgfscope}%
\begin{pgfscope}%
\pgfpathrectangle{\pgfqpoint{0.100000in}{0.220728in}}{\pgfqpoint{3.696000in}{3.696000in}}%
\pgfusepath{clip}%
\pgfsetbuttcap%
\pgfsetroundjoin%
\definecolor{currentfill}{rgb}{0.121569,0.466667,0.705882}%
\pgfsetfillcolor{currentfill}%
\pgfsetfillopacity{0.616596}%
\pgfsetlinewidth{1.003750pt}%
\definecolor{currentstroke}{rgb}{0.121569,0.466667,0.705882}%
\pgfsetstrokecolor{currentstroke}%
\pgfsetstrokeopacity{0.616596}%
\pgfsetdash{}{0pt}%
\pgfpathmoveto{\pgfqpoint{0.831052in}{1.213574in}}%
\pgfpathcurveto{\pgfqpoint{0.839288in}{1.213574in}}{\pgfqpoint{0.847189in}{1.216846in}}{\pgfqpoint{0.853012in}{1.222670in}}%
\pgfpathcurveto{\pgfqpoint{0.858836in}{1.228494in}}{\pgfqpoint{0.862109in}{1.236394in}}{\pgfqpoint{0.862109in}{1.244631in}}%
\pgfpathcurveto{\pgfqpoint{0.862109in}{1.252867in}}{\pgfqpoint{0.858836in}{1.260767in}}{\pgfqpoint{0.853012in}{1.266591in}}%
\pgfpathcurveto{\pgfqpoint{0.847189in}{1.272415in}}{\pgfqpoint{0.839288in}{1.275687in}}{\pgfqpoint{0.831052in}{1.275687in}}%
\pgfpathcurveto{\pgfqpoint{0.822816in}{1.275687in}}{\pgfqpoint{0.814916in}{1.272415in}}{\pgfqpoint{0.809092in}{1.266591in}}%
\pgfpathcurveto{\pgfqpoint{0.803268in}{1.260767in}}{\pgfqpoint{0.799996in}{1.252867in}}{\pgfqpoint{0.799996in}{1.244631in}}%
\pgfpathcurveto{\pgfqpoint{0.799996in}{1.236394in}}{\pgfqpoint{0.803268in}{1.228494in}}{\pgfqpoint{0.809092in}{1.222670in}}%
\pgfpathcurveto{\pgfqpoint{0.814916in}{1.216846in}}{\pgfqpoint{0.822816in}{1.213574in}}{\pgfqpoint{0.831052in}{1.213574in}}%
\pgfpathclose%
\pgfusepath{stroke,fill}%
\end{pgfscope}%
\begin{pgfscope}%
\pgfpathrectangle{\pgfqpoint{0.100000in}{0.220728in}}{\pgfqpoint{3.696000in}{3.696000in}}%
\pgfusepath{clip}%
\pgfsetbuttcap%
\pgfsetroundjoin%
\definecolor{currentfill}{rgb}{0.121569,0.466667,0.705882}%
\pgfsetfillcolor{currentfill}%
\pgfsetfillopacity{0.616596}%
\pgfsetlinewidth{1.003750pt}%
\definecolor{currentstroke}{rgb}{0.121569,0.466667,0.705882}%
\pgfsetstrokecolor{currentstroke}%
\pgfsetstrokeopacity{0.616596}%
\pgfsetdash{}{0pt}%
\pgfpathmoveto{\pgfqpoint{0.831052in}{1.213574in}}%
\pgfpathcurveto{\pgfqpoint{0.839288in}{1.213574in}}{\pgfqpoint{0.847189in}{1.216846in}}{\pgfqpoint{0.853012in}{1.222670in}}%
\pgfpathcurveto{\pgfqpoint{0.858836in}{1.228494in}}{\pgfqpoint{0.862109in}{1.236394in}}{\pgfqpoint{0.862109in}{1.244631in}}%
\pgfpathcurveto{\pgfqpoint{0.862109in}{1.252867in}}{\pgfqpoint{0.858836in}{1.260767in}}{\pgfqpoint{0.853012in}{1.266591in}}%
\pgfpathcurveto{\pgfqpoint{0.847189in}{1.272415in}}{\pgfqpoint{0.839288in}{1.275687in}}{\pgfqpoint{0.831052in}{1.275687in}}%
\pgfpathcurveto{\pgfqpoint{0.822816in}{1.275687in}}{\pgfqpoint{0.814916in}{1.272415in}}{\pgfqpoint{0.809092in}{1.266591in}}%
\pgfpathcurveto{\pgfqpoint{0.803268in}{1.260767in}}{\pgfqpoint{0.799996in}{1.252867in}}{\pgfqpoint{0.799996in}{1.244631in}}%
\pgfpathcurveto{\pgfqpoint{0.799996in}{1.236394in}}{\pgfqpoint{0.803268in}{1.228494in}}{\pgfqpoint{0.809092in}{1.222670in}}%
\pgfpathcurveto{\pgfqpoint{0.814916in}{1.216846in}}{\pgfqpoint{0.822816in}{1.213574in}}{\pgfqpoint{0.831052in}{1.213574in}}%
\pgfpathclose%
\pgfusepath{stroke,fill}%
\end{pgfscope}%
\begin{pgfscope}%
\pgfpathrectangle{\pgfqpoint{0.100000in}{0.220728in}}{\pgfqpoint{3.696000in}{3.696000in}}%
\pgfusepath{clip}%
\pgfsetbuttcap%
\pgfsetroundjoin%
\definecolor{currentfill}{rgb}{0.121569,0.466667,0.705882}%
\pgfsetfillcolor{currentfill}%
\pgfsetfillopacity{0.616596}%
\pgfsetlinewidth{1.003750pt}%
\definecolor{currentstroke}{rgb}{0.121569,0.466667,0.705882}%
\pgfsetstrokecolor{currentstroke}%
\pgfsetstrokeopacity{0.616596}%
\pgfsetdash{}{0pt}%
\pgfpathmoveto{\pgfqpoint{0.831052in}{1.213574in}}%
\pgfpathcurveto{\pgfqpoint{0.839288in}{1.213574in}}{\pgfqpoint{0.847189in}{1.216846in}}{\pgfqpoint{0.853012in}{1.222670in}}%
\pgfpathcurveto{\pgfqpoint{0.858836in}{1.228494in}}{\pgfqpoint{0.862109in}{1.236394in}}{\pgfqpoint{0.862109in}{1.244631in}}%
\pgfpathcurveto{\pgfqpoint{0.862109in}{1.252867in}}{\pgfqpoint{0.858836in}{1.260767in}}{\pgfqpoint{0.853012in}{1.266591in}}%
\pgfpathcurveto{\pgfqpoint{0.847189in}{1.272415in}}{\pgfqpoint{0.839288in}{1.275687in}}{\pgfqpoint{0.831052in}{1.275687in}}%
\pgfpathcurveto{\pgfqpoint{0.822816in}{1.275687in}}{\pgfqpoint{0.814916in}{1.272415in}}{\pgfqpoint{0.809092in}{1.266591in}}%
\pgfpathcurveto{\pgfqpoint{0.803268in}{1.260767in}}{\pgfqpoint{0.799996in}{1.252867in}}{\pgfqpoint{0.799996in}{1.244631in}}%
\pgfpathcurveto{\pgfqpoint{0.799996in}{1.236394in}}{\pgfqpoint{0.803268in}{1.228494in}}{\pgfqpoint{0.809092in}{1.222670in}}%
\pgfpathcurveto{\pgfqpoint{0.814916in}{1.216846in}}{\pgfqpoint{0.822816in}{1.213574in}}{\pgfqpoint{0.831052in}{1.213574in}}%
\pgfpathclose%
\pgfusepath{stroke,fill}%
\end{pgfscope}%
\begin{pgfscope}%
\pgfpathrectangle{\pgfqpoint{0.100000in}{0.220728in}}{\pgfqpoint{3.696000in}{3.696000in}}%
\pgfusepath{clip}%
\pgfsetbuttcap%
\pgfsetroundjoin%
\definecolor{currentfill}{rgb}{0.121569,0.466667,0.705882}%
\pgfsetfillcolor{currentfill}%
\pgfsetfillopacity{0.616596}%
\pgfsetlinewidth{1.003750pt}%
\definecolor{currentstroke}{rgb}{0.121569,0.466667,0.705882}%
\pgfsetstrokecolor{currentstroke}%
\pgfsetstrokeopacity{0.616596}%
\pgfsetdash{}{0pt}%
\pgfpathmoveto{\pgfqpoint{0.831052in}{1.213574in}}%
\pgfpathcurveto{\pgfqpoint{0.839288in}{1.213574in}}{\pgfqpoint{0.847189in}{1.216846in}}{\pgfqpoint{0.853012in}{1.222670in}}%
\pgfpathcurveto{\pgfqpoint{0.858836in}{1.228494in}}{\pgfqpoint{0.862109in}{1.236394in}}{\pgfqpoint{0.862109in}{1.244631in}}%
\pgfpathcurveto{\pgfqpoint{0.862109in}{1.252867in}}{\pgfqpoint{0.858836in}{1.260767in}}{\pgfqpoint{0.853012in}{1.266591in}}%
\pgfpathcurveto{\pgfqpoint{0.847189in}{1.272415in}}{\pgfqpoint{0.839288in}{1.275687in}}{\pgfqpoint{0.831052in}{1.275687in}}%
\pgfpathcurveto{\pgfqpoint{0.822816in}{1.275687in}}{\pgfqpoint{0.814916in}{1.272415in}}{\pgfqpoint{0.809092in}{1.266591in}}%
\pgfpathcurveto{\pgfqpoint{0.803268in}{1.260767in}}{\pgfqpoint{0.799996in}{1.252867in}}{\pgfqpoint{0.799996in}{1.244631in}}%
\pgfpathcurveto{\pgfqpoint{0.799996in}{1.236394in}}{\pgfqpoint{0.803268in}{1.228494in}}{\pgfqpoint{0.809092in}{1.222670in}}%
\pgfpathcurveto{\pgfqpoint{0.814916in}{1.216846in}}{\pgfqpoint{0.822816in}{1.213574in}}{\pgfqpoint{0.831052in}{1.213574in}}%
\pgfpathclose%
\pgfusepath{stroke,fill}%
\end{pgfscope}%
\begin{pgfscope}%
\pgfpathrectangle{\pgfqpoint{0.100000in}{0.220728in}}{\pgfqpoint{3.696000in}{3.696000in}}%
\pgfusepath{clip}%
\pgfsetbuttcap%
\pgfsetroundjoin%
\definecolor{currentfill}{rgb}{0.121569,0.466667,0.705882}%
\pgfsetfillcolor{currentfill}%
\pgfsetfillopacity{0.616596}%
\pgfsetlinewidth{1.003750pt}%
\definecolor{currentstroke}{rgb}{0.121569,0.466667,0.705882}%
\pgfsetstrokecolor{currentstroke}%
\pgfsetstrokeopacity{0.616596}%
\pgfsetdash{}{0pt}%
\pgfpathmoveto{\pgfqpoint{0.831052in}{1.213574in}}%
\pgfpathcurveto{\pgfqpoint{0.839288in}{1.213574in}}{\pgfqpoint{0.847189in}{1.216846in}}{\pgfqpoint{0.853012in}{1.222670in}}%
\pgfpathcurveto{\pgfqpoint{0.858836in}{1.228494in}}{\pgfqpoint{0.862109in}{1.236394in}}{\pgfqpoint{0.862109in}{1.244631in}}%
\pgfpathcurveto{\pgfqpoint{0.862109in}{1.252867in}}{\pgfqpoint{0.858836in}{1.260767in}}{\pgfqpoint{0.853012in}{1.266591in}}%
\pgfpathcurveto{\pgfqpoint{0.847189in}{1.272415in}}{\pgfqpoint{0.839288in}{1.275687in}}{\pgfqpoint{0.831052in}{1.275687in}}%
\pgfpathcurveto{\pgfqpoint{0.822816in}{1.275687in}}{\pgfqpoint{0.814916in}{1.272415in}}{\pgfqpoint{0.809092in}{1.266591in}}%
\pgfpathcurveto{\pgfqpoint{0.803268in}{1.260767in}}{\pgfqpoint{0.799996in}{1.252867in}}{\pgfqpoint{0.799996in}{1.244631in}}%
\pgfpathcurveto{\pgfqpoint{0.799996in}{1.236394in}}{\pgfqpoint{0.803268in}{1.228494in}}{\pgfqpoint{0.809092in}{1.222670in}}%
\pgfpathcurveto{\pgfqpoint{0.814916in}{1.216846in}}{\pgfqpoint{0.822816in}{1.213574in}}{\pgfqpoint{0.831052in}{1.213574in}}%
\pgfpathclose%
\pgfusepath{stroke,fill}%
\end{pgfscope}%
\begin{pgfscope}%
\pgfpathrectangle{\pgfqpoint{0.100000in}{0.220728in}}{\pgfqpoint{3.696000in}{3.696000in}}%
\pgfusepath{clip}%
\pgfsetbuttcap%
\pgfsetroundjoin%
\definecolor{currentfill}{rgb}{0.121569,0.466667,0.705882}%
\pgfsetfillcolor{currentfill}%
\pgfsetfillopacity{0.616596}%
\pgfsetlinewidth{1.003750pt}%
\definecolor{currentstroke}{rgb}{0.121569,0.466667,0.705882}%
\pgfsetstrokecolor{currentstroke}%
\pgfsetstrokeopacity{0.616596}%
\pgfsetdash{}{0pt}%
\pgfpathmoveto{\pgfqpoint{0.831052in}{1.213574in}}%
\pgfpathcurveto{\pgfqpoint{0.839288in}{1.213574in}}{\pgfqpoint{0.847189in}{1.216846in}}{\pgfqpoint{0.853012in}{1.222670in}}%
\pgfpathcurveto{\pgfqpoint{0.858836in}{1.228494in}}{\pgfqpoint{0.862109in}{1.236394in}}{\pgfqpoint{0.862109in}{1.244631in}}%
\pgfpathcurveto{\pgfqpoint{0.862109in}{1.252867in}}{\pgfqpoint{0.858836in}{1.260767in}}{\pgfqpoint{0.853012in}{1.266591in}}%
\pgfpathcurveto{\pgfqpoint{0.847189in}{1.272415in}}{\pgfqpoint{0.839288in}{1.275687in}}{\pgfqpoint{0.831052in}{1.275687in}}%
\pgfpathcurveto{\pgfqpoint{0.822816in}{1.275687in}}{\pgfqpoint{0.814916in}{1.272415in}}{\pgfqpoint{0.809092in}{1.266591in}}%
\pgfpathcurveto{\pgfqpoint{0.803268in}{1.260767in}}{\pgfqpoint{0.799996in}{1.252867in}}{\pgfqpoint{0.799996in}{1.244631in}}%
\pgfpathcurveto{\pgfqpoint{0.799996in}{1.236394in}}{\pgfqpoint{0.803268in}{1.228494in}}{\pgfqpoint{0.809092in}{1.222670in}}%
\pgfpathcurveto{\pgfqpoint{0.814916in}{1.216846in}}{\pgfqpoint{0.822816in}{1.213574in}}{\pgfqpoint{0.831052in}{1.213574in}}%
\pgfpathclose%
\pgfusepath{stroke,fill}%
\end{pgfscope}%
\begin{pgfscope}%
\pgfpathrectangle{\pgfqpoint{0.100000in}{0.220728in}}{\pgfqpoint{3.696000in}{3.696000in}}%
\pgfusepath{clip}%
\pgfsetbuttcap%
\pgfsetroundjoin%
\definecolor{currentfill}{rgb}{0.121569,0.466667,0.705882}%
\pgfsetfillcolor{currentfill}%
\pgfsetfillopacity{0.616596}%
\pgfsetlinewidth{1.003750pt}%
\definecolor{currentstroke}{rgb}{0.121569,0.466667,0.705882}%
\pgfsetstrokecolor{currentstroke}%
\pgfsetstrokeopacity{0.616596}%
\pgfsetdash{}{0pt}%
\pgfpathmoveto{\pgfqpoint{0.831052in}{1.213574in}}%
\pgfpathcurveto{\pgfqpoint{0.839288in}{1.213574in}}{\pgfqpoint{0.847189in}{1.216846in}}{\pgfqpoint{0.853012in}{1.222670in}}%
\pgfpathcurveto{\pgfqpoint{0.858836in}{1.228494in}}{\pgfqpoint{0.862109in}{1.236394in}}{\pgfqpoint{0.862109in}{1.244631in}}%
\pgfpathcurveto{\pgfqpoint{0.862109in}{1.252867in}}{\pgfqpoint{0.858836in}{1.260767in}}{\pgfqpoint{0.853012in}{1.266591in}}%
\pgfpathcurveto{\pgfqpoint{0.847189in}{1.272415in}}{\pgfqpoint{0.839288in}{1.275687in}}{\pgfqpoint{0.831052in}{1.275687in}}%
\pgfpathcurveto{\pgfqpoint{0.822816in}{1.275687in}}{\pgfqpoint{0.814916in}{1.272415in}}{\pgfqpoint{0.809092in}{1.266591in}}%
\pgfpathcurveto{\pgfqpoint{0.803268in}{1.260767in}}{\pgfqpoint{0.799996in}{1.252867in}}{\pgfqpoint{0.799996in}{1.244631in}}%
\pgfpathcurveto{\pgfqpoint{0.799996in}{1.236394in}}{\pgfqpoint{0.803268in}{1.228494in}}{\pgfqpoint{0.809092in}{1.222670in}}%
\pgfpathcurveto{\pgfqpoint{0.814916in}{1.216846in}}{\pgfqpoint{0.822816in}{1.213574in}}{\pgfqpoint{0.831052in}{1.213574in}}%
\pgfpathclose%
\pgfusepath{stroke,fill}%
\end{pgfscope}%
\begin{pgfscope}%
\pgfpathrectangle{\pgfqpoint{0.100000in}{0.220728in}}{\pgfqpoint{3.696000in}{3.696000in}}%
\pgfusepath{clip}%
\pgfsetbuttcap%
\pgfsetroundjoin%
\definecolor{currentfill}{rgb}{0.121569,0.466667,0.705882}%
\pgfsetfillcolor{currentfill}%
\pgfsetfillopacity{0.616596}%
\pgfsetlinewidth{1.003750pt}%
\definecolor{currentstroke}{rgb}{0.121569,0.466667,0.705882}%
\pgfsetstrokecolor{currentstroke}%
\pgfsetstrokeopacity{0.616596}%
\pgfsetdash{}{0pt}%
\pgfpathmoveto{\pgfqpoint{0.831052in}{1.213574in}}%
\pgfpathcurveto{\pgfqpoint{0.839288in}{1.213574in}}{\pgfqpoint{0.847189in}{1.216846in}}{\pgfqpoint{0.853012in}{1.222670in}}%
\pgfpathcurveto{\pgfqpoint{0.858836in}{1.228494in}}{\pgfqpoint{0.862109in}{1.236394in}}{\pgfqpoint{0.862109in}{1.244631in}}%
\pgfpathcurveto{\pgfqpoint{0.862109in}{1.252867in}}{\pgfqpoint{0.858836in}{1.260767in}}{\pgfqpoint{0.853012in}{1.266591in}}%
\pgfpathcurveto{\pgfqpoint{0.847189in}{1.272415in}}{\pgfqpoint{0.839288in}{1.275687in}}{\pgfqpoint{0.831052in}{1.275687in}}%
\pgfpathcurveto{\pgfqpoint{0.822816in}{1.275687in}}{\pgfqpoint{0.814916in}{1.272415in}}{\pgfqpoint{0.809092in}{1.266591in}}%
\pgfpathcurveto{\pgfqpoint{0.803268in}{1.260767in}}{\pgfqpoint{0.799996in}{1.252867in}}{\pgfqpoint{0.799996in}{1.244631in}}%
\pgfpathcurveto{\pgfqpoint{0.799996in}{1.236394in}}{\pgfqpoint{0.803268in}{1.228494in}}{\pgfqpoint{0.809092in}{1.222670in}}%
\pgfpathcurveto{\pgfqpoint{0.814916in}{1.216846in}}{\pgfqpoint{0.822816in}{1.213574in}}{\pgfqpoint{0.831052in}{1.213574in}}%
\pgfpathclose%
\pgfusepath{stroke,fill}%
\end{pgfscope}%
\begin{pgfscope}%
\pgfpathrectangle{\pgfqpoint{0.100000in}{0.220728in}}{\pgfqpoint{3.696000in}{3.696000in}}%
\pgfusepath{clip}%
\pgfsetbuttcap%
\pgfsetroundjoin%
\definecolor{currentfill}{rgb}{0.121569,0.466667,0.705882}%
\pgfsetfillcolor{currentfill}%
\pgfsetfillopacity{0.616596}%
\pgfsetlinewidth{1.003750pt}%
\definecolor{currentstroke}{rgb}{0.121569,0.466667,0.705882}%
\pgfsetstrokecolor{currentstroke}%
\pgfsetstrokeopacity{0.616596}%
\pgfsetdash{}{0pt}%
\pgfpathmoveto{\pgfqpoint{0.831052in}{1.213574in}}%
\pgfpathcurveto{\pgfqpoint{0.839288in}{1.213574in}}{\pgfqpoint{0.847189in}{1.216846in}}{\pgfqpoint{0.853012in}{1.222670in}}%
\pgfpathcurveto{\pgfqpoint{0.858836in}{1.228494in}}{\pgfqpoint{0.862109in}{1.236394in}}{\pgfqpoint{0.862109in}{1.244631in}}%
\pgfpathcurveto{\pgfqpoint{0.862109in}{1.252867in}}{\pgfqpoint{0.858836in}{1.260767in}}{\pgfqpoint{0.853012in}{1.266591in}}%
\pgfpathcurveto{\pgfqpoint{0.847189in}{1.272415in}}{\pgfqpoint{0.839288in}{1.275687in}}{\pgfqpoint{0.831052in}{1.275687in}}%
\pgfpathcurveto{\pgfqpoint{0.822816in}{1.275687in}}{\pgfqpoint{0.814916in}{1.272415in}}{\pgfqpoint{0.809092in}{1.266591in}}%
\pgfpathcurveto{\pgfqpoint{0.803268in}{1.260767in}}{\pgfqpoint{0.799996in}{1.252867in}}{\pgfqpoint{0.799996in}{1.244631in}}%
\pgfpathcurveto{\pgfqpoint{0.799996in}{1.236394in}}{\pgfqpoint{0.803268in}{1.228494in}}{\pgfqpoint{0.809092in}{1.222670in}}%
\pgfpathcurveto{\pgfqpoint{0.814916in}{1.216846in}}{\pgfqpoint{0.822816in}{1.213574in}}{\pgfqpoint{0.831052in}{1.213574in}}%
\pgfpathclose%
\pgfusepath{stroke,fill}%
\end{pgfscope}%
\begin{pgfscope}%
\pgfpathrectangle{\pgfqpoint{0.100000in}{0.220728in}}{\pgfqpoint{3.696000in}{3.696000in}}%
\pgfusepath{clip}%
\pgfsetbuttcap%
\pgfsetroundjoin%
\definecolor{currentfill}{rgb}{0.121569,0.466667,0.705882}%
\pgfsetfillcolor{currentfill}%
\pgfsetfillopacity{0.616596}%
\pgfsetlinewidth{1.003750pt}%
\definecolor{currentstroke}{rgb}{0.121569,0.466667,0.705882}%
\pgfsetstrokecolor{currentstroke}%
\pgfsetstrokeopacity{0.616596}%
\pgfsetdash{}{0pt}%
\pgfpathmoveto{\pgfqpoint{0.831052in}{1.213574in}}%
\pgfpathcurveto{\pgfqpoint{0.839288in}{1.213574in}}{\pgfqpoint{0.847189in}{1.216846in}}{\pgfqpoint{0.853012in}{1.222670in}}%
\pgfpathcurveto{\pgfqpoint{0.858836in}{1.228494in}}{\pgfqpoint{0.862109in}{1.236394in}}{\pgfqpoint{0.862109in}{1.244631in}}%
\pgfpathcurveto{\pgfqpoint{0.862109in}{1.252867in}}{\pgfqpoint{0.858836in}{1.260767in}}{\pgfqpoint{0.853012in}{1.266591in}}%
\pgfpathcurveto{\pgfqpoint{0.847189in}{1.272415in}}{\pgfqpoint{0.839288in}{1.275687in}}{\pgfqpoint{0.831052in}{1.275687in}}%
\pgfpathcurveto{\pgfqpoint{0.822816in}{1.275687in}}{\pgfqpoint{0.814916in}{1.272415in}}{\pgfqpoint{0.809092in}{1.266591in}}%
\pgfpathcurveto{\pgfqpoint{0.803268in}{1.260767in}}{\pgfqpoint{0.799996in}{1.252867in}}{\pgfqpoint{0.799996in}{1.244631in}}%
\pgfpathcurveto{\pgfqpoint{0.799996in}{1.236394in}}{\pgfqpoint{0.803268in}{1.228494in}}{\pgfqpoint{0.809092in}{1.222670in}}%
\pgfpathcurveto{\pgfqpoint{0.814916in}{1.216846in}}{\pgfqpoint{0.822816in}{1.213574in}}{\pgfqpoint{0.831052in}{1.213574in}}%
\pgfpathclose%
\pgfusepath{stroke,fill}%
\end{pgfscope}%
\begin{pgfscope}%
\pgfpathrectangle{\pgfqpoint{0.100000in}{0.220728in}}{\pgfqpoint{3.696000in}{3.696000in}}%
\pgfusepath{clip}%
\pgfsetbuttcap%
\pgfsetroundjoin%
\definecolor{currentfill}{rgb}{0.121569,0.466667,0.705882}%
\pgfsetfillcolor{currentfill}%
\pgfsetfillopacity{0.616596}%
\pgfsetlinewidth{1.003750pt}%
\definecolor{currentstroke}{rgb}{0.121569,0.466667,0.705882}%
\pgfsetstrokecolor{currentstroke}%
\pgfsetstrokeopacity{0.616596}%
\pgfsetdash{}{0pt}%
\pgfpathmoveto{\pgfqpoint{0.831052in}{1.213574in}}%
\pgfpathcurveto{\pgfqpoint{0.839288in}{1.213574in}}{\pgfqpoint{0.847189in}{1.216846in}}{\pgfqpoint{0.853012in}{1.222670in}}%
\pgfpathcurveto{\pgfqpoint{0.858836in}{1.228494in}}{\pgfqpoint{0.862109in}{1.236394in}}{\pgfqpoint{0.862109in}{1.244631in}}%
\pgfpathcurveto{\pgfqpoint{0.862109in}{1.252867in}}{\pgfqpoint{0.858836in}{1.260767in}}{\pgfqpoint{0.853012in}{1.266591in}}%
\pgfpathcurveto{\pgfqpoint{0.847189in}{1.272415in}}{\pgfqpoint{0.839288in}{1.275687in}}{\pgfqpoint{0.831052in}{1.275687in}}%
\pgfpathcurveto{\pgfqpoint{0.822816in}{1.275687in}}{\pgfqpoint{0.814916in}{1.272415in}}{\pgfqpoint{0.809092in}{1.266591in}}%
\pgfpathcurveto{\pgfqpoint{0.803268in}{1.260767in}}{\pgfqpoint{0.799996in}{1.252867in}}{\pgfqpoint{0.799996in}{1.244631in}}%
\pgfpathcurveto{\pgfqpoint{0.799996in}{1.236394in}}{\pgfqpoint{0.803268in}{1.228494in}}{\pgfqpoint{0.809092in}{1.222670in}}%
\pgfpathcurveto{\pgfqpoint{0.814916in}{1.216846in}}{\pgfqpoint{0.822816in}{1.213574in}}{\pgfqpoint{0.831052in}{1.213574in}}%
\pgfpathclose%
\pgfusepath{stroke,fill}%
\end{pgfscope}%
\begin{pgfscope}%
\pgfpathrectangle{\pgfqpoint{0.100000in}{0.220728in}}{\pgfqpoint{3.696000in}{3.696000in}}%
\pgfusepath{clip}%
\pgfsetbuttcap%
\pgfsetroundjoin%
\definecolor{currentfill}{rgb}{0.121569,0.466667,0.705882}%
\pgfsetfillcolor{currentfill}%
\pgfsetfillopacity{0.616596}%
\pgfsetlinewidth{1.003750pt}%
\definecolor{currentstroke}{rgb}{0.121569,0.466667,0.705882}%
\pgfsetstrokecolor{currentstroke}%
\pgfsetstrokeopacity{0.616596}%
\pgfsetdash{}{0pt}%
\pgfpathmoveto{\pgfqpoint{0.831052in}{1.213574in}}%
\pgfpathcurveto{\pgfqpoint{0.839288in}{1.213574in}}{\pgfqpoint{0.847189in}{1.216846in}}{\pgfqpoint{0.853012in}{1.222670in}}%
\pgfpathcurveto{\pgfqpoint{0.858836in}{1.228494in}}{\pgfqpoint{0.862109in}{1.236394in}}{\pgfqpoint{0.862109in}{1.244631in}}%
\pgfpathcurveto{\pgfqpoint{0.862109in}{1.252867in}}{\pgfqpoint{0.858836in}{1.260767in}}{\pgfqpoint{0.853012in}{1.266591in}}%
\pgfpathcurveto{\pgfqpoint{0.847189in}{1.272415in}}{\pgfqpoint{0.839288in}{1.275687in}}{\pgfqpoint{0.831052in}{1.275687in}}%
\pgfpathcurveto{\pgfqpoint{0.822816in}{1.275687in}}{\pgfqpoint{0.814916in}{1.272415in}}{\pgfqpoint{0.809092in}{1.266591in}}%
\pgfpathcurveto{\pgfqpoint{0.803268in}{1.260767in}}{\pgfqpoint{0.799996in}{1.252867in}}{\pgfqpoint{0.799996in}{1.244631in}}%
\pgfpathcurveto{\pgfqpoint{0.799996in}{1.236394in}}{\pgfqpoint{0.803268in}{1.228494in}}{\pgfqpoint{0.809092in}{1.222670in}}%
\pgfpathcurveto{\pgfqpoint{0.814916in}{1.216846in}}{\pgfqpoint{0.822816in}{1.213574in}}{\pgfqpoint{0.831052in}{1.213574in}}%
\pgfpathclose%
\pgfusepath{stroke,fill}%
\end{pgfscope}%
\begin{pgfscope}%
\pgfpathrectangle{\pgfqpoint{0.100000in}{0.220728in}}{\pgfqpoint{3.696000in}{3.696000in}}%
\pgfusepath{clip}%
\pgfsetbuttcap%
\pgfsetroundjoin%
\definecolor{currentfill}{rgb}{0.121569,0.466667,0.705882}%
\pgfsetfillcolor{currentfill}%
\pgfsetfillopacity{0.616596}%
\pgfsetlinewidth{1.003750pt}%
\definecolor{currentstroke}{rgb}{0.121569,0.466667,0.705882}%
\pgfsetstrokecolor{currentstroke}%
\pgfsetstrokeopacity{0.616596}%
\pgfsetdash{}{0pt}%
\pgfpathmoveto{\pgfqpoint{0.831052in}{1.213574in}}%
\pgfpathcurveto{\pgfqpoint{0.839288in}{1.213574in}}{\pgfqpoint{0.847189in}{1.216846in}}{\pgfqpoint{0.853012in}{1.222670in}}%
\pgfpathcurveto{\pgfqpoint{0.858836in}{1.228494in}}{\pgfqpoint{0.862109in}{1.236394in}}{\pgfqpoint{0.862109in}{1.244631in}}%
\pgfpathcurveto{\pgfqpoint{0.862109in}{1.252867in}}{\pgfqpoint{0.858836in}{1.260767in}}{\pgfqpoint{0.853012in}{1.266591in}}%
\pgfpathcurveto{\pgfqpoint{0.847189in}{1.272415in}}{\pgfqpoint{0.839288in}{1.275687in}}{\pgfqpoint{0.831052in}{1.275687in}}%
\pgfpathcurveto{\pgfqpoint{0.822816in}{1.275687in}}{\pgfqpoint{0.814916in}{1.272415in}}{\pgfqpoint{0.809092in}{1.266591in}}%
\pgfpathcurveto{\pgfqpoint{0.803268in}{1.260767in}}{\pgfqpoint{0.799996in}{1.252867in}}{\pgfqpoint{0.799996in}{1.244631in}}%
\pgfpathcurveto{\pgfqpoint{0.799996in}{1.236394in}}{\pgfqpoint{0.803268in}{1.228494in}}{\pgfqpoint{0.809092in}{1.222670in}}%
\pgfpathcurveto{\pgfqpoint{0.814916in}{1.216846in}}{\pgfqpoint{0.822816in}{1.213574in}}{\pgfqpoint{0.831052in}{1.213574in}}%
\pgfpathclose%
\pgfusepath{stroke,fill}%
\end{pgfscope}%
\begin{pgfscope}%
\pgfpathrectangle{\pgfqpoint{0.100000in}{0.220728in}}{\pgfqpoint{3.696000in}{3.696000in}}%
\pgfusepath{clip}%
\pgfsetbuttcap%
\pgfsetroundjoin%
\definecolor{currentfill}{rgb}{0.121569,0.466667,0.705882}%
\pgfsetfillcolor{currentfill}%
\pgfsetfillopacity{0.616596}%
\pgfsetlinewidth{1.003750pt}%
\definecolor{currentstroke}{rgb}{0.121569,0.466667,0.705882}%
\pgfsetstrokecolor{currentstroke}%
\pgfsetstrokeopacity{0.616596}%
\pgfsetdash{}{0pt}%
\pgfpathmoveto{\pgfqpoint{0.831052in}{1.213574in}}%
\pgfpathcurveto{\pgfqpoint{0.839288in}{1.213574in}}{\pgfqpoint{0.847189in}{1.216846in}}{\pgfqpoint{0.853012in}{1.222670in}}%
\pgfpathcurveto{\pgfqpoint{0.858836in}{1.228494in}}{\pgfqpoint{0.862109in}{1.236394in}}{\pgfqpoint{0.862109in}{1.244631in}}%
\pgfpathcurveto{\pgfqpoint{0.862109in}{1.252867in}}{\pgfqpoint{0.858836in}{1.260767in}}{\pgfqpoint{0.853012in}{1.266591in}}%
\pgfpathcurveto{\pgfqpoint{0.847189in}{1.272415in}}{\pgfqpoint{0.839288in}{1.275687in}}{\pgfqpoint{0.831052in}{1.275687in}}%
\pgfpathcurveto{\pgfqpoint{0.822816in}{1.275687in}}{\pgfqpoint{0.814916in}{1.272415in}}{\pgfqpoint{0.809092in}{1.266591in}}%
\pgfpathcurveto{\pgfqpoint{0.803268in}{1.260767in}}{\pgfqpoint{0.799996in}{1.252867in}}{\pgfqpoint{0.799996in}{1.244631in}}%
\pgfpathcurveto{\pgfqpoint{0.799996in}{1.236394in}}{\pgfqpoint{0.803268in}{1.228494in}}{\pgfqpoint{0.809092in}{1.222670in}}%
\pgfpathcurveto{\pgfqpoint{0.814916in}{1.216846in}}{\pgfqpoint{0.822816in}{1.213574in}}{\pgfqpoint{0.831052in}{1.213574in}}%
\pgfpathclose%
\pgfusepath{stroke,fill}%
\end{pgfscope}%
\begin{pgfscope}%
\pgfpathrectangle{\pgfqpoint{0.100000in}{0.220728in}}{\pgfqpoint{3.696000in}{3.696000in}}%
\pgfusepath{clip}%
\pgfsetbuttcap%
\pgfsetroundjoin%
\definecolor{currentfill}{rgb}{0.121569,0.466667,0.705882}%
\pgfsetfillcolor{currentfill}%
\pgfsetfillopacity{0.616596}%
\pgfsetlinewidth{1.003750pt}%
\definecolor{currentstroke}{rgb}{0.121569,0.466667,0.705882}%
\pgfsetstrokecolor{currentstroke}%
\pgfsetstrokeopacity{0.616596}%
\pgfsetdash{}{0pt}%
\pgfpathmoveto{\pgfqpoint{0.831052in}{1.213574in}}%
\pgfpathcurveto{\pgfqpoint{0.839288in}{1.213574in}}{\pgfqpoint{0.847189in}{1.216846in}}{\pgfqpoint{0.853012in}{1.222670in}}%
\pgfpathcurveto{\pgfqpoint{0.858836in}{1.228494in}}{\pgfqpoint{0.862109in}{1.236394in}}{\pgfqpoint{0.862109in}{1.244631in}}%
\pgfpathcurveto{\pgfqpoint{0.862109in}{1.252867in}}{\pgfqpoint{0.858836in}{1.260767in}}{\pgfqpoint{0.853012in}{1.266591in}}%
\pgfpathcurveto{\pgfqpoint{0.847189in}{1.272415in}}{\pgfqpoint{0.839288in}{1.275687in}}{\pgfqpoint{0.831052in}{1.275687in}}%
\pgfpathcurveto{\pgfqpoint{0.822816in}{1.275687in}}{\pgfqpoint{0.814916in}{1.272415in}}{\pgfqpoint{0.809092in}{1.266591in}}%
\pgfpathcurveto{\pgfqpoint{0.803268in}{1.260767in}}{\pgfqpoint{0.799996in}{1.252867in}}{\pgfqpoint{0.799996in}{1.244631in}}%
\pgfpathcurveto{\pgfqpoint{0.799996in}{1.236394in}}{\pgfqpoint{0.803268in}{1.228494in}}{\pgfqpoint{0.809092in}{1.222670in}}%
\pgfpathcurveto{\pgfqpoint{0.814916in}{1.216846in}}{\pgfqpoint{0.822816in}{1.213574in}}{\pgfqpoint{0.831052in}{1.213574in}}%
\pgfpathclose%
\pgfusepath{stroke,fill}%
\end{pgfscope}%
\begin{pgfscope}%
\pgfpathrectangle{\pgfqpoint{0.100000in}{0.220728in}}{\pgfqpoint{3.696000in}{3.696000in}}%
\pgfusepath{clip}%
\pgfsetbuttcap%
\pgfsetroundjoin%
\definecolor{currentfill}{rgb}{0.121569,0.466667,0.705882}%
\pgfsetfillcolor{currentfill}%
\pgfsetfillopacity{0.616596}%
\pgfsetlinewidth{1.003750pt}%
\definecolor{currentstroke}{rgb}{0.121569,0.466667,0.705882}%
\pgfsetstrokecolor{currentstroke}%
\pgfsetstrokeopacity{0.616596}%
\pgfsetdash{}{0pt}%
\pgfpathmoveto{\pgfqpoint{0.831052in}{1.213574in}}%
\pgfpathcurveto{\pgfqpoint{0.839288in}{1.213574in}}{\pgfqpoint{0.847189in}{1.216846in}}{\pgfqpoint{0.853012in}{1.222670in}}%
\pgfpathcurveto{\pgfqpoint{0.858836in}{1.228494in}}{\pgfqpoint{0.862109in}{1.236394in}}{\pgfqpoint{0.862109in}{1.244631in}}%
\pgfpathcurveto{\pgfqpoint{0.862109in}{1.252867in}}{\pgfqpoint{0.858836in}{1.260767in}}{\pgfqpoint{0.853012in}{1.266591in}}%
\pgfpathcurveto{\pgfqpoint{0.847189in}{1.272415in}}{\pgfqpoint{0.839288in}{1.275687in}}{\pgfqpoint{0.831052in}{1.275687in}}%
\pgfpathcurveto{\pgfqpoint{0.822816in}{1.275687in}}{\pgfqpoint{0.814916in}{1.272415in}}{\pgfqpoint{0.809092in}{1.266591in}}%
\pgfpathcurveto{\pgfqpoint{0.803268in}{1.260767in}}{\pgfqpoint{0.799996in}{1.252867in}}{\pgfqpoint{0.799996in}{1.244631in}}%
\pgfpathcurveto{\pgfqpoint{0.799996in}{1.236394in}}{\pgfqpoint{0.803268in}{1.228494in}}{\pgfqpoint{0.809092in}{1.222670in}}%
\pgfpathcurveto{\pgfqpoint{0.814916in}{1.216846in}}{\pgfqpoint{0.822816in}{1.213574in}}{\pgfqpoint{0.831052in}{1.213574in}}%
\pgfpathclose%
\pgfusepath{stroke,fill}%
\end{pgfscope}%
\begin{pgfscope}%
\pgfpathrectangle{\pgfqpoint{0.100000in}{0.220728in}}{\pgfqpoint{3.696000in}{3.696000in}}%
\pgfusepath{clip}%
\pgfsetbuttcap%
\pgfsetroundjoin%
\definecolor{currentfill}{rgb}{0.121569,0.466667,0.705882}%
\pgfsetfillcolor{currentfill}%
\pgfsetfillopacity{0.616596}%
\pgfsetlinewidth{1.003750pt}%
\definecolor{currentstroke}{rgb}{0.121569,0.466667,0.705882}%
\pgfsetstrokecolor{currentstroke}%
\pgfsetstrokeopacity{0.616596}%
\pgfsetdash{}{0pt}%
\pgfpathmoveto{\pgfqpoint{0.831052in}{1.213574in}}%
\pgfpathcurveto{\pgfqpoint{0.839288in}{1.213574in}}{\pgfqpoint{0.847189in}{1.216846in}}{\pgfqpoint{0.853012in}{1.222670in}}%
\pgfpathcurveto{\pgfqpoint{0.858836in}{1.228494in}}{\pgfqpoint{0.862109in}{1.236394in}}{\pgfqpoint{0.862109in}{1.244631in}}%
\pgfpathcurveto{\pgfqpoint{0.862109in}{1.252867in}}{\pgfqpoint{0.858836in}{1.260767in}}{\pgfqpoint{0.853012in}{1.266591in}}%
\pgfpathcurveto{\pgfqpoint{0.847189in}{1.272415in}}{\pgfqpoint{0.839288in}{1.275687in}}{\pgfqpoint{0.831052in}{1.275687in}}%
\pgfpathcurveto{\pgfqpoint{0.822816in}{1.275687in}}{\pgfqpoint{0.814916in}{1.272415in}}{\pgfqpoint{0.809092in}{1.266591in}}%
\pgfpathcurveto{\pgfqpoint{0.803268in}{1.260767in}}{\pgfqpoint{0.799996in}{1.252867in}}{\pgfqpoint{0.799996in}{1.244631in}}%
\pgfpathcurveto{\pgfqpoint{0.799996in}{1.236394in}}{\pgfqpoint{0.803268in}{1.228494in}}{\pgfqpoint{0.809092in}{1.222670in}}%
\pgfpathcurveto{\pgfqpoint{0.814916in}{1.216846in}}{\pgfqpoint{0.822816in}{1.213574in}}{\pgfqpoint{0.831052in}{1.213574in}}%
\pgfpathclose%
\pgfusepath{stroke,fill}%
\end{pgfscope}%
\begin{pgfscope}%
\pgfpathrectangle{\pgfqpoint{0.100000in}{0.220728in}}{\pgfqpoint{3.696000in}{3.696000in}}%
\pgfusepath{clip}%
\pgfsetbuttcap%
\pgfsetroundjoin%
\definecolor{currentfill}{rgb}{0.121569,0.466667,0.705882}%
\pgfsetfillcolor{currentfill}%
\pgfsetfillopacity{0.616596}%
\pgfsetlinewidth{1.003750pt}%
\definecolor{currentstroke}{rgb}{0.121569,0.466667,0.705882}%
\pgfsetstrokecolor{currentstroke}%
\pgfsetstrokeopacity{0.616596}%
\pgfsetdash{}{0pt}%
\pgfpathmoveto{\pgfqpoint{0.831052in}{1.213574in}}%
\pgfpathcurveto{\pgfqpoint{0.839288in}{1.213574in}}{\pgfqpoint{0.847189in}{1.216846in}}{\pgfqpoint{0.853012in}{1.222670in}}%
\pgfpathcurveto{\pgfqpoint{0.858836in}{1.228494in}}{\pgfqpoint{0.862109in}{1.236394in}}{\pgfqpoint{0.862109in}{1.244631in}}%
\pgfpathcurveto{\pgfqpoint{0.862109in}{1.252867in}}{\pgfqpoint{0.858836in}{1.260767in}}{\pgfqpoint{0.853012in}{1.266591in}}%
\pgfpathcurveto{\pgfqpoint{0.847189in}{1.272415in}}{\pgfqpoint{0.839288in}{1.275687in}}{\pgfqpoint{0.831052in}{1.275687in}}%
\pgfpathcurveto{\pgfqpoint{0.822816in}{1.275687in}}{\pgfqpoint{0.814916in}{1.272415in}}{\pgfqpoint{0.809092in}{1.266591in}}%
\pgfpathcurveto{\pgfqpoint{0.803268in}{1.260767in}}{\pgfqpoint{0.799996in}{1.252867in}}{\pgfqpoint{0.799996in}{1.244631in}}%
\pgfpathcurveto{\pgfqpoint{0.799996in}{1.236394in}}{\pgfqpoint{0.803268in}{1.228494in}}{\pgfqpoint{0.809092in}{1.222670in}}%
\pgfpathcurveto{\pgfqpoint{0.814916in}{1.216846in}}{\pgfqpoint{0.822816in}{1.213574in}}{\pgfqpoint{0.831052in}{1.213574in}}%
\pgfpathclose%
\pgfusepath{stroke,fill}%
\end{pgfscope}%
\begin{pgfscope}%
\pgfpathrectangle{\pgfqpoint{0.100000in}{0.220728in}}{\pgfqpoint{3.696000in}{3.696000in}}%
\pgfusepath{clip}%
\pgfsetbuttcap%
\pgfsetroundjoin%
\definecolor{currentfill}{rgb}{0.121569,0.466667,0.705882}%
\pgfsetfillcolor{currentfill}%
\pgfsetfillopacity{0.616596}%
\pgfsetlinewidth{1.003750pt}%
\definecolor{currentstroke}{rgb}{0.121569,0.466667,0.705882}%
\pgfsetstrokecolor{currentstroke}%
\pgfsetstrokeopacity{0.616596}%
\pgfsetdash{}{0pt}%
\pgfpathmoveto{\pgfqpoint{0.831052in}{1.213574in}}%
\pgfpathcurveto{\pgfqpoint{0.839288in}{1.213574in}}{\pgfqpoint{0.847189in}{1.216846in}}{\pgfqpoint{0.853012in}{1.222670in}}%
\pgfpathcurveto{\pgfqpoint{0.858836in}{1.228494in}}{\pgfqpoint{0.862109in}{1.236394in}}{\pgfqpoint{0.862109in}{1.244631in}}%
\pgfpathcurveto{\pgfqpoint{0.862109in}{1.252867in}}{\pgfqpoint{0.858836in}{1.260767in}}{\pgfqpoint{0.853012in}{1.266591in}}%
\pgfpathcurveto{\pgfqpoint{0.847189in}{1.272415in}}{\pgfqpoint{0.839288in}{1.275687in}}{\pgfqpoint{0.831052in}{1.275687in}}%
\pgfpathcurveto{\pgfqpoint{0.822816in}{1.275687in}}{\pgfqpoint{0.814916in}{1.272415in}}{\pgfqpoint{0.809092in}{1.266591in}}%
\pgfpathcurveto{\pgfqpoint{0.803268in}{1.260767in}}{\pgfqpoint{0.799996in}{1.252867in}}{\pgfqpoint{0.799996in}{1.244631in}}%
\pgfpathcurveto{\pgfqpoint{0.799996in}{1.236394in}}{\pgfqpoint{0.803268in}{1.228494in}}{\pgfqpoint{0.809092in}{1.222670in}}%
\pgfpathcurveto{\pgfqpoint{0.814916in}{1.216846in}}{\pgfqpoint{0.822816in}{1.213574in}}{\pgfqpoint{0.831052in}{1.213574in}}%
\pgfpathclose%
\pgfusepath{stroke,fill}%
\end{pgfscope}%
\begin{pgfscope}%
\pgfpathrectangle{\pgfqpoint{0.100000in}{0.220728in}}{\pgfqpoint{3.696000in}{3.696000in}}%
\pgfusepath{clip}%
\pgfsetbuttcap%
\pgfsetroundjoin%
\definecolor{currentfill}{rgb}{0.121569,0.466667,0.705882}%
\pgfsetfillcolor{currentfill}%
\pgfsetfillopacity{0.616636}%
\pgfsetlinewidth{1.003750pt}%
\definecolor{currentstroke}{rgb}{0.121569,0.466667,0.705882}%
\pgfsetstrokecolor{currentstroke}%
\pgfsetstrokeopacity{0.616636}%
\pgfsetdash{}{0pt}%
\pgfpathmoveto{\pgfqpoint{0.828691in}{1.212634in}}%
\pgfpathcurveto{\pgfqpoint{0.836927in}{1.212634in}}{\pgfqpoint{0.844827in}{1.215907in}}{\pgfqpoint{0.850651in}{1.221731in}}%
\pgfpathcurveto{\pgfqpoint{0.856475in}{1.227554in}}{\pgfqpoint{0.859748in}{1.235454in}}{\pgfqpoint{0.859748in}{1.243691in}}%
\pgfpathcurveto{\pgfqpoint{0.859748in}{1.251927in}}{\pgfqpoint{0.856475in}{1.259827in}}{\pgfqpoint{0.850651in}{1.265651in}}%
\pgfpathcurveto{\pgfqpoint{0.844827in}{1.271475in}}{\pgfqpoint{0.836927in}{1.274747in}}{\pgfqpoint{0.828691in}{1.274747in}}%
\pgfpathcurveto{\pgfqpoint{0.820455in}{1.274747in}}{\pgfqpoint{0.812555in}{1.271475in}}{\pgfqpoint{0.806731in}{1.265651in}}%
\pgfpathcurveto{\pgfqpoint{0.800907in}{1.259827in}}{\pgfqpoint{0.797635in}{1.251927in}}{\pgfqpoint{0.797635in}{1.243691in}}%
\pgfpathcurveto{\pgfqpoint{0.797635in}{1.235454in}}{\pgfqpoint{0.800907in}{1.227554in}}{\pgfqpoint{0.806731in}{1.221731in}}%
\pgfpathcurveto{\pgfqpoint{0.812555in}{1.215907in}}{\pgfqpoint{0.820455in}{1.212634in}}{\pgfqpoint{0.828691in}{1.212634in}}%
\pgfpathclose%
\pgfusepath{stroke,fill}%
\end{pgfscope}%
\begin{pgfscope}%
\pgfpathrectangle{\pgfqpoint{0.100000in}{0.220728in}}{\pgfqpoint{3.696000in}{3.696000in}}%
\pgfusepath{clip}%
\pgfsetbuttcap%
\pgfsetroundjoin%
\definecolor{currentfill}{rgb}{0.121569,0.466667,0.705882}%
\pgfsetfillcolor{currentfill}%
\pgfsetfillopacity{0.618213}%
\pgfsetlinewidth{1.003750pt}%
\definecolor{currentstroke}{rgb}{0.121569,0.466667,0.705882}%
\pgfsetstrokecolor{currentstroke}%
\pgfsetstrokeopacity{0.618213}%
\pgfsetdash{}{0pt}%
\pgfpathmoveto{\pgfqpoint{3.135636in}{2.979479in}}%
\pgfpathcurveto{\pgfqpoint{3.143873in}{2.979479in}}{\pgfqpoint{3.151773in}{2.982751in}}{\pgfqpoint{3.157597in}{2.988575in}}%
\pgfpathcurveto{\pgfqpoint{3.163421in}{2.994399in}}{\pgfqpoint{3.166693in}{3.002299in}}{\pgfqpoint{3.166693in}{3.010536in}}%
\pgfpathcurveto{\pgfqpoint{3.166693in}{3.018772in}}{\pgfqpoint{3.163421in}{3.026672in}}{\pgfqpoint{3.157597in}{3.032496in}}%
\pgfpathcurveto{\pgfqpoint{3.151773in}{3.038320in}}{\pgfqpoint{3.143873in}{3.041592in}}{\pgfqpoint{3.135636in}{3.041592in}}%
\pgfpathcurveto{\pgfqpoint{3.127400in}{3.041592in}}{\pgfqpoint{3.119500in}{3.038320in}}{\pgfqpoint{3.113676in}{3.032496in}}%
\pgfpathcurveto{\pgfqpoint{3.107852in}{3.026672in}}{\pgfqpoint{3.104580in}{3.018772in}}{\pgfqpoint{3.104580in}{3.010536in}}%
\pgfpathcurveto{\pgfqpoint{3.104580in}{3.002299in}}{\pgfqpoint{3.107852in}{2.994399in}}{\pgfqpoint{3.113676in}{2.988575in}}%
\pgfpathcurveto{\pgfqpoint{3.119500in}{2.982751in}}{\pgfqpoint{3.127400in}{2.979479in}}{\pgfqpoint{3.135636in}{2.979479in}}%
\pgfpathclose%
\pgfusepath{stroke,fill}%
\end{pgfscope}%
\begin{pgfscope}%
\pgfpathrectangle{\pgfqpoint{0.100000in}{0.220728in}}{\pgfqpoint{3.696000in}{3.696000in}}%
\pgfusepath{clip}%
\pgfsetbuttcap%
\pgfsetroundjoin%
\definecolor{currentfill}{rgb}{0.121569,0.466667,0.705882}%
\pgfsetfillcolor{currentfill}%
\pgfsetfillopacity{0.622365}%
\pgfsetlinewidth{1.003750pt}%
\definecolor{currentstroke}{rgb}{0.121569,0.466667,0.705882}%
\pgfsetstrokecolor{currentstroke}%
\pgfsetstrokeopacity{0.622365}%
\pgfsetdash{}{0pt}%
\pgfpathmoveto{\pgfqpoint{3.152059in}{2.976570in}}%
\pgfpathcurveto{\pgfqpoint{3.160296in}{2.976570in}}{\pgfqpoint{3.168196in}{2.979843in}}{\pgfqpoint{3.174019in}{2.985667in}}%
\pgfpathcurveto{\pgfqpoint{3.179843in}{2.991491in}}{\pgfqpoint{3.183116in}{2.999391in}}{\pgfqpoint{3.183116in}{3.007627in}}%
\pgfpathcurveto{\pgfqpoint{3.183116in}{3.015863in}}{\pgfqpoint{3.179843in}{3.023763in}}{\pgfqpoint{3.174019in}{3.029587in}}%
\pgfpathcurveto{\pgfqpoint{3.168196in}{3.035411in}}{\pgfqpoint{3.160296in}{3.038683in}}{\pgfqpoint{3.152059in}{3.038683in}}%
\pgfpathcurveto{\pgfqpoint{3.143823in}{3.038683in}}{\pgfqpoint{3.135923in}{3.035411in}}{\pgfqpoint{3.130099in}{3.029587in}}%
\pgfpathcurveto{\pgfqpoint{3.124275in}{3.023763in}}{\pgfqpoint{3.121003in}{3.015863in}}{\pgfqpoint{3.121003in}{3.007627in}}%
\pgfpathcurveto{\pgfqpoint{3.121003in}{2.999391in}}{\pgfqpoint{3.124275in}{2.991491in}}{\pgfqpoint{3.130099in}{2.985667in}}%
\pgfpathcurveto{\pgfqpoint{3.135923in}{2.979843in}}{\pgfqpoint{3.143823in}{2.976570in}}{\pgfqpoint{3.152059in}{2.976570in}}%
\pgfpathclose%
\pgfusepath{stroke,fill}%
\end{pgfscope}%
\begin{pgfscope}%
\pgfpathrectangle{\pgfqpoint{0.100000in}{0.220728in}}{\pgfqpoint{3.696000in}{3.696000in}}%
\pgfusepath{clip}%
\pgfsetbuttcap%
\pgfsetroundjoin%
\definecolor{currentfill}{rgb}{0.121569,0.466667,0.705882}%
\pgfsetfillcolor{currentfill}%
\pgfsetfillopacity{0.622827}%
\pgfsetlinewidth{1.003750pt}%
\definecolor{currentstroke}{rgb}{0.121569,0.466667,0.705882}%
\pgfsetstrokecolor{currentstroke}%
\pgfsetstrokeopacity{0.622827}%
\pgfsetdash{}{0pt}%
\pgfpathmoveto{\pgfqpoint{3.172542in}{2.973651in}}%
\pgfpathcurveto{\pgfqpoint{3.180778in}{2.973651in}}{\pgfqpoint{3.188678in}{2.976924in}}{\pgfqpoint{3.194502in}{2.982748in}}%
\pgfpathcurveto{\pgfqpoint{3.200326in}{2.988572in}}{\pgfqpoint{3.203598in}{2.996472in}}{\pgfqpoint{3.203598in}{3.004708in}}%
\pgfpathcurveto{\pgfqpoint{3.203598in}{3.012944in}}{\pgfqpoint{3.200326in}{3.020844in}}{\pgfqpoint{3.194502in}{3.026668in}}%
\pgfpathcurveto{\pgfqpoint{3.188678in}{3.032492in}}{\pgfqpoint{3.180778in}{3.035764in}}{\pgfqpoint{3.172542in}{3.035764in}}%
\pgfpathcurveto{\pgfqpoint{3.164305in}{3.035764in}}{\pgfqpoint{3.156405in}{3.032492in}}{\pgfqpoint{3.150581in}{3.026668in}}%
\pgfpathcurveto{\pgfqpoint{3.144757in}{3.020844in}}{\pgfqpoint{3.141485in}{3.012944in}}{\pgfqpoint{3.141485in}{3.004708in}}%
\pgfpathcurveto{\pgfqpoint{3.141485in}{2.996472in}}{\pgfqpoint{3.144757in}{2.988572in}}{\pgfqpoint{3.150581in}{2.982748in}}%
\pgfpathcurveto{\pgfqpoint{3.156405in}{2.976924in}}{\pgfqpoint{3.164305in}{2.973651in}}{\pgfqpoint{3.172542in}{2.973651in}}%
\pgfpathclose%
\pgfusepath{stroke,fill}%
\end{pgfscope}%
\begin{pgfscope}%
\pgfpathrectangle{\pgfqpoint{0.100000in}{0.220728in}}{\pgfqpoint{3.696000in}{3.696000in}}%
\pgfusepath{clip}%
\pgfsetbuttcap%
\pgfsetroundjoin%
\definecolor{currentfill}{rgb}{0.121569,0.466667,0.705882}%
\pgfsetfillcolor{currentfill}%
\pgfsetfillopacity{0.627913}%
\pgfsetlinewidth{1.003750pt}%
\definecolor{currentstroke}{rgb}{0.121569,0.466667,0.705882}%
\pgfsetstrokecolor{currentstroke}%
\pgfsetstrokeopacity{0.627913}%
\pgfsetdash{}{0pt}%
\pgfpathmoveto{\pgfqpoint{3.193063in}{2.972624in}}%
\pgfpathcurveto{\pgfqpoint{3.201300in}{2.972624in}}{\pgfqpoint{3.209200in}{2.975897in}}{\pgfqpoint{3.215023in}{2.981721in}}%
\pgfpathcurveto{\pgfqpoint{3.220847in}{2.987545in}}{\pgfqpoint{3.224120in}{2.995445in}}{\pgfqpoint{3.224120in}{3.003681in}}%
\pgfpathcurveto{\pgfqpoint{3.224120in}{3.011917in}}{\pgfqpoint{3.220847in}{3.019817in}}{\pgfqpoint{3.215023in}{3.025641in}}%
\pgfpathcurveto{\pgfqpoint{3.209200in}{3.031465in}}{\pgfqpoint{3.201300in}{3.034737in}}{\pgfqpoint{3.193063in}{3.034737in}}%
\pgfpathcurveto{\pgfqpoint{3.184827in}{3.034737in}}{\pgfqpoint{3.176927in}{3.031465in}}{\pgfqpoint{3.171103in}{3.025641in}}%
\pgfpathcurveto{\pgfqpoint{3.165279in}{3.019817in}}{\pgfqpoint{3.162007in}{3.011917in}}{\pgfqpoint{3.162007in}{3.003681in}}%
\pgfpathcurveto{\pgfqpoint{3.162007in}{2.995445in}}{\pgfqpoint{3.165279in}{2.987545in}}{\pgfqpoint{3.171103in}{2.981721in}}%
\pgfpathcurveto{\pgfqpoint{3.176927in}{2.975897in}}{\pgfqpoint{3.184827in}{2.972624in}}{\pgfqpoint{3.193063in}{2.972624in}}%
\pgfpathclose%
\pgfusepath{stroke,fill}%
\end{pgfscope}%
\begin{pgfscope}%
\pgfpathrectangle{\pgfqpoint{0.100000in}{0.220728in}}{\pgfqpoint{3.696000in}{3.696000in}}%
\pgfusepath{clip}%
\pgfsetbuttcap%
\pgfsetroundjoin%
\definecolor{currentfill}{rgb}{0.121569,0.466667,0.705882}%
\pgfsetfillcolor{currentfill}%
\pgfsetfillopacity{0.630219}%
\pgfsetlinewidth{1.003750pt}%
\definecolor{currentstroke}{rgb}{0.121569,0.466667,0.705882}%
\pgfsetstrokecolor{currentstroke}%
\pgfsetstrokeopacity{0.630219}%
\pgfsetdash{}{0pt}%
\pgfpathmoveto{\pgfqpoint{3.204397in}{2.970358in}}%
\pgfpathcurveto{\pgfqpoint{3.212633in}{2.970358in}}{\pgfqpoint{3.220533in}{2.973631in}}{\pgfqpoint{3.226357in}{2.979455in}}%
\pgfpathcurveto{\pgfqpoint{3.232181in}{2.985279in}}{\pgfqpoint{3.235453in}{2.993179in}}{\pgfqpoint{3.235453in}{3.001415in}}%
\pgfpathcurveto{\pgfqpoint{3.235453in}{3.009651in}}{\pgfqpoint{3.232181in}{3.017551in}}{\pgfqpoint{3.226357in}{3.023375in}}%
\pgfpathcurveto{\pgfqpoint{3.220533in}{3.029199in}}{\pgfqpoint{3.212633in}{3.032471in}}{\pgfqpoint{3.204397in}{3.032471in}}%
\pgfpathcurveto{\pgfqpoint{3.196161in}{3.032471in}}{\pgfqpoint{3.188261in}{3.029199in}}{\pgfqpoint{3.182437in}{3.023375in}}%
\pgfpathcurveto{\pgfqpoint{3.176613in}{3.017551in}}{\pgfqpoint{3.173340in}{3.009651in}}{\pgfqpoint{3.173340in}{3.001415in}}%
\pgfpathcurveto{\pgfqpoint{3.173340in}{2.993179in}}{\pgfqpoint{3.176613in}{2.985279in}}{\pgfqpoint{3.182437in}{2.979455in}}%
\pgfpathcurveto{\pgfqpoint{3.188261in}{2.973631in}}{\pgfqpoint{3.196161in}{2.970358in}}{\pgfqpoint{3.204397in}{2.970358in}}%
\pgfpathclose%
\pgfusepath{stroke,fill}%
\end{pgfscope}%
\begin{pgfscope}%
\pgfpathrectangle{\pgfqpoint{0.100000in}{0.220728in}}{\pgfqpoint{3.696000in}{3.696000in}}%
\pgfusepath{clip}%
\pgfsetbuttcap%
\pgfsetroundjoin%
\definecolor{currentfill}{rgb}{0.121569,0.466667,0.705882}%
\pgfsetfillcolor{currentfill}%
\pgfsetfillopacity{0.630608}%
\pgfsetlinewidth{1.003750pt}%
\definecolor{currentstroke}{rgb}{0.121569,0.466667,0.705882}%
\pgfsetstrokecolor{currentstroke}%
\pgfsetstrokeopacity{0.630608}%
\pgfsetdash{}{0pt}%
\pgfpathmoveto{\pgfqpoint{3.211279in}{2.969322in}}%
\pgfpathcurveto{\pgfqpoint{3.219516in}{2.969322in}}{\pgfqpoint{3.227416in}{2.972595in}}{\pgfqpoint{3.233240in}{2.978418in}}%
\pgfpathcurveto{\pgfqpoint{3.239064in}{2.984242in}}{\pgfqpoint{3.242336in}{2.992142in}}{\pgfqpoint{3.242336in}{3.000379in}}%
\pgfpathcurveto{\pgfqpoint{3.242336in}{3.008615in}}{\pgfqpoint{3.239064in}{3.016515in}}{\pgfqpoint{3.233240in}{3.022339in}}%
\pgfpathcurveto{\pgfqpoint{3.227416in}{3.028163in}}{\pgfqpoint{3.219516in}{3.031435in}}{\pgfqpoint{3.211279in}{3.031435in}}%
\pgfpathcurveto{\pgfqpoint{3.203043in}{3.031435in}}{\pgfqpoint{3.195143in}{3.028163in}}{\pgfqpoint{3.189319in}{3.022339in}}%
\pgfpathcurveto{\pgfqpoint{3.183495in}{3.016515in}}{\pgfqpoint{3.180223in}{3.008615in}}{\pgfqpoint{3.180223in}{3.000379in}}%
\pgfpathcurveto{\pgfqpoint{3.180223in}{2.992142in}}{\pgfqpoint{3.183495in}{2.984242in}}{\pgfqpoint{3.189319in}{2.978418in}}%
\pgfpathcurveto{\pgfqpoint{3.195143in}{2.972595in}}{\pgfqpoint{3.203043in}{2.969322in}}{\pgfqpoint{3.211279in}{2.969322in}}%
\pgfpathclose%
\pgfusepath{stroke,fill}%
\end{pgfscope}%
\begin{pgfscope}%
\pgfpathrectangle{\pgfqpoint{0.100000in}{0.220728in}}{\pgfqpoint{3.696000in}{3.696000in}}%
\pgfusepath{clip}%
\pgfsetbuttcap%
\pgfsetroundjoin%
\definecolor{currentfill}{rgb}{0.121569,0.466667,0.705882}%
\pgfsetfillcolor{currentfill}%
\pgfsetfillopacity{0.631466}%
\pgfsetlinewidth{1.003750pt}%
\definecolor{currentstroke}{rgb}{0.121569,0.466667,0.705882}%
\pgfsetstrokecolor{currentstroke}%
\pgfsetstrokeopacity{0.631466}%
\pgfsetdash{}{0pt}%
\pgfpathmoveto{\pgfqpoint{3.219945in}{2.967290in}}%
\pgfpathcurveto{\pgfqpoint{3.228181in}{2.967290in}}{\pgfqpoint{3.236081in}{2.970562in}}{\pgfqpoint{3.241905in}{2.976386in}}%
\pgfpathcurveto{\pgfqpoint{3.247729in}{2.982210in}}{\pgfqpoint{3.251001in}{2.990110in}}{\pgfqpoint{3.251001in}{2.998346in}}%
\pgfpathcurveto{\pgfqpoint{3.251001in}{3.006582in}}{\pgfqpoint{3.247729in}{3.014482in}}{\pgfqpoint{3.241905in}{3.020306in}}%
\pgfpathcurveto{\pgfqpoint{3.236081in}{3.026130in}}{\pgfqpoint{3.228181in}{3.029403in}}{\pgfqpoint{3.219945in}{3.029403in}}%
\pgfpathcurveto{\pgfqpoint{3.211708in}{3.029403in}}{\pgfqpoint{3.203808in}{3.026130in}}{\pgfqpoint{3.197984in}{3.020306in}}%
\pgfpathcurveto{\pgfqpoint{3.192161in}{3.014482in}}{\pgfqpoint{3.188888in}{3.006582in}}{\pgfqpoint{3.188888in}{2.998346in}}%
\pgfpathcurveto{\pgfqpoint{3.188888in}{2.990110in}}{\pgfqpoint{3.192161in}{2.982210in}}{\pgfqpoint{3.197984in}{2.976386in}}%
\pgfpathcurveto{\pgfqpoint{3.203808in}{2.970562in}}{\pgfqpoint{3.211708in}{2.967290in}}{\pgfqpoint{3.219945in}{2.967290in}}%
\pgfpathclose%
\pgfusepath{stroke,fill}%
\end{pgfscope}%
\begin{pgfscope}%
\pgfpathrectangle{\pgfqpoint{0.100000in}{0.220728in}}{\pgfqpoint{3.696000in}{3.696000in}}%
\pgfusepath{clip}%
\pgfsetbuttcap%
\pgfsetroundjoin%
\definecolor{currentfill}{rgb}{0.121569,0.466667,0.705882}%
\pgfsetfillcolor{currentfill}%
\pgfsetfillopacity{0.634207}%
\pgfsetlinewidth{1.003750pt}%
\definecolor{currentstroke}{rgb}{0.121569,0.466667,0.705882}%
\pgfsetstrokecolor{currentstroke}%
\pgfsetstrokeopacity{0.634207}%
\pgfsetdash{}{0pt}%
\pgfpathmoveto{\pgfqpoint{3.231114in}{2.965999in}}%
\pgfpathcurveto{\pgfqpoint{3.239350in}{2.965999in}}{\pgfqpoint{3.247250in}{2.969271in}}{\pgfqpoint{3.253074in}{2.975095in}}%
\pgfpathcurveto{\pgfqpoint{3.258898in}{2.980919in}}{\pgfqpoint{3.262170in}{2.988819in}}{\pgfqpoint{3.262170in}{2.997055in}}%
\pgfpathcurveto{\pgfqpoint{3.262170in}{3.005292in}}{\pgfqpoint{3.258898in}{3.013192in}}{\pgfqpoint{3.253074in}{3.019016in}}%
\pgfpathcurveto{\pgfqpoint{3.247250in}{3.024840in}}{\pgfqpoint{3.239350in}{3.028112in}}{\pgfqpoint{3.231114in}{3.028112in}}%
\pgfpathcurveto{\pgfqpoint{3.222878in}{3.028112in}}{\pgfqpoint{3.214977in}{3.024840in}}{\pgfqpoint{3.209154in}{3.019016in}}%
\pgfpathcurveto{\pgfqpoint{3.203330in}{3.013192in}}{\pgfqpoint{3.200057in}{3.005292in}}{\pgfqpoint{3.200057in}{2.997055in}}%
\pgfpathcurveto{\pgfqpoint{3.200057in}{2.988819in}}{\pgfqpoint{3.203330in}{2.980919in}}{\pgfqpoint{3.209154in}{2.975095in}}%
\pgfpathcurveto{\pgfqpoint{3.214977in}{2.969271in}}{\pgfqpoint{3.222878in}{2.965999in}}{\pgfqpoint{3.231114in}{2.965999in}}%
\pgfpathclose%
\pgfusepath{stroke,fill}%
\end{pgfscope}%
\begin{pgfscope}%
\pgfpathrectangle{\pgfqpoint{0.100000in}{0.220728in}}{\pgfqpoint{3.696000in}{3.696000in}}%
\pgfusepath{clip}%
\pgfsetbuttcap%
\pgfsetroundjoin%
\definecolor{currentfill}{rgb}{0.121569,0.466667,0.705882}%
\pgfsetfillcolor{currentfill}%
\pgfsetfillopacity{0.635547}%
\pgfsetlinewidth{1.003750pt}%
\definecolor{currentstroke}{rgb}{0.121569,0.466667,0.705882}%
\pgfsetstrokecolor{currentstroke}%
\pgfsetstrokeopacity{0.635547}%
\pgfsetdash{}{0pt}%
\pgfpathmoveto{\pgfqpoint{3.243553in}{2.962014in}}%
\pgfpathcurveto{\pgfqpoint{3.251789in}{2.962014in}}{\pgfqpoint{3.259689in}{2.965287in}}{\pgfqpoint{3.265513in}{2.971111in}}%
\pgfpathcurveto{\pgfqpoint{3.271337in}{2.976934in}}{\pgfqpoint{3.274609in}{2.984835in}}{\pgfqpoint{3.274609in}{2.993071in}}%
\pgfpathcurveto{\pgfqpoint{3.274609in}{3.001307in}}{\pgfqpoint{3.271337in}{3.009207in}}{\pgfqpoint{3.265513in}{3.015031in}}%
\pgfpathcurveto{\pgfqpoint{3.259689in}{3.020855in}}{\pgfqpoint{3.251789in}{3.024127in}}{\pgfqpoint{3.243553in}{3.024127in}}%
\pgfpathcurveto{\pgfqpoint{3.235316in}{3.024127in}}{\pgfqpoint{3.227416in}{3.020855in}}{\pgfqpoint{3.221593in}{3.015031in}}%
\pgfpathcurveto{\pgfqpoint{3.215769in}{3.009207in}}{\pgfqpoint{3.212496in}{3.001307in}}{\pgfqpoint{3.212496in}{2.993071in}}%
\pgfpathcurveto{\pgfqpoint{3.212496in}{2.984835in}}{\pgfqpoint{3.215769in}{2.976934in}}{\pgfqpoint{3.221593in}{2.971111in}}%
\pgfpathcurveto{\pgfqpoint{3.227416in}{2.965287in}}{\pgfqpoint{3.235316in}{2.962014in}}{\pgfqpoint{3.243553in}{2.962014in}}%
\pgfpathclose%
\pgfusepath{stroke,fill}%
\end{pgfscope}%
\begin{pgfscope}%
\pgfpathrectangle{\pgfqpoint{0.100000in}{0.220728in}}{\pgfqpoint{3.696000in}{3.696000in}}%
\pgfusepath{clip}%
\pgfsetbuttcap%
\pgfsetroundjoin%
\definecolor{currentfill}{rgb}{0.121569,0.466667,0.705882}%
\pgfsetfillcolor{currentfill}%
\pgfsetfillopacity{0.637182}%
\pgfsetlinewidth{1.003750pt}%
\definecolor{currentstroke}{rgb}{0.121569,0.466667,0.705882}%
\pgfsetstrokecolor{currentstroke}%
\pgfsetstrokeopacity{0.637182}%
\pgfsetdash{}{0pt}%
\pgfpathmoveto{\pgfqpoint{3.249056in}{2.958910in}}%
\pgfpathcurveto{\pgfqpoint{3.257292in}{2.958910in}}{\pgfqpoint{3.265192in}{2.962182in}}{\pgfqpoint{3.271016in}{2.968006in}}%
\pgfpathcurveto{\pgfqpoint{3.276840in}{2.973830in}}{\pgfqpoint{3.280112in}{2.981730in}}{\pgfqpoint{3.280112in}{2.989967in}}%
\pgfpathcurveto{\pgfqpoint{3.280112in}{2.998203in}}{\pgfqpoint{3.276840in}{3.006103in}}{\pgfqpoint{3.271016in}{3.011927in}}%
\pgfpathcurveto{\pgfqpoint{3.265192in}{3.017751in}}{\pgfqpoint{3.257292in}{3.021023in}}{\pgfqpoint{3.249056in}{3.021023in}}%
\pgfpathcurveto{\pgfqpoint{3.240819in}{3.021023in}}{\pgfqpoint{3.232919in}{3.017751in}}{\pgfqpoint{3.227095in}{3.011927in}}%
\pgfpathcurveto{\pgfqpoint{3.221271in}{3.006103in}}{\pgfqpoint{3.217999in}{2.998203in}}{\pgfqpoint{3.217999in}{2.989967in}}%
\pgfpathcurveto{\pgfqpoint{3.217999in}{2.981730in}}{\pgfqpoint{3.221271in}{2.973830in}}{\pgfqpoint{3.227095in}{2.968006in}}%
\pgfpathcurveto{\pgfqpoint{3.232919in}{2.962182in}}{\pgfqpoint{3.240819in}{2.958910in}}{\pgfqpoint{3.249056in}{2.958910in}}%
\pgfpathclose%
\pgfusepath{stroke,fill}%
\end{pgfscope}%
\begin{pgfscope}%
\pgfpathrectangle{\pgfqpoint{0.100000in}{0.220728in}}{\pgfqpoint{3.696000in}{3.696000in}}%
\pgfusepath{clip}%
\pgfsetbuttcap%
\pgfsetroundjoin%
\definecolor{currentfill}{rgb}{0.121569,0.466667,0.705882}%
\pgfsetfillcolor{currentfill}%
\pgfsetfillopacity{0.638421}%
\pgfsetlinewidth{1.003750pt}%
\definecolor{currentstroke}{rgb}{0.121569,0.466667,0.705882}%
\pgfsetstrokecolor{currentstroke}%
\pgfsetstrokeopacity{0.638421}%
\pgfsetdash{}{0pt}%
\pgfpathmoveto{\pgfqpoint{3.257237in}{2.958322in}}%
\pgfpathcurveto{\pgfqpoint{3.265473in}{2.958322in}}{\pgfqpoint{3.273373in}{2.961595in}}{\pgfqpoint{3.279197in}{2.967419in}}%
\pgfpathcurveto{\pgfqpoint{3.285021in}{2.973242in}}{\pgfqpoint{3.288294in}{2.981143in}}{\pgfqpoint{3.288294in}{2.989379in}}%
\pgfpathcurveto{\pgfqpoint{3.288294in}{2.997615in}}{\pgfqpoint{3.285021in}{3.005515in}}{\pgfqpoint{3.279197in}{3.011339in}}%
\pgfpathcurveto{\pgfqpoint{3.273373in}{3.017163in}}{\pgfqpoint{3.265473in}{3.020435in}}{\pgfqpoint{3.257237in}{3.020435in}}%
\pgfpathcurveto{\pgfqpoint{3.249001in}{3.020435in}}{\pgfqpoint{3.241101in}{3.017163in}}{\pgfqpoint{3.235277in}{3.011339in}}%
\pgfpathcurveto{\pgfqpoint{3.229453in}{3.005515in}}{\pgfqpoint{3.226181in}{2.997615in}}{\pgfqpoint{3.226181in}{2.989379in}}%
\pgfpathcurveto{\pgfqpoint{3.226181in}{2.981143in}}{\pgfqpoint{3.229453in}{2.973242in}}{\pgfqpoint{3.235277in}{2.967419in}}%
\pgfpathcurveto{\pgfqpoint{3.241101in}{2.961595in}}{\pgfqpoint{3.249001in}{2.958322in}}{\pgfqpoint{3.257237in}{2.958322in}}%
\pgfpathclose%
\pgfusepath{stroke,fill}%
\end{pgfscope}%
\begin{pgfscope}%
\pgfpathrectangle{\pgfqpoint{0.100000in}{0.220728in}}{\pgfqpoint{3.696000in}{3.696000in}}%
\pgfusepath{clip}%
\pgfsetbuttcap%
\pgfsetroundjoin%
\definecolor{currentfill}{rgb}{0.121569,0.466667,0.705882}%
\pgfsetfillcolor{currentfill}%
\pgfsetfillopacity{0.639143}%
\pgfsetlinewidth{1.003750pt}%
\definecolor{currentstroke}{rgb}{0.121569,0.466667,0.705882}%
\pgfsetstrokecolor{currentstroke}%
\pgfsetstrokeopacity{0.639143}%
\pgfsetdash{}{0pt}%
\pgfpathmoveto{\pgfqpoint{3.267955in}{2.958313in}}%
\pgfpathcurveto{\pgfqpoint{3.276191in}{2.958313in}}{\pgfqpoint{3.284091in}{2.961585in}}{\pgfqpoint{3.289915in}{2.967409in}}%
\pgfpathcurveto{\pgfqpoint{3.295739in}{2.973233in}}{\pgfqpoint{3.299011in}{2.981133in}}{\pgfqpoint{3.299011in}{2.989370in}}%
\pgfpathcurveto{\pgfqpoint{3.299011in}{2.997606in}}{\pgfqpoint{3.295739in}{3.005506in}}{\pgfqpoint{3.289915in}{3.011330in}}%
\pgfpathcurveto{\pgfqpoint{3.284091in}{3.017154in}}{\pgfqpoint{3.276191in}{3.020426in}}{\pgfqpoint{3.267955in}{3.020426in}}%
\pgfpathcurveto{\pgfqpoint{3.259718in}{3.020426in}}{\pgfqpoint{3.251818in}{3.017154in}}{\pgfqpoint{3.245994in}{3.011330in}}%
\pgfpathcurveto{\pgfqpoint{3.240170in}{3.005506in}}{\pgfqpoint{3.236898in}{2.997606in}}{\pgfqpoint{3.236898in}{2.989370in}}%
\pgfpathcurveto{\pgfqpoint{3.236898in}{2.981133in}}{\pgfqpoint{3.240170in}{2.973233in}}{\pgfqpoint{3.245994in}{2.967409in}}%
\pgfpathcurveto{\pgfqpoint{3.251818in}{2.961585in}}{\pgfqpoint{3.259718in}{2.958313in}}{\pgfqpoint{3.267955in}{2.958313in}}%
\pgfpathclose%
\pgfusepath{stroke,fill}%
\end{pgfscope}%
\begin{pgfscope}%
\pgfpathrectangle{\pgfqpoint{0.100000in}{0.220728in}}{\pgfqpoint{3.696000in}{3.696000in}}%
\pgfusepath{clip}%
\pgfsetbuttcap%
\pgfsetroundjoin%
\definecolor{currentfill}{rgb}{0.121569,0.466667,0.705882}%
\pgfsetfillcolor{currentfill}%
\pgfsetfillopacity{0.641564}%
\pgfsetlinewidth{1.003750pt}%
\definecolor{currentstroke}{rgb}{0.121569,0.466667,0.705882}%
\pgfsetstrokecolor{currentstroke}%
\pgfsetstrokeopacity{0.641564}%
\pgfsetdash{}{0pt}%
\pgfpathmoveto{\pgfqpoint{3.278795in}{2.957575in}}%
\pgfpathcurveto{\pgfqpoint{3.287031in}{2.957575in}}{\pgfqpoint{3.294931in}{2.960848in}}{\pgfqpoint{3.300755in}{2.966671in}}%
\pgfpathcurveto{\pgfqpoint{3.306579in}{2.972495in}}{\pgfqpoint{3.309851in}{2.980395in}}{\pgfqpoint{3.309851in}{2.988632in}}%
\pgfpathcurveto{\pgfqpoint{3.309851in}{2.996868in}}{\pgfqpoint{3.306579in}{3.004768in}}{\pgfqpoint{3.300755in}{3.010592in}}%
\pgfpathcurveto{\pgfqpoint{3.294931in}{3.016416in}}{\pgfqpoint{3.287031in}{3.019688in}}{\pgfqpoint{3.278795in}{3.019688in}}%
\pgfpathcurveto{\pgfqpoint{3.270558in}{3.019688in}}{\pgfqpoint{3.262658in}{3.016416in}}{\pgfqpoint{3.256834in}{3.010592in}}%
\pgfpathcurveto{\pgfqpoint{3.251010in}{3.004768in}}{\pgfqpoint{3.247738in}{2.996868in}}{\pgfqpoint{3.247738in}{2.988632in}}%
\pgfpathcurveto{\pgfqpoint{3.247738in}{2.980395in}}{\pgfqpoint{3.251010in}{2.972495in}}{\pgfqpoint{3.256834in}{2.966671in}}%
\pgfpathcurveto{\pgfqpoint{3.262658in}{2.960848in}}{\pgfqpoint{3.270558in}{2.957575in}}{\pgfqpoint{3.278795in}{2.957575in}}%
\pgfpathclose%
\pgfusepath{stroke,fill}%
\end{pgfscope}%
\begin{pgfscope}%
\pgfpathrectangle{\pgfqpoint{0.100000in}{0.220728in}}{\pgfqpoint{3.696000in}{3.696000in}}%
\pgfusepath{clip}%
\pgfsetbuttcap%
\pgfsetroundjoin%
\definecolor{currentfill}{rgb}{0.121569,0.466667,0.705882}%
\pgfsetfillcolor{currentfill}%
\pgfsetfillopacity{0.644500}%
\pgfsetlinewidth{1.003750pt}%
\definecolor{currentstroke}{rgb}{0.121569,0.466667,0.705882}%
\pgfsetstrokecolor{currentstroke}%
\pgfsetstrokeopacity{0.644500}%
\pgfsetdash{}{0pt}%
\pgfpathmoveto{\pgfqpoint{3.290112in}{2.956452in}}%
\pgfpathcurveto{\pgfqpoint{3.298348in}{2.956452in}}{\pgfqpoint{3.306248in}{2.959725in}}{\pgfqpoint{3.312072in}{2.965548in}}%
\pgfpathcurveto{\pgfqpoint{3.317896in}{2.971372in}}{\pgfqpoint{3.321168in}{2.979272in}}{\pgfqpoint{3.321168in}{2.987509in}}%
\pgfpathcurveto{\pgfqpoint{3.321168in}{2.995745in}}{\pgfqpoint{3.317896in}{3.003645in}}{\pgfqpoint{3.312072in}{3.009469in}}%
\pgfpathcurveto{\pgfqpoint{3.306248in}{3.015293in}}{\pgfqpoint{3.298348in}{3.018565in}}{\pgfqpoint{3.290112in}{3.018565in}}%
\pgfpathcurveto{\pgfqpoint{3.281875in}{3.018565in}}{\pgfqpoint{3.273975in}{3.015293in}}{\pgfqpoint{3.268151in}{3.009469in}}%
\pgfpathcurveto{\pgfqpoint{3.262327in}{3.003645in}}{\pgfqpoint{3.259055in}{2.995745in}}{\pgfqpoint{3.259055in}{2.987509in}}%
\pgfpathcurveto{\pgfqpoint{3.259055in}{2.979272in}}{\pgfqpoint{3.262327in}{2.971372in}}{\pgfqpoint{3.268151in}{2.965548in}}%
\pgfpathcurveto{\pgfqpoint{3.273975in}{2.959725in}}{\pgfqpoint{3.281875in}{2.956452in}}{\pgfqpoint{3.290112in}{2.956452in}}%
\pgfpathclose%
\pgfusepath{stroke,fill}%
\end{pgfscope}%
\begin{pgfscope}%
\pgfpathrectangle{\pgfqpoint{0.100000in}{0.220728in}}{\pgfqpoint{3.696000in}{3.696000in}}%
\pgfusepath{clip}%
\pgfsetbuttcap%
\pgfsetroundjoin%
\definecolor{currentfill}{rgb}{0.121569,0.466667,0.705882}%
\pgfsetfillcolor{currentfill}%
\pgfsetfillopacity{0.650908}%
\pgfsetlinewidth{1.003750pt}%
\definecolor{currentstroke}{rgb}{0.121569,0.466667,0.705882}%
\pgfsetstrokecolor{currentstroke}%
\pgfsetstrokeopacity{0.650908}%
\pgfsetdash{}{0pt}%
\pgfpathmoveto{\pgfqpoint{3.286572in}{2.955810in}}%
\pgfpathcurveto{\pgfqpoint{3.294808in}{2.955810in}}{\pgfqpoint{3.302708in}{2.959083in}}{\pgfqpoint{3.308532in}{2.964907in}}%
\pgfpathcurveto{\pgfqpoint{3.314356in}{2.970731in}}{\pgfqpoint{3.317629in}{2.978631in}}{\pgfqpoint{3.317629in}{2.986867in}}%
\pgfpathcurveto{\pgfqpoint{3.317629in}{2.995103in}}{\pgfqpoint{3.314356in}{3.003003in}}{\pgfqpoint{3.308532in}{3.008827in}}%
\pgfpathcurveto{\pgfqpoint{3.302708in}{3.014651in}}{\pgfqpoint{3.294808in}{3.017923in}}{\pgfqpoint{3.286572in}{3.017923in}}%
\pgfpathcurveto{\pgfqpoint{3.278336in}{3.017923in}}{\pgfqpoint{3.270436in}{3.014651in}}{\pgfqpoint{3.264612in}{3.008827in}}%
\pgfpathcurveto{\pgfqpoint{3.258788in}{3.003003in}}{\pgfqpoint{3.255516in}{2.995103in}}{\pgfqpoint{3.255516in}{2.986867in}}%
\pgfpathcurveto{\pgfqpoint{3.255516in}{2.978631in}}{\pgfqpoint{3.258788in}{2.970731in}}{\pgfqpoint{3.264612in}{2.964907in}}%
\pgfpathcurveto{\pgfqpoint{3.270436in}{2.959083in}}{\pgfqpoint{3.278336in}{2.955810in}}{\pgfqpoint{3.286572in}{2.955810in}}%
\pgfpathclose%
\pgfusepath{stroke,fill}%
\end{pgfscope}%
\begin{pgfscope}%
\pgfpathrectangle{\pgfqpoint{0.100000in}{0.220728in}}{\pgfqpoint{3.696000in}{3.696000in}}%
\pgfusepath{clip}%
\pgfsetbuttcap%
\pgfsetroundjoin%
\definecolor{currentfill}{rgb}{0.121569,0.466667,0.705882}%
\pgfsetfillcolor{currentfill}%
\pgfsetfillopacity{0.651477}%
\pgfsetlinewidth{1.003750pt}%
\definecolor{currentstroke}{rgb}{0.121569,0.466667,0.705882}%
\pgfsetstrokecolor{currentstroke}%
\pgfsetstrokeopacity{0.651477}%
\pgfsetdash{}{0pt}%
\pgfpathmoveto{\pgfqpoint{0.592749in}{1.458234in}}%
\pgfpathcurveto{\pgfqpoint{0.600985in}{1.458234in}}{\pgfqpoint{0.608886in}{1.461506in}}{\pgfqpoint{0.614709in}{1.467330in}}%
\pgfpathcurveto{\pgfqpoint{0.620533in}{1.473154in}}{\pgfqpoint{0.623806in}{1.481054in}}{\pgfqpoint{0.623806in}{1.489290in}}%
\pgfpathcurveto{\pgfqpoint{0.623806in}{1.497526in}}{\pgfqpoint{0.620533in}{1.505426in}}{\pgfqpoint{0.614709in}{1.511250in}}%
\pgfpathcurveto{\pgfqpoint{0.608886in}{1.517074in}}{\pgfqpoint{0.600985in}{1.520347in}}{\pgfqpoint{0.592749in}{1.520347in}}%
\pgfpathcurveto{\pgfqpoint{0.584513in}{1.520347in}}{\pgfqpoint{0.576613in}{1.517074in}}{\pgfqpoint{0.570789in}{1.511250in}}%
\pgfpathcurveto{\pgfqpoint{0.564965in}{1.505426in}}{\pgfqpoint{0.561693in}{1.497526in}}{\pgfqpoint{0.561693in}{1.489290in}}%
\pgfpathcurveto{\pgfqpoint{0.561693in}{1.481054in}}{\pgfqpoint{0.564965in}{1.473154in}}{\pgfqpoint{0.570789in}{1.467330in}}%
\pgfpathcurveto{\pgfqpoint{0.576613in}{1.461506in}}{\pgfqpoint{0.584513in}{1.458234in}}{\pgfqpoint{0.592749in}{1.458234in}}%
\pgfpathclose%
\pgfusepath{stroke,fill}%
\end{pgfscope}%
\begin{pgfscope}%
\pgfpathrectangle{\pgfqpoint{0.100000in}{0.220728in}}{\pgfqpoint{3.696000in}{3.696000in}}%
\pgfusepath{clip}%
\pgfsetbuttcap%
\pgfsetroundjoin%
\definecolor{currentfill}{rgb}{0.121569,0.466667,0.705882}%
\pgfsetfillcolor{currentfill}%
\pgfsetfillopacity{0.652157}%
\pgfsetlinewidth{1.003750pt}%
\definecolor{currentstroke}{rgb}{0.121569,0.466667,0.705882}%
\pgfsetstrokecolor{currentstroke}%
\pgfsetstrokeopacity{0.652157}%
\pgfsetdash{}{0pt}%
\pgfpathmoveto{\pgfqpoint{0.596003in}{1.459896in}}%
\pgfpathcurveto{\pgfqpoint{0.604240in}{1.459896in}}{\pgfqpoint{0.612140in}{1.463169in}}{\pgfqpoint{0.617964in}{1.468993in}}%
\pgfpathcurveto{\pgfqpoint{0.623787in}{1.474817in}}{\pgfqpoint{0.627060in}{1.482717in}}{\pgfqpoint{0.627060in}{1.490953in}}%
\pgfpathcurveto{\pgfqpoint{0.627060in}{1.499189in}}{\pgfqpoint{0.623787in}{1.507089in}}{\pgfqpoint{0.617964in}{1.512913in}}%
\pgfpathcurveto{\pgfqpoint{0.612140in}{1.518737in}}{\pgfqpoint{0.604240in}{1.522009in}}{\pgfqpoint{0.596003in}{1.522009in}}%
\pgfpathcurveto{\pgfqpoint{0.587767in}{1.522009in}}{\pgfqpoint{0.579867in}{1.518737in}}{\pgfqpoint{0.574043in}{1.512913in}}%
\pgfpathcurveto{\pgfqpoint{0.568219in}{1.507089in}}{\pgfqpoint{0.564947in}{1.499189in}}{\pgfqpoint{0.564947in}{1.490953in}}%
\pgfpathcurveto{\pgfqpoint{0.564947in}{1.482717in}}{\pgfqpoint{0.568219in}{1.474817in}}{\pgfqpoint{0.574043in}{1.468993in}}%
\pgfpathcurveto{\pgfqpoint{0.579867in}{1.463169in}}{\pgfqpoint{0.587767in}{1.459896in}}{\pgfqpoint{0.596003in}{1.459896in}}%
\pgfpathclose%
\pgfusepath{stroke,fill}%
\end{pgfscope}%
\begin{pgfscope}%
\pgfpathrectangle{\pgfqpoint{0.100000in}{0.220728in}}{\pgfqpoint{3.696000in}{3.696000in}}%
\pgfusepath{clip}%
\pgfsetbuttcap%
\pgfsetroundjoin%
\definecolor{currentfill}{rgb}{0.121569,0.466667,0.705882}%
\pgfsetfillcolor{currentfill}%
\pgfsetfillopacity{0.652814}%
\pgfsetlinewidth{1.003750pt}%
\definecolor{currentstroke}{rgb}{0.121569,0.466667,0.705882}%
\pgfsetstrokecolor{currentstroke}%
\pgfsetstrokeopacity{0.652814}%
\pgfsetdash{}{0pt}%
\pgfpathmoveto{\pgfqpoint{0.598648in}{1.461341in}}%
\pgfpathcurveto{\pgfqpoint{0.606885in}{1.461341in}}{\pgfqpoint{0.614785in}{1.464614in}}{\pgfqpoint{0.620609in}{1.470438in}}%
\pgfpathcurveto{\pgfqpoint{0.626432in}{1.476261in}}{\pgfqpoint{0.629705in}{1.484162in}}{\pgfqpoint{0.629705in}{1.492398in}}%
\pgfpathcurveto{\pgfqpoint{0.629705in}{1.500634in}}{\pgfqpoint{0.626432in}{1.508534in}}{\pgfqpoint{0.620609in}{1.514358in}}%
\pgfpathcurveto{\pgfqpoint{0.614785in}{1.520182in}}{\pgfqpoint{0.606885in}{1.523454in}}{\pgfqpoint{0.598648in}{1.523454in}}%
\pgfpathcurveto{\pgfqpoint{0.590412in}{1.523454in}}{\pgfqpoint{0.582512in}{1.520182in}}{\pgfqpoint{0.576688in}{1.514358in}}%
\pgfpathcurveto{\pgfqpoint{0.570864in}{1.508534in}}{\pgfqpoint{0.567592in}{1.500634in}}{\pgfqpoint{0.567592in}{1.492398in}}%
\pgfpathcurveto{\pgfqpoint{0.567592in}{1.484162in}}{\pgfqpoint{0.570864in}{1.476261in}}{\pgfqpoint{0.576688in}{1.470438in}}%
\pgfpathcurveto{\pgfqpoint{0.582512in}{1.464614in}}{\pgfqpoint{0.590412in}{1.461341in}}{\pgfqpoint{0.598648in}{1.461341in}}%
\pgfpathclose%
\pgfusepath{stroke,fill}%
\end{pgfscope}%
\begin{pgfscope}%
\pgfpathrectangle{\pgfqpoint{0.100000in}{0.220728in}}{\pgfqpoint{3.696000in}{3.696000in}}%
\pgfusepath{clip}%
\pgfsetbuttcap%
\pgfsetroundjoin%
\definecolor{currentfill}{rgb}{0.121569,0.466667,0.705882}%
\pgfsetfillcolor{currentfill}%
\pgfsetfillopacity{0.653733}%
\pgfsetlinewidth{1.003750pt}%
\definecolor{currentstroke}{rgb}{0.121569,0.466667,0.705882}%
\pgfsetstrokecolor{currentstroke}%
\pgfsetstrokeopacity{0.653733}%
\pgfsetdash{}{0pt}%
\pgfpathmoveto{\pgfqpoint{3.299959in}{2.954873in}}%
\pgfpathcurveto{\pgfqpoint{3.308195in}{2.954873in}}{\pgfqpoint{3.316095in}{2.958146in}}{\pgfqpoint{3.321919in}{2.963969in}}%
\pgfpathcurveto{\pgfqpoint{3.327743in}{2.969793in}}{\pgfqpoint{3.331015in}{2.977693in}}{\pgfqpoint{3.331015in}{2.985930in}}%
\pgfpathcurveto{\pgfqpoint{3.331015in}{2.994166in}}{\pgfqpoint{3.327743in}{3.002066in}}{\pgfqpoint{3.321919in}{3.007890in}}%
\pgfpathcurveto{\pgfqpoint{3.316095in}{3.013714in}}{\pgfqpoint{3.308195in}{3.016986in}}{\pgfqpoint{3.299959in}{3.016986in}}%
\pgfpathcurveto{\pgfqpoint{3.291723in}{3.016986in}}{\pgfqpoint{3.283823in}{3.013714in}}{\pgfqpoint{3.277999in}{3.007890in}}%
\pgfpathcurveto{\pgfqpoint{3.272175in}{3.002066in}}{\pgfqpoint{3.268902in}{2.994166in}}{\pgfqpoint{3.268902in}{2.985930in}}%
\pgfpathcurveto{\pgfqpoint{3.268902in}{2.977693in}}{\pgfqpoint{3.272175in}{2.969793in}}{\pgfqpoint{3.277999in}{2.963969in}}%
\pgfpathcurveto{\pgfqpoint{3.283823in}{2.958146in}}{\pgfqpoint{3.291723in}{2.954873in}}{\pgfqpoint{3.299959in}{2.954873in}}%
\pgfpathclose%
\pgfusepath{stroke,fill}%
\end{pgfscope}%
\begin{pgfscope}%
\pgfpathrectangle{\pgfqpoint{0.100000in}{0.220728in}}{\pgfqpoint{3.696000in}{3.696000in}}%
\pgfusepath{clip}%
\pgfsetbuttcap%
\pgfsetroundjoin%
\definecolor{currentfill}{rgb}{0.121569,0.466667,0.705882}%
\pgfsetfillcolor{currentfill}%
\pgfsetfillopacity{0.655168}%
\pgfsetlinewidth{1.003750pt}%
\definecolor{currentstroke}{rgb}{0.121569,0.466667,0.705882}%
\pgfsetstrokecolor{currentstroke}%
\pgfsetstrokeopacity{0.655168}%
\pgfsetdash{}{0pt}%
\pgfpathmoveto{\pgfqpoint{0.602904in}{1.466665in}}%
\pgfpathcurveto{\pgfqpoint{0.611141in}{1.466665in}}{\pgfqpoint{0.619041in}{1.469937in}}{\pgfqpoint{0.624865in}{1.475761in}}%
\pgfpathcurveto{\pgfqpoint{0.630689in}{1.481585in}}{\pgfqpoint{0.633961in}{1.489485in}}{\pgfqpoint{0.633961in}{1.497721in}}%
\pgfpathcurveto{\pgfqpoint{0.633961in}{1.505957in}}{\pgfqpoint{0.630689in}{1.513857in}}{\pgfqpoint{0.624865in}{1.519681in}}%
\pgfpathcurveto{\pgfqpoint{0.619041in}{1.525505in}}{\pgfqpoint{0.611141in}{1.528778in}}{\pgfqpoint{0.602904in}{1.528778in}}%
\pgfpathcurveto{\pgfqpoint{0.594668in}{1.528778in}}{\pgfqpoint{0.586768in}{1.525505in}}{\pgfqpoint{0.580944in}{1.519681in}}%
\pgfpathcurveto{\pgfqpoint{0.575120in}{1.513857in}}{\pgfqpoint{0.571848in}{1.505957in}}{\pgfqpoint{0.571848in}{1.497721in}}%
\pgfpathcurveto{\pgfqpoint{0.571848in}{1.489485in}}{\pgfqpoint{0.575120in}{1.481585in}}{\pgfqpoint{0.580944in}{1.475761in}}%
\pgfpathcurveto{\pgfqpoint{0.586768in}{1.469937in}}{\pgfqpoint{0.594668in}{1.466665in}}{\pgfqpoint{0.602904in}{1.466665in}}%
\pgfpathclose%
\pgfusepath{stroke,fill}%
\end{pgfscope}%
\begin{pgfscope}%
\pgfpathrectangle{\pgfqpoint{0.100000in}{0.220728in}}{\pgfqpoint{3.696000in}{3.696000in}}%
\pgfusepath{clip}%
\pgfsetbuttcap%
\pgfsetroundjoin%
\definecolor{currentfill}{rgb}{0.121569,0.466667,0.705882}%
\pgfsetfillcolor{currentfill}%
\pgfsetfillopacity{0.655329}%
\pgfsetlinewidth{1.003750pt}%
\definecolor{currentstroke}{rgb}{0.121569,0.466667,0.705882}%
\pgfsetstrokecolor{currentstroke}%
\pgfsetstrokeopacity{0.655329}%
\pgfsetdash{}{0pt}%
\pgfpathmoveto{\pgfqpoint{3.331233in}{2.953830in}}%
\pgfpathcurveto{\pgfqpoint{3.339469in}{2.953830in}}{\pgfqpoint{3.347369in}{2.957102in}}{\pgfqpoint{3.353193in}{2.962926in}}%
\pgfpathcurveto{\pgfqpoint{3.359017in}{2.968750in}}{\pgfqpoint{3.362289in}{2.976650in}}{\pgfqpoint{3.362289in}{2.984886in}}%
\pgfpathcurveto{\pgfqpoint{3.362289in}{2.993123in}}{\pgfqpoint{3.359017in}{3.001023in}}{\pgfqpoint{3.353193in}{3.006847in}}%
\pgfpathcurveto{\pgfqpoint{3.347369in}{3.012671in}}{\pgfqpoint{3.339469in}{3.015943in}}{\pgfqpoint{3.331233in}{3.015943in}}%
\pgfpathcurveto{\pgfqpoint{3.322997in}{3.015943in}}{\pgfqpoint{3.315097in}{3.012671in}}{\pgfqpoint{3.309273in}{3.006847in}}%
\pgfpathcurveto{\pgfqpoint{3.303449in}{3.001023in}}{\pgfqpoint{3.300176in}{2.993123in}}{\pgfqpoint{3.300176in}{2.984886in}}%
\pgfpathcurveto{\pgfqpoint{3.300176in}{2.976650in}}{\pgfqpoint{3.303449in}{2.968750in}}{\pgfqpoint{3.309273in}{2.962926in}}%
\pgfpathcurveto{\pgfqpoint{3.315097in}{2.957102in}}{\pgfqpoint{3.322997in}{2.953830in}}{\pgfqpoint{3.331233in}{2.953830in}}%
\pgfpathclose%
\pgfusepath{stroke,fill}%
\end{pgfscope}%
\begin{pgfscope}%
\pgfpathrectangle{\pgfqpoint{0.100000in}{0.220728in}}{\pgfqpoint{3.696000in}{3.696000in}}%
\pgfusepath{clip}%
\pgfsetbuttcap%
\pgfsetroundjoin%
\definecolor{currentfill}{rgb}{0.121569,0.466667,0.705882}%
\pgfsetfillcolor{currentfill}%
\pgfsetfillopacity{0.656693}%
\pgfsetlinewidth{1.003750pt}%
\definecolor{currentstroke}{rgb}{0.121569,0.466667,0.705882}%
\pgfsetstrokecolor{currentstroke}%
\pgfsetstrokeopacity{0.656693}%
\pgfsetdash{}{0pt}%
\pgfpathmoveto{\pgfqpoint{3.314645in}{2.953323in}}%
\pgfpathcurveto{\pgfqpoint{3.322881in}{2.953323in}}{\pgfqpoint{3.330781in}{2.956595in}}{\pgfqpoint{3.336605in}{2.962419in}}%
\pgfpathcurveto{\pgfqpoint{3.342429in}{2.968243in}}{\pgfqpoint{3.345701in}{2.976143in}}{\pgfqpoint{3.345701in}{2.984379in}}%
\pgfpathcurveto{\pgfqpoint{3.345701in}{2.992616in}}{\pgfqpoint{3.342429in}{3.000516in}}{\pgfqpoint{3.336605in}{3.006340in}}%
\pgfpathcurveto{\pgfqpoint{3.330781in}{3.012164in}}{\pgfqpoint{3.322881in}{3.015436in}}{\pgfqpoint{3.314645in}{3.015436in}}%
\pgfpathcurveto{\pgfqpoint{3.306408in}{3.015436in}}{\pgfqpoint{3.298508in}{3.012164in}}{\pgfqpoint{3.292684in}{3.006340in}}%
\pgfpathcurveto{\pgfqpoint{3.286860in}{3.000516in}}{\pgfqpoint{3.283588in}{2.992616in}}{\pgfqpoint{3.283588in}{2.984379in}}%
\pgfpathcurveto{\pgfqpoint{3.283588in}{2.976143in}}{\pgfqpoint{3.286860in}{2.968243in}}{\pgfqpoint{3.292684in}{2.962419in}}%
\pgfpathcurveto{\pgfqpoint{3.298508in}{2.956595in}}{\pgfqpoint{3.306408in}{2.953323in}}{\pgfqpoint{3.314645in}{2.953323in}}%
\pgfpathclose%
\pgfusepath{stroke,fill}%
\end{pgfscope}%
\begin{pgfscope}%
\pgfpathrectangle{\pgfqpoint{0.100000in}{0.220728in}}{\pgfqpoint{3.696000in}{3.696000in}}%
\pgfusepath{clip}%
\pgfsetbuttcap%
\pgfsetroundjoin%
\definecolor{currentfill}{rgb}{0.121569,0.466667,0.705882}%
\pgfsetfillcolor{currentfill}%
\pgfsetfillopacity{0.656735}%
\pgfsetlinewidth{1.003750pt}%
\definecolor{currentstroke}{rgb}{0.121569,0.466667,0.705882}%
\pgfsetstrokecolor{currentstroke}%
\pgfsetstrokeopacity{0.656735}%
\pgfsetdash{}{0pt}%
\pgfpathmoveto{\pgfqpoint{0.606727in}{1.469309in}}%
\pgfpathcurveto{\pgfqpoint{0.614963in}{1.469309in}}{\pgfqpoint{0.622864in}{1.472582in}}{\pgfqpoint{0.628687in}{1.478406in}}%
\pgfpathcurveto{\pgfqpoint{0.634511in}{1.484229in}}{\pgfqpoint{0.637784in}{1.492130in}}{\pgfqpoint{0.637784in}{1.500366in}}%
\pgfpathcurveto{\pgfqpoint{0.637784in}{1.508602in}}{\pgfqpoint{0.634511in}{1.516502in}}{\pgfqpoint{0.628687in}{1.522326in}}%
\pgfpathcurveto{\pgfqpoint{0.622864in}{1.528150in}}{\pgfqpoint{0.614963in}{1.531422in}}{\pgfqpoint{0.606727in}{1.531422in}}%
\pgfpathcurveto{\pgfqpoint{0.598491in}{1.531422in}}{\pgfqpoint{0.590591in}{1.528150in}}{\pgfqpoint{0.584767in}{1.522326in}}%
\pgfpathcurveto{\pgfqpoint{0.578943in}{1.516502in}}{\pgfqpoint{0.575671in}{1.508602in}}{\pgfqpoint{0.575671in}{1.500366in}}%
\pgfpathcurveto{\pgfqpoint{0.575671in}{1.492130in}}{\pgfqpoint{0.578943in}{1.484229in}}{\pgfqpoint{0.584767in}{1.478406in}}%
\pgfpathcurveto{\pgfqpoint{0.590591in}{1.472582in}}{\pgfqpoint{0.598491in}{1.469309in}}{\pgfqpoint{0.606727in}{1.469309in}}%
\pgfpathclose%
\pgfusepath{stroke,fill}%
\end{pgfscope}%
\begin{pgfscope}%
\pgfpathrectangle{\pgfqpoint{0.100000in}{0.220728in}}{\pgfqpoint{3.696000in}{3.696000in}}%
\pgfusepath{clip}%
\pgfsetbuttcap%
\pgfsetroundjoin%
\definecolor{currentfill}{rgb}{0.121569,0.466667,0.705882}%
\pgfsetfillcolor{currentfill}%
\pgfsetfillopacity{0.658276}%
\pgfsetlinewidth{1.003750pt}%
\definecolor{currentstroke}{rgb}{0.121569,0.466667,0.705882}%
\pgfsetstrokecolor{currentstroke}%
\pgfsetstrokeopacity{0.658276}%
\pgfsetdash{}{0pt}%
\pgfpathmoveto{\pgfqpoint{0.614205in}{1.470305in}}%
\pgfpathcurveto{\pgfqpoint{0.622441in}{1.470305in}}{\pgfqpoint{0.630341in}{1.473578in}}{\pgfqpoint{0.636165in}{1.479402in}}%
\pgfpathcurveto{\pgfqpoint{0.641989in}{1.485226in}}{\pgfqpoint{0.645261in}{1.493126in}}{\pgfqpoint{0.645261in}{1.501362in}}%
\pgfpathcurveto{\pgfqpoint{0.645261in}{1.509598in}}{\pgfqpoint{0.641989in}{1.517498in}}{\pgfqpoint{0.636165in}{1.523322in}}%
\pgfpathcurveto{\pgfqpoint{0.630341in}{1.529146in}}{\pgfqpoint{0.622441in}{1.532418in}}{\pgfqpoint{0.614205in}{1.532418in}}%
\pgfpathcurveto{\pgfqpoint{0.605969in}{1.532418in}}{\pgfqpoint{0.598069in}{1.529146in}}{\pgfqpoint{0.592245in}{1.523322in}}%
\pgfpathcurveto{\pgfqpoint{0.586421in}{1.517498in}}{\pgfqpoint{0.583148in}{1.509598in}}{\pgfqpoint{0.583148in}{1.501362in}}%
\pgfpathcurveto{\pgfqpoint{0.583148in}{1.493126in}}{\pgfqpoint{0.586421in}{1.485226in}}{\pgfqpoint{0.592245in}{1.479402in}}%
\pgfpathcurveto{\pgfqpoint{0.598069in}{1.473578in}}{\pgfqpoint{0.605969in}{1.470305in}}{\pgfqpoint{0.614205in}{1.470305in}}%
\pgfpathclose%
\pgfusepath{stroke,fill}%
\end{pgfscope}%
\begin{pgfscope}%
\pgfpathrectangle{\pgfqpoint{0.100000in}{0.220728in}}{\pgfqpoint{3.696000in}{3.696000in}}%
\pgfusepath{clip}%
\pgfsetbuttcap%
\pgfsetroundjoin%
\definecolor{currentfill}{rgb}{0.121569,0.466667,0.705882}%
\pgfsetfillcolor{currentfill}%
\pgfsetfillopacity{0.658318}%
\pgfsetlinewidth{1.003750pt}%
\definecolor{currentstroke}{rgb}{0.121569,0.466667,0.705882}%
\pgfsetstrokecolor{currentstroke}%
\pgfsetstrokeopacity{0.658318}%
\pgfsetdash{}{0pt}%
\pgfpathmoveto{\pgfqpoint{0.624870in}{1.460544in}}%
\pgfpathcurveto{\pgfqpoint{0.633106in}{1.460544in}}{\pgfqpoint{0.641006in}{1.463816in}}{\pgfqpoint{0.646830in}{1.469640in}}%
\pgfpathcurveto{\pgfqpoint{0.652654in}{1.475464in}}{\pgfqpoint{0.655926in}{1.483364in}}{\pgfqpoint{0.655926in}{1.491601in}}%
\pgfpathcurveto{\pgfqpoint{0.655926in}{1.499837in}}{\pgfqpoint{0.652654in}{1.507737in}}{\pgfqpoint{0.646830in}{1.513561in}}%
\pgfpathcurveto{\pgfqpoint{0.641006in}{1.519385in}}{\pgfqpoint{0.633106in}{1.522657in}}{\pgfqpoint{0.624870in}{1.522657in}}%
\pgfpathcurveto{\pgfqpoint{0.616633in}{1.522657in}}{\pgfqpoint{0.608733in}{1.519385in}}{\pgfqpoint{0.602909in}{1.513561in}}%
\pgfpathcurveto{\pgfqpoint{0.597085in}{1.507737in}}{\pgfqpoint{0.593813in}{1.499837in}}{\pgfqpoint{0.593813in}{1.491601in}}%
\pgfpathcurveto{\pgfqpoint{0.593813in}{1.483364in}}{\pgfqpoint{0.597085in}{1.475464in}}{\pgfqpoint{0.602909in}{1.469640in}}%
\pgfpathcurveto{\pgfqpoint{0.608733in}{1.463816in}}{\pgfqpoint{0.616633in}{1.460544in}}{\pgfqpoint{0.624870in}{1.460544in}}%
\pgfpathclose%
\pgfusepath{stroke,fill}%
\end{pgfscope}%
\begin{pgfscope}%
\pgfpathrectangle{\pgfqpoint{0.100000in}{0.220728in}}{\pgfqpoint{3.696000in}{3.696000in}}%
\pgfusepath{clip}%
\pgfsetbuttcap%
\pgfsetroundjoin%
\definecolor{currentfill}{rgb}{0.121569,0.466667,0.705882}%
\pgfsetfillcolor{currentfill}%
\pgfsetfillopacity{0.658524}%
\pgfsetlinewidth{1.003750pt}%
\definecolor{currentstroke}{rgb}{0.121569,0.466667,0.705882}%
\pgfsetstrokecolor{currentstroke}%
\pgfsetstrokeopacity{0.658524}%
\pgfsetdash{}{0pt}%
\pgfpathmoveto{\pgfqpoint{0.626311in}{1.458672in}}%
\pgfpathcurveto{\pgfqpoint{0.634548in}{1.458672in}}{\pgfqpoint{0.642448in}{1.461944in}}{\pgfqpoint{0.648272in}{1.467768in}}%
\pgfpathcurveto{\pgfqpoint{0.654096in}{1.473592in}}{\pgfqpoint{0.657368in}{1.481492in}}{\pgfqpoint{0.657368in}{1.489728in}}%
\pgfpathcurveto{\pgfqpoint{0.657368in}{1.497965in}}{\pgfqpoint{0.654096in}{1.505865in}}{\pgfqpoint{0.648272in}{1.511689in}}%
\pgfpathcurveto{\pgfqpoint{0.642448in}{1.517512in}}{\pgfqpoint{0.634548in}{1.520785in}}{\pgfqpoint{0.626311in}{1.520785in}}%
\pgfpathcurveto{\pgfqpoint{0.618075in}{1.520785in}}{\pgfqpoint{0.610175in}{1.517512in}}{\pgfqpoint{0.604351in}{1.511689in}}%
\pgfpathcurveto{\pgfqpoint{0.598527in}{1.505865in}}{\pgfqpoint{0.595255in}{1.497965in}}{\pgfqpoint{0.595255in}{1.489728in}}%
\pgfpathcurveto{\pgfqpoint{0.595255in}{1.481492in}}{\pgfqpoint{0.598527in}{1.473592in}}{\pgfqpoint{0.604351in}{1.467768in}}%
\pgfpathcurveto{\pgfqpoint{0.610175in}{1.461944in}}{\pgfqpoint{0.618075in}{1.458672in}}{\pgfqpoint{0.626311in}{1.458672in}}%
\pgfpathclose%
\pgfusepath{stroke,fill}%
\end{pgfscope}%
\begin{pgfscope}%
\pgfpathrectangle{\pgfqpoint{0.100000in}{0.220728in}}{\pgfqpoint{3.696000in}{3.696000in}}%
\pgfusepath{clip}%
\pgfsetbuttcap%
\pgfsetroundjoin%
\definecolor{currentfill}{rgb}{0.121569,0.466667,0.705882}%
\pgfsetfillcolor{currentfill}%
\pgfsetfillopacity{0.658525}%
\pgfsetlinewidth{1.003750pt}%
\definecolor{currentstroke}{rgb}{0.121569,0.466667,0.705882}%
\pgfsetstrokecolor{currentstroke}%
\pgfsetstrokeopacity{0.658525}%
\pgfsetdash{}{0pt}%
\pgfpathmoveto{\pgfqpoint{3.347257in}{2.949768in}}%
\pgfpathcurveto{\pgfqpoint{3.355493in}{2.949768in}}{\pgfqpoint{3.363393in}{2.953040in}}{\pgfqpoint{3.369217in}{2.958864in}}%
\pgfpathcurveto{\pgfqpoint{3.375041in}{2.964688in}}{\pgfqpoint{3.378314in}{2.972588in}}{\pgfqpoint{3.378314in}{2.980824in}}%
\pgfpathcurveto{\pgfqpoint{3.378314in}{2.989061in}}{\pgfqpoint{3.375041in}{2.996961in}}{\pgfqpoint{3.369217in}{3.002785in}}%
\pgfpathcurveto{\pgfqpoint{3.363393in}{3.008609in}}{\pgfqpoint{3.355493in}{3.011881in}}{\pgfqpoint{3.347257in}{3.011881in}}%
\pgfpathcurveto{\pgfqpoint{3.339021in}{3.011881in}}{\pgfqpoint{3.331121in}{3.008609in}}{\pgfqpoint{3.325297in}{3.002785in}}%
\pgfpathcurveto{\pgfqpoint{3.319473in}{2.996961in}}{\pgfqpoint{3.316201in}{2.989061in}}{\pgfqpoint{3.316201in}{2.980824in}}%
\pgfpathcurveto{\pgfqpoint{3.316201in}{2.972588in}}{\pgfqpoint{3.319473in}{2.964688in}}{\pgfqpoint{3.325297in}{2.958864in}}%
\pgfpathcurveto{\pgfqpoint{3.331121in}{2.953040in}}{\pgfqpoint{3.339021in}{2.949768in}}{\pgfqpoint{3.347257in}{2.949768in}}%
\pgfpathclose%
\pgfusepath{stroke,fill}%
\end{pgfscope}%
\begin{pgfscope}%
\pgfpathrectangle{\pgfqpoint{0.100000in}{0.220728in}}{\pgfqpoint{3.696000in}{3.696000in}}%
\pgfusepath{clip}%
\pgfsetbuttcap%
\pgfsetroundjoin%
\definecolor{currentfill}{rgb}{0.121569,0.466667,0.705882}%
\pgfsetfillcolor{currentfill}%
\pgfsetfillopacity{0.659146}%
\pgfsetlinewidth{1.003750pt}%
\definecolor{currentstroke}{rgb}{0.121569,0.466667,0.705882}%
\pgfsetstrokecolor{currentstroke}%
\pgfsetstrokeopacity{0.659146}%
\pgfsetdash{}{0pt}%
\pgfpathmoveto{\pgfqpoint{0.620330in}{1.469471in}}%
\pgfpathcurveto{\pgfqpoint{0.628567in}{1.469471in}}{\pgfqpoint{0.636467in}{1.472743in}}{\pgfqpoint{0.642291in}{1.478567in}}%
\pgfpathcurveto{\pgfqpoint{0.648115in}{1.484391in}}{\pgfqpoint{0.651387in}{1.492291in}}{\pgfqpoint{0.651387in}{1.500527in}}%
\pgfpathcurveto{\pgfqpoint{0.651387in}{1.508763in}}{\pgfqpoint{0.648115in}{1.516663in}}{\pgfqpoint{0.642291in}{1.522487in}}%
\pgfpathcurveto{\pgfqpoint{0.636467in}{1.528311in}}{\pgfqpoint{0.628567in}{1.531584in}}{\pgfqpoint{0.620330in}{1.531584in}}%
\pgfpathcurveto{\pgfqpoint{0.612094in}{1.531584in}}{\pgfqpoint{0.604194in}{1.528311in}}{\pgfqpoint{0.598370in}{1.522487in}}%
\pgfpathcurveto{\pgfqpoint{0.592546in}{1.516663in}}{\pgfqpoint{0.589274in}{1.508763in}}{\pgfqpoint{0.589274in}{1.500527in}}%
\pgfpathcurveto{\pgfqpoint{0.589274in}{1.492291in}}{\pgfqpoint{0.592546in}{1.484391in}}{\pgfqpoint{0.598370in}{1.478567in}}%
\pgfpathcurveto{\pgfqpoint{0.604194in}{1.472743in}}{\pgfqpoint{0.612094in}{1.469471in}}{\pgfqpoint{0.620330in}{1.469471in}}%
\pgfpathclose%
\pgfusepath{stroke,fill}%
\end{pgfscope}%
\begin{pgfscope}%
\pgfpathrectangle{\pgfqpoint{0.100000in}{0.220728in}}{\pgfqpoint{3.696000in}{3.696000in}}%
\pgfusepath{clip}%
\pgfsetbuttcap%
\pgfsetroundjoin%
\definecolor{currentfill}{rgb}{0.121569,0.466667,0.705882}%
\pgfsetfillcolor{currentfill}%
\pgfsetfillopacity{0.659834}%
\pgfsetlinewidth{1.003750pt}%
\definecolor{currentstroke}{rgb}{0.121569,0.466667,0.705882}%
\pgfsetstrokecolor{currentstroke}%
\pgfsetstrokeopacity{0.659834}%
\pgfsetdash{}{0pt}%
\pgfpathmoveto{\pgfqpoint{0.628511in}{1.458447in}}%
\pgfpathcurveto{\pgfqpoint{0.636747in}{1.458447in}}{\pgfqpoint{0.644648in}{1.461719in}}{\pgfqpoint{0.650471in}{1.467543in}}%
\pgfpathcurveto{\pgfqpoint{0.656295in}{1.473367in}}{\pgfqpoint{0.659568in}{1.481267in}}{\pgfqpoint{0.659568in}{1.489503in}}%
\pgfpathcurveto{\pgfqpoint{0.659568in}{1.497740in}}{\pgfqpoint{0.656295in}{1.505640in}}{\pgfqpoint{0.650471in}{1.511464in}}%
\pgfpathcurveto{\pgfqpoint{0.644648in}{1.517288in}}{\pgfqpoint{0.636747in}{1.520560in}}{\pgfqpoint{0.628511in}{1.520560in}}%
\pgfpathcurveto{\pgfqpoint{0.620275in}{1.520560in}}{\pgfqpoint{0.612375in}{1.517288in}}{\pgfqpoint{0.606551in}{1.511464in}}%
\pgfpathcurveto{\pgfqpoint{0.600727in}{1.505640in}}{\pgfqpoint{0.597455in}{1.497740in}}{\pgfqpoint{0.597455in}{1.489503in}}%
\pgfpathcurveto{\pgfqpoint{0.597455in}{1.481267in}}{\pgfqpoint{0.600727in}{1.473367in}}{\pgfqpoint{0.606551in}{1.467543in}}%
\pgfpathcurveto{\pgfqpoint{0.612375in}{1.461719in}}{\pgfqpoint{0.620275in}{1.458447in}}{\pgfqpoint{0.628511in}{1.458447in}}%
\pgfpathclose%
\pgfusepath{stroke,fill}%
\end{pgfscope}%
\begin{pgfscope}%
\pgfpathrectangle{\pgfqpoint{0.100000in}{0.220728in}}{\pgfqpoint{3.696000in}{3.696000in}}%
\pgfusepath{clip}%
\pgfsetbuttcap%
\pgfsetroundjoin%
\definecolor{currentfill}{rgb}{0.121569,0.466667,0.705882}%
\pgfsetfillcolor{currentfill}%
\pgfsetfillopacity{0.660905}%
\pgfsetlinewidth{1.003750pt}%
\definecolor{currentstroke}{rgb}{0.121569,0.466667,0.705882}%
\pgfsetstrokecolor{currentstroke}%
\pgfsetstrokeopacity{0.660905}%
\pgfsetdash{}{0pt}%
\pgfpathmoveto{\pgfqpoint{0.630037in}{1.458301in}}%
\pgfpathcurveto{\pgfqpoint{0.638273in}{1.458301in}}{\pgfqpoint{0.646173in}{1.461573in}}{\pgfqpoint{0.651997in}{1.467397in}}%
\pgfpathcurveto{\pgfqpoint{0.657821in}{1.473221in}}{\pgfqpoint{0.661093in}{1.481121in}}{\pgfqpoint{0.661093in}{1.489358in}}%
\pgfpathcurveto{\pgfqpoint{0.661093in}{1.497594in}}{\pgfqpoint{0.657821in}{1.505494in}}{\pgfqpoint{0.651997in}{1.511318in}}%
\pgfpathcurveto{\pgfqpoint{0.646173in}{1.517142in}}{\pgfqpoint{0.638273in}{1.520414in}}{\pgfqpoint{0.630037in}{1.520414in}}%
\pgfpathcurveto{\pgfqpoint{0.621800in}{1.520414in}}{\pgfqpoint{0.613900in}{1.517142in}}{\pgfqpoint{0.608076in}{1.511318in}}%
\pgfpathcurveto{\pgfqpoint{0.602252in}{1.505494in}}{\pgfqpoint{0.598980in}{1.497594in}}{\pgfqpoint{0.598980in}{1.489358in}}%
\pgfpathcurveto{\pgfqpoint{0.598980in}{1.481121in}}{\pgfqpoint{0.602252in}{1.473221in}}{\pgfqpoint{0.608076in}{1.467397in}}%
\pgfpathcurveto{\pgfqpoint{0.613900in}{1.461573in}}{\pgfqpoint{0.621800in}{1.458301in}}{\pgfqpoint{0.630037in}{1.458301in}}%
\pgfpathclose%
\pgfusepath{stroke,fill}%
\end{pgfscope}%
\begin{pgfscope}%
\pgfpathrectangle{\pgfqpoint{0.100000in}{0.220728in}}{\pgfqpoint{3.696000in}{3.696000in}}%
\pgfusepath{clip}%
\pgfsetbuttcap%
\pgfsetroundjoin%
\definecolor{currentfill}{rgb}{0.121569,0.466667,0.705882}%
\pgfsetfillcolor{currentfill}%
\pgfsetfillopacity{0.662348}%
\pgfsetlinewidth{1.003750pt}%
\definecolor{currentstroke}{rgb}{0.121569,0.466667,0.705882}%
\pgfsetstrokecolor{currentstroke}%
\pgfsetstrokeopacity{0.662348}%
\pgfsetdash{}{0pt}%
\pgfpathmoveto{\pgfqpoint{3.364369in}{2.948492in}}%
\pgfpathcurveto{\pgfqpoint{3.372606in}{2.948492in}}{\pgfqpoint{3.380506in}{2.951764in}}{\pgfqpoint{3.386330in}{2.957588in}}%
\pgfpathcurveto{\pgfqpoint{3.392154in}{2.963412in}}{\pgfqpoint{3.395426in}{2.971312in}}{\pgfqpoint{3.395426in}{2.979548in}}%
\pgfpathcurveto{\pgfqpoint{3.395426in}{2.987785in}}{\pgfqpoint{3.392154in}{2.995685in}}{\pgfqpoint{3.386330in}{3.001509in}}%
\pgfpathcurveto{\pgfqpoint{3.380506in}{3.007333in}}{\pgfqpoint{3.372606in}{3.010605in}}{\pgfqpoint{3.364369in}{3.010605in}}%
\pgfpathcurveto{\pgfqpoint{3.356133in}{3.010605in}}{\pgfqpoint{3.348233in}{3.007333in}}{\pgfqpoint{3.342409in}{3.001509in}}%
\pgfpathcurveto{\pgfqpoint{3.336585in}{2.995685in}}{\pgfqpoint{3.333313in}{2.987785in}}{\pgfqpoint{3.333313in}{2.979548in}}%
\pgfpathcurveto{\pgfqpoint{3.333313in}{2.971312in}}{\pgfqpoint{3.336585in}{2.963412in}}{\pgfqpoint{3.342409in}{2.957588in}}%
\pgfpathcurveto{\pgfqpoint{3.348233in}{2.951764in}}{\pgfqpoint{3.356133in}{2.948492in}}{\pgfqpoint{3.364369in}{2.948492in}}%
\pgfpathclose%
\pgfusepath{stroke,fill}%
\end{pgfscope}%
\begin{pgfscope}%
\pgfpathrectangle{\pgfqpoint{0.100000in}{0.220728in}}{\pgfqpoint{3.696000in}{3.696000in}}%
\pgfusepath{clip}%
\pgfsetbuttcap%
\pgfsetroundjoin%
\definecolor{currentfill}{rgb}{0.121569,0.466667,0.705882}%
\pgfsetfillcolor{currentfill}%
\pgfsetfillopacity{0.662484}%
\pgfsetlinewidth{1.003750pt}%
\definecolor{currentstroke}{rgb}{0.121569,0.466667,0.705882}%
\pgfsetstrokecolor{currentstroke}%
\pgfsetstrokeopacity{0.662484}%
\pgfsetdash{}{0pt}%
\pgfpathmoveto{\pgfqpoint{0.633219in}{1.457205in}}%
\pgfpathcurveto{\pgfqpoint{0.641456in}{1.457205in}}{\pgfqpoint{0.649356in}{1.460478in}}{\pgfqpoint{0.655180in}{1.466301in}}%
\pgfpathcurveto{\pgfqpoint{0.661003in}{1.472125in}}{\pgfqpoint{0.664276in}{1.480025in}}{\pgfqpoint{0.664276in}{1.488262in}}%
\pgfpathcurveto{\pgfqpoint{0.664276in}{1.496498in}}{\pgfqpoint{0.661003in}{1.504398in}}{\pgfqpoint{0.655180in}{1.510222in}}%
\pgfpathcurveto{\pgfqpoint{0.649356in}{1.516046in}}{\pgfqpoint{0.641456in}{1.519318in}}{\pgfqpoint{0.633219in}{1.519318in}}%
\pgfpathcurveto{\pgfqpoint{0.624983in}{1.519318in}}{\pgfqpoint{0.617083in}{1.516046in}}{\pgfqpoint{0.611259in}{1.510222in}}%
\pgfpathcurveto{\pgfqpoint{0.605435in}{1.504398in}}{\pgfqpoint{0.602163in}{1.496498in}}{\pgfqpoint{0.602163in}{1.488262in}}%
\pgfpathcurveto{\pgfqpoint{0.602163in}{1.480025in}}{\pgfqpoint{0.605435in}{1.472125in}}{\pgfqpoint{0.611259in}{1.466301in}}%
\pgfpathcurveto{\pgfqpoint{0.617083in}{1.460478in}}{\pgfqpoint{0.624983in}{1.457205in}}{\pgfqpoint{0.633219in}{1.457205in}}%
\pgfpathclose%
\pgfusepath{stroke,fill}%
\end{pgfscope}%
\begin{pgfscope}%
\pgfpathrectangle{\pgfqpoint{0.100000in}{0.220728in}}{\pgfqpoint{3.696000in}{3.696000in}}%
\pgfusepath{clip}%
\pgfsetbuttcap%
\pgfsetroundjoin%
\definecolor{currentfill}{rgb}{0.121569,0.466667,0.705882}%
\pgfsetfillcolor{currentfill}%
\pgfsetfillopacity{0.663134}%
\pgfsetlinewidth{1.003750pt}%
\definecolor{currentstroke}{rgb}{0.121569,0.466667,0.705882}%
\pgfsetstrokecolor{currentstroke}%
\pgfsetstrokeopacity{0.663134}%
\pgfsetdash{}{0pt}%
\pgfpathmoveto{\pgfqpoint{3.374320in}{2.945638in}}%
\pgfpathcurveto{\pgfqpoint{3.382556in}{2.945638in}}{\pgfqpoint{3.390456in}{2.948910in}}{\pgfqpoint{3.396280in}{2.954734in}}%
\pgfpathcurveto{\pgfqpoint{3.402104in}{2.960558in}}{\pgfqpoint{3.405376in}{2.968458in}}{\pgfqpoint{3.405376in}{2.976695in}}%
\pgfpathcurveto{\pgfqpoint{3.405376in}{2.984931in}}{\pgfqpoint{3.402104in}{2.992831in}}{\pgfqpoint{3.396280in}{2.998655in}}%
\pgfpathcurveto{\pgfqpoint{3.390456in}{3.004479in}}{\pgfqpoint{3.382556in}{3.007751in}}{\pgfqpoint{3.374320in}{3.007751in}}%
\pgfpathcurveto{\pgfqpoint{3.366084in}{3.007751in}}{\pgfqpoint{3.358184in}{3.004479in}}{\pgfqpoint{3.352360in}{2.998655in}}%
\pgfpathcurveto{\pgfqpoint{3.346536in}{2.992831in}}{\pgfqpoint{3.343263in}{2.984931in}}{\pgfqpoint{3.343263in}{2.976695in}}%
\pgfpathcurveto{\pgfqpoint{3.343263in}{2.968458in}}{\pgfqpoint{3.346536in}{2.960558in}}{\pgfqpoint{3.352360in}{2.954734in}}%
\pgfpathcurveto{\pgfqpoint{3.358184in}{2.948910in}}{\pgfqpoint{3.366084in}{2.945638in}}{\pgfqpoint{3.374320in}{2.945638in}}%
\pgfpathclose%
\pgfusepath{stroke,fill}%
\end{pgfscope}%
\begin{pgfscope}%
\pgfpathrectangle{\pgfqpoint{0.100000in}{0.220728in}}{\pgfqpoint{3.696000in}{3.696000in}}%
\pgfusepath{clip}%
\pgfsetbuttcap%
\pgfsetroundjoin%
\definecolor{currentfill}{rgb}{0.121569,0.466667,0.705882}%
\pgfsetfillcolor{currentfill}%
\pgfsetfillopacity{0.663279}%
\pgfsetlinewidth{1.003750pt}%
\definecolor{currentstroke}{rgb}{0.121569,0.466667,0.705882}%
\pgfsetstrokecolor{currentstroke}%
\pgfsetstrokeopacity{0.663279}%
\pgfsetdash{}{0pt}%
\pgfpathmoveto{\pgfqpoint{0.640977in}{1.451110in}}%
\pgfpathcurveto{\pgfqpoint{0.649214in}{1.451110in}}{\pgfqpoint{0.657114in}{1.454382in}}{\pgfqpoint{0.662938in}{1.460206in}}%
\pgfpathcurveto{\pgfqpoint{0.668762in}{1.466030in}}{\pgfqpoint{0.672034in}{1.473930in}}{\pgfqpoint{0.672034in}{1.482166in}}%
\pgfpathcurveto{\pgfqpoint{0.672034in}{1.490402in}}{\pgfqpoint{0.668762in}{1.498302in}}{\pgfqpoint{0.662938in}{1.504126in}}%
\pgfpathcurveto{\pgfqpoint{0.657114in}{1.509950in}}{\pgfqpoint{0.649214in}{1.513223in}}{\pgfqpoint{0.640977in}{1.513223in}}%
\pgfpathcurveto{\pgfqpoint{0.632741in}{1.513223in}}{\pgfqpoint{0.624841in}{1.509950in}}{\pgfqpoint{0.619017in}{1.504126in}}%
\pgfpathcurveto{\pgfqpoint{0.613193in}{1.498302in}}{\pgfqpoint{0.609921in}{1.490402in}}{\pgfqpoint{0.609921in}{1.482166in}}%
\pgfpathcurveto{\pgfqpoint{0.609921in}{1.473930in}}{\pgfqpoint{0.613193in}{1.466030in}}{\pgfqpoint{0.619017in}{1.460206in}}%
\pgfpathcurveto{\pgfqpoint{0.624841in}{1.454382in}}{\pgfqpoint{0.632741in}{1.451110in}}{\pgfqpoint{0.640977in}{1.451110in}}%
\pgfpathclose%
\pgfusepath{stroke,fill}%
\end{pgfscope}%
\begin{pgfscope}%
\pgfpathrectangle{\pgfqpoint{0.100000in}{0.220728in}}{\pgfqpoint{3.696000in}{3.696000in}}%
\pgfusepath{clip}%
\pgfsetbuttcap%
\pgfsetroundjoin%
\definecolor{currentfill}{rgb}{0.121569,0.466667,0.705882}%
\pgfsetfillcolor{currentfill}%
\pgfsetfillopacity{0.664347}%
\pgfsetlinewidth{1.003750pt}%
\definecolor{currentstroke}{rgb}{0.121569,0.466667,0.705882}%
\pgfsetstrokecolor{currentstroke}%
\pgfsetstrokeopacity{0.664347}%
\pgfsetdash{}{0pt}%
\pgfpathmoveto{\pgfqpoint{3.379283in}{2.944697in}}%
\pgfpathcurveto{\pgfqpoint{3.387520in}{2.944697in}}{\pgfqpoint{3.395420in}{2.947969in}}{\pgfqpoint{3.401244in}{2.953793in}}%
\pgfpathcurveto{\pgfqpoint{3.407068in}{2.959617in}}{\pgfqpoint{3.410340in}{2.967517in}}{\pgfqpoint{3.410340in}{2.975753in}}%
\pgfpathcurveto{\pgfqpoint{3.410340in}{2.983990in}}{\pgfqpoint{3.407068in}{2.991890in}}{\pgfqpoint{3.401244in}{2.997714in}}%
\pgfpathcurveto{\pgfqpoint{3.395420in}{3.003538in}}{\pgfqpoint{3.387520in}{3.006810in}}{\pgfqpoint{3.379283in}{3.006810in}}%
\pgfpathcurveto{\pgfqpoint{3.371047in}{3.006810in}}{\pgfqpoint{3.363147in}{3.003538in}}{\pgfqpoint{3.357323in}{2.997714in}}%
\pgfpathcurveto{\pgfqpoint{3.351499in}{2.991890in}}{\pgfqpoint{3.348227in}{2.983990in}}{\pgfqpoint{3.348227in}{2.975753in}}%
\pgfpathcurveto{\pgfqpoint{3.348227in}{2.967517in}}{\pgfqpoint{3.351499in}{2.959617in}}{\pgfqpoint{3.357323in}{2.953793in}}%
\pgfpathcurveto{\pgfqpoint{3.363147in}{2.947969in}}{\pgfqpoint{3.371047in}{2.944697in}}{\pgfqpoint{3.379283in}{2.944697in}}%
\pgfpathclose%
\pgfusepath{stroke,fill}%
\end{pgfscope}%
\begin{pgfscope}%
\pgfpathrectangle{\pgfqpoint{0.100000in}{0.220728in}}{\pgfqpoint{3.696000in}{3.696000in}}%
\pgfusepath{clip}%
\pgfsetbuttcap%
\pgfsetroundjoin%
\definecolor{currentfill}{rgb}{0.121569,0.466667,0.705882}%
\pgfsetfillcolor{currentfill}%
\pgfsetfillopacity{0.665603}%
\pgfsetlinewidth{1.003750pt}%
\definecolor{currentstroke}{rgb}{0.121569,0.466667,0.705882}%
\pgfsetstrokecolor{currentstroke}%
\pgfsetstrokeopacity{0.665603}%
\pgfsetdash{}{0pt}%
\pgfpathmoveto{\pgfqpoint{3.385348in}{2.943175in}}%
\pgfpathcurveto{\pgfqpoint{3.393585in}{2.943175in}}{\pgfqpoint{3.401485in}{2.946447in}}{\pgfqpoint{3.407309in}{2.952271in}}%
\pgfpathcurveto{\pgfqpoint{3.413132in}{2.958095in}}{\pgfqpoint{3.416405in}{2.965995in}}{\pgfqpoint{3.416405in}{2.974231in}}%
\pgfpathcurveto{\pgfqpoint{3.416405in}{2.982468in}}{\pgfqpoint{3.413132in}{2.990368in}}{\pgfqpoint{3.407309in}{2.996192in}}%
\pgfpathcurveto{\pgfqpoint{3.401485in}{3.002016in}}{\pgfqpoint{3.393585in}{3.005288in}}{\pgfqpoint{3.385348in}{3.005288in}}%
\pgfpathcurveto{\pgfqpoint{3.377112in}{3.005288in}}{\pgfqpoint{3.369212in}{3.002016in}}{\pgfqpoint{3.363388in}{2.996192in}}%
\pgfpathcurveto{\pgfqpoint{3.357564in}{2.990368in}}{\pgfqpoint{3.354292in}{2.982468in}}{\pgfqpoint{3.354292in}{2.974231in}}%
\pgfpathcurveto{\pgfqpoint{3.354292in}{2.965995in}}{\pgfqpoint{3.357564in}{2.958095in}}{\pgfqpoint{3.363388in}{2.952271in}}%
\pgfpathcurveto{\pgfqpoint{3.369212in}{2.946447in}}{\pgfqpoint{3.377112in}{2.943175in}}{\pgfqpoint{3.385348in}{2.943175in}}%
\pgfpathclose%
\pgfusepath{stroke,fill}%
\end{pgfscope}%
\begin{pgfscope}%
\pgfpathrectangle{\pgfqpoint{0.100000in}{0.220728in}}{\pgfqpoint{3.696000in}{3.696000in}}%
\pgfusepath{clip}%
\pgfsetbuttcap%
\pgfsetroundjoin%
\definecolor{currentfill}{rgb}{0.121569,0.466667,0.705882}%
\pgfsetfillcolor{currentfill}%
\pgfsetfillopacity{0.665731}%
\pgfsetlinewidth{1.003750pt}%
\definecolor{currentstroke}{rgb}{0.121569,0.466667,0.705882}%
\pgfsetstrokecolor{currentstroke}%
\pgfsetstrokeopacity{0.665731}%
\pgfsetdash{}{0pt}%
\pgfpathmoveto{\pgfqpoint{0.654998in}{1.444319in}}%
\pgfpathcurveto{\pgfqpoint{0.663234in}{1.444319in}}{\pgfqpoint{0.671134in}{1.447591in}}{\pgfqpoint{0.676958in}{1.453415in}}%
\pgfpathcurveto{\pgfqpoint{0.682782in}{1.459239in}}{\pgfqpoint{0.686054in}{1.467139in}}{\pgfqpoint{0.686054in}{1.475375in}}%
\pgfpathcurveto{\pgfqpoint{0.686054in}{1.483612in}}{\pgfqpoint{0.682782in}{1.491512in}}{\pgfqpoint{0.676958in}{1.497336in}}%
\pgfpathcurveto{\pgfqpoint{0.671134in}{1.503160in}}{\pgfqpoint{0.663234in}{1.506432in}}{\pgfqpoint{0.654998in}{1.506432in}}%
\pgfpathcurveto{\pgfqpoint{0.646762in}{1.506432in}}{\pgfqpoint{0.638861in}{1.503160in}}{\pgfqpoint{0.633038in}{1.497336in}}%
\pgfpathcurveto{\pgfqpoint{0.627214in}{1.491512in}}{\pgfqpoint{0.623941in}{1.483612in}}{\pgfqpoint{0.623941in}{1.475375in}}%
\pgfpathcurveto{\pgfqpoint{0.623941in}{1.467139in}}{\pgfqpoint{0.627214in}{1.459239in}}{\pgfqpoint{0.633038in}{1.453415in}}%
\pgfpathcurveto{\pgfqpoint{0.638861in}{1.447591in}}{\pgfqpoint{0.646762in}{1.444319in}}{\pgfqpoint{0.654998in}{1.444319in}}%
\pgfpathclose%
\pgfusepath{stroke,fill}%
\end{pgfscope}%
\begin{pgfscope}%
\pgfpathrectangle{\pgfqpoint{0.100000in}{0.220728in}}{\pgfqpoint{3.696000in}{3.696000in}}%
\pgfusepath{clip}%
\pgfsetbuttcap%
\pgfsetroundjoin%
\definecolor{currentfill}{rgb}{0.121569,0.466667,0.705882}%
\pgfsetfillcolor{currentfill}%
\pgfsetfillopacity{0.665892}%
\pgfsetlinewidth{1.003750pt}%
\definecolor{currentstroke}{rgb}{0.121569,0.466667,0.705882}%
\pgfsetstrokecolor{currentstroke}%
\pgfsetstrokeopacity{0.665892}%
\pgfsetdash{}{0pt}%
\pgfpathmoveto{\pgfqpoint{3.388991in}{2.942295in}}%
\pgfpathcurveto{\pgfqpoint{3.397227in}{2.942295in}}{\pgfqpoint{3.405127in}{2.945567in}}{\pgfqpoint{3.410951in}{2.951391in}}%
\pgfpathcurveto{\pgfqpoint{3.416775in}{2.957215in}}{\pgfqpoint{3.420047in}{2.965115in}}{\pgfqpoint{3.420047in}{2.973352in}}%
\pgfpathcurveto{\pgfqpoint{3.420047in}{2.981588in}}{\pgfqpoint{3.416775in}{2.989488in}}{\pgfqpoint{3.410951in}{2.995312in}}%
\pgfpathcurveto{\pgfqpoint{3.405127in}{3.001136in}}{\pgfqpoint{3.397227in}{3.004408in}}{\pgfqpoint{3.388991in}{3.004408in}}%
\pgfpathcurveto{\pgfqpoint{3.380754in}{3.004408in}}{\pgfqpoint{3.372854in}{3.001136in}}{\pgfqpoint{3.367030in}{2.995312in}}%
\pgfpathcurveto{\pgfqpoint{3.361207in}{2.989488in}}{\pgfqpoint{3.357934in}{2.981588in}}{\pgfqpoint{3.357934in}{2.973352in}}%
\pgfpathcurveto{\pgfqpoint{3.357934in}{2.965115in}}{\pgfqpoint{3.361207in}{2.957215in}}{\pgfqpoint{3.367030in}{2.951391in}}%
\pgfpathcurveto{\pgfqpoint{3.372854in}{2.945567in}}{\pgfqpoint{3.380754in}{2.942295in}}{\pgfqpoint{3.388991in}{2.942295in}}%
\pgfpathclose%
\pgfusepath{stroke,fill}%
\end{pgfscope}%
\begin{pgfscope}%
\pgfpathrectangle{\pgfqpoint{0.100000in}{0.220728in}}{\pgfqpoint{3.696000in}{3.696000in}}%
\pgfusepath{clip}%
\pgfsetbuttcap%
\pgfsetroundjoin%
\definecolor{currentfill}{rgb}{0.121569,0.466667,0.705882}%
\pgfsetfillcolor{currentfill}%
\pgfsetfillopacity{0.666988}%
\pgfsetlinewidth{1.003750pt}%
\definecolor{currentstroke}{rgb}{0.121569,0.466667,0.705882}%
\pgfsetstrokecolor{currentstroke}%
\pgfsetstrokeopacity{0.666988}%
\pgfsetdash{}{0pt}%
\pgfpathmoveto{\pgfqpoint{3.393066in}{2.941607in}}%
\pgfpathcurveto{\pgfqpoint{3.401303in}{2.941607in}}{\pgfqpoint{3.409203in}{2.944880in}}{\pgfqpoint{3.415027in}{2.950704in}}%
\pgfpathcurveto{\pgfqpoint{3.420851in}{2.956527in}}{\pgfqpoint{3.424123in}{2.964427in}}{\pgfqpoint{3.424123in}{2.972664in}}%
\pgfpathcurveto{\pgfqpoint{3.424123in}{2.980900in}}{\pgfqpoint{3.420851in}{2.988800in}}{\pgfqpoint{3.415027in}{2.994624in}}%
\pgfpathcurveto{\pgfqpoint{3.409203in}{3.000448in}}{\pgfqpoint{3.401303in}{3.003720in}}{\pgfqpoint{3.393066in}{3.003720in}}%
\pgfpathcurveto{\pgfqpoint{3.384830in}{3.003720in}}{\pgfqpoint{3.376930in}{3.000448in}}{\pgfqpoint{3.371106in}{2.994624in}}%
\pgfpathcurveto{\pgfqpoint{3.365282in}{2.988800in}}{\pgfqpoint{3.362010in}{2.980900in}}{\pgfqpoint{3.362010in}{2.972664in}}%
\pgfpathcurveto{\pgfqpoint{3.362010in}{2.964427in}}{\pgfqpoint{3.365282in}{2.956527in}}{\pgfqpoint{3.371106in}{2.950704in}}%
\pgfpathcurveto{\pgfqpoint{3.376930in}{2.944880in}}{\pgfqpoint{3.384830in}{2.941607in}}{\pgfqpoint{3.393066in}{2.941607in}}%
\pgfpathclose%
\pgfusepath{stroke,fill}%
\end{pgfscope}%
\begin{pgfscope}%
\pgfpathrectangle{\pgfqpoint{0.100000in}{0.220728in}}{\pgfqpoint{3.696000in}{3.696000in}}%
\pgfusepath{clip}%
\pgfsetbuttcap%
\pgfsetroundjoin%
\definecolor{currentfill}{rgb}{0.121569,0.466667,0.705882}%
\pgfsetfillcolor{currentfill}%
\pgfsetfillopacity{0.667142}%
\pgfsetlinewidth{1.003750pt}%
\definecolor{currentstroke}{rgb}{0.121569,0.466667,0.705882}%
\pgfsetstrokecolor{currentstroke}%
\pgfsetstrokeopacity{0.667142}%
\pgfsetdash{}{0pt}%
\pgfpathmoveto{\pgfqpoint{0.669318in}{1.436598in}}%
\pgfpathcurveto{\pgfqpoint{0.677554in}{1.436598in}}{\pgfqpoint{0.685454in}{1.439871in}}{\pgfqpoint{0.691278in}{1.445694in}}%
\pgfpathcurveto{\pgfqpoint{0.697102in}{1.451518in}}{\pgfqpoint{0.700375in}{1.459418in}}{\pgfqpoint{0.700375in}{1.467655in}}%
\pgfpathcurveto{\pgfqpoint{0.700375in}{1.475891in}}{\pgfqpoint{0.697102in}{1.483791in}}{\pgfqpoint{0.691278in}{1.489615in}}%
\pgfpathcurveto{\pgfqpoint{0.685454in}{1.495439in}}{\pgfqpoint{0.677554in}{1.498711in}}{\pgfqpoint{0.669318in}{1.498711in}}%
\pgfpathcurveto{\pgfqpoint{0.661082in}{1.498711in}}{\pgfqpoint{0.653182in}{1.495439in}}{\pgfqpoint{0.647358in}{1.489615in}}%
\pgfpathcurveto{\pgfqpoint{0.641534in}{1.483791in}}{\pgfqpoint{0.638262in}{1.475891in}}{\pgfqpoint{0.638262in}{1.467655in}}%
\pgfpathcurveto{\pgfqpoint{0.638262in}{1.459418in}}{\pgfqpoint{0.641534in}{1.451518in}}{\pgfqpoint{0.647358in}{1.445694in}}%
\pgfpathcurveto{\pgfqpoint{0.653182in}{1.439871in}}{\pgfqpoint{0.661082in}{1.436598in}}{\pgfqpoint{0.669318in}{1.436598in}}%
\pgfpathclose%
\pgfusepath{stroke,fill}%
\end{pgfscope}%
\begin{pgfscope}%
\pgfpathrectangle{\pgfqpoint{0.100000in}{0.220728in}}{\pgfqpoint{3.696000in}{3.696000in}}%
\pgfusepath{clip}%
\pgfsetbuttcap%
\pgfsetroundjoin%
\definecolor{currentfill}{rgb}{0.121569,0.466667,0.705882}%
\pgfsetfillcolor{currentfill}%
\pgfsetfillopacity{0.667564}%
\pgfsetlinewidth{1.003750pt}%
\definecolor{currentstroke}{rgb}{0.121569,0.466667,0.705882}%
\pgfsetstrokecolor{currentstroke}%
\pgfsetstrokeopacity{0.667564}%
\pgfsetdash{}{0pt}%
\pgfpathmoveto{\pgfqpoint{3.395350in}{2.941257in}}%
\pgfpathcurveto{\pgfqpoint{3.403586in}{2.941257in}}{\pgfqpoint{3.411486in}{2.944530in}}{\pgfqpoint{3.417310in}{2.950353in}}%
\pgfpathcurveto{\pgfqpoint{3.423134in}{2.956177in}}{\pgfqpoint{3.426407in}{2.964077in}}{\pgfqpoint{3.426407in}{2.972314in}}%
\pgfpathcurveto{\pgfqpoint{3.426407in}{2.980550in}}{\pgfqpoint{3.423134in}{2.988450in}}{\pgfqpoint{3.417310in}{2.994274in}}%
\pgfpathcurveto{\pgfqpoint{3.411486in}{3.000098in}}{\pgfqpoint{3.403586in}{3.003370in}}{\pgfqpoint{3.395350in}{3.003370in}}%
\pgfpathcurveto{\pgfqpoint{3.387114in}{3.003370in}}{\pgfqpoint{3.379214in}{3.000098in}}{\pgfqpoint{3.373390in}{2.994274in}}%
\pgfpathcurveto{\pgfqpoint{3.367566in}{2.988450in}}{\pgfqpoint{3.364294in}{2.980550in}}{\pgfqpoint{3.364294in}{2.972314in}}%
\pgfpathcurveto{\pgfqpoint{3.364294in}{2.964077in}}{\pgfqpoint{3.367566in}{2.956177in}}{\pgfqpoint{3.373390in}{2.950353in}}%
\pgfpathcurveto{\pgfqpoint{3.379214in}{2.944530in}}{\pgfqpoint{3.387114in}{2.941257in}}{\pgfqpoint{3.395350in}{2.941257in}}%
\pgfpathclose%
\pgfusepath{stroke,fill}%
\end{pgfscope}%
\begin{pgfscope}%
\pgfpathrectangle{\pgfqpoint{0.100000in}{0.220728in}}{\pgfqpoint{3.696000in}{3.696000in}}%
\pgfusepath{clip}%
\pgfsetbuttcap%
\pgfsetroundjoin%
\definecolor{currentfill}{rgb}{0.121569,0.466667,0.705882}%
\pgfsetfillcolor{currentfill}%
\pgfsetfillopacity{0.667922}%
\pgfsetlinewidth{1.003750pt}%
\definecolor{currentstroke}{rgb}{0.121569,0.466667,0.705882}%
\pgfsetstrokecolor{currentstroke}%
\pgfsetstrokeopacity{0.667922}%
\pgfsetdash{}{0pt}%
\pgfpathmoveto{\pgfqpoint{3.396547in}{2.941043in}}%
\pgfpathcurveto{\pgfqpoint{3.404783in}{2.941043in}}{\pgfqpoint{3.412683in}{2.944316in}}{\pgfqpoint{3.418507in}{2.950140in}}%
\pgfpathcurveto{\pgfqpoint{3.424331in}{2.955964in}}{\pgfqpoint{3.427603in}{2.963864in}}{\pgfqpoint{3.427603in}{2.972100in}}%
\pgfpathcurveto{\pgfqpoint{3.427603in}{2.980336in}}{\pgfqpoint{3.424331in}{2.988236in}}{\pgfqpoint{3.418507in}{2.994060in}}%
\pgfpathcurveto{\pgfqpoint{3.412683in}{2.999884in}}{\pgfqpoint{3.404783in}{3.003156in}}{\pgfqpoint{3.396547in}{3.003156in}}%
\pgfpathcurveto{\pgfqpoint{3.388311in}{3.003156in}}{\pgfqpoint{3.380410in}{2.999884in}}{\pgfqpoint{3.374587in}{2.994060in}}%
\pgfpathcurveto{\pgfqpoint{3.368763in}{2.988236in}}{\pgfqpoint{3.365490in}{2.980336in}}{\pgfqpoint{3.365490in}{2.972100in}}%
\pgfpathcurveto{\pgfqpoint{3.365490in}{2.963864in}}{\pgfqpoint{3.368763in}{2.955964in}}{\pgfqpoint{3.374587in}{2.950140in}}%
\pgfpathcurveto{\pgfqpoint{3.380410in}{2.944316in}}{\pgfqpoint{3.388311in}{2.941043in}}{\pgfqpoint{3.396547in}{2.941043in}}%
\pgfpathclose%
\pgfusepath{stroke,fill}%
\end{pgfscope}%
\begin{pgfscope}%
\pgfpathrectangle{\pgfqpoint{0.100000in}{0.220728in}}{\pgfqpoint{3.696000in}{3.696000in}}%
\pgfusepath{clip}%
\pgfsetbuttcap%
\pgfsetroundjoin%
\definecolor{currentfill}{rgb}{0.121569,0.466667,0.705882}%
\pgfsetfillcolor{currentfill}%
\pgfsetfillopacity{0.668276}%
\pgfsetlinewidth{1.003750pt}%
\definecolor{currentstroke}{rgb}{0.121569,0.466667,0.705882}%
\pgfsetstrokecolor{currentstroke}%
\pgfsetstrokeopacity{0.668276}%
\pgfsetdash{}{0pt}%
\pgfpathmoveto{\pgfqpoint{3.397097in}{2.941285in}}%
\pgfpathcurveto{\pgfqpoint{3.405333in}{2.941285in}}{\pgfqpoint{3.413233in}{2.944557in}}{\pgfqpoint{3.419057in}{2.950381in}}%
\pgfpathcurveto{\pgfqpoint{3.424881in}{2.956205in}}{\pgfqpoint{3.428153in}{2.964105in}}{\pgfqpoint{3.428153in}{2.972341in}}%
\pgfpathcurveto{\pgfqpoint{3.428153in}{2.980577in}}{\pgfqpoint{3.424881in}{2.988478in}}{\pgfqpoint{3.419057in}{2.994301in}}%
\pgfpathcurveto{\pgfqpoint{3.413233in}{3.000125in}}{\pgfqpoint{3.405333in}{3.003398in}}{\pgfqpoint{3.397097in}{3.003398in}}%
\pgfpathcurveto{\pgfqpoint{3.388860in}{3.003398in}}{\pgfqpoint{3.380960in}{3.000125in}}{\pgfqpoint{3.375136in}{2.994301in}}%
\pgfpathcurveto{\pgfqpoint{3.369312in}{2.988478in}}{\pgfqpoint{3.366040in}{2.980577in}}{\pgfqpoint{3.366040in}{2.972341in}}%
\pgfpathcurveto{\pgfqpoint{3.366040in}{2.964105in}}{\pgfqpoint{3.369312in}{2.956205in}}{\pgfqpoint{3.375136in}{2.950381in}}%
\pgfpathcurveto{\pgfqpoint{3.380960in}{2.944557in}}{\pgfqpoint{3.388860in}{2.941285in}}{\pgfqpoint{3.397097in}{2.941285in}}%
\pgfpathclose%
\pgfusepath{stroke,fill}%
\end{pgfscope}%
\begin{pgfscope}%
\pgfpathrectangle{\pgfqpoint{0.100000in}{0.220728in}}{\pgfqpoint{3.696000in}{3.696000in}}%
\pgfusepath{clip}%
\pgfsetbuttcap%
\pgfsetroundjoin%
\definecolor{currentfill}{rgb}{0.121569,0.466667,0.705882}%
\pgfsetfillcolor{currentfill}%
\pgfsetfillopacity{0.668667}%
\pgfsetlinewidth{1.003750pt}%
\definecolor{currentstroke}{rgb}{0.121569,0.466667,0.705882}%
\pgfsetstrokecolor{currentstroke}%
\pgfsetstrokeopacity{0.668667}%
\pgfsetdash{}{0pt}%
\pgfpathmoveto{\pgfqpoint{3.398464in}{2.940560in}}%
\pgfpathcurveto{\pgfqpoint{3.406700in}{2.940560in}}{\pgfqpoint{3.414600in}{2.943832in}}{\pgfqpoint{3.420424in}{2.949656in}}%
\pgfpathcurveto{\pgfqpoint{3.426248in}{2.955480in}}{\pgfqpoint{3.429520in}{2.963380in}}{\pgfqpoint{3.429520in}{2.971616in}}%
\pgfpathcurveto{\pgfqpoint{3.429520in}{2.979852in}}{\pgfqpoint{3.426248in}{2.987752in}}{\pgfqpoint{3.420424in}{2.993576in}}%
\pgfpathcurveto{\pgfqpoint{3.414600in}{2.999400in}}{\pgfqpoint{3.406700in}{3.002673in}}{\pgfqpoint{3.398464in}{3.002673in}}%
\pgfpathcurveto{\pgfqpoint{3.390227in}{3.002673in}}{\pgfqpoint{3.382327in}{2.999400in}}{\pgfqpoint{3.376503in}{2.993576in}}%
\pgfpathcurveto{\pgfqpoint{3.370679in}{2.987752in}}{\pgfqpoint{3.367407in}{2.979852in}}{\pgfqpoint{3.367407in}{2.971616in}}%
\pgfpathcurveto{\pgfqpoint{3.367407in}{2.963380in}}{\pgfqpoint{3.370679in}{2.955480in}}{\pgfqpoint{3.376503in}{2.949656in}}%
\pgfpathcurveto{\pgfqpoint{3.382327in}{2.943832in}}{\pgfqpoint{3.390227in}{2.940560in}}{\pgfqpoint{3.398464in}{2.940560in}}%
\pgfpathclose%
\pgfusepath{stroke,fill}%
\end{pgfscope}%
\begin{pgfscope}%
\pgfpathrectangle{\pgfqpoint{0.100000in}{0.220728in}}{\pgfqpoint{3.696000in}{3.696000in}}%
\pgfusepath{clip}%
\pgfsetbuttcap%
\pgfsetroundjoin%
\definecolor{currentfill}{rgb}{0.121569,0.466667,0.705882}%
\pgfsetfillcolor{currentfill}%
\pgfsetfillopacity{0.669075}%
\pgfsetlinewidth{1.003750pt}%
\definecolor{currentstroke}{rgb}{0.121569,0.466667,0.705882}%
\pgfsetstrokecolor{currentstroke}%
\pgfsetstrokeopacity{0.669075}%
\pgfsetdash{}{0pt}%
\pgfpathmoveto{\pgfqpoint{0.680479in}{1.431696in}}%
\pgfpathcurveto{\pgfqpoint{0.688716in}{1.431696in}}{\pgfqpoint{0.696616in}{1.434968in}}{\pgfqpoint{0.702440in}{1.440792in}}%
\pgfpathcurveto{\pgfqpoint{0.708263in}{1.446616in}}{\pgfqpoint{0.711536in}{1.454516in}}{\pgfqpoint{0.711536in}{1.462752in}}%
\pgfpathcurveto{\pgfqpoint{0.711536in}{1.470988in}}{\pgfqpoint{0.708263in}{1.478888in}}{\pgfqpoint{0.702440in}{1.484712in}}%
\pgfpathcurveto{\pgfqpoint{0.696616in}{1.490536in}}{\pgfqpoint{0.688716in}{1.493809in}}{\pgfqpoint{0.680479in}{1.493809in}}%
\pgfpathcurveto{\pgfqpoint{0.672243in}{1.493809in}}{\pgfqpoint{0.664343in}{1.490536in}}{\pgfqpoint{0.658519in}{1.484712in}}%
\pgfpathcurveto{\pgfqpoint{0.652695in}{1.478888in}}{\pgfqpoint{0.649423in}{1.470988in}}{\pgfqpoint{0.649423in}{1.462752in}}%
\pgfpathcurveto{\pgfqpoint{0.649423in}{1.454516in}}{\pgfqpoint{0.652695in}{1.446616in}}{\pgfqpoint{0.658519in}{1.440792in}}%
\pgfpathcurveto{\pgfqpoint{0.664343in}{1.434968in}}{\pgfqpoint{0.672243in}{1.431696in}}{\pgfqpoint{0.680479in}{1.431696in}}%
\pgfpathclose%
\pgfusepath{stroke,fill}%
\end{pgfscope}%
\begin{pgfscope}%
\pgfpathrectangle{\pgfqpoint{0.100000in}{0.220728in}}{\pgfqpoint{3.696000in}{3.696000in}}%
\pgfusepath{clip}%
\pgfsetbuttcap%
\pgfsetroundjoin%
\definecolor{currentfill}{rgb}{0.121569,0.466667,0.705882}%
\pgfsetfillcolor{currentfill}%
\pgfsetfillopacity{0.669351}%
\pgfsetlinewidth{1.003750pt}%
\definecolor{currentstroke}{rgb}{0.121569,0.466667,0.705882}%
\pgfsetstrokecolor{currentstroke}%
\pgfsetstrokeopacity{0.669351}%
\pgfsetdash{}{0pt}%
\pgfpathmoveto{\pgfqpoint{3.400310in}{2.939329in}}%
\pgfpathcurveto{\pgfqpoint{3.408546in}{2.939329in}}{\pgfqpoint{3.416446in}{2.942602in}}{\pgfqpoint{3.422270in}{2.948426in}}%
\pgfpathcurveto{\pgfqpoint{3.428094in}{2.954250in}}{\pgfqpoint{3.431366in}{2.962150in}}{\pgfqpoint{3.431366in}{2.970386in}}%
\pgfpathcurveto{\pgfqpoint{3.431366in}{2.978622in}}{\pgfqpoint{3.428094in}{2.986522in}}{\pgfqpoint{3.422270in}{2.992346in}}%
\pgfpathcurveto{\pgfqpoint{3.416446in}{2.998170in}}{\pgfqpoint{3.408546in}{3.001442in}}{\pgfqpoint{3.400310in}{3.001442in}}%
\pgfpathcurveto{\pgfqpoint{3.392074in}{3.001442in}}{\pgfqpoint{3.384174in}{2.998170in}}{\pgfqpoint{3.378350in}{2.992346in}}%
\pgfpathcurveto{\pgfqpoint{3.372526in}{2.986522in}}{\pgfqpoint{3.369253in}{2.978622in}}{\pgfqpoint{3.369253in}{2.970386in}}%
\pgfpathcurveto{\pgfqpoint{3.369253in}{2.962150in}}{\pgfqpoint{3.372526in}{2.954250in}}{\pgfqpoint{3.378350in}{2.948426in}}%
\pgfpathcurveto{\pgfqpoint{3.384174in}{2.942602in}}{\pgfqpoint{3.392074in}{2.939329in}}{\pgfqpoint{3.400310in}{2.939329in}}%
\pgfpathclose%
\pgfusepath{stroke,fill}%
\end{pgfscope}%
\begin{pgfscope}%
\pgfpathrectangle{\pgfqpoint{0.100000in}{0.220728in}}{\pgfqpoint{3.696000in}{3.696000in}}%
\pgfusepath{clip}%
\pgfsetbuttcap%
\pgfsetroundjoin%
\definecolor{currentfill}{rgb}{0.121569,0.466667,0.705882}%
\pgfsetfillcolor{currentfill}%
\pgfsetfillopacity{0.669738}%
\pgfsetlinewidth{1.003750pt}%
\definecolor{currentstroke}{rgb}{0.121569,0.466667,0.705882}%
\pgfsetstrokecolor{currentstroke}%
\pgfsetstrokeopacity{0.669738}%
\pgfsetdash{}{0pt}%
\pgfpathmoveto{\pgfqpoint{3.401171in}{2.938437in}}%
\pgfpathcurveto{\pgfqpoint{3.409407in}{2.938437in}}{\pgfqpoint{3.417307in}{2.941709in}}{\pgfqpoint{3.423131in}{2.947533in}}%
\pgfpathcurveto{\pgfqpoint{3.428955in}{2.953357in}}{\pgfqpoint{3.432227in}{2.961257in}}{\pgfqpoint{3.432227in}{2.969493in}}%
\pgfpathcurveto{\pgfqpoint{3.432227in}{2.977730in}}{\pgfqpoint{3.428955in}{2.985630in}}{\pgfqpoint{3.423131in}{2.991453in}}%
\pgfpathcurveto{\pgfqpoint{3.417307in}{2.997277in}}{\pgfqpoint{3.409407in}{3.000550in}}{\pgfqpoint{3.401171in}{3.000550in}}%
\pgfpathcurveto{\pgfqpoint{3.392935in}{3.000550in}}{\pgfqpoint{3.385035in}{2.997277in}}{\pgfqpoint{3.379211in}{2.991453in}}%
\pgfpathcurveto{\pgfqpoint{3.373387in}{2.985630in}}{\pgfqpoint{3.370114in}{2.977730in}}{\pgfqpoint{3.370114in}{2.969493in}}%
\pgfpathcurveto{\pgfqpoint{3.370114in}{2.961257in}}{\pgfqpoint{3.373387in}{2.953357in}}{\pgfqpoint{3.379211in}{2.947533in}}%
\pgfpathcurveto{\pgfqpoint{3.385035in}{2.941709in}}{\pgfqpoint{3.392935in}{2.938437in}}{\pgfqpoint{3.401171in}{2.938437in}}%
\pgfpathclose%
\pgfusepath{stroke,fill}%
\end{pgfscope}%
\begin{pgfscope}%
\pgfpathrectangle{\pgfqpoint{0.100000in}{0.220728in}}{\pgfqpoint{3.696000in}{3.696000in}}%
\pgfusepath{clip}%
\pgfsetbuttcap%
\pgfsetroundjoin%
\definecolor{currentfill}{rgb}{0.121569,0.466667,0.705882}%
\pgfsetfillcolor{currentfill}%
\pgfsetfillopacity{0.670025}%
\pgfsetlinewidth{1.003750pt}%
\definecolor{currentstroke}{rgb}{0.121569,0.466667,0.705882}%
\pgfsetstrokecolor{currentstroke}%
\pgfsetstrokeopacity{0.670025}%
\pgfsetdash{}{0pt}%
\pgfpathmoveto{\pgfqpoint{3.401463in}{2.938039in}}%
\pgfpathcurveto{\pgfqpoint{3.409700in}{2.938039in}}{\pgfqpoint{3.417600in}{2.941311in}}{\pgfqpoint{3.423424in}{2.947135in}}%
\pgfpathcurveto{\pgfqpoint{3.429248in}{2.952959in}}{\pgfqpoint{3.432520in}{2.960859in}}{\pgfqpoint{3.432520in}{2.969095in}}%
\pgfpathcurveto{\pgfqpoint{3.432520in}{2.977332in}}{\pgfqpoint{3.429248in}{2.985232in}}{\pgfqpoint{3.423424in}{2.991056in}}%
\pgfpathcurveto{\pgfqpoint{3.417600in}{2.996880in}}{\pgfqpoint{3.409700in}{3.000152in}}{\pgfqpoint{3.401463in}{3.000152in}}%
\pgfpathcurveto{\pgfqpoint{3.393227in}{3.000152in}}{\pgfqpoint{3.385327in}{2.996880in}}{\pgfqpoint{3.379503in}{2.991056in}}%
\pgfpathcurveto{\pgfqpoint{3.373679in}{2.985232in}}{\pgfqpoint{3.370407in}{2.977332in}}{\pgfqpoint{3.370407in}{2.969095in}}%
\pgfpathcurveto{\pgfqpoint{3.370407in}{2.960859in}}{\pgfqpoint{3.373679in}{2.952959in}}{\pgfqpoint{3.379503in}{2.947135in}}%
\pgfpathcurveto{\pgfqpoint{3.385327in}{2.941311in}}{\pgfqpoint{3.393227in}{2.938039in}}{\pgfqpoint{3.401463in}{2.938039in}}%
\pgfpathclose%
\pgfusepath{stroke,fill}%
\end{pgfscope}%
\begin{pgfscope}%
\pgfpathrectangle{\pgfqpoint{0.100000in}{0.220728in}}{\pgfqpoint{3.696000in}{3.696000in}}%
\pgfusepath{clip}%
\pgfsetbuttcap%
\pgfsetroundjoin%
\definecolor{currentfill}{rgb}{0.121569,0.466667,0.705882}%
\pgfsetfillcolor{currentfill}%
\pgfsetfillopacity{0.670113}%
\pgfsetlinewidth{1.003750pt}%
\definecolor{currentstroke}{rgb}{0.121569,0.466667,0.705882}%
\pgfsetstrokecolor{currentstroke}%
\pgfsetstrokeopacity{0.670113}%
\pgfsetdash{}{0pt}%
\pgfpathmoveto{\pgfqpoint{3.401565in}{2.937492in}}%
\pgfpathcurveto{\pgfqpoint{3.409801in}{2.937492in}}{\pgfqpoint{3.417701in}{2.940765in}}{\pgfqpoint{3.423525in}{2.946588in}}%
\pgfpathcurveto{\pgfqpoint{3.429349in}{2.952412in}}{\pgfqpoint{3.432621in}{2.960312in}}{\pgfqpoint{3.432621in}{2.968549in}}%
\pgfpathcurveto{\pgfqpoint{3.432621in}{2.976785in}}{\pgfqpoint{3.429349in}{2.984685in}}{\pgfqpoint{3.423525in}{2.990509in}}%
\pgfpathcurveto{\pgfqpoint{3.417701in}{2.996333in}}{\pgfqpoint{3.409801in}{2.999605in}}{\pgfqpoint{3.401565in}{2.999605in}}%
\pgfpathcurveto{\pgfqpoint{3.393329in}{2.999605in}}{\pgfqpoint{3.385429in}{2.996333in}}{\pgfqpoint{3.379605in}{2.990509in}}%
\pgfpathcurveto{\pgfqpoint{3.373781in}{2.984685in}}{\pgfqpoint{3.370508in}{2.976785in}}{\pgfqpoint{3.370508in}{2.968549in}}%
\pgfpathcurveto{\pgfqpoint{3.370508in}{2.960312in}}{\pgfqpoint{3.373781in}{2.952412in}}{\pgfqpoint{3.379605in}{2.946588in}}%
\pgfpathcurveto{\pgfqpoint{3.385429in}{2.940765in}}{\pgfqpoint{3.393329in}{2.937492in}}{\pgfqpoint{3.401565in}{2.937492in}}%
\pgfpathclose%
\pgfusepath{stroke,fill}%
\end{pgfscope}%
\begin{pgfscope}%
\pgfpathrectangle{\pgfqpoint{0.100000in}{0.220728in}}{\pgfqpoint{3.696000in}{3.696000in}}%
\pgfusepath{clip}%
\pgfsetbuttcap%
\pgfsetroundjoin%
\definecolor{currentfill}{rgb}{0.121569,0.466667,0.705882}%
\pgfsetfillcolor{currentfill}%
\pgfsetfillopacity{0.670406}%
\pgfsetlinewidth{1.003750pt}%
\definecolor{currentstroke}{rgb}{0.121569,0.466667,0.705882}%
\pgfsetstrokecolor{currentstroke}%
\pgfsetstrokeopacity{0.670406}%
\pgfsetdash{}{0pt}%
\pgfpathmoveto{\pgfqpoint{3.401466in}{2.935438in}}%
\pgfpathcurveto{\pgfqpoint{3.409702in}{2.935438in}}{\pgfqpoint{3.417603in}{2.938711in}}{\pgfqpoint{3.423426in}{2.944534in}}%
\pgfpathcurveto{\pgfqpoint{3.429250in}{2.950358in}}{\pgfqpoint{3.432523in}{2.958258in}}{\pgfqpoint{3.432523in}{2.966495in}}%
\pgfpathcurveto{\pgfqpoint{3.432523in}{2.974731in}}{\pgfqpoint{3.429250in}{2.982631in}}{\pgfqpoint{3.423426in}{2.988455in}}%
\pgfpathcurveto{\pgfqpoint{3.417603in}{2.994279in}}{\pgfqpoint{3.409702in}{2.997551in}}{\pgfqpoint{3.401466in}{2.997551in}}%
\pgfpathcurveto{\pgfqpoint{3.393230in}{2.997551in}}{\pgfqpoint{3.385330in}{2.994279in}}{\pgfqpoint{3.379506in}{2.988455in}}%
\pgfpathcurveto{\pgfqpoint{3.373682in}{2.982631in}}{\pgfqpoint{3.370410in}{2.974731in}}{\pgfqpoint{3.370410in}{2.966495in}}%
\pgfpathcurveto{\pgfqpoint{3.370410in}{2.958258in}}{\pgfqpoint{3.373682in}{2.950358in}}{\pgfqpoint{3.379506in}{2.944534in}}%
\pgfpathcurveto{\pgfqpoint{3.385330in}{2.938711in}}{\pgfqpoint{3.393230in}{2.935438in}}{\pgfqpoint{3.401466in}{2.935438in}}%
\pgfpathclose%
\pgfusepath{stroke,fill}%
\end{pgfscope}%
\begin{pgfscope}%
\pgfpathrectangle{\pgfqpoint{0.100000in}{0.220728in}}{\pgfqpoint{3.696000in}{3.696000in}}%
\pgfusepath{clip}%
\pgfsetbuttcap%
\pgfsetroundjoin%
\definecolor{currentfill}{rgb}{0.121569,0.466667,0.705882}%
\pgfsetfillcolor{currentfill}%
\pgfsetfillopacity{0.670869}%
\pgfsetlinewidth{1.003750pt}%
\definecolor{currentstroke}{rgb}{0.121569,0.466667,0.705882}%
\pgfsetstrokecolor{currentstroke}%
\pgfsetstrokeopacity{0.670869}%
\pgfsetdash{}{0pt}%
\pgfpathmoveto{\pgfqpoint{3.400263in}{2.932634in}}%
\pgfpathcurveto{\pgfqpoint{3.408499in}{2.932634in}}{\pgfqpoint{3.416399in}{2.935906in}}{\pgfqpoint{3.422223in}{2.941730in}}%
\pgfpathcurveto{\pgfqpoint{3.428047in}{2.947554in}}{\pgfqpoint{3.431319in}{2.955454in}}{\pgfqpoint{3.431319in}{2.963690in}}%
\pgfpathcurveto{\pgfqpoint{3.431319in}{2.971926in}}{\pgfqpoint{3.428047in}{2.979826in}}{\pgfqpoint{3.422223in}{2.985650in}}%
\pgfpathcurveto{\pgfqpoint{3.416399in}{2.991474in}}{\pgfqpoint{3.408499in}{2.994747in}}{\pgfqpoint{3.400263in}{2.994747in}}%
\pgfpathcurveto{\pgfqpoint{3.392027in}{2.994747in}}{\pgfqpoint{3.384127in}{2.991474in}}{\pgfqpoint{3.378303in}{2.985650in}}%
\pgfpathcurveto{\pgfqpoint{3.372479in}{2.979826in}}{\pgfqpoint{3.369206in}{2.971926in}}{\pgfqpoint{3.369206in}{2.963690in}}%
\pgfpathcurveto{\pgfqpoint{3.369206in}{2.955454in}}{\pgfqpoint{3.372479in}{2.947554in}}{\pgfqpoint{3.378303in}{2.941730in}}%
\pgfpathcurveto{\pgfqpoint{3.384127in}{2.935906in}}{\pgfqpoint{3.392027in}{2.932634in}}{\pgfqpoint{3.400263in}{2.932634in}}%
\pgfpathclose%
\pgfusepath{stroke,fill}%
\end{pgfscope}%
\begin{pgfscope}%
\pgfpathrectangle{\pgfqpoint{0.100000in}{0.220728in}}{\pgfqpoint{3.696000in}{3.696000in}}%
\pgfusepath{clip}%
\pgfsetbuttcap%
\pgfsetroundjoin%
\definecolor{currentfill}{rgb}{0.121569,0.466667,0.705882}%
\pgfsetfillcolor{currentfill}%
\pgfsetfillopacity{0.671231}%
\pgfsetlinewidth{1.003750pt}%
\definecolor{currentstroke}{rgb}{0.121569,0.466667,0.705882}%
\pgfsetstrokecolor{currentstroke}%
\pgfsetstrokeopacity{0.671231}%
\pgfsetdash{}{0pt}%
\pgfpathmoveto{\pgfqpoint{0.690190in}{1.426879in}}%
\pgfpathcurveto{\pgfqpoint{0.698426in}{1.426879in}}{\pgfqpoint{0.706326in}{1.430152in}}{\pgfqpoint{0.712150in}{1.435976in}}%
\pgfpathcurveto{\pgfqpoint{0.717974in}{1.441800in}}{\pgfqpoint{0.721247in}{1.449700in}}{\pgfqpoint{0.721247in}{1.457936in}}%
\pgfpathcurveto{\pgfqpoint{0.721247in}{1.466172in}}{\pgfqpoint{0.717974in}{1.474072in}}{\pgfqpoint{0.712150in}{1.479896in}}%
\pgfpathcurveto{\pgfqpoint{0.706326in}{1.485720in}}{\pgfqpoint{0.698426in}{1.488992in}}{\pgfqpoint{0.690190in}{1.488992in}}%
\pgfpathcurveto{\pgfqpoint{0.681954in}{1.488992in}}{\pgfqpoint{0.674054in}{1.485720in}}{\pgfqpoint{0.668230in}{1.479896in}}%
\pgfpathcurveto{\pgfqpoint{0.662406in}{1.474072in}}{\pgfqpoint{0.659134in}{1.466172in}}{\pgfqpoint{0.659134in}{1.457936in}}%
\pgfpathcurveto{\pgfqpoint{0.659134in}{1.449700in}}{\pgfqpoint{0.662406in}{1.441800in}}{\pgfqpoint{0.668230in}{1.435976in}}%
\pgfpathcurveto{\pgfqpoint{0.674054in}{1.430152in}}{\pgfqpoint{0.681954in}{1.426879in}}{\pgfqpoint{0.690190in}{1.426879in}}%
\pgfpathclose%
\pgfusepath{stroke,fill}%
\end{pgfscope}%
\begin{pgfscope}%
\pgfpathrectangle{\pgfqpoint{0.100000in}{0.220728in}}{\pgfqpoint{3.696000in}{3.696000in}}%
\pgfusepath{clip}%
\pgfsetbuttcap%
\pgfsetroundjoin%
\definecolor{currentfill}{rgb}{0.121569,0.466667,0.705882}%
\pgfsetfillcolor{currentfill}%
\pgfsetfillopacity{0.671478}%
\pgfsetlinewidth{1.003750pt}%
\definecolor{currentstroke}{rgb}{0.121569,0.466667,0.705882}%
\pgfsetstrokecolor{currentstroke}%
\pgfsetstrokeopacity{0.671478}%
\pgfsetdash{}{0pt}%
\pgfpathmoveto{\pgfqpoint{3.397538in}{2.927401in}}%
\pgfpathcurveto{\pgfqpoint{3.405774in}{2.927401in}}{\pgfqpoint{3.413674in}{2.930674in}}{\pgfqpoint{3.419498in}{2.936498in}}%
\pgfpathcurveto{\pgfqpoint{3.425322in}{2.942322in}}{\pgfqpoint{3.428594in}{2.950222in}}{\pgfqpoint{3.428594in}{2.958458in}}%
\pgfpathcurveto{\pgfqpoint{3.428594in}{2.966694in}}{\pgfqpoint{3.425322in}{2.974594in}}{\pgfqpoint{3.419498in}{2.980418in}}%
\pgfpathcurveto{\pgfqpoint{3.413674in}{2.986242in}}{\pgfqpoint{3.405774in}{2.989514in}}{\pgfqpoint{3.397538in}{2.989514in}}%
\pgfpathcurveto{\pgfqpoint{3.389301in}{2.989514in}}{\pgfqpoint{3.381401in}{2.986242in}}{\pgfqpoint{3.375577in}{2.980418in}}%
\pgfpathcurveto{\pgfqpoint{3.369753in}{2.974594in}}{\pgfqpoint{3.366481in}{2.966694in}}{\pgfqpoint{3.366481in}{2.958458in}}%
\pgfpathcurveto{\pgfqpoint{3.366481in}{2.950222in}}{\pgfqpoint{3.369753in}{2.942322in}}{\pgfqpoint{3.375577in}{2.936498in}}%
\pgfpathcurveto{\pgfqpoint{3.381401in}{2.930674in}}{\pgfqpoint{3.389301in}{2.927401in}}{\pgfqpoint{3.397538in}{2.927401in}}%
\pgfpathclose%
\pgfusepath{stroke,fill}%
\end{pgfscope}%
\begin{pgfscope}%
\pgfpathrectangle{\pgfqpoint{0.100000in}{0.220728in}}{\pgfqpoint{3.696000in}{3.696000in}}%
\pgfusepath{clip}%
\pgfsetbuttcap%
\pgfsetroundjoin%
\definecolor{currentfill}{rgb}{0.121569,0.466667,0.705882}%
\pgfsetfillcolor{currentfill}%
\pgfsetfillopacity{0.672272}%
\pgfsetlinewidth{1.003750pt}%
\definecolor{currentstroke}{rgb}{0.121569,0.466667,0.705882}%
\pgfsetstrokecolor{currentstroke}%
\pgfsetstrokeopacity{0.672272}%
\pgfsetdash{}{0pt}%
\pgfpathmoveto{\pgfqpoint{3.394075in}{2.922201in}}%
\pgfpathcurveto{\pgfqpoint{3.402312in}{2.922201in}}{\pgfqpoint{3.410212in}{2.925473in}}{\pgfqpoint{3.416035in}{2.931297in}}%
\pgfpathcurveto{\pgfqpoint{3.421859in}{2.937121in}}{\pgfqpoint{3.425132in}{2.945021in}}{\pgfqpoint{3.425132in}{2.953258in}}%
\pgfpathcurveto{\pgfqpoint{3.425132in}{2.961494in}}{\pgfqpoint{3.421859in}{2.969394in}}{\pgfqpoint{3.416035in}{2.975218in}}%
\pgfpathcurveto{\pgfqpoint{3.410212in}{2.981042in}}{\pgfqpoint{3.402312in}{2.984314in}}{\pgfqpoint{3.394075in}{2.984314in}}%
\pgfpathcurveto{\pgfqpoint{3.385839in}{2.984314in}}{\pgfqpoint{3.377939in}{2.981042in}}{\pgfqpoint{3.372115in}{2.975218in}}%
\pgfpathcurveto{\pgfqpoint{3.366291in}{2.969394in}}{\pgfqpoint{3.363019in}{2.961494in}}{\pgfqpoint{3.363019in}{2.953258in}}%
\pgfpathcurveto{\pgfqpoint{3.363019in}{2.945021in}}{\pgfqpoint{3.366291in}{2.937121in}}{\pgfqpoint{3.372115in}{2.931297in}}%
\pgfpathcurveto{\pgfqpoint{3.377939in}{2.925473in}}{\pgfqpoint{3.385839in}{2.922201in}}{\pgfqpoint{3.394075in}{2.922201in}}%
\pgfpathclose%
\pgfusepath{stroke,fill}%
\end{pgfscope}%
\begin{pgfscope}%
\pgfpathrectangle{\pgfqpoint{0.100000in}{0.220728in}}{\pgfqpoint{3.696000in}{3.696000in}}%
\pgfusepath{clip}%
\pgfsetbuttcap%
\pgfsetroundjoin%
\definecolor{currentfill}{rgb}{0.121569,0.466667,0.705882}%
\pgfsetfillcolor{currentfill}%
\pgfsetfillopacity{0.672734}%
\pgfsetlinewidth{1.003750pt}%
\definecolor{currentstroke}{rgb}{0.121569,0.466667,0.705882}%
\pgfsetstrokecolor{currentstroke}%
\pgfsetstrokeopacity{0.672734}%
\pgfsetdash{}{0pt}%
\pgfpathmoveto{\pgfqpoint{3.392438in}{2.919032in}}%
\pgfpathcurveto{\pgfqpoint{3.400674in}{2.919032in}}{\pgfqpoint{3.408574in}{2.922305in}}{\pgfqpoint{3.414398in}{2.928129in}}%
\pgfpathcurveto{\pgfqpoint{3.420222in}{2.933953in}}{\pgfqpoint{3.423494in}{2.941853in}}{\pgfqpoint{3.423494in}{2.950089in}}%
\pgfpathcurveto{\pgfqpoint{3.423494in}{2.958325in}}{\pgfqpoint{3.420222in}{2.966225in}}{\pgfqpoint{3.414398in}{2.972049in}}%
\pgfpathcurveto{\pgfqpoint{3.408574in}{2.977873in}}{\pgfqpoint{3.400674in}{2.981145in}}{\pgfqpoint{3.392438in}{2.981145in}}%
\pgfpathcurveto{\pgfqpoint{3.384202in}{2.981145in}}{\pgfqpoint{3.376302in}{2.977873in}}{\pgfqpoint{3.370478in}{2.972049in}}%
\pgfpathcurveto{\pgfqpoint{3.364654in}{2.966225in}}{\pgfqpoint{3.361381in}{2.958325in}}{\pgfqpoint{3.361381in}{2.950089in}}%
\pgfpathcurveto{\pgfqpoint{3.361381in}{2.941853in}}{\pgfqpoint{3.364654in}{2.933953in}}{\pgfqpoint{3.370478in}{2.928129in}}%
\pgfpathcurveto{\pgfqpoint{3.376302in}{2.922305in}}{\pgfqpoint{3.384202in}{2.919032in}}{\pgfqpoint{3.392438in}{2.919032in}}%
\pgfpathclose%
\pgfusepath{stroke,fill}%
\end{pgfscope}%
\begin{pgfscope}%
\pgfpathrectangle{\pgfqpoint{0.100000in}{0.220728in}}{\pgfqpoint{3.696000in}{3.696000in}}%
\pgfusepath{clip}%
\pgfsetbuttcap%
\pgfsetroundjoin%
\definecolor{currentfill}{rgb}{0.121569,0.466667,0.705882}%
\pgfsetfillcolor{currentfill}%
\pgfsetfillopacity{0.672977}%
\pgfsetlinewidth{1.003750pt}%
\definecolor{currentstroke}{rgb}{0.121569,0.466667,0.705882}%
\pgfsetstrokecolor{currentstroke}%
\pgfsetstrokeopacity{0.672977}%
\pgfsetdash{}{0pt}%
\pgfpathmoveto{\pgfqpoint{3.391414in}{2.917432in}}%
\pgfpathcurveto{\pgfqpoint{3.399650in}{2.917432in}}{\pgfqpoint{3.407550in}{2.920704in}}{\pgfqpoint{3.413374in}{2.926528in}}%
\pgfpathcurveto{\pgfqpoint{3.419198in}{2.932352in}}{\pgfqpoint{3.422470in}{2.940252in}}{\pgfqpoint{3.422470in}{2.948488in}}%
\pgfpathcurveto{\pgfqpoint{3.422470in}{2.956725in}}{\pgfqpoint{3.419198in}{2.964625in}}{\pgfqpoint{3.413374in}{2.970449in}}%
\pgfpathcurveto{\pgfqpoint{3.407550in}{2.976273in}}{\pgfqpoint{3.399650in}{2.979545in}}{\pgfqpoint{3.391414in}{2.979545in}}%
\pgfpathcurveto{\pgfqpoint{3.383178in}{2.979545in}}{\pgfqpoint{3.375278in}{2.976273in}}{\pgfqpoint{3.369454in}{2.970449in}}%
\pgfpathcurveto{\pgfqpoint{3.363630in}{2.964625in}}{\pgfqpoint{3.360357in}{2.956725in}}{\pgfqpoint{3.360357in}{2.948488in}}%
\pgfpathcurveto{\pgfqpoint{3.360357in}{2.940252in}}{\pgfqpoint{3.363630in}{2.932352in}}{\pgfqpoint{3.369454in}{2.926528in}}%
\pgfpathcurveto{\pgfqpoint{3.375278in}{2.920704in}}{\pgfqpoint{3.383178in}{2.917432in}}{\pgfqpoint{3.391414in}{2.917432in}}%
\pgfpathclose%
\pgfusepath{stroke,fill}%
\end{pgfscope}%
\begin{pgfscope}%
\pgfpathrectangle{\pgfqpoint{0.100000in}{0.220728in}}{\pgfqpoint{3.696000in}{3.696000in}}%
\pgfusepath{clip}%
\pgfsetbuttcap%
\pgfsetroundjoin%
\definecolor{currentfill}{rgb}{0.121569,0.466667,0.705882}%
\pgfsetfillcolor{currentfill}%
\pgfsetfillopacity{0.673117}%
\pgfsetlinewidth{1.003750pt}%
\definecolor{currentstroke}{rgb}{0.121569,0.466667,0.705882}%
\pgfsetstrokecolor{currentstroke}%
\pgfsetstrokeopacity{0.673117}%
\pgfsetdash{}{0pt}%
\pgfpathmoveto{\pgfqpoint{3.390887in}{2.916521in}}%
\pgfpathcurveto{\pgfqpoint{3.399123in}{2.916521in}}{\pgfqpoint{3.407023in}{2.919793in}}{\pgfqpoint{3.412847in}{2.925617in}}%
\pgfpathcurveto{\pgfqpoint{3.418671in}{2.931441in}}{\pgfqpoint{3.421943in}{2.939341in}}{\pgfqpoint{3.421943in}{2.947578in}}%
\pgfpathcurveto{\pgfqpoint{3.421943in}{2.955814in}}{\pgfqpoint{3.418671in}{2.963714in}}{\pgfqpoint{3.412847in}{2.969538in}}%
\pgfpathcurveto{\pgfqpoint{3.407023in}{2.975362in}}{\pgfqpoint{3.399123in}{2.978634in}}{\pgfqpoint{3.390887in}{2.978634in}}%
\pgfpathcurveto{\pgfqpoint{3.382650in}{2.978634in}}{\pgfqpoint{3.374750in}{2.975362in}}{\pgfqpoint{3.368926in}{2.969538in}}%
\pgfpathcurveto{\pgfqpoint{3.363102in}{2.963714in}}{\pgfqpoint{3.359830in}{2.955814in}}{\pgfqpoint{3.359830in}{2.947578in}}%
\pgfpathcurveto{\pgfqpoint{3.359830in}{2.939341in}}{\pgfqpoint{3.363102in}{2.931441in}}{\pgfqpoint{3.368926in}{2.925617in}}%
\pgfpathcurveto{\pgfqpoint{3.374750in}{2.919793in}}{\pgfqpoint{3.382650in}{2.916521in}}{\pgfqpoint{3.390887in}{2.916521in}}%
\pgfpathclose%
\pgfusepath{stroke,fill}%
\end{pgfscope}%
\begin{pgfscope}%
\pgfpathrectangle{\pgfqpoint{0.100000in}{0.220728in}}{\pgfqpoint{3.696000in}{3.696000in}}%
\pgfusepath{clip}%
\pgfsetbuttcap%
\pgfsetroundjoin%
\definecolor{currentfill}{rgb}{0.121569,0.466667,0.705882}%
\pgfsetfillcolor{currentfill}%
\pgfsetfillopacity{0.673209}%
\pgfsetlinewidth{1.003750pt}%
\definecolor{currentstroke}{rgb}{0.121569,0.466667,0.705882}%
\pgfsetstrokecolor{currentstroke}%
\pgfsetstrokeopacity{0.673209}%
\pgfsetdash{}{0pt}%
\pgfpathmoveto{\pgfqpoint{3.390607in}{2.916071in}}%
\pgfpathcurveto{\pgfqpoint{3.398844in}{2.916071in}}{\pgfqpoint{3.406744in}{2.919343in}}{\pgfqpoint{3.412568in}{2.925167in}}%
\pgfpathcurveto{\pgfqpoint{3.418392in}{2.930991in}}{\pgfqpoint{3.421664in}{2.938891in}}{\pgfqpoint{3.421664in}{2.947127in}}%
\pgfpathcurveto{\pgfqpoint{3.421664in}{2.955363in}}{\pgfqpoint{3.418392in}{2.963263in}}{\pgfqpoint{3.412568in}{2.969087in}}%
\pgfpathcurveto{\pgfqpoint{3.406744in}{2.974911in}}{\pgfqpoint{3.398844in}{2.978184in}}{\pgfqpoint{3.390607in}{2.978184in}}%
\pgfpathcurveto{\pgfqpoint{3.382371in}{2.978184in}}{\pgfqpoint{3.374471in}{2.974911in}}{\pgfqpoint{3.368647in}{2.969087in}}%
\pgfpathcurveto{\pgfqpoint{3.362823in}{2.963263in}}{\pgfqpoint{3.359551in}{2.955363in}}{\pgfqpoint{3.359551in}{2.947127in}}%
\pgfpathcurveto{\pgfqpoint{3.359551in}{2.938891in}}{\pgfqpoint{3.362823in}{2.930991in}}{\pgfqpoint{3.368647in}{2.925167in}}%
\pgfpathcurveto{\pgfqpoint{3.374471in}{2.919343in}}{\pgfqpoint{3.382371in}{2.916071in}}{\pgfqpoint{3.390607in}{2.916071in}}%
\pgfpathclose%
\pgfusepath{stroke,fill}%
\end{pgfscope}%
\begin{pgfscope}%
\pgfpathrectangle{\pgfqpoint{0.100000in}{0.220728in}}{\pgfqpoint{3.696000in}{3.696000in}}%
\pgfusepath{clip}%
\pgfsetbuttcap%
\pgfsetroundjoin%
\definecolor{currentfill}{rgb}{0.121569,0.466667,0.705882}%
\pgfsetfillcolor{currentfill}%
\pgfsetfillopacity{0.673258}%
\pgfsetlinewidth{1.003750pt}%
\definecolor{currentstroke}{rgb}{0.121569,0.466667,0.705882}%
\pgfsetstrokecolor{currentstroke}%
\pgfsetstrokeopacity{0.673258}%
\pgfsetdash{}{0pt}%
\pgfpathmoveto{\pgfqpoint{0.698559in}{1.423949in}}%
\pgfpathcurveto{\pgfqpoint{0.706795in}{1.423949in}}{\pgfqpoint{0.714695in}{1.427222in}}{\pgfqpoint{0.720519in}{1.433046in}}%
\pgfpathcurveto{\pgfqpoint{0.726343in}{1.438870in}}{\pgfqpoint{0.729616in}{1.446770in}}{\pgfqpoint{0.729616in}{1.455006in}}%
\pgfpathcurveto{\pgfqpoint{0.729616in}{1.463242in}}{\pgfqpoint{0.726343in}{1.471142in}}{\pgfqpoint{0.720519in}{1.476966in}}%
\pgfpathcurveto{\pgfqpoint{0.714695in}{1.482790in}}{\pgfqpoint{0.706795in}{1.486062in}}{\pgfqpoint{0.698559in}{1.486062in}}%
\pgfpathcurveto{\pgfqpoint{0.690323in}{1.486062in}}{\pgfqpoint{0.682423in}{1.482790in}}{\pgfqpoint{0.676599in}{1.476966in}}%
\pgfpathcurveto{\pgfqpoint{0.670775in}{1.471142in}}{\pgfqpoint{0.667503in}{1.463242in}}{\pgfqpoint{0.667503in}{1.455006in}}%
\pgfpathcurveto{\pgfqpoint{0.667503in}{1.446770in}}{\pgfqpoint{0.670775in}{1.438870in}}{\pgfqpoint{0.676599in}{1.433046in}}%
\pgfpathcurveto{\pgfqpoint{0.682423in}{1.427222in}}{\pgfqpoint{0.690323in}{1.423949in}}{\pgfqpoint{0.698559in}{1.423949in}}%
\pgfpathclose%
\pgfusepath{stroke,fill}%
\end{pgfscope}%
\begin{pgfscope}%
\pgfpathrectangle{\pgfqpoint{0.100000in}{0.220728in}}{\pgfqpoint{3.696000in}{3.696000in}}%
\pgfusepath{clip}%
\pgfsetbuttcap%
\pgfsetroundjoin%
\definecolor{currentfill}{rgb}{0.121569,0.466667,0.705882}%
\pgfsetfillcolor{currentfill}%
\pgfsetfillopacity{0.673259}%
\pgfsetlinewidth{1.003750pt}%
\definecolor{currentstroke}{rgb}{0.121569,0.466667,0.705882}%
\pgfsetstrokecolor{currentstroke}%
\pgfsetstrokeopacity{0.673259}%
\pgfsetdash{}{0pt}%
\pgfpathmoveto{\pgfqpoint{3.390453in}{2.915821in}}%
\pgfpathcurveto{\pgfqpoint{3.398690in}{2.915821in}}{\pgfqpoint{3.406590in}{2.919093in}}{\pgfqpoint{3.412414in}{2.924917in}}%
\pgfpathcurveto{\pgfqpoint{3.418238in}{2.930741in}}{\pgfqpoint{3.421510in}{2.938641in}}{\pgfqpoint{3.421510in}{2.946878in}}%
\pgfpathcurveto{\pgfqpoint{3.421510in}{2.955114in}}{\pgfqpoint{3.418238in}{2.963014in}}{\pgfqpoint{3.412414in}{2.968838in}}%
\pgfpathcurveto{\pgfqpoint{3.406590in}{2.974662in}}{\pgfqpoint{3.398690in}{2.977934in}}{\pgfqpoint{3.390453in}{2.977934in}}%
\pgfpathcurveto{\pgfqpoint{3.382217in}{2.977934in}}{\pgfqpoint{3.374317in}{2.974662in}}{\pgfqpoint{3.368493in}{2.968838in}}%
\pgfpathcurveto{\pgfqpoint{3.362669in}{2.963014in}}{\pgfqpoint{3.359397in}{2.955114in}}{\pgfqpoint{3.359397in}{2.946878in}}%
\pgfpathcurveto{\pgfqpoint{3.359397in}{2.938641in}}{\pgfqpoint{3.362669in}{2.930741in}}{\pgfqpoint{3.368493in}{2.924917in}}%
\pgfpathcurveto{\pgfqpoint{3.374317in}{2.919093in}}{\pgfqpoint{3.382217in}{2.915821in}}{\pgfqpoint{3.390453in}{2.915821in}}%
\pgfpathclose%
\pgfusepath{stroke,fill}%
\end{pgfscope}%
\begin{pgfscope}%
\pgfpathrectangle{\pgfqpoint{0.100000in}{0.220728in}}{\pgfqpoint{3.696000in}{3.696000in}}%
\pgfusepath{clip}%
\pgfsetbuttcap%
\pgfsetroundjoin%
\definecolor{currentfill}{rgb}{0.121569,0.466667,0.705882}%
\pgfsetfillcolor{currentfill}%
\pgfsetfillopacity{0.673281}%
\pgfsetlinewidth{1.003750pt}%
\definecolor{currentstroke}{rgb}{0.121569,0.466667,0.705882}%
\pgfsetstrokecolor{currentstroke}%
\pgfsetstrokeopacity{0.673281}%
\pgfsetdash{}{0pt}%
\pgfpathmoveto{\pgfqpoint{3.390351in}{2.915686in}}%
\pgfpathcurveto{\pgfqpoint{3.398587in}{2.915686in}}{\pgfqpoint{3.406487in}{2.918958in}}{\pgfqpoint{3.412311in}{2.924782in}}%
\pgfpathcurveto{\pgfqpoint{3.418135in}{2.930606in}}{\pgfqpoint{3.421407in}{2.938506in}}{\pgfqpoint{3.421407in}{2.946742in}}%
\pgfpathcurveto{\pgfqpoint{3.421407in}{2.954979in}}{\pgfqpoint{3.418135in}{2.962879in}}{\pgfqpoint{3.412311in}{2.968703in}}%
\pgfpathcurveto{\pgfqpoint{3.406487in}{2.974527in}}{\pgfqpoint{3.398587in}{2.977799in}}{\pgfqpoint{3.390351in}{2.977799in}}%
\pgfpathcurveto{\pgfqpoint{3.382115in}{2.977799in}}{\pgfqpoint{3.374215in}{2.974527in}}{\pgfqpoint{3.368391in}{2.968703in}}%
\pgfpathcurveto{\pgfqpoint{3.362567in}{2.962879in}}{\pgfqpoint{3.359294in}{2.954979in}}{\pgfqpoint{3.359294in}{2.946742in}}%
\pgfpathcurveto{\pgfqpoint{3.359294in}{2.938506in}}{\pgfqpoint{3.362567in}{2.930606in}}{\pgfqpoint{3.368391in}{2.924782in}}%
\pgfpathcurveto{\pgfqpoint{3.374215in}{2.918958in}}{\pgfqpoint{3.382115in}{2.915686in}}{\pgfqpoint{3.390351in}{2.915686in}}%
\pgfpathclose%
\pgfusepath{stroke,fill}%
\end{pgfscope}%
\begin{pgfscope}%
\pgfpathrectangle{\pgfqpoint{0.100000in}{0.220728in}}{\pgfqpoint{3.696000in}{3.696000in}}%
\pgfusepath{clip}%
\pgfsetbuttcap%
\pgfsetroundjoin%
\definecolor{currentfill}{rgb}{0.121569,0.466667,0.705882}%
\pgfsetfillcolor{currentfill}%
\pgfsetfillopacity{0.673297}%
\pgfsetlinewidth{1.003750pt}%
\definecolor{currentstroke}{rgb}{0.121569,0.466667,0.705882}%
\pgfsetstrokecolor{currentstroke}%
\pgfsetstrokeopacity{0.673297}%
\pgfsetdash{}{0pt}%
\pgfpathmoveto{\pgfqpoint{3.390315in}{2.915604in}}%
\pgfpathcurveto{\pgfqpoint{3.398551in}{2.915604in}}{\pgfqpoint{3.406451in}{2.918877in}}{\pgfqpoint{3.412275in}{2.924701in}}%
\pgfpathcurveto{\pgfqpoint{3.418099in}{2.930524in}}{\pgfqpoint{3.421371in}{2.938424in}}{\pgfqpoint{3.421371in}{2.946661in}}%
\pgfpathcurveto{\pgfqpoint{3.421371in}{2.954897in}}{\pgfqpoint{3.418099in}{2.962797in}}{\pgfqpoint{3.412275in}{2.968621in}}%
\pgfpathcurveto{\pgfqpoint{3.406451in}{2.974445in}}{\pgfqpoint{3.398551in}{2.977717in}}{\pgfqpoint{3.390315in}{2.977717in}}%
\pgfpathcurveto{\pgfqpoint{3.382079in}{2.977717in}}{\pgfqpoint{3.374178in}{2.974445in}}{\pgfqpoint{3.368355in}{2.968621in}}%
\pgfpathcurveto{\pgfqpoint{3.362531in}{2.962797in}}{\pgfqpoint{3.359258in}{2.954897in}}{\pgfqpoint{3.359258in}{2.946661in}}%
\pgfpathcurveto{\pgfqpoint{3.359258in}{2.938424in}}{\pgfqpoint{3.362531in}{2.930524in}}{\pgfqpoint{3.368355in}{2.924701in}}%
\pgfpathcurveto{\pgfqpoint{3.374178in}{2.918877in}}{\pgfqpoint{3.382079in}{2.915604in}}{\pgfqpoint{3.390315in}{2.915604in}}%
\pgfpathclose%
\pgfusepath{stroke,fill}%
\end{pgfscope}%
\begin{pgfscope}%
\pgfpathrectangle{\pgfqpoint{0.100000in}{0.220728in}}{\pgfqpoint{3.696000in}{3.696000in}}%
\pgfusepath{clip}%
\pgfsetbuttcap%
\pgfsetroundjoin%
\definecolor{currentfill}{rgb}{0.121569,0.466667,0.705882}%
\pgfsetfillcolor{currentfill}%
\pgfsetfillopacity{0.673401}%
\pgfsetlinewidth{1.003750pt}%
\definecolor{currentstroke}{rgb}{0.121569,0.466667,0.705882}%
\pgfsetstrokecolor{currentstroke}%
\pgfsetstrokeopacity{0.673401}%
\pgfsetdash{}{0pt}%
\pgfpathmoveto{\pgfqpoint{3.389807in}{2.914906in}}%
\pgfpathcurveto{\pgfqpoint{3.398043in}{2.914906in}}{\pgfqpoint{3.405943in}{2.918179in}}{\pgfqpoint{3.411767in}{2.924003in}}%
\pgfpathcurveto{\pgfqpoint{3.417591in}{2.929827in}}{\pgfqpoint{3.420863in}{2.937727in}}{\pgfqpoint{3.420863in}{2.945963in}}%
\pgfpathcurveto{\pgfqpoint{3.420863in}{2.954199in}}{\pgfqpoint{3.417591in}{2.962099in}}{\pgfqpoint{3.411767in}{2.967923in}}%
\pgfpathcurveto{\pgfqpoint{3.405943in}{2.973747in}}{\pgfqpoint{3.398043in}{2.977019in}}{\pgfqpoint{3.389807in}{2.977019in}}%
\pgfpathcurveto{\pgfqpoint{3.381571in}{2.977019in}}{\pgfqpoint{3.373671in}{2.973747in}}{\pgfqpoint{3.367847in}{2.967923in}}%
\pgfpathcurveto{\pgfqpoint{3.362023in}{2.962099in}}{\pgfqpoint{3.358750in}{2.954199in}}{\pgfqpoint{3.358750in}{2.945963in}}%
\pgfpathcurveto{\pgfqpoint{3.358750in}{2.937727in}}{\pgfqpoint{3.362023in}{2.929827in}}{\pgfqpoint{3.367847in}{2.924003in}}%
\pgfpathcurveto{\pgfqpoint{3.373671in}{2.918179in}}{\pgfqpoint{3.381571in}{2.914906in}}{\pgfqpoint{3.389807in}{2.914906in}}%
\pgfpathclose%
\pgfusepath{stroke,fill}%
\end{pgfscope}%
\begin{pgfscope}%
\pgfpathrectangle{\pgfqpoint{0.100000in}{0.220728in}}{\pgfqpoint{3.696000in}{3.696000in}}%
\pgfusepath{clip}%
\pgfsetbuttcap%
\pgfsetroundjoin%
\definecolor{currentfill}{rgb}{0.121569,0.466667,0.705882}%
\pgfsetfillcolor{currentfill}%
\pgfsetfillopacity{0.673914}%
\pgfsetlinewidth{1.003750pt}%
\definecolor{currentstroke}{rgb}{0.121569,0.466667,0.705882}%
\pgfsetstrokecolor{currentstroke}%
\pgfsetstrokeopacity{0.673914}%
\pgfsetdash{}{0pt}%
\pgfpathmoveto{\pgfqpoint{3.388657in}{2.912776in}}%
\pgfpathcurveto{\pgfqpoint{3.396894in}{2.912776in}}{\pgfqpoint{3.404794in}{2.916048in}}{\pgfqpoint{3.410618in}{2.921872in}}%
\pgfpathcurveto{\pgfqpoint{3.416441in}{2.927696in}}{\pgfqpoint{3.419714in}{2.935596in}}{\pgfqpoint{3.419714in}{2.943832in}}%
\pgfpathcurveto{\pgfqpoint{3.419714in}{2.952069in}}{\pgfqpoint{3.416441in}{2.959969in}}{\pgfqpoint{3.410618in}{2.965793in}}%
\pgfpathcurveto{\pgfqpoint{3.404794in}{2.971617in}}{\pgfqpoint{3.396894in}{2.974889in}}{\pgfqpoint{3.388657in}{2.974889in}}%
\pgfpathcurveto{\pgfqpoint{3.380421in}{2.974889in}}{\pgfqpoint{3.372521in}{2.971617in}}{\pgfqpoint{3.366697in}{2.965793in}}%
\pgfpathcurveto{\pgfqpoint{3.360873in}{2.959969in}}{\pgfqpoint{3.357601in}{2.952069in}}{\pgfqpoint{3.357601in}{2.943832in}}%
\pgfpathcurveto{\pgfqpoint{3.357601in}{2.935596in}}{\pgfqpoint{3.360873in}{2.927696in}}{\pgfqpoint{3.366697in}{2.921872in}}%
\pgfpathcurveto{\pgfqpoint{3.372521in}{2.916048in}}{\pgfqpoint{3.380421in}{2.912776in}}{\pgfqpoint{3.388657in}{2.912776in}}%
\pgfpathclose%
\pgfusepath{stroke,fill}%
\end{pgfscope}%
\begin{pgfscope}%
\pgfpathrectangle{\pgfqpoint{0.100000in}{0.220728in}}{\pgfqpoint{3.696000in}{3.696000in}}%
\pgfusepath{clip}%
\pgfsetbuttcap%
\pgfsetroundjoin%
\definecolor{currentfill}{rgb}{0.121569,0.466667,0.705882}%
\pgfsetfillcolor{currentfill}%
\pgfsetfillopacity{0.674380}%
\pgfsetlinewidth{1.003750pt}%
\definecolor{currentstroke}{rgb}{0.121569,0.466667,0.705882}%
\pgfsetstrokecolor{currentstroke}%
\pgfsetstrokeopacity{0.674380}%
\pgfsetdash{}{0pt}%
\pgfpathmoveto{\pgfqpoint{3.386943in}{2.909837in}}%
\pgfpathcurveto{\pgfqpoint{3.395180in}{2.909837in}}{\pgfqpoint{3.403080in}{2.913109in}}{\pgfqpoint{3.408904in}{2.918933in}}%
\pgfpathcurveto{\pgfqpoint{3.414727in}{2.924757in}}{\pgfqpoint{3.418000in}{2.932657in}}{\pgfqpoint{3.418000in}{2.940894in}}%
\pgfpathcurveto{\pgfqpoint{3.418000in}{2.949130in}}{\pgfqpoint{3.414727in}{2.957030in}}{\pgfqpoint{3.408904in}{2.962854in}}%
\pgfpathcurveto{\pgfqpoint{3.403080in}{2.968678in}}{\pgfqpoint{3.395180in}{2.971950in}}{\pgfqpoint{3.386943in}{2.971950in}}%
\pgfpathcurveto{\pgfqpoint{3.378707in}{2.971950in}}{\pgfqpoint{3.370807in}{2.968678in}}{\pgfqpoint{3.364983in}{2.962854in}}%
\pgfpathcurveto{\pgfqpoint{3.359159in}{2.957030in}}{\pgfqpoint{3.355887in}{2.949130in}}{\pgfqpoint{3.355887in}{2.940894in}}%
\pgfpathcurveto{\pgfqpoint{3.355887in}{2.932657in}}{\pgfqpoint{3.359159in}{2.924757in}}{\pgfqpoint{3.364983in}{2.918933in}}%
\pgfpathcurveto{\pgfqpoint{3.370807in}{2.913109in}}{\pgfqpoint{3.378707in}{2.909837in}}{\pgfqpoint{3.386943in}{2.909837in}}%
\pgfpathclose%
\pgfusepath{stroke,fill}%
\end{pgfscope}%
\begin{pgfscope}%
\pgfpathrectangle{\pgfqpoint{0.100000in}{0.220728in}}{\pgfqpoint{3.696000in}{3.696000in}}%
\pgfusepath{clip}%
\pgfsetbuttcap%
\pgfsetroundjoin%
\definecolor{currentfill}{rgb}{0.121569,0.466667,0.705882}%
\pgfsetfillcolor{currentfill}%
\pgfsetfillopacity{0.674587}%
\pgfsetlinewidth{1.003750pt}%
\definecolor{currentstroke}{rgb}{0.121569,0.466667,0.705882}%
\pgfsetstrokecolor{currentstroke}%
\pgfsetstrokeopacity{0.674587}%
\pgfsetdash{}{0pt}%
\pgfpathmoveto{\pgfqpoint{3.385816in}{2.908337in}}%
\pgfpathcurveto{\pgfqpoint{3.394052in}{2.908337in}}{\pgfqpoint{3.401952in}{2.911609in}}{\pgfqpoint{3.407776in}{2.917433in}}%
\pgfpathcurveto{\pgfqpoint{3.413600in}{2.923257in}}{\pgfqpoint{3.416872in}{2.931157in}}{\pgfqpoint{3.416872in}{2.939393in}}%
\pgfpathcurveto{\pgfqpoint{3.416872in}{2.947630in}}{\pgfqpoint{3.413600in}{2.955530in}}{\pgfqpoint{3.407776in}{2.961354in}}%
\pgfpathcurveto{\pgfqpoint{3.401952in}{2.967178in}}{\pgfqpoint{3.394052in}{2.970450in}}{\pgfqpoint{3.385816in}{2.970450in}}%
\pgfpathcurveto{\pgfqpoint{3.377579in}{2.970450in}}{\pgfqpoint{3.369679in}{2.967178in}}{\pgfqpoint{3.363855in}{2.961354in}}%
\pgfpathcurveto{\pgfqpoint{3.358031in}{2.955530in}}{\pgfqpoint{3.354759in}{2.947630in}}{\pgfqpoint{3.354759in}{2.939393in}}%
\pgfpathcurveto{\pgfqpoint{3.354759in}{2.931157in}}{\pgfqpoint{3.358031in}{2.923257in}}{\pgfqpoint{3.363855in}{2.917433in}}%
\pgfpathcurveto{\pgfqpoint{3.369679in}{2.911609in}}{\pgfqpoint{3.377579in}{2.908337in}}{\pgfqpoint{3.385816in}{2.908337in}}%
\pgfpathclose%
\pgfusepath{stroke,fill}%
\end{pgfscope}%
\begin{pgfscope}%
\pgfpathrectangle{\pgfqpoint{0.100000in}{0.220728in}}{\pgfqpoint{3.696000in}{3.696000in}}%
\pgfusepath{clip}%
\pgfsetbuttcap%
\pgfsetroundjoin%
\definecolor{currentfill}{rgb}{0.121569,0.466667,0.705882}%
\pgfsetfillcolor{currentfill}%
\pgfsetfillopacity{0.674767}%
\pgfsetlinewidth{1.003750pt}%
\definecolor{currentstroke}{rgb}{0.121569,0.466667,0.705882}%
\pgfsetstrokecolor{currentstroke}%
\pgfsetstrokeopacity{0.674767}%
\pgfsetdash{}{0pt}%
\pgfpathmoveto{\pgfqpoint{3.385444in}{2.907442in}}%
\pgfpathcurveto{\pgfqpoint{3.393680in}{2.907442in}}{\pgfqpoint{3.401580in}{2.910714in}}{\pgfqpoint{3.407404in}{2.916538in}}%
\pgfpathcurveto{\pgfqpoint{3.413228in}{2.922362in}}{\pgfqpoint{3.416500in}{2.930262in}}{\pgfqpoint{3.416500in}{2.938498in}}%
\pgfpathcurveto{\pgfqpoint{3.416500in}{2.946735in}}{\pgfqpoint{3.413228in}{2.954635in}}{\pgfqpoint{3.407404in}{2.960459in}}%
\pgfpathcurveto{\pgfqpoint{3.401580in}{2.966282in}}{\pgfqpoint{3.393680in}{2.969555in}}{\pgfqpoint{3.385444in}{2.969555in}}%
\pgfpathcurveto{\pgfqpoint{3.377207in}{2.969555in}}{\pgfqpoint{3.369307in}{2.966282in}}{\pgfqpoint{3.363483in}{2.960459in}}%
\pgfpathcurveto{\pgfqpoint{3.357659in}{2.954635in}}{\pgfqpoint{3.354387in}{2.946735in}}{\pgfqpoint{3.354387in}{2.938498in}}%
\pgfpathcurveto{\pgfqpoint{3.354387in}{2.930262in}}{\pgfqpoint{3.357659in}{2.922362in}}{\pgfqpoint{3.363483in}{2.916538in}}%
\pgfpathcurveto{\pgfqpoint{3.369307in}{2.910714in}}{\pgfqpoint{3.377207in}{2.907442in}}{\pgfqpoint{3.385444in}{2.907442in}}%
\pgfpathclose%
\pgfusepath{stroke,fill}%
\end{pgfscope}%
\begin{pgfscope}%
\pgfpathrectangle{\pgfqpoint{0.100000in}{0.220728in}}{\pgfqpoint{3.696000in}{3.696000in}}%
\pgfusepath{clip}%
\pgfsetbuttcap%
\pgfsetroundjoin%
\definecolor{currentfill}{rgb}{0.121569,0.466667,0.705882}%
\pgfsetfillcolor{currentfill}%
\pgfsetfillopacity{0.675103}%
\pgfsetlinewidth{1.003750pt}%
\definecolor{currentstroke}{rgb}{0.121569,0.466667,0.705882}%
\pgfsetstrokecolor{currentstroke}%
\pgfsetstrokeopacity{0.675103}%
\pgfsetdash{}{0pt}%
\pgfpathmoveto{\pgfqpoint{3.383819in}{2.905178in}}%
\pgfpathcurveto{\pgfqpoint{3.392055in}{2.905178in}}{\pgfqpoint{3.399955in}{2.908450in}}{\pgfqpoint{3.405779in}{2.914274in}}%
\pgfpathcurveto{\pgfqpoint{3.411603in}{2.920098in}}{\pgfqpoint{3.414875in}{2.927998in}}{\pgfqpoint{3.414875in}{2.936235in}}%
\pgfpathcurveto{\pgfqpoint{3.414875in}{2.944471in}}{\pgfqpoint{3.411603in}{2.952371in}}{\pgfqpoint{3.405779in}{2.958195in}}%
\pgfpathcurveto{\pgfqpoint{3.399955in}{2.964019in}}{\pgfqpoint{3.392055in}{2.967291in}}{\pgfqpoint{3.383819in}{2.967291in}}%
\pgfpathcurveto{\pgfqpoint{3.375583in}{2.967291in}}{\pgfqpoint{3.367683in}{2.964019in}}{\pgfqpoint{3.361859in}{2.958195in}}%
\pgfpathcurveto{\pgfqpoint{3.356035in}{2.952371in}}{\pgfqpoint{3.352762in}{2.944471in}}{\pgfqpoint{3.352762in}{2.936235in}}%
\pgfpathcurveto{\pgfqpoint{3.352762in}{2.927998in}}{\pgfqpoint{3.356035in}{2.920098in}}{\pgfqpoint{3.361859in}{2.914274in}}%
\pgfpathcurveto{\pgfqpoint{3.367683in}{2.908450in}}{\pgfqpoint{3.375583in}{2.905178in}}{\pgfqpoint{3.383819in}{2.905178in}}%
\pgfpathclose%
\pgfusepath{stroke,fill}%
\end{pgfscope}%
\begin{pgfscope}%
\pgfpathrectangle{\pgfqpoint{0.100000in}{0.220728in}}{\pgfqpoint{3.696000in}{3.696000in}}%
\pgfusepath{clip}%
\pgfsetbuttcap%
\pgfsetroundjoin%
\definecolor{currentfill}{rgb}{0.121569,0.466667,0.705882}%
\pgfsetfillcolor{currentfill}%
\pgfsetfillopacity{0.675776}%
\pgfsetlinewidth{1.003750pt}%
\definecolor{currentstroke}{rgb}{0.121569,0.466667,0.705882}%
\pgfsetstrokecolor{currentstroke}%
\pgfsetstrokeopacity{0.675776}%
\pgfsetdash{}{0pt}%
\pgfpathmoveto{\pgfqpoint{3.382103in}{2.901785in}}%
\pgfpathcurveto{\pgfqpoint{3.390339in}{2.901785in}}{\pgfqpoint{3.398239in}{2.905058in}}{\pgfqpoint{3.404063in}{2.910882in}}%
\pgfpathcurveto{\pgfqpoint{3.409887in}{2.916706in}}{\pgfqpoint{3.413159in}{2.924606in}}{\pgfqpoint{3.413159in}{2.932842in}}%
\pgfpathcurveto{\pgfqpoint{3.413159in}{2.941078in}}{\pgfqpoint{3.409887in}{2.948978in}}{\pgfqpoint{3.404063in}{2.954802in}}%
\pgfpathcurveto{\pgfqpoint{3.398239in}{2.960626in}}{\pgfqpoint{3.390339in}{2.963898in}}{\pgfqpoint{3.382103in}{2.963898in}}%
\pgfpathcurveto{\pgfqpoint{3.373867in}{2.963898in}}{\pgfqpoint{3.365967in}{2.960626in}}{\pgfqpoint{3.360143in}{2.954802in}}%
\pgfpathcurveto{\pgfqpoint{3.354319in}{2.948978in}}{\pgfqpoint{3.351046in}{2.941078in}}{\pgfqpoint{3.351046in}{2.932842in}}%
\pgfpathcurveto{\pgfqpoint{3.351046in}{2.924606in}}{\pgfqpoint{3.354319in}{2.916706in}}{\pgfqpoint{3.360143in}{2.910882in}}%
\pgfpathcurveto{\pgfqpoint{3.365967in}{2.905058in}}{\pgfqpoint{3.373867in}{2.901785in}}{\pgfqpoint{3.382103in}{2.901785in}}%
\pgfpathclose%
\pgfusepath{stroke,fill}%
\end{pgfscope}%
\begin{pgfscope}%
\pgfpathrectangle{\pgfqpoint{0.100000in}{0.220728in}}{\pgfqpoint{3.696000in}{3.696000in}}%
\pgfusepath{clip}%
\pgfsetbuttcap%
\pgfsetroundjoin%
\definecolor{currentfill}{rgb}{0.121569,0.466667,0.705882}%
\pgfsetfillcolor{currentfill}%
\pgfsetfillopacity{0.676303}%
\pgfsetlinewidth{1.003750pt}%
\definecolor{currentstroke}{rgb}{0.121569,0.466667,0.705882}%
\pgfsetstrokecolor{currentstroke}%
\pgfsetstrokeopacity{0.676303}%
\pgfsetdash{}{0pt}%
\pgfpathmoveto{\pgfqpoint{0.714164in}{1.416956in}}%
\pgfpathcurveto{\pgfqpoint{0.722400in}{1.416956in}}{\pgfqpoint{0.730300in}{1.420228in}}{\pgfqpoint{0.736124in}{1.426052in}}%
\pgfpathcurveto{\pgfqpoint{0.741948in}{1.431876in}}{\pgfqpoint{0.745220in}{1.439776in}}{\pgfqpoint{0.745220in}{1.448012in}}%
\pgfpathcurveto{\pgfqpoint{0.745220in}{1.456249in}}{\pgfqpoint{0.741948in}{1.464149in}}{\pgfqpoint{0.736124in}{1.469973in}}%
\pgfpathcurveto{\pgfqpoint{0.730300in}{1.475797in}}{\pgfqpoint{0.722400in}{1.479069in}}{\pgfqpoint{0.714164in}{1.479069in}}%
\pgfpathcurveto{\pgfqpoint{0.705928in}{1.479069in}}{\pgfqpoint{0.698028in}{1.475797in}}{\pgfqpoint{0.692204in}{1.469973in}}%
\pgfpathcurveto{\pgfqpoint{0.686380in}{1.464149in}}{\pgfqpoint{0.683107in}{1.456249in}}{\pgfqpoint{0.683107in}{1.448012in}}%
\pgfpathcurveto{\pgfqpoint{0.683107in}{1.439776in}}{\pgfqpoint{0.686380in}{1.431876in}}{\pgfqpoint{0.692204in}{1.426052in}}%
\pgfpathcurveto{\pgfqpoint{0.698028in}{1.420228in}}{\pgfqpoint{0.705928in}{1.416956in}}{\pgfqpoint{0.714164in}{1.416956in}}%
\pgfpathclose%
\pgfusepath{stroke,fill}%
\end{pgfscope}%
\begin{pgfscope}%
\pgfpathrectangle{\pgfqpoint{0.100000in}{0.220728in}}{\pgfqpoint{3.696000in}{3.696000in}}%
\pgfusepath{clip}%
\pgfsetbuttcap%
\pgfsetroundjoin%
\definecolor{currentfill}{rgb}{0.121569,0.466667,0.705882}%
\pgfsetfillcolor{currentfill}%
\pgfsetfillopacity{0.676680}%
\pgfsetlinewidth{1.003750pt}%
\definecolor{currentstroke}{rgb}{0.121569,0.466667,0.705882}%
\pgfsetstrokecolor{currentstroke}%
\pgfsetstrokeopacity{0.676680}%
\pgfsetdash{}{0pt}%
\pgfpathmoveto{\pgfqpoint{3.380300in}{2.897773in}}%
\pgfpathcurveto{\pgfqpoint{3.388537in}{2.897773in}}{\pgfqpoint{3.396437in}{2.901045in}}{\pgfqpoint{3.402261in}{2.906869in}}%
\pgfpathcurveto{\pgfqpoint{3.408084in}{2.912693in}}{\pgfqpoint{3.411357in}{2.920593in}}{\pgfqpoint{3.411357in}{2.928830in}}%
\pgfpathcurveto{\pgfqpoint{3.411357in}{2.937066in}}{\pgfqpoint{3.408084in}{2.944966in}}{\pgfqpoint{3.402261in}{2.950790in}}%
\pgfpathcurveto{\pgfqpoint{3.396437in}{2.956614in}}{\pgfqpoint{3.388537in}{2.959886in}}{\pgfqpoint{3.380300in}{2.959886in}}%
\pgfpathcurveto{\pgfqpoint{3.372064in}{2.959886in}}{\pgfqpoint{3.364164in}{2.956614in}}{\pgfqpoint{3.358340in}{2.950790in}}%
\pgfpathcurveto{\pgfqpoint{3.352516in}{2.944966in}}{\pgfqpoint{3.349244in}{2.937066in}}{\pgfqpoint{3.349244in}{2.928830in}}%
\pgfpathcurveto{\pgfqpoint{3.349244in}{2.920593in}}{\pgfqpoint{3.352516in}{2.912693in}}{\pgfqpoint{3.358340in}{2.906869in}}%
\pgfpathcurveto{\pgfqpoint{3.364164in}{2.901045in}}{\pgfqpoint{3.372064in}{2.897773in}}{\pgfqpoint{3.380300in}{2.897773in}}%
\pgfpathclose%
\pgfusepath{stroke,fill}%
\end{pgfscope}%
\begin{pgfscope}%
\pgfpathrectangle{\pgfqpoint{0.100000in}{0.220728in}}{\pgfqpoint{3.696000in}{3.696000in}}%
\pgfusepath{clip}%
\pgfsetbuttcap%
\pgfsetroundjoin%
\definecolor{currentfill}{rgb}{0.121569,0.466667,0.705882}%
\pgfsetfillcolor{currentfill}%
\pgfsetfillopacity{0.677533}%
\pgfsetlinewidth{1.003750pt}%
\definecolor{currentstroke}{rgb}{0.121569,0.466667,0.705882}%
\pgfsetstrokecolor{currentstroke}%
\pgfsetstrokeopacity{0.677533}%
\pgfsetdash{}{0pt}%
\pgfpathmoveto{\pgfqpoint{3.377210in}{2.893593in}}%
\pgfpathcurveto{\pgfqpoint{3.385446in}{2.893593in}}{\pgfqpoint{3.393346in}{2.896865in}}{\pgfqpoint{3.399170in}{2.902689in}}%
\pgfpathcurveto{\pgfqpoint{3.404994in}{2.908513in}}{\pgfqpoint{3.408266in}{2.916413in}}{\pgfqpoint{3.408266in}{2.924649in}}%
\pgfpathcurveto{\pgfqpoint{3.408266in}{2.932885in}}{\pgfqpoint{3.404994in}{2.940785in}}{\pgfqpoint{3.399170in}{2.946609in}}%
\pgfpathcurveto{\pgfqpoint{3.393346in}{2.952433in}}{\pgfqpoint{3.385446in}{2.955706in}}{\pgfqpoint{3.377210in}{2.955706in}}%
\pgfpathcurveto{\pgfqpoint{3.368973in}{2.955706in}}{\pgfqpoint{3.361073in}{2.952433in}}{\pgfqpoint{3.355249in}{2.946609in}}%
\pgfpathcurveto{\pgfqpoint{3.349425in}{2.940785in}}{\pgfqpoint{3.346153in}{2.932885in}}{\pgfqpoint{3.346153in}{2.924649in}}%
\pgfpathcurveto{\pgfqpoint{3.346153in}{2.916413in}}{\pgfqpoint{3.349425in}{2.908513in}}{\pgfqpoint{3.355249in}{2.902689in}}%
\pgfpathcurveto{\pgfqpoint{3.361073in}{2.896865in}}{\pgfqpoint{3.368973in}{2.893593in}}{\pgfqpoint{3.377210in}{2.893593in}}%
\pgfpathclose%
\pgfusepath{stroke,fill}%
\end{pgfscope}%
\begin{pgfscope}%
\pgfpathrectangle{\pgfqpoint{0.100000in}{0.220728in}}{\pgfqpoint{3.696000in}{3.696000in}}%
\pgfusepath{clip}%
\pgfsetbuttcap%
\pgfsetroundjoin%
\definecolor{currentfill}{rgb}{0.121569,0.466667,0.705882}%
\pgfsetfillcolor{currentfill}%
\pgfsetfillopacity{0.678805}%
\pgfsetlinewidth{1.003750pt}%
\definecolor{currentstroke}{rgb}{0.121569,0.466667,0.705882}%
\pgfsetstrokecolor{currentstroke}%
\pgfsetstrokeopacity{0.678805}%
\pgfsetdash{}{0pt}%
\pgfpathmoveto{\pgfqpoint{0.726761in}{1.407837in}}%
\pgfpathcurveto{\pgfqpoint{0.734998in}{1.407837in}}{\pgfqpoint{0.742898in}{1.411110in}}{\pgfqpoint{0.748722in}{1.416934in}}%
\pgfpathcurveto{\pgfqpoint{0.754546in}{1.422758in}}{\pgfqpoint{0.757818in}{1.430658in}}{\pgfqpoint{0.757818in}{1.438894in}}%
\pgfpathcurveto{\pgfqpoint{0.757818in}{1.447130in}}{\pgfqpoint{0.754546in}{1.455030in}}{\pgfqpoint{0.748722in}{1.460854in}}%
\pgfpathcurveto{\pgfqpoint{0.742898in}{1.466678in}}{\pgfqpoint{0.734998in}{1.469950in}}{\pgfqpoint{0.726761in}{1.469950in}}%
\pgfpathcurveto{\pgfqpoint{0.718525in}{1.469950in}}{\pgfqpoint{0.710625in}{1.466678in}}{\pgfqpoint{0.704801in}{1.460854in}}%
\pgfpathcurveto{\pgfqpoint{0.698977in}{1.455030in}}{\pgfqpoint{0.695705in}{1.447130in}}{\pgfqpoint{0.695705in}{1.438894in}}%
\pgfpathcurveto{\pgfqpoint{0.695705in}{1.430658in}}{\pgfqpoint{0.698977in}{1.422758in}}{\pgfqpoint{0.704801in}{1.416934in}}%
\pgfpathcurveto{\pgfqpoint{0.710625in}{1.411110in}}{\pgfqpoint{0.718525in}{1.407837in}}{\pgfqpoint{0.726761in}{1.407837in}}%
\pgfpathclose%
\pgfusepath{stroke,fill}%
\end{pgfscope}%
\begin{pgfscope}%
\pgfpathrectangle{\pgfqpoint{0.100000in}{0.220728in}}{\pgfqpoint{3.696000in}{3.696000in}}%
\pgfusepath{clip}%
\pgfsetbuttcap%
\pgfsetroundjoin%
\definecolor{currentfill}{rgb}{0.121569,0.466667,0.705882}%
\pgfsetfillcolor{currentfill}%
\pgfsetfillopacity{0.678874}%
\pgfsetlinewidth{1.003750pt}%
\definecolor{currentstroke}{rgb}{0.121569,0.466667,0.705882}%
\pgfsetstrokecolor{currentstroke}%
\pgfsetstrokeopacity{0.678874}%
\pgfsetdash{}{0pt}%
\pgfpathmoveto{\pgfqpoint{3.374261in}{2.885883in}}%
\pgfpathcurveto{\pgfqpoint{3.382498in}{2.885883in}}{\pgfqpoint{3.390398in}{2.889156in}}{\pgfqpoint{3.396222in}{2.894980in}}%
\pgfpathcurveto{\pgfqpoint{3.402046in}{2.900803in}}{\pgfqpoint{3.405318in}{2.908704in}}{\pgfqpoint{3.405318in}{2.916940in}}%
\pgfpathcurveto{\pgfqpoint{3.405318in}{2.925176in}}{\pgfqpoint{3.402046in}{2.933076in}}{\pgfqpoint{3.396222in}{2.938900in}}%
\pgfpathcurveto{\pgfqpoint{3.390398in}{2.944724in}}{\pgfqpoint{3.382498in}{2.947996in}}{\pgfqpoint{3.374261in}{2.947996in}}%
\pgfpathcurveto{\pgfqpoint{3.366025in}{2.947996in}}{\pgfqpoint{3.358125in}{2.944724in}}{\pgfqpoint{3.352301in}{2.938900in}}%
\pgfpathcurveto{\pgfqpoint{3.346477in}{2.933076in}}{\pgfqpoint{3.343205in}{2.925176in}}{\pgfqpoint{3.343205in}{2.916940in}}%
\pgfpathcurveto{\pgfqpoint{3.343205in}{2.908704in}}{\pgfqpoint{3.346477in}{2.900803in}}{\pgfqpoint{3.352301in}{2.894980in}}%
\pgfpathcurveto{\pgfqpoint{3.358125in}{2.889156in}}{\pgfqpoint{3.366025in}{2.885883in}}{\pgfqpoint{3.374261in}{2.885883in}}%
\pgfpathclose%
\pgfusepath{stroke,fill}%
\end{pgfscope}%
\begin{pgfscope}%
\pgfpathrectangle{\pgfqpoint{0.100000in}{0.220728in}}{\pgfqpoint{3.696000in}{3.696000in}}%
\pgfusepath{clip}%
\pgfsetbuttcap%
\pgfsetroundjoin%
\definecolor{currentfill}{rgb}{0.121569,0.466667,0.705882}%
\pgfsetfillcolor{currentfill}%
\pgfsetfillopacity{0.679515}%
\pgfsetlinewidth{1.003750pt}%
\definecolor{currentstroke}{rgb}{0.121569,0.466667,0.705882}%
\pgfsetstrokecolor{currentstroke}%
\pgfsetstrokeopacity{0.679515}%
\pgfsetdash{}{0pt}%
\pgfpathmoveto{\pgfqpoint{3.371947in}{2.882061in}}%
\pgfpathcurveto{\pgfqpoint{3.380183in}{2.882061in}}{\pgfqpoint{3.388083in}{2.885334in}}{\pgfqpoint{3.393907in}{2.891158in}}%
\pgfpathcurveto{\pgfqpoint{3.399731in}{2.896981in}}{\pgfqpoint{3.403003in}{2.904881in}}{\pgfqpoint{3.403003in}{2.913118in}}%
\pgfpathcurveto{\pgfqpoint{3.403003in}{2.921354in}}{\pgfqpoint{3.399731in}{2.929254in}}{\pgfqpoint{3.393907in}{2.935078in}}%
\pgfpathcurveto{\pgfqpoint{3.388083in}{2.940902in}}{\pgfqpoint{3.380183in}{2.944174in}}{\pgfqpoint{3.371947in}{2.944174in}}%
\pgfpathcurveto{\pgfqpoint{3.363710in}{2.944174in}}{\pgfqpoint{3.355810in}{2.940902in}}{\pgfqpoint{3.349986in}{2.935078in}}%
\pgfpathcurveto{\pgfqpoint{3.344162in}{2.929254in}}{\pgfqpoint{3.340890in}{2.921354in}}{\pgfqpoint{3.340890in}{2.913118in}}%
\pgfpathcurveto{\pgfqpoint{3.340890in}{2.904881in}}{\pgfqpoint{3.344162in}{2.896981in}}{\pgfqpoint{3.349986in}{2.891158in}}%
\pgfpathcurveto{\pgfqpoint{3.355810in}{2.885334in}}{\pgfqpoint{3.363710in}{2.882061in}}{\pgfqpoint{3.371947in}{2.882061in}}%
\pgfpathclose%
\pgfusepath{stroke,fill}%
\end{pgfscope}%
\begin{pgfscope}%
\pgfpathrectangle{\pgfqpoint{0.100000in}{0.220728in}}{\pgfqpoint{3.696000in}{3.696000in}}%
\pgfusepath{clip}%
\pgfsetbuttcap%
\pgfsetroundjoin%
\definecolor{currentfill}{rgb}{0.121569,0.466667,0.705882}%
\pgfsetfillcolor{currentfill}%
\pgfsetfillopacity{0.680416}%
\pgfsetlinewidth{1.003750pt}%
\definecolor{currentstroke}{rgb}{0.121569,0.466667,0.705882}%
\pgfsetstrokecolor{currentstroke}%
\pgfsetstrokeopacity{0.680416}%
\pgfsetdash{}{0pt}%
\pgfpathmoveto{\pgfqpoint{3.369409in}{2.877459in}}%
\pgfpathcurveto{\pgfqpoint{3.377645in}{2.877459in}}{\pgfqpoint{3.385545in}{2.880732in}}{\pgfqpoint{3.391369in}{2.886556in}}%
\pgfpathcurveto{\pgfqpoint{3.397193in}{2.892380in}}{\pgfqpoint{3.400465in}{2.900280in}}{\pgfqpoint{3.400465in}{2.908516in}}%
\pgfpathcurveto{\pgfqpoint{3.400465in}{2.916752in}}{\pgfqpoint{3.397193in}{2.924652in}}{\pgfqpoint{3.391369in}{2.930476in}}%
\pgfpathcurveto{\pgfqpoint{3.385545in}{2.936300in}}{\pgfqpoint{3.377645in}{2.939572in}}{\pgfqpoint{3.369409in}{2.939572in}}%
\pgfpathcurveto{\pgfqpoint{3.361173in}{2.939572in}}{\pgfqpoint{3.353273in}{2.936300in}}{\pgfqpoint{3.347449in}{2.930476in}}%
\pgfpathcurveto{\pgfqpoint{3.341625in}{2.924652in}}{\pgfqpoint{3.338352in}{2.916752in}}{\pgfqpoint{3.338352in}{2.908516in}}%
\pgfpathcurveto{\pgfqpoint{3.338352in}{2.900280in}}{\pgfqpoint{3.341625in}{2.892380in}}{\pgfqpoint{3.347449in}{2.886556in}}%
\pgfpathcurveto{\pgfqpoint{3.353273in}{2.880732in}}{\pgfqpoint{3.361173in}{2.877459in}}{\pgfqpoint{3.369409in}{2.877459in}}%
\pgfpathclose%
\pgfusepath{stroke,fill}%
\end{pgfscope}%
\begin{pgfscope}%
\pgfpathrectangle{\pgfqpoint{0.100000in}{0.220728in}}{\pgfqpoint{3.696000in}{3.696000in}}%
\pgfusepath{clip}%
\pgfsetbuttcap%
\pgfsetroundjoin%
\definecolor{currentfill}{rgb}{0.121569,0.466667,0.705882}%
\pgfsetfillcolor{currentfill}%
\pgfsetfillopacity{0.681083}%
\pgfsetlinewidth{1.003750pt}%
\definecolor{currentstroke}{rgb}{0.121569,0.466667,0.705882}%
\pgfsetstrokecolor{currentstroke}%
\pgfsetstrokeopacity{0.681083}%
\pgfsetdash{}{0pt}%
\pgfpathmoveto{\pgfqpoint{0.738996in}{1.402639in}}%
\pgfpathcurveto{\pgfqpoint{0.747233in}{1.402639in}}{\pgfqpoint{0.755133in}{1.405912in}}{\pgfqpoint{0.760957in}{1.411735in}}%
\pgfpathcurveto{\pgfqpoint{0.766781in}{1.417559in}}{\pgfqpoint{0.770053in}{1.425459in}}{\pgfqpoint{0.770053in}{1.433696in}}%
\pgfpathcurveto{\pgfqpoint{0.770053in}{1.441932in}}{\pgfqpoint{0.766781in}{1.449832in}}{\pgfqpoint{0.760957in}{1.455656in}}%
\pgfpathcurveto{\pgfqpoint{0.755133in}{1.461480in}}{\pgfqpoint{0.747233in}{1.464752in}}{\pgfqpoint{0.738996in}{1.464752in}}%
\pgfpathcurveto{\pgfqpoint{0.730760in}{1.464752in}}{\pgfqpoint{0.722860in}{1.461480in}}{\pgfqpoint{0.717036in}{1.455656in}}%
\pgfpathcurveto{\pgfqpoint{0.711212in}{1.449832in}}{\pgfqpoint{0.707940in}{1.441932in}}{\pgfqpoint{0.707940in}{1.433696in}}%
\pgfpathcurveto{\pgfqpoint{0.707940in}{1.425459in}}{\pgfqpoint{0.711212in}{1.417559in}}{\pgfqpoint{0.717036in}{1.411735in}}%
\pgfpathcurveto{\pgfqpoint{0.722860in}{1.405912in}}{\pgfqpoint{0.730760in}{1.402639in}}{\pgfqpoint{0.738996in}{1.402639in}}%
\pgfpathclose%
\pgfusepath{stroke,fill}%
\end{pgfscope}%
\begin{pgfscope}%
\pgfpathrectangle{\pgfqpoint{0.100000in}{0.220728in}}{\pgfqpoint{3.696000in}{3.696000in}}%
\pgfusepath{clip}%
\pgfsetbuttcap%
\pgfsetroundjoin%
\definecolor{currentfill}{rgb}{0.121569,0.466667,0.705882}%
\pgfsetfillcolor{currentfill}%
\pgfsetfillopacity{0.681401}%
\pgfsetlinewidth{1.003750pt}%
\definecolor{currentstroke}{rgb}{0.121569,0.466667,0.705882}%
\pgfsetstrokecolor{currentstroke}%
\pgfsetstrokeopacity{0.681401}%
\pgfsetdash{}{0pt}%
\pgfpathmoveto{\pgfqpoint{3.367115in}{2.871586in}}%
\pgfpathcurveto{\pgfqpoint{3.375352in}{2.871586in}}{\pgfqpoint{3.383252in}{2.874859in}}{\pgfqpoint{3.389076in}{2.880683in}}%
\pgfpathcurveto{\pgfqpoint{3.394899in}{2.886507in}}{\pgfqpoint{3.398172in}{2.894407in}}{\pgfqpoint{3.398172in}{2.902643in}}%
\pgfpathcurveto{\pgfqpoint{3.398172in}{2.910879in}}{\pgfqpoint{3.394899in}{2.918779in}}{\pgfqpoint{3.389076in}{2.924603in}}%
\pgfpathcurveto{\pgfqpoint{3.383252in}{2.930427in}}{\pgfqpoint{3.375352in}{2.933699in}}{\pgfqpoint{3.367115in}{2.933699in}}%
\pgfpathcurveto{\pgfqpoint{3.358879in}{2.933699in}}{\pgfqpoint{3.350979in}{2.930427in}}{\pgfqpoint{3.345155in}{2.924603in}}%
\pgfpathcurveto{\pgfqpoint{3.339331in}{2.918779in}}{\pgfqpoint{3.336059in}{2.910879in}}{\pgfqpoint{3.336059in}{2.902643in}}%
\pgfpathcurveto{\pgfqpoint{3.336059in}{2.894407in}}{\pgfqpoint{3.339331in}{2.886507in}}{\pgfqpoint{3.345155in}{2.880683in}}%
\pgfpathcurveto{\pgfqpoint{3.350979in}{2.874859in}}{\pgfqpoint{3.358879in}{2.871586in}}{\pgfqpoint{3.367115in}{2.871586in}}%
\pgfpathclose%
\pgfusepath{stroke,fill}%
\end{pgfscope}%
\begin{pgfscope}%
\pgfpathrectangle{\pgfqpoint{0.100000in}{0.220728in}}{\pgfqpoint{3.696000in}{3.696000in}}%
\pgfusepath{clip}%
\pgfsetbuttcap%
\pgfsetroundjoin%
\definecolor{currentfill}{rgb}{0.121569,0.466667,0.705882}%
\pgfsetfillcolor{currentfill}%
\pgfsetfillopacity{0.682268}%
\pgfsetlinewidth{1.003750pt}%
\definecolor{currentstroke}{rgb}{0.121569,0.466667,0.705882}%
\pgfsetstrokecolor{currentstroke}%
\pgfsetstrokeopacity{0.682268}%
\pgfsetdash{}{0pt}%
\pgfpathmoveto{\pgfqpoint{3.362381in}{2.864947in}}%
\pgfpathcurveto{\pgfqpoint{3.370618in}{2.864947in}}{\pgfqpoint{3.378518in}{2.868220in}}{\pgfqpoint{3.384342in}{2.874044in}}%
\pgfpathcurveto{\pgfqpoint{3.390166in}{2.879867in}}{\pgfqpoint{3.393438in}{2.887768in}}{\pgfqpoint{3.393438in}{2.896004in}}%
\pgfpathcurveto{\pgfqpoint{3.393438in}{2.904240in}}{\pgfqpoint{3.390166in}{2.912140in}}{\pgfqpoint{3.384342in}{2.917964in}}%
\pgfpathcurveto{\pgfqpoint{3.378518in}{2.923788in}}{\pgfqpoint{3.370618in}{2.927060in}}{\pgfqpoint{3.362381in}{2.927060in}}%
\pgfpathcurveto{\pgfqpoint{3.354145in}{2.927060in}}{\pgfqpoint{3.346245in}{2.923788in}}{\pgfqpoint{3.340421in}{2.917964in}}%
\pgfpathcurveto{\pgfqpoint{3.334597in}{2.912140in}}{\pgfqpoint{3.331325in}{2.904240in}}{\pgfqpoint{3.331325in}{2.896004in}}%
\pgfpathcurveto{\pgfqpoint{3.331325in}{2.887768in}}{\pgfqpoint{3.334597in}{2.879867in}}{\pgfqpoint{3.340421in}{2.874044in}}%
\pgfpathcurveto{\pgfqpoint{3.346245in}{2.868220in}}{\pgfqpoint{3.354145in}{2.864947in}}{\pgfqpoint{3.362381in}{2.864947in}}%
\pgfpathclose%
\pgfusepath{stroke,fill}%
\end{pgfscope}%
\begin{pgfscope}%
\pgfpathrectangle{\pgfqpoint{0.100000in}{0.220728in}}{\pgfqpoint{3.696000in}{3.696000in}}%
\pgfusepath{clip}%
\pgfsetbuttcap%
\pgfsetroundjoin%
\definecolor{currentfill}{rgb}{0.121569,0.466667,0.705882}%
\pgfsetfillcolor{currentfill}%
\pgfsetfillopacity{0.682435}%
\pgfsetlinewidth{1.003750pt}%
\definecolor{currentstroke}{rgb}{0.121569,0.466667,0.705882}%
\pgfsetstrokecolor{currentstroke}%
\pgfsetstrokeopacity{0.682435}%
\pgfsetdash{}{0pt}%
\pgfpathmoveto{\pgfqpoint{0.748318in}{1.395917in}}%
\pgfpathcurveto{\pgfqpoint{0.756554in}{1.395917in}}{\pgfqpoint{0.764454in}{1.399189in}}{\pgfqpoint{0.770278in}{1.405013in}}%
\pgfpathcurveto{\pgfqpoint{0.776102in}{1.410837in}}{\pgfqpoint{0.779375in}{1.418737in}}{\pgfqpoint{0.779375in}{1.426974in}}%
\pgfpathcurveto{\pgfqpoint{0.779375in}{1.435210in}}{\pgfqpoint{0.776102in}{1.443110in}}{\pgfqpoint{0.770278in}{1.448934in}}%
\pgfpathcurveto{\pgfqpoint{0.764454in}{1.454758in}}{\pgfqpoint{0.756554in}{1.458030in}}{\pgfqpoint{0.748318in}{1.458030in}}%
\pgfpathcurveto{\pgfqpoint{0.740082in}{1.458030in}}{\pgfqpoint{0.732182in}{1.454758in}}{\pgfqpoint{0.726358in}{1.448934in}}%
\pgfpathcurveto{\pgfqpoint{0.720534in}{1.443110in}}{\pgfqpoint{0.717262in}{1.435210in}}{\pgfqpoint{0.717262in}{1.426974in}}%
\pgfpathcurveto{\pgfqpoint{0.717262in}{1.418737in}}{\pgfqpoint{0.720534in}{1.410837in}}{\pgfqpoint{0.726358in}{1.405013in}}%
\pgfpathcurveto{\pgfqpoint{0.732182in}{1.399189in}}{\pgfqpoint{0.740082in}{1.395917in}}{\pgfqpoint{0.748318in}{1.395917in}}%
\pgfpathclose%
\pgfusepath{stroke,fill}%
\end{pgfscope}%
\begin{pgfscope}%
\pgfpathrectangle{\pgfqpoint{0.100000in}{0.220728in}}{\pgfqpoint{3.696000in}{3.696000in}}%
\pgfusepath{clip}%
\pgfsetbuttcap%
\pgfsetroundjoin%
\definecolor{currentfill}{rgb}{0.121569,0.466667,0.705882}%
\pgfsetfillcolor{currentfill}%
\pgfsetfillopacity{0.683809}%
\pgfsetlinewidth{1.003750pt}%
\definecolor{currentstroke}{rgb}{0.121569,0.466667,0.705882}%
\pgfsetstrokecolor{currentstroke}%
\pgfsetstrokeopacity{0.683809}%
\pgfsetdash{}{0pt}%
\pgfpathmoveto{\pgfqpoint{3.359135in}{2.857613in}}%
\pgfpathcurveto{\pgfqpoint{3.367371in}{2.857613in}}{\pgfqpoint{3.375271in}{2.860885in}}{\pgfqpoint{3.381095in}{2.866709in}}%
\pgfpathcurveto{\pgfqpoint{3.386919in}{2.872533in}}{\pgfqpoint{3.390191in}{2.880433in}}{\pgfqpoint{3.390191in}{2.888669in}}%
\pgfpathcurveto{\pgfqpoint{3.390191in}{2.896905in}}{\pgfqpoint{3.386919in}{2.904805in}}{\pgfqpoint{3.381095in}{2.910629in}}%
\pgfpathcurveto{\pgfqpoint{3.375271in}{2.916453in}}{\pgfqpoint{3.367371in}{2.919726in}}{\pgfqpoint{3.359135in}{2.919726in}}%
\pgfpathcurveto{\pgfqpoint{3.350899in}{2.919726in}}{\pgfqpoint{3.342999in}{2.916453in}}{\pgfqpoint{3.337175in}{2.910629in}}%
\pgfpathcurveto{\pgfqpoint{3.331351in}{2.904805in}}{\pgfqpoint{3.328078in}{2.896905in}}{\pgfqpoint{3.328078in}{2.888669in}}%
\pgfpathcurveto{\pgfqpoint{3.328078in}{2.880433in}}{\pgfqpoint{3.331351in}{2.872533in}}{\pgfqpoint{3.337175in}{2.866709in}}%
\pgfpathcurveto{\pgfqpoint{3.342999in}{2.860885in}}{\pgfqpoint{3.350899in}{2.857613in}}{\pgfqpoint{3.359135in}{2.857613in}}%
\pgfpathclose%
\pgfusepath{stroke,fill}%
\end{pgfscope}%
\begin{pgfscope}%
\pgfpathrectangle{\pgfqpoint{0.100000in}{0.220728in}}{\pgfqpoint{3.696000in}{3.696000in}}%
\pgfusepath{clip}%
\pgfsetbuttcap%
\pgfsetroundjoin%
\definecolor{currentfill}{rgb}{0.121569,0.466667,0.705882}%
\pgfsetfillcolor{currentfill}%
\pgfsetfillopacity{0.684544}%
\pgfsetlinewidth{1.003750pt}%
\definecolor{currentstroke}{rgb}{0.121569,0.466667,0.705882}%
\pgfsetstrokecolor{currentstroke}%
\pgfsetstrokeopacity{0.684544}%
\pgfsetdash{}{0pt}%
\pgfpathmoveto{\pgfqpoint{0.756239in}{1.391706in}}%
\pgfpathcurveto{\pgfqpoint{0.764476in}{1.391706in}}{\pgfqpoint{0.772376in}{1.394979in}}{\pgfqpoint{0.778200in}{1.400803in}}%
\pgfpathcurveto{\pgfqpoint{0.784024in}{1.406627in}}{\pgfqpoint{0.787296in}{1.414527in}}{\pgfqpoint{0.787296in}{1.422763in}}%
\pgfpathcurveto{\pgfqpoint{0.787296in}{1.430999in}}{\pgfqpoint{0.784024in}{1.438899in}}{\pgfqpoint{0.778200in}{1.444723in}}%
\pgfpathcurveto{\pgfqpoint{0.772376in}{1.450547in}}{\pgfqpoint{0.764476in}{1.453819in}}{\pgfqpoint{0.756239in}{1.453819in}}%
\pgfpathcurveto{\pgfqpoint{0.748003in}{1.453819in}}{\pgfqpoint{0.740103in}{1.450547in}}{\pgfqpoint{0.734279in}{1.444723in}}%
\pgfpathcurveto{\pgfqpoint{0.728455in}{1.438899in}}{\pgfqpoint{0.725183in}{1.430999in}}{\pgfqpoint{0.725183in}{1.422763in}}%
\pgfpathcurveto{\pgfqpoint{0.725183in}{1.414527in}}{\pgfqpoint{0.728455in}{1.406627in}}{\pgfqpoint{0.734279in}{1.400803in}}%
\pgfpathcurveto{\pgfqpoint{0.740103in}{1.394979in}}{\pgfqpoint{0.748003in}{1.391706in}}{\pgfqpoint{0.756239in}{1.391706in}}%
\pgfpathclose%
\pgfusepath{stroke,fill}%
\end{pgfscope}%
\begin{pgfscope}%
\pgfpathrectangle{\pgfqpoint{0.100000in}{0.220728in}}{\pgfqpoint{3.696000in}{3.696000in}}%
\pgfusepath{clip}%
\pgfsetbuttcap%
\pgfsetroundjoin%
\definecolor{currentfill}{rgb}{0.121569,0.466667,0.705882}%
\pgfsetfillcolor{currentfill}%
\pgfsetfillopacity{0.685474}%
\pgfsetlinewidth{1.003750pt}%
\definecolor{currentstroke}{rgb}{0.121569,0.466667,0.705882}%
\pgfsetstrokecolor{currentstroke}%
\pgfsetstrokeopacity{0.685474}%
\pgfsetdash{}{0pt}%
\pgfpathmoveto{\pgfqpoint{3.355055in}{2.850101in}}%
\pgfpathcurveto{\pgfqpoint{3.363291in}{2.850101in}}{\pgfqpoint{3.371191in}{2.853373in}}{\pgfqpoint{3.377015in}{2.859197in}}%
\pgfpathcurveto{\pgfqpoint{3.382839in}{2.865021in}}{\pgfqpoint{3.386112in}{2.872921in}}{\pgfqpoint{3.386112in}{2.881158in}}%
\pgfpathcurveto{\pgfqpoint{3.386112in}{2.889394in}}{\pgfqpoint{3.382839in}{2.897294in}}{\pgfqpoint{3.377015in}{2.903118in}}%
\pgfpathcurveto{\pgfqpoint{3.371191in}{2.908942in}}{\pgfqpoint{3.363291in}{2.912214in}}{\pgfqpoint{3.355055in}{2.912214in}}%
\pgfpathcurveto{\pgfqpoint{3.346819in}{2.912214in}}{\pgfqpoint{3.338919in}{2.908942in}}{\pgfqpoint{3.333095in}{2.903118in}}%
\pgfpathcurveto{\pgfqpoint{3.327271in}{2.897294in}}{\pgfqpoint{3.323999in}{2.889394in}}{\pgfqpoint{3.323999in}{2.881158in}}%
\pgfpathcurveto{\pgfqpoint{3.323999in}{2.872921in}}{\pgfqpoint{3.327271in}{2.865021in}}{\pgfqpoint{3.333095in}{2.859197in}}%
\pgfpathcurveto{\pgfqpoint{3.338919in}{2.853373in}}{\pgfqpoint{3.346819in}{2.850101in}}{\pgfqpoint{3.355055in}{2.850101in}}%
\pgfpathclose%
\pgfusepath{stroke,fill}%
\end{pgfscope}%
\begin{pgfscope}%
\pgfpathrectangle{\pgfqpoint{0.100000in}{0.220728in}}{\pgfqpoint{3.696000in}{3.696000in}}%
\pgfusepath{clip}%
\pgfsetbuttcap%
\pgfsetroundjoin%
\definecolor{currentfill}{rgb}{0.121569,0.466667,0.705882}%
\pgfsetfillcolor{currentfill}%
\pgfsetfillopacity{0.686719}%
\pgfsetlinewidth{1.003750pt}%
\definecolor{currentstroke}{rgb}{0.121569,0.466667,0.705882}%
\pgfsetstrokecolor{currentstroke}%
\pgfsetstrokeopacity{0.686719}%
\pgfsetdash{}{0pt}%
\pgfpathmoveto{\pgfqpoint{3.349222in}{2.841680in}}%
\pgfpathcurveto{\pgfqpoint{3.357458in}{2.841680in}}{\pgfqpoint{3.365358in}{2.844952in}}{\pgfqpoint{3.371182in}{2.850776in}}%
\pgfpathcurveto{\pgfqpoint{3.377006in}{2.856600in}}{\pgfqpoint{3.380278in}{2.864500in}}{\pgfqpoint{3.380278in}{2.872736in}}%
\pgfpathcurveto{\pgfqpoint{3.380278in}{2.880972in}}{\pgfqpoint{3.377006in}{2.888872in}}{\pgfqpoint{3.371182in}{2.894696in}}%
\pgfpathcurveto{\pgfqpoint{3.365358in}{2.900520in}}{\pgfqpoint{3.357458in}{2.903793in}}{\pgfqpoint{3.349222in}{2.903793in}}%
\pgfpathcurveto{\pgfqpoint{3.340985in}{2.903793in}}{\pgfqpoint{3.333085in}{2.900520in}}{\pgfqpoint{3.327261in}{2.894696in}}%
\pgfpathcurveto{\pgfqpoint{3.321437in}{2.888872in}}{\pgfqpoint{3.318165in}{2.880972in}}{\pgfqpoint{3.318165in}{2.872736in}}%
\pgfpathcurveto{\pgfqpoint{3.318165in}{2.864500in}}{\pgfqpoint{3.321437in}{2.856600in}}{\pgfqpoint{3.327261in}{2.850776in}}%
\pgfpathcurveto{\pgfqpoint{3.333085in}{2.844952in}}{\pgfqpoint{3.340985in}{2.841680in}}{\pgfqpoint{3.349222in}{2.841680in}}%
\pgfpathclose%
\pgfusepath{stroke,fill}%
\end{pgfscope}%
\begin{pgfscope}%
\pgfpathrectangle{\pgfqpoint{0.100000in}{0.220728in}}{\pgfqpoint{3.696000in}{3.696000in}}%
\pgfusepath{clip}%
\pgfsetbuttcap%
\pgfsetroundjoin%
\definecolor{currentfill}{rgb}{0.121569,0.466667,0.705882}%
\pgfsetfillcolor{currentfill}%
\pgfsetfillopacity{0.688649}%
\pgfsetlinewidth{1.003750pt}%
\definecolor{currentstroke}{rgb}{0.121569,0.466667,0.705882}%
\pgfsetstrokecolor{currentstroke}%
\pgfsetstrokeopacity{0.688649}%
\pgfsetdash{}{0pt}%
\pgfpathmoveto{\pgfqpoint{3.344383in}{2.830249in}}%
\pgfpathcurveto{\pgfqpoint{3.352619in}{2.830249in}}{\pgfqpoint{3.360519in}{2.833521in}}{\pgfqpoint{3.366343in}{2.839345in}}%
\pgfpathcurveto{\pgfqpoint{3.372167in}{2.845169in}}{\pgfqpoint{3.375439in}{2.853069in}}{\pgfqpoint{3.375439in}{2.861305in}}%
\pgfpathcurveto{\pgfqpoint{3.375439in}{2.869542in}}{\pgfqpoint{3.372167in}{2.877442in}}{\pgfqpoint{3.366343in}{2.883266in}}%
\pgfpathcurveto{\pgfqpoint{3.360519in}{2.889090in}}{\pgfqpoint{3.352619in}{2.892362in}}{\pgfqpoint{3.344383in}{2.892362in}}%
\pgfpathcurveto{\pgfqpoint{3.336147in}{2.892362in}}{\pgfqpoint{3.328247in}{2.889090in}}{\pgfqpoint{3.322423in}{2.883266in}}%
\pgfpathcurveto{\pgfqpoint{3.316599in}{2.877442in}}{\pgfqpoint{3.313326in}{2.869542in}}{\pgfqpoint{3.313326in}{2.861305in}}%
\pgfpathcurveto{\pgfqpoint{3.313326in}{2.853069in}}{\pgfqpoint{3.316599in}{2.845169in}}{\pgfqpoint{3.322423in}{2.839345in}}%
\pgfpathcurveto{\pgfqpoint{3.328247in}{2.833521in}}{\pgfqpoint{3.336147in}{2.830249in}}{\pgfqpoint{3.344383in}{2.830249in}}%
\pgfpathclose%
\pgfusepath{stroke,fill}%
\end{pgfscope}%
\begin{pgfscope}%
\pgfpathrectangle{\pgfqpoint{0.100000in}{0.220728in}}{\pgfqpoint{3.696000in}{3.696000in}}%
\pgfusepath{clip}%
\pgfsetbuttcap%
\pgfsetroundjoin%
\definecolor{currentfill}{rgb}{0.121569,0.466667,0.705882}%
\pgfsetfillcolor{currentfill}%
\pgfsetfillopacity{0.688788}%
\pgfsetlinewidth{1.003750pt}%
\definecolor{currentstroke}{rgb}{0.121569,0.466667,0.705882}%
\pgfsetstrokecolor{currentstroke}%
\pgfsetstrokeopacity{0.688788}%
\pgfsetdash{}{0pt}%
\pgfpathmoveto{\pgfqpoint{0.771002in}{1.386825in}}%
\pgfpathcurveto{\pgfqpoint{0.779239in}{1.386825in}}{\pgfqpoint{0.787139in}{1.390098in}}{\pgfqpoint{0.792963in}{1.395922in}}%
\pgfpathcurveto{\pgfqpoint{0.798787in}{1.401746in}}{\pgfqpoint{0.802059in}{1.409646in}}{\pgfqpoint{0.802059in}{1.417882in}}%
\pgfpathcurveto{\pgfqpoint{0.802059in}{1.426118in}}{\pgfqpoint{0.798787in}{1.434018in}}{\pgfqpoint{0.792963in}{1.439842in}}%
\pgfpathcurveto{\pgfqpoint{0.787139in}{1.445666in}}{\pgfqpoint{0.779239in}{1.448938in}}{\pgfqpoint{0.771002in}{1.448938in}}%
\pgfpathcurveto{\pgfqpoint{0.762766in}{1.448938in}}{\pgfqpoint{0.754866in}{1.445666in}}{\pgfqpoint{0.749042in}{1.439842in}}%
\pgfpathcurveto{\pgfqpoint{0.743218in}{1.434018in}}{\pgfqpoint{0.739946in}{1.426118in}}{\pgfqpoint{0.739946in}{1.417882in}}%
\pgfpathcurveto{\pgfqpoint{0.739946in}{1.409646in}}{\pgfqpoint{0.743218in}{1.401746in}}{\pgfqpoint{0.749042in}{1.395922in}}%
\pgfpathcurveto{\pgfqpoint{0.754866in}{1.390098in}}{\pgfqpoint{0.762766in}{1.386825in}}{\pgfqpoint{0.771002in}{1.386825in}}%
\pgfpathclose%
\pgfusepath{stroke,fill}%
\end{pgfscope}%
\begin{pgfscope}%
\pgfpathrectangle{\pgfqpoint{0.100000in}{0.220728in}}{\pgfqpoint{3.696000in}{3.696000in}}%
\pgfusepath{clip}%
\pgfsetbuttcap%
\pgfsetroundjoin%
\definecolor{currentfill}{rgb}{0.121569,0.466667,0.705882}%
\pgfsetfillcolor{currentfill}%
\pgfsetfillopacity{0.689498}%
\pgfsetlinewidth{1.003750pt}%
\definecolor{currentstroke}{rgb}{0.121569,0.466667,0.705882}%
\pgfsetstrokecolor{currentstroke}%
\pgfsetstrokeopacity{0.689498}%
\pgfsetdash{}{0pt}%
\pgfpathmoveto{\pgfqpoint{3.340778in}{2.824280in}}%
\pgfpathcurveto{\pgfqpoint{3.349014in}{2.824280in}}{\pgfqpoint{3.356914in}{2.827553in}}{\pgfqpoint{3.362738in}{2.833377in}}%
\pgfpathcurveto{\pgfqpoint{3.368562in}{2.839201in}}{\pgfqpoint{3.371835in}{2.847101in}}{\pgfqpoint{3.371835in}{2.855337in}}%
\pgfpathcurveto{\pgfqpoint{3.371835in}{2.863573in}}{\pgfqpoint{3.368562in}{2.871473in}}{\pgfqpoint{3.362738in}{2.877297in}}%
\pgfpathcurveto{\pgfqpoint{3.356914in}{2.883121in}}{\pgfqpoint{3.349014in}{2.886393in}}{\pgfqpoint{3.340778in}{2.886393in}}%
\pgfpathcurveto{\pgfqpoint{3.332542in}{2.886393in}}{\pgfqpoint{3.324642in}{2.883121in}}{\pgfqpoint{3.318818in}{2.877297in}}%
\pgfpathcurveto{\pgfqpoint{3.312994in}{2.871473in}}{\pgfqpoint{3.309722in}{2.863573in}}{\pgfqpoint{3.309722in}{2.855337in}}%
\pgfpathcurveto{\pgfqpoint{3.309722in}{2.847101in}}{\pgfqpoint{3.312994in}{2.839201in}}{\pgfqpoint{3.318818in}{2.833377in}}%
\pgfpathcurveto{\pgfqpoint{3.324642in}{2.827553in}}{\pgfqpoint{3.332542in}{2.824280in}}{\pgfqpoint{3.340778in}{2.824280in}}%
\pgfpathclose%
\pgfusepath{stroke,fill}%
\end{pgfscope}%
\begin{pgfscope}%
\pgfpathrectangle{\pgfqpoint{0.100000in}{0.220728in}}{\pgfqpoint{3.696000in}{3.696000in}}%
\pgfusepath{clip}%
\pgfsetbuttcap%
\pgfsetroundjoin%
\definecolor{currentfill}{rgb}{0.121569,0.466667,0.705882}%
\pgfsetfillcolor{currentfill}%
\pgfsetfillopacity{0.689997}%
\pgfsetlinewidth{1.003750pt}%
\definecolor{currentstroke}{rgb}{0.121569,0.466667,0.705882}%
\pgfsetstrokecolor{currentstroke}%
\pgfsetstrokeopacity{0.689997}%
\pgfsetdash{}{0pt}%
\pgfpathmoveto{\pgfqpoint{3.338810in}{2.821112in}}%
\pgfpathcurveto{\pgfqpoint{3.347046in}{2.821112in}}{\pgfqpoint{3.354946in}{2.824384in}}{\pgfqpoint{3.360770in}{2.830208in}}%
\pgfpathcurveto{\pgfqpoint{3.366594in}{2.836032in}}{\pgfqpoint{3.369866in}{2.843932in}}{\pgfqpoint{3.369866in}{2.852168in}}%
\pgfpathcurveto{\pgfqpoint{3.369866in}{2.860404in}}{\pgfqpoint{3.366594in}{2.868304in}}{\pgfqpoint{3.360770in}{2.874128in}}%
\pgfpathcurveto{\pgfqpoint{3.354946in}{2.879952in}}{\pgfqpoint{3.347046in}{2.883225in}}{\pgfqpoint{3.338810in}{2.883225in}}%
\pgfpathcurveto{\pgfqpoint{3.330573in}{2.883225in}}{\pgfqpoint{3.322673in}{2.879952in}}{\pgfqpoint{3.316849in}{2.874128in}}%
\pgfpathcurveto{\pgfqpoint{3.311025in}{2.868304in}}{\pgfqpoint{3.307753in}{2.860404in}}{\pgfqpoint{3.307753in}{2.852168in}}%
\pgfpathcurveto{\pgfqpoint{3.307753in}{2.843932in}}{\pgfqpoint{3.311025in}{2.836032in}}{\pgfqpoint{3.316849in}{2.830208in}}%
\pgfpathcurveto{\pgfqpoint{3.322673in}{2.824384in}}{\pgfqpoint{3.330573in}{2.821112in}}{\pgfqpoint{3.338810in}{2.821112in}}%
\pgfpathclose%
\pgfusepath{stroke,fill}%
\end{pgfscope}%
\begin{pgfscope}%
\pgfpathrectangle{\pgfqpoint{0.100000in}{0.220728in}}{\pgfqpoint{3.696000in}{3.696000in}}%
\pgfusepath{clip}%
\pgfsetbuttcap%
\pgfsetroundjoin%
\definecolor{currentfill}{rgb}{0.121569,0.466667,0.705882}%
\pgfsetfillcolor{currentfill}%
\pgfsetfillopacity{0.690285}%
\pgfsetlinewidth{1.003750pt}%
\definecolor{currentstroke}{rgb}{0.121569,0.466667,0.705882}%
\pgfsetstrokecolor{currentstroke}%
\pgfsetstrokeopacity{0.690285}%
\pgfsetdash{}{0pt}%
\pgfpathmoveto{\pgfqpoint{3.338056in}{2.819013in}}%
\pgfpathcurveto{\pgfqpoint{3.346293in}{2.819013in}}{\pgfqpoint{3.354193in}{2.822286in}}{\pgfqpoint{3.360017in}{2.828110in}}%
\pgfpathcurveto{\pgfqpoint{3.365841in}{2.833933in}}{\pgfqpoint{3.369113in}{2.841834in}}{\pgfqpoint{3.369113in}{2.850070in}}%
\pgfpathcurveto{\pgfqpoint{3.369113in}{2.858306in}}{\pgfqpoint{3.365841in}{2.866206in}}{\pgfqpoint{3.360017in}{2.872030in}}%
\pgfpathcurveto{\pgfqpoint{3.354193in}{2.877854in}}{\pgfqpoint{3.346293in}{2.881126in}}{\pgfqpoint{3.338056in}{2.881126in}}%
\pgfpathcurveto{\pgfqpoint{3.329820in}{2.881126in}}{\pgfqpoint{3.321920in}{2.877854in}}{\pgfqpoint{3.316096in}{2.872030in}}%
\pgfpathcurveto{\pgfqpoint{3.310272in}{2.866206in}}{\pgfqpoint{3.307000in}{2.858306in}}{\pgfqpoint{3.307000in}{2.850070in}}%
\pgfpathcurveto{\pgfqpoint{3.307000in}{2.841834in}}{\pgfqpoint{3.310272in}{2.833933in}}{\pgfqpoint{3.316096in}{2.828110in}}%
\pgfpathcurveto{\pgfqpoint{3.321920in}{2.822286in}}{\pgfqpoint{3.329820in}{2.819013in}}{\pgfqpoint{3.338056in}{2.819013in}}%
\pgfpathclose%
\pgfusepath{stroke,fill}%
\end{pgfscope}%
\begin{pgfscope}%
\pgfpathrectangle{\pgfqpoint{0.100000in}{0.220728in}}{\pgfqpoint{3.696000in}{3.696000in}}%
\pgfusepath{clip}%
\pgfsetbuttcap%
\pgfsetroundjoin%
\definecolor{currentfill}{rgb}{0.121569,0.466667,0.705882}%
\pgfsetfillcolor{currentfill}%
\pgfsetfillopacity{0.690805}%
\pgfsetlinewidth{1.003750pt}%
\definecolor{currentstroke}{rgb}{0.121569,0.466667,0.705882}%
\pgfsetstrokecolor{currentstroke}%
\pgfsetstrokeopacity{0.690805}%
\pgfsetdash{}{0pt}%
\pgfpathmoveto{\pgfqpoint{3.335168in}{2.814632in}}%
\pgfpathcurveto{\pgfqpoint{3.343404in}{2.814632in}}{\pgfqpoint{3.351304in}{2.817905in}}{\pgfqpoint{3.357128in}{2.823729in}}%
\pgfpathcurveto{\pgfqpoint{3.362952in}{2.829552in}}{\pgfqpoint{3.366224in}{2.837452in}}{\pgfqpoint{3.366224in}{2.845689in}}%
\pgfpathcurveto{\pgfqpoint{3.366224in}{2.853925in}}{\pgfqpoint{3.362952in}{2.861825in}}{\pgfqpoint{3.357128in}{2.867649in}}%
\pgfpathcurveto{\pgfqpoint{3.351304in}{2.873473in}}{\pgfqpoint{3.343404in}{2.876745in}}{\pgfqpoint{3.335168in}{2.876745in}}%
\pgfpathcurveto{\pgfqpoint{3.326931in}{2.876745in}}{\pgfqpoint{3.319031in}{2.873473in}}{\pgfqpoint{3.313207in}{2.867649in}}%
\pgfpathcurveto{\pgfqpoint{3.307383in}{2.861825in}}{\pgfqpoint{3.304111in}{2.853925in}}{\pgfqpoint{3.304111in}{2.845689in}}%
\pgfpathcurveto{\pgfqpoint{3.304111in}{2.837452in}}{\pgfqpoint{3.307383in}{2.829552in}}{\pgfqpoint{3.313207in}{2.823729in}}%
\pgfpathcurveto{\pgfqpoint{3.319031in}{2.817905in}}{\pgfqpoint{3.326931in}{2.814632in}}{\pgfqpoint{3.335168in}{2.814632in}}%
\pgfpathclose%
\pgfusepath{stroke,fill}%
\end{pgfscope}%
\begin{pgfscope}%
\pgfpathrectangle{\pgfqpoint{0.100000in}{0.220728in}}{\pgfqpoint{3.696000in}{3.696000in}}%
\pgfusepath{clip}%
\pgfsetbuttcap%
\pgfsetroundjoin%
\definecolor{currentfill}{rgb}{0.121569,0.466667,0.705882}%
\pgfsetfillcolor{currentfill}%
\pgfsetfillopacity{0.691207}%
\pgfsetlinewidth{1.003750pt}%
\definecolor{currentstroke}{rgb}{0.121569,0.466667,0.705882}%
\pgfsetstrokecolor{currentstroke}%
\pgfsetstrokeopacity{0.691207}%
\pgfsetdash{}{0pt}%
\pgfpathmoveto{\pgfqpoint{3.333972in}{2.812122in}}%
\pgfpathcurveto{\pgfqpoint{3.342209in}{2.812122in}}{\pgfqpoint{3.350109in}{2.815394in}}{\pgfqpoint{3.355933in}{2.821218in}}%
\pgfpathcurveto{\pgfqpoint{3.361757in}{2.827042in}}{\pgfqpoint{3.365029in}{2.834942in}}{\pgfqpoint{3.365029in}{2.843178in}}%
\pgfpathcurveto{\pgfqpoint{3.365029in}{2.851415in}}{\pgfqpoint{3.361757in}{2.859315in}}{\pgfqpoint{3.355933in}{2.865139in}}%
\pgfpathcurveto{\pgfqpoint{3.350109in}{2.870963in}}{\pgfqpoint{3.342209in}{2.874235in}}{\pgfqpoint{3.333972in}{2.874235in}}%
\pgfpathcurveto{\pgfqpoint{3.325736in}{2.874235in}}{\pgfqpoint{3.317836in}{2.870963in}}{\pgfqpoint{3.312012in}{2.865139in}}%
\pgfpathcurveto{\pgfqpoint{3.306188in}{2.859315in}}{\pgfqpoint{3.302916in}{2.851415in}}{\pgfqpoint{3.302916in}{2.843178in}}%
\pgfpathcurveto{\pgfqpoint{3.302916in}{2.834942in}}{\pgfqpoint{3.306188in}{2.827042in}}{\pgfqpoint{3.312012in}{2.821218in}}%
\pgfpathcurveto{\pgfqpoint{3.317836in}{2.815394in}}{\pgfqpoint{3.325736in}{2.812122in}}{\pgfqpoint{3.333972in}{2.812122in}}%
\pgfpathclose%
\pgfusepath{stroke,fill}%
\end{pgfscope}%
\begin{pgfscope}%
\pgfpathrectangle{\pgfqpoint{0.100000in}{0.220728in}}{\pgfqpoint{3.696000in}{3.696000in}}%
\pgfusepath{clip}%
\pgfsetbuttcap%
\pgfsetroundjoin%
\definecolor{currentfill}{rgb}{0.121569,0.466667,0.705882}%
\pgfsetfillcolor{currentfill}%
\pgfsetfillopacity{0.691262}%
\pgfsetlinewidth{1.003750pt}%
\definecolor{currentstroke}{rgb}{0.121569,0.466667,0.705882}%
\pgfsetstrokecolor{currentstroke}%
\pgfsetstrokeopacity{0.691262}%
\pgfsetdash{}{0pt}%
\pgfpathmoveto{\pgfqpoint{0.783661in}{1.377612in}}%
\pgfpathcurveto{\pgfqpoint{0.791897in}{1.377612in}}{\pgfqpoint{0.799797in}{1.380885in}}{\pgfqpoint{0.805621in}{1.386709in}}%
\pgfpathcurveto{\pgfqpoint{0.811445in}{1.392533in}}{\pgfqpoint{0.814718in}{1.400433in}}{\pgfqpoint{0.814718in}{1.408669in}}%
\pgfpathcurveto{\pgfqpoint{0.814718in}{1.416905in}}{\pgfqpoint{0.811445in}{1.424805in}}{\pgfqpoint{0.805621in}{1.430629in}}%
\pgfpathcurveto{\pgfqpoint{0.799797in}{1.436453in}}{\pgfqpoint{0.791897in}{1.439725in}}{\pgfqpoint{0.783661in}{1.439725in}}%
\pgfpathcurveto{\pgfqpoint{0.775425in}{1.439725in}}{\pgfqpoint{0.767525in}{1.436453in}}{\pgfqpoint{0.761701in}{1.430629in}}%
\pgfpathcurveto{\pgfqpoint{0.755877in}{1.424805in}}{\pgfqpoint{0.752605in}{1.416905in}}{\pgfqpoint{0.752605in}{1.408669in}}%
\pgfpathcurveto{\pgfqpoint{0.752605in}{1.400433in}}{\pgfqpoint{0.755877in}{1.392533in}}{\pgfqpoint{0.761701in}{1.386709in}}%
\pgfpathcurveto{\pgfqpoint{0.767525in}{1.380885in}}{\pgfqpoint{0.775425in}{1.377612in}}{\pgfqpoint{0.783661in}{1.377612in}}%
\pgfpathclose%
\pgfusepath{stroke,fill}%
\end{pgfscope}%
\begin{pgfscope}%
\pgfpathrectangle{\pgfqpoint{0.100000in}{0.220728in}}{\pgfqpoint{3.696000in}{3.696000in}}%
\pgfusepath{clip}%
\pgfsetbuttcap%
\pgfsetroundjoin%
\definecolor{currentfill}{rgb}{0.121569,0.466667,0.705882}%
\pgfsetfillcolor{currentfill}%
\pgfsetfillopacity{0.691462}%
\pgfsetlinewidth{1.003750pt}%
\definecolor{currentstroke}{rgb}{0.121569,0.466667,0.705882}%
\pgfsetstrokecolor{currentstroke}%
\pgfsetstrokeopacity{0.691462}%
\pgfsetdash{}{0pt}%
\pgfpathmoveto{\pgfqpoint{3.333421in}{2.810769in}}%
\pgfpathcurveto{\pgfqpoint{3.341658in}{2.810769in}}{\pgfqpoint{3.349558in}{2.814041in}}{\pgfqpoint{3.355382in}{2.819865in}}%
\pgfpathcurveto{\pgfqpoint{3.361206in}{2.825689in}}{\pgfqpoint{3.364478in}{2.833589in}}{\pgfqpoint{3.364478in}{2.841825in}}%
\pgfpathcurveto{\pgfqpoint{3.364478in}{2.850062in}}{\pgfqpoint{3.361206in}{2.857962in}}{\pgfqpoint{3.355382in}{2.863786in}}%
\pgfpathcurveto{\pgfqpoint{3.349558in}{2.869610in}}{\pgfqpoint{3.341658in}{2.872882in}}{\pgfqpoint{3.333421in}{2.872882in}}%
\pgfpathcurveto{\pgfqpoint{3.325185in}{2.872882in}}{\pgfqpoint{3.317285in}{2.869610in}}{\pgfqpoint{3.311461in}{2.863786in}}%
\pgfpathcurveto{\pgfqpoint{3.305637in}{2.857962in}}{\pgfqpoint{3.302365in}{2.850062in}}{\pgfqpoint{3.302365in}{2.841825in}}%
\pgfpathcurveto{\pgfqpoint{3.302365in}{2.833589in}}{\pgfqpoint{3.305637in}{2.825689in}}{\pgfqpoint{3.311461in}{2.819865in}}%
\pgfpathcurveto{\pgfqpoint{3.317285in}{2.814041in}}{\pgfqpoint{3.325185in}{2.810769in}}{\pgfqpoint{3.333421in}{2.810769in}}%
\pgfpathclose%
\pgfusepath{stroke,fill}%
\end{pgfscope}%
\begin{pgfscope}%
\pgfpathrectangle{\pgfqpoint{0.100000in}{0.220728in}}{\pgfqpoint{3.696000in}{3.696000in}}%
\pgfusepath{clip}%
\pgfsetbuttcap%
\pgfsetroundjoin%
\definecolor{currentfill}{rgb}{0.121569,0.466667,0.705882}%
\pgfsetfillcolor{currentfill}%
\pgfsetfillopacity{0.691542}%
\pgfsetlinewidth{1.003750pt}%
\definecolor{currentstroke}{rgb}{0.121569,0.466667,0.705882}%
\pgfsetstrokecolor{currentstroke}%
\pgfsetstrokeopacity{0.691542}%
\pgfsetdash{}{0pt}%
\pgfpathmoveto{\pgfqpoint{3.332931in}{2.810032in}}%
\pgfpathcurveto{\pgfqpoint{3.341167in}{2.810032in}}{\pgfqpoint{3.349067in}{2.813305in}}{\pgfqpoint{3.354891in}{2.819128in}}%
\pgfpathcurveto{\pgfqpoint{3.360715in}{2.824952in}}{\pgfqpoint{3.363987in}{2.832852in}}{\pgfqpoint{3.363987in}{2.841089in}}%
\pgfpathcurveto{\pgfqpoint{3.363987in}{2.849325in}}{\pgfqpoint{3.360715in}{2.857225in}}{\pgfqpoint{3.354891in}{2.863049in}}%
\pgfpathcurveto{\pgfqpoint{3.349067in}{2.868873in}}{\pgfqpoint{3.341167in}{2.872145in}}{\pgfqpoint{3.332931in}{2.872145in}}%
\pgfpathcurveto{\pgfqpoint{3.324694in}{2.872145in}}{\pgfqpoint{3.316794in}{2.868873in}}{\pgfqpoint{3.310970in}{2.863049in}}%
\pgfpathcurveto{\pgfqpoint{3.305147in}{2.857225in}}{\pgfqpoint{3.301874in}{2.849325in}}{\pgfqpoint{3.301874in}{2.841089in}}%
\pgfpathcurveto{\pgfqpoint{3.301874in}{2.832852in}}{\pgfqpoint{3.305147in}{2.824952in}}{\pgfqpoint{3.310970in}{2.819128in}}%
\pgfpathcurveto{\pgfqpoint{3.316794in}{2.813305in}}{\pgfqpoint{3.324694in}{2.810032in}}{\pgfqpoint{3.332931in}{2.810032in}}%
\pgfpathclose%
\pgfusepath{stroke,fill}%
\end{pgfscope}%
\begin{pgfscope}%
\pgfpathrectangle{\pgfqpoint{0.100000in}{0.220728in}}{\pgfqpoint{3.696000in}{3.696000in}}%
\pgfusepath{clip}%
\pgfsetbuttcap%
\pgfsetroundjoin%
\definecolor{currentfill}{rgb}{0.121569,0.466667,0.705882}%
\pgfsetfillcolor{currentfill}%
\pgfsetfillopacity{0.692026}%
\pgfsetlinewidth{1.003750pt}%
\definecolor{currentstroke}{rgb}{0.121569,0.466667,0.705882}%
\pgfsetstrokecolor{currentstroke}%
\pgfsetstrokeopacity{0.692026}%
\pgfsetdash{}{0pt}%
\pgfpathmoveto{\pgfqpoint{3.331774in}{2.807276in}}%
\pgfpathcurveto{\pgfqpoint{3.340010in}{2.807276in}}{\pgfqpoint{3.347910in}{2.810548in}}{\pgfqpoint{3.353734in}{2.816372in}}%
\pgfpathcurveto{\pgfqpoint{3.359558in}{2.822196in}}{\pgfqpoint{3.362830in}{2.830096in}}{\pgfqpoint{3.362830in}{2.838332in}}%
\pgfpathcurveto{\pgfqpoint{3.362830in}{2.846569in}}{\pgfqpoint{3.359558in}{2.854469in}}{\pgfqpoint{3.353734in}{2.860293in}}%
\pgfpathcurveto{\pgfqpoint{3.347910in}{2.866117in}}{\pgfqpoint{3.340010in}{2.869389in}}{\pgfqpoint{3.331774in}{2.869389in}}%
\pgfpathcurveto{\pgfqpoint{3.323537in}{2.869389in}}{\pgfqpoint{3.315637in}{2.866117in}}{\pgfqpoint{3.309814in}{2.860293in}}%
\pgfpathcurveto{\pgfqpoint{3.303990in}{2.854469in}}{\pgfqpoint{3.300717in}{2.846569in}}{\pgfqpoint{3.300717in}{2.838332in}}%
\pgfpathcurveto{\pgfqpoint{3.300717in}{2.830096in}}{\pgfqpoint{3.303990in}{2.822196in}}{\pgfqpoint{3.309814in}{2.816372in}}%
\pgfpathcurveto{\pgfqpoint{3.315637in}{2.810548in}}{\pgfqpoint{3.323537in}{2.807276in}}{\pgfqpoint{3.331774in}{2.807276in}}%
\pgfpathclose%
\pgfusepath{stroke,fill}%
\end{pgfscope}%
\begin{pgfscope}%
\pgfpathrectangle{\pgfqpoint{0.100000in}{0.220728in}}{\pgfqpoint{3.696000in}{3.696000in}}%
\pgfusepath{clip}%
\pgfsetbuttcap%
\pgfsetroundjoin%
\definecolor{currentfill}{rgb}{0.121569,0.466667,0.705882}%
\pgfsetfillcolor{currentfill}%
\pgfsetfillopacity{0.692579}%
\pgfsetlinewidth{1.003750pt}%
\definecolor{currentstroke}{rgb}{0.121569,0.466667,0.705882}%
\pgfsetstrokecolor{currentstroke}%
\pgfsetstrokeopacity{0.692579}%
\pgfsetdash{}{0pt}%
\pgfpathmoveto{\pgfqpoint{3.330003in}{2.804258in}}%
\pgfpathcurveto{\pgfqpoint{3.338239in}{2.804258in}}{\pgfqpoint{3.346139in}{2.807530in}}{\pgfqpoint{3.351963in}{2.813354in}}%
\pgfpathcurveto{\pgfqpoint{3.357787in}{2.819178in}}{\pgfqpoint{3.361059in}{2.827078in}}{\pgfqpoint{3.361059in}{2.835314in}}%
\pgfpathcurveto{\pgfqpoint{3.361059in}{2.843550in}}{\pgfqpoint{3.357787in}{2.851450in}}{\pgfqpoint{3.351963in}{2.857274in}}%
\pgfpathcurveto{\pgfqpoint{3.346139in}{2.863098in}}{\pgfqpoint{3.338239in}{2.866371in}}{\pgfqpoint{3.330003in}{2.866371in}}%
\pgfpathcurveto{\pgfqpoint{3.321767in}{2.866371in}}{\pgfqpoint{3.313866in}{2.863098in}}{\pgfqpoint{3.308043in}{2.857274in}}%
\pgfpathcurveto{\pgfqpoint{3.302219in}{2.851450in}}{\pgfqpoint{3.298946in}{2.843550in}}{\pgfqpoint{3.298946in}{2.835314in}}%
\pgfpathcurveto{\pgfqpoint{3.298946in}{2.827078in}}{\pgfqpoint{3.302219in}{2.819178in}}{\pgfqpoint{3.308043in}{2.813354in}}%
\pgfpathcurveto{\pgfqpoint{3.313866in}{2.807530in}}{\pgfqpoint{3.321767in}{2.804258in}}{\pgfqpoint{3.330003in}{2.804258in}}%
\pgfpathclose%
\pgfusepath{stroke,fill}%
\end{pgfscope}%
\begin{pgfscope}%
\pgfpathrectangle{\pgfqpoint{0.100000in}{0.220728in}}{\pgfqpoint{3.696000in}{3.696000in}}%
\pgfusepath{clip}%
\pgfsetbuttcap%
\pgfsetroundjoin%
\definecolor{currentfill}{rgb}{0.121569,0.466667,0.705882}%
\pgfsetfillcolor{currentfill}%
\pgfsetfillopacity{0.692739}%
\pgfsetlinewidth{1.003750pt}%
\definecolor{currentstroke}{rgb}{0.121569,0.466667,0.705882}%
\pgfsetstrokecolor{currentstroke}%
\pgfsetstrokeopacity{0.692739}%
\pgfsetdash{}{0pt}%
\pgfpathmoveto{\pgfqpoint{3.328806in}{2.802344in}}%
\pgfpathcurveto{\pgfqpoint{3.337042in}{2.802344in}}{\pgfqpoint{3.344942in}{2.805616in}}{\pgfqpoint{3.350766in}{2.811440in}}%
\pgfpathcurveto{\pgfqpoint{3.356590in}{2.817264in}}{\pgfqpoint{3.359862in}{2.825164in}}{\pgfqpoint{3.359862in}{2.833400in}}%
\pgfpathcurveto{\pgfqpoint{3.359862in}{2.841637in}}{\pgfqpoint{3.356590in}{2.849537in}}{\pgfqpoint{3.350766in}{2.855361in}}%
\pgfpathcurveto{\pgfqpoint{3.344942in}{2.861185in}}{\pgfqpoint{3.337042in}{2.864457in}}{\pgfqpoint{3.328806in}{2.864457in}}%
\pgfpathcurveto{\pgfqpoint{3.320569in}{2.864457in}}{\pgfqpoint{3.312669in}{2.861185in}}{\pgfqpoint{3.306845in}{2.855361in}}%
\pgfpathcurveto{\pgfqpoint{3.301021in}{2.849537in}}{\pgfqpoint{3.297749in}{2.841637in}}{\pgfqpoint{3.297749in}{2.833400in}}%
\pgfpathcurveto{\pgfqpoint{3.297749in}{2.825164in}}{\pgfqpoint{3.301021in}{2.817264in}}{\pgfqpoint{3.306845in}{2.811440in}}%
\pgfpathcurveto{\pgfqpoint{3.312669in}{2.805616in}}{\pgfqpoint{3.320569in}{2.802344in}}{\pgfqpoint{3.328806in}{2.802344in}}%
\pgfpathclose%
\pgfusepath{stroke,fill}%
\end{pgfscope}%
\begin{pgfscope}%
\pgfpathrectangle{\pgfqpoint{0.100000in}{0.220728in}}{\pgfqpoint{3.696000in}{3.696000in}}%
\pgfusepath{clip}%
\pgfsetbuttcap%
\pgfsetroundjoin%
\definecolor{currentfill}{rgb}{0.121569,0.466667,0.705882}%
\pgfsetfillcolor{currentfill}%
\pgfsetfillopacity{0.693224}%
\pgfsetlinewidth{1.003750pt}%
\definecolor{currentstroke}{rgb}{0.121569,0.466667,0.705882}%
\pgfsetstrokecolor{currentstroke}%
\pgfsetstrokeopacity{0.693224}%
\pgfsetdash{}{0pt}%
\pgfpathmoveto{\pgfqpoint{3.327804in}{2.799246in}}%
\pgfpathcurveto{\pgfqpoint{3.336041in}{2.799246in}}{\pgfqpoint{3.343941in}{2.802518in}}{\pgfqpoint{3.349765in}{2.808342in}}%
\pgfpathcurveto{\pgfqpoint{3.355589in}{2.814166in}}{\pgfqpoint{3.358861in}{2.822066in}}{\pgfqpoint{3.358861in}{2.830302in}}%
\pgfpathcurveto{\pgfqpoint{3.358861in}{2.838538in}}{\pgfqpoint{3.355589in}{2.846438in}}{\pgfqpoint{3.349765in}{2.852262in}}%
\pgfpathcurveto{\pgfqpoint{3.343941in}{2.858086in}}{\pgfqpoint{3.336041in}{2.861359in}}{\pgfqpoint{3.327804in}{2.861359in}}%
\pgfpathcurveto{\pgfqpoint{3.319568in}{2.861359in}}{\pgfqpoint{3.311668in}{2.858086in}}{\pgfqpoint{3.305844in}{2.852262in}}%
\pgfpathcurveto{\pgfqpoint{3.300020in}{2.846438in}}{\pgfqpoint{3.296748in}{2.838538in}}{\pgfqpoint{3.296748in}{2.830302in}}%
\pgfpathcurveto{\pgfqpoint{3.296748in}{2.822066in}}{\pgfqpoint{3.300020in}{2.814166in}}{\pgfqpoint{3.305844in}{2.808342in}}%
\pgfpathcurveto{\pgfqpoint{3.311668in}{2.802518in}}{\pgfqpoint{3.319568in}{2.799246in}}{\pgfqpoint{3.327804in}{2.799246in}}%
\pgfpathclose%
\pgfusepath{stroke,fill}%
\end{pgfscope}%
\begin{pgfscope}%
\pgfpathrectangle{\pgfqpoint{0.100000in}{0.220728in}}{\pgfqpoint{3.696000in}{3.696000in}}%
\pgfusepath{clip}%
\pgfsetbuttcap%
\pgfsetroundjoin%
\definecolor{currentfill}{rgb}{0.121569,0.466667,0.705882}%
\pgfsetfillcolor{currentfill}%
\pgfsetfillopacity{0.693554}%
\pgfsetlinewidth{1.003750pt}%
\definecolor{currentstroke}{rgb}{0.121569,0.466667,0.705882}%
\pgfsetstrokecolor{currentstroke}%
\pgfsetstrokeopacity{0.693554}%
\pgfsetdash{}{0pt}%
\pgfpathmoveto{\pgfqpoint{0.793728in}{1.373425in}}%
\pgfpathcurveto{\pgfqpoint{0.801964in}{1.373425in}}{\pgfqpoint{0.809864in}{1.376697in}}{\pgfqpoint{0.815688in}{1.382521in}}%
\pgfpathcurveto{\pgfqpoint{0.821512in}{1.388345in}}{\pgfqpoint{0.824784in}{1.396245in}}{\pgfqpoint{0.824784in}{1.404481in}}%
\pgfpathcurveto{\pgfqpoint{0.824784in}{1.412718in}}{\pgfqpoint{0.821512in}{1.420618in}}{\pgfqpoint{0.815688in}{1.426442in}}%
\pgfpathcurveto{\pgfqpoint{0.809864in}{1.432265in}}{\pgfqpoint{0.801964in}{1.435538in}}{\pgfqpoint{0.793728in}{1.435538in}}%
\pgfpathcurveto{\pgfqpoint{0.785491in}{1.435538in}}{\pgfqpoint{0.777591in}{1.432265in}}{\pgfqpoint{0.771767in}{1.426442in}}%
\pgfpathcurveto{\pgfqpoint{0.765943in}{1.420618in}}{\pgfqpoint{0.762671in}{1.412718in}}{\pgfqpoint{0.762671in}{1.404481in}}%
\pgfpathcurveto{\pgfqpoint{0.762671in}{1.396245in}}{\pgfqpoint{0.765943in}{1.388345in}}{\pgfqpoint{0.771767in}{1.382521in}}%
\pgfpathcurveto{\pgfqpoint{0.777591in}{1.376697in}}{\pgfqpoint{0.785491in}{1.373425in}}{\pgfqpoint{0.793728in}{1.373425in}}%
\pgfpathclose%
\pgfusepath{stroke,fill}%
\end{pgfscope}%
\begin{pgfscope}%
\pgfpathrectangle{\pgfqpoint{0.100000in}{0.220728in}}{\pgfqpoint{3.696000in}{3.696000in}}%
\pgfusepath{clip}%
\pgfsetbuttcap%
\pgfsetroundjoin%
\definecolor{currentfill}{rgb}{0.121569,0.466667,0.705882}%
\pgfsetfillcolor{currentfill}%
\pgfsetfillopacity{0.693860}%
\pgfsetlinewidth{1.003750pt}%
\definecolor{currentstroke}{rgb}{0.121569,0.466667,0.705882}%
\pgfsetstrokecolor{currentstroke}%
\pgfsetstrokeopacity{0.693860}%
\pgfsetdash{}{0pt}%
\pgfpathmoveto{\pgfqpoint{3.324617in}{2.794689in}}%
\pgfpathcurveto{\pgfqpoint{3.332854in}{2.794689in}}{\pgfqpoint{3.340754in}{2.797961in}}{\pgfqpoint{3.346578in}{2.803785in}}%
\pgfpathcurveto{\pgfqpoint{3.352401in}{2.809609in}}{\pgfqpoint{3.355674in}{2.817509in}}{\pgfqpoint{3.355674in}{2.825745in}}%
\pgfpathcurveto{\pgfqpoint{3.355674in}{2.833982in}}{\pgfqpoint{3.352401in}{2.841882in}}{\pgfqpoint{3.346578in}{2.847706in}}%
\pgfpathcurveto{\pgfqpoint{3.340754in}{2.853529in}}{\pgfqpoint{3.332854in}{2.856802in}}{\pgfqpoint{3.324617in}{2.856802in}}%
\pgfpathcurveto{\pgfqpoint{3.316381in}{2.856802in}}{\pgfqpoint{3.308481in}{2.853529in}}{\pgfqpoint{3.302657in}{2.847706in}}%
\pgfpathcurveto{\pgfqpoint{3.296833in}{2.841882in}}{\pgfqpoint{3.293561in}{2.833982in}}{\pgfqpoint{3.293561in}{2.825745in}}%
\pgfpathcurveto{\pgfqpoint{3.293561in}{2.817509in}}{\pgfqpoint{3.296833in}{2.809609in}}{\pgfqpoint{3.302657in}{2.803785in}}%
\pgfpathcurveto{\pgfqpoint{3.308481in}{2.797961in}}{\pgfqpoint{3.316381in}{2.794689in}}{\pgfqpoint{3.324617in}{2.794689in}}%
\pgfpathclose%
\pgfusepath{stroke,fill}%
\end{pgfscope}%
\begin{pgfscope}%
\pgfpathrectangle{\pgfqpoint{0.100000in}{0.220728in}}{\pgfqpoint{3.696000in}{3.696000in}}%
\pgfusepath{clip}%
\pgfsetbuttcap%
\pgfsetroundjoin%
\definecolor{currentfill}{rgb}{0.121569,0.466667,0.705882}%
\pgfsetfillcolor{currentfill}%
\pgfsetfillopacity{0.694343}%
\pgfsetlinewidth{1.003750pt}%
\definecolor{currentstroke}{rgb}{0.121569,0.466667,0.705882}%
\pgfsetstrokecolor{currentstroke}%
\pgfsetstrokeopacity{0.694343}%
\pgfsetdash{}{0pt}%
\pgfpathmoveto{\pgfqpoint{3.323232in}{2.792173in}}%
\pgfpathcurveto{\pgfqpoint{3.331468in}{2.792173in}}{\pgfqpoint{3.339368in}{2.795446in}}{\pgfqpoint{3.345192in}{2.801270in}}%
\pgfpathcurveto{\pgfqpoint{3.351016in}{2.807094in}}{\pgfqpoint{3.354288in}{2.814994in}}{\pgfqpoint{3.354288in}{2.823230in}}%
\pgfpathcurveto{\pgfqpoint{3.354288in}{2.831466in}}{\pgfqpoint{3.351016in}{2.839366in}}{\pgfqpoint{3.345192in}{2.845190in}}%
\pgfpathcurveto{\pgfqpoint{3.339368in}{2.851014in}}{\pgfqpoint{3.331468in}{2.854286in}}{\pgfqpoint{3.323232in}{2.854286in}}%
\pgfpathcurveto{\pgfqpoint{3.314996in}{2.854286in}}{\pgfqpoint{3.307095in}{2.851014in}}{\pgfqpoint{3.301272in}{2.845190in}}%
\pgfpathcurveto{\pgfqpoint{3.295448in}{2.839366in}}{\pgfqpoint{3.292175in}{2.831466in}}{\pgfqpoint{3.292175in}{2.823230in}}%
\pgfpathcurveto{\pgfqpoint{3.292175in}{2.814994in}}{\pgfqpoint{3.295448in}{2.807094in}}{\pgfqpoint{3.301272in}{2.801270in}}%
\pgfpathcurveto{\pgfqpoint{3.307095in}{2.795446in}}{\pgfqpoint{3.314996in}{2.792173in}}{\pgfqpoint{3.323232in}{2.792173in}}%
\pgfpathclose%
\pgfusepath{stroke,fill}%
\end{pgfscope}%
\begin{pgfscope}%
\pgfpathrectangle{\pgfqpoint{0.100000in}{0.220728in}}{\pgfqpoint{3.696000in}{3.696000in}}%
\pgfusepath{clip}%
\pgfsetbuttcap%
\pgfsetroundjoin%
\definecolor{currentfill}{rgb}{0.121569,0.466667,0.705882}%
\pgfsetfillcolor{currentfill}%
\pgfsetfillopacity{0.694664}%
\pgfsetlinewidth{1.003750pt}%
\definecolor{currentstroke}{rgb}{0.121569,0.466667,0.705882}%
\pgfsetstrokecolor{currentstroke}%
\pgfsetstrokeopacity{0.694664}%
\pgfsetdash{}{0pt}%
\pgfpathmoveto{\pgfqpoint{3.322643in}{2.790837in}}%
\pgfpathcurveto{\pgfqpoint{3.330879in}{2.790837in}}{\pgfqpoint{3.338779in}{2.794110in}}{\pgfqpoint{3.344603in}{2.799934in}}%
\pgfpathcurveto{\pgfqpoint{3.350427in}{2.805757in}}{\pgfqpoint{3.353699in}{2.813658in}}{\pgfqpoint{3.353699in}{2.821894in}}%
\pgfpathcurveto{\pgfqpoint{3.353699in}{2.830130in}}{\pgfqpoint{3.350427in}{2.838030in}}{\pgfqpoint{3.344603in}{2.843854in}}%
\pgfpathcurveto{\pgfqpoint{3.338779in}{2.849678in}}{\pgfqpoint{3.330879in}{2.852950in}}{\pgfqpoint{3.322643in}{2.852950in}}%
\pgfpathcurveto{\pgfqpoint{3.314406in}{2.852950in}}{\pgfqpoint{3.306506in}{2.849678in}}{\pgfqpoint{3.300682in}{2.843854in}}%
\pgfpathcurveto{\pgfqpoint{3.294858in}{2.838030in}}{\pgfqpoint{3.291586in}{2.830130in}}{\pgfqpoint{3.291586in}{2.821894in}}%
\pgfpathcurveto{\pgfqpoint{3.291586in}{2.813658in}}{\pgfqpoint{3.294858in}{2.805757in}}{\pgfqpoint{3.300682in}{2.799934in}}%
\pgfpathcurveto{\pgfqpoint{3.306506in}{2.794110in}}{\pgfqpoint{3.314406in}{2.790837in}}{\pgfqpoint{3.322643in}{2.790837in}}%
\pgfpathclose%
\pgfusepath{stroke,fill}%
\end{pgfscope}%
\begin{pgfscope}%
\pgfpathrectangle{\pgfqpoint{0.100000in}{0.220728in}}{\pgfqpoint{3.696000in}{3.696000in}}%
\pgfusepath{clip}%
\pgfsetbuttcap%
\pgfsetroundjoin%
\definecolor{currentfill}{rgb}{0.121569,0.466667,0.705882}%
\pgfsetfillcolor{currentfill}%
\pgfsetfillopacity{0.694792}%
\pgfsetlinewidth{1.003750pt}%
\definecolor{currentstroke}{rgb}{0.121569,0.466667,0.705882}%
\pgfsetstrokecolor{currentstroke}%
\pgfsetstrokeopacity{0.694792}%
\pgfsetdash{}{0pt}%
\pgfpathmoveto{\pgfqpoint{3.322140in}{2.790126in}}%
\pgfpathcurveto{\pgfqpoint{3.330377in}{2.790126in}}{\pgfqpoint{3.338277in}{2.793398in}}{\pgfqpoint{3.344100in}{2.799222in}}%
\pgfpathcurveto{\pgfqpoint{3.349924in}{2.805046in}}{\pgfqpoint{3.353197in}{2.812946in}}{\pgfqpoint{3.353197in}{2.821182in}}%
\pgfpathcurveto{\pgfqpoint{3.353197in}{2.829419in}}{\pgfqpoint{3.349924in}{2.837319in}}{\pgfqpoint{3.344100in}{2.843143in}}%
\pgfpathcurveto{\pgfqpoint{3.338277in}{2.848966in}}{\pgfqpoint{3.330377in}{2.852239in}}{\pgfqpoint{3.322140in}{2.852239in}}%
\pgfpathcurveto{\pgfqpoint{3.313904in}{2.852239in}}{\pgfqpoint{3.306004in}{2.848966in}}{\pgfqpoint{3.300180in}{2.843143in}}%
\pgfpathcurveto{\pgfqpoint{3.294356in}{2.837319in}}{\pgfqpoint{3.291084in}{2.829419in}}{\pgfqpoint{3.291084in}{2.821182in}}%
\pgfpathcurveto{\pgfqpoint{3.291084in}{2.812946in}}{\pgfqpoint{3.294356in}{2.805046in}}{\pgfqpoint{3.300180in}{2.799222in}}%
\pgfpathcurveto{\pgfqpoint{3.306004in}{2.793398in}}{\pgfqpoint{3.313904in}{2.790126in}}{\pgfqpoint{3.322140in}{2.790126in}}%
\pgfpathclose%
\pgfusepath{stroke,fill}%
\end{pgfscope}%
\begin{pgfscope}%
\pgfpathrectangle{\pgfqpoint{0.100000in}{0.220728in}}{\pgfqpoint{3.696000in}{3.696000in}}%
\pgfusepath{clip}%
\pgfsetbuttcap%
\pgfsetroundjoin%
\definecolor{currentfill}{rgb}{0.121569,0.466667,0.705882}%
\pgfsetfillcolor{currentfill}%
\pgfsetfillopacity{0.694894}%
\pgfsetlinewidth{1.003750pt}%
\definecolor{currentstroke}{rgb}{0.121569,0.466667,0.705882}%
\pgfsetstrokecolor{currentstroke}%
\pgfsetstrokeopacity{0.694894}%
\pgfsetdash{}{0pt}%
\pgfpathmoveto{\pgfqpoint{0.801730in}{1.369460in}}%
\pgfpathcurveto{\pgfqpoint{0.809966in}{1.369460in}}{\pgfqpoint{0.817866in}{1.372732in}}{\pgfqpoint{0.823690in}{1.378556in}}%
\pgfpathcurveto{\pgfqpoint{0.829514in}{1.384380in}}{\pgfqpoint{0.832786in}{1.392280in}}{\pgfqpoint{0.832786in}{1.400517in}}%
\pgfpathcurveto{\pgfqpoint{0.832786in}{1.408753in}}{\pgfqpoint{0.829514in}{1.416653in}}{\pgfqpoint{0.823690in}{1.422477in}}%
\pgfpathcurveto{\pgfqpoint{0.817866in}{1.428301in}}{\pgfqpoint{0.809966in}{1.431573in}}{\pgfqpoint{0.801730in}{1.431573in}}%
\pgfpathcurveto{\pgfqpoint{0.793494in}{1.431573in}}{\pgfqpoint{0.785594in}{1.428301in}}{\pgfqpoint{0.779770in}{1.422477in}}%
\pgfpathcurveto{\pgfqpoint{0.773946in}{1.416653in}}{\pgfqpoint{0.770673in}{1.408753in}}{\pgfqpoint{0.770673in}{1.400517in}}%
\pgfpathcurveto{\pgfqpoint{0.770673in}{1.392280in}}{\pgfqpoint{0.773946in}{1.384380in}}{\pgfqpoint{0.779770in}{1.378556in}}%
\pgfpathcurveto{\pgfqpoint{0.785594in}{1.372732in}}{\pgfqpoint{0.793494in}{1.369460in}}{\pgfqpoint{0.801730in}{1.369460in}}%
\pgfpathclose%
\pgfusepath{stroke,fill}%
\end{pgfscope}%
\begin{pgfscope}%
\pgfpathrectangle{\pgfqpoint{0.100000in}{0.220728in}}{\pgfqpoint{3.696000in}{3.696000in}}%
\pgfusepath{clip}%
\pgfsetbuttcap%
\pgfsetroundjoin%
\definecolor{currentfill}{rgb}{0.121569,0.466667,0.705882}%
\pgfsetfillcolor{currentfill}%
\pgfsetfillopacity{0.695319}%
\pgfsetlinewidth{1.003750pt}%
\definecolor{currentstroke}{rgb}{0.121569,0.466667,0.705882}%
\pgfsetstrokecolor{currentstroke}%
\pgfsetstrokeopacity{0.695319}%
\pgfsetdash{}{0pt}%
\pgfpathmoveto{\pgfqpoint{3.320978in}{2.787819in}}%
\pgfpathcurveto{\pgfqpoint{3.329214in}{2.787819in}}{\pgfqpoint{3.337114in}{2.791091in}}{\pgfqpoint{3.342938in}{2.796915in}}%
\pgfpathcurveto{\pgfqpoint{3.348762in}{2.802739in}}{\pgfqpoint{3.352034in}{2.810639in}}{\pgfqpoint{3.352034in}{2.818875in}}%
\pgfpathcurveto{\pgfqpoint{3.352034in}{2.827112in}}{\pgfqpoint{3.348762in}{2.835012in}}{\pgfqpoint{3.342938in}{2.840836in}}%
\pgfpathcurveto{\pgfqpoint{3.337114in}{2.846659in}}{\pgfqpoint{3.329214in}{2.849932in}}{\pgfqpoint{3.320978in}{2.849932in}}%
\pgfpathcurveto{\pgfqpoint{3.312742in}{2.849932in}}{\pgfqpoint{3.304841in}{2.846659in}}{\pgfqpoint{3.299018in}{2.840836in}}%
\pgfpathcurveto{\pgfqpoint{3.293194in}{2.835012in}}{\pgfqpoint{3.289921in}{2.827112in}}{\pgfqpoint{3.289921in}{2.818875in}}%
\pgfpathcurveto{\pgfqpoint{3.289921in}{2.810639in}}{\pgfqpoint{3.293194in}{2.802739in}}{\pgfqpoint{3.299018in}{2.796915in}}%
\pgfpathcurveto{\pgfqpoint{3.304841in}{2.791091in}}{\pgfqpoint{3.312742in}{2.787819in}}{\pgfqpoint{3.320978in}{2.787819in}}%
\pgfpathclose%
\pgfusepath{stroke,fill}%
\end{pgfscope}%
\begin{pgfscope}%
\pgfpathrectangle{\pgfqpoint{0.100000in}{0.220728in}}{\pgfqpoint{3.696000in}{3.696000in}}%
\pgfusepath{clip}%
\pgfsetbuttcap%
\pgfsetroundjoin%
\definecolor{currentfill}{rgb}{0.121569,0.466667,0.705882}%
\pgfsetfillcolor{currentfill}%
\pgfsetfillopacity{0.695598}%
\pgfsetlinewidth{1.003750pt}%
\definecolor{currentstroke}{rgb}{0.121569,0.466667,0.705882}%
\pgfsetstrokecolor{currentstroke}%
\pgfsetstrokeopacity{0.695598}%
\pgfsetdash{}{0pt}%
\pgfpathmoveto{\pgfqpoint{3.320248in}{2.786615in}}%
\pgfpathcurveto{\pgfqpoint{3.328484in}{2.786615in}}{\pgfqpoint{3.336384in}{2.789887in}}{\pgfqpoint{3.342208in}{2.795711in}}%
\pgfpathcurveto{\pgfqpoint{3.348032in}{2.801535in}}{\pgfqpoint{3.351304in}{2.809435in}}{\pgfqpoint{3.351304in}{2.817672in}}%
\pgfpathcurveto{\pgfqpoint{3.351304in}{2.825908in}}{\pgfqpoint{3.348032in}{2.833808in}}{\pgfqpoint{3.342208in}{2.839632in}}%
\pgfpathcurveto{\pgfqpoint{3.336384in}{2.845456in}}{\pgfqpoint{3.328484in}{2.848728in}}{\pgfqpoint{3.320248in}{2.848728in}}%
\pgfpathcurveto{\pgfqpoint{3.312012in}{2.848728in}}{\pgfqpoint{3.304112in}{2.845456in}}{\pgfqpoint{3.298288in}{2.839632in}}%
\pgfpathcurveto{\pgfqpoint{3.292464in}{2.833808in}}{\pgfqpoint{3.289191in}{2.825908in}}{\pgfqpoint{3.289191in}{2.817672in}}%
\pgfpathcurveto{\pgfqpoint{3.289191in}{2.809435in}}{\pgfqpoint{3.292464in}{2.801535in}}{\pgfqpoint{3.298288in}{2.795711in}}%
\pgfpathcurveto{\pgfqpoint{3.304112in}{2.789887in}}{\pgfqpoint{3.312012in}{2.786615in}}{\pgfqpoint{3.320248in}{2.786615in}}%
\pgfpathclose%
\pgfusepath{stroke,fill}%
\end{pgfscope}%
\begin{pgfscope}%
\pgfpathrectangle{\pgfqpoint{0.100000in}{0.220728in}}{\pgfqpoint{3.696000in}{3.696000in}}%
\pgfusepath{clip}%
\pgfsetbuttcap%
\pgfsetroundjoin%
\definecolor{currentfill}{rgb}{0.121569,0.466667,0.705882}%
\pgfsetfillcolor{currentfill}%
\pgfsetfillopacity{0.695737}%
\pgfsetlinewidth{1.003750pt}%
\definecolor{currentstroke}{rgb}{0.121569,0.466667,0.705882}%
\pgfsetstrokecolor{currentstroke}%
\pgfsetstrokeopacity{0.695737}%
\pgfsetdash{}{0pt}%
\pgfpathmoveto{\pgfqpoint{3.319802in}{2.785953in}}%
\pgfpathcurveto{\pgfqpoint{3.328039in}{2.785953in}}{\pgfqpoint{3.335939in}{2.789225in}}{\pgfqpoint{3.341762in}{2.795049in}}%
\pgfpathcurveto{\pgfqpoint{3.347586in}{2.800873in}}{\pgfqpoint{3.350859in}{2.808773in}}{\pgfqpoint{3.350859in}{2.817010in}}%
\pgfpathcurveto{\pgfqpoint{3.350859in}{2.825246in}}{\pgfqpoint{3.347586in}{2.833146in}}{\pgfqpoint{3.341762in}{2.838970in}}%
\pgfpathcurveto{\pgfqpoint{3.335939in}{2.844794in}}{\pgfqpoint{3.328039in}{2.848066in}}{\pgfqpoint{3.319802in}{2.848066in}}%
\pgfpathcurveto{\pgfqpoint{3.311566in}{2.848066in}}{\pgfqpoint{3.303666in}{2.844794in}}{\pgfqpoint{3.297842in}{2.838970in}}%
\pgfpathcurveto{\pgfqpoint{3.292018in}{2.833146in}}{\pgfqpoint{3.288746in}{2.825246in}}{\pgfqpoint{3.288746in}{2.817010in}}%
\pgfpathcurveto{\pgfqpoint{3.288746in}{2.808773in}}{\pgfqpoint{3.292018in}{2.800873in}}{\pgfqpoint{3.297842in}{2.795049in}}%
\pgfpathcurveto{\pgfqpoint{3.303666in}{2.789225in}}{\pgfqpoint{3.311566in}{2.785953in}}{\pgfqpoint{3.319802in}{2.785953in}}%
\pgfpathclose%
\pgfusepath{stroke,fill}%
\end{pgfscope}%
\begin{pgfscope}%
\pgfpathrectangle{\pgfqpoint{0.100000in}{0.220728in}}{\pgfqpoint{3.696000in}{3.696000in}}%
\pgfusepath{clip}%
\pgfsetbuttcap%
\pgfsetroundjoin%
\definecolor{currentfill}{rgb}{0.121569,0.466667,0.705882}%
\pgfsetfillcolor{currentfill}%
\pgfsetfillopacity{0.695815}%
\pgfsetlinewidth{1.003750pt}%
\definecolor{currentstroke}{rgb}{0.121569,0.466667,0.705882}%
\pgfsetstrokecolor{currentstroke}%
\pgfsetstrokeopacity{0.695815}%
\pgfsetdash{}{0pt}%
\pgfpathmoveto{\pgfqpoint{3.319649in}{2.785494in}}%
\pgfpathcurveto{\pgfqpoint{3.327885in}{2.785494in}}{\pgfqpoint{3.335785in}{2.788766in}}{\pgfqpoint{3.341609in}{2.794590in}}%
\pgfpathcurveto{\pgfqpoint{3.347433in}{2.800414in}}{\pgfqpoint{3.350705in}{2.808314in}}{\pgfqpoint{3.350705in}{2.816550in}}%
\pgfpathcurveto{\pgfqpoint{3.350705in}{2.824787in}}{\pgfqpoint{3.347433in}{2.832687in}}{\pgfqpoint{3.341609in}{2.838511in}}%
\pgfpathcurveto{\pgfqpoint{3.335785in}{2.844334in}}{\pgfqpoint{3.327885in}{2.847607in}}{\pgfqpoint{3.319649in}{2.847607in}}%
\pgfpathcurveto{\pgfqpoint{3.311412in}{2.847607in}}{\pgfqpoint{3.303512in}{2.844334in}}{\pgfqpoint{3.297688in}{2.838511in}}%
\pgfpathcurveto{\pgfqpoint{3.291864in}{2.832687in}}{\pgfqpoint{3.288592in}{2.824787in}}{\pgfqpoint{3.288592in}{2.816550in}}%
\pgfpathcurveto{\pgfqpoint{3.288592in}{2.808314in}}{\pgfqpoint{3.291864in}{2.800414in}}{\pgfqpoint{3.297688in}{2.794590in}}%
\pgfpathcurveto{\pgfqpoint{3.303512in}{2.788766in}}{\pgfqpoint{3.311412in}{2.785494in}}{\pgfqpoint{3.319649in}{2.785494in}}%
\pgfpathclose%
\pgfusepath{stroke,fill}%
\end{pgfscope}%
\begin{pgfscope}%
\pgfpathrectangle{\pgfqpoint{0.100000in}{0.220728in}}{\pgfqpoint{3.696000in}{3.696000in}}%
\pgfusepath{clip}%
\pgfsetbuttcap%
\pgfsetroundjoin%
\definecolor{currentfill}{rgb}{0.121569,0.466667,0.705882}%
\pgfsetfillcolor{currentfill}%
\pgfsetfillopacity{0.696230}%
\pgfsetlinewidth{1.003750pt}%
\definecolor{currentstroke}{rgb}{0.121569,0.466667,0.705882}%
\pgfsetstrokecolor{currentstroke}%
\pgfsetstrokeopacity{0.696230}%
\pgfsetdash{}{0pt}%
\pgfpathmoveto{\pgfqpoint{3.317885in}{2.782867in}}%
\pgfpathcurveto{\pgfqpoint{3.326121in}{2.782867in}}{\pgfqpoint{3.334021in}{2.786139in}}{\pgfqpoint{3.339845in}{2.791963in}}%
\pgfpathcurveto{\pgfqpoint{3.345669in}{2.797787in}}{\pgfqpoint{3.348941in}{2.805687in}}{\pgfqpoint{3.348941in}{2.813923in}}%
\pgfpathcurveto{\pgfqpoint{3.348941in}{2.822159in}}{\pgfqpoint{3.345669in}{2.830059in}}{\pgfqpoint{3.339845in}{2.835883in}}%
\pgfpathcurveto{\pgfqpoint{3.334021in}{2.841707in}}{\pgfqpoint{3.326121in}{2.844980in}}{\pgfqpoint{3.317885in}{2.844980in}}%
\pgfpathcurveto{\pgfqpoint{3.309648in}{2.844980in}}{\pgfqpoint{3.301748in}{2.841707in}}{\pgfqpoint{3.295924in}{2.835883in}}%
\pgfpathcurveto{\pgfqpoint{3.290100in}{2.830059in}}{\pgfqpoint{3.286828in}{2.822159in}}{\pgfqpoint{3.286828in}{2.813923in}}%
\pgfpathcurveto{\pgfqpoint{3.286828in}{2.805687in}}{\pgfqpoint{3.290100in}{2.797787in}}{\pgfqpoint{3.295924in}{2.791963in}}%
\pgfpathcurveto{\pgfqpoint{3.301748in}{2.786139in}}{\pgfqpoint{3.309648in}{2.782867in}}{\pgfqpoint{3.317885in}{2.782867in}}%
\pgfpathclose%
\pgfusepath{stroke,fill}%
\end{pgfscope}%
\begin{pgfscope}%
\pgfpathrectangle{\pgfqpoint{0.100000in}{0.220728in}}{\pgfqpoint{3.696000in}{3.696000in}}%
\pgfusepath{clip}%
\pgfsetbuttcap%
\pgfsetroundjoin%
\definecolor{currentfill}{rgb}{0.121569,0.466667,0.705882}%
\pgfsetfillcolor{currentfill}%
\pgfsetfillopacity{0.696489}%
\pgfsetlinewidth{1.003750pt}%
\definecolor{currentstroke}{rgb}{0.121569,0.466667,0.705882}%
\pgfsetstrokecolor{currentstroke}%
\pgfsetstrokeopacity{0.696489}%
\pgfsetdash{}{0pt}%
\pgfpathmoveto{\pgfqpoint{3.317080in}{2.781302in}}%
\pgfpathcurveto{\pgfqpoint{3.325316in}{2.781302in}}{\pgfqpoint{3.333216in}{2.784574in}}{\pgfqpoint{3.339040in}{2.790398in}}%
\pgfpathcurveto{\pgfqpoint{3.344864in}{2.796222in}}{\pgfqpoint{3.348136in}{2.804122in}}{\pgfqpoint{3.348136in}{2.812358in}}%
\pgfpathcurveto{\pgfqpoint{3.348136in}{2.820595in}}{\pgfqpoint{3.344864in}{2.828495in}}{\pgfqpoint{3.339040in}{2.834319in}}%
\pgfpathcurveto{\pgfqpoint{3.333216in}{2.840143in}}{\pgfqpoint{3.325316in}{2.843415in}}{\pgfqpoint{3.317080in}{2.843415in}}%
\pgfpathcurveto{\pgfqpoint{3.308843in}{2.843415in}}{\pgfqpoint{3.300943in}{2.840143in}}{\pgfqpoint{3.295119in}{2.834319in}}%
\pgfpathcurveto{\pgfqpoint{3.289296in}{2.828495in}}{\pgfqpoint{3.286023in}{2.820595in}}{\pgfqpoint{3.286023in}{2.812358in}}%
\pgfpathcurveto{\pgfqpoint{3.286023in}{2.804122in}}{\pgfqpoint{3.289296in}{2.796222in}}{\pgfqpoint{3.295119in}{2.790398in}}%
\pgfpathcurveto{\pgfqpoint{3.300943in}{2.784574in}}{\pgfqpoint{3.308843in}{2.781302in}}{\pgfqpoint{3.317080in}{2.781302in}}%
\pgfpathclose%
\pgfusepath{stroke,fill}%
\end{pgfscope}%
\begin{pgfscope}%
\pgfpathrectangle{\pgfqpoint{0.100000in}{0.220728in}}{\pgfqpoint{3.696000in}{3.696000in}}%
\pgfusepath{clip}%
\pgfsetbuttcap%
\pgfsetroundjoin%
\definecolor{currentfill}{rgb}{0.121569,0.466667,0.705882}%
\pgfsetfillcolor{currentfill}%
\pgfsetfillopacity{0.696861}%
\pgfsetlinewidth{1.003750pt}%
\definecolor{currentstroke}{rgb}{0.121569,0.466667,0.705882}%
\pgfsetstrokecolor{currentstroke}%
\pgfsetstrokeopacity{0.696861}%
\pgfsetdash{}{0pt}%
\pgfpathmoveto{\pgfqpoint{3.316064in}{2.779079in}}%
\pgfpathcurveto{\pgfqpoint{3.324301in}{2.779079in}}{\pgfqpoint{3.332201in}{2.782352in}}{\pgfqpoint{3.338024in}{2.788176in}}%
\pgfpathcurveto{\pgfqpoint{3.343848in}{2.793999in}}{\pgfqpoint{3.347121in}{2.801900in}}{\pgfqpoint{3.347121in}{2.810136in}}%
\pgfpathcurveto{\pgfqpoint{3.347121in}{2.818372in}}{\pgfqpoint{3.343848in}{2.826272in}}{\pgfqpoint{3.338024in}{2.832096in}}%
\pgfpathcurveto{\pgfqpoint{3.332201in}{2.837920in}}{\pgfqpoint{3.324301in}{2.841192in}}{\pgfqpoint{3.316064in}{2.841192in}}%
\pgfpathcurveto{\pgfqpoint{3.307828in}{2.841192in}}{\pgfqpoint{3.299928in}{2.837920in}}{\pgfqpoint{3.294104in}{2.832096in}}%
\pgfpathcurveto{\pgfqpoint{3.288280in}{2.826272in}}{\pgfqpoint{3.285008in}{2.818372in}}{\pgfqpoint{3.285008in}{2.810136in}}%
\pgfpathcurveto{\pgfqpoint{3.285008in}{2.801900in}}{\pgfqpoint{3.288280in}{2.793999in}}{\pgfqpoint{3.294104in}{2.788176in}}%
\pgfpathcurveto{\pgfqpoint{3.299928in}{2.782352in}}{\pgfqpoint{3.307828in}{2.779079in}}{\pgfqpoint{3.316064in}{2.779079in}}%
\pgfpathclose%
\pgfusepath{stroke,fill}%
\end{pgfscope}%
\begin{pgfscope}%
\pgfpathrectangle{\pgfqpoint{0.100000in}{0.220728in}}{\pgfqpoint{3.696000in}{3.696000in}}%
\pgfusepath{clip}%
\pgfsetbuttcap%
\pgfsetroundjoin%
\definecolor{currentfill}{rgb}{0.121569,0.466667,0.705882}%
\pgfsetfillcolor{currentfill}%
\pgfsetfillopacity{0.697042}%
\pgfsetlinewidth{1.003750pt}%
\definecolor{currentstroke}{rgb}{0.121569,0.466667,0.705882}%
\pgfsetstrokecolor{currentstroke}%
\pgfsetstrokeopacity{0.697042}%
\pgfsetdash{}{0pt}%
\pgfpathmoveto{\pgfqpoint{3.315302in}{2.778055in}}%
\pgfpathcurveto{\pgfqpoint{3.323538in}{2.778055in}}{\pgfqpoint{3.331438in}{2.781328in}}{\pgfqpoint{3.337262in}{2.787151in}}%
\pgfpathcurveto{\pgfqpoint{3.343086in}{2.792975in}}{\pgfqpoint{3.346358in}{2.800875in}}{\pgfqpoint{3.346358in}{2.809112in}}%
\pgfpathcurveto{\pgfqpoint{3.346358in}{2.817348in}}{\pgfqpoint{3.343086in}{2.825248in}}{\pgfqpoint{3.337262in}{2.831072in}}%
\pgfpathcurveto{\pgfqpoint{3.331438in}{2.836896in}}{\pgfqpoint{3.323538in}{2.840168in}}{\pgfqpoint{3.315302in}{2.840168in}}%
\pgfpathcurveto{\pgfqpoint{3.307065in}{2.840168in}}{\pgfqpoint{3.299165in}{2.836896in}}{\pgfqpoint{3.293341in}{2.831072in}}%
\pgfpathcurveto{\pgfqpoint{3.287518in}{2.825248in}}{\pgfqpoint{3.284245in}{2.817348in}}{\pgfqpoint{3.284245in}{2.809112in}}%
\pgfpathcurveto{\pgfqpoint{3.284245in}{2.800875in}}{\pgfqpoint{3.287518in}{2.792975in}}{\pgfqpoint{3.293341in}{2.787151in}}%
\pgfpathcurveto{\pgfqpoint{3.299165in}{2.781328in}}{\pgfqpoint{3.307065in}{2.778055in}}{\pgfqpoint{3.315302in}{2.778055in}}%
\pgfpathclose%
\pgfusepath{stroke,fill}%
\end{pgfscope}%
\begin{pgfscope}%
\pgfpathrectangle{\pgfqpoint{0.100000in}{0.220728in}}{\pgfqpoint{3.696000in}{3.696000in}}%
\pgfusepath{clip}%
\pgfsetbuttcap%
\pgfsetroundjoin%
\definecolor{currentfill}{rgb}{0.121569,0.466667,0.705882}%
\pgfsetfillcolor{currentfill}%
\pgfsetfillopacity{0.697203}%
\pgfsetlinewidth{1.003750pt}%
\definecolor{currentstroke}{rgb}{0.121569,0.466667,0.705882}%
\pgfsetstrokecolor{currentstroke}%
\pgfsetstrokeopacity{0.697203}%
\pgfsetdash{}{0pt}%
\pgfpathmoveto{\pgfqpoint{0.816284in}{1.361670in}}%
\pgfpathcurveto{\pgfqpoint{0.824520in}{1.361670in}}{\pgfqpoint{0.832421in}{1.364943in}}{\pgfqpoint{0.838244in}{1.370767in}}%
\pgfpathcurveto{\pgfqpoint{0.844068in}{1.376591in}}{\pgfqpoint{0.847341in}{1.384491in}}{\pgfqpoint{0.847341in}{1.392727in}}%
\pgfpathcurveto{\pgfqpoint{0.847341in}{1.400963in}}{\pgfqpoint{0.844068in}{1.408863in}}{\pgfqpoint{0.838244in}{1.414687in}}%
\pgfpathcurveto{\pgfqpoint{0.832421in}{1.420511in}}{\pgfqpoint{0.824520in}{1.423783in}}{\pgfqpoint{0.816284in}{1.423783in}}%
\pgfpathcurveto{\pgfqpoint{0.808048in}{1.423783in}}{\pgfqpoint{0.800148in}{1.420511in}}{\pgfqpoint{0.794324in}{1.414687in}}%
\pgfpathcurveto{\pgfqpoint{0.788500in}{1.408863in}}{\pgfqpoint{0.785228in}{1.400963in}}{\pgfqpoint{0.785228in}{1.392727in}}%
\pgfpathcurveto{\pgfqpoint{0.785228in}{1.384491in}}{\pgfqpoint{0.788500in}{1.376591in}}{\pgfqpoint{0.794324in}{1.370767in}}%
\pgfpathcurveto{\pgfqpoint{0.800148in}{1.364943in}}{\pgfqpoint{0.808048in}{1.361670in}}{\pgfqpoint{0.816284in}{1.361670in}}%
\pgfpathclose%
\pgfusepath{stroke,fill}%
\end{pgfscope}%
\begin{pgfscope}%
\pgfpathrectangle{\pgfqpoint{0.100000in}{0.220728in}}{\pgfqpoint{3.696000in}{3.696000in}}%
\pgfusepath{clip}%
\pgfsetbuttcap%
\pgfsetroundjoin%
\definecolor{currentfill}{rgb}{0.121569,0.466667,0.705882}%
\pgfsetfillcolor{currentfill}%
\pgfsetfillopacity{0.697587}%
\pgfsetlinewidth{1.003750pt}%
\definecolor{currentstroke}{rgb}{0.121569,0.466667,0.705882}%
\pgfsetstrokecolor{currentstroke}%
\pgfsetstrokeopacity{0.697587}%
\pgfsetdash{}{0pt}%
\pgfpathmoveto{\pgfqpoint{3.313984in}{2.774884in}}%
\pgfpathcurveto{\pgfqpoint{3.322220in}{2.774884in}}{\pgfqpoint{3.330121in}{2.778156in}}{\pgfqpoint{3.335944in}{2.783980in}}%
\pgfpathcurveto{\pgfqpoint{3.341768in}{2.789804in}}{\pgfqpoint{3.345041in}{2.797704in}}{\pgfqpoint{3.345041in}{2.805940in}}%
\pgfpathcurveto{\pgfqpoint{3.345041in}{2.814176in}}{\pgfqpoint{3.341768in}{2.822076in}}{\pgfqpoint{3.335944in}{2.827900in}}%
\pgfpathcurveto{\pgfqpoint{3.330121in}{2.833724in}}{\pgfqpoint{3.322220in}{2.836997in}}{\pgfqpoint{3.313984in}{2.836997in}}%
\pgfpathcurveto{\pgfqpoint{3.305748in}{2.836997in}}{\pgfqpoint{3.297848in}{2.833724in}}{\pgfqpoint{3.292024in}{2.827900in}}%
\pgfpathcurveto{\pgfqpoint{3.286200in}{2.822076in}}{\pgfqpoint{3.282928in}{2.814176in}}{\pgfqpoint{3.282928in}{2.805940in}}%
\pgfpathcurveto{\pgfqpoint{3.282928in}{2.797704in}}{\pgfqpoint{3.286200in}{2.789804in}}{\pgfqpoint{3.292024in}{2.783980in}}%
\pgfpathcurveto{\pgfqpoint{3.297848in}{2.778156in}}{\pgfqpoint{3.305748in}{2.774884in}}{\pgfqpoint{3.313984in}{2.774884in}}%
\pgfpathclose%
\pgfusepath{stroke,fill}%
\end{pgfscope}%
\begin{pgfscope}%
\pgfpathrectangle{\pgfqpoint{0.100000in}{0.220728in}}{\pgfqpoint{3.696000in}{3.696000in}}%
\pgfusepath{clip}%
\pgfsetbuttcap%
\pgfsetroundjoin%
\definecolor{currentfill}{rgb}{0.121569,0.466667,0.705882}%
\pgfsetfillcolor{currentfill}%
\pgfsetfillopacity{0.697878}%
\pgfsetlinewidth{1.003750pt}%
\definecolor{currentstroke}{rgb}{0.121569,0.466667,0.705882}%
\pgfsetstrokecolor{currentstroke}%
\pgfsetstrokeopacity{0.697878}%
\pgfsetdash{}{0pt}%
\pgfpathmoveto{\pgfqpoint{3.313079in}{2.773316in}}%
\pgfpathcurveto{\pgfqpoint{3.321315in}{2.773316in}}{\pgfqpoint{3.329215in}{2.776589in}}{\pgfqpoint{3.335039in}{2.782413in}}%
\pgfpathcurveto{\pgfqpoint{3.340863in}{2.788237in}}{\pgfqpoint{3.344135in}{2.796137in}}{\pgfqpoint{3.344135in}{2.804373in}}%
\pgfpathcurveto{\pgfqpoint{3.344135in}{2.812609in}}{\pgfqpoint{3.340863in}{2.820509in}}{\pgfqpoint{3.335039in}{2.826333in}}%
\pgfpathcurveto{\pgfqpoint{3.329215in}{2.832157in}}{\pgfqpoint{3.321315in}{2.835429in}}{\pgfqpoint{3.313079in}{2.835429in}}%
\pgfpathcurveto{\pgfqpoint{3.304843in}{2.835429in}}{\pgfqpoint{3.296943in}{2.832157in}}{\pgfqpoint{3.291119in}{2.826333in}}%
\pgfpathcurveto{\pgfqpoint{3.285295in}{2.820509in}}{\pgfqpoint{3.282022in}{2.812609in}}{\pgfqpoint{3.282022in}{2.804373in}}%
\pgfpathcurveto{\pgfqpoint{3.282022in}{2.796137in}}{\pgfqpoint{3.285295in}{2.788237in}}{\pgfqpoint{3.291119in}{2.782413in}}%
\pgfpathcurveto{\pgfqpoint{3.296943in}{2.776589in}}{\pgfqpoint{3.304843in}{2.773316in}}{\pgfqpoint{3.313079in}{2.773316in}}%
\pgfpathclose%
\pgfusepath{stroke,fill}%
\end{pgfscope}%
\begin{pgfscope}%
\pgfpathrectangle{\pgfqpoint{0.100000in}{0.220728in}}{\pgfqpoint{3.696000in}{3.696000in}}%
\pgfusepath{clip}%
\pgfsetbuttcap%
\pgfsetroundjoin%
\definecolor{currentfill}{rgb}{0.121569,0.466667,0.705882}%
\pgfsetfillcolor{currentfill}%
\pgfsetfillopacity{0.698044}%
\pgfsetlinewidth{1.003750pt}%
\definecolor{currentstroke}{rgb}{0.121569,0.466667,0.705882}%
\pgfsetstrokecolor{currentstroke}%
\pgfsetstrokeopacity{0.698044}%
\pgfsetdash{}{0pt}%
\pgfpathmoveto{\pgfqpoint{3.312528in}{2.772564in}}%
\pgfpathcurveto{\pgfqpoint{3.320764in}{2.772564in}}{\pgfqpoint{3.328665in}{2.775837in}}{\pgfqpoint{3.334488in}{2.781661in}}%
\pgfpathcurveto{\pgfqpoint{3.340312in}{2.787485in}}{\pgfqpoint{3.343585in}{2.795385in}}{\pgfqpoint{3.343585in}{2.803621in}}%
\pgfpathcurveto{\pgfqpoint{3.343585in}{2.811857in}}{\pgfqpoint{3.340312in}{2.819757in}}{\pgfqpoint{3.334488in}{2.825581in}}%
\pgfpathcurveto{\pgfqpoint{3.328665in}{2.831405in}}{\pgfqpoint{3.320764in}{2.834677in}}{\pgfqpoint{3.312528in}{2.834677in}}%
\pgfpathcurveto{\pgfqpoint{3.304292in}{2.834677in}}{\pgfqpoint{3.296392in}{2.831405in}}{\pgfqpoint{3.290568in}{2.825581in}}%
\pgfpathcurveto{\pgfqpoint{3.284744in}{2.819757in}}{\pgfqpoint{3.281472in}{2.811857in}}{\pgfqpoint{3.281472in}{2.803621in}}%
\pgfpathcurveto{\pgfqpoint{3.281472in}{2.795385in}}{\pgfqpoint{3.284744in}{2.787485in}}{\pgfqpoint{3.290568in}{2.781661in}}%
\pgfpathcurveto{\pgfqpoint{3.296392in}{2.775837in}}{\pgfqpoint{3.304292in}{2.772564in}}{\pgfqpoint{3.312528in}{2.772564in}}%
\pgfpathclose%
\pgfusepath{stroke,fill}%
\end{pgfscope}%
\begin{pgfscope}%
\pgfpathrectangle{\pgfqpoint{0.100000in}{0.220728in}}{\pgfqpoint{3.696000in}{3.696000in}}%
\pgfusepath{clip}%
\pgfsetbuttcap%
\pgfsetroundjoin%
\definecolor{currentfill}{rgb}{0.121569,0.466667,0.705882}%
\pgfsetfillcolor{currentfill}%
\pgfsetfillopacity{0.698349}%
\pgfsetlinewidth{1.003750pt}%
\definecolor{currentstroke}{rgb}{0.121569,0.466667,0.705882}%
\pgfsetstrokecolor{currentstroke}%
\pgfsetstrokeopacity{0.698349}%
\pgfsetdash{}{0pt}%
\pgfpathmoveto{\pgfqpoint{3.311618in}{2.770312in}}%
\pgfpathcurveto{\pgfqpoint{3.319854in}{2.770312in}}{\pgfqpoint{3.327754in}{2.773584in}}{\pgfqpoint{3.333578in}{2.779408in}}%
\pgfpathcurveto{\pgfqpoint{3.339402in}{2.785232in}}{\pgfqpoint{3.342675in}{2.793132in}}{\pgfqpoint{3.342675in}{2.801369in}}%
\pgfpathcurveto{\pgfqpoint{3.342675in}{2.809605in}}{\pgfqpoint{3.339402in}{2.817505in}}{\pgfqpoint{3.333578in}{2.823329in}}%
\pgfpathcurveto{\pgfqpoint{3.327754in}{2.829153in}}{\pgfqpoint{3.319854in}{2.832425in}}{\pgfqpoint{3.311618in}{2.832425in}}%
\pgfpathcurveto{\pgfqpoint{3.303382in}{2.832425in}}{\pgfqpoint{3.295482in}{2.829153in}}{\pgfqpoint{3.289658in}{2.823329in}}%
\pgfpathcurveto{\pgfqpoint{3.283834in}{2.817505in}}{\pgfqpoint{3.280562in}{2.809605in}}{\pgfqpoint{3.280562in}{2.801369in}}%
\pgfpathcurveto{\pgfqpoint{3.280562in}{2.793132in}}{\pgfqpoint{3.283834in}{2.785232in}}{\pgfqpoint{3.289658in}{2.779408in}}%
\pgfpathcurveto{\pgfqpoint{3.295482in}{2.773584in}}{\pgfqpoint{3.303382in}{2.770312in}}{\pgfqpoint{3.311618in}{2.770312in}}%
\pgfpathclose%
\pgfusepath{stroke,fill}%
\end{pgfscope}%
\begin{pgfscope}%
\pgfpathrectangle{\pgfqpoint{0.100000in}{0.220728in}}{\pgfqpoint{3.696000in}{3.696000in}}%
\pgfusepath{clip}%
\pgfsetbuttcap%
\pgfsetroundjoin%
\definecolor{currentfill}{rgb}{0.121569,0.466667,0.705882}%
\pgfsetfillcolor{currentfill}%
\pgfsetfillopacity{0.698828}%
\pgfsetlinewidth{1.003750pt}%
\definecolor{currentstroke}{rgb}{0.121569,0.466667,0.705882}%
\pgfsetstrokecolor{currentstroke}%
\pgfsetstrokeopacity{0.698828}%
\pgfsetdash{}{0pt}%
\pgfpathmoveto{\pgfqpoint{3.309615in}{2.767164in}}%
\pgfpathcurveto{\pgfqpoint{3.317851in}{2.767164in}}{\pgfqpoint{3.325752in}{2.770436in}}{\pgfqpoint{3.331575in}{2.776260in}}%
\pgfpathcurveto{\pgfqpoint{3.337399in}{2.782084in}}{\pgfqpoint{3.340672in}{2.789984in}}{\pgfqpoint{3.340672in}{2.798220in}}%
\pgfpathcurveto{\pgfqpoint{3.340672in}{2.806456in}}{\pgfqpoint{3.337399in}{2.814356in}}{\pgfqpoint{3.331575in}{2.820180in}}%
\pgfpathcurveto{\pgfqpoint{3.325752in}{2.826004in}}{\pgfqpoint{3.317851in}{2.829277in}}{\pgfqpoint{3.309615in}{2.829277in}}%
\pgfpathcurveto{\pgfqpoint{3.301379in}{2.829277in}}{\pgfqpoint{3.293479in}{2.826004in}}{\pgfqpoint{3.287655in}{2.820180in}}%
\pgfpathcurveto{\pgfqpoint{3.281831in}{2.814356in}}{\pgfqpoint{3.278559in}{2.806456in}}{\pgfqpoint{3.278559in}{2.798220in}}%
\pgfpathcurveto{\pgfqpoint{3.278559in}{2.789984in}}{\pgfqpoint{3.281831in}{2.782084in}}{\pgfqpoint{3.287655in}{2.776260in}}%
\pgfpathcurveto{\pgfqpoint{3.293479in}{2.770436in}}{\pgfqpoint{3.301379in}{2.767164in}}{\pgfqpoint{3.309615in}{2.767164in}}%
\pgfpathclose%
\pgfusepath{stroke,fill}%
\end{pgfscope}%
\begin{pgfscope}%
\pgfpathrectangle{\pgfqpoint{0.100000in}{0.220728in}}{\pgfqpoint{3.696000in}{3.696000in}}%
\pgfusepath{clip}%
\pgfsetbuttcap%
\pgfsetroundjoin%
\definecolor{currentfill}{rgb}{0.121569,0.466667,0.705882}%
\pgfsetfillcolor{currentfill}%
\pgfsetfillopacity{0.699130}%
\pgfsetlinewidth{1.003750pt}%
\definecolor{currentstroke}{rgb}{0.121569,0.466667,0.705882}%
\pgfsetstrokecolor{currentstroke}%
\pgfsetstrokeopacity{0.699130}%
\pgfsetdash{}{0pt}%
\pgfpathmoveto{\pgfqpoint{3.308587in}{2.765487in}}%
\pgfpathcurveto{\pgfqpoint{3.316823in}{2.765487in}}{\pgfqpoint{3.324723in}{2.768760in}}{\pgfqpoint{3.330547in}{2.774583in}}%
\pgfpathcurveto{\pgfqpoint{3.336371in}{2.780407in}}{\pgfqpoint{3.339643in}{2.788307in}}{\pgfqpoint{3.339643in}{2.796544in}}%
\pgfpathcurveto{\pgfqpoint{3.339643in}{2.804780in}}{\pgfqpoint{3.336371in}{2.812680in}}{\pgfqpoint{3.330547in}{2.818504in}}%
\pgfpathcurveto{\pgfqpoint{3.324723in}{2.824328in}}{\pgfqpoint{3.316823in}{2.827600in}}{\pgfqpoint{3.308587in}{2.827600in}}%
\pgfpathcurveto{\pgfqpoint{3.300350in}{2.827600in}}{\pgfqpoint{3.292450in}{2.824328in}}{\pgfqpoint{3.286626in}{2.818504in}}%
\pgfpathcurveto{\pgfqpoint{3.280802in}{2.812680in}}{\pgfqpoint{3.277530in}{2.804780in}}{\pgfqpoint{3.277530in}{2.796544in}}%
\pgfpathcurveto{\pgfqpoint{3.277530in}{2.788307in}}{\pgfqpoint{3.280802in}{2.780407in}}{\pgfqpoint{3.286626in}{2.774583in}}%
\pgfpathcurveto{\pgfqpoint{3.292450in}{2.768760in}}{\pgfqpoint{3.300350in}{2.765487in}}{\pgfqpoint{3.308587in}{2.765487in}}%
\pgfpathclose%
\pgfusepath{stroke,fill}%
\end{pgfscope}%
\begin{pgfscope}%
\pgfpathrectangle{\pgfqpoint{0.100000in}{0.220728in}}{\pgfqpoint{3.696000in}{3.696000in}}%
\pgfusepath{clip}%
\pgfsetbuttcap%
\pgfsetroundjoin%
\definecolor{currentfill}{rgb}{0.121569,0.466667,0.705882}%
\pgfsetfillcolor{currentfill}%
\pgfsetfillopacity{0.699309}%
\pgfsetlinewidth{1.003750pt}%
\definecolor{currentstroke}{rgb}{0.121569,0.466667,0.705882}%
\pgfsetstrokecolor{currentstroke}%
\pgfsetstrokeopacity{0.699309}%
\pgfsetdash{}{0pt}%
\pgfpathmoveto{\pgfqpoint{3.308131in}{2.764475in}}%
\pgfpathcurveto{\pgfqpoint{3.316367in}{2.764475in}}{\pgfqpoint{3.324267in}{2.767747in}}{\pgfqpoint{3.330091in}{2.773571in}}%
\pgfpathcurveto{\pgfqpoint{3.335915in}{2.779395in}}{\pgfqpoint{3.339187in}{2.787295in}}{\pgfqpoint{3.339187in}{2.795532in}}%
\pgfpathcurveto{\pgfqpoint{3.339187in}{2.803768in}}{\pgfqpoint{3.335915in}{2.811668in}}{\pgfqpoint{3.330091in}{2.817492in}}%
\pgfpathcurveto{\pgfqpoint{3.324267in}{2.823316in}}{\pgfqpoint{3.316367in}{2.826588in}}{\pgfqpoint{3.308131in}{2.826588in}}%
\pgfpathcurveto{\pgfqpoint{3.299894in}{2.826588in}}{\pgfqpoint{3.291994in}{2.823316in}}{\pgfqpoint{3.286170in}{2.817492in}}%
\pgfpathcurveto{\pgfqpoint{3.280346in}{2.811668in}}{\pgfqpoint{3.277074in}{2.803768in}}{\pgfqpoint{3.277074in}{2.795532in}}%
\pgfpathcurveto{\pgfqpoint{3.277074in}{2.787295in}}{\pgfqpoint{3.280346in}{2.779395in}}{\pgfqpoint{3.286170in}{2.773571in}}%
\pgfpathcurveto{\pgfqpoint{3.291994in}{2.767747in}}{\pgfqpoint{3.299894in}{2.764475in}}{\pgfqpoint{3.308131in}{2.764475in}}%
\pgfpathclose%
\pgfusepath{stroke,fill}%
\end{pgfscope}%
\begin{pgfscope}%
\pgfpathrectangle{\pgfqpoint{0.100000in}{0.220728in}}{\pgfqpoint{3.696000in}{3.696000in}}%
\pgfusepath{clip}%
\pgfsetbuttcap%
\pgfsetroundjoin%
\definecolor{currentfill}{rgb}{0.121569,0.466667,0.705882}%
\pgfsetfillcolor{currentfill}%
\pgfsetfillopacity{0.699616}%
\pgfsetlinewidth{1.003750pt}%
\definecolor{currentstroke}{rgb}{0.121569,0.466667,0.705882}%
\pgfsetstrokecolor{currentstroke}%
\pgfsetstrokeopacity{0.699616}%
\pgfsetdash{}{0pt}%
\pgfpathmoveto{\pgfqpoint{3.306785in}{2.762551in}}%
\pgfpathcurveto{\pgfqpoint{3.315021in}{2.762551in}}{\pgfqpoint{3.322921in}{2.765823in}}{\pgfqpoint{3.328745in}{2.771647in}}%
\pgfpathcurveto{\pgfqpoint{3.334569in}{2.777471in}}{\pgfqpoint{3.337841in}{2.785371in}}{\pgfqpoint{3.337841in}{2.793608in}}%
\pgfpathcurveto{\pgfqpoint{3.337841in}{2.801844in}}{\pgfqpoint{3.334569in}{2.809744in}}{\pgfqpoint{3.328745in}{2.815568in}}%
\pgfpathcurveto{\pgfqpoint{3.322921in}{2.821392in}}{\pgfqpoint{3.315021in}{2.824664in}}{\pgfqpoint{3.306785in}{2.824664in}}%
\pgfpathcurveto{\pgfqpoint{3.298548in}{2.824664in}}{\pgfqpoint{3.290648in}{2.821392in}}{\pgfqpoint{3.284824in}{2.815568in}}%
\pgfpathcurveto{\pgfqpoint{3.279000in}{2.809744in}}{\pgfqpoint{3.275728in}{2.801844in}}{\pgfqpoint{3.275728in}{2.793608in}}%
\pgfpathcurveto{\pgfqpoint{3.275728in}{2.785371in}}{\pgfqpoint{3.279000in}{2.777471in}}{\pgfqpoint{3.284824in}{2.771647in}}%
\pgfpathcurveto{\pgfqpoint{3.290648in}{2.765823in}}{\pgfqpoint{3.298548in}{2.762551in}}{\pgfqpoint{3.306785in}{2.762551in}}%
\pgfpathclose%
\pgfusepath{stroke,fill}%
\end{pgfscope}%
\begin{pgfscope}%
\pgfpathrectangle{\pgfqpoint{0.100000in}{0.220728in}}{\pgfqpoint{3.696000in}{3.696000in}}%
\pgfusepath{clip}%
\pgfsetbuttcap%
\pgfsetroundjoin%
\definecolor{currentfill}{rgb}{0.121569,0.466667,0.705882}%
\pgfsetfillcolor{currentfill}%
\pgfsetfillopacity{0.699737}%
\pgfsetlinewidth{1.003750pt}%
\definecolor{currentstroke}{rgb}{0.121569,0.466667,0.705882}%
\pgfsetstrokecolor{currentstroke}%
\pgfsetstrokeopacity{0.699737}%
\pgfsetdash{}{0pt}%
\pgfpathmoveto{\pgfqpoint{0.828118in}{1.355636in}}%
\pgfpathcurveto{\pgfqpoint{0.836354in}{1.355636in}}{\pgfqpoint{0.844254in}{1.358908in}}{\pgfqpoint{0.850078in}{1.364732in}}%
\pgfpathcurveto{\pgfqpoint{0.855902in}{1.370556in}}{\pgfqpoint{0.859174in}{1.378456in}}{\pgfqpoint{0.859174in}{1.386692in}}%
\pgfpathcurveto{\pgfqpoint{0.859174in}{1.394929in}}{\pgfqpoint{0.855902in}{1.402829in}}{\pgfqpoint{0.850078in}{1.408653in}}%
\pgfpathcurveto{\pgfqpoint{0.844254in}{1.414477in}}{\pgfqpoint{0.836354in}{1.417749in}}{\pgfqpoint{0.828118in}{1.417749in}}%
\pgfpathcurveto{\pgfqpoint{0.819881in}{1.417749in}}{\pgfqpoint{0.811981in}{1.414477in}}{\pgfqpoint{0.806157in}{1.408653in}}%
\pgfpathcurveto{\pgfqpoint{0.800334in}{1.402829in}}{\pgfqpoint{0.797061in}{1.394929in}}{\pgfqpoint{0.797061in}{1.386692in}}%
\pgfpathcurveto{\pgfqpoint{0.797061in}{1.378456in}}{\pgfqpoint{0.800334in}{1.370556in}}{\pgfqpoint{0.806157in}{1.364732in}}%
\pgfpathcurveto{\pgfqpoint{0.811981in}{1.358908in}}{\pgfqpoint{0.819881in}{1.355636in}}{\pgfqpoint{0.828118in}{1.355636in}}%
\pgfpathclose%
\pgfusepath{stroke,fill}%
\end{pgfscope}%
\begin{pgfscope}%
\pgfpathrectangle{\pgfqpoint{0.100000in}{0.220728in}}{\pgfqpoint{3.696000in}{3.696000in}}%
\pgfusepath{clip}%
\pgfsetbuttcap%
\pgfsetroundjoin%
\definecolor{currentfill}{rgb}{0.121569,0.466667,0.705882}%
\pgfsetfillcolor{currentfill}%
\pgfsetfillopacity{0.700334}%
\pgfsetlinewidth{1.003750pt}%
\definecolor{currentstroke}{rgb}{0.121569,0.466667,0.705882}%
\pgfsetstrokecolor{currentstroke}%
\pgfsetstrokeopacity{0.700334}%
\pgfsetdash{}{0pt}%
\pgfpathmoveto{\pgfqpoint{3.305391in}{2.758513in}}%
\pgfpathcurveto{\pgfqpoint{3.313627in}{2.758513in}}{\pgfqpoint{3.321527in}{2.761785in}}{\pgfqpoint{3.327351in}{2.767609in}}%
\pgfpathcurveto{\pgfqpoint{3.333175in}{2.773433in}}{\pgfqpoint{3.336447in}{2.781333in}}{\pgfqpoint{3.336447in}{2.789569in}}%
\pgfpathcurveto{\pgfqpoint{3.336447in}{2.797805in}}{\pgfqpoint{3.333175in}{2.805705in}}{\pgfqpoint{3.327351in}{2.811529in}}%
\pgfpathcurveto{\pgfqpoint{3.321527in}{2.817353in}}{\pgfqpoint{3.313627in}{2.820626in}}{\pgfqpoint{3.305391in}{2.820626in}}%
\pgfpathcurveto{\pgfqpoint{3.297154in}{2.820626in}}{\pgfqpoint{3.289254in}{2.817353in}}{\pgfqpoint{3.283430in}{2.811529in}}%
\pgfpathcurveto{\pgfqpoint{3.277607in}{2.805705in}}{\pgfqpoint{3.274334in}{2.797805in}}{\pgfqpoint{3.274334in}{2.789569in}}%
\pgfpathcurveto{\pgfqpoint{3.274334in}{2.781333in}}{\pgfqpoint{3.277607in}{2.773433in}}{\pgfqpoint{3.283430in}{2.767609in}}%
\pgfpathcurveto{\pgfqpoint{3.289254in}{2.761785in}}{\pgfqpoint{3.297154in}{2.758513in}}{\pgfqpoint{3.305391in}{2.758513in}}%
\pgfpathclose%
\pgfusepath{stroke,fill}%
\end{pgfscope}%
\begin{pgfscope}%
\pgfpathrectangle{\pgfqpoint{0.100000in}{0.220728in}}{\pgfqpoint{3.696000in}{3.696000in}}%
\pgfusepath{clip}%
\pgfsetbuttcap%
\pgfsetroundjoin%
\definecolor{currentfill}{rgb}{0.121569,0.466667,0.705882}%
\pgfsetfillcolor{currentfill}%
\pgfsetfillopacity{0.701064}%
\pgfsetlinewidth{1.003750pt}%
\definecolor{currentstroke}{rgb}{0.121569,0.466667,0.705882}%
\pgfsetstrokecolor{currentstroke}%
\pgfsetstrokeopacity{0.701064}%
\pgfsetdash{}{0pt}%
\pgfpathmoveto{\pgfqpoint{3.302593in}{2.754070in}}%
\pgfpathcurveto{\pgfqpoint{3.310829in}{2.754070in}}{\pgfqpoint{3.318729in}{2.757342in}}{\pgfqpoint{3.324553in}{2.763166in}}%
\pgfpathcurveto{\pgfqpoint{3.330377in}{2.768990in}}{\pgfqpoint{3.333650in}{2.776890in}}{\pgfqpoint{3.333650in}{2.785126in}}%
\pgfpathcurveto{\pgfqpoint{3.333650in}{2.793363in}}{\pgfqpoint{3.330377in}{2.801263in}}{\pgfqpoint{3.324553in}{2.807087in}}%
\pgfpathcurveto{\pgfqpoint{3.318729in}{2.812910in}}{\pgfqpoint{3.310829in}{2.816183in}}{\pgfqpoint{3.302593in}{2.816183in}}%
\pgfpathcurveto{\pgfqpoint{3.294357in}{2.816183in}}{\pgfqpoint{3.286457in}{2.812910in}}{\pgfqpoint{3.280633in}{2.807087in}}%
\pgfpathcurveto{\pgfqpoint{3.274809in}{2.801263in}}{\pgfqpoint{3.271537in}{2.793363in}}{\pgfqpoint{3.271537in}{2.785126in}}%
\pgfpathcurveto{\pgfqpoint{3.271537in}{2.776890in}}{\pgfqpoint{3.274809in}{2.768990in}}{\pgfqpoint{3.280633in}{2.763166in}}%
\pgfpathcurveto{\pgfqpoint{3.286457in}{2.757342in}}{\pgfqpoint{3.294357in}{2.754070in}}{\pgfqpoint{3.302593in}{2.754070in}}%
\pgfpathclose%
\pgfusepath{stroke,fill}%
\end{pgfscope}%
\begin{pgfscope}%
\pgfpathrectangle{\pgfqpoint{0.100000in}{0.220728in}}{\pgfqpoint{3.696000in}{3.696000in}}%
\pgfusepath{clip}%
\pgfsetbuttcap%
\pgfsetroundjoin%
\definecolor{currentfill}{rgb}{0.121569,0.466667,0.705882}%
\pgfsetfillcolor{currentfill}%
\pgfsetfillopacity{0.701462}%
\pgfsetlinewidth{1.003750pt}%
\definecolor{currentstroke}{rgb}{0.121569,0.466667,0.705882}%
\pgfsetstrokecolor{currentstroke}%
\pgfsetstrokeopacity{0.701462}%
\pgfsetdash{}{0pt}%
\pgfpathmoveto{\pgfqpoint{3.300995in}{2.751710in}}%
\pgfpathcurveto{\pgfqpoint{3.309231in}{2.751710in}}{\pgfqpoint{3.317131in}{2.754983in}}{\pgfqpoint{3.322955in}{2.760806in}}%
\pgfpathcurveto{\pgfqpoint{3.328779in}{2.766630in}}{\pgfqpoint{3.332052in}{2.774530in}}{\pgfqpoint{3.332052in}{2.782767in}}%
\pgfpathcurveto{\pgfqpoint{3.332052in}{2.791003in}}{\pgfqpoint{3.328779in}{2.798903in}}{\pgfqpoint{3.322955in}{2.804727in}}%
\pgfpathcurveto{\pgfqpoint{3.317131in}{2.810551in}}{\pgfqpoint{3.309231in}{2.813823in}}{\pgfqpoint{3.300995in}{2.813823in}}%
\pgfpathcurveto{\pgfqpoint{3.292759in}{2.813823in}}{\pgfqpoint{3.284859in}{2.810551in}}{\pgfqpoint{3.279035in}{2.804727in}}%
\pgfpathcurveto{\pgfqpoint{3.273211in}{2.798903in}}{\pgfqpoint{3.269939in}{2.791003in}}{\pgfqpoint{3.269939in}{2.782767in}}%
\pgfpathcurveto{\pgfqpoint{3.269939in}{2.774530in}}{\pgfqpoint{3.273211in}{2.766630in}}{\pgfqpoint{3.279035in}{2.760806in}}%
\pgfpathcurveto{\pgfqpoint{3.284859in}{2.754983in}}{\pgfqpoint{3.292759in}{2.751710in}}{\pgfqpoint{3.300995in}{2.751710in}}%
\pgfpathclose%
\pgfusepath{stroke,fill}%
\end{pgfscope}%
\begin{pgfscope}%
\pgfpathrectangle{\pgfqpoint{0.100000in}{0.220728in}}{\pgfqpoint{3.696000in}{3.696000in}}%
\pgfusepath{clip}%
\pgfsetbuttcap%
\pgfsetroundjoin%
\definecolor{currentfill}{rgb}{0.121569,0.466667,0.705882}%
\pgfsetfillcolor{currentfill}%
\pgfsetfillopacity{0.701710}%
\pgfsetlinewidth{1.003750pt}%
\definecolor{currentstroke}{rgb}{0.121569,0.466667,0.705882}%
\pgfsetstrokecolor{currentstroke}%
\pgfsetstrokeopacity{0.701710}%
\pgfsetdash{}{0pt}%
\pgfpathmoveto{\pgfqpoint{3.300395in}{2.750168in}}%
\pgfpathcurveto{\pgfqpoint{3.308632in}{2.750168in}}{\pgfqpoint{3.316532in}{2.753440in}}{\pgfqpoint{3.322356in}{2.759264in}}%
\pgfpathcurveto{\pgfqpoint{3.328180in}{2.765088in}}{\pgfqpoint{3.331452in}{2.772988in}}{\pgfqpoint{3.331452in}{2.781224in}}%
\pgfpathcurveto{\pgfqpoint{3.331452in}{2.789461in}}{\pgfqpoint{3.328180in}{2.797361in}}{\pgfqpoint{3.322356in}{2.803185in}}%
\pgfpathcurveto{\pgfqpoint{3.316532in}{2.809009in}}{\pgfqpoint{3.308632in}{2.812281in}}{\pgfqpoint{3.300395in}{2.812281in}}%
\pgfpathcurveto{\pgfqpoint{3.292159in}{2.812281in}}{\pgfqpoint{3.284259in}{2.809009in}}{\pgfqpoint{3.278435in}{2.803185in}}%
\pgfpathcurveto{\pgfqpoint{3.272611in}{2.797361in}}{\pgfqpoint{3.269339in}{2.789461in}}{\pgfqpoint{3.269339in}{2.781224in}}%
\pgfpathcurveto{\pgfqpoint{3.269339in}{2.772988in}}{\pgfqpoint{3.272611in}{2.765088in}}{\pgfqpoint{3.278435in}{2.759264in}}%
\pgfpathcurveto{\pgfqpoint{3.284259in}{2.753440in}}{\pgfqpoint{3.292159in}{2.750168in}}{\pgfqpoint{3.300395in}{2.750168in}}%
\pgfpathclose%
\pgfusepath{stroke,fill}%
\end{pgfscope}%
\begin{pgfscope}%
\pgfpathrectangle{\pgfqpoint{0.100000in}{0.220728in}}{\pgfqpoint{3.696000in}{3.696000in}}%
\pgfusepath{clip}%
\pgfsetbuttcap%
\pgfsetroundjoin%
\definecolor{currentfill}{rgb}{0.121569,0.466667,0.705882}%
\pgfsetfillcolor{currentfill}%
\pgfsetfillopacity{0.701979}%
\pgfsetlinewidth{1.003750pt}%
\definecolor{currentstroke}{rgb}{0.121569,0.466667,0.705882}%
\pgfsetstrokecolor{currentstroke}%
\pgfsetstrokeopacity{0.701979}%
\pgfsetdash{}{0pt}%
\pgfpathmoveto{\pgfqpoint{0.837952in}{1.350904in}}%
\pgfpathcurveto{\pgfqpoint{0.846188in}{1.350904in}}{\pgfqpoint{0.854088in}{1.354176in}}{\pgfqpoint{0.859912in}{1.360000in}}%
\pgfpathcurveto{\pgfqpoint{0.865736in}{1.365824in}}{\pgfqpoint{0.869008in}{1.373724in}}{\pgfqpoint{0.869008in}{1.381961in}}%
\pgfpathcurveto{\pgfqpoint{0.869008in}{1.390197in}}{\pgfqpoint{0.865736in}{1.398097in}}{\pgfqpoint{0.859912in}{1.403921in}}%
\pgfpathcurveto{\pgfqpoint{0.854088in}{1.409745in}}{\pgfqpoint{0.846188in}{1.413017in}}{\pgfqpoint{0.837952in}{1.413017in}}%
\pgfpathcurveto{\pgfqpoint{0.829716in}{1.413017in}}{\pgfqpoint{0.821816in}{1.409745in}}{\pgfqpoint{0.815992in}{1.403921in}}%
\pgfpathcurveto{\pgfqpoint{0.810168in}{1.398097in}}{\pgfqpoint{0.806895in}{1.390197in}}{\pgfqpoint{0.806895in}{1.381961in}}%
\pgfpathcurveto{\pgfqpoint{0.806895in}{1.373724in}}{\pgfqpoint{0.810168in}{1.365824in}}{\pgfqpoint{0.815992in}{1.360000in}}%
\pgfpathcurveto{\pgfqpoint{0.821816in}{1.354176in}}{\pgfqpoint{0.829716in}{1.350904in}}{\pgfqpoint{0.837952in}{1.350904in}}%
\pgfpathclose%
\pgfusepath{stroke,fill}%
\end{pgfscope}%
\begin{pgfscope}%
\pgfpathrectangle{\pgfqpoint{0.100000in}{0.220728in}}{\pgfqpoint{3.696000in}{3.696000in}}%
\pgfusepath{clip}%
\pgfsetbuttcap%
\pgfsetroundjoin%
\definecolor{currentfill}{rgb}{0.121569,0.466667,0.705882}%
\pgfsetfillcolor{currentfill}%
\pgfsetfillopacity{0.702196}%
\pgfsetlinewidth{1.003750pt}%
\definecolor{currentstroke}{rgb}{0.121569,0.466667,0.705882}%
\pgfsetstrokecolor{currentstroke}%
\pgfsetstrokeopacity{0.702196}%
\pgfsetdash{}{0pt}%
\pgfpathmoveto{\pgfqpoint{3.298414in}{2.747122in}}%
\pgfpathcurveto{\pgfqpoint{3.306650in}{2.747122in}}{\pgfqpoint{3.314550in}{2.750394in}}{\pgfqpoint{3.320374in}{2.756218in}}%
\pgfpathcurveto{\pgfqpoint{3.326198in}{2.762042in}}{\pgfqpoint{3.329470in}{2.769942in}}{\pgfqpoint{3.329470in}{2.778179in}}%
\pgfpathcurveto{\pgfqpoint{3.329470in}{2.786415in}}{\pgfqpoint{3.326198in}{2.794315in}}{\pgfqpoint{3.320374in}{2.800139in}}%
\pgfpathcurveto{\pgfqpoint{3.314550in}{2.805963in}}{\pgfqpoint{3.306650in}{2.809235in}}{\pgfqpoint{3.298414in}{2.809235in}}%
\pgfpathcurveto{\pgfqpoint{3.290177in}{2.809235in}}{\pgfqpoint{3.282277in}{2.805963in}}{\pgfqpoint{3.276454in}{2.800139in}}%
\pgfpathcurveto{\pgfqpoint{3.270630in}{2.794315in}}{\pgfqpoint{3.267357in}{2.786415in}}{\pgfqpoint{3.267357in}{2.778179in}}%
\pgfpathcurveto{\pgfqpoint{3.267357in}{2.769942in}}{\pgfqpoint{3.270630in}{2.762042in}}{\pgfqpoint{3.276454in}{2.756218in}}%
\pgfpathcurveto{\pgfqpoint{3.282277in}{2.750394in}}{\pgfqpoint{3.290177in}{2.747122in}}{\pgfqpoint{3.298414in}{2.747122in}}%
\pgfpathclose%
\pgfusepath{stroke,fill}%
\end{pgfscope}%
\begin{pgfscope}%
\pgfpathrectangle{\pgfqpoint{0.100000in}{0.220728in}}{\pgfqpoint{3.696000in}{3.696000in}}%
\pgfusepath{clip}%
\pgfsetbuttcap%
\pgfsetroundjoin%
\definecolor{currentfill}{rgb}{0.121569,0.466667,0.705882}%
\pgfsetfillcolor{currentfill}%
\pgfsetfillopacity{0.702485}%
\pgfsetlinewidth{1.003750pt}%
\definecolor{currentstroke}{rgb}{0.121569,0.466667,0.705882}%
\pgfsetstrokecolor{currentstroke}%
\pgfsetstrokeopacity{0.702485}%
\pgfsetdash{}{0pt}%
\pgfpathmoveto{\pgfqpoint{3.297491in}{2.745292in}}%
\pgfpathcurveto{\pgfqpoint{3.305727in}{2.745292in}}{\pgfqpoint{3.313627in}{2.748565in}}{\pgfqpoint{3.319451in}{2.754388in}}%
\pgfpathcurveto{\pgfqpoint{3.325275in}{2.760212in}}{\pgfqpoint{3.328547in}{2.768112in}}{\pgfqpoint{3.328547in}{2.776349in}}%
\pgfpathcurveto{\pgfqpoint{3.328547in}{2.784585in}}{\pgfqpoint{3.325275in}{2.792485in}}{\pgfqpoint{3.319451in}{2.798309in}}%
\pgfpathcurveto{\pgfqpoint{3.313627in}{2.804133in}}{\pgfqpoint{3.305727in}{2.807405in}}{\pgfqpoint{3.297491in}{2.807405in}}%
\pgfpathcurveto{\pgfqpoint{3.289255in}{2.807405in}}{\pgfqpoint{3.281354in}{2.804133in}}{\pgfqpoint{3.275531in}{2.798309in}}%
\pgfpathcurveto{\pgfqpoint{3.269707in}{2.792485in}}{\pgfqpoint{3.266434in}{2.784585in}}{\pgfqpoint{3.266434in}{2.776349in}}%
\pgfpathcurveto{\pgfqpoint{3.266434in}{2.768112in}}{\pgfqpoint{3.269707in}{2.760212in}}{\pgfqpoint{3.275531in}{2.754388in}}%
\pgfpathcurveto{\pgfqpoint{3.281354in}{2.748565in}}{\pgfqpoint{3.289255in}{2.745292in}}{\pgfqpoint{3.297491in}{2.745292in}}%
\pgfpathclose%
\pgfusepath{stroke,fill}%
\end{pgfscope}%
\begin{pgfscope}%
\pgfpathrectangle{\pgfqpoint{0.100000in}{0.220728in}}{\pgfqpoint{3.696000in}{3.696000in}}%
\pgfusepath{clip}%
\pgfsetbuttcap%
\pgfsetroundjoin%
\definecolor{currentfill}{rgb}{0.121569,0.466667,0.705882}%
\pgfsetfillcolor{currentfill}%
\pgfsetfillopacity{0.702680}%
\pgfsetlinewidth{1.003750pt}%
\definecolor{currentstroke}{rgb}{0.121569,0.466667,0.705882}%
\pgfsetstrokecolor{currentstroke}%
\pgfsetstrokeopacity{0.702680}%
\pgfsetdash{}{0pt}%
\pgfpathmoveto{\pgfqpoint{3.297064in}{2.744348in}}%
\pgfpathcurveto{\pgfqpoint{3.305300in}{2.744348in}}{\pgfqpoint{3.313200in}{2.747620in}}{\pgfqpoint{3.319024in}{2.753444in}}%
\pgfpathcurveto{\pgfqpoint{3.324848in}{2.759268in}}{\pgfqpoint{3.328120in}{2.767168in}}{\pgfqpoint{3.328120in}{2.775405in}}%
\pgfpathcurveto{\pgfqpoint{3.328120in}{2.783641in}}{\pgfqpoint{3.324848in}{2.791541in}}{\pgfqpoint{3.319024in}{2.797365in}}%
\pgfpathcurveto{\pgfqpoint{3.313200in}{2.803189in}}{\pgfqpoint{3.305300in}{2.806461in}}{\pgfqpoint{3.297064in}{2.806461in}}%
\pgfpathcurveto{\pgfqpoint{3.288827in}{2.806461in}}{\pgfqpoint{3.280927in}{2.803189in}}{\pgfqpoint{3.275103in}{2.797365in}}%
\pgfpathcurveto{\pgfqpoint{3.269279in}{2.791541in}}{\pgfqpoint{3.266007in}{2.783641in}}{\pgfqpoint{3.266007in}{2.775405in}}%
\pgfpathcurveto{\pgfqpoint{3.266007in}{2.767168in}}{\pgfqpoint{3.269279in}{2.759268in}}{\pgfqpoint{3.275103in}{2.753444in}}%
\pgfpathcurveto{\pgfqpoint{3.280927in}{2.747620in}}{\pgfqpoint{3.288827in}{2.744348in}}{\pgfqpoint{3.297064in}{2.744348in}}%
\pgfpathclose%
\pgfusepath{stroke,fill}%
\end{pgfscope}%
\begin{pgfscope}%
\pgfpathrectangle{\pgfqpoint{0.100000in}{0.220728in}}{\pgfqpoint{3.696000in}{3.696000in}}%
\pgfusepath{clip}%
\pgfsetbuttcap%
\pgfsetroundjoin%
\definecolor{currentfill}{rgb}{0.121569,0.466667,0.705882}%
\pgfsetfillcolor{currentfill}%
\pgfsetfillopacity{0.702760}%
\pgfsetlinewidth{1.003750pt}%
\definecolor{currentstroke}{rgb}{0.121569,0.466667,0.705882}%
\pgfsetstrokecolor{currentstroke}%
\pgfsetstrokeopacity{0.702760}%
\pgfsetdash{}{0pt}%
\pgfpathmoveto{\pgfqpoint{3.296726in}{2.743849in}}%
\pgfpathcurveto{\pgfqpoint{3.304962in}{2.743849in}}{\pgfqpoint{3.312862in}{2.747121in}}{\pgfqpoint{3.318686in}{2.752945in}}%
\pgfpathcurveto{\pgfqpoint{3.324510in}{2.758769in}}{\pgfqpoint{3.327783in}{2.766669in}}{\pgfqpoint{3.327783in}{2.774906in}}%
\pgfpathcurveto{\pgfqpoint{3.327783in}{2.783142in}}{\pgfqpoint{3.324510in}{2.791042in}}{\pgfqpoint{3.318686in}{2.796866in}}%
\pgfpathcurveto{\pgfqpoint{3.312862in}{2.802690in}}{\pgfqpoint{3.304962in}{2.805962in}}{\pgfqpoint{3.296726in}{2.805962in}}%
\pgfpathcurveto{\pgfqpoint{3.288490in}{2.805962in}}{\pgfqpoint{3.280590in}{2.802690in}}{\pgfqpoint{3.274766in}{2.796866in}}%
\pgfpathcurveto{\pgfqpoint{3.268942in}{2.791042in}}{\pgfqpoint{3.265670in}{2.783142in}}{\pgfqpoint{3.265670in}{2.774906in}}%
\pgfpathcurveto{\pgfqpoint{3.265670in}{2.766669in}}{\pgfqpoint{3.268942in}{2.758769in}}{\pgfqpoint{3.274766in}{2.752945in}}%
\pgfpathcurveto{\pgfqpoint{3.280590in}{2.747121in}}{\pgfqpoint{3.288490in}{2.743849in}}{\pgfqpoint{3.296726in}{2.743849in}}%
\pgfpathclose%
\pgfusepath{stroke,fill}%
\end{pgfscope}%
\begin{pgfscope}%
\pgfpathrectangle{\pgfqpoint{0.100000in}{0.220728in}}{\pgfqpoint{3.696000in}{3.696000in}}%
\pgfusepath{clip}%
\pgfsetbuttcap%
\pgfsetroundjoin%
\definecolor{currentfill}{rgb}{0.121569,0.466667,0.705882}%
\pgfsetfillcolor{currentfill}%
\pgfsetfillopacity{0.703303}%
\pgfsetlinewidth{1.003750pt}%
\definecolor{currentstroke}{rgb}{0.121569,0.466667,0.705882}%
\pgfsetstrokecolor{currentstroke}%
\pgfsetstrokeopacity{0.703303}%
\pgfsetdash{}{0pt}%
\pgfpathmoveto{\pgfqpoint{3.295447in}{2.740333in}}%
\pgfpathcurveto{\pgfqpoint{3.303683in}{2.740333in}}{\pgfqpoint{3.311583in}{2.743605in}}{\pgfqpoint{3.317407in}{2.749429in}}%
\pgfpathcurveto{\pgfqpoint{3.323231in}{2.755253in}}{\pgfqpoint{3.326504in}{2.763153in}}{\pgfqpoint{3.326504in}{2.771389in}}%
\pgfpathcurveto{\pgfqpoint{3.326504in}{2.779626in}}{\pgfqpoint{3.323231in}{2.787526in}}{\pgfqpoint{3.317407in}{2.793350in}}%
\pgfpathcurveto{\pgfqpoint{3.311583in}{2.799174in}}{\pgfqpoint{3.303683in}{2.802446in}}{\pgfqpoint{3.295447in}{2.802446in}}%
\pgfpathcurveto{\pgfqpoint{3.287211in}{2.802446in}}{\pgfqpoint{3.279311in}{2.799174in}}{\pgfqpoint{3.273487in}{2.793350in}}%
\pgfpathcurveto{\pgfqpoint{3.267663in}{2.787526in}}{\pgfqpoint{3.264391in}{2.779626in}}{\pgfqpoint{3.264391in}{2.771389in}}%
\pgfpathcurveto{\pgfqpoint{3.264391in}{2.763153in}}{\pgfqpoint{3.267663in}{2.755253in}}{\pgfqpoint{3.273487in}{2.749429in}}%
\pgfpathcurveto{\pgfqpoint{3.279311in}{2.743605in}}{\pgfqpoint{3.287211in}{2.740333in}}{\pgfqpoint{3.295447in}{2.740333in}}%
\pgfpathclose%
\pgfusepath{stroke,fill}%
\end{pgfscope}%
\begin{pgfscope}%
\pgfpathrectangle{\pgfqpoint{0.100000in}{0.220728in}}{\pgfqpoint{3.696000in}{3.696000in}}%
\pgfusepath{clip}%
\pgfsetbuttcap%
\pgfsetroundjoin%
\definecolor{currentfill}{rgb}{0.121569,0.466667,0.705882}%
\pgfsetfillcolor{currentfill}%
\pgfsetfillopacity{0.703615}%
\pgfsetlinewidth{1.003750pt}%
\definecolor{currentstroke}{rgb}{0.121569,0.466667,0.705882}%
\pgfsetstrokecolor{currentstroke}%
\pgfsetstrokeopacity{0.703615}%
\pgfsetdash{}{0pt}%
\pgfpathmoveto{\pgfqpoint{3.294480in}{2.738768in}}%
\pgfpathcurveto{\pgfqpoint{3.302716in}{2.738768in}}{\pgfqpoint{3.310616in}{2.742041in}}{\pgfqpoint{3.316440in}{2.747864in}}%
\pgfpathcurveto{\pgfqpoint{3.322264in}{2.753688in}}{\pgfqpoint{3.325536in}{2.761588in}}{\pgfqpoint{3.325536in}{2.769825in}}%
\pgfpathcurveto{\pgfqpoint{3.325536in}{2.778061in}}{\pgfqpoint{3.322264in}{2.785961in}}{\pgfqpoint{3.316440in}{2.791785in}}%
\pgfpathcurveto{\pgfqpoint{3.310616in}{2.797609in}}{\pgfqpoint{3.302716in}{2.800881in}}{\pgfqpoint{3.294480in}{2.800881in}}%
\pgfpathcurveto{\pgfqpoint{3.286243in}{2.800881in}}{\pgfqpoint{3.278343in}{2.797609in}}{\pgfqpoint{3.272519in}{2.791785in}}%
\pgfpathcurveto{\pgfqpoint{3.266695in}{2.785961in}}{\pgfqpoint{3.263423in}{2.778061in}}{\pgfqpoint{3.263423in}{2.769825in}}%
\pgfpathcurveto{\pgfqpoint{3.263423in}{2.761588in}}{\pgfqpoint{3.266695in}{2.753688in}}{\pgfqpoint{3.272519in}{2.747864in}}%
\pgfpathcurveto{\pgfqpoint{3.278343in}{2.742041in}}{\pgfqpoint{3.286243in}{2.738768in}}{\pgfqpoint{3.294480in}{2.738768in}}%
\pgfpathclose%
\pgfusepath{stroke,fill}%
\end{pgfscope}%
\begin{pgfscope}%
\pgfpathrectangle{\pgfqpoint{0.100000in}{0.220728in}}{\pgfqpoint{3.696000in}{3.696000in}}%
\pgfusepath{clip}%
\pgfsetbuttcap%
\pgfsetroundjoin%
\definecolor{currentfill}{rgb}{0.121569,0.466667,0.705882}%
\pgfsetfillcolor{currentfill}%
\pgfsetfillopacity{0.703723}%
\pgfsetlinewidth{1.003750pt}%
\definecolor{currentstroke}{rgb}{0.121569,0.466667,0.705882}%
\pgfsetstrokecolor{currentstroke}%
\pgfsetstrokeopacity{0.703723}%
\pgfsetdash{}{0pt}%
\pgfpathmoveto{\pgfqpoint{3.293835in}{2.737822in}}%
\pgfpathcurveto{\pgfqpoint{3.302071in}{2.737822in}}{\pgfqpoint{3.309971in}{2.741094in}}{\pgfqpoint{3.315795in}{2.746918in}}%
\pgfpathcurveto{\pgfqpoint{3.321619in}{2.752742in}}{\pgfqpoint{3.324891in}{2.760642in}}{\pgfqpoint{3.324891in}{2.768878in}}%
\pgfpathcurveto{\pgfqpoint{3.324891in}{2.777115in}}{\pgfqpoint{3.321619in}{2.785015in}}{\pgfqpoint{3.315795in}{2.790839in}}%
\pgfpathcurveto{\pgfqpoint{3.309971in}{2.796662in}}{\pgfqpoint{3.302071in}{2.799935in}}{\pgfqpoint{3.293835in}{2.799935in}}%
\pgfpathcurveto{\pgfqpoint{3.285598in}{2.799935in}}{\pgfqpoint{3.277698in}{2.796662in}}{\pgfqpoint{3.271874in}{2.790839in}}%
\pgfpathcurveto{\pgfqpoint{3.266051in}{2.785015in}}{\pgfqpoint{3.262778in}{2.777115in}}{\pgfqpoint{3.262778in}{2.768878in}}%
\pgfpathcurveto{\pgfqpoint{3.262778in}{2.760642in}}{\pgfqpoint{3.266051in}{2.752742in}}{\pgfqpoint{3.271874in}{2.746918in}}%
\pgfpathcurveto{\pgfqpoint{3.277698in}{2.741094in}}{\pgfqpoint{3.285598in}{2.737822in}}{\pgfqpoint{3.293835in}{2.737822in}}%
\pgfpathclose%
\pgfusepath{stroke,fill}%
\end{pgfscope}%
\begin{pgfscope}%
\pgfpathrectangle{\pgfqpoint{0.100000in}{0.220728in}}{\pgfqpoint{3.696000in}{3.696000in}}%
\pgfusepath{clip}%
\pgfsetbuttcap%
\pgfsetroundjoin%
\definecolor{currentfill}{rgb}{0.121569,0.466667,0.705882}%
\pgfsetfillcolor{currentfill}%
\pgfsetfillopacity{0.703814}%
\pgfsetlinewidth{1.003750pt}%
\definecolor{currentstroke}{rgb}{0.121569,0.466667,0.705882}%
\pgfsetstrokecolor{currentstroke}%
\pgfsetstrokeopacity{0.703814}%
\pgfsetdash{}{0pt}%
\pgfpathmoveto{\pgfqpoint{3.293622in}{2.737238in}}%
\pgfpathcurveto{\pgfqpoint{3.301859in}{2.737238in}}{\pgfqpoint{3.309759in}{2.740510in}}{\pgfqpoint{3.315583in}{2.746334in}}%
\pgfpathcurveto{\pgfqpoint{3.321407in}{2.752158in}}{\pgfqpoint{3.324679in}{2.760058in}}{\pgfqpoint{3.324679in}{2.768294in}}%
\pgfpathcurveto{\pgfqpoint{3.324679in}{2.776531in}}{\pgfqpoint{3.321407in}{2.784431in}}{\pgfqpoint{3.315583in}{2.790255in}}%
\pgfpathcurveto{\pgfqpoint{3.309759in}{2.796079in}}{\pgfqpoint{3.301859in}{2.799351in}}{\pgfqpoint{3.293622in}{2.799351in}}%
\pgfpathcurveto{\pgfqpoint{3.285386in}{2.799351in}}{\pgfqpoint{3.277486in}{2.796079in}}{\pgfqpoint{3.271662in}{2.790255in}}%
\pgfpathcurveto{\pgfqpoint{3.265838in}{2.784431in}}{\pgfqpoint{3.262566in}{2.776531in}}{\pgfqpoint{3.262566in}{2.768294in}}%
\pgfpathcurveto{\pgfqpoint{3.262566in}{2.760058in}}{\pgfqpoint{3.265838in}{2.752158in}}{\pgfqpoint{3.271662in}{2.746334in}}%
\pgfpathcurveto{\pgfqpoint{3.277486in}{2.740510in}}{\pgfqpoint{3.285386in}{2.737238in}}{\pgfqpoint{3.293622in}{2.737238in}}%
\pgfpathclose%
\pgfusepath{stroke,fill}%
\end{pgfscope}%
\begin{pgfscope}%
\pgfpathrectangle{\pgfqpoint{0.100000in}{0.220728in}}{\pgfqpoint{3.696000in}{3.696000in}}%
\pgfusepath{clip}%
\pgfsetbuttcap%
\pgfsetroundjoin%
\definecolor{currentfill}{rgb}{0.121569,0.466667,0.705882}%
\pgfsetfillcolor{currentfill}%
\pgfsetfillopacity{0.704005}%
\pgfsetlinewidth{1.003750pt}%
\definecolor{currentstroke}{rgb}{0.121569,0.466667,0.705882}%
\pgfsetstrokecolor{currentstroke}%
\pgfsetstrokeopacity{0.704005}%
\pgfsetdash{}{0pt}%
\pgfpathmoveto{\pgfqpoint{0.847406in}{1.346977in}}%
\pgfpathcurveto{\pgfqpoint{0.855642in}{1.346977in}}{\pgfqpoint{0.863542in}{1.350250in}}{\pgfqpoint{0.869366in}{1.356074in}}%
\pgfpathcurveto{\pgfqpoint{0.875190in}{1.361897in}}{\pgfqpoint{0.878462in}{1.369798in}}{\pgfqpoint{0.878462in}{1.378034in}}%
\pgfpathcurveto{\pgfqpoint{0.878462in}{1.386270in}}{\pgfqpoint{0.875190in}{1.394170in}}{\pgfqpoint{0.869366in}{1.399994in}}%
\pgfpathcurveto{\pgfqpoint{0.863542in}{1.405818in}}{\pgfqpoint{0.855642in}{1.409090in}}{\pgfqpoint{0.847406in}{1.409090in}}%
\pgfpathcurveto{\pgfqpoint{0.839169in}{1.409090in}}{\pgfqpoint{0.831269in}{1.405818in}}{\pgfqpoint{0.825445in}{1.399994in}}%
\pgfpathcurveto{\pgfqpoint{0.819622in}{1.394170in}}{\pgfqpoint{0.816349in}{1.386270in}}{\pgfqpoint{0.816349in}{1.378034in}}%
\pgfpathcurveto{\pgfqpoint{0.816349in}{1.369798in}}{\pgfqpoint{0.819622in}{1.361897in}}{\pgfqpoint{0.825445in}{1.356074in}}%
\pgfpathcurveto{\pgfqpoint{0.831269in}{1.350250in}}{\pgfqpoint{0.839169in}{1.346977in}}{\pgfqpoint{0.847406in}{1.346977in}}%
\pgfpathclose%
\pgfusepath{stroke,fill}%
\end{pgfscope}%
\begin{pgfscope}%
\pgfpathrectangle{\pgfqpoint{0.100000in}{0.220728in}}{\pgfqpoint{3.696000in}{3.696000in}}%
\pgfusepath{clip}%
\pgfsetbuttcap%
\pgfsetroundjoin%
\definecolor{currentfill}{rgb}{0.121569,0.466667,0.705882}%
\pgfsetfillcolor{currentfill}%
\pgfsetfillopacity{0.704097}%
\pgfsetlinewidth{1.003750pt}%
\definecolor{currentstroke}{rgb}{0.121569,0.466667,0.705882}%
\pgfsetstrokecolor{currentstroke}%
\pgfsetstrokeopacity{0.704097}%
\pgfsetdash{}{0pt}%
\pgfpathmoveto{\pgfqpoint{3.292011in}{2.735016in}}%
\pgfpathcurveto{\pgfqpoint{3.300247in}{2.735016in}}{\pgfqpoint{3.308147in}{2.738288in}}{\pgfqpoint{3.313971in}{2.744112in}}%
\pgfpathcurveto{\pgfqpoint{3.319795in}{2.749936in}}{\pgfqpoint{3.323067in}{2.757836in}}{\pgfqpoint{3.323067in}{2.766072in}}%
\pgfpathcurveto{\pgfqpoint{3.323067in}{2.774308in}}{\pgfqpoint{3.319795in}{2.782208in}}{\pgfqpoint{3.313971in}{2.788032in}}%
\pgfpathcurveto{\pgfqpoint{3.308147in}{2.793856in}}{\pgfqpoint{3.300247in}{2.797129in}}{\pgfqpoint{3.292011in}{2.797129in}}%
\pgfpathcurveto{\pgfqpoint{3.283775in}{2.797129in}}{\pgfqpoint{3.275875in}{2.793856in}}{\pgfqpoint{3.270051in}{2.788032in}}%
\pgfpathcurveto{\pgfqpoint{3.264227in}{2.782208in}}{\pgfqpoint{3.260954in}{2.774308in}}{\pgfqpoint{3.260954in}{2.766072in}}%
\pgfpathcurveto{\pgfqpoint{3.260954in}{2.757836in}}{\pgfqpoint{3.264227in}{2.749936in}}{\pgfqpoint{3.270051in}{2.744112in}}%
\pgfpathcurveto{\pgfqpoint{3.275875in}{2.738288in}}{\pgfqpoint{3.283775in}{2.735016in}}{\pgfqpoint{3.292011in}{2.735016in}}%
\pgfpathclose%
\pgfusepath{stroke,fill}%
\end{pgfscope}%
\begin{pgfscope}%
\pgfpathrectangle{\pgfqpoint{0.100000in}{0.220728in}}{\pgfqpoint{3.696000in}{3.696000in}}%
\pgfusepath{clip}%
\pgfsetbuttcap%
\pgfsetroundjoin%
\definecolor{currentfill}{rgb}{0.121569,0.466667,0.705882}%
\pgfsetfillcolor{currentfill}%
\pgfsetfillopacity{0.704372}%
\pgfsetlinewidth{1.003750pt}%
\definecolor{currentstroke}{rgb}{0.121569,0.466667,0.705882}%
\pgfsetstrokecolor{currentstroke}%
\pgfsetstrokeopacity{0.704372}%
\pgfsetdash{}{0pt}%
\pgfpathmoveto{\pgfqpoint{3.291462in}{2.733814in}}%
\pgfpathcurveto{\pgfqpoint{3.299698in}{2.733814in}}{\pgfqpoint{3.307598in}{2.737087in}}{\pgfqpoint{3.313422in}{2.742911in}}%
\pgfpathcurveto{\pgfqpoint{3.319246in}{2.748735in}}{\pgfqpoint{3.322518in}{2.756635in}}{\pgfqpoint{3.322518in}{2.764871in}}%
\pgfpathcurveto{\pgfqpoint{3.322518in}{2.773107in}}{\pgfqpoint{3.319246in}{2.781007in}}{\pgfqpoint{3.313422in}{2.786831in}}%
\pgfpathcurveto{\pgfqpoint{3.307598in}{2.792655in}}{\pgfqpoint{3.299698in}{2.795927in}}{\pgfqpoint{3.291462in}{2.795927in}}%
\pgfpathcurveto{\pgfqpoint{3.283225in}{2.795927in}}{\pgfqpoint{3.275325in}{2.792655in}}{\pgfqpoint{3.269501in}{2.786831in}}%
\pgfpathcurveto{\pgfqpoint{3.263677in}{2.781007in}}{\pgfqpoint{3.260405in}{2.773107in}}{\pgfqpoint{3.260405in}{2.764871in}}%
\pgfpathcurveto{\pgfqpoint{3.260405in}{2.756635in}}{\pgfqpoint{3.263677in}{2.748735in}}{\pgfqpoint{3.269501in}{2.742911in}}%
\pgfpathcurveto{\pgfqpoint{3.275325in}{2.737087in}}{\pgfqpoint{3.283225in}{2.733814in}}{\pgfqpoint{3.291462in}{2.733814in}}%
\pgfpathclose%
\pgfusepath{stroke,fill}%
\end{pgfscope}%
\begin{pgfscope}%
\pgfpathrectangle{\pgfqpoint{0.100000in}{0.220728in}}{\pgfqpoint{3.696000in}{3.696000in}}%
\pgfusepath{clip}%
\pgfsetbuttcap%
\pgfsetroundjoin%
\definecolor{currentfill}{rgb}{0.121569,0.466667,0.705882}%
\pgfsetfillcolor{currentfill}%
\pgfsetfillopacity{0.704507}%
\pgfsetlinewidth{1.003750pt}%
\definecolor{currentstroke}{rgb}{0.121569,0.466667,0.705882}%
\pgfsetstrokecolor{currentstroke}%
\pgfsetstrokeopacity{0.704507}%
\pgfsetdash{}{0pt}%
\pgfpathmoveto{\pgfqpoint{3.291134in}{2.733110in}}%
\pgfpathcurveto{\pgfqpoint{3.299370in}{2.733110in}}{\pgfqpoint{3.307270in}{2.736382in}}{\pgfqpoint{3.313094in}{2.742206in}}%
\pgfpathcurveto{\pgfqpoint{3.318918in}{2.748030in}}{\pgfqpoint{3.322191in}{2.755930in}}{\pgfqpoint{3.322191in}{2.764167in}}%
\pgfpathcurveto{\pgfqpoint{3.322191in}{2.772403in}}{\pgfqpoint{3.318918in}{2.780303in}}{\pgfqpoint{3.313094in}{2.786127in}}%
\pgfpathcurveto{\pgfqpoint{3.307270in}{2.791951in}}{\pgfqpoint{3.299370in}{2.795223in}}{\pgfqpoint{3.291134in}{2.795223in}}%
\pgfpathcurveto{\pgfqpoint{3.282898in}{2.795223in}}{\pgfqpoint{3.274998in}{2.791951in}}{\pgfqpoint{3.269174in}{2.786127in}}%
\pgfpathcurveto{\pgfqpoint{3.263350in}{2.780303in}}{\pgfqpoint{3.260078in}{2.772403in}}{\pgfqpoint{3.260078in}{2.764167in}}%
\pgfpathcurveto{\pgfqpoint{3.260078in}{2.755930in}}{\pgfqpoint{3.263350in}{2.748030in}}{\pgfqpoint{3.269174in}{2.742206in}}%
\pgfpathcurveto{\pgfqpoint{3.274998in}{2.736382in}}{\pgfqpoint{3.282898in}{2.733110in}}{\pgfqpoint{3.291134in}{2.733110in}}%
\pgfpathclose%
\pgfusepath{stroke,fill}%
\end{pgfscope}%
\begin{pgfscope}%
\pgfpathrectangle{\pgfqpoint{0.100000in}{0.220728in}}{\pgfqpoint{3.696000in}{3.696000in}}%
\pgfusepath{clip}%
\pgfsetbuttcap%
\pgfsetroundjoin%
\definecolor{currentfill}{rgb}{0.121569,0.466667,0.705882}%
\pgfsetfillcolor{currentfill}%
\pgfsetfillopacity{0.704758}%
\pgfsetlinewidth{1.003750pt}%
\definecolor{currentstroke}{rgb}{0.121569,0.466667,0.705882}%
\pgfsetstrokecolor{currentstroke}%
\pgfsetstrokeopacity{0.704758}%
\pgfsetdash{}{0pt}%
\pgfpathmoveto{\pgfqpoint{3.290123in}{2.731608in}}%
\pgfpathcurveto{\pgfqpoint{3.298359in}{2.731608in}}{\pgfqpoint{3.306259in}{2.734881in}}{\pgfqpoint{3.312083in}{2.740705in}}%
\pgfpathcurveto{\pgfqpoint{3.317907in}{2.746529in}}{\pgfqpoint{3.321180in}{2.754429in}}{\pgfqpoint{3.321180in}{2.762665in}}%
\pgfpathcurveto{\pgfqpoint{3.321180in}{2.770901in}}{\pgfqpoint{3.317907in}{2.778801in}}{\pgfqpoint{3.312083in}{2.784625in}}%
\pgfpathcurveto{\pgfqpoint{3.306259in}{2.790449in}}{\pgfqpoint{3.298359in}{2.793721in}}{\pgfqpoint{3.290123in}{2.793721in}}%
\pgfpathcurveto{\pgfqpoint{3.281887in}{2.793721in}}{\pgfqpoint{3.273987in}{2.790449in}}{\pgfqpoint{3.268163in}{2.784625in}}%
\pgfpathcurveto{\pgfqpoint{3.262339in}{2.778801in}}{\pgfqpoint{3.259067in}{2.770901in}}{\pgfqpoint{3.259067in}{2.762665in}}%
\pgfpathcurveto{\pgfqpoint{3.259067in}{2.754429in}}{\pgfqpoint{3.262339in}{2.746529in}}{\pgfqpoint{3.268163in}{2.740705in}}%
\pgfpathcurveto{\pgfqpoint{3.273987in}{2.734881in}}{\pgfqpoint{3.281887in}{2.731608in}}{\pgfqpoint{3.290123in}{2.731608in}}%
\pgfpathclose%
\pgfusepath{stroke,fill}%
\end{pgfscope}%
\begin{pgfscope}%
\pgfpathrectangle{\pgfqpoint{0.100000in}{0.220728in}}{\pgfqpoint{3.696000in}{3.696000in}}%
\pgfusepath{clip}%
\pgfsetbuttcap%
\pgfsetroundjoin%
\definecolor{currentfill}{rgb}{0.121569,0.466667,0.705882}%
\pgfsetfillcolor{currentfill}%
\pgfsetfillopacity{0.705246}%
\pgfsetlinewidth{1.003750pt}%
\definecolor{currentstroke}{rgb}{0.121569,0.466667,0.705882}%
\pgfsetstrokecolor{currentstroke}%
\pgfsetstrokeopacity{0.705246}%
\pgfsetdash{}{0pt}%
\pgfpathmoveto{\pgfqpoint{3.288512in}{2.729396in}}%
\pgfpathcurveto{\pgfqpoint{3.296749in}{2.729396in}}{\pgfqpoint{3.304649in}{2.732668in}}{\pgfqpoint{3.310473in}{2.738492in}}%
\pgfpathcurveto{\pgfqpoint{3.316297in}{2.744316in}}{\pgfqpoint{3.319569in}{2.752216in}}{\pgfqpoint{3.319569in}{2.760452in}}%
\pgfpathcurveto{\pgfqpoint{3.319569in}{2.768689in}}{\pgfqpoint{3.316297in}{2.776589in}}{\pgfqpoint{3.310473in}{2.782413in}}%
\pgfpathcurveto{\pgfqpoint{3.304649in}{2.788237in}}{\pgfqpoint{3.296749in}{2.791509in}}{\pgfqpoint{3.288512in}{2.791509in}}%
\pgfpathcurveto{\pgfqpoint{3.280276in}{2.791509in}}{\pgfqpoint{3.272376in}{2.788237in}}{\pgfqpoint{3.266552in}{2.782413in}}%
\pgfpathcurveto{\pgfqpoint{3.260728in}{2.776589in}}{\pgfqpoint{3.257456in}{2.768689in}}{\pgfqpoint{3.257456in}{2.760452in}}%
\pgfpathcurveto{\pgfqpoint{3.257456in}{2.752216in}}{\pgfqpoint{3.260728in}{2.744316in}}{\pgfqpoint{3.266552in}{2.738492in}}%
\pgfpathcurveto{\pgfqpoint{3.272376in}{2.732668in}}{\pgfqpoint{3.280276in}{2.729396in}}{\pgfqpoint{3.288512in}{2.729396in}}%
\pgfpathclose%
\pgfusepath{stroke,fill}%
\end{pgfscope}%
\begin{pgfscope}%
\pgfpathrectangle{\pgfqpoint{0.100000in}{0.220728in}}{\pgfqpoint{3.696000in}{3.696000in}}%
\pgfusepath{clip}%
\pgfsetbuttcap%
\pgfsetroundjoin%
\definecolor{currentfill}{rgb}{0.121569,0.466667,0.705882}%
\pgfsetfillcolor{currentfill}%
\pgfsetfillopacity{0.705531}%
\pgfsetlinewidth{1.003750pt}%
\definecolor{currentstroke}{rgb}{0.121569,0.466667,0.705882}%
\pgfsetstrokecolor{currentstroke}%
\pgfsetstrokeopacity{0.705531}%
\pgfsetdash{}{0pt}%
\pgfpathmoveto{\pgfqpoint{3.287984in}{2.727831in}}%
\pgfpathcurveto{\pgfqpoint{3.296220in}{2.727831in}}{\pgfqpoint{3.304120in}{2.731103in}}{\pgfqpoint{3.309944in}{2.736927in}}%
\pgfpathcurveto{\pgfqpoint{3.315768in}{2.742751in}}{\pgfqpoint{3.319040in}{2.750651in}}{\pgfqpoint{3.319040in}{2.758887in}}%
\pgfpathcurveto{\pgfqpoint{3.319040in}{2.767123in}}{\pgfqpoint{3.315768in}{2.775023in}}{\pgfqpoint{3.309944in}{2.780847in}}%
\pgfpathcurveto{\pgfqpoint{3.304120in}{2.786671in}}{\pgfqpoint{3.296220in}{2.789944in}}{\pgfqpoint{3.287984in}{2.789944in}}%
\pgfpathcurveto{\pgfqpoint{3.279747in}{2.789944in}}{\pgfqpoint{3.271847in}{2.786671in}}{\pgfqpoint{3.266023in}{2.780847in}}%
\pgfpathcurveto{\pgfqpoint{3.260200in}{2.775023in}}{\pgfqpoint{3.256927in}{2.767123in}}{\pgfqpoint{3.256927in}{2.758887in}}%
\pgfpathcurveto{\pgfqpoint{3.256927in}{2.750651in}}{\pgfqpoint{3.260200in}{2.742751in}}{\pgfqpoint{3.266023in}{2.736927in}}%
\pgfpathcurveto{\pgfqpoint{3.271847in}{2.731103in}}{\pgfqpoint{3.279747in}{2.727831in}}{\pgfqpoint{3.287984in}{2.727831in}}%
\pgfpathclose%
\pgfusepath{stroke,fill}%
\end{pgfscope}%
\begin{pgfscope}%
\pgfpathrectangle{\pgfqpoint{0.100000in}{0.220728in}}{\pgfqpoint{3.696000in}{3.696000in}}%
\pgfusepath{clip}%
\pgfsetbuttcap%
\pgfsetroundjoin%
\definecolor{currentfill}{rgb}{0.121569,0.466667,0.705882}%
\pgfsetfillcolor{currentfill}%
\pgfsetfillopacity{0.705940}%
\pgfsetlinewidth{1.003750pt}%
\definecolor{currentstroke}{rgb}{0.121569,0.466667,0.705882}%
\pgfsetstrokecolor{currentstroke}%
\pgfsetstrokeopacity{0.705940}%
\pgfsetdash{}{0pt}%
\pgfpathmoveto{\pgfqpoint{3.285857in}{2.724571in}}%
\pgfpathcurveto{\pgfqpoint{3.294094in}{2.724571in}}{\pgfqpoint{3.301994in}{2.727843in}}{\pgfqpoint{3.307817in}{2.733667in}}%
\pgfpathcurveto{\pgfqpoint{3.313641in}{2.739491in}}{\pgfqpoint{3.316914in}{2.747391in}}{\pgfqpoint{3.316914in}{2.755627in}}%
\pgfpathcurveto{\pgfqpoint{3.316914in}{2.763863in}}{\pgfqpoint{3.313641in}{2.771763in}}{\pgfqpoint{3.307817in}{2.777587in}}%
\pgfpathcurveto{\pgfqpoint{3.301994in}{2.783411in}}{\pgfqpoint{3.294094in}{2.786684in}}{\pgfqpoint{3.285857in}{2.786684in}}%
\pgfpathcurveto{\pgfqpoint{3.277621in}{2.786684in}}{\pgfqpoint{3.269721in}{2.783411in}}{\pgfqpoint{3.263897in}{2.777587in}}%
\pgfpathcurveto{\pgfqpoint{3.258073in}{2.771763in}}{\pgfqpoint{3.254801in}{2.763863in}}{\pgfqpoint{3.254801in}{2.755627in}}%
\pgfpathcurveto{\pgfqpoint{3.254801in}{2.747391in}}{\pgfqpoint{3.258073in}{2.739491in}}{\pgfqpoint{3.263897in}{2.733667in}}%
\pgfpathcurveto{\pgfqpoint{3.269721in}{2.727843in}}{\pgfqpoint{3.277621in}{2.724571in}}{\pgfqpoint{3.285857in}{2.724571in}}%
\pgfpathclose%
\pgfusepath{stroke,fill}%
\end{pgfscope}%
\begin{pgfscope}%
\pgfpathrectangle{\pgfqpoint{0.100000in}{0.220728in}}{\pgfqpoint{3.696000in}{3.696000in}}%
\pgfusepath{clip}%
\pgfsetbuttcap%
\pgfsetroundjoin%
\definecolor{currentfill}{rgb}{0.121569,0.466667,0.705882}%
\pgfsetfillcolor{currentfill}%
\pgfsetfillopacity{0.706703}%
\pgfsetlinewidth{1.003750pt}%
\definecolor{currentstroke}{rgb}{0.121569,0.466667,0.705882}%
\pgfsetstrokecolor{currentstroke}%
\pgfsetstrokeopacity{0.706703}%
\pgfsetdash{}{0pt}%
\pgfpathmoveto{\pgfqpoint{3.283906in}{2.720521in}}%
\pgfpathcurveto{\pgfqpoint{3.292142in}{2.720521in}}{\pgfqpoint{3.300042in}{2.723794in}}{\pgfqpoint{3.305866in}{2.729618in}}%
\pgfpathcurveto{\pgfqpoint{3.311690in}{2.735441in}}{\pgfqpoint{3.314962in}{2.743342in}}{\pgfqpoint{3.314962in}{2.751578in}}%
\pgfpathcurveto{\pgfqpoint{3.314962in}{2.759814in}}{\pgfqpoint{3.311690in}{2.767714in}}{\pgfqpoint{3.305866in}{2.773538in}}%
\pgfpathcurveto{\pgfqpoint{3.300042in}{2.779362in}}{\pgfqpoint{3.292142in}{2.782634in}}{\pgfqpoint{3.283906in}{2.782634in}}%
\pgfpathcurveto{\pgfqpoint{3.275670in}{2.782634in}}{\pgfqpoint{3.267770in}{2.779362in}}{\pgfqpoint{3.261946in}{2.773538in}}%
\pgfpathcurveto{\pgfqpoint{3.256122in}{2.767714in}}{\pgfqpoint{3.252849in}{2.759814in}}{\pgfqpoint{3.252849in}{2.751578in}}%
\pgfpathcurveto{\pgfqpoint{3.252849in}{2.743342in}}{\pgfqpoint{3.256122in}{2.735441in}}{\pgfqpoint{3.261946in}{2.729618in}}%
\pgfpathcurveto{\pgfqpoint{3.267770in}{2.723794in}}{\pgfqpoint{3.275670in}{2.720521in}}{\pgfqpoint{3.283906in}{2.720521in}}%
\pgfpathclose%
\pgfusepath{stroke,fill}%
\end{pgfscope}%
\begin{pgfscope}%
\pgfpathrectangle{\pgfqpoint{0.100000in}{0.220728in}}{\pgfqpoint{3.696000in}{3.696000in}}%
\pgfusepath{clip}%
\pgfsetbuttcap%
\pgfsetroundjoin%
\definecolor{currentfill}{rgb}{0.121569,0.466667,0.705882}%
\pgfsetfillcolor{currentfill}%
\pgfsetfillopacity{0.707125}%
\pgfsetlinewidth{1.003750pt}%
\definecolor{currentstroke}{rgb}{0.121569,0.466667,0.705882}%
\pgfsetstrokecolor{currentstroke}%
\pgfsetstrokeopacity{0.707125}%
\pgfsetdash{}{0pt}%
\pgfpathmoveto{\pgfqpoint{3.282855in}{2.718282in}}%
\pgfpathcurveto{\pgfqpoint{3.291091in}{2.718282in}}{\pgfqpoint{3.298991in}{2.721554in}}{\pgfqpoint{3.304815in}{2.727378in}}%
\pgfpathcurveto{\pgfqpoint{3.310639in}{2.733202in}}{\pgfqpoint{3.313911in}{2.741102in}}{\pgfqpoint{3.313911in}{2.749338in}}%
\pgfpathcurveto{\pgfqpoint{3.313911in}{2.757575in}}{\pgfqpoint{3.310639in}{2.765475in}}{\pgfqpoint{3.304815in}{2.771299in}}%
\pgfpathcurveto{\pgfqpoint{3.298991in}{2.777122in}}{\pgfqpoint{3.291091in}{2.780395in}}{\pgfqpoint{3.282855in}{2.780395in}}%
\pgfpathcurveto{\pgfqpoint{3.274619in}{2.780395in}}{\pgfqpoint{3.266719in}{2.777122in}}{\pgfqpoint{3.260895in}{2.771299in}}%
\pgfpathcurveto{\pgfqpoint{3.255071in}{2.765475in}}{\pgfqpoint{3.251798in}{2.757575in}}{\pgfqpoint{3.251798in}{2.749338in}}%
\pgfpathcurveto{\pgfqpoint{3.251798in}{2.741102in}}{\pgfqpoint{3.255071in}{2.733202in}}{\pgfqpoint{3.260895in}{2.727378in}}%
\pgfpathcurveto{\pgfqpoint{3.266719in}{2.721554in}}{\pgfqpoint{3.274619in}{2.718282in}}{\pgfqpoint{3.282855in}{2.718282in}}%
\pgfpathclose%
\pgfusepath{stroke,fill}%
\end{pgfscope}%
\begin{pgfscope}%
\pgfpathrectangle{\pgfqpoint{0.100000in}{0.220728in}}{\pgfqpoint{3.696000in}{3.696000in}}%
\pgfusepath{clip}%
\pgfsetbuttcap%
\pgfsetroundjoin%
\definecolor{currentfill}{rgb}{0.121569,0.466667,0.705882}%
\pgfsetfillcolor{currentfill}%
\pgfsetfillopacity{0.707348}%
\pgfsetlinewidth{1.003750pt}%
\definecolor{currentstroke}{rgb}{0.121569,0.466667,0.705882}%
\pgfsetstrokecolor{currentstroke}%
\pgfsetstrokeopacity{0.707348}%
\pgfsetdash{}{0pt}%
\pgfpathmoveto{\pgfqpoint{3.282065in}{2.717318in}}%
\pgfpathcurveto{\pgfqpoint{3.290301in}{2.717318in}}{\pgfqpoint{3.298201in}{2.720590in}}{\pgfqpoint{3.304025in}{2.726414in}}%
\pgfpathcurveto{\pgfqpoint{3.309849in}{2.732238in}}{\pgfqpoint{3.313121in}{2.740138in}}{\pgfqpoint{3.313121in}{2.748374in}}%
\pgfpathcurveto{\pgfqpoint{3.313121in}{2.756611in}}{\pgfqpoint{3.309849in}{2.764511in}}{\pgfqpoint{3.304025in}{2.770335in}}%
\pgfpathcurveto{\pgfqpoint{3.298201in}{2.776159in}}{\pgfqpoint{3.290301in}{2.779431in}}{\pgfqpoint{3.282065in}{2.779431in}}%
\pgfpathcurveto{\pgfqpoint{3.273828in}{2.779431in}}{\pgfqpoint{3.265928in}{2.776159in}}{\pgfqpoint{3.260104in}{2.770335in}}%
\pgfpathcurveto{\pgfqpoint{3.254280in}{2.764511in}}{\pgfqpoint{3.251008in}{2.756611in}}{\pgfqpoint{3.251008in}{2.748374in}}%
\pgfpathcurveto{\pgfqpoint{3.251008in}{2.740138in}}{\pgfqpoint{3.254280in}{2.732238in}}{\pgfqpoint{3.260104in}{2.726414in}}%
\pgfpathcurveto{\pgfqpoint{3.265928in}{2.720590in}}{\pgfqpoint{3.273828in}{2.717318in}}{\pgfqpoint{3.282065in}{2.717318in}}%
\pgfpathclose%
\pgfusepath{stroke,fill}%
\end{pgfscope}%
\begin{pgfscope}%
\pgfpathrectangle{\pgfqpoint{0.100000in}{0.220728in}}{\pgfqpoint{3.696000in}{3.696000in}}%
\pgfusepath{clip}%
\pgfsetbuttcap%
\pgfsetroundjoin%
\definecolor{currentfill}{rgb}{0.121569,0.466667,0.705882}%
\pgfsetfillcolor{currentfill}%
\pgfsetfillopacity{0.708009}%
\pgfsetlinewidth{1.003750pt}%
\definecolor{currentstroke}{rgb}{0.121569,0.466667,0.705882}%
\pgfsetstrokecolor{currentstroke}%
\pgfsetstrokeopacity{0.708009}%
\pgfsetdash{}{0pt}%
\pgfpathmoveto{\pgfqpoint{3.280414in}{2.712581in}}%
\pgfpathcurveto{\pgfqpoint{3.288650in}{2.712581in}}{\pgfqpoint{3.296550in}{2.715853in}}{\pgfqpoint{3.302374in}{2.721677in}}%
\pgfpathcurveto{\pgfqpoint{3.308198in}{2.727501in}}{\pgfqpoint{3.311471in}{2.735401in}}{\pgfqpoint{3.311471in}{2.743638in}}%
\pgfpathcurveto{\pgfqpoint{3.311471in}{2.751874in}}{\pgfqpoint{3.308198in}{2.759774in}}{\pgfqpoint{3.302374in}{2.765598in}}%
\pgfpathcurveto{\pgfqpoint{3.296550in}{2.771422in}}{\pgfqpoint{3.288650in}{2.774694in}}{\pgfqpoint{3.280414in}{2.774694in}}%
\pgfpathcurveto{\pgfqpoint{3.272178in}{2.774694in}}{\pgfqpoint{3.264278in}{2.771422in}}{\pgfqpoint{3.258454in}{2.765598in}}%
\pgfpathcurveto{\pgfqpoint{3.252630in}{2.759774in}}{\pgfqpoint{3.249358in}{2.751874in}}{\pgfqpoint{3.249358in}{2.743638in}}%
\pgfpathcurveto{\pgfqpoint{3.249358in}{2.735401in}}{\pgfqpoint{3.252630in}{2.727501in}}{\pgfqpoint{3.258454in}{2.721677in}}%
\pgfpathcurveto{\pgfqpoint{3.264278in}{2.715853in}}{\pgfqpoint{3.272178in}{2.712581in}}{\pgfqpoint{3.280414in}{2.712581in}}%
\pgfpathclose%
\pgfusepath{stroke,fill}%
\end{pgfscope}%
\begin{pgfscope}%
\pgfpathrectangle{\pgfqpoint{0.100000in}{0.220728in}}{\pgfqpoint{3.696000in}{3.696000in}}%
\pgfusepath{clip}%
\pgfsetbuttcap%
\pgfsetroundjoin%
\definecolor{currentfill}{rgb}{0.121569,0.466667,0.705882}%
\pgfsetfillcolor{currentfill}%
\pgfsetfillopacity{0.708197}%
\pgfsetlinewidth{1.003750pt}%
\definecolor{currentstroke}{rgb}{0.121569,0.466667,0.705882}%
\pgfsetstrokecolor{currentstroke}%
\pgfsetstrokeopacity{0.708197}%
\pgfsetdash{}{0pt}%
\pgfpathmoveto{\pgfqpoint{0.863920in}{1.339922in}}%
\pgfpathcurveto{\pgfqpoint{0.872156in}{1.339922in}}{\pgfqpoint{0.880056in}{1.343194in}}{\pgfqpoint{0.885880in}{1.349018in}}%
\pgfpathcurveto{\pgfqpoint{0.891704in}{1.354842in}}{\pgfqpoint{0.894976in}{1.362742in}}{\pgfqpoint{0.894976in}{1.370978in}}%
\pgfpathcurveto{\pgfqpoint{0.894976in}{1.379214in}}{\pgfqpoint{0.891704in}{1.387115in}}{\pgfqpoint{0.885880in}{1.392938in}}%
\pgfpathcurveto{\pgfqpoint{0.880056in}{1.398762in}}{\pgfqpoint{0.872156in}{1.402035in}}{\pgfqpoint{0.863920in}{1.402035in}}%
\pgfpathcurveto{\pgfqpoint{0.855683in}{1.402035in}}{\pgfqpoint{0.847783in}{1.398762in}}{\pgfqpoint{0.841959in}{1.392938in}}%
\pgfpathcurveto{\pgfqpoint{0.836135in}{1.387115in}}{\pgfqpoint{0.832863in}{1.379214in}}{\pgfqpoint{0.832863in}{1.370978in}}%
\pgfpathcurveto{\pgfqpoint{0.832863in}{1.362742in}}{\pgfqpoint{0.836135in}{1.354842in}}{\pgfqpoint{0.841959in}{1.349018in}}%
\pgfpathcurveto{\pgfqpoint{0.847783in}{1.343194in}}{\pgfqpoint{0.855683in}{1.339922in}}{\pgfqpoint{0.863920in}{1.339922in}}%
\pgfpathclose%
\pgfusepath{stroke,fill}%
\end{pgfscope}%
\begin{pgfscope}%
\pgfpathrectangle{\pgfqpoint{0.100000in}{0.220728in}}{\pgfqpoint{3.696000in}{3.696000in}}%
\pgfusepath{clip}%
\pgfsetbuttcap%
\pgfsetroundjoin%
\definecolor{currentfill}{rgb}{0.121569,0.466667,0.705882}%
\pgfsetfillcolor{currentfill}%
\pgfsetfillopacity{0.708312}%
\pgfsetlinewidth{1.003750pt}%
\definecolor{currentstroke}{rgb}{0.121569,0.466667,0.705882}%
\pgfsetstrokecolor{currentstroke}%
\pgfsetstrokeopacity{0.708312}%
\pgfsetdash{}{0pt}%
\pgfpathmoveto{\pgfqpoint{3.279021in}{2.710335in}}%
\pgfpathcurveto{\pgfqpoint{3.287257in}{2.710335in}}{\pgfqpoint{3.295157in}{2.713607in}}{\pgfqpoint{3.300981in}{2.719431in}}%
\pgfpathcurveto{\pgfqpoint{3.306805in}{2.725255in}}{\pgfqpoint{3.310077in}{2.733155in}}{\pgfqpoint{3.310077in}{2.741392in}}%
\pgfpathcurveto{\pgfqpoint{3.310077in}{2.749628in}}{\pgfqpoint{3.306805in}{2.757528in}}{\pgfqpoint{3.300981in}{2.763352in}}%
\pgfpathcurveto{\pgfqpoint{3.295157in}{2.769176in}}{\pgfqpoint{3.287257in}{2.772448in}}{\pgfqpoint{3.279021in}{2.772448in}}%
\pgfpathcurveto{\pgfqpoint{3.270784in}{2.772448in}}{\pgfqpoint{3.262884in}{2.769176in}}{\pgfqpoint{3.257060in}{2.763352in}}%
\pgfpathcurveto{\pgfqpoint{3.251237in}{2.757528in}}{\pgfqpoint{3.247964in}{2.749628in}}{\pgfqpoint{3.247964in}{2.741392in}}%
\pgfpathcurveto{\pgfqpoint{3.247964in}{2.733155in}}{\pgfqpoint{3.251237in}{2.725255in}}{\pgfqpoint{3.257060in}{2.719431in}}%
\pgfpathcurveto{\pgfqpoint{3.262884in}{2.713607in}}{\pgfqpoint{3.270784in}{2.710335in}}{\pgfqpoint{3.279021in}{2.710335in}}%
\pgfpathclose%
\pgfusepath{stroke,fill}%
\end{pgfscope}%
\begin{pgfscope}%
\pgfpathrectangle{\pgfqpoint{0.100000in}{0.220728in}}{\pgfqpoint{3.696000in}{3.696000in}}%
\pgfusepath{clip}%
\pgfsetbuttcap%
\pgfsetroundjoin%
\definecolor{currentfill}{rgb}{0.121569,0.466667,0.705882}%
\pgfsetfillcolor{currentfill}%
\pgfsetfillopacity{0.708864}%
\pgfsetlinewidth{1.003750pt}%
\definecolor{currentstroke}{rgb}{0.121569,0.466667,0.705882}%
\pgfsetstrokecolor{currentstroke}%
\pgfsetstrokeopacity{0.708864}%
\pgfsetdash{}{0pt}%
\pgfpathmoveto{\pgfqpoint{3.277535in}{2.707480in}}%
\pgfpathcurveto{\pgfqpoint{3.285771in}{2.707480in}}{\pgfqpoint{3.293671in}{2.710753in}}{\pgfqpoint{3.299495in}{2.716577in}}%
\pgfpathcurveto{\pgfqpoint{3.305319in}{2.722401in}}{\pgfqpoint{3.308591in}{2.730301in}}{\pgfqpoint{3.308591in}{2.738537in}}%
\pgfpathcurveto{\pgfqpoint{3.308591in}{2.746773in}}{\pgfqpoint{3.305319in}{2.754673in}}{\pgfqpoint{3.299495in}{2.760497in}}%
\pgfpathcurveto{\pgfqpoint{3.293671in}{2.766321in}}{\pgfqpoint{3.285771in}{2.769593in}}{\pgfqpoint{3.277535in}{2.769593in}}%
\pgfpathcurveto{\pgfqpoint{3.269299in}{2.769593in}}{\pgfqpoint{3.261399in}{2.766321in}}{\pgfqpoint{3.255575in}{2.760497in}}%
\pgfpathcurveto{\pgfqpoint{3.249751in}{2.754673in}}{\pgfqpoint{3.246478in}{2.746773in}}{\pgfqpoint{3.246478in}{2.738537in}}%
\pgfpathcurveto{\pgfqpoint{3.246478in}{2.730301in}}{\pgfqpoint{3.249751in}{2.722401in}}{\pgfqpoint{3.255575in}{2.716577in}}%
\pgfpathcurveto{\pgfqpoint{3.261399in}{2.710753in}}{\pgfqpoint{3.269299in}{2.707480in}}{\pgfqpoint{3.277535in}{2.707480in}}%
\pgfpathclose%
\pgfusepath{stroke,fill}%
\end{pgfscope}%
\begin{pgfscope}%
\pgfpathrectangle{\pgfqpoint{0.100000in}{0.220728in}}{\pgfqpoint{3.696000in}{3.696000in}}%
\pgfusepath{clip}%
\pgfsetbuttcap%
\pgfsetroundjoin%
\definecolor{currentfill}{rgb}{0.121569,0.466667,0.705882}%
\pgfsetfillcolor{currentfill}%
\pgfsetfillopacity{0.709205}%
\pgfsetlinewidth{1.003750pt}%
\definecolor{currentstroke}{rgb}{0.121569,0.466667,0.705882}%
\pgfsetstrokecolor{currentstroke}%
\pgfsetstrokeopacity{0.709205}%
\pgfsetdash{}{0pt}%
\pgfpathmoveto{\pgfqpoint{3.276919in}{2.705864in}}%
\pgfpathcurveto{\pgfqpoint{3.285155in}{2.705864in}}{\pgfqpoint{3.293056in}{2.709136in}}{\pgfqpoint{3.298879in}{2.714960in}}%
\pgfpathcurveto{\pgfqpoint{3.304703in}{2.720784in}}{\pgfqpoint{3.307976in}{2.728684in}}{\pgfqpoint{3.307976in}{2.736921in}}%
\pgfpathcurveto{\pgfqpoint{3.307976in}{2.745157in}}{\pgfqpoint{3.304703in}{2.753057in}}{\pgfqpoint{3.298879in}{2.758881in}}%
\pgfpathcurveto{\pgfqpoint{3.293056in}{2.764705in}}{\pgfqpoint{3.285155in}{2.767977in}}{\pgfqpoint{3.276919in}{2.767977in}}%
\pgfpathcurveto{\pgfqpoint{3.268683in}{2.767977in}}{\pgfqpoint{3.260783in}{2.764705in}}{\pgfqpoint{3.254959in}{2.758881in}}%
\pgfpathcurveto{\pgfqpoint{3.249135in}{2.753057in}}{\pgfqpoint{3.245863in}{2.745157in}}{\pgfqpoint{3.245863in}{2.736921in}}%
\pgfpathcurveto{\pgfqpoint{3.245863in}{2.728684in}}{\pgfqpoint{3.249135in}{2.720784in}}{\pgfqpoint{3.254959in}{2.714960in}}%
\pgfpathcurveto{\pgfqpoint{3.260783in}{2.709136in}}{\pgfqpoint{3.268683in}{2.705864in}}{\pgfqpoint{3.276919in}{2.705864in}}%
\pgfpathclose%
\pgfusepath{stroke,fill}%
\end{pgfscope}%
\begin{pgfscope}%
\pgfpathrectangle{\pgfqpoint{0.100000in}{0.220728in}}{\pgfqpoint{3.696000in}{3.696000in}}%
\pgfusepath{clip}%
\pgfsetbuttcap%
\pgfsetroundjoin%
\definecolor{currentfill}{rgb}{0.121569,0.466667,0.705882}%
\pgfsetfillcolor{currentfill}%
\pgfsetfillopacity{0.709521}%
\pgfsetlinewidth{1.003750pt}%
\definecolor{currentstroke}{rgb}{0.121569,0.466667,0.705882}%
\pgfsetstrokecolor{currentstroke}%
\pgfsetstrokeopacity{0.709521}%
\pgfsetdash{}{0pt}%
\pgfpathmoveto{\pgfqpoint{3.275375in}{2.703771in}}%
\pgfpathcurveto{\pgfqpoint{3.283612in}{2.703771in}}{\pgfqpoint{3.291512in}{2.707044in}}{\pgfqpoint{3.297336in}{2.712867in}}%
\pgfpathcurveto{\pgfqpoint{3.303160in}{2.718691in}}{\pgfqpoint{3.306432in}{2.726591in}}{\pgfqpoint{3.306432in}{2.734828in}}%
\pgfpathcurveto{\pgfqpoint{3.306432in}{2.743064in}}{\pgfqpoint{3.303160in}{2.750964in}}{\pgfqpoint{3.297336in}{2.756788in}}%
\pgfpathcurveto{\pgfqpoint{3.291512in}{2.762612in}}{\pgfqpoint{3.283612in}{2.765884in}}{\pgfqpoint{3.275375in}{2.765884in}}%
\pgfpathcurveto{\pgfqpoint{3.267139in}{2.765884in}}{\pgfqpoint{3.259239in}{2.762612in}}{\pgfqpoint{3.253415in}{2.756788in}}%
\pgfpathcurveto{\pgfqpoint{3.247591in}{2.750964in}}{\pgfqpoint{3.244319in}{2.743064in}}{\pgfqpoint{3.244319in}{2.734828in}}%
\pgfpathcurveto{\pgfqpoint{3.244319in}{2.726591in}}{\pgfqpoint{3.247591in}{2.718691in}}{\pgfqpoint{3.253415in}{2.712867in}}%
\pgfpathcurveto{\pgfqpoint{3.259239in}{2.707044in}}{\pgfqpoint{3.267139in}{2.703771in}}{\pgfqpoint{3.275375in}{2.703771in}}%
\pgfpathclose%
\pgfusepath{stroke,fill}%
\end{pgfscope}%
\begin{pgfscope}%
\pgfpathrectangle{\pgfqpoint{0.100000in}{0.220728in}}{\pgfqpoint{3.696000in}{3.696000in}}%
\pgfusepath{clip}%
\pgfsetbuttcap%
\pgfsetroundjoin%
\definecolor{currentfill}{rgb}{0.121569,0.466667,0.705882}%
\pgfsetfillcolor{currentfill}%
\pgfsetfillopacity{0.710252}%
\pgfsetlinewidth{1.003750pt}%
\definecolor{currentstroke}{rgb}{0.121569,0.466667,0.705882}%
\pgfsetstrokecolor{currentstroke}%
\pgfsetstrokeopacity{0.710252}%
\pgfsetdash{}{0pt}%
\pgfpathmoveto{\pgfqpoint{3.273864in}{2.699261in}}%
\pgfpathcurveto{\pgfqpoint{3.282101in}{2.699261in}}{\pgfqpoint{3.290001in}{2.702533in}}{\pgfqpoint{3.295825in}{2.708357in}}%
\pgfpathcurveto{\pgfqpoint{3.301649in}{2.714181in}}{\pgfqpoint{3.304921in}{2.722081in}}{\pgfqpoint{3.304921in}{2.730318in}}%
\pgfpathcurveto{\pgfqpoint{3.304921in}{2.738554in}}{\pgfqpoint{3.301649in}{2.746454in}}{\pgfqpoint{3.295825in}{2.752278in}}%
\pgfpathcurveto{\pgfqpoint{3.290001in}{2.758102in}}{\pgfqpoint{3.282101in}{2.761374in}}{\pgfqpoint{3.273864in}{2.761374in}}%
\pgfpathcurveto{\pgfqpoint{3.265628in}{2.761374in}}{\pgfqpoint{3.257728in}{2.758102in}}{\pgfqpoint{3.251904in}{2.752278in}}%
\pgfpathcurveto{\pgfqpoint{3.246080in}{2.746454in}}{\pgfqpoint{3.242808in}{2.738554in}}{\pgfqpoint{3.242808in}{2.730318in}}%
\pgfpathcurveto{\pgfqpoint{3.242808in}{2.722081in}}{\pgfqpoint{3.246080in}{2.714181in}}{\pgfqpoint{3.251904in}{2.708357in}}%
\pgfpathcurveto{\pgfqpoint{3.257728in}{2.702533in}}{\pgfqpoint{3.265628in}{2.699261in}}{\pgfqpoint{3.273864in}{2.699261in}}%
\pgfpathclose%
\pgfusepath{stroke,fill}%
\end{pgfscope}%
\begin{pgfscope}%
\pgfpathrectangle{\pgfqpoint{0.100000in}{0.220728in}}{\pgfqpoint{3.696000in}{3.696000in}}%
\pgfusepath{clip}%
\pgfsetbuttcap%
\pgfsetroundjoin%
\definecolor{currentfill}{rgb}{0.121569,0.466667,0.705882}%
\pgfsetfillcolor{currentfill}%
\pgfsetfillopacity{0.710688}%
\pgfsetlinewidth{1.003750pt}%
\definecolor{currentstroke}{rgb}{0.121569,0.466667,0.705882}%
\pgfsetstrokecolor{currentstroke}%
\pgfsetstrokeopacity{0.710688}%
\pgfsetdash{}{0pt}%
\pgfpathmoveto{\pgfqpoint{3.272711in}{2.697260in}}%
\pgfpathcurveto{\pgfqpoint{3.280947in}{2.697260in}}{\pgfqpoint{3.288847in}{2.700532in}}{\pgfqpoint{3.294671in}{2.706356in}}%
\pgfpathcurveto{\pgfqpoint{3.300495in}{2.712180in}}{\pgfqpoint{3.303767in}{2.720080in}}{\pgfqpoint{3.303767in}{2.728316in}}%
\pgfpathcurveto{\pgfqpoint{3.303767in}{2.736552in}}{\pgfqpoint{3.300495in}{2.744453in}}{\pgfqpoint{3.294671in}{2.750276in}}%
\pgfpathcurveto{\pgfqpoint{3.288847in}{2.756100in}}{\pgfqpoint{3.280947in}{2.759373in}}{\pgfqpoint{3.272711in}{2.759373in}}%
\pgfpathcurveto{\pgfqpoint{3.264474in}{2.759373in}}{\pgfqpoint{3.256574in}{2.756100in}}{\pgfqpoint{3.250750in}{2.750276in}}%
\pgfpathcurveto{\pgfqpoint{3.244926in}{2.744453in}}{\pgfqpoint{3.241654in}{2.736552in}}{\pgfqpoint{3.241654in}{2.728316in}}%
\pgfpathcurveto{\pgfqpoint{3.241654in}{2.720080in}}{\pgfqpoint{3.244926in}{2.712180in}}{\pgfqpoint{3.250750in}{2.706356in}}%
\pgfpathcurveto{\pgfqpoint{3.256574in}{2.700532in}}{\pgfqpoint{3.264474in}{2.697260in}}{\pgfqpoint{3.272711in}{2.697260in}}%
\pgfpathclose%
\pgfusepath{stroke,fill}%
\end{pgfscope}%
\begin{pgfscope}%
\pgfpathrectangle{\pgfqpoint{0.100000in}{0.220728in}}{\pgfqpoint{3.696000in}{3.696000in}}%
\pgfusepath{clip}%
\pgfsetbuttcap%
\pgfsetroundjoin%
\definecolor{currentfill}{rgb}{0.121569,0.466667,0.705882}%
\pgfsetfillcolor{currentfill}%
\pgfsetfillopacity{0.710941}%
\pgfsetlinewidth{1.003750pt}%
\definecolor{currentstroke}{rgb}{0.121569,0.466667,0.705882}%
\pgfsetstrokecolor{currentstroke}%
\pgfsetstrokeopacity{0.710941}%
\pgfsetdash{}{0pt}%
\pgfpathmoveto{\pgfqpoint{3.272004in}{2.696323in}}%
\pgfpathcurveto{\pgfqpoint{3.280240in}{2.696323in}}{\pgfqpoint{3.288140in}{2.699595in}}{\pgfqpoint{3.293964in}{2.705419in}}%
\pgfpathcurveto{\pgfqpoint{3.299788in}{2.711243in}}{\pgfqpoint{3.303060in}{2.719143in}}{\pgfqpoint{3.303060in}{2.727379in}}%
\pgfpathcurveto{\pgfqpoint{3.303060in}{2.735616in}}{\pgfqpoint{3.299788in}{2.743516in}}{\pgfqpoint{3.293964in}{2.749340in}}%
\pgfpathcurveto{\pgfqpoint{3.288140in}{2.755164in}}{\pgfqpoint{3.280240in}{2.758436in}}{\pgfqpoint{3.272004in}{2.758436in}}%
\pgfpathcurveto{\pgfqpoint{3.263767in}{2.758436in}}{\pgfqpoint{3.255867in}{2.755164in}}{\pgfqpoint{3.250043in}{2.749340in}}%
\pgfpathcurveto{\pgfqpoint{3.244219in}{2.743516in}}{\pgfqpoint{3.240947in}{2.735616in}}{\pgfqpoint{3.240947in}{2.727379in}}%
\pgfpathcurveto{\pgfqpoint{3.240947in}{2.719143in}}{\pgfqpoint{3.244219in}{2.711243in}}{\pgfqpoint{3.250043in}{2.705419in}}%
\pgfpathcurveto{\pgfqpoint{3.255867in}{2.699595in}}{\pgfqpoint{3.263767in}{2.696323in}}{\pgfqpoint{3.272004in}{2.696323in}}%
\pgfpathclose%
\pgfusepath{stroke,fill}%
\end{pgfscope}%
\begin{pgfscope}%
\pgfpathrectangle{\pgfqpoint{0.100000in}{0.220728in}}{\pgfqpoint{3.696000in}{3.696000in}}%
\pgfusepath{clip}%
\pgfsetbuttcap%
\pgfsetroundjoin%
\definecolor{currentfill}{rgb}{0.121569,0.466667,0.705882}%
\pgfsetfillcolor{currentfill}%
\pgfsetfillopacity{0.711017}%
\pgfsetlinewidth{1.003750pt}%
\definecolor{currentstroke}{rgb}{0.121569,0.466667,0.705882}%
\pgfsetstrokecolor{currentstroke}%
\pgfsetstrokeopacity{0.711017}%
\pgfsetdash{}{0pt}%
\pgfpathmoveto{\pgfqpoint{0.880277in}{1.329815in}}%
\pgfpathcurveto{\pgfqpoint{0.888513in}{1.329815in}}{\pgfqpoint{0.896413in}{1.333088in}}{\pgfqpoint{0.902237in}{1.338911in}}%
\pgfpathcurveto{\pgfqpoint{0.908061in}{1.344735in}}{\pgfqpoint{0.911333in}{1.352635in}}{\pgfqpoint{0.911333in}{1.360872in}}%
\pgfpathcurveto{\pgfqpoint{0.911333in}{1.369108in}}{\pgfqpoint{0.908061in}{1.377008in}}{\pgfqpoint{0.902237in}{1.382832in}}%
\pgfpathcurveto{\pgfqpoint{0.896413in}{1.388656in}}{\pgfqpoint{0.888513in}{1.391928in}}{\pgfqpoint{0.880277in}{1.391928in}}%
\pgfpathcurveto{\pgfqpoint{0.872040in}{1.391928in}}{\pgfqpoint{0.864140in}{1.388656in}}{\pgfqpoint{0.858316in}{1.382832in}}%
\pgfpathcurveto{\pgfqpoint{0.852492in}{1.377008in}}{\pgfqpoint{0.849220in}{1.369108in}}{\pgfqpoint{0.849220in}{1.360872in}}%
\pgfpathcurveto{\pgfqpoint{0.849220in}{1.352635in}}{\pgfqpoint{0.852492in}{1.344735in}}{\pgfqpoint{0.858316in}{1.338911in}}%
\pgfpathcurveto{\pgfqpoint{0.864140in}{1.333088in}}{\pgfqpoint{0.872040in}{1.329815in}}{\pgfqpoint{0.880277in}{1.329815in}}%
\pgfpathclose%
\pgfusepath{stroke,fill}%
\end{pgfscope}%
\begin{pgfscope}%
\pgfpathrectangle{\pgfqpoint{0.100000in}{0.220728in}}{\pgfqpoint{3.696000in}{3.696000in}}%
\pgfusepath{clip}%
\pgfsetbuttcap%
\pgfsetroundjoin%
\definecolor{currentfill}{rgb}{0.121569,0.466667,0.705882}%
\pgfsetfillcolor{currentfill}%
\pgfsetfillopacity{0.711468}%
\pgfsetlinewidth{1.003750pt}%
\definecolor{currentstroke}{rgb}{0.121569,0.466667,0.705882}%
\pgfsetstrokecolor{currentstroke}%
\pgfsetstrokeopacity{0.711468}%
\pgfsetdash{}{0pt}%
\pgfpathmoveto{\pgfqpoint{3.271012in}{2.693376in}}%
\pgfpathcurveto{\pgfqpoint{3.279248in}{2.693376in}}{\pgfqpoint{3.287148in}{2.696648in}}{\pgfqpoint{3.292972in}{2.702472in}}%
\pgfpathcurveto{\pgfqpoint{3.298796in}{2.708296in}}{\pgfqpoint{3.302068in}{2.716196in}}{\pgfqpoint{3.302068in}{2.724433in}}%
\pgfpathcurveto{\pgfqpoint{3.302068in}{2.732669in}}{\pgfqpoint{3.298796in}{2.740569in}}{\pgfqpoint{3.292972in}{2.746393in}}%
\pgfpathcurveto{\pgfqpoint{3.287148in}{2.752217in}}{\pgfqpoint{3.279248in}{2.755489in}}{\pgfqpoint{3.271012in}{2.755489in}}%
\pgfpathcurveto{\pgfqpoint{3.262775in}{2.755489in}}{\pgfqpoint{3.254875in}{2.752217in}}{\pgfqpoint{3.249051in}{2.746393in}}%
\pgfpathcurveto{\pgfqpoint{3.243227in}{2.740569in}}{\pgfqpoint{3.239955in}{2.732669in}}{\pgfqpoint{3.239955in}{2.724433in}}%
\pgfpathcurveto{\pgfqpoint{3.239955in}{2.716196in}}{\pgfqpoint{3.243227in}{2.708296in}}{\pgfqpoint{3.249051in}{2.702472in}}%
\pgfpathcurveto{\pgfqpoint{3.254875in}{2.696648in}}{\pgfqpoint{3.262775in}{2.693376in}}{\pgfqpoint{3.271012in}{2.693376in}}%
\pgfpathclose%
\pgfusepath{stroke,fill}%
\end{pgfscope}%
\begin{pgfscope}%
\pgfpathrectangle{\pgfqpoint{0.100000in}{0.220728in}}{\pgfqpoint{3.696000in}{3.696000in}}%
\pgfusepath{clip}%
\pgfsetbuttcap%
\pgfsetroundjoin%
\definecolor{currentfill}{rgb}{0.121569,0.466667,0.705882}%
\pgfsetfillcolor{currentfill}%
\pgfsetfillopacity{0.712110}%
\pgfsetlinewidth{1.003750pt}%
\definecolor{currentstroke}{rgb}{0.121569,0.466667,0.705882}%
\pgfsetstrokecolor{currentstroke}%
\pgfsetstrokeopacity{0.712110}%
\pgfsetdash{}{0pt}%
\pgfpathmoveto{\pgfqpoint{3.268394in}{2.689855in}}%
\pgfpathcurveto{\pgfqpoint{3.276630in}{2.689855in}}{\pgfqpoint{3.284530in}{2.693127in}}{\pgfqpoint{3.290354in}{2.698951in}}%
\pgfpathcurveto{\pgfqpoint{3.296178in}{2.704775in}}{\pgfqpoint{3.299450in}{2.712675in}}{\pgfqpoint{3.299450in}{2.720911in}}%
\pgfpathcurveto{\pgfqpoint{3.299450in}{2.729148in}}{\pgfqpoint{3.296178in}{2.737048in}}{\pgfqpoint{3.290354in}{2.742872in}}%
\pgfpathcurveto{\pgfqpoint{3.284530in}{2.748696in}}{\pgfqpoint{3.276630in}{2.751968in}}{\pgfqpoint{3.268394in}{2.751968in}}%
\pgfpathcurveto{\pgfqpoint{3.260157in}{2.751968in}}{\pgfqpoint{3.252257in}{2.748696in}}{\pgfqpoint{3.246433in}{2.742872in}}%
\pgfpathcurveto{\pgfqpoint{3.240609in}{2.737048in}}{\pgfqpoint{3.237337in}{2.729148in}}{\pgfqpoint{3.237337in}{2.720911in}}%
\pgfpathcurveto{\pgfqpoint{3.237337in}{2.712675in}}{\pgfqpoint{3.240609in}{2.704775in}}{\pgfqpoint{3.246433in}{2.698951in}}%
\pgfpathcurveto{\pgfqpoint{3.252257in}{2.693127in}}{\pgfqpoint{3.260157in}{2.689855in}}{\pgfqpoint{3.268394in}{2.689855in}}%
\pgfpathclose%
\pgfusepath{stroke,fill}%
\end{pgfscope}%
\begin{pgfscope}%
\pgfpathrectangle{\pgfqpoint{0.100000in}{0.220728in}}{\pgfqpoint{3.696000in}{3.696000in}}%
\pgfusepath{clip}%
\pgfsetbuttcap%
\pgfsetroundjoin%
\definecolor{currentfill}{rgb}{0.121569,0.466667,0.705882}%
\pgfsetfillcolor{currentfill}%
\pgfsetfillopacity{0.712636}%
\pgfsetlinewidth{1.003750pt}%
\definecolor{currentstroke}{rgb}{0.121569,0.466667,0.705882}%
\pgfsetstrokecolor{currentstroke}%
\pgfsetstrokeopacity{0.712636}%
\pgfsetdash{}{0pt}%
\pgfpathmoveto{\pgfqpoint{3.267340in}{2.688098in}}%
\pgfpathcurveto{\pgfqpoint{3.275576in}{2.688098in}}{\pgfqpoint{3.283476in}{2.691371in}}{\pgfqpoint{3.289300in}{2.697195in}}%
\pgfpathcurveto{\pgfqpoint{3.295124in}{2.703018in}}{\pgfqpoint{3.298396in}{2.710919in}}{\pgfqpoint{3.298396in}{2.719155in}}%
\pgfpathcurveto{\pgfqpoint{3.298396in}{2.727391in}}{\pgfqpoint{3.295124in}{2.735291in}}{\pgfqpoint{3.289300in}{2.741115in}}%
\pgfpathcurveto{\pgfqpoint{3.283476in}{2.746939in}}{\pgfqpoint{3.275576in}{2.750211in}}{\pgfqpoint{3.267340in}{2.750211in}}%
\pgfpathcurveto{\pgfqpoint{3.259104in}{2.750211in}}{\pgfqpoint{3.251204in}{2.746939in}}{\pgfqpoint{3.245380in}{2.741115in}}%
\pgfpathcurveto{\pgfqpoint{3.239556in}{2.735291in}}{\pgfqpoint{3.236283in}{2.727391in}}{\pgfqpoint{3.236283in}{2.719155in}}%
\pgfpathcurveto{\pgfqpoint{3.236283in}{2.710919in}}{\pgfqpoint{3.239556in}{2.703018in}}{\pgfqpoint{3.245380in}{2.697195in}}%
\pgfpathcurveto{\pgfqpoint{3.251204in}{2.691371in}}{\pgfqpoint{3.259104in}{2.688098in}}{\pgfqpoint{3.267340in}{2.688098in}}%
\pgfpathclose%
\pgfusepath{stroke,fill}%
\end{pgfscope}%
\begin{pgfscope}%
\pgfpathrectangle{\pgfqpoint{0.100000in}{0.220728in}}{\pgfqpoint{3.696000in}{3.696000in}}%
\pgfusepath{clip}%
\pgfsetbuttcap%
\pgfsetroundjoin%
\definecolor{currentfill}{rgb}{0.121569,0.466667,0.705882}%
\pgfsetfillcolor{currentfill}%
\pgfsetfillopacity{0.712907}%
\pgfsetlinewidth{1.003750pt}%
\definecolor{currentstroke}{rgb}{0.121569,0.466667,0.705882}%
\pgfsetstrokecolor{currentstroke}%
\pgfsetstrokeopacity{0.712907}%
\pgfsetdash{}{0pt}%
\pgfpathmoveto{\pgfqpoint{3.266922in}{2.686904in}}%
\pgfpathcurveto{\pgfqpoint{3.275158in}{2.686904in}}{\pgfqpoint{3.283058in}{2.690176in}}{\pgfqpoint{3.288882in}{2.696000in}}%
\pgfpathcurveto{\pgfqpoint{3.294706in}{2.701824in}}{\pgfqpoint{3.297978in}{2.709724in}}{\pgfqpoint{3.297978in}{2.717961in}}%
\pgfpathcurveto{\pgfqpoint{3.297978in}{2.726197in}}{\pgfqpoint{3.294706in}{2.734097in}}{\pgfqpoint{3.288882in}{2.739921in}}%
\pgfpathcurveto{\pgfqpoint{3.283058in}{2.745745in}}{\pgfqpoint{3.275158in}{2.749017in}}{\pgfqpoint{3.266922in}{2.749017in}}%
\pgfpathcurveto{\pgfqpoint{3.258685in}{2.749017in}}{\pgfqpoint{3.250785in}{2.745745in}}{\pgfqpoint{3.244962in}{2.739921in}}%
\pgfpathcurveto{\pgfqpoint{3.239138in}{2.734097in}}{\pgfqpoint{3.235865in}{2.726197in}}{\pgfqpoint{3.235865in}{2.717961in}}%
\pgfpathcurveto{\pgfqpoint{3.235865in}{2.709724in}}{\pgfqpoint{3.239138in}{2.701824in}}{\pgfqpoint{3.244962in}{2.696000in}}%
\pgfpathcurveto{\pgfqpoint{3.250785in}{2.690176in}}{\pgfqpoint{3.258685in}{2.686904in}}{\pgfqpoint{3.266922in}{2.686904in}}%
\pgfpathclose%
\pgfusepath{stroke,fill}%
\end{pgfscope}%
\begin{pgfscope}%
\pgfpathrectangle{\pgfqpoint{0.100000in}{0.220728in}}{\pgfqpoint{3.696000in}{3.696000in}}%
\pgfusepath{clip}%
\pgfsetbuttcap%
\pgfsetroundjoin%
\definecolor{currentfill}{rgb}{0.121569,0.466667,0.705882}%
\pgfsetfillcolor{currentfill}%
\pgfsetfillopacity{0.713216}%
\pgfsetlinewidth{1.003750pt}%
\definecolor{currentstroke}{rgb}{0.121569,0.466667,0.705882}%
\pgfsetstrokecolor{currentstroke}%
\pgfsetstrokeopacity{0.713216}%
\pgfsetdash{}{0pt}%
\pgfpathmoveto{\pgfqpoint{3.265685in}{2.685033in}}%
\pgfpathcurveto{\pgfqpoint{3.273921in}{2.685033in}}{\pgfqpoint{3.281821in}{2.688306in}}{\pgfqpoint{3.287645in}{2.694130in}}%
\pgfpathcurveto{\pgfqpoint{3.293469in}{2.699954in}}{\pgfqpoint{3.296742in}{2.707854in}}{\pgfqpoint{3.296742in}{2.716090in}}%
\pgfpathcurveto{\pgfqpoint{3.296742in}{2.724326in}}{\pgfqpoint{3.293469in}{2.732226in}}{\pgfqpoint{3.287645in}{2.738050in}}%
\pgfpathcurveto{\pgfqpoint{3.281821in}{2.743874in}}{\pgfqpoint{3.273921in}{2.747146in}}{\pgfqpoint{3.265685in}{2.747146in}}%
\pgfpathcurveto{\pgfqpoint{3.257449in}{2.747146in}}{\pgfqpoint{3.249549in}{2.743874in}}{\pgfqpoint{3.243725in}{2.738050in}}%
\pgfpathcurveto{\pgfqpoint{3.237901in}{2.732226in}}{\pgfqpoint{3.234629in}{2.724326in}}{\pgfqpoint{3.234629in}{2.716090in}}%
\pgfpathcurveto{\pgfqpoint{3.234629in}{2.707854in}}{\pgfqpoint{3.237901in}{2.699954in}}{\pgfqpoint{3.243725in}{2.694130in}}%
\pgfpathcurveto{\pgfqpoint{3.249549in}{2.688306in}}{\pgfqpoint{3.257449in}{2.685033in}}{\pgfqpoint{3.265685in}{2.685033in}}%
\pgfpathclose%
\pgfusepath{stroke,fill}%
\end{pgfscope}%
\begin{pgfscope}%
\pgfpathrectangle{\pgfqpoint{0.100000in}{0.220728in}}{\pgfqpoint{3.696000in}{3.696000in}}%
\pgfusepath{clip}%
\pgfsetbuttcap%
\pgfsetroundjoin%
\definecolor{currentfill}{rgb}{0.121569,0.466667,0.705882}%
\pgfsetfillcolor{currentfill}%
\pgfsetfillopacity{0.713837}%
\pgfsetlinewidth{1.003750pt}%
\definecolor{currentstroke}{rgb}{0.121569,0.466667,0.705882}%
\pgfsetstrokecolor{currentstroke}%
\pgfsetstrokeopacity{0.713837}%
\pgfsetdash{}{0pt}%
\pgfpathmoveto{\pgfqpoint{0.893794in}{1.322953in}}%
\pgfpathcurveto{\pgfqpoint{0.902031in}{1.322953in}}{\pgfqpoint{0.909931in}{1.326225in}}{\pgfqpoint{0.915755in}{1.332049in}}%
\pgfpathcurveto{\pgfqpoint{0.921579in}{1.337873in}}{\pgfqpoint{0.924851in}{1.345773in}}{\pgfqpoint{0.924851in}{1.354009in}}%
\pgfpathcurveto{\pgfqpoint{0.924851in}{1.362245in}}{\pgfqpoint{0.921579in}{1.370145in}}{\pgfqpoint{0.915755in}{1.375969in}}%
\pgfpathcurveto{\pgfqpoint{0.909931in}{1.381793in}}{\pgfqpoint{0.902031in}{1.385066in}}{\pgfqpoint{0.893794in}{1.385066in}}%
\pgfpathcurveto{\pgfqpoint{0.885558in}{1.385066in}}{\pgfqpoint{0.877658in}{1.381793in}}{\pgfqpoint{0.871834in}{1.375969in}}%
\pgfpathcurveto{\pgfqpoint{0.866010in}{1.370145in}}{\pgfqpoint{0.862738in}{1.362245in}}{\pgfqpoint{0.862738in}{1.354009in}}%
\pgfpathcurveto{\pgfqpoint{0.862738in}{1.345773in}}{\pgfqpoint{0.866010in}{1.337873in}}{\pgfqpoint{0.871834in}{1.332049in}}%
\pgfpathcurveto{\pgfqpoint{0.877658in}{1.326225in}}{\pgfqpoint{0.885558in}{1.322953in}}{\pgfqpoint{0.893794in}{1.322953in}}%
\pgfpathclose%
\pgfusepath{stroke,fill}%
\end{pgfscope}%
\begin{pgfscope}%
\pgfpathrectangle{\pgfqpoint{0.100000in}{0.220728in}}{\pgfqpoint{3.696000in}{3.696000in}}%
\pgfusepath{clip}%
\pgfsetbuttcap%
\pgfsetroundjoin%
\definecolor{currentfill}{rgb}{0.121569,0.466667,0.705882}%
\pgfsetfillcolor{currentfill}%
\pgfsetfillopacity{0.714094}%
\pgfsetlinewidth{1.003750pt}%
\definecolor{currentstroke}{rgb}{0.121569,0.466667,0.705882}%
\pgfsetstrokecolor{currentstroke}%
\pgfsetstrokeopacity{0.714094}%
\pgfsetdash{}{0pt}%
\pgfpathmoveto{\pgfqpoint{3.264342in}{2.680874in}}%
\pgfpathcurveto{\pgfqpoint{3.272579in}{2.680874in}}{\pgfqpoint{3.280479in}{2.684146in}}{\pgfqpoint{3.286303in}{2.689970in}}%
\pgfpathcurveto{\pgfqpoint{3.292127in}{2.695794in}}{\pgfqpoint{3.295399in}{2.703694in}}{\pgfqpoint{3.295399in}{2.711931in}}%
\pgfpathcurveto{\pgfqpoint{3.295399in}{2.720167in}}{\pgfqpoint{3.292127in}{2.728067in}}{\pgfqpoint{3.286303in}{2.733891in}}%
\pgfpathcurveto{\pgfqpoint{3.280479in}{2.739715in}}{\pgfqpoint{3.272579in}{2.742987in}}{\pgfqpoint{3.264342in}{2.742987in}}%
\pgfpathcurveto{\pgfqpoint{3.256106in}{2.742987in}}{\pgfqpoint{3.248206in}{2.739715in}}{\pgfqpoint{3.242382in}{2.733891in}}%
\pgfpathcurveto{\pgfqpoint{3.236558in}{2.728067in}}{\pgfqpoint{3.233286in}{2.720167in}}{\pgfqpoint{3.233286in}{2.711931in}}%
\pgfpathcurveto{\pgfqpoint{3.233286in}{2.703694in}}{\pgfqpoint{3.236558in}{2.695794in}}{\pgfqpoint{3.242382in}{2.689970in}}%
\pgfpathcurveto{\pgfqpoint{3.248206in}{2.684146in}}{\pgfqpoint{3.256106in}{2.680874in}}{\pgfqpoint{3.264342in}{2.680874in}}%
\pgfpathclose%
\pgfusepath{stroke,fill}%
\end{pgfscope}%
\begin{pgfscope}%
\pgfpathrectangle{\pgfqpoint{0.100000in}{0.220728in}}{\pgfqpoint{3.696000in}{3.696000in}}%
\pgfusepath{clip}%
\pgfsetbuttcap%
\pgfsetroundjoin%
\definecolor{currentfill}{rgb}{0.121569,0.466667,0.705882}%
\pgfsetfillcolor{currentfill}%
\pgfsetfillopacity{0.715027}%
\pgfsetlinewidth{1.003750pt}%
\definecolor{currentstroke}{rgb}{0.121569,0.466667,0.705882}%
\pgfsetstrokecolor{currentstroke}%
\pgfsetstrokeopacity{0.715027}%
\pgfsetdash{}{0pt}%
\pgfpathmoveto{\pgfqpoint{3.262108in}{2.676468in}}%
\pgfpathcurveto{\pgfqpoint{3.270345in}{2.676468in}}{\pgfqpoint{3.278245in}{2.679740in}}{\pgfqpoint{3.284069in}{2.685564in}}%
\pgfpathcurveto{\pgfqpoint{3.289893in}{2.691388in}}{\pgfqpoint{3.293165in}{2.699288in}}{\pgfqpoint{3.293165in}{2.707524in}}%
\pgfpathcurveto{\pgfqpoint{3.293165in}{2.715761in}}{\pgfqpoint{3.289893in}{2.723661in}}{\pgfqpoint{3.284069in}{2.729485in}}%
\pgfpathcurveto{\pgfqpoint{3.278245in}{2.735309in}}{\pgfqpoint{3.270345in}{2.738581in}}{\pgfqpoint{3.262108in}{2.738581in}}%
\pgfpathcurveto{\pgfqpoint{3.253872in}{2.738581in}}{\pgfqpoint{3.245972in}{2.735309in}}{\pgfqpoint{3.240148in}{2.729485in}}%
\pgfpathcurveto{\pgfqpoint{3.234324in}{2.723661in}}{\pgfqpoint{3.231052in}{2.715761in}}{\pgfqpoint{3.231052in}{2.707524in}}%
\pgfpathcurveto{\pgfqpoint{3.231052in}{2.699288in}}{\pgfqpoint{3.234324in}{2.691388in}}{\pgfqpoint{3.240148in}{2.685564in}}%
\pgfpathcurveto{\pgfqpoint{3.245972in}{2.679740in}}{\pgfqpoint{3.253872in}{2.676468in}}{\pgfqpoint{3.262108in}{2.676468in}}%
\pgfpathclose%
\pgfusepath{stroke,fill}%
\end{pgfscope}%
\begin{pgfscope}%
\pgfpathrectangle{\pgfqpoint{0.100000in}{0.220728in}}{\pgfqpoint{3.696000in}{3.696000in}}%
\pgfusepath{clip}%
\pgfsetbuttcap%
\pgfsetroundjoin%
\definecolor{currentfill}{rgb}{0.121569,0.466667,0.705882}%
\pgfsetfillcolor{currentfill}%
\pgfsetfillopacity{0.715291}%
\pgfsetlinewidth{1.003750pt}%
\definecolor{currentstroke}{rgb}{0.121569,0.466667,0.705882}%
\pgfsetstrokecolor{currentstroke}%
\pgfsetstrokeopacity{0.715291}%
\pgfsetdash{}{0pt}%
\pgfpathmoveto{\pgfqpoint{0.904858in}{1.313684in}}%
\pgfpathcurveto{\pgfqpoint{0.913095in}{1.313684in}}{\pgfqpoint{0.920995in}{1.316956in}}{\pgfqpoint{0.926819in}{1.322780in}}%
\pgfpathcurveto{\pgfqpoint{0.932642in}{1.328604in}}{\pgfqpoint{0.935915in}{1.336504in}}{\pgfqpoint{0.935915in}{1.344741in}}%
\pgfpathcurveto{\pgfqpoint{0.935915in}{1.352977in}}{\pgfqpoint{0.932642in}{1.360877in}}{\pgfqpoint{0.926819in}{1.366701in}}%
\pgfpathcurveto{\pgfqpoint{0.920995in}{1.372525in}}{\pgfqpoint{0.913095in}{1.375797in}}{\pgfqpoint{0.904858in}{1.375797in}}%
\pgfpathcurveto{\pgfqpoint{0.896622in}{1.375797in}}{\pgfqpoint{0.888722in}{1.372525in}}{\pgfqpoint{0.882898in}{1.366701in}}%
\pgfpathcurveto{\pgfqpoint{0.877074in}{1.360877in}}{\pgfqpoint{0.873802in}{1.352977in}}{\pgfqpoint{0.873802in}{1.344741in}}%
\pgfpathcurveto{\pgfqpoint{0.873802in}{1.336504in}}{\pgfqpoint{0.877074in}{1.328604in}}{\pgfqpoint{0.882898in}{1.322780in}}%
\pgfpathcurveto{\pgfqpoint{0.888722in}{1.316956in}}{\pgfqpoint{0.896622in}{1.313684in}}{\pgfqpoint{0.904858in}{1.313684in}}%
\pgfpathclose%
\pgfusepath{stroke,fill}%
\end{pgfscope}%
\begin{pgfscope}%
\pgfpathrectangle{\pgfqpoint{0.100000in}{0.220728in}}{\pgfqpoint{3.696000in}{3.696000in}}%
\pgfusepath{clip}%
\pgfsetbuttcap%
\pgfsetroundjoin%
\definecolor{currentfill}{rgb}{0.121569,0.466667,0.705882}%
\pgfsetfillcolor{currentfill}%
\pgfsetfillopacity{0.715499}%
\pgfsetlinewidth{1.003750pt}%
\definecolor{currentstroke}{rgb}{0.121569,0.466667,0.705882}%
\pgfsetstrokecolor{currentstroke}%
\pgfsetstrokeopacity{0.715499}%
\pgfsetdash{}{0pt}%
\pgfpathmoveto{\pgfqpoint{3.260493in}{2.674429in}}%
\pgfpathcurveto{\pgfqpoint{3.268729in}{2.674429in}}{\pgfqpoint{3.276629in}{2.677702in}}{\pgfqpoint{3.282453in}{2.683526in}}%
\pgfpathcurveto{\pgfqpoint{3.288277in}{2.689350in}}{\pgfqpoint{3.291549in}{2.697250in}}{\pgfqpoint{3.291549in}{2.705486in}}%
\pgfpathcurveto{\pgfqpoint{3.291549in}{2.713722in}}{\pgfqpoint{3.288277in}{2.721622in}}{\pgfqpoint{3.282453in}{2.727446in}}%
\pgfpathcurveto{\pgfqpoint{3.276629in}{2.733270in}}{\pgfqpoint{3.268729in}{2.736542in}}{\pgfqpoint{3.260493in}{2.736542in}}%
\pgfpathcurveto{\pgfqpoint{3.252256in}{2.736542in}}{\pgfqpoint{3.244356in}{2.733270in}}{\pgfqpoint{3.238532in}{2.727446in}}%
\pgfpathcurveto{\pgfqpoint{3.232708in}{2.721622in}}{\pgfqpoint{3.229436in}{2.713722in}}{\pgfqpoint{3.229436in}{2.705486in}}%
\pgfpathcurveto{\pgfqpoint{3.229436in}{2.697250in}}{\pgfqpoint{3.232708in}{2.689350in}}{\pgfqpoint{3.238532in}{2.683526in}}%
\pgfpathcurveto{\pgfqpoint{3.244356in}{2.677702in}}{\pgfqpoint{3.252256in}{2.674429in}}{\pgfqpoint{3.260493in}{2.674429in}}%
\pgfpathclose%
\pgfusepath{stroke,fill}%
\end{pgfscope}%
\begin{pgfscope}%
\pgfpathrectangle{\pgfqpoint{0.100000in}{0.220728in}}{\pgfqpoint{3.696000in}{3.696000in}}%
\pgfusepath{clip}%
\pgfsetbuttcap%
\pgfsetroundjoin%
\definecolor{currentfill}{rgb}{0.121569,0.466667,0.705882}%
\pgfsetfillcolor{currentfill}%
\pgfsetfillopacity{0.716332}%
\pgfsetlinewidth{1.003750pt}%
\definecolor{currentstroke}{rgb}{0.121569,0.466667,0.705882}%
\pgfsetstrokecolor{currentstroke}%
\pgfsetstrokeopacity{0.716332}%
\pgfsetdash{}{0pt}%
\pgfpathmoveto{\pgfqpoint{3.258673in}{2.669676in}}%
\pgfpathcurveto{\pgfqpoint{3.266909in}{2.669676in}}{\pgfqpoint{3.274809in}{2.672949in}}{\pgfqpoint{3.280633in}{2.678773in}}%
\pgfpathcurveto{\pgfqpoint{3.286457in}{2.684597in}}{\pgfqpoint{3.289730in}{2.692497in}}{\pgfqpoint{3.289730in}{2.700733in}}%
\pgfpathcurveto{\pgfqpoint{3.289730in}{2.708969in}}{\pgfqpoint{3.286457in}{2.716869in}}{\pgfqpoint{3.280633in}{2.722693in}}%
\pgfpathcurveto{\pgfqpoint{3.274809in}{2.728517in}}{\pgfqpoint{3.266909in}{2.731789in}}{\pgfqpoint{3.258673in}{2.731789in}}%
\pgfpathcurveto{\pgfqpoint{3.250437in}{2.731789in}}{\pgfqpoint{3.242537in}{2.728517in}}{\pgfqpoint{3.236713in}{2.722693in}}%
\pgfpathcurveto{\pgfqpoint{3.230889in}{2.716869in}}{\pgfqpoint{3.227617in}{2.708969in}}{\pgfqpoint{3.227617in}{2.700733in}}%
\pgfpathcurveto{\pgfqpoint{3.227617in}{2.692497in}}{\pgfqpoint{3.230889in}{2.684597in}}{\pgfqpoint{3.236713in}{2.678773in}}%
\pgfpathcurveto{\pgfqpoint{3.242537in}{2.672949in}}{\pgfqpoint{3.250437in}{2.669676in}}{\pgfqpoint{3.258673in}{2.669676in}}%
\pgfpathclose%
\pgfusepath{stroke,fill}%
\end{pgfscope}%
\begin{pgfscope}%
\pgfpathrectangle{\pgfqpoint{0.100000in}{0.220728in}}{\pgfqpoint{3.696000in}{3.696000in}}%
\pgfusepath{clip}%
\pgfsetbuttcap%
\pgfsetroundjoin%
\definecolor{currentfill}{rgb}{0.121569,0.466667,0.705882}%
\pgfsetfillcolor{currentfill}%
\pgfsetfillopacity{0.716808}%
\pgfsetlinewidth{1.003750pt}%
\definecolor{currentstroke}{rgb}{0.121569,0.466667,0.705882}%
\pgfsetstrokecolor{currentstroke}%
\pgfsetstrokeopacity{0.716808}%
\pgfsetdash{}{0pt}%
\pgfpathmoveto{\pgfqpoint{3.257264in}{2.667631in}}%
\pgfpathcurveto{\pgfqpoint{3.265501in}{2.667631in}}{\pgfqpoint{3.273401in}{2.670903in}}{\pgfqpoint{3.279225in}{2.676727in}}%
\pgfpathcurveto{\pgfqpoint{3.285049in}{2.682551in}}{\pgfqpoint{3.288321in}{2.690451in}}{\pgfqpoint{3.288321in}{2.698687in}}%
\pgfpathcurveto{\pgfqpoint{3.288321in}{2.706923in}}{\pgfqpoint{3.285049in}{2.714823in}}{\pgfqpoint{3.279225in}{2.720647in}}%
\pgfpathcurveto{\pgfqpoint{3.273401in}{2.726471in}}{\pgfqpoint{3.265501in}{2.729744in}}{\pgfqpoint{3.257264in}{2.729744in}}%
\pgfpathcurveto{\pgfqpoint{3.249028in}{2.729744in}}{\pgfqpoint{3.241128in}{2.726471in}}{\pgfqpoint{3.235304in}{2.720647in}}%
\pgfpathcurveto{\pgfqpoint{3.229480in}{2.714823in}}{\pgfqpoint{3.226208in}{2.706923in}}{\pgfqpoint{3.226208in}{2.698687in}}%
\pgfpathcurveto{\pgfqpoint{3.226208in}{2.690451in}}{\pgfqpoint{3.229480in}{2.682551in}}{\pgfqpoint{3.235304in}{2.676727in}}%
\pgfpathcurveto{\pgfqpoint{3.241128in}{2.670903in}}{\pgfqpoint{3.249028in}{2.667631in}}{\pgfqpoint{3.257264in}{2.667631in}}%
\pgfpathclose%
\pgfusepath{stroke,fill}%
\end{pgfscope}%
\begin{pgfscope}%
\pgfpathrectangle{\pgfqpoint{0.100000in}{0.220728in}}{\pgfqpoint{3.696000in}{3.696000in}}%
\pgfusepath{clip}%
\pgfsetbuttcap%
\pgfsetroundjoin%
\definecolor{currentfill}{rgb}{0.121569,0.466667,0.705882}%
\pgfsetfillcolor{currentfill}%
\pgfsetfillopacity{0.717074}%
\pgfsetlinewidth{1.003750pt}%
\definecolor{currentstroke}{rgb}{0.121569,0.466667,0.705882}%
\pgfsetstrokecolor{currentstroke}%
\pgfsetstrokeopacity{0.717074}%
\pgfsetdash{}{0pt}%
\pgfpathmoveto{\pgfqpoint{3.256491in}{2.666520in}}%
\pgfpathcurveto{\pgfqpoint{3.264727in}{2.666520in}}{\pgfqpoint{3.272627in}{2.669792in}}{\pgfqpoint{3.278451in}{2.675616in}}%
\pgfpathcurveto{\pgfqpoint{3.284275in}{2.681440in}}{\pgfqpoint{3.287547in}{2.689340in}}{\pgfqpoint{3.287547in}{2.697576in}}%
\pgfpathcurveto{\pgfqpoint{3.287547in}{2.705812in}}{\pgfqpoint{3.284275in}{2.713712in}}{\pgfqpoint{3.278451in}{2.719536in}}%
\pgfpathcurveto{\pgfqpoint{3.272627in}{2.725360in}}{\pgfqpoint{3.264727in}{2.728633in}}{\pgfqpoint{3.256491in}{2.728633in}}%
\pgfpathcurveto{\pgfqpoint{3.248255in}{2.728633in}}{\pgfqpoint{3.240355in}{2.725360in}}{\pgfqpoint{3.234531in}{2.719536in}}%
\pgfpathcurveto{\pgfqpoint{3.228707in}{2.713712in}}{\pgfqpoint{3.225434in}{2.705812in}}{\pgfqpoint{3.225434in}{2.697576in}}%
\pgfpathcurveto{\pgfqpoint{3.225434in}{2.689340in}}{\pgfqpoint{3.228707in}{2.681440in}}{\pgfqpoint{3.234531in}{2.675616in}}%
\pgfpathcurveto{\pgfqpoint{3.240355in}{2.669792in}}{\pgfqpoint{3.248255in}{2.666520in}}{\pgfqpoint{3.256491in}{2.666520in}}%
\pgfpathclose%
\pgfusepath{stroke,fill}%
\end{pgfscope}%
\begin{pgfscope}%
\pgfpathrectangle{\pgfqpoint{0.100000in}{0.220728in}}{\pgfqpoint{3.696000in}{3.696000in}}%
\pgfusepath{clip}%
\pgfsetbuttcap%
\pgfsetroundjoin%
\definecolor{currentfill}{rgb}{0.121569,0.466667,0.705882}%
\pgfsetfillcolor{currentfill}%
\pgfsetfillopacity{0.717212}%
\pgfsetlinewidth{1.003750pt}%
\definecolor{currentstroke}{rgb}{0.121569,0.466667,0.705882}%
\pgfsetstrokecolor{currentstroke}%
\pgfsetstrokeopacity{0.717212}%
\pgfsetdash{}{0pt}%
\pgfpathmoveto{\pgfqpoint{3.256190in}{2.665727in}}%
\pgfpathcurveto{\pgfqpoint{3.264426in}{2.665727in}}{\pgfqpoint{3.272326in}{2.668999in}}{\pgfqpoint{3.278150in}{2.674823in}}%
\pgfpathcurveto{\pgfqpoint{3.283974in}{2.680647in}}{\pgfqpoint{3.287246in}{2.688547in}}{\pgfqpoint{3.287246in}{2.696783in}}%
\pgfpathcurveto{\pgfqpoint{3.287246in}{2.705019in}}{\pgfqpoint{3.283974in}{2.712919in}}{\pgfqpoint{3.278150in}{2.718743in}}%
\pgfpathcurveto{\pgfqpoint{3.272326in}{2.724567in}}{\pgfqpoint{3.264426in}{2.727840in}}{\pgfqpoint{3.256190in}{2.727840in}}%
\pgfpathcurveto{\pgfqpoint{3.247953in}{2.727840in}}{\pgfqpoint{3.240053in}{2.724567in}}{\pgfqpoint{3.234230in}{2.718743in}}%
\pgfpathcurveto{\pgfqpoint{3.228406in}{2.712919in}}{\pgfqpoint{3.225133in}{2.705019in}}{\pgfqpoint{3.225133in}{2.696783in}}%
\pgfpathcurveto{\pgfqpoint{3.225133in}{2.688547in}}{\pgfqpoint{3.228406in}{2.680647in}}{\pgfqpoint{3.234230in}{2.674823in}}%
\pgfpathcurveto{\pgfqpoint{3.240053in}{2.668999in}}{\pgfqpoint{3.247953in}{2.665727in}}{\pgfqpoint{3.256190in}{2.665727in}}%
\pgfpathclose%
\pgfusepath{stroke,fill}%
\end{pgfscope}%
\begin{pgfscope}%
\pgfpathrectangle{\pgfqpoint{0.100000in}{0.220728in}}{\pgfqpoint{3.696000in}{3.696000in}}%
\pgfusepath{clip}%
\pgfsetbuttcap%
\pgfsetroundjoin%
\definecolor{currentfill}{rgb}{0.121569,0.466667,0.705882}%
\pgfsetfillcolor{currentfill}%
\pgfsetfillopacity{0.717662}%
\pgfsetlinewidth{1.003750pt}%
\definecolor{currentstroke}{rgb}{0.121569,0.466667,0.705882}%
\pgfsetstrokecolor{currentstroke}%
\pgfsetstrokeopacity{0.717662}%
\pgfsetdash{}{0pt}%
\pgfpathmoveto{\pgfqpoint{3.253969in}{2.662869in}}%
\pgfpathcurveto{\pgfqpoint{3.262205in}{2.662869in}}{\pgfqpoint{3.270105in}{2.666141in}}{\pgfqpoint{3.275929in}{2.671965in}}%
\pgfpathcurveto{\pgfqpoint{3.281753in}{2.677789in}}{\pgfqpoint{3.285025in}{2.685689in}}{\pgfqpoint{3.285025in}{2.693925in}}%
\pgfpathcurveto{\pgfqpoint{3.285025in}{2.702161in}}{\pgfqpoint{3.281753in}{2.710061in}}{\pgfqpoint{3.275929in}{2.715885in}}%
\pgfpathcurveto{\pgfqpoint{3.270105in}{2.721709in}}{\pgfqpoint{3.262205in}{2.724982in}}{\pgfqpoint{3.253969in}{2.724982in}}%
\pgfpathcurveto{\pgfqpoint{3.245732in}{2.724982in}}{\pgfqpoint{3.237832in}{2.721709in}}{\pgfqpoint{3.232009in}{2.715885in}}%
\pgfpathcurveto{\pgfqpoint{3.226185in}{2.710061in}}{\pgfqpoint{3.222912in}{2.702161in}}{\pgfqpoint{3.222912in}{2.693925in}}%
\pgfpathcurveto{\pgfqpoint{3.222912in}{2.685689in}}{\pgfqpoint{3.226185in}{2.677789in}}{\pgfqpoint{3.232009in}{2.671965in}}%
\pgfpathcurveto{\pgfqpoint{3.237832in}{2.666141in}}{\pgfqpoint{3.245732in}{2.662869in}}{\pgfqpoint{3.253969in}{2.662869in}}%
\pgfpathclose%
\pgfusepath{stroke,fill}%
\end{pgfscope}%
\begin{pgfscope}%
\pgfpathrectangle{\pgfqpoint{0.100000in}{0.220728in}}{\pgfqpoint{3.696000in}{3.696000in}}%
\pgfusepath{clip}%
\pgfsetbuttcap%
\pgfsetroundjoin%
\definecolor{currentfill}{rgb}{0.121569,0.466667,0.705882}%
\pgfsetfillcolor{currentfill}%
\pgfsetfillopacity{0.718004}%
\pgfsetlinewidth{1.003750pt}%
\definecolor{currentstroke}{rgb}{0.121569,0.466667,0.705882}%
\pgfsetstrokecolor{currentstroke}%
\pgfsetstrokeopacity{0.718004}%
\pgfsetdash{}{0pt}%
\pgfpathmoveto{\pgfqpoint{3.253130in}{2.661110in}}%
\pgfpathcurveto{\pgfqpoint{3.261367in}{2.661110in}}{\pgfqpoint{3.269267in}{2.664382in}}{\pgfqpoint{3.275091in}{2.670206in}}%
\pgfpathcurveto{\pgfqpoint{3.280915in}{2.676030in}}{\pgfqpoint{3.284187in}{2.683930in}}{\pgfqpoint{3.284187in}{2.692167in}}%
\pgfpathcurveto{\pgfqpoint{3.284187in}{2.700403in}}{\pgfqpoint{3.280915in}{2.708303in}}{\pgfqpoint{3.275091in}{2.714127in}}%
\pgfpathcurveto{\pgfqpoint{3.269267in}{2.719951in}}{\pgfqpoint{3.261367in}{2.723223in}}{\pgfqpoint{3.253130in}{2.723223in}}%
\pgfpathcurveto{\pgfqpoint{3.244894in}{2.723223in}}{\pgfqpoint{3.236994in}{2.719951in}}{\pgfqpoint{3.231170in}{2.714127in}}%
\pgfpathcurveto{\pgfqpoint{3.225346in}{2.708303in}}{\pgfqpoint{3.222074in}{2.700403in}}{\pgfqpoint{3.222074in}{2.692167in}}%
\pgfpathcurveto{\pgfqpoint{3.222074in}{2.683930in}}{\pgfqpoint{3.225346in}{2.676030in}}{\pgfqpoint{3.231170in}{2.670206in}}%
\pgfpathcurveto{\pgfqpoint{3.236994in}{2.664382in}}{\pgfqpoint{3.244894in}{2.661110in}}{\pgfqpoint{3.253130in}{2.661110in}}%
\pgfpathclose%
\pgfusepath{stroke,fill}%
\end{pgfscope}%
\begin{pgfscope}%
\pgfpathrectangle{\pgfqpoint{0.100000in}{0.220728in}}{\pgfqpoint{3.696000in}{3.696000in}}%
\pgfusepath{clip}%
\pgfsetbuttcap%
\pgfsetroundjoin%
\definecolor{currentfill}{rgb}{0.121569,0.466667,0.705882}%
\pgfsetfillcolor{currentfill}%
\pgfsetfillopacity{0.718177}%
\pgfsetlinewidth{1.003750pt}%
\definecolor{currentstroke}{rgb}{0.121569,0.466667,0.705882}%
\pgfsetstrokecolor{currentstroke}%
\pgfsetstrokeopacity{0.718177}%
\pgfsetdash{}{0pt}%
\pgfpathmoveto{\pgfqpoint{3.252655in}{2.660098in}}%
\pgfpathcurveto{\pgfqpoint{3.260891in}{2.660098in}}{\pgfqpoint{3.268791in}{2.663370in}}{\pgfqpoint{3.274615in}{2.669194in}}%
\pgfpathcurveto{\pgfqpoint{3.280439in}{2.675018in}}{\pgfqpoint{3.283711in}{2.682918in}}{\pgfqpoint{3.283711in}{2.691154in}}%
\pgfpathcurveto{\pgfqpoint{3.283711in}{2.699390in}}{\pgfqpoint{3.280439in}{2.707290in}}{\pgfqpoint{3.274615in}{2.713114in}}%
\pgfpathcurveto{\pgfqpoint{3.268791in}{2.718938in}}{\pgfqpoint{3.260891in}{2.722211in}}{\pgfqpoint{3.252655in}{2.722211in}}%
\pgfpathcurveto{\pgfqpoint{3.244419in}{2.722211in}}{\pgfqpoint{3.236519in}{2.718938in}}{\pgfqpoint{3.230695in}{2.713114in}}%
\pgfpathcurveto{\pgfqpoint{3.224871in}{2.707290in}}{\pgfqpoint{3.221598in}{2.699390in}}{\pgfqpoint{3.221598in}{2.691154in}}%
\pgfpathcurveto{\pgfqpoint{3.221598in}{2.682918in}}{\pgfqpoint{3.224871in}{2.675018in}}{\pgfqpoint{3.230695in}{2.669194in}}%
\pgfpathcurveto{\pgfqpoint{3.236519in}{2.663370in}}{\pgfqpoint{3.244419in}{2.660098in}}{\pgfqpoint{3.252655in}{2.660098in}}%
\pgfpathclose%
\pgfusepath{stroke,fill}%
\end{pgfscope}%
\begin{pgfscope}%
\pgfpathrectangle{\pgfqpoint{0.100000in}{0.220728in}}{\pgfqpoint{3.696000in}{3.696000in}}%
\pgfusepath{clip}%
\pgfsetbuttcap%
\pgfsetroundjoin%
\definecolor{currentfill}{rgb}{0.121569,0.466667,0.705882}%
\pgfsetfillcolor{currentfill}%
\pgfsetfillopacity{0.718405}%
\pgfsetlinewidth{1.003750pt}%
\definecolor{currentstroke}{rgb}{0.121569,0.466667,0.705882}%
\pgfsetstrokecolor{currentstroke}%
\pgfsetstrokeopacity{0.718405}%
\pgfsetdash{}{0pt}%
\pgfpathmoveto{\pgfqpoint{0.925771in}{1.301443in}}%
\pgfpathcurveto{\pgfqpoint{0.934007in}{1.301443in}}{\pgfqpoint{0.941907in}{1.304715in}}{\pgfqpoint{0.947731in}{1.310539in}}%
\pgfpathcurveto{\pgfqpoint{0.953555in}{1.316363in}}{\pgfqpoint{0.956827in}{1.324263in}}{\pgfqpoint{0.956827in}{1.332499in}}%
\pgfpathcurveto{\pgfqpoint{0.956827in}{1.340735in}}{\pgfqpoint{0.953555in}{1.348636in}}{\pgfqpoint{0.947731in}{1.354459in}}%
\pgfpathcurveto{\pgfqpoint{0.941907in}{1.360283in}}{\pgfqpoint{0.934007in}{1.363556in}}{\pgfqpoint{0.925771in}{1.363556in}}%
\pgfpathcurveto{\pgfqpoint{0.917534in}{1.363556in}}{\pgfqpoint{0.909634in}{1.360283in}}{\pgfqpoint{0.903810in}{1.354459in}}%
\pgfpathcurveto{\pgfqpoint{0.897986in}{1.348636in}}{\pgfqpoint{0.894714in}{1.340735in}}{\pgfqpoint{0.894714in}{1.332499in}}%
\pgfpathcurveto{\pgfqpoint{0.894714in}{1.324263in}}{\pgfqpoint{0.897986in}{1.316363in}}{\pgfqpoint{0.903810in}{1.310539in}}%
\pgfpathcurveto{\pgfqpoint{0.909634in}{1.304715in}}{\pgfqpoint{0.917534in}{1.301443in}}{\pgfqpoint{0.925771in}{1.301443in}}%
\pgfpathclose%
\pgfusepath{stroke,fill}%
\end{pgfscope}%
\begin{pgfscope}%
\pgfpathrectangle{\pgfqpoint{0.100000in}{0.220728in}}{\pgfqpoint{3.696000in}{3.696000in}}%
\pgfusepath{clip}%
\pgfsetbuttcap%
\pgfsetroundjoin%
\definecolor{currentfill}{rgb}{0.121569,0.466667,0.705882}%
\pgfsetfillcolor{currentfill}%
\pgfsetfillopacity{0.718471}%
\pgfsetlinewidth{1.003750pt}%
\definecolor{currentstroke}{rgb}{0.121569,0.466667,0.705882}%
\pgfsetstrokecolor{currentstroke}%
\pgfsetstrokeopacity{0.718471}%
\pgfsetdash{}{0pt}%
\pgfpathmoveto{\pgfqpoint{3.251652in}{2.658790in}}%
\pgfpathcurveto{\pgfqpoint{3.259888in}{2.658790in}}{\pgfqpoint{3.267788in}{2.662062in}}{\pgfqpoint{3.273612in}{2.667886in}}%
\pgfpathcurveto{\pgfqpoint{3.279436in}{2.673710in}}{\pgfqpoint{3.282708in}{2.681610in}}{\pgfqpoint{3.282708in}{2.689846in}}%
\pgfpathcurveto{\pgfqpoint{3.282708in}{2.698082in}}{\pgfqpoint{3.279436in}{2.705983in}}{\pgfqpoint{3.273612in}{2.711806in}}%
\pgfpathcurveto{\pgfqpoint{3.267788in}{2.717630in}}{\pgfqpoint{3.259888in}{2.720903in}}{\pgfqpoint{3.251652in}{2.720903in}}%
\pgfpathcurveto{\pgfqpoint{3.243415in}{2.720903in}}{\pgfqpoint{3.235515in}{2.717630in}}{\pgfqpoint{3.229691in}{2.711806in}}%
\pgfpathcurveto{\pgfqpoint{3.223867in}{2.705983in}}{\pgfqpoint{3.220595in}{2.698082in}}{\pgfqpoint{3.220595in}{2.689846in}}%
\pgfpathcurveto{\pgfqpoint{3.220595in}{2.681610in}}{\pgfqpoint{3.223867in}{2.673710in}}{\pgfqpoint{3.229691in}{2.667886in}}%
\pgfpathcurveto{\pgfqpoint{3.235515in}{2.662062in}}{\pgfqpoint{3.243415in}{2.658790in}}{\pgfqpoint{3.251652in}{2.658790in}}%
\pgfpathclose%
\pgfusepath{stroke,fill}%
\end{pgfscope}%
\begin{pgfscope}%
\pgfpathrectangle{\pgfqpoint{0.100000in}{0.220728in}}{\pgfqpoint{3.696000in}{3.696000in}}%
\pgfusepath{clip}%
\pgfsetbuttcap%
\pgfsetroundjoin%
\definecolor{currentfill}{rgb}{0.121569,0.466667,0.705882}%
\pgfsetfillcolor{currentfill}%
\pgfsetfillopacity{0.719004}%
\pgfsetlinewidth{1.003750pt}%
\definecolor{currentstroke}{rgb}{0.121569,0.466667,0.705882}%
\pgfsetstrokecolor{currentstroke}%
\pgfsetstrokeopacity{0.719004}%
\pgfsetdash{}{0pt}%
\pgfpathmoveto{\pgfqpoint{3.250301in}{2.656543in}}%
\pgfpathcurveto{\pgfqpoint{3.258537in}{2.656543in}}{\pgfqpoint{3.266437in}{2.659815in}}{\pgfqpoint{3.272261in}{2.665639in}}%
\pgfpathcurveto{\pgfqpoint{3.278085in}{2.671463in}}{\pgfqpoint{3.281357in}{2.679363in}}{\pgfqpoint{3.281357in}{2.687600in}}%
\pgfpathcurveto{\pgfqpoint{3.281357in}{2.695836in}}{\pgfqpoint{3.278085in}{2.703736in}}{\pgfqpoint{3.272261in}{2.709560in}}%
\pgfpathcurveto{\pgfqpoint{3.266437in}{2.715384in}}{\pgfqpoint{3.258537in}{2.718656in}}{\pgfqpoint{3.250301in}{2.718656in}}%
\pgfpathcurveto{\pgfqpoint{3.242064in}{2.718656in}}{\pgfqpoint{3.234164in}{2.715384in}}{\pgfqpoint{3.228340in}{2.709560in}}%
\pgfpathcurveto{\pgfqpoint{3.222516in}{2.703736in}}{\pgfqpoint{3.219244in}{2.695836in}}{\pgfqpoint{3.219244in}{2.687600in}}%
\pgfpathcurveto{\pgfqpoint{3.219244in}{2.679363in}}{\pgfqpoint{3.222516in}{2.671463in}}{\pgfqpoint{3.228340in}{2.665639in}}%
\pgfpathcurveto{\pgfqpoint{3.234164in}{2.659815in}}{\pgfqpoint{3.242064in}{2.656543in}}{\pgfqpoint{3.250301in}{2.656543in}}%
\pgfpathclose%
\pgfusepath{stroke,fill}%
\end{pgfscope}%
\begin{pgfscope}%
\pgfpathrectangle{\pgfqpoint{0.100000in}{0.220728in}}{\pgfqpoint{3.696000in}{3.696000in}}%
\pgfusepath{clip}%
\pgfsetbuttcap%
\pgfsetroundjoin%
\definecolor{currentfill}{rgb}{0.121569,0.466667,0.705882}%
\pgfsetfillcolor{currentfill}%
\pgfsetfillopacity{0.719639}%
\pgfsetlinewidth{1.003750pt}%
\definecolor{currentstroke}{rgb}{0.121569,0.466667,0.705882}%
\pgfsetstrokecolor{currentstroke}%
\pgfsetstrokeopacity{0.719639}%
\pgfsetdash{}{0pt}%
\pgfpathmoveto{\pgfqpoint{3.249195in}{2.652981in}}%
\pgfpathcurveto{\pgfqpoint{3.257431in}{2.652981in}}{\pgfqpoint{3.265331in}{2.656253in}}{\pgfqpoint{3.271155in}{2.662077in}}%
\pgfpathcurveto{\pgfqpoint{3.276979in}{2.667901in}}{\pgfqpoint{3.280252in}{2.675801in}}{\pgfqpoint{3.280252in}{2.684037in}}%
\pgfpathcurveto{\pgfqpoint{3.280252in}{2.692274in}}{\pgfqpoint{3.276979in}{2.700174in}}{\pgfqpoint{3.271155in}{2.705998in}}%
\pgfpathcurveto{\pgfqpoint{3.265331in}{2.711822in}}{\pgfqpoint{3.257431in}{2.715094in}}{\pgfqpoint{3.249195in}{2.715094in}}%
\pgfpathcurveto{\pgfqpoint{3.240959in}{2.715094in}}{\pgfqpoint{3.233059in}{2.711822in}}{\pgfqpoint{3.227235in}{2.705998in}}%
\pgfpathcurveto{\pgfqpoint{3.221411in}{2.700174in}}{\pgfqpoint{3.218139in}{2.692274in}}{\pgfqpoint{3.218139in}{2.684037in}}%
\pgfpathcurveto{\pgfqpoint{3.218139in}{2.675801in}}{\pgfqpoint{3.221411in}{2.667901in}}{\pgfqpoint{3.227235in}{2.662077in}}%
\pgfpathcurveto{\pgfqpoint{3.233059in}{2.656253in}}{\pgfqpoint{3.240959in}{2.652981in}}{\pgfqpoint{3.249195in}{2.652981in}}%
\pgfpathclose%
\pgfusepath{stroke,fill}%
\end{pgfscope}%
\begin{pgfscope}%
\pgfpathrectangle{\pgfqpoint{0.100000in}{0.220728in}}{\pgfqpoint{3.696000in}{3.696000in}}%
\pgfusepath{clip}%
\pgfsetbuttcap%
\pgfsetroundjoin%
\definecolor{currentfill}{rgb}{0.121569,0.466667,0.705882}%
\pgfsetfillcolor{currentfill}%
\pgfsetfillopacity{0.720492}%
\pgfsetlinewidth{1.003750pt}%
\definecolor{currentstroke}{rgb}{0.121569,0.466667,0.705882}%
\pgfsetstrokecolor{currentstroke}%
\pgfsetstrokeopacity{0.720492}%
\pgfsetdash{}{0pt}%
\pgfpathmoveto{\pgfqpoint{3.246124in}{2.648701in}}%
\pgfpathcurveto{\pgfqpoint{3.254361in}{2.648701in}}{\pgfqpoint{3.262261in}{2.651974in}}{\pgfqpoint{3.268085in}{2.657797in}}%
\pgfpathcurveto{\pgfqpoint{3.273909in}{2.663621in}}{\pgfqpoint{3.277181in}{2.671521in}}{\pgfqpoint{3.277181in}{2.679758in}}%
\pgfpathcurveto{\pgfqpoint{3.277181in}{2.687994in}}{\pgfqpoint{3.273909in}{2.695894in}}{\pgfqpoint{3.268085in}{2.701718in}}%
\pgfpathcurveto{\pgfqpoint{3.262261in}{2.707542in}}{\pgfqpoint{3.254361in}{2.710814in}}{\pgfqpoint{3.246124in}{2.710814in}}%
\pgfpathcurveto{\pgfqpoint{3.237888in}{2.710814in}}{\pgfqpoint{3.229988in}{2.707542in}}{\pgfqpoint{3.224164in}{2.701718in}}%
\pgfpathcurveto{\pgfqpoint{3.218340in}{2.695894in}}{\pgfqpoint{3.215068in}{2.687994in}}{\pgfqpoint{3.215068in}{2.679758in}}%
\pgfpathcurveto{\pgfqpoint{3.215068in}{2.671521in}}{\pgfqpoint{3.218340in}{2.663621in}}{\pgfqpoint{3.224164in}{2.657797in}}%
\pgfpathcurveto{\pgfqpoint{3.229988in}{2.651974in}}{\pgfqpoint{3.237888in}{2.648701in}}{\pgfqpoint{3.246124in}{2.648701in}}%
\pgfpathclose%
\pgfusepath{stroke,fill}%
\end{pgfscope}%
\begin{pgfscope}%
\pgfpathrectangle{\pgfqpoint{0.100000in}{0.220728in}}{\pgfqpoint{3.696000in}{3.696000in}}%
\pgfusepath{clip}%
\pgfsetbuttcap%
\pgfsetroundjoin%
\definecolor{currentfill}{rgb}{0.121569,0.466667,0.705882}%
\pgfsetfillcolor{currentfill}%
\pgfsetfillopacity{0.721073}%
\pgfsetlinewidth{1.003750pt}%
\definecolor{currentstroke}{rgb}{0.121569,0.466667,0.705882}%
\pgfsetstrokecolor{currentstroke}%
\pgfsetstrokeopacity{0.721073}%
\pgfsetdash{}{0pt}%
\pgfpathmoveto{\pgfqpoint{3.244624in}{2.646537in}}%
\pgfpathcurveto{\pgfqpoint{3.252860in}{2.646537in}}{\pgfqpoint{3.260760in}{2.649810in}}{\pgfqpoint{3.266584in}{2.655634in}}%
\pgfpathcurveto{\pgfqpoint{3.272408in}{2.661457in}}{\pgfqpoint{3.275681in}{2.669358in}}{\pgfqpoint{3.275681in}{2.677594in}}%
\pgfpathcurveto{\pgfqpoint{3.275681in}{2.685830in}}{\pgfqpoint{3.272408in}{2.693730in}}{\pgfqpoint{3.266584in}{2.699554in}}%
\pgfpathcurveto{\pgfqpoint{3.260760in}{2.705378in}}{\pgfqpoint{3.252860in}{2.708650in}}{\pgfqpoint{3.244624in}{2.708650in}}%
\pgfpathcurveto{\pgfqpoint{3.236388in}{2.708650in}}{\pgfqpoint{3.228488in}{2.705378in}}{\pgfqpoint{3.222664in}{2.699554in}}%
\pgfpathcurveto{\pgfqpoint{3.216840in}{2.693730in}}{\pgfqpoint{3.213568in}{2.685830in}}{\pgfqpoint{3.213568in}{2.677594in}}%
\pgfpathcurveto{\pgfqpoint{3.213568in}{2.669358in}}{\pgfqpoint{3.216840in}{2.661457in}}{\pgfqpoint{3.222664in}{2.655634in}}%
\pgfpathcurveto{\pgfqpoint{3.228488in}{2.649810in}}{\pgfqpoint{3.236388in}{2.646537in}}{\pgfqpoint{3.244624in}{2.646537in}}%
\pgfpathclose%
\pgfusepath{stroke,fill}%
\end{pgfscope}%
\begin{pgfscope}%
\pgfpathrectangle{\pgfqpoint{0.100000in}{0.220728in}}{\pgfqpoint{3.696000in}{3.696000in}}%
\pgfusepath{clip}%
\pgfsetbuttcap%
\pgfsetroundjoin%
\definecolor{currentfill}{rgb}{0.121569,0.466667,0.705882}%
\pgfsetfillcolor{currentfill}%
\pgfsetfillopacity{0.721611}%
\pgfsetlinewidth{1.003750pt}%
\definecolor{currentstroke}{rgb}{0.121569,0.466667,0.705882}%
\pgfsetstrokecolor{currentstroke}%
\pgfsetstrokeopacity{0.721611}%
\pgfsetdash{}{0pt}%
\pgfpathmoveto{\pgfqpoint{3.242935in}{2.642846in}}%
\pgfpathcurveto{\pgfqpoint{3.251171in}{2.642846in}}{\pgfqpoint{3.259072in}{2.646118in}}{\pgfqpoint{3.264895in}{2.651942in}}%
\pgfpathcurveto{\pgfqpoint{3.270719in}{2.657766in}}{\pgfqpoint{3.273992in}{2.665666in}}{\pgfqpoint{3.273992in}{2.673902in}}%
\pgfpathcurveto{\pgfqpoint{3.273992in}{2.682139in}}{\pgfqpoint{3.270719in}{2.690039in}}{\pgfqpoint{3.264895in}{2.695863in}}%
\pgfpathcurveto{\pgfqpoint{3.259072in}{2.701686in}}{\pgfqpoint{3.251171in}{2.704959in}}{\pgfqpoint{3.242935in}{2.704959in}}%
\pgfpathcurveto{\pgfqpoint{3.234699in}{2.704959in}}{\pgfqpoint{3.226799in}{2.701686in}}{\pgfqpoint{3.220975in}{2.695863in}}%
\pgfpathcurveto{\pgfqpoint{3.215151in}{2.690039in}}{\pgfqpoint{3.211879in}{2.682139in}}{\pgfqpoint{3.211879in}{2.673902in}}%
\pgfpathcurveto{\pgfqpoint{3.211879in}{2.665666in}}{\pgfqpoint{3.215151in}{2.657766in}}{\pgfqpoint{3.220975in}{2.651942in}}%
\pgfpathcurveto{\pgfqpoint{3.226799in}{2.646118in}}{\pgfqpoint{3.234699in}{2.642846in}}{\pgfqpoint{3.242935in}{2.642846in}}%
\pgfpathclose%
\pgfusepath{stroke,fill}%
\end{pgfscope}%
\begin{pgfscope}%
\pgfpathrectangle{\pgfqpoint{0.100000in}{0.220728in}}{\pgfqpoint{3.696000in}{3.696000in}}%
\pgfusepath{clip}%
\pgfsetbuttcap%
\pgfsetroundjoin%
\definecolor{currentfill}{rgb}{0.121569,0.466667,0.705882}%
\pgfsetfillcolor{currentfill}%
\pgfsetfillopacity{0.722390}%
\pgfsetlinewidth{1.003750pt}%
\definecolor{currentstroke}{rgb}{0.121569,0.466667,0.705882}%
\pgfsetstrokecolor{currentstroke}%
\pgfsetstrokeopacity{0.722390}%
\pgfsetdash{}{0pt}%
\pgfpathmoveto{\pgfqpoint{3.238698in}{2.637558in}}%
\pgfpathcurveto{\pgfqpoint{3.246935in}{2.637558in}}{\pgfqpoint{3.254835in}{2.640831in}}{\pgfqpoint{3.260659in}{2.646654in}}%
\pgfpathcurveto{\pgfqpoint{3.266482in}{2.652478in}}{\pgfqpoint{3.269755in}{2.660378in}}{\pgfqpoint{3.269755in}{2.668615in}}%
\pgfpathcurveto{\pgfqpoint{3.269755in}{2.676851in}}{\pgfqpoint{3.266482in}{2.684751in}}{\pgfqpoint{3.260659in}{2.690575in}}%
\pgfpathcurveto{\pgfqpoint{3.254835in}{2.696399in}}{\pgfqpoint{3.246935in}{2.699671in}}{\pgfqpoint{3.238698in}{2.699671in}}%
\pgfpathcurveto{\pgfqpoint{3.230462in}{2.699671in}}{\pgfqpoint{3.222562in}{2.696399in}}{\pgfqpoint{3.216738in}{2.690575in}}%
\pgfpathcurveto{\pgfqpoint{3.210914in}{2.684751in}}{\pgfqpoint{3.207642in}{2.676851in}}{\pgfqpoint{3.207642in}{2.668615in}}%
\pgfpathcurveto{\pgfqpoint{3.207642in}{2.660378in}}{\pgfqpoint{3.210914in}{2.652478in}}{\pgfqpoint{3.216738in}{2.646654in}}%
\pgfpathcurveto{\pgfqpoint{3.222562in}{2.640831in}}{\pgfqpoint{3.230462in}{2.637558in}}{\pgfqpoint{3.238698in}{2.637558in}}%
\pgfpathclose%
\pgfusepath{stroke,fill}%
\end{pgfscope}%
\begin{pgfscope}%
\pgfpathrectangle{\pgfqpoint{0.100000in}{0.220728in}}{\pgfqpoint{3.696000in}{3.696000in}}%
\pgfusepath{clip}%
\pgfsetbuttcap%
\pgfsetroundjoin%
\definecolor{currentfill}{rgb}{0.121569,0.466667,0.705882}%
\pgfsetfillcolor{currentfill}%
\pgfsetfillopacity{0.723010}%
\pgfsetlinewidth{1.003750pt}%
\definecolor{currentstroke}{rgb}{0.121569,0.466667,0.705882}%
\pgfsetstrokecolor{currentstroke}%
\pgfsetstrokeopacity{0.723010}%
\pgfsetdash{}{0pt}%
\pgfpathmoveto{\pgfqpoint{0.944952in}{1.293166in}}%
\pgfpathcurveto{\pgfqpoint{0.953188in}{1.293166in}}{\pgfqpoint{0.961088in}{1.296438in}}{\pgfqpoint{0.966912in}{1.302262in}}%
\pgfpathcurveto{\pgfqpoint{0.972736in}{1.308086in}}{\pgfqpoint{0.976008in}{1.315986in}}{\pgfqpoint{0.976008in}{1.324223in}}%
\pgfpathcurveto{\pgfqpoint{0.976008in}{1.332459in}}{\pgfqpoint{0.972736in}{1.340359in}}{\pgfqpoint{0.966912in}{1.346183in}}%
\pgfpathcurveto{\pgfqpoint{0.961088in}{1.352007in}}{\pgfqpoint{0.953188in}{1.355279in}}{\pgfqpoint{0.944952in}{1.355279in}}%
\pgfpathcurveto{\pgfqpoint{0.936715in}{1.355279in}}{\pgfqpoint{0.928815in}{1.352007in}}{\pgfqpoint{0.922991in}{1.346183in}}%
\pgfpathcurveto{\pgfqpoint{0.917167in}{1.340359in}}{\pgfqpoint{0.913895in}{1.332459in}}{\pgfqpoint{0.913895in}{1.324223in}}%
\pgfpathcurveto{\pgfqpoint{0.913895in}{1.315986in}}{\pgfqpoint{0.917167in}{1.308086in}}{\pgfqpoint{0.922991in}{1.302262in}}%
\pgfpathcurveto{\pgfqpoint{0.928815in}{1.296438in}}{\pgfqpoint{0.936715in}{1.293166in}}{\pgfqpoint{0.944952in}{1.293166in}}%
\pgfpathclose%
\pgfusepath{stroke,fill}%
\end{pgfscope}%
\begin{pgfscope}%
\pgfpathrectangle{\pgfqpoint{0.100000in}{0.220728in}}{\pgfqpoint{3.696000in}{3.696000in}}%
\pgfusepath{clip}%
\pgfsetbuttcap%
\pgfsetroundjoin%
\definecolor{currentfill}{rgb}{0.121569,0.466667,0.705882}%
\pgfsetfillcolor{currentfill}%
\pgfsetfillopacity{0.723068}%
\pgfsetlinewidth{1.003750pt}%
\definecolor{currentstroke}{rgb}{0.121569,0.466667,0.705882}%
\pgfsetstrokecolor{currentstroke}%
\pgfsetstrokeopacity{0.723068}%
\pgfsetdash{}{0pt}%
\pgfpathmoveto{\pgfqpoint{3.236682in}{2.635131in}}%
\pgfpathcurveto{\pgfqpoint{3.244918in}{2.635131in}}{\pgfqpoint{3.252819in}{2.638404in}}{\pgfqpoint{3.258642in}{2.644228in}}%
\pgfpathcurveto{\pgfqpoint{3.264466in}{2.650052in}}{\pgfqpoint{3.267739in}{2.657952in}}{\pgfqpoint{3.267739in}{2.666188in}}%
\pgfpathcurveto{\pgfqpoint{3.267739in}{2.674424in}}{\pgfqpoint{3.264466in}{2.682324in}}{\pgfqpoint{3.258642in}{2.688148in}}%
\pgfpathcurveto{\pgfqpoint{3.252819in}{2.693972in}}{\pgfqpoint{3.244918in}{2.697244in}}{\pgfqpoint{3.236682in}{2.697244in}}%
\pgfpathcurveto{\pgfqpoint{3.228446in}{2.697244in}}{\pgfqpoint{3.220546in}{2.693972in}}{\pgfqpoint{3.214722in}{2.688148in}}%
\pgfpathcurveto{\pgfqpoint{3.208898in}{2.682324in}}{\pgfqpoint{3.205626in}{2.674424in}}{\pgfqpoint{3.205626in}{2.666188in}}%
\pgfpathcurveto{\pgfqpoint{3.205626in}{2.657952in}}{\pgfqpoint{3.208898in}{2.650052in}}{\pgfqpoint{3.214722in}{2.644228in}}%
\pgfpathcurveto{\pgfqpoint{3.220546in}{2.638404in}}{\pgfqpoint{3.228446in}{2.635131in}}{\pgfqpoint{3.236682in}{2.635131in}}%
\pgfpathclose%
\pgfusepath{stroke,fill}%
\end{pgfscope}%
\begin{pgfscope}%
\pgfpathrectangle{\pgfqpoint{0.100000in}{0.220728in}}{\pgfqpoint{3.696000in}{3.696000in}}%
\pgfusepath{clip}%
\pgfsetbuttcap%
\pgfsetroundjoin%
\definecolor{currentfill}{rgb}{0.121569,0.466667,0.705882}%
\pgfsetfillcolor{currentfill}%
\pgfsetfillopacity{0.723327}%
\pgfsetlinewidth{1.003750pt}%
\definecolor{currentstroke}{rgb}{0.121569,0.466667,0.705882}%
\pgfsetstrokecolor{currentstroke}%
\pgfsetstrokeopacity{0.723327}%
\pgfsetdash{}{0pt}%
\pgfpathmoveto{\pgfqpoint{3.235763in}{2.633024in}}%
\pgfpathcurveto{\pgfqpoint{3.244000in}{2.633024in}}{\pgfqpoint{3.251900in}{2.636297in}}{\pgfqpoint{3.257724in}{2.642121in}}%
\pgfpathcurveto{\pgfqpoint{3.263547in}{2.647944in}}{\pgfqpoint{3.266820in}{2.655844in}}{\pgfqpoint{3.266820in}{2.664081in}}%
\pgfpathcurveto{\pgfqpoint{3.266820in}{2.672317in}}{\pgfqpoint{3.263547in}{2.680217in}}{\pgfqpoint{3.257724in}{2.686041in}}%
\pgfpathcurveto{\pgfqpoint{3.251900in}{2.691865in}}{\pgfqpoint{3.244000in}{2.695137in}}{\pgfqpoint{3.235763in}{2.695137in}}%
\pgfpathcurveto{\pgfqpoint{3.227527in}{2.695137in}}{\pgfqpoint{3.219627in}{2.691865in}}{\pgfqpoint{3.213803in}{2.686041in}}%
\pgfpathcurveto{\pgfqpoint{3.207979in}{2.680217in}}{\pgfqpoint{3.204707in}{2.672317in}}{\pgfqpoint{3.204707in}{2.664081in}}%
\pgfpathcurveto{\pgfqpoint{3.204707in}{2.655844in}}{\pgfqpoint{3.207979in}{2.647944in}}{\pgfqpoint{3.213803in}{2.642121in}}%
\pgfpathcurveto{\pgfqpoint{3.219627in}{2.636297in}}{\pgfqpoint{3.227527in}{2.633024in}}{\pgfqpoint{3.235763in}{2.633024in}}%
\pgfpathclose%
\pgfusepath{stroke,fill}%
\end{pgfscope}%
\begin{pgfscope}%
\pgfpathrectangle{\pgfqpoint{0.100000in}{0.220728in}}{\pgfqpoint{3.696000in}{3.696000in}}%
\pgfusepath{clip}%
\pgfsetbuttcap%
\pgfsetroundjoin%
\definecolor{currentfill}{rgb}{0.121569,0.466667,0.705882}%
\pgfsetfillcolor{currentfill}%
\pgfsetfillopacity{0.723830}%
\pgfsetlinewidth{1.003750pt}%
\definecolor{currentstroke}{rgb}{0.121569,0.466667,0.705882}%
\pgfsetstrokecolor{currentstroke}%
\pgfsetstrokeopacity{0.723830}%
\pgfsetdash{}{0pt}%
\pgfpathmoveto{\pgfqpoint{3.232125in}{2.628455in}}%
\pgfpathcurveto{\pgfqpoint{3.240361in}{2.628455in}}{\pgfqpoint{3.248261in}{2.631727in}}{\pgfqpoint{3.254085in}{2.637551in}}%
\pgfpathcurveto{\pgfqpoint{3.259909in}{2.643375in}}{\pgfqpoint{3.263182in}{2.651275in}}{\pgfqpoint{3.263182in}{2.659511in}}%
\pgfpathcurveto{\pgfqpoint{3.263182in}{2.667747in}}{\pgfqpoint{3.259909in}{2.675647in}}{\pgfqpoint{3.254085in}{2.681471in}}%
\pgfpathcurveto{\pgfqpoint{3.248261in}{2.687295in}}{\pgfqpoint{3.240361in}{2.690568in}}{\pgfqpoint{3.232125in}{2.690568in}}%
\pgfpathcurveto{\pgfqpoint{3.223889in}{2.690568in}}{\pgfqpoint{3.215989in}{2.687295in}}{\pgfqpoint{3.210165in}{2.681471in}}%
\pgfpathcurveto{\pgfqpoint{3.204341in}{2.675647in}}{\pgfqpoint{3.201069in}{2.667747in}}{\pgfqpoint{3.201069in}{2.659511in}}%
\pgfpathcurveto{\pgfqpoint{3.201069in}{2.651275in}}{\pgfqpoint{3.204341in}{2.643375in}}{\pgfqpoint{3.210165in}{2.637551in}}%
\pgfpathcurveto{\pgfqpoint{3.215989in}{2.631727in}}{\pgfqpoint{3.223889in}{2.628455in}}{\pgfqpoint{3.232125in}{2.628455in}}%
\pgfpathclose%
\pgfusepath{stroke,fill}%
\end{pgfscope}%
\begin{pgfscope}%
\pgfpathrectangle{\pgfqpoint{0.100000in}{0.220728in}}{\pgfqpoint{3.696000in}{3.696000in}}%
\pgfusepath{clip}%
\pgfsetbuttcap%
\pgfsetroundjoin%
\definecolor{currentfill}{rgb}{0.121569,0.466667,0.705882}%
\pgfsetfillcolor{currentfill}%
\pgfsetfillopacity{0.724916}%
\pgfsetlinewidth{1.003750pt}%
\definecolor{currentstroke}{rgb}{0.121569,0.466667,0.705882}%
\pgfsetstrokecolor{currentstroke}%
\pgfsetstrokeopacity{0.724916}%
\pgfsetdash{}{0pt}%
\pgfpathmoveto{\pgfqpoint{3.229157in}{2.622935in}}%
\pgfpathcurveto{\pgfqpoint{3.237394in}{2.622935in}}{\pgfqpoint{3.245294in}{2.626207in}}{\pgfqpoint{3.251118in}{2.632031in}}%
\pgfpathcurveto{\pgfqpoint{3.256942in}{2.637855in}}{\pgfqpoint{3.260214in}{2.645755in}}{\pgfqpoint{3.260214in}{2.653991in}}%
\pgfpathcurveto{\pgfqpoint{3.260214in}{2.662227in}}{\pgfqpoint{3.256942in}{2.670128in}}{\pgfqpoint{3.251118in}{2.675951in}}%
\pgfpathcurveto{\pgfqpoint{3.245294in}{2.681775in}}{\pgfqpoint{3.237394in}{2.685048in}}{\pgfqpoint{3.229157in}{2.685048in}}%
\pgfpathcurveto{\pgfqpoint{3.220921in}{2.685048in}}{\pgfqpoint{3.213021in}{2.681775in}}{\pgfqpoint{3.207197in}{2.675951in}}%
\pgfpathcurveto{\pgfqpoint{3.201373in}{2.670128in}}{\pgfqpoint{3.198101in}{2.662227in}}{\pgfqpoint{3.198101in}{2.653991in}}%
\pgfpathcurveto{\pgfqpoint{3.198101in}{2.645755in}}{\pgfqpoint{3.201373in}{2.637855in}}{\pgfqpoint{3.207197in}{2.632031in}}%
\pgfpathcurveto{\pgfqpoint{3.213021in}{2.626207in}}{\pgfqpoint{3.220921in}{2.622935in}}{\pgfqpoint{3.229157in}{2.622935in}}%
\pgfpathclose%
\pgfusepath{stroke,fill}%
\end{pgfscope}%
\begin{pgfscope}%
\pgfpathrectangle{\pgfqpoint{0.100000in}{0.220728in}}{\pgfqpoint{3.696000in}{3.696000in}}%
\pgfusepath{clip}%
\pgfsetbuttcap%
\pgfsetroundjoin%
\definecolor{currentfill}{rgb}{0.121569,0.466667,0.705882}%
\pgfsetfillcolor{currentfill}%
\pgfsetfillopacity{0.725470}%
\pgfsetlinewidth{1.003750pt}%
\definecolor{currentstroke}{rgb}{0.121569,0.466667,0.705882}%
\pgfsetstrokecolor{currentstroke}%
\pgfsetstrokeopacity{0.725470}%
\pgfsetdash{}{0pt}%
\pgfpathmoveto{\pgfqpoint{3.227585in}{2.619633in}}%
\pgfpathcurveto{\pgfqpoint{3.235821in}{2.619633in}}{\pgfqpoint{3.243721in}{2.622906in}}{\pgfqpoint{3.249545in}{2.628730in}}%
\pgfpathcurveto{\pgfqpoint{3.255369in}{2.634554in}}{\pgfqpoint{3.258641in}{2.642454in}}{\pgfqpoint{3.258641in}{2.650690in}}%
\pgfpathcurveto{\pgfqpoint{3.258641in}{2.658926in}}{\pgfqpoint{3.255369in}{2.666826in}}{\pgfqpoint{3.249545in}{2.672650in}}%
\pgfpathcurveto{\pgfqpoint{3.243721in}{2.678474in}}{\pgfqpoint{3.235821in}{2.681746in}}{\pgfqpoint{3.227585in}{2.681746in}}%
\pgfpathcurveto{\pgfqpoint{3.219348in}{2.681746in}}{\pgfqpoint{3.211448in}{2.678474in}}{\pgfqpoint{3.205624in}{2.672650in}}%
\pgfpathcurveto{\pgfqpoint{3.199800in}{2.666826in}}{\pgfqpoint{3.196528in}{2.658926in}}{\pgfqpoint{3.196528in}{2.650690in}}%
\pgfpathcurveto{\pgfqpoint{3.196528in}{2.642454in}}{\pgfqpoint{3.199800in}{2.634554in}}{\pgfqpoint{3.205624in}{2.628730in}}%
\pgfpathcurveto{\pgfqpoint{3.211448in}{2.622906in}}{\pgfqpoint{3.219348in}{2.619633in}}{\pgfqpoint{3.227585in}{2.619633in}}%
\pgfpathclose%
\pgfusepath{stroke,fill}%
\end{pgfscope}%
\begin{pgfscope}%
\pgfpathrectangle{\pgfqpoint{0.100000in}{0.220728in}}{\pgfqpoint{3.696000in}{3.696000in}}%
\pgfusepath{clip}%
\pgfsetbuttcap%
\pgfsetroundjoin%
\definecolor{currentfill}{rgb}{0.121569,0.466667,0.705882}%
\pgfsetfillcolor{currentfill}%
\pgfsetfillopacity{0.726144}%
\pgfsetlinewidth{1.003750pt}%
\definecolor{currentstroke}{rgb}{0.121569,0.466667,0.705882}%
\pgfsetstrokecolor{currentstroke}%
\pgfsetstrokeopacity{0.726144}%
\pgfsetdash{}{0pt}%
\pgfpathmoveto{\pgfqpoint{3.224631in}{2.615997in}}%
\pgfpathcurveto{\pgfqpoint{3.232867in}{2.615997in}}{\pgfqpoint{3.240767in}{2.619269in}}{\pgfqpoint{3.246591in}{2.625093in}}%
\pgfpathcurveto{\pgfqpoint{3.252415in}{2.630917in}}{\pgfqpoint{3.255687in}{2.638817in}}{\pgfqpoint{3.255687in}{2.647054in}}%
\pgfpathcurveto{\pgfqpoint{3.255687in}{2.655290in}}{\pgfqpoint{3.252415in}{2.663190in}}{\pgfqpoint{3.246591in}{2.669014in}}%
\pgfpathcurveto{\pgfqpoint{3.240767in}{2.674838in}}{\pgfqpoint{3.232867in}{2.678110in}}{\pgfqpoint{3.224631in}{2.678110in}}%
\pgfpathcurveto{\pgfqpoint{3.216395in}{2.678110in}}{\pgfqpoint{3.208495in}{2.674838in}}{\pgfqpoint{3.202671in}{2.669014in}}%
\pgfpathcurveto{\pgfqpoint{3.196847in}{2.663190in}}{\pgfqpoint{3.193574in}{2.655290in}}{\pgfqpoint{3.193574in}{2.647054in}}%
\pgfpathcurveto{\pgfqpoint{3.193574in}{2.638817in}}{\pgfqpoint{3.196847in}{2.630917in}}{\pgfqpoint{3.202671in}{2.625093in}}%
\pgfpathcurveto{\pgfqpoint{3.208495in}{2.619269in}}{\pgfqpoint{3.216395in}{2.615997in}}{\pgfqpoint{3.224631in}{2.615997in}}%
\pgfpathclose%
\pgfusepath{stroke,fill}%
\end{pgfscope}%
\begin{pgfscope}%
\pgfpathrectangle{\pgfqpoint{0.100000in}{0.220728in}}{\pgfqpoint{3.696000in}{3.696000in}}%
\pgfusepath{clip}%
\pgfsetbuttcap%
\pgfsetroundjoin%
\definecolor{currentfill}{rgb}{0.121569,0.466667,0.705882}%
\pgfsetfillcolor{currentfill}%
\pgfsetfillopacity{0.726632}%
\pgfsetlinewidth{1.003750pt}%
\definecolor{currentstroke}{rgb}{0.121569,0.466667,0.705882}%
\pgfsetstrokecolor{currentstroke}%
\pgfsetstrokeopacity{0.726632}%
\pgfsetdash{}{0pt}%
\pgfpathmoveto{\pgfqpoint{0.963341in}{1.281569in}}%
\pgfpathcurveto{\pgfqpoint{0.971577in}{1.281569in}}{\pgfqpoint{0.979478in}{1.284841in}}{\pgfqpoint{0.985301in}{1.290665in}}%
\pgfpathcurveto{\pgfqpoint{0.991125in}{1.296489in}}{\pgfqpoint{0.994398in}{1.304389in}}{\pgfqpoint{0.994398in}{1.312625in}}%
\pgfpathcurveto{\pgfqpoint{0.994398in}{1.320861in}}{\pgfqpoint{0.991125in}{1.328762in}}{\pgfqpoint{0.985301in}{1.334585in}}%
\pgfpathcurveto{\pgfqpoint{0.979478in}{1.340409in}}{\pgfqpoint{0.971577in}{1.343682in}}{\pgfqpoint{0.963341in}{1.343682in}}%
\pgfpathcurveto{\pgfqpoint{0.955105in}{1.343682in}}{\pgfqpoint{0.947205in}{1.340409in}}{\pgfqpoint{0.941381in}{1.334585in}}%
\pgfpathcurveto{\pgfqpoint{0.935557in}{1.328762in}}{\pgfqpoint{0.932285in}{1.320861in}}{\pgfqpoint{0.932285in}{1.312625in}}%
\pgfpathcurveto{\pgfqpoint{0.932285in}{1.304389in}}{\pgfqpoint{0.935557in}{1.296489in}}{\pgfqpoint{0.941381in}{1.290665in}}%
\pgfpathcurveto{\pgfqpoint{0.947205in}{1.284841in}}{\pgfqpoint{0.955105in}{1.281569in}}{\pgfqpoint{0.963341in}{1.281569in}}%
\pgfpathclose%
\pgfusepath{stroke,fill}%
\end{pgfscope}%
\begin{pgfscope}%
\pgfpathrectangle{\pgfqpoint{0.100000in}{0.220728in}}{\pgfqpoint{3.696000in}{3.696000in}}%
\pgfusepath{clip}%
\pgfsetbuttcap%
\pgfsetroundjoin%
\definecolor{currentfill}{rgb}{0.121569,0.466667,0.705882}%
\pgfsetfillcolor{currentfill}%
\pgfsetfillopacity{0.727134}%
\pgfsetlinewidth{1.003750pt}%
\definecolor{currentstroke}{rgb}{0.121569,0.466667,0.705882}%
\pgfsetstrokecolor{currentstroke}%
\pgfsetstrokeopacity{0.727134}%
\pgfsetdash{}{0pt}%
\pgfpathmoveto{\pgfqpoint{3.221480in}{2.607245in}}%
\pgfpathcurveto{\pgfqpoint{3.229716in}{2.607245in}}{\pgfqpoint{3.237616in}{2.610517in}}{\pgfqpoint{3.243440in}{2.616341in}}%
\pgfpathcurveto{\pgfqpoint{3.249264in}{2.622165in}}{\pgfqpoint{3.252536in}{2.630065in}}{\pgfqpoint{3.252536in}{2.638302in}}%
\pgfpathcurveto{\pgfqpoint{3.252536in}{2.646538in}}{\pgfqpoint{3.249264in}{2.654438in}}{\pgfqpoint{3.243440in}{2.660262in}}%
\pgfpathcurveto{\pgfqpoint{3.237616in}{2.666086in}}{\pgfqpoint{3.229716in}{2.669358in}}{\pgfqpoint{3.221480in}{2.669358in}}%
\pgfpathcurveto{\pgfqpoint{3.213243in}{2.669358in}}{\pgfqpoint{3.205343in}{2.666086in}}{\pgfqpoint{3.199519in}{2.660262in}}%
\pgfpathcurveto{\pgfqpoint{3.193695in}{2.654438in}}{\pgfqpoint{3.190423in}{2.646538in}}{\pgfqpoint{3.190423in}{2.638302in}}%
\pgfpathcurveto{\pgfqpoint{3.190423in}{2.630065in}}{\pgfqpoint{3.193695in}{2.622165in}}{\pgfqpoint{3.199519in}{2.616341in}}%
\pgfpathcurveto{\pgfqpoint{3.205343in}{2.610517in}}{\pgfqpoint{3.213243in}{2.607245in}}{\pgfqpoint{3.221480in}{2.607245in}}%
\pgfpathclose%
\pgfusepath{stroke,fill}%
\end{pgfscope}%
\begin{pgfscope}%
\pgfpathrectangle{\pgfqpoint{0.100000in}{0.220728in}}{\pgfqpoint{3.696000in}{3.696000in}}%
\pgfusepath{clip}%
\pgfsetbuttcap%
\pgfsetroundjoin%
\definecolor{currentfill}{rgb}{0.121569,0.466667,0.705882}%
\pgfsetfillcolor{currentfill}%
\pgfsetfillopacity{0.727938}%
\pgfsetlinewidth{1.003750pt}%
\definecolor{currentstroke}{rgb}{0.121569,0.466667,0.705882}%
\pgfsetstrokecolor{currentstroke}%
\pgfsetstrokeopacity{0.727938}%
\pgfsetdash{}{0pt}%
\pgfpathmoveto{\pgfqpoint{3.215183in}{2.598549in}}%
\pgfpathcurveto{\pgfqpoint{3.223419in}{2.598549in}}{\pgfqpoint{3.231319in}{2.601822in}}{\pgfqpoint{3.237143in}{2.607646in}}%
\pgfpathcurveto{\pgfqpoint{3.242967in}{2.613470in}}{\pgfqpoint{3.246239in}{2.621370in}}{\pgfqpoint{3.246239in}{2.629606in}}%
\pgfpathcurveto{\pgfqpoint{3.246239in}{2.637842in}}{\pgfqpoint{3.242967in}{2.645742in}}{\pgfqpoint{3.237143in}{2.651566in}}%
\pgfpathcurveto{\pgfqpoint{3.231319in}{2.657390in}}{\pgfqpoint{3.223419in}{2.660662in}}{\pgfqpoint{3.215183in}{2.660662in}}%
\pgfpathcurveto{\pgfqpoint{3.206947in}{2.660662in}}{\pgfqpoint{3.199047in}{2.657390in}}{\pgfqpoint{3.193223in}{2.651566in}}%
\pgfpathcurveto{\pgfqpoint{3.187399in}{2.645742in}}{\pgfqpoint{3.184126in}{2.637842in}}{\pgfqpoint{3.184126in}{2.629606in}}%
\pgfpathcurveto{\pgfqpoint{3.184126in}{2.621370in}}{\pgfqpoint{3.187399in}{2.613470in}}{\pgfqpoint{3.193223in}{2.607646in}}%
\pgfpathcurveto{\pgfqpoint{3.199047in}{2.601822in}}{\pgfqpoint{3.206947in}{2.598549in}}{\pgfqpoint{3.215183in}{2.598549in}}%
\pgfpathclose%
\pgfusepath{stroke,fill}%
\end{pgfscope}%
\begin{pgfscope}%
\pgfpathrectangle{\pgfqpoint{0.100000in}{0.220728in}}{\pgfqpoint{3.696000in}{3.696000in}}%
\pgfusepath{clip}%
\pgfsetbuttcap%
\pgfsetroundjoin%
\definecolor{currentfill}{rgb}{0.121569,0.466667,0.705882}%
\pgfsetfillcolor{currentfill}%
\pgfsetfillopacity{0.729514}%
\pgfsetlinewidth{1.003750pt}%
\definecolor{currentstroke}{rgb}{0.121569,0.466667,0.705882}%
\pgfsetstrokecolor{currentstroke}%
\pgfsetstrokeopacity{0.729514}%
\pgfsetdash{}{0pt}%
\pgfpathmoveto{\pgfqpoint{0.978419in}{1.275471in}}%
\pgfpathcurveto{\pgfqpoint{0.986656in}{1.275471in}}{\pgfqpoint{0.994556in}{1.278743in}}{\pgfqpoint{1.000380in}{1.284567in}}%
\pgfpathcurveto{\pgfqpoint{1.006204in}{1.290391in}}{\pgfqpoint{1.009476in}{1.298291in}}{\pgfqpoint{1.009476in}{1.306527in}}%
\pgfpathcurveto{\pgfqpoint{1.009476in}{1.314763in}}{\pgfqpoint{1.006204in}{1.322663in}}{\pgfqpoint{1.000380in}{1.328487in}}%
\pgfpathcurveto{\pgfqpoint{0.994556in}{1.334311in}}{\pgfqpoint{0.986656in}{1.337584in}}{\pgfqpoint{0.978419in}{1.337584in}}%
\pgfpathcurveto{\pgfqpoint{0.970183in}{1.337584in}}{\pgfqpoint{0.962283in}{1.334311in}}{\pgfqpoint{0.956459in}{1.328487in}}%
\pgfpathcurveto{\pgfqpoint{0.950635in}{1.322663in}}{\pgfqpoint{0.947363in}{1.314763in}}{\pgfqpoint{0.947363in}{1.306527in}}%
\pgfpathcurveto{\pgfqpoint{0.947363in}{1.298291in}}{\pgfqpoint{0.950635in}{1.290391in}}{\pgfqpoint{0.956459in}{1.284567in}}%
\pgfpathcurveto{\pgfqpoint{0.962283in}{1.278743in}}{\pgfqpoint{0.970183in}{1.275471in}}{\pgfqpoint{0.978419in}{1.275471in}}%
\pgfpathclose%
\pgfusepath{stroke,fill}%
\end{pgfscope}%
\begin{pgfscope}%
\pgfpathrectangle{\pgfqpoint{0.100000in}{0.220728in}}{\pgfqpoint{3.696000in}{3.696000in}}%
\pgfusepath{clip}%
\pgfsetbuttcap%
\pgfsetroundjoin%
\definecolor{currentfill}{rgb}{0.121569,0.466667,0.705882}%
\pgfsetfillcolor{currentfill}%
\pgfsetfillopacity{0.729693}%
\pgfsetlinewidth{1.003750pt}%
\definecolor{currentstroke}{rgb}{0.121569,0.466667,0.705882}%
\pgfsetstrokecolor{currentstroke}%
\pgfsetstrokeopacity{0.729693}%
\pgfsetdash{}{0pt}%
\pgfpathmoveto{\pgfqpoint{3.209723in}{2.590062in}}%
\pgfpathcurveto{\pgfqpoint{3.217960in}{2.590062in}}{\pgfqpoint{3.225860in}{2.593334in}}{\pgfqpoint{3.231684in}{2.599158in}}%
\pgfpathcurveto{\pgfqpoint{3.237508in}{2.604982in}}{\pgfqpoint{3.240780in}{2.612882in}}{\pgfqpoint{3.240780in}{2.621119in}}%
\pgfpathcurveto{\pgfqpoint{3.240780in}{2.629355in}}{\pgfqpoint{3.237508in}{2.637255in}}{\pgfqpoint{3.231684in}{2.643079in}}%
\pgfpathcurveto{\pgfqpoint{3.225860in}{2.648903in}}{\pgfqpoint{3.217960in}{2.652175in}}{\pgfqpoint{3.209723in}{2.652175in}}%
\pgfpathcurveto{\pgfqpoint{3.201487in}{2.652175in}}{\pgfqpoint{3.193587in}{2.648903in}}{\pgfqpoint{3.187763in}{2.643079in}}%
\pgfpathcurveto{\pgfqpoint{3.181939in}{2.637255in}}{\pgfqpoint{3.178667in}{2.629355in}}{\pgfqpoint{3.178667in}{2.621119in}}%
\pgfpathcurveto{\pgfqpoint{3.178667in}{2.612882in}}{\pgfqpoint{3.181939in}{2.604982in}}{\pgfqpoint{3.187763in}{2.599158in}}%
\pgfpathcurveto{\pgfqpoint{3.193587in}{2.593334in}}{\pgfqpoint{3.201487in}{2.590062in}}{\pgfqpoint{3.209723in}{2.590062in}}%
\pgfpathclose%
\pgfusepath{stroke,fill}%
\end{pgfscope}%
\begin{pgfscope}%
\pgfpathrectangle{\pgfqpoint{0.100000in}{0.220728in}}{\pgfqpoint{3.696000in}{3.696000in}}%
\pgfusepath{clip}%
\pgfsetbuttcap%
\pgfsetroundjoin%
\definecolor{currentfill}{rgb}{0.121569,0.466667,0.705882}%
\pgfsetfillcolor{currentfill}%
\pgfsetfillopacity{0.730598}%
\pgfsetlinewidth{1.003750pt}%
\definecolor{currentstroke}{rgb}{0.121569,0.466667,0.705882}%
\pgfsetstrokecolor{currentstroke}%
\pgfsetstrokeopacity{0.730598}%
\pgfsetdash{}{0pt}%
\pgfpathmoveto{\pgfqpoint{3.207330in}{2.584355in}}%
\pgfpathcurveto{\pgfqpoint{3.215566in}{2.584355in}}{\pgfqpoint{3.223466in}{2.587627in}}{\pgfqpoint{3.229290in}{2.593451in}}%
\pgfpathcurveto{\pgfqpoint{3.235114in}{2.599275in}}{\pgfqpoint{3.238386in}{2.607175in}}{\pgfqpoint{3.238386in}{2.615411in}}%
\pgfpathcurveto{\pgfqpoint{3.238386in}{2.623647in}}{\pgfqpoint{3.235114in}{2.631548in}}{\pgfqpoint{3.229290in}{2.637371in}}%
\pgfpathcurveto{\pgfqpoint{3.223466in}{2.643195in}}{\pgfqpoint{3.215566in}{2.646468in}}{\pgfqpoint{3.207330in}{2.646468in}}%
\pgfpathcurveto{\pgfqpoint{3.199093in}{2.646468in}}{\pgfqpoint{3.191193in}{2.643195in}}{\pgfqpoint{3.185369in}{2.637371in}}%
\pgfpathcurveto{\pgfqpoint{3.179545in}{2.631548in}}{\pgfqpoint{3.176273in}{2.623647in}}{\pgfqpoint{3.176273in}{2.615411in}}%
\pgfpathcurveto{\pgfqpoint{3.176273in}{2.607175in}}{\pgfqpoint{3.179545in}{2.599275in}}{\pgfqpoint{3.185369in}{2.593451in}}%
\pgfpathcurveto{\pgfqpoint{3.191193in}{2.587627in}}{\pgfqpoint{3.199093in}{2.584355in}}{\pgfqpoint{3.207330in}{2.584355in}}%
\pgfpathclose%
\pgfusepath{stroke,fill}%
\end{pgfscope}%
\begin{pgfscope}%
\pgfpathrectangle{\pgfqpoint{0.100000in}{0.220728in}}{\pgfqpoint{3.696000in}{3.696000in}}%
\pgfusepath{clip}%
\pgfsetbuttcap%
\pgfsetroundjoin%
\definecolor{currentfill}{rgb}{0.121569,0.466667,0.705882}%
\pgfsetfillcolor{currentfill}%
\pgfsetfillopacity{0.731469}%
\pgfsetlinewidth{1.003750pt}%
\definecolor{currentstroke}{rgb}{0.121569,0.466667,0.705882}%
\pgfsetstrokecolor{currentstroke}%
\pgfsetstrokeopacity{0.731469}%
\pgfsetdash{}{0pt}%
\pgfpathmoveto{\pgfqpoint{3.201284in}{2.576670in}}%
\pgfpathcurveto{\pgfqpoint{3.209520in}{2.576670in}}{\pgfqpoint{3.217420in}{2.579942in}}{\pgfqpoint{3.223244in}{2.585766in}}%
\pgfpathcurveto{\pgfqpoint{3.229068in}{2.591590in}}{\pgfqpoint{3.232341in}{2.599490in}}{\pgfqpoint{3.232341in}{2.607727in}}%
\pgfpathcurveto{\pgfqpoint{3.232341in}{2.615963in}}{\pgfqpoint{3.229068in}{2.623863in}}{\pgfqpoint{3.223244in}{2.629687in}}%
\pgfpathcurveto{\pgfqpoint{3.217420in}{2.635511in}}{\pgfqpoint{3.209520in}{2.638783in}}{\pgfqpoint{3.201284in}{2.638783in}}%
\pgfpathcurveto{\pgfqpoint{3.193048in}{2.638783in}}{\pgfqpoint{3.185148in}{2.635511in}}{\pgfqpoint{3.179324in}{2.629687in}}%
\pgfpathcurveto{\pgfqpoint{3.173500in}{2.623863in}}{\pgfqpoint{3.170228in}{2.615963in}}{\pgfqpoint{3.170228in}{2.607727in}}%
\pgfpathcurveto{\pgfqpoint{3.170228in}{2.599490in}}{\pgfqpoint{3.173500in}{2.591590in}}{\pgfqpoint{3.179324in}{2.585766in}}%
\pgfpathcurveto{\pgfqpoint{3.185148in}{2.579942in}}{\pgfqpoint{3.193048in}{2.576670in}}{\pgfqpoint{3.201284in}{2.576670in}}%
\pgfpathclose%
\pgfusepath{stroke,fill}%
\end{pgfscope}%
\begin{pgfscope}%
\pgfpathrectangle{\pgfqpoint{0.100000in}{0.220728in}}{\pgfqpoint{3.696000in}{3.696000in}}%
\pgfusepath{clip}%
\pgfsetbuttcap%
\pgfsetroundjoin%
\definecolor{currentfill}{rgb}{0.121569,0.466667,0.705882}%
\pgfsetfillcolor{currentfill}%
\pgfsetfillopacity{0.732592}%
\pgfsetlinewidth{1.003750pt}%
\definecolor{currentstroke}{rgb}{0.121569,0.466667,0.705882}%
\pgfsetstrokecolor{currentstroke}%
\pgfsetstrokeopacity{0.732592}%
\pgfsetdash{}{0pt}%
\pgfpathmoveto{\pgfqpoint{0.991361in}{1.268050in}}%
\pgfpathcurveto{\pgfqpoint{0.999598in}{1.268050in}}{\pgfqpoint{1.007498in}{1.271323in}}{\pgfqpoint{1.013322in}{1.277146in}}%
\pgfpathcurveto{\pgfqpoint{1.019145in}{1.282970in}}{\pgfqpoint{1.022418in}{1.290870in}}{\pgfqpoint{1.022418in}{1.299107in}}%
\pgfpathcurveto{\pgfqpoint{1.022418in}{1.307343in}}{\pgfqpoint{1.019145in}{1.315243in}}{\pgfqpoint{1.013322in}{1.321067in}}%
\pgfpathcurveto{\pgfqpoint{1.007498in}{1.326891in}}{\pgfqpoint{0.999598in}{1.330163in}}{\pgfqpoint{0.991361in}{1.330163in}}%
\pgfpathcurveto{\pgfqpoint{0.983125in}{1.330163in}}{\pgfqpoint{0.975225in}{1.326891in}}{\pgfqpoint{0.969401in}{1.321067in}}%
\pgfpathcurveto{\pgfqpoint{0.963577in}{1.315243in}}{\pgfqpoint{0.960305in}{1.307343in}}{\pgfqpoint{0.960305in}{1.299107in}}%
\pgfpathcurveto{\pgfqpoint{0.960305in}{1.290870in}}{\pgfqpoint{0.963577in}{1.282970in}}{\pgfqpoint{0.969401in}{1.277146in}}%
\pgfpathcurveto{\pgfqpoint{0.975225in}{1.271323in}}{\pgfqpoint{0.983125in}{1.268050in}}{\pgfqpoint{0.991361in}{1.268050in}}%
\pgfpathclose%
\pgfusepath{stroke,fill}%
\end{pgfscope}%
\begin{pgfscope}%
\pgfpathrectangle{\pgfqpoint{0.100000in}{0.220728in}}{\pgfqpoint{3.696000in}{3.696000in}}%
\pgfusepath{clip}%
\pgfsetbuttcap%
\pgfsetroundjoin%
\definecolor{currentfill}{rgb}{0.121569,0.466667,0.705882}%
\pgfsetfillcolor{currentfill}%
\pgfsetfillopacity{0.733552}%
\pgfsetlinewidth{1.003750pt}%
\definecolor{currentstroke}{rgb}{0.121569,0.466667,0.705882}%
\pgfsetstrokecolor{currentstroke}%
\pgfsetstrokeopacity{0.733552}%
\pgfsetdash{}{0pt}%
\pgfpathmoveto{\pgfqpoint{3.197052in}{2.566508in}}%
\pgfpathcurveto{\pgfqpoint{3.205288in}{2.566508in}}{\pgfqpoint{3.213188in}{2.569780in}}{\pgfqpoint{3.219012in}{2.575604in}}%
\pgfpathcurveto{\pgfqpoint{3.224836in}{2.581428in}}{\pgfqpoint{3.228108in}{2.589328in}}{\pgfqpoint{3.228108in}{2.597564in}}%
\pgfpathcurveto{\pgfqpoint{3.228108in}{2.605800in}}{\pgfqpoint{3.224836in}{2.613700in}}{\pgfqpoint{3.219012in}{2.619524in}}%
\pgfpathcurveto{\pgfqpoint{3.213188in}{2.625348in}}{\pgfqpoint{3.205288in}{2.628621in}}{\pgfqpoint{3.197052in}{2.628621in}}%
\pgfpathcurveto{\pgfqpoint{3.188816in}{2.628621in}}{\pgfqpoint{3.180916in}{2.625348in}}{\pgfqpoint{3.175092in}{2.619524in}}%
\pgfpathcurveto{\pgfqpoint{3.169268in}{2.613700in}}{\pgfqpoint{3.165995in}{2.605800in}}{\pgfqpoint{3.165995in}{2.597564in}}%
\pgfpathcurveto{\pgfqpoint{3.165995in}{2.589328in}}{\pgfqpoint{3.169268in}{2.581428in}}{\pgfqpoint{3.175092in}{2.575604in}}%
\pgfpathcurveto{\pgfqpoint{3.180916in}{2.569780in}}{\pgfqpoint{3.188816in}{2.566508in}}{\pgfqpoint{3.197052in}{2.566508in}}%
\pgfpathclose%
\pgfusepath{stroke,fill}%
\end{pgfscope}%
\begin{pgfscope}%
\pgfpathrectangle{\pgfqpoint{0.100000in}{0.220728in}}{\pgfqpoint{3.696000in}{3.696000in}}%
\pgfusepath{clip}%
\pgfsetbuttcap%
\pgfsetroundjoin%
\definecolor{currentfill}{rgb}{0.121569,0.466667,0.705882}%
\pgfsetfillcolor{currentfill}%
\pgfsetfillopacity{0.734578}%
\pgfsetlinewidth{1.003750pt}%
\definecolor{currentstroke}{rgb}{0.121569,0.466667,0.705882}%
\pgfsetstrokecolor{currentstroke}%
\pgfsetstrokeopacity{0.734578}%
\pgfsetdash{}{0pt}%
\pgfpathmoveto{\pgfqpoint{3.194284in}{2.560866in}}%
\pgfpathcurveto{\pgfqpoint{3.202520in}{2.560866in}}{\pgfqpoint{3.210420in}{2.564138in}}{\pgfqpoint{3.216244in}{2.569962in}}%
\pgfpathcurveto{\pgfqpoint{3.222068in}{2.575786in}}{\pgfqpoint{3.225341in}{2.583686in}}{\pgfqpoint{3.225341in}{2.591922in}}%
\pgfpathcurveto{\pgfqpoint{3.225341in}{2.600158in}}{\pgfqpoint{3.222068in}{2.608058in}}{\pgfqpoint{3.216244in}{2.613882in}}%
\pgfpathcurveto{\pgfqpoint{3.210420in}{2.619706in}}{\pgfqpoint{3.202520in}{2.622979in}}{\pgfqpoint{3.194284in}{2.622979in}}%
\pgfpathcurveto{\pgfqpoint{3.186048in}{2.622979in}}{\pgfqpoint{3.178148in}{2.619706in}}{\pgfqpoint{3.172324in}{2.613882in}}%
\pgfpathcurveto{\pgfqpoint{3.166500in}{2.608058in}}{\pgfqpoint{3.163228in}{2.600158in}}{\pgfqpoint{3.163228in}{2.591922in}}%
\pgfpathcurveto{\pgfqpoint{3.163228in}{2.583686in}}{\pgfqpoint{3.166500in}{2.575786in}}{\pgfqpoint{3.172324in}{2.569962in}}%
\pgfpathcurveto{\pgfqpoint{3.178148in}{2.564138in}}{\pgfqpoint{3.186048in}{2.560866in}}{\pgfqpoint{3.194284in}{2.560866in}}%
\pgfpathclose%
\pgfusepath{stroke,fill}%
\end{pgfscope}%
\begin{pgfscope}%
\pgfpathrectangle{\pgfqpoint{0.100000in}{0.220728in}}{\pgfqpoint{3.696000in}{3.696000in}}%
\pgfusepath{clip}%
\pgfsetbuttcap%
\pgfsetroundjoin%
\definecolor{currentfill}{rgb}{0.121569,0.466667,0.705882}%
\pgfsetfillcolor{currentfill}%
\pgfsetfillopacity{0.735070}%
\pgfsetlinewidth{1.003750pt}%
\definecolor{currentstroke}{rgb}{0.121569,0.466667,0.705882}%
\pgfsetstrokecolor{currentstroke}%
\pgfsetstrokeopacity{0.735070}%
\pgfsetdash{}{0pt}%
\pgfpathmoveto{\pgfqpoint{3.192151in}{2.558442in}}%
\pgfpathcurveto{\pgfqpoint{3.200387in}{2.558442in}}{\pgfqpoint{3.208287in}{2.561714in}}{\pgfqpoint{3.214111in}{2.567538in}}%
\pgfpathcurveto{\pgfqpoint{3.219935in}{2.573362in}}{\pgfqpoint{3.223207in}{2.581262in}}{\pgfqpoint{3.223207in}{2.589499in}}%
\pgfpathcurveto{\pgfqpoint{3.223207in}{2.597735in}}{\pgfqpoint{3.219935in}{2.605635in}}{\pgfqpoint{3.214111in}{2.611459in}}%
\pgfpathcurveto{\pgfqpoint{3.208287in}{2.617283in}}{\pgfqpoint{3.200387in}{2.620555in}}{\pgfqpoint{3.192151in}{2.620555in}}%
\pgfpathcurveto{\pgfqpoint{3.183914in}{2.620555in}}{\pgfqpoint{3.176014in}{2.617283in}}{\pgfqpoint{3.170190in}{2.611459in}}%
\pgfpathcurveto{\pgfqpoint{3.164366in}{2.605635in}}{\pgfqpoint{3.161094in}{2.597735in}}{\pgfqpoint{3.161094in}{2.589499in}}%
\pgfpathcurveto{\pgfqpoint{3.161094in}{2.581262in}}{\pgfqpoint{3.164366in}{2.573362in}}{\pgfqpoint{3.170190in}{2.567538in}}%
\pgfpathcurveto{\pgfqpoint{3.176014in}{2.561714in}}{\pgfqpoint{3.183914in}{2.558442in}}{\pgfqpoint{3.192151in}{2.558442in}}%
\pgfpathclose%
\pgfusepath{stroke,fill}%
\end{pgfscope}%
\begin{pgfscope}%
\pgfpathrectangle{\pgfqpoint{0.100000in}{0.220728in}}{\pgfqpoint{3.696000in}{3.696000in}}%
\pgfusepath{clip}%
\pgfsetbuttcap%
\pgfsetroundjoin%
\definecolor{currentfill}{rgb}{0.121569,0.466667,0.705882}%
\pgfsetfillcolor{currentfill}%
\pgfsetfillopacity{0.735852}%
\pgfsetlinewidth{1.003750pt}%
\definecolor{currentstroke}{rgb}{0.121569,0.466667,0.705882}%
\pgfsetstrokecolor{currentstroke}%
\pgfsetstrokeopacity{0.735852}%
\pgfsetdash{}{0pt}%
\pgfpathmoveto{\pgfqpoint{3.189758in}{2.552522in}}%
\pgfpathcurveto{\pgfqpoint{3.197995in}{2.552522in}}{\pgfqpoint{3.205895in}{2.555794in}}{\pgfqpoint{3.211718in}{2.561618in}}%
\pgfpathcurveto{\pgfqpoint{3.217542in}{2.567442in}}{\pgfqpoint{3.220815in}{2.575342in}}{\pgfqpoint{3.220815in}{2.583578in}}%
\pgfpathcurveto{\pgfqpoint{3.220815in}{2.591815in}}{\pgfqpoint{3.217542in}{2.599715in}}{\pgfqpoint{3.211718in}{2.605539in}}%
\pgfpathcurveto{\pgfqpoint{3.205895in}{2.611363in}}{\pgfqpoint{3.197995in}{2.614635in}}{\pgfqpoint{3.189758in}{2.614635in}}%
\pgfpathcurveto{\pgfqpoint{3.181522in}{2.614635in}}{\pgfqpoint{3.173622in}{2.611363in}}{\pgfqpoint{3.167798in}{2.605539in}}%
\pgfpathcurveto{\pgfqpoint{3.161974in}{2.599715in}}{\pgfqpoint{3.158702in}{2.591815in}}{\pgfqpoint{3.158702in}{2.583578in}}%
\pgfpathcurveto{\pgfqpoint{3.158702in}{2.575342in}}{\pgfqpoint{3.161974in}{2.567442in}}{\pgfqpoint{3.167798in}{2.561618in}}%
\pgfpathcurveto{\pgfqpoint{3.173622in}{2.555794in}}{\pgfqpoint{3.181522in}{2.552522in}}{\pgfqpoint{3.189758in}{2.552522in}}%
\pgfpathclose%
\pgfusepath{stroke,fill}%
\end{pgfscope}%
\begin{pgfscope}%
\pgfpathrectangle{\pgfqpoint{0.100000in}{0.220728in}}{\pgfqpoint{3.696000in}{3.696000in}}%
\pgfusepath{clip}%
\pgfsetbuttcap%
\pgfsetroundjoin%
\definecolor{currentfill}{rgb}{0.121569,0.466667,0.705882}%
\pgfsetfillcolor{currentfill}%
\pgfsetfillopacity{0.736235}%
\pgfsetlinewidth{1.003750pt}%
\definecolor{currentstroke}{rgb}{0.121569,0.466667,0.705882}%
\pgfsetstrokecolor{currentstroke}%
\pgfsetstrokeopacity{0.736235}%
\pgfsetdash{}{0pt}%
\pgfpathmoveto{\pgfqpoint{3.187983in}{2.549705in}}%
\pgfpathcurveto{\pgfqpoint{3.196219in}{2.549705in}}{\pgfqpoint{3.204120in}{2.552978in}}{\pgfqpoint{3.209943in}{2.558802in}}%
\pgfpathcurveto{\pgfqpoint{3.215767in}{2.564626in}}{\pgfqpoint{3.219040in}{2.572526in}}{\pgfqpoint{3.219040in}{2.580762in}}%
\pgfpathcurveto{\pgfqpoint{3.219040in}{2.588998in}}{\pgfqpoint{3.215767in}{2.596898in}}{\pgfqpoint{3.209943in}{2.602722in}}%
\pgfpathcurveto{\pgfqpoint{3.204120in}{2.608546in}}{\pgfqpoint{3.196219in}{2.611818in}}{\pgfqpoint{3.187983in}{2.611818in}}%
\pgfpathcurveto{\pgfqpoint{3.179747in}{2.611818in}}{\pgfqpoint{3.171847in}{2.608546in}}{\pgfqpoint{3.166023in}{2.602722in}}%
\pgfpathcurveto{\pgfqpoint{3.160199in}{2.596898in}}{\pgfqpoint{3.156927in}{2.588998in}}{\pgfqpoint{3.156927in}{2.580762in}}%
\pgfpathcurveto{\pgfqpoint{3.156927in}{2.572526in}}{\pgfqpoint{3.160199in}{2.564626in}}{\pgfqpoint{3.166023in}{2.558802in}}%
\pgfpathcurveto{\pgfqpoint{3.171847in}{2.552978in}}{\pgfqpoint{3.179747in}{2.549705in}}{\pgfqpoint{3.187983in}{2.549705in}}%
\pgfpathclose%
\pgfusepath{stroke,fill}%
\end{pgfscope}%
\begin{pgfscope}%
\pgfpathrectangle{\pgfqpoint{0.100000in}{0.220728in}}{\pgfqpoint{3.696000in}{3.696000in}}%
\pgfusepath{clip}%
\pgfsetbuttcap%
\pgfsetroundjoin%
\definecolor{currentfill}{rgb}{0.121569,0.466667,0.705882}%
\pgfsetfillcolor{currentfill}%
\pgfsetfillopacity{0.736865}%
\pgfsetlinewidth{1.003750pt}%
\definecolor{currentstroke}{rgb}{0.121569,0.466667,0.705882}%
\pgfsetstrokecolor{currentstroke}%
\pgfsetstrokeopacity{0.736865}%
\pgfsetdash{}{0pt}%
\pgfpathmoveto{\pgfqpoint{1.016815in}{1.254687in}}%
\pgfpathcurveto{\pgfqpoint{1.025051in}{1.254687in}}{\pgfqpoint{1.032951in}{1.257959in}}{\pgfqpoint{1.038775in}{1.263783in}}%
\pgfpathcurveto{\pgfqpoint{1.044599in}{1.269607in}}{\pgfqpoint{1.047871in}{1.277507in}}{\pgfqpoint{1.047871in}{1.285743in}}%
\pgfpathcurveto{\pgfqpoint{1.047871in}{1.293980in}}{\pgfqpoint{1.044599in}{1.301880in}}{\pgfqpoint{1.038775in}{1.307704in}}%
\pgfpathcurveto{\pgfqpoint{1.032951in}{1.313528in}}{\pgfqpoint{1.025051in}{1.316800in}}{\pgfqpoint{1.016815in}{1.316800in}}%
\pgfpathcurveto{\pgfqpoint{1.008579in}{1.316800in}}{\pgfqpoint{1.000678in}{1.313528in}}{\pgfqpoint{0.994855in}{1.307704in}}%
\pgfpathcurveto{\pgfqpoint{0.989031in}{1.301880in}}{\pgfqpoint{0.985758in}{1.293980in}}{\pgfqpoint{0.985758in}{1.285743in}}%
\pgfpathcurveto{\pgfqpoint{0.985758in}{1.277507in}}{\pgfqpoint{0.989031in}{1.269607in}}{\pgfqpoint{0.994855in}{1.263783in}}%
\pgfpathcurveto{\pgfqpoint{1.000678in}{1.257959in}}{\pgfqpoint{1.008579in}{1.254687in}}{\pgfqpoint{1.016815in}{1.254687in}}%
\pgfpathclose%
\pgfusepath{stroke,fill}%
\end{pgfscope}%
\begin{pgfscope}%
\pgfpathrectangle{\pgfqpoint{0.100000in}{0.220728in}}{\pgfqpoint{3.696000in}{3.696000in}}%
\pgfusepath{clip}%
\pgfsetbuttcap%
\pgfsetroundjoin%
\definecolor{currentfill}{rgb}{0.121569,0.466667,0.705882}%
\pgfsetfillcolor{currentfill}%
\pgfsetfillopacity{0.736920}%
\pgfsetlinewidth{1.003750pt}%
\definecolor{currentstroke}{rgb}{0.121569,0.466667,0.705882}%
\pgfsetstrokecolor{currentstroke}%
\pgfsetstrokeopacity{0.736920}%
\pgfsetdash{}{0pt}%
\pgfpathmoveto{\pgfqpoint{3.185937in}{2.546382in}}%
\pgfpathcurveto{\pgfqpoint{3.194174in}{2.546382in}}{\pgfqpoint{3.202074in}{2.549655in}}{\pgfqpoint{3.207898in}{2.555479in}}%
\pgfpathcurveto{\pgfqpoint{3.213722in}{2.561302in}}{\pgfqpoint{3.216994in}{2.569203in}}{\pgfqpoint{3.216994in}{2.577439in}}%
\pgfpathcurveto{\pgfqpoint{3.216994in}{2.585675in}}{\pgfqpoint{3.213722in}{2.593575in}}{\pgfqpoint{3.207898in}{2.599399in}}%
\pgfpathcurveto{\pgfqpoint{3.202074in}{2.605223in}}{\pgfqpoint{3.194174in}{2.608495in}}{\pgfqpoint{3.185937in}{2.608495in}}%
\pgfpathcurveto{\pgfqpoint{3.177701in}{2.608495in}}{\pgfqpoint{3.169801in}{2.605223in}}{\pgfqpoint{3.163977in}{2.599399in}}%
\pgfpathcurveto{\pgfqpoint{3.158153in}{2.593575in}}{\pgfqpoint{3.154881in}{2.585675in}}{\pgfqpoint{3.154881in}{2.577439in}}%
\pgfpathcurveto{\pgfqpoint{3.154881in}{2.569203in}}{\pgfqpoint{3.158153in}{2.561302in}}{\pgfqpoint{3.163977in}{2.555479in}}%
\pgfpathcurveto{\pgfqpoint{3.169801in}{2.549655in}}{\pgfqpoint{3.177701in}{2.546382in}}{\pgfqpoint{3.185937in}{2.546382in}}%
\pgfpathclose%
\pgfusepath{stroke,fill}%
\end{pgfscope}%
\begin{pgfscope}%
\pgfpathrectangle{\pgfqpoint{0.100000in}{0.220728in}}{\pgfqpoint{3.696000in}{3.696000in}}%
\pgfusepath{clip}%
\pgfsetbuttcap%
\pgfsetroundjoin%
\definecolor{currentfill}{rgb}{0.121569,0.466667,0.705882}%
\pgfsetfillcolor{currentfill}%
\pgfsetfillopacity{0.737280}%
\pgfsetlinewidth{1.003750pt}%
\definecolor{currentstroke}{rgb}{0.121569,0.466667,0.705882}%
\pgfsetstrokecolor{currentstroke}%
\pgfsetstrokeopacity{0.737280}%
\pgfsetdash{}{0pt}%
\pgfpathmoveto{\pgfqpoint{3.184958in}{2.544295in}}%
\pgfpathcurveto{\pgfqpoint{3.193195in}{2.544295in}}{\pgfqpoint{3.201095in}{2.547568in}}{\pgfqpoint{3.206919in}{2.553392in}}%
\pgfpathcurveto{\pgfqpoint{3.212743in}{2.559216in}}{\pgfqpoint{3.216015in}{2.567116in}}{\pgfqpoint{3.216015in}{2.575352in}}%
\pgfpathcurveto{\pgfqpoint{3.216015in}{2.583588in}}{\pgfqpoint{3.212743in}{2.591488in}}{\pgfqpoint{3.206919in}{2.597312in}}%
\pgfpathcurveto{\pgfqpoint{3.201095in}{2.603136in}}{\pgfqpoint{3.193195in}{2.606408in}}{\pgfqpoint{3.184958in}{2.606408in}}%
\pgfpathcurveto{\pgfqpoint{3.176722in}{2.606408in}}{\pgfqpoint{3.168822in}{2.603136in}}{\pgfqpoint{3.162998in}{2.597312in}}%
\pgfpathcurveto{\pgfqpoint{3.157174in}{2.591488in}}{\pgfqpoint{3.153902in}{2.583588in}}{\pgfqpoint{3.153902in}{2.575352in}}%
\pgfpathcurveto{\pgfqpoint{3.153902in}{2.567116in}}{\pgfqpoint{3.157174in}{2.559216in}}{\pgfqpoint{3.162998in}{2.553392in}}%
\pgfpathcurveto{\pgfqpoint{3.168822in}{2.547568in}}{\pgfqpoint{3.176722in}{2.544295in}}{\pgfqpoint{3.184958in}{2.544295in}}%
\pgfpathclose%
\pgfusepath{stroke,fill}%
\end{pgfscope}%
\begin{pgfscope}%
\pgfpathrectangle{\pgfqpoint{0.100000in}{0.220728in}}{\pgfqpoint{3.696000in}{3.696000in}}%
\pgfusepath{clip}%
\pgfsetbuttcap%
\pgfsetroundjoin%
\definecolor{currentfill}{rgb}{0.121569,0.466667,0.705882}%
\pgfsetfillcolor{currentfill}%
\pgfsetfillopacity{0.737761}%
\pgfsetlinewidth{1.003750pt}%
\definecolor{currentstroke}{rgb}{0.121569,0.466667,0.705882}%
\pgfsetstrokecolor{currentstroke}%
\pgfsetstrokeopacity{0.737761}%
\pgfsetdash{}{0pt}%
\pgfpathmoveto{\pgfqpoint{3.182681in}{2.541468in}}%
\pgfpathcurveto{\pgfqpoint{3.190918in}{2.541468in}}{\pgfqpoint{3.198818in}{2.544740in}}{\pgfqpoint{3.204642in}{2.550564in}}%
\pgfpathcurveto{\pgfqpoint{3.210466in}{2.556388in}}{\pgfqpoint{3.213738in}{2.564288in}}{\pgfqpoint{3.213738in}{2.572525in}}%
\pgfpathcurveto{\pgfqpoint{3.213738in}{2.580761in}}{\pgfqpoint{3.210466in}{2.588661in}}{\pgfqpoint{3.204642in}{2.594485in}}%
\pgfpathcurveto{\pgfqpoint{3.198818in}{2.600309in}}{\pgfqpoint{3.190918in}{2.603581in}}{\pgfqpoint{3.182681in}{2.603581in}}%
\pgfpathcurveto{\pgfqpoint{3.174445in}{2.603581in}}{\pgfqpoint{3.166545in}{2.600309in}}{\pgfqpoint{3.160721in}{2.594485in}}%
\pgfpathcurveto{\pgfqpoint{3.154897in}{2.588661in}}{\pgfqpoint{3.151625in}{2.580761in}}{\pgfqpoint{3.151625in}{2.572525in}}%
\pgfpathcurveto{\pgfqpoint{3.151625in}{2.564288in}}{\pgfqpoint{3.154897in}{2.556388in}}{\pgfqpoint{3.160721in}{2.550564in}}%
\pgfpathcurveto{\pgfqpoint{3.166545in}{2.544740in}}{\pgfqpoint{3.174445in}{2.541468in}}{\pgfqpoint{3.182681in}{2.541468in}}%
\pgfpathclose%
\pgfusepath{stroke,fill}%
\end{pgfscope}%
\begin{pgfscope}%
\pgfpathrectangle{\pgfqpoint{0.100000in}{0.220728in}}{\pgfqpoint{3.696000in}{3.696000in}}%
\pgfusepath{clip}%
\pgfsetbuttcap%
\pgfsetroundjoin%
\definecolor{currentfill}{rgb}{0.121569,0.466667,0.705882}%
\pgfsetfillcolor{currentfill}%
\pgfsetfillopacity{0.738643}%
\pgfsetlinewidth{1.003750pt}%
\definecolor{currentstroke}{rgb}{0.121569,0.466667,0.705882}%
\pgfsetstrokecolor{currentstroke}%
\pgfsetstrokeopacity{0.738643}%
\pgfsetdash{}{0pt}%
\pgfpathmoveto{\pgfqpoint{3.180709in}{2.534683in}}%
\pgfpathcurveto{\pgfqpoint{3.188945in}{2.534683in}}{\pgfqpoint{3.196846in}{2.537955in}}{\pgfqpoint{3.202669in}{2.543779in}}%
\pgfpathcurveto{\pgfqpoint{3.208493in}{2.549603in}}{\pgfqpoint{3.211766in}{2.557503in}}{\pgfqpoint{3.211766in}{2.565739in}}%
\pgfpathcurveto{\pgfqpoint{3.211766in}{2.573975in}}{\pgfqpoint{3.208493in}{2.581876in}}{\pgfqpoint{3.202669in}{2.587699in}}%
\pgfpathcurveto{\pgfqpoint{3.196846in}{2.593523in}}{\pgfqpoint{3.188945in}{2.596796in}}{\pgfqpoint{3.180709in}{2.596796in}}%
\pgfpathcurveto{\pgfqpoint{3.172473in}{2.596796in}}{\pgfqpoint{3.164573in}{2.593523in}}{\pgfqpoint{3.158749in}{2.587699in}}%
\pgfpathcurveto{\pgfqpoint{3.152925in}{2.581876in}}{\pgfqpoint{3.149653in}{2.573975in}}{\pgfqpoint{3.149653in}{2.565739in}}%
\pgfpathcurveto{\pgfqpoint{3.149653in}{2.557503in}}{\pgfqpoint{3.152925in}{2.549603in}}{\pgfqpoint{3.158749in}{2.543779in}}%
\pgfpathcurveto{\pgfqpoint{3.164573in}{2.537955in}}{\pgfqpoint{3.172473in}{2.534683in}}{\pgfqpoint{3.180709in}{2.534683in}}%
\pgfpathclose%
\pgfusepath{stroke,fill}%
\end{pgfscope}%
\begin{pgfscope}%
\pgfpathrectangle{\pgfqpoint{0.100000in}{0.220728in}}{\pgfqpoint{3.696000in}{3.696000in}}%
\pgfusepath{clip}%
\pgfsetbuttcap%
\pgfsetroundjoin%
\definecolor{currentfill}{rgb}{0.121569,0.466667,0.705882}%
\pgfsetfillcolor{currentfill}%
\pgfsetfillopacity{0.739114}%
\pgfsetlinewidth{1.003750pt}%
\definecolor{currentstroke}{rgb}{0.121569,0.466667,0.705882}%
\pgfsetstrokecolor{currentstroke}%
\pgfsetstrokeopacity{0.739114}%
\pgfsetdash{}{0pt}%
\pgfpathmoveto{\pgfqpoint{3.179031in}{2.531517in}}%
\pgfpathcurveto{\pgfqpoint{3.187267in}{2.531517in}}{\pgfqpoint{3.195167in}{2.534789in}}{\pgfqpoint{3.200991in}{2.540613in}}%
\pgfpathcurveto{\pgfqpoint{3.206815in}{2.546437in}}{\pgfqpoint{3.210087in}{2.554337in}}{\pgfqpoint{3.210087in}{2.562573in}}%
\pgfpathcurveto{\pgfqpoint{3.210087in}{2.570810in}}{\pgfqpoint{3.206815in}{2.578710in}}{\pgfqpoint{3.200991in}{2.584534in}}%
\pgfpathcurveto{\pgfqpoint{3.195167in}{2.590358in}}{\pgfqpoint{3.187267in}{2.593630in}}{\pgfqpoint{3.179031in}{2.593630in}}%
\pgfpathcurveto{\pgfqpoint{3.170795in}{2.593630in}}{\pgfqpoint{3.162894in}{2.590358in}}{\pgfqpoint{3.157071in}{2.584534in}}%
\pgfpathcurveto{\pgfqpoint{3.151247in}{2.578710in}}{\pgfqpoint{3.147974in}{2.570810in}}{\pgfqpoint{3.147974in}{2.562573in}}%
\pgfpathcurveto{\pgfqpoint{3.147974in}{2.554337in}}{\pgfqpoint{3.151247in}{2.546437in}}{\pgfqpoint{3.157071in}{2.540613in}}%
\pgfpathcurveto{\pgfqpoint{3.162894in}{2.534789in}}{\pgfqpoint{3.170795in}{2.531517in}}{\pgfqpoint{3.179031in}{2.531517in}}%
\pgfpathclose%
\pgfusepath{stroke,fill}%
\end{pgfscope}%
\begin{pgfscope}%
\pgfpathrectangle{\pgfqpoint{0.100000in}{0.220728in}}{\pgfqpoint{3.696000in}{3.696000in}}%
\pgfusepath{clip}%
\pgfsetbuttcap%
\pgfsetroundjoin%
\definecolor{currentfill}{rgb}{0.121569,0.466667,0.705882}%
\pgfsetfillcolor{currentfill}%
\pgfsetfillopacity{0.739444}%
\pgfsetlinewidth{1.003750pt}%
\definecolor{currentstroke}{rgb}{0.121569,0.466667,0.705882}%
\pgfsetstrokecolor{currentstroke}%
\pgfsetstrokeopacity{0.739444}%
\pgfsetdash{}{0pt}%
\pgfpathmoveto{\pgfqpoint{3.178006in}{2.530236in}}%
\pgfpathcurveto{\pgfqpoint{3.186242in}{2.530236in}}{\pgfqpoint{3.194142in}{2.533508in}}{\pgfqpoint{3.199966in}{2.539332in}}%
\pgfpathcurveto{\pgfqpoint{3.205790in}{2.545156in}}{\pgfqpoint{3.209063in}{2.553056in}}{\pgfqpoint{3.209063in}{2.561292in}}%
\pgfpathcurveto{\pgfqpoint{3.209063in}{2.569529in}}{\pgfqpoint{3.205790in}{2.577429in}}{\pgfqpoint{3.199966in}{2.583253in}}%
\pgfpathcurveto{\pgfqpoint{3.194142in}{2.589077in}}{\pgfqpoint{3.186242in}{2.592349in}}{\pgfqpoint{3.178006in}{2.592349in}}%
\pgfpathcurveto{\pgfqpoint{3.169770in}{2.592349in}}{\pgfqpoint{3.161870in}{2.589077in}}{\pgfqpoint{3.156046in}{2.583253in}}%
\pgfpathcurveto{\pgfqpoint{3.150222in}{2.577429in}}{\pgfqpoint{3.146950in}{2.569529in}}{\pgfqpoint{3.146950in}{2.561292in}}%
\pgfpathcurveto{\pgfqpoint{3.146950in}{2.553056in}}{\pgfqpoint{3.150222in}{2.545156in}}{\pgfqpoint{3.156046in}{2.539332in}}%
\pgfpathcurveto{\pgfqpoint{3.161870in}{2.533508in}}{\pgfqpoint{3.169770in}{2.530236in}}{\pgfqpoint{3.178006in}{2.530236in}}%
\pgfpathclose%
\pgfusepath{stroke,fill}%
\end{pgfscope}%
\begin{pgfscope}%
\pgfpathrectangle{\pgfqpoint{0.100000in}{0.220728in}}{\pgfqpoint{3.696000in}{3.696000in}}%
\pgfusepath{clip}%
\pgfsetbuttcap%
\pgfsetroundjoin%
\definecolor{currentfill}{rgb}{0.121569,0.466667,0.705882}%
\pgfsetfillcolor{currentfill}%
\pgfsetfillopacity{0.739601}%
\pgfsetlinewidth{1.003750pt}%
\definecolor{currentstroke}{rgb}{0.121569,0.466667,0.705882}%
\pgfsetstrokecolor{currentstroke}%
\pgfsetstrokeopacity{0.739601}%
\pgfsetdash{}{0pt}%
\pgfpathmoveto{\pgfqpoint{3.177519in}{2.529313in}}%
\pgfpathcurveto{\pgfqpoint{3.185755in}{2.529313in}}{\pgfqpoint{3.193655in}{2.532586in}}{\pgfqpoint{3.199479in}{2.538410in}}%
\pgfpathcurveto{\pgfqpoint{3.205303in}{2.544234in}}{\pgfqpoint{3.208575in}{2.552134in}}{\pgfqpoint{3.208575in}{2.560370in}}%
\pgfpathcurveto{\pgfqpoint{3.208575in}{2.568606in}}{\pgfqpoint{3.205303in}{2.576506in}}{\pgfqpoint{3.199479in}{2.582330in}}%
\pgfpathcurveto{\pgfqpoint{3.193655in}{2.588154in}}{\pgfqpoint{3.185755in}{2.591426in}}{\pgfqpoint{3.177519in}{2.591426in}}%
\pgfpathcurveto{\pgfqpoint{3.169282in}{2.591426in}}{\pgfqpoint{3.161382in}{2.588154in}}{\pgfqpoint{3.155558in}{2.582330in}}%
\pgfpathcurveto{\pgfqpoint{3.149735in}{2.576506in}}{\pgfqpoint{3.146462in}{2.568606in}}{\pgfqpoint{3.146462in}{2.560370in}}%
\pgfpathcurveto{\pgfqpoint{3.146462in}{2.552134in}}{\pgfqpoint{3.149735in}{2.544234in}}{\pgfqpoint{3.155558in}{2.538410in}}%
\pgfpathcurveto{\pgfqpoint{3.161382in}{2.532586in}}{\pgfqpoint{3.169282in}{2.529313in}}{\pgfqpoint{3.177519in}{2.529313in}}%
\pgfpathclose%
\pgfusepath{stroke,fill}%
\end{pgfscope}%
\begin{pgfscope}%
\pgfpathrectangle{\pgfqpoint{0.100000in}{0.220728in}}{\pgfqpoint{3.696000in}{3.696000in}}%
\pgfusepath{clip}%
\pgfsetbuttcap%
\pgfsetroundjoin%
\definecolor{currentfill}{rgb}{0.121569,0.466667,0.705882}%
\pgfsetfillcolor{currentfill}%
\pgfsetfillopacity{0.740050}%
\pgfsetlinewidth{1.003750pt}%
\definecolor{currentstroke}{rgb}{0.121569,0.466667,0.705882}%
\pgfsetstrokecolor{currentstroke}%
\pgfsetstrokeopacity{0.740050}%
\pgfsetdash{}{0pt}%
\pgfpathmoveto{\pgfqpoint{3.176220in}{2.527740in}}%
\pgfpathcurveto{\pgfqpoint{3.184456in}{2.527740in}}{\pgfqpoint{3.192356in}{2.531012in}}{\pgfqpoint{3.198180in}{2.536836in}}%
\pgfpathcurveto{\pgfqpoint{3.204004in}{2.542660in}}{\pgfqpoint{3.207276in}{2.550560in}}{\pgfqpoint{3.207276in}{2.558796in}}%
\pgfpathcurveto{\pgfqpoint{3.207276in}{2.567032in}}{\pgfqpoint{3.204004in}{2.574932in}}{\pgfqpoint{3.198180in}{2.580756in}}%
\pgfpathcurveto{\pgfqpoint{3.192356in}{2.586580in}}{\pgfqpoint{3.184456in}{2.589853in}}{\pgfqpoint{3.176220in}{2.589853in}}%
\pgfpathcurveto{\pgfqpoint{3.167984in}{2.589853in}}{\pgfqpoint{3.160084in}{2.586580in}}{\pgfqpoint{3.154260in}{2.580756in}}%
\pgfpathcurveto{\pgfqpoint{3.148436in}{2.574932in}}{\pgfqpoint{3.145163in}{2.567032in}}{\pgfqpoint{3.145163in}{2.558796in}}%
\pgfpathcurveto{\pgfqpoint{3.145163in}{2.550560in}}{\pgfqpoint{3.148436in}{2.542660in}}{\pgfqpoint{3.154260in}{2.536836in}}%
\pgfpathcurveto{\pgfqpoint{3.160084in}{2.531012in}}{\pgfqpoint{3.167984in}{2.527740in}}{\pgfqpoint{3.176220in}{2.527740in}}%
\pgfpathclose%
\pgfusepath{stroke,fill}%
\end{pgfscope}%
\begin{pgfscope}%
\pgfpathrectangle{\pgfqpoint{0.100000in}{0.220728in}}{\pgfqpoint{3.696000in}{3.696000in}}%
\pgfusepath{clip}%
\pgfsetbuttcap%
\pgfsetroundjoin%
\definecolor{currentfill}{rgb}{0.121569,0.466667,0.705882}%
\pgfsetfillcolor{currentfill}%
\pgfsetfillopacity{0.740500}%
\pgfsetlinewidth{1.003750pt}%
\definecolor{currentstroke}{rgb}{0.121569,0.466667,0.705882}%
\pgfsetstrokecolor{currentstroke}%
\pgfsetstrokeopacity{0.740500}%
\pgfsetdash{}{0pt}%
\pgfpathmoveto{\pgfqpoint{3.174632in}{2.523756in}}%
\pgfpathcurveto{\pgfqpoint{3.182868in}{2.523756in}}{\pgfqpoint{3.190768in}{2.527028in}}{\pgfqpoint{3.196592in}{2.532852in}}%
\pgfpathcurveto{\pgfqpoint{3.202416in}{2.538676in}}{\pgfqpoint{3.205689in}{2.546576in}}{\pgfqpoint{3.205689in}{2.554812in}}%
\pgfpathcurveto{\pgfqpoint{3.205689in}{2.563049in}}{\pgfqpoint{3.202416in}{2.570949in}}{\pgfqpoint{3.196592in}{2.576773in}}%
\pgfpathcurveto{\pgfqpoint{3.190768in}{2.582596in}}{\pgfqpoint{3.182868in}{2.585869in}}{\pgfqpoint{3.174632in}{2.585869in}}%
\pgfpathcurveto{\pgfqpoint{3.166396in}{2.585869in}}{\pgfqpoint{3.158496in}{2.582596in}}{\pgfqpoint{3.152672in}{2.576773in}}%
\pgfpathcurveto{\pgfqpoint{3.146848in}{2.570949in}}{\pgfqpoint{3.143576in}{2.563049in}}{\pgfqpoint{3.143576in}{2.554812in}}%
\pgfpathcurveto{\pgfqpoint{3.143576in}{2.546576in}}{\pgfqpoint{3.146848in}{2.538676in}}{\pgfqpoint{3.152672in}{2.532852in}}%
\pgfpathcurveto{\pgfqpoint{3.158496in}{2.527028in}}{\pgfqpoint{3.166396in}{2.523756in}}{\pgfqpoint{3.174632in}{2.523756in}}%
\pgfpathclose%
\pgfusepath{stroke,fill}%
\end{pgfscope}%
\begin{pgfscope}%
\pgfpathrectangle{\pgfqpoint{0.100000in}{0.220728in}}{\pgfqpoint{3.696000in}{3.696000in}}%
\pgfusepath{clip}%
\pgfsetbuttcap%
\pgfsetroundjoin%
\definecolor{currentfill}{rgb}{0.121569,0.466667,0.705882}%
\pgfsetfillcolor{currentfill}%
\pgfsetfillopacity{0.741215}%
\pgfsetlinewidth{1.003750pt}%
\definecolor{currentstroke}{rgb}{0.121569,0.466667,0.705882}%
\pgfsetstrokecolor{currentstroke}%
\pgfsetstrokeopacity{0.741215}%
\pgfsetdash{}{0pt}%
\pgfpathmoveto{\pgfqpoint{3.171507in}{2.519587in}}%
\pgfpathcurveto{\pgfqpoint{3.179744in}{2.519587in}}{\pgfqpoint{3.187644in}{2.522860in}}{\pgfqpoint{3.193468in}{2.528684in}}%
\pgfpathcurveto{\pgfqpoint{3.199291in}{2.534508in}}{\pgfqpoint{3.202564in}{2.542408in}}{\pgfqpoint{3.202564in}{2.550644in}}%
\pgfpathcurveto{\pgfqpoint{3.202564in}{2.558880in}}{\pgfqpoint{3.199291in}{2.566780in}}{\pgfqpoint{3.193468in}{2.572604in}}%
\pgfpathcurveto{\pgfqpoint{3.187644in}{2.578428in}}{\pgfqpoint{3.179744in}{2.581700in}}{\pgfqpoint{3.171507in}{2.581700in}}%
\pgfpathcurveto{\pgfqpoint{3.163271in}{2.581700in}}{\pgfqpoint{3.155371in}{2.578428in}}{\pgfqpoint{3.149547in}{2.572604in}}%
\pgfpathcurveto{\pgfqpoint{3.143723in}{2.566780in}}{\pgfqpoint{3.140451in}{2.558880in}}{\pgfqpoint{3.140451in}{2.550644in}}%
\pgfpathcurveto{\pgfqpoint{3.140451in}{2.542408in}}{\pgfqpoint{3.143723in}{2.534508in}}{\pgfqpoint{3.149547in}{2.528684in}}%
\pgfpathcurveto{\pgfqpoint{3.155371in}{2.522860in}}{\pgfqpoint{3.163271in}{2.519587in}}{\pgfqpoint{3.171507in}{2.519587in}}%
\pgfpathclose%
\pgfusepath{stroke,fill}%
\end{pgfscope}%
\begin{pgfscope}%
\pgfpathrectangle{\pgfqpoint{0.100000in}{0.220728in}}{\pgfqpoint{3.696000in}{3.696000in}}%
\pgfusepath{clip}%
\pgfsetbuttcap%
\pgfsetroundjoin%
\definecolor{currentfill}{rgb}{0.121569,0.466667,0.705882}%
\pgfsetfillcolor{currentfill}%
\pgfsetfillopacity{0.741262}%
\pgfsetlinewidth{1.003750pt}%
\definecolor{currentstroke}{rgb}{0.121569,0.466667,0.705882}%
\pgfsetstrokecolor{currentstroke}%
\pgfsetstrokeopacity{0.741262}%
\pgfsetdash{}{0pt}%
\pgfpathmoveto{\pgfqpoint{1.042010in}{1.246566in}}%
\pgfpathcurveto{\pgfqpoint{1.050246in}{1.246566in}}{\pgfqpoint{1.058146in}{1.249838in}}{\pgfqpoint{1.063970in}{1.255662in}}%
\pgfpathcurveto{\pgfqpoint{1.069794in}{1.261486in}}{\pgfqpoint{1.073066in}{1.269386in}}{\pgfqpoint{1.073066in}{1.277622in}}%
\pgfpathcurveto{\pgfqpoint{1.073066in}{1.285859in}}{\pgfqpoint{1.069794in}{1.293759in}}{\pgfqpoint{1.063970in}{1.299583in}}%
\pgfpathcurveto{\pgfqpoint{1.058146in}{1.305407in}}{\pgfqpoint{1.050246in}{1.308679in}}{\pgfqpoint{1.042010in}{1.308679in}}%
\pgfpathcurveto{\pgfqpoint{1.033773in}{1.308679in}}{\pgfqpoint{1.025873in}{1.305407in}}{\pgfqpoint{1.020049in}{1.299583in}}%
\pgfpathcurveto{\pgfqpoint{1.014225in}{1.293759in}}{\pgfqpoint{1.010953in}{1.285859in}}{\pgfqpoint{1.010953in}{1.277622in}}%
\pgfpathcurveto{\pgfqpoint{1.010953in}{1.269386in}}{\pgfqpoint{1.014225in}{1.261486in}}{\pgfqpoint{1.020049in}{1.255662in}}%
\pgfpathcurveto{\pgfqpoint{1.025873in}{1.249838in}}{\pgfqpoint{1.033773in}{1.246566in}}{\pgfqpoint{1.042010in}{1.246566in}}%
\pgfpathclose%
\pgfusepath{stroke,fill}%
\end{pgfscope}%
\begin{pgfscope}%
\pgfpathrectangle{\pgfqpoint{0.100000in}{0.220728in}}{\pgfqpoint{3.696000in}{3.696000in}}%
\pgfusepath{clip}%
\pgfsetbuttcap%
\pgfsetroundjoin%
\definecolor{currentfill}{rgb}{0.121569,0.466667,0.705882}%
\pgfsetfillcolor{currentfill}%
\pgfsetfillopacity{0.742376}%
\pgfsetlinewidth{1.003750pt}%
\definecolor{currentstroke}{rgb}{0.121569,0.466667,0.705882}%
\pgfsetstrokecolor{currentstroke}%
\pgfsetstrokeopacity{0.742376}%
\pgfsetdash{}{0pt}%
\pgfpathmoveto{\pgfqpoint{3.168780in}{2.514765in}}%
\pgfpathcurveto{\pgfqpoint{3.177016in}{2.514765in}}{\pgfqpoint{3.184916in}{2.518037in}}{\pgfqpoint{3.190740in}{2.523861in}}%
\pgfpathcurveto{\pgfqpoint{3.196564in}{2.529685in}}{\pgfqpoint{3.199836in}{2.537585in}}{\pgfqpoint{3.199836in}{2.545821in}}%
\pgfpathcurveto{\pgfqpoint{3.199836in}{2.554058in}}{\pgfqpoint{3.196564in}{2.561958in}}{\pgfqpoint{3.190740in}{2.567782in}}%
\pgfpathcurveto{\pgfqpoint{3.184916in}{2.573606in}}{\pgfqpoint{3.177016in}{2.576878in}}{\pgfqpoint{3.168780in}{2.576878in}}%
\pgfpathcurveto{\pgfqpoint{3.160543in}{2.576878in}}{\pgfqpoint{3.152643in}{2.573606in}}{\pgfqpoint{3.146819in}{2.567782in}}%
\pgfpathcurveto{\pgfqpoint{3.140995in}{2.561958in}}{\pgfqpoint{3.137723in}{2.554058in}}{\pgfqpoint{3.137723in}{2.545821in}}%
\pgfpathcurveto{\pgfqpoint{3.137723in}{2.537585in}}{\pgfqpoint{3.140995in}{2.529685in}}{\pgfqpoint{3.146819in}{2.523861in}}%
\pgfpathcurveto{\pgfqpoint{3.152643in}{2.518037in}}{\pgfqpoint{3.160543in}{2.514765in}}{\pgfqpoint{3.168780in}{2.514765in}}%
\pgfpathclose%
\pgfusepath{stroke,fill}%
\end{pgfscope}%
\begin{pgfscope}%
\pgfpathrectangle{\pgfqpoint{0.100000in}{0.220728in}}{\pgfqpoint{3.696000in}{3.696000in}}%
\pgfusepath{clip}%
\pgfsetbuttcap%
\pgfsetroundjoin%
\definecolor{currentfill}{rgb}{0.121569,0.466667,0.705882}%
\pgfsetfillcolor{currentfill}%
\pgfsetfillopacity{0.742950}%
\pgfsetlinewidth{1.003750pt}%
\definecolor{currentstroke}{rgb}{0.121569,0.466667,0.705882}%
\pgfsetstrokecolor{currentstroke}%
\pgfsetstrokeopacity{0.742950}%
\pgfsetdash{}{0pt}%
\pgfpathmoveto{\pgfqpoint{3.167421in}{2.511681in}}%
\pgfpathcurveto{\pgfqpoint{3.175657in}{2.511681in}}{\pgfqpoint{3.183557in}{2.514953in}}{\pgfqpoint{3.189381in}{2.520777in}}%
\pgfpathcurveto{\pgfqpoint{3.195205in}{2.526601in}}{\pgfqpoint{3.198477in}{2.534501in}}{\pgfqpoint{3.198477in}{2.542737in}}%
\pgfpathcurveto{\pgfqpoint{3.198477in}{2.550974in}}{\pgfqpoint{3.195205in}{2.558874in}}{\pgfqpoint{3.189381in}{2.564698in}}%
\pgfpathcurveto{\pgfqpoint{3.183557in}{2.570522in}}{\pgfqpoint{3.175657in}{2.573794in}}{\pgfqpoint{3.167421in}{2.573794in}}%
\pgfpathcurveto{\pgfqpoint{3.159185in}{2.573794in}}{\pgfqpoint{3.151285in}{2.570522in}}{\pgfqpoint{3.145461in}{2.564698in}}%
\pgfpathcurveto{\pgfqpoint{3.139637in}{2.558874in}}{\pgfqpoint{3.136364in}{2.550974in}}{\pgfqpoint{3.136364in}{2.542737in}}%
\pgfpathcurveto{\pgfqpoint{3.136364in}{2.534501in}}{\pgfqpoint{3.139637in}{2.526601in}}{\pgfqpoint{3.145461in}{2.520777in}}%
\pgfpathcurveto{\pgfqpoint{3.151285in}{2.514953in}}{\pgfqpoint{3.159185in}{2.511681in}}{\pgfqpoint{3.167421in}{2.511681in}}%
\pgfpathclose%
\pgfusepath{stroke,fill}%
\end{pgfscope}%
\begin{pgfscope}%
\pgfpathrectangle{\pgfqpoint{0.100000in}{0.220728in}}{\pgfqpoint{3.696000in}{3.696000in}}%
\pgfusepath{clip}%
\pgfsetbuttcap%
\pgfsetroundjoin%
\definecolor{currentfill}{rgb}{0.121569,0.466667,0.705882}%
\pgfsetfillcolor{currentfill}%
\pgfsetfillopacity{0.743453}%
\pgfsetlinewidth{1.003750pt}%
\definecolor{currentstroke}{rgb}{0.121569,0.466667,0.705882}%
\pgfsetstrokecolor{currentstroke}%
\pgfsetstrokeopacity{0.743453}%
\pgfsetdash{}{0pt}%
\pgfpathmoveto{\pgfqpoint{3.164790in}{2.508454in}}%
\pgfpathcurveto{\pgfqpoint{3.173026in}{2.508454in}}{\pgfqpoint{3.180926in}{2.511727in}}{\pgfqpoint{3.186750in}{2.517551in}}%
\pgfpathcurveto{\pgfqpoint{3.192574in}{2.523375in}}{\pgfqpoint{3.195846in}{2.531275in}}{\pgfqpoint{3.195846in}{2.539511in}}%
\pgfpathcurveto{\pgfqpoint{3.195846in}{2.547747in}}{\pgfqpoint{3.192574in}{2.555647in}}{\pgfqpoint{3.186750in}{2.561471in}}%
\pgfpathcurveto{\pgfqpoint{3.180926in}{2.567295in}}{\pgfqpoint{3.173026in}{2.570567in}}{\pgfqpoint{3.164790in}{2.570567in}}%
\pgfpathcurveto{\pgfqpoint{3.156554in}{2.570567in}}{\pgfqpoint{3.148654in}{2.567295in}}{\pgfqpoint{3.142830in}{2.561471in}}%
\pgfpathcurveto{\pgfqpoint{3.137006in}{2.555647in}}{\pgfqpoint{3.133733in}{2.547747in}}{\pgfqpoint{3.133733in}{2.539511in}}%
\pgfpathcurveto{\pgfqpoint{3.133733in}{2.531275in}}{\pgfqpoint{3.137006in}{2.523375in}}{\pgfqpoint{3.142830in}{2.517551in}}%
\pgfpathcurveto{\pgfqpoint{3.148654in}{2.511727in}}{\pgfqpoint{3.156554in}{2.508454in}}{\pgfqpoint{3.164790in}{2.508454in}}%
\pgfpathclose%
\pgfusepath{stroke,fill}%
\end{pgfscope}%
\begin{pgfscope}%
\pgfpathrectangle{\pgfqpoint{0.100000in}{0.220728in}}{\pgfqpoint{3.696000in}{3.696000in}}%
\pgfusepath{clip}%
\pgfsetbuttcap%
\pgfsetroundjoin%
\definecolor{currentfill}{rgb}{0.121569,0.466667,0.705882}%
\pgfsetfillcolor{currentfill}%
\pgfsetfillopacity{0.744462}%
\pgfsetlinewidth{1.003750pt}%
\definecolor{currentstroke}{rgb}{0.121569,0.466667,0.705882}%
\pgfsetstrokecolor{currentstroke}%
\pgfsetstrokeopacity{0.744462}%
\pgfsetdash{}{0pt}%
\pgfpathmoveto{\pgfqpoint{3.162855in}{2.502322in}}%
\pgfpathcurveto{\pgfqpoint{3.171092in}{2.502322in}}{\pgfqpoint{3.178992in}{2.505594in}}{\pgfqpoint{3.184816in}{2.511418in}}%
\pgfpathcurveto{\pgfqpoint{3.190640in}{2.517242in}}{\pgfqpoint{3.193912in}{2.525142in}}{\pgfqpoint{3.193912in}{2.533379in}}%
\pgfpathcurveto{\pgfqpoint{3.193912in}{2.541615in}}{\pgfqpoint{3.190640in}{2.549515in}}{\pgfqpoint{3.184816in}{2.555339in}}%
\pgfpathcurveto{\pgfqpoint{3.178992in}{2.561163in}}{\pgfqpoint{3.171092in}{2.564435in}}{\pgfqpoint{3.162855in}{2.564435in}}%
\pgfpathcurveto{\pgfqpoint{3.154619in}{2.564435in}}{\pgfqpoint{3.146719in}{2.561163in}}{\pgfqpoint{3.140895in}{2.555339in}}%
\pgfpathcurveto{\pgfqpoint{3.135071in}{2.549515in}}{\pgfqpoint{3.131799in}{2.541615in}}{\pgfqpoint{3.131799in}{2.533379in}}%
\pgfpathcurveto{\pgfqpoint{3.131799in}{2.525142in}}{\pgfqpoint{3.135071in}{2.517242in}}{\pgfqpoint{3.140895in}{2.511418in}}%
\pgfpathcurveto{\pgfqpoint{3.146719in}{2.505594in}}{\pgfqpoint{3.154619in}{2.502322in}}{\pgfqpoint{3.162855in}{2.502322in}}%
\pgfpathclose%
\pgfusepath{stroke,fill}%
\end{pgfscope}%
\begin{pgfscope}%
\pgfpathrectangle{\pgfqpoint{0.100000in}{0.220728in}}{\pgfqpoint{3.696000in}{3.696000in}}%
\pgfusepath{clip}%
\pgfsetbuttcap%
\pgfsetroundjoin%
\definecolor{currentfill}{rgb}{0.121569,0.466667,0.705882}%
\pgfsetfillcolor{currentfill}%
\pgfsetfillopacity{0.744992}%
\pgfsetlinewidth{1.003750pt}%
\definecolor{currentstroke}{rgb}{0.121569,0.466667,0.705882}%
\pgfsetstrokecolor{currentstroke}%
\pgfsetstrokeopacity{0.744992}%
\pgfsetdash{}{0pt}%
\pgfpathmoveto{\pgfqpoint{3.161169in}{2.499511in}}%
\pgfpathcurveto{\pgfqpoint{3.169405in}{2.499511in}}{\pgfqpoint{3.177305in}{2.502784in}}{\pgfqpoint{3.183129in}{2.508608in}}%
\pgfpathcurveto{\pgfqpoint{3.188953in}{2.514432in}}{\pgfqpoint{3.192225in}{2.522332in}}{\pgfqpoint{3.192225in}{2.530568in}}%
\pgfpathcurveto{\pgfqpoint{3.192225in}{2.538804in}}{\pgfqpoint{3.188953in}{2.546704in}}{\pgfqpoint{3.183129in}{2.552528in}}%
\pgfpathcurveto{\pgfqpoint{3.177305in}{2.558352in}}{\pgfqpoint{3.169405in}{2.561624in}}{\pgfqpoint{3.161169in}{2.561624in}}%
\pgfpathcurveto{\pgfqpoint{3.152932in}{2.561624in}}{\pgfqpoint{3.145032in}{2.558352in}}{\pgfqpoint{3.139208in}{2.552528in}}%
\pgfpathcurveto{\pgfqpoint{3.133385in}{2.546704in}}{\pgfqpoint{3.130112in}{2.538804in}}{\pgfqpoint{3.130112in}{2.530568in}}%
\pgfpathcurveto{\pgfqpoint{3.130112in}{2.522332in}}{\pgfqpoint{3.133385in}{2.514432in}}{\pgfqpoint{3.139208in}{2.508608in}}%
\pgfpathcurveto{\pgfqpoint{3.145032in}{2.502784in}}{\pgfqpoint{3.152932in}{2.499511in}}{\pgfqpoint{3.161169in}{2.499511in}}%
\pgfpathclose%
\pgfusepath{stroke,fill}%
\end{pgfscope}%
\begin{pgfscope}%
\pgfpathrectangle{\pgfqpoint{0.100000in}{0.220728in}}{\pgfqpoint{3.696000in}{3.696000in}}%
\pgfusepath{clip}%
\pgfsetbuttcap%
\pgfsetroundjoin%
\definecolor{currentfill}{rgb}{0.121569,0.466667,0.705882}%
\pgfsetfillcolor{currentfill}%
\pgfsetfillopacity{0.745266}%
\pgfsetlinewidth{1.003750pt}%
\definecolor{currentstroke}{rgb}{0.121569,0.466667,0.705882}%
\pgfsetstrokecolor{currentstroke}%
\pgfsetstrokeopacity{0.745266}%
\pgfsetdash{}{0pt}%
\pgfpathmoveto{\pgfqpoint{3.160170in}{2.498000in}}%
\pgfpathcurveto{\pgfqpoint{3.168407in}{2.498000in}}{\pgfqpoint{3.176307in}{2.501272in}}{\pgfqpoint{3.182131in}{2.507096in}}%
\pgfpathcurveto{\pgfqpoint{3.187955in}{2.512920in}}{\pgfqpoint{3.191227in}{2.520820in}}{\pgfqpoint{3.191227in}{2.529057in}}%
\pgfpathcurveto{\pgfqpoint{3.191227in}{2.537293in}}{\pgfqpoint{3.187955in}{2.545193in}}{\pgfqpoint{3.182131in}{2.551017in}}%
\pgfpathcurveto{\pgfqpoint{3.176307in}{2.556841in}}{\pgfqpoint{3.168407in}{2.560113in}}{\pgfqpoint{3.160170in}{2.560113in}}%
\pgfpathcurveto{\pgfqpoint{3.151934in}{2.560113in}}{\pgfqpoint{3.144034in}{2.556841in}}{\pgfqpoint{3.138210in}{2.551017in}}%
\pgfpathcurveto{\pgfqpoint{3.132386in}{2.545193in}}{\pgfqpoint{3.129114in}{2.537293in}}{\pgfqpoint{3.129114in}{2.529057in}}%
\pgfpathcurveto{\pgfqpoint{3.129114in}{2.520820in}}{\pgfqpoint{3.132386in}{2.512920in}}{\pgfqpoint{3.138210in}{2.507096in}}%
\pgfpathcurveto{\pgfqpoint{3.144034in}{2.501272in}}{\pgfqpoint{3.151934in}{2.498000in}}{\pgfqpoint{3.160170in}{2.498000in}}%
\pgfpathclose%
\pgfusepath{stroke,fill}%
\end{pgfscope}%
\begin{pgfscope}%
\pgfpathrectangle{\pgfqpoint{0.100000in}{0.220728in}}{\pgfqpoint{3.696000in}{3.696000in}}%
\pgfusepath{clip}%
\pgfsetbuttcap%
\pgfsetroundjoin%
\definecolor{currentfill}{rgb}{0.121569,0.466667,0.705882}%
\pgfsetfillcolor{currentfill}%
\pgfsetfillopacity{0.745288}%
\pgfsetlinewidth{1.003750pt}%
\definecolor{currentstroke}{rgb}{0.121569,0.466667,0.705882}%
\pgfsetstrokecolor{currentstroke}%
\pgfsetstrokeopacity{0.745288}%
\pgfsetdash{}{0pt}%
\pgfpathmoveto{\pgfqpoint{1.062559in}{1.235962in}}%
\pgfpathcurveto{\pgfqpoint{1.070795in}{1.235962in}}{\pgfqpoint{1.078695in}{1.239234in}}{\pgfqpoint{1.084519in}{1.245058in}}%
\pgfpathcurveto{\pgfqpoint{1.090343in}{1.250882in}}{\pgfqpoint{1.093615in}{1.258782in}}{\pgfqpoint{1.093615in}{1.267018in}}%
\pgfpathcurveto{\pgfqpoint{1.093615in}{1.275255in}}{\pgfqpoint{1.090343in}{1.283155in}}{\pgfqpoint{1.084519in}{1.288979in}}%
\pgfpathcurveto{\pgfqpoint{1.078695in}{1.294803in}}{\pgfqpoint{1.070795in}{1.298075in}}{\pgfqpoint{1.062559in}{1.298075in}}%
\pgfpathcurveto{\pgfqpoint{1.054322in}{1.298075in}}{\pgfqpoint{1.046422in}{1.294803in}}{\pgfqpoint{1.040598in}{1.288979in}}%
\pgfpathcurveto{\pgfqpoint{1.034774in}{1.283155in}}{\pgfqpoint{1.031502in}{1.275255in}}{\pgfqpoint{1.031502in}{1.267018in}}%
\pgfpathcurveto{\pgfqpoint{1.031502in}{1.258782in}}{\pgfqpoint{1.034774in}{1.250882in}}{\pgfqpoint{1.040598in}{1.245058in}}%
\pgfpathcurveto{\pgfqpoint{1.046422in}{1.239234in}}{\pgfqpoint{1.054322in}{1.235962in}}{\pgfqpoint{1.062559in}{1.235962in}}%
\pgfpathclose%
\pgfusepath{stroke,fill}%
\end{pgfscope}%
\begin{pgfscope}%
\pgfpathrectangle{\pgfqpoint{0.100000in}{0.220728in}}{\pgfqpoint{3.696000in}{3.696000in}}%
\pgfusepath{clip}%
\pgfsetbuttcap%
\pgfsetroundjoin%
\definecolor{currentfill}{rgb}{0.121569,0.466667,0.705882}%
\pgfsetfillcolor{currentfill}%
\pgfsetfillopacity{0.745438}%
\pgfsetlinewidth{1.003750pt}%
\definecolor{currentstroke}{rgb}{0.121569,0.466667,0.705882}%
\pgfsetstrokecolor{currentstroke}%
\pgfsetstrokeopacity{0.745438}%
\pgfsetdash{}{0pt}%
\pgfpathmoveto{\pgfqpoint{3.159850in}{2.496999in}}%
\pgfpathcurveto{\pgfqpoint{3.168086in}{2.496999in}}{\pgfqpoint{3.175986in}{2.500271in}}{\pgfqpoint{3.181810in}{2.506095in}}%
\pgfpathcurveto{\pgfqpoint{3.187634in}{2.511919in}}{\pgfqpoint{3.190906in}{2.519819in}}{\pgfqpoint{3.190906in}{2.528055in}}%
\pgfpathcurveto{\pgfqpoint{3.190906in}{2.536292in}}{\pgfqpoint{3.187634in}{2.544192in}}{\pgfqpoint{3.181810in}{2.550015in}}%
\pgfpathcurveto{\pgfqpoint{3.175986in}{2.555839in}}{\pgfqpoint{3.168086in}{2.559112in}}{\pgfqpoint{3.159850in}{2.559112in}}%
\pgfpathcurveto{\pgfqpoint{3.151613in}{2.559112in}}{\pgfqpoint{3.143713in}{2.555839in}}{\pgfqpoint{3.137889in}{2.550015in}}%
\pgfpathcurveto{\pgfqpoint{3.132065in}{2.544192in}}{\pgfqpoint{3.128793in}{2.536292in}}{\pgfqpoint{3.128793in}{2.528055in}}%
\pgfpathcurveto{\pgfqpoint{3.128793in}{2.519819in}}{\pgfqpoint{3.132065in}{2.511919in}}{\pgfqpoint{3.137889in}{2.506095in}}%
\pgfpathcurveto{\pgfqpoint{3.143713in}{2.500271in}}{\pgfqpoint{3.151613in}{2.496999in}}{\pgfqpoint{3.159850in}{2.496999in}}%
\pgfpathclose%
\pgfusepath{stroke,fill}%
\end{pgfscope}%
\begin{pgfscope}%
\pgfpathrectangle{\pgfqpoint{0.100000in}{0.220728in}}{\pgfqpoint{3.696000in}{3.696000in}}%
\pgfusepath{clip}%
\pgfsetbuttcap%
\pgfsetroundjoin%
\definecolor{currentfill}{rgb}{0.121569,0.466667,0.705882}%
\pgfsetfillcolor{currentfill}%
\pgfsetfillopacity{0.746029}%
\pgfsetlinewidth{1.003750pt}%
\definecolor{currentstroke}{rgb}{0.121569,0.466667,0.705882}%
\pgfsetstrokecolor{currentstroke}%
\pgfsetstrokeopacity{0.746029}%
\pgfsetdash{}{0pt}%
\pgfpathmoveto{\pgfqpoint{3.157986in}{2.494151in}}%
\pgfpathcurveto{\pgfqpoint{3.166223in}{2.494151in}}{\pgfqpoint{3.174123in}{2.497423in}}{\pgfqpoint{3.179947in}{2.503247in}}%
\pgfpathcurveto{\pgfqpoint{3.185771in}{2.509071in}}{\pgfqpoint{3.189043in}{2.516971in}}{\pgfqpoint{3.189043in}{2.525207in}}%
\pgfpathcurveto{\pgfqpoint{3.189043in}{2.533444in}}{\pgfqpoint{3.185771in}{2.541344in}}{\pgfqpoint{3.179947in}{2.547168in}}%
\pgfpathcurveto{\pgfqpoint{3.174123in}{2.552991in}}{\pgfqpoint{3.166223in}{2.556264in}}{\pgfqpoint{3.157986in}{2.556264in}}%
\pgfpathcurveto{\pgfqpoint{3.149750in}{2.556264in}}{\pgfqpoint{3.141850in}{2.552991in}}{\pgfqpoint{3.136026in}{2.547168in}}%
\pgfpathcurveto{\pgfqpoint{3.130202in}{2.541344in}}{\pgfqpoint{3.126930in}{2.533444in}}{\pgfqpoint{3.126930in}{2.525207in}}%
\pgfpathcurveto{\pgfqpoint{3.126930in}{2.516971in}}{\pgfqpoint{3.130202in}{2.509071in}}{\pgfqpoint{3.136026in}{2.503247in}}%
\pgfpathcurveto{\pgfqpoint{3.141850in}{2.497423in}}{\pgfqpoint{3.149750in}{2.494151in}}{\pgfqpoint{3.157986in}{2.494151in}}%
\pgfpathclose%
\pgfusepath{stroke,fill}%
\end{pgfscope}%
\begin{pgfscope}%
\pgfpathrectangle{\pgfqpoint{0.100000in}{0.220728in}}{\pgfqpoint{3.696000in}{3.696000in}}%
\pgfusepath{clip}%
\pgfsetbuttcap%
\pgfsetroundjoin%
\definecolor{currentfill}{rgb}{0.121569,0.466667,0.705882}%
\pgfsetfillcolor{currentfill}%
\pgfsetfillopacity{0.746887}%
\pgfsetlinewidth{1.003750pt}%
\definecolor{currentstroke}{rgb}{0.121569,0.466667,0.705882}%
\pgfsetstrokecolor{currentstroke}%
\pgfsetstrokeopacity{0.746887}%
\pgfsetdash{}{0pt}%
\pgfpathmoveto{\pgfqpoint{3.155944in}{2.490126in}}%
\pgfpathcurveto{\pgfqpoint{3.164181in}{2.490126in}}{\pgfqpoint{3.172081in}{2.493399in}}{\pgfqpoint{3.177905in}{2.499222in}}%
\pgfpathcurveto{\pgfqpoint{3.183729in}{2.505046in}}{\pgfqpoint{3.187001in}{2.512946in}}{\pgfqpoint{3.187001in}{2.521183in}}%
\pgfpathcurveto{\pgfqpoint{3.187001in}{2.529419in}}{\pgfqpoint{3.183729in}{2.537319in}}{\pgfqpoint{3.177905in}{2.543143in}}%
\pgfpathcurveto{\pgfqpoint{3.172081in}{2.548967in}}{\pgfqpoint{3.164181in}{2.552239in}}{\pgfqpoint{3.155944in}{2.552239in}}%
\pgfpathcurveto{\pgfqpoint{3.147708in}{2.552239in}}{\pgfqpoint{3.139808in}{2.548967in}}{\pgfqpoint{3.133984in}{2.543143in}}%
\pgfpathcurveto{\pgfqpoint{3.128160in}{2.537319in}}{\pgfqpoint{3.124888in}{2.529419in}}{\pgfqpoint{3.124888in}{2.521183in}}%
\pgfpathcurveto{\pgfqpoint{3.124888in}{2.512946in}}{\pgfqpoint{3.128160in}{2.505046in}}{\pgfqpoint{3.133984in}{2.499222in}}%
\pgfpathcurveto{\pgfqpoint{3.139808in}{2.493399in}}{\pgfqpoint{3.147708in}{2.490126in}}{\pgfqpoint{3.155944in}{2.490126in}}%
\pgfpathclose%
\pgfusepath{stroke,fill}%
\end{pgfscope}%
\begin{pgfscope}%
\pgfpathrectangle{\pgfqpoint{0.100000in}{0.220728in}}{\pgfqpoint{3.696000in}{3.696000in}}%
\pgfusepath{clip}%
\pgfsetbuttcap%
\pgfsetroundjoin%
\definecolor{currentfill}{rgb}{0.121569,0.466667,0.705882}%
\pgfsetfillcolor{currentfill}%
\pgfsetfillopacity{0.747330}%
\pgfsetlinewidth{1.003750pt}%
\definecolor{currentstroke}{rgb}{0.121569,0.466667,0.705882}%
\pgfsetstrokecolor{currentstroke}%
\pgfsetstrokeopacity{0.747330}%
\pgfsetdash{}{0pt}%
\pgfpathmoveto{\pgfqpoint{3.154753in}{2.487867in}}%
\pgfpathcurveto{\pgfqpoint{3.162989in}{2.487867in}}{\pgfqpoint{3.170889in}{2.491139in}}{\pgfqpoint{3.176713in}{2.496963in}}%
\pgfpathcurveto{\pgfqpoint{3.182537in}{2.502787in}}{\pgfqpoint{3.185809in}{2.510687in}}{\pgfqpoint{3.185809in}{2.518923in}}%
\pgfpathcurveto{\pgfqpoint{3.185809in}{2.527159in}}{\pgfqpoint{3.182537in}{2.535059in}}{\pgfqpoint{3.176713in}{2.540883in}}%
\pgfpathcurveto{\pgfqpoint{3.170889in}{2.546707in}}{\pgfqpoint{3.162989in}{2.549980in}}{\pgfqpoint{3.154753in}{2.549980in}}%
\pgfpathcurveto{\pgfqpoint{3.146516in}{2.549980in}}{\pgfqpoint{3.138616in}{2.546707in}}{\pgfqpoint{3.132792in}{2.540883in}}%
\pgfpathcurveto{\pgfqpoint{3.126968in}{2.535059in}}{\pgfqpoint{3.123696in}{2.527159in}}{\pgfqpoint{3.123696in}{2.518923in}}%
\pgfpathcurveto{\pgfqpoint{3.123696in}{2.510687in}}{\pgfqpoint{3.126968in}{2.502787in}}{\pgfqpoint{3.132792in}{2.496963in}}%
\pgfpathcurveto{\pgfqpoint{3.138616in}{2.491139in}}{\pgfqpoint{3.146516in}{2.487867in}}{\pgfqpoint{3.154753in}{2.487867in}}%
\pgfpathclose%
\pgfusepath{stroke,fill}%
\end{pgfscope}%
\begin{pgfscope}%
\pgfpathrectangle{\pgfqpoint{0.100000in}{0.220728in}}{\pgfqpoint{3.696000in}{3.696000in}}%
\pgfusepath{clip}%
\pgfsetbuttcap%
\pgfsetroundjoin%
\definecolor{currentfill}{rgb}{0.121569,0.466667,0.705882}%
\pgfsetfillcolor{currentfill}%
\pgfsetfillopacity{0.747608}%
\pgfsetlinewidth{1.003750pt}%
\definecolor{currentstroke}{rgb}{0.121569,0.466667,0.705882}%
\pgfsetstrokecolor{currentstroke}%
\pgfsetstrokeopacity{0.747608}%
\pgfsetdash{}{0pt}%
\pgfpathmoveto{\pgfqpoint{3.154001in}{2.486909in}}%
\pgfpathcurveto{\pgfqpoint{3.162237in}{2.486909in}}{\pgfqpoint{3.170137in}{2.490182in}}{\pgfqpoint{3.175961in}{2.496006in}}%
\pgfpathcurveto{\pgfqpoint{3.181785in}{2.501830in}}{\pgfqpoint{3.185058in}{2.509730in}}{\pgfqpoint{3.185058in}{2.517966in}}%
\pgfpathcurveto{\pgfqpoint{3.185058in}{2.526202in}}{\pgfqpoint{3.181785in}{2.534102in}}{\pgfqpoint{3.175961in}{2.539926in}}%
\pgfpathcurveto{\pgfqpoint{3.170137in}{2.545750in}}{\pgfqpoint{3.162237in}{2.549022in}}{\pgfqpoint{3.154001in}{2.549022in}}%
\pgfpathcurveto{\pgfqpoint{3.145765in}{2.549022in}}{\pgfqpoint{3.137865in}{2.545750in}}{\pgfqpoint{3.132041in}{2.539926in}}%
\pgfpathcurveto{\pgfqpoint{3.126217in}{2.534102in}}{\pgfqpoint{3.122945in}{2.526202in}}{\pgfqpoint{3.122945in}{2.517966in}}%
\pgfpathcurveto{\pgfqpoint{3.122945in}{2.509730in}}{\pgfqpoint{3.126217in}{2.501830in}}{\pgfqpoint{3.132041in}{2.496006in}}%
\pgfpathcurveto{\pgfqpoint{3.137865in}{2.490182in}}{\pgfqpoint{3.145765in}{2.486909in}}{\pgfqpoint{3.154001in}{2.486909in}}%
\pgfpathclose%
\pgfusepath{stroke,fill}%
\end{pgfscope}%
\begin{pgfscope}%
\pgfpathrectangle{\pgfqpoint{0.100000in}{0.220728in}}{\pgfqpoint{3.696000in}{3.696000in}}%
\pgfusepath{clip}%
\pgfsetbuttcap%
\pgfsetroundjoin%
\definecolor{currentfill}{rgb}{0.121569,0.466667,0.705882}%
\pgfsetfillcolor{currentfill}%
\pgfsetfillopacity{0.748264}%
\pgfsetlinewidth{1.003750pt}%
\definecolor{currentstroke}{rgb}{0.121569,0.466667,0.705882}%
\pgfsetstrokecolor{currentstroke}%
\pgfsetstrokeopacity{0.748264}%
\pgfsetdash{}{0pt}%
\pgfpathmoveto{\pgfqpoint{3.152558in}{2.483186in}}%
\pgfpathcurveto{\pgfqpoint{3.160794in}{2.483186in}}{\pgfqpoint{3.168694in}{2.486458in}}{\pgfqpoint{3.174518in}{2.492282in}}%
\pgfpathcurveto{\pgfqpoint{3.180342in}{2.498106in}}{\pgfqpoint{3.183614in}{2.506006in}}{\pgfqpoint{3.183614in}{2.514242in}}%
\pgfpathcurveto{\pgfqpoint{3.183614in}{2.522478in}}{\pgfqpoint{3.180342in}{2.530378in}}{\pgfqpoint{3.174518in}{2.536202in}}%
\pgfpathcurveto{\pgfqpoint{3.168694in}{2.542026in}}{\pgfqpoint{3.160794in}{2.545299in}}{\pgfqpoint{3.152558in}{2.545299in}}%
\pgfpathcurveto{\pgfqpoint{3.144321in}{2.545299in}}{\pgfqpoint{3.136421in}{2.542026in}}{\pgfqpoint{3.130597in}{2.536202in}}%
\pgfpathcurveto{\pgfqpoint{3.124773in}{2.530378in}}{\pgfqpoint{3.121501in}{2.522478in}}{\pgfqpoint{3.121501in}{2.514242in}}%
\pgfpathcurveto{\pgfqpoint{3.121501in}{2.506006in}}{\pgfqpoint{3.124773in}{2.498106in}}{\pgfqpoint{3.130597in}{2.492282in}}%
\pgfpathcurveto{\pgfqpoint{3.136421in}{2.486458in}}{\pgfqpoint{3.144321in}{2.483186in}}{\pgfqpoint{3.152558in}{2.483186in}}%
\pgfpathclose%
\pgfusepath{stroke,fill}%
\end{pgfscope}%
\begin{pgfscope}%
\pgfpathrectangle{\pgfqpoint{0.100000in}{0.220728in}}{\pgfqpoint{3.696000in}{3.696000in}}%
\pgfusepath{clip}%
\pgfsetbuttcap%
\pgfsetroundjoin%
\definecolor{currentfill}{rgb}{0.121569,0.466667,0.705882}%
\pgfsetfillcolor{currentfill}%
\pgfsetfillopacity{0.748634}%
\pgfsetlinewidth{1.003750pt}%
\definecolor{currentstroke}{rgb}{0.121569,0.466667,0.705882}%
\pgfsetstrokecolor{currentstroke}%
\pgfsetstrokeopacity{0.748634}%
\pgfsetdash{}{0pt}%
\pgfpathmoveto{\pgfqpoint{3.151542in}{2.481425in}}%
\pgfpathcurveto{\pgfqpoint{3.159778in}{2.481425in}}{\pgfqpoint{3.167678in}{2.484697in}}{\pgfqpoint{3.173502in}{2.490521in}}%
\pgfpathcurveto{\pgfqpoint{3.179326in}{2.496345in}}{\pgfqpoint{3.182599in}{2.504245in}}{\pgfqpoint{3.182599in}{2.512481in}}%
\pgfpathcurveto{\pgfqpoint{3.182599in}{2.520718in}}{\pgfqpoint{3.179326in}{2.528618in}}{\pgfqpoint{3.173502in}{2.534442in}}%
\pgfpathcurveto{\pgfqpoint{3.167678in}{2.540266in}}{\pgfqpoint{3.159778in}{2.543538in}}{\pgfqpoint{3.151542in}{2.543538in}}%
\pgfpathcurveto{\pgfqpoint{3.143306in}{2.543538in}}{\pgfqpoint{3.135406in}{2.540266in}}{\pgfqpoint{3.129582in}{2.534442in}}%
\pgfpathcurveto{\pgfqpoint{3.123758in}{2.528618in}}{\pgfqpoint{3.120486in}{2.520718in}}{\pgfqpoint{3.120486in}{2.512481in}}%
\pgfpathcurveto{\pgfqpoint{3.120486in}{2.504245in}}{\pgfqpoint{3.123758in}{2.496345in}}{\pgfqpoint{3.129582in}{2.490521in}}%
\pgfpathcurveto{\pgfqpoint{3.135406in}{2.484697in}}{\pgfqpoint{3.143306in}{2.481425in}}{\pgfqpoint{3.151542in}{2.481425in}}%
\pgfpathclose%
\pgfusepath{stroke,fill}%
\end{pgfscope}%
\begin{pgfscope}%
\pgfpathrectangle{\pgfqpoint{0.100000in}{0.220728in}}{\pgfqpoint{3.696000in}{3.696000in}}%
\pgfusepath{clip}%
\pgfsetbuttcap%
\pgfsetroundjoin%
\definecolor{currentfill}{rgb}{0.121569,0.466667,0.705882}%
\pgfsetfillcolor{currentfill}%
\pgfsetfillopacity{0.748834}%
\pgfsetlinewidth{1.003750pt}%
\definecolor{currentstroke}{rgb}{0.121569,0.466667,0.705882}%
\pgfsetstrokecolor{currentstroke}%
\pgfsetstrokeopacity{0.748834}%
\pgfsetdash{}{0pt}%
\pgfpathmoveto{\pgfqpoint{3.150928in}{2.480523in}}%
\pgfpathcurveto{\pgfqpoint{3.159165in}{2.480523in}}{\pgfqpoint{3.167065in}{2.483796in}}{\pgfqpoint{3.172889in}{2.489619in}}%
\pgfpathcurveto{\pgfqpoint{3.178712in}{2.495443in}}{\pgfqpoint{3.181985in}{2.503343in}}{\pgfqpoint{3.181985in}{2.511580in}}%
\pgfpathcurveto{\pgfqpoint{3.181985in}{2.519816in}}{\pgfqpoint{3.178712in}{2.527716in}}{\pgfqpoint{3.172889in}{2.533540in}}%
\pgfpathcurveto{\pgfqpoint{3.167065in}{2.539364in}}{\pgfqpoint{3.159165in}{2.542636in}}{\pgfqpoint{3.150928in}{2.542636in}}%
\pgfpathcurveto{\pgfqpoint{3.142692in}{2.542636in}}{\pgfqpoint{3.134792in}{2.539364in}}{\pgfqpoint{3.128968in}{2.533540in}}%
\pgfpathcurveto{\pgfqpoint{3.123144in}{2.527716in}}{\pgfqpoint{3.119872in}{2.519816in}}{\pgfqpoint{3.119872in}{2.511580in}}%
\pgfpathcurveto{\pgfqpoint{3.119872in}{2.503343in}}{\pgfqpoint{3.123144in}{2.495443in}}{\pgfqpoint{3.128968in}{2.489619in}}%
\pgfpathcurveto{\pgfqpoint{3.134792in}{2.483796in}}{\pgfqpoint{3.142692in}{2.480523in}}{\pgfqpoint{3.150928in}{2.480523in}}%
\pgfpathclose%
\pgfusepath{stroke,fill}%
\end{pgfscope}%
\begin{pgfscope}%
\pgfpathrectangle{\pgfqpoint{0.100000in}{0.220728in}}{\pgfqpoint{3.696000in}{3.696000in}}%
\pgfusepath{clip}%
\pgfsetbuttcap%
\pgfsetroundjoin%
\definecolor{currentfill}{rgb}{0.121569,0.466667,0.705882}%
\pgfsetfillcolor{currentfill}%
\pgfsetfillopacity{0.749170}%
\pgfsetlinewidth{1.003750pt}%
\definecolor{currentstroke}{rgb}{0.121569,0.466667,0.705882}%
\pgfsetstrokecolor{currentstroke}%
\pgfsetstrokeopacity{0.749170}%
\pgfsetdash{}{0pt}%
\pgfpathmoveto{\pgfqpoint{3.150260in}{2.478885in}}%
\pgfpathcurveto{\pgfqpoint{3.158497in}{2.478885in}}{\pgfqpoint{3.166397in}{2.482157in}}{\pgfqpoint{3.172221in}{2.487981in}}%
\pgfpathcurveto{\pgfqpoint{3.178045in}{2.493805in}}{\pgfqpoint{3.181317in}{2.501705in}}{\pgfqpoint{3.181317in}{2.509941in}}%
\pgfpathcurveto{\pgfqpoint{3.181317in}{2.518177in}}{\pgfqpoint{3.178045in}{2.526077in}}{\pgfqpoint{3.172221in}{2.531901in}}%
\pgfpathcurveto{\pgfqpoint{3.166397in}{2.537725in}}{\pgfqpoint{3.158497in}{2.540998in}}{\pgfqpoint{3.150260in}{2.540998in}}%
\pgfpathcurveto{\pgfqpoint{3.142024in}{2.540998in}}{\pgfqpoint{3.134124in}{2.537725in}}{\pgfqpoint{3.128300in}{2.531901in}}%
\pgfpathcurveto{\pgfqpoint{3.122476in}{2.526077in}}{\pgfqpoint{3.119204in}{2.518177in}}{\pgfqpoint{3.119204in}{2.509941in}}%
\pgfpathcurveto{\pgfqpoint{3.119204in}{2.501705in}}{\pgfqpoint{3.122476in}{2.493805in}}{\pgfqpoint{3.128300in}{2.487981in}}%
\pgfpathcurveto{\pgfqpoint{3.134124in}{2.482157in}}{\pgfqpoint{3.142024in}{2.478885in}}{\pgfqpoint{3.150260in}{2.478885in}}%
\pgfpathclose%
\pgfusepath{stroke,fill}%
\end{pgfscope}%
\begin{pgfscope}%
\pgfpathrectangle{\pgfqpoint{0.100000in}{0.220728in}}{\pgfqpoint{3.696000in}{3.696000in}}%
\pgfusepath{clip}%
\pgfsetbuttcap%
\pgfsetroundjoin%
\definecolor{currentfill}{rgb}{0.121569,0.466667,0.705882}%
\pgfsetfillcolor{currentfill}%
\pgfsetfillopacity{0.749706}%
\pgfsetlinewidth{1.003750pt}%
\definecolor{currentstroke}{rgb}{0.121569,0.466667,0.705882}%
\pgfsetstrokecolor{currentstroke}%
\pgfsetstrokeopacity{0.749706}%
\pgfsetdash{}{0pt}%
\pgfpathmoveto{\pgfqpoint{3.148227in}{2.475974in}}%
\pgfpathcurveto{\pgfqpoint{3.156463in}{2.475974in}}{\pgfqpoint{3.164363in}{2.479246in}}{\pgfqpoint{3.170187in}{2.485070in}}%
\pgfpathcurveto{\pgfqpoint{3.176011in}{2.490894in}}{\pgfqpoint{3.179283in}{2.498794in}}{\pgfqpoint{3.179283in}{2.507030in}}%
\pgfpathcurveto{\pgfqpoint{3.179283in}{2.515266in}}{\pgfqpoint{3.176011in}{2.523166in}}{\pgfqpoint{3.170187in}{2.528990in}}%
\pgfpathcurveto{\pgfqpoint{3.164363in}{2.534814in}}{\pgfqpoint{3.156463in}{2.538087in}}{\pgfqpoint{3.148227in}{2.538087in}}%
\pgfpathcurveto{\pgfqpoint{3.139991in}{2.538087in}}{\pgfqpoint{3.132090in}{2.534814in}}{\pgfqpoint{3.126267in}{2.528990in}}%
\pgfpathcurveto{\pgfqpoint{3.120443in}{2.523166in}}{\pgfqpoint{3.117170in}{2.515266in}}{\pgfqpoint{3.117170in}{2.507030in}}%
\pgfpathcurveto{\pgfqpoint{3.117170in}{2.498794in}}{\pgfqpoint{3.120443in}{2.490894in}}{\pgfqpoint{3.126267in}{2.485070in}}%
\pgfpathcurveto{\pgfqpoint{3.132090in}{2.479246in}}{\pgfqpoint{3.139991in}{2.475974in}}{\pgfqpoint{3.148227in}{2.475974in}}%
\pgfpathclose%
\pgfusepath{stroke,fill}%
\end{pgfscope}%
\begin{pgfscope}%
\pgfpathrectangle{\pgfqpoint{0.100000in}{0.220728in}}{\pgfqpoint{3.696000in}{3.696000in}}%
\pgfusepath{clip}%
\pgfsetbuttcap%
\pgfsetroundjoin%
\definecolor{currentfill}{rgb}{0.121569,0.466667,0.705882}%
\pgfsetfillcolor{currentfill}%
\pgfsetfillopacity{0.750451}%
\pgfsetlinewidth{1.003750pt}%
\definecolor{currentstroke}{rgb}{0.121569,0.466667,0.705882}%
\pgfsetstrokecolor{currentstroke}%
\pgfsetstrokeopacity{0.750451}%
\pgfsetdash{}{0pt}%
\pgfpathmoveto{\pgfqpoint{1.081876in}{1.232821in}}%
\pgfpathcurveto{\pgfqpoint{1.090112in}{1.232821in}}{\pgfqpoint{1.098012in}{1.236094in}}{\pgfqpoint{1.103836in}{1.241918in}}%
\pgfpathcurveto{\pgfqpoint{1.109660in}{1.247742in}}{\pgfqpoint{1.112932in}{1.255642in}}{\pgfqpoint{1.112932in}{1.263878in}}%
\pgfpathcurveto{\pgfqpoint{1.112932in}{1.272114in}}{\pgfqpoint{1.109660in}{1.280014in}}{\pgfqpoint{1.103836in}{1.285838in}}%
\pgfpathcurveto{\pgfqpoint{1.098012in}{1.291662in}}{\pgfqpoint{1.090112in}{1.294934in}}{\pgfqpoint{1.081876in}{1.294934in}}%
\pgfpathcurveto{\pgfqpoint{1.073639in}{1.294934in}}{\pgfqpoint{1.065739in}{1.291662in}}{\pgfqpoint{1.059915in}{1.285838in}}%
\pgfpathcurveto{\pgfqpoint{1.054092in}{1.280014in}}{\pgfqpoint{1.050819in}{1.272114in}}{\pgfqpoint{1.050819in}{1.263878in}}%
\pgfpathcurveto{\pgfqpoint{1.050819in}{1.255642in}}{\pgfqpoint{1.054092in}{1.247742in}}{\pgfqpoint{1.059915in}{1.241918in}}%
\pgfpathcurveto{\pgfqpoint{1.065739in}{1.236094in}}{\pgfqpoint{1.073639in}{1.232821in}}{\pgfqpoint{1.081876in}{1.232821in}}%
\pgfpathclose%
\pgfusepath{stroke,fill}%
\end{pgfscope}%
\begin{pgfscope}%
\pgfpathrectangle{\pgfqpoint{0.100000in}{0.220728in}}{\pgfqpoint{3.696000in}{3.696000in}}%
\pgfusepath{clip}%
\pgfsetbuttcap%
\pgfsetroundjoin%
\definecolor{currentfill}{rgb}{0.121569,0.466667,0.705882}%
\pgfsetfillcolor{currentfill}%
\pgfsetfillopacity{0.750647}%
\pgfsetlinewidth{1.003750pt}%
\definecolor{currentstroke}{rgb}{0.121569,0.466667,0.705882}%
\pgfsetstrokecolor{currentstroke}%
\pgfsetstrokeopacity{0.750647}%
\pgfsetdash{}{0pt}%
\pgfpathmoveto{\pgfqpoint{3.146583in}{2.471784in}}%
\pgfpathcurveto{\pgfqpoint{3.154819in}{2.471784in}}{\pgfqpoint{3.162719in}{2.475056in}}{\pgfqpoint{3.168543in}{2.480880in}}%
\pgfpathcurveto{\pgfqpoint{3.174367in}{2.486704in}}{\pgfqpoint{3.177639in}{2.494604in}}{\pgfqpoint{3.177639in}{2.502841in}}%
\pgfpathcurveto{\pgfqpoint{3.177639in}{2.511077in}}{\pgfqpoint{3.174367in}{2.518977in}}{\pgfqpoint{3.168543in}{2.524801in}}%
\pgfpathcurveto{\pgfqpoint{3.162719in}{2.530625in}}{\pgfqpoint{3.154819in}{2.533897in}}{\pgfqpoint{3.146583in}{2.533897in}}%
\pgfpathcurveto{\pgfqpoint{3.138347in}{2.533897in}}{\pgfqpoint{3.130446in}{2.530625in}}{\pgfqpoint{3.124623in}{2.524801in}}%
\pgfpathcurveto{\pgfqpoint{3.118799in}{2.518977in}}{\pgfqpoint{3.115526in}{2.511077in}}{\pgfqpoint{3.115526in}{2.502841in}}%
\pgfpathcurveto{\pgfqpoint{3.115526in}{2.494604in}}{\pgfqpoint{3.118799in}{2.486704in}}{\pgfqpoint{3.124623in}{2.480880in}}%
\pgfpathcurveto{\pgfqpoint{3.130446in}{2.475056in}}{\pgfqpoint{3.138347in}{2.471784in}}{\pgfqpoint{3.146583in}{2.471784in}}%
\pgfpathclose%
\pgfusepath{stroke,fill}%
\end{pgfscope}%
\begin{pgfscope}%
\pgfpathrectangle{\pgfqpoint{0.100000in}{0.220728in}}{\pgfqpoint{3.696000in}{3.696000in}}%
\pgfusepath{clip}%
\pgfsetbuttcap%
\pgfsetroundjoin%
\definecolor{currentfill}{rgb}{0.121569,0.466667,0.705882}%
\pgfsetfillcolor{currentfill}%
\pgfsetfillopacity{0.751106}%
\pgfsetlinewidth{1.003750pt}%
\definecolor{currentstroke}{rgb}{0.121569,0.466667,0.705882}%
\pgfsetstrokecolor{currentstroke}%
\pgfsetstrokeopacity{0.751106}%
\pgfsetdash{}{0pt}%
\pgfpathmoveto{\pgfqpoint{3.145396in}{2.469518in}}%
\pgfpathcurveto{\pgfqpoint{3.153633in}{2.469518in}}{\pgfqpoint{3.161533in}{2.472791in}}{\pgfqpoint{3.167357in}{2.478615in}}%
\pgfpathcurveto{\pgfqpoint{3.173181in}{2.484439in}}{\pgfqpoint{3.176453in}{2.492339in}}{\pgfqpoint{3.176453in}{2.500575in}}%
\pgfpathcurveto{\pgfqpoint{3.176453in}{2.508811in}}{\pgfqpoint{3.173181in}{2.516711in}}{\pgfqpoint{3.167357in}{2.522535in}}%
\pgfpathcurveto{\pgfqpoint{3.161533in}{2.528359in}}{\pgfqpoint{3.153633in}{2.531631in}}{\pgfqpoint{3.145396in}{2.531631in}}%
\pgfpathcurveto{\pgfqpoint{3.137160in}{2.531631in}}{\pgfqpoint{3.129260in}{2.528359in}}{\pgfqpoint{3.123436in}{2.522535in}}%
\pgfpathcurveto{\pgfqpoint{3.117612in}{2.516711in}}{\pgfqpoint{3.114340in}{2.508811in}}{\pgfqpoint{3.114340in}{2.500575in}}%
\pgfpathcurveto{\pgfqpoint{3.114340in}{2.492339in}}{\pgfqpoint{3.117612in}{2.484439in}}{\pgfqpoint{3.123436in}{2.478615in}}%
\pgfpathcurveto{\pgfqpoint{3.129260in}{2.472791in}}{\pgfqpoint{3.137160in}{2.469518in}}{\pgfqpoint{3.145396in}{2.469518in}}%
\pgfpathclose%
\pgfusepath{stroke,fill}%
\end{pgfscope}%
\begin{pgfscope}%
\pgfpathrectangle{\pgfqpoint{0.100000in}{0.220728in}}{\pgfqpoint{3.696000in}{3.696000in}}%
\pgfusepath{clip}%
\pgfsetbuttcap%
\pgfsetroundjoin%
\definecolor{currentfill}{rgb}{0.121569,0.466667,0.705882}%
\pgfsetfillcolor{currentfill}%
\pgfsetfillopacity{0.751343}%
\pgfsetlinewidth{1.003750pt}%
\definecolor{currentstroke}{rgb}{0.121569,0.466667,0.705882}%
\pgfsetstrokecolor{currentstroke}%
\pgfsetstrokeopacity{0.751343}%
\pgfsetdash{}{0pt}%
\pgfpathmoveto{\pgfqpoint{3.144633in}{2.468354in}}%
\pgfpathcurveto{\pgfqpoint{3.152870in}{2.468354in}}{\pgfqpoint{3.160770in}{2.471627in}}{\pgfqpoint{3.166594in}{2.477450in}}%
\pgfpathcurveto{\pgfqpoint{3.172418in}{2.483274in}}{\pgfqpoint{3.175690in}{2.491174in}}{\pgfqpoint{3.175690in}{2.499411in}}%
\pgfpathcurveto{\pgfqpoint{3.175690in}{2.507647in}}{\pgfqpoint{3.172418in}{2.515547in}}{\pgfqpoint{3.166594in}{2.521371in}}%
\pgfpathcurveto{\pgfqpoint{3.160770in}{2.527195in}}{\pgfqpoint{3.152870in}{2.530467in}}{\pgfqpoint{3.144633in}{2.530467in}}%
\pgfpathcurveto{\pgfqpoint{3.136397in}{2.530467in}}{\pgfqpoint{3.128497in}{2.527195in}}{\pgfqpoint{3.122673in}{2.521371in}}%
\pgfpathcurveto{\pgfqpoint{3.116849in}{2.515547in}}{\pgfqpoint{3.113577in}{2.507647in}}{\pgfqpoint{3.113577in}{2.499411in}}%
\pgfpathcurveto{\pgfqpoint{3.113577in}{2.491174in}}{\pgfqpoint{3.116849in}{2.483274in}}{\pgfqpoint{3.122673in}{2.477450in}}%
\pgfpathcurveto{\pgfqpoint{3.128497in}{2.471627in}}{\pgfqpoint{3.136397in}{2.468354in}}{\pgfqpoint{3.144633in}{2.468354in}}%
\pgfpathclose%
\pgfusepath{stroke,fill}%
\end{pgfscope}%
\begin{pgfscope}%
\pgfpathrectangle{\pgfqpoint{0.100000in}{0.220728in}}{\pgfqpoint{3.696000in}{3.696000in}}%
\pgfusepath{clip}%
\pgfsetbuttcap%
\pgfsetroundjoin%
\definecolor{currentfill}{rgb}{0.121569,0.466667,0.705882}%
\pgfsetfillcolor{currentfill}%
\pgfsetfillopacity{0.751476}%
\pgfsetlinewidth{1.003750pt}%
\definecolor{currentstroke}{rgb}{0.121569,0.466667,0.705882}%
\pgfsetstrokecolor{currentstroke}%
\pgfsetstrokeopacity{0.751476}%
\pgfsetdash{}{0pt}%
\pgfpathmoveto{\pgfqpoint{3.144312in}{2.467601in}}%
\pgfpathcurveto{\pgfqpoint{3.152549in}{2.467601in}}{\pgfqpoint{3.160449in}{2.470874in}}{\pgfqpoint{3.166273in}{2.476698in}}%
\pgfpathcurveto{\pgfqpoint{3.172097in}{2.482522in}}{\pgfqpoint{3.175369in}{2.490422in}}{\pgfqpoint{3.175369in}{2.498658in}}%
\pgfpathcurveto{\pgfqpoint{3.175369in}{2.506894in}}{\pgfqpoint{3.172097in}{2.514794in}}{\pgfqpoint{3.166273in}{2.520618in}}%
\pgfpathcurveto{\pgfqpoint{3.160449in}{2.526442in}}{\pgfqpoint{3.152549in}{2.529714in}}{\pgfqpoint{3.144312in}{2.529714in}}%
\pgfpathcurveto{\pgfqpoint{3.136076in}{2.529714in}}{\pgfqpoint{3.128176in}{2.526442in}}{\pgfqpoint{3.122352in}{2.520618in}}%
\pgfpathcurveto{\pgfqpoint{3.116528in}{2.514794in}}{\pgfqpoint{3.113256in}{2.506894in}}{\pgfqpoint{3.113256in}{2.498658in}}%
\pgfpathcurveto{\pgfqpoint{3.113256in}{2.490422in}}{\pgfqpoint{3.116528in}{2.482522in}}{\pgfqpoint{3.122352in}{2.476698in}}%
\pgfpathcurveto{\pgfqpoint{3.128176in}{2.470874in}}{\pgfqpoint{3.136076in}{2.467601in}}{\pgfqpoint{3.144312in}{2.467601in}}%
\pgfpathclose%
\pgfusepath{stroke,fill}%
\end{pgfscope}%
\begin{pgfscope}%
\pgfpathrectangle{\pgfqpoint{0.100000in}{0.220728in}}{\pgfqpoint{3.696000in}{3.696000in}}%
\pgfusepath{clip}%
\pgfsetbuttcap%
\pgfsetroundjoin%
\definecolor{currentfill}{rgb}{0.121569,0.466667,0.705882}%
\pgfsetfillcolor{currentfill}%
\pgfsetfillopacity{0.751721}%
\pgfsetlinewidth{1.003750pt}%
\definecolor{currentstroke}{rgb}{0.121569,0.466667,0.705882}%
\pgfsetstrokecolor{currentstroke}%
\pgfsetstrokeopacity{0.751721}%
\pgfsetdash{}{0pt}%
\pgfpathmoveto{\pgfqpoint{3.143301in}{2.466091in}}%
\pgfpathcurveto{\pgfqpoint{3.151537in}{2.466091in}}{\pgfqpoint{3.159437in}{2.469363in}}{\pgfqpoint{3.165261in}{2.475187in}}%
\pgfpathcurveto{\pgfqpoint{3.171085in}{2.481011in}}{\pgfqpoint{3.174358in}{2.488911in}}{\pgfqpoint{3.174358in}{2.497147in}}%
\pgfpathcurveto{\pgfqpoint{3.174358in}{2.505384in}}{\pgfqpoint{3.171085in}{2.513284in}}{\pgfqpoint{3.165261in}{2.519108in}}%
\pgfpathcurveto{\pgfqpoint{3.159437in}{2.524932in}}{\pgfqpoint{3.151537in}{2.528204in}}{\pgfqpoint{3.143301in}{2.528204in}}%
\pgfpathcurveto{\pgfqpoint{3.135065in}{2.528204in}}{\pgfqpoint{3.127165in}{2.524932in}}{\pgfqpoint{3.121341in}{2.519108in}}%
\pgfpathcurveto{\pgfqpoint{3.115517in}{2.513284in}}{\pgfqpoint{3.112245in}{2.505384in}}{\pgfqpoint{3.112245in}{2.497147in}}%
\pgfpathcurveto{\pgfqpoint{3.112245in}{2.488911in}}{\pgfqpoint{3.115517in}{2.481011in}}{\pgfqpoint{3.121341in}{2.475187in}}%
\pgfpathcurveto{\pgfqpoint{3.127165in}{2.469363in}}{\pgfqpoint{3.135065in}{2.466091in}}{\pgfqpoint{3.143301in}{2.466091in}}%
\pgfpathclose%
\pgfusepath{stroke,fill}%
\end{pgfscope}%
\begin{pgfscope}%
\pgfpathrectangle{\pgfqpoint{0.100000in}{0.220728in}}{\pgfqpoint{3.696000in}{3.696000in}}%
\pgfusepath{clip}%
\pgfsetbuttcap%
\pgfsetroundjoin%
\definecolor{currentfill}{rgb}{0.121569,0.466667,0.705882}%
\pgfsetfillcolor{currentfill}%
\pgfsetfillopacity{0.751886}%
\pgfsetlinewidth{1.003750pt}%
\definecolor{currentstroke}{rgb}{0.121569,0.466667,0.705882}%
\pgfsetstrokecolor{currentstroke}%
\pgfsetstrokeopacity{0.751886}%
\pgfsetdash{}{0pt}%
\pgfpathmoveto{\pgfqpoint{3.142870in}{2.465209in}}%
\pgfpathcurveto{\pgfqpoint{3.151106in}{2.465209in}}{\pgfqpoint{3.159006in}{2.468482in}}{\pgfqpoint{3.164830in}{2.474306in}}%
\pgfpathcurveto{\pgfqpoint{3.170654in}{2.480129in}}{\pgfqpoint{3.173926in}{2.488030in}}{\pgfqpoint{3.173926in}{2.496266in}}%
\pgfpathcurveto{\pgfqpoint{3.173926in}{2.504502in}}{\pgfqpoint{3.170654in}{2.512402in}}{\pgfqpoint{3.164830in}{2.518226in}}%
\pgfpathcurveto{\pgfqpoint{3.159006in}{2.524050in}}{\pgfqpoint{3.151106in}{2.527322in}}{\pgfqpoint{3.142870in}{2.527322in}}%
\pgfpathcurveto{\pgfqpoint{3.134633in}{2.527322in}}{\pgfqpoint{3.126733in}{2.524050in}}{\pgfqpoint{3.120909in}{2.518226in}}%
\pgfpathcurveto{\pgfqpoint{3.115085in}{2.512402in}}{\pgfqpoint{3.111813in}{2.504502in}}{\pgfqpoint{3.111813in}{2.496266in}}%
\pgfpathcurveto{\pgfqpoint{3.111813in}{2.488030in}}{\pgfqpoint{3.115085in}{2.480129in}}{\pgfqpoint{3.120909in}{2.474306in}}%
\pgfpathcurveto{\pgfqpoint{3.126733in}{2.468482in}}{\pgfqpoint{3.134633in}{2.465209in}}{\pgfqpoint{3.142870in}{2.465209in}}%
\pgfpathclose%
\pgfusepath{stroke,fill}%
\end{pgfscope}%
\begin{pgfscope}%
\pgfpathrectangle{\pgfqpoint{0.100000in}{0.220728in}}{\pgfqpoint{3.696000in}{3.696000in}}%
\pgfusepath{clip}%
\pgfsetbuttcap%
\pgfsetroundjoin%
\definecolor{currentfill}{rgb}{0.121569,0.466667,0.705882}%
\pgfsetfillcolor{currentfill}%
\pgfsetfillopacity{0.751977}%
\pgfsetlinewidth{1.003750pt}%
\definecolor{currentstroke}{rgb}{0.121569,0.466667,0.705882}%
\pgfsetstrokecolor{currentstroke}%
\pgfsetstrokeopacity{0.751977}%
\pgfsetdash{}{0pt}%
\pgfpathmoveto{\pgfqpoint{3.142605in}{2.464761in}}%
\pgfpathcurveto{\pgfqpoint{3.150842in}{2.464761in}}{\pgfqpoint{3.158742in}{2.468033in}}{\pgfqpoint{3.164566in}{2.473857in}}%
\pgfpathcurveto{\pgfqpoint{3.170389in}{2.479681in}}{\pgfqpoint{3.173662in}{2.487581in}}{\pgfqpoint{3.173662in}{2.495817in}}%
\pgfpathcurveto{\pgfqpoint{3.173662in}{2.504054in}}{\pgfqpoint{3.170389in}{2.511954in}}{\pgfqpoint{3.164566in}{2.517777in}}%
\pgfpathcurveto{\pgfqpoint{3.158742in}{2.523601in}}{\pgfqpoint{3.150842in}{2.526874in}}{\pgfqpoint{3.142605in}{2.526874in}}%
\pgfpathcurveto{\pgfqpoint{3.134369in}{2.526874in}}{\pgfqpoint{3.126469in}{2.523601in}}{\pgfqpoint{3.120645in}{2.517777in}}%
\pgfpathcurveto{\pgfqpoint{3.114821in}{2.511954in}}{\pgfqpoint{3.111549in}{2.504054in}}{\pgfqpoint{3.111549in}{2.495817in}}%
\pgfpathcurveto{\pgfqpoint{3.111549in}{2.487581in}}{\pgfqpoint{3.114821in}{2.479681in}}{\pgfqpoint{3.120645in}{2.473857in}}%
\pgfpathcurveto{\pgfqpoint{3.126469in}{2.468033in}}{\pgfqpoint{3.134369in}{2.464761in}}{\pgfqpoint{3.142605in}{2.464761in}}%
\pgfpathclose%
\pgfusepath{stroke,fill}%
\end{pgfscope}%
\begin{pgfscope}%
\pgfpathrectangle{\pgfqpoint{0.100000in}{0.220728in}}{\pgfqpoint{3.696000in}{3.696000in}}%
\pgfusepath{clip}%
\pgfsetbuttcap%
\pgfsetroundjoin%
\definecolor{currentfill}{rgb}{0.121569,0.466667,0.705882}%
\pgfsetfillcolor{currentfill}%
\pgfsetfillopacity{0.752030}%
\pgfsetlinewidth{1.003750pt}%
\definecolor{currentstroke}{rgb}{0.121569,0.466667,0.705882}%
\pgfsetstrokecolor{currentstroke}%
\pgfsetstrokeopacity{0.752030}%
\pgfsetdash{}{0pt}%
\pgfpathmoveto{\pgfqpoint{3.142445in}{2.464547in}}%
\pgfpathcurveto{\pgfqpoint{3.150682in}{2.464547in}}{\pgfqpoint{3.158582in}{2.467820in}}{\pgfqpoint{3.164406in}{2.473644in}}%
\pgfpathcurveto{\pgfqpoint{3.170229in}{2.479468in}}{\pgfqpoint{3.173502in}{2.487368in}}{\pgfqpoint{3.173502in}{2.495604in}}%
\pgfpathcurveto{\pgfqpoint{3.173502in}{2.503840in}}{\pgfqpoint{3.170229in}{2.511740in}}{\pgfqpoint{3.164406in}{2.517564in}}%
\pgfpathcurveto{\pgfqpoint{3.158582in}{2.523388in}}{\pgfqpoint{3.150682in}{2.526660in}}{\pgfqpoint{3.142445in}{2.526660in}}%
\pgfpathcurveto{\pgfqpoint{3.134209in}{2.526660in}}{\pgfqpoint{3.126309in}{2.523388in}}{\pgfqpoint{3.120485in}{2.517564in}}%
\pgfpathcurveto{\pgfqpoint{3.114661in}{2.511740in}}{\pgfqpoint{3.111389in}{2.503840in}}{\pgfqpoint{3.111389in}{2.495604in}}%
\pgfpathcurveto{\pgfqpoint{3.111389in}{2.487368in}}{\pgfqpoint{3.114661in}{2.479468in}}{\pgfqpoint{3.120485in}{2.473644in}}%
\pgfpathcurveto{\pgfqpoint{3.126309in}{2.467820in}}{\pgfqpoint{3.134209in}{2.464547in}}{\pgfqpoint{3.142445in}{2.464547in}}%
\pgfpathclose%
\pgfusepath{stroke,fill}%
\end{pgfscope}%
\begin{pgfscope}%
\pgfpathrectangle{\pgfqpoint{0.100000in}{0.220728in}}{\pgfqpoint{3.696000in}{3.696000in}}%
\pgfusepath{clip}%
\pgfsetbuttcap%
\pgfsetroundjoin%
\definecolor{currentfill}{rgb}{0.121569,0.466667,0.705882}%
\pgfsetfillcolor{currentfill}%
\pgfsetfillopacity{0.752060}%
\pgfsetlinewidth{1.003750pt}%
\definecolor{currentstroke}{rgb}{0.121569,0.466667,0.705882}%
\pgfsetstrokecolor{currentstroke}%
\pgfsetstrokeopacity{0.752060}%
\pgfsetdash{}{0pt}%
\pgfpathmoveto{\pgfqpoint{3.142385in}{2.464399in}}%
\pgfpathcurveto{\pgfqpoint{3.150621in}{2.464399in}}{\pgfqpoint{3.158521in}{2.467671in}}{\pgfqpoint{3.164345in}{2.473495in}}%
\pgfpathcurveto{\pgfqpoint{3.170169in}{2.479319in}}{\pgfqpoint{3.173441in}{2.487219in}}{\pgfqpoint{3.173441in}{2.495456in}}%
\pgfpathcurveto{\pgfqpoint{3.173441in}{2.503692in}}{\pgfqpoint{3.170169in}{2.511592in}}{\pgfqpoint{3.164345in}{2.517416in}}%
\pgfpathcurveto{\pgfqpoint{3.158521in}{2.523240in}}{\pgfqpoint{3.150621in}{2.526512in}}{\pgfqpoint{3.142385in}{2.526512in}}%
\pgfpathcurveto{\pgfqpoint{3.134149in}{2.526512in}}{\pgfqpoint{3.126249in}{2.523240in}}{\pgfqpoint{3.120425in}{2.517416in}}%
\pgfpathcurveto{\pgfqpoint{3.114601in}{2.511592in}}{\pgfqpoint{3.111328in}{2.503692in}}{\pgfqpoint{3.111328in}{2.495456in}}%
\pgfpathcurveto{\pgfqpoint{3.111328in}{2.487219in}}{\pgfqpoint{3.114601in}{2.479319in}}{\pgfqpoint{3.120425in}{2.473495in}}%
\pgfpathcurveto{\pgfqpoint{3.126249in}{2.467671in}}{\pgfqpoint{3.134149in}{2.464399in}}{\pgfqpoint{3.142385in}{2.464399in}}%
\pgfpathclose%
\pgfusepath{stroke,fill}%
\end{pgfscope}%
\begin{pgfscope}%
\pgfpathrectangle{\pgfqpoint{0.100000in}{0.220728in}}{\pgfqpoint{3.696000in}{3.696000in}}%
\pgfusepath{clip}%
\pgfsetbuttcap%
\pgfsetroundjoin%
\definecolor{currentfill}{rgb}{0.121569,0.466667,0.705882}%
\pgfsetfillcolor{currentfill}%
\pgfsetfillopacity{0.752073}%
\pgfsetlinewidth{1.003750pt}%
\definecolor{currentstroke}{rgb}{0.121569,0.466667,0.705882}%
\pgfsetstrokecolor{currentstroke}%
\pgfsetstrokeopacity{0.752073}%
\pgfsetdash{}{0pt}%
\pgfpathmoveto{\pgfqpoint{3.142335in}{2.464324in}}%
\pgfpathcurveto{\pgfqpoint{3.150571in}{2.464324in}}{\pgfqpoint{3.158471in}{2.467597in}}{\pgfqpoint{3.164295in}{2.473421in}}%
\pgfpathcurveto{\pgfqpoint{3.170119in}{2.479245in}}{\pgfqpoint{3.173391in}{2.487145in}}{\pgfqpoint{3.173391in}{2.495381in}}%
\pgfpathcurveto{\pgfqpoint{3.173391in}{2.503617in}}{\pgfqpoint{3.170119in}{2.511517in}}{\pgfqpoint{3.164295in}{2.517341in}}%
\pgfpathcurveto{\pgfqpoint{3.158471in}{2.523165in}}{\pgfqpoint{3.150571in}{2.526437in}}{\pgfqpoint{3.142335in}{2.526437in}}%
\pgfpathcurveto{\pgfqpoint{3.134099in}{2.526437in}}{\pgfqpoint{3.126199in}{2.523165in}}{\pgfqpoint{3.120375in}{2.517341in}}%
\pgfpathcurveto{\pgfqpoint{3.114551in}{2.511517in}}{\pgfqpoint{3.111278in}{2.503617in}}{\pgfqpoint{3.111278in}{2.495381in}}%
\pgfpathcurveto{\pgfqpoint{3.111278in}{2.487145in}}{\pgfqpoint{3.114551in}{2.479245in}}{\pgfqpoint{3.120375in}{2.473421in}}%
\pgfpathcurveto{\pgfqpoint{3.126199in}{2.467597in}}{\pgfqpoint{3.134099in}{2.464324in}}{\pgfqpoint{3.142335in}{2.464324in}}%
\pgfpathclose%
\pgfusepath{stroke,fill}%
\end{pgfscope}%
\begin{pgfscope}%
\pgfpathrectangle{\pgfqpoint{0.100000in}{0.220728in}}{\pgfqpoint{3.696000in}{3.696000in}}%
\pgfusepath{clip}%
\pgfsetbuttcap%
\pgfsetroundjoin%
\definecolor{currentfill}{rgb}{0.121569,0.466667,0.705882}%
\pgfsetfillcolor{currentfill}%
\pgfsetfillopacity{0.752081}%
\pgfsetlinewidth{1.003750pt}%
\definecolor{currentstroke}{rgb}{0.121569,0.466667,0.705882}%
\pgfsetstrokecolor{currentstroke}%
\pgfsetstrokeopacity{0.752081}%
\pgfsetdash{}{0pt}%
\pgfpathmoveto{\pgfqpoint{3.142311in}{2.464285in}}%
\pgfpathcurveto{\pgfqpoint{3.150547in}{2.464285in}}{\pgfqpoint{3.158447in}{2.467557in}}{\pgfqpoint{3.164271in}{2.473381in}}%
\pgfpathcurveto{\pgfqpoint{3.170095in}{2.479205in}}{\pgfqpoint{3.173367in}{2.487105in}}{\pgfqpoint{3.173367in}{2.495341in}}%
\pgfpathcurveto{\pgfqpoint{3.173367in}{2.503578in}}{\pgfqpoint{3.170095in}{2.511478in}}{\pgfqpoint{3.164271in}{2.517302in}}%
\pgfpathcurveto{\pgfqpoint{3.158447in}{2.523126in}}{\pgfqpoint{3.150547in}{2.526398in}}{\pgfqpoint{3.142311in}{2.526398in}}%
\pgfpathcurveto{\pgfqpoint{3.134074in}{2.526398in}}{\pgfqpoint{3.126174in}{2.523126in}}{\pgfqpoint{3.120350in}{2.517302in}}%
\pgfpathcurveto{\pgfqpoint{3.114526in}{2.511478in}}{\pgfqpoint{3.111254in}{2.503578in}}{\pgfqpoint{3.111254in}{2.495341in}}%
\pgfpathcurveto{\pgfqpoint{3.111254in}{2.487105in}}{\pgfqpoint{3.114526in}{2.479205in}}{\pgfqpoint{3.120350in}{2.473381in}}%
\pgfpathcurveto{\pgfqpoint{3.126174in}{2.467557in}}{\pgfqpoint{3.134074in}{2.464285in}}{\pgfqpoint{3.142311in}{2.464285in}}%
\pgfpathclose%
\pgfusepath{stroke,fill}%
\end{pgfscope}%
\begin{pgfscope}%
\pgfpathrectangle{\pgfqpoint{0.100000in}{0.220728in}}{\pgfqpoint{3.696000in}{3.696000in}}%
\pgfusepath{clip}%
\pgfsetbuttcap%
\pgfsetroundjoin%
\definecolor{currentfill}{rgb}{0.121569,0.466667,0.705882}%
\pgfsetfillcolor{currentfill}%
\pgfsetfillopacity{0.752086}%
\pgfsetlinewidth{1.003750pt}%
\definecolor{currentstroke}{rgb}{0.121569,0.466667,0.705882}%
\pgfsetstrokecolor{currentstroke}%
\pgfsetstrokeopacity{0.752086}%
\pgfsetdash{}{0pt}%
\pgfpathmoveto{\pgfqpoint{3.142298in}{2.464262in}}%
\pgfpathcurveto{\pgfqpoint{3.150534in}{2.464262in}}{\pgfqpoint{3.158434in}{2.467534in}}{\pgfqpoint{3.164258in}{2.473358in}}%
\pgfpathcurveto{\pgfqpoint{3.170082in}{2.479182in}}{\pgfqpoint{3.173354in}{2.487082in}}{\pgfqpoint{3.173354in}{2.495318in}}%
\pgfpathcurveto{\pgfqpoint{3.173354in}{2.503555in}}{\pgfqpoint{3.170082in}{2.511455in}}{\pgfqpoint{3.164258in}{2.517279in}}%
\pgfpathcurveto{\pgfqpoint{3.158434in}{2.523103in}}{\pgfqpoint{3.150534in}{2.526375in}}{\pgfqpoint{3.142298in}{2.526375in}}%
\pgfpathcurveto{\pgfqpoint{3.134061in}{2.526375in}}{\pgfqpoint{3.126161in}{2.523103in}}{\pgfqpoint{3.120337in}{2.517279in}}%
\pgfpathcurveto{\pgfqpoint{3.114513in}{2.511455in}}{\pgfqpoint{3.111241in}{2.503555in}}{\pgfqpoint{3.111241in}{2.495318in}}%
\pgfpathcurveto{\pgfqpoint{3.111241in}{2.487082in}}{\pgfqpoint{3.114513in}{2.479182in}}{\pgfqpoint{3.120337in}{2.473358in}}%
\pgfpathcurveto{\pgfqpoint{3.126161in}{2.467534in}}{\pgfqpoint{3.134061in}{2.464262in}}{\pgfqpoint{3.142298in}{2.464262in}}%
\pgfpathclose%
\pgfusepath{stroke,fill}%
\end{pgfscope}%
\begin{pgfscope}%
\pgfpathrectangle{\pgfqpoint{0.100000in}{0.220728in}}{\pgfqpoint{3.696000in}{3.696000in}}%
\pgfusepath{clip}%
\pgfsetbuttcap%
\pgfsetroundjoin%
\definecolor{currentfill}{rgb}{0.121569,0.466667,0.705882}%
\pgfsetfillcolor{currentfill}%
\pgfsetfillopacity{0.752088}%
\pgfsetlinewidth{1.003750pt}%
\definecolor{currentstroke}{rgb}{0.121569,0.466667,0.705882}%
\pgfsetstrokecolor{currentstroke}%
\pgfsetstrokeopacity{0.752088}%
\pgfsetdash{}{0pt}%
\pgfpathmoveto{\pgfqpoint{3.142290in}{2.464251in}}%
\pgfpathcurveto{\pgfqpoint{3.150526in}{2.464251in}}{\pgfqpoint{3.158426in}{2.467523in}}{\pgfqpoint{3.164250in}{2.473347in}}%
\pgfpathcurveto{\pgfqpoint{3.170074in}{2.479171in}}{\pgfqpoint{3.173346in}{2.487071in}}{\pgfqpoint{3.173346in}{2.495308in}}%
\pgfpathcurveto{\pgfqpoint{3.173346in}{2.503544in}}{\pgfqpoint{3.170074in}{2.511444in}}{\pgfqpoint{3.164250in}{2.517268in}}%
\pgfpathcurveto{\pgfqpoint{3.158426in}{2.523092in}}{\pgfqpoint{3.150526in}{2.526364in}}{\pgfqpoint{3.142290in}{2.526364in}}%
\pgfpathcurveto{\pgfqpoint{3.134053in}{2.526364in}}{\pgfqpoint{3.126153in}{2.523092in}}{\pgfqpoint{3.120329in}{2.517268in}}%
\pgfpathcurveto{\pgfqpoint{3.114505in}{2.511444in}}{\pgfqpoint{3.111233in}{2.503544in}}{\pgfqpoint{3.111233in}{2.495308in}}%
\pgfpathcurveto{\pgfqpoint{3.111233in}{2.487071in}}{\pgfqpoint{3.114505in}{2.479171in}}{\pgfqpoint{3.120329in}{2.473347in}}%
\pgfpathcurveto{\pgfqpoint{3.126153in}{2.467523in}}{\pgfqpoint{3.134053in}{2.464251in}}{\pgfqpoint{3.142290in}{2.464251in}}%
\pgfpathclose%
\pgfusepath{stroke,fill}%
\end{pgfscope}%
\begin{pgfscope}%
\pgfpathrectangle{\pgfqpoint{0.100000in}{0.220728in}}{\pgfqpoint{3.696000in}{3.696000in}}%
\pgfusepath{clip}%
\pgfsetbuttcap%
\pgfsetroundjoin%
\definecolor{currentfill}{rgb}{0.121569,0.466667,0.705882}%
\pgfsetfillcolor{currentfill}%
\pgfsetfillopacity{0.752311}%
\pgfsetlinewidth{1.003750pt}%
\definecolor{currentstroke}{rgb}{0.121569,0.466667,0.705882}%
\pgfsetstrokecolor{currentstroke}%
\pgfsetstrokeopacity{0.752311}%
\pgfsetdash{}{0pt}%
\pgfpathmoveto{\pgfqpoint{3.141739in}{2.462977in}}%
\pgfpathcurveto{\pgfqpoint{3.149975in}{2.462977in}}{\pgfqpoint{3.157875in}{2.466249in}}{\pgfqpoint{3.163699in}{2.472073in}}%
\pgfpathcurveto{\pgfqpoint{3.169523in}{2.477897in}}{\pgfqpoint{3.172795in}{2.485797in}}{\pgfqpoint{3.172795in}{2.494033in}}%
\pgfpathcurveto{\pgfqpoint{3.172795in}{2.502270in}}{\pgfqpoint{3.169523in}{2.510170in}}{\pgfqpoint{3.163699in}{2.515994in}}%
\pgfpathcurveto{\pgfqpoint{3.157875in}{2.521818in}}{\pgfqpoint{3.149975in}{2.525090in}}{\pgfqpoint{3.141739in}{2.525090in}}%
\pgfpathcurveto{\pgfqpoint{3.133502in}{2.525090in}}{\pgfqpoint{3.125602in}{2.521818in}}{\pgfqpoint{3.119778in}{2.515994in}}%
\pgfpathcurveto{\pgfqpoint{3.113954in}{2.510170in}}{\pgfqpoint{3.110682in}{2.502270in}}{\pgfqpoint{3.110682in}{2.494033in}}%
\pgfpathcurveto{\pgfqpoint{3.110682in}{2.485797in}}{\pgfqpoint{3.113954in}{2.477897in}}{\pgfqpoint{3.119778in}{2.472073in}}%
\pgfpathcurveto{\pgfqpoint{3.125602in}{2.466249in}}{\pgfqpoint{3.133502in}{2.462977in}}{\pgfqpoint{3.141739in}{2.462977in}}%
\pgfpathclose%
\pgfusepath{stroke,fill}%
\end{pgfscope}%
\begin{pgfscope}%
\pgfpathrectangle{\pgfqpoint{0.100000in}{0.220728in}}{\pgfqpoint{3.696000in}{3.696000in}}%
\pgfusepath{clip}%
\pgfsetbuttcap%
\pgfsetroundjoin%
\definecolor{currentfill}{rgb}{0.121569,0.466667,0.705882}%
\pgfsetfillcolor{currentfill}%
\pgfsetfillopacity{0.752420}%
\pgfsetlinewidth{1.003750pt}%
\definecolor{currentstroke}{rgb}{0.121569,0.466667,0.705882}%
\pgfsetstrokecolor{currentstroke}%
\pgfsetstrokeopacity{0.752420}%
\pgfsetdash{}{0pt}%
\pgfpathmoveto{\pgfqpoint{3.141359in}{2.462319in}}%
\pgfpathcurveto{\pgfqpoint{3.149595in}{2.462319in}}{\pgfqpoint{3.157495in}{2.465592in}}{\pgfqpoint{3.163319in}{2.471416in}}%
\pgfpathcurveto{\pgfqpoint{3.169143in}{2.477240in}}{\pgfqpoint{3.172415in}{2.485140in}}{\pgfqpoint{3.172415in}{2.493376in}}%
\pgfpathcurveto{\pgfqpoint{3.172415in}{2.501612in}}{\pgfqpoint{3.169143in}{2.509512in}}{\pgfqpoint{3.163319in}{2.515336in}}%
\pgfpathcurveto{\pgfqpoint{3.157495in}{2.521160in}}{\pgfqpoint{3.149595in}{2.524432in}}{\pgfqpoint{3.141359in}{2.524432in}}%
\pgfpathcurveto{\pgfqpoint{3.133122in}{2.524432in}}{\pgfqpoint{3.125222in}{2.521160in}}{\pgfqpoint{3.119398in}{2.515336in}}%
\pgfpathcurveto{\pgfqpoint{3.113575in}{2.509512in}}{\pgfqpoint{3.110302in}{2.501612in}}{\pgfqpoint{3.110302in}{2.493376in}}%
\pgfpathcurveto{\pgfqpoint{3.110302in}{2.485140in}}{\pgfqpoint{3.113575in}{2.477240in}}{\pgfqpoint{3.119398in}{2.471416in}}%
\pgfpathcurveto{\pgfqpoint{3.125222in}{2.465592in}}{\pgfqpoint{3.133122in}{2.462319in}}{\pgfqpoint{3.141359in}{2.462319in}}%
\pgfpathclose%
\pgfusepath{stroke,fill}%
\end{pgfscope}%
\begin{pgfscope}%
\pgfpathrectangle{\pgfqpoint{0.100000in}{0.220728in}}{\pgfqpoint{3.696000in}{3.696000in}}%
\pgfusepath{clip}%
\pgfsetbuttcap%
\pgfsetroundjoin%
\definecolor{currentfill}{rgb}{0.121569,0.466667,0.705882}%
\pgfsetfillcolor{currentfill}%
\pgfsetfillopacity{0.752486}%
\pgfsetlinewidth{1.003750pt}%
\definecolor{currentstroke}{rgb}{0.121569,0.466667,0.705882}%
\pgfsetstrokecolor{currentstroke}%
\pgfsetstrokeopacity{0.752486}%
\pgfsetdash{}{0pt}%
\pgfpathmoveto{\pgfqpoint{3.141141in}{2.461992in}}%
\pgfpathcurveto{\pgfqpoint{3.149378in}{2.461992in}}{\pgfqpoint{3.157278in}{2.465265in}}{\pgfqpoint{3.163102in}{2.471089in}}%
\pgfpathcurveto{\pgfqpoint{3.168926in}{2.476913in}}{\pgfqpoint{3.172198in}{2.484813in}}{\pgfqpoint{3.172198in}{2.493049in}}%
\pgfpathcurveto{\pgfqpoint{3.172198in}{2.501285in}}{\pgfqpoint{3.168926in}{2.509185in}}{\pgfqpoint{3.163102in}{2.515009in}}%
\pgfpathcurveto{\pgfqpoint{3.157278in}{2.520833in}}{\pgfqpoint{3.149378in}{2.524105in}}{\pgfqpoint{3.141141in}{2.524105in}}%
\pgfpathcurveto{\pgfqpoint{3.132905in}{2.524105in}}{\pgfqpoint{3.125005in}{2.520833in}}{\pgfqpoint{3.119181in}{2.515009in}}%
\pgfpathcurveto{\pgfqpoint{3.113357in}{2.509185in}}{\pgfqpoint{3.110085in}{2.501285in}}{\pgfqpoint{3.110085in}{2.493049in}}%
\pgfpathcurveto{\pgfqpoint{3.110085in}{2.484813in}}{\pgfqpoint{3.113357in}{2.476913in}}{\pgfqpoint{3.119181in}{2.471089in}}%
\pgfpathcurveto{\pgfqpoint{3.125005in}{2.465265in}}{\pgfqpoint{3.132905in}{2.461992in}}{\pgfqpoint{3.141141in}{2.461992in}}%
\pgfpathclose%
\pgfusepath{stroke,fill}%
\end{pgfscope}%
\begin{pgfscope}%
\pgfpathrectangle{\pgfqpoint{0.100000in}{0.220728in}}{\pgfqpoint{3.696000in}{3.696000in}}%
\pgfusepath{clip}%
\pgfsetbuttcap%
\pgfsetroundjoin%
\definecolor{currentfill}{rgb}{0.121569,0.466667,0.705882}%
\pgfsetfillcolor{currentfill}%
\pgfsetfillopacity{0.752522}%
\pgfsetlinewidth{1.003750pt}%
\definecolor{currentstroke}{rgb}{0.121569,0.466667,0.705882}%
\pgfsetstrokecolor{currentstroke}%
\pgfsetstrokeopacity{0.752522}%
\pgfsetdash{}{0pt}%
\pgfpathmoveto{\pgfqpoint{3.141056in}{2.461770in}}%
\pgfpathcurveto{\pgfqpoint{3.149292in}{2.461770in}}{\pgfqpoint{3.157192in}{2.465042in}}{\pgfqpoint{3.163016in}{2.470866in}}%
\pgfpathcurveto{\pgfqpoint{3.168840in}{2.476690in}}{\pgfqpoint{3.172112in}{2.484590in}}{\pgfqpoint{3.172112in}{2.492827in}}%
\pgfpathcurveto{\pgfqpoint{3.172112in}{2.501063in}}{\pgfqpoint{3.168840in}{2.508963in}}{\pgfqpoint{3.163016in}{2.514787in}}%
\pgfpathcurveto{\pgfqpoint{3.157192in}{2.520611in}}{\pgfqpoint{3.149292in}{2.523883in}}{\pgfqpoint{3.141056in}{2.523883in}}%
\pgfpathcurveto{\pgfqpoint{3.132820in}{2.523883in}}{\pgfqpoint{3.124920in}{2.520611in}}{\pgfqpoint{3.119096in}{2.514787in}}%
\pgfpathcurveto{\pgfqpoint{3.113272in}{2.508963in}}{\pgfqpoint{3.109999in}{2.501063in}}{\pgfqpoint{3.109999in}{2.492827in}}%
\pgfpathcurveto{\pgfqpoint{3.109999in}{2.484590in}}{\pgfqpoint{3.113272in}{2.476690in}}{\pgfqpoint{3.119096in}{2.470866in}}%
\pgfpathcurveto{\pgfqpoint{3.124920in}{2.465042in}}{\pgfqpoint{3.132820in}{2.461770in}}{\pgfqpoint{3.141056in}{2.461770in}}%
\pgfpathclose%
\pgfusepath{stroke,fill}%
\end{pgfscope}%
\begin{pgfscope}%
\pgfpathrectangle{\pgfqpoint{0.100000in}{0.220728in}}{\pgfqpoint{3.696000in}{3.696000in}}%
\pgfusepath{clip}%
\pgfsetbuttcap%
\pgfsetroundjoin%
\definecolor{currentfill}{rgb}{0.121569,0.466667,0.705882}%
\pgfsetfillcolor{currentfill}%
\pgfsetfillopacity{0.752800}%
\pgfsetlinewidth{1.003750pt}%
\definecolor{currentstroke}{rgb}{0.121569,0.466667,0.705882}%
\pgfsetstrokecolor{currentstroke}%
\pgfsetstrokeopacity{0.752800}%
\pgfsetdash{}{0pt}%
\pgfpathmoveto{\pgfqpoint{3.140081in}{2.460220in}}%
\pgfpathcurveto{\pgfqpoint{3.148317in}{2.460220in}}{\pgfqpoint{3.156218in}{2.463493in}}{\pgfqpoint{3.162041in}{2.469317in}}%
\pgfpathcurveto{\pgfqpoint{3.167865in}{2.475141in}}{\pgfqpoint{3.171138in}{2.483041in}}{\pgfqpoint{3.171138in}{2.491277in}}%
\pgfpathcurveto{\pgfqpoint{3.171138in}{2.499513in}}{\pgfqpoint{3.167865in}{2.507413in}}{\pgfqpoint{3.162041in}{2.513237in}}%
\pgfpathcurveto{\pgfqpoint{3.156218in}{2.519061in}}{\pgfqpoint{3.148317in}{2.522333in}}{\pgfqpoint{3.140081in}{2.522333in}}%
\pgfpathcurveto{\pgfqpoint{3.131845in}{2.522333in}}{\pgfqpoint{3.123945in}{2.519061in}}{\pgfqpoint{3.118121in}{2.513237in}}%
\pgfpathcurveto{\pgfqpoint{3.112297in}{2.507413in}}{\pgfqpoint{3.109025in}{2.499513in}}{\pgfqpoint{3.109025in}{2.491277in}}%
\pgfpathcurveto{\pgfqpoint{3.109025in}{2.483041in}}{\pgfqpoint{3.112297in}{2.475141in}}{\pgfqpoint{3.118121in}{2.469317in}}%
\pgfpathcurveto{\pgfqpoint{3.123945in}{2.463493in}}{\pgfqpoint{3.131845in}{2.460220in}}{\pgfqpoint{3.140081in}{2.460220in}}%
\pgfpathclose%
\pgfusepath{stroke,fill}%
\end{pgfscope}%
\begin{pgfscope}%
\pgfpathrectangle{\pgfqpoint{0.100000in}{0.220728in}}{\pgfqpoint{3.696000in}{3.696000in}}%
\pgfusepath{clip}%
\pgfsetbuttcap%
\pgfsetroundjoin%
\definecolor{currentfill}{rgb}{0.121569,0.466667,0.705882}%
\pgfsetfillcolor{currentfill}%
\pgfsetfillopacity{0.752984}%
\pgfsetlinewidth{1.003750pt}%
\definecolor{currentstroke}{rgb}{0.121569,0.466667,0.705882}%
\pgfsetstrokecolor{currentstroke}%
\pgfsetstrokeopacity{0.752984}%
\pgfsetdash{}{0pt}%
\pgfpathmoveto{\pgfqpoint{3.139626in}{2.459390in}}%
\pgfpathcurveto{\pgfqpoint{3.147862in}{2.459390in}}{\pgfqpoint{3.155762in}{2.462663in}}{\pgfqpoint{3.161586in}{2.468487in}}%
\pgfpathcurveto{\pgfqpoint{3.167410in}{2.474311in}}{\pgfqpoint{3.170683in}{2.482211in}}{\pgfqpoint{3.170683in}{2.490447in}}%
\pgfpathcurveto{\pgfqpoint{3.170683in}{2.498683in}}{\pgfqpoint{3.167410in}{2.506583in}}{\pgfqpoint{3.161586in}{2.512407in}}%
\pgfpathcurveto{\pgfqpoint{3.155762in}{2.518231in}}{\pgfqpoint{3.147862in}{2.521503in}}{\pgfqpoint{3.139626in}{2.521503in}}%
\pgfpathcurveto{\pgfqpoint{3.131390in}{2.521503in}}{\pgfqpoint{3.123490in}{2.518231in}}{\pgfqpoint{3.117666in}{2.512407in}}%
\pgfpathcurveto{\pgfqpoint{3.111842in}{2.506583in}}{\pgfqpoint{3.108570in}{2.498683in}}{\pgfqpoint{3.108570in}{2.490447in}}%
\pgfpathcurveto{\pgfqpoint{3.108570in}{2.482211in}}{\pgfqpoint{3.111842in}{2.474311in}}{\pgfqpoint{3.117666in}{2.468487in}}%
\pgfpathcurveto{\pgfqpoint{3.123490in}{2.462663in}}{\pgfqpoint{3.131390in}{2.459390in}}{\pgfqpoint{3.139626in}{2.459390in}}%
\pgfpathclose%
\pgfusepath{stroke,fill}%
\end{pgfscope}%
\begin{pgfscope}%
\pgfpathrectangle{\pgfqpoint{0.100000in}{0.220728in}}{\pgfqpoint{3.696000in}{3.696000in}}%
\pgfusepath{clip}%
\pgfsetbuttcap%
\pgfsetroundjoin%
\definecolor{currentfill}{rgb}{0.121569,0.466667,0.705882}%
\pgfsetfillcolor{currentfill}%
\pgfsetfillopacity{0.753080}%
\pgfsetlinewidth{1.003750pt}%
\definecolor{currentstroke}{rgb}{0.121569,0.466667,0.705882}%
\pgfsetstrokecolor{currentstroke}%
\pgfsetstrokeopacity{0.753080}%
\pgfsetdash{}{0pt}%
\pgfpathmoveto{\pgfqpoint{3.139408in}{2.458875in}}%
\pgfpathcurveto{\pgfqpoint{3.147644in}{2.458875in}}{\pgfqpoint{3.155544in}{2.462147in}}{\pgfqpoint{3.161368in}{2.467971in}}%
\pgfpathcurveto{\pgfqpoint{3.167192in}{2.473795in}}{\pgfqpoint{3.170464in}{2.481695in}}{\pgfqpoint{3.170464in}{2.489931in}}%
\pgfpathcurveto{\pgfqpoint{3.170464in}{2.498167in}}{\pgfqpoint{3.167192in}{2.506067in}}{\pgfqpoint{3.161368in}{2.511891in}}%
\pgfpathcurveto{\pgfqpoint{3.155544in}{2.517715in}}{\pgfqpoint{3.147644in}{2.520988in}}{\pgfqpoint{3.139408in}{2.520988in}}%
\pgfpathcurveto{\pgfqpoint{3.131172in}{2.520988in}}{\pgfqpoint{3.123272in}{2.517715in}}{\pgfqpoint{3.117448in}{2.511891in}}%
\pgfpathcurveto{\pgfqpoint{3.111624in}{2.506067in}}{\pgfqpoint{3.108351in}{2.498167in}}{\pgfqpoint{3.108351in}{2.489931in}}%
\pgfpathcurveto{\pgfqpoint{3.108351in}{2.481695in}}{\pgfqpoint{3.111624in}{2.473795in}}{\pgfqpoint{3.117448in}{2.467971in}}%
\pgfpathcurveto{\pgfqpoint{3.123272in}{2.462147in}}{\pgfqpoint{3.131172in}{2.458875in}}{\pgfqpoint{3.139408in}{2.458875in}}%
\pgfpathclose%
\pgfusepath{stroke,fill}%
\end{pgfscope}%
\begin{pgfscope}%
\pgfpathrectangle{\pgfqpoint{0.100000in}{0.220728in}}{\pgfqpoint{3.696000in}{3.696000in}}%
\pgfusepath{clip}%
\pgfsetbuttcap%
\pgfsetroundjoin%
\definecolor{currentfill}{rgb}{0.121569,0.466667,0.705882}%
\pgfsetfillcolor{currentfill}%
\pgfsetfillopacity{0.753161}%
\pgfsetlinewidth{1.003750pt}%
\definecolor{currentstroke}{rgb}{0.121569,0.466667,0.705882}%
\pgfsetstrokecolor{currentstroke}%
\pgfsetstrokeopacity{0.753161}%
\pgfsetdash{}{0pt}%
\pgfpathmoveto{\pgfqpoint{1.100470in}{1.221431in}}%
\pgfpathcurveto{\pgfqpoint{1.108706in}{1.221431in}}{\pgfqpoint{1.116606in}{1.224703in}}{\pgfqpoint{1.122430in}{1.230527in}}%
\pgfpathcurveto{\pgfqpoint{1.128254in}{1.236351in}}{\pgfqpoint{1.131526in}{1.244251in}}{\pgfqpoint{1.131526in}{1.252487in}}%
\pgfpathcurveto{\pgfqpoint{1.131526in}{1.260723in}}{\pgfqpoint{1.128254in}{1.268623in}}{\pgfqpoint{1.122430in}{1.274447in}}%
\pgfpathcurveto{\pgfqpoint{1.116606in}{1.280271in}}{\pgfqpoint{1.108706in}{1.283544in}}{\pgfqpoint{1.100470in}{1.283544in}}%
\pgfpathcurveto{\pgfqpoint{1.092234in}{1.283544in}}{\pgfqpoint{1.084334in}{1.280271in}}{\pgfqpoint{1.078510in}{1.274447in}}%
\pgfpathcurveto{\pgfqpoint{1.072686in}{1.268623in}}{\pgfqpoint{1.069413in}{1.260723in}}{\pgfqpoint{1.069413in}{1.252487in}}%
\pgfpathcurveto{\pgfqpoint{1.069413in}{1.244251in}}{\pgfqpoint{1.072686in}{1.236351in}}{\pgfqpoint{1.078510in}{1.230527in}}%
\pgfpathcurveto{\pgfqpoint{1.084334in}{1.224703in}}{\pgfqpoint{1.092234in}{1.221431in}}{\pgfqpoint{1.100470in}{1.221431in}}%
\pgfpathclose%
\pgfusepath{stroke,fill}%
\end{pgfscope}%
\begin{pgfscope}%
\pgfpathrectangle{\pgfqpoint{0.100000in}{0.220728in}}{\pgfqpoint{3.696000in}{3.696000in}}%
\pgfusepath{clip}%
\pgfsetbuttcap%
\pgfsetroundjoin%
\definecolor{currentfill}{rgb}{0.121569,0.466667,0.705882}%
\pgfsetfillcolor{currentfill}%
\pgfsetfillopacity{0.753296}%
\pgfsetlinewidth{1.003750pt}%
\definecolor{currentstroke}{rgb}{0.121569,0.466667,0.705882}%
\pgfsetstrokecolor{currentstroke}%
\pgfsetstrokeopacity{0.753296}%
\pgfsetdash{}{0pt}%
\pgfpathmoveto{\pgfqpoint{3.138497in}{2.457612in}}%
\pgfpathcurveto{\pgfqpoint{3.146733in}{2.457612in}}{\pgfqpoint{3.154633in}{2.460884in}}{\pgfqpoint{3.160457in}{2.466708in}}%
\pgfpathcurveto{\pgfqpoint{3.166281in}{2.472532in}}{\pgfqpoint{3.169553in}{2.480432in}}{\pgfqpoint{3.169553in}{2.488669in}}%
\pgfpathcurveto{\pgfqpoint{3.169553in}{2.496905in}}{\pgfqpoint{3.166281in}{2.504805in}}{\pgfqpoint{3.160457in}{2.510629in}}%
\pgfpathcurveto{\pgfqpoint{3.154633in}{2.516453in}}{\pgfqpoint{3.146733in}{2.519725in}}{\pgfqpoint{3.138497in}{2.519725in}}%
\pgfpathcurveto{\pgfqpoint{3.130261in}{2.519725in}}{\pgfqpoint{3.122361in}{2.516453in}}{\pgfqpoint{3.116537in}{2.510629in}}%
\pgfpathcurveto{\pgfqpoint{3.110713in}{2.504805in}}{\pgfqpoint{3.107440in}{2.496905in}}{\pgfqpoint{3.107440in}{2.488669in}}%
\pgfpathcurveto{\pgfqpoint{3.107440in}{2.480432in}}{\pgfqpoint{3.110713in}{2.472532in}}{\pgfqpoint{3.116537in}{2.466708in}}%
\pgfpathcurveto{\pgfqpoint{3.122361in}{2.460884in}}{\pgfqpoint{3.130261in}{2.457612in}}{\pgfqpoint{3.138497in}{2.457612in}}%
\pgfpathclose%
\pgfusepath{stroke,fill}%
\end{pgfscope}%
\begin{pgfscope}%
\pgfpathrectangle{\pgfqpoint{0.100000in}{0.220728in}}{\pgfqpoint{3.696000in}{3.696000in}}%
\pgfusepath{clip}%
\pgfsetbuttcap%
\pgfsetroundjoin%
\definecolor{currentfill}{rgb}{0.121569,0.466667,0.705882}%
\pgfsetfillcolor{currentfill}%
\pgfsetfillopacity{0.753756}%
\pgfsetlinewidth{1.003750pt}%
\definecolor{currentstroke}{rgb}{0.121569,0.466667,0.705882}%
\pgfsetstrokecolor{currentstroke}%
\pgfsetstrokeopacity{0.753756}%
\pgfsetdash{}{0pt}%
\pgfpathmoveto{\pgfqpoint{3.137339in}{2.454726in}}%
\pgfpathcurveto{\pgfqpoint{3.145576in}{2.454726in}}{\pgfqpoint{3.153476in}{2.457998in}}{\pgfqpoint{3.159300in}{2.463822in}}%
\pgfpathcurveto{\pgfqpoint{3.165124in}{2.469646in}}{\pgfqpoint{3.168396in}{2.477546in}}{\pgfqpoint{3.168396in}{2.485782in}}%
\pgfpathcurveto{\pgfqpoint{3.168396in}{2.494019in}}{\pgfqpoint{3.165124in}{2.501919in}}{\pgfqpoint{3.159300in}{2.507743in}}%
\pgfpathcurveto{\pgfqpoint{3.153476in}{2.513567in}}{\pgfqpoint{3.145576in}{2.516839in}}{\pgfqpoint{3.137339in}{2.516839in}}%
\pgfpathcurveto{\pgfqpoint{3.129103in}{2.516839in}}{\pgfqpoint{3.121203in}{2.513567in}}{\pgfqpoint{3.115379in}{2.507743in}}%
\pgfpathcurveto{\pgfqpoint{3.109555in}{2.501919in}}{\pgfqpoint{3.106283in}{2.494019in}}{\pgfqpoint{3.106283in}{2.485782in}}%
\pgfpathcurveto{\pgfqpoint{3.106283in}{2.477546in}}{\pgfqpoint{3.109555in}{2.469646in}}{\pgfqpoint{3.115379in}{2.463822in}}%
\pgfpathcurveto{\pgfqpoint{3.121203in}{2.457998in}}{\pgfqpoint{3.129103in}{2.454726in}}{\pgfqpoint{3.137339in}{2.454726in}}%
\pgfpathclose%
\pgfusepath{stroke,fill}%
\end{pgfscope}%
\begin{pgfscope}%
\pgfpathrectangle{\pgfqpoint{0.100000in}{0.220728in}}{\pgfqpoint{3.696000in}{3.696000in}}%
\pgfusepath{clip}%
\pgfsetbuttcap%
\pgfsetroundjoin%
\definecolor{currentfill}{rgb}{0.121569,0.466667,0.705882}%
\pgfsetfillcolor{currentfill}%
\pgfsetfillopacity{0.754030}%
\pgfsetlinewidth{1.003750pt}%
\definecolor{currentstroke}{rgb}{0.121569,0.466667,0.705882}%
\pgfsetstrokecolor{currentstroke}%
\pgfsetstrokeopacity{0.754030}%
\pgfsetdash{}{0pt}%
\pgfpathmoveto{\pgfqpoint{3.136634in}{2.453303in}}%
\pgfpathcurveto{\pgfqpoint{3.144870in}{2.453303in}}{\pgfqpoint{3.152770in}{2.456575in}}{\pgfqpoint{3.158594in}{2.462399in}}%
\pgfpathcurveto{\pgfqpoint{3.164418in}{2.468223in}}{\pgfqpoint{3.167690in}{2.476123in}}{\pgfqpoint{3.167690in}{2.484359in}}%
\pgfpathcurveto{\pgfqpoint{3.167690in}{2.492595in}}{\pgfqpoint{3.164418in}{2.500495in}}{\pgfqpoint{3.158594in}{2.506319in}}%
\pgfpathcurveto{\pgfqpoint{3.152770in}{2.512143in}}{\pgfqpoint{3.144870in}{2.515416in}}{\pgfqpoint{3.136634in}{2.515416in}}%
\pgfpathcurveto{\pgfqpoint{3.128397in}{2.515416in}}{\pgfqpoint{3.120497in}{2.512143in}}{\pgfqpoint{3.114673in}{2.506319in}}%
\pgfpathcurveto{\pgfqpoint{3.108850in}{2.500495in}}{\pgfqpoint{3.105577in}{2.492595in}}{\pgfqpoint{3.105577in}{2.484359in}}%
\pgfpathcurveto{\pgfqpoint{3.105577in}{2.476123in}}{\pgfqpoint{3.108850in}{2.468223in}}{\pgfqpoint{3.114673in}{2.462399in}}%
\pgfpathcurveto{\pgfqpoint{3.120497in}{2.456575in}}{\pgfqpoint{3.128397in}{2.453303in}}{\pgfqpoint{3.136634in}{2.453303in}}%
\pgfpathclose%
\pgfusepath{stroke,fill}%
\end{pgfscope}%
\begin{pgfscope}%
\pgfpathrectangle{\pgfqpoint{0.100000in}{0.220728in}}{\pgfqpoint{3.696000in}{3.696000in}}%
\pgfusepath{clip}%
\pgfsetbuttcap%
\pgfsetroundjoin%
\definecolor{currentfill}{rgb}{0.121569,0.466667,0.705882}%
\pgfsetfillcolor{currentfill}%
\pgfsetfillopacity{0.754184}%
\pgfsetlinewidth{1.003750pt}%
\definecolor{currentstroke}{rgb}{0.121569,0.466667,0.705882}%
\pgfsetstrokecolor{currentstroke}%
\pgfsetstrokeopacity{0.754184}%
\pgfsetdash{}{0pt}%
\pgfpathmoveto{\pgfqpoint{3.136159in}{2.452652in}}%
\pgfpathcurveto{\pgfqpoint{3.144395in}{2.452652in}}{\pgfqpoint{3.152295in}{2.455924in}}{\pgfqpoint{3.158119in}{2.461748in}}%
\pgfpathcurveto{\pgfqpoint{3.163943in}{2.467572in}}{\pgfqpoint{3.167215in}{2.475472in}}{\pgfqpoint{3.167215in}{2.483708in}}%
\pgfpathcurveto{\pgfqpoint{3.167215in}{2.491945in}}{\pgfqpoint{3.163943in}{2.499845in}}{\pgfqpoint{3.158119in}{2.505669in}}%
\pgfpathcurveto{\pgfqpoint{3.152295in}{2.511492in}}{\pgfqpoint{3.144395in}{2.514765in}}{\pgfqpoint{3.136159in}{2.514765in}}%
\pgfpathcurveto{\pgfqpoint{3.127923in}{2.514765in}}{\pgfqpoint{3.120023in}{2.511492in}}{\pgfqpoint{3.114199in}{2.505669in}}%
\pgfpathcurveto{\pgfqpoint{3.108375in}{2.499845in}}{\pgfqpoint{3.105102in}{2.491945in}}{\pgfqpoint{3.105102in}{2.483708in}}%
\pgfpathcurveto{\pgfqpoint{3.105102in}{2.475472in}}{\pgfqpoint{3.108375in}{2.467572in}}{\pgfqpoint{3.114199in}{2.461748in}}%
\pgfpathcurveto{\pgfqpoint{3.120023in}{2.455924in}}{\pgfqpoint{3.127923in}{2.452652in}}{\pgfqpoint{3.136159in}{2.452652in}}%
\pgfpathclose%
\pgfusepath{stroke,fill}%
\end{pgfscope}%
\begin{pgfscope}%
\pgfpathrectangle{\pgfqpoint{0.100000in}{0.220728in}}{\pgfqpoint{3.696000in}{3.696000in}}%
\pgfusepath{clip}%
\pgfsetbuttcap%
\pgfsetroundjoin%
\definecolor{currentfill}{rgb}{0.121569,0.466667,0.705882}%
\pgfsetfillcolor{currentfill}%
\pgfsetfillopacity{0.754268}%
\pgfsetlinewidth{1.003750pt}%
\definecolor{currentstroke}{rgb}{0.121569,0.466667,0.705882}%
\pgfsetstrokecolor{currentstroke}%
\pgfsetstrokeopacity{0.754268}%
\pgfsetdash{}{0pt}%
\pgfpathmoveto{\pgfqpoint{3.135989in}{2.452183in}}%
\pgfpathcurveto{\pgfqpoint{3.144226in}{2.452183in}}{\pgfqpoint{3.152126in}{2.455455in}}{\pgfqpoint{3.157950in}{2.461279in}}%
\pgfpathcurveto{\pgfqpoint{3.163773in}{2.467103in}}{\pgfqpoint{3.167046in}{2.475003in}}{\pgfqpoint{3.167046in}{2.483239in}}%
\pgfpathcurveto{\pgfqpoint{3.167046in}{2.491475in}}{\pgfqpoint{3.163773in}{2.499375in}}{\pgfqpoint{3.157950in}{2.505199in}}%
\pgfpathcurveto{\pgfqpoint{3.152126in}{2.511023in}}{\pgfqpoint{3.144226in}{2.514296in}}{\pgfqpoint{3.135989in}{2.514296in}}%
\pgfpathcurveto{\pgfqpoint{3.127753in}{2.514296in}}{\pgfqpoint{3.119853in}{2.511023in}}{\pgfqpoint{3.114029in}{2.505199in}}%
\pgfpathcurveto{\pgfqpoint{3.108205in}{2.499375in}}{\pgfqpoint{3.104933in}{2.491475in}}{\pgfqpoint{3.104933in}{2.483239in}}%
\pgfpathcurveto{\pgfqpoint{3.104933in}{2.475003in}}{\pgfqpoint{3.108205in}{2.467103in}}{\pgfqpoint{3.114029in}{2.461279in}}%
\pgfpathcurveto{\pgfqpoint{3.119853in}{2.455455in}}{\pgfqpoint{3.127753in}{2.452183in}}{\pgfqpoint{3.135989in}{2.452183in}}%
\pgfpathclose%
\pgfusepath{stroke,fill}%
\end{pgfscope}%
\begin{pgfscope}%
\pgfpathrectangle{\pgfqpoint{0.100000in}{0.220728in}}{\pgfqpoint{3.696000in}{3.696000in}}%
\pgfusepath{clip}%
\pgfsetbuttcap%
\pgfsetroundjoin%
\definecolor{currentfill}{rgb}{0.121569,0.466667,0.705882}%
\pgfsetfillcolor{currentfill}%
\pgfsetfillopacity{0.754543}%
\pgfsetlinewidth{1.003750pt}%
\definecolor{currentstroke}{rgb}{0.121569,0.466667,0.705882}%
\pgfsetstrokecolor{currentstroke}%
\pgfsetstrokeopacity{0.754543}%
\pgfsetdash{}{0pt}%
\pgfpathmoveto{\pgfqpoint{3.135018in}{2.450551in}}%
\pgfpathcurveto{\pgfqpoint{3.143254in}{2.450551in}}{\pgfqpoint{3.151154in}{2.453823in}}{\pgfqpoint{3.156978in}{2.459647in}}%
\pgfpathcurveto{\pgfqpoint{3.162802in}{2.465471in}}{\pgfqpoint{3.166074in}{2.473371in}}{\pgfqpoint{3.166074in}{2.481608in}}%
\pgfpathcurveto{\pgfqpoint{3.166074in}{2.489844in}}{\pgfqpoint{3.162802in}{2.497744in}}{\pgfqpoint{3.156978in}{2.503568in}}%
\pgfpathcurveto{\pgfqpoint{3.151154in}{2.509392in}}{\pgfqpoint{3.143254in}{2.512664in}}{\pgfqpoint{3.135018in}{2.512664in}}%
\pgfpathcurveto{\pgfqpoint{3.126781in}{2.512664in}}{\pgfqpoint{3.118881in}{2.509392in}}{\pgfqpoint{3.113057in}{2.503568in}}%
\pgfpathcurveto{\pgfqpoint{3.107233in}{2.497744in}}{\pgfqpoint{3.103961in}{2.489844in}}{\pgfqpoint{3.103961in}{2.481608in}}%
\pgfpathcurveto{\pgfqpoint{3.103961in}{2.473371in}}{\pgfqpoint{3.107233in}{2.465471in}}{\pgfqpoint{3.113057in}{2.459647in}}%
\pgfpathcurveto{\pgfqpoint{3.118881in}{2.453823in}}{\pgfqpoint{3.126781in}{2.450551in}}{\pgfqpoint{3.135018in}{2.450551in}}%
\pgfpathclose%
\pgfusepath{stroke,fill}%
\end{pgfscope}%
\begin{pgfscope}%
\pgfpathrectangle{\pgfqpoint{0.100000in}{0.220728in}}{\pgfqpoint{3.696000in}{3.696000in}}%
\pgfusepath{clip}%
\pgfsetbuttcap%
\pgfsetroundjoin%
\definecolor{currentfill}{rgb}{0.121569,0.466667,0.705882}%
\pgfsetfillcolor{currentfill}%
\pgfsetfillopacity{0.754750}%
\pgfsetlinewidth{1.003750pt}%
\definecolor{currentstroke}{rgb}{0.121569,0.466667,0.705882}%
\pgfsetstrokecolor{currentstroke}%
\pgfsetstrokeopacity{0.754750}%
\pgfsetdash{}{0pt}%
\pgfpathmoveto{\pgfqpoint{3.134611in}{2.449733in}}%
\pgfpathcurveto{\pgfqpoint{3.142847in}{2.449733in}}{\pgfqpoint{3.150747in}{2.453005in}}{\pgfqpoint{3.156571in}{2.458829in}}%
\pgfpathcurveto{\pgfqpoint{3.162395in}{2.464653in}}{\pgfqpoint{3.165667in}{2.472553in}}{\pgfqpoint{3.165667in}{2.480789in}}%
\pgfpathcurveto{\pgfqpoint{3.165667in}{2.489025in}}{\pgfqpoint{3.162395in}{2.496925in}}{\pgfqpoint{3.156571in}{2.502749in}}%
\pgfpathcurveto{\pgfqpoint{3.150747in}{2.508573in}}{\pgfqpoint{3.142847in}{2.511846in}}{\pgfqpoint{3.134611in}{2.511846in}}%
\pgfpathcurveto{\pgfqpoint{3.126374in}{2.511846in}}{\pgfqpoint{3.118474in}{2.508573in}}{\pgfqpoint{3.112650in}{2.502749in}}%
\pgfpathcurveto{\pgfqpoint{3.106826in}{2.496925in}}{\pgfqpoint{3.103554in}{2.489025in}}{\pgfqpoint{3.103554in}{2.480789in}}%
\pgfpathcurveto{\pgfqpoint{3.103554in}{2.472553in}}{\pgfqpoint{3.106826in}{2.464653in}}{\pgfqpoint{3.112650in}{2.458829in}}%
\pgfpathcurveto{\pgfqpoint{3.118474in}{2.453005in}}{\pgfqpoint{3.126374in}{2.449733in}}{\pgfqpoint{3.134611in}{2.449733in}}%
\pgfpathclose%
\pgfusepath{stroke,fill}%
\end{pgfscope}%
\begin{pgfscope}%
\pgfpathrectangle{\pgfqpoint{0.100000in}{0.220728in}}{\pgfqpoint{3.696000in}{3.696000in}}%
\pgfusepath{clip}%
\pgfsetbuttcap%
\pgfsetroundjoin%
\definecolor{currentfill}{rgb}{0.121569,0.466667,0.705882}%
\pgfsetfillcolor{currentfill}%
\pgfsetfillopacity{0.754846}%
\pgfsetlinewidth{1.003750pt}%
\definecolor{currentstroke}{rgb}{0.121569,0.466667,0.705882}%
\pgfsetstrokecolor{currentstroke}%
\pgfsetstrokeopacity{0.754846}%
\pgfsetdash{}{0pt}%
\pgfpathmoveto{\pgfqpoint{3.134381in}{2.449211in}}%
\pgfpathcurveto{\pgfqpoint{3.142618in}{2.449211in}}{\pgfqpoint{3.150518in}{2.452484in}}{\pgfqpoint{3.156342in}{2.458308in}}%
\pgfpathcurveto{\pgfqpoint{3.162166in}{2.464131in}}{\pgfqpoint{3.165438in}{2.472032in}}{\pgfqpoint{3.165438in}{2.480268in}}%
\pgfpathcurveto{\pgfqpoint{3.165438in}{2.488504in}}{\pgfqpoint{3.162166in}{2.496404in}}{\pgfqpoint{3.156342in}{2.502228in}}%
\pgfpathcurveto{\pgfqpoint{3.150518in}{2.508052in}}{\pgfqpoint{3.142618in}{2.511324in}}{\pgfqpoint{3.134381in}{2.511324in}}%
\pgfpathcurveto{\pgfqpoint{3.126145in}{2.511324in}}{\pgfqpoint{3.118245in}{2.508052in}}{\pgfqpoint{3.112421in}{2.502228in}}%
\pgfpathcurveto{\pgfqpoint{3.106597in}{2.496404in}}{\pgfqpoint{3.103325in}{2.488504in}}{\pgfqpoint{3.103325in}{2.480268in}}%
\pgfpathcurveto{\pgfqpoint{3.103325in}{2.472032in}}{\pgfqpoint{3.106597in}{2.464131in}}{\pgfqpoint{3.112421in}{2.458308in}}%
\pgfpathcurveto{\pgfqpoint{3.118245in}{2.452484in}}{\pgfqpoint{3.126145in}{2.449211in}}{\pgfqpoint{3.134381in}{2.449211in}}%
\pgfpathclose%
\pgfusepath{stroke,fill}%
\end{pgfscope}%
\begin{pgfscope}%
\pgfpathrectangle{\pgfqpoint{0.100000in}{0.220728in}}{\pgfqpoint{3.696000in}{3.696000in}}%
\pgfusepath{clip}%
\pgfsetbuttcap%
\pgfsetroundjoin%
\definecolor{currentfill}{rgb}{0.121569,0.466667,0.705882}%
\pgfsetfillcolor{currentfill}%
\pgfsetfillopacity{0.754891}%
\pgfsetlinewidth{1.003750pt}%
\definecolor{currentstroke}{rgb}{0.121569,0.466667,0.705882}%
\pgfsetstrokecolor{currentstroke}%
\pgfsetstrokeopacity{0.754891}%
\pgfsetdash{}{0pt}%
\pgfpathmoveto{\pgfqpoint{3.134203in}{2.448963in}}%
\pgfpathcurveto{\pgfqpoint{3.142440in}{2.448963in}}{\pgfqpoint{3.150340in}{2.452235in}}{\pgfqpoint{3.156164in}{2.458059in}}%
\pgfpathcurveto{\pgfqpoint{3.161988in}{2.463883in}}{\pgfqpoint{3.165260in}{2.471783in}}{\pgfqpoint{3.165260in}{2.480019in}}%
\pgfpathcurveto{\pgfqpoint{3.165260in}{2.488256in}}{\pgfqpoint{3.161988in}{2.496156in}}{\pgfqpoint{3.156164in}{2.501980in}}%
\pgfpathcurveto{\pgfqpoint{3.150340in}{2.507804in}}{\pgfqpoint{3.142440in}{2.511076in}}{\pgfqpoint{3.134203in}{2.511076in}}%
\pgfpathcurveto{\pgfqpoint{3.125967in}{2.511076in}}{\pgfqpoint{3.118067in}{2.507804in}}{\pgfqpoint{3.112243in}{2.501980in}}%
\pgfpathcurveto{\pgfqpoint{3.106419in}{2.496156in}}{\pgfqpoint{3.103147in}{2.488256in}}{\pgfqpoint{3.103147in}{2.480019in}}%
\pgfpathcurveto{\pgfqpoint{3.103147in}{2.471783in}}{\pgfqpoint{3.106419in}{2.463883in}}{\pgfqpoint{3.112243in}{2.458059in}}%
\pgfpathcurveto{\pgfqpoint{3.118067in}{2.452235in}}{\pgfqpoint{3.125967in}{2.448963in}}{\pgfqpoint{3.134203in}{2.448963in}}%
\pgfpathclose%
\pgfusepath{stroke,fill}%
\end{pgfscope}%
\begin{pgfscope}%
\pgfpathrectangle{\pgfqpoint{0.100000in}{0.220728in}}{\pgfqpoint{3.696000in}{3.696000in}}%
\pgfusepath{clip}%
\pgfsetbuttcap%
\pgfsetroundjoin%
\definecolor{currentfill}{rgb}{0.121569,0.466667,0.705882}%
\pgfsetfillcolor{currentfill}%
\pgfsetfillopacity{0.755303}%
\pgfsetlinewidth{1.003750pt}%
\definecolor{currentstroke}{rgb}{0.121569,0.466667,0.705882}%
\pgfsetstrokecolor{currentstroke}%
\pgfsetstrokeopacity{0.755303}%
\pgfsetdash{}{0pt}%
\pgfpathmoveto{\pgfqpoint{3.133335in}{2.446837in}}%
\pgfpathcurveto{\pgfqpoint{3.141572in}{2.446837in}}{\pgfqpoint{3.149472in}{2.450109in}}{\pgfqpoint{3.155296in}{2.455933in}}%
\pgfpathcurveto{\pgfqpoint{3.161119in}{2.461757in}}{\pgfqpoint{3.164392in}{2.469657in}}{\pgfqpoint{3.164392in}{2.477893in}}%
\pgfpathcurveto{\pgfqpoint{3.164392in}{2.486129in}}{\pgfqpoint{3.161119in}{2.494029in}}{\pgfqpoint{3.155296in}{2.499853in}}%
\pgfpathcurveto{\pgfqpoint{3.149472in}{2.505677in}}{\pgfqpoint{3.141572in}{2.508950in}}{\pgfqpoint{3.133335in}{2.508950in}}%
\pgfpathcurveto{\pgfqpoint{3.125099in}{2.508950in}}{\pgfqpoint{3.117199in}{2.505677in}}{\pgfqpoint{3.111375in}{2.499853in}}%
\pgfpathcurveto{\pgfqpoint{3.105551in}{2.494029in}}{\pgfqpoint{3.102279in}{2.486129in}}{\pgfqpoint{3.102279in}{2.477893in}}%
\pgfpathcurveto{\pgfqpoint{3.102279in}{2.469657in}}{\pgfqpoint{3.105551in}{2.461757in}}{\pgfqpoint{3.111375in}{2.455933in}}%
\pgfpathcurveto{\pgfqpoint{3.117199in}{2.450109in}}{\pgfqpoint{3.125099in}{2.446837in}}{\pgfqpoint{3.133335in}{2.446837in}}%
\pgfpathclose%
\pgfusepath{stroke,fill}%
\end{pgfscope}%
\begin{pgfscope}%
\pgfpathrectangle{\pgfqpoint{0.100000in}{0.220728in}}{\pgfqpoint{3.696000in}{3.696000in}}%
\pgfusepath{clip}%
\pgfsetbuttcap%
\pgfsetroundjoin%
\definecolor{currentfill}{rgb}{0.121569,0.466667,0.705882}%
\pgfsetfillcolor{currentfill}%
\pgfsetfillopacity{0.755346}%
\pgfsetlinewidth{1.003750pt}%
\definecolor{currentstroke}{rgb}{0.121569,0.466667,0.705882}%
\pgfsetstrokecolor{currentstroke}%
\pgfsetstrokeopacity{0.755346}%
\pgfsetdash{}{0pt}%
\pgfpathmoveto{\pgfqpoint{1.119045in}{1.210556in}}%
\pgfpathcurveto{\pgfqpoint{1.127282in}{1.210556in}}{\pgfqpoint{1.135182in}{1.213828in}}{\pgfqpoint{1.141006in}{1.219652in}}%
\pgfpathcurveto{\pgfqpoint{1.146830in}{1.225476in}}{\pgfqpoint{1.150102in}{1.233376in}}{\pgfqpoint{1.150102in}{1.241612in}}%
\pgfpathcurveto{\pgfqpoint{1.150102in}{1.249849in}}{\pgfqpoint{1.146830in}{1.257749in}}{\pgfqpoint{1.141006in}{1.263573in}}%
\pgfpathcurveto{\pgfqpoint{1.135182in}{1.269397in}}{\pgfqpoint{1.127282in}{1.272669in}}{\pgfqpoint{1.119045in}{1.272669in}}%
\pgfpathcurveto{\pgfqpoint{1.110809in}{1.272669in}}{\pgfqpoint{1.102909in}{1.269397in}}{\pgfqpoint{1.097085in}{1.263573in}}%
\pgfpathcurveto{\pgfqpoint{1.091261in}{1.257749in}}{\pgfqpoint{1.087989in}{1.249849in}}{\pgfqpoint{1.087989in}{1.241612in}}%
\pgfpathcurveto{\pgfqpoint{1.087989in}{1.233376in}}{\pgfqpoint{1.091261in}{1.225476in}}{\pgfqpoint{1.097085in}{1.219652in}}%
\pgfpathcurveto{\pgfqpoint{1.102909in}{1.213828in}}{\pgfqpoint{1.110809in}{1.210556in}}{\pgfqpoint{1.119045in}{1.210556in}}%
\pgfpathclose%
\pgfusepath{stroke,fill}%
\end{pgfscope}%
\begin{pgfscope}%
\pgfpathrectangle{\pgfqpoint{0.100000in}{0.220728in}}{\pgfqpoint{3.696000in}{3.696000in}}%
\pgfusepath{clip}%
\pgfsetbuttcap%
\pgfsetroundjoin%
\definecolor{currentfill}{rgb}{0.121569,0.466667,0.705882}%
\pgfsetfillcolor{currentfill}%
\pgfsetfillopacity{0.755514}%
\pgfsetlinewidth{1.003750pt}%
\definecolor{currentstroke}{rgb}{0.121569,0.466667,0.705882}%
\pgfsetstrokecolor{currentstroke}%
\pgfsetstrokeopacity{0.755514}%
\pgfsetdash{}{0pt}%
\pgfpathmoveto{\pgfqpoint{3.132767in}{2.445699in}}%
\pgfpathcurveto{\pgfqpoint{3.141004in}{2.445699in}}{\pgfqpoint{3.148904in}{2.448971in}}{\pgfqpoint{3.154728in}{2.454795in}}%
\pgfpathcurveto{\pgfqpoint{3.160551in}{2.460619in}}{\pgfqpoint{3.163824in}{2.468519in}}{\pgfqpoint{3.163824in}{2.476755in}}%
\pgfpathcurveto{\pgfqpoint{3.163824in}{2.484992in}}{\pgfqpoint{3.160551in}{2.492892in}}{\pgfqpoint{3.154728in}{2.498715in}}%
\pgfpathcurveto{\pgfqpoint{3.148904in}{2.504539in}}{\pgfqpoint{3.141004in}{2.507812in}}{\pgfqpoint{3.132767in}{2.507812in}}%
\pgfpathcurveto{\pgfqpoint{3.124531in}{2.507812in}}{\pgfqpoint{3.116631in}{2.504539in}}{\pgfqpoint{3.110807in}{2.498715in}}%
\pgfpathcurveto{\pgfqpoint{3.104983in}{2.492892in}}{\pgfqpoint{3.101711in}{2.484992in}}{\pgfqpoint{3.101711in}{2.476755in}}%
\pgfpathcurveto{\pgfqpoint{3.101711in}{2.468519in}}{\pgfqpoint{3.104983in}{2.460619in}}{\pgfqpoint{3.110807in}{2.454795in}}%
\pgfpathcurveto{\pgfqpoint{3.116631in}{2.448971in}}{\pgfqpoint{3.124531in}{2.445699in}}{\pgfqpoint{3.132767in}{2.445699in}}%
\pgfpathclose%
\pgfusepath{stroke,fill}%
\end{pgfscope}%
\begin{pgfscope}%
\pgfpathrectangle{\pgfqpoint{0.100000in}{0.220728in}}{\pgfqpoint{3.696000in}{3.696000in}}%
\pgfusepath{clip}%
\pgfsetbuttcap%
\pgfsetroundjoin%
\definecolor{currentfill}{rgb}{0.121569,0.466667,0.705882}%
\pgfsetfillcolor{currentfill}%
\pgfsetfillopacity{0.755633}%
\pgfsetlinewidth{1.003750pt}%
\definecolor{currentstroke}{rgb}{0.121569,0.466667,0.705882}%
\pgfsetstrokecolor{currentstroke}%
\pgfsetstrokeopacity{0.755633}%
\pgfsetdash{}{0pt}%
\pgfpathmoveto{\pgfqpoint{3.132394in}{2.445169in}}%
\pgfpathcurveto{\pgfqpoint{3.140631in}{2.445169in}}{\pgfqpoint{3.148531in}{2.448442in}}{\pgfqpoint{3.154355in}{2.454265in}}%
\pgfpathcurveto{\pgfqpoint{3.160179in}{2.460089in}}{\pgfqpoint{3.163451in}{2.467989in}}{\pgfqpoint{3.163451in}{2.476226in}}%
\pgfpathcurveto{\pgfqpoint{3.163451in}{2.484462in}}{\pgfqpoint{3.160179in}{2.492362in}}{\pgfqpoint{3.154355in}{2.498186in}}%
\pgfpathcurveto{\pgfqpoint{3.148531in}{2.504010in}}{\pgfqpoint{3.140631in}{2.507282in}}{\pgfqpoint{3.132394in}{2.507282in}}%
\pgfpathcurveto{\pgfqpoint{3.124158in}{2.507282in}}{\pgfqpoint{3.116258in}{2.504010in}}{\pgfqpoint{3.110434in}{2.498186in}}%
\pgfpathcurveto{\pgfqpoint{3.104610in}{2.492362in}}{\pgfqpoint{3.101338in}{2.484462in}}{\pgfqpoint{3.101338in}{2.476226in}}%
\pgfpathcurveto{\pgfqpoint{3.101338in}{2.467989in}}{\pgfqpoint{3.104610in}{2.460089in}}{\pgfqpoint{3.110434in}{2.454265in}}%
\pgfpathcurveto{\pgfqpoint{3.116258in}{2.448442in}}{\pgfqpoint{3.124158in}{2.445169in}}{\pgfqpoint{3.132394in}{2.445169in}}%
\pgfpathclose%
\pgfusepath{stroke,fill}%
\end{pgfscope}%
\begin{pgfscope}%
\pgfpathrectangle{\pgfqpoint{0.100000in}{0.220728in}}{\pgfqpoint{3.696000in}{3.696000in}}%
\pgfusepath{clip}%
\pgfsetbuttcap%
\pgfsetroundjoin%
\definecolor{currentfill}{rgb}{0.121569,0.466667,0.705882}%
\pgfsetfillcolor{currentfill}%
\pgfsetfillopacity{0.756058}%
\pgfsetlinewidth{1.003750pt}%
\definecolor{currentstroke}{rgb}{0.121569,0.466667,0.705882}%
\pgfsetstrokecolor{currentstroke}%
\pgfsetstrokeopacity{0.756058}%
\pgfsetdash{}{0pt}%
\pgfpathmoveto{\pgfqpoint{3.131577in}{2.443087in}}%
\pgfpathcurveto{\pgfqpoint{3.139813in}{2.443087in}}{\pgfqpoint{3.147713in}{2.446359in}}{\pgfqpoint{3.153537in}{2.452183in}}%
\pgfpathcurveto{\pgfqpoint{3.159361in}{2.458007in}}{\pgfqpoint{3.162633in}{2.465907in}}{\pgfqpoint{3.162633in}{2.474143in}}%
\pgfpathcurveto{\pgfqpoint{3.162633in}{2.482379in}}{\pgfqpoint{3.159361in}{2.490279in}}{\pgfqpoint{3.153537in}{2.496103in}}%
\pgfpathcurveto{\pgfqpoint{3.147713in}{2.501927in}}{\pgfqpoint{3.139813in}{2.505200in}}{\pgfqpoint{3.131577in}{2.505200in}}%
\pgfpathcurveto{\pgfqpoint{3.123341in}{2.505200in}}{\pgfqpoint{3.115441in}{2.501927in}}{\pgfqpoint{3.109617in}{2.496103in}}%
\pgfpathcurveto{\pgfqpoint{3.103793in}{2.490279in}}{\pgfqpoint{3.100520in}{2.482379in}}{\pgfqpoint{3.100520in}{2.474143in}}%
\pgfpathcurveto{\pgfqpoint{3.100520in}{2.465907in}}{\pgfqpoint{3.103793in}{2.458007in}}{\pgfqpoint{3.109617in}{2.452183in}}%
\pgfpathcurveto{\pgfqpoint{3.115441in}{2.446359in}}{\pgfqpoint{3.123341in}{2.443087in}}{\pgfqpoint{3.131577in}{2.443087in}}%
\pgfpathclose%
\pgfusepath{stroke,fill}%
\end{pgfscope}%
\begin{pgfscope}%
\pgfpathrectangle{\pgfqpoint{0.100000in}{0.220728in}}{\pgfqpoint{3.696000in}{3.696000in}}%
\pgfusepath{clip}%
\pgfsetbuttcap%
\pgfsetroundjoin%
\definecolor{currentfill}{rgb}{0.121569,0.466667,0.705882}%
\pgfsetfillcolor{currentfill}%
\pgfsetfillopacity{0.756621}%
\pgfsetlinewidth{1.003750pt}%
\definecolor{currentstroke}{rgb}{0.121569,0.466667,0.705882}%
\pgfsetstrokecolor{currentstroke}%
\pgfsetstrokeopacity{0.756621}%
\pgfsetdash{}{0pt}%
\pgfpathmoveto{\pgfqpoint{3.130094in}{2.440338in}}%
\pgfpathcurveto{\pgfqpoint{3.138330in}{2.440338in}}{\pgfqpoint{3.146231in}{2.443611in}}{\pgfqpoint{3.152054in}{2.449435in}}%
\pgfpathcurveto{\pgfqpoint{3.157878in}{2.455258in}}{\pgfqpoint{3.161151in}{2.463159in}}{\pgfqpoint{3.161151in}{2.471395in}}%
\pgfpathcurveto{\pgfqpoint{3.161151in}{2.479631in}}{\pgfqpoint{3.157878in}{2.487531in}}{\pgfqpoint{3.152054in}{2.493355in}}%
\pgfpathcurveto{\pgfqpoint{3.146231in}{2.499179in}}{\pgfqpoint{3.138330in}{2.502451in}}{\pgfqpoint{3.130094in}{2.502451in}}%
\pgfpathcurveto{\pgfqpoint{3.121858in}{2.502451in}}{\pgfqpoint{3.113958in}{2.499179in}}{\pgfqpoint{3.108134in}{2.493355in}}%
\pgfpathcurveto{\pgfqpoint{3.102310in}{2.487531in}}{\pgfqpoint{3.099038in}{2.479631in}}{\pgfqpoint{3.099038in}{2.471395in}}%
\pgfpathcurveto{\pgfqpoint{3.099038in}{2.463159in}}{\pgfqpoint{3.102310in}{2.455258in}}{\pgfqpoint{3.108134in}{2.449435in}}%
\pgfpathcurveto{\pgfqpoint{3.113958in}{2.443611in}}{\pgfqpoint{3.121858in}{2.440338in}}{\pgfqpoint{3.130094in}{2.440338in}}%
\pgfpathclose%
\pgfusepath{stroke,fill}%
\end{pgfscope}%
\begin{pgfscope}%
\pgfpathrectangle{\pgfqpoint{0.100000in}{0.220728in}}{\pgfqpoint{3.696000in}{3.696000in}}%
\pgfusepath{clip}%
\pgfsetbuttcap%
\pgfsetroundjoin%
\definecolor{currentfill}{rgb}{0.121569,0.466667,0.705882}%
\pgfsetfillcolor{currentfill}%
\pgfsetfillopacity{0.756953}%
\pgfsetlinewidth{1.003750pt}%
\definecolor{currentstroke}{rgb}{0.121569,0.466667,0.705882}%
\pgfsetstrokecolor{currentstroke}%
\pgfsetstrokeopacity{0.756953}%
\pgfsetdash{}{0pt}%
\pgfpathmoveto{\pgfqpoint{3.129200in}{2.439027in}}%
\pgfpathcurveto{\pgfqpoint{3.137436in}{2.439027in}}{\pgfqpoint{3.145336in}{2.442299in}}{\pgfqpoint{3.151160in}{2.448123in}}%
\pgfpathcurveto{\pgfqpoint{3.156984in}{2.453947in}}{\pgfqpoint{3.160256in}{2.461847in}}{\pgfqpoint{3.160256in}{2.470083in}}%
\pgfpathcurveto{\pgfqpoint{3.160256in}{2.478319in}}{\pgfqpoint{3.156984in}{2.486220in}}{\pgfqpoint{3.151160in}{2.492043in}}%
\pgfpathcurveto{\pgfqpoint{3.145336in}{2.497867in}}{\pgfqpoint{3.137436in}{2.501140in}}{\pgfqpoint{3.129200in}{2.501140in}}%
\pgfpathcurveto{\pgfqpoint{3.120963in}{2.501140in}}{\pgfqpoint{3.113063in}{2.497867in}}{\pgfqpoint{3.107239in}{2.492043in}}%
\pgfpathcurveto{\pgfqpoint{3.101416in}{2.486220in}}{\pgfqpoint{3.098143in}{2.478319in}}{\pgfqpoint{3.098143in}{2.470083in}}%
\pgfpathcurveto{\pgfqpoint{3.098143in}{2.461847in}}{\pgfqpoint{3.101416in}{2.453947in}}{\pgfqpoint{3.107239in}{2.448123in}}%
\pgfpathcurveto{\pgfqpoint{3.113063in}{2.442299in}}{\pgfqpoint{3.120963in}{2.439027in}}{\pgfqpoint{3.129200in}{2.439027in}}%
\pgfpathclose%
\pgfusepath{stroke,fill}%
\end{pgfscope}%
\begin{pgfscope}%
\pgfpathrectangle{\pgfqpoint{0.100000in}{0.220728in}}{\pgfqpoint{3.696000in}{3.696000in}}%
\pgfusepath{clip}%
\pgfsetbuttcap%
\pgfsetroundjoin%
\definecolor{currentfill}{rgb}{0.121569,0.466667,0.705882}%
\pgfsetfillcolor{currentfill}%
\pgfsetfillopacity{0.757120}%
\pgfsetlinewidth{1.003750pt}%
\definecolor{currentstroke}{rgb}{0.121569,0.466667,0.705882}%
\pgfsetstrokecolor{currentstroke}%
\pgfsetstrokeopacity{0.757120}%
\pgfsetdash{}{0pt}%
\pgfpathmoveto{\pgfqpoint{3.128866in}{2.438060in}}%
\pgfpathcurveto{\pgfqpoint{3.137103in}{2.438060in}}{\pgfqpoint{3.145003in}{2.441333in}}{\pgfqpoint{3.150827in}{2.447157in}}%
\pgfpathcurveto{\pgfqpoint{3.156650in}{2.452981in}}{\pgfqpoint{3.159923in}{2.460881in}}{\pgfqpoint{3.159923in}{2.469117in}}%
\pgfpathcurveto{\pgfqpoint{3.159923in}{2.477353in}}{\pgfqpoint{3.156650in}{2.485253in}}{\pgfqpoint{3.150827in}{2.491077in}}%
\pgfpathcurveto{\pgfqpoint{3.145003in}{2.496901in}}{\pgfqpoint{3.137103in}{2.500173in}}{\pgfqpoint{3.128866in}{2.500173in}}%
\pgfpathcurveto{\pgfqpoint{3.120630in}{2.500173in}}{\pgfqpoint{3.112730in}{2.496901in}}{\pgfqpoint{3.106906in}{2.491077in}}%
\pgfpathcurveto{\pgfqpoint{3.101082in}{2.485253in}}{\pgfqpoint{3.097810in}{2.477353in}}{\pgfqpoint{3.097810in}{2.469117in}}%
\pgfpathcurveto{\pgfqpoint{3.097810in}{2.460881in}}{\pgfqpoint{3.101082in}{2.452981in}}{\pgfqpoint{3.106906in}{2.447157in}}%
\pgfpathcurveto{\pgfqpoint{3.112730in}{2.441333in}}{\pgfqpoint{3.120630in}{2.438060in}}{\pgfqpoint{3.128866in}{2.438060in}}%
\pgfpathclose%
\pgfusepath{stroke,fill}%
\end{pgfscope}%
\begin{pgfscope}%
\pgfpathrectangle{\pgfqpoint{0.100000in}{0.220728in}}{\pgfqpoint{3.696000in}{3.696000in}}%
\pgfusepath{clip}%
\pgfsetbuttcap%
\pgfsetroundjoin%
\definecolor{currentfill}{rgb}{0.121569,0.466667,0.705882}%
\pgfsetfillcolor{currentfill}%
\pgfsetfillopacity{0.757457}%
\pgfsetlinewidth{1.003750pt}%
\definecolor{currentstroke}{rgb}{0.121569,0.466667,0.705882}%
\pgfsetstrokecolor{currentstroke}%
\pgfsetstrokeopacity{0.757457}%
\pgfsetdash{}{0pt}%
\pgfpathmoveto{\pgfqpoint{3.127750in}{2.436049in}}%
\pgfpathcurveto{\pgfqpoint{3.135987in}{2.436049in}}{\pgfqpoint{3.143887in}{2.439322in}}{\pgfqpoint{3.149711in}{2.445146in}}%
\pgfpathcurveto{\pgfqpoint{3.155535in}{2.450970in}}{\pgfqpoint{3.158807in}{2.458870in}}{\pgfqpoint{3.158807in}{2.467106in}}%
\pgfpathcurveto{\pgfqpoint{3.158807in}{2.475342in}}{\pgfqpoint{3.155535in}{2.483242in}}{\pgfqpoint{3.149711in}{2.489066in}}%
\pgfpathcurveto{\pgfqpoint{3.143887in}{2.494890in}}{\pgfqpoint{3.135987in}{2.498162in}}{\pgfqpoint{3.127750in}{2.498162in}}%
\pgfpathcurveto{\pgfqpoint{3.119514in}{2.498162in}}{\pgfqpoint{3.111614in}{2.494890in}}{\pgfqpoint{3.105790in}{2.489066in}}%
\pgfpathcurveto{\pgfqpoint{3.099966in}{2.483242in}}{\pgfqpoint{3.096694in}{2.475342in}}{\pgfqpoint{3.096694in}{2.467106in}}%
\pgfpathcurveto{\pgfqpoint{3.096694in}{2.458870in}}{\pgfqpoint{3.099966in}{2.450970in}}{\pgfqpoint{3.105790in}{2.445146in}}%
\pgfpathcurveto{\pgfqpoint{3.111614in}{2.439322in}}{\pgfqpoint{3.119514in}{2.436049in}}{\pgfqpoint{3.127750in}{2.436049in}}%
\pgfpathclose%
\pgfusepath{stroke,fill}%
\end{pgfscope}%
\begin{pgfscope}%
\pgfpathrectangle{\pgfqpoint{0.100000in}{0.220728in}}{\pgfqpoint{3.696000in}{3.696000in}}%
\pgfusepath{clip}%
\pgfsetbuttcap%
\pgfsetroundjoin%
\definecolor{currentfill}{rgb}{0.121569,0.466667,0.705882}%
\pgfsetfillcolor{currentfill}%
\pgfsetfillopacity{0.757671}%
\pgfsetlinewidth{1.003750pt}%
\definecolor{currentstroke}{rgb}{0.121569,0.466667,0.705882}%
\pgfsetstrokecolor{currentstroke}%
\pgfsetstrokeopacity{0.757671}%
\pgfsetdash{}{0pt}%
\pgfpathmoveto{\pgfqpoint{3.127142in}{2.435059in}}%
\pgfpathcurveto{\pgfqpoint{3.135379in}{2.435059in}}{\pgfqpoint{3.143279in}{2.438331in}}{\pgfqpoint{3.149103in}{2.444155in}}%
\pgfpathcurveto{\pgfqpoint{3.154927in}{2.449979in}}{\pgfqpoint{3.158199in}{2.457879in}}{\pgfqpoint{3.158199in}{2.466115in}}%
\pgfpathcurveto{\pgfqpoint{3.158199in}{2.474352in}}{\pgfqpoint{3.154927in}{2.482252in}}{\pgfqpoint{3.149103in}{2.488076in}}%
\pgfpathcurveto{\pgfqpoint{3.143279in}{2.493900in}}{\pgfqpoint{3.135379in}{2.497172in}}{\pgfqpoint{3.127142in}{2.497172in}}%
\pgfpathcurveto{\pgfqpoint{3.118906in}{2.497172in}}{\pgfqpoint{3.111006in}{2.493900in}}{\pgfqpoint{3.105182in}{2.488076in}}%
\pgfpathcurveto{\pgfqpoint{3.099358in}{2.482252in}}{\pgfqpoint{3.096086in}{2.474352in}}{\pgfqpoint{3.096086in}{2.466115in}}%
\pgfpathcurveto{\pgfqpoint{3.096086in}{2.457879in}}{\pgfqpoint{3.099358in}{2.449979in}}{\pgfqpoint{3.105182in}{2.444155in}}%
\pgfpathcurveto{\pgfqpoint{3.111006in}{2.438331in}}{\pgfqpoint{3.118906in}{2.435059in}}{\pgfqpoint{3.127142in}{2.435059in}}%
\pgfpathclose%
\pgfusepath{stroke,fill}%
\end{pgfscope}%
\begin{pgfscope}%
\pgfpathrectangle{\pgfqpoint{0.100000in}{0.220728in}}{\pgfqpoint{3.696000in}{3.696000in}}%
\pgfusepath{clip}%
\pgfsetbuttcap%
\pgfsetroundjoin%
\definecolor{currentfill}{rgb}{0.121569,0.466667,0.705882}%
\pgfsetfillcolor{currentfill}%
\pgfsetfillopacity{0.758041}%
\pgfsetlinewidth{1.003750pt}%
\definecolor{currentstroke}{rgb}{0.121569,0.466667,0.705882}%
\pgfsetstrokecolor{currentstroke}%
\pgfsetstrokeopacity{0.758041}%
\pgfsetdash{}{0pt}%
\pgfpathmoveto{\pgfqpoint{3.126505in}{2.433123in}}%
\pgfpathcurveto{\pgfqpoint{3.134741in}{2.433123in}}{\pgfqpoint{3.142641in}{2.436396in}}{\pgfqpoint{3.148465in}{2.442220in}}%
\pgfpathcurveto{\pgfqpoint{3.154289in}{2.448043in}}{\pgfqpoint{3.157562in}{2.455944in}}{\pgfqpoint{3.157562in}{2.464180in}}%
\pgfpathcurveto{\pgfqpoint{3.157562in}{2.472416in}}{\pgfqpoint{3.154289in}{2.480316in}}{\pgfqpoint{3.148465in}{2.486140in}}%
\pgfpathcurveto{\pgfqpoint{3.142641in}{2.491964in}}{\pgfqpoint{3.134741in}{2.495236in}}{\pgfqpoint{3.126505in}{2.495236in}}%
\pgfpathcurveto{\pgfqpoint{3.118269in}{2.495236in}}{\pgfqpoint{3.110369in}{2.491964in}}{\pgfqpoint{3.104545in}{2.486140in}}%
\pgfpathcurveto{\pgfqpoint{3.098721in}{2.480316in}}{\pgfqpoint{3.095449in}{2.472416in}}{\pgfqpoint{3.095449in}{2.464180in}}%
\pgfpathcurveto{\pgfqpoint{3.095449in}{2.455944in}}{\pgfqpoint{3.098721in}{2.448043in}}{\pgfqpoint{3.104545in}{2.442220in}}%
\pgfpathcurveto{\pgfqpoint{3.110369in}{2.436396in}}{\pgfqpoint{3.118269in}{2.433123in}}{\pgfqpoint{3.126505in}{2.433123in}}%
\pgfpathclose%
\pgfusepath{stroke,fill}%
\end{pgfscope}%
\begin{pgfscope}%
\pgfpathrectangle{\pgfqpoint{0.100000in}{0.220728in}}{\pgfqpoint{3.696000in}{3.696000in}}%
\pgfusepath{clip}%
\pgfsetbuttcap%
\pgfsetroundjoin%
\definecolor{currentfill}{rgb}{0.121569,0.466667,0.705882}%
\pgfsetfillcolor{currentfill}%
\pgfsetfillopacity{0.758614}%
\pgfsetlinewidth{1.003750pt}%
\definecolor{currentstroke}{rgb}{0.121569,0.466667,0.705882}%
\pgfsetstrokecolor{currentstroke}%
\pgfsetstrokeopacity{0.758614}%
\pgfsetdash{}{0pt}%
\pgfpathmoveto{\pgfqpoint{3.124524in}{2.429847in}}%
\pgfpathcurveto{\pgfqpoint{3.132760in}{2.429847in}}{\pgfqpoint{3.140660in}{2.433119in}}{\pgfqpoint{3.146484in}{2.438943in}}%
\pgfpathcurveto{\pgfqpoint{3.152308in}{2.444767in}}{\pgfqpoint{3.155580in}{2.452667in}}{\pgfqpoint{3.155580in}{2.460903in}}%
\pgfpathcurveto{\pgfqpoint{3.155580in}{2.469140in}}{\pgfqpoint{3.152308in}{2.477040in}}{\pgfqpoint{3.146484in}{2.482864in}}%
\pgfpathcurveto{\pgfqpoint{3.140660in}{2.488688in}}{\pgfqpoint{3.132760in}{2.491960in}}{\pgfqpoint{3.124524in}{2.491960in}}%
\pgfpathcurveto{\pgfqpoint{3.116288in}{2.491960in}}{\pgfqpoint{3.108387in}{2.488688in}}{\pgfqpoint{3.102564in}{2.482864in}}%
\pgfpathcurveto{\pgfqpoint{3.096740in}{2.477040in}}{\pgfqpoint{3.093467in}{2.469140in}}{\pgfqpoint{3.093467in}{2.460903in}}%
\pgfpathcurveto{\pgfqpoint{3.093467in}{2.452667in}}{\pgfqpoint{3.096740in}{2.444767in}}{\pgfqpoint{3.102564in}{2.438943in}}%
\pgfpathcurveto{\pgfqpoint{3.108387in}{2.433119in}}{\pgfqpoint{3.116288in}{2.429847in}}{\pgfqpoint{3.124524in}{2.429847in}}%
\pgfpathclose%
\pgfusepath{stroke,fill}%
\end{pgfscope}%
\begin{pgfscope}%
\pgfpathrectangle{\pgfqpoint{0.100000in}{0.220728in}}{\pgfqpoint{3.696000in}{3.696000in}}%
\pgfusepath{clip}%
\pgfsetbuttcap%
\pgfsetroundjoin%
\definecolor{currentfill}{rgb}{0.121569,0.466667,0.705882}%
\pgfsetfillcolor{currentfill}%
\pgfsetfillopacity{0.758802}%
\pgfsetlinewidth{1.003750pt}%
\definecolor{currentstroke}{rgb}{0.121569,0.466667,0.705882}%
\pgfsetstrokecolor{currentstroke}%
\pgfsetstrokeopacity{0.758802}%
\pgfsetdash{}{0pt}%
\pgfpathmoveto{\pgfqpoint{1.134358in}{1.203786in}}%
\pgfpathcurveto{\pgfqpoint{1.142594in}{1.203786in}}{\pgfqpoint{1.150494in}{1.207059in}}{\pgfqpoint{1.156318in}{1.212883in}}%
\pgfpathcurveto{\pgfqpoint{1.162142in}{1.218707in}}{\pgfqpoint{1.165414in}{1.226607in}}{\pgfqpoint{1.165414in}{1.234843in}}%
\pgfpathcurveto{\pgfqpoint{1.165414in}{1.243079in}}{\pgfqpoint{1.162142in}{1.250979in}}{\pgfqpoint{1.156318in}{1.256803in}}%
\pgfpathcurveto{\pgfqpoint{1.150494in}{1.262627in}}{\pgfqpoint{1.142594in}{1.265899in}}{\pgfqpoint{1.134358in}{1.265899in}}%
\pgfpathcurveto{\pgfqpoint{1.126122in}{1.265899in}}{\pgfqpoint{1.118222in}{1.262627in}}{\pgfqpoint{1.112398in}{1.256803in}}%
\pgfpathcurveto{\pgfqpoint{1.106574in}{1.250979in}}{\pgfqpoint{1.103301in}{1.243079in}}{\pgfqpoint{1.103301in}{1.234843in}}%
\pgfpathcurveto{\pgfqpoint{1.103301in}{1.226607in}}{\pgfqpoint{1.106574in}{1.218707in}}{\pgfqpoint{1.112398in}{1.212883in}}%
\pgfpathcurveto{\pgfqpoint{1.118222in}{1.207059in}}{\pgfqpoint{1.126122in}{1.203786in}}{\pgfqpoint{1.134358in}{1.203786in}}%
\pgfpathclose%
\pgfusepath{stroke,fill}%
\end{pgfscope}%
\begin{pgfscope}%
\pgfpathrectangle{\pgfqpoint{0.100000in}{0.220728in}}{\pgfqpoint{3.696000in}{3.696000in}}%
\pgfusepath{clip}%
\pgfsetbuttcap%
\pgfsetroundjoin%
\definecolor{currentfill}{rgb}{0.121569,0.466667,0.705882}%
\pgfsetfillcolor{currentfill}%
\pgfsetfillopacity{0.758992}%
\pgfsetlinewidth{1.003750pt}%
\definecolor{currentstroke}{rgb}{0.121569,0.466667,0.705882}%
\pgfsetstrokecolor{currentstroke}%
\pgfsetstrokeopacity{0.758992}%
\pgfsetdash{}{0pt}%
\pgfpathmoveto{\pgfqpoint{3.123733in}{2.427952in}}%
\pgfpathcurveto{\pgfqpoint{3.131969in}{2.427952in}}{\pgfqpoint{3.139870in}{2.431224in}}{\pgfqpoint{3.145693in}{2.437048in}}%
\pgfpathcurveto{\pgfqpoint{3.151517in}{2.442872in}}{\pgfqpoint{3.154790in}{2.450772in}}{\pgfqpoint{3.154790in}{2.459008in}}%
\pgfpathcurveto{\pgfqpoint{3.154790in}{2.467244in}}{\pgfqpoint{3.151517in}{2.475144in}}{\pgfqpoint{3.145693in}{2.480968in}}%
\pgfpathcurveto{\pgfqpoint{3.139870in}{2.486792in}}{\pgfqpoint{3.131969in}{2.490065in}}{\pgfqpoint{3.123733in}{2.490065in}}%
\pgfpathcurveto{\pgfqpoint{3.115497in}{2.490065in}}{\pgfqpoint{3.107597in}{2.486792in}}{\pgfqpoint{3.101773in}{2.480968in}}%
\pgfpathcurveto{\pgfqpoint{3.095949in}{2.475144in}}{\pgfqpoint{3.092677in}{2.467244in}}{\pgfqpoint{3.092677in}{2.459008in}}%
\pgfpathcurveto{\pgfqpoint{3.092677in}{2.450772in}}{\pgfqpoint{3.095949in}{2.442872in}}{\pgfqpoint{3.101773in}{2.437048in}}%
\pgfpathcurveto{\pgfqpoint{3.107597in}{2.431224in}}{\pgfqpoint{3.115497in}{2.427952in}}{\pgfqpoint{3.123733in}{2.427952in}}%
\pgfpathclose%
\pgfusepath{stroke,fill}%
\end{pgfscope}%
\begin{pgfscope}%
\pgfpathrectangle{\pgfqpoint{0.100000in}{0.220728in}}{\pgfqpoint{3.696000in}{3.696000in}}%
\pgfusepath{clip}%
\pgfsetbuttcap%
\pgfsetroundjoin%
\definecolor{currentfill}{rgb}{0.121569,0.466667,0.705882}%
\pgfsetfillcolor{currentfill}%
\pgfsetfillopacity{0.759211}%
\pgfsetlinewidth{1.003750pt}%
\definecolor{currentstroke}{rgb}{0.121569,0.466667,0.705882}%
\pgfsetstrokecolor{currentstroke}%
\pgfsetstrokeopacity{0.759211}%
\pgfsetdash{}{0pt}%
\pgfpathmoveto{\pgfqpoint{3.123330in}{2.426926in}}%
\pgfpathcurveto{\pgfqpoint{3.131566in}{2.426926in}}{\pgfqpoint{3.139466in}{2.430198in}}{\pgfqpoint{3.145290in}{2.436022in}}%
\pgfpathcurveto{\pgfqpoint{3.151114in}{2.441846in}}{\pgfqpoint{3.154386in}{2.449746in}}{\pgfqpoint{3.154386in}{2.457982in}}%
\pgfpathcurveto{\pgfqpoint{3.154386in}{2.466219in}}{\pgfqpoint{3.151114in}{2.474119in}}{\pgfqpoint{3.145290in}{2.479942in}}%
\pgfpathcurveto{\pgfqpoint{3.139466in}{2.485766in}}{\pgfqpoint{3.131566in}{2.489039in}}{\pgfqpoint{3.123330in}{2.489039in}}%
\pgfpathcurveto{\pgfqpoint{3.115093in}{2.489039in}}{\pgfqpoint{3.107193in}{2.485766in}}{\pgfqpoint{3.101369in}{2.479942in}}%
\pgfpathcurveto{\pgfqpoint{3.095546in}{2.474119in}}{\pgfqpoint{3.092273in}{2.466219in}}{\pgfqpoint{3.092273in}{2.457982in}}%
\pgfpathcurveto{\pgfqpoint{3.092273in}{2.449746in}}{\pgfqpoint{3.095546in}{2.441846in}}{\pgfqpoint{3.101369in}{2.436022in}}%
\pgfpathcurveto{\pgfqpoint{3.107193in}{2.430198in}}{\pgfqpoint{3.115093in}{2.426926in}}{\pgfqpoint{3.123330in}{2.426926in}}%
\pgfpathclose%
\pgfusepath{stroke,fill}%
\end{pgfscope}%
\begin{pgfscope}%
\pgfpathrectangle{\pgfqpoint{0.100000in}{0.220728in}}{\pgfqpoint{3.696000in}{3.696000in}}%
\pgfusepath{clip}%
\pgfsetbuttcap%
\pgfsetroundjoin%
\definecolor{currentfill}{rgb}{0.121569,0.466667,0.705882}%
\pgfsetfillcolor{currentfill}%
\pgfsetfillopacity{0.759311}%
\pgfsetlinewidth{1.003750pt}%
\definecolor{currentstroke}{rgb}{0.121569,0.466667,0.705882}%
\pgfsetstrokecolor{currentstroke}%
\pgfsetstrokeopacity{0.759311}%
\pgfsetdash{}{0pt}%
\pgfpathmoveto{\pgfqpoint{3.123002in}{2.426395in}}%
\pgfpathcurveto{\pgfqpoint{3.131238in}{2.426395in}}{\pgfqpoint{3.139138in}{2.429668in}}{\pgfqpoint{3.144962in}{2.435492in}}%
\pgfpathcurveto{\pgfqpoint{3.150786in}{2.441315in}}{\pgfqpoint{3.154058in}{2.449215in}}{\pgfqpoint{3.154058in}{2.457452in}}%
\pgfpathcurveto{\pgfqpoint{3.154058in}{2.465688in}}{\pgfqpoint{3.150786in}{2.473588in}}{\pgfqpoint{3.144962in}{2.479412in}}%
\pgfpathcurveto{\pgfqpoint{3.139138in}{2.485236in}}{\pgfqpoint{3.131238in}{2.488508in}}{\pgfqpoint{3.123002in}{2.488508in}}%
\pgfpathcurveto{\pgfqpoint{3.114765in}{2.488508in}}{\pgfqpoint{3.106865in}{2.485236in}}{\pgfqpoint{3.101042in}{2.479412in}}%
\pgfpathcurveto{\pgfqpoint{3.095218in}{2.473588in}}{\pgfqpoint{3.091945in}{2.465688in}}{\pgfqpoint{3.091945in}{2.457452in}}%
\pgfpathcurveto{\pgfqpoint{3.091945in}{2.449215in}}{\pgfqpoint{3.095218in}{2.441315in}}{\pgfqpoint{3.101042in}{2.435492in}}%
\pgfpathcurveto{\pgfqpoint{3.106865in}{2.429668in}}{\pgfqpoint{3.114765in}{2.426395in}}{\pgfqpoint{3.123002in}{2.426395in}}%
\pgfpathclose%
\pgfusepath{stroke,fill}%
\end{pgfscope}%
\begin{pgfscope}%
\pgfpathrectangle{\pgfqpoint{0.100000in}{0.220728in}}{\pgfqpoint{3.696000in}{3.696000in}}%
\pgfusepath{clip}%
\pgfsetbuttcap%
\pgfsetroundjoin%
\definecolor{currentfill}{rgb}{0.121569,0.466667,0.705882}%
\pgfsetfillcolor{currentfill}%
\pgfsetfillopacity{0.759654}%
\pgfsetlinewidth{1.003750pt}%
\definecolor{currentstroke}{rgb}{0.121569,0.466667,0.705882}%
\pgfsetstrokecolor{currentstroke}%
\pgfsetstrokeopacity{0.759654}%
\pgfsetdash{}{0pt}%
\pgfpathmoveto{\pgfqpoint{3.122327in}{2.424664in}}%
\pgfpathcurveto{\pgfqpoint{3.130563in}{2.424664in}}{\pgfqpoint{3.138463in}{2.427937in}}{\pgfqpoint{3.144287in}{2.433760in}}%
\pgfpathcurveto{\pgfqpoint{3.150111in}{2.439584in}}{\pgfqpoint{3.153384in}{2.447484in}}{\pgfqpoint{3.153384in}{2.455721in}}%
\pgfpathcurveto{\pgfqpoint{3.153384in}{2.463957in}}{\pgfqpoint{3.150111in}{2.471857in}}{\pgfqpoint{3.144287in}{2.477681in}}%
\pgfpathcurveto{\pgfqpoint{3.138463in}{2.483505in}}{\pgfqpoint{3.130563in}{2.486777in}}{\pgfqpoint{3.122327in}{2.486777in}}%
\pgfpathcurveto{\pgfqpoint{3.114091in}{2.486777in}}{\pgfqpoint{3.106191in}{2.483505in}}{\pgfqpoint{3.100367in}{2.477681in}}%
\pgfpathcurveto{\pgfqpoint{3.094543in}{2.471857in}}{\pgfqpoint{3.091271in}{2.463957in}}{\pgfqpoint{3.091271in}{2.455721in}}%
\pgfpathcurveto{\pgfqpoint{3.091271in}{2.447484in}}{\pgfqpoint{3.094543in}{2.439584in}}{\pgfqpoint{3.100367in}{2.433760in}}%
\pgfpathcurveto{\pgfqpoint{3.106191in}{2.427937in}}{\pgfqpoint{3.114091in}{2.424664in}}{\pgfqpoint{3.122327in}{2.424664in}}%
\pgfpathclose%
\pgfusepath{stroke,fill}%
\end{pgfscope}%
\begin{pgfscope}%
\pgfpathrectangle{\pgfqpoint{0.100000in}{0.220728in}}{\pgfqpoint{3.696000in}{3.696000in}}%
\pgfusepath{clip}%
\pgfsetbuttcap%
\pgfsetroundjoin%
\definecolor{currentfill}{rgb}{0.121569,0.466667,0.705882}%
\pgfsetfillcolor{currentfill}%
\pgfsetfillopacity{0.759842}%
\pgfsetlinewidth{1.003750pt}%
\definecolor{currentstroke}{rgb}{0.121569,0.466667,0.705882}%
\pgfsetstrokecolor{currentstroke}%
\pgfsetstrokeopacity{0.759842}%
\pgfsetdash{}{0pt}%
\pgfpathmoveto{\pgfqpoint{3.121910in}{2.423753in}}%
\pgfpathcurveto{\pgfqpoint{3.130146in}{2.423753in}}{\pgfqpoint{3.138046in}{2.427025in}}{\pgfqpoint{3.143870in}{2.432849in}}%
\pgfpathcurveto{\pgfqpoint{3.149694in}{2.438673in}}{\pgfqpoint{3.152966in}{2.446573in}}{\pgfqpoint{3.152966in}{2.454810in}}%
\pgfpathcurveto{\pgfqpoint{3.152966in}{2.463046in}}{\pgfqpoint{3.149694in}{2.470946in}}{\pgfqpoint{3.143870in}{2.476770in}}%
\pgfpathcurveto{\pgfqpoint{3.138046in}{2.482594in}}{\pgfqpoint{3.130146in}{2.485866in}}{\pgfqpoint{3.121910in}{2.485866in}}%
\pgfpathcurveto{\pgfqpoint{3.113674in}{2.485866in}}{\pgfqpoint{3.105774in}{2.482594in}}{\pgfqpoint{3.099950in}{2.476770in}}%
\pgfpathcurveto{\pgfqpoint{3.094126in}{2.470946in}}{\pgfqpoint{3.090853in}{2.463046in}}{\pgfqpoint{3.090853in}{2.454810in}}%
\pgfpathcurveto{\pgfqpoint{3.090853in}{2.446573in}}{\pgfqpoint{3.094126in}{2.438673in}}{\pgfqpoint{3.099950in}{2.432849in}}%
\pgfpathcurveto{\pgfqpoint{3.105774in}{2.427025in}}{\pgfqpoint{3.113674in}{2.423753in}}{\pgfqpoint{3.121910in}{2.423753in}}%
\pgfpathclose%
\pgfusepath{stroke,fill}%
\end{pgfscope}%
\begin{pgfscope}%
\pgfpathrectangle{\pgfqpoint{0.100000in}{0.220728in}}{\pgfqpoint{3.696000in}{3.696000in}}%
\pgfusepath{clip}%
\pgfsetbuttcap%
\pgfsetroundjoin%
\definecolor{currentfill}{rgb}{0.121569,0.466667,0.705882}%
\pgfsetfillcolor{currentfill}%
\pgfsetfillopacity{0.759928}%
\pgfsetlinewidth{1.003750pt}%
\definecolor{currentstroke}{rgb}{0.121569,0.466667,0.705882}%
\pgfsetstrokecolor{currentstroke}%
\pgfsetstrokeopacity{0.759928}%
\pgfsetdash{}{0pt}%
\pgfpathmoveto{\pgfqpoint{3.121600in}{2.423287in}}%
\pgfpathcurveto{\pgfqpoint{3.129836in}{2.423287in}}{\pgfqpoint{3.137736in}{2.426559in}}{\pgfqpoint{3.143560in}{2.432383in}}%
\pgfpathcurveto{\pgfqpoint{3.149384in}{2.438207in}}{\pgfqpoint{3.152656in}{2.446107in}}{\pgfqpoint{3.152656in}{2.454343in}}%
\pgfpathcurveto{\pgfqpoint{3.152656in}{2.462580in}}{\pgfqpoint{3.149384in}{2.470480in}}{\pgfqpoint{3.143560in}{2.476304in}}%
\pgfpathcurveto{\pgfqpoint{3.137736in}{2.482127in}}{\pgfqpoint{3.129836in}{2.485400in}}{\pgfqpoint{3.121600in}{2.485400in}}%
\pgfpathcurveto{\pgfqpoint{3.113363in}{2.485400in}}{\pgfqpoint{3.105463in}{2.482127in}}{\pgfqpoint{3.099639in}{2.476304in}}%
\pgfpathcurveto{\pgfqpoint{3.093815in}{2.470480in}}{\pgfqpoint{3.090543in}{2.462580in}}{\pgfqpoint{3.090543in}{2.454343in}}%
\pgfpathcurveto{\pgfqpoint{3.090543in}{2.446107in}}{\pgfqpoint{3.093815in}{2.438207in}}{\pgfqpoint{3.099639in}{2.432383in}}%
\pgfpathcurveto{\pgfqpoint{3.105463in}{2.426559in}}{\pgfqpoint{3.113363in}{2.423287in}}{\pgfqpoint{3.121600in}{2.423287in}}%
\pgfpathclose%
\pgfusepath{stroke,fill}%
\end{pgfscope}%
\begin{pgfscope}%
\pgfpathrectangle{\pgfqpoint{0.100000in}{0.220728in}}{\pgfqpoint{3.696000in}{3.696000in}}%
\pgfusepath{clip}%
\pgfsetbuttcap%
\pgfsetroundjoin%
\definecolor{currentfill}{rgb}{0.121569,0.466667,0.705882}%
\pgfsetfillcolor{currentfill}%
\pgfsetfillopacity{0.760308}%
\pgfsetlinewidth{1.003750pt}%
\definecolor{currentstroke}{rgb}{0.121569,0.466667,0.705882}%
\pgfsetstrokecolor{currentstroke}%
\pgfsetstrokeopacity{0.760308}%
\pgfsetdash{}{0pt}%
\pgfpathmoveto{\pgfqpoint{3.120862in}{2.421194in}}%
\pgfpathcurveto{\pgfqpoint{3.129098in}{2.421194in}}{\pgfqpoint{3.136998in}{2.424466in}}{\pgfqpoint{3.142822in}{2.430290in}}%
\pgfpathcurveto{\pgfqpoint{3.148646in}{2.436114in}}{\pgfqpoint{3.151918in}{2.444014in}}{\pgfqpoint{3.151918in}{2.452250in}}%
\pgfpathcurveto{\pgfqpoint{3.151918in}{2.460486in}}{\pgfqpoint{3.148646in}{2.468386in}}{\pgfqpoint{3.142822in}{2.474210in}}%
\pgfpathcurveto{\pgfqpoint{3.136998in}{2.480034in}}{\pgfqpoint{3.129098in}{2.483307in}}{\pgfqpoint{3.120862in}{2.483307in}}%
\pgfpathcurveto{\pgfqpoint{3.112625in}{2.483307in}}{\pgfqpoint{3.104725in}{2.480034in}}{\pgfqpoint{3.098901in}{2.474210in}}%
\pgfpathcurveto{\pgfqpoint{3.093077in}{2.468386in}}{\pgfqpoint{3.089805in}{2.460486in}}{\pgfqpoint{3.089805in}{2.452250in}}%
\pgfpathcurveto{\pgfqpoint{3.089805in}{2.444014in}}{\pgfqpoint{3.093077in}{2.436114in}}{\pgfqpoint{3.098901in}{2.430290in}}%
\pgfpathcurveto{\pgfqpoint{3.104725in}{2.424466in}}{\pgfqpoint{3.112625in}{2.421194in}}{\pgfqpoint{3.120862in}{2.421194in}}%
\pgfpathclose%
\pgfusepath{stroke,fill}%
\end{pgfscope}%
\begin{pgfscope}%
\pgfpathrectangle{\pgfqpoint{0.100000in}{0.220728in}}{\pgfqpoint{3.696000in}{3.696000in}}%
\pgfusepath{clip}%
\pgfsetbuttcap%
\pgfsetroundjoin%
\definecolor{currentfill}{rgb}{0.121569,0.466667,0.705882}%
\pgfsetfillcolor{currentfill}%
\pgfsetfillopacity{0.760851}%
\pgfsetlinewidth{1.003750pt}%
\definecolor{currentstroke}{rgb}{0.121569,0.466667,0.705882}%
\pgfsetstrokecolor{currentstroke}%
\pgfsetstrokeopacity{0.760851}%
\pgfsetdash{}{0pt}%
\pgfpathmoveto{\pgfqpoint{3.119637in}{2.418372in}}%
\pgfpathcurveto{\pgfqpoint{3.127873in}{2.418372in}}{\pgfqpoint{3.135773in}{2.421645in}}{\pgfqpoint{3.141597in}{2.427469in}}%
\pgfpathcurveto{\pgfqpoint{3.147421in}{2.433293in}}{\pgfqpoint{3.150694in}{2.441193in}}{\pgfqpoint{3.150694in}{2.449429in}}%
\pgfpathcurveto{\pgfqpoint{3.150694in}{2.457665in}}{\pgfqpoint{3.147421in}{2.465565in}}{\pgfqpoint{3.141597in}{2.471389in}}%
\pgfpathcurveto{\pgfqpoint{3.135773in}{2.477213in}}{\pgfqpoint{3.127873in}{2.480485in}}{\pgfqpoint{3.119637in}{2.480485in}}%
\pgfpathcurveto{\pgfqpoint{3.111401in}{2.480485in}}{\pgfqpoint{3.103501in}{2.477213in}}{\pgfqpoint{3.097677in}{2.471389in}}%
\pgfpathcurveto{\pgfqpoint{3.091853in}{2.465565in}}{\pgfqpoint{3.088581in}{2.457665in}}{\pgfqpoint{3.088581in}{2.449429in}}%
\pgfpathcurveto{\pgfqpoint{3.088581in}{2.441193in}}{\pgfqpoint{3.091853in}{2.433293in}}{\pgfqpoint{3.097677in}{2.427469in}}%
\pgfpathcurveto{\pgfqpoint{3.103501in}{2.421645in}}{\pgfqpoint{3.111401in}{2.418372in}}{\pgfqpoint{3.119637in}{2.418372in}}%
\pgfpathclose%
\pgfusepath{stroke,fill}%
\end{pgfscope}%
\begin{pgfscope}%
\pgfpathrectangle{\pgfqpoint{0.100000in}{0.220728in}}{\pgfqpoint{3.696000in}{3.696000in}}%
\pgfusepath{clip}%
\pgfsetbuttcap%
\pgfsetroundjoin%
\definecolor{currentfill}{rgb}{0.121569,0.466667,0.705882}%
\pgfsetfillcolor{currentfill}%
\pgfsetfillopacity{0.761477}%
\pgfsetlinewidth{1.003750pt}%
\definecolor{currentstroke}{rgb}{0.121569,0.466667,0.705882}%
\pgfsetstrokecolor{currentstroke}%
\pgfsetstrokeopacity{0.761477}%
\pgfsetdash{}{0pt}%
\pgfpathmoveto{\pgfqpoint{3.117592in}{2.415243in}}%
\pgfpathcurveto{\pgfqpoint{3.125828in}{2.415243in}}{\pgfqpoint{3.133729in}{2.418516in}}{\pgfqpoint{3.139552in}{2.424340in}}%
\pgfpathcurveto{\pgfqpoint{3.145376in}{2.430164in}}{\pgfqpoint{3.148649in}{2.438064in}}{\pgfqpoint{3.148649in}{2.446300in}}%
\pgfpathcurveto{\pgfqpoint{3.148649in}{2.454536in}}{\pgfqpoint{3.145376in}{2.462436in}}{\pgfqpoint{3.139552in}{2.468260in}}%
\pgfpathcurveto{\pgfqpoint{3.133729in}{2.474084in}}{\pgfqpoint{3.125828in}{2.477356in}}{\pgfqpoint{3.117592in}{2.477356in}}%
\pgfpathcurveto{\pgfqpoint{3.109356in}{2.477356in}}{\pgfqpoint{3.101456in}{2.474084in}}{\pgfqpoint{3.095632in}{2.468260in}}%
\pgfpathcurveto{\pgfqpoint{3.089808in}{2.462436in}}{\pgfqpoint{3.086536in}{2.454536in}}{\pgfqpoint{3.086536in}{2.446300in}}%
\pgfpathcurveto{\pgfqpoint{3.086536in}{2.438064in}}{\pgfqpoint{3.089808in}{2.430164in}}{\pgfqpoint{3.095632in}{2.424340in}}%
\pgfpathcurveto{\pgfqpoint{3.101456in}{2.418516in}}{\pgfqpoint{3.109356in}{2.415243in}}{\pgfqpoint{3.117592in}{2.415243in}}%
\pgfpathclose%
\pgfusepath{stroke,fill}%
\end{pgfscope}%
\begin{pgfscope}%
\pgfpathrectangle{\pgfqpoint{0.100000in}{0.220728in}}{\pgfqpoint{3.696000in}{3.696000in}}%
\pgfusepath{clip}%
\pgfsetbuttcap%
\pgfsetroundjoin%
\definecolor{currentfill}{rgb}{0.121569,0.466667,0.705882}%
\pgfsetfillcolor{currentfill}%
\pgfsetfillopacity{0.761584}%
\pgfsetlinewidth{1.003750pt}%
\definecolor{currentstroke}{rgb}{0.121569,0.466667,0.705882}%
\pgfsetstrokecolor{currentstroke}%
\pgfsetstrokeopacity{0.761584}%
\pgfsetdash{}{0pt}%
\pgfpathmoveto{\pgfqpoint{1.147272in}{1.199057in}}%
\pgfpathcurveto{\pgfqpoint{1.155508in}{1.199057in}}{\pgfqpoint{1.163408in}{1.202329in}}{\pgfqpoint{1.169232in}{1.208153in}}%
\pgfpathcurveto{\pgfqpoint{1.175056in}{1.213977in}}{\pgfqpoint{1.178329in}{1.221877in}}{\pgfqpoint{1.178329in}{1.230113in}}%
\pgfpathcurveto{\pgfqpoint{1.178329in}{1.238350in}}{\pgfqpoint{1.175056in}{1.246250in}}{\pgfqpoint{1.169232in}{1.252074in}}%
\pgfpathcurveto{\pgfqpoint{1.163408in}{1.257898in}}{\pgfqpoint{1.155508in}{1.261170in}}{\pgfqpoint{1.147272in}{1.261170in}}%
\pgfpathcurveto{\pgfqpoint{1.139036in}{1.261170in}}{\pgfqpoint{1.131136in}{1.257898in}}{\pgfqpoint{1.125312in}{1.252074in}}%
\pgfpathcurveto{\pgfqpoint{1.119488in}{1.246250in}}{\pgfqpoint{1.116216in}{1.238350in}}{\pgfqpoint{1.116216in}{1.230113in}}%
\pgfpathcurveto{\pgfqpoint{1.116216in}{1.221877in}}{\pgfqpoint{1.119488in}{1.213977in}}{\pgfqpoint{1.125312in}{1.208153in}}%
\pgfpathcurveto{\pgfqpoint{1.131136in}{1.202329in}}{\pgfqpoint{1.139036in}{1.199057in}}{\pgfqpoint{1.147272in}{1.199057in}}%
\pgfpathclose%
\pgfusepath{stroke,fill}%
\end{pgfscope}%
\begin{pgfscope}%
\pgfpathrectangle{\pgfqpoint{0.100000in}{0.220728in}}{\pgfqpoint{3.696000in}{3.696000in}}%
\pgfusepath{clip}%
\pgfsetbuttcap%
\pgfsetroundjoin%
\definecolor{currentfill}{rgb}{0.121569,0.466667,0.705882}%
\pgfsetfillcolor{currentfill}%
\pgfsetfillopacity{0.762564}%
\pgfsetlinewidth{1.003750pt}%
\definecolor{currentstroke}{rgb}{0.121569,0.466667,0.705882}%
\pgfsetstrokecolor{currentstroke}%
\pgfsetstrokeopacity{0.762564}%
\pgfsetdash{}{0pt}%
\pgfpathmoveto{\pgfqpoint{3.115687in}{2.408797in}}%
\pgfpathcurveto{\pgfqpoint{3.123923in}{2.408797in}}{\pgfqpoint{3.131823in}{2.412069in}}{\pgfqpoint{3.137647in}{2.417893in}}%
\pgfpathcurveto{\pgfqpoint{3.143471in}{2.423717in}}{\pgfqpoint{3.146743in}{2.431617in}}{\pgfqpoint{3.146743in}{2.439854in}}%
\pgfpathcurveto{\pgfqpoint{3.146743in}{2.448090in}}{\pgfqpoint{3.143471in}{2.455990in}}{\pgfqpoint{3.137647in}{2.461814in}}%
\pgfpathcurveto{\pgfqpoint{3.131823in}{2.467638in}}{\pgfqpoint{3.123923in}{2.470910in}}{\pgfqpoint{3.115687in}{2.470910in}}%
\pgfpathcurveto{\pgfqpoint{3.107450in}{2.470910in}}{\pgfqpoint{3.099550in}{2.467638in}}{\pgfqpoint{3.093726in}{2.461814in}}%
\pgfpathcurveto{\pgfqpoint{3.087902in}{2.455990in}}{\pgfqpoint{3.084630in}{2.448090in}}{\pgfqpoint{3.084630in}{2.439854in}}%
\pgfpathcurveto{\pgfqpoint{3.084630in}{2.431617in}}{\pgfqpoint{3.087902in}{2.423717in}}{\pgfqpoint{3.093726in}{2.417893in}}%
\pgfpathcurveto{\pgfqpoint{3.099550in}{2.412069in}}{\pgfqpoint{3.107450in}{2.408797in}}{\pgfqpoint{3.115687in}{2.408797in}}%
\pgfpathclose%
\pgfusepath{stroke,fill}%
\end{pgfscope}%
\begin{pgfscope}%
\pgfpathrectangle{\pgfqpoint{0.100000in}{0.220728in}}{\pgfqpoint{3.696000in}{3.696000in}}%
\pgfusepath{clip}%
\pgfsetbuttcap%
\pgfsetroundjoin%
\definecolor{currentfill}{rgb}{0.121569,0.466667,0.705882}%
\pgfsetfillcolor{currentfill}%
\pgfsetfillopacity{0.763150}%
\pgfsetlinewidth{1.003750pt}%
\definecolor{currentstroke}{rgb}{0.121569,0.466667,0.705882}%
\pgfsetstrokecolor{currentstroke}%
\pgfsetstrokeopacity{0.763150}%
\pgfsetdash{}{0pt}%
\pgfpathmoveto{\pgfqpoint{3.114050in}{2.405768in}}%
\pgfpathcurveto{\pgfqpoint{3.122287in}{2.405768in}}{\pgfqpoint{3.130187in}{2.409041in}}{\pgfqpoint{3.136011in}{2.414865in}}%
\pgfpathcurveto{\pgfqpoint{3.141835in}{2.420689in}}{\pgfqpoint{3.145107in}{2.428589in}}{\pgfqpoint{3.145107in}{2.436825in}}%
\pgfpathcurveto{\pgfqpoint{3.145107in}{2.445061in}}{\pgfqpoint{3.141835in}{2.452961in}}{\pgfqpoint{3.136011in}{2.458785in}}%
\pgfpathcurveto{\pgfqpoint{3.130187in}{2.464609in}}{\pgfqpoint{3.122287in}{2.467881in}}{\pgfqpoint{3.114050in}{2.467881in}}%
\pgfpathcurveto{\pgfqpoint{3.105814in}{2.467881in}}{\pgfqpoint{3.097914in}{2.464609in}}{\pgfqpoint{3.092090in}{2.458785in}}%
\pgfpathcurveto{\pgfqpoint{3.086266in}{2.452961in}}{\pgfqpoint{3.082994in}{2.445061in}}{\pgfqpoint{3.082994in}{2.436825in}}%
\pgfpathcurveto{\pgfqpoint{3.082994in}{2.428589in}}{\pgfqpoint{3.086266in}{2.420689in}}{\pgfqpoint{3.092090in}{2.414865in}}%
\pgfpathcurveto{\pgfqpoint{3.097914in}{2.409041in}}{\pgfqpoint{3.105814in}{2.405768in}}{\pgfqpoint{3.114050in}{2.405768in}}%
\pgfpathclose%
\pgfusepath{stroke,fill}%
\end{pgfscope}%
\begin{pgfscope}%
\pgfpathrectangle{\pgfqpoint{0.100000in}{0.220728in}}{\pgfqpoint{3.696000in}{3.696000in}}%
\pgfusepath{clip}%
\pgfsetbuttcap%
\pgfsetroundjoin%
\definecolor{currentfill}{rgb}{0.121569,0.466667,0.705882}%
\pgfsetfillcolor{currentfill}%
\pgfsetfillopacity{0.763437}%
\pgfsetlinewidth{1.003750pt}%
\definecolor{currentstroke}{rgb}{0.121569,0.466667,0.705882}%
\pgfsetstrokecolor{currentstroke}%
\pgfsetstrokeopacity{0.763437}%
\pgfsetdash{}{0pt}%
\pgfpathmoveto{\pgfqpoint{3.113028in}{2.404124in}}%
\pgfpathcurveto{\pgfqpoint{3.121264in}{2.404124in}}{\pgfqpoint{3.129164in}{2.407396in}}{\pgfqpoint{3.134988in}{2.413220in}}%
\pgfpathcurveto{\pgfqpoint{3.140812in}{2.419044in}}{\pgfqpoint{3.144084in}{2.426944in}}{\pgfqpoint{3.144084in}{2.435180in}}%
\pgfpathcurveto{\pgfqpoint{3.144084in}{2.443416in}}{\pgfqpoint{3.140812in}{2.451316in}}{\pgfqpoint{3.134988in}{2.457140in}}%
\pgfpathcurveto{\pgfqpoint{3.129164in}{2.462964in}}{\pgfqpoint{3.121264in}{2.466237in}}{\pgfqpoint{3.113028in}{2.466237in}}%
\pgfpathcurveto{\pgfqpoint{3.104792in}{2.466237in}}{\pgfqpoint{3.096892in}{2.462964in}}{\pgfqpoint{3.091068in}{2.457140in}}%
\pgfpathcurveto{\pgfqpoint{3.085244in}{2.451316in}}{\pgfqpoint{3.081971in}{2.443416in}}{\pgfqpoint{3.081971in}{2.435180in}}%
\pgfpathcurveto{\pgfqpoint{3.081971in}{2.426944in}}{\pgfqpoint{3.085244in}{2.419044in}}{\pgfqpoint{3.091068in}{2.413220in}}%
\pgfpathcurveto{\pgfqpoint{3.096892in}{2.407396in}}{\pgfqpoint{3.104792in}{2.404124in}}{\pgfqpoint{3.113028in}{2.404124in}}%
\pgfpathclose%
\pgfusepath{stroke,fill}%
\end{pgfscope}%
\begin{pgfscope}%
\pgfpathrectangle{\pgfqpoint{0.100000in}{0.220728in}}{\pgfqpoint{3.696000in}{3.696000in}}%
\pgfusepath{clip}%
\pgfsetbuttcap%
\pgfsetroundjoin%
\definecolor{currentfill}{rgb}{0.121569,0.466667,0.705882}%
\pgfsetfillcolor{currentfill}%
\pgfsetfillopacity{0.763622}%
\pgfsetlinewidth{1.003750pt}%
\definecolor{currentstroke}{rgb}{0.121569,0.466667,0.705882}%
\pgfsetstrokecolor{currentstroke}%
\pgfsetstrokeopacity{0.763622}%
\pgfsetdash{}{0pt}%
\pgfpathmoveto{\pgfqpoint{3.112755in}{2.403042in}}%
\pgfpathcurveto{\pgfqpoint{3.120991in}{2.403042in}}{\pgfqpoint{3.128891in}{2.406314in}}{\pgfqpoint{3.134715in}{2.412138in}}%
\pgfpathcurveto{\pgfqpoint{3.140539in}{2.417962in}}{\pgfqpoint{3.143811in}{2.425862in}}{\pgfqpoint{3.143811in}{2.434098in}}%
\pgfpathcurveto{\pgfqpoint{3.143811in}{2.442334in}}{\pgfqpoint{3.140539in}{2.450234in}}{\pgfqpoint{3.134715in}{2.456058in}}%
\pgfpathcurveto{\pgfqpoint{3.128891in}{2.461882in}}{\pgfqpoint{3.120991in}{2.465155in}}{\pgfqpoint{3.112755in}{2.465155in}}%
\pgfpathcurveto{\pgfqpoint{3.104519in}{2.465155in}}{\pgfqpoint{3.096619in}{2.461882in}}{\pgfqpoint{3.090795in}{2.456058in}}%
\pgfpathcurveto{\pgfqpoint{3.084971in}{2.450234in}}{\pgfqpoint{3.081698in}{2.442334in}}{\pgfqpoint{3.081698in}{2.434098in}}%
\pgfpathcurveto{\pgfqpoint{3.081698in}{2.425862in}}{\pgfqpoint{3.084971in}{2.417962in}}{\pgfqpoint{3.090795in}{2.412138in}}%
\pgfpathcurveto{\pgfqpoint{3.096619in}{2.406314in}}{\pgfqpoint{3.104519in}{2.403042in}}{\pgfqpoint{3.112755in}{2.403042in}}%
\pgfpathclose%
\pgfusepath{stroke,fill}%
\end{pgfscope}%
\begin{pgfscope}%
\pgfpathrectangle{\pgfqpoint{0.100000in}{0.220728in}}{\pgfqpoint{3.696000in}{3.696000in}}%
\pgfusepath{clip}%
\pgfsetbuttcap%
\pgfsetroundjoin%
\definecolor{currentfill}{rgb}{0.121569,0.466667,0.705882}%
\pgfsetfillcolor{currentfill}%
\pgfsetfillopacity{0.763978}%
\pgfsetlinewidth{1.003750pt}%
\definecolor{currentstroke}{rgb}{0.121569,0.466667,0.705882}%
\pgfsetstrokecolor{currentstroke}%
\pgfsetstrokeopacity{0.763978}%
\pgfsetdash{}{0pt}%
\pgfpathmoveto{\pgfqpoint{3.111629in}{2.401079in}}%
\pgfpathcurveto{\pgfqpoint{3.119866in}{2.401079in}}{\pgfqpoint{3.127766in}{2.404351in}}{\pgfqpoint{3.133590in}{2.410175in}}%
\pgfpathcurveto{\pgfqpoint{3.139413in}{2.415999in}}{\pgfqpoint{3.142686in}{2.423899in}}{\pgfqpoint{3.142686in}{2.432135in}}%
\pgfpathcurveto{\pgfqpoint{3.142686in}{2.440372in}}{\pgfqpoint{3.139413in}{2.448272in}}{\pgfqpoint{3.133590in}{2.454096in}}%
\pgfpathcurveto{\pgfqpoint{3.127766in}{2.459920in}}{\pgfqpoint{3.119866in}{2.463192in}}{\pgfqpoint{3.111629in}{2.463192in}}%
\pgfpathcurveto{\pgfqpoint{3.103393in}{2.463192in}}{\pgfqpoint{3.095493in}{2.459920in}}{\pgfqpoint{3.089669in}{2.454096in}}%
\pgfpathcurveto{\pgfqpoint{3.083845in}{2.448272in}}{\pgfqpoint{3.080573in}{2.440372in}}{\pgfqpoint{3.080573in}{2.432135in}}%
\pgfpathcurveto{\pgfqpoint{3.080573in}{2.423899in}}{\pgfqpoint{3.083845in}{2.415999in}}{\pgfqpoint{3.089669in}{2.410175in}}%
\pgfpathcurveto{\pgfqpoint{3.095493in}{2.404351in}}{\pgfqpoint{3.103393in}{2.401079in}}{\pgfqpoint{3.111629in}{2.401079in}}%
\pgfpathclose%
\pgfusepath{stroke,fill}%
\end{pgfscope}%
\begin{pgfscope}%
\pgfpathrectangle{\pgfqpoint{0.100000in}{0.220728in}}{\pgfqpoint{3.696000in}{3.696000in}}%
\pgfusepath{clip}%
\pgfsetbuttcap%
\pgfsetroundjoin%
\definecolor{currentfill}{rgb}{0.121569,0.466667,0.705882}%
\pgfsetfillcolor{currentfill}%
\pgfsetfillopacity{0.764205}%
\pgfsetlinewidth{1.003750pt}%
\definecolor{currentstroke}{rgb}{0.121569,0.466667,0.705882}%
\pgfsetstrokecolor{currentstroke}%
\pgfsetstrokeopacity{0.764205}%
\pgfsetdash{}{0pt}%
\pgfpathmoveto{\pgfqpoint{3.111036in}{2.400098in}}%
\pgfpathcurveto{\pgfqpoint{3.119273in}{2.400098in}}{\pgfqpoint{3.127173in}{2.403371in}}{\pgfqpoint{3.132997in}{2.409195in}}%
\pgfpathcurveto{\pgfqpoint{3.138821in}{2.415019in}}{\pgfqpoint{3.142093in}{2.422919in}}{\pgfqpoint{3.142093in}{2.431155in}}%
\pgfpathcurveto{\pgfqpoint{3.142093in}{2.439391in}}{\pgfqpoint{3.138821in}{2.447291in}}{\pgfqpoint{3.132997in}{2.453115in}}%
\pgfpathcurveto{\pgfqpoint{3.127173in}{2.458939in}}{\pgfqpoint{3.119273in}{2.462211in}}{\pgfqpoint{3.111036in}{2.462211in}}%
\pgfpathcurveto{\pgfqpoint{3.102800in}{2.462211in}}{\pgfqpoint{3.094900in}{2.458939in}}{\pgfqpoint{3.089076in}{2.453115in}}%
\pgfpathcurveto{\pgfqpoint{3.083252in}{2.447291in}}{\pgfqpoint{3.079980in}{2.439391in}}{\pgfqpoint{3.079980in}{2.431155in}}%
\pgfpathcurveto{\pgfqpoint{3.079980in}{2.422919in}}{\pgfqpoint{3.083252in}{2.415019in}}{\pgfqpoint{3.089076in}{2.409195in}}%
\pgfpathcurveto{\pgfqpoint{3.094900in}{2.403371in}}{\pgfqpoint{3.102800in}{2.400098in}}{\pgfqpoint{3.111036in}{2.400098in}}%
\pgfpathclose%
\pgfusepath{stroke,fill}%
\end{pgfscope}%
\begin{pgfscope}%
\pgfpathrectangle{\pgfqpoint{0.100000in}{0.220728in}}{\pgfqpoint{3.696000in}{3.696000in}}%
\pgfusepath{clip}%
\pgfsetbuttcap%
\pgfsetroundjoin%
\definecolor{currentfill}{rgb}{0.121569,0.466667,0.705882}%
\pgfsetfillcolor{currentfill}%
\pgfsetfillopacity{0.764323}%
\pgfsetlinewidth{1.003750pt}%
\definecolor{currentstroke}{rgb}{0.121569,0.466667,0.705882}%
\pgfsetstrokecolor{currentstroke}%
\pgfsetstrokeopacity{0.764323}%
\pgfsetdash{}{0pt}%
\pgfpathmoveto{\pgfqpoint{3.110833in}{2.399410in}}%
\pgfpathcurveto{\pgfqpoint{3.119069in}{2.399410in}}{\pgfqpoint{3.126969in}{2.402682in}}{\pgfqpoint{3.132793in}{2.408506in}}%
\pgfpathcurveto{\pgfqpoint{3.138617in}{2.414330in}}{\pgfqpoint{3.141890in}{2.422230in}}{\pgfqpoint{3.141890in}{2.430467in}}%
\pgfpathcurveto{\pgfqpoint{3.141890in}{2.438703in}}{\pgfqpoint{3.138617in}{2.446603in}}{\pgfqpoint{3.132793in}{2.452427in}}%
\pgfpathcurveto{\pgfqpoint{3.126969in}{2.458251in}}{\pgfqpoint{3.119069in}{2.461523in}}{\pgfqpoint{3.110833in}{2.461523in}}%
\pgfpathcurveto{\pgfqpoint{3.102597in}{2.461523in}}{\pgfqpoint{3.094697in}{2.458251in}}{\pgfqpoint{3.088873in}{2.452427in}}%
\pgfpathcurveto{\pgfqpoint{3.083049in}{2.446603in}}{\pgfqpoint{3.079777in}{2.438703in}}{\pgfqpoint{3.079777in}{2.430467in}}%
\pgfpathcurveto{\pgfqpoint{3.079777in}{2.422230in}}{\pgfqpoint{3.083049in}{2.414330in}}{\pgfqpoint{3.088873in}{2.408506in}}%
\pgfpathcurveto{\pgfqpoint{3.094697in}{2.402682in}}{\pgfqpoint{3.102597in}{2.399410in}}{\pgfqpoint{3.110833in}{2.399410in}}%
\pgfpathclose%
\pgfusepath{stroke,fill}%
\end{pgfscope}%
\begin{pgfscope}%
\pgfpathrectangle{\pgfqpoint{0.100000in}{0.220728in}}{\pgfqpoint{3.696000in}{3.696000in}}%
\pgfusepath{clip}%
\pgfsetbuttcap%
\pgfsetroundjoin%
\definecolor{currentfill}{rgb}{0.121569,0.466667,0.705882}%
\pgfsetfillcolor{currentfill}%
\pgfsetfillopacity{0.764750}%
\pgfsetlinewidth{1.003750pt}%
\definecolor{currentstroke}{rgb}{0.121569,0.466667,0.705882}%
\pgfsetstrokecolor{currentstroke}%
\pgfsetstrokeopacity{0.764750}%
\pgfsetdash{}{0pt}%
\pgfpathmoveto{\pgfqpoint{3.109484in}{2.397190in}}%
\pgfpathcurveto{\pgfqpoint{3.117720in}{2.397190in}}{\pgfqpoint{3.125621in}{2.400463in}}{\pgfqpoint{3.131444in}{2.406287in}}%
\pgfpathcurveto{\pgfqpoint{3.137268in}{2.412111in}}{\pgfqpoint{3.140541in}{2.420011in}}{\pgfqpoint{3.140541in}{2.428247in}}%
\pgfpathcurveto{\pgfqpoint{3.140541in}{2.436483in}}{\pgfqpoint{3.137268in}{2.444383in}}{\pgfqpoint{3.131444in}{2.450207in}}%
\pgfpathcurveto{\pgfqpoint{3.125621in}{2.456031in}}{\pgfqpoint{3.117720in}{2.459303in}}{\pgfqpoint{3.109484in}{2.459303in}}%
\pgfpathcurveto{\pgfqpoint{3.101248in}{2.459303in}}{\pgfqpoint{3.093348in}{2.456031in}}{\pgfqpoint{3.087524in}{2.450207in}}%
\pgfpathcurveto{\pgfqpoint{3.081700in}{2.444383in}}{\pgfqpoint{3.078428in}{2.436483in}}{\pgfqpoint{3.078428in}{2.428247in}}%
\pgfpathcurveto{\pgfqpoint{3.078428in}{2.420011in}}{\pgfqpoint{3.081700in}{2.412111in}}{\pgfqpoint{3.087524in}{2.406287in}}%
\pgfpathcurveto{\pgfqpoint{3.093348in}{2.400463in}}{\pgfqpoint{3.101248in}{2.397190in}}{\pgfqpoint{3.109484in}{2.397190in}}%
\pgfpathclose%
\pgfusepath{stroke,fill}%
\end{pgfscope}%
\begin{pgfscope}%
\pgfpathrectangle{\pgfqpoint{0.100000in}{0.220728in}}{\pgfqpoint{3.696000in}{3.696000in}}%
\pgfusepath{clip}%
\pgfsetbuttcap%
\pgfsetroundjoin%
\definecolor{currentfill}{rgb}{0.121569,0.466667,0.705882}%
\pgfsetfillcolor{currentfill}%
\pgfsetfillopacity{0.764767}%
\pgfsetlinewidth{1.003750pt}%
\definecolor{currentstroke}{rgb}{0.121569,0.466667,0.705882}%
\pgfsetstrokecolor{currentstroke}%
\pgfsetstrokeopacity{0.764767}%
\pgfsetdash{}{0pt}%
\pgfpathmoveto{\pgfqpoint{1.159641in}{1.196253in}}%
\pgfpathcurveto{\pgfqpoint{1.167877in}{1.196253in}}{\pgfqpoint{1.175778in}{1.199525in}}{\pgfqpoint{1.181601in}{1.205349in}}%
\pgfpathcurveto{\pgfqpoint{1.187425in}{1.211173in}}{\pgfqpoint{1.190698in}{1.219073in}}{\pgfqpoint{1.190698in}{1.227309in}}%
\pgfpathcurveto{\pgfqpoint{1.190698in}{1.235545in}}{\pgfqpoint{1.187425in}{1.243445in}}{\pgfqpoint{1.181601in}{1.249269in}}%
\pgfpathcurveto{\pgfqpoint{1.175778in}{1.255093in}}{\pgfqpoint{1.167877in}{1.258366in}}{\pgfqpoint{1.159641in}{1.258366in}}%
\pgfpathcurveto{\pgfqpoint{1.151405in}{1.258366in}}{\pgfqpoint{1.143505in}{1.255093in}}{\pgfqpoint{1.137681in}{1.249269in}}%
\pgfpathcurveto{\pgfqpoint{1.131857in}{1.243445in}}{\pgfqpoint{1.128585in}{1.235545in}}{\pgfqpoint{1.128585in}{1.227309in}}%
\pgfpathcurveto{\pgfqpoint{1.128585in}{1.219073in}}{\pgfqpoint{1.131857in}{1.211173in}}{\pgfqpoint{1.137681in}{1.205349in}}%
\pgfpathcurveto{\pgfqpoint{1.143505in}{1.199525in}}{\pgfqpoint{1.151405in}{1.196253in}}{\pgfqpoint{1.159641in}{1.196253in}}%
\pgfpathclose%
\pgfusepath{stroke,fill}%
\end{pgfscope}%
\begin{pgfscope}%
\pgfpathrectangle{\pgfqpoint{0.100000in}{0.220728in}}{\pgfqpoint{3.696000in}{3.696000in}}%
\pgfusepath{clip}%
\pgfsetbuttcap%
\pgfsetroundjoin%
\definecolor{currentfill}{rgb}{0.121569,0.466667,0.705882}%
\pgfsetfillcolor{currentfill}%
\pgfsetfillopacity{0.765032}%
\pgfsetlinewidth{1.003750pt}%
\definecolor{currentstroke}{rgb}{0.121569,0.466667,0.705882}%
\pgfsetstrokecolor{currentstroke}%
\pgfsetstrokeopacity{0.765032}%
\pgfsetdash{}{0pt}%
\pgfpathmoveto{\pgfqpoint{3.108912in}{2.395960in}}%
\pgfpathcurveto{\pgfqpoint{3.117148in}{2.395960in}}{\pgfqpoint{3.125048in}{2.399232in}}{\pgfqpoint{3.130872in}{2.405056in}}%
\pgfpathcurveto{\pgfqpoint{3.136696in}{2.410880in}}{\pgfqpoint{3.139968in}{2.418780in}}{\pgfqpoint{3.139968in}{2.427016in}}%
\pgfpathcurveto{\pgfqpoint{3.139968in}{2.435253in}}{\pgfqpoint{3.136696in}{2.443153in}}{\pgfqpoint{3.130872in}{2.448977in}}%
\pgfpathcurveto{\pgfqpoint{3.125048in}{2.454801in}}{\pgfqpoint{3.117148in}{2.458073in}}{\pgfqpoint{3.108912in}{2.458073in}}%
\pgfpathcurveto{\pgfqpoint{3.100675in}{2.458073in}}{\pgfqpoint{3.092775in}{2.454801in}}{\pgfqpoint{3.086951in}{2.448977in}}%
\pgfpathcurveto{\pgfqpoint{3.081128in}{2.443153in}}{\pgfqpoint{3.077855in}{2.435253in}}{\pgfqpoint{3.077855in}{2.427016in}}%
\pgfpathcurveto{\pgfqpoint{3.077855in}{2.418780in}}{\pgfqpoint{3.081128in}{2.410880in}}{\pgfqpoint{3.086951in}{2.405056in}}%
\pgfpathcurveto{\pgfqpoint{3.092775in}{2.399232in}}{\pgfqpoint{3.100675in}{2.395960in}}{\pgfqpoint{3.108912in}{2.395960in}}%
\pgfpathclose%
\pgfusepath{stroke,fill}%
\end{pgfscope}%
\begin{pgfscope}%
\pgfpathrectangle{\pgfqpoint{0.100000in}{0.220728in}}{\pgfqpoint{3.696000in}{3.696000in}}%
\pgfusepath{clip}%
\pgfsetbuttcap%
\pgfsetroundjoin%
\definecolor{currentfill}{rgb}{0.121569,0.466667,0.705882}%
\pgfsetfillcolor{currentfill}%
\pgfsetfillopacity{0.765177}%
\pgfsetlinewidth{1.003750pt}%
\definecolor{currentstroke}{rgb}{0.121569,0.466667,0.705882}%
\pgfsetstrokecolor{currentstroke}%
\pgfsetstrokeopacity{0.765177}%
\pgfsetdash{}{0pt}%
\pgfpathmoveto{\pgfqpoint{3.108640in}{2.395205in}}%
\pgfpathcurveto{\pgfqpoint{3.116876in}{2.395205in}}{\pgfqpoint{3.124776in}{2.398477in}}{\pgfqpoint{3.130600in}{2.404301in}}%
\pgfpathcurveto{\pgfqpoint{3.136424in}{2.410125in}}{\pgfqpoint{3.139696in}{2.418025in}}{\pgfqpoint{3.139696in}{2.426261in}}%
\pgfpathcurveto{\pgfqpoint{3.139696in}{2.434498in}}{\pgfqpoint{3.136424in}{2.442398in}}{\pgfqpoint{3.130600in}{2.448222in}}%
\pgfpathcurveto{\pgfqpoint{3.124776in}{2.454045in}}{\pgfqpoint{3.116876in}{2.457318in}}{\pgfqpoint{3.108640in}{2.457318in}}%
\pgfpathcurveto{\pgfqpoint{3.100404in}{2.457318in}}{\pgfqpoint{3.092504in}{2.454045in}}{\pgfqpoint{3.086680in}{2.448222in}}%
\pgfpathcurveto{\pgfqpoint{3.080856in}{2.442398in}}{\pgfqpoint{3.077583in}{2.434498in}}{\pgfqpoint{3.077583in}{2.426261in}}%
\pgfpathcurveto{\pgfqpoint{3.077583in}{2.418025in}}{\pgfqpoint{3.080856in}{2.410125in}}{\pgfqpoint{3.086680in}{2.404301in}}%
\pgfpathcurveto{\pgfqpoint{3.092504in}{2.398477in}}{\pgfqpoint{3.100404in}{2.395205in}}{\pgfqpoint{3.108640in}{2.395205in}}%
\pgfpathclose%
\pgfusepath{stroke,fill}%
\end{pgfscope}%
\begin{pgfscope}%
\pgfpathrectangle{\pgfqpoint{0.100000in}{0.220728in}}{\pgfqpoint{3.696000in}{3.696000in}}%
\pgfusepath{clip}%
\pgfsetbuttcap%
\pgfsetroundjoin%
\definecolor{currentfill}{rgb}{0.121569,0.466667,0.705882}%
\pgfsetfillcolor{currentfill}%
\pgfsetfillopacity{0.765381}%
\pgfsetlinewidth{1.003750pt}%
\definecolor{currentstroke}{rgb}{0.121569,0.466667,0.705882}%
\pgfsetstrokecolor{currentstroke}%
\pgfsetstrokeopacity{0.765381}%
\pgfsetdash{}{0pt}%
\pgfpathmoveto{\pgfqpoint{3.107830in}{2.393854in}}%
\pgfpathcurveto{\pgfqpoint{3.116066in}{2.393854in}}{\pgfqpoint{3.123966in}{2.397126in}}{\pgfqpoint{3.129790in}{2.402950in}}%
\pgfpathcurveto{\pgfqpoint{3.135614in}{2.408774in}}{\pgfqpoint{3.138886in}{2.416674in}}{\pgfqpoint{3.138886in}{2.424911in}}%
\pgfpathcurveto{\pgfqpoint{3.138886in}{2.433147in}}{\pgfqpoint{3.135614in}{2.441047in}}{\pgfqpoint{3.129790in}{2.446871in}}%
\pgfpathcurveto{\pgfqpoint{3.123966in}{2.452695in}}{\pgfqpoint{3.116066in}{2.455967in}}{\pgfqpoint{3.107830in}{2.455967in}}%
\pgfpathcurveto{\pgfqpoint{3.099593in}{2.455967in}}{\pgfqpoint{3.091693in}{2.452695in}}{\pgfqpoint{3.085869in}{2.446871in}}%
\pgfpathcurveto{\pgfqpoint{3.080046in}{2.441047in}}{\pgfqpoint{3.076773in}{2.433147in}}{\pgfqpoint{3.076773in}{2.424911in}}%
\pgfpathcurveto{\pgfqpoint{3.076773in}{2.416674in}}{\pgfqpoint{3.080046in}{2.408774in}}{\pgfqpoint{3.085869in}{2.402950in}}%
\pgfpathcurveto{\pgfqpoint{3.091693in}{2.397126in}}{\pgfqpoint{3.099593in}{2.393854in}}{\pgfqpoint{3.107830in}{2.393854in}}%
\pgfpathclose%
\pgfusepath{stroke,fill}%
\end{pgfscope}%
\begin{pgfscope}%
\pgfpathrectangle{\pgfqpoint{0.100000in}{0.220728in}}{\pgfqpoint{3.696000in}{3.696000in}}%
\pgfusepath{clip}%
\pgfsetbuttcap%
\pgfsetroundjoin%
\definecolor{currentfill}{rgb}{0.121569,0.466667,0.705882}%
\pgfsetfillcolor{currentfill}%
\pgfsetfillopacity{0.765856}%
\pgfsetlinewidth{1.003750pt}%
\definecolor{currentstroke}{rgb}{0.121569,0.466667,0.705882}%
\pgfsetstrokecolor{currentstroke}%
\pgfsetstrokeopacity{0.765856}%
\pgfsetdash{}{0pt}%
\pgfpathmoveto{\pgfqpoint{3.106851in}{2.391348in}}%
\pgfpathcurveto{\pgfqpoint{3.115087in}{2.391348in}}{\pgfqpoint{3.122987in}{2.394621in}}{\pgfqpoint{3.128811in}{2.400445in}}%
\pgfpathcurveto{\pgfqpoint{3.134635in}{2.406269in}}{\pgfqpoint{3.137907in}{2.414169in}}{\pgfqpoint{3.137907in}{2.422405in}}%
\pgfpathcurveto{\pgfqpoint{3.137907in}{2.430641in}}{\pgfqpoint{3.134635in}{2.438541in}}{\pgfqpoint{3.128811in}{2.444365in}}%
\pgfpathcurveto{\pgfqpoint{3.122987in}{2.450189in}}{\pgfqpoint{3.115087in}{2.453461in}}{\pgfqpoint{3.106851in}{2.453461in}}%
\pgfpathcurveto{\pgfqpoint{3.098614in}{2.453461in}}{\pgfqpoint{3.090714in}{2.450189in}}{\pgfqpoint{3.084890in}{2.444365in}}%
\pgfpathcurveto{\pgfqpoint{3.079066in}{2.438541in}}{\pgfqpoint{3.075794in}{2.430641in}}{\pgfqpoint{3.075794in}{2.422405in}}%
\pgfpathcurveto{\pgfqpoint{3.075794in}{2.414169in}}{\pgfqpoint{3.079066in}{2.406269in}}{\pgfqpoint{3.084890in}{2.400445in}}%
\pgfpathcurveto{\pgfqpoint{3.090714in}{2.394621in}}{\pgfqpoint{3.098614in}{2.391348in}}{\pgfqpoint{3.106851in}{2.391348in}}%
\pgfpathclose%
\pgfusepath{stroke,fill}%
\end{pgfscope}%
\begin{pgfscope}%
\pgfpathrectangle{\pgfqpoint{0.100000in}{0.220728in}}{\pgfqpoint{3.696000in}{3.696000in}}%
\pgfusepath{clip}%
\pgfsetbuttcap%
\pgfsetroundjoin%
\definecolor{currentfill}{rgb}{0.121569,0.466667,0.705882}%
\pgfsetfillcolor{currentfill}%
\pgfsetfillopacity{0.766472}%
\pgfsetlinewidth{1.003750pt}%
\definecolor{currentstroke}{rgb}{0.121569,0.466667,0.705882}%
\pgfsetstrokecolor{currentstroke}%
\pgfsetstrokeopacity{0.766472}%
\pgfsetdash{}{0pt}%
\pgfpathmoveto{\pgfqpoint{3.105420in}{2.388068in}}%
\pgfpathcurveto{\pgfqpoint{3.113656in}{2.388068in}}{\pgfqpoint{3.121556in}{2.391340in}}{\pgfqpoint{3.127380in}{2.397164in}}%
\pgfpathcurveto{\pgfqpoint{3.133204in}{2.402988in}}{\pgfqpoint{3.136477in}{2.410888in}}{\pgfqpoint{3.136477in}{2.419124in}}%
\pgfpathcurveto{\pgfqpoint{3.136477in}{2.427361in}}{\pgfqpoint{3.133204in}{2.435261in}}{\pgfqpoint{3.127380in}{2.441085in}}%
\pgfpathcurveto{\pgfqpoint{3.121556in}{2.446909in}}{\pgfqpoint{3.113656in}{2.450181in}}{\pgfqpoint{3.105420in}{2.450181in}}%
\pgfpathcurveto{\pgfqpoint{3.097184in}{2.450181in}}{\pgfqpoint{3.089284in}{2.446909in}}{\pgfqpoint{3.083460in}{2.441085in}}%
\pgfpathcurveto{\pgfqpoint{3.077636in}{2.435261in}}{\pgfqpoint{3.074364in}{2.427361in}}{\pgfqpoint{3.074364in}{2.419124in}}%
\pgfpathcurveto{\pgfqpoint{3.074364in}{2.410888in}}{\pgfqpoint{3.077636in}{2.402988in}}{\pgfqpoint{3.083460in}{2.397164in}}%
\pgfpathcurveto{\pgfqpoint{3.089284in}{2.391340in}}{\pgfqpoint{3.097184in}{2.388068in}}{\pgfqpoint{3.105420in}{2.388068in}}%
\pgfpathclose%
\pgfusepath{stroke,fill}%
\end{pgfscope}%
\begin{pgfscope}%
\pgfpathrectangle{\pgfqpoint{0.100000in}{0.220728in}}{\pgfqpoint{3.696000in}{3.696000in}}%
\pgfusepath{clip}%
\pgfsetbuttcap%
\pgfsetroundjoin%
\definecolor{currentfill}{rgb}{0.121569,0.466667,0.705882}%
\pgfsetfillcolor{currentfill}%
\pgfsetfillopacity{0.767078}%
\pgfsetlinewidth{1.003750pt}%
\definecolor{currentstroke}{rgb}{0.121569,0.466667,0.705882}%
\pgfsetstrokecolor{currentstroke}%
\pgfsetstrokeopacity{0.767078}%
\pgfsetdash{}{0pt}%
\pgfpathmoveto{\pgfqpoint{3.103116in}{2.384469in}}%
\pgfpathcurveto{\pgfqpoint{3.111352in}{2.384469in}}{\pgfqpoint{3.119252in}{2.387741in}}{\pgfqpoint{3.125076in}{2.393565in}}%
\pgfpathcurveto{\pgfqpoint{3.130900in}{2.399389in}}{\pgfqpoint{3.134172in}{2.407289in}}{\pgfqpoint{3.134172in}{2.415526in}}%
\pgfpathcurveto{\pgfqpoint{3.134172in}{2.423762in}}{\pgfqpoint{3.130900in}{2.431662in}}{\pgfqpoint{3.125076in}{2.437486in}}%
\pgfpathcurveto{\pgfqpoint{3.119252in}{2.443310in}}{\pgfqpoint{3.111352in}{2.446582in}}{\pgfqpoint{3.103116in}{2.446582in}}%
\pgfpathcurveto{\pgfqpoint{3.094880in}{2.446582in}}{\pgfqpoint{3.086979in}{2.443310in}}{\pgfqpoint{3.081156in}{2.437486in}}%
\pgfpathcurveto{\pgfqpoint{3.075332in}{2.431662in}}{\pgfqpoint{3.072059in}{2.423762in}}{\pgfqpoint{3.072059in}{2.415526in}}%
\pgfpathcurveto{\pgfqpoint{3.072059in}{2.407289in}}{\pgfqpoint{3.075332in}{2.399389in}}{\pgfqpoint{3.081156in}{2.393565in}}%
\pgfpathcurveto{\pgfqpoint{3.086979in}{2.387741in}}{\pgfqpoint{3.094880in}{2.384469in}}{\pgfqpoint{3.103116in}{2.384469in}}%
\pgfpathclose%
\pgfusepath{stroke,fill}%
\end{pgfscope}%
\begin{pgfscope}%
\pgfpathrectangle{\pgfqpoint{0.100000in}{0.220728in}}{\pgfqpoint{3.696000in}{3.696000in}}%
\pgfusepath{clip}%
\pgfsetbuttcap%
\pgfsetroundjoin%
\definecolor{currentfill}{rgb}{0.121569,0.466667,0.705882}%
\pgfsetfillcolor{currentfill}%
\pgfsetfillopacity{0.768179}%
\pgfsetlinewidth{1.003750pt}%
\definecolor{currentstroke}{rgb}{0.121569,0.466667,0.705882}%
\pgfsetstrokecolor{currentstroke}%
\pgfsetstrokeopacity{0.768179}%
\pgfsetdash{}{0pt}%
\pgfpathmoveto{\pgfqpoint{3.101138in}{2.378874in}}%
\pgfpathcurveto{\pgfqpoint{3.109375in}{2.378874in}}{\pgfqpoint{3.117275in}{2.382146in}}{\pgfqpoint{3.123099in}{2.387970in}}%
\pgfpathcurveto{\pgfqpoint{3.128923in}{2.393794in}}{\pgfqpoint{3.132195in}{2.401694in}}{\pgfqpoint{3.132195in}{2.409930in}}%
\pgfpathcurveto{\pgfqpoint{3.132195in}{2.418166in}}{\pgfqpoint{3.128923in}{2.426066in}}{\pgfqpoint{3.123099in}{2.431890in}}%
\pgfpathcurveto{\pgfqpoint{3.117275in}{2.437714in}}{\pgfqpoint{3.109375in}{2.440987in}}{\pgfqpoint{3.101138in}{2.440987in}}%
\pgfpathcurveto{\pgfqpoint{3.092902in}{2.440987in}}{\pgfqpoint{3.085002in}{2.437714in}}{\pgfqpoint{3.079178in}{2.431890in}}%
\pgfpathcurveto{\pgfqpoint{3.073354in}{2.426066in}}{\pgfqpoint{3.070082in}{2.418166in}}{\pgfqpoint{3.070082in}{2.409930in}}%
\pgfpathcurveto{\pgfqpoint{3.070082in}{2.401694in}}{\pgfqpoint{3.073354in}{2.393794in}}{\pgfqpoint{3.079178in}{2.387970in}}%
\pgfpathcurveto{\pgfqpoint{3.085002in}{2.382146in}}{\pgfqpoint{3.092902in}{2.378874in}}{\pgfqpoint{3.101138in}{2.378874in}}%
\pgfpathclose%
\pgfusepath{stroke,fill}%
\end{pgfscope}%
\begin{pgfscope}%
\pgfpathrectangle{\pgfqpoint{0.100000in}{0.220728in}}{\pgfqpoint{3.696000in}{3.696000in}}%
\pgfusepath{clip}%
\pgfsetbuttcap%
\pgfsetroundjoin%
\definecolor{currentfill}{rgb}{0.121569,0.466667,0.705882}%
\pgfsetfillcolor{currentfill}%
\pgfsetfillopacity{0.769339}%
\pgfsetlinewidth{1.003750pt}%
\definecolor{currentstroke}{rgb}{0.121569,0.466667,0.705882}%
\pgfsetstrokecolor{currentstroke}%
\pgfsetstrokeopacity{0.769339}%
\pgfsetdash{}{0pt}%
\pgfpathmoveto{\pgfqpoint{3.098333in}{2.372995in}}%
\pgfpathcurveto{\pgfqpoint{3.106569in}{2.372995in}}{\pgfqpoint{3.114469in}{2.376267in}}{\pgfqpoint{3.120293in}{2.382091in}}%
\pgfpathcurveto{\pgfqpoint{3.126117in}{2.387915in}}{\pgfqpoint{3.129389in}{2.395815in}}{\pgfqpoint{3.129389in}{2.404051in}}%
\pgfpathcurveto{\pgfqpoint{3.129389in}{2.412287in}}{\pgfqpoint{3.126117in}{2.420187in}}{\pgfqpoint{3.120293in}{2.426011in}}%
\pgfpathcurveto{\pgfqpoint{3.114469in}{2.431835in}}{\pgfqpoint{3.106569in}{2.435108in}}{\pgfqpoint{3.098333in}{2.435108in}}%
\pgfpathcurveto{\pgfqpoint{3.090097in}{2.435108in}}{\pgfqpoint{3.082196in}{2.431835in}}{\pgfqpoint{3.076373in}{2.426011in}}%
\pgfpathcurveto{\pgfqpoint{3.070549in}{2.420187in}}{\pgfqpoint{3.067276in}{2.412287in}}{\pgfqpoint{3.067276in}{2.404051in}}%
\pgfpathcurveto{\pgfqpoint{3.067276in}{2.395815in}}{\pgfqpoint{3.070549in}{2.387915in}}{\pgfqpoint{3.076373in}{2.382091in}}%
\pgfpathcurveto{\pgfqpoint{3.082196in}{2.376267in}}{\pgfqpoint{3.090097in}{2.372995in}}{\pgfqpoint{3.098333in}{2.372995in}}%
\pgfpathclose%
\pgfusepath{stroke,fill}%
\end{pgfscope}%
\begin{pgfscope}%
\pgfpathrectangle{\pgfqpoint{0.100000in}{0.220728in}}{\pgfqpoint{3.696000in}{3.696000in}}%
\pgfusepath{clip}%
\pgfsetbuttcap%
\pgfsetroundjoin%
\definecolor{currentfill}{rgb}{0.121569,0.466667,0.705882}%
\pgfsetfillcolor{currentfill}%
\pgfsetfillopacity{0.770327}%
\pgfsetlinewidth{1.003750pt}%
\definecolor{currentstroke}{rgb}{0.121569,0.466667,0.705882}%
\pgfsetstrokecolor{currentstroke}%
\pgfsetstrokeopacity{0.770327}%
\pgfsetdash{}{0pt}%
\pgfpathmoveto{\pgfqpoint{1.182361in}{1.190847in}}%
\pgfpathcurveto{\pgfqpoint{1.190597in}{1.190847in}}{\pgfqpoint{1.198497in}{1.194119in}}{\pgfqpoint{1.204321in}{1.199943in}}%
\pgfpathcurveto{\pgfqpoint{1.210145in}{1.205767in}}{\pgfqpoint{1.213418in}{1.213667in}}{\pgfqpoint{1.213418in}{1.221903in}}%
\pgfpathcurveto{\pgfqpoint{1.213418in}{1.230140in}}{\pgfqpoint{1.210145in}{1.238040in}}{\pgfqpoint{1.204321in}{1.243863in}}%
\pgfpathcurveto{\pgfqpoint{1.198497in}{1.249687in}}{\pgfqpoint{1.190597in}{1.252960in}}{\pgfqpoint{1.182361in}{1.252960in}}%
\pgfpathcurveto{\pgfqpoint{1.174125in}{1.252960in}}{\pgfqpoint{1.166225in}{1.249687in}}{\pgfqpoint{1.160401in}{1.243863in}}%
\pgfpathcurveto{\pgfqpoint{1.154577in}{1.238040in}}{\pgfqpoint{1.151305in}{1.230140in}}{\pgfqpoint{1.151305in}{1.221903in}}%
\pgfpathcurveto{\pgfqpoint{1.151305in}{1.213667in}}{\pgfqpoint{1.154577in}{1.205767in}}{\pgfqpoint{1.160401in}{1.199943in}}%
\pgfpathcurveto{\pgfqpoint{1.166225in}{1.194119in}}{\pgfqpoint{1.174125in}{1.190847in}}{\pgfqpoint{1.182361in}{1.190847in}}%
\pgfpathclose%
\pgfusepath{stroke,fill}%
\end{pgfscope}%
\begin{pgfscope}%
\pgfpathrectangle{\pgfqpoint{0.100000in}{0.220728in}}{\pgfqpoint{3.696000in}{3.696000in}}%
\pgfusepath{clip}%
\pgfsetbuttcap%
\pgfsetroundjoin%
\definecolor{currentfill}{rgb}{0.121569,0.466667,0.705882}%
\pgfsetfillcolor{currentfill}%
\pgfsetfillopacity{0.770517}%
\pgfsetlinewidth{1.003750pt}%
\definecolor{currentstroke}{rgb}{0.121569,0.466667,0.705882}%
\pgfsetstrokecolor{currentstroke}%
\pgfsetstrokeopacity{0.770517}%
\pgfsetdash{}{0pt}%
\pgfpathmoveto{\pgfqpoint{3.094440in}{2.366457in}}%
\pgfpathcurveto{\pgfqpoint{3.102676in}{2.366457in}}{\pgfqpoint{3.110576in}{2.369729in}}{\pgfqpoint{3.116400in}{2.375553in}}%
\pgfpathcurveto{\pgfqpoint{3.122224in}{2.381377in}}{\pgfqpoint{3.125496in}{2.389277in}}{\pgfqpoint{3.125496in}{2.397513in}}%
\pgfpathcurveto{\pgfqpoint{3.125496in}{2.405749in}}{\pgfqpoint{3.122224in}{2.413649in}}{\pgfqpoint{3.116400in}{2.419473in}}%
\pgfpathcurveto{\pgfqpoint{3.110576in}{2.425297in}}{\pgfqpoint{3.102676in}{2.428570in}}{\pgfqpoint{3.094440in}{2.428570in}}%
\pgfpathcurveto{\pgfqpoint{3.086204in}{2.428570in}}{\pgfqpoint{3.078304in}{2.425297in}}{\pgfqpoint{3.072480in}{2.419473in}}%
\pgfpathcurveto{\pgfqpoint{3.066656in}{2.413649in}}{\pgfqpoint{3.063383in}{2.405749in}}{\pgfqpoint{3.063383in}{2.397513in}}%
\pgfpathcurveto{\pgfqpoint{3.063383in}{2.389277in}}{\pgfqpoint{3.066656in}{2.381377in}}{\pgfqpoint{3.072480in}{2.375553in}}%
\pgfpathcurveto{\pgfqpoint{3.078304in}{2.369729in}}{\pgfqpoint{3.086204in}{2.366457in}}{\pgfqpoint{3.094440in}{2.366457in}}%
\pgfpathclose%
\pgfusepath{stroke,fill}%
\end{pgfscope}%
\begin{pgfscope}%
\pgfpathrectangle{\pgfqpoint{0.100000in}{0.220728in}}{\pgfqpoint{3.696000in}{3.696000in}}%
\pgfusepath{clip}%
\pgfsetbuttcap%
\pgfsetroundjoin%
\definecolor{currentfill}{rgb}{0.121569,0.466667,0.705882}%
\pgfsetfillcolor{currentfill}%
\pgfsetfillopacity{0.772108}%
\pgfsetlinewidth{1.003750pt}%
\definecolor{currentstroke}{rgb}{0.121569,0.466667,0.705882}%
\pgfsetstrokecolor{currentstroke}%
\pgfsetstrokeopacity{0.772108}%
\pgfsetdash{}{0pt}%
\pgfpathmoveto{\pgfqpoint{3.091143in}{2.357266in}}%
\pgfpathcurveto{\pgfqpoint{3.099379in}{2.357266in}}{\pgfqpoint{3.107280in}{2.360538in}}{\pgfqpoint{3.113103in}{2.366362in}}%
\pgfpathcurveto{\pgfqpoint{3.118927in}{2.372186in}}{\pgfqpoint{3.122200in}{2.380086in}}{\pgfqpoint{3.122200in}{2.388322in}}%
\pgfpathcurveto{\pgfqpoint{3.122200in}{2.396559in}}{\pgfqpoint{3.118927in}{2.404459in}}{\pgfqpoint{3.113103in}{2.410283in}}%
\pgfpathcurveto{\pgfqpoint{3.107280in}{2.416106in}}{\pgfqpoint{3.099379in}{2.419379in}}{\pgfqpoint{3.091143in}{2.419379in}}%
\pgfpathcurveto{\pgfqpoint{3.082907in}{2.419379in}}{\pgfqpoint{3.075007in}{2.416106in}}{\pgfqpoint{3.069183in}{2.410283in}}%
\pgfpathcurveto{\pgfqpoint{3.063359in}{2.404459in}}{\pgfqpoint{3.060087in}{2.396559in}}{\pgfqpoint{3.060087in}{2.388322in}}%
\pgfpathcurveto{\pgfqpoint{3.060087in}{2.380086in}}{\pgfqpoint{3.063359in}{2.372186in}}{\pgfqpoint{3.069183in}{2.366362in}}%
\pgfpathcurveto{\pgfqpoint{3.075007in}{2.360538in}}{\pgfqpoint{3.082907in}{2.357266in}}{\pgfqpoint{3.091143in}{2.357266in}}%
\pgfpathclose%
\pgfusepath{stroke,fill}%
\end{pgfscope}%
\begin{pgfscope}%
\pgfpathrectangle{\pgfqpoint{0.100000in}{0.220728in}}{\pgfqpoint{3.696000in}{3.696000in}}%
\pgfusepath{clip}%
\pgfsetbuttcap%
\pgfsetroundjoin%
\definecolor{currentfill}{rgb}{0.121569,0.466667,0.705882}%
\pgfsetfillcolor{currentfill}%
\pgfsetfillopacity{0.773011}%
\pgfsetlinewidth{1.003750pt}%
\definecolor{currentstroke}{rgb}{0.121569,0.466667,0.705882}%
\pgfsetstrokecolor{currentstroke}%
\pgfsetstrokeopacity{0.773011}%
\pgfsetdash{}{0pt}%
\pgfpathmoveto{\pgfqpoint{3.088846in}{2.352814in}}%
\pgfpathcurveto{\pgfqpoint{3.097082in}{2.352814in}}{\pgfqpoint{3.104982in}{2.356086in}}{\pgfqpoint{3.110806in}{2.361910in}}%
\pgfpathcurveto{\pgfqpoint{3.116630in}{2.367734in}}{\pgfqpoint{3.119903in}{2.375634in}}{\pgfqpoint{3.119903in}{2.383870in}}%
\pgfpathcurveto{\pgfqpoint{3.119903in}{2.392107in}}{\pgfqpoint{3.116630in}{2.400007in}}{\pgfqpoint{3.110806in}{2.405831in}}%
\pgfpathcurveto{\pgfqpoint{3.104982in}{2.411654in}}{\pgfqpoint{3.097082in}{2.414927in}}{\pgfqpoint{3.088846in}{2.414927in}}%
\pgfpathcurveto{\pgfqpoint{3.080610in}{2.414927in}}{\pgfqpoint{3.072710in}{2.411654in}}{\pgfqpoint{3.066886in}{2.405831in}}%
\pgfpathcurveto{\pgfqpoint{3.061062in}{2.400007in}}{\pgfqpoint{3.057790in}{2.392107in}}{\pgfqpoint{3.057790in}{2.383870in}}%
\pgfpathcurveto{\pgfqpoint{3.057790in}{2.375634in}}{\pgfqpoint{3.061062in}{2.367734in}}{\pgfqpoint{3.066886in}{2.361910in}}%
\pgfpathcurveto{\pgfqpoint{3.072710in}{2.356086in}}{\pgfqpoint{3.080610in}{2.352814in}}{\pgfqpoint{3.088846in}{2.352814in}}%
\pgfpathclose%
\pgfusepath{stroke,fill}%
\end{pgfscope}%
\begin{pgfscope}%
\pgfpathrectangle{\pgfqpoint{0.100000in}{0.220728in}}{\pgfqpoint{3.696000in}{3.696000in}}%
\pgfusepath{clip}%
\pgfsetbuttcap%
\pgfsetroundjoin%
\definecolor{currentfill}{rgb}{0.121569,0.466667,0.705882}%
\pgfsetfillcolor{currentfill}%
\pgfsetfillopacity{0.773495}%
\pgfsetlinewidth{1.003750pt}%
\definecolor{currentstroke}{rgb}{0.121569,0.466667,0.705882}%
\pgfsetstrokecolor{currentstroke}%
\pgfsetstrokeopacity{0.773495}%
\pgfsetdash{}{0pt}%
\pgfpathmoveto{\pgfqpoint{3.087433in}{2.350504in}}%
\pgfpathcurveto{\pgfqpoint{3.095669in}{2.350504in}}{\pgfqpoint{3.103569in}{2.353777in}}{\pgfqpoint{3.109393in}{2.359601in}}%
\pgfpathcurveto{\pgfqpoint{3.115217in}{2.365425in}}{\pgfqpoint{3.118489in}{2.373325in}}{\pgfqpoint{3.118489in}{2.381561in}}%
\pgfpathcurveto{\pgfqpoint{3.118489in}{2.389797in}}{\pgfqpoint{3.115217in}{2.397697in}}{\pgfqpoint{3.109393in}{2.403521in}}%
\pgfpathcurveto{\pgfqpoint{3.103569in}{2.409345in}}{\pgfqpoint{3.095669in}{2.412617in}}{\pgfqpoint{3.087433in}{2.412617in}}%
\pgfpathcurveto{\pgfqpoint{3.079197in}{2.412617in}}{\pgfqpoint{3.071296in}{2.409345in}}{\pgfqpoint{3.065473in}{2.403521in}}%
\pgfpathcurveto{\pgfqpoint{3.059649in}{2.397697in}}{\pgfqpoint{3.056376in}{2.389797in}}{\pgfqpoint{3.056376in}{2.381561in}}%
\pgfpathcurveto{\pgfqpoint{3.056376in}{2.373325in}}{\pgfqpoint{3.059649in}{2.365425in}}{\pgfqpoint{3.065473in}{2.359601in}}%
\pgfpathcurveto{\pgfqpoint{3.071296in}{2.353777in}}{\pgfqpoint{3.079197in}{2.350504in}}{\pgfqpoint{3.087433in}{2.350504in}}%
\pgfpathclose%
\pgfusepath{stroke,fill}%
\end{pgfscope}%
\begin{pgfscope}%
\pgfpathrectangle{\pgfqpoint{0.100000in}{0.220728in}}{\pgfqpoint{3.696000in}{3.696000in}}%
\pgfusepath{clip}%
\pgfsetbuttcap%
\pgfsetroundjoin%
\definecolor{currentfill}{rgb}{0.121569,0.466667,0.705882}%
\pgfsetfillcolor{currentfill}%
\pgfsetfillopacity{0.773752}%
\pgfsetlinewidth{1.003750pt}%
\definecolor{currentstroke}{rgb}{0.121569,0.466667,0.705882}%
\pgfsetstrokecolor{currentstroke}%
\pgfsetstrokeopacity{0.773752}%
\pgfsetdash{}{0pt}%
\pgfpathmoveto{\pgfqpoint{3.086956in}{2.348887in}}%
\pgfpathcurveto{\pgfqpoint{3.095192in}{2.348887in}}{\pgfqpoint{3.103093in}{2.352159in}}{\pgfqpoint{3.108916in}{2.357983in}}%
\pgfpathcurveto{\pgfqpoint{3.114740in}{2.363807in}}{\pgfqpoint{3.118013in}{2.371707in}}{\pgfqpoint{3.118013in}{2.379943in}}%
\pgfpathcurveto{\pgfqpoint{3.118013in}{2.388179in}}{\pgfqpoint{3.114740in}{2.396079in}}{\pgfqpoint{3.108916in}{2.401903in}}%
\pgfpathcurveto{\pgfqpoint{3.103093in}{2.407727in}}{\pgfqpoint{3.095192in}{2.411000in}}{\pgfqpoint{3.086956in}{2.411000in}}%
\pgfpathcurveto{\pgfqpoint{3.078720in}{2.411000in}}{\pgfqpoint{3.070820in}{2.407727in}}{\pgfqpoint{3.064996in}{2.401903in}}%
\pgfpathcurveto{\pgfqpoint{3.059172in}{2.396079in}}{\pgfqpoint{3.055900in}{2.388179in}}{\pgfqpoint{3.055900in}{2.379943in}}%
\pgfpathcurveto{\pgfqpoint{3.055900in}{2.371707in}}{\pgfqpoint{3.059172in}{2.363807in}}{\pgfqpoint{3.064996in}{2.357983in}}%
\pgfpathcurveto{\pgfqpoint{3.070820in}{2.352159in}}{\pgfqpoint{3.078720in}{2.348887in}}{\pgfqpoint{3.086956in}{2.348887in}}%
\pgfpathclose%
\pgfusepath{stroke,fill}%
\end{pgfscope}%
\begin{pgfscope}%
\pgfpathrectangle{\pgfqpoint{0.100000in}{0.220728in}}{\pgfqpoint{3.696000in}{3.696000in}}%
\pgfusepath{clip}%
\pgfsetbuttcap%
\pgfsetroundjoin%
\definecolor{currentfill}{rgb}{0.121569,0.466667,0.705882}%
\pgfsetfillcolor{currentfill}%
\pgfsetfillopacity{0.773960}%
\pgfsetlinewidth{1.003750pt}%
\definecolor{currentstroke}{rgb}{0.121569,0.466667,0.705882}%
\pgfsetstrokecolor{currentstroke}%
\pgfsetstrokeopacity{0.773960}%
\pgfsetdash{}{0pt}%
\pgfpathmoveto{\pgfqpoint{1.203164in}{1.181739in}}%
\pgfpathcurveto{\pgfqpoint{1.211400in}{1.181739in}}{\pgfqpoint{1.219300in}{1.185011in}}{\pgfqpoint{1.225124in}{1.190835in}}%
\pgfpathcurveto{\pgfqpoint{1.230948in}{1.196659in}}{\pgfqpoint{1.234221in}{1.204559in}}{\pgfqpoint{1.234221in}{1.212795in}}%
\pgfpathcurveto{\pgfqpoint{1.234221in}{1.221031in}}{\pgfqpoint{1.230948in}{1.228931in}}{\pgfqpoint{1.225124in}{1.234755in}}%
\pgfpathcurveto{\pgfqpoint{1.219300in}{1.240579in}}{\pgfqpoint{1.211400in}{1.243852in}}{\pgfqpoint{1.203164in}{1.243852in}}%
\pgfpathcurveto{\pgfqpoint{1.194928in}{1.243852in}}{\pgfqpoint{1.187028in}{1.240579in}}{\pgfqpoint{1.181204in}{1.234755in}}%
\pgfpathcurveto{\pgfqpoint{1.175380in}{1.228931in}}{\pgfqpoint{1.172108in}{1.221031in}}{\pgfqpoint{1.172108in}{1.212795in}}%
\pgfpathcurveto{\pgfqpoint{1.172108in}{1.204559in}}{\pgfqpoint{1.175380in}{1.196659in}}{\pgfqpoint{1.181204in}{1.190835in}}%
\pgfpathcurveto{\pgfqpoint{1.187028in}{1.185011in}}{\pgfqpoint{1.194928in}{1.181739in}}{\pgfqpoint{1.203164in}{1.181739in}}%
\pgfpathclose%
\pgfusepath{stroke,fill}%
\end{pgfscope}%
\begin{pgfscope}%
\pgfpathrectangle{\pgfqpoint{0.100000in}{0.220728in}}{\pgfqpoint{3.696000in}{3.696000in}}%
\pgfusepath{clip}%
\pgfsetbuttcap%
\pgfsetroundjoin%
\definecolor{currentfill}{rgb}{0.121569,0.466667,0.705882}%
\pgfsetfillcolor{currentfill}%
\pgfsetfillopacity{0.774237}%
\pgfsetlinewidth{1.003750pt}%
\definecolor{currentstroke}{rgb}{0.121569,0.466667,0.705882}%
\pgfsetstrokecolor{currentstroke}%
\pgfsetstrokeopacity{0.774237}%
\pgfsetdash{}{0pt}%
\pgfpathmoveto{\pgfqpoint{3.085463in}{2.346069in}}%
\pgfpathcurveto{\pgfqpoint{3.093699in}{2.346069in}}{\pgfqpoint{3.101599in}{2.349341in}}{\pgfqpoint{3.107423in}{2.355165in}}%
\pgfpathcurveto{\pgfqpoint{3.113247in}{2.360989in}}{\pgfqpoint{3.116519in}{2.368889in}}{\pgfqpoint{3.116519in}{2.377125in}}%
\pgfpathcurveto{\pgfqpoint{3.116519in}{2.385361in}}{\pgfqpoint{3.113247in}{2.393261in}}{\pgfqpoint{3.107423in}{2.399085in}}%
\pgfpathcurveto{\pgfqpoint{3.101599in}{2.404909in}}{\pgfqpoint{3.093699in}{2.408182in}}{\pgfqpoint{3.085463in}{2.408182in}}%
\pgfpathcurveto{\pgfqpoint{3.077226in}{2.408182in}}{\pgfqpoint{3.069326in}{2.404909in}}{\pgfqpoint{3.063502in}{2.399085in}}%
\pgfpathcurveto{\pgfqpoint{3.057678in}{2.393261in}}{\pgfqpoint{3.054406in}{2.385361in}}{\pgfqpoint{3.054406in}{2.377125in}}%
\pgfpathcurveto{\pgfqpoint{3.054406in}{2.368889in}}{\pgfqpoint{3.057678in}{2.360989in}}{\pgfqpoint{3.063502in}{2.355165in}}%
\pgfpathcurveto{\pgfqpoint{3.069326in}{2.349341in}}{\pgfqpoint{3.077226in}{2.346069in}}{\pgfqpoint{3.085463in}{2.346069in}}%
\pgfpathclose%
\pgfusepath{stroke,fill}%
\end{pgfscope}%
\begin{pgfscope}%
\pgfpathrectangle{\pgfqpoint{0.100000in}{0.220728in}}{\pgfqpoint{3.696000in}{3.696000in}}%
\pgfusepath{clip}%
\pgfsetbuttcap%
\pgfsetroundjoin%
\definecolor{currentfill}{rgb}{0.121569,0.466667,0.705882}%
\pgfsetfillcolor{currentfill}%
\pgfsetfillopacity{0.774507}%
\pgfsetlinewidth{1.003750pt}%
\definecolor{currentstroke}{rgb}{0.121569,0.466667,0.705882}%
\pgfsetstrokecolor{currentstroke}%
\pgfsetstrokeopacity{0.774507}%
\pgfsetdash{}{0pt}%
\pgfpathmoveto{\pgfqpoint{3.084640in}{2.344534in}}%
\pgfpathcurveto{\pgfqpoint{3.092876in}{2.344534in}}{\pgfqpoint{3.100777in}{2.347807in}}{\pgfqpoint{3.106600in}{2.353630in}}%
\pgfpathcurveto{\pgfqpoint{3.112424in}{2.359454in}}{\pgfqpoint{3.115697in}{2.367354in}}{\pgfqpoint{3.115697in}{2.375591in}}%
\pgfpathcurveto{\pgfqpoint{3.115697in}{2.383827in}}{\pgfqpoint{3.112424in}{2.391727in}}{\pgfqpoint{3.106600in}{2.397551in}}%
\pgfpathcurveto{\pgfqpoint{3.100777in}{2.403375in}}{\pgfqpoint{3.092876in}{2.406647in}}{\pgfqpoint{3.084640in}{2.406647in}}%
\pgfpathcurveto{\pgfqpoint{3.076404in}{2.406647in}}{\pgfqpoint{3.068504in}{2.403375in}}{\pgfqpoint{3.062680in}{2.397551in}}%
\pgfpathcurveto{\pgfqpoint{3.056856in}{2.391727in}}{\pgfqpoint{3.053584in}{2.383827in}}{\pgfqpoint{3.053584in}{2.375591in}}%
\pgfpathcurveto{\pgfqpoint{3.053584in}{2.367354in}}{\pgfqpoint{3.056856in}{2.359454in}}{\pgfqpoint{3.062680in}{2.353630in}}%
\pgfpathcurveto{\pgfqpoint{3.068504in}{2.347807in}}{\pgfqpoint{3.076404in}{2.344534in}}{\pgfqpoint{3.084640in}{2.344534in}}%
\pgfpathclose%
\pgfusepath{stroke,fill}%
\end{pgfscope}%
\begin{pgfscope}%
\pgfpathrectangle{\pgfqpoint{0.100000in}{0.220728in}}{\pgfqpoint{3.696000in}{3.696000in}}%
\pgfusepath{clip}%
\pgfsetbuttcap%
\pgfsetroundjoin%
\definecolor{currentfill}{rgb}{0.121569,0.466667,0.705882}%
\pgfsetfillcolor{currentfill}%
\pgfsetfillopacity{0.774680}%
\pgfsetlinewidth{1.003750pt}%
\definecolor{currentstroke}{rgb}{0.121569,0.466667,0.705882}%
\pgfsetstrokecolor{currentstroke}%
\pgfsetstrokeopacity{0.774680}%
\pgfsetdash{}{0pt}%
\pgfpathmoveto{\pgfqpoint{3.084303in}{2.343667in}}%
\pgfpathcurveto{\pgfqpoint{3.092539in}{2.343667in}}{\pgfqpoint{3.100439in}{2.346939in}}{\pgfqpoint{3.106263in}{2.352763in}}%
\pgfpathcurveto{\pgfqpoint{3.112087in}{2.358587in}}{\pgfqpoint{3.115359in}{2.366487in}}{\pgfqpoint{3.115359in}{2.374723in}}%
\pgfpathcurveto{\pgfqpoint{3.115359in}{2.382960in}}{\pgfqpoint{3.112087in}{2.390860in}}{\pgfqpoint{3.106263in}{2.396684in}}%
\pgfpathcurveto{\pgfqpoint{3.100439in}{2.402508in}}{\pgfqpoint{3.092539in}{2.405780in}}{\pgfqpoint{3.084303in}{2.405780in}}%
\pgfpathcurveto{\pgfqpoint{3.076067in}{2.405780in}}{\pgfqpoint{3.068167in}{2.402508in}}{\pgfqpoint{3.062343in}{2.396684in}}%
\pgfpathcurveto{\pgfqpoint{3.056519in}{2.390860in}}{\pgfqpoint{3.053246in}{2.382960in}}{\pgfqpoint{3.053246in}{2.374723in}}%
\pgfpathcurveto{\pgfqpoint{3.053246in}{2.366487in}}{\pgfqpoint{3.056519in}{2.358587in}}{\pgfqpoint{3.062343in}{2.352763in}}%
\pgfpathcurveto{\pgfqpoint{3.068167in}{2.346939in}}{\pgfqpoint{3.076067in}{2.343667in}}{\pgfqpoint{3.084303in}{2.343667in}}%
\pgfpathclose%
\pgfusepath{stroke,fill}%
\end{pgfscope}%
\begin{pgfscope}%
\pgfpathrectangle{\pgfqpoint{0.100000in}{0.220728in}}{\pgfqpoint{3.696000in}{3.696000in}}%
\pgfusepath{clip}%
\pgfsetbuttcap%
\pgfsetroundjoin%
\definecolor{currentfill}{rgb}{0.121569,0.466667,0.705882}%
\pgfsetfillcolor{currentfill}%
\pgfsetfillopacity{0.774989}%
\pgfsetlinewidth{1.003750pt}%
\definecolor{currentstroke}{rgb}{0.121569,0.466667,0.705882}%
\pgfsetstrokecolor{currentstroke}%
\pgfsetstrokeopacity{0.774989}%
\pgfsetdash{}{0pt}%
\pgfpathmoveto{\pgfqpoint{3.083153in}{2.341675in}}%
\pgfpathcurveto{\pgfqpoint{3.091390in}{2.341675in}}{\pgfqpoint{3.099290in}{2.344948in}}{\pgfqpoint{3.105113in}{2.350772in}}%
\pgfpathcurveto{\pgfqpoint{3.110937in}{2.356596in}}{\pgfqpoint{3.114210in}{2.364496in}}{\pgfqpoint{3.114210in}{2.372732in}}%
\pgfpathcurveto{\pgfqpoint{3.114210in}{2.380968in}}{\pgfqpoint{3.110937in}{2.388868in}}{\pgfqpoint{3.105113in}{2.394692in}}%
\pgfpathcurveto{\pgfqpoint{3.099290in}{2.400516in}}{\pgfqpoint{3.091390in}{2.403788in}}{\pgfqpoint{3.083153in}{2.403788in}}%
\pgfpathcurveto{\pgfqpoint{3.074917in}{2.403788in}}{\pgfqpoint{3.067017in}{2.400516in}}{\pgfqpoint{3.061193in}{2.394692in}}%
\pgfpathcurveto{\pgfqpoint{3.055369in}{2.388868in}}{\pgfqpoint{3.052097in}{2.380968in}}{\pgfqpoint{3.052097in}{2.372732in}}%
\pgfpathcurveto{\pgfqpoint{3.052097in}{2.364496in}}{\pgfqpoint{3.055369in}{2.356596in}}{\pgfqpoint{3.061193in}{2.350772in}}%
\pgfpathcurveto{\pgfqpoint{3.067017in}{2.344948in}}{\pgfqpoint{3.074917in}{2.341675in}}{\pgfqpoint{3.083153in}{2.341675in}}%
\pgfpathclose%
\pgfusepath{stroke,fill}%
\end{pgfscope}%
\begin{pgfscope}%
\pgfpathrectangle{\pgfqpoint{0.100000in}{0.220728in}}{\pgfqpoint{3.696000in}{3.696000in}}%
\pgfusepath{clip}%
\pgfsetbuttcap%
\pgfsetroundjoin%
\definecolor{currentfill}{rgb}{0.121569,0.466667,0.705882}%
\pgfsetfillcolor{currentfill}%
\pgfsetfillopacity{0.775208}%
\pgfsetlinewidth{1.003750pt}%
\definecolor{currentstroke}{rgb}{0.121569,0.466667,0.705882}%
\pgfsetstrokecolor{currentstroke}%
\pgfsetstrokeopacity{0.775208}%
\pgfsetdash{}{0pt}%
\pgfpathmoveto{\pgfqpoint{3.082662in}{2.340605in}}%
\pgfpathcurveto{\pgfqpoint{3.090898in}{2.340605in}}{\pgfqpoint{3.098798in}{2.343877in}}{\pgfqpoint{3.104622in}{2.349701in}}%
\pgfpathcurveto{\pgfqpoint{3.110446in}{2.355525in}}{\pgfqpoint{3.113718in}{2.363425in}}{\pgfqpoint{3.113718in}{2.371662in}}%
\pgfpathcurveto{\pgfqpoint{3.113718in}{2.379898in}}{\pgfqpoint{3.110446in}{2.387798in}}{\pgfqpoint{3.104622in}{2.393622in}}%
\pgfpathcurveto{\pgfqpoint{3.098798in}{2.399446in}}{\pgfqpoint{3.090898in}{2.402718in}}{\pgfqpoint{3.082662in}{2.402718in}}%
\pgfpathcurveto{\pgfqpoint{3.074425in}{2.402718in}}{\pgfqpoint{3.066525in}{2.399446in}}{\pgfqpoint{3.060701in}{2.393622in}}%
\pgfpathcurveto{\pgfqpoint{3.054877in}{2.387798in}}{\pgfqpoint{3.051605in}{2.379898in}}{\pgfqpoint{3.051605in}{2.371662in}}%
\pgfpathcurveto{\pgfqpoint{3.051605in}{2.363425in}}{\pgfqpoint{3.054877in}{2.355525in}}{\pgfqpoint{3.060701in}{2.349701in}}%
\pgfpathcurveto{\pgfqpoint{3.066525in}{2.343877in}}{\pgfqpoint{3.074425in}{2.340605in}}{\pgfqpoint{3.082662in}{2.340605in}}%
\pgfpathclose%
\pgfusepath{stroke,fill}%
\end{pgfscope}%
\begin{pgfscope}%
\pgfpathrectangle{\pgfqpoint{0.100000in}{0.220728in}}{\pgfqpoint{3.696000in}{3.696000in}}%
\pgfusepath{clip}%
\pgfsetbuttcap%
\pgfsetroundjoin%
\definecolor{currentfill}{rgb}{0.121569,0.466667,0.705882}%
\pgfsetfillcolor{currentfill}%
\pgfsetfillopacity{0.775322}%
\pgfsetlinewidth{1.003750pt}%
\definecolor{currentstroke}{rgb}{0.121569,0.466667,0.705882}%
\pgfsetstrokecolor{currentstroke}%
\pgfsetstrokeopacity{0.775322}%
\pgfsetdash{}{0pt}%
\pgfpathmoveto{\pgfqpoint{3.082430in}{2.339954in}}%
\pgfpathcurveto{\pgfqpoint{3.090666in}{2.339954in}}{\pgfqpoint{3.098566in}{2.343227in}}{\pgfqpoint{3.104390in}{2.349051in}}%
\pgfpathcurveto{\pgfqpoint{3.110214in}{2.354874in}}{\pgfqpoint{3.113487in}{2.362774in}}{\pgfqpoint{3.113487in}{2.371011in}}%
\pgfpathcurveto{\pgfqpoint{3.113487in}{2.379247in}}{\pgfqpoint{3.110214in}{2.387147in}}{\pgfqpoint{3.104390in}{2.392971in}}%
\pgfpathcurveto{\pgfqpoint{3.098566in}{2.398795in}}{\pgfqpoint{3.090666in}{2.402067in}}{\pgfqpoint{3.082430in}{2.402067in}}%
\pgfpathcurveto{\pgfqpoint{3.074194in}{2.402067in}}{\pgfqpoint{3.066294in}{2.398795in}}{\pgfqpoint{3.060470in}{2.392971in}}%
\pgfpathcurveto{\pgfqpoint{3.054646in}{2.387147in}}{\pgfqpoint{3.051374in}{2.379247in}}{\pgfqpoint{3.051374in}{2.371011in}}%
\pgfpathcurveto{\pgfqpoint{3.051374in}{2.362774in}}{\pgfqpoint{3.054646in}{2.354874in}}{\pgfqpoint{3.060470in}{2.349051in}}%
\pgfpathcurveto{\pgfqpoint{3.066294in}{2.343227in}}{\pgfqpoint{3.074194in}{2.339954in}}{\pgfqpoint{3.082430in}{2.339954in}}%
\pgfpathclose%
\pgfusepath{stroke,fill}%
\end{pgfscope}%
\begin{pgfscope}%
\pgfpathrectangle{\pgfqpoint{0.100000in}{0.220728in}}{\pgfqpoint{3.696000in}{3.696000in}}%
\pgfusepath{clip}%
\pgfsetbuttcap%
\pgfsetroundjoin%
\definecolor{currentfill}{rgb}{0.121569,0.466667,0.705882}%
\pgfsetfillcolor{currentfill}%
\pgfsetfillopacity{0.775573}%
\pgfsetlinewidth{1.003750pt}%
\definecolor{currentstroke}{rgb}{0.121569,0.466667,0.705882}%
\pgfsetstrokecolor{currentstroke}%
\pgfsetstrokeopacity{0.775573}%
\pgfsetdash{}{0pt}%
\pgfpathmoveto{\pgfqpoint{3.081543in}{2.338394in}}%
\pgfpathcurveto{\pgfqpoint{3.089779in}{2.338394in}}{\pgfqpoint{3.097679in}{2.341666in}}{\pgfqpoint{3.103503in}{2.347490in}}%
\pgfpathcurveto{\pgfqpoint{3.109327in}{2.353314in}}{\pgfqpoint{3.112599in}{2.361214in}}{\pgfqpoint{3.112599in}{2.369450in}}%
\pgfpathcurveto{\pgfqpoint{3.112599in}{2.377687in}}{\pgfqpoint{3.109327in}{2.385587in}}{\pgfqpoint{3.103503in}{2.391411in}}%
\pgfpathcurveto{\pgfqpoint{3.097679in}{2.397235in}}{\pgfqpoint{3.089779in}{2.400507in}}{\pgfqpoint{3.081543in}{2.400507in}}%
\pgfpathcurveto{\pgfqpoint{3.073306in}{2.400507in}}{\pgfqpoint{3.065406in}{2.397235in}}{\pgfqpoint{3.059582in}{2.391411in}}%
\pgfpathcurveto{\pgfqpoint{3.053758in}{2.385587in}}{\pgfqpoint{3.050486in}{2.377687in}}{\pgfqpoint{3.050486in}{2.369450in}}%
\pgfpathcurveto{\pgfqpoint{3.050486in}{2.361214in}}{\pgfqpoint{3.053758in}{2.353314in}}{\pgfqpoint{3.059582in}{2.347490in}}%
\pgfpathcurveto{\pgfqpoint{3.065406in}{2.341666in}}{\pgfqpoint{3.073306in}{2.338394in}}{\pgfqpoint{3.081543in}{2.338394in}}%
\pgfpathclose%
\pgfusepath{stroke,fill}%
\end{pgfscope}%
\begin{pgfscope}%
\pgfpathrectangle{\pgfqpoint{0.100000in}{0.220728in}}{\pgfqpoint{3.696000in}{3.696000in}}%
\pgfusepath{clip}%
\pgfsetbuttcap%
\pgfsetroundjoin%
\definecolor{currentfill}{rgb}{0.121569,0.466667,0.705882}%
\pgfsetfillcolor{currentfill}%
\pgfsetfillopacity{0.775749}%
\pgfsetlinewidth{1.003750pt}%
\definecolor{currentstroke}{rgb}{0.121569,0.466667,0.705882}%
\pgfsetstrokecolor{currentstroke}%
\pgfsetstrokeopacity{0.775749}%
\pgfsetdash{}{0pt}%
\pgfpathmoveto{\pgfqpoint{3.081187in}{2.337541in}}%
\pgfpathcurveto{\pgfqpoint{3.089423in}{2.337541in}}{\pgfqpoint{3.097323in}{2.340813in}}{\pgfqpoint{3.103147in}{2.346637in}}%
\pgfpathcurveto{\pgfqpoint{3.108971in}{2.352461in}}{\pgfqpoint{3.112243in}{2.360361in}}{\pgfqpoint{3.112243in}{2.368597in}}%
\pgfpathcurveto{\pgfqpoint{3.112243in}{2.376833in}}{\pgfqpoint{3.108971in}{2.384734in}}{\pgfqpoint{3.103147in}{2.390557in}}%
\pgfpathcurveto{\pgfqpoint{3.097323in}{2.396381in}}{\pgfqpoint{3.089423in}{2.399654in}}{\pgfqpoint{3.081187in}{2.399654in}}%
\pgfpathcurveto{\pgfqpoint{3.072950in}{2.399654in}}{\pgfqpoint{3.065050in}{2.396381in}}{\pgfqpoint{3.059226in}{2.390557in}}%
\pgfpathcurveto{\pgfqpoint{3.053402in}{2.384734in}}{\pgfqpoint{3.050130in}{2.376833in}}{\pgfqpoint{3.050130in}{2.368597in}}%
\pgfpathcurveto{\pgfqpoint{3.050130in}{2.360361in}}{\pgfqpoint{3.053402in}{2.352461in}}{\pgfqpoint{3.059226in}{2.346637in}}%
\pgfpathcurveto{\pgfqpoint{3.065050in}{2.340813in}}{\pgfqpoint{3.072950in}{2.337541in}}{\pgfqpoint{3.081187in}{2.337541in}}%
\pgfpathclose%
\pgfusepath{stroke,fill}%
\end{pgfscope}%
\begin{pgfscope}%
\pgfpathrectangle{\pgfqpoint{0.100000in}{0.220728in}}{\pgfqpoint{3.696000in}{3.696000in}}%
\pgfusepath{clip}%
\pgfsetbuttcap%
\pgfsetroundjoin%
\definecolor{currentfill}{rgb}{0.121569,0.466667,0.705882}%
\pgfsetfillcolor{currentfill}%
\pgfsetfillopacity{0.775850}%
\pgfsetlinewidth{1.003750pt}%
\definecolor{currentstroke}{rgb}{0.121569,0.466667,0.705882}%
\pgfsetstrokecolor{currentstroke}%
\pgfsetstrokeopacity{0.775850}%
\pgfsetdash{}{0pt}%
\pgfpathmoveto{\pgfqpoint{3.080982in}{2.337095in}}%
\pgfpathcurveto{\pgfqpoint{3.089218in}{2.337095in}}{\pgfqpoint{3.097118in}{2.340367in}}{\pgfqpoint{3.102942in}{2.346191in}}%
\pgfpathcurveto{\pgfqpoint{3.108766in}{2.352015in}}{\pgfqpoint{3.112038in}{2.359915in}}{\pgfqpoint{3.112038in}{2.368151in}}%
\pgfpathcurveto{\pgfqpoint{3.112038in}{2.376388in}}{\pgfqpoint{3.108766in}{2.384288in}}{\pgfqpoint{3.102942in}{2.390112in}}%
\pgfpathcurveto{\pgfqpoint{3.097118in}{2.395936in}}{\pgfqpoint{3.089218in}{2.399208in}}{\pgfqpoint{3.080982in}{2.399208in}}%
\pgfpathcurveto{\pgfqpoint{3.072745in}{2.399208in}}{\pgfqpoint{3.064845in}{2.395936in}}{\pgfqpoint{3.059022in}{2.390112in}}%
\pgfpathcurveto{\pgfqpoint{3.053198in}{2.384288in}}{\pgfqpoint{3.049925in}{2.376388in}}{\pgfqpoint{3.049925in}{2.368151in}}%
\pgfpathcurveto{\pgfqpoint{3.049925in}{2.359915in}}{\pgfqpoint{3.053198in}{2.352015in}}{\pgfqpoint{3.059022in}{2.346191in}}%
\pgfpathcurveto{\pgfqpoint{3.064845in}{2.340367in}}{\pgfqpoint{3.072745in}{2.337095in}}{\pgfqpoint{3.080982in}{2.337095in}}%
\pgfpathclose%
\pgfusepath{stroke,fill}%
\end{pgfscope}%
\begin{pgfscope}%
\pgfpathrectangle{\pgfqpoint{0.100000in}{0.220728in}}{\pgfqpoint{3.696000in}{3.696000in}}%
\pgfusepath{clip}%
\pgfsetbuttcap%
\pgfsetroundjoin%
\definecolor{currentfill}{rgb}{0.121569,0.466667,0.705882}%
\pgfsetfillcolor{currentfill}%
\pgfsetfillopacity{0.775897}%
\pgfsetlinewidth{1.003750pt}%
\definecolor{currentstroke}{rgb}{0.121569,0.466667,0.705882}%
\pgfsetstrokecolor{currentstroke}%
\pgfsetstrokeopacity{0.775897}%
\pgfsetdash{}{0pt}%
\pgfpathmoveto{\pgfqpoint{3.080834in}{2.336858in}}%
\pgfpathcurveto{\pgfqpoint{3.089071in}{2.336858in}}{\pgfqpoint{3.096971in}{2.340131in}}{\pgfqpoint{3.102795in}{2.345954in}}%
\pgfpathcurveto{\pgfqpoint{3.108618in}{2.351778in}}{\pgfqpoint{3.111891in}{2.359678in}}{\pgfqpoint{3.111891in}{2.367915in}}%
\pgfpathcurveto{\pgfqpoint{3.111891in}{2.376151in}}{\pgfqpoint{3.108618in}{2.384051in}}{\pgfqpoint{3.102795in}{2.389875in}}%
\pgfpathcurveto{\pgfqpoint{3.096971in}{2.395699in}}{\pgfqpoint{3.089071in}{2.398971in}}{\pgfqpoint{3.080834in}{2.398971in}}%
\pgfpathcurveto{\pgfqpoint{3.072598in}{2.398971in}}{\pgfqpoint{3.064698in}{2.395699in}}{\pgfqpoint{3.058874in}{2.389875in}}%
\pgfpathcurveto{\pgfqpoint{3.053050in}{2.384051in}}{\pgfqpoint{3.049778in}{2.376151in}}{\pgfqpoint{3.049778in}{2.367915in}}%
\pgfpathcurveto{\pgfqpoint{3.049778in}{2.359678in}}{\pgfqpoint{3.053050in}{2.351778in}}{\pgfqpoint{3.058874in}{2.345954in}}%
\pgfpathcurveto{\pgfqpoint{3.064698in}{2.340131in}}{\pgfqpoint{3.072598in}{2.336858in}}{\pgfqpoint{3.080834in}{2.336858in}}%
\pgfpathclose%
\pgfusepath{stroke,fill}%
\end{pgfscope}%
\begin{pgfscope}%
\pgfpathrectangle{\pgfqpoint{0.100000in}{0.220728in}}{\pgfqpoint{3.696000in}{3.696000in}}%
\pgfusepath{clip}%
\pgfsetbuttcap%
\pgfsetroundjoin%
\definecolor{currentfill}{rgb}{0.121569,0.466667,0.705882}%
\pgfsetfillcolor{currentfill}%
\pgfsetfillopacity{0.776259}%
\pgfsetlinewidth{1.003750pt}%
\definecolor{currentstroke}{rgb}{0.121569,0.466667,0.705882}%
\pgfsetstrokecolor{currentstroke}%
\pgfsetstrokeopacity{0.776259}%
\pgfsetdash{}{0pt}%
\pgfpathmoveto{\pgfqpoint{3.080148in}{2.334951in}}%
\pgfpathcurveto{\pgfqpoint{3.088385in}{2.334951in}}{\pgfqpoint{3.096285in}{2.338223in}}{\pgfqpoint{3.102109in}{2.344047in}}%
\pgfpathcurveto{\pgfqpoint{3.107933in}{2.349871in}}{\pgfqpoint{3.111205in}{2.357771in}}{\pgfqpoint{3.111205in}{2.366007in}}%
\pgfpathcurveto{\pgfqpoint{3.111205in}{2.374244in}}{\pgfqpoint{3.107933in}{2.382144in}}{\pgfqpoint{3.102109in}{2.387968in}}%
\pgfpathcurveto{\pgfqpoint{3.096285in}{2.393792in}}{\pgfqpoint{3.088385in}{2.397064in}}{\pgfqpoint{3.080148in}{2.397064in}}%
\pgfpathcurveto{\pgfqpoint{3.071912in}{2.397064in}}{\pgfqpoint{3.064012in}{2.393792in}}{\pgfqpoint{3.058188in}{2.387968in}}%
\pgfpathcurveto{\pgfqpoint{3.052364in}{2.382144in}}{\pgfqpoint{3.049092in}{2.374244in}}{\pgfqpoint{3.049092in}{2.366007in}}%
\pgfpathcurveto{\pgfqpoint{3.049092in}{2.357771in}}{\pgfqpoint{3.052364in}{2.349871in}}{\pgfqpoint{3.058188in}{2.344047in}}%
\pgfpathcurveto{\pgfqpoint{3.064012in}{2.338223in}}{\pgfqpoint{3.071912in}{2.334951in}}{\pgfqpoint{3.080148in}{2.334951in}}%
\pgfpathclose%
\pgfusepath{stroke,fill}%
\end{pgfscope}%
\begin{pgfscope}%
\pgfpathrectangle{\pgfqpoint{0.100000in}{0.220728in}}{\pgfqpoint{3.696000in}{3.696000in}}%
\pgfusepath{clip}%
\pgfsetbuttcap%
\pgfsetroundjoin%
\definecolor{currentfill}{rgb}{0.121569,0.466667,0.705882}%
\pgfsetfillcolor{currentfill}%
\pgfsetfillopacity{0.776744}%
\pgfsetlinewidth{1.003750pt}%
\definecolor{currentstroke}{rgb}{0.121569,0.466667,0.705882}%
\pgfsetstrokecolor{currentstroke}%
\pgfsetstrokeopacity{0.776744}%
\pgfsetdash{}{0pt}%
\pgfpathmoveto{\pgfqpoint{3.078865in}{2.332248in}}%
\pgfpathcurveto{\pgfqpoint{3.087101in}{2.332248in}}{\pgfqpoint{3.095001in}{2.335520in}}{\pgfqpoint{3.100825in}{2.341344in}}%
\pgfpathcurveto{\pgfqpoint{3.106649in}{2.347168in}}{\pgfqpoint{3.109921in}{2.355068in}}{\pgfqpoint{3.109921in}{2.363304in}}%
\pgfpathcurveto{\pgfqpoint{3.109921in}{2.371541in}}{\pgfqpoint{3.106649in}{2.379441in}}{\pgfqpoint{3.100825in}{2.385265in}}%
\pgfpathcurveto{\pgfqpoint{3.095001in}{2.391088in}}{\pgfqpoint{3.087101in}{2.394361in}}{\pgfqpoint{3.078865in}{2.394361in}}%
\pgfpathcurveto{\pgfqpoint{3.070629in}{2.394361in}}{\pgfqpoint{3.062729in}{2.391088in}}{\pgfqpoint{3.056905in}{2.385265in}}%
\pgfpathcurveto{\pgfqpoint{3.051081in}{2.379441in}}{\pgfqpoint{3.047808in}{2.371541in}}{\pgfqpoint{3.047808in}{2.363304in}}%
\pgfpathcurveto{\pgfqpoint{3.047808in}{2.355068in}}{\pgfqpoint{3.051081in}{2.347168in}}{\pgfqpoint{3.056905in}{2.341344in}}%
\pgfpathcurveto{\pgfqpoint{3.062729in}{2.335520in}}{\pgfqpoint{3.070629in}{2.332248in}}{\pgfqpoint{3.078865in}{2.332248in}}%
\pgfpathclose%
\pgfusepath{stroke,fill}%
\end{pgfscope}%
\begin{pgfscope}%
\pgfpathrectangle{\pgfqpoint{0.100000in}{0.220728in}}{\pgfqpoint{3.696000in}{3.696000in}}%
\pgfusepath{clip}%
\pgfsetbuttcap%
\pgfsetroundjoin%
\definecolor{currentfill}{rgb}{0.121569,0.466667,0.705882}%
\pgfsetfillcolor{currentfill}%
\pgfsetfillopacity{0.777412}%
\pgfsetlinewidth{1.003750pt}%
\definecolor{currentstroke}{rgb}{0.121569,0.466667,0.705882}%
\pgfsetstrokecolor{currentstroke}%
\pgfsetstrokeopacity{0.777412}%
\pgfsetdash{}{0pt}%
\pgfpathmoveto{\pgfqpoint{3.077020in}{2.329550in}}%
\pgfpathcurveto{\pgfqpoint{3.085256in}{2.329550in}}{\pgfqpoint{3.093156in}{2.332823in}}{\pgfqpoint{3.098980in}{2.338646in}}%
\pgfpathcurveto{\pgfqpoint{3.104804in}{2.344470in}}{\pgfqpoint{3.108076in}{2.352370in}}{\pgfqpoint{3.108076in}{2.360607in}}%
\pgfpathcurveto{\pgfqpoint{3.108076in}{2.368843in}}{\pgfqpoint{3.104804in}{2.376743in}}{\pgfqpoint{3.098980in}{2.382567in}}%
\pgfpathcurveto{\pgfqpoint{3.093156in}{2.388391in}}{\pgfqpoint{3.085256in}{2.391663in}}{\pgfqpoint{3.077020in}{2.391663in}}%
\pgfpathcurveto{\pgfqpoint{3.068783in}{2.391663in}}{\pgfqpoint{3.060883in}{2.388391in}}{\pgfqpoint{3.055059in}{2.382567in}}%
\pgfpathcurveto{\pgfqpoint{3.049235in}{2.376743in}}{\pgfqpoint{3.045963in}{2.368843in}}{\pgfqpoint{3.045963in}{2.360607in}}%
\pgfpathcurveto{\pgfqpoint{3.045963in}{2.352370in}}{\pgfqpoint{3.049235in}{2.344470in}}{\pgfqpoint{3.055059in}{2.338646in}}%
\pgfpathcurveto{\pgfqpoint{3.060883in}{2.332823in}}{\pgfqpoint{3.068783in}{2.329550in}}{\pgfqpoint{3.077020in}{2.329550in}}%
\pgfpathclose%
\pgfusepath{stroke,fill}%
\end{pgfscope}%
\begin{pgfscope}%
\pgfpathrectangle{\pgfqpoint{0.100000in}{0.220728in}}{\pgfqpoint{3.696000in}{3.696000in}}%
\pgfusepath{clip}%
\pgfsetbuttcap%
\pgfsetroundjoin%
\definecolor{currentfill}{rgb}{0.121569,0.466667,0.705882}%
\pgfsetfillcolor{currentfill}%
\pgfsetfillopacity{0.777533}%
\pgfsetlinewidth{1.003750pt}%
\definecolor{currentstroke}{rgb}{0.121569,0.466667,0.705882}%
\pgfsetstrokecolor{currentstroke}%
\pgfsetstrokeopacity{0.777533}%
\pgfsetdash{}{0pt}%
\pgfpathmoveto{\pgfqpoint{1.220763in}{1.177313in}}%
\pgfpathcurveto{\pgfqpoint{1.228999in}{1.177313in}}{\pgfqpoint{1.236899in}{1.180586in}}{\pgfqpoint{1.242723in}{1.186410in}}%
\pgfpathcurveto{\pgfqpoint{1.248547in}{1.192234in}}{\pgfqpoint{1.251820in}{1.200134in}}{\pgfqpoint{1.251820in}{1.208370in}}%
\pgfpathcurveto{\pgfqpoint{1.251820in}{1.216606in}}{\pgfqpoint{1.248547in}{1.224506in}}{\pgfqpoint{1.242723in}{1.230330in}}%
\pgfpathcurveto{\pgfqpoint{1.236899in}{1.236154in}}{\pgfqpoint{1.228999in}{1.239426in}}{\pgfqpoint{1.220763in}{1.239426in}}%
\pgfpathcurveto{\pgfqpoint{1.212527in}{1.239426in}}{\pgfqpoint{1.204627in}{1.236154in}}{\pgfqpoint{1.198803in}{1.230330in}}%
\pgfpathcurveto{\pgfqpoint{1.192979in}{1.224506in}}{\pgfqpoint{1.189707in}{1.216606in}}{\pgfqpoint{1.189707in}{1.208370in}}%
\pgfpathcurveto{\pgfqpoint{1.189707in}{1.200134in}}{\pgfqpoint{1.192979in}{1.192234in}}{\pgfqpoint{1.198803in}{1.186410in}}%
\pgfpathcurveto{\pgfqpoint{1.204627in}{1.180586in}}{\pgfqpoint{1.212527in}{1.177313in}}{\pgfqpoint{1.220763in}{1.177313in}}%
\pgfpathclose%
\pgfusepath{stroke,fill}%
\end{pgfscope}%
\begin{pgfscope}%
\pgfpathrectangle{\pgfqpoint{0.100000in}{0.220728in}}{\pgfqpoint{3.696000in}{3.696000in}}%
\pgfusepath{clip}%
\pgfsetbuttcap%
\pgfsetroundjoin%
\definecolor{currentfill}{rgb}{0.121569,0.466667,0.705882}%
\pgfsetfillcolor{currentfill}%
\pgfsetfillopacity{0.778257}%
\pgfsetlinewidth{1.003750pt}%
\definecolor{currentstroke}{rgb}{0.121569,0.466667,0.705882}%
\pgfsetstrokecolor{currentstroke}%
\pgfsetstrokeopacity{0.778257}%
\pgfsetdash{}{0pt}%
\pgfpathmoveto{\pgfqpoint{3.075530in}{2.323517in}}%
\pgfpathcurveto{\pgfqpoint{3.083767in}{2.323517in}}{\pgfqpoint{3.091667in}{2.326790in}}{\pgfqpoint{3.097491in}{2.332614in}}%
\pgfpathcurveto{\pgfqpoint{3.103315in}{2.338438in}}{\pgfqpoint{3.106587in}{2.346338in}}{\pgfqpoint{3.106587in}{2.354574in}}%
\pgfpathcurveto{\pgfqpoint{3.106587in}{2.362810in}}{\pgfqpoint{3.103315in}{2.370710in}}{\pgfqpoint{3.097491in}{2.376534in}}%
\pgfpathcurveto{\pgfqpoint{3.091667in}{2.382358in}}{\pgfqpoint{3.083767in}{2.385630in}}{\pgfqpoint{3.075530in}{2.385630in}}%
\pgfpathcurveto{\pgfqpoint{3.067294in}{2.385630in}}{\pgfqpoint{3.059394in}{2.382358in}}{\pgfqpoint{3.053570in}{2.376534in}}%
\pgfpathcurveto{\pgfqpoint{3.047746in}{2.370710in}}{\pgfqpoint{3.044474in}{2.362810in}}{\pgfqpoint{3.044474in}{2.354574in}}%
\pgfpathcurveto{\pgfqpoint{3.044474in}{2.346338in}}{\pgfqpoint{3.047746in}{2.338438in}}{\pgfqpoint{3.053570in}{2.332614in}}%
\pgfpathcurveto{\pgfqpoint{3.059394in}{2.326790in}}{\pgfqpoint{3.067294in}{2.323517in}}{\pgfqpoint{3.075530in}{2.323517in}}%
\pgfpathclose%
\pgfusepath{stroke,fill}%
\end{pgfscope}%
\begin{pgfscope}%
\pgfpathrectangle{\pgfqpoint{0.100000in}{0.220728in}}{\pgfqpoint{3.696000in}{3.696000in}}%
\pgfusepath{clip}%
\pgfsetbuttcap%
\pgfsetroundjoin%
\definecolor{currentfill}{rgb}{0.121569,0.466667,0.705882}%
\pgfsetfillcolor{currentfill}%
\pgfsetfillopacity{0.779141}%
\pgfsetlinewidth{1.003750pt}%
\definecolor{currentstroke}{rgb}{0.121569,0.466667,0.705882}%
\pgfsetstrokecolor{currentstroke}%
\pgfsetstrokeopacity{0.779141}%
\pgfsetdash{}{0pt}%
\pgfpathmoveto{\pgfqpoint{1.234312in}{1.167914in}}%
\pgfpathcurveto{\pgfqpoint{1.242548in}{1.167914in}}{\pgfqpoint{1.250448in}{1.171186in}}{\pgfqpoint{1.256272in}{1.177010in}}%
\pgfpathcurveto{\pgfqpoint{1.262096in}{1.182834in}}{\pgfqpoint{1.265368in}{1.190734in}}{\pgfqpoint{1.265368in}{1.198971in}}%
\pgfpathcurveto{\pgfqpoint{1.265368in}{1.207207in}}{\pgfqpoint{1.262096in}{1.215107in}}{\pgfqpoint{1.256272in}{1.220931in}}%
\pgfpathcurveto{\pgfqpoint{1.250448in}{1.226755in}}{\pgfqpoint{1.242548in}{1.230027in}}{\pgfqpoint{1.234312in}{1.230027in}}%
\pgfpathcurveto{\pgfqpoint{1.226075in}{1.230027in}}{\pgfqpoint{1.218175in}{1.226755in}}{\pgfqpoint{1.212351in}{1.220931in}}%
\pgfpathcurveto{\pgfqpoint{1.206528in}{1.215107in}}{\pgfqpoint{1.203255in}{1.207207in}}{\pgfqpoint{1.203255in}{1.198971in}}%
\pgfpathcurveto{\pgfqpoint{1.203255in}{1.190734in}}{\pgfqpoint{1.206528in}{1.182834in}}{\pgfqpoint{1.212351in}{1.177010in}}%
\pgfpathcurveto{\pgfqpoint{1.218175in}{1.171186in}}{\pgfqpoint{1.226075in}{1.167914in}}{\pgfqpoint{1.234312in}{1.167914in}}%
\pgfpathclose%
\pgfusepath{stroke,fill}%
\end{pgfscope}%
\begin{pgfscope}%
\pgfpathrectangle{\pgfqpoint{0.100000in}{0.220728in}}{\pgfqpoint{3.696000in}{3.696000in}}%
\pgfusepath{clip}%
\pgfsetbuttcap%
\pgfsetroundjoin%
\definecolor{currentfill}{rgb}{0.121569,0.466667,0.705882}%
\pgfsetfillcolor{currentfill}%
\pgfsetfillopacity{0.779378}%
\pgfsetlinewidth{1.003750pt}%
\definecolor{currentstroke}{rgb}{0.121569,0.466667,0.705882}%
\pgfsetstrokecolor{currentstroke}%
\pgfsetstrokeopacity{0.779378}%
\pgfsetdash{}{0pt}%
\pgfpathmoveto{\pgfqpoint{3.072763in}{2.318252in}}%
\pgfpathcurveto{\pgfqpoint{3.080999in}{2.318252in}}{\pgfqpoint{3.088899in}{2.321525in}}{\pgfqpoint{3.094723in}{2.327349in}}%
\pgfpathcurveto{\pgfqpoint{3.100547in}{2.333172in}}{\pgfqpoint{3.103819in}{2.341073in}}{\pgfqpoint{3.103819in}{2.349309in}}%
\pgfpathcurveto{\pgfqpoint{3.103819in}{2.357545in}}{\pgfqpoint{3.100547in}{2.365445in}}{\pgfqpoint{3.094723in}{2.371269in}}%
\pgfpathcurveto{\pgfqpoint{3.088899in}{2.377093in}}{\pgfqpoint{3.080999in}{2.380365in}}{\pgfqpoint{3.072763in}{2.380365in}}%
\pgfpathcurveto{\pgfqpoint{3.064526in}{2.380365in}}{\pgfqpoint{3.056626in}{2.377093in}}{\pgfqpoint{3.050802in}{2.371269in}}%
\pgfpathcurveto{\pgfqpoint{3.044978in}{2.365445in}}{\pgfqpoint{3.041706in}{2.357545in}}{\pgfqpoint{3.041706in}{2.349309in}}%
\pgfpathcurveto{\pgfqpoint{3.041706in}{2.341073in}}{\pgfqpoint{3.044978in}{2.333172in}}{\pgfqpoint{3.050802in}{2.327349in}}%
\pgfpathcurveto{\pgfqpoint{3.056626in}{2.321525in}}{\pgfqpoint{3.064526in}{2.318252in}}{\pgfqpoint{3.072763in}{2.318252in}}%
\pgfpathclose%
\pgfusepath{stroke,fill}%
\end{pgfscope}%
\begin{pgfscope}%
\pgfpathrectangle{\pgfqpoint{0.100000in}{0.220728in}}{\pgfqpoint{3.696000in}{3.696000in}}%
\pgfusepath{clip}%
\pgfsetbuttcap%
\pgfsetroundjoin%
\definecolor{currentfill}{rgb}{0.121569,0.466667,0.705882}%
\pgfsetfillcolor{currentfill}%
\pgfsetfillopacity{0.779924}%
\pgfsetlinewidth{1.003750pt}%
\definecolor{currentstroke}{rgb}{0.121569,0.466667,0.705882}%
\pgfsetstrokecolor{currentstroke}%
\pgfsetstrokeopacity{0.779924}%
\pgfsetdash{}{0pt}%
\pgfpathmoveto{\pgfqpoint{3.070990in}{2.315396in}}%
\pgfpathcurveto{\pgfqpoint{3.079226in}{2.315396in}}{\pgfqpoint{3.087126in}{2.318668in}}{\pgfqpoint{3.092950in}{2.324492in}}%
\pgfpathcurveto{\pgfqpoint{3.098774in}{2.330316in}}{\pgfqpoint{3.102047in}{2.338216in}}{\pgfqpoint{3.102047in}{2.346452in}}%
\pgfpathcurveto{\pgfqpoint{3.102047in}{2.354689in}}{\pgfqpoint{3.098774in}{2.362589in}}{\pgfqpoint{3.092950in}{2.368413in}}%
\pgfpathcurveto{\pgfqpoint{3.087126in}{2.374237in}}{\pgfqpoint{3.079226in}{2.377509in}}{\pgfqpoint{3.070990in}{2.377509in}}%
\pgfpathcurveto{\pgfqpoint{3.062754in}{2.377509in}}{\pgfqpoint{3.054854in}{2.374237in}}{\pgfqpoint{3.049030in}{2.368413in}}%
\pgfpathcurveto{\pgfqpoint{3.043206in}{2.362589in}}{\pgfqpoint{3.039934in}{2.354689in}}{\pgfqpoint{3.039934in}{2.346452in}}%
\pgfpathcurveto{\pgfqpoint{3.039934in}{2.338216in}}{\pgfqpoint{3.043206in}{2.330316in}}{\pgfqpoint{3.049030in}{2.324492in}}%
\pgfpathcurveto{\pgfqpoint{3.054854in}{2.318668in}}{\pgfqpoint{3.062754in}{2.315396in}}{\pgfqpoint{3.070990in}{2.315396in}}%
\pgfpathclose%
\pgfusepath{stroke,fill}%
\end{pgfscope}%
\begin{pgfscope}%
\pgfpathrectangle{\pgfqpoint{0.100000in}{0.220728in}}{\pgfqpoint{3.696000in}{3.696000in}}%
\pgfusepath{clip}%
\pgfsetbuttcap%
\pgfsetroundjoin%
\definecolor{currentfill}{rgb}{0.121569,0.466667,0.705882}%
\pgfsetfillcolor{currentfill}%
\pgfsetfillopacity{0.780721}%
\pgfsetlinewidth{1.003750pt}%
\definecolor{currentstroke}{rgb}{0.121569,0.466667,0.705882}%
\pgfsetstrokecolor{currentstroke}%
\pgfsetstrokeopacity{0.780721}%
\pgfsetdash{}{0pt}%
\pgfpathmoveto{\pgfqpoint{3.069741in}{2.311112in}}%
\pgfpathcurveto{\pgfqpoint{3.077978in}{2.311112in}}{\pgfqpoint{3.085878in}{2.314385in}}{\pgfqpoint{3.091702in}{2.320209in}}%
\pgfpathcurveto{\pgfqpoint{3.097526in}{2.326032in}}{\pgfqpoint{3.100798in}{2.333932in}}{\pgfqpoint{3.100798in}{2.342169in}}%
\pgfpathcurveto{\pgfqpoint{3.100798in}{2.350405in}}{\pgfqpoint{3.097526in}{2.358305in}}{\pgfqpoint{3.091702in}{2.364129in}}%
\pgfpathcurveto{\pgfqpoint{3.085878in}{2.369953in}}{\pgfqpoint{3.077978in}{2.373225in}}{\pgfqpoint{3.069741in}{2.373225in}}%
\pgfpathcurveto{\pgfqpoint{3.061505in}{2.373225in}}{\pgfqpoint{3.053605in}{2.369953in}}{\pgfqpoint{3.047781in}{2.364129in}}%
\pgfpathcurveto{\pgfqpoint{3.041957in}{2.358305in}}{\pgfqpoint{3.038685in}{2.350405in}}{\pgfqpoint{3.038685in}{2.342169in}}%
\pgfpathcurveto{\pgfqpoint{3.038685in}{2.333932in}}{\pgfqpoint{3.041957in}{2.326032in}}{\pgfqpoint{3.047781in}{2.320209in}}%
\pgfpathcurveto{\pgfqpoint{3.053605in}{2.314385in}}{\pgfqpoint{3.061505in}{2.311112in}}{\pgfqpoint{3.069741in}{2.311112in}}%
\pgfpathclose%
\pgfusepath{stroke,fill}%
\end{pgfscope}%
\begin{pgfscope}%
\pgfpathrectangle{\pgfqpoint{0.100000in}{0.220728in}}{\pgfqpoint{3.696000in}{3.696000in}}%
\pgfusepath{clip}%
\pgfsetbuttcap%
\pgfsetroundjoin%
\definecolor{currentfill}{rgb}{0.121569,0.466667,0.705882}%
\pgfsetfillcolor{currentfill}%
\pgfsetfillopacity{0.781746}%
\pgfsetlinewidth{1.003750pt}%
\definecolor{currentstroke}{rgb}{0.121569,0.466667,0.705882}%
\pgfsetstrokecolor{currentstroke}%
\pgfsetstrokeopacity{0.781746}%
\pgfsetdash{}{0pt}%
\pgfpathmoveto{\pgfqpoint{3.066509in}{2.305165in}}%
\pgfpathcurveto{\pgfqpoint{3.074745in}{2.305165in}}{\pgfqpoint{3.082645in}{2.308437in}}{\pgfqpoint{3.088469in}{2.314261in}}%
\pgfpathcurveto{\pgfqpoint{3.094293in}{2.320085in}}{\pgfqpoint{3.097565in}{2.327985in}}{\pgfqpoint{3.097565in}{2.336222in}}%
\pgfpathcurveto{\pgfqpoint{3.097565in}{2.344458in}}{\pgfqpoint{3.094293in}{2.352358in}}{\pgfqpoint{3.088469in}{2.358182in}}%
\pgfpathcurveto{\pgfqpoint{3.082645in}{2.364006in}}{\pgfqpoint{3.074745in}{2.367278in}}{\pgfqpoint{3.066509in}{2.367278in}}%
\pgfpathcurveto{\pgfqpoint{3.058272in}{2.367278in}}{\pgfqpoint{3.050372in}{2.364006in}}{\pgfqpoint{3.044548in}{2.358182in}}%
\pgfpathcurveto{\pgfqpoint{3.038724in}{2.352358in}}{\pgfqpoint{3.035452in}{2.344458in}}{\pgfqpoint{3.035452in}{2.336222in}}%
\pgfpathcurveto{\pgfqpoint{3.035452in}{2.327985in}}{\pgfqpoint{3.038724in}{2.320085in}}{\pgfqpoint{3.044548in}{2.314261in}}%
\pgfpathcurveto{\pgfqpoint{3.050372in}{2.308437in}}{\pgfqpoint{3.058272in}{2.305165in}}{\pgfqpoint{3.066509in}{2.305165in}}%
\pgfpathclose%
\pgfusepath{stroke,fill}%
\end{pgfscope}%
\begin{pgfscope}%
\pgfpathrectangle{\pgfqpoint{0.100000in}{0.220728in}}{\pgfqpoint{3.696000in}{3.696000in}}%
\pgfusepath{clip}%
\pgfsetbuttcap%
\pgfsetroundjoin%
\definecolor{currentfill}{rgb}{0.121569,0.466667,0.705882}%
\pgfsetfillcolor{currentfill}%
\pgfsetfillopacity{0.782334}%
\pgfsetlinewidth{1.003750pt}%
\definecolor{currentstroke}{rgb}{0.121569,0.466667,0.705882}%
\pgfsetstrokecolor{currentstroke}%
\pgfsetstrokeopacity{0.782334}%
\pgfsetdash{}{0pt}%
\pgfpathmoveto{\pgfqpoint{3.064713in}{2.302022in}}%
\pgfpathcurveto{\pgfqpoint{3.072949in}{2.302022in}}{\pgfqpoint{3.080849in}{2.305294in}}{\pgfqpoint{3.086673in}{2.311118in}}%
\pgfpathcurveto{\pgfqpoint{3.092497in}{2.316942in}}{\pgfqpoint{3.095769in}{2.324842in}}{\pgfqpoint{3.095769in}{2.333079in}}%
\pgfpathcurveto{\pgfqpoint{3.095769in}{2.341315in}}{\pgfqpoint{3.092497in}{2.349215in}}{\pgfqpoint{3.086673in}{2.355039in}}%
\pgfpathcurveto{\pgfqpoint{3.080849in}{2.360863in}}{\pgfqpoint{3.072949in}{2.364135in}}{\pgfqpoint{3.064713in}{2.364135in}}%
\pgfpathcurveto{\pgfqpoint{3.056477in}{2.364135in}}{\pgfqpoint{3.048577in}{2.360863in}}{\pgfqpoint{3.042753in}{2.355039in}}%
\pgfpathcurveto{\pgfqpoint{3.036929in}{2.349215in}}{\pgfqpoint{3.033656in}{2.341315in}}{\pgfqpoint{3.033656in}{2.333079in}}%
\pgfpathcurveto{\pgfqpoint{3.033656in}{2.324842in}}{\pgfqpoint{3.036929in}{2.316942in}}{\pgfqpoint{3.042753in}{2.311118in}}%
\pgfpathcurveto{\pgfqpoint{3.048577in}{2.305294in}}{\pgfqpoint{3.056477in}{2.302022in}}{\pgfqpoint{3.064713in}{2.302022in}}%
\pgfpathclose%
\pgfusepath{stroke,fill}%
\end{pgfscope}%
\begin{pgfscope}%
\pgfpathrectangle{\pgfqpoint{0.100000in}{0.220728in}}{\pgfqpoint{3.696000in}{3.696000in}}%
\pgfusepath{clip}%
\pgfsetbuttcap%
\pgfsetroundjoin%
\definecolor{currentfill}{rgb}{0.121569,0.466667,0.705882}%
\pgfsetfillcolor{currentfill}%
\pgfsetfillopacity{0.782380}%
\pgfsetlinewidth{1.003750pt}%
\definecolor{currentstroke}{rgb}{0.121569,0.466667,0.705882}%
\pgfsetstrokecolor{currentstroke}%
\pgfsetstrokeopacity{0.782380}%
\pgfsetdash{}{0pt}%
\pgfpathmoveto{\pgfqpoint{1.259436in}{1.153875in}}%
\pgfpathcurveto{\pgfqpoint{1.267673in}{1.153875in}}{\pgfqpoint{1.275573in}{1.157148in}}{\pgfqpoint{1.281397in}{1.162972in}}%
\pgfpathcurveto{\pgfqpoint{1.287221in}{1.168796in}}{\pgfqpoint{1.290493in}{1.176696in}}{\pgfqpoint{1.290493in}{1.184932in}}%
\pgfpathcurveto{\pgfqpoint{1.290493in}{1.193168in}}{\pgfqpoint{1.287221in}{1.201068in}}{\pgfqpoint{1.281397in}{1.206892in}}%
\pgfpathcurveto{\pgfqpoint{1.275573in}{1.212716in}}{\pgfqpoint{1.267673in}{1.215988in}}{\pgfqpoint{1.259436in}{1.215988in}}%
\pgfpathcurveto{\pgfqpoint{1.251200in}{1.215988in}}{\pgfqpoint{1.243300in}{1.212716in}}{\pgfqpoint{1.237476in}{1.206892in}}%
\pgfpathcurveto{\pgfqpoint{1.231652in}{1.201068in}}{\pgfqpoint{1.228380in}{1.193168in}}{\pgfqpoint{1.228380in}{1.184932in}}%
\pgfpathcurveto{\pgfqpoint{1.228380in}{1.176696in}}{\pgfqpoint{1.231652in}{1.168796in}}{\pgfqpoint{1.237476in}{1.162972in}}%
\pgfpathcurveto{\pgfqpoint{1.243300in}{1.157148in}}{\pgfqpoint{1.251200in}{1.153875in}}{\pgfqpoint{1.259436in}{1.153875in}}%
\pgfpathclose%
\pgfusepath{stroke,fill}%
\end{pgfscope}%
\begin{pgfscope}%
\pgfpathrectangle{\pgfqpoint{0.100000in}{0.220728in}}{\pgfqpoint{3.696000in}{3.696000in}}%
\pgfusepath{clip}%
\pgfsetbuttcap%
\pgfsetroundjoin%
\definecolor{currentfill}{rgb}{0.121569,0.466667,0.705882}%
\pgfsetfillcolor{currentfill}%
\pgfsetfillopacity{0.782691}%
\pgfsetlinewidth{1.003750pt}%
\definecolor{currentstroke}{rgb}{0.121569,0.466667,0.705882}%
\pgfsetstrokecolor{currentstroke}%
\pgfsetstrokeopacity{0.782691}%
\pgfsetdash{}{0pt}%
\pgfpathmoveto{\pgfqpoint{3.064026in}{2.300105in}}%
\pgfpathcurveto{\pgfqpoint{3.072262in}{2.300105in}}{\pgfqpoint{3.080162in}{2.303377in}}{\pgfqpoint{3.085986in}{2.309201in}}%
\pgfpathcurveto{\pgfqpoint{3.091810in}{2.315025in}}{\pgfqpoint{3.095082in}{2.322925in}}{\pgfqpoint{3.095082in}{2.331161in}}%
\pgfpathcurveto{\pgfqpoint{3.095082in}{2.339398in}}{\pgfqpoint{3.091810in}{2.347298in}}{\pgfqpoint{3.085986in}{2.353122in}}%
\pgfpathcurveto{\pgfqpoint{3.080162in}{2.358946in}}{\pgfqpoint{3.072262in}{2.362218in}}{\pgfqpoint{3.064026in}{2.362218in}}%
\pgfpathcurveto{\pgfqpoint{3.055789in}{2.362218in}}{\pgfqpoint{3.047889in}{2.358946in}}{\pgfqpoint{3.042065in}{2.353122in}}%
\pgfpathcurveto{\pgfqpoint{3.036241in}{2.347298in}}{\pgfqpoint{3.032969in}{2.339398in}}{\pgfqpoint{3.032969in}{2.331161in}}%
\pgfpathcurveto{\pgfqpoint{3.032969in}{2.322925in}}{\pgfqpoint{3.036241in}{2.315025in}}{\pgfqpoint{3.042065in}{2.309201in}}%
\pgfpathcurveto{\pgfqpoint{3.047889in}{2.303377in}}{\pgfqpoint{3.055789in}{2.300105in}}{\pgfqpoint{3.064026in}{2.300105in}}%
\pgfpathclose%
\pgfusepath{stroke,fill}%
\end{pgfscope}%
\begin{pgfscope}%
\pgfpathrectangle{\pgfqpoint{0.100000in}{0.220728in}}{\pgfqpoint{3.696000in}{3.696000in}}%
\pgfusepath{clip}%
\pgfsetbuttcap%
\pgfsetroundjoin%
\definecolor{currentfill}{rgb}{0.121569,0.466667,0.705882}%
\pgfsetfillcolor{currentfill}%
\pgfsetfillopacity{0.783129}%
\pgfsetlinewidth{1.003750pt}%
\definecolor{currentstroke}{rgb}{0.121569,0.466667,0.705882}%
\pgfsetstrokecolor{currentstroke}%
\pgfsetstrokeopacity{0.783129}%
\pgfsetdash{}{0pt}%
\pgfpathmoveto{\pgfqpoint{3.062574in}{2.297534in}}%
\pgfpathcurveto{\pgfqpoint{3.070810in}{2.297534in}}{\pgfqpoint{3.078710in}{2.300806in}}{\pgfqpoint{3.084534in}{2.306630in}}%
\pgfpathcurveto{\pgfqpoint{3.090358in}{2.312454in}}{\pgfqpoint{3.093630in}{2.320354in}}{\pgfqpoint{3.093630in}{2.328590in}}%
\pgfpathcurveto{\pgfqpoint{3.093630in}{2.336826in}}{\pgfqpoint{3.090358in}{2.344726in}}{\pgfqpoint{3.084534in}{2.350550in}}%
\pgfpathcurveto{\pgfqpoint{3.078710in}{2.356374in}}{\pgfqpoint{3.070810in}{2.359647in}}{\pgfqpoint{3.062574in}{2.359647in}}%
\pgfpathcurveto{\pgfqpoint{3.054337in}{2.359647in}}{\pgfqpoint{3.046437in}{2.356374in}}{\pgfqpoint{3.040613in}{2.350550in}}%
\pgfpathcurveto{\pgfqpoint{3.034790in}{2.344726in}}{\pgfqpoint{3.031517in}{2.336826in}}{\pgfqpoint{3.031517in}{2.328590in}}%
\pgfpathcurveto{\pgfqpoint{3.031517in}{2.320354in}}{\pgfqpoint{3.034790in}{2.312454in}}{\pgfqpoint{3.040613in}{2.306630in}}%
\pgfpathcurveto{\pgfqpoint{3.046437in}{2.300806in}}{\pgfqpoint{3.054337in}{2.297534in}}{\pgfqpoint{3.062574in}{2.297534in}}%
\pgfpathclose%
\pgfusepath{stroke,fill}%
\end{pgfscope}%
\begin{pgfscope}%
\pgfpathrectangle{\pgfqpoint{0.100000in}{0.220728in}}{\pgfqpoint{3.696000in}{3.696000in}}%
\pgfusepath{clip}%
\pgfsetbuttcap%
\pgfsetroundjoin%
\definecolor{currentfill}{rgb}{0.121569,0.466667,0.705882}%
\pgfsetfillcolor{currentfill}%
\pgfsetfillopacity{0.783883}%
\pgfsetlinewidth{1.003750pt}%
\definecolor{currentstroke}{rgb}{0.121569,0.466667,0.705882}%
\pgfsetstrokecolor{currentstroke}%
\pgfsetstrokeopacity{0.783883}%
\pgfsetdash{}{0pt}%
\pgfpathmoveto{\pgfqpoint{3.060814in}{2.293712in}}%
\pgfpathcurveto{\pgfqpoint{3.069050in}{2.293712in}}{\pgfqpoint{3.076950in}{2.296984in}}{\pgfqpoint{3.082774in}{2.302808in}}%
\pgfpathcurveto{\pgfqpoint{3.088598in}{2.308632in}}{\pgfqpoint{3.091870in}{2.316532in}}{\pgfqpoint{3.091870in}{2.324768in}}%
\pgfpathcurveto{\pgfqpoint{3.091870in}{2.333004in}}{\pgfqpoint{3.088598in}{2.340904in}}{\pgfqpoint{3.082774in}{2.346728in}}%
\pgfpathcurveto{\pgfqpoint{3.076950in}{2.352552in}}{\pgfqpoint{3.069050in}{2.355825in}}{\pgfqpoint{3.060814in}{2.355825in}}%
\pgfpathcurveto{\pgfqpoint{3.052578in}{2.355825in}}{\pgfqpoint{3.044677in}{2.352552in}}{\pgfqpoint{3.038854in}{2.346728in}}%
\pgfpathcurveto{\pgfqpoint{3.033030in}{2.340904in}}{\pgfqpoint{3.029757in}{2.333004in}}{\pgfqpoint{3.029757in}{2.324768in}}%
\pgfpathcurveto{\pgfqpoint{3.029757in}{2.316532in}}{\pgfqpoint{3.033030in}{2.308632in}}{\pgfqpoint{3.038854in}{2.302808in}}%
\pgfpathcurveto{\pgfqpoint{3.044677in}{2.296984in}}{\pgfqpoint{3.052578in}{2.293712in}}{\pgfqpoint{3.060814in}{2.293712in}}%
\pgfpathclose%
\pgfusepath{stroke,fill}%
\end{pgfscope}%
\begin{pgfscope}%
\pgfpathrectangle{\pgfqpoint{0.100000in}{0.220728in}}{\pgfqpoint{3.696000in}{3.696000in}}%
\pgfusepath{clip}%
\pgfsetbuttcap%
\pgfsetroundjoin%
\definecolor{currentfill}{rgb}{0.121569,0.466667,0.705882}%
\pgfsetfillcolor{currentfill}%
\pgfsetfillopacity{0.784316}%
\pgfsetlinewidth{1.003750pt}%
\definecolor{currentstroke}{rgb}{0.121569,0.466667,0.705882}%
\pgfsetstrokecolor{currentstroke}%
\pgfsetstrokeopacity{0.784316}%
\pgfsetdash{}{0pt}%
\pgfpathmoveto{\pgfqpoint{3.059969in}{2.291567in}}%
\pgfpathcurveto{\pgfqpoint{3.068205in}{2.291567in}}{\pgfqpoint{3.076105in}{2.294839in}}{\pgfqpoint{3.081929in}{2.300663in}}%
\pgfpathcurveto{\pgfqpoint{3.087753in}{2.306487in}}{\pgfqpoint{3.091025in}{2.314387in}}{\pgfqpoint{3.091025in}{2.322623in}}%
\pgfpathcurveto{\pgfqpoint{3.091025in}{2.330860in}}{\pgfqpoint{3.087753in}{2.338760in}}{\pgfqpoint{3.081929in}{2.344584in}}%
\pgfpathcurveto{\pgfqpoint{3.076105in}{2.350408in}}{\pgfqpoint{3.068205in}{2.353680in}}{\pgfqpoint{3.059969in}{2.353680in}}%
\pgfpathcurveto{\pgfqpoint{3.051732in}{2.353680in}}{\pgfqpoint{3.043832in}{2.350408in}}{\pgfqpoint{3.038008in}{2.344584in}}%
\pgfpathcurveto{\pgfqpoint{3.032185in}{2.338760in}}{\pgfqpoint{3.028912in}{2.330860in}}{\pgfqpoint{3.028912in}{2.322623in}}%
\pgfpathcurveto{\pgfqpoint{3.028912in}{2.314387in}}{\pgfqpoint{3.032185in}{2.306487in}}{\pgfqpoint{3.038008in}{2.300663in}}%
\pgfpathcurveto{\pgfqpoint{3.043832in}{2.294839in}}{\pgfqpoint{3.051732in}{2.291567in}}{\pgfqpoint{3.059969in}{2.291567in}}%
\pgfpathclose%
\pgfusepath{stroke,fill}%
\end{pgfscope}%
\begin{pgfscope}%
\pgfpathrectangle{\pgfqpoint{0.100000in}{0.220728in}}{\pgfqpoint{3.696000in}{3.696000in}}%
\pgfusepath{clip}%
\pgfsetbuttcap%
\pgfsetroundjoin%
\definecolor{currentfill}{rgb}{0.121569,0.466667,0.705882}%
\pgfsetfillcolor{currentfill}%
\pgfsetfillopacity{0.784504}%
\pgfsetlinewidth{1.003750pt}%
\definecolor{currentstroke}{rgb}{0.121569,0.466667,0.705882}%
\pgfsetstrokecolor{currentstroke}%
\pgfsetstrokeopacity{0.784504}%
\pgfsetdash{}{0pt}%
\pgfpathmoveto{\pgfqpoint{3.059269in}{2.290462in}}%
\pgfpathcurveto{\pgfqpoint{3.067506in}{2.290462in}}{\pgfqpoint{3.075406in}{2.293735in}}{\pgfqpoint{3.081230in}{2.299559in}}%
\pgfpathcurveto{\pgfqpoint{3.087053in}{2.305383in}}{\pgfqpoint{3.090326in}{2.313283in}}{\pgfqpoint{3.090326in}{2.321519in}}%
\pgfpathcurveto{\pgfqpoint{3.090326in}{2.329755in}}{\pgfqpoint{3.087053in}{2.337655in}}{\pgfqpoint{3.081230in}{2.343479in}}%
\pgfpathcurveto{\pgfqpoint{3.075406in}{2.349303in}}{\pgfqpoint{3.067506in}{2.352575in}}{\pgfqpoint{3.059269in}{2.352575in}}%
\pgfpathcurveto{\pgfqpoint{3.051033in}{2.352575in}}{\pgfqpoint{3.043133in}{2.349303in}}{\pgfqpoint{3.037309in}{2.343479in}}%
\pgfpathcurveto{\pgfqpoint{3.031485in}{2.337655in}}{\pgfqpoint{3.028213in}{2.329755in}}{\pgfqpoint{3.028213in}{2.321519in}}%
\pgfpathcurveto{\pgfqpoint{3.028213in}{2.313283in}}{\pgfqpoint{3.031485in}{2.305383in}}{\pgfqpoint{3.037309in}{2.299559in}}%
\pgfpathcurveto{\pgfqpoint{3.043133in}{2.293735in}}{\pgfqpoint{3.051033in}{2.290462in}}{\pgfqpoint{3.059269in}{2.290462in}}%
\pgfpathclose%
\pgfusepath{stroke,fill}%
\end{pgfscope}%
\begin{pgfscope}%
\pgfpathrectangle{\pgfqpoint{0.100000in}{0.220728in}}{\pgfqpoint{3.696000in}{3.696000in}}%
\pgfusepath{clip}%
\pgfsetbuttcap%
\pgfsetroundjoin%
\definecolor{currentfill}{rgb}{0.121569,0.466667,0.705882}%
\pgfsetfillcolor{currentfill}%
\pgfsetfillopacity{0.785026}%
\pgfsetlinewidth{1.003750pt}%
\definecolor{currentstroke}{rgb}{0.121569,0.466667,0.705882}%
\pgfsetstrokecolor{currentstroke}%
\pgfsetstrokeopacity{0.785026}%
\pgfsetdash{}{0pt}%
\pgfpathmoveto{\pgfqpoint{3.057948in}{2.287216in}}%
\pgfpathcurveto{\pgfqpoint{3.066184in}{2.287216in}}{\pgfqpoint{3.074084in}{2.290488in}}{\pgfqpoint{3.079908in}{2.296312in}}%
\pgfpathcurveto{\pgfqpoint{3.085732in}{2.302136in}}{\pgfqpoint{3.089004in}{2.310036in}}{\pgfqpoint{3.089004in}{2.318272in}}%
\pgfpathcurveto{\pgfqpoint{3.089004in}{2.326508in}}{\pgfqpoint{3.085732in}{2.334408in}}{\pgfqpoint{3.079908in}{2.340232in}}%
\pgfpathcurveto{\pgfqpoint{3.074084in}{2.346056in}}{\pgfqpoint{3.066184in}{2.349329in}}{\pgfqpoint{3.057948in}{2.349329in}}%
\pgfpathcurveto{\pgfqpoint{3.049711in}{2.349329in}}{\pgfqpoint{3.041811in}{2.346056in}}{\pgfqpoint{3.035987in}{2.340232in}}%
\pgfpathcurveto{\pgfqpoint{3.030164in}{2.334408in}}{\pgfqpoint{3.026891in}{2.326508in}}{\pgfqpoint{3.026891in}{2.318272in}}%
\pgfpathcurveto{\pgfqpoint{3.026891in}{2.310036in}}{\pgfqpoint{3.030164in}{2.302136in}}{\pgfqpoint{3.035987in}{2.296312in}}%
\pgfpathcurveto{\pgfqpoint{3.041811in}{2.290488in}}{\pgfqpoint{3.049711in}{2.287216in}}{\pgfqpoint{3.057948in}{2.287216in}}%
\pgfpathclose%
\pgfusepath{stroke,fill}%
\end{pgfscope}%
\begin{pgfscope}%
\pgfpathrectangle{\pgfqpoint{0.100000in}{0.220728in}}{\pgfqpoint{3.696000in}{3.696000in}}%
\pgfusepath{clip}%
\pgfsetbuttcap%
\pgfsetroundjoin%
\definecolor{currentfill}{rgb}{0.121569,0.466667,0.705882}%
\pgfsetfillcolor{currentfill}%
\pgfsetfillopacity{0.785334}%
\pgfsetlinewidth{1.003750pt}%
\definecolor{currentstroke}{rgb}{0.121569,0.466667,0.705882}%
\pgfsetstrokecolor{currentstroke}%
\pgfsetstrokeopacity{0.785334}%
\pgfsetdash{}{0pt}%
\pgfpathmoveto{\pgfqpoint{3.057159in}{2.285585in}}%
\pgfpathcurveto{\pgfqpoint{3.065396in}{2.285585in}}{\pgfqpoint{3.073296in}{2.288858in}}{\pgfqpoint{3.079120in}{2.294682in}}%
\pgfpathcurveto{\pgfqpoint{3.084944in}{2.300506in}}{\pgfqpoint{3.088216in}{2.308406in}}{\pgfqpoint{3.088216in}{2.316642in}}%
\pgfpathcurveto{\pgfqpoint{3.088216in}{2.324878in}}{\pgfqpoint{3.084944in}{2.332778in}}{\pgfqpoint{3.079120in}{2.338602in}}%
\pgfpathcurveto{\pgfqpoint{3.073296in}{2.344426in}}{\pgfqpoint{3.065396in}{2.347698in}}{\pgfqpoint{3.057159in}{2.347698in}}%
\pgfpathcurveto{\pgfqpoint{3.048923in}{2.347698in}}{\pgfqpoint{3.041023in}{2.344426in}}{\pgfqpoint{3.035199in}{2.338602in}}%
\pgfpathcurveto{\pgfqpoint{3.029375in}{2.332778in}}{\pgfqpoint{3.026103in}{2.324878in}}{\pgfqpoint{3.026103in}{2.316642in}}%
\pgfpathcurveto{\pgfqpoint{3.026103in}{2.308406in}}{\pgfqpoint{3.029375in}{2.300506in}}{\pgfqpoint{3.035199in}{2.294682in}}%
\pgfpathcurveto{\pgfqpoint{3.041023in}{2.288858in}}{\pgfqpoint{3.048923in}{2.285585in}}{\pgfqpoint{3.057159in}{2.285585in}}%
\pgfpathclose%
\pgfusepath{stroke,fill}%
\end{pgfscope}%
\begin{pgfscope}%
\pgfpathrectangle{\pgfqpoint{0.100000in}{0.220728in}}{\pgfqpoint{3.696000in}{3.696000in}}%
\pgfusepath{clip}%
\pgfsetbuttcap%
\pgfsetroundjoin%
\definecolor{currentfill}{rgb}{0.121569,0.466667,0.705882}%
\pgfsetfillcolor{currentfill}%
\pgfsetfillopacity{0.785505}%
\pgfsetlinewidth{1.003750pt}%
\definecolor{currentstroke}{rgb}{0.121569,0.466667,0.705882}%
\pgfsetstrokecolor{currentstroke}%
\pgfsetstrokeopacity{0.785505}%
\pgfsetdash{}{0pt}%
\pgfpathmoveto{\pgfqpoint{3.056649in}{2.284791in}}%
\pgfpathcurveto{\pgfqpoint{3.064885in}{2.284791in}}{\pgfqpoint{3.072785in}{2.288064in}}{\pgfqpoint{3.078609in}{2.293888in}}%
\pgfpathcurveto{\pgfqpoint{3.084433in}{2.299712in}}{\pgfqpoint{3.087705in}{2.307612in}}{\pgfqpoint{3.087705in}{2.315848in}}%
\pgfpathcurveto{\pgfqpoint{3.087705in}{2.324084in}}{\pgfqpoint{3.084433in}{2.331984in}}{\pgfqpoint{3.078609in}{2.337808in}}%
\pgfpathcurveto{\pgfqpoint{3.072785in}{2.343632in}}{\pgfqpoint{3.064885in}{2.346904in}}{\pgfqpoint{3.056649in}{2.346904in}}%
\pgfpathcurveto{\pgfqpoint{3.048413in}{2.346904in}}{\pgfqpoint{3.040513in}{2.343632in}}{\pgfqpoint{3.034689in}{2.337808in}}%
\pgfpathcurveto{\pgfqpoint{3.028865in}{2.331984in}}{\pgfqpoint{3.025592in}{2.324084in}}{\pgfqpoint{3.025592in}{2.315848in}}%
\pgfpathcurveto{\pgfqpoint{3.025592in}{2.307612in}}{\pgfqpoint{3.028865in}{2.299712in}}{\pgfqpoint{3.034689in}{2.293888in}}%
\pgfpathcurveto{\pgfqpoint{3.040513in}{2.288064in}}{\pgfqpoint{3.048413in}{2.284791in}}{\pgfqpoint{3.056649in}{2.284791in}}%
\pgfpathclose%
\pgfusepath{stroke,fill}%
\end{pgfscope}%
\begin{pgfscope}%
\pgfpathrectangle{\pgfqpoint{0.100000in}{0.220728in}}{\pgfqpoint{3.696000in}{3.696000in}}%
\pgfusepath{clip}%
\pgfsetbuttcap%
\pgfsetroundjoin%
\definecolor{currentfill}{rgb}{0.121569,0.466667,0.705882}%
\pgfsetfillcolor{currentfill}%
\pgfsetfillopacity{0.786104}%
\pgfsetlinewidth{1.003750pt}%
\definecolor{currentstroke}{rgb}{0.121569,0.466667,0.705882}%
\pgfsetstrokecolor{currentstroke}%
\pgfsetstrokeopacity{0.786104}%
\pgfsetdash{}{0pt}%
\pgfpathmoveto{\pgfqpoint{3.055534in}{2.281524in}}%
\pgfpathcurveto{\pgfqpoint{3.063770in}{2.281524in}}{\pgfqpoint{3.071670in}{2.284796in}}{\pgfqpoint{3.077494in}{2.290620in}}%
\pgfpathcurveto{\pgfqpoint{3.083318in}{2.296444in}}{\pgfqpoint{3.086590in}{2.304344in}}{\pgfqpoint{3.086590in}{2.312580in}}%
\pgfpathcurveto{\pgfqpoint{3.086590in}{2.320816in}}{\pgfqpoint{3.083318in}{2.328717in}}{\pgfqpoint{3.077494in}{2.334540in}}%
\pgfpathcurveto{\pgfqpoint{3.071670in}{2.340364in}}{\pgfqpoint{3.063770in}{2.343637in}}{\pgfqpoint{3.055534in}{2.343637in}}%
\pgfpathcurveto{\pgfqpoint{3.047297in}{2.343637in}}{\pgfqpoint{3.039397in}{2.340364in}}{\pgfqpoint{3.033573in}{2.334540in}}%
\pgfpathcurveto{\pgfqpoint{3.027750in}{2.328717in}}{\pgfqpoint{3.024477in}{2.320816in}}{\pgfqpoint{3.024477in}{2.312580in}}%
\pgfpathcurveto{\pgfqpoint{3.024477in}{2.304344in}}{\pgfqpoint{3.027750in}{2.296444in}}{\pgfqpoint{3.033573in}{2.290620in}}%
\pgfpathcurveto{\pgfqpoint{3.039397in}{2.284796in}}{\pgfqpoint{3.047297in}{2.281524in}}{\pgfqpoint{3.055534in}{2.281524in}}%
\pgfpathclose%
\pgfusepath{stroke,fill}%
\end{pgfscope}%
\begin{pgfscope}%
\pgfpathrectangle{\pgfqpoint{0.100000in}{0.220728in}}{\pgfqpoint{3.696000in}{3.696000in}}%
\pgfusepath{clip}%
\pgfsetbuttcap%
\pgfsetroundjoin%
\definecolor{currentfill}{rgb}{0.121569,0.466667,0.705882}%
\pgfsetfillcolor{currentfill}%
\pgfsetfillopacity{0.786839}%
\pgfsetlinewidth{1.003750pt}%
\definecolor{currentstroke}{rgb}{0.121569,0.466667,0.705882}%
\pgfsetstrokecolor{currentstroke}%
\pgfsetstrokeopacity{0.786839}%
\pgfsetdash{}{0pt}%
\pgfpathmoveto{\pgfqpoint{3.053425in}{2.277686in}}%
\pgfpathcurveto{\pgfqpoint{3.061661in}{2.277686in}}{\pgfqpoint{3.069561in}{2.280958in}}{\pgfqpoint{3.075385in}{2.286782in}}%
\pgfpathcurveto{\pgfqpoint{3.081209in}{2.292606in}}{\pgfqpoint{3.084481in}{2.300506in}}{\pgfqpoint{3.084481in}{2.308742in}}%
\pgfpathcurveto{\pgfqpoint{3.084481in}{2.316978in}}{\pgfqpoint{3.081209in}{2.324878in}}{\pgfqpoint{3.075385in}{2.330702in}}%
\pgfpathcurveto{\pgfqpoint{3.069561in}{2.336526in}}{\pgfqpoint{3.061661in}{2.339799in}}{\pgfqpoint{3.053425in}{2.339799in}}%
\pgfpathcurveto{\pgfqpoint{3.045188in}{2.339799in}}{\pgfqpoint{3.037288in}{2.336526in}}{\pgfqpoint{3.031464in}{2.330702in}}%
\pgfpathcurveto{\pgfqpoint{3.025640in}{2.324878in}}{\pgfqpoint{3.022368in}{2.316978in}}{\pgfqpoint{3.022368in}{2.308742in}}%
\pgfpathcurveto{\pgfqpoint{3.022368in}{2.300506in}}{\pgfqpoint{3.025640in}{2.292606in}}{\pgfqpoint{3.031464in}{2.286782in}}%
\pgfpathcurveto{\pgfqpoint{3.037288in}{2.280958in}}{\pgfqpoint{3.045188in}{2.277686in}}{\pgfqpoint{3.053425in}{2.277686in}}%
\pgfpathclose%
\pgfusepath{stroke,fill}%
\end{pgfscope}%
\begin{pgfscope}%
\pgfpathrectangle{\pgfqpoint{0.100000in}{0.220728in}}{\pgfqpoint{3.696000in}{3.696000in}}%
\pgfusepath{clip}%
\pgfsetbuttcap%
\pgfsetroundjoin%
\definecolor{currentfill}{rgb}{0.121569,0.466667,0.705882}%
\pgfsetfillcolor{currentfill}%
\pgfsetfillopacity{0.787839}%
\pgfsetlinewidth{1.003750pt}%
\definecolor{currentstroke}{rgb}{0.121569,0.466667,0.705882}%
\pgfsetstrokecolor{currentstroke}%
\pgfsetstrokeopacity{0.787839}%
\pgfsetdash{}{0pt}%
\pgfpathmoveto{\pgfqpoint{3.050152in}{2.272166in}}%
\pgfpathcurveto{\pgfqpoint{3.058388in}{2.272166in}}{\pgfqpoint{3.066289in}{2.275438in}}{\pgfqpoint{3.072112in}{2.281262in}}%
\pgfpathcurveto{\pgfqpoint{3.077936in}{2.287086in}}{\pgfqpoint{3.081209in}{2.294986in}}{\pgfqpoint{3.081209in}{2.303223in}}%
\pgfpathcurveto{\pgfqpoint{3.081209in}{2.311459in}}{\pgfqpoint{3.077936in}{2.319359in}}{\pgfqpoint{3.072112in}{2.325183in}}%
\pgfpathcurveto{\pgfqpoint{3.066289in}{2.331007in}}{\pgfqpoint{3.058388in}{2.334279in}}{\pgfqpoint{3.050152in}{2.334279in}}%
\pgfpathcurveto{\pgfqpoint{3.041916in}{2.334279in}}{\pgfqpoint{3.034016in}{2.331007in}}{\pgfqpoint{3.028192in}{2.325183in}}%
\pgfpathcurveto{\pgfqpoint{3.022368in}{2.319359in}}{\pgfqpoint{3.019096in}{2.311459in}}{\pgfqpoint{3.019096in}{2.303223in}}%
\pgfpathcurveto{\pgfqpoint{3.019096in}{2.294986in}}{\pgfqpoint{3.022368in}{2.287086in}}{\pgfqpoint{3.028192in}{2.281262in}}%
\pgfpathcurveto{\pgfqpoint{3.034016in}{2.275438in}}{\pgfqpoint{3.041916in}{2.272166in}}{\pgfqpoint{3.050152in}{2.272166in}}%
\pgfpathclose%
\pgfusepath{stroke,fill}%
\end{pgfscope}%
\begin{pgfscope}%
\pgfpathrectangle{\pgfqpoint{0.100000in}{0.220728in}}{\pgfqpoint{3.696000in}{3.696000in}}%
\pgfusepath{clip}%
\pgfsetbuttcap%
\pgfsetroundjoin%
\definecolor{currentfill}{rgb}{0.121569,0.466667,0.705882}%
\pgfsetfillcolor{currentfill}%
\pgfsetfillopacity{0.788197}%
\pgfsetlinewidth{1.003750pt}%
\definecolor{currentstroke}{rgb}{0.121569,0.466667,0.705882}%
\pgfsetstrokecolor{currentstroke}%
\pgfsetstrokeopacity{0.788197}%
\pgfsetdash{}{0pt}%
\pgfpathmoveto{\pgfqpoint{1.280863in}{1.145287in}}%
\pgfpathcurveto{\pgfqpoint{1.289099in}{1.145287in}}{\pgfqpoint{1.296999in}{1.148559in}}{\pgfqpoint{1.302823in}{1.154383in}}%
\pgfpathcurveto{\pgfqpoint{1.308647in}{1.160207in}}{\pgfqpoint{1.311920in}{1.168107in}}{\pgfqpoint{1.311920in}{1.176343in}}%
\pgfpathcurveto{\pgfqpoint{1.311920in}{1.184580in}}{\pgfqpoint{1.308647in}{1.192480in}}{\pgfqpoint{1.302823in}{1.198304in}}%
\pgfpathcurveto{\pgfqpoint{1.296999in}{1.204128in}}{\pgfqpoint{1.289099in}{1.207400in}}{\pgfqpoint{1.280863in}{1.207400in}}%
\pgfpathcurveto{\pgfqpoint{1.272627in}{1.207400in}}{\pgfqpoint{1.264727in}{1.204128in}}{\pgfqpoint{1.258903in}{1.198304in}}%
\pgfpathcurveto{\pgfqpoint{1.253079in}{1.192480in}}{\pgfqpoint{1.249807in}{1.184580in}}{\pgfqpoint{1.249807in}{1.176343in}}%
\pgfpathcurveto{\pgfqpoint{1.249807in}{1.168107in}}{\pgfqpoint{1.253079in}{1.160207in}}{\pgfqpoint{1.258903in}{1.154383in}}%
\pgfpathcurveto{\pgfqpoint{1.264727in}{1.148559in}}{\pgfqpoint{1.272627in}{1.145287in}}{\pgfqpoint{1.280863in}{1.145287in}}%
\pgfpathclose%
\pgfusepath{stroke,fill}%
\end{pgfscope}%
\begin{pgfscope}%
\pgfpathrectangle{\pgfqpoint{0.100000in}{0.220728in}}{\pgfqpoint{3.696000in}{3.696000in}}%
\pgfusepath{clip}%
\pgfsetbuttcap%
\pgfsetroundjoin%
\definecolor{currentfill}{rgb}{0.121569,0.466667,0.705882}%
\pgfsetfillcolor{currentfill}%
\pgfsetfillopacity{0.789331}%
\pgfsetlinewidth{1.003750pt}%
\definecolor{currentstroke}{rgb}{0.121569,0.466667,0.705882}%
\pgfsetstrokecolor{currentstroke}%
\pgfsetstrokeopacity{0.789331}%
\pgfsetdash{}{0pt}%
\pgfpathmoveto{\pgfqpoint{3.046561in}{2.264734in}}%
\pgfpathcurveto{\pgfqpoint{3.054797in}{2.264734in}}{\pgfqpoint{3.062697in}{2.268006in}}{\pgfqpoint{3.068521in}{2.273830in}}%
\pgfpathcurveto{\pgfqpoint{3.074345in}{2.279654in}}{\pgfqpoint{3.077618in}{2.287554in}}{\pgfqpoint{3.077618in}{2.295790in}}%
\pgfpathcurveto{\pgfqpoint{3.077618in}{2.304026in}}{\pgfqpoint{3.074345in}{2.311927in}}{\pgfqpoint{3.068521in}{2.317750in}}%
\pgfpathcurveto{\pgfqpoint{3.062697in}{2.323574in}}{\pgfqpoint{3.054797in}{2.326847in}}{\pgfqpoint{3.046561in}{2.326847in}}%
\pgfpathcurveto{\pgfqpoint{3.038325in}{2.326847in}}{\pgfqpoint{3.030425in}{2.323574in}}{\pgfqpoint{3.024601in}{2.317750in}}%
\pgfpathcurveto{\pgfqpoint{3.018777in}{2.311927in}}{\pgfqpoint{3.015505in}{2.304026in}}{\pgfqpoint{3.015505in}{2.295790in}}%
\pgfpathcurveto{\pgfqpoint{3.015505in}{2.287554in}}{\pgfqpoint{3.018777in}{2.279654in}}{\pgfqpoint{3.024601in}{2.273830in}}%
\pgfpathcurveto{\pgfqpoint{3.030425in}{2.268006in}}{\pgfqpoint{3.038325in}{2.264734in}}{\pgfqpoint{3.046561in}{2.264734in}}%
\pgfpathclose%
\pgfusepath{stroke,fill}%
\end{pgfscope}%
\begin{pgfscope}%
\pgfpathrectangle{\pgfqpoint{0.100000in}{0.220728in}}{\pgfqpoint{3.696000in}{3.696000in}}%
\pgfusepath{clip}%
\pgfsetbuttcap%
\pgfsetroundjoin%
\definecolor{currentfill}{rgb}{0.121569,0.466667,0.705882}%
\pgfsetfillcolor{currentfill}%
\pgfsetfillopacity{0.790195}%
\pgfsetlinewidth{1.003750pt}%
\definecolor{currentstroke}{rgb}{0.121569,0.466667,0.705882}%
\pgfsetstrokecolor{currentstroke}%
\pgfsetstrokeopacity{0.790195}%
\pgfsetdash{}{0pt}%
\pgfpathmoveto{\pgfqpoint{3.045110in}{2.260347in}}%
\pgfpathcurveto{\pgfqpoint{3.053346in}{2.260347in}}{\pgfqpoint{3.061246in}{2.263620in}}{\pgfqpoint{3.067070in}{2.269444in}}%
\pgfpathcurveto{\pgfqpoint{3.072894in}{2.275268in}}{\pgfqpoint{3.076166in}{2.283168in}}{\pgfqpoint{3.076166in}{2.291404in}}%
\pgfpathcurveto{\pgfqpoint{3.076166in}{2.299640in}}{\pgfqpoint{3.072894in}{2.307540in}}{\pgfqpoint{3.067070in}{2.313364in}}%
\pgfpathcurveto{\pgfqpoint{3.061246in}{2.319188in}}{\pgfqpoint{3.053346in}{2.322460in}}{\pgfqpoint{3.045110in}{2.322460in}}%
\pgfpathcurveto{\pgfqpoint{3.036873in}{2.322460in}}{\pgfqpoint{3.028973in}{2.319188in}}{\pgfqpoint{3.023149in}{2.313364in}}%
\pgfpathcurveto{\pgfqpoint{3.017326in}{2.307540in}}{\pgfqpoint{3.014053in}{2.299640in}}{\pgfqpoint{3.014053in}{2.291404in}}%
\pgfpathcurveto{\pgfqpoint{3.014053in}{2.283168in}}{\pgfqpoint{3.017326in}{2.275268in}}{\pgfqpoint{3.023149in}{2.269444in}}%
\pgfpathcurveto{\pgfqpoint{3.028973in}{2.263620in}}{\pgfqpoint{3.036873in}{2.260347in}}{\pgfqpoint{3.045110in}{2.260347in}}%
\pgfpathclose%
\pgfusepath{stroke,fill}%
\end{pgfscope}%
\begin{pgfscope}%
\pgfpathrectangle{\pgfqpoint{0.100000in}{0.220728in}}{\pgfqpoint{3.696000in}{3.696000in}}%
\pgfusepath{clip}%
\pgfsetbuttcap%
\pgfsetroundjoin%
\definecolor{currentfill}{rgb}{0.121569,0.466667,0.705882}%
\pgfsetfillcolor{currentfill}%
\pgfsetfillopacity{0.791077}%
\pgfsetlinewidth{1.003750pt}%
\definecolor{currentstroke}{rgb}{0.121569,0.466667,0.705882}%
\pgfsetstrokecolor{currentstroke}%
\pgfsetstrokeopacity{0.791077}%
\pgfsetdash{}{0pt}%
\pgfpathmoveto{\pgfqpoint{3.042266in}{2.255789in}}%
\pgfpathcurveto{\pgfqpoint{3.050502in}{2.255789in}}{\pgfqpoint{3.058402in}{2.259061in}}{\pgfqpoint{3.064226in}{2.264885in}}%
\pgfpathcurveto{\pgfqpoint{3.070050in}{2.270709in}}{\pgfqpoint{3.073322in}{2.278609in}}{\pgfqpoint{3.073322in}{2.286846in}}%
\pgfpathcurveto{\pgfqpoint{3.073322in}{2.295082in}}{\pgfqpoint{3.070050in}{2.302982in}}{\pgfqpoint{3.064226in}{2.308806in}}%
\pgfpathcurveto{\pgfqpoint{3.058402in}{2.314630in}}{\pgfqpoint{3.050502in}{2.317902in}}{\pgfqpoint{3.042266in}{2.317902in}}%
\pgfpathcurveto{\pgfqpoint{3.034029in}{2.317902in}}{\pgfqpoint{3.026129in}{2.314630in}}{\pgfqpoint{3.020305in}{2.308806in}}%
\pgfpathcurveto{\pgfqpoint{3.014481in}{2.302982in}}{\pgfqpoint{3.011209in}{2.295082in}}{\pgfqpoint{3.011209in}{2.286846in}}%
\pgfpathcurveto{\pgfqpoint{3.011209in}{2.278609in}}{\pgfqpoint{3.014481in}{2.270709in}}{\pgfqpoint{3.020305in}{2.264885in}}%
\pgfpathcurveto{\pgfqpoint{3.026129in}{2.259061in}}{\pgfqpoint{3.034029in}{2.255789in}}{\pgfqpoint{3.042266in}{2.255789in}}%
\pgfpathclose%
\pgfusepath{stroke,fill}%
\end{pgfscope}%
\begin{pgfscope}%
\pgfpathrectangle{\pgfqpoint{0.100000in}{0.220728in}}{\pgfqpoint{3.696000in}{3.696000in}}%
\pgfusepath{clip}%
\pgfsetbuttcap%
\pgfsetroundjoin%
\definecolor{currentfill}{rgb}{0.121569,0.466667,0.705882}%
\pgfsetfillcolor{currentfill}%
\pgfsetfillopacity{0.792343}%
\pgfsetlinewidth{1.003750pt}%
\definecolor{currentstroke}{rgb}{0.121569,0.466667,0.705882}%
\pgfsetstrokecolor{currentstroke}%
\pgfsetstrokeopacity{0.792343}%
\pgfsetdash{}{0pt}%
\pgfpathmoveto{\pgfqpoint{3.039836in}{2.249103in}}%
\pgfpathcurveto{\pgfqpoint{3.048073in}{2.249103in}}{\pgfqpoint{3.055973in}{2.252375in}}{\pgfqpoint{3.061796in}{2.258199in}}%
\pgfpathcurveto{\pgfqpoint{3.067620in}{2.264023in}}{\pgfqpoint{3.070893in}{2.271923in}}{\pgfqpoint{3.070893in}{2.280159in}}%
\pgfpathcurveto{\pgfqpoint{3.070893in}{2.288395in}}{\pgfqpoint{3.067620in}{2.296295in}}{\pgfqpoint{3.061796in}{2.302119in}}%
\pgfpathcurveto{\pgfqpoint{3.055973in}{2.307943in}}{\pgfqpoint{3.048073in}{2.311216in}}{\pgfqpoint{3.039836in}{2.311216in}}%
\pgfpathcurveto{\pgfqpoint{3.031600in}{2.311216in}}{\pgfqpoint{3.023700in}{2.307943in}}{\pgfqpoint{3.017876in}{2.302119in}}%
\pgfpathcurveto{\pgfqpoint{3.012052in}{2.296295in}}{\pgfqpoint{3.008780in}{2.288395in}}{\pgfqpoint{3.008780in}{2.280159in}}%
\pgfpathcurveto{\pgfqpoint{3.008780in}{2.271923in}}{\pgfqpoint{3.012052in}{2.264023in}}{\pgfqpoint{3.017876in}{2.258199in}}%
\pgfpathcurveto{\pgfqpoint{3.023700in}{2.252375in}}{\pgfqpoint{3.031600in}{2.249103in}}{\pgfqpoint{3.039836in}{2.249103in}}%
\pgfpathclose%
\pgfusepath{stroke,fill}%
\end{pgfscope}%
\begin{pgfscope}%
\pgfpathrectangle{\pgfqpoint{0.100000in}{0.220728in}}{\pgfqpoint{3.696000in}{3.696000in}}%
\pgfusepath{clip}%
\pgfsetbuttcap%
\pgfsetroundjoin%
\definecolor{currentfill}{rgb}{0.121569,0.466667,0.705882}%
\pgfsetfillcolor{currentfill}%
\pgfsetfillopacity{0.792395}%
\pgfsetlinewidth{1.003750pt}%
\definecolor{currentstroke}{rgb}{0.121569,0.466667,0.705882}%
\pgfsetstrokecolor{currentstroke}%
\pgfsetstrokeopacity{0.792395}%
\pgfsetdash{}{0pt}%
\pgfpathmoveto{\pgfqpoint{1.301483in}{1.137122in}}%
\pgfpathcurveto{\pgfqpoint{1.309720in}{1.137122in}}{\pgfqpoint{1.317620in}{1.140394in}}{\pgfqpoint{1.323444in}{1.146218in}}%
\pgfpathcurveto{\pgfqpoint{1.329268in}{1.152042in}}{\pgfqpoint{1.332540in}{1.159942in}}{\pgfqpoint{1.332540in}{1.168179in}}%
\pgfpathcurveto{\pgfqpoint{1.332540in}{1.176415in}}{\pgfqpoint{1.329268in}{1.184315in}}{\pgfqpoint{1.323444in}{1.190139in}}%
\pgfpathcurveto{\pgfqpoint{1.317620in}{1.195963in}}{\pgfqpoint{1.309720in}{1.199235in}}{\pgfqpoint{1.301483in}{1.199235in}}%
\pgfpathcurveto{\pgfqpoint{1.293247in}{1.199235in}}{\pgfqpoint{1.285347in}{1.195963in}}{\pgfqpoint{1.279523in}{1.190139in}}%
\pgfpathcurveto{\pgfqpoint{1.273699in}{1.184315in}}{\pgfqpoint{1.270427in}{1.176415in}}{\pgfqpoint{1.270427in}{1.168179in}}%
\pgfpathcurveto{\pgfqpoint{1.270427in}{1.159942in}}{\pgfqpoint{1.273699in}{1.152042in}}{\pgfqpoint{1.279523in}{1.146218in}}%
\pgfpathcurveto{\pgfqpoint{1.285347in}{1.140394in}}{\pgfqpoint{1.293247in}{1.137122in}}{\pgfqpoint{1.301483in}{1.137122in}}%
\pgfpathclose%
\pgfusepath{stroke,fill}%
\end{pgfscope}%
\begin{pgfscope}%
\pgfpathrectangle{\pgfqpoint{0.100000in}{0.220728in}}{\pgfqpoint{3.696000in}{3.696000in}}%
\pgfusepath{clip}%
\pgfsetbuttcap%
\pgfsetroundjoin%
\definecolor{currentfill}{rgb}{0.121569,0.466667,0.705882}%
\pgfsetfillcolor{currentfill}%
\pgfsetfillopacity{0.793808}%
\pgfsetlinewidth{1.003750pt}%
\definecolor{currentstroke}{rgb}{0.121569,0.466667,0.705882}%
\pgfsetstrokecolor{currentstroke}%
\pgfsetstrokeopacity{0.793808}%
\pgfsetdash{}{0pt}%
\pgfpathmoveto{\pgfqpoint{3.037076in}{2.242085in}}%
\pgfpathcurveto{\pgfqpoint{3.045312in}{2.242085in}}{\pgfqpoint{3.053212in}{2.245358in}}{\pgfqpoint{3.059036in}{2.251182in}}%
\pgfpathcurveto{\pgfqpoint{3.064860in}{2.257005in}}{\pgfqpoint{3.068132in}{2.264906in}}{\pgfqpoint{3.068132in}{2.273142in}}%
\pgfpathcurveto{\pgfqpoint{3.068132in}{2.281378in}}{\pgfqpoint{3.064860in}{2.289278in}}{\pgfqpoint{3.059036in}{2.295102in}}%
\pgfpathcurveto{\pgfqpoint{3.053212in}{2.300926in}}{\pgfqpoint{3.045312in}{2.304198in}}{\pgfqpoint{3.037076in}{2.304198in}}%
\pgfpathcurveto{\pgfqpoint{3.028840in}{2.304198in}}{\pgfqpoint{3.020940in}{2.300926in}}{\pgfqpoint{3.015116in}{2.295102in}}%
\pgfpathcurveto{\pgfqpoint{3.009292in}{2.289278in}}{\pgfqpoint{3.006019in}{2.281378in}}{\pgfqpoint{3.006019in}{2.273142in}}%
\pgfpathcurveto{\pgfqpoint{3.006019in}{2.264906in}}{\pgfqpoint{3.009292in}{2.257005in}}{\pgfqpoint{3.015116in}{2.251182in}}%
\pgfpathcurveto{\pgfqpoint{3.020940in}{2.245358in}}{\pgfqpoint{3.028840in}{2.242085in}}{\pgfqpoint{3.037076in}{2.242085in}}%
\pgfpathclose%
\pgfusepath{stroke,fill}%
\end{pgfscope}%
\begin{pgfscope}%
\pgfpathrectangle{\pgfqpoint{0.100000in}{0.220728in}}{\pgfqpoint{3.696000in}{3.696000in}}%
\pgfusepath{clip}%
\pgfsetbuttcap%
\pgfsetroundjoin%
\definecolor{currentfill}{rgb}{0.121569,0.466667,0.705882}%
\pgfsetfillcolor{currentfill}%
\pgfsetfillopacity{0.794486}%
\pgfsetlinewidth{1.003750pt}%
\definecolor{currentstroke}{rgb}{0.121569,0.466667,0.705882}%
\pgfsetstrokecolor{currentstroke}%
\pgfsetstrokeopacity{0.794486}%
\pgfsetdash{}{0pt}%
\pgfpathmoveto{\pgfqpoint{3.034846in}{2.238525in}}%
\pgfpathcurveto{\pgfqpoint{3.043082in}{2.238525in}}{\pgfqpoint{3.050982in}{2.241797in}}{\pgfqpoint{3.056806in}{2.247621in}}%
\pgfpathcurveto{\pgfqpoint{3.062630in}{2.253445in}}{\pgfqpoint{3.065903in}{2.261345in}}{\pgfqpoint{3.065903in}{2.269581in}}%
\pgfpathcurveto{\pgfqpoint{3.065903in}{2.277818in}}{\pgfqpoint{3.062630in}{2.285718in}}{\pgfqpoint{3.056806in}{2.291542in}}%
\pgfpathcurveto{\pgfqpoint{3.050982in}{2.297365in}}{\pgfqpoint{3.043082in}{2.300638in}}{\pgfqpoint{3.034846in}{2.300638in}}%
\pgfpathcurveto{\pgfqpoint{3.026610in}{2.300638in}}{\pgfqpoint{3.018710in}{2.297365in}}{\pgfqpoint{3.012886in}{2.291542in}}%
\pgfpathcurveto{\pgfqpoint{3.007062in}{2.285718in}}{\pgfqpoint{3.003790in}{2.277818in}}{\pgfqpoint{3.003790in}{2.269581in}}%
\pgfpathcurveto{\pgfqpoint{3.003790in}{2.261345in}}{\pgfqpoint{3.007062in}{2.253445in}}{\pgfqpoint{3.012886in}{2.247621in}}%
\pgfpathcurveto{\pgfqpoint{3.018710in}{2.241797in}}{\pgfqpoint{3.026610in}{2.238525in}}{\pgfqpoint{3.034846in}{2.238525in}}%
\pgfpathclose%
\pgfusepath{stroke,fill}%
\end{pgfscope}%
\begin{pgfscope}%
\pgfpathrectangle{\pgfqpoint{0.100000in}{0.220728in}}{\pgfqpoint{3.696000in}{3.696000in}}%
\pgfusepath{clip}%
\pgfsetbuttcap%
\pgfsetroundjoin%
\definecolor{currentfill}{rgb}{0.121569,0.466667,0.705882}%
\pgfsetfillcolor{currentfill}%
\pgfsetfillopacity{0.795618}%
\pgfsetlinewidth{1.003750pt}%
\definecolor{currentstroke}{rgb}{0.121569,0.466667,0.705882}%
\pgfsetstrokecolor{currentstroke}%
\pgfsetstrokeopacity{0.795618}%
\pgfsetdash{}{0pt}%
\pgfpathmoveto{\pgfqpoint{3.033006in}{2.233075in}}%
\pgfpathcurveto{\pgfqpoint{3.041243in}{2.233075in}}{\pgfqpoint{3.049143in}{2.236347in}}{\pgfqpoint{3.054966in}{2.242171in}}%
\pgfpathcurveto{\pgfqpoint{3.060790in}{2.247995in}}{\pgfqpoint{3.064063in}{2.255895in}}{\pgfqpoint{3.064063in}{2.264131in}}%
\pgfpathcurveto{\pgfqpoint{3.064063in}{2.272367in}}{\pgfqpoint{3.060790in}{2.280267in}}{\pgfqpoint{3.054966in}{2.286091in}}%
\pgfpathcurveto{\pgfqpoint{3.049143in}{2.291915in}}{\pgfqpoint{3.041243in}{2.295188in}}{\pgfqpoint{3.033006in}{2.295188in}}%
\pgfpathcurveto{\pgfqpoint{3.024770in}{2.295188in}}{\pgfqpoint{3.016870in}{2.291915in}}{\pgfqpoint{3.011046in}{2.286091in}}%
\pgfpathcurveto{\pgfqpoint{3.005222in}{2.280267in}}{\pgfqpoint{3.001950in}{2.272367in}}{\pgfqpoint{3.001950in}{2.264131in}}%
\pgfpathcurveto{\pgfqpoint{3.001950in}{2.255895in}}{\pgfqpoint{3.005222in}{2.247995in}}{\pgfqpoint{3.011046in}{2.242171in}}%
\pgfpathcurveto{\pgfqpoint{3.016870in}{2.236347in}}{\pgfqpoint{3.024770in}{2.233075in}}{\pgfqpoint{3.033006in}{2.233075in}}%
\pgfpathclose%
\pgfusepath{stroke,fill}%
\end{pgfscope}%
\begin{pgfscope}%
\pgfpathrectangle{\pgfqpoint{0.100000in}{0.220728in}}{\pgfqpoint{3.696000in}{3.696000in}}%
\pgfusepath{clip}%
\pgfsetbuttcap%
\pgfsetroundjoin%
\definecolor{currentfill}{rgb}{0.121569,0.466667,0.705882}%
\pgfsetfillcolor{currentfill}%
\pgfsetfillopacity{0.796200}%
\pgfsetlinewidth{1.003750pt}%
\definecolor{currentstroke}{rgb}{0.121569,0.466667,0.705882}%
\pgfsetstrokecolor{currentstroke}%
\pgfsetstrokeopacity{0.796200}%
\pgfsetdash{}{0pt}%
\pgfpathmoveto{\pgfqpoint{3.031623in}{2.230241in}}%
\pgfpathcurveto{\pgfqpoint{3.039859in}{2.230241in}}{\pgfqpoint{3.047760in}{2.233513in}}{\pgfqpoint{3.053583in}{2.239337in}}%
\pgfpathcurveto{\pgfqpoint{3.059407in}{2.245161in}}{\pgfqpoint{3.062680in}{2.253061in}}{\pgfqpoint{3.062680in}{2.261298in}}%
\pgfpathcurveto{\pgfqpoint{3.062680in}{2.269534in}}{\pgfqpoint{3.059407in}{2.277434in}}{\pgfqpoint{3.053583in}{2.283258in}}%
\pgfpathcurveto{\pgfqpoint{3.047760in}{2.289082in}}{\pgfqpoint{3.039859in}{2.292354in}}{\pgfqpoint{3.031623in}{2.292354in}}%
\pgfpathcurveto{\pgfqpoint{3.023387in}{2.292354in}}{\pgfqpoint{3.015487in}{2.289082in}}{\pgfqpoint{3.009663in}{2.283258in}}%
\pgfpathcurveto{\pgfqpoint{3.003839in}{2.277434in}}{\pgfqpoint{3.000567in}{2.269534in}}{\pgfqpoint{3.000567in}{2.261298in}}%
\pgfpathcurveto{\pgfqpoint{3.000567in}{2.253061in}}{\pgfqpoint{3.003839in}{2.245161in}}{\pgfqpoint{3.009663in}{2.239337in}}%
\pgfpathcurveto{\pgfqpoint{3.015487in}{2.233513in}}{\pgfqpoint{3.023387in}{2.230241in}}{\pgfqpoint{3.031623in}{2.230241in}}%
\pgfpathclose%
\pgfusepath{stroke,fill}%
\end{pgfscope}%
\begin{pgfscope}%
\pgfpathrectangle{\pgfqpoint{0.100000in}{0.220728in}}{\pgfqpoint{3.696000in}{3.696000in}}%
\pgfusepath{clip}%
\pgfsetbuttcap%
\pgfsetroundjoin%
\definecolor{currentfill}{rgb}{0.121569,0.466667,0.705882}%
\pgfsetfillcolor{currentfill}%
\pgfsetfillopacity{0.796615}%
\pgfsetlinewidth{1.003750pt}%
\definecolor{currentstroke}{rgb}{0.121569,0.466667,0.705882}%
\pgfsetstrokecolor{currentstroke}%
\pgfsetstrokeopacity{0.796615}%
\pgfsetdash{}{0pt}%
\pgfpathmoveto{\pgfqpoint{1.320250in}{1.132314in}}%
\pgfpathcurveto{\pgfqpoint{1.328486in}{1.132314in}}{\pgfqpoint{1.336386in}{1.135586in}}{\pgfqpoint{1.342210in}{1.141410in}}%
\pgfpathcurveto{\pgfqpoint{1.348034in}{1.147234in}}{\pgfqpoint{1.351307in}{1.155134in}}{\pgfqpoint{1.351307in}{1.163370in}}%
\pgfpathcurveto{\pgfqpoint{1.351307in}{1.171607in}}{\pgfqpoint{1.348034in}{1.179507in}}{\pgfqpoint{1.342210in}{1.185331in}}%
\pgfpathcurveto{\pgfqpoint{1.336386in}{1.191155in}}{\pgfqpoint{1.328486in}{1.194427in}}{\pgfqpoint{1.320250in}{1.194427in}}%
\pgfpathcurveto{\pgfqpoint{1.312014in}{1.194427in}}{\pgfqpoint{1.304114in}{1.191155in}}{\pgfqpoint{1.298290in}{1.185331in}}%
\pgfpathcurveto{\pgfqpoint{1.292466in}{1.179507in}}{\pgfqpoint{1.289194in}{1.171607in}}{\pgfqpoint{1.289194in}{1.163370in}}%
\pgfpathcurveto{\pgfqpoint{1.289194in}{1.155134in}}{\pgfqpoint{1.292466in}{1.147234in}}{\pgfqpoint{1.298290in}{1.141410in}}%
\pgfpathcurveto{\pgfqpoint{1.304114in}{1.135586in}}{\pgfqpoint{1.312014in}{1.132314in}}{\pgfqpoint{1.320250in}{1.132314in}}%
\pgfpathclose%
\pgfusepath{stroke,fill}%
\end{pgfscope}%
\begin{pgfscope}%
\pgfpathrectangle{\pgfqpoint{0.100000in}{0.220728in}}{\pgfqpoint{3.696000in}{3.696000in}}%
\pgfusepath{clip}%
\pgfsetbuttcap%
\pgfsetroundjoin%
\definecolor{currentfill}{rgb}{0.121569,0.466667,0.705882}%
\pgfsetfillcolor{currentfill}%
\pgfsetfillopacity{0.796952}%
\pgfsetlinewidth{1.003750pt}%
\definecolor{currentstroke}{rgb}{0.121569,0.466667,0.705882}%
\pgfsetstrokecolor{currentstroke}%
\pgfsetstrokeopacity{0.796952}%
\pgfsetdash{}{0pt}%
\pgfpathmoveto{\pgfqpoint{3.029589in}{2.227014in}}%
\pgfpathcurveto{\pgfqpoint{3.037825in}{2.227014in}}{\pgfqpoint{3.045725in}{2.230286in}}{\pgfqpoint{3.051549in}{2.236110in}}%
\pgfpathcurveto{\pgfqpoint{3.057373in}{2.241934in}}{\pgfqpoint{3.060645in}{2.249834in}}{\pgfqpoint{3.060645in}{2.258071in}}%
\pgfpathcurveto{\pgfqpoint{3.060645in}{2.266307in}}{\pgfqpoint{3.057373in}{2.274207in}}{\pgfqpoint{3.051549in}{2.280031in}}%
\pgfpathcurveto{\pgfqpoint{3.045725in}{2.285855in}}{\pgfqpoint{3.037825in}{2.289127in}}{\pgfqpoint{3.029589in}{2.289127in}}%
\pgfpathcurveto{\pgfqpoint{3.021352in}{2.289127in}}{\pgfqpoint{3.013452in}{2.285855in}}{\pgfqpoint{3.007628in}{2.280031in}}%
\pgfpathcurveto{\pgfqpoint{3.001804in}{2.274207in}}{\pgfqpoint{2.998532in}{2.266307in}}{\pgfqpoint{2.998532in}{2.258071in}}%
\pgfpathcurveto{\pgfqpoint{2.998532in}{2.249834in}}{\pgfqpoint{3.001804in}{2.241934in}}{\pgfqpoint{3.007628in}{2.236110in}}%
\pgfpathcurveto{\pgfqpoint{3.013452in}{2.230286in}}{\pgfqpoint{3.021352in}{2.227014in}}{\pgfqpoint{3.029589in}{2.227014in}}%
\pgfpathclose%
\pgfusepath{stroke,fill}%
\end{pgfscope}%
\begin{pgfscope}%
\pgfpathrectangle{\pgfqpoint{0.100000in}{0.220728in}}{\pgfqpoint{3.696000in}{3.696000in}}%
\pgfusepath{clip}%
\pgfsetbuttcap%
\pgfsetroundjoin%
\definecolor{currentfill}{rgb}{0.121569,0.466667,0.705882}%
\pgfsetfillcolor{currentfill}%
\pgfsetfillopacity{0.797851}%
\pgfsetlinewidth{1.003750pt}%
\definecolor{currentstroke}{rgb}{0.121569,0.466667,0.705882}%
\pgfsetstrokecolor{currentstroke}%
\pgfsetstrokeopacity{0.797851}%
\pgfsetdash{}{0pt}%
\pgfpathmoveto{\pgfqpoint{3.027736in}{2.221728in}}%
\pgfpathcurveto{\pgfqpoint{3.035972in}{2.221728in}}{\pgfqpoint{3.043872in}{2.225000in}}{\pgfqpoint{3.049696in}{2.230824in}}%
\pgfpathcurveto{\pgfqpoint{3.055520in}{2.236648in}}{\pgfqpoint{3.058793in}{2.244548in}}{\pgfqpoint{3.058793in}{2.252784in}}%
\pgfpathcurveto{\pgfqpoint{3.058793in}{2.261021in}}{\pgfqpoint{3.055520in}{2.268921in}}{\pgfqpoint{3.049696in}{2.274745in}}%
\pgfpathcurveto{\pgfqpoint{3.043872in}{2.280569in}}{\pgfqpoint{3.035972in}{2.283841in}}{\pgfqpoint{3.027736in}{2.283841in}}%
\pgfpathcurveto{\pgfqpoint{3.019500in}{2.283841in}}{\pgfqpoint{3.011600in}{2.280569in}}{\pgfqpoint{3.005776in}{2.274745in}}%
\pgfpathcurveto{\pgfqpoint{2.999952in}{2.268921in}}{\pgfqpoint{2.996680in}{2.261021in}}{\pgfqpoint{2.996680in}{2.252784in}}%
\pgfpathcurveto{\pgfqpoint{2.996680in}{2.244548in}}{\pgfqpoint{2.999952in}{2.236648in}}{\pgfqpoint{3.005776in}{2.230824in}}%
\pgfpathcurveto{\pgfqpoint{3.011600in}{2.225000in}}{\pgfqpoint{3.019500in}{2.221728in}}{\pgfqpoint{3.027736in}{2.221728in}}%
\pgfpathclose%
\pgfusepath{stroke,fill}%
\end{pgfscope}%
\begin{pgfscope}%
\pgfpathrectangle{\pgfqpoint{0.100000in}{0.220728in}}{\pgfqpoint{3.696000in}{3.696000in}}%
\pgfusepath{clip}%
\pgfsetbuttcap%
\pgfsetroundjoin%
\definecolor{currentfill}{rgb}{0.121569,0.466667,0.705882}%
\pgfsetfillcolor{currentfill}%
\pgfsetfillopacity{0.798361}%
\pgfsetlinewidth{1.003750pt}%
\definecolor{currentstroke}{rgb}{0.121569,0.466667,0.705882}%
\pgfsetstrokecolor{currentstroke}%
\pgfsetstrokeopacity{0.798361}%
\pgfsetdash{}{0pt}%
\pgfpathmoveto{\pgfqpoint{3.026417in}{2.219184in}}%
\pgfpathcurveto{\pgfqpoint{3.034653in}{2.219184in}}{\pgfqpoint{3.042553in}{2.222457in}}{\pgfqpoint{3.048377in}{2.228281in}}%
\pgfpathcurveto{\pgfqpoint{3.054201in}{2.234104in}}{\pgfqpoint{3.057473in}{2.242005in}}{\pgfqpoint{3.057473in}{2.250241in}}%
\pgfpathcurveto{\pgfqpoint{3.057473in}{2.258477in}}{\pgfqpoint{3.054201in}{2.266377in}}{\pgfqpoint{3.048377in}{2.272201in}}%
\pgfpathcurveto{\pgfqpoint{3.042553in}{2.278025in}}{\pgfqpoint{3.034653in}{2.281297in}}{\pgfqpoint{3.026417in}{2.281297in}}%
\pgfpathcurveto{\pgfqpoint{3.018180in}{2.281297in}}{\pgfqpoint{3.010280in}{2.278025in}}{\pgfqpoint{3.004456in}{2.272201in}}%
\pgfpathcurveto{\pgfqpoint{2.998633in}{2.266377in}}{\pgfqpoint{2.995360in}{2.258477in}}{\pgfqpoint{2.995360in}{2.250241in}}%
\pgfpathcurveto{\pgfqpoint{2.995360in}{2.242005in}}{\pgfqpoint{2.998633in}{2.234104in}}{\pgfqpoint{3.004456in}{2.228281in}}%
\pgfpathcurveto{\pgfqpoint{3.010280in}{2.222457in}}{\pgfqpoint{3.018180in}{2.219184in}}{\pgfqpoint{3.026417in}{2.219184in}}%
\pgfpathclose%
\pgfusepath{stroke,fill}%
\end{pgfscope}%
\begin{pgfscope}%
\pgfpathrectangle{\pgfqpoint{0.100000in}{0.220728in}}{\pgfqpoint{3.696000in}{3.696000in}}%
\pgfusepath{clip}%
\pgfsetbuttcap%
\pgfsetroundjoin%
\definecolor{currentfill}{rgb}{0.121569,0.466667,0.705882}%
\pgfsetfillcolor{currentfill}%
\pgfsetfillopacity{0.798624}%
\pgfsetlinewidth{1.003750pt}%
\definecolor{currentstroke}{rgb}{0.121569,0.466667,0.705882}%
\pgfsetstrokecolor{currentstroke}%
\pgfsetstrokeopacity{0.798624}%
\pgfsetdash{}{0pt}%
\pgfpathmoveto{\pgfqpoint{3.025633in}{2.217789in}}%
\pgfpathcurveto{\pgfqpoint{3.033869in}{2.217789in}}{\pgfqpoint{3.041769in}{2.221062in}}{\pgfqpoint{3.047593in}{2.226886in}}%
\pgfpathcurveto{\pgfqpoint{3.053417in}{2.232710in}}{\pgfqpoint{3.056689in}{2.240610in}}{\pgfqpoint{3.056689in}{2.248846in}}%
\pgfpathcurveto{\pgfqpoint{3.056689in}{2.257082in}}{\pgfqpoint{3.053417in}{2.264982in}}{\pgfqpoint{3.047593in}{2.270806in}}%
\pgfpathcurveto{\pgfqpoint{3.041769in}{2.276630in}}{\pgfqpoint{3.033869in}{2.279902in}}{\pgfqpoint{3.025633in}{2.279902in}}%
\pgfpathcurveto{\pgfqpoint{3.017397in}{2.279902in}}{\pgfqpoint{3.009497in}{2.276630in}}{\pgfqpoint{3.003673in}{2.270806in}}%
\pgfpathcurveto{\pgfqpoint{2.997849in}{2.264982in}}{\pgfqpoint{2.994576in}{2.257082in}}{\pgfqpoint{2.994576in}{2.248846in}}%
\pgfpathcurveto{\pgfqpoint{2.994576in}{2.240610in}}{\pgfqpoint{2.997849in}{2.232710in}}{\pgfqpoint{3.003673in}{2.226886in}}%
\pgfpathcurveto{\pgfqpoint{3.009497in}{2.221062in}}{\pgfqpoint{3.017397in}{2.217789in}}{\pgfqpoint{3.025633in}{2.217789in}}%
\pgfpathclose%
\pgfusepath{stroke,fill}%
\end{pgfscope}%
\begin{pgfscope}%
\pgfpathrectangle{\pgfqpoint{0.100000in}{0.220728in}}{\pgfqpoint{3.696000in}{3.696000in}}%
\pgfusepath{clip}%
\pgfsetbuttcap%
\pgfsetroundjoin%
\definecolor{currentfill}{rgb}{0.121569,0.466667,0.705882}%
\pgfsetfillcolor{currentfill}%
\pgfsetfillopacity{0.798778}%
\pgfsetlinewidth{1.003750pt}%
\definecolor{currentstroke}{rgb}{0.121569,0.466667,0.705882}%
\pgfsetstrokecolor{currentstroke}%
\pgfsetstrokeopacity{0.798778}%
\pgfsetdash{}{0pt}%
\pgfpathmoveto{\pgfqpoint{3.025318in}{2.216932in}}%
\pgfpathcurveto{\pgfqpoint{3.033555in}{2.216932in}}{\pgfqpoint{3.041455in}{2.220204in}}{\pgfqpoint{3.047279in}{2.226028in}}%
\pgfpathcurveto{\pgfqpoint{3.053102in}{2.231852in}}{\pgfqpoint{3.056375in}{2.239752in}}{\pgfqpoint{3.056375in}{2.247988in}}%
\pgfpathcurveto{\pgfqpoint{3.056375in}{2.256225in}}{\pgfqpoint{3.053102in}{2.264125in}}{\pgfqpoint{3.047279in}{2.269949in}}%
\pgfpathcurveto{\pgfqpoint{3.041455in}{2.275773in}}{\pgfqpoint{3.033555in}{2.279045in}}{\pgfqpoint{3.025318in}{2.279045in}}%
\pgfpathcurveto{\pgfqpoint{3.017082in}{2.279045in}}{\pgfqpoint{3.009182in}{2.275773in}}{\pgfqpoint{3.003358in}{2.269949in}}%
\pgfpathcurveto{\pgfqpoint{2.997534in}{2.264125in}}{\pgfqpoint{2.994262in}{2.256225in}}{\pgfqpoint{2.994262in}{2.247988in}}%
\pgfpathcurveto{\pgfqpoint{2.994262in}{2.239752in}}{\pgfqpoint{2.997534in}{2.231852in}}{\pgfqpoint{3.003358in}{2.226028in}}%
\pgfpathcurveto{\pgfqpoint{3.009182in}{2.220204in}}{\pgfqpoint{3.017082in}{2.216932in}}{\pgfqpoint{3.025318in}{2.216932in}}%
\pgfpathclose%
\pgfusepath{stroke,fill}%
\end{pgfscope}%
\begin{pgfscope}%
\pgfpathrectangle{\pgfqpoint{0.100000in}{0.220728in}}{\pgfqpoint{3.696000in}{3.696000in}}%
\pgfusepath{clip}%
\pgfsetbuttcap%
\pgfsetroundjoin%
\definecolor{currentfill}{rgb}{0.121569,0.466667,0.705882}%
\pgfsetfillcolor{currentfill}%
\pgfsetfillopacity{0.799133}%
\pgfsetlinewidth{1.003750pt}%
\definecolor{currentstroke}{rgb}{0.121569,0.466667,0.705882}%
\pgfsetstrokecolor{currentstroke}%
\pgfsetstrokeopacity{0.799133}%
\pgfsetdash{}{0pt}%
\pgfpathmoveto{\pgfqpoint{3.024183in}{2.214929in}}%
\pgfpathcurveto{\pgfqpoint{3.032420in}{2.214929in}}{\pgfqpoint{3.040320in}{2.218202in}}{\pgfqpoint{3.046144in}{2.224025in}}%
\pgfpathcurveto{\pgfqpoint{3.051968in}{2.229849in}}{\pgfqpoint{3.055240in}{2.237749in}}{\pgfqpoint{3.055240in}{2.245986in}}%
\pgfpathcurveto{\pgfqpoint{3.055240in}{2.254222in}}{\pgfqpoint{3.051968in}{2.262122in}}{\pgfqpoint{3.046144in}{2.267946in}}%
\pgfpathcurveto{\pgfqpoint{3.040320in}{2.273770in}}{\pgfqpoint{3.032420in}{2.277042in}}{\pgfqpoint{3.024183in}{2.277042in}}%
\pgfpathcurveto{\pgfqpoint{3.015947in}{2.277042in}}{\pgfqpoint{3.008047in}{2.273770in}}{\pgfqpoint{3.002223in}{2.267946in}}%
\pgfpathcurveto{\pgfqpoint{2.996399in}{2.262122in}}{\pgfqpoint{2.993127in}{2.254222in}}{\pgfqpoint{2.993127in}{2.245986in}}%
\pgfpathcurveto{\pgfqpoint{2.993127in}{2.237749in}}{\pgfqpoint{2.996399in}{2.229849in}}{\pgfqpoint{3.002223in}{2.224025in}}%
\pgfpathcurveto{\pgfqpoint{3.008047in}{2.218202in}}{\pgfqpoint{3.015947in}{2.214929in}}{\pgfqpoint{3.024183in}{2.214929in}}%
\pgfpathclose%
\pgfusepath{stroke,fill}%
\end{pgfscope}%
\begin{pgfscope}%
\pgfpathrectangle{\pgfqpoint{0.100000in}{0.220728in}}{\pgfqpoint{3.696000in}{3.696000in}}%
\pgfusepath{clip}%
\pgfsetbuttcap%
\pgfsetroundjoin%
\definecolor{currentfill}{rgb}{0.121569,0.466667,0.705882}%
\pgfsetfillcolor{currentfill}%
\pgfsetfillopacity{0.799347}%
\pgfsetlinewidth{1.003750pt}%
\definecolor{currentstroke}{rgb}{0.121569,0.466667,0.705882}%
\pgfsetstrokecolor{currentstroke}%
\pgfsetstrokeopacity{0.799347}%
\pgfsetdash{}{0pt}%
\pgfpathmoveto{\pgfqpoint{3.023621in}{2.213823in}}%
\pgfpathcurveto{\pgfqpoint{3.031858in}{2.213823in}}{\pgfqpoint{3.039758in}{2.217096in}}{\pgfqpoint{3.045582in}{2.222920in}}%
\pgfpathcurveto{\pgfqpoint{3.051406in}{2.228744in}}{\pgfqpoint{3.054678in}{2.236644in}}{\pgfqpoint{3.054678in}{2.244880in}}%
\pgfpathcurveto{\pgfqpoint{3.054678in}{2.253116in}}{\pgfqpoint{3.051406in}{2.261016in}}{\pgfqpoint{3.045582in}{2.266840in}}%
\pgfpathcurveto{\pgfqpoint{3.039758in}{2.272664in}}{\pgfqpoint{3.031858in}{2.275936in}}{\pgfqpoint{3.023621in}{2.275936in}}%
\pgfpathcurveto{\pgfqpoint{3.015385in}{2.275936in}}{\pgfqpoint{3.007485in}{2.272664in}}{\pgfqpoint{3.001661in}{2.266840in}}%
\pgfpathcurveto{\pgfqpoint{2.995837in}{2.261016in}}{\pgfqpoint{2.992565in}{2.253116in}}{\pgfqpoint{2.992565in}{2.244880in}}%
\pgfpathcurveto{\pgfqpoint{2.992565in}{2.236644in}}{\pgfqpoint{2.995837in}{2.228744in}}{\pgfqpoint{3.001661in}{2.222920in}}%
\pgfpathcurveto{\pgfqpoint{3.007485in}{2.217096in}}{\pgfqpoint{3.015385in}{2.213823in}}{\pgfqpoint{3.023621in}{2.213823in}}%
\pgfpathclose%
\pgfusepath{stroke,fill}%
\end{pgfscope}%
\begin{pgfscope}%
\pgfpathrectangle{\pgfqpoint{0.100000in}{0.220728in}}{\pgfqpoint{3.696000in}{3.696000in}}%
\pgfusepath{clip}%
\pgfsetbuttcap%
\pgfsetroundjoin%
\definecolor{currentfill}{rgb}{0.121569,0.466667,0.705882}%
\pgfsetfillcolor{currentfill}%
\pgfsetfillopacity{0.799687}%
\pgfsetlinewidth{1.003750pt}%
\definecolor{currentstroke}{rgb}{0.121569,0.466667,0.705882}%
\pgfsetstrokecolor{currentstroke}%
\pgfsetstrokeopacity{0.799687}%
\pgfsetdash{}{0pt}%
\pgfpathmoveto{\pgfqpoint{3.022907in}{2.212041in}}%
\pgfpathcurveto{\pgfqpoint{3.031143in}{2.212041in}}{\pgfqpoint{3.039043in}{2.215313in}}{\pgfqpoint{3.044867in}{2.221137in}}%
\pgfpathcurveto{\pgfqpoint{3.050691in}{2.226961in}}{\pgfqpoint{3.053963in}{2.234861in}}{\pgfqpoint{3.053963in}{2.243097in}}%
\pgfpathcurveto{\pgfqpoint{3.053963in}{2.251333in}}{\pgfqpoint{3.050691in}{2.259233in}}{\pgfqpoint{3.044867in}{2.265057in}}%
\pgfpathcurveto{\pgfqpoint{3.039043in}{2.270881in}}{\pgfqpoint{3.031143in}{2.274154in}}{\pgfqpoint{3.022907in}{2.274154in}}%
\pgfpathcurveto{\pgfqpoint{3.014670in}{2.274154in}}{\pgfqpoint{3.006770in}{2.270881in}}{\pgfqpoint{3.000947in}{2.265057in}}%
\pgfpathcurveto{\pgfqpoint{2.995123in}{2.259233in}}{\pgfqpoint{2.991850in}{2.251333in}}{\pgfqpoint{2.991850in}{2.243097in}}%
\pgfpathcurveto{\pgfqpoint{2.991850in}{2.234861in}}{\pgfqpoint{2.995123in}{2.226961in}}{\pgfqpoint{3.000947in}{2.221137in}}%
\pgfpathcurveto{\pgfqpoint{3.006770in}{2.215313in}}{\pgfqpoint{3.014670in}{2.212041in}}{\pgfqpoint{3.022907in}{2.212041in}}%
\pgfpathclose%
\pgfusepath{stroke,fill}%
\end{pgfscope}%
\begin{pgfscope}%
\pgfpathrectangle{\pgfqpoint{0.100000in}{0.220728in}}{\pgfqpoint{3.696000in}{3.696000in}}%
\pgfusepath{clip}%
\pgfsetbuttcap%
\pgfsetroundjoin%
\definecolor{currentfill}{rgb}{0.121569,0.466667,0.705882}%
\pgfsetfillcolor{currentfill}%
\pgfsetfillopacity{0.800097}%
\pgfsetlinewidth{1.003750pt}%
\definecolor{currentstroke}{rgb}{0.121569,0.466667,0.705882}%
\pgfsetstrokecolor{currentstroke}%
\pgfsetstrokeopacity{0.800097}%
\pgfsetdash{}{0pt}%
\pgfpathmoveto{\pgfqpoint{3.021308in}{2.209212in}}%
\pgfpathcurveto{\pgfqpoint{3.029544in}{2.209212in}}{\pgfqpoint{3.037444in}{2.212484in}}{\pgfqpoint{3.043268in}{2.218308in}}%
\pgfpathcurveto{\pgfqpoint{3.049092in}{2.224132in}}{\pgfqpoint{3.052364in}{2.232032in}}{\pgfqpoint{3.052364in}{2.240268in}}%
\pgfpathcurveto{\pgfqpoint{3.052364in}{2.248505in}}{\pgfqpoint{3.049092in}{2.256405in}}{\pgfqpoint{3.043268in}{2.262229in}}%
\pgfpathcurveto{\pgfqpoint{3.037444in}{2.268053in}}{\pgfqpoint{3.029544in}{2.271325in}}{\pgfqpoint{3.021308in}{2.271325in}}%
\pgfpathcurveto{\pgfqpoint{3.013072in}{2.271325in}}{\pgfqpoint{3.005171in}{2.268053in}}{\pgfqpoint{2.999348in}{2.262229in}}%
\pgfpathcurveto{\pgfqpoint{2.993524in}{2.256405in}}{\pgfqpoint{2.990251in}{2.248505in}}{\pgfqpoint{2.990251in}{2.240268in}}%
\pgfpathcurveto{\pgfqpoint{2.990251in}{2.232032in}}{\pgfqpoint{2.993524in}{2.224132in}}{\pgfqpoint{2.999348in}{2.218308in}}%
\pgfpathcurveto{\pgfqpoint{3.005171in}{2.212484in}}{\pgfqpoint{3.013072in}{2.209212in}}{\pgfqpoint{3.021308in}{2.209212in}}%
\pgfpathclose%
\pgfusepath{stroke,fill}%
\end{pgfscope}%
\begin{pgfscope}%
\pgfpathrectangle{\pgfqpoint{0.100000in}{0.220728in}}{\pgfqpoint{3.696000in}{3.696000in}}%
\pgfusepath{clip}%
\pgfsetbuttcap%
\pgfsetroundjoin%
\definecolor{currentfill}{rgb}{0.121569,0.466667,0.705882}%
\pgfsetfillcolor{currentfill}%
\pgfsetfillopacity{0.800425}%
\pgfsetlinewidth{1.003750pt}%
\definecolor{currentstroke}{rgb}{0.121569,0.466667,0.705882}%
\pgfsetstrokecolor{currentstroke}%
\pgfsetstrokeopacity{0.800425}%
\pgfsetdash{}{0pt}%
\pgfpathmoveto{\pgfqpoint{3.020660in}{2.207798in}}%
\pgfpathcurveto{\pgfqpoint{3.028897in}{2.207798in}}{\pgfqpoint{3.036797in}{2.211070in}}{\pgfqpoint{3.042621in}{2.216894in}}%
\pgfpathcurveto{\pgfqpoint{3.048445in}{2.222718in}}{\pgfqpoint{3.051717in}{2.230618in}}{\pgfqpoint{3.051717in}{2.238855in}}%
\pgfpathcurveto{\pgfqpoint{3.051717in}{2.247091in}}{\pgfqpoint{3.048445in}{2.254991in}}{\pgfqpoint{3.042621in}{2.260815in}}%
\pgfpathcurveto{\pgfqpoint{3.036797in}{2.266639in}}{\pgfqpoint{3.028897in}{2.269911in}}{\pgfqpoint{3.020660in}{2.269911in}}%
\pgfpathcurveto{\pgfqpoint{3.012424in}{2.269911in}}{\pgfqpoint{3.004524in}{2.266639in}}{\pgfqpoint{2.998700in}{2.260815in}}%
\pgfpathcurveto{\pgfqpoint{2.992876in}{2.254991in}}{\pgfqpoint{2.989604in}{2.247091in}}{\pgfqpoint{2.989604in}{2.238855in}}%
\pgfpathcurveto{\pgfqpoint{2.989604in}{2.230618in}}{\pgfqpoint{2.992876in}{2.222718in}}{\pgfqpoint{2.998700in}{2.216894in}}%
\pgfpathcurveto{\pgfqpoint{3.004524in}{2.211070in}}{\pgfqpoint{3.012424in}{2.207798in}}{\pgfqpoint{3.020660in}{2.207798in}}%
\pgfpathclose%
\pgfusepath{stroke,fill}%
\end{pgfscope}%
\begin{pgfscope}%
\pgfpathrectangle{\pgfqpoint{0.100000in}{0.220728in}}{\pgfqpoint{3.696000in}{3.696000in}}%
\pgfusepath{clip}%
\pgfsetbuttcap%
\pgfsetroundjoin%
\definecolor{currentfill}{rgb}{0.121569,0.466667,0.705882}%
\pgfsetfillcolor{currentfill}%
\pgfsetfillopacity{0.800601}%
\pgfsetlinewidth{1.003750pt}%
\definecolor{currentstroke}{rgb}{0.121569,0.466667,0.705882}%
\pgfsetstrokecolor{currentstroke}%
\pgfsetstrokeopacity{0.800601}%
\pgfsetdash{}{0pt}%
\pgfpathmoveto{\pgfqpoint{3.020336in}{2.206972in}}%
\pgfpathcurveto{\pgfqpoint{3.028572in}{2.206972in}}{\pgfqpoint{3.036472in}{2.210244in}}{\pgfqpoint{3.042296in}{2.216068in}}%
\pgfpathcurveto{\pgfqpoint{3.048120in}{2.221892in}}{\pgfqpoint{3.051392in}{2.229792in}}{\pgfqpoint{3.051392in}{2.238028in}}%
\pgfpathcurveto{\pgfqpoint{3.051392in}{2.246265in}}{\pgfqpoint{3.048120in}{2.254165in}}{\pgfqpoint{3.042296in}{2.259989in}}%
\pgfpathcurveto{\pgfqpoint{3.036472in}{2.265813in}}{\pgfqpoint{3.028572in}{2.269085in}}{\pgfqpoint{3.020336in}{2.269085in}}%
\pgfpathcurveto{\pgfqpoint{3.012099in}{2.269085in}}{\pgfqpoint{3.004199in}{2.265813in}}{\pgfqpoint{2.998375in}{2.259989in}}%
\pgfpathcurveto{\pgfqpoint{2.992552in}{2.254165in}}{\pgfqpoint{2.989279in}{2.246265in}}{\pgfqpoint{2.989279in}{2.238028in}}%
\pgfpathcurveto{\pgfqpoint{2.989279in}{2.229792in}}{\pgfqpoint{2.992552in}{2.221892in}}{\pgfqpoint{2.998375in}{2.216068in}}%
\pgfpathcurveto{\pgfqpoint{3.004199in}{2.210244in}}{\pgfqpoint{3.012099in}{2.206972in}}{\pgfqpoint{3.020336in}{2.206972in}}%
\pgfpathclose%
\pgfusepath{stroke,fill}%
\end{pgfscope}%
\begin{pgfscope}%
\pgfpathrectangle{\pgfqpoint{0.100000in}{0.220728in}}{\pgfqpoint{3.696000in}{3.696000in}}%
\pgfusepath{clip}%
\pgfsetbuttcap%
\pgfsetroundjoin%
\definecolor{currentfill}{rgb}{0.121569,0.466667,0.705882}%
\pgfsetfillcolor{currentfill}%
\pgfsetfillopacity{0.800682}%
\pgfsetlinewidth{1.003750pt}%
\definecolor{currentstroke}{rgb}{0.121569,0.466667,0.705882}%
\pgfsetstrokecolor{currentstroke}%
\pgfsetstrokeopacity{0.800682}%
\pgfsetdash{}{0pt}%
\pgfpathmoveto{\pgfqpoint{3.020064in}{2.206565in}}%
\pgfpathcurveto{\pgfqpoint{3.028301in}{2.206565in}}{\pgfqpoint{3.036201in}{2.209837in}}{\pgfqpoint{3.042025in}{2.215661in}}%
\pgfpathcurveto{\pgfqpoint{3.047848in}{2.221485in}}{\pgfqpoint{3.051121in}{2.229385in}}{\pgfqpoint{3.051121in}{2.237621in}}%
\pgfpathcurveto{\pgfqpoint{3.051121in}{2.245857in}}{\pgfqpoint{3.047848in}{2.253757in}}{\pgfqpoint{3.042025in}{2.259581in}}%
\pgfpathcurveto{\pgfqpoint{3.036201in}{2.265405in}}{\pgfqpoint{3.028301in}{2.268678in}}{\pgfqpoint{3.020064in}{2.268678in}}%
\pgfpathcurveto{\pgfqpoint{3.011828in}{2.268678in}}{\pgfqpoint{3.003928in}{2.265405in}}{\pgfqpoint{2.998104in}{2.259581in}}%
\pgfpathcurveto{\pgfqpoint{2.992280in}{2.253757in}}{\pgfqpoint{2.989008in}{2.245857in}}{\pgfqpoint{2.989008in}{2.237621in}}%
\pgfpathcurveto{\pgfqpoint{2.989008in}{2.229385in}}{\pgfqpoint{2.992280in}{2.221485in}}{\pgfqpoint{2.998104in}{2.215661in}}%
\pgfpathcurveto{\pgfqpoint{3.003928in}{2.209837in}}{\pgfqpoint{3.011828in}{2.206565in}}{\pgfqpoint{3.020064in}{2.206565in}}%
\pgfpathclose%
\pgfusepath{stroke,fill}%
\end{pgfscope}%
\begin{pgfscope}%
\pgfpathrectangle{\pgfqpoint{0.100000in}{0.220728in}}{\pgfqpoint{3.696000in}{3.696000in}}%
\pgfusepath{clip}%
\pgfsetbuttcap%
\pgfsetroundjoin%
\definecolor{currentfill}{rgb}{0.121569,0.466667,0.705882}%
\pgfsetfillcolor{currentfill}%
\pgfsetfillopacity{0.801009}%
\pgfsetlinewidth{1.003750pt}%
\definecolor{currentstroke}{rgb}{0.121569,0.466667,0.705882}%
\pgfsetstrokecolor{currentstroke}%
\pgfsetstrokeopacity{0.801009}%
\pgfsetdash{}{0pt}%
\pgfpathmoveto{\pgfqpoint{3.019370in}{2.204773in}}%
\pgfpathcurveto{\pgfqpoint{3.027606in}{2.204773in}}{\pgfqpoint{3.035506in}{2.208045in}}{\pgfqpoint{3.041330in}{2.213869in}}%
\pgfpathcurveto{\pgfqpoint{3.047154in}{2.219693in}}{\pgfqpoint{3.050426in}{2.227593in}}{\pgfqpoint{3.050426in}{2.235829in}}%
\pgfpathcurveto{\pgfqpoint{3.050426in}{2.244066in}}{\pgfqpoint{3.047154in}{2.251966in}}{\pgfqpoint{3.041330in}{2.257790in}}%
\pgfpathcurveto{\pgfqpoint{3.035506in}{2.263613in}}{\pgfqpoint{3.027606in}{2.266886in}}{\pgfqpoint{3.019370in}{2.266886in}}%
\pgfpathcurveto{\pgfqpoint{3.011134in}{2.266886in}}{\pgfqpoint{3.003234in}{2.263613in}}{\pgfqpoint{2.997410in}{2.257790in}}%
\pgfpathcurveto{\pgfqpoint{2.991586in}{2.251966in}}{\pgfqpoint{2.988313in}{2.244066in}}{\pgfqpoint{2.988313in}{2.235829in}}%
\pgfpathcurveto{\pgfqpoint{2.988313in}{2.227593in}}{\pgfqpoint{2.991586in}{2.219693in}}{\pgfqpoint{2.997410in}{2.213869in}}%
\pgfpathcurveto{\pgfqpoint{3.003234in}{2.208045in}}{\pgfqpoint{3.011134in}{2.204773in}}{\pgfqpoint{3.019370in}{2.204773in}}%
\pgfpathclose%
\pgfusepath{stroke,fill}%
\end{pgfscope}%
\begin{pgfscope}%
\pgfpathrectangle{\pgfqpoint{0.100000in}{0.220728in}}{\pgfqpoint{3.696000in}{3.696000in}}%
\pgfusepath{clip}%
\pgfsetbuttcap%
\pgfsetroundjoin%
\definecolor{currentfill}{rgb}{0.121569,0.466667,0.705882}%
\pgfsetfillcolor{currentfill}%
\pgfsetfillopacity{0.801040}%
\pgfsetlinewidth{1.003750pt}%
\definecolor{currentstroke}{rgb}{0.121569,0.466667,0.705882}%
\pgfsetstrokecolor{currentstroke}%
\pgfsetstrokeopacity{0.801040}%
\pgfsetdash{}{0pt}%
\pgfpathmoveto{\pgfqpoint{1.337713in}{1.126926in}}%
\pgfpathcurveto{\pgfqpoint{1.345949in}{1.126926in}}{\pgfqpoint{1.353849in}{1.130199in}}{\pgfqpoint{1.359673in}{1.136023in}}%
\pgfpathcurveto{\pgfqpoint{1.365497in}{1.141847in}}{\pgfqpoint{1.368769in}{1.149747in}}{\pgfqpoint{1.368769in}{1.157983in}}%
\pgfpathcurveto{\pgfqpoint{1.368769in}{1.166219in}}{\pgfqpoint{1.365497in}{1.174119in}}{\pgfqpoint{1.359673in}{1.179943in}}%
\pgfpathcurveto{\pgfqpoint{1.353849in}{1.185767in}}{\pgfqpoint{1.345949in}{1.189039in}}{\pgfqpoint{1.337713in}{1.189039in}}%
\pgfpathcurveto{\pgfqpoint{1.329477in}{1.189039in}}{\pgfqpoint{1.321576in}{1.185767in}}{\pgfqpoint{1.315753in}{1.179943in}}%
\pgfpathcurveto{\pgfqpoint{1.309929in}{1.174119in}}{\pgfqpoint{1.306656in}{1.166219in}}{\pgfqpoint{1.306656in}{1.157983in}}%
\pgfpathcurveto{\pgfqpoint{1.306656in}{1.149747in}}{\pgfqpoint{1.309929in}{1.141847in}}{\pgfqpoint{1.315753in}{1.136023in}}%
\pgfpathcurveto{\pgfqpoint{1.321576in}{1.130199in}}{\pgfqpoint{1.329477in}{1.126926in}}{\pgfqpoint{1.337713in}{1.126926in}}%
\pgfpathclose%
\pgfusepath{stroke,fill}%
\end{pgfscope}%
\begin{pgfscope}%
\pgfpathrectangle{\pgfqpoint{0.100000in}{0.220728in}}{\pgfqpoint{3.696000in}{3.696000in}}%
\pgfusepath{clip}%
\pgfsetbuttcap%
\pgfsetroundjoin%
\definecolor{currentfill}{rgb}{0.121569,0.466667,0.705882}%
\pgfsetfillcolor{currentfill}%
\pgfsetfillopacity{0.801499}%
\pgfsetlinewidth{1.003750pt}%
\definecolor{currentstroke}{rgb}{0.121569,0.466667,0.705882}%
\pgfsetstrokecolor{currentstroke}%
\pgfsetstrokeopacity{0.801499}%
\pgfsetdash{}{0pt}%
\pgfpathmoveto{\pgfqpoint{3.018252in}{2.202492in}}%
\pgfpathcurveto{\pgfqpoint{3.026488in}{2.202492in}}{\pgfqpoint{3.034389in}{2.205765in}}{\pgfqpoint{3.040212in}{2.211589in}}%
\pgfpathcurveto{\pgfqpoint{3.046036in}{2.217413in}}{\pgfqpoint{3.049309in}{2.225313in}}{\pgfqpoint{3.049309in}{2.233549in}}%
\pgfpathcurveto{\pgfqpoint{3.049309in}{2.241785in}}{\pgfqpoint{3.046036in}{2.249685in}}{\pgfqpoint{3.040212in}{2.255509in}}%
\pgfpathcurveto{\pgfqpoint{3.034389in}{2.261333in}}{\pgfqpoint{3.026488in}{2.264605in}}{\pgfqpoint{3.018252in}{2.264605in}}%
\pgfpathcurveto{\pgfqpoint{3.010016in}{2.264605in}}{\pgfqpoint{3.002116in}{2.261333in}}{\pgfqpoint{2.996292in}{2.255509in}}%
\pgfpathcurveto{\pgfqpoint{2.990468in}{2.249685in}}{\pgfqpoint{2.987196in}{2.241785in}}{\pgfqpoint{2.987196in}{2.233549in}}%
\pgfpathcurveto{\pgfqpoint{2.987196in}{2.225313in}}{\pgfqpoint{2.990468in}{2.217413in}}{\pgfqpoint{2.996292in}{2.211589in}}%
\pgfpathcurveto{\pgfqpoint{3.002116in}{2.205765in}}{\pgfqpoint{3.010016in}{2.202492in}}{\pgfqpoint{3.018252in}{2.202492in}}%
\pgfpathclose%
\pgfusepath{stroke,fill}%
\end{pgfscope}%
\begin{pgfscope}%
\pgfpathrectangle{\pgfqpoint{0.100000in}{0.220728in}}{\pgfqpoint{3.696000in}{3.696000in}}%
\pgfusepath{clip}%
\pgfsetbuttcap%
\pgfsetroundjoin%
\definecolor{currentfill}{rgb}{0.121569,0.466667,0.705882}%
\pgfsetfillcolor{currentfill}%
\pgfsetfillopacity{0.802029}%
\pgfsetlinewidth{1.003750pt}%
\definecolor{currentstroke}{rgb}{0.121569,0.466667,0.705882}%
\pgfsetstrokecolor{currentstroke}%
\pgfsetstrokeopacity{0.802029}%
\pgfsetdash{}{0pt}%
\pgfpathmoveto{\pgfqpoint{3.016482in}{2.199814in}}%
\pgfpathcurveto{\pgfqpoint{3.024718in}{2.199814in}}{\pgfqpoint{3.032618in}{2.203086in}}{\pgfqpoint{3.038442in}{2.208910in}}%
\pgfpathcurveto{\pgfqpoint{3.044266in}{2.214734in}}{\pgfqpoint{3.047539in}{2.222634in}}{\pgfqpoint{3.047539in}{2.230870in}}%
\pgfpathcurveto{\pgfqpoint{3.047539in}{2.239107in}}{\pgfqpoint{3.044266in}{2.247007in}}{\pgfqpoint{3.038442in}{2.252831in}}%
\pgfpathcurveto{\pgfqpoint{3.032618in}{2.258655in}}{\pgfqpoint{3.024718in}{2.261927in}}{\pgfqpoint{3.016482in}{2.261927in}}%
\pgfpathcurveto{\pgfqpoint{3.008246in}{2.261927in}}{\pgfqpoint{3.000346in}{2.258655in}}{\pgfqpoint{2.994522in}{2.252831in}}%
\pgfpathcurveto{\pgfqpoint{2.988698in}{2.247007in}}{\pgfqpoint{2.985426in}{2.239107in}}{\pgfqpoint{2.985426in}{2.230870in}}%
\pgfpathcurveto{\pgfqpoint{2.985426in}{2.222634in}}{\pgfqpoint{2.988698in}{2.214734in}}{\pgfqpoint{2.994522in}{2.208910in}}%
\pgfpathcurveto{\pgfqpoint{3.000346in}{2.203086in}}{\pgfqpoint{3.008246in}{2.199814in}}{\pgfqpoint{3.016482in}{2.199814in}}%
\pgfpathclose%
\pgfusepath{stroke,fill}%
\end{pgfscope}%
\begin{pgfscope}%
\pgfpathrectangle{\pgfqpoint{0.100000in}{0.220728in}}{\pgfqpoint{3.696000in}{3.696000in}}%
\pgfusepath{clip}%
\pgfsetbuttcap%
\pgfsetroundjoin%
\definecolor{currentfill}{rgb}{0.121569,0.466667,0.705882}%
\pgfsetfillcolor{currentfill}%
\pgfsetfillopacity{0.802937}%
\pgfsetlinewidth{1.003750pt}%
\definecolor{currentstroke}{rgb}{0.121569,0.466667,0.705882}%
\pgfsetstrokecolor{currentstroke}%
\pgfsetstrokeopacity{0.802937}%
\pgfsetdash{}{0pt}%
\pgfpathmoveto{\pgfqpoint{3.014743in}{2.194561in}}%
\pgfpathcurveto{\pgfqpoint{3.022980in}{2.194561in}}{\pgfqpoint{3.030880in}{2.197833in}}{\pgfqpoint{3.036703in}{2.203657in}}%
\pgfpathcurveto{\pgfqpoint{3.042527in}{2.209481in}}{\pgfqpoint{3.045800in}{2.217381in}}{\pgfqpoint{3.045800in}{2.225617in}}%
\pgfpathcurveto{\pgfqpoint{3.045800in}{2.233853in}}{\pgfqpoint{3.042527in}{2.241753in}}{\pgfqpoint{3.036703in}{2.247577in}}%
\pgfpathcurveto{\pgfqpoint{3.030880in}{2.253401in}}{\pgfqpoint{3.022980in}{2.256674in}}{\pgfqpoint{3.014743in}{2.256674in}}%
\pgfpathcurveto{\pgfqpoint{3.006507in}{2.256674in}}{\pgfqpoint{2.998607in}{2.253401in}}{\pgfqpoint{2.992783in}{2.247577in}}%
\pgfpathcurveto{\pgfqpoint{2.986959in}{2.241753in}}{\pgfqpoint{2.983687in}{2.233853in}}{\pgfqpoint{2.983687in}{2.225617in}}%
\pgfpathcurveto{\pgfqpoint{2.983687in}{2.217381in}}{\pgfqpoint{2.986959in}{2.209481in}}{\pgfqpoint{2.992783in}{2.203657in}}%
\pgfpathcurveto{\pgfqpoint{2.998607in}{2.197833in}}{\pgfqpoint{3.006507in}{2.194561in}}{\pgfqpoint{3.014743in}{2.194561in}}%
\pgfpathclose%
\pgfusepath{stroke,fill}%
\end{pgfscope}%
\begin{pgfscope}%
\pgfpathrectangle{\pgfqpoint{0.100000in}{0.220728in}}{\pgfqpoint{3.696000in}{3.696000in}}%
\pgfusepath{clip}%
\pgfsetbuttcap%
\pgfsetroundjoin%
\definecolor{currentfill}{rgb}{0.121569,0.466667,0.705882}%
\pgfsetfillcolor{currentfill}%
\pgfsetfillopacity{0.803684}%
\pgfsetlinewidth{1.003750pt}%
\definecolor{currentstroke}{rgb}{0.121569,0.466667,0.705882}%
\pgfsetstrokecolor{currentstroke}%
\pgfsetstrokeopacity{0.803684}%
\pgfsetdash{}{0pt}%
\pgfpathmoveto{\pgfqpoint{1.354627in}{1.118463in}}%
\pgfpathcurveto{\pgfqpoint{1.362864in}{1.118463in}}{\pgfqpoint{1.370764in}{1.121735in}}{\pgfqpoint{1.376588in}{1.127559in}}%
\pgfpathcurveto{\pgfqpoint{1.382412in}{1.133383in}}{\pgfqpoint{1.385684in}{1.141283in}}{\pgfqpoint{1.385684in}{1.149520in}}%
\pgfpathcurveto{\pgfqpoint{1.385684in}{1.157756in}}{\pgfqpoint{1.382412in}{1.165656in}}{\pgfqpoint{1.376588in}{1.171480in}}%
\pgfpathcurveto{\pgfqpoint{1.370764in}{1.177304in}}{\pgfqpoint{1.362864in}{1.180576in}}{\pgfqpoint{1.354627in}{1.180576in}}%
\pgfpathcurveto{\pgfqpoint{1.346391in}{1.180576in}}{\pgfqpoint{1.338491in}{1.177304in}}{\pgfqpoint{1.332667in}{1.171480in}}%
\pgfpathcurveto{\pgfqpoint{1.326843in}{1.165656in}}{\pgfqpoint{1.323571in}{1.157756in}}{\pgfqpoint{1.323571in}{1.149520in}}%
\pgfpathcurveto{\pgfqpoint{1.323571in}{1.141283in}}{\pgfqpoint{1.326843in}{1.133383in}}{\pgfqpoint{1.332667in}{1.127559in}}%
\pgfpathcurveto{\pgfqpoint{1.338491in}{1.121735in}}{\pgfqpoint{1.346391in}{1.118463in}}{\pgfqpoint{1.354627in}{1.118463in}}%
\pgfpathclose%
\pgfusepath{stroke,fill}%
\end{pgfscope}%
\begin{pgfscope}%
\pgfpathrectangle{\pgfqpoint{0.100000in}{0.220728in}}{\pgfqpoint{3.696000in}{3.696000in}}%
\pgfusepath{clip}%
\pgfsetbuttcap%
\pgfsetroundjoin%
\definecolor{currentfill}{rgb}{0.121569,0.466667,0.705882}%
\pgfsetfillcolor{currentfill}%
\pgfsetfillopacity{0.804055}%
\pgfsetlinewidth{1.003750pt}%
\definecolor{currentstroke}{rgb}{0.121569,0.466667,0.705882}%
\pgfsetstrokecolor{currentstroke}%
\pgfsetstrokeopacity{0.804055}%
\pgfsetdash{}{0pt}%
\pgfpathmoveto{\pgfqpoint{3.012186in}{2.188975in}}%
\pgfpathcurveto{\pgfqpoint{3.020422in}{2.188975in}}{\pgfqpoint{3.028322in}{2.192247in}}{\pgfqpoint{3.034146in}{2.198071in}}%
\pgfpathcurveto{\pgfqpoint{3.039970in}{2.203895in}}{\pgfqpoint{3.043242in}{2.211795in}}{\pgfqpoint{3.043242in}{2.220031in}}%
\pgfpathcurveto{\pgfqpoint{3.043242in}{2.228267in}}{\pgfqpoint{3.039970in}{2.236167in}}{\pgfqpoint{3.034146in}{2.241991in}}%
\pgfpathcurveto{\pgfqpoint{3.028322in}{2.247815in}}{\pgfqpoint{3.020422in}{2.251088in}}{\pgfqpoint{3.012186in}{2.251088in}}%
\pgfpathcurveto{\pgfqpoint{3.003949in}{2.251088in}}{\pgfqpoint{2.996049in}{2.247815in}}{\pgfqpoint{2.990225in}{2.241991in}}%
\pgfpathcurveto{\pgfqpoint{2.984401in}{2.236167in}}{\pgfqpoint{2.981129in}{2.228267in}}{\pgfqpoint{2.981129in}{2.220031in}}%
\pgfpathcurveto{\pgfqpoint{2.981129in}{2.211795in}}{\pgfqpoint{2.984401in}{2.203895in}}{\pgfqpoint{2.990225in}{2.198071in}}%
\pgfpathcurveto{\pgfqpoint{2.996049in}{2.192247in}}{\pgfqpoint{3.003949in}{2.188975in}}{\pgfqpoint{3.012186in}{2.188975in}}%
\pgfpathclose%
\pgfusepath{stroke,fill}%
\end{pgfscope}%
\begin{pgfscope}%
\pgfpathrectangle{\pgfqpoint{0.100000in}{0.220728in}}{\pgfqpoint{3.696000in}{3.696000in}}%
\pgfusepath{clip}%
\pgfsetbuttcap%
\pgfsetroundjoin%
\definecolor{currentfill}{rgb}{0.121569,0.466667,0.705882}%
\pgfsetfillcolor{currentfill}%
\pgfsetfillopacity{0.805033}%
\pgfsetlinewidth{1.003750pt}%
\definecolor{currentstroke}{rgb}{0.121569,0.466667,0.705882}%
\pgfsetstrokecolor{currentstroke}%
\pgfsetstrokeopacity{0.805033}%
\pgfsetdash{}{0pt}%
\pgfpathmoveto{\pgfqpoint{3.008293in}{2.183157in}}%
\pgfpathcurveto{\pgfqpoint{3.016530in}{2.183157in}}{\pgfqpoint{3.024430in}{2.186429in}}{\pgfqpoint{3.030254in}{2.192253in}}%
\pgfpathcurveto{\pgfqpoint{3.036078in}{2.198077in}}{\pgfqpoint{3.039350in}{2.205977in}}{\pgfqpoint{3.039350in}{2.214213in}}%
\pgfpathcurveto{\pgfqpoint{3.039350in}{2.222449in}}{\pgfqpoint{3.036078in}{2.230350in}}{\pgfqpoint{3.030254in}{2.236173in}}%
\pgfpathcurveto{\pgfqpoint{3.024430in}{2.241997in}}{\pgfqpoint{3.016530in}{2.245270in}}{\pgfqpoint{3.008293in}{2.245270in}}%
\pgfpathcurveto{\pgfqpoint{3.000057in}{2.245270in}}{\pgfqpoint{2.992157in}{2.241997in}}{\pgfqpoint{2.986333in}{2.236173in}}%
\pgfpathcurveto{\pgfqpoint{2.980509in}{2.230350in}}{\pgfqpoint{2.977237in}{2.222449in}}{\pgfqpoint{2.977237in}{2.214213in}}%
\pgfpathcurveto{\pgfqpoint{2.977237in}{2.205977in}}{\pgfqpoint{2.980509in}{2.198077in}}{\pgfqpoint{2.986333in}{2.192253in}}%
\pgfpathcurveto{\pgfqpoint{2.992157in}{2.186429in}}{\pgfqpoint{3.000057in}{2.183157in}}{\pgfqpoint{3.008293in}{2.183157in}}%
\pgfpathclose%
\pgfusepath{stroke,fill}%
\end{pgfscope}%
\begin{pgfscope}%
\pgfpathrectangle{\pgfqpoint{0.100000in}{0.220728in}}{\pgfqpoint{3.696000in}{3.696000in}}%
\pgfusepath{clip}%
\pgfsetbuttcap%
\pgfsetroundjoin%
\definecolor{currentfill}{rgb}{0.121569,0.466667,0.705882}%
\pgfsetfillcolor{currentfill}%
\pgfsetfillopacity{0.806413}%
\pgfsetlinewidth{1.003750pt}%
\definecolor{currentstroke}{rgb}{0.121569,0.466667,0.705882}%
\pgfsetstrokecolor{currentstroke}%
\pgfsetstrokeopacity{0.806413}%
\pgfsetdash{}{0pt}%
\pgfpathmoveto{\pgfqpoint{3.005370in}{2.174578in}}%
\pgfpathcurveto{\pgfqpoint{3.013606in}{2.174578in}}{\pgfqpoint{3.021506in}{2.177850in}}{\pgfqpoint{3.027330in}{2.183674in}}%
\pgfpathcurveto{\pgfqpoint{3.033154in}{2.189498in}}{\pgfqpoint{3.036426in}{2.197398in}}{\pgfqpoint{3.036426in}{2.205635in}}%
\pgfpathcurveto{\pgfqpoint{3.036426in}{2.213871in}}{\pgfqpoint{3.033154in}{2.221771in}}{\pgfqpoint{3.027330in}{2.227595in}}%
\pgfpathcurveto{\pgfqpoint{3.021506in}{2.233419in}}{\pgfqpoint{3.013606in}{2.236691in}}{\pgfqpoint{3.005370in}{2.236691in}}%
\pgfpathcurveto{\pgfqpoint{2.997134in}{2.236691in}}{\pgfqpoint{2.989234in}{2.233419in}}{\pgfqpoint{2.983410in}{2.227595in}}%
\pgfpathcurveto{\pgfqpoint{2.977586in}{2.221771in}}{\pgfqpoint{2.974313in}{2.213871in}}{\pgfqpoint{2.974313in}{2.205635in}}%
\pgfpathcurveto{\pgfqpoint{2.974313in}{2.197398in}}{\pgfqpoint{2.977586in}{2.189498in}}{\pgfqpoint{2.983410in}{2.183674in}}%
\pgfpathcurveto{\pgfqpoint{2.989234in}{2.177850in}}{\pgfqpoint{2.997134in}{2.174578in}}{\pgfqpoint{3.005370in}{2.174578in}}%
\pgfpathclose%
\pgfusepath{stroke,fill}%
\end{pgfscope}%
\begin{pgfscope}%
\pgfpathrectangle{\pgfqpoint{0.100000in}{0.220728in}}{\pgfqpoint{3.696000in}{3.696000in}}%
\pgfusepath{clip}%
\pgfsetbuttcap%
\pgfsetroundjoin%
\definecolor{currentfill}{rgb}{0.121569,0.466667,0.705882}%
\pgfsetfillcolor{currentfill}%
\pgfsetfillopacity{0.806921}%
\pgfsetlinewidth{1.003750pt}%
\definecolor{currentstroke}{rgb}{0.121569,0.466667,0.705882}%
\pgfsetstrokecolor{currentstroke}%
\pgfsetstrokeopacity{0.806921}%
\pgfsetdash{}{0pt}%
\pgfpathmoveto{\pgfqpoint{1.369270in}{1.115552in}}%
\pgfpathcurveto{\pgfqpoint{1.377506in}{1.115552in}}{\pgfqpoint{1.385406in}{1.118824in}}{\pgfqpoint{1.391230in}{1.124648in}}%
\pgfpathcurveto{\pgfqpoint{1.397054in}{1.130472in}}{\pgfqpoint{1.400326in}{1.138372in}}{\pgfqpoint{1.400326in}{1.146608in}}%
\pgfpathcurveto{\pgfqpoint{1.400326in}{1.154844in}}{\pgfqpoint{1.397054in}{1.162744in}}{\pgfqpoint{1.391230in}{1.168568in}}%
\pgfpathcurveto{\pgfqpoint{1.385406in}{1.174392in}}{\pgfqpoint{1.377506in}{1.177665in}}{\pgfqpoint{1.369270in}{1.177665in}}%
\pgfpathcurveto{\pgfqpoint{1.361033in}{1.177665in}}{\pgfqpoint{1.353133in}{1.174392in}}{\pgfqpoint{1.347309in}{1.168568in}}%
\pgfpathcurveto{\pgfqpoint{1.341486in}{1.162744in}}{\pgfqpoint{1.338213in}{1.154844in}}{\pgfqpoint{1.338213in}{1.146608in}}%
\pgfpathcurveto{\pgfqpoint{1.338213in}{1.138372in}}{\pgfqpoint{1.341486in}{1.130472in}}{\pgfqpoint{1.347309in}{1.124648in}}%
\pgfpathcurveto{\pgfqpoint{1.353133in}{1.118824in}}{\pgfqpoint{1.361033in}{1.115552in}}{\pgfqpoint{1.369270in}{1.115552in}}%
\pgfpathclose%
\pgfusepath{stroke,fill}%
\end{pgfscope}%
\begin{pgfscope}%
\pgfpathrectangle{\pgfqpoint{0.100000in}{0.220728in}}{\pgfqpoint{3.696000in}{3.696000in}}%
\pgfusepath{clip}%
\pgfsetbuttcap%
\pgfsetroundjoin%
\definecolor{currentfill}{rgb}{0.121569,0.466667,0.705882}%
\pgfsetfillcolor{currentfill}%
\pgfsetfillopacity{0.807144}%
\pgfsetlinewidth{1.003750pt}%
\definecolor{currentstroke}{rgb}{0.121569,0.466667,0.705882}%
\pgfsetstrokecolor{currentstroke}%
\pgfsetstrokeopacity{0.807144}%
\pgfsetdash{}{0pt}%
\pgfpathmoveto{\pgfqpoint{3.003225in}{2.170291in}}%
\pgfpathcurveto{\pgfqpoint{3.011461in}{2.170291in}}{\pgfqpoint{3.019361in}{2.173564in}}{\pgfqpoint{3.025185in}{2.179388in}}%
\pgfpathcurveto{\pgfqpoint{3.031009in}{2.185211in}}{\pgfqpoint{3.034281in}{2.193112in}}{\pgfqpoint{3.034281in}{2.201348in}}%
\pgfpathcurveto{\pgfqpoint{3.034281in}{2.209584in}}{\pgfqpoint{3.031009in}{2.217484in}}{\pgfqpoint{3.025185in}{2.223308in}}%
\pgfpathcurveto{\pgfqpoint{3.019361in}{2.229132in}}{\pgfqpoint{3.011461in}{2.232404in}}{\pgfqpoint{3.003225in}{2.232404in}}%
\pgfpathcurveto{\pgfqpoint{2.994988in}{2.232404in}}{\pgfqpoint{2.987088in}{2.229132in}}{\pgfqpoint{2.981264in}{2.223308in}}%
\pgfpathcurveto{\pgfqpoint{2.975441in}{2.217484in}}{\pgfqpoint{2.972168in}{2.209584in}}{\pgfqpoint{2.972168in}{2.201348in}}%
\pgfpathcurveto{\pgfqpoint{2.972168in}{2.193112in}}{\pgfqpoint{2.975441in}{2.185211in}}{\pgfqpoint{2.981264in}{2.179388in}}%
\pgfpathcurveto{\pgfqpoint{2.987088in}{2.173564in}}{\pgfqpoint{2.994988in}{2.170291in}}{\pgfqpoint{3.003225in}{2.170291in}}%
\pgfpathclose%
\pgfusepath{stroke,fill}%
\end{pgfscope}%
\begin{pgfscope}%
\pgfpathrectangle{\pgfqpoint{0.100000in}{0.220728in}}{\pgfqpoint{3.696000in}{3.696000in}}%
\pgfusepath{clip}%
\pgfsetbuttcap%
\pgfsetroundjoin%
\definecolor{currentfill}{rgb}{0.121569,0.466667,0.705882}%
\pgfsetfillcolor{currentfill}%
\pgfsetfillopacity{0.807600}%
\pgfsetlinewidth{1.003750pt}%
\definecolor{currentstroke}{rgb}{0.121569,0.466667,0.705882}%
\pgfsetstrokecolor{currentstroke}%
\pgfsetstrokeopacity{0.807600}%
\pgfsetdash{}{0pt}%
\pgfpathmoveto{\pgfqpoint{3.001945in}{2.168292in}}%
\pgfpathcurveto{\pgfqpoint{3.010182in}{2.168292in}}{\pgfqpoint{3.018082in}{2.171565in}}{\pgfqpoint{3.023906in}{2.177389in}}%
\pgfpathcurveto{\pgfqpoint{3.029730in}{2.183213in}}{\pgfqpoint{3.033002in}{2.191113in}}{\pgfqpoint{3.033002in}{2.199349in}}%
\pgfpathcurveto{\pgfqpoint{3.033002in}{2.207585in}}{\pgfqpoint{3.029730in}{2.215485in}}{\pgfqpoint{3.023906in}{2.221309in}}%
\pgfpathcurveto{\pgfqpoint{3.018082in}{2.227133in}}{\pgfqpoint{3.010182in}{2.230405in}}{\pgfqpoint{3.001945in}{2.230405in}}%
\pgfpathcurveto{\pgfqpoint{2.993709in}{2.230405in}}{\pgfqpoint{2.985809in}{2.227133in}}{\pgfqpoint{2.979985in}{2.221309in}}%
\pgfpathcurveto{\pgfqpoint{2.974161in}{2.215485in}}{\pgfqpoint{2.970889in}{2.207585in}}{\pgfqpoint{2.970889in}{2.199349in}}%
\pgfpathcurveto{\pgfqpoint{2.970889in}{2.191113in}}{\pgfqpoint{2.974161in}{2.183213in}}{\pgfqpoint{2.979985in}{2.177389in}}%
\pgfpathcurveto{\pgfqpoint{2.985809in}{2.171565in}}{\pgfqpoint{2.993709in}{2.168292in}}{\pgfqpoint{3.001945in}{2.168292in}}%
\pgfpathclose%
\pgfusepath{stroke,fill}%
\end{pgfscope}%
\begin{pgfscope}%
\pgfpathrectangle{\pgfqpoint{0.100000in}{0.220728in}}{\pgfqpoint{3.696000in}{3.696000in}}%
\pgfusepath{clip}%
\pgfsetbuttcap%
\pgfsetroundjoin%
\definecolor{currentfill}{rgb}{0.121569,0.466667,0.705882}%
\pgfsetfillcolor{currentfill}%
\pgfsetfillopacity{0.807830}%
\pgfsetlinewidth{1.003750pt}%
\definecolor{currentstroke}{rgb}{0.121569,0.466667,0.705882}%
\pgfsetstrokecolor{currentstroke}%
\pgfsetstrokeopacity{0.807830}%
\pgfsetdash{}{0pt}%
\pgfpathmoveto{\pgfqpoint{3.001474in}{2.166851in}}%
\pgfpathcurveto{\pgfqpoint{3.009711in}{2.166851in}}{\pgfqpoint{3.017611in}{2.170123in}}{\pgfqpoint{3.023435in}{2.175947in}}%
\pgfpathcurveto{\pgfqpoint{3.029259in}{2.181771in}}{\pgfqpoint{3.032531in}{2.189671in}}{\pgfqpoint{3.032531in}{2.197907in}}%
\pgfpathcurveto{\pgfqpoint{3.032531in}{2.206143in}}{\pgfqpoint{3.029259in}{2.214043in}}{\pgfqpoint{3.023435in}{2.219867in}}%
\pgfpathcurveto{\pgfqpoint{3.017611in}{2.225691in}}{\pgfqpoint{3.009711in}{2.228964in}}{\pgfqpoint{3.001474in}{2.228964in}}%
\pgfpathcurveto{\pgfqpoint{2.993238in}{2.228964in}}{\pgfqpoint{2.985338in}{2.225691in}}{\pgfqpoint{2.979514in}{2.219867in}}%
\pgfpathcurveto{\pgfqpoint{2.973690in}{2.214043in}}{\pgfqpoint{2.970418in}{2.206143in}}{\pgfqpoint{2.970418in}{2.197907in}}%
\pgfpathcurveto{\pgfqpoint{2.970418in}{2.189671in}}{\pgfqpoint{2.973690in}{2.181771in}}{\pgfqpoint{2.979514in}{2.175947in}}%
\pgfpathcurveto{\pgfqpoint{2.985338in}{2.170123in}}{\pgfqpoint{2.993238in}{2.166851in}}{\pgfqpoint{3.001474in}{2.166851in}}%
\pgfpathclose%
\pgfusepath{stroke,fill}%
\end{pgfscope}%
\begin{pgfscope}%
\pgfpathrectangle{\pgfqpoint{0.100000in}{0.220728in}}{\pgfqpoint{3.696000in}{3.696000in}}%
\pgfusepath{clip}%
\pgfsetbuttcap%
\pgfsetroundjoin%
\definecolor{currentfill}{rgb}{0.121569,0.466667,0.705882}%
\pgfsetfillcolor{currentfill}%
\pgfsetfillopacity{0.808282}%
\pgfsetlinewidth{1.003750pt}%
\definecolor{currentstroke}{rgb}{0.121569,0.466667,0.705882}%
\pgfsetstrokecolor{currentstroke}%
\pgfsetstrokeopacity{0.808282}%
\pgfsetdash{}{0pt}%
\pgfpathmoveto{\pgfqpoint{3.000068in}{2.164379in}}%
\pgfpathcurveto{\pgfqpoint{3.008304in}{2.164379in}}{\pgfqpoint{3.016204in}{2.167651in}}{\pgfqpoint{3.022028in}{2.173475in}}%
\pgfpathcurveto{\pgfqpoint{3.027852in}{2.179299in}}{\pgfqpoint{3.031124in}{2.187199in}}{\pgfqpoint{3.031124in}{2.195435in}}%
\pgfpathcurveto{\pgfqpoint{3.031124in}{2.203672in}}{\pgfqpoint{3.027852in}{2.211572in}}{\pgfqpoint{3.022028in}{2.217396in}}%
\pgfpathcurveto{\pgfqpoint{3.016204in}{2.223220in}}{\pgfqpoint{3.008304in}{2.226492in}}{\pgfqpoint{3.000068in}{2.226492in}}%
\pgfpathcurveto{\pgfqpoint{2.991831in}{2.226492in}}{\pgfqpoint{2.983931in}{2.223220in}}{\pgfqpoint{2.978107in}{2.217396in}}%
\pgfpathcurveto{\pgfqpoint{2.972284in}{2.211572in}}{\pgfqpoint{2.969011in}{2.203672in}}{\pgfqpoint{2.969011in}{2.195435in}}%
\pgfpathcurveto{\pgfqpoint{2.969011in}{2.187199in}}{\pgfqpoint{2.972284in}{2.179299in}}{\pgfqpoint{2.978107in}{2.173475in}}%
\pgfpathcurveto{\pgfqpoint{2.983931in}{2.167651in}}{\pgfqpoint{2.991831in}{2.164379in}}{\pgfqpoint{3.000068in}{2.164379in}}%
\pgfpathclose%
\pgfusepath{stroke,fill}%
\end{pgfscope}%
\begin{pgfscope}%
\pgfpathrectangle{\pgfqpoint{0.100000in}{0.220728in}}{\pgfqpoint{3.696000in}{3.696000in}}%
\pgfusepath{clip}%
\pgfsetbuttcap%
\pgfsetroundjoin%
\definecolor{currentfill}{rgb}{0.121569,0.466667,0.705882}%
\pgfsetfillcolor{currentfill}%
\pgfsetfillopacity{0.808522}%
\pgfsetlinewidth{1.003750pt}%
\definecolor{currentstroke}{rgb}{0.121569,0.466667,0.705882}%
\pgfsetstrokecolor{currentstroke}%
\pgfsetstrokeopacity{0.808522}%
\pgfsetdash{}{0pt}%
\pgfpathmoveto{\pgfqpoint{2.999317in}{2.162950in}}%
\pgfpathcurveto{\pgfqpoint{3.007553in}{2.162950in}}{\pgfqpoint{3.015453in}{2.166222in}}{\pgfqpoint{3.021277in}{2.172046in}}%
\pgfpathcurveto{\pgfqpoint{3.027101in}{2.177870in}}{\pgfqpoint{3.030374in}{2.185770in}}{\pgfqpoint{3.030374in}{2.194006in}}%
\pgfpathcurveto{\pgfqpoint{3.030374in}{2.202243in}}{\pgfqpoint{3.027101in}{2.210143in}}{\pgfqpoint{3.021277in}{2.215967in}}%
\pgfpathcurveto{\pgfqpoint{3.015453in}{2.221790in}}{\pgfqpoint{3.007553in}{2.225063in}}{\pgfqpoint{2.999317in}{2.225063in}}%
\pgfpathcurveto{\pgfqpoint{2.991081in}{2.225063in}}{\pgfqpoint{2.983181in}{2.221790in}}{\pgfqpoint{2.977357in}{2.215967in}}%
\pgfpathcurveto{\pgfqpoint{2.971533in}{2.210143in}}{\pgfqpoint{2.968261in}{2.202243in}}{\pgfqpoint{2.968261in}{2.194006in}}%
\pgfpathcurveto{\pgfqpoint{2.968261in}{2.185770in}}{\pgfqpoint{2.971533in}{2.177870in}}{\pgfqpoint{2.977357in}{2.172046in}}%
\pgfpathcurveto{\pgfqpoint{2.983181in}{2.166222in}}{\pgfqpoint{2.991081in}{2.162950in}}{\pgfqpoint{2.999317in}{2.162950in}}%
\pgfpathclose%
\pgfusepath{stroke,fill}%
\end{pgfscope}%
\begin{pgfscope}%
\pgfpathrectangle{\pgfqpoint{0.100000in}{0.220728in}}{\pgfqpoint{3.696000in}{3.696000in}}%
\pgfusepath{clip}%
\pgfsetbuttcap%
\pgfsetroundjoin%
\definecolor{currentfill}{rgb}{0.121569,0.466667,0.705882}%
\pgfsetfillcolor{currentfill}%
\pgfsetfillopacity{0.808634}%
\pgfsetlinewidth{1.003750pt}%
\definecolor{currentstroke}{rgb}{0.121569,0.466667,0.705882}%
\pgfsetstrokecolor{currentstroke}%
\pgfsetstrokeopacity{0.808634}%
\pgfsetdash{}{0pt}%
\pgfpathmoveto{\pgfqpoint{1.381528in}{1.108854in}}%
\pgfpathcurveto{\pgfqpoint{1.389764in}{1.108854in}}{\pgfqpoint{1.397664in}{1.112127in}}{\pgfqpoint{1.403488in}{1.117951in}}%
\pgfpathcurveto{\pgfqpoint{1.409312in}{1.123775in}}{\pgfqpoint{1.412584in}{1.131675in}}{\pgfqpoint{1.412584in}{1.139911in}}%
\pgfpathcurveto{\pgfqpoint{1.412584in}{1.148147in}}{\pgfqpoint{1.409312in}{1.156047in}}{\pgfqpoint{1.403488in}{1.161871in}}%
\pgfpathcurveto{\pgfqpoint{1.397664in}{1.167695in}}{\pgfqpoint{1.389764in}{1.170967in}}{\pgfqpoint{1.381528in}{1.170967in}}%
\pgfpathcurveto{\pgfqpoint{1.373292in}{1.170967in}}{\pgfqpoint{1.365392in}{1.167695in}}{\pgfqpoint{1.359568in}{1.161871in}}%
\pgfpathcurveto{\pgfqpoint{1.353744in}{1.156047in}}{\pgfqpoint{1.350471in}{1.148147in}}{\pgfqpoint{1.350471in}{1.139911in}}%
\pgfpathcurveto{\pgfqpoint{1.350471in}{1.131675in}}{\pgfqpoint{1.353744in}{1.123775in}}{\pgfqpoint{1.359568in}{1.117951in}}%
\pgfpathcurveto{\pgfqpoint{1.365392in}{1.112127in}}{\pgfqpoint{1.373292in}{1.108854in}}{\pgfqpoint{1.381528in}{1.108854in}}%
\pgfpathclose%
\pgfusepath{stroke,fill}%
\end{pgfscope}%
\begin{pgfscope}%
\pgfpathrectangle{\pgfqpoint{0.100000in}{0.220728in}}{\pgfqpoint{3.696000in}{3.696000in}}%
\pgfusepath{clip}%
\pgfsetbuttcap%
\pgfsetroundjoin%
\definecolor{currentfill}{rgb}{0.121569,0.466667,0.705882}%
\pgfsetfillcolor{currentfill}%
\pgfsetfillopacity{0.808666}%
\pgfsetlinewidth{1.003750pt}%
\definecolor{currentstroke}{rgb}{0.121569,0.466667,0.705882}%
\pgfsetstrokecolor{currentstroke}%
\pgfsetstrokeopacity{0.808666}%
\pgfsetdash{}{0pt}%
\pgfpathmoveto{\pgfqpoint{2.998988in}{2.162122in}}%
\pgfpathcurveto{\pgfqpoint{3.007224in}{2.162122in}}{\pgfqpoint{3.015124in}{2.165395in}}{\pgfqpoint{3.020948in}{2.171219in}}%
\pgfpathcurveto{\pgfqpoint{3.026772in}{2.177043in}}{\pgfqpoint{3.030044in}{2.184943in}}{\pgfqpoint{3.030044in}{2.193179in}}%
\pgfpathcurveto{\pgfqpoint{3.030044in}{2.201415in}}{\pgfqpoint{3.026772in}{2.209315in}}{\pgfqpoint{3.020948in}{2.215139in}}%
\pgfpathcurveto{\pgfqpoint{3.015124in}{2.220963in}}{\pgfqpoint{3.007224in}{2.224235in}}{\pgfqpoint{2.998988in}{2.224235in}}%
\pgfpathcurveto{\pgfqpoint{2.990752in}{2.224235in}}{\pgfqpoint{2.982852in}{2.220963in}}{\pgfqpoint{2.977028in}{2.215139in}}%
\pgfpathcurveto{\pgfqpoint{2.971204in}{2.209315in}}{\pgfqpoint{2.967931in}{2.201415in}}{\pgfqpoint{2.967931in}{2.193179in}}%
\pgfpathcurveto{\pgfqpoint{2.967931in}{2.184943in}}{\pgfqpoint{2.971204in}{2.177043in}}{\pgfqpoint{2.977028in}{2.171219in}}%
\pgfpathcurveto{\pgfqpoint{2.982852in}{2.165395in}}{\pgfqpoint{2.990752in}{2.162122in}}{\pgfqpoint{2.998988in}{2.162122in}}%
\pgfpathclose%
\pgfusepath{stroke,fill}%
\end{pgfscope}%
\begin{pgfscope}%
\pgfpathrectangle{\pgfqpoint{0.100000in}{0.220728in}}{\pgfqpoint{3.696000in}{3.696000in}}%
\pgfusepath{clip}%
\pgfsetbuttcap%
\pgfsetroundjoin%
\definecolor{currentfill}{rgb}{0.121569,0.466667,0.705882}%
\pgfsetfillcolor{currentfill}%
\pgfsetfillopacity{0.809043}%
\pgfsetlinewidth{1.003750pt}%
\definecolor{currentstroke}{rgb}{0.121569,0.466667,0.705882}%
\pgfsetstrokecolor{currentstroke}%
\pgfsetstrokeopacity{0.809043}%
\pgfsetdash{}{0pt}%
\pgfpathmoveto{\pgfqpoint{2.997711in}{2.159865in}}%
\pgfpathcurveto{\pgfqpoint{3.005948in}{2.159865in}}{\pgfqpoint{3.013848in}{2.163137in}}{\pgfqpoint{3.019672in}{2.168961in}}%
\pgfpathcurveto{\pgfqpoint{3.025496in}{2.174785in}}{\pgfqpoint{3.028768in}{2.182685in}}{\pgfqpoint{3.028768in}{2.190921in}}%
\pgfpathcurveto{\pgfqpoint{3.028768in}{2.199158in}}{\pgfqpoint{3.025496in}{2.207058in}}{\pgfqpoint{3.019672in}{2.212882in}}%
\pgfpathcurveto{\pgfqpoint{3.013848in}{2.218706in}}{\pgfqpoint{3.005948in}{2.221978in}}{\pgfqpoint{2.997711in}{2.221978in}}%
\pgfpathcurveto{\pgfqpoint{2.989475in}{2.221978in}}{\pgfqpoint{2.981575in}{2.218706in}}{\pgfqpoint{2.975751in}{2.212882in}}%
\pgfpathcurveto{\pgfqpoint{2.969927in}{2.207058in}}{\pgfqpoint{2.966655in}{2.199158in}}{\pgfqpoint{2.966655in}{2.190921in}}%
\pgfpathcurveto{\pgfqpoint{2.966655in}{2.182685in}}{\pgfqpoint{2.969927in}{2.174785in}}{\pgfqpoint{2.975751in}{2.168961in}}%
\pgfpathcurveto{\pgfqpoint{2.981575in}{2.163137in}}{\pgfqpoint{2.989475in}{2.159865in}}{\pgfqpoint{2.997711in}{2.159865in}}%
\pgfpathclose%
\pgfusepath{stroke,fill}%
\end{pgfscope}%
\begin{pgfscope}%
\pgfpathrectangle{\pgfqpoint{0.100000in}{0.220728in}}{\pgfqpoint{3.696000in}{3.696000in}}%
\pgfusepath{clip}%
\pgfsetbuttcap%
\pgfsetroundjoin%
\definecolor{currentfill}{rgb}{0.121569,0.466667,0.705882}%
\pgfsetfillcolor{currentfill}%
\pgfsetfillopacity{0.809751}%
\pgfsetlinewidth{1.003750pt}%
\definecolor{currentstroke}{rgb}{0.121569,0.466667,0.705882}%
\pgfsetstrokecolor{currentstroke}%
\pgfsetstrokeopacity{0.809751}%
\pgfsetdash{}{0pt}%
\pgfpathmoveto{\pgfqpoint{2.996343in}{2.155530in}}%
\pgfpathcurveto{\pgfqpoint{3.004580in}{2.155530in}}{\pgfqpoint{3.012480in}{2.158803in}}{\pgfqpoint{3.018304in}{2.164627in}}%
\pgfpathcurveto{\pgfqpoint{3.024128in}{2.170451in}}{\pgfqpoint{3.027400in}{2.178351in}}{\pgfqpoint{3.027400in}{2.186587in}}%
\pgfpathcurveto{\pgfqpoint{3.027400in}{2.194823in}}{\pgfqpoint{3.024128in}{2.202723in}}{\pgfqpoint{3.018304in}{2.208547in}}%
\pgfpathcurveto{\pgfqpoint{3.012480in}{2.214371in}}{\pgfqpoint{3.004580in}{2.217643in}}{\pgfqpoint{2.996343in}{2.217643in}}%
\pgfpathcurveto{\pgfqpoint{2.988107in}{2.217643in}}{\pgfqpoint{2.980207in}{2.214371in}}{\pgfqpoint{2.974383in}{2.208547in}}%
\pgfpathcurveto{\pgfqpoint{2.968559in}{2.202723in}}{\pgfqpoint{2.965287in}{2.194823in}}{\pgfqpoint{2.965287in}{2.186587in}}%
\pgfpathcurveto{\pgfqpoint{2.965287in}{2.178351in}}{\pgfqpoint{2.968559in}{2.170451in}}{\pgfqpoint{2.974383in}{2.164627in}}%
\pgfpathcurveto{\pgfqpoint{2.980207in}{2.158803in}}{\pgfqpoint{2.988107in}{2.155530in}}{\pgfqpoint{2.996343in}{2.155530in}}%
\pgfpathclose%
\pgfusepath{stroke,fill}%
\end{pgfscope}%
\begin{pgfscope}%
\pgfpathrectangle{\pgfqpoint{0.100000in}{0.220728in}}{\pgfqpoint{3.696000in}{3.696000in}}%
\pgfusepath{clip}%
\pgfsetbuttcap%
\pgfsetroundjoin%
\definecolor{currentfill}{rgb}{0.121569,0.466667,0.705882}%
\pgfsetfillcolor{currentfill}%
\pgfsetfillopacity{0.810339}%
\pgfsetlinewidth{1.003750pt}%
\definecolor{currentstroke}{rgb}{0.121569,0.466667,0.705882}%
\pgfsetstrokecolor{currentstroke}%
\pgfsetstrokeopacity{0.810339}%
\pgfsetdash{}{0pt}%
\pgfpathmoveto{\pgfqpoint{1.393316in}{1.104920in}}%
\pgfpathcurveto{\pgfqpoint{1.401552in}{1.104920in}}{\pgfqpoint{1.409452in}{1.108193in}}{\pgfqpoint{1.415276in}{1.114017in}}%
\pgfpathcurveto{\pgfqpoint{1.421100in}{1.119841in}}{\pgfqpoint{1.424372in}{1.127741in}}{\pgfqpoint{1.424372in}{1.135977in}}%
\pgfpathcurveto{\pgfqpoint{1.424372in}{1.144213in}}{\pgfqpoint{1.421100in}{1.152113in}}{\pgfqpoint{1.415276in}{1.157937in}}%
\pgfpathcurveto{\pgfqpoint{1.409452in}{1.163761in}}{\pgfqpoint{1.401552in}{1.167033in}}{\pgfqpoint{1.393316in}{1.167033in}}%
\pgfpathcurveto{\pgfqpoint{1.385079in}{1.167033in}}{\pgfqpoint{1.377179in}{1.163761in}}{\pgfqpoint{1.371355in}{1.157937in}}%
\pgfpathcurveto{\pgfqpoint{1.365531in}{1.152113in}}{\pgfqpoint{1.362259in}{1.144213in}}{\pgfqpoint{1.362259in}{1.135977in}}%
\pgfpathcurveto{\pgfqpoint{1.362259in}{1.127741in}}{\pgfqpoint{1.365531in}{1.119841in}}{\pgfqpoint{1.371355in}{1.114017in}}%
\pgfpathcurveto{\pgfqpoint{1.377179in}{1.108193in}}{\pgfqpoint{1.385079in}{1.104920in}}{\pgfqpoint{1.393316in}{1.104920in}}%
\pgfpathclose%
\pgfusepath{stroke,fill}%
\end{pgfscope}%
\begin{pgfscope}%
\pgfpathrectangle{\pgfqpoint{0.100000in}{0.220728in}}{\pgfqpoint{3.696000in}{3.696000in}}%
\pgfusepath{clip}%
\pgfsetbuttcap%
\pgfsetroundjoin%
\definecolor{currentfill}{rgb}{0.121569,0.466667,0.705882}%
\pgfsetfillcolor{currentfill}%
\pgfsetfillopacity{0.810609}%
\pgfsetlinewidth{1.003750pt}%
\definecolor{currentstroke}{rgb}{0.121569,0.466667,0.705882}%
\pgfsetstrokecolor{currentstroke}%
\pgfsetstrokeopacity{0.810609}%
\pgfsetdash{}{0pt}%
\pgfpathmoveto{\pgfqpoint{2.994032in}{2.150541in}}%
\pgfpathcurveto{\pgfqpoint{3.002268in}{2.150541in}}{\pgfqpoint{3.010168in}{2.153813in}}{\pgfqpoint{3.015992in}{2.159637in}}%
\pgfpathcurveto{\pgfqpoint{3.021816in}{2.165461in}}{\pgfqpoint{3.025088in}{2.173361in}}{\pgfqpoint{3.025088in}{2.181597in}}%
\pgfpathcurveto{\pgfqpoint{3.025088in}{2.189833in}}{\pgfqpoint{3.021816in}{2.197733in}}{\pgfqpoint{3.015992in}{2.203557in}}%
\pgfpathcurveto{\pgfqpoint{3.010168in}{2.209381in}}{\pgfqpoint{3.002268in}{2.212654in}}{\pgfqpoint{2.994032in}{2.212654in}}%
\pgfpathcurveto{\pgfqpoint{2.985795in}{2.212654in}}{\pgfqpoint{2.977895in}{2.209381in}}{\pgfqpoint{2.972071in}{2.203557in}}%
\pgfpathcurveto{\pgfqpoint{2.966247in}{2.197733in}}{\pgfqpoint{2.962975in}{2.189833in}}{\pgfqpoint{2.962975in}{2.181597in}}%
\pgfpathcurveto{\pgfqpoint{2.962975in}{2.173361in}}{\pgfqpoint{2.966247in}{2.165461in}}{\pgfqpoint{2.972071in}{2.159637in}}%
\pgfpathcurveto{\pgfqpoint{2.977895in}{2.153813in}}{\pgfqpoint{2.985795in}{2.150541in}}{\pgfqpoint{2.994032in}{2.150541in}}%
\pgfpathclose%
\pgfusepath{stroke,fill}%
\end{pgfscope}%
\begin{pgfscope}%
\pgfpathrectangle{\pgfqpoint{0.100000in}{0.220728in}}{\pgfqpoint{3.696000in}{3.696000in}}%
\pgfusepath{clip}%
\pgfsetbuttcap%
\pgfsetroundjoin%
\definecolor{currentfill}{rgb}{0.121569,0.466667,0.705882}%
\pgfsetfillcolor{currentfill}%
\pgfsetfillopacity{0.811593}%
\pgfsetlinewidth{1.003750pt}%
\definecolor{currentstroke}{rgb}{0.121569,0.466667,0.705882}%
\pgfsetstrokecolor{currentstroke}%
\pgfsetstrokeopacity{0.811593}%
\pgfsetdash{}{0pt}%
\pgfpathmoveto{\pgfqpoint{2.990970in}{2.145355in}}%
\pgfpathcurveto{\pgfqpoint{2.999207in}{2.145355in}}{\pgfqpoint{3.007107in}{2.148627in}}{\pgfqpoint{3.012931in}{2.154451in}}%
\pgfpathcurveto{\pgfqpoint{3.018754in}{2.160275in}}{\pgfqpoint{3.022027in}{2.168175in}}{\pgfqpoint{3.022027in}{2.176411in}}%
\pgfpathcurveto{\pgfqpoint{3.022027in}{2.184648in}}{\pgfqpoint{3.018754in}{2.192548in}}{\pgfqpoint{3.012931in}{2.198372in}}%
\pgfpathcurveto{\pgfqpoint{3.007107in}{2.204196in}}{\pgfqpoint{2.999207in}{2.207468in}}{\pgfqpoint{2.990970in}{2.207468in}}%
\pgfpathcurveto{\pgfqpoint{2.982734in}{2.207468in}}{\pgfqpoint{2.974834in}{2.204196in}}{\pgfqpoint{2.969010in}{2.198372in}}%
\pgfpathcurveto{\pgfqpoint{2.963186in}{2.192548in}}{\pgfqpoint{2.959914in}{2.184648in}}{\pgfqpoint{2.959914in}{2.176411in}}%
\pgfpathcurveto{\pgfqpoint{2.959914in}{2.168175in}}{\pgfqpoint{2.963186in}{2.160275in}}{\pgfqpoint{2.969010in}{2.154451in}}%
\pgfpathcurveto{\pgfqpoint{2.974834in}{2.148627in}}{\pgfqpoint{2.982734in}{2.145355in}}{\pgfqpoint{2.990970in}{2.145355in}}%
\pgfpathclose%
\pgfusepath{stroke,fill}%
\end{pgfscope}%
\begin{pgfscope}%
\pgfpathrectangle{\pgfqpoint{0.100000in}{0.220728in}}{\pgfqpoint{3.696000in}{3.696000in}}%
\pgfusepath{clip}%
\pgfsetbuttcap%
\pgfsetroundjoin%
\definecolor{currentfill}{rgb}{0.121569,0.466667,0.705882}%
\pgfsetfillcolor{currentfill}%
\pgfsetfillopacity{0.812795}%
\pgfsetlinewidth{1.003750pt}%
\definecolor{currentstroke}{rgb}{0.121569,0.466667,0.705882}%
\pgfsetstrokecolor{currentstroke}%
\pgfsetstrokeopacity{0.812795}%
\pgfsetdash{}{0pt}%
\pgfpathmoveto{\pgfqpoint{1.403786in}{1.101441in}}%
\pgfpathcurveto{\pgfqpoint{1.412022in}{1.101441in}}{\pgfqpoint{1.419922in}{1.104713in}}{\pgfqpoint{1.425746in}{1.110537in}}%
\pgfpathcurveto{\pgfqpoint{1.431570in}{1.116361in}}{\pgfqpoint{1.434842in}{1.124261in}}{\pgfqpoint{1.434842in}{1.132497in}}%
\pgfpathcurveto{\pgfqpoint{1.434842in}{1.140733in}}{\pgfqpoint{1.431570in}{1.148634in}}{\pgfqpoint{1.425746in}{1.154457in}}%
\pgfpathcurveto{\pgfqpoint{1.419922in}{1.160281in}}{\pgfqpoint{1.412022in}{1.163554in}}{\pgfqpoint{1.403786in}{1.163554in}}%
\pgfpathcurveto{\pgfqpoint{1.395550in}{1.163554in}}{\pgfqpoint{1.387650in}{1.160281in}}{\pgfqpoint{1.381826in}{1.154457in}}%
\pgfpathcurveto{\pgfqpoint{1.376002in}{1.148634in}}{\pgfqpoint{1.372729in}{1.140733in}}{\pgfqpoint{1.372729in}{1.132497in}}%
\pgfpathcurveto{\pgfqpoint{1.372729in}{1.124261in}}{\pgfqpoint{1.376002in}{1.116361in}}{\pgfqpoint{1.381826in}{1.110537in}}%
\pgfpathcurveto{\pgfqpoint{1.387650in}{1.104713in}}{\pgfqpoint{1.395550in}{1.101441in}}{\pgfqpoint{1.403786in}{1.101441in}}%
\pgfpathclose%
\pgfusepath{stroke,fill}%
\end{pgfscope}%
\begin{pgfscope}%
\pgfpathrectangle{\pgfqpoint{0.100000in}{0.220728in}}{\pgfqpoint{3.696000in}{3.696000in}}%
\pgfusepath{clip}%
\pgfsetbuttcap%
\pgfsetroundjoin%
\definecolor{currentfill}{rgb}{0.121569,0.466667,0.705882}%
\pgfsetfillcolor{currentfill}%
\pgfsetfillopacity{0.813103}%
\pgfsetlinewidth{1.003750pt}%
\definecolor{currentstroke}{rgb}{0.121569,0.466667,0.705882}%
\pgfsetstrokecolor{currentstroke}%
\pgfsetstrokeopacity{0.813103}%
\pgfsetdash{}{0pt}%
\pgfpathmoveto{\pgfqpoint{2.988474in}{2.136625in}}%
\pgfpathcurveto{\pgfqpoint{2.996711in}{2.136625in}}{\pgfqpoint{3.004611in}{2.139897in}}{\pgfqpoint{3.010435in}{2.145721in}}%
\pgfpathcurveto{\pgfqpoint{3.016259in}{2.151545in}}{\pgfqpoint{3.019531in}{2.159445in}}{\pgfqpoint{3.019531in}{2.167682in}}%
\pgfpathcurveto{\pgfqpoint{3.019531in}{2.175918in}}{\pgfqpoint{3.016259in}{2.183818in}}{\pgfqpoint{3.010435in}{2.189642in}}%
\pgfpathcurveto{\pgfqpoint{3.004611in}{2.195466in}}{\pgfqpoint{2.996711in}{2.198738in}}{\pgfqpoint{2.988474in}{2.198738in}}%
\pgfpathcurveto{\pgfqpoint{2.980238in}{2.198738in}}{\pgfqpoint{2.972338in}{2.195466in}}{\pgfqpoint{2.966514in}{2.189642in}}%
\pgfpathcurveto{\pgfqpoint{2.960690in}{2.183818in}}{\pgfqpoint{2.957418in}{2.175918in}}{\pgfqpoint{2.957418in}{2.167682in}}%
\pgfpathcurveto{\pgfqpoint{2.957418in}{2.159445in}}{\pgfqpoint{2.960690in}{2.151545in}}{\pgfqpoint{2.966514in}{2.145721in}}%
\pgfpathcurveto{\pgfqpoint{2.972338in}{2.139897in}}{\pgfqpoint{2.980238in}{2.136625in}}{\pgfqpoint{2.988474in}{2.136625in}}%
\pgfpathclose%
\pgfusepath{stroke,fill}%
\end{pgfscope}%
\begin{pgfscope}%
\pgfpathrectangle{\pgfqpoint{0.100000in}{0.220728in}}{\pgfqpoint{3.696000in}{3.696000in}}%
\pgfusepath{clip}%
\pgfsetbuttcap%
\pgfsetroundjoin%
\definecolor{currentfill}{rgb}{0.121569,0.466667,0.705882}%
\pgfsetfillcolor{currentfill}%
\pgfsetfillopacity{0.813830}%
\pgfsetlinewidth{1.003750pt}%
\definecolor{currentstroke}{rgb}{0.121569,0.466667,0.705882}%
\pgfsetstrokecolor{currentstroke}%
\pgfsetstrokeopacity{0.813830}%
\pgfsetdash{}{0pt}%
\pgfpathmoveto{\pgfqpoint{2.986211in}{2.132238in}}%
\pgfpathcurveto{\pgfqpoint{2.994448in}{2.132238in}}{\pgfqpoint{3.002348in}{2.135510in}}{\pgfqpoint{3.008172in}{2.141334in}}%
\pgfpathcurveto{\pgfqpoint{3.013996in}{2.147158in}}{\pgfqpoint{3.017268in}{2.155058in}}{\pgfqpoint{3.017268in}{2.163294in}}%
\pgfpathcurveto{\pgfqpoint{3.017268in}{2.171531in}}{\pgfqpoint{3.013996in}{2.179431in}}{\pgfqpoint{3.008172in}{2.185255in}}%
\pgfpathcurveto{\pgfqpoint{3.002348in}{2.191079in}}{\pgfqpoint{2.994448in}{2.194351in}}{\pgfqpoint{2.986211in}{2.194351in}}%
\pgfpathcurveto{\pgfqpoint{2.977975in}{2.194351in}}{\pgfqpoint{2.970075in}{2.191079in}}{\pgfqpoint{2.964251in}{2.185255in}}%
\pgfpathcurveto{\pgfqpoint{2.958427in}{2.179431in}}{\pgfqpoint{2.955155in}{2.171531in}}{\pgfqpoint{2.955155in}{2.163294in}}%
\pgfpathcurveto{\pgfqpoint{2.955155in}{2.155058in}}{\pgfqpoint{2.958427in}{2.147158in}}{\pgfqpoint{2.964251in}{2.141334in}}%
\pgfpathcurveto{\pgfqpoint{2.970075in}{2.135510in}}{\pgfqpoint{2.977975in}{2.132238in}}{\pgfqpoint{2.986211in}{2.132238in}}%
\pgfpathclose%
\pgfusepath{stroke,fill}%
\end{pgfscope}%
\begin{pgfscope}%
\pgfpathrectangle{\pgfqpoint{0.100000in}{0.220728in}}{\pgfqpoint{3.696000in}{3.696000in}}%
\pgfusepath{clip}%
\pgfsetbuttcap%
\pgfsetroundjoin%
\definecolor{currentfill}{rgb}{0.121569,0.466667,0.705882}%
\pgfsetfillcolor{currentfill}%
\pgfsetfillopacity{0.814277}%
\pgfsetlinewidth{1.003750pt}%
\definecolor{currentstroke}{rgb}{0.121569,0.466667,0.705882}%
\pgfsetstrokecolor{currentstroke}%
\pgfsetstrokeopacity{0.814277}%
\pgfsetdash{}{0pt}%
\pgfpathmoveto{\pgfqpoint{2.984956in}{2.130037in}}%
\pgfpathcurveto{\pgfqpoint{2.993192in}{2.130037in}}{\pgfqpoint{3.001092in}{2.133310in}}{\pgfqpoint{3.006916in}{2.139134in}}%
\pgfpathcurveto{\pgfqpoint{3.012740in}{2.144958in}}{\pgfqpoint{3.016012in}{2.152858in}}{\pgfqpoint{3.016012in}{2.161094in}}%
\pgfpathcurveto{\pgfqpoint{3.016012in}{2.169330in}}{\pgfqpoint{3.012740in}{2.177230in}}{\pgfqpoint{3.006916in}{2.183054in}}%
\pgfpathcurveto{\pgfqpoint{3.001092in}{2.188878in}}{\pgfqpoint{2.993192in}{2.192150in}}{\pgfqpoint{2.984956in}{2.192150in}}%
\pgfpathcurveto{\pgfqpoint{2.976720in}{2.192150in}}{\pgfqpoint{2.968820in}{2.188878in}}{\pgfqpoint{2.962996in}{2.183054in}}%
\pgfpathcurveto{\pgfqpoint{2.957172in}{2.177230in}}{\pgfqpoint{2.953899in}{2.169330in}}{\pgfqpoint{2.953899in}{2.161094in}}%
\pgfpathcurveto{\pgfqpoint{2.953899in}{2.152858in}}{\pgfqpoint{2.957172in}{2.144958in}}{\pgfqpoint{2.962996in}{2.139134in}}%
\pgfpathcurveto{\pgfqpoint{2.968820in}{2.133310in}}{\pgfqpoint{2.976720in}{2.130037in}}{\pgfqpoint{2.984956in}{2.130037in}}%
\pgfpathclose%
\pgfusepath{stroke,fill}%
\end{pgfscope}%
\begin{pgfscope}%
\pgfpathrectangle{\pgfqpoint{0.100000in}{0.220728in}}{\pgfqpoint{3.696000in}{3.696000in}}%
\pgfusepath{clip}%
\pgfsetbuttcap%
\pgfsetroundjoin%
\definecolor{currentfill}{rgb}{0.121569,0.466667,0.705882}%
\pgfsetfillcolor{currentfill}%
\pgfsetfillopacity{0.814519}%
\pgfsetlinewidth{1.003750pt}%
\definecolor{currentstroke}{rgb}{0.121569,0.466667,0.705882}%
\pgfsetstrokecolor{currentstroke}%
\pgfsetstrokeopacity{0.814519}%
\pgfsetdash{}{0pt}%
\pgfpathmoveto{\pgfqpoint{2.984482in}{2.128589in}}%
\pgfpathcurveto{\pgfqpoint{2.992718in}{2.128589in}}{\pgfqpoint{3.000618in}{2.131861in}}{\pgfqpoint{3.006442in}{2.137685in}}%
\pgfpathcurveto{\pgfqpoint{3.012266in}{2.143509in}}{\pgfqpoint{3.015539in}{2.151409in}}{\pgfqpoint{3.015539in}{2.159645in}}%
\pgfpathcurveto{\pgfqpoint{3.015539in}{2.167881in}}{\pgfqpoint{3.012266in}{2.175781in}}{\pgfqpoint{3.006442in}{2.181605in}}%
\pgfpathcurveto{\pgfqpoint{3.000618in}{2.187429in}}{\pgfqpoint{2.992718in}{2.190702in}}{\pgfqpoint{2.984482in}{2.190702in}}%
\pgfpathcurveto{\pgfqpoint{2.976246in}{2.190702in}}{\pgfqpoint{2.968346in}{2.187429in}}{\pgfqpoint{2.962522in}{2.181605in}}%
\pgfpathcurveto{\pgfqpoint{2.956698in}{2.175781in}}{\pgfqpoint{2.953426in}{2.167881in}}{\pgfqpoint{2.953426in}{2.159645in}}%
\pgfpathcurveto{\pgfqpoint{2.953426in}{2.151409in}}{\pgfqpoint{2.956698in}{2.143509in}}{\pgfqpoint{2.962522in}{2.137685in}}%
\pgfpathcurveto{\pgfqpoint{2.968346in}{2.131861in}}{\pgfqpoint{2.976246in}{2.128589in}}{\pgfqpoint{2.984482in}{2.128589in}}%
\pgfpathclose%
\pgfusepath{stroke,fill}%
\end{pgfscope}%
\begin{pgfscope}%
\pgfpathrectangle{\pgfqpoint{0.100000in}{0.220728in}}{\pgfqpoint{3.696000in}{3.696000in}}%
\pgfusepath{clip}%
\pgfsetbuttcap%
\pgfsetroundjoin%
\definecolor{currentfill}{rgb}{0.121569,0.466667,0.705882}%
\pgfsetfillcolor{currentfill}%
\pgfsetfillopacity{0.814850}%
\pgfsetlinewidth{1.003750pt}%
\definecolor{currentstroke}{rgb}{0.121569,0.466667,0.705882}%
\pgfsetstrokecolor{currentstroke}%
\pgfsetstrokeopacity{0.814850}%
\pgfsetdash{}{0pt}%
\pgfpathmoveto{\pgfqpoint{1.412753in}{1.098856in}}%
\pgfpathcurveto{\pgfqpoint{1.420989in}{1.098856in}}{\pgfqpoint{1.428889in}{1.102128in}}{\pgfqpoint{1.434713in}{1.107952in}}%
\pgfpathcurveto{\pgfqpoint{1.440537in}{1.113776in}}{\pgfqpoint{1.443810in}{1.121676in}}{\pgfqpoint{1.443810in}{1.129913in}}%
\pgfpathcurveto{\pgfqpoint{1.443810in}{1.138149in}}{\pgfqpoint{1.440537in}{1.146049in}}{\pgfqpoint{1.434713in}{1.151873in}}%
\pgfpathcurveto{\pgfqpoint{1.428889in}{1.157697in}}{\pgfqpoint{1.420989in}{1.160969in}}{\pgfqpoint{1.412753in}{1.160969in}}%
\pgfpathcurveto{\pgfqpoint{1.404517in}{1.160969in}}{\pgfqpoint{1.396617in}{1.157697in}}{\pgfqpoint{1.390793in}{1.151873in}}%
\pgfpathcurveto{\pgfqpoint{1.384969in}{1.146049in}}{\pgfqpoint{1.381697in}{1.138149in}}{\pgfqpoint{1.381697in}{1.129913in}}%
\pgfpathcurveto{\pgfqpoint{1.381697in}{1.121676in}}{\pgfqpoint{1.384969in}{1.113776in}}{\pgfqpoint{1.390793in}{1.107952in}}%
\pgfpathcurveto{\pgfqpoint{1.396617in}{1.102128in}}{\pgfqpoint{1.404517in}{1.098856in}}{\pgfqpoint{1.412753in}{1.098856in}}%
\pgfpathclose%
\pgfusepath{stroke,fill}%
\end{pgfscope}%
\begin{pgfscope}%
\pgfpathrectangle{\pgfqpoint{0.100000in}{0.220728in}}{\pgfqpoint{3.696000in}{3.696000in}}%
\pgfusepath{clip}%
\pgfsetbuttcap%
\pgfsetroundjoin%
\definecolor{currentfill}{rgb}{0.121569,0.466667,0.705882}%
\pgfsetfillcolor{currentfill}%
\pgfsetfillopacity{0.815024}%
\pgfsetlinewidth{1.003750pt}%
\definecolor{currentstroke}{rgb}{0.121569,0.466667,0.705882}%
\pgfsetstrokecolor{currentstroke}%
\pgfsetstrokeopacity{0.815024}%
\pgfsetdash{}{0pt}%
\pgfpathmoveto{\pgfqpoint{2.982523in}{2.125120in}}%
\pgfpathcurveto{\pgfqpoint{2.990759in}{2.125120in}}{\pgfqpoint{2.998659in}{2.128392in}}{\pgfqpoint{3.004483in}{2.134216in}}%
\pgfpathcurveto{\pgfqpoint{3.010307in}{2.140040in}}{\pgfqpoint{3.013579in}{2.147940in}}{\pgfqpoint{3.013579in}{2.156176in}}%
\pgfpathcurveto{\pgfqpoint{3.013579in}{2.164413in}}{\pgfqpoint{3.010307in}{2.172313in}}{\pgfqpoint{3.004483in}{2.178137in}}%
\pgfpathcurveto{\pgfqpoint{2.998659in}{2.183960in}}{\pgfqpoint{2.990759in}{2.187233in}}{\pgfqpoint{2.982523in}{2.187233in}}%
\pgfpathcurveto{\pgfqpoint{2.974287in}{2.187233in}}{\pgfqpoint{2.966387in}{2.183960in}}{\pgfqpoint{2.960563in}{2.178137in}}%
\pgfpathcurveto{\pgfqpoint{2.954739in}{2.172313in}}{\pgfqpoint{2.951466in}{2.164413in}}{\pgfqpoint{2.951466in}{2.156176in}}%
\pgfpathcurveto{\pgfqpoint{2.951466in}{2.147940in}}{\pgfqpoint{2.954739in}{2.140040in}}{\pgfqpoint{2.960563in}{2.134216in}}%
\pgfpathcurveto{\pgfqpoint{2.966387in}{2.128392in}}{\pgfqpoint{2.974287in}{2.125120in}}{\pgfqpoint{2.982523in}{2.125120in}}%
\pgfpathclose%
\pgfusepath{stroke,fill}%
\end{pgfscope}%
\begin{pgfscope}%
\pgfpathrectangle{\pgfqpoint{0.100000in}{0.220728in}}{\pgfqpoint{3.696000in}{3.696000in}}%
\pgfusepath{clip}%
\pgfsetbuttcap%
\pgfsetroundjoin%
\definecolor{currentfill}{rgb}{0.121569,0.466667,0.705882}%
\pgfsetfillcolor{currentfill}%
\pgfsetfillopacity{0.815363}%
\pgfsetlinewidth{1.003750pt}%
\definecolor{currentstroke}{rgb}{0.121569,0.466667,0.705882}%
\pgfsetstrokecolor{currentstroke}%
\pgfsetstrokeopacity{0.815363}%
\pgfsetdash{}{0pt}%
\pgfpathmoveto{\pgfqpoint{2.981654in}{2.123198in}}%
\pgfpathcurveto{\pgfqpoint{2.989890in}{2.123198in}}{\pgfqpoint{2.997790in}{2.126471in}}{\pgfqpoint{3.003614in}{2.132295in}}%
\pgfpathcurveto{\pgfqpoint{3.009438in}{2.138119in}}{\pgfqpoint{3.012711in}{2.146019in}}{\pgfqpoint{3.012711in}{2.154255in}}%
\pgfpathcurveto{\pgfqpoint{3.012711in}{2.162491in}}{\pgfqpoint{3.009438in}{2.170391in}}{\pgfqpoint{3.003614in}{2.176215in}}%
\pgfpathcurveto{\pgfqpoint{2.997790in}{2.182039in}}{\pgfqpoint{2.989890in}{2.185311in}}{\pgfqpoint{2.981654in}{2.185311in}}%
\pgfpathcurveto{\pgfqpoint{2.973418in}{2.185311in}}{\pgfqpoint{2.965518in}{2.182039in}}{\pgfqpoint{2.959694in}{2.176215in}}%
\pgfpathcurveto{\pgfqpoint{2.953870in}{2.170391in}}{\pgfqpoint{2.950598in}{2.162491in}}{\pgfqpoint{2.950598in}{2.154255in}}%
\pgfpathcurveto{\pgfqpoint{2.950598in}{2.146019in}}{\pgfqpoint{2.953870in}{2.138119in}}{\pgfqpoint{2.959694in}{2.132295in}}%
\pgfpathcurveto{\pgfqpoint{2.965518in}{2.126471in}}{\pgfqpoint{2.973418in}{2.123198in}}{\pgfqpoint{2.981654in}{2.123198in}}%
\pgfpathclose%
\pgfusepath{stroke,fill}%
\end{pgfscope}%
\begin{pgfscope}%
\pgfpathrectangle{\pgfqpoint{0.100000in}{0.220728in}}{\pgfqpoint{3.696000in}{3.696000in}}%
\pgfusepath{clip}%
\pgfsetbuttcap%
\pgfsetroundjoin%
\definecolor{currentfill}{rgb}{0.121569,0.466667,0.705882}%
\pgfsetfillcolor{currentfill}%
\pgfsetfillopacity{0.815861}%
\pgfsetlinewidth{1.003750pt}%
\definecolor{currentstroke}{rgb}{0.121569,0.466667,0.705882}%
\pgfsetstrokecolor{currentstroke}%
\pgfsetstrokeopacity{0.815861}%
\pgfsetdash{}{0pt}%
\pgfpathmoveto{\pgfqpoint{2.980589in}{2.120418in}}%
\pgfpathcurveto{\pgfqpoint{2.988825in}{2.120418in}}{\pgfqpoint{2.996726in}{2.123691in}}{\pgfqpoint{3.002549in}{2.129515in}}%
\pgfpathcurveto{\pgfqpoint{3.008373in}{2.135338in}}{\pgfqpoint{3.011646in}{2.143239in}}{\pgfqpoint{3.011646in}{2.151475in}}%
\pgfpathcurveto{\pgfqpoint{3.011646in}{2.159711in}}{\pgfqpoint{3.008373in}{2.167611in}}{\pgfqpoint{3.002549in}{2.173435in}}%
\pgfpathcurveto{\pgfqpoint{2.996726in}{2.179259in}}{\pgfqpoint{2.988825in}{2.182531in}}{\pgfqpoint{2.980589in}{2.182531in}}%
\pgfpathcurveto{\pgfqpoint{2.972353in}{2.182531in}}{\pgfqpoint{2.964453in}{2.179259in}}{\pgfqpoint{2.958629in}{2.173435in}}%
\pgfpathcurveto{\pgfqpoint{2.952805in}{2.167611in}}{\pgfqpoint{2.949533in}{2.159711in}}{\pgfqpoint{2.949533in}{2.151475in}}%
\pgfpathcurveto{\pgfqpoint{2.949533in}{2.143239in}}{\pgfqpoint{2.952805in}{2.135338in}}{\pgfqpoint{2.958629in}{2.129515in}}%
\pgfpathcurveto{\pgfqpoint{2.964453in}{2.123691in}}{\pgfqpoint{2.972353in}{2.120418in}}{\pgfqpoint{2.980589in}{2.120418in}}%
\pgfpathclose%
\pgfusepath{stroke,fill}%
\end{pgfscope}%
\begin{pgfscope}%
\pgfpathrectangle{\pgfqpoint{0.100000in}{0.220728in}}{\pgfqpoint{3.696000in}{3.696000in}}%
\pgfusepath{clip}%
\pgfsetbuttcap%
\pgfsetroundjoin%
\definecolor{currentfill}{rgb}{0.121569,0.466667,0.705882}%
\pgfsetfillcolor{currentfill}%
\pgfsetfillopacity{0.816083}%
\pgfsetlinewidth{1.003750pt}%
\definecolor{currentstroke}{rgb}{0.121569,0.466667,0.705882}%
\pgfsetstrokecolor{currentstroke}%
\pgfsetstrokeopacity{0.816083}%
\pgfsetdash{}{0pt}%
\pgfpathmoveto{\pgfqpoint{2.979741in}{2.118988in}}%
\pgfpathcurveto{\pgfqpoint{2.987978in}{2.118988in}}{\pgfqpoint{2.995878in}{2.122261in}}{\pgfqpoint{3.001702in}{2.128085in}}%
\pgfpathcurveto{\pgfqpoint{3.007525in}{2.133909in}}{\pgfqpoint{3.010798in}{2.141809in}}{\pgfqpoint{3.010798in}{2.150045in}}%
\pgfpathcurveto{\pgfqpoint{3.010798in}{2.158281in}}{\pgfqpoint{3.007525in}{2.166181in}}{\pgfqpoint{3.001702in}{2.172005in}}%
\pgfpathcurveto{\pgfqpoint{2.995878in}{2.177829in}}{\pgfqpoint{2.987978in}{2.181101in}}{\pgfqpoint{2.979741in}{2.181101in}}%
\pgfpathcurveto{\pgfqpoint{2.971505in}{2.181101in}}{\pgfqpoint{2.963605in}{2.177829in}}{\pgfqpoint{2.957781in}{2.172005in}}%
\pgfpathcurveto{\pgfqpoint{2.951957in}{2.166181in}}{\pgfqpoint{2.948685in}{2.158281in}}{\pgfqpoint{2.948685in}{2.150045in}}%
\pgfpathcurveto{\pgfqpoint{2.948685in}{2.141809in}}{\pgfqpoint{2.951957in}{2.133909in}}{\pgfqpoint{2.957781in}{2.128085in}}%
\pgfpathcurveto{\pgfqpoint{2.963605in}{2.122261in}}{\pgfqpoint{2.971505in}{2.118988in}}{\pgfqpoint{2.979741in}{2.118988in}}%
\pgfpathclose%
\pgfusepath{stroke,fill}%
\end{pgfscope}%
\begin{pgfscope}%
\pgfpathrectangle{\pgfqpoint{0.100000in}{0.220728in}}{\pgfqpoint{3.696000in}{3.696000in}}%
\pgfusepath{clip}%
\pgfsetbuttcap%
\pgfsetroundjoin%
\definecolor{currentfill}{rgb}{0.121569,0.466667,0.705882}%
\pgfsetfillcolor{currentfill}%
\pgfsetfillopacity{0.816414}%
\pgfsetlinewidth{1.003750pt}%
\definecolor{currentstroke}{rgb}{0.121569,0.466667,0.705882}%
\pgfsetstrokecolor{currentstroke}%
\pgfsetstrokeopacity{0.816414}%
\pgfsetdash{}{0pt}%
\pgfpathmoveto{\pgfqpoint{1.419394in}{1.096673in}}%
\pgfpathcurveto{\pgfqpoint{1.427630in}{1.096673in}}{\pgfqpoint{1.435530in}{1.099945in}}{\pgfqpoint{1.441354in}{1.105769in}}%
\pgfpathcurveto{\pgfqpoint{1.447178in}{1.111593in}}{\pgfqpoint{1.450450in}{1.119493in}}{\pgfqpoint{1.450450in}{1.127730in}}%
\pgfpathcurveto{\pgfqpoint{1.450450in}{1.135966in}}{\pgfqpoint{1.447178in}{1.143866in}}{\pgfqpoint{1.441354in}{1.149690in}}%
\pgfpathcurveto{\pgfqpoint{1.435530in}{1.155514in}}{\pgfqpoint{1.427630in}{1.158786in}}{\pgfqpoint{1.419394in}{1.158786in}}%
\pgfpathcurveto{\pgfqpoint{1.411158in}{1.158786in}}{\pgfqpoint{1.403258in}{1.155514in}}{\pgfqpoint{1.397434in}{1.149690in}}%
\pgfpathcurveto{\pgfqpoint{1.391610in}{1.143866in}}{\pgfqpoint{1.388337in}{1.135966in}}{\pgfqpoint{1.388337in}{1.127730in}}%
\pgfpathcurveto{\pgfqpoint{1.388337in}{1.119493in}}{\pgfqpoint{1.391610in}{1.111593in}}{\pgfqpoint{1.397434in}{1.105769in}}%
\pgfpathcurveto{\pgfqpoint{1.403258in}{1.099945in}}{\pgfqpoint{1.411158in}{1.096673in}}{\pgfqpoint{1.419394in}{1.096673in}}%
\pgfpathclose%
\pgfusepath{stroke,fill}%
\end{pgfscope}%
\begin{pgfscope}%
\pgfpathrectangle{\pgfqpoint{0.100000in}{0.220728in}}{\pgfqpoint{3.696000in}{3.696000in}}%
\pgfusepath{clip}%
\pgfsetbuttcap%
\pgfsetroundjoin%
\definecolor{currentfill}{rgb}{0.121569,0.466667,0.705882}%
\pgfsetfillcolor{currentfill}%
\pgfsetfillopacity{0.816661}%
\pgfsetlinewidth{1.003750pt}%
\definecolor{currentstroke}{rgb}{0.121569,0.466667,0.705882}%
\pgfsetstrokecolor{currentstroke}%
\pgfsetstrokeopacity{0.816661}%
\pgfsetdash{}{0pt}%
\pgfpathmoveto{\pgfqpoint{2.978510in}{2.115832in}}%
\pgfpathcurveto{\pgfqpoint{2.986747in}{2.115832in}}{\pgfqpoint{2.994647in}{2.119105in}}{\pgfqpoint{3.000471in}{2.124929in}}%
\pgfpathcurveto{\pgfqpoint{3.006295in}{2.130753in}}{\pgfqpoint{3.009567in}{2.138653in}}{\pgfqpoint{3.009567in}{2.146889in}}%
\pgfpathcurveto{\pgfqpoint{3.009567in}{2.155125in}}{\pgfqpoint{3.006295in}{2.163025in}}{\pgfqpoint{3.000471in}{2.168849in}}%
\pgfpathcurveto{\pgfqpoint{2.994647in}{2.174673in}}{\pgfqpoint{2.986747in}{2.177945in}}{\pgfqpoint{2.978510in}{2.177945in}}%
\pgfpathcurveto{\pgfqpoint{2.970274in}{2.177945in}}{\pgfqpoint{2.962374in}{2.174673in}}{\pgfqpoint{2.956550in}{2.168849in}}%
\pgfpathcurveto{\pgfqpoint{2.950726in}{2.163025in}}{\pgfqpoint{2.947454in}{2.155125in}}{\pgfqpoint{2.947454in}{2.146889in}}%
\pgfpathcurveto{\pgfqpoint{2.947454in}{2.138653in}}{\pgfqpoint{2.950726in}{2.130753in}}{\pgfqpoint{2.956550in}{2.124929in}}%
\pgfpathcurveto{\pgfqpoint{2.962374in}{2.119105in}}{\pgfqpoint{2.970274in}{2.115832in}}{\pgfqpoint{2.978510in}{2.115832in}}%
\pgfpathclose%
\pgfusepath{stroke,fill}%
\end{pgfscope}%
\begin{pgfscope}%
\pgfpathrectangle{\pgfqpoint{0.100000in}{0.220728in}}{\pgfqpoint{3.696000in}{3.696000in}}%
\pgfusepath{clip}%
\pgfsetbuttcap%
\pgfsetroundjoin%
\definecolor{currentfill}{rgb}{0.121569,0.466667,0.705882}%
\pgfsetfillcolor{currentfill}%
\pgfsetfillopacity{0.817382}%
\pgfsetlinewidth{1.003750pt}%
\definecolor{currentstroke}{rgb}{0.121569,0.466667,0.705882}%
\pgfsetstrokecolor{currentstroke}%
\pgfsetstrokeopacity{0.817382}%
\pgfsetdash{}{0pt}%
\pgfpathmoveto{\pgfqpoint{2.976699in}{2.112121in}}%
\pgfpathcurveto{\pgfqpoint{2.984935in}{2.112121in}}{\pgfqpoint{2.992836in}{2.115394in}}{\pgfqpoint{2.998659in}{2.121218in}}%
\pgfpathcurveto{\pgfqpoint{3.004483in}{2.127041in}}{\pgfqpoint{3.007756in}{2.134942in}}{\pgfqpoint{3.007756in}{2.143178in}}%
\pgfpathcurveto{\pgfqpoint{3.007756in}{2.151414in}}{\pgfqpoint{3.004483in}{2.159314in}}{\pgfqpoint{2.998659in}{2.165138in}}%
\pgfpathcurveto{\pgfqpoint{2.992836in}{2.170962in}}{\pgfqpoint{2.984935in}{2.174234in}}{\pgfqpoint{2.976699in}{2.174234in}}%
\pgfpathcurveto{\pgfqpoint{2.968463in}{2.174234in}}{\pgfqpoint{2.960563in}{2.170962in}}{\pgfqpoint{2.954739in}{2.165138in}}%
\pgfpathcurveto{\pgfqpoint{2.948915in}{2.159314in}}{\pgfqpoint{2.945643in}{2.151414in}}{\pgfqpoint{2.945643in}{2.143178in}}%
\pgfpathcurveto{\pgfqpoint{2.945643in}{2.134942in}}{\pgfqpoint{2.948915in}{2.127041in}}{\pgfqpoint{2.954739in}{2.121218in}}%
\pgfpathcurveto{\pgfqpoint{2.960563in}{2.115394in}}{\pgfqpoint{2.968463in}{2.112121in}}{\pgfqpoint{2.976699in}{2.112121in}}%
\pgfpathclose%
\pgfusepath{stroke,fill}%
\end{pgfscope}%
\begin{pgfscope}%
\pgfpathrectangle{\pgfqpoint{0.100000in}{0.220728in}}{\pgfqpoint{3.696000in}{3.696000in}}%
\pgfusepath{clip}%
\pgfsetbuttcap%
\pgfsetroundjoin%
\definecolor{currentfill}{rgb}{0.121569,0.466667,0.705882}%
\pgfsetfillcolor{currentfill}%
\pgfsetfillopacity{0.817466}%
\pgfsetlinewidth{1.003750pt}%
\definecolor{currentstroke}{rgb}{0.121569,0.466667,0.705882}%
\pgfsetstrokecolor{currentstroke}%
\pgfsetstrokeopacity{0.817466}%
\pgfsetdash{}{0pt}%
\pgfpathmoveto{\pgfqpoint{1.424925in}{1.093578in}}%
\pgfpathcurveto{\pgfqpoint{1.433161in}{1.093578in}}{\pgfqpoint{1.441061in}{1.096850in}}{\pgfqpoint{1.446885in}{1.102674in}}%
\pgfpathcurveto{\pgfqpoint{1.452709in}{1.108498in}}{\pgfqpoint{1.455982in}{1.116398in}}{\pgfqpoint{1.455982in}{1.124634in}}%
\pgfpathcurveto{\pgfqpoint{1.455982in}{1.132870in}}{\pgfqpoint{1.452709in}{1.140770in}}{\pgfqpoint{1.446885in}{1.146594in}}%
\pgfpathcurveto{\pgfqpoint{1.441061in}{1.152418in}}{\pgfqpoint{1.433161in}{1.155691in}}{\pgfqpoint{1.424925in}{1.155691in}}%
\pgfpathcurveto{\pgfqpoint{1.416689in}{1.155691in}}{\pgfqpoint{1.408789in}{1.152418in}}{\pgfqpoint{1.402965in}{1.146594in}}%
\pgfpathcurveto{\pgfqpoint{1.397141in}{1.140770in}}{\pgfqpoint{1.393869in}{1.132870in}}{\pgfqpoint{1.393869in}{1.124634in}}%
\pgfpathcurveto{\pgfqpoint{1.393869in}{1.116398in}}{\pgfqpoint{1.397141in}{1.108498in}}{\pgfqpoint{1.402965in}{1.102674in}}%
\pgfpathcurveto{\pgfqpoint{1.408789in}{1.096850in}}{\pgfqpoint{1.416689in}{1.093578in}}{\pgfqpoint{1.424925in}{1.093578in}}%
\pgfpathclose%
\pgfusepath{stroke,fill}%
\end{pgfscope}%
\begin{pgfscope}%
\pgfpathrectangle{\pgfqpoint{0.100000in}{0.220728in}}{\pgfqpoint{3.696000in}{3.696000in}}%
\pgfusepath{clip}%
\pgfsetbuttcap%
\pgfsetroundjoin%
\definecolor{currentfill}{rgb}{0.121569,0.466667,0.705882}%
\pgfsetfillcolor{currentfill}%
\pgfsetfillopacity{0.818222}%
\pgfsetlinewidth{1.003750pt}%
\definecolor{currentstroke}{rgb}{0.121569,0.466667,0.705882}%
\pgfsetstrokecolor{currentstroke}%
\pgfsetstrokeopacity{0.818222}%
\pgfsetdash{}{0pt}%
\pgfpathmoveto{\pgfqpoint{2.974272in}{2.108275in}}%
\pgfpathcurveto{\pgfqpoint{2.982508in}{2.108275in}}{\pgfqpoint{2.990408in}{2.111547in}}{\pgfqpoint{2.996232in}{2.117371in}}%
\pgfpathcurveto{\pgfqpoint{3.002056in}{2.123195in}}{\pgfqpoint{3.005328in}{2.131095in}}{\pgfqpoint{3.005328in}{2.139331in}}%
\pgfpathcurveto{\pgfqpoint{3.005328in}{2.147567in}}{\pgfqpoint{3.002056in}{2.155467in}}{\pgfqpoint{2.996232in}{2.161291in}}%
\pgfpathcurveto{\pgfqpoint{2.990408in}{2.167115in}}{\pgfqpoint{2.982508in}{2.170388in}}{\pgfqpoint{2.974272in}{2.170388in}}%
\pgfpathcurveto{\pgfqpoint{2.966035in}{2.170388in}}{\pgfqpoint{2.958135in}{2.167115in}}{\pgfqpoint{2.952311in}{2.161291in}}%
\pgfpathcurveto{\pgfqpoint{2.946487in}{2.155467in}}{\pgfqpoint{2.943215in}{2.147567in}}{\pgfqpoint{2.943215in}{2.139331in}}%
\pgfpathcurveto{\pgfqpoint{2.943215in}{2.131095in}}{\pgfqpoint{2.946487in}{2.123195in}}{\pgfqpoint{2.952311in}{2.117371in}}%
\pgfpathcurveto{\pgfqpoint{2.958135in}{2.111547in}}{\pgfqpoint{2.966035in}{2.108275in}}{\pgfqpoint{2.974272in}{2.108275in}}%
\pgfpathclose%
\pgfusepath{stroke,fill}%
\end{pgfscope}%
\begin{pgfscope}%
\pgfpathrectangle{\pgfqpoint{0.100000in}{0.220728in}}{\pgfqpoint{3.696000in}{3.696000in}}%
\pgfusepath{clip}%
\pgfsetbuttcap%
\pgfsetroundjoin%
\definecolor{currentfill}{rgb}{0.121569,0.466667,0.705882}%
\pgfsetfillcolor{currentfill}%
\pgfsetfillopacity{0.819104}%
\pgfsetlinewidth{1.003750pt}%
\definecolor{currentstroke}{rgb}{0.121569,0.466667,0.705882}%
\pgfsetstrokecolor{currentstroke}%
\pgfsetstrokeopacity{0.819104}%
\pgfsetdash{}{0pt}%
\pgfpathmoveto{\pgfqpoint{2.972229in}{2.101927in}}%
\pgfpathcurveto{\pgfqpoint{2.980465in}{2.101927in}}{\pgfqpoint{2.988365in}{2.105199in}}{\pgfqpoint{2.994189in}{2.111023in}}%
\pgfpathcurveto{\pgfqpoint{3.000013in}{2.116847in}}{\pgfqpoint{3.003285in}{2.124747in}}{\pgfqpoint{3.003285in}{2.132984in}}%
\pgfpathcurveto{\pgfqpoint{3.003285in}{2.141220in}}{\pgfqpoint{3.000013in}{2.149120in}}{\pgfqpoint{2.994189in}{2.154944in}}%
\pgfpathcurveto{\pgfqpoint{2.988365in}{2.160768in}}{\pgfqpoint{2.980465in}{2.164040in}}{\pgfqpoint{2.972229in}{2.164040in}}%
\pgfpathcurveto{\pgfqpoint{2.963993in}{2.164040in}}{\pgfqpoint{2.956093in}{2.160768in}}{\pgfqpoint{2.950269in}{2.154944in}}%
\pgfpathcurveto{\pgfqpoint{2.944445in}{2.149120in}}{\pgfqpoint{2.941172in}{2.141220in}}{\pgfqpoint{2.941172in}{2.132984in}}%
\pgfpathcurveto{\pgfqpoint{2.941172in}{2.124747in}}{\pgfqpoint{2.944445in}{2.116847in}}{\pgfqpoint{2.950269in}{2.111023in}}%
\pgfpathcurveto{\pgfqpoint{2.956093in}{2.105199in}}{\pgfqpoint{2.963993in}{2.101927in}}{\pgfqpoint{2.972229in}{2.101927in}}%
\pgfpathclose%
\pgfusepath{stroke,fill}%
\end{pgfscope}%
\begin{pgfscope}%
\pgfpathrectangle{\pgfqpoint{0.100000in}{0.220728in}}{\pgfqpoint{3.696000in}{3.696000in}}%
\pgfusepath{clip}%
\pgfsetbuttcap%
\pgfsetroundjoin%
\definecolor{currentfill}{rgb}{0.121569,0.466667,0.705882}%
\pgfsetfillcolor{currentfill}%
\pgfsetfillopacity{0.819602}%
\pgfsetlinewidth{1.003750pt}%
\definecolor{currentstroke}{rgb}{0.121569,0.466667,0.705882}%
\pgfsetstrokecolor{currentstroke}%
\pgfsetstrokeopacity{0.819602}%
\pgfsetdash{}{0pt}%
\pgfpathmoveto{\pgfqpoint{1.434686in}{1.087955in}}%
\pgfpathcurveto{\pgfqpoint{1.442922in}{1.087955in}}{\pgfqpoint{1.450822in}{1.091227in}}{\pgfqpoint{1.456646in}{1.097051in}}%
\pgfpathcurveto{\pgfqpoint{1.462470in}{1.102875in}}{\pgfqpoint{1.465742in}{1.110775in}}{\pgfqpoint{1.465742in}{1.119011in}}%
\pgfpathcurveto{\pgfqpoint{1.465742in}{1.127247in}}{\pgfqpoint{1.462470in}{1.135148in}}{\pgfqpoint{1.456646in}{1.140971in}}%
\pgfpathcurveto{\pgfqpoint{1.450822in}{1.146795in}}{\pgfqpoint{1.442922in}{1.150068in}}{\pgfqpoint{1.434686in}{1.150068in}}%
\pgfpathcurveto{\pgfqpoint{1.426450in}{1.150068in}}{\pgfqpoint{1.418549in}{1.146795in}}{\pgfqpoint{1.412726in}{1.140971in}}%
\pgfpathcurveto{\pgfqpoint{1.406902in}{1.135148in}}{\pgfqpoint{1.403629in}{1.127247in}}{\pgfqpoint{1.403629in}{1.119011in}}%
\pgfpathcurveto{\pgfqpoint{1.403629in}{1.110775in}}{\pgfqpoint{1.406902in}{1.102875in}}{\pgfqpoint{1.412726in}{1.097051in}}%
\pgfpathcurveto{\pgfqpoint{1.418549in}{1.091227in}}{\pgfqpoint{1.426450in}{1.087955in}}{\pgfqpoint{1.434686in}{1.087955in}}%
\pgfpathclose%
\pgfusepath{stroke,fill}%
\end{pgfscope}%
\begin{pgfscope}%
\pgfpathrectangle{\pgfqpoint{0.100000in}{0.220728in}}{\pgfqpoint{3.696000in}{3.696000in}}%
\pgfusepath{clip}%
\pgfsetbuttcap%
\pgfsetroundjoin%
\definecolor{currentfill}{rgb}{0.121569,0.466667,0.705882}%
\pgfsetfillcolor{currentfill}%
\pgfsetfillopacity{0.820159}%
\pgfsetlinewidth{1.003750pt}%
\definecolor{currentstroke}{rgb}{0.121569,0.466667,0.705882}%
\pgfsetstrokecolor{currentstroke}%
\pgfsetstrokeopacity{0.820159}%
\pgfsetdash{}{0pt}%
\pgfpathmoveto{\pgfqpoint{2.968615in}{2.095633in}}%
\pgfpathcurveto{\pgfqpoint{2.976851in}{2.095633in}}{\pgfqpoint{2.984751in}{2.098905in}}{\pgfqpoint{2.990575in}{2.104729in}}%
\pgfpathcurveto{\pgfqpoint{2.996399in}{2.110553in}}{\pgfqpoint{2.999671in}{2.118453in}}{\pgfqpoint{2.999671in}{2.126689in}}%
\pgfpathcurveto{\pgfqpoint{2.999671in}{2.134926in}}{\pgfqpoint{2.996399in}{2.142826in}}{\pgfqpoint{2.990575in}{2.148650in}}%
\pgfpathcurveto{\pgfqpoint{2.984751in}{2.154473in}}{\pgfqpoint{2.976851in}{2.157746in}}{\pgfqpoint{2.968615in}{2.157746in}}%
\pgfpathcurveto{\pgfqpoint{2.960379in}{2.157746in}}{\pgfqpoint{2.952479in}{2.154473in}}{\pgfqpoint{2.946655in}{2.148650in}}%
\pgfpathcurveto{\pgfqpoint{2.940831in}{2.142826in}}{\pgfqpoint{2.937558in}{2.134926in}}{\pgfqpoint{2.937558in}{2.126689in}}%
\pgfpathcurveto{\pgfqpoint{2.937558in}{2.118453in}}{\pgfqpoint{2.940831in}{2.110553in}}{\pgfqpoint{2.946655in}{2.104729in}}%
\pgfpathcurveto{\pgfqpoint{2.952479in}{2.098905in}}{\pgfqpoint{2.960379in}{2.095633in}}{\pgfqpoint{2.968615in}{2.095633in}}%
\pgfpathclose%
\pgfusepath{stroke,fill}%
\end{pgfscope}%
\begin{pgfscope}%
\pgfpathrectangle{\pgfqpoint{0.100000in}{0.220728in}}{\pgfqpoint{3.696000in}{3.696000in}}%
\pgfusepath{clip}%
\pgfsetbuttcap%
\pgfsetroundjoin%
\definecolor{currentfill}{rgb}{0.121569,0.466667,0.705882}%
\pgfsetfillcolor{currentfill}%
\pgfsetfillopacity{0.821441}%
\pgfsetlinewidth{1.003750pt}%
\definecolor{currentstroke}{rgb}{0.121569,0.466667,0.705882}%
\pgfsetstrokecolor{currentstroke}%
\pgfsetstrokeopacity{0.821441}%
\pgfsetdash{}{0pt}%
\pgfpathmoveto{\pgfqpoint{2.964794in}{2.087334in}}%
\pgfpathcurveto{\pgfqpoint{2.973030in}{2.087334in}}{\pgfqpoint{2.980930in}{2.090607in}}{\pgfqpoint{2.986754in}{2.096430in}}%
\pgfpathcurveto{\pgfqpoint{2.992578in}{2.102254in}}{\pgfqpoint{2.995850in}{2.110154in}}{\pgfqpoint{2.995850in}{2.118391in}}%
\pgfpathcurveto{\pgfqpoint{2.995850in}{2.126627in}}{\pgfqpoint{2.992578in}{2.134527in}}{\pgfqpoint{2.986754in}{2.140351in}}%
\pgfpathcurveto{\pgfqpoint{2.980930in}{2.146175in}}{\pgfqpoint{2.973030in}{2.149447in}}{\pgfqpoint{2.964794in}{2.149447in}}%
\pgfpathcurveto{\pgfqpoint{2.956557in}{2.149447in}}{\pgfqpoint{2.948657in}{2.146175in}}{\pgfqpoint{2.942833in}{2.140351in}}%
\pgfpathcurveto{\pgfqpoint{2.937009in}{2.134527in}}{\pgfqpoint{2.933737in}{2.126627in}}{\pgfqpoint{2.933737in}{2.118391in}}%
\pgfpathcurveto{\pgfqpoint{2.933737in}{2.110154in}}{\pgfqpoint{2.937009in}{2.102254in}}{\pgfqpoint{2.942833in}{2.096430in}}%
\pgfpathcurveto{\pgfqpoint{2.948657in}{2.090607in}}{\pgfqpoint{2.956557in}{2.087334in}}{\pgfqpoint{2.964794in}{2.087334in}}%
\pgfpathclose%
\pgfusepath{stroke,fill}%
\end{pgfscope}%
\begin{pgfscope}%
\pgfpathrectangle{\pgfqpoint{0.100000in}{0.220728in}}{\pgfqpoint{3.696000in}{3.696000in}}%
\pgfusepath{clip}%
\pgfsetbuttcap%
\pgfsetroundjoin%
\definecolor{currentfill}{rgb}{0.121569,0.466667,0.705882}%
\pgfsetfillcolor{currentfill}%
\pgfsetfillopacity{0.821820}%
\pgfsetlinewidth{1.003750pt}%
\definecolor{currentstroke}{rgb}{0.121569,0.466667,0.705882}%
\pgfsetstrokecolor{currentstroke}%
\pgfsetstrokeopacity{0.821820}%
\pgfsetdash{}{0pt}%
\pgfpathmoveto{\pgfqpoint{1.444091in}{1.085777in}}%
\pgfpathcurveto{\pgfqpoint{1.452327in}{1.085777in}}{\pgfqpoint{1.460227in}{1.089050in}}{\pgfqpoint{1.466051in}{1.094874in}}%
\pgfpathcurveto{\pgfqpoint{1.471875in}{1.100698in}}{\pgfqpoint{1.475147in}{1.108598in}}{\pgfqpoint{1.475147in}{1.116834in}}%
\pgfpathcurveto{\pgfqpoint{1.475147in}{1.125070in}}{\pgfqpoint{1.471875in}{1.132970in}}{\pgfqpoint{1.466051in}{1.138794in}}%
\pgfpathcurveto{\pgfqpoint{1.460227in}{1.144618in}}{\pgfqpoint{1.452327in}{1.147890in}}{\pgfqpoint{1.444091in}{1.147890in}}%
\pgfpathcurveto{\pgfqpoint{1.435854in}{1.147890in}}{\pgfqpoint{1.427954in}{1.144618in}}{\pgfqpoint{1.422130in}{1.138794in}}%
\pgfpathcurveto{\pgfqpoint{1.416306in}{1.132970in}}{\pgfqpoint{1.413034in}{1.125070in}}{\pgfqpoint{1.413034in}{1.116834in}}%
\pgfpathcurveto{\pgfqpoint{1.413034in}{1.108598in}}{\pgfqpoint{1.416306in}{1.100698in}}{\pgfqpoint{1.422130in}{1.094874in}}%
\pgfpathcurveto{\pgfqpoint{1.427954in}{1.089050in}}{\pgfqpoint{1.435854in}{1.085777in}}{\pgfqpoint{1.444091in}{1.085777in}}%
\pgfpathclose%
\pgfusepath{stroke,fill}%
\end{pgfscope}%
\begin{pgfscope}%
\pgfpathrectangle{\pgfqpoint{0.100000in}{0.220728in}}{\pgfqpoint{3.696000in}{3.696000in}}%
\pgfusepath{clip}%
\pgfsetbuttcap%
\pgfsetroundjoin%
\definecolor{currentfill}{rgb}{0.121569,0.466667,0.705882}%
\pgfsetfillcolor{currentfill}%
\pgfsetfillopacity{0.823132}%
\pgfsetlinewidth{1.003750pt}%
\definecolor{currentstroke}{rgb}{0.121569,0.466667,0.705882}%
\pgfsetstrokecolor{currentstroke}%
\pgfsetstrokeopacity{0.823132}%
\pgfsetdash{}{0pt}%
\pgfpathmoveto{\pgfqpoint{2.961324in}{2.078936in}}%
\pgfpathcurveto{\pgfqpoint{2.969560in}{2.078936in}}{\pgfqpoint{2.977461in}{2.082208in}}{\pgfqpoint{2.983284in}{2.088032in}}%
\pgfpathcurveto{\pgfqpoint{2.989108in}{2.093856in}}{\pgfqpoint{2.992381in}{2.101756in}}{\pgfqpoint{2.992381in}{2.109992in}}%
\pgfpathcurveto{\pgfqpoint{2.992381in}{2.118229in}}{\pgfqpoint{2.989108in}{2.126129in}}{\pgfqpoint{2.983284in}{2.131953in}}%
\pgfpathcurveto{\pgfqpoint{2.977461in}{2.137777in}}{\pgfqpoint{2.969560in}{2.141049in}}{\pgfqpoint{2.961324in}{2.141049in}}%
\pgfpathcurveto{\pgfqpoint{2.953088in}{2.141049in}}{\pgfqpoint{2.945188in}{2.137777in}}{\pgfqpoint{2.939364in}{2.131953in}}%
\pgfpathcurveto{\pgfqpoint{2.933540in}{2.126129in}}{\pgfqpoint{2.930268in}{2.118229in}}{\pgfqpoint{2.930268in}{2.109992in}}%
\pgfpathcurveto{\pgfqpoint{2.930268in}{2.101756in}}{\pgfqpoint{2.933540in}{2.093856in}}{\pgfqpoint{2.939364in}{2.088032in}}%
\pgfpathcurveto{\pgfqpoint{2.945188in}{2.082208in}}{\pgfqpoint{2.953088in}{2.078936in}}{\pgfqpoint{2.961324in}{2.078936in}}%
\pgfpathclose%
\pgfusepath{stroke,fill}%
\end{pgfscope}%
\begin{pgfscope}%
\pgfpathrectangle{\pgfqpoint{0.100000in}{0.220728in}}{\pgfqpoint{3.696000in}{3.696000in}}%
\pgfusepath{clip}%
\pgfsetbuttcap%
\pgfsetroundjoin%
\definecolor{currentfill}{rgb}{0.121569,0.466667,0.705882}%
\pgfsetfillcolor{currentfill}%
\pgfsetfillopacity{0.823431}%
\pgfsetlinewidth{1.003750pt}%
\definecolor{currentstroke}{rgb}{0.121569,0.466667,0.705882}%
\pgfsetstrokecolor{currentstroke}%
\pgfsetstrokeopacity{0.823431}%
\pgfsetdash{}{0pt}%
\pgfpathmoveto{\pgfqpoint{1.451968in}{1.082242in}}%
\pgfpathcurveto{\pgfqpoint{1.460204in}{1.082242in}}{\pgfqpoint{1.468104in}{1.085514in}}{\pgfqpoint{1.473928in}{1.091338in}}%
\pgfpathcurveto{\pgfqpoint{1.479752in}{1.097162in}}{\pgfqpoint{1.483025in}{1.105062in}}{\pgfqpoint{1.483025in}{1.113298in}}%
\pgfpathcurveto{\pgfqpoint{1.483025in}{1.121535in}}{\pgfqpoint{1.479752in}{1.129435in}}{\pgfqpoint{1.473928in}{1.135259in}}%
\pgfpathcurveto{\pgfqpoint{1.468104in}{1.141083in}}{\pgfqpoint{1.460204in}{1.144355in}}{\pgfqpoint{1.451968in}{1.144355in}}%
\pgfpathcurveto{\pgfqpoint{1.443732in}{1.144355in}}{\pgfqpoint{1.435832in}{1.141083in}}{\pgfqpoint{1.430008in}{1.135259in}}%
\pgfpathcurveto{\pgfqpoint{1.424184in}{1.129435in}}{\pgfqpoint{1.420912in}{1.121535in}}{\pgfqpoint{1.420912in}{1.113298in}}%
\pgfpathcurveto{\pgfqpoint{1.420912in}{1.105062in}}{\pgfqpoint{1.424184in}{1.097162in}}{\pgfqpoint{1.430008in}{1.091338in}}%
\pgfpathcurveto{\pgfqpoint{1.435832in}{1.085514in}}{\pgfqpoint{1.443732in}{1.082242in}}{\pgfqpoint{1.451968in}{1.082242in}}%
\pgfpathclose%
\pgfusepath{stroke,fill}%
\end{pgfscope}%
\begin{pgfscope}%
\pgfpathrectangle{\pgfqpoint{0.100000in}{0.220728in}}{\pgfqpoint{3.696000in}{3.696000in}}%
\pgfusepath{clip}%
\pgfsetbuttcap%
\pgfsetroundjoin%
\definecolor{currentfill}{rgb}{0.121569,0.466667,0.705882}%
\pgfsetfillcolor{currentfill}%
\pgfsetfillopacity{0.823783}%
\pgfsetlinewidth{1.003750pt}%
\definecolor{currentstroke}{rgb}{0.121569,0.466667,0.705882}%
\pgfsetstrokecolor{currentstroke}%
\pgfsetstrokeopacity{0.823783}%
\pgfsetdash{}{0pt}%
\pgfpathmoveto{\pgfqpoint{2.958499in}{2.074329in}}%
\pgfpathcurveto{\pgfqpoint{2.966735in}{2.074329in}}{\pgfqpoint{2.974635in}{2.077602in}}{\pgfqpoint{2.980459in}{2.083426in}}%
\pgfpathcurveto{\pgfqpoint{2.986283in}{2.089250in}}{\pgfqpoint{2.989555in}{2.097150in}}{\pgfqpoint{2.989555in}{2.105386in}}%
\pgfpathcurveto{\pgfqpoint{2.989555in}{2.113622in}}{\pgfqpoint{2.986283in}{2.121522in}}{\pgfqpoint{2.980459in}{2.127346in}}%
\pgfpathcurveto{\pgfqpoint{2.974635in}{2.133170in}}{\pgfqpoint{2.966735in}{2.136442in}}{\pgfqpoint{2.958499in}{2.136442in}}%
\pgfpathcurveto{\pgfqpoint{2.950262in}{2.136442in}}{\pgfqpoint{2.942362in}{2.133170in}}{\pgfqpoint{2.936538in}{2.127346in}}%
\pgfpathcurveto{\pgfqpoint{2.930714in}{2.121522in}}{\pgfqpoint{2.927442in}{2.113622in}}{\pgfqpoint{2.927442in}{2.105386in}}%
\pgfpathcurveto{\pgfqpoint{2.927442in}{2.097150in}}{\pgfqpoint{2.930714in}{2.089250in}}{\pgfqpoint{2.936538in}{2.083426in}}%
\pgfpathcurveto{\pgfqpoint{2.942362in}{2.077602in}}{\pgfqpoint{2.950262in}{2.074329in}}{\pgfqpoint{2.958499in}{2.074329in}}%
\pgfpathclose%
\pgfusepath{stroke,fill}%
\end{pgfscope}%
\begin{pgfscope}%
\pgfpathrectangle{\pgfqpoint{0.100000in}{0.220728in}}{\pgfqpoint{3.696000in}{3.696000in}}%
\pgfusepath{clip}%
\pgfsetbuttcap%
\pgfsetroundjoin%
\definecolor{currentfill}{rgb}{0.121569,0.466667,0.705882}%
\pgfsetfillcolor{currentfill}%
\pgfsetfillopacity{0.825135}%
\pgfsetlinewidth{1.003750pt}%
\definecolor{currentstroke}{rgb}{0.121569,0.466667,0.705882}%
\pgfsetstrokecolor{currentstroke}%
\pgfsetstrokeopacity{0.825135}%
\pgfsetdash{}{0pt}%
\pgfpathmoveto{\pgfqpoint{2.956200in}{2.066425in}}%
\pgfpathcurveto{\pgfqpoint{2.964437in}{2.066425in}}{\pgfqpoint{2.972337in}{2.069697in}}{\pgfqpoint{2.978161in}{2.075521in}}%
\pgfpathcurveto{\pgfqpoint{2.983984in}{2.081345in}}{\pgfqpoint{2.987257in}{2.089245in}}{\pgfqpoint{2.987257in}{2.097481in}}%
\pgfpathcurveto{\pgfqpoint{2.987257in}{2.105717in}}{\pgfqpoint{2.983984in}{2.113617in}}{\pgfqpoint{2.978161in}{2.119441in}}%
\pgfpathcurveto{\pgfqpoint{2.972337in}{2.125265in}}{\pgfqpoint{2.964437in}{2.128538in}}{\pgfqpoint{2.956200in}{2.128538in}}%
\pgfpathcurveto{\pgfqpoint{2.947964in}{2.128538in}}{\pgfqpoint{2.940064in}{2.125265in}}{\pgfqpoint{2.934240in}{2.119441in}}%
\pgfpathcurveto{\pgfqpoint{2.928416in}{2.113617in}}{\pgfqpoint{2.925144in}{2.105717in}}{\pgfqpoint{2.925144in}{2.097481in}}%
\pgfpathcurveto{\pgfqpoint{2.925144in}{2.089245in}}{\pgfqpoint{2.928416in}{2.081345in}}{\pgfqpoint{2.934240in}{2.075521in}}%
\pgfpathcurveto{\pgfqpoint{2.940064in}{2.069697in}}{\pgfqpoint{2.947964in}{2.066425in}}{\pgfqpoint{2.956200in}{2.066425in}}%
\pgfpathclose%
\pgfusepath{stroke,fill}%
\end{pgfscope}%
\begin{pgfscope}%
\pgfpathrectangle{\pgfqpoint{0.100000in}{0.220728in}}{\pgfqpoint{3.696000in}{3.696000in}}%
\pgfusepath{clip}%
\pgfsetbuttcap%
\pgfsetroundjoin%
\definecolor{currentfill}{rgb}{0.121569,0.466667,0.705882}%
\pgfsetfillcolor{currentfill}%
\pgfsetfillopacity{0.825245}%
\pgfsetlinewidth{1.003750pt}%
\definecolor{currentstroke}{rgb}{0.121569,0.466667,0.705882}%
\pgfsetstrokecolor{currentstroke}%
\pgfsetstrokeopacity{0.825245}%
\pgfsetdash{}{0pt}%
\pgfpathmoveto{\pgfqpoint{1.458724in}{1.080984in}}%
\pgfpathcurveto{\pgfqpoint{1.466960in}{1.080984in}}{\pgfqpoint{1.474860in}{1.084256in}}{\pgfqpoint{1.480684in}{1.090080in}}%
\pgfpathcurveto{\pgfqpoint{1.486508in}{1.095904in}}{\pgfqpoint{1.489780in}{1.103804in}}{\pgfqpoint{1.489780in}{1.112041in}}%
\pgfpathcurveto{\pgfqpoint{1.489780in}{1.120277in}}{\pgfqpoint{1.486508in}{1.128177in}}{\pgfqpoint{1.480684in}{1.134001in}}%
\pgfpathcurveto{\pgfqpoint{1.474860in}{1.139825in}}{\pgfqpoint{1.466960in}{1.143097in}}{\pgfqpoint{1.458724in}{1.143097in}}%
\pgfpathcurveto{\pgfqpoint{1.450487in}{1.143097in}}{\pgfqpoint{1.442587in}{1.139825in}}{\pgfqpoint{1.436763in}{1.134001in}}%
\pgfpathcurveto{\pgfqpoint{1.430939in}{1.128177in}}{\pgfqpoint{1.427667in}{1.120277in}}{\pgfqpoint{1.427667in}{1.112041in}}%
\pgfpathcurveto{\pgfqpoint{1.427667in}{1.103804in}}{\pgfqpoint{1.430939in}{1.095904in}}{\pgfqpoint{1.436763in}{1.090080in}}%
\pgfpathcurveto{\pgfqpoint{1.442587in}{1.084256in}}{\pgfqpoint{1.450487in}{1.080984in}}{\pgfqpoint{1.458724in}{1.080984in}}%
\pgfpathclose%
\pgfusepath{stroke,fill}%
\end{pgfscope}%
\begin{pgfscope}%
\pgfpathrectangle{\pgfqpoint{0.100000in}{0.220728in}}{\pgfqpoint{3.696000in}{3.696000in}}%
\pgfusepath{clip}%
\pgfsetbuttcap%
\pgfsetroundjoin%
\definecolor{currentfill}{rgb}{0.121569,0.466667,0.705882}%
\pgfsetfillcolor{currentfill}%
\pgfsetfillopacity{0.825821}%
\pgfsetlinewidth{1.003750pt}%
\definecolor{currentstroke}{rgb}{0.121569,0.466667,0.705882}%
\pgfsetstrokecolor{currentstroke}%
\pgfsetstrokeopacity{0.825821}%
\pgfsetdash{}{0pt}%
\pgfpathmoveto{\pgfqpoint{2.954294in}{2.062420in}}%
\pgfpathcurveto{\pgfqpoint{2.962530in}{2.062420in}}{\pgfqpoint{2.970430in}{2.065692in}}{\pgfqpoint{2.976254in}{2.071516in}}%
\pgfpathcurveto{\pgfqpoint{2.982078in}{2.077340in}}{\pgfqpoint{2.985351in}{2.085240in}}{\pgfqpoint{2.985351in}{2.093476in}}%
\pgfpathcurveto{\pgfqpoint{2.985351in}{2.101713in}}{\pgfqpoint{2.982078in}{2.109613in}}{\pgfqpoint{2.976254in}{2.115437in}}%
\pgfpathcurveto{\pgfqpoint{2.970430in}{2.121261in}}{\pgfqpoint{2.962530in}{2.124533in}}{\pgfqpoint{2.954294in}{2.124533in}}%
\pgfpathcurveto{\pgfqpoint{2.946058in}{2.124533in}}{\pgfqpoint{2.938158in}{2.121261in}}{\pgfqpoint{2.932334in}{2.115437in}}%
\pgfpathcurveto{\pgfqpoint{2.926510in}{2.109613in}}{\pgfqpoint{2.923238in}{2.101713in}}{\pgfqpoint{2.923238in}{2.093476in}}%
\pgfpathcurveto{\pgfqpoint{2.923238in}{2.085240in}}{\pgfqpoint{2.926510in}{2.077340in}}{\pgfqpoint{2.932334in}{2.071516in}}%
\pgfpathcurveto{\pgfqpoint{2.938158in}{2.065692in}}{\pgfqpoint{2.946058in}{2.062420in}}{\pgfqpoint{2.954294in}{2.062420in}}%
\pgfpathclose%
\pgfusepath{stroke,fill}%
\end{pgfscope}%
\begin{pgfscope}%
\pgfpathrectangle{\pgfqpoint{0.100000in}{0.220728in}}{\pgfqpoint{3.696000in}{3.696000in}}%
\pgfusepath{clip}%
\pgfsetbuttcap%
\pgfsetroundjoin%
\definecolor{currentfill}{rgb}{0.121569,0.466667,0.705882}%
\pgfsetfillcolor{currentfill}%
\pgfsetfillopacity{0.826014}%
\pgfsetlinewidth{1.003750pt}%
\definecolor{currentstroke}{rgb}{0.121569,0.466667,0.705882}%
\pgfsetstrokecolor{currentstroke}%
\pgfsetstrokeopacity{0.826014}%
\pgfsetdash{}{0pt}%
\pgfpathmoveto{\pgfqpoint{1.464704in}{1.076787in}}%
\pgfpathcurveto{\pgfqpoint{1.472940in}{1.076787in}}{\pgfqpoint{1.480840in}{1.080059in}}{\pgfqpoint{1.486664in}{1.085883in}}%
\pgfpathcurveto{\pgfqpoint{1.492488in}{1.091707in}}{\pgfqpoint{1.495760in}{1.099607in}}{\pgfqpoint{1.495760in}{1.107844in}}%
\pgfpathcurveto{\pgfqpoint{1.495760in}{1.116080in}}{\pgfqpoint{1.492488in}{1.123980in}}{\pgfqpoint{1.486664in}{1.129804in}}%
\pgfpathcurveto{\pgfqpoint{1.480840in}{1.135628in}}{\pgfqpoint{1.472940in}{1.138900in}}{\pgfqpoint{1.464704in}{1.138900in}}%
\pgfpathcurveto{\pgfqpoint{1.456468in}{1.138900in}}{\pgfqpoint{1.448568in}{1.135628in}}{\pgfqpoint{1.442744in}{1.129804in}}%
\pgfpathcurveto{\pgfqpoint{1.436920in}{1.123980in}}{\pgfqpoint{1.433647in}{1.116080in}}{\pgfqpoint{1.433647in}{1.107844in}}%
\pgfpathcurveto{\pgfqpoint{1.433647in}{1.099607in}}{\pgfqpoint{1.436920in}{1.091707in}}{\pgfqpoint{1.442744in}{1.085883in}}%
\pgfpathcurveto{\pgfqpoint{1.448568in}{1.080059in}}{\pgfqpoint{1.456468in}{1.076787in}}{\pgfqpoint{1.464704in}{1.076787in}}%
\pgfpathclose%
\pgfusepath{stroke,fill}%
\end{pgfscope}%
\begin{pgfscope}%
\pgfpathrectangle{\pgfqpoint{0.100000in}{0.220728in}}{\pgfqpoint{3.696000in}{3.696000in}}%
\pgfusepath{clip}%
\pgfsetbuttcap%
\pgfsetroundjoin%
\definecolor{currentfill}{rgb}{0.121569,0.466667,0.705882}%
\pgfsetfillcolor{currentfill}%
\pgfsetfillopacity{0.826213}%
\pgfsetlinewidth{1.003750pt}%
\definecolor{currentstroke}{rgb}{0.121569,0.466667,0.705882}%
\pgfsetstrokecolor{currentstroke}%
\pgfsetstrokeopacity{0.826213}%
\pgfsetdash{}{0pt}%
\pgfpathmoveto{\pgfqpoint{2.953238in}{2.060284in}}%
\pgfpathcurveto{\pgfqpoint{2.961474in}{2.060284in}}{\pgfqpoint{2.969374in}{2.063556in}}{\pgfqpoint{2.975198in}{2.069380in}}%
\pgfpathcurveto{\pgfqpoint{2.981022in}{2.075204in}}{\pgfqpoint{2.984294in}{2.083104in}}{\pgfqpoint{2.984294in}{2.091340in}}%
\pgfpathcurveto{\pgfqpoint{2.984294in}{2.099576in}}{\pgfqpoint{2.981022in}{2.107476in}}{\pgfqpoint{2.975198in}{2.113300in}}%
\pgfpathcurveto{\pgfqpoint{2.969374in}{2.119124in}}{\pgfqpoint{2.961474in}{2.122397in}}{\pgfqpoint{2.953238in}{2.122397in}}%
\pgfpathcurveto{\pgfqpoint{2.945001in}{2.122397in}}{\pgfqpoint{2.937101in}{2.119124in}}{\pgfqpoint{2.931277in}{2.113300in}}%
\pgfpathcurveto{\pgfqpoint{2.925453in}{2.107476in}}{\pgfqpoint{2.922181in}{2.099576in}}{\pgfqpoint{2.922181in}{2.091340in}}%
\pgfpathcurveto{\pgfqpoint{2.922181in}{2.083104in}}{\pgfqpoint{2.925453in}{2.075204in}}{\pgfqpoint{2.931277in}{2.069380in}}%
\pgfpathcurveto{\pgfqpoint{2.937101in}{2.063556in}}{\pgfqpoint{2.945001in}{2.060284in}}{\pgfqpoint{2.953238in}{2.060284in}}%
\pgfpathclose%
\pgfusepath{stroke,fill}%
\end{pgfscope}%
\begin{pgfscope}%
\pgfpathrectangle{\pgfqpoint{0.100000in}{0.220728in}}{\pgfqpoint{3.696000in}{3.696000in}}%
\pgfusepath{clip}%
\pgfsetbuttcap%
\pgfsetroundjoin%
\definecolor{currentfill}{rgb}{0.121569,0.466667,0.705882}%
\pgfsetfillcolor{currentfill}%
\pgfsetfillopacity{0.826428}%
\pgfsetlinewidth{1.003750pt}%
\definecolor{currentstroke}{rgb}{0.121569,0.466667,0.705882}%
\pgfsetstrokecolor{currentstroke}%
\pgfsetstrokeopacity{0.826428}%
\pgfsetdash{}{0pt}%
\pgfpathmoveto{\pgfqpoint{2.952745in}{2.059017in}}%
\pgfpathcurveto{\pgfqpoint{2.960981in}{2.059017in}}{\pgfqpoint{2.968881in}{2.062289in}}{\pgfqpoint{2.974705in}{2.068113in}}%
\pgfpathcurveto{\pgfqpoint{2.980529in}{2.073937in}}{\pgfqpoint{2.983801in}{2.081837in}}{\pgfqpoint{2.983801in}{2.090074in}}%
\pgfpathcurveto{\pgfqpoint{2.983801in}{2.098310in}}{\pgfqpoint{2.980529in}{2.106210in}}{\pgfqpoint{2.974705in}{2.112034in}}%
\pgfpathcurveto{\pgfqpoint{2.968881in}{2.117858in}}{\pgfqpoint{2.960981in}{2.121130in}}{\pgfqpoint{2.952745in}{2.121130in}}%
\pgfpathcurveto{\pgfqpoint{2.944508in}{2.121130in}}{\pgfqpoint{2.936608in}{2.117858in}}{\pgfqpoint{2.930784in}{2.112034in}}%
\pgfpathcurveto{\pgfqpoint{2.924960in}{2.106210in}}{\pgfqpoint{2.921688in}{2.098310in}}{\pgfqpoint{2.921688in}{2.090074in}}%
\pgfpathcurveto{\pgfqpoint{2.921688in}{2.081837in}}{\pgfqpoint{2.924960in}{2.073937in}}{\pgfqpoint{2.930784in}{2.068113in}}%
\pgfpathcurveto{\pgfqpoint{2.936608in}{2.062289in}}{\pgfqpoint{2.944508in}{2.059017in}}{\pgfqpoint{2.952745in}{2.059017in}}%
\pgfpathclose%
\pgfusepath{stroke,fill}%
\end{pgfscope}%
\begin{pgfscope}%
\pgfpathrectangle{\pgfqpoint{0.100000in}{0.220728in}}{\pgfqpoint{3.696000in}{3.696000in}}%
\pgfusepath{clip}%
\pgfsetbuttcap%
\pgfsetroundjoin%
\definecolor{currentfill}{rgb}{0.121569,0.466667,0.705882}%
\pgfsetfillcolor{currentfill}%
\pgfsetfillopacity{0.826834}%
\pgfsetlinewidth{1.003750pt}%
\definecolor{currentstroke}{rgb}{0.121569,0.466667,0.705882}%
\pgfsetstrokecolor{currentstroke}%
\pgfsetstrokeopacity{0.826834}%
\pgfsetdash{}{0pt}%
\pgfpathmoveto{\pgfqpoint{2.951182in}{2.056306in}}%
\pgfpathcurveto{\pgfqpoint{2.959418in}{2.056306in}}{\pgfqpoint{2.967318in}{2.059578in}}{\pgfqpoint{2.973142in}{2.065402in}}%
\pgfpathcurveto{\pgfqpoint{2.978966in}{2.071226in}}{\pgfqpoint{2.982238in}{2.079126in}}{\pgfqpoint{2.982238in}{2.087362in}}%
\pgfpathcurveto{\pgfqpoint{2.982238in}{2.095599in}}{\pgfqpoint{2.978966in}{2.103499in}}{\pgfqpoint{2.973142in}{2.109323in}}%
\pgfpathcurveto{\pgfqpoint{2.967318in}{2.115146in}}{\pgfqpoint{2.959418in}{2.118419in}}{\pgfqpoint{2.951182in}{2.118419in}}%
\pgfpathcurveto{\pgfqpoint{2.942945in}{2.118419in}}{\pgfqpoint{2.935045in}{2.115146in}}{\pgfqpoint{2.929221in}{2.109323in}}%
\pgfpathcurveto{\pgfqpoint{2.923398in}{2.103499in}}{\pgfqpoint{2.920125in}{2.095599in}}{\pgfqpoint{2.920125in}{2.087362in}}%
\pgfpathcurveto{\pgfqpoint{2.920125in}{2.079126in}}{\pgfqpoint{2.923398in}{2.071226in}}{\pgfqpoint{2.929221in}{2.065402in}}%
\pgfpathcurveto{\pgfqpoint{2.935045in}{2.059578in}}{\pgfqpoint{2.942945in}{2.056306in}}{\pgfqpoint{2.951182in}{2.056306in}}%
\pgfpathclose%
\pgfusepath{stroke,fill}%
\end{pgfscope}%
\begin{pgfscope}%
\pgfpathrectangle{\pgfqpoint{0.100000in}{0.220728in}}{\pgfqpoint{3.696000in}{3.696000in}}%
\pgfusepath{clip}%
\pgfsetbuttcap%
\pgfsetroundjoin%
\definecolor{currentfill}{rgb}{0.121569,0.466667,0.705882}%
\pgfsetfillcolor{currentfill}%
\pgfsetfillopacity{0.827521}%
\pgfsetlinewidth{1.003750pt}%
\definecolor{currentstroke}{rgb}{0.121569,0.466667,0.705882}%
\pgfsetstrokecolor{currentstroke}%
\pgfsetstrokeopacity{0.827521}%
\pgfsetdash{}{0pt}%
\pgfpathmoveto{\pgfqpoint{1.475828in}{1.070467in}}%
\pgfpathcurveto{\pgfqpoint{1.484064in}{1.070467in}}{\pgfqpoint{1.491965in}{1.073739in}}{\pgfqpoint{1.497788in}{1.079563in}}%
\pgfpathcurveto{\pgfqpoint{1.503612in}{1.085387in}}{\pgfqpoint{1.506885in}{1.093287in}}{\pgfqpoint{1.506885in}{1.101523in}}%
\pgfpathcurveto{\pgfqpoint{1.506885in}{1.109759in}}{\pgfqpoint{1.503612in}{1.117659in}}{\pgfqpoint{1.497788in}{1.123483in}}%
\pgfpathcurveto{\pgfqpoint{1.491965in}{1.129307in}}{\pgfqpoint{1.484064in}{1.132580in}}{\pgfqpoint{1.475828in}{1.132580in}}%
\pgfpathcurveto{\pgfqpoint{1.467592in}{1.132580in}}{\pgfqpoint{1.459692in}{1.129307in}}{\pgfqpoint{1.453868in}{1.123483in}}%
\pgfpathcurveto{\pgfqpoint{1.448044in}{1.117659in}}{\pgfqpoint{1.444772in}{1.109759in}}{\pgfqpoint{1.444772in}{1.101523in}}%
\pgfpathcurveto{\pgfqpoint{1.444772in}{1.093287in}}{\pgfqpoint{1.448044in}{1.085387in}}{\pgfqpoint{1.453868in}{1.079563in}}%
\pgfpathcurveto{\pgfqpoint{1.459692in}{1.073739in}}{\pgfqpoint{1.467592in}{1.070467in}}{\pgfqpoint{1.475828in}{1.070467in}}%
\pgfpathclose%
\pgfusepath{stroke,fill}%
\end{pgfscope}%
\begin{pgfscope}%
\pgfpathrectangle{\pgfqpoint{0.100000in}{0.220728in}}{\pgfqpoint{3.696000in}{3.696000in}}%
\pgfusepath{clip}%
\pgfsetbuttcap%
\pgfsetroundjoin%
\definecolor{currentfill}{rgb}{0.121569,0.466667,0.705882}%
\pgfsetfillcolor{currentfill}%
\pgfsetfillopacity{0.827561}%
\pgfsetlinewidth{1.003750pt}%
\definecolor{currentstroke}{rgb}{0.121569,0.466667,0.705882}%
\pgfsetstrokecolor{currentstroke}%
\pgfsetstrokeopacity{0.827561}%
\pgfsetdash{}{0pt}%
\pgfpathmoveto{\pgfqpoint{2.949881in}{2.052063in}}%
\pgfpathcurveto{\pgfqpoint{2.958118in}{2.052063in}}{\pgfqpoint{2.966018in}{2.055336in}}{\pgfqpoint{2.971842in}{2.061160in}}%
\pgfpathcurveto{\pgfqpoint{2.977665in}{2.066984in}}{\pgfqpoint{2.980938in}{2.074884in}}{\pgfqpoint{2.980938in}{2.083120in}}%
\pgfpathcurveto{\pgfqpoint{2.980938in}{2.091356in}}{\pgfqpoint{2.977665in}{2.099256in}}{\pgfqpoint{2.971842in}{2.105080in}}%
\pgfpathcurveto{\pgfqpoint{2.966018in}{2.110904in}}{\pgfqpoint{2.958118in}{2.114176in}}{\pgfqpoint{2.949881in}{2.114176in}}%
\pgfpathcurveto{\pgfqpoint{2.941645in}{2.114176in}}{\pgfqpoint{2.933745in}{2.110904in}}{\pgfqpoint{2.927921in}{2.105080in}}%
\pgfpathcurveto{\pgfqpoint{2.922097in}{2.099256in}}{\pgfqpoint{2.918825in}{2.091356in}}{\pgfqpoint{2.918825in}{2.083120in}}%
\pgfpathcurveto{\pgfqpoint{2.918825in}{2.074884in}}{\pgfqpoint{2.922097in}{2.066984in}}{\pgfqpoint{2.927921in}{2.061160in}}%
\pgfpathcurveto{\pgfqpoint{2.933745in}{2.055336in}}{\pgfqpoint{2.941645in}{2.052063in}}{\pgfqpoint{2.949881in}{2.052063in}}%
\pgfpathclose%
\pgfusepath{stroke,fill}%
\end{pgfscope}%
\begin{pgfscope}%
\pgfpathrectangle{\pgfqpoint{0.100000in}{0.220728in}}{\pgfqpoint{3.696000in}{3.696000in}}%
\pgfusepath{clip}%
\pgfsetbuttcap%
\pgfsetroundjoin%
\definecolor{currentfill}{rgb}{0.121569,0.466667,0.705882}%
\pgfsetfillcolor{currentfill}%
\pgfsetfillopacity{0.828347}%
\pgfsetlinewidth{1.003750pt}%
\definecolor{currentstroke}{rgb}{0.121569,0.466667,0.705882}%
\pgfsetstrokecolor{currentstroke}%
\pgfsetstrokeopacity{0.828347}%
\pgfsetdash{}{0pt}%
\pgfpathmoveto{\pgfqpoint{2.947792in}{2.047190in}}%
\pgfpathcurveto{\pgfqpoint{2.956028in}{2.047190in}}{\pgfqpoint{2.963928in}{2.050463in}}{\pgfqpoint{2.969752in}{2.056287in}}%
\pgfpathcurveto{\pgfqpoint{2.975576in}{2.062111in}}{\pgfqpoint{2.978848in}{2.070011in}}{\pgfqpoint{2.978848in}{2.078247in}}%
\pgfpathcurveto{\pgfqpoint{2.978848in}{2.086483in}}{\pgfqpoint{2.975576in}{2.094383in}}{\pgfqpoint{2.969752in}{2.100207in}}%
\pgfpathcurveto{\pgfqpoint{2.963928in}{2.106031in}}{\pgfqpoint{2.956028in}{2.109303in}}{\pgfqpoint{2.947792in}{2.109303in}}%
\pgfpathcurveto{\pgfqpoint{2.939556in}{2.109303in}}{\pgfqpoint{2.931656in}{2.106031in}}{\pgfqpoint{2.925832in}{2.100207in}}%
\pgfpathcurveto{\pgfqpoint{2.920008in}{2.094383in}}{\pgfqpoint{2.916735in}{2.086483in}}{\pgfqpoint{2.916735in}{2.078247in}}%
\pgfpathcurveto{\pgfqpoint{2.916735in}{2.070011in}}{\pgfqpoint{2.920008in}{2.062111in}}{\pgfqpoint{2.925832in}{2.056287in}}%
\pgfpathcurveto{\pgfqpoint{2.931656in}{2.050463in}}{\pgfqpoint{2.939556in}{2.047190in}}{\pgfqpoint{2.947792in}{2.047190in}}%
\pgfpathclose%
\pgfusepath{stroke,fill}%
\end{pgfscope}%
\begin{pgfscope}%
\pgfpathrectangle{\pgfqpoint{0.100000in}{0.220728in}}{\pgfqpoint{3.696000in}{3.696000in}}%
\pgfusepath{clip}%
\pgfsetbuttcap%
\pgfsetroundjoin%
\definecolor{currentfill}{rgb}{0.121569,0.466667,0.705882}%
\pgfsetfillcolor{currentfill}%
\pgfsetfillopacity{0.828903}%
\pgfsetlinewidth{1.003750pt}%
\definecolor{currentstroke}{rgb}{0.121569,0.466667,0.705882}%
\pgfsetstrokecolor{currentstroke}%
\pgfsetstrokeopacity{0.828903}%
\pgfsetdash{}{0pt}%
\pgfpathmoveto{\pgfqpoint{1.484534in}{1.066717in}}%
\pgfpathcurveto{\pgfqpoint{1.492770in}{1.066717in}}{\pgfqpoint{1.500670in}{1.069989in}}{\pgfqpoint{1.506494in}{1.075813in}}%
\pgfpathcurveto{\pgfqpoint{1.512318in}{1.081637in}}{\pgfqpoint{1.515590in}{1.089537in}}{\pgfqpoint{1.515590in}{1.097773in}}%
\pgfpathcurveto{\pgfqpoint{1.515590in}{1.106009in}}{\pgfqpoint{1.512318in}{1.113910in}}{\pgfqpoint{1.506494in}{1.119733in}}%
\pgfpathcurveto{\pgfqpoint{1.500670in}{1.125557in}}{\pgfqpoint{1.492770in}{1.128830in}}{\pgfqpoint{1.484534in}{1.128830in}}%
\pgfpathcurveto{\pgfqpoint{1.476297in}{1.128830in}}{\pgfqpoint{1.468397in}{1.125557in}}{\pgfqpoint{1.462573in}{1.119733in}}%
\pgfpathcurveto{\pgfqpoint{1.456749in}{1.113910in}}{\pgfqpoint{1.453477in}{1.106009in}}{\pgfqpoint{1.453477in}{1.097773in}}%
\pgfpathcurveto{\pgfqpoint{1.453477in}{1.089537in}}{\pgfqpoint{1.456749in}{1.081637in}}{\pgfqpoint{1.462573in}{1.075813in}}%
\pgfpathcurveto{\pgfqpoint{1.468397in}{1.069989in}}{\pgfqpoint{1.476297in}{1.066717in}}{\pgfqpoint{1.484534in}{1.066717in}}%
\pgfpathclose%
\pgfusepath{stroke,fill}%
\end{pgfscope}%
\begin{pgfscope}%
\pgfpathrectangle{\pgfqpoint{0.100000in}{0.220728in}}{\pgfqpoint{3.696000in}{3.696000in}}%
\pgfusepath{clip}%
\pgfsetbuttcap%
\pgfsetroundjoin%
\definecolor{currentfill}{rgb}{0.121569,0.466667,0.705882}%
\pgfsetfillcolor{currentfill}%
\pgfsetfillopacity{0.829353}%
\pgfsetlinewidth{1.003750pt}%
\definecolor{currentstroke}{rgb}{0.121569,0.466667,0.705882}%
\pgfsetstrokecolor{currentstroke}%
\pgfsetstrokeopacity{0.829353}%
\pgfsetdash{}{0pt}%
\pgfpathmoveto{\pgfqpoint{2.944972in}{2.042685in}}%
\pgfpathcurveto{\pgfqpoint{2.953208in}{2.042685in}}{\pgfqpoint{2.961108in}{2.045957in}}{\pgfqpoint{2.966932in}{2.051781in}}%
\pgfpathcurveto{\pgfqpoint{2.972756in}{2.057605in}}{\pgfqpoint{2.976028in}{2.065505in}}{\pgfqpoint{2.976028in}{2.073742in}}%
\pgfpathcurveto{\pgfqpoint{2.976028in}{2.081978in}}{\pgfqpoint{2.972756in}{2.089878in}}{\pgfqpoint{2.966932in}{2.095702in}}%
\pgfpathcurveto{\pgfqpoint{2.961108in}{2.101526in}}{\pgfqpoint{2.953208in}{2.104798in}}{\pgfqpoint{2.944972in}{2.104798in}}%
\pgfpathcurveto{\pgfqpoint{2.936735in}{2.104798in}}{\pgfqpoint{2.928835in}{2.101526in}}{\pgfqpoint{2.923011in}{2.095702in}}%
\pgfpathcurveto{\pgfqpoint{2.917187in}{2.089878in}}{\pgfqpoint{2.913915in}{2.081978in}}{\pgfqpoint{2.913915in}{2.073742in}}%
\pgfpathcurveto{\pgfqpoint{2.913915in}{2.065505in}}{\pgfqpoint{2.917187in}{2.057605in}}{\pgfqpoint{2.923011in}{2.051781in}}%
\pgfpathcurveto{\pgfqpoint{2.928835in}{2.045957in}}{\pgfqpoint{2.936735in}{2.042685in}}{\pgfqpoint{2.944972in}{2.042685in}}%
\pgfpathclose%
\pgfusepath{stroke,fill}%
\end{pgfscope}%
\begin{pgfscope}%
\pgfpathrectangle{\pgfqpoint{0.100000in}{0.220728in}}{\pgfqpoint{3.696000in}{3.696000in}}%
\pgfusepath{clip}%
\pgfsetbuttcap%
\pgfsetroundjoin%
\definecolor{currentfill}{rgb}{0.121569,0.466667,0.705882}%
\pgfsetfillcolor{currentfill}%
\pgfsetfillopacity{0.830147}%
\pgfsetlinewidth{1.003750pt}%
\definecolor{currentstroke}{rgb}{0.121569,0.466667,0.705882}%
\pgfsetstrokecolor{currentstroke}%
\pgfsetstrokeopacity{0.830147}%
\pgfsetdash{}{0pt}%
\pgfpathmoveto{\pgfqpoint{1.491044in}{1.063788in}}%
\pgfpathcurveto{\pgfqpoint{1.499280in}{1.063788in}}{\pgfqpoint{1.507180in}{1.067061in}}{\pgfqpoint{1.513004in}{1.072885in}}%
\pgfpathcurveto{\pgfqpoint{1.518828in}{1.078709in}}{\pgfqpoint{1.522100in}{1.086609in}}{\pgfqpoint{1.522100in}{1.094845in}}%
\pgfpathcurveto{\pgfqpoint{1.522100in}{1.103081in}}{\pgfqpoint{1.518828in}{1.110981in}}{\pgfqpoint{1.513004in}{1.116805in}}%
\pgfpathcurveto{\pgfqpoint{1.507180in}{1.122629in}}{\pgfqpoint{1.499280in}{1.125901in}}{\pgfqpoint{1.491044in}{1.125901in}}%
\pgfpathcurveto{\pgfqpoint{1.482808in}{1.125901in}}{\pgfqpoint{1.474908in}{1.122629in}}{\pgfqpoint{1.469084in}{1.116805in}}%
\pgfpathcurveto{\pgfqpoint{1.463260in}{1.110981in}}{\pgfqpoint{1.459987in}{1.103081in}}{\pgfqpoint{1.459987in}{1.094845in}}%
\pgfpathcurveto{\pgfqpoint{1.459987in}{1.086609in}}{\pgfqpoint{1.463260in}{1.078709in}}{\pgfqpoint{1.469084in}{1.072885in}}%
\pgfpathcurveto{\pgfqpoint{1.474908in}{1.067061in}}{\pgfqpoint{1.482808in}{1.063788in}}{\pgfqpoint{1.491044in}{1.063788in}}%
\pgfpathclose%
\pgfusepath{stroke,fill}%
\end{pgfscope}%
\begin{pgfscope}%
\pgfpathrectangle{\pgfqpoint{0.100000in}{0.220728in}}{\pgfqpoint{3.696000in}{3.696000in}}%
\pgfusepath{clip}%
\pgfsetbuttcap%
\pgfsetroundjoin%
\definecolor{currentfill}{rgb}{0.121569,0.466667,0.705882}%
\pgfsetfillcolor{currentfill}%
\pgfsetfillopacity{0.830652}%
\pgfsetlinewidth{1.003750pt}%
\definecolor{currentstroke}{rgb}{0.121569,0.466667,0.705882}%
\pgfsetstrokecolor{currentstroke}%
\pgfsetstrokeopacity{0.830652}%
\pgfsetdash{}{0pt}%
\pgfpathmoveto{\pgfqpoint{2.942548in}{2.035220in}}%
\pgfpathcurveto{\pgfqpoint{2.950785in}{2.035220in}}{\pgfqpoint{2.958685in}{2.038492in}}{\pgfqpoint{2.964509in}{2.044316in}}%
\pgfpathcurveto{\pgfqpoint{2.970333in}{2.050140in}}{\pgfqpoint{2.973605in}{2.058040in}}{\pgfqpoint{2.973605in}{2.066276in}}%
\pgfpathcurveto{\pgfqpoint{2.973605in}{2.074512in}}{\pgfqpoint{2.970333in}{2.082413in}}{\pgfqpoint{2.964509in}{2.088236in}}%
\pgfpathcurveto{\pgfqpoint{2.958685in}{2.094060in}}{\pgfqpoint{2.950785in}{2.097333in}}{\pgfqpoint{2.942548in}{2.097333in}}%
\pgfpathcurveto{\pgfqpoint{2.934312in}{2.097333in}}{\pgfqpoint{2.926412in}{2.094060in}}{\pgfqpoint{2.920588in}{2.088236in}}%
\pgfpathcurveto{\pgfqpoint{2.914764in}{2.082413in}}{\pgfqpoint{2.911492in}{2.074512in}}{\pgfqpoint{2.911492in}{2.066276in}}%
\pgfpathcurveto{\pgfqpoint{2.911492in}{2.058040in}}{\pgfqpoint{2.914764in}{2.050140in}}{\pgfqpoint{2.920588in}{2.044316in}}%
\pgfpathcurveto{\pgfqpoint{2.926412in}{2.038492in}}{\pgfqpoint{2.934312in}{2.035220in}}{\pgfqpoint{2.942548in}{2.035220in}}%
\pgfpathclose%
\pgfusepath{stroke,fill}%
\end{pgfscope}%
\begin{pgfscope}%
\pgfpathrectangle{\pgfqpoint{0.100000in}{0.220728in}}{\pgfqpoint{3.696000in}{3.696000in}}%
\pgfusepath{clip}%
\pgfsetbuttcap%
\pgfsetroundjoin%
\definecolor{currentfill}{rgb}{0.121569,0.466667,0.705882}%
\pgfsetfillcolor{currentfill}%
\pgfsetfillopacity{0.831954}%
\pgfsetlinewidth{1.003750pt}%
\definecolor{currentstroke}{rgb}{0.121569,0.466667,0.705882}%
\pgfsetstrokecolor{currentstroke}%
\pgfsetstrokeopacity{0.831954}%
\pgfsetdash{}{0pt}%
\pgfpathmoveto{\pgfqpoint{2.938543in}{2.027210in}}%
\pgfpathcurveto{\pgfqpoint{2.946779in}{2.027210in}}{\pgfqpoint{2.954679in}{2.030482in}}{\pgfqpoint{2.960503in}{2.036306in}}%
\pgfpathcurveto{\pgfqpoint{2.966327in}{2.042130in}}{\pgfqpoint{2.969600in}{2.050030in}}{\pgfqpoint{2.969600in}{2.058266in}}%
\pgfpathcurveto{\pgfqpoint{2.969600in}{2.066502in}}{\pgfqpoint{2.966327in}{2.074403in}}{\pgfqpoint{2.960503in}{2.080226in}}%
\pgfpathcurveto{\pgfqpoint{2.954679in}{2.086050in}}{\pgfqpoint{2.946779in}{2.089323in}}{\pgfqpoint{2.938543in}{2.089323in}}%
\pgfpathcurveto{\pgfqpoint{2.930307in}{2.089323in}}{\pgfqpoint{2.922407in}{2.086050in}}{\pgfqpoint{2.916583in}{2.080226in}}%
\pgfpathcurveto{\pgfqpoint{2.910759in}{2.074403in}}{\pgfqpoint{2.907487in}{2.066502in}}{\pgfqpoint{2.907487in}{2.058266in}}%
\pgfpathcurveto{\pgfqpoint{2.907487in}{2.050030in}}{\pgfqpoint{2.910759in}{2.042130in}}{\pgfqpoint{2.916583in}{2.036306in}}%
\pgfpathcurveto{\pgfqpoint{2.922407in}{2.030482in}}{\pgfqpoint{2.930307in}{2.027210in}}{\pgfqpoint{2.938543in}{2.027210in}}%
\pgfpathclose%
\pgfusepath{stroke,fill}%
\end{pgfscope}%
\begin{pgfscope}%
\pgfpathrectangle{\pgfqpoint{0.100000in}{0.220728in}}{\pgfqpoint{3.696000in}{3.696000in}}%
\pgfusepath{clip}%
\pgfsetbuttcap%
\pgfsetroundjoin%
\definecolor{currentfill}{rgb}{0.121569,0.466667,0.705882}%
\pgfsetfillcolor{currentfill}%
\pgfsetfillopacity{0.832752}%
\pgfsetlinewidth{1.003750pt}%
\definecolor{currentstroke}{rgb}{0.121569,0.466667,0.705882}%
\pgfsetstrokecolor{currentstroke}%
\pgfsetstrokeopacity{0.832752}%
\pgfsetdash{}{0pt}%
\pgfpathmoveto{\pgfqpoint{2.936195in}{2.023339in}}%
\pgfpathcurveto{\pgfqpoint{2.944432in}{2.023339in}}{\pgfqpoint{2.952332in}{2.026611in}}{\pgfqpoint{2.958156in}{2.032435in}}%
\pgfpathcurveto{\pgfqpoint{2.963980in}{2.038259in}}{\pgfqpoint{2.967252in}{2.046159in}}{\pgfqpoint{2.967252in}{2.054395in}}%
\pgfpathcurveto{\pgfqpoint{2.967252in}{2.062631in}}{\pgfqpoint{2.963980in}{2.070531in}}{\pgfqpoint{2.958156in}{2.076355in}}%
\pgfpathcurveto{\pgfqpoint{2.952332in}{2.082179in}}{\pgfqpoint{2.944432in}{2.085452in}}{\pgfqpoint{2.936195in}{2.085452in}}%
\pgfpathcurveto{\pgfqpoint{2.927959in}{2.085452in}}{\pgfqpoint{2.920059in}{2.082179in}}{\pgfqpoint{2.914235in}{2.076355in}}%
\pgfpathcurveto{\pgfqpoint{2.908411in}{2.070531in}}{\pgfqpoint{2.905139in}{2.062631in}}{\pgfqpoint{2.905139in}{2.054395in}}%
\pgfpathcurveto{\pgfqpoint{2.905139in}{2.046159in}}{\pgfqpoint{2.908411in}{2.038259in}}{\pgfqpoint{2.914235in}{2.032435in}}%
\pgfpathcurveto{\pgfqpoint{2.920059in}{2.026611in}}{\pgfqpoint{2.927959in}{2.023339in}}{\pgfqpoint{2.936195in}{2.023339in}}%
\pgfpathclose%
\pgfusepath{stroke,fill}%
\end{pgfscope}%
\begin{pgfscope}%
\pgfpathrectangle{\pgfqpoint{0.100000in}{0.220728in}}{\pgfqpoint{3.696000in}{3.696000in}}%
\pgfusepath{clip}%
\pgfsetbuttcap%
\pgfsetroundjoin%
\definecolor{currentfill}{rgb}{0.121569,0.466667,0.705882}%
\pgfsetfillcolor{currentfill}%
\pgfsetfillopacity{0.832777}%
\pgfsetlinewidth{1.003750pt}%
\definecolor{currentstroke}{rgb}{0.121569,0.466667,0.705882}%
\pgfsetstrokecolor{currentstroke}%
\pgfsetstrokeopacity{0.832777}%
\pgfsetdash{}{0pt}%
\pgfpathmoveto{\pgfqpoint{1.502940in}{1.060156in}}%
\pgfpathcurveto{\pgfqpoint{1.511177in}{1.060156in}}{\pgfqpoint{1.519077in}{1.063428in}}{\pgfqpoint{1.524901in}{1.069252in}}%
\pgfpathcurveto{\pgfqpoint{1.530725in}{1.075076in}}{\pgfqpoint{1.533997in}{1.082976in}}{\pgfqpoint{1.533997in}{1.091212in}}%
\pgfpathcurveto{\pgfqpoint{1.533997in}{1.099449in}}{\pgfqpoint{1.530725in}{1.107349in}}{\pgfqpoint{1.524901in}{1.113173in}}%
\pgfpathcurveto{\pgfqpoint{1.519077in}{1.118996in}}{\pgfqpoint{1.511177in}{1.122269in}}{\pgfqpoint{1.502940in}{1.122269in}}%
\pgfpathcurveto{\pgfqpoint{1.494704in}{1.122269in}}{\pgfqpoint{1.486804in}{1.118996in}}{\pgfqpoint{1.480980in}{1.113173in}}%
\pgfpathcurveto{\pgfqpoint{1.475156in}{1.107349in}}{\pgfqpoint{1.471884in}{1.099449in}}{\pgfqpoint{1.471884in}{1.091212in}}%
\pgfpathcurveto{\pgfqpoint{1.471884in}{1.082976in}}{\pgfqpoint{1.475156in}{1.075076in}}{\pgfqpoint{1.480980in}{1.069252in}}%
\pgfpathcurveto{\pgfqpoint{1.486804in}{1.063428in}}{\pgfqpoint{1.494704in}{1.060156in}}{\pgfqpoint{1.502940in}{1.060156in}}%
\pgfpathclose%
\pgfusepath{stroke,fill}%
\end{pgfscope}%
\begin{pgfscope}%
\pgfpathrectangle{\pgfqpoint{0.100000in}{0.220728in}}{\pgfqpoint{3.696000in}{3.696000in}}%
\pgfusepath{clip}%
\pgfsetbuttcap%
\pgfsetroundjoin%
\definecolor{currentfill}{rgb}{0.121569,0.466667,0.705882}%
\pgfsetfillcolor{currentfill}%
\pgfsetfillopacity{0.833657}%
\pgfsetlinewidth{1.003750pt}%
\definecolor{currentstroke}{rgb}{0.121569,0.466667,0.705882}%
\pgfsetstrokecolor{currentstroke}%
\pgfsetstrokeopacity{0.833657}%
\pgfsetdash{}{0pt}%
\pgfpathmoveto{\pgfqpoint{2.934557in}{2.017734in}}%
\pgfpathcurveto{\pgfqpoint{2.942793in}{2.017734in}}{\pgfqpoint{2.950693in}{2.021006in}}{\pgfqpoint{2.956517in}{2.026830in}}%
\pgfpathcurveto{\pgfqpoint{2.962341in}{2.032654in}}{\pgfqpoint{2.965613in}{2.040554in}}{\pgfqpoint{2.965613in}{2.048790in}}%
\pgfpathcurveto{\pgfqpoint{2.965613in}{2.057026in}}{\pgfqpoint{2.962341in}{2.064926in}}{\pgfqpoint{2.956517in}{2.070750in}}%
\pgfpathcurveto{\pgfqpoint{2.950693in}{2.076574in}}{\pgfqpoint{2.942793in}{2.079847in}}{\pgfqpoint{2.934557in}{2.079847in}}%
\pgfpathcurveto{\pgfqpoint{2.926320in}{2.079847in}}{\pgfqpoint{2.918420in}{2.076574in}}{\pgfqpoint{2.912596in}{2.070750in}}%
\pgfpathcurveto{\pgfqpoint{2.906773in}{2.064926in}}{\pgfqpoint{2.903500in}{2.057026in}}{\pgfqpoint{2.903500in}{2.048790in}}%
\pgfpathcurveto{\pgfqpoint{2.903500in}{2.040554in}}{\pgfqpoint{2.906773in}{2.032654in}}{\pgfqpoint{2.912596in}{2.026830in}}%
\pgfpathcurveto{\pgfqpoint{2.918420in}{2.021006in}}{\pgfqpoint{2.926320in}{2.017734in}}{\pgfqpoint{2.934557in}{2.017734in}}%
\pgfpathclose%
\pgfusepath{stroke,fill}%
\end{pgfscope}%
\begin{pgfscope}%
\pgfpathrectangle{\pgfqpoint{0.100000in}{0.220728in}}{\pgfqpoint{3.696000in}{3.696000in}}%
\pgfusepath{clip}%
\pgfsetbuttcap%
\pgfsetroundjoin%
\definecolor{currentfill}{rgb}{0.121569,0.466667,0.705882}%
\pgfsetfillcolor{currentfill}%
\pgfsetfillopacity{0.834722}%
\pgfsetlinewidth{1.003750pt}%
\definecolor{currentstroke}{rgb}{0.121569,0.466667,0.705882}%
\pgfsetstrokecolor{currentstroke}%
\pgfsetstrokeopacity{0.834722}%
\pgfsetdash{}{0pt}%
\pgfpathmoveto{\pgfqpoint{2.931122in}{2.011134in}}%
\pgfpathcurveto{\pgfqpoint{2.939359in}{2.011134in}}{\pgfqpoint{2.947259in}{2.014406in}}{\pgfqpoint{2.953083in}{2.020230in}}%
\pgfpathcurveto{\pgfqpoint{2.958907in}{2.026054in}}{\pgfqpoint{2.962179in}{2.033954in}}{\pgfqpoint{2.962179in}{2.042191in}}%
\pgfpathcurveto{\pgfqpoint{2.962179in}{2.050427in}}{\pgfqpoint{2.958907in}{2.058327in}}{\pgfqpoint{2.953083in}{2.064151in}}%
\pgfpathcurveto{\pgfqpoint{2.947259in}{2.069975in}}{\pgfqpoint{2.939359in}{2.073247in}}{\pgfqpoint{2.931122in}{2.073247in}}%
\pgfpathcurveto{\pgfqpoint{2.922886in}{2.073247in}}{\pgfqpoint{2.914986in}{2.069975in}}{\pgfqpoint{2.909162in}{2.064151in}}%
\pgfpathcurveto{\pgfqpoint{2.903338in}{2.058327in}}{\pgfqpoint{2.900066in}{2.050427in}}{\pgfqpoint{2.900066in}{2.042191in}}%
\pgfpathcurveto{\pgfqpoint{2.900066in}{2.033954in}}{\pgfqpoint{2.903338in}{2.026054in}}{\pgfqpoint{2.909162in}{2.020230in}}%
\pgfpathcurveto{\pgfqpoint{2.914986in}{2.014406in}}{\pgfqpoint{2.922886in}{2.011134in}}{\pgfqpoint{2.931122in}{2.011134in}}%
\pgfpathclose%
\pgfusepath{stroke,fill}%
\end{pgfscope}%
\begin{pgfscope}%
\pgfpathrectangle{\pgfqpoint{0.100000in}{0.220728in}}{\pgfqpoint{3.696000in}{3.696000in}}%
\pgfusepath{clip}%
\pgfsetbuttcap%
\pgfsetroundjoin%
\definecolor{currentfill}{rgb}{0.121569,0.466667,0.705882}%
\pgfsetfillcolor{currentfill}%
\pgfsetfillopacity{0.835391}%
\pgfsetlinewidth{1.003750pt}%
\definecolor{currentstroke}{rgb}{0.121569,0.466667,0.705882}%
\pgfsetstrokecolor{currentstroke}%
\pgfsetstrokeopacity{0.835391}%
\pgfsetdash{}{0pt}%
\pgfpathmoveto{\pgfqpoint{2.929569in}{2.007457in}}%
\pgfpathcurveto{\pgfqpoint{2.937806in}{2.007457in}}{\pgfqpoint{2.945706in}{2.010729in}}{\pgfqpoint{2.951530in}{2.016553in}}%
\pgfpathcurveto{\pgfqpoint{2.957353in}{2.022377in}}{\pgfqpoint{2.960626in}{2.030277in}}{\pgfqpoint{2.960626in}{2.038513in}}%
\pgfpathcurveto{\pgfqpoint{2.960626in}{2.046750in}}{\pgfqpoint{2.957353in}{2.054650in}}{\pgfqpoint{2.951530in}{2.060473in}}%
\pgfpathcurveto{\pgfqpoint{2.945706in}{2.066297in}}{\pgfqpoint{2.937806in}{2.069570in}}{\pgfqpoint{2.929569in}{2.069570in}}%
\pgfpathcurveto{\pgfqpoint{2.921333in}{2.069570in}}{\pgfqpoint{2.913433in}{2.066297in}}{\pgfqpoint{2.907609in}{2.060473in}}%
\pgfpathcurveto{\pgfqpoint{2.901785in}{2.054650in}}{\pgfqpoint{2.898513in}{2.046750in}}{\pgfqpoint{2.898513in}{2.038513in}}%
\pgfpathcurveto{\pgfqpoint{2.898513in}{2.030277in}}{\pgfqpoint{2.901785in}{2.022377in}}{\pgfqpoint{2.907609in}{2.016553in}}%
\pgfpathcurveto{\pgfqpoint{2.913433in}{2.010729in}}{\pgfqpoint{2.921333in}{2.007457in}}{\pgfqpoint{2.929569in}{2.007457in}}%
\pgfpathclose%
\pgfusepath{stroke,fill}%
\end{pgfscope}%
\begin{pgfscope}%
\pgfpathrectangle{\pgfqpoint{0.100000in}{0.220728in}}{\pgfqpoint{3.696000in}{3.696000in}}%
\pgfusepath{clip}%
\pgfsetbuttcap%
\pgfsetroundjoin%
\definecolor{currentfill}{rgb}{0.121569,0.466667,0.705882}%
\pgfsetfillcolor{currentfill}%
\pgfsetfillopacity{0.835772}%
\pgfsetlinewidth{1.003750pt}%
\definecolor{currentstroke}{rgb}{0.121569,0.466667,0.705882}%
\pgfsetstrokecolor{currentstroke}%
\pgfsetstrokeopacity{0.835772}%
\pgfsetdash{}{0pt}%
\pgfpathmoveto{\pgfqpoint{2.928861in}{2.005361in}}%
\pgfpathcurveto{\pgfqpoint{2.937097in}{2.005361in}}{\pgfqpoint{2.944997in}{2.008634in}}{\pgfqpoint{2.950821in}{2.014458in}}%
\pgfpathcurveto{\pgfqpoint{2.956645in}{2.020282in}}{\pgfqpoint{2.959917in}{2.028182in}}{\pgfqpoint{2.959917in}{2.036418in}}%
\pgfpathcurveto{\pgfqpoint{2.959917in}{2.044654in}}{\pgfqpoint{2.956645in}{2.052554in}}{\pgfqpoint{2.950821in}{2.058378in}}%
\pgfpathcurveto{\pgfqpoint{2.944997in}{2.064202in}}{\pgfqpoint{2.937097in}{2.067474in}}{\pgfqpoint{2.928861in}{2.067474in}}%
\pgfpathcurveto{\pgfqpoint{2.920625in}{2.067474in}}{\pgfqpoint{2.912724in}{2.064202in}}{\pgfqpoint{2.906901in}{2.058378in}}%
\pgfpathcurveto{\pgfqpoint{2.901077in}{2.052554in}}{\pgfqpoint{2.897804in}{2.044654in}}{\pgfqpoint{2.897804in}{2.036418in}}%
\pgfpathcurveto{\pgfqpoint{2.897804in}{2.028182in}}{\pgfqpoint{2.901077in}{2.020282in}}{\pgfqpoint{2.906901in}{2.014458in}}%
\pgfpathcurveto{\pgfqpoint{2.912724in}{2.008634in}}{\pgfqpoint{2.920625in}{2.005361in}}{\pgfqpoint{2.928861in}{2.005361in}}%
\pgfpathclose%
\pgfusepath{stroke,fill}%
\end{pgfscope}%
\begin{pgfscope}%
\pgfpathrectangle{\pgfqpoint{0.100000in}{0.220728in}}{\pgfqpoint{3.696000in}{3.696000in}}%
\pgfusepath{clip}%
\pgfsetbuttcap%
\pgfsetroundjoin%
\definecolor{currentfill}{rgb}{0.121569,0.466667,0.705882}%
\pgfsetfillcolor{currentfill}%
\pgfsetfillopacity{0.835949}%
\pgfsetlinewidth{1.003750pt}%
\definecolor{currentstroke}{rgb}{0.121569,0.466667,0.705882}%
\pgfsetstrokecolor{currentstroke}%
\pgfsetstrokeopacity{0.835949}%
\pgfsetdash{}{0pt}%
\pgfpathmoveto{\pgfqpoint{2.928246in}{2.004324in}}%
\pgfpathcurveto{\pgfqpoint{2.936482in}{2.004324in}}{\pgfqpoint{2.944382in}{2.007597in}}{\pgfqpoint{2.950206in}{2.013421in}}%
\pgfpathcurveto{\pgfqpoint{2.956030in}{2.019245in}}{\pgfqpoint{2.959302in}{2.027145in}}{\pgfqpoint{2.959302in}{2.035381in}}%
\pgfpathcurveto{\pgfqpoint{2.959302in}{2.043617in}}{\pgfqpoint{2.956030in}{2.051517in}}{\pgfqpoint{2.950206in}{2.057341in}}%
\pgfpathcurveto{\pgfqpoint{2.944382in}{2.063165in}}{\pgfqpoint{2.936482in}{2.066437in}}{\pgfqpoint{2.928246in}{2.066437in}}%
\pgfpathcurveto{\pgfqpoint{2.920009in}{2.066437in}}{\pgfqpoint{2.912109in}{2.063165in}}{\pgfqpoint{2.906285in}{2.057341in}}%
\pgfpathcurveto{\pgfqpoint{2.900462in}{2.051517in}}{\pgfqpoint{2.897189in}{2.043617in}}{\pgfqpoint{2.897189in}{2.035381in}}%
\pgfpathcurveto{\pgfqpoint{2.897189in}{2.027145in}}{\pgfqpoint{2.900462in}{2.019245in}}{\pgfqpoint{2.906285in}{2.013421in}}%
\pgfpathcurveto{\pgfqpoint{2.912109in}{2.007597in}}{\pgfqpoint{2.920009in}{2.004324in}}{\pgfqpoint{2.928246in}{2.004324in}}%
\pgfpathclose%
\pgfusepath{stroke,fill}%
\end{pgfscope}%
\begin{pgfscope}%
\pgfpathrectangle{\pgfqpoint{0.100000in}{0.220728in}}{\pgfqpoint{3.696000in}{3.696000in}}%
\pgfusepath{clip}%
\pgfsetbuttcap%
\pgfsetroundjoin%
\definecolor{currentfill}{rgb}{0.121569,0.466667,0.705882}%
\pgfsetfillcolor{currentfill}%
\pgfsetfillopacity{0.836425}%
\pgfsetlinewidth{1.003750pt}%
\definecolor{currentstroke}{rgb}{0.121569,0.466667,0.705882}%
\pgfsetstrokecolor{currentstroke}%
\pgfsetstrokeopacity{0.836425}%
\pgfsetdash{}{0pt}%
\pgfpathmoveto{\pgfqpoint{2.927342in}{2.001513in}}%
\pgfpathcurveto{\pgfqpoint{2.935579in}{2.001513in}}{\pgfqpoint{2.943479in}{2.004785in}}{\pgfqpoint{2.949303in}{2.010609in}}%
\pgfpathcurveto{\pgfqpoint{2.955127in}{2.016433in}}{\pgfqpoint{2.958399in}{2.024333in}}{\pgfqpoint{2.958399in}{2.032570in}}%
\pgfpathcurveto{\pgfqpoint{2.958399in}{2.040806in}}{\pgfqpoint{2.955127in}{2.048706in}}{\pgfqpoint{2.949303in}{2.054530in}}%
\pgfpathcurveto{\pgfqpoint{2.943479in}{2.060354in}}{\pgfqpoint{2.935579in}{2.063626in}}{\pgfqpoint{2.927342in}{2.063626in}}%
\pgfpathcurveto{\pgfqpoint{2.919106in}{2.063626in}}{\pgfqpoint{2.911206in}{2.060354in}}{\pgfqpoint{2.905382in}{2.054530in}}%
\pgfpathcurveto{\pgfqpoint{2.899558in}{2.048706in}}{\pgfqpoint{2.896286in}{2.040806in}}{\pgfqpoint{2.896286in}{2.032570in}}%
\pgfpathcurveto{\pgfqpoint{2.896286in}{2.024333in}}{\pgfqpoint{2.899558in}{2.016433in}}{\pgfqpoint{2.905382in}{2.010609in}}%
\pgfpathcurveto{\pgfqpoint{2.911206in}{2.004785in}}{\pgfqpoint{2.919106in}{2.001513in}}{\pgfqpoint{2.927342in}{2.001513in}}%
\pgfpathclose%
\pgfusepath{stroke,fill}%
\end{pgfscope}%
\begin{pgfscope}%
\pgfpathrectangle{\pgfqpoint{0.100000in}{0.220728in}}{\pgfqpoint{3.696000in}{3.696000in}}%
\pgfusepath{clip}%
\pgfsetbuttcap%
\pgfsetroundjoin%
\definecolor{currentfill}{rgb}{0.121569,0.466667,0.705882}%
\pgfsetfillcolor{currentfill}%
\pgfsetfillopacity{0.836667}%
\pgfsetlinewidth{1.003750pt}%
\definecolor{currentstroke}{rgb}{0.121569,0.466667,0.705882}%
\pgfsetstrokecolor{currentstroke}%
\pgfsetstrokeopacity{0.836667}%
\pgfsetdash{}{0pt}%
\pgfpathmoveto{\pgfqpoint{2.926701in}{2.000018in}}%
\pgfpathcurveto{\pgfqpoint{2.934937in}{2.000018in}}{\pgfqpoint{2.942837in}{2.003291in}}{\pgfqpoint{2.948661in}{2.009115in}}%
\pgfpathcurveto{\pgfqpoint{2.954485in}{2.014938in}}{\pgfqpoint{2.957757in}{2.022839in}}{\pgfqpoint{2.957757in}{2.031075in}}%
\pgfpathcurveto{\pgfqpoint{2.957757in}{2.039311in}}{\pgfqpoint{2.954485in}{2.047211in}}{\pgfqpoint{2.948661in}{2.053035in}}%
\pgfpathcurveto{\pgfqpoint{2.942837in}{2.058859in}}{\pgfqpoint{2.934937in}{2.062131in}}{\pgfqpoint{2.926701in}{2.062131in}}%
\pgfpathcurveto{\pgfqpoint{2.918464in}{2.062131in}}{\pgfqpoint{2.910564in}{2.058859in}}{\pgfqpoint{2.904740in}{2.053035in}}%
\pgfpathcurveto{\pgfqpoint{2.898916in}{2.047211in}}{\pgfqpoint{2.895644in}{2.039311in}}{\pgfqpoint{2.895644in}{2.031075in}}%
\pgfpathcurveto{\pgfqpoint{2.895644in}{2.022839in}}{\pgfqpoint{2.898916in}{2.014938in}}{\pgfqpoint{2.904740in}{2.009115in}}%
\pgfpathcurveto{\pgfqpoint{2.910564in}{2.003291in}}{\pgfqpoint{2.918464in}{2.000018in}}{\pgfqpoint{2.926701in}{2.000018in}}%
\pgfpathclose%
\pgfusepath{stroke,fill}%
\end{pgfscope}%
\begin{pgfscope}%
\pgfpathrectangle{\pgfqpoint{0.100000in}{0.220728in}}{\pgfqpoint{3.696000in}{3.696000in}}%
\pgfusepath{clip}%
\pgfsetbuttcap%
\pgfsetroundjoin%
\definecolor{currentfill}{rgb}{0.121569,0.466667,0.705882}%
\pgfsetfillcolor{currentfill}%
\pgfsetfillopacity{0.837131}%
\pgfsetlinewidth{1.003750pt}%
\definecolor{currentstroke}{rgb}{0.121569,0.466667,0.705882}%
\pgfsetstrokecolor{currentstroke}%
\pgfsetstrokeopacity{0.837131}%
\pgfsetdash{}{0pt}%
\pgfpathmoveto{\pgfqpoint{2.925324in}{1.997742in}}%
\pgfpathcurveto{\pgfqpoint{2.933560in}{1.997742in}}{\pgfqpoint{2.941460in}{2.001014in}}{\pgfqpoint{2.947284in}{2.006838in}}%
\pgfpathcurveto{\pgfqpoint{2.953108in}{2.012662in}}{\pgfqpoint{2.956381in}{2.020562in}}{\pgfqpoint{2.956381in}{2.028798in}}%
\pgfpathcurveto{\pgfqpoint{2.956381in}{2.037035in}}{\pgfqpoint{2.953108in}{2.044935in}}{\pgfqpoint{2.947284in}{2.050758in}}%
\pgfpathcurveto{\pgfqpoint{2.941460in}{2.056582in}}{\pgfqpoint{2.933560in}{2.059855in}}{\pgfqpoint{2.925324in}{2.059855in}}%
\pgfpathcurveto{\pgfqpoint{2.917088in}{2.059855in}}{\pgfqpoint{2.909188in}{2.056582in}}{\pgfqpoint{2.903364in}{2.050758in}}%
\pgfpathcurveto{\pgfqpoint{2.897540in}{2.044935in}}{\pgfqpoint{2.894268in}{2.037035in}}{\pgfqpoint{2.894268in}{2.028798in}}%
\pgfpathcurveto{\pgfqpoint{2.894268in}{2.020562in}}{\pgfqpoint{2.897540in}{2.012662in}}{\pgfqpoint{2.903364in}{2.006838in}}%
\pgfpathcurveto{\pgfqpoint{2.909188in}{2.001014in}}{\pgfqpoint{2.917088in}{1.997742in}}{\pgfqpoint{2.925324in}{1.997742in}}%
\pgfpathclose%
\pgfusepath{stroke,fill}%
\end{pgfscope}%
\begin{pgfscope}%
\pgfpathrectangle{\pgfqpoint{0.100000in}{0.220728in}}{\pgfqpoint{3.696000in}{3.696000in}}%
\pgfusepath{clip}%
\pgfsetbuttcap%
\pgfsetroundjoin%
\definecolor{currentfill}{rgb}{0.121569,0.466667,0.705882}%
\pgfsetfillcolor{currentfill}%
\pgfsetfillopacity{0.837973}%
\pgfsetlinewidth{1.003750pt}%
\definecolor{currentstroke}{rgb}{0.121569,0.466667,0.705882}%
\pgfsetstrokecolor{currentstroke}%
\pgfsetstrokeopacity{0.837973}%
\pgfsetdash{}{0pt}%
\pgfpathmoveto{\pgfqpoint{2.923829in}{1.992407in}}%
\pgfpathcurveto{\pgfqpoint{2.932065in}{1.992407in}}{\pgfqpoint{2.939965in}{1.995679in}}{\pgfqpoint{2.945789in}{2.001503in}}%
\pgfpathcurveto{\pgfqpoint{2.951613in}{2.007327in}}{\pgfqpoint{2.954885in}{2.015227in}}{\pgfqpoint{2.954885in}{2.023464in}}%
\pgfpathcurveto{\pgfqpoint{2.954885in}{2.031700in}}{\pgfqpoint{2.951613in}{2.039600in}}{\pgfqpoint{2.945789in}{2.045424in}}%
\pgfpathcurveto{\pgfqpoint{2.939965in}{2.051248in}}{\pgfqpoint{2.932065in}{2.054520in}}{\pgfqpoint{2.923829in}{2.054520in}}%
\pgfpathcurveto{\pgfqpoint{2.915592in}{2.054520in}}{\pgfqpoint{2.907692in}{2.051248in}}{\pgfqpoint{2.901868in}{2.045424in}}%
\pgfpathcurveto{\pgfqpoint{2.896044in}{2.039600in}}{\pgfqpoint{2.892772in}{2.031700in}}{\pgfqpoint{2.892772in}{2.023464in}}%
\pgfpathcurveto{\pgfqpoint{2.892772in}{2.015227in}}{\pgfqpoint{2.896044in}{2.007327in}}{\pgfqpoint{2.901868in}{2.001503in}}%
\pgfpathcurveto{\pgfqpoint{2.907692in}{1.995679in}}{\pgfqpoint{2.915592in}{1.992407in}}{\pgfqpoint{2.923829in}{1.992407in}}%
\pgfpathclose%
\pgfusepath{stroke,fill}%
\end{pgfscope}%
\begin{pgfscope}%
\pgfpathrectangle{\pgfqpoint{0.100000in}{0.220728in}}{\pgfqpoint{3.696000in}{3.696000in}}%
\pgfusepath{clip}%
\pgfsetbuttcap%
\pgfsetroundjoin%
\definecolor{currentfill}{rgb}{0.121569,0.466667,0.705882}%
\pgfsetfillcolor{currentfill}%
\pgfsetfillopacity{0.838130}%
\pgfsetlinewidth{1.003750pt}%
\definecolor{currentstroke}{rgb}{0.121569,0.466667,0.705882}%
\pgfsetstrokecolor{currentstroke}%
\pgfsetstrokeopacity{0.838130}%
\pgfsetdash{}{0pt}%
\pgfpathmoveto{\pgfqpoint{1.523940in}{1.053709in}}%
\pgfpathcurveto{\pgfqpoint{1.532176in}{1.053709in}}{\pgfqpoint{1.540076in}{1.056981in}}{\pgfqpoint{1.545900in}{1.062805in}}%
\pgfpathcurveto{\pgfqpoint{1.551724in}{1.068629in}}{\pgfqpoint{1.554996in}{1.076529in}}{\pgfqpoint{1.554996in}{1.084765in}}%
\pgfpathcurveto{\pgfqpoint{1.554996in}{1.093002in}}{\pgfqpoint{1.551724in}{1.100902in}}{\pgfqpoint{1.545900in}{1.106726in}}%
\pgfpathcurveto{\pgfqpoint{1.540076in}{1.112549in}}{\pgfqpoint{1.532176in}{1.115822in}}{\pgfqpoint{1.523940in}{1.115822in}}%
\pgfpathcurveto{\pgfqpoint{1.515703in}{1.115822in}}{\pgfqpoint{1.507803in}{1.112549in}}{\pgfqpoint{1.501979in}{1.106726in}}%
\pgfpathcurveto{\pgfqpoint{1.496155in}{1.100902in}}{\pgfqpoint{1.492883in}{1.093002in}}{\pgfqpoint{1.492883in}{1.084765in}}%
\pgfpathcurveto{\pgfqpoint{1.492883in}{1.076529in}}{\pgfqpoint{1.496155in}{1.068629in}}{\pgfqpoint{1.501979in}{1.062805in}}%
\pgfpathcurveto{\pgfqpoint{1.507803in}{1.056981in}}{\pgfqpoint{1.515703in}{1.053709in}}{\pgfqpoint{1.523940in}{1.053709in}}%
\pgfpathclose%
\pgfusepath{stroke,fill}%
\end{pgfscope}%
\begin{pgfscope}%
\pgfpathrectangle{\pgfqpoint{0.100000in}{0.220728in}}{\pgfqpoint{3.696000in}{3.696000in}}%
\pgfusepath{clip}%
\pgfsetbuttcap%
\pgfsetroundjoin%
\definecolor{currentfill}{rgb}{0.121569,0.466667,0.705882}%
\pgfsetfillcolor{currentfill}%
\pgfsetfillopacity{0.838398}%
\pgfsetlinewidth{1.003750pt}%
\definecolor{currentstroke}{rgb}{0.121569,0.466667,0.705882}%
\pgfsetstrokecolor{currentstroke}%
\pgfsetstrokeopacity{0.838398}%
\pgfsetdash{}{0pt}%
\pgfpathmoveto{\pgfqpoint{2.922500in}{1.989798in}}%
\pgfpathcurveto{\pgfqpoint{2.930736in}{1.989798in}}{\pgfqpoint{2.938637in}{1.993070in}}{\pgfqpoint{2.944460in}{1.998894in}}%
\pgfpathcurveto{\pgfqpoint{2.950284in}{2.004718in}}{\pgfqpoint{2.953557in}{2.012618in}}{\pgfqpoint{2.953557in}{2.020854in}}%
\pgfpathcurveto{\pgfqpoint{2.953557in}{2.029090in}}{\pgfqpoint{2.950284in}{2.036990in}}{\pgfqpoint{2.944460in}{2.042814in}}%
\pgfpathcurveto{\pgfqpoint{2.938637in}{2.048638in}}{\pgfqpoint{2.930736in}{2.051911in}}{\pgfqpoint{2.922500in}{2.051911in}}%
\pgfpathcurveto{\pgfqpoint{2.914264in}{2.051911in}}{\pgfqpoint{2.906364in}{2.048638in}}{\pgfqpoint{2.900540in}{2.042814in}}%
\pgfpathcurveto{\pgfqpoint{2.894716in}{2.036990in}}{\pgfqpoint{2.891444in}{2.029090in}}{\pgfqpoint{2.891444in}{2.020854in}}%
\pgfpathcurveto{\pgfqpoint{2.891444in}{2.012618in}}{\pgfqpoint{2.894716in}{2.004718in}}{\pgfqpoint{2.900540in}{1.998894in}}%
\pgfpathcurveto{\pgfqpoint{2.906364in}{1.993070in}}{\pgfqpoint{2.914264in}{1.989798in}}{\pgfqpoint{2.922500in}{1.989798in}}%
\pgfpathclose%
\pgfusepath{stroke,fill}%
\end{pgfscope}%
\begin{pgfscope}%
\pgfpathrectangle{\pgfqpoint{0.100000in}{0.220728in}}{\pgfqpoint{3.696000in}{3.696000in}}%
\pgfusepath{clip}%
\pgfsetbuttcap%
\pgfsetroundjoin%
\definecolor{currentfill}{rgb}{0.121569,0.466667,0.705882}%
\pgfsetfillcolor{currentfill}%
\pgfsetfillopacity{0.839016}%
\pgfsetlinewidth{1.003750pt}%
\definecolor{currentstroke}{rgb}{0.121569,0.466667,0.705882}%
\pgfsetstrokecolor{currentstroke}%
\pgfsetstrokeopacity{0.839016}%
\pgfsetdash{}{0pt}%
\pgfpathmoveto{\pgfqpoint{2.920782in}{1.986055in}}%
\pgfpathcurveto{\pgfqpoint{2.929018in}{1.986055in}}{\pgfqpoint{2.936918in}{1.989327in}}{\pgfqpoint{2.942742in}{1.995151in}}%
\pgfpathcurveto{\pgfqpoint{2.948566in}{2.000975in}}{\pgfqpoint{2.951839in}{2.008875in}}{\pgfqpoint{2.951839in}{2.017111in}}%
\pgfpathcurveto{\pgfqpoint{2.951839in}{2.025347in}}{\pgfqpoint{2.948566in}{2.033248in}}{\pgfqpoint{2.942742in}{2.039071in}}%
\pgfpathcurveto{\pgfqpoint{2.936918in}{2.044895in}}{\pgfqpoint{2.929018in}{2.048168in}}{\pgfqpoint{2.920782in}{2.048168in}}%
\pgfpathcurveto{\pgfqpoint{2.912546in}{2.048168in}}{\pgfqpoint{2.904646in}{2.044895in}}{\pgfqpoint{2.898822in}{2.039071in}}%
\pgfpathcurveto{\pgfqpoint{2.892998in}{2.033248in}}{\pgfqpoint{2.889726in}{2.025347in}}{\pgfqpoint{2.889726in}{2.017111in}}%
\pgfpathcurveto{\pgfqpoint{2.889726in}{2.008875in}}{\pgfqpoint{2.892998in}{2.000975in}}{\pgfqpoint{2.898822in}{1.995151in}}%
\pgfpathcurveto{\pgfqpoint{2.904646in}{1.989327in}}{\pgfqpoint{2.912546in}{1.986055in}}{\pgfqpoint{2.920782in}{1.986055in}}%
\pgfpathclose%
\pgfusepath{stroke,fill}%
\end{pgfscope}%
\begin{pgfscope}%
\pgfpathrectangle{\pgfqpoint{0.100000in}{0.220728in}}{\pgfqpoint{3.696000in}{3.696000in}}%
\pgfusepath{clip}%
\pgfsetbuttcap%
\pgfsetroundjoin%
\definecolor{currentfill}{rgb}{0.121569,0.466667,0.705882}%
\pgfsetfillcolor{currentfill}%
\pgfsetfillopacity{0.839396}%
\pgfsetlinewidth{1.003750pt}%
\definecolor{currentstroke}{rgb}{0.121569,0.466667,0.705882}%
\pgfsetstrokecolor{currentstroke}%
\pgfsetstrokeopacity{0.839396}%
\pgfsetdash{}{0pt}%
\pgfpathmoveto{\pgfqpoint{2.920115in}{1.983907in}}%
\pgfpathcurveto{\pgfqpoint{2.928351in}{1.983907in}}{\pgfqpoint{2.936251in}{1.987179in}}{\pgfqpoint{2.942075in}{1.993003in}}%
\pgfpathcurveto{\pgfqpoint{2.947899in}{1.998827in}}{\pgfqpoint{2.951171in}{2.006727in}}{\pgfqpoint{2.951171in}{2.014963in}}%
\pgfpathcurveto{\pgfqpoint{2.951171in}{2.023200in}}{\pgfqpoint{2.947899in}{2.031100in}}{\pgfqpoint{2.942075in}{2.036924in}}%
\pgfpathcurveto{\pgfqpoint{2.936251in}{2.042748in}}{\pgfqpoint{2.928351in}{2.046020in}}{\pgfqpoint{2.920115in}{2.046020in}}%
\pgfpathcurveto{\pgfqpoint{2.911878in}{2.046020in}}{\pgfqpoint{2.903978in}{2.042748in}}{\pgfqpoint{2.898154in}{2.036924in}}%
\pgfpathcurveto{\pgfqpoint{2.892330in}{2.031100in}}{\pgfqpoint{2.889058in}{2.023200in}}{\pgfqpoint{2.889058in}{2.014963in}}%
\pgfpathcurveto{\pgfqpoint{2.889058in}{2.006727in}}{\pgfqpoint{2.892330in}{1.998827in}}{\pgfqpoint{2.898154in}{1.993003in}}%
\pgfpathcurveto{\pgfqpoint{2.903978in}{1.987179in}}{\pgfqpoint{2.911878in}{1.983907in}}{\pgfqpoint{2.920115in}{1.983907in}}%
\pgfpathclose%
\pgfusepath{stroke,fill}%
\end{pgfscope}%
\begin{pgfscope}%
\pgfpathrectangle{\pgfqpoint{0.100000in}{0.220728in}}{\pgfqpoint{3.696000in}{3.696000in}}%
\pgfusepath{clip}%
\pgfsetbuttcap%
\pgfsetroundjoin%
\definecolor{currentfill}{rgb}{0.121569,0.466667,0.705882}%
\pgfsetfillcolor{currentfill}%
\pgfsetfillopacity{0.839858}%
\pgfsetlinewidth{1.003750pt}%
\definecolor{currentstroke}{rgb}{0.121569,0.466667,0.705882}%
\pgfsetstrokecolor{currentstroke}%
\pgfsetstrokeopacity{0.839858}%
\pgfsetdash{}{0pt}%
\pgfpathmoveto{\pgfqpoint{2.918361in}{1.980768in}}%
\pgfpathcurveto{\pgfqpoint{2.926598in}{1.980768in}}{\pgfqpoint{2.934498in}{1.984040in}}{\pgfqpoint{2.940322in}{1.989864in}}%
\pgfpathcurveto{\pgfqpoint{2.946146in}{1.995688in}}{\pgfqpoint{2.949418in}{2.003588in}}{\pgfqpoint{2.949418in}{2.011824in}}%
\pgfpathcurveto{\pgfqpoint{2.949418in}{2.020060in}}{\pgfqpoint{2.946146in}{2.027960in}}{\pgfqpoint{2.940322in}{2.033784in}}%
\pgfpathcurveto{\pgfqpoint{2.934498in}{2.039608in}}{\pgfqpoint{2.926598in}{2.042881in}}{\pgfqpoint{2.918361in}{2.042881in}}%
\pgfpathcurveto{\pgfqpoint{2.910125in}{2.042881in}}{\pgfqpoint{2.902225in}{2.039608in}}{\pgfqpoint{2.896401in}{2.033784in}}%
\pgfpathcurveto{\pgfqpoint{2.890577in}{2.027960in}}{\pgfqpoint{2.887305in}{2.020060in}}{\pgfqpoint{2.887305in}{2.011824in}}%
\pgfpathcurveto{\pgfqpoint{2.887305in}{2.003588in}}{\pgfqpoint{2.890577in}{1.995688in}}{\pgfqpoint{2.896401in}{1.989864in}}%
\pgfpathcurveto{\pgfqpoint{2.902225in}{1.984040in}}{\pgfqpoint{2.910125in}{1.980768in}}{\pgfqpoint{2.918361in}{1.980768in}}%
\pgfpathclose%
\pgfusepath{stroke,fill}%
\end{pgfscope}%
\begin{pgfscope}%
\pgfpathrectangle{\pgfqpoint{0.100000in}{0.220728in}}{\pgfqpoint{3.696000in}{3.696000in}}%
\pgfusepath{clip}%
\pgfsetbuttcap%
\pgfsetroundjoin%
\definecolor{currentfill}{rgb}{0.121569,0.466667,0.705882}%
\pgfsetfillcolor{currentfill}%
\pgfsetfillopacity{0.840758}%
\pgfsetlinewidth{1.003750pt}%
\definecolor{currentstroke}{rgb}{0.121569,0.466667,0.705882}%
\pgfsetstrokecolor{currentstroke}%
\pgfsetstrokeopacity{0.840758}%
\pgfsetdash{}{0pt}%
\pgfpathmoveto{\pgfqpoint{2.916763in}{1.976131in}}%
\pgfpathcurveto{\pgfqpoint{2.925000in}{1.976131in}}{\pgfqpoint{2.932900in}{1.979403in}}{\pgfqpoint{2.938724in}{1.985227in}}%
\pgfpathcurveto{\pgfqpoint{2.944547in}{1.991051in}}{\pgfqpoint{2.947820in}{1.998951in}}{\pgfqpoint{2.947820in}{2.007187in}}%
\pgfpathcurveto{\pgfqpoint{2.947820in}{2.015423in}}{\pgfqpoint{2.944547in}{2.023324in}}{\pgfqpoint{2.938724in}{2.029147in}}%
\pgfpathcurveto{\pgfqpoint{2.932900in}{2.034971in}}{\pgfqpoint{2.925000in}{2.038244in}}{\pgfqpoint{2.916763in}{2.038244in}}%
\pgfpathcurveto{\pgfqpoint{2.908527in}{2.038244in}}{\pgfqpoint{2.900627in}{2.034971in}}{\pgfqpoint{2.894803in}{2.029147in}}%
\pgfpathcurveto{\pgfqpoint{2.888979in}{2.023324in}}{\pgfqpoint{2.885707in}{2.015423in}}{\pgfqpoint{2.885707in}{2.007187in}}%
\pgfpathcurveto{\pgfqpoint{2.885707in}{1.998951in}}{\pgfqpoint{2.888979in}{1.991051in}}{\pgfqpoint{2.894803in}{1.985227in}}%
\pgfpathcurveto{\pgfqpoint{2.900627in}{1.979403in}}{\pgfqpoint{2.908527in}{1.976131in}}{\pgfqpoint{2.916763in}{1.976131in}}%
\pgfpathclose%
\pgfusepath{stroke,fill}%
\end{pgfscope}%
\begin{pgfscope}%
\pgfpathrectangle{\pgfqpoint{0.100000in}{0.220728in}}{\pgfqpoint{3.696000in}{3.696000in}}%
\pgfusepath{clip}%
\pgfsetbuttcap%
\pgfsetroundjoin%
\definecolor{currentfill}{rgb}{0.121569,0.466667,0.705882}%
\pgfsetfillcolor{currentfill}%
\pgfsetfillopacity{0.841475}%
\pgfsetlinewidth{1.003750pt}%
\definecolor{currentstroke}{rgb}{0.121569,0.466667,0.705882}%
\pgfsetstrokecolor{currentstroke}%
\pgfsetstrokeopacity{0.841475}%
\pgfsetdash{}{0pt}%
\pgfpathmoveto{\pgfqpoint{1.544363in}{1.042889in}}%
\pgfpathcurveto{\pgfqpoint{1.552600in}{1.042889in}}{\pgfqpoint{1.560500in}{1.046161in}}{\pgfqpoint{1.566324in}{1.051985in}}%
\pgfpathcurveto{\pgfqpoint{1.572148in}{1.057809in}}{\pgfqpoint{1.575420in}{1.065709in}}{\pgfqpoint{1.575420in}{1.073946in}}%
\pgfpathcurveto{\pgfqpoint{1.575420in}{1.082182in}}{\pgfqpoint{1.572148in}{1.090082in}}{\pgfqpoint{1.566324in}{1.095906in}}%
\pgfpathcurveto{\pgfqpoint{1.560500in}{1.101730in}}{\pgfqpoint{1.552600in}{1.105002in}}{\pgfqpoint{1.544363in}{1.105002in}}%
\pgfpathcurveto{\pgfqpoint{1.536127in}{1.105002in}}{\pgfqpoint{1.528227in}{1.101730in}}{\pgfqpoint{1.522403in}{1.095906in}}%
\pgfpathcurveto{\pgfqpoint{1.516579in}{1.090082in}}{\pgfqpoint{1.513307in}{1.082182in}}{\pgfqpoint{1.513307in}{1.073946in}}%
\pgfpathcurveto{\pgfqpoint{1.513307in}{1.065709in}}{\pgfqpoint{1.516579in}{1.057809in}}{\pgfqpoint{1.522403in}{1.051985in}}%
\pgfpathcurveto{\pgfqpoint{1.528227in}{1.046161in}}{\pgfqpoint{1.536127in}{1.042889in}}{\pgfqpoint{1.544363in}{1.042889in}}%
\pgfpathclose%
\pgfusepath{stroke,fill}%
\end{pgfscope}%
\begin{pgfscope}%
\pgfpathrectangle{\pgfqpoint{0.100000in}{0.220728in}}{\pgfqpoint{3.696000in}{3.696000in}}%
\pgfusepath{clip}%
\pgfsetbuttcap%
\pgfsetroundjoin%
\definecolor{currentfill}{rgb}{0.121569,0.466667,0.705882}%
\pgfsetfillcolor{currentfill}%
\pgfsetfillopacity{0.841730}%
\pgfsetlinewidth{1.003750pt}%
\definecolor{currentstroke}{rgb}{0.121569,0.466667,0.705882}%
\pgfsetstrokecolor{currentstroke}%
\pgfsetstrokeopacity{0.841730}%
\pgfsetdash{}{0pt}%
\pgfpathmoveto{\pgfqpoint{2.914528in}{1.970939in}}%
\pgfpathcurveto{\pgfqpoint{2.922764in}{1.970939in}}{\pgfqpoint{2.930665in}{1.974211in}}{\pgfqpoint{2.936488in}{1.980035in}}%
\pgfpathcurveto{\pgfqpoint{2.942312in}{1.985859in}}{\pgfqpoint{2.945585in}{1.993759in}}{\pgfqpoint{2.945585in}{2.001995in}}%
\pgfpathcurveto{\pgfqpoint{2.945585in}{2.010232in}}{\pgfqpoint{2.942312in}{2.018132in}}{\pgfqpoint{2.936488in}{2.023956in}}%
\pgfpathcurveto{\pgfqpoint{2.930665in}{2.029780in}}{\pgfqpoint{2.922764in}{2.033052in}}{\pgfqpoint{2.914528in}{2.033052in}}%
\pgfpathcurveto{\pgfqpoint{2.906292in}{2.033052in}}{\pgfqpoint{2.898392in}{2.029780in}}{\pgfqpoint{2.892568in}{2.023956in}}%
\pgfpathcurveto{\pgfqpoint{2.886744in}{2.018132in}}{\pgfqpoint{2.883472in}{2.010232in}}{\pgfqpoint{2.883472in}{2.001995in}}%
\pgfpathcurveto{\pgfqpoint{2.883472in}{1.993759in}}{\pgfqpoint{2.886744in}{1.985859in}}{\pgfqpoint{2.892568in}{1.980035in}}%
\pgfpathcurveto{\pgfqpoint{2.898392in}{1.974211in}}{\pgfqpoint{2.906292in}{1.970939in}}{\pgfqpoint{2.914528in}{1.970939in}}%
\pgfpathclose%
\pgfusepath{stroke,fill}%
\end{pgfscope}%
\begin{pgfscope}%
\pgfpathrectangle{\pgfqpoint{0.100000in}{0.220728in}}{\pgfqpoint{3.696000in}{3.696000in}}%
\pgfusepath{clip}%
\pgfsetbuttcap%
\pgfsetroundjoin%
\definecolor{currentfill}{rgb}{0.121569,0.466667,0.705882}%
\pgfsetfillcolor{currentfill}%
\pgfsetfillopacity{0.842802}%
\pgfsetlinewidth{1.003750pt}%
\definecolor{currentstroke}{rgb}{0.121569,0.466667,0.705882}%
\pgfsetstrokecolor{currentstroke}%
\pgfsetstrokeopacity{0.842802}%
\pgfsetdash{}{0pt}%
\pgfpathmoveto{\pgfqpoint{2.911281in}{1.964959in}}%
\pgfpathcurveto{\pgfqpoint{2.919517in}{1.964959in}}{\pgfqpoint{2.927417in}{1.968231in}}{\pgfqpoint{2.933241in}{1.974055in}}%
\pgfpathcurveto{\pgfqpoint{2.939065in}{1.979879in}}{\pgfqpoint{2.942337in}{1.987779in}}{\pgfqpoint{2.942337in}{1.996015in}}%
\pgfpathcurveto{\pgfqpoint{2.942337in}{2.004252in}}{\pgfqpoint{2.939065in}{2.012152in}}{\pgfqpoint{2.933241in}{2.017976in}}%
\pgfpathcurveto{\pgfqpoint{2.927417in}{2.023800in}}{\pgfqpoint{2.919517in}{2.027072in}}{\pgfqpoint{2.911281in}{2.027072in}}%
\pgfpathcurveto{\pgfqpoint{2.903044in}{2.027072in}}{\pgfqpoint{2.895144in}{2.023800in}}{\pgfqpoint{2.889320in}{2.017976in}}%
\pgfpathcurveto{\pgfqpoint{2.883496in}{2.012152in}}{\pgfqpoint{2.880224in}{2.004252in}}{\pgfqpoint{2.880224in}{1.996015in}}%
\pgfpathcurveto{\pgfqpoint{2.880224in}{1.987779in}}{\pgfqpoint{2.883496in}{1.979879in}}{\pgfqpoint{2.889320in}{1.974055in}}%
\pgfpathcurveto{\pgfqpoint{2.895144in}{1.968231in}}{\pgfqpoint{2.903044in}{1.964959in}}{\pgfqpoint{2.911281in}{1.964959in}}%
\pgfpathclose%
\pgfusepath{stroke,fill}%
\end{pgfscope}%
\begin{pgfscope}%
\pgfpathrectangle{\pgfqpoint{0.100000in}{0.220728in}}{\pgfqpoint{3.696000in}{3.696000in}}%
\pgfusepath{clip}%
\pgfsetbuttcap%
\pgfsetroundjoin%
\definecolor{currentfill}{rgb}{0.121569,0.466667,0.705882}%
\pgfsetfillcolor{currentfill}%
\pgfsetfillopacity{0.844470}%
\pgfsetlinewidth{1.003750pt}%
\definecolor{currentstroke}{rgb}{0.121569,0.466667,0.705882}%
\pgfsetstrokecolor{currentstroke}%
\pgfsetstrokeopacity{0.844470}%
\pgfsetdash{}{0pt}%
\pgfpathmoveto{\pgfqpoint{2.908478in}{1.955830in}}%
\pgfpathcurveto{\pgfqpoint{2.916714in}{1.955830in}}{\pgfqpoint{2.924614in}{1.959102in}}{\pgfqpoint{2.930438in}{1.964926in}}%
\pgfpathcurveto{\pgfqpoint{2.936262in}{1.970750in}}{\pgfqpoint{2.939534in}{1.978650in}}{\pgfqpoint{2.939534in}{1.986886in}}%
\pgfpathcurveto{\pgfqpoint{2.939534in}{1.995123in}}{\pgfqpoint{2.936262in}{2.003023in}}{\pgfqpoint{2.930438in}{2.008847in}}%
\pgfpathcurveto{\pgfqpoint{2.924614in}{2.014671in}}{\pgfqpoint{2.916714in}{2.017943in}}{\pgfqpoint{2.908478in}{2.017943in}}%
\pgfpathcurveto{\pgfqpoint{2.900241in}{2.017943in}}{\pgfqpoint{2.892341in}{2.014671in}}{\pgfqpoint{2.886517in}{2.008847in}}%
\pgfpathcurveto{\pgfqpoint{2.880694in}{2.003023in}}{\pgfqpoint{2.877421in}{1.995123in}}{\pgfqpoint{2.877421in}{1.986886in}}%
\pgfpathcurveto{\pgfqpoint{2.877421in}{1.978650in}}{\pgfqpoint{2.880694in}{1.970750in}}{\pgfqpoint{2.886517in}{1.964926in}}%
\pgfpathcurveto{\pgfqpoint{2.892341in}{1.959102in}}{\pgfqpoint{2.900241in}{1.955830in}}{\pgfqpoint{2.908478in}{1.955830in}}%
\pgfpathclose%
\pgfusepath{stroke,fill}%
\end{pgfscope}%
\begin{pgfscope}%
\pgfpathrectangle{\pgfqpoint{0.100000in}{0.220728in}}{\pgfqpoint{3.696000in}{3.696000in}}%
\pgfusepath{clip}%
\pgfsetbuttcap%
\pgfsetroundjoin%
\definecolor{currentfill}{rgb}{0.121569,0.466667,0.705882}%
\pgfsetfillcolor{currentfill}%
\pgfsetfillopacity{0.844544}%
\pgfsetlinewidth{1.003750pt}%
\definecolor{currentstroke}{rgb}{0.121569,0.466667,0.705882}%
\pgfsetstrokecolor{currentstroke}%
\pgfsetstrokeopacity{0.844544}%
\pgfsetdash{}{0pt}%
\pgfpathmoveto{\pgfqpoint{1.560663in}{1.036019in}}%
\pgfpathcurveto{\pgfqpoint{1.568899in}{1.036019in}}{\pgfqpoint{1.576799in}{1.039291in}}{\pgfqpoint{1.582623in}{1.045115in}}%
\pgfpathcurveto{\pgfqpoint{1.588447in}{1.050939in}}{\pgfqpoint{1.591719in}{1.058839in}}{\pgfqpoint{1.591719in}{1.067076in}}%
\pgfpathcurveto{\pgfqpoint{1.591719in}{1.075312in}}{\pgfqpoint{1.588447in}{1.083212in}}{\pgfqpoint{1.582623in}{1.089036in}}%
\pgfpathcurveto{\pgfqpoint{1.576799in}{1.094860in}}{\pgfqpoint{1.568899in}{1.098132in}}{\pgfqpoint{1.560663in}{1.098132in}}%
\pgfpathcurveto{\pgfqpoint{1.552427in}{1.098132in}}{\pgfqpoint{1.544527in}{1.094860in}}{\pgfqpoint{1.538703in}{1.089036in}}%
\pgfpathcurveto{\pgfqpoint{1.532879in}{1.083212in}}{\pgfqpoint{1.529606in}{1.075312in}}{\pgfqpoint{1.529606in}{1.067076in}}%
\pgfpathcurveto{\pgfqpoint{1.529606in}{1.058839in}}{\pgfqpoint{1.532879in}{1.050939in}}{\pgfqpoint{1.538703in}{1.045115in}}%
\pgfpathcurveto{\pgfqpoint{1.544527in}{1.039291in}}{\pgfqpoint{1.552427in}{1.036019in}}{\pgfqpoint{1.560663in}{1.036019in}}%
\pgfpathclose%
\pgfusepath{stroke,fill}%
\end{pgfscope}%
\begin{pgfscope}%
\pgfpathrectangle{\pgfqpoint{0.100000in}{0.220728in}}{\pgfqpoint{3.696000in}{3.696000in}}%
\pgfusepath{clip}%
\pgfsetbuttcap%
\pgfsetroundjoin%
\definecolor{currentfill}{rgb}{0.121569,0.466667,0.705882}%
\pgfsetfillcolor{currentfill}%
\pgfsetfillopacity{0.845204}%
\pgfsetlinewidth{1.003750pt}%
\definecolor{currentstroke}{rgb}{0.121569,0.466667,0.705882}%
\pgfsetstrokecolor{currentstroke}%
\pgfsetstrokeopacity{0.845204}%
\pgfsetdash{}{0pt}%
\pgfpathmoveto{\pgfqpoint{2.906002in}{1.950983in}}%
\pgfpathcurveto{\pgfqpoint{2.914238in}{1.950983in}}{\pgfqpoint{2.922138in}{1.954256in}}{\pgfqpoint{2.927962in}{1.960080in}}%
\pgfpathcurveto{\pgfqpoint{2.933786in}{1.965904in}}{\pgfqpoint{2.937058in}{1.973804in}}{\pgfqpoint{2.937058in}{1.982040in}}%
\pgfpathcurveto{\pgfqpoint{2.937058in}{1.990276in}}{\pgfqpoint{2.933786in}{1.998176in}}{\pgfqpoint{2.927962in}{2.004000in}}%
\pgfpathcurveto{\pgfqpoint{2.922138in}{2.009824in}}{\pgfqpoint{2.914238in}{2.013096in}}{\pgfqpoint{2.906002in}{2.013096in}}%
\pgfpathcurveto{\pgfqpoint{2.897765in}{2.013096in}}{\pgfqpoint{2.889865in}{2.009824in}}{\pgfqpoint{2.884041in}{2.004000in}}%
\pgfpathcurveto{\pgfqpoint{2.878218in}{1.998176in}}{\pgfqpoint{2.874945in}{1.990276in}}{\pgfqpoint{2.874945in}{1.982040in}}%
\pgfpathcurveto{\pgfqpoint{2.874945in}{1.973804in}}{\pgfqpoint{2.878218in}{1.965904in}}{\pgfqpoint{2.884041in}{1.960080in}}%
\pgfpathcurveto{\pgfqpoint{2.889865in}{1.954256in}}{\pgfqpoint{2.897765in}{1.950983in}}{\pgfqpoint{2.906002in}{1.950983in}}%
\pgfpathclose%
\pgfusepath{stroke,fill}%
\end{pgfscope}%
\begin{pgfscope}%
\pgfpathrectangle{\pgfqpoint{0.100000in}{0.220728in}}{\pgfqpoint{3.696000in}{3.696000in}}%
\pgfusepath{clip}%
\pgfsetbuttcap%
\pgfsetroundjoin%
\definecolor{currentfill}{rgb}{0.121569,0.466667,0.705882}%
\pgfsetfillcolor{currentfill}%
\pgfsetfillopacity{0.845655}%
\pgfsetlinewidth{1.003750pt}%
\definecolor{currentstroke}{rgb}{0.121569,0.466667,0.705882}%
\pgfsetstrokecolor{currentstroke}%
\pgfsetstrokeopacity{0.845655}%
\pgfsetdash{}{0pt}%
\pgfpathmoveto{\pgfqpoint{2.904504in}{1.948713in}}%
\pgfpathcurveto{\pgfqpoint{2.912740in}{1.948713in}}{\pgfqpoint{2.920640in}{1.951985in}}{\pgfqpoint{2.926464in}{1.957809in}}%
\pgfpathcurveto{\pgfqpoint{2.932288in}{1.963633in}}{\pgfqpoint{2.935560in}{1.971533in}}{\pgfqpoint{2.935560in}{1.979769in}}%
\pgfpathcurveto{\pgfqpoint{2.935560in}{1.988006in}}{\pgfqpoint{2.932288in}{1.995906in}}{\pgfqpoint{2.926464in}{2.001730in}}%
\pgfpathcurveto{\pgfqpoint{2.920640in}{2.007554in}}{\pgfqpoint{2.912740in}{2.010826in}}{\pgfqpoint{2.904504in}{2.010826in}}%
\pgfpathcurveto{\pgfqpoint{2.896267in}{2.010826in}}{\pgfqpoint{2.888367in}{2.007554in}}{\pgfqpoint{2.882543in}{2.001730in}}%
\pgfpathcurveto{\pgfqpoint{2.876719in}{1.995906in}}{\pgfqpoint{2.873447in}{1.988006in}}{\pgfqpoint{2.873447in}{1.979769in}}%
\pgfpathcurveto{\pgfqpoint{2.873447in}{1.971533in}}{\pgfqpoint{2.876719in}{1.963633in}}{\pgfqpoint{2.882543in}{1.957809in}}%
\pgfpathcurveto{\pgfqpoint{2.888367in}{1.951985in}}{\pgfqpoint{2.896267in}{1.948713in}}{\pgfqpoint{2.904504in}{1.948713in}}%
\pgfpathclose%
\pgfusepath{stroke,fill}%
\end{pgfscope}%
\begin{pgfscope}%
\pgfpathrectangle{\pgfqpoint{0.100000in}{0.220728in}}{\pgfqpoint{3.696000in}{3.696000in}}%
\pgfusepath{clip}%
\pgfsetbuttcap%
\pgfsetroundjoin%
\definecolor{currentfill}{rgb}{0.121569,0.466667,0.705882}%
\pgfsetfillcolor{currentfill}%
\pgfsetfillopacity{0.846157}%
\pgfsetlinewidth{1.003750pt}%
\definecolor{currentstroke}{rgb}{0.121569,0.466667,0.705882}%
\pgfsetstrokecolor{currentstroke}%
\pgfsetstrokeopacity{0.846157}%
\pgfsetdash{}{0pt}%
\pgfpathmoveto{\pgfqpoint{1.575949in}{1.025817in}}%
\pgfpathcurveto{\pgfqpoint{1.584186in}{1.025817in}}{\pgfqpoint{1.592086in}{1.029089in}}{\pgfqpoint{1.597910in}{1.034913in}}%
\pgfpathcurveto{\pgfqpoint{1.603734in}{1.040737in}}{\pgfqpoint{1.607006in}{1.048637in}}{\pgfqpoint{1.607006in}{1.056874in}}%
\pgfpathcurveto{\pgfqpoint{1.607006in}{1.065110in}}{\pgfqpoint{1.603734in}{1.073010in}}{\pgfqpoint{1.597910in}{1.078834in}}%
\pgfpathcurveto{\pgfqpoint{1.592086in}{1.084658in}}{\pgfqpoint{1.584186in}{1.087930in}}{\pgfqpoint{1.575949in}{1.087930in}}%
\pgfpathcurveto{\pgfqpoint{1.567713in}{1.087930in}}{\pgfqpoint{1.559813in}{1.084658in}}{\pgfqpoint{1.553989in}{1.078834in}}%
\pgfpathcurveto{\pgfqpoint{1.548165in}{1.073010in}}{\pgfqpoint{1.544893in}{1.065110in}}{\pgfqpoint{1.544893in}{1.056874in}}%
\pgfpathcurveto{\pgfqpoint{1.544893in}{1.048637in}}{\pgfqpoint{1.548165in}{1.040737in}}{\pgfqpoint{1.553989in}{1.034913in}}%
\pgfpathcurveto{\pgfqpoint{1.559813in}{1.029089in}}{\pgfqpoint{1.567713in}{1.025817in}}{\pgfqpoint{1.575949in}{1.025817in}}%
\pgfpathclose%
\pgfusepath{stroke,fill}%
\end{pgfscope}%
\begin{pgfscope}%
\pgfpathrectangle{\pgfqpoint{0.100000in}{0.220728in}}{\pgfqpoint{3.696000in}{3.696000in}}%
\pgfusepath{clip}%
\pgfsetbuttcap%
\pgfsetroundjoin%
\definecolor{currentfill}{rgb}{0.121569,0.466667,0.705882}%
\pgfsetfillcolor{currentfill}%
\pgfsetfillopacity{0.846575}%
\pgfsetlinewidth{1.003750pt}%
\definecolor{currentstroke}{rgb}{0.121569,0.466667,0.705882}%
\pgfsetstrokecolor{currentstroke}%
\pgfsetstrokeopacity{0.846575}%
\pgfsetdash{}{0pt}%
\pgfpathmoveto{\pgfqpoint{2.903244in}{1.942571in}}%
\pgfpathcurveto{\pgfqpoint{2.911480in}{1.942571in}}{\pgfqpoint{2.919380in}{1.945843in}}{\pgfqpoint{2.925204in}{1.951667in}}%
\pgfpathcurveto{\pgfqpoint{2.931028in}{1.957491in}}{\pgfqpoint{2.934300in}{1.965391in}}{\pgfqpoint{2.934300in}{1.973627in}}%
\pgfpathcurveto{\pgfqpoint{2.934300in}{1.981864in}}{\pgfqpoint{2.931028in}{1.989764in}}{\pgfqpoint{2.925204in}{1.995587in}}%
\pgfpathcurveto{\pgfqpoint{2.919380in}{2.001411in}}{\pgfqpoint{2.911480in}{2.004684in}}{\pgfqpoint{2.903244in}{2.004684in}}%
\pgfpathcurveto{\pgfqpoint{2.895008in}{2.004684in}}{\pgfqpoint{2.887108in}{2.001411in}}{\pgfqpoint{2.881284in}{1.995587in}}%
\pgfpathcurveto{\pgfqpoint{2.875460in}{1.989764in}}{\pgfqpoint{2.872187in}{1.981864in}}{\pgfqpoint{2.872187in}{1.973627in}}%
\pgfpathcurveto{\pgfqpoint{2.872187in}{1.965391in}}{\pgfqpoint{2.875460in}{1.957491in}}{\pgfqpoint{2.881284in}{1.951667in}}%
\pgfpathcurveto{\pgfqpoint{2.887108in}{1.945843in}}{\pgfqpoint{2.895008in}{1.942571in}}{\pgfqpoint{2.903244in}{1.942571in}}%
\pgfpathclose%
\pgfusepath{stroke,fill}%
\end{pgfscope}%
\begin{pgfscope}%
\pgfpathrectangle{\pgfqpoint{0.100000in}{0.220728in}}{\pgfqpoint{3.696000in}{3.696000in}}%
\pgfusepath{clip}%
\pgfsetbuttcap%
\pgfsetroundjoin%
\definecolor{currentfill}{rgb}{0.121569,0.466667,0.705882}%
\pgfsetfillcolor{currentfill}%
\pgfsetfillopacity{0.847132}%
\pgfsetlinewidth{1.003750pt}%
\definecolor{currentstroke}{rgb}{0.121569,0.466667,0.705882}%
\pgfsetstrokecolor{currentstroke}%
\pgfsetstrokeopacity{0.847132}%
\pgfsetdash{}{0pt}%
\pgfpathmoveto{\pgfqpoint{2.902191in}{1.939623in}}%
\pgfpathcurveto{\pgfqpoint{2.910427in}{1.939623in}}{\pgfqpoint{2.918327in}{1.942895in}}{\pgfqpoint{2.924151in}{1.948719in}}%
\pgfpathcurveto{\pgfqpoint{2.929975in}{1.954543in}}{\pgfqpoint{2.933247in}{1.962443in}}{\pgfqpoint{2.933247in}{1.970679in}}%
\pgfpathcurveto{\pgfqpoint{2.933247in}{1.978915in}}{\pgfqpoint{2.929975in}{1.986815in}}{\pgfqpoint{2.924151in}{1.992639in}}%
\pgfpathcurveto{\pgfqpoint{2.918327in}{1.998463in}}{\pgfqpoint{2.910427in}{2.001736in}}{\pgfqpoint{2.902191in}{2.001736in}}%
\pgfpathcurveto{\pgfqpoint{2.893955in}{2.001736in}}{\pgfqpoint{2.886055in}{1.998463in}}{\pgfqpoint{2.880231in}{1.992639in}}%
\pgfpathcurveto{\pgfqpoint{2.874407in}{1.986815in}}{\pgfqpoint{2.871135in}{1.978915in}}{\pgfqpoint{2.871135in}{1.970679in}}%
\pgfpathcurveto{\pgfqpoint{2.871135in}{1.962443in}}{\pgfqpoint{2.874407in}{1.954543in}}{\pgfqpoint{2.880231in}{1.948719in}}%
\pgfpathcurveto{\pgfqpoint{2.886055in}{1.942895in}}{\pgfqpoint{2.893955in}{1.939623in}}{\pgfqpoint{2.902191in}{1.939623in}}%
\pgfpathclose%
\pgfusepath{stroke,fill}%
\end{pgfscope}%
\begin{pgfscope}%
\pgfpathrectangle{\pgfqpoint{0.100000in}{0.220728in}}{\pgfqpoint{3.696000in}{3.696000in}}%
\pgfusepath{clip}%
\pgfsetbuttcap%
\pgfsetroundjoin%
\definecolor{currentfill}{rgb}{0.121569,0.466667,0.705882}%
\pgfsetfillcolor{currentfill}%
\pgfsetfillopacity{0.847253}%
\pgfsetlinewidth{1.003750pt}%
\definecolor{currentstroke}{rgb}{0.121569,0.466667,0.705882}%
\pgfsetstrokecolor{currentstroke}%
\pgfsetstrokeopacity{0.847253}%
\pgfsetdash{}{0pt}%
\pgfpathmoveto{\pgfqpoint{2.901129in}{1.937898in}}%
\pgfpathcurveto{\pgfqpoint{2.909365in}{1.937898in}}{\pgfqpoint{2.917265in}{1.941170in}}{\pgfqpoint{2.923089in}{1.946994in}}%
\pgfpathcurveto{\pgfqpoint{2.928913in}{1.952818in}}{\pgfqpoint{2.932186in}{1.960718in}}{\pgfqpoint{2.932186in}{1.968954in}}%
\pgfpathcurveto{\pgfqpoint{2.932186in}{1.977190in}}{\pgfqpoint{2.928913in}{1.985090in}}{\pgfqpoint{2.923089in}{1.990914in}}%
\pgfpathcurveto{\pgfqpoint{2.917265in}{1.996738in}}{\pgfqpoint{2.909365in}{2.000011in}}{\pgfqpoint{2.901129in}{2.000011in}}%
\pgfpathcurveto{\pgfqpoint{2.892893in}{2.000011in}}{\pgfqpoint{2.884993in}{1.996738in}}{\pgfqpoint{2.879169in}{1.990914in}}%
\pgfpathcurveto{\pgfqpoint{2.873345in}{1.985090in}}{\pgfqpoint{2.870073in}{1.977190in}}{\pgfqpoint{2.870073in}{1.968954in}}%
\pgfpathcurveto{\pgfqpoint{2.870073in}{1.960718in}}{\pgfqpoint{2.873345in}{1.952818in}}{\pgfqpoint{2.879169in}{1.946994in}}%
\pgfpathcurveto{\pgfqpoint{2.884993in}{1.941170in}}{\pgfqpoint{2.892893in}{1.937898in}}{\pgfqpoint{2.901129in}{1.937898in}}%
\pgfpathclose%
\pgfusepath{stroke,fill}%
\end{pgfscope}%
\begin{pgfscope}%
\pgfpathrectangle{\pgfqpoint{0.100000in}{0.220728in}}{\pgfqpoint{3.696000in}{3.696000in}}%
\pgfusepath{clip}%
\pgfsetbuttcap%
\pgfsetroundjoin%
\definecolor{currentfill}{rgb}{0.121569,0.466667,0.705882}%
\pgfsetfillcolor{currentfill}%
\pgfsetfillopacity{0.848087}%
\pgfsetlinewidth{1.003750pt}%
\definecolor{currentstroke}{rgb}{0.121569,0.466667,0.705882}%
\pgfsetstrokecolor{currentstroke}%
\pgfsetstrokeopacity{0.848087}%
\pgfsetdash{}{0pt}%
\pgfpathmoveto{\pgfqpoint{2.898725in}{1.933462in}}%
\pgfpathcurveto{\pgfqpoint{2.906961in}{1.933462in}}{\pgfqpoint{2.914861in}{1.936734in}}{\pgfqpoint{2.920685in}{1.942558in}}%
\pgfpathcurveto{\pgfqpoint{2.926509in}{1.948382in}}{\pgfqpoint{2.929782in}{1.956282in}}{\pgfqpoint{2.929782in}{1.964519in}}%
\pgfpathcurveto{\pgfqpoint{2.929782in}{1.972755in}}{\pgfqpoint{2.926509in}{1.980655in}}{\pgfqpoint{2.920685in}{1.986479in}}%
\pgfpathcurveto{\pgfqpoint{2.914861in}{1.992303in}}{\pgfqpoint{2.906961in}{1.995575in}}{\pgfqpoint{2.898725in}{1.995575in}}%
\pgfpathcurveto{\pgfqpoint{2.890489in}{1.995575in}}{\pgfqpoint{2.882589in}{1.992303in}}{\pgfqpoint{2.876765in}{1.986479in}}%
\pgfpathcurveto{\pgfqpoint{2.870941in}{1.980655in}}{\pgfqpoint{2.867669in}{1.972755in}}{\pgfqpoint{2.867669in}{1.964519in}}%
\pgfpathcurveto{\pgfqpoint{2.867669in}{1.956282in}}{\pgfqpoint{2.870941in}{1.948382in}}{\pgfqpoint{2.876765in}{1.942558in}}%
\pgfpathcurveto{\pgfqpoint{2.882589in}{1.936734in}}{\pgfqpoint{2.890489in}{1.933462in}}{\pgfqpoint{2.898725in}{1.933462in}}%
\pgfpathclose%
\pgfusepath{stroke,fill}%
\end{pgfscope}%
\begin{pgfscope}%
\pgfpathrectangle{\pgfqpoint{0.100000in}{0.220728in}}{\pgfqpoint{3.696000in}{3.696000in}}%
\pgfusepath{clip}%
\pgfsetbuttcap%
\pgfsetroundjoin%
\definecolor{currentfill}{rgb}{0.121569,0.466667,0.705882}%
\pgfsetfillcolor{currentfill}%
\pgfsetfillopacity{0.849164}%
\pgfsetlinewidth{1.003750pt}%
\definecolor{currentstroke}{rgb}{0.121569,0.466667,0.705882}%
\pgfsetstrokecolor{currentstroke}%
\pgfsetstrokeopacity{0.849164}%
\pgfsetdash{}{0pt}%
\pgfpathmoveto{\pgfqpoint{1.604188in}{1.009192in}}%
\pgfpathcurveto{\pgfqpoint{1.612424in}{1.009192in}}{\pgfqpoint{1.620324in}{1.012464in}}{\pgfqpoint{1.626148in}{1.018288in}}%
\pgfpathcurveto{\pgfqpoint{1.631972in}{1.024112in}}{\pgfqpoint{1.635244in}{1.032012in}}{\pgfqpoint{1.635244in}{1.040249in}}%
\pgfpathcurveto{\pgfqpoint{1.635244in}{1.048485in}}{\pgfqpoint{1.631972in}{1.056385in}}{\pgfqpoint{1.626148in}{1.062209in}}%
\pgfpathcurveto{\pgfqpoint{1.620324in}{1.068033in}}{\pgfqpoint{1.612424in}{1.071305in}}{\pgfqpoint{1.604188in}{1.071305in}}%
\pgfpathcurveto{\pgfqpoint{1.595952in}{1.071305in}}{\pgfqpoint{1.588052in}{1.068033in}}{\pgfqpoint{1.582228in}{1.062209in}}%
\pgfpathcurveto{\pgfqpoint{1.576404in}{1.056385in}}{\pgfqpoint{1.573131in}{1.048485in}}{\pgfqpoint{1.573131in}{1.040249in}}%
\pgfpathcurveto{\pgfqpoint{1.573131in}{1.032012in}}{\pgfqpoint{1.576404in}{1.024112in}}{\pgfqpoint{1.582228in}{1.018288in}}%
\pgfpathcurveto{\pgfqpoint{1.588052in}{1.012464in}}{\pgfqpoint{1.595952in}{1.009192in}}{\pgfqpoint{1.604188in}{1.009192in}}%
\pgfpathclose%
\pgfusepath{stroke,fill}%
\end{pgfscope}%
\begin{pgfscope}%
\pgfpathrectangle{\pgfqpoint{0.100000in}{0.220728in}}{\pgfqpoint{3.696000in}{3.696000in}}%
\pgfusepath{clip}%
\pgfsetbuttcap%
\pgfsetroundjoin%
\definecolor{currentfill}{rgb}{0.121569,0.466667,0.705882}%
\pgfsetfillcolor{currentfill}%
\pgfsetfillopacity{0.849295}%
\pgfsetlinewidth{1.003750pt}%
\definecolor{currentstroke}{rgb}{0.121569,0.466667,0.705882}%
\pgfsetstrokecolor{currentstroke}%
\pgfsetstrokeopacity{0.849295}%
\pgfsetdash{}{0pt}%
\pgfpathmoveto{\pgfqpoint{2.897199in}{1.924773in}}%
\pgfpathcurveto{\pgfqpoint{2.905436in}{1.924773in}}{\pgfqpoint{2.913336in}{1.928045in}}{\pgfqpoint{2.919160in}{1.933869in}}%
\pgfpathcurveto{\pgfqpoint{2.924983in}{1.939693in}}{\pgfqpoint{2.928256in}{1.947593in}}{\pgfqpoint{2.928256in}{1.955829in}}%
\pgfpathcurveto{\pgfqpoint{2.928256in}{1.964066in}}{\pgfqpoint{2.924983in}{1.971966in}}{\pgfqpoint{2.919160in}{1.977790in}}%
\pgfpathcurveto{\pgfqpoint{2.913336in}{1.983613in}}{\pgfqpoint{2.905436in}{1.986886in}}{\pgfqpoint{2.897199in}{1.986886in}}%
\pgfpathcurveto{\pgfqpoint{2.888963in}{1.986886in}}{\pgfqpoint{2.881063in}{1.983613in}}{\pgfqpoint{2.875239in}{1.977790in}}%
\pgfpathcurveto{\pgfqpoint{2.869415in}{1.971966in}}{\pgfqpoint{2.866143in}{1.964066in}}{\pgfqpoint{2.866143in}{1.955829in}}%
\pgfpathcurveto{\pgfqpoint{2.866143in}{1.947593in}}{\pgfqpoint{2.869415in}{1.939693in}}{\pgfqpoint{2.875239in}{1.933869in}}%
\pgfpathcurveto{\pgfqpoint{2.881063in}{1.928045in}}{\pgfqpoint{2.888963in}{1.924773in}}{\pgfqpoint{2.897199in}{1.924773in}}%
\pgfpathclose%
\pgfusepath{stroke,fill}%
\end{pgfscope}%
\begin{pgfscope}%
\pgfpathrectangle{\pgfqpoint{0.100000in}{0.220728in}}{\pgfqpoint{3.696000in}{3.696000in}}%
\pgfusepath{clip}%
\pgfsetbuttcap%
\pgfsetroundjoin%
\definecolor{currentfill}{rgb}{0.121569,0.466667,0.705882}%
\pgfsetfillcolor{currentfill}%
\pgfsetfillopacity{0.850489}%
\pgfsetlinewidth{1.003750pt}%
\definecolor{currentstroke}{rgb}{0.121569,0.466667,0.705882}%
\pgfsetstrokecolor{currentstroke}%
\pgfsetstrokeopacity{0.850489}%
\pgfsetdash{}{0pt}%
\pgfpathmoveto{\pgfqpoint{2.890801in}{1.913103in}}%
\pgfpathcurveto{\pgfqpoint{2.899037in}{1.913103in}}{\pgfqpoint{2.906938in}{1.916375in}}{\pgfqpoint{2.912761in}{1.922199in}}%
\pgfpathcurveto{\pgfqpoint{2.918585in}{1.928023in}}{\pgfqpoint{2.921858in}{1.935923in}}{\pgfqpoint{2.921858in}{1.944160in}}%
\pgfpathcurveto{\pgfqpoint{2.921858in}{1.952396in}}{\pgfqpoint{2.918585in}{1.960296in}}{\pgfqpoint{2.912761in}{1.966120in}}%
\pgfpathcurveto{\pgfqpoint{2.906938in}{1.971944in}}{\pgfqpoint{2.899037in}{1.975216in}}{\pgfqpoint{2.890801in}{1.975216in}}%
\pgfpathcurveto{\pgfqpoint{2.882565in}{1.975216in}}{\pgfqpoint{2.874665in}{1.971944in}}{\pgfqpoint{2.868841in}{1.966120in}}%
\pgfpathcurveto{\pgfqpoint{2.863017in}{1.960296in}}{\pgfqpoint{2.859745in}{1.952396in}}{\pgfqpoint{2.859745in}{1.944160in}}%
\pgfpathcurveto{\pgfqpoint{2.859745in}{1.935923in}}{\pgfqpoint{2.863017in}{1.928023in}}{\pgfqpoint{2.868841in}{1.922199in}}%
\pgfpathcurveto{\pgfqpoint{2.874665in}{1.916375in}}{\pgfqpoint{2.882565in}{1.913103in}}{\pgfqpoint{2.890801in}{1.913103in}}%
\pgfpathclose%
\pgfusepath{stroke,fill}%
\end{pgfscope}%
\begin{pgfscope}%
\pgfpathrectangle{\pgfqpoint{0.100000in}{0.220728in}}{\pgfqpoint{3.696000in}{3.696000in}}%
\pgfusepath{clip}%
\pgfsetbuttcap%
\pgfsetroundjoin%
\definecolor{currentfill}{rgb}{0.121569,0.466667,0.705882}%
\pgfsetfillcolor{currentfill}%
\pgfsetfillopacity{0.851240}%
\pgfsetlinewidth{1.003750pt}%
\definecolor{currentstroke}{rgb}{0.121569,0.466667,0.705882}%
\pgfsetstrokecolor{currentstroke}%
\pgfsetstrokeopacity{0.851240}%
\pgfsetdash{}{0pt}%
\pgfpathmoveto{\pgfqpoint{2.887161in}{1.907282in}}%
\pgfpathcurveto{\pgfqpoint{2.895397in}{1.907282in}}{\pgfqpoint{2.903298in}{1.910554in}}{\pgfqpoint{2.909121in}{1.916378in}}%
\pgfpathcurveto{\pgfqpoint{2.914945in}{1.922202in}}{\pgfqpoint{2.918218in}{1.930102in}}{\pgfqpoint{2.918218in}{1.938338in}}%
\pgfpathcurveto{\pgfqpoint{2.918218in}{1.946575in}}{\pgfqpoint{2.914945in}{1.954475in}}{\pgfqpoint{2.909121in}{1.960299in}}%
\pgfpathcurveto{\pgfqpoint{2.903298in}{1.966123in}}{\pgfqpoint{2.895397in}{1.969395in}}{\pgfqpoint{2.887161in}{1.969395in}}%
\pgfpathcurveto{\pgfqpoint{2.878925in}{1.969395in}}{\pgfqpoint{2.871025in}{1.966123in}}{\pgfqpoint{2.865201in}{1.960299in}}%
\pgfpathcurveto{\pgfqpoint{2.859377in}{1.954475in}}{\pgfqpoint{2.856105in}{1.946575in}}{\pgfqpoint{2.856105in}{1.938338in}}%
\pgfpathcurveto{\pgfqpoint{2.856105in}{1.930102in}}{\pgfqpoint{2.859377in}{1.922202in}}{\pgfqpoint{2.865201in}{1.916378in}}%
\pgfpathcurveto{\pgfqpoint{2.871025in}{1.910554in}}{\pgfqpoint{2.878925in}{1.907282in}}{\pgfqpoint{2.887161in}{1.907282in}}%
\pgfpathclose%
\pgfusepath{stroke,fill}%
\end{pgfscope}%
\begin{pgfscope}%
\pgfpathrectangle{\pgfqpoint{0.100000in}{0.220728in}}{\pgfqpoint{3.696000in}{3.696000in}}%
\pgfusepath{clip}%
\pgfsetbuttcap%
\pgfsetroundjoin%
\definecolor{currentfill}{rgb}{0.121569,0.466667,0.705882}%
\pgfsetfillcolor{currentfill}%
\pgfsetfillopacity{0.852697}%
\pgfsetlinewidth{1.003750pt}%
\definecolor{currentstroke}{rgb}{0.121569,0.466667,0.705882}%
\pgfsetstrokecolor{currentstroke}%
\pgfsetstrokeopacity{0.852697}%
\pgfsetdash{}{0pt}%
\pgfpathmoveto{\pgfqpoint{2.884954in}{1.894496in}}%
\pgfpathcurveto{\pgfqpoint{2.893190in}{1.894496in}}{\pgfqpoint{2.901090in}{1.897768in}}{\pgfqpoint{2.906914in}{1.903592in}}%
\pgfpathcurveto{\pgfqpoint{2.912738in}{1.909416in}}{\pgfqpoint{2.916010in}{1.917316in}}{\pgfqpoint{2.916010in}{1.925552in}}%
\pgfpathcurveto{\pgfqpoint{2.916010in}{1.933789in}}{\pgfqpoint{2.912738in}{1.941689in}}{\pgfqpoint{2.906914in}{1.947513in}}%
\pgfpathcurveto{\pgfqpoint{2.901090in}{1.953336in}}{\pgfqpoint{2.893190in}{1.956609in}}{\pgfqpoint{2.884954in}{1.956609in}}%
\pgfpathcurveto{\pgfqpoint{2.876718in}{1.956609in}}{\pgfqpoint{2.868818in}{1.953336in}}{\pgfqpoint{2.862994in}{1.947513in}}%
\pgfpathcurveto{\pgfqpoint{2.857170in}{1.941689in}}{\pgfqpoint{2.853897in}{1.933789in}}{\pgfqpoint{2.853897in}{1.925552in}}%
\pgfpathcurveto{\pgfqpoint{2.853897in}{1.917316in}}{\pgfqpoint{2.857170in}{1.909416in}}{\pgfqpoint{2.862994in}{1.903592in}}%
\pgfpathcurveto{\pgfqpoint{2.868818in}{1.897768in}}{\pgfqpoint{2.876718in}{1.894496in}}{\pgfqpoint{2.884954in}{1.894496in}}%
\pgfpathclose%
\pgfusepath{stroke,fill}%
\end{pgfscope}%
\begin{pgfscope}%
\pgfpathrectangle{\pgfqpoint{0.100000in}{0.220728in}}{\pgfqpoint{3.696000in}{3.696000in}}%
\pgfusepath{clip}%
\pgfsetbuttcap%
\pgfsetroundjoin%
\definecolor{currentfill}{rgb}{0.121569,0.466667,0.705882}%
\pgfsetfillcolor{currentfill}%
\pgfsetfillopacity{0.854345}%
\pgfsetlinewidth{1.003750pt}%
\definecolor{currentstroke}{rgb}{0.121569,0.466667,0.705882}%
\pgfsetstrokecolor{currentstroke}%
\pgfsetstrokeopacity{0.854345}%
\pgfsetdash{}{0pt}%
\pgfpathmoveto{\pgfqpoint{2.878530in}{1.883234in}}%
\pgfpathcurveto{\pgfqpoint{2.886766in}{1.883234in}}{\pgfqpoint{2.894666in}{1.886506in}}{\pgfqpoint{2.900490in}{1.892330in}}%
\pgfpathcurveto{\pgfqpoint{2.906314in}{1.898154in}}{\pgfqpoint{2.909586in}{1.906054in}}{\pgfqpoint{2.909586in}{1.914290in}}%
\pgfpathcurveto{\pgfqpoint{2.909586in}{1.922527in}}{\pgfqpoint{2.906314in}{1.930427in}}{\pgfqpoint{2.900490in}{1.936251in}}%
\pgfpathcurveto{\pgfqpoint{2.894666in}{1.942075in}}{\pgfqpoint{2.886766in}{1.945347in}}{\pgfqpoint{2.878530in}{1.945347in}}%
\pgfpathcurveto{\pgfqpoint{2.870293in}{1.945347in}}{\pgfqpoint{2.862393in}{1.942075in}}{\pgfqpoint{2.856569in}{1.936251in}}%
\pgfpathcurveto{\pgfqpoint{2.850746in}{1.930427in}}{\pgfqpoint{2.847473in}{1.922527in}}{\pgfqpoint{2.847473in}{1.914290in}}%
\pgfpathcurveto{\pgfqpoint{2.847473in}{1.906054in}}{\pgfqpoint{2.850746in}{1.898154in}}{\pgfqpoint{2.856569in}{1.892330in}}%
\pgfpathcurveto{\pgfqpoint{2.862393in}{1.886506in}}{\pgfqpoint{2.870293in}{1.883234in}}{\pgfqpoint{2.878530in}{1.883234in}}%
\pgfpathclose%
\pgfusepath{stroke,fill}%
\end{pgfscope}%
\begin{pgfscope}%
\pgfpathrectangle{\pgfqpoint{0.100000in}{0.220728in}}{\pgfqpoint{3.696000in}{3.696000in}}%
\pgfusepath{clip}%
\pgfsetbuttcap%
\pgfsetroundjoin%
\definecolor{currentfill}{rgb}{0.121569,0.466667,0.705882}%
\pgfsetfillcolor{currentfill}%
\pgfsetfillopacity{0.855144}%
\pgfsetlinewidth{1.003750pt}%
\definecolor{currentstroke}{rgb}{0.121569,0.466667,0.705882}%
\pgfsetstrokecolor{currentstroke}%
\pgfsetstrokeopacity{0.855144}%
\pgfsetdash{}{0pt}%
\pgfpathmoveto{\pgfqpoint{2.874446in}{1.877569in}}%
\pgfpathcurveto{\pgfqpoint{2.882683in}{1.877569in}}{\pgfqpoint{2.890583in}{1.880842in}}{\pgfqpoint{2.896407in}{1.886666in}}%
\pgfpathcurveto{\pgfqpoint{2.902231in}{1.892490in}}{\pgfqpoint{2.905503in}{1.900390in}}{\pgfqpoint{2.905503in}{1.908626in}}%
\pgfpathcurveto{\pgfqpoint{2.905503in}{1.916862in}}{\pgfqpoint{2.902231in}{1.924762in}}{\pgfqpoint{2.896407in}{1.930586in}}%
\pgfpathcurveto{\pgfqpoint{2.890583in}{1.936410in}}{\pgfqpoint{2.882683in}{1.939682in}}{\pgfqpoint{2.874446in}{1.939682in}}%
\pgfpathcurveto{\pgfqpoint{2.866210in}{1.939682in}}{\pgfqpoint{2.858310in}{1.936410in}}{\pgfqpoint{2.852486in}{1.930586in}}%
\pgfpathcurveto{\pgfqpoint{2.846662in}{1.924762in}}{\pgfqpoint{2.843390in}{1.916862in}}{\pgfqpoint{2.843390in}{1.908626in}}%
\pgfpathcurveto{\pgfqpoint{2.843390in}{1.900390in}}{\pgfqpoint{2.846662in}{1.892490in}}{\pgfqpoint{2.852486in}{1.886666in}}%
\pgfpathcurveto{\pgfqpoint{2.858310in}{1.880842in}}{\pgfqpoint{2.866210in}{1.877569in}}{\pgfqpoint{2.874446in}{1.877569in}}%
\pgfpathclose%
\pgfusepath{stroke,fill}%
\end{pgfscope}%
\begin{pgfscope}%
\pgfpathrectangle{\pgfqpoint{0.100000in}{0.220728in}}{\pgfqpoint{3.696000in}{3.696000in}}%
\pgfusepath{clip}%
\pgfsetbuttcap%
\pgfsetroundjoin%
\definecolor{currentfill}{rgb}{0.121569,0.466667,0.705882}%
\pgfsetfillcolor{currentfill}%
\pgfsetfillopacity{0.855324}%
\pgfsetlinewidth{1.003750pt}%
\definecolor{currentstroke}{rgb}{0.121569,0.466667,0.705882}%
\pgfsetstrokecolor{currentstroke}%
\pgfsetstrokeopacity{0.855324}%
\pgfsetdash{}{0pt}%
\pgfpathmoveto{\pgfqpoint{1.629904in}{1.000625in}}%
\pgfpathcurveto{\pgfqpoint{1.638140in}{1.000625in}}{\pgfqpoint{1.646040in}{1.003898in}}{\pgfqpoint{1.651864in}{1.009721in}}%
\pgfpathcurveto{\pgfqpoint{1.657688in}{1.015545in}}{\pgfqpoint{1.660960in}{1.023445in}}{\pgfqpoint{1.660960in}{1.031682in}}%
\pgfpathcurveto{\pgfqpoint{1.660960in}{1.039918in}}{\pgfqpoint{1.657688in}{1.047818in}}{\pgfqpoint{1.651864in}{1.053642in}}%
\pgfpathcurveto{\pgfqpoint{1.646040in}{1.059466in}}{\pgfqpoint{1.638140in}{1.062738in}}{\pgfqpoint{1.629904in}{1.062738in}}%
\pgfpathcurveto{\pgfqpoint{1.621668in}{1.062738in}}{\pgfqpoint{1.613767in}{1.059466in}}{\pgfqpoint{1.607944in}{1.053642in}}%
\pgfpathcurveto{\pgfqpoint{1.602120in}{1.047818in}}{\pgfqpoint{1.598847in}{1.039918in}}{\pgfqpoint{1.598847in}{1.031682in}}%
\pgfpathcurveto{\pgfqpoint{1.598847in}{1.023445in}}{\pgfqpoint{1.602120in}{1.015545in}}{\pgfqpoint{1.607944in}{1.009721in}}%
\pgfpathcurveto{\pgfqpoint{1.613767in}{1.003898in}}{\pgfqpoint{1.621668in}{1.000625in}}{\pgfqpoint{1.629904in}{1.000625in}}%
\pgfpathclose%
\pgfusepath{stroke,fill}%
\end{pgfscope}%
\begin{pgfscope}%
\pgfpathrectangle{\pgfqpoint{0.100000in}{0.220728in}}{\pgfqpoint{3.696000in}{3.696000in}}%
\pgfusepath{clip}%
\pgfsetbuttcap%
\pgfsetroundjoin%
\definecolor{currentfill}{rgb}{0.121569,0.466667,0.705882}%
\pgfsetfillcolor{currentfill}%
\pgfsetfillopacity{0.857114}%
\pgfsetlinewidth{1.003750pt}%
\definecolor{currentstroke}{rgb}{0.121569,0.466667,0.705882}%
\pgfsetstrokecolor{currentstroke}%
\pgfsetstrokeopacity{0.857114}%
\pgfsetdash{}{0pt}%
\pgfpathmoveto{\pgfqpoint{2.871061in}{1.862685in}}%
\pgfpathcurveto{\pgfqpoint{2.879297in}{1.862685in}}{\pgfqpoint{2.887197in}{1.865957in}}{\pgfqpoint{2.893021in}{1.871781in}}%
\pgfpathcurveto{\pgfqpoint{2.898845in}{1.877605in}}{\pgfqpoint{2.902117in}{1.885505in}}{\pgfqpoint{2.902117in}{1.893741in}}%
\pgfpathcurveto{\pgfqpoint{2.902117in}{1.901977in}}{\pgfqpoint{2.898845in}{1.909877in}}{\pgfqpoint{2.893021in}{1.915701in}}%
\pgfpathcurveto{\pgfqpoint{2.887197in}{1.921525in}}{\pgfqpoint{2.879297in}{1.924798in}}{\pgfqpoint{2.871061in}{1.924798in}}%
\pgfpathcurveto{\pgfqpoint{2.862824in}{1.924798in}}{\pgfqpoint{2.854924in}{1.921525in}}{\pgfqpoint{2.849100in}{1.915701in}}%
\pgfpathcurveto{\pgfqpoint{2.843277in}{1.909877in}}{\pgfqpoint{2.840004in}{1.901977in}}{\pgfqpoint{2.840004in}{1.893741in}}%
\pgfpathcurveto{\pgfqpoint{2.840004in}{1.885505in}}{\pgfqpoint{2.843277in}{1.877605in}}{\pgfqpoint{2.849100in}{1.871781in}}%
\pgfpathcurveto{\pgfqpoint{2.854924in}{1.865957in}}{\pgfqpoint{2.862824in}{1.862685in}}{\pgfqpoint{2.871061in}{1.862685in}}%
\pgfpathclose%
\pgfusepath{stroke,fill}%
\end{pgfscope}%
\begin{pgfscope}%
\pgfpathrectangle{\pgfqpoint{0.100000in}{0.220728in}}{\pgfqpoint{3.696000in}{3.696000in}}%
\pgfusepath{clip}%
\pgfsetbuttcap%
\pgfsetroundjoin%
\definecolor{currentfill}{rgb}{0.121569,0.466667,0.705882}%
\pgfsetfillcolor{currentfill}%
\pgfsetfillopacity{0.858075}%
\pgfsetlinewidth{1.003750pt}%
\definecolor{currentstroke}{rgb}{0.121569,0.466667,0.705882}%
\pgfsetstrokecolor{currentstroke}%
\pgfsetstrokeopacity{0.858075}%
\pgfsetdash{}{0pt}%
\pgfpathmoveto{\pgfqpoint{2.867361in}{1.855757in}}%
\pgfpathcurveto{\pgfqpoint{2.875597in}{1.855757in}}{\pgfqpoint{2.883497in}{1.859029in}}{\pgfqpoint{2.889321in}{1.864853in}}%
\pgfpathcurveto{\pgfqpoint{2.895145in}{1.870677in}}{\pgfqpoint{2.898417in}{1.878577in}}{\pgfqpoint{2.898417in}{1.886813in}}%
\pgfpathcurveto{\pgfqpoint{2.898417in}{1.895050in}}{\pgfqpoint{2.895145in}{1.902950in}}{\pgfqpoint{2.889321in}{1.908774in}}%
\pgfpathcurveto{\pgfqpoint{2.883497in}{1.914598in}}{\pgfqpoint{2.875597in}{1.917870in}}{\pgfqpoint{2.867361in}{1.917870in}}%
\pgfpathcurveto{\pgfqpoint{2.859125in}{1.917870in}}{\pgfqpoint{2.851225in}{1.914598in}}{\pgfqpoint{2.845401in}{1.908774in}}%
\pgfpathcurveto{\pgfqpoint{2.839577in}{1.902950in}}{\pgfqpoint{2.836304in}{1.895050in}}{\pgfqpoint{2.836304in}{1.886813in}}%
\pgfpathcurveto{\pgfqpoint{2.836304in}{1.878577in}}{\pgfqpoint{2.839577in}{1.870677in}}{\pgfqpoint{2.845401in}{1.864853in}}%
\pgfpathcurveto{\pgfqpoint{2.851225in}{1.859029in}}{\pgfqpoint{2.859125in}{1.855757in}}{\pgfqpoint{2.867361in}{1.855757in}}%
\pgfpathclose%
\pgfusepath{stroke,fill}%
\end{pgfscope}%
\begin{pgfscope}%
\pgfpathrectangle{\pgfqpoint{0.100000in}{0.220728in}}{\pgfqpoint{3.696000in}{3.696000in}}%
\pgfusepath{clip}%
\pgfsetbuttcap%
\pgfsetroundjoin%
\definecolor{currentfill}{rgb}{0.121569,0.466667,0.705882}%
\pgfsetfillcolor{currentfill}%
\pgfsetfillopacity{0.858833}%
\pgfsetlinewidth{1.003750pt}%
\definecolor{currentstroke}{rgb}{0.121569,0.466667,0.705882}%
\pgfsetstrokecolor{currentstroke}%
\pgfsetstrokeopacity{0.858833}%
\pgfsetdash{}{0pt}%
\pgfpathmoveto{\pgfqpoint{2.862009in}{1.848572in}}%
\pgfpathcurveto{\pgfqpoint{2.870245in}{1.848572in}}{\pgfqpoint{2.878145in}{1.851845in}}{\pgfqpoint{2.883969in}{1.857669in}}%
\pgfpathcurveto{\pgfqpoint{2.889793in}{1.863493in}}{\pgfqpoint{2.893065in}{1.871393in}}{\pgfqpoint{2.893065in}{1.879629in}}%
\pgfpathcurveto{\pgfqpoint{2.893065in}{1.887865in}}{\pgfqpoint{2.889793in}{1.895765in}}{\pgfqpoint{2.883969in}{1.901589in}}%
\pgfpathcurveto{\pgfqpoint{2.878145in}{1.907413in}}{\pgfqpoint{2.870245in}{1.910685in}}{\pgfqpoint{2.862009in}{1.910685in}}%
\pgfpathcurveto{\pgfqpoint{2.853773in}{1.910685in}}{\pgfqpoint{2.845873in}{1.907413in}}{\pgfqpoint{2.840049in}{1.901589in}}%
\pgfpathcurveto{\pgfqpoint{2.834225in}{1.895765in}}{\pgfqpoint{2.830952in}{1.887865in}}{\pgfqpoint{2.830952in}{1.879629in}}%
\pgfpathcurveto{\pgfqpoint{2.830952in}{1.871393in}}{\pgfqpoint{2.834225in}{1.863493in}}{\pgfqpoint{2.840049in}{1.857669in}}%
\pgfpathcurveto{\pgfqpoint{2.845873in}{1.851845in}}{\pgfqpoint{2.853773in}{1.848572in}}{\pgfqpoint{2.862009in}{1.848572in}}%
\pgfpathclose%
\pgfusepath{stroke,fill}%
\end{pgfscope}%
\begin{pgfscope}%
\pgfpathrectangle{\pgfqpoint{0.100000in}{0.220728in}}{\pgfqpoint{3.696000in}{3.696000in}}%
\pgfusepath{clip}%
\pgfsetbuttcap%
\pgfsetroundjoin%
\definecolor{currentfill}{rgb}{0.121569,0.466667,0.705882}%
\pgfsetfillcolor{currentfill}%
\pgfsetfillopacity{0.859831}%
\pgfsetlinewidth{1.003750pt}%
\definecolor{currentstroke}{rgb}{0.121569,0.466667,0.705882}%
\pgfsetstrokecolor{currentstroke}%
\pgfsetstrokeopacity{0.859831}%
\pgfsetdash{}{0pt}%
\pgfpathmoveto{\pgfqpoint{1.654282in}{0.988796in}}%
\pgfpathcurveto{\pgfqpoint{1.662519in}{0.988796in}}{\pgfqpoint{1.670419in}{0.992069in}}{\pgfqpoint{1.676243in}{0.997893in}}%
\pgfpathcurveto{\pgfqpoint{1.682067in}{1.003716in}}{\pgfqpoint{1.685339in}{1.011617in}}{\pgfqpoint{1.685339in}{1.019853in}}%
\pgfpathcurveto{\pgfqpoint{1.685339in}{1.028089in}}{\pgfqpoint{1.682067in}{1.035989in}}{\pgfqpoint{1.676243in}{1.041813in}}%
\pgfpathcurveto{\pgfqpoint{1.670419in}{1.047637in}}{\pgfqpoint{1.662519in}{1.050909in}}{\pgfqpoint{1.654282in}{1.050909in}}%
\pgfpathcurveto{\pgfqpoint{1.646046in}{1.050909in}}{\pgfqpoint{1.638146in}{1.047637in}}{\pgfqpoint{1.632322in}{1.041813in}}%
\pgfpathcurveto{\pgfqpoint{1.626498in}{1.035989in}}{\pgfqpoint{1.623226in}{1.028089in}}{\pgfqpoint{1.623226in}{1.019853in}}%
\pgfpathcurveto{\pgfqpoint{1.623226in}{1.011617in}}{\pgfqpoint{1.626498in}{1.003716in}}{\pgfqpoint{1.632322in}{0.997893in}}%
\pgfpathcurveto{\pgfqpoint{1.638146in}{0.992069in}}{\pgfqpoint{1.646046in}{0.988796in}}{\pgfqpoint{1.654282in}{0.988796in}}%
\pgfpathclose%
\pgfusepath{stroke,fill}%
\end{pgfscope}%
\begin{pgfscope}%
\pgfpathrectangle{\pgfqpoint{0.100000in}{0.220728in}}{\pgfqpoint{3.696000in}{3.696000in}}%
\pgfusepath{clip}%
\pgfsetbuttcap%
\pgfsetroundjoin%
\definecolor{currentfill}{rgb}{0.121569,0.466667,0.705882}%
\pgfsetfillcolor{currentfill}%
\pgfsetfillopacity{0.860375}%
\pgfsetlinewidth{1.003750pt}%
\definecolor{currentstroke}{rgb}{0.121569,0.466667,0.705882}%
\pgfsetstrokecolor{currentstroke}%
\pgfsetstrokeopacity{0.860375}%
\pgfsetdash{}{0pt}%
\pgfpathmoveto{\pgfqpoint{2.858543in}{1.834386in}}%
\pgfpathcurveto{\pgfqpoint{2.866779in}{1.834386in}}{\pgfqpoint{2.874679in}{1.837658in}}{\pgfqpoint{2.880503in}{1.843482in}}%
\pgfpathcurveto{\pgfqpoint{2.886327in}{1.849306in}}{\pgfqpoint{2.889599in}{1.857206in}}{\pgfqpoint{2.889599in}{1.865442in}}%
\pgfpathcurveto{\pgfqpoint{2.889599in}{1.873679in}}{\pgfqpoint{2.886327in}{1.881579in}}{\pgfqpoint{2.880503in}{1.887403in}}%
\pgfpathcurveto{\pgfqpoint{2.874679in}{1.893226in}}{\pgfqpoint{2.866779in}{1.896499in}}{\pgfqpoint{2.858543in}{1.896499in}}%
\pgfpathcurveto{\pgfqpoint{2.850306in}{1.896499in}}{\pgfqpoint{2.842406in}{1.893226in}}{\pgfqpoint{2.836582in}{1.887403in}}%
\pgfpathcurveto{\pgfqpoint{2.830759in}{1.881579in}}{\pgfqpoint{2.827486in}{1.873679in}}{\pgfqpoint{2.827486in}{1.865442in}}%
\pgfpathcurveto{\pgfqpoint{2.827486in}{1.857206in}}{\pgfqpoint{2.830759in}{1.849306in}}{\pgfqpoint{2.836582in}{1.843482in}}%
\pgfpathcurveto{\pgfqpoint{2.842406in}{1.837658in}}{\pgfqpoint{2.850306in}{1.834386in}}{\pgfqpoint{2.858543in}{1.834386in}}%
\pgfpathclose%
\pgfusepath{stroke,fill}%
\end{pgfscope}%
\begin{pgfscope}%
\pgfpathrectangle{\pgfqpoint{0.100000in}{0.220728in}}{\pgfqpoint{3.696000in}{3.696000in}}%
\pgfusepath{clip}%
\pgfsetbuttcap%
\pgfsetroundjoin%
\definecolor{currentfill}{rgb}{0.121569,0.466667,0.705882}%
\pgfsetfillcolor{currentfill}%
\pgfsetfillopacity{0.862204}%
\pgfsetlinewidth{1.003750pt}%
\definecolor{currentstroke}{rgb}{0.121569,0.466667,0.705882}%
\pgfsetstrokecolor{currentstroke}%
\pgfsetstrokeopacity{0.862204}%
\pgfsetdash{}{0pt}%
\pgfpathmoveto{\pgfqpoint{2.850374in}{1.822464in}}%
\pgfpathcurveto{\pgfqpoint{2.858611in}{1.822464in}}{\pgfqpoint{2.866511in}{1.825736in}}{\pgfqpoint{2.872335in}{1.831560in}}%
\pgfpathcurveto{\pgfqpoint{2.878159in}{1.837384in}}{\pgfqpoint{2.881431in}{1.845284in}}{\pgfqpoint{2.881431in}{1.853521in}}%
\pgfpathcurveto{\pgfqpoint{2.881431in}{1.861757in}}{\pgfqpoint{2.878159in}{1.869657in}}{\pgfqpoint{2.872335in}{1.875481in}}%
\pgfpathcurveto{\pgfqpoint{2.866511in}{1.881305in}}{\pgfqpoint{2.858611in}{1.884577in}}{\pgfqpoint{2.850374in}{1.884577in}}%
\pgfpathcurveto{\pgfqpoint{2.842138in}{1.884577in}}{\pgfqpoint{2.834238in}{1.881305in}}{\pgfqpoint{2.828414in}{1.875481in}}%
\pgfpathcurveto{\pgfqpoint{2.822590in}{1.869657in}}{\pgfqpoint{2.819318in}{1.861757in}}{\pgfqpoint{2.819318in}{1.853521in}}%
\pgfpathcurveto{\pgfqpoint{2.819318in}{1.845284in}}{\pgfqpoint{2.822590in}{1.837384in}}{\pgfqpoint{2.828414in}{1.831560in}}%
\pgfpathcurveto{\pgfqpoint{2.834238in}{1.825736in}}{\pgfqpoint{2.842138in}{1.822464in}}{\pgfqpoint{2.850374in}{1.822464in}}%
\pgfpathclose%
\pgfusepath{stroke,fill}%
\end{pgfscope}%
\begin{pgfscope}%
\pgfpathrectangle{\pgfqpoint{0.100000in}{0.220728in}}{\pgfqpoint{3.696000in}{3.696000in}}%
\pgfusepath{clip}%
\pgfsetbuttcap%
\pgfsetroundjoin%
\definecolor{currentfill}{rgb}{0.121569,0.466667,0.705882}%
\pgfsetfillcolor{currentfill}%
\pgfsetfillopacity{0.863142}%
\pgfsetlinewidth{1.003750pt}%
\definecolor{currentstroke}{rgb}{0.121569,0.466667,0.705882}%
\pgfsetstrokecolor{currentstroke}%
\pgfsetstrokeopacity{0.863142}%
\pgfsetdash{}{0pt}%
\pgfpathmoveto{\pgfqpoint{2.846044in}{1.815328in}}%
\pgfpathcurveto{\pgfqpoint{2.854280in}{1.815328in}}{\pgfqpoint{2.862180in}{1.818600in}}{\pgfqpoint{2.868004in}{1.824424in}}%
\pgfpathcurveto{\pgfqpoint{2.873828in}{1.830248in}}{\pgfqpoint{2.877100in}{1.838148in}}{\pgfqpoint{2.877100in}{1.846384in}}%
\pgfpathcurveto{\pgfqpoint{2.877100in}{1.854621in}}{\pgfqpoint{2.873828in}{1.862521in}}{\pgfqpoint{2.868004in}{1.868345in}}%
\pgfpathcurveto{\pgfqpoint{2.862180in}{1.874168in}}{\pgfqpoint{2.854280in}{1.877441in}}{\pgfqpoint{2.846044in}{1.877441in}}%
\pgfpathcurveto{\pgfqpoint{2.837807in}{1.877441in}}{\pgfqpoint{2.829907in}{1.874168in}}{\pgfqpoint{2.824083in}{1.868345in}}%
\pgfpathcurveto{\pgfqpoint{2.818259in}{1.862521in}}{\pgfqpoint{2.814987in}{1.854621in}}{\pgfqpoint{2.814987in}{1.846384in}}%
\pgfpathcurveto{\pgfqpoint{2.814987in}{1.838148in}}{\pgfqpoint{2.818259in}{1.830248in}}{\pgfqpoint{2.824083in}{1.824424in}}%
\pgfpathcurveto{\pgfqpoint{2.829907in}{1.818600in}}{\pgfqpoint{2.837807in}{1.815328in}}{\pgfqpoint{2.846044in}{1.815328in}}%
\pgfpathclose%
\pgfusepath{stroke,fill}%
\end{pgfscope}%
\begin{pgfscope}%
\pgfpathrectangle{\pgfqpoint{0.100000in}{0.220728in}}{\pgfqpoint{3.696000in}{3.696000in}}%
\pgfusepath{clip}%
\pgfsetbuttcap%
\pgfsetroundjoin%
\definecolor{currentfill}{rgb}{0.121569,0.466667,0.705882}%
\pgfsetfillcolor{currentfill}%
\pgfsetfillopacity{0.863747}%
\pgfsetlinewidth{1.003750pt}%
\definecolor{currentstroke}{rgb}{0.121569,0.466667,0.705882}%
\pgfsetstrokecolor{currentstroke}%
\pgfsetstrokeopacity{0.863747}%
\pgfsetdash{}{0pt}%
\pgfpathmoveto{\pgfqpoint{2.844743in}{1.810520in}}%
\pgfpathcurveto{\pgfqpoint{2.852980in}{1.810520in}}{\pgfqpoint{2.860880in}{1.813792in}}{\pgfqpoint{2.866704in}{1.819616in}}%
\pgfpathcurveto{\pgfqpoint{2.872528in}{1.825440in}}{\pgfqpoint{2.875800in}{1.833340in}}{\pgfqpoint{2.875800in}{1.841577in}}%
\pgfpathcurveto{\pgfqpoint{2.875800in}{1.849813in}}{\pgfqpoint{2.872528in}{1.857713in}}{\pgfqpoint{2.866704in}{1.863537in}}%
\pgfpathcurveto{\pgfqpoint{2.860880in}{1.869361in}}{\pgfqpoint{2.852980in}{1.872633in}}{\pgfqpoint{2.844743in}{1.872633in}}%
\pgfpathcurveto{\pgfqpoint{2.836507in}{1.872633in}}{\pgfqpoint{2.828607in}{1.869361in}}{\pgfqpoint{2.822783in}{1.863537in}}%
\pgfpathcurveto{\pgfqpoint{2.816959in}{1.857713in}}{\pgfqpoint{2.813687in}{1.849813in}}{\pgfqpoint{2.813687in}{1.841577in}}%
\pgfpathcurveto{\pgfqpoint{2.813687in}{1.833340in}}{\pgfqpoint{2.816959in}{1.825440in}}{\pgfqpoint{2.822783in}{1.819616in}}%
\pgfpathcurveto{\pgfqpoint{2.828607in}{1.813792in}}{\pgfqpoint{2.836507in}{1.810520in}}{\pgfqpoint{2.844743in}{1.810520in}}%
\pgfpathclose%
\pgfusepath{stroke,fill}%
\end{pgfscope}%
\begin{pgfscope}%
\pgfpathrectangle{\pgfqpoint{0.100000in}{0.220728in}}{\pgfqpoint{3.696000in}{3.696000in}}%
\pgfusepath{clip}%
\pgfsetbuttcap%
\pgfsetroundjoin%
\definecolor{currentfill}{rgb}{0.121569,0.466667,0.705882}%
\pgfsetfillcolor{currentfill}%
\pgfsetfillopacity{0.864452}%
\pgfsetlinewidth{1.003750pt}%
\definecolor{currentstroke}{rgb}{0.121569,0.466667,0.705882}%
\pgfsetstrokecolor{currentstroke}%
\pgfsetstrokeopacity{0.864452}%
\pgfsetdash{}{0pt}%
\pgfpathmoveto{\pgfqpoint{2.839321in}{1.801600in}}%
\pgfpathcurveto{\pgfqpoint{2.847557in}{1.801600in}}{\pgfqpoint{2.855457in}{1.804872in}}{\pgfqpoint{2.861281in}{1.810696in}}%
\pgfpathcurveto{\pgfqpoint{2.867105in}{1.816520in}}{\pgfqpoint{2.870377in}{1.824420in}}{\pgfqpoint{2.870377in}{1.832656in}}%
\pgfpathcurveto{\pgfqpoint{2.870377in}{1.840892in}}{\pgfqpoint{2.867105in}{1.848792in}}{\pgfqpoint{2.861281in}{1.854616in}}%
\pgfpathcurveto{\pgfqpoint{2.855457in}{1.860440in}}{\pgfqpoint{2.847557in}{1.863713in}}{\pgfqpoint{2.839321in}{1.863713in}}%
\pgfpathcurveto{\pgfqpoint{2.831084in}{1.863713in}}{\pgfqpoint{2.823184in}{1.860440in}}{\pgfqpoint{2.817360in}{1.854616in}}%
\pgfpathcurveto{\pgfqpoint{2.811536in}{1.848792in}}{\pgfqpoint{2.808264in}{1.840892in}}{\pgfqpoint{2.808264in}{1.832656in}}%
\pgfpathcurveto{\pgfqpoint{2.808264in}{1.824420in}}{\pgfqpoint{2.811536in}{1.816520in}}{\pgfqpoint{2.817360in}{1.810696in}}%
\pgfpathcurveto{\pgfqpoint{2.823184in}{1.804872in}}{\pgfqpoint{2.831084in}{1.801600in}}{\pgfqpoint{2.839321in}{1.801600in}}%
\pgfpathclose%
\pgfusepath{stroke,fill}%
\end{pgfscope}%
\begin{pgfscope}%
\pgfpathrectangle{\pgfqpoint{0.100000in}{0.220728in}}{\pgfqpoint{3.696000in}{3.696000in}}%
\pgfusepath{clip}%
\pgfsetbuttcap%
\pgfsetroundjoin%
\definecolor{currentfill}{rgb}{0.121569,0.466667,0.705882}%
\pgfsetfillcolor{currentfill}%
\pgfsetfillopacity{0.864880}%
\pgfsetlinewidth{1.003750pt}%
\definecolor{currentstroke}{rgb}{0.121569,0.466667,0.705882}%
\pgfsetstrokecolor{currentstroke}%
\pgfsetstrokeopacity{0.864880}%
\pgfsetdash{}{0pt}%
\pgfpathmoveto{\pgfqpoint{1.675882in}{0.984425in}}%
\pgfpathcurveto{\pgfqpoint{1.684118in}{0.984425in}}{\pgfqpoint{1.692018in}{0.987698in}}{\pgfqpoint{1.697842in}{0.993521in}}%
\pgfpathcurveto{\pgfqpoint{1.703666in}{0.999345in}}{\pgfqpoint{1.706939in}{1.007245in}}{\pgfqpoint{1.706939in}{1.015482in}}%
\pgfpathcurveto{\pgfqpoint{1.706939in}{1.023718in}}{\pgfqpoint{1.703666in}{1.031618in}}{\pgfqpoint{1.697842in}{1.037442in}}%
\pgfpathcurveto{\pgfqpoint{1.692018in}{1.043266in}}{\pgfqpoint{1.684118in}{1.046538in}}{\pgfqpoint{1.675882in}{1.046538in}}%
\pgfpathcurveto{\pgfqpoint{1.667646in}{1.046538in}}{\pgfqpoint{1.659746in}{1.043266in}}{\pgfqpoint{1.653922in}{1.037442in}}%
\pgfpathcurveto{\pgfqpoint{1.648098in}{1.031618in}}{\pgfqpoint{1.644826in}{1.023718in}}{\pgfqpoint{1.644826in}{1.015482in}}%
\pgfpathcurveto{\pgfqpoint{1.644826in}{1.007245in}}{\pgfqpoint{1.648098in}{0.999345in}}{\pgfqpoint{1.653922in}{0.993521in}}%
\pgfpathcurveto{\pgfqpoint{1.659746in}{0.987698in}}{\pgfqpoint{1.667646in}{0.984425in}}{\pgfqpoint{1.675882in}{0.984425in}}%
\pgfpathclose%
\pgfusepath{stroke,fill}%
\end{pgfscope}%
\begin{pgfscope}%
\pgfpathrectangle{\pgfqpoint{0.100000in}{0.220728in}}{\pgfqpoint{3.696000in}{3.696000in}}%
\pgfusepath{clip}%
\pgfsetbuttcap%
\pgfsetroundjoin%
\definecolor{currentfill}{rgb}{0.121569,0.466667,0.705882}%
\pgfsetfillcolor{currentfill}%
\pgfsetfillopacity{0.865855}%
\pgfsetlinewidth{1.003750pt}%
\definecolor{currentstroke}{rgb}{0.121569,0.466667,0.705882}%
\pgfsetstrokecolor{currentstroke}%
\pgfsetstrokeopacity{0.865855}%
\pgfsetdash{}{0pt}%
\pgfpathmoveto{\pgfqpoint{2.834834in}{1.791722in}}%
\pgfpathcurveto{\pgfqpoint{2.843071in}{1.791722in}}{\pgfqpoint{2.850971in}{1.794995in}}{\pgfqpoint{2.856795in}{1.800818in}}%
\pgfpathcurveto{\pgfqpoint{2.862619in}{1.806642in}}{\pgfqpoint{2.865891in}{1.814542in}}{\pgfqpoint{2.865891in}{1.822779in}}%
\pgfpathcurveto{\pgfqpoint{2.865891in}{1.831015in}}{\pgfqpoint{2.862619in}{1.838915in}}{\pgfqpoint{2.856795in}{1.844739in}}%
\pgfpathcurveto{\pgfqpoint{2.850971in}{1.850563in}}{\pgfqpoint{2.843071in}{1.853835in}}{\pgfqpoint{2.834834in}{1.853835in}}%
\pgfpathcurveto{\pgfqpoint{2.826598in}{1.853835in}}{\pgfqpoint{2.818698in}{1.850563in}}{\pgfqpoint{2.812874in}{1.844739in}}%
\pgfpathcurveto{\pgfqpoint{2.807050in}{1.838915in}}{\pgfqpoint{2.803778in}{1.831015in}}{\pgfqpoint{2.803778in}{1.822779in}}%
\pgfpathcurveto{\pgfqpoint{2.803778in}{1.814542in}}{\pgfqpoint{2.807050in}{1.806642in}}{\pgfqpoint{2.812874in}{1.800818in}}%
\pgfpathcurveto{\pgfqpoint{2.818698in}{1.794995in}}{\pgfqpoint{2.826598in}{1.791722in}}{\pgfqpoint{2.834834in}{1.791722in}}%
\pgfpathclose%
\pgfusepath{stroke,fill}%
\end{pgfscope}%
\begin{pgfscope}%
\pgfpathrectangle{\pgfqpoint{0.100000in}{0.220728in}}{\pgfqpoint{3.696000in}{3.696000in}}%
\pgfusepath{clip}%
\pgfsetbuttcap%
\pgfsetroundjoin%
\definecolor{currentfill}{rgb}{0.121569,0.466667,0.705882}%
\pgfsetfillcolor{currentfill}%
\pgfsetfillopacity{0.867293}%
\pgfsetlinewidth{1.003750pt}%
\definecolor{currentstroke}{rgb}{0.121569,0.466667,0.705882}%
\pgfsetstrokecolor{currentstroke}%
\pgfsetstrokeopacity{0.867293}%
\pgfsetdash{}{0pt}%
\pgfpathmoveto{\pgfqpoint{2.831201in}{1.779644in}}%
\pgfpathcurveto{\pgfqpoint{2.839437in}{1.779644in}}{\pgfqpoint{2.847337in}{1.782916in}}{\pgfqpoint{2.853161in}{1.788740in}}%
\pgfpathcurveto{\pgfqpoint{2.858985in}{1.794564in}}{\pgfqpoint{2.862258in}{1.802464in}}{\pgfqpoint{2.862258in}{1.810701in}}%
\pgfpathcurveto{\pgfqpoint{2.862258in}{1.818937in}}{\pgfqpoint{2.858985in}{1.826837in}}{\pgfqpoint{2.853161in}{1.832661in}}%
\pgfpathcurveto{\pgfqpoint{2.847337in}{1.838485in}}{\pgfqpoint{2.839437in}{1.841757in}}{\pgfqpoint{2.831201in}{1.841757in}}%
\pgfpathcurveto{\pgfqpoint{2.822965in}{1.841757in}}{\pgfqpoint{2.815065in}{1.838485in}}{\pgfqpoint{2.809241in}{1.832661in}}%
\pgfpathcurveto{\pgfqpoint{2.803417in}{1.826837in}}{\pgfqpoint{2.800145in}{1.818937in}}{\pgfqpoint{2.800145in}{1.810701in}}%
\pgfpathcurveto{\pgfqpoint{2.800145in}{1.802464in}}{\pgfqpoint{2.803417in}{1.794564in}}{\pgfqpoint{2.809241in}{1.788740in}}%
\pgfpathcurveto{\pgfqpoint{2.815065in}{1.782916in}}{\pgfqpoint{2.822965in}{1.779644in}}{\pgfqpoint{2.831201in}{1.779644in}}%
\pgfpathclose%
\pgfusepath{stroke,fill}%
\end{pgfscope}%
\begin{pgfscope}%
\pgfpathrectangle{\pgfqpoint{0.100000in}{0.220728in}}{\pgfqpoint{3.696000in}{3.696000in}}%
\pgfusepath{clip}%
\pgfsetbuttcap%
\pgfsetroundjoin%
\definecolor{currentfill}{rgb}{0.121569,0.466667,0.705882}%
\pgfsetfillcolor{currentfill}%
\pgfsetfillopacity{0.867727}%
\pgfsetlinewidth{1.003750pt}%
\definecolor{currentstroke}{rgb}{0.121569,0.466667,0.705882}%
\pgfsetstrokecolor{currentstroke}%
\pgfsetstrokeopacity{0.867727}%
\pgfsetdash{}{0pt}%
\pgfpathmoveto{\pgfqpoint{2.827552in}{1.773791in}}%
\pgfpathcurveto{\pgfqpoint{2.835789in}{1.773791in}}{\pgfqpoint{2.843689in}{1.777063in}}{\pgfqpoint{2.849513in}{1.782887in}}%
\pgfpathcurveto{\pgfqpoint{2.855337in}{1.788711in}}{\pgfqpoint{2.858609in}{1.796611in}}{\pgfqpoint{2.858609in}{1.804847in}}%
\pgfpathcurveto{\pgfqpoint{2.858609in}{1.813083in}}{\pgfqpoint{2.855337in}{1.820983in}}{\pgfqpoint{2.849513in}{1.826807in}}%
\pgfpathcurveto{\pgfqpoint{2.843689in}{1.832631in}}{\pgfqpoint{2.835789in}{1.835904in}}{\pgfqpoint{2.827552in}{1.835904in}}%
\pgfpathcurveto{\pgfqpoint{2.819316in}{1.835904in}}{\pgfqpoint{2.811416in}{1.832631in}}{\pgfqpoint{2.805592in}{1.826807in}}%
\pgfpathcurveto{\pgfqpoint{2.799768in}{1.820983in}}{\pgfqpoint{2.796496in}{1.813083in}}{\pgfqpoint{2.796496in}{1.804847in}}%
\pgfpathcurveto{\pgfqpoint{2.796496in}{1.796611in}}{\pgfqpoint{2.799768in}{1.788711in}}{\pgfqpoint{2.805592in}{1.782887in}}%
\pgfpathcurveto{\pgfqpoint{2.811416in}{1.777063in}}{\pgfqpoint{2.819316in}{1.773791in}}{\pgfqpoint{2.827552in}{1.773791in}}%
\pgfpathclose%
\pgfusepath{stroke,fill}%
\end{pgfscope}%
\begin{pgfscope}%
\pgfpathrectangle{\pgfqpoint{0.100000in}{0.220728in}}{\pgfqpoint{3.696000in}{3.696000in}}%
\pgfusepath{clip}%
\pgfsetbuttcap%
\pgfsetroundjoin%
\definecolor{currentfill}{rgb}{0.121569,0.466667,0.705882}%
\pgfsetfillcolor{currentfill}%
\pgfsetfillopacity{0.869194}%
\pgfsetlinewidth{1.003750pt}%
\definecolor{currentstroke}{rgb}{0.121569,0.466667,0.705882}%
\pgfsetstrokecolor{currentstroke}%
\pgfsetstrokeopacity{0.869194}%
\pgfsetdash{}{0pt}%
\pgfpathmoveto{\pgfqpoint{2.824086in}{1.764254in}}%
\pgfpathcurveto{\pgfqpoint{2.832322in}{1.764254in}}{\pgfqpoint{2.840222in}{1.767527in}}{\pgfqpoint{2.846046in}{1.773351in}}%
\pgfpathcurveto{\pgfqpoint{2.851870in}{1.779175in}}{\pgfqpoint{2.855143in}{1.787075in}}{\pgfqpoint{2.855143in}{1.795311in}}%
\pgfpathcurveto{\pgfqpoint{2.855143in}{1.803547in}}{\pgfqpoint{2.851870in}{1.811447in}}{\pgfqpoint{2.846046in}{1.817271in}}%
\pgfpathcurveto{\pgfqpoint{2.840222in}{1.823095in}}{\pgfqpoint{2.832322in}{1.826367in}}{\pgfqpoint{2.824086in}{1.826367in}}%
\pgfpathcurveto{\pgfqpoint{2.815850in}{1.826367in}}{\pgfqpoint{2.807950in}{1.823095in}}{\pgfqpoint{2.802126in}{1.817271in}}%
\pgfpathcurveto{\pgfqpoint{2.796302in}{1.811447in}}{\pgfqpoint{2.793030in}{1.803547in}}{\pgfqpoint{2.793030in}{1.795311in}}%
\pgfpathcurveto{\pgfqpoint{2.793030in}{1.787075in}}{\pgfqpoint{2.796302in}{1.779175in}}{\pgfqpoint{2.802126in}{1.773351in}}%
\pgfpathcurveto{\pgfqpoint{2.807950in}{1.767527in}}{\pgfqpoint{2.815850in}{1.764254in}}{\pgfqpoint{2.824086in}{1.764254in}}%
\pgfpathclose%
\pgfusepath{stroke,fill}%
\end{pgfscope}%
\begin{pgfscope}%
\pgfpathrectangle{\pgfqpoint{0.100000in}{0.220728in}}{\pgfqpoint{3.696000in}{3.696000in}}%
\pgfusepath{clip}%
\pgfsetbuttcap%
\pgfsetroundjoin%
\definecolor{currentfill}{rgb}{0.121569,0.466667,0.705882}%
\pgfsetfillcolor{currentfill}%
\pgfsetfillopacity{0.870090}%
\pgfsetlinewidth{1.003750pt}%
\definecolor{currentstroke}{rgb}{0.121569,0.466667,0.705882}%
\pgfsetstrokecolor{currentstroke}%
\pgfsetstrokeopacity{0.870090}%
\pgfsetdash{}{0pt}%
\pgfpathmoveto{\pgfqpoint{1.695629in}{0.978773in}}%
\pgfpathcurveto{\pgfqpoint{1.703865in}{0.978773in}}{\pgfqpoint{1.711765in}{0.982045in}}{\pgfqpoint{1.717589in}{0.987869in}}%
\pgfpathcurveto{\pgfqpoint{1.723413in}{0.993693in}}{\pgfqpoint{1.726685in}{1.001593in}}{\pgfqpoint{1.726685in}{1.009829in}}%
\pgfpathcurveto{\pgfqpoint{1.726685in}{1.018065in}}{\pgfqpoint{1.723413in}{1.025965in}}{\pgfqpoint{1.717589in}{1.031789in}}%
\pgfpathcurveto{\pgfqpoint{1.711765in}{1.037613in}}{\pgfqpoint{1.703865in}{1.040886in}}{\pgfqpoint{1.695629in}{1.040886in}}%
\pgfpathcurveto{\pgfqpoint{1.687393in}{1.040886in}}{\pgfqpoint{1.679493in}{1.037613in}}{\pgfqpoint{1.673669in}{1.031789in}}%
\pgfpathcurveto{\pgfqpoint{1.667845in}{1.025965in}}{\pgfqpoint{1.664572in}{1.018065in}}{\pgfqpoint{1.664572in}{1.009829in}}%
\pgfpathcurveto{\pgfqpoint{1.664572in}{1.001593in}}{\pgfqpoint{1.667845in}{0.993693in}}{\pgfqpoint{1.673669in}{0.987869in}}%
\pgfpathcurveto{\pgfqpoint{1.679493in}{0.982045in}}{\pgfqpoint{1.687393in}{0.978773in}}{\pgfqpoint{1.695629in}{0.978773in}}%
\pgfpathclose%
\pgfusepath{stroke,fill}%
\end{pgfscope}%
\begin{pgfscope}%
\pgfpathrectangle{\pgfqpoint{0.100000in}{0.220728in}}{\pgfqpoint{3.696000in}{3.696000in}}%
\pgfusepath{clip}%
\pgfsetbuttcap%
\pgfsetroundjoin%
\definecolor{currentfill}{rgb}{0.121569,0.466667,0.705882}%
\pgfsetfillcolor{currentfill}%
\pgfsetfillopacity{0.870334}%
\pgfsetlinewidth{1.003750pt}%
\definecolor{currentstroke}{rgb}{0.121569,0.466667,0.705882}%
\pgfsetstrokecolor{currentstroke}%
\pgfsetstrokeopacity{0.870334}%
\pgfsetdash{}{0pt}%
\pgfpathmoveto{\pgfqpoint{2.818895in}{1.753733in}}%
\pgfpathcurveto{\pgfqpoint{2.827132in}{1.753733in}}{\pgfqpoint{2.835032in}{1.757006in}}{\pgfqpoint{2.840856in}{1.762830in}}%
\pgfpathcurveto{\pgfqpoint{2.846680in}{1.768653in}}{\pgfqpoint{2.849952in}{1.776553in}}{\pgfqpoint{2.849952in}{1.784790in}}%
\pgfpathcurveto{\pgfqpoint{2.849952in}{1.793026in}}{\pgfqpoint{2.846680in}{1.800926in}}{\pgfqpoint{2.840856in}{1.806750in}}%
\pgfpathcurveto{\pgfqpoint{2.835032in}{1.812574in}}{\pgfqpoint{2.827132in}{1.815846in}}{\pgfqpoint{2.818895in}{1.815846in}}%
\pgfpathcurveto{\pgfqpoint{2.810659in}{1.815846in}}{\pgfqpoint{2.802759in}{1.812574in}}{\pgfqpoint{2.796935in}{1.806750in}}%
\pgfpathcurveto{\pgfqpoint{2.791111in}{1.800926in}}{\pgfqpoint{2.787839in}{1.793026in}}{\pgfqpoint{2.787839in}{1.784790in}}%
\pgfpathcurveto{\pgfqpoint{2.787839in}{1.776553in}}{\pgfqpoint{2.791111in}{1.768653in}}{\pgfqpoint{2.796935in}{1.762830in}}%
\pgfpathcurveto{\pgfqpoint{2.802759in}{1.757006in}}{\pgfqpoint{2.810659in}{1.753733in}}{\pgfqpoint{2.818895in}{1.753733in}}%
\pgfpathclose%
\pgfusepath{stroke,fill}%
\end{pgfscope}%
\begin{pgfscope}%
\pgfpathrectangle{\pgfqpoint{0.100000in}{0.220728in}}{\pgfqpoint{3.696000in}{3.696000in}}%
\pgfusepath{clip}%
\pgfsetbuttcap%
\pgfsetroundjoin%
\definecolor{currentfill}{rgb}{0.121569,0.466667,0.705882}%
\pgfsetfillcolor{currentfill}%
\pgfsetfillopacity{0.870902}%
\pgfsetlinewidth{1.003750pt}%
\definecolor{currentstroke}{rgb}{0.121569,0.466667,0.705882}%
\pgfsetstrokecolor{currentstroke}%
\pgfsetstrokeopacity{0.870902}%
\pgfsetdash{}{0pt}%
\pgfpathmoveto{\pgfqpoint{2.815422in}{1.748806in}}%
\pgfpathcurveto{\pgfqpoint{2.823658in}{1.748806in}}{\pgfqpoint{2.831558in}{1.752079in}}{\pgfqpoint{2.837382in}{1.757903in}}%
\pgfpathcurveto{\pgfqpoint{2.843206in}{1.763727in}}{\pgfqpoint{2.846478in}{1.771627in}}{\pgfqpoint{2.846478in}{1.779863in}}%
\pgfpathcurveto{\pgfqpoint{2.846478in}{1.788099in}}{\pgfqpoint{2.843206in}{1.795999in}}{\pgfqpoint{2.837382in}{1.801823in}}%
\pgfpathcurveto{\pgfqpoint{2.831558in}{1.807647in}}{\pgfqpoint{2.823658in}{1.810919in}}{\pgfqpoint{2.815422in}{1.810919in}}%
\pgfpathcurveto{\pgfqpoint{2.807186in}{1.810919in}}{\pgfqpoint{2.799286in}{1.807647in}}{\pgfqpoint{2.793462in}{1.801823in}}%
\pgfpathcurveto{\pgfqpoint{2.787638in}{1.795999in}}{\pgfqpoint{2.784365in}{1.788099in}}{\pgfqpoint{2.784365in}{1.779863in}}%
\pgfpathcurveto{\pgfqpoint{2.784365in}{1.771627in}}{\pgfqpoint{2.787638in}{1.763727in}}{\pgfqpoint{2.793462in}{1.757903in}}%
\pgfpathcurveto{\pgfqpoint{2.799286in}{1.752079in}}{\pgfqpoint{2.807186in}{1.748806in}}{\pgfqpoint{2.815422in}{1.748806in}}%
\pgfpathclose%
\pgfusepath{stroke,fill}%
\end{pgfscope}%
\begin{pgfscope}%
\pgfpathrectangle{\pgfqpoint{0.100000in}{0.220728in}}{\pgfqpoint{3.696000in}{3.696000in}}%
\pgfusepath{clip}%
\pgfsetbuttcap%
\pgfsetroundjoin%
\definecolor{currentfill}{rgb}{0.121569,0.466667,0.705882}%
\pgfsetfillcolor{currentfill}%
\pgfsetfillopacity{0.872350}%
\pgfsetlinewidth{1.003750pt}%
\definecolor{currentstroke}{rgb}{0.121569,0.466667,0.705882}%
\pgfsetstrokecolor{currentstroke}%
\pgfsetstrokeopacity{0.872350}%
\pgfsetdash{}{0pt}%
\pgfpathmoveto{\pgfqpoint{2.812394in}{1.738709in}}%
\pgfpathcurveto{\pgfqpoint{2.820630in}{1.738709in}}{\pgfqpoint{2.828530in}{1.741981in}}{\pgfqpoint{2.834354in}{1.747805in}}%
\pgfpathcurveto{\pgfqpoint{2.840178in}{1.753629in}}{\pgfqpoint{2.843450in}{1.761529in}}{\pgfqpoint{2.843450in}{1.769765in}}%
\pgfpathcurveto{\pgfqpoint{2.843450in}{1.778001in}}{\pgfqpoint{2.840178in}{1.785901in}}{\pgfqpoint{2.834354in}{1.791725in}}%
\pgfpathcurveto{\pgfqpoint{2.828530in}{1.797549in}}{\pgfqpoint{2.820630in}{1.800822in}}{\pgfqpoint{2.812394in}{1.800822in}}%
\pgfpathcurveto{\pgfqpoint{2.804158in}{1.800822in}}{\pgfqpoint{2.796258in}{1.797549in}}{\pgfqpoint{2.790434in}{1.791725in}}%
\pgfpathcurveto{\pgfqpoint{2.784610in}{1.785901in}}{\pgfqpoint{2.781337in}{1.778001in}}{\pgfqpoint{2.781337in}{1.769765in}}%
\pgfpathcurveto{\pgfqpoint{2.781337in}{1.761529in}}{\pgfqpoint{2.784610in}{1.753629in}}{\pgfqpoint{2.790434in}{1.747805in}}%
\pgfpathcurveto{\pgfqpoint{2.796258in}{1.741981in}}{\pgfqpoint{2.804158in}{1.738709in}}{\pgfqpoint{2.812394in}{1.738709in}}%
\pgfpathclose%
\pgfusepath{stroke,fill}%
\end{pgfscope}%
\begin{pgfscope}%
\pgfpathrectangle{\pgfqpoint{0.100000in}{0.220728in}}{\pgfqpoint{3.696000in}{3.696000in}}%
\pgfusepath{clip}%
\pgfsetbuttcap%
\pgfsetroundjoin%
\definecolor{currentfill}{rgb}{0.121569,0.466667,0.705882}%
\pgfsetfillcolor{currentfill}%
\pgfsetfillopacity{0.873549}%
\pgfsetlinewidth{1.003750pt}%
\definecolor{currentstroke}{rgb}{0.121569,0.466667,0.705882}%
\pgfsetstrokecolor{currentstroke}%
\pgfsetstrokeopacity{0.873549}%
\pgfsetdash{}{0pt}%
\pgfpathmoveto{\pgfqpoint{2.807622in}{1.727826in}}%
\pgfpathcurveto{\pgfqpoint{2.815859in}{1.727826in}}{\pgfqpoint{2.823759in}{1.731098in}}{\pgfqpoint{2.829583in}{1.736922in}}%
\pgfpathcurveto{\pgfqpoint{2.835407in}{1.742746in}}{\pgfqpoint{2.838679in}{1.750646in}}{\pgfqpoint{2.838679in}{1.758882in}}%
\pgfpathcurveto{\pgfqpoint{2.838679in}{1.767118in}}{\pgfqpoint{2.835407in}{1.775019in}}{\pgfqpoint{2.829583in}{1.780842in}}%
\pgfpathcurveto{\pgfqpoint{2.823759in}{1.786666in}}{\pgfqpoint{2.815859in}{1.789939in}}{\pgfqpoint{2.807622in}{1.789939in}}%
\pgfpathcurveto{\pgfqpoint{2.799386in}{1.789939in}}{\pgfqpoint{2.791486in}{1.786666in}}{\pgfqpoint{2.785662in}{1.780842in}}%
\pgfpathcurveto{\pgfqpoint{2.779838in}{1.775019in}}{\pgfqpoint{2.776566in}{1.767118in}}{\pgfqpoint{2.776566in}{1.758882in}}%
\pgfpathcurveto{\pgfqpoint{2.776566in}{1.750646in}}{\pgfqpoint{2.779838in}{1.742746in}}{\pgfqpoint{2.785662in}{1.736922in}}%
\pgfpathcurveto{\pgfqpoint{2.791486in}{1.731098in}}{\pgfqpoint{2.799386in}{1.727826in}}{\pgfqpoint{2.807622in}{1.727826in}}%
\pgfpathclose%
\pgfusepath{stroke,fill}%
\end{pgfscope}%
\begin{pgfscope}%
\pgfpathrectangle{\pgfqpoint{0.100000in}{0.220728in}}{\pgfqpoint{3.696000in}{3.696000in}}%
\pgfusepath{clip}%
\pgfsetbuttcap%
\pgfsetroundjoin%
\definecolor{currentfill}{rgb}{0.121569,0.466667,0.705882}%
\pgfsetfillcolor{currentfill}%
\pgfsetfillopacity{0.874104}%
\pgfsetlinewidth{1.003750pt}%
\definecolor{currentstroke}{rgb}{0.121569,0.466667,0.705882}%
\pgfsetstrokecolor{currentstroke}%
\pgfsetstrokeopacity{0.874104}%
\pgfsetdash{}{0pt}%
\pgfpathmoveto{\pgfqpoint{1.713253in}{0.972407in}}%
\pgfpathcurveto{\pgfqpoint{1.721489in}{0.972407in}}{\pgfqpoint{1.729389in}{0.975679in}}{\pgfqpoint{1.735213in}{0.981503in}}%
\pgfpathcurveto{\pgfqpoint{1.741037in}{0.987327in}}{\pgfqpoint{1.744309in}{0.995227in}}{\pgfqpoint{1.744309in}{1.003463in}}%
\pgfpathcurveto{\pgfqpoint{1.744309in}{1.011700in}}{\pgfqpoint{1.741037in}{1.019600in}}{\pgfqpoint{1.735213in}{1.025424in}}%
\pgfpathcurveto{\pgfqpoint{1.729389in}{1.031248in}}{\pgfqpoint{1.721489in}{1.034520in}}{\pgfqpoint{1.713253in}{1.034520in}}%
\pgfpathcurveto{\pgfqpoint{1.705017in}{1.034520in}}{\pgfqpoint{1.697117in}{1.031248in}}{\pgfqpoint{1.691293in}{1.025424in}}%
\pgfpathcurveto{\pgfqpoint{1.685469in}{1.019600in}}{\pgfqpoint{1.682196in}{1.011700in}}{\pgfqpoint{1.682196in}{1.003463in}}%
\pgfpathcurveto{\pgfqpoint{1.682196in}{0.995227in}}{\pgfqpoint{1.685469in}{0.987327in}}{\pgfqpoint{1.691293in}{0.981503in}}%
\pgfpathcurveto{\pgfqpoint{1.697117in}{0.975679in}}{\pgfqpoint{1.705017in}{0.972407in}}{\pgfqpoint{1.713253in}{0.972407in}}%
\pgfpathclose%
\pgfusepath{stroke,fill}%
\end{pgfscope}%
\begin{pgfscope}%
\pgfpathrectangle{\pgfqpoint{0.100000in}{0.220728in}}{\pgfqpoint{3.696000in}{3.696000in}}%
\pgfusepath{clip}%
\pgfsetbuttcap%
\pgfsetroundjoin%
\definecolor{currentfill}{rgb}{0.121569,0.466667,0.705882}%
\pgfsetfillcolor{currentfill}%
\pgfsetfillopacity{0.874718}%
\pgfsetlinewidth{1.003750pt}%
\definecolor{currentstroke}{rgb}{0.121569,0.466667,0.705882}%
\pgfsetstrokecolor{currentstroke}%
\pgfsetstrokeopacity{0.874718}%
\pgfsetdash{}{0pt}%
\pgfpathmoveto{\pgfqpoint{2.801161in}{1.717939in}}%
\pgfpathcurveto{\pgfqpoint{2.809398in}{1.717939in}}{\pgfqpoint{2.817298in}{1.721211in}}{\pgfqpoint{2.823122in}{1.727035in}}%
\pgfpathcurveto{\pgfqpoint{2.828946in}{1.732859in}}{\pgfqpoint{2.832218in}{1.740759in}}{\pgfqpoint{2.832218in}{1.748995in}}%
\pgfpathcurveto{\pgfqpoint{2.832218in}{1.757232in}}{\pgfqpoint{2.828946in}{1.765132in}}{\pgfqpoint{2.823122in}{1.770956in}}%
\pgfpathcurveto{\pgfqpoint{2.817298in}{1.776780in}}{\pgfqpoint{2.809398in}{1.780052in}}{\pgfqpoint{2.801161in}{1.780052in}}%
\pgfpathcurveto{\pgfqpoint{2.792925in}{1.780052in}}{\pgfqpoint{2.785025in}{1.776780in}}{\pgfqpoint{2.779201in}{1.770956in}}%
\pgfpathcurveto{\pgfqpoint{2.773377in}{1.765132in}}{\pgfqpoint{2.770105in}{1.757232in}}{\pgfqpoint{2.770105in}{1.748995in}}%
\pgfpathcurveto{\pgfqpoint{2.770105in}{1.740759in}}{\pgfqpoint{2.773377in}{1.732859in}}{\pgfqpoint{2.779201in}{1.727035in}}%
\pgfpathcurveto{\pgfqpoint{2.785025in}{1.721211in}}{\pgfqpoint{2.792925in}{1.717939in}}{\pgfqpoint{2.801161in}{1.717939in}}%
\pgfpathclose%
\pgfusepath{stroke,fill}%
\end{pgfscope}%
\begin{pgfscope}%
\pgfpathrectangle{\pgfqpoint{0.100000in}{0.220728in}}{\pgfqpoint{3.696000in}{3.696000in}}%
\pgfusepath{clip}%
\pgfsetbuttcap%
\pgfsetroundjoin%
\definecolor{currentfill}{rgb}{0.121569,0.466667,0.705882}%
\pgfsetfillcolor{currentfill}%
\pgfsetfillopacity{0.876669}%
\pgfsetlinewidth{1.003750pt}%
\definecolor{currentstroke}{rgb}{0.121569,0.466667,0.705882}%
\pgfsetstrokecolor{currentstroke}%
\pgfsetstrokeopacity{0.876669}%
\pgfsetdash{}{0pt}%
\pgfpathmoveto{\pgfqpoint{2.796504in}{1.699183in}}%
\pgfpathcurveto{\pgfqpoint{2.804741in}{1.699183in}}{\pgfqpoint{2.812641in}{1.702456in}}{\pgfqpoint{2.818465in}{1.708279in}}%
\pgfpathcurveto{\pgfqpoint{2.824289in}{1.714103in}}{\pgfqpoint{2.827561in}{1.722003in}}{\pgfqpoint{2.827561in}{1.730240in}}%
\pgfpathcurveto{\pgfqpoint{2.827561in}{1.738476in}}{\pgfqpoint{2.824289in}{1.746376in}}{\pgfqpoint{2.818465in}{1.752200in}}%
\pgfpathcurveto{\pgfqpoint{2.812641in}{1.758024in}}{\pgfqpoint{2.804741in}{1.761296in}}{\pgfqpoint{2.796504in}{1.761296in}}%
\pgfpathcurveto{\pgfqpoint{2.788268in}{1.761296in}}{\pgfqpoint{2.780368in}{1.758024in}}{\pgfqpoint{2.774544in}{1.752200in}}%
\pgfpathcurveto{\pgfqpoint{2.768720in}{1.746376in}}{\pgfqpoint{2.765448in}{1.738476in}}{\pgfqpoint{2.765448in}{1.730240in}}%
\pgfpathcurveto{\pgfqpoint{2.765448in}{1.722003in}}{\pgfqpoint{2.768720in}{1.714103in}}{\pgfqpoint{2.774544in}{1.708279in}}%
\pgfpathcurveto{\pgfqpoint{2.780368in}{1.702456in}}{\pgfqpoint{2.788268in}{1.699183in}}{\pgfqpoint{2.796504in}{1.699183in}}%
\pgfpathclose%
\pgfusepath{stroke,fill}%
\end{pgfscope}%
\begin{pgfscope}%
\pgfpathrectangle{\pgfqpoint{0.100000in}{0.220728in}}{\pgfqpoint{3.696000in}{3.696000in}}%
\pgfusepath{clip}%
\pgfsetbuttcap%
\pgfsetroundjoin%
\definecolor{currentfill}{rgb}{0.121569,0.466667,0.705882}%
\pgfsetfillcolor{currentfill}%
\pgfsetfillopacity{0.877511}%
\pgfsetlinewidth{1.003750pt}%
\definecolor{currentstroke}{rgb}{0.121569,0.466667,0.705882}%
\pgfsetstrokecolor{currentstroke}%
\pgfsetstrokeopacity{0.877511}%
\pgfsetdash{}{0pt}%
\pgfpathmoveto{\pgfqpoint{2.791881in}{1.690152in}}%
\pgfpathcurveto{\pgfqpoint{2.800117in}{1.690152in}}{\pgfqpoint{2.808018in}{1.693425in}}{\pgfqpoint{2.813841in}{1.699249in}}%
\pgfpathcurveto{\pgfqpoint{2.819665in}{1.705073in}}{\pgfqpoint{2.822938in}{1.712973in}}{\pgfqpoint{2.822938in}{1.721209in}}%
\pgfpathcurveto{\pgfqpoint{2.822938in}{1.729445in}}{\pgfqpoint{2.819665in}{1.737345in}}{\pgfqpoint{2.813841in}{1.743169in}}%
\pgfpathcurveto{\pgfqpoint{2.808018in}{1.748993in}}{\pgfqpoint{2.800117in}{1.752265in}}{\pgfqpoint{2.791881in}{1.752265in}}%
\pgfpathcurveto{\pgfqpoint{2.783645in}{1.752265in}}{\pgfqpoint{2.775745in}{1.748993in}}{\pgfqpoint{2.769921in}{1.743169in}}%
\pgfpathcurveto{\pgfqpoint{2.764097in}{1.737345in}}{\pgfqpoint{2.760825in}{1.729445in}}{\pgfqpoint{2.760825in}{1.721209in}}%
\pgfpathcurveto{\pgfqpoint{2.760825in}{1.712973in}}{\pgfqpoint{2.764097in}{1.705073in}}{\pgfqpoint{2.769921in}{1.699249in}}%
\pgfpathcurveto{\pgfqpoint{2.775745in}{1.693425in}}{\pgfqpoint{2.783645in}{1.690152in}}{\pgfqpoint{2.791881in}{1.690152in}}%
\pgfpathclose%
\pgfusepath{stroke,fill}%
\end{pgfscope}%
\begin{pgfscope}%
\pgfpathrectangle{\pgfqpoint{0.100000in}{0.220728in}}{\pgfqpoint{3.696000in}{3.696000in}}%
\pgfusepath{clip}%
\pgfsetbuttcap%
\pgfsetroundjoin%
\definecolor{currentfill}{rgb}{0.121569,0.466667,0.705882}%
\pgfsetfillcolor{currentfill}%
\pgfsetfillopacity{0.877529}%
\pgfsetlinewidth{1.003750pt}%
\definecolor{currentstroke}{rgb}{0.121569,0.466667,0.705882}%
\pgfsetstrokecolor{currentstroke}%
\pgfsetstrokeopacity{0.877529}%
\pgfsetdash{}{0pt}%
\pgfpathmoveto{\pgfqpoint{1.729261in}{0.969333in}}%
\pgfpathcurveto{\pgfqpoint{1.737498in}{0.969333in}}{\pgfqpoint{1.745398in}{0.972605in}}{\pgfqpoint{1.751222in}{0.978429in}}%
\pgfpathcurveto{\pgfqpoint{1.757046in}{0.984253in}}{\pgfqpoint{1.760318in}{0.992153in}}{\pgfqpoint{1.760318in}{1.000390in}}%
\pgfpathcurveto{\pgfqpoint{1.760318in}{1.008626in}}{\pgfqpoint{1.757046in}{1.016526in}}{\pgfqpoint{1.751222in}{1.022350in}}%
\pgfpathcurveto{\pgfqpoint{1.745398in}{1.028174in}}{\pgfqpoint{1.737498in}{1.031446in}}{\pgfqpoint{1.729261in}{1.031446in}}%
\pgfpathcurveto{\pgfqpoint{1.721025in}{1.031446in}}{\pgfqpoint{1.713125in}{1.028174in}}{\pgfqpoint{1.707301in}{1.022350in}}%
\pgfpathcurveto{\pgfqpoint{1.701477in}{1.016526in}}{\pgfqpoint{1.698205in}{1.008626in}}{\pgfqpoint{1.698205in}{1.000390in}}%
\pgfpathcurveto{\pgfqpoint{1.698205in}{0.992153in}}{\pgfqpoint{1.701477in}{0.984253in}}{\pgfqpoint{1.707301in}{0.978429in}}%
\pgfpathcurveto{\pgfqpoint{1.713125in}{0.972605in}}{\pgfqpoint{1.721025in}{0.969333in}}{\pgfqpoint{1.729261in}{0.969333in}}%
\pgfpathclose%
\pgfusepath{stroke,fill}%
\end{pgfscope}%
\begin{pgfscope}%
\pgfpathrectangle{\pgfqpoint{0.100000in}{0.220728in}}{\pgfqpoint{3.696000in}{3.696000in}}%
\pgfusepath{clip}%
\pgfsetbuttcap%
\pgfsetroundjoin%
\definecolor{currentfill}{rgb}{0.121569,0.466667,0.705882}%
\pgfsetfillcolor{currentfill}%
\pgfsetfillopacity{0.878588}%
\pgfsetlinewidth{1.003750pt}%
\definecolor{currentstroke}{rgb}{0.121569,0.466667,0.705882}%
\pgfsetstrokecolor{currentstroke}%
\pgfsetstrokeopacity{0.878588}%
\pgfsetdash{}{0pt}%
\pgfpathmoveto{\pgfqpoint{2.786520in}{1.681730in}}%
\pgfpathcurveto{\pgfqpoint{2.794756in}{1.681730in}}{\pgfqpoint{2.802657in}{1.685002in}}{\pgfqpoint{2.808480in}{1.690826in}}%
\pgfpathcurveto{\pgfqpoint{2.814304in}{1.696650in}}{\pgfqpoint{2.817577in}{1.704550in}}{\pgfqpoint{2.817577in}{1.712786in}}%
\pgfpathcurveto{\pgfqpoint{2.817577in}{1.721023in}}{\pgfqpoint{2.814304in}{1.728923in}}{\pgfqpoint{2.808480in}{1.734747in}}%
\pgfpathcurveto{\pgfqpoint{2.802657in}{1.740571in}}{\pgfqpoint{2.794756in}{1.743843in}}{\pgfqpoint{2.786520in}{1.743843in}}%
\pgfpathcurveto{\pgfqpoint{2.778284in}{1.743843in}}{\pgfqpoint{2.770384in}{1.740571in}}{\pgfqpoint{2.764560in}{1.734747in}}%
\pgfpathcurveto{\pgfqpoint{2.758736in}{1.728923in}}{\pgfqpoint{2.755464in}{1.721023in}}{\pgfqpoint{2.755464in}{1.712786in}}%
\pgfpathcurveto{\pgfqpoint{2.755464in}{1.704550in}}{\pgfqpoint{2.758736in}{1.696650in}}{\pgfqpoint{2.764560in}{1.690826in}}%
\pgfpathcurveto{\pgfqpoint{2.770384in}{1.685002in}}{\pgfqpoint{2.778284in}{1.681730in}}{\pgfqpoint{2.786520in}{1.681730in}}%
\pgfpathclose%
\pgfusepath{stroke,fill}%
\end{pgfscope}%
\begin{pgfscope}%
\pgfpathrectangle{\pgfqpoint{0.100000in}{0.220728in}}{\pgfqpoint{3.696000in}{3.696000in}}%
\pgfusepath{clip}%
\pgfsetbuttcap%
\pgfsetroundjoin%
\definecolor{currentfill}{rgb}{0.121569,0.466667,0.705882}%
\pgfsetfillcolor{currentfill}%
\pgfsetfillopacity{0.879802}%
\pgfsetlinewidth{1.003750pt}%
\definecolor{currentstroke}{rgb}{0.121569,0.466667,0.705882}%
\pgfsetstrokecolor{currentstroke}%
\pgfsetstrokeopacity{0.879802}%
\pgfsetdash{}{0pt}%
\pgfpathmoveto{\pgfqpoint{1.742954in}{0.961737in}}%
\pgfpathcurveto{\pgfqpoint{1.751190in}{0.961737in}}{\pgfqpoint{1.759090in}{0.965009in}}{\pgfqpoint{1.764914in}{0.970833in}}%
\pgfpathcurveto{\pgfqpoint{1.770738in}{0.976657in}}{\pgfqpoint{1.774010in}{0.984557in}}{\pgfqpoint{1.774010in}{0.992793in}}%
\pgfpathcurveto{\pgfqpoint{1.774010in}{1.001030in}}{\pgfqpoint{1.770738in}{1.008930in}}{\pgfqpoint{1.764914in}{1.014754in}}%
\pgfpathcurveto{\pgfqpoint{1.759090in}{1.020578in}}{\pgfqpoint{1.751190in}{1.023850in}}{\pgfqpoint{1.742954in}{1.023850in}}%
\pgfpathcurveto{\pgfqpoint{1.734718in}{1.023850in}}{\pgfqpoint{1.726818in}{1.020578in}}{\pgfqpoint{1.720994in}{1.014754in}}%
\pgfpathcurveto{\pgfqpoint{1.715170in}{1.008930in}}{\pgfqpoint{1.711897in}{1.001030in}}{\pgfqpoint{1.711897in}{0.992793in}}%
\pgfpathcurveto{\pgfqpoint{1.711897in}{0.984557in}}{\pgfqpoint{1.715170in}{0.976657in}}{\pgfqpoint{1.720994in}{0.970833in}}%
\pgfpathcurveto{\pgfqpoint{1.726818in}{0.965009in}}{\pgfqpoint{1.734718in}{0.961737in}}{\pgfqpoint{1.742954in}{0.961737in}}%
\pgfpathclose%
\pgfusepath{stroke,fill}%
\end{pgfscope}%
\begin{pgfscope}%
\pgfpathrectangle{\pgfqpoint{0.100000in}{0.220728in}}{\pgfqpoint{3.696000in}{3.696000in}}%
\pgfusepath{clip}%
\pgfsetbuttcap%
\pgfsetroundjoin%
\definecolor{currentfill}{rgb}{0.121569,0.466667,0.705882}%
\pgfsetfillcolor{currentfill}%
\pgfsetfillopacity{0.880199}%
\pgfsetlinewidth{1.003750pt}%
\definecolor{currentstroke}{rgb}{0.121569,0.466667,0.705882}%
\pgfsetstrokecolor{currentstroke}%
\pgfsetstrokeopacity{0.880199}%
\pgfsetdash{}{0pt}%
\pgfpathmoveto{\pgfqpoint{2.782990in}{1.666986in}}%
\pgfpathcurveto{\pgfqpoint{2.791226in}{1.666986in}}{\pgfqpoint{2.799127in}{1.670259in}}{\pgfqpoint{2.804950in}{1.676083in}}%
\pgfpathcurveto{\pgfqpoint{2.810774in}{1.681906in}}{\pgfqpoint{2.814047in}{1.689807in}}{\pgfqpoint{2.814047in}{1.698043in}}%
\pgfpathcurveto{\pgfqpoint{2.814047in}{1.706279in}}{\pgfqpoint{2.810774in}{1.714179in}}{\pgfqpoint{2.804950in}{1.720003in}}%
\pgfpathcurveto{\pgfqpoint{2.799127in}{1.725827in}}{\pgfqpoint{2.791226in}{1.729099in}}{\pgfqpoint{2.782990in}{1.729099in}}%
\pgfpathcurveto{\pgfqpoint{2.774754in}{1.729099in}}{\pgfqpoint{2.766854in}{1.725827in}}{\pgfqpoint{2.761030in}{1.720003in}}%
\pgfpathcurveto{\pgfqpoint{2.755206in}{1.714179in}}{\pgfqpoint{2.751934in}{1.706279in}}{\pgfqpoint{2.751934in}{1.698043in}}%
\pgfpathcurveto{\pgfqpoint{2.751934in}{1.689807in}}{\pgfqpoint{2.755206in}{1.681906in}}{\pgfqpoint{2.761030in}{1.676083in}}%
\pgfpathcurveto{\pgfqpoint{2.766854in}{1.670259in}}{\pgfqpoint{2.774754in}{1.666986in}}{\pgfqpoint{2.782990in}{1.666986in}}%
\pgfpathclose%
\pgfusepath{stroke,fill}%
\end{pgfscope}%
\begin{pgfscope}%
\pgfpathrectangle{\pgfqpoint{0.100000in}{0.220728in}}{\pgfqpoint{3.696000in}{3.696000in}}%
\pgfusepath{clip}%
\pgfsetbuttcap%
\pgfsetroundjoin%
\definecolor{currentfill}{rgb}{0.121569,0.466667,0.705882}%
\pgfsetfillcolor{currentfill}%
\pgfsetfillopacity{0.881758}%
\pgfsetlinewidth{1.003750pt}%
\definecolor{currentstroke}{rgb}{0.121569,0.466667,0.705882}%
\pgfsetstrokecolor{currentstroke}%
\pgfsetstrokeopacity{0.881758}%
\pgfsetdash{}{0pt}%
\pgfpathmoveto{\pgfqpoint{2.776483in}{1.652917in}}%
\pgfpathcurveto{\pgfqpoint{2.784720in}{1.652917in}}{\pgfqpoint{2.792620in}{1.656190in}}{\pgfqpoint{2.798444in}{1.662014in}}%
\pgfpathcurveto{\pgfqpoint{2.804267in}{1.667838in}}{\pgfqpoint{2.807540in}{1.675738in}}{\pgfqpoint{2.807540in}{1.683974in}}%
\pgfpathcurveto{\pgfqpoint{2.807540in}{1.692210in}}{\pgfqpoint{2.804267in}{1.700110in}}{\pgfqpoint{2.798444in}{1.705934in}}%
\pgfpathcurveto{\pgfqpoint{2.792620in}{1.711758in}}{\pgfqpoint{2.784720in}{1.715030in}}{\pgfqpoint{2.776483in}{1.715030in}}%
\pgfpathcurveto{\pgfqpoint{2.768247in}{1.715030in}}{\pgfqpoint{2.760347in}{1.711758in}}{\pgfqpoint{2.754523in}{1.705934in}}%
\pgfpathcurveto{\pgfqpoint{2.748699in}{1.700110in}}{\pgfqpoint{2.745427in}{1.692210in}}{\pgfqpoint{2.745427in}{1.683974in}}%
\pgfpathcurveto{\pgfqpoint{2.745427in}{1.675738in}}{\pgfqpoint{2.748699in}{1.667838in}}{\pgfqpoint{2.754523in}{1.662014in}}%
\pgfpathcurveto{\pgfqpoint{2.760347in}{1.656190in}}{\pgfqpoint{2.768247in}{1.652917in}}{\pgfqpoint{2.776483in}{1.652917in}}%
\pgfpathclose%
\pgfusepath{stroke,fill}%
\end{pgfscope}%
\begin{pgfscope}%
\pgfpathrectangle{\pgfqpoint{0.100000in}{0.220728in}}{\pgfqpoint{3.696000in}{3.696000in}}%
\pgfusepath{clip}%
\pgfsetbuttcap%
\pgfsetroundjoin%
\definecolor{currentfill}{rgb}{0.121569,0.466667,0.705882}%
\pgfsetfillcolor{currentfill}%
\pgfsetfillopacity{0.882789}%
\pgfsetlinewidth{1.003750pt}%
\definecolor{currentstroke}{rgb}{0.121569,0.466667,0.705882}%
\pgfsetstrokecolor{currentstroke}%
\pgfsetstrokeopacity{0.882789}%
\pgfsetdash{}{0pt}%
\pgfpathmoveto{\pgfqpoint{2.772662in}{1.646232in}}%
\pgfpathcurveto{\pgfqpoint{2.780898in}{1.646232in}}{\pgfqpoint{2.788798in}{1.649504in}}{\pgfqpoint{2.794622in}{1.655328in}}%
\pgfpathcurveto{\pgfqpoint{2.800446in}{1.661152in}}{\pgfqpoint{2.803718in}{1.669052in}}{\pgfqpoint{2.803718in}{1.677288in}}%
\pgfpathcurveto{\pgfqpoint{2.803718in}{1.685525in}}{\pgfqpoint{2.800446in}{1.693425in}}{\pgfqpoint{2.794622in}{1.699249in}}%
\pgfpathcurveto{\pgfqpoint{2.788798in}{1.705072in}}{\pgfqpoint{2.780898in}{1.708345in}}{\pgfqpoint{2.772662in}{1.708345in}}%
\pgfpathcurveto{\pgfqpoint{2.764425in}{1.708345in}}{\pgfqpoint{2.756525in}{1.705072in}}{\pgfqpoint{2.750701in}{1.699249in}}%
\pgfpathcurveto{\pgfqpoint{2.744877in}{1.693425in}}{\pgfqpoint{2.741605in}{1.685525in}}{\pgfqpoint{2.741605in}{1.677288in}}%
\pgfpathcurveto{\pgfqpoint{2.741605in}{1.669052in}}{\pgfqpoint{2.744877in}{1.661152in}}{\pgfqpoint{2.750701in}{1.655328in}}%
\pgfpathcurveto{\pgfqpoint{2.756525in}{1.649504in}}{\pgfqpoint{2.764425in}{1.646232in}}{\pgfqpoint{2.772662in}{1.646232in}}%
\pgfpathclose%
\pgfusepath{stroke,fill}%
\end{pgfscope}%
\begin{pgfscope}%
\pgfpathrectangle{\pgfqpoint{0.100000in}{0.220728in}}{\pgfqpoint{3.696000in}{3.696000in}}%
\pgfusepath{clip}%
\pgfsetbuttcap%
\pgfsetroundjoin%
\definecolor{currentfill}{rgb}{0.121569,0.466667,0.705882}%
\pgfsetfillcolor{currentfill}%
\pgfsetfillopacity{0.883283}%
\pgfsetlinewidth{1.003750pt}%
\definecolor{currentstroke}{rgb}{0.121569,0.466667,0.705882}%
\pgfsetstrokecolor{currentstroke}%
\pgfsetstrokeopacity{0.883283}%
\pgfsetdash{}{0pt}%
\pgfpathmoveto{\pgfqpoint{2.771695in}{1.641186in}}%
\pgfpathcurveto{\pgfqpoint{2.779931in}{1.641186in}}{\pgfqpoint{2.787831in}{1.644458in}}{\pgfqpoint{2.793655in}{1.650282in}}%
\pgfpathcurveto{\pgfqpoint{2.799479in}{1.656106in}}{\pgfqpoint{2.802751in}{1.664006in}}{\pgfqpoint{2.802751in}{1.672242in}}%
\pgfpathcurveto{\pgfqpoint{2.802751in}{1.680479in}}{\pgfqpoint{2.799479in}{1.688379in}}{\pgfqpoint{2.793655in}{1.694203in}}%
\pgfpathcurveto{\pgfqpoint{2.787831in}{1.700026in}}{\pgfqpoint{2.779931in}{1.703299in}}{\pgfqpoint{2.771695in}{1.703299in}}%
\pgfpathcurveto{\pgfqpoint{2.763458in}{1.703299in}}{\pgfqpoint{2.755558in}{1.700026in}}{\pgfqpoint{2.749734in}{1.694203in}}%
\pgfpathcurveto{\pgfqpoint{2.743910in}{1.688379in}}{\pgfqpoint{2.740638in}{1.680479in}}{\pgfqpoint{2.740638in}{1.672242in}}%
\pgfpathcurveto{\pgfqpoint{2.740638in}{1.664006in}}{\pgfqpoint{2.743910in}{1.656106in}}{\pgfqpoint{2.749734in}{1.650282in}}%
\pgfpathcurveto{\pgfqpoint{2.755558in}{1.644458in}}{\pgfqpoint{2.763458in}{1.641186in}}{\pgfqpoint{2.771695in}{1.641186in}}%
\pgfpathclose%
\pgfusepath{stroke,fill}%
\end{pgfscope}%
\begin{pgfscope}%
\pgfpathrectangle{\pgfqpoint{0.100000in}{0.220728in}}{\pgfqpoint{3.696000in}{3.696000in}}%
\pgfusepath{clip}%
\pgfsetbuttcap%
\pgfsetroundjoin%
\definecolor{currentfill}{rgb}{0.121569,0.466667,0.705882}%
\pgfsetfillcolor{currentfill}%
\pgfsetfillopacity{0.884325}%
\pgfsetlinewidth{1.003750pt}%
\definecolor{currentstroke}{rgb}{0.121569,0.466667,0.705882}%
\pgfsetstrokecolor{currentstroke}%
\pgfsetstrokeopacity{0.884325}%
\pgfsetdash{}{0pt}%
\pgfpathmoveto{\pgfqpoint{2.767093in}{1.633269in}}%
\pgfpathcurveto{\pgfqpoint{2.775329in}{1.633269in}}{\pgfqpoint{2.783229in}{1.636541in}}{\pgfqpoint{2.789053in}{1.642365in}}%
\pgfpathcurveto{\pgfqpoint{2.794877in}{1.648189in}}{\pgfqpoint{2.798149in}{1.656089in}}{\pgfqpoint{2.798149in}{1.664326in}}%
\pgfpathcurveto{\pgfqpoint{2.798149in}{1.672562in}}{\pgfqpoint{2.794877in}{1.680462in}}{\pgfqpoint{2.789053in}{1.686286in}}%
\pgfpathcurveto{\pgfqpoint{2.783229in}{1.692110in}}{\pgfqpoint{2.775329in}{1.695382in}}{\pgfqpoint{2.767093in}{1.695382in}}%
\pgfpathcurveto{\pgfqpoint{2.758856in}{1.695382in}}{\pgfqpoint{2.750956in}{1.692110in}}{\pgfqpoint{2.745133in}{1.686286in}}%
\pgfpathcurveto{\pgfqpoint{2.739309in}{1.680462in}}{\pgfqpoint{2.736036in}{1.672562in}}{\pgfqpoint{2.736036in}{1.664326in}}%
\pgfpathcurveto{\pgfqpoint{2.736036in}{1.656089in}}{\pgfqpoint{2.739309in}{1.648189in}}{\pgfqpoint{2.745133in}{1.642365in}}%
\pgfpathcurveto{\pgfqpoint{2.750956in}{1.636541in}}{\pgfqpoint{2.758856in}{1.633269in}}{\pgfqpoint{2.767093in}{1.633269in}}%
\pgfpathclose%
\pgfusepath{stroke,fill}%
\end{pgfscope}%
\begin{pgfscope}%
\pgfpathrectangle{\pgfqpoint{0.100000in}{0.220728in}}{\pgfqpoint{3.696000in}{3.696000in}}%
\pgfusepath{clip}%
\pgfsetbuttcap%
\pgfsetroundjoin%
\definecolor{currentfill}{rgb}{0.121569,0.466667,0.705882}%
\pgfsetfillcolor{currentfill}%
\pgfsetfillopacity{0.884383}%
\pgfsetlinewidth{1.003750pt}%
\definecolor{currentstroke}{rgb}{0.121569,0.466667,0.705882}%
\pgfsetstrokecolor{currentstroke}%
\pgfsetstrokeopacity{0.884383}%
\pgfsetdash{}{0pt}%
\pgfpathmoveto{\pgfqpoint{1.767535in}{0.948566in}}%
\pgfpathcurveto{\pgfqpoint{1.775771in}{0.948566in}}{\pgfqpoint{1.783671in}{0.951838in}}{\pgfqpoint{1.789495in}{0.957662in}}%
\pgfpathcurveto{\pgfqpoint{1.795319in}{0.963486in}}{\pgfqpoint{1.798591in}{0.971386in}}{\pgfqpoint{1.798591in}{0.979622in}}%
\pgfpathcurveto{\pgfqpoint{1.798591in}{0.987859in}}{\pgfqpoint{1.795319in}{0.995759in}}{\pgfqpoint{1.789495in}{1.001583in}}%
\pgfpathcurveto{\pgfqpoint{1.783671in}{1.007407in}}{\pgfqpoint{1.775771in}{1.010679in}}{\pgfqpoint{1.767535in}{1.010679in}}%
\pgfpathcurveto{\pgfqpoint{1.759298in}{1.010679in}}{\pgfqpoint{1.751398in}{1.007407in}}{\pgfqpoint{1.745574in}{1.001583in}}%
\pgfpathcurveto{\pgfqpoint{1.739750in}{0.995759in}}{\pgfqpoint{1.736478in}{0.987859in}}{\pgfqpoint{1.736478in}{0.979622in}}%
\pgfpathcurveto{\pgfqpoint{1.736478in}{0.971386in}}{\pgfqpoint{1.739750in}{0.963486in}}{\pgfqpoint{1.745574in}{0.957662in}}%
\pgfpathcurveto{\pgfqpoint{1.751398in}{0.951838in}}{\pgfqpoint{1.759298in}{0.948566in}}{\pgfqpoint{1.767535in}{0.948566in}}%
\pgfpathclose%
\pgfusepath{stroke,fill}%
\end{pgfscope}%
\begin{pgfscope}%
\pgfpathrectangle{\pgfqpoint{0.100000in}{0.220728in}}{\pgfqpoint{3.696000in}{3.696000in}}%
\pgfusepath{clip}%
\pgfsetbuttcap%
\pgfsetroundjoin%
\definecolor{currentfill}{rgb}{0.121569,0.466667,0.705882}%
\pgfsetfillcolor{currentfill}%
\pgfsetfillopacity{0.885394}%
\pgfsetlinewidth{1.003750pt}%
\definecolor{currentstroke}{rgb}{0.121569,0.466667,0.705882}%
\pgfsetstrokecolor{currentstroke}%
\pgfsetstrokeopacity{0.885394}%
\pgfsetdash{}{0pt}%
\pgfpathmoveto{\pgfqpoint{2.763249in}{1.622251in}}%
\pgfpathcurveto{\pgfqpoint{2.771485in}{1.622251in}}{\pgfqpoint{2.779385in}{1.625523in}}{\pgfqpoint{2.785209in}{1.631347in}}%
\pgfpathcurveto{\pgfqpoint{2.791033in}{1.637171in}}{\pgfqpoint{2.794306in}{1.645071in}}{\pgfqpoint{2.794306in}{1.653307in}}%
\pgfpathcurveto{\pgfqpoint{2.794306in}{1.661544in}}{\pgfqpoint{2.791033in}{1.669444in}}{\pgfqpoint{2.785209in}{1.675268in}}%
\pgfpathcurveto{\pgfqpoint{2.779385in}{1.681092in}}{\pgfqpoint{2.771485in}{1.684364in}}{\pgfqpoint{2.763249in}{1.684364in}}%
\pgfpathcurveto{\pgfqpoint{2.755013in}{1.684364in}}{\pgfqpoint{2.747113in}{1.681092in}}{\pgfqpoint{2.741289in}{1.675268in}}%
\pgfpathcurveto{\pgfqpoint{2.735465in}{1.669444in}}{\pgfqpoint{2.732193in}{1.661544in}}{\pgfqpoint{2.732193in}{1.653307in}}%
\pgfpathcurveto{\pgfqpoint{2.732193in}{1.645071in}}{\pgfqpoint{2.735465in}{1.637171in}}{\pgfqpoint{2.741289in}{1.631347in}}%
\pgfpathcurveto{\pgfqpoint{2.747113in}{1.625523in}}{\pgfqpoint{2.755013in}{1.622251in}}{\pgfqpoint{2.763249in}{1.622251in}}%
\pgfpathclose%
\pgfusepath{stroke,fill}%
\end{pgfscope}%
\begin{pgfscope}%
\pgfpathrectangle{\pgfqpoint{0.100000in}{0.220728in}}{\pgfqpoint{3.696000in}{3.696000in}}%
\pgfusepath{clip}%
\pgfsetbuttcap%
\pgfsetroundjoin%
\definecolor{currentfill}{rgb}{0.121569,0.466667,0.705882}%
\pgfsetfillcolor{currentfill}%
\pgfsetfillopacity{0.886239}%
\pgfsetlinewidth{1.003750pt}%
\definecolor{currentstroke}{rgb}{0.121569,0.466667,0.705882}%
\pgfsetstrokecolor{currentstroke}%
\pgfsetstrokeopacity{0.886239}%
\pgfsetdash{}{0pt}%
\pgfpathmoveto{\pgfqpoint{2.761620in}{1.616807in}}%
\pgfpathcurveto{\pgfqpoint{2.769856in}{1.616807in}}{\pgfqpoint{2.777756in}{1.620079in}}{\pgfqpoint{2.783580in}{1.625903in}}%
\pgfpathcurveto{\pgfqpoint{2.789404in}{1.631727in}}{\pgfqpoint{2.792676in}{1.639627in}}{\pgfqpoint{2.792676in}{1.647863in}}%
\pgfpathcurveto{\pgfqpoint{2.792676in}{1.656100in}}{\pgfqpoint{2.789404in}{1.664000in}}{\pgfqpoint{2.783580in}{1.669824in}}%
\pgfpathcurveto{\pgfqpoint{2.777756in}{1.675648in}}{\pgfqpoint{2.769856in}{1.678920in}}{\pgfqpoint{2.761620in}{1.678920in}}%
\pgfpathcurveto{\pgfqpoint{2.753383in}{1.678920in}}{\pgfqpoint{2.745483in}{1.675648in}}{\pgfqpoint{2.739659in}{1.669824in}}%
\pgfpathcurveto{\pgfqpoint{2.733835in}{1.664000in}}{\pgfqpoint{2.730563in}{1.656100in}}{\pgfqpoint{2.730563in}{1.647863in}}%
\pgfpathcurveto{\pgfqpoint{2.730563in}{1.639627in}}{\pgfqpoint{2.733835in}{1.631727in}}{\pgfqpoint{2.739659in}{1.625903in}}%
\pgfpathcurveto{\pgfqpoint{2.745483in}{1.620079in}}{\pgfqpoint{2.753383in}{1.616807in}}{\pgfqpoint{2.761620in}{1.616807in}}%
\pgfpathclose%
\pgfusepath{stroke,fill}%
\end{pgfscope}%
\begin{pgfscope}%
\pgfpathrectangle{\pgfqpoint{0.100000in}{0.220728in}}{\pgfqpoint{3.696000in}{3.696000in}}%
\pgfusepath{clip}%
\pgfsetbuttcap%
\pgfsetroundjoin%
\definecolor{currentfill}{rgb}{0.121569,0.466667,0.705882}%
\pgfsetfillcolor{currentfill}%
\pgfsetfillopacity{0.887044}%
\pgfsetlinewidth{1.003750pt}%
\definecolor{currentstroke}{rgb}{0.121569,0.466667,0.705882}%
\pgfsetstrokecolor{currentstroke}%
\pgfsetstrokeopacity{0.887044}%
\pgfsetdash{}{0pt}%
\pgfpathmoveto{\pgfqpoint{2.757482in}{1.609556in}}%
\pgfpathcurveto{\pgfqpoint{2.765718in}{1.609556in}}{\pgfqpoint{2.773618in}{1.612828in}}{\pgfqpoint{2.779442in}{1.618652in}}%
\pgfpathcurveto{\pgfqpoint{2.785266in}{1.624476in}}{\pgfqpoint{2.788538in}{1.632376in}}{\pgfqpoint{2.788538in}{1.640612in}}%
\pgfpathcurveto{\pgfqpoint{2.788538in}{1.648849in}}{\pgfqpoint{2.785266in}{1.656749in}}{\pgfqpoint{2.779442in}{1.662573in}}%
\pgfpathcurveto{\pgfqpoint{2.773618in}{1.668396in}}{\pgfqpoint{2.765718in}{1.671669in}}{\pgfqpoint{2.757482in}{1.671669in}}%
\pgfpathcurveto{\pgfqpoint{2.749245in}{1.671669in}}{\pgfqpoint{2.741345in}{1.668396in}}{\pgfqpoint{2.735521in}{1.662573in}}%
\pgfpathcurveto{\pgfqpoint{2.729698in}{1.656749in}}{\pgfqpoint{2.726425in}{1.648849in}}{\pgfqpoint{2.726425in}{1.640612in}}%
\pgfpathcurveto{\pgfqpoint{2.726425in}{1.632376in}}{\pgfqpoint{2.729698in}{1.624476in}}{\pgfqpoint{2.735521in}{1.618652in}}%
\pgfpathcurveto{\pgfqpoint{2.741345in}{1.612828in}}{\pgfqpoint{2.749245in}{1.609556in}}{\pgfqpoint{2.757482in}{1.609556in}}%
\pgfpathclose%
\pgfusepath{stroke,fill}%
\end{pgfscope}%
\begin{pgfscope}%
\pgfpathrectangle{\pgfqpoint{0.100000in}{0.220728in}}{\pgfqpoint{3.696000in}{3.696000in}}%
\pgfusepath{clip}%
\pgfsetbuttcap%
\pgfsetroundjoin%
\definecolor{currentfill}{rgb}{0.121569,0.466667,0.705882}%
\pgfsetfillcolor{currentfill}%
\pgfsetfillopacity{0.888661}%
\pgfsetlinewidth{1.003750pt}%
\definecolor{currentstroke}{rgb}{0.121569,0.466667,0.705882}%
\pgfsetstrokecolor{currentstroke}%
\pgfsetstrokeopacity{0.888661}%
\pgfsetdash{}{0pt}%
\pgfpathmoveto{\pgfqpoint{2.754168in}{1.598454in}}%
\pgfpathcurveto{\pgfqpoint{2.762405in}{1.598454in}}{\pgfqpoint{2.770305in}{1.601726in}}{\pgfqpoint{2.776129in}{1.607550in}}%
\pgfpathcurveto{\pgfqpoint{2.781953in}{1.613374in}}{\pgfqpoint{2.785225in}{1.621274in}}{\pgfqpoint{2.785225in}{1.629510in}}%
\pgfpathcurveto{\pgfqpoint{2.785225in}{1.637746in}}{\pgfqpoint{2.781953in}{1.645646in}}{\pgfqpoint{2.776129in}{1.651470in}}%
\pgfpathcurveto{\pgfqpoint{2.770305in}{1.657294in}}{\pgfqpoint{2.762405in}{1.660567in}}{\pgfqpoint{2.754168in}{1.660567in}}%
\pgfpathcurveto{\pgfqpoint{2.745932in}{1.660567in}}{\pgfqpoint{2.738032in}{1.657294in}}{\pgfqpoint{2.732208in}{1.651470in}}%
\pgfpathcurveto{\pgfqpoint{2.726384in}{1.645646in}}{\pgfqpoint{2.723112in}{1.637746in}}{\pgfqpoint{2.723112in}{1.629510in}}%
\pgfpathcurveto{\pgfqpoint{2.723112in}{1.621274in}}{\pgfqpoint{2.726384in}{1.613374in}}{\pgfqpoint{2.732208in}{1.607550in}}%
\pgfpathcurveto{\pgfqpoint{2.738032in}{1.601726in}}{\pgfqpoint{2.745932in}{1.598454in}}{\pgfqpoint{2.754168in}{1.598454in}}%
\pgfpathclose%
\pgfusepath{stroke,fill}%
\end{pgfscope}%
\begin{pgfscope}%
\pgfpathrectangle{\pgfqpoint{0.100000in}{0.220728in}}{\pgfqpoint{3.696000in}{3.696000in}}%
\pgfusepath{clip}%
\pgfsetbuttcap%
\pgfsetroundjoin%
\definecolor{currentfill}{rgb}{0.121569,0.466667,0.705882}%
\pgfsetfillcolor{currentfill}%
\pgfsetfillopacity{0.888822}%
\pgfsetlinewidth{1.003750pt}%
\definecolor{currentstroke}{rgb}{0.121569,0.466667,0.705882}%
\pgfsetstrokecolor{currentstroke}%
\pgfsetstrokeopacity{0.888822}%
\pgfsetdash{}{0pt}%
\pgfpathmoveto{\pgfqpoint{1.790688in}{0.939695in}}%
\pgfpathcurveto{\pgfqpoint{1.798924in}{0.939695in}}{\pgfqpoint{1.806824in}{0.942967in}}{\pgfqpoint{1.812648in}{0.948791in}}%
\pgfpathcurveto{\pgfqpoint{1.818472in}{0.954615in}}{\pgfqpoint{1.821744in}{0.962515in}}{\pgfqpoint{1.821744in}{0.970751in}}%
\pgfpathcurveto{\pgfqpoint{1.821744in}{0.978987in}}{\pgfqpoint{1.818472in}{0.986887in}}{\pgfqpoint{1.812648in}{0.992711in}}%
\pgfpathcurveto{\pgfqpoint{1.806824in}{0.998535in}}{\pgfqpoint{1.798924in}{1.001808in}}{\pgfqpoint{1.790688in}{1.001808in}}%
\pgfpathcurveto{\pgfqpoint{1.782452in}{1.001808in}}{\pgfqpoint{1.774552in}{0.998535in}}{\pgfqpoint{1.768728in}{0.992711in}}%
\pgfpathcurveto{\pgfqpoint{1.762904in}{0.986887in}}{\pgfqpoint{1.759631in}{0.978987in}}{\pgfqpoint{1.759631in}{0.970751in}}%
\pgfpathcurveto{\pgfqpoint{1.759631in}{0.962515in}}{\pgfqpoint{1.762904in}{0.954615in}}{\pgfqpoint{1.768728in}{0.948791in}}%
\pgfpathcurveto{\pgfqpoint{1.774552in}{0.942967in}}{\pgfqpoint{1.782452in}{0.939695in}}{\pgfqpoint{1.790688in}{0.939695in}}%
\pgfpathclose%
\pgfusepath{stroke,fill}%
\end{pgfscope}%
\begin{pgfscope}%
\pgfpathrectangle{\pgfqpoint{0.100000in}{0.220728in}}{\pgfqpoint{3.696000in}{3.696000in}}%
\pgfusepath{clip}%
\pgfsetbuttcap%
\pgfsetroundjoin%
\definecolor{currentfill}{rgb}{0.121569,0.466667,0.705882}%
\pgfsetfillcolor{currentfill}%
\pgfsetfillopacity{0.889517}%
\pgfsetlinewidth{1.003750pt}%
\definecolor{currentstroke}{rgb}{0.121569,0.466667,0.705882}%
\pgfsetstrokecolor{currentstroke}%
\pgfsetstrokeopacity{0.889517}%
\pgfsetdash{}{0pt}%
\pgfpathmoveto{\pgfqpoint{2.751882in}{1.592610in}}%
\pgfpathcurveto{\pgfqpoint{2.760118in}{1.592610in}}{\pgfqpoint{2.768018in}{1.595882in}}{\pgfqpoint{2.773842in}{1.601706in}}%
\pgfpathcurveto{\pgfqpoint{2.779666in}{1.607530in}}{\pgfqpoint{2.782938in}{1.615430in}}{\pgfqpoint{2.782938in}{1.623666in}}%
\pgfpathcurveto{\pgfqpoint{2.782938in}{1.631902in}}{\pgfqpoint{2.779666in}{1.639803in}}{\pgfqpoint{2.773842in}{1.645626in}}%
\pgfpathcurveto{\pgfqpoint{2.768018in}{1.651450in}}{\pgfqpoint{2.760118in}{1.654723in}}{\pgfqpoint{2.751882in}{1.654723in}}%
\pgfpathcurveto{\pgfqpoint{2.743645in}{1.654723in}}{\pgfqpoint{2.735745in}{1.651450in}}{\pgfqpoint{2.729921in}{1.645626in}}%
\pgfpathcurveto{\pgfqpoint{2.724098in}{1.639803in}}{\pgfqpoint{2.720825in}{1.631902in}}{\pgfqpoint{2.720825in}{1.623666in}}%
\pgfpathcurveto{\pgfqpoint{2.720825in}{1.615430in}}{\pgfqpoint{2.724098in}{1.607530in}}{\pgfqpoint{2.729921in}{1.601706in}}%
\pgfpathcurveto{\pgfqpoint{2.735745in}{1.595882in}}{\pgfqpoint{2.743645in}{1.592610in}}{\pgfqpoint{2.751882in}{1.592610in}}%
\pgfpathclose%
\pgfusepath{stroke,fill}%
\end{pgfscope}%
\begin{pgfscope}%
\pgfpathrectangle{\pgfqpoint{0.100000in}{0.220728in}}{\pgfqpoint{3.696000in}{3.696000in}}%
\pgfusepath{clip}%
\pgfsetbuttcap%
\pgfsetroundjoin%
\definecolor{currentfill}{rgb}{0.121569,0.466667,0.705882}%
\pgfsetfillcolor{currentfill}%
\pgfsetfillopacity{0.890391}%
\pgfsetlinewidth{1.003750pt}%
\definecolor{currentstroke}{rgb}{0.121569,0.466667,0.705882}%
\pgfsetstrokecolor{currentstroke}%
\pgfsetstrokeopacity{0.890391}%
\pgfsetdash{}{0pt}%
\pgfpathmoveto{\pgfqpoint{2.748101in}{1.587546in}}%
\pgfpathcurveto{\pgfqpoint{2.756337in}{1.587546in}}{\pgfqpoint{2.764237in}{1.590818in}}{\pgfqpoint{2.770061in}{1.596642in}}%
\pgfpathcurveto{\pgfqpoint{2.775885in}{1.602466in}}{\pgfqpoint{2.779158in}{1.610366in}}{\pgfqpoint{2.779158in}{1.618602in}}%
\pgfpathcurveto{\pgfqpoint{2.779158in}{1.626838in}}{\pgfqpoint{2.775885in}{1.634738in}}{\pgfqpoint{2.770061in}{1.640562in}}%
\pgfpathcurveto{\pgfqpoint{2.764237in}{1.646386in}}{\pgfqpoint{2.756337in}{1.649659in}}{\pgfqpoint{2.748101in}{1.649659in}}%
\pgfpathcurveto{\pgfqpoint{2.739865in}{1.649659in}}{\pgfqpoint{2.731965in}{1.646386in}}{\pgfqpoint{2.726141in}{1.640562in}}%
\pgfpathcurveto{\pgfqpoint{2.720317in}{1.634738in}}{\pgfqpoint{2.717045in}{1.626838in}}{\pgfqpoint{2.717045in}{1.618602in}}%
\pgfpathcurveto{\pgfqpoint{2.717045in}{1.610366in}}{\pgfqpoint{2.720317in}{1.602466in}}{\pgfqpoint{2.726141in}{1.596642in}}%
\pgfpathcurveto{\pgfqpoint{2.731965in}{1.590818in}}{\pgfqpoint{2.739865in}{1.587546in}}{\pgfqpoint{2.748101in}{1.587546in}}%
\pgfpathclose%
\pgfusepath{stroke,fill}%
\end{pgfscope}%
\begin{pgfscope}%
\pgfpathrectangle{\pgfqpoint{0.100000in}{0.220728in}}{\pgfqpoint{3.696000in}{3.696000in}}%
\pgfusepath{clip}%
\pgfsetbuttcap%
\pgfsetroundjoin%
\definecolor{currentfill}{rgb}{0.121569,0.466667,0.705882}%
\pgfsetfillcolor{currentfill}%
\pgfsetfillopacity{0.891476}%
\pgfsetlinewidth{1.003750pt}%
\definecolor{currentstroke}{rgb}{0.121569,0.466667,0.705882}%
\pgfsetstrokecolor{currentstroke}%
\pgfsetstrokeopacity{0.891476}%
\pgfsetdash{}{0pt}%
\pgfpathmoveto{\pgfqpoint{1.809624in}{0.927401in}}%
\pgfpathcurveto{\pgfqpoint{1.817860in}{0.927401in}}{\pgfqpoint{1.825760in}{0.930673in}}{\pgfqpoint{1.831584in}{0.936497in}}%
\pgfpathcurveto{\pgfqpoint{1.837408in}{0.942321in}}{\pgfqpoint{1.840680in}{0.950221in}}{\pgfqpoint{1.840680in}{0.958458in}}%
\pgfpathcurveto{\pgfqpoint{1.840680in}{0.966694in}}{\pgfqpoint{1.837408in}{0.974594in}}{\pgfqpoint{1.831584in}{0.980418in}}%
\pgfpathcurveto{\pgfqpoint{1.825760in}{0.986242in}}{\pgfqpoint{1.817860in}{0.989514in}}{\pgfqpoint{1.809624in}{0.989514in}}%
\pgfpathcurveto{\pgfqpoint{1.801388in}{0.989514in}}{\pgfqpoint{1.793488in}{0.986242in}}{\pgfqpoint{1.787664in}{0.980418in}}%
\pgfpathcurveto{\pgfqpoint{1.781840in}{0.974594in}}{\pgfqpoint{1.778567in}{0.966694in}}{\pgfqpoint{1.778567in}{0.958458in}}%
\pgfpathcurveto{\pgfqpoint{1.778567in}{0.950221in}}{\pgfqpoint{1.781840in}{0.942321in}}{\pgfqpoint{1.787664in}{0.936497in}}%
\pgfpathcurveto{\pgfqpoint{1.793488in}{0.930673in}}{\pgfqpoint{1.801388in}{0.927401in}}{\pgfqpoint{1.809624in}{0.927401in}}%
\pgfpathclose%
\pgfusepath{stroke,fill}%
\end{pgfscope}%
\begin{pgfscope}%
\pgfpathrectangle{\pgfqpoint{0.100000in}{0.220728in}}{\pgfqpoint{3.696000in}{3.696000in}}%
\pgfusepath{clip}%
\pgfsetbuttcap%
\pgfsetroundjoin%
\definecolor{currentfill}{rgb}{0.121569,0.466667,0.705882}%
\pgfsetfillcolor{currentfill}%
\pgfsetfillopacity{0.891776}%
\pgfsetlinewidth{1.003750pt}%
\definecolor{currentstroke}{rgb}{0.121569,0.466667,0.705882}%
\pgfsetstrokecolor{currentstroke}%
\pgfsetstrokeopacity{0.891776}%
\pgfsetdash{}{0pt}%
\pgfpathmoveto{\pgfqpoint{2.745246in}{1.573839in}}%
\pgfpathcurveto{\pgfqpoint{2.753482in}{1.573839in}}{\pgfqpoint{2.761382in}{1.577111in}}{\pgfqpoint{2.767206in}{1.582935in}}%
\pgfpathcurveto{\pgfqpoint{2.773030in}{1.588759in}}{\pgfqpoint{2.776302in}{1.596659in}}{\pgfqpoint{2.776302in}{1.604895in}}%
\pgfpathcurveto{\pgfqpoint{2.776302in}{1.613131in}}{\pgfqpoint{2.773030in}{1.621031in}}{\pgfqpoint{2.767206in}{1.626855in}}%
\pgfpathcurveto{\pgfqpoint{2.761382in}{1.632679in}}{\pgfqpoint{2.753482in}{1.635952in}}{\pgfqpoint{2.745246in}{1.635952in}}%
\pgfpathcurveto{\pgfqpoint{2.737009in}{1.635952in}}{\pgfqpoint{2.729109in}{1.632679in}}{\pgfqpoint{2.723285in}{1.626855in}}%
\pgfpathcurveto{\pgfqpoint{2.717461in}{1.621031in}}{\pgfqpoint{2.714189in}{1.613131in}}{\pgfqpoint{2.714189in}{1.604895in}}%
\pgfpathcurveto{\pgfqpoint{2.714189in}{1.596659in}}{\pgfqpoint{2.717461in}{1.588759in}}{\pgfqpoint{2.723285in}{1.582935in}}%
\pgfpathcurveto{\pgfqpoint{2.729109in}{1.577111in}}{\pgfqpoint{2.737009in}{1.573839in}}{\pgfqpoint{2.745246in}{1.573839in}}%
\pgfpathclose%
\pgfusepath{stroke,fill}%
\end{pgfscope}%
\begin{pgfscope}%
\pgfpathrectangle{\pgfqpoint{0.100000in}{0.220728in}}{\pgfqpoint{3.696000in}{3.696000in}}%
\pgfusepath{clip}%
\pgfsetbuttcap%
\pgfsetroundjoin%
\definecolor{currentfill}{rgb}{0.121569,0.466667,0.705882}%
\pgfsetfillcolor{currentfill}%
\pgfsetfillopacity{0.893347}%
\pgfsetlinewidth{1.003750pt}%
\definecolor{currentstroke}{rgb}{0.121569,0.466667,0.705882}%
\pgfsetstrokecolor{currentstroke}%
\pgfsetstrokeopacity{0.893347}%
\pgfsetdash{}{0pt}%
\pgfpathmoveto{\pgfqpoint{2.738916in}{1.562823in}}%
\pgfpathcurveto{\pgfqpoint{2.747153in}{1.562823in}}{\pgfqpoint{2.755053in}{1.566096in}}{\pgfqpoint{2.760877in}{1.571920in}}%
\pgfpathcurveto{\pgfqpoint{2.766700in}{1.577744in}}{\pgfqpoint{2.769973in}{1.585644in}}{\pgfqpoint{2.769973in}{1.593880in}}%
\pgfpathcurveto{\pgfqpoint{2.769973in}{1.602116in}}{\pgfqpoint{2.766700in}{1.610016in}}{\pgfqpoint{2.760877in}{1.615840in}}%
\pgfpathcurveto{\pgfqpoint{2.755053in}{1.621664in}}{\pgfqpoint{2.747153in}{1.624936in}}{\pgfqpoint{2.738916in}{1.624936in}}%
\pgfpathcurveto{\pgfqpoint{2.730680in}{1.624936in}}{\pgfqpoint{2.722780in}{1.621664in}}{\pgfqpoint{2.716956in}{1.615840in}}%
\pgfpathcurveto{\pgfqpoint{2.711132in}{1.610016in}}{\pgfqpoint{2.707860in}{1.602116in}}{\pgfqpoint{2.707860in}{1.593880in}}%
\pgfpathcurveto{\pgfqpoint{2.707860in}{1.585644in}}{\pgfqpoint{2.711132in}{1.577744in}}{\pgfqpoint{2.716956in}{1.571920in}}%
\pgfpathcurveto{\pgfqpoint{2.722780in}{1.566096in}}{\pgfqpoint{2.730680in}{1.562823in}}{\pgfqpoint{2.738916in}{1.562823in}}%
\pgfpathclose%
\pgfusepath{stroke,fill}%
\end{pgfscope}%
\begin{pgfscope}%
\pgfpathrectangle{\pgfqpoint{0.100000in}{0.220728in}}{\pgfqpoint{3.696000in}{3.696000in}}%
\pgfusepath{clip}%
\pgfsetbuttcap%
\pgfsetroundjoin%
\definecolor{currentfill}{rgb}{0.121569,0.466667,0.705882}%
\pgfsetfillcolor{currentfill}%
\pgfsetfillopacity{0.894765}%
\pgfsetlinewidth{1.003750pt}%
\definecolor{currentstroke}{rgb}{0.121569,0.466667,0.705882}%
\pgfsetstrokecolor{currentstroke}%
\pgfsetstrokeopacity{0.894765}%
\pgfsetdash{}{0pt}%
\pgfpathmoveto{\pgfqpoint{2.732383in}{1.549786in}}%
\pgfpathcurveto{\pgfqpoint{2.740619in}{1.549786in}}{\pgfqpoint{2.748519in}{1.553058in}}{\pgfqpoint{2.754343in}{1.558882in}}%
\pgfpathcurveto{\pgfqpoint{2.760167in}{1.564706in}}{\pgfqpoint{2.763439in}{1.572606in}}{\pgfqpoint{2.763439in}{1.580843in}}%
\pgfpathcurveto{\pgfqpoint{2.763439in}{1.589079in}}{\pgfqpoint{2.760167in}{1.596979in}}{\pgfqpoint{2.754343in}{1.602803in}}%
\pgfpathcurveto{\pgfqpoint{2.748519in}{1.608627in}}{\pgfqpoint{2.740619in}{1.611899in}}{\pgfqpoint{2.732383in}{1.611899in}}%
\pgfpathcurveto{\pgfqpoint{2.724146in}{1.611899in}}{\pgfqpoint{2.716246in}{1.608627in}}{\pgfqpoint{2.710422in}{1.602803in}}%
\pgfpathcurveto{\pgfqpoint{2.704598in}{1.596979in}}{\pgfqpoint{2.701326in}{1.589079in}}{\pgfqpoint{2.701326in}{1.580843in}}%
\pgfpathcurveto{\pgfqpoint{2.701326in}{1.572606in}}{\pgfqpoint{2.704598in}{1.564706in}}{\pgfqpoint{2.710422in}{1.558882in}}%
\pgfpathcurveto{\pgfqpoint{2.716246in}{1.553058in}}{\pgfqpoint{2.724146in}{1.549786in}}{\pgfqpoint{2.732383in}{1.549786in}}%
\pgfpathclose%
\pgfusepath{stroke,fill}%
\end{pgfscope}%
\begin{pgfscope}%
\pgfpathrectangle{\pgfqpoint{0.100000in}{0.220728in}}{\pgfqpoint{3.696000in}{3.696000in}}%
\pgfusepath{clip}%
\pgfsetbuttcap%
\pgfsetroundjoin%
\definecolor{currentfill}{rgb}{0.121569,0.466667,0.705882}%
\pgfsetfillcolor{currentfill}%
\pgfsetfillopacity{0.895305}%
\pgfsetlinewidth{1.003750pt}%
\definecolor{currentstroke}{rgb}{0.121569,0.466667,0.705882}%
\pgfsetstrokecolor{currentstroke}%
\pgfsetstrokeopacity{0.895305}%
\pgfsetdash{}{0pt}%
\pgfpathmoveto{\pgfqpoint{1.827553in}{0.920331in}}%
\pgfpathcurveto{\pgfqpoint{1.835790in}{0.920331in}}{\pgfqpoint{1.843690in}{0.923604in}}{\pgfqpoint{1.849514in}{0.929428in}}%
\pgfpathcurveto{\pgfqpoint{1.855338in}{0.935252in}}{\pgfqpoint{1.858610in}{0.943152in}}{\pgfqpoint{1.858610in}{0.951388in}}%
\pgfpathcurveto{\pgfqpoint{1.858610in}{0.959624in}}{\pgfqpoint{1.855338in}{0.967524in}}{\pgfqpoint{1.849514in}{0.973348in}}%
\pgfpathcurveto{\pgfqpoint{1.843690in}{0.979172in}}{\pgfqpoint{1.835790in}{0.982444in}}{\pgfqpoint{1.827553in}{0.982444in}}%
\pgfpathcurveto{\pgfqpoint{1.819317in}{0.982444in}}{\pgfqpoint{1.811417in}{0.979172in}}{\pgfqpoint{1.805593in}{0.973348in}}%
\pgfpathcurveto{\pgfqpoint{1.799769in}{0.967524in}}{\pgfqpoint{1.796497in}{0.959624in}}{\pgfqpoint{1.796497in}{0.951388in}}%
\pgfpathcurveto{\pgfqpoint{1.796497in}{0.943152in}}{\pgfqpoint{1.799769in}{0.935252in}}{\pgfqpoint{1.805593in}{0.929428in}}%
\pgfpathcurveto{\pgfqpoint{1.811417in}{0.923604in}}{\pgfqpoint{1.819317in}{0.920331in}}{\pgfqpoint{1.827553in}{0.920331in}}%
\pgfpathclose%
\pgfusepath{stroke,fill}%
\end{pgfscope}%
\begin{pgfscope}%
\pgfpathrectangle{\pgfqpoint{0.100000in}{0.220728in}}{\pgfqpoint{3.696000in}{3.696000in}}%
\pgfusepath{clip}%
\pgfsetbuttcap%
\pgfsetroundjoin%
\definecolor{currentfill}{rgb}{0.121569,0.466667,0.705882}%
\pgfsetfillcolor{currentfill}%
\pgfsetfillopacity{0.895656}%
\pgfsetlinewidth{1.003750pt}%
\definecolor{currentstroke}{rgb}{0.121569,0.466667,0.705882}%
\pgfsetstrokecolor{currentstroke}%
\pgfsetstrokeopacity{0.895656}%
\pgfsetdash{}{0pt}%
\pgfpathmoveto{\pgfqpoint{2.730038in}{1.541673in}}%
\pgfpathcurveto{\pgfqpoint{2.738275in}{1.541673in}}{\pgfqpoint{2.746175in}{1.544946in}}{\pgfqpoint{2.751999in}{1.550769in}}%
\pgfpathcurveto{\pgfqpoint{2.757823in}{1.556593in}}{\pgfqpoint{2.761095in}{1.564493in}}{\pgfqpoint{2.761095in}{1.572730in}}%
\pgfpathcurveto{\pgfqpoint{2.761095in}{1.580966in}}{\pgfqpoint{2.757823in}{1.588866in}}{\pgfqpoint{2.751999in}{1.594690in}}%
\pgfpathcurveto{\pgfqpoint{2.746175in}{1.600514in}}{\pgfqpoint{2.738275in}{1.603786in}}{\pgfqpoint{2.730038in}{1.603786in}}%
\pgfpathcurveto{\pgfqpoint{2.721802in}{1.603786in}}{\pgfqpoint{2.713902in}{1.600514in}}{\pgfqpoint{2.708078in}{1.594690in}}%
\pgfpathcurveto{\pgfqpoint{2.702254in}{1.588866in}}{\pgfqpoint{2.698982in}{1.580966in}}{\pgfqpoint{2.698982in}{1.572730in}}%
\pgfpathcurveto{\pgfqpoint{2.698982in}{1.564493in}}{\pgfqpoint{2.702254in}{1.556593in}}{\pgfqpoint{2.708078in}{1.550769in}}%
\pgfpathcurveto{\pgfqpoint{2.713902in}{1.544946in}}{\pgfqpoint{2.721802in}{1.541673in}}{\pgfqpoint{2.730038in}{1.541673in}}%
\pgfpathclose%
\pgfusepath{stroke,fill}%
\end{pgfscope}%
\begin{pgfscope}%
\pgfpathrectangle{\pgfqpoint{0.100000in}{0.220728in}}{\pgfqpoint{3.696000in}{3.696000in}}%
\pgfusepath{clip}%
\pgfsetbuttcap%
\pgfsetroundjoin%
\definecolor{currentfill}{rgb}{0.121569,0.466667,0.705882}%
\pgfsetfillcolor{currentfill}%
\pgfsetfillopacity{0.896640}%
\pgfsetlinewidth{1.003750pt}%
\definecolor{currentstroke}{rgb}{0.121569,0.466667,0.705882}%
\pgfsetstrokecolor{currentstroke}%
\pgfsetstrokeopacity{0.896640}%
\pgfsetdash{}{0pt}%
\pgfpathmoveto{\pgfqpoint{2.723124in}{1.529664in}}%
\pgfpathcurveto{\pgfqpoint{2.731360in}{1.529664in}}{\pgfqpoint{2.739260in}{1.532937in}}{\pgfqpoint{2.745084in}{1.538761in}}%
\pgfpathcurveto{\pgfqpoint{2.750908in}{1.544584in}}{\pgfqpoint{2.754180in}{1.552485in}}{\pgfqpoint{2.754180in}{1.560721in}}%
\pgfpathcurveto{\pgfqpoint{2.754180in}{1.568957in}}{\pgfqpoint{2.750908in}{1.576857in}}{\pgfqpoint{2.745084in}{1.582681in}}%
\pgfpathcurveto{\pgfqpoint{2.739260in}{1.588505in}}{\pgfqpoint{2.731360in}{1.591777in}}{\pgfqpoint{2.723124in}{1.591777in}}%
\pgfpathcurveto{\pgfqpoint{2.714888in}{1.591777in}}{\pgfqpoint{2.706988in}{1.588505in}}{\pgfqpoint{2.701164in}{1.582681in}}%
\pgfpathcurveto{\pgfqpoint{2.695340in}{1.576857in}}{\pgfqpoint{2.692067in}{1.568957in}}{\pgfqpoint{2.692067in}{1.560721in}}%
\pgfpathcurveto{\pgfqpoint{2.692067in}{1.552485in}}{\pgfqpoint{2.695340in}{1.544584in}}{\pgfqpoint{2.701164in}{1.538761in}}%
\pgfpathcurveto{\pgfqpoint{2.706988in}{1.532937in}}{\pgfqpoint{2.714888in}{1.529664in}}{\pgfqpoint{2.723124in}{1.529664in}}%
\pgfpathclose%
\pgfusepath{stroke,fill}%
\end{pgfscope}%
\begin{pgfscope}%
\pgfpathrectangle{\pgfqpoint{0.100000in}{0.220728in}}{\pgfqpoint{3.696000in}{3.696000in}}%
\pgfusepath{clip}%
\pgfsetbuttcap%
\pgfsetroundjoin%
\definecolor{currentfill}{rgb}{0.121569,0.466667,0.705882}%
\pgfsetfillcolor{currentfill}%
\pgfsetfillopacity{0.898505}%
\pgfsetlinewidth{1.003750pt}%
\definecolor{currentstroke}{rgb}{0.121569,0.466667,0.705882}%
\pgfsetstrokecolor{currentstroke}%
\pgfsetstrokeopacity{0.898505}%
\pgfsetdash{}{0pt}%
\pgfpathmoveto{\pgfqpoint{2.719309in}{1.515694in}}%
\pgfpathcurveto{\pgfqpoint{2.727545in}{1.515694in}}{\pgfqpoint{2.735445in}{1.518966in}}{\pgfqpoint{2.741269in}{1.524790in}}%
\pgfpathcurveto{\pgfqpoint{2.747093in}{1.530614in}}{\pgfqpoint{2.750365in}{1.538514in}}{\pgfqpoint{2.750365in}{1.546750in}}%
\pgfpathcurveto{\pgfqpoint{2.750365in}{1.554987in}}{\pgfqpoint{2.747093in}{1.562887in}}{\pgfqpoint{2.741269in}{1.568711in}}%
\pgfpathcurveto{\pgfqpoint{2.735445in}{1.574535in}}{\pgfqpoint{2.727545in}{1.577807in}}{\pgfqpoint{2.719309in}{1.577807in}}%
\pgfpathcurveto{\pgfqpoint{2.711072in}{1.577807in}}{\pgfqpoint{2.703172in}{1.574535in}}{\pgfqpoint{2.697348in}{1.568711in}}%
\pgfpathcurveto{\pgfqpoint{2.691524in}{1.562887in}}{\pgfqpoint{2.688252in}{1.554987in}}{\pgfqpoint{2.688252in}{1.546750in}}%
\pgfpathcurveto{\pgfqpoint{2.688252in}{1.538514in}}{\pgfqpoint{2.691524in}{1.530614in}}{\pgfqpoint{2.697348in}{1.524790in}}%
\pgfpathcurveto{\pgfqpoint{2.703172in}{1.518966in}}{\pgfqpoint{2.711072in}{1.515694in}}{\pgfqpoint{2.719309in}{1.515694in}}%
\pgfpathclose%
\pgfusepath{stroke,fill}%
\end{pgfscope}%
\begin{pgfscope}%
\pgfpathrectangle{\pgfqpoint{0.100000in}{0.220728in}}{\pgfqpoint{3.696000in}{3.696000in}}%
\pgfusepath{clip}%
\pgfsetbuttcap%
\pgfsetroundjoin%
\definecolor{currentfill}{rgb}{0.121569,0.466667,0.705882}%
\pgfsetfillcolor{currentfill}%
\pgfsetfillopacity{0.899465}%
\pgfsetlinewidth{1.003750pt}%
\definecolor{currentstroke}{rgb}{0.121569,0.466667,0.705882}%
\pgfsetstrokecolor{currentstroke}%
\pgfsetstrokeopacity{0.899465}%
\pgfsetdash{}{0pt}%
\pgfpathmoveto{\pgfqpoint{2.716352in}{1.508493in}}%
\pgfpathcurveto{\pgfqpoint{2.724589in}{1.508493in}}{\pgfqpoint{2.732489in}{1.511765in}}{\pgfqpoint{2.738313in}{1.517589in}}%
\pgfpathcurveto{\pgfqpoint{2.744137in}{1.523413in}}{\pgfqpoint{2.747409in}{1.531313in}}{\pgfqpoint{2.747409in}{1.539549in}}%
\pgfpathcurveto{\pgfqpoint{2.747409in}{1.547786in}}{\pgfqpoint{2.744137in}{1.555686in}}{\pgfqpoint{2.738313in}{1.561510in}}%
\pgfpathcurveto{\pgfqpoint{2.732489in}{1.567334in}}{\pgfqpoint{2.724589in}{1.570606in}}{\pgfqpoint{2.716352in}{1.570606in}}%
\pgfpathcurveto{\pgfqpoint{2.708116in}{1.570606in}}{\pgfqpoint{2.700216in}{1.567334in}}{\pgfqpoint{2.694392in}{1.561510in}}%
\pgfpathcurveto{\pgfqpoint{2.688568in}{1.555686in}}{\pgfqpoint{2.685296in}{1.547786in}}{\pgfqpoint{2.685296in}{1.539549in}}%
\pgfpathcurveto{\pgfqpoint{2.685296in}{1.531313in}}{\pgfqpoint{2.688568in}{1.523413in}}{\pgfqpoint{2.694392in}{1.517589in}}%
\pgfpathcurveto{\pgfqpoint{2.700216in}{1.511765in}}{\pgfqpoint{2.708116in}{1.508493in}}{\pgfqpoint{2.716352in}{1.508493in}}%
\pgfpathclose%
\pgfusepath{stroke,fill}%
\end{pgfscope}%
\begin{pgfscope}%
\pgfpathrectangle{\pgfqpoint{0.100000in}{0.220728in}}{\pgfqpoint{3.696000in}{3.696000in}}%
\pgfusepath{clip}%
\pgfsetbuttcap%
\pgfsetroundjoin%
\definecolor{currentfill}{rgb}{0.121569,0.466667,0.705882}%
\pgfsetfillcolor{currentfill}%
\pgfsetfillopacity{0.899797}%
\pgfsetlinewidth{1.003750pt}%
\definecolor{currentstroke}{rgb}{0.121569,0.466667,0.705882}%
\pgfsetstrokecolor{currentstroke}%
\pgfsetstrokeopacity{0.899797}%
\pgfsetdash{}{0pt}%
\pgfpathmoveto{\pgfqpoint{1.844379in}{0.914444in}}%
\pgfpathcurveto{\pgfqpoint{1.852615in}{0.914444in}}{\pgfqpoint{1.860515in}{0.917716in}}{\pgfqpoint{1.866339in}{0.923540in}}%
\pgfpathcurveto{\pgfqpoint{1.872163in}{0.929364in}}{\pgfqpoint{1.875435in}{0.937264in}}{\pgfqpoint{1.875435in}{0.945500in}}%
\pgfpathcurveto{\pgfqpoint{1.875435in}{0.953736in}}{\pgfqpoint{1.872163in}{0.961637in}}{\pgfqpoint{1.866339in}{0.967460in}}%
\pgfpathcurveto{\pgfqpoint{1.860515in}{0.973284in}}{\pgfqpoint{1.852615in}{0.976557in}}{\pgfqpoint{1.844379in}{0.976557in}}%
\pgfpathcurveto{\pgfqpoint{1.836143in}{0.976557in}}{\pgfqpoint{1.828243in}{0.973284in}}{\pgfqpoint{1.822419in}{0.967460in}}%
\pgfpathcurveto{\pgfqpoint{1.816595in}{0.961637in}}{\pgfqpoint{1.813322in}{0.953736in}}{\pgfqpoint{1.813322in}{0.945500in}}%
\pgfpathcurveto{\pgfqpoint{1.813322in}{0.937264in}}{\pgfqpoint{1.816595in}{0.929364in}}{\pgfqpoint{1.822419in}{0.923540in}}%
\pgfpathcurveto{\pgfqpoint{1.828243in}{0.917716in}}{\pgfqpoint{1.836143in}{0.914444in}}{\pgfqpoint{1.844379in}{0.914444in}}%
\pgfpathclose%
\pgfusepath{stroke,fill}%
\end{pgfscope}%
\begin{pgfscope}%
\pgfpathrectangle{\pgfqpoint{0.100000in}{0.220728in}}{\pgfqpoint{3.696000in}{3.696000in}}%
\pgfusepath{clip}%
\pgfsetbuttcap%
\pgfsetroundjoin%
\definecolor{currentfill}{rgb}{0.121569,0.466667,0.705882}%
\pgfsetfillcolor{currentfill}%
\pgfsetfillopacity{0.899919}%
\pgfsetlinewidth{1.003750pt}%
\definecolor{currentstroke}{rgb}{0.121569,0.466667,0.705882}%
\pgfsetstrokecolor{currentstroke}%
\pgfsetstrokeopacity{0.899919}%
\pgfsetdash{}{0pt}%
\pgfpathmoveto{\pgfqpoint{2.714174in}{1.505018in}}%
\pgfpathcurveto{\pgfqpoint{2.722410in}{1.505018in}}{\pgfqpoint{2.730310in}{1.508290in}}{\pgfqpoint{2.736134in}{1.514114in}}%
\pgfpathcurveto{\pgfqpoint{2.741958in}{1.519938in}}{\pgfqpoint{2.745230in}{1.527838in}}{\pgfqpoint{2.745230in}{1.536074in}}%
\pgfpathcurveto{\pgfqpoint{2.745230in}{1.544311in}}{\pgfqpoint{2.741958in}{1.552211in}}{\pgfqpoint{2.736134in}{1.558035in}}%
\pgfpathcurveto{\pgfqpoint{2.730310in}{1.563859in}}{\pgfqpoint{2.722410in}{1.567131in}}{\pgfqpoint{2.714174in}{1.567131in}}%
\pgfpathcurveto{\pgfqpoint{2.705937in}{1.567131in}}{\pgfqpoint{2.698037in}{1.563859in}}{\pgfqpoint{2.692213in}{1.558035in}}%
\pgfpathcurveto{\pgfqpoint{2.686389in}{1.552211in}}{\pgfqpoint{2.683117in}{1.544311in}}{\pgfqpoint{2.683117in}{1.536074in}}%
\pgfpathcurveto{\pgfqpoint{2.683117in}{1.527838in}}{\pgfqpoint{2.686389in}{1.519938in}}{\pgfqpoint{2.692213in}{1.514114in}}%
\pgfpathcurveto{\pgfqpoint{2.698037in}{1.508290in}}{\pgfqpoint{2.705937in}{1.505018in}}{\pgfqpoint{2.714174in}{1.505018in}}%
\pgfpathclose%
\pgfusepath{stroke,fill}%
\end{pgfscope}%
\begin{pgfscope}%
\pgfpathrectangle{\pgfqpoint{0.100000in}{0.220728in}}{\pgfqpoint{3.696000in}{3.696000in}}%
\pgfusepath{clip}%
\pgfsetbuttcap%
\pgfsetroundjoin%
\definecolor{currentfill}{rgb}{0.121569,0.466667,0.705882}%
\pgfsetfillcolor{currentfill}%
\pgfsetfillopacity{0.901026}%
\pgfsetlinewidth{1.003750pt}%
\definecolor{currentstroke}{rgb}{0.121569,0.466667,0.705882}%
\pgfsetstrokecolor{currentstroke}%
\pgfsetstrokeopacity{0.901026}%
\pgfsetdash{}{0pt}%
\pgfpathmoveto{\pgfqpoint{2.712017in}{1.496584in}}%
\pgfpathcurveto{\pgfqpoint{2.720253in}{1.496584in}}{\pgfqpoint{2.728153in}{1.499856in}}{\pgfqpoint{2.733977in}{1.505680in}}%
\pgfpathcurveto{\pgfqpoint{2.739801in}{1.511504in}}{\pgfqpoint{2.743074in}{1.519404in}}{\pgfqpoint{2.743074in}{1.527640in}}%
\pgfpathcurveto{\pgfqpoint{2.743074in}{1.535876in}}{\pgfqpoint{2.739801in}{1.543776in}}{\pgfqpoint{2.733977in}{1.549600in}}%
\pgfpathcurveto{\pgfqpoint{2.728153in}{1.555424in}}{\pgfqpoint{2.720253in}{1.558697in}}{\pgfqpoint{2.712017in}{1.558697in}}%
\pgfpathcurveto{\pgfqpoint{2.703781in}{1.558697in}}{\pgfqpoint{2.695881in}{1.555424in}}{\pgfqpoint{2.690057in}{1.549600in}}%
\pgfpathcurveto{\pgfqpoint{2.684233in}{1.543776in}}{\pgfqpoint{2.680961in}{1.535876in}}{\pgfqpoint{2.680961in}{1.527640in}}%
\pgfpathcurveto{\pgfqpoint{2.680961in}{1.519404in}}{\pgfqpoint{2.684233in}{1.511504in}}{\pgfqpoint{2.690057in}{1.505680in}}%
\pgfpathcurveto{\pgfqpoint{2.695881in}{1.499856in}}{\pgfqpoint{2.703781in}{1.496584in}}{\pgfqpoint{2.712017in}{1.496584in}}%
\pgfpathclose%
\pgfusepath{stroke,fill}%
\end{pgfscope}%
\begin{pgfscope}%
\pgfpathrectangle{\pgfqpoint{0.100000in}{0.220728in}}{\pgfqpoint{3.696000in}{3.696000in}}%
\pgfusepath{clip}%
\pgfsetbuttcap%
\pgfsetroundjoin%
\definecolor{currentfill}{rgb}{0.121569,0.466667,0.705882}%
\pgfsetfillcolor{currentfill}%
\pgfsetfillopacity{0.902014}%
\pgfsetlinewidth{1.003750pt}%
\definecolor{currentstroke}{rgb}{0.121569,0.466667,0.705882}%
\pgfsetstrokecolor{currentstroke}%
\pgfsetstrokeopacity{0.902014}%
\pgfsetdash{}{0pt}%
\pgfpathmoveto{\pgfqpoint{2.707323in}{1.488389in}}%
\pgfpathcurveto{\pgfqpoint{2.715560in}{1.488389in}}{\pgfqpoint{2.723460in}{1.491661in}}{\pgfqpoint{2.729284in}{1.497485in}}%
\pgfpathcurveto{\pgfqpoint{2.735108in}{1.503309in}}{\pgfqpoint{2.738380in}{1.511209in}}{\pgfqpoint{2.738380in}{1.519445in}}%
\pgfpathcurveto{\pgfqpoint{2.738380in}{1.527682in}}{\pgfqpoint{2.735108in}{1.535582in}}{\pgfqpoint{2.729284in}{1.541406in}}%
\pgfpathcurveto{\pgfqpoint{2.723460in}{1.547230in}}{\pgfqpoint{2.715560in}{1.550502in}}{\pgfqpoint{2.707323in}{1.550502in}}%
\pgfpathcurveto{\pgfqpoint{2.699087in}{1.550502in}}{\pgfqpoint{2.691187in}{1.547230in}}{\pgfqpoint{2.685363in}{1.541406in}}%
\pgfpathcurveto{\pgfqpoint{2.679539in}{1.535582in}}{\pgfqpoint{2.676267in}{1.527682in}}{\pgfqpoint{2.676267in}{1.519445in}}%
\pgfpathcurveto{\pgfqpoint{2.676267in}{1.511209in}}{\pgfqpoint{2.679539in}{1.503309in}}{\pgfqpoint{2.685363in}{1.497485in}}%
\pgfpathcurveto{\pgfqpoint{2.691187in}{1.491661in}}{\pgfqpoint{2.699087in}{1.488389in}}{\pgfqpoint{2.707323in}{1.488389in}}%
\pgfpathclose%
\pgfusepath{stroke,fill}%
\end{pgfscope}%
\begin{pgfscope}%
\pgfpathrectangle{\pgfqpoint{0.100000in}{0.220728in}}{\pgfqpoint{3.696000in}{3.696000in}}%
\pgfusepath{clip}%
\pgfsetbuttcap%
\pgfsetroundjoin%
\definecolor{currentfill}{rgb}{0.121569,0.466667,0.705882}%
\pgfsetfillcolor{currentfill}%
\pgfsetfillopacity{0.903389}%
\pgfsetlinewidth{1.003750pt}%
\definecolor{currentstroke}{rgb}{0.121569,0.466667,0.705882}%
\pgfsetstrokecolor{currentstroke}%
\pgfsetstrokeopacity{0.903389}%
\pgfsetdash{}{0pt}%
\pgfpathmoveto{\pgfqpoint{2.702370in}{1.478780in}}%
\pgfpathcurveto{\pgfqpoint{2.710606in}{1.478780in}}{\pgfqpoint{2.718506in}{1.482052in}}{\pgfqpoint{2.724330in}{1.487876in}}%
\pgfpathcurveto{\pgfqpoint{2.730154in}{1.493700in}}{\pgfqpoint{2.733427in}{1.501600in}}{\pgfqpoint{2.733427in}{1.509836in}}%
\pgfpathcurveto{\pgfqpoint{2.733427in}{1.518073in}}{\pgfqpoint{2.730154in}{1.525973in}}{\pgfqpoint{2.724330in}{1.531796in}}%
\pgfpathcurveto{\pgfqpoint{2.718506in}{1.537620in}}{\pgfqpoint{2.710606in}{1.540893in}}{\pgfqpoint{2.702370in}{1.540893in}}%
\pgfpathcurveto{\pgfqpoint{2.694134in}{1.540893in}}{\pgfqpoint{2.686234in}{1.537620in}}{\pgfqpoint{2.680410in}{1.531796in}}%
\pgfpathcurveto{\pgfqpoint{2.674586in}{1.525973in}}{\pgfqpoint{2.671314in}{1.518073in}}{\pgfqpoint{2.671314in}{1.509836in}}%
\pgfpathcurveto{\pgfqpoint{2.671314in}{1.501600in}}{\pgfqpoint{2.674586in}{1.493700in}}{\pgfqpoint{2.680410in}{1.487876in}}%
\pgfpathcurveto{\pgfqpoint{2.686234in}{1.482052in}}{\pgfqpoint{2.694134in}{1.478780in}}{\pgfqpoint{2.702370in}{1.478780in}}%
\pgfpathclose%
\pgfusepath{stroke,fill}%
\end{pgfscope}%
\begin{pgfscope}%
\pgfpathrectangle{\pgfqpoint{0.100000in}{0.220728in}}{\pgfqpoint{3.696000in}{3.696000in}}%
\pgfusepath{clip}%
\pgfsetbuttcap%
\pgfsetroundjoin%
\definecolor{currentfill}{rgb}{0.121569,0.466667,0.705882}%
\pgfsetfillcolor{currentfill}%
\pgfsetfillopacity{0.903463}%
\pgfsetlinewidth{1.003750pt}%
\definecolor{currentstroke}{rgb}{0.121569,0.466667,0.705882}%
\pgfsetstrokecolor{currentstroke}%
\pgfsetstrokeopacity{0.903463}%
\pgfsetdash{}{0pt}%
\pgfpathmoveto{\pgfqpoint{1.860310in}{0.909279in}}%
\pgfpathcurveto{\pgfqpoint{1.868546in}{0.909279in}}{\pgfqpoint{1.876446in}{0.912551in}}{\pgfqpoint{1.882270in}{0.918375in}}%
\pgfpathcurveto{\pgfqpoint{1.888094in}{0.924199in}}{\pgfqpoint{1.891366in}{0.932099in}}{\pgfqpoint{1.891366in}{0.940335in}}%
\pgfpathcurveto{\pgfqpoint{1.891366in}{0.948571in}}{\pgfqpoint{1.888094in}{0.956471in}}{\pgfqpoint{1.882270in}{0.962295in}}%
\pgfpathcurveto{\pgfqpoint{1.876446in}{0.968119in}}{\pgfqpoint{1.868546in}{0.971392in}}{\pgfqpoint{1.860310in}{0.971392in}}%
\pgfpathcurveto{\pgfqpoint{1.852073in}{0.971392in}}{\pgfqpoint{1.844173in}{0.968119in}}{\pgfqpoint{1.838349in}{0.962295in}}%
\pgfpathcurveto{\pgfqpoint{1.832525in}{0.956471in}}{\pgfqpoint{1.829253in}{0.948571in}}{\pgfqpoint{1.829253in}{0.940335in}}%
\pgfpathcurveto{\pgfqpoint{1.829253in}{0.932099in}}{\pgfqpoint{1.832525in}{0.924199in}}{\pgfqpoint{1.838349in}{0.918375in}}%
\pgfpathcurveto{\pgfqpoint{1.844173in}{0.912551in}}{\pgfqpoint{1.852073in}{0.909279in}}{\pgfqpoint{1.860310in}{0.909279in}}%
\pgfpathclose%
\pgfusepath{stroke,fill}%
\end{pgfscope}%
\begin{pgfscope}%
\pgfpathrectangle{\pgfqpoint{0.100000in}{0.220728in}}{\pgfqpoint{3.696000in}{3.696000in}}%
\pgfusepath{clip}%
\pgfsetbuttcap%
\pgfsetroundjoin%
\definecolor{currentfill}{rgb}{0.121569,0.466667,0.705882}%
\pgfsetfillcolor{currentfill}%
\pgfsetfillopacity{0.904077}%
\pgfsetlinewidth{1.003750pt}%
\definecolor{currentstroke}{rgb}{0.121569,0.466667,0.705882}%
\pgfsetstrokecolor{currentstroke}%
\pgfsetstrokeopacity{0.904077}%
\pgfsetdash{}{0pt}%
\pgfpathmoveto{\pgfqpoint{2.700776in}{1.472115in}}%
\pgfpathcurveto{\pgfqpoint{2.709012in}{1.472115in}}{\pgfqpoint{2.716912in}{1.475387in}}{\pgfqpoint{2.722736in}{1.481211in}}%
\pgfpathcurveto{\pgfqpoint{2.728560in}{1.487035in}}{\pgfqpoint{2.731833in}{1.494935in}}{\pgfqpoint{2.731833in}{1.503171in}}%
\pgfpathcurveto{\pgfqpoint{2.731833in}{1.511408in}}{\pgfqpoint{2.728560in}{1.519308in}}{\pgfqpoint{2.722736in}{1.525132in}}%
\pgfpathcurveto{\pgfqpoint{2.716912in}{1.530956in}}{\pgfqpoint{2.709012in}{1.534228in}}{\pgfqpoint{2.700776in}{1.534228in}}%
\pgfpathcurveto{\pgfqpoint{2.692540in}{1.534228in}}{\pgfqpoint{2.684640in}{1.530956in}}{\pgfqpoint{2.678816in}{1.525132in}}%
\pgfpathcurveto{\pgfqpoint{2.672992in}{1.519308in}}{\pgfqpoint{2.669720in}{1.511408in}}{\pgfqpoint{2.669720in}{1.503171in}}%
\pgfpathcurveto{\pgfqpoint{2.669720in}{1.494935in}}{\pgfqpoint{2.672992in}{1.487035in}}{\pgfqpoint{2.678816in}{1.481211in}}%
\pgfpathcurveto{\pgfqpoint{2.684640in}{1.475387in}}{\pgfqpoint{2.692540in}{1.472115in}}{\pgfqpoint{2.700776in}{1.472115in}}%
\pgfpathclose%
\pgfusepath{stroke,fill}%
\end{pgfscope}%
\begin{pgfscope}%
\pgfpathrectangle{\pgfqpoint{0.100000in}{0.220728in}}{\pgfqpoint{3.696000in}{3.696000in}}%
\pgfusepath{clip}%
\pgfsetbuttcap%
\pgfsetroundjoin%
\definecolor{currentfill}{rgb}{0.121569,0.466667,0.705882}%
\pgfsetfillcolor{currentfill}%
\pgfsetfillopacity{0.905493}%
\pgfsetlinewidth{1.003750pt}%
\definecolor{currentstroke}{rgb}{0.121569,0.466667,0.705882}%
\pgfsetstrokecolor{currentstroke}%
\pgfsetstrokeopacity{0.905493}%
\pgfsetdash{}{0pt}%
\pgfpathmoveto{\pgfqpoint{2.695646in}{1.462757in}}%
\pgfpathcurveto{\pgfqpoint{2.703882in}{1.462757in}}{\pgfqpoint{2.711782in}{1.466029in}}{\pgfqpoint{2.717606in}{1.471853in}}%
\pgfpathcurveto{\pgfqpoint{2.723430in}{1.477677in}}{\pgfqpoint{2.726703in}{1.485577in}}{\pgfqpoint{2.726703in}{1.493813in}}%
\pgfpathcurveto{\pgfqpoint{2.726703in}{1.502049in}}{\pgfqpoint{2.723430in}{1.509950in}}{\pgfqpoint{2.717606in}{1.515773in}}%
\pgfpathcurveto{\pgfqpoint{2.711782in}{1.521597in}}{\pgfqpoint{2.703882in}{1.524870in}}{\pgfqpoint{2.695646in}{1.524870in}}%
\pgfpathcurveto{\pgfqpoint{2.687410in}{1.524870in}}{\pgfqpoint{2.679510in}{1.521597in}}{\pgfqpoint{2.673686in}{1.515773in}}%
\pgfpathcurveto{\pgfqpoint{2.667862in}{1.509950in}}{\pgfqpoint{2.664590in}{1.502049in}}{\pgfqpoint{2.664590in}{1.493813in}}%
\pgfpathcurveto{\pgfqpoint{2.664590in}{1.485577in}}{\pgfqpoint{2.667862in}{1.477677in}}{\pgfqpoint{2.673686in}{1.471853in}}%
\pgfpathcurveto{\pgfqpoint{2.679510in}{1.466029in}}{\pgfqpoint{2.687410in}{1.462757in}}{\pgfqpoint{2.695646in}{1.462757in}}%
\pgfpathclose%
\pgfusepath{stroke,fill}%
\end{pgfscope}%
\begin{pgfscope}%
\pgfpathrectangle{\pgfqpoint{0.100000in}{0.220728in}}{\pgfqpoint{3.696000in}{3.696000in}}%
\pgfusepath{clip}%
\pgfsetbuttcap%
\pgfsetroundjoin%
\definecolor{currentfill}{rgb}{0.121569,0.466667,0.705882}%
\pgfsetfillcolor{currentfill}%
\pgfsetfillopacity{0.906276}%
\pgfsetlinewidth{1.003750pt}%
\definecolor{currentstroke}{rgb}{0.121569,0.466667,0.705882}%
\pgfsetstrokecolor{currentstroke}%
\pgfsetstrokeopacity{0.906276}%
\pgfsetdash{}{0pt}%
\pgfpathmoveto{\pgfqpoint{2.693169in}{1.457171in}}%
\pgfpathcurveto{\pgfqpoint{2.701405in}{1.457171in}}{\pgfqpoint{2.709305in}{1.460444in}}{\pgfqpoint{2.715129in}{1.466268in}}%
\pgfpathcurveto{\pgfqpoint{2.720953in}{1.472092in}}{\pgfqpoint{2.724225in}{1.479992in}}{\pgfqpoint{2.724225in}{1.488228in}}%
\pgfpathcurveto{\pgfqpoint{2.724225in}{1.496464in}}{\pgfqpoint{2.720953in}{1.504364in}}{\pgfqpoint{2.715129in}{1.510188in}}%
\pgfpathcurveto{\pgfqpoint{2.709305in}{1.516012in}}{\pgfqpoint{2.701405in}{1.519284in}}{\pgfqpoint{2.693169in}{1.519284in}}%
\pgfpathcurveto{\pgfqpoint{2.684933in}{1.519284in}}{\pgfqpoint{2.677033in}{1.516012in}}{\pgfqpoint{2.671209in}{1.510188in}}%
\pgfpathcurveto{\pgfqpoint{2.665385in}{1.504364in}}{\pgfqpoint{2.662112in}{1.496464in}}{\pgfqpoint{2.662112in}{1.488228in}}%
\pgfpathcurveto{\pgfqpoint{2.662112in}{1.479992in}}{\pgfqpoint{2.665385in}{1.472092in}}{\pgfqpoint{2.671209in}{1.466268in}}%
\pgfpathcurveto{\pgfqpoint{2.677033in}{1.460444in}}{\pgfqpoint{2.684933in}{1.457171in}}{\pgfqpoint{2.693169in}{1.457171in}}%
\pgfpathclose%
\pgfusepath{stroke,fill}%
\end{pgfscope}%
\begin{pgfscope}%
\pgfpathrectangle{\pgfqpoint{0.100000in}{0.220728in}}{\pgfqpoint{3.696000in}{3.696000in}}%
\pgfusepath{clip}%
\pgfsetbuttcap%
\pgfsetroundjoin%
\definecolor{currentfill}{rgb}{0.121569,0.466667,0.705882}%
\pgfsetfillcolor{currentfill}%
\pgfsetfillopacity{0.906666}%
\pgfsetlinewidth{1.003750pt}%
\definecolor{currentstroke}{rgb}{0.121569,0.466667,0.705882}%
\pgfsetstrokecolor{currentstroke}%
\pgfsetstrokeopacity{0.906666}%
\pgfsetdash{}{0pt}%
\pgfpathmoveto{\pgfqpoint{2.692330in}{1.453485in}}%
\pgfpathcurveto{\pgfqpoint{2.700566in}{1.453485in}}{\pgfqpoint{2.708466in}{1.456758in}}{\pgfqpoint{2.714290in}{1.462582in}}%
\pgfpathcurveto{\pgfqpoint{2.720114in}{1.468405in}}{\pgfqpoint{2.723386in}{1.476305in}}{\pgfqpoint{2.723386in}{1.484542in}}%
\pgfpathcurveto{\pgfqpoint{2.723386in}{1.492778in}}{\pgfqpoint{2.720114in}{1.500678in}}{\pgfqpoint{2.714290in}{1.506502in}}%
\pgfpathcurveto{\pgfqpoint{2.708466in}{1.512326in}}{\pgfqpoint{2.700566in}{1.515598in}}{\pgfqpoint{2.692330in}{1.515598in}}%
\pgfpathcurveto{\pgfqpoint{2.684094in}{1.515598in}}{\pgfqpoint{2.676194in}{1.512326in}}{\pgfqpoint{2.670370in}{1.506502in}}%
\pgfpathcurveto{\pgfqpoint{2.664546in}{1.500678in}}{\pgfqpoint{2.661273in}{1.492778in}}{\pgfqpoint{2.661273in}{1.484542in}}%
\pgfpathcurveto{\pgfqpoint{2.661273in}{1.476305in}}{\pgfqpoint{2.664546in}{1.468405in}}{\pgfqpoint{2.670370in}{1.462582in}}%
\pgfpathcurveto{\pgfqpoint{2.676194in}{1.456758in}}{\pgfqpoint{2.684094in}{1.453485in}}{\pgfqpoint{2.692330in}{1.453485in}}%
\pgfpathclose%
\pgfusepath{stroke,fill}%
\end{pgfscope}%
\begin{pgfscope}%
\pgfpathrectangle{\pgfqpoint{0.100000in}{0.220728in}}{\pgfqpoint{3.696000in}{3.696000in}}%
\pgfusepath{clip}%
\pgfsetbuttcap%
\pgfsetroundjoin%
\definecolor{currentfill}{rgb}{0.121569,0.466667,0.705882}%
\pgfsetfillcolor{currentfill}%
\pgfsetfillopacity{0.907619}%
\pgfsetlinewidth{1.003750pt}%
\definecolor{currentstroke}{rgb}{0.121569,0.466667,0.705882}%
\pgfsetstrokecolor{currentstroke}%
\pgfsetstrokeopacity{0.907619}%
\pgfsetdash{}{0pt}%
\pgfpathmoveto{\pgfqpoint{2.689111in}{1.447348in}}%
\pgfpathcurveto{\pgfqpoint{2.697347in}{1.447348in}}{\pgfqpoint{2.705247in}{1.450620in}}{\pgfqpoint{2.711071in}{1.456444in}}%
\pgfpathcurveto{\pgfqpoint{2.716895in}{1.462268in}}{\pgfqpoint{2.720167in}{1.470168in}}{\pgfqpoint{2.720167in}{1.478404in}}%
\pgfpathcurveto{\pgfqpoint{2.720167in}{1.486641in}}{\pgfqpoint{2.716895in}{1.494541in}}{\pgfqpoint{2.711071in}{1.500365in}}%
\pgfpathcurveto{\pgfqpoint{2.705247in}{1.506189in}}{\pgfqpoint{2.697347in}{1.509461in}}{\pgfqpoint{2.689111in}{1.509461in}}%
\pgfpathcurveto{\pgfqpoint{2.680874in}{1.509461in}}{\pgfqpoint{2.672974in}{1.506189in}}{\pgfqpoint{2.667150in}{1.500365in}}%
\pgfpathcurveto{\pgfqpoint{2.661327in}{1.494541in}}{\pgfqpoint{2.658054in}{1.486641in}}{\pgfqpoint{2.658054in}{1.478404in}}%
\pgfpathcurveto{\pgfqpoint{2.658054in}{1.470168in}}{\pgfqpoint{2.661327in}{1.462268in}}{\pgfqpoint{2.667150in}{1.456444in}}%
\pgfpathcurveto{\pgfqpoint{2.672974in}{1.450620in}}{\pgfqpoint{2.680874in}{1.447348in}}{\pgfqpoint{2.689111in}{1.447348in}}%
\pgfpathclose%
\pgfusepath{stroke,fill}%
\end{pgfscope}%
\begin{pgfscope}%
\pgfpathrectangle{\pgfqpoint{0.100000in}{0.220728in}}{\pgfqpoint{3.696000in}{3.696000in}}%
\pgfusepath{clip}%
\pgfsetbuttcap%
\pgfsetroundjoin%
\definecolor{currentfill}{rgb}{0.121569,0.466667,0.705882}%
\pgfsetfillcolor{currentfill}%
\pgfsetfillopacity{0.907873}%
\pgfsetlinewidth{1.003750pt}%
\definecolor{currentstroke}{rgb}{0.121569,0.466667,0.705882}%
\pgfsetstrokecolor{currentstroke}%
\pgfsetstrokeopacity{0.907873}%
\pgfsetdash{}{0pt}%
\pgfpathmoveto{\pgfqpoint{1.874070in}{0.905196in}}%
\pgfpathcurveto{\pgfqpoint{1.882306in}{0.905196in}}{\pgfqpoint{1.890206in}{0.908468in}}{\pgfqpoint{1.896030in}{0.914292in}}%
\pgfpathcurveto{\pgfqpoint{1.901854in}{0.920116in}}{\pgfqpoint{1.905126in}{0.928016in}}{\pgfqpoint{1.905126in}{0.936252in}}%
\pgfpathcurveto{\pgfqpoint{1.905126in}{0.944488in}}{\pgfqpoint{1.901854in}{0.952388in}}{\pgfqpoint{1.896030in}{0.958212in}}%
\pgfpathcurveto{\pgfqpoint{1.890206in}{0.964036in}}{\pgfqpoint{1.882306in}{0.967309in}}{\pgfqpoint{1.874070in}{0.967309in}}%
\pgfpathcurveto{\pgfqpoint{1.865834in}{0.967309in}}{\pgfqpoint{1.857934in}{0.964036in}}{\pgfqpoint{1.852110in}{0.958212in}}%
\pgfpathcurveto{\pgfqpoint{1.846286in}{0.952388in}}{\pgfqpoint{1.843013in}{0.944488in}}{\pgfqpoint{1.843013in}{0.936252in}}%
\pgfpathcurveto{\pgfqpoint{1.843013in}{0.928016in}}{\pgfqpoint{1.846286in}{0.920116in}}{\pgfqpoint{1.852110in}{0.914292in}}%
\pgfpathcurveto{\pgfqpoint{1.857934in}{0.908468in}}{\pgfqpoint{1.865834in}{0.905196in}}{\pgfqpoint{1.874070in}{0.905196in}}%
\pgfpathclose%
\pgfusepath{stroke,fill}%
\end{pgfscope}%
\begin{pgfscope}%
\pgfpathrectangle{\pgfqpoint{0.100000in}{0.220728in}}{\pgfqpoint{3.696000in}{3.696000in}}%
\pgfusepath{clip}%
\pgfsetbuttcap%
\pgfsetroundjoin%
\definecolor{currentfill}{rgb}{0.121569,0.466667,0.705882}%
\pgfsetfillcolor{currentfill}%
\pgfsetfillopacity{0.908708}%
\pgfsetlinewidth{1.003750pt}%
\definecolor{currentstroke}{rgb}{0.121569,0.466667,0.705882}%
\pgfsetstrokecolor{currentstroke}%
\pgfsetstrokeopacity{0.908708}%
\pgfsetdash{}{0pt}%
\pgfpathmoveto{\pgfqpoint{2.686440in}{1.438795in}}%
\pgfpathcurveto{\pgfqpoint{2.694676in}{1.438795in}}{\pgfqpoint{2.702576in}{1.442067in}}{\pgfqpoint{2.708400in}{1.447891in}}%
\pgfpathcurveto{\pgfqpoint{2.714224in}{1.453715in}}{\pgfqpoint{2.717497in}{1.461615in}}{\pgfqpoint{2.717497in}{1.469851in}}%
\pgfpathcurveto{\pgfqpoint{2.717497in}{1.478087in}}{\pgfqpoint{2.714224in}{1.485987in}}{\pgfqpoint{2.708400in}{1.491811in}}%
\pgfpathcurveto{\pgfqpoint{2.702576in}{1.497635in}}{\pgfqpoint{2.694676in}{1.500908in}}{\pgfqpoint{2.686440in}{1.500908in}}%
\pgfpathcurveto{\pgfqpoint{2.678204in}{1.500908in}}{\pgfqpoint{2.670304in}{1.497635in}}{\pgfqpoint{2.664480in}{1.491811in}}%
\pgfpathcurveto{\pgfqpoint{2.658656in}{1.485987in}}{\pgfqpoint{2.655384in}{1.478087in}}{\pgfqpoint{2.655384in}{1.469851in}}%
\pgfpathcurveto{\pgfqpoint{2.655384in}{1.461615in}}{\pgfqpoint{2.658656in}{1.453715in}}{\pgfqpoint{2.664480in}{1.447891in}}%
\pgfpathcurveto{\pgfqpoint{2.670304in}{1.442067in}}{\pgfqpoint{2.678204in}{1.438795in}}{\pgfqpoint{2.686440in}{1.438795in}}%
\pgfpathclose%
\pgfusepath{stroke,fill}%
\end{pgfscope}%
\begin{pgfscope}%
\pgfpathrectangle{\pgfqpoint{0.100000in}{0.220728in}}{\pgfqpoint{3.696000in}{3.696000in}}%
\pgfusepath{clip}%
\pgfsetbuttcap%
\pgfsetroundjoin%
\definecolor{currentfill}{rgb}{0.121569,0.466667,0.705882}%
\pgfsetfillcolor{currentfill}%
\pgfsetfillopacity{0.909333}%
\pgfsetlinewidth{1.003750pt}%
\definecolor{currentstroke}{rgb}{0.121569,0.466667,0.705882}%
\pgfsetstrokecolor{currentstroke}%
\pgfsetstrokeopacity{0.909333}%
\pgfsetdash{}{0pt}%
\pgfpathmoveto{\pgfqpoint{2.685082in}{1.434110in}}%
\pgfpathcurveto{\pgfqpoint{2.693318in}{1.434110in}}{\pgfqpoint{2.701218in}{1.437382in}}{\pgfqpoint{2.707042in}{1.443206in}}%
\pgfpathcurveto{\pgfqpoint{2.712866in}{1.449030in}}{\pgfqpoint{2.716139in}{1.456930in}}{\pgfqpoint{2.716139in}{1.465166in}}%
\pgfpathcurveto{\pgfqpoint{2.716139in}{1.473403in}}{\pgfqpoint{2.712866in}{1.481303in}}{\pgfqpoint{2.707042in}{1.487127in}}%
\pgfpathcurveto{\pgfqpoint{2.701218in}{1.492950in}}{\pgfqpoint{2.693318in}{1.496223in}}{\pgfqpoint{2.685082in}{1.496223in}}%
\pgfpathcurveto{\pgfqpoint{2.676846in}{1.496223in}}{\pgfqpoint{2.668946in}{1.492950in}}{\pgfqpoint{2.663122in}{1.487127in}}%
\pgfpathcurveto{\pgfqpoint{2.657298in}{1.481303in}}{\pgfqpoint{2.654026in}{1.473403in}}{\pgfqpoint{2.654026in}{1.465166in}}%
\pgfpathcurveto{\pgfqpoint{2.654026in}{1.456930in}}{\pgfqpoint{2.657298in}{1.449030in}}{\pgfqpoint{2.663122in}{1.443206in}}%
\pgfpathcurveto{\pgfqpoint{2.668946in}{1.437382in}}{\pgfqpoint{2.676846in}{1.434110in}}{\pgfqpoint{2.685082in}{1.434110in}}%
\pgfpathclose%
\pgfusepath{stroke,fill}%
\end{pgfscope}%
\begin{pgfscope}%
\pgfpathrectangle{\pgfqpoint{0.100000in}{0.220728in}}{\pgfqpoint{3.696000in}{3.696000in}}%
\pgfusepath{clip}%
\pgfsetbuttcap%
\pgfsetroundjoin%
\definecolor{currentfill}{rgb}{0.121569,0.466667,0.705882}%
\pgfsetfillcolor{currentfill}%
\pgfsetfillopacity{0.910293}%
\pgfsetlinewidth{1.003750pt}%
\definecolor{currentstroke}{rgb}{0.121569,0.466667,0.705882}%
\pgfsetstrokecolor{currentstroke}%
\pgfsetstrokeopacity{0.910293}%
\pgfsetdash{}{0pt}%
\pgfpathmoveto{\pgfqpoint{2.681417in}{1.428725in}}%
\pgfpathcurveto{\pgfqpoint{2.689654in}{1.428725in}}{\pgfqpoint{2.697554in}{1.431997in}}{\pgfqpoint{2.703378in}{1.437821in}}%
\pgfpathcurveto{\pgfqpoint{2.709201in}{1.443645in}}{\pgfqpoint{2.712474in}{1.451545in}}{\pgfqpoint{2.712474in}{1.459781in}}%
\pgfpathcurveto{\pgfqpoint{2.712474in}{1.468017in}}{\pgfqpoint{2.709201in}{1.475918in}}{\pgfqpoint{2.703378in}{1.481741in}}%
\pgfpathcurveto{\pgfqpoint{2.697554in}{1.487565in}}{\pgfqpoint{2.689654in}{1.490838in}}{\pgfqpoint{2.681417in}{1.490838in}}%
\pgfpathcurveto{\pgfqpoint{2.673181in}{1.490838in}}{\pgfqpoint{2.665281in}{1.487565in}}{\pgfqpoint{2.659457in}{1.481741in}}%
\pgfpathcurveto{\pgfqpoint{2.653633in}{1.475918in}}{\pgfqpoint{2.650361in}{1.468017in}}{\pgfqpoint{2.650361in}{1.459781in}}%
\pgfpathcurveto{\pgfqpoint{2.650361in}{1.451545in}}{\pgfqpoint{2.653633in}{1.443645in}}{\pgfqpoint{2.659457in}{1.437821in}}%
\pgfpathcurveto{\pgfqpoint{2.665281in}{1.431997in}}{\pgfqpoint{2.673181in}{1.428725in}}{\pgfqpoint{2.681417in}{1.428725in}}%
\pgfpathclose%
\pgfusepath{stroke,fill}%
\end{pgfscope}%
\begin{pgfscope}%
\pgfpathrectangle{\pgfqpoint{0.100000in}{0.220728in}}{\pgfqpoint{3.696000in}{3.696000in}}%
\pgfusepath{clip}%
\pgfsetbuttcap%
\pgfsetroundjoin%
\definecolor{currentfill}{rgb}{0.121569,0.466667,0.705882}%
\pgfsetfillcolor{currentfill}%
\pgfsetfillopacity{0.911545}%
\pgfsetlinewidth{1.003750pt}%
\definecolor{currentstroke}{rgb}{0.121569,0.466667,0.705882}%
\pgfsetstrokecolor{currentstroke}%
\pgfsetstrokeopacity{0.911545}%
\pgfsetdash{}{0pt}%
\pgfpathmoveto{\pgfqpoint{2.678641in}{1.418728in}}%
\pgfpathcurveto{\pgfqpoint{2.686877in}{1.418728in}}{\pgfqpoint{2.694777in}{1.422000in}}{\pgfqpoint{2.700601in}{1.427824in}}%
\pgfpathcurveto{\pgfqpoint{2.706425in}{1.433648in}}{\pgfqpoint{2.709697in}{1.441548in}}{\pgfqpoint{2.709697in}{1.449784in}}%
\pgfpathcurveto{\pgfqpoint{2.709697in}{1.458021in}}{\pgfqpoint{2.706425in}{1.465921in}}{\pgfqpoint{2.700601in}{1.471745in}}%
\pgfpathcurveto{\pgfqpoint{2.694777in}{1.477568in}}{\pgfqpoint{2.686877in}{1.480841in}}{\pgfqpoint{2.678641in}{1.480841in}}%
\pgfpathcurveto{\pgfqpoint{2.670405in}{1.480841in}}{\pgfqpoint{2.662505in}{1.477568in}}{\pgfqpoint{2.656681in}{1.471745in}}%
\pgfpathcurveto{\pgfqpoint{2.650857in}{1.465921in}}{\pgfqpoint{2.647584in}{1.458021in}}{\pgfqpoint{2.647584in}{1.449784in}}%
\pgfpathcurveto{\pgfqpoint{2.647584in}{1.441548in}}{\pgfqpoint{2.650857in}{1.433648in}}{\pgfqpoint{2.656681in}{1.427824in}}%
\pgfpathcurveto{\pgfqpoint{2.662505in}{1.422000in}}{\pgfqpoint{2.670405in}{1.418728in}}{\pgfqpoint{2.678641in}{1.418728in}}%
\pgfpathclose%
\pgfusepath{stroke,fill}%
\end{pgfscope}%
\begin{pgfscope}%
\pgfpathrectangle{\pgfqpoint{0.100000in}{0.220728in}}{\pgfqpoint{3.696000in}{3.696000in}}%
\pgfusepath{clip}%
\pgfsetbuttcap%
\pgfsetroundjoin%
\definecolor{currentfill}{rgb}{0.121569,0.466667,0.705882}%
\pgfsetfillcolor{currentfill}%
\pgfsetfillopacity{0.911776}%
\pgfsetlinewidth{1.003750pt}%
\definecolor{currentstroke}{rgb}{0.121569,0.466667,0.705882}%
\pgfsetstrokecolor{currentstroke}%
\pgfsetstrokeopacity{0.911776}%
\pgfsetdash{}{0pt}%
\pgfpathmoveto{\pgfqpoint{1.887392in}{0.902569in}}%
\pgfpathcurveto{\pgfqpoint{1.895628in}{0.902569in}}{\pgfqpoint{1.903528in}{0.905841in}}{\pgfqpoint{1.909352in}{0.911665in}}%
\pgfpathcurveto{\pgfqpoint{1.915176in}{0.917489in}}{\pgfqpoint{1.918448in}{0.925389in}}{\pgfqpoint{1.918448in}{0.933625in}}%
\pgfpathcurveto{\pgfqpoint{1.918448in}{0.941862in}}{\pgfqpoint{1.915176in}{0.949762in}}{\pgfqpoint{1.909352in}{0.955586in}}%
\pgfpathcurveto{\pgfqpoint{1.903528in}{0.961410in}}{\pgfqpoint{1.895628in}{0.964682in}}{\pgfqpoint{1.887392in}{0.964682in}}%
\pgfpathcurveto{\pgfqpoint{1.879155in}{0.964682in}}{\pgfqpoint{1.871255in}{0.961410in}}{\pgfqpoint{1.865431in}{0.955586in}}%
\pgfpathcurveto{\pgfqpoint{1.859607in}{0.949762in}}{\pgfqpoint{1.856335in}{0.941862in}}{\pgfqpoint{1.856335in}{0.933625in}}%
\pgfpathcurveto{\pgfqpoint{1.856335in}{0.925389in}}{\pgfqpoint{1.859607in}{0.917489in}}{\pgfqpoint{1.865431in}{0.911665in}}%
\pgfpathcurveto{\pgfqpoint{1.871255in}{0.905841in}}{\pgfqpoint{1.879155in}{0.902569in}}{\pgfqpoint{1.887392in}{0.902569in}}%
\pgfpathclose%
\pgfusepath{stroke,fill}%
\end{pgfscope}%
\begin{pgfscope}%
\pgfpathrectangle{\pgfqpoint{0.100000in}{0.220728in}}{\pgfqpoint{3.696000in}{3.696000in}}%
\pgfusepath{clip}%
\pgfsetbuttcap%
\pgfsetroundjoin%
\definecolor{currentfill}{rgb}{0.121569,0.466667,0.705882}%
\pgfsetfillcolor{currentfill}%
\pgfsetfillopacity{0.912217}%
\pgfsetlinewidth{1.003750pt}%
\definecolor{currentstroke}{rgb}{0.121569,0.466667,0.705882}%
\pgfsetstrokecolor{currentstroke}%
\pgfsetstrokeopacity{0.912217}%
\pgfsetdash{}{0pt}%
\pgfpathmoveto{\pgfqpoint{2.676586in}{1.413632in}}%
\pgfpathcurveto{\pgfqpoint{2.684822in}{1.413632in}}{\pgfqpoint{2.692722in}{1.416904in}}{\pgfqpoint{2.698546in}{1.422728in}}%
\pgfpathcurveto{\pgfqpoint{2.704370in}{1.428552in}}{\pgfqpoint{2.707642in}{1.436452in}}{\pgfqpoint{2.707642in}{1.444688in}}%
\pgfpathcurveto{\pgfqpoint{2.707642in}{1.452925in}}{\pgfqpoint{2.704370in}{1.460825in}}{\pgfqpoint{2.698546in}{1.466649in}}%
\pgfpathcurveto{\pgfqpoint{2.692722in}{1.472473in}}{\pgfqpoint{2.684822in}{1.475745in}}{\pgfqpoint{2.676586in}{1.475745in}}%
\pgfpathcurveto{\pgfqpoint{2.668349in}{1.475745in}}{\pgfqpoint{2.660449in}{1.472473in}}{\pgfqpoint{2.654625in}{1.466649in}}%
\pgfpathcurveto{\pgfqpoint{2.648801in}{1.460825in}}{\pgfqpoint{2.645529in}{1.452925in}}{\pgfqpoint{2.645529in}{1.444688in}}%
\pgfpathcurveto{\pgfqpoint{2.645529in}{1.436452in}}{\pgfqpoint{2.648801in}{1.428552in}}{\pgfqpoint{2.654625in}{1.422728in}}%
\pgfpathcurveto{\pgfqpoint{2.660449in}{1.416904in}}{\pgfqpoint{2.668349in}{1.413632in}}{\pgfqpoint{2.676586in}{1.413632in}}%
\pgfpathclose%
\pgfusepath{stroke,fill}%
\end{pgfscope}%
\begin{pgfscope}%
\pgfpathrectangle{\pgfqpoint{0.100000in}{0.220728in}}{\pgfqpoint{3.696000in}{3.696000in}}%
\pgfusepath{clip}%
\pgfsetbuttcap%
\pgfsetroundjoin%
\definecolor{currentfill}{rgb}{0.121569,0.466667,0.705882}%
\pgfsetfillcolor{currentfill}%
\pgfsetfillopacity{0.912658}%
\pgfsetlinewidth{1.003750pt}%
\definecolor{currentstroke}{rgb}{0.121569,0.466667,0.705882}%
\pgfsetstrokecolor{currentstroke}%
\pgfsetstrokeopacity{0.912658}%
\pgfsetdash{}{0pt}%
\pgfpathmoveto{\pgfqpoint{2.675180in}{1.411470in}}%
\pgfpathcurveto{\pgfqpoint{2.683416in}{1.411470in}}{\pgfqpoint{2.691316in}{1.414743in}}{\pgfqpoint{2.697140in}{1.420567in}}%
\pgfpathcurveto{\pgfqpoint{2.702964in}{1.426391in}}{\pgfqpoint{2.706236in}{1.434291in}}{\pgfqpoint{2.706236in}{1.442527in}}%
\pgfpathcurveto{\pgfqpoint{2.706236in}{1.450763in}}{\pgfqpoint{2.702964in}{1.458663in}}{\pgfqpoint{2.697140in}{1.464487in}}%
\pgfpathcurveto{\pgfqpoint{2.691316in}{1.470311in}}{\pgfqpoint{2.683416in}{1.473583in}}{\pgfqpoint{2.675180in}{1.473583in}}%
\pgfpathcurveto{\pgfqpoint{2.666943in}{1.473583in}}{\pgfqpoint{2.659043in}{1.470311in}}{\pgfqpoint{2.653220in}{1.464487in}}%
\pgfpathcurveto{\pgfqpoint{2.647396in}{1.458663in}}{\pgfqpoint{2.644123in}{1.450763in}}{\pgfqpoint{2.644123in}{1.442527in}}%
\pgfpathcurveto{\pgfqpoint{2.644123in}{1.434291in}}{\pgfqpoint{2.647396in}{1.426391in}}{\pgfqpoint{2.653220in}{1.420567in}}%
\pgfpathcurveto{\pgfqpoint{2.659043in}{1.414743in}}{\pgfqpoint{2.666943in}{1.411470in}}{\pgfqpoint{2.675180in}{1.411470in}}%
\pgfpathclose%
\pgfusepath{stroke,fill}%
\end{pgfscope}%
\begin{pgfscope}%
\pgfpathrectangle{\pgfqpoint{0.100000in}{0.220728in}}{\pgfqpoint{3.696000in}{3.696000in}}%
\pgfusepath{clip}%
\pgfsetbuttcap%
\pgfsetroundjoin%
\definecolor{currentfill}{rgb}{0.121569,0.466667,0.705882}%
\pgfsetfillcolor{currentfill}%
\pgfsetfillopacity{0.913311}%
\pgfsetlinewidth{1.003750pt}%
\definecolor{currentstroke}{rgb}{0.121569,0.466667,0.705882}%
\pgfsetstrokecolor{currentstroke}%
\pgfsetstrokeopacity{0.913311}%
\pgfsetdash{}{0pt}%
\pgfpathmoveto{\pgfqpoint{2.673838in}{1.406005in}}%
\pgfpathcurveto{\pgfqpoint{2.682074in}{1.406005in}}{\pgfqpoint{2.689974in}{1.409277in}}{\pgfqpoint{2.695798in}{1.415101in}}%
\pgfpathcurveto{\pgfqpoint{2.701622in}{1.420925in}}{\pgfqpoint{2.704894in}{1.428825in}}{\pgfqpoint{2.704894in}{1.437061in}}%
\pgfpathcurveto{\pgfqpoint{2.704894in}{1.445298in}}{\pgfqpoint{2.701622in}{1.453198in}}{\pgfqpoint{2.695798in}{1.459021in}}%
\pgfpathcurveto{\pgfqpoint{2.689974in}{1.464845in}}{\pgfqpoint{2.682074in}{1.468118in}}{\pgfqpoint{2.673838in}{1.468118in}}%
\pgfpathcurveto{\pgfqpoint{2.665602in}{1.468118in}}{\pgfqpoint{2.657702in}{1.464845in}}{\pgfqpoint{2.651878in}{1.459021in}}%
\pgfpathcurveto{\pgfqpoint{2.646054in}{1.453198in}}{\pgfqpoint{2.642781in}{1.445298in}}{\pgfqpoint{2.642781in}{1.437061in}}%
\pgfpathcurveto{\pgfqpoint{2.642781in}{1.428825in}}{\pgfqpoint{2.646054in}{1.420925in}}{\pgfqpoint{2.651878in}{1.415101in}}%
\pgfpathcurveto{\pgfqpoint{2.657702in}{1.409277in}}{\pgfqpoint{2.665602in}{1.406005in}}{\pgfqpoint{2.673838in}{1.406005in}}%
\pgfpathclose%
\pgfusepath{stroke,fill}%
\end{pgfscope}%
\begin{pgfscope}%
\pgfpathrectangle{\pgfqpoint{0.100000in}{0.220728in}}{\pgfqpoint{3.696000in}{3.696000in}}%
\pgfusepath{clip}%
\pgfsetbuttcap%
\pgfsetroundjoin%
\definecolor{currentfill}{rgb}{0.121569,0.466667,0.705882}%
\pgfsetfillcolor{currentfill}%
\pgfsetfillopacity{0.913746}%
\pgfsetlinewidth{1.003750pt}%
\definecolor{currentstroke}{rgb}{0.121569,0.466667,0.705882}%
\pgfsetstrokecolor{currentstroke}%
\pgfsetstrokeopacity{0.913746}%
\pgfsetdash{}{0pt}%
\pgfpathmoveto{\pgfqpoint{1.900572in}{0.895762in}}%
\pgfpathcurveto{\pgfqpoint{1.908808in}{0.895762in}}{\pgfqpoint{1.916708in}{0.899034in}}{\pgfqpoint{1.922532in}{0.904858in}}%
\pgfpathcurveto{\pgfqpoint{1.928356in}{0.910682in}}{\pgfqpoint{1.931629in}{0.918582in}}{\pgfqpoint{1.931629in}{0.926818in}}%
\pgfpathcurveto{\pgfqpoint{1.931629in}{0.935054in}}{\pgfqpoint{1.928356in}{0.942954in}}{\pgfqpoint{1.922532in}{0.948778in}}%
\pgfpathcurveto{\pgfqpoint{1.916708in}{0.954602in}}{\pgfqpoint{1.908808in}{0.957875in}}{\pgfqpoint{1.900572in}{0.957875in}}%
\pgfpathcurveto{\pgfqpoint{1.892336in}{0.957875in}}{\pgfqpoint{1.884436in}{0.954602in}}{\pgfqpoint{1.878612in}{0.948778in}}%
\pgfpathcurveto{\pgfqpoint{1.872788in}{0.942954in}}{\pgfqpoint{1.869516in}{0.935054in}}{\pgfqpoint{1.869516in}{0.926818in}}%
\pgfpathcurveto{\pgfqpoint{1.869516in}{0.918582in}}{\pgfqpoint{1.872788in}{0.910682in}}{\pgfqpoint{1.878612in}{0.904858in}}%
\pgfpathcurveto{\pgfqpoint{1.884436in}{0.899034in}}{\pgfqpoint{1.892336in}{0.895762in}}{\pgfqpoint{1.900572in}{0.895762in}}%
\pgfpathclose%
\pgfusepath{stroke,fill}%
\end{pgfscope}%
\begin{pgfscope}%
\pgfpathrectangle{\pgfqpoint{0.100000in}{0.220728in}}{\pgfqpoint{3.696000in}{3.696000in}}%
\pgfusepath{clip}%
\pgfsetbuttcap%
\pgfsetroundjoin%
\definecolor{currentfill}{rgb}{0.121569,0.466667,0.705882}%
\pgfsetfillcolor{currentfill}%
\pgfsetfillopacity{0.914116}%
\pgfsetlinewidth{1.003750pt}%
\definecolor{currentstroke}{rgb}{0.121569,0.466667,0.705882}%
\pgfsetstrokecolor{currentstroke}%
\pgfsetstrokeopacity{0.914116}%
\pgfsetdash{}{0pt}%
\pgfpathmoveto{\pgfqpoint{2.670201in}{1.399172in}}%
\pgfpathcurveto{\pgfqpoint{2.678437in}{1.399172in}}{\pgfqpoint{2.686337in}{1.402445in}}{\pgfqpoint{2.692161in}{1.408269in}}%
\pgfpathcurveto{\pgfqpoint{2.697985in}{1.414093in}}{\pgfqpoint{2.701257in}{1.421993in}}{\pgfqpoint{2.701257in}{1.430229in}}%
\pgfpathcurveto{\pgfqpoint{2.701257in}{1.438465in}}{\pgfqpoint{2.697985in}{1.446365in}}{\pgfqpoint{2.692161in}{1.452189in}}%
\pgfpathcurveto{\pgfqpoint{2.686337in}{1.458013in}}{\pgfqpoint{2.678437in}{1.461285in}}{\pgfqpoint{2.670201in}{1.461285in}}%
\pgfpathcurveto{\pgfqpoint{2.661965in}{1.461285in}}{\pgfqpoint{2.654065in}{1.458013in}}{\pgfqpoint{2.648241in}{1.452189in}}%
\pgfpathcurveto{\pgfqpoint{2.642417in}{1.446365in}}{\pgfqpoint{2.639144in}{1.438465in}}{\pgfqpoint{2.639144in}{1.430229in}}%
\pgfpathcurveto{\pgfqpoint{2.639144in}{1.421993in}}{\pgfqpoint{2.642417in}{1.414093in}}{\pgfqpoint{2.648241in}{1.408269in}}%
\pgfpathcurveto{\pgfqpoint{2.654065in}{1.402445in}}{\pgfqpoint{2.661965in}{1.399172in}}{\pgfqpoint{2.670201in}{1.399172in}}%
\pgfpathclose%
\pgfusepath{stroke,fill}%
\end{pgfscope}%
\begin{pgfscope}%
\pgfpathrectangle{\pgfqpoint{0.100000in}{0.220728in}}{\pgfqpoint{3.696000in}{3.696000in}}%
\pgfusepath{clip}%
\pgfsetbuttcap%
\pgfsetroundjoin%
\definecolor{currentfill}{rgb}{0.121569,0.466667,0.705882}%
\pgfsetfillcolor{currentfill}%
\pgfsetfillopacity{0.915131}%
\pgfsetlinewidth{1.003750pt}%
\definecolor{currentstroke}{rgb}{0.121569,0.466667,0.705882}%
\pgfsetstrokecolor{currentstroke}%
\pgfsetstrokeopacity{0.915131}%
\pgfsetdash{}{0pt}%
\pgfpathmoveto{\pgfqpoint{2.665884in}{1.390140in}}%
\pgfpathcurveto{\pgfqpoint{2.674120in}{1.390140in}}{\pgfqpoint{2.682020in}{1.393412in}}{\pgfqpoint{2.687844in}{1.399236in}}%
\pgfpathcurveto{\pgfqpoint{2.693668in}{1.405060in}}{\pgfqpoint{2.696940in}{1.412960in}}{\pgfqpoint{2.696940in}{1.421197in}}%
\pgfpathcurveto{\pgfqpoint{2.696940in}{1.429433in}}{\pgfqpoint{2.693668in}{1.437333in}}{\pgfqpoint{2.687844in}{1.443157in}}%
\pgfpathcurveto{\pgfqpoint{2.682020in}{1.448981in}}{\pgfqpoint{2.674120in}{1.452253in}}{\pgfqpoint{2.665884in}{1.452253in}}%
\pgfpathcurveto{\pgfqpoint{2.657648in}{1.452253in}}{\pgfqpoint{2.649748in}{1.448981in}}{\pgfqpoint{2.643924in}{1.443157in}}%
\pgfpathcurveto{\pgfqpoint{2.638100in}{1.437333in}}{\pgfqpoint{2.634827in}{1.429433in}}{\pgfqpoint{2.634827in}{1.421197in}}%
\pgfpathcurveto{\pgfqpoint{2.634827in}{1.412960in}}{\pgfqpoint{2.638100in}{1.405060in}}{\pgfqpoint{2.643924in}{1.399236in}}%
\pgfpathcurveto{\pgfqpoint{2.649748in}{1.393412in}}{\pgfqpoint{2.657648in}{1.390140in}}{\pgfqpoint{2.665884in}{1.390140in}}%
\pgfpathclose%
\pgfusepath{stroke,fill}%
\end{pgfscope}%
\begin{pgfscope}%
\pgfpathrectangle{\pgfqpoint{0.100000in}{0.220728in}}{\pgfqpoint{3.696000in}{3.696000in}}%
\pgfusepath{clip}%
\pgfsetbuttcap%
\pgfsetroundjoin%
\definecolor{currentfill}{rgb}{0.121569,0.466667,0.705882}%
\pgfsetfillcolor{currentfill}%
\pgfsetfillopacity{0.916219}%
\pgfsetlinewidth{1.003750pt}%
\definecolor{currentstroke}{rgb}{0.121569,0.466667,0.705882}%
\pgfsetstrokecolor{currentstroke}%
\pgfsetstrokeopacity{0.916219}%
\pgfsetdash{}{0pt}%
\pgfpathmoveto{\pgfqpoint{1.911550in}{0.893919in}}%
\pgfpathcurveto{\pgfqpoint{1.919787in}{0.893919in}}{\pgfqpoint{1.927687in}{0.897191in}}{\pgfqpoint{1.933511in}{0.903015in}}%
\pgfpathcurveto{\pgfqpoint{1.939335in}{0.908839in}}{\pgfqpoint{1.942607in}{0.916739in}}{\pgfqpoint{1.942607in}{0.924975in}}%
\pgfpathcurveto{\pgfqpoint{1.942607in}{0.933212in}}{\pgfqpoint{1.939335in}{0.941112in}}{\pgfqpoint{1.933511in}{0.946936in}}%
\pgfpathcurveto{\pgfqpoint{1.927687in}{0.952760in}}{\pgfqpoint{1.919787in}{0.956032in}}{\pgfqpoint{1.911550in}{0.956032in}}%
\pgfpathcurveto{\pgfqpoint{1.903314in}{0.956032in}}{\pgfqpoint{1.895414in}{0.952760in}}{\pgfqpoint{1.889590in}{0.946936in}}%
\pgfpathcurveto{\pgfqpoint{1.883766in}{0.941112in}}{\pgfqpoint{1.880494in}{0.933212in}}{\pgfqpoint{1.880494in}{0.924975in}}%
\pgfpathcurveto{\pgfqpoint{1.880494in}{0.916739in}}{\pgfqpoint{1.883766in}{0.908839in}}{\pgfqpoint{1.889590in}{0.903015in}}%
\pgfpathcurveto{\pgfqpoint{1.895414in}{0.897191in}}{\pgfqpoint{1.903314in}{0.893919in}}{\pgfqpoint{1.911550in}{0.893919in}}%
\pgfpathclose%
\pgfusepath{stroke,fill}%
\end{pgfscope}%
\begin{pgfscope}%
\pgfpathrectangle{\pgfqpoint{0.100000in}{0.220728in}}{\pgfqpoint{3.696000in}{3.696000in}}%
\pgfusepath{clip}%
\pgfsetbuttcap%
\pgfsetroundjoin%
\definecolor{currentfill}{rgb}{0.121569,0.466667,0.705882}%
\pgfsetfillcolor{currentfill}%
\pgfsetfillopacity{0.916515}%
\pgfsetlinewidth{1.003750pt}%
\definecolor{currentstroke}{rgb}{0.121569,0.466667,0.705882}%
\pgfsetstrokecolor{currentstroke}%
\pgfsetstrokeopacity{0.916515}%
\pgfsetdash{}{0pt}%
\pgfpathmoveto{\pgfqpoint{2.662560in}{1.379924in}}%
\pgfpathcurveto{\pgfqpoint{2.670796in}{1.379924in}}{\pgfqpoint{2.678696in}{1.383196in}}{\pgfqpoint{2.684520in}{1.389020in}}%
\pgfpathcurveto{\pgfqpoint{2.690344in}{1.394844in}}{\pgfqpoint{2.693616in}{1.402744in}}{\pgfqpoint{2.693616in}{1.410980in}}%
\pgfpathcurveto{\pgfqpoint{2.693616in}{1.419217in}}{\pgfqpoint{2.690344in}{1.427117in}}{\pgfqpoint{2.684520in}{1.432941in}}%
\pgfpathcurveto{\pgfqpoint{2.678696in}{1.438764in}}{\pgfqpoint{2.670796in}{1.442037in}}{\pgfqpoint{2.662560in}{1.442037in}}%
\pgfpathcurveto{\pgfqpoint{2.654324in}{1.442037in}}{\pgfqpoint{2.646424in}{1.438764in}}{\pgfqpoint{2.640600in}{1.432941in}}%
\pgfpathcurveto{\pgfqpoint{2.634776in}{1.427117in}}{\pgfqpoint{2.631503in}{1.419217in}}{\pgfqpoint{2.631503in}{1.410980in}}%
\pgfpathcurveto{\pgfqpoint{2.631503in}{1.402744in}}{\pgfqpoint{2.634776in}{1.394844in}}{\pgfqpoint{2.640600in}{1.389020in}}%
\pgfpathcurveto{\pgfqpoint{2.646424in}{1.383196in}}{\pgfqpoint{2.654324in}{1.379924in}}{\pgfqpoint{2.662560in}{1.379924in}}%
\pgfpathclose%
\pgfusepath{stroke,fill}%
\end{pgfscope}%
\begin{pgfscope}%
\pgfpathrectangle{\pgfqpoint{0.100000in}{0.220728in}}{\pgfqpoint{3.696000in}{3.696000in}}%
\pgfusepath{clip}%
\pgfsetbuttcap%
\pgfsetroundjoin%
\definecolor{currentfill}{rgb}{0.121569,0.466667,0.705882}%
\pgfsetfillcolor{currentfill}%
\pgfsetfillopacity{0.917558}%
\pgfsetlinewidth{1.003750pt}%
\definecolor{currentstroke}{rgb}{0.121569,0.466667,0.705882}%
\pgfsetstrokecolor{currentstroke}%
\pgfsetstrokeopacity{0.917558}%
\pgfsetdash{}{0pt}%
\pgfpathmoveto{\pgfqpoint{1.920146in}{0.888819in}}%
\pgfpathcurveto{\pgfqpoint{1.928382in}{0.888819in}}{\pgfqpoint{1.936282in}{0.892091in}}{\pgfqpoint{1.942106in}{0.897915in}}%
\pgfpathcurveto{\pgfqpoint{1.947930in}{0.903739in}}{\pgfqpoint{1.951202in}{0.911639in}}{\pgfqpoint{1.951202in}{0.919876in}}%
\pgfpathcurveto{\pgfqpoint{1.951202in}{0.928112in}}{\pgfqpoint{1.947930in}{0.936012in}}{\pgfqpoint{1.942106in}{0.941836in}}%
\pgfpathcurveto{\pgfqpoint{1.936282in}{0.947660in}}{\pgfqpoint{1.928382in}{0.950932in}}{\pgfqpoint{1.920146in}{0.950932in}}%
\pgfpathcurveto{\pgfqpoint{1.911910in}{0.950932in}}{\pgfqpoint{1.904010in}{0.947660in}}{\pgfqpoint{1.898186in}{0.941836in}}%
\pgfpathcurveto{\pgfqpoint{1.892362in}{0.936012in}}{\pgfqpoint{1.889089in}{0.928112in}}{\pgfqpoint{1.889089in}{0.919876in}}%
\pgfpathcurveto{\pgfqpoint{1.889089in}{0.911639in}}{\pgfqpoint{1.892362in}{0.903739in}}{\pgfqpoint{1.898186in}{0.897915in}}%
\pgfpathcurveto{\pgfqpoint{1.904010in}{0.892091in}}{\pgfqpoint{1.911910in}{0.888819in}}{\pgfqpoint{1.920146in}{0.888819in}}%
\pgfpathclose%
\pgfusepath{stroke,fill}%
\end{pgfscope}%
\begin{pgfscope}%
\pgfpathrectangle{\pgfqpoint{0.100000in}{0.220728in}}{\pgfqpoint{3.696000in}{3.696000in}}%
\pgfusepath{clip}%
\pgfsetbuttcap%
\pgfsetroundjoin%
\definecolor{currentfill}{rgb}{0.121569,0.466667,0.705882}%
\pgfsetfillcolor{currentfill}%
\pgfsetfillopacity{0.917649}%
\pgfsetlinewidth{1.003750pt}%
\definecolor{currentstroke}{rgb}{0.121569,0.466667,0.705882}%
\pgfsetstrokecolor{currentstroke}%
\pgfsetstrokeopacity{0.917649}%
\pgfsetdash{}{0pt}%
\pgfpathmoveto{\pgfqpoint{2.653078in}{1.366577in}}%
\pgfpathcurveto{\pgfqpoint{2.661314in}{1.366577in}}{\pgfqpoint{2.669215in}{1.369850in}}{\pgfqpoint{2.675038in}{1.375674in}}%
\pgfpathcurveto{\pgfqpoint{2.680862in}{1.381498in}}{\pgfqpoint{2.684135in}{1.389398in}}{\pgfqpoint{2.684135in}{1.397634in}}%
\pgfpathcurveto{\pgfqpoint{2.684135in}{1.405870in}}{\pgfqpoint{2.680862in}{1.413770in}}{\pgfqpoint{2.675038in}{1.419594in}}%
\pgfpathcurveto{\pgfqpoint{2.669215in}{1.425418in}}{\pgfqpoint{2.661314in}{1.428690in}}{\pgfqpoint{2.653078in}{1.428690in}}%
\pgfpathcurveto{\pgfqpoint{2.644842in}{1.428690in}}{\pgfqpoint{2.636942in}{1.425418in}}{\pgfqpoint{2.631118in}{1.419594in}}%
\pgfpathcurveto{\pgfqpoint{2.625294in}{1.413770in}}{\pgfqpoint{2.622022in}{1.405870in}}{\pgfqpoint{2.622022in}{1.397634in}}%
\pgfpathcurveto{\pgfqpoint{2.622022in}{1.389398in}}{\pgfqpoint{2.625294in}{1.381498in}}{\pgfqpoint{2.631118in}{1.375674in}}%
\pgfpathcurveto{\pgfqpoint{2.636942in}{1.369850in}}{\pgfqpoint{2.644842in}{1.366577in}}{\pgfqpoint{2.653078in}{1.366577in}}%
\pgfpathclose%
\pgfusepath{stroke,fill}%
\end{pgfscope}%
\begin{pgfscope}%
\pgfpathrectangle{\pgfqpoint{0.100000in}{0.220728in}}{\pgfqpoint{3.696000in}{3.696000in}}%
\pgfusepath{clip}%
\pgfsetbuttcap%
\pgfsetroundjoin%
\definecolor{currentfill}{rgb}{0.121569,0.466667,0.705882}%
\pgfsetfillcolor{currentfill}%
\pgfsetfillopacity{0.919290}%
\pgfsetlinewidth{1.003750pt}%
\definecolor{currentstroke}{rgb}{0.121569,0.466667,0.705882}%
\pgfsetstrokecolor{currentstroke}%
\pgfsetstrokeopacity{0.919290}%
\pgfsetdash{}{0pt}%
\pgfpathmoveto{\pgfqpoint{1.928341in}{0.886509in}}%
\pgfpathcurveto{\pgfqpoint{1.936577in}{0.886509in}}{\pgfqpoint{1.944477in}{0.889782in}}{\pgfqpoint{1.950301in}{0.895606in}}%
\pgfpathcurveto{\pgfqpoint{1.956125in}{0.901429in}}{\pgfqpoint{1.959398in}{0.909330in}}{\pgfqpoint{1.959398in}{0.917566in}}%
\pgfpathcurveto{\pgfqpoint{1.959398in}{0.925802in}}{\pgfqpoint{1.956125in}{0.933702in}}{\pgfqpoint{1.950301in}{0.939526in}}%
\pgfpathcurveto{\pgfqpoint{1.944477in}{0.945350in}}{\pgfqpoint{1.936577in}{0.948622in}}{\pgfqpoint{1.928341in}{0.948622in}}%
\pgfpathcurveto{\pgfqpoint{1.920105in}{0.948622in}}{\pgfqpoint{1.912205in}{0.945350in}}{\pgfqpoint{1.906381in}{0.939526in}}%
\pgfpathcurveto{\pgfqpoint{1.900557in}{0.933702in}}{\pgfqpoint{1.897285in}{0.925802in}}{\pgfqpoint{1.897285in}{0.917566in}}%
\pgfpathcurveto{\pgfqpoint{1.897285in}{0.909330in}}{\pgfqpoint{1.900557in}{0.901429in}}{\pgfqpoint{1.906381in}{0.895606in}}%
\pgfpathcurveto{\pgfqpoint{1.912205in}{0.889782in}}{\pgfqpoint{1.920105in}{0.886509in}}{\pgfqpoint{1.928341in}{0.886509in}}%
\pgfpathclose%
\pgfusepath{stroke,fill}%
\end{pgfscope}%
\begin{pgfscope}%
\pgfpathrectangle{\pgfqpoint{0.100000in}{0.220728in}}{\pgfqpoint{3.696000in}{3.696000in}}%
\pgfusepath{clip}%
\pgfsetbuttcap%
\pgfsetroundjoin%
\definecolor{currentfill}{rgb}{0.121569,0.466667,0.705882}%
\pgfsetfillcolor{currentfill}%
\pgfsetfillopacity{0.920120}%
\pgfsetlinewidth{1.003750pt}%
\definecolor{currentstroke}{rgb}{0.121569,0.466667,0.705882}%
\pgfsetstrokecolor{currentstroke}%
\pgfsetstrokeopacity{0.920120}%
\pgfsetdash{}{0pt}%
\pgfpathmoveto{\pgfqpoint{2.648431in}{1.347658in}}%
\pgfpathcurveto{\pgfqpoint{2.656667in}{1.347658in}}{\pgfqpoint{2.664567in}{1.350930in}}{\pgfqpoint{2.670391in}{1.356754in}}%
\pgfpathcurveto{\pgfqpoint{2.676215in}{1.362578in}}{\pgfqpoint{2.679487in}{1.370478in}}{\pgfqpoint{2.679487in}{1.378715in}}%
\pgfpathcurveto{\pgfqpoint{2.679487in}{1.386951in}}{\pgfqpoint{2.676215in}{1.394851in}}{\pgfqpoint{2.670391in}{1.400675in}}%
\pgfpathcurveto{\pgfqpoint{2.664567in}{1.406499in}}{\pgfqpoint{2.656667in}{1.409771in}}{\pgfqpoint{2.648431in}{1.409771in}}%
\pgfpathcurveto{\pgfqpoint{2.640194in}{1.409771in}}{\pgfqpoint{2.632294in}{1.406499in}}{\pgfqpoint{2.626470in}{1.400675in}}%
\pgfpathcurveto{\pgfqpoint{2.620646in}{1.394851in}}{\pgfqpoint{2.617374in}{1.386951in}}{\pgfqpoint{2.617374in}{1.378715in}}%
\pgfpathcurveto{\pgfqpoint{2.617374in}{1.370478in}}{\pgfqpoint{2.620646in}{1.362578in}}{\pgfqpoint{2.626470in}{1.356754in}}%
\pgfpathcurveto{\pgfqpoint{2.632294in}{1.350930in}}{\pgfqpoint{2.640194in}{1.347658in}}{\pgfqpoint{2.648431in}{1.347658in}}%
\pgfpathclose%
\pgfusepath{stroke,fill}%
\end{pgfscope}%
\begin{pgfscope}%
\pgfpathrectangle{\pgfqpoint{0.100000in}{0.220728in}}{\pgfqpoint{3.696000in}{3.696000in}}%
\pgfusepath{clip}%
\pgfsetbuttcap%
\pgfsetroundjoin%
\definecolor{currentfill}{rgb}{0.121569,0.466667,0.705882}%
\pgfsetfillcolor{currentfill}%
\pgfsetfillopacity{0.921840}%
\pgfsetlinewidth{1.003750pt}%
\definecolor{currentstroke}{rgb}{0.121569,0.466667,0.705882}%
\pgfsetstrokecolor{currentstroke}%
\pgfsetstrokeopacity{0.921840}%
\pgfsetdash{}{0pt}%
\pgfpathmoveto{\pgfqpoint{2.637930in}{1.328777in}}%
\pgfpathcurveto{\pgfqpoint{2.646166in}{1.328777in}}{\pgfqpoint{2.654067in}{1.332050in}}{\pgfqpoint{2.659890in}{1.337874in}}%
\pgfpathcurveto{\pgfqpoint{2.665714in}{1.343698in}}{\pgfqpoint{2.668987in}{1.351598in}}{\pgfqpoint{2.668987in}{1.359834in}}%
\pgfpathcurveto{\pgfqpoint{2.668987in}{1.368070in}}{\pgfqpoint{2.665714in}{1.375970in}}{\pgfqpoint{2.659890in}{1.381794in}}%
\pgfpathcurveto{\pgfqpoint{2.654067in}{1.387618in}}{\pgfqpoint{2.646166in}{1.390890in}}{\pgfqpoint{2.637930in}{1.390890in}}%
\pgfpathcurveto{\pgfqpoint{2.629694in}{1.390890in}}{\pgfqpoint{2.621794in}{1.387618in}}{\pgfqpoint{2.615970in}{1.381794in}}%
\pgfpathcurveto{\pgfqpoint{2.610146in}{1.375970in}}{\pgfqpoint{2.606874in}{1.368070in}}{\pgfqpoint{2.606874in}{1.359834in}}%
\pgfpathcurveto{\pgfqpoint{2.606874in}{1.351598in}}{\pgfqpoint{2.610146in}{1.343698in}}{\pgfqpoint{2.615970in}{1.337874in}}%
\pgfpathcurveto{\pgfqpoint{2.621794in}{1.332050in}}{\pgfqpoint{2.629694in}{1.328777in}}{\pgfqpoint{2.637930in}{1.328777in}}%
\pgfpathclose%
\pgfusepath{stroke,fill}%
\end{pgfscope}%
\begin{pgfscope}%
\pgfpathrectangle{\pgfqpoint{0.100000in}{0.220728in}}{\pgfqpoint{3.696000in}{3.696000in}}%
\pgfusepath{clip}%
\pgfsetbuttcap%
\pgfsetroundjoin%
\definecolor{currentfill}{rgb}{0.121569,0.466667,0.705882}%
\pgfsetfillcolor{currentfill}%
\pgfsetfillopacity{0.922093}%
\pgfsetlinewidth{1.003750pt}%
\definecolor{currentstroke}{rgb}{0.121569,0.466667,0.705882}%
\pgfsetstrokecolor{currentstroke}%
\pgfsetstrokeopacity{0.922093}%
\pgfsetdash{}{0pt}%
\pgfpathmoveto{\pgfqpoint{1.942716in}{0.879144in}}%
\pgfpathcurveto{\pgfqpoint{1.950952in}{0.879144in}}{\pgfqpoint{1.958852in}{0.882416in}}{\pgfqpoint{1.964676in}{0.888240in}}%
\pgfpathcurveto{\pgfqpoint{1.970500in}{0.894064in}}{\pgfqpoint{1.973772in}{0.901964in}}{\pgfqpoint{1.973772in}{0.910200in}}%
\pgfpathcurveto{\pgfqpoint{1.973772in}{0.918437in}}{\pgfqpoint{1.970500in}{0.926337in}}{\pgfqpoint{1.964676in}{0.932161in}}%
\pgfpathcurveto{\pgfqpoint{1.958852in}{0.937985in}}{\pgfqpoint{1.950952in}{0.941257in}}{\pgfqpoint{1.942716in}{0.941257in}}%
\pgfpathcurveto{\pgfqpoint{1.934480in}{0.941257in}}{\pgfqpoint{1.926580in}{0.937985in}}{\pgfqpoint{1.920756in}{0.932161in}}%
\pgfpathcurveto{\pgfqpoint{1.914932in}{0.926337in}}{\pgfqpoint{1.911659in}{0.918437in}}{\pgfqpoint{1.911659in}{0.910200in}}%
\pgfpathcurveto{\pgfqpoint{1.911659in}{0.901964in}}{\pgfqpoint{1.914932in}{0.894064in}}{\pgfqpoint{1.920756in}{0.888240in}}%
\pgfpathcurveto{\pgfqpoint{1.926580in}{0.882416in}}{\pgfqpoint{1.934480in}{0.879144in}}{\pgfqpoint{1.942716in}{0.879144in}}%
\pgfpathclose%
\pgfusepath{stroke,fill}%
\end{pgfscope}%
\begin{pgfscope}%
\pgfpathrectangle{\pgfqpoint{0.100000in}{0.220728in}}{\pgfqpoint{3.696000in}{3.696000in}}%
\pgfusepath{clip}%
\pgfsetbuttcap%
\pgfsetroundjoin%
\definecolor{currentfill}{rgb}{0.121569,0.466667,0.705882}%
\pgfsetfillcolor{currentfill}%
\pgfsetfillopacity{0.923345}%
\pgfsetlinewidth{1.003750pt}%
\definecolor{currentstroke}{rgb}{0.121569,0.466667,0.705882}%
\pgfsetstrokecolor{currentstroke}%
\pgfsetstrokeopacity{0.923345}%
\pgfsetdash{}{0pt}%
\pgfpathmoveto{\pgfqpoint{2.632611in}{1.319876in}}%
\pgfpathcurveto{\pgfqpoint{2.640847in}{1.319876in}}{\pgfqpoint{2.648747in}{1.323148in}}{\pgfqpoint{2.654571in}{1.328972in}}%
\pgfpathcurveto{\pgfqpoint{2.660395in}{1.334796in}}{\pgfqpoint{2.663667in}{1.342696in}}{\pgfqpoint{2.663667in}{1.350932in}}%
\pgfpathcurveto{\pgfqpoint{2.663667in}{1.359169in}}{\pgfqpoint{2.660395in}{1.367069in}}{\pgfqpoint{2.654571in}{1.372893in}}%
\pgfpathcurveto{\pgfqpoint{2.648747in}{1.378717in}}{\pgfqpoint{2.640847in}{1.381989in}}{\pgfqpoint{2.632611in}{1.381989in}}%
\pgfpathcurveto{\pgfqpoint{2.624375in}{1.381989in}}{\pgfqpoint{2.616475in}{1.378717in}}{\pgfqpoint{2.610651in}{1.372893in}}%
\pgfpathcurveto{\pgfqpoint{2.604827in}{1.367069in}}{\pgfqpoint{2.601554in}{1.359169in}}{\pgfqpoint{2.601554in}{1.350932in}}%
\pgfpathcurveto{\pgfqpoint{2.601554in}{1.342696in}}{\pgfqpoint{2.604827in}{1.334796in}}{\pgfqpoint{2.610651in}{1.328972in}}%
\pgfpathcurveto{\pgfqpoint{2.616475in}{1.323148in}}{\pgfqpoint{2.624375in}{1.319876in}}{\pgfqpoint{2.632611in}{1.319876in}}%
\pgfpathclose%
\pgfusepath{stroke,fill}%
\end{pgfscope}%
\begin{pgfscope}%
\pgfpathrectangle{\pgfqpoint{0.100000in}{0.220728in}}{\pgfqpoint{3.696000in}{3.696000in}}%
\pgfusepath{clip}%
\pgfsetbuttcap%
\pgfsetroundjoin%
\definecolor{currentfill}{rgb}{0.121569,0.466667,0.705882}%
\pgfsetfillcolor{currentfill}%
\pgfsetfillopacity{0.924186}%
\pgfsetlinewidth{1.003750pt}%
\definecolor{currentstroke}{rgb}{0.121569,0.466667,0.705882}%
\pgfsetstrokecolor{currentstroke}%
\pgfsetstrokeopacity{0.924186}%
\pgfsetdash{}{0pt}%
\pgfpathmoveto{\pgfqpoint{1.954784in}{0.871728in}}%
\pgfpathcurveto{\pgfqpoint{1.963020in}{0.871728in}}{\pgfqpoint{1.970920in}{0.875000in}}{\pgfqpoint{1.976744in}{0.880824in}}%
\pgfpathcurveto{\pgfqpoint{1.982568in}{0.886648in}}{\pgfqpoint{1.985840in}{0.894548in}}{\pgfqpoint{1.985840in}{0.902784in}}%
\pgfpathcurveto{\pgfqpoint{1.985840in}{0.911020in}}{\pgfqpoint{1.982568in}{0.918920in}}{\pgfqpoint{1.976744in}{0.924744in}}%
\pgfpathcurveto{\pgfqpoint{1.970920in}{0.930568in}}{\pgfqpoint{1.963020in}{0.933841in}}{\pgfqpoint{1.954784in}{0.933841in}}%
\pgfpathcurveto{\pgfqpoint{1.946548in}{0.933841in}}{\pgfqpoint{1.938648in}{0.930568in}}{\pgfqpoint{1.932824in}{0.924744in}}%
\pgfpathcurveto{\pgfqpoint{1.927000in}{0.918920in}}{\pgfqpoint{1.923727in}{0.911020in}}{\pgfqpoint{1.923727in}{0.902784in}}%
\pgfpathcurveto{\pgfqpoint{1.923727in}{0.894548in}}{\pgfqpoint{1.927000in}{0.886648in}}{\pgfqpoint{1.932824in}{0.880824in}}%
\pgfpathcurveto{\pgfqpoint{1.938648in}{0.875000in}}{\pgfqpoint{1.946548in}{0.871728in}}{\pgfqpoint{1.954784in}{0.871728in}}%
\pgfpathclose%
\pgfusepath{stroke,fill}%
\end{pgfscope}%
\begin{pgfscope}%
\pgfpathrectangle{\pgfqpoint{0.100000in}{0.220728in}}{\pgfqpoint{3.696000in}{3.696000in}}%
\pgfusepath{clip}%
\pgfsetbuttcap%
\pgfsetroundjoin%
\definecolor{currentfill}{rgb}{0.121569,0.466667,0.705882}%
\pgfsetfillcolor{currentfill}%
\pgfsetfillopacity{0.924701}%
\pgfsetlinewidth{1.003750pt}%
\definecolor{currentstroke}{rgb}{0.121569,0.466667,0.705882}%
\pgfsetstrokecolor{currentstroke}%
\pgfsetstrokeopacity{0.924701}%
\pgfsetdash{}{0pt}%
\pgfpathmoveto{\pgfqpoint{2.628870in}{1.306030in}}%
\pgfpathcurveto{\pgfqpoint{2.637107in}{1.306030in}}{\pgfqpoint{2.645007in}{1.309303in}}{\pgfqpoint{2.650831in}{1.315127in}}%
\pgfpathcurveto{\pgfqpoint{2.656655in}{1.320950in}}{\pgfqpoint{2.659927in}{1.328850in}}{\pgfqpoint{2.659927in}{1.337087in}}%
\pgfpathcurveto{\pgfqpoint{2.659927in}{1.345323in}}{\pgfqpoint{2.656655in}{1.353223in}}{\pgfqpoint{2.650831in}{1.359047in}}%
\pgfpathcurveto{\pgfqpoint{2.645007in}{1.364871in}}{\pgfqpoint{2.637107in}{1.368143in}}{\pgfqpoint{2.628870in}{1.368143in}}%
\pgfpathcurveto{\pgfqpoint{2.620634in}{1.368143in}}{\pgfqpoint{2.612734in}{1.364871in}}{\pgfqpoint{2.606910in}{1.359047in}}%
\pgfpathcurveto{\pgfqpoint{2.601086in}{1.353223in}}{\pgfqpoint{2.597814in}{1.345323in}}{\pgfqpoint{2.597814in}{1.337087in}}%
\pgfpathcurveto{\pgfqpoint{2.597814in}{1.328850in}}{\pgfqpoint{2.601086in}{1.320950in}}{\pgfqpoint{2.606910in}{1.315127in}}%
\pgfpathcurveto{\pgfqpoint{2.612734in}{1.309303in}}{\pgfqpoint{2.620634in}{1.306030in}}{\pgfqpoint{2.628870in}{1.306030in}}%
\pgfpathclose%
\pgfusepath{stroke,fill}%
\end{pgfscope}%
\begin{pgfscope}%
\pgfpathrectangle{\pgfqpoint{0.100000in}{0.220728in}}{\pgfqpoint{3.696000in}{3.696000in}}%
\pgfusepath{clip}%
\pgfsetbuttcap%
\pgfsetroundjoin%
\definecolor{currentfill}{rgb}{0.121569,0.466667,0.705882}%
\pgfsetfillcolor{currentfill}%
\pgfsetfillopacity{0.926592}%
\pgfsetlinewidth{1.003750pt}%
\definecolor{currentstroke}{rgb}{0.121569,0.466667,0.705882}%
\pgfsetstrokecolor{currentstroke}%
\pgfsetstrokeopacity{0.926592}%
\pgfsetdash{}{0pt}%
\pgfpathmoveto{\pgfqpoint{1.964367in}{0.869414in}}%
\pgfpathcurveto{\pgfqpoint{1.972603in}{0.869414in}}{\pgfqpoint{1.980504in}{0.872686in}}{\pgfqpoint{1.986327in}{0.878510in}}%
\pgfpathcurveto{\pgfqpoint{1.992151in}{0.884334in}}{\pgfqpoint{1.995424in}{0.892234in}}{\pgfqpoint{1.995424in}{0.900471in}}%
\pgfpathcurveto{\pgfqpoint{1.995424in}{0.908707in}}{\pgfqpoint{1.992151in}{0.916607in}}{\pgfqpoint{1.986327in}{0.922431in}}%
\pgfpathcurveto{\pgfqpoint{1.980504in}{0.928255in}}{\pgfqpoint{1.972603in}{0.931527in}}{\pgfqpoint{1.964367in}{0.931527in}}%
\pgfpathcurveto{\pgfqpoint{1.956131in}{0.931527in}}{\pgfqpoint{1.948231in}{0.928255in}}{\pgfqpoint{1.942407in}{0.922431in}}%
\pgfpathcurveto{\pgfqpoint{1.936583in}{0.916607in}}{\pgfqpoint{1.933311in}{0.908707in}}{\pgfqpoint{1.933311in}{0.900471in}}%
\pgfpathcurveto{\pgfqpoint{1.933311in}{0.892234in}}{\pgfqpoint{1.936583in}{0.884334in}}{\pgfqpoint{1.942407in}{0.878510in}}%
\pgfpathcurveto{\pgfqpoint{1.948231in}{0.872686in}}{\pgfqpoint{1.956131in}{0.869414in}}{\pgfqpoint{1.964367in}{0.869414in}}%
\pgfpathclose%
\pgfusepath{stroke,fill}%
\end{pgfscope}%
\begin{pgfscope}%
\pgfpathrectangle{\pgfqpoint{0.100000in}{0.220728in}}{\pgfqpoint{3.696000in}{3.696000in}}%
\pgfusepath{clip}%
\pgfsetbuttcap%
\pgfsetroundjoin%
\definecolor{currentfill}{rgb}{0.121569,0.466667,0.705882}%
\pgfsetfillcolor{currentfill}%
\pgfsetfillopacity{0.926735}%
\pgfsetlinewidth{1.003750pt}%
\definecolor{currentstroke}{rgb}{0.121569,0.466667,0.705882}%
\pgfsetstrokecolor{currentstroke}%
\pgfsetstrokeopacity{0.926735}%
\pgfsetdash{}{0pt}%
\pgfpathmoveto{\pgfqpoint{2.620654in}{1.292710in}}%
\pgfpathcurveto{\pgfqpoint{2.628891in}{1.292710in}}{\pgfqpoint{2.636791in}{1.295982in}}{\pgfqpoint{2.642615in}{1.301806in}}%
\pgfpathcurveto{\pgfqpoint{2.648439in}{1.307630in}}{\pgfqpoint{2.651711in}{1.315530in}}{\pgfqpoint{2.651711in}{1.323767in}}%
\pgfpathcurveto{\pgfqpoint{2.651711in}{1.332003in}}{\pgfqpoint{2.648439in}{1.339903in}}{\pgfqpoint{2.642615in}{1.345727in}}%
\pgfpathcurveto{\pgfqpoint{2.636791in}{1.351551in}}{\pgfqpoint{2.628891in}{1.354823in}}{\pgfqpoint{2.620654in}{1.354823in}}%
\pgfpathcurveto{\pgfqpoint{2.612418in}{1.354823in}}{\pgfqpoint{2.604518in}{1.351551in}}{\pgfqpoint{2.598694in}{1.345727in}}%
\pgfpathcurveto{\pgfqpoint{2.592870in}{1.339903in}}{\pgfqpoint{2.589598in}{1.332003in}}{\pgfqpoint{2.589598in}{1.323767in}}%
\pgfpathcurveto{\pgfqpoint{2.589598in}{1.315530in}}{\pgfqpoint{2.592870in}{1.307630in}}{\pgfqpoint{2.598694in}{1.301806in}}%
\pgfpathcurveto{\pgfqpoint{2.604518in}{1.295982in}}{\pgfqpoint{2.612418in}{1.292710in}}{\pgfqpoint{2.620654in}{1.292710in}}%
\pgfpathclose%
\pgfusepath{stroke,fill}%
\end{pgfscope}%
\begin{pgfscope}%
\pgfpathrectangle{\pgfqpoint{0.100000in}{0.220728in}}{\pgfqpoint{3.696000in}{3.696000in}}%
\pgfusepath{clip}%
\pgfsetbuttcap%
\pgfsetroundjoin%
\definecolor{currentfill}{rgb}{0.121569,0.466667,0.705882}%
\pgfsetfillcolor{currentfill}%
\pgfsetfillopacity{0.927810}%
\pgfsetlinewidth{1.003750pt}%
\definecolor{currentstroke}{rgb}{0.121569,0.466667,0.705882}%
\pgfsetstrokecolor{currentstroke}%
\pgfsetstrokeopacity{0.927810}%
\pgfsetdash{}{0pt}%
\pgfpathmoveto{\pgfqpoint{1.972930in}{0.864424in}}%
\pgfpathcurveto{\pgfqpoint{1.981166in}{0.864424in}}{\pgfqpoint{1.989066in}{0.867696in}}{\pgfqpoint{1.994890in}{0.873520in}}%
\pgfpathcurveto{\pgfqpoint{2.000714in}{0.879344in}}{\pgfqpoint{2.003986in}{0.887244in}}{\pgfqpoint{2.003986in}{0.895480in}}%
\pgfpathcurveto{\pgfqpoint{2.003986in}{0.903717in}}{\pgfqpoint{2.000714in}{0.911617in}}{\pgfqpoint{1.994890in}{0.917441in}}%
\pgfpathcurveto{\pgfqpoint{1.989066in}{0.923265in}}{\pgfqpoint{1.981166in}{0.926537in}}{\pgfqpoint{1.972930in}{0.926537in}}%
\pgfpathcurveto{\pgfqpoint{1.964693in}{0.926537in}}{\pgfqpoint{1.956793in}{0.923265in}}{\pgfqpoint{1.950969in}{0.917441in}}%
\pgfpathcurveto{\pgfqpoint{1.945145in}{0.911617in}}{\pgfqpoint{1.941873in}{0.903717in}}{\pgfqpoint{1.941873in}{0.895480in}}%
\pgfpathcurveto{\pgfqpoint{1.941873in}{0.887244in}}{\pgfqpoint{1.945145in}{0.879344in}}{\pgfqpoint{1.950969in}{0.873520in}}%
\pgfpathcurveto{\pgfqpoint{1.956793in}{0.867696in}}{\pgfqpoint{1.964693in}{0.864424in}}{\pgfqpoint{1.972930in}{0.864424in}}%
\pgfpathclose%
\pgfusepath{stroke,fill}%
\end{pgfscope}%
\begin{pgfscope}%
\pgfpathrectangle{\pgfqpoint{0.100000in}{0.220728in}}{\pgfqpoint{3.696000in}{3.696000in}}%
\pgfusepath{clip}%
\pgfsetbuttcap%
\pgfsetroundjoin%
\definecolor{currentfill}{rgb}{0.121569,0.466667,0.705882}%
\pgfsetfillcolor{currentfill}%
\pgfsetfillopacity{0.928735}%
\pgfsetlinewidth{1.003750pt}%
\definecolor{currentstroke}{rgb}{0.121569,0.466667,0.705882}%
\pgfsetstrokecolor{currentstroke}%
\pgfsetstrokeopacity{0.928735}%
\pgfsetdash{}{0pt}%
\pgfpathmoveto{\pgfqpoint{1.980997in}{0.858872in}}%
\pgfpathcurveto{\pgfqpoint{1.989233in}{0.858872in}}{\pgfqpoint{1.997133in}{0.862144in}}{\pgfqpoint{2.002957in}{0.867968in}}%
\pgfpathcurveto{\pgfqpoint{2.008781in}{0.873792in}}{\pgfqpoint{2.012053in}{0.881692in}}{\pgfqpoint{2.012053in}{0.889929in}}%
\pgfpathcurveto{\pgfqpoint{2.012053in}{0.898165in}}{\pgfqpoint{2.008781in}{0.906065in}}{\pgfqpoint{2.002957in}{0.911889in}}%
\pgfpathcurveto{\pgfqpoint{1.997133in}{0.917713in}}{\pgfqpoint{1.989233in}{0.920985in}}{\pgfqpoint{1.980997in}{0.920985in}}%
\pgfpathcurveto{\pgfqpoint{1.972760in}{0.920985in}}{\pgfqpoint{1.964860in}{0.917713in}}{\pgfqpoint{1.959037in}{0.911889in}}%
\pgfpathcurveto{\pgfqpoint{1.953213in}{0.906065in}}{\pgfqpoint{1.949940in}{0.898165in}}{\pgfqpoint{1.949940in}{0.889929in}}%
\pgfpathcurveto{\pgfqpoint{1.949940in}{0.881692in}}{\pgfqpoint{1.953213in}{0.873792in}}{\pgfqpoint{1.959037in}{0.867968in}}%
\pgfpathcurveto{\pgfqpoint{1.964860in}{0.862144in}}{\pgfqpoint{1.972760in}{0.858872in}}{\pgfqpoint{1.980997in}{0.858872in}}%
\pgfpathclose%
\pgfusepath{stroke,fill}%
\end{pgfscope}%
\begin{pgfscope}%
\pgfpathrectangle{\pgfqpoint{0.100000in}{0.220728in}}{\pgfqpoint{3.696000in}{3.696000in}}%
\pgfusepath{clip}%
\pgfsetbuttcap%
\pgfsetroundjoin%
\definecolor{currentfill}{rgb}{0.121569,0.466667,0.705882}%
\pgfsetfillcolor{currentfill}%
\pgfsetfillopacity{0.929110}%
\pgfsetlinewidth{1.003750pt}%
\definecolor{currentstroke}{rgb}{0.121569,0.466667,0.705882}%
\pgfsetstrokecolor{currentstroke}%
\pgfsetstrokeopacity{0.929110}%
\pgfsetdash{}{0pt}%
\pgfpathmoveto{\pgfqpoint{2.614363in}{1.275351in}}%
\pgfpathcurveto{\pgfqpoint{2.622600in}{1.275351in}}{\pgfqpoint{2.630500in}{1.278623in}}{\pgfqpoint{2.636324in}{1.284447in}}%
\pgfpathcurveto{\pgfqpoint{2.642147in}{1.290271in}}{\pgfqpoint{2.645420in}{1.298171in}}{\pgfqpoint{2.645420in}{1.306407in}}%
\pgfpathcurveto{\pgfqpoint{2.645420in}{1.314644in}}{\pgfqpoint{2.642147in}{1.322544in}}{\pgfqpoint{2.636324in}{1.328368in}}%
\pgfpathcurveto{\pgfqpoint{2.630500in}{1.334192in}}{\pgfqpoint{2.622600in}{1.337464in}}{\pgfqpoint{2.614363in}{1.337464in}}%
\pgfpathcurveto{\pgfqpoint{2.606127in}{1.337464in}}{\pgfqpoint{2.598227in}{1.334192in}}{\pgfqpoint{2.592403in}{1.328368in}}%
\pgfpathcurveto{\pgfqpoint{2.586579in}{1.322544in}}{\pgfqpoint{2.583307in}{1.314644in}}{\pgfqpoint{2.583307in}{1.306407in}}%
\pgfpathcurveto{\pgfqpoint{2.583307in}{1.298171in}}{\pgfqpoint{2.586579in}{1.290271in}}{\pgfqpoint{2.592403in}{1.284447in}}%
\pgfpathcurveto{\pgfqpoint{2.598227in}{1.278623in}}{\pgfqpoint{2.606127in}{1.275351in}}{\pgfqpoint{2.614363in}{1.275351in}}%
\pgfpathclose%
\pgfusepath{stroke,fill}%
\end{pgfscope}%
\begin{pgfscope}%
\pgfpathrectangle{\pgfqpoint{0.100000in}{0.220728in}}{\pgfqpoint{3.696000in}{3.696000in}}%
\pgfusepath{clip}%
\pgfsetbuttcap%
\pgfsetroundjoin%
\definecolor{currentfill}{rgb}{0.121569,0.466667,0.705882}%
\pgfsetfillcolor{currentfill}%
\pgfsetfillopacity{0.930249}%
\pgfsetlinewidth{1.003750pt}%
\definecolor{currentstroke}{rgb}{0.121569,0.466667,0.705882}%
\pgfsetstrokecolor{currentstroke}%
\pgfsetstrokeopacity{0.930249}%
\pgfsetdash{}{0pt}%
\pgfpathmoveto{\pgfqpoint{1.987886in}{0.855568in}}%
\pgfpathcurveto{\pgfqpoint{1.996123in}{0.855568in}}{\pgfqpoint{2.004023in}{0.858841in}}{\pgfqpoint{2.009847in}{0.864665in}}%
\pgfpathcurveto{\pgfqpoint{2.015671in}{0.870489in}}{\pgfqpoint{2.018943in}{0.878389in}}{\pgfqpoint{2.018943in}{0.886625in}}%
\pgfpathcurveto{\pgfqpoint{2.018943in}{0.894861in}}{\pgfqpoint{2.015671in}{0.902761in}}{\pgfqpoint{2.009847in}{0.908585in}}%
\pgfpathcurveto{\pgfqpoint{2.004023in}{0.914409in}}{\pgfqpoint{1.996123in}{0.917681in}}{\pgfqpoint{1.987886in}{0.917681in}}%
\pgfpathcurveto{\pgfqpoint{1.979650in}{0.917681in}}{\pgfqpoint{1.971750in}{0.914409in}}{\pgfqpoint{1.965926in}{0.908585in}}%
\pgfpathcurveto{\pgfqpoint{1.960102in}{0.902761in}}{\pgfqpoint{1.956830in}{0.894861in}}{\pgfqpoint{1.956830in}{0.886625in}}%
\pgfpathcurveto{\pgfqpoint{1.956830in}{0.878389in}}{\pgfqpoint{1.960102in}{0.870489in}}{\pgfqpoint{1.965926in}{0.864665in}}%
\pgfpathcurveto{\pgfqpoint{1.971750in}{0.858841in}}{\pgfqpoint{1.979650in}{0.855568in}}{\pgfqpoint{1.987886in}{0.855568in}}%
\pgfpathclose%
\pgfusepath{stroke,fill}%
\end{pgfscope}%
\begin{pgfscope}%
\pgfpathrectangle{\pgfqpoint{0.100000in}{0.220728in}}{\pgfqpoint{3.696000in}{3.696000in}}%
\pgfusepath{clip}%
\pgfsetbuttcap%
\pgfsetroundjoin%
\definecolor{currentfill}{rgb}{0.121569,0.466667,0.705882}%
\pgfsetfillcolor{currentfill}%
\pgfsetfillopacity{0.931169}%
\pgfsetlinewidth{1.003750pt}%
\definecolor{currentstroke}{rgb}{0.121569,0.466667,0.705882}%
\pgfsetstrokecolor{currentstroke}%
\pgfsetstrokeopacity{0.931169}%
\pgfsetdash{}{0pt}%
\pgfpathmoveto{\pgfqpoint{1.993140in}{0.852114in}}%
\pgfpathcurveto{\pgfqpoint{2.001376in}{0.852114in}}{\pgfqpoint{2.009276in}{0.855386in}}{\pgfqpoint{2.015100in}{0.861210in}}%
\pgfpathcurveto{\pgfqpoint{2.020924in}{0.867034in}}{\pgfqpoint{2.024197in}{0.874934in}}{\pgfqpoint{2.024197in}{0.883171in}}%
\pgfpathcurveto{\pgfqpoint{2.024197in}{0.891407in}}{\pgfqpoint{2.020924in}{0.899307in}}{\pgfqpoint{2.015100in}{0.905131in}}%
\pgfpathcurveto{\pgfqpoint{2.009276in}{0.910955in}}{\pgfqpoint{2.001376in}{0.914227in}}{\pgfqpoint{1.993140in}{0.914227in}}%
\pgfpathcurveto{\pgfqpoint{1.984904in}{0.914227in}}{\pgfqpoint{1.977004in}{0.910955in}}{\pgfqpoint{1.971180in}{0.905131in}}%
\pgfpathcurveto{\pgfqpoint{1.965356in}{0.899307in}}{\pgfqpoint{1.962084in}{0.891407in}}{\pgfqpoint{1.962084in}{0.883171in}}%
\pgfpathcurveto{\pgfqpoint{1.962084in}{0.874934in}}{\pgfqpoint{1.965356in}{0.867034in}}{\pgfqpoint{1.971180in}{0.861210in}}%
\pgfpathcurveto{\pgfqpoint{1.977004in}{0.855386in}}{\pgfqpoint{1.984904in}{0.852114in}}{\pgfqpoint{1.993140in}{0.852114in}}%
\pgfpathclose%
\pgfusepath{stroke,fill}%
\end{pgfscope}%
\begin{pgfscope}%
\pgfpathrectangle{\pgfqpoint{0.100000in}{0.220728in}}{\pgfqpoint{3.696000in}{3.696000in}}%
\pgfusepath{clip}%
\pgfsetbuttcap%
\pgfsetroundjoin%
\definecolor{currentfill}{rgb}{0.121569,0.466667,0.705882}%
\pgfsetfillcolor{currentfill}%
\pgfsetfillopacity{0.931864}%
\pgfsetlinewidth{1.003750pt}%
\definecolor{currentstroke}{rgb}{0.121569,0.466667,0.705882}%
\pgfsetstrokecolor{currentstroke}%
\pgfsetstrokeopacity{0.931864}%
\pgfsetdash{}{0pt}%
\pgfpathmoveto{\pgfqpoint{2.607356in}{1.257901in}}%
\pgfpathcurveto{\pgfqpoint{2.615592in}{1.257901in}}{\pgfqpoint{2.623492in}{1.261174in}}{\pgfqpoint{2.629316in}{1.266998in}}%
\pgfpathcurveto{\pgfqpoint{2.635140in}{1.272822in}}{\pgfqpoint{2.638413in}{1.280722in}}{\pgfqpoint{2.638413in}{1.288958in}}%
\pgfpathcurveto{\pgfqpoint{2.638413in}{1.297194in}}{\pgfqpoint{2.635140in}{1.305094in}}{\pgfqpoint{2.629316in}{1.310918in}}%
\pgfpathcurveto{\pgfqpoint{2.623492in}{1.316742in}}{\pgfqpoint{2.615592in}{1.320014in}}{\pgfqpoint{2.607356in}{1.320014in}}%
\pgfpathcurveto{\pgfqpoint{2.599120in}{1.320014in}}{\pgfqpoint{2.591220in}{1.316742in}}{\pgfqpoint{2.585396in}{1.310918in}}%
\pgfpathcurveto{\pgfqpoint{2.579572in}{1.305094in}}{\pgfqpoint{2.576300in}{1.297194in}}{\pgfqpoint{2.576300in}{1.288958in}}%
\pgfpathcurveto{\pgfqpoint{2.576300in}{1.280722in}}{\pgfqpoint{2.579572in}{1.272822in}}{\pgfqpoint{2.585396in}{1.266998in}}%
\pgfpathcurveto{\pgfqpoint{2.591220in}{1.261174in}}{\pgfqpoint{2.599120in}{1.257901in}}{\pgfqpoint{2.607356in}{1.257901in}}%
\pgfpathclose%
\pgfusepath{stroke,fill}%
\end{pgfscope}%
\begin{pgfscope}%
\pgfpathrectangle{\pgfqpoint{0.100000in}{0.220728in}}{\pgfqpoint{3.696000in}{3.696000in}}%
\pgfusepath{clip}%
\pgfsetbuttcap%
\pgfsetroundjoin%
\definecolor{currentfill}{rgb}{0.121569,0.466667,0.705882}%
\pgfsetfillcolor{currentfill}%
\pgfsetfillopacity{0.932868}%
\pgfsetlinewidth{1.003750pt}%
\definecolor{currentstroke}{rgb}{0.121569,0.466667,0.705882}%
\pgfsetstrokecolor{currentstroke}%
\pgfsetstrokeopacity{0.932868}%
\pgfsetdash{}{0pt}%
\pgfpathmoveto{\pgfqpoint{2.601619in}{1.248954in}}%
\pgfpathcurveto{\pgfqpoint{2.609856in}{1.248954in}}{\pgfqpoint{2.617756in}{1.252227in}}{\pgfqpoint{2.623580in}{1.258051in}}%
\pgfpathcurveto{\pgfqpoint{2.629404in}{1.263875in}}{\pgfqpoint{2.632676in}{1.271775in}}{\pgfqpoint{2.632676in}{1.280011in}}%
\pgfpathcurveto{\pgfqpoint{2.632676in}{1.288247in}}{\pgfqpoint{2.629404in}{1.296147in}}{\pgfqpoint{2.623580in}{1.301971in}}%
\pgfpathcurveto{\pgfqpoint{2.617756in}{1.307795in}}{\pgfqpoint{2.609856in}{1.311067in}}{\pgfqpoint{2.601619in}{1.311067in}}%
\pgfpathcurveto{\pgfqpoint{2.593383in}{1.311067in}}{\pgfqpoint{2.585483in}{1.307795in}}{\pgfqpoint{2.579659in}{1.301971in}}%
\pgfpathcurveto{\pgfqpoint{2.573835in}{1.296147in}}{\pgfqpoint{2.570563in}{1.288247in}}{\pgfqpoint{2.570563in}{1.280011in}}%
\pgfpathcurveto{\pgfqpoint{2.570563in}{1.271775in}}{\pgfqpoint{2.573835in}{1.263875in}}{\pgfqpoint{2.579659in}{1.258051in}}%
\pgfpathcurveto{\pgfqpoint{2.585483in}{1.252227in}}{\pgfqpoint{2.593383in}{1.248954in}}{\pgfqpoint{2.601619in}{1.248954in}}%
\pgfpathclose%
\pgfusepath{stroke,fill}%
\end{pgfscope}%
\begin{pgfscope}%
\pgfpathrectangle{\pgfqpoint{0.100000in}{0.220728in}}{\pgfqpoint{3.696000in}{3.696000in}}%
\pgfusepath{clip}%
\pgfsetbuttcap%
\pgfsetroundjoin%
\definecolor{currentfill}{rgb}{0.121569,0.466667,0.705882}%
\pgfsetfillcolor{currentfill}%
\pgfsetfillopacity{0.933597}%
\pgfsetlinewidth{1.003750pt}%
\definecolor{currentstroke}{rgb}{0.121569,0.466667,0.705882}%
\pgfsetstrokecolor{currentstroke}%
\pgfsetstrokeopacity{0.933597}%
\pgfsetdash{}{0pt}%
\pgfpathmoveto{\pgfqpoint{2.003030in}{0.849815in}}%
\pgfpathcurveto{\pgfqpoint{2.011267in}{0.849815in}}{\pgfqpoint{2.019167in}{0.853087in}}{\pgfqpoint{2.024991in}{0.858911in}}%
\pgfpathcurveto{\pgfqpoint{2.030814in}{0.864735in}}{\pgfqpoint{2.034087in}{0.872635in}}{\pgfqpoint{2.034087in}{0.880871in}}%
\pgfpathcurveto{\pgfqpoint{2.034087in}{0.889108in}}{\pgfqpoint{2.030814in}{0.897008in}}{\pgfqpoint{2.024991in}{0.902831in}}%
\pgfpathcurveto{\pgfqpoint{2.019167in}{0.908655in}}{\pgfqpoint{2.011267in}{0.911928in}}{\pgfqpoint{2.003030in}{0.911928in}}%
\pgfpathcurveto{\pgfqpoint{1.994794in}{0.911928in}}{\pgfqpoint{1.986894in}{0.908655in}}{\pgfqpoint{1.981070in}{0.902831in}}%
\pgfpathcurveto{\pgfqpoint{1.975246in}{0.897008in}}{\pgfqpoint{1.971974in}{0.889108in}}{\pgfqpoint{1.971974in}{0.880871in}}%
\pgfpathcurveto{\pgfqpoint{1.971974in}{0.872635in}}{\pgfqpoint{1.975246in}{0.864735in}}{\pgfqpoint{1.981070in}{0.858911in}}%
\pgfpathcurveto{\pgfqpoint{1.986894in}{0.853087in}}{\pgfqpoint{1.994794in}{0.849815in}}{\pgfqpoint{2.003030in}{0.849815in}}%
\pgfpathclose%
\pgfusepath{stroke,fill}%
\end{pgfscope}%
\begin{pgfscope}%
\pgfpathrectangle{\pgfqpoint{0.100000in}{0.220728in}}{\pgfqpoint{3.696000in}{3.696000in}}%
\pgfusepath{clip}%
\pgfsetbuttcap%
\pgfsetroundjoin%
\definecolor{currentfill}{rgb}{0.121569,0.466667,0.705882}%
\pgfsetfillcolor{currentfill}%
\pgfsetfillopacity{0.934807}%
\pgfsetlinewidth{1.003750pt}%
\definecolor{currentstroke}{rgb}{0.121569,0.466667,0.705882}%
\pgfsetstrokecolor{currentstroke}%
\pgfsetstrokeopacity{0.934807}%
\pgfsetdash{}{0pt}%
\pgfpathmoveto{\pgfqpoint{2.598010in}{1.232843in}}%
\pgfpathcurveto{\pgfqpoint{2.606246in}{1.232843in}}{\pgfqpoint{2.614146in}{1.236115in}}{\pgfqpoint{2.619970in}{1.241939in}}%
\pgfpathcurveto{\pgfqpoint{2.625794in}{1.247763in}}{\pgfqpoint{2.629066in}{1.255663in}}{\pgfqpoint{2.629066in}{1.263899in}}%
\pgfpathcurveto{\pgfqpoint{2.629066in}{1.272135in}}{\pgfqpoint{2.625794in}{1.280035in}}{\pgfqpoint{2.619970in}{1.285859in}}%
\pgfpathcurveto{\pgfqpoint{2.614146in}{1.291683in}}{\pgfqpoint{2.606246in}{1.294956in}}{\pgfqpoint{2.598010in}{1.294956in}}%
\pgfpathcurveto{\pgfqpoint{2.589774in}{1.294956in}}{\pgfqpoint{2.581874in}{1.291683in}}{\pgfqpoint{2.576050in}{1.285859in}}%
\pgfpathcurveto{\pgfqpoint{2.570226in}{1.280035in}}{\pgfqpoint{2.566953in}{1.272135in}}{\pgfqpoint{2.566953in}{1.263899in}}%
\pgfpathcurveto{\pgfqpoint{2.566953in}{1.255663in}}{\pgfqpoint{2.570226in}{1.247763in}}{\pgfqpoint{2.576050in}{1.241939in}}%
\pgfpathcurveto{\pgfqpoint{2.581874in}{1.236115in}}{\pgfqpoint{2.589774in}{1.232843in}}{\pgfqpoint{2.598010in}{1.232843in}}%
\pgfpathclose%
\pgfusepath{stroke,fill}%
\end{pgfscope}%
\begin{pgfscope}%
\pgfpathrectangle{\pgfqpoint{0.100000in}{0.220728in}}{\pgfqpoint{3.696000in}{3.696000in}}%
\pgfusepath{clip}%
\pgfsetbuttcap%
\pgfsetroundjoin%
\definecolor{currentfill}{rgb}{0.121569,0.466667,0.705882}%
\pgfsetfillcolor{currentfill}%
\pgfsetfillopacity{0.935595}%
\pgfsetlinewidth{1.003750pt}%
\definecolor{currentstroke}{rgb}{0.121569,0.466667,0.705882}%
\pgfsetstrokecolor{currentstroke}%
\pgfsetstrokeopacity{0.935595}%
\pgfsetdash{}{0pt}%
\pgfpathmoveto{\pgfqpoint{2.594021in}{1.224863in}}%
\pgfpathcurveto{\pgfqpoint{2.602257in}{1.224863in}}{\pgfqpoint{2.610157in}{1.228136in}}{\pgfqpoint{2.615981in}{1.233959in}}%
\pgfpathcurveto{\pgfqpoint{2.621805in}{1.239783in}}{\pgfqpoint{2.625077in}{1.247683in}}{\pgfqpoint{2.625077in}{1.255920in}}%
\pgfpathcurveto{\pgfqpoint{2.625077in}{1.264156in}}{\pgfqpoint{2.621805in}{1.272056in}}{\pgfqpoint{2.615981in}{1.277880in}}%
\pgfpathcurveto{\pgfqpoint{2.610157in}{1.283704in}}{\pgfqpoint{2.602257in}{1.286976in}}{\pgfqpoint{2.594021in}{1.286976in}}%
\pgfpathcurveto{\pgfqpoint{2.585784in}{1.286976in}}{\pgfqpoint{2.577884in}{1.283704in}}{\pgfqpoint{2.572060in}{1.277880in}}%
\pgfpathcurveto{\pgfqpoint{2.566236in}{1.272056in}}{\pgfqpoint{2.562964in}{1.264156in}}{\pgfqpoint{2.562964in}{1.255920in}}%
\pgfpathcurveto{\pgfqpoint{2.562964in}{1.247683in}}{\pgfqpoint{2.566236in}{1.239783in}}{\pgfqpoint{2.572060in}{1.233959in}}%
\pgfpathcurveto{\pgfqpoint{2.577884in}{1.228136in}}{\pgfqpoint{2.585784in}{1.224863in}}{\pgfqpoint{2.594021in}{1.224863in}}%
\pgfpathclose%
\pgfusepath{stroke,fill}%
\end{pgfscope}%
\begin{pgfscope}%
\pgfpathrectangle{\pgfqpoint{0.100000in}{0.220728in}}{\pgfqpoint{3.696000in}{3.696000in}}%
\pgfusepath{clip}%
\pgfsetbuttcap%
\pgfsetroundjoin%
\definecolor{currentfill}{rgb}{0.121569,0.466667,0.705882}%
\pgfsetfillcolor{currentfill}%
\pgfsetfillopacity{0.936156}%
\pgfsetlinewidth{1.003750pt}%
\definecolor{currentstroke}{rgb}{0.121569,0.466667,0.705882}%
\pgfsetstrokecolor{currentstroke}%
\pgfsetstrokeopacity{0.936156}%
\pgfsetdash{}{0pt}%
\pgfpathmoveto{\pgfqpoint{2.591780in}{1.221055in}}%
\pgfpathcurveto{\pgfqpoint{2.600016in}{1.221055in}}{\pgfqpoint{2.607916in}{1.224327in}}{\pgfqpoint{2.613740in}{1.230151in}}%
\pgfpathcurveto{\pgfqpoint{2.619564in}{1.235975in}}{\pgfqpoint{2.622836in}{1.243875in}}{\pgfqpoint{2.622836in}{1.252111in}}%
\pgfpathcurveto{\pgfqpoint{2.622836in}{1.260347in}}{\pgfqpoint{2.619564in}{1.268248in}}{\pgfqpoint{2.613740in}{1.274071in}}%
\pgfpathcurveto{\pgfqpoint{2.607916in}{1.279895in}}{\pgfqpoint{2.600016in}{1.283168in}}{\pgfqpoint{2.591780in}{1.283168in}}%
\pgfpathcurveto{\pgfqpoint{2.583544in}{1.283168in}}{\pgfqpoint{2.575644in}{1.279895in}}{\pgfqpoint{2.569820in}{1.274071in}}%
\pgfpathcurveto{\pgfqpoint{2.563996in}{1.268248in}}{\pgfqpoint{2.560723in}{1.260347in}}{\pgfqpoint{2.560723in}{1.252111in}}%
\pgfpathcurveto{\pgfqpoint{2.560723in}{1.243875in}}{\pgfqpoint{2.563996in}{1.235975in}}{\pgfqpoint{2.569820in}{1.230151in}}%
\pgfpathcurveto{\pgfqpoint{2.575644in}{1.224327in}}{\pgfqpoint{2.583544in}{1.221055in}}{\pgfqpoint{2.591780in}{1.221055in}}%
\pgfpathclose%
\pgfusepath{stroke,fill}%
\end{pgfscope}%
\begin{pgfscope}%
\pgfpathrectangle{\pgfqpoint{0.100000in}{0.220728in}}{\pgfqpoint{3.696000in}{3.696000in}}%
\pgfusepath{clip}%
\pgfsetbuttcap%
\pgfsetroundjoin%
\definecolor{currentfill}{rgb}{0.121569,0.466667,0.705882}%
\pgfsetfillcolor{currentfill}%
\pgfsetfillopacity{0.936857}%
\pgfsetlinewidth{1.003750pt}%
\definecolor{currentstroke}{rgb}{0.121569,0.466667,0.705882}%
\pgfsetstrokecolor{currentstroke}%
\pgfsetstrokeopacity{0.936857}%
\pgfsetdash{}{0pt}%
\pgfpathmoveto{\pgfqpoint{2.590517in}{1.213980in}}%
\pgfpathcurveto{\pgfqpoint{2.598754in}{1.213980in}}{\pgfqpoint{2.606654in}{1.217253in}}{\pgfqpoint{2.612478in}{1.223077in}}%
\pgfpathcurveto{\pgfqpoint{2.618301in}{1.228900in}}{\pgfqpoint{2.621574in}{1.236801in}}{\pgfqpoint{2.621574in}{1.245037in}}%
\pgfpathcurveto{\pgfqpoint{2.621574in}{1.253273in}}{\pgfqpoint{2.618301in}{1.261173in}}{\pgfqpoint{2.612478in}{1.266997in}}%
\pgfpathcurveto{\pgfqpoint{2.606654in}{1.272821in}}{\pgfqpoint{2.598754in}{1.276093in}}{\pgfqpoint{2.590517in}{1.276093in}}%
\pgfpathcurveto{\pgfqpoint{2.582281in}{1.276093in}}{\pgfqpoint{2.574381in}{1.272821in}}{\pgfqpoint{2.568557in}{1.266997in}}%
\pgfpathcurveto{\pgfqpoint{2.562733in}{1.261173in}}{\pgfqpoint{2.559461in}{1.253273in}}{\pgfqpoint{2.559461in}{1.245037in}}%
\pgfpathcurveto{\pgfqpoint{2.559461in}{1.236801in}}{\pgfqpoint{2.562733in}{1.228900in}}{\pgfqpoint{2.568557in}{1.223077in}}%
\pgfpathcurveto{\pgfqpoint{2.574381in}{1.217253in}}{\pgfqpoint{2.582281in}{1.213980in}}{\pgfqpoint{2.590517in}{1.213980in}}%
\pgfpathclose%
\pgfusepath{stroke,fill}%
\end{pgfscope}%
\begin{pgfscope}%
\pgfpathrectangle{\pgfqpoint{0.100000in}{0.220728in}}{\pgfqpoint{3.696000in}{3.696000in}}%
\pgfusepath{clip}%
\pgfsetbuttcap%
\pgfsetroundjoin%
\definecolor{currentfill}{rgb}{0.121569,0.466667,0.705882}%
\pgfsetfillcolor{currentfill}%
\pgfsetfillopacity{0.937003}%
\pgfsetlinewidth{1.003750pt}%
\definecolor{currentstroke}{rgb}{0.121569,0.466667,0.705882}%
\pgfsetstrokecolor{currentstroke}%
\pgfsetstrokeopacity{0.937003}%
\pgfsetdash{}{0pt}%
\pgfpathmoveto{\pgfqpoint{2.021207in}{0.842386in}}%
\pgfpathcurveto{\pgfqpoint{2.029443in}{0.842386in}}{\pgfqpoint{2.037344in}{0.845658in}}{\pgfqpoint{2.043167in}{0.851482in}}%
\pgfpathcurveto{\pgfqpoint{2.048991in}{0.857306in}}{\pgfqpoint{2.052264in}{0.865206in}}{\pgfqpoint{2.052264in}{0.873442in}}%
\pgfpathcurveto{\pgfqpoint{2.052264in}{0.881679in}}{\pgfqpoint{2.048991in}{0.889579in}}{\pgfqpoint{2.043167in}{0.895403in}}%
\pgfpathcurveto{\pgfqpoint{2.037344in}{0.901226in}}{\pgfqpoint{2.029443in}{0.904499in}}{\pgfqpoint{2.021207in}{0.904499in}}%
\pgfpathcurveto{\pgfqpoint{2.012971in}{0.904499in}}{\pgfqpoint{2.005071in}{0.901226in}}{\pgfqpoint{1.999247in}{0.895403in}}%
\pgfpathcurveto{\pgfqpoint{1.993423in}{0.889579in}}{\pgfqpoint{1.990151in}{0.881679in}}{\pgfqpoint{1.990151in}{0.873442in}}%
\pgfpathcurveto{\pgfqpoint{1.990151in}{0.865206in}}{\pgfqpoint{1.993423in}{0.857306in}}{\pgfqpoint{1.999247in}{0.851482in}}%
\pgfpathcurveto{\pgfqpoint{2.005071in}{0.845658in}}{\pgfqpoint{2.012971in}{0.842386in}}{\pgfqpoint{2.021207in}{0.842386in}}%
\pgfpathclose%
\pgfusepath{stroke,fill}%
\end{pgfscope}%
\begin{pgfscope}%
\pgfpathrectangle{\pgfqpoint{0.100000in}{0.220728in}}{\pgfqpoint{3.696000in}{3.696000in}}%
\pgfusepath{clip}%
\pgfsetbuttcap%
\pgfsetroundjoin%
\definecolor{currentfill}{rgb}{0.121569,0.466667,0.705882}%
\pgfsetfillcolor{currentfill}%
\pgfsetfillopacity{0.937865}%
\pgfsetlinewidth{1.003750pt}%
\definecolor{currentstroke}{rgb}{0.121569,0.466667,0.705882}%
\pgfsetstrokecolor{currentstroke}%
\pgfsetstrokeopacity{0.937865}%
\pgfsetdash{}{0pt}%
\pgfpathmoveto{\pgfqpoint{2.586702in}{1.206158in}}%
\pgfpathcurveto{\pgfqpoint{2.594939in}{1.206158in}}{\pgfqpoint{2.602839in}{1.209431in}}{\pgfqpoint{2.608663in}{1.215255in}}%
\pgfpathcurveto{\pgfqpoint{2.614486in}{1.221079in}}{\pgfqpoint{2.617759in}{1.228979in}}{\pgfqpoint{2.617759in}{1.237215in}}%
\pgfpathcurveto{\pgfqpoint{2.617759in}{1.245451in}}{\pgfqpoint{2.614486in}{1.253351in}}{\pgfqpoint{2.608663in}{1.259175in}}%
\pgfpathcurveto{\pgfqpoint{2.602839in}{1.264999in}}{\pgfqpoint{2.594939in}{1.268271in}}{\pgfqpoint{2.586702in}{1.268271in}}%
\pgfpathcurveto{\pgfqpoint{2.578466in}{1.268271in}}{\pgfqpoint{2.570566in}{1.264999in}}{\pgfqpoint{2.564742in}{1.259175in}}%
\pgfpathcurveto{\pgfqpoint{2.558918in}{1.253351in}}{\pgfqpoint{2.555646in}{1.245451in}}{\pgfqpoint{2.555646in}{1.237215in}}%
\pgfpathcurveto{\pgfqpoint{2.555646in}{1.228979in}}{\pgfqpoint{2.558918in}{1.221079in}}{\pgfqpoint{2.564742in}{1.215255in}}%
\pgfpathcurveto{\pgfqpoint{2.570566in}{1.209431in}}{\pgfqpoint{2.578466in}{1.206158in}}{\pgfqpoint{2.586702in}{1.206158in}}%
\pgfpathclose%
\pgfusepath{stroke,fill}%
\end{pgfscope}%
\begin{pgfscope}%
\pgfpathrectangle{\pgfqpoint{0.100000in}{0.220728in}}{\pgfqpoint{3.696000in}{3.696000in}}%
\pgfusepath{clip}%
\pgfsetbuttcap%
\pgfsetroundjoin%
\definecolor{currentfill}{rgb}{0.121569,0.466667,0.705882}%
\pgfsetfillcolor{currentfill}%
\pgfsetfillopacity{0.938474}%
\pgfsetlinewidth{1.003750pt}%
\definecolor{currentstroke}{rgb}{0.121569,0.466667,0.705882}%
\pgfsetstrokecolor{currentstroke}%
\pgfsetstrokeopacity{0.938474}%
\pgfsetdash{}{0pt}%
\pgfpathmoveto{\pgfqpoint{2.584776in}{1.201849in}}%
\pgfpathcurveto{\pgfqpoint{2.593012in}{1.201849in}}{\pgfqpoint{2.600912in}{1.205122in}}{\pgfqpoint{2.606736in}{1.210946in}}%
\pgfpathcurveto{\pgfqpoint{2.612560in}{1.216770in}}{\pgfqpoint{2.615832in}{1.224670in}}{\pgfqpoint{2.615832in}{1.232906in}}%
\pgfpathcurveto{\pgfqpoint{2.615832in}{1.241142in}}{\pgfqpoint{2.612560in}{1.249042in}}{\pgfqpoint{2.606736in}{1.254866in}}%
\pgfpathcurveto{\pgfqpoint{2.600912in}{1.260690in}}{\pgfqpoint{2.593012in}{1.263962in}}{\pgfqpoint{2.584776in}{1.263962in}}%
\pgfpathcurveto{\pgfqpoint{2.576539in}{1.263962in}}{\pgfqpoint{2.568639in}{1.260690in}}{\pgfqpoint{2.562815in}{1.254866in}}%
\pgfpathcurveto{\pgfqpoint{2.556991in}{1.249042in}}{\pgfqpoint{2.553719in}{1.241142in}}{\pgfqpoint{2.553719in}{1.232906in}}%
\pgfpathcurveto{\pgfqpoint{2.553719in}{1.224670in}}{\pgfqpoint{2.556991in}{1.216770in}}{\pgfqpoint{2.562815in}{1.210946in}}%
\pgfpathcurveto{\pgfqpoint{2.568639in}{1.205122in}}{\pgfqpoint{2.576539in}{1.201849in}}{\pgfqpoint{2.584776in}{1.201849in}}%
\pgfpathclose%
\pgfusepath{stroke,fill}%
\end{pgfscope}%
\begin{pgfscope}%
\pgfpathrectangle{\pgfqpoint{0.100000in}{0.220728in}}{\pgfqpoint{3.696000in}{3.696000in}}%
\pgfusepath{clip}%
\pgfsetbuttcap%
\pgfsetroundjoin%
\definecolor{currentfill}{rgb}{0.121569,0.466667,0.705882}%
\pgfsetfillcolor{currentfill}%
\pgfsetfillopacity{0.939181}%
\pgfsetlinewidth{1.003750pt}%
\definecolor{currentstroke}{rgb}{0.121569,0.466667,0.705882}%
\pgfsetstrokecolor{currentstroke}%
\pgfsetstrokeopacity{0.939181}%
\pgfsetdash{}{0pt}%
\pgfpathmoveto{\pgfqpoint{2.583290in}{1.195836in}}%
\pgfpathcurveto{\pgfqpoint{2.591527in}{1.195836in}}{\pgfqpoint{2.599427in}{1.199108in}}{\pgfqpoint{2.605251in}{1.204932in}}%
\pgfpathcurveto{\pgfqpoint{2.611074in}{1.210756in}}{\pgfqpoint{2.614347in}{1.218656in}}{\pgfqpoint{2.614347in}{1.226893in}}%
\pgfpathcurveto{\pgfqpoint{2.614347in}{1.235129in}}{\pgfqpoint{2.611074in}{1.243029in}}{\pgfqpoint{2.605251in}{1.248853in}}%
\pgfpathcurveto{\pgfqpoint{2.599427in}{1.254677in}}{\pgfqpoint{2.591527in}{1.257949in}}{\pgfqpoint{2.583290in}{1.257949in}}%
\pgfpathcurveto{\pgfqpoint{2.575054in}{1.257949in}}{\pgfqpoint{2.567154in}{1.254677in}}{\pgfqpoint{2.561330in}{1.248853in}}%
\pgfpathcurveto{\pgfqpoint{2.555506in}{1.243029in}}{\pgfqpoint{2.552234in}{1.235129in}}{\pgfqpoint{2.552234in}{1.226893in}}%
\pgfpathcurveto{\pgfqpoint{2.552234in}{1.218656in}}{\pgfqpoint{2.555506in}{1.210756in}}{\pgfqpoint{2.561330in}{1.204932in}}%
\pgfpathcurveto{\pgfqpoint{2.567154in}{1.199108in}}{\pgfqpoint{2.575054in}{1.195836in}}{\pgfqpoint{2.583290in}{1.195836in}}%
\pgfpathclose%
\pgfusepath{stroke,fill}%
\end{pgfscope}%
\begin{pgfscope}%
\pgfpathrectangle{\pgfqpoint{0.100000in}{0.220728in}}{\pgfqpoint{3.696000in}{3.696000in}}%
\pgfusepath{clip}%
\pgfsetbuttcap%
\pgfsetroundjoin%
\definecolor{currentfill}{rgb}{0.121569,0.466667,0.705882}%
\pgfsetfillcolor{currentfill}%
\pgfsetfillopacity{0.939855}%
\pgfsetlinewidth{1.003750pt}%
\definecolor{currentstroke}{rgb}{0.121569,0.466667,0.705882}%
\pgfsetstrokecolor{currentstroke}%
\pgfsetstrokeopacity{0.939855}%
\pgfsetdash{}{0pt}%
\pgfpathmoveto{\pgfqpoint{2.037928in}{0.833684in}}%
\pgfpathcurveto{\pgfqpoint{2.046164in}{0.833684in}}{\pgfqpoint{2.054064in}{0.836956in}}{\pgfqpoint{2.059888in}{0.842780in}}%
\pgfpathcurveto{\pgfqpoint{2.065712in}{0.848604in}}{\pgfqpoint{2.068984in}{0.856504in}}{\pgfqpoint{2.068984in}{0.864740in}}%
\pgfpathcurveto{\pgfqpoint{2.068984in}{0.872977in}}{\pgfqpoint{2.065712in}{0.880877in}}{\pgfqpoint{2.059888in}{0.886701in}}%
\pgfpathcurveto{\pgfqpoint{2.054064in}{0.892525in}}{\pgfqpoint{2.046164in}{0.895797in}}{\pgfqpoint{2.037928in}{0.895797in}}%
\pgfpathcurveto{\pgfqpoint{2.029691in}{0.895797in}}{\pgfqpoint{2.021791in}{0.892525in}}{\pgfqpoint{2.015967in}{0.886701in}}%
\pgfpathcurveto{\pgfqpoint{2.010143in}{0.880877in}}{\pgfqpoint{2.006871in}{0.872977in}}{\pgfqpoint{2.006871in}{0.864740in}}%
\pgfpathcurveto{\pgfqpoint{2.006871in}{0.856504in}}{\pgfqpoint{2.010143in}{0.848604in}}{\pgfqpoint{2.015967in}{0.842780in}}%
\pgfpathcurveto{\pgfqpoint{2.021791in}{0.836956in}}{\pgfqpoint{2.029691in}{0.833684in}}{\pgfqpoint{2.037928in}{0.833684in}}%
\pgfpathclose%
\pgfusepath{stroke,fill}%
\end{pgfscope}%
\begin{pgfscope}%
\pgfpathrectangle{\pgfqpoint{0.100000in}{0.220728in}}{\pgfqpoint{3.696000in}{3.696000in}}%
\pgfusepath{clip}%
\pgfsetbuttcap%
\pgfsetroundjoin%
\definecolor{currentfill}{rgb}{0.121569,0.466667,0.705882}%
\pgfsetfillcolor{currentfill}%
\pgfsetfillopacity{0.940585}%
\pgfsetlinewidth{1.003750pt}%
\definecolor{currentstroke}{rgb}{0.121569,0.466667,0.705882}%
\pgfsetstrokecolor{currentstroke}%
\pgfsetstrokeopacity{0.940585}%
\pgfsetdash{}{0pt}%
\pgfpathmoveto{\pgfqpoint{2.577912in}{1.186752in}}%
\pgfpathcurveto{\pgfqpoint{2.586148in}{1.186752in}}{\pgfqpoint{2.594048in}{1.190025in}}{\pgfqpoint{2.599872in}{1.195849in}}%
\pgfpathcurveto{\pgfqpoint{2.605696in}{1.201673in}}{\pgfqpoint{2.608968in}{1.209573in}}{\pgfqpoint{2.608968in}{1.217809in}}%
\pgfpathcurveto{\pgfqpoint{2.608968in}{1.226045in}}{\pgfqpoint{2.605696in}{1.233945in}}{\pgfqpoint{2.599872in}{1.239769in}}%
\pgfpathcurveto{\pgfqpoint{2.594048in}{1.245593in}}{\pgfqpoint{2.586148in}{1.248865in}}{\pgfqpoint{2.577912in}{1.248865in}}%
\pgfpathcurveto{\pgfqpoint{2.569675in}{1.248865in}}{\pgfqpoint{2.561775in}{1.245593in}}{\pgfqpoint{2.555951in}{1.239769in}}%
\pgfpathcurveto{\pgfqpoint{2.550128in}{1.233945in}}{\pgfqpoint{2.546855in}{1.226045in}}{\pgfqpoint{2.546855in}{1.217809in}}%
\pgfpathcurveto{\pgfqpoint{2.546855in}{1.209573in}}{\pgfqpoint{2.550128in}{1.201673in}}{\pgfqpoint{2.555951in}{1.195849in}}%
\pgfpathcurveto{\pgfqpoint{2.561775in}{1.190025in}}{\pgfqpoint{2.569675in}{1.186752in}}{\pgfqpoint{2.577912in}{1.186752in}}%
\pgfpathclose%
\pgfusepath{stroke,fill}%
\end{pgfscope}%
\begin{pgfscope}%
\pgfpathrectangle{\pgfqpoint{0.100000in}{0.220728in}}{\pgfqpoint{3.696000in}{3.696000in}}%
\pgfusepath{clip}%
\pgfsetbuttcap%
\pgfsetroundjoin%
\definecolor{currentfill}{rgb}{0.121569,0.466667,0.705882}%
\pgfsetfillcolor{currentfill}%
\pgfsetfillopacity{0.942093}%
\pgfsetlinewidth{1.003750pt}%
\definecolor{currentstroke}{rgb}{0.121569,0.466667,0.705882}%
\pgfsetstrokecolor{currentstroke}%
\pgfsetstrokeopacity{0.942093}%
\pgfsetdash{}{0pt}%
\pgfpathmoveto{\pgfqpoint{2.574395in}{1.174207in}}%
\pgfpathcurveto{\pgfqpoint{2.582631in}{1.174207in}}{\pgfqpoint{2.590531in}{1.177479in}}{\pgfqpoint{2.596355in}{1.183303in}}%
\pgfpathcurveto{\pgfqpoint{2.602179in}{1.189127in}}{\pgfqpoint{2.605452in}{1.197027in}}{\pgfqpoint{2.605452in}{1.205264in}}%
\pgfpathcurveto{\pgfqpoint{2.605452in}{1.213500in}}{\pgfqpoint{2.602179in}{1.221400in}}{\pgfqpoint{2.596355in}{1.227224in}}%
\pgfpathcurveto{\pgfqpoint{2.590531in}{1.233048in}}{\pgfqpoint{2.582631in}{1.236320in}}{\pgfqpoint{2.574395in}{1.236320in}}%
\pgfpathcurveto{\pgfqpoint{2.566159in}{1.236320in}}{\pgfqpoint{2.558259in}{1.233048in}}{\pgfqpoint{2.552435in}{1.227224in}}%
\pgfpathcurveto{\pgfqpoint{2.546611in}{1.221400in}}{\pgfqpoint{2.543339in}{1.213500in}}{\pgfqpoint{2.543339in}{1.205264in}}%
\pgfpathcurveto{\pgfqpoint{2.543339in}{1.197027in}}{\pgfqpoint{2.546611in}{1.189127in}}{\pgfqpoint{2.552435in}{1.183303in}}%
\pgfpathcurveto{\pgfqpoint{2.558259in}{1.177479in}}{\pgfqpoint{2.566159in}{1.174207in}}{\pgfqpoint{2.574395in}{1.174207in}}%
\pgfpathclose%
\pgfusepath{stroke,fill}%
\end{pgfscope}%
\begin{pgfscope}%
\pgfpathrectangle{\pgfqpoint{0.100000in}{0.220728in}}{\pgfqpoint{3.696000in}{3.696000in}}%
\pgfusepath{clip}%
\pgfsetbuttcap%
\pgfsetroundjoin%
\definecolor{currentfill}{rgb}{0.121569,0.466667,0.705882}%
\pgfsetfillcolor{currentfill}%
\pgfsetfillopacity{0.943098}%
\pgfsetlinewidth{1.003750pt}%
\definecolor{currentstroke}{rgb}{0.121569,0.466667,0.705882}%
\pgfsetstrokecolor{currentstroke}%
\pgfsetstrokeopacity{0.943098}%
\pgfsetdash{}{0pt}%
\pgfpathmoveto{\pgfqpoint{2.572396in}{1.168041in}}%
\pgfpathcurveto{\pgfqpoint{2.580633in}{1.168041in}}{\pgfqpoint{2.588533in}{1.171313in}}{\pgfqpoint{2.594357in}{1.177137in}}%
\pgfpathcurveto{\pgfqpoint{2.600180in}{1.182961in}}{\pgfqpoint{2.603453in}{1.190861in}}{\pgfqpoint{2.603453in}{1.199097in}}%
\pgfpathcurveto{\pgfqpoint{2.603453in}{1.207334in}}{\pgfqpoint{2.600180in}{1.215234in}}{\pgfqpoint{2.594357in}{1.221058in}}%
\pgfpathcurveto{\pgfqpoint{2.588533in}{1.226882in}}{\pgfqpoint{2.580633in}{1.230154in}}{\pgfqpoint{2.572396in}{1.230154in}}%
\pgfpathcurveto{\pgfqpoint{2.564160in}{1.230154in}}{\pgfqpoint{2.556260in}{1.226882in}}{\pgfqpoint{2.550436in}{1.221058in}}%
\pgfpathcurveto{\pgfqpoint{2.544612in}{1.215234in}}{\pgfqpoint{2.541340in}{1.207334in}}{\pgfqpoint{2.541340in}{1.199097in}}%
\pgfpathcurveto{\pgfqpoint{2.541340in}{1.190861in}}{\pgfqpoint{2.544612in}{1.182961in}}{\pgfqpoint{2.550436in}{1.177137in}}%
\pgfpathcurveto{\pgfqpoint{2.556260in}{1.171313in}}{\pgfqpoint{2.564160in}{1.168041in}}{\pgfqpoint{2.572396in}{1.168041in}}%
\pgfpathclose%
\pgfusepath{stroke,fill}%
\end{pgfscope}%
\begin{pgfscope}%
\pgfpathrectangle{\pgfqpoint{0.100000in}{0.220728in}}{\pgfqpoint{3.696000in}{3.696000in}}%
\pgfusepath{clip}%
\pgfsetbuttcap%
\pgfsetroundjoin%
\definecolor{currentfill}{rgb}{0.121569,0.466667,0.705882}%
\pgfsetfillcolor{currentfill}%
\pgfsetfillopacity{0.943549}%
\pgfsetlinewidth{1.003750pt}%
\definecolor{currentstroke}{rgb}{0.121569,0.466667,0.705882}%
\pgfsetstrokecolor{currentstroke}%
\pgfsetstrokeopacity{0.943549}%
\pgfsetdash{}{0pt}%
\pgfpathmoveto{\pgfqpoint{2.051286in}{0.830495in}}%
\pgfpathcurveto{\pgfqpoint{2.059522in}{0.830495in}}{\pgfqpoint{2.067422in}{0.833767in}}{\pgfqpoint{2.073246in}{0.839591in}}%
\pgfpathcurveto{\pgfqpoint{2.079070in}{0.845415in}}{\pgfqpoint{2.082342in}{0.853315in}}{\pgfqpoint{2.082342in}{0.861551in}}%
\pgfpathcurveto{\pgfqpoint{2.082342in}{0.869787in}}{\pgfqpoint{2.079070in}{0.877688in}}{\pgfqpoint{2.073246in}{0.883511in}}%
\pgfpathcurveto{\pgfqpoint{2.067422in}{0.889335in}}{\pgfqpoint{2.059522in}{0.892608in}}{\pgfqpoint{2.051286in}{0.892608in}}%
\pgfpathcurveto{\pgfqpoint{2.043049in}{0.892608in}}{\pgfqpoint{2.035149in}{0.889335in}}{\pgfqpoint{2.029325in}{0.883511in}}%
\pgfpathcurveto{\pgfqpoint{2.023501in}{0.877688in}}{\pgfqpoint{2.020229in}{0.869787in}}{\pgfqpoint{2.020229in}{0.861551in}}%
\pgfpathcurveto{\pgfqpoint{2.020229in}{0.853315in}}{\pgfqpoint{2.023501in}{0.845415in}}{\pgfqpoint{2.029325in}{0.839591in}}%
\pgfpathcurveto{\pgfqpoint{2.035149in}{0.833767in}}{\pgfqpoint{2.043049in}{0.830495in}}{\pgfqpoint{2.051286in}{0.830495in}}%
\pgfpathclose%
\pgfusepath{stroke,fill}%
\end{pgfscope}%
\begin{pgfscope}%
\pgfpathrectangle{\pgfqpoint{0.100000in}{0.220728in}}{\pgfqpoint{3.696000in}{3.696000in}}%
\pgfusepath{clip}%
\pgfsetbuttcap%
\pgfsetroundjoin%
\definecolor{currentfill}{rgb}{0.121569,0.466667,0.705882}%
\pgfsetfillcolor{currentfill}%
\pgfsetfillopacity{0.943827}%
\pgfsetlinewidth{1.003750pt}%
\definecolor{currentstroke}{rgb}{0.121569,0.466667,0.705882}%
\pgfsetstrokecolor{currentstroke}%
\pgfsetstrokeopacity{0.943827}%
\pgfsetdash{}{0pt}%
\pgfpathmoveto{\pgfqpoint{2.568268in}{1.161630in}}%
\pgfpathcurveto{\pgfqpoint{2.576504in}{1.161630in}}{\pgfqpoint{2.584404in}{1.164903in}}{\pgfqpoint{2.590228in}{1.170727in}}%
\pgfpathcurveto{\pgfqpoint{2.596052in}{1.176550in}}{\pgfqpoint{2.599325in}{1.184451in}}{\pgfqpoint{2.599325in}{1.192687in}}%
\pgfpathcurveto{\pgfqpoint{2.599325in}{1.200923in}}{\pgfqpoint{2.596052in}{1.208823in}}{\pgfqpoint{2.590228in}{1.214647in}}%
\pgfpathcurveto{\pgfqpoint{2.584404in}{1.220471in}}{\pgfqpoint{2.576504in}{1.223743in}}{\pgfqpoint{2.568268in}{1.223743in}}%
\pgfpathcurveto{\pgfqpoint{2.560032in}{1.223743in}}{\pgfqpoint{2.552132in}{1.220471in}}{\pgfqpoint{2.546308in}{1.214647in}}%
\pgfpathcurveto{\pgfqpoint{2.540484in}{1.208823in}}{\pgfqpoint{2.537212in}{1.200923in}}{\pgfqpoint{2.537212in}{1.192687in}}%
\pgfpathcurveto{\pgfqpoint{2.537212in}{1.184451in}}{\pgfqpoint{2.540484in}{1.176550in}}{\pgfqpoint{2.546308in}{1.170727in}}%
\pgfpathcurveto{\pgfqpoint{2.552132in}{1.164903in}}{\pgfqpoint{2.560032in}{1.161630in}}{\pgfqpoint{2.568268in}{1.161630in}}%
\pgfpathclose%
\pgfusepath{stroke,fill}%
\end{pgfscope}%
\begin{pgfscope}%
\pgfpathrectangle{\pgfqpoint{0.100000in}{0.220728in}}{\pgfqpoint{3.696000in}{3.696000in}}%
\pgfusepath{clip}%
\pgfsetbuttcap%
\pgfsetroundjoin%
\definecolor{currentfill}{rgb}{0.121569,0.466667,0.705882}%
\pgfsetfillcolor{currentfill}%
\pgfsetfillopacity{0.945545}%
\pgfsetlinewidth{1.003750pt}%
\definecolor{currentstroke}{rgb}{0.121569,0.466667,0.705882}%
\pgfsetstrokecolor{currentstroke}%
\pgfsetstrokeopacity{0.945545}%
\pgfsetdash{}{0pt}%
\pgfpathmoveto{\pgfqpoint{2.565259in}{1.150376in}}%
\pgfpathcurveto{\pgfqpoint{2.573496in}{1.150376in}}{\pgfqpoint{2.581396in}{1.153648in}}{\pgfqpoint{2.587220in}{1.159472in}}%
\pgfpathcurveto{\pgfqpoint{2.593044in}{1.165296in}}{\pgfqpoint{2.596316in}{1.173196in}}{\pgfqpoint{2.596316in}{1.181433in}}%
\pgfpathcurveto{\pgfqpoint{2.596316in}{1.189669in}}{\pgfqpoint{2.593044in}{1.197569in}}{\pgfqpoint{2.587220in}{1.203393in}}%
\pgfpathcurveto{\pgfqpoint{2.581396in}{1.209217in}}{\pgfqpoint{2.573496in}{1.212489in}}{\pgfqpoint{2.565259in}{1.212489in}}%
\pgfpathcurveto{\pgfqpoint{2.557023in}{1.212489in}}{\pgfqpoint{2.549123in}{1.209217in}}{\pgfqpoint{2.543299in}{1.203393in}}%
\pgfpathcurveto{\pgfqpoint{2.537475in}{1.197569in}}{\pgfqpoint{2.534203in}{1.189669in}}{\pgfqpoint{2.534203in}{1.181433in}}%
\pgfpathcurveto{\pgfqpoint{2.534203in}{1.173196in}}{\pgfqpoint{2.537475in}{1.165296in}}{\pgfqpoint{2.543299in}{1.159472in}}%
\pgfpathcurveto{\pgfqpoint{2.549123in}{1.153648in}}{\pgfqpoint{2.557023in}{1.150376in}}{\pgfqpoint{2.565259in}{1.150376in}}%
\pgfpathclose%
\pgfusepath{stroke,fill}%
\end{pgfscope}%
\begin{pgfscope}%
\pgfpathrectangle{\pgfqpoint{0.100000in}{0.220728in}}{\pgfqpoint{3.696000in}{3.696000in}}%
\pgfusepath{clip}%
\pgfsetbuttcap%
\pgfsetroundjoin%
\definecolor{currentfill}{rgb}{0.121569,0.466667,0.705882}%
\pgfsetfillcolor{currentfill}%
\pgfsetfillopacity{0.945715}%
\pgfsetlinewidth{1.003750pt}%
\definecolor{currentstroke}{rgb}{0.121569,0.466667,0.705882}%
\pgfsetstrokecolor{currentstroke}%
\pgfsetstrokeopacity{0.945715}%
\pgfsetdash{}{0pt}%
\pgfpathmoveto{\pgfqpoint{2.063747in}{0.825520in}}%
\pgfpathcurveto{\pgfqpoint{2.071984in}{0.825520in}}{\pgfqpoint{2.079884in}{0.828792in}}{\pgfqpoint{2.085708in}{0.834616in}}%
\pgfpathcurveto{\pgfqpoint{2.091532in}{0.840440in}}{\pgfqpoint{2.094804in}{0.848340in}}{\pgfqpoint{2.094804in}{0.856576in}}%
\pgfpathcurveto{\pgfqpoint{2.094804in}{0.864812in}}{\pgfqpoint{2.091532in}{0.872712in}}{\pgfqpoint{2.085708in}{0.878536in}}%
\pgfpathcurveto{\pgfqpoint{2.079884in}{0.884360in}}{\pgfqpoint{2.071984in}{0.887633in}}{\pgfqpoint{2.063747in}{0.887633in}}%
\pgfpathcurveto{\pgfqpoint{2.055511in}{0.887633in}}{\pgfqpoint{2.047611in}{0.884360in}}{\pgfqpoint{2.041787in}{0.878536in}}%
\pgfpathcurveto{\pgfqpoint{2.035963in}{0.872712in}}{\pgfqpoint{2.032691in}{0.864812in}}{\pgfqpoint{2.032691in}{0.856576in}}%
\pgfpathcurveto{\pgfqpoint{2.032691in}{0.848340in}}{\pgfqpoint{2.035963in}{0.840440in}}{\pgfqpoint{2.041787in}{0.834616in}}%
\pgfpathcurveto{\pgfqpoint{2.047611in}{0.828792in}}{\pgfqpoint{2.055511in}{0.825520in}}{\pgfqpoint{2.063747in}{0.825520in}}%
\pgfpathclose%
\pgfusepath{stroke,fill}%
\end{pgfscope}%
\begin{pgfscope}%
\pgfpathrectangle{\pgfqpoint{0.100000in}{0.220728in}}{\pgfqpoint{3.696000in}{3.696000in}}%
\pgfusepath{clip}%
\pgfsetbuttcap%
\pgfsetroundjoin%
\definecolor{currentfill}{rgb}{0.121569,0.466667,0.705882}%
\pgfsetfillcolor{currentfill}%
\pgfsetfillopacity{0.946337}%
\pgfsetlinewidth{1.003750pt}%
\definecolor{currentstroke}{rgb}{0.121569,0.466667,0.705882}%
\pgfsetstrokecolor{currentstroke}%
\pgfsetstrokeopacity{0.946337}%
\pgfsetdash{}{0pt}%
\pgfpathmoveto{\pgfqpoint{2.562994in}{1.144060in}}%
\pgfpathcurveto{\pgfqpoint{2.571230in}{1.144060in}}{\pgfqpoint{2.579130in}{1.147333in}}{\pgfqpoint{2.584954in}{1.153156in}}%
\pgfpathcurveto{\pgfqpoint{2.590778in}{1.158980in}}{\pgfqpoint{2.594050in}{1.166880in}}{\pgfqpoint{2.594050in}{1.175117in}}%
\pgfpathcurveto{\pgfqpoint{2.594050in}{1.183353in}}{\pgfqpoint{2.590778in}{1.191253in}}{\pgfqpoint{2.584954in}{1.197077in}}%
\pgfpathcurveto{\pgfqpoint{2.579130in}{1.202901in}}{\pgfqpoint{2.571230in}{1.206173in}}{\pgfqpoint{2.562994in}{1.206173in}}%
\pgfpathcurveto{\pgfqpoint{2.554757in}{1.206173in}}{\pgfqpoint{2.546857in}{1.202901in}}{\pgfqpoint{2.541033in}{1.197077in}}%
\pgfpathcurveto{\pgfqpoint{2.535209in}{1.191253in}}{\pgfqpoint{2.531937in}{1.183353in}}{\pgfqpoint{2.531937in}{1.175117in}}%
\pgfpathcurveto{\pgfqpoint{2.531937in}{1.166880in}}{\pgfqpoint{2.535209in}{1.158980in}}{\pgfqpoint{2.541033in}{1.153156in}}%
\pgfpathcurveto{\pgfqpoint{2.546857in}{1.147333in}}{\pgfqpoint{2.554757in}{1.144060in}}{\pgfqpoint{2.562994in}{1.144060in}}%
\pgfpathclose%
\pgfusepath{stroke,fill}%
\end{pgfscope}%
\begin{pgfscope}%
\pgfpathrectangle{\pgfqpoint{0.100000in}{0.220728in}}{\pgfqpoint{3.696000in}{3.696000in}}%
\pgfusepath{clip}%
\pgfsetbuttcap%
\pgfsetroundjoin%
\definecolor{currentfill}{rgb}{0.121569,0.466667,0.705882}%
\pgfsetfillcolor{currentfill}%
\pgfsetfillopacity{0.946703}%
\pgfsetlinewidth{1.003750pt}%
\definecolor{currentstroke}{rgb}{0.121569,0.466667,0.705882}%
\pgfsetstrokecolor{currentstroke}%
\pgfsetstrokeopacity{0.946703}%
\pgfsetdash{}{0pt}%
\pgfpathmoveto{\pgfqpoint{2.561139in}{1.141143in}}%
\pgfpathcurveto{\pgfqpoint{2.569375in}{1.141143in}}{\pgfqpoint{2.577276in}{1.144416in}}{\pgfqpoint{2.583099in}{1.150240in}}%
\pgfpathcurveto{\pgfqpoint{2.588923in}{1.156064in}}{\pgfqpoint{2.592196in}{1.163964in}}{\pgfqpoint{2.592196in}{1.172200in}}%
\pgfpathcurveto{\pgfqpoint{2.592196in}{1.180436in}}{\pgfqpoint{2.588923in}{1.188336in}}{\pgfqpoint{2.583099in}{1.194160in}}%
\pgfpathcurveto{\pgfqpoint{2.577276in}{1.199984in}}{\pgfqpoint{2.569375in}{1.203256in}}{\pgfqpoint{2.561139in}{1.203256in}}%
\pgfpathcurveto{\pgfqpoint{2.552903in}{1.203256in}}{\pgfqpoint{2.545003in}{1.199984in}}{\pgfqpoint{2.539179in}{1.194160in}}%
\pgfpathcurveto{\pgfqpoint{2.533355in}{1.188336in}}{\pgfqpoint{2.530083in}{1.180436in}}{\pgfqpoint{2.530083in}{1.172200in}}%
\pgfpathcurveto{\pgfqpoint{2.530083in}{1.163964in}}{\pgfqpoint{2.533355in}{1.156064in}}{\pgfqpoint{2.539179in}{1.150240in}}%
\pgfpathcurveto{\pgfqpoint{2.545003in}{1.144416in}}{\pgfqpoint{2.552903in}{1.141143in}}{\pgfqpoint{2.561139in}{1.141143in}}%
\pgfpathclose%
\pgfusepath{stroke,fill}%
\end{pgfscope}%
\begin{pgfscope}%
\pgfpathrectangle{\pgfqpoint{0.100000in}{0.220728in}}{\pgfqpoint{3.696000in}{3.696000in}}%
\pgfusepath{clip}%
\pgfsetbuttcap%
\pgfsetroundjoin%
\definecolor{currentfill}{rgb}{0.121569,0.466667,0.705882}%
\pgfsetfillcolor{currentfill}%
\pgfsetfillopacity{0.947389}%
\pgfsetlinewidth{1.003750pt}%
\definecolor{currentstroke}{rgb}{0.121569,0.466667,0.705882}%
\pgfsetstrokecolor{currentstroke}%
\pgfsetstrokeopacity{0.947389}%
\pgfsetdash{}{0pt}%
\pgfpathmoveto{\pgfqpoint{2.559888in}{1.135611in}}%
\pgfpathcurveto{\pgfqpoint{2.568124in}{1.135611in}}{\pgfqpoint{2.576024in}{1.138883in}}{\pgfqpoint{2.581848in}{1.144707in}}%
\pgfpathcurveto{\pgfqpoint{2.587672in}{1.150531in}}{\pgfqpoint{2.590944in}{1.158431in}}{\pgfqpoint{2.590944in}{1.166667in}}%
\pgfpathcurveto{\pgfqpoint{2.590944in}{1.174903in}}{\pgfqpoint{2.587672in}{1.182804in}}{\pgfqpoint{2.581848in}{1.188627in}}%
\pgfpathcurveto{\pgfqpoint{2.576024in}{1.194451in}}{\pgfqpoint{2.568124in}{1.197724in}}{\pgfqpoint{2.559888in}{1.197724in}}%
\pgfpathcurveto{\pgfqpoint{2.551651in}{1.197724in}}{\pgfqpoint{2.543751in}{1.194451in}}{\pgfqpoint{2.537928in}{1.188627in}}%
\pgfpathcurveto{\pgfqpoint{2.532104in}{1.182804in}}{\pgfqpoint{2.528831in}{1.174903in}}{\pgfqpoint{2.528831in}{1.166667in}}%
\pgfpathcurveto{\pgfqpoint{2.528831in}{1.158431in}}{\pgfqpoint{2.532104in}{1.150531in}}{\pgfqpoint{2.537928in}{1.144707in}}%
\pgfpathcurveto{\pgfqpoint{2.543751in}{1.138883in}}{\pgfqpoint{2.551651in}{1.135611in}}{\pgfqpoint{2.559888in}{1.135611in}}%
\pgfpathclose%
\pgfusepath{stroke,fill}%
\end{pgfscope}%
\begin{pgfscope}%
\pgfpathrectangle{\pgfqpoint{0.100000in}{0.220728in}}{\pgfqpoint{3.696000in}{3.696000in}}%
\pgfusepath{clip}%
\pgfsetbuttcap%
\pgfsetroundjoin%
\definecolor{currentfill}{rgb}{0.121569,0.466667,0.705882}%
\pgfsetfillcolor{currentfill}%
\pgfsetfillopacity{0.947983}%
\pgfsetlinewidth{1.003750pt}%
\definecolor{currentstroke}{rgb}{0.121569,0.466667,0.705882}%
\pgfsetstrokecolor{currentstroke}%
\pgfsetstrokeopacity{0.947983}%
\pgfsetdash{}{0pt}%
\pgfpathmoveto{\pgfqpoint{2.075019in}{0.819065in}}%
\pgfpathcurveto{\pgfqpoint{2.083256in}{0.819065in}}{\pgfqpoint{2.091156in}{0.822337in}}{\pgfqpoint{2.096980in}{0.828161in}}%
\pgfpathcurveto{\pgfqpoint{2.102803in}{0.833985in}}{\pgfqpoint{2.106076in}{0.841885in}}{\pgfqpoint{2.106076in}{0.850121in}}%
\pgfpathcurveto{\pgfqpoint{2.106076in}{0.858357in}}{\pgfqpoint{2.102803in}{0.866257in}}{\pgfqpoint{2.096980in}{0.872081in}}%
\pgfpathcurveto{\pgfqpoint{2.091156in}{0.877905in}}{\pgfqpoint{2.083256in}{0.881178in}}{\pgfqpoint{2.075019in}{0.881178in}}%
\pgfpathcurveto{\pgfqpoint{2.066783in}{0.881178in}}{\pgfqpoint{2.058883in}{0.877905in}}{\pgfqpoint{2.053059in}{0.872081in}}%
\pgfpathcurveto{\pgfqpoint{2.047235in}{0.866257in}}{\pgfqpoint{2.043963in}{0.858357in}}{\pgfqpoint{2.043963in}{0.850121in}}%
\pgfpathcurveto{\pgfqpoint{2.043963in}{0.841885in}}{\pgfqpoint{2.047235in}{0.833985in}}{\pgfqpoint{2.053059in}{0.828161in}}%
\pgfpathcurveto{\pgfqpoint{2.058883in}{0.822337in}}{\pgfqpoint{2.066783in}{0.819065in}}{\pgfqpoint{2.075019in}{0.819065in}}%
\pgfpathclose%
\pgfusepath{stroke,fill}%
\end{pgfscope}%
\begin{pgfscope}%
\pgfpathrectangle{\pgfqpoint{0.100000in}{0.220728in}}{\pgfqpoint{3.696000in}{3.696000in}}%
\pgfusepath{clip}%
\pgfsetbuttcap%
\pgfsetroundjoin%
\definecolor{currentfill}{rgb}{0.121569,0.466667,0.705882}%
\pgfsetfillcolor{currentfill}%
\pgfsetfillopacity{0.948086}%
\pgfsetlinewidth{1.003750pt}%
\definecolor{currentstroke}{rgb}{0.121569,0.466667,0.705882}%
\pgfsetstrokecolor{currentstroke}%
\pgfsetstrokeopacity{0.948086}%
\pgfsetdash{}{0pt}%
\pgfpathmoveto{\pgfqpoint{2.555566in}{1.128367in}}%
\pgfpathcurveto{\pgfqpoint{2.563802in}{1.128367in}}{\pgfqpoint{2.571702in}{1.131639in}}{\pgfqpoint{2.577526in}{1.137463in}}%
\pgfpathcurveto{\pgfqpoint{2.583350in}{1.143287in}}{\pgfqpoint{2.586622in}{1.151187in}}{\pgfqpoint{2.586622in}{1.159423in}}%
\pgfpathcurveto{\pgfqpoint{2.586622in}{1.167659in}}{\pgfqpoint{2.583350in}{1.175559in}}{\pgfqpoint{2.577526in}{1.181383in}}%
\pgfpathcurveto{\pgfqpoint{2.571702in}{1.187207in}}{\pgfqpoint{2.563802in}{1.190480in}}{\pgfqpoint{2.555566in}{1.190480in}}%
\pgfpathcurveto{\pgfqpoint{2.547329in}{1.190480in}}{\pgfqpoint{2.539429in}{1.187207in}}{\pgfqpoint{2.533605in}{1.181383in}}%
\pgfpathcurveto{\pgfqpoint{2.527781in}{1.175559in}}{\pgfqpoint{2.524509in}{1.167659in}}{\pgfqpoint{2.524509in}{1.159423in}}%
\pgfpathcurveto{\pgfqpoint{2.524509in}{1.151187in}}{\pgfqpoint{2.527781in}{1.143287in}}{\pgfqpoint{2.533605in}{1.137463in}}%
\pgfpathcurveto{\pgfqpoint{2.539429in}{1.131639in}}{\pgfqpoint{2.547329in}{1.128367in}}{\pgfqpoint{2.555566in}{1.128367in}}%
\pgfpathclose%
\pgfusepath{stroke,fill}%
\end{pgfscope}%
\begin{pgfscope}%
\pgfpathrectangle{\pgfqpoint{0.100000in}{0.220728in}}{\pgfqpoint{3.696000in}{3.696000in}}%
\pgfusepath{clip}%
\pgfsetbuttcap%
\pgfsetroundjoin%
\definecolor{currentfill}{rgb}{0.121569,0.466667,0.705882}%
\pgfsetfillcolor{currentfill}%
\pgfsetfillopacity{0.948836}%
\pgfsetlinewidth{1.003750pt}%
\definecolor{currentstroke}{rgb}{0.121569,0.466667,0.705882}%
\pgfsetstrokecolor{currentstroke}%
\pgfsetstrokeopacity{0.948836}%
\pgfsetdash{}{0pt}%
\pgfpathmoveto{\pgfqpoint{2.550357in}{1.120476in}}%
\pgfpathcurveto{\pgfqpoint{2.558593in}{1.120476in}}{\pgfqpoint{2.566493in}{1.123748in}}{\pgfqpoint{2.572317in}{1.129572in}}%
\pgfpathcurveto{\pgfqpoint{2.578141in}{1.135396in}}{\pgfqpoint{2.581414in}{1.143296in}}{\pgfqpoint{2.581414in}{1.151532in}}%
\pgfpathcurveto{\pgfqpoint{2.581414in}{1.159769in}}{\pgfqpoint{2.578141in}{1.167669in}}{\pgfqpoint{2.572317in}{1.173493in}}%
\pgfpathcurveto{\pgfqpoint{2.566493in}{1.179317in}}{\pgfqpoint{2.558593in}{1.182589in}}{\pgfqpoint{2.550357in}{1.182589in}}%
\pgfpathcurveto{\pgfqpoint{2.542121in}{1.182589in}}{\pgfqpoint{2.534221in}{1.179317in}}{\pgfqpoint{2.528397in}{1.173493in}}%
\pgfpathcurveto{\pgfqpoint{2.522573in}{1.167669in}}{\pgfqpoint{2.519301in}{1.159769in}}{\pgfqpoint{2.519301in}{1.151532in}}%
\pgfpathcurveto{\pgfqpoint{2.519301in}{1.143296in}}{\pgfqpoint{2.522573in}{1.135396in}}{\pgfqpoint{2.528397in}{1.129572in}}%
\pgfpathcurveto{\pgfqpoint{2.534221in}{1.123748in}}{\pgfqpoint{2.542121in}{1.120476in}}{\pgfqpoint{2.550357in}{1.120476in}}%
\pgfpathclose%
\pgfusepath{stroke,fill}%
\end{pgfscope}%
\begin{pgfscope}%
\pgfpathrectangle{\pgfqpoint{0.100000in}{0.220728in}}{\pgfqpoint{3.696000in}{3.696000in}}%
\pgfusepath{clip}%
\pgfsetbuttcap%
\pgfsetroundjoin%
\definecolor{currentfill}{rgb}{0.121569,0.466667,0.705882}%
\pgfsetfillcolor{currentfill}%
\pgfsetfillopacity{0.950227}%
\pgfsetlinewidth{1.003750pt}%
\definecolor{currentstroke}{rgb}{0.121569,0.466667,0.705882}%
\pgfsetstrokecolor{currentstroke}%
\pgfsetstrokeopacity{0.950227}%
\pgfsetdash{}{0pt}%
\pgfpathmoveto{\pgfqpoint{2.547568in}{1.110576in}}%
\pgfpathcurveto{\pgfqpoint{2.555804in}{1.110576in}}{\pgfqpoint{2.563704in}{1.113849in}}{\pgfqpoint{2.569528in}{1.119673in}}%
\pgfpathcurveto{\pgfqpoint{2.575352in}{1.125497in}}{\pgfqpoint{2.578625in}{1.133397in}}{\pgfqpoint{2.578625in}{1.141633in}}%
\pgfpathcurveto{\pgfqpoint{2.578625in}{1.149869in}}{\pgfqpoint{2.575352in}{1.157769in}}{\pgfqpoint{2.569528in}{1.163593in}}%
\pgfpathcurveto{\pgfqpoint{2.563704in}{1.169417in}}{\pgfqpoint{2.555804in}{1.172689in}}{\pgfqpoint{2.547568in}{1.172689in}}%
\pgfpathcurveto{\pgfqpoint{2.539332in}{1.172689in}}{\pgfqpoint{2.531432in}{1.169417in}}{\pgfqpoint{2.525608in}{1.163593in}}%
\pgfpathcurveto{\pgfqpoint{2.519784in}{1.157769in}}{\pgfqpoint{2.516512in}{1.149869in}}{\pgfqpoint{2.516512in}{1.141633in}}%
\pgfpathcurveto{\pgfqpoint{2.516512in}{1.133397in}}{\pgfqpoint{2.519784in}{1.125497in}}{\pgfqpoint{2.525608in}{1.119673in}}%
\pgfpathcurveto{\pgfqpoint{2.531432in}{1.113849in}}{\pgfqpoint{2.539332in}{1.110576in}}{\pgfqpoint{2.547568in}{1.110576in}}%
\pgfpathclose%
\pgfusepath{stroke,fill}%
\end{pgfscope}%
\begin{pgfscope}%
\pgfpathrectangle{\pgfqpoint{0.100000in}{0.220728in}}{\pgfqpoint{3.696000in}{3.696000in}}%
\pgfusepath{clip}%
\pgfsetbuttcap%
\pgfsetroundjoin%
\definecolor{currentfill}{rgb}{0.121569,0.466667,0.705882}%
\pgfsetfillcolor{currentfill}%
\pgfsetfillopacity{0.950474}%
\pgfsetlinewidth{1.003750pt}%
\definecolor{currentstroke}{rgb}{0.121569,0.466667,0.705882}%
\pgfsetstrokecolor{currentstroke}%
\pgfsetstrokeopacity{0.950474}%
\pgfsetdash{}{0pt}%
\pgfpathmoveto{\pgfqpoint{2.085826in}{0.815041in}}%
\pgfpathcurveto{\pgfqpoint{2.094062in}{0.815041in}}{\pgfqpoint{2.101962in}{0.818314in}}{\pgfqpoint{2.107786in}{0.824138in}}%
\pgfpathcurveto{\pgfqpoint{2.113610in}{0.829962in}}{\pgfqpoint{2.116883in}{0.837862in}}{\pgfqpoint{2.116883in}{0.846098in}}%
\pgfpathcurveto{\pgfqpoint{2.116883in}{0.854334in}}{\pgfqpoint{2.113610in}{0.862234in}}{\pgfqpoint{2.107786in}{0.868058in}}%
\pgfpathcurveto{\pgfqpoint{2.101962in}{0.873882in}}{\pgfqpoint{2.094062in}{0.877154in}}{\pgfqpoint{2.085826in}{0.877154in}}%
\pgfpathcurveto{\pgfqpoint{2.077590in}{0.877154in}}{\pgfqpoint{2.069690in}{0.873882in}}{\pgfqpoint{2.063866in}{0.868058in}}%
\pgfpathcurveto{\pgfqpoint{2.058042in}{0.862234in}}{\pgfqpoint{2.054770in}{0.854334in}}{\pgfqpoint{2.054770in}{0.846098in}}%
\pgfpathcurveto{\pgfqpoint{2.054770in}{0.837862in}}{\pgfqpoint{2.058042in}{0.829962in}}{\pgfqpoint{2.063866in}{0.824138in}}%
\pgfpathcurveto{\pgfqpoint{2.069690in}{0.818314in}}{\pgfqpoint{2.077590in}{0.815041in}}{\pgfqpoint{2.085826in}{0.815041in}}%
\pgfpathclose%
\pgfusepath{stroke,fill}%
\end{pgfscope}%
\begin{pgfscope}%
\pgfpathrectangle{\pgfqpoint{0.100000in}{0.220728in}}{\pgfqpoint{3.696000in}{3.696000in}}%
\pgfusepath{clip}%
\pgfsetbuttcap%
\pgfsetroundjoin%
\definecolor{currentfill}{rgb}{0.121569,0.466667,0.705882}%
\pgfsetfillcolor{currentfill}%
\pgfsetfillopacity{0.951763}%
\pgfsetlinewidth{1.003750pt}%
\definecolor{currentstroke}{rgb}{0.121569,0.466667,0.705882}%
\pgfsetstrokecolor{currentstroke}%
\pgfsetstrokeopacity{0.951763}%
\pgfsetdash{}{0pt}%
\pgfpathmoveto{\pgfqpoint{2.542875in}{1.099403in}}%
\pgfpathcurveto{\pgfqpoint{2.551111in}{1.099403in}}{\pgfqpoint{2.559011in}{1.102675in}}{\pgfqpoint{2.564835in}{1.108499in}}%
\pgfpathcurveto{\pgfqpoint{2.570659in}{1.114323in}}{\pgfqpoint{2.573931in}{1.122223in}}{\pgfqpoint{2.573931in}{1.130459in}}%
\pgfpathcurveto{\pgfqpoint{2.573931in}{1.138695in}}{\pgfqpoint{2.570659in}{1.146596in}}{\pgfqpoint{2.564835in}{1.152419in}}%
\pgfpathcurveto{\pgfqpoint{2.559011in}{1.158243in}}{\pgfqpoint{2.551111in}{1.161516in}}{\pgfqpoint{2.542875in}{1.161516in}}%
\pgfpathcurveto{\pgfqpoint{2.534639in}{1.161516in}}{\pgfqpoint{2.526739in}{1.158243in}}{\pgfqpoint{2.520915in}{1.152419in}}%
\pgfpathcurveto{\pgfqpoint{2.515091in}{1.146596in}}{\pgfqpoint{2.511818in}{1.138695in}}{\pgfqpoint{2.511818in}{1.130459in}}%
\pgfpathcurveto{\pgfqpoint{2.511818in}{1.122223in}}{\pgfqpoint{2.515091in}{1.114323in}}{\pgfqpoint{2.520915in}{1.108499in}}%
\pgfpathcurveto{\pgfqpoint{2.526739in}{1.102675in}}{\pgfqpoint{2.534639in}{1.099403in}}{\pgfqpoint{2.542875in}{1.099403in}}%
\pgfpathclose%
\pgfusepath{stroke,fill}%
\end{pgfscope}%
\begin{pgfscope}%
\pgfpathrectangle{\pgfqpoint{0.100000in}{0.220728in}}{\pgfqpoint{3.696000in}{3.696000in}}%
\pgfusepath{clip}%
\pgfsetbuttcap%
\pgfsetroundjoin%
\definecolor{currentfill}{rgb}{0.121569,0.466667,0.705882}%
\pgfsetfillcolor{currentfill}%
\pgfsetfillopacity{0.952184}%
\pgfsetlinewidth{1.003750pt}%
\definecolor{currentstroke}{rgb}{0.121569,0.466667,0.705882}%
\pgfsetstrokecolor{currentstroke}%
\pgfsetstrokeopacity{0.952184}%
\pgfsetdash{}{0pt}%
\pgfpathmoveto{\pgfqpoint{2.094437in}{0.809811in}}%
\pgfpathcurveto{\pgfqpoint{2.102673in}{0.809811in}}{\pgfqpoint{2.110573in}{0.813084in}}{\pgfqpoint{2.116397in}{0.818908in}}%
\pgfpathcurveto{\pgfqpoint{2.122221in}{0.824732in}}{\pgfqpoint{2.125493in}{0.832632in}}{\pgfqpoint{2.125493in}{0.840868in}}%
\pgfpathcurveto{\pgfqpoint{2.125493in}{0.849104in}}{\pgfqpoint{2.122221in}{0.857004in}}{\pgfqpoint{2.116397in}{0.862828in}}%
\pgfpathcurveto{\pgfqpoint{2.110573in}{0.868652in}}{\pgfqpoint{2.102673in}{0.871924in}}{\pgfqpoint{2.094437in}{0.871924in}}%
\pgfpathcurveto{\pgfqpoint{2.086201in}{0.871924in}}{\pgfqpoint{2.078301in}{0.868652in}}{\pgfqpoint{2.072477in}{0.862828in}}%
\pgfpathcurveto{\pgfqpoint{2.066653in}{0.857004in}}{\pgfqpoint{2.063380in}{0.849104in}}{\pgfqpoint{2.063380in}{0.840868in}}%
\pgfpathcurveto{\pgfqpoint{2.063380in}{0.832632in}}{\pgfqpoint{2.066653in}{0.824732in}}{\pgfqpoint{2.072477in}{0.818908in}}%
\pgfpathcurveto{\pgfqpoint{2.078301in}{0.813084in}}{\pgfqpoint{2.086201in}{0.809811in}}{\pgfqpoint{2.094437in}{0.809811in}}%
\pgfpathclose%
\pgfusepath{stroke,fill}%
\end{pgfscope}%
\begin{pgfscope}%
\pgfpathrectangle{\pgfqpoint{0.100000in}{0.220728in}}{\pgfqpoint{3.696000in}{3.696000in}}%
\pgfusepath{clip}%
\pgfsetbuttcap%
\pgfsetroundjoin%
\definecolor{currentfill}{rgb}{0.121569,0.466667,0.705882}%
\pgfsetfillcolor{currentfill}%
\pgfsetfillopacity{0.953127}%
\pgfsetlinewidth{1.003750pt}%
\definecolor{currentstroke}{rgb}{0.121569,0.466667,0.705882}%
\pgfsetstrokecolor{currentstroke}%
\pgfsetstrokeopacity{0.953127}%
\pgfsetdash{}{0pt}%
\pgfpathmoveto{\pgfqpoint{2.535905in}{1.089089in}}%
\pgfpathcurveto{\pgfqpoint{2.544141in}{1.089089in}}{\pgfqpoint{2.552041in}{1.092362in}}{\pgfqpoint{2.557865in}{1.098185in}}%
\pgfpathcurveto{\pgfqpoint{2.563689in}{1.104009in}}{\pgfqpoint{2.566961in}{1.111909in}}{\pgfqpoint{2.566961in}{1.120146in}}%
\pgfpathcurveto{\pgfqpoint{2.566961in}{1.128382in}}{\pgfqpoint{2.563689in}{1.136282in}}{\pgfqpoint{2.557865in}{1.142106in}}%
\pgfpathcurveto{\pgfqpoint{2.552041in}{1.147930in}}{\pgfqpoint{2.544141in}{1.151202in}}{\pgfqpoint{2.535905in}{1.151202in}}%
\pgfpathcurveto{\pgfqpoint{2.527668in}{1.151202in}}{\pgfqpoint{2.519768in}{1.147930in}}{\pgfqpoint{2.513944in}{1.142106in}}%
\pgfpathcurveto{\pgfqpoint{2.508120in}{1.136282in}}{\pgfqpoint{2.504848in}{1.128382in}}{\pgfqpoint{2.504848in}{1.120146in}}%
\pgfpathcurveto{\pgfqpoint{2.504848in}{1.111909in}}{\pgfqpoint{2.508120in}{1.104009in}}{\pgfqpoint{2.513944in}{1.098185in}}%
\pgfpathcurveto{\pgfqpoint{2.519768in}{1.092362in}}{\pgfqpoint{2.527668in}{1.089089in}}{\pgfqpoint{2.535905in}{1.089089in}}%
\pgfpathclose%
\pgfusepath{stroke,fill}%
\end{pgfscope}%
\begin{pgfscope}%
\pgfpathrectangle{\pgfqpoint{0.100000in}{0.220728in}}{\pgfqpoint{3.696000in}{3.696000in}}%
\pgfusepath{clip}%
\pgfsetbuttcap%
\pgfsetroundjoin%
\definecolor{currentfill}{rgb}{0.121569,0.466667,0.705882}%
\pgfsetfillcolor{currentfill}%
\pgfsetfillopacity{0.953705}%
\pgfsetlinewidth{1.003750pt}%
\definecolor{currentstroke}{rgb}{0.121569,0.466667,0.705882}%
\pgfsetstrokecolor{currentstroke}%
\pgfsetstrokeopacity{0.953705}%
\pgfsetdash{}{0pt}%
\pgfpathmoveto{\pgfqpoint{2.100503in}{0.807483in}}%
\pgfpathcurveto{\pgfqpoint{2.108739in}{0.807483in}}{\pgfqpoint{2.116639in}{0.810755in}}{\pgfqpoint{2.122463in}{0.816579in}}%
\pgfpathcurveto{\pgfqpoint{2.128287in}{0.822403in}}{\pgfqpoint{2.131559in}{0.830303in}}{\pgfqpoint{2.131559in}{0.838540in}}%
\pgfpathcurveto{\pgfqpoint{2.131559in}{0.846776in}}{\pgfqpoint{2.128287in}{0.854676in}}{\pgfqpoint{2.122463in}{0.860500in}}%
\pgfpathcurveto{\pgfqpoint{2.116639in}{0.866324in}}{\pgfqpoint{2.108739in}{0.869596in}}{\pgfqpoint{2.100503in}{0.869596in}}%
\pgfpathcurveto{\pgfqpoint{2.092267in}{0.869596in}}{\pgfqpoint{2.084366in}{0.866324in}}{\pgfqpoint{2.078543in}{0.860500in}}%
\pgfpathcurveto{\pgfqpoint{2.072719in}{0.854676in}}{\pgfqpoint{2.069446in}{0.846776in}}{\pgfqpoint{2.069446in}{0.838540in}}%
\pgfpathcurveto{\pgfqpoint{2.069446in}{0.830303in}}{\pgfqpoint{2.072719in}{0.822403in}}{\pgfqpoint{2.078543in}{0.816579in}}%
\pgfpathcurveto{\pgfqpoint{2.084366in}{0.810755in}}{\pgfqpoint{2.092267in}{0.807483in}}{\pgfqpoint{2.100503in}{0.807483in}}%
\pgfpathclose%
\pgfusepath{stroke,fill}%
\end{pgfscope}%
\begin{pgfscope}%
\pgfpathrectangle{\pgfqpoint{0.100000in}{0.220728in}}{\pgfqpoint{3.696000in}{3.696000in}}%
\pgfusepath{clip}%
\pgfsetbuttcap%
\pgfsetroundjoin%
\definecolor{currentfill}{rgb}{0.121569,0.466667,0.705882}%
\pgfsetfillcolor{currentfill}%
\pgfsetfillopacity{0.955402}%
\pgfsetlinewidth{1.003750pt}%
\definecolor{currentstroke}{rgb}{0.121569,0.466667,0.705882}%
\pgfsetstrokecolor{currentstroke}%
\pgfsetstrokeopacity{0.955402}%
\pgfsetdash{}{0pt}%
\pgfpathmoveto{\pgfqpoint{2.112230in}{0.801542in}}%
\pgfpathcurveto{\pgfqpoint{2.120466in}{0.801542in}}{\pgfqpoint{2.128366in}{0.804814in}}{\pgfqpoint{2.134190in}{0.810638in}}%
\pgfpathcurveto{\pgfqpoint{2.140014in}{0.816462in}}{\pgfqpoint{2.143286in}{0.824362in}}{\pgfqpoint{2.143286in}{0.832598in}}%
\pgfpathcurveto{\pgfqpoint{2.143286in}{0.840834in}}{\pgfqpoint{2.140014in}{0.848735in}}{\pgfqpoint{2.134190in}{0.854558in}}%
\pgfpathcurveto{\pgfqpoint{2.128366in}{0.860382in}}{\pgfqpoint{2.120466in}{0.863655in}}{\pgfqpoint{2.112230in}{0.863655in}}%
\pgfpathcurveto{\pgfqpoint{2.103994in}{0.863655in}}{\pgfqpoint{2.096093in}{0.860382in}}{\pgfqpoint{2.090270in}{0.854558in}}%
\pgfpathcurveto{\pgfqpoint{2.084446in}{0.848735in}}{\pgfqpoint{2.081173in}{0.840834in}}{\pgfqpoint{2.081173in}{0.832598in}}%
\pgfpathcurveto{\pgfqpoint{2.081173in}{0.824362in}}{\pgfqpoint{2.084446in}{0.816462in}}{\pgfqpoint{2.090270in}{0.810638in}}%
\pgfpathcurveto{\pgfqpoint{2.096093in}{0.804814in}}{\pgfqpoint{2.103994in}{0.801542in}}{\pgfqpoint{2.112230in}{0.801542in}}%
\pgfpathclose%
\pgfusepath{stroke,fill}%
\end{pgfscope}%
\begin{pgfscope}%
\pgfpathrectangle{\pgfqpoint{0.100000in}{0.220728in}}{\pgfqpoint{3.696000in}{3.696000in}}%
\pgfusepath{clip}%
\pgfsetbuttcap%
\pgfsetroundjoin%
\definecolor{currentfill}{rgb}{0.121569,0.466667,0.705882}%
\pgfsetfillcolor{currentfill}%
\pgfsetfillopacity{0.955677}%
\pgfsetlinewidth{1.003750pt}%
\definecolor{currentstroke}{rgb}{0.121569,0.466667,0.705882}%
\pgfsetstrokecolor{currentstroke}%
\pgfsetstrokeopacity{0.955677}%
\pgfsetdash{}{0pt}%
\pgfpathmoveto{\pgfqpoint{2.532597in}{1.071124in}}%
\pgfpathcurveto{\pgfqpoint{2.540834in}{1.071124in}}{\pgfqpoint{2.548734in}{1.074397in}}{\pgfqpoint{2.554558in}{1.080220in}}%
\pgfpathcurveto{\pgfqpoint{2.560382in}{1.086044in}}{\pgfqpoint{2.563654in}{1.093944in}}{\pgfqpoint{2.563654in}{1.102181in}}%
\pgfpathcurveto{\pgfqpoint{2.563654in}{1.110417in}}{\pgfqpoint{2.560382in}{1.118317in}}{\pgfqpoint{2.554558in}{1.124141in}}%
\pgfpathcurveto{\pgfqpoint{2.548734in}{1.129965in}}{\pgfqpoint{2.540834in}{1.133237in}}{\pgfqpoint{2.532597in}{1.133237in}}%
\pgfpathcurveto{\pgfqpoint{2.524361in}{1.133237in}}{\pgfqpoint{2.516461in}{1.129965in}}{\pgfqpoint{2.510637in}{1.124141in}}%
\pgfpathcurveto{\pgfqpoint{2.504813in}{1.118317in}}{\pgfqpoint{2.501541in}{1.110417in}}{\pgfqpoint{2.501541in}{1.102181in}}%
\pgfpathcurveto{\pgfqpoint{2.501541in}{1.093944in}}{\pgfqpoint{2.504813in}{1.086044in}}{\pgfqpoint{2.510637in}{1.080220in}}%
\pgfpathcurveto{\pgfqpoint{2.516461in}{1.074397in}}{\pgfqpoint{2.524361in}{1.071124in}}{\pgfqpoint{2.532597in}{1.071124in}}%
\pgfpathclose%
\pgfusepath{stroke,fill}%
\end{pgfscope}%
\begin{pgfscope}%
\pgfpathrectangle{\pgfqpoint{0.100000in}{0.220728in}}{\pgfqpoint{3.696000in}{3.696000in}}%
\pgfusepath{clip}%
\pgfsetbuttcap%
\pgfsetroundjoin%
\definecolor{currentfill}{rgb}{0.121569,0.466667,0.705882}%
\pgfsetfillcolor{currentfill}%
\pgfsetfillopacity{0.956262}%
\pgfsetlinewidth{1.003750pt}%
\definecolor{currentstroke}{rgb}{0.121569,0.466667,0.705882}%
\pgfsetstrokecolor{currentstroke}%
\pgfsetstrokeopacity{0.956262}%
\pgfsetdash{}{0pt}%
\pgfpathmoveto{\pgfqpoint{2.518211in}{1.053632in}}%
\pgfpathcurveto{\pgfqpoint{2.526447in}{1.053632in}}{\pgfqpoint{2.534347in}{1.056904in}}{\pgfqpoint{2.540171in}{1.062728in}}%
\pgfpathcurveto{\pgfqpoint{2.545995in}{1.068552in}}{\pgfqpoint{2.549267in}{1.076452in}}{\pgfqpoint{2.549267in}{1.084688in}}%
\pgfpathcurveto{\pgfqpoint{2.549267in}{1.092924in}}{\pgfqpoint{2.545995in}{1.100824in}}{\pgfqpoint{2.540171in}{1.106648in}}%
\pgfpathcurveto{\pgfqpoint{2.534347in}{1.112472in}}{\pgfqpoint{2.526447in}{1.115745in}}{\pgfqpoint{2.518211in}{1.115745in}}%
\pgfpathcurveto{\pgfqpoint{2.509974in}{1.115745in}}{\pgfqpoint{2.502074in}{1.112472in}}{\pgfqpoint{2.496250in}{1.106648in}}%
\pgfpathcurveto{\pgfqpoint{2.490426in}{1.100824in}}{\pgfqpoint{2.487154in}{1.092924in}}{\pgfqpoint{2.487154in}{1.084688in}}%
\pgfpathcurveto{\pgfqpoint{2.487154in}{1.076452in}}{\pgfqpoint{2.490426in}{1.068552in}}{\pgfqpoint{2.496250in}{1.062728in}}%
\pgfpathcurveto{\pgfqpoint{2.502074in}{1.056904in}}{\pgfqpoint{2.509974in}{1.053632in}}{\pgfqpoint{2.518211in}{1.053632in}}%
\pgfpathclose%
\pgfusepath{stroke,fill}%
\end{pgfscope}%
\begin{pgfscope}%
\pgfpathrectangle{\pgfqpoint{0.100000in}{0.220728in}}{\pgfqpoint{3.696000in}{3.696000in}}%
\pgfusepath{clip}%
\pgfsetbuttcap%
\pgfsetroundjoin%
\definecolor{currentfill}{rgb}{0.121569,0.466667,0.705882}%
\pgfsetfillcolor{currentfill}%
\pgfsetfillopacity{0.957525}%
\pgfsetlinewidth{1.003750pt}%
\definecolor{currentstroke}{rgb}{0.121569,0.466667,0.705882}%
\pgfsetstrokecolor{currentstroke}%
\pgfsetstrokeopacity{0.957525}%
\pgfsetdash{}{0pt}%
\pgfpathmoveto{\pgfqpoint{2.122367in}{0.796486in}}%
\pgfpathcurveto{\pgfqpoint{2.130603in}{0.796486in}}{\pgfqpoint{2.138503in}{0.799758in}}{\pgfqpoint{2.144327in}{0.805582in}}%
\pgfpathcurveto{\pgfqpoint{2.150151in}{0.811406in}}{\pgfqpoint{2.153424in}{0.819306in}}{\pgfqpoint{2.153424in}{0.827543in}}%
\pgfpathcurveto{\pgfqpoint{2.153424in}{0.835779in}}{\pgfqpoint{2.150151in}{0.843679in}}{\pgfqpoint{2.144327in}{0.849503in}}%
\pgfpathcurveto{\pgfqpoint{2.138503in}{0.855327in}}{\pgfqpoint{2.130603in}{0.858599in}}{\pgfqpoint{2.122367in}{0.858599in}}%
\pgfpathcurveto{\pgfqpoint{2.114131in}{0.858599in}}{\pgfqpoint{2.106231in}{0.855327in}}{\pgfqpoint{2.100407in}{0.849503in}}%
\pgfpathcurveto{\pgfqpoint{2.094583in}{0.843679in}}{\pgfqpoint{2.091311in}{0.835779in}}{\pgfqpoint{2.091311in}{0.827543in}}%
\pgfpathcurveto{\pgfqpoint{2.091311in}{0.819306in}}{\pgfqpoint{2.094583in}{0.811406in}}{\pgfqpoint{2.100407in}{0.805582in}}%
\pgfpathcurveto{\pgfqpoint{2.106231in}{0.799758in}}{\pgfqpoint{2.114131in}{0.796486in}}{\pgfqpoint{2.122367in}{0.796486in}}%
\pgfpathclose%
\pgfusepath{stroke,fill}%
\end{pgfscope}%
\begin{pgfscope}%
\pgfpathrectangle{\pgfqpoint{0.100000in}{0.220728in}}{\pgfqpoint{3.696000in}{3.696000in}}%
\pgfusepath{clip}%
\pgfsetbuttcap%
\pgfsetroundjoin%
\definecolor{currentfill}{rgb}{0.121569,0.466667,0.705882}%
\pgfsetfillcolor{currentfill}%
\pgfsetfillopacity{0.959635}%
\pgfsetlinewidth{1.003750pt}%
\definecolor{currentstroke}{rgb}{0.121569,0.466667,0.705882}%
\pgfsetstrokecolor{currentstroke}%
\pgfsetstrokeopacity{0.959635}%
\pgfsetdash{}{0pt}%
\pgfpathmoveto{\pgfqpoint{2.131781in}{0.793028in}}%
\pgfpathcurveto{\pgfqpoint{2.140017in}{0.793028in}}{\pgfqpoint{2.147917in}{0.796300in}}{\pgfqpoint{2.153741in}{0.802124in}}%
\pgfpathcurveto{\pgfqpoint{2.159565in}{0.807948in}}{\pgfqpoint{2.162837in}{0.815848in}}{\pgfqpoint{2.162837in}{0.824084in}}%
\pgfpathcurveto{\pgfqpoint{2.162837in}{0.832320in}}{\pgfqpoint{2.159565in}{0.840221in}}{\pgfqpoint{2.153741in}{0.846044in}}%
\pgfpathcurveto{\pgfqpoint{2.147917in}{0.851868in}}{\pgfqpoint{2.140017in}{0.855141in}}{\pgfqpoint{2.131781in}{0.855141in}}%
\pgfpathcurveto{\pgfqpoint{2.123545in}{0.855141in}}{\pgfqpoint{2.115645in}{0.851868in}}{\pgfqpoint{2.109821in}{0.846044in}}%
\pgfpathcurveto{\pgfqpoint{2.103997in}{0.840221in}}{\pgfqpoint{2.100724in}{0.832320in}}{\pgfqpoint{2.100724in}{0.824084in}}%
\pgfpathcurveto{\pgfqpoint{2.100724in}{0.815848in}}{\pgfqpoint{2.103997in}{0.807948in}}{\pgfqpoint{2.109821in}{0.802124in}}%
\pgfpathcurveto{\pgfqpoint{2.115645in}{0.796300in}}{\pgfqpoint{2.123545in}{0.793028in}}{\pgfqpoint{2.131781in}{0.793028in}}%
\pgfpathclose%
\pgfusepath{stroke,fill}%
\end{pgfscope}%
\begin{pgfscope}%
\pgfpathrectangle{\pgfqpoint{0.100000in}{0.220728in}}{\pgfqpoint{3.696000in}{3.696000in}}%
\pgfusepath{clip}%
\pgfsetbuttcap%
\pgfsetroundjoin%
\definecolor{currentfill}{rgb}{0.121569,0.466667,0.705882}%
\pgfsetfillcolor{currentfill}%
\pgfsetfillopacity{0.960016}%
\pgfsetlinewidth{1.003750pt}%
\definecolor{currentstroke}{rgb}{0.121569,0.466667,0.705882}%
\pgfsetstrokecolor{currentstroke}%
\pgfsetstrokeopacity{0.960016}%
\pgfsetdash{}{0pt}%
\pgfpathmoveto{\pgfqpoint{2.512584in}{1.025681in}}%
\pgfpathcurveto{\pgfqpoint{2.520820in}{1.025681in}}{\pgfqpoint{2.528720in}{1.028954in}}{\pgfqpoint{2.534544in}{1.034778in}}%
\pgfpathcurveto{\pgfqpoint{2.540368in}{1.040602in}}{\pgfqpoint{2.543641in}{1.048502in}}{\pgfqpoint{2.543641in}{1.056738in}}%
\pgfpathcurveto{\pgfqpoint{2.543641in}{1.064974in}}{\pgfqpoint{2.540368in}{1.072874in}}{\pgfqpoint{2.534544in}{1.078698in}}%
\pgfpathcurveto{\pgfqpoint{2.528720in}{1.084522in}}{\pgfqpoint{2.520820in}{1.087794in}}{\pgfqpoint{2.512584in}{1.087794in}}%
\pgfpathcurveto{\pgfqpoint{2.504348in}{1.087794in}}{\pgfqpoint{2.496448in}{1.084522in}}{\pgfqpoint{2.490624in}{1.078698in}}%
\pgfpathcurveto{\pgfqpoint{2.484800in}{1.072874in}}{\pgfqpoint{2.481528in}{1.064974in}}{\pgfqpoint{2.481528in}{1.056738in}}%
\pgfpathcurveto{\pgfqpoint{2.481528in}{1.048502in}}{\pgfqpoint{2.484800in}{1.040602in}}{\pgfqpoint{2.490624in}{1.034778in}}%
\pgfpathcurveto{\pgfqpoint{2.496448in}{1.028954in}}{\pgfqpoint{2.504348in}{1.025681in}}{\pgfqpoint{2.512584in}{1.025681in}}%
\pgfpathclose%
\pgfusepath{stroke,fill}%
\end{pgfscope}%
\begin{pgfscope}%
\pgfpathrectangle{\pgfqpoint{0.100000in}{0.220728in}}{\pgfqpoint{3.696000in}{3.696000in}}%
\pgfusepath{clip}%
\pgfsetbuttcap%
\pgfsetroundjoin%
\definecolor{currentfill}{rgb}{0.121569,0.466667,0.705882}%
\pgfsetfillcolor{currentfill}%
\pgfsetfillopacity{0.961248}%
\pgfsetlinewidth{1.003750pt}%
\definecolor{currentstroke}{rgb}{0.121569,0.466667,0.705882}%
\pgfsetstrokecolor{currentstroke}%
\pgfsetstrokeopacity{0.961248}%
\pgfsetdash{}{0pt}%
\pgfpathmoveto{\pgfqpoint{2.139386in}{0.789373in}}%
\pgfpathcurveto{\pgfqpoint{2.147622in}{0.789373in}}{\pgfqpoint{2.155522in}{0.792646in}}{\pgfqpoint{2.161346in}{0.798470in}}%
\pgfpathcurveto{\pgfqpoint{2.167170in}{0.804294in}}{\pgfqpoint{2.170443in}{0.812194in}}{\pgfqpoint{2.170443in}{0.820430in}}%
\pgfpathcurveto{\pgfqpoint{2.170443in}{0.828666in}}{\pgfqpoint{2.167170in}{0.836566in}}{\pgfqpoint{2.161346in}{0.842390in}}%
\pgfpathcurveto{\pgfqpoint{2.155522in}{0.848214in}}{\pgfqpoint{2.147622in}{0.851486in}}{\pgfqpoint{2.139386in}{0.851486in}}%
\pgfpathcurveto{\pgfqpoint{2.131150in}{0.851486in}}{\pgfqpoint{2.123250in}{0.848214in}}{\pgfqpoint{2.117426in}{0.842390in}}%
\pgfpathcurveto{\pgfqpoint{2.111602in}{0.836566in}}{\pgfqpoint{2.108330in}{0.828666in}}{\pgfqpoint{2.108330in}{0.820430in}}%
\pgfpathcurveto{\pgfqpoint{2.108330in}{0.812194in}}{\pgfqpoint{2.111602in}{0.804294in}}{\pgfqpoint{2.117426in}{0.798470in}}%
\pgfpathcurveto{\pgfqpoint{2.123250in}{0.792646in}}{\pgfqpoint{2.131150in}{0.789373in}}{\pgfqpoint{2.139386in}{0.789373in}}%
\pgfpathclose%
\pgfusepath{stroke,fill}%
\end{pgfscope}%
\begin{pgfscope}%
\pgfpathrectangle{\pgfqpoint{0.100000in}{0.220728in}}{\pgfqpoint{3.696000in}{3.696000in}}%
\pgfusepath{clip}%
\pgfsetbuttcap%
\pgfsetroundjoin%
\definecolor{currentfill}{rgb}{0.121569,0.466667,0.705882}%
\pgfsetfillcolor{currentfill}%
\pgfsetfillopacity{0.962575}%
\pgfsetlinewidth{1.003750pt}%
\definecolor{currentstroke}{rgb}{0.121569,0.466667,0.705882}%
\pgfsetstrokecolor{currentstroke}%
\pgfsetstrokeopacity{0.962575}%
\pgfsetdash{}{0pt}%
\pgfpathmoveto{\pgfqpoint{2.496863in}{0.998801in}}%
\pgfpathcurveto{\pgfqpoint{2.505099in}{0.998801in}}{\pgfqpoint{2.512999in}{1.002073in}}{\pgfqpoint{2.518823in}{1.007897in}}%
\pgfpathcurveto{\pgfqpoint{2.524647in}{1.013721in}}{\pgfqpoint{2.527920in}{1.021621in}}{\pgfqpoint{2.527920in}{1.029857in}}%
\pgfpathcurveto{\pgfqpoint{2.527920in}{1.038094in}}{\pgfqpoint{2.524647in}{1.045994in}}{\pgfqpoint{2.518823in}{1.051818in}}%
\pgfpathcurveto{\pgfqpoint{2.512999in}{1.057642in}}{\pgfqpoint{2.505099in}{1.060914in}}{\pgfqpoint{2.496863in}{1.060914in}}%
\pgfpathcurveto{\pgfqpoint{2.488627in}{1.060914in}}{\pgfqpoint{2.480727in}{1.057642in}}{\pgfqpoint{2.474903in}{1.051818in}}%
\pgfpathcurveto{\pgfqpoint{2.469079in}{1.045994in}}{\pgfqpoint{2.465807in}{1.038094in}}{\pgfqpoint{2.465807in}{1.029857in}}%
\pgfpathcurveto{\pgfqpoint{2.465807in}{1.021621in}}{\pgfqpoint{2.469079in}{1.013721in}}{\pgfqpoint{2.474903in}{1.007897in}}%
\pgfpathcurveto{\pgfqpoint{2.480727in}{1.002073in}}{\pgfqpoint{2.488627in}{0.998801in}}{\pgfqpoint{2.496863in}{0.998801in}}%
\pgfpathclose%
\pgfusepath{stroke,fill}%
\end{pgfscope}%
\begin{pgfscope}%
\pgfpathrectangle{\pgfqpoint{0.100000in}{0.220728in}}{\pgfqpoint{3.696000in}{3.696000in}}%
\pgfusepath{clip}%
\pgfsetbuttcap%
\pgfsetroundjoin%
\definecolor{currentfill}{rgb}{0.121569,0.466667,0.705882}%
\pgfsetfillcolor{currentfill}%
\pgfsetfillopacity{0.962954}%
\pgfsetlinewidth{1.003750pt}%
\definecolor{currentstroke}{rgb}{0.121569,0.466667,0.705882}%
\pgfsetstrokecolor{currentstroke}%
\pgfsetstrokeopacity{0.962954}%
\pgfsetdash{}{0pt}%
\pgfpathmoveto{\pgfqpoint{2.146047in}{0.787469in}}%
\pgfpathcurveto{\pgfqpoint{2.154284in}{0.787469in}}{\pgfqpoint{2.162184in}{0.790742in}}{\pgfqpoint{2.168008in}{0.796566in}}%
\pgfpathcurveto{\pgfqpoint{2.173832in}{0.802389in}}{\pgfqpoint{2.177104in}{0.810290in}}{\pgfqpoint{2.177104in}{0.818526in}}%
\pgfpathcurveto{\pgfqpoint{2.177104in}{0.826762in}}{\pgfqpoint{2.173832in}{0.834662in}}{\pgfqpoint{2.168008in}{0.840486in}}%
\pgfpathcurveto{\pgfqpoint{2.162184in}{0.846310in}}{\pgfqpoint{2.154284in}{0.849582in}}{\pgfqpoint{2.146047in}{0.849582in}}%
\pgfpathcurveto{\pgfqpoint{2.137811in}{0.849582in}}{\pgfqpoint{2.129911in}{0.846310in}}{\pgfqpoint{2.124087in}{0.840486in}}%
\pgfpathcurveto{\pgfqpoint{2.118263in}{0.834662in}}{\pgfqpoint{2.114991in}{0.826762in}}{\pgfqpoint{2.114991in}{0.818526in}}%
\pgfpathcurveto{\pgfqpoint{2.114991in}{0.810290in}}{\pgfqpoint{2.118263in}{0.802389in}}{\pgfqpoint{2.124087in}{0.796566in}}%
\pgfpathcurveto{\pgfqpoint{2.129911in}{0.790742in}}{\pgfqpoint{2.137811in}{0.787469in}}{\pgfqpoint{2.146047in}{0.787469in}}%
\pgfpathclose%
\pgfusepath{stroke,fill}%
\end{pgfscope}%
\begin{pgfscope}%
\pgfpathrectangle{\pgfqpoint{0.100000in}{0.220728in}}{\pgfqpoint{3.696000in}{3.696000in}}%
\pgfusepath{clip}%
\pgfsetbuttcap%
\pgfsetroundjoin%
\definecolor{currentfill}{rgb}{0.121569,0.466667,0.705882}%
\pgfsetfillcolor{currentfill}%
\pgfsetfillopacity{0.965760}%
\pgfsetlinewidth{1.003750pt}%
\definecolor{currentstroke}{rgb}{0.121569,0.466667,0.705882}%
\pgfsetstrokecolor{currentstroke}%
\pgfsetstrokeopacity{0.965760}%
\pgfsetdash{}{0pt}%
\pgfpathmoveto{\pgfqpoint{2.158130in}{0.782779in}}%
\pgfpathcurveto{\pgfqpoint{2.166366in}{0.782779in}}{\pgfqpoint{2.174266in}{0.786051in}}{\pgfqpoint{2.180090in}{0.791875in}}%
\pgfpathcurveto{\pgfqpoint{2.185914in}{0.797699in}}{\pgfqpoint{2.189187in}{0.805599in}}{\pgfqpoint{2.189187in}{0.813835in}}%
\pgfpathcurveto{\pgfqpoint{2.189187in}{0.822071in}}{\pgfqpoint{2.185914in}{0.829972in}}{\pgfqpoint{2.180090in}{0.835795in}}%
\pgfpathcurveto{\pgfqpoint{2.174266in}{0.841619in}}{\pgfqpoint{2.166366in}{0.844892in}}{\pgfqpoint{2.158130in}{0.844892in}}%
\pgfpathcurveto{\pgfqpoint{2.149894in}{0.844892in}}{\pgfqpoint{2.141994in}{0.841619in}}{\pgfqpoint{2.136170in}{0.835795in}}%
\pgfpathcurveto{\pgfqpoint{2.130346in}{0.829972in}}{\pgfqpoint{2.127074in}{0.822071in}}{\pgfqpoint{2.127074in}{0.813835in}}%
\pgfpathcurveto{\pgfqpoint{2.127074in}{0.805599in}}{\pgfqpoint{2.130346in}{0.797699in}}{\pgfqpoint{2.136170in}{0.791875in}}%
\pgfpathcurveto{\pgfqpoint{2.141994in}{0.786051in}}{\pgfqpoint{2.149894in}{0.782779in}}{\pgfqpoint{2.158130in}{0.782779in}}%
\pgfpathclose%
\pgfusepath{stroke,fill}%
\end{pgfscope}%
\begin{pgfscope}%
\pgfpathrectangle{\pgfqpoint{0.100000in}{0.220728in}}{\pgfqpoint{3.696000in}{3.696000in}}%
\pgfusepath{clip}%
\pgfsetbuttcap%
\pgfsetroundjoin%
\definecolor{currentfill}{rgb}{0.121569,0.466667,0.705882}%
\pgfsetfillcolor{currentfill}%
\pgfsetfillopacity{0.967193}%
\pgfsetlinewidth{1.003750pt}%
\definecolor{currentstroke}{rgb}{0.121569,0.466667,0.705882}%
\pgfsetstrokecolor{currentstroke}%
\pgfsetstrokeopacity{0.967193}%
\pgfsetdash{}{0pt}%
\pgfpathmoveto{\pgfqpoint{2.489239in}{0.964313in}}%
\pgfpathcurveto{\pgfqpoint{2.497476in}{0.964313in}}{\pgfqpoint{2.505376in}{0.967585in}}{\pgfqpoint{2.511200in}{0.973409in}}%
\pgfpathcurveto{\pgfqpoint{2.517024in}{0.979233in}}{\pgfqpoint{2.520296in}{0.987133in}}{\pgfqpoint{2.520296in}{0.995370in}}%
\pgfpathcurveto{\pgfqpoint{2.520296in}{1.003606in}}{\pgfqpoint{2.517024in}{1.011506in}}{\pgfqpoint{2.511200in}{1.017330in}}%
\pgfpathcurveto{\pgfqpoint{2.505376in}{1.023154in}}{\pgfqpoint{2.497476in}{1.026426in}}{\pgfqpoint{2.489239in}{1.026426in}}%
\pgfpathcurveto{\pgfqpoint{2.481003in}{1.026426in}}{\pgfqpoint{2.473103in}{1.023154in}}{\pgfqpoint{2.467279in}{1.017330in}}%
\pgfpathcurveto{\pgfqpoint{2.461455in}{1.011506in}}{\pgfqpoint{2.458183in}{1.003606in}}{\pgfqpoint{2.458183in}{0.995370in}}%
\pgfpathcurveto{\pgfqpoint{2.458183in}{0.987133in}}{\pgfqpoint{2.461455in}{0.979233in}}{\pgfqpoint{2.467279in}{0.973409in}}%
\pgfpathcurveto{\pgfqpoint{2.473103in}{0.967585in}}{\pgfqpoint{2.481003in}{0.964313in}}{\pgfqpoint{2.489239in}{0.964313in}}%
\pgfpathclose%
\pgfusepath{stroke,fill}%
\end{pgfscope}%
\begin{pgfscope}%
\pgfpathrectangle{\pgfqpoint{0.100000in}{0.220728in}}{\pgfqpoint{3.696000in}{3.696000in}}%
\pgfusepath{clip}%
\pgfsetbuttcap%
\pgfsetroundjoin%
\definecolor{currentfill}{rgb}{0.121569,0.466667,0.705882}%
\pgfsetfillcolor{currentfill}%
\pgfsetfillopacity{0.968039}%
\pgfsetlinewidth{1.003750pt}%
\definecolor{currentstroke}{rgb}{0.121569,0.466667,0.705882}%
\pgfsetstrokecolor{currentstroke}%
\pgfsetstrokeopacity{0.968039}%
\pgfsetdash{}{0pt}%
\pgfpathmoveto{\pgfqpoint{2.168659in}{0.778550in}}%
\pgfpathcurveto{\pgfqpoint{2.176895in}{0.778550in}}{\pgfqpoint{2.184795in}{0.781823in}}{\pgfqpoint{2.190619in}{0.787646in}}%
\pgfpathcurveto{\pgfqpoint{2.196443in}{0.793470in}}{\pgfqpoint{2.199716in}{0.801370in}}{\pgfqpoint{2.199716in}{0.809607in}}%
\pgfpathcurveto{\pgfqpoint{2.199716in}{0.817843in}}{\pgfqpoint{2.196443in}{0.825743in}}{\pgfqpoint{2.190619in}{0.831567in}}%
\pgfpathcurveto{\pgfqpoint{2.184795in}{0.837391in}}{\pgfqpoint{2.176895in}{0.840663in}}{\pgfqpoint{2.168659in}{0.840663in}}%
\pgfpathcurveto{\pgfqpoint{2.160423in}{0.840663in}}{\pgfqpoint{2.152523in}{0.837391in}}{\pgfqpoint{2.146699in}{0.831567in}}%
\pgfpathcurveto{\pgfqpoint{2.140875in}{0.825743in}}{\pgfqpoint{2.137603in}{0.817843in}}{\pgfqpoint{2.137603in}{0.809607in}}%
\pgfpathcurveto{\pgfqpoint{2.137603in}{0.801370in}}{\pgfqpoint{2.140875in}{0.793470in}}{\pgfqpoint{2.146699in}{0.787646in}}%
\pgfpathcurveto{\pgfqpoint{2.152523in}{0.781823in}}{\pgfqpoint{2.160423in}{0.778550in}}{\pgfqpoint{2.168659in}{0.778550in}}%
\pgfpathclose%
\pgfusepath{stroke,fill}%
\end{pgfscope}%
\begin{pgfscope}%
\pgfpathrectangle{\pgfqpoint{0.100000in}{0.220728in}}{\pgfqpoint{3.696000in}{3.696000in}}%
\pgfusepath{clip}%
\pgfsetbuttcap%
\pgfsetroundjoin%
\definecolor{currentfill}{rgb}{0.121569,0.466667,0.705882}%
\pgfsetfillcolor{currentfill}%
\pgfsetfillopacity{0.969131}%
\pgfsetlinewidth{1.003750pt}%
\definecolor{currentstroke}{rgb}{0.121569,0.466667,0.705882}%
\pgfsetstrokecolor{currentstroke}%
\pgfsetstrokeopacity{0.969131}%
\pgfsetdash{}{0pt}%
\pgfpathmoveto{\pgfqpoint{2.481690in}{0.945754in}}%
\pgfpathcurveto{\pgfqpoint{2.489926in}{0.945754in}}{\pgfqpoint{2.497826in}{0.949027in}}{\pgfqpoint{2.503650in}{0.954850in}}%
\pgfpathcurveto{\pgfqpoint{2.509474in}{0.960674in}}{\pgfqpoint{2.512746in}{0.968574in}}{\pgfqpoint{2.512746in}{0.976811in}}%
\pgfpathcurveto{\pgfqpoint{2.512746in}{0.985047in}}{\pgfqpoint{2.509474in}{0.992947in}}{\pgfqpoint{2.503650in}{0.998771in}}%
\pgfpathcurveto{\pgfqpoint{2.497826in}{1.004595in}}{\pgfqpoint{2.489926in}{1.007867in}}{\pgfqpoint{2.481690in}{1.007867in}}%
\pgfpathcurveto{\pgfqpoint{2.473454in}{1.007867in}}{\pgfqpoint{2.465554in}{1.004595in}}{\pgfqpoint{2.459730in}{0.998771in}}%
\pgfpathcurveto{\pgfqpoint{2.453906in}{0.992947in}}{\pgfqpoint{2.450633in}{0.985047in}}{\pgfqpoint{2.450633in}{0.976811in}}%
\pgfpathcurveto{\pgfqpoint{2.450633in}{0.968574in}}{\pgfqpoint{2.453906in}{0.960674in}}{\pgfqpoint{2.459730in}{0.954850in}}%
\pgfpathcurveto{\pgfqpoint{2.465554in}{0.949027in}}{\pgfqpoint{2.473454in}{0.945754in}}{\pgfqpoint{2.481690in}{0.945754in}}%
\pgfpathclose%
\pgfusepath{stroke,fill}%
\end{pgfscope}%
\begin{pgfscope}%
\pgfpathrectangle{\pgfqpoint{0.100000in}{0.220728in}}{\pgfqpoint{3.696000in}{3.696000in}}%
\pgfusepath{clip}%
\pgfsetbuttcap%
\pgfsetroundjoin%
\definecolor{currentfill}{rgb}{0.121569,0.466667,0.705882}%
\pgfsetfillcolor{currentfill}%
\pgfsetfillopacity{0.969822}%
\pgfsetlinewidth{1.003750pt}%
\definecolor{currentstroke}{rgb}{0.121569,0.466667,0.705882}%
\pgfsetstrokecolor{currentstroke}%
\pgfsetstrokeopacity{0.969822}%
\pgfsetdash{}{0pt}%
\pgfpathmoveto{\pgfqpoint{2.176273in}{0.776950in}}%
\pgfpathcurveto{\pgfqpoint{2.184509in}{0.776950in}}{\pgfqpoint{2.192410in}{0.780222in}}{\pgfqpoint{2.198233in}{0.786046in}}%
\pgfpathcurveto{\pgfqpoint{2.204057in}{0.791870in}}{\pgfqpoint{2.207330in}{0.799770in}}{\pgfqpoint{2.207330in}{0.808006in}}%
\pgfpathcurveto{\pgfqpoint{2.207330in}{0.816243in}}{\pgfqpoint{2.204057in}{0.824143in}}{\pgfqpoint{2.198233in}{0.829967in}}%
\pgfpathcurveto{\pgfqpoint{2.192410in}{0.835791in}}{\pgfqpoint{2.184509in}{0.839063in}}{\pgfqpoint{2.176273in}{0.839063in}}%
\pgfpathcurveto{\pgfqpoint{2.168037in}{0.839063in}}{\pgfqpoint{2.160137in}{0.835791in}}{\pgfqpoint{2.154313in}{0.829967in}}%
\pgfpathcurveto{\pgfqpoint{2.148489in}{0.824143in}}{\pgfqpoint{2.145217in}{0.816243in}}{\pgfqpoint{2.145217in}{0.808006in}}%
\pgfpathcurveto{\pgfqpoint{2.145217in}{0.799770in}}{\pgfqpoint{2.148489in}{0.791870in}}{\pgfqpoint{2.154313in}{0.786046in}}%
\pgfpathcurveto{\pgfqpoint{2.160137in}{0.780222in}}{\pgfqpoint{2.168037in}{0.776950in}}{\pgfqpoint{2.176273in}{0.776950in}}%
\pgfpathclose%
\pgfusepath{stroke,fill}%
\end{pgfscope}%
\begin{pgfscope}%
\pgfpathrectangle{\pgfqpoint{0.100000in}{0.220728in}}{\pgfqpoint{3.696000in}{3.696000in}}%
\pgfusepath{clip}%
\pgfsetbuttcap%
\pgfsetroundjoin%
\definecolor{currentfill}{rgb}{0.121569,0.466667,0.705882}%
\pgfsetfillcolor{currentfill}%
\pgfsetfillopacity{0.970319}%
\pgfsetlinewidth{1.003750pt}%
\definecolor{currentstroke}{rgb}{0.121569,0.466667,0.705882}%
\pgfsetstrokecolor{currentstroke}%
\pgfsetstrokeopacity{0.970319}%
\pgfsetdash{}{0pt}%
\pgfpathmoveto{\pgfqpoint{2.476641in}{0.937288in}}%
\pgfpathcurveto{\pgfqpoint{2.484877in}{0.937288in}}{\pgfqpoint{2.492777in}{0.940560in}}{\pgfqpoint{2.498601in}{0.946384in}}%
\pgfpathcurveto{\pgfqpoint{2.504425in}{0.952208in}}{\pgfqpoint{2.507697in}{0.960108in}}{\pgfqpoint{2.507697in}{0.968344in}}%
\pgfpathcurveto{\pgfqpoint{2.507697in}{0.976581in}}{\pgfqpoint{2.504425in}{0.984481in}}{\pgfqpoint{2.498601in}{0.990305in}}%
\pgfpathcurveto{\pgfqpoint{2.492777in}{0.996129in}}{\pgfqpoint{2.484877in}{0.999401in}}{\pgfqpoint{2.476641in}{0.999401in}}%
\pgfpathcurveto{\pgfqpoint{2.468404in}{0.999401in}}{\pgfqpoint{2.460504in}{0.996129in}}{\pgfqpoint{2.454680in}{0.990305in}}%
\pgfpathcurveto{\pgfqpoint{2.448856in}{0.984481in}}{\pgfqpoint{2.445584in}{0.976581in}}{\pgfqpoint{2.445584in}{0.968344in}}%
\pgfpathcurveto{\pgfqpoint{2.445584in}{0.960108in}}{\pgfqpoint{2.448856in}{0.952208in}}{\pgfqpoint{2.454680in}{0.946384in}}%
\pgfpathcurveto{\pgfqpoint{2.460504in}{0.940560in}}{\pgfqpoint{2.468404in}{0.937288in}}{\pgfqpoint{2.476641in}{0.937288in}}%
\pgfpathclose%
\pgfusepath{stroke,fill}%
\end{pgfscope}%
\begin{pgfscope}%
\pgfpathrectangle{\pgfqpoint{0.100000in}{0.220728in}}{\pgfqpoint{3.696000in}{3.696000in}}%
\pgfusepath{clip}%
\pgfsetbuttcap%
\pgfsetroundjoin%
\definecolor{currentfill}{rgb}{0.121569,0.466667,0.705882}%
\pgfsetfillcolor{currentfill}%
\pgfsetfillopacity{0.970577}%
\pgfsetlinewidth{1.003750pt}%
\definecolor{currentstroke}{rgb}{0.121569,0.466667,0.705882}%
\pgfsetstrokecolor{currentstroke}%
\pgfsetstrokeopacity{0.970577}%
\pgfsetdash{}{0pt}%
\pgfpathmoveto{\pgfqpoint{2.182529in}{0.773293in}}%
\pgfpathcurveto{\pgfqpoint{2.190765in}{0.773293in}}{\pgfqpoint{2.198665in}{0.776566in}}{\pgfqpoint{2.204489in}{0.782390in}}%
\pgfpathcurveto{\pgfqpoint{2.210313in}{0.788214in}}{\pgfqpoint{2.213586in}{0.796114in}}{\pgfqpoint{2.213586in}{0.804350in}}%
\pgfpathcurveto{\pgfqpoint{2.213586in}{0.812586in}}{\pgfqpoint{2.210313in}{0.820486in}}{\pgfqpoint{2.204489in}{0.826310in}}%
\pgfpathcurveto{\pgfqpoint{2.198665in}{0.832134in}}{\pgfqpoint{2.190765in}{0.835406in}}{\pgfqpoint{2.182529in}{0.835406in}}%
\pgfpathcurveto{\pgfqpoint{2.174293in}{0.835406in}}{\pgfqpoint{2.166393in}{0.832134in}}{\pgfqpoint{2.160569in}{0.826310in}}%
\pgfpathcurveto{\pgfqpoint{2.154745in}{0.820486in}}{\pgfqpoint{2.151473in}{0.812586in}}{\pgfqpoint{2.151473in}{0.804350in}}%
\pgfpathcurveto{\pgfqpoint{2.151473in}{0.796114in}}{\pgfqpoint{2.154745in}{0.788214in}}{\pgfqpoint{2.160569in}{0.782390in}}%
\pgfpathcurveto{\pgfqpoint{2.166393in}{0.776566in}}{\pgfqpoint{2.174293in}{0.773293in}}{\pgfqpoint{2.182529in}{0.773293in}}%
\pgfpathclose%
\pgfusepath{stroke,fill}%
\end{pgfscope}%
\begin{pgfscope}%
\pgfpathrectangle{\pgfqpoint{0.100000in}{0.220728in}}{\pgfqpoint{3.696000in}{3.696000in}}%
\pgfusepath{clip}%
\pgfsetbuttcap%
\pgfsetroundjoin%
\definecolor{currentfill}{rgb}{0.121569,0.466667,0.705882}%
\pgfsetfillcolor{currentfill}%
\pgfsetfillopacity{0.971778}%
\pgfsetlinewidth{1.003750pt}%
\definecolor{currentstroke}{rgb}{0.121569,0.466667,0.705882}%
\pgfsetstrokecolor{currentstroke}%
\pgfsetstrokeopacity{0.971778}%
\pgfsetdash{}{0pt}%
\pgfpathmoveto{\pgfqpoint{2.474032in}{0.926018in}}%
\pgfpathcurveto{\pgfqpoint{2.482269in}{0.926018in}}{\pgfqpoint{2.490169in}{0.929290in}}{\pgfqpoint{2.495993in}{0.935114in}}%
\pgfpathcurveto{\pgfqpoint{2.501816in}{0.940938in}}{\pgfqpoint{2.505089in}{0.948838in}}{\pgfqpoint{2.505089in}{0.957074in}}%
\pgfpathcurveto{\pgfqpoint{2.505089in}{0.965310in}}{\pgfqpoint{2.501816in}{0.973211in}}{\pgfqpoint{2.495993in}{0.979034in}}%
\pgfpathcurveto{\pgfqpoint{2.490169in}{0.984858in}}{\pgfqpoint{2.482269in}{0.988131in}}{\pgfqpoint{2.474032in}{0.988131in}}%
\pgfpathcurveto{\pgfqpoint{2.465796in}{0.988131in}}{\pgfqpoint{2.457896in}{0.984858in}}{\pgfqpoint{2.452072in}{0.979034in}}%
\pgfpathcurveto{\pgfqpoint{2.446248in}{0.973211in}}{\pgfqpoint{2.442976in}{0.965310in}}{\pgfqpoint{2.442976in}{0.957074in}}%
\pgfpathcurveto{\pgfqpoint{2.442976in}{0.948838in}}{\pgfqpoint{2.446248in}{0.940938in}}{\pgfqpoint{2.452072in}{0.935114in}}%
\pgfpathcurveto{\pgfqpoint{2.457896in}{0.929290in}}{\pgfqpoint{2.465796in}{0.926018in}}{\pgfqpoint{2.474032in}{0.926018in}}%
\pgfpathclose%
\pgfusepath{stroke,fill}%
\end{pgfscope}%
\begin{pgfscope}%
\pgfpathrectangle{\pgfqpoint{0.100000in}{0.220728in}}{\pgfqpoint{3.696000in}{3.696000in}}%
\pgfusepath{clip}%
\pgfsetbuttcap%
\pgfsetroundjoin%
\definecolor{currentfill}{rgb}{0.121569,0.466667,0.705882}%
\pgfsetfillcolor{currentfill}%
\pgfsetfillopacity{0.972198}%
\pgfsetlinewidth{1.003750pt}%
\definecolor{currentstroke}{rgb}{0.121569,0.466667,0.705882}%
\pgfsetstrokecolor{currentstroke}%
\pgfsetstrokeopacity{0.972198}%
\pgfsetdash{}{0pt}%
\pgfpathmoveto{\pgfqpoint{2.194246in}{0.768918in}}%
\pgfpathcurveto{\pgfqpoint{2.202482in}{0.768918in}}{\pgfqpoint{2.210382in}{0.772191in}}{\pgfqpoint{2.216206in}{0.778015in}}%
\pgfpathcurveto{\pgfqpoint{2.222030in}{0.783839in}}{\pgfqpoint{2.225302in}{0.791739in}}{\pgfqpoint{2.225302in}{0.799975in}}%
\pgfpathcurveto{\pgfqpoint{2.225302in}{0.808211in}}{\pgfqpoint{2.222030in}{0.816111in}}{\pgfqpoint{2.216206in}{0.821935in}}%
\pgfpathcurveto{\pgfqpoint{2.210382in}{0.827759in}}{\pgfqpoint{2.202482in}{0.831031in}}{\pgfqpoint{2.194246in}{0.831031in}}%
\pgfpathcurveto{\pgfqpoint{2.186009in}{0.831031in}}{\pgfqpoint{2.178109in}{0.827759in}}{\pgfqpoint{2.172285in}{0.821935in}}%
\pgfpathcurveto{\pgfqpoint{2.166461in}{0.816111in}}{\pgfqpoint{2.163189in}{0.808211in}}{\pgfqpoint{2.163189in}{0.799975in}}%
\pgfpathcurveto{\pgfqpoint{2.163189in}{0.791739in}}{\pgfqpoint{2.166461in}{0.783839in}}{\pgfqpoint{2.172285in}{0.778015in}}%
\pgfpathcurveto{\pgfqpoint{2.178109in}{0.772191in}}{\pgfqpoint{2.186009in}{0.768918in}}{\pgfqpoint{2.194246in}{0.768918in}}%
\pgfpathclose%
\pgfusepath{stroke,fill}%
\end{pgfscope}%
\begin{pgfscope}%
\pgfpathrectangle{\pgfqpoint{0.100000in}{0.220728in}}{\pgfqpoint{3.696000in}{3.696000in}}%
\pgfusepath{clip}%
\pgfsetbuttcap%
\pgfsetroundjoin%
\definecolor{currentfill}{rgb}{0.121569,0.466667,0.705882}%
\pgfsetfillcolor{currentfill}%
\pgfsetfillopacity{0.973394}%
\pgfsetlinewidth{1.003750pt}%
\definecolor{currentstroke}{rgb}{0.121569,0.466667,0.705882}%
\pgfsetstrokecolor{currentstroke}%
\pgfsetstrokeopacity{0.973394}%
\pgfsetdash{}{0pt}%
\pgfpathmoveto{\pgfqpoint{2.463508in}{0.911333in}}%
\pgfpathcurveto{\pgfqpoint{2.471744in}{0.911333in}}{\pgfqpoint{2.479644in}{0.914605in}}{\pgfqpoint{2.485468in}{0.920429in}}%
\pgfpathcurveto{\pgfqpoint{2.491292in}{0.926253in}}{\pgfqpoint{2.494564in}{0.934153in}}{\pgfqpoint{2.494564in}{0.942390in}}%
\pgfpathcurveto{\pgfqpoint{2.494564in}{0.950626in}}{\pgfqpoint{2.491292in}{0.958526in}}{\pgfqpoint{2.485468in}{0.964350in}}%
\pgfpathcurveto{\pgfqpoint{2.479644in}{0.970174in}}{\pgfqpoint{2.471744in}{0.973446in}}{\pgfqpoint{2.463508in}{0.973446in}}%
\pgfpathcurveto{\pgfqpoint{2.455271in}{0.973446in}}{\pgfqpoint{2.447371in}{0.970174in}}{\pgfqpoint{2.441548in}{0.964350in}}%
\pgfpathcurveto{\pgfqpoint{2.435724in}{0.958526in}}{\pgfqpoint{2.432451in}{0.950626in}}{\pgfqpoint{2.432451in}{0.942390in}}%
\pgfpathcurveto{\pgfqpoint{2.432451in}{0.934153in}}{\pgfqpoint{2.435724in}{0.926253in}}{\pgfqpoint{2.441548in}{0.920429in}}%
\pgfpathcurveto{\pgfqpoint{2.447371in}{0.914605in}}{\pgfqpoint{2.455271in}{0.911333in}}{\pgfqpoint{2.463508in}{0.911333in}}%
\pgfpathclose%
\pgfusepath{stroke,fill}%
\end{pgfscope}%
\begin{pgfscope}%
\pgfpathrectangle{\pgfqpoint{0.100000in}{0.220728in}}{\pgfqpoint{3.696000in}{3.696000in}}%
\pgfusepath{clip}%
\pgfsetbuttcap%
\pgfsetroundjoin%
\definecolor{currentfill}{rgb}{0.121569,0.466667,0.705882}%
\pgfsetfillcolor{currentfill}%
\pgfsetfillopacity{0.974034}%
\pgfsetlinewidth{1.003750pt}%
\definecolor{currentstroke}{rgb}{0.121569,0.466667,0.705882}%
\pgfsetstrokecolor{currentstroke}%
\pgfsetstrokeopacity{0.974034}%
\pgfsetdash{}{0pt}%
\pgfpathmoveto{\pgfqpoint{2.216657in}{0.762246in}}%
\pgfpathcurveto{\pgfqpoint{2.224893in}{0.762246in}}{\pgfqpoint{2.232793in}{0.765519in}}{\pgfqpoint{2.238617in}{0.771343in}}%
\pgfpathcurveto{\pgfqpoint{2.244441in}{0.777167in}}{\pgfqpoint{2.247713in}{0.785067in}}{\pgfqpoint{2.247713in}{0.793303in}}%
\pgfpathcurveto{\pgfqpoint{2.247713in}{0.801539in}}{\pgfqpoint{2.244441in}{0.809439in}}{\pgfqpoint{2.238617in}{0.815263in}}%
\pgfpathcurveto{\pgfqpoint{2.232793in}{0.821087in}}{\pgfqpoint{2.224893in}{0.824359in}}{\pgfqpoint{2.216657in}{0.824359in}}%
\pgfpathcurveto{\pgfqpoint{2.208421in}{0.824359in}}{\pgfqpoint{2.200521in}{0.821087in}}{\pgfqpoint{2.194697in}{0.815263in}}%
\pgfpathcurveto{\pgfqpoint{2.188873in}{0.809439in}}{\pgfqpoint{2.185600in}{0.801539in}}{\pgfqpoint{2.185600in}{0.793303in}}%
\pgfpathcurveto{\pgfqpoint{2.185600in}{0.785067in}}{\pgfqpoint{2.188873in}{0.777167in}}{\pgfqpoint{2.194697in}{0.771343in}}%
\pgfpathcurveto{\pgfqpoint{2.200521in}{0.765519in}}{\pgfqpoint{2.208421in}{0.762246in}}{\pgfqpoint{2.216657in}{0.762246in}}%
\pgfpathclose%
\pgfusepath{stroke,fill}%
\end{pgfscope}%
\begin{pgfscope}%
\pgfpathrectangle{\pgfqpoint{0.100000in}{0.220728in}}{\pgfqpoint{3.696000in}{3.696000in}}%
\pgfusepath{clip}%
\pgfsetbuttcap%
\pgfsetroundjoin%
\definecolor{currentfill}{rgb}{0.121569,0.466667,0.705882}%
\pgfsetfillcolor{currentfill}%
\pgfsetfillopacity{0.975953}%
\pgfsetlinewidth{1.003750pt}%
\definecolor{currentstroke}{rgb}{0.121569,0.466667,0.705882}%
\pgfsetstrokecolor{currentstroke}%
\pgfsetstrokeopacity{0.975953}%
\pgfsetdash{}{0pt}%
\pgfpathmoveto{\pgfqpoint{2.458328in}{0.889415in}}%
\pgfpathcurveto{\pgfqpoint{2.466565in}{0.889415in}}{\pgfqpoint{2.474465in}{0.892688in}}{\pgfqpoint{2.480289in}{0.898512in}}%
\pgfpathcurveto{\pgfqpoint{2.486112in}{0.904335in}}{\pgfqpoint{2.489385in}{0.912236in}}{\pgfqpoint{2.489385in}{0.920472in}}%
\pgfpathcurveto{\pgfqpoint{2.489385in}{0.928708in}}{\pgfqpoint{2.486112in}{0.936608in}}{\pgfqpoint{2.480289in}{0.942432in}}%
\pgfpathcurveto{\pgfqpoint{2.474465in}{0.948256in}}{\pgfqpoint{2.466565in}{0.951528in}}{\pgfqpoint{2.458328in}{0.951528in}}%
\pgfpathcurveto{\pgfqpoint{2.450092in}{0.951528in}}{\pgfqpoint{2.442192in}{0.948256in}}{\pgfqpoint{2.436368in}{0.942432in}}%
\pgfpathcurveto{\pgfqpoint{2.430544in}{0.936608in}}{\pgfqpoint{2.427272in}{0.928708in}}{\pgfqpoint{2.427272in}{0.920472in}}%
\pgfpathcurveto{\pgfqpoint{2.427272in}{0.912236in}}{\pgfqpoint{2.430544in}{0.904335in}}{\pgfqpoint{2.436368in}{0.898512in}}%
\pgfpathcurveto{\pgfqpoint{2.442192in}{0.892688in}}{\pgfqpoint{2.450092in}{0.889415in}}{\pgfqpoint{2.458328in}{0.889415in}}%
\pgfpathclose%
\pgfusepath{stroke,fill}%
\end{pgfscope}%
\begin{pgfscope}%
\pgfpathrectangle{\pgfqpoint{0.100000in}{0.220728in}}{\pgfqpoint{3.696000in}{3.696000in}}%
\pgfusepath{clip}%
\pgfsetbuttcap%
\pgfsetroundjoin%
\definecolor{currentfill}{rgb}{0.121569,0.466667,0.705882}%
\pgfsetfillcolor{currentfill}%
\pgfsetfillopacity{0.977543}%
\pgfsetlinewidth{1.003750pt}%
\definecolor{currentstroke}{rgb}{0.121569,0.466667,0.705882}%
\pgfsetstrokecolor{currentstroke}%
\pgfsetstrokeopacity{0.977543}%
\pgfsetdash{}{0pt}%
\pgfpathmoveto{\pgfqpoint{2.235427in}{0.757723in}}%
\pgfpathcurveto{\pgfqpoint{2.243663in}{0.757723in}}{\pgfqpoint{2.251563in}{0.760995in}}{\pgfqpoint{2.257387in}{0.766819in}}%
\pgfpathcurveto{\pgfqpoint{2.263211in}{0.772643in}}{\pgfqpoint{2.266484in}{0.780543in}}{\pgfqpoint{2.266484in}{0.788779in}}%
\pgfpathcurveto{\pgfqpoint{2.266484in}{0.797016in}}{\pgfqpoint{2.263211in}{0.804916in}}{\pgfqpoint{2.257387in}{0.810740in}}%
\pgfpathcurveto{\pgfqpoint{2.251563in}{0.816563in}}{\pgfqpoint{2.243663in}{0.819836in}}{\pgfqpoint{2.235427in}{0.819836in}}%
\pgfpathcurveto{\pgfqpoint{2.227191in}{0.819836in}}{\pgfqpoint{2.219291in}{0.816563in}}{\pgfqpoint{2.213467in}{0.810740in}}%
\pgfpathcurveto{\pgfqpoint{2.207643in}{0.804916in}}{\pgfqpoint{2.204371in}{0.797016in}}{\pgfqpoint{2.204371in}{0.788779in}}%
\pgfpathcurveto{\pgfqpoint{2.204371in}{0.780543in}}{\pgfqpoint{2.207643in}{0.772643in}}{\pgfqpoint{2.213467in}{0.766819in}}%
\pgfpathcurveto{\pgfqpoint{2.219291in}{0.760995in}}{\pgfqpoint{2.227191in}{0.757723in}}{\pgfqpoint{2.235427in}{0.757723in}}%
\pgfpathclose%
\pgfusepath{stroke,fill}%
\end{pgfscope}%
\begin{pgfscope}%
\pgfpathrectangle{\pgfqpoint{0.100000in}{0.220728in}}{\pgfqpoint{3.696000in}{3.696000in}}%
\pgfusepath{clip}%
\pgfsetbuttcap%
\pgfsetroundjoin%
\definecolor{currentfill}{rgb}{0.121569,0.466667,0.705882}%
\pgfsetfillcolor{currentfill}%
\pgfsetfillopacity{0.977944}%
\pgfsetlinewidth{1.003750pt}%
\definecolor{currentstroke}{rgb}{0.121569,0.466667,0.705882}%
\pgfsetstrokecolor{currentstroke}%
\pgfsetstrokeopacity{0.977944}%
\pgfsetdash{}{0pt}%
\pgfpathmoveto{\pgfqpoint{2.444419in}{0.867530in}}%
\pgfpathcurveto{\pgfqpoint{2.452655in}{0.867530in}}{\pgfqpoint{2.460555in}{0.870803in}}{\pgfqpoint{2.466379in}{0.876626in}}%
\pgfpathcurveto{\pgfqpoint{2.472203in}{0.882450in}}{\pgfqpoint{2.475475in}{0.890350in}}{\pgfqpoint{2.475475in}{0.898587in}}%
\pgfpathcurveto{\pgfqpoint{2.475475in}{0.906823in}}{\pgfqpoint{2.472203in}{0.914723in}}{\pgfqpoint{2.466379in}{0.920547in}}%
\pgfpathcurveto{\pgfqpoint{2.460555in}{0.926371in}}{\pgfqpoint{2.452655in}{0.929643in}}{\pgfqpoint{2.444419in}{0.929643in}}%
\pgfpathcurveto{\pgfqpoint{2.436183in}{0.929643in}}{\pgfqpoint{2.428283in}{0.926371in}}{\pgfqpoint{2.422459in}{0.920547in}}%
\pgfpathcurveto{\pgfqpoint{2.416635in}{0.914723in}}{\pgfqpoint{2.413362in}{0.906823in}}{\pgfqpoint{2.413362in}{0.898587in}}%
\pgfpathcurveto{\pgfqpoint{2.413362in}{0.890350in}}{\pgfqpoint{2.416635in}{0.882450in}}{\pgfqpoint{2.422459in}{0.876626in}}%
\pgfpathcurveto{\pgfqpoint{2.428283in}{0.870803in}}{\pgfqpoint{2.436183in}{0.867530in}}{\pgfqpoint{2.444419in}{0.867530in}}%
\pgfpathclose%
\pgfusepath{stroke,fill}%
\end{pgfscope}%
\begin{pgfscope}%
\pgfpathrectangle{\pgfqpoint{0.100000in}{0.220728in}}{\pgfqpoint{3.696000in}{3.696000in}}%
\pgfusepath{clip}%
\pgfsetbuttcap%
\pgfsetroundjoin%
\definecolor{currentfill}{rgb}{0.121569,0.466667,0.705882}%
\pgfsetfillcolor{currentfill}%
\pgfsetfillopacity{0.980935}%
\pgfsetlinewidth{1.003750pt}%
\definecolor{currentstroke}{rgb}{0.121569,0.466667,0.705882}%
\pgfsetstrokecolor{currentstroke}%
\pgfsetstrokeopacity{0.980935}%
\pgfsetdash{}{0pt}%
\pgfpathmoveto{\pgfqpoint{2.251351in}{0.757345in}}%
\pgfpathcurveto{\pgfqpoint{2.259587in}{0.757345in}}{\pgfqpoint{2.267487in}{0.760618in}}{\pgfqpoint{2.273311in}{0.766442in}}%
\pgfpathcurveto{\pgfqpoint{2.279135in}{0.772266in}}{\pgfqpoint{2.282408in}{0.780166in}}{\pgfqpoint{2.282408in}{0.788402in}}%
\pgfpathcurveto{\pgfqpoint{2.282408in}{0.796638in}}{\pgfqpoint{2.279135in}{0.804538in}}{\pgfqpoint{2.273311in}{0.810362in}}%
\pgfpathcurveto{\pgfqpoint{2.267487in}{0.816186in}}{\pgfqpoint{2.259587in}{0.819458in}}{\pgfqpoint{2.251351in}{0.819458in}}%
\pgfpathcurveto{\pgfqpoint{2.243115in}{0.819458in}}{\pgfqpoint{2.235215in}{0.816186in}}{\pgfqpoint{2.229391in}{0.810362in}}%
\pgfpathcurveto{\pgfqpoint{2.223567in}{0.804538in}}{\pgfqpoint{2.220295in}{0.796638in}}{\pgfqpoint{2.220295in}{0.788402in}}%
\pgfpathcurveto{\pgfqpoint{2.220295in}{0.780166in}}{\pgfqpoint{2.223567in}{0.772266in}}{\pgfqpoint{2.229391in}{0.766442in}}%
\pgfpathcurveto{\pgfqpoint{2.235215in}{0.760618in}}{\pgfqpoint{2.243115in}{0.757345in}}{\pgfqpoint{2.251351in}{0.757345in}}%
\pgfpathclose%
\pgfusepath{stroke,fill}%
\end{pgfscope}%
\begin{pgfscope}%
\pgfpathrectangle{\pgfqpoint{0.100000in}{0.220728in}}{\pgfqpoint{3.696000in}{3.696000in}}%
\pgfusepath{clip}%
\pgfsetbuttcap%
\pgfsetroundjoin%
\definecolor{currentfill}{rgb}{0.121569,0.466667,0.705882}%
\pgfsetfillcolor{currentfill}%
\pgfsetfillopacity{0.982406}%
\pgfsetlinewidth{1.003750pt}%
\definecolor{currentstroke}{rgb}{0.121569,0.466667,0.705882}%
\pgfsetstrokecolor{currentstroke}%
\pgfsetstrokeopacity{0.982406}%
\pgfsetdash{}{0pt}%
\pgfpathmoveto{\pgfqpoint{2.266525in}{0.750225in}}%
\pgfpathcurveto{\pgfqpoint{2.274762in}{0.750225in}}{\pgfqpoint{2.282662in}{0.753497in}}{\pgfqpoint{2.288485in}{0.759321in}}%
\pgfpathcurveto{\pgfqpoint{2.294309in}{0.765145in}}{\pgfqpoint{2.297582in}{0.773045in}}{\pgfqpoint{2.297582in}{0.781282in}}%
\pgfpathcurveto{\pgfqpoint{2.297582in}{0.789518in}}{\pgfqpoint{2.294309in}{0.797418in}}{\pgfqpoint{2.288485in}{0.803242in}}%
\pgfpathcurveto{\pgfqpoint{2.282662in}{0.809066in}}{\pgfqpoint{2.274762in}{0.812338in}}{\pgfqpoint{2.266525in}{0.812338in}}%
\pgfpathcurveto{\pgfqpoint{2.258289in}{0.812338in}}{\pgfqpoint{2.250389in}{0.809066in}}{\pgfqpoint{2.244565in}{0.803242in}}%
\pgfpathcurveto{\pgfqpoint{2.238741in}{0.797418in}}{\pgfqpoint{2.235469in}{0.789518in}}{\pgfqpoint{2.235469in}{0.781282in}}%
\pgfpathcurveto{\pgfqpoint{2.235469in}{0.773045in}}{\pgfqpoint{2.238741in}{0.765145in}}{\pgfqpoint{2.244565in}{0.759321in}}%
\pgfpathcurveto{\pgfqpoint{2.250389in}{0.753497in}}{\pgfqpoint{2.258289in}{0.750225in}}{\pgfqpoint{2.266525in}{0.750225in}}%
\pgfpathclose%
\pgfusepath{stroke,fill}%
\end{pgfscope}%
\begin{pgfscope}%
\pgfpathrectangle{\pgfqpoint{0.100000in}{0.220728in}}{\pgfqpoint{3.696000in}{3.696000in}}%
\pgfusepath{clip}%
\pgfsetbuttcap%
\pgfsetroundjoin%
\definecolor{currentfill}{rgb}{0.121569,0.466667,0.705882}%
\pgfsetfillcolor{currentfill}%
\pgfsetfillopacity{0.982609}%
\pgfsetlinewidth{1.003750pt}%
\definecolor{currentstroke}{rgb}{0.121569,0.466667,0.705882}%
\pgfsetstrokecolor{currentstroke}%
\pgfsetstrokeopacity{0.982609}%
\pgfsetdash{}{0pt}%
\pgfpathmoveto{\pgfqpoint{2.436953in}{0.844835in}}%
\pgfpathcurveto{\pgfqpoint{2.445189in}{0.844835in}}{\pgfqpoint{2.453089in}{0.848107in}}{\pgfqpoint{2.458913in}{0.853931in}}%
\pgfpathcurveto{\pgfqpoint{2.464737in}{0.859755in}}{\pgfqpoint{2.468009in}{0.867655in}}{\pgfqpoint{2.468009in}{0.875891in}}%
\pgfpathcurveto{\pgfqpoint{2.468009in}{0.884128in}}{\pgfqpoint{2.464737in}{0.892028in}}{\pgfqpoint{2.458913in}{0.897852in}}%
\pgfpathcurveto{\pgfqpoint{2.453089in}{0.903676in}}{\pgfqpoint{2.445189in}{0.906948in}}{\pgfqpoint{2.436953in}{0.906948in}}%
\pgfpathcurveto{\pgfqpoint{2.428716in}{0.906948in}}{\pgfqpoint{2.420816in}{0.903676in}}{\pgfqpoint{2.414992in}{0.897852in}}%
\pgfpathcurveto{\pgfqpoint{2.409168in}{0.892028in}}{\pgfqpoint{2.405896in}{0.884128in}}{\pgfqpoint{2.405896in}{0.875891in}}%
\pgfpathcurveto{\pgfqpoint{2.405896in}{0.867655in}}{\pgfqpoint{2.409168in}{0.859755in}}{\pgfqpoint{2.414992in}{0.853931in}}%
\pgfpathcurveto{\pgfqpoint{2.420816in}{0.848107in}}{\pgfqpoint{2.428716in}{0.844835in}}{\pgfqpoint{2.436953in}{0.844835in}}%
\pgfpathclose%
\pgfusepath{stroke,fill}%
\end{pgfscope}%
\begin{pgfscope}%
\pgfpathrectangle{\pgfqpoint{0.100000in}{0.220728in}}{\pgfqpoint{3.696000in}{3.696000in}}%
\pgfusepath{clip}%
\pgfsetbuttcap%
\pgfsetroundjoin%
\definecolor{currentfill}{rgb}{0.121569,0.466667,0.705882}%
\pgfsetfillcolor{currentfill}%
\pgfsetfillopacity{0.984737}%
\pgfsetlinewidth{1.003750pt}%
\definecolor{currentstroke}{rgb}{0.121569,0.466667,0.705882}%
\pgfsetstrokecolor{currentstroke}%
\pgfsetstrokeopacity{0.984737}%
\pgfsetdash{}{0pt}%
\pgfpathmoveto{\pgfqpoint{2.280647in}{0.745834in}}%
\pgfpathcurveto{\pgfqpoint{2.288883in}{0.745834in}}{\pgfqpoint{2.296783in}{0.749106in}}{\pgfqpoint{2.302607in}{0.754930in}}%
\pgfpathcurveto{\pgfqpoint{2.308431in}{0.760754in}}{\pgfqpoint{2.311703in}{0.768654in}}{\pgfqpoint{2.311703in}{0.776890in}}%
\pgfpathcurveto{\pgfqpoint{2.311703in}{0.785126in}}{\pgfqpoint{2.308431in}{0.793026in}}{\pgfqpoint{2.302607in}{0.798850in}}%
\pgfpathcurveto{\pgfqpoint{2.296783in}{0.804674in}}{\pgfqpoint{2.288883in}{0.807947in}}{\pgfqpoint{2.280647in}{0.807947in}}%
\pgfpathcurveto{\pgfqpoint{2.272411in}{0.807947in}}{\pgfqpoint{2.264511in}{0.804674in}}{\pgfqpoint{2.258687in}{0.798850in}}%
\pgfpathcurveto{\pgfqpoint{2.252863in}{0.793026in}}{\pgfqpoint{2.249590in}{0.785126in}}{\pgfqpoint{2.249590in}{0.776890in}}%
\pgfpathcurveto{\pgfqpoint{2.249590in}{0.768654in}}{\pgfqpoint{2.252863in}{0.760754in}}{\pgfqpoint{2.258687in}{0.754930in}}%
\pgfpathcurveto{\pgfqpoint{2.264511in}{0.749106in}}{\pgfqpoint{2.272411in}{0.745834in}}{\pgfqpoint{2.280647in}{0.745834in}}%
\pgfpathclose%
\pgfusepath{stroke,fill}%
\end{pgfscope}%
\begin{pgfscope}%
\pgfpathrectangle{\pgfqpoint{0.100000in}{0.220728in}}{\pgfqpoint{3.696000in}{3.696000in}}%
\pgfusepath{clip}%
\pgfsetbuttcap%
\pgfsetroundjoin%
\definecolor{currentfill}{rgb}{0.121569,0.466667,0.705882}%
\pgfsetfillcolor{currentfill}%
\pgfsetfillopacity{0.984917}%
\pgfsetlinewidth{1.003750pt}%
\definecolor{currentstroke}{rgb}{0.121569,0.466667,0.705882}%
\pgfsetstrokecolor{currentstroke}%
\pgfsetstrokeopacity{0.984917}%
\pgfsetdash{}{0pt}%
\pgfpathmoveto{\pgfqpoint{2.431455in}{0.832538in}}%
\pgfpathcurveto{\pgfqpoint{2.439692in}{0.832538in}}{\pgfqpoint{2.447592in}{0.835810in}}{\pgfqpoint{2.453416in}{0.841634in}}%
\pgfpathcurveto{\pgfqpoint{2.459240in}{0.847458in}}{\pgfqpoint{2.462512in}{0.855358in}}{\pgfqpoint{2.462512in}{0.863595in}}%
\pgfpathcurveto{\pgfqpoint{2.462512in}{0.871831in}}{\pgfqpoint{2.459240in}{0.879731in}}{\pgfqpoint{2.453416in}{0.885555in}}%
\pgfpathcurveto{\pgfqpoint{2.447592in}{0.891379in}}{\pgfqpoint{2.439692in}{0.894651in}}{\pgfqpoint{2.431455in}{0.894651in}}%
\pgfpathcurveto{\pgfqpoint{2.423219in}{0.894651in}}{\pgfqpoint{2.415319in}{0.891379in}}{\pgfqpoint{2.409495in}{0.885555in}}%
\pgfpathcurveto{\pgfqpoint{2.403671in}{0.879731in}}{\pgfqpoint{2.400399in}{0.871831in}}{\pgfqpoint{2.400399in}{0.863595in}}%
\pgfpathcurveto{\pgfqpoint{2.400399in}{0.855358in}}{\pgfqpoint{2.403671in}{0.847458in}}{\pgfqpoint{2.409495in}{0.841634in}}%
\pgfpathcurveto{\pgfqpoint{2.415319in}{0.835810in}}{\pgfqpoint{2.423219in}{0.832538in}}{\pgfqpoint{2.431455in}{0.832538in}}%
\pgfpathclose%
\pgfusepath{stroke,fill}%
\end{pgfscope}%
\begin{pgfscope}%
\pgfpathrectangle{\pgfqpoint{0.100000in}{0.220728in}}{\pgfqpoint{3.696000in}{3.696000in}}%
\pgfusepath{clip}%
\pgfsetbuttcap%
\pgfsetroundjoin%
\definecolor{currentfill}{rgb}{0.121569,0.466667,0.705882}%
\pgfsetfillcolor{currentfill}%
\pgfsetfillopacity{0.985421}%
\pgfsetlinewidth{1.003750pt}%
\definecolor{currentstroke}{rgb}{0.121569,0.466667,0.705882}%
\pgfsetstrokecolor{currentstroke}%
\pgfsetstrokeopacity{0.985421}%
\pgfsetdash{}{0pt}%
\pgfpathmoveto{\pgfqpoint{2.426965in}{0.825014in}}%
\pgfpathcurveto{\pgfqpoint{2.435201in}{0.825014in}}{\pgfqpoint{2.443101in}{0.828287in}}{\pgfqpoint{2.448925in}{0.834111in}}%
\pgfpathcurveto{\pgfqpoint{2.454749in}{0.839935in}}{\pgfqpoint{2.458021in}{0.847835in}}{\pgfqpoint{2.458021in}{0.856071in}}%
\pgfpathcurveto{\pgfqpoint{2.458021in}{0.864307in}}{\pgfqpoint{2.454749in}{0.872207in}}{\pgfqpoint{2.448925in}{0.878031in}}%
\pgfpathcurveto{\pgfqpoint{2.443101in}{0.883855in}}{\pgfqpoint{2.435201in}{0.887127in}}{\pgfqpoint{2.426965in}{0.887127in}}%
\pgfpathcurveto{\pgfqpoint{2.418729in}{0.887127in}}{\pgfqpoint{2.410829in}{0.883855in}}{\pgfqpoint{2.405005in}{0.878031in}}%
\pgfpathcurveto{\pgfqpoint{2.399181in}{0.872207in}}{\pgfqpoint{2.395908in}{0.864307in}}{\pgfqpoint{2.395908in}{0.856071in}}%
\pgfpathcurveto{\pgfqpoint{2.395908in}{0.847835in}}{\pgfqpoint{2.399181in}{0.839935in}}{\pgfqpoint{2.405005in}{0.834111in}}%
\pgfpathcurveto{\pgfqpoint{2.410829in}{0.828287in}}{\pgfqpoint{2.418729in}{0.825014in}}{\pgfqpoint{2.426965in}{0.825014in}}%
\pgfpathclose%
\pgfusepath{stroke,fill}%
\end{pgfscope}%
\begin{pgfscope}%
\pgfpathrectangle{\pgfqpoint{0.100000in}{0.220728in}}{\pgfqpoint{3.696000in}{3.696000in}}%
\pgfusepath{clip}%
\pgfsetbuttcap%
\pgfsetroundjoin%
\definecolor{currentfill}{rgb}{0.121569,0.466667,0.705882}%
\pgfsetfillcolor{currentfill}%
\pgfsetfillopacity{0.986001}%
\pgfsetlinewidth{1.003750pt}%
\definecolor{currentstroke}{rgb}{0.121569,0.466667,0.705882}%
\pgfsetstrokecolor{currentstroke}%
\pgfsetstrokeopacity{0.986001}%
\pgfsetdash{}{0pt}%
\pgfpathmoveto{\pgfqpoint{2.425804in}{0.820422in}}%
\pgfpathcurveto{\pgfqpoint{2.434040in}{0.820422in}}{\pgfqpoint{2.441940in}{0.823694in}}{\pgfqpoint{2.447764in}{0.829518in}}%
\pgfpathcurveto{\pgfqpoint{2.453588in}{0.835342in}}{\pgfqpoint{2.456860in}{0.843242in}}{\pgfqpoint{2.456860in}{0.851478in}}%
\pgfpathcurveto{\pgfqpoint{2.456860in}{0.859714in}}{\pgfqpoint{2.453588in}{0.867614in}}{\pgfqpoint{2.447764in}{0.873438in}}%
\pgfpathcurveto{\pgfqpoint{2.441940in}{0.879262in}}{\pgfqpoint{2.434040in}{0.882535in}}{\pgfqpoint{2.425804in}{0.882535in}}%
\pgfpathcurveto{\pgfqpoint{2.417568in}{0.882535in}}{\pgfqpoint{2.409668in}{0.879262in}}{\pgfqpoint{2.403844in}{0.873438in}}%
\pgfpathcurveto{\pgfqpoint{2.398020in}{0.867614in}}{\pgfqpoint{2.394747in}{0.859714in}}{\pgfqpoint{2.394747in}{0.851478in}}%
\pgfpathcurveto{\pgfqpoint{2.394747in}{0.843242in}}{\pgfqpoint{2.398020in}{0.835342in}}{\pgfqpoint{2.403844in}{0.829518in}}%
\pgfpathcurveto{\pgfqpoint{2.409668in}{0.823694in}}{\pgfqpoint{2.417568in}{0.820422in}}{\pgfqpoint{2.425804in}{0.820422in}}%
\pgfpathclose%
\pgfusepath{stroke,fill}%
\end{pgfscope}%
\begin{pgfscope}%
\pgfpathrectangle{\pgfqpoint{0.100000in}{0.220728in}}{\pgfqpoint{3.696000in}{3.696000in}}%
\pgfusepath{clip}%
\pgfsetbuttcap%
\pgfsetroundjoin%
\definecolor{currentfill}{rgb}{0.121569,0.466667,0.705882}%
\pgfsetfillcolor{currentfill}%
\pgfsetfillopacity{0.986671}%
\pgfsetlinewidth{1.003750pt}%
\definecolor{currentstroke}{rgb}{0.121569,0.466667,0.705882}%
\pgfsetstrokecolor{currentstroke}%
\pgfsetstrokeopacity{0.986671}%
\pgfsetdash{}{0pt}%
\pgfpathmoveto{\pgfqpoint{2.291404in}{0.744175in}}%
\pgfpathcurveto{\pgfqpoint{2.299641in}{0.744175in}}{\pgfqpoint{2.307541in}{0.747448in}}{\pgfqpoint{2.313365in}{0.753272in}}%
\pgfpathcurveto{\pgfqpoint{2.319189in}{0.759096in}}{\pgfqpoint{2.322461in}{0.766996in}}{\pgfqpoint{2.322461in}{0.775232in}}%
\pgfpathcurveto{\pgfqpoint{2.322461in}{0.783468in}}{\pgfqpoint{2.319189in}{0.791368in}}{\pgfqpoint{2.313365in}{0.797192in}}%
\pgfpathcurveto{\pgfqpoint{2.307541in}{0.803016in}}{\pgfqpoint{2.299641in}{0.806288in}}{\pgfqpoint{2.291404in}{0.806288in}}%
\pgfpathcurveto{\pgfqpoint{2.283168in}{0.806288in}}{\pgfqpoint{2.275268in}{0.803016in}}{\pgfqpoint{2.269444in}{0.797192in}}%
\pgfpathcurveto{\pgfqpoint{2.263620in}{0.791368in}}{\pgfqpoint{2.260348in}{0.783468in}}{\pgfqpoint{2.260348in}{0.775232in}}%
\pgfpathcurveto{\pgfqpoint{2.260348in}{0.766996in}}{\pgfqpoint{2.263620in}{0.759096in}}{\pgfqpoint{2.269444in}{0.753272in}}%
\pgfpathcurveto{\pgfqpoint{2.275268in}{0.747448in}}{\pgfqpoint{2.283168in}{0.744175in}}{\pgfqpoint{2.291404in}{0.744175in}}%
\pgfpathclose%
\pgfusepath{stroke,fill}%
\end{pgfscope}%
\begin{pgfscope}%
\pgfpathrectangle{\pgfqpoint{0.100000in}{0.220728in}}{\pgfqpoint{3.696000in}{3.696000in}}%
\pgfusepath{clip}%
\pgfsetbuttcap%
\pgfsetroundjoin%
\definecolor{currentfill}{rgb}{0.121569,0.466667,0.705882}%
\pgfsetfillcolor{currentfill}%
\pgfsetfillopacity{0.986721}%
\pgfsetlinewidth{1.003750pt}%
\definecolor{currentstroke}{rgb}{0.121569,0.466667,0.705882}%
\pgfsetstrokecolor{currentstroke}%
\pgfsetstrokeopacity{0.986721}%
\pgfsetdash{}{0pt}%
\pgfpathmoveto{\pgfqpoint{2.419192in}{0.810400in}}%
\pgfpathcurveto{\pgfqpoint{2.427429in}{0.810400in}}{\pgfqpoint{2.435329in}{0.813672in}}{\pgfqpoint{2.441153in}{0.819496in}}%
\pgfpathcurveto{\pgfqpoint{2.446977in}{0.825320in}}{\pgfqpoint{2.450249in}{0.833220in}}{\pgfqpoint{2.450249in}{0.841456in}}%
\pgfpathcurveto{\pgfqpoint{2.450249in}{0.849693in}}{\pgfqpoint{2.446977in}{0.857593in}}{\pgfqpoint{2.441153in}{0.863416in}}%
\pgfpathcurveto{\pgfqpoint{2.435329in}{0.869240in}}{\pgfqpoint{2.427429in}{0.872513in}}{\pgfqpoint{2.419192in}{0.872513in}}%
\pgfpathcurveto{\pgfqpoint{2.410956in}{0.872513in}}{\pgfqpoint{2.403056in}{0.869240in}}{\pgfqpoint{2.397232in}{0.863416in}}%
\pgfpathcurveto{\pgfqpoint{2.391408in}{0.857593in}}{\pgfqpoint{2.388136in}{0.849693in}}{\pgfqpoint{2.388136in}{0.841456in}}%
\pgfpathcurveto{\pgfqpoint{2.388136in}{0.833220in}}{\pgfqpoint{2.391408in}{0.825320in}}{\pgfqpoint{2.397232in}{0.819496in}}%
\pgfpathcurveto{\pgfqpoint{2.403056in}{0.813672in}}{\pgfqpoint{2.410956in}{0.810400in}}{\pgfqpoint{2.419192in}{0.810400in}}%
\pgfpathclose%
\pgfusepath{stroke,fill}%
\end{pgfscope}%
\begin{pgfscope}%
\pgfpathrectangle{\pgfqpoint{0.100000in}{0.220728in}}{\pgfqpoint{3.696000in}{3.696000in}}%
\pgfusepath{clip}%
\pgfsetbuttcap%
\pgfsetroundjoin%
\definecolor{currentfill}{rgb}{0.121569,0.466667,0.705882}%
\pgfsetfillcolor{currentfill}%
\pgfsetfillopacity{0.987393}%
\pgfsetlinewidth{1.003750pt}%
\definecolor{currentstroke}{rgb}{0.121569,0.466667,0.705882}%
\pgfsetstrokecolor{currentstroke}%
\pgfsetstrokeopacity{0.987393}%
\pgfsetdash{}{0pt}%
\pgfpathmoveto{\pgfqpoint{2.299419in}{0.738700in}}%
\pgfpathcurveto{\pgfqpoint{2.307655in}{0.738700in}}{\pgfqpoint{2.315555in}{0.741972in}}{\pgfqpoint{2.321379in}{0.747796in}}%
\pgfpathcurveto{\pgfqpoint{2.327203in}{0.753620in}}{\pgfqpoint{2.330476in}{0.761520in}}{\pgfqpoint{2.330476in}{0.769757in}}%
\pgfpathcurveto{\pgfqpoint{2.330476in}{0.777993in}}{\pgfqpoint{2.327203in}{0.785893in}}{\pgfqpoint{2.321379in}{0.791717in}}%
\pgfpathcurveto{\pgfqpoint{2.315555in}{0.797541in}}{\pgfqpoint{2.307655in}{0.800813in}}{\pgfqpoint{2.299419in}{0.800813in}}%
\pgfpathcurveto{\pgfqpoint{2.291183in}{0.800813in}}{\pgfqpoint{2.283283in}{0.797541in}}{\pgfqpoint{2.277459in}{0.791717in}}%
\pgfpathcurveto{\pgfqpoint{2.271635in}{0.785893in}}{\pgfqpoint{2.268363in}{0.777993in}}{\pgfqpoint{2.268363in}{0.769757in}}%
\pgfpathcurveto{\pgfqpoint{2.268363in}{0.761520in}}{\pgfqpoint{2.271635in}{0.753620in}}{\pgfqpoint{2.277459in}{0.747796in}}%
\pgfpathcurveto{\pgfqpoint{2.283283in}{0.741972in}}{\pgfqpoint{2.291183in}{0.738700in}}{\pgfqpoint{2.299419in}{0.738700in}}%
\pgfpathclose%
\pgfusepath{stroke,fill}%
\end{pgfscope}%
\begin{pgfscope}%
\pgfpathrectangle{\pgfqpoint{0.100000in}{0.220728in}}{\pgfqpoint{3.696000in}{3.696000in}}%
\pgfusepath{clip}%
\pgfsetbuttcap%
\pgfsetroundjoin%
\definecolor{currentfill}{rgb}{0.121569,0.466667,0.705882}%
\pgfsetfillcolor{currentfill}%
\pgfsetfillopacity{0.988482}%
\pgfsetlinewidth{1.003750pt}%
\definecolor{currentstroke}{rgb}{0.121569,0.466667,0.705882}%
\pgfsetstrokecolor{currentstroke}%
\pgfsetstrokeopacity{0.988482}%
\pgfsetdash{}{0pt}%
\pgfpathmoveto{\pgfqpoint{2.415689in}{0.796427in}}%
\pgfpathcurveto{\pgfqpoint{2.423926in}{0.796427in}}{\pgfqpoint{2.431826in}{0.799699in}}{\pgfqpoint{2.437650in}{0.805523in}}%
\pgfpathcurveto{\pgfqpoint{2.443474in}{0.811347in}}{\pgfqpoint{2.446746in}{0.819247in}}{\pgfqpoint{2.446746in}{0.827484in}}%
\pgfpathcurveto{\pgfqpoint{2.446746in}{0.835720in}}{\pgfqpoint{2.443474in}{0.843620in}}{\pgfqpoint{2.437650in}{0.849444in}}%
\pgfpathcurveto{\pgfqpoint{2.431826in}{0.855268in}}{\pgfqpoint{2.423926in}{0.858540in}}{\pgfqpoint{2.415689in}{0.858540in}}%
\pgfpathcurveto{\pgfqpoint{2.407453in}{0.858540in}}{\pgfqpoint{2.399553in}{0.855268in}}{\pgfqpoint{2.393729in}{0.849444in}}%
\pgfpathcurveto{\pgfqpoint{2.387905in}{0.843620in}}{\pgfqpoint{2.384633in}{0.835720in}}{\pgfqpoint{2.384633in}{0.827484in}}%
\pgfpathcurveto{\pgfqpoint{2.384633in}{0.819247in}}{\pgfqpoint{2.387905in}{0.811347in}}{\pgfqpoint{2.393729in}{0.805523in}}%
\pgfpathcurveto{\pgfqpoint{2.399553in}{0.799699in}}{\pgfqpoint{2.407453in}{0.796427in}}{\pgfqpoint{2.415689in}{0.796427in}}%
\pgfpathclose%
\pgfusepath{stroke,fill}%
\end{pgfscope}%
\begin{pgfscope}%
\pgfpathrectangle{\pgfqpoint{0.100000in}{0.220728in}}{\pgfqpoint{3.696000in}{3.696000in}}%
\pgfusepath{clip}%
\pgfsetbuttcap%
\pgfsetroundjoin%
\definecolor{currentfill}{rgb}{0.121569,0.466667,0.705882}%
\pgfsetfillcolor{currentfill}%
\pgfsetfillopacity{0.990119}%
\pgfsetlinewidth{1.003750pt}%
\definecolor{currentstroke}{rgb}{0.121569,0.466667,0.705882}%
\pgfsetstrokecolor{currentstroke}%
\pgfsetstrokeopacity{0.990119}%
\pgfsetdash{}{0pt}%
\pgfpathmoveto{\pgfqpoint{2.314435in}{0.735680in}}%
\pgfpathcurveto{\pgfqpoint{2.322671in}{0.735680in}}{\pgfqpoint{2.330572in}{0.738953in}}{\pgfqpoint{2.336395in}{0.744776in}}%
\pgfpathcurveto{\pgfqpoint{2.342219in}{0.750600in}}{\pgfqpoint{2.345492in}{0.758500in}}{\pgfqpoint{2.345492in}{0.766737in}}%
\pgfpathcurveto{\pgfqpoint{2.345492in}{0.774973in}}{\pgfqpoint{2.342219in}{0.782873in}}{\pgfqpoint{2.336395in}{0.788697in}}%
\pgfpathcurveto{\pgfqpoint{2.330572in}{0.794521in}}{\pgfqpoint{2.322671in}{0.797793in}}{\pgfqpoint{2.314435in}{0.797793in}}%
\pgfpathcurveto{\pgfqpoint{2.306199in}{0.797793in}}{\pgfqpoint{2.298299in}{0.794521in}}{\pgfqpoint{2.292475in}{0.788697in}}%
\pgfpathcurveto{\pgfqpoint{2.286651in}{0.782873in}}{\pgfqpoint{2.283379in}{0.774973in}}{\pgfqpoint{2.283379in}{0.766737in}}%
\pgfpathcurveto{\pgfqpoint{2.283379in}{0.758500in}}{\pgfqpoint{2.286651in}{0.750600in}}{\pgfqpoint{2.292475in}{0.744776in}}%
\pgfpathcurveto{\pgfqpoint{2.298299in}{0.738953in}}{\pgfqpoint{2.306199in}{0.735680in}}{\pgfqpoint{2.314435in}{0.735680in}}%
\pgfpathclose%
\pgfusepath{stroke,fill}%
\end{pgfscope}%
\begin{pgfscope}%
\pgfpathrectangle{\pgfqpoint{0.100000in}{0.220728in}}{\pgfqpoint{3.696000in}{3.696000in}}%
\pgfusepath{clip}%
\pgfsetbuttcap%
\pgfsetroundjoin%
\definecolor{currentfill}{rgb}{0.121569,0.466667,0.705882}%
\pgfsetfillcolor{currentfill}%
\pgfsetfillopacity{0.990167}%
\pgfsetlinewidth{1.003750pt}%
\definecolor{currentstroke}{rgb}{0.121569,0.466667,0.705882}%
\pgfsetstrokecolor{currentstroke}%
\pgfsetstrokeopacity{0.990167}%
\pgfsetdash{}{0pt}%
\pgfpathmoveto{\pgfqpoint{2.407575in}{0.779988in}}%
\pgfpathcurveto{\pgfqpoint{2.415812in}{0.779988in}}{\pgfqpoint{2.423712in}{0.783260in}}{\pgfqpoint{2.429536in}{0.789084in}}%
\pgfpathcurveto{\pgfqpoint{2.435360in}{0.794908in}}{\pgfqpoint{2.438632in}{0.802808in}}{\pgfqpoint{2.438632in}{0.811045in}}%
\pgfpathcurveto{\pgfqpoint{2.438632in}{0.819281in}}{\pgfqpoint{2.435360in}{0.827181in}}{\pgfqpoint{2.429536in}{0.833005in}}%
\pgfpathcurveto{\pgfqpoint{2.423712in}{0.838829in}}{\pgfqpoint{2.415812in}{0.842101in}}{\pgfqpoint{2.407575in}{0.842101in}}%
\pgfpathcurveto{\pgfqpoint{2.399339in}{0.842101in}}{\pgfqpoint{2.391439in}{0.838829in}}{\pgfqpoint{2.385615in}{0.833005in}}%
\pgfpathcurveto{\pgfqpoint{2.379791in}{0.827181in}}{\pgfqpoint{2.376519in}{0.819281in}}{\pgfqpoint{2.376519in}{0.811045in}}%
\pgfpathcurveto{\pgfqpoint{2.376519in}{0.802808in}}{\pgfqpoint{2.379791in}{0.794908in}}{\pgfqpoint{2.385615in}{0.789084in}}%
\pgfpathcurveto{\pgfqpoint{2.391439in}{0.783260in}}{\pgfqpoint{2.399339in}{0.779988in}}{\pgfqpoint{2.407575in}{0.779988in}}%
\pgfpathclose%
\pgfusepath{stroke,fill}%
\end{pgfscope}%
\begin{pgfscope}%
\pgfpathrectangle{\pgfqpoint{0.100000in}{0.220728in}}{\pgfqpoint{3.696000in}{3.696000in}}%
\pgfusepath{clip}%
\pgfsetbuttcap%
\pgfsetroundjoin%
\definecolor{currentfill}{rgb}{0.121569,0.466667,0.705882}%
\pgfsetfillcolor{currentfill}%
\pgfsetfillopacity{0.991498}%
\pgfsetlinewidth{1.003750pt}%
\definecolor{currentstroke}{rgb}{0.121569,0.466667,0.705882}%
\pgfsetstrokecolor{currentstroke}%
\pgfsetstrokeopacity{0.991498}%
\pgfsetdash{}{0pt}%
\pgfpathmoveto{\pgfqpoint{2.328924in}{0.731832in}}%
\pgfpathcurveto{\pgfqpoint{2.337160in}{0.731832in}}{\pgfqpoint{2.345060in}{0.735104in}}{\pgfqpoint{2.350884in}{0.740928in}}%
\pgfpathcurveto{\pgfqpoint{2.356708in}{0.746752in}}{\pgfqpoint{2.359981in}{0.754652in}}{\pgfqpoint{2.359981in}{0.762888in}}%
\pgfpathcurveto{\pgfqpoint{2.359981in}{0.771124in}}{\pgfqpoint{2.356708in}{0.779025in}}{\pgfqpoint{2.350884in}{0.784848in}}%
\pgfpathcurveto{\pgfqpoint{2.345060in}{0.790672in}}{\pgfqpoint{2.337160in}{0.793945in}}{\pgfqpoint{2.328924in}{0.793945in}}%
\pgfpathcurveto{\pgfqpoint{2.320688in}{0.793945in}}{\pgfqpoint{2.312788in}{0.790672in}}{\pgfqpoint{2.306964in}{0.784848in}}%
\pgfpathcurveto{\pgfqpoint{2.301140in}{0.779025in}}{\pgfqpoint{2.297868in}{0.771124in}}{\pgfqpoint{2.297868in}{0.762888in}}%
\pgfpathcurveto{\pgfqpoint{2.297868in}{0.754652in}}{\pgfqpoint{2.301140in}{0.746752in}}{\pgfqpoint{2.306964in}{0.740928in}}%
\pgfpathcurveto{\pgfqpoint{2.312788in}{0.735104in}}{\pgfqpoint{2.320688in}{0.731832in}}{\pgfqpoint{2.328924in}{0.731832in}}%
\pgfpathclose%
\pgfusepath{stroke,fill}%
\end{pgfscope}%
\begin{pgfscope}%
\pgfpathrectangle{\pgfqpoint{0.100000in}{0.220728in}}{\pgfqpoint{3.696000in}{3.696000in}}%
\pgfusepath{clip}%
\pgfsetbuttcap%
\pgfsetroundjoin%
\definecolor{currentfill}{rgb}{0.121569,0.466667,0.705882}%
\pgfsetfillcolor{currentfill}%
\pgfsetfillopacity{0.992687}%
\pgfsetlinewidth{1.003750pt}%
\definecolor{currentstroke}{rgb}{0.121569,0.466667,0.705882}%
\pgfsetstrokecolor{currentstroke}%
\pgfsetstrokeopacity{0.992687}%
\pgfsetdash{}{0pt}%
\pgfpathmoveto{\pgfqpoint{2.402374in}{0.762229in}}%
\pgfpathcurveto{\pgfqpoint{2.410611in}{0.762229in}}{\pgfqpoint{2.418511in}{0.765501in}}{\pgfqpoint{2.424335in}{0.771325in}}%
\pgfpathcurveto{\pgfqpoint{2.430158in}{0.777149in}}{\pgfqpoint{2.433431in}{0.785049in}}{\pgfqpoint{2.433431in}{0.793285in}}%
\pgfpathcurveto{\pgfqpoint{2.433431in}{0.801522in}}{\pgfqpoint{2.430158in}{0.809422in}}{\pgfqpoint{2.424335in}{0.815246in}}%
\pgfpathcurveto{\pgfqpoint{2.418511in}{0.821070in}}{\pgfqpoint{2.410611in}{0.824342in}}{\pgfqpoint{2.402374in}{0.824342in}}%
\pgfpathcurveto{\pgfqpoint{2.394138in}{0.824342in}}{\pgfqpoint{2.386238in}{0.821070in}}{\pgfqpoint{2.380414in}{0.815246in}}%
\pgfpathcurveto{\pgfqpoint{2.374590in}{0.809422in}}{\pgfqpoint{2.371318in}{0.801522in}}{\pgfqpoint{2.371318in}{0.793285in}}%
\pgfpathcurveto{\pgfqpoint{2.371318in}{0.785049in}}{\pgfqpoint{2.374590in}{0.777149in}}{\pgfqpoint{2.380414in}{0.771325in}}%
\pgfpathcurveto{\pgfqpoint{2.386238in}{0.765501in}}{\pgfqpoint{2.394138in}{0.762229in}}{\pgfqpoint{2.402374in}{0.762229in}}%
\pgfpathclose%
\pgfusepath{stroke,fill}%
\end{pgfscope}%
\begin{pgfscope}%
\pgfpathrectangle{\pgfqpoint{0.100000in}{0.220728in}}{\pgfqpoint{3.696000in}{3.696000in}}%
\pgfusepath{clip}%
\pgfsetbuttcap%
\pgfsetroundjoin%
\definecolor{currentfill}{rgb}{0.121569,0.466667,0.705882}%
\pgfsetfillcolor{currentfill}%
\pgfsetfillopacity{0.993306}%
\pgfsetlinewidth{1.003750pt}%
\definecolor{currentstroke}{rgb}{0.121569,0.466667,0.705882}%
\pgfsetstrokecolor{currentstroke}%
\pgfsetstrokeopacity{0.993306}%
\pgfsetdash{}{0pt}%
\pgfpathmoveto{\pgfqpoint{2.339377in}{0.728685in}}%
\pgfpathcurveto{\pgfqpoint{2.347614in}{0.728685in}}{\pgfqpoint{2.355514in}{0.731957in}}{\pgfqpoint{2.361338in}{0.737781in}}%
\pgfpathcurveto{\pgfqpoint{2.367162in}{0.743605in}}{\pgfqpoint{2.370434in}{0.751505in}}{\pgfqpoint{2.370434in}{0.759742in}}%
\pgfpathcurveto{\pgfqpoint{2.370434in}{0.767978in}}{\pgfqpoint{2.367162in}{0.775878in}}{\pgfqpoint{2.361338in}{0.781702in}}%
\pgfpathcurveto{\pgfqpoint{2.355514in}{0.787526in}}{\pgfqpoint{2.347614in}{0.790798in}}{\pgfqpoint{2.339377in}{0.790798in}}%
\pgfpathcurveto{\pgfqpoint{2.331141in}{0.790798in}}{\pgfqpoint{2.323241in}{0.787526in}}{\pgfqpoint{2.317417in}{0.781702in}}%
\pgfpathcurveto{\pgfqpoint{2.311593in}{0.775878in}}{\pgfqpoint{2.308321in}{0.767978in}}{\pgfqpoint{2.308321in}{0.759742in}}%
\pgfpathcurveto{\pgfqpoint{2.308321in}{0.751505in}}{\pgfqpoint{2.311593in}{0.743605in}}{\pgfqpoint{2.317417in}{0.737781in}}%
\pgfpathcurveto{\pgfqpoint{2.323241in}{0.731957in}}{\pgfqpoint{2.331141in}{0.728685in}}{\pgfqpoint{2.339377in}{0.728685in}}%
\pgfpathclose%
\pgfusepath{stroke,fill}%
\end{pgfscope}%
\begin{pgfscope}%
\pgfpathrectangle{\pgfqpoint{0.100000in}{0.220728in}}{\pgfqpoint{3.696000in}{3.696000in}}%
\pgfusepath{clip}%
\pgfsetbuttcap%
\pgfsetroundjoin%
\definecolor{currentfill}{rgb}{0.121569,0.466667,0.705882}%
\pgfsetfillcolor{currentfill}%
\pgfsetfillopacity{0.994921}%
\pgfsetlinewidth{1.003750pt}%
\definecolor{currentstroke}{rgb}{0.121569,0.466667,0.705882}%
\pgfsetstrokecolor{currentstroke}%
\pgfsetstrokeopacity{0.994921}%
\pgfsetdash{}{0pt}%
\pgfpathmoveto{\pgfqpoint{2.347361in}{0.727187in}}%
\pgfpathcurveto{\pgfqpoint{2.355597in}{0.727187in}}{\pgfqpoint{2.363497in}{0.730459in}}{\pgfqpoint{2.369321in}{0.736283in}}%
\pgfpathcurveto{\pgfqpoint{2.375145in}{0.742107in}}{\pgfqpoint{2.378418in}{0.750007in}}{\pgfqpoint{2.378418in}{0.758243in}}%
\pgfpathcurveto{\pgfqpoint{2.378418in}{0.766480in}}{\pgfqpoint{2.375145in}{0.774380in}}{\pgfqpoint{2.369321in}{0.780204in}}%
\pgfpathcurveto{\pgfqpoint{2.363497in}{0.786027in}}{\pgfqpoint{2.355597in}{0.789300in}}{\pgfqpoint{2.347361in}{0.789300in}}%
\pgfpathcurveto{\pgfqpoint{2.339125in}{0.789300in}}{\pgfqpoint{2.331225in}{0.786027in}}{\pgfqpoint{2.325401in}{0.780204in}}%
\pgfpathcurveto{\pgfqpoint{2.319577in}{0.774380in}}{\pgfqpoint{2.316305in}{0.766480in}}{\pgfqpoint{2.316305in}{0.758243in}}%
\pgfpathcurveto{\pgfqpoint{2.316305in}{0.750007in}}{\pgfqpoint{2.319577in}{0.742107in}}{\pgfqpoint{2.325401in}{0.736283in}}%
\pgfpathcurveto{\pgfqpoint{2.331225in}{0.730459in}}{\pgfqpoint{2.339125in}{0.727187in}}{\pgfqpoint{2.347361in}{0.727187in}}%
\pgfpathclose%
\pgfusepath{stroke,fill}%
\end{pgfscope}%
\begin{pgfscope}%
\pgfpathrectangle{\pgfqpoint{0.100000in}{0.220728in}}{\pgfqpoint{3.696000in}{3.696000in}}%
\pgfusepath{clip}%
\pgfsetbuttcap%
\pgfsetroundjoin%
\definecolor{currentfill}{rgb}{0.121569,0.466667,0.705882}%
\pgfsetfillcolor{currentfill}%
\pgfsetfillopacity{0.995805}%
\pgfsetlinewidth{1.003750pt}%
\definecolor{currentstroke}{rgb}{0.121569,0.466667,0.705882}%
\pgfsetstrokecolor{currentstroke}%
\pgfsetstrokeopacity{0.995805}%
\pgfsetdash{}{0pt}%
\pgfpathmoveto{\pgfqpoint{2.397285in}{0.745054in}}%
\pgfpathcurveto{\pgfqpoint{2.405522in}{0.745054in}}{\pgfqpoint{2.413422in}{0.748326in}}{\pgfqpoint{2.419246in}{0.754150in}}%
\pgfpathcurveto{\pgfqpoint{2.425070in}{0.759974in}}{\pgfqpoint{2.428342in}{0.767874in}}{\pgfqpoint{2.428342in}{0.776111in}}%
\pgfpathcurveto{\pgfqpoint{2.428342in}{0.784347in}}{\pgfqpoint{2.425070in}{0.792247in}}{\pgfqpoint{2.419246in}{0.798071in}}%
\pgfpathcurveto{\pgfqpoint{2.413422in}{0.803895in}}{\pgfqpoint{2.405522in}{0.807167in}}{\pgfqpoint{2.397285in}{0.807167in}}%
\pgfpathcurveto{\pgfqpoint{2.389049in}{0.807167in}}{\pgfqpoint{2.381149in}{0.803895in}}{\pgfqpoint{2.375325in}{0.798071in}}%
\pgfpathcurveto{\pgfqpoint{2.369501in}{0.792247in}}{\pgfqpoint{2.366229in}{0.784347in}}{\pgfqpoint{2.366229in}{0.776111in}}%
\pgfpathcurveto{\pgfqpoint{2.366229in}{0.767874in}}{\pgfqpoint{2.369501in}{0.759974in}}{\pgfqpoint{2.375325in}{0.754150in}}%
\pgfpathcurveto{\pgfqpoint{2.381149in}{0.748326in}}{\pgfqpoint{2.389049in}{0.745054in}}{\pgfqpoint{2.397285in}{0.745054in}}%
\pgfpathclose%
\pgfusepath{stroke,fill}%
\end{pgfscope}%
\begin{pgfscope}%
\pgfpathrectangle{\pgfqpoint{0.100000in}{0.220728in}}{\pgfqpoint{3.696000in}{3.696000in}}%
\pgfusepath{clip}%
\pgfsetbuttcap%
\pgfsetroundjoin%
\definecolor{currentfill}{rgb}{0.121569,0.466667,0.705882}%
\pgfsetfillcolor{currentfill}%
\pgfsetfillopacity{0.996083}%
\pgfsetlinewidth{1.003750pt}%
\definecolor{currentstroke}{rgb}{0.121569,0.466667,0.705882}%
\pgfsetstrokecolor{currentstroke}%
\pgfsetstrokeopacity{0.996083}%
\pgfsetdash{}{0pt}%
\pgfpathmoveto{\pgfqpoint{2.354481in}{0.724417in}}%
\pgfpathcurveto{\pgfqpoint{2.362717in}{0.724417in}}{\pgfqpoint{2.370617in}{0.727689in}}{\pgfqpoint{2.376441in}{0.733513in}}%
\pgfpathcurveto{\pgfqpoint{2.382265in}{0.739337in}}{\pgfqpoint{2.385537in}{0.747237in}}{\pgfqpoint{2.385537in}{0.755473in}}%
\pgfpathcurveto{\pgfqpoint{2.385537in}{0.763709in}}{\pgfqpoint{2.382265in}{0.771609in}}{\pgfqpoint{2.376441in}{0.777433in}}%
\pgfpathcurveto{\pgfqpoint{2.370617in}{0.783257in}}{\pgfqpoint{2.362717in}{0.786530in}}{\pgfqpoint{2.354481in}{0.786530in}}%
\pgfpathcurveto{\pgfqpoint{2.346244in}{0.786530in}}{\pgfqpoint{2.338344in}{0.783257in}}{\pgfqpoint{2.332520in}{0.777433in}}%
\pgfpathcurveto{\pgfqpoint{2.326696in}{0.771609in}}{\pgfqpoint{2.323424in}{0.763709in}}{\pgfqpoint{2.323424in}{0.755473in}}%
\pgfpathcurveto{\pgfqpoint{2.323424in}{0.747237in}}{\pgfqpoint{2.326696in}{0.739337in}}{\pgfqpoint{2.332520in}{0.733513in}}%
\pgfpathcurveto{\pgfqpoint{2.338344in}{0.727689in}}{\pgfqpoint{2.346244in}{0.724417in}}{\pgfqpoint{2.354481in}{0.724417in}}%
\pgfpathclose%
\pgfusepath{stroke,fill}%
\end{pgfscope}%
\begin{pgfscope}%
\pgfpathrectangle{\pgfqpoint{0.100000in}{0.220728in}}{\pgfqpoint{3.696000in}{3.696000in}}%
\pgfusepath{clip}%
\pgfsetbuttcap%
\pgfsetroundjoin%
\definecolor{currentfill}{rgb}{0.121569,0.466667,0.705882}%
\pgfsetfillcolor{currentfill}%
\pgfsetfillopacity{0.997068}%
\pgfsetlinewidth{1.003750pt}%
\definecolor{currentstroke}{rgb}{0.121569,0.466667,0.705882}%
\pgfsetstrokecolor{currentstroke}%
\pgfsetstrokeopacity{0.997068}%
\pgfsetdash{}{0pt}%
\pgfpathmoveto{\pgfqpoint{2.391881in}{0.736867in}}%
\pgfpathcurveto{\pgfqpoint{2.400117in}{0.736867in}}{\pgfqpoint{2.408017in}{0.740139in}}{\pgfqpoint{2.413841in}{0.745963in}}%
\pgfpathcurveto{\pgfqpoint{2.419665in}{0.751787in}}{\pgfqpoint{2.422937in}{0.759687in}}{\pgfqpoint{2.422937in}{0.767924in}}%
\pgfpathcurveto{\pgfqpoint{2.422937in}{0.776160in}}{\pgfqpoint{2.419665in}{0.784060in}}{\pgfqpoint{2.413841in}{0.789884in}}%
\pgfpathcurveto{\pgfqpoint{2.408017in}{0.795708in}}{\pgfqpoint{2.400117in}{0.798980in}}{\pgfqpoint{2.391881in}{0.798980in}}%
\pgfpathcurveto{\pgfqpoint{2.383645in}{0.798980in}}{\pgfqpoint{2.375744in}{0.795708in}}{\pgfqpoint{2.369921in}{0.789884in}}%
\pgfpathcurveto{\pgfqpoint{2.364097in}{0.784060in}}{\pgfqpoint{2.360824in}{0.776160in}}{\pgfqpoint{2.360824in}{0.767924in}}%
\pgfpathcurveto{\pgfqpoint{2.360824in}{0.759687in}}{\pgfqpoint{2.364097in}{0.751787in}}{\pgfqpoint{2.369921in}{0.745963in}}%
\pgfpathcurveto{\pgfqpoint{2.375744in}{0.740139in}}{\pgfqpoint{2.383645in}{0.736867in}}{\pgfqpoint{2.391881in}{0.736867in}}%
\pgfpathclose%
\pgfusepath{stroke,fill}%
\end{pgfscope}%
\begin{pgfscope}%
\pgfpathrectangle{\pgfqpoint{0.100000in}{0.220728in}}{\pgfqpoint{3.696000in}{3.696000in}}%
\pgfusepath{clip}%
\pgfsetbuttcap%
\pgfsetroundjoin%
\definecolor{currentfill}{rgb}{0.121569,0.466667,0.705882}%
\pgfsetfillcolor{currentfill}%
\pgfsetfillopacity{0.997085}%
\pgfsetlinewidth{1.003750pt}%
\definecolor{currentstroke}{rgb}{0.121569,0.466667,0.705882}%
\pgfsetstrokecolor{currentstroke}%
\pgfsetstrokeopacity{0.997085}%
\pgfsetdash{}{0pt}%
\pgfpathmoveto{\pgfqpoint{2.360623in}{0.721366in}}%
\pgfpathcurveto{\pgfqpoint{2.368859in}{0.721366in}}{\pgfqpoint{2.376759in}{0.724638in}}{\pgfqpoint{2.382583in}{0.730462in}}%
\pgfpathcurveto{\pgfqpoint{2.388407in}{0.736286in}}{\pgfqpoint{2.391679in}{0.744186in}}{\pgfqpoint{2.391679in}{0.752423in}}%
\pgfpathcurveto{\pgfqpoint{2.391679in}{0.760659in}}{\pgfqpoint{2.388407in}{0.768559in}}{\pgfqpoint{2.382583in}{0.774383in}}%
\pgfpathcurveto{\pgfqpoint{2.376759in}{0.780207in}}{\pgfqpoint{2.368859in}{0.783479in}}{\pgfqpoint{2.360623in}{0.783479in}}%
\pgfpathcurveto{\pgfqpoint{2.352387in}{0.783479in}}{\pgfqpoint{2.344487in}{0.780207in}}{\pgfqpoint{2.338663in}{0.774383in}}%
\pgfpathcurveto{\pgfqpoint{2.332839in}{0.768559in}}{\pgfqpoint{2.329566in}{0.760659in}}{\pgfqpoint{2.329566in}{0.752423in}}%
\pgfpathcurveto{\pgfqpoint{2.329566in}{0.744186in}}{\pgfqpoint{2.332839in}{0.736286in}}{\pgfqpoint{2.338663in}{0.730462in}}%
\pgfpathcurveto{\pgfqpoint{2.344487in}{0.724638in}}{\pgfqpoint{2.352387in}{0.721366in}}{\pgfqpoint{2.360623in}{0.721366in}}%
\pgfpathclose%
\pgfusepath{stroke,fill}%
\end{pgfscope}%
\begin{pgfscope}%
\pgfpathrectangle{\pgfqpoint{0.100000in}{0.220728in}}{\pgfqpoint{3.696000in}{3.696000in}}%
\pgfusepath{clip}%
\pgfsetbuttcap%
\pgfsetroundjoin%
\definecolor{currentfill}{rgb}{0.121569,0.466667,0.705882}%
\pgfsetfillcolor{currentfill}%
\pgfsetfillopacity{0.997897}%
\pgfsetlinewidth{1.003750pt}%
\definecolor{currentstroke}{rgb}{0.121569,0.466667,0.705882}%
\pgfsetstrokecolor{currentstroke}%
\pgfsetstrokeopacity{0.997897}%
\pgfsetdash{}{0pt}%
\pgfpathmoveto{\pgfqpoint{2.389768in}{0.731700in}}%
\pgfpathcurveto{\pgfqpoint{2.398005in}{0.731700in}}{\pgfqpoint{2.405905in}{0.734973in}}{\pgfqpoint{2.411729in}{0.740797in}}%
\pgfpathcurveto{\pgfqpoint{2.417552in}{0.746621in}}{\pgfqpoint{2.420825in}{0.754521in}}{\pgfqpoint{2.420825in}{0.762757in}}%
\pgfpathcurveto{\pgfqpoint{2.420825in}{0.770993in}}{\pgfqpoint{2.417552in}{0.778893in}}{\pgfqpoint{2.411729in}{0.784717in}}%
\pgfpathcurveto{\pgfqpoint{2.405905in}{0.790541in}}{\pgfqpoint{2.398005in}{0.793813in}}{\pgfqpoint{2.389768in}{0.793813in}}%
\pgfpathcurveto{\pgfqpoint{2.381532in}{0.793813in}}{\pgfqpoint{2.373632in}{0.790541in}}{\pgfqpoint{2.367808in}{0.784717in}}%
\pgfpathcurveto{\pgfqpoint{2.361984in}{0.778893in}}{\pgfqpoint{2.358712in}{0.770993in}}{\pgfqpoint{2.358712in}{0.762757in}}%
\pgfpathcurveto{\pgfqpoint{2.358712in}{0.754521in}}{\pgfqpoint{2.361984in}{0.746621in}}{\pgfqpoint{2.367808in}{0.740797in}}%
\pgfpathcurveto{\pgfqpoint{2.373632in}{0.734973in}}{\pgfqpoint{2.381532in}{0.731700in}}{\pgfqpoint{2.389768in}{0.731700in}}%
\pgfpathclose%
\pgfusepath{stroke,fill}%
\end{pgfscope}%
\begin{pgfscope}%
\pgfpathrectangle{\pgfqpoint{0.100000in}{0.220728in}}{\pgfqpoint{3.696000in}{3.696000in}}%
\pgfusepath{clip}%
\pgfsetbuttcap%
\pgfsetroundjoin%
\definecolor{currentfill}{rgb}{0.121569,0.466667,0.705882}%
\pgfsetfillcolor{currentfill}%
\pgfsetfillopacity{0.997929}%
\pgfsetlinewidth{1.003750pt}%
\definecolor{currentstroke}{rgb}{0.121569,0.466667,0.705882}%
\pgfsetstrokecolor{currentstroke}%
\pgfsetstrokeopacity{0.997929}%
\pgfsetdash{}{0pt}%
\pgfpathmoveto{\pgfqpoint{2.365174in}{0.718694in}}%
\pgfpathcurveto{\pgfqpoint{2.373410in}{0.718694in}}{\pgfqpoint{2.381310in}{0.721967in}}{\pgfqpoint{2.387134in}{0.727791in}}%
\pgfpathcurveto{\pgfqpoint{2.392958in}{0.733615in}}{\pgfqpoint{2.396231in}{0.741515in}}{\pgfqpoint{2.396231in}{0.749751in}}%
\pgfpathcurveto{\pgfqpoint{2.396231in}{0.757987in}}{\pgfqpoint{2.392958in}{0.765887in}}{\pgfqpoint{2.387134in}{0.771711in}}%
\pgfpathcurveto{\pgfqpoint{2.381310in}{0.777535in}}{\pgfqpoint{2.373410in}{0.780807in}}{\pgfqpoint{2.365174in}{0.780807in}}%
\pgfpathcurveto{\pgfqpoint{2.356938in}{0.780807in}}{\pgfqpoint{2.349038in}{0.777535in}}{\pgfqpoint{2.343214in}{0.771711in}}%
\pgfpathcurveto{\pgfqpoint{2.337390in}{0.765887in}}{\pgfqpoint{2.334118in}{0.757987in}}{\pgfqpoint{2.334118in}{0.749751in}}%
\pgfpathcurveto{\pgfqpoint{2.334118in}{0.741515in}}{\pgfqpoint{2.337390in}{0.733615in}}{\pgfqpoint{2.343214in}{0.727791in}}%
\pgfpathcurveto{\pgfqpoint{2.349038in}{0.721967in}}{\pgfqpoint{2.356938in}{0.718694in}}{\pgfqpoint{2.365174in}{0.718694in}}%
\pgfpathclose%
\pgfusepath{stroke,fill}%
\end{pgfscope}%
\begin{pgfscope}%
\pgfpathrectangle{\pgfqpoint{0.100000in}{0.220728in}}{\pgfqpoint{3.696000in}{3.696000in}}%
\pgfusepath{clip}%
\pgfsetbuttcap%
\pgfsetroundjoin%
\definecolor{currentfill}{rgb}{0.121569,0.466667,0.705882}%
\pgfsetfillcolor{currentfill}%
\pgfsetfillopacity{0.998410}%
\pgfsetlinewidth{1.003750pt}%
\definecolor{currentstroke}{rgb}{0.121569,0.466667,0.705882}%
\pgfsetstrokecolor{currentstroke}%
\pgfsetstrokeopacity{0.998410}%
\pgfsetdash{}{0pt}%
\pgfpathmoveto{\pgfqpoint{2.367557in}{0.717587in}}%
\pgfpathcurveto{\pgfqpoint{2.375793in}{0.717587in}}{\pgfqpoint{2.383693in}{0.720859in}}{\pgfqpoint{2.389517in}{0.726683in}}%
\pgfpathcurveto{\pgfqpoint{2.395341in}{0.732507in}}{\pgfqpoint{2.398613in}{0.740407in}}{\pgfqpoint{2.398613in}{0.748643in}}%
\pgfpathcurveto{\pgfqpoint{2.398613in}{0.756880in}}{\pgfqpoint{2.395341in}{0.764780in}}{\pgfqpoint{2.389517in}{0.770604in}}%
\pgfpathcurveto{\pgfqpoint{2.383693in}{0.776428in}}{\pgfqpoint{2.375793in}{0.779700in}}{\pgfqpoint{2.367557in}{0.779700in}}%
\pgfpathcurveto{\pgfqpoint{2.359321in}{0.779700in}}{\pgfqpoint{2.351421in}{0.776428in}}{\pgfqpoint{2.345597in}{0.770604in}}%
\pgfpathcurveto{\pgfqpoint{2.339773in}{0.764780in}}{\pgfqpoint{2.336500in}{0.756880in}}{\pgfqpoint{2.336500in}{0.748643in}}%
\pgfpathcurveto{\pgfqpoint{2.336500in}{0.740407in}}{\pgfqpoint{2.339773in}{0.732507in}}{\pgfqpoint{2.345597in}{0.726683in}}%
\pgfpathcurveto{\pgfqpoint{2.351421in}{0.720859in}}{\pgfqpoint{2.359321in}{0.717587in}}{\pgfqpoint{2.367557in}{0.717587in}}%
\pgfpathclose%
\pgfusepath{stroke,fill}%
\end{pgfscope}%
\begin{pgfscope}%
\pgfpathrectangle{\pgfqpoint{0.100000in}{0.220728in}}{\pgfqpoint{3.696000in}{3.696000in}}%
\pgfusepath{clip}%
\pgfsetbuttcap%
\pgfsetroundjoin%
\definecolor{currentfill}{rgb}{0.121569,0.466667,0.705882}%
\pgfsetfillcolor{currentfill}%
\pgfsetfillopacity{0.998423}%
\pgfsetlinewidth{1.003750pt}%
\definecolor{currentstroke}{rgb}{0.121569,0.466667,0.705882}%
\pgfsetstrokecolor{currentstroke}%
\pgfsetstrokeopacity{0.998423}%
\pgfsetdash{}{0pt}%
\pgfpathmoveto{\pgfqpoint{2.367615in}{0.717553in}}%
\pgfpathcurveto{\pgfqpoint{2.375851in}{0.717553in}}{\pgfqpoint{2.383751in}{0.720825in}}{\pgfqpoint{2.389575in}{0.726649in}}%
\pgfpathcurveto{\pgfqpoint{2.395399in}{0.732473in}}{\pgfqpoint{2.398671in}{0.740373in}}{\pgfqpoint{2.398671in}{0.748609in}}%
\pgfpathcurveto{\pgfqpoint{2.398671in}{0.756846in}}{\pgfqpoint{2.395399in}{0.764746in}}{\pgfqpoint{2.389575in}{0.770570in}}%
\pgfpathcurveto{\pgfqpoint{2.383751in}{0.776394in}}{\pgfqpoint{2.375851in}{0.779666in}}{\pgfqpoint{2.367615in}{0.779666in}}%
\pgfpathcurveto{\pgfqpoint{2.359378in}{0.779666in}}{\pgfqpoint{2.351478in}{0.776394in}}{\pgfqpoint{2.345654in}{0.770570in}}%
\pgfpathcurveto{\pgfqpoint{2.339831in}{0.764746in}}{\pgfqpoint{2.336558in}{0.756846in}}{\pgfqpoint{2.336558in}{0.748609in}}%
\pgfpathcurveto{\pgfqpoint{2.336558in}{0.740373in}}{\pgfqpoint{2.339831in}{0.732473in}}{\pgfqpoint{2.345654in}{0.726649in}}%
\pgfpathcurveto{\pgfqpoint{2.351478in}{0.720825in}}{\pgfqpoint{2.359378in}{0.717553in}}{\pgfqpoint{2.367615in}{0.717553in}}%
\pgfpathclose%
\pgfusepath{stroke,fill}%
\end{pgfscope}%
\begin{pgfscope}%
\pgfpathrectangle{\pgfqpoint{0.100000in}{0.220728in}}{\pgfqpoint{3.696000in}{3.696000in}}%
\pgfusepath{clip}%
\pgfsetbuttcap%
\pgfsetroundjoin%
\definecolor{currentfill}{rgb}{0.121569,0.466667,0.705882}%
\pgfsetfillcolor{currentfill}%
\pgfsetfillopacity{0.998444}%
\pgfsetlinewidth{1.003750pt}%
\definecolor{currentstroke}{rgb}{0.121569,0.466667,0.705882}%
\pgfsetstrokecolor{currentstroke}%
\pgfsetstrokeopacity{0.998444}%
\pgfsetdash{}{0pt}%
\pgfpathmoveto{\pgfqpoint{2.367718in}{0.717484in}}%
\pgfpathcurveto{\pgfqpoint{2.375954in}{0.717484in}}{\pgfqpoint{2.383855in}{0.720756in}}{\pgfqpoint{2.389678in}{0.726580in}}%
\pgfpathcurveto{\pgfqpoint{2.395502in}{0.732404in}}{\pgfqpoint{2.398775in}{0.740304in}}{\pgfqpoint{2.398775in}{0.748540in}}%
\pgfpathcurveto{\pgfqpoint{2.398775in}{0.756777in}}{\pgfqpoint{2.395502in}{0.764677in}}{\pgfqpoint{2.389678in}{0.770501in}}%
\pgfpathcurveto{\pgfqpoint{2.383855in}{0.776324in}}{\pgfqpoint{2.375954in}{0.779597in}}{\pgfqpoint{2.367718in}{0.779597in}}%
\pgfpathcurveto{\pgfqpoint{2.359482in}{0.779597in}}{\pgfqpoint{2.351582in}{0.776324in}}{\pgfqpoint{2.345758in}{0.770501in}}%
\pgfpathcurveto{\pgfqpoint{2.339934in}{0.764677in}}{\pgfqpoint{2.336662in}{0.756777in}}{\pgfqpoint{2.336662in}{0.748540in}}%
\pgfpathcurveto{\pgfqpoint{2.336662in}{0.740304in}}{\pgfqpoint{2.339934in}{0.732404in}}{\pgfqpoint{2.345758in}{0.726580in}}%
\pgfpathcurveto{\pgfqpoint{2.351582in}{0.720756in}}{\pgfqpoint{2.359482in}{0.717484in}}{\pgfqpoint{2.367718in}{0.717484in}}%
\pgfpathclose%
\pgfusepath{stroke,fill}%
\end{pgfscope}%
\begin{pgfscope}%
\pgfpathrectangle{\pgfqpoint{0.100000in}{0.220728in}}{\pgfqpoint{3.696000in}{3.696000in}}%
\pgfusepath{clip}%
\pgfsetbuttcap%
\pgfsetroundjoin%
\definecolor{currentfill}{rgb}{0.121569,0.466667,0.705882}%
\pgfsetfillcolor{currentfill}%
\pgfsetfillopacity{0.998484}%
\pgfsetlinewidth{1.003750pt}%
\definecolor{currentstroke}{rgb}{0.121569,0.466667,0.705882}%
\pgfsetstrokecolor{currentstroke}%
\pgfsetstrokeopacity{0.998484}%
\pgfsetdash{}{0pt}%
\pgfpathmoveto{\pgfqpoint{2.367911in}{0.717374in}}%
\pgfpathcurveto{\pgfqpoint{2.376147in}{0.717374in}}{\pgfqpoint{2.384047in}{0.720646in}}{\pgfqpoint{2.389871in}{0.726470in}}%
\pgfpathcurveto{\pgfqpoint{2.395695in}{0.732294in}}{\pgfqpoint{2.398967in}{0.740194in}}{\pgfqpoint{2.398967in}{0.748431in}}%
\pgfpathcurveto{\pgfqpoint{2.398967in}{0.756667in}}{\pgfqpoint{2.395695in}{0.764567in}}{\pgfqpoint{2.389871in}{0.770391in}}%
\pgfpathcurveto{\pgfqpoint{2.384047in}{0.776215in}}{\pgfqpoint{2.376147in}{0.779487in}}{\pgfqpoint{2.367911in}{0.779487in}}%
\pgfpathcurveto{\pgfqpoint{2.359674in}{0.779487in}}{\pgfqpoint{2.351774in}{0.776215in}}{\pgfqpoint{2.345950in}{0.770391in}}%
\pgfpathcurveto{\pgfqpoint{2.340126in}{0.764567in}}{\pgfqpoint{2.336854in}{0.756667in}}{\pgfqpoint{2.336854in}{0.748431in}}%
\pgfpathcurveto{\pgfqpoint{2.336854in}{0.740194in}}{\pgfqpoint{2.340126in}{0.732294in}}{\pgfqpoint{2.345950in}{0.726470in}}%
\pgfpathcurveto{\pgfqpoint{2.351774in}{0.720646in}}{\pgfqpoint{2.359674in}{0.717374in}}{\pgfqpoint{2.367911in}{0.717374in}}%
\pgfpathclose%
\pgfusepath{stroke,fill}%
\end{pgfscope}%
\begin{pgfscope}%
\pgfpathrectangle{\pgfqpoint{0.100000in}{0.220728in}}{\pgfqpoint{3.696000in}{3.696000in}}%
\pgfusepath{clip}%
\pgfsetbuttcap%
\pgfsetroundjoin%
\definecolor{currentfill}{rgb}{0.121569,0.466667,0.705882}%
\pgfsetfillcolor{currentfill}%
\pgfsetfillopacity{0.998590}%
\pgfsetlinewidth{1.003750pt}%
\definecolor{currentstroke}{rgb}{0.121569,0.466667,0.705882}%
\pgfsetstrokecolor{currentstroke}%
\pgfsetstrokeopacity{0.998590}%
\pgfsetdash{}{0pt}%
\pgfpathmoveto{\pgfqpoint{2.368229in}{0.717212in}}%
\pgfpathcurveto{\pgfqpoint{2.376466in}{0.717212in}}{\pgfqpoint{2.384366in}{0.720484in}}{\pgfqpoint{2.390190in}{0.726308in}}%
\pgfpathcurveto{\pgfqpoint{2.396014in}{0.732132in}}{\pgfqpoint{2.399286in}{0.740032in}}{\pgfqpoint{2.399286in}{0.748269in}}%
\pgfpathcurveto{\pgfqpoint{2.399286in}{0.756505in}}{\pgfqpoint{2.396014in}{0.764405in}}{\pgfqpoint{2.390190in}{0.770229in}}%
\pgfpathcurveto{\pgfqpoint{2.384366in}{0.776053in}}{\pgfqpoint{2.376466in}{0.779325in}}{\pgfqpoint{2.368229in}{0.779325in}}%
\pgfpathcurveto{\pgfqpoint{2.359993in}{0.779325in}}{\pgfqpoint{2.352093in}{0.776053in}}{\pgfqpoint{2.346269in}{0.770229in}}%
\pgfpathcurveto{\pgfqpoint{2.340445in}{0.764405in}}{\pgfqpoint{2.337173in}{0.756505in}}{\pgfqpoint{2.337173in}{0.748269in}}%
\pgfpathcurveto{\pgfqpoint{2.337173in}{0.740032in}}{\pgfqpoint{2.340445in}{0.732132in}}{\pgfqpoint{2.346269in}{0.726308in}}%
\pgfpathcurveto{\pgfqpoint{2.352093in}{0.720484in}}{\pgfqpoint{2.359993in}{0.717212in}}{\pgfqpoint{2.368229in}{0.717212in}}%
\pgfpathclose%
\pgfusepath{stroke,fill}%
\end{pgfscope}%
\begin{pgfscope}%
\pgfpathrectangle{\pgfqpoint{0.100000in}{0.220728in}}{\pgfqpoint{3.696000in}{3.696000in}}%
\pgfusepath{clip}%
\pgfsetbuttcap%
\pgfsetroundjoin%
\definecolor{currentfill}{rgb}{0.121569,0.466667,0.705882}%
\pgfsetfillcolor{currentfill}%
\pgfsetfillopacity{0.998696}%
\pgfsetlinewidth{1.003750pt}%
\definecolor{currentstroke}{rgb}{0.121569,0.466667,0.705882}%
\pgfsetstrokecolor{currentstroke}%
\pgfsetstrokeopacity{0.998696}%
\pgfsetdash{}{0pt}%
\pgfpathmoveto{\pgfqpoint{2.368880in}{0.716801in}}%
\pgfpathcurveto{\pgfqpoint{2.377117in}{0.716801in}}{\pgfqpoint{2.385017in}{0.720073in}}{\pgfqpoint{2.390841in}{0.725897in}}%
\pgfpathcurveto{\pgfqpoint{2.396665in}{0.731721in}}{\pgfqpoint{2.399937in}{0.739621in}}{\pgfqpoint{2.399937in}{0.747857in}}%
\pgfpathcurveto{\pgfqpoint{2.399937in}{0.756094in}}{\pgfqpoint{2.396665in}{0.763994in}}{\pgfqpoint{2.390841in}{0.769818in}}%
\pgfpathcurveto{\pgfqpoint{2.385017in}{0.775641in}}{\pgfqpoint{2.377117in}{0.778914in}}{\pgfqpoint{2.368880in}{0.778914in}}%
\pgfpathcurveto{\pgfqpoint{2.360644in}{0.778914in}}{\pgfqpoint{2.352744in}{0.775641in}}{\pgfqpoint{2.346920in}{0.769818in}}%
\pgfpathcurveto{\pgfqpoint{2.341096in}{0.763994in}}{\pgfqpoint{2.337824in}{0.756094in}}{\pgfqpoint{2.337824in}{0.747857in}}%
\pgfpathcurveto{\pgfqpoint{2.337824in}{0.739621in}}{\pgfqpoint{2.341096in}{0.731721in}}{\pgfqpoint{2.346920in}{0.725897in}}%
\pgfpathcurveto{\pgfqpoint{2.352744in}{0.720073in}}{\pgfqpoint{2.360644in}{0.716801in}}{\pgfqpoint{2.368880in}{0.716801in}}%
\pgfpathclose%
\pgfusepath{stroke,fill}%
\end{pgfscope}%
\begin{pgfscope}%
\pgfpathrectangle{\pgfqpoint{0.100000in}{0.220728in}}{\pgfqpoint{3.696000in}{3.696000in}}%
\pgfusepath{clip}%
\pgfsetbuttcap%
\pgfsetroundjoin%
\definecolor{currentfill}{rgb}{0.121569,0.466667,0.705882}%
\pgfsetfillcolor{currentfill}%
\pgfsetfillopacity{0.998873}%
\pgfsetlinewidth{1.003750pt}%
\definecolor{currentstroke}{rgb}{0.121569,0.466667,0.705882}%
\pgfsetstrokecolor{currentstroke}%
\pgfsetstrokeopacity{0.998873}%
\pgfsetdash{}{0pt}%
\pgfpathmoveto{\pgfqpoint{2.387495in}{0.725646in}}%
\pgfpathcurveto{\pgfqpoint{2.395731in}{0.725646in}}{\pgfqpoint{2.403631in}{0.728918in}}{\pgfqpoint{2.409455in}{0.734742in}}%
\pgfpathcurveto{\pgfqpoint{2.415279in}{0.740566in}}{\pgfqpoint{2.418552in}{0.748466in}}{\pgfqpoint{2.418552in}{0.756702in}}%
\pgfpathcurveto{\pgfqpoint{2.418552in}{0.764939in}}{\pgfqpoint{2.415279in}{0.772839in}}{\pgfqpoint{2.409455in}{0.778663in}}%
\pgfpathcurveto{\pgfqpoint{2.403631in}{0.784487in}}{\pgfqpoint{2.395731in}{0.787759in}}{\pgfqpoint{2.387495in}{0.787759in}}%
\pgfpathcurveto{\pgfqpoint{2.379259in}{0.787759in}}{\pgfqpoint{2.371359in}{0.784487in}}{\pgfqpoint{2.365535in}{0.778663in}}%
\pgfpathcurveto{\pgfqpoint{2.359711in}{0.772839in}}{\pgfqpoint{2.356439in}{0.764939in}}{\pgfqpoint{2.356439in}{0.756702in}}%
\pgfpathcurveto{\pgfqpoint{2.356439in}{0.748466in}}{\pgfqpoint{2.359711in}{0.740566in}}{\pgfqpoint{2.365535in}{0.734742in}}%
\pgfpathcurveto{\pgfqpoint{2.371359in}{0.728918in}}{\pgfqpoint{2.379259in}{0.725646in}}{\pgfqpoint{2.387495in}{0.725646in}}%
\pgfpathclose%
\pgfusepath{stroke,fill}%
\end{pgfscope}%
\begin{pgfscope}%
\pgfpathrectangle{\pgfqpoint{0.100000in}{0.220728in}}{\pgfqpoint{3.696000in}{3.696000in}}%
\pgfusepath{clip}%
\pgfsetbuttcap%
\pgfsetroundjoin%
\definecolor{currentfill}{rgb}{0.121569,0.466667,0.705882}%
\pgfsetfillcolor{currentfill}%
\pgfsetfillopacity{0.998875}%
\pgfsetlinewidth{1.003750pt}%
\definecolor{currentstroke}{rgb}{0.121569,0.466667,0.705882}%
\pgfsetstrokecolor{currentstroke}%
\pgfsetstrokeopacity{0.998875}%
\pgfsetdash{}{0pt}%
\pgfpathmoveto{\pgfqpoint{2.370102in}{0.716128in}}%
\pgfpathcurveto{\pgfqpoint{2.378338in}{0.716128in}}{\pgfqpoint{2.386238in}{0.719400in}}{\pgfqpoint{2.392062in}{0.725224in}}%
\pgfpathcurveto{\pgfqpoint{2.397886in}{0.731048in}}{\pgfqpoint{2.401159in}{0.738948in}}{\pgfqpoint{2.401159in}{0.747184in}}%
\pgfpathcurveto{\pgfqpoint{2.401159in}{0.755420in}}{\pgfqpoint{2.397886in}{0.763320in}}{\pgfqpoint{2.392062in}{0.769144in}}%
\pgfpathcurveto{\pgfqpoint{2.386238in}{0.774968in}}{\pgfqpoint{2.378338in}{0.778241in}}{\pgfqpoint{2.370102in}{0.778241in}}%
\pgfpathcurveto{\pgfqpoint{2.361866in}{0.778241in}}{\pgfqpoint{2.353966in}{0.774968in}}{\pgfqpoint{2.348142in}{0.769144in}}%
\pgfpathcurveto{\pgfqpoint{2.342318in}{0.763320in}}{\pgfqpoint{2.339046in}{0.755420in}}{\pgfqpoint{2.339046in}{0.747184in}}%
\pgfpathcurveto{\pgfqpoint{2.339046in}{0.738948in}}{\pgfqpoint{2.342318in}{0.731048in}}{\pgfqpoint{2.348142in}{0.725224in}}%
\pgfpathcurveto{\pgfqpoint{2.353966in}{0.719400in}}{\pgfqpoint{2.361866in}{0.716128in}}{\pgfqpoint{2.370102in}{0.716128in}}%
\pgfpathclose%
\pgfusepath{stroke,fill}%
\end{pgfscope}%
\begin{pgfscope}%
\pgfpathrectangle{\pgfqpoint{0.100000in}{0.220728in}}{\pgfqpoint{3.696000in}{3.696000in}}%
\pgfusepath{clip}%
\pgfsetbuttcap%
\pgfsetroundjoin%
\definecolor{currentfill}{rgb}{0.121569,0.466667,0.705882}%
\pgfsetfillcolor{currentfill}%
\pgfsetfillopacity{0.999123}%
\pgfsetlinewidth{1.003750pt}%
\definecolor{currentstroke}{rgb}{0.121569,0.466667,0.705882}%
\pgfsetstrokecolor{currentstroke}%
\pgfsetstrokeopacity{0.999123}%
\pgfsetdash{}{0pt}%
\pgfpathmoveto{\pgfqpoint{2.372413in}{0.714970in}}%
\pgfpathcurveto{\pgfqpoint{2.380649in}{0.714970in}}{\pgfqpoint{2.388549in}{0.718243in}}{\pgfqpoint{2.394373in}{0.724067in}}%
\pgfpathcurveto{\pgfqpoint{2.400197in}{0.729891in}}{\pgfqpoint{2.403469in}{0.737791in}}{\pgfqpoint{2.403469in}{0.746027in}}%
\pgfpathcurveto{\pgfqpoint{2.403469in}{0.754263in}}{\pgfqpoint{2.400197in}{0.762163in}}{\pgfqpoint{2.394373in}{0.767987in}}%
\pgfpathcurveto{\pgfqpoint{2.388549in}{0.773811in}}{\pgfqpoint{2.380649in}{0.777083in}}{\pgfqpoint{2.372413in}{0.777083in}}%
\pgfpathcurveto{\pgfqpoint{2.364176in}{0.777083in}}{\pgfqpoint{2.356276in}{0.773811in}}{\pgfqpoint{2.350452in}{0.767987in}}%
\pgfpathcurveto{\pgfqpoint{2.344628in}{0.762163in}}{\pgfqpoint{2.341356in}{0.754263in}}{\pgfqpoint{2.341356in}{0.746027in}}%
\pgfpathcurveto{\pgfqpoint{2.341356in}{0.737791in}}{\pgfqpoint{2.344628in}{0.729891in}}{\pgfqpoint{2.350452in}{0.724067in}}%
\pgfpathcurveto{\pgfqpoint{2.356276in}{0.718243in}}{\pgfqpoint{2.364176in}{0.714970in}}{\pgfqpoint{2.372413in}{0.714970in}}%
\pgfpathclose%
\pgfusepath{stroke,fill}%
\end{pgfscope}%
\begin{pgfscope}%
\pgfpathrectangle{\pgfqpoint{0.100000in}{0.220728in}}{\pgfqpoint{3.696000in}{3.696000in}}%
\pgfusepath{clip}%
\pgfsetbuttcap%
\pgfsetroundjoin%
\definecolor{currentfill}{rgb}{0.121569,0.466667,0.705882}%
\pgfsetfillcolor{currentfill}%
\pgfsetfillopacity{0.999348}%
\pgfsetlinewidth{1.003750pt}%
\definecolor{currentstroke}{rgb}{0.121569,0.466667,0.705882}%
\pgfsetstrokecolor{currentstroke}%
\pgfsetstrokeopacity{0.999348}%
\pgfsetdash{}{0pt}%
\pgfpathmoveto{\pgfqpoint{2.386127in}{0.722195in}}%
\pgfpathcurveto{\pgfqpoint{2.394363in}{0.722195in}}{\pgfqpoint{2.402263in}{0.725467in}}{\pgfqpoint{2.408087in}{0.731291in}}%
\pgfpathcurveto{\pgfqpoint{2.413911in}{0.737115in}}{\pgfqpoint{2.417183in}{0.745015in}}{\pgfqpoint{2.417183in}{0.753251in}}%
\pgfpathcurveto{\pgfqpoint{2.417183in}{0.761487in}}{\pgfqpoint{2.413911in}{0.769387in}}{\pgfqpoint{2.408087in}{0.775211in}}%
\pgfpathcurveto{\pgfqpoint{2.402263in}{0.781035in}}{\pgfqpoint{2.394363in}{0.784308in}}{\pgfqpoint{2.386127in}{0.784308in}}%
\pgfpathcurveto{\pgfqpoint{2.377890in}{0.784308in}}{\pgfqpoint{2.369990in}{0.781035in}}{\pgfqpoint{2.364166in}{0.775211in}}%
\pgfpathcurveto{\pgfqpoint{2.358342in}{0.769387in}}{\pgfqpoint{2.355070in}{0.761487in}}{\pgfqpoint{2.355070in}{0.753251in}}%
\pgfpathcurveto{\pgfqpoint{2.355070in}{0.745015in}}{\pgfqpoint{2.358342in}{0.737115in}}{\pgfqpoint{2.364166in}{0.731291in}}%
\pgfpathcurveto{\pgfqpoint{2.369990in}{0.725467in}}{\pgfqpoint{2.377890in}{0.722195in}}{\pgfqpoint{2.386127in}{0.722195in}}%
\pgfpathclose%
\pgfusepath{stroke,fill}%
\end{pgfscope}%
\begin{pgfscope}%
\pgfpathrectangle{\pgfqpoint{0.100000in}{0.220728in}}{\pgfqpoint{3.696000in}{3.696000in}}%
\pgfusepath{clip}%
\pgfsetbuttcap%
\pgfsetroundjoin%
\definecolor{currentfill}{rgb}{0.121569,0.466667,0.705882}%
\pgfsetfillcolor{currentfill}%
\pgfsetfillopacity{0.999380}%
\pgfsetlinewidth{1.003750pt}%
\definecolor{currentstroke}{rgb}{0.121569,0.466667,0.705882}%
\pgfsetstrokecolor{currentstroke}%
\pgfsetstrokeopacity{0.999380}%
\pgfsetdash{}{0pt}%
\pgfpathmoveto{\pgfqpoint{2.374275in}{0.714569in}}%
\pgfpathcurveto{\pgfqpoint{2.382511in}{0.714569in}}{\pgfqpoint{2.390411in}{0.717841in}}{\pgfqpoint{2.396235in}{0.723665in}}%
\pgfpathcurveto{\pgfqpoint{2.402059in}{0.729489in}}{\pgfqpoint{2.405331in}{0.737389in}}{\pgfqpoint{2.405331in}{0.745625in}}%
\pgfpathcurveto{\pgfqpoint{2.405331in}{0.753861in}}{\pgfqpoint{2.402059in}{0.761761in}}{\pgfqpoint{2.396235in}{0.767585in}}%
\pgfpathcurveto{\pgfqpoint{2.390411in}{0.773409in}}{\pgfqpoint{2.382511in}{0.776682in}}{\pgfqpoint{2.374275in}{0.776682in}}%
\pgfpathcurveto{\pgfqpoint{2.366039in}{0.776682in}}{\pgfqpoint{2.358138in}{0.773409in}}{\pgfqpoint{2.352315in}{0.767585in}}%
\pgfpathcurveto{\pgfqpoint{2.346491in}{0.761761in}}{\pgfqpoint{2.343218in}{0.753861in}}{\pgfqpoint{2.343218in}{0.745625in}}%
\pgfpathcurveto{\pgfqpoint{2.343218in}{0.737389in}}{\pgfqpoint{2.346491in}{0.729489in}}{\pgfqpoint{2.352315in}{0.723665in}}%
\pgfpathcurveto{\pgfqpoint{2.358138in}{0.717841in}}{\pgfqpoint{2.366039in}{0.714569in}}{\pgfqpoint{2.374275in}{0.714569in}}%
\pgfpathclose%
\pgfusepath{stroke,fill}%
\end{pgfscope}%
\begin{pgfscope}%
\pgfpathrectangle{\pgfqpoint{0.100000in}{0.220728in}}{\pgfqpoint{3.696000in}{3.696000in}}%
\pgfusepath{clip}%
\pgfsetbuttcap%
\pgfsetroundjoin%
\definecolor{currentfill}{rgb}{0.121569,0.466667,0.705882}%
\pgfsetfillcolor{currentfill}%
\pgfsetfillopacity{0.999609}%
\pgfsetlinewidth{1.003750pt}%
\definecolor{currentstroke}{rgb}{0.121569,0.466667,0.705882}%
\pgfsetstrokecolor{currentstroke}%
\pgfsetstrokeopacity{0.999609}%
\pgfsetdash{}{0pt}%
\pgfpathmoveto{\pgfqpoint{2.385425in}{0.720243in}}%
\pgfpathcurveto{\pgfqpoint{2.393661in}{0.720243in}}{\pgfqpoint{2.401561in}{0.723515in}}{\pgfqpoint{2.407385in}{0.729339in}}%
\pgfpathcurveto{\pgfqpoint{2.413209in}{0.735163in}}{\pgfqpoint{2.416481in}{0.743063in}}{\pgfqpoint{2.416481in}{0.751299in}}%
\pgfpathcurveto{\pgfqpoint{2.416481in}{0.759535in}}{\pgfqpoint{2.413209in}{0.767435in}}{\pgfqpoint{2.407385in}{0.773259in}}%
\pgfpathcurveto{\pgfqpoint{2.401561in}{0.779083in}}{\pgfqpoint{2.393661in}{0.782356in}}{\pgfqpoint{2.385425in}{0.782356in}}%
\pgfpathcurveto{\pgfqpoint{2.377189in}{0.782356in}}{\pgfqpoint{2.369289in}{0.779083in}}{\pgfqpoint{2.363465in}{0.773259in}}%
\pgfpathcurveto{\pgfqpoint{2.357641in}{0.767435in}}{\pgfqpoint{2.354368in}{0.759535in}}{\pgfqpoint{2.354368in}{0.751299in}}%
\pgfpathcurveto{\pgfqpoint{2.354368in}{0.743063in}}{\pgfqpoint{2.357641in}{0.735163in}}{\pgfqpoint{2.363465in}{0.729339in}}%
\pgfpathcurveto{\pgfqpoint{2.369289in}{0.723515in}}{\pgfqpoint{2.377189in}{0.720243in}}{\pgfqpoint{2.385425in}{0.720243in}}%
\pgfpathclose%
\pgfusepath{stroke,fill}%
\end{pgfscope}%
\begin{pgfscope}%
\pgfpathrectangle{\pgfqpoint{0.100000in}{0.220728in}}{\pgfqpoint{3.696000in}{3.696000in}}%
\pgfusepath{clip}%
\pgfsetbuttcap%
\pgfsetroundjoin%
\definecolor{currentfill}{rgb}{0.121569,0.466667,0.705882}%
\pgfsetfillcolor{currentfill}%
\pgfsetfillopacity{0.999755}%
\pgfsetlinewidth{1.003750pt}%
\definecolor{currentstroke}{rgb}{0.121569,0.466667,0.705882}%
\pgfsetstrokecolor{currentstroke}%
\pgfsetstrokeopacity{0.999755}%
\pgfsetdash{}{0pt}%
\pgfpathmoveto{\pgfqpoint{2.385007in}{0.719213in}}%
\pgfpathcurveto{\pgfqpoint{2.393243in}{0.719213in}}{\pgfqpoint{2.401143in}{0.722486in}}{\pgfqpoint{2.406967in}{0.728310in}}%
\pgfpathcurveto{\pgfqpoint{2.412791in}{0.734134in}}{\pgfqpoint{2.416063in}{0.742034in}}{\pgfqpoint{2.416063in}{0.750270in}}%
\pgfpathcurveto{\pgfqpoint{2.416063in}{0.758506in}}{\pgfqpoint{2.412791in}{0.766406in}}{\pgfqpoint{2.406967in}{0.772230in}}%
\pgfpathcurveto{\pgfqpoint{2.401143in}{0.778054in}}{\pgfqpoint{2.393243in}{0.781326in}}{\pgfqpoint{2.385007in}{0.781326in}}%
\pgfpathcurveto{\pgfqpoint{2.376770in}{0.781326in}}{\pgfqpoint{2.368870in}{0.778054in}}{\pgfqpoint{2.363046in}{0.772230in}}%
\pgfpathcurveto{\pgfqpoint{2.357223in}{0.766406in}}{\pgfqpoint{2.353950in}{0.758506in}}{\pgfqpoint{2.353950in}{0.750270in}}%
\pgfpathcurveto{\pgfqpoint{2.353950in}{0.742034in}}{\pgfqpoint{2.357223in}{0.734134in}}{\pgfqpoint{2.363046in}{0.728310in}}%
\pgfpathcurveto{\pgfqpoint{2.368870in}{0.722486in}}{\pgfqpoint{2.376770in}{0.719213in}}{\pgfqpoint{2.385007in}{0.719213in}}%
\pgfpathclose%
\pgfusepath{stroke,fill}%
\end{pgfscope}%
\begin{pgfscope}%
\pgfpathrectangle{\pgfqpoint{0.100000in}{0.220728in}}{\pgfqpoint{3.696000in}{3.696000in}}%
\pgfusepath{clip}%
\pgfsetbuttcap%
\pgfsetroundjoin%
\definecolor{currentfill}{rgb}{0.121569,0.466667,0.705882}%
\pgfsetfillcolor{currentfill}%
\pgfsetfillopacity{0.999836}%
\pgfsetlinewidth{1.003750pt}%
\definecolor{currentstroke}{rgb}{0.121569,0.466667,0.705882}%
\pgfsetstrokecolor{currentstroke}%
\pgfsetstrokeopacity{0.999836}%
\pgfsetdash{}{0pt}%
\pgfpathmoveto{\pgfqpoint{2.384779in}{0.718645in}}%
\pgfpathcurveto{\pgfqpoint{2.393015in}{0.718645in}}{\pgfqpoint{2.400915in}{0.721917in}}{\pgfqpoint{2.406739in}{0.727741in}}%
\pgfpathcurveto{\pgfqpoint{2.412563in}{0.733565in}}{\pgfqpoint{2.415835in}{0.741465in}}{\pgfqpoint{2.415835in}{0.749701in}}%
\pgfpathcurveto{\pgfqpoint{2.415835in}{0.757937in}}{\pgfqpoint{2.412563in}{0.765837in}}{\pgfqpoint{2.406739in}{0.771661in}}%
\pgfpathcurveto{\pgfqpoint{2.400915in}{0.777485in}}{\pgfqpoint{2.393015in}{0.780758in}}{\pgfqpoint{2.384779in}{0.780758in}}%
\pgfpathcurveto{\pgfqpoint{2.376542in}{0.780758in}}{\pgfqpoint{2.368642in}{0.777485in}}{\pgfqpoint{2.362818in}{0.771661in}}%
\pgfpathcurveto{\pgfqpoint{2.356994in}{0.765837in}}{\pgfqpoint{2.353722in}{0.757937in}}{\pgfqpoint{2.353722in}{0.749701in}}%
\pgfpathcurveto{\pgfqpoint{2.353722in}{0.741465in}}{\pgfqpoint{2.356994in}{0.733565in}}{\pgfqpoint{2.362818in}{0.727741in}}%
\pgfpathcurveto{\pgfqpoint{2.368642in}{0.721917in}}{\pgfqpoint{2.376542in}{0.718645in}}{\pgfqpoint{2.384779in}{0.718645in}}%
\pgfpathclose%
\pgfusepath{stroke,fill}%
\end{pgfscope}%
\begin{pgfscope}%
\pgfpathrectangle{\pgfqpoint{0.100000in}{0.220728in}}{\pgfqpoint{3.696000in}{3.696000in}}%
\pgfusepath{clip}%
\pgfsetbuttcap%
\pgfsetroundjoin%
\definecolor{currentfill}{rgb}{0.121569,0.466667,0.705882}%
\pgfsetfillcolor{currentfill}%
\pgfsetfillopacity{0.999880}%
\pgfsetlinewidth{1.003750pt}%
\definecolor{currentstroke}{rgb}{0.121569,0.466667,0.705882}%
\pgfsetstrokecolor{currentstroke}%
\pgfsetstrokeopacity{0.999880}%
\pgfsetdash{}{0pt}%
\pgfpathmoveto{\pgfqpoint{2.384649in}{0.718337in}}%
\pgfpathcurveto{\pgfqpoint{2.392885in}{0.718337in}}{\pgfqpoint{2.400785in}{0.721610in}}{\pgfqpoint{2.406609in}{0.727434in}}%
\pgfpathcurveto{\pgfqpoint{2.412433in}{0.733258in}}{\pgfqpoint{2.415705in}{0.741158in}}{\pgfqpoint{2.415705in}{0.749394in}}%
\pgfpathcurveto{\pgfqpoint{2.415705in}{0.757630in}}{\pgfqpoint{2.412433in}{0.765530in}}{\pgfqpoint{2.406609in}{0.771354in}}%
\pgfpathcurveto{\pgfqpoint{2.400785in}{0.777178in}}{\pgfqpoint{2.392885in}{0.780450in}}{\pgfqpoint{2.384649in}{0.780450in}}%
\pgfpathcurveto{\pgfqpoint{2.376412in}{0.780450in}}{\pgfqpoint{2.368512in}{0.777178in}}{\pgfqpoint{2.362688in}{0.771354in}}%
\pgfpathcurveto{\pgfqpoint{2.356865in}{0.765530in}}{\pgfqpoint{2.353592in}{0.757630in}}{\pgfqpoint{2.353592in}{0.749394in}}%
\pgfpathcurveto{\pgfqpoint{2.353592in}{0.741158in}}{\pgfqpoint{2.356865in}{0.733258in}}{\pgfqpoint{2.362688in}{0.727434in}}%
\pgfpathcurveto{\pgfqpoint{2.368512in}{0.721610in}}{\pgfqpoint{2.376412in}{0.718337in}}{\pgfqpoint{2.384649in}{0.718337in}}%
\pgfpathclose%
\pgfusepath{stroke,fill}%
\end{pgfscope}%
\begin{pgfscope}%
\pgfpathrectangle{\pgfqpoint{0.100000in}{0.220728in}}{\pgfqpoint{3.696000in}{3.696000in}}%
\pgfusepath{clip}%
\pgfsetbuttcap%
\pgfsetroundjoin%
\definecolor{currentfill}{rgb}{0.121569,0.466667,0.705882}%
\pgfsetfillcolor{currentfill}%
\pgfsetfillopacity{0.999893}%
\pgfsetlinewidth{1.003750pt}%
\definecolor{currentstroke}{rgb}{0.121569,0.466667,0.705882}%
\pgfsetstrokecolor{currentstroke}%
\pgfsetstrokeopacity{0.999893}%
\pgfsetdash{}{0pt}%
\pgfpathmoveto{\pgfqpoint{2.377688in}{0.714138in}}%
\pgfpathcurveto{\pgfqpoint{2.385924in}{0.714138in}}{\pgfqpoint{2.393824in}{0.717411in}}{\pgfqpoint{2.399648in}{0.723235in}}%
\pgfpathcurveto{\pgfqpoint{2.405472in}{0.729059in}}{\pgfqpoint{2.408744in}{0.736959in}}{\pgfqpoint{2.408744in}{0.745195in}}%
\pgfpathcurveto{\pgfqpoint{2.408744in}{0.753431in}}{\pgfqpoint{2.405472in}{0.761331in}}{\pgfqpoint{2.399648in}{0.767155in}}%
\pgfpathcurveto{\pgfqpoint{2.393824in}{0.772979in}}{\pgfqpoint{2.385924in}{0.776251in}}{\pgfqpoint{2.377688in}{0.776251in}}%
\pgfpathcurveto{\pgfqpoint{2.369451in}{0.776251in}}{\pgfqpoint{2.361551in}{0.772979in}}{\pgfqpoint{2.355727in}{0.767155in}}%
\pgfpathcurveto{\pgfqpoint{2.349903in}{0.761331in}}{\pgfqpoint{2.346631in}{0.753431in}}{\pgfqpoint{2.346631in}{0.745195in}}%
\pgfpathcurveto{\pgfqpoint{2.346631in}{0.736959in}}{\pgfqpoint{2.349903in}{0.729059in}}{\pgfqpoint{2.355727in}{0.723235in}}%
\pgfpathcurveto{\pgfqpoint{2.361551in}{0.717411in}}{\pgfqpoint{2.369451in}{0.714138in}}{\pgfqpoint{2.377688in}{0.714138in}}%
\pgfpathclose%
\pgfusepath{stroke,fill}%
\end{pgfscope}%
\begin{pgfscope}%
\pgfpathrectangle{\pgfqpoint{0.100000in}{0.220728in}}{\pgfqpoint{3.696000in}{3.696000in}}%
\pgfusepath{clip}%
\pgfsetbuttcap%
\pgfsetroundjoin%
\definecolor{currentfill}{rgb}{0.121569,0.466667,0.705882}%
\pgfsetfillcolor{currentfill}%
\pgfsetfillopacity{0.999906}%
\pgfsetlinewidth{1.003750pt}%
\definecolor{currentstroke}{rgb}{0.121569,0.466667,0.705882}%
\pgfsetstrokecolor{currentstroke}%
\pgfsetstrokeopacity{0.999906}%
\pgfsetdash{}{0pt}%
\pgfpathmoveto{\pgfqpoint{2.384580in}{0.718171in}}%
\pgfpathcurveto{\pgfqpoint{2.392816in}{0.718171in}}{\pgfqpoint{2.400717in}{0.721443in}}{\pgfqpoint{2.406540in}{0.727267in}}%
\pgfpathcurveto{\pgfqpoint{2.412364in}{0.733091in}}{\pgfqpoint{2.415637in}{0.740991in}}{\pgfqpoint{2.415637in}{0.749227in}}%
\pgfpathcurveto{\pgfqpoint{2.415637in}{0.757463in}}{\pgfqpoint{2.412364in}{0.765363in}}{\pgfqpoint{2.406540in}{0.771187in}}%
\pgfpathcurveto{\pgfqpoint{2.400717in}{0.777011in}}{\pgfqpoint{2.392816in}{0.780284in}}{\pgfqpoint{2.384580in}{0.780284in}}%
\pgfpathcurveto{\pgfqpoint{2.376344in}{0.780284in}}{\pgfqpoint{2.368444in}{0.777011in}}{\pgfqpoint{2.362620in}{0.771187in}}%
\pgfpathcurveto{\pgfqpoint{2.356796in}{0.765363in}}{\pgfqpoint{2.353524in}{0.757463in}}{\pgfqpoint{2.353524in}{0.749227in}}%
\pgfpathcurveto{\pgfqpoint{2.353524in}{0.740991in}}{\pgfqpoint{2.356796in}{0.733091in}}{\pgfqpoint{2.362620in}{0.727267in}}%
\pgfpathcurveto{\pgfqpoint{2.368444in}{0.721443in}}{\pgfqpoint{2.376344in}{0.718171in}}{\pgfqpoint{2.384580in}{0.718171in}}%
\pgfpathclose%
\pgfusepath{stroke,fill}%
\end{pgfscope}%
\begin{pgfscope}%
\pgfpathrectangle{\pgfqpoint{0.100000in}{0.220728in}}{\pgfqpoint{3.696000in}{3.696000in}}%
\pgfusepath{clip}%
\pgfsetbuttcap%
\pgfsetroundjoin%
\definecolor{currentfill}{rgb}{0.121569,0.466667,0.705882}%
\pgfsetfillcolor{currentfill}%
\pgfsetfillopacity{0.999921}%
\pgfsetlinewidth{1.003750pt}%
\definecolor{currentstroke}{rgb}{0.121569,0.466667,0.705882}%
\pgfsetstrokecolor{currentstroke}%
\pgfsetstrokeopacity{0.999921}%
\pgfsetdash{}{0pt}%
\pgfpathmoveto{\pgfqpoint{2.379966in}{0.713896in}}%
\pgfpathcurveto{\pgfqpoint{2.388203in}{0.713896in}}{\pgfqpoint{2.396103in}{0.717168in}}{\pgfqpoint{2.401927in}{0.722992in}}%
\pgfpathcurveto{\pgfqpoint{2.407750in}{0.728816in}}{\pgfqpoint{2.411023in}{0.736716in}}{\pgfqpoint{2.411023in}{0.744952in}}%
\pgfpathcurveto{\pgfqpoint{2.411023in}{0.753188in}}{\pgfqpoint{2.407750in}{0.761089in}}{\pgfqpoint{2.401927in}{0.766912in}}%
\pgfpathcurveto{\pgfqpoint{2.396103in}{0.772736in}}{\pgfqpoint{2.388203in}{0.776009in}}{\pgfqpoint{2.379966in}{0.776009in}}%
\pgfpathcurveto{\pgfqpoint{2.371730in}{0.776009in}}{\pgfqpoint{2.363830in}{0.772736in}}{\pgfqpoint{2.358006in}{0.766912in}}%
\pgfpathcurveto{\pgfqpoint{2.352182in}{0.761089in}}{\pgfqpoint{2.348910in}{0.753188in}}{\pgfqpoint{2.348910in}{0.744952in}}%
\pgfpathcurveto{\pgfqpoint{2.348910in}{0.736716in}}{\pgfqpoint{2.352182in}{0.728816in}}{\pgfqpoint{2.358006in}{0.722992in}}%
\pgfpathcurveto{\pgfqpoint{2.363830in}{0.717168in}}{\pgfqpoint{2.371730in}{0.713896in}}{\pgfqpoint{2.379966in}{0.713896in}}%
\pgfpathclose%
\pgfusepath{stroke,fill}%
\end{pgfscope}%
\begin{pgfscope}%
\pgfpathrectangle{\pgfqpoint{0.100000in}{0.220728in}}{\pgfqpoint{3.696000in}{3.696000in}}%
\pgfusepath{clip}%
\pgfsetbuttcap%
\pgfsetroundjoin%
\definecolor{currentfill}{rgb}{0.121569,0.466667,0.705882}%
\pgfsetfillcolor{currentfill}%
\pgfsetfillopacity{0.999921}%
\pgfsetlinewidth{1.003750pt}%
\definecolor{currentstroke}{rgb}{0.121569,0.466667,0.705882}%
\pgfsetstrokecolor{currentstroke}%
\pgfsetstrokeopacity{0.999921}%
\pgfsetdash{}{0pt}%
\pgfpathmoveto{\pgfqpoint{2.384544in}{0.718080in}}%
\pgfpathcurveto{\pgfqpoint{2.392781in}{0.718080in}}{\pgfqpoint{2.400681in}{0.721352in}}{\pgfqpoint{2.406505in}{0.727176in}}%
\pgfpathcurveto{\pgfqpoint{2.412329in}{0.733000in}}{\pgfqpoint{2.415601in}{0.740900in}}{\pgfqpoint{2.415601in}{0.749136in}}%
\pgfpathcurveto{\pgfqpoint{2.415601in}{0.757372in}}{\pgfqpoint{2.412329in}{0.765272in}}{\pgfqpoint{2.406505in}{0.771096in}}%
\pgfpathcurveto{\pgfqpoint{2.400681in}{0.776920in}}{\pgfqpoint{2.392781in}{0.780193in}}{\pgfqpoint{2.384544in}{0.780193in}}%
\pgfpathcurveto{\pgfqpoint{2.376308in}{0.780193in}}{\pgfqpoint{2.368408in}{0.776920in}}{\pgfqpoint{2.362584in}{0.771096in}}%
\pgfpathcurveto{\pgfqpoint{2.356760in}{0.765272in}}{\pgfqpoint{2.353488in}{0.757372in}}{\pgfqpoint{2.353488in}{0.749136in}}%
\pgfpathcurveto{\pgfqpoint{2.353488in}{0.740900in}}{\pgfqpoint{2.356760in}{0.733000in}}{\pgfqpoint{2.362584in}{0.727176in}}%
\pgfpathcurveto{\pgfqpoint{2.368408in}{0.721352in}}{\pgfqpoint{2.376308in}{0.718080in}}{\pgfqpoint{2.384544in}{0.718080in}}%
\pgfpathclose%
\pgfusepath{stroke,fill}%
\end{pgfscope}%
\begin{pgfscope}%
\pgfpathrectangle{\pgfqpoint{0.100000in}{0.220728in}}{\pgfqpoint{3.696000in}{3.696000in}}%
\pgfusepath{clip}%
\pgfsetbuttcap%
\pgfsetroundjoin%
\definecolor{currentfill}{rgb}{0.121569,0.466667,0.705882}%
\pgfsetfillcolor{currentfill}%
\pgfsetfillopacity{0.999927}%
\pgfsetlinewidth{1.003750pt}%
\definecolor{currentstroke}{rgb}{0.121569,0.466667,0.705882}%
\pgfsetstrokecolor{currentstroke}%
\pgfsetstrokeopacity{0.999927}%
\pgfsetdash{}{0pt}%
\pgfpathmoveto{\pgfqpoint{2.384513in}{0.718040in}}%
\pgfpathcurveto{\pgfqpoint{2.392749in}{0.718040in}}{\pgfqpoint{2.400649in}{0.721313in}}{\pgfqpoint{2.406473in}{0.727136in}}%
\pgfpathcurveto{\pgfqpoint{2.412297in}{0.732960in}}{\pgfqpoint{2.415569in}{0.740860in}}{\pgfqpoint{2.415569in}{0.749097in}}%
\pgfpathcurveto{\pgfqpoint{2.415569in}{0.757333in}}{\pgfqpoint{2.412297in}{0.765233in}}{\pgfqpoint{2.406473in}{0.771057in}}%
\pgfpathcurveto{\pgfqpoint{2.400649in}{0.776881in}}{\pgfqpoint{2.392749in}{0.780153in}}{\pgfqpoint{2.384513in}{0.780153in}}%
\pgfpathcurveto{\pgfqpoint{2.376276in}{0.780153in}}{\pgfqpoint{2.368376in}{0.776881in}}{\pgfqpoint{2.362552in}{0.771057in}}%
\pgfpathcurveto{\pgfqpoint{2.356728in}{0.765233in}}{\pgfqpoint{2.353456in}{0.757333in}}{\pgfqpoint{2.353456in}{0.749097in}}%
\pgfpathcurveto{\pgfqpoint{2.353456in}{0.740860in}}{\pgfqpoint{2.356728in}{0.732960in}}{\pgfqpoint{2.362552in}{0.727136in}}%
\pgfpathcurveto{\pgfqpoint{2.368376in}{0.721313in}}{\pgfqpoint{2.376276in}{0.718040in}}{\pgfqpoint{2.384513in}{0.718040in}}%
\pgfpathclose%
\pgfusepath{stroke,fill}%
\end{pgfscope}%
\begin{pgfscope}%
\pgfpathrectangle{\pgfqpoint{0.100000in}{0.220728in}}{\pgfqpoint{3.696000in}{3.696000in}}%
\pgfusepath{clip}%
\pgfsetbuttcap%
\pgfsetroundjoin%
\definecolor{currentfill}{rgb}{0.121569,0.466667,0.705882}%
\pgfsetfillcolor{currentfill}%
\pgfsetfillopacity{0.999929}%
\pgfsetlinewidth{1.003750pt}%
\definecolor{currentstroke}{rgb}{0.121569,0.466667,0.705882}%
\pgfsetstrokecolor{currentstroke}%
\pgfsetstrokeopacity{0.999929}%
\pgfsetdash{}{0pt}%
\pgfpathmoveto{\pgfqpoint{2.384493in}{0.718016in}}%
\pgfpathcurveto{\pgfqpoint{2.392730in}{0.718016in}}{\pgfqpoint{2.400630in}{0.721288in}}{\pgfqpoint{2.406454in}{0.727112in}}%
\pgfpathcurveto{\pgfqpoint{2.412278in}{0.732936in}}{\pgfqpoint{2.415550in}{0.740836in}}{\pgfqpoint{2.415550in}{0.749072in}}%
\pgfpathcurveto{\pgfqpoint{2.415550in}{0.757308in}}{\pgfqpoint{2.412278in}{0.765208in}}{\pgfqpoint{2.406454in}{0.771032in}}%
\pgfpathcurveto{\pgfqpoint{2.400630in}{0.776856in}}{\pgfqpoint{2.392730in}{0.780129in}}{\pgfqpoint{2.384493in}{0.780129in}}%
\pgfpathcurveto{\pgfqpoint{2.376257in}{0.780129in}}{\pgfqpoint{2.368357in}{0.776856in}}{\pgfqpoint{2.362533in}{0.771032in}}%
\pgfpathcurveto{\pgfqpoint{2.356709in}{0.765208in}}{\pgfqpoint{2.353437in}{0.757308in}}{\pgfqpoint{2.353437in}{0.749072in}}%
\pgfpathcurveto{\pgfqpoint{2.353437in}{0.740836in}}{\pgfqpoint{2.356709in}{0.732936in}}{\pgfqpoint{2.362533in}{0.727112in}}%
\pgfpathcurveto{\pgfqpoint{2.368357in}{0.721288in}}{\pgfqpoint{2.376257in}{0.718016in}}{\pgfqpoint{2.384493in}{0.718016in}}%
\pgfpathclose%
\pgfusepath{stroke,fill}%
\end{pgfscope}%
\begin{pgfscope}%
\pgfpathrectangle{\pgfqpoint{0.100000in}{0.220728in}}{\pgfqpoint{3.696000in}{3.696000in}}%
\pgfusepath{clip}%
\pgfsetbuttcap%
\pgfsetroundjoin%
\definecolor{currentfill}{rgb}{0.121569,0.466667,0.705882}%
\pgfsetfillcolor{currentfill}%
\pgfsetlinewidth{1.003750pt}%
\definecolor{currentstroke}{rgb}{0.121569,0.466667,0.705882}%
\pgfsetstrokecolor{currentstroke}%
\pgfsetdash{}{0pt}%
\pgfpathmoveto{\pgfqpoint{2.381668in}{0.715037in}}%
\pgfpathcurveto{\pgfqpoint{2.389904in}{0.715037in}}{\pgfqpoint{2.397804in}{0.718310in}}{\pgfqpoint{2.403628in}{0.724134in}}%
\pgfpathcurveto{\pgfqpoint{2.409452in}{0.729957in}}{\pgfqpoint{2.412725in}{0.737858in}}{\pgfqpoint{2.412725in}{0.746094in}}%
\pgfpathcurveto{\pgfqpoint{2.412725in}{0.754330in}}{\pgfqpoint{2.409452in}{0.762230in}}{\pgfqpoint{2.403628in}{0.768054in}}%
\pgfpathcurveto{\pgfqpoint{2.397804in}{0.773878in}}{\pgfqpoint{2.389904in}{0.777150in}}{\pgfqpoint{2.381668in}{0.777150in}}%
\pgfpathcurveto{\pgfqpoint{2.373432in}{0.777150in}}{\pgfqpoint{2.365532in}{0.773878in}}{\pgfqpoint{2.359708in}{0.768054in}}%
\pgfpathcurveto{\pgfqpoint{2.353884in}{0.762230in}}{\pgfqpoint{2.350612in}{0.754330in}}{\pgfqpoint{2.350612in}{0.746094in}}%
\pgfpathcurveto{\pgfqpoint{2.350612in}{0.737858in}}{\pgfqpoint{2.353884in}{0.729957in}}{\pgfqpoint{2.359708in}{0.724134in}}%
\pgfpathcurveto{\pgfqpoint{2.365532in}{0.718310in}}{\pgfqpoint{2.373432in}{0.715037in}}{\pgfqpoint{2.381668in}{0.715037in}}%
\pgfpathclose%
\pgfusepath{stroke,fill}%
\end{pgfscope}%
\begin{pgfscope}%
\pgfsetbuttcap%
\pgfsetmiterjoin%
\definecolor{currentfill}{rgb}{1.000000,1.000000,1.000000}%
\pgfsetfillcolor{currentfill}%
\pgfsetfillopacity{0.800000}%
\pgfsetlinewidth{1.003750pt}%
\definecolor{currentstroke}{rgb}{0.800000,0.800000,0.800000}%
\pgfsetstrokecolor{currentstroke}%
\pgfsetstrokeopacity{0.800000}%
\pgfsetdash{}{0pt}%
\pgfpathmoveto{\pgfqpoint{0.197222in}{0.290172in}}%
\pgfpathlineto{\pgfqpoint{1.937579in}{0.290172in}}%
\pgfpathquadraticcurveto{\pgfqpoint{1.965356in}{0.290172in}}{\pgfqpoint{1.965356in}{0.317950in}}%
\pgfpathlineto{\pgfqpoint{1.965356in}{0.915633in}}%
\pgfpathquadraticcurveto{\pgfqpoint{1.965356in}{0.943411in}}{\pgfqpoint{1.937579in}{0.943411in}}%
\pgfpathlineto{\pgfqpoint{0.197222in}{0.943411in}}%
\pgfpathquadraticcurveto{\pgfqpoint{0.169444in}{0.943411in}}{\pgfqpoint{0.169444in}{0.915633in}}%
\pgfpathlineto{\pgfqpoint{0.169444in}{0.317950in}}%
\pgfpathquadraticcurveto{\pgfqpoint{0.169444in}{0.290172in}}{\pgfqpoint{0.197222in}{0.290172in}}%
\pgfpathclose%
\pgfusepath{stroke,fill}%
\end{pgfscope}%
\begin{pgfscope}%
\pgfsetrectcap%
\pgfsetroundjoin%
\pgfsetlinewidth{1.505625pt}%
\definecolor{currentstroke}{rgb}{0.121569,0.466667,0.705882}%
\pgfsetstrokecolor{currentstroke}%
\pgfsetdash{}{0pt}%
\pgfpathmoveto{\pgfqpoint{0.225000in}{0.830943in}}%
\pgfpathlineto{\pgfqpoint{0.502778in}{0.830943in}}%
\pgfusepath{stroke}%
\end{pgfscope}%
\begin{pgfscope}%
\definecolor{textcolor}{rgb}{0.000000,0.000000,0.000000}%
\pgfsetstrokecolor{textcolor}%
\pgfsetfillcolor{textcolor}%
\pgftext[x=0.613889in,y=0.782332in,left,base]{\color{textcolor}\sffamily\fontsize{10.000000}{12.000000}\selectfont Ground truth}%
\end{pgfscope}%
\begin{pgfscope}%
\pgfsetbuttcap%
\pgfsetroundjoin%
\definecolor{currentfill}{rgb}{0.121569,0.466667,0.705882}%
\pgfsetfillcolor{currentfill}%
\pgfsetlinewidth{1.003750pt}%
\definecolor{currentstroke}{rgb}{0.121569,0.466667,0.705882}%
\pgfsetstrokecolor{currentstroke}%
\pgfsetdash{}{0pt}%
\pgfsys@defobject{currentmarker}{\pgfqpoint{-0.031056in}{-0.031056in}}{\pgfqpoint{0.031056in}{0.031056in}}{%
\pgfpathmoveto{\pgfqpoint{0.000000in}{-0.031056in}}%
\pgfpathcurveto{\pgfqpoint{0.008236in}{-0.031056in}}{\pgfqpoint{0.016136in}{-0.027784in}}{\pgfqpoint{0.021960in}{-0.021960in}}%
\pgfpathcurveto{\pgfqpoint{0.027784in}{-0.016136in}}{\pgfqpoint{0.031056in}{-0.008236in}}{\pgfqpoint{0.031056in}{0.000000in}}%
\pgfpathcurveto{\pgfqpoint{0.031056in}{0.008236in}}{\pgfqpoint{0.027784in}{0.016136in}}{\pgfqpoint{0.021960in}{0.021960in}}%
\pgfpathcurveto{\pgfqpoint{0.016136in}{0.027784in}}{\pgfqpoint{0.008236in}{0.031056in}}{\pgfqpoint{0.000000in}{0.031056in}}%
\pgfpathcurveto{\pgfqpoint{-0.008236in}{0.031056in}}{\pgfqpoint{-0.016136in}{0.027784in}}{\pgfqpoint{-0.021960in}{0.021960in}}%
\pgfpathcurveto{\pgfqpoint{-0.027784in}{0.016136in}}{\pgfqpoint{-0.031056in}{0.008236in}}{\pgfqpoint{-0.031056in}{0.000000in}}%
\pgfpathcurveto{\pgfqpoint{-0.031056in}{-0.008236in}}{\pgfqpoint{-0.027784in}{-0.016136in}}{\pgfqpoint{-0.021960in}{-0.021960in}}%
\pgfpathcurveto{\pgfqpoint{-0.016136in}{-0.027784in}}{\pgfqpoint{-0.008236in}{-0.031056in}}{\pgfqpoint{0.000000in}{-0.031056in}}%
\pgfpathclose%
\pgfusepath{stroke,fill}%
}%
\begin{pgfscope}%
\pgfsys@transformshift{0.363889in}{0.614933in}%
\pgfsys@useobject{currentmarker}{}%
\end{pgfscope}%
\end{pgfscope}%
\begin{pgfscope}%
\definecolor{textcolor}{rgb}{0.000000,0.000000,0.000000}%
\pgfsetstrokecolor{textcolor}%
\pgfsetfillcolor{textcolor}%
\pgftext[x=0.613889in,y=0.578475in,left,base]{\color{textcolor}\sffamily\fontsize{10.000000}{12.000000}\selectfont Estimated position}%
\end{pgfscope}%
\begin{pgfscope}%
\pgfsetbuttcap%
\pgfsetroundjoin%
\definecolor{currentfill}{rgb}{1.000000,0.498039,0.054902}%
\pgfsetfillcolor{currentfill}%
\pgfsetlinewidth{1.003750pt}%
\definecolor{currentstroke}{rgb}{1.000000,0.498039,0.054902}%
\pgfsetstrokecolor{currentstroke}%
\pgfsetdash{}{0pt}%
\pgfsys@defobject{currentmarker}{\pgfqpoint{-0.031056in}{-0.031056in}}{\pgfqpoint{0.031056in}{0.031056in}}{%
\pgfpathmoveto{\pgfqpoint{0.000000in}{-0.031056in}}%
\pgfpathcurveto{\pgfqpoint{0.008236in}{-0.031056in}}{\pgfqpoint{0.016136in}{-0.027784in}}{\pgfqpoint{0.021960in}{-0.021960in}}%
\pgfpathcurveto{\pgfqpoint{0.027784in}{-0.016136in}}{\pgfqpoint{0.031056in}{-0.008236in}}{\pgfqpoint{0.031056in}{0.000000in}}%
\pgfpathcurveto{\pgfqpoint{0.031056in}{0.008236in}}{\pgfqpoint{0.027784in}{0.016136in}}{\pgfqpoint{0.021960in}{0.021960in}}%
\pgfpathcurveto{\pgfqpoint{0.016136in}{0.027784in}}{\pgfqpoint{0.008236in}{0.031056in}}{\pgfqpoint{0.000000in}{0.031056in}}%
\pgfpathcurveto{\pgfqpoint{-0.008236in}{0.031056in}}{\pgfqpoint{-0.016136in}{0.027784in}}{\pgfqpoint{-0.021960in}{0.021960in}}%
\pgfpathcurveto{\pgfqpoint{-0.027784in}{0.016136in}}{\pgfqpoint{-0.031056in}{0.008236in}}{\pgfqpoint{-0.031056in}{0.000000in}}%
\pgfpathcurveto{\pgfqpoint{-0.031056in}{-0.008236in}}{\pgfqpoint{-0.027784in}{-0.016136in}}{\pgfqpoint{-0.021960in}{-0.021960in}}%
\pgfpathcurveto{\pgfqpoint{-0.016136in}{-0.027784in}}{\pgfqpoint{-0.008236in}{-0.031056in}}{\pgfqpoint{0.000000in}{-0.031056in}}%
\pgfpathclose%
\pgfusepath{stroke,fill}%
}%
\begin{pgfscope}%
\pgfsys@transformshift{0.363889in}{0.411076in}%
\pgfsys@useobject{currentmarker}{}%
\end{pgfscope}%
\end{pgfscope}%
\begin{pgfscope}%
\definecolor{textcolor}{rgb}{0.000000,0.000000,0.000000}%
\pgfsetstrokecolor{textcolor}%
\pgfsetfillcolor{textcolor}%
\pgftext[x=0.613889in,y=0.374618in,left,base]{\color{textcolor}\sffamily\fontsize{10.000000}{12.000000}\selectfont Estimated turn}%
\end{pgfscope}%
\end{pgfpicture}%
\makeatother%
\endgroup%
}
%         \caption{MPU-9250 Breakout}
%         \label{fig:square163D}
%     \end{subfigure}
%     \caption{Position estimation by the best performing algorithms in the 4-meter line experiment.}
%     \label{fig:square16}
% \end{figure}

% %\subsubsection{28 meter}

% % For the 16-meter line experiment, the Mahony algorithm which had the lowest displacement error with an average of 0.48 meters (16\% of error margin), and ROLEQ with an average of 0.24 meters of turn error (7\% of error margin).

% % \begin{figure}[!h]
% %     \centering
% %     \begin{table}[H]
    \begin{center}
    \resizebox{1\linewidth}{!}{

        \begin{tabular}[t]{lcccc}
            \hline
            Algorithm                   & Displacement Error[$m$] & Displacement Error[\%]      & Turn Error[$m$]  & Turn Error[\%]             \\
            \hline 
            AngularRate            & 34.07  & 70.98 & 39.11 & 81.47              \\            AQUA            & 11.82  & 24.63 & 14.64 & 30.51              \\            Complementary            & 12.75  & 26.56 & 14.68 & 30.59              \\            Davenport            & 2.32  & 4.84 & 6.44 & 13.43              \\            EKF            & 3.81  & 7.95 & 6.39 & 13.32              \\            FAMC            & 31.54  & 65.70 & 41.11 & 85.66              \\            FLAE            & 2.28  & 4.74 & 6.59 & 13.72              \\            Fourati            & 54.07  & 112.65 & 56.38 & 117.45              \\            Madgwick            & 3.35  & 6.98 & 6.41 & 13.36              \\            Mahony            & 2.63  & 5.47 & 6.42 & 13.37              \\            OLEQ            & 2.67  & 5.56 & 7.16 & 14.92              \\            QUEST            & 22.92  & 47.74 & 33.83 & 70.48              \\            ROLEQ            & 2.85  & 5.94 & 7.54 & 15.71              \\            SAAM            & 2.65  & 5.52 & 6.36 & 13.24              \\            Tilt            & 2.65  & 5.52 & 6.36 & 13.24              \\
            \hline
            Average & 12.82 & 26.72 & 17.30 & 36.03
        \end{tabular}
        }
        \caption{Accelerometer Specifications. }
        \label{tab:accelerometer_specification}
    \end{center}
\end{table}
% % \end{figure}

% % % \begin{figure}[!h]
% % %     \centering
% % %     \begin{subfigure}{0.49\textwidth}
% % %         \centering
% % %         \resizebox{1\linewidth}{!}{%% Creator: Matplotlib, PGF backend
%%
%% To include the figure in your LaTeX document, write
%%   \input{<filename>.pgf}
%%
%% Make sure the required packages are loaded in your preamble
%%   \usepackage{pgf}
%%
%% and, on pdftex
%%   \usepackage[utf8]{inputenc}\DeclareUnicodeCharacter{2212}{-}
%%
%% or, on luatex and xetex
%%   \usepackage{unicode-math}
%%
%% Figures using additional raster images can only be included by \input if
%% they are in the same directory as the main LaTeX file. For loading figures
%% from other directories you can use the `import` package
%%   \usepackage{import}
%%
%% and then include the figures with
%%   \import{<path to file>}{<filename>.pgf}
%%
%% Matplotlib used the following preamble
%%   \usepackage{fontspec}
%%
\begingroup%
\makeatletter%
\begin{pgfpicture}%
\pgfpathrectangle{\pgfpointorigin}{\pgfqpoint{4.342355in}{4.209289in}}%
\pgfusepath{use as bounding box, clip}%
\begin{pgfscope}%
\pgfsetbuttcap%
\pgfsetmiterjoin%
\definecolor{currentfill}{rgb}{1.000000,1.000000,1.000000}%
\pgfsetfillcolor{currentfill}%
\pgfsetlinewidth{0.000000pt}%
\definecolor{currentstroke}{rgb}{1.000000,1.000000,1.000000}%
\pgfsetstrokecolor{currentstroke}%
\pgfsetdash{}{0pt}%
\pgfpathmoveto{\pgfqpoint{0.000000in}{-0.000000in}}%
\pgfpathlineto{\pgfqpoint{4.342355in}{-0.000000in}}%
\pgfpathlineto{\pgfqpoint{4.342355in}{4.209289in}}%
\pgfpathlineto{\pgfqpoint{0.000000in}{4.209289in}}%
\pgfpathclose%
\pgfusepath{fill}%
\end{pgfscope}%
\begin{pgfscope}%
\pgfsetbuttcap%
\pgfsetmiterjoin%
\definecolor{currentfill}{rgb}{1.000000,1.000000,1.000000}%
\pgfsetfillcolor{currentfill}%
\pgfsetlinewidth{0.000000pt}%
\definecolor{currentstroke}{rgb}{0.000000,0.000000,0.000000}%
\pgfsetstrokecolor{currentstroke}%
\pgfsetstrokeopacity{0.000000}%
\pgfsetdash{}{0pt}%
\pgfpathmoveto{\pgfqpoint{0.100000in}{0.212622in}}%
\pgfpathlineto{\pgfqpoint{3.796000in}{0.212622in}}%
\pgfpathlineto{\pgfqpoint{3.796000in}{3.908622in}}%
\pgfpathlineto{\pgfqpoint{0.100000in}{3.908622in}}%
\pgfpathclose%
\pgfusepath{fill}%
\end{pgfscope}%
\begin{pgfscope}%
\pgfsetbuttcap%
\pgfsetmiterjoin%
\definecolor{currentfill}{rgb}{0.950000,0.950000,0.950000}%
\pgfsetfillcolor{currentfill}%
\pgfsetfillopacity{0.500000}%
\pgfsetlinewidth{1.003750pt}%
\definecolor{currentstroke}{rgb}{0.950000,0.950000,0.950000}%
\pgfsetstrokecolor{currentstroke}%
\pgfsetstrokeopacity{0.500000}%
\pgfsetdash{}{0pt}%
\pgfpathmoveto{\pgfqpoint{0.379073in}{1.123938in}}%
\pgfpathlineto{\pgfqpoint{1.599613in}{2.147018in}}%
\pgfpathlineto{\pgfqpoint{1.582647in}{3.622484in}}%
\pgfpathlineto{\pgfqpoint{0.303698in}{2.689165in}}%
\pgfusepath{stroke,fill}%
\end{pgfscope}%
\begin{pgfscope}%
\pgfsetbuttcap%
\pgfsetmiterjoin%
\definecolor{currentfill}{rgb}{0.900000,0.900000,0.900000}%
\pgfsetfillcolor{currentfill}%
\pgfsetfillopacity{0.500000}%
\pgfsetlinewidth{1.003750pt}%
\definecolor{currentstroke}{rgb}{0.900000,0.900000,0.900000}%
\pgfsetstrokecolor{currentstroke}%
\pgfsetstrokeopacity{0.500000}%
\pgfsetdash{}{0pt}%
\pgfpathmoveto{\pgfqpoint{1.599613in}{2.147018in}}%
\pgfpathlineto{\pgfqpoint{3.558144in}{1.577751in}}%
\pgfpathlineto{\pgfqpoint{3.628038in}{3.104037in}}%
\pgfpathlineto{\pgfqpoint{1.582647in}{3.622484in}}%
\pgfusepath{stroke,fill}%
\end{pgfscope}%
\begin{pgfscope}%
\pgfsetbuttcap%
\pgfsetmiterjoin%
\definecolor{currentfill}{rgb}{0.925000,0.925000,0.925000}%
\pgfsetfillcolor{currentfill}%
\pgfsetfillopacity{0.500000}%
\pgfsetlinewidth{1.003750pt}%
\definecolor{currentstroke}{rgb}{0.925000,0.925000,0.925000}%
\pgfsetstrokecolor{currentstroke}%
\pgfsetstrokeopacity{0.500000}%
\pgfsetdash{}{0pt}%
\pgfpathmoveto{\pgfqpoint{0.379073in}{1.123938in}}%
\pgfpathlineto{\pgfqpoint{2.455212in}{0.445871in}}%
\pgfpathlineto{\pgfqpoint{3.558144in}{1.577751in}}%
\pgfpathlineto{\pgfqpoint{1.599613in}{2.147018in}}%
\pgfusepath{stroke,fill}%
\end{pgfscope}%
\begin{pgfscope}%
\pgfsetrectcap%
\pgfsetroundjoin%
\pgfsetlinewidth{0.803000pt}%
\definecolor{currentstroke}{rgb}{0.000000,0.000000,0.000000}%
\pgfsetstrokecolor{currentstroke}%
\pgfsetdash{}{0pt}%
\pgfpathmoveto{\pgfqpoint{0.379073in}{1.123938in}}%
\pgfpathlineto{\pgfqpoint{2.455212in}{0.445871in}}%
\pgfusepath{stroke}%
\end{pgfscope}%
\begin{pgfscope}%
\definecolor{textcolor}{rgb}{0.000000,0.000000,0.000000}%
\pgfsetstrokecolor{textcolor}%
\pgfsetfillcolor{textcolor}%
\pgftext[x=0.730374in, y=0.408886in, left, base,rotate=341.912962]{\color{textcolor}\rmfamily\fontsize{10.000000}{12.000000}\selectfont Position X [\(\displaystyle m\)]}%
\end{pgfscope}%
\begin{pgfscope}%
\pgfsetbuttcap%
\pgfsetroundjoin%
\pgfsetlinewidth{0.803000pt}%
\definecolor{currentstroke}{rgb}{0.690196,0.690196,0.690196}%
\pgfsetstrokecolor{currentstroke}%
\pgfsetdash{}{0pt}%
\pgfpathmoveto{\pgfqpoint{0.445276in}{1.102316in}}%
\pgfpathlineto{\pgfqpoint{1.662333in}{2.128788in}}%
\pgfpathlineto{\pgfqpoint{1.648014in}{3.605915in}}%
\pgfusepath{stroke}%
\end{pgfscope}%
\begin{pgfscope}%
\pgfsetbuttcap%
\pgfsetroundjoin%
\pgfsetlinewidth{0.803000pt}%
\definecolor{currentstroke}{rgb}{0.690196,0.690196,0.690196}%
\pgfsetstrokecolor{currentstroke}%
\pgfsetdash{}{0pt}%
\pgfpathmoveto{\pgfqpoint{0.696767in}{1.020179in}}%
\pgfpathlineto{\pgfqpoint{1.900432in}{2.059582in}}%
\pgfpathlineto{\pgfqpoint{1.896246in}{3.542995in}}%
\pgfusepath{stroke}%
\end{pgfscope}%
\begin{pgfscope}%
\pgfsetbuttcap%
\pgfsetroundjoin%
\pgfsetlinewidth{0.803000pt}%
\definecolor{currentstroke}{rgb}{0.690196,0.690196,0.690196}%
\pgfsetstrokecolor{currentstroke}%
\pgfsetdash{}{0pt}%
\pgfpathmoveto{\pgfqpoint{0.951573in}{0.936959in}}%
\pgfpathlineto{\pgfqpoint{2.141411in}{1.989539in}}%
\pgfpathlineto{\pgfqpoint{2.147609in}{3.479282in}}%
\pgfusepath{stroke}%
\end{pgfscope}%
\begin{pgfscope}%
\pgfsetbuttcap%
\pgfsetroundjoin%
\pgfsetlinewidth{0.803000pt}%
\definecolor{currentstroke}{rgb}{0.690196,0.690196,0.690196}%
\pgfsetstrokecolor{currentstroke}%
\pgfsetdash{}{0pt}%
\pgfpathmoveto{\pgfqpoint{1.209761in}{0.852635in}}%
\pgfpathlineto{\pgfqpoint{2.385321in}{1.918644in}}%
\pgfpathlineto{\pgfqpoint{2.402163in}{3.414760in}}%
\pgfusepath{stroke}%
\end{pgfscope}%
\begin{pgfscope}%
\pgfsetbuttcap%
\pgfsetroundjoin%
\pgfsetlinewidth{0.803000pt}%
\definecolor{currentstroke}{rgb}{0.690196,0.690196,0.690196}%
\pgfsetstrokecolor{currentstroke}%
\pgfsetdash{}{0pt}%
\pgfpathmoveto{\pgfqpoint{1.471398in}{0.767184in}}%
\pgfpathlineto{\pgfqpoint{2.632216in}{1.846881in}}%
\pgfpathlineto{\pgfqpoint{2.659969in}{3.349414in}}%
\pgfusepath{stroke}%
\end{pgfscope}%
\begin{pgfscope}%
\pgfsetbuttcap%
\pgfsetroundjoin%
\pgfsetlinewidth{0.803000pt}%
\definecolor{currentstroke}{rgb}{0.690196,0.690196,0.690196}%
\pgfsetstrokecolor{currentstroke}%
\pgfsetdash{}{0pt}%
\pgfpathmoveto{\pgfqpoint{1.736554in}{0.680584in}}%
\pgfpathlineto{\pgfqpoint{2.882152in}{1.774235in}}%
\pgfpathlineto{\pgfqpoint{2.921089in}{3.283228in}}%
\pgfusepath{stroke}%
\end{pgfscope}%
\begin{pgfscope}%
\pgfsetbuttcap%
\pgfsetroundjoin%
\pgfsetlinewidth{0.803000pt}%
\definecolor{currentstroke}{rgb}{0.690196,0.690196,0.690196}%
\pgfsetstrokecolor{currentstroke}%
\pgfsetdash{}{0pt}%
\pgfpathmoveto{\pgfqpoint{2.005300in}{0.592812in}}%
\pgfpathlineto{\pgfqpoint{3.135185in}{1.700688in}}%
\pgfpathlineto{\pgfqpoint{3.185589in}{3.216185in}}%
\pgfusepath{stroke}%
\end{pgfscope}%
\begin{pgfscope}%
\pgfsetbuttcap%
\pgfsetroundjoin%
\pgfsetlinewidth{0.803000pt}%
\definecolor{currentstroke}{rgb}{0.690196,0.690196,0.690196}%
\pgfsetstrokecolor{currentstroke}%
\pgfsetdash{}{0pt}%
\pgfpathmoveto{\pgfqpoint{2.277711in}{0.503843in}}%
\pgfpathlineto{\pgfqpoint{3.391373in}{1.626225in}}%
\pgfpathlineto{\pgfqpoint{3.453533in}{3.148269in}}%
\pgfusepath{stroke}%
\end{pgfscope}%
\begin{pgfscope}%
\pgfsetrectcap%
\pgfsetroundjoin%
\pgfsetlinewidth{0.803000pt}%
\definecolor{currentstroke}{rgb}{0.000000,0.000000,0.000000}%
\pgfsetstrokecolor{currentstroke}%
\pgfsetdash{}{0pt}%
\pgfpathmoveto{\pgfqpoint{0.455872in}{1.111253in}}%
\pgfpathlineto{\pgfqpoint{0.424037in}{1.084403in}}%
\pgfusepath{stroke}%
\end{pgfscope}%
\begin{pgfscope}%
\definecolor{textcolor}{rgb}{0.000000,0.000000,0.000000}%
\pgfsetstrokecolor{textcolor}%
\pgfsetfillcolor{textcolor}%
\pgftext[x=0.340647in,y=0.884523in,,top]{\color{textcolor}\rmfamily\fontsize{10.000000}{12.000000}\selectfont \(\displaystyle {−5}\)}%
\end{pgfscope}%
\begin{pgfscope}%
\pgfsetrectcap%
\pgfsetroundjoin%
\pgfsetlinewidth{0.803000pt}%
\definecolor{currentstroke}{rgb}{0.000000,0.000000,0.000000}%
\pgfsetstrokecolor{currentstroke}%
\pgfsetdash{}{0pt}%
\pgfpathmoveto{\pgfqpoint{0.707252in}{1.029234in}}%
\pgfpathlineto{\pgfqpoint{0.675750in}{1.002031in}}%
\pgfusepath{stroke}%
\end{pgfscope}%
\begin{pgfscope}%
\definecolor{textcolor}{rgb}{0.000000,0.000000,0.000000}%
\pgfsetstrokecolor{textcolor}%
\pgfsetfillcolor{textcolor}%
\pgftext[x=0.592390in,y=0.800647in,,top]{\color{textcolor}\rmfamily\fontsize{10.000000}{12.000000}\selectfont \(\displaystyle {0}\)}%
\end{pgfscope}%
\begin{pgfscope}%
\pgfsetrectcap%
\pgfsetroundjoin%
\pgfsetlinewidth{0.803000pt}%
\definecolor{currentstroke}{rgb}{0.000000,0.000000,0.000000}%
\pgfsetstrokecolor{currentstroke}%
\pgfsetdash{}{0pt}%
\pgfpathmoveto{\pgfqpoint{0.961944in}{0.946134in}}%
\pgfpathlineto{\pgfqpoint{0.930786in}{0.918571in}}%
\pgfusepath{stroke}%
\end{pgfscope}%
\begin{pgfscope}%
\definecolor{textcolor}{rgb}{0.000000,0.000000,0.000000}%
\pgfsetstrokecolor{textcolor}%
\pgfsetfillcolor{textcolor}%
\pgftext[x=0.847464in,y=0.715662in,,top]{\color{textcolor}\rmfamily\fontsize{10.000000}{12.000000}\selectfont \(\displaystyle {5}\)}%
\end{pgfscope}%
\begin{pgfscope}%
\pgfsetrectcap%
\pgfsetroundjoin%
\pgfsetlinewidth{0.803000pt}%
\definecolor{currentstroke}{rgb}{0.000000,0.000000,0.000000}%
\pgfsetstrokecolor{currentstroke}%
\pgfsetdash{}{0pt}%
\pgfpathmoveto{\pgfqpoint{1.220013in}{0.861932in}}%
\pgfpathlineto{\pgfqpoint{1.189212in}{0.834001in}}%
\pgfusepath{stroke}%
\end{pgfscope}%
\begin{pgfscope}%
\definecolor{textcolor}{rgb}{0.000000,0.000000,0.000000}%
\pgfsetstrokecolor{textcolor}%
\pgfsetfillcolor{textcolor}%
\pgftext[x=1.105936in,y=0.629545in,,top]{\color{textcolor}\rmfamily\fontsize{10.000000}{12.000000}\selectfont \(\displaystyle {10}\)}%
\end{pgfscope}%
\begin{pgfscope}%
\pgfsetrectcap%
\pgfsetroundjoin%
\pgfsetlinewidth{0.803000pt}%
\definecolor{currentstroke}{rgb}{0.000000,0.000000,0.000000}%
\pgfsetstrokecolor{currentstroke}%
\pgfsetdash{}{0pt}%
\pgfpathmoveto{\pgfqpoint{1.481527in}{0.776606in}}%
\pgfpathlineto{\pgfqpoint{1.451096in}{0.748301in}}%
\pgfusepath{stroke}%
\end{pgfscope}%
\begin{pgfscope}%
\definecolor{textcolor}{rgb}{0.000000,0.000000,0.000000}%
\pgfsetstrokecolor{textcolor}%
\pgfsetfillcolor{textcolor}%
\pgftext[x=1.367874in,y=0.542273in,,top]{\color{textcolor}\rmfamily\fontsize{10.000000}{12.000000}\selectfont \(\displaystyle {15}\)}%
\end{pgfscope}%
\begin{pgfscope}%
\pgfsetrectcap%
\pgfsetroundjoin%
\pgfsetlinewidth{0.803000pt}%
\definecolor{currentstroke}{rgb}{0.000000,0.000000,0.000000}%
\pgfsetstrokecolor{currentstroke}%
\pgfsetdash{}{0pt}%
\pgfpathmoveto{\pgfqpoint{1.746556in}{0.690133in}}%
\pgfpathlineto{\pgfqpoint{1.716506in}{0.661446in}}%
\pgfusepath{stroke}%
\end{pgfscope}%
\begin{pgfscope}%
\definecolor{textcolor}{rgb}{0.000000,0.000000,0.000000}%
\pgfsetstrokecolor{textcolor}%
\pgfsetfillcolor{textcolor}%
\pgftext[x=1.633348in,y=0.453822in,,top]{\color{textcolor}\rmfamily\fontsize{10.000000}{12.000000}\selectfont \(\displaystyle {20}\)}%
\end{pgfscope}%
\begin{pgfscope}%
\pgfsetrectcap%
\pgfsetroundjoin%
\pgfsetlinewidth{0.803000pt}%
\definecolor{currentstroke}{rgb}{0.000000,0.000000,0.000000}%
\pgfsetstrokecolor{currentstroke}%
\pgfsetdash{}{0pt}%
\pgfpathmoveto{\pgfqpoint{2.015171in}{0.602490in}}%
\pgfpathlineto{\pgfqpoint{1.985516in}{0.573413in}}%
\pgfusepath{stroke}%
\end{pgfscope}%
\begin{pgfscope}%
\definecolor{textcolor}{rgb}{0.000000,0.000000,0.000000}%
\pgfsetstrokecolor{textcolor}%
\pgfsetfillcolor{textcolor}%
\pgftext[x=1.902431in,y=0.364169in,,top]{\color{textcolor}\rmfamily\fontsize{10.000000}{12.000000}\selectfont \(\displaystyle {25}\)}%
\end{pgfscope}%
\begin{pgfscope}%
\pgfsetrectcap%
\pgfsetroundjoin%
\pgfsetlinewidth{0.803000pt}%
\definecolor{currentstroke}{rgb}{0.000000,0.000000,0.000000}%
\pgfsetstrokecolor{currentstroke}%
\pgfsetdash{}{0pt}%
\pgfpathmoveto{\pgfqpoint{2.287445in}{0.513653in}}%
\pgfpathlineto{\pgfqpoint{2.258199in}{0.484178in}}%
\pgfusepath{stroke}%
\end{pgfscope}%
\begin{pgfscope}%
\definecolor{textcolor}{rgb}{0.000000,0.000000,0.000000}%
\pgfsetstrokecolor{textcolor}%
\pgfsetfillcolor{textcolor}%
\pgftext[x=2.175197in,y=0.273290in,,top]{\color{textcolor}\rmfamily\fontsize{10.000000}{12.000000}\selectfont \(\displaystyle {30}\)}%
\end{pgfscope}%
\begin{pgfscope}%
\pgfsetrectcap%
\pgfsetroundjoin%
\pgfsetlinewidth{0.803000pt}%
\definecolor{currentstroke}{rgb}{0.000000,0.000000,0.000000}%
\pgfsetstrokecolor{currentstroke}%
\pgfsetdash{}{0pt}%
\pgfpathmoveto{\pgfqpoint{3.558144in}{1.577751in}}%
\pgfpathlineto{\pgfqpoint{2.455212in}{0.445871in}}%
\pgfusepath{stroke}%
\end{pgfscope}%
\begin{pgfscope}%
\definecolor{textcolor}{rgb}{0.000000,0.000000,0.000000}%
\pgfsetstrokecolor{textcolor}%
\pgfsetfillcolor{textcolor}%
\pgftext[x=3.120747in, y=0.305657in, left, base,rotate=45.742112]{\color{textcolor}\rmfamily\fontsize{10.000000}{12.000000}\selectfont Position Y [\(\displaystyle m\)]}%
\end{pgfscope}%
\begin{pgfscope}%
\pgfsetbuttcap%
\pgfsetroundjoin%
\pgfsetlinewidth{0.803000pt}%
\definecolor{currentstroke}{rgb}{0.690196,0.690196,0.690196}%
\pgfsetstrokecolor{currentstroke}%
\pgfsetdash{}{0pt}%
\pgfpathmoveto{\pgfqpoint{0.412766in}{2.768758in}}%
\pgfpathlineto{\pgfqpoint{0.482800in}{1.210883in}}%
\pgfpathlineto{\pgfqpoint{2.549323in}{0.542452in}}%
\pgfusepath{stroke}%
\end{pgfscope}%
\begin{pgfscope}%
\pgfsetbuttcap%
\pgfsetroundjoin%
\pgfsetlinewidth{0.803000pt}%
\definecolor{currentstroke}{rgb}{0.690196,0.690196,0.690196}%
\pgfsetstrokecolor{currentstroke}%
\pgfsetdash{}{0pt}%
\pgfpathmoveto{\pgfqpoint{0.586945in}{2.895866in}}%
\pgfpathlineto{\pgfqpoint{0.648587in}{1.349849in}}%
\pgfpathlineto{\pgfqpoint{2.699595in}{0.696667in}}%
\pgfusepath{stroke}%
\end{pgfscope}%
\begin{pgfscope}%
\pgfsetbuttcap%
\pgfsetroundjoin%
\pgfsetlinewidth{0.803000pt}%
\definecolor{currentstroke}{rgb}{0.690196,0.690196,0.690196}%
\pgfsetstrokecolor{currentstroke}%
\pgfsetdash{}{0pt}%
\pgfpathmoveto{\pgfqpoint{0.757099in}{3.020037in}}%
\pgfpathlineto{\pgfqpoint{0.810708in}{1.485743in}}%
\pgfpathlineto{\pgfqpoint{2.846370in}{0.847295in}}%
\pgfusepath{stroke}%
\end{pgfscope}%
\begin{pgfscope}%
\pgfsetbuttcap%
\pgfsetroundjoin%
\pgfsetlinewidth{0.803000pt}%
\definecolor{currentstroke}{rgb}{0.690196,0.690196,0.690196}%
\pgfsetstrokecolor{currentstroke}%
\pgfsetdash{}{0pt}%
\pgfpathmoveto{\pgfqpoint{0.923368in}{3.141372in}}%
\pgfpathlineto{\pgfqpoint{0.969286in}{1.618665in}}%
\pgfpathlineto{\pgfqpoint{2.989769in}{0.994458in}}%
\pgfusepath{stroke}%
\end{pgfscope}%
\begin{pgfscope}%
\pgfsetbuttcap%
\pgfsetroundjoin%
\pgfsetlinewidth{0.803000pt}%
\definecolor{currentstroke}{rgb}{0.690196,0.690196,0.690196}%
\pgfsetstrokecolor{currentstroke}%
\pgfsetdash{}{0pt}%
\pgfpathmoveto{\pgfqpoint{1.085881in}{3.259967in}}%
\pgfpathlineto{\pgfqpoint{1.124434in}{1.748713in}}%
\pgfpathlineto{\pgfqpoint{3.129909in}{1.138275in}}%
\pgfusepath{stroke}%
\end{pgfscope}%
\begin{pgfscope}%
\pgfsetbuttcap%
\pgfsetroundjoin%
\pgfsetlinewidth{0.803000pt}%
\definecolor{currentstroke}{rgb}{0.690196,0.690196,0.690196}%
\pgfsetstrokecolor{currentstroke}%
\pgfsetdash{}{0pt}%
\pgfpathmoveto{\pgfqpoint{1.244766in}{3.375914in}}%
\pgfpathlineto{\pgfqpoint{1.276262in}{1.875979in}}%
\pgfpathlineto{\pgfqpoint{3.266897in}{1.278860in}}%
\pgfusepath{stroke}%
\end{pgfscope}%
\begin{pgfscope}%
\pgfsetbuttcap%
\pgfsetroundjoin%
\pgfsetlinewidth{0.803000pt}%
\definecolor{currentstroke}{rgb}{0.690196,0.690196,0.690196}%
\pgfsetstrokecolor{currentstroke}%
\pgfsetdash{}{0pt}%
\pgfpathmoveto{\pgfqpoint{1.400142in}{3.489300in}}%
\pgfpathlineto{\pgfqpoint{1.424876in}{2.000550in}}%
\pgfpathlineto{\pgfqpoint{3.400841in}{1.416318in}}%
\pgfusepath{stroke}%
\end{pgfscope}%
\begin{pgfscope}%
\pgfsetbuttcap%
\pgfsetroundjoin%
\pgfsetlinewidth{0.803000pt}%
\definecolor{currentstroke}{rgb}{0.690196,0.690196,0.690196}%
\pgfsetstrokecolor{currentstroke}%
\pgfsetdash{}{0pt}%
\pgfpathmoveto{\pgfqpoint{1.552125in}{3.600210in}}%
\pgfpathlineto{\pgfqpoint{1.570377in}{2.122512in}}%
\pgfpathlineto{\pgfqpoint{3.531839in}{1.550755in}}%
\pgfusepath{stroke}%
\end{pgfscope}%
\begin{pgfscope}%
\pgfsetrectcap%
\pgfsetroundjoin%
\pgfsetlinewidth{0.803000pt}%
\definecolor{currentstroke}{rgb}{0.000000,0.000000,0.000000}%
\pgfsetstrokecolor{currentstroke}%
\pgfsetdash{}{0pt}%
\pgfpathmoveto{\pgfqpoint{2.531910in}{0.548084in}}%
\pgfpathlineto{\pgfqpoint{2.584195in}{0.531172in}}%
\pgfusepath{stroke}%
\end{pgfscope}%
\begin{pgfscope}%
\definecolor{textcolor}{rgb}{0.000000,0.000000,0.000000}%
\pgfsetstrokecolor{textcolor}%
\pgfsetfillcolor{textcolor}%
\pgftext[x=2.727982in,y=0.356288in,,top]{\color{textcolor}\rmfamily\fontsize{10.000000}{12.000000}\selectfont \(\displaystyle {0}\)}%
\end{pgfscope}%
\begin{pgfscope}%
\pgfsetrectcap%
\pgfsetroundjoin%
\pgfsetlinewidth{0.803000pt}%
\definecolor{currentstroke}{rgb}{0.000000,0.000000,0.000000}%
\pgfsetstrokecolor{currentstroke}%
\pgfsetdash{}{0pt}%
\pgfpathmoveto{\pgfqpoint{2.682322in}{0.702168in}}%
\pgfpathlineto{\pgfqpoint{2.734183in}{0.685652in}}%
\pgfusepath{stroke}%
\end{pgfscope}%
\begin{pgfscope}%
\definecolor{textcolor}{rgb}{0.000000,0.000000,0.000000}%
\pgfsetstrokecolor{textcolor}%
\pgfsetfillcolor{textcolor}%
\pgftext[x=2.876238in,y=0.512787in,,top]{\color{textcolor}\rmfamily\fontsize{10.000000}{12.000000}\selectfont \(\displaystyle {5}\)}%
\end{pgfscope}%
\begin{pgfscope}%
\pgfsetrectcap%
\pgfsetroundjoin%
\pgfsetlinewidth{0.803000pt}%
\definecolor{currentstroke}{rgb}{0.000000,0.000000,0.000000}%
\pgfsetstrokecolor{currentstroke}%
\pgfsetdash{}{0pt}%
\pgfpathmoveto{\pgfqpoint{2.829237in}{0.852668in}}%
\pgfpathlineto{\pgfqpoint{2.880679in}{0.836534in}}%
\pgfusepath{stroke}%
\end{pgfscope}%
\begin{pgfscope}%
\definecolor{textcolor}{rgb}{0.000000,0.000000,0.000000}%
\pgfsetstrokecolor{textcolor}%
\pgfsetfillcolor{textcolor}%
\pgftext[x=3.021043in,y=0.665643in,,top]{\color{textcolor}\rmfamily\fontsize{10.000000}{12.000000}\selectfont \(\displaystyle {10}\)}%
\end{pgfscope}%
\begin{pgfscope}%
\pgfsetrectcap%
\pgfsetroundjoin%
\pgfsetlinewidth{0.803000pt}%
\definecolor{currentstroke}{rgb}{0.000000,0.000000,0.000000}%
\pgfsetstrokecolor{currentstroke}%
\pgfsetdash{}{0pt}%
\pgfpathmoveto{\pgfqpoint{2.972774in}{0.999709in}}%
\pgfpathlineto{\pgfqpoint{3.023803in}{0.983944in}}%
\pgfusepath{stroke}%
\end{pgfscope}%
\begin{pgfscope}%
\definecolor{textcolor}{rgb}{0.000000,0.000000,0.000000}%
\pgfsetstrokecolor{textcolor}%
\pgfsetfillcolor{textcolor}%
\pgftext[x=3.162516in,y=0.814981in,,top]{\color{textcolor}\rmfamily\fontsize{10.000000}{12.000000}\selectfont \(\displaystyle {15}\)}%
\end{pgfscope}%
\begin{pgfscope}%
\pgfsetrectcap%
\pgfsetroundjoin%
\pgfsetlinewidth{0.803000pt}%
\definecolor{currentstroke}{rgb}{0.000000,0.000000,0.000000}%
\pgfsetstrokecolor{currentstroke}%
\pgfsetdash{}{0pt}%
\pgfpathmoveto{\pgfqpoint{3.113049in}{1.143407in}}%
\pgfpathlineto{\pgfqpoint{3.163671in}{1.127999in}}%
\pgfusepath{stroke}%
\end{pgfscope}%
\begin{pgfscope}%
\definecolor{textcolor}{rgb}{0.000000,0.000000,0.000000}%
\pgfsetstrokecolor{textcolor}%
\pgfsetfillcolor{textcolor}%
\pgftext[x=3.300770in,y=0.960922in,,top]{\color{textcolor}\rmfamily\fontsize{10.000000}{12.000000}\selectfont \(\displaystyle {20}\)}%
\end{pgfscope}%
\begin{pgfscope}%
\pgfsetrectcap%
\pgfsetroundjoin%
\pgfsetlinewidth{0.803000pt}%
\definecolor{currentstroke}{rgb}{0.000000,0.000000,0.000000}%
\pgfsetstrokecolor{currentstroke}%
\pgfsetdash{}{0pt}%
\pgfpathmoveto{\pgfqpoint{3.250171in}{1.283877in}}%
\pgfpathlineto{\pgfqpoint{3.300391in}{1.268813in}}%
\pgfusepath{stroke}%
\end{pgfscope}%
\begin{pgfscope}%
\definecolor{textcolor}{rgb}{0.000000,0.000000,0.000000}%
\pgfsetstrokecolor{textcolor}%
\pgfsetfillcolor{textcolor}%
\pgftext[x=3.435915in,y=1.103580in,,top]{\color{textcolor}\rmfamily\fontsize{10.000000}{12.000000}\selectfont \(\displaystyle {25}\)}%
\end{pgfscope}%
\begin{pgfscope}%
\pgfsetrectcap%
\pgfsetroundjoin%
\pgfsetlinewidth{0.803000pt}%
\definecolor{currentstroke}{rgb}{0.000000,0.000000,0.000000}%
\pgfsetstrokecolor{currentstroke}%
\pgfsetdash{}{0pt}%
\pgfpathmoveto{\pgfqpoint{3.384247in}{1.421225in}}%
\pgfpathlineto{\pgfqpoint{3.434069in}{1.406494in}}%
\pgfusepath{stroke}%
\end{pgfscope}%
\begin{pgfscope}%
\definecolor{textcolor}{rgb}{0.000000,0.000000,0.000000}%
\pgfsetstrokecolor{textcolor}%
\pgfsetfillcolor{textcolor}%
\pgftext[x=3.568053in,y=1.243065in,,top]{\color{textcolor}\rmfamily\fontsize{10.000000}{12.000000}\selectfont \(\displaystyle {30}\)}%
\end{pgfscope}%
\begin{pgfscope}%
\pgfsetrectcap%
\pgfsetroundjoin%
\pgfsetlinewidth{0.803000pt}%
\definecolor{currentstroke}{rgb}{0.000000,0.000000,0.000000}%
\pgfsetstrokecolor{currentstroke}%
\pgfsetdash{}{0pt}%
\pgfpathmoveto{\pgfqpoint{3.515375in}{1.555554in}}%
\pgfpathlineto{\pgfqpoint{3.564806in}{1.541145in}}%
\pgfusepath{stroke}%
\end{pgfscope}%
\begin{pgfscope}%
\definecolor{textcolor}{rgb}{0.000000,0.000000,0.000000}%
\pgfsetstrokecolor{textcolor}%
\pgfsetfillcolor{textcolor}%
\pgftext[x=3.697284in,y=1.379481in,,top]{\color{textcolor}\rmfamily\fontsize{10.000000}{12.000000}\selectfont \(\displaystyle {35}\)}%
\end{pgfscope}%
\begin{pgfscope}%
\pgfsetrectcap%
\pgfsetroundjoin%
\pgfsetlinewidth{0.803000pt}%
\definecolor{currentstroke}{rgb}{0.000000,0.000000,0.000000}%
\pgfsetstrokecolor{currentstroke}%
\pgfsetdash{}{0pt}%
\pgfpathmoveto{\pgfqpoint{3.558144in}{1.577751in}}%
\pgfpathlineto{\pgfqpoint{3.628038in}{3.104037in}}%
\pgfusepath{stroke}%
\end{pgfscope}%
\begin{pgfscope}%
\definecolor{textcolor}{rgb}{0.000000,0.000000,0.000000}%
\pgfsetstrokecolor{textcolor}%
\pgfsetfillcolor{textcolor}%
\pgftext[x=4.167903in, y=1.963517in, left, base,rotate=87.378092]{\color{textcolor}\rmfamily\fontsize{10.000000}{12.000000}\selectfont Position Z [\(\displaystyle m\)]}%
\end{pgfscope}%
\begin{pgfscope}%
\pgfsetbuttcap%
\pgfsetroundjoin%
\pgfsetlinewidth{0.803000pt}%
\definecolor{currentstroke}{rgb}{0.690196,0.690196,0.690196}%
\pgfsetstrokecolor{currentstroke}%
\pgfsetdash{}{0pt}%
\pgfpathmoveto{\pgfqpoint{3.563508in}{1.694866in}}%
\pgfpathlineto{\pgfqpoint{1.598309in}{2.260455in}}%
\pgfpathlineto{\pgfqpoint{0.373299in}{1.243854in}}%
\pgfusepath{stroke}%
\end{pgfscope}%
\begin{pgfscope}%
\pgfsetbuttcap%
\pgfsetroundjoin%
\pgfsetlinewidth{0.803000pt}%
\definecolor{currentstroke}{rgb}{0.690196,0.690196,0.690196}%
\pgfsetstrokecolor{currentstroke}%
\pgfsetdash{}{0pt}%
\pgfpathmoveto{\pgfqpoint{3.572035in}{1.881081in}}%
\pgfpathlineto{\pgfqpoint{1.596236in}{2.440745in}}%
\pgfpathlineto{\pgfqpoint{0.364114in}{1.434587in}}%
\pgfusepath{stroke}%
\end{pgfscope}%
\begin{pgfscope}%
\pgfsetbuttcap%
\pgfsetroundjoin%
\pgfsetlinewidth{0.803000pt}%
\definecolor{currentstroke}{rgb}{0.690196,0.690196,0.690196}%
\pgfsetstrokecolor{currentstroke}%
\pgfsetdash{}{0pt}%
\pgfpathmoveto{\pgfqpoint{3.580656in}{2.069335in}}%
\pgfpathlineto{\pgfqpoint{1.594141in}{2.622915in}}%
\pgfpathlineto{\pgfqpoint{0.354824in}{1.627488in}}%
\pgfusepath{stroke}%
\end{pgfscope}%
\begin{pgfscope}%
\pgfsetbuttcap%
\pgfsetroundjoin%
\pgfsetlinewidth{0.803000pt}%
\definecolor{currentstroke}{rgb}{0.690196,0.690196,0.690196}%
\pgfsetstrokecolor{currentstroke}%
\pgfsetdash{}{0pt}%
\pgfpathmoveto{\pgfqpoint{3.589371in}{2.259663in}}%
\pgfpathlineto{\pgfqpoint{1.592024in}{2.806995in}}%
\pgfpathlineto{\pgfqpoint{0.345429in}{1.822595in}}%
\pgfusepath{stroke}%
\end{pgfscope}%
\begin{pgfscope}%
\pgfsetbuttcap%
\pgfsetroundjoin%
\pgfsetlinewidth{0.803000pt}%
\definecolor{currentstroke}{rgb}{0.690196,0.690196,0.690196}%
\pgfsetstrokecolor{currentstroke}%
\pgfsetdash{}{0pt}%
\pgfpathmoveto{\pgfqpoint{3.598183in}{2.452098in}}%
\pgfpathlineto{\pgfqpoint{1.589885in}{2.993015in}}%
\pgfpathlineto{\pgfqpoint{0.335925in}{2.019947in}}%
\pgfusepath{stroke}%
\end{pgfscope}%
\begin{pgfscope}%
\pgfsetbuttcap%
\pgfsetroundjoin%
\pgfsetlinewidth{0.803000pt}%
\definecolor{currentstroke}{rgb}{0.690196,0.690196,0.690196}%
\pgfsetstrokecolor{currentstroke}%
\pgfsetdash{}{0pt}%
\pgfpathmoveto{\pgfqpoint{3.607094in}{2.646676in}}%
\pgfpathlineto{\pgfqpoint{1.587723in}{3.181005in}}%
\pgfpathlineto{\pgfqpoint{0.326311in}{2.219581in}}%
\pgfusepath{stroke}%
\end{pgfscope}%
\begin{pgfscope}%
\pgfsetbuttcap%
\pgfsetroundjoin%
\pgfsetlinewidth{0.803000pt}%
\definecolor{currentstroke}{rgb}{0.690196,0.690196,0.690196}%
\pgfsetstrokecolor{currentstroke}%
\pgfsetdash{}{0pt}%
\pgfpathmoveto{\pgfqpoint{3.616104in}{2.843434in}}%
\pgfpathlineto{\pgfqpoint{1.585539in}{3.370999in}}%
\pgfpathlineto{\pgfqpoint{0.316586in}{2.421539in}}%
\pgfusepath{stroke}%
\end{pgfscope}%
\begin{pgfscope}%
\pgfsetbuttcap%
\pgfsetroundjoin%
\pgfsetlinewidth{0.803000pt}%
\definecolor{currentstroke}{rgb}{0.690196,0.690196,0.690196}%
\pgfsetstrokecolor{currentstroke}%
\pgfsetdash{}{0pt}%
\pgfpathmoveto{\pgfqpoint{3.625215in}{3.042407in}}%
\pgfpathlineto{\pgfqpoint{1.583330in}{3.563026in}}%
\pgfpathlineto{\pgfqpoint{0.306746in}{2.625860in}}%
\pgfusepath{stroke}%
\end{pgfscope}%
\begin{pgfscope}%
\pgfsetrectcap%
\pgfsetroundjoin%
\pgfsetlinewidth{0.803000pt}%
\definecolor{currentstroke}{rgb}{0.000000,0.000000,0.000000}%
\pgfsetstrokecolor{currentstroke}%
\pgfsetdash{}{0pt}%
\pgfpathmoveto{\pgfqpoint{3.547011in}{1.699614in}}%
\pgfpathlineto{\pgfqpoint{3.596539in}{1.685360in}}%
\pgfusepath{stroke}%
\end{pgfscope}%
\begin{pgfscope}%
\definecolor{textcolor}{rgb}{0.000000,0.000000,0.000000}%
\pgfsetstrokecolor{textcolor}%
\pgfsetfillcolor{textcolor}%
\pgftext[x=3.817813in,y=1.730819in,,top]{\color{textcolor}\rmfamily\fontsize{10.000000}{12.000000}\selectfont \(\displaystyle {−5}\)}%
\end{pgfscope}%
\begin{pgfscope}%
\pgfsetrectcap%
\pgfsetroundjoin%
\pgfsetlinewidth{0.803000pt}%
\definecolor{currentstroke}{rgb}{0.000000,0.000000,0.000000}%
\pgfsetstrokecolor{currentstroke}%
\pgfsetdash{}{0pt}%
\pgfpathmoveto{\pgfqpoint{3.555446in}{1.885780in}}%
\pgfpathlineto{\pgfqpoint{3.605253in}{1.871671in}}%
\pgfusepath{stroke}%
\end{pgfscope}%
\begin{pgfscope}%
\definecolor{textcolor}{rgb}{0.000000,0.000000,0.000000}%
\pgfsetstrokecolor{textcolor}%
\pgfsetfillcolor{textcolor}%
\pgftext[x=3.827696in,y=1.916665in,,top]{\color{textcolor}\rmfamily\fontsize{10.000000}{12.000000}\selectfont \(\displaystyle {−4}\)}%
\end{pgfscope}%
\begin{pgfscope}%
\pgfsetrectcap%
\pgfsetroundjoin%
\pgfsetlinewidth{0.803000pt}%
\definecolor{currentstroke}{rgb}{0.000000,0.000000,0.000000}%
\pgfsetstrokecolor{currentstroke}%
\pgfsetdash{}{0pt}%
\pgfpathmoveto{\pgfqpoint{3.563972in}{2.073984in}}%
\pgfpathlineto{\pgfqpoint{3.614063in}{2.060025in}}%
\pgfusepath{stroke}%
\end{pgfscope}%
\begin{pgfscope}%
\definecolor{textcolor}{rgb}{0.000000,0.000000,0.000000}%
\pgfsetstrokecolor{textcolor}%
\pgfsetfillcolor{textcolor}%
\pgftext[x=3.837686in,y=2.104541in,,top]{\color{textcolor}\rmfamily\fontsize{10.000000}{12.000000}\selectfont \(\displaystyle {−3}\)}%
\end{pgfscope}%
\begin{pgfscope}%
\pgfsetrectcap%
\pgfsetroundjoin%
\pgfsetlinewidth{0.803000pt}%
\definecolor{currentstroke}{rgb}{0.000000,0.000000,0.000000}%
\pgfsetstrokecolor{currentstroke}%
\pgfsetdash{}{0pt}%
\pgfpathmoveto{\pgfqpoint{3.572592in}{2.264261in}}%
\pgfpathlineto{\pgfqpoint{3.622970in}{2.250456in}}%
\pgfusepath{stroke}%
\end{pgfscope}%
\begin{pgfscope}%
\definecolor{textcolor}{rgb}{0.000000,0.000000,0.000000}%
\pgfsetstrokecolor{textcolor}%
\pgfsetfillcolor{textcolor}%
\pgftext[x=3.847786in,y=2.294480in,,top]{\color{textcolor}\rmfamily\fontsize{10.000000}{12.000000}\selectfont \(\displaystyle {−2}\)}%
\end{pgfscope}%
\begin{pgfscope}%
\pgfsetrectcap%
\pgfsetroundjoin%
\pgfsetlinewidth{0.803000pt}%
\definecolor{currentstroke}{rgb}{0.000000,0.000000,0.000000}%
\pgfsetstrokecolor{currentstroke}%
\pgfsetdash{}{0pt}%
\pgfpathmoveto{\pgfqpoint{3.581308in}{2.456643in}}%
\pgfpathlineto{\pgfqpoint{3.631975in}{2.442996in}}%
\pgfusepath{stroke}%
\end{pgfscope}%
\begin{pgfscope}%
\definecolor{textcolor}{rgb}{0.000000,0.000000,0.000000}%
\pgfsetstrokecolor{textcolor}%
\pgfsetfillcolor{textcolor}%
\pgftext[x=3.857998in,y=2.486516in,,top]{\color{textcolor}\rmfamily\fontsize{10.000000}{12.000000}\selectfont \(\displaystyle {−1}\)}%
\end{pgfscope}%
\begin{pgfscope}%
\pgfsetrectcap%
\pgfsetroundjoin%
\pgfsetlinewidth{0.803000pt}%
\definecolor{currentstroke}{rgb}{0.000000,0.000000,0.000000}%
\pgfsetstrokecolor{currentstroke}%
\pgfsetdash{}{0pt}%
\pgfpathmoveto{\pgfqpoint{3.590121in}{2.651167in}}%
\pgfpathlineto{\pgfqpoint{3.641081in}{2.637683in}}%
\pgfusepath{stroke}%
\end{pgfscope}%
\begin{pgfscope}%
\definecolor{textcolor}{rgb}{0.000000,0.000000,0.000000}%
\pgfsetstrokecolor{textcolor}%
\pgfsetfillcolor{textcolor}%
\pgftext[x=3.868323in,y=2.680683in,,top]{\color{textcolor}\rmfamily\fontsize{10.000000}{12.000000}\selectfont \(\displaystyle {0}\)}%
\end{pgfscope}%
\begin{pgfscope}%
\pgfsetrectcap%
\pgfsetroundjoin%
\pgfsetlinewidth{0.803000pt}%
\definecolor{currentstroke}{rgb}{0.000000,0.000000,0.000000}%
\pgfsetstrokecolor{currentstroke}%
\pgfsetdash{}{0pt}%
\pgfpathmoveto{\pgfqpoint{3.599032in}{2.847869in}}%
\pgfpathlineto{\pgfqpoint{3.650289in}{2.834552in}}%
\pgfusepath{stroke}%
\end{pgfscope}%
\begin{pgfscope}%
\definecolor{textcolor}{rgb}{0.000000,0.000000,0.000000}%
\pgfsetstrokecolor{textcolor}%
\pgfsetfillcolor{textcolor}%
\pgftext[x=3.878763in,y=2.877019in,,top]{\color{textcolor}\rmfamily\fontsize{10.000000}{12.000000}\selectfont \(\displaystyle {1}\)}%
\end{pgfscope}%
\begin{pgfscope}%
\pgfsetrectcap%
\pgfsetroundjoin%
\pgfsetlinewidth{0.803000pt}%
\definecolor{currentstroke}{rgb}{0.000000,0.000000,0.000000}%
\pgfsetstrokecolor{currentstroke}%
\pgfsetdash{}{0pt}%
\pgfpathmoveto{\pgfqpoint{3.608044in}{3.046785in}}%
\pgfpathlineto{\pgfqpoint{3.659601in}{3.033640in}}%
\pgfusepath{stroke}%
\end{pgfscope}%
\begin{pgfscope}%
\definecolor{textcolor}{rgb}{0.000000,0.000000,0.000000}%
\pgfsetstrokecolor{textcolor}%
\pgfsetfillcolor{textcolor}%
\pgftext[x=3.889320in,y=3.075558in,,top]{\color{textcolor}\rmfamily\fontsize{10.000000}{12.000000}\selectfont \(\displaystyle {2}\)}%
\end{pgfscope}%
\begin{pgfscope}%
\pgfpathrectangle{\pgfqpoint{0.100000in}{0.212622in}}{\pgfqpoint{3.696000in}{3.696000in}}%
\pgfusepath{clip}%
\pgfsetrectcap%
\pgfsetroundjoin%
\pgfsetlinewidth{1.505625pt}%
\definecolor{currentstroke}{rgb}{0.121569,0.466667,0.705882}%
\pgfsetstrokecolor{currentstroke}%
\pgfsetdash{}{0pt}%
\pgfpathmoveto{\pgfqpoint{0.760001in}{2.205098in}}%
\pgfpathlineto{\pgfqpoint{1.659946in}{2.912350in}}%
\pgfpathlineto{\pgfqpoint{3.108757in}{2.519866in}}%
\pgfpathlineto{\pgfqpoint{2.272850in}{1.758106in}}%
\pgfpathlineto{\pgfqpoint{0.760001in}{2.205098in}}%
\pgfusepath{stroke}%
\end{pgfscope}%
\begin{pgfscope}%
\pgfpathrectangle{\pgfqpoint{0.100000in}{0.212622in}}{\pgfqpoint{3.696000in}{3.696000in}}%
\pgfusepath{clip}%
\pgfsetrectcap%
\pgfsetroundjoin%
\pgfsetlinewidth{1.505625pt}%
\definecolor{currentstroke}{rgb}{1.000000,0.000000,0.000000}%
\pgfsetstrokecolor{currentstroke}%
\pgfsetdash{}{0pt}%
\pgfpathmoveto{\pgfqpoint{0.760001in}{2.205098in}}%
\pgfpathlineto{\pgfqpoint{0.760001in}{2.205098in}}%
\pgfusepath{stroke}%
\end{pgfscope}%
\begin{pgfscope}%
\pgfpathrectangle{\pgfqpoint{0.100000in}{0.212622in}}{\pgfqpoint{3.696000in}{3.696000in}}%
\pgfusepath{clip}%
\pgfsetrectcap%
\pgfsetroundjoin%
\pgfsetlinewidth{1.505625pt}%
\definecolor{currentstroke}{rgb}{1.000000,0.000000,0.000000}%
\pgfsetstrokecolor{currentstroke}%
\pgfsetdash{}{0pt}%
\pgfpathmoveto{\pgfqpoint{0.760001in}{2.205098in}}%
\pgfpathlineto{\pgfqpoint{0.760001in}{2.205098in}}%
\pgfusepath{stroke}%
\end{pgfscope}%
\begin{pgfscope}%
\pgfpathrectangle{\pgfqpoint{0.100000in}{0.212622in}}{\pgfqpoint{3.696000in}{3.696000in}}%
\pgfusepath{clip}%
\pgfsetrectcap%
\pgfsetroundjoin%
\pgfsetlinewidth{1.505625pt}%
\definecolor{currentstroke}{rgb}{1.000000,0.000000,0.000000}%
\pgfsetstrokecolor{currentstroke}%
\pgfsetdash{}{0pt}%
\pgfpathmoveto{\pgfqpoint{0.760001in}{2.205098in}}%
\pgfpathlineto{\pgfqpoint{0.760001in}{2.205098in}}%
\pgfusepath{stroke}%
\end{pgfscope}%
\begin{pgfscope}%
\pgfpathrectangle{\pgfqpoint{0.100000in}{0.212622in}}{\pgfqpoint{3.696000in}{3.696000in}}%
\pgfusepath{clip}%
\pgfsetrectcap%
\pgfsetroundjoin%
\pgfsetlinewidth{1.505625pt}%
\definecolor{currentstroke}{rgb}{1.000000,0.000000,0.000000}%
\pgfsetstrokecolor{currentstroke}%
\pgfsetdash{}{0pt}%
\pgfpathmoveto{\pgfqpoint{0.760001in}{2.205098in}}%
\pgfpathlineto{\pgfqpoint{0.760001in}{2.205098in}}%
\pgfusepath{stroke}%
\end{pgfscope}%
\begin{pgfscope}%
\pgfpathrectangle{\pgfqpoint{0.100000in}{0.212622in}}{\pgfqpoint{3.696000in}{3.696000in}}%
\pgfusepath{clip}%
\pgfsetrectcap%
\pgfsetroundjoin%
\pgfsetlinewidth{1.505625pt}%
\definecolor{currentstroke}{rgb}{1.000000,0.000000,0.000000}%
\pgfsetstrokecolor{currentstroke}%
\pgfsetdash{}{0pt}%
\pgfpathmoveto{\pgfqpoint{0.758913in}{2.205452in}}%
\pgfpathlineto{\pgfqpoint{0.760001in}{2.205098in}}%
\pgfusepath{stroke}%
\end{pgfscope}%
\begin{pgfscope}%
\pgfpathrectangle{\pgfqpoint{0.100000in}{0.212622in}}{\pgfqpoint{3.696000in}{3.696000in}}%
\pgfusepath{clip}%
\pgfsetrectcap%
\pgfsetroundjoin%
\pgfsetlinewidth{1.505625pt}%
\definecolor{currentstroke}{rgb}{1.000000,0.000000,0.000000}%
\pgfsetstrokecolor{currentstroke}%
\pgfsetdash{}{0pt}%
\pgfpathmoveto{\pgfqpoint{0.757702in}{2.205882in}}%
\pgfpathlineto{\pgfqpoint{0.760001in}{2.205098in}}%
\pgfusepath{stroke}%
\end{pgfscope}%
\begin{pgfscope}%
\pgfpathrectangle{\pgfqpoint{0.100000in}{0.212622in}}{\pgfqpoint{3.696000in}{3.696000in}}%
\pgfusepath{clip}%
\pgfsetrectcap%
\pgfsetroundjoin%
\pgfsetlinewidth{1.505625pt}%
\definecolor{currentstroke}{rgb}{1.000000,0.000000,0.000000}%
\pgfsetstrokecolor{currentstroke}%
\pgfsetdash{}{0pt}%
\pgfpathmoveto{\pgfqpoint{0.756877in}{2.206303in}}%
\pgfpathlineto{\pgfqpoint{0.760001in}{2.205098in}}%
\pgfusepath{stroke}%
\end{pgfscope}%
\begin{pgfscope}%
\pgfpathrectangle{\pgfqpoint{0.100000in}{0.212622in}}{\pgfqpoint{3.696000in}{3.696000in}}%
\pgfusepath{clip}%
\pgfsetrectcap%
\pgfsetroundjoin%
\pgfsetlinewidth{1.505625pt}%
\definecolor{currentstroke}{rgb}{1.000000,0.000000,0.000000}%
\pgfsetstrokecolor{currentstroke}%
\pgfsetdash{}{0pt}%
\pgfpathmoveto{\pgfqpoint{0.756943in}{2.206393in}}%
\pgfpathlineto{\pgfqpoint{0.760001in}{2.205098in}}%
\pgfusepath{stroke}%
\end{pgfscope}%
\begin{pgfscope}%
\pgfpathrectangle{\pgfqpoint{0.100000in}{0.212622in}}{\pgfqpoint{3.696000in}{3.696000in}}%
\pgfusepath{clip}%
\pgfsetrectcap%
\pgfsetroundjoin%
\pgfsetlinewidth{1.505625pt}%
\definecolor{currentstroke}{rgb}{1.000000,0.000000,0.000000}%
\pgfsetstrokecolor{currentstroke}%
\pgfsetdash{}{0pt}%
\pgfpathmoveto{\pgfqpoint{0.757589in}{2.206433in}}%
\pgfpathlineto{\pgfqpoint{0.760001in}{2.205098in}}%
\pgfusepath{stroke}%
\end{pgfscope}%
\begin{pgfscope}%
\pgfpathrectangle{\pgfqpoint{0.100000in}{0.212622in}}{\pgfqpoint{3.696000in}{3.696000in}}%
\pgfusepath{clip}%
\pgfsetrectcap%
\pgfsetroundjoin%
\pgfsetlinewidth{1.505625pt}%
\definecolor{currentstroke}{rgb}{1.000000,0.000000,0.000000}%
\pgfsetstrokecolor{currentstroke}%
\pgfsetdash{}{0pt}%
\pgfpathmoveto{\pgfqpoint{0.758774in}{2.206601in}}%
\pgfpathlineto{\pgfqpoint{0.760001in}{2.205098in}}%
\pgfusepath{stroke}%
\end{pgfscope}%
\begin{pgfscope}%
\pgfpathrectangle{\pgfqpoint{0.100000in}{0.212622in}}{\pgfqpoint{3.696000in}{3.696000in}}%
\pgfusepath{clip}%
\pgfsetrectcap%
\pgfsetroundjoin%
\pgfsetlinewidth{1.505625pt}%
\definecolor{currentstroke}{rgb}{1.000000,0.000000,0.000000}%
\pgfsetstrokecolor{currentstroke}%
\pgfsetdash{}{0pt}%
\pgfpathmoveto{\pgfqpoint{0.760589in}{2.206667in}}%
\pgfpathlineto{\pgfqpoint{0.769684in}{2.212707in}}%
\pgfusepath{stroke}%
\end{pgfscope}%
\begin{pgfscope}%
\pgfpathrectangle{\pgfqpoint{0.100000in}{0.212622in}}{\pgfqpoint{3.696000in}{3.696000in}}%
\pgfusepath{clip}%
\pgfsetrectcap%
\pgfsetroundjoin%
\pgfsetlinewidth{1.505625pt}%
\definecolor{currentstroke}{rgb}{1.000000,0.000000,0.000000}%
\pgfsetstrokecolor{currentstroke}%
\pgfsetdash{}{0pt}%
\pgfpathmoveto{\pgfqpoint{0.762785in}{2.206702in}}%
\pgfpathlineto{\pgfqpoint{0.769684in}{2.212707in}}%
\pgfusepath{stroke}%
\end{pgfscope}%
\begin{pgfscope}%
\pgfpathrectangle{\pgfqpoint{0.100000in}{0.212622in}}{\pgfqpoint{3.696000in}{3.696000in}}%
\pgfusepath{clip}%
\pgfsetrectcap%
\pgfsetroundjoin%
\pgfsetlinewidth{1.505625pt}%
\definecolor{currentstroke}{rgb}{1.000000,0.000000,0.000000}%
\pgfsetstrokecolor{currentstroke}%
\pgfsetdash{}{0pt}%
\pgfpathmoveto{\pgfqpoint{0.765461in}{2.206924in}}%
\pgfpathlineto{\pgfqpoint{0.769684in}{2.212707in}}%
\pgfusepath{stroke}%
\end{pgfscope}%
\begin{pgfscope}%
\pgfpathrectangle{\pgfqpoint{0.100000in}{0.212622in}}{\pgfqpoint{3.696000in}{3.696000in}}%
\pgfusepath{clip}%
\pgfsetrectcap%
\pgfsetroundjoin%
\pgfsetlinewidth{1.505625pt}%
\definecolor{currentstroke}{rgb}{1.000000,0.000000,0.000000}%
\pgfsetstrokecolor{currentstroke}%
\pgfsetdash{}{0pt}%
\pgfpathmoveto{\pgfqpoint{0.768091in}{2.206970in}}%
\pgfpathlineto{\pgfqpoint{0.779354in}{2.220307in}}%
\pgfusepath{stroke}%
\end{pgfscope}%
\begin{pgfscope}%
\pgfpathrectangle{\pgfqpoint{0.100000in}{0.212622in}}{\pgfqpoint{3.696000in}{3.696000in}}%
\pgfusepath{clip}%
\pgfsetrectcap%
\pgfsetroundjoin%
\pgfsetlinewidth{1.505625pt}%
\definecolor{currentstroke}{rgb}{1.000000,0.000000,0.000000}%
\pgfsetstrokecolor{currentstroke}%
\pgfsetdash{}{0pt}%
\pgfpathmoveto{\pgfqpoint{0.769784in}{2.207003in}}%
\pgfpathlineto{\pgfqpoint{0.779354in}{2.220307in}}%
\pgfusepath{stroke}%
\end{pgfscope}%
\begin{pgfscope}%
\pgfpathrectangle{\pgfqpoint{0.100000in}{0.212622in}}{\pgfqpoint{3.696000in}{3.696000in}}%
\pgfusepath{clip}%
\pgfsetrectcap%
\pgfsetroundjoin%
\pgfsetlinewidth{1.505625pt}%
\definecolor{currentstroke}{rgb}{1.000000,0.000000,0.000000}%
\pgfsetstrokecolor{currentstroke}%
\pgfsetdash{}{0pt}%
\pgfpathmoveto{\pgfqpoint{0.770601in}{2.207041in}}%
\pgfpathlineto{\pgfqpoint{0.779354in}{2.220307in}}%
\pgfusepath{stroke}%
\end{pgfscope}%
\begin{pgfscope}%
\pgfpathrectangle{\pgfqpoint{0.100000in}{0.212622in}}{\pgfqpoint{3.696000in}{3.696000in}}%
\pgfusepath{clip}%
\pgfsetrectcap%
\pgfsetroundjoin%
\pgfsetlinewidth{1.505625pt}%
\definecolor{currentstroke}{rgb}{1.000000,0.000000,0.000000}%
\pgfsetstrokecolor{currentstroke}%
\pgfsetdash{}{0pt}%
\pgfpathmoveto{\pgfqpoint{0.771078in}{2.207036in}}%
\pgfpathlineto{\pgfqpoint{0.779354in}{2.220307in}}%
\pgfusepath{stroke}%
\end{pgfscope}%
\begin{pgfscope}%
\pgfpathrectangle{\pgfqpoint{0.100000in}{0.212622in}}{\pgfqpoint{3.696000in}{3.696000in}}%
\pgfusepath{clip}%
\pgfsetrectcap%
\pgfsetroundjoin%
\pgfsetlinewidth{1.505625pt}%
\definecolor{currentstroke}{rgb}{1.000000,0.000000,0.000000}%
\pgfsetstrokecolor{currentstroke}%
\pgfsetdash{}{0pt}%
\pgfpathmoveto{\pgfqpoint{0.771843in}{2.207002in}}%
\pgfpathlineto{\pgfqpoint{0.779354in}{2.220307in}}%
\pgfusepath{stroke}%
\end{pgfscope}%
\begin{pgfscope}%
\pgfpathrectangle{\pgfqpoint{0.100000in}{0.212622in}}{\pgfqpoint{3.696000in}{3.696000in}}%
\pgfusepath{clip}%
\pgfsetrectcap%
\pgfsetroundjoin%
\pgfsetlinewidth{1.505625pt}%
\definecolor{currentstroke}{rgb}{1.000000,0.000000,0.000000}%
\pgfsetstrokecolor{currentstroke}%
\pgfsetdash{}{0pt}%
\pgfpathmoveto{\pgfqpoint{0.772843in}{2.206941in}}%
\pgfpathlineto{\pgfqpoint{0.779354in}{2.220307in}}%
\pgfusepath{stroke}%
\end{pgfscope}%
\begin{pgfscope}%
\pgfpathrectangle{\pgfqpoint{0.100000in}{0.212622in}}{\pgfqpoint{3.696000in}{3.696000in}}%
\pgfusepath{clip}%
\pgfsetrectcap%
\pgfsetroundjoin%
\pgfsetlinewidth{1.505625pt}%
\definecolor{currentstroke}{rgb}{1.000000,0.000000,0.000000}%
\pgfsetstrokecolor{currentstroke}%
\pgfsetdash{}{0pt}%
\pgfpathmoveto{\pgfqpoint{0.773372in}{2.206908in}}%
\pgfpathlineto{\pgfqpoint{0.779354in}{2.220307in}}%
\pgfusepath{stroke}%
\end{pgfscope}%
\begin{pgfscope}%
\pgfpathrectangle{\pgfqpoint{0.100000in}{0.212622in}}{\pgfqpoint{3.696000in}{3.696000in}}%
\pgfusepath{clip}%
\pgfsetrectcap%
\pgfsetroundjoin%
\pgfsetlinewidth{1.505625pt}%
\definecolor{currentstroke}{rgb}{1.000000,0.000000,0.000000}%
\pgfsetstrokecolor{currentstroke}%
\pgfsetdash{}{0pt}%
\pgfpathmoveto{\pgfqpoint{0.773669in}{2.206893in}}%
\pgfpathlineto{\pgfqpoint{0.779354in}{2.220307in}}%
\pgfusepath{stroke}%
\end{pgfscope}%
\begin{pgfscope}%
\pgfpathrectangle{\pgfqpoint{0.100000in}{0.212622in}}{\pgfqpoint{3.696000in}{3.696000in}}%
\pgfusepath{clip}%
\pgfsetrectcap%
\pgfsetroundjoin%
\pgfsetlinewidth{1.505625pt}%
\definecolor{currentstroke}{rgb}{1.000000,0.000000,0.000000}%
\pgfsetstrokecolor{currentstroke}%
\pgfsetdash{}{0pt}%
\pgfpathmoveto{\pgfqpoint{0.773843in}{2.206878in}}%
\pgfpathlineto{\pgfqpoint{0.779354in}{2.220307in}}%
\pgfusepath{stroke}%
\end{pgfscope}%
\begin{pgfscope}%
\pgfpathrectangle{\pgfqpoint{0.100000in}{0.212622in}}{\pgfqpoint{3.696000in}{3.696000in}}%
\pgfusepath{clip}%
\pgfsetrectcap%
\pgfsetroundjoin%
\pgfsetlinewidth{1.505625pt}%
\definecolor{currentstroke}{rgb}{1.000000,0.000000,0.000000}%
\pgfsetstrokecolor{currentstroke}%
\pgfsetdash{}{0pt}%
\pgfpathmoveto{\pgfqpoint{0.774216in}{2.206857in}}%
\pgfpathlineto{\pgfqpoint{0.779354in}{2.220307in}}%
\pgfusepath{stroke}%
\end{pgfscope}%
\begin{pgfscope}%
\pgfpathrectangle{\pgfqpoint{0.100000in}{0.212622in}}{\pgfqpoint{3.696000in}{3.696000in}}%
\pgfusepath{clip}%
\pgfsetrectcap%
\pgfsetroundjoin%
\pgfsetlinewidth{1.505625pt}%
\definecolor{currentstroke}{rgb}{1.000000,0.000000,0.000000}%
\pgfsetstrokecolor{currentstroke}%
\pgfsetdash{}{0pt}%
\pgfpathmoveto{\pgfqpoint{0.774405in}{2.206852in}}%
\pgfpathlineto{\pgfqpoint{0.779354in}{2.220307in}}%
\pgfusepath{stroke}%
\end{pgfscope}%
\begin{pgfscope}%
\pgfpathrectangle{\pgfqpoint{0.100000in}{0.212622in}}{\pgfqpoint{3.696000in}{3.696000in}}%
\pgfusepath{clip}%
\pgfsetrectcap%
\pgfsetroundjoin%
\pgfsetlinewidth{1.505625pt}%
\definecolor{currentstroke}{rgb}{1.000000,0.000000,0.000000}%
\pgfsetstrokecolor{currentstroke}%
\pgfsetdash{}{0pt}%
\pgfpathmoveto{\pgfqpoint{0.774514in}{2.206847in}}%
\pgfpathlineto{\pgfqpoint{0.779354in}{2.220307in}}%
\pgfusepath{stroke}%
\end{pgfscope}%
\begin{pgfscope}%
\pgfpathrectangle{\pgfqpoint{0.100000in}{0.212622in}}{\pgfqpoint{3.696000in}{3.696000in}}%
\pgfusepath{clip}%
\pgfsetrectcap%
\pgfsetroundjoin%
\pgfsetlinewidth{1.505625pt}%
\definecolor{currentstroke}{rgb}{1.000000,0.000000,0.000000}%
\pgfsetstrokecolor{currentstroke}%
\pgfsetdash{}{0pt}%
\pgfpathmoveto{\pgfqpoint{0.774565in}{2.206848in}}%
\pgfpathlineto{\pgfqpoint{0.779354in}{2.220307in}}%
\pgfusepath{stroke}%
\end{pgfscope}%
\begin{pgfscope}%
\pgfpathrectangle{\pgfqpoint{0.100000in}{0.212622in}}{\pgfqpoint{3.696000in}{3.696000in}}%
\pgfusepath{clip}%
\pgfsetrectcap%
\pgfsetroundjoin%
\pgfsetlinewidth{1.505625pt}%
\definecolor{currentstroke}{rgb}{1.000000,0.000000,0.000000}%
\pgfsetstrokecolor{currentstroke}%
\pgfsetdash{}{0pt}%
\pgfpathmoveto{\pgfqpoint{0.774597in}{2.206846in}}%
\pgfpathlineto{\pgfqpoint{0.779354in}{2.220307in}}%
\pgfusepath{stroke}%
\end{pgfscope}%
\begin{pgfscope}%
\pgfpathrectangle{\pgfqpoint{0.100000in}{0.212622in}}{\pgfqpoint{3.696000in}{3.696000in}}%
\pgfusepath{clip}%
\pgfsetrectcap%
\pgfsetroundjoin%
\pgfsetlinewidth{1.505625pt}%
\definecolor{currentstroke}{rgb}{1.000000,0.000000,0.000000}%
\pgfsetstrokecolor{currentstroke}%
\pgfsetdash{}{0pt}%
\pgfpathmoveto{\pgfqpoint{0.774615in}{2.206845in}}%
\pgfpathlineto{\pgfqpoint{0.779354in}{2.220307in}}%
\pgfusepath{stroke}%
\end{pgfscope}%
\begin{pgfscope}%
\pgfpathrectangle{\pgfqpoint{0.100000in}{0.212622in}}{\pgfqpoint{3.696000in}{3.696000in}}%
\pgfusepath{clip}%
\pgfsetrectcap%
\pgfsetroundjoin%
\pgfsetlinewidth{1.505625pt}%
\definecolor{currentstroke}{rgb}{1.000000,0.000000,0.000000}%
\pgfsetstrokecolor{currentstroke}%
\pgfsetdash{}{0pt}%
\pgfpathmoveto{\pgfqpoint{0.774625in}{2.206845in}}%
\pgfpathlineto{\pgfqpoint{0.779354in}{2.220307in}}%
\pgfusepath{stroke}%
\end{pgfscope}%
\begin{pgfscope}%
\pgfpathrectangle{\pgfqpoint{0.100000in}{0.212622in}}{\pgfqpoint{3.696000in}{3.696000in}}%
\pgfusepath{clip}%
\pgfsetrectcap%
\pgfsetroundjoin%
\pgfsetlinewidth{1.505625pt}%
\definecolor{currentstroke}{rgb}{1.000000,0.000000,0.000000}%
\pgfsetstrokecolor{currentstroke}%
\pgfsetdash{}{0pt}%
\pgfpathmoveto{\pgfqpoint{0.774630in}{2.206845in}}%
\pgfpathlineto{\pgfqpoint{0.779354in}{2.220307in}}%
\pgfusepath{stroke}%
\end{pgfscope}%
\begin{pgfscope}%
\pgfpathrectangle{\pgfqpoint{0.100000in}{0.212622in}}{\pgfqpoint{3.696000in}{3.696000in}}%
\pgfusepath{clip}%
\pgfsetrectcap%
\pgfsetroundjoin%
\pgfsetlinewidth{1.505625pt}%
\definecolor{currentstroke}{rgb}{1.000000,0.000000,0.000000}%
\pgfsetstrokecolor{currentstroke}%
\pgfsetdash{}{0pt}%
\pgfpathmoveto{\pgfqpoint{0.774633in}{2.206845in}}%
\pgfpathlineto{\pgfqpoint{0.779354in}{2.220307in}}%
\pgfusepath{stroke}%
\end{pgfscope}%
\begin{pgfscope}%
\pgfpathrectangle{\pgfqpoint{0.100000in}{0.212622in}}{\pgfqpoint{3.696000in}{3.696000in}}%
\pgfusepath{clip}%
\pgfsetrectcap%
\pgfsetroundjoin%
\pgfsetlinewidth{1.505625pt}%
\definecolor{currentstroke}{rgb}{1.000000,0.000000,0.000000}%
\pgfsetstrokecolor{currentstroke}%
\pgfsetdash{}{0pt}%
\pgfpathmoveto{\pgfqpoint{0.774634in}{2.206845in}}%
\pgfpathlineto{\pgfqpoint{0.779354in}{2.220307in}}%
\pgfusepath{stroke}%
\end{pgfscope}%
\begin{pgfscope}%
\pgfpathrectangle{\pgfqpoint{0.100000in}{0.212622in}}{\pgfqpoint{3.696000in}{3.696000in}}%
\pgfusepath{clip}%
\pgfsetrectcap%
\pgfsetroundjoin%
\pgfsetlinewidth{1.505625pt}%
\definecolor{currentstroke}{rgb}{1.000000,0.000000,0.000000}%
\pgfsetstrokecolor{currentstroke}%
\pgfsetdash{}{0pt}%
\pgfpathmoveto{\pgfqpoint{0.774635in}{2.206845in}}%
\pgfpathlineto{\pgfqpoint{0.779354in}{2.220307in}}%
\pgfusepath{stroke}%
\end{pgfscope}%
\begin{pgfscope}%
\pgfpathrectangle{\pgfqpoint{0.100000in}{0.212622in}}{\pgfqpoint{3.696000in}{3.696000in}}%
\pgfusepath{clip}%
\pgfsetrectcap%
\pgfsetroundjoin%
\pgfsetlinewidth{1.505625pt}%
\definecolor{currentstroke}{rgb}{1.000000,0.000000,0.000000}%
\pgfsetstrokecolor{currentstroke}%
\pgfsetdash{}{0pt}%
\pgfpathmoveto{\pgfqpoint{0.774636in}{2.206845in}}%
\pgfpathlineto{\pgfqpoint{0.779354in}{2.220307in}}%
\pgfusepath{stroke}%
\end{pgfscope}%
\begin{pgfscope}%
\pgfpathrectangle{\pgfqpoint{0.100000in}{0.212622in}}{\pgfqpoint{3.696000in}{3.696000in}}%
\pgfusepath{clip}%
\pgfsetrectcap%
\pgfsetroundjoin%
\pgfsetlinewidth{1.505625pt}%
\definecolor{currentstroke}{rgb}{1.000000,0.000000,0.000000}%
\pgfsetstrokecolor{currentstroke}%
\pgfsetdash{}{0pt}%
\pgfpathmoveto{\pgfqpoint{0.774636in}{2.206844in}}%
\pgfpathlineto{\pgfqpoint{0.779354in}{2.220307in}}%
\pgfusepath{stroke}%
\end{pgfscope}%
\begin{pgfscope}%
\pgfpathrectangle{\pgfqpoint{0.100000in}{0.212622in}}{\pgfqpoint{3.696000in}{3.696000in}}%
\pgfusepath{clip}%
\pgfsetrectcap%
\pgfsetroundjoin%
\pgfsetlinewidth{1.505625pt}%
\definecolor{currentstroke}{rgb}{1.000000,0.000000,0.000000}%
\pgfsetstrokecolor{currentstroke}%
\pgfsetdash{}{0pt}%
\pgfpathmoveto{\pgfqpoint{0.774636in}{2.206844in}}%
\pgfpathlineto{\pgfqpoint{0.779354in}{2.220307in}}%
\pgfusepath{stroke}%
\end{pgfscope}%
\begin{pgfscope}%
\pgfpathrectangle{\pgfqpoint{0.100000in}{0.212622in}}{\pgfqpoint{3.696000in}{3.696000in}}%
\pgfusepath{clip}%
\pgfsetrectcap%
\pgfsetroundjoin%
\pgfsetlinewidth{1.505625pt}%
\definecolor{currentstroke}{rgb}{1.000000,0.000000,0.000000}%
\pgfsetstrokecolor{currentstroke}%
\pgfsetdash{}{0pt}%
\pgfpathmoveto{\pgfqpoint{0.774758in}{2.206821in}}%
\pgfpathlineto{\pgfqpoint{0.779354in}{2.220307in}}%
\pgfusepath{stroke}%
\end{pgfscope}%
\begin{pgfscope}%
\pgfpathrectangle{\pgfqpoint{0.100000in}{0.212622in}}{\pgfqpoint{3.696000in}{3.696000in}}%
\pgfusepath{clip}%
\pgfsetrectcap%
\pgfsetroundjoin%
\pgfsetlinewidth{1.505625pt}%
\definecolor{currentstroke}{rgb}{1.000000,0.000000,0.000000}%
\pgfsetstrokecolor{currentstroke}%
\pgfsetdash{}{0pt}%
\pgfpathmoveto{\pgfqpoint{0.776976in}{2.206398in}}%
\pgfpathlineto{\pgfqpoint{0.789011in}{2.227896in}}%
\pgfusepath{stroke}%
\end{pgfscope}%
\begin{pgfscope}%
\pgfpathrectangle{\pgfqpoint{0.100000in}{0.212622in}}{\pgfqpoint{3.696000in}{3.696000in}}%
\pgfusepath{clip}%
\pgfsetrectcap%
\pgfsetroundjoin%
\pgfsetlinewidth{1.505625pt}%
\definecolor{currentstroke}{rgb}{1.000000,0.000000,0.000000}%
\pgfsetstrokecolor{currentstroke}%
\pgfsetdash{}{0pt}%
\pgfpathmoveto{\pgfqpoint{0.778853in}{2.206254in}}%
\pgfpathlineto{\pgfqpoint{0.789011in}{2.227896in}}%
\pgfusepath{stroke}%
\end{pgfscope}%
\begin{pgfscope}%
\pgfpathrectangle{\pgfqpoint{0.100000in}{0.212622in}}{\pgfqpoint{3.696000in}{3.696000in}}%
\pgfusepath{clip}%
\pgfsetrectcap%
\pgfsetroundjoin%
\pgfsetlinewidth{1.505625pt}%
\definecolor{currentstroke}{rgb}{1.000000,0.000000,0.000000}%
\pgfsetstrokecolor{currentstroke}%
\pgfsetdash{}{0pt}%
\pgfpathmoveto{\pgfqpoint{0.783582in}{2.205203in}}%
\pgfpathlineto{\pgfqpoint{0.789011in}{2.227896in}}%
\pgfusepath{stroke}%
\end{pgfscope}%
\begin{pgfscope}%
\pgfpathrectangle{\pgfqpoint{0.100000in}{0.212622in}}{\pgfqpoint{3.696000in}{3.696000in}}%
\pgfusepath{clip}%
\pgfsetrectcap%
\pgfsetroundjoin%
\pgfsetlinewidth{1.505625pt}%
\definecolor{currentstroke}{rgb}{1.000000,0.000000,0.000000}%
\pgfsetstrokecolor{currentstroke}%
\pgfsetdash{}{0pt}%
\pgfpathmoveto{\pgfqpoint{0.786550in}{2.204878in}}%
\pgfpathlineto{\pgfqpoint{0.798655in}{2.235476in}}%
\pgfusepath{stroke}%
\end{pgfscope}%
\begin{pgfscope}%
\pgfpathrectangle{\pgfqpoint{0.100000in}{0.212622in}}{\pgfqpoint{3.696000in}{3.696000in}}%
\pgfusepath{clip}%
\pgfsetrectcap%
\pgfsetroundjoin%
\pgfsetlinewidth{1.505625pt}%
\definecolor{currentstroke}{rgb}{1.000000,0.000000,0.000000}%
\pgfsetstrokecolor{currentstroke}%
\pgfsetdash{}{0pt}%
\pgfpathmoveto{\pgfqpoint{0.793059in}{2.204026in}}%
\pgfpathlineto{\pgfqpoint{0.798655in}{2.235476in}}%
\pgfusepath{stroke}%
\end{pgfscope}%
\begin{pgfscope}%
\pgfpathrectangle{\pgfqpoint{0.100000in}{0.212622in}}{\pgfqpoint{3.696000in}{3.696000in}}%
\pgfusepath{clip}%
\pgfsetrectcap%
\pgfsetroundjoin%
\pgfsetlinewidth{1.505625pt}%
\definecolor{currentstroke}{rgb}{1.000000,0.000000,0.000000}%
\pgfsetstrokecolor{currentstroke}%
\pgfsetdash{}{0pt}%
\pgfpathmoveto{\pgfqpoint{0.798360in}{2.203890in}}%
\pgfpathlineto{\pgfqpoint{0.808287in}{2.243045in}}%
\pgfusepath{stroke}%
\end{pgfscope}%
\begin{pgfscope}%
\pgfpathrectangle{\pgfqpoint{0.100000in}{0.212622in}}{\pgfqpoint{3.696000in}{3.696000in}}%
\pgfusepath{clip}%
\pgfsetrectcap%
\pgfsetroundjoin%
\pgfsetlinewidth{1.505625pt}%
\definecolor{currentstroke}{rgb}{1.000000,0.000000,0.000000}%
\pgfsetstrokecolor{currentstroke}%
\pgfsetdash{}{0pt}%
\pgfpathmoveto{\pgfqpoint{0.804117in}{2.203527in}}%
\pgfpathlineto{\pgfqpoint{0.817906in}{2.250604in}}%
\pgfusepath{stroke}%
\end{pgfscope}%
\begin{pgfscope}%
\pgfpathrectangle{\pgfqpoint{0.100000in}{0.212622in}}{\pgfqpoint{3.696000in}{3.696000in}}%
\pgfusepath{clip}%
\pgfsetrectcap%
\pgfsetroundjoin%
\pgfsetlinewidth{1.505625pt}%
\definecolor{currentstroke}{rgb}{1.000000,0.000000,0.000000}%
\pgfsetstrokecolor{currentstroke}%
\pgfsetdash{}{0pt}%
\pgfpathmoveto{\pgfqpoint{0.811481in}{2.202864in}}%
\pgfpathlineto{\pgfqpoint{0.817906in}{2.250604in}}%
\pgfusepath{stroke}%
\end{pgfscope}%
\begin{pgfscope}%
\pgfpathrectangle{\pgfqpoint{0.100000in}{0.212622in}}{\pgfqpoint{3.696000in}{3.696000in}}%
\pgfusepath{clip}%
\pgfsetrectcap%
\pgfsetroundjoin%
\pgfsetlinewidth{1.505625pt}%
\definecolor{currentstroke}{rgb}{1.000000,0.000000,0.000000}%
\pgfsetstrokecolor{currentstroke}%
\pgfsetdash{}{0pt}%
\pgfpathmoveto{\pgfqpoint{0.816369in}{2.202435in}}%
\pgfpathlineto{\pgfqpoint{0.827512in}{2.258153in}}%
\pgfusepath{stroke}%
\end{pgfscope}%
\begin{pgfscope}%
\pgfpathrectangle{\pgfqpoint{0.100000in}{0.212622in}}{\pgfqpoint{3.696000in}{3.696000in}}%
\pgfusepath{clip}%
\pgfsetrectcap%
\pgfsetroundjoin%
\pgfsetlinewidth{1.505625pt}%
\definecolor{currentstroke}{rgb}{1.000000,0.000000,0.000000}%
\pgfsetstrokecolor{currentstroke}%
\pgfsetdash{}{0pt}%
\pgfpathmoveto{\pgfqpoint{0.820669in}{2.201949in}}%
\pgfpathlineto{\pgfqpoint{0.827512in}{2.258153in}}%
\pgfusepath{stroke}%
\end{pgfscope}%
\begin{pgfscope}%
\pgfpathrectangle{\pgfqpoint{0.100000in}{0.212622in}}{\pgfqpoint{3.696000in}{3.696000in}}%
\pgfusepath{clip}%
\pgfsetrectcap%
\pgfsetroundjoin%
\pgfsetlinewidth{1.505625pt}%
\definecolor{currentstroke}{rgb}{1.000000,0.000000,0.000000}%
\pgfsetstrokecolor{currentstroke}%
\pgfsetdash{}{0pt}%
\pgfpathmoveto{\pgfqpoint{0.823984in}{2.201849in}}%
\pgfpathlineto{\pgfqpoint{0.837105in}{2.265693in}}%
\pgfusepath{stroke}%
\end{pgfscope}%
\begin{pgfscope}%
\pgfpathrectangle{\pgfqpoint{0.100000in}{0.212622in}}{\pgfqpoint{3.696000in}{3.696000in}}%
\pgfusepath{clip}%
\pgfsetrectcap%
\pgfsetroundjoin%
\pgfsetlinewidth{1.505625pt}%
\definecolor{currentstroke}{rgb}{1.000000,0.000000,0.000000}%
\pgfsetstrokecolor{currentstroke}%
\pgfsetdash{}{0pt}%
\pgfpathmoveto{\pgfqpoint{0.829603in}{2.201270in}}%
\pgfpathlineto{\pgfqpoint{0.837105in}{2.265693in}}%
\pgfusepath{stroke}%
\end{pgfscope}%
\begin{pgfscope}%
\pgfpathrectangle{\pgfqpoint{0.100000in}{0.212622in}}{\pgfqpoint{3.696000in}{3.696000in}}%
\pgfusepath{clip}%
\pgfsetrectcap%
\pgfsetroundjoin%
\pgfsetlinewidth{1.505625pt}%
\definecolor{currentstroke}{rgb}{1.000000,0.000000,0.000000}%
\pgfsetstrokecolor{currentstroke}%
\pgfsetdash{}{0pt}%
\pgfpathmoveto{\pgfqpoint{0.831368in}{2.201117in}}%
\pgfpathlineto{\pgfqpoint{0.846686in}{2.273222in}}%
\pgfusepath{stroke}%
\end{pgfscope}%
\begin{pgfscope}%
\pgfpathrectangle{\pgfqpoint{0.100000in}{0.212622in}}{\pgfqpoint{3.696000in}{3.696000in}}%
\pgfusepath{clip}%
\pgfsetrectcap%
\pgfsetroundjoin%
\pgfsetlinewidth{1.505625pt}%
\definecolor{currentstroke}{rgb}{1.000000,0.000000,0.000000}%
\pgfsetstrokecolor{currentstroke}%
\pgfsetdash{}{0pt}%
\pgfpathmoveto{\pgfqpoint{0.832921in}{2.200970in}}%
\pgfpathlineto{\pgfqpoint{0.846686in}{2.273222in}}%
\pgfusepath{stroke}%
\end{pgfscope}%
\begin{pgfscope}%
\pgfpathrectangle{\pgfqpoint{0.100000in}{0.212622in}}{\pgfqpoint{3.696000in}{3.696000in}}%
\pgfusepath{clip}%
\pgfsetrectcap%
\pgfsetroundjoin%
\pgfsetlinewidth{1.505625pt}%
\definecolor{currentstroke}{rgb}{1.000000,0.000000,0.000000}%
\pgfsetstrokecolor{currentstroke}%
\pgfsetdash{}{0pt}%
\pgfpathmoveto{\pgfqpoint{0.835044in}{2.201003in}}%
\pgfpathlineto{\pgfqpoint{0.846686in}{2.273222in}}%
\pgfusepath{stroke}%
\end{pgfscope}%
\begin{pgfscope}%
\pgfpathrectangle{\pgfqpoint{0.100000in}{0.212622in}}{\pgfqpoint{3.696000in}{3.696000in}}%
\pgfusepath{clip}%
\pgfsetrectcap%
\pgfsetroundjoin%
\pgfsetlinewidth{1.505625pt}%
\definecolor{currentstroke}{rgb}{1.000000,0.000000,0.000000}%
\pgfsetstrokecolor{currentstroke}%
\pgfsetdash{}{0pt}%
\pgfpathmoveto{\pgfqpoint{0.837742in}{2.201006in}}%
\pgfpathlineto{\pgfqpoint{0.846686in}{2.273222in}}%
\pgfusepath{stroke}%
\end{pgfscope}%
\begin{pgfscope}%
\pgfpathrectangle{\pgfqpoint{0.100000in}{0.212622in}}{\pgfqpoint{3.696000in}{3.696000in}}%
\pgfusepath{clip}%
\pgfsetrectcap%
\pgfsetroundjoin%
\pgfsetlinewidth{1.505625pt}%
\definecolor{currentstroke}{rgb}{1.000000,0.000000,0.000000}%
\pgfsetstrokecolor{currentstroke}%
\pgfsetdash{}{0pt}%
\pgfpathmoveto{\pgfqpoint{0.839242in}{2.200943in}}%
\pgfpathlineto{\pgfqpoint{0.846686in}{2.273222in}}%
\pgfusepath{stroke}%
\end{pgfscope}%
\begin{pgfscope}%
\pgfpathrectangle{\pgfqpoint{0.100000in}{0.212622in}}{\pgfqpoint{3.696000in}{3.696000in}}%
\pgfusepath{clip}%
\pgfsetrectcap%
\pgfsetroundjoin%
\pgfsetlinewidth{1.505625pt}%
\definecolor{currentstroke}{rgb}{1.000000,0.000000,0.000000}%
\pgfsetstrokecolor{currentstroke}%
\pgfsetdash{}{0pt}%
\pgfpathmoveto{\pgfqpoint{0.839521in}{2.200673in}}%
\pgfpathlineto{\pgfqpoint{0.856254in}{2.280741in}}%
\pgfusepath{stroke}%
\end{pgfscope}%
\begin{pgfscope}%
\pgfpathrectangle{\pgfqpoint{0.100000in}{0.212622in}}{\pgfqpoint{3.696000in}{3.696000in}}%
\pgfusepath{clip}%
\pgfsetrectcap%
\pgfsetroundjoin%
\pgfsetlinewidth{1.505625pt}%
\definecolor{currentstroke}{rgb}{1.000000,0.000000,0.000000}%
\pgfsetstrokecolor{currentstroke}%
\pgfsetdash{}{0pt}%
\pgfpathmoveto{\pgfqpoint{0.842755in}{2.200402in}}%
\pgfpathlineto{\pgfqpoint{0.856254in}{2.280741in}}%
\pgfusepath{stroke}%
\end{pgfscope}%
\begin{pgfscope}%
\pgfpathrectangle{\pgfqpoint{0.100000in}{0.212622in}}{\pgfqpoint{3.696000in}{3.696000in}}%
\pgfusepath{clip}%
\pgfsetrectcap%
\pgfsetroundjoin%
\pgfsetlinewidth{1.505625pt}%
\definecolor{currentstroke}{rgb}{1.000000,0.000000,0.000000}%
\pgfsetstrokecolor{currentstroke}%
\pgfsetdash{}{0pt}%
\pgfpathmoveto{\pgfqpoint{0.845670in}{2.200266in}}%
\pgfpathlineto{\pgfqpoint{0.856254in}{2.280741in}}%
\pgfusepath{stroke}%
\end{pgfscope}%
\begin{pgfscope}%
\pgfpathrectangle{\pgfqpoint{0.100000in}{0.212622in}}{\pgfqpoint{3.696000in}{3.696000in}}%
\pgfusepath{clip}%
\pgfsetrectcap%
\pgfsetroundjoin%
\pgfsetlinewidth{1.505625pt}%
\definecolor{currentstroke}{rgb}{1.000000,0.000000,0.000000}%
\pgfsetstrokecolor{currentstroke}%
\pgfsetdash{}{0pt}%
\pgfpathmoveto{\pgfqpoint{0.849687in}{2.200265in}}%
\pgfpathlineto{\pgfqpoint{0.865809in}{2.288251in}}%
\pgfusepath{stroke}%
\end{pgfscope}%
\begin{pgfscope}%
\pgfpathrectangle{\pgfqpoint{0.100000in}{0.212622in}}{\pgfqpoint{3.696000in}{3.696000in}}%
\pgfusepath{clip}%
\pgfsetrectcap%
\pgfsetroundjoin%
\pgfsetlinewidth{1.505625pt}%
\definecolor{currentstroke}{rgb}{1.000000,0.000000,0.000000}%
\pgfsetstrokecolor{currentstroke}%
\pgfsetdash{}{0pt}%
\pgfpathmoveto{\pgfqpoint{0.852005in}{2.200096in}}%
\pgfpathlineto{\pgfqpoint{0.865809in}{2.288251in}}%
\pgfusepath{stroke}%
\end{pgfscope}%
\begin{pgfscope}%
\pgfpathrectangle{\pgfqpoint{0.100000in}{0.212622in}}{\pgfqpoint{3.696000in}{3.696000in}}%
\pgfusepath{clip}%
\pgfsetrectcap%
\pgfsetroundjoin%
\pgfsetlinewidth{1.505625pt}%
\definecolor{currentstroke}{rgb}{1.000000,0.000000,0.000000}%
\pgfsetstrokecolor{currentstroke}%
\pgfsetdash{}{0pt}%
\pgfpathmoveto{\pgfqpoint{0.853097in}{2.199808in}}%
\pgfpathlineto{\pgfqpoint{0.865809in}{2.288251in}}%
\pgfusepath{stroke}%
\end{pgfscope}%
\begin{pgfscope}%
\pgfpathrectangle{\pgfqpoint{0.100000in}{0.212622in}}{\pgfqpoint{3.696000in}{3.696000in}}%
\pgfusepath{clip}%
\pgfsetrectcap%
\pgfsetroundjoin%
\pgfsetlinewidth{1.505625pt}%
\definecolor{currentstroke}{rgb}{1.000000,0.000000,0.000000}%
\pgfsetstrokecolor{currentstroke}%
\pgfsetdash{}{0pt}%
\pgfpathmoveto{\pgfqpoint{0.857202in}{2.199236in}}%
\pgfpathlineto{\pgfqpoint{0.875352in}{2.295750in}}%
\pgfusepath{stroke}%
\end{pgfscope}%
\begin{pgfscope}%
\pgfpathrectangle{\pgfqpoint{0.100000in}{0.212622in}}{\pgfqpoint{3.696000in}{3.696000in}}%
\pgfusepath{clip}%
\pgfsetrectcap%
\pgfsetroundjoin%
\pgfsetlinewidth{1.505625pt}%
\definecolor{currentstroke}{rgb}{1.000000,0.000000,0.000000}%
\pgfsetstrokecolor{currentstroke}%
\pgfsetdash{}{0pt}%
\pgfpathmoveto{\pgfqpoint{0.860095in}{2.198989in}}%
\pgfpathlineto{\pgfqpoint{0.875352in}{2.295750in}}%
\pgfusepath{stroke}%
\end{pgfscope}%
\begin{pgfscope}%
\pgfpathrectangle{\pgfqpoint{0.100000in}{0.212622in}}{\pgfqpoint{3.696000in}{3.696000in}}%
\pgfusepath{clip}%
\pgfsetrectcap%
\pgfsetroundjoin%
\pgfsetlinewidth{1.505625pt}%
\definecolor{currentstroke}{rgb}{1.000000,0.000000,0.000000}%
\pgfsetstrokecolor{currentstroke}%
\pgfsetdash{}{0pt}%
\pgfpathmoveto{\pgfqpoint{0.865055in}{2.198502in}}%
\pgfpathlineto{\pgfqpoint{0.884882in}{2.303240in}}%
\pgfusepath{stroke}%
\end{pgfscope}%
\begin{pgfscope}%
\pgfpathrectangle{\pgfqpoint{0.100000in}{0.212622in}}{\pgfqpoint{3.696000in}{3.696000in}}%
\pgfusepath{clip}%
\pgfsetrectcap%
\pgfsetroundjoin%
\pgfsetlinewidth{1.505625pt}%
\definecolor{currentstroke}{rgb}{1.000000,0.000000,0.000000}%
\pgfsetstrokecolor{currentstroke}%
\pgfsetdash{}{0pt}%
\pgfpathmoveto{\pgfqpoint{0.867546in}{2.198332in}}%
\pgfpathlineto{\pgfqpoint{0.884882in}{2.303240in}}%
\pgfusepath{stroke}%
\end{pgfscope}%
\begin{pgfscope}%
\pgfpathrectangle{\pgfqpoint{0.100000in}{0.212622in}}{\pgfqpoint{3.696000in}{3.696000in}}%
\pgfusepath{clip}%
\pgfsetrectcap%
\pgfsetroundjoin%
\pgfsetlinewidth{1.505625pt}%
\definecolor{currentstroke}{rgb}{1.000000,0.000000,0.000000}%
\pgfsetstrokecolor{currentstroke}%
\pgfsetdash{}{0pt}%
\pgfpathmoveto{\pgfqpoint{0.869665in}{2.198205in}}%
\pgfpathlineto{\pgfqpoint{0.884882in}{2.303240in}}%
\pgfusepath{stroke}%
\end{pgfscope}%
\begin{pgfscope}%
\pgfpathrectangle{\pgfqpoint{0.100000in}{0.212622in}}{\pgfqpoint{3.696000in}{3.696000in}}%
\pgfusepath{clip}%
\pgfsetrectcap%
\pgfsetroundjoin%
\pgfsetlinewidth{1.505625pt}%
\definecolor{currentstroke}{rgb}{1.000000,0.000000,0.000000}%
\pgfsetstrokecolor{currentstroke}%
\pgfsetdash{}{0pt}%
\pgfpathmoveto{\pgfqpoint{0.871411in}{2.197991in}}%
\pgfpathlineto{\pgfqpoint{0.884882in}{2.303240in}}%
\pgfusepath{stroke}%
\end{pgfscope}%
\begin{pgfscope}%
\pgfpathrectangle{\pgfqpoint{0.100000in}{0.212622in}}{\pgfqpoint{3.696000in}{3.696000in}}%
\pgfusepath{clip}%
\pgfsetrectcap%
\pgfsetroundjoin%
\pgfsetlinewidth{1.505625pt}%
\definecolor{currentstroke}{rgb}{1.000000,0.000000,0.000000}%
\pgfsetstrokecolor{currentstroke}%
\pgfsetdash{}{0pt}%
\pgfpathmoveto{\pgfqpoint{0.871961in}{2.197964in}}%
\pgfpathlineto{\pgfqpoint{0.884882in}{2.303240in}}%
\pgfusepath{stroke}%
\end{pgfscope}%
\begin{pgfscope}%
\pgfpathrectangle{\pgfqpoint{0.100000in}{0.212622in}}{\pgfqpoint{3.696000in}{3.696000in}}%
\pgfusepath{clip}%
\pgfsetrectcap%
\pgfsetroundjoin%
\pgfsetlinewidth{1.505625pt}%
\definecolor{currentstroke}{rgb}{1.000000,0.000000,0.000000}%
\pgfsetstrokecolor{currentstroke}%
\pgfsetdash{}{0pt}%
\pgfpathmoveto{\pgfqpoint{0.873378in}{2.197835in}}%
\pgfpathlineto{\pgfqpoint{0.884882in}{2.303240in}}%
\pgfusepath{stroke}%
\end{pgfscope}%
\begin{pgfscope}%
\pgfpathrectangle{\pgfqpoint{0.100000in}{0.212622in}}{\pgfqpoint{3.696000in}{3.696000in}}%
\pgfusepath{clip}%
\pgfsetrectcap%
\pgfsetroundjoin%
\pgfsetlinewidth{1.505625pt}%
\definecolor{currentstroke}{rgb}{1.000000,0.000000,0.000000}%
\pgfsetstrokecolor{currentstroke}%
\pgfsetdash{}{0pt}%
\pgfpathmoveto{\pgfqpoint{0.875353in}{2.197786in}}%
\pgfpathlineto{\pgfqpoint{0.894400in}{2.310719in}}%
\pgfusepath{stroke}%
\end{pgfscope}%
\begin{pgfscope}%
\pgfpathrectangle{\pgfqpoint{0.100000in}{0.212622in}}{\pgfqpoint{3.696000in}{3.696000in}}%
\pgfusepath{clip}%
\pgfsetrectcap%
\pgfsetroundjoin%
\pgfsetlinewidth{1.505625pt}%
\definecolor{currentstroke}{rgb}{1.000000,0.000000,0.000000}%
\pgfsetstrokecolor{currentstroke}%
\pgfsetdash{}{0pt}%
\pgfpathmoveto{\pgfqpoint{0.877896in}{2.197801in}}%
\pgfpathlineto{\pgfqpoint{0.894400in}{2.310719in}}%
\pgfusepath{stroke}%
\end{pgfscope}%
\begin{pgfscope}%
\pgfpathrectangle{\pgfqpoint{0.100000in}{0.212622in}}{\pgfqpoint{3.696000in}{3.696000in}}%
\pgfusepath{clip}%
\pgfsetrectcap%
\pgfsetroundjoin%
\pgfsetlinewidth{1.505625pt}%
\definecolor{currentstroke}{rgb}{1.000000,0.000000,0.000000}%
\pgfsetstrokecolor{currentstroke}%
\pgfsetdash{}{0pt}%
\pgfpathmoveto{\pgfqpoint{0.879586in}{2.197661in}}%
\pgfpathlineto{\pgfqpoint{0.894400in}{2.310719in}}%
\pgfusepath{stroke}%
\end{pgfscope}%
\begin{pgfscope}%
\pgfpathrectangle{\pgfqpoint{0.100000in}{0.212622in}}{\pgfqpoint{3.696000in}{3.696000in}}%
\pgfusepath{clip}%
\pgfsetrectcap%
\pgfsetroundjoin%
\pgfsetlinewidth{1.505625pt}%
\definecolor{currentstroke}{rgb}{1.000000,0.000000,0.000000}%
\pgfsetstrokecolor{currentstroke}%
\pgfsetdash{}{0pt}%
\pgfpathmoveto{\pgfqpoint{0.877916in}{2.196835in}}%
\pgfpathlineto{\pgfqpoint{0.894400in}{2.310719in}}%
\pgfusepath{stroke}%
\end{pgfscope}%
\begin{pgfscope}%
\pgfpathrectangle{\pgfqpoint{0.100000in}{0.212622in}}{\pgfqpoint{3.696000in}{3.696000in}}%
\pgfusepath{clip}%
\pgfsetrectcap%
\pgfsetroundjoin%
\pgfsetlinewidth{1.505625pt}%
\definecolor{currentstroke}{rgb}{1.000000,0.000000,0.000000}%
\pgfsetstrokecolor{currentstroke}%
\pgfsetdash{}{0pt}%
\pgfpathmoveto{\pgfqpoint{0.881719in}{2.196309in}}%
\pgfpathlineto{\pgfqpoint{0.903905in}{2.318189in}}%
\pgfusepath{stroke}%
\end{pgfscope}%
\begin{pgfscope}%
\pgfpathrectangle{\pgfqpoint{0.100000in}{0.212622in}}{\pgfqpoint{3.696000in}{3.696000in}}%
\pgfusepath{clip}%
\pgfsetrectcap%
\pgfsetroundjoin%
\pgfsetlinewidth{1.505625pt}%
\definecolor{currentstroke}{rgb}{1.000000,0.000000,0.000000}%
\pgfsetstrokecolor{currentstroke}%
\pgfsetdash{}{0pt}%
\pgfpathmoveto{\pgfqpoint{0.884462in}{2.195987in}}%
\pgfpathlineto{\pgfqpoint{0.903905in}{2.318189in}}%
\pgfusepath{stroke}%
\end{pgfscope}%
\begin{pgfscope}%
\pgfpathrectangle{\pgfqpoint{0.100000in}{0.212622in}}{\pgfqpoint{3.696000in}{3.696000in}}%
\pgfusepath{clip}%
\pgfsetrectcap%
\pgfsetroundjoin%
\pgfsetlinewidth{1.505625pt}%
\definecolor{currentstroke}{rgb}{1.000000,0.000000,0.000000}%
\pgfsetstrokecolor{currentstroke}%
\pgfsetdash{}{0pt}%
\pgfpathmoveto{\pgfqpoint{0.889231in}{2.195497in}}%
\pgfpathlineto{\pgfqpoint{0.903905in}{2.318189in}}%
\pgfusepath{stroke}%
\end{pgfscope}%
\begin{pgfscope}%
\pgfpathrectangle{\pgfqpoint{0.100000in}{0.212622in}}{\pgfqpoint{3.696000in}{3.696000in}}%
\pgfusepath{clip}%
\pgfsetrectcap%
\pgfsetroundjoin%
\pgfsetlinewidth{1.505625pt}%
\definecolor{currentstroke}{rgb}{1.000000,0.000000,0.000000}%
\pgfsetstrokecolor{currentstroke}%
\pgfsetdash{}{0pt}%
\pgfpathmoveto{\pgfqpoint{0.891647in}{2.195397in}}%
\pgfpathlineto{\pgfqpoint{0.913397in}{2.325649in}}%
\pgfusepath{stroke}%
\end{pgfscope}%
\begin{pgfscope}%
\pgfpathrectangle{\pgfqpoint{0.100000in}{0.212622in}}{\pgfqpoint{3.696000in}{3.696000in}}%
\pgfusepath{clip}%
\pgfsetrectcap%
\pgfsetroundjoin%
\pgfsetlinewidth{1.505625pt}%
\definecolor{currentstroke}{rgb}{1.000000,0.000000,0.000000}%
\pgfsetstrokecolor{currentstroke}%
\pgfsetdash{}{0pt}%
\pgfpathmoveto{\pgfqpoint{0.894229in}{2.195192in}}%
\pgfpathlineto{\pgfqpoint{0.913397in}{2.325649in}}%
\pgfusepath{stroke}%
\end{pgfscope}%
\begin{pgfscope}%
\pgfpathrectangle{\pgfqpoint{0.100000in}{0.212622in}}{\pgfqpoint{3.696000in}{3.696000in}}%
\pgfusepath{clip}%
\pgfsetrectcap%
\pgfsetroundjoin%
\pgfsetlinewidth{1.505625pt}%
\definecolor{currentstroke}{rgb}{1.000000,0.000000,0.000000}%
\pgfsetstrokecolor{currentstroke}%
\pgfsetdash{}{0pt}%
\pgfpathmoveto{\pgfqpoint{0.896019in}{2.194985in}}%
\pgfpathlineto{\pgfqpoint{0.913397in}{2.325649in}}%
\pgfusepath{stroke}%
\end{pgfscope}%
\begin{pgfscope}%
\pgfpathrectangle{\pgfqpoint{0.100000in}{0.212622in}}{\pgfqpoint{3.696000in}{3.696000in}}%
\pgfusepath{clip}%
\pgfsetrectcap%
\pgfsetroundjoin%
\pgfsetlinewidth{1.505625pt}%
\definecolor{currentstroke}{rgb}{1.000000,0.000000,0.000000}%
\pgfsetstrokecolor{currentstroke}%
\pgfsetdash{}{0pt}%
\pgfpathmoveto{\pgfqpoint{0.897911in}{2.194857in}}%
\pgfpathlineto{\pgfqpoint{0.913397in}{2.325649in}}%
\pgfusepath{stroke}%
\end{pgfscope}%
\begin{pgfscope}%
\pgfpathrectangle{\pgfqpoint{0.100000in}{0.212622in}}{\pgfqpoint{3.696000in}{3.696000in}}%
\pgfusepath{clip}%
\pgfsetrectcap%
\pgfsetroundjoin%
\pgfsetlinewidth{1.505625pt}%
\definecolor{currentstroke}{rgb}{1.000000,0.000000,0.000000}%
\pgfsetstrokecolor{currentstroke}%
\pgfsetdash{}{0pt}%
\pgfpathmoveto{\pgfqpoint{0.901249in}{2.194459in}}%
\pgfpathlineto{\pgfqpoint{0.922877in}{2.333099in}}%
\pgfusepath{stroke}%
\end{pgfscope}%
\begin{pgfscope}%
\pgfpathrectangle{\pgfqpoint{0.100000in}{0.212622in}}{\pgfqpoint{3.696000in}{3.696000in}}%
\pgfusepath{clip}%
\pgfsetrectcap%
\pgfsetroundjoin%
\pgfsetlinewidth{1.505625pt}%
\definecolor{currentstroke}{rgb}{1.000000,0.000000,0.000000}%
\pgfsetstrokecolor{currentstroke}%
\pgfsetdash{}{0pt}%
\pgfpathmoveto{\pgfqpoint{0.903832in}{2.194235in}}%
\pgfpathlineto{\pgfqpoint{0.922877in}{2.333099in}}%
\pgfusepath{stroke}%
\end{pgfscope}%
\begin{pgfscope}%
\pgfpathrectangle{\pgfqpoint{0.100000in}{0.212622in}}{\pgfqpoint{3.696000in}{3.696000in}}%
\pgfusepath{clip}%
\pgfsetrectcap%
\pgfsetroundjoin%
\pgfsetlinewidth{1.505625pt}%
\definecolor{currentstroke}{rgb}{1.000000,0.000000,0.000000}%
\pgfsetstrokecolor{currentstroke}%
\pgfsetdash{}{0pt}%
\pgfpathmoveto{\pgfqpoint{0.906037in}{2.194140in}}%
\pgfpathlineto{\pgfqpoint{0.922877in}{2.333099in}}%
\pgfusepath{stroke}%
\end{pgfscope}%
\begin{pgfscope}%
\pgfpathrectangle{\pgfqpoint{0.100000in}{0.212622in}}{\pgfqpoint{3.696000in}{3.696000in}}%
\pgfusepath{clip}%
\pgfsetrectcap%
\pgfsetroundjoin%
\pgfsetlinewidth{1.505625pt}%
\definecolor{currentstroke}{rgb}{1.000000,0.000000,0.000000}%
\pgfsetstrokecolor{currentstroke}%
\pgfsetdash{}{0pt}%
\pgfpathmoveto{\pgfqpoint{0.908504in}{2.193932in}}%
\pgfpathlineto{\pgfqpoint{0.932345in}{2.340540in}}%
\pgfusepath{stroke}%
\end{pgfscope}%
\begin{pgfscope}%
\pgfpathrectangle{\pgfqpoint{0.100000in}{0.212622in}}{\pgfqpoint{3.696000in}{3.696000in}}%
\pgfusepath{clip}%
\pgfsetrectcap%
\pgfsetroundjoin%
\pgfsetlinewidth{1.505625pt}%
\definecolor{currentstroke}{rgb}{1.000000,0.000000,0.000000}%
\pgfsetstrokecolor{currentstroke}%
\pgfsetdash{}{0pt}%
\pgfpathmoveto{\pgfqpoint{0.909318in}{2.193938in}}%
\pgfpathlineto{\pgfqpoint{0.932345in}{2.340540in}}%
\pgfusepath{stroke}%
\end{pgfscope}%
\begin{pgfscope}%
\pgfpathrectangle{\pgfqpoint{0.100000in}{0.212622in}}{\pgfqpoint{3.696000in}{3.696000in}}%
\pgfusepath{clip}%
\pgfsetrectcap%
\pgfsetroundjoin%
\pgfsetlinewidth{1.505625pt}%
\definecolor{currentstroke}{rgb}{1.000000,0.000000,0.000000}%
\pgfsetstrokecolor{currentstroke}%
\pgfsetdash{}{0pt}%
\pgfpathmoveto{\pgfqpoint{0.911414in}{2.193647in}}%
\pgfpathlineto{\pgfqpoint{0.932345in}{2.340540in}}%
\pgfusepath{stroke}%
\end{pgfscope}%
\begin{pgfscope}%
\pgfpathrectangle{\pgfqpoint{0.100000in}{0.212622in}}{\pgfqpoint{3.696000in}{3.696000in}}%
\pgfusepath{clip}%
\pgfsetrectcap%
\pgfsetroundjoin%
\pgfsetlinewidth{1.505625pt}%
\definecolor{currentstroke}{rgb}{1.000000,0.000000,0.000000}%
\pgfsetstrokecolor{currentstroke}%
\pgfsetdash{}{0pt}%
\pgfpathmoveto{\pgfqpoint{0.913261in}{2.193473in}}%
\pgfpathlineto{\pgfqpoint{0.932345in}{2.340540in}}%
\pgfusepath{stroke}%
\end{pgfscope}%
\begin{pgfscope}%
\pgfpathrectangle{\pgfqpoint{0.100000in}{0.212622in}}{\pgfqpoint{3.696000in}{3.696000in}}%
\pgfusepath{clip}%
\pgfsetrectcap%
\pgfsetroundjoin%
\pgfsetlinewidth{1.505625pt}%
\definecolor{currentstroke}{rgb}{1.000000,0.000000,0.000000}%
\pgfsetstrokecolor{currentstroke}%
\pgfsetdash{}{0pt}%
\pgfpathmoveto{\pgfqpoint{0.916687in}{2.193119in}}%
\pgfpathlineto{\pgfqpoint{0.932345in}{2.340540in}}%
\pgfusepath{stroke}%
\end{pgfscope}%
\begin{pgfscope}%
\pgfpathrectangle{\pgfqpoint{0.100000in}{0.212622in}}{\pgfqpoint{3.696000in}{3.696000in}}%
\pgfusepath{clip}%
\pgfsetrectcap%
\pgfsetroundjoin%
\pgfsetlinewidth{1.505625pt}%
\definecolor{currentstroke}{rgb}{1.000000,0.000000,0.000000}%
\pgfsetstrokecolor{currentstroke}%
\pgfsetdash{}{0pt}%
\pgfpathmoveto{\pgfqpoint{0.920379in}{2.192871in}}%
\pgfpathlineto{\pgfqpoint{0.941800in}{2.347970in}}%
\pgfusepath{stroke}%
\end{pgfscope}%
\begin{pgfscope}%
\pgfpathrectangle{\pgfqpoint{0.100000in}{0.212622in}}{\pgfqpoint{3.696000in}{3.696000in}}%
\pgfusepath{clip}%
\pgfsetrectcap%
\pgfsetroundjoin%
\pgfsetlinewidth{1.505625pt}%
\definecolor{currentstroke}{rgb}{1.000000,0.000000,0.000000}%
\pgfsetstrokecolor{currentstroke}%
\pgfsetdash{}{0pt}%
\pgfpathmoveto{\pgfqpoint{0.922811in}{2.192699in}}%
\pgfpathlineto{\pgfqpoint{0.941800in}{2.347970in}}%
\pgfusepath{stroke}%
\end{pgfscope}%
\begin{pgfscope}%
\pgfpathrectangle{\pgfqpoint{0.100000in}{0.212622in}}{\pgfqpoint{3.696000in}{3.696000in}}%
\pgfusepath{clip}%
\pgfsetrectcap%
\pgfsetroundjoin%
\pgfsetlinewidth{1.505625pt}%
\definecolor{currentstroke}{rgb}{1.000000,0.000000,0.000000}%
\pgfsetstrokecolor{currentstroke}%
\pgfsetdash{}{0pt}%
\pgfpathmoveto{\pgfqpoint{0.924194in}{2.192594in}}%
\pgfpathlineto{\pgfqpoint{0.941800in}{2.347970in}}%
\pgfusepath{stroke}%
\end{pgfscope}%
\begin{pgfscope}%
\pgfpathrectangle{\pgfqpoint{0.100000in}{0.212622in}}{\pgfqpoint{3.696000in}{3.696000in}}%
\pgfusepath{clip}%
\pgfsetrectcap%
\pgfsetroundjoin%
\pgfsetlinewidth{1.505625pt}%
\definecolor{currentstroke}{rgb}{1.000000,0.000000,0.000000}%
\pgfsetstrokecolor{currentstroke}%
\pgfsetdash{}{0pt}%
\pgfpathmoveto{\pgfqpoint{0.924795in}{2.192541in}}%
\pgfpathlineto{\pgfqpoint{0.941800in}{2.347970in}}%
\pgfusepath{stroke}%
\end{pgfscope}%
\begin{pgfscope}%
\pgfpathrectangle{\pgfqpoint{0.100000in}{0.212622in}}{\pgfqpoint{3.696000in}{3.696000in}}%
\pgfusepath{clip}%
\pgfsetrectcap%
\pgfsetroundjoin%
\pgfsetlinewidth{1.505625pt}%
\definecolor{currentstroke}{rgb}{1.000000,0.000000,0.000000}%
\pgfsetstrokecolor{currentstroke}%
\pgfsetdash{}{0pt}%
\pgfpathmoveto{\pgfqpoint{0.926110in}{2.192353in}}%
\pgfpathlineto{\pgfqpoint{0.951243in}{2.355391in}}%
\pgfusepath{stroke}%
\end{pgfscope}%
\begin{pgfscope}%
\pgfpathrectangle{\pgfqpoint{0.100000in}{0.212622in}}{\pgfqpoint{3.696000in}{3.696000in}}%
\pgfusepath{clip}%
\pgfsetrectcap%
\pgfsetroundjoin%
\pgfsetlinewidth{1.505625pt}%
\definecolor{currentstroke}{rgb}{1.000000,0.000000,0.000000}%
\pgfsetstrokecolor{currentstroke}%
\pgfsetdash{}{0pt}%
\pgfpathmoveto{\pgfqpoint{0.927068in}{2.192153in}}%
\pgfpathlineto{\pgfqpoint{0.951243in}{2.355391in}}%
\pgfusepath{stroke}%
\end{pgfscope}%
\begin{pgfscope}%
\pgfpathrectangle{\pgfqpoint{0.100000in}{0.212622in}}{\pgfqpoint{3.696000in}{3.696000in}}%
\pgfusepath{clip}%
\pgfsetrectcap%
\pgfsetroundjoin%
\pgfsetlinewidth{1.505625pt}%
\definecolor{currentstroke}{rgb}{1.000000,0.000000,0.000000}%
\pgfsetstrokecolor{currentstroke}%
\pgfsetdash{}{0pt}%
\pgfpathmoveto{\pgfqpoint{0.929450in}{2.191925in}}%
\pgfpathlineto{\pgfqpoint{0.951243in}{2.355391in}}%
\pgfusepath{stroke}%
\end{pgfscope}%
\begin{pgfscope}%
\pgfpathrectangle{\pgfqpoint{0.100000in}{0.212622in}}{\pgfqpoint{3.696000in}{3.696000in}}%
\pgfusepath{clip}%
\pgfsetrectcap%
\pgfsetroundjoin%
\pgfsetlinewidth{1.505625pt}%
\definecolor{currentstroke}{rgb}{1.000000,0.000000,0.000000}%
\pgfsetstrokecolor{currentstroke}%
\pgfsetdash{}{0pt}%
\pgfpathmoveto{\pgfqpoint{0.930633in}{2.191883in}}%
\pgfpathlineto{\pgfqpoint{0.951243in}{2.355391in}}%
\pgfusepath{stroke}%
\end{pgfscope}%
\begin{pgfscope}%
\pgfpathrectangle{\pgfqpoint{0.100000in}{0.212622in}}{\pgfqpoint{3.696000in}{3.696000in}}%
\pgfusepath{clip}%
\pgfsetrectcap%
\pgfsetroundjoin%
\pgfsetlinewidth{1.505625pt}%
\definecolor{currentstroke}{rgb}{1.000000,0.000000,0.000000}%
\pgfsetstrokecolor{currentstroke}%
\pgfsetdash{}{0pt}%
\pgfpathmoveto{\pgfqpoint{0.931012in}{2.191504in}}%
\pgfpathlineto{\pgfqpoint{0.951243in}{2.355391in}}%
\pgfusepath{stroke}%
\end{pgfscope}%
\begin{pgfscope}%
\pgfpathrectangle{\pgfqpoint{0.100000in}{0.212622in}}{\pgfqpoint{3.696000in}{3.696000in}}%
\pgfusepath{clip}%
\pgfsetrectcap%
\pgfsetroundjoin%
\pgfsetlinewidth{1.505625pt}%
\definecolor{currentstroke}{rgb}{1.000000,0.000000,0.000000}%
\pgfsetstrokecolor{currentstroke}%
\pgfsetdash{}{0pt}%
\pgfpathmoveto{\pgfqpoint{0.933032in}{2.191149in}}%
\pgfpathlineto{\pgfqpoint{0.951243in}{2.355391in}}%
\pgfusepath{stroke}%
\end{pgfscope}%
\begin{pgfscope}%
\pgfpathrectangle{\pgfqpoint{0.100000in}{0.212622in}}{\pgfqpoint{3.696000in}{3.696000in}}%
\pgfusepath{clip}%
\pgfsetrectcap%
\pgfsetroundjoin%
\pgfsetlinewidth{1.505625pt}%
\definecolor{currentstroke}{rgb}{1.000000,0.000000,0.000000}%
\pgfsetstrokecolor{currentstroke}%
\pgfsetdash{}{0pt}%
\pgfpathmoveto{\pgfqpoint{0.935142in}{2.190968in}}%
\pgfpathlineto{\pgfqpoint{0.960673in}{2.362802in}}%
\pgfusepath{stroke}%
\end{pgfscope}%
\begin{pgfscope}%
\pgfpathrectangle{\pgfqpoint{0.100000in}{0.212622in}}{\pgfqpoint{3.696000in}{3.696000in}}%
\pgfusepath{clip}%
\pgfsetrectcap%
\pgfsetroundjoin%
\pgfsetlinewidth{1.505625pt}%
\definecolor{currentstroke}{rgb}{1.000000,0.000000,0.000000}%
\pgfsetstrokecolor{currentstroke}%
\pgfsetdash{}{0pt}%
\pgfpathmoveto{\pgfqpoint{0.936757in}{2.190852in}}%
\pgfpathlineto{\pgfqpoint{0.960673in}{2.362802in}}%
\pgfusepath{stroke}%
\end{pgfscope}%
\begin{pgfscope}%
\pgfpathrectangle{\pgfqpoint{0.100000in}{0.212622in}}{\pgfqpoint{3.696000in}{3.696000in}}%
\pgfusepath{clip}%
\pgfsetrectcap%
\pgfsetroundjoin%
\pgfsetlinewidth{1.505625pt}%
\definecolor{currentstroke}{rgb}{1.000000,0.000000,0.000000}%
\pgfsetstrokecolor{currentstroke}%
\pgfsetdash{}{0pt}%
\pgfpathmoveto{\pgfqpoint{0.937687in}{2.190794in}}%
\pgfpathlineto{\pgfqpoint{0.960673in}{2.362802in}}%
\pgfusepath{stroke}%
\end{pgfscope}%
\begin{pgfscope}%
\pgfpathrectangle{\pgfqpoint{0.100000in}{0.212622in}}{\pgfqpoint{3.696000in}{3.696000in}}%
\pgfusepath{clip}%
\pgfsetrectcap%
\pgfsetroundjoin%
\pgfsetlinewidth{1.505625pt}%
\definecolor{currentstroke}{rgb}{1.000000,0.000000,0.000000}%
\pgfsetstrokecolor{currentstroke}%
\pgfsetdash{}{0pt}%
\pgfpathmoveto{\pgfqpoint{0.936215in}{2.190192in}}%
\pgfpathlineto{\pgfqpoint{0.960673in}{2.362802in}}%
\pgfusepath{stroke}%
\end{pgfscope}%
\begin{pgfscope}%
\pgfpathrectangle{\pgfqpoint{0.100000in}{0.212622in}}{\pgfqpoint{3.696000in}{3.696000in}}%
\pgfusepath{clip}%
\pgfsetrectcap%
\pgfsetroundjoin%
\pgfsetlinewidth{1.505625pt}%
\definecolor{currentstroke}{rgb}{1.000000,0.000000,0.000000}%
\pgfsetstrokecolor{currentstroke}%
\pgfsetdash{}{0pt}%
\pgfpathmoveto{\pgfqpoint{0.936945in}{2.190110in}}%
\pgfpathlineto{\pgfqpoint{0.960673in}{2.362802in}}%
\pgfusepath{stroke}%
\end{pgfscope}%
\begin{pgfscope}%
\pgfpathrectangle{\pgfqpoint{0.100000in}{0.212622in}}{\pgfqpoint{3.696000in}{3.696000in}}%
\pgfusepath{clip}%
\pgfsetrectcap%
\pgfsetroundjoin%
\pgfsetlinewidth{1.505625pt}%
\definecolor{currentstroke}{rgb}{1.000000,0.000000,0.000000}%
\pgfsetstrokecolor{currentstroke}%
\pgfsetdash{}{0pt}%
\pgfpathmoveto{\pgfqpoint{0.937944in}{2.190022in}}%
\pgfpathlineto{\pgfqpoint{0.960673in}{2.362802in}}%
\pgfusepath{stroke}%
\end{pgfscope}%
\begin{pgfscope}%
\pgfpathrectangle{\pgfqpoint{0.100000in}{0.212622in}}{\pgfqpoint{3.696000in}{3.696000in}}%
\pgfusepath{clip}%
\pgfsetrectcap%
\pgfsetroundjoin%
\pgfsetlinewidth{1.505625pt}%
\definecolor{currentstroke}{rgb}{1.000000,0.000000,0.000000}%
\pgfsetstrokecolor{currentstroke}%
\pgfsetdash{}{0pt}%
\pgfpathmoveto{\pgfqpoint{0.939929in}{2.189864in}}%
\pgfpathlineto{\pgfqpoint{0.960673in}{2.362802in}}%
\pgfusepath{stroke}%
\end{pgfscope}%
\begin{pgfscope}%
\pgfpathrectangle{\pgfqpoint{0.100000in}{0.212622in}}{\pgfqpoint{3.696000in}{3.696000in}}%
\pgfusepath{clip}%
\pgfsetrectcap%
\pgfsetroundjoin%
\pgfsetlinewidth{1.505625pt}%
\definecolor{currentstroke}{rgb}{1.000000,0.000000,0.000000}%
\pgfsetstrokecolor{currentstroke}%
\pgfsetdash{}{0pt}%
\pgfpathmoveto{\pgfqpoint{0.942624in}{2.189777in}}%
\pgfpathlineto{\pgfqpoint{0.970091in}{2.370204in}}%
\pgfusepath{stroke}%
\end{pgfscope}%
\begin{pgfscope}%
\pgfpathrectangle{\pgfqpoint{0.100000in}{0.212622in}}{\pgfqpoint{3.696000in}{3.696000in}}%
\pgfusepath{clip}%
\pgfsetrectcap%
\pgfsetroundjoin%
\pgfsetlinewidth{1.505625pt}%
\definecolor{currentstroke}{rgb}{1.000000,0.000000,0.000000}%
\pgfsetstrokecolor{currentstroke}%
\pgfsetdash{}{0pt}%
\pgfpathmoveto{\pgfqpoint{0.944230in}{2.189669in}}%
\pgfpathlineto{\pgfqpoint{0.970091in}{2.370204in}}%
\pgfusepath{stroke}%
\end{pgfscope}%
\begin{pgfscope}%
\pgfpathrectangle{\pgfqpoint{0.100000in}{0.212622in}}{\pgfqpoint{3.696000in}{3.696000in}}%
\pgfusepath{clip}%
\pgfsetrectcap%
\pgfsetroundjoin%
\pgfsetlinewidth{1.505625pt}%
\definecolor{currentstroke}{rgb}{1.000000,0.000000,0.000000}%
\pgfsetstrokecolor{currentstroke}%
\pgfsetdash{}{0pt}%
\pgfpathmoveto{\pgfqpoint{0.945968in}{2.189614in}}%
\pgfpathlineto{\pgfqpoint{0.970091in}{2.370204in}}%
\pgfusepath{stroke}%
\end{pgfscope}%
\begin{pgfscope}%
\pgfpathrectangle{\pgfqpoint{0.100000in}{0.212622in}}{\pgfqpoint{3.696000in}{3.696000in}}%
\pgfusepath{clip}%
\pgfsetrectcap%
\pgfsetroundjoin%
\pgfsetlinewidth{1.505625pt}%
\definecolor{currentstroke}{rgb}{1.000000,0.000000,0.000000}%
\pgfsetstrokecolor{currentstroke}%
\pgfsetdash{}{0pt}%
\pgfpathmoveto{\pgfqpoint{0.944919in}{2.189166in}}%
\pgfpathlineto{\pgfqpoint{0.970091in}{2.370204in}}%
\pgfusepath{stroke}%
\end{pgfscope}%
\begin{pgfscope}%
\pgfpathrectangle{\pgfqpoint{0.100000in}{0.212622in}}{\pgfqpoint{3.696000in}{3.696000in}}%
\pgfusepath{clip}%
\pgfsetrectcap%
\pgfsetroundjoin%
\pgfsetlinewidth{1.505625pt}%
\definecolor{currentstroke}{rgb}{1.000000,0.000000,0.000000}%
\pgfsetstrokecolor{currentstroke}%
\pgfsetdash{}{0pt}%
\pgfpathmoveto{\pgfqpoint{0.946509in}{2.189002in}}%
\pgfpathlineto{\pgfqpoint{0.970091in}{2.370204in}}%
\pgfusepath{stroke}%
\end{pgfscope}%
\begin{pgfscope}%
\pgfpathrectangle{\pgfqpoint{0.100000in}{0.212622in}}{\pgfqpoint{3.696000in}{3.696000in}}%
\pgfusepath{clip}%
\pgfsetrectcap%
\pgfsetroundjoin%
\pgfsetlinewidth{1.505625pt}%
\definecolor{currentstroke}{rgb}{1.000000,0.000000,0.000000}%
\pgfsetstrokecolor{currentstroke}%
\pgfsetdash{}{0pt}%
\pgfpathmoveto{\pgfqpoint{0.947868in}{2.188799in}}%
\pgfpathlineto{\pgfqpoint{0.970091in}{2.370204in}}%
\pgfusepath{stroke}%
\end{pgfscope}%
\begin{pgfscope}%
\pgfpathrectangle{\pgfqpoint{0.100000in}{0.212622in}}{\pgfqpoint{3.696000in}{3.696000in}}%
\pgfusepath{clip}%
\pgfsetrectcap%
\pgfsetroundjoin%
\pgfsetlinewidth{1.505625pt}%
\definecolor{currentstroke}{rgb}{1.000000,0.000000,0.000000}%
\pgfsetstrokecolor{currentstroke}%
\pgfsetdash{}{0pt}%
\pgfpathmoveto{\pgfqpoint{0.950599in}{2.188603in}}%
\pgfpathlineto{\pgfqpoint{0.979496in}{2.377595in}}%
\pgfusepath{stroke}%
\end{pgfscope}%
\begin{pgfscope}%
\pgfpathrectangle{\pgfqpoint{0.100000in}{0.212622in}}{\pgfqpoint{3.696000in}{3.696000in}}%
\pgfusepath{clip}%
\pgfsetrectcap%
\pgfsetroundjoin%
\pgfsetlinewidth{1.505625pt}%
\definecolor{currentstroke}{rgb}{1.000000,0.000000,0.000000}%
\pgfsetstrokecolor{currentstroke}%
\pgfsetdash{}{0pt}%
\pgfpathmoveto{\pgfqpoint{0.951965in}{2.188522in}}%
\pgfpathlineto{\pgfqpoint{0.979496in}{2.377595in}}%
\pgfusepath{stroke}%
\end{pgfscope}%
\begin{pgfscope}%
\pgfpathrectangle{\pgfqpoint{0.100000in}{0.212622in}}{\pgfqpoint{3.696000in}{3.696000in}}%
\pgfusepath{clip}%
\pgfsetrectcap%
\pgfsetroundjoin%
\pgfsetlinewidth{1.505625pt}%
\definecolor{currentstroke}{rgb}{1.000000,0.000000,0.000000}%
\pgfsetstrokecolor{currentstroke}%
\pgfsetdash{}{0pt}%
\pgfpathmoveto{\pgfqpoint{0.953524in}{2.188430in}}%
\pgfpathlineto{\pgfqpoint{0.979496in}{2.377595in}}%
\pgfusepath{stroke}%
\end{pgfscope}%
\begin{pgfscope}%
\pgfpathrectangle{\pgfqpoint{0.100000in}{0.212622in}}{\pgfqpoint{3.696000in}{3.696000in}}%
\pgfusepath{clip}%
\pgfsetrectcap%
\pgfsetroundjoin%
\pgfsetlinewidth{1.505625pt}%
\definecolor{currentstroke}{rgb}{1.000000,0.000000,0.000000}%
\pgfsetstrokecolor{currentstroke}%
\pgfsetdash{}{0pt}%
\pgfpathmoveto{\pgfqpoint{0.954566in}{2.188361in}}%
\pgfpathlineto{\pgfqpoint{0.979496in}{2.377595in}}%
\pgfusepath{stroke}%
\end{pgfscope}%
\begin{pgfscope}%
\pgfpathrectangle{\pgfqpoint{0.100000in}{0.212622in}}{\pgfqpoint{3.696000in}{3.696000in}}%
\pgfusepath{clip}%
\pgfsetrectcap%
\pgfsetroundjoin%
\pgfsetlinewidth{1.505625pt}%
\definecolor{currentstroke}{rgb}{1.000000,0.000000,0.000000}%
\pgfsetstrokecolor{currentstroke}%
\pgfsetdash{}{0pt}%
\pgfpathmoveto{\pgfqpoint{0.955358in}{2.188259in}}%
\pgfpathlineto{\pgfqpoint{0.979496in}{2.377595in}}%
\pgfusepath{stroke}%
\end{pgfscope}%
\begin{pgfscope}%
\pgfpathrectangle{\pgfqpoint{0.100000in}{0.212622in}}{\pgfqpoint{3.696000in}{3.696000in}}%
\pgfusepath{clip}%
\pgfsetrectcap%
\pgfsetroundjoin%
\pgfsetlinewidth{1.505625pt}%
\definecolor{currentstroke}{rgb}{1.000000,0.000000,0.000000}%
\pgfsetstrokecolor{currentstroke}%
\pgfsetdash{}{0pt}%
\pgfpathmoveto{\pgfqpoint{0.957085in}{2.188137in}}%
\pgfpathlineto{\pgfqpoint{0.979496in}{2.377595in}}%
\pgfusepath{stroke}%
\end{pgfscope}%
\begin{pgfscope}%
\pgfpathrectangle{\pgfqpoint{0.100000in}{0.212622in}}{\pgfqpoint{3.696000in}{3.696000in}}%
\pgfusepath{clip}%
\pgfsetrectcap%
\pgfsetroundjoin%
\pgfsetlinewidth{1.505625pt}%
\definecolor{currentstroke}{rgb}{1.000000,0.000000,0.000000}%
\pgfsetstrokecolor{currentstroke}%
\pgfsetdash{}{0pt}%
\pgfpathmoveto{\pgfqpoint{0.959536in}{2.188012in}}%
\pgfpathlineto{\pgfqpoint{0.988889in}{2.384977in}}%
\pgfusepath{stroke}%
\end{pgfscope}%
\begin{pgfscope}%
\pgfpathrectangle{\pgfqpoint{0.100000in}{0.212622in}}{\pgfqpoint{3.696000in}{3.696000in}}%
\pgfusepath{clip}%
\pgfsetrectcap%
\pgfsetroundjoin%
\pgfsetlinewidth{1.505625pt}%
\definecolor{currentstroke}{rgb}{1.000000,0.000000,0.000000}%
\pgfsetstrokecolor{currentstroke}%
\pgfsetdash{}{0pt}%
\pgfpathmoveto{\pgfqpoint{0.962338in}{2.187903in}}%
\pgfpathlineto{\pgfqpoint{0.988889in}{2.384977in}}%
\pgfusepath{stroke}%
\end{pgfscope}%
\begin{pgfscope}%
\pgfpathrectangle{\pgfqpoint{0.100000in}{0.212622in}}{\pgfqpoint{3.696000in}{3.696000in}}%
\pgfusepath{clip}%
\pgfsetrectcap%
\pgfsetroundjoin%
\pgfsetlinewidth{1.505625pt}%
\definecolor{currentstroke}{rgb}{1.000000,0.000000,0.000000}%
\pgfsetstrokecolor{currentstroke}%
\pgfsetdash{}{0pt}%
\pgfpathmoveto{\pgfqpoint{0.963954in}{2.187792in}}%
\pgfpathlineto{\pgfqpoint{0.988889in}{2.384977in}}%
\pgfusepath{stroke}%
\end{pgfscope}%
\begin{pgfscope}%
\pgfpathrectangle{\pgfqpoint{0.100000in}{0.212622in}}{\pgfqpoint{3.696000in}{3.696000in}}%
\pgfusepath{clip}%
\pgfsetrectcap%
\pgfsetroundjoin%
\pgfsetlinewidth{1.505625pt}%
\definecolor{currentstroke}{rgb}{1.000000,0.000000,0.000000}%
\pgfsetstrokecolor{currentstroke}%
\pgfsetdash{}{0pt}%
\pgfpathmoveto{\pgfqpoint{0.963149in}{2.187442in}}%
\pgfpathlineto{\pgfqpoint{0.988889in}{2.384977in}}%
\pgfusepath{stroke}%
\end{pgfscope}%
\begin{pgfscope}%
\pgfpathrectangle{\pgfqpoint{0.100000in}{0.212622in}}{\pgfqpoint{3.696000in}{3.696000in}}%
\pgfusepath{clip}%
\pgfsetrectcap%
\pgfsetroundjoin%
\pgfsetlinewidth{1.505625pt}%
\definecolor{currentstroke}{rgb}{1.000000,0.000000,0.000000}%
\pgfsetstrokecolor{currentstroke}%
\pgfsetdash{}{0pt}%
\pgfpathmoveto{\pgfqpoint{0.965825in}{2.187064in}}%
\pgfpathlineto{\pgfqpoint{0.988889in}{2.384977in}}%
\pgfusepath{stroke}%
\end{pgfscope}%
\begin{pgfscope}%
\pgfpathrectangle{\pgfqpoint{0.100000in}{0.212622in}}{\pgfqpoint{3.696000in}{3.696000in}}%
\pgfusepath{clip}%
\pgfsetrectcap%
\pgfsetroundjoin%
\pgfsetlinewidth{1.505625pt}%
\definecolor{currentstroke}{rgb}{1.000000,0.000000,0.000000}%
\pgfsetstrokecolor{currentstroke}%
\pgfsetdash{}{0pt}%
\pgfpathmoveto{\pgfqpoint{0.967065in}{2.186799in}}%
\pgfpathlineto{\pgfqpoint{0.998270in}{2.392350in}}%
\pgfusepath{stroke}%
\end{pgfscope}%
\begin{pgfscope}%
\pgfpathrectangle{\pgfqpoint{0.100000in}{0.212622in}}{\pgfqpoint{3.696000in}{3.696000in}}%
\pgfusepath{clip}%
\pgfsetrectcap%
\pgfsetroundjoin%
\pgfsetlinewidth{1.505625pt}%
\definecolor{currentstroke}{rgb}{1.000000,0.000000,0.000000}%
\pgfsetstrokecolor{currentstroke}%
\pgfsetdash{}{0pt}%
\pgfpathmoveto{\pgfqpoint{0.970674in}{2.186434in}}%
\pgfpathlineto{\pgfqpoint{0.998270in}{2.392350in}}%
\pgfusepath{stroke}%
\end{pgfscope}%
\begin{pgfscope}%
\pgfpathrectangle{\pgfqpoint{0.100000in}{0.212622in}}{\pgfqpoint{3.696000in}{3.696000in}}%
\pgfusepath{clip}%
\pgfsetrectcap%
\pgfsetroundjoin%
\pgfsetlinewidth{1.505625pt}%
\definecolor{currentstroke}{rgb}{1.000000,0.000000,0.000000}%
\pgfsetstrokecolor{currentstroke}%
\pgfsetdash{}{0pt}%
\pgfpathmoveto{\pgfqpoint{0.974127in}{2.186176in}}%
\pgfpathlineto{\pgfqpoint{0.998270in}{2.392350in}}%
\pgfusepath{stroke}%
\end{pgfscope}%
\begin{pgfscope}%
\pgfpathrectangle{\pgfqpoint{0.100000in}{0.212622in}}{\pgfqpoint{3.696000in}{3.696000in}}%
\pgfusepath{clip}%
\pgfsetrectcap%
\pgfsetroundjoin%
\pgfsetlinewidth{1.505625pt}%
\definecolor{currentstroke}{rgb}{1.000000,0.000000,0.000000}%
\pgfsetstrokecolor{currentstroke}%
\pgfsetdash{}{0pt}%
\pgfpathmoveto{\pgfqpoint{0.972725in}{2.184990in}}%
\pgfpathlineto{\pgfqpoint{1.007639in}{2.399712in}}%
\pgfusepath{stroke}%
\end{pgfscope}%
\begin{pgfscope}%
\pgfpathrectangle{\pgfqpoint{0.100000in}{0.212622in}}{\pgfqpoint{3.696000in}{3.696000in}}%
\pgfusepath{clip}%
\pgfsetrectcap%
\pgfsetroundjoin%
\pgfsetlinewidth{1.505625pt}%
\definecolor{currentstroke}{rgb}{1.000000,0.000000,0.000000}%
\pgfsetstrokecolor{currentstroke}%
\pgfsetdash{}{0pt}%
\pgfpathmoveto{\pgfqpoint{0.975432in}{2.184659in}}%
\pgfpathlineto{\pgfqpoint{1.007639in}{2.399712in}}%
\pgfusepath{stroke}%
\end{pgfscope}%
\begin{pgfscope}%
\pgfpathrectangle{\pgfqpoint{0.100000in}{0.212622in}}{\pgfqpoint{3.696000in}{3.696000in}}%
\pgfusepath{clip}%
\pgfsetrectcap%
\pgfsetroundjoin%
\pgfsetlinewidth{1.505625pt}%
\definecolor{currentstroke}{rgb}{1.000000,0.000000,0.000000}%
\pgfsetstrokecolor{currentstroke}%
\pgfsetdash{}{0pt}%
\pgfpathmoveto{\pgfqpoint{0.977823in}{2.184393in}}%
\pgfpathlineto{\pgfqpoint{1.007639in}{2.399712in}}%
\pgfusepath{stroke}%
\end{pgfscope}%
\begin{pgfscope}%
\pgfpathrectangle{\pgfqpoint{0.100000in}{0.212622in}}{\pgfqpoint{3.696000in}{3.696000in}}%
\pgfusepath{clip}%
\pgfsetrectcap%
\pgfsetroundjoin%
\pgfsetlinewidth{1.505625pt}%
\definecolor{currentstroke}{rgb}{1.000000,0.000000,0.000000}%
\pgfsetstrokecolor{currentstroke}%
\pgfsetdash{}{0pt}%
\pgfpathmoveto{\pgfqpoint{0.981789in}{2.183977in}}%
\pgfpathlineto{\pgfqpoint{1.016995in}{2.407065in}}%
\pgfusepath{stroke}%
\end{pgfscope}%
\begin{pgfscope}%
\pgfpathrectangle{\pgfqpoint{0.100000in}{0.212622in}}{\pgfqpoint{3.696000in}{3.696000in}}%
\pgfusepath{clip}%
\pgfsetrectcap%
\pgfsetroundjoin%
\pgfsetlinewidth{1.505625pt}%
\definecolor{currentstroke}{rgb}{1.000000,0.000000,0.000000}%
\pgfsetstrokecolor{currentstroke}%
\pgfsetdash{}{0pt}%
\pgfpathmoveto{\pgfqpoint{0.985927in}{2.184004in}}%
\pgfpathlineto{\pgfqpoint{1.016995in}{2.407065in}}%
\pgfusepath{stroke}%
\end{pgfscope}%
\begin{pgfscope}%
\pgfpathrectangle{\pgfqpoint{0.100000in}{0.212622in}}{\pgfqpoint{3.696000in}{3.696000in}}%
\pgfusepath{clip}%
\pgfsetrectcap%
\pgfsetroundjoin%
\pgfsetlinewidth{1.505625pt}%
\definecolor{currentstroke}{rgb}{1.000000,0.000000,0.000000}%
\pgfsetstrokecolor{currentstroke}%
\pgfsetdash{}{0pt}%
\pgfpathmoveto{\pgfqpoint{0.990144in}{2.183897in}}%
\pgfpathlineto{\pgfqpoint{1.026340in}{2.414409in}}%
\pgfusepath{stroke}%
\end{pgfscope}%
\begin{pgfscope}%
\pgfpathrectangle{\pgfqpoint{0.100000in}{0.212622in}}{\pgfqpoint{3.696000in}{3.696000in}}%
\pgfusepath{clip}%
\pgfsetrectcap%
\pgfsetroundjoin%
\pgfsetlinewidth{1.505625pt}%
\definecolor{currentstroke}{rgb}{1.000000,0.000000,0.000000}%
\pgfsetstrokecolor{currentstroke}%
\pgfsetdash{}{0pt}%
\pgfpathmoveto{\pgfqpoint{0.993173in}{2.183582in}}%
\pgfpathlineto{\pgfqpoint{1.026340in}{2.414409in}}%
\pgfusepath{stroke}%
\end{pgfscope}%
\begin{pgfscope}%
\pgfpathrectangle{\pgfqpoint{0.100000in}{0.212622in}}{\pgfqpoint{3.696000in}{3.696000in}}%
\pgfusepath{clip}%
\pgfsetrectcap%
\pgfsetroundjoin%
\pgfsetlinewidth{1.505625pt}%
\definecolor{currentstroke}{rgb}{1.000000,0.000000,0.000000}%
\pgfsetstrokecolor{currentstroke}%
\pgfsetdash{}{0pt}%
\pgfpathmoveto{\pgfqpoint{0.996803in}{2.183560in}}%
\pgfpathlineto{\pgfqpoint{1.026340in}{2.414409in}}%
\pgfusepath{stroke}%
\end{pgfscope}%
\begin{pgfscope}%
\pgfpathrectangle{\pgfqpoint{0.100000in}{0.212622in}}{\pgfqpoint{3.696000in}{3.696000in}}%
\pgfusepath{clip}%
\pgfsetrectcap%
\pgfsetroundjoin%
\pgfsetlinewidth{1.505625pt}%
\definecolor{currentstroke}{rgb}{1.000000,0.000000,0.000000}%
\pgfsetstrokecolor{currentstroke}%
\pgfsetdash{}{0pt}%
\pgfpathmoveto{\pgfqpoint{1.003853in}{2.182553in}}%
\pgfpathlineto{\pgfqpoint{1.035672in}{2.421743in}}%
\pgfusepath{stroke}%
\end{pgfscope}%
\begin{pgfscope}%
\pgfpathrectangle{\pgfqpoint{0.100000in}{0.212622in}}{\pgfqpoint{3.696000in}{3.696000in}}%
\pgfusepath{clip}%
\pgfsetrectcap%
\pgfsetroundjoin%
\pgfsetlinewidth{1.505625pt}%
\definecolor{currentstroke}{rgb}{1.000000,0.000000,0.000000}%
\pgfsetstrokecolor{currentstroke}%
\pgfsetdash{}{0pt}%
\pgfpathmoveto{\pgfqpoint{1.007144in}{2.181807in}}%
\pgfpathlineto{\pgfqpoint{1.035672in}{2.421743in}}%
\pgfusepath{stroke}%
\end{pgfscope}%
\begin{pgfscope}%
\pgfpathrectangle{\pgfqpoint{0.100000in}{0.212622in}}{\pgfqpoint{3.696000in}{3.696000in}}%
\pgfusepath{clip}%
\pgfsetrectcap%
\pgfsetroundjoin%
\pgfsetlinewidth{1.505625pt}%
\definecolor{currentstroke}{rgb}{1.000000,0.000000,0.000000}%
\pgfsetstrokecolor{currentstroke}%
\pgfsetdash{}{0pt}%
\pgfpathmoveto{\pgfqpoint{1.014990in}{2.181057in}}%
\pgfpathlineto{\pgfqpoint{1.044991in}{2.429067in}}%
\pgfusepath{stroke}%
\end{pgfscope}%
\begin{pgfscope}%
\pgfpathrectangle{\pgfqpoint{0.100000in}{0.212622in}}{\pgfqpoint{3.696000in}{3.696000in}}%
\pgfusepath{clip}%
\pgfsetrectcap%
\pgfsetroundjoin%
\pgfsetlinewidth{1.505625pt}%
\definecolor{currentstroke}{rgb}{1.000000,0.000000,0.000000}%
\pgfsetstrokecolor{currentstroke}%
\pgfsetdash{}{0pt}%
\pgfpathmoveto{\pgfqpoint{1.021347in}{2.180729in}}%
\pgfpathlineto{\pgfqpoint{1.054299in}{2.436382in}}%
\pgfusepath{stroke}%
\end{pgfscope}%
\begin{pgfscope}%
\pgfpathrectangle{\pgfqpoint{0.100000in}{0.212622in}}{\pgfqpoint{3.696000in}{3.696000in}}%
\pgfusepath{clip}%
\pgfsetrectcap%
\pgfsetroundjoin%
\pgfsetlinewidth{1.505625pt}%
\definecolor{currentstroke}{rgb}{1.000000,0.000000,0.000000}%
\pgfsetstrokecolor{currentstroke}%
\pgfsetdash{}{0pt}%
\pgfpathmoveto{\pgfqpoint{1.022542in}{2.178827in}}%
\pgfpathlineto{\pgfqpoint{1.063595in}{2.443687in}}%
\pgfusepath{stroke}%
\end{pgfscope}%
\begin{pgfscope}%
\pgfpathrectangle{\pgfqpoint{0.100000in}{0.212622in}}{\pgfqpoint{3.696000in}{3.696000in}}%
\pgfusepath{clip}%
\pgfsetrectcap%
\pgfsetroundjoin%
\pgfsetlinewidth{1.505625pt}%
\definecolor{currentstroke}{rgb}{1.000000,0.000000,0.000000}%
\pgfsetstrokecolor{currentstroke}%
\pgfsetdash{}{0pt}%
\pgfpathmoveto{\pgfqpoint{1.031069in}{2.177818in}}%
\pgfpathlineto{\pgfqpoint{1.063595in}{2.443687in}}%
\pgfusepath{stroke}%
\end{pgfscope}%
\begin{pgfscope}%
\pgfpathrectangle{\pgfqpoint{0.100000in}{0.212622in}}{\pgfqpoint{3.696000in}{3.696000in}}%
\pgfusepath{clip}%
\pgfsetrectcap%
\pgfsetroundjoin%
\pgfsetlinewidth{1.505625pt}%
\definecolor{currentstroke}{rgb}{1.000000,0.000000,0.000000}%
\pgfsetstrokecolor{currentstroke}%
\pgfsetdash{}{0pt}%
\pgfpathmoveto{\pgfqpoint{1.037406in}{2.176857in}}%
\pgfpathlineto{\pgfqpoint{1.072878in}{2.450983in}}%
\pgfusepath{stroke}%
\end{pgfscope}%
\begin{pgfscope}%
\pgfpathrectangle{\pgfqpoint{0.100000in}{0.212622in}}{\pgfqpoint{3.696000in}{3.696000in}}%
\pgfusepath{clip}%
\pgfsetrectcap%
\pgfsetroundjoin%
\pgfsetlinewidth{1.505625pt}%
\definecolor{currentstroke}{rgb}{1.000000,0.000000,0.000000}%
\pgfsetstrokecolor{currentstroke}%
\pgfsetdash{}{0pt}%
\pgfpathmoveto{\pgfqpoint{1.042345in}{2.176489in}}%
\pgfpathlineto{\pgfqpoint{1.082149in}{2.458269in}}%
\pgfusepath{stroke}%
\end{pgfscope}%
\begin{pgfscope}%
\pgfpathrectangle{\pgfqpoint{0.100000in}{0.212622in}}{\pgfqpoint{3.696000in}{3.696000in}}%
\pgfusepath{clip}%
\pgfsetrectcap%
\pgfsetroundjoin%
\pgfsetlinewidth{1.505625pt}%
\definecolor{currentstroke}{rgb}{1.000000,0.000000,0.000000}%
\pgfsetstrokecolor{currentstroke}%
\pgfsetdash{}{0pt}%
\pgfpathmoveto{\pgfqpoint{1.047554in}{2.176146in}}%
\pgfpathlineto{\pgfqpoint{1.082149in}{2.458269in}}%
\pgfusepath{stroke}%
\end{pgfscope}%
\begin{pgfscope}%
\pgfpathrectangle{\pgfqpoint{0.100000in}{0.212622in}}{\pgfqpoint{3.696000in}{3.696000in}}%
\pgfusepath{clip}%
\pgfsetrectcap%
\pgfsetroundjoin%
\pgfsetlinewidth{1.505625pt}%
\definecolor{currentstroke}{rgb}{1.000000,0.000000,0.000000}%
\pgfsetstrokecolor{currentstroke}%
\pgfsetdash{}{0pt}%
\pgfpathmoveto{\pgfqpoint{1.050087in}{2.176153in}}%
\pgfpathlineto{\pgfqpoint{1.091408in}{2.465545in}}%
\pgfusepath{stroke}%
\end{pgfscope}%
\begin{pgfscope}%
\pgfpathrectangle{\pgfqpoint{0.100000in}{0.212622in}}{\pgfqpoint{3.696000in}{3.696000in}}%
\pgfusepath{clip}%
\pgfsetrectcap%
\pgfsetroundjoin%
\pgfsetlinewidth{1.505625pt}%
\definecolor{currentstroke}{rgb}{1.000000,0.000000,0.000000}%
\pgfsetstrokecolor{currentstroke}%
\pgfsetdash{}{0pt}%
\pgfpathmoveto{\pgfqpoint{1.053519in}{2.175901in}}%
\pgfpathlineto{\pgfqpoint{1.091408in}{2.465545in}}%
\pgfusepath{stroke}%
\end{pgfscope}%
\begin{pgfscope}%
\pgfpathrectangle{\pgfqpoint{0.100000in}{0.212622in}}{\pgfqpoint{3.696000in}{3.696000in}}%
\pgfusepath{clip}%
\pgfsetrectcap%
\pgfsetroundjoin%
\pgfsetlinewidth{1.505625pt}%
\definecolor{currentstroke}{rgb}{1.000000,0.000000,0.000000}%
\pgfsetstrokecolor{currentstroke}%
\pgfsetdash{}{0pt}%
\pgfpathmoveto{\pgfqpoint{1.057496in}{2.175745in}}%
\pgfpathlineto{\pgfqpoint{1.091408in}{2.465545in}}%
\pgfusepath{stroke}%
\end{pgfscope}%
\begin{pgfscope}%
\pgfpathrectangle{\pgfqpoint{0.100000in}{0.212622in}}{\pgfqpoint{3.696000in}{3.696000in}}%
\pgfusepath{clip}%
\pgfsetrectcap%
\pgfsetroundjoin%
\pgfsetlinewidth{1.505625pt}%
\definecolor{currentstroke}{rgb}{1.000000,0.000000,0.000000}%
\pgfsetstrokecolor{currentstroke}%
\pgfsetdash{}{0pt}%
\pgfpathmoveto{\pgfqpoint{1.063767in}{2.175050in}}%
\pgfpathlineto{\pgfqpoint{1.100656in}{2.472813in}}%
\pgfusepath{stroke}%
\end{pgfscope}%
\begin{pgfscope}%
\pgfpathrectangle{\pgfqpoint{0.100000in}{0.212622in}}{\pgfqpoint{3.696000in}{3.696000in}}%
\pgfusepath{clip}%
\pgfsetrectcap%
\pgfsetroundjoin%
\pgfsetlinewidth{1.505625pt}%
\definecolor{currentstroke}{rgb}{1.000000,0.000000,0.000000}%
\pgfsetstrokecolor{currentstroke}%
\pgfsetdash{}{0pt}%
\pgfpathmoveto{\pgfqpoint{1.068144in}{2.174569in}}%
\pgfpathlineto{\pgfqpoint{1.109891in}{2.480070in}}%
\pgfusepath{stroke}%
\end{pgfscope}%
\begin{pgfscope}%
\pgfpathrectangle{\pgfqpoint{0.100000in}{0.212622in}}{\pgfqpoint{3.696000in}{3.696000in}}%
\pgfusepath{clip}%
\pgfsetrectcap%
\pgfsetroundjoin%
\pgfsetlinewidth{1.505625pt}%
\definecolor{currentstroke}{rgb}{1.000000,0.000000,0.000000}%
\pgfsetstrokecolor{currentstroke}%
\pgfsetdash{}{0pt}%
\pgfpathmoveto{\pgfqpoint{1.075637in}{2.173963in}}%
\pgfpathlineto{\pgfqpoint{1.109891in}{2.480070in}}%
\pgfusepath{stroke}%
\end{pgfscope}%
\begin{pgfscope}%
\pgfpathrectangle{\pgfqpoint{0.100000in}{0.212622in}}{\pgfqpoint{3.696000in}{3.696000in}}%
\pgfusepath{clip}%
\pgfsetrectcap%
\pgfsetroundjoin%
\pgfsetlinewidth{1.505625pt}%
\definecolor{currentstroke}{rgb}{1.000000,0.000000,0.000000}%
\pgfsetstrokecolor{currentstroke}%
\pgfsetdash{}{0pt}%
\pgfpathmoveto{\pgfqpoint{1.082747in}{2.173381in}}%
\pgfpathlineto{\pgfqpoint{1.119114in}{2.487319in}}%
\pgfusepath{stroke}%
\end{pgfscope}%
\begin{pgfscope}%
\pgfpathrectangle{\pgfqpoint{0.100000in}{0.212622in}}{\pgfqpoint{3.696000in}{3.696000in}}%
\pgfusepath{clip}%
\pgfsetrectcap%
\pgfsetroundjoin%
\pgfsetlinewidth{1.505625pt}%
\definecolor{currentstroke}{rgb}{1.000000,0.000000,0.000000}%
\pgfsetstrokecolor{currentstroke}%
\pgfsetdash{}{0pt}%
\pgfpathmoveto{\pgfqpoint{1.086005in}{2.171341in}}%
\pgfpathlineto{\pgfqpoint{1.128325in}{2.494558in}}%
\pgfusepath{stroke}%
\end{pgfscope}%
\begin{pgfscope}%
\pgfpathrectangle{\pgfqpoint{0.100000in}{0.212622in}}{\pgfqpoint{3.696000in}{3.696000in}}%
\pgfusepath{clip}%
\pgfsetrectcap%
\pgfsetroundjoin%
\pgfsetlinewidth{1.505625pt}%
\definecolor{currentstroke}{rgb}{1.000000,0.000000,0.000000}%
\pgfsetstrokecolor{currentstroke}%
\pgfsetdash{}{0pt}%
\pgfpathmoveto{\pgfqpoint{1.090916in}{2.170692in}}%
\pgfpathlineto{\pgfqpoint{1.128325in}{2.494558in}}%
\pgfusepath{stroke}%
\end{pgfscope}%
\begin{pgfscope}%
\pgfpathrectangle{\pgfqpoint{0.100000in}{0.212622in}}{\pgfqpoint{3.696000in}{3.696000in}}%
\pgfusepath{clip}%
\pgfsetrectcap%
\pgfsetroundjoin%
\pgfsetlinewidth{1.505625pt}%
\definecolor{currentstroke}{rgb}{1.000000,0.000000,0.000000}%
\pgfsetstrokecolor{currentstroke}%
\pgfsetdash{}{0pt}%
\pgfpathmoveto{\pgfqpoint{1.092490in}{2.170520in}}%
\pgfpathlineto{\pgfqpoint{1.128325in}{2.494558in}}%
\pgfusepath{stroke}%
\end{pgfscope}%
\begin{pgfscope}%
\pgfpathrectangle{\pgfqpoint{0.100000in}{0.212622in}}{\pgfqpoint{3.696000in}{3.696000in}}%
\pgfusepath{clip}%
\pgfsetrectcap%
\pgfsetroundjoin%
\pgfsetlinewidth{1.505625pt}%
\definecolor{currentstroke}{rgb}{1.000000,0.000000,0.000000}%
\pgfsetstrokecolor{currentstroke}%
\pgfsetdash{}{0pt}%
\pgfpathmoveto{\pgfqpoint{1.093871in}{2.170356in}}%
\pgfpathlineto{\pgfqpoint{1.137525in}{2.501787in}}%
\pgfusepath{stroke}%
\end{pgfscope}%
\begin{pgfscope}%
\pgfpathrectangle{\pgfqpoint{0.100000in}{0.212622in}}{\pgfqpoint{3.696000in}{3.696000in}}%
\pgfusepath{clip}%
\pgfsetrectcap%
\pgfsetroundjoin%
\pgfsetlinewidth{1.505625pt}%
\definecolor{currentstroke}{rgb}{1.000000,0.000000,0.000000}%
\pgfsetstrokecolor{currentstroke}%
\pgfsetdash{}{0pt}%
\pgfpathmoveto{\pgfqpoint{1.096131in}{2.170469in}}%
\pgfpathlineto{\pgfqpoint{1.137525in}{2.501787in}}%
\pgfusepath{stroke}%
\end{pgfscope}%
\begin{pgfscope}%
\pgfpathrectangle{\pgfqpoint{0.100000in}{0.212622in}}{\pgfqpoint{3.696000in}{3.696000in}}%
\pgfusepath{clip}%
\pgfsetrectcap%
\pgfsetroundjoin%
\pgfsetlinewidth{1.505625pt}%
\definecolor{currentstroke}{rgb}{1.000000,0.000000,0.000000}%
\pgfsetstrokecolor{currentstroke}%
\pgfsetdash{}{0pt}%
\pgfpathmoveto{\pgfqpoint{1.097669in}{2.170388in}}%
\pgfpathlineto{\pgfqpoint{1.137525in}{2.501787in}}%
\pgfusepath{stroke}%
\end{pgfscope}%
\begin{pgfscope}%
\pgfpathrectangle{\pgfqpoint{0.100000in}{0.212622in}}{\pgfqpoint{3.696000in}{3.696000in}}%
\pgfusepath{clip}%
\pgfsetrectcap%
\pgfsetroundjoin%
\pgfsetlinewidth{1.505625pt}%
\definecolor{currentstroke}{rgb}{1.000000,0.000000,0.000000}%
\pgfsetstrokecolor{currentstroke}%
\pgfsetdash{}{0pt}%
\pgfpathmoveto{\pgfqpoint{1.098529in}{2.170329in}}%
\pgfpathlineto{\pgfqpoint{1.137525in}{2.501787in}}%
\pgfusepath{stroke}%
\end{pgfscope}%
\begin{pgfscope}%
\pgfpathrectangle{\pgfqpoint{0.100000in}{0.212622in}}{\pgfqpoint{3.696000in}{3.696000in}}%
\pgfusepath{clip}%
\pgfsetrectcap%
\pgfsetroundjoin%
\pgfsetlinewidth{1.505625pt}%
\definecolor{currentstroke}{rgb}{1.000000,0.000000,0.000000}%
\pgfsetstrokecolor{currentstroke}%
\pgfsetdash{}{0pt}%
\pgfpathmoveto{\pgfqpoint{1.097258in}{2.169792in}}%
\pgfpathlineto{\pgfqpoint{1.137525in}{2.501787in}}%
\pgfusepath{stroke}%
\end{pgfscope}%
\begin{pgfscope}%
\pgfpathrectangle{\pgfqpoint{0.100000in}{0.212622in}}{\pgfqpoint{3.696000in}{3.696000in}}%
\pgfusepath{clip}%
\pgfsetrectcap%
\pgfsetroundjoin%
\pgfsetlinewidth{1.505625pt}%
\definecolor{currentstroke}{rgb}{1.000000,0.000000,0.000000}%
\pgfsetstrokecolor{currentstroke}%
\pgfsetdash{}{0pt}%
\pgfpathmoveto{\pgfqpoint{1.097998in}{2.169680in}}%
\pgfpathlineto{\pgfqpoint{1.137525in}{2.501787in}}%
\pgfusepath{stroke}%
\end{pgfscope}%
\begin{pgfscope}%
\pgfpathrectangle{\pgfqpoint{0.100000in}{0.212622in}}{\pgfqpoint{3.696000in}{3.696000in}}%
\pgfusepath{clip}%
\pgfsetrectcap%
\pgfsetroundjoin%
\pgfsetlinewidth{1.505625pt}%
\definecolor{currentstroke}{rgb}{1.000000,0.000000,0.000000}%
\pgfsetstrokecolor{currentstroke}%
\pgfsetdash{}{0pt}%
\pgfpathmoveto{\pgfqpoint{1.099124in}{2.169512in}}%
\pgfpathlineto{\pgfqpoint{1.137525in}{2.501787in}}%
\pgfusepath{stroke}%
\end{pgfscope}%
\begin{pgfscope}%
\pgfpathrectangle{\pgfqpoint{0.100000in}{0.212622in}}{\pgfqpoint{3.696000in}{3.696000in}}%
\pgfusepath{clip}%
\pgfsetrectcap%
\pgfsetroundjoin%
\pgfsetlinewidth{1.505625pt}%
\definecolor{currentstroke}{rgb}{1.000000,0.000000,0.000000}%
\pgfsetstrokecolor{currentstroke}%
\pgfsetdash{}{0pt}%
\pgfpathmoveto{\pgfqpoint{1.100266in}{2.169391in}}%
\pgfpathlineto{\pgfqpoint{1.146712in}{2.509007in}}%
\pgfusepath{stroke}%
\end{pgfscope}%
\begin{pgfscope}%
\pgfpathrectangle{\pgfqpoint{0.100000in}{0.212622in}}{\pgfqpoint{3.696000in}{3.696000in}}%
\pgfusepath{clip}%
\pgfsetrectcap%
\pgfsetroundjoin%
\pgfsetlinewidth{1.505625pt}%
\definecolor{currentstroke}{rgb}{1.000000,0.000000,0.000000}%
\pgfsetstrokecolor{currentstroke}%
\pgfsetdash{}{0pt}%
\pgfpathmoveto{\pgfqpoint{1.100827in}{2.169358in}}%
\pgfpathlineto{\pgfqpoint{1.146712in}{2.509007in}}%
\pgfusepath{stroke}%
\end{pgfscope}%
\begin{pgfscope}%
\pgfpathrectangle{\pgfqpoint{0.100000in}{0.212622in}}{\pgfqpoint{3.696000in}{3.696000in}}%
\pgfusepath{clip}%
\pgfsetrectcap%
\pgfsetroundjoin%
\pgfsetlinewidth{1.505625pt}%
\definecolor{currentstroke}{rgb}{1.000000,0.000000,0.000000}%
\pgfsetstrokecolor{currentstroke}%
\pgfsetdash{}{0pt}%
\pgfpathmoveto{\pgfqpoint{1.101503in}{2.169318in}}%
\pgfpathlineto{\pgfqpoint{1.146712in}{2.509007in}}%
\pgfusepath{stroke}%
\end{pgfscope}%
\begin{pgfscope}%
\pgfpathrectangle{\pgfqpoint{0.100000in}{0.212622in}}{\pgfqpoint{3.696000in}{3.696000in}}%
\pgfusepath{clip}%
\pgfsetrectcap%
\pgfsetroundjoin%
\pgfsetlinewidth{1.505625pt}%
\definecolor{currentstroke}{rgb}{1.000000,0.000000,0.000000}%
\pgfsetstrokecolor{currentstroke}%
\pgfsetdash{}{0pt}%
\pgfpathmoveto{\pgfqpoint{1.102620in}{2.169182in}}%
\pgfpathlineto{\pgfqpoint{1.146712in}{2.509007in}}%
\pgfusepath{stroke}%
\end{pgfscope}%
\begin{pgfscope}%
\pgfpathrectangle{\pgfqpoint{0.100000in}{0.212622in}}{\pgfqpoint{3.696000in}{3.696000in}}%
\pgfusepath{clip}%
\pgfsetrectcap%
\pgfsetroundjoin%
\pgfsetlinewidth{1.505625pt}%
\definecolor{currentstroke}{rgb}{1.000000,0.000000,0.000000}%
\pgfsetstrokecolor{currentstroke}%
\pgfsetdash{}{0pt}%
\pgfpathmoveto{\pgfqpoint{1.103784in}{2.169185in}}%
\pgfpathlineto{\pgfqpoint{1.146712in}{2.509007in}}%
\pgfusepath{stroke}%
\end{pgfscope}%
\begin{pgfscope}%
\pgfpathrectangle{\pgfqpoint{0.100000in}{0.212622in}}{\pgfqpoint{3.696000in}{3.696000in}}%
\pgfusepath{clip}%
\pgfsetrectcap%
\pgfsetroundjoin%
\pgfsetlinewidth{1.505625pt}%
\definecolor{currentstroke}{rgb}{1.000000,0.000000,0.000000}%
\pgfsetstrokecolor{currentstroke}%
\pgfsetdash{}{0pt}%
\pgfpathmoveto{\pgfqpoint{1.106080in}{2.168892in}}%
\pgfpathlineto{\pgfqpoint{1.146712in}{2.509007in}}%
\pgfusepath{stroke}%
\end{pgfscope}%
\begin{pgfscope}%
\pgfpathrectangle{\pgfqpoint{0.100000in}{0.212622in}}{\pgfqpoint{3.696000in}{3.696000in}}%
\pgfusepath{clip}%
\pgfsetrectcap%
\pgfsetroundjoin%
\pgfsetlinewidth{1.505625pt}%
\definecolor{currentstroke}{rgb}{1.000000,0.000000,0.000000}%
\pgfsetstrokecolor{currentstroke}%
\pgfsetdash{}{0pt}%
\pgfpathmoveto{\pgfqpoint{1.108669in}{2.168852in}}%
\pgfpathlineto{\pgfqpoint{1.146712in}{2.509007in}}%
\pgfusepath{stroke}%
\end{pgfscope}%
\begin{pgfscope}%
\pgfpathrectangle{\pgfqpoint{0.100000in}{0.212622in}}{\pgfqpoint{3.696000in}{3.696000in}}%
\pgfusepath{clip}%
\pgfsetrectcap%
\pgfsetroundjoin%
\pgfsetlinewidth{1.505625pt}%
\definecolor{currentstroke}{rgb}{1.000000,0.000000,0.000000}%
\pgfsetstrokecolor{currentstroke}%
\pgfsetdash{}{0pt}%
\pgfpathmoveto{\pgfqpoint{1.110576in}{2.168765in}}%
\pgfpathlineto{\pgfqpoint{1.155887in}{2.516218in}}%
\pgfusepath{stroke}%
\end{pgfscope}%
\begin{pgfscope}%
\pgfpathrectangle{\pgfqpoint{0.100000in}{0.212622in}}{\pgfqpoint{3.696000in}{3.696000in}}%
\pgfusepath{clip}%
\pgfsetrectcap%
\pgfsetroundjoin%
\pgfsetlinewidth{1.505625pt}%
\definecolor{currentstroke}{rgb}{1.000000,0.000000,0.000000}%
\pgfsetstrokecolor{currentstroke}%
\pgfsetdash{}{0pt}%
\pgfpathmoveto{\pgfqpoint{1.111490in}{2.168719in}}%
\pgfpathlineto{\pgfqpoint{1.155887in}{2.516218in}}%
\pgfusepath{stroke}%
\end{pgfscope}%
\begin{pgfscope}%
\pgfpathrectangle{\pgfqpoint{0.100000in}{0.212622in}}{\pgfqpoint{3.696000in}{3.696000in}}%
\pgfusepath{clip}%
\pgfsetrectcap%
\pgfsetroundjoin%
\pgfsetlinewidth{1.505625pt}%
\definecolor{currentstroke}{rgb}{1.000000,0.000000,0.000000}%
\pgfsetstrokecolor{currentstroke}%
\pgfsetdash{}{0pt}%
\pgfpathmoveto{\pgfqpoint{1.113874in}{2.168689in}}%
\pgfpathlineto{\pgfqpoint{1.155887in}{2.516218in}}%
\pgfusepath{stroke}%
\end{pgfscope}%
\begin{pgfscope}%
\pgfpathrectangle{\pgfqpoint{0.100000in}{0.212622in}}{\pgfqpoint{3.696000in}{3.696000in}}%
\pgfusepath{clip}%
\pgfsetrectcap%
\pgfsetroundjoin%
\pgfsetlinewidth{1.505625pt}%
\definecolor{currentstroke}{rgb}{1.000000,0.000000,0.000000}%
\pgfsetstrokecolor{currentstroke}%
\pgfsetdash{}{0pt}%
\pgfpathmoveto{\pgfqpoint{1.117951in}{2.168307in}}%
\pgfpathlineto{\pgfqpoint{1.155887in}{2.516218in}}%
\pgfusepath{stroke}%
\end{pgfscope}%
\begin{pgfscope}%
\pgfpathrectangle{\pgfqpoint{0.100000in}{0.212622in}}{\pgfqpoint{3.696000in}{3.696000in}}%
\pgfusepath{clip}%
\pgfsetrectcap%
\pgfsetroundjoin%
\pgfsetlinewidth{1.505625pt}%
\definecolor{currentstroke}{rgb}{1.000000,0.000000,0.000000}%
\pgfsetstrokecolor{currentstroke}%
\pgfsetdash{}{0pt}%
\pgfpathmoveto{\pgfqpoint{1.120418in}{2.168052in}}%
\pgfpathlineto{\pgfqpoint{1.165051in}{2.523420in}}%
\pgfusepath{stroke}%
\end{pgfscope}%
\begin{pgfscope}%
\pgfpathrectangle{\pgfqpoint{0.100000in}{0.212622in}}{\pgfqpoint{3.696000in}{3.696000in}}%
\pgfusepath{clip}%
\pgfsetrectcap%
\pgfsetroundjoin%
\pgfsetlinewidth{1.505625pt}%
\definecolor{currentstroke}{rgb}{1.000000,0.000000,0.000000}%
\pgfsetstrokecolor{currentstroke}%
\pgfsetdash{}{0pt}%
\pgfpathmoveto{\pgfqpoint{1.124994in}{2.167812in}}%
\pgfpathlineto{\pgfqpoint{1.165051in}{2.523420in}}%
\pgfusepath{stroke}%
\end{pgfscope}%
\begin{pgfscope}%
\pgfpathrectangle{\pgfqpoint{0.100000in}{0.212622in}}{\pgfqpoint{3.696000in}{3.696000in}}%
\pgfusepath{clip}%
\pgfsetrectcap%
\pgfsetroundjoin%
\pgfsetlinewidth{1.505625pt}%
\definecolor{currentstroke}{rgb}{1.000000,0.000000,0.000000}%
\pgfsetstrokecolor{currentstroke}%
\pgfsetdash{}{0pt}%
\pgfpathmoveto{\pgfqpoint{1.127465in}{2.167612in}}%
\pgfpathlineto{\pgfqpoint{1.174203in}{2.530612in}}%
\pgfusepath{stroke}%
\end{pgfscope}%
\begin{pgfscope}%
\pgfpathrectangle{\pgfqpoint{0.100000in}{0.212622in}}{\pgfqpoint{3.696000in}{3.696000in}}%
\pgfusepath{clip}%
\pgfsetrectcap%
\pgfsetroundjoin%
\pgfsetlinewidth{1.505625pt}%
\definecolor{currentstroke}{rgb}{1.000000,0.000000,0.000000}%
\pgfsetstrokecolor{currentstroke}%
\pgfsetdash{}{0pt}%
\pgfpathmoveto{\pgfqpoint{1.123768in}{2.166247in}}%
\pgfpathlineto{\pgfqpoint{1.174203in}{2.530612in}}%
\pgfusepath{stroke}%
\end{pgfscope}%
\begin{pgfscope}%
\pgfpathrectangle{\pgfqpoint{0.100000in}{0.212622in}}{\pgfqpoint{3.696000in}{3.696000in}}%
\pgfusepath{clip}%
\pgfsetrectcap%
\pgfsetroundjoin%
\pgfsetlinewidth{1.505625pt}%
\definecolor{currentstroke}{rgb}{1.000000,0.000000,0.000000}%
\pgfsetstrokecolor{currentstroke}%
\pgfsetdash{}{0pt}%
\pgfpathmoveto{\pgfqpoint{1.127129in}{2.165927in}}%
\pgfpathlineto{\pgfqpoint{1.174203in}{2.530612in}}%
\pgfusepath{stroke}%
\end{pgfscope}%
\begin{pgfscope}%
\pgfpathrectangle{\pgfqpoint{0.100000in}{0.212622in}}{\pgfqpoint{3.696000in}{3.696000in}}%
\pgfusepath{clip}%
\pgfsetrectcap%
\pgfsetroundjoin%
\pgfsetlinewidth{1.505625pt}%
\definecolor{currentstroke}{rgb}{1.000000,0.000000,0.000000}%
\pgfsetstrokecolor{currentstroke}%
\pgfsetdash{}{0pt}%
\pgfpathmoveto{\pgfqpoint{1.129443in}{2.165643in}}%
\pgfpathlineto{\pgfqpoint{1.174203in}{2.530612in}}%
\pgfusepath{stroke}%
\end{pgfscope}%
\begin{pgfscope}%
\pgfpathrectangle{\pgfqpoint{0.100000in}{0.212622in}}{\pgfqpoint{3.696000in}{3.696000in}}%
\pgfusepath{clip}%
\pgfsetrectcap%
\pgfsetroundjoin%
\pgfsetlinewidth{1.505625pt}%
\definecolor{currentstroke}{rgb}{1.000000,0.000000,0.000000}%
\pgfsetstrokecolor{currentstroke}%
\pgfsetdash{}{0pt}%
\pgfpathmoveto{\pgfqpoint{1.131491in}{2.165502in}}%
\pgfpathlineto{\pgfqpoint{1.174203in}{2.530612in}}%
\pgfusepath{stroke}%
\end{pgfscope}%
\begin{pgfscope}%
\pgfpathrectangle{\pgfqpoint{0.100000in}{0.212622in}}{\pgfqpoint{3.696000in}{3.696000in}}%
\pgfusepath{clip}%
\pgfsetrectcap%
\pgfsetroundjoin%
\pgfsetlinewidth{1.505625pt}%
\definecolor{currentstroke}{rgb}{1.000000,0.000000,0.000000}%
\pgfsetstrokecolor{currentstroke}%
\pgfsetdash{}{0pt}%
\pgfpathmoveto{\pgfqpoint{1.133596in}{2.165311in}}%
\pgfpathlineto{\pgfqpoint{1.183343in}{2.537795in}}%
\pgfusepath{stroke}%
\end{pgfscope}%
\begin{pgfscope}%
\pgfpathrectangle{\pgfqpoint{0.100000in}{0.212622in}}{\pgfqpoint{3.696000in}{3.696000in}}%
\pgfusepath{clip}%
\pgfsetrectcap%
\pgfsetroundjoin%
\pgfsetlinewidth{1.505625pt}%
\definecolor{currentstroke}{rgb}{1.000000,0.000000,0.000000}%
\pgfsetstrokecolor{currentstroke}%
\pgfsetdash{}{0pt}%
\pgfpathmoveto{\pgfqpoint{1.134708in}{2.165370in}}%
\pgfpathlineto{\pgfqpoint{1.183343in}{2.537795in}}%
\pgfusepath{stroke}%
\end{pgfscope}%
\begin{pgfscope}%
\pgfpathrectangle{\pgfqpoint{0.100000in}{0.212622in}}{\pgfqpoint{3.696000in}{3.696000in}}%
\pgfusepath{clip}%
\pgfsetrectcap%
\pgfsetroundjoin%
\pgfsetlinewidth{1.505625pt}%
\definecolor{currentstroke}{rgb}{1.000000,0.000000,0.000000}%
\pgfsetstrokecolor{currentstroke}%
\pgfsetdash{}{0pt}%
\pgfpathmoveto{\pgfqpoint{1.136430in}{2.165213in}}%
\pgfpathlineto{\pgfqpoint{1.183343in}{2.537795in}}%
\pgfusepath{stroke}%
\end{pgfscope}%
\begin{pgfscope}%
\pgfpathrectangle{\pgfqpoint{0.100000in}{0.212622in}}{\pgfqpoint{3.696000in}{3.696000in}}%
\pgfusepath{clip}%
\pgfsetrectcap%
\pgfsetroundjoin%
\pgfsetlinewidth{1.505625pt}%
\definecolor{currentstroke}{rgb}{1.000000,0.000000,0.000000}%
\pgfsetstrokecolor{currentstroke}%
\pgfsetdash{}{0pt}%
\pgfpathmoveto{\pgfqpoint{1.138133in}{2.165092in}}%
\pgfpathlineto{\pgfqpoint{1.183343in}{2.537795in}}%
\pgfusepath{stroke}%
\end{pgfscope}%
\begin{pgfscope}%
\pgfpathrectangle{\pgfqpoint{0.100000in}{0.212622in}}{\pgfqpoint{3.696000in}{3.696000in}}%
\pgfusepath{clip}%
\pgfsetrectcap%
\pgfsetroundjoin%
\pgfsetlinewidth{1.505625pt}%
\definecolor{currentstroke}{rgb}{1.000000,0.000000,0.000000}%
\pgfsetstrokecolor{currentstroke}%
\pgfsetdash{}{0pt}%
\pgfpathmoveto{\pgfqpoint{1.141362in}{2.164728in}}%
\pgfpathlineto{\pgfqpoint{1.192471in}{2.544969in}}%
\pgfusepath{stroke}%
\end{pgfscope}%
\begin{pgfscope}%
\pgfpathrectangle{\pgfqpoint{0.100000in}{0.212622in}}{\pgfqpoint{3.696000in}{3.696000in}}%
\pgfusepath{clip}%
\pgfsetrectcap%
\pgfsetroundjoin%
\pgfsetlinewidth{1.505625pt}%
\definecolor{currentstroke}{rgb}{1.000000,0.000000,0.000000}%
\pgfsetstrokecolor{currentstroke}%
\pgfsetdash{}{0pt}%
\pgfpathmoveto{\pgfqpoint{1.144831in}{2.164724in}}%
\pgfpathlineto{\pgfqpoint{1.192471in}{2.544969in}}%
\pgfusepath{stroke}%
\end{pgfscope}%
\begin{pgfscope}%
\pgfpathrectangle{\pgfqpoint{0.100000in}{0.212622in}}{\pgfqpoint{3.696000in}{3.696000in}}%
\pgfusepath{clip}%
\pgfsetrectcap%
\pgfsetroundjoin%
\pgfsetlinewidth{1.505625pt}%
\definecolor{currentstroke}{rgb}{1.000000,0.000000,0.000000}%
\pgfsetstrokecolor{currentstroke}%
\pgfsetdash{}{0pt}%
\pgfpathmoveto{\pgfqpoint{1.149272in}{2.164605in}}%
\pgfpathlineto{\pgfqpoint{1.192471in}{2.544969in}}%
\pgfusepath{stroke}%
\end{pgfscope}%
\begin{pgfscope}%
\pgfpathrectangle{\pgfqpoint{0.100000in}{0.212622in}}{\pgfqpoint{3.696000in}{3.696000in}}%
\pgfusepath{clip}%
\pgfsetrectcap%
\pgfsetroundjoin%
\pgfsetlinewidth{1.505625pt}%
\definecolor{currentstroke}{rgb}{1.000000,0.000000,0.000000}%
\pgfsetstrokecolor{currentstroke}%
\pgfsetdash{}{0pt}%
\pgfpathmoveto{\pgfqpoint{1.151732in}{2.164483in}}%
\pgfpathlineto{\pgfqpoint{1.201587in}{2.552133in}}%
\pgfusepath{stroke}%
\end{pgfscope}%
\begin{pgfscope}%
\pgfpathrectangle{\pgfqpoint{0.100000in}{0.212622in}}{\pgfqpoint{3.696000in}{3.696000in}}%
\pgfusepath{clip}%
\pgfsetrectcap%
\pgfsetroundjoin%
\pgfsetlinewidth{1.505625pt}%
\definecolor{currentstroke}{rgb}{1.000000,0.000000,0.000000}%
\pgfsetstrokecolor{currentstroke}%
\pgfsetdash{}{0pt}%
\pgfpathmoveto{\pgfqpoint{1.154482in}{2.164341in}}%
\pgfpathlineto{\pgfqpoint{1.201587in}{2.552133in}}%
\pgfusepath{stroke}%
\end{pgfscope}%
\begin{pgfscope}%
\pgfpathrectangle{\pgfqpoint{0.100000in}{0.212622in}}{\pgfqpoint{3.696000in}{3.696000in}}%
\pgfusepath{clip}%
\pgfsetrectcap%
\pgfsetroundjoin%
\pgfsetlinewidth{1.505625pt}%
\definecolor{currentstroke}{rgb}{1.000000,0.000000,0.000000}%
\pgfsetstrokecolor{currentstroke}%
\pgfsetdash{}{0pt}%
\pgfpathmoveto{\pgfqpoint{1.158982in}{2.163975in}}%
\pgfpathlineto{\pgfqpoint{1.210692in}{2.559288in}}%
\pgfusepath{stroke}%
\end{pgfscope}%
\begin{pgfscope}%
\pgfpathrectangle{\pgfqpoint{0.100000in}{0.212622in}}{\pgfqpoint{3.696000in}{3.696000in}}%
\pgfusepath{clip}%
\pgfsetrectcap%
\pgfsetroundjoin%
\pgfsetlinewidth{1.505625pt}%
\definecolor{currentstroke}{rgb}{1.000000,0.000000,0.000000}%
\pgfsetstrokecolor{currentstroke}%
\pgfsetdash{}{0pt}%
\pgfpathmoveto{\pgfqpoint{1.162814in}{2.163864in}}%
\pgfpathlineto{\pgfqpoint{1.210692in}{2.559288in}}%
\pgfusepath{stroke}%
\end{pgfscope}%
\begin{pgfscope}%
\pgfpathrectangle{\pgfqpoint{0.100000in}{0.212622in}}{\pgfqpoint{3.696000in}{3.696000in}}%
\pgfusepath{clip}%
\pgfsetrectcap%
\pgfsetroundjoin%
\pgfsetlinewidth{1.505625pt}%
\definecolor{currentstroke}{rgb}{1.000000,0.000000,0.000000}%
\pgfsetstrokecolor{currentstroke}%
\pgfsetdash{}{0pt}%
\pgfpathmoveto{\pgfqpoint{1.169360in}{2.163401in}}%
\pgfpathlineto{\pgfqpoint{1.219785in}{2.566434in}}%
\pgfusepath{stroke}%
\end{pgfscope}%
\begin{pgfscope}%
\pgfpathrectangle{\pgfqpoint{0.100000in}{0.212622in}}{\pgfqpoint{3.696000in}{3.696000in}}%
\pgfusepath{clip}%
\pgfsetrectcap%
\pgfsetroundjoin%
\pgfsetlinewidth{1.505625pt}%
\definecolor{currentstroke}{rgb}{1.000000,0.000000,0.000000}%
\pgfsetstrokecolor{currentstroke}%
\pgfsetdash{}{0pt}%
\pgfpathmoveto{\pgfqpoint{1.172918in}{2.163090in}}%
\pgfpathlineto{\pgfqpoint{1.219785in}{2.566434in}}%
\pgfusepath{stroke}%
\end{pgfscope}%
\begin{pgfscope}%
\pgfpathrectangle{\pgfqpoint{0.100000in}{0.212622in}}{\pgfqpoint{3.696000in}{3.696000in}}%
\pgfusepath{clip}%
\pgfsetrectcap%
\pgfsetroundjoin%
\pgfsetlinewidth{1.505625pt}%
\definecolor{currentstroke}{rgb}{1.000000,0.000000,0.000000}%
\pgfsetstrokecolor{currentstroke}%
\pgfsetdash{}{0pt}%
\pgfpathmoveto{\pgfqpoint{1.170733in}{2.162178in}}%
\pgfpathlineto{\pgfqpoint{1.219785in}{2.566434in}}%
\pgfusepath{stroke}%
\end{pgfscope}%
\begin{pgfscope}%
\pgfpathrectangle{\pgfqpoint{0.100000in}{0.212622in}}{\pgfqpoint{3.696000in}{3.696000in}}%
\pgfusepath{clip}%
\pgfsetrectcap%
\pgfsetroundjoin%
\pgfsetlinewidth{1.505625pt}%
\definecolor{currentstroke}{rgb}{1.000000,0.000000,0.000000}%
\pgfsetstrokecolor{currentstroke}%
\pgfsetdash{}{0pt}%
\pgfpathmoveto{\pgfqpoint{1.174005in}{2.161704in}}%
\pgfpathlineto{\pgfqpoint{1.228866in}{2.573571in}}%
\pgfusepath{stroke}%
\end{pgfscope}%
\begin{pgfscope}%
\pgfpathrectangle{\pgfqpoint{0.100000in}{0.212622in}}{\pgfqpoint{3.696000in}{3.696000in}}%
\pgfusepath{clip}%
\pgfsetrectcap%
\pgfsetroundjoin%
\pgfsetlinewidth{1.505625pt}%
\definecolor{currentstroke}{rgb}{1.000000,0.000000,0.000000}%
\pgfsetstrokecolor{currentstroke}%
\pgfsetdash{}{0pt}%
\pgfpathmoveto{\pgfqpoint{1.175445in}{2.161404in}}%
\pgfpathlineto{\pgfqpoint{1.228866in}{2.573571in}}%
\pgfusepath{stroke}%
\end{pgfscope}%
\begin{pgfscope}%
\pgfpathrectangle{\pgfqpoint{0.100000in}{0.212622in}}{\pgfqpoint{3.696000in}{3.696000in}}%
\pgfusepath{clip}%
\pgfsetrectcap%
\pgfsetroundjoin%
\pgfsetlinewidth{1.505625pt}%
\definecolor{currentstroke}{rgb}{1.000000,0.000000,0.000000}%
\pgfsetstrokecolor{currentstroke}%
\pgfsetdash{}{0pt}%
\pgfpathmoveto{\pgfqpoint{1.180140in}{2.160984in}}%
\pgfpathlineto{\pgfqpoint{1.228866in}{2.573571in}}%
\pgfusepath{stroke}%
\end{pgfscope}%
\begin{pgfscope}%
\pgfpathrectangle{\pgfqpoint{0.100000in}{0.212622in}}{\pgfqpoint{3.696000in}{3.696000in}}%
\pgfusepath{clip}%
\pgfsetrectcap%
\pgfsetroundjoin%
\pgfsetlinewidth{1.505625pt}%
\definecolor{currentstroke}{rgb}{1.000000,0.000000,0.000000}%
\pgfsetstrokecolor{currentstroke}%
\pgfsetdash{}{0pt}%
\pgfpathmoveto{\pgfqpoint{1.182108in}{2.160939in}}%
\pgfpathlineto{\pgfqpoint{1.237935in}{2.580698in}}%
\pgfusepath{stroke}%
\end{pgfscope}%
\begin{pgfscope}%
\pgfpathrectangle{\pgfqpoint{0.100000in}{0.212622in}}{\pgfqpoint{3.696000in}{3.696000in}}%
\pgfusepath{clip}%
\pgfsetrectcap%
\pgfsetroundjoin%
\pgfsetlinewidth{1.505625pt}%
\definecolor{currentstroke}{rgb}{1.000000,0.000000,0.000000}%
\pgfsetstrokecolor{currentstroke}%
\pgfsetdash{}{0pt}%
\pgfpathmoveto{\pgfqpoint{1.184292in}{2.160848in}}%
\pgfpathlineto{\pgfqpoint{1.237935in}{2.580698in}}%
\pgfusepath{stroke}%
\end{pgfscope}%
\begin{pgfscope}%
\pgfpathrectangle{\pgfqpoint{0.100000in}{0.212622in}}{\pgfqpoint{3.696000in}{3.696000in}}%
\pgfusepath{clip}%
\pgfsetrectcap%
\pgfsetroundjoin%
\pgfsetlinewidth{1.505625pt}%
\definecolor{currentstroke}{rgb}{1.000000,0.000000,0.000000}%
\pgfsetstrokecolor{currentstroke}%
\pgfsetdash{}{0pt}%
\pgfpathmoveto{\pgfqpoint{1.186146in}{2.160580in}}%
\pgfpathlineto{\pgfqpoint{1.237935in}{2.580698in}}%
\pgfusepath{stroke}%
\end{pgfscope}%
\begin{pgfscope}%
\pgfpathrectangle{\pgfqpoint{0.100000in}{0.212622in}}{\pgfqpoint{3.696000in}{3.696000in}}%
\pgfusepath{clip}%
\pgfsetrectcap%
\pgfsetroundjoin%
\pgfsetlinewidth{1.505625pt}%
\definecolor{currentstroke}{rgb}{1.000000,0.000000,0.000000}%
\pgfsetstrokecolor{currentstroke}%
\pgfsetdash{}{0pt}%
\pgfpathmoveto{\pgfqpoint{1.187636in}{2.160429in}}%
\pgfpathlineto{\pgfqpoint{1.237935in}{2.580698in}}%
\pgfusepath{stroke}%
\end{pgfscope}%
\begin{pgfscope}%
\pgfpathrectangle{\pgfqpoint{0.100000in}{0.212622in}}{\pgfqpoint{3.696000in}{3.696000in}}%
\pgfusepath{clip}%
\pgfsetrectcap%
\pgfsetroundjoin%
\pgfsetlinewidth{1.505625pt}%
\definecolor{currentstroke}{rgb}{1.000000,0.000000,0.000000}%
\pgfsetstrokecolor{currentstroke}%
\pgfsetdash{}{0pt}%
\pgfpathmoveto{\pgfqpoint{1.191393in}{2.159937in}}%
\pgfpathlineto{\pgfqpoint{1.246993in}{2.587816in}}%
\pgfusepath{stroke}%
\end{pgfscope}%
\begin{pgfscope}%
\pgfpathrectangle{\pgfqpoint{0.100000in}{0.212622in}}{\pgfqpoint{3.696000in}{3.696000in}}%
\pgfusepath{clip}%
\pgfsetrectcap%
\pgfsetroundjoin%
\pgfsetlinewidth{1.505625pt}%
\definecolor{currentstroke}{rgb}{1.000000,0.000000,0.000000}%
\pgfsetstrokecolor{currentstroke}%
\pgfsetdash{}{0pt}%
\pgfpathmoveto{\pgfqpoint{1.194690in}{2.159825in}}%
\pgfpathlineto{\pgfqpoint{1.246993in}{2.587816in}}%
\pgfusepath{stroke}%
\end{pgfscope}%
\begin{pgfscope}%
\pgfpathrectangle{\pgfqpoint{0.100000in}{0.212622in}}{\pgfqpoint{3.696000in}{3.696000in}}%
\pgfusepath{clip}%
\pgfsetrectcap%
\pgfsetroundjoin%
\pgfsetlinewidth{1.505625pt}%
\definecolor{currentstroke}{rgb}{1.000000,0.000000,0.000000}%
\pgfsetstrokecolor{currentstroke}%
\pgfsetdash{}{0pt}%
\pgfpathmoveto{\pgfqpoint{1.196855in}{2.159723in}}%
\pgfpathlineto{\pgfqpoint{1.246993in}{2.587816in}}%
\pgfusepath{stroke}%
\end{pgfscope}%
\begin{pgfscope}%
\pgfpathrectangle{\pgfqpoint{0.100000in}{0.212622in}}{\pgfqpoint{3.696000in}{3.696000in}}%
\pgfusepath{clip}%
\pgfsetrectcap%
\pgfsetroundjoin%
\pgfsetlinewidth{1.505625pt}%
\definecolor{currentstroke}{rgb}{1.000000,0.000000,0.000000}%
\pgfsetstrokecolor{currentstroke}%
\pgfsetdash{}{0pt}%
\pgfpathmoveto{\pgfqpoint{1.198042in}{2.159647in}}%
\pgfpathlineto{\pgfqpoint{1.246993in}{2.587816in}}%
\pgfusepath{stroke}%
\end{pgfscope}%
\begin{pgfscope}%
\pgfpathrectangle{\pgfqpoint{0.100000in}{0.212622in}}{\pgfqpoint{3.696000in}{3.696000in}}%
\pgfusepath{clip}%
\pgfsetrectcap%
\pgfsetroundjoin%
\pgfsetlinewidth{1.505625pt}%
\definecolor{currentstroke}{rgb}{1.000000,0.000000,0.000000}%
\pgfsetstrokecolor{currentstroke}%
\pgfsetdash{}{0pt}%
\pgfpathmoveto{\pgfqpoint{1.197233in}{2.159332in}}%
\pgfpathlineto{\pgfqpoint{1.246993in}{2.587816in}}%
\pgfusepath{stroke}%
\end{pgfscope}%
\begin{pgfscope}%
\pgfpathrectangle{\pgfqpoint{0.100000in}{0.212622in}}{\pgfqpoint{3.696000in}{3.696000in}}%
\pgfusepath{clip}%
\pgfsetrectcap%
\pgfsetroundjoin%
\pgfsetlinewidth{1.505625pt}%
\definecolor{currentstroke}{rgb}{1.000000,0.000000,0.000000}%
\pgfsetstrokecolor{currentstroke}%
\pgfsetdash{}{0pt}%
\pgfpathmoveto{\pgfqpoint{1.199207in}{2.159105in}}%
\pgfpathlineto{\pgfqpoint{1.256039in}{2.594926in}}%
\pgfusepath{stroke}%
\end{pgfscope}%
\begin{pgfscope}%
\pgfpathrectangle{\pgfqpoint{0.100000in}{0.212622in}}{\pgfqpoint{3.696000in}{3.696000in}}%
\pgfusepath{clip}%
\pgfsetrectcap%
\pgfsetroundjoin%
\pgfsetlinewidth{1.505625pt}%
\definecolor{currentstroke}{rgb}{1.000000,0.000000,0.000000}%
\pgfsetstrokecolor{currentstroke}%
\pgfsetdash{}{0pt}%
\pgfpathmoveto{\pgfqpoint{1.200504in}{2.158942in}}%
\pgfpathlineto{\pgfqpoint{1.256039in}{2.594926in}}%
\pgfusepath{stroke}%
\end{pgfscope}%
\begin{pgfscope}%
\pgfpathrectangle{\pgfqpoint{0.100000in}{0.212622in}}{\pgfqpoint{3.696000in}{3.696000in}}%
\pgfusepath{clip}%
\pgfsetrectcap%
\pgfsetroundjoin%
\pgfsetlinewidth{1.505625pt}%
\definecolor{currentstroke}{rgb}{1.000000,0.000000,0.000000}%
\pgfsetstrokecolor{currentstroke}%
\pgfsetdash{}{0pt}%
\pgfpathmoveto{\pgfqpoint{1.204318in}{2.158639in}}%
\pgfpathlineto{\pgfqpoint{1.256039in}{2.594926in}}%
\pgfusepath{stroke}%
\end{pgfscope}%
\begin{pgfscope}%
\pgfpathrectangle{\pgfqpoint{0.100000in}{0.212622in}}{\pgfqpoint{3.696000in}{3.696000in}}%
\pgfusepath{clip}%
\pgfsetrectcap%
\pgfsetroundjoin%
\pgfsetlinewidth{1.505625pt}%
\definecolor{currentstroke}{rgb}{1.000000,0.000000,0.000000}%
\pgfsetstrokecolor{currentstroke}%
\pgfsetdash{}{0pt}%
\pgfpathmoveto{\pgfqpoint{1.207274in}{2.158806in}}%
\pgfpathlineto{\pgfqpoint{1.265073in}{2.602026in}}%
\pgfusepath{stroke}%
\end{pgfscope}%
\begin{pgfscope}%
\pgfpathrectangle{\pgfqpoint{0.100000in}{0.212622in}}{\pgfqpoint{3.696000in}{3.696000in}}%
\pgfusepath{clip}%
\pgfsetrectcap%
\pgfsetroundjoin%
\pgfsetlinewidth{1.505625pt}%
\definecolor{currentstroke}{rgb}{1.000000,0.000000,0.000000}%
\pgfsetstrokecolor{currentstroke}%
\pgfsetdash{}{0pt}%
\pgfpathmoveto{\pgfqpoint{1.211066in}{2.158693in}}%
\pgfpathlineto{\pgfqpoint{1.265073in}{2.602026in}}%
\pgfusepath{stroke}%
\end{pgfscope}%
\begin{pgfscope}%
\pgfpathrectangle{\pgfqpoint{0.100000in}{0.212622in}}{\pgfqpoint{3.696000in}{3.696000in}}%
\pgfusepath{clip}%
\pgfsetrectcap%
\pgfsetroundjoin%
\pgfsetlinewidth{1.505625pt}%
\definecolor{currentstroke}{rgb}{1.000000,0.000000,0.000000}%
\pgfsetstrokecolor{currentstroke}%
\pgfsetdash{}{0pt}%
\pgfpathmoveto{\pgfqpoint{1.214925in}{2.158518in}}%
\pgfpathlineto{\pgfqpoint{1.265073in}{2.602026in}}%
\pgfusepath{stroke}%
\end{pgfscope}%
\begin{pgfscope}%
\pgfpathrectangle{\pgfqpoint{0.100000in}{0.212622in}}{\pgfqpoint{3.696000in}{3.696000in}}%
\pgfusepath{clip}%
\pgfsetrectcap%
\pgfsetroundjoin%
\pgfsetlinewidth{1.505625pt}%
\definecolor{currentstroke}{rgb}{1.000000,0.000000,0.000000}%
\pgfsetstrokecolor{currentstroke}%
\pgfsetdash{}{0pt}%
\pgfpathmoveto{\pgfqpoint{1.220149in}{2.158131in}}%
\pgfpathlineto{\pgfqpoint{1.274096in}{2.609117in}}%
\pgfusepath{stroke}%
\end{pgfscope}%
\begin{pgfscope}%
\pgfpathrectangle{\pgfqpoint{0.100000in}{0.212622in}}{\pgfqpoint{3.696000in}{3.696000in}}%
\pgfusepath{clip}%
\pgfsetrectcap%
\pgfsetroundjoin%
\pgfsetlinewidth{1.505625pt}%
\definecolor{currentstroke}{rgb}{1.000000,0.000000,0.000000}%
\pgfsetstrokecolor{currentstroke}%
\pgfsetdash{}{0pt}%
\pgfpathmoveto{\pgfqpoint{1.227444in}{2.157247in}}%
\pgfpathlineto{\pgfqpoint{1.283108in}{2.616198in}}%
\pgfusepath{stroke}%
\end{pgfscope}%
\begin{pgfscope}%
\pgfpathrectangle{\pgfqpoint{0.100000in}{0.212622in}}{\pgfqpoint{3.696000in}{3.696000in}}%
\pgfusepath{clip}%
\pgfsetrectcap%
\pgfsetroundjoin%
\pgfsetlinewidth{1.505625pt}%
\definecolor{currentstroke}{rgb}{1.000000,0.000000,0.000000}%
\pgfsetstrokecolor{currentstroke}%
\pgfsetdash{}{0pt}%
\pgfpathmoveto{\pgfqpoint{1.231388in}{2.156859in}}%
\pgfpathlineto{\pgfqpoint{1.283108in}{2.616198in}}%
\pgfusepath{stroke}%
\end{pgfscope}%
\begin{pgfscope}%
\pgfpathrectangle{\pgfqpoint{0.100000in}{0.212622in}}{\pgfqpoint{3.696000in}{3.696000in}}%
\pgfusepath{clip}%
\pgfsetrectcap%
\pgfsetroundjoin%
\pgfsetlinewidth{1.505625pt}%
\definecolor{currentstroke}{rgb}{1.000000,0.000000,0.000000}%
\pgfsetstrokecolor{currentstroke}%
\pgfsetdash{}{0pt}%
\pgfpathmoveto{\pgfqpoint{1.238972in}{2.156711in}}%
\pgfpathlineto{\pgfqpoint{1.292107in}{2.623271in}}%
\pgfusepath{stroke}%
\end{pgfscope}%
\begin{pgfscope}%
\pgfpathrectangle{\pgfqpoint{0.100000in}{0.212622in}}{\pgfqpoint{3.696000in}{3.696000in}}%
\pgfusepath{clip}%
\pgfsetrectcap%
\pgfsetroundjoin%
\pgfsetlinewidth{1.505625pt}%
\definecolor{currentstroke}{rgb}{1.000000,0.000000,0.000000}%
\pgfsetstrokecolor{currentstroke}%
\pgfsetdash{}{0pt}%
\pgfpathmoveto{\pgfqpoint{1.243206in}{2.156562in}}%
\pgfpathlineto{\pgfqpoint{1.301096in}{2.630335in}}%
\pgfusepath{stroke}%
\end{pgfscope}%
\begin{pgfscope}%
\pgfpathrectangle{\pgfqpoint{0.100000in}{0.212622in}}{\pgfqpoint{3.696000in}{3.696000in}}%
\pgfusepath{clip}%
\pgfsetrectcap%
\pgfsetroundjoin%
\pgfsetlinewidth{1.505625pt}%
\definecolor{currentstroke}{rgb}{1.000000,0.000000,0.000000}%
\pgfsetstrokecolor{currentstroke}%
\pgfsetdash{}{0pt}%
\pgfpathmoveto{\pgfqpoint{1.245080in}{2.156448in}}%
\pgfpathlineto{\pgfqpoint{1.301096in}{2.630335in}}%
\pgfusepath{stroke}%
\end{pgfscope}%
\begin{pgfscope}%
\pgfpathrectangle{\pgfqpoint{0.100000in}{0.212622in}}{\pgfqpoint{3.696000in}{3.696000in}}%
\pgfusepath{clip}%
\pgfsetrectcap%
\pgfsetroundjoin%
\pgfsetlinewidth{1.505625pt}%
\definecolor{currentstroke}{rgb}{1.000000,0.000000,0.000000}%
\pgfsetstrokecolor{currentstroke}%
\pgfsetdash{}{0pt}%
\pgfpathmoveto{\pgfqpoint{1.246503in}{2.156310in}}%
\pgfpathlineto{\pgfqpoint{1.301096in}{2.630335in}}%
\pgfusepath{stroke}%
\end{pgfscope}%
\begin{pgfscope}%
\pgfpathrectangle{\pgfqpoint{0.100000in}{0.212622in}}{\pgfqpoint{3.696000in}{3.696000in}}%
\pgfusepath{clip}%
\pgfsetrectcap%
\pgfsetroundjoin%
\pgfsetlinewidth{1.505625pt}%
\definecolor{currentstroke}{rgb}{1.000000,0.000000,0.000000}%
\pgfsetstrokecolor{currentstroke}%
\pgfsetdash{}{0pt}%
\pgfpathmoveto{\pgfqpoint{1.246902in}{2.156305in}}%
\pgfpathlineto{\pgfqpoint{1.301096in}{2.630335in}}%
\pgfusepath{stroke}%
\end{pgfscope}%
\begin{pgfscope}%
\pgfpathrectangle{\pgfqpoint{0.100000in}{0.212622in}}{\pgfqpoint{3.696000in}{3.696000in}}%
\pgfusepath{clip}%
\pgfsetrectcap%
\pgfsetroundjoin%
\pgfsetlinewidth{1.505625pt}%
\definecolor{currentstroke}{rgb}{1.000000,0.000000,0.000000}%
\pgfsetstrokecolor{currentstroke}%
\pgfsetdash{}{0pt}%
\pgfpathmoveto{\pgfqpoint{1.247845in}{2.156246in}}%
\pgfpathlineto{\pgfqpoint{1.301096in}{2.630335in}}%
\pgfusepath{stroke}%
\end{pgfscope}%
\begin{pgfscope}%
\pgfpathrectangle{\pgfqpoint{0.100000in}{0.212622in}}{\pgfqpoint{3.696000in}{3.696000in}}%
\pgfusepath{clip}%
\pgfsetrectcap%
\pgfsetroundjoin%
\pgfsetlinewidth{1.505625pt}%
\definecolor{currentstroke}{rgb}{1.000000,0.000000,0.000000}%
\pgfsetstrokecolor{currentstroke}%
\pgfsetdash{}{0pt}%
\pgfpathmoveto{\pgfqpoint{1.249626in}{2.156265in}}%
\pgfpathlineto{\pgfqpoint{1.301096in}{2.630335in}}%
\pgfusepath{stroke}%
\end{pgfscope}%
\begin{pgfscope}%
\pgfpathrectangle{\pgfqpoint{0.100000in}{0.212622in}}{\pgfqpoint{3.696000in}{3.696000in}}%
\pgfusepath{clip}%
\pgfsetrectcap%
\pgfsetroundjoin%
\pgfsetlinewidth{1.505625pt}%
\definecolor{currentstroke}{rgb}{1.000000,0.000000,0.000000}%
\pgfsetstrokecolor{currentstroke}%
\pgfsetdash{}{0pt}%
\pgfpathmoveto{\pgfqpoint{1.251534in}{2.156202in}}%
\pgfpathlineto{\pgfqpoint{1.310072in}{2.637389in}}%
\pgfusepath{stroke}%
\end{pgfscope}%
\begin{pgfscope}%
\pgfpathrectangle{\pgfqpoint{0.100000in}{0.212622in}}{\pgfqpoint{3.696000in}{3.696000in}}%
\pgfusepath{clip}%
\pgfsetrectcap%
\pgfsetroundjoin%
\pgfsetlinewidth{1.505625pt}%
\definecolor{currentstroke}{rgb}{1.000000,0.000000,0.000000}%
\pgfsetstrokecolor{currentstroke}%
\pgfsetdash{}{0pt}%
\pgfpathmoveto{\pgfqpoint{1.252799in}{2.156152in}}%
\pgfpathlineto{\pgfqpoint{1.310072in}{2.637389in}}%
\pgfusepath{stroke}%
\end{pgfscope}%
\begin{pgfscope}%
\pgfpathrectangle{\pgfqpoint{0.100000in}{0.212622in}}{\pgfqpoint{3.696000in}{3.696000in}}%
\pgfusepath{clip}%
\pgfsetrectcap%
\pgfsetroundjoin%
\pgfsetlinewidth{1.505625pt}%
\definecolor{currentstroke}{rgb}{1.000000,0.000000,0.000000}%
\pgfsetstrokecolor{currentstroke}%
\pgfsetdash{}{0pt}%
\pgfpathmoveto{\pgfqpoint{1.254706in}{2.156181in}}%
\pgfpathlineto{\pgfqpoint{1.310072in}{2.637389in}}%
\pgfusepath{stroke}%
\end{pgfscope}%
\begin{pgfscope}%
\pgfpathrectangle{\pgfqpoint{0.100000in}{0.212622in}}{\pgfqpoint{3.696000in}{3.696000in}}%
\pgfusepath{clip}%
\pgfsetrectcap%
\pgfsetroundjoin%
\pgfsetlinewidth{1.505625pt}%
\definecolor{currentstroke}{rgb}{1.000000,0.000000,0.000000}%
\pgfsetstrokecolor{currentstroke}%
\pgfsetdash{}{0pt}%
\pgfpathmoveto{\pgfqpoint{1.258651in}{2.155877in}}%
\pgfpathlineto{\pgfqpoint{1.310072in}{2.637389in}}%
\pgfusepath{stroke}%
\end{pgfscope}%
\begin{pgfscope}%
\pgfpathrectangle{\pgfqpoint{0.100000in}{0.212622in}}{\pgfqpoint{3.696000in}{3.696000in}}%
\pgfusepath{clip}%
\pgfsetrectcap%
\pgfsetroundjoin%
\pgfsetlinewidth{1.505625pt}%
\definecolor{currentstroke}{rgb}{1.000000,0.000000,0.000000}%
\pgfsetstrokecolor{currentstroke}%
\pgfsetdash{}{0pt}%
\pgfpathmoveto{\pgfqpoint{1.261662in}{2.155904in}}%
\pgfpathlineto{\pgfqpoint{1.319037in}{2.644435in}}%
\pgfusepath{stroke}%
\end{pgfscope}%
\begin{pgfscope}%
\pgfpathrectangle{\pgfqpoint{0.100000in}{0.212622in}}{\pgfqpoint{3.696000in}{3.696000in}}%
\pgfusepath{clip}%
\pgfsetrectcap%
\pgfsetroundjoin%
\pgfsetlinewidth{1.505625pt}%
\definecolor{currentstroke}{rgb}{1.000000,0.000000,0.000000}%
\pgfsetstrokecolor{currentstroke}%
\pgfsetdash{}{0pt}%
\pgfpathmoveto{\pgfqpoint{1.266772in}{2.155629in}}%
\pgfpathlineto{\pgfqpoint{1.319037in}{2.644435in}}%
\pgfusepath{stroke}%
\end{pgfscope}%
\begin{pgfscope}%
\pgfpathrectangle{\pgfqpoint{0.100000in}{0.212622in}}{\pgfqpoint{3.696000in}{3.696000in}}%
\pgfusepath{clip}%
\pgfsetrectcap%
\pgfsetroundjoin%
\pgfsetlinewidth{1.505625pt}%
\definecolor{currentstroke}{rgb}{1.000000,0.000000,0.000000}%
\pgfsetstrokecolor{currentstroke}%
\pgfsetdash{}{0pt}%
\pgfpathmoveto{\pgfqpoint{1.269324in}{2.155555in}}%
\pgfpathlineto{\pgfqpoint{1.327991in}{2.651472in}}%
\pgfusepath{stroke}%
\end{pgfscope}%
\begin{pgfscope}%
\pgfpathrectangle{\pgfqpoint{0.100000in}{0.212622in}}{\pgfqpoint{3.696000in}{3.696000in}}%
\pgfusepath{clip}%
\pgfsetrectcap%
\pgfsetroundjoin%
\pgfsetlinewidth{1.505625pt}%
\definecolor{currentstroke}{rgb}{1.000000,0.000000,0.000000}%
\pgfsetstrokecolor{currentstroke}%
\pgfsetdash{}{0pt}%
\pgfpathmoveto{\pgfqpoint{1.273279in}{2.155739in}}%
\pgfpathlineto{\pgfqpoint{1.327991in}{2.651472in}}%
\pgfusepath{stroke}%
\end{pgfscope}%
\begin{pgfscope}%
\pgfpathrectangle{\pgfqpoint{0.100000in}{0.212622in}}{\pgfqpoint{3.696000in}{3.696000in}}%
\pgfusepath{clip}%
\pgfsetrectcap%
\pgfsetroundjoin%
\pgfsetlinewidth{1.505625pt}%
\definecolor{currentstroke}{rgb}{1.000000,0.000000,0.000000}%
\pgfsetstrokecolor{currentstroke}%
\pgfsetdash{}{0pt}%
\pgfpathmoveto{\pgfqpoint{1.279051in}{2.155282in}}%
\pgfpathlineto{\pgfqpoint{1.336933in}{2.658499in}}%
\pgfusepath{stroke}%
\end{pgfscope}%
\begin{pgfscope}%
\pgfpathrectangle{\pgfqpoint{0.100000in}{0.212622in}}{\pgfqpoint{3.696000in}{3.696000in}}%
\pgfusepath{clip}%
\pgfsetrectcap%
\pgfsetroundjoin%
\pgfsetlinewidth{1.505625pt}%
\definecolor{currentstroke}{rgb}{1.000000,0.000000,0.000000}%
\pgfsetstrokecolor{currentstroke}%
\pgfsetdash{}{0pt}%
\pgfpathmoveto{\pgfqpoint{1.282371in}{2.154788in}}%
\pgfpathlineto{\pgfqpoint{1.336933in}{2.658499in}}%
\pgfusepath{stroke}%
\end{pgfscope}%
\begin{pgfscope}%
\pgfpathrectangle{\pgfqpoint{0.100000in}{0.212622in}}{\pgfqpoint{3.696000in}{3.696000in}}%
\pgfusepath{clip}%
\pgfsetrectcap%
\pgfsetroundjoin%
\pgfsetlinewidth{1.505625pt}%
\definecolor{currentstroke}{rgb}{1.000000,0.000000,0.000000}%
\pgfsetstrokecolor{currentstroke}%
\pgfsetdash{}{0pt}%
\pgfpathmoveto{\pgfqpoint{1.288605in}{2.154796in}}%
\pgfpathlineto{\pgfqpoint{1.345864in}{2.665518in}}%
\pgfusepath{stroke}%
\end{pgfscope}%
\begin{pgfscope}%
\pgfpathrectangle{\pgfqpoint{0.100000in}{0.212622in}}{\pgfqpoint{3.696000in}{3.696000in}}%
\pgfusepath{clip}%
\pgfsetrectcap%
\pgfsetroundjoin%
\pgfsetlinewidth{1.505625pt}%
\definecolor{currentstroke}{rgb}{1.000000,0.000000,0.000000}%
\pgfsetstrokecolor{currentstroke}%
\pgfsetdash{}{0pt}%
\pgfpathmoveto{\pgfqpoint{1.294186in}{2.154646in}}%
\pgfpathlineto{\pgfqpoint{1.354784in}{2.672527in}}%
\pgfusepath{stroke}%
\end{pgfscope}%
\begin{pgfscope}%
\pgfpathrectangle{\pgfqpoint{0.100000in}{0.212622in}}{\pgfqpoint{3.696000in}{3.696000in}}%
\pgfusepath{clip}%
\pgfsetrectcap%
\pgfsetroundjoin%
\pgfsetlinewidth{1.505625pt}%
\definecolor{currentstroke}{rgb}{1.000000,0.000000,0.000000}%
\pgfsetstrokecolor{currentstroke}%
\pgfsetdash{}{0pt}%
\pgfpathmoveto{\pgfqpoint{1.296113in}{2.152913in}}%
\pgfpathlineto{\pgfqpoint{1.363692in}{2.679528in}}%
\pgfusepath{stroke}%
\end{pgfscope}%
\begin{pgfscope}%
\pgfpathrectangle{\pgfqpoint{0.100000in}{0.212622in}}{\pgfqpoint{3.696000in}{3.696000in}}%
\pgfusepath{clip}%
\pgfsetrectcap%
\pgfsetroundjoin%
\pgfsetlinewidth{1.505625pt}%
\definecolor{currentstroke}{rgb}{1.000000,0.000000,0.000000}%
\pgfsetstrokecolor{currentstroke}%
\pgfsetdash{}{0pt}%
\pgfpathmoveto{\pgfqpoint{1.303583in}{2.152157in}}%
\pgfpathlineto{\pgfqpoint{1.363692in}{2.679528in}}%
\pgfusepath{stroke}%
\end{pgfscope}%
\begin{pgfscope}%
\pgfpathrectangle{\pgfqpoint{0.100000in}{0.212622in}}{\pgfqpoint{3.696000in}{3.696000in}}%
\pgfusepath{clip}%
\pgfsetrectcap%
\pgfsetroundjoin%
\pgfsetlinewidth{1.505625pt}%
\definecolor{currentstroke}{rgb}{1.000000,0.000000,0.000000}%
\pgfsetstrokecolor{currentstroke}%
\pgfsetdash{}{0pt}%
\pgfpathmoveto{\pgfqpoint{1.307807in}{2.151654in}}%
\pgfpathlineto{\pgfqpoint{1.372589in}{2.686520in}}%
\pgfusepath{stroke}%
\end{pgfscope}%
\begin{pgfscope}%
\pgfpathrectangle{\pgfqpoint{0.100000in}{0.212622in}}{\pgfqpoint{3.696000in}{3.696000in}}%
\pgfusepath{clip}%
\pgfsetrectcap%
\pgfsetroundjoin%
\pgfsetlinewidth{1.505625pt}%
\definecolor{currentstroke}{rgb}{1.000000,0.000000,0.000000}%
\pgfsetstrokecolor{currentstroke}%
\pgfsetdash{}{0pt}%
\pgfpathmoveto{\pgfqpoint{1.316517in}{2.151009in}}%
\pgfpathlineto{\pgfqpoint{1.381474in}{2.693503in}}%
\pgfusepath{stroke}%
\end{pgfscope}%
\begin{pgfscope}%
\pgfpathrectangle{\pgfqpoint{0.100000in}{0.212622in}}{\pgfqpoint{3.696000in}{3.696000in}}%
\pgfusepath{clip}%
\pgfsetrectcap%
\pgfsetroundjoin%
\pgfsetlinewidth{1.505625pt}%
\definecolor{currentstroke}{rgb}{1.000000,0.000000,0.000000}%
\pgfsetstrokecolor{currentstroke}%
\pgfsetdash{}{0pt}%
\pgfpathmoveto{\pgfqpoint{1.322987in}{2.151085in}}%
\pgfpathlineto{\pgfqpoint{1.390348in}{2.700477in}}%
\pgfusepath{stroke}%
\end{pgfscope}%
\begin{pgfscope}%
\pgfpathrectangle{\pgfqpoint{0.100000in}{0.212622in}}{\pgfqpoint{3.696000in}{3.696000in}}%
\pgfusepath{clip}%
\pgfsetrectcap%
\pgfsetroundjoin%
\pgfsetlinewidth{1.505625pt}%
\definecolor{currentstroke}{rgb}{1.000000,0.000000,0.000000}%
\pgfsetstrokecolor{currentstroke}%
\pgfsetdash{}{0pt}%
\pgfpathmoveto{\pgfqpoint{1.326468in}{2.150802in}}%
\pgfpathlineto{\pgfqpoint{1.390348in}{2.700477in}}%
\pgfusepath{stroke}%
\end{pgfscope}%
\begin{pgfscope}%
\pgfpathrectangle{\pgfqpoint{0.100000in}{0.212622in}}{\pgfqpoint{3.696000in}{3.696000in}}%
\pgfusepath{clip}%
\pgfsetrectcap%
\pgfsetroundjoin%
\pgfsetlinewidth{1.505625pt}%
\definecolor{currentstroke}{rgb}{1.000000,0.000000,0.000000}%
\pgfsetstrokecolor{currentstroke}%
\pgfsetdash{}{0pt}%
\pgfpathmoveto{\pgfqpoint{1.329111in}{2.150561in}}%
\pgfpathlineto{\pgfqpoint{1.399211in}{2.707442in}}%
\pgfusepath{stroke}%
\end{pgfscope}%
\begin{pgfscope}%
\pgfpathrectangle{\pgfqpoint{0.100000in}{0.212622in}}{\pgfqpoint{3.696000in}{3.696000in}}%
\pgfusepath{clip}%
\pgfsetrectcap%
\pgfsetroundjoin%
\pgfsetlinewidth{1.505625pt}%
\definecolor{currentstroke}{rgb}{1.000000,0.000000,0.000000}%
\pgfsetstrokecolor{currentstroke}%
\pgfsetdash{}{0pt}%
\pgfpathmoveto{\pgfqpoint{1.331473in}{2.150366in}}%
\pgfpathlineto{\pgfqpoint{1.399211in}{2.707442in}}%
\pgfusepath{stroke}%
\end{pgfscope}%
\begin{pgfscope}%
\pgfpathrectangle{\pgfqpoint{0.100000in}{0.212622in}}{\pgfqpoint{3.696000in}{3.696000in}}%
\pgfusepath{clip}%
\pgfsetrectcap%
\pgfsetroundjoin%
\pgfsetlinewidth{1.505625pt}%
\definecolor{currentstroke}{rgb}{1.000000,0.000000,0.000000}%
\pgfsetstrokecolor{currentstroke}%
\pgfsetdash{}{0pt}%
\pgfpathmoveto{\pgfqpoint{1.334849in}{2.150347in}}%
\pgfpathlineto{\pgfqpoint{1.399211in}{2.707442in}}%
\pgfusepath{stroke}%
\end{pgfscope}%
\begin{pgfscope}%
\pgfpathrectangle{\pgfqpoint{0.100000in}{0.212622in}}{\pgfqpoint{3.696000in}{3.696000in}}%
\pgfusepath{clip}%
\pgfsetrectcap%
\pgfsetroundjoin%
\pgfsetlinewidth{1.505625pt}%
\definecolor{currentstroke}{rgb}{1.000000,0.000000,0.000000}%
\pgfsetstrokecolor{currentstroke}%
\pgfsetdash{}{0pt}%
\pgfpathmoveto{\pgfqpoint{1.336649in}{2.150249in}}%
\pgfpathlineto{\pgfqpoint{1.408062in}{2.714398in}}%
\pgfusepath{stroke}%
\end{pgfscope}%
\begin{pgfscope}%
\pgfpathrectangle{\pgfqpoint{0.100000in}{0.212622in}}{\pgfqpoint{3.696000in}{3.696000in}}%
\pgfusepath{clip}%
\pgfsetrectcap%
\pgfsetroundjoin%
\pgfsetlinewidth{1.505625pt}%
\definecolor{currentstroke}{rgb}{1.000000,0.000000,0.000000}%
\pgfsetstrokecolor{currentstroke}%
\pgfsetdash{}{0pt}%
\pgfpathmoveto{\pgfqpoint{1.335115in}{2.149497in}}%
\pgfpathlineto{\pgfqpoint{1.408062in}{2.714398in}}%
\pgfusepath{stroke}%
\end{pgfscope}%
\begin{pgfscope}%
\pgfpathrectangle{\pgfqpoint{0.100000in}{0.212622in}}{\pgfqpoint{3.696000in}{3.696000in}}%
\pgfusepath{clip}%
\pgfsetrectcap%
\pgfsetroundjoin%
\pgfsetlinewidth{1.505625pt}%
\definecolor{currentstroke}{rgb}{1.000000,0.000000,0.000000}%
\pgfsetstrokecolor{currentstroke}%
\pgfsetdash{}{0pt}%
\pgfpathmoveto{\pgfqpoint{1.338049in}{2.149301in}}%
\pgfpathlineto{\pgfqpoint{1.408062in}{2.714398in}}%
\pgfusepath{stroke}%
\end{pgfscope}%
\begin{pgfscope}%
\pgfpathrectangle{\pgfqpoint{0.100000in}{0.212622in}}{\pgfqpoint{3.696000in}{3.696000in}}%
\pgfusepath{clip}%
\pgfsetrectcap%
\pgfsetroundjoin%
\pgfsetlinewidth{1.505625pt}%
\definecolor{currentstroke}{rgb}{1.000000,0.000000,0.000000}%
\pgfsetstrokecolor{currentstroke}%
\pgfsetdash{}{0pt}%
\pgfpathmoveto{\pgfqpoint{1.340030in}{2.149170in}}%
\pgfpathlineto{\pgfqpoint{1.408062in}{2.714398in}}%
\pgfusepath{stroke}%
\end{pgfscope}%
\begin{pgfscope}%
\pgfpathrectangle{\pgfqpoint{0.100000in}{0.212622in}}{\pgfqpoint{3.696000in}{3.696000in}}%
\pgfusepath{clip}%
\pgfsetrectcap%
\pgfsetroundjoin%
\pgfsetlinewidth{1.505625pt}%
\definecolor{currentstroke}{rgb}{1.000000,0.000000,0.000000}%
\pgfsetstrokecolor{currentstroke}%
\pgfsetdash{}{0pt}%
\pgfpathmoveto{\pgfqpoint{1.343387in}{2.149089in}}%
\pgfpathlineto{\pgfqpoint{1.416902in}{2.721346in}}%
\pgfusepath{stroke}%
\end{pgfscope}%
\begin{pgfscope}%
\pgfpathrectangle{\pgfqpoint{0.100000in}{0.212622in}}{\pgfqpoint{3.696000in}{3.696000in}}%
\pgfusepath{clip}%
\pgfsetrectcap%
\pgfsetroundjoin%
\pgfsetlinewidth{1.505625pt}%
\definecolor{currentstroke}{rgb}{1.000000,0.000000,0.000000}%
\pgfsetstrokecolor{currentstroke}%
\pgfsetdash{}{0pt}%
\pgfpathmoveto{\pgfqpoint{1.345148in}{2.149051in}}%
\pgfpathlineto{\pgfqpoint{1.416902in}{2.721346in}}%
\pgfusepath{stroke}%
\end{pgfscope}%
\begin{pgfscope}%
\pgfpathrectangle{\pgfqpoint{0.100000in}{0.212622in}}{\pgfqpoint{3.696000in}{3.696000in}}%
\pgfusepath{clip}%
\pgfsetrectcap%
\pgfsetroundjoin%
\pgfsetlinewidth{1.505625pt}%
\definecolor{currentstroke}{rgb}{1.000000,0.000000,0.000000}%
\pgfsetstrokecolor{currentstroke}%
\pgfsetdash{}{0pt}%
\pgfpathmoveto{\pgfqpoint{1.347963in}{2.149080in}}%
\pgfpathlineto{\pgfqpoint{1.416902in}{2.721346in}}%
\pgfusepath{stroke}%
\end{pgfscope}%
\begin{pgfscope}%
\pgfpathrectangle{\pgfqpoint{0.100000in}{0.212622in}}{\pgfqpoint{3.696000in}{3.696000in}}%
\pgfusepath{clip}%
\pgfsetrectcap%
\pgfsetroundjoin%
\pgfsetlinewidth{1.505625pt}%
\definecolor{currentstroke}{rgb}{1.000000,0.000000,0.000000}%
\pgfsetstrokecolor{currentstroke}%
\pgfsetdash{}{0pt}%
\pgfpathmoveto{\pgfqpoint{1.352733in}{2.148539in}}%
\pgfpathlineto{\pgfqpoint{1.425731in}{2.728284in}}%
\pgfusepath{stroke}%
\end{pgfscope}%
\begin{pgfscope}%
\pgfpathrectangle{\pgfqpoint{0.100000in}{0.212622in}}{\pgfqpoint{3.696000in}{3.696000in}}%
\pgfusepath{clip}%
\pgfsetrectcap%
\pgfsetroundjoin%
\pgfsetlinewidth{1.505625pt}%
\definecolor{currentstroke}{rgb}{1.000000,0.000000,0.000000}%
\pgfsetstrokecolor{currentstroke}%
\pgfsetdash{}{0pt}%
\pgfpathmoveto{\pgfqpoint{1.356471in}{2.148637in}}%
\pgfpathlineto{\pgfqpoint{1.425731in}{2.728284in}}%
\pgfusepath{stroke}%
\end{pgfscope}%
\begin{pgfscope}%
\pgfpathrectangle{\pgfqpoint{0.100000in}{0.212622in}}{\pgfqpoint{3.696000in}{3.696000in}}%
\pgfusepath{clip}%
\pgfsetrectcap%
\pgfsetroundjoin%
\pgfsetlinewidth{1.505625pt}%
\definecolor{currentstroke}{rgb}{1.000000,0.000000,0.000000}%
\pgfsetstrokecolor{currentstroke}%
\pgfsetdash{}{0pt}%
\pgfpathmoveto{\pgfqpoint{1.361017in}{2.148554in}}%
\pgfpathlineto{\pgfqpoint{1.434549in}{2.735214in}}%
\pgfusepath{stroke}%
\end{pgfscope}%
\begin{pgfscope}%
\pgfpathrectangle{\pgfqpoint{0.100000in}{0.212622in}}{\pgfqpoint{3.696000in}{3.696000in}}%
\pgfusepath{clip}%
\pgfsetrectcap%
\pgfsetroundjoin%
\pgfsetlinewidth{1.505625pt}%
\definecolor{currentstroke}{rgb}{1.000000,0.000000,0.000000}%
\pgfsetstrokecolor{currentstroke}%
\pgfsetdash{}{0pt}%
\pgfpathmoveto{\pgfqpoint{1.363994in}{2.148593in}}%
\pgfpathlineto{\pgfqpoint{1.434549in}{2.735214in}}%
\pgfusepath{stroke}%
\end{pgfscope}%
\begin{pgfscope}%
\pgfpathrectangle{\pgfqpoint{0.100000in}{0.212622in}}{\pgfqpoint{3.696000in}{3.696000in}}%
\pgfusepath{clip}%
\pgfsetrectcap%
\pgfsetroundjoin%
\pgfsetlinewidth{1.505625pt}%
\definecolor{currentstroke}{rgb}{1.000000,0.000000,0.000000}%
\pgfsetstrokecolor{currentstroke}%
\pgfsetdash{}{0pt}%
\pgfpathmoveto{\pgfqpoint{1.366164in}{2.148500in}}%
\pgfpathlineto{\pgfqpoint{1.443356in}{2.742135in}}%
\pgfusepath{stroke}%
\end{pgfscope}%
\begin{pgfscope}%
\pgfpathrectangle{\pgfqpoint{0.100000in}{0.212622in}}{\pgfqpoint{3.696000in}{3.696000in}}%
\pgfusepath{clip}%
\pgfsetrectcap%
\pgfsetroundjoin%
\pgfsetlinewidth{1.505625pt}%
\definecolor{currentstroke}{rgb}{1.000000,0.000000,0.000000}%
\pgfsetstrokecolor{currentstroke}%
\pgfsetdash{}{0pt}%
\pgfpathmoveto{\pgfqpoint{1.369794in}{2.148183in}}%
\pgfpathlineto{\pgfqpoint{1.443356in}{2.742135in}}%
\pgfusepath{stroke}%
\end{pgfscope}%
\begin{pgfscope}%
\pgfpathrectangle{\pgfqpoint{0.100000in}{0.212622in}}{\pgfqpoint{3.696000in}{3.696000in}}%
\pgfusepath{clip}%
\pgfsetrectcap%
\pgfsetroundjoin%
\pgfsetlinewidth{1.505625pt}%
\definecolor{currentstroke}{rgb}{1.000000,0.000000,0.000000}%
\pgfsetstrokecolor{currentstroke}%
\pgfsetdash{}{0pt}%
\pgfpathmoveto{\pgfqpoint{1.372702in}{2.148212in}}%
\pgfpathlineto{\pgfqpoint{1.443356in}{2.742135in}}%
\pgfusepath{stroke}%
\end{pgfscope}%
\begin{pgfscope}%
\pgfpathrectangle{\pgfqpoint{0.100000in}{0.212622in}}{\pgfqpoint{3.696000in}{3.696000in}}%
\pgfusepath{clip}%
\pgfsetrectcap%
\pgfsetroundjoin%
\pgfsetlinewidth{1.505625pt}%
\definecolor{currentstroke}{rgb}{1.000000,0.000000,0.000000}%
\pgfsetstrokecolor{currentstroke}%
\pgfsetdash{}{0pt}%
\pgfpathmoveto{\pgfqpoint{1.376257in}{2.147942in}}%
\pgfpathlineto{\pgfqpoint{1.452151in}{2.749047in}}%
\pgfusepath{stroke}%
\end{pgfscope}%
\begin{pgfscope}%
\pgfpathrectangle{\pgfqpoint{0.100000in}{0.212622in}}{\pgfqpoint{3.696000in}{3.696000in}}%
\pgfusepath{clip}%
\pgfsetrectcap%
\pgfsetroundjoin%
\pgfsetlinewidth{1.505625pt}%
\definecolor{currentstroke}{rgb}{1.000000,0.000000,0.000000}%
\pgfsetstrokecolor{currentstroke}%
\pgfsetdash{}{0pt}%
\pgfpathmoveto{\pgfqpoint{1.378631in}{2.147780in}}%
\pgfpathlineto{\pgfqpoint{1.452151in}{2.749047in}}%
\pgfusepath{stroke}%
\end{pgfscope}%
\begin{pgfscope}%
\pgfpathrectangle{\pgfqpoint{0.100000in}{0.212622in}}{\pgfqpoint{3.696000in}{3.696000in}}%
\pgfusepath{clip}%
\pgfsetrectcap%
\pgfsetroundjoin%
\pgfsetlinewidth{1.505625pt}%
\definecolor{currentstroke}{rgb}{1.000000,0.000000,0.000000}%
\pgfsetstrokecolor{currentstroke}%
\pgfsetdash{}{0pt}%
\pgfpathmoveto{\pgfqpoint{1.380171in}{2.147599in}}%
\pgfpathlineto{\pgfqpoint{1.452151in}{2.749047in}}%
\pgfusepath{stroke}%
\end{pgfscope}%
\begin{pgfscope}%
\pgfpathrectangle{\pgfqpoint{0.100000in}{0.212622in}}{\pgfqpoint{3.696000in}{3.696000in}}%
\pgfusepath{clip}%
\pgfsetrectcap%
\pgfsetroundjoin%
\pgfsetlinewidth{1.505625pt}%
\definecolor{currentstroke}{rgb}{1.000000,0.000000,0.000000}%
\pgfsetstrokecolor{currentstroke}%
\pgfsetdash{}{0pt}%
\pgfpathmoveto{\pgfqpoint{1.383281in}{2.147514in}}%
\pgfpathlineto{\pgfqpoint{1.460936in}{2.755950in}}%
\pgfusepath{stroke}%
\end{pgfscope}%
\begin{pgfscope}%
\pgfpathrectangle{\pgfqpoint{0.100000in}{0.212622in}}{\pgfqpoint{3.696000in}{3.696000in}}%
\pgfusepath{clip}%
\pgfsetrectcap%
\pgfsetroundjoin%
\pgfsetlinewidth{1.505625pt}%
\definecolor{currentstroke}{rgb}{1.000000,0.000000,0.000000}%
\pgfsetstrokecolor{currentstroke}%
\pgfsetdash{}{0pt}%
\pgfpathmoveto{\pgfqpoint{1.384870in}{2.147454in}}%
\pgfpathlineto{\pgfqpoint{1.460936in}{2.755950in}}%
\pgfusepath{stroke}%
\end{pgfscope}%
\begin{pgfscope}%
\pgfpathrectangle{\pgfqpoint{0.100000in}{0.212622in}}{\pgfqpoint{3.696000in}{3.696000in}}%
\pgfusepath{clip}%
\pgfsetrectcap%
\pgfsetroundjoin%
\pgfsetlinewidth{1.505625pt}%
\definecolor{currentstroke}{rgb}{1.000000,0.000000,0.000000}%
\pgfsetstrokecolor{currentstroke}%
\pgfsetdash{}{0pt}%
\pgfpathmoveto{\pgfqpoint{1.383189in}{2.146686in}}%
\pgfpathlineto{\pgfqpoint{1.460936in}{2.755950in}}%
\pgfusepath{stroke}%
\end{pgfscope}%
\begin{pgfscope}%
\pgfpathrectangle{\pgfqpoint{0.100000in}{0.212622in}}{\pgfqpoint{3.696000in}{3.696000in}}%
\pgfusepath{clip}%
\pgfsetrectcap%
\pgfsetroundjoin%
\pgfsetlinewidth{1.505625pt}%
\definecolor{currentstroke}{rgb}{1.000000,0.000000,0.000000}%
\pgfsetstrokecolor{currentstroke}%
\pgfsetdash{}{0pt}%
\pgfpathmoveto{\pgfqpoint{1.385843in}{2.146492in}}%
\pgfpathlineto{\pgfqpoint{1.460936in}{2.755950in}}%
\pgfusepath{stroke}%
\end{pgfscope}%
\begin{pgfscope}%
\pgfpathrectangle{\pgfqpoint{0.100000in}{0.212622in}}{\pgfqpoint{3.696000in}{3.696000in}}%
\pgfusepath{clip}%
\pgfsetrectcap%
\pgfsetroundjoin%
\pgfsetlinewidth{1.505625pt}%
\definecolor{currentstroke}{rgb}{1.000000,0.000000,0.000000}%
\pgfsetstrokecolor{currentstroke}%
\pgfsetdash{}{0pt}%
\pgfpathmoveto{\pgfqpoint{1.387624in}{2.146309in}}%
\pgfpathlineto{\pgfqpoint{1.469709in}{2.762845in}}%
\pgfusepath{stroke}%
\end{pgfscope}%
\begin{pgfscope}%
\pgfpathrectangle{\pgfqpoint{0.100000in}{0.212622in}}{\pgfqpoint{3.696000in}{3.696000in}}%
\pgfusepath{clip}%
\pgfsetrectcap%
\pgfsetroundjoin%
\pgfsetlinewidth{1.505625pt}%
\definecolor{currentstroke}{rgb}{1.000000,0.000000,0.000000}%
\pgfsetstrokecolor{currentstroke}%
\pgfsetdash{}{0pt}%
\pgfpathmoveto{\pgfqpoint{1.390917in}{2.146162in}}%
\pgfpathlineto{\pgfqpoint{1.469709in}{2.762845in}}%
\pgfusepath{stroke}%
\end{pgfscope}%
\begin{pgfscope}%
\pgfpathrectangle{\pgfqpoint{0.100000in}{0.212622in}}{\pgfqpoint{3.696000in}{3.696000in}}%
\pgfusepath{clip}%
\pgfsetrectcap%
\pgfsetroundjoin%
\pgfsetlinewidth{1.505625pt}%
\definecolor{currentstroke}{rgb}{1.000000,0.000000,0.000000}%
\pgfsetstrokecolor{currentstroke}%
\pgfsetdash{}{0pt}%
\pgfpathmoveto{\pgfqpoint{1.394079in}{2.146046in}}%
\pgfpathlineto{\pgfqpoint{1.478471in}{2.769731in}}%
\pgfusepath{stroke}%
\end{pgfscope}%
\begin{pgfscope}%
\pgfpathrectangle{\pgfqpoint{0.100000in}{0.212622in}}{\pgfqpoint{3.696000in}{3.696000in}}%
\pgfusepath{clip}%
\pgfsetrectcap%
\pgfsetroundjoin%
\pgfsetlinewidth{1.505625pt}%
\definecolor{currentstroke}{rgb}{1.000000,0.000000,0.000000}%
\pgfsetstrokecolor{currentstroke}%
\pgfsetdash{}{0pt}%
\pgfpathmoveto{\pgfqpoint{1.396390in}{2.145854in}}%
\pgfpathlineto{\pgfqpoint{1.478471in}{2.769731in}}%
\pgfusepath{stroke}%
\end{pgfscope}%
\begin{pgfscope}%
\pgfpathrectangle{\pgfqpoint{0.100000in}{0.212622in}}{\pgfqpoint{3.696000in}{3.696000in}}%
\pgfusepath{clip}%
\pgfsetrectcap%
\pgfsetroundjoin%
\pgfsetlinewidth{1.505625pt}%
\definecolor{currentstroke}{rgb}{1.000000,0.000000,0.000000}%
\pgfsetstrokecolor{currentstroke}%
\pgfsetdash{}{0pt}%
\pgfpathmoveto{\pgfqpoint{1.401711in}{2.145475in}}%
\pgfpathlineto{\pgfqpoint{1.487222in}{2.776608in}}%
\pgfusepath{stroke}%
\end{pgfscope}%
\begin{pgfscope}%
\pgfpathrectangle{\pgfqpoint{0.100000in}{0.212622in}}{\pgfqpoint{3.696000in}{3.696000in}}%
\pgfusepath{clip}%
\pgfsetrectcap%
\pgfsetroundjoin%
\pgfsetlinewidth{1.505625pt}%
\definecolor{currentstroke}{rgb}{1.000000,0.000000,0.000000}%
\pgfsetstrokecolor{currentstroke}%
\pgfsetdash{}{0pt}%
\pgfpathmoveto{\pgfqpoint{1.406191in}{2.145398in}}%
\pgfpathlineto{\pgfqpoint{1.487222in}{2.776608in}}%
\pgfusepath{stroke}%
\end{pgfscope}%
\begin{pgfscope}%
\pgfpathrectangle{\pgfqpoint{0.100000in}{0.212622in}}{\pgfqpoint{3.696000in}{3.696000in}}%
\pgfusepath{clip}%
\pgfsetrectcap%
\pgfsetroundjoin%
\pgfsetlinewidth{1.505625pt}%
\definecolor{currentstroke}{rgb}{1.000000,0.000000,0.000000}%
\pgfsetstrokecolor{currentstroke}%
\pgfsetdash{}{0pt}%
\pgfpathmoveto{\pgfqpoint{1.410364in}{2.145359in}}%
\pgfpathlineto{\pgfqpoint{1.495962in}{2.783477in}}%
\pgfusepath{stroke}%
\end{pgfscope}%
\begin{pgfscope}%
\pgfpathrectangle{\pgfqpoint{0.100000in}{0.212622in}}{\pgfqpoint{3.696000in}{3.696000in}}%
\pgfusepath{clip}%
\pgfsetrectcap%
\pgfsetroundjoin%
\pgfsetlinewidth{1.505625pt}%
\definecolor{currentstroke}{rgb}{1.000000,0.000000,0.000000}%
\pgfsetstrokecolor{currentstroke}%
\pgfsetdash{}{0pt}%
\pgfpathmoveto{\pgfqpoint{1.413595in}{2.145080in}}%
\pgfpathlineto{\pgfqpoint{1.495962in}{2.783477in}}%
\pgfusepath{stroke}%
\end{pgfscope}%
\begin{pgfscope}%
\pgfpathrectangle{\pgfqpoint{0.100000in}{0.212622in}}{\pgfqpoint{3.696000in}{3.696000in}}%
\pgfusepath{clip}%
\pgfsetrectcap%
\pgfsetroundjoin%
\pgfsetlinewidth{1.505625pt}%
\definecolor{currentstroke}{rgb}{1.000000,0.000000,0.000000}%
\pgfsetstrokecolor{currentstroke}%
\pgfsetdash{}{0pt}%
\pgfpathmoveto{\pgfqpoint{1.414846in}{2.144996in}}%
\pgfpathlineto{\pgfqpoint{1.495962in}{2.783477in}}%
\pgfusepath{stroke}%
\end{pgfscope}%
\begin{pgfscope}%
\pgfpathrectangle{\pgfqpoint{0.100000in}{0.212622in}}{\pgfqpoint{3.696000in}{3.696000in}}%
\pgfusepath{clip}%
\pgfsetrectcap%
\pgfsetroundjoin%
\pgfsetlinewidth{1.505625pt}%
\definecolor{currentstroke}{rgb}{1.000000,0.000000,0.000000}%
\pgfsetstrokecolor{currentstroke}%
\pgfsetdash{}{0pt}%
\pgfpathmoveto{\pgfqpoint{1.416923in}{2.144888in}}%
\pgfpathlineto{\pgfqpoint{1.504691in}{2.790337in}}%
\pgfusepath{stroke}%
\end{pgfscope}%
\begin{pgfscope}%
\pgfpathrectangle{\pgfqpoint{0.100000in}{0.212622in}}{\pgfqpoint{3.696000in}{3.696000in}}%
\pgfusepath{clip}%
\pgfsetrectcap%
\pgfsetroundjoin%
\pgfsetlinewidth{1.505625pt}%
\definecolor{currentstroke}{rgb}{1.000000,0.000000,0.000000}%
\pgfsetstrokecolor{currentstroke}%
\pgfsetdash{}{0pt}%
\pgfpathmoveto{\pgfqpoint{1.417872in}{2.144873in}}%
\pgfpathlineto{\pgfqpoint{1.504691in}{2.790337in}}%
\pgfusepath{stroke}%
\end{pgfscope}%
\begin{pgfscope}%
\pgfpathrectangle{\pgfqpoint{0.100000in}{0.212622in}}{\pgfqpoint{3.696000in}{3.696000in}}%
\pgfusepath{clip}%
\pgfsetrectcap%
\pgfsetroundjoin%
\pgfsetlinewidth{1.505625pt}%
\definecolor{currentstroke}{rgb}{1.000000,0.000000,0.000000}%
\pgfsetstrokecolor{currentstroke}%
\pgfsetdash{}{0pt}%
\pgfpathmoveto{\pgfqpoint{1.417320in}{2.144632in}}%
\pgfpathlineto{\pgfqpoint{1.504691in}{2.790337in}}%
\pgfusepath{stroke}%
\end{pgfscope}%
\begin{pgfscope}%
\pgfpathrectangle{\pgfqpoint{0.100000in}{0.212622in}}{\pgfqpoint{3.696000in}{3.696000in}}%
\pgfusepath{clip}%
\pgfsetrectcap%
\pgfsetroundjoin%
\pgfsetlinewidth{1.505625pt}%
\definecolor{currentstroke}{rgb}{1.000000,0.000000,0.000000}%
\pgfsetstrokecolor{currentstroke}%
\pgfsetdash{}{0pt}%
\pgfpathmoveto{\pgfqpoint{1.418192in}{2.144583in}}%
\pgfpathlineto{\pgfqpoint{1.504691in}{2.790337in}}%
\pgfusepath{stroke}%
\end{pgfscope}%
\begin{pgfscope}%
\pgfpathrectangle{\pgfqpoint{0.100000in}{0.212622in}}{\pgfqpoint{3.696000in}{3.696000in}}%
\pgfusepath{clip}%
\pgfsetrectcap%
\pgfsetroundjoin%
\pgfsetlinewidth{1.505625pt}%
\definecolor{currentstroke}{rgb}{1.000000,0.000000,0.000000}%
\pgfsetstrokecolor{currentstroke}%
\pgfsetdash{}{0pt}%
\pgfpathmoveto{\pgfqpoint{1.419032in}{2.144475in}}%
\pgfpathlineto{\pgfqpoint{1.504691in}{2.790337in}}%
\pgfusepath{stroke}%
\end{pgfscope}%
\begin{pgfscope}%
\pgfpathrectangle{\pgfqpoint{0.100000in}{0.212622in}}{\pgfqpoint{3.696000in}{3.696000in}}%
\pgfusepath{clip}%
\pgfsetrectcap%
\pgfsetroundjoin%
\pgfsetlinewidth{1.505625pt}%
\definecolor{currentstroke}{rgb}{1.000000,0.000000,0.000000}%
\pgfsetstrokecolor{currentstroke}%
\pgfsetdash{}{0pt}%
\pgfpathmoveto{\pgfqpoint{1.420538in}{2.144418in}}%
\pgfpathlineto{\pgfqpoint{1.504691in}{2.790337in}}%
\pgfusepath{stroke}%
\end{pgfscope}%
\begin{pgfscope}%
\pgfpathrectangle{\pgfqpoint{0.100000in}{0.212622in}}{\pgfqpoint{3.696000in}{3.696000in}}%
\pgfusepath{clip}%
\pgfsetrectcap%
\pgfsetroundjoin%
\pgfsetlinewidth{1.505625pt}%
\definecolor{currentstroke}{rgb}{1.000000,0.000000,0.000000}%
\pgfsetstrokecolor{currentstroke}%
\pgfsetdash{}{0pt}%
\pgfpathmoveto{\pgfqpoint{1.421300in}{2.144359in}}%
\pgfpathlineto{\pgfqpoint{1.504691in}{2.790337in}}%
\pgfusepath{stroke}%
\end{pgfscope}%
\begin{pgfscope}%
\pgfpathrectangle{\pgfqpoint{0.100000in}{0.212622in}}{\pgfqpoint{3.696000in}{3.696000in}}%
\pgfusepath{clip}%
\pgfsetrectcap%
\pgfsetroundjoin%
\pgfsetlinewidth{1.505625pt}%
\definecolor{currentstroke}{rgb}{1.000000,0.000000,0.000000}%
\pgfsetstrokecolor{currentstroke}%
\pgfsetdash{}{0pt}%
\pgfpathmoveto{\pgfqpoint{1.422786in}{2.144442in}}%
\pgfpathlineto{\pgfqpoint{1.504691in}{2.790337in}}%
\pgfusepath{stroke}%
\end{pgfscope}%
\begin{pgfscope}%
\pgfpathrectangle{\pgfqpoint{0.100000in}{0.212622in}}{\pgfqpoint{3.696000in}{3.696000in}}%
\pgfusepath{clip}%
\pgfsetrectcap%
\pgfsetroundjoin%
\pgfsetlinewidth{1.505625pt}%
\definecolor{currentstroke}{rgb}{1.000000,0.000000,0.000000}%
\pgfsetstrokecolor{currentstroke}%
\pgfsetdash{}{0pt}%
\pgfpathmoveto{\pgfqpoint{1.425238in}{2.144316in}}%
\pgfpathlineto{\pgfqpoint{1.513409in}{2.797188in}}%
\pgfusepath{stroke}%
\end{pgfscope}%
\begin{pgfscope}%
\pgfpathrectangle{\pgfqpoint{0.100000in}{0.212622in}}{\pgfqpoint{3.696000in}{3.696000in}}%
\pgfusepath{clip}%
\pgfsetrectcap%
\pgfsetroundjoin%
\pgfsetlinewidth{1.505625pt}%
\definecolor{currentstroke}{rgb}{1.000000,0.000000,0.000000}%
\pgfsetstrokecolor{currentstroke}%
\pgfsetdash{}{0pt}%
\pgfpathmoveto{\pgfqpoint{1.427443in}{2.144221in}}%
\pgfpathlineto{\pgfqpoint{1.513409in}{2.797188in}}%
\pgfusepath{stroke}%
\end{pgfscope}%
\begin{pgfscope}%
\pgfpathrectangle{\pgfqpoint{0.100000in}{0.212622in}}{\pgfqpoint{3.696000in}{3.696000in}}%
\pgfusepath{clip}%
\pgfsetrectcap%
\pgfsetroundjoin%
\pgfsetlinewidth{1.505625pt}%
\definecolor{currentstroke}{rgb}{1.000000,0.000000,0.000000}%
\pgfsetstrokecolor{currentstroke}%
\pgfsetdash{}{0pt}%
\pgfpathmoveto{\pgfqpoint{1.430162in}{2.144113in}}%
\pgfpathlineto{\pgfqpoint{1.513409in}{2.797188in}}%
\pgfusepath{stroke}%
\end{pgfscope}%
\begin{pgfscope}%
\pgfpathrectangle{\pgfqpoint{0.100000in}{0.212622in}}{\pgfqpoint{3.696000in}{3.696000in}}%
\pgfusepath{clip}%
\pgfsetrectcap%
\pgfsetroundjoin%
\pgfsetlinewidth{1.505625pt}%
\definecolor{currentstroke}{rgb}{1.000000,0.000000,0.000000}%
\pgfsetstrokecolor{currentstroke}%
\pgfsetdash{}{0pt}%
\pgfpathmoveto{\pgfqpoint{1.431900in}{2.144023in}}%
\pgfpathlineto{\pgfqpoint{1.522115in}{2.804031in}}%
\pgfusepath{stroke}%
\end{pgfscope}%
\begin{pgfscope}%
\pgfpathrectangle{\pgfqpoint{0.100000in}{0.212622in}}{\pgfqpoint{3.696000in}{3.696000in}}%
\pgfusepath{clip}%
\pgfsetrectcap%
\pgfsetroundjoin%
\pgfsetlinewidth{1.505625pt}%
\definecolor{currentstroke}{rgb}{1.000000,0.000000,0.000000}%
\pgfsetstrokecolor{currentstroke}%
\pgfsetdash{}{0pt}%
\pgfpathmoveto{\pgfqpoint{1.433470in}{2.143938in}}%
\pgfpathlineto{\pgfqpoint{1.522115in}{2.804031in}}%
\pgfusepath{stroke}%
\end{pgfscope}%
\begin{pgfscope}%
\pgfpathrectangle{\pgfqpoint{0.100000in}{0.212622in}}{\pgfqpoint{3.696000in}{3.696000in}}%
\pgfusepath{clip}%
\pgfsetrectcap%
\pgfsetroundjoin%
\pgfsetlinewidth{1.505625pt}%
\definecolor{currentstroke}{rgb}{1.000000,0.000000,0.000000}%
\pgfsetstrokecolor{currentstroke}%
\pgfsetdash{}{0pt}%
\pgfpathmoveto{\pgfqpoint{1.436005in}{2.143822in}}%
\pgfpathlineto{\pgfqpoint{1.522115in}{2.804031in}}%
\pgfusepath{stroke}%
\end{pgfscope}%
\begin{pgfscope}%
\pgfpathrectangle{\pgfqpoint{0.100000in}{0.212622in}}{\pgfqpoint{3.696000in}{3.696000in}}%
\pgfusepath{clip}%
\pgfsetrectcap%
\pgfsetroundjoin%
\pgfsetlinewidth{1.505625pt}%
\definecolor{currentstroke}{rgb}{1.000000,0.000000,0.000000}%
\pgfsetstrokecolor{currentstroke}%
\pgfsetdash{}{0pt}%
\pgfpathmoveto{\pgfqpoint{1.437235in}{2.143745in}}%
\pgfpathlineto{\pgfqpoint{1.522115in}{2.804031in}}%
\pgfusepath{stroke}%
\end{pgfscope}%
\begin{pgfscope}%
\pgfpathrectangle{\pgfqpoint{0.100000in}{0.212622in}}{\pgfqpoint{3.696000in}{3.696000in}}%
\pgfusepath{clip}%
\pgfsetrectcap%
\pgfsetroundjoin%
\pgfsetlinewidth{1.505625pt}%
\definecolor{currentstroke}{rgb}{1.000000,0.000000,0.000000}%
\pgfsetstrokecolor{currentstroke}%
\pgfsetdash{}{0pt}%
\pgfpathmoveto{\pgfqpoint{1.439576in}{2.143680in}}%
\pgfpathlineto{\pgfqpoint{1.530811in}{2.810865in}}%
\pgfusepath{stroke}%
\end{pgfscope}%
\begin{pgfscope}%
\pgfpathrectangle{\pgfqpoint{0.100000in}{0.212622in}}{\pgfqpoint{3.696000in}{3.696000in}}%
\pgfusepath{clip}%
\pgfsetrectcap%
\pgfsetroundjoin%
\pgfsetlinewidth{1.505625pt}%
\definecolor{currentstroke}{rgb}{1.000000,0.000000,0.000000}%
\pgfsetstrokecolor{currentstroke}%
\pgfsetdash{}{0pt}%
\pgfpathmoveto{\pgfqpoint{1.444015in}{2.143203in}}%
\pgfpathlineto{\pgfqpoint{1.530811in}{2.810865in}}%
\pgfusepath{stroke}%
\end{pgfscope}%
\begin{pgfscope}%
\pgfpathrectangle{\pgfqpoint{0.100000in}{0.212622in}}{\pgfqpoint{3.696000in}{3.696000in}}%
\pgfusepath{clip}%
\pgfsetrectcap%
\pgfsetroundjoin%
\pgfsetlinewidth{1.505625pt}%
\definecolor{currentstroke}{rgb}{1.000000,0.000000,0.000000}%
\pgfsetstrokecolor{currentstroke}%
\pgfsetdash{}{0pt}%
\pgfpathmoveto{\pgfqpoint{1.446979in}{2.143193in}}%
\pgfpathlineto{\pgfqpoint{1.539496in}{2.817690in}}%
\pgfusepath{stroke}%
\end{pgfscope}%
\begin{pgfscope}%
\pgfpathrectangle{\pgfqpoint{0.100000in}{0.212622in}}{\pgfqpoint{3.696000in}{3.696000in}}%
\pgfusepath{clip}%
\pgfsetrectcap%
\pgfsetroundjoin%
\pgfsetlinewidth{1.505625pt}%
\definecolor{currentstroke}{rgb}{1.000000,0.000000,0.000000}%
\pgfsetstrokecolor{currentstroke}%
\pgfsetdash{}{0pt}%
\pgfpathmoveto{\pgfqpoint{1.450905in}{2.143042in}}%
\pgfpathlineto{\pgfqpoint{1.539496in}{2.817690in}}%
\pgfusepath{stroke}%
\end{pgfscope}%
\begin{pgfscope}%
\pgfpathrectangle{\pgfqpoint{0.100000in}{0.212622in}}{\pgfqpoint{3.696000in}{3.696000in}}%
\pgfusepath{clip}%
\pgfsetrectcap%
\pgfsetroundjoin%
\pgfsetlinewidth{1.505625pt}%
\definecolor{currentstroke}{rgb}{1.000000,0.000000,0.000000}%
\pgfsetstrokecolor{currentstroke}%
\pgfsetdash{}{0pt}%
\pgfpathmoveto{\pgfqpoint{1.453091in}{2.142955in}}%
\pgfpathlineto{\pgfqpoint{1.539496in}{2.817690in}}%
\pgfusepath{stroke}%
\end{pgfscope}%
\begin{pgfscope}%
\pgfpathrectangle{\pgfqpoint{0.100000in}{0.212622in}}{\pgfqpoint{3.696000in}{3.696000in}}%
\pgfusepath{clip}%
\pgfsetrectcap%
\pgfsetroundjoin%
\pgfsetlinewidth{1.505625pt}%
\definecolor{currentstroke}{rgb}{1.000000,0.000000,0.000000}%
\pgfsetstrokecolor{currentstroke}%
\pgfsetdash{}{0pt}%
\pgfpathmoveto{\pgfqpoint{1.453802in}{2.142915in}}%
\pgfpathlineto{\pgfqpoint{1.539496in}{2.817690in}}%
\pgfusepath{stroke}%
\end{pgfscope}%
\begin{pgfscope}%
\pgfpathrectangle{\pgfqpoint{0.100000in}{0.212622in}}{\pgfqpoint{3.696000in}{3.696000in}}%
\pgfusepath{clip}%
\pgfsetrectcap%
\pgfsetroundjoin%
\pgfsetlinewidth{1.505625pt}%
\definecolor{currentstroke}{rgb}{1.000000,0.000000,0.000000}%
\pgfsetstrokecolor{currentstroke}%
\pgfsetdash{}{0pt}%
\pgfpathmoveto{\pgfqpoint{1.454559in}{2.142896in}}%
\pgfpathlineto{\pgfqpoint{1.548170in}{2.824507in}}%
\pgfusepath{stroke}%
\end{pgfscope}%
\begin{pgfscope}%
\pgfpathrectangle{\pgfqpoint{0.100000in}{0.212622in}}{\pgfqpoint{3.696000in}{3.696000in}}%
\pgfusepath{clip}%
\pgfsetrectcap%
\pgfsetroundjoin%
\pgfsetlinewidth{1.505625pt}%
\definecolor{currentstroke}{rgb}{1.000000,0.000000,0.000000}%
\pgfsetstrokecolor{currentstroke}%
\pgfsetdash{}{0pt}%
\pgfpathmoveto{\pgfqpoint{1.454904in}{2.142892in}}%
\pgfpathlineto{\pgfqpoint{1.548170in}{2.824507in}}%
\pgfusepath{stroke}%
\end{pgfscope}%
\begin{pgfscope}%
\pgfpathrectangle{\pgfqpoint{0.100000in}{0.212622in}}{\pgfqpoint{3.696000in}{3.696000in}}%
\pgfusepath{clip}%
\pgfsetrectcap%
\pgfsetroundjoin%
\pgfsetlinewidth{1.505625pt}%
\definecolor{currentstroke}{rgb}{1.000000,0.000000,0.000000}%
\pgfsetstrokecolor{currentstroke}%
\pgfsetdash{}{0pt}%
\pgfpathmoveto{\pgfqpoint{1.454852in}{2.142788in}}%
\pgfpathlineto{\pgfqpoint{1.548170in}{2.824507in}}%
\pgfusepath{stroke}%
\end{pgfscope}%
\begin{pgfscope}%
\pgfpathrectangle{\pgfqpoint{0.100000in}{0.212622in}}{\pgfqpoint{3.696000in}{3.696000in}}%
\pgfusepath{clip}%
\pgfsetrectcap%
\pgfsetroundjoin%
\pgfsetlinewidth{1.505625pt}%
\definecolor{currentstroke}{rgb}{1.000000,0.000000,0.000000}%
\pgfsetstrokecolor{currentstroke}%
\pgfsetdash{}{0pt}%
\pgfpathmoveto{\pgfqpoint{1.456069in}{2.142682in}}%
\pgfpathlineto{\pgfqpoint{1.548170in}{2.824507in}}%
\pgfusepath{stroke}%
\end{pgfscope}%
\begin{pgfscope}%
\pgfpathrectangle{\pgfqpoint{0.100000in}{0.212622in}}{\pgfqpoint{3.696000in}{3.696000in}}%
\pgfusepath{clip}%
\pgfsetrectcap%
\pgfsetroundjoin%
\pgfsetlinewidth{1.505625pt}%
\definecolor{currentstroke}{rgb}{1.000000,0.000000,0.000000}%
\pgfsetstrokecolor{currentstroke}%
\pgfsetdash{}{0pt}%
\pgfpathmoveto{\pgfqpoint{1.457420in}{2.142706in}}%
\pgfpathlineto{\pgfqpoint{1.548170in}{2.824507in}}%
\pgfusepath{stroke}%
\end{pgfscope}%
\begin{pgfscope}%
\pgfpathrectangle{\pgfqpoint{0.100000in}{0.212622in}}{\pgfqpoint{3.696000in}{3.696000in}}%
\pgfusepath{clip}%
\pgfsetrectcap%
\pgfsetroundjoin%
\pgfsetlinewidth{1.505625pt}%
\definecolor{currentstroke}{rgb}{1.000000,0.000000,0.000000}%
\pgfsetstrokecolor{currentstroke}%
\pgfsetdash{}{0pt}%
\pgfpathmoveto{\pgfqpoint{1.458089in}{2.142693in}}%
\pgfpathlineto{\pgfqpoint{1.548170in}{2.824507in}}%
\pgfusepath{stroke}%
\end{pgfscope}%
\begin{pgfscope}%
\pgfpathrectangle{\pgfqpoint{0.100000in}{0.212622in}}{\pgfqpoint{3.696000in}{3.696000in}}%
\pgfusepath{clip}%
\pgfsetrectcap%
\pgfsetroundjoin%
\pgfsetlinewidth{1.505625pt}%
\definecolor{currentstroke}{rgb}{1.000000,0.000000,0.000000}%
\pgfsetstrokecolor{currentstroke}%
\pgfsetdash{}{0pt}%
\pgfpathmoveto{\pgfqpoint{1.458606in}{2.142656in}}%
\pgfpathlineto{\pgfqpoint{1.548170in}{2.824507in}}%
\pgfusepath{stroke}%
\end{pgfscope}%
\begin{pgfscope}%
\pgfpathrectangle{\pgfqpoint{0.100000in}{0.212622in}}{\pgfqpoint{3.696000in}{3.696000in}}%
\pgfusepath{clip}%
\pgfsetrectcap%
\pgfsetroundjoin%
\pgfsetlinewidth{1.505625pt}%
\definecolor{currentstroke}{rgb}{1.000000,0.000000,0.000000}%
\pgfsetstrokecolor{currentstroke}%
\pgfsetdash{}{0pt}%
\pgfpathmoveto{\pgfqpoint{1.459298in}{2.142625in}}%
\pgfpathlineto{\pgfqpoint{1.548170in}{2.824507in}}%
\pgfusepath{stroke}%
\end{pgfscope}%
\begin{pgfscope}%
\pgfpathrectangle{\pgfqpoint{0.100000in}{0.212622in}}{\pgfqpoint{3.696000in}{3.696000in}}%
\pgfusepath{clip}%
\pgfsetrectcap%
\pgfsetroundjoin%
\pgfsetlinewidth{1.505625pt}%
\definecolor{currentstroke}{rgb}{1.000000,0.000000,0.000000}%
\pgfsetstrokecolor{currentstroke}%
\pgfsetdash{}{0pt}%
\pgfpathmoveto{\pgfqpoint{1.460715in}{2.142528in}}%
\pgfpathlineto{\pgfqpoint{1.548170in}{2.824507in}}%
\pgfusepath{stroke}%
\end{pgfscope}%
\begin{pgfscope}%
\pgfpathrectangle{\pgfqpoint{0.100000in}{0.212622in}}{\pgfqpoint{3.696000in}{3.696000in}}%
\pgfusepath{clip}%
\pgfsetrectcap%
\pgfsetroundjoin%
\pgfsetlinewidth{1.505625pt}%
\definecolor{currentstroke}{rgb}{1.000000,0.000000,0.000000}%
\pgfsetstrokecolor{currentstroke}%
\pgfsetdash{}{0pt}%
\pgfpathmoveto{\pgfqpoint{1.462157in}{2.142467in}}%
\pgfpathlineto{\pgfqpoint{1.556833in}{2.831315in}}%
\pgfusepath{stroke}%
\end{pgfscope}%
\begin{pgfscope}%
\pgfpathrectangle{\pgfqpoint{0.100000in}{0.212622in}}{\pgfqpoint{3.696000in}{3.696000in}}%
\pgfusepath{clip}%
\pgfsetrectcap%
\pgfsetroundjoin%
\pgfsetlinewidth{1.505625pt}%
\definecolor{currentstroke}{rgb}{1.000000,0.000000,0.000000}%
\pgfsetstrokecolor{currentstroke}%
\pgfsetdash{}{0pt}%
\pgfpathmoveto{\pgfqpoint{1.462884in}{2.142512in}}%
\pgfpathlineto{\pgfqpoint{1.556833in}{2.831315in}}%
\pgfusepath{stroke}%
\end{pgfscope}%
\begin{pgfscope}%
\pgfpathrectangle{\pgfqpoint{0.100000in}{0.212622in}}{\pgfqpoint{3.696000in}{3.696000in}}%
\pgfusepath{clip}%
\pgfsetrectcap%
\pgfsetroundjoin%
\pgfsetlinewidth{1.505625pt}%
\definecolor{currentstroke}{rgb}{1.000000,0.000000,0.000000}%
\pgfsetstrokecolor{currentstroke}%
\pgfsetdash{}{0pt}%
\pgfpathmoveto{\pgfqpoint{1.464770in}{2.142400in}}%
\pgfpathlineto{\pgfqpoint{1.556833in}{2.831315in}}%
\pgfusepath{stroke}%
\end{pgfscope}%
\begin{pgfscope}%
\pgfpathrectangle{\pgfqpoint{0.100000in}{0.212622in}}{\pgfqpoint{3.696000in}{3.696000in}}%
\pgfusepath{clip}%
\pgfsetrectcap%
\pgfsetroundjoin%
\pgfsetlinewidth{1.505625pt}%
\definecolor{currentstroke}{rgb}{1.000000,0.000000,0.000000}%
\pgfsetstrokecolor{currentstroke}%
\pgfsetdash{}{0pt}%
\pgfpathmoveto{\pgfqpoint{1.466207in}{2.142461in}}%
\pgfpathlineto{\pgfqpoint{1.556833in}{2.831315in}}%
\pgfusepath{stroke}%
\end{pgfscope}%
\begin{pgfscope}%
\pgfpathrectangle{\pgfqpoint{0.100000in}{0.212622in}}{\pgfqpoint{3.696000in}{3.696000in}}%
\pgfusepath{clip}%
\pgfsetrectcap%
\pgfsetroundjoin%
\pgfsetlinewidth{1.505625pt}%
\definecolor{currentstroke}{rgb}{1.000000,0.000000,0.000000}%
\pgfsetstrokecolor{currentstroke}%
\pgfsetdash{}{0pt}%
\pgfpathmoveto{\pgfqpoint{1.467081in}{2.142430in}}%
\pgfpathlineto{\pgfqpoint{1.556833in}{2.831315in}}%
\pgfusepath{stroke}%
\end{pgfscope}%
\begin{pgfscope}%
\pgfpathrectangle{\pgfqpoint{0.100000in}{0.212622in}}{\pgfqpoint{3.696000in}{3.696000in}}%
\pgfusepath{clip}%
\pgfsetrectcap%
\pgfsetroundjoin%
\pgfsetlinewidth{1.505625pt}%
\definecolor{currentstroke}{rgb}{1.000000,0.000000,0.000000}%
\pgfsetstrokecolor{currentstroke}%
\pgfsetdash{}{0pt}%
\pgfpathmoveto{\pgfqpoint{1.467689in}{2.142402in}}%
\pgfpathlineto{\pgfqpoint{1.556833in}{2.831315in}}%
\pgfusepath{stroke}%
\end{pgfscope}%
\begin{pgfscope}%
\pgfpathrectangle{\pgfqpoint{0.100000in}{0.212622in}}{\pgfqpoint{3.696000in}{3.696000in}}%
\pgfusepath{clip}%
\pgfsetrectcap%
\pgfsetroundjoin%
\pgfsetlinewidth{1.505625pt}%
\definecolor{currentstroke}{rgb}{1.000000,0.000000,0.000000}%
\pgfsetstrokecolor{currentstroke}%
\pgfsetdash{}{0pt}%
\pgfpathmoveto{\pgfqpoint{1.468073in}{2.142335in}}%
\pgfpathlineto{\pgfqpoint{1.556833in}{2.831315in}}%
\pgfusepath{stroke}%
\end{pgfscope}%
\begin{pgfscope}%
\pgfpathrectangle{\pgfqpoint{0.100000in}{0.212622in}}{\pgfqpoint{3.696000in}{3.696000in}}%
\pgfusepath{clip}%
\pgfsetrectcap%
\pgfsetroundjoin%
\pgfsetlinewidth{1.505625pt}%
\definecolor{currentstroke}{rgb}{1.000000,0.000000,0.000000}%
\pgfsetstrokecolor{currentstroke}%
\pgfsetdash{}{0pt}%
\pgfpathmoveto{\pgfqpoint{1.469043in}{2.142316in}}%
\pgfpathlineto{\pgfqpoint{1.556833in}{2.831315in}}%
\pgfusepath{stroke}%
\end{pgfscope}%
\begin{pgfscope}%
\pgfpathrectangle{\pgfqpoint{0.100000in}{0.212622in}}{\pgfqpoint{3.696000in}{3.696000in}}%
\pgfusepath{clip}%
\pgfsetrectcap%
\pgfsetroundjoin%
\pgfsetlinewidth{1.505625pt}%
\definecolor{currentstroke}{rgb}{1.000000,0.000000,0.000000}%
\pgfsetstrokecolor{currentstroke}%
\pgfsetdash{}{0pt}%
\pgfpathmoveto{\pgfqpoint{1.470084in}{2.142279in}}%
\pgfpathlineto{\pgfqpoint{1.565486in}{2.838114in}}%
\pgfusepath{stroke}%
\end{pgfscope}%
\begin{pgfscope}%
\pgfpathrectangle{\pgfqpoint{0.100000in}{0.212622in}}{\pgfqpoint{3.696000in}{3.696000in}}%
\pgfusepath{clip}%
\pgfsetrectcap%
\pgfsetroundjoin%
\pgfsetlinewidth{1.505625pt}%
\definecolor{currentstroke}{rgb}{1.000000,0.000000,0.000000}%
\pgfsetstrokecolor{currentstroke}%
\pgfsetdash{}{0pt}%
\pgfpathmoveto{\pgfqpoint{1.470499in}{2.142246in}}%
\pgfpathlineto{\pgfqpoint{1.565486in}{2.838114in}}%
\pgfusepath{stroke}%
\end{pgfscope}%
\begin{pgfscope}%
\pgfpathrectangle{\pgfqpoint{0.100000in}{0.212622in}}{\pgfqpoint{3.696000in}{3.696000in}}%
\pgfusepath{clip}%
\pgfsetrectcap%
\pgfsetroundjoin%
\pgfsetlinewidth{1.505625pt}%
\definecolor{currentstroke}{rgb}{1.000000,0.000000,0.000000}%
\pgfsetstrokecolor{currentstroke}%
\pgfsetdash{}{0pt}%
\pgfpathmoveto{\pgfqpoint{1.470888in}{2.142221in}}%
\pgfpathlineto{\pgfqpoint{1.565486in}{2.838114in}}%
\pgfusepath{stroke}%
\end{pgfscope}%
\begin{pgfscope}%
\pgfpathrectangle{\pgfqpoint{0.100000in}{0.212622in}}{\pgfqpoint{3.696000in}{3.696000in}}%
\pgfusepath{clip}%
\pgfsetrectcap%
\pgfsetroundjoin%
\pgfsetlinewidth{1.505625pt}%
\definecolor{currentstroke}{rgb}{1.000000,0.000000,0.000000}%
\pgfsetstrokecolor{currentstroke}%
\pgfsetdash{}{0pt}%
\pgfpathmoveto{\pgfqpoint{1.471496in}{2.142203in}}%
\pgfpathlineto{\pgfqpoint{1.565486in}{2.838114in}}%
\pgfusepath{stroke}%
\end{pgfscope}%
\begin{pgfscope}%
\pgfpathrectangle{\pgfqpoint{0.100000in}{0.212622in}}{\pgfqpoint{3.696000in}{3.696000in}}%
\pgfusepath{clip}%
\pgfsetrectcap%
\pgfsetroundjoin%
\pgfsetlinewidth{1.505625pt}%
\definecolor{currentstroke}{rgb}{1.000000,0.000000,0.000000}%
\pgfsetstrokecolor{currentstroke}%
\pgfsetdash{}{0pt}%
\pgfpathmoveto{\pgfqpoint{1.472042in}{2.142002in}}%
\pgfpathlineto{\pgfqpoint{1.565486in}{2.838114in}}%
\pgfusepath{stroke}%
\end{pgfscope}%
\begin{pgfscope}%
\pgfpathrectangle{\pgfqpoint{0.100000in}{0.212622in}}{\pgfqpoint{3.696000in}{3.696000in}}%
\pgfusepath{clip}%
\pgfsetrectcap%
\pgfsetroundjoin%
\pgfsetlinewidth{1.505625pt}%
\definecolor{currentstroke}{rgb}{1.000000,0.000000,0.000000}%
\pgfsetstrokecolor{currentstroke}%
\pgfsetdash{}{0pt}%
\pgfpathmoveto{\pgfqpoint{1.473382in}{2.141885in}}%
\pgfpathlineto{\pgfqpoint{1.565486in}{2.838114in}}%
\pgfusepath{stroke}%
\end{pgfscope}%
\begin{pgfscope}%
\pgfpathrectangle{\pgfqpoint{0.100000in}{0.212622in}}{\pgfqpoint{3.696000in}{3.696000in}}%
\pgfusepath{clip}%
\pgfsetrectcap%
\pgfsetroundjoin%
\pgfsetlinewidth{1.505625pt}%
\definecolor{currentstroke}{rgb}{1.000000,0.000000,0.000000}%
\pgfsetstrokecolor{currentstroke}%
\pgfsetdash{}{0pt}%
\pgfpathmoveto{\pgfqpoint{1.474423in}{2.141820in}}%
\pgfpathlineto{\pgfqpoint{1.565486in}{2.838114in}}%
\pgfusepath{stroke}%
\end{pgfscope}%
\begin{pgfscope}%
\pgfpathrectangle{\pgfqpoint{0.100000in}{0.212622in}}{\pgfqpoint{3.696000in}{3.696000in}}%
\pgfusepath{clip}%
\pgfsetrectcap%
\pgfsetroundjoin%
\pgfsetlinewidth{1.505625pt}%
\definecolor{currentstroke}{rgb}{1.000000,0.000000,0.000000}%
\pgfsetstrokecolor{currentstroke}%
\pgfsetdash{}{0pt}%
\pgfpathmoveto{\pgfqpoint{1.476183in}{2.141761in}}%
\pgfpathlineto{\pgfqpoint{1.565486in}{2.838114in}}%
\pgfusepath{stroke}%
\end{pgfscope}%
\begin{pgfscope}%
\pgfpathrectangle{\pgfqpoint{0.100000in}{0.212622in}}{\pgfqpoint{3.696000in}{3.696000in}}%
\pgfusepath{clip}%
\pgfsetrectcap%
\pgfsetroundjoin%
\pgfsetlinewidth{1.505625pt}%
\definecolor{currentstroke}{rgb}{1.000000,0.000000,0.000000}%
\pgfsetstrokecolor{currentstroke}%
\pgfsetdash{}{0pt}%
\pgfpathmoveto{\pgfqpoint{1.477197in}{2.141729in}}%
\pgfpathlineto{\pgfqpoint{1.574127in}{2.844906in}}%
\pgfusepath{stroke}%
\end{pgfscope}%
\begin{pgfscope}%
\pgfpathrectangle{\pgfqpoint{0.100000in}{0.212622in}}{\pgfqpoint{3.696000in}{3.696000in}}%
\pgfusepath{clip}%
\pgfsetrectcap%
\pgfsetroundjoin%
\pgfsetlinewidth{1.505625pt}%
\definecolor{currentstroke}{rgb}{1.000000,0.000000,0.000000}%
\pgfsetstrokecolor{currentstroke}%
\pgfsetdash{}{0pt}%
\pgfpathmoveto{\pgfqpoint{1.477542in}{2.141678in}}%
\pgfpathlineto{\pgfqpoint{1.574127in}{2.844906in}}%
\pgfusepath{stroke}%
\end{pgfscope}%
\begin{pgfscope}%
\pgfpathrectangle{\pgfqpoint{0.100000in}{0.212622in}}{\pgfqpoint{3.696000in}{3.696000in}}%
\pgfusepath{clip}%
\pgfsetrectcap%
\pgfsetroundjoin%
\pgfsetlinewidth{1.505625pt}%
\definecolor{currentstroke}{rgb}{1.000000,0.000000,0.000000}%
\pgfsetstrokecolor{currentstroke}%
\pgfsetdash{}{0pt}%
\pgfpathmoveto{\pgfqpoint{1.477854in}{2.141661in}}%
\pgfpathlineto{\pgfqpoint{1.574127in}{2.844906in}}%
\pgfusepath{stroke}%
\end{pgfscope}%
\begin{pgfscope}%
\pgfpathrectangle{\pgfqpoint{0.100000in}{0.212622in}}{\pgfqpoint{3.696000in}{3.696000in}}%
\pgfusepath{clip}%
\pgfsetrectcap%
\pgfsetroundjoin%
\pgfsetlinewidth{1.505625pt}%
\definecolor{currentstroke}{rgb}{1.000000,0.000000,0.000000}%
\pgfsetstrokecolor{currentstroke}%
\pgfsetdash{}{0pt}%
\pgfpathmoveto{\pgfqpoint{1.478031in}{2.141652in}}%
\pgfpathlineto{\pgfqpoint{1.574127in}{2.844906in}}%
\pgfusepath{stroke}%
\end{pgfscope}%
\begin{pgfscope}%
\pgfpathrectangle{\pgfqpoint{0.100000in}{0.212622in}}{\pgfqpoint{3.696000in}{3.696000in}}%
\pgfusepath{clip}%
\pgfsetrectcap%
\pgfsetroundjoin%
\pgfsetlinewidth{1.505625pt}%
\definecolor{currentstroke}{rgb}{1.000000,0.000000,0.000000}%
\pgfsetstrokecolor{currentstroke}%
\pgfsetdash{}{0pt}%
\pgfpathmoveto{\pgfqpoint{1.479149in}{2.141601in}}%
\pgfpathlineto{\pgfqpoint{1.574127in}{2.844906in}}%
\pgfusepath{stroke}%
\end{pgfscope}%
\begin{pgfscope}%
\pgfpathrectangle{\pgfqpoint{0.100000in}{0.212622in}}{\pgfqpoint{3.696000in}{3.696000in}}%
\pgfusepath{clip}%
\pgfsetrectcap%
\pgfsetroundjoin%
\pgfsetlinewidth{1.505625pt}%
\definecolor{currentstroke}{rgb}{1.000000,0.000000,0.000000}%
\pgfsetstrokecolor{currentstroke}%
\pgfsetdash{}{0pt}%
\pgfpathmoveto{\pgfqpoint{1.481797in}{2.141407in}}%
\pgfpathlineto{\pgfqpoint{1.574127in}{2.844906in}}%
\pgfusepath{stroke}%
\end{pgfscope}%
\begin{pgfscope}%
\pgfpathrectangle{\pgfqpoint{0.100000in}{0.212622in}}{\pgfqpoint{3.696000in}{3.696000in}}%
\pgfusepath{clip}%
\pgfsetrectcap%
\pgfsetroundjoin%
\pgfsetlinewidth{1.505625pt}%
\definecolor{currentstroke}{rgb}{1.000000,0.000000,0.000000}%
\pgfsetstrokecolor{currentstroke}%
\pgfsetdash{}{0pt}%
\pgfpathmoveto{\pgfqpoint{1.484030in}{2.141317in}}%
\pgfpathlineto{\pgfqpoint{1.582757in}{2.851688in}}%
\pgfusepath{stroke}%
\end{pgfscope}%
\begin{pgfscope}%
\pgfpathrectangle{\pgfqpoint{0.100000in}{0.212622in}}{\pgfqpoint{3.696000in}{3.696000in}}%
\pgfusepath{clip}%
\pgfsetrectcap%
\pgfsetroundjoin%
\pgfsetlinewidth{1.505625pt}%
\definecolor{currentstroke}{rgb}{1.000000,0.000000,0.000000}%
\pgfsetstrokecolor{currentstroke}%
\pgfsetdash{}{0pt}%
\pgfpathmoveto{\pgfqpoint{1.484843in}{2.140937in}}%
\pgfpathlineto{\pgfqpoint{1.582757in}{2.851688in}}%
\pgfusepath{stroke}%
\end{pgfscope}%
\begin{pgfscope}%
\pgfpathrectangle{\pgfqpoint{0.100000in}{0.212622in}}{\pgfqpoint{3.696000in}{3.696000in}}%
\pgfusepath{clip}%
\pgfsetrectcap%
\pgfsetroundjoin%
\pgfsetlinewidth{1.505625pt}%
\definecolor{currentstroke}{rgb}{1.000000,0.000000,0.000000}%
\pgfsetstrokecolor{currentstroke}%
\pgfsetdash{}{0pt}%
\pgfpathmoveto{\pgfqpoint{1.485722in}{2.140838in}}%
\pgfpathlineto{\pgfqpoint{1.582757in}{2.851688in}}%
\pgfusepath{stroke}%
\end{pgfscope}%
\begin{pgfscope}%
\pgfpathrectangle{\pgfqpoint{0.100000in}{0.212622in}}{\pgfqpoint{3.696000in}{3.696000in}}%
\pgfusepath{clip}%
\pgfsetrectcap%
\pgfsetroundjoin%
\pgfsetlinewidth{1.505625pt}%
\definecolor{currentstroke}{rgb}{1.000000,0.000000,0.000000}%
\pgfsetstrokecolor{currentstroke}%
\pgfsetdash{}{0pt}%
\pgfpathmoveto{\pgfqpoint{1.487035in}{2.140739in}}%
\pgfpathlineto{\pgfqpoint{1.582757in}{2.851688in}}%
\pgfusepath{stroke}%
\end{pgfscope}%
\begin{pgfscope}%
\pgfpathrectangle{\pgfqpoint{0.100000in}{0.212622in}}{\pgfqpoint{3.696000in}{3.696000in}}%
\pgfusepath{clip}%
\pgfsetrectcap%
\pgfsetroundjoin%
\pgfsetlinewidth{1.505625pt}%
\definecolor{currentstroke}{rgb}{1.000000,0.000000,0.000000}%
\pgfsetstrokecolor{currentstroke}%
\pgfsetdash{}{0pt}%
\pgfpathmoveto{\pgfqpoint{1.488690in}{2.140694in}}%
\pgfpathlineto{\pgfqpoint{1.582757in}{2.851688in}}%
\pgfusepath{stroke}%
\end{pgfscope}%
\begin{pgfscope}%
\pgfpathrectangle{\pgfqpoint{0.100000in}{0.212622in}}{\pgfqpoint{3.696000in}{3.696000in}}%
\pgfusepath{clip}%
\pgfsetrectcap%
\pgfsetroundjoin%
\pgfsetlinewidth{1.505625pt}%
\definecolor{currentstroke}{rgb}{1.000000,0.000000,0.000000}%
\pgfsetstrokecolor{currentstroke}%
\pgfsetdash{}{0pt}%
\pgfpathmoveto{\pgfqpoint{1.489830in}{2.140589in}}%
\pgfpathlineto{\pgfqpoint{1.582757in}{2.851688in}}%
\pgfusepath{stroke}%
\end{pgfscope}%
\begin{pgfscope}%
\pgfpathrectangle{\pgfqpoint{0.100000in}{0.212622in}}{\pgfqpoint{3.696000in}{3.696000in}}%
\pgfusepath{clip}%
\pgfsetrectcap%
\pgfsetroundjoin%
\pgfsetlinewidth{1.505625pt}%
\definecolor{currentstroke}{rgb}{1.000000,0.000000,0.000000}%
\pgfsetstrokecolor{currentstroke}%
\pgfsetdash{}{0pt}%
\pgfpathmoveto{\pgfqpoint{1.491032in}{2.140516in}}%
\pgfpathlineto{\pgfqpoint{1.591377in}{2.858462in}}%
\pgfusepath{stroke}%
\end{pgfscope}%
\begin{pgfscope}%
\pgfpathrectangle{\pgfqpoint{0.100000in}{0.212622in}}{\pgfqpoint{3.696000in}{3.696000in}}%
\pgfusepath{clip}%
\pgfsetrectcap%
\pgfsetroundjoin%
\pgfsetlinewidth{1.505625pt}%
\definecolor{currentstroke}{rgb}{1.000000,0.000000,0.000000}%
\pgfsetstrokecolor{currentstroke}%
\pgfsetdash{}{0pt}%
\pgfpathmoveto{\pgfqpoint{1.493134in}{2.140345in}}%
\pgfpathlineto{\pgfqpoint{1.591377in}{2.858462in}}%
\pgfusepath{stroke}%
\end{pgfscope}%
\begin{pgfscope}%
\pgfpathrectangle{\pgfqpoint{0.100000in}{0.212622in}}{\pgfqpoint{3.696000in}{3.696000in}}%
\pgfusepath{clip}%
\pgfsetrectcap%
\pgfsetroundjoin%
\pgfsetlinewidth{1.505625pt}%
\definecolor{currentstroke}{rgb}{1.000000,0.000000,0.000000}%
\pgfsetstrokecolor{currentstroke}%
\pgfsetdash{}{0pt}%
\pgfpathmoveto{\pgfqpoint{1.494248in}{2.140304in}}%
\pgfpathlineto{\pgfqpoint{1.591377in}{2.858462in}}%
\pgfusepath{stroke}%
\end{pgfscope}%
\begin{pgfscope}%
\pgfpathrectangle{\pgfqpoint{0.100000in}{0.212622in}}{\pgfqpoint{3.696000in}{3.696000in}}%
\pgfusepath{clip}%
\pgfsetrectcap%
\pgfsetroundjoin%
\pgfsetlinewidth{1.505625pt}%
\definecolor{currentstroke}{rgb}{1.000000,0.000000,0.000000}%
\pgfsetstrokecolor{currentstroke}%
\pgfsetdash{}{0pt}%
\pgfpathmoveto{\pgfqpoint{1.495658in}{2.140252in}}%
\pgfpathlineto{\pgfqpoint{1.591377in}{2.858462in}}%
\pgfusepath{stroke}%
\end{pgfscope}%
\begin{pgfscope}%
\pgfpathrectangle{\pgfqpoint{0.100000in}{0.212622in}}{\pgfqpoint{3.696000in}{3.696000in}}%
\pgfusepath{clip}%
\pgfsetrectcap%
\pgfsetroundjoin%
\pgfsetlinewidth{1.505625pt}%
\definecolor{currentstroke}{rgb}{1.000000,0.000000,0.000000}%
\pgfsetstrokecolor{currentstroke}%
\pgfsetdash{}{0pt}%
\pgfpathmoveto{\pgfqpoint{1.499094in}{2.139962in}}%
\pgfpathlineto{\pgfqpoint{1.599986in}{2.865228in}}%
\pgfusepath{stroke}%
\end{pgfscope}%
\begin{pgfscope}%
\pgfpathrectangle{\pgfqpoint{0.100000in}{0.212622in}}{\pgfqpoint{3.696000in}{3.696000in}}%
\pgfusepath{clip}%
\pgfsetrectcap%
\pgfsetroundjoin%
\pgfsetlinewidth{1.505625pt}%
\definecolor{currentstroke}{rgb}{1.000000,0.000000,0.000000}%
\pgfsetstrokecolor{currentstroke}%
\pgfsetdash{}{0pt}%
\pgfpathmoveto{\pgfqpoint{1.501940in}{2.139892in}}%
\pgfpathlineto{\pgfqpoint{1.599986in}{2.865228in}}%
\pgfusepath{stroke}%
\end{pgfscope}%
\begin{pgfscope}%
\pgfpathrectangle{\pgfqpoint{0.100000in}{0.212622in}}{\pgfqpoint{3.696000in}{3.696000in}}%
\pgfusepath{clip}%
\pgfsetrectcap%
\pgfsetroundjoin%
\pgfsetlinewidth{1.505625pt}%
\definecolor{currentstroke}{rgb}{1.000000,0.000000,0.000000}%
\pgfsetstrokecolor{currentstroke}%
\pgfsetdash{}{0pt}%
\pgfpathmoveto{\pgfqpoint{1.499273in}{2.138569in}}%
\pgfpathlineto{\pgfqpoint{1.599986in}{2.865228in}}%
\pgfusepath{stroke}%
\end{pgfscope}%
\begin{pgfscope}%
\pgfpathrectangle{\pgfqpoint{0.100000in}{0.212622in}}{\pgfqpoint{3.696000in}{3.696000in}}%
\pgfusepath{clip}%
\pgfsetrectcap%
\pgfsetroundjoin%
\pgfsetlinewidth{1.505625pt}%
\definecolor{currentstroke}{rgb}{1.000000,0.000000,0.000000}%
\pgfsetstrokecolor{currentstroke}%
\pgfsetdash{}{0pt}%
\pgfpathmoveto{\pgfqpoint{1.503282in}{2.138281in}}%
\pgfpathlineto{\pgfqpoint{1.608584in}{2.871985in}}%
\pgfusepath{stroke}%
\end{pgfscope}%
\begin{pgfscope}%
\pgfpathrectangle{\pgfqpoint{0.100000in}{0.212622in}}{\pgfqpoint{3.696000in}{3.696000in}}%
\pgfusepath{clip}%
\pgfsetrectcap%
\pgfsetroundjoin%
\pgfsetlinewidth{1.505625pt}%
\definecolor{currentstroke}{rgb}{1.000000,0.000000,0.000000}%
\pgfsetstrokecolor{currentstroke}%
\pgfsetdash{}{0pt}%
\pgfpathmoveto{\pgfqpoint{1.506879in}{2.138200in}}%
\pgfpathlineto{\pgfqpoint{1.608584in}{2.871985in}}%
\pgfusepath{stroke}%
\end{pgfscope}%
\begin{pgfscope}%
\pgfpathrectangle{\pgfqpoint{0.100000in}{0.212622in}}{\pgfqpoint{3.696000in}{3.696000in}}%
\pgfusepath{clip}%
\pgfsetrectcap%
\pgfsetroundjoin%
\pgfsetlinewidth{1.505625pt}%
\definecolor{currentstroke}{rgb}{1.000000,0.000000,0.000000}%
\pgfsetstrokecolor{currentstroke}%
\pgfsetdash{}{0pt}%
\pgfpathmoveto{\pgfqpoint{1.507514in}{2.137555in}}%
\pgfpathlineto{\pgfqpoint{1.608584in}{2.871985in}}%
\pgfusepath{stroke}%
\end{pgfscope}%
\begin{pgfscope}%
\pgfpathrectangle{\pgfqpoint{0.100000in}{0.212622in}}{\pgfqpoint{3.696000in}{3.696000in}}%
\pgfusepath{clip}%
\pgfsetrectcap%
\pgfsetroundjoin%
\pgfsetlinewidth{1.505625pt}%
\definecolor{currentstroke}{rgb}{1.000000,0.000000,0.000000}%
\pgfsetstrokecolor{currentstroke}%
\pgfsetdash{}{0pt}%
\pgfpathmoveto{\pgfqpoint{1.510234in}{2.137353in}}%
\pgfpathlineto{\pgfqpoint{1.617171in}{2.878733in}}%
\pgfusepath{stroke}%
\end{pgfscope}%
\begin{pgfscope}%
\pgfpathrectangle{\pgfqpoint{0.100000in}{0.212622in}}{\pgfqpoint{3.696000in}{3.696000in}}%
\pgfusepath{clip}%
\pgfsetrectcap%
\pgfsetroundjoin%
\pgfsetlinewidth{1.505625pt}%
\definecolor{currentstroke}{rgb}{1.000000,0.000000,0.000000}%
\pgfsetstrokecolor{currentstroke}%
\pgfsetdash{}{0pt}%
\pgfpathmoveto{\pgfqpoint{1.512255in}{2.137183in}}%
\pgfpathlineto{\pgfqpoint{1.617171in}{2.878733in}}%
\pgfusepath{stroke}%
\end{pgfscope}%
\begin{pgfscope}%
\pgfpathrectangle{\pgfqpoint{0.100000in}{0.212622in}}{\pgfqpoint{3.696000in}{3.696000in}}%
\pgfusepath{clip}%
\pgfsetrectcap%
\pgfsetroundjoin%
\pgfsetlinewidth{1.505625pt}%
\definecolor{currentstroke}{rgb}{1.000000,0.000000,0.000000}%
\pgfsetstrokecolor{currentstroke}%
\pgfsetdash{}{0pt}%
\pgfpathmoveto{\pgfqpoint{1.513969in}{2.137091in}}%
\pgfpathlineto{\pgfqpoint{1.617171in}{2.878733in}}%
\pgfusepath{stroke}%
\end{pgfscope}%
\begin{pgfscope}%
\pgfpathrectangle{\pgfqpoint{0.100000in}{0.212622in}}{\pgfqpoint{3.696000in}{3.696000in}}%
\pgfusepath{clip}%
\pgfsetrectcap%
\pgfsetroundjoin%
\pgfsetlinewidth{1.505625pt}%
\definecolor{currentstroke}{rgb}{1.000000,0.000000,0.000000}%
\pgfsetstrokecolor{currentstroke}%
\pgfsetdash{}{0pt}%
\pgfpathmoveto{\pgfqpoint{1.515878in}{2.137042in}}%
\pgfpathlineto{\pgfqpoint{1.625748in}{2.885473in}}%
\pgfusepath{stroke}%
\end{pgfscope}%
\begin{pgfscope}%
\pgfpathrectangle{\pgfqpoint{0.100000in}{0.212622in}}{\pgfqpoint{3.696000in}{3.696000in}}%
\pgfusepath{clip}%
\pgfsetrectcap%
\pgfsetroundjoin%
\pgfsetlinewidth{1.505625pt}%
\definecolor{currentstroke}{rgb}{1.000000,0.000000,0.000000}%
\pgfsetstrokecolor{currentstroke}%
\pgfsetdash{}{0pt}%
\pgfpathmoveto{\pgfqpoint{1.516658in}{2.137051in}}%
\pgfpathlineto{\pgfqpoint{1.625748in}{2.885473in}}%
\pgfusepath{stroke}%
\end{pgfscope}%
\begin{pgfscope}%
\pgfpathrectangle{\pgfqpoint{0.100000in}{0.212622in}}{\pgfqpoint{3.696000in}{3.696000in}}%
\pgfusepath{clip}%
\pgfsetrectcap%
\pgfsetroundjoin%
\pgfsetlinewidth{1.505625pt}%
\definecolor{currentstroke}{rgb}{1.000000,0.000000,0.000000}%
\pgfsetstrokecolor{currentstroke}%
\pgfsetdash{}{0pt}%
\pgfpathmoveto{\pgfqpoint{1.517246in}{2.137014in}}%
\pgfpathlineto{\pgfqpoint{1.625748in}{2.885473in}}%
\pgfusepath{stroke}%
\end{pgfscope}%
\begin{pgfscope}%
\pgfpathrectangle{\pgfqpoint{0.100000in}{0.212622in}}{\pgfqpoint{3.696000in}{3.696000in}}%
\pgfusepath{clip}%
\pgfsetrectcap%
\pgfsetroundjoin%
\pgfsetlinewidth{1.505625pt}%
\definecolor{currentstroke}{rgb}{1.000000,0.000000,0.000000}%
\pgfsetstrokecolor{currentstroke}%
\pgfsetdash{}{0pt}%
\pgfpathmoveto{\pgfqpoint{1.517966in}{2.136996in}}%
\pgfpathlineto{\pgfqpoint{1.625748in}{2.885473in}}%
\pgfusepath{stroke}%
\end{pgfscope}%
\begin{pgfscope}%
\pgfpathrectangle{\pgfqpoint{0.100000in}{0.212622in}}{\pgfqpoint{3.696000in}{3.696000in}}%
\pgfusepath{clip}%
\pgfsetrectcap%
\pgfsetroundjoin%
\pgfsetlinewidth{1.505625pt}%
\definecolor{currentstroke}{rgb}{1.000000,0.000000,0.000000}%
\pgfsetstrokecolor{currentstroke}%
\pgfsetdash{}{0pt}%
\pgfpathmoveto{\pgfqpoint{1.519602in}{2.136983in}}%
\pgfpathlineto{\pgfqpoint{1.625748in}{2.885473in}}%
\pgfusepath{stroke}%
\end{pgfscope}%
\begin{pgfscope}%
\pgfpathrectangle{\pgfqpoint{0.100000in}{0.212622in}}{\pgfqpoint{3.696000in}{3.696000in}}%
\pgfusepath{clip}%
\pgfsetrectcap%
\pgfsetroundjoin%
\pgfsetlinewidth{1.505625pt}%
\definecolor{currentstroke}{rgb}{1.000000,0.000000,0.000000}%
\pgfsetstrokecolor{currentstroke}%
\pgfsetdash{}{0pt}%
\pgfpathmoveto{\pgfqpoint{1.522552in}{2.136733in}}%
\pgfpathlineto{\pgfqpoint{1.625748in}{2.885473in}}%
\pgfusepath{stroke}%
\end{pgfscope}%
\begin{pgfscope}%
\pgfpathrectangle{\pgfqpoint{0.100000in}{0.212622in}}{\pgfqpoint{3.696000in}{3.696000in}}%
\pgfusepath{clip}%
\pgfsetrectcap%
\pgfsetroundjoin%
\pgfsetlinewidth{1.505625pt}%
\definecolor{currentstroke}{rgb}{1.000000,0.000000,0.000000}%
\pgfsetstrokecolor{currentstroke}%
\pgfsetdash{}{0pt}%
\pgfpathmoveto{\pgfqpoint{1.525118in}{2.136659in}}%
\pgfpathlineto{\pgfqpoint{1.634313in}{2.892205in}}%
\pgfusepath{stroke}%
\end{pgfscope}%
\begin{pgfscope}%
\pgfpathrectangle{\pgfqpoint{0.100000in}{0.212622in}}{\pgfqpoint{3.696000in}{3.696000in}}%
\pgfusepath{clip}%
\pgfsetrectcap%
\pgfsetroundjoin%
\pgfsetlinewidth{1.505625pt}%
\definecolor{currentstroke}{rgb}{1.000000,0.000000,0.000000}%
\pgfsetstrokecolor{currentstroke}%
\pgfsetdash{}{0pt}%
\pgfpathmoveto{\pgfqpoint{1.526305in}{2.136815in}}%
\pgfpathlineto{\pgfqpoint{1.634313in}{2.892205in}}%
\pgfusepath{stroke}%
\end{pgfscope}%
\begin{pgfscope}%
\pgfpathrectangle{\pgfqpoint{0.100000in}{0.212622in}}{\pgfqpoint{3.696000in}{3.696000in}}%
\pgfusepath{clip}%
\pgfsetrectcap%
\pgfsetroundjoin%
\pgfsetlinewidth{1.505625pt}%
\definecolor{currentstroke}{rgb}{1.000000,0.000000,0.000000}%
\pgfsetstrokecolor{currentstroke}%
\pgfsetdash{}{0pt}%
\pgfpathmoveto{\pgfqpoint{1.528196in}{2.136676in}}%
\pgfpathlineto{\pgfqpoint{1.634313in}{2.892205in}}%
\pgfusepath{stroke}%
\end{pgfscope}%
\begin{pgfscope}%
\pgfpathrectangle{\pgfqpoint{0.100000in}{0.212622in}}{\pgfqpoint{3.696000in}{3.696000in}}%
\pgfusepath{clip}%
\pgfsetrectcap%
\pgfsetroundjoin%
\pgfsetlinewidth{1.505625pt}%
\definecolor{currentstroke}{rgb}{1.000000,0.000000,0.000000}%
\pgfsetstrokecolor{currentstroke}%
\pgfsetdash{}{0pt}%
\pgfpathmoveto{\pgfqpoint{1.530259in}{2.136630in}}%
\pgfpathlineto{\pgfqpoint{1.634313in}{2.892205in}}%
\pgfusepath{stroke}%
\end{pgfscope}%
\begin{pgfscope}%
\pgfpathrectangle{\pgfqpoint{0.100000in}{0.212622in}}{\pgfqpoint{3.696000in}{3.696000in}}%
\pgfusepath{clip}%
\pgfsetrectcap%
\pgfsetroundjoin%
\pgfsetlinewidth{1.505625pt}%
\definecolor{currentstroke}{rgb}{1.000000,0.000000,0.000000}%
\pgfsetstrokecolor{currentstroke}%
\pgfsetdash{}{0pt}%
\pgfpathmoveto{\pgfqpoint{1.532213in}{2.136590in}}%
\pgfpathlineto{\pgfqpoint{1.642868in}{2.898928in}}%
\pgfusepath{stroke}%
\end{pgfscope}%
\begin{pgfscope}%
\pgfpathrectangle{\pgfqpoint{0.100000in}{0.212622in}}{\pgfqpoint{3.696000in}{3.696000in}}%
\pgfusepath{clip}%
\pgfsetrectcap%
\pgfsetroundjoin%
\pgfsetlinewidth{1.505625pt}%
\definecolor{currentstroke}{rgb}{1.000000,0.000000,0.000000}%
\pgfsetstrokecolor{currentstroke}%
\pgfsetdash{}{0pt}%
\pgfpathmoveto{\pgfqpoint{1.533858in}{2.136444in}}%
\pgfpathlineto{\pgfqpoint{1.642868in}{2.898928in}}%
\pgfusepath{stroke}%
\end{pgfscope}%
\begin{pgfscope}%
\pgfpathrectangle{\pgfqpoint{0.100000in}{0.212622in}}{\pgfqpoint{3.696000in}{3.696000in}}%
\pgfusepath{clip}%
\pgfsetrectcap%
\pgfsetroundjoin%
\pgfsetlinewidth{1.505625pt}%
\definecolor{currentstroke}{rgb}{1.000000,0.000000,0.000000}%
\pgfsetstrokecolor{currentstroke}%
\pgfsetdash{}{0pt}%
\pgfpathmoveto{\pgfqpoint{1.534336in}{2.136412in}}%
\pgfpathlineto{\pgfqpoint{1.642868in}{2.898928in}}%
\pgfusepath{stroke}%
\end{pgfscope}%
\begin{pgfscope}%
\pgfpathrectangle{\pgfqpoint{0.100000in}{0.212622in}}{\pgfqpoint{3.696000in}{3.696000in}}%
\pgfusepath{clip}%
\pgfsetrectcap%
\pgfsetroundjoin%
\pgfsetlinewidth{1.505625pt}%
\definecolor{currentstroke}{rgb}{1.000000,0.000000,0.000000}%
\pgfsetstrokecolor{currentstroke}%
\pgfsetdash{}{0pt}%
\pgfpathmoveto{\pgfqpoint{1.534978in}{2.136386in}}%
\pgfpathlineto{\pgfqpoint{1.642868in}{2.898928in}}%
\pgfusepath{stroke}%
\end{pgfscope}%
\begin{pgfscope}%
\pgfpathrectangle{\pgfqpoint{0.100000in}{0.212622in}}{\pgfqpoint{3.696000in}{3.696000in}}%
\pgfusepath{clip}%
\pgfsetrectcap%
\pgfsetroundjoin%
\pgfsetlinewidth{1.505625pt}%
\definecolor{currentstroke}{rgb}{1.000000,0.000000,0.000000}%
\pgfsetstrokecolor{currentstroke}%
\pgfsetdash{}{0pt}%
\pgfpathmoveto{\pgfqpoint{1.536224in}{2.136328in}}%
\pgfpathlineto{\pgfqpoint{1.642868in}{2.898928in}}%
\pgfusepath{stroke}%
\end{pgfscope}%
\begin{pgfscope}%
\pgfpathrectangle{\pgfqpoint{0.100000in}{0.212622in}}{\pgfqpoint{3.696000in}{3.696000in}}%
\pgfusepath{clip}%
\pgfsetrectcap%
\pgfsetroundjoin%
\pgfsetlinewidth{1.505625pt}%
\definecolor{currentstroke}{rgb}{1.000000,0.000000,0.000000}%
\pgfsetstrokecolor{currentstroke}%
\pgfsetdash{}{0pt}%
\pgfpathmoveto{\pgfqpoint{1.536823in}{2.136292in}}%
\pgfpathlineto{\pgfqpoint{1.642868in}{2.898928in}}%
\pgfusepath{stroke}%
\end{pgfscope}%
\begin{pgfscope}%
\pgfpathrectangle{\pgfqpoint{0.100000in}{0.212622in}}{\pgfqpoint{3.696000in}{3.696000in}}%
\pgfusepath{clip}%
\pgfsetrectcap%
\pgfsetroundjoin%
\pgfsetlinewidth{1.505625pt}%
\definecolor{currentstroke}{rgb}{1.000000,0.000000,0.000000}%
\pgfsetstrokecolor{currentstroke}%
\pgfsetdash{}{0pt}%
\pgfpathmoveto{\pgfqpoint{1.538494in}{2.136255in}}%
\pgfpathlineto{\pgfqpoint{1.651413in}{2.905643in}}%
\pgfusepath{stroke}%
\end{pgfscope}%
\begin{pgfscope}%
\pgfpathrectangle{\pgfqpoint{0.100000in}{0.212622in}}{\pgfqpoint{3.696000in}{3.696000in}}%
\pgfusepath{clip}%
\pgfsetrectcap%
\pgfsetroundjoin%
\pgfsetlinewidth{1.505625pt}%
\definecolor{currentstroke}{rgb}{1.000000,0.000000,0.000000}%
\pgfsetstrokecolor{currentstroke}%
\pgfsetdash{}{0pt}%
\pgfpathmoveto{\pgfqpoint{1.541316in}{2.135969in}}%
\pgfpathlineto{\pgfqpoint{1.651413in}{2.905643in}}%
\pgfusepath{stroke}%
\end{pgfscope}%
\begin{pgfscope}%
\pgfpathrectangle{\pgfqpoint{0.100000in}{0.212622in}}{\pgfqpoint{3.696000in}{3.696000in}}%
\pgfusepath{clip}%
\pgfsetrectcap%
\pgfsetroundjoin%
\pgfsetlinewidth{1.505625pt}%
\definecolor{currentstroke}{rgb}{1.000000,0.000000,0.000000}%
\pgfsetstrokecolor{currentstroke}%
\pgfsetdash{}{0pt}%
\pgfpathmoveto{\pgfqpoint{1.543771in}{2.135899in}}%
\pgfpathlineto{\pgfqpoint{1.651413in}{2.905643in}}%
\pgfusepath{stroke}%
\end{pgfscope}%
\begin{pgfscope}%
\pgfpathrectangle{\pgfqpoint{0.100000in}{0.212622in}}{\pgfqpoint{3.696000in}{3.696000in}}%
\pgfusepath{clip}%
\pgfsetrectcap%
\pgfsetroundjoin%
\pgfsetlinewidth{1.505625pt}%
\definecolor{currentstroke}{rgb}{1.000000,0.000000,0.000000}%
\pgfsetstrokecolor{currentstroke}%
\pgfsetdash{}{0pt}%
\pgfpathmoveto{\pgfqpoint{1.544690in}{2.135936in}}%
\pgfpathlineto{\pgfqpoint{1.651413in}{2.905643in}}%
\pgfusepath{stroke}%
\end{pgfscope}%
\begin{pgfscope}%
\pgfpathrectangle{\pgfqpoint{0.100000in}{0.212622in}}{\pgfqpoint{3.696000in}{3.696000in}}%
\pgfusepath{clip}%
\pgfsetrectcap%
\pgfsetroundjoin%
\pgfsetlinewidth{1.505625pt}%
\definecolor{currentstroke}{rgb}{1.000000,0.000000,0.000000}%
\pgfsetstrokecolor{currentstroke}%
\pgfsetdash{}{0pt}%
\pgfpathmoveto{\pgfqpoint{1.546642in}{2.135793in}}%
\pgfpathlineto{\pgfqpoint{1.659946in}{2.912350in}}%
\pgfusepath{stroke}%
\end{pgfscope}%
\begin{pgfscope}%
\pgfpathrectangle{\pgfqpoint{0.100000in}{0.212622in}}{\pgfqpoint{3.696000in}{3.696000in}}%
\pgfusepath{clip}%
\pgfsetrectcap%
\pgfsetroundjoin%
\pgfsetlinewidth{1.505625pt}%
\definecolor{currentstroke}{rgb}{1.000000,0.000000,0.000000}%
\pgfsetstrokecolor{currentstroke}%
\pgfsetdash{}{0pt}%
\pgfpathmoveto{\pgfqpoint{1.548583in}{2.135708in}}%
\pgfpathlineto{\pgfqpoint{1.659946in}{2.912350in}}%
\pgfusepath{stroke}%
\end{pgfscope}%
\begin{pgfscope}%
\pgfpathrectangle{\pgfqpoint{0.100000in}{0.212622in}}{\pgfqpoint{3.696000in}{3.696000in}}%
\pgfusepath{clip}%
\pgfsetrectcap%
\pgfsetroundjoin%
\pgfsetlinewidth{1.505625pt}%
\definecolor{currentstroke}{rgb}{1.000000,0.000000,0.000000}%
\pgfsetstrokecolor{currentstroke}%
\pgfsetdash{}{0pt}%
\pgfpathmoveto{\pgfqpoint{1.547446in}{2.134990in}}%
\pgfpathlineto{\pgfqpoint{1.659946in}{2.912350in}}%
\pgfusepath{stroke}%
\end{pgfscope}%
\begin{pgfscope}%
\pgfpathrectangle{\pgfqpoint{0.100000in}{0.212622in}}{\pgfqpoint{3.696000in}{3.696000in}}%
\pgfusepath{clip}%
\pgfsetrectcap%
\pgfsetroundjoin%
\pgfsetlinewidth{1.505625pt}%
\definecolor{currentstroke}{rgb}{1.000000,0.000000,0.000000}%
\pgfsetstrokecolor{currentstroke}%
\pgfsetdash{}{0pt}%
\pgfpathmoveto{\pgfqpoint{1.549983in}{2.134937in}}%
\pgfpathlineto{\pgfqpoint{1.659946in}{2.912350in}}%
\pgfusepath{stroke}%
\end{pgfscope}%
\begin{pgfscope}%
\pgfpathrectangle{\pgfqpoint{0.100000in}{0.212622in}}{\pgfqpoint{3.696000in}{3.696000in}}%
\pgfusepath{clip}%
\pgfsetrectcap%
\pgfsetroundjoin%
\pgfsetlinewidth{1.505625pt}%
\definecolor{currentstroke}{rgb}{1.000000,0.000000,0.000000}%
\pgfsetstrokecolor{currentstroke}%
\pgfsetdash{}{0pt}%
\pgfpathmoveto{\pgfqpoint{1.552002in}{2.134709in}}%
\pgfpathlineto{\pgfqpoint{1.659946in}{2.912350in}}%
\pgfusepath{stroke}%
\end{pgfscope}%
\begin{pgfscope}%
\pgfpathrectangle{\pgfqpoint{0.100000in}{0.212622in}}{\pgfqpoint{3.696000in}{3.696000in}}%
\pgfusepath{clip}%
\pgfsetrectcap%
\pgfsetroundjoin%
\pgfsetlinewidth{1.505625pt}%
\definecolor{currentstroke}{rgb}{1.000000,0.000000,0.000000}%
\pgfsetstrokecolor{currentstroke}%
\pgfsetdash{}{0pt}%
\pgfpathmoveto{\pgfqpoint{1.553548in}{2.134653in}}%
\pgfpathlineto{\pgfqpoint{1.659946in}{2.912350in}}%
\pgfusepath{stroke}%
\end{pgfscope}%
\begin{pgfscope}%
\pgfpathrectangle{\pgfqpoint{0.100000in}{0.212622in}}{\pgfqpoint{3.696000in}{3.696000in}}%
\pgfusepath{clip}%
\pgfsetrectcap%
\pgfsetroundjoin%
\pgfsetlinewidth{1.505625pt}%
\definecolor{currentstroke}{rgb}{1.000000,0.000000,0.000000}%
\pgfsetstrokecolor{currentstroke}%
\pgfsetdash{}{0pt}%
\pgfpathmoveto{\pgfqpoint{1.555256in}{2.134549in}}%
\pgfpathlineto{\pgfqpoint{1.659946in}{2.912350in}}%
\pgfusepath{stroke}%
\end{pgfscope}%
\begin{pgfscope}%
\pgfpathrectangle{\pgfqpoint{0.100000in}{0.212622in}}{\pgfqpoint{3.696000in}{3.696000in}}%
\pgfusepath{clip}%
\pgfsetrectcap%
\pgfsetroundjoin%
\pgfsetlinewidth{1.505625pt}%
\definecolor{currentstroke}{rgb}{1.000000,0.000000,0.000000}%
\pgfsetstrokecolor{currentstroke}%
\pgfsetdash{}{0pt}%
\pgfpathmoveto{\pgfqpoint{1.555852in}{2.134459in}}%
\pgfpathlineto{\pgfqpoint{1.659946in}{2.912350in}}%
\pgfusepath{stroke}%
\end{pgfscope}%
\begin{pgfscope}%
\pgfpathrectangle{\pgfqpoint{0.100000in}{0.212622in}}{\pgfqpoint{3.696000in}{3.696000in}}%
\pgfusepath{clip}%
\pgfsetrectcap%
\pgfsetroundjoin%
\pgfsetlinewidth{1.505625pt}%
\definecolor{currentstroke}{rgb}{1.000000,0.000000,0.000000}%
\pgfsetstrokecolor{currentstroke}%
\pgfsetdash{}{0pt}%
\pgfpathmoveto{\pgfqpoint{1.557187in}{2.134410in}}%
\pgfpathlineto{\pgfqpoint{1.659946in}{2.912350in}}%
\pgfusepath{stroke}%
\end{pgfscope}%
\begin{pgfscope}%
\pgfpathrectangle{\pgfqpoint{0.100000in}{0.212622in}}{\pgfqpoint{3.696000in}{3.696000in}}%
\pgfusepath{clip}%
\pgfsetrectcap%
\pgfsetroundjoin%
\pgfsetlinewidth{1.505625pt}%
\definecolor{currentstroke}{rgb}{1.000000,0.000000,0.000000}%
\pgfsetstrokecolor{currentstroke}%
\pgfsetdash{}{0pt}%
\pgfpathmoveto{\pgfqpoint{1.557957in}{2.134346in}}%
\pgfpathlineto{\pgfqpoint{1.659946in}{2.912350in}}%
\pgfusepath{stroke}%
\end{pgfscope}%
\begin{pgfscope}%
\pgfpathrectangle{\pgfqpoint{0.100000in}{0.212622in}}{\pgfqpoint{3.696000in}{3.696000in}}%
\pgfusepath{clip}%
\pgfsetrectcap%
\pgfsetroundjoin%
\pgfsetlinewidth{1.505625pt}%
\definecolor{currentstroke}{rgb}{1.000000,0.000000,0.000000}%
\pgfsetstrokecolor{currentstroke}%
\pgfsetdash{}{0pt}%
\pgfpathmoveto{\pgfqpoint{1.558245in}{2.134318in}}%
\pgfpathlineto{\pgfqpoint{1.659946in}{2.912350in}}%
\pgfusepath{stroke}%
\end{pgfscope}%
\begin{pgfscope}%
\pgfpathrectangle{\pgfqpoint{0.100000in}{0.212622in}}{\pgfqpoint{3.696000in}{3.696000in}}%
\pgfusepath{clip}%
\pgfsetrectcap%
\pgfsetroundjoin%
\pgfsetlinewidth{1.505625pt}%
\definecolor{currentstroke}{rgb}{1.000000,0.000000,0.000000}%
\pgfsetstrokecolor{currentstroke}%
\pgfsetdash{}{0pt}%
\pgfpathmoveto{\pgfqpoint{1.559379in}{2.134268in}}%
\pgfpathlineto{\pgfqpoint{1.659946in}{2.912350in}}%
\pgfusepath{stroke}%
\end{pgfscope}%
\begin{pgfscope}%
\pgfpathrectangle{\pgfqpoint{0.100000in}{0.212622in}}{\pgfqpoint{3.696000in}{3.696000in}}%
\pgfusepath{clip}%
\pgfsetrectcap%
\pgfsetroundjoin%
\pgfsetlinewidth{1.505625pt}%
\definecolor{currentstroke}{rgb}{1.000000,0.000000,0.000000}%
\pgfsetstrokecolor{currentstroke}%
\pgfsetdash{}{0pt}%
\pgfpathmoveto{\pgfqpoint{1.559951in}{2.134230in}}%
\pgfpathlineto{\pgfqpoint{1.659946in}{2.912350in}}%
\pgfusepath{stroke}%
\end{pgfscope}%
\begin{pgfscope}%
\pgfpathrectangle{\pgfqpoint{0.100000in}{0.212622in}}{\pgfqpoint{3.696000in}{3.696000in}}%
\pgfusepath{clip}%
\pgfsetrectcap%
\pgfsetroundjoin%
\pgfsetlinewidth{1.505625pt}%
\definecolor{currentstroke}{rgb}{1.000000,0.000000,0.000000}%
\pgfsetstrokecolor{currentstroke}%
\pgfsetdash{}{0pt}%
\pgfpathmoveto{\pgfqpoint{1.560212in}{2.134234in}}%
\pgfpathlineto{\pgfqpoint{1.659946in}{2.912350in}}%
\pgfusepath{stroke}%
\end{pgfscope}%
\begin{pgfscope}%
\pgfpathrectangle{\pgfqpoint{0.100000in}{0.212622in}}{\pgfqpoint{3.696000in}{3.696000in}}%
\pgfusepath{clip}%
\pgfsetrectcap%
\pgfsetroundjoin%
\pgfsetlinewidth{1.505625pt}%
\definecolor{currentstroke}{rgb}{1.000000,0.000000,0.000000}%
\pgfsetstrokecolor{currentstroke}%
\pgfsetdash{}{0pt}%
\pgfpathmoveto{\pgfqpoint{1.561184in}{2.134208in}}%
\pgfpathlineto{\pgfqpoint{1.659946in}{2.912350in}}%
\pgfusepath{stroke}%
\end{pgfscope}%
\begin{pgfscope}%
\pgfpathrectangle{\pgfqpoint{0.100000in}{0.212622in}}{\pgfqpoint{3.696000in}{3.696000in}}%
\pgfusepath{clip}%
\pgfsetrectcap%
\pgfsetroundjoin%
\pgfsetlinewidth{1.505625pt}%
\definecolor{currentstroke}{rgb}{1.000000,0.000000,0.000000}%
\pgfsetstrokecolor{currentstroke}%
\pgfsetdash{}{0pt}%
\pgfpathmoveto{\pgfqpoint{1.561684in}{2.134169in}}%
\pgfpathlineto{\pgfqpoint{1.659946in}{2.912350in}}%
\pgfusepath{stroke}%
\end{pgfscope}%
\begin{pgfscope}%
\pgfpathrectangle{\pgfqpoint{0.100000in}{0.212622in}}{\pgfqpoint{3.696000in}{3.696000in}}%
\pgfusepath{clip}%
\pgfsetrectcap%
\pgfsetroundjoin%
\pgfsetlinewidth{1.505625pt}%
\definecolor{currentstroke}{rgb}{1.000000,0.000000,0.000000}%
\pgfsetstrokecolor{currentstroke}%
\pgfsetdash{}{0pt}%
\pgfpathmoveto{\pgfqpoint{1.562280in}{2.134117in}}%
\pgfpathlineto{\pgfqpoint{1.659946in}{2.912350in}}%
\pgfusepath{stroke}%
\end{pgfscope}%
\begin{pgfscope}%
\pgfpathrectangle{\pgfqpoint{0.100000in}{0.212622in}}{\pgfqpoint{3.696000in}{3.696000in}}%
\pgfusepath{clip}%
\pgfsetrectcap%
\pgfsetroundjoin%
\pgfsetlinewidth{1.505625pt}%
\definecolor{currentstroke}{rgb}{1.000000,0.000000,0.000000}%
\pgfsetstrokecolor{currentstroke}%
\pgfsetdash{}{0pt}%
\pgfpathmoveto{\pgfqpoint{1.563462in}{2.134056in}}%
\pgfpathlineto{\pgfqpoint{1.659946in}{2.912350in}}%
\pgfusepath{stroke}%
\end{pgfscope}%
\begin{pgfscope}%
\pgfpathrectangle{\pgfqpoint{0.100000in}{0.212622in}}{\pgfqpoint{3.696000in}{3.696000in}}%
\pgfusepath{clip}%
\pgfsetrectcap%
\pgfsetroundjoin%
\pgfsetlinewidth{1.505625pt}%
\definecolor{currentstroke}{rgb}{1.000000,0.000000,0.000000}%
\pgfsetstrokecolor{currentstroke}%
\pgfsetdash{}{0pt}%
\pgfpathmoveto{\pgfqpoint{1.564791in}{2.133989in}}%
\pgfpathlineto{\pgfqpoint{1.659946in}{2.912350in}}%
\pgfusepath{stroke}%
\end{pgfscope}%
\begin{pgfscope}%
\pgfpathrectangle{\pgfqpoint{0.100000in}{0.212622in}}{\pgfqpoint{3.696000in}{3.696000in}}%
\pgfusepath{clip}%
\pgfsetrectcap%
\pgfsetroundjoin%
\pgfsetlinewidth{1.505625pt}%
\definecolor{currentstroke}{rgb}{1.000000,0.000000,0.000000}%
\pgfsetstrokecolor{currentstroke}%
\pgfsetdash{}{0pt}%
\pgfpathmoveto{\pgfqpoint{1.565819in}{2.134090in}}%
\pgfpathlineto{\pgfqpoint{1.659946in}{2.912350in}}%
\pgfusepath{stroke}%
\end{pgfscope}%
\begin{pgfscope}%
\pgfpathrectangle{\pgfqpoint{0.100000in}{0.212622in}}{\pgfqpoint{3.696000in}{3.696000in}}%
\pgfusepath{clip}%
\pgfsetrectcap%
\pgfsetroundjoin%
\pgfsetlinewidth{1.505625pt}%
\definecolor{currentstroke}{rgb}{1.000000,0.000000,0.000000}%
\pgfsetstrokecolor{currentstroke}%
\pgfsetdash{}{0pt}%
\pgfpathmoveto{\pgfqpoint{1.568245in}{2.133818in}}%
\pgfpathlineto{\pgfqpoint{1.659946in}{2.912350in}}%
\pgfusepath{stroke}%
\end{pgfscope}%
\begin{pgfscope}%
\pgfpathrectangle{\pgfqpoint{0.100000in}{0.212622in}}{\pgfqpoint{3.696000in}{3.696000in}}%
\pgfusepath{clip}%
\pgfsetrectcap%
\pgfsetroundjoin%
\pgfsetlinewidth{1.505625pt}%
\definecolor{currentstroke}{rgb}{1.000000,0.000000,0.000000}%
\pgfsetstrokecolor{currentstroke}%
\pgfsetdash{}{0pt}%
\pgfpathmoveto{\pgfqpoint{1.570201in}{2.133768in}}%
\pgfpathlineto{\pgfqpoint{1.659946in}{2.912350in}}%
\pgfusepath{stroke}%
\end{pgfscope}%
\begin{pgfscope}%
\pgfpathrectangle{\pgfqpoint{0.100000in}{0.212622in}}{\pgfqpoint{3.696000in}{3.696000in}}%
\pgfusepath{clip}%
\pgfsetrectcap%
\pgfsetroundjoin%
\pgfsetlinewidth{1.505625pt}%
\definecolor{currentstroke}{rgb}{1.000000,0.000000,0.000000}%
\pgfsetstrokecolor{currentstroke}%
\pgfsetdash{}{0pt}%
\pgfpathmoveto{\pgfqpoint{1.571305in}{2.133753in}}%
\pgfpathlineto{\pgfqpoint{1.659946in}{2.912350in}}%
\pgfusepath{stroke}%
\end{pgfscope}%
\begin{pgfscope}%
\pgfpathrectangle{\pgfqpoint{0.100000in}{0.212622in}}{\pgfqpoint{3.696000in}{3.696000in}}%
\pgfusepath{clip}%
\pgfsetrectcap%
\pgfsetroundjoin%
\pgfsetlinewidth{1.505625pt}%
\definecolor{currentstroke}{rgb}{1.000000,0.000000,0.000000}%
\pgfsetstrokecolor{currentstroke}%
\pgfsetdash{}{0pt}%
\pgfpathmoveto{\pgfqpoint{1.572131in}{2.133682in}}%
\pgfpathlineto{\pgfqpoint{1.659946in}{2.912350in}}%
\pgfusepath{stroke}%
\end{pgfscope}%
\begin{pgfscope}%
\pgfpathrectangle{\pgfqpoint{0.100000in}{0.212622in}}{\pgfqpoint{3.696000in}{3.696000in}}%
\pgfusepath{clip}%
\pgfsetrectcap%
\pgfsetroundjoin%
\pgfsetlinewidth{1.505625pt}%
\definecolor{currentstroke}{rgb}{1.000000,0.000000,0.000000}%
\pgfsetstrokecolor{currentstroke}%
\pgfsetdash{}{0pt}%
\pgfpathmoveto{\pgfqpoint{1.572909in}{2.133640in}}%
\pgfpathlineto{\pgfqpoint{1.659946in}{2.912350in}}%
\pgfusepath{stroke}%
\end{pgfscope}%
\begin{pgfscope}%
\pgfpathrectangle{\pgfqpoint{0.100000in}{0.212622in}}{\pgfqpoint{3.696000in}{3.696000in}}%
\pgfusepath{clip}%
\pgfsetrectcap%
\pgfsetroundjoin%
\pgfsetlinewidth{1.505625pt}%
\definecolor{currentstroke}{rgb}{1.000000,0.000000,0.000000}%
\pgfsetstrokecolor{currentstroke}%
\pgfsetdash{}{0pt}%
\pgfpathmoveto{\pgfqpoint{1.574292in}{2.133607in}}%
\pgfpathlineto{\pgfqpoint{1.659946in}{2.912350in}}%
\pgfusepath{stroke}%
\end{pgfscope}%
\begin{pgfscope}%
\pgfpathrectangle{\pgfqpoint{0.100000in}{0.212622in}}{\pgfqpoint{3.696000in}{3.696000in}}%
\pgfusepath{clip}%
\pgfsetrectcap%
\pgfsetroundjoin%
\pgfsetlinewidth{1.505625pt}%
\definecolor{currentstroke}{rgb}{1.000000,0.000000,0.000000}%
\pgfsetstrokecolor{currentstroke}%
\pgfsetdash{}{0pt}%
\pgfpathmoveto{\pgfqpoint{1.575842in}{2.133516in}}%
\pgfpathlineto{\pgfqpoint{1.659946in}{2.912350in}}%
\pgfusepath{stroke}%
\end{pgfscope}%
\begin{pgfscope}%
\pgfpathrectangle{\pgfqpoint{0.100000in}{0.212622in}}{\pgfqpoint{3.696000in}{3.696000in}}%
\pgfusepath{clip}%
\pgfsetrectcap%
\pgfsetroundjoin%
\pgfsetlinewidth{1.505625pt}%
\definecolor{currentstroke}{rgb}{1.000000,0.000000,0.000000}%
\pgfsetstrokecolor{currentstroke}%
\pgfsetdash{}{0pt}%
\pgfpathmoveto{\pgfqpoint{1.576511in}{2.133502in}}%
\pgfpathlineto{\pgfqpoint{1.659946in}{2.912350in}}%
\pgfusepath{stroke}%
\end{pgfscope}%
\begin{pgfscope}%
\pgfpathrectangle{\pgfqpoint{0.100000in}{0.212622in}}{\pgfqpoint{3.696000in}{3.696000in}}%
\pgfusepath{clip}%
\pgfsetrectcap%
\pgfsetroundjoin%
\pgfsetlinewidth{1.505625pt}%
\definecolor{currentstroke}{rgb}{1.000000,0.000000,0.000000}%
\pgfsetstrokecolor{currentstroke}%
\pgfsetdash{}{0pt}%
\pgfpathmoveto{\pgfqpoint{1.578228in}{2.133420in}}%
\pgfpathlineto{\pgfqpoint{1.659946in}{2.912350in}}%
\pgfusepath{stroke}%
\end{pgfscope}%
\begin{pgfscope}%
\pgfpathrectangle{\pgfqpoint{0.100000in}{0.212622in}}{\pgfqpoint{3.696000in}{3.696000in}}%
\pgfusepath{clip}%
\pgfsetrectcap%
\pgfsetroundjoin%
\pgfsetlinewidth{1.505625pt}%
\definecolor{currentstroke}{rgb}{1.000000,0.000000,0.000000}%
\pgfsetstrokecolor{currentstroke}%
\pgfsetdash{}{0pt}%
\pgfpathmoveto{\pgfqpoint{1.579008in}{2.133378in}}%
\pgfpathlineto{\pgfqpoint{1.659946in}{2.912350in}}%
\pgfusepath{stroke}%
\end{pgfscope}%
\begin{pgfscope}%
\pgfpathrectangle{\pgfqpoint{0.100000in}{0.212622in}}{\pgfqpoint{3.696000in}{3.696000in}}%
\pgfusepath{clip}%
\pgfsetrectcap%
\pgfsetroundjoin%
\pgfsetlinewidth{1.505625pt}%
\definecolor{currentstroke}{rgb}{1.000000,0.000000,0.000000}%
\pgfsetstrokecolor{currentstroke}%
\pgfsetdash{}{0pt}%
\pgfpathmoveto{\pgfqpoint{1.580625in}{2.133477in}}%
\pgfpathlineto{\pgfqpoint{1.659946in}{2.912350in}}%
\pgfusepath{stroke}%
\end{pgfscope}%
\begin{pgfscope}%
\pgfpathrectangle{\pgfqpoint{0.100000in}{0.212622in}}{\pgfqpoint{3.696000in}{3.696000in}}%
\pgfusepath{clip}%
\pgfsetrectcap%
\pgfsetroundjoin%
\pgfsetlinewidth{1.505625pt}%
\definecolor{currentstroke}{rgb}{1.000000,0.000000,0.000000}%
\pgfsetstrokecolor{currentstroke}%
\pgfsetdash{}{0pt}%
\pgfpathmoveto{\pgfqpoint{1.583802in}{2.133122in}}%
\pgfpathlineto{\pgfqpoint{1.659946in}{2.912350in}}%
\pgfusepath{stroke}%
\end{pgfscope}%
\begin{pgfscope}%
\pgfpathrectangle{\pgfqpoint{0.100000in}{0.212622in}}{\pgfqpoint{3.696000in}{3.696000in}}%
\pgfusepath{clip}%
\pgfsetrectcap%
\pgfsetroundjoin%
\pgfsetlinewidth{1.505625pt}%
\definecolor{currentstroke}{rgb}{1.000000,0.000000,0.000000}%
\pgfsetstrokecolor{currentstroke}%
\pgfsetdash{}{0pt}%
\pgfpathmoveto{\pgfqpoint{1.584943in}{2.133119in}}%
\pgfpathlineto{\pgfqpoint{1.659946in}{2.912350in}}%
\pgfusepath{stroke}%
\end{pgfscope}%
\begin{pgfscope}%
\pgfpathrectangle{\pgfqpoint{0.100000in}{0.212622in}}{\pgfqpoint{3.696000in}{3.696000in}}%
\pgfusepath{clip}%
\pgfsetrectcap%
\pgfsetroundjoin%
\pgfsetlinewidth{1.505625pt}%
\definecolor{currentstroke}{rgb}{1.000000,0.000000,0.000000}%
\pgfsetstrokecolor{currentstroke}%
\pgfsetdash{}{0pt}%
\pgfpathmoveto{\pgfqpoint{1.585495in}{2.132889in}}%
\pgfpathlineto{\pgfqpoint{1.659946in}{2.912350in}}%
\pgfusepath{stroke}%
\end{pgfscope}%
\begin{pgfscope}%
\pgfpathrectangle{\pgfqpoint{0.100000in}{0.212622in}}{\pgfqpoint{3.696000in}{3.696000in}}%
\pgfusepath{clip}%
\pgfsetrectcap%
\pgfsetroundjoin%
\pgfsetlinewidth{1.505625pt}%
\definecolor{currentstroke}{rgb}{1.000000,0.000000,0.000000}%
\pgfsetstrokecolor{currentstroke}%
\pgfsetdash{}{0pt}%
\pgfpathmoveto{\pgfqpoint{1.586542in}{2.132833in}}%
\pgfpathlineto{\pgfqpoint{1.659946in}{2.912350in}}%
\pgfusepath{stroke}%
\end{pgfscope}%
\begin{pgfscope}%
\pgfpathrectangle{\pgfqpoint{0.100000in}{0.212622in}}{\pgfqpoint{3.696000in}{3.696000in}}%
\pgfusepath{clip}%
\pgfsetrectcap%
\pgfsetroundjoin%
\pgfsetlinewidth{1.505625pt}%
\definecolor{currentstroke}{rgb}{1.000000,0.000000,0.000000}%
\pgfsetstrokecolor{currentstroke}%
\pgfsetdash{}{0pt}%
\pgfpathmoveto{\pgfqpoint{1.587956in}{2.132775in}}%
\pgfpathlineto{\pgfqpoint{1.659946in}{2.912350in}}%
\pgfusepath{stroke}%
\end{pgfscope}%
\begin{pgfscope}%
\pgfpathrectangle{\pgfqpoint{0.100000in}{0.212622in}}{\pgfqpoint{3.696000in}{3.696000in}}%
\pgfusepath{clip}%
\pgfsetrectcap%
\pgfsetroundjoin%
\pgfsetlinewidth{1.505625pt}%
\definecolor{currentstroke}{rgb}{1.000000,0.000000,0.000000}%
\pgfsetstrokecolor{currentstroke}%
\pgfsetdash{}{0pt}%
\pgfpathmoveto{\pgfqpoint{1.588661in}{2.132725in}}%
\pgfpathlineto{\pgfqpoint{1.659946in}{2.912350in}}%
\pgfusepath{stroke}%
\end{pgfscope}%
\begin{pgfscope}%
\pgfpathrectangle{\pgfqpoint{0.100000in}{0.212622in}}{\pgfqpoint{3.696000in}{3.696000in}}%
\pgfusepath{clip}%
\pgfsetrectcap%
\pgfsetroundjoin%
\pgfsetlinewidth{1.505625pt}%
\definecolor{currentstroke}{rgb}{1.000000,0.000000,0.000000}%
\pgfsetstrokecolor{currentstroke}%
\pgfsetdash{}{0pt}%
\pgfpathmoveto{\pgfqpoint{1.589214in}{2.132698in}}%
\pgfpathlineto{\pgfqpoint{1.659946in}{2.912350in}}%
\pgfusepath{stroke}%
\end{pgfscope}%
\begin{pgfscope}%
\pgfpathrectangle{\pgfqpoint{0.100000in}{0.212622in}}{\pgfqpoint{3.696000in}{3.696000in}}%
\pgfusepath{clip}%
\pgfsetrectcap%
\pgfsetroundjoin%
\pgfsetlinewidth{1.505625pt}%
\definecolor{currentstroke}{rgb}{1.000000,0.000000,0.000000}%
\pgfsetstrokecolor{currentstroke}%
\pgfsetdash{}{0pt}%
\pgfpathmoveto{\pgfqpoint{1.589400in}{2.132677in}}%
\pgfpathlineto{\pgfqpoint{1.659946in}{2.912350in}}%
\pgfusepath{stroke}%
\end{pgfscope}%
\begin{pgfscope}%
\pgfpathrectangle{\pgfqpoint{0.100000in}{0.212622in}}{\pgfqpoint{3.696000in}{3.696000in}}%
\pgfusepath{clip}%
\pgfsetrectcap%
\pgfsetroundjoin%
\pgfsetlinewidth{1.505625pt}%
\definecolor{currentstroke}{rgb}{1.000000,0.000000,0.000000}%
\pgfsetstrokecolor{currentstroke}%
\pgfsetdash{}{0pt}%
\pgfpathmoveto{\pgfqpoint{1.589887in}{2.132684in}}%
\pgfpathlineto{\pgfqpoint{1.659946in}{2.912350in}}%
\pgfusepath{stroke}%
\end{pgfscope}%
\begin{pgfscope}%
\pgfpathrectangle{\pgfqpoint{0.100000in}{0.212622in}}{\pgfqpoint{3.696000in}{3.696000in}}%
\pgfusepath{clip}%
\pgfsetrectcap%
\pgfsetroundjoin%
\pgfsetlinewidth{1.505625pt}%
\definecolor{currentstroke}{rgb}{1.000000,0.000000,0.000000}%
\pgfsetstrokecolor{currentstroke}%
\pgfsetdash{}{0pt}%
\pgfpathmoveto{\pgfqpoint{1.590213in}{2.132672in}}%
\pgfpathlineto{\pgfqpoint{1.659946in}{2.912350in}}%
\pgfusepath{stroke}%
\end{pgfscope}%
\begin{pgfscope}%
\pgfpathrectangle{\pgfqpoint{0.100000in}{0.212622in}}{\pgfqpoint{3.696000in}{3.696000in}}%
\pgfusepath{clip}%
\pgfsetrectcap%
\pgfsetroundjoin%
\pgfsetlinewidth{1.505625pt}%
\definecolor{currentstroke}{rgb}{1.000000,0.000000,0.000000}%
\pgfsetstrokecolor{currentstroke}%
\pgfsetdash{}{0pt}%
\pgfpathmoveto{\pgfqpoint{1.590327in}{2.132663in}}%
\pgfpathlineto{\pgfqpoint{1.659946in}{2.912350in}}%
\pgfusepath{stroke}%
\end{pgfscope}%
\begin{pgfscope}%
\pgfpathrectangle{\pgfqpoint{0.100000in}{0.212622in}}{\pgfqpoint{3.696000in}{3.696000in}}%
\pgfusepath{clip}%
\pgfsetrectcap%
\pgfsetroundjoin%
\pgfsetlinewidth{1.505625pt}%
\definecolor{currentstroke}{rgb}{1.000000,0.000000,0.000000}%
\pgfsetstrokecolor{currentstroke}%
\pgfsetdash{}{0pt}%
\pgfpathmoveto{\pgfqpoint{1.590901in}{2.132635in}}%
\pgfpathlineto{\pgfqpoint{1.659946in}{2.912350in}}%
\pgfusepath{stroke}%
\end{pgfscope}%
\begin{pgfscope}%
\pgfpathrectangle{\pgfqpoint{0.100000in}{0.212622in}}{\pgfqpoint{3.696000in}{3.696000in}}%
\pgfusepath{clip}%
\pgfsetrectcap%
\pgfsetroundjoin%
\pgfsetlinewidth{1.505625pt}%
\definecolor{currentstroke}{rgb}{1.000000,0.000000,0.000000}%
\pgfsetstrokecolor{currentstroke}%
\pgfsetdash{}{0pt}%
\pgfpathmoveto{\pgfqpoint{1.591218in}{2.132623in}}%
\pgfpathlineto{\pgfqpoint{1.659946in}{2.912350in}}%
\pgfusepath{stroke}%
\end{pgfscope}%
\begin{pgfscope}%
\pgfpathrectangle{\pgfqpoint{0.100000in}{0.212622in}}{\pgfqpoint{3.696000in}{3.696000in}}%
\pgfusepath{clip}%
\pgfsetrectcap%
\pgfsetroundjoin%
\pgfsetlinewidth{1.505625pt}%
\definecolor{currentstroke}{rgb}{1.000000,0.000000,0.000000}%
\pgfsetstrokecolor{currentstroke}%
\pgfsetdash{}{0pt}%
\pgfpathmoveto{\pgfqpoint{1.591348in}{2.132622in}}%
\pgfpathlineto{\pgfqpoint{1.659946in}{2.912350in}}%
\pgfusepath{stroke}%
\end{pgfscope}%
\begin{pgfscope}%
\pgfpathrectangle{\pgfqpoint{0.100000in}{0.212622in}}{\pgfqpoint{3.696000in}{3.696000in}}%
\pgfusepath{clip}%
\pgfsetrectcap%
\pgfsetroundjoin%
\pgfsetlinewidth{1.505625pt}%
\definecolor{currentstroke}{rgb}{1.000000,0.000000,0.000000}%
\pgfsetstrokecolor{currentstroke}%
\pgfsetdash{}{0pt}%
\pgfpathmoveto{\pgfqpoint{1.592389in}{2.132609in}}%
\pgfpathlineto{\pgfqpoint{1.659946in}{2.912350in}}%
\pgfusepath{stroke}%
\end{pgfscope}%
\begin{pgfscope}%
\pgfpathrectangle{\pgfqpoint{0.100000in}{0.212622in}}{\pgfqpoint{3.696000in}{3.696000in}}%
\pgfusepath{clip}%
\pgfsetrectcap%
\pgfsetroundjoin%
\pgfsetlinewidth{1.505625pt}%
\definecolor{currentstroke}{rgb}{1.000000,0.000000,0.000000}%
\pgfsetstrokecolor{currentstroke}%
\pgfsetdash{}{0pt}%
\pgfpathmoveto{\pgfqpoint{1.592900in}{2.132599in}}%
\pgfpathlineto{\pgfqpoint{1.659946in}{2.912350in}}%
\pgfusepath{stroke}%
\end{pgfscope}%
\begin{pgfscope}%
\pgfpathrectangle{\pgfqpoint{0.100000in}{0.212622in}}{\pgfqpoint{3.696000in}{3.696000in}}%
\pgfusepath{clip}%
\pgfsetrectcap%
\pgfsetroundjoin%
\pgfsetlinewidth{1.505625pt}%
\definecolor{currentstroke}{rgb}{1.000000,0.000000,0.000000}%
\pgfsetstrokecolor{currentstroke}%
\pgfsetdash{}{0pt}%
\pgfpathmoveto{\pgfqpoint{1.593118in}{2.132594in}}%
\pgfpathlineto{\pgfqpoint{1.659946in}{2.912350in}}%
\pgfusepath{stroke}%
\end{pgfscope}%
\begin{pgfscope}%
\pgfpathrectangle{\pgfqpoint{0.100000in}{0.212622in}}{\pgfqpoint{3.696000in}{3.696000in}}%
\pgfusepath{clip}%
\pgfsetrectcap%
\pgfsetroundjoin%
\pgfsetlinewidth{1.505625pt}%
\definecolor{currentstroke}{rgb}{1.000000,0.000000,0.000000}%
\pgfsetstrokecolor{currentstroke}%
\pgfsetdash{}{0pt}%
\pgfpathmoveto{\pgfqpoint{1.593299in}{2.132587in}}%
\pgfpathlineto{\pgfqpoint{1.659946in}{2.912350in}}%
\pgfusepath{stroke}%
\end{pgfscope}%
\begin{pgfscope}%
\pgfpathrectangle{\pgfqpoint{0.100000in}{0.212622in}}{\pgfqpoint{3.696000in}{3.696000in}}%
\pgfusepath{clip}%
\pgfsetrectcap%
\pgfsetroundjoin%
\pgfsetlinewidth{1.505625pt}%
\definecolor{currentstroke}{rgb}{1.000000,0.000000,0.000000}%
\pgfsetstrokecolor{currentstroke}%
\pgfsetdash{}{0pt}%
\pgfpathmoveto{\pgfqpoint{1.593383in}{2.132582in}}%
\pgfpathlineto{\pgfqpoint{1.659946in}{2.912350in}}%
\pgfusepath{stroke}%
\end{pgfscope}%
\begin{pgfscope}%
\pgfpathrectangle{\pgfqpoint{0.100000in}{0.212622in}}{\pgfqpoint{3.696000in}{3.696000in}}%
\pgfusepath{clip}%
\pgfsetrectcap%
\pgfsetroundjoin%
\pgfsetlinewidth{1.505625pt}%
\definecolor{currentstroke}{rgb}{1.000000,0.000000,0.000000}%
\pgfsetstrokecolor{currentstroke}%
\pgfsetdash{}{0pt}%
\pgfpathmoveto{\pgfqpoint{1.594325in}{2.132498in}}%
\pgfpathlineto{\pgfqpoint{1.659946in}{2.912350in}}%
\pgfusepath{stroke}%
\end{pgfscope}%
\begin{pgfscope}%
\pgfpathrectangle{\pgfqpoint{0.100000in}{0.212622in}}{\pgfqpoint{3.696000in}{3.696000in}}%
\pgfusepath{clip}%
\pgfsetrectcap%
\pgfsetroundjoin%
\pgfsetlinewidth{1.505625pt}%
\definecolor{currentstroke}{rgb}{1.000000,0.000000,0.000000}%
\pgfsetstrokecolor{currentstroke}%
\pgfsetdash{}{0pt}%
\pgfpathmoveto{\pgfqpoint{1.597175in}{2.132094in}}%
\pgfpathlineto{\pgfqpoint{1.659946in}{2.912350in}}%
\pgfusepath{stroke}%
\end{pgfscope}%
\begin{pgfscope}%
\pgfpathrectangle{\pgfqpoint{0.100000in}{0.212622in}}{\pgfqpoint{3.696000in}{3.696000in}}%
\pgfusepath{clip}%
\pgfsetrectcap%
\pgfsetroundjoin%
\pgfsetlinewidth{1.505625pt}%
\definecolor{currentstroke}{rgb}{1.000000,0.000000,0.000000}%
\pgfsetstrokecolor{currentstroke}%
\pgfsetdash{}{0pt}%
\pgfpathmoveto{\pgfqpoint{1.598352in}{2.132038in}}%
\pgfpathlineto{\pgfqpoint{1.659946in}{2.912350in}}%
\pgfusepath{stroke}%
\end{pgfscope}%
\begin{pgfscope}%
\pgfpathrectangle{\pgfqpoint{0.100000in}{0.212622in}}{\pgfqpoint{3.696000in}{3.696000in}}%
\pgfusepath{clip}%
\pgfsetrectcap%
\pgfsetroundjoin%
\pgfsetlinewidth{1.505625pt}%
\definecolor{currentstroke}{rgb}{1.000000,0.000000,0.000000}%
\pgfsetstrokecolor{currentstroke}%
\pgfsetdash{}{0pt}%
\pgfpathmoveto{\pgfqpoint{1.600565in}{2.132078in}}%
\pgfpathlineto{\pgfqpoint{1.659946in}{2.912350in}}%
\pgfusepath{stroke}%
\end{pgfscope}%
\begin{pgfscope}%
\pgfpathrectangle{\pgfqpoint{0.100000in}{0.212622in}}{\pgfqpoint{3.696000in}{3.696000in}}%
\pgfusepath{clip}%
\pgfsetrectcap%
\pgfsetroundjoin%
\pgfsetlinewidth{1.505625pt}%
\definecolor{currentstroke}{rgb}{1.000000,0.000000,0.000000}%
\pgfsetstrokecolor{currentstroke}%
\pgfsetdash{}{0pt}%
\pgfpathmoveto{\pgfqpoint{1.604615in}{2.131727in}}%
\pgfpathlineto{\pgfqpoint{1.659946in}{2.912350in}}%
\pgfusepath{stroke}%
\end{pgfscope}%
\begin{pgfscope}%
\pgfpathrectangle{\pgfqpoint{0.100000in}{0.212622in}}{\pgfqpoint{3.696000in}{3.696000in}}%
\pgfusepath{clip}%
\pgfsetrectcap%
\pgfsetroundjoin%
\pgfsetlinewidth{1.505625pt}%
\definecolor{currentstroke}{rgb}{1.000000,0.000000,0.000000}%
\pgfsetstrokecolor{currentstroke}%
\pgfsetdash{}{0pt}%
\pgfpathmoveto{\pgfqpoint{1.607866in}{2.131570in}}%
\pgfpathlineto{\pgfqpoint{1.659946in}{2.912350in}}%
\pgfusepath{stroke}%
\end{pgfscope}%
\begin{pgfscope}%
\pgfpathrectangle{\pgfqpoint{0.100000in}{0.212622in}}{\pgfqpoint{3.696000in}{3.696000in}}%
\pgfusepath{clip}%
\pgfsetrectcap%
\pgfsetroundjoin%
\pgfsetlinewidth{1.505625pt}%
\definecolor{currentstroke}{rgb}{1.000000,0.000000,0.000000}%
\pgfsetstrokecolor{currentstroke}%
\pgfsetdash{}{0pt}%
\pgfpathmoveto{\pgfqpoint{1.609121in}{2.131559in}}%
\pgfpathlineto{\pgfqpoint{1.659946in}{2.912350in}}%
\pgfusepath{stroke}%
\end{pgfscope}%
\begin{pgfscope}%
\pgfpathrectangle{\pgfqpoint{0.100000in}{0.212622in}}{\pgfqpoint{3.696000in}{3.696000in}}%
\pgfusepath{clip}%
\pgfsetrectcap%
\pgfsetroundjoin%
\pgfsetlinewidth{1.505625pt}%
\definecolor{currentstroke}{rgb}{1.000000,0.000000,0.000000}%
\pgfsetstrokecolor{currentstroke}%
\pgfsetdash{}{0pt}%
\pgfpathmoveto{\pgfqpoint{1.614287in}{2.130985in}}%
\pgfpathlineto{\pgfqpoint{1.659946in}{2.912350in}}%
\pgfusepath{stroke}%
\end{pgfscope}%
\begin{pgfscope}%
\pgfpathrectangle{\pgfqpoint{0.100000in}{0.212622in}}{\pgfqpoint{3.696000in}{3.696000in}}%
\pgfusepath{clip}%
\pgfsetrectcap%
\pgfsetroundjoin%
\pgfsetlinewidth{1.505625pt}%
\definecolor{currentstroke}{rgb}{1.000000,0.000000,0.000000}%
\pgfsetstrokecolor{currentstroke}%
\pgfsetdash{}{0pt}%
\pgfpathmoveto{\pgfqpoint{1.618743in}{2.130731in}}%
\pgfpathlineto{\pgfqpoint{1.659946in}{2.912350in}}%
\pgfusepath{stroke}%
\end{pgfscope}%
\begin{pgfscope}%
\pgfpathrectangle{\pgfqpoint{0.100000in}{0.212622in}}{\pgfqpoint{3.696000in}{3.696000in}}%
\pgfusepath{clip}%
\pgfsetrectcap%
\pgfsetroundjoin%
\pgfsetlinewidth{1.505625pt}%
\definecolor{currentstroke}{rgb}{1.000000,0.000000,0.000000}%
\pgfsetstrokecolor{currentstroke}%
\pgfsetdash{}{0pt}%
\pgfpathmoveto{\pgfqpoint{1.620931in}{2.130905in}}%
\pgfpathlineto{\pgfqpoint{1.659946in}{2.912350in}}%
\pgfusepath{stroke}%
\end{pgfscope}%
\begin{pgfscope}%
\pgfpathrectangle{\pgfqpoint{0.100000in}{0.212622in}}{\pgfqpoint{3.696000in}{3.696000in}}%
\pgfusepath{clip}%
\pgfsetrectcap%
\pgfsetroundjoin%
\pgfsetlinewidth{1.505625pt}%
\definecolor{currentstroke}{rgb}{1.000000,0.000000,0.000000}%
\pgfsetstrokecolor{currentstroke}%
\pgfsetdash{}{0pt}%
\pgfpathmoveto{\pgfqpoint{1.624249in}{2.130602in}}%
\pgfpathlineto{\pgfqpoint{1.659946in}{2.912350in}}%
\pgfusepath{stroke}%
\end{pgfscope}%
\begin{pgfscope}%
\pgfpathrectangle{\pgfqpoint{0.100000in}{0.212622in}}{\pgfqpoint{3.696000in}{3.696000in}}%
\pgfusepath{clip}%
\pgfsetrectcap%
\pgfsetroundjoin%
\pgfsetlinewidth{1.505625pt}%
\definecolor{currentstroke}{rgb}{1.000000,0.000000,0.000000}%
\pgfsetstrokecolor{currentstroke}%
\pgfsetdash{}{0pt}%
\pgfpathmoveto{\pgfqpoint{1.627503in}{2.130519in}}%
\pgfpathlineto{\pgfqpoint{1.659946in}{2.912350in}}%
\pgfusepath{stroke}%
\end{pgfscope}%
\begin{pgfscope}%
\pgfpathrectangle{\pgfqpoint{0.100000in}{0.212622in}}{\pgfqpoint{3.696000in}{3.696000in}}%
\pgfusepath{clip}%
\pgfsetrectcap%
\pgfsetroundjoin%
\pgfsetlinewidth{1.505625pt}%
\definecolor{currentstroke}{rgb}{1.000000,0.000000,0.000000}%
\pgfsetstrokecolor{currentstroke}%
\pgfsetdash{}{0pt}%
\pgfpathmoveto{\pgfqpoint{1.630049in}{2.130414in}}%
\pgfpathlineto{\pgfqpoint{1.659946in}{2.912350in}}%
\pgfusepath{stroke}%
\end{pgfscope}%
\begin{pgfscope}%
\pgfpathrectangle{\pgfqpoint{0.100000in}{0.212622in}}{\pgfqpoint{3.696000in}{3.696000in}}%
\pgfusepath{clip}%
\pgfsetrectcap%
\pgfsetroundjoin%
\pgfsetlinewidth{1.505625pt}%
\definecolor{currentstroke}{rgb}{1.000000,0.000000,0.000000}%
\pgfsetstrokecolor{currentstroke}%
\pgfsetdash{}{0pt}%
\pgfpathmoveto{\pgfqpoint{1.632329in}{2.130145in}}%
\pgfpathlineto{\pgfqpoint{1.659946in}{2.912350in}}%
\pgfusepath{stroke}%
\end{pgfscope}%
\begin{pgfscope}%
\pgfpathrectangle{\pgfqpoint{0.100000in}{0.212622in}}{\pgfqpoint{3.696000in}{3.696000in}}%
\pgfusepath{clip}%
\pgfsetrectcap%
\pgfsetroundjoin%
\pgfsetlinewidth{1.505625pt}%
\definecolor{currentstroke}{rgb}{1.000000,0.000000,0.000000}%
\pgfsetstrokecolor{currentstroke}%
\pgfsetdash{}{0pt}%
\pgfpathmoveto{\pgfqpoint{1.634731in}{2.130069in}}%
\pgfpathlineto{\pgfqpoint{1.659946in}{2.912350in}}%
\pgfusepath{stroke}%
\end{pgfscope}%
\begin{pgfscope}%
\pgfpathrectangle{\pgfqpoint{0.100000in}{0.212622in}}{\pgfqpoint{3.696000in}{3.696000in}}%
\pgfusepath{clip}%
\pgfsetrectcap%
\pgfsetroundjoin%
\pgfsetlinewidth{1.505625pt}%
\definecolor{currentstroke}{rgb}{1.000000,0.000000,0.000000}%
\pgfsetstrokecolor{currentstroke}%
\pgfsetdash{}{0pt}%
\pgfpathmoveto{\pgfqpoint{1.637459in}{2.129840in}}%
\pgfpathlineto{\pgfqpoint{1.659946in}{2.912350in}}%
\pgfusepath{stroke}%
\end{pgfscope}%
\begin{pgfscope}%
\pgfpathrectangle{\pgfqpoint{0.100000in}{0.212622in}}{\pgfqpoint{3.696000in}{3.696000in}}%
\pgfusepath{clip}%
\pgfsetrectcap%
\pgfsetroundjoin%
\pgfsetlinewidth{1.505625pt}%
\definecolor{currentstroke}{rgb}{1.000000,0.000000,0.000000}%
\pgfsetstrokecolor{currentstroke}%
\pgfsetdash{}{0pt}%
\pgfpathmoveto{\pgfqpoint{1.639460in}{2.129652in}}%
\pgfpathlineto{\pgfqpoint{1.659946in}{2.912350in}}%
\pgfusepath{stroke}%
\end{pgfscope}%
\begin{pgfscope}%
\pgfpathrectangle{\pgfqpoint{0.100000in}{0.212622in}}{\pgfqpoint{3.696000in}{3.696000in}}%
\pgfusepath{clip}%
\pgfsetrectcap%
\pgfsetroundjoin%
\pgfsetlinewidth{1.505625pt}%
\definecolor{currentstroke}{rgb}{1.000000,0.000000,0.000000}%
\pgfsetstrokecolor{currentstroke}%
\pgfsetdash{}{0pt}%
\pgfpathmoveto{\pgfqpoint{1.641260in}{2.129576in}}%
\pgfpathlineto{\pgfqpoint{1.659946in}{2.912350in}}%
\pgfusepath{stroke}%
\end{pgfscope}%
\begin{pgfscope}%
\pgfpathrectangle{\pgfqpoint{0.100000in}{0.212622in}}{\pgfqpoint{3.696000in}{3.696000in}}%
\pgfusepath{clip}%
\pgfsetrectcap%
\pgfsetroundjoin%
\pgfsetlinewidth{1.505625pt}%
\definecolor{currentstroke}{rgb}{1.000000,0.000000,0.000000}%
\pgfsetstrokecolor{currentstroke}%
\pgfsetdash{}{0pt}%
\pgfpathmoveto{\pgfqpoint{1.642530in}{2.129553in}}%
\pgfpathlineto{\pgfqpoint{1.659946in}{2.912350in}}%
\pgfusepath{stroke}%
\end{pgfscope}%
\begin{pgfscope}%
\pgfpathrectangle{\pgfqpoint{0.100000in}{0.212622in}}{\pgfqpoint{3.696000in}{3.696000in}}%
\pgfusepath{clip}%
\pgfsetrectcap%
\pgfsetroundjoin%
\pgfsetlinewidth{1.505625pt}%
\definecolor{currentstroke}{rgb}{1.000000,0.000000,0.000000}%
\pgfsetstrokecolor{currentstroke}%
\pgfsetdash{}{0pt}%
\pgfpathmoveto{\pgfqpoint{1.643392in}{2.129471in}}%
\pgfpathlineto{\pgfqpoint{1.659946in}{2.912350in}}%
\pgfusepath{stroke}%
\end{pgfscope}%
\begin{pgfscope}%
\pgfpathrectangle{\pgfqpoint{0.100000in}{0.212622in}}{\pgfqpoint{3.696000in}{3.696000in}}%
\pgfusepath{clip}%
\pgfsetrectcap%
\pgfsetroundjoin%
\pgfsetlinewidth{1.505625pt}%
\definecolor{currentstroke}{rgb}{1.000000,0.000000,0.000000}%
\pgfsetstrokecolor{currentstroke}%
\pgfsetdash{}{0pt}%
\pgfpathmoveto{\pgfqpoint{1.644135in}{2.129351in}}%
\pgfpathlineto{\pgfqpoint{1.659946in}{2.912350in}}%
\pgfusepath{stroke}%
\end{pgfscope}%
\begin{pgfscope}%
\pgfpathrectangle{\pgfqpoint{0.100000in}{0.212622in}}{\pgfqpoint{3.696000in}{3.696000in}}%
\pgfusepath{clip}%
\pgfsetrectcap%
\pgfsetroundjoin%
\pgfsetlinewidth{1.505625pt}%
\definecolor{currentstroke}{rgb}{1.000000,0.000000,0.000000}%
\pgfsetstrokecolor{currentstroke}%
\pgfsetdash{}{0pt}%
\pgfpathmoveto{\pgfqpoint{1.646047in}{2.129204in}}%
\pgfpathlineto{\pgfqpoint{1.659946in}{2.912350in}}%
\pgfusepath{stroke}%
\end{pgfscope}%
\begin{pgfscope}%
\pgfpathrectangle{\pgfqpoint{0.100000in}{0.212622in}}{\pgfqpoint{3.696000in}{3.696000in}}%
\pgfusepath{clip}%
\pgfsetrectcap%
\pgfsetroundjoin%
\pgfsetlinewidth{1.505625pt}%
\definecolor{currentstroke}{rgb}{1.000000,0.000000,0.000000}%
\pgfsetstrokecolor{currentstroke}%
\pgfsetdash{}{0pt}%
\pgfpathmoveto{\pgfqpoint{1.647845in}{2.129082in}}%
\pgfpathlineto{\pgfqpoint{1.659946in}{2.912350in}}%
\pgfusepath{stroke}%
\end{pgfscope}%
\begin{pgfscope}%
\pgfpathrectangle{\pgfqpoint{0.100000in}{0.212622in}}{\pgfqpoint{3.696000in}{3.696000in}}%
\pgfusepath{clip}%
\pgfsetrectcap%
\pgfsetroundjoin%
\pgfsetlinewidth{1.505625pt}%
\definecolor{currentstroke}{rgb}{1.000000,0.000000,0.000000}%
\pgfsetstrokecolor{currentstroke}%
\pgfsetdash{}{0pt}%
\pgfpathmoveto{\pgfqpoint{1.648534in}{2.129111in}}%
\pgfpathlineto{\pgfqpoint{1.659946in}{2.912350in}}%
\pgfusepath{stroke}%
\end{pgfscope}%
\begin{pgfscope}%
\pgfpathrectangle{\pgfqpoint{0.100000in}{0.212622in}}{\pgfqpoint{3.696000in}{3.696000in}}%
\pgfusepath{clip}%
\pgfsetrectcap%
\pgfsetroundjoin%
\pgfsetlinewidth{1.505625pt}%
\definecolor{currentstroke}{rgb}{1.000000,0.000000,0.000000}%
\pgfsetstrokecolor{currentstroke}%
\pgfsetdash{}{0pt}%
\pgfpathmoveto{\pgfqpoint{1.649778in}{2.129098in}}%
\pgfpathlineto{\pgfqpoint{1.659946in}{2.912350in}}%
\pgfusepath{stroke}%
\end{pgfscope}%
\begin{pgfscope}%
\pgfpathrectangle{\pgfqpoint{0.100000in}{0.212622in}}{\pgfqpoint{3.696000in}{3.696000in}}%
\pgfusepath{clip}%
\pgfsetrectcap%
\pgfsetroundjoin%
\pgfsetlinewidth{1.505625pt}%
\definecolor{currentstroke}{rgb}{1.000000,0.000000,0.000000}%
\pgfsetstrokecolor{currentstroke}%
\pgfsetdash{}{0pt}%
\pgfpathmoveto{\pgfqpoint{1.650383in}{2.129076in}}%
\pgfpathlineto{\pgfqpoint{1.659946in}{2.912350in}}%
\pgfusepath{stroke}%
\end{pgfscope}%
\begin{pgfscope}%
\pgfpathrectangle{\pgfqpoint{0.100000in}{0.212622in}}{\pgfqpoint{3.696000in}{3.696000in}}%
\pgfusepath{clip}%
\pgfsetrectcap%
\pgfsetroundjoin%
\pgfsetlinewidth{1.505625pt}%
\definecolor{currentstroke}{rgb}{1.000000,0.000000,0.000000}%
\pgfsetstrokecolor{currentstroke}%
\pgfsetdash{}{0pt}%
\pgfpathmoveto{\pgfqpoint{1.651485in}{2.129022in}}%
\pgfpathlineto{\pgfqpoint{1.659946in}{2.912350in}}%
\pgfusepath{stroke}%
\end{pgfscope}%
\begin{pgfscope}%
\pgfpathrectangle{\pgfqpoint{0.100000in}{0.212622in}}{\pgfqpoint{3.696000in}{3.696000in}}%
\pgfusepath{clip}%
\pgfsetrectcap%
\pgfsetroundjoin%
\pgfsetlinewidth{1.505625pt}%
\definecolor{currentstroke}{rgb}{1.000000,0.000000,0.000000}%
\pgfsetstrokecolor{currentstroke}%
\pgfsetdash{}{0pt}%
\pgfpathmoveto{\pgfqpoint{1.654963in}{2.128679in}}%
\pgfpathlineto{\pgfqpoint{1.659946in}{2.912350in}}%
\pgfusepath{stroke}%
\end{pgfscope}%
\begin{pgfscope}%
\pgfpathrectangle{\pgfqpoint{0.100000in}{0.212622in}}{\pgfqpoint{3.696000in}{3.696000in}}%
\pgfusepath{clip}%
\pgfsetrectcap%
\pgfsetroundjoin%
\pgfsetlinewidth{1.505625pt}%
\definecolor{currentstroke}{rgb}{1.000000,0.000000,0.000000}%
\pgfsetstrokecolor{currentstroke}%
\pgfsetdash{}{0pt}%
\pgfpathmoveto{\pgfqpoint{1.656517in}{2.128602in}}%
\pgfpathlineto{\pgfqpoint{1.659946in}{2.912350in}}%
\pgfusepath{stroke}%
\end{pgfscope}%
\begin{pgfscope}%
\pgfpathrectangle{\pgfqpoint{0.100000in}{0.212622in}}{\pgfqpoint{3.696000in}{3.696000in}}%
\pgfusepath{clip}%
\pgfsetrectcap%
\pgfsetroundjoin%
\pgfsetlinewidth{1.505625pt}%
\definecolor{currentstroke}{rgb}{1.000000,0.000000,0.000000}%
\pgfsetstrokecolor{currentstroke}%
\pgfsetdash{}{0pt}%
\pgfpathmoveto{\pgfqpoint{1.659283in}{2.128625in}}%
\pgfpathlineto{\pgfqpoint{1.659946in}{2.912350in}}%
\pgfusepath{stroke}%
\end{pgfscope}%
\begin{pgfscope}%
\pgfpathrectangle{\pgfqpoint{0.100000in}{0.212622in}}{\pgfqpoint{3.696000in}{3.696000in}}%
\pgfusepath{clip}%
\pgfsetrectcap%
\pgfsetroundjoin%
\pgfsetlinewidth{1.505625pt}%
\definecolor{currentstroke}{rgb}{1.000000,0.000000,0.000000}%
\pgfsetstrokecolor{currentstroke}%
\pgfsetdash{}{0pt}%
\pgfpathmoveto{\pgfqpoint{1.662959in}{2.128452in}}%
\pgfpathlineto{\pgfqpoint{1.659946in}{2.912350in}}%
\pgfusepath{stroke}%
\end{pgfscope}%
\begin{pgfscope}%
\pgfpathrectangle{\pgfqpoint{0.100000in}{0.212622in}}{\pgfqpoint{3.696000in}{3.696000in}}%
\pgfusepath{clip}%
\pgfsetrectcap%
\pgfsetroundjoin%
\pgfsetlinewidth{1.505625pt}%
\definecolor{currentstroke}{rgb}{1.000000,0.000000,0.000000}%
\pgfsetstrokecolor{currentstroke}%
\pgfsetdash{}{0pt}%
\pgfpathmoveto{\pgfqpoint{1.666571in}{2.128163in}}%
\pgfpathlineto{\pgfqpoint{1.659946in}{2.912350in}}%
\pgfusepath{stroke}%
\end{pgfscope}%
\begin{pgfscope}%
\pgfpathrectangle{\pgfqpoint{0.100000in}{0.212622in}}{\pgfqpoint{3.696000in}{3.696000in}}%
\pgfusepath{clip}%
\pgfsetrectcap%
\pgfsetroundjoin%
\pgfsetlinewidth{1.505625pt}%
\definecolor{currentstroke}{rgb}{1.000000,0.000000,0.000000}%
\pgfsetstrokecolor{currentstroke}%
\pgfsetdash{}{0pt}%
\pgfpathmoveto{\pgfqpoint{1.662056in}{2.126299in}}%
\pgfpathlineto{\pgfqpoint{1.659946in}{2.912350in}}%
\pgfusepath{stroke}%
\end{pgfscope}%
\begin{pgfscope}%
\pgfpathrectangle{\pgfqpoint{0.100000in}{0.212622in}}{\pgfqpoint{3.696000in}{3.696000in}}%
\pgfusepath{clip}%
\pgfsetrectcap%
\pgfsetroundjoin%
\pgfsetlinewidth{1.505625pt}%
\definecolor{currentstroke}{rgb}{1.000000,0.000000,0.000000}%
\pgfsetstrokecolor{currentstroke}%
\pgfsetdash{}{0pt}%
\pgfpathmoveto{\pgfqpoint{1.664347in}{2.126123in}}%
\pgfpathlineto{\pgfqpoint{1.659946in}{2.912350in}}%
\pgfusepath{stroke}%
\end{pgfscope}%
\begin{pgfscope}%
\pgfpathrectangle{\pgfqpoint{0.100000in}{0.212622in}}{\pgfqpoint{3.696000in}{3.696000in}}%
\pgfusepath{clip}%
\pgfsetrectcap%
\pgfsetroundjoin%
\pgfsetlinewidth{1.505625pt}%
\definecolor{currentstroke}{rgb}{1.000000,0.000000,0.000000}%
\pgfsetstrokecolor{currentstroke}%
\pgfsetdash{}{0pt}%
\pgfpathmoveto{\pgfqpoint{1.666413in}{2.126128in}}%
\pgfpathlineto{\pgfqpoint{1.659946in}{2.912350in}}%
\pgfusepath{stroke}%
\end{pgfscope}%
\begin{pgfscope}%
\pgfpathrectangle{\pgfqpoint{0.100000in}{0.212622in}}{\pgfqpoint{3.696000in}{3.696000in}}%
\pgfusepath{clip}%
\pgfsetrectcap%
\pgfsetroundjoin%
\pgfsetlinewidth{1.505625pt}%
\definecolor{currentstroke}{rgb}{1.000000,0.000000,0.000000}%
\pgfsetstrokecolor{currentstroke}%
\pgfsetdash{}{0pt}%
\pgfpathmoveto{\pgfqpoint{1.667813in}{2.126109in}}%
\pgfpathlineto{\pgfqpoint{1.659946in}{2.912350in}}%
\pgfusepath{stroke}%
\end{pgfscope}%
\begin{pgfscope}%
\pgfpathrectangle{\pgfqpoint{0.100000in}{0.212622in}}{\pgfqpoint{3.696000in}{3.696000in}}%
\pgfusepath{clip}%
\pgfsetrectcap%
\pgfsetroundjoin%
\pgfsetlinewidth{1.505625pt}%
\definecolor{currentstroke}{rgb}{1.000000,0.000000,0.000000}%
\pgfsetstrokecolor{currentstroke}%
\pgfsetdash{}{0pt}%
\pgfpathmoveto{\pgfqpoint{1.668672in}{2.126071in}}%
\pgfpathlineto{\pgfqpoint{1.659946in}{2.912350in}}%
\pgfusepath{stroke}%
\end{pgfscope}%
\begin{pgfscope}%
\pgfpathrectangle{\pgfqpoint{0.100000in}{0.212622in}}{\pgfqpoint{3.696000in}{3.696000in}}%
\pgfusepath{clip}%
\pgfsetrectcap%
\pgfsetroundjoin%
\pgfsetlinewidth{1.505625pt}%
\definecolor{currentstroke}{rgb}{1.000000,0.000000,0.000000}%
\pgfsetstrokecolor{currentstroke}%
\pgfsetdash{}{0pt}%
\pgfpathmoveto{\pgfqpoint{1.669004in}{2.126061in}}%
\pgfpathlineto{\pgfqpoint{1.659946in}{2.912350in}}%
\pgfusepath{stroke}%
\end{pgfscope}%
\begin{pgfscope}%
\pgfpathrectangle{\pgfqpoint{0.100000in}{0.212622in}}{\pgfqpoint{3.696000in}{3.696000in}}%
\pgfusepath{clip}%
\pgfsetrectcap%
\pgfsetroundjoin%
\pgfsetlinewidth{1.505625pt}%
\definecolor{currentstroke}{rgb}{1.000000,0.000000,0.000000}%
\pgfsetstrokecolor{currentstroke}%
\pgfsetdash{}{0pt}%
\pgfpathmoveto{\pgfqpoint{1.670023in}{2.126059in}}%
\pgfpathlineto{\pgfqpoint{1.659946in}{2.912350in}}%
\pgfusepath{stroke}%
\end{pgfscope}%
\begin{pgfscope}%
\pgfpathrectangle{\pgfqpoint{0.100000in}{0.212622in}}{\pgfqpoint{3.696000in}{3.696000in}}%
\pgfusepath{clip}%
\pgfsetrectcap%
\pgfsetroundjoin%
\pgfsetlinewidth{1.505625pt}%
\definecolor{currentstroke}{rgb}{1.000000,0.000000,0.000000}%
\pgfsetstrokecolor{currentstroke}%
\pgfsetdash{}{0pt}%
\pgfpathmoveto{\pgfqpoint{1.670598in}{2.126050in}}%
\pgfpathlineto{\pgfqpoint{1.659946in}{2.912350in}}%
\pgfusepath{stroke}%
\end{pgfscope}%
\begin{pgfscope}%
\pgfpathrectangle{\pgfqpoint{0.100000in}{0.212622in}}{\pgfqpoint{3.696000in}{3.696000in}}%
\pgfusepath{clip}%
\pgfsetrectcap%
\pgfsetroundjoin%
\pgfsetlinewidth{1.505625pt}%
\definecolor{currentstroke}{rgb}{1.000000,0.000000,0.000000}%
\pgfsetstrokecolor{currentstroke}%
\pgfsetdash{}{0pt}%
\pgfpathmoveto{\pgfqpoint{1.670823in}{2.126043in}}%
\pgfpathlineto{\pgfqpoint{1.659946in}{2.912350in}}%
\pgfusepath{stroke}%
\end{pgfscope}%
\begin{pgfscope}%
\pgfpathrectangle{\pgfqpoint{0.100000in}{0.212622in}}{\pgfqpoint{3.696000in}{3.696000in}}%
\pgfusepath{clip}%
\pgfsetrectcap%
\pgfsetroundjoin%
\pgfsetlinewidth{1.505625pt}%
\definecolor{currentstroke}{rgb}{1.000000,0.000000,0.000000}%
\pgfsetstrokecolor{currentstroke}%
\pgfsetdash{}{0pt}%
\pgfpathmoveto{\pgfqpoint{1.671644in}{2.125976in}}%
\pgfpathlineto{\pgfqpoint{1.659946in}{2.912350in}}%
\pgfusepath{stroke}%
\end{pgfscope}%
\begin{pgfscope}%
\pgfpathrectangle{\pgfqpoint{0.100000in}{0.212622in}}{\pgfqpoint{3.696000in}{3.696000in}}%
\pgfusepath{clip}%
\pgfsetrectcap%
\pgfsetroundjoin%
\pgfsetlinewidth{1.505625pt}%
\definecolor{currentstroke}{rgb}{1.000000,0.000000,0.000000}%
\pgfsetstrokecolor{currentstroke}%
\pgfsetdash{}{0pt}%
\pgfpathmoveto{\pgfqpoint{1.672441in}{2.125972in}}%
\pgfpathlineto{\pgfqpoint{1.659946in}{2.912350in}}%
\pgfusepath{stroke}%
\end{pgfscope}%
\begin{pgfscope}%
\pgfpathrectangle{\pgfqpoint{0.100000in}{0.212622in}}{\pgfqpoint{3.696000in}{3.696000in}}%
\pgfusepath{clip}%
\pgfsetrectcap%
\pgfsetroundjoin%
\pgfsetlinewidth{1.505625pt}%
\definecolor{currentstroke}{rgb}{1.000000,0.000000,0.000000}%
\pgfsetstrokecolor{currentstroke}%
\pgfsetdash{}{0pt}%
\pgfpathmoveto{\pgfqpoint{1.673118in}{2.125940in}}%
\pgfpathlineto{\pgfqpoint{1.659946in}{2.912350in}}%
\pgfusepath{stroke}%
\end{pgfscope}%
\begin{pgfscope}%
\pgfpathrectangle{\pgfqpoint{0.100000in}{0.212622in}}{\pgfqpoint{3.696000in}{3.696000in}}%
\pgfusepath{clip}%
\pgfsetrectcap%
\pgfsetroundjoin%
\pgfsetlinewidth{1.505625pt}%
\definecolor{currentstroke}{rgb}{1.000000,0.000000,0.000000}%
\pgfsetstrokecolor{currentstroke}%
\pgfsetdash{}{0pt}%
\pgfpathmoveto{\pgfqpoint{1.673748in}{2.125902in}}%
\pgfpathlineto{\pgfqpoint{1.659946in}{2.912350in}}%
\pgfusepath{stroke}%
\end{pgfscope}%
\begin{pgfscope}%
\pgfpathrectangle{\pgfqpoint{0.100000in}{0.212622in}}{\pgfqpoint{3.696000in}{3.696000in}}%
\pgfusepath{clip}%
\pgfsetrectcap%
\pgfsetroundjoin%
\pgfsetlinewidth{1.505625pt}%
\definecolor{currentstroke}{rgb}{1.000000,0.000000,0.000000}%
\pgfsetstrokecolor{currentstroke}%
\pgfsetdash{}{0pt}%
\pgfpathmoveto{\pgfqpoint{1.674490in}{2.125863in}}%
\pgfpathlineto{\pgfqpoint{1.659946in}{2.912350in}}%
\pgfusepath{stroke}%
\end{pgfscope}%
\begin{pgfscope}%
\pgfpathrectangle{\pgfqpoint{0.100000in}{0.212622in}}{\pgfqpoint{3.696000in}{3.696000in}}%
\pgfusepath{clip}%
\pgfsetrectcap%
\pgfsetroundjoin%
\pgfsetlinewidth{1.505625pt}%
\definecolor{currentstroke}{rgb}{1.000000,0.000000,0.000000}%
\pgfsetstrokecolor{currentstroke}%
\pgfsetdash{}{0pt}%
\pgfpathmoveto{\pgfqpoint{1.675540in}{2.125844in}}%
\pgfpathlineto{\pgfqpoint{1.659946in}{2.912350in}}%
\pgfusepath{stroke}%
\end{pgfscope}%
\begin{pgfscope}%
\pgfpathrectangle{\pgfqpoint{0.100000in}{0.212622in}}{\pgfqpoint{3.696000in}{3.696000in}}%
\pgfusepath{clip}%
\pgfsetrectcap%
\pgfsetroundjoin%
\pgfsetlinewidth{1.505625pt}%
\definecolor{currentstroke}{rgb}{1.000000,0.000000,0.000000}%
\pgfsetstrokecolor{currentstroke}%
\pgfsetdash{}{0pt}%
\pgfpathmoveto{\pgfqpoint{1.676028in}{2.125845in}}%
\pgfpathlineto{\pgfqpoint{1.659946in}{2.912350in}}%
\pgfusepath{stroke}%
\end{pgfscope}%
\begin{pgfscope}%
\pgfpathrectangle{\pgfqpoint{0.100000in}{0.212622in}}{\pgfqpoint{3.696000in}{3.696000in}}%
\pgfusepath{clip}%
\pgfsetrectcap%
\pgfsetroundjoin%
\pgfsetlinewidth{1.505625pt}%
\definecolor{currentstroke}{rgb}{1.000000,0.000000,0.000000}%
\pgfsetstrokecolor{currentstroke}%
\pgfsetdash{}{0pt}%
\pgfpathmoveto{\pgfqpoint{1.676306in}{2.125827in}}%
\pgfpathlineto{\pgfqpoint{1.659946in}{2.912350in}}%
\pgfusepath{stroke}%
\end{pgfscope}%
\begin{pgfscope}%
\pgfpathrectangle{\pgfqpoint{0.100000in}{0.212622in}}{\pgfqpoint{3.696000in}{3.696000in}}%
\pgfusepath{clip}%
\pgfsetrectcap%
\pgfsetroundjoin%
\pgfsetlinewidth{1.505625pt}%
\definecolor{currentstroke}{rgb}{1.000000,0.000000,0.000000}%
\pgfsetstrokecolor{currentstroke}%
\pgfsetdash{}{0pt}%
\pgfpathmoveto{\pgfqpoint{1.676432in}{2.125825in}}%
\pgfpathlineto{\pgfqpoint{1.659946in}{2.912350in}}%
\pgfusepath{stroke}%
\end{pgfscope}%
\begin{pgfscope}%
\pgfpathrectangle{\pgfqpoint{0.100000in}{0.212622in}}{\pgfqpoint{3.696000in}{3.696000in}}%
\pgfusepath{clip}%
\pgfsetrectcap%
\pgfsetroundjoin%
\pgfsetlinewidth{1.505625pt}%
\definecolor{currentstroke}{rgb}{1.000000,0.000000,0.000000}%
\pgfsetstrokecolor{currentstroke}%
\pgfsetdash{}{0pt}%
\pgfpathmoveto{\pgfqpoint{1.676505in}{2.125827in}}%
\pgfpathlineto{\pgfqpoint{1.659946in}{2.912350in}}%
\pgfusepath{stroke}%
\end{pgfscope}%
\begin{pgfscope}%
\pgfpathrectangle{\pgfqpoint{0.100000in}{0.212622in}}{\pgfqpoint{3.696000in}{3.696000in}}%
\pgfusepath{clip}%
\pgfsetrectcap%
\pgfsetroundjoin%
\pgfsetlinewidth{1.505625pt}%
\definecolor{currentstroke}{rgb}{1.000000,0.000000,0.000000}%
\pgfsetstrokecolor{currentstroke}%
\pgfsetdash{}{0pt}%
\pgfpathmoveto{\pgfqpoint{1.676548in}{2.125825in}}%
\pgfpathlineto{\pgfqpoint{1.659946in}{2.912350in}}%
\pgfusepath{stroke}%
\end{pgfscope}%
\begin{pgfscope}%
\pgfpathrectangle{\pgfqpoint{0.100000in}{0.212622in}}{\pgfqpoint{3.696000in}{3.696000in}}%
\pgfusepath{clip}%
\pgfsetrectcap%
\pgfsetroundjoin%
\pgfsetlinewidth{1.505625pt}%
\definecolor{currentstroke}{rgb}{1.000000,0.000000,0.000000}%
\pgfsetstrokecolor{currentstroke}%
\pgfsetdash{}{0pt}%
\pgfpathmoveto{\pgfqpoint{1.676575in}{2.125824in}}%
\pgfpathlineto{\pgfqpoint{1.659946in}{2.912350in}}%
\pgfusepath{stroke}%
\end{pgfscope}%
\begin{pgfscope}%
\pgfpathrectangle{\pgfqpoint{0.100000in}{0.212622in}}{\pgfqpoint{3.696000in}{3.696000in}}%
\pgfusepath{clip}%
\pgfsetrectcap%
\pgfsetroundjoin%
\pgfsetlinewidth{1.505625pt}%
\definecolor{currentstroke}{rgb}{1.000000,0.000000,0.000000}%
\pgfsetstrokecolor{currentstroke}%
\pgfsetdash{}{0pt}%
\pgfpathmoveto{\pgfqpoint{1.676587in}{2.125824in}}%
\pgfpathlineto{\pgfqpoint{1.659946in}{2.912350in}}%
\pgfusepath{stroke}%
\end{pgfscope}%
\begin{pgfscope}%
\pgfpathrectangle{\pgfqpoint{0.100000in}{0.212622in}}{\pgfqpoint{3.696000in}{3.696000in}}%
\pgfusepath{clip}%
\pgfsetrectcap%
\pgfsetroundjoin%
\pgfsetlinewidth{1.505625pt}%
\definecolor{currentstroke}{rgb}{1.000000,0.000000,0.000000}%
\pgfsetstrokecolor{currentstroke}%
\pgfsetdash{}{0pt}%
\pgfpathmoveto{\pgfqpoint{1.676595in}{2.125824in}}%
\pgfpathlineto{\pgfqpoint{1.659946in}{2.912350in}}%
\pgfusepath{stroke}%
\end{pgfscope}%
\begin{pgfscope}%
\pgfpathrectangle{\pgfqpoint{0.100000in}{0.212622in}}{\pgfqpoint{3.696000in}{3.696000in}}%
\pgfusepath{clip}%
\pgfsetrectcap%
\pgfsetroundjoin%
\pgfsetlinewidth{1.505625pt}%
\definecolor{currentstroke}{rgb}{1.000000,0.000000,0.000000}%
\pgfsetstrokecolor{currentstroke}%
\pgfsetdash{}{0pt}%
\pgfpathmoveto{\pgfqpoint{1.676599in}{2.125824in}}%
\pgfpathlineto{\pgfqpoint{1.659946in}{2.912350in}}%
\pgfusepath{stroke}%
\end{pgfscope}%
\begin{pgfscope}%
\pgfpathrectangle{\pgfqpoint{0.100000in}{0.212622in}}{\pgfqpoint{3.696000in}{3.696000in}}%
\pgfusepath{clip}%
\pgfsetrectcap%
\pgfsetroundjoin%
\pgfsetlinewidth{1.505625pt}%
\definecolor{currentstroke}{rgb}{1.000000,0.000000,0.000000}%
\pgfsetstrokecolor{currentstroke}%
\pgfsetdash{}{0pt}%
\pgfpathmoveto{\pgfqpoint{1.676602in}{2.125824in}}%
\pgfpathlineto{\pgfqpoint{1.659946in}{2.912350in}}%
\pgfusepath{stroke}%
\end{pgfscope}%
\begin{pgfscope}%
\pgfpathrectangle{\pgfqpoint{0.100000in}{0.212622in}}{\pgfqpoint{3.696000in}{3.696000in}}%
\pgfusepath{clip}%
\pgfsetrectcap%
\pgfsetroundjoin%
\pgfsetlinewidth{1.505625pt}%
\definecolor{currentstroke}{rgb}{1.000000,0.000000,0.000000}%
\pgfsetstrokecolor{currentstroke}%
\pgfsetdash{}{0pt}%
\pgfpathmoveto{\pgfqpoint{1.677090in}{2.125799in}}%
\pgfpathlineto{\pgfqpoint{1.659946in}{2.912350in}}%
\pgfusepath{stroke}%
\end{pgfscope}%
\begin{pgfscope}%
\pgfpathrectangle{\pgfqpoint{0.100000in}{0.212622in}}{\pgfqpoint{3.696000in}{3.696000in}}%
\pgfusepath{clip}%
\pgfsetrectcap%
\pgfsetroundjoin%
\pgfsetlinewidth{1.505625pt}%
\definecolor{currentstroke}{rgb}{1.000000,0.000000,0.000000}%
\pgfsetstrokecolor{currentstroke}%
\pgfsetdash{}{0pt}%
\pgfpathmoveto{\pgfqpoint{1.677363in}{2.125780in}}%
\pgfpathlineto{\pgfqpoint{1.659946in}{2.912350in}}%
\pgfusepath{stroke}%
\end{pgfscope}%
\begin{pgfscope}%
\pgfpathrectangle{\pgfqpoint{0.100000in}{0.212622in}}{\pgfqpoint{3.696000in}{3.696000in}}%
\pgfusepath{clip}%
\pgfsetrectcap%
\pgfsetroundjoin%
\pgfsetlinewidth{1.505625pt}%
\definecolor{currentstroke}{rgb}{1.000000,0.000000,0.000000}%
\pgfsetstrokecolor{currentstroke}%
\pgfsetdash{}{0pt}%
\pgfpathmoveto{\pgfqpoint{1.677533in}{2.125765in}}%
\pgfpathlineto{\pgfqpoint{1.659946in}{2.912350in}}%
\pgfusepath{stroke}%
\end{pgfscope}%
\begin{pgfscope}%
\pgfpathrectangle{\pgfqpoint{0.100000in}{0.212622in}}{\pgfqpoint{3.696000in}{3.696000in}}%
\pgfusepath{clip}%
\pgfsetrectcap%
\pgfsetroundjoin%
\pgfsetlinewidth{1.505625pt}%
\definecolor{currentstroke}{rgb}{1.000000,0.000000,0.000000}%
\pgfsetstrokecolor{currentstroke}%
\pgfsetdash{}{0pt}%
\pgfpathmoveto{\pgfqpoint{1.677634in}{2.125756in}}%
\pgfpathlineto{\pgfqpoint{1.659946in}{2.912350in}}%
\pgfusepath{stroke}%
\end{pgfscope}%
\begin{pgfscope}%
\pgfpathrectangle{\pgfqpoint{0.100000in}{0.212622in}}{\pgfqpoint{3.696000in}{3.696000in}}%
\pgfusepath{clip}%
\pgfsetrectcap%
\pgfsetroundjoin%
\pgfsetlinewidth{1.505625pt}%
\definecolor{currentstroke}{rgb}{1.000000,0.000000,0.000000}%
\pgfsetstrokecolor{currentstroke}%
\pgfsetdash{}{0pt}%
\pgfpathmoveto{\pgfqpoint{1.678267in}{2.125687in}}%
\pgfpathlineto{\pgfqpoint{1.659946in}{2.912350in}}%
\pgfusepath{stroke}%
\end{pgfscope}%
\begin{pgfscope}%
\pgfpathrectangle{\pgfqpoint{0.100000in}{0.212622in}}{\pgfqpoint{3.696000in}{3.696000in}}%
\pgfusepath{clip}%
\pgfsetrectcap%
\pgfsetroundjoin%
\pgfsetlinewidth{1.505625pt}%
\definecolor{currentstroke}{rgb}{1.000000,0.000000,0.000000}%
\pgfsetstrokecolor{currentstroke}%
\pgfsetdash{}{0pt}%
\pgfpathmoveto{\pgfqpoint{1.679706in}{2.125464in}}%
\pgfpathlineto{\pgfqpoint{1.659946in}{2.912350in}}%
\pgfusepath{stroke}%
\end{pgfscope}%
\begin{pgfscope}%
\pgfpathrectangle{\pgfqpoint{0.100000in}{0.212622in}}{\pgfqpoint{3.696000in}{3.696000in}}%
\pgfusepath{clip}%
\pgfsetrectcap%
\pgfsetroundjoin%
\pgfsetlinewidth{1.505625pt}%
\definecolor{currentstroke}{rgb}{1.000000,0.000000,0.000000}%
\pgfsetstrokecolor{currentstroke}%
\pgfsetdash{}{0pt}%
\pgfpathmoveto{\pgfqpoint{1.680525in}{2.125379in}}%
\pgfpathlineto{\pgfqpoint{1.659946in}{2.912350in}}%
\pgfusepath{stroke}%
\end{pgfscope}%
\begin{pgfscope}%
\pgfpathrectangle{\pgfqpoint{0.100000in}{0.212622in}}{\pgfqpoint{3.696000in}{3.696000in}}%
\pgfusepath{clip}%
\pgfsetrectcap%
\pgfsetroundjoin%
\pgfsetlinewidth{1.505625pt}%
\definecolor{currentstroke}{rgb}{1.000000,0.000000,0.000000}%
\pgfsetstrokecolor{currentstroke}%
\pgfsetdash{}{0pt}%
\pgfpathmoveto{\pgfqpoint{1.680974in}{2.125314in}}%
\pgfpathlineto{\pgfqpoint{1.659946in}{2.912350in}}%
\pgfusepath{stroke}%
\end{pgfscope}%
\begin{pgfscope}%
\pgfpathrectangle{\pgfqpoint{0.100000in}{0.212622in}}{\pgfqpoint{3.696000in}{3.696000in}}%
\pgfusepath{clip}%
\pgfsetrectcap%
\pgfsetroundjoin%
\pgfsetlinewidth{1.505625pt}%
\definecolor{currentstroke}{rgb}{1.000000,0.000000,0.000000}%
\pgfsetstrokecolor{currentstroke}%
\pgfsetdash{}{0pt}%
\pgfpathmoveto{\pgfqpoint{1.681790in}{2.125207in}}%
\pgfpathlineto{\pgfqpoint{1.659946in}{2.912350in}}%
\pgfusepath{stroke}%
\end{pgfscope}%
\begin{pgfscope}%
\pgfpathrectangle{\pgfqpoint{0.100000in}{0.212622in}}{\pgfqpoint{3.696000in}{3.696000in}}%
\pgfusepath{clip}%
\pgfsetrectcap%
\pgfsetroundjoin%
\pgfsetlinewidth{1.505625pt}%
\definecolor{currentstroke}{rgb}{1.000000,0.000000,0.000000}%
\pgfsetstrokecolor{currentstroke}%
\pgfsetdash{}{0pt}%
\pgfpathmoveto{\pgfqpoint{1.682906in}{2.125003in}}%
\pgfpathlineto{\pgfqpoint{1.659946in}{2.912350in}}%
\pgfusepath{stroke}%
\end{pgfscope}%
\begin{pgfscope}%
\pgfpathrectangle{\pgfqpoint{0.100000in}{0.212622in}}{\pgfqpoint{3.696000in}{3.696000in}}%
\pgfusepath{clip}%
\pgfsetrectcap%
\pgfsetroundjoin%
\pgfsetlinewidth{1.505625pt}%
\definecolor{currentstroke}{rgb}{1.000000,0.000000,0.000000}%
\pgfsetstrokecolor{currentstroke}%
\pgfsetdash{}{0pt}%
\pgfpathmoveto{\pgfqpoint{1.683482in}{2.124882in}}%
\pgfpathlineto{\pgfqpoint{1.659946in}{2.912350in}}%
\pgfusepath{stroke}%
\end{pgfscope}%
\begin{pgfscope}%
\pgfpathrectangle{\pgfqpoint{0.100000in}{0.212622in}}{\pgfqpoint{3.696000in}{3.696000in}}%
\pgfusepath{clip}%
\pgfsetrectcap%
\pgfsetroundjoin%
\pgfsetlinewidth{1.505625pt}%
\definecolor{currentstroke}{rgb}{1.000000,0.000000,0.000000}%
\pgfsetstrokecolor{currentstroke}%
\pgfsetdash{}{0pt}%
\pgfpathmoveto{\pgfqpoint{1.683856in}{2.124769in}}%
\pgfpathlineto{\pgfqpoint{1.659946in}{2.912350in}}%
\pgfusepath{stroke}%
\end{pgfscope}%
\begin{pgfscope}%
\pgfpathrectangle{\pgfqpoint{0.100000in}{0.212622in}}{\pgfqpoint{3.696000in}{3.696000in}}%
\pgfusepath{clip}%
\pgfsetrectcap%
\pgfsetroundjoin%
\pgfsetlinewidth{1.505625pt}%
\definecolor{currentstroke}{rgb}{1.000000,0.000000,0.000000}%
\pgfsetstrokecolor{currentstroke}%
\pgfsetdash{}{0pt}%
\pgfpathmoveto{\pgfqpoint{1.684036in}{2.124738in}}%
\pgfpathlineto{\pgfqpoint{1.659946in}{2.912350in}}%
\pgfusepath{stroke}%
\end{pgfscope}%
\begin{pgfscope}%
\pgfpathrectangle{\pgfqpoint{0.100000in}{0.212622in}}{\pgfqpoint{3.696000in}{3.696000in}}%
\pgfusepath{clip}%
\pgfsetrectcap%
\pgfsetroundjoin%
\pgfsetlinewidth{1.505625pt}%
\definecolor{currentstroke}{rgb}{1.000000,0.000000,0.000000}%
\pgfsetstrokecolor{currentstroke}%
\pgfsetdash{}{0pt}%
\pgfpathmoveto{\pgfqpoint{1.684147in}{2.124711in}}%
\pgfpathlineto{\pgfqpoint{1.659946in}{2.912350in}}%
\pgfusepath{stroke}%
\end{pgfscope}%
\begin{pgfscope}%
\pgfpathrectangle{\pgfqpoint{0.100000in}{0.212622in}}{\pgfqpoint{3.696000in}{3.696000in}}%
\pgfusepath{clip}%
\pgfsetrectcap%
\pgfsetroundjoin%
\pgfsetlinewidth{1.505625pt}%
\definecolor{currentstroke}{rgb}{1.000000,0.000000,0.000000}%
\pgfsetstrokecolor{currentstroke}%
\pgfsetdash{}{0pt}%
\pgfpathmoveto{\pgfqpoint{1.684201in}{2.124705in}}%
\pgfpathlineto{\pgfqpoint{1.659946in}{2.912350in}}%
\pgfusepath{stroke}%
\end{pgfscope}%
\begin{pgfscope}%
\pgfpathrectangle{\pgfqpoint{0.100000in}{0.212622in}}{\pgfqpoint{3.696000in}{3.696000in}}%
\pgfusepath{clip}%
\pgfsetrectcap%
\pgfsetroundjoin%
\pgfsetlinewidth{1.505625pt}%
\definecolor{currentstroke}{rgb}{1.000000,0.000000,0.000000}%
\pgfsetstrokecolor{currentstroke}%
\pgfsetdash{}{0pt}%
\pgfpathmoveto{\pgfqpoint{1.685205in}{2.124578in}}%
\pgfpathlineto{\pgfqpoint{1.659946in}{2.912350in}}%
\pgfusepath{stroke}%
\end{pgfscope}%
\begin{pgfscope}%
\pgfpathrectangle{\pgfqpoint{0.100000in}{0.212622in}}{\pgfqpoint{3.696000in}{3.696000in}}%
\pgfusepath{clip}%
\pgfsetrectcap%
\pgfsetroundjoin%
\pgfsetlinewidth{1.505625pt}%
\definecolor{currentstroke}{rgb}{1.000000,0.000000,0.000000}%
\pgfsetstrokecolor{currentstroke}%
\pgfsetdash{}{0pt}%
\pgfpathmoveto{\pgfqpoint{1.685789in}{2.124482in}}%
\pgfpathlineto{\pgfqpoint{1.659946in}{2.912350in}}%
\pgfusepath{stroke}%
\end{pgfscope}%
\begin{pgfscope}%
\pgfpathrectangle{\pgfqpoint{0.100000in}{0.212622in}}{\pgfqpoint{3.696000in}{3.696000in}}%
\pgfusepath{clip}%
\pgfsetrectcap%
\pgfsetroundjoin%
\pgfsetlinewidth{1.505625pt}%
\definecolor{currentstroke}{rgb}{1.000000,0.000000,0.000000}%
\pgfsetstrokecolor{currentstroke}%
\pgfsetdash{}{0pt}%
\pgfpathmoveto{\pgfqpoint{1.687658in}{2.124107in}}%
\pgfpathlineto{\pgfqpoint{1.659946in}{2.912350in}}%
\pgfusepath{stroke}%
\end{pgfscope}%
\begin{pgfscope}%
\pgfpathrectangle{\pgfqpoint{0.100000in}{0.212622in}}{\pgfqpoint{3.696000in}{3.696000in}}%
\pgfusepath{clip}%
\pgfsetrectcap%
\pgfsetroundjoin%
\pgfsetlinewidth{1.505625pt}%
\definecolor{currentstroke}{rgb}{1.000000,0.000000,0.000000}%
\pgfsetstrokecolor{currentstroke}%
\pgfsetdash{}{0pt}%
\pgfpathmoveto{\pgfqpoint{1.689905in}{2.123804in}}%
\pgfpathlineto{\pgfqpoint{1.659946in}{2.912350in}}%
\pgfusepath{stroke}%
\end{pgfscope}%
\begin{pgfscope}%
\pgfpathrectangle{\pgfqpoint{0.100000in}{0.212622in}}{\pgfqpoint{3.696000in}{3.696000in}}%
\pgfusepath{clip}%
\pgfsetrectcap%
\pgfsetroundjoin%
\pgfsetlinewidth{1.505625pt}%
\definecolor{currentstroke}{rgb}{1.000000,0.000000,0.000000}%
\pgfsetstrokecolor{currentstroke}%
\pgfsetdash{}{0pt}%
\pgfpathmoveto{\pgfqpoint{1.692824in}{2.123271in}}%
\pgfpathlineto{\pgfqpoint{1.659946in}{2.912350in}}%
\pgfusepath{stroke}%
\end{pgfscope}%
\begin{pgfscope}%
\pgfpathrectangle{\pgfqpoint{0.100000in}{0.212622in}}{\pgfqpoint{3.696000in}{3.696000in}}%
\pgfusepath{clip}%
\pgfsetrectcap%
\pgfsetroundjoin%
\pgfsetlinewidth{1.505625pt}%
\definecolor{currentstroke}{rgb}{1.000000,0.000000,0.000000}%
\pgfsetstrokecolor{currentstroke}%
\pgfsetdash{}{0pt}%
\pgfpathmoveto{\pgfqpoint{1.694417in}{2.123016in}}%
\pgfpathlineto{\pgfqpoint{1.659946in}{2.912350in}}%
\pgfusepath{stroke}%
\end{pgfscope}%
\begin{pgfscope}%
\pgfpathrectangle{\pgfqpoint{0.100000in}{0.212622in}}{\pgfqpoint{3.696000in}{3.696000in}}%
\pgfusepath{clip}%
\pgfsetrectcap%
\pgfsetroundjoin%
\pgfsetlinewidth{1.505625pt}%
\definecolor{currentstroke}{rgb}{1.000000,0.000000,0.000000}%
\pgfsetstrokecolor{currentstroke}%
\pgfsetdash{}{0pt}%
\pgfpathmoveto{\pgfqpoint{1.696697in}{2.122653in}}%
\pgfpathlineto{\pgfqpoint{1.659946in}{2.912350in}}%
\pgfusepath{stroke}%
\end{pgfscope}%
\begin{pgfscope}%
\pgfpathrectangle{\pgfqpoint{0.100000in}{0.212622in}}{\pgfqpoint{3.696000in}{3.696000in}}%
\pgfusepath{clip}%
\pgfsetrectcap%
\pgfsetroundjoin%
\pgfsetlinewidth{1.505625pt}%
\definecolor{currentstroke}{rgb}{1.000000,0.000000,0.000000}%
\pgfsetstrokecolor{currentstroke}%
\pgfsetdash{}{0pt}%
\pgfpathmoveto{\pgfqpoint{1.697843in}{2.122525in}}%
\pgfpathlineto{\pgfqpoint{1.659946in}{2.912350in}}%
\pgfusepath{stroke}%
\end{pgfscope}%
\begin{pgfscope}%
\pgfpathrectangle{\pgfqpoint{0.100000in}{0.212622in}}{\pgfqpoint{3.696000in}{3.696000in}}%
\pgfusepath{clip}%
\pgfsetrectcap%
\pgfsetroundjoin%
\pgfsetlinewidth{1.505625pt}%
\definecolor{currentstroke}{rgb}{1.000000,0.000000,0.000000}%
\pgfsetstrokecolor{currentstroke}%
\pgfsetdash{}{0pt}%
\pgfpathmoveto{\pgfqpoint{1.699636in}{2.122309in}}%
\pgfpathlineto{\pgfqpoint{1.659946in}{2.912350in}}%
\pgfusepath{stroke}%
\end{pgfscope}%
\begin{pgfscope}%
\pgfpathrectangle{\pgfqpoint{0.100000in}{0.212622in}}{\pgfqpoint{3.696000in}{3.696000in}}%
\pgfusepath{clip}%
\pgfsetrectcap%
\pgfsetroundjoin%
\pgfsetlinewidth{1.505625pt}%
\definecolor{currentstroke}{rgb}{1.000000,0.000000,0.000000}%
\pgfsetstrokecolor{currentstroke}%
\pgfsetdash{}{0pt}%
\pgfpathmoveto{\pgfqpoint{1.701767in}{2.121940in}}%
\pgfpathlineto{\pgfqpoint{1.659946in}{2.912350in}}%
\pgfusepath{stroke}%
\end{pgfscope}%
\begin{pgfscope}%
\pgfpathrectangle{\pgfqpoint{0.100000in}{0.212622in}}{\pgfqpoint{3.696000in}{3.696000in}}%
\pgfusepath{clip}%
\pgfsetrectcap%
\pgfsetroundjoin%
\pgfsetlinewidth{1.505625pt}%
\definecolor{currentstroke}{rgb}{1.000000,0.000000,0.000000}%
\pgfsetstrokecolor{currentstroke}%
\pgfsetdash{}{0pt}%
\pgfpathmoveto{\pgfqpoint{1.704573in}{2.121347in}}%
\pgfpathlineto{\pgfqpoint{1.659946in}{2.912350in}}%
\pgfusepath{stroke}%
\end{pgfscope}%
\begin{pgfscope}%
\pgfpathrectangle{\pgfqpoint{0.100000in}{0.212622in}}{\pgfqpoint{3.696000in}{3.696000in}}%
\pgfusepath{clip}%
\pgfsetrectcap%
\pgfsetroundjoin%
\pgfsetlinewidth{1.505625pt}%
\definecolor{currentstroke}{rgb}{1.000000,0.000000,0.000000}%
\pgfsetstrokecolor{currentstroke}%
\pgfsetdash{}{0pt}%
\pgfpathmoveto{\pgfqpoint{1.707838in}{2.120871in}}%
\pgfpathlineto{\pgfqpoint{1.659946in}{2.912350in}}%
\pgfusepath{stroke}%
\end{pgfscope}%
\begin{pgfscope}%
\pgfpathrectangle{\pgfqpoint{0.100000in}{0.212622in}}{\pgfqpoint{3.696000in}{3.696000in}}%
\pgfusepath{clip}%
\pgfsetrectcap%
\pgfsetroundjoin%
\pgfsetlinewidth{1.505625pt}%
\definecolor{currentstroke}{rgb}{1.000000,0.000000,0.000000}%
\pgfsetstrokecolor{currentstroke}%
\pgfsetdash{}{0pt}%
\pgfpathmoveto{\pgfqpoint{1.709643in}{2.120625in}}%
\pgfpathlineto{\pgfqpoint{1.659946in}{2.912350in}}%
\pgfusepath{stroke}%
\end{pgfscope}%
\begin{pgfscope}%
\pgfpathrectangle{\pgfqpoint{0.100000in}{0.212622in}}{\pgfqpoint{3.696000in}{3.696000in}}%
\pgfusepath{clip}%
\pgfsetrectcap%
\pgfsetroundjoin%
\pgfsetlinewidth{1.505625pt}%
\definecolor{currentstroke}{rgb}{1.000000,0.000000,0.000000}%
\pgfsetstrokecolor{currentstroke}%
\pgfsetdash{}{0pt}%
\pgfpathmoveto{\pgfqpoint{1.712716in}{2.119958in}}%
\pgfpathlineto{\pgfqpoint{1.659946in}{2.912350in}}%
\pgfusepath{stroke}%
\end{pgfscope}%
\begin{pgfscope}%
\pgfpathrectangle{\pgfqpoint{0.100000in}{0.212622in}}{\pgfqpoint{3.696000in}{3.696000in}}%
\pgfusepath{clip}%
\pgfsetrectcap%
\pgfsetroundjoin%
\pgfsetlinewidth{1.505625pt}%
\definecolor{currentstroke}{rgb}{1.000000,0.000000,0.000000}%
\pgfsetstrokecolor{currentstroke}%
\pgfsetdash{}{0pt}%
\pgfpathmoveto{\pgfqpoint{1.717703in}{2.119248in}}%
\pgfpathlineto{\pgfqpoint{1.659946in}{2.912350in}}%
\pgfusepath{stroke}%
\end{pgfscope}%
\begin{pgfscope}%
\pgfpathrectangle{\pgfqpoint{0.100000in}{0.212622in}}{\pgfqpoint{3.696000in}{3.696000in}}%
\pgfusepath{clip}%
\pgfsetrectcap%
\pgfsetroundjoin%
\pgfsetlinewidth{1.505625pt}%
\definecolor{currentstroke}{rgb}{1.000000,0.000000,0.000000}%
\pgfsetstrokecolor{currentstroke}%
\pgfsetdash{}{0pt}%
\pgfpathmoveto{\pgfqpoint{1.720421in}{2.118874in}}%
\pgfpathlineto{\pgfqpoint{1.659946in}{2.912350in}}%
\pgfusepath{stroke}%
\end{pgfscope}%
\begin{pgfscope}%
\pgfpathrectangle{\pgfqpoint{0.100000in}{0.212622in}}{\pgfqpoint{3.696000in}{3.696000in}}%
\pgfusepath{clip}%
\pgfsetrectcap%
\pgfsetroundjoin%
\pgfsetlinewidth{1.505625pt}%
\definecolor{currentstroke}{rgb}{1.000000,0.000000,0.000000}%
\pgfsetstrokecolor{currentstroke}%
\pgfsetdash{}{0pt}%
\pgfpathmoveto{\pgfqpoint{1.723742in}{2.118490in}}%
\pgfpathlineto{\pgfqpoint{1.659946in}{2.912350in}}%
\pgfusepath{stroke}%
\end{pgfscope}%
\begin{pgfscope}%
\pgfpathrectangle{\pgfqpoint{0.100000in}{0.212622in}}{\pgfqpoint{3.696000in}{3.696000in}}%
\pgfusepath{clip}%
\pgfsetrectcap%
\pgfsetroundjoin%
\pgfsetlinewidth{1.505625pt}%
\definecolor{currentstroke}{rgb}{1.000000,0.000000,0.000000}%
\pgfsetstrokecolor{currentstroke}%
\pgfsetdash{}{0pt}%
\pgfpathmoveto{\pgfqpoint{1.728017in}{2.117657in}}%
\pgfpathlineto{\pgfqpoint{1.659946in}{2.912350in}}%
\pgfusepath{stroke}%
\end{pgfscope}%
\begin{pgfscope}%
\pgfpathrectangle{\pgfqpoint{0.100000in}{0.212622in}}{\pgfqpoint{3.696000in}{3.696000in}}%
\pgfusepath{clip}%
\pgfsetrectcap%
\pgfsetroundjoin%
\pgfsetlinewidth{1.505625pt}%
\definecolor{currentstroke}{rgb}{1.000000,0.000000,0.000000}%
\pgfsetstrokecolor{currentstroke}%
\pgfsetdash{}{0pt}%
\pgfpathmoveto{\pgfqpoint{1.733217in}{2.116789in}}%
\pgfpathlineto{\pgfqpoint{1.659946in}{2.912350in}}%
\pgfusepath{stroke}%
\end{pgfscope}%
\begin{pgfscope}%
\pgfpathrectangle{\pgfqpoint{0.100000in}{0.212622in}}{\pgfqpoint{3.696000in}{3.696000in}}%
\pgfusepath{clip}%
\pgfsetrectcap%
\pgfsetroundjoin%
\pgfsetlinewidth{1.505625pt}%
\definecolor{currentstroke}{rgb}{1.000000,0.000000,0.000000}%
\pgfsetstrokecolor{currentstroke}%
\pgfsetdash{}{0pt}%
\pgfpathmoveto{\pgfqpoint{1.736015in}{2.116349in}}%
\pgfpathlineto{\pgfqpoint{1.659946in}{2.912350in}}%
\pgfusepath{stroke}%
\end{pgfscope}%
\begin{pgfscope}%
\pgfpathrectangle{\pgfqpoint{0.100000in}{0.212622in}}{\pgfqpoint{3.696000in}{3.696000in}}%
\pgfusepath{clip}%
\pgfsetrectcap%
\pgfsetroundjoin%
\pgfsetlinewidth{1.505625pt}%
\definecolor{currentstroke}{rgb}{1.000000,0.000000,0.000000}%
\pgfsetstrokecolor{currentstroke}%
\pgfsetdash{}{0pt}%
\pgfpathmoveto{\pgfqpoint{1.738937in}{2.115953in}}%
\pgfpathlineto{\pgfqpoint{1.659946in}{2.912350in}}%
\pgfusepath{stroke}%
\end{pgfscope}%
\begin{pgfscope}%
\pgfpathrectangle{\pgfqpoint{0.100000in}{0.212622in}}{\pgfqpoint{3.696000in}{3.696000in}}%
\pgfusepath{clip}%
\pgfsetrectcap%
\pgfsetroundjoin%
\pgfsetlinewidth{1.505625pt}%
\definecolor{currentstroke}{rgb}{1.000000,0.000000,0.000000}%
\pgfsetstrokecolor{currentstroke}%
\pgfsetdash{}{0pt}%
\pgfpathmoveto{\pgfqpoint{1.742582in}{2.115351in}}%
\pgfpathlineto{\pgfqpoint{1.659946in}{2.912350in}}%
\pgfusepath{stroke}%
\end{pgfscope}%
\begin{pgfscope}%
\pgfpathrectangle{\pgfqpoint{0.100000in}{0.212622in}}{\pgfqpoint{3.696000in}{3.696000in}}%
\pgfusepath{clip}%
\pgfsetrectcap%
\pgfsetroundjoin%
\pgfsetlinewidth{1.505625pt}%
\definecolor{currentstroke}{rgb}{1.000000,0.000000,0.000000}%
\pgfsetstrokecolor{currentstroke}%
\pgfsetdash{}{0pt}%
\pgfpathmoveto{\pgfqpoint{1.747533in}{2.114721in}}%
\pgfpathlineto{\pgfqpoint{1.659946in}{2.912350in}}%
\pgfusepath{stroke}%
\end{pgfscope}%
\begin{pgfscope}%
\pgfpathrectangle{\pgfqpoint{0.100000in}{0.212622in}}{\pgfqpoint{3.696000in}{3.696000in}}%
\pgfusepath{clip}%
\pgfsetrectcap%
\pgfsetroundjoin%
\pgfsetlinewidth{1.505625pt}%
\definecolor{currentstroke}{rgb}{1.000000,0.000000,0.000000}%
\pgfsetstrokecolor{currentstroke}%
\pgfsetdash{}{0pt}%
\pgfpathmoveto{\pgfqpoint{1.752799in}{2.114058in}}%
\pgfpathlineto{\pgfqpoint{1.659946in}{2.912350in}}%
\pgfusepath{stroke}%
\end{pgfscope}%
\begin{pgfscope}%
\pgfpathrectangle{\pgfqpoint{0.100000in}{0.212622in}}{\pgfqpoint{3.696000in}{3.696000in}}%
\pgfusepath{clip}%
\pgfsetrectcap%
\pgfsetroundjoin%
\pgfsetlinewidth{1.505625pt}%
\definecolor{currentstroke}{rgb}{1.000000,0.000000,0.000000}%
\pgfsetstrokecolor{currentstroke}%
\pgfsetdash{}{0pt}%
\pgfpathmoveto{\pgfqpoint{1.758865in}{2.113584in}}%
\pgfpathlineto{\pgfqpoint{1.659946in}{2.912350in}}%
\pgfusepath{stroke}%
\end{pgfscope}%
\begin{pgfscope}%
\pgfpathrectangle{\pgfqpoint{0.100000in}{0.212622in}}{\pgfqpoint{3.696000in}{3.696000in}}%
\pgfusepath{clip}%
\pgfsetrectcap%
\pgfsetroundjoin%
\pgfsetlinewidth{1.505625pt}%
\definecolor{currentstroke}{rgb}{1.000000,0.000000,0.000000}%
\pgfsetstrokecolor{currentstroke}%
\pgfsetdash{}{0pt}%
\pgfpathmoveto{\pgfqpoint{1.766165in}{2.112042in}}%
\pgfpathlineto{\pgfqpoint{1.659946in}{2.912350in}}%
\pgfusepath{stroke}%
\end{pgfscope}%
\begin{pgfscope}%
\pgfpathrectangle{\pgfqpoint{0.100000in}{0.212622in}}{\pgfqpoint{3.696000in}{3.696000in}}%
\pgfusepath{clip}%
\pgfsetrectcap%
\pgfsetroundjoin%
\pgfsetlinewidth{1.505625pt}%
\definecolor{currentstroke}{rgb}{1.000000,0.000000,0.000000}%
\pgfsetstrokecolor{currentstroke}%
\pgfsetdash{}{0pt}%
\pgfpathmoveto{\pgfqpoint{1.774176in}{2.110526in}}%
\pgfpathlineto{\pgfqpoint{1.659946in}{2.912350in}}%
\pgfusepath{stroke}%
\end{pgfscope}%
\begin{pgfscope}%
\pgfpathrectangle{\pgfqpoint{0.100000in}{0.212622in}}{\pgfqpoint{3.696000in}{3.696000in}}%
\pgfusepath{clip}%
\pgfsetrectcap%
\pgfsetroundjoin%
\pgfsetlinewidth{1.505625pt}%
\definecolor{currentstroke}{rgb}{1.000000,0.000000,0.000000}%
\pgfsetstrokecolor{currentstroke}%
\pgfsetdash{}{0pt}%
\pgfpathmoveto{\pgfqpoint{1.782570in}{2.109093in}}%
\pgfpathlineto{\pgfqpoint{1.659946in}{2.912350in}}%
\pgfusepath{stroke}%
\end{pgfscope}%
\begin{pgfscope}%
\pgfpathrectangle{\pgfqpoint{0.100000in}{0.212622in}}{\pgfqpoint{3.696000in}{3.696000in}}%
\pgfusepath{clip}%
\pgfsetrectcap%
\pgfsetroundjoin%
\pgfsetlinewidth{1.505625pt}%
\definecolor{currentstroke}{rgb}{1.000000,0.000000,0.000000}%
\pgfsetstrokecolor{currentstroke}%
\pgfsetdash{}{0pt}%
\pgfpathmoveto{\pgfqpoint{1.791001in}{2.107750in}}%
\pgfpathlineto{\pgfqpoint{1.659946in}{2.912350in}}%
\pgfusepath{stroke}%
\end{pgfscope}%
\begin{pgfscope}%
\pgfpathrectangle{\pgfqpoint{0.100000in}{0.212622in}}{\pgfqpoint{3.696000in}{3.696000in}}%
\pgfusepath{clip}%
\pgfsetrectcap%
\pgfsetroundjoin%
\pgfsetlinewidth{1.505625pt}%
\definecolor{currentstroke}{rgb}{1.000000,0.000000,0.000000}%
\pgfsetstrokecolor{currentstroke}%
\pgfsetdash{}{0pt}%
\pgfpathmoveto{\pgfqpoint{1.799919in}{2.106455in}}%
\pgfpathlineto{\pgfqpoint{1.659946in}{2.912350in}}%
\pgfusepath{stroke}%
\end{pgfscope}%
\begin{pgfscope}%
\pgfpathrectangle{\pgfqpoint{0.100000in}{0.212622in}}{\pgfqpoint{3.696000in}{3.696000in}}%
\pgfusepath{clip}%
\pgfsetrectcap%
\pgfsetroundjoin%
\pgfsetlinewidth{1.505625pt}%
\definecolor{currentstroke}{rgb}{1.000000,0.000000,0.000000}%
\pgfsetstrokecolor{currentstroke}%
\pgfsetdash{}{0pt}%
\pgfpathmoveto{\pgfqpoint{1.809630in}{2.105557in}}%
\pgfpathlineto{\pgfqpoint{1.659946in}{2.912350in}}%
\pgfusepath{stroke}%
\end{pgfscope}%
\begin{pgfscope}%
\pgfpathrectangle{\pgfqpoint{0.100000in}{0.212622in}}{\pgfqpoint{3.696000in}{3.696000in}}%
\pgfusepath{clip}%
\pgfsetrectcap%
\pgfsetroundjoin%
\pgfsetlinewidth{1.505625pt}%
\definecolor{currentstroke}{rgb}{1.000000,0.000000,0.000000}%
\pgfsetstrokecolor{currentstroke}%
\pgfsetdash{}{0pt}%
\pgfpathmoveto{\pgfqpoint{1.814209in}{2.105004in}}%
\pgfpathlineto{\pgfqpoint{1.659946in}{2.912350in}}%
\pgfusepath{stroke}%
\end{pgfscope}%
\begin{pgfscope}%
\pgfpathrectangle{\pgfqpoint{0.100000in}{0.212622in}}{\pgfqpoint{3.696000in}{3.696000in}}%
\pgfusepath{clip}%
\pgfsetrectcap%
\pgfsetroundjoin%
\pgfsetlinewidth{1.505625pt}%
\definecolor{currentstroke}{rgb}{1.000000,0.000000,0.000000}%
\pgfsetstrokecolor{currentstroke}%
\pgfsetdash{}{0pt}%
\pgfpathmoveto{\pgfqpoint{1.817129in}{2.104596in}}%
\pgfpathlineto{\pgfqpoint{1.659946in}{2.912350in}}%
\pgfusepath{stroke}%
\end{pgfscope}%
\begin{pgfscope}%
\pgfpathrectangle{\pgfqpoint{0.100000in}{0.212622in}}{\pgfqpoint{3.696000in}{3.696000in}}%
\pgfusepath{clip}%
\pgfsetrectcap%
\pgfsetroundjoin%
\pgfsetlinewidth{1.505625pt}%
\definecolor{currentstroke}{rgb}{1.000000,0.000000,0.000000}%
\pgfsetstrokecolor{currentstroke}%
\pgfsetdash{}{0pt}%
\pgfpathmoveto{\pgfqpoint{1.821677in}{2.103497in}}%
\pgfpathlineto{\pgfqpoint{1.659946in}{2.912350in}}%
\pgfusepath{stroke}%
\end{pgfscope}%
\begin{pgfscope}%
\pgfpathrectangle{\pgfqpoint{0.100000in}{0.212622in}}{\pgfqpoint{3.696000in}{3.696000in}}%
\pgfusepath{clip}%
\pgfsetrectcap%
\pgfsetroundjoin%
\pgfsetlinewidth{1.505625pt}%
\definecolor{currentstroke}{rgb}{1.000000,0.000000,0.000000}%
\pgfsetstrokecolor{currentstroke}%
\pgfsetdash{}{0pt}%
\pgfpathmoveto{\pgfqpoint{1.827567in}{2.102502in}}%
\pgfpathlineto{\pgfqpoint{1.659946in}{2.912350in}}%
\pgfusepath{stroke}%
\end{pgfscope}%
\begin{pgfscope}%
\pgfpathrectangle{\pgfqpoint{0.100000in}{0.212622in}}{\pgfqpoint{3.696000in}{3.696000in}}%
\pgfusepath{clip}%
\pgfsetrectcap%
\pgfsetroundjoin%
\pgfsetlinewidth{1.505625pt}%
\definecolor{currentstroke}{rgb}{1.000000,0.000000,0.000000}%
\pgfsetstrokecolor{currentstroke}%
\pgfsetdash{}{0pt}%
\pgfpathmoveto{\pgfqpoint{1.830614in}{2.102020in}}%
\pgfpathlineto{\pgfqpoint{1.659946in}{2.912350in}}%
\pgfusepath{stroke}%
\end{pgfscope}%
\begin{pgfscope}%
\pgfpathrectangle{\pgfqpoint{0.100000in}{0.212622in}}{\pgfqpoint{3.696000in}{3.696000in}}%
\pgfusepath{clip}%
\pgfsetrectcap%
\pgfsetroundjoin%
\pgfsetlinewidth{1.505625pt}%
\definecolor{currentstroke}{rgb}{1.000000,0.000000,0.000000}%
\pgfsetstrokecolor{currentstroke}%
\pgfsetdash{}{0pt}%
\pgfpathmoveto{\pgfqpoint{1.834084in}{2.101451in}}%
\pgfpathlineto{\pgfqpoint{1.659946in}{2.912350in}}%
\pgfusepath{stroke}%
\end{pgfscope}%
\begin{pgfscope}%
\pgfpathrectangle{\pgfqpoint{0.100000in}{0.212622in}}{\pgfqpoint{3.696000in}{3.696000in}}%
\pgfusepath{clip}%
\pgfsetrectcap%
\pgfsetroundjoin%
\pgfsetlinewidth{1.505625pt}%
\definecolor{currentstroke}{rgb}{1.000000,0.000000,0.000000}%
\pgfsetstrokecolor{currentstroke}%
\pgfsetdash{}{0pt}%
\pgfpathmoveto{\pgfqpoint{1.838239in}{2.100797in}}%
\pgfpathlineto{\pgfqpoint{1.674070in}{2.908524in}}%
\pgfusepath{stroke}%
\end{pgfscope}%
\begin{pgfscope}%
\pgfpathrectangle{\pgfqpoint{0.100000in}{0.212622in}}{\pgfqpoint{3.696000in}{3.696000in}}%
\pgfusepath{clip}%
\pgfsetrectcap%
\pgfsetroundjoin%
\pgfsetlinewidth{1.505625pt}%
\definecolor{currentstroke}{rgb}{1.000000,0.000000,0.000000}%
\pgfsetstrokecolor{currentstroke}%
\pgfsetdash{}{0pt}%
\pgfpathmoveto{\pgfqpoint{1.843930in}{2.100044in}}%
\pgfpathlineto{\pgfqpoint{1.674070in}{2.908524in}}%
\pgfusepath{stroke}%
\end{pgfscope}%
\begin{pgfscope}%
\pgfpathrectangle{\pgfqpoint{0.100000in}{0.212622in}}{\pgfqpoint{3.696000in}{3.696000in}}%
\pgfusepath{clip}%
\pgfsetrectcap%
\pgfsetroundjoin%
\pgfsetlinewidth{1.505625pt}%
\definecolor{currentstroke}{rgb}{1.000000,0.000000,0.000000}%
\pgfsetstrokecolor{currentstroke}%
\pgfsetdash{}{0pt}%
\pgfpathmoveto{\pgfqpoint{1.850217in}{2.098888in}}%
\pgfpathlineto{\pgfqpoint{1.688203in}{2.904695in}}%
\pgfusepath{stroke}%
\end{pgfscope}%
\begin{pgfscope}%
\pgfpathrectangle{\pgfqpoint{0.100000in}{0.212622in}}{\pgfqpoint{3.696000in}{3.696000in}}%
\pgfusepath{clip}%
\pgfsetrectcap%
\pgfsetroundjoin%
\pgfsetlinewidth{1.505625pt}%
\definecolor{currentstroke}{rgb}{1.000000,0.000000,0.000000}%
\pgfsetstrokecolor{currentstroke}%
\pgfsetdash{}{0pt}%
\pgfpathmoveto{\pgfqpoint{1.853477in}{2.098419in}}%
\pgfpathlineto{\pgfqpoint{1.688203in}{2.904695in}}%
\pgfusepath{stroke}%
\end{pgfscope}%
\begin{pgfscope}%
\pgfpathrectangle{\pgfqpoint{0.100000in}{0.212622in}}{\pgfqpoint{3.696000in}{3.696000in}}%
\pgfusepath{clip}%
\pgfsetrectcap%
\pgfsetroundjoin%
\pgfsetlinewidth{1.505625pt}%
\definecolor{currentstroke}{rgb}{1.000000,0.000000,0.000000}%
\pgfsetstrokecolor{currentstroke}%
\pgfsetdash{}{0pt}%
\pgfpathmoveto{\pgfqpoint{1.857041in}{2.097852in}}%
\pgfpathlineto{\pgfqpoint{1.688203in}{2.904695in}}%
\pgfusepath{stroke}%
\end{pgfscope}%
\begin{pgfscope}%
\pgfpathrectangle{\pgfqpoint{0.100000in}{0.212622in}}{\pgfqpoint{3.696000in}{3.696000in}}%
\pgfusepath{clip}%
\pgfsetrectcap%
\pgfsetroundjoin%
\pgfsetlinewidth{1.505625pt}%
\definecolor{currentstroke}{rgb}{1.000000,0.000000,0.000000}%
\pgfsetstrokecolor{currentstroke}%
\pgfsetdash{}{0pt}%
\pgfpathmoveto{\pgfqpoint{1.862417in}{2.096833in}}%
\pgfpathlineto{\pgfqpoint{1.702346in}{2.900864in}}%
\pgfusepath{stroke}%
\end{pgfscope}%
\begin{pgfscope}%
\pgfpathrectangle{\pgfqpoint{0.100000in}{0.212622in}}{\pgfqpoint{3.696000in}{3.696000in}}%
\pgfusepath{clip}%
\pgfsetrectcap%
\pgfsetroundjoin%
\pgfsetlinewidth{1.505625pt}%
\definecolor{currentstroke}{rgb}{1.000000,0.000000,0.000000}%
\pgfsetstrokecolor{currentstroke}%
\pgfsetdash{}{0pt}%
\pgfpathmoveto{\pgfqpoint{1.868129in}{2.095873in}}%
\pgfpathlineto{\pgfqpoint{1.702346in}{2.900864in}}%
\pgfusepath{stroke}%
\end{pgfscope}%
\begin{pgfscope}%
\pgfpathrectangle{\pgfqpoint{0.100000in}{0.212622in}}{\pgfqpoint{3.696000in}{3.696000in}}%
\pgfusepath{clip}%
\pgfsetrectcap%
\pgfsetroundjoin%
\pgfsetlinewidth{1.505625pt}%
\definecolor{currentstroke}{rgb}{1.000000,0.000000,0.000000}%
\pgfsetstrokecolor{currentstroke}%
\pgfsetdash{}{0pt}%
\pgfpathmoveto{\pgfqpoint{1.874654in}{2.094389in}}%
\pgfpathlineto{\pgfqpoint{1.702346in}{2.900864in}}%
\pgfusepath{stroke}%
\end{pgfscope}%
\begin{pgfscope}%
\pgfpathrectangle{\pgfqpoint{0.100000in}{0.212622in}}{\pgfqpoint{3.696000in}{3.696000in}}%
\pgfusepath{clip}%
\pgfsetrectcap%
\pgfsetroundjoin%
\pgfsetlinewidth{1.505625pt}%
\definecolor{currentstroke}{rgb}{1.000000,0.000000,0.000000}%
\pgfsetstrokecolor{currentstroke}%
\pgfsetdash{}{0pt}%
\pgfpathmoveto{\pgfqpoint{1.879898in}{2.093661in}}%
\pgfpathlineto{\pgfqpoint{1.716500in}{2.897029in}}%
\pgfusepath{stroke}%
\end{pgfscope}%
\begin{pgfscope}%
\pgfpathrectangle{\pgfqpoint{0.100000in}{0.212622in}}{\pgfqpoint{3.696000in}{3.696000in}}%
\pgfusepath{clip}%
\pgfsetrectcap%
\pgfsetroundjoin%
\pgfsetlinewidth{1.505625pt}%
\definecolor{currentstroke}{rgb}{1.000000,0.000000,0.000000}%
\pgfsetstrokecolor{currentstroke}%
\pgfsetdash{}{0pt}%
\pgfpathmoveto{\pgfqpoint{1.887305in}{2.091967in}}%
\pgfpathlineto{\pgfqpoint{1.730663in}{2.893193in}}%
\pgfusepath{stroke}%
\end{pgfscope}%
\begin{pgfscope}%
\pgfpathrectangle{\pgfqpoint{0.100000in}{0.212622in}}{\pgfqpoint{3.696000in}{3.696000in}}%
\pgfusepath{clip}%
\pgfsetrectcap%
\pgfsetroundjoin%
\pgfsetlinewidth{1.505625pt}%
\definecolor{currentstroke}{rgb}{1.000000,0.000000,0.000000}%
\pgfsetstrokecolor{currentstroke}%
\pgfsetdash{}{0pt}%
\pgfpathmoveto{\pgfqpoint{1.895105in}{2.091140in}}%
\pgfpathlineto{\pgfqpoint{1.730663in}{2.893193in}}%
\pgfusepath{stroke}%
\end{pgfscope}%
\begin{pgfscope}%
\pgfpathrectangle{\pgfqpoint{0.100000in}{0.212622in}}{\pgfqpoint{3.696000in}{3.696000in}}%
\pgfusepath{clip}%
\pgfsetrectcap%
\pgfsetroundjoin%
\pgfsetlinewidth{1.505625pt}%
\definecolor{currentstroke}{rgb}{1.000000,0.000000,0.000000}%
\pgfsetstrokecolor{currentstroke}%
\pgfsetdash{}{0pt}%
\pgfpathmoveto{\pgfqpoint{1.904674in}{2.089023in}}%
\pgfpathlineto{\pgfqpoint{1.744837in}{2.889353in}}%
\pgfusepath{stroke}%
\end{pgfscope}%
\begin{pgfscope}%
\pgfpathrectangle{\pgfqpoint{0.100000in}{0.212622in}}{\pgfqpoint{3.696000in}{3.696000in}}%
\pgfusepath{clip}%
\pgfsetrectcap%
\pgfsetroundjoin%
\pgfsetlinewidth{1.505625pt}%
\definecolor{currentstroke}{rgb}{1.000000,0.000000,0.000000}%
\pgfsetstrokecolor{currentstroke}%
\pgfsetdash{}{0pt}%
\pgfpathmoveto{\pgfqpoint{1.915292in}{2.086757in}}%
\pgfpathlineto{\pgfqpoint{1.759021in}{2.885510in}}%
\pgfusepath{stroke}%
\end{pgfscope}%
\begin{pgfscope}%
\pgfpathrectangle{\pgfqpoint{0.100000in}{0.212622in}}{\pgfqpoint{3.696000in}{3.696000in}}%
\pgfusepath{clip}%
\pgfsetrectcap%
\pgfsetroundjoin%
\pgfsetlinewidth{1.505625pt}%
\definecolor{currentstroke}{rgb}{1.000000,0.000000,0.000000}%
\pgfsetstrokecolor{currentstroke}%
\pgfsetdash{}{0pt}%
\pgfpathmoveto{\pgfqpoint{1.921308in}{2.085239in}}%
\pgfpathlineto{\pgfqpoint{1.759021in}{2.885510in}}%
\pgfusepath{stroke}%
\end{pgfscope}%
\begin{pgfscope}%
\pgfpathrectangle{\pgfqpoint{0.100000in}{0.212622in}}{\pgfqpoint{3.696000in}{3.696000in}}%
\pgfusepath{clip}%
\pgfsetrectcap%
\pgfsetroundjoin%
\pgfsetlinewidth{1.505625pt}%
\definecolor{currentstroke}{rgb}{1.000000,0.000000,0.000000}%
\pgfsetstrokecolor{currentstroke}%
\pgfsetdash{}{0pt}%
\pgfpathmoveto{\pgfqpoint{1.924635in}{2.084518in}}%
\pgfpathlineto{\pgfqpoint{1.759021in}{2.885510in}}%
\pgfusepath{stroke}%
\end{pgfscope}%
\begin{pgfscope}%
\pgfpathrectangle{\pgfqpoint{0.100000in}{0.212622in}}{\pgfqpoint{3.696000in}{3.696000in}}%
\pgfusepath{clip}%
\pgfsetrectcap%
\pgfsetroundjoin%
\pgfsetlinewidth{1.505625pt}%
\definecolor{currentstroke}{rgb}{1.000000,0.000000,0.000000}%
\pgfsetstrokecolor{currentstroke}%
\pgfsetdash{}{0pt}%
\pgfpathmoveto{\pgfqpoint{1.928552in}{2.083920in}}%
\pgfpathlineto{\pgfqpoint{1.773215in}{2.881665in}}%
\pgfusepath{stroke}%
\end{pgfscope}%
\begin{pgfscope}%
\pgfpathrectangle{\pgfqpoint{0.100000in}{0.212622in}}{\pgfqpoint{3.696000in}{3.696000in}}%
\pgfusepath{clip}%
\pgfsetrectcap%
\pgfsetroundjoin%
\pgfsetlinewidth{1.505625pt}%
\definecolor{currentstroke}{rgb}{1.000000,0.000000,0.000000}%
\pgfsetstrokecolor{currentstroke}%
\pgfsetdash{}{0pt}%
\pgfpathmoveto{\pgfqpoint{1.932760in}{2.083578in}}%
\pgfpathlineto{\pgfqpoint{1.773215in}{2.881665in}}%
\pgfusepath{stroke}%
\end{pgfscope}%
\begin{pgfscope}%
\pgfpathrectangle{\pgfqpoint{0.100000in}{0.212622in}}{\pgfqpoint{3.696000in}{3.696000in}}%
\pgfusepath{clip}%
\pgfsetrectcap%
\pgfsetroundjoin%
\pgfsetlinewidth{1.505625pt}%
\definecolor{currentstroke}{rgb}{1.000000,0.000000,0.000000}%
\pgfsetstrokecolor{currentstroke}%
\pgfsetdash{}{0pt}%
\pgfpathmoveto{\pgfqpoint{1.938515in}{2.082422in}}%
\pgfpathlineto{\pgfqpoint{1.773215in}{2.881665in}}%
\pgfusepath{stroke}%
\end{pgfscope}%
\begin{pgfscope}%
\pgfpathrectangle{\pgfqpoint{0.100000in}{0.212622in}}{\pgfqpoint{3.696000in}{3.696000in}}%
\pgfusepath{clip}%
\pgfsetrectcap%
\pgfsetroundjoin%
\pgfsetlinewidth{1.505625pt}%
\definecolor{currentstroke}{rgb}{1.000000,0.000000,0.000000}%
\pgfsetstrokecolor{currentstroke}%
\pgfsetdash{}{0pt}%
\pgfpathmoveto{\pgfqpoint{1.941847in}{2.081652in}}%
\pgfpathlineto{\pgfqpoint{1.787419in}{2.877817in}}%
\pgfusepath{stroke}%
\end{pgfscope}%
\begin{pgfscope}%
\pgfpathrectangle{\pgfqpoint{0.100000in}{0.212622in}}{\pgfqpoint{3.696000in}{3.696000in}}%
\pgfusepath{clip}%
\pgfsetrectcap%
\pgfsetroundjoin%
\pgfsetlinewidth{1.505625pt}%
\definecolor{currentstroke}{rgb}{1.000000,0.000000,0.000000}%
\pgfsetstrokecolor{currentstroke}%
\pgfsetdash{}{0pt}%
\pgfpathmoveto{\pgfqpoint{1.945277in}{2.081069in}}%
\pgfpathlineto{\pgfqpoint{1.787419in}{2.877817in}}%
\pgfusepath{stroke}%
\end{pgfscope}%
\begin{pgfscope}%
\pgfpathrectangle{\pgfqpoint{0.100000in}{0.212622in}}{\pgfqpoint{3.696000in}{3.696000in}}%
\pgfusepath{clip}%
\pgfsetrectcap%
\pgfsetroundjoin%
\pgfsetlinewidth{1.505625pt}%
\definecolor{currentstroke}{rgb}{1.000000,0.000000,0.000000}%
\pgfsetstrokecolor{currentstroke}%
\pgfsetdash{}{0pt}%
\pgfpathmoveto{\pgfqpoint{1.949444in}{2.080172in}}%
\pgfpathlineto{\pgfqpoint{1.787419in}{2.877817in}}%
\pgfusepath{stroke}%
\end{pgfscope}%
\begin{pgfscope}%
\pgfpathrectangle{\pgfqpoint{0.100000in}{0.212622in}}{\pgfqpoint{3.696000in}{3.696000in}}%
\pgfusepath{clip}%
\pgfsetrectcap%
\pgfsetroundjoin%
\pgfsetlinewidth{1.505625pt}%
\definecolor{currentstroke}{rgb}{1.000000,0.000000,0.000000}%
\pgfsetstrokecolor{currentstroke}%
\pgfsetdash{}{0pt}%
\pgfpathmoveto{\pgfqpoint{1.951579in}{2.079684in}}%
\pgfpathlineto{\pgfqpoint{1.787419in}{2.877817in}}%
\pgfusepath{stroke}%
\end{pgfscope}%
\begin{pgfscope}%
\pgfpathrectangle{\pgfqpoint{0.100000in}{0.212622in}}{\pgfqpoint{3.696000in}{3.696000in}}%
\pgfusepath{clip}%
\pgfsetrectcap%
\pgfsetroundjoin%
\pgfsetlinewidth{1.505625pt}%
\definecolor{currentstroke}{rgb}{1.000000,0.000000,0.000000}%
\pgfsetstrokecolor{currentstroke}%
\pgfsetdash{}{0pt}%
\pgfpathmoveto{\pgfqpoint{1.952722in}{2.079490in}}%
\pgfpathlineto{\pgfqpoint{1.787419in}{2.877817in}}%
\pgfusepath{stroke}%
\end{pgfscope}%
\begin{pgfscope}%
\pgfpathrectangle{\pgfqpoint{0.100000in}{0.212622in}}{\pgfqpoint{3.696000in}{3.696000in}}%
\pgfusepath{clip}%
\pgfsetrectcap%
\pgfsetroundjoin%
\pgfsetlinewidth{1.505625pt}%
\definecolor{currentstroke}{rgb}{1.000000,0.000000,0.000000}%
\pgfsetstrokecolor{currentstroke}%
\pgfsetdash{}{0pt}%
\pgfpathmoveto{\pgfqpoint{1.955712in}{2.078582in}}%
\pgfpathlineto{\pgfqpoint{1.801633in}{2.873967in}}%
\pgfusepath{stroke}%
\end{pgfscope}%
\begin{pgfscope}%
\pgfpathrectangle{\pgfqpoint{0.100000in}{0.212622in}}{\pgfqpoint{3.696000in}{3.696000in}}%
\pgfusepath{clip}%
\pgfsetrectcap%
\pgfsetroundjoin%
\pgfsetlinewidth{1.505625pt}%
\definecolor{currentstroke}{rgb}{1.000000,0.000000,0.000000}%
\pgfsetstrokecolor{currentstroke}%
\pgfsetdash{}{0pt}%
\pgfpathmoveto{\pgfqpoint{1.958493in}{2.078186in}}%
\pgfpathlineto{\pgfqpoint{1.801633in}{2.873967in}}%
\pgfusepath{stroke}%
\end{pgfscope}%
\begin{pgfscope}%
\pgfpathrectangle{\pgfqpoint{0.100000in}{0.212622in}}{\pgfqpoint{3.696000in}{3.696000in}}%
\pgfusepath{clip}%
\pgfsetrectcap%
\pgfsetroundjoin%
\pgfsetlinewidth{1.505625pt}%
\definecolor{currentstroke}{rgb}{1.000000,0.000000,0.000000}%
\pgfsetstrokecolor{currentstroke}%
\pgfsetdash{}{0pt}%
\pgfpathmoveto{\pgfqpoint{1.962050in}{2.077841in}}%
\pgfpathlineto{\pgfqpoint{1.801633in}{2.873967in}}%
\pgfusepath{stroke}%
\end{pgfscope}%
\begin{pgfscope}%
\pgfpathrectangle{\pgfqpoint{0.100000in}{0.212622in}}{\pgfqpoint{3.696000in}{3.696000in}}%
\pgfusepath{clip}%
\pgfsetrectcap%
\pgfsetroundjoin%
\pgfsetlinewidth{1.505625pt}%
\definecolor{currentstroke}{rgb}{1.000000,0.000000,0.000000}%
\pgfsetstrokecolor{currentstroke}%
\pgfsetdash{}{0pt}%
\pgfpathmoveto{\pgfqpoint{1.968682in}{2.076592in}}%
\pgfpathlineto{\pgfqpoint{1.815857in}{2.870113in}}%
\pgfusepath{stroke}%
\end{pgfscope}%
\begin{pgfscope}%
\pgfpathrectangle{\pgfqpoint{0.100000in}{0.212622in}}{\pgfqpoint{3.696000in}{3.696000in}}%
\pgfusepath{clip}%
\pgfsetrectcap%
\pgfsetroundjoin%
\pgfsetlinewidth{1.505625pt}%
\definecolor{currentstroke}{rgb}{1.000000,0.000000,0.000000}%
\pgfsetstrokecolor{currentstroke}%
\pgfsetdash{}{0pt}%
\pgfpathmoveto{\pgfqpoint{1.972105in}{2.076163in}}%
\pgfpathlineto{\pgfqpoint{1.815857in}{2.870113in}}%
\pgfusepath{stroke}%
\end{pgfscope}%
\begin{pgfscope}%
\pgfpathrectangle{\pgfqpoint{0.100000in}{0.212622in}}{\pgfqpoint{3.696000in}{3.696000in}}%
\pgfusepath{clip}%
\pgfsetrectcap%
\pgfsetroundjoin%
\pgfsetlinewidth{1.505625pt}%
\definecolor{currentstroke}{rgb}{1.000000,0.000000,0.000000}%
\pgfsetstrokecolor{currentstroke}%
\pgfsetdash{}{0pt}%
\pgfpathmoveto{\pgfqpoint{1.974208in}{2.075754in}}%
\pgfpathlineto{\pgfqpoint{1.815857in}{2.870113in}}%
\pgfusepath{stroke}%
\end{pgfscope}%
\begin{pgfscope}%
\pgfpathrectangle{\pgfqpoint{0.100000in}{0.212622in}}{\pgfqpoint{3.696000in}{3.696000in}}%
\pgfusepath{clip}%
\pgfsetrectcap%
\pgfsetroundjoin%
\pgfsetlinewidth{1.505625pt}%
\definecolor{currentstroke}{rgb}{1.000000,0.000000,0.000000}%
\pgfsetstrokecolor{currentstroke}%
\pgfsetdash{}{0pt}%
\pgfpathmoveto{\pgfqpoint{1.976616in}{2.075283in}}%
\pgfpathlineto{\pgfqpoint{1.815857in}{2.870113in}}%
\pgfusepath{stroke}%
\end{pgfscope}%
\begin{pgfscope}%
\pgfpathrectangle{\pgfqpoint{0.100000in}{0.212622in}}{\pgfqpoint{3.696000in}{3.696000in}}%
\pgfusepath{clip}%
\pgfsetrectcap%
\pgfsetroundjoin%
\pgfsetlinewidth{1.505625pt}%
\definecolor{currentstroke}{rgb}{1.000000,0.000000,0.000000}%
\pgfsetstrokecolor{currentstroke}%
\pgfsetdash{}{0pt}%
\pgfpathmoveto{\pgfqpoint{1.979814in}{2.074765in}}%
\pgfpathlineto{\pgfqpoint{1.815857in}{2.870113in}}%
\pgfusepath{stroke}%
\end{pgfscope}%
\begin{pgfscope}%
\pgfpathrectangle{\pgfqpoint{0.100000in}{0.212622in}}{\pgfqpoint{3.696000in}{3.696000in}}%
\pgfusepath{clip}%
\pgfsetrectcap%
\pgfsetroundjoin%
\pgfsetlinewidth{1.505625pt}%
\definecolor{currentstroke}{rgb}{1.000000,0.000000,0.000000}%
\pgfsetstrokecolor{currentstroke}%
\pgfsetdash{}{0pt}%
\pgfpathmoveto{\pgfqpoint{1.981820in}{2.074198in}}%
\pgfpathlineto{\pgfqpoint{1.830092in}{2.866257in}}%
\pgfusepath{stroke}%
\end{pgfscope}%
\begin{pgfscope}%
\pgfpathrectangle{\pgfqpoint{0.100000in}{0.212622in}}{\pgfqpoint{3.696000in}{3.696000in}}%
\pgfusepath{clip}%
\pgfsetrectcap%
\pgfsetroundjoin%
\pgfsetlinewidth{1.505625pt}%
\definecolor{currentstroke}{rgb}{1.000000,0.000000,0.000000}%
\pgfsetstrokecolor{currentstroke}%
\pgfsetdash{}{0pt}%
\pgfpathmoveto{\pgfqpoint{1.984289in}{2.073719in}}%
\pgfpathlineto{\pgfqpoint{1.830092in}{2.866257in}}%
\pgfusepath{stroke}%
\end{pgfscope}%
\begin{pgfscope}%
\pgfpathrectangle{\pgfqpoint{0.100000in}{0.212622in}}{\pgfqpoint{3.696000in}{3.696000in}}%
\pgfusepath{clip}%
\pgfsetrectcap%
\pgfsetroundjoin%
\pgfsetlinewidth{1.505625pt}%
\definecolor{currentstroke}{rgb}{1.000000,0.000000,0.000000}%
\pgfsetstrokecolor{currentstroke}%
\pgfsetdash{}{0pt}%
\pgfpathmoveto{\pgfqpoint{1.987371in}{2.073132in}}%
\pgfpathlineto{\pgfqpoint{1.830092in}{2.866257in}}%
\pgfusepath{stroke}%
\end{pgfscope}%
\begin{pgfscope}%
\pgfpathrectangle{\pgfqpoint{0.100000in}{0.212622in}}{\pgfqpoint{3.696000in}{3.696000in}}%
\pgfusepath{clip}%
\pgfsetrectcap%
\pgfsetroundjoin%
\pgfsetlinewidth{1.505625pt}%
\definecolor{currentstroke}{rgb}{1.000000,0.000000,0.000000}%
\pgfsetstrokecolor{currentstroke}%
\pgfsetdash{}{0pt}%
\pgfpathmoveto{\pgfqpoint{1.991880in}{2.072119in}}%
\pgfpathlineto{\pgfqpoint{1.830092in}{2.866257in}}%
\pgfusepath{stroke}%
\end{pgfscope}%
\begin{pgfscope}%
\pgfpathrectangle{\pgfqpoint{0.100000in}{0.212622in}}{\pgfqpoint{3.696000in}{3.696000in}}%
\pgfusepath{clip}%
\pgfsetrectcap%
\pgfsetroundjoin%
\pgfsetlinewidth{1.505625pt}%
\definecolor{currentstroke}{rgb}{1.000000,0.000000,0.000000}%
\pgfsetstrokecolor{currentstroke}%
\pgfsetdash{}{0pt}%
\pgfpathmoveto{\pgfqpoint{1.996486in}{2.071351in}}%
\pgfpathlineto{\pgfqpoint{1.844337in}{2.862398in}}%
\pgfusepath{stroke}%
\end{pgfscope}%
\begin{pgfscope}%
\pgfpathrectangle{\pgfqpoint{0.100000in}{0.212622in}}{\pgfqpoint{3.696000in}{3.696000in}}%
\pgfusepath{clip}%
\pgfsetrectcap%
\pgfsetroundjoin%
\pgfsetlinewidth{1.505625pt}%
\definecolor{currentstroke}{rgb}{1.000000,0.000000,0.000000}%
\pgfsetstrokecolor{currentstroke}%
\pgfsetdash{}{0pt}%
\pgfpathmoveto{\pgfqpoint{2.002080in}{2.070362in}}%
\pgfpathlineto{\pgfqpoint{1.844337in}{2.862398in}}%
\pgfusepath{stroke}%
\end{pgfscope}%
\begin{pgfscope}%
\pgfpathrectangle{\pgfqpoint{0.100000in}{0.212622in}}{\pgfqpoint{3.696000in}{3.696000in}}%
\pgfusepath{clip}%
\pgfsetrectcap%
\pgfsetroundjoin%
\pgfsetlinewidth{1.505625pt}%
\definecolor{currentstroke}{rgb}{1.000000,0.000000,0.000000}%
\pgfsetstrokecolor{currentstroke}%
\pgfsetdash{}{0pt}%
\pgfpathmoveto{\pgfqpoint{2.007632in}{2.069499in}}%
\pgfpathlineto{\pgfqpoint{1.858592in}{2.858537in}}%
\pgfusepath{stroke}%
\end{pgfscope}%
\begin{pgfscope}%
\pgfpathrectangle{\pgfqpoint{0.100000in}{0.212622in}}{\pgfqpoint{3.696000in}{3.696000in}}%
\pgfusepath{clip}%
\pgfsetrectcap%
\pgfsetroundjoin%
\pgfsetlinewidth{1.505625pt}%
\definecolor{currentstroke}{rgb}{1.000000,0.000000,0.000000}%
\pgfsetstrokecolor{currentstroke}%
\pgfsetdash{}{0pt}%
\pgfpathmoveto{\pgfqpoint{2.013261in}{2.069071in}}%
\pgfpathlineto{\pgfqpoint{1.858592in}{2.858537in}}%
\pgfusepath{stroke}%
\end{pgfscope}%
\begin{pgfscope}%
\pgfpathrectangle{\pgfqpoint{0.100000in}{0.212622in}}{\pgfqpoint{3.696000in}{3.696000in}}%
\pgfusepath{clip}%
\pgfsetrectcap%
\pgfsetroundjoin%
\pgfsetlinewidth{1.505625pt}%
\definecolor{currentstroke}{rgb}{1.000000,0.000000,0.000000}%
\pgfsetstrokecolor{currentstroke}%
\pgfsetdash{}{0pt}%
\pgfpathmoveto{\pgfqpoint{2.020618in}{2.067571in}}%
\pgfpathlineto{\pgfqpoint{1.872857in}{2.854672in}}%
\pgfusepath{stroke}%
\end{pgfscope}%
\begin{pgfscope}%
\pgfpathrectangle{\pgfqpoint{0.100000in}{0.212622in}}{\pgfqpoint{3.696000in}{3.696000in}}%
\pgfusepath{clip}%
\pgfsetrectcap%
\pgfsetroundjoin%
\pgfsetlinewidth{1.505625pt}%
\definecolor{currentstroke}{rgb}{1.000000,0.000000,0.000000}%
\pgfsetstrokecolor{currentstroke}%
\pgfsetdash{}{0pt}%
\pgfpathmoveto{\pgfqpoint{2.028732in}{2.065833in}}%
\pgfpathlineto{\pgfqpoint{1.872857in}{2.854672in}}%
\pgfusepath{stroke}%
\end{pgfscope}%
\begin{pgfscope}%
\pgfpathrectangle{\pgfqpoint{0.100000in}{0.212622in}}{\pgfqpoint{3.696000in}{3.696000in}}%
\pgfusepath{clip}%
\pgfsetrectcap%
\pgfsetroundjoin%
\pgfsetlinewidth{1.505625pt}%
\definecolor{currentstroke}{rgb}{1.000000,0.000000,0.000000}%
\pgfsetstrokecolor{currentstroke}%
\pgfsetdash{}{0pt}%
\pgfpathmoveto{\pgfqpoint{2.038565in}{2.063881in}}%
\pgfpathlineto{\pgfqpoint{1.887132in}{2.850805in}}%
\pgfusepath{stroke}%
\end{pgfscope}%
\begin{pgfscope}%
\pgfpathrectangle{\pgfqpoint{0.100000in}{0.212622in}}{\pgfqpoint{3.696000in}{3.696000in}}%
\pgfusepath{clip}%
\pgfsetrectcap%
\pgfsetroundjoin%
\pgfsetlinewidth{1.505625pt}%
\definecolor{currentstroke}{rgb}{1.000000,0.000000,0.000000}%
\pgfsetstrokecolor{currentstroke}%
\pgfsetdash{}{0pt}%
\pgfpathmoveto{\pgfqpoint{2.049465in}{2.061316in}}%
\pgfpathlineto{\pgfqpoint{1.901418in}{2.846935in}}%
\pgfusepath{stroke}%
\end{pgfscope}%
\begin{pgfscope}%
\pgfpathrectangle{\pgfqpoint{0.100000in}{0.212622in}}{\pgfqpoint{3.696000in}{3.696000in}}%
\pgfusepath{clip}%
\pgfsetrectcap%
\pgfsetroundjoin%
\pgfsetlinewidth{1.505625pt}%
\definecolor{currentstroke}{rgb}{1.000000,0.000000,0.000000}%
\pgfsetstrokecolor{currentstroke}%
\pgfsetdash{}{0pt}%
\pgfpathmoveto{\pgfqpoint{2.061282in}{2.058769in}}%
\pgfpathlineto{\pgfqpoint{1.915714in}{2.843062in}}%
\pgfusepath{stroke}%
\end{pgfscope}%
\begin{pgfscope}%
\pgfpathrectangle{\pgfqpoint{0.100000in}{0.212622in}}{\pgfqpoint{3.696000in}{3.696000in}}%
\pgfusepath{clip}%
\pgfsetrectcap%
\pgfsetroundjoin%
\pgfsetlinewidth{1.505625pt}%
\definecolor{currentstroke}{rgb}{1.000000,0.000000,0.000000}%
\pgfsetstrokecolor{currentstroke}%
\pgfsetdash{}{0pt}%
\pgfpathmoveto{\pgfqpoint{2.074451in}{2.055635in}}%
\pgfpathlineto{\pgfqpoint{1.930020in}{2.839187in}}%
\pgfusepath{stroke}%
\end{pgfscope}%
\begin{pgfscope}%
\pgfpathrectangle{\pgfqpoint{0.100000in}{0.212622in}}{\pgfqpoint{3.696000in}{3.696000in}}%
\pgfusepath{clip}%
\pgfsetrectcap%
\pgfsetroundjoin%
\pgfsetlinewidth{1.505625pt}%
\definecolor{currentstroke}{rgb}{1.000000,0.000000,0.000000}%
\pgfsetstrokecolor{currentstroke}%
\pgfsetdash{}{0pt}%
\pgfpathmoveto{\pgfqpoint{2.087634in}{2.053178in}}%
\pgfpathlineto{\pgfqpoint{1.944337in}{2.835308in}}%
\pgfusepath{stroke}%
\end{pgfscope}%
\begin{pgfscope}%
\pgfpathrectangle{\pgfqpoint{0.100000in}{0.212622in}}{\pgfqpoint{3.696000in}{3.696000in}}%
\pgfusepath{clip}%
\pgfsetrectcap%
\pgfsetroundjoin%
\pgfsetlinewidth{1.505625pt}%
\definecolor{currentstroke}{rgb}{1.000000,0.000000,0.000000}%
\pgfsetstrokecolor{currentstroke}%
\pgfsetdash{}{0pt}%
\pgfpathmoveto{\pgfqpoint{2.102638in}{2.049886in}}%
\pgfpathlineto{\pgfqpoint{1.958663in}{2.831427in}}%
\pgfusepath{stroke}%
\end{pgfscope}%
\begin{pgfscope}%
\pgfpathrectangle{\pgfqpoint{0.100000in}{0.212622in}}{\pgfqpoint{3.696000in}{3.696000in}}%
\pgfusepath{clip}%
\pgfsetrectcap%
\pgfsetroundjoin%
\pgfsetlinewidth{1.505625pt}%
\definecolor{currentstroke}{rgb}{1.000000,0.000000,0.000000}%
\pgfsetstrokecolor{currentstroke}%
\pgfsetdash{}{0pt}%
\pgfpathmoveto{\pgfqpoint{2.109392in}{2.049570in}}%
\pgfpathlineto{\pgfqpoint{1.973000in}{2.827543in}}%
\pgfusepath{stroke}%
\end{pgfscope}%
\begin{pgfscope}%
\pgfpathrectangle{\pgfqpoint{0.100000in}{0.212622in}}{\pgfqpoint{3.696000in}{3.696000in}}%
\pgfusepath{clip}%
\pgfsetrectcap%
\pgfsetroundjoin%
\pgfsetlinewidth{1.505625pt}%
\definecolor{currentstroke}{rgb}{1.000000,0.000000,0.000000}%
\pgfsetstrokecolor{currentstroke}%
\pgfsetdash{}{0pt}%
\pgfpathmoveto{\pgfqpoint{2.118397in}{2.047541in}}%
\pgfpathlineto{\pgfqpoint{1.973000in}{2.827543in}}%
\pgfusepath{stroke}%
\end{pgfscope}%
\begin{pgfscope}%
\pgfpathrectangle{\pgfqpoint{0.100000in}{0.212622in}}{\pgfqpoint{3.696000in}{3.696000in}}%
\pgfusepath{clip}%
\pgfsetrectcap%
\pgfsetroundjoin%
\pgfsetlinewidth{1.505625pt}%
\definecolor{currentstroke}{rgb}{1.000000,0.000000,0.000000}%
\pgfsetstrokecolor{currentstroke}%
\pgfsetdash{}{0pt}%
\pgfpathmoveto{\pgfqpoint{2.128236in}{2.044549in}}%
\pgfpathlineto{\pgfqpoint{1.987348in}{2.823657in}}%
\pgfusepath{stroke}%
\end{pgfscope}%
\begin{pgfscope}%
\pgfpathrectangle{\pgfqpoint{0.100000in}{0.212622in}}{\pgfqpoint{3.696000in}{3.696000in}}%
\pgfusepath{clip}%
\pgfsetrectcap%
\pgfsetroundjoin%
\pgfsetlinewidth{1.505625pt}%
\definecolor{currentstroke}{rgb}{1.000000,0.000000,0.000000}%
\pgfsetstrokecolor{currentstroke}%
\pgfsetdash{}{0pt}%
\pgfpathmoveto{\pgfqpoint{2.138280in}{2.042321in}}%
\pgfpathlineto{\pgfqpoint{2.001705in}{2.819767in}}%
\pgfusepath{stroke}%
\end{pgfscope}%
\begin{pgfscope}%
\pgfpathrectangle{\pgfqpoint{0.100000in}{0.212622in}}{\pgfqpoint{3.696000in}{3.696000in}}%
\pgfusepath{clip}%
\pgfsetrectcap%
\pgfsetroundjoin%
\pgfsetlinewidth{1.505625pt}%
\definecolor{currentstroke}{rgb}{1.000000,0.000000,0.000000}%
\pgfsetstrokecolor{currentstroke}%
\pgfsetdash{}{0pt}%
\pgfpathmoveto{\pgfqpoint{2.149027in}{2.040104in}}%
\pgfpathlineto{\pgfqpoint{2.001705in}{2.819767in}}%
\pgfusepath{stroke}%
\end{pgfscope}%
\begin{pgfscope}%
\pgfpathrectangle{\pgfqpoint{0.100000in}{0.212622in}}{\pgfqpoint{3.696000in}{3.696000in}}%
\pgfusepath{clip}%
\pgfsetrectcap%
\pgfsetroundjoin%
\pgfsetlinewidth{1.505625pt}%
\definecolor{currentstroke}{rgb}{1.000000,0.000000,0.000000}%
\pgfsetstrokecolor{currentstroke}%
\pgfsetdash{}{0pt}%
\pgfpathmoveto{\pgfqpoint{2.160389in}{2.038483in}}%
\pgfpathlineto{\pgfqpoint{2.016073in}{2.815875in}}%
\pgfusepath{stroke}%
\end{pgfscope}%
\begin{pgfscope}%
\pgfpathrectangle{\pgfqpoint{0.100000in}{0.212622in}}{\pgfqpoint{3.696000in}{3.696000in}}%
\pgfusepath{clip}%
\pgfsetrectcap%
\pgfsetroundjoin%
\pgfsetlinewidth{1.505625pt}%
\definecolor{currentstroke}{rgb}{1.000000,0.000000,0.000000}%
\pgfsetstrokecolor{currentstroke}%
\pgfsetdash{}{0pt}%
\pgfpathmoveto{\pgfqpoint{2.173267in}{2.035480in}}%
\pgfpathlineto{\pgfqpoint{2.030452in}{2.811980in}}%
\pgfusepath{stroke}%
\end{pgfscope}%
\begin{pgfscope}%
\pgfpathrectangle{\pgfqpoint{0.100000in}{0.212622in}}{\pgfqpoint{3.696000in}{3.696000in}}%
\pgfusepath{clip}%
\pgfsetrectcap%
\pgfsetroundjoin%
\pgfsetlinewidth{1.505625pt}%
\definecolor{currentstroke}{rgb}{1.000000,0.000000,0.000000}%
\pgfsetstrokecolor{currentstroke}%
\pgfsetdash{}{0pt}%
\pgfpathmoveto{\pgfqpoint{2.185978in}{2.032743in}}%
\pgfpathlineto{\pgfqpoint{2.044840in}{2.808082in}}%
\pgfusepath{stroke}%
\end{pgfscope}%
\begin{pgfscope}%
\pgfpathrectangle{\pgfqpoint{0.100000in}{0.212622in}}{\pgfqpoint{3.696000in}{3.696000in}}%
\pgfusepath{clip}%
\pgfsetrectcap%
\pgfsetroundjoin%
\pgfsetlinewidth{1.505625pt}%
\definecolor{currentstroke}{rgb}{1.000000,0.000000,0.000000}%
\pgfsetstrokecolor{currentstroke}%
\pgfsetdash{}{0pt}%
\pgfpathmoveto{\pgfqpoint{2.197226in}{2.031722in}}%
\pgfpathlineto{\pgfqpoint{2.059239in}{2.804181in}}%
\pgfusepath{stroke}%
\end{pgfscope}%
\begin{pgfscope}%
\pgfpathrectangle{\pgfqpoint{0.100000in}{0.212622in}}{\pgfqpoint{3.696000in}{3.696000in}}%
\pgfusepath{clip}%
\pgfsetrectcap%
\pgfsetroundjoin%
\pgfsetlinewidth{1.505625pt}%
\definecolor{currentstroke}{rgb}{1.000000,0.000000,0.000000}%
\pgfsetstrokecolor{currentstroke}%
\pgfsetdash{}{0pt}%
\pgfpathmoveto{\pgfqpoint{2.211906in}{2.028476in}}%
\pgfpathlineto{\pgfqpoint{2.073649in}{2.800278in}}%
\pgfusepath{stroke}%
\end{pgfscope}%
\begin{pgfscope}%
\pgfpathrectangle{\pgfqpoint{0.100000in}{0.212622in}}{\pgfqpoint{3.696000in}{3.696000in}}%
\pgfusepath{clip}%
\pgfsetrectcap%
\pgfsetroundjoin%
\pgfsetlinewidth{1.505625pt}%
\definecolor{currentstroke}{rgb}{1.000000,0.000000,0.000000}%
\pgfsetstrokecolor{currentstroke}%
\pgfsetdash{}{0pt}%
\pgfpathmoveto{\pgfqpoint{2.219209in}{2.027237in}}%
\pgfpathlineto{\pgfqpoint{2.088068in}{2.796371in}}%
\pgfusepath{stroke}%
\end{pgfscope}%
\begin{pgfscope}%
\pgfpathrectangle{\pgfqpoint{0.100000in}{0.212622in}}{\pgfqpoint{3.696000in}{3.696000in}}%
\pgfusepath{clip}%
\pgfsetrectcap%
\pgfsetroundjoin%
\pgfsetlinewidth{1.505625pt}%
\definecolor{currentstroke}{rgb}{1.000000,0.000000,0.000000}%
\pgfsetstrokecolor{currentstroke}%
\pgfsetdash{}{0pt}%
\pgfpathmoveto{\pgfqpoint{2.228392in}{2.025924in}}%
\pgfpathlineto{\pgfqpoint{2.102498in}{2.792462in}}%
\pgfusepath{stroke}%
\end{pgfscope}%
\begin{pgfscope}%
\pgfpathrectangle{\pgfqpoint{0.100000in}{0.212622in}}{\pgfqpoint{3.696000in}{3.696000in}}%
\pgfusepath{clip}%
\pgfsetrectcap%
\pgfsetroundjoin%
\pgfsetlinewidth{1.505625pt}%
\definecolor{currentstroke}{rgb}{1.000000,0.000000,0.000000}%
\pgfsetstrokecolor{currentstroke}%
\pgfsetdash{}{0pt}%
\pgfpathmoveto{\pgfqpoint{2.238481in}{2.024365in}}%
\pgfpathlineto{\pgfqpoint{2.116939in}{2.788550in}}%
\pgfusepath{stroke}%
\end{pgfscope}%
\begin{pgfscope}%
\pgfpathrectangle{\pgfqpoint{0.100000in}{0.212622in}}{\pgfqpoint{3.696000in}{3.696000in}}%
\pgfusepath{clip}%
\pgfsetrectcap%
\pgfsetroundjoin%
\pgfsetlinewidth{1.505625pt}%
\definecolor{currentstroke}{rgb}{1.000000,0.000000,0.000000}%
\pgfsetstrokecolor{currentstroke}%
\pgfsetdash{}{0pt}%
\pgfpathmoveto{\pgfqpoint{2.247897in}{2.023320in}}%
\pgfpathlineto{\pgfqpoint{2.116939in}{2.788550in}}%
\pgfusepath{stroke}%
\end{pgfscope}%
\begin{pgfscope}%
\pgfpathrectangle{\pgfqpoint{0.100000in}{0.212622in}}{\pgfqpoint{3.696000in}{3.696000in}}%
\pgfusepath{clip}%
\pgfsetrectcap%
\pgfsetroundjoin%
\pgfsetlinewidth{1.505625pt}%
\definecolor{currentstroke}{rgb}{1.000000,0.000000,0.000000}%
\pgfsetstrokecolor{currentstroke}%
\pgfsetdash{}{0pt}%
\pgfpathmoveto{\pgfqpoint{2.258788in}{2.021546in}}%
\pgfpathlineto{\pgfqpoint{2.131390in}{2.784635in}}%
\pgfusepath{stroke}%
\end{pgfscope}%
\begin{pgfscope}%
\pgfpathrectangle{\pgfqpoint{0.100000in}{0.212622in}}{\pgfqpoint{3.696000in}{3.696000in}}%
\pgfusepath{clip}%
\pgfsetrectcap%
\pgfsetroundjoin%
\pgfsetlinewidth{1.505625pt}%
\definecolor{currentstroke}{rgb}{1.000000,0.000000,0.000000}%
\pgfsetstrokecolor{currentstroke}%
\pgfsetdash{}{0pt}%
\pgfpathmoveto{\pgfqpoint{2.265901in}{2.019949in}}%
\pgfpathlineto{\pgfqpoint{2.145851in}{2.780718in}}%
\pgfusepath{stroke}%
\end{pgfscope}%
\begin{pgfscope}%
\pgfpathrectangle{\pgfqpoint{0.100000in}{0.212622in}}{\pgfqpoint{3.696000in}{3.696000in}}%
\pgfusepath{clip}%
\pgfsetrectcap%
\pgfsetroundjoin%
\pgfsetlinewidth{1.505625pt}%
\definecolor{currentstroke}{rgb}{1.000000,0.000000,0.000000}%
\pgfsetstrokecolor{currentstroke}%
\pgfsetdash{}{0pt}%
\pgfpathmoveto{\pgfqpoint{2.272676in}{2.018998in}}%
\pgfpathlineto{\pgfqpoint{2.160323in}{2.776797in}}%
\pgfusepath{stroke}%
\end{pgfscope}%
\begin{pgfscope}%
\pgfpathrectangle{\pgfqpoint{0.100000in}{0.212622in}}{\pgfqpoint{3.696000in}{3.696000in}}%
\pgfusepath{clip}%
\pgfsetrectcap%
\pgfsetroundjoin%
\pgfsetlinewidth{1.505625pt}%
\definecolor{currentstroke}{rgb}{1.000000,0.000000,0.000000}%
\pgfsetstrokecolor{currentstroke}%
\pgfsetdash{}{0pt}%
\pgfpathmoveto{\pgfqpoint{2.280019in}{2.017667in}}%
\pgfpathlineto{\pgfqpoint{2.160323in}{2.776797in}}%
\pgfusepath{stroke}%
\end{pgfscope}%
\begin{pgfscope}%
\pgfpathrectangle{\pgfqpoint{0.100000in}{0.212622in}}{\pgfqpoint{3.696000in}{3.696000in}}%
\pgfusepath{clip}%
\pgfsetrectcap%
\pgfsetroundjoin%
\pgfsetlinewidth{1.505625pt}%
\definecolor{currentstroke}{rgb}{1.000000,0.000000,0.000000}%
\pgfsetstrokecolor{currentstroke}%
\pgfsetdash{}{0pt}%
\pgfpathmoveto{\pgfqpoint{2.285043in}{2.016194in}}%
\pgfpathlineto{\pgfqpoint{2.174805in}{2.772874in}}%
\pgfusepath{stroke}%
\end{pgfscope}%
\begin{pgfscope}%
\pgfpathrectangle{\pgfqpoint{0.100000in}{0.212622in}}{\pgfqpoint{3.696000in}{3.696000in}}%
\pgfusepath{clip}%
\pgfsetrectcap%
\pgfsetroundjoin%
\pgfsetlinewidth{1.505625pt}%
\definecolor{currentstroke}{rgb}{1.000000,0.000000,0.000000}%
\pgfsetstrokecolor{currentstroke}%
\pgfsetdash{}{0pt}%
\pgfpathmoveto{\pgfqpoint{2.291032in}{2.014850in}}%
\pgfpathlineto{\pgfqpoint{2.174805in}{2.772874in}}%
\pgfusepath{stroke}%
\end{pgfscope}%
\begin{pgfscope}%
\pgfpathrectangle{\pgfqpoint{0.100000in}{0.212622in}}{\pgfqpoint{3.696000in}{3.696000in}}%
\pgfusepath{clip}%
\pgfsetrectcap%
\pgfsetroundjoin%
\pgfsetlinewidth{1.505625pt}%
\definecolor{currentstroke}{rgb}{1.000000,0.000000,0.000000}%
\pgfsetstrokecolor{currentstroke}%
\pgfsetdash{}{0pt}%
\pgfpathmoveto{\pgfqpoint{2.294161in}{2.014330in}}%
\pgfpathlineto{\pgfqpoint{2.174805in}{2.772874in}}%
\pgfusepath{stroke}%
\end{pgfscope}%
\begin{pgfscope}%
\pgfpathrectangle{\pgfqpoint{0.100000in}{0.212622in}}{\pgfqpoint{3.696000in}{3.696000in}}%
\pgfusepath{clip}%
\pgfsetrectcap%
\pgfsetroundjoin%
\pgfsetlinewidth{1.505625pt}%
\definecolor{currentstroke}{rgb}{1.000000,0.000000,0.000000}%
\pgfsetstrokecolor{currentstroke}%
\pgfsetdash{}{0pt}%
\pgfpathmoveto{\pgfqpoint{2.297790in}{2.013417in}}%
\pgfpathlineto{\pgfqpoint{2.189298in}{2.768948in}}%
\pgfusepath{stroke}%
\end{pgfscope}%
\begin{pgfscope}%
\pgfpathrectangle{\pgfqpoint{0.100000in}{0.212622in}}{\pgfqpoint{3.696000in}{3.696000in}}%
\pgfusepath{clip}%
\pgfsetrectcap%
\pgfsetroundjoin%
\pgfsetlinewidth{1.505625pt}%
\definecolor{currentstroke}{rgb}{1.000000,0.000000,0.000000}%
\pgfsetstrokecolor{currentstroke}%
\pgfsetdash{}{0pt}%
\pgfpathmoveto{\pgfqpoint{2.301563in}{2.012519in}}%
\pgfpathlineto{\pgfqpoint{2.189298in}{2.768948in}}%
\pgfusepath{stroke}%
\end{pgfscope}%
\begin{pgfscope}%
\pgfpathrectangle{\pgfqpoint{0.100000in}{0.212622in}}{\pgfqpoint{3.696000in}{3.696000in}}%
\pgfusepath{clip}%
\pgfsetrectcap%
\pgfsetroundjoin%
\pgfsetlinewidth{1.505625pt}%
\definecolor{currentstroke}{rgb}{1.000000,0.000000,0.000000}%
\pgfsetstrokecolor{currentstroke}%
\pgfsetdash{}{0pt}%
\pgfpathmoveto{\pgfqpoint{2.306998in}{2.011357in}}%
\pgfpathlineto{\pgfqpoint{2.189298in}{2.768948in}}%
\pgfusepath{stroke}%
\end{pgfscope}%
\begin{pgfscope}%
\pgfpathrectangle{\pgfqpoint{0.100000in}{0.212622in}}{\pgfqpoint{3.696000in}{3.696000in}}%
\pgfusepath{clip}%
\pgfsetrectcap%
\pgfsetroundjoin%
\pgfsetlinewidth{1.505625pt}%
\definecolor{currentstroke}{rgb}{1.000000,0.000000,0.000000}%
\pgfsetstrokecolor{currentstroke}%
\pgfsetdash{}{0pt}%
\pgfpathmoveto{\pgfqpoint{2.313344in}{2.009481in}}%
\pgfpathlineto{\pgfqpoint{2.203801in}{2.765019in}}%
\pgfusepath{stroke}%
\end{pgfscope}%
\begin{pgfscope}%
\pgfpathrectangle{\pgfqpoint{0.100000in}{0.212622in}}{\pgfqpoint{3.696000in}{3.696000in}}%
\pgfusepath{clip}%
\pgfsetrectcap%
\pgfsetroundjoin%
\pgfsetlinewidth{1.505625pt}%
\definecolor{currentstroke}{rgb}{1.000000,0.000000,0.000000}%
\pgfsetstrokecolor{currentstroke}%
\pgfsetdash{}{0pt}%
\pgfpathmoveto{\pgfqpoint{2.319449in}{2.007977in}}%
\pgfpathlineto{\pgfqpoint{2.203801in}{2.765019in}}%
\pgfusepath{stroke}%
\end{pgfscope}%
\begin{pgfscope}%
\pgfpathrectangle{\pgfqpoint{0.100000in}{0.212622in}}{\pgfqpoint{3.696000in}{3.696000in}}%
\pgfusepath{clip}%
\pgfsetrectcap%
\pgfsetroundjoin%
\pgfsetlinewidth{1.505625pt}%
\definecolor{currentstroke}{rgb}{1.000000,0.000000,0.000000}%
\pgfsetstrokecolor{currentstroke}%
\pgfsetdash{}{0pt}%
\pgfpathmoveto{\pgfqpoint{2.322138in}{2.007824in}}%
\pgfpathlineto{\pgfqpoint{2.203801in}{2.765019in}}%
\pgfusepath{stroke}%
\end{pgfscope}%
\begin{pgfscope}%
\pgfpathrectangle{\pgfqpoint{0.100000in}{0.212622in}}{\pgfqpoint{3.696000in}{3.696000in}}%
\pgfusepath{clip}%
\pgfsetrectcap%
\pgfsetroundjoin%
\pgfsetlinewidth{1.505625pt}%
\definecolor{currentstroke}{rgb}{1.000000,0.000000,0.000000}%
\pgfsetstrokecolor{currentstroke}%
\pgfsetdash{}{0pt}%
\pgfpathmoveto{\pgfqpoint{2.326743in}{2.006669in}}%
\pgfpathlineto{\pgfqpoint{2.218315in}{2.761087in}}%
\pgfusepath{stroke}%
\end{pgfscope}%
\begin{pgfscope}%
\pgfpathrectangle{\pgfqpoint{0.100000in}{0.212622in}}{\pgfqpoint{3.696000in}{3.696000in}}%
\pgfusepath{clip}%
\pgfsetrectcap%
\pgfsetroundjoin%
\pgfsetlinewidth{1.505625pt}%
\definecolor{currentstroke}{rgb}{1.000000,0.000000,0.000000}%
\pgfsetstrokecolor{currentstroke}%
\pgfsetdash{}{0pt}%
\pgfpathmoveto{\pgfqpoint{2.331740in}{2.005329in}}%
\pgfpathlineto{\pgfqpoint{2.218315in}{2.761087in}}%
\pgfusepath{stroke}%
\end{pgfscope}%
\begin{pgfscope}%
\pgfpathrectangle{\pgfqpoint{0.100000in}{0.212622in}}{\pgfqpoint{3.696000in}{3.696000in}}%
\pgfusepath{clip}%
\pgfsetrectcap%
\pgfsetroundjoin%
\pgfsetlinewidth{1.505625pt}%
\definecolor{currentstroke}{rgb}{1.000000,0.000000,0.000000}%
\pgfsetstrokecolor{currentstroke}%
\pgfsetdash{}{0pt}%
\pgfpathmoveto{\pgfqpoint{2.334567in}{2.004746in}}%
\pgfpathlineto{\pgfqpoint{2.218315in}{2.761087in}}%
\pgfusepath{stroke}%
\end{pgfscope}%
\begin{pgfscope}%
\pgfpathrectangle{\pgfqpoint{0.100000in}{0.212622in}}{\pgfqpoint{3.696000in}{3.696000in}}%
\pgfusepath{clip}%
\pgfsetrectcap%
\pgfsetroundjoin%
\pgfsetlinewidth{1.505625pt}%
\definecolor{currentstroke}{rgb}{1.000000,0.000000,0.000000}%
\pgfsetstrokecolor{currentstroke}%
\pgfsetdash{}{0pt}%
\pgfpathmoveto{\pgfqpoint{2.337818in}{2.004034in}}%
\pgfpathlineto{\pgfqpoint{2.218315in}{2.761087in}}%
\pgfusepath{stroke}%
\end{pgfscope}%
\begin{pgfscope}%
\pgfpathrectangle{\pgfqpoint{0.100000in}{0.212622in}}{\pgfqpoint{3.696000in}{3.696000in}}%
\pgfusepath{clip}%
\pgfsetrectcap%
\pgfsetroundjoin%
\pgfsetlinewidth{1.505625pt}%
\definecolor{currentstroke}{rgb}{1.000000,0.000000,0.000000}%
\pgfsetstrokecolor{currentstroke}%
\pgfsetdash{}{0pt}%
\pgfpathmoveto{\pgfqpoint{2.339673in}{2.003628in}}%
\pgfpathlineto{\pgfqpoint{2.232839in}{2.757153in}}%
\pgfusepath{stroke}%
\end{pgfscope}%
\begin{pgfscope}%
\pgfpathrectangle{\pgfqpoint{0.100000in}{0.212622in}}{\pgfqpoint{3.696000in}{3.696000in}}%
\pgfusepath{clip}%
\pgfsetrectcap%
\pgfsetroundjoin%
\pgfsetlinewidth{1.505625pt}%
\definecolor{currentstroke}{rgb}{1.000000,0.000000,0.000000}%
\pgfsetstrokecolor{currentstroke}%
\pgfsetdash{}{0pt}%
\pgfpathmoveto{\pgfqpoint{2.341914in}{2.003254in}}%
\pgfpathlineto{\pgfqpoint{2.232839in}{2.757153in}}%
\pgfusepath{stroke}%
\end{pgfscope}%
\begin{pgfscope}%
\pgfpathrectangle{\pgfqpoint{0.100000in}{0.212622in}}{\pgfqpoint{3.696000in}{3.696000in}}%
\pgfusepath{clip}%
\pgfsetrectcap%
\pgfsetroundjoin%
\pgfsetlinewidth{1.505625pt}%
\definecolor{currentstroke}{rgb}{1.000000,0.000000,0.000000}%
\pgfsetstrokecolor{currentstroke}%
\pgfsetdash{}{0pt}%
\pgfpathmoveto{\pgfqpoint{2.344786in}{2.002628in}}%
\pgfpathlineto{\pgfqpoint{2.232839in}{2.757153in}}%
\pgfusepath{stroke}%
\end{pgfscope}%
\begin{pgfscope}%
\pgfpathrectangle{\pgfqpoint{0.100000in}{0.212622in}}{\pgfqpoint{3.696000in}{3.696000in}}%
\pgfusepath{clip}%
\pgfsetrectcap%
\pgfsetroundjoin%
\pgfsetlinewidth{1.505625pt}%
\definecolor{currentstroke}{rgb}{1.000000,0.000000,0.000000}%
\pgfsetstrokecolor{currentstroke}%
\pgfsetdash{}{0pt}%
\pgfpathmoveto{\pgfqpoint{2.348732in}{2.001868in}}%
\pgfpathlineto{\pgfqpoint{2.232839in}{2.757153in}}%
\pgfusepath{stroke}%
\end{pgfscope}%
\begin{pgfscope}%
\pgfpathrectangle{\pgfqpoint{0.100000in}{0.212622in}}{\pgfqpoint{3.696000in}{3.696000in}}%
\pgfusepath{clip}%
\pgfsetrectcap%
\pgfsetroundjoin%
\pgfsetlinewidth{1.505625pt}%
\definecolor{currentstroke}{rgb}{1.000000,0.000000,0.000000}%
\pgfsetstrokecolor{currentstroke}%
\pgfsetdash{}{0pt}%
\pgfpathmoveto{\pgfqpoint{2.353311in}{2.000982in}}%
\pgfpathlineto{\pgfqpoint{2.247374in}{2.753215in}}%
\pgfusepath{stroke}%
\end{pgfscope}%
\begin{pgfscope}%
\pgfpathrectangle{\pgfqpoint{0.100000in}{0.212622in}}{\pgfqpoint{3.696000in}{3.696000in}}%
\pgfusepath{clip}%
\pgfsetrectcap%
\pgfsetroundjoin%
\pgfsetlinewidth{1.505625pt}%
\definecolor{currentstroke}{rgb}{1.000000,0.000000,0.000000}%
\pgfsetstrokecolor{currentstroke}%
\pgfsetdash{}{0pt}%
\pgfpathmoveto{\pgfqpoint{2.358253in}{1.999759in}}%
\pgfpathlineto{\pgfqpoint{2.247374in}{2.753215in}}%
\pgfusepath{stroke}%
\end{pgfscope}%
\begin{pgfscope}%
\pgfpathrectangle{\pgfqpoint{0.100000in}{0.212622in}}{\pgfqpoint{3.696000in}{3.696000in}}%
\pgfusepath{clip}%
\pgfsetrectcap%
\pgfsetroundjoin%
\pgfsetlinewidth{1.505625pt}%
\definecolor{currentstroke}{rgb}{1.000000,0.000000,0.000000}%
\pgfsetstrokecolor{currentstroke}%
\pgfsetdash{}{0pt}%
\pgfpathmoveto{\pgfqpoint{2.365487in}{1.998340in}}%
\pgfpathlineto{\pgfqpoint{2.261919in}{2.749275in}}%
\pgfusepath{stroke}%
\end{pgfscope}%
\begin{pgfscope}%
\pgfpathrectangle{\pgfqpoint{0.100000in}{0.212622in}}{\pgfqpoint{3.696000in}{3.696000in}}%
\pgfusepath{clip}%
\pgfsetrectcap%
\pgfsetroundjoin%
\pgfsetlinewidth{1.505625pt}%
\definecolor{currentstroke}{rgb}{1.000000,0.000000,0.000000}%
\pgfsetstrokecolor{currentstroke}%
\pgfsetdash{}{0pt}%
\pgfpathmoveto{\pgfqpoint{2.375041in}{1.995375in}}%
\pgfpathlineto{\pgfqpoint{2.261919in}{2.749275in}}%
\pgfusepath{stroke}%
\end{pgfscope}%
\begin{pgfscope}%
\pgfpathrectangle{\pgfqpoint{0.100000in}{0.212622in}}{\pgfqpoint{3.696000in}{3.696000in}}%
\pgfusepath{clip}%
\pgfsetrectcap%
\pgfsetroundjoin%
\pgfsetlinewidth{1.505625pt}%
\definecolor{currentstroke}{rgb}{1.000000,0.000000,0.000000}%
\pgfsetstrokecolor{currentstroke}%
\pgfsetdash{}{0pt}%
\pgfpathmoveto{\pgfqpoint{2.380120in}{1.994000in}}%
\pgfpathlineto{\pgfqpoint{2.276475in}{2.745332in}}%
\pgfusepath{stroke}%
\end{pgfscope}%
\begin{pgfscope}%
\pgfpathrectangle{\pgfqpoint{0.100000in}{0.212622in}}{\pgfqpoint{3.696000in}{3.696000in}}%
\pgfusepath{clip}%
\pgfsetrectcap%
\pgfsetroundjoin%
\pgfsetlinewidth{1.505625pt}%
\definecolor{currentstroke}{rgb}{1.000000,0.000000,0.000000}%
\pgfsetstrokecolor{currentstroke}%
\pgfsetdash{}{0pt}%
\pgfpathmoveto{\pgfqpoint{2.382747in}{1.993446in}}%
\pgfpathlineto{\pgfqpoint{2.276475in}{2.745332in}}%
\pgfusepath{stroke}%
\end{pgfscope}%
\begin{pgfscope}%
\pgfpathrectangle{\pgfqpoint{0.100000in}{0.212622in}}{\pgfqpoint{3.696000in}{3.696000in}}%
\pgfusepath{clip}%
\pgfsetrectcap%
\pgfsetroundjoin%
\pgfsetlinewidth{1.505625pt}%
\definecolor{currentstroke}{rgb}{1.000000,0.000000,0.000000}%
\pgfsetstrokecolor{currentstroke}%
\pgfsetdash{}{0pt}%
\pgfpathmoveto{\pgfqpoint{2.386486in}{1.992509in}}%
\pgfpathlineto{\pgfqpoint{2.276475in}{2.745332in}}%
\pgfusepath{stroke}%
\end{pgfscope}%
\begin{pgfscope}%
\pgfpathrectangle{\pgfqpoint{0.100000in}{0.212622in}}{\pgfqpoint{3.696000in}{3.696000in}}%
\pgfusepath{clip}%
\pgfsetrectcap%
\pgfsetroundjoin%
\pgfsetlinewidth{1.505625pt}%
\definecolor{currentstroke}{rgb}{1.000000,0.000000,0.000000}%
\pgfsetstrokecolor{currentstroke}%
\pgfsetdash{}{0pt}%
\pgfpathmoveto{\pgfqpoint{2.391040in}{1.990851in}}%
\pgfpathlineto{\pgfqpoint{2.276475in}{2.745332in}}%
\pgfusepath{stroke}%
\end{pgfscope}%
\begin{pgfscope}%
\pgfpathrectangle{\pgfqpoint{0.100000in}{0.212622in}}{\pgfqpoint{3.696000in}{3.696000in}}%
\pgfusepath{clip}%
\pgfsetrectcap%
\pgfsetroundjoin%
\pgfsetlinewidth{1.505625pt}%
\definecolor{currentstroke}{rgb}{1.000000,0.000000,0.000000}%
\pgfsetstrokecolor{currentstroke}%
\pgfsetdash{}{0pt}%
\pgfpathmoveto{\pgfqpoint{2.396619in}{1.989523in}}%
\pgfpathlineto{\pgfqpoint{2.291041in}{2.741386in}}%
\pgfusepath{stroke}%
\end{pgfscope}%
\begin{pgfscope}%
\pgfpathrectangle{\pgfqpoint{0.100000in}{0.212622in}}{\pgfqpoint{3.696000in}{3.696000in}}%
\pgfusepath{clip}%
\pgfsetrectcap%
\pgfsetroundjoin%
\pgfsetlinewidth{1.505625pt}%
\definecolor{currentstroke}{rgb}{1.000000,0.000000,0.000000}%
\pgfsetstrokecolor{currentstroke}%
\pgfsetdash{}{0pt}%
\pgfpathmoveto{\pgfqpoint{2.399854in}{1.988603in}}%
\pgfpathlineto{\pgfqpoint{2.291041in}{2.741386in}}%
\pgfusepath{stroke}%
\end{pgfscope}%
\begin{pgfscope}%
\pgfpathrectangle{\pgfqpoint{0.100000in}{0.212622in}}{\pgfqpoint{3.696000in}{3.696000in}}%
\pgfusepath{clip}%
\pgfsetrectcap%
\pgfsetroundjoin%
\pgfsetlinewidth{1.505625pt}%
\definecolor{currentstroke}{rgb}{1.000000,0.000000,0.000000}%
\pgfsetstrokecolor{currentstroke}%
\pgfsetdash{}{0pt}%
\pgfpathmoveto{\pgfqpoint{2.403667in}{1.987578in}}%
\pgfpathlineto{\pgfqpoint{2.291041in}{2.741386in}}%
\pgfusepath{stroke}%
\end{pgfscope}%
\begin{pgfscope}%
\pgfpathrectangle{\pgfqpoint{0.100000in}{0.212622in}}{\pgfqpoint{3.696000in}{3.696000in}}%
\pgfusepath{clip}%
\pgfsetrectcap%
\pgfsetroundjoin%
\pgfsetlinewidth{1.505625pt}%
\definecolor{currentstroke}{rgb}{1.000000,0.000000,0.000000}%
\pgfsetstrokecolor{currentstroke}%
\pgfsetdash{}{0pt}%
\pgfpathmoveto{\pgfqpoint{2.405630in}{1.987232in}}%
\pgfpathlineto{\pgfqpoint{2.291041in}{2.741386in}}%
\pgfusepath{stroke}%
\end{pgfscope}%
\begin{pgfscope}%
\pgfpathrectangle{\pgfqpoint{0.100000in}{0.212622in}}{\pgfqpoint{3.696000in}{3.696000in}}%
\pgfusepath{clip}%
\pgfsetrectcap%
\pgfsetroundjoin%
\pgfsetlinewidth{1.505625pt}%
\definecolor{currentstroke}{rgb}{1.000000,0.000000,0.000000}%
\pgfsetstrokecolor{currentstroke}%
\pgfsetdash{}{0pt}%
\pgfpathmoveto{\pgfqpoint{2.409132in}{1.985998in}}%
\pgfpathlineto{\pgfqpoint{2.305618in}{2.737437in}}%
\pgfusepath{stroke}%
\end{pgfscope}%
\begin{pgfscope}%
\pgfpathrectangle{\pgfqpoint{0.100000in}{0.212622in}}{\pgfqpoint{3.696000in}{3.696000in}}%
\pgfusepath{clip}%
\pgfsetrectcap%
\pgfsetroundjoin%
\pgfsetlinewidth{1.505625pt}%
\definecolor{currentstroke}{rgb}{1.000000,0.000000,0.000000}%
\pgfsetstrokecolor{currentstroke}%
\pgfsetdash{}{0pt}%
\pgfpathmoveto{\pgfqpoint{2.410999in}{1.985503in}}%
\pgfpathlineto{\pgfqpoint{2.305618in}{2.737437in}}%
\pgfusepath{stroke}%
\end{pgfscope}%
\begin{pgfscope}%
\pgfpathrectangle{\pgfqpoint{0.100000in}{0.212622in}}{\pgfqpoint{3.696000in}{3.696000in}}%
\pgfusepath{clip}%
\pgfsetrectcap%
\pgfsetroundjoin%
\pgfsetlinewidth{1.505625pt}%
\definecolor{currentstroke}{rgb}{1.000000,0.000000,0.000000}%
\pgfsetstrokecolor{currentstroke}%
\pgfsetdash{}{0pt}%
\pgfpathmoveto{\pgfqpoint{2.411840in}{1.985399in}}%
\pgfpathlineto{\pgfqpoint{2.305618in}{2.737437in}}%
\pgfusepath{stroke}%
\end{pgfscope}%
\begin{pgfscope}%
\pgfpathrectangle{\pgfqpoint{0.100000in}{0.212622in}}{\pgfqpoint{3.696000in}{3.696000in}}%
\pgfusepath{clip}%
\pgfsetrectcap%
\pgfsetroundjoin%
\pgfsetlinewidth{1.505625pt}%
\definecolor{currentstroke}{rgb}{1.000000,0.000000,0.000000}%
\pgfsetstrokecolor{currentstroke}%
\pgfsetdash{}{0pt}%
\pgfpathmoveto{\pgfqpoint{2.413939in}{1.984892in}}%
\pgfpathlineto{\pgfqpoint{2.305618in}{2.737437in}}%
\pgfusepath{stroke}%
\end{pgfscope}%
\begin{pgfscope}%
\pgfpathrectangle{\pgfqpoint{0.100000in}{0.212622in}}{\pgfqpoint{3.696000in}{3.696000in}}%
\pgfusepath{clip}%
\pgfsetrectcap%
\pgfsetroundjoin%
\pgfsetlinewidth{1.505625pt}%
\definecolor{currentstroke}{rgb}{1.000000,0.000000,0.000000}%
\pgfsetstrokecolor{currentstroke}%
\pgfsetdash{}{0pt}%
\pgfpathmoveto{\pgfqpoint{2.416571in}{1.984124in}}%
\pgfpathlineto{\pgfqpoint{2.305618in}{2.737437in}}%
\pgfusepath{stroke}%
\end{pgfscope}%
\begin{pgfscope}%
\pgfpathrectangle{\pgfqpoint{0.100000in}{0.212622in}}{\pgfqpoint{3.696000in}{3.696000in}}%
\pgfusepath{clip}%
\pgfsetrectcap%
\pgfsetroundjoin%
\pgfsetlinewidth{1.505625pt}%
\definecolor{currentstroke}{rgb}{1.000000,0.000000,0.000000}%
\pgfsetstrokecolor{currentstroke}%
\pgfsetdash{}{0pt}%
\pgfpathmoveto{\pgfqpoint{2.417911in}{1.983844in}}%
\pgfpathlineto{\pgfqpoint{2.305618in}{2.737437in}}%
\pgfusepath{stroke}%
\end{pgfscope}%
\begin{pgfscope}%
\pgfpathrectangle{\pgfqpoint{0.100000in}{0.212622in}}{\pgfqpoint{3.696000in}{3.696000in}}%
\pgfusepath{clip}%
\pgfsetrectcap%
\pgfsetroundjoin%
\pgfsetlinewidth{1.505625pt}%
\definecolor{currentstroke}{rgb}{1.000000,0.000000,0.000000}%
\pgfsetstrokecolor{currentstroke}%
\pgfsetdash{}{0pt}%
\pgfpathmoveto{\pgfqpoint{2.420471in}{1.983246in}}%
\pgfpathlineto{\pgfqpoint{2.305618in}{2.737437in}}%
\pgfusepath{stroke}%
\end{pgfscope}%
\begin{pgfscope}%
\pgfpathrectangle{\pgfqpoint{0.100000in}{0.212622in}}{\pgfqpoint{3.696000in}{3.696000in}}%
\pgfusepath{clip}%
\pgfsetrectcap%
\pgfsetroundjoin%
\pgfsetlinewidth{1.505625pt}%
\definecolor{currentstroke}{rgb}{1.000000,0.000000,0.000000}%
\pgfsetstrokecolor{currentstroke}%
\pgfsetdash{}{0pt}%
\pgfpathmoveto{\pgfqpoint{2.421844in}{1.982940in}}%
\pgfpathlineto{\pgfqpoint{2.305618in}{2.737437in}}%
\pgfusepath{stroke}%
\end{pgfscope}%
\begin{pgfscope}%
\pgfpathrectangle{\pgfqpoint{0.100000in}{0.212622in}}{\pgfqpoint{3.696000in}{3.696000in}}%
\pgfusepath{clip}%
\pgfsetrectcap%
\pgfsetroundjoin%
\pgfsetlinewidth{1.505625pt}%
\definecolor{currentstroke}{rgb}{1.000000,0.000000,0.000000}%
\pgfsetstrokecolor{currentstroke}%
\pgfsetdash{}{0pt}%
\pgfpathmoveto{\pgfqpoint{2.422668in}{1.982717in}}%
\pgfpathlineto{\pgfqpoint{2.320205in}{2.733485in}}%
\pgfusepath{stroke}%
\end{pgfscope}%
\begin{pgfscope}%
\pgfpathrectangle{\pgfqpoint{0.100000in}{0.212622in}}{\pgfqpoint{3.696000in}{3.696000in}}%
\pgfusepath{clip}%
\pgfsetrectcap%
\pgfsetroundjoin%
\pgfsetlinewidth{1.505625pt}%
\definecolor{currentstroke}{rgb}{1.000000,0.000000,0.000000}%
\pgfsetstrokecolor{currentstroke}%
\pgfsetdash{}{0pt}%
\pgfpathmoveto{\pgfqpoint{2.424468in}{1.982320in}}%
\pgfpathlineto{\pgfqpoint{2.320205in}{2.733485in}}%
\pgfusepath{stroke}%
\end{pgfscope}%
\begin{pgfscope}%
\pgfpathrectangle{\pgfqpoint{0.100000in}{0.212622in}}{\pgfqpoint{3.696000in}{3.696000in}}%
\pgfusepath{clip}%
\pgfsetrectcap%
\pgfsetroundjoin%
\pgfsetlinewidth{1.505625pt}%
\definecolor{currentstroke}{rgb}{1.000000,0.000000,0.000000}%
\pgfsetstrokecolor{currentstroke}%
\pgfsetdash{}{0pt}%
\pgfpathmoveto{\pgfqpoint{2.425399in}{1.982120in}}%
\pgfpathlineto{\pgfqpoint{2.320205in}{2.733485in}}%
\pgfusepath{stroke}%
\end{pgfscope}%
\begin{pgfscope}%
\pgfpathrectangle{\pgfqpoint{0.100000in}{0.212622in}}{\pgfqpoint{3.696000in}{3.696000in}}%
\pgfusepath{clip}%
\pgfsetrectcap%
\pgfsetroundjoin%
\pgfsetlinewidth{1.505625pt}%
\definecolor{currentstroke}{rgb}{1.000000,0.000000,0.000000}%
\pgfsetstrokecolor{currentstroke}%
\pgfsetdash{}{0pt}%
\pgfpathmoveto{\pgfqpoint{2.426904in}{1.981639in}}%
\pgfpathlineto{\pgfqpoint{2.320205in}{2.733485in}}%
\pgfusepath{stroke}%
\end{pgfscope}%
\begin{pgfscope}%
\pgfpathrectangle{\pgfqpoint{0.100000in}{0.212622in}}{\pgfqpoint{3.696000in}{3.696000in}}%
\pgfusepath{clip}%
\pgfsetrectcap%
\pgfsetroundjoin%
\pgfsetlinewidth{1.505625pt}%
\definecolor{currentstroke}{rgb}{1.000000,0.000000,0.000000}%
\pgfsetstrokecolor{currentstroke}%
\pgfsetdash{}{0pt}%
\pgfpathmoveto{\pgfqpoint{2.429070in}{1.981129in}}%
\pgfpathlineto{\pgfqpoint{2.320205in}{2.733485in}}%
\pgfusepath{stroke}%
\end{pgfscope}%
\begin{pgfscope}%
\pgfpathrectangle{\pgfqpoint{0.100000in}{0.212622in}}{\pgfqpoint{3.696000in}{3.696000in}}%
\pgfusepath{clip}%
\pgfsetrectcap%
\pgfsetroundjoin%
\pgfsetlinewidth{1.505625pt}%
\definecolor{currentstroke}{rgb}{1.000000,0.000000,0.000000}%
\pgfsetstrokecolor{currentstroke}%
\pgfsetdash{}{0pt}%
\pgfpathmoveto{\pgfqpoint{2.430328in}{1.980845in}}%
\pgfpathlineto{\pgfqpoint{2.320205in}{2.733485in}}%
\pgfusepath{stroke}%
\end{pgfscope}%
\begin{pgfscope}%
\pgfpathrectangle{\pgfqpoint{0.100000in}{0.212622in}}{\pgfqpoint{3.696000in}{3.696000in}}%
\pgfusepath{clip}%
\pgfsetrectcap%
\pgfsetroundjoin%
\pgfsetlinewidth{1.505625pt}%
\definecolor{currentstroke}{rgb}{1.000000,0.000000,0.000000}%
\pgfsetstrokecolor{currentstroke}%
\pgfsetdash{}{0pt}%
\pgfpathmoveto{\pgfqpoint{2.432477in}{1.980221in}}%
\pgfpathlineto{\pgfqpoint{2.320205in}{2.733485in}}%
\pgfusepath{stroke}%
\end{pgfscope}%
\begin{pgfscope}%
\pgfpathrectangle{\pgfqpoint{0.100000in}{0.212622in}}{\pgfqpoint{3.696000in}{3.696000in}}%
\pgfusepath{clip}%
\pgfsetrectcap%
\pgfsetroundjoin%
\pgfsetlinewidth{1.505625pt}%
\definecolor{currentstroke}{rgb}{1.000000,0.000000,0.000000}%
\pgfsetstrokecolor{currentstroke}%
\pgfsetdash{}{0pt}%
\pgfpathmoveto{\pgfqpoint{2.436834in}{1.979268in}}%
\pgfpathlineto{\pgfqpoint{2.320205in}{2.733485in}}%
\pgfusepath{stroke}%
\end{pgfscope}%
\begin{pgfscope}%
\pgfpathrectangle{\pgfqpoint{0.100000in}{0.212622in}}{\pgfqpoint{3.696000in}{3.696000in}}%
\pgfusepath{clip}%
\pgfsetrectcap%
\pgfsetroundjoin%
\pgfsetlinewidth{1.505625pt}%
\definecolor{currentstroke}{rgb}{1.000000,0.000000,0.000000}%
\pgfsetstrokecolor{currentstroke}%
\pgfsetdash{}{0pt}%
\pgfpathmoveto{\pgfqpoint{2.441587in}{1.978115in}}%
\pgfpathlineto{\pgfqpoint{2.334803in}{2.729531in}}%
\pgfusepath{stroke}%
\end{pgfscope}%
\begin{pgfscope}%
\pgfpathrectangle{\pgfqpoint{0.100000in}{0.212622in}}{\pgfqpoint{3.696000in}{3.696000in}}%
\pgfusepath{clip}%
\pgfsetrectcap%
\pgfsetroundjoin%
\pgfsetlinewidth{1.505625pt}%
\definecolor{currentstroke}{rgb}{1.000000,0.000000,0.000000}%
\pgfsetstrokecolor{currentstroke}%
\pgfsetdash{}{0pt}%
\pgfpathmoveto{\pgfqpoint{2.444416in}{1.977447in}}%
\pgfpathlineto{\pgfqpoint{2.334803in}{2.729531in}}%
\pgfusepath{stroke}%
\end{pgfscope}%
\begin{pgfscope}%
\pgfpathrectangle{\pgfqpoint{0.100000in}{0.212622in}}{\pgfqpoint{3.696000in}{3.696000in}}%
\pgfusepath{clip}%
\pgfsetrectcap%
\pgfsetroundjoin%
\pgfsetlinewidth{1.505625pt}%
\definecolor{currentstroke}{rgb}{1.000000,0.000000,0.000000}%
\pgfsetstrokecolor{currentstroke}%
\pgfsetdash{}{0pt}%
\pgfpathmoveto{\pgfqpoint{2.448066in}{1.976568in}}%
\pgfpathlineto{\pgfqpoint{2.334803in}{2.729531in}}%
\pgfusepath{stroke}%
\end{pgfscope}%
\begin{pgfscope}%
\pgfpathrectangle{\pgfqpoint{0.100000in}{0.212622in}}{\pgfqpoint{3.696000in}{3.696000in}}%
\pgfusepath{clip}%
\pgfsetrectcap%
\pgfsetroundjoin%
\pgfsetlinewidth{1.505625pt}%
\definecolor{currentstroke}{rgb}{1.000000,0.000000,0.000000}%
\pgfsetstrokecolor{currentstroke}%
\pgfsetdash{}{0pt}%
\pgfpathmoveto{\pgfqpoint{2.449928in}{1.976040in}}%
\pgfpathlineto{\pgfqpoint{2.334803in}{2.729531in}}%
\pgfusepath{stroke}%
\end{pgfscope}%
\begin{pgfscope}%
\pgfpathrectangle{\pgfqpoint{0.100000in}{0.212622in}}{\pgfqpoint{3.696000in}{3.696000in}}%
\pgfusepath{clip}%
\pgfsetrectcap%
\pgfsetroundjoin%
\pgfsetlinewidth{1.505625pt}%
\definecolor{currentstroke}{rgb}{1.000000,0.000000,0.000000}%
\pgfsetstrokecolor{currentstroke}%
\pgfsetdash{}{0pt}%
\pgfpathmoveto{\pgfqpoint{2.452870in}{1.975116in}}%
\pgfpathlineto{\pgfqpoint{2.349412in}{2.725573in}}%
\pgfusepath{stroke}%
\end{pgfscope}%
\begin{pgfscope}%
\pgfpathrectangle{\pgfqpoint{0.100000in}{0.212622in}}{\pgfqpoint{3.696000in}{3.696000in}}%
\pgfusepath{clip}%
\pgfsetrectcap%
\pgfsetroundjoin%
\pgfsetlinewidth{1.505625pt}%
\definecolor{currentstroke}{rgb}{1.000000,0.000000,0.000000}%
\pgfsetstrokecolor{currentstroke}%
\pgfsetdash{}{0pt}%
\pgfpathmoveto{\pgfqpoint{2.457843in}{1.973634in}}%
\pgfpathlineto{\pgfqpoint{2.349412in}{2.725573in}}%
\pgfusepath{stroke}%
\end{pgfscope}%
\begin{pgfscope}%
\pgfpathrectangle{\pgfqpoint{0.100000in}{0.212622in}}{\pgfqpoint{3.696000in}{3.696000in}}%
\pgfusepath{clip}%
\pgfsetrectcap%
\pgfsetroundjoin%
\pgfsetlinewidth{1.505625pt}%
\definecolor{currentstroke}{rgb}{1.000000,0.000000,0.000000}%
\pgfsetstrokecolor{currentstroke}%
\pgfsetdash{}{0pt}%
\pgfpathmoveto{\pgfqpoint{2.464302in}{1.971209in}}%
\pgfpathlineto{\pgfqpoint{2.349412in}{2.725573in}}%
\pgfusepath{stroke}%
\end{pgfscope}%
\begin{pgfscope}%
\pgfpathrectangle{\pgfqpoint{0.100000in}{0.212622in}}{\pgfqpoint{3.696000in}{3.696000in}}%
\pgfusepath{clip}%
\pgfsetrectcap%
\pgfsetroundjoin%
\pgfsetlinewidth{1.505625pt}%
\definecolor{currentstroke}{rgb}{1.000000,0.000000,0.000000}%
\pgfsetstrokecolor{currentstroke}%
\pgfsetdash{}{0pt}%
\pgfpathmoveto{\pgfqpoint{2.471184in}{1.969303in}}%
\pgfpathlineto{\pgfqpoint{2.364031in}{2.721613in}}%
\pgfusepath{stroke}%
\end{pgfscope}%
\begin{pgfscope}%
\pgfpathrectangle{\pgfqpoint{0.100000in}{0.212622in}}{\pgfqpoint{3.696000in}{3.696000in}}%
\pgfusepath{clip}%
\pgfsetrectcap%
\pgfsetroundjoin%
\pgfsetlinewidth{1.505625pt}%
\definecolor{currentstroke}{rgb}{1.000000,0.000000,0.000000}%
\pgfsetstrokecolor{currentstroke}%
\pgfsetdash{}{0pt}%
\pgfpathmoveto{\pgfqpoint{2.478809in}{1.967037in}}%
\pgfpathlineto{\pgfqpoint{2.364031in}{2.721613in}}%
\pgfusepath{stroke}%
\end{pgfscope}%
\begin{pgfscope}%
\pgfpathrectangle{\pgfqpoint{0.100000in}{0.212622in}}{\pgfqpoint{3.696000in}{3.696000in}}%
\pgfusepath{clip}%
\pgfsetrectcap%
\pgfsetroundjoin%
\pgfsetlinewidth{1.505625pt}%
\definecolor{currentstroke}{rgb}{1.000000,0.000000,0.000000}%
\pgfsetstrokecolor{currentstroke}%
\pgfsetdash{}{0pt}%
\pgfpathmoveto{\pgfqpoint{2.487349in}{1.964226in}}%
\pgfpathlineto{\pgfqpoint{2.378661in}{2.717650in}}%
\pgfusepath{stroke}%
\end{pgfscope}%
\begin{pgfscope}%
\pgfpathrectangle{\pgfqpoint{0.100000in}{0.212622in}}{\pgfqpoint{3.696000in}{3.696000in}}%
\pgfusepath{clip}%
\pgfsetrectcap%
\pgfsetroundjoin%
\pgfsetlinewidth{1.505625pt}%
\definecolor{currentstroke}{rgb}{1.000000,0.000000,0.000000}%
\pgfsetstrokecolor{currentstroke}%
\pgfsetdash{}{0pt}%
\pgfpathmoveto{\pgfqpoint{2.495773in}{1.962082in}}%
\pgfpathlineto{\pgfqpoint{2.393301in}{2.713684in}}%
\pgfusepath{stroke}%
\end{pgfscope}%
\begin{pgfscope}%
\pgfpathrectangle{\pgfqpoint{0.100000in}{0.212622in}}{\pgfqpoint{3.696000in}{3.696000in}}%
\pgfusepath{clip}%
\pgfsetrectcap%
\pgfsetroundjoin%
\pgfsetlinewidth{1.505625pt}%
\definecolor{currentstroke}{rgb}{1.000000,0.000000,0.000000}%
\pgfsetstrokecolor{currentstroke}%
\pgfsetdash{}{0pt}%
\pgfpathmoveto{\pgfqpoint{2.506893in}{1.958518in}}%
\pgfpathlineto{\pgfqpoint{2.393301in}{2.713684in}}%
\pgfusepath{stroke}%
\end{pgfscope}%
\begin{pgfscope}%
\pgfpathrectangle{\pgfqpoint{0.100000in}{0.212622in}}{\pgfqpoint{3.696000in}{3.696000in}}%
\pgfusepath{clip}%
\pgfsetrectcap%
\pgfsetroundjoin%
\pgfsetlinewidth{1.505625pt}%
\definecolor{currentstroke}{rgb}{1.000000,0.000000,0.000000}%
\pgfsetstrokecolor{currentstroke}%
\pgfsetdash{}{0pt}%
\pgfpathmoveto{\pgfqpoint{2.518536in}{1.954674in}}%
\pgfpathlineto{\pgfqpoint{2.407952in}{2.709715in}}%
\pgfusepath{stroke}%
\end{pgfscope}%
\begin{pgfscope}%
\pgfpathrectangle{\pgfqpoint{0.100000in}{0.212622in}}{\pgfqpoint{3.696000in}{3.696000in}}%
\pgfusepath{clip}%
\pgfsetrectcap%
\pgfsetroundjoin%
\pgfsetlinewidth{1.505625pt}%
\definecolor{currentstroke}{rgb}{1.000000,0.000000,0.000000}%
\pgfsetstrokecolor{currentstroke}%
\pgfsetdash{}{0pt}%
\pgfpathmoveto{\pgfqpoint{2.532498in}{1.950724in}}%
\pgfpathlineto{\pgfqpoint{2.422613in}{2.705743in}}%
\pgfusepath{stroke}%
\end{pgfscope}%
\begin{pgfscope}%
\pgfpathrectangle{\pgfqpoint{0.100000in}{0.212622in}}{\pgfqpoint{3.696000in}{3.696000in}}%
\pgfusepath{clip}%
\pgfsetrectcap%
\pgfsetroundjoin%
\pgfsetlinewidth{1.505625pt}%
\definecolor{currentstroke}{rgb}{1.000000,0.000000,0.000000}%
\pgfsetstrokecolor{currentstroke}%
\pgfsetdash{}{0pt}%
\pgfpathmoveto{\pgfqpoint{2.539406in}{1.948884in}}%
\pgfpathlineto{\pgfqpoint{2.422613in}{2.705743in}}%
\pgfusepath{stroke}%
\end{pgfscope}%
\begin{pgfscope}%
\pgfpathrectangle{\pgfqpoint{0.100000in}{0.212622in}}{\pgfqpoint{3.696000in}{3.696000in}}%
\pgfusepath{clip}%
\pgfsetrectcap%
\pgfsetroundjoin%
\pgfsetlinewidth{1.505625pt}%
\definecolor{currentstroke}{rgb}{1.000000,0.000000,0.000000}%
\pgfsetstrokecolor{currentstroke}%
\pgfsetdash{}{0pt}%
\pgfpathmoveto{\pgfqpoint{2.547104in}{1.946810in}}%
\pgfpathlineto{\pgfqpoint{2.437286in}{2.701768in}}%
\pgfusepath{stroke}%
\end{pgfscope}%
\begin{pgfscope}%
\pgfpathrectangle{\pgfqpoint{0.100000in}{0.212622in}}{\pgfqpoint{3.696000in}{3.696000in}}%
\pgfusepath{clip}%
\pgfsetrectcap%
\pgfsetroundjoin%
\pgfsetlinewidth{1.505625pt}%
\definecolor{currentstroke}{rgb}{1.000000,0.000000,0.000000}%
\pgfsetstrokecolor{currentstroke}%
\pgfsetdash{}{0pt}%
\pgfpathmoveto{\pgfqpoint{2.550885in}{1.946096in}}%
\pgfpathlineto{\pgfqpoint{2.437286in}{2.701768in}}%
\pgfusepath{stroke}%
\end{pgfscope}%
\begin{pgfscope}%
\pgfpathrectangle{\pgfqpoint{0.100000in}{0.212622in}}{\pgfqpoint{3.696000in}{3.696000in}}%
\pgfusepath{clip}%
\pgfsetrectcap%
\pgfsetroundjoin%
\pgfsetlinewidth{1.505625pt}%
\definecolor{currentstroke}{rgb}{1.000000,0.000000,0.000000}%
\pgfsetstrokecolor{currentstroke}%
\pgfsetdash{}{0pt}%
\pgfpathmoveto{\pgfqpoint{2.556724in}{1.944525in}}%
\pgfpathlineto{\pgfqpoint{2.451969in}{2.697791in}}%
\pgfusepath{stroke}%
\end{pgfscope}%
\begin{pgfscope}%
\pgfpathrectangle{\pgfqpoint{0.100000in}{0.212622in}}{\pgfqpoint{3.696000in}{3.696000in}}%
\pgfusepath{clip}%
\pgfsetrectcap%
\pgfsetroundjoin%
\pgfsetlinewidth{1.505625pt}%
\definecolor{currentstroke}{rgb}{1.000000,0.000000,0.000000}%
\pgfsetstrokecolor{currentstroke}%
\pgfsetdash{}{0pt}%
\pgfpathmoveto{\pgfqpoint{2.559957in}{1.943557in}}%
\pgfpathlineto{\pgfqpoint{2.451969in}{2.697791in}}%
\pgfusepath{stroke}%
\end{pgfscope}%
\begin{pgfscope}%
\pgfpathrectangle{\pgfqpoint{0.100000in}{0.212622in}}{\pgfqpoint{3.696000in}{3.696000in}}%
\pgfusepath{clip}%
\pgfsetrectcap%
\pgfsetroundjoin%
\pgfsetlinewidth{1.505625pt}%
\definecolor{currentstroke}{rgb}{1.000000,0.000000,0.000000}%
\pgfsetstrokecolor{currentstroke}%
\pgfsetdash{}{0pt}%
\pgfpathmoveto{\pgfqpoint{2.564120in}{1.942593in}}%
\pgfpathlineto{\pgfqpoint{2.451969in}{2.697791in}}%
\pgfusepath{stroke}%
\end{pgfscope}%
\begin{pgfscope}%
\pgfpathrectangle{\pgfqpoint{0.100000in}{0.212622in}}{\pgfqpoint{3.696000in}{3.696000in}}%
\pgfusepath{clip}%
\pgfsetrectcap%
\pgfsetroundjoin%
\pgfsetlinewidth{1.505625pt}%
\definecolor{currentstroke}{rgb}{1.000000,0.000000,0.000000}%
\pgfsetstrokecolor{currentstroke}%
\pgfsetdash{}{0pt}%
\pgfpathmoveto{\pgfqpoint{2.566589in}{1.941963in}}%
\pgfpathlineto{\pgfqpoint{2.451969in}{2.697791in}}%
\pgfusepath{stroke}%
\end{pgfscope}%
\begin{pgfscope}%
\pgfpathrectangle{\pgfqpoint{0.100000in}{0.212622in}}{\pgfqpoint{3.696000in}{3.696000in}}%
\pgfusepath{clip}%
\pgfsetrectcap%
\pgfsetroundjoin%
\pgfsetlinewidth{1.505625pt}%
\definecolor{currentstroke}{rgb}{1.000000,0.000000,0.000000}%
\pgfsetstrokecolor{currentstroke}%
\pgfsetdash{}{0pt}%
\pgfpathmoveto{\pgfqpoint{2.569838in}{1.941245in}}%
\pgfpathlineto{\pgfqpoint{2.466662in}{2.693810in}}%
\pgfusepath{stroke}%
\end{pgfscope}%
\begin{pgfscope}%
\pgfpathrectangle{\pgfqpoint{0.100000in}{0.212622in}}{\pgfqpoint{3.696000in}{3.696000in}}%
\pgfusepath{clip}%
\pgfsetrectcap%
\pgfsetroundjoin%
\pgfsetlinewidth{1.505625pt}%
\definecolor{currentstroke}{rgb}{1.000000,0.000000,0.000000}%
\pgfsetstrokecolor{currentstroke}%
\pgfsetdash{}{0pt}%
\pgfpathmoveto{\pgfqpoint{2.574662in}{1.940294in}}%
\pgfpathlineto{\pgfqpoint{2.466662in}{2.693810in}}%
\pgfusepath{stroke}%
\end{pgfscope}%
\begin{pgfscope}%
\pgfpathrectangle{\pgfqpoint{0.100000in}{0.212622in}}{\pgfqpoint{3.696000in}{3.696000in}}%
\pgfusepath{clip}%
\pgfsetrectcap%
\pgfsetroundjoin%
\pgfsetlinewidth{1.505625pt}%
\definecolor{currentstroke}{rgb}{1.000000,0.000000,0.000000}%
\pgfsetstrokecolor{currentstroke}%
\pgfsetdash{}{0pt}%
\pgfpathmoveto{\pgfqpoint{2.580428in}{1.938849in}}%
\pgfpathlineto{\pgfqpoint{2.466662in}{2.693810in}}%
\pgfusepath{stroke}%
\end{pgfscope}%
\begin{pgfscope}%
\pgfpathrectangle{\pgfqpoint{0.100000in}{0.212622in}}{\pgfqpoint{3.696000in}{3.696000in}}%
\pgfusepath{clip}%
\pgfsetrectcap%
\pgfsetroundjoin%
\pgfsetlinewidth{1.505625pt}%
\definecolor{currentstroke}{rgb}{1.000000,0.000000,0.000000}%
\pgfsetstrokecolor{currentstroke}%
\pgfsetdash{}{0pt}%
\pgfpathmoveto{\pgfqpoint{2.586209in}{1.938007in}}%
\pgfpathlineto{\pgfqpoint{2.481367in}{2.689827in}}%
\pgfusepath{stroke}%
\end{pgfscope}%
\begin{pgfscope}%
\pgfpathrectangle{\pgfqpoint{0.100000in}{0.212622in}}{\pgfqpoint{3.696000in}{3.696000in}}%
\pgfusepath{clip}%
\pgfsetrectcap%
\pgfsetroundjoin%
\pgfsetlinewidth{1.505625pt}%
\definecolor{currentstroke}{rgb}{1.000000,0.000000,0.000000}%
\pgfsetstrokecolor{currentstroke}%
\pgfsetdash{}{0pt}%
\pgfpathmoveto{\pgfqpoint{2.593351in}{1.936482in}}%
\pgfpathlineto{\pgfqpoint{2.481367in}{2.689827in}}%
\pgfusepath{stroke}%
\end{pgfscope}%
\begin{pgfscope}%
\pgfpathrectangle{\pgfqpoint{0.100000in}{0.212622in}}{\pgfqpoint{3.696000in}{3.696000in}}%
\pgfusepath{clip}%
\pgfsetrectcap%
\pgfsetroundjoin%
\pgfsetlinewidth{1.505625pt}%
\definecolor{currentstroke}{rgb}{1.000000,0.000000,0.000000}%
\pgfsetstrokecolor{currentstroke}%
\pgfsetdash{}{0pt}%
\pgfpathmoveto{\pgfqpoint{2.602817in}{1.934039in}}%
\pgfpathlineto{\pgfqpoint{2.496082in}{2.685840in}}%
\pgfusepath{stroke}%
\end{pgfscope}%
\begin{pgfscope}%
\pgfpathrectangle{\pgfqpoint{0.100000in}{0.212622in}}{\pgfqpoint{3.696000in}{3.696000in}}%
\pgfusepath{clip}%
\pgfsetrectcap%
\pgfsetroundjoin%
\pgfsetlinewidth{1.505625pt}%
\definecolor{currentstroke}{rgb}{1.000000,0.000000,0.000000}%
\pgfsetstrokecolor{currentstroke}%
\pgfsetdash{}{0pt}%
\pgfpathmoveto{\pgfqpoint{2.613692in}{1.932050in}}%
\pgfpathlineto{\pgfqpoint{2.510807in}{2.681851in}}%
\pgfusepath{stroke}%
\end{pgfscope}%
\begin{pgfscope}%
\pgfpathrectangle{\pgfqpoint{0.100000in}{0.212622in}}{\pgfqpoint{3.696000in}{3.696000in}}%
\pgfusepath{clip}%
\pgfsetrectcap%
\pgfsetroundjoin%
\pgfsetlinewidth{1.505625pt}%
\definecolor{currentstroke}{rgb}{1.000000,0.000000,0.000000}%
\pgfsetstrokecolor{currentstroke}%
\pgfsetdash{}{0pt}%
\pgfpathmoveto{\pgfqpoint{2.625613in}{1.930030in}}%
\pgfpathlineto{\pgfqpoint{2.525544in}{2.677859in}}%
\pgfusepath{stroke}%
\end{pgfscope}%
\begin{pgfscope}%
\pgfpathrectangle{\pgfqpoint{0.100000in}{0.212622in}}{\pgfqpoint{3.696000in}{3.696000in}}%
\pgfusepath{clip}%
\pgfsetrectcap%
\pgfsetroundjoin%
\pgfsetlinewidth{1.505625pt}%
\definecolor{currentstroke}{rgb}{1.000000,0.000000,0.000000}%
\pgfsetstrokecolor{currentstroke}%
\pgfsetdash{}{0pt}%
\pgfpathmoveto{\pgfqpoint{2.638088in}{1.927146in}}%
\pgfpathlineto{\pgfqpoint{2.540291in}{2.673864in}}%
\pgfusepath{stroke}%
\end{pgfscope}%
\begin{pgfscope}%
\pgfpathrectangle{\pgfqpoint{0.100000in}{0.212622in}}{\pgfqpoint{3.696000in}{3.696000in}}%
\pgfusepath{clip}%
\pgfsetrectcap%
\pgfsetroundjoin%
\pgfsetlinewidth{1.505625pt}%
\definecolor{currentstroke}{rgb}{1.000000,0.000000,0.000000}%
\pgfsetstrokecolor{currentstroke}%
\pgfsetdash{}{0pt}%
\pgfpathmoveto{\pgfqpoint{2.651853in}{1.923088in}}%
\pgfpathlineto{\pgfqpoint{2.555049in}{2.669866in}}%
\pgfusepath{stroke}%
\end{pgfscope}%
\begin{pgfscope}%
\pgfpathrectangle{\pgfqpoint{0.100000in}{0.212622in}}{\pgfqpoint{3.696000in}{3.696000in}}%
\pgfusepath{clip}%
\pgfsetrectcap%
\pgfsetroundjoin%
\pgfsetlinewidth{1.505625pt}%
\definecolor{currentstroke}{rgb}{1.000000,0.000000,0.000000}%
\pgfsetstrokecolor{currentstroke}%
\pgfsetdash{}{0pt}%
\pgfpathmoveto{\pgfqpoint{2.664933in}{1.919884in}}%
\pgfpathlineto{\pgfqpoint{2.555049in}{2.669866in}}%
\pgfusepath{stroke}%
\end{pgfscope}%
\begin{pgfscope}%
\pgfpathrectangle{\pgfqpoint{0.100000in}{0.212622in}}{\pgfqpoint{3.696000in}{3.696000in}}%
\pgfusepath{clip}%
\pgfsetrectcap%
\pgfsetroundjoin%
\pgfsetlinewidth{1.505625pt}%
\definecolor{currentstroke}{rgb}{1.000000,0.000000,0.000000}%
\pgfsetstrokecolor{currentstroke}%
\pgfsetdash{}{0pt}%
\pgfpathmoveto{\pgfqpoint{2.672603in}{1.917866in}}%
\pgfpathlineto{\pgfqpoint{2.569818in}{2.665865in}}%
\pgfusepath{stroke}%
\end{pgfscope}%
\begin{pgfscope}%
\pgfpathrectangle{\pgfqpoint{0.100000in}{0.212622in}}{\pgfqpoint{3.696000in}{3.696000in}}%
\pgfusepath{clip}%
\pgfsetrectcap%
\pgfsetroundjoin%
\pgfsetlinewidth{1.505625pt}%
\definecolor{currentstroke}{rgb}{1.000000,0.000000,0.000000}%
\pgfsetstrokecolor{currentstroke}%
\pgfsetdash{}{0pt}%
\pgfpathmoveto{\pgfqpoint{2.676959in}{1.916692in}}%
\pgfpathlineto{\pgfqpoint{2.569818in}{2.665865in}}%
\pgfusepath{stroke}%
\end{pgfscope}%
\begin{pgfscope}%
\pgfpathrectangle{\pgfqpoint{0.100000in}{0.212622in}}{\pgfqpoint{3.696000in}{3.696000in}}%
\pgfusepath{clip}%
\pgfsetrectcap%
\pgfsetroundjoin%
\pgfsetlinewidth{1.505625pt}%
\definecolor{currentstroke}{rgb}{1.000000,0.000000,0.000000}%
\pgfsetstrokecolor{currentstroke}%
\pgfsetdash{}{0pt}%
\pgfpathmoveto{\pgfqpoint{2.681534in}{1.915404in}}%
\pgfpathlineto{\pgfqpoint{2.584597in}{2.661861in}}%
\pgfusepath{stroke}%
\end{pgfscope}%
\begin{pgfscope}%
\pgfpathrectangle{\pgfqpoint{0.100000in}{0.212622in}}{\pgfqpoint{3.696000in}{3.696000in}}%
\pgfusepath{clip}%
\pgfsetrectcap%
\pgfsetroundjoin%
\pgfsetlinewidth{1.505625pt}%
\definecolor{currentstroke}{rgb}{1.000000,0.000000,0.000000}%
\pgfsetstrokecolor{currentstroke}%
\pgfsetdash{}{0pt}%
\pgfpathmoveto{\pgfqpoint{2.686560in}{1.914336in}}%
\pgfpathlineto{\pgfqpoint{2.584597in}{2.661861in}}%
\pgfusepath{stroke}%
\end{pgfscope}%
\begin{pgfscope}%
\pgfpathrectangle{\pgfqpoint{0.100000in}{0.212622in}}{\pgfqpoint{3.696000in}{3.696000in}}%
\pgfusepath{clip}%
\pgfsetrectcap%
\pgfsetroundjoin%
\pgfsetlinewidth{1.505625pt}%
\definecolor{currentstroke}{rgb}{1.000000,0.000000,0.000000}%
\pgfsetstrokecolor{currentstroke}%
\pgfsetdash{}{0pt}%
\pgfpathmoveto{\pgfqpoint{2.692047in}{1.913108in}}%
\pgfpathlineto{\pgfqpoint{2.584597in}{2.661861in}}%
\pgfusepath{stroke}%
\end{pgfscope}%
\begin{pgfscope}%
\pgfpathrectangle{\pgfqpoint{0.100000in}{0.212622in}}{\pgfqpoint{3.696000in}{3.696000in}}%
\pgfusepath{clip}%
\pgfsetrectcap%
\pgfsetroundjoin%
\pgfsetlinewidth{1.505625pt}%
\definecolor{currentstroke}{rgb}{1.000000,0.000000,0.000000}%
\pgfsetstrokecolor{currentstroke}%
\pgfsetdash{}{0pt}%
\pgfpathmoveto{\pgfqpoint{2.694607in}{1.912706in}}%
\pgfpathlineto{\pgfqpoint{2.599387in}{2.657855in}}%
\pgfusepath{stroke}%
\end{pgfscope}%
\begin{pgfscope}%
\pgfpathrectangle{\pgfqpoint{0.100000in}{0.212622in}}{\pgfqpoint{3.696000in}{3.696000in}}%
\pgfusepath{clip}%
\pgfsetrectcap%
\pgfsetroundjoin%
\pgfsetlinewidth{1.505625pt}%
\definecolor{currentstroke}{rgb}{1.000000,0.000000,0.000000}%
\pgfsetstrokecolor{currentstroke}%
\pgfsetdash{}{0pt}%
\pgfpathmoveto{\pgfqpoint{2.699650in}{1.911349in}}%
\pgfpathlineto{\pgfqpoint{2.599387in}{2.657855in}}%
\pgfusepath{stroke}%
\end{pgfscope}%
\begin{pgfscope}%
\pgfpathrectangle{\pgfqpoint{0.100000in}{0.212622in}}{\pgfqpoint{3.696000in}{3.696000in}}%
\pgfusepath{clip}%
\pgfsetrectcap%
\pgfsetroundjoin%
\pgfsetlinewidth{1.505625pt}%
\definecolor{currentstroke}{rgb}{1.000000,0.000000,0.000000}%
\pgfsetstrokecolor{currentstroke}%
\pgfsetdash{}{0pt}%
\pgfpathmoveto{\pgfqpoint{2.702400in}{1.910789in}}%
\pgfpathlineto{\pgfqpoint{2.599387in}{2.657855in}}%
\pgfusepath{stroke}%
\end{pgfscope}%
\begin{pgfscope}%
\pgfpathrectangle{\pgfqpoint{0.100000in}{0.212622in}}{\pgfqpoint{3.696000in}{3.696000in}}%
\pgfusepath{clip}%
\pgfsetrectcap%
\pgfsetroundjoin%
\pgfsetlinewidth{1.505625pt}%
\definecolor{currentstroke}{rgb}{1.000000,0.000000,0.000000}%
\pgfsetstrokecolor{currentstroke}%
\pgfsetdash{}{0pt}%
\pgfpathmoveto{\pgfqpoint{2.705998in}{1.909967in}}%
\pgfpathlineto{\pgfqpoint{2.599387in}{2.657855in}}%
\pgfusepath{stroke}%
\end{pgfscope}%
\begin{pgfscope}%
\pgfpathrectangle{\pgfqpoint{0.100000in}{0.212622in}}{\pgfqpoint{3.696000in}{3.696000in}}%
\pgfusepath{clip}%
\pgfsetrectcap%
\pgfsetroundjoin%
\pgfsetlinewidth{1.505625pt}%
\definecolor{currentstroke}{rgb}{1.000000,0.000000,0.000000}%
\pgfsetstrokecolor{currentstroke}%
\pgfsetdash{}{0pt}%
\pgfpathmoveto{\pgfqpoint{2.710487in}{1.908974in}}%
\pgfpathlineto{\pgfqpoint{2.614188in}{2.653845in}}%
\pgfusepath{stroke}%
\end{pgfscope}%
\begin{pgfscope}%
\pgfpathrectangle{\pgfqpoint{0.100000in}{0.212622in}}{\pgfqpoint{3.696000in}{3.696000in}}%
\pgfusepath{clip}%
\pgfsetrectcap%
\pgfsetroundjoin%
\pgfsetlinewidth{1.505625pt}%
\definecolor{currentstroke}{rgb}{1.000000,0.000000,0.000000}%
\pgfsetstrokecolor{currentstroke}%
\pgfsetdash{}{0pt}%
\pgfpathmoveto{\pgfqpoint{2.715153in}{1.908335in}}%
\pgfpathlineto{\pgfqpoint{2.614188in}{2.653845in}}%
\pgfusepath{stroke}%
\end{pgfscope}%
\begin{pgfscope}%
\pgfpathrectangle{\pgfqpoint{0.100000in}{0.212622in}}{\pgfqpoint{3.696000in}{3.696000in}}%
\pgfusepath{clip}%
\pgfsetrectcap%
\pgfsetroundjoin%
\pgfsetlinewidth{1.505625pt}%
\definecolor{currentstroke}{rgb}{1.000000,0.000000,0.000000}%
\pgfsetstrokecolor{currentstroke}%
\pgfsetdash{}{0pt}%
\pgfpathmoveto{\pgfqpoint{2.720523in}{1.907243in}}%
\pgfpathlineto{\pgfqpoint{2.614188in}{2.653845in}}%
\pgfusepath{stroke}%
\end{pgfscope}%
\begin{pgfscope}%
\pgfpathrectangle{\pgfqpoint{0.100000in}{0.212622in}}{\pgfqpoint{3.696000in}{3.696000in}}%
\pgfusepath{clip}%
\pgfsetrectcap%
\pgfsetroundjoin%
\pgfsetlinewidth{1.505625pt}%
\definecolor{currentstroke}{rgb}{1.000000,0.000000,0.000000}%
\pgfsetstrokecolor{currentstroke}%
\pgfsetdash{}{0pt}%
\pgfpathmoveto{\pgfqpoint{2.725782in}{1.906756in}}%
\pgfpathlineto{\pgfqpoint{2.629000in}{2.649833in}}%
\pgfusepath{stroke}%
\end{pgfscope}%
\begin{pgfscope}%
\pgfpathrectangle{\pgfqpoint{0.100000in}{0.212622in}}{\pgfqpoint{3.696000in}{3.696000in}}%
\pgfusepath{clip}%
\pgfsetrectcap%
\pgfsetroundjoin%
\pgfsetlinewidth{1.505625pt}%
\definecolor{currentstroke}{rgb}{1.000000,0.000000,0.000000}%
\pgfsetstrokecolor{currentstroke}%
\pgfsetdash{}{0pt}%
\pgfpathmoveto{\pgfqpoint{2.732348in}{1.905378in}}%
\pgfpathlineto{\pgfqpoint{2.629000in}{2.649833in}}%
\pgfusepath{stroke}%
\end{pgfscope}%
\begin{pgfscope}%
\pgfpathrectangle{\pgfqpoint{0.100000in}{0.212622in}}{\pgfqpoint{3.696000in}{3.696000in}}%
\pgfusepath{clip}%
\pgfsetrectcap%
\pgfsetroundjoin%
\pgfsetlinewidth{1.505625pt}%
\definecolor{currentstroke}{rgb}{1.000000,0.000000,0.000000}%
\pgfsetstrokecolor{currentstroke}%
\pgfsetdash{}{0pt}%
\pgfpathmoveto{\pgfqpoint{2.740084in}{1.903807in}}%
\pgfpathlineto{\pgfqpoint{2.643823in}{2.645817in}}%
\pgfusepath{stroke}%
\end{pgfscope}%
\begin{pgfscope}%
\pgfpathrectangle{\pgfqpoint{0.100000in}{0.212622in}}{\pgfqpoint{3.696000in}{3.696000in}}%
\pgfusepath{clip}%
\pgfsetrectcap%
\pgfsetroundjoin%
\pgfsetlinewidth{1.505625pt}%
\definecolor{currentstroke}{rgb}{1.000000,0.000000,0.000000}%
\pgfsetstrokecolor{currentstroke}%
\pgfsetdash{}{0pt}%
\pgfpathmoveto{\pgfqpoint{2.748692in}{1.901527in}}%
\pgfpathlineto{\pgfqpoint{2.643823in}{2.645817in}}%
\pgfusepath{stroke}%
\end{pgfscope}%
\begin{pgfscope}%
\pgfpathrectangle{\pgfqpoint{0.100000in}{0.212622in}}{\pgfqpoint{3.696000in}{3.696000in}}%
\pgfusepath{clip}%
\pgfsetrectcap%
\pgfsetroundjoin%
\pgfsetlinewidth{1.505625pt}%
\definecolor{currentstroke}{rgb}{1.000000,0.000000,0.000000}%
\pgfsetstrokecolor{currentstroke}%
\pgfsetdash{}{0pt}%
\pgfpathmoveto{\pgfqpoint{2.757020in}{1.900246in}}%
\pgfpathlineto{\pgfqpoint{2.658656in}{2.641799in}}%
\pgfusepath{stroke}%
\end{pgfscope}%
\begin{pgfscope}%
\pgfpathrectangle{\pgfqpoint{0.100000in}{0.212622in}}{\pgfqpoint{3.696000in}{3.696000in}}%
\pgfusepath{clip}%
\pgfsetrectcap%
\pgfsetroundjoin%
\pgfsetlinewidth{1.505625pt}%
\definecolor{currentstroke}{rgb}{1.000000,0.000000,0.000000}%
\pgfsetstrokecolor{currentstroke}%
\pgfsetdash{}{0pt}%
\pgfpathmoveto{\pgfqpoint{2.766382in}{1.897675in}}%
\pgfpathlineto{\pgfqpoint{2.673501in}{2.637777in}}%
\pgfusepath{stroke}%
\end{pgfscope}%
\begin{pgfscope}%
\pgfpathrectangle{\pgfqpoint{0.100000in}{0.212622in}}{\pgfqpoint{3.696000in}{3.696000in}}%
\pgfusepath{clip}%
\pgfsetrectcap%
\pgfsetroundjoin%
\pgfsetlinewidth{1.505625pt}%
\definecolor{currentstroke}{rgb}{1.000000,0.000000,0.000000}%
\pgfsetstrokecolor{currentstroke}%
\pgfsetdash{}{0pt}%
\pgfpathmoveto{\pgfqpoint{2.771675in}{1.896353in}}%
\pgfpathlineto{\pgfqpoint{2.673501in}{2.637777in}}%
\pgfusepath{stroke}%
\end{pgfscope}%
\begin{pgfscope}%
\pgfpathrectangle{\pgfqpoint{0.100000in}{0.212622in}}{\pgfqpoint{3.696000in}{3.696000in}}%
\pgfusepath{clip}%
\pgfsetrectcap%
\pgfsetroundjoin%
\pgfsetlinewidth{1.505625pt}%
\definecolor{currentstroke}{rgb}{1.000000,0.000000,0.000000}%
\pgfsetstrokecolor{currentstroke}%
\pgfsetdash{}{0pt}%
\pgfpathmoveto{\pgfqpoint{2.777058in}{1.895184in}}%
\pgfpathlineto{\pgfqpoint{2.673501in}{2.637777in}}%
\pgfusepath{stroke}%
\end{pgfscope}%
\begin{pgfscope}%
\pgfpathrectangle{\pgfqpoint{0.100000in}{0.212622in}}{\pgfqpoint{3.696000in}{3.696000in}}%
\pgfusepath{clip}%
\pgfsetrectcap%
\pgfsetroundjoin%
\pgfsetlinewidth{1.505625pt}%
\definecolor{currentstroke}{rgb}{1.000000,0.000000,0.000000}%
\pgfsetstrokecolor{currentstroke}%
\pgfsetdash{}{0pt}%
\pgfpathmoveto{\pgfqpoint{2.784234in}{1.893284in}}%
\pgfpathlineto{\pgfqpoint{2.688356in}{2.633753in}}%
\pgfusepath{stroke}%
\end{pgfscope}%
\begin{pgfscope}%
\pgfpathrectangle{\pgfqpoint{0.100000in}{0.212622in}}{\pgfqpoint{3.696000in}{3.696000in}}%
\pgfusepath{clip}%
\pgfsetrectcap%
\pgfsetroundjoin%
\pgfsetlinewidth{1.505625pt}%
\definecolor{currentstroke}{rgb}{1.000000,0.000000,0.000000}%
\pgfsetstrokecolor{currentstroke}%
\pgfsetdash{}{0pt}%
\pgfpathmoveto{\pgfqpoint{2.791536in}{1.891738in}}%
\pgfpathlineto{\pgfqpoint{2.688356in}{2.633753in}}%
\pgfusepath{stroke}%
\end{pgfscope}%
\begin{pgfscope}%
\pgfpathrectangle{\pgfqpoint{0.100000in}{0.212622in}}{\pgfqpoint{3.696000in}{3.696000in}}%
\pgfusepath{clip}%
\pgfsetrectcap%
\pgfsetroundjoin%
\pgfsetlinewidth{1.505625pt}%
\definecolor{currentstroke}{rgb}{1.000000,0.000000,0.000000}%
\pgfsetstrokecolor{currentstroke}%
\pgfsetdash{}{0pt}%
\pgfpathmoveto{\pgfqpoint{2.795363in}{1.890879in}}%
\pgfpathlineto{\pgfqpoint{2.703222in}{2.629726in}}%
\pgfusepath{stroke}%
\end{pgfscope}%
\begin{pgfscope}%
\pgfpathrectangle{\pgfqpoint{0.100000in}{0.212622in}}{\pgfqpoint{3.696000in}{3.696000in}}%
\pgfusepath{clip}%
\pgfsetrectcap%
\pgfsetroundjoin%
\pgfsetlinewidth{1.505625pt}%
\definecolor{currentstroke}{rgb}{1.000000,0.000000,0.000000}%
\pgfsetstrokecolor{currentstroke}%
\pgfsetdash{}{0pt}%
\pgfpathmoveto{\pgfqpoint{2.800248in}{1.889953in}}%
\pgfpathlineto{\pgfqpoint{2.703222in}{2.629726in}}%
\pgfusepath{stroke}%
\end{pgfscope}%
\begin{pgfscope}%
\pgfpathrectangle{\pgfqpoint{0.100000in}{0.212622in}}{\pgfqpoint{3.696000in}{3.696000in}}%
\pgfusepath{clip}%
\pgfsetrectcap%
\pgfsetroundjoin%
\pgfsetlinewidth{1.505625pt}%
\definecolor{currentstroke}{rgb}{1.000000,0.000000,0.000000}%
\pgfsetstrokecolor{currentstroke}%
\pgfsetdash{}{0pt}%
\pgfpathmoveto{\pgfqpoint{2.805122in}{1.889206in}}%
\pgfpathlineto{\pgfqpoint{2.703222in}{2.629726in}}%
\pgfusepath{stroke}%
\end{pgfscope}%
\begin{pgfscope}%
\pgfpathrectangle{\pgfqpoint{0.100000in}{0.212622in}}{\pgfqpoint{3.696000in}{3.696000in}}%
\pgfusepath{clip}%
\pgfsetrectcap%
\pgfsetroundjoin%
\pgfsetlinewidth{1.505625pt}%
\definecolor{currentstroke}{rgb}{1.000000,0.000000,0.000000}%
\pgfsetstrokecolor{currentstroke}%
\pgfsetdash{}{0pt}%
\pgfpathmoveto{\pgfqpoint{2.807682in}{1.888842in}}%
\pgfpathlineto{\pgfqpoint{2.718099in}{2.625696in}}%
\pgfusepath{stroke}%
\end{pgfscope}%
\begin{pgfscope}%
\pgfpathrectangle{\pgfqpoint{0.100000in}{0.212622in}}{\pgfqpoint{3.696000in}{3.696000in}}%
\pgfusepath{clip}%
\pgfsetrectcap%
\pgfsetroundjoin%
\pgfsetlinewidth{1.505625pt}%
\definecolor{currentstroke}{rgb}{1.000000,0.000000,0.000000}%
\pgfsetstrokecolor{currentstroke}%
\pgfsetdash{}{0pt}%
\pgfpathmoveto{\pgfqpoint{2.811121in}{1.888129in}}%
\pgfpathlineto{\pgfqpoint{2.718099in}{2.625696in}}%
\pgfusepath{stroke}%
\end{pgfscope}%
\begin{pgfscope}%
\pgfpathrectangle{\pgfqpoint{0.100000in}{0.212622in}}{\pgfqpoint{3.696000in}{3.696000in}}%
\pgfusepath{clip}%
\pgfsetrectcap%
\pgfsetroundjoin%
\pgfsetlinewidth{1.505625pt}%
\definecolor{currentstroke}{rgb}{1.000000,0.000000,0.000000}%
\pgfsetstrokecolor{currentstroke}%
\pgfsetdash{}{0pt}%
\pgfpathmoveto{\pgfqpoint{2.812928in}{1.887994in}}%
\pgfpathlineto{\pgfqpoint{2.718099in}{2.625696in}}%
\pgfusepath{stroke}%
\end{pgfscope}%
\begin{pgfscope}%
\pgfpathrectangle{\pgfqpoint{0.100000in}{0.212622in}}{\pgfqpoint{3.696000in}{3.696000in}}%
\pgfusepath{clip}%
\pgfsetrectcap%
\pgfsetroundjoin%
\pgfsetlinewidth{1.505625pt}%
\definecolor{currentstroke}{rgb}{1.000000,0.000000,0.000000}%
\pgfsetstrokecolor{currentstroke}%
\pgfsetdash{}{0pt}%
\pgfpathmoveto{\pgfqpoint{2.814110in}{1.887839in}}%
\pgfpathlineto{\pgfqpoint{2.718099in}{2.625696in}}%
\pgfusepath{stroke}%
\end{pgfscope}%
\begin{pgfscope}%
\pgfpathrectangle{\pgfqpoint{0.100000in}{0.212622in}}{\pgfqpoint{3.696000in}{3.696000in}}%
\pgfusepath{clip}%
\pgfsetrectcap%
\pgfsetroundjoin%
\pgfsetlinewidth{1.505625pt}%
\definecolor{currentstroke}{rgb}{1.000000,0.000000,0.000000}%
\pgfsetstrokecolor{currentstroke}%
\pgfsetdash{}{0pt}%
\pgfpathmoveto{\pgfqpoint{2.815779in}{1.887538in}}%
\pgfpathlineto{\pgfqpoint{2.718099in}{2.625696in}}%
\pgfusepath{stroke}%
\end{pgfscope}%
\begin{pgfscope}%
\pgfpathrectangle{\pgfqpoint{0.100000in}{0.212622in}}{\pgfqpoint{3.696000in}{3.696000in}}%
\pgfusepath{clip}%
\pgfsetrectcap%
\pgfsetroundjoin%
\pgfsetlinewidth{1.505625pt}%
\definecolor{currentstroke}{rgb}{1.000000,0.000000,0.000000}%
\pgfsetstrokecolor{currentstroke}%
\pgfsetdash{}{0pt}%
\pgfpathmoveto{\pgfqpoint{2.817915in}{1.887157in}}%
\pgfpathlineto{\pgfqpoint{2.718099in}{2.625696in}}%
\pgfusepath{stroke}%
\end{pgfscope}%
\begin{pgfscope}%
\pgfpathrectangle{\pgfqpoint{0.100000in}{0.212622in}}{\pgfqpoint{3.696000in}{3.696000in}}%
\pgfusepath{clip}%
\pgfsetrectcap%
\pgfsetroundjoin%
\pgfsetlinewidth{1.505625pt}%
\definecolor{currentstroke}{rgb}{1.000000,0.000000,0.000000}%
\pgfsetstrokecolor{currentstroke}%
\pgfsetdash{}{0pt}%
\pgfpathmoveto{\pgfqpoint{2.821177in}{1.886679in}}%
\pgfpathlineto{\pgfqpoint{2.732987in}{2.621663in}}%
\pgfusepath{stroke}%
\end{pgfscope}%
\begin{pgfscope}%
\pgfpathrectangle{\pgfqpoint{0.100000in}{0.212622in}}{\pgfqpoint{3.696000in}{3.696000in}}%
\pgfusepath{clip}%
\pgfsetrectcap%
\pgfsetroundjoin%
\pgfsetlinewidth{1.505625pt}%
\definecolor{currentstroke}{rgb}{1.000000,0.000000,0.000000}%
\pgfsetstrokecolor{currentstroke}%
\pgfsetdash{}{0pt}%
\pgfpathmoveto{\pgfqpoint{2.826105in}{1.885603in}}%
\pgfpathlineto{\pgfqpoint{2.732987in}{2.621663in}}%
\pgfusepath{stroke}%
\end{pgfscope}%
\begin{pgfscope}%
\pgfpathrectangle{\pgfqpoint{0.100000in}{0.212622in}}{\pgfqpoint{3.696000in}{3.696000in}}%
\pgfusepath{clip}%
\pgfsetrectcap%
\pgfsetroundjoin%
\pgfsetlinewidth{1.505625pt}%
\definecolor{currentstroke}{rgb}{1.000000,0.000000,0.000000}%
\pgfsetstrokecolor{currentstroke}%
\pgfsetdash{}{0pt}%
\pgfpathmoveto{\pgfqpoint{2.832649in}{1.884355in}}%
\pgfpathlineto{\pgfqpoint{2.732987in}{2.621663in}}%
\pgfusepath{stroke}%
\end{pgfscope}%
\begin{pgfscope}%
\pgfpathrectangle{\pgfqpoint{0.100000in}{0.212622in}}{\pgfqpoint{3.696000in}{3.696000in}}%
\pgfusepath{clip}%
\pgfsetrectcap%
\pgfsetroundjoin%
\pgfsetlinewidth{1.505625pt}%
\definecolor{currentstroke}{rgb}{1.000000,0.000000,0.000000}%
\pgfsetstrokecolor{currentstroke}%
\pgfsetdash{}{0pt}%
\pgfpathmoveto{\pgfqpoint{2.840069in}{1.882869in}}%
\pgfpathlineto{\pgfqpoint{2.747886in}{2.617626in}}%
\pgfusepath{stroke}%
\end{pgfscope}%
\begin{pgfscope}%
\pgfpathrectangle{\pgfqpoint{0.100000in}{0.212622in}}{\pgfqpoint{3.696000in}{3.696000in}}%
\pgfusepath{clip}%
\pgfsetrectcap%
\pgfsetroundjoin%
\pgfsetlinewidth{1.505625pt}%
\definecolor{currentstroke}{rgb}{1.000000,0.000000,0.000000}%
\pgfsetstrokecolor{currentstroke}%
\pgfsetdash{}{0pt}%
\pgfpathmoveto{\pgfqpoint{2.843919in}{1.882289in}}%
\pgfpathlineto{\pgfqpoint{2.747886in}{2.617626in}}%
\pgfusepath{stroke}%
\end{pgfscope}%
\begin{pgfscope}%
\pgfpathrectangle{\pgfqpoint{0.100000in}{0.212622in}}{\pgfqpoint{3.696000in}{3.696000in}}%
\pgfusepath{clip}%
\pgfsetrectcap%
\pgfsetroundjoin%
\pgfsetlinewidth{1.505625pt}%
\definecolor{currentstroke}{rgb}{1.000000,0.000000,0.000000}%
\pgfsetstrokecolor{currentstroke}%
\pgfsetdash{}{0pt}%
\pgfpathmoveto{\pgfqpoint{2.848754in}{1.880860in}}%
\pgfpathlineto{\pgfqpoint{2.747886in}{2.617626in}}%
\pgfusepath{stroke}%
\end{pgfscope}%
\begin{pgfscope}%
\pgfpathrectangle{\pgfqpoint{0.100000in}{0.212622in}}{\pgfqpoint{3.696000in}{3.696000in}}%
\pgfusepath{clip}%
\pgfsetrectcap%
\pgfsetroundjoin%
\pgfsetlinewidth{1.505625pt}%
\definecolor{currentstroke}{rgb}{1.000000,0.000000,0.000000}%
\pgfsetstrokecolor{currentstroke}%
\pgfsetdash{}{0pt}%
\pgfpathmoveto{\pgfqpoint{2.853985in}{1.879142in}}%
\pgfpathlineto{\pgfqpoint{2.762795in}{2.613587in}}%
\pgfusepath{stroke}%
\end{pgfscope}%
\begin{pgfscope}%
\pgfpathrectangle{\pgfqpoint{0.100000in}{0.212622in}}{\pgfqpoint{3.696000in}{3.696000in}}%
\pgfusepath{clip}%
\pgfsetrectcap%
\pgfsetroundjoin%
\pgfsetlinewidth{1.505625pt}%
\definecolor{currentstroke}{rgb}{1.000000,0.000000,0.000000}%
\pgfsetstrokecolor{currentstroke}%
\pgfsetdash{}{0pt}%
\pgfpathmoveto{\pgfqpoint{2.858959in}{1.878217in}}%
\pgfpathlineto{\pgfqpoint{2.762795in}{2.613587in}}%
\pgfusepath{stroke}%
\end{pgfscope}%
\begin{pgfscope}%
\pgfpathrectangle{\pgfqpoint{0.100000in}{0.212622in}}{\pgfqpoint{3.696000in}{3.696000in}}%
\pgfusepath{clip}%
\pgfsetrectcap%
\pgfsetroundjoin%
\pgfsetlinewidth{1.505625pt}%
\definecolor{currentstroke}{rgb}{1.000000,0.000000,0.000000}%
\pgfsetstrokecolor{currentstroke}%
\pgfsetdash{}{0pt}%
\pgfpathmoveto{\pgfqpoint{2.864801in}{1.876960in}}%
\pgfpathlineto{\pgfqpoint{2.762795in}{2.613587in}}%
\pgfusepath{stroke}%
\end{pgfscope}%
\begin{pgfscope}%
\pgfpathrectangle{\pgfqpoint{0.100000in}{0.212622in}}{\pgfqpoint{3.696000in}{3.696000in}}%
\pgfusepath{clip}%
\pgfsetrectcap%
\pgfsetroundjoin%
\pgfsetlinewidth{1.505625pt}%
\definecolor{currentstroke}{rgb}{1.000000,0.000000,0.000000}%
\pgfsetstrokecolor{currentstroke}%
\pgfsetdash{}{0pt}%
\pgfpathmoveto{\pgfqpoint{2.870967in}{1.875777in}}%
\pgfpathlineto{\pgfqpoint{2.777716in}{2.609545in}}%
\pgfusepath{stroke}%
\end{pgfscope}%
\begin{pgfscope}%
\pgfpathrectangle{\pgfqpoint{0.100000in}{0.212622in}}{\pgfqpoint{3.696000in}{3.696000in}}%
\pgfusepath{clip}%
\pgfsetrectcap%
\pgfsetroundjoin%
\pgfsetlinewidth{1.505625pt}%
\definecolor{currentstroke}{rgb}{1.000000,0.000000,0.000000}%
\pgfsetstrokecolor{currentstroke}%
\pgfsetdash{}{0pt}%
\pgfpathmoveto{\pgfqpoint{2.877272in}{1.874560in}}%
\pgfpathlineto{\pgfqpoint{2.777716in}{2.609545in}}%
\pgfusepath{stroke}%
\end{pgfscope}%
\begin{pgfscope}%
\pgfpathrectangle{\pgfqpoint{0.100000in}{0.212622in}}{\pgfqpoint{3.696000in}{3.696000in}}%
\pgfusepath{clip}%
\pgfsetrectcap%
\pgfsetroundjoin%
\pgfsetlinewidth{1.505625pt}%
\definecolor{currentstroke}{rgb}{1.000000,0.000000,0.000000}%
\pgfsetstrokecolor{currentstroke}%
\pgfsetdash{}{0pt}%
\pgfpathmoveto{\pgfqpoint{2.881030in}{1.873544in}}%
\pgfpathlineto{\pgfqpoint{2.792647in}{2.605501in}}%
\pgfusepath{stroke}%
\end{pgfscope}%
\begin{pgfscope}%
\pgfpathrectangle{\pgfqpoint{0.100000in}{0.212622in}}{\pgfqpoint{3.696000in}{3.696000in}}%
\pgfusepath{clip}%
\pgfsetrectcap%
\pgfsetroundjoin%
\pgfsetlinewidth{1.505625pt}%
\definecolor{currentstroke}{rgb}{1.000000,0.000000,0.000000}%
\pgfsetstrokecolor{currentstroke}%
\pgfsetdash{}{0pt}%
\pgfpathmoveto{\pgfqpoint{2.885078in}{1.872554in}}%
\pgfpathlineto{\pgfqpoint{2.792647in}{2.605501in}}%
\pgfusepath{stroke}%
\end{pgfscope}%
\begin{pgfscope}%
\pgfpathrectangle{\pgfqpoint{0.100000in}{0.212622in}}{\pgfqpoint{3.696000in}{3.696000in}}%
\pgfusepath{clip}%
\pgfsetrectcap%
\pgfsetroundjoin%
\pgfsetlinewidth{1.505625pt}%
\definecolor{currentstroke}{rgb}{1.000000,0.000000,0.000000}%
\pgfsetstrokecolor{currentstroke}%
\pgfsetdash{}{0pt}%
\pgfpathmoveto{\pgfqpoint{2.889523in}{1.871498in}}%
\pgfpathlineto{\pgfqpoint{2.792647in}{2.605501in}}%
\pgfusepath{stroke}%
\end{pgfscope}%
\begin{pgfscope}%
\pgfpathrectangle{\pgfqpoint{0.100000in}{0.212622in}}{\pgfqpoint{3.696000in}{3.696000in}}%
\pgfusepath{clip}%
\pgfsetrectcap%
\pgfsetroundjoin%
\pgfsetlinewidth{1.505625pt}%
\definecolor{currentstroke}{rgb}{1.000000,0.000000,0.000000}%
\pgfsetstrokecolor{currentstroke}%
\pgfsetdash{}{0pt}%
\pgfpathmoveto{\pgfqpoint{2.891802in}{1.871104in}}%
\pgfpathlineto{\pgfqpoint{2.792647in}{2.605501in}}%
\pgfusepath{stroke}%
\end{pgfscope}%
\begin{pgfscope}%
\pgfpathrectangle{\pgfqpoint{0.100000in}{0.212622in}}{\pgfqpoint{3.696000in}{3.696000in}}%
\pgfusepath{clip}%
\pgfsetrectcap%
\pgfsetroundjoin%
\pgfsetlinewidth{1.505625pt}%
\definecolor{currentstroke}{rgb}{1.000000,0.000000,0.000000}%
\pgfsetstrokecolor{currentstroke}%
\pgfsetdash{}{0pt}%
\pgfpathmoveto{\pgfqpoint{2.894539in}{1.870720in}}%
\pgfpathlineto{\pgfqpoint{2.792647in}{2.605501in}}%
\pgfusepath{stroke}%
\end{pgfscope}%
\begin{pgfscope}%
\pgfpathrectangle{\pgfqpoint{0.100000in}{0.212622in}}{\pgfqpoint{3.696000in}{3.696000in}}%
\pgfusepath{clip}%
\pgfsetrectcap%
\pgfsetroundjoin%
\pgfsetlinewidth{1.505625pt}%
\definecolor{currentstroke}{rgb}{1.000000,0.000000,0.000000}%
\pgfsetstrokecolor{currentstroke}%
\pgfsetdash{}{0pt}%
\pgfpathmoveto{\pgfqpoint{2.896158in}{1.870402in}}%
\pgfpathlineto{\pgfqpoint{2.807590in}{2.601453in}}%
\pgfusepath{stroke}%
\end{pgfscope}%
\begin{pgfscope}%
\pgfpathrectangle{\pgfqpoint{0.100000in}{0.212622in}}{\pgfqpoint{3.696000in}{3.696000in}}%
\pgfusepath{clip}%
\pgfsetrectcap%
\pgfsetroundjoin%
\pgfsetlinewidth{1.505625pt}%
\definecolor{currentstroke}{rgb}{1.000000,0.000000,0.000000}%
\pgfsetstrokecolor{currentstroke}%
\pgfsetdash{}{0pt}%
\pgfpathmoveto{\pgfqpoint{2.898903in}{1.869959in}}%
\pgfpathlineto{\pgfqpoint{2.807590in}{2.601453in}}%
\pgfusepath{stroke}%
\end{pgfscope}%
\begin{pgfscope}%
\pgfpathrectangle{\pgfqpoint{0.100000in}{0.212622in}}{\pgfqpoint{3.696000in}{3.696000in}}%
\pgfusepath{clip}%
\pgfsetrectcap%
\pgfsetroundjoin%
\pgfsetlinewidth{1.505625pt}%
\definecolor{currentstroke}{rgb}{1.000000,0.000000,0.000000}%
\pgfsetstrokecolor{currentstroke}%
\pgfsetdash{}{0pt}%
\pgfpathmoveto{\pgfqpoint{2.902163in}{1.869442in}}%
\pgfpathlineto{\pgfqpoint{2.807590in}{2.601453in}}%
\pgfusepath{stroke}%
\end{pgfscope}%
\begin{pgfscope}%
\pgfpathrectangle{\pgfqpoint{0.100000in}{0.212622in}}{\pgfqpoint{3.696000in}{3.696000in}}%
\pgfusepath{clip}%
\pgfsetrectcap%
\pgfsetroundjoin%
\pgfsetlinewidth{1.505625pt}%
\definecolor{currentstroke}{rgb}{1.000000,0.000000,0.000000}%
\pgfsetstrokecolor{currentstroke}%
\pgfsetdash{}{0pt}%
\pgfpathmoveto{\pgfqpoint{2.904048in}{1.869044in}}%
\pgfpathlineto{\pgfqpoint{2.807590in}{2.601453in}}%
\pgfusepath{stroke}%
\end{pgfscope}%
\begin{pgfscope}%
\pgfpathrectangle{\pgfqpoint{0.100000in}{0.212622in}}{\pgfqpoint{3.696000in}{3.696000in}}%
\pgfusepath{clip}%
\pgfsetrectcap%
\pgfsetroundjoin%
\pgfsetlinewidth{1.505625pt}%
\definecolor{currentstroke}{rgb}{1.000000,0.000000,0.000000}%
\pgfsetstrokecolor{currentstroke}%
\pgfsetdash{}{0pt}%
\pgfpathmoveto{\pgfqpoint{2.906168in}{1.868716in}}%
\pgfpathlineto{\pgfqpoint{2.807590in}{2.601453in}}%
\pgfusepath{stroke}%
\end{pgfscope}%
\begin{pgfscope}%
\pgfpathrectangle{\pgfqpoint{0.100000in}{0.212622in}}{\pgfqpoint{3.696000in}{3.696000in}}%
\pgfusepath{clip}%
\pgfsetrectcap%
\pgfsetroundjoin%
\pgfsetlinewidth{1.505625pt}%
\definecolor{currentstroke}{rgb}{1.000000,0.000000,0.000000}%
\pgfsetstrokecolor{currentstroke}%
\pgfsetdash{}{0pt}%
\pgfpathmoveto{\pgfqpoint{2.908860in}{1.868200in}}%
\pgfpathlineto{\pgfqpoint{2.807590in}{2.601453in}}%
\pgfusepath{stroke}%
\end{pgfscope}%
\begin{pgfscope}%
\pgfpathrectangle{\pgfqpoint{0.100000in}{0.212622in}}{\pgfqpoint{3.696000in}{3.696000in}}%
\pgfusepath{clip}%
\pgfsetrectcap%
\pgfsetroundjoin%
\pgfsetlinewidth{1.505625pt}%
\definecolor{currentstroke}{rgb}{1.000000,0.000000,0.000000}%
\pgfsetstrokecolor{currentstroke}%
\pgfsetdash{}{0pt}%
\pgfpathmoveto{\pgfqpoint{2.912071in}{1.867543in}}%
\pgfpathlineto{\pgfqpoint{2.822543in}{2.597402in}}%
\pgfusepath{stroke}%
\end{pgfscope}%
\begin{pgfscope}%
\pgfpathrectangle{\pgfqpoint{0.100000in}{0.212622in}}{\pgfqpoint{3.696000in}{3.696000in}}%
\pgfusepath{clip}%
\pgfsetrectcap%
\pgfsetroundjoin%
\pgfsetlinewidth{1.505625pt}%
\definecolor{currentstroke}{rgb}{1.000000,0.000000,0.000000}%
\pgfsetstrokecolor{currentstroke}%
\pgfsetdash{}{0pt}%
\pgfpathmoveto{\pgfqpoint{2.916338in}{1.866621in}}%
\pgfpathlineto{\pgfqpoint{2.822543in}{2.597402in}}%
\pgfusepath{stroke}%
\end{pgfscope}%
\begin{pgfscope}%
\pgfpathrectangle{\pgfqpoint{0.100000in}{0.212622in}}{\pgfqpoint{3.696000in}{3.696000in}}%
\pgfusepath{clip}%
\pgfsetrectcap%
\pgfsetroundjoin%
\pgfsetlinewidth{1.505625pt}%
\definecolor{currentstroke}{rgb}{1.000000,0.000000,0.000000}%
\pgfsetstrokecolor{currentstroke}%
\pgfsetdash{}{0pt}%
\pgfpathmoveto{\pgfqpoint{2.922215in}{1.865288in}}%
\pgfpathlineto{\pgfqpoint{2.822543in}{2.597402in}}%
\pgfusepath{stroke}%
\end{pgfscope}%
\begin{pgfscope}%
\pgfpathrectangle{\pgfqpoint{0.100000in}{0.212622in}}{\pgfqpoint{3.696000in}{3.696000in}}%
\pgfusepath{clip}%
\pgfsetrectcap%
\pgfsetroundjoin%
\pgfsetlinewidth{1.505625pt}%
\definecolor{currentstroke}{rgb}{1.000000,0.000000,0.000000}%
\pgfsetstrokecolor{currentstroke}%
\pgfsetdash{}{0pt}%
\pgfpathmoveto{\pgfqpoint{2.928450in}{1.863745in}}%
\pgfpathlineto{\pgfqpoint{2.837508in}{2.593348in}}%
\pgfusepath{stroke}%
\end{pgfscope}%
\begin{pgfscope}%
\pgfpathrectangle{\pgfqpoint{0.100000in}{0.212622in}}{\pgfqpoint{3.696000in}{3.696000in}}%
\pgfusepath{clip}%
\pgfsetrectcap%
\pgfsetroundjoin%
\pgfsetlinewidth{1.505625pt}%
\definecolor{currentstroke}{rgb}{1.000000,0.000000,0.000000}%
\pgfsetstrokecolor{currentstroke}%
\pgfsetdash{}{0pt}%
\pgfpathmoveto{\pgfqpoint{2.934935in}{1.862502in}}%
\pgfpathlineto{\pgfqpoint{2.837508in}{2.593348in}}%
\pgfusepath{stroke}%
\end{pgfscope}%
\begin{pgfscope}%
\pgfpathrectangle{\pgfqpoint{0.100000in}{0.212622in}}{\pgfqpoint{3.696000in}{3.696000in}}%
\pgfusepath{clip}%
\pgfsetrectcap%
\pgfsetroundjoin%
\pgfsetlinewidth{1.505625pt}%
\definecolor{currentstroke}{rgb}{1.000000,0.000000,0.000000}%
\pgfsetstrokecolor{currentstroke}%
\pgfsetdash{}{0pt}%
\pgfpathmoveto{\pgfqpoint{2.942203in}{1.860758in}}%
\pgfpathlineto{\pgfqpoint{2.852483in}{2.589291in}}%
\pgfusepath{stroke}%
\end{pgfscope}%
\begin{pgfscope}%
\pgfpathrectangle{\pgfqpoint{0.100000in}{0.212622in}}{\pgfqpoint{3.696000in}{3.696000in}}%
\pgfusepath{clip}%
\pgfsetrectcap%
\pgfsetroundjoin%
\pgfsetlinewidth{1.505625pt}%
\definecolor{currentstroke}{rgb}{1.000000,0.000000,0.000000}%
\pgfsetstrokecolor{currentstroke}%
\pgfsetdash{}{0pt}%
\pgfpathmoveto{\pgfqpoint{2.945816in}{1.860120in}}%
\pgfpathlineto{\pgfqpoint{2.852483in}{2.589291in}}%
\pgfusepath{stroke}%
\end{pgfscope}%
\begin{pgfscope}%
\pgfpathrectangle{\pgfqpoint{0.100000in}{0.212622in}}{\pgfqpoint{3.696000in}{3.696000in}}%
\pgfusepath{clip}%
\pgfsetrectcap%
\pgfsetroundjoin%
\pgfsetlinewidth{1.505625pt}%
\definecolor{currentstroke}{rgb}{1.000000,0.000000,0.000000}%
\pgfsetstrokecolor{currentstroke}%
\pgfsetdash{}{0pt}%
\pgfpathmoveto{\pgfqpoint{2.950948in}{1.859061in}}%
\pgfpathlineto{\pgfqpoint{2.852483in}{2.589291in}}%
\pgfusepath{stroke}%
\end{pgfscope}%
\begin{pgfscope}%
\pgfpathrectangle{\pgfqpoint{0.100000in}{0.212622in}}{\pgfqpoint{3.696000in}{3.696000in}}%
\pgfusepath{clip}%
\pgfsetrectcap%
\pgfsetroundjoin%
\pgfsetlinewidth{1.505625pt}%
\definecolor{currentstroke}{rgb}{1.000000,0.000000,0.000000}%
\pgfsetstrokecolor{currentstroke}%
\pgfsetdash{}{0pt}%
\pgfpathmoveto{\pgfqpoint{2.958054in}{1.857267in}}%
\pgfpathlineto{\pgfqpoint{2.867469in}{2.585231in}}%
\pgfusepath{stroke}%
\end{pgfscope}%
\begin{pgfscope}%
\pgfpathrectangle{\pgfqpoint{0.100000in}{0.212622in}}{\pgfqpoint{3.696000in}{3.696000in}}%
\pgfusepath{clip}%
\pgfsetrectcap%
\pgfsetroundjoin%
\pgfsetlinewidth{1.505625pt}%
\definecolor{currentstroke}{rgb}{1.000000,0.000000,0.000000}%
\pgfsetstrokecolor{currentstroke}%
\pgfsetdash{}{0pt}%
\pgfpathmoveto{\pgfqpoint{2.965541in}{1.855921in}}%
\pgfpathlineto{\pgfqpoint{2.867469in}{2.585231in}}%
\pgfusepath{stroke}%
\end{pgfscope}%
\begin{pgfscope}%
\pgfpathrectangle{\pgfqpoint{0.100000in}{0.212622in}}{\pgfqpoint{3.696000in}{3.696000in}}%
\pgfusepath{clip}%
\pgfsetrectcap%
\pgfsetroundjoin%
\pgfsetlinewidth{1.505625pt}%
\definecolor{currentstroke}{rgb}{1.000000,0.000000,0.000000}%
\pgfsetstrokecolor{currentstroke}%
\pgfsetdash{}{0pt}%
\pgfpathmoveto{\pgfqpoint{2.969615in}{1.855227in}}%
\pgfpathlineto{\pgfqpoint{2.867469in}{2.585231in}}%
\pgfusepath{stroke}%
\end{pgfscope}%
\begin{pgfscope}%
\pgfpathrectangle{\pgfqpoint{0.100000in}{0.212622in}}{\pgfqpoint{3.696000in}{3.696000in}}%
\pgfusepath{clip}%
\pgfsetrectcap%
\pgfsetroundjoin%
\pgfsetlinewidth{1.505625pt}%
\definecolor{currentstroke}{rgb}{1.000000,0.000000,0.000000}%
\pgfsetstrokecolor{currentstroke}%
\pgfsetdash{}{0pt}%
\pgfpathmoveto{\pgfqpoint{2.974164in}{1.854195in}}%
\pgfpathlineto{\pgfqpoint{2.882467in}{2.581168in}}%
\pgfusepath{stroke}%
\end{pgfscope}%
\begin{pgfscope}%
\pgfpathrectangle{\pgfqpoint{0.100000in}{0.212622in}}{\pgfqpoint{3.696000in}{3.696000in}}%
\pgfusepath{clip}%
\pgfsetrectcap%
\pgfsetroundjoin%
\pgfsetlinewidth{1.505625pt}%
\definecolor{currentstroke}{rgb}{1.000000,0.000000,0.000000}%
\pgfsetstrokecolor{currentstroke}%
\pgfsetdash{}{0pt}%
\pgfpathmoveto{\pgfqpoint{2.976491in}{1.853781in}}%
\pgfpathlineto{\pgfqpoint{2.882467in}{2.581168in}}%
\pgfusepath{stroke}%
\end{pgfscope}%
\begin{pgfscope}%
\pgfpathrectangle{\pgfqpoint{0.100000in}{0.212622in}}{\pgfqpoint{3.696000in}{3.696000in}}%
\pgfusepath{clip}%
\pgfsetrectcap%
\pgfsetroundjoin%
\pgfsetlinewidth{1.505625pt}%
\definecolor{currentstroke}{rgb}{1.000000,0.000000,0.000000}%
\pgfsetstrokecolor{currentstroke}%
\pgfsetdash{}{0pt}%
\pgfpathmoveto{\pgfqpoint{2.980033in}{1.853169in}}%
\pgfpathlineto{\pgfqpoint{2.882467in}{2.581168in}}%
\pgfusepath{stroke}%
\end{pgfscope}%
\begin{pgfscope}%
\pgfpathrectangle{\pgfqpoint{0.100000in}{0.212622in}}{\pgfqpoint{3.696000in}{3.696000in}}%
\pgfusepath{clip}%
\pgfsetrectcap%
\pgfsetroundjoin%
\pgfsetlinewidth{1.505625pt}%
\definecolor{currentstroke}{rgb}{1.000000,0.000000,0.000000}%
\pgfsetstrokecolor{currentstroke}%
\pgfsetdash{}{0pt}%
\pgfpathmoveto{\pgfqpoint{2.985357in}{1.851939in}}%
\pgfpathlineto{\pgfqpoint{2.882467in}{2.581168in}}%
\pgfusepath{stroke}%
\end{pgfscope}%
\begin{pgfscope}%
\pgfpathrectangle{\pgfqpoint{0.100000in}{0.212622in}}{\pgfqpoint{3.696000in}{3.696000in}}%
\pgfusepath{clip}%
\pgfsetrectcap%
\pgfsetroundjoin%
\pgfsetlinewidth{1.505625pt}%
\definecolor{currentstroke}{rgb}{1.000000,0.000000,0.000000}%
\pgfsetstrokecolor{currentstroke}%
\pgfsetdash{}{0pt}%
\pgfpathmoveto{\pgfqpoint{2.991384in}{1.850541in}}%
\pgfpathlineto{\pgfqpoint{2.897475in}{2.577103in}}%
\pgfusepath{stroke}%
\end{pgfscope}%
\begin{pgfscope}%
\pgfpathrectangle{\pgfqpoint{0.100000in}{0.212622in}}{\pgfqpoint{3.696000in}{3.696000in}}%
\pgfusepath{clip}%
\pgfsetrectcap%
\pgfsetroundjoin%
\pgfsetlinewidth{1.505625pt}%
\definecolor{currentstroke}{rgb}{1.000000,0.000000,0.000000}%
\pgfsetstrokecolor{currentstroke}%
\pgfsetdash{}{0pt}%
\pgfpathmoveto{\pgfqpoint{2.997593in}{1.849305in}}%
\pgfpathlineto{\pgfqpoint{2.897475in}{2.577103in}}%
\pgfusepath{stroke}%
\end{pgfscope}%
\begin{pgfscope}%
\pgfpathrectangle{\pgfqpoint{0.100000in}{0.212622in}}{\pgfqpoint{3.696000in}{3.696000in}}%
\pgfusepath{clip}%
\pgfsetrectcap%
\pgfsetroundjoin%
\pgfsetlinewidth{1.505625pt}%
\definecolor{currentstroke}{rgb}{1.000000,0.000000,0.000000}%
\pgfsetstrokecolor{currentstroke}%
\pgfsetdash{}{0pt}%
\pgfpathmoveto{\pgfqpoint{3.004160in}{1.847860in}}%
\pgfpathlineto{\pgfqpoint{2.912495in}{2.573034in}}%
\pgfusepath{stroke}%
\end{pgfscope}%
\begin{pgfscope}%
\pgfpathrectangle{\pgfqpoint{0.100000in}{0.212622in}}{\pgfqpoint{3.696000in}{3.696000in}}%
\pgfusepath{clip}%
\pgfsetrectcap%
\pgfsetroundjoin%
\pgfsetlinewidth{1.505625pt}%
\definecolor{currentstroke}{rgb}{1.000000,0.000000,0.000000}%
\pgfsetstrokecolor{currentstroke}%
\pgfsetdash{}{0pt}%
\pgfpathmoveto{\pgfqpoint{3.007949in}{1.847046in}}%
\pgfpathlineto{\pgfqpoint{2.912495in}{2.573034in}}%
\pgfusepath{stroke}%
\end{pgfscope}%
\begin{pgfscope}%
\pgfpathrectangle{\pgfqpoint{0.100000in}{0.212622in}}{\pgfqpoint{3.696000in}{3.696000in}}%
\pgfusepath{clip}%
\pgfsetrectcap%
\pgfsetroundjoin%
\pgfsetlinewidth{1.505625pt}%
\definecolor{currentstroke}{rgb}{1.000000,0.000000,0.000000}%
\pgfsetstrokecolor{currentstroke}%
\pgfsetdash{}{0pt}%
\pgfpathmoveto{\pgfqpoint{3.013092in}{1.846096in}}%
\pgfpathlineto{\pgfqpoint{2.912495in}{2.573034in}}%
\pgfusepath{stroke}%
\end{pgfscope}%
\begin{pgfscope}%
\pgfpathrectangle{\pgfqpoint{0.100000in}{0.212622in}}{\pgfqpoint{3.696000in}{3.696000in}}%
\pgfusepath{clip}%
\pgfsetrectcap%
\pgfsetroundjoin%
\pgfsetlinewidth{1.505625pt}%
\definecolor{currentstroke}{rgb}{1.000000,0.000000,0.000000}%
\pgfsetstrokecolor{currentstroke}%
\pgfsetdash{}{0pt}%
\pgfpathmoveto{\pgfqpoint{3.020410in}{1.844705in}}%
\pgfpathlineto{\pgfqpoint{2.927525in}{2.568962in}}%
\pgfusepath{stroke}%
\end{pgfscope}%
\begin{pgfscope}%
\pgfpathrectangle{\pgfqpoint{0.100000in}{0.212622in}}{\pgfqpoint{3.696000in}{3.696000in}}%
\pgfusepath{clip}%
\pgfsetrectcap%
\pgfsetroundjoin%
\pgfsetlinewidth{1.505625pt}%
\definecolor{currentstroke}{rgb}{1.000000,0.000000,0.000000}%
\pgfsetstrokecolor{currentstroke}%
\pgfsetdash{}{0pt}%
\pgfpathmoveto{\pgfqpoint{3.029405in}{1.842679in}}%
\pgfpathlineto{\pgfqpoint{2.927525in}{2.568962in}}%
\pgfusepath{stroke}%
\end{pgfscope}%
\begin{pgfscope}%
\pgfpathrectangle{\pgfqpoint{0.100000in}{0.212622in}}{\pgfqpoint{3.696000in}{3.696000in}}%
\pgfusepath{clip}%
\pgfsetrectcap%
\pgfsetroundjoin%
\pgfsetlinewidth{1.505625pt}%
\definecolor{currentstroke}{rgb}{1.000000,0.000000,0.000000}%
\pgfsetstrokecolor{currentstroke}%
\pgfsetdash{}{0pt}%
\pgfpathmoveto{\pgfqpoint{3.038063in}{1.841612in}}%
\pgfpathlineto{\pgfqpoint{2.942567in}{2.564887in}}%
\pgfusepath{stroke}%
\end{pgfscope}%
\begin{pgfscope}%
\pgfpathrectangle{\pgfqpoint{0.100000in}{0.212622in}}{\pgfqpoint{3.696000in}{3.696000in}}%
\pgfusepath{clip}%
\pgfsetrectcap%
\pgfsetroundjoin%
\pgfsetlinewidth{1.505625pt}%
\definecolor{currentstroke}{rgb}{1.000000,0.000000,0.000000}%
\pgfsetstrokecolor{currentstroke}%
\pgfsetdash{}{0pt}%
\pgfpathmoveto{\pgfqpoint{3.047858in}{1.839404in}}%
\pgfpathlineto{\pgfqpoint{2.957619in}{2.560810in}}%
\pgfusepath{stroke}%
\end{pgfscope}%
\begin{pgfscope}%
\pgfpathrectangle{\pgfqpoint{0.100000in}{0.212622in}}{\pgfqpoint{3.696000in}{3.696000in}}%
\pgfusepath{clip}%
\pgfsetrectcap%
\pgfsetroundjoin%
\pgfsetlinewidth{1.505625pt}%
\definecolor{currentstroke}{rgb}{1.000000,0.000000,0.000000}%
\pgfsetstrokecolor{currentstroke}%
\pgfsetdash{}{0pt}%
\pgfpathmoveto{\pgfqpoint{3.053083in}{1.838376in}}%
\pgfpathlineto{\pgfqpoint{2.957619in}{2.560810in}}%
\pgfusepath{stroke}%
\end{pgfscope}%
\begin{pgfscope}%
\pgfpathrectangle{\pgfqpoint{0.100000in}{0.212622in}}{\pgfqpoint{3.696000in}{3.696000in}}%
\pgfusepath{clip}%
\pgfsetrectcap%
\pgfsetroundjoin%
\pgfsetlinewidth{1.505625pt}%
\definecolor{currentstroke}{rgb}{1.000000,0.000000,0.000000}%
\pgfsetstrokecolor{currentstroke}%
\pgfsetdash{}{0pt}%
\pgfpathmoveto{\pgfqpoint{3.059297in}{1.837189in}}%
\pgfpathlineto{\pgfqpoint{2.957619in}{2.560810in}}%
\pgfusepath{stroke}%
\end{pgfscope}%
\begin{pgfscope}%
\pgfpathrectangle{\pgfqpoint{0.100000in}{0.212622in}}{\pgfqpoint{3.696000in}{3.696000in}}%
\pgfusepath{clip}%
\pgfsetrectcap%
\pgfsetroundjoin%
\pgfsetlinewidth{1.505625pt}%
\definecolor{currentstroke}{rgb}{1.000000,0.000000,0.000000}%
\pgfsetstrokecolor{currentstroke}%
\pgfsetdash{}{0pt}%
\pgfpathmoveto{\pgfqpoint{3.067014in}{1.835421in}}%
\pgfpathlineto{\pgfqpoint{2.972683in}{2.556729in}}%
\pgfusepath{stroke}%
\end{pgfscope}%
\begin{pgfscope}%
\pgfpathrectangle{\pgfqpoint{0.100000in}{0.212622in}}{\pgfqpoint{3.696000in}{3.696000in}}%
\pgfusepath{clip}%
\pgfsetrectcap%
\pgfsetroundjoin%
\pgfsetlinewidth{1.505625pt}%
\definecolor{currentstroke}{rgb}{1.000000,0.000000,0.000000}%
\pgfsetstrokecolor{currentstroke}%
\pgfsetdash{}{0pt}%
\pgfpathmoveto{\pgfqpoint{3.075679in}{1.833077in}}%
\pgfpathlineto{\pgfqpoint{2.987758in}{2.552645in}}%
\pgfusepath{stroke}%
\end{pgfscope}%
\begin{pgfscope}%
\pgfpathrectangle{\pgfqpoint{0.100000in}{0.212622in}}{\pgfqpoint{3.696000in}{3.696000in}}%
\pgfusepath{clip}%
\pgfsetrectcap%
\pgfsetroundjoin%
\pgfsetlinewidth{1.505625pt}%
\definecolor{currentstroke}{rgb}{1.000000,0.000000,0.000000}%
\pgfsetstrokecolor{currentstroke}%
\pgfsetdash{}{0pt}%
\pgfpathmoveto{\pgfqpoint{3.080251in}{1.832404in}}%
\pgfpathlineto{\pgfqpoint{2.987758in}{2.552645in}}%
\pgfusepath{stroke}%
\end{pgfscope}%
\begin{pgfscope}%
\pgfpathrectangle{\pgfqpoint{0.100000in}{0.212622in}}{\pgfqpoint{3.696000in}{3.696000in}}%
\pgfusepath{clip}%
\pgfsetrectcap%
\pgfsetroundjoin%
\pgfsetlinewidth{1.505625pt}%
\definecolor{currentstroke}{rgb}{1.000000,0.000000,0.000000}%
\pgfsetstrokecolor{currentstroke}%
\pgfsetdash{}{0pt}%
\pgfpathmoveto{\pgfqpoint{3.085393in}{1.831549in}}%
\pgfpathlineto{\pgfqpoint{2.987758in}{2.552645in}}%
\pgfusepath{stroke}%
\end{pgfscope}%
\begin{pgfscope}%
\pgfpathrectangle{\pgfqpoint{0.100000in}{0.212622in}}{\pgfqpoint{3.696000in}{3.696000in}}%
\pgfusepath{clip}%
\pgfsetrectcap%
\pgfsetroundjoin%
\pgfsetlinewidth{1.505625pt}%
\definecolor{currentstroke}{rgb}{1.000000,0.000000,0.000000}%
\pgfsetstrokecolor{currentstroke}%
\pgfsetdash{}{0pt}%
\pgfpathmoveto{\pgfqpoint{3.091504in}{1.830185in}}%
\pgfpathlineto{\pgfqpoint{3.002844in}{2.548558in}}%
\pgfusepath{stroke}%
\end{pgfscope}%
\begin{pgfscope}%
\pgfpathrectangle{\pgfqpoint{0.100000in}{0.212622in}}{\pgfqpoint{3.696000in}{3.696000in}}%
\pgfusepath{clip}%
\pgfsetrectcap%
\pgfsetroundjoin%
\pgfsetlinewidth{1.505625pt}%
\definecolor{currentstroke}{rgb}{1.000000,0.000000,0.000000}%
\pgfsetstrokecolor{currentstroke}%
\pgfsetdash{}{0pt}%
\pgfpathmoveto{\pgfqpoint{3.097591in}{1.829065in}}%
\pgfpathlineto{\pgfqpoint{3.002844in}{2.548558in}}%
\pgfusepath{stroke}%
\end{pgfscope}%
\begin{pgfscope}%
\pgfpathrectangle{\pgfqpoint{0.100000in}{0.212622in}}{\pgfqpoint{3.696000in}{3.696000in}}%
\pgfusepath{clip}%
\pgfsetrectcap%
\pgfsetroundjoin%
\pgfsetlinewidth{1.505625pt}%
\definecolor{currentstroke}{rgb}{1.000000,0.000000,0.000000}%
\pgfsetstrokecolor{currentstroke}%
\pgfsetdash{}{0pt}%
\pgfpathmoveto{\pgfqpoint{3.104346in}{1.827955in}}%
\pgfpathlineto{\pgfqpoint{3.017941in}{2.544468in}}%
\pgfusepath{stroke}%
\end{pgfscope}%
\begin{pgfscope}%
\pgfpathrectangle{\pgfqpoint{0.100000in}{0.212622in}}{\pgfqpoint{3.696000in}{3.696000in}}%
\pgfusepath{clip}%
\pgfsetrectcap%
\pgfsetroundjoin%
\pgfsetlinewidth{1.505625pt}%
\definecolor{currentstroke}{rgb}{1.000000,0.000000,0.000000}%
\pgfsetstrokecolor{currentstroke}%
\pgfsetdash{}{0pt}%
\pgfpathmoveto{\pgfqpoint{3.112048in}{1.826630in}}%
\pgfpathlineto{\pgfqpoint{3.017941in}{2.544468in}}%
\pgfusepath{stroke}%
\end{pgfscope}%
\begin{pgfscope}%
\pgfpathrectangle{\pgfqpoint{0.100000in}{0.212622in}}{\pgfqpoint{3.696000in}{3.696000in}}%
\pgfusepath{clip}%
\pgfsetrectcap%
\pgfsetroundjoin%
\pgfsetlinewidth{1.505625pt}%
\definecolor{currentstroke}{rgb}{1.000000,0.000000,0.000000}%
\pgfsetstrokecolor{currentstroke}%
\pgfsetdash{}{0pt}%
\pgfpathmoveto{\pgfqpoint{3.120582in}{1.825252in}}%
\pgfpathlineto{\pgfqpoint{3.033049in}{2.540376in}}%
\pgfusepath{stroke}%
\end{pgfscope}%
\begin{pgfscope}%
\pgfpathrectangle{\pgfqpoint{0.100000in}{0.212622in}}{\pgfqpoint{3.696000in}{3.696000in}}%
\pgfusepath{clip}%
\pgfsetrectcap%
\pgfsetroundjoin%
\pgfsetlinewidth{1.505625pt}%
\definecolor{currentstroke}{rgb}{1.000000,0.000000,0.000000}%
\pgfsetstrokecolor{currentstroke}%
\pgfsetdash{}{0pt}%
\pgfpathmoveto{\pgfqpoint{3.125255in}{1.824329in}}%
\pgfpathlineto{\pgfqpoint{3.033049in}{2.540376in}}%
\pgfusepath{stroke}%
\end{pgfscope}%
\begin{pgfscope}%
\pgfpathrectangle{\pgfqpoint{0.100000in}{0.212622in}}{\pgfqpoint{3.696000in}{3.696000in}}%
\pgfusepath{clip}%
\pgfsetrectcap%
\pgfsetroundjoin%
\pgfsetlinewidth{1.505625pt}%
\definecolor{currentstroke}{rgb}{1.000000,0.000000,0.000000}%
\pgfsetstrokecolor{currentstroke}%
\pgfsetdash{}{0pt}%
\pgfpathmoveto{\pgfqpoint{3.130904in}{1.823395in}}%
\pgfpathlineto{\pgfqpoint{3.033049in}{2.540376in}}%
\pgfusepath{stroke}%
\end{pgfscope}%
\begin{pgfscope}%
\pgfpathrectangle{\pgfqpoint{0.100000in}{0.212622in}}{\pgfqpoint{3.696000in}{3.696000in}}%
\pgfusepath{clip}%
\pgfsetrectcap%
\pgfsetroundjoin%
\pgfsetlinewidth{1.505625pt}%
\definecolor{currentstroke}{rgb}{1.000000,0.000000,0.000000}%
\pgfsetstrokecolor{currentstroke}%
\pgfsetdash{}{0pt}%
\pgfpathmoveto{\pgfqpoint{3.137639in}{1.821755in}}%
\pgfpathlineto{\pgfqpoint{3.048168in}{2.536280in}}%
\pgfusepath{stroke}%
\end{pgfscope}%
\begin{pgfscope}%
\pgfpathrectangle{\pgfqpoint{0.100000in}{0.212622in}}{\pgfqpoint{3.696000in}{3.696000in}}%
\pgfusepath{clip}%
\pgfsetrectcap%
\pgfsetroundjoin%
\pgfsetlinewidth{1.505625pt}%
\definecolor{currentstroke}{rgb}{1.000000,0.000000,0.000000}%
\pgfsetstrokecolor{currentstroke}%
\pgfsetdash{}{0pt}%
\pgfpathmoveto{\pgfqpoint{3.140904in}{1.821172in}}%
\pgfpathlineto{\pgfqpoint{3.048168in}{2.536280in}}%
\pgfusepath{stroke}%
\end{pgfscope}%
\begin{pgfscope}%
\pgfpathrectangle{\pgfqpoint{0.100000in}{0.212622in}}{\pgfqpoint{3.696000in}{3.696000in}}%
\pgfusepath{clip}%
\pgfsetrectcap%
\pgfsetroundjoin%
\pgfsetlinewidth{1.505625pt}%
\definecolor{currentstroke}{rgb}{1.000000,0.000000,0.000000}%
\pgfsetstrokecolor{currentstroke}%
\pgfsetdash{}{0pt}%
\pgfpathmoveto{\pgfqpoint{3.145275in}{1.820603in}}%
\pgfpathlineto{\pgfqpoint{3.048168in}{2.536280in}}%
\pgfusepath{stroke}%
\end{pgfscope}%
\begin{pgfscope}%
\pgfpathrectangle{\pgfqpoint{0.100000in}{0.212622in}}{\pgfqpoint{3.696000in}{3.696000in}}%
\pgfusepath{clip}%
\pgfsetrectcap%
\pgfsetroundjoin%
\pgfsetlinewidth{1.505625pt}%
\definecolor{currentstroke}{rgb}{1.000000,0.000000,0.000000}%
\pgfsetstrokecolor{currentstroke}%
\pgfsetdash{}{0pt}%
\pgfpathmoveto{\pgfqpoint{3.150878in}{1.819310in}}%
\pgfpathlineto{\pgfqpoint{3.063299in}{2.532181in}}%
\pgfusepath{stroke}%
\end{pgfscope}%
\begin{pgfscope}%
\pgfpathrectangle{\pgfqpoint{0.100000in}{0.212622in}}{\pgfqpoint{3.696000in}{3.696000in}}%
\pgfusepath{clip}%
\pgfsetrectcap%
\pgfsetroundjoin%
\pgfsetlinewidth{1.505625pt}%
\definecolor{currentstroke}{rgb}{1.000000,0.000000,0.000000}%
\pgfsetstrokecolor{currentstroke}%
\pgfsetdash{}{0pt}%
\pgfpathmoveto{\pgfqpoint{3.153984in}{1.818287in}}%
\pgfpathlineto{\pgfqpoint{3.063299in}{2.532181in}}%
\pgfusepath{stroke}%
\end{pgfscope}%
\begin{pgfscope}%
\pgfpathrectangle{\pgfqpoint{0.100000in}{0.212622in}}{\pgfqpoint{3.696000in}{3.696000in}}%
\pgfusepath{clip}%
\pgfsetrectcap%
\pgfsetroundjoin%
\pgfsetlinewidth{1.505625pt}%
\definecolor{currentstroke}{rgb}{1.000000,0.000000,0.000000}%
\pgfsetstrokecolor{currentstroke}%
\pgfsetdash{}{0pt}%
\pgfpathmoveto{\pgfqpoint{3.156755in}{1.818151in}}%
\pgfpathlineto{\pgfqpoint{3.063299in}{2.532181in}}%
\pgfusepath{stroke}%
\end{pgfscope}%
\begin{pgfscope}%
\pgfpathrectangle{\pgfqpoint{0.100000in}{0.212622in}}{\pgfqpoint{3.696000in}{3.696000in}}%
\pgfusepath{clip}%
\pgfsetrectcap%
\pgfsetroundjoin%
\pgfsetlinewidth{1.505625pt}%
\definecolor{currentstroke}{rgb}{1.000000,0.000000,0.000000}%
\pgfsetstrokecolor{currentstroke}%
\pgfsetdash{}{0pt}%
\pgfpathmoveto{\pgfqpoint{3.161725in}{1.816740in}}%
\pgfpathlineto{\pgfqpoint{3.063299in}{2.532181in}}%
\pgfusepath{stroke}%
\end{pgfscope}%
\begin{pgfscope}%
\pgfpathrectangle{\pgfqpoint{0.100000in}{0.212622in}}{\pgfqpoint{3.696000in}{3.696000in}}%
\pgfusepath{clip}%
\pgfsetrectcap%
\pgfsetroundjoin%
\pgfsetlinewidth{1.505625pt}%
\definecolor{currentstroke}{rgb}{1.000000,0.000000,0.000000}%
\pgfsetstrokecolor{currentstroke}%
\pgfsetdash{}{0pt}%
\pgfpathmoveto{\pgfqpoint{3.167285in}{1.815502in}}%
\pgfpathlineto{\pgfqpoint{3.078440in}{2.528079in}}%
\pgfusepath{stroke}%
\end{pgfscope}%
\begin{pgfscope}%
\pgfpathrectangle{\pgfqpoint{0.100000in}{0.212622in}}{\pgfqpoint{3.696000in}{3.696000in}}%
\pgfusepath{clip}%
\pgfsetrectcap%
\pgfsetroundjoin%
\pgfsetlinewidth{1.505625pt}%
\definecolor{currentstroke}{rgb}{1.000000,0.000000,0.000000}%
\pgfsetstrokecolor{currentstroke}%
\pgfsetdash{}{0pt}%
\pgfpathmoveto{\pgfqpoint{3.170298in}{1.814779in}}%
\pgfpathlineto{\pgfqpoint{3.078440in}{2.528079in}}%
\pgfusepath{stroke}%
\end{pgfscope}%
\begin{pgfscope}%
\pgfpathrectangle{\pgfqpoint{0.100000in}{0.212622in}}{\pgfqpoint{3.696000in}{3.696000in}}%
\pgfusepath{clip}%
\pgfsetrectcap%
\pgfsetroundjoin%
\pgfsetlinewidth{1.505625pt}%
\definecolor{currentstroke}{rgb}{1.000000,0.000000,0.000000}%
\pgfsetstrokecolor{currentstroke}%
\pgfsetdash{}{0pt}%
\pgfpathmoveto{\pgfqpoint{3.174682in}{1.813659in}}%
\pgfpathlineto{\pgfqpoint{3.078440in}{2.528079in}}%
\pgfusepath{stroke}%
\end{pgfscope}%
\begin{pgfscope}%
\pgfpathrectangle{\pgfqpoint{0.100000in}{0.212622in}}{\pgfqpoint{3.696000in}{3.696000in}}%
\pgfusepath{clip}%
\pgfsetrectcap%
\pgfsetroundjoin%
\pgfsetlinewidth{1.505625pt}%
\definecolor{currentstroke}{rgb}{1.000000,0.000000,0.000000}%
\pgfsetstrokecolor{currentstroke}%
\pgfsetdash{}{0pt}%
\pgfpathmoveto{\pgfqpoint{3.179744in}{1.812360in}}%
\pgfpathlineto{\pgfqpoint{3.093593in}{2.523974in}}%
\pgfusepath{stroke}%
\end{pgfscope}%
\begin{pgfscope}%
\pgfpathrectangle{\pgfqpoint{0.100000in}{0.212622in}}{\pgfqpoint{3.696000in}{3.696000in}}%
\pgfusepath{clip}%
\pgfsetrectcap%
\pgfsetroundjoin%
\pgfsetlinewidth{1.505625pt}%
\definecolor{currentstroke}{rgb}{1.000000,0.000000,0.000000}%
\pgfsetstrokecolor{currentstroke}%
\pgfsetdash{}{0pt}%
\pgfpathmoveto{\pgfqpoint{3.182335in}{1.811846in}}%
\pgfpathlineto{\pgfqpoint{3.093593in}{2.523974in}}%
\pgfusepath{stroke}%
\end{pgfscope}%
\begin{pgfscope}%
\pgfpathrectangle{\pgfqpoint{0.100000in}{0.212622in}}{\pgfqpoint{3.696000in}{3.696000in}}%
\pgfusepath{clip}%
\pgfsetrectcap%
\pgfsetroundjoin%
\pgfsetlinewidth{1.505625pt}%
\definecolor{currentstroke}{rgb}{1.000000,0.000000,0.000000}%
\pgfsetstrokecolor{currentstroke}%
\pgfsetdash{}{0pt}%
\pgfpathmoveto{\pgfqpoint{3.186556in}{1.810874in}}%
\pgfpathlineto{\pgfqpoint{3.093593in}{2.523974in}}%
\pgfusepath{stroke}%
\end{pgfscope}%
\begin{pgfscope}%
\pgfpathrectangle{\pgfqpoint{0.100000in}{0.212622in}}{\pgfqpoint{3.696000in}{3.696000in}}%
\pgfusepath{clip}%
\pgfsetrectcap%
\pgfsetroundjoin%
\pgfsetlinewidth{1.505625pt}%
\definecolor{currentstroke}{rgb}{1.000000,0.000000,0.000000}%
\pgfsetstrokecolor{currentstroke}%
\pgfsetdash{}{0pt}%
\pgfpathmoveto{\pgfqpoint{3.188717in}{1.810486in}}%
\pgfpathlineto{\pgfqpoint{3.093593in}{2.523974in}}%
\pgfusepath{stroke}%
\end{pgfscope}%
\begin{pgfscope}%
\pgfpathrectangle{\pgfqpoint{0.100000in}{0.212622in}}{\pgfqpoint{3.696000in}{3.696000in}}%
\pgfusepath{clip}%
\pgfsetrectcap%
\pgfsetroundjoin%
\pgfsetlinewidth{1.505625pt}%
\definecolor{currentstroke}{rgb}{1.000000,0.000000,0.000000}%
\pgfsetstrokecolor{currentstroke}%
\pgfsetdash{}{0pt}%
\pgfpathmoveto{\pgfqpoint{3.191285in}{1.810134in}}%
\pgfpathlineto{\pgfqpoint{3.093593in}{2.523974in}}%
\pgfusepath{stroke}%
\end{pgfscope}%
\begin{pgfscope}%
\pgfpathrectangle{\pgfqpoint{0.100000in}{0.212622in}}{\pgfqpoint{3.696000in}{3.696000in}}%
\pgfusepath{clip}%
\pgfsetrectcap%
\pgfsetroundjoin%
\pgfsetlinewidth{1.505625pt}%
\definecolor{currentstroke}{rgb}{1.000000,0.000000,0.000000}%
\pgfsetstrokecolor{currentstroke}%
\pgfsetdash{}{0pt}%
\pgfpathmoveto{\pgfqpoint{3.194606in}{1.809399in}}%
\pgfpathlineto{\pgfqpoint{3.108757in}{2.519866in}}%
\pgfusepath{stroke}%
\end{pgfscope}%
\begin{pgfscope}%
\pgfpathrectangle{\pgfqpoint{0.100000in}{0.212622in}}{\pgfqpoint{3.696000in}{3.696000in}}%
\pgfusepath{clip}%
\pgfsetrectcap%
\pgfsetroundjoin%
\pgfsetlinewidth{1.505625pt}%
\definecolor{currentstroke}{rgb}{1.000000,0.000000,0.000000}%
\pgfsetstrokecolor{currentstroke}%
\pgfsetdash{}{0pt}%
\pgfpathmoveto{\pgfqpoint{3.198294in}{1.808840in}}%
\pgfpathlineto{\pgfqpoint{3.108757in}{2.519866in}}%
\pgfusepath{stroke}%
\end{pgfscope}%
\begin{pgfscope}%
\pgfpathrectangle{\pgfqpoint{0.100000in}{0.212622in}}{\pgfqpoint{3.696000in}{3.696000in}}%
\pgfusepath{clip}%
\pgfsetrectcap%
\pgfsetroundjoin%
\pgfsetlinewidth{1.505625pt}%
\definecolor{currentstroke}{rgb}{1.000000,0.000000,0.000000}%
\pgfsetstrokecolor{currentstroke}%
\pgfsetdash{}{0pt}%
\pgfpathmoveto{\pgfqpoint{3.200359in}{1.808489in}}%
\pgfpathlineto{\pgfqpoint{3.108757in}{2.519866in}}%
\pgfusepath{stroke}%
\end{pgfscope}%
\begin{pgfscope}%
\pgfpathrectangle{\pgfqpoint{0.100000in}{0.212622in}}{\pgfqpoint{3.696000in}{3.696000in}}%
\pgfusepath{clip}%
\pgfsetrectcap%
\pgfsetroundjoin%
\pgfsetlinewidth{1.505625pt}%
\definecolor{currentstroke}{rgb}{1.000000,0.000000,0.000000}%
\pgfsetstrokecolor{currentstroke}%
\pgfsetdash{}{0pt}%
\pgfpathmoveto{\pgfqpoint{3.203123in}{1.807821in}}%
\pgfpathlineto{\pgfqpoint{3.108757in}{2.519866in}}%
\pgfusepath{stroke}%
\end{pgfscope}%
\begin{pgfscope}%
\pgfpathrectangle{\pgfqpoint{0.100000in}{0.212622in}}{\pgfqpoint{3.696000in}{3.696000in}}%
\pgfusepath{clip}%
\pgfsetrectcap%
\pgfsetroundjoin%
\pgfsetlinewidth{1.505625pt}%
\definecolor{currentstroke}{rgb}{1.000000,0.000000,0.000000}%
\pgfsetstrokecolor{currentstroke}%
\pgfsetdash{}{0pt}%
\pgfpathmoveto{\pgfqpoint{3.205904in}{1.807495in}}%
\pgfpathlineto{\pgfqpoint{3.108757in}{2.519866in}}%
\pgfusepath{stroke}%
\end{pgfscope}%
\begin{pgfscope}%
\pgfpathrectangle{\pgfqpoint{0.100000in}{0.212622in}}{\pgfqpoint{3.696000in}{3.696000in}}%
\pgfusepath{clip}%
\pgfsetrectcap%
\pgfsetroundjoin%
\pgfsetlinewidth{1.505625pt}%
\definecolor{currentstroke}{rgb}{1.000000,0.000000,0.000000}%
\pgfsetstrokecolor{currentstroke}%
\pgfsetdash{}{0pt}%
\pgfpathmoveto{\pgfqpoint{3.209984in}{1.806477in}}%
\pgfpathlineto{\pgfqpoint{3.108757in}{2.519866in}}%
\pgfusepath{stroke}%
\end{pgfscope}%
\begin{pgfscope}%
\pgfpathrectangle{\pgfqpoint{0.100000in}{0.212622in}}{\pgfqpoint{3.696000in}{3.696000in}}%
\pgfusepath{clip}%
\pgfsetrectcap%
\pgfsetroundjoin%
\pgfsetlinewidth{1.505625pt}%
\definecolor{currentstroke}{rgb}{1.000000,0.000000,0.000000}%
\pgfsetstrokecolor{currentstroke}%
\pgfsetdash{}{0pt}%
\pgfpathmoveto{\pgfqpoint{3.214681in}{1.805394in}}%
\pgfpathlineto{\pgfqpoint{3.108757in}{2.519866in}}%
\pgfusepath{stroke}%
\end{pgfscope}%
\begin{pgfscope}%
\pgfpathrectangle{\pgfqpoint{0.100000in}{0.212622in}}{\pgfqpoint{3.696000in}{3.696000in}}%
\pgfusepath{clip}%
\pgfsetrectcap%
\pgfsetroundjoin%
\pgfsetlinewidth{1.505625pt}%
\definecolor{currentstroke}{rgb}{1.000000,0.000000,0.000000}%
\pgfsetstrokecolor{currentstroke}%
\pgfsetdash{}{0pt}%
\pgfpathmoveto{\pgfqpoint{3.220105in}{1.804564in}}%
\pgfpathlineto{\pgfqpoint{3.108757in}{2.519866in}}%
\pgfusepath{stroke}%
\end{pgfscope}%
\begin{pgfscope}%
\pgfpathrectangle{\pgfqpoint{0.100000in}{0.212622in}}{\pgfqpoint{3.696000in}{3.696000in}}%
\pgfusepath{clip}%
\pgfsetrectcap%
\pgfsetroundjoin%
\pgfsetlinewidth{1.505625pt}%
\definecolor{currentstroke}{rgb}{1.000000,0.000000,0.000000}%
\pgfsetstrokecolor{currentstroke}%
\pgfsetdash{}{0pt}%
\pgfpathmoveto{\pgfqpoint{3.228185in}{1.802270in}}%
\pgfpathlineto{\pgfqpoint{3.108757in}{2.519866in}}%
\pgfusepath{stroke}%
\end{pgfscope}%
\begin{pgfscope}%
\pgfpathrectangle{\pgfqpoint{0.100000in}{0.212622in}}{\pgfqpoint{3.696000in}{3.696000in}}%
\pgfusepath{clip}%
\pgfsetrectcap%
\pgfsetroundjoin%
\pgfsetlinewidth{1.505625pt}%
\definecolor{currentstroke}{rgb}{1.000000,0.000000,0.000000}%
\pgfsetstrokecolor{currentstroke}%
\pgfsetdash{}{0pt}%
\pgfpathmoveto{\pgfqpoint{3.237507in}{1.799957in}}%
\pgfpathlineto{\pgfqpoint{3.108757in}{2.519866in}}%
\pgfusepath{stroke}%
\end{pgfscope}%
\begin{pgfscope}%
\pgfpathrectangle{\pgfqpoint{0.100000in}{0.212622in}}{\pgfqpoint{3.696000in}{3.696000in}}%
\pgfusepath{clip}%
\pgfsetrectcap%
\pgfsetroundjoin%
\pgfsetlinewidth{1.505625pt}%
\definecolor{currentstroke}{rgb}{1.000000,0.000000,0.000000}%
\pgfsetstrokecolor{currentstroke}%
\pgfsetdash{}{0pt}%
\pgfpathmoveto{\pgfqpoint{3.246604in}{1.798470in}}%
\pgfpathlineto{\pgfqpoint{3.108757in}{2.519866in}}%
\pgfusepath{stroke}%
\end{pgfscope}%
\begin{pgfscope}%
\pgfpathrectangle{\pgfqpoint{0.100000in}{0.212622in}}{\pgfqpoint{3.696000in}{3.696000in}}%
\pgfusepath{clip}%
\pgfsetrectcap%
\pgfsetroundjoin%
\pgfsetlinewidth{1.505625pt}%
\definecolor{currentstroke}{rgb}{1.000000,0.000000,0.000000}%
\pgfsetstrokecolor{currentstroke}%
\pgfsetdash{}{0pt}%
\pgfpathmoveto{\pgfqpoint{3.256623in}{1.796404in}}%
\pgfpathlineto{\pgfqpoint{3.108757in}{2.519866in}}%
\pgfusepath{stroke}%
\end{pgfscope}%
\begin{pgfscope}%
\pgfpathrectangle{\pgfqpoint{0.100000in}{0.212622in}}{\pgfqpoint{3.696000in}{3.696000in}}%
\pgfusepath{clip}%
\pgfsetrectcap%
\pgfsetroundjoin%
\pgfsetlinewidth{1.505625pt}%
\definecolor{currentstroke}{rgb}{1.000000,0.000000,0.000000}%
\pgfsetstrokecolor{currentstroke}%
\pgfsetdash{}{0pt}%
\pgfpathmoveto{\pgfqpoint{3.266805in}{1.794687in}}%
\pgfpathlineto{\pgfqpoint{3.108757in}{2.519866in}}%
\pgfusepath{stroke}%
\end{pgfscope}%
\begin{pgfscope}%
\pgfpathrectangle{\pgfqpoint{0.100000in}{0.212622in}}{\pgfqpoint{3.696000in}{3.696000in}}%
\pgfusepath{clip}%
\pgfsetrectcap%
\pgfsetroundjoin%
\pgfsetlinewidth{1.505625pt}%
\definecolor{currentstroke}{rgb}{1.000000,0.000000,0.000000}%
\pgfsetstrokecolor{currentstroke}%
\pgfsetdash{}{0pt}%
\pgfpathmoveto{\pgfqpoint{3.278143in}{1.792767in}}%
\pgfpathlineto{\pgfqpoint{3.108757in}{2.519866in}}%
\pgfusepath{stroke}%
\end{pgfscope}%
\begin{pgfscope}%
\pgfpathrectangle{\pgfqpoint{0.100000in}{0.212622in}}{\pgfqpoint{3.696000in}{3.696000in}}%
\pgfusepath{clip}%
\pgfsetrectcap%
\pgfsetroundjoin%
\pgfsetlinewidth{1.505625pt}%
\definecolor{currentstroke}{rgb}{1.000000,0.000000,0.000000}%
\pgfsetstrokecolor{currentstroke}%
\pgfsetdash{}{0pt}%
\pgfpathmoveto{\pgfqpoint{3.284378in}{1.791471in}}%
\pgfpathlineto{\pgfqpoint{3.108757in}{2.519866in}}%
\pgfusepath{stroke}%
\end{pgfscope}%
\begin{pgfscope}%
\pgfpathrectangle{\pgfqpoint{0.100000in}{0.212622in}}{\pgfqpoint{3.696000in}{3.696000in}}%
\pgfusepath{clip}%
\pgfsetrectcap%
\pgfsetroundjoin%
\pgfsetlinewidth{1.505625pt}%
\definecolor{currentstroke}{rgb}{1.000000,0.000000,0.000000}%
\pgfsetstrokecolor{currentstroke}%
\pgfsetdash{}{0pt}%
\pgfpathmoveto{\pgfqpoint{3.291017in}{1.790143in}}%
\pgfpathlineto{\pgfqpoint{3.108757in}{2.519866in}}%
\pgfusepath{stroke}%
\end{pgfscope}%
\begin{pgfscope}%
\pgfpathrectangle{\pgfqpoint{0.100000in}{0.212622in}}{\pgfqpoint{3.696000in}{3.696000in}}%
\pgfusepath{clip}%
\pgfsetrectcap%
\pgfsetroundjoin%
\pgfsetlinewidth{1.505625pt}%
\definecolor{currentstroke}{rgb}{1.000000,0.000000,0.000000}%
\pgfsetstrokecolor{currentstroke}%
\pgfsetdash{}{0pt}%
\pgfpathmoveto{\pgfqpoint{3.298265in}{1.788858in}}%
\pgfpathlineto{\pgfqpoint{3.108757in}{2.519866in}}%
\pgfusepath{stroke}%
\end{pgfscope}%
\begin{pgfscope}%
\pgfpathrectangle{\pgfqpoint{0.100000in}{0.212622in}}{\pgfqpoint{3.696000in}{3.696000in}}%
\pgfusepath{clip}%
\pgfsetrectcap%
\pgfsetroundjoin%
\pgfsetlinewidth{1.505625pt}%
\definecolor{currentstroke}{rgb}{1.000000,0.000000,0.000000}%
\pgfsetstrokecolor{currentstroke}%
\pgfsetdash{}{0pt}%
\pgfpathmoveto{\pgfqpoint{3.305615in}{1.787841in}}%
\pgfpathlineto{\pgfqpoint{3.108757in}{2.519866in}}%
\pgfusepath{stroke}%
\end{pgfscope}%
\begin{pgfscope}%
\pgfpathrectangle{\pgfqpoint{0.100000in}{0.212622in}}{\pgfqpoint{3.696000in}{3.696000in}}%
\pgfusepath{clip}%
\pgfsetrectcap%
\pgfsetroundjoin%
\pgfsetlinewidth{1.505625pt}%
\definecolor{currentstroke}{rgb}{1.000000,0.000000,0.000000}%
\pgfsetstrokecolor{currentstroke}%
\pgfsetdash{}{0pt}%
\pgfpathmoveto{\pgfqpoint{3.309660in}{1.787257in}}%
\pgfpathlineto{\pgfqpoint{3.108757in}{2.519866in}}%
\pgfusepath{stroke}%
\end{pgfscope}%
\begin{pgfscope}%
\pgfpathrectangle{\pgfqpoint{0.100000in}{0.212622in}}{\pgfqpoint{3.696000in}{3.696000in}}%
\pgfusepath{clip}%
\pgfsetrectcap%
\pgfsetroundjoin%
\pgfsetlinewidth{1.505625pt}%
\definecolor{currentstroke}{rgb}{1.000000,0.000000,0.000000}%
\pgfsetstrokecolor{currentstroke}%
\pgfsetdash{}{0pt}%
\pgfpathmoveto{\pgfqpoint{3.314090in}{1.786490in}}%
\pgfpathlineto{\pgfqpoint{3.108757in}{2.519866in}}%
\pgfusepath{stroke}%
\end{pgfscope}%
\begin{pgfscope}%
\pgfpathrectangle{\pgfqpoint{0.100000in}{0.212622in}}{\pgfqpoint{3.696000in}{3.696000in}}%
\pgfusepath{clip}%
\pgfsetrectcap%
\pgfsetroundjoin%
\pgfsetlinewidth{1.505625pt}%
\definecolor{currentstroke}{rgb}{1.000000,0.000000,0.000000}%
\pgfsetstrokecolor{currentstroke}%
\pgfsetdash{}{0pt}%
\pgfpathmoveto{\pgfqpoint{3.316770in}{1.785929in}}%
\pgfpathlineto{\pgfqpoint{3.108757in}{2.519866in}}%
\pgfusepath{stroke}%
\end{pgfscope}%
\begin{pgfscope}%
\pgfpathrectangle{\pgfqpoint{0.100000in}{0.212622in}}{\pgfqpoint{3.696000in}{3.696000in}}%
\pgfusepath{clip}%
\pgfsetrectcap%
\pgfsetroundjoin%
\pgfsetlinewidth{1.505625pt}%
\definecolor{currentstroke}{rgb}{1.000000,0.000000,0.000000}%
\pgfsetstrokecolor{currentstroke}%
\pgfsetdash{}{0pt}%
\pgfpathmoveto{\pgfqpoint{3.318134in}{1.785649in}}%
\pgfpathlineto{\pgfqpoint{3.108757in}{2.519866in}}%
\pgfusepath{stroke}%
\end{pgfscope}%
\begin{pgfscope}%
\pgfpathrectangle{\pgfqpoint{0.100000in}{0.212622in}}{\pgfqpoint{3.696000in}{3.696000in}}%
\pgfusepath{clip}%
\pgfsetrectcap%
\pgfsetroundjoin%
\pgfsetlinewidth{1.505625pt}%
\definecolor{currentstroke}{rgb}{1.000000,0.000000,0.000000}%
\pgfsetstrokecolor{currentstroke}%
\pgfsetdash{}{0pt}%
\pgfpathmoveto{\pgfqpoint{3.318846in}{1.785507in}}%
\pgfpathlineto{\pgfqpoint{3.108757in}{2.519866in}}%
\pgfusepath{stroke}%
\end{pgfscope}%
\begin{pgfscope}%
\pgfpathrectangle{\pgfqpoint{0.100000in}{0.212622in}}{\pgfqpoint{3.696000in}{3.696000in}}%
\pgfusepath{clip}%
\pgfsetrectcap%
\pgfsetroundjoin%
\pgfsetlinewidth{1.505625pt}%
\definecolor{currentstroke}{rgb}{1.000000,0.000000,0.000000}%
\pgfsetstrokecolor{currentstroke}%
\pgfsetdash{}{0pt}%
\pgfpathmoveto{\pgfqpoint{3.319254in}{1.785431in}}%
\pgfpathlineto{\pgfqpoint{3.108757in}{2.519866in}}%
\pgfusepath{stroke}%
\end{pgfscope}%
\begin{pgfscope}%
\pgfpathrectangle{\pgfqpoint{0.100000in}{0.212622in}}{\pgfqpoint{3.696000in}{3.696000in}}%
\pgfusepath{clip}%
\pgfsetrectcap%
\pgfsetroundjoin%
\pgfsetlinewidth{1.505625pt}%
\definecolor{currentstroke}{rgb}{1.000000,0.000000,0.000000}%
\pgfsetstrokecolor{currentstroke}%
\pgfsetdash{}{0pt}%
\pgfpathmoveto{\pgfqpoint{3.319473in}{1.785391in}}%
\pgfpathlineto{\pgfqpoint{3.108757in}{2.519866in}}%
\pgfusepath{stroke}%
\end{pgfscope}%
\begin{pgfscope}%
\pgfpathrectangle{\pgfqpoint{0.100000in}{0.212622in}}{\pgfqpoint{3.696000in}{3.696000in}}%
\pgfusepath{clip}%
\pgfsetrectcap%
\pgfsetroundjoin%
\pgfsetlinewidth{1.505625pt}%
\definecolor{currentstroke}{rgb}{1.000000,0.000000,0.000000}%
\pgfsetstrokecolor{currentstroke}%
\pgfsetdash{}{0pt}%
\pgfpathmoveto{\pgfqpoint{3.319596in}{1.785370in}}%
\pgfpathlineto{\pgfqpoint{3.108757in}{2.519866in}}%
\pgfusepath{stroke}%
\end{pgfscope}%
\begin{pgfscope}%
\pgfpathrectangle{\pgfqpoint{0.100000in}{0.212622in}}{\pgfqpoint{3.696000in}{3.696000in}}%
\pgfusepath{clip}%
\pgfsetrectcap%
\pgfsetroundjoin%
\pgfsetlinewidth{1.505625pt}%
\definecolor{currentstroke}{rgb}{1.000000,0.000000,0.000000}%
\pgfsetstrokecolor{currentstroke}%
\pgfsetdash{}{0pt}%
\pgfpathmoveto{\pgfqpoint{3.319668in}{1.785356in}}%
\pgfpathlineto{\pgfqpoint{3.108757in}{2.519866in}}%
\pgfusepath{stroke}%
\end{pgfscope}%
\begin{pgfscope}%
\pgfpathrectangle{\pgfqpoint{0.100000in}{0.212622in}}{\pgfqpoint{3.696000in}{3.696000in}}%
\pgfusepath{clip}%
\pgfsetrectcap%
\pgfsetroundjoin%
\pgfsetlinewidth{1.505625pt}%
\definecolor{currentstroke}{rgb}{1.000000,0.000000,0.000000}%
\pgfsetstrokecolor{currentstroke}%
\pgfsetdash{}{0pt}%
\pgfpathmoveto{\pgfqpoint{3.319706in}{1.785349in}}%
\pgfpathlineto{\pgfqpoint{3.108757in}{2.519866in}}%
\pgfusepath{stroke}%
\end{pgfscope}%
\begin{pgfscope}%
\pgfpathrectangle{\pgfqpoint{0.100000in}{0.212622in}}{\pgfqpoint{3.696000in}{3.696000in}}%
\pgfusepath{clip}%
\pgfsetrectcap%
\pgfsetroundjoin%
\pgfsetlinewidth{1.505625pt}%
\definecolor{currentstroke}{rgb}{1.000000,0.000000,0.000000}%
\pgfsetstrokecolor{currentstroke}%
\pgfsetdash{}{0pt}%
\pgfpathmoveto{\pgfqpoint{3.319726in}{1.785345in}}%
\pgfpathlineto{\pgfqpoint{3.108757in}{2.519866in}}%
\pgfusepath{stroke}%
\end{pgfscope}%
\begin{pgfscope}%
\pgfpathrectangle{\pgfqpoint{0.100000in}{0.212622in}}{\pgfqpoint{3.696000in}{3.696000in}}%
\pgfusepath{clip}%
\pgfsetrectcap%
\pgfsetroundjoin%
\pgfsetlinewidth{1.505625pt}%
\definecolor{currentstroke}{rgb}{1.000000,0.000000,0.000000}%
\pgfsetstrokecolor{currentstroke}%
\pgfsetdash{}{0pt}%
\pgfpathmoveto{\pgfqpoint{3.319735in}{1.785343in}}%
\pgfpathlineto{\pgfqpoint{3.108757in}{2.519866in}}%
\pgfusepath{stroke}%
\end{pgfscope}%
\begin{pgfscope}%
\pgfpathrectangle{\pgfqpoint{0.100000in}{0.212622in}}{\pgfqpoint{3.696000in}{3.696000in}}%
\pgfusepath{clip}%
\pgfsetrectcap%
\pgfsetroundjoin%
\pgfsetlinewidth{1.505625pt}%
\definecolor{currentstroke}{rgb}{1.000000,0.000000,0.000000}%
\pgfsetstrokecolor{currentstroke}%
\pgfsetdash{}{0pt}%
\pgfpathmoveto{\pgfqpoint{3.319738in}{1.785343in}}%
\pgfpathlineto{\pgfqpoint{3.108757in}{2.519866in}}%
\pgfusepath{stroke}%
\end{pgfscope}%
\begin{pgfscope}%
\pgfpathrectangle{\pgfqpoint{0.100000in}{0.212622in}}{\pgfqpoint{3.696000in}{3.696000in}}%
\pgfusepath{clip}%
\pgfsetrectcap%
\pgfsetroundjoin%
\pgfsetlinewidth{1.505625pt}%
\definecolor{currentstroke}{rgb}{1.000000,0.000000,0.000000}%
\pgfsetstrokecolor{currentstroke}%
\pgfsetdash{}{0pt}%
\pgfpathmoveto{\pgfqpoint{3.319819in}{1.785340in}}%
\pgfpathlineto{\pgfqpoint{3.108757in}{2.519866in}}%
\pgfusepath{stroke}%
\end{pgfscope}%
\begin{pgfscope}%
\pgfpathrectangle{\pgfqpoint{0.100000in}{0.212622in}}{\pgfqpoint{3.696000in}{3.696000in}}%
\pgfusepath{clip}%
\pgfsetrectcap%
\pgfsetroundjoin%
\pgfsetlinewidth{1.505625pt}%
\definecolor{currentstroke}{rgb}{1.000000,0.000000,0.000000}%
\pgfsetstrokecolor{currentstroke}%
\pgfsetdash{}{0pt}%
\pgfpathmoveto{\pgfqpoint{3.319852in}{1.785335in}}%
\pgfpathlineto{\pgfqpoint{3.108757in}{2.519866in}}%
\pgfusepath{stroke}%
\end{pgfscope}%
\begin{pgfscope}%
\pgfpathrectangle{\pgfqpoint{0.100000in}{0.212622in}}{\pgfqpoint{3.696000in}{3.696000in}}%
\pgfusepath{clip}%
\pgfsetrectcap%
\pgfsetroundjoin%
\pgfsetlinewidth{1.505625pt}%
\definecolor{currentstroke}{rgb}{1.000000,0.000000,0.000000}%
\pgfsetstrokecolor{currentstroke}%
\pgfsetdash{}{0pt}%
\pgfpathmoveto{\pgfqpoint{3.319895in}{1.785307in}}%
\pgfpathlineto{\pgfqpoint{3.108757in}{2.519866in}}%
\pgfusepath{stroke}%
\end{pgfscope}%
\begin{pgfscope}%
\pgfpathrectangle{\pgfqpoint{0.100000in}{0.212622in}}{\pgfqpoint{3.696000in}{3.696000in}}%
\pgfusepath{clip}%
\pgfsetrectcap%
\pgfsetroundjoin%
\pgfsetlinewidth{1.505625pt}%
\definecolor{currentstroke}{rgb}{1.000000,0.000000,0.000000}%
\pgfsetstrokecolor{currentstroke}%
\pgfsetdash{}{0pt}%
\pgfpathmoveto{\pgfqpoint{3.319891in}{1.785182in}}%
\pgfpathlineto{\pgfqpoint{3.108757in}{2.519866in}}%
\pgfusepath{stroke}%
\end{pgfscope}%
\begin{pgfscope}%
\pgfpathrectangle{\pgfqpoint{0.100000in}{0.212622in}}{\pgfqpoint{3.696000in}{3.696000in}}%
\pgfusepath{clip}%
\pgfsetrectcap%
\pgfsetroundjoin%
\pgfsetlinewidth{1.505625pt}%
\definecolor{currentstroke}{rgb}{1.000000,0.000000,0.000000}%
\pgfsetstrokecolor{currentstroke}%
\pgfsetdash{}{0pt}%
\pgfpathmoveto{\pgfqpoint{3.319505in}{1.785102in}}%
\pgfpathlineto{\pgfqpoint{3.108757in}{2.519866in}}%
\pgfusepath{stroke}%
\end{pgfscope}%
\begin{pgfscope}%
\pgfpathrectangle{\pgfqpoint{0.100000in}{0.212622in}}{\pgfqpoint{3.696000in}{3.696000in}}%
\pgfusepath{clip}%
\pgfsetrectcap%
\pgfsetroundjoin%
\pgfsetlinewidth{1.505625pt}%
\definecolor{currentstroke}{rgb}{1.000000,0.000000,0.000000}%
\pgfsetstrokecolor{currentstroke}%
\pgfsetdash{}{0pt}%
\pgfpathmoveto{\pgfqpoint{3.318606in}{1.785133in}}%
\pgfpathlineto{\pgfqpoint{3.108757in}{2.519866in}}%
\pgfusepath{stroke}%
\end{pgfscope}%
\begin{pgfscope}%
\pgfpathrectangle{\pgfqpoint{0.100000in}{0.212622in}}{\pgfqpoint{3.696000in}{3.696000in}}%
\pgfusepath{clip}%
\pgfsetrectcap%
\pgfsetroundjoin%
\pgfsetlinewidth{1.505625pt}%
\definecolor{currentstroke}{rgb}{1.000000,0.000000,0.000000}%
\pgfsetstrokecolor{currentstroke}%
\pgfsetdash{}{0pt}%
\pgfpathmoveto{\pgfqpoint{3.317302in}{1.785251in}}%
\pgfpathlineto{\pgfqpoint{3.108757in}{2.519866in}}%
\pgfusepath{stroke}%
\end{pgfscope}%
\begin{pgfscope}%
\pgfpathrectangle{\pgfqpoint{0.100000in}{0.212622in}}{\pgfqpoint{3.696000in}{3.696000in}}%
\pgfusepath{clip}%
\pgfsetrectcap%
\pgfsetroundjoin%
\pgfsetlinewidth{1.505625pt}%
\definecolor{currentstroke}{rgb}{1.000000,0.000000,0.000000}%
\pgfsetstrokecolor{currentstroke}%
\pgfsetdash{}{0pt}%
\pgfpathmoveto{\pgfqpoint{3.315603in}{1.785536in}}%
\pgfpathlineto{\pgfqpoint{3.108757in}{2.519866in}}%
\pgfusepath{stroke}%
\end{pgfscope}%
\begin{pgfscope}%
\pgfpathrectangle{\pgfqpoint{0.100000in}{0.212622in}}{\pgfqpoint{3.696000in}{3.696000in}}%
\pgfusepath{clip}%
\pgfsetrectcap%
\pgfsetroundjoin%
\pgfsetlinewidth{1.505625pt}%
\definecolor{currentstroke}{rgb}{1.000000,0.000000,0.000000}%
\pgfsetstrokecolor{currentstroke}%
\pgfsetdash{}{0pt}%
\pgfpathmoveto{\pgfqpoint{3.314972in}{1.785740in}}%
\pgfpathlineto{\pgfqpoint{3.108757in}{2.519866in}}%
\pgfusepath{stroke}%
\end{pgfscope}%
\begin{pgfscope}%
\pgfpathrectangle{\pgfqpoint{0.100000in}{0.212622in}}{\pgfqpoint{3.696000in}{3.696000in}}%
\pgfusepath{clip}%
\pgfsetrectcap%
\pgfsetroundjoin%
\pgfsetlinewidth{1.505625pt}%
\definecolor{currentstroke}{rgb}{1.000000,0.000000,0.000000}%
\pgfsetstrokecolor{currentstroke}%
\pgfsetdash{}{0pt}%
\pgfpathmoveto{\pgfqpoint{3.314458in}{1.785798in}}%
\pgfpathlineto{\pgfqpoint{3.108757in}{2.519866in}}%
\pgfusepath{stroke}%
\end{pgfscope}%
\begin{pgfscope}%
\pgfpathrectangle{\pgfqpoint{0.100000in}{0.212622in}}{\pgfqpoint{3.696000in}{3.696000in}}%
\pgfusepath{clip}%
\pgfsetrectcap%
\pgfsetroundjoin%
\pgfsetlinewidth{1.505625pt}%
\definecolor{currentstroke}{rgb}{1.000000,0.000000,0.000000}%
\pgfsetstrokecolor{currentstroke}%
\pgfsetdash{}{0pt}%
\pgfpathmoveto{\pgfqpoint{3.313711in}{1.785944in}}%
\pgfpathlineto{\pgfqpoint{3.108757in}{2.519866in}}%
\pgfusepath{stroke}%
\end{pgfscope}%
\begin{pgfscope}%
\pgfpathrectangle{\pgfqpoint{0.100000in}{0.212622in}}{\pgfqpoint{3.696000in}{3.696000in}}%
\pgfusepath{clip}%
\pgfsetrectcap%
\pgfsetroundjoin%
\pgfsetlinewidth{1.505625pt}%
\definecolor{currentstroke}{rgb}{1.000000,0.000000,0.000000}%
\pgfsetstrokecolor{currentstroke}%
\pgfsetdash{}{0pt}%
\pgfpathmoveto{\pgfqpoint{3.313311in}{1.786011in}}%
\pgfpathlineto{\pgfqpoint{3.108757in}{2.519866in}}%
\pgfusepath{stroke}%
\end{pgfscope}%
\begin{pgfscope}%
\pgfpathrectangle{\pgfqpoint{0.100000in}{0.212622in}}{\pgfqpoint{3.696000in}{3.696000in}}%
\pgfusepath{clip}%
\pgfsetrectcap%
\pgfsetroundjoin%
\pgfsetlinewidth{1.505625pt}%
\definecolor{currentstroke}{rgb}{1.000000,0.000000,0.000000}%
\pgfsetstrokecolor{currentstroke}%
\pgfsetdash{}{0pt}%
\pgfpathmoveto{\pgfqpoint{3.312378in}{1.786233in}}%
\pgfpathlineto{\pgfqpoint{3.108757in}{2.519866in}}%
\pgfusepath{stroke}%
\end{pgfscope}%
\begin{pgfscope}%
\pgfpathrectangle{\pgfqpoint{0.100000in}{0.212622in}}{\pgfqpoint{3.696000in}{3.696000in}}%
\pgfusepath{clip}%
\pgfsetrectcap%
\pgfsetroundjoin%
\pgfsetlinewidth{1.505625pt}%
\definecolor{currentstroke}{rgb}{1.000000,0.000000,0.000000}%
\pgfsetstrokecolor{currentstroke}%
\pgfsetdash{}{0pt}%
\pgfpathmoveto{\pgfqpoint{3.311911in}{1.786298in}}%
\pgfpathlineto{\pgfqpoint{3.108757in}{2.519866in}}%
\pgfusepath{stroke}%
\end{pgfscope}%
\begin{pgfscope}%
\pgfpathrectangle{\pgfqpoint{0.100000in}{0.212622in}}{\pgfqpoint{3.696000in}{3.696000in}}%
\pgfusepath{clip}%
\pgfsetrectcap%
\pgfsetroundjoin%
\pgfsetlinewidth{1.505625pt}%
\definecolor{currentstroke}{rgb}{1.000000,0.000000,0.000000}%
\pgfsetstrokecolor{currentstroke}%
\pgfsetdash{}{0pt}%
\pgfpathmoveto{\pgfqpoint{3.311664in}{1.786339in}}%
\pgfpathlineto{\pgfqpoint{3.108757in}{2.519866in}}%
\pgfusepath{stroke}%
\end{pgfscope}%
\begin{pgfscope}%
\pgfpathrectangle{\pgfqpoint{0.100000in}{0.212622in}}{\pgfqpoint{3.696000in}{3.696000in}}%
\pgfusepath{clip}%
\pgfsetrectcap%
\pgfsetroundjoin%
\pgfsetlinewidth{1.505625pt}%
\definecolor{currentstroke}{rgb}{1.000000,0.000000,0.000000}%
\pgfsetstrokecolor{currentstroke}%
\pgfsetdash{}{0pt}%
\pgfpathmoveto{\pgfqpoint{3.310705in}{1.786577in}}%
\pgfpathlineto{\pgfqpoint{3.108757in}{2.519866in}}%
\pgfusepath{stroke}%
\end{pgfscope}%
\begin{pgfscope}%
\pgfpathrectangle{\pgfqpoint{0.100000in}{0.212622in}}{\pgfqpoint{3.696000in}{3.696000in}}%
\pgfusepath{clip}%
\pgfsetrectcap%
\pgfsetroundjoin%
\pgfsetlinewidth{1.505625pt}%
\definecolor{currentstroke}{rgb}{1.000000,0.000000,0.000000}%
\pgfsetstrokecolor{currentstroke}%
\pgfsetdash{}{0pt}%
\pgfpathmoveto{\pgfqpoint{3.309715in}{1.786723in}}%
\pgfpathlineto{\pgfqpoint{3.108757in}{2.519866in}}%
\pgfusepath{stroke}%
\end{pgfscope}%
\begin{pgfscope}%
\pgfpathrectangle{\pgfqpoint{0.100000in}{0.212622in}}{\pgfqpoint{3.696000in}{3.696000in}}%
\pgfusepath{clip}%
\pgfsetrectcap%
\pgfsetroundjoin%
\pgfsetlinewidth{1.505625pt}%
\definecolor{currentstroke}{rgb}{1.000000,0.000000,0.000000}%
\pgfsetstrokecolor{currentstroke}%
\pgfsetdash{}{0pt}%
\pgfpathmoveto{\pgfqpoint{3.309229in}{1.786792in}}%
\pgfpathlineto{\pgfqpoint{3.108757in}{2.519866in}}%
\pgfusepath{stroke}%
\end{pgfscope}%
\begin{pgfscope}%
\pgfpathrectangle{\pgfqpoint{0.100000in}{0.212622in}}{\pgfqpoint{3.696000in}{3.696000in}}%
\pgfusepath{clip}%
\pgfsetrectcap%
\pgfsetroundjoin%
\pgfsetlinewidth{1.505625pt}%
\definecolor{currentstroke}{rgb}{1.000000,0.000000,0.000000}%
\pgfsetstrokecolor{currentstroke}%
\pgfsetdash{}{0pt}%
\pgfpathmoveto{\pgfqpoint{3.307286in}{1.787372in}}%
\pgfpathlineto{\pgfqpoint{3.108757in}{2.519866in}}%
\pgfusepath{stroke}%
\end{pgfscope}%
\begin{pgfscope}%
\pgfpathrectangle{\pgfqpoint{0.100000in}{0.212622in}}{\pgfqpoint{3.696000in}{3.696000in}}%
\pgfusepath{clip}%
\pgfsetrectcap%
\pgfsetroundjoin%
\pgfsetlinewidth{1.505625pt}%
\definecolor{currentstroke}{rgb}{1.000000,0.000000,0.000000}%
\pgfsetstrokecolor{currentstroke}%
\pgfsetdash{}{0pt}%
\pgfpathmoveto{\pgfqpoint{3.306343in}{1.787552in}}%
\pgfpathlineto{\pgfqpoint{3.108757in}{2.519866in}}%
\pgfusepath{stroke}%
\end{pgfscope}%
\begin{pgfscope}%
\pgfpathrectangle{\pgfqpoint{0.100000in}{0.212622in}}{\pgfqpoint{3.696000in}{3.696000in}}%
\pgfusepath{clip}%
\pgfsetrectcap%
\pgfsetroundjoin%
\pgfsetlinewidth{1.505625pt}%
\definecolor{currentstroke}{rgb}{1.000000,0.000000,0.000000}%
\pgfsetstrokecolor{currentstroke}%
\pgfsetdash{}{0pt}%
\pgfpathmoveto{\pgfqpoint{3.305957in}{1.787577in}}%
\pgfpathlineto{\pgfqpoint{3.108757in}{2.519866in}}%
\pgfusepath{stroke}%
\end{pgfscope}%
\begin{pgfscope}%
\pgfpathrectangle{\pgfqpoint{0.100000in}{0.212622in}}{\pgfqpoint{3.696000in}{3.696000in}}%
\pgfusepath{clip}%
\pgfsetrectcap%
\pgfsetroundjoin%
\pgfsetlinewidth{1.505625pt}%
\definecolor{currentstroke}{rgb}{1.000000,0.000000,0.000000}%
\pgfsetstrokecolor{currentstroke}%
\pgfsetdash{}{0pt}%
\pgfpathmoveto{\pgfqpoint{3.304000in}{1.788027in}}%
\pgfpathlineto{\pgfqpoint{3.108757in}{2.519866in}}%
\pgfusepath{stroke}%
\end{pgfscope}%
\begin{pgfscope}%
\pgfpathrectangle{\pgfqpoint{0.100000in}{0.212622in}}{\pgfqpoint{3.696000in}{3.696000in}}%
\pgfusepath{clip}%
\pgfsetrectcap%
\pgfsetroundjoin%
\pgfsetlinewidth{1.505625pt}%
\definecolor{currentstroke}{rgb}{1.000000,0.000000,0.000000}%
\pgfsetstrokecolor{currentstroke}%
\pgfsetdash{}{0pt}%
\pgfpathmoveto{\pgfqpoint{3.302901in}{1.788292in}}%
\pgfpathlineto{\pgfqpoint{3.108757in}{2.519866in}}%
\pgfusepath{stroke}%
\end{pgfscope}%
\begin{pgfscope}%
\pgfpathrectangle{\pgfqpoint{0.100000in}{0.212622in}}{\pgfqpoint{3.696000in}{3.696000in}}%
\pgfusepath{clip}%
\pgfsetrectcap%
\pgfsetroundjoin%
\pgfsetlinewidth{1.505625pt}%
\definecolor{currentstroke}{rgb}{1.000000,0.000000,0.000000}%
\pgfsetstrokecolor{currentstroke}%
\pgfsetdash{}{0pt}%
\pgfpathmoveto{\pgfqpoint{3.301556in}{1.788433in}}%
\pgfpathlineto{\pgfqpoint{3.108757in}{2.519866in}}%
\pgfusepath{stroke}%
\end{pgfscope}%
\begin{pgfscope}%
\pgfpathrectangle{\pgfqpoint{0.100000in}{0.212622in}}{\pgfqpoint{3.696000in}{3.696000in}}%
\pgfusepath{clip}%
\pgfsetrectcap%
\pgfsetroundjoin%
\pgfsetlinewidth{1.505625pt}%
\definecolor{currentstroke}{rgb}{1.000000,0.000000,0.000000}%
\pgfsetstrokecolor{currentstroke}%
\pgfsetdash{}{0pt}%
\pgfpathmoveto{\pgfqpoint{3.300646in}{1.788539in}}%
\pgfpathlineto{\pgfqpoint{3.108757in}{2.519866in}}%
\pgfusepath{stroke}%
\end{pgfscope}%
\begin{pgfscope}%
\pgfpathrectangle{\pgfqpoint{0.100000in}{0.212622in}}{\pgfqpoint{3.696000in}{3.696000in}}%
\pgfusepath{clip}%
\pgfsetrectcap%
\pgfsetroundjoin%
\pgfsetlinewidth{1.505625pt}%
\definecolor{currentstroke}{rgb}{1.000000,0.000000,0.000000}%
\pgfsetstrokecolor{currentstroke}%
\pgfsetdash{}{0pt}%
\pgfpathmoveto{\pgfqpoint{3.299171in}{1.789015in}}%
\pgfpathlineto{\pgfqpoint{3.108757in}{2.519866in}}%
\pgfusepath{stroke}%
\end{pgfscope}%
\begin{pgfscope}%
\pgfpathrectangle{\pgfqpoint{0.100000in}{0.212622in}}{\pgfqpoint{3.696000in}{3.696000in}}%
\pgfusepath{clip}%
\pgfsetrectcap%
\pgfsetroundjoin%
\pgfsetlinewidth{1.505625pt}%
\definecolor{currentstroke}{rgb}{1.000000,0.000000,0.000000}%
\pgfsetstrokecolor{currentstroke}%
\pgfsetdash{}{0pt}%
\pgfpathmoveto{\pgfqpoint{3.298676in}{1.789046in}}%
\pgfpathlineto{\pgfqpoint{3.108757in}{2.519866in}}%
\pgfusepath{stroke}%
\end{pgfscope}%
\begin{pgfscope}%
\pgfpathrectangle{\pgfqpoint{0.100000in}{0.212622in}}{\pgfqpoint{3.696000in}{3.696000in}}%
\pgfusepath{clip}%
\pgfsetrectcap%
\pgfsetroundjoin%
\pgfsetlinewidth{1.505625pt}%
\definecolor{currentstroke}{rgb}{1.000000,0.000000,0.000000}%
\pgfsetstrokecolor{currentstroke}%
\pgfsetdash{}{0pt}%
\pgfpathmoveto{\pgfqpoint{3.298059in}{1.789118in}}%
\pgfpathlineto{\pgfqpoint{3.108757in}{2.519866in}}%
\pgfusepath{stroke}%
\end{pgfscope}%
\begin{pgfscope}%
\pgfpathrectangle{\pgfqpoint{0.100000in}{0.212622in}}{\pgfqpoint{3.696000in}{3.696000in}}%
\pgfusepath{clip}%
\pgfsetrectcap%
\pgfsetroundjoin%
\pgfsetlinewidth{1.505625pt}%
\definecolor{currentstroke}{rgb}{1.000000,0.000000,0.000000}%
\pgfsetstrokecolor{currentstroke}%
\pgfsetdash{}{0pt}%
\pgfpathmoveto{\pgfqpoint{3.296973in}{1.789426in}}%
\pgfpathlineto{\pgfqpoint{3.108757in}{2.519866in}}%
\pgfusepath{stroke}%
\end{pgfscope}%
\begin{pgfscope}%
\pgfpathrectangle{\pgfqpoint{0.100000in}{0.212622in}}{\pgfqpoint{3.696000in}{3.696000in}}%
\pgfusepath{clip}%
\pgfsetrectcap%
\pgfsetroundjoin%
\pgfsetlinewidth{1.505625pt}%
\definecolor{currentstroke}{rgb}{1.000000,0.000000,0.000000}%
\pgfsetstrokecolor{currentstroke}%
\pgfsetdash{}{0pt}%
\pgfpathmoveto{\pgfqpoint{3.295539in}{1.789676in}}%
\pgfpathlineto{\pgfqpoint{3.108757in}{2.519866in}}%
\pgfusepath{stroke}%
\end{pgfscope}%
\begin{pgfscope}%
\pgfpathrectangle{\pgfqpoint{0.100000in}{0.212622in}}{\pgfqpoint{3.696000in}{3.696000in}}%
\pgfusepath{clip}%
\pgfsetrectcap%
\pgfsetroundjoin%
\pgfsetlinewidth{1.505625pt}%
\definecolor{currentstroke}{rgb}{1.000000,0.000000,0.000000}%
\pgfsetstrokecolor{currentstroke}%
\pgfsetdash{}{0pt}%
\pgfpathmoveto{\pgfqpoint{3.294994in}{1.789764in}}%
\pgfpathlineto{\pgfqpoint{3.108757in}{2.519866in}}%
\pgfusepath{stroke}%
\end{pgfscope}%
\begin{pgfscope}%
\pgfpathrectangle{\pgfqpoint{0.100000in}{0.212622in}}{\pgfqpoint{3.696000in}{3.696000in}}%
\pgfusepath{clip}%
\pgfsetrectcap%
\pgfsetroundjoin%
\pgfsetlinewidth{1.505625pt}%
\definecolor{currentstroke}{rgb}{1.000000,0.000000,0.000000}%
\pgfsetstrokecolor{currentstroke}%
\pgfsetdash{}{0pt}%
\pgfpathmoveto{\pgfqpoint{3.294485in}{1.789893in}}%
\pgfpathlineto{\pgfqpoint{3.108757in}{2.519866in}}%
\pgfusepath{stroke}%
\end{pgfscope}%
\begin{pgfscope}%
\pgfpathrectangle{\pgfqpoint{0.100000in}{0.212622in}}{\pgfqpoint{3.696000in}{3.696000in}}%
\pgfusepath{clip}%
\pgfsetrectcap%
\pgfsetroundjoin%
\pgfsetlinewidth{1.505625pt}%
\definecolor{currentstroke}{rgb}{1.000000,0.000000,0.000000}%
\pgfsetstrokecolor{currentstroke}%
\pgfsetdash{}{0pt}%
\pgfpathmoveto{\pgfqpoint{3.294236in}{1.789936in}}%
\pgfpathlineto{\pgfqpoint{3.108757in}{2.519866in}}%
\pgfusepath{stroke}%
\end{pgfscope}%
\begin{pgfscope}%
\pgfpathrectangle{\pgfqpoint{0.100000in}{0.212622in}}{\pgfqpoint{3.696000in}{3.696000in}}%
\pgfusepath{clip}%
\pgfsetrectcap%
\pgfsetroundjoin%
\pgfsetlinewidth{1.505625pt}%
\definecolor{currentstroke}{rgb}{1.000000,0.000000,0.000000}%
\pgfsetstrokecolor{currentstroke}%
\pgfsetdash{}{0pt}%
\pgfpathmoveto{\pgfqpoint{3.293926in}{1.789965in}}%
\pgfpathlineto{\pgfqpoint{3.108757in}{2.519866in}}%
\pgfusepath{stroke}%
\end{pgfscope}%
\begin{pgfscope}%
\pgfpathrectangle{\pgfqpoint{0.100000in}{0.212622in}}{\pgfqpoint{3.696000in}{3.696000in}}%
\pgfusepath{clip}%
\pgfsetrectcap%
\pgfsetroundjoin%
\pgfsetlinewidth{1.505625pt}%
\definecolor{currentstroke}{rgb}{1.000000,0.000000,0.000000}%
\pgfsetstrokecolor{currentstroke}%
\pgfsetdash{}{0pt}%
\pgfpathmoveto{\pgfqpoint{3.292008in}{1.790397in}}%
\pgfpathlineto{\pgfqpoint{3.108757in}{2.519866in}}%
\pgfusepath{stroke}%
\end{pgfscope}%
\begin{pgfscope}%
\pgfpathrectangle{\pgfqpoint{0.100000in}{0.212622in}}{\pgfqpoint{3.696000in}{3.696000in}}%
\pgfusepath{clip}%
\pgfsetrectcap%
\pgfsetroundjoin%
\pgfsetlinewidth{1.505625pt}%
\definecolor{currentstroke}{rgb}{1.000000,0.000000,0.000000}%
\pgfsetstrokecolor{currentstroke}%
\pgfsetdash{}{0pt}%
\pgfpathmoveto{\pgfqpoint{3.289962in}{1.790811in}}%
\pgfpathlineto{\pgfqpoint{3.108757in}{2.519866in}}%
\pgfusepath{stroke}%
\end{pgfscope}%
\begin{pgfscope}%
\pgfpathrectangle{\pgfqpoint{0.100000in}{0.212622in}}{\pgfqpoint{3.696000in}{3.696000in}}%
\pgfusepath{clip}%
\pgfsetrectcap%
\pgfsetroundjoin%
\pgfsetlinewidth{1.505625pt}%
\definecolor{currentstroke}{rgb}{1.000000,0.000000,0.000000}%
\pgfsetstrokecolor{currentstroke}%
\pgfsetdash{}{0pt}%
\pgfpathmoveto{\pgfqpoint{3.288221in}{1.791012in}}%
\pgfpathlineto{\pgfqpoint{3.108757in}{2.519866in}}%
\pgfusepath{stroke}%
\end{pgfscope}%
\begin{pgfscope}%
\pgfpathrectangle{\pgfqpoint{0.100000in}{0.212622in}}{\pgfqpoint{3.696000in}{3.696000in}}%
\pgfusepath{clip}%
\pgfsetrectcap%
\pgfsetroundjoin%
\pgfsetlinewidth{1.505625pt}%
\definecolor{currentstroke}{rgb}{1.000000,0.000000,0.000000}%
\pgfsetstrokecolor{currentstroke}%
\pgfsetdash{}{0pt}%
\pgfpathmoveto{\pgfqpoint{3.287187in}{1.791191in}}%
\pgfpathlineto{\pgfqpoint{3.108757in}{2.519866in}}%
\pgfusepath{stroke}%
\end{pgfscope}%
\begin{pgfscope}%
\pgfpathrectangle{\pgfqpoint{0.100000in}{0.212622in}}{\pgfqpoint{3.696000in}{3.696000in}}%
\pgfusepath{clip}%
\pgfsetrectcap%
\pgfsetroundjoin%
\pgfsetlinewidth{1.505625pt}%
\definecolor{currentstroke}{rgb}{1.000000,0.000000,0.000000}%
\pgfsetstrokecolor{currentstroke}%
\pgfsetdash{}{0pt}%
\pgfpathmoveto{\pgfqpoint{3.285215in}{1.791696in}}%
\pgfpathlineto{\pgfqpoint{3.108757in}{2.519866in}}%
\pgfusepath{stroke}%
\end{pgfscope}%
\begin{pgfscope}%
\pgfpathrectangle{\pgfqpoint{0.100000in}{0.212622in}}{\pgfqpoint{3.696000in}{3.696000in}}%
\pgfusepath{clip}%
\pgfsetrectcap%
\pgfsetroundjoin%
\pgfsetlinewidth{1.505625pt}%
\definecolor{currentstroke}{rgb}{1.000000,0.000000,0.000000}%
\pgfsetstrokecolor{currentstroke}%
\pgfsetdash{}{0pt}%
\pgfpathmoveto{\pgfqpoint{3.283360in}{1.791878in}}%
\pgfpathlineto{\pgfqpoint{3.108757in}{2.519866in}}%
\pgfusepath{stroke}%
\end{pgfscope}%
\begin{pgfscope}%
\pgfpathrectangle{\pgfqpoint{0.100000in}{0.212622in}}{\pgfqpoint{3.696000in}{3.696000in}}%
\pgfusepath{clip}%
\pgfsetrectcap%
\pgfsetroundjoin%
\pgfsetlinewidth{1.505625pt}%
\definecolor{currentstroke}{rgb}{1.000000,0.000000,0.000000}%
\pgfsetstrokecolor{currentstroke}%
\pgfsetdash{}{0pt}%
\pgfpathmoveto{\pgfqpoint{3.282675in}{1.791956in}}%
\pgfpathlineto{\pgfqpoint{3.108757in}{2.519866in}}%
\pgfusepath{stroke}%
\end{pgfscope}%
\begin{pgfscope}%
\pgfpathrectangle{\pgfqpoint{0.100000in}{0.212622in}}{\pgfqpoint{3.696000in}{3.696000in}}%
\pgfusepath{clip}%
\pgfsetrectcap%
\pgfsetroundjoin%
\pgfsetlinewidth{1.505625pt}%
\definecolor{currentstroke}{rgb}{1.000000,0.000000,0.000000}%
\pgfsetstrokecolor{currentstroke}%
\pgfsetdash{}{0pt}%
\pgfpathmoveto{\pgfqpoint{3.280678in}{1.792484in}}%
\pgfpathlineto{\pgfqpoint{3.108757in}{2.519866in}}%
\pgfusepath{stroke}%
\end{pgfscope}%
\begin{pgfscope}%
\pgfpathrectangle{\pgfqpoint{0.100000in}{0.212622in}}{\pgfqpoint{3.696000in}{3.696000in}}%
\pgfusepath{clip}%
\pgfsetrectcap%
\pgfsetroundjoin%
\pgfsetlinewidth{1.505625pt}%
\definecolor{currentstroke}{rgb}{1.000000,0.000000,0.000000}%
\pgfsetstrokecolor{currentstroke}%
\pgfsetdash{}{0pt}%
\pgfpathmoveto{\pgfqpoint{3.279677in}{1.792687in}}%
\pgfpathlineto{\pgfqpoint{3.108757in}{2.519866in}}%
\pgfusepath{stroke}%
\end{pgfscope}%
\begin{pgfscope}%
\pgfpathrectangle{\pgfqpoint{0.100000in}{0.212622in}}{\pgfqpoint{3.696000in}{3.696000in}}%
\pgfusepath{clip}%
\pgfsetrectcap%
\pgfsetroundjoin%
\pgfsetlinewidth{1.505625pt}%
\definecolor{currentstroke}{rgb}{1.000000,0.000000,0.000000}%
\pgfsetstrokecolor{currentstroke}%
\pgfsetdash{}{0pt}%
\pgfpathmoveto{\pgfqpoint{3.278942in}{1.792783in}}%
\pgfpathlineto{\pgfqpoint{3.108757in}{2.519866in}}%
\pgfusepath{stroke}%
\end{pgfscope}%
\begin{pgfscope}%
\pgfpathrectangle{\pgfqpoint{0.100000in}{0.212622in}}{\pgfqpoint{3.696000in}{3.696000in}}%
\pgfusepath{clip}%
\pgfsetrectcap%
\pgfsetroundjoin%
\pgfsetlinewidth{1.505625pt}%
\definecolor{currentstroke}{rgb}{1.000000,0.000000,0.000000}%
\pgfsetstrokecolor{currentstroke}%
\pgfsetdash{}{0pt}%
\pgfpathmoveto{\pgfqpoint{3.277237in}{1.793091in}}%
\pgfpathlineto{\pgfqpoint{3.108757in}{2.519866in}}%
\pgfusepath{stroke}%
\end{pgfscope}%
\begin{pgfscope}%
\pgfpathrectangle{\pgfqpoint{0.100000in}{0.212622in}}{\pgfqpoint{3.696000in}{3.696000in}}%
\pgfusepath{clip}%
\pgfsetrectcap%
\pgfsetroundjoin%
\pgfsetlinewidth{1.505625pt}%
\definecolor{currentstroke}{rgb}{1.000000,0.000000,0.000000}%
\pgfsetstrokecolor{currentstroke}%
\pgfsetdash{}{0pt}%
\pgfpathmoveto{\pgfqpoint{3.275123in}{1.793560in}}%
\pgfpathlineto{\pgfqpoint{3.108757in}{2.519866in}}%
\pgfusepath{stroke}%
\end{pgfscope}%
\begin{pgfscope}%
\pgfpathrectangle{\pgfqpoint{0.100000in}{0.212622in}}{\pgfqpoint{3.696000in}{3.696000in}}%
\pgfusepath{clip}%
\pgfsetrectcap%
\pgfsetroundjoin%
\pgfsetlinewidth{1.505625pt}%
\definecolor{currentstroke}{rgb}{1.000000,0.000000,0.000000}%
\pgfsetstrokecolor{currentstroke}%
\pgfsetdash{}{0pt}%
\pgfpathmoveto{\pgfqpoint{3.273050in}{1.793891in}}%
\pgfpathlineto{\pgfqpoint{3.108757in}{2.519866in}}%
\pgfusepath{stroke}%
\end{pgfscope}%
\begin{pgfscope}%
\pgfpathrectangle{\pgfqpoint{0.100000in}{0.212622in}}{\pgfqpoint{3.696000in}{3.696000in}}%
\pgfusepath{clip}%
\pgfsetrectcap%
\pgfsetroundjoin%
\pgfsetlinewidth{1.505625pt}%
\definecolor{currentstroke}{rgb}{1.000000,0.000000,0.000000}%
\pgfsetstrokecolor{currentstroke}%
\pgfsetdash{}{0pt}%
\pgfpathmoveto{\pgfqpoint{3.270822in}{1.794119in}}%
\pgfpathlineto{\pgfqpoint{3.108757in}{2.519866in}}%
\pgfusepath{stroke}%
\end{pgfscope}%
\begin{pgfscope}%
\pgfpathrectangle{\pgfqpoint{0.100000in}{0.212622in}}{\pgfqpoint{3.696000in}{3.696000in}}%
\pgfusepath{clip}%
\pgfsetrectcap%
\pgfsetroundjoin%
\pgfsetlinewidth{1.505625pt}%
\definecolor{currentstroke}{rgb}{1.000000,0.000000,0.000000}%
\pgfsetstrokecolor{currentstroke}%
\pgfsetdash{}{0pt}%
\pgfpathmoveto{\pgfqpoint{3.268814in}{1.794596in}}%
\pgfpathlineto{\pgfqpoint{3.108757in}{2.519866in}}%
\pgfusepath{stroke}%
\end{pgfscope}%
\begin{pgfscope}%
\pgfpathrectangle{\pgfqpoint{0.100000in}{0.212622in}}{\pgfqpoint{3.696000in}{3.696000in}}%
\pgfusepath{clip}%
\pgfsetrectcap%
\pgfsetroundjoin%
\pgfsetlinewidth{1.505625pt}%
\definecolor{currentstroke}{rgb}{1.000000,0.000000,0.000000}%
\pgfsetstrokecolor{currentstroke}%
\pgfsetdash{}{0pt}%
\pgfpathmoveto{\pgfqpoint{3.266403in}{1.795069in}}%
\pgfpathlineto{\pgfqpoint{3.108757in}{2.519866in}}%
\pgfusepath{stroke}%
\end{pgfscope}%
\begin{pgfscope}%
\pgfpathrectangle{\pgfqpoint{0.100000in}{0.212622in}}{\pgfqpoint{3.696000in}{3.696000in}}%
\pgfusepath{clip}%
\pgfsetrectcap%
\pgfsetroundjoin%
\pgfsetlinewidth{1.505625pt}%
\definecolor{currentstroke}{rgb}{1.000000,0.000000,0.000000}%
\pgfsetstrokecolor{currentstroke}%
\pgfsetdash{}{0pt}%
\pgfpathmoveto{\pgfqpoint{3.264960in}{1.795195in}}%
\pgfpathlineto{\pgfqpoint{3.108757in}{2.519866in}}%
\pgfusepath{stroke}%
\end{pgfscope}%
\begin{pgfscope}%
\pgfpathrectangle{\pgfqpoint{0.100000in}{0.212622in}}{\pgfqpoint{3.696000in}{3.696000in}}%
\pgfusepath{clip}%
\pgfsetrectcap%
\pgfsetroundjoin%
\pgfsetlinewidth{1.505625pt}%
\definecolor{currentstroke}{rgb}{1.000000,0.000000,0.000000}%
\pgfsetstrokecolor{currentstroke}%
\pgfsetdash{}{0pt}%
\pgfpathmoveto{\pgfqpoint{3.261317in}{1.795961in}}%
\pgfpathlineto{\pgfqpoint{3.108757in}{2.519866in}}%
\pgfusepath{stroke}%
\end{pgfscope}%
\begin{pgfscope}%
\pgfpathrectangle{\pgfqpoint{0.100000in}{0.212622in}}{\pgfqpoint{3.696000in}{3.696000in}}%
\pgfusepath{clip}%
\pgfsetrectcap%
\pgfsetroundjoin%
\pgfsetlinewidth{1.505625pt}%
\definecolor{currentstroke}{rgb}{1.000000,0.000000,0.000000}%
\pgfsetstrokecolor{currentstroke}%
\pgfsetdash{}{0pt}%
\pgfpathmoveto{\pgfqpoint{3.256840in}{1.797044in}}%
\pgfpathlineto{\pgfqpoint{3.108757in}{2.519866in}}%
\pgfusepath{stroke}%
\end{pgfscope}%
\begin{pgfscope}%
\pgfpathrectangle{\pgfqpoint{0.100000in}{0.212622in}}{\pgfqpoint{3.696000in}{3.696000in}}%
\pgfusepath{clip}%
\pgfsetrectcap%
\pgfsetroundjoin%
\pgfsetlinewidth{1.505625pt}%
\definecolor{currentstroke}{rgb}{1.000000,0.000000,0.000000}%
\pgfsetstrokecolor{currentstroke}%
\pgfsetdash{}{0pt}%
\pgfpathmoveto{\pgfqpoint{3.253831in}{1.797406in}}%
\pgfpathlineto{\pgfqpoint{3.108757in}{2.519866in}}%
\pgfusepath{stroke}%
\end{pgfscope}%
\begin{pgfscope}%
\pgfpathrectangle{\pgfqpoint{0.100000in}{0.212622in}}{\pgfqpoint{3.696000in}{3.696000in}}%
\pgfusepath{clip}%
\pgfsetrectcap%
\pgfsetroundjoin%
\pgfsetlinewidth{1.505625pt}%
\definecolor{currentstroke}{rgb}{1.000000,0.000000,0.000000}%
\pgfsetstrokecolor{currentstroke}%
\pgfsetdash{}{0pt}%
\pgfpathmoveto{\pgfqpoint{3.249567in}{1.797863in}}%
\pgfpathlineto{\pgfqpoint{3.108757in}{2.519866in}}%
\pgfusepath{stroke}%
\end{pgfscope}%
\begin{pgfscope}%
\pgfpathrectangle{\pgfqpoint{0.100000in}{0.212622in}}{\pgfqpoint{3.696000in}{3.696000in}}%
\pgfusepath{clip}%
\pgfsetrectcap%
\pgfsetroundjoin%
\pgfsetlinewidth{1.505625pt}%
\definecolor{currentstroke}{rgb}{1.000000,0.000000,0.000000}%
\pgfsetstrokecolor{currentstroke}%
\pgfsetdash{}{0pt}%
\pgfpathmoveto{\pgfqpoint{3.243682in}{1.799225in}}%
\pgfpathlineto{\pgfqpoint{3.108757in}{2.519866in}}%
\pgfusepath{stroke}%
\end{pgfscope}%
\begin{pgfscope}%
\pgfpathrectangle{\pgfqpoint{0.100000in}{0.212622in}}{\pgfqpoint{3.696000in}{3.696000in}}%
\pgfusepath{clip}%
\pgfsetrectcap%
\pgfsetroundjoin%
\pgfsetlinewidth{1.505625pt}%
\definecolor{currentstroke}{rgb}{1.000000,0.000000,0.000000}%
\pgfsetstrokecolor{currentstroke}%
\pgfsetdash{}{0pt}%
\pgfpathmoveto{\pgfqpoint{3.238720in}{1.799705in}}%
\pgfpathlineto{\pgfqpoint{3.108757in}{2.519866in}}%
\pgfusepath{stroke}%
\end{pgfscope}%
\begin{pgfscope}%
\pgfpathrectangle{\pgfqpoint{0.100000in}{0.212622in}}{\pgfqpoint{3.696000in}{3.696000in}}%
\pgfusepath{clip}%
\pgfsetrectcap%
\pgfsetroundjoin%
\pgfsetlinewidth{1.505625pt}%
\definecolor{currentstroke}{rgb}{1.000000,0.000000,0.000000}%
\pgfsetstrokecolor{currentstroke}%
\pgfsetdash{}{0pt}%
\pgfpathmoveto{\pgfqpoint{3.235272in}{1.800066in}}%
\pgfpathlineto{\pgfqpoint{3.108757in}{2.519866in}}%
\pgfusepath{stroke}%
\end{pgfscope}%
\begin{pgfscope}%
\pgfpathrectangle{\pgfqpoint{0.100000in}{0.212622in}}{\pgfqpoint{3.696000in}{3.696000in}}%
\pgfusepath{clip}%
\pgfsetrectcap%
\pgfsetroundjoin%
\pgfsetlinewidth{1.505625pt}%
\definecolor{currentstroke}{rgb}{1.000000,0.000000,0.000000}%
\pgfsetstrokecolor{currentstroke}%
\pgfsetdash{}{0pt}%
\pgfpathmoveto{\pgfqpoint{3.227773in}{1.801621in}}%
\pgfpathlineto{\pgfqpoint{3.108757in}{2.519866in}}%
\pgfusepath{stroke}%
\end{pgfscope}%
\begin{pgfscope}%
\pgfpathrectangle{\pgfqpoint{0.100000in}{0.212622in}}{\pgfqpoint{3.696000in}{3.696000in}}%
\pgfusepath{clip}%
\pgfsetrectcap%
\pgfsetroundjoin%
\pgfsetlinewidth{1.505625pt}%
\definecolor{currentstroke}{rgb}{1.000000,0.000000,0.000000}%
\pgfsetstrokecolor{currentstroke}%
\pgfsetdash{}{0pt}%
\pgfpathmoveto{\pgfqpoint{3.223839in}{1.802336in}}%
\pgfpathlineto{\pgfqpoint{3.108757in}{2.519866in}}%
\pgfusepath{stroke}%
\end{pgfscope}%
\begin{pgfscope}%
\pgfpathrectangle{\pgfqpoint{0.100000in}{0.212622in}}{\pgfqpoint{3.696000in}{3.696000in}}%
\pgfusepath{clip}%
\pgfsetrectcap%
\pgfsetroundjoin%
\pgfsetlinewidth{1.505625pt}%
\definecolor{currentstroke}{rgb}{1.000000,0.000000,0.000000}%
\pgfsetstrokecolor{currentstroke}%
\pgfsetdash{}{0pt}%
\pgfpathmoveto{\pgfqpoint{3.222550in}{1.802509in}}%
\pgfpathlineto{\pgfqpoint{3.108757in}{2.519866in}}%
\pgfusepath{stroke}%
\end{pgfscope}%
\begin{pgfscope}%
\pgfpathrectangle{\pgfqpoint{0.100000in}{0.212622in}}{\pgfqpoint{3.696000in}{3.696000in}}%
\pgfusepath{clip}%
\pgfsetrectcap%
\pgfsetroundjoin%
\pgfsetlinewidth{1.505625pt}%
\definecolor{currentstroke}{rgb}{1.000000,0.000000,0.000000}%
\pgfsetstrokecolor{currentstroke}%
\pgfsetdash{}{0pt}%
\pgfpathmoveto{\pgfqpoint{3.219198in}{1.803179in}}%
\pgfpathlineto{\pgfqpoint{3.108757in}{2.519866in}}%
\pgfusepath{stroke}%
\end{pgfscope}%
\begin{pgfscope}%
\pgfpathrectangle{\pgfqpoint{0.100000in}{0.212622in}}{\pgfqpoint{3.696000in}{3.696000in}}%
\pgfusepath{clip}%
\pgfsetrectcap%
\pgfsetroundjoin%
\pgfsetlinewidth{1.505625pt}%
\definecolor{currentstroke}{rgb}{1.000000,0.000000,0.000000}%
\pgfsetstrokecolor{currentstroke}%
\pgfsetdash{}{0pt}%
\pgfpathmoveto{\pgfqpoint{3.217019in}{1.803810in}}%
\pgfpathlineto{\pgfqpoint{3.108757in}{2.519866in}}%
\pgfusepath{stroke}%
\end{pgfscope}%
\begin{pgfscope}%
\pgfpathrectangle{\pgfqpoint{0.100000in}{0.212622in}}{\pgfqpoint{3.696000in}{3.696000in}}%
\pgfusepath{clip}%
\pgfsetrectcap%
\pgfsetroundjoin%
\pgfsetlinewidth{1.505625pt}%
\definecolor{currentstroke}{rgb}{1.000000,0.000000,0.000000}%
\pgfsetstrokecolor{currentstroke}%
\pgfsetdash{}{0pt}%
\pgfpathmoveto{\pgfqpoint{3.215695in}{1.803992in}}%
\pgfpathlineto{\pgfqpoint{3.108757in}{2.519866in}}%
\pgfusepath{stroke}%
\end{pgfscope}%
\begin{pgfscope}%
\pgfpathrectangle{\pgfqpoint{0.100000in}{0.212622in}}{\pgfqpoint{3.696000in}{3.696000in}}%
\pgfusepath{clip}%
\pgfsetrectcap%
\pgfsetroundjoin%
\pgfsetlinewidth{1.505625pt}%
\definecolor{currentstroke}{rgb}{1.000000,0.000000,0.000000}%
\pgfsetstrokecolor{currentstroke}%
\pgfsetdash{}{0pt}%
\pgfpathmoveto{\pgfqpoint{3.212590in}{1.804620in}}%
\pgfpathlineto{\pgfqpoint{3.108757in}{2.519866in}}%
\pgfusepath{stroke}%
\end{pgfscope}%
\begin{pgfscope}%
\pgfpathrectangle{\pgfqpoint{0.100000in}{0.212622in}}{\pgfqpoint{3.696000in}{3.696000in}}%
\pgfusepath{clip}%
\pgfsetrectcap%
\pgfsetroundjoin%
\pgfsetlinewidth{1.505625pt}%
\definecolor{currentstroke}{rgb}{1.000000,0.000000,0.000000}%
\pgfsetstrokecolor{currentstroke}%
\pgfsetdash{}{0pt}%
\pgfpathmoveto{\pgfqpoint{3.208879in}{1.805375in}}%
\pgfpathlineto{\pgfqpoint{3.108757in}{2.519866in}}%
\pgfusepath{stroke}%
\end{pgfscope}%
\begin{pgfscope}%
\pgfpathrectangle{\pgfqpoint{0.100000in}{0.212622in}}{\pgfqpoint{3.696000in}{3.696000in}}%
\pgfusepath{clip}%
\pgfsetrectcap%
\pgfsetroundjoin%
\pgfsetlinewidth{1.505625pt}%
\definecolor{currentstroke}{rgb}{1.000000,0.000000,0.000000}%
\pgfsetstrokecolor{currentstroke}%
\pgfsetdash{}{0pt}%
\pgfpathmoveto{\pgfqpoint{3.205486in}{1.805641in}}%
\pgfpathlineto{\pgfqpoint{3.108757in}{2.519866in}}%
\pgfusepath{stroke}%
\end{pgfscope}%
\begin{pgfscope}%
\pgfpathrectangle{\pgfqpoint{0.100000in}{0.212622in}}{\pgfqpoint{3.696000in}{3.696000in}}%
\pgfusepath{clip}%
\pgfsetrectcap%
\pgfsetroundjoin%
\pgfsetlinewidth{1.505625pt}%
\definecolor{currentstroke}{rgb}{1.000000,0.000000,0.000000}%
\pgfsetstrokecolor{currentstroke}%
\pgfsetdash{}{0pt}%
\pgfpathmoveto{\pgfqpoint{3.203650in}{1.805821in}}%
\pgfpathlineto{\pgfqpoint{3.108757in}{2.519866in}}%
\pgfusepath{stroke}%
\end{pgfscope}%
\begin{pgfscope}%
\pgfpathrectangle{\pgfqpoint{0.100000in}{0.212622in}}{\pgfqpoint{3.696000in}{3.696000in}}%
\pgfusepath{clip}%
\pgfsetrectcap%
\pgfsetroundjoin%
\pgfsetlinewidth{1.505625pt}%
\definecolor{currentstroke}{rgb}{1.000000,0.000000,0.000000}%
\pgfsetstrokecolor{currentstroke}%
\pgfsetdash{}{0pt}%
\pgfpathmoveto{\pgfqpoint{3.200864in}{1.806404in}}%
\pgfpathlineto{\pgfqpoint{3.108757in}{2.519866in}}%
\pgfusepath{stroke}%
\end{pgfscope}%
\begin{pgfscope}%
\pgfpathrectangle{\pgfqpoint{0.100000in}{0.212622in}}{\pgfqpoint{3.696000in}{3.696000in}}%
\pgfusepath{clip}%
\pgfsetrectcap%
\pgfsetroundjoin%
\pgfsetlinewidth{1.505625pt}%
\definecolor{currentstroke}{rgb}{1.000000,0.000000,0.000000}%
\pgfsetstrokecolor{currentstroke}%
\pgfsetdash{}{0pt}%
\pgfpathmoveto{\pgfqpoint{3.199633in}{1.806563in}}%
\pgfpathlineto{\pgfqpoint{3.108757in}{2.519866in}}%
\pgfusepath{stroke}%
\end{pgfscope}%
\begin{pgfscope}%
\pgfpathrectangle{\pgfqpoint{0.100000in}{0.212622in}}{\pgfqpoint{3.696000in}{3.696000in}}%
\pgfusepath{clip}%
\pgfsetrectcap%
\pgfsetroundjoin%
\pgfsetlinewidth{1.505625pt}%
\definecolor{currentstroke}{rgb}{1.000000,0.000000,0.000000}%
\pgfsetstrokecolor{currentstroke}%
\pgfsetdash{}{0pt}%
\pgfpathmoveto{\pgfqpoint{3.199085in}{1.806638in}}%
\pgfpathlineto{\pgfqpoint{3.100850in}{2.512660in}}%
\pgfusepath{stroke}%
\end{pgfscope}%
\begin{pgfscope}%
\pgfpathrectangle{\pgfqpoint{0.100000in}{0.212622in}}{\pgfqpoint{3.696000in}{3.696000in}}%
\pgfusepath{clip}%
\pgfsetrectcap%
\pgfsetroundjoin%
\pgfsetlinewidth{1.505625pt}%
\definecolor{currentstroke}{rgb}{1.000000,0.000000,0.000000}%
\pgfsetstrokecolor{currentstroke}%
\pgfsetdash{}{0pt}%
\pgfpathmoveto{\pgfqpoint{3.197447in}{1.806950in}}%
\pgfpathlineto{\pgfqpoint{3.100850in}{2.512660in}}%
\pgfusepath{stroke}%
\end{pgfscope}%
\begin{pgfscope}%
\pgfpathrectangle{\pgfqpoint{0.100000in}{0.212622in}}{\pgfqpoint{3.696000in}{3.696000in}}%
\pgfusepath{clip}%
\pgfsetrectcap%
\pgfsetroundjoin%
\pgfsetlinewidth{1.505625pt}%
\definecolor{currentstroke}{rgb}{1.000000,0.000000,0.000000}%
\pgfsetstrokecolor{currentstroke}%
\pgfsetdash{}{0pt}%
\pgfpathmoveto{\pgfqpoint{3.196500in}{1.807116in}}%
\pgfpathlineto{\pgfqpoint{3.100850in}{2.512660in}}%
\pgfusepath{stroke}%
\end{pgfscope}%
\begin{pgfscope}%
\pgfpathrectangle{\pgfqpoint{0.100000in}{0.212622in}}{\pgfqpoint{3.696000in}{3.696000in}}%
\pgfusepath{clip}%
\pgfsetrectcap%
\pgfsetroundjoin%
\pgfsetlinewidth{1.505625pt}%
\definecolor{currentstroke}{rgb}{1.000000,0.000000,0.000000}%
\pgfsetstrokecolor{currentstroke}%
\pgfsetdash{}{0pt}%
\pgfpathmoveto{\pgfqpoint{3.195879in}{1.807174in}}%
\pgfpathlineto{\pgfqpoint{3.100850in}{2.512660in}}%
\pgfusepath{stroke}%
\end{pgfscope}%
\begin{pgfscope}%
\pgfpathrectangle{\pgfqpoint{0.100000in}{0.212622in}}{\pgfqpoint{3.696000in}{3.696000in}}%
\pgfusepath{clip}%
\pgfsetrectcap%
\pgfsetroundjoin%
\pgfsetlinewidth{1.505625pt}%
\definecolor{currentstroke}{rgb}{1.000000,0.000000,0.000000}%
\pgfsetstrokecolor{currentstroke}%
\pgfsetdash{}{0pt}%
\pgfpathmoveto{\pgfqpoint{3.194752in}{1.807306in}}%
\pgfpathlineto{\pgfqpoint{3.100850in}{2.512660in}}%
\pgfusepath{stroke}%
\end{pgfscope}%
\begin{pgfscope}%
\pgfpathrectangle{\pgfqpoint{0.100000in}{0.212622in}}{\pgfqpoint{3.696000in}{3.696000in}}%
\pgfusepath{clip}%
\pgfsetrectcap%
\pgfsetroundjoin%
\pgfsetlinewidth{1.505625pt}%
\definecolor{currentstroke}{rgb}{1.000000,0.000000,0.000000}%
\pgfsetstrokecolor{currentstroke}%
\pgfsetdash{}{0pt}%
\pgfpathmoveto{\pgfqpoint{3.193943in}{1.807470in}}%
\pgfpathlineto{\pgfqpoint{3.100850in}{2.512660in}}%
\pgfusepath{stroke}%
\end{pgfscope}%
\begin{pgfscope}%
\pgfpathrectangle{\pgfqpoint{0.100000in}{0.212622in}}{\pgfqpoint{3.696000in}{3.696000in}}%
\pgfusepath{clip}%
\pgfsetrectcap%
\pgfsetroundjoin%
\pgfsetlinewidth{1.505625pt}%
\definecolor{currentstroke}{rgb}{1.000000,0.000000,0.000000}%
\pgfsetstrokecolor{currentstroke}%
\pgfsetdash{}{0pt}%
\pgfpathmoveto{\pgfqpoint{3.192624in}{1.807625in}}%
\pgfpathlineto{\pgfqpoint{3.100850in}{2.512660in}}%
\pgfusepath{stroke}%
\end{pgfscope}%
\begin{pgfscope}%
\pgfpathrectangle{\pgfqpoint{0.100000in}{0.212622in}}{\pgfqpoint{3.696000in}{3.696000in}}%
\pgfusepath{clip}%
\pgfsetrectcap%
\pgfsetroundjoin%
\pgfsetlinewidth{1.505625pt}%
\definecolor{currentstroke}{rgb}{1.000000,0.000000,0.000000}%
\pgfsetstrokecolor{currentstroke}%
\pgfsetdash{}{0pt}%
\pgfpathmoveto{\pgfqpoint{3.191229in}{1.807812in}}%
\pgfpathlineto{\pgfqpoint{3.092932in}{2.505445in}}%
\pgfusepath{stroke}%
\end{pgfscope}%
\begin{pgfscope}%
\pgfpathrectangle{\pgfqpoint{0.100000in}{0.212622in}}{\pgfqpoint{3.696000in}{3.696000in}}%
\pgfusepath{clip}%
\pgfsetrectcap%
\pgfsetroundjoin%
\pgfsetlinewidth{1.505625pt}%
\definecolor{currentstroke}{rgb}{1.000000,0.000000,0.000000}%
\pgfsetstrokecolor{currentstroke}%
\pgfsetdash{}{0pt}%
\pgfpathmoveto{\pgfqpoint{3.190027in}{1.808073in}}%
\pgfpathlineto{\pgfqpoint{3.092932in}{2.505445in}}%
\pgfusepath{stroke}%
\end{pgfscope}%
\begin{pgfscope}%
\pgfpathrectangle{\pgfqpoint{0.100000in}{0.212622in}}{\pgfqpoint{3.696000in}{3.696000in}}%
\pgfusepath{clip}%
\pgfsetrectcap%
\pgfsetroundjoin%
\pgfsetlinewidth{1.505625pt}%
\definecolor{currentstroke}{rgb}{1.000000,0.000000,0.000000}%
\pgfsetstrokecolor{currentstroke}%
\pgfsetdash{}{0pt}%
\pgfpathmoveto{\pgfqpoint{3.188481in}{1.808201in}}%
\pgfpathlineto{\pgfqpoint{3.092932in}{2.505445in}}%
\pgfusepath{stroke}%
\end{pgfscope}%
\begin{pgfscope}%
\pgfpathrectangle{\pgfqpoint{0.100000in}{0.212622in}}{\pgfqpoint{3.696000in}{3.696000in}}%
\pgfusepath{clip}%
\pgfsetrectcap%
\pgfsetroundjoin%
\pgfsetlinewidth{1.505625pt}%
\definecolor{currentstroke}{rgb}{1.000000,0.000000,0.000000}%
\pgfsetstrokecolor{currentstroke}%
\pgfsetdash{}{0pt}%
\pgfpathmoveto{\pgfqpoint{3.186989in}{1.808396in}}%
\pgfpathlineto{\pgfqpoint{3.092932in}{2.505445in}}%
\pgfusepath{stroke}%
\end{pgfscope}%
\begin{pgfscope}%
\pgfpathrectangle{\pgfqpoint{0.100000in}{0.212622in}}{\pgfqpoint{3.696000in}{3.696000in}}%
\pgfusepath{clip}%
\pgfsetrectcap%
\pgfsetroundjoin%
\pgfsetlinewidth{1.505625pt}%
\definecolor{currentstroke}{rgb}{1.000000,0.000000,0.000000}%
\pgfsetstrokecolor{currentstroke}%
\pgfsetdash{}{0pt}%
\pgfpathmoveto{\pgfqpoint{3.184542in}{1.808821in}}%
\pgfpathlineto{\pgfqpoint{3.085004in}{2.498220in}}%
\pgfusepath{stroke}%
\end{pgfscope}%
\begin{pgfscope}%
\pgfpathrectangle{\pgfqpoint{0.100000in}{0.212622in}}{\pgfqpoint{3.696000in}{3.696000in}}%
\pgfusepath{clip}%
\pgfsetrectcap%
\pgfsetroundjoin%
\pgfsetlinewidth{1.505625pt}%
\definecolor{currentstroke}{rgb}{1.000000,0.000000,0.000000}%
\pgfsetstrokecolor{currentstroke}%
\pgfsetdash{}{0pt}%
\pgfpathmoveto{\pgfqpoint{3.182086in}{1.809132in}}%
\pgfpathlineto{\pgfqpoint{3.085004in}{2.498220in}}%
\pgfusepath{stroke}%
\end{pgfscope}%
\begin{pgfscope}%
\pgfpathrectangle{\pgfqpoint{0.100000in}{0.212622in}}{\pgfqpoint{3.696000in}{3.696000in}}%
\pgfusepath{clip}%
\pgfsetrectcap%
\pgfsetroundjoin%
\pgfsetlinewidth{1.505625pt}%
\definecolor{currentstroke}{rgb}{1.000000,0.000000,0.000000}%
\pgfsetstrokecolor{currentstroke}%
\pgfsetdash{}{0pt}%
\pgfpathmoveto{\pgfqpoint{3.180474in}{1.809339in}}%
\pgfpathlineto{\pgfqpoint{3.085004in}{2.498220in}}%
\pgfusepath{stroke}%
\end{pgfscope}%
\begin{pgfscope}%
\pgfpathrectangle{\pgfqpoint{0.100000in}{0.212622in}}{\pgfqpoint{3.696000in}{3.696000in}}%
\pgfusepath{clip}%
\pgfsetrectcap%
\pgfsetroundjoin%
\pgfsetlinewidth{1.505625pt}%
\definecolor{currentstroke}{rgb}{1.000000,0.000000,0.000000}%
\pgfsetstrokecolor{currentstroke}%
\pgfsetdash{}{0pt}%
\pgfpathmoveto{\pgfqpoint{3.176668in}{1.809941in}}%
\pgfpathlineto{\pgfqpoint{3.077065in}{2.490986in}}%
\pgfusepath{stroke}%
\end{pgfscope}%
\begin{pgfscope}%
\pgfpathrectangle{\pgfqpoint{0.100000in}{0.212622in}}{\pgfqpoint{3.696000in}{3.696000in}}%
\pgfusepath{clip}%
\pgfsetrectcap%
\pgfsetroundjoin%
\pgfsetlinewidth{1.505625pt}%
\definecolor{currentstroke}{rgb}{1.000000,0.000000,0.000000}%
\pgfsetstrokecolor{currentstroke}%
\pgfsetdash{}{0pt}%
\pgfpathmoveto{\pgfqpoint{3.174120in}{1.810487in}}%
\pgfpathlineto{\pgfqpoint{3.077065in}{2.490986in}}%
\pgfusepath{stroke}%
\end{pgfscope}%
\begin{pgfscope}%
\pgfpathrectangle{\pgfqpoint{0.100000in}{0.212622in}}{\pgfqpoint{3.696000in}{3.696000in}}%
\pgfusepath{clip}%
\pgfsetrectcap%
\pgfsetroundjoin%
\pgfsetlinewidth{1.505625pt}%
\definecolor{currentstroke}{rgb}{1.000000,0.000000,0.000000}%
\pgfsetstrokecolor{currentstroke}%
\pgfsetdash{}{0pt}%
\pgfpathmoveto{\pgfqpoint{3.171339in}{1.810896in}}%
\pgfpathlineto{\pgfqpoint{3.077065in}{2.490986in}}%
\pgfusepath{stroke}%
\end{pgfscope}%
\begin{pgfscope}%
\pgfpathrectangle{\pgfqpoint{0.100000in}{0.212622in}}{\pgfqpoint{3.696000in}{3.696000in}}%
\pgfusepath{clip}%
\pgfsetrectcap%
\pgfsetroundjoin%
\pgfsetlinewidth{1.505625pt}%
\definecolor{currentstroke}{rgb}{1.000000,0.000000,0.000000}%
\pgfsetstrokecolor{currentstroke}%
\pgfsetdash{}{0pt}%
\pgfpathmoveto{\pgfqpoint{3.170608in}{1.810955in}}%
\pgfpathlineto{\pgfqpoint{3.069117in}{2.483742in}}%
\pgfusepath{stroke}%
\end{pgfscope}%
\begin{pgfscope}%
\pgfpathrectangle{\pgfqpoint{0.100000in}{0.212622in}}{\pgfqpoint{3.696000in}{3.696000in}}%
\pgfusepath{clip}%
\pgfsetrectcap%
\pgfsetroundjoin%
\pgfsetlinewidth{1.505625pt}%
\definecolor{currentstroke}{rgb}{1.000000,0.000000,0.000000}%
\pgfsetstrokecolor{currentstroke}%
\pgfsetdash{}{0pt}%
\pgfpathmoveto{\pgfqpoint{3.168439in}{1.811338in}}%
\pgfpathlineto{\pgfqpoint{3.069117in}{2.483742in}}%
\pgfusepath{stroke}%
\end{pgfscope}%
\begin{pgfscope}%
\pgfpathrectangle{\pgfqpoint{0.100000in}{0.212622in}}{\pgfqpoint{3.696000in}{3.696000in}}%
\pgfusepath{clip}%
\pgfsetrectcap%
\pgfsetroundjoin%
\pgfsetlinewidth{1.505625pt}%
\definecolor{currentstroke}{rgb}{1.000000,0.000000,0.000000}%
\pgfsetstrokecolor{currentstroke}%
\pgfsetdash{}{0pt}%
\pgfpathmoveto{\pgfqpoint{3.167023in}{1.811594in}}%
\pgfpathlineto{\pgfqpoint{3.069117in}{2.483742in}}%
\pgfusepath{stroke}%
\end{pgfscope}%
\begin{pgfscope}%
\pgfpathrectangle{\pgfqpoint{0.100000in}{0.212622in}}{\pgfqpoint{3.696000in}{3.696000in}}%
\pgfusepath{clip}%
\pgfsetrectcap%
\pgfsetroundjoin%
\pgfsetlinewidth{1.505625pt}%
\definecolor{currentstroke}{rgb}{1.000000,0.000000,0.000000}%
\pgfsetstrokecolor{currentstroke}%
\pgfsetdash{}{0pt}%
\pgfpathmoveto{\pgfqpoint{3.165722in}{1.811678in}}%
\pgfpathlineto{\pgfqpoint{3.069117in}{2.483742in}}%
\pgfusepath{stroke}%
\end{pgfscope}%
\begin{pgfscope}%
\pgfpathrectangle{\pgfqpoint{0.100000in}{0.212622in}}{\pgfqpoint{3.696000in}{3.696000in}}%
\pgfusepath{clip}%
\pgfsetrectcap%
\pgfsetroundjoin%
\pgfsetlinewidth{1.505625pt}%
\definecolor{currentstroke}{rgb}{1.000000,0.000000,0.000000}%
\pgfsetstrokecolor{currentstroke}%
\pgfsetdash{}{0pt}%
\pgfpathmoveto{\pgfqpoint{3.165052in}{1.811743in}}%
\pgfpathlineto{\pgfqpoint{3.069117in}{2.483742in}}%
\pgfusepath{stroke}%
\end{pgfscope}%
\begin{pgfscope}%
\pgfpathrectangle{\pgfqpoint{0.100000in}{0.212622in}}{\pgfqpoint{3.696000in}{3.696000in}}%
\pgfusepath{clip}%
\pgfsetrectcap%
\pgfsetroundjoin%
\pgfsetlinewidth{1.505625pt}%
\definecolor{currentstroke}{rgb}{1.000000,0.000000,0.000000}%
\pgfsetstrokecolor{currentstroke}%
\pgfsetdash{}{0pt}%
\pgfpathmoveto{\pgfqpoint{3.163667in}{1.811943in}}%
\pgfpathlineto{\pgfqpoint{3.069117in}{2.483742in}}%
\pgfusepath{stroke}%
\end{pgfscope}%
\begin{pgfscope}%
\pgfpathrectangle{\pgfqpoint{0.100000in}{0.212622in}}{\pgfqpoint{3.696000in}{3.696000in}}%
\pgfusepath{clip}%
\pgfsetrectcap%
\pgfsetroundjoin%
\pgfsetlinewidth{1.505625pt}%
\definecolor{currentstroke}{rgb}{1.000000,0.000000,0.000000}%
\pgfsetstrokecolor{currentstroke}%
\pgfsetdash{}{0pt}%
\pgfpathmoveto{\pgfqpoint{3.162936in}{1.812032in}}%
\pgfpathlineto{\pgfqpoint{3.061158in}{2.476489in}}%
\pgfusepath{stroke}%
\end{pgfscope}%
\begin{pgfscope}%
\pgfpathrectangle{\pgfqpoint{0.100000in}{0.212622in}}{\pgfqpoint{3.696000in}{3.696000in}}%
\pgfusepath{clip}%
\pgfsetrectcap%
\pgfsetroundjoin%
\pgfsetlinewidth{1.505625pt}%
\definecolor{currentstroke}{rgb}{1.000000,0.000000,0.000000}%
\pgfsetstrokecolor{currentstroke}%
\pgfsetdash{}{0pt}%
\pgfpathmoveto{\pgfqpoint{3.162483in}{1.812084in}}%
\pgfpathlineto{\pgfqpoint{3.061158in}{2.476489in}}%
\pgfusepath{stroke}%
\end{pgfscope}%
\begin{pgfscope}%
\pgfpathrectangle{\pgfqpoint{0.100000in}{0.212622in}}{\pgfqpoint{3.696000in}{3.696000in}}%
\pgfusepath{clip}%
\pgfsetrectcap%
\pgfsetroundjoin%
\pgfsetlinewidth{1.505625pt}%
\definecolor{currentstroke}{rgb}{1.000000,0.000000,0.000000}%
\pgfsetstrokecolor{currentstroke}%
\pgfsetdash{}{0pt}%
\pgfpathmoveto{\pgfqpoint{3.160763in}{1.812400in}}%
\pgfpathlineto{\pgfqpoint{3.061158in}{2.476489in}}%
\pgfusepath{stroke}%
\end{pgfscope}%
\begin{pgfscope}%
\pgfpathrectangle{\pgfqpoint{0.100000in}{0.212622in}}{\pgfqpoint{3.696000in}{3.696000in}}%
\pgfusepath{clip}%
\pgfsetrectcap%
\pgfsetroundjoin%
\pgfsetlinewidth{1.505625pt}%
\definecolor{currentstroke}{rgb}{1.000000,0.000000,0.000000}%
\pgfsetstrokecolor{currentstroke}%
\pgfsetdash{}{0pt}%
\pgfpathmoveto{\pgfqpoint{3.158331in}{1.813016in}}%
\pgfpathlineto{\pgfqpoint{3.061158in}{2.476489in}}%
\pgfusepath{stroke}%
\end{pgfscope}%
\begin{pgfscope}%
\pgfpathrectangle{\pgfqpoint{0.100000in}{0.212622in}}{\pgfqpoint{3.696000in}{3.696000in}}%
\pgfusepath{clip}%
\pgfsetrectcap%
\pgfsetroundjoin%
\pgfsetlinewidth{1.505625pt}%
\definecolor{currentstroke}{rgb}{1.000000,0.000000,0.000000}%
\pgfsetstrokecolor{currentstroke}%
\pgfsetdash{}{0pt}%
\pgfpathmoveto{\pgfqpoint{3.156106in}{1.813171in}}%
\pgfpathlineto{\pgfqpoint{3.061158in}{2.476489in}}%
\pgfusepath{stroke}%
\end{pgfscope}%
\begin{pgfscope}%
\pgfpathrectangle{\pgfqpoint{0.100000in}{0.212622in}}{\pgfqpoint{3.696000in}{3.696000in}}%
\pgfusepath{clip}%
\pgfsetrectcap%
\pgfsetroundjoin%
\pgfsetlinewidth{1.505625pt}%
\definecolor{currentstroke}{rgb}{1.000000,0.000000,0.000000}%
\pgfsetstrokecolor{currentstroke}%
\pgfsetdash{}{0pt}%
\pgfpathmoveto{\pgfqpoint{3.155162in}{1.813190in}}%
\pgfpathlineto{\pgfqpoint{3.053188in}{2.469227in}}%
\pgfusepath{stroke}%
\end{pgfscope}%
\begin{pgfscope}%
\pgfpathrectangle{\pgfqpoint{0.100000in}{0.212622in}}{\pgfqpoint{3.696000in}{3.696000in}}%
\pgfusepath{clip}%
\pgfsetrectcap%
\pgfsetroundjoin%
\pgfsetlinewidth{1.505625pt}%
\definecolor{currentstroke}{rgb}{1.000000,0.000000,0.000000}%
\pgfsetstrokecolor{currentstroke}%
\pgfsetdash{}{0pt}%
\pgfpathmoveto{\pgfqpoint{3.152499in}{1.813672in}}%
\pgfpathlineto{\pgfqpoint{3.053188in}{2.469227in}}%
\pgfusepath{stroke}%
\end{pgfscope}%
\begin{pgfscope}%
\pgfpathrectangle{\pgfqpoint{0.100000in}{0.212622in}}{\pgfqpoint{3.696000in}{3.696000in}}%
\pgfusepath{clip}%
\pgfsetrectcap%
\pgfsetroundjoin%
\pgfsetlinewidth{1.505625pt}%
\definecolor{currentstroke}{rgb}{1.000000,0.000000,0.000000}%
\pgfsetstrokecolor{currentstroke}%
\pgfsetdash{}{0pt}%
\pgfpathmoveto{\pgfqpoint{3.151172in}{1.813889in}}%
\pgfpathlineto{\pgfqpoint{3.053188in}{2.469227in}}%
\pgfusepath{stroke}%
\end{pgfscope}%
\begin{pgfscope}%
\pgfpathrectangle{\pgfqpoint{0.100000in}{0.212622in}}{\pgfqpoint{3.696000in}{3.696000in}}%
\pgfusepath{clip}%
\pgfsetrectcap%
\pgfsetroundjoin%
\pgfsetlinewidth{1.505625pt}%
\definecolor{currentstroke}{rgb}{1.000000,0.000000,0.000000}%
\pgfsetstrokecolor{currentstroke}%
\pgfsetdash{}{0pt}%
\pgfpathmoveto{\pgfqpoint{3.150229in}{1.814021in}}%
\pgfpathlineto{\pgfqpoint{3.053188in}{2.469227in}}%
\pgfusepath{stroke}%
\end{pgfscope}%
\begin{pgfscope}%
\pgfpathrectangle{\pgfqpoint{0.100000in}{0.212622in}}{\pgfqpoint{3.696000in}{3.696000in}}%
\pgfusepath{clip}%
\pgfsetrectcap%
\pgfsetroundjoin%
\pgfsetlinewidth{1.505625pt}%
\definecolor{currentstroke}{rgb}{1.000000,0.000000,0.000000}%
\pgfsetstrokecolor{currentstroke}%
\pgfsetdash{}{0pt}%
\pgfpathmoveto{\pgfqpoint{3.147697in}{1.814283in}}%
\pgfpathlineto{\pgfqpoint{3.045209in}{2.461955in}}%
\pgfusepath{stroke}%
\end{pgfscope}%
\begin{pgfscope}%
\pgfpathrectangle{\pgfqpoint{0.100000in}{0.212622in}}{\pgfqpoint{3.696000in}{3.696000in}}%
\pgfusepath{clip}%
\pgfsetrectcap%
\pgfsetroundjoin%
\pgfsetlinewidth{1.505625pt}%
\definecolor{currentstroke}{rgb}{1.000000,0.000000,0.000000}%
\pgfsetstrokecolor{currentstroke}%
\pgfsetdash{}{0pt}%
\pgfpathmoveto{\pgfqpoint{3.144242in}{1.815136in}}%
\pgfpathlineto{\pgfqpoint{3.045209in}{2.461955in}}%
\pgfusepath{stroke}%
\end{pgfscope}%
\begin{pgfscope}%
\pgfpathrectangle{\pgfqpoint{0.100000in}{0.212622in}}{\pgfqpoint{3.696000in}{3.696000in}}%
\pgfusepath{clip}%
\pgfsetrectcap%
\pgfsetroundjoin%
\pgfsetlinewidth{1.505625pt}%
\definecolor{currentstroke}{rgb}{1.000000,0.000000,0.000000}%
\pgfsetstrokecolor{currentstroke}%
\pgfsetdash{}{0pt}%
\pgfpathmoveto{\pgfqpoint{3.141286in}{1.815442in}}%
\pgfpathlineto{\pgfqpoint{3.045209in}{2.461955in}}%
\pgfusepath{stroke}%
\end{pgfscope}%
\begin{pgfscope}%
\pgfpathrectangle{\pgfqpoint{0.100000in}{0.212622in}}{\pgfqpoint{3.696000in}{3.696000in}}%
\pgfusepath{clip}%
\pgfsetrectcap%
\pgfsetroundjoin%
\pgfsetlinewidth{1.505625pt}%
\definecolor{currentstroke}{rgb}{1.000000,0.000000,0.000000}%
\pgfsetstrokecolor{currentstroke}%
\pgfsetdash{}{0pt}%
\pgfpathmoveto{\pgfqpoint{3.138470in}{1.815607in}}%
\pgfpathlineto{\pgfqpoint{3.037219in}{2.454673in}}%
\pgfusepath{stroke}%
\end{pgfscope}%
\begin{pgfscope}%
\pgfpathrectangle{\pgfqpoint{0.100000in}{0.212622in}}{\pgfqpoint{3.696000in}{3.696000in}}%
\pgfusepath{clip}%
\pgfsetrectcap%
\pgfsetroundjoin%
\pgfsetlinewidth{1.505625pt}%
\definecolor{currentstroke}{rgb}{1.000000,0.000000,0.000000}%
\pgfsetstrokecolor{currentstroke}%
\pgfsetdash{}{0pt}%
\pgfpathmoveto{\pgfqpoint{3.135953in}{1.816112in}}%
\pgfpathlineto{\pgfqpoint{3.037219in}{2.454673in}}%
\pgfusepath{stroke}%
\end{pgfscope}%
\begin{pgfscope}%
\pgfpathrectangle{\pgfqpoint{0.100000in}{0.212622in}}{\pgfqpoint{3.696000in}{3.696000in}}%
\pgfusepath{clip}%
\pgfsetrectcap%
\pgfsetroundjoin%
\pgfsetlinewidth{1.505625pt}%
\definecolor{currentstroke}{rgb}{1.000000,0.000000,0.000000}%
\pgfsetstrokecolor{currentstroke}%
\pgfsetdash{}{0pt}%
\pgfpathmoveto{\pgfqpoint{3.133158in}{1.816686in}}%
\pgfpathlineto{\pgfqpoint{3.037219in}{2.454673in}}%
\pgfusepath{stroke}%
\end{pgfscope}%
\begin{pgfscope}%
\pgfpathrectangle{\pgfqpoint{0.100000in}{0.212622in}}{\pgfqpoint{3.696000in}{3.696000in}}%
\pgfusepath{clip}%
\pgfsetrectcap%
\pgfsetroundjoin%
\pgfsetlinewidth{1.505625pt}%
\definecolor{currentstroke}{rgb}{1.000000,0.000000,0.000000}%
\pgfsetstrokecolor{currentstroke}%
\pgfsetdash{}{0pt}%
\pgfpathmoveto{\pgfqpoint{3.131209in}{1.816812in}}%
\pgfpathlineto{\pgfqpoint{3.029218in}{2.447382in}}%
\pgfusepath{stroke}%
\end{pgfscope}%
\begin{pgfscope}%
\pgfpathrectangle{\pgfqpoint{0.100000in}{0.212622in}}{\pgfqpoint{3.696000in}{3.696000in}}%
\pgfusepath{clip}%
\pgfsetrectcap%
\pgfsetroundjoin%
\pgfsetlinewidth{1.505625pt}%
\definecolor{currentstroke}{rgb}{1.000000,0.000000,0.000000}%
\pgfsetstrokecolor{currentstroke}%
\pgfsetdash{}{0pt}%
\pgfpathmoveto{\pgfqpoint{3.127384in}{1.817537in}}%
\pgfpathlineto{\pgfqpoint{3.029218in}{2.447382in}}%
\pgfusepath{stroke}%
\end{pgfscope}%
\begin{pgfscope}%
\pgfpathrectangle{\pgfqpoint{0.100000in}{0.212622in}}{\pgfqpoint{3.696000in}{3.696000in}}%
\pgfusepath{clip}%
\pgfsetrectcap%
\pgfsetroundjoin%
\pgfsetlinewidth{1.505625pt}%
\definecolor{currentstroke}{rgb}{1.000000,0.000000,0.000000}%
\pgfsetstrokecolor{currentstroke}%
\pgfsetdash{}{0pt}%
\pgfpathmoveto{\pgfqpoint{3.121993in}{1.819006in}}%
\pgfpathlineto{\pgfqpoint{3.021207in}{2.440082in}}%
\pgfusepath{stroke}%
\end{pgfscope}%
\begin{pgfscope}%
\pgfpathrectangle{\pgfqpoint{0.100000in}{0.212622in}}{\pgfqpoint{3.696000in}{3.696000in}}%
\pgfusepath{clip}%
\pgfsetrectcap%
\pgfsetroundjoin%
\pgfsetlinewidth{1.505625pt}%
\definecolor{currentstroke}{rgb}{1.000000,0.000000,0.000000}%
\pgfsetstrokecolor{currentstroke}%
\pgfsetdash{}{0pt}%
\pgfpathmoveto{\pgfqpoint{3.118903in}{1.819102in}}%
\pgfpathlineto{\pgfqpoint{3.021207in}{2.440082in}}%
\pgfusepath{stroke}%
\end{pgfscope}%
\begin{pgfscope}%
\pgfpathrectangle{\pgfqpoint{0.100000in}{0.212622in}}{\pgfqpoint{3.696000in}{3.696000in}}%
\pgfusepath{clip}%
\pgfsetrectcap%
\pgfsetroundjoin%
\pgfsetlinewidth{1.505625pt}%
\definecolor{currentstroke}{rgb}{1.000000,0.000000,0.000000}%
\pgfsetstrokecolor{currentstroke}%
\pgfsetdash{}{0pt}%
\pgfpathmoveto{\pgfqpoint{3.114910in}{1.819284in}}%
\pgfpathlineto{\pgfqpoint{3.013186in}{2.432772in}}%
\pgfusepath{stroke}%
\end{pgfscope}%
\begin{pgfscope}%
\pgfpathrectangle{\pgfqpoint{0.100000in}{0.212622in}}{\pgfqpoint{3.696000in}{3.696000in}}%
\pgfusepath{clip}%
\pgfsetrectcap%
\pgfsetroundjoin%
\pgfsetlinewidth{1.505625pt}%
\definecolor{currentstroke}{rgb}{1.000000,0.000000,0.000000}%
\pgfsetstrokecolor{currentstroke}%
\pgfsetdash{}{0pt}%
\pgfpathmoveto{\pgfqpoint{3.108364in}{1.820427in}}%
\pgfpathlineto{\pgfqpoint{3.005154in}{2.425453in}}%
\pgfusepath{stroke}%
\end{pgfscope}%
\begin{pgfscope}%
\pgfpathrectangle{\pgfqpoint{0.100000in}{0.212622in}}{\pgfqpoint{3.696000in}{3.696000in}}%
\pgfusepath{clip}%
\pgfsetrectcap%
\pgfsetroundjoin%
\pgfsetlinewidth{1.505625pt}%
\definecolor{currentstroke}{rgb}{1.000000,0.000000,0.000000}%
\pgfsetstrokecolor{currentstroke}%
\pgfsetdash{}{0pt}%
\pgfpathmoveto{\pgfqpoint{3.102251in}{1.821093in}}%
\pgfpathlineto{\pgfqpoint{2.997112in}{2.418124in}}%
\pgfusepath{stroke}%
\end{pgfscope}%
\begin{pgfscope}%
\pgfpathrectangle{\pgfqpoint{0.100000in}{0.212622in}}{\pgfqpoint{3.696000in}{3.696000in}}%
\pgfusepath{clip}%
\pgfsetrectcap%
\pgfsetroundjoin%
\pgfsetlinewidth{1.505625pt}%
\definecolor{currentstroke}{rgb}{1.000000,0.000000,0.000000}%
\pgfsetstrokecolor{currentstroke}%
\pgfsetdash{}{0pt}%
\pgfpathmoveto{\pgfqpoint{3.099496in}{1.821456in}}%
\pgfpathlineto{\pgfqpoint{2.997112in}{2.418124in}}%
\pgfusepath{stroke}%
\end{pgfscope}%
\begin{pgfscope}%
\pgfpathrectangle{\pgfqpoint{0.100000in}{0.212622in}}{\pgfqpoint{3.696000in}{3.696000in}}%
\pgfusepath{clip}%
\pgfsetrectcap%
\pgfsetroundjoin%
\pgfsetlinewidth{1.505625pt}%
\definecolor{currentstroke}{rgb}{1.000000,0.000000,0.000000}%
\pgfsetstrokecolor{currentstroke}%
\pgfsetdash{}{0pt}%
\pgfpathmoveto{\pgfqpoint{3.092247in}{1.822712in}}%
\pgfpathlineto{\pgfqpoint{2.989059in}{2.410785in}}%
\pgfusepath{stroke}%
\end{pgfscope}%
\begin{pgfscope}%
\pgfpathrectangle{\pgfqpoint{0.100000in}{0.212622in}}{\pgfqpoint{3.696000in}{3.696000in}}%
\pgfusepath{clip}%
\pgfsetrectcap%
\pgfsetroundjoin%
\pgfsetlinewidth{1.505625pt}%
\definecolor{currentstroke}{rgb}{1.000000,0.000000,0.000000}%
\pgfsetstrokecolor{currentstroke}%
\pgfsetdash{}{0pt}%
\pgfpathmoveto{\pgfqpoint{3.083118in}{1.824868in}}%
\pgfpathlineto{\pgfqpoint{2.980995in}{2.403437in}}%
\pgfusepath{stroke}%
\end{pgfscope}%
\begin{pgfscope}%
\pgfpathrectangle{\pgfqpoint{0.100000in}{0.212622in}}{\pgfqpoint{3.696000in}{3.696000in}}%
\pgfusepath{clip}%
\pgfsetrectcap%
\pgfsetroundjoin%
\pgfsetlinewidth{1.505625pt}%
\definecolor{currentstroke}{rgb}{1.000000,0.000000,0.000000}%
\pgfsetstrokecolor{currentstroke}%
\pgfsetdash{}{0pt}%
\pgfpathmoveto{\pgfqpoint{3.076842in}{1.824834in}}%
\pgfpathlineto{\pgfqpoint{2.972921in}{2.396079in}}%
\pgfusepath{stroke}%
\end{pgfscope}%
\begin{pgfscope}%
\pgfpathrectangle{\pgfqpoint{0.100000in}{0.212622in}}{\pgfqpoint{3.696000in}{3.696000in}}%
\pgfusepath{clip}%
\pgfsetrectcap%
\pgfsetroundjoin%
\pgfsetlinewidth{1.505625pt}%
\definecolor{currentstroke}{rgb}{1.000000,0.000000,0.000000}%
\pgfsetstrokecolor{currentstroke}%
\pgfsetdash{}{0pt}%
\pgfpathmoveto{\pgfqpoint{3.072270in}{1.824877in}}%
\pgfpathlineto{\pgfqpoint{2.964837in}{2.388712in}}%
\pgfusepath{stroke}%
\end{pgfscope}%
\begin{pgfscope}%
\pgfpathrectangle{\pgfqpoint{0.100000in}{0.212622in}}{\pgfqpoint{3.696000in}{3.696000in}}%
\pgfusepath{clip}%
\pgfsetrectcap%
\pgfsetroundjoin%
\pgfsetlinewidth{1.505625pt}%
\definecolor{currentstroke}{rgb}{1.000000,0.000000,0.000000}%
\pgfsetstrokecolor{currentstroke}%
\pgfsetdash{}{0pt}%
\pgfpathmoveto{\pgfqpoint{3.062324in}{1.826548in}}%
\pgfpathlineto{\pgfqpoint{2.956742in}{2.381335in}}%
\pgfusepath{stroke}%
\end{pgfscope}%
\begin{pgfscope}%
\pgfpathrectangle{\pgfqpoint{0.100000in}{0.212622in}}{\pgfqpoint{3.696000in}{3.696000in}}%
\pgfusepath{clip}%
\pgfsetrectcap%
\pgfsetroundjoin%
\pgfsetlinewidth{1.505625pt}%
\definecolor{currentstroke}{rgb}{1.000000,0.000000,0.000000}%
\pgfsetstrokecolor{currentstroke}%
\pgfsetdash{}{0pt}%
\pgfpathmoveto{\pgfqpoint{3.051188in}{1.829042in}}%
\pgfpathlineto{\pgfqpoint{2.948636in}{2.373948in}}%
\pgfusepath{stroke}%
\end{pgfscope}%
\begin{pgfscope}%
\pgfpathrectangle{\pgfqpoint{0.100000in}{0.212622in}}{\pgfqpoint{3.696000in}{3.696000in}}%
\pgfusepath{clip}%
\pgfsetrectcap%
\pgfsetroundjoin%
\pgfsetlinewidth{1.505625pt}%
\definecolor{currentstroke}{rgb}{1.000000,0.000000,0.000000}%
\pgfsetstrokecolor{currentstroke}%
\pgfsetdash{}{0pt}%
\pgfpathmoveto{\pgfqpoint{3.046154in}{1.829519in}}%
\pgfpathlineto{\pgfqpoint{2.932392in}{2.359145in}}%
\pgfusepath{stroke}%
\end{pgfscope}%
\begin{pgfscope}%
\pgfpathrectangle{\pgfqpoint{0.100000in}{0.212622in}}{\pgfqpoint{3.696000in}{3.696000in}}%
\pgfusepath{clip}%
\pgfsetrectcap%
\pgfsetroundjoin%
\pgfsetlinewidth{1.505625pt}%
\definecolor{currentstroke}{rgb}{1.000000,0.000000,0.000000}%
\pgfsetstrokecolor{currentstroke}%
\pgfsetdash{}{0pt}%
\pgfpathmoveto{\pgfqpoint{3.041474in}{1.829744in}}%
\pgfpathlineto{\pgfqpoint{2.932392in}{2.359145in}}%
\pgfusepath{stroke}%
\end{pgfscope}%
\begin{pgfscope}%
\pgfpathrectangle{\pgfqpoint{0.100000in}{0.212622in}}{\pgfqpoint{3.696000in}{3.696000in}}%
\pgfusepath{clip}%
\pgfsetrectcap%
\pgfsetroundjoin%
\pgfsetlinewidth{1.505625pt}%
\definecolor{currentstroke}{rgb}{1.000000,0.000000,0.000000}%
\pgfsetstrokecolor{currentstroke}%
\pgfsetdash{}{0pt}%
\pgfpathmoveto{\pgfqpoint{3.038121in}{1.830188in}}%
\pgfpathlineto{\pgfqpoint{2.924255in}{2.351730in}}%
\pgfusepath{stroke}%
\end{pgfscope}%
\begin{pgfscope}%
\pgfpathrectangle{\pgfqpoint{0.100000in}{0.212622in}}{\pgfqpoint{3.696000in}{3.696000in}}%
\pgfusepath{clip}%
\pgfsetrectcap%
\pgfsetroundjoin%
\pgfsetlinewidth{1.505625pt}%
\definecolor{currentstroke}{rgb}{1.000000,0.000000,0.000000}%
\pgfsetstrokecolor{currentstroke}%
\pgfsetdash{}{0pt}%
\pgfpathmoveto{\pgfqpoint{3.034342in}{1.830526in}}%
\pgfpathlineto{\pgfqpoint{2.924255in}{2.351730in}}%
\pgfusepath{stroke}%
\end{pgfscope}%
\begin{pgfscope}%
\pgfpathrectangle{\pgfqpoint{0.100000in}{0.212622in}}{\pgfqpoint{3.696000in}{3.696000in}}%
\pgfusepath{clip}%
\pgfsetrectcap%
\pgfsetroundjoin%
\pgfsetlinewidth{1.505625pt}%
\definecolor{currentstroke}{rgb}{1.000000,0.000000,0.000000}%
\pgfsetstrokecolor{currentstroke}%
\pgfsetdash{}{0pt}%
\pgfpathmoveto{\pgfqpoint{3.031916in}{1.830646in}}%
\pgfpathlineto{\pgfqpoint{2.916106in}{2.344304in}}%
\pgfusepath{stroke}%
\end{pgfscope}%
\begin{pgfscope}%
\pgfpathrectangle{\pgfqpoint{0.100000in}{0.212622in}}{\pgfqpoint{3.696000in}{3.696000in}}%
\pgfusepath{clip}%
\pgfsetrectcap%
\pgfsetroundjoin%
\pgfsetlinewidth{1.505625pt}%
\definecolor{currentstroke}{rgb}{1.000000,0.000000,0.000000}%
\pgfsetstrokecolor{currentstroke}%
\pgfsetdash{}{0pt}%
\pgfpathmoveto{\pgfqpoint{3.026544in}{1.831277in}}%
\pgfpathlineto{\pgfqpoint{2.916106in}{2.344304in}}%
\pgfusepath{stroke}%
\end{pgfscope}%
\begin{pgfscope}%
\pgfpathrectangle{\pgfqpoint{0.100000in}{0.212622in}}{\pgfqpoint{3.696000in}{3.696000in}}%
\pgfusepath{clip}%
\pgfsetrectcap%
\pgfsetroundjoin%
\pgfsetlinewidth{1.505625pt}%
\definecolor{currentstroke}{rgb}{1.000000,0.000000,0.000000}%
\pgfsetstrokecolor{currentstroke}%
\pgfsetdash{}{0pt}%
\pgfpathmoveto{\pgfqpoint{3.019661in}{1.832912in}}%
\pgfpathlineto{\pgfqpoint{2.907947in}{2.336868in}}%
\pgfusepath{stroke}%
\end{pgfscope}%
\begin{pgfscope}%
\pgfpathrectangle{\pgfqpoint{0.100000in}{0.212622in}}{\pgfqpoint{3.696000in}{3.696000in}}%
\pgfusepath{clip}%
\pgfsetrectcap%
\pgfsetroundjoin%
\pgfsetlinewidth{1.505625pt}%
\definecolor{currentstroke}{rgb}{1.000000,0.000000,0.000000}%
\pgfsetstrokecolor{currentstroke}%
\pgfsetdash{}{0pt}%
\pgfpathmoveto{\pgfqpoint{3.016158in}{1.833218in}}%
\pgfpathlineto{\pgfqpoint{2.899777in}{2.329423in}}%
\pgfusepath{stroke}%
\end{pgfscope}%
\begin{pgfscope}%
\pgfpathrectangle{\pgfqpoint{0.100000in}{0.212622in}}{\pgfqpoint{3.696000in}{3.696000in}}%
\pgfusepath{clip}%
\pgfsetrectcap%
\pgfsetroundjoin%
\pgfsetlinewidth{1.505625pt}%
\definecolor{currentstroke}{rgb}{1.000000,0.000000,0.000000}%
\pgfsetstrokecolor{currentstroke}%
\pgfsetdash{}{0pt}%
\pgfpathmoveto{\pgfqpoint{3.012958in}{1.833526in}}%
\pgfpathlineto{\pgfqpoint{2.899777in}{2.329423in}}%
\pgfusepath{stroke}%
\end{pgfscope}%
\begin{pgfscope}%
\pgfpathrectangle{\pgfqpoint{0.100000in}{0.212622in}}{\pgfqpoint{3.696000in}{3.696000in}}%
\pgfusepath{clip}%
\pgfsetrectcap%
\pgfsetroundjoin%
\pgfsetlinewidth{1.505625pt}%
\definecolor{currentstroke}{rgb}{1.000000,0.000000,0.000000}%
\pgfsetstrokecolor{currentstroke}%
\pgfsetdash{}{0pt}%
\pgfpathmoveto{\pgfqpoint{3.010553in}{1.834042in}}%
\pgfpathlineto{\pgfqpoint{2.891597in}{2.321968in}}%
\pgfusepath{stroke}%
\end{pgfscope}%
\begin{pgfscope}%
\pgfpathrectangle{\pgfqpoint{0.100000in}{0.212622in}}{\pgfqpoint{3.696000in}{3.696000in}}%
\pgfusepath{clip}%
\pgfsetrectcap%
\pgfsetroundjoin%
\pgfsetlinewidth{1.505625pt}%
\definecolor{currentstroke}{rgb}{1.000000,0.000000,0.000000}%
\pgfsetstrokecolor{currentstroke}%
\pgfsetdash{}{0pt}%
\pgfpathmoveto{\pgfqpoint{3.007786in}{1.834223in}}%
\pgfpathlineto{\pgfqpoint{2.891597in}{2.321968in}}%
\pgfusepath{stroke}%
\end{pgfscope}%
\begin{pgfscope}%
\pgfpathrectangle{\pgfqpoint{0.100000in}{0.212622in}}{\pgfqpoint{3.696000in}{3.696000in}}%
\pgfusepath{clip}%
\pgfsetrectcap%
\pgfsetroundjoin%
\pgfsetlinewidth{1.505625pt}%
\definecolor{currentstroke}{rgb}{1.000000,0.000000,0.000000}%
\pgfsetstrokecolor{currentstroke}%
\pgfsetdash{}{0pt}%
\pgfpathmoveto{\pgfqpoint{3.006961in}{1.834312in}}%
\pgfpathlineto{\pgfqpoint{2.891597in}{2.321968in}}%
\pgfusepath{stroke}%
\end{pgfscope}%
\begin{pgfscope}%
\pgfpathrectangle{\pgfqpoint{0.100000in}{0.212622in}}{\pgfqpoint{3.696000in}{3.696000in}}%
\pgfusepath{clip}%
\pgfsetrectcap%
\pgfsetroundjoin%
\pgfsetlinewidth{1.505625pt}%
\definecolor{currentstroke}{rgb}{1.000000,0.000000,0.000000}%
\pgfsetstrokecolor{currentstroke}%
\pgfsetdash{}{0pt}%
\pgfpathmoveto{\pgfqpoint{3.004469in}{1.834733in}}%
\pgfpathlineto{\pgfqpoint{2.883405in}{2.314503in}}%
\pgfusepath{stroke}%
\end{pgfscope}%
\begin{pgfscope}%
\pgfpathrectangle{\pgfqpoint{0.100000in}{0.212622in}}{\pgfqpoint{3.696000in}{3.696000in}}%
\pgfusepath{clip}%
\pgfsetrectcap%
\pgfsetroundjoin%
\pgfsetlinewidth{1.505625pt}%
\definecolor{currentstroke}{rgb}{1.000000,0.000000,0.000000}%
\pgfsetstrokecolor{currentstroke}%
\pgfsetdash{}{0pt}%
\pgfpathmoveto{\pgfqpoint{3.001230in}{1.835328in}}%
\pgfpathlineto{\pgfqpoint{2.883405in}{2.314503in}}%
\pgfusepath{stroke}%
\end{pgfscope}%
\begin{pgfscope}%
\pgfpathrectangle{\pgfqpoint{0.100000in}{0.212622in}}{\pgfqpoint{3.696000in}{3.696000in}}%
\pgfusepath{clip}%
\pgfsetrectcap%
\pgfsetroundjoin%
\pgfsetlinewidth{1.505625pt}%
\definecolor{currentstroke}{rgb}{1.000000,0.000000,0.000000}%
\pgfsetstrokecolor{currentstroke}%
\pgfsetdash{}{0pt}%
\pgfpathmoveto{\pgfqpoint{2.999330in}{1.835474in}}%
\pgfpathlineto{\pgfqpoint{2.883405in}{2.314503in}}%
\pgfusepath{stroke}%
\end{pgfscope}%
\begin{pgfscope}%
\pgfpathrectangle{\pgfqpoint{0.100000in}{0.212622in}}{\pgfqpoint{3.696000in}{3.696000in}}%
\pgfusepath{clip}%
\pgfsetrectcap%
\pgfsetroundjoin%
\pgfsetlinewidth{1.505625pt}%
\definecolor{currentstroke}{rgb}{1.000000,0.000000,0.000000}%
\pgfsetstrokecolor{currentstroke}%
\pgfsetdash{}{0pt}%
\pgfpathmoveto{\pgfqpoint{2.997049in}{1.835879in}}%
\pgfpathlineto{\pgfqpoint{2.875203in}{2.307029in}}%
\pgfusepath{stroke}%
\end{pgfscope}%
\begin{pgfscope}%
\pgfpathrectangle{\pgfqpoint{0.100000in}{0.212622in}}{\pgfqpoint{3.696000in}{3.696000in}}%
\pgfusepath{clip}%
\pgfsetrectcap%
\pgfsetroundjoin%
\pgfsetlinewidth{1.505625pt}%
\definecolor{currentstroke}{rgb}{1.000000,0.000000,0.000000}%
\pgfsetstrokecolor{currentstroke}%
\pgfsetdash{}{0pt}%
\pgfpathmoveto{\pgfqpoint{2.995477in}{1.836232in}}%
\pgfpathlineto{\pgfqpoint{2.875203in}{2.307029in}}%
\pgfusepath{stroke}%
\end{pgfscope}%
\begin{pgfscope}%
\pgfpathrectangle{\pgfqpoint{0.100000in}{0.212622in}}{\pgfqpoint{3.696000in}{3.696000in}}%
\pgfusepath{clip}%
\pgfsetrectcap%
\pgfsetroundjoin%
\pgfsetlinewidth{1.505625pt}%
\definecolor{currentstroke}{rgb}{1.000000,0.000000,0.000000}%
\pgfsetstrokecolor{currentstroke}%
\pgfsetdash{}{0pt}%
\pgfpathmoveto{\pgfqpoint{2.993790in}{1.836293in}}%
\pgfpathlineto{\pgfqpoint{2.875203in}{2.307029in}}%
\pgfusepath{stroke}%
\end{pgfscope}%
\begin{pgfscope}%
\pgfpathrectangle{\pgfqpoint{0.100000in}{0.212622in}}{\pgfqpoint{3.696000in}{3.696000in}}%
\pgfusepath{clip}%
\pgfsetrectcap%
\pgfsetroundjoin%
\pgfsetlinewidth{1.505625pt}%
\definecolor{currentstroke}{rgb}{1.000000,0.000000,0.000000}%
\pgfsetstrokecolor{currentstroke}%
\pgfsetdash{}{0pt}%
\pgfpathmoveto{\pgfqpoint{2.990969in}{1.836613in}}%
\pgfpathlineto{\pgfqpoint{2.866990in}{2.299544in}}%
\pgfusepath{stroke}%
\end{pgfscope}%
\begin{pgfscope}%
\pgfpathrectangle{\pgfqpoint{0.100000in}{0.212622in}}{\pgfqpoint{3.696000in}{3.696000in}}%
\pgfusepath{clip}%
\pgfsetrectcap%
\pgfsetroundjoin%
\pgfsetlinewidth{1.505625pt}%
\definecolor{currentstroke}{rgb}{1.000000,0.000000,0.000000}%
\pgfsetstrokecolor{currentstroke}%
\pgfsetdash{}{0pt}%
\pgfpathmoveto{\pgfqpoint{2.988783in}{1.837029in}}%
\pgfpathlineto{\pgfqpoint{2.866990in}{2.299544in}}%
\pgfusepath{stroke}%
\end{pgfscope}%
\begin{pgfscope}%
\pgfpathrectangle{\pgfqpoint{0.100000in}{0.212622in}}{\pgfqpoint{3.696000in}{3.696000in}}%
\pgfusepath{clip}%
\pgfsetrectcap%
\pgfsetroundjoin%
\pgfsetlinewidth{1.505625pt}%
\definecolor{currentstroke}{rgb}{1.000000,0.000000,0.000000}%
\pgfsetstrokecolor{currentstroke}%
\pgfsetdash{}{0pt}%
\pgfpathmoveto{\pgfqpoint{2.985729in}{1.837312in}}%
\pgfpathlineto{\pgfqpoint{2.866990in}{2.299544in}}%
\pgfusepath{stroke}%
\end{pgfscope}%
\begin{pgfscope}%
\pgfpathrectangle{\pgfqpoint{0.100000in}{0.212622in}}{\pgfqpoint{3.696000in}{3.696000in}}%
\pgfusepath{clip}%
\pgfsetrectcap%
\pgfsetroundjoin%
\pgfsetlinewidth{1.505625pt}%
\definecolor{currentstroke}{rgb}{1.000000,0.000000,0.000000}%
\pgfsetstrokecolor{currentstroke}%
\pgfsetdash{}{0pt}%
\pgfpathmoveto{\pgfqpoint{2.984697in}{1.837355in}}%
\pgfpathlineto{\pgfqpoint{2.866990in}{2.299544in}}%
\pgfusepath{stroke}%
\end{pgfscope}%
\begin{pgfscope}%
\pgfpathrectangle{\pgfqpoint{0.100000in}{0.212622in}}{\pgfqpoint{3.696000in}{3.696000in}}%
\pgfusepath{clip}%
\pgfsetrectcap%
\pgfsetroundjoin%
\pgfsetlinewidth{1.505625pt}%
\definecolor{currentstroke}{rgb}{1.000000,0.000000,0.000000}%
\pgfsetstrokecolor{currentstroke}%
\pgfsetdash{}{0pt}%
\pgfpathmoveto{\pgfqpoint{2.982187in}{1.837764in}}%
\pgfpathlineto{\pgfqpoint{2.858766in}{2.292050in}}%
\pgfusepath{stroke}%
\end{pgfscope}%
\begin{pgfscope}%
\pgfpathrectangle{\pgfqpoint{0.100000in}{0.212622in}}{\pgfqpoint{3.696000in}{3.696000in}}%
\pgfusepath{clip}%
\pgfsetrectcap%
\pgfsetroundjoin%
\pgfsetlinewidth{1.505625pt}%
\definecolor{currentstroke}{rgb}{1.000000,0.000000,0.000000}%
\pgfsetstrokecolor{currentstroke}%
\pgfsetdash{}{0pt}%
\pgfpathmoveto{\pgfqpoint{2.980706in}{1.838088in}}%
\pgfpathlineto{\pgfqpoint{2.858766in}{2.292050in}}%
\pgfusepath{stroke}%
\end{pgfscope}%
\begin{pgfscope}%
\pgfpathrectangle{\pgfqpoint{0.100000in}{0.212622in}}{\pgfqpoint{3.696000in}{3.696000in}}%
\pgfusepath{clip}%
\pgfsetrectcap%
\pgfsetroundjoin%
\pgfsetlinewidth{1.505625pt}%
\definecolor{currentstroke}{rgb}{1.000000,0.000000,0.000000}%
\pgfsetstrokecolor{currentstroke}%
\pgfsetdash{}{0pt}%
\pgfpathmoveto{\pgfqpoint{2.979716in}{1.838206in}}%
\pgfpathlineto{\pgfqpoint{2.858766in}{2.292050in}}%
\pgfusepath{stroke}%
\end{pgfscope}%
\begin{pgfscope}%
\pgfpathrectangle{\pgfqpoint{0.100000in}{0.212622in}}{\pgfqpoint{3.696000in}{3.696000in}}%
\pgfusepath{clip}%
\pgfsetrectcap%
\pgfsetroundjoin%
\pgfsetlinewidth{1.505625pt}%
\definecolor{currentstroke}{rgb}{1.000000,0.000000,0.000000}%
\pgfsetstrokecolor{currentstroke}%
\pgfsetdash{}{0pt}%
\pgfpathmoveto{\pgfqpoint{2.977292in}{1.838478in}}%
\pgfpathlineto{\pgfqpoint{2.858766in}{2.292050in}}%
\pgfusepath{stroke}%
\end{pgfscope}%
\begin{pgfscope}%
\pgfpathrectangle{\pgfqpoint{0.100000in}{0.212622in}}{\pgfqpoint{3.696000in}{3.696000in}}%
\pgfusepath{clip}%
\pgfsetrectcap%
\pgfsetroundjoin%
\pgfsetlinewidth{1.505625pt}%
\definecolor{currentstroke}{rgb}{1.000000,0.000000,0.000000}%
\pgfsetstrokecolor{currentstroke}%
\pgfsetdash{}{0pt}%
\pgfpathmoveto{\pgfqpoint{2.973882in}{1.839206in}}%
\pgfpathlineto{\pgfqpoint{2.850531in}{2.284545in}}%
\pgfusepath{stroke}%
\end{pgfscope}%
\begin{pgfscope}%
\pgfpathrectangle{\pgfqpoint{0.100000in}{0.212622in}}{\pgfqpoint{3.696000in}{3.696000in}}%
\pgfusepath{clip}%
\pgfsetrectcap%
\pgfsetroundjoin%
\pgfsetlinewidth{1.505625pt}%
\definecolor{currentstroke}{rgb}{1.000000,0.000000,0.000000}%
\pgfsetstrokecolor{currentstroke}%
\pgfsetdash{}{0pt}%
\pgfpathmoveto{\pgfqpoint{2.971649in}{1.839407in}}%
\pgfpathlineto{\pgfqpoint{2.850531in}{2.284545in}}%
\pgfusepath{stroke}%
\end{pgfscope}%
\begin{pgfscope}%
\pgfpathrectangle{\pgfqpoint{0.100000in}{0.212622in}}{\pgfqpoint{3.696000in}{3.696000in}}%
\pgfusepath{clip}%
\pgfsetrectcap%
\pgfsetroundjoin%
\pgfsetlinewidth{1.505625pt}%
\definecolor{currentstroke}{rgb}{1.000000,0.000000,0.000000}%
\pgfsetstrokecolor{currentstroke}%
\pgfsetdash{}{0pt}%
\pgfpathmoveto{\pgfqpoint{2.968038in}{1.839610in}}%
\pgfpathlineto{\pgfqpoint{2.842285in}{2.277031in}}%
\pgfusepath{stroke}%
\end{pgfscope}%
\begin{pgfscope}%
\pgfpathrectangle{\pgfqpoint{0.100000in}{0.212622in}}{\pgfqpoint{3.696000in}{3.696000in}}%
\pgfusepath{clip}%
\pgfsetrectcap%
\pgfsetroundjoin%
\pgfsetlinewidth{1.505625pt}%
\definecolor{currentstroke}{rgb}{1.000000,0.000000,0.000000}%
\pgfsetstrokecolor{currentstroke}%
\pgfsetdash{}{0pt}%
\pgfpathmoveto{\pgfqpoint{2.965309in}{1.839986in}}%
\pgfpathlineto{\pgfqpoint{2.842285in}{2.277031in}}%
\pgfusepath{stroke}%
\end{pgfscope}%
\begin{pgfscope}%
\pgfpathrectangle{\pgfqpoint{0.100000in}{0.212622in}}{\pgfqpoint{3.696000in}{3.696000in}}%
\pgfusepath{clip}%
\pgfsetrectcap%
\pgfsetroundjoin%
\pgfsetlinewidth{1.505625pt}%
\definecolor{currentstroke}{rgb}{1.000000,0.000000,0.000000}%
\pgfsetstrokecolor{currentstroke}%
\pgfsetdash{}{0pt}%
\pgfpathmoveto{\pgfqpoint{2.962011in}{1.840150in}}%
\pgfpathlineto{\pgfqpoint{2.842285in}{2.277031in}}%
\pgfusepath{stroke}%
\end{pgfscope}%
\begin{pgfscope}%
\pgfpathrectangle{\pgfqpoint{0.100000in}{0.212622in}}{\pgfqpoint{3.696000in}{3.696000in}}%
\pgfusepath{clip}%
\pgfsetrectcap%
\pgfsetroundjoin%
\pgfsetlinewidth{1.505625pt}%
\definecolor{currentstroke}{rgb}{1.000000,0.000000,0.000000}%
\pgfsetstrokecolor{currentstroke}%
\pgfsetdash{}{0pt}%
\pgfpathmoveto{\pgfqpoint{2.960142in}{1.840229in}}%
\pgfpathlineto{\pgfqpoint{2.834029in}{2.269507in}}%
\pgfusepath{stroke}%
\end{pgfscope}%
\begin{pgfscope}%
\pgfpathrectangle{\pgfqpoint{0.100000in}{0.212622in}}{\pgfqpoint{3.696000in}{3.696000in}}%
\pgfusepath{clip}%
\pgfsetrectcap%
\pgfsetroundjoin%
\pgfsetlinewidth{1.505625pt}%
\definecolor{currentstroke}{rgb}{1.000000,0.000000,0.000000}%
\pgfsetstrokecolor{currentstroke}%
\pgfsetdash{}{0pt}%
\pgfpathmoveto{\pgfqpoint{2.955350in}{1.840896in}}%
\pgfpathlineto{\pgfqpoint{2.834029in}{2.269507in}}%
\pgfusepath{stroke}%
\end{pgfscope}%
\begin{pgfscope}%
\pgfpathrectangle{\pgfqpoint{0.100000in}{0.212622in}}{\pgfqpoint{3.696000in}{3.696000in}}%
\pgfusepath{clip}%
\pgfsetrectcap%
\pgfsetroundjoin%
\pgfsetlinewidth{1.505625pt}%
\definecolor{currentstroke}{rgb}{1.000000,0.000000,0.000000}%
\pgfsetstrokecolor{currentstroke}%
\pgfsetdash{}{0pt}%
\pgfpathmoveto{\pgfqpoint{2.949565in}{1.841845in}}%
\pgfpathlineto{\pgfqpoint{2.825761in}{2.261973in}}%
\pgfusepath{stroke}%
\end{pgfscope}%
\begin{pgfscope}%
\pgfpathrectangle{\pgfqpoint{0.100000in}{0.212622in}}{\pgfqpoint{3.696000in}{3.696000in}}%
\pgfusepath{clip}%
\pgfsetrectcap%
\pgfsetroundjoin%
\pgfsetlinewidth{1.505625pt}%
\definecolor{currentstroke}{rgb}{1.000000,0.000000,0.000000}%
\pgfsetstrokecolor{currentstroke}%
\pgfsetdash{}{0pt}%
\pgfpathmoveto{\pgfqpoint{2.946878in}{1.841818in}}%
\pgfpathlineto{\pgfqpoint{2.817483in}{2.254428in}}%
\pgfusepath{stroke}%
\end{pgfscope}%
\begin{pgfscope}%
\pgfpathrectangle{\pgfqpoint{0.100000in}{0.212622in}}{\pgfqpoint{3.696000in}{3.696000in}}%
\pgfusepath{clip}%
\pgfsetrectcap%
\pgfsetroundjoin%
\pgfsetlinewidth{1.505625pt}%
\definecolor{currentstroke}{rgb}{1.000000,0.000000,0.000000}%
\pgfsetstrokecolor{currentstroke}%
\pgfsetdash{}{0pt}%
\pgfpathmoveto{\pgfqpoint{2.941350in}{1.842588in}}%
\pgfpathlineto{\pgfqpoint{2.817483in}{2.254428in}}%
\pgfusepath{stroke}%
\end{pgfscope}%
\begin{pgfscope}%
\pgfpathrectangle{\pgfqpoint{0.100000in}{0.212622in}}{\pgfqpoint{3.696000in}{3.696000in}}%
\pgfusepath{clip}%
\pgfsetrectcap%
\pgfsetroundjoin%
\pgfsetlinewidth{1.505625pt}%
\definecolor{currentstroke}{rgb}{1.000000,0.000000,0.000000}%
\pgfsetstrokecolor{currentstroke}%
\pgfsetdash{}{0pt}%
\pgfpathmoveto{\pgfqpoint{2.934237in}{1.843939in}}%
\pgfpathlineto{\pgfqpoint{2.809193in}{2.246874in}}%
\pgfusepath{stroke}%
\end{pgfscope}%
\begin{pgfscope}%
\pgfpathrectangle{\pgfqpoint{0.100000in}{0.212622in}}{\pgfqpoint{3.696000in}{3.696000in}}%
\pgfusepath{clip}%
\pgfsetrectcap%
\pgfsetroundjoin%
\pgfsetlinewidth{1.505625pt}%
\definecolor{currentstroke}{rgb}{1.000000,0.000000,0.000000}%
\pgfsetstrokecolor{currentstroke}%
\pgfsetdash{}{0pt}%
\pgfpathmoveto{\pgfqpoint{2.929615in}{1.843941in}}%
\pgfpathlineto{\pgfqpoint{2.800893in}{2.239310in}}%
\pgfusepath{stroke}%
\end{pgfscope}%
\begin{pgfscope}%
\pgfpathrectangle{\pgfqpoint{0.100000in}{0.212622in}}{\pgfqpoint{3.696000in}{3.696000in}}%
\pgfusepath{clip}%
\pgfsetrectcap%
\pgfsetroundjoin%
\pgfsetlinewidth{1.505625pt}%
\definecolor{currentstroke}{rgb}{1.000000,0.000000,0.000000}%
\pgfsetstrokecolor{currentstroke}%
\pgfsetdash{}{0pt}%
\pgfpathmoveto{\pgfqpoint{2.927454in}{1.843850in}}%
\pgfpathlineto{\pgfqpoint{2.800893in}{2.239310in}}%
\pgfusepath{stroke}%
\end{pgfscope}%
\begin{pgfscope}%
\pgfpathrectangle{\pgfqpoint{0.100000in}{0.212622in}}{\pgfqpoint{3.696000in}{3.696000in}}%
\pgfusepath{clip}%
\pgfsetrectcap%
\pgfsetroundjoin%
\pgfsetlinewidth{1.505625pt}%
\definecolor{currentstroke}{rgb}{1.000000,0.000000,0.000000}%
\pgfsetstrokecolor{currentstroke}%
\pgfsetdash{}{0pt}%
\pgfpathmoveto{\pgfqpoint{2.922363in}{1.844732in}}%
\pgfpathlineto{\pgfqpoint{2.792581in}{2.231736in}}%
\pgfusepath{stroke}%
\end{pgfscope}%
\begin{pgfscope}%
\pgfpathrectangle{\pgfqpoint{0.100000in}{0.212622in}}{\pgfqpoint{3.696000in}{3.696000in}}%
\pgfusepath{clip}%
\pgfsetrectcap%
\pgfsetroundjoin%
\pgfsetlinewidth{1.505625pt}%
\definecolor{currentstroke}{rgb}{1.000000,0.000000,0.000000}%
\pgfsetstrokecolor{currentstroke}%
\pgfsetdash{}{0pt}%
\pgfpathmoveto{\pgfqpoint{2.919838in}{1.845113in}}%
\pgfpathlineto{\pgfqpoint{2.792581in}{2.231736in}}%
\pgfusepath{stroke}%
\end{pgfscope}%
\begin{pgfscope}%
\pgfpathrectangle{\pgfqpoint{0.100000in}{0.212622in}}{\pgfqpoint{3.696000in}{3.696000in}}%
\pgfusepath{clip}%
\pgfsetrectcap%
\pgfsetroundjoin%
\pgfsetlinewidth{1.505625pt}%
\definecolor{currentstroke}{rgb}{1.000000,0.000000,0.000000}%
\pgfsetstrokecolor{currentstroke}%
\pgfsetdash{}{0pt}%
\pgfpathmoveto{\pgfqpoint{2.918333in}{1.845209in}}%
\pgfpathlineto{\pgfqpoint{2.784259in}{2.224151in}}%
\pgfusepath{stroke}%
\end{pgfscope}%
\begin{pgfscope}%
\pgfpathrectangle{\pgfqpoint{0.100000in}{0.212622in}}{\pgfqpoint{3.696000in}{3.696000in}}%
\pgfusepath{clip}%
\pgfsetrectcap%
\pgfsetroundjoin%
\pgfsetlinewidth{1.505625pt}%
\definecolor{currentstroke}{rgb}{1.000000,0.000000,0.000000}%
\pgfsetstrokecolor{currentstroke}%
\pgfsetdash{}{0pt}%
\pgfpathmoveto{\pgfqpoint{2.914403in}{1.845826in}}%
\pgfpathlineto{\pgfqpoint{2.784259in}{2.224151in}}%
\pgfusepath{stroke}%
\end{pgfscope}%
\begin{pgfscope}%
\pgfpathrectangle{\pgfqpoint{0.100000in}{0.212622in}}{\pgfqpoint{3.696000in}{3.696000in}}%
\pgfusepath{clip}%
\pgfsetrectcap%
\pgfsetroundjoin%
\pgfsetlinewidth{1.505625pt}%
\definecolor{currentstroke}{rgb}{1.000000,0.000000,0.000000}%
\pgfsetstrokecolor{currentstroke}%
\pgfsetdash{}{0pt}%
\pgfpathmoveto{\pgfqpoint{2.911789in}{1.846371in}}%
\pgfpathlineto{\pgfqpoint{2.784259in}{2.224151in}}%
\pgfusepath{stroke}%
\end{pgfscope}%
\begin{pgfscope}%
\pgfpathrectangle{\pgfqpoint{0.100000in}{0.212622in}}{\pgfqpoint{3.696000in}{3.696000in}}%
\pgfusepath{clip}%
\pgfsetrectcap%
\pgfsetroundjoin%
\pgfsetlinewidth{1.505625pt}%
\definecolor{currentstroke}{rgb}{1.000000,0.000000,0.000000}%
\pgfsetstrokecolor{currentstroke}%
\pgfsetdash{}{0pt}%
\pgfpathmoveto{\pgfqpoint{2.910291in}{1.846534in}}%
\pgfpathlineto{\pgfqpoint{2.775925in}{2.216557in}}%
\pgfusepath{stroke}%
\end{pgfscope}%
\begin{pgfscope}%
\pgfpathrectangle{\pgfqpoint{0.100000in}{0.212622in}}{\pgfqpoint{3.696000in}{3.696000in}}%
\pgfusepath{clip}%
\pgfsetrectcap%
\pgfsetroundjoin%
\pgfsetlinewidth{1.505625pt}%
\definecolor{currentstroke}{rgb}{1.000000,0.000000,0.000000}%
\pgfsetstrokecolor{currentstroke}%
\pgfsetdash{}{0pt}%
\pgfpathmoveto{\pgfqpoint{2.908929in}{1.846542in}}%
\pgfpathlineto{\pgfqpoint{2.775925in}{2.216557in}}%
\pgfusepath{stroke}%
\end{pgfscope}%
\begin{pgfscope}%
\pgfpathrectangle{\pgfqpoint{0.100000in}{0.212622in}}{\pgfqpoint{3.696000in}{3.696000in}}%
\pgfusepath{clip}%
\pgfsetrectcap%
\pgfsetroundjoin%
\pgfsetlinewidth{1.505625pt}%
\definecolor{currentstroke}{rgb}{1.000000,0.000000,0.000000}%
\pgfsetstrokecolor{currentstroke}%
\pgfsetdash{}{0pt}%
\pgfpathmoveto{\pgfqpoint{2.907652in}{1.846880in}}%
\pgfpathlineto{\pgfqpoint{2.775925in}{2.216557in}}%
\pgfusepath{stroke}%
\end{pgfscope}%
\begin{pgfscope}%
\pgfpathrectangle{\pgfqpoint{0.100000in}{0.212622in}}{\pgfqpoint{3.696000in}{3.696000in}}%
\pgfusepath{clip}%
\pgfsetrectcap%
\pgfsetroundjoin%
\pgfsetlinewidth{1.505625pt}%
\definecolor{currentstroke}{rgb}{1.000000,0.000000,0.000000}%
\pgfsetstrokecolor{currentstroke}%
\pgfsetdash{}{0pt}%
\pgfpathmoveto{\pgfqpoint{2.906008in}{1.846949in}}%
\pgfpathlineto{\pgfqpoint{2.775925in}{2.216557in}}%
\pgfusepath{stroke}%
\end{pgfscope}%
\begin{pgfscope}%
\pgfpathrectangle{\pgfqpoint{0.100000in}{0.212622in}}{\pgfqpoint{3.696000in}{3.696000in}}%
\pgfusepath{clip}%
\pgfsetrectcap%
\pgfsetroundjoin%
\pgfsetlinewidth{1.505625pt}%
\definecolor{currentstroke}{rgb}{1.000000,0.000000,0.000000}%
\pgfsetstrokecolor{currentstroke}%
\pgfsetdash{}{0pt}%
\pgfpathmoveto{\pgfqpoint{2.905396in}{1.846970in}}%
\pgfpathlineto{\pgfqpoint{2.775925in}{2.216557in}}%
\pgfusepath{stroke}%
\end{pgfscope}%
\begin{pgfscope}%
\pgfpathrectangle{\pgfqpoint{0.100000in}{0.212622in}}{\pgfqpoint{3.696000in}{3.696000in}}%
\pgfusepath{clip}%
\pgfsetrectcap%
\pgfsetroundjoin%
\pgfsetlinewidth{1.505625pt}%
\definecolor{currentstroke}{rgb}{1.000000,0.000000,0.000000}%
\pgfsetstrokecolor{currentstroke}%
\pgfsetdash{}{0pt}%
\pgfpathmoveto{\pgfqpoint{2.903083in}{1.847268in}}%
\pgfpathlineto{\pgfqpoint{2.767580in}{2.208952in}}%
\pgfusepath{stroke}%
\end{pgfscope}%
\begin{pgfscope}%
\pgfpathrectangle{\pgfqpoint{0.100000in}{0.212622in}}{\pgfqpoint{3.696000in}{3.696000in}}%
\pgfusepath{clip}%
\pgfsetrectcap%
\pgfsetroundjoin%
\pgfsetlinewidth{1.505625pt}%
\definecolor{currentstroke}{rgb}{1.000000,0.000000,0.000000}%
\pgfsetstrokecolor{currentstroke}%
\pgfsetdash{}{0pt}%
\pgfpathmoveto{\pgfqpoint{2.901749in}{1.847465in}}%
\pgfpathlineto{\pgfqpoint{2.767580in}{2.208952in}}%
\pgfusepath{stroke}%
\end{pgfscope}%
\begin{pgfscope}%
\pgfpathrectangle{\pgfqpoint{0.100000in}{0.212622in}}{\pgfqpoint{3.696000in}{3.696000in}}%
\pgfusepath{clip}%
\pgfsetrectcap%
\pgfsetroundjoin%
\pgfsetlinewidth{1.505625pt}%
\definecolor{currentstroke}{rgb}{1.000000,0.000000,0.000000}%
\pgfsetstrokecolor{currentstroke}%
\pgfsetdash{}{0pt}%
\pgfpathmoveto{\pgfqpoint{2.901173in}{1.847461in}}%
\pgfpathlineto{\pgfqpoint{2.767580in}{2.208952in}}%
\pgfusepath{stroke}%
\end{pgfscope}%
\begin{pgfscope}%
\pgfpathrectangle{\pgfqpoint{0.100000in}{0.212622in}}{\pgfqpoint{3.696000in}{3.696000in}}%
\pgfusepath{clip}%
\pgfsetrectcap%
\pgfsetroundjoin%
\pgfsetlinewidth{1.505625pt}%
\definecolor{currentstroke}{rgb}{1.000000,0.000000,0.000000}%
\pgfsetstrokecolor{currentstroke}%
\pgfsetdash{}{0pt}%
\pgfpathmoveto{\pgfqpoint{2.898836in}{1.847821in}}%
\pgfpathlineto{\pgfqpoint{2.767580in}{2.208952in}}%
\pgfusepath{stroke}%
\end{pgfscope}%
\begin{pgfscope}%
\pgfpathrectangle{\pgfqpoint{0.100000in}{0.212622in}}{\pgfqpoint{3.696000in}{3.696000in}}%
\pgfusepath{clip}%
\pgfsetrectcap%
\pgfsetroundjoin%
\pgfsetlinewidth{1.505625pt}%
\definecolor{currentstroke}{rgb}{1.000000,0.000000,0.000000}%
\pgfsetstrokecolor{currentstroke}%
\pgfsetdash{}{0pt}%
\pgfpathmoveto{\pgfqpoint{2.895513in}{1.848772in}}%
\pgfpathlineto{\pgfqpoint{2.759224in}{2.201338in}}%
\pgfusepath{stroke}%
\end{pgfscope}%
\begin{pgfscope}%
\pgfpathrectangle{\pgfqpoint{0.100000in}{0.212622in}}{\pgfqpoint{3.696000in}{3.696000in}}%
\pgfusepath{clip}%
\pgfsetrectcap%
\pgfsetroundjoin%
\pgfsetlinewidth{1.505625pt}%
\definecolor{currentstroke}{rgb}{1.000000,0.000000,0.000000}%
\pgfsetstrokecolor{currentstroke}%
\pgfsetdash{}{0pt}%
\pgfpathmoveto{\pgfqpoint{2.893699in}{1.848795in}}%
\pgfpathlineto{\pgfqpoint{2.759224in}{2.201338in}}%
\pgfusepath{stroke}%
\end{pgfscope}%
\begin{pgfscope}%
\pgfpathrectangle{\pgfqpoint{0.100000in}{0.212622in}}{\pgfqpoint{3.696000in}{3.696000in}}%
\pgfusepath{clip}%
\pgfsetrectcap%
\pgfsetroundjoin%
\pgfsetlinewidth{1.505625pt}%
\definecolor{currentstroke}{rgb}{1.000000,0.000000,0.000000}%
\pgfsetstrokecolor{currentstroke}%
\pgfsetdash{}{0pt}%
\pgfpathmoveto{\pgfqpoint{2.891738in}{1.849044in}}%
\pgfpathlineto{\pgfqpoint{2.759224in}{2.201338in}}%
\pgfusepath{stroke}%
\end{pgfscope}%
\begin{pgfscope}%
\pgfpathrectangle{\pgfqpoint{0.100000in}{0.212622in}}{\pgfqpoint{3.696000in}{3.696000in}}%
\pgfusepath{clip}%
\pgfsetrectcap%
\pgfsetroundjoin%
\pgfsetlinewidth{1.505625pt}%
\definecolor{currentstroke}{rgb}{1.000000,0.000000,0.000000}%
\pgfsetstrokecolor{currentstroke}%
\pgfsetdash{}{0pt}%
\pgfpathmoveto{\pgfqpoint{2.888912in}{1.849656in}}%
\pgfpathlineto{\pgfqpoint{2.750857in}{2.193713in}}%
\pgfusepath{stroke}%
\end{pgfscope}%
\begin{pgfscope}%
\pgfpathrectangle{\pgfqpoint{0.100000in}{0.212622in}}{\pgfqpoint{3.696000in}{3.696000in}}%
\pgfusepath{clip}%
\pgfsetrectcap%
\pgfsetroundjoin%
\pgfsetlinewidth{1.505625pt}%
\definecolor{currentstroke}{rgb}{1.000000,0.000000,0.000000}%
\pgfsetstrokecolor{currentstroke}%
\pgfsetdash{}{0pt}%
\pgfpathmoveto{\pgfqpoint{2.887180in}{1.849854in}}%
\pgfpathlineto{\pgfqpoint{2.750857in}{2.193713in}}%
\pgfusepath{stroke}%
\end{pgfscope}%
\begin{pgfscope}%
\pgfpathrectangle{\pgfqpoint{0.100000in}{0.212622in}}{\pgfqpoint{3.696000in}{3.696000in}}%
\pgfusepath{clip}%
\pgfsetrectcap%
\pgfsetroundjoin%
\pgfsetlinewidth{1.505625pt}%
\definecolor{currentstroke}{rgb}{1.000000,0.000000,0.000000}%
\pgfsetstrokecolor{currentstroke}%
\pgfsetdash{}{0pt}%
\pgfpathmoveto{\pgfqpoint{2.885049in}{1.850152in}}%
\pgfpathlineto{\pgfqpoint{2.750857in}{2.193713in}}%
\pgfusepath{stroke}%
\end{pgfscope}%
\begin{pgfscope}%
\pgfpathrectangle{\pgfqpoint{0.100000in}{0.212622in}}{\pgfqpoint{3.696000in}{3.696000in}}%
\pgfusepath{clip}%
\pgfsetrectcap%
\pgfsetroundjoin%
\pgfsetlinewidth{1.505625pt}%
\definecolor{currentstroke}{rgb}{1.000000,0.000000,0.000000}%
\pgfsetstrokecolor{currentstroke}%
\pgfsetdash{}{0pt}%
\pgfpathmoveto{\pgfqpoint{2.883390in}{1.850604in}}%
\pgfpathlineto{\pgfqpoint{2.742479in}{2.186078in}}%
\pgfusepath{stroke}%
\end{pgfscope}%
\begin{pgfscope}%
\pgfpathrectangle{\pgfqpoint{0.100000in}{0.212622in}}{\pgfqpoint{3.696000in}{3.696000in}}%
\pgfusepath{clip}%
\pgfsetrectcap%
\pgfsetroundjoin%
\pgfsetlinewidth{1.505625pt}%
\definecolor{currentstroke}{rgb}{1.000000,0.000000,0.000000}%
\pgfsetstrokecolor{currentstroke}%
\pgfsetdash{}{0pt}%
\pgfpathmoveto{\pgfqpoint{2.882060in}{1.850618in}}%
\pgfpathlineto{\pgfqpoint{2.742479in}{2.186078in}}%
\pgfusepath{stroke}%
\end{pgfscope}%
\begin{pgfscope}%
\pgfpathrectangle{\pgfqpoint{0.100000in}{0.212622in}}{\pgfqpoint{3.696000in}{3.696000in}}%
\pgfusepath{clip}%
\pgfsetrectcap%
\pgfsetroundjoin%
\pgfsetlinewidth{1.505625pt}%
\definecolor{currentstroke}{rgb}{1.000000,0.000000,0.000000}%
\pgfsetstrokecolor{currentstroke}%
\pgfsetdash{}{0pt}%
\pgfpathmoveto{\pgfqpoint{2.880529in}{1.850769in}}%
\pgfpathlineto{\pgfqpoint{2.742479in}{2.186078in}}%
\pgfusepath{stroke}%
\end{pgfscope}%
\begin{pgfscope}%
\pgfpathrectangle{\pgfqpoint{0.100000in}{0.212622in}}{\pgfqpoint{3.696000in}{3.696000in}}%
\pgfusepath{clip}%
\pgfsetrectcap%
\pgfsetroundjoin%
\pgfsetlinewidth{1.505625pt}%
\definecolor{currentstroke}{rgb}{1.000000,0.000000,0.000000}%
\pgfsetstrokecolor{currentstroke}%
\pgfsetdash{}{0pt}%
\pgfpathmoveto{\pgfqpoint{2.877837in}{1.851462in}}%
\pgfpathlineto{\pgfqpoint{2.742479in}{2.186078in}}%
\pgfusepath{stroke}%
\end{pgfscope}%
\begin{pgfscope}%
\pgfpathrectangle{\pgfqpoint{0.100000in}{0.212622in}}{\pgfqpoint{3.696000in}{3.696000in}}%
\pgfusepath{clip}%
\pgfsetrectcap%
\pgfsetroundjoin%
\pgfsetlinewidth{1.505625pt}%
\definecolor{currentstroke}{rgb}{1.000000,0.000000,0.000000}%
\pgfsetstrokecolor{currentstroke}%
\pgfsetdash{}{0pt}%
\pgfpathmoveto{\pgfqpoint{2.875504in}{1.851419in}}%
\pgfpathlineto{\pgfqpoint{2.734090in}{2.178433in}}%
\pgfusepath{stroke}%
\end{pgfscope}%
\begin{pgfscope}%
\pgfpathrectangle{\pgfqpoint{0.100000in}{0.212622in}}{\pgfqpoint{3.696000in}{3.696000in}}%
\pgfusepath{clip}%
\pgfsetrectcap%
\pgfsetroundjoin%
\pgfsetlinewidth{1.505625pt}%
\definecolor{currentstroke}{rgb}{1.000000,0.000000,0.000000}%
\pgfsetstrokecolor{currentstroke}%
\pgfsetdash{}{0pt}%
\pgfpathmoveto{\pgfqpoint{2.874356in}{1.851437in}}%
\pgfpathlineto{\pgfqpoint{2.734090in}{2.178433in}}%
\pgfusepath{stroke}%
\end{pgfscope}%
\begin{pgfscope}%
\pgfpathrectangle{\pgfqpoint{0.100000in}{0.212622in}}{\pgfqpoint{3.696000in}{3.696000in}}%
\pgfusepath{clip}%
\pgfsetrectcap%
\pgfsetroundjoin%
\pgfsetlinewidth{1.505625pt}%
\definecolor{currentstroke}{rgb}{1.000000,0.000000,0.000000}%
\pgfsetstrokecolor{currentstroke}%
\pgfsetdash{}{0pt}%
\pgfpathmoveto{\pgfqpoint{2.871678in}{1.851852in}}%
\pgfpathlineto{\pgfqpoint{2.734090in}{2.178433in}}%
\pgfusepath{stroke}%
\end{pgfscope}%
\begin{pgfscope}%
\pgfpathrectangle{\pgfqpoint{0.100000in}{0.212622in}}{\pgfqpoint{3.696000in}{3.696000in}}%
\pgfusepath{clip}%
\pgfsetrectcap%
\pgfsetroundjoin%
\pgfsetlinewidth{1.505625pt}%
\definecolor{currentstroke}{rgb}{1.000000,0.000000,0.000000}%
\pgfsetstrokecolor{currentstroke}%
\pgfsetdash{}{0pt}%
\pgfpathmoveto{\pgfqpoint{2.868423in}{1.852277in}}%
\pgfpathlineto{\pgfqpoint{2.725689in}{2.170777in}}%
\pgfusepath{stroke}%
\end{pgfscope}%
\begin{pgfscope}%
\pgfpathrectangle{\pgfqpoint{0.100000in}{0.212622in}}{\pgfqpoint{3.696000in}{3.696000in}}%
\pgfusepath{clip}%
\pgfsetrectcap%
\pgfsetroundjoin%
\pgfsetlinewidth{1.505625pt}%
\definecolor{currentstroke}{rgb}{1.000000,0.000000,0.000000}%
\pgfsetstrokecolor{currentstroke}%
\pgfsetdash{}{0pt}%
\pgfpathmoveto{\pgfqpoint{2.866493in}{1.852439in}}%
\pgfpathlineto{\pgfqpoint{2.725689in}{2.170777in}}%
\pgfusepath{stroke}%
\end{pgfscope}%
\begin{pgfscope}%
\pgfpathrectangle{\pgfqpoint{0.100000in}{0.212622in}}{\pgfqpoint{3.696000in}{3.696000in}}%
\pgfusepath{clip}%
\pgfsetrectcap%
\pgfsetroundjoin%
\pgfsetlinewidth{1.505625pt}%
\definecolor{currentstroke}{rgb}{1.000000,0.000000,0.000000}%
\pgfsetstrokecolor{currentstroke}%
\pgfsetdash{}{0pt}%
\pgfpathmoveto{\pgfqpoint{2.860569in}{1.853951in}}%
\pgfpathlineto{\pgfqpoint{2.717278in}{2.163112in}}%
\pgfusepath{stroke}%
\end{pgfscope}%
\begin{pgfscope}%
\pgfpathrectangle{\pgfqpoint{0.100000in}{0.212622in}}{\pgfqpoint{3.696000in}{3.696000in}}%
\pgfusepath{clip}%
\pgfsetrectcap%
\pgfsetroundjoin%
\pgfsetlinewidth{1.505625pt}%
\definecolor{currentstroke}{rgb}{1.000000,0.000000,0.000000}%
\pgfsetstrokecolor{currentstroke}%
\pgfsetdash{}{0pt}%
\pgfpathmoveto{\pgfqpoint{2.854886in}{1.855005in}}%
\pgfpathlineto{\pgfqpoint{2.717278in}{2.163112in}}%
\pgfusepath{stroke}%
\end{pgfscope}%
\begin{pgfscope}%
\pgfpathrectangle{\pgfqpoint{0.100000in}{0.212622in}}{\pgfqpoint{3.696000in}{3.696000in}}%
\pgfusepath{clip}%
\pgfsetrectcap%
\pgfsetroundjoin%
\pgfsetlinewidth{1.505625pt}%
\definecolor{currentstroke}{rgb}{1.000000,0.000000,0.000000}%
\pgfsetstrokecolor{currentstroke}%
\pgfsetdash{}{0pt}%
\pgfpathmoveto{\pgfqpoint{2.851218in}{1.855232in}}%
\pgfpathlineto{\pgfqpoint{2.708855in}{2.155436in}}%
\pgfusepath{stroke}%
\end{pgfscope}%
\begin{pgfscope}%
\pgfpathrectangle{\pgfqpoint{0.100000in}{0.212622in}}{\pgfqpoint{3.696000in}{3.696000in}}%
\pgfusepath{clip}%
\pgfsetrectcap%
\pgfsetroundjoin%
\pgfsetlinewidth{1.505625pt}%
\definecolor{currentstroke}{rgb}{1.000000,0.000000,0.000000}%
\pgfsetstrokecolor{currentstroke}%
\pgfsetdash{}{0pt}%
\pgfpathmoveto{\pgfqpoint{2.842574in}{1.857458in}}%
\pgfpathlineto{\pgfqpoint{2.700420in}{2.147750in}}%
\pgfusepath{stroke}%
\end{pgfscope}%
\begin{pgfscope}%
\pgfpathrectangle{\pgfqpoint{0.100000in}{0.212622in}}{\pgfqpoint{3.696000in}{3.696000in}}%
\pgfusepath{clip}%
\pgfsetrectcap%
\pgfsetroundjoin%
\pgfsetlinewidth{1.505625pt}%
\definecolor{currentstroke}{rgb}{1.000000,0.000000,0.000000}%
\pgfsetstrokecolor{currentstroke}%
\pgfsetdash{}{0pt}%
\pgfpathmoveto{\pgfqpoint{2.838349in}{1.858363in}}%
\pgfpathlineto{\pgfqpoint{2.700420in}{2.147750in}}%
\pgfusepath{stroke}%
\end{pgfscope}%
\begin{pgfscope}%
\pgfpathrectangle{\pgfqpoint{0.100000in}{0.212622in}}{\pgfqpoint{3.696000in}{3.696000in}}%
\pgfusepath{clip}%
\pgfsetrectcap%
\pgfsetroundjoin%
\pgfsetlinewidth{1.505625pt}%
\definecolor{currentstroke}{rgb}{1.000000,0.000000,0.000000}%
\pgfsetstrokecolor{currentstroke}%
\pgfsetdash{}{0pt}%
\pgfpathmoveto{\pgfqpoint{2.835554in}{1.858676in}}%
\pgfpathlineto{\pgfqpoint{2.691975in}{2.140053in}}%
\pgfusepath{stroke}%
\end{pgfscope}%
\begin{pgfscope}%
\pgfpathrectangle{\pgfqpoint{0.100000in}{0.212622in}}{\pgfqpoint{3.696000in}{3.696000in}}%
\pgfusepath{clip}%
\pgfsetrectcap%
\pgfsetroundjoin%
\pgfsetlinewidth{1.505625pt}%
\definecolor{currentstroke}{rgb}{1.000000,0.000000,0.000000}%
\pgfsetstrokecolor{currentstroke}%
\pgfsetdash{}{0pt}%
\pgfpathmoveto{\pgfqpoint{2.829348in}{1.860113in}}%
\pgfpathlineto{\pgfqpoint{2.691975in}{2.140053in}}%
\pgfusepath{stroke}%
\end{pgfscope}%
\begin{pgfscope}%
\pgfpathrectangle{\pgfqpoint{0.100000in}{0.212622in}}{\pgfqpoint{3.696000in}{3.696000in}}%
\pgfusepath{clip}%
\pgfsetrectcap%
\pgfsetroundjoin%
\pgfsetlinewidth{1.505625pt}%
\definecolor{currentstroke}{rgb}{1.000000,0.000000,0.000000}%
\pgfsetstrokecolor{currentstroke}%
\pgfsetdash{}{0pt}%
\pgfpathmoveto{\pgfqpoint{2.821866in}{1.862141in}}%
\pgfpathlineto{\pgfqpoint{2.683518in}{2.132347in}}%
\pgfusepath{stroke}%
\end{pgfscope}%
\begin{pgfscope}%
\pgfpathrectangle{\pgfqpoint{0.100000in}{0.212622in}}{\pgfqpoint{3.696000in}{3.696000in}}%
\pgfusepath{clip}%
\pgfsetrectcap%
\pgfsetroundjoin%
\pgfsetlinewidth{1.505625pt}%
\definecolor{currentstroke}{rgb}{1.000000,0.000000,0.000000}%
\pgfsetstrokecolor{currentstroke}%
\pgfsetdash{}{0pt}%
\pgfpathmoveto{\pgfqpoint{2.817250in}{1.862784in}}%
\pgfpathlineto{\pgfqpoint{2.675050in}{2.124630in}}%
\pgfusepath{stroke}%
\end{pgfscope}%
\begin{pgfscope}%
\pgfpathrectangle{\pgfqpoint{0.100000in}{0.212622in}}{\pgfqpoint{3.696000in}{3.696000in}}%
\pgfusepath{clip}%
\pgfsetrectcap%
\pgfsetroundjoin%
\pgfsetlinewidth{1.505625pt}%
\definecolor{currentstroke}{rgb}{1.000000,0.000000,0.000000}%
\pgfsetstrokecolor{currentstroke}%
\pgfsetdash{}{0pt}%
\pgfpathmoveto{\pgfqpoint{2.809822in}{1.864276in}}%
\pgfpathlineto{\pgfqpoint{2.666570in}{2.116902in}}%
\pgfusepath{stroke}%
\end{pgfscope}%
\begin{pgfscope}%
\pgfpathrectangle{\pgfqpoint{0.100000in}{0.212622in}}{\pgfqpoint{3.696000in}{3.696000in}}%
\pgfusepath{clip}%
\pgfsetrectcap%
\pgfsetroundjoin%
\pgfsetlinewidth{1.505625pt}%
\definecolor{currentstroke}{rgb}{1.000000,0.000000,0.000000}%
\pgfsetstrokecolor{currentstroke}%
\pgfsetdash{}{0pt}%
\pgfpathmoveto{\pgfqpoint{2.799799in}{1.867084in}}%
\pgfpathlineto{\pgfqpoint{2.658079in}{2.109164in}}%
\pgfusepath{stroke}%
\end{pgfscope}%
\begin{pgfscope}%
\pgfpathrectangle{\pgfqpoint{0.100000in}{0.212622in}}{\pgfqpoint{3.696000in}{3.696000in}}%
\pgfusepath{clip}%
\pgfsetrectcap%
\pgfsetroundjoin%
\pgfsetlinewidth{1.505625pt}%
\definecolor{currentstroke}{rgb}{1.000000,0.000000,0.000000}%
\pgfsetstrokecolor{currentstroke}%
\pgfsetdash{}{0pt}%
\pgfpathmoveto{\pgfqpoint{2.795067in}{1.867410in}}%
\pgfpathlineto{\pgfqpoint{2.649577in}{2.101416in}}%
\pgfusepath{stroke}%
\end{pgfscope}%
\begin{pgfscope}%
\pgfpathrectangle{\pgfqpoint{0.100000in}{0.212622in}}{\pgfqpoint{3.696000in}{3.696000in}}%
\pgfusepath{clip}%
\pgfsetrectcap%
\pgfsetroundjoin%
\pgfsetlinewidth{1.505625pt}%
\definecolor{currentstroke}{rgb}{1.000000,0.000000,0.000000}%
\pgfsetstrokecolor{currentstroke}%
\pgfsetdash{}{0pt}%
\pgfpathmoveto{\pgfqpoint{2.789644in}{1.868441in}}%
\pgfpathlineto{\pgfqpoint{2.649577in}{2.101416in}}%
\pgfusepath{stroke}%
\end{pgfscope}%
\begin{pgfscope}%
\pgfpathrectangle{\pgfqpoint{0.100000in}{0.212622in}}{\pgfqpoint{3.696000in}{3.696000in}}%
\pgfusepath{clip}%
\pgfsetrectcap%
\pgfsetroundjoin%
\pgfsetlinewidth{1.505625pt}%
\definecolor{currentstroke}{rgb}{1.000000,0.000000,0.000000}%
\pgfsetstrokecolor{currentstroke}%
\pgfsetdash{}{0pt}%
\pgfpathmoveto{\pgfqpoint{2.782379in}{1.870620in}}%
\pgfpathlineto{\pgfqpoint{2.641063in}{2.093658in}}%
\pgfusepath{stroke}%
\end{pgfscope}%
\begin{pgfscope}%
\pgfpathrectangle{\pgfqpoint{0.100000in}{0.212622in}}{\pgfqpoint{3.696000in}{3.696000in}}%
\pgfusepath{clip}%
\pgfsetrectcap%
\pgfsetroundjoin%
\pgfsetlinewidth{1.505625pt}%
\definecolor{currentstroke}{rgb}{1.000000,0.000000,0.000000}%
\pgfsetstrokecolor{currentstroke}%
\pgfsetdash{}{0pt}%
\pgfpathmoveto{\pgfqpoint{2.777926in}{1.870783in}}%
\pgfpathlineto{\pgfqpoint{2.632538in}{2.085889in}}%
\pgfusepath{stroke}%
\end{pgfscope}%
\begin{pgfscope}%
\pgfpathrectangle{\pgfqpoint{0.100000in}{0.212622in}}{\pgfqpoint{3.696000in}{3.696000in}}%
\pgfusepath{clip}%
\pgfsetrectcap%
\pgfsetroundjoin%
\pgfsetlinewidth{1.505625pt}%
\definecolor{currentstroke}{rgb}{1.000000,0.000000,0.000000}%
\pgfsetstrokecolor{currentstroke}%
\pgfsetdash{}{0pt}%
\pgfpathmoveto{\pgfqpoint{2.769883in}{1.871862in}}%
\pgfpathlineto{\pgfqpoint{2.624001in}{2.078109in}}%
\pgfusepath{stroke}%
\end{pgfscope}%
\begin{pgfscope}%
\pgfpathrectangle{\pgfqpoint{0.100000in}{0.212622in}}{\pgfqpoint{3.696000in}{3.696000in}}%
\pgfusepath{clip}%
\pgfsetrectcap%
\pgfsetroundjoin%
\pgfsetlinewidth{1.505625pt}%
\definecolor{currentstroke}{rgb}{1.000000,0.000000,0.000000}%
\pgfsetstrokecolor{currentstroke}%
\pgfsetdash{}{0pt}%
\pgfpathmoveto{\pgfqpoint{2.759258in}{1.874760in}}%
\pgfpathlineto{\pgfqpoint{2.615453in}{2.070319in}}%
\pgfusepath{stroke}%
\end{pgfscope}%
\begin{pgfscope}%
\pgfpathrectangle{\pgfqpoint{0.100000in}{0.212622in}}{\pgfqpoint{3.696000in}{3.696000in}}%
\pgfusepath{clip}%
\pgfsetrectcap%
\pgfsetroundjoin%
\pgfsetlinewidth{1.505625pt}%
\definecolor{currentstroke}{rgb}{1.000000,0.000000,0.000000}%
\pgfsetstrokecolor{currentstroke}%
\pgfsetdash{}{0pt}%
\pgfpathmoveto{\pgfqpoint{2.753982in}{1.874815in}}%
\pgfpathlineto{\pgfqpoint{2.606893in}{2.062519in}}%
\pgfusepath{stroke}%
\end{pgfscope}%
\begin{pgfscope}%
\pgfpathrectangle{\pgfqpoint{0.100000in}{0.212622in}}{\pgfqpoint{3.696000in}{3.696000in}}%
\pgfusepath{clip}%
\pgfsetrectcap%
\pgfsetroundjoin%
\pgfsetlinewidth{1.505625pt}%
\definecolor{currentstroke}{rgb}{1.000000,0.000000,0.000000}%
\pgfsetstrokecolor{currentstroke}%
\pgfsetdash{}{0pt}%
\pgfpathmoveto{\pgfqpoint{2.748341in}{1.876083in}}%
\pgfpathlineto{\pgfqpoint{2.606893in}{2.062519in}}%
\pgfusepath{stroke}%
\end{pgfscope}%
\begin{pgfscope}%
\pgfpathrectangle{\pgfqpoint{0.100000in}{0.212622in}}{\pgfqpoint{3.696000in}{3.696000in}}%
\pgfusepath{clip}%
\pgfsetrectcap%
\pgfsetroundjoin%
\pgfsetlinewidth{1.505625pt}%
\definecolor{currentstroke}{rgb}{1.000000,0.000000,0.000000}%
\pgfsetstrokecolor{currentstroke}%
\pgfsetdash{}{0pt}%
\pgfpathmoveto{\pgfqpoint{2.740943in}{1.878493in}}%
\pgfpathlineto{\pgfqpoint{2.598322in}{2.054708in}}%
\pgfusepath{stroke}%
\end{pgfscope}%
\begin{pgfscope}%
\pgfpathrectangle{\pgfqpoint{0.100000in}{0.212622in}}{\pgfqpoint{3.696000in}{3.696000in}}%
\pgfusepath{clip}%
\pgfsetrectcap%
\pgfsetroundjoin%
\pgfsetlinewidth{1.505625pt}%
\definecolor{currentstroke}{rgb}{1.000000,0.000000,0.000000}%
\pgfsetstrokecolor{currentstroke}%
\pgfsetdash{}{0pt}%
\pgfpathmoveto{\pgfqpoint{2.735259in}{1.878908in}}%
\pgfpathlineto{\pgfqpoint{2.589739in}{2.046886in}}%
\pgfusepath{stroke}%
\end{pgfscope}%
\begin{pgfscope}%
\pgfpathrectangle{\pgfqpoint{0.100000in}{0.212622in}}{\pgfqpoint{3.696000in}{3.696000in}}%
\pgfusepath{clip}%
\pgfsetrectcap%
\pgfsetroundjoin%
\pgfsetlinewidth{1.505625pt}%
\definecolor{currentstroke}{rgb}{1.000000,0.000000,0.000000}%
\pgfsetstrokecolor{currentstroke}%
\pgfsetdash{}{0pt}%
\pgfpathmoveto{\pgfqpoint{2.730181in}{1.879869in}}%
\pgfpathlineto{\pgfqpoint{2.581145in}{2.039054in}}%
\pgfusepath{stroke}%
\end{pgfscope}%
\begin{pgfscope}%
\pgfpathrectangle{\pgfqpoint{0.100000in}{0.212622in}}{\pgfqpoint{3.696000in}{3.696000in}}%
\pgfusepath{clip}%
\pgfsetrectcap%
\pgfsetroundjoin%
\pgfsetlinewidth{1.505625pt}%
\definecolor{currentstroke}{rgb}{1.000000,0.000000,0.000000}%
\pgfsetstrokecolor{currentstroke}%
\pgfsetdash{}{0pt}%
\pgfpathmoveto{\pgfqpoint{2.724312in}{1.881504in}}%
\pgfpathlineto{\pgfqpoint{2.581145in}{2.039054in}}%
\pgfusepath{stroke}%
\end{pgfscope}%
\begin{pgfscope}%
\pgfpathrectangle{\pgfqpoint{0.100000in}{0.212622in}}{\pgfqpoint{3.696000in}{3.696000in}}%
\pgfusepath{clip}%
\pgfsetrectcap%
\pgfsetroundjoin%
\pgfsetlinewidth{1.505625pt}%
\definecolor{currentstroke}{rgb}{1.000000,0.000000,0.000000}%
\pgfsetstrokecolor{currentstroke}%
\pgfsetdash{}{0pt}%
\pgfpathmoveto{\pgfqpoint{2.719011in}{1.882102in}}%
\pgfpathlineto{\pgfqpoint{2.572539in}{2.031211in}}%
\pgfusepath{stroke}%
\end{pgfscope}%
\begin{pgfscope}%
\pgfpathrectangle{\pgfqpoint{0.100000in}{0.212622in}}{\pgfqpoint{3.696000in}{3.696000in}}%
\pgfusepath{clip}%
\pgfsetrectcap%
\pgfsetroundjoin%
\pgfsetlinewidth{1.505625pt}%
\definecolor{currentstroke}{rgb}{1.000000,0.000000,0.000000}%
\pgfsetstrokecolor{currentstroke}%
\pgfsetdash{}{0pt}%
\pgfpathmoveto{\pgfqpoint{2.714347in}{1.883229in}}%
\pgfpathlineto{\pgfqpoint{2.572539in}{2.031211in}}%
\pgfusepath{stroke}%
\end{pgfscope}%
\begin{pgfscope}%
\pgfpathrectangle{\pgfqpoint{0.100000in}{0.212622in}}{\pgfqpoint{3.696000in}{3.696000in}}%
\pgfusepath{clip}%
\pgfsetrectcap%
\pgfsetroundjoin%
\pgfsetlinewidth{1.505625pt}%
\definecolor{currentstroke}{rgb}{1.000000,0.000000,0.000000}%
\pgfsetstrokecolor{currentstroke}%
\pgfsetdash{}{0pt}%
\pgfpathmoveto{\pgfqpoint{2.709374in}{1.884474in}}%
\pgfpathlineto{\pgfqpoint{2.563921in}{2.023358in}}%
\pgfusepath{stroke}%
\end{pgfscope}%
\begin{pgfscope}%
\pgfpathrectangle{\pgfqpoint{0.100000in}{0.212622in}}{\pgfqpoint{3.696000in}{3.696000in}}%
\pgfusepath{clip}%
\pgfsetrectcap%
\pgfsetroundjoin%
\pgfsetlinewidth{1.505625pt}%
\definecolor{currentstroke}{rgb}{1.000000,0.000000,0.000000}%
\pgfsetstrokecolor{currentstroke}%
\pgfsetdash{}{0pt}%
\pgfpathmoveto{\pgfqpoint{2.704029in}{1.885060in}}%
\pgfpathlineto{\pgfqpoint{2.563921in}{2.023358in}}%
\pgfusepath{stroke}%
\end{pgfscope}%
\begin{pgfscope}%
\pgfpathrectangle{\pgfqpoint{0.100000in}{0.212622in}}{\pgfqpoint{3.696000in}{3.696000in}}%
\pgfusepath{clip}%
\pgfsetrectcap%
\pgfsetroundjoin%
\pgfsetlinewidth{1.505625pt}%
\definecolor{currentstroke}{rgb}{1.000000,0.000000,0.000000}%
\pgfsetstrokecolor{currentstroke}%
\pgfsetdash{}{0pt}%
\pgfpathmoveto{\pgfqpoint{2.696426in}{1.887027in}}%
\pgfpathlineto{\pgfqpoint{2.555292in}{2.015494in}}%
\pgfusepath{stroke}%
\end{pgfscope}%
\begin{pgfscope}%
\pgfpathrectangle{\pgfqpoint{0.100000in}{0.212622in}}{\pgfqpoint{3.696000in}{3.696000in}}%
\pgfusepath{clip}%
\pgfsetrectcap%
\pgfsetroundjoin%
\pgfsetlinewidth{1.505625pt}%
\definecolor{currentstroke}{rgb}{1.000000,0.000000,0.000000}%
\pgfsetstrokecolor{currentstroke}%
\pgfsetdash{}{0pt}%
\pgfpathmoveto{\pgfqpoint{2.687943in}{1.889464in}}%
\pgfpathlineto{\pgfqpoint{2.555292in}{2.015494in}}%
\pgfusepath{stroke}%
\end{pgfscope}%
\begin{pgfscope}%
\pgfpathrectangle{\pgfqpoint{0.100000in}{0.212622in}}{\pgfqpoint{3.696000in}{3.696000in}}%
\pgfusepath{clip}%
\pgfsetrectcap%
\pgfsetroundjoin%
\pgfsetlinewidth{1.505625pt}%
\definecolor{currentstroke}{rgb}{1.000000,0.000000,0.000000}%
\pgfsetstrokecolor{currentstroke}%
\pgfsetdash{}{0pt}%
\pgfpathmoveto{\pgfqpoint{2.681345in}{1.890080in}}%
\pgfpathlineto{\pgfqpoint{2.546651in}{2.007620in}}%
\pgfusepath{stroke}%
\end{pgfscope}%
\begin{pgfscope}%
\pgfpathrectangle{\pgfqpoint{0.100000in}{0.212622in}}{\pgfqpoint{3.696000in}{3.696000in}}%
\pgfusepath{clip}%
\pgfsetrectcap%
\pgfsetroundjoin%
\pgfsetlinewidth{1.505625pt}%
\definecolor{currentstroke}{rgb}{1.000000,0.000000,0.000000}%
\pgfsetstrokecolor{currentstroke}%
\pgfsetdash{}{0pt}%
\pgfpathmoveto{\pgfqpoint{2.676965in}{1.890691in}}%
\pgfpathlineto{\pgfqpoint{2.537998in}{1.999735in}}%
\pgfusepath{stroke}%
\end{pgfscope}%
\begin{pgfscope}%
\pgfpathrectangle{\pgfqpoint{0.100000in}{0.212622in}}{\pgfqpoint{3.696000in}{3.696000in}}%
\pgfusepath{clip}%
\pgfsetrectcap%
\pgfsetroundjoin%
\pgfsetlinewidth{1.505625pt}%
\definecolor{currentstroke}{rgb}{1.000000,0.000000,0.000000}%
\pgfsetstrokecolor{currentstroke}%
\pgfsetdash{}{0pt}%
\pgfpathmoveto{\pgfqpoint{2.670619in}{1.892724in}}%
\pgfpathlineto{\pgfqpoint{2.537998in}{1.999735in}}%
\pgfusepath{stroke}%
\end{pgfscope}%
\begin{pgfscope}%
\pgfpathrectangle{\pgfqpoint{0.100000in}{0.212622in}}{\pgfqpoint{3.696000in}{3.696000in}}%
\pgfusepath{clip}%
\pgfsetrectcap%
\pgfsetroundjoin%
\pgfsetlinewidth{1.505625pt}%
\definecolor{currentstroke}{rgb}{1.000000,0.000000,0.000000}%
\pgfsetstrokecolor{currentstroke}%
\pgfsetdash{}{0pt}%
\pgfpathmoveto{\pgfqpoint{2.664549in}{1.893308in}}%
\pgfpathlineto{\pgfqpoint{2.529334in}{1.991839in}}%
\pgfusepath{stroke}%
\end{pgfscope}%
\begin{pgfscope}%
\pgfpathrectangle{\pgfqpoint{0.100000in}{0.212622in}}{\pgfqpoint{3.696000in}{3.696000in}}%
\pgfusepath{clip}%
\pgfsetrectcap%
\pgfsetroundjoin%
\pgfsetlinewidth{1.505625pt}%
\definecolor{currentstroke}{rgb}{1.000000,0.000000,0.000000}%
\pgfsetstrokecolor{currentstroke}%
\pgfsetdash{}{0pt}%
\pgfpathmoveto{\pgfqpoint{2.659622in}{1.893788in}}%
\pgfpathlineto{\pgfqpoint{2.520657in}{1.983932in}}%
\pgfusepath{stroke}%
\end{pgfscope}%
\begin{pgfscope}%
\pgfpathrectangle{\pgfqpoint{0.100000in}{0.212622in}}{\pgfqpoint{3.696000in}{3.696000in}}%
\pgfusepath{clip}%
\pgfsetrectcap%
\pgfsetroundjoin%
\pgfsetlinewidth{1.505625pt}%
\definecolor{currentstroke}{rgb}{1.000000,0.000000,0.000000}%
\pgfsetstrokecolor{currentstroke}%
\pgfsetdash{}{0pt}%
\pgfpathmoveto{\pgfqpoint{2.649800in}{1.896245in}}%
\pgfpathlineto{\pgfqpoint{2.520657in}{1.983932in}}%
\pgfusepath{stroke}%
\end{pgfscope}%
\begin{pgfscope}%
\pgfpathrectangle{\pgfqpoint{0.100000in}{0.212622in}}{\pgfqpoint{3.696000in}{3.696000in}}%
\pgfusepath{clip}%
\pgfsetrectcap%
\pgfsetroundjoin%
\pgfsetlinewidth{1.505625pt}%
\definecolor{currentstroke}{rgb}{1.000000,0.000000,0.000000}%
\pgfsetstrokecolor{currentstroke}%
\pgfsetdash{}{0pt}%
\pgfpathmoveto{\pgfqpoint{2.644603in}{1.897488in}}%
\pgfpathlineto{\pgfqpoint{2.511969in}{1.976015in}}%
\pgfusepath{stroke}%
\end{pgfscope}%
\begin{pgfscope}%
\pgfpathrectangle{\pgfqpoint{0.100000in}{0.212622in}}{\pgfqpoint{3.696000in}{3.696000in}}%
\pgfusepath{clip}%
\pgfsetrectcap%
\pgfsetroundjoin%
\pgfsetlinewidth{1.505625pt}%
\definecolor{currentstroke}{rgb}{1.000000,0.000000,0.000000}%
\pgfsetstrokecolor{currentstroke}%
\pgfsetdash{}{0pt}%
\pgfpathmoveto{\pgfqpoint{2.641108in}{1.897733in}}%
\pgfpathlineto{\pgfqpoint{2.511969in}{1.976015in}}%
\pgfusepath{stroke}%
\end{pgfscope}%
\begin{pgfscope}%
\pgfpathrectangle{\pgfqpoint{0.100000in}{0.212622in}}{\pgfqpoint{3.696000in}{3.696000in}}%
\pgfusepath{clip}%
\pgfsetrectcap%
\pgfsetroundjoin%
\pgfsetlinewidth{1.505625pt}%
\definecolor{currentstroke}{rgb}{1.000000,0.000000,0.000000}%
\pgfsetstrokecolor{currentstroke}%
\pgfsetdash{}{0pt}%
\pgfpathmoveto{\pgfqpoint{2.634819in}{1.898939in}}%
\pgfpathlineto{\pgfqpoint{2.503269in}{1.968086in}}%
\pgfusepath{stroke}%
\end{pgfscope}%
\begin{pgfscope}%
\pgfpathrectangle{\pgfqpoint{0.100000in}{0.212622in}}{\pgfqpoint{3.696000in}{3.696000in}}%
\pgfusepath{clip}%
\pgfsetrectcap%
\pgfsetroundjoin%
\pgfsetlinewidth{1.505625pt}%
\definecolor{currentstroke}{rgb}{1.000000,0.000000,0.000000}%
\pgfsetstrokecolor{currentstroke}%
\pgfsetdash{}{0pt}%
\pgfpathmoveto{\pgfqpoint{2.626097in}{1.901686in}}%
\pgfpathlineto{\pgfqpoint{2.494558in}{1.960148in}}%
\pgfusepath{stroke}%
\end{pgfscope}%
\begin{pgfscope}%
\pgfpathrectangle{\pgfqpoint{0.100000in}{0.212622in}}{\pgfqpoint{3.696000in}{3.696000in}}%
\pgfusepath{clip}%
\pgfsetrectcap%
\pgfsetroundjoin%
\pgfsetlinewidth{1.505625pt}%
\definecolor{currentstroke}{rgb}{1.000000,0.000000,0.000000}%
\pgfsetstrokecolor{currentstroke}%
\pgfsetdash{}{0pt}%
\pgfpathmoveto{\pgfqpoint{2.621597in}{1.902013in}}%
\pgfpathlineto{\pgfqpoint{2.494558in}{1.960148in}}%
\pgfusepath{stroke}%
\end{pgfscope}%
\begin{pgfscope}%
\pgfpathrectangle{\pgfqpoint{0.100000in}{0.212622in}}{\pgfqpoint{3.696000in}{3.696000in}}%
\pgfusepath{clip}%
\pgfsetrectcap%
\pgfsetroundjoin%
\pgfsetlinewidth{1.505625pt}%
\definecolor{currentstroke}{rgb}{1.000000,0.000000,0.000000}%
\pgfsetstrokecolor{currentstroke}%
\pgfsetdash{}{0pt}%
\pgfpathmoveto{\pgfqpoint{2.617583in}{1.902494in}}%
\pgfpathlineto{\pgfqpoint{2.485834in}{1.952198in}}%
\pgfusepath{stroke}%
\end{pgfscope}%
\begin{pgfscope}%
\pgfpathrectangle{\pgfqpoint{0.100000in}{0.212622in}}{\pgfqpoint{3.696000in}{3.696000in}}%
\pgfusepath{clip}%
\pgfsetrectcap%
\pgfsetroundjoin%
\pgfsetlinewidth{1.505625pt}%
\definecolor{currentstroke}{rgb}{1.000000,0.000000,0.000000}%
\pgfsetstrokecolor{currentstroke}%
\pgfsetdash{}{0pt}%
\pgfpathmoveto{\pgfqpoint{2.610910in}{1.904302in}}%
\pgfpathlineto{\pgfqpoint{2.485834in}{1.952198in}}%
\pgfusepath{stroke}%
\end{pgfscope}%
\begin{pgfscope}%
\pgfpathrectangle{\pgfqpoint{0.100000in}{0.212622in}}{\pgfqpoint{3.696000in}{3.696000in}}%
\pgfusepath{clip}%
\pgfsetrectcap%
\pgfsetroundjoin%
\pgfsetlinewidth{1.505625pt}%
\definecolor{currentstroke}{rgb}{1.000000,0.000000,0.000000}%
\pgfsetstrokecolor{currentstroke}%
\pgfsetdash{}{0pt}%
\pgfpathmoveto{\pgfqpoint{2.604360in}{1.904900in}}%
\pgfpathlineto{\pgfqpoint{2.477099in}{1.944237in}}%
\pgfusepath{stroke}%
\end{pgfscope}%
\begin{pgfscope}%
\pgfpathrectangle{\pgfqpoint{0.100000in}{0.212622in}}{\pgfqpoint{3.696000in}{3.696000in}}%
\pgfusepath{clip}%
\pgfsetrectcap%
\pgfsetroundjoin%
\pgfsetlinewidth{1.505625pt}%
\definecolor{currentstroke}{rgb}{1.000000,0.000000,0.000000}%
\pgfsetstrokecolor{currentstroke}%
\pgfsetdash{}{0pt}%
\pgfpathmoveto{\pgfqpoint{2.601635in}{1.905197in}}%
\pgfpathlineto{\pgfqpoint{2.477099in}{1.944237in}}%
\pgfusepath{stroke}%
\end{pgfscope}%
\begin{pgfscope}%
\pgfpathrectangle{\pgfqpoint{0.100000in}{0.212622in}}{\pgfqpoint{3.696000in}{3.696000in}}%
\pgfusepath{clip}%
\pgfsetrectcap%
\pgfsetroundjoin%
\pgfsetlinewidth{1.505625pt}%
\definecolor{currentstroke}{rgb}{1.000000,0.000000,0.000000}%
\pgfsetstrokecolor{currentstroke}%
\pgfsetdash{}{0pt}%
\pgfpathmoveto{\pgfqpoint{2.596294in}{1.906667in}}%
\pgfpathlineto{\pgfqpoint{2.468352in}{1.936266in}}%
\pgfusepath{stroke}%
\end{pgfscope}%
\begin{pgfscope}%
\pgfpathrectangle{\pgfqpoint{0.100000in}{0.212622in}}{\pgfqpoint{3.696000in}{3.696000in}}%
\pgfusepath{clip}%
\pgfsetrectcap%
\pgfsetroundjoin%
\pgfsetlinewidth{1.505625pt}%
\definecolor{currentstroke}{rgb}{1.000000,0.000000,0.000000}%
\pgfsetstrokecolor{currentstroke}%
\pgfsetdash{}{0pt}%
\pgfpathmoveto{\pgfqpoint{2.591208in}{1.907781in}}%
\pgfpathlineto{\pgfqpoint{2.468352in}{1.936266in}}%
\pgfusepath{stroke}%
\end{pgfscope}%
\begin{pgfscope}%
\pgfpathrectangle{\pgfqpoint{0.100000in}{0.212622in}}{\pgfqpoint{3.696000in}{3.696000in}}%
\pgfusepath{clip}%
\pgfsetrectcap%
\pgfsetroundjoin%
\pgfsetlinewidth{1.505625pt}%
\definecolor{currentstroke}{rgb}{1.000000,0.000000,0.000000}%
\pgfsetstrokecolor{currentstroke}%
\pgfsetdash{}{0pt}%
\pgfpathmoveto{\pgfqpoint{2.587840in}{1.908141in}}%
\pgfpathlineto{\pgfqpoint{2.459592in}{1.928284in}}%
\pgfusepath{stroke}%
\end{pgfscope}%
\begin{pgfscope}%
\pgfpathrectangle{\pgfqpoint{0.100000in}{0.212622in}}{\pgfqpoint{3.696000in}{3.696000in}}%
\pgfusepath{clip}%
\pgfsetrectcap%
\pgfsetroundjoin%
\pgfsetlinewidth{1.505625pt}%
\definecolor{currentstroke}{rgb}{1.000000,0.000000,0.000000}%
\pgfsetstrokecolor{currentstroke}%
\pgfsetdash{}{0pt}%
\pgfpathmoveto{\pgfqpoint{2.582006in}{1.909206in}}%
\pgfpathlineto{\pgfqpoint{2.459592in}{1.928284in}}%
\pgfusepath{stroke}%
\end{pgfscope}%
\begin{pgfscope}%
\pgfpathrectangle{\pgfqpoint{0.100000in}{0.212622in}}{\pgfqpoint{3.696000in}{3.696000in}}%
\pgfusepath{clip}%
\pgfsetrectcap%
\pgfsetroundjoin%
\pgfsetlinewidth{1.505625pt}%
\definecolor{currentstroke}{rgb}{1.000000,0.000000,0.000000}%
\pgfsetstrokecolor{currentstroke}%
\pgfsetdash{}{0pt}%
\pgfpathmoveto{\pgfqpoint{2.578204in}{1.910213in}}%
\pgfpathlineto{\pgfqpoint{2.450821in}{1.920291in}}%
\pgfusepath{stroke}%
\end{pgfscope}%
\begin{pgfscope}%
\pgfpathrectangle{\pgfqpoint{0.100000in}{0.212622in}}{\pgfqpoint{3.696000in}{3.696000in}}%
\pgfusepath{clip}%
\pgfsetrectcap%
\pgfsetroundjoin%
\pgfsetlinewidth{1.505625pt}%
\definecolor{currentstroke}{rgb}{1.000000,0.000000,0.000000}%
\pgfsetstrokecolor{currentstroke}%
\pgfsetdash{}{0pt}%
\pgfpathmoveto{\pgfqpoint{2.574696in}{1.910625in}}%
\pgfpathlineto{\pgfqpoint{2.450821in}{1.920291in}}%
\pgfusepath{stroke}%
\end{pgfscope}%
\begin{pgfscope}%
\pgfpathrectangle{\pgfqpoint{0.100000in}{0.212622in}}{\pgfqpoint{3.696000in}{3.696000in}}%
\pgfusepath{clip}%
\pgfsetrectcap%
\pgfsetroundjoin%
\pgfsetlinewidth{1.505625pt}%
\definecolor{currentstroke}{rgb}{1.000000,0.000000,0.000000}%
\pgfsetstrokecolor{currentstroke}%
\pgfsetdash{}{0pt}%
\pgfpathmoveto{\pgfqpoint{2.570619in}{1.910955in}}%
\pgfpathlineto{\pgfqpoint{2.442038in}{1.912287in}}%
\pgfusepath{stroke}%
\end{pgfscope}%
\begin{pgfscope}%
\pgfpathrectangle{\pgfqpoint{0.100000in}{0.212622in}}{\pgfqpoint{3.696000in}{3.696000in}}%
\pgfusepath{clip}%
\pgfsetrectcap%
\pgfsetroundjoin%
\pgfsetlinewidth{1.505625pt}%
\definecolor{currentstroke}{rgb}{1.000000,0.000000,0.000000}%
\pgfsetstrokecolor{currentstroke}%
\pgfsetdash{}{0pt}%
\pgfpathmoveto{\pgfqpoint{2.564774in}{1.912237in}}%
\pgfpathlineto{\pgfqpoint{2.442038in}{1.912287in}}%
\pgfusepath{stroke}%
\end{pgfscope}%
\begin{pgfscope}%
\pgfpathrectangle{\pgfqpoint{0.100000in}{0.212622in}}{\pgfqpoint{3.696000in}{3.696000in}}%
\pgfusepath{clip}%
\pgfsetrectcap%
\pgfsetroundjoin%
\pgfsetlinewidth{1.505625pt}%
\definecolor{currentstroke}{rgb}{1.000000,0.000000,0.000000}%
\pgfsetstrokecolor{currentstroke}%
\pgfsetdash{}{0pt}%
\pgfpathmoveto{\pgfqpoint{2.558384in}{1.913005in}}%
\pgfpathlineto{\pgfqpoint{2.433243in}{1.904272in}}%
\pgfusepath{stroke}%
\end{pgfscope}%
\begin{pgfscope}%
\pgfpathrectangle{\pgfqpoint{0.100000in}{0.212622in}}{\pgfqpoint{3.696000in}{3.696000in}}%
\pgfusepath{clip}%
\pgfsetrectcap%
\pgfsetroundjoin%
\pgfsetlinewidth{1.505625pt}%
\definecolor{currentstroke}{rgb}{1.000000,0.000000,0.000000}%
\pgfsetstrokecolor{currentstroke}%
\pgfsetdash{}{0pt}%
\pgfpathmoveto{\pgfqpoint{2.555980in}{1.913139in}}%
\pgfpathlineto{\pgfqpoint{2.433243in}{1.904272in}}%
\pgfusepath{stroke}%
\end{pgfscope}%
\begin{pgfscope}%
\pgfpathrectangle{\pgfqpoint{0.100000in}{0.212622in}}{\pgfqpoint{3.696000in}{3.696000in}}%
\pgfusepath{clip}%
\pgfsetrectcap%
\pgfsetroundjoin%
\pgfsetlinewidth{1.505625pt}%
\definecolor{currentstroke}{rgb}{1.000000,0.000000,0.000000}%
\pgfsetstrokecolor{currentstroke}%
\pgfsetdash{}{0pt}%
\pgfpathmoveto{\pgfqpoint{2.550876in}{1.914153in}}%
\pgfpathlineto{\pgfqpoint{2.424436in}{1.896246in}}%
\pgfusepath{stroke}%
\end{pgfscope}%
\begin{pgfscope}%
\pgfpathrectangle{\pgfqpoint{0.100000in}{0.212622in}}{\pgfqpoint{3.696000in}{3.696000in}}%
\pgfusepath{clip}%
\pgfsetrectcap%
\pgfsetroundjoin%
\pgfsetlinewidth{1.505625pt}%
\definecolor{currentstroke}{rgb}{1.000000,0.000000,0.000000}%
\pgfsetstrokecolor{currentstroke}%
\pgfsetdash{}{0pt}%
\pgfpathmoveto{\pgfqpoint{2.544293in}{1.916212in}}%
\pgfpathlineto{\pgfqpoint{2.424436in}{1.896246in}}%
\pgfusepath{stroke}%
\end{pgfscope}%
\begin{pgfscope}%
\pgfpathrectangle{\pgfqpoint{0.100000in}{0.212622in}}{\pgfqpoint{3.696000in}{3.696000in}}%
\pgfusepath{clip}%
\pgfsetrectcap%
\pgfsetroundjoin%
\pgfsetlinewidth{1.505625pt}%
\definecolor{currentstroke}{rgb}{1.000000,0.000000,0.000000}%
\pgfsetstrokecolor{currentstroke}%
\pgfsetdash{}{0pt}%
\pgfpathmoveto{\pgfqpoint{2.540615in}{1.916396in}}%
\pgfpathlineto{\pgfqpoint{2.415617in}{1.888209in}}%
\pgfusepath{stroke}%
\end{pgfscope}%
\begin{pgfscope}%
\pgfpathrectangle{\pgfqpoint{0.100000in}{0.212622in}}{\pgfqpoint{3.696000in}{3.696000in}}%
\pgfusepath{clip}%
\pgfsetrectcap%
\pgfsetroundjoin%
\pgfsetlinewidth{1.505625pt}%
\definecolor{currentstroke}{rgb}{1.000000,0.000000,0.000000}%
\pgfsetstrokecolor{currentstroke}%
\pgfsetdash{}{0pt}%
\pgfpathmoveto{\pgfqpoint{2.537355in}{1.916943in}}%
\pgfpathlineto{\pgfqpoint{2.415617in}{1.888209in}}%
\pgfusepath{stroke}%
\end{pgfscope}%
\begin{pgfscope}%
\pgfpathrectangle{\pgfqpoint{0.100000in}{0.212622in}}{\pgfqpoint{3.696000in}{3.696000in}}%
\pgfusepath{clip}%
\pgfsetrectcap%
\pgfsetroundjoin%
\pgfsetlinewidth{1.505625pt}%
\definecolor{currentstroke}{rgb}{1.000000,0.000000,0.000000}%
\pgfsetstrokecolor{currentstroke}%
\pgfsetdash{}{0pt}%
\pgfpathmoveto{\pgfqpoint{2.535043in}{1.917599in}}%
\pgfpathlineto{\pgfqpoint{2.406785in}{1.880161in}}%
\pgfusepath{stroke}%
\end{pgfscope}%
\begin{pgfscope}%
\pgfpathrectangle{\pgfqpoint{0.100000in}{0.212622in}}{\pgfqpoint{3.696000in}{3.696000in}}%
\pgfusepath{clip}%
\pgfsetrectcap%
\pgfsetroundjoin%
\pgfsetlinewidth{1.505625pt}%
\definecolor{currentstroke}{rgb}{1.000000,0.000000,0.000000}%
\pgfsetstrokecolor{currentstroke}%
\pgfsetdash{}{0pt}%
\pgfpathmoveto{\pgfqpoint{2.532393in}{1.917780in}}%
\pgfpathlineto{\pgfqpoint{2.406785in}{1.880161in}}%
\pgfusepath{stroke}%
\end{pgfscope}%
\begin{pgfscope}%
\pgfpathrectangle{\pgfqpoint{0.100000in}{0.212622in}}{\pgfqpoint{3.696000in}{3.696000in}}%
\pgfusepath{clip}%
\pgfsetrectcap%
\pgfsetroundjoin%
\pgfsetlinewidth{1.505625pt}%
\definecolor{currentstroke}{rgb}{1.000000,0.000000,0.000000}%
\pgfsetstrokecolor{currentstroke}%
\pgfsetdash{}{0pt}%
\pgfpathmoveto{\pgfqpoint{2.531276in}{1.917907in}}%
\pgfpathlineto{\pgfqpoint{2.406785in}{1.880161in}}%
\pgfusepath{stroke}%
\end{pgfscope}%
\begin{pgfscope}%
\pgfpathrectangle{\pgfqpoint{0.100000in}{0.212622in}}{\pgfqpoint{3.696000in}{3.696000in}}%
\pgfusepath{clip}%
\pgfsetrectcap%
\pgfsetroundjoin%
\pgfsetlinewidth{1.505625pt}%
\definecolor{currentstroke}{rgb}{1.000000,0.000000,0.000000}%
\pgfsetstrokecolor{currentstroke}%
\pgfsetdash{}{0pt}%
\pgfpathmoveto{\pgfqpoint{2.528395in}{1.918504in}}%
\pgfpathlineto{\pgfqpoint{2.406785in}{1.880161in}}%
\pgfusepath{stroke}%
\end{pgfscope}%
\begin{pgfscope}%
\pgfpathrectangle{\pgfqpoint{0.100000in}{0.212622in}}{\pgfqpoint{3.696000in}{3.696000in}}%
\pgfusepath{clip}%
\pgfsetrectcap%
\pgfsetroundjoin%
\pgfsetlinewidth{1.505625pt}%
\definecolor{currentstroke}{rgb}{1.000000,0.000000,0.000000}%
\pgfsetstrokecolor{currentstroke}%
\pgfsetdash{}{0pt}%
\pgfpathmoveto{\pgfqpoint{2.524743in}{1.919418in}}%
\pgfpathlineto{\pgfqpoint{2.397942in}{1.872102in}}%
\pgfusepath{stroke}%
\end{pgfscope}%
\begin{pgfscope}%
\pgfpathrectangle{\pgfqpoint{0.100000in}{0.212622in}}{\pgfqpoint{3.696000in}{3.696000in}}%
\pgfusepath{clip}%
\pgfsetrectcap%
\pgfsetroundjoin%
\pgfsetlinewidth{1.505625pt}%
\definecolor{currentstroke}{rgb}{1.000000,0.000000,0.000000}%
\pgfsetstrokecolor{currentstroke}%
\pgfsetdash{}{0pt}%
\pgfpathmoveto{\pgfqpoint{2.521745in}{1.919651in}}%
\pgfpathlineto{\pgfqpoint{2.397942in}{1.872102in}}%
\pgfusepath{stroke}%
\end{pgfscope}%
\begin{pgfscope}%
\pgfpathrectangle{\pgfqpoint{0.100000in}{0.212622in}}{\pgfqpoint{3.696000in}{3.696000in}}%
\pgfusepath{clip}%
\pgfsetrectcap%
\pgfsetroundjoin%
\pgfsetlinewidth{1.505625pt}%
\definecolor{currentstroke}{rgb}{1.000000,0.000000,0.000000}%
\pgfsetstrokecolor{currentstroke}%
\pgfsetdash{}{0pt}%
\pgfpathmoveto{\pgfqpoint{2.519816in}{1.919806in}}%
\pgfpathlineto{\pgfqpoint{2.397942in}{1.872102in}}%
\pgfusepath{stroke}%
\end{pgfscope}%
\begin{pgfscope}%
\pgfpathrectangle{\pgfqpoint{0.100000in}{0.212622in}}{\pgfqpoint{3.696000in}{3.696000in}}%
\pgfusepath{clip}%
\pgfsetrectcap%
\pgfsetroundjoin%
\pgfsetlinewidth{1.505625pt}%
\definecolor{currentstroke}{rgb}{1.000000,0.000000,0.000000}%
\pgfsetstrokecolor{currentstroke}%
\pgfsetdash{}{0pt}%
\pgfpathmoveto{\pgfqpoint{2.518222in}{1.920217in}}%
\pgfpathlineto{\pgfqpoint{2.397942in}{1.872102in}}%
\pgfusepath{stroke}%
\end{pgfscope}%
\begin{pgfscope}%
\pgfpathrectangle{\pgfqpoint{0.100000in}{0.212622in}}{\pgfqpoint{3.696000in}{3.696000in}}%
\pgfusepath{clip}%
\pgfsetrectcap%
\pgfsetroundjoin%
\pgfsetlinewidth{1.505625pt}%
\definecolor{currentstroke}{rgb}{1.000000,0.000000,0.000000}%
\pgfsetstrokecolor{currentstroke}%
\pgfsetdash{}{0pt}%
\pgfpathmoveto{\pgfqpoint{2.516001in}{1.920400in}}%
\pgfpathlineto{\pgfqpoint{2.389086in}{1.864032in}}%
\pgfusepath{stroke}%
\end{pgfscope}%
\begin{pgfscope}%
\pgfpathrectangle{\pgfqpoint{0.100000in}{0.212622in}}{\pgfqpoint{3.696000in}{3.696000in}}%
\pgfusepath{clip}%
\pgfsetrectcap%
\pgfsetroundjoin%
\pgfsetlinewidth{1.505625pt}%
\definecolor{currentstroke}{rgb}{1.000000,0.000000,0.000000}%
\pgfsetstrokecolor{currentstroke}%
\pgfsetdash{}{0pt}%
\pgfpathmoveto{\pgfqpoint{2.513945in}{1.920593in}}%
\pgfpathlineto{\pgfqpoint{2.389086in}{1.864032in}}%
\pgfusepath{stroke}%
\end{pgfscope}%
\begin{pgfscope}%
\pgfpathrectangle{\pgfqpoint{0.100000in}{0.212622in}}{\pgfqpoint{3.696000in}{3.696000in}}%
\pgfusepath{clip}%
\pgfsetrectcap%
\pgfsetroundjoin%
\pgfsetlinewidth{1.505625pt}%
\definecolor{currentstroke}{rgb}{1.000000,0.000000,0.000000}%
\pgfsetstrokecolor{currentstroke}%
\pgfsetdash{}{0pt}%
\pgfpathmoveto{\pgfqpoint{2.510054in}{1.921377in}}%
\pgfpathlineto{\pgfqpoint{2.389086in}{1.864032in}}%
\pgfusepath{stroke}%
\end{pgfscope}%
\begin{pgfscope}%
\pgfpathrectangle{\pgfqpoint{0.100000in}{0.212622in}}{\pgfqpoint{3.696000in}{3.696000in}}%
\pgfusepath{clip}%
\pgfsetrectcap%
\pgfsetroundjoin%
\pgfsetlinewidth{1.505625pt}%
\definecolor{currentstroke}{rgb}{1.000000,0.000000,0.000000}%
\pgfsetstrokecolor{currentstroke}%
\pgfsetdash{}{0pt}%
\pgfpathmoveto{\pgfqpoint{2.505011in}{1.922668in}}%
\pgfpathlineto{\pgfqpoint{2.380219in}{1.855951in}}%
\pgfusepath{stroke}%
\end{pgfscope}%
\begin{pgfscope}%
\pgfpathrectangle{\pgfqpoint{0.100000in}{0.212622in}}{\pgfqpoint{3.696000in}{3.696000in}}%
\pgfusepath{clip}%
\pgfsetrectcap%
\pgfsetroundjoin%
\pgfsetlinewidth{1.505625pt}%
\definecolor{currentstroke}{rgb}{1.000000,0.000000,0.000000}%
\pgfsetstrokecolor{currentstroke}%
\pgfsetdash{}{0pt}%
\pgfpathmoveto{\pgfqpoint{2.501266in}{1.923079in}}%
\pgfpathlineto{\pgfqpoint{2.380219in}{1.855951in}}%
\pgfusepath{stroke}%
\end{pgfscope}%
\begin{pgfscope}%
\pgfpathrectangle{\pgfqpoint{0.100000in}{0.212622in}}{\pgfqpoint{3.696000in}{3.696000in}}%
\pgfusepath{clip}%
\pgfsetrectcap%
\pgfsetroundjoin%
\pgfsetlinewidth{1.505625pt}%
\definecolor{currentstroke}{rgb}{1.000000,0.000000,0.000000}%
\pgfsetstrokecolor{currentstroke}%
\pgfsetdash{}{0pt}%
\pgfpathmoveto{\pgfqpoint{2.499188in}{1.923069in}}%
\pgfpathlineto{\pgfqpoint{2.371339in}{1.847859in}}%
\pgfusepath{stroke}%
\end{pgfscope}%
\begin{pgfscope}%
\pgfpathrectangle{\pgfqpoint{0.100000in}{0.212622in}}{\pgfqpoint{3.696000in}{3.696000in}}%
\pgfusepath{clip}%
\pgfsetrectcap%
\pgfsetroundjoin%
\pgfsetlinewidth{1.505625pt}%
\definecolor{currentstroke}{rgb}{1.000000,0.000000,0.000000}%
\pgfsetstrokecolor{currentstroke}%
\pgfsetdash{}{0pt}%
\pgfpathmoveto{\pgfqpoint{2.495660in}{1.923598in}}%
\pgfpathlineto{\pgfqpoint{2.371339in}{1.847859in}}%
\pgfusepath{stroke}%
\end{pgfscope}%
\begin{pgfscope}%
\pgfpathrectangle{\pgfqpoint{0.100000in}{0.212622in}}{\pgfqpoint{3.696000in}{3.696000in}}%
\pgfusepath{clip}%
\pgfsetrectcap%
\pgfsetroundjoin%
\pgfsetlinewidth{1.505625pt}%
\definecolor{currentstroke}{rgb}{1.000000,0.000000,0.000000}%
\pgfsetstrokecolor{currentstroke}%
\pgfsetdash{}{0pt}%
\pgfpathmoveto{\pgfqpoint{2.490510in}{1.924886in}}%
\pgfpathlineto{\pgfqpoint{2.362447in}{1.839755in}}%
\pgfusepath{stroke}%
\end{pgfscope}%
\begin{pgfscope}%
\pgfpathrectangle{\pgfqpoint{0.100000in}{0.212622in}}{\pgfqpoint{3.696000in}{3.696000in}}%
\pgfusepath{clip}%
\pgfsetrectcap%
\pgfsetroundjoin%
\pgfsetlinewidth{1.505625pt}%
\definecolor{currentstroke}{rgb}{1.000000,0.000000,0.000000}%
\pgfsetstrokecolor{currentstroke}%
\pgfsetdash{}{0pt}%
\pgfpathmoveto{\pgfqpoint{2.484815in}{1.925436in}}%
\pgfpathlineto{\pgfqpoint{2.362447in}{1.839755in}}%
\pgfusepath{stroke}%
\end{pgfscope}%
\begin{pgfscope}%
\pgfpathrectangle{\pgfqpoint{0.100000in}{0.212622in}}{\pgfqpoint{3.696000in}{3.696000in}}%
\pgfusepath{clip}%
\pgfsetrectcap%
\pgfsetroundjoin%
\pgfsetlinewidth{1.505625pt}%
\definecolor{currentstroke}{rgb}{1.000000,0.000000,0.000000}%
\pgfsetstrokecolor{currentstroke}%
\pgfsetdash{}{0pt}%
\pgfpathmoveto{\pgfqpoint{2.480812in}{1.925768in}}%
\pgfpathlineto{\pgfqpoint{2.353543in}{1.831641in}}%
\pgfusepath{stroke}%
\end{pgfscope}%
\begin{pgfscope}%
\pgfpathrectangle{\pgfqpoint{0.100000in}{0.212622in}}{\pgfqpoint{3.696000in}{3.696000in}}%
\pgfusepath{clip}%
\pgfsetrectcap%
\pgfsetroundjoin%
\pgfsetlinewidth{1.505625pt}%
\definecolor{currentstroke}{rgb}{1.000000,0.000000,0.000000}%
\pgfsetstrokecolor{currentstroke}%
\pgfsetdash{}{0pt}%
\pgfpathmoveto{\pgfqpoint{2.473212in}{1.927103in}}%
\pgfpathlineto{\pgfqpoint{2.344626in}{1.823515in}}%
\pgfusepath{stroke}%
\end{pgfscope}%
\begin{pgfscope}%
\pgfpathrectangle{\pgfqpoint{0.100000in}{0.212622in}}{\pgfqpoint{3.696000in}{3.696000in}}%
\pgfusepath{clip}%
\pgfsetrectcap%
\pgfsetroundjoin%
\pgfsetlinewidth{1.505625pt}%
\definecolor{currentstroke}{rgb}{1.000000,0.000000,0.000000}%
\pgfsetstrokecolor{currentstroke}%
\pgfsetdash{}{0pt}%
\pgfpathmoveto{\pgfqpoint{2.469275in}{1.927649in}}%
\pgfpathlineto{\pgfqpoint{2.344626in}{1.823515in}}%
\pgfusepath{stroke}%
\end{pgfscope}%
\begin{pgfscope}%
\pgfpathrectangle{\pgfqpoint{0.100000in}{0.212622in}}{\pgfqpoint{3.696000in}{3.696000in}}%
\pgfusepath{clip}%
\pgfsetrectcap%
\pgfsetroundjoin%
\pgfsetlinewidth{1.505625pt}%
\definecolor{currentstroke}{rgb}{1.000000,0.000000,0.000000}%
\pgfsetstrokecolor{currentstroke}%
\pgfsetdash{}{0pt}%
\pgfpathmoveto{\pgfqpoint{2.466198in}{1.927844in}}%
\pgfpathlineto{\pgfqpoint{2.344626in}{1.823515in}}%
\pgfusepath{stroke}%
\end{pgfscope}%
\begin{pgfscope}%
\pgfpathrectangle{\pgfqpoint{0.100000in}{0.212622in}}{\pgfqpoint{3.696000in}{3.696000in}}%
\pgfusepath{clip}%
\pgfsetrectcap%
\pgfsetroundjoin%
\pgfsetlinewidth{1.505625pt}%
\definecolor{currentstroke}{rgb}{1.000000,0.000000,0.000000}%
\pgfsetstrokecolor{currentstroke}%
\pgfsetdash{}{0pt}%
\pgfpathmoveto{\pgfqpoint{2.464280in}{1.928011in}}%
\pgfpathlineto{\pgfqpoint{2.335697in}{1.815378in}}%
\pgfusepath{stroke}%
\end{pgfscope}%
\begin{pgfscope}%
\pgfpathrectangle{\pgfqpoint{0.100000in}{0.212622in}}{\pgfqpoint{3.696000in}{3.696000in}}%
\pgfusepath{clip}%
\pgfsetrectcap%
\pgfsetroundjoin%
\pgfsetlinewidth{1.505625pt}%
\definecolor{currentstroke}{rgb}{1.000000,0.000000,0.000000}%
\pgfsetstrokecolor{currentstroke}%
\pgfsetdash{}{0pt}%
\pgfpathmoveto{\pgfqpoint{2.463180in}{1.928123in}}%
\pgfpathlineto{\pgfqpoint{2.335697in}{1.815378in}}%
\pgfusepath{stroke}%
\end{pgfscope}%
\begin{pgfscope}%
\pgfpathrectangle{\pgfqpoint{0.100000in}{0.212622in}}{\pgfqpoint{3.696000in}{3.696000in}}%
\pgfusepath{clip}%
\pgfsetrectcap%
\pgfsetroundjoin%
\pgfsetlinewidth{1.505625pt}%
\definecolor{currentstroke}{rgb}{1.000000,0.000000,0.000000}%
\pgfsetstrokecolor{currentstroke}%
\pgfsetdash{}{0pt}%
\pgfpathmoveto{\pgfqpoint{2.461909in}{1.928221in}}%
\pgfpathlineto{\pgfqpoint{2.335697in}{1.815378in}}%
\pgfusepath{stroke}%
\end{pgfscope}%
\begin{pgfscope}%
\pgfpathrectangle{\pgfqpoint{0.100000in}{0.212622in}}{\pgfqpoint{3.696000in}{3.696000in}}%
\pgfusepath{clip}%
\pgfsetrectcap%
\pgfsetroundjoin%
\pgfsetlinewidth{1.505625pt}%
\definecolor{currentstroke}{rgb}{1.000000,0.000000,0.000000}%
\pgfsetstrokecolor{currentstroke}%
\pgfsetdash{}{0pt}%
\pgfpathmoveto{\pgfqpoint{2.460245in}{1.928446in}}%
\pgfpathlineto{\pgfqpoint{2.335697in}{1.815378in}}%
\pgfusepath{stroke}%
\end{pgfscope}%
\begin{pgfscope}%
\pgfpathrectangle{\pgfqpoint{0.100000in}{0.212622in}}{\pgfqpoint{3.696000in}{3.696000in}}%
\pgfusepath{clip}%
\pgfsetrectcap%
\pgfsetroundjoin%
\pgfsetlinewidth{1.505625pt}%
\definecolor{currentstroke}{rgb}{1.000000,0.000000,0.000000}%
\pgfsetstrokecolor{currentstroke}%
\pgfsetdash{}{0pt}%
\pgfpathmoveto{\pgfqpoint{2.459099in}{1.928722in}}%
\pgfpathlineto{\pgfqpoint{2.335697in}{1.815378in}}%
\pgfusepath{stroke}%
\end{pgfscope}%
\begin{pgfscope}%
\pgfpathrectangle{\pgfqpoint{0.100000in}{0.212622in}}{\pgfqpoint{3.696000in}{3.696000in}}%
\pgfusepath{clip}%
\pgfsetrectcap%
\pgfsetroundjoin%
\pgfsetlinewidth{1.505625pt}%
\definecolor{currentstroke}{rgb}{1.000000,0.000000,0.000000}%
\pgfsetstrokecolor{currentstroke}%
\pgfsetdash{}{0pt}%
\pgfpathmoveto{\pgfqpoint{2.457078in}{1.928903in}}%
\pgfpathlineto{\pgfqpoint{2.335697in}{1.815378in}}%
\pgfusepath{stroke}%
\end{pgfscope}%
\begin{pgfscope}%
\pgfpathrectangle{\pgfqpoint{0.100000in}{0.212622in}}{\pgfqpoint{3.696000in}{3.696000in}}%
\pgfusepath{clip}%
\pgfsetrectcap%
\pgfsetroundjoin%
\pgfsetlinewidth{1.505625pt}%
\definecolor{currentstroke}{rgb}{1.000000,0.000000,0.000000}%
\pgfsetstrokecolor{currentstroke}%
\pgfsetdash{}{0pt}%
\pgfpathmoveto{\pgfqpoint{2.454315in}{1.929319in}}%
\pgfpathlineto{\pgfqpoint{2.326756in}{1.807230in}}%
\pgfusepath{stroke}%
\end{pgfscope}%
\begin{pgfscope}%
\pgfpathrectangle{\pgfqpoint{0.100000in}{0.212622in}}{\pgfqpoint{3.696000in}{3.696000in}}%
\pgfusepath{clip}%
\pgfsetrectcap%
\pgfsetroundjoin%
\pgfsetlinewidth{1.505625pt}%
\definecolor{currentstroke}{rgb}{1.000000,0.000000,0.000000}%
\pgfsetstrokecolor{currentstroke}%
\pgfsetdash{}{0pt}%
\pgfpathmoveto{\pgfqpoint{2.450771in}{1.930007in}}%
\pgfpathlineto{\pgfqpoint{2.326756in}{1.807230in}}%
\pgfusepath{stroke}%
\end{pgfscope}%
\begin{pgfscope}%
\pgfpathrectangle{\pgfqpoint{0.100000in}{0.212622in}}{\pgfqpoint{3.696000in}{3.696000in}}%
\pgfusepath{clip}%
\pgfsetrectcap%
\pgfsetroundjoin%
\pgfsetlinewidth{1.505625pt}%
\definecolor{currentstroke}{rgb}{1.000000,0.000000,0.000000}%
\pgfsetstrokecolor{currentstroke}%
\pgfsetdash{}{0pt}%
\pgfpathmoveto{\pgfqpoint{2.448054in}{1.930216in}}%
\pgfpathlineto{\pgfqpoint{2.326756in}{1.807230in}}%
\pgfusepath{stroke}%
\end{pgfscope}%
\begin{pgfscope}%
\pgfpathrectangle{\pgfqpoint{0.100000in}{0.212622in}}{\pgfqpoint{3.696000in}{3.696000in}}%
\pgfusepath{clip}%
\pgfsetrectcap%
\pgfsetroundjoin%
\pgfsetlinewidth{1.505625pt}%
\definecolor{currentstroke}{rgb}{1.000000,0.000000,0.000000}%
\pgfsetstrokecolor{currentstroke}%
\pgfsetdash{}{0pt}%
\pgfpathmoveto{\pgfqpoint{2.444803in}{1.930473in}}%
\pgfpathlineto{\pgfqpoint{2.317803in}{1.799071in}}%
\pgfusepath{stroke}%
\end{pgfscope}%
\begin{pgfscope}%
\pgfpathrectangle{\pgfqpoint{0.100000in}{0.212622in}}{\pgfqpoint{3.696000in}{3.696000in}}%
\pgfusepath{clip}%
\pgfsetrectcap%
\pgfsetroundjoin%
\pgfsetlinewidth{1.505625pt}%
\definecolor{currentstroke}{rgb}{1.000000,0.000000,0.000000}%
\pgfsetstrokecolor{currentstroke}%
\pgfsetdash{}{0pt}%
\pgfpathmoveto{\pgfqpoint{2.442872in}{1.930656in}}%
\pgfpathlineto{\pgfqpoint{2.317803in}{1.799071in}}%
\pgfusepath{stroke}%
\end{pgfscope}%
\begin{pgfscope}%
\pgfpathrectangle{\pgfqpoint{0.100000in}{0.212622in}}{\pgfqpoint{3.696000in}{3.696000in}}%
\pgfusepath{clip}%
\pgfsetrectcap%
\pgfsetroundjoin%
\pgfsetlinewidth{1.505625pt}%
\definecolor{currentstroke}{rgb}{1.000000,0.000000,0.000000}%
\pgfsetstrokecolor{currentstroke}%
\pgfsetdash{}{0pt}%
\pgfpathmoveto{\pgfqpoint{2.440812in}{1.930825in}}%
\pgfpathlineto{\pgfqpoint{2.317803in}{1.799071in}}%
\pgfusepath{stroke}%
\end{pgfscope}%
\begin{pgfscope}%
\pgfpathrectangle{\pgfqpoint{0.100000in}{0.212622in}}{\pgfqpoint{3.696000in}{3.696000in}}%
\pgfusepath{clip}%
\pgfsetrectcap%
\pgfsetroundjoin%
\pgfsetlinewidth{1.505625pt}%
\definecolor{currentstroke}{rgb}{1.000000,0.000000,0.000000}%
\pgfsetstrokecolor{currentstroke}%
\pgfsetdash{}{0pt}%
\pgfpathmoveto{\pgfqpoint{2.439650in}{1.930914in}}%
\pgfpathlineto{\pgfqpoint{2.317803in}{1.799071in}}%
\pgfusepath{stroke}%
\end{pgfscope}%
\begin{pgfscope}%
\pgfpathrectangle{\pgfqpoint{0.100000in}{0.212622in}}{\pgfqpoint{3.696000in}{3.696000in}}%
\pgfusepath{clip}%
\pgfsetrectcap%
\pgfsetroundjoin%
\pgfsetlinewidth{1.505625pt}%
\definecolor{currentstroke}{rgb}{1.000000,0.000000,0.000000}%
\pgfsetstrokecolor{currentstroke}%
\pgfsetdash{}{0pt}%
\pgfpathmoveto{\pgfqpoint{2.439010in}{1.930974in}}%
\pgfpathlineto{\pgfqpoint{2.317803in}{1.799071in}}%
\pgfusepath{stroke}%
\end{pgfscope}%
\begin{pgfscope}%
\pgfpathrectangle{\pgfqpoint{0.100000in}{0.212622in}}{\pgfqpoint{3.696000in}{3.696000in}}%
\pgfusepath{clip}%
\pgfsetrectcap%
\pgfsetroundjoin%
\pgfsetlinewidth{1.505625pt}%
\definecolor{currentstroke}{rgb}{1.000000,0.000000,0.000000}%
\pgfsetstrokecolor{currentstroke}%
\pgfsetdash{}{0pt}%
\pgfpathmoveto{\pgfqpoint{2.438660in}{1.931012in}}%
\pgfpathlineto{\pgfqpoint{2.317803in}{1.799071in}}%
\pgfusepath{stroke}%
\end{pgfscope}%
\begin{pgfscope}%
\pgfpathrectangle{\pgfqpoint{0.100000in}{0.212622in}}{\pgfqpoint{3.696000in}{3.696000in}}%
\pgfusepath{clip}%
\pgfsetrectcap%
\pgfsetroundjoin%
\pgfsetlinewidth{1.505625pt}%
\definecolor{currentstroke}{rgb}{1.000000,0.000000,0.000000}%
\pgfsetstrokecolor{currentstroke}%
\pgfsetdash{}{0pt}%
\pgfpathmoveto{\pgfqpoint{2.438469in}{1.931023in}}%
\pgfpathlineto{\pgfqpoint{2.317803in}{1.799071in}}%
\pgfusepath{stroke}%
\end{pgfscope}%
\begin{pgfscope}%
\pgfpathrectangle{\pgfqpoint{0.100000in}{0.212622in}}{\pgfqpoint{3.696000in}{3.696000in}}%
\pgfusepath{clip}%
\pgfsetrectcap%
\pgfsetroundjoin%
\pgfsetlinewidth{1.505625pt}%
\definecolor{currentstroke}{rgb}{1.000000,0.000000,0.000000}%
\pgfsetstrokecolor{currentstroke}%
\pgfsetdash{}{0pt}%
\pgfpathmoveto{\pgfqpoint{2.438346in}{1.931035in}}%
\pgfpathlineto{\pgfqpoint{2.317803in}{1.799071in}}%
\pgfusepath{stroke}%
\end{pgfscope}%
\begin{pgfscope}%
\pgfpathrectangle{\pgfqpoint{0.100000in}{0.212622in}}{\pgfqpoint{3.696000in}{3.696000in}}%
\pgfusepath{clip}%
\pgfsetrectcap%
\pgfsetroundjoin%
\pgfsetlinewidth{1.505625pt}%
\definecolor{currentstroke}{rgb}{1.000000,0.000000,0.000000}%
\pgfsetstrokecolor{currentstroke}%
\pgfsetdash{}{0pt}%
\pgfpathmoveto{\pgfqpoint{2.437909in}{1.931087in}}%
\pgfpathlineto{\pgfqpoint{2.317803in}{1.799071in}}%
\pgfusepath{stroke}%
\end{pgfscope}%
\begin{pgfscope}%
\pgfpathrectangle{\pgfqpoint{0.100000in}{0.212622in}}{\pgfqpoint{3.696000in}{3.696000in}}%
\pgfusepath{clip}%
\pgfsetrectcap%
\pgfsetroundjoin%
\pgfsetlinewidth{1.505625pt}%
\definecolor{currentstroke}{rgb}{1.000000,0.000000,0.000000}%
\pgfsetstrokecolor{currentstroke}%
\pgfsetdash{}{0pt}%
\pgfpathmoveto{\pgfqpoint{2.436964in}{1.931238in}}%
\pgfpathlineto{\pgfqpoint{2.308837in}{1.790901in}}%
\pgfusepath{stroke}%
\end{pgfscope}%
\begin{pgfscope}%
\pgfpathrectangle{\pgfqpoint{0.100000in}{0.212622in}}{\pgfqpoint{3.696000in}{3.696000in}}%
\pgfusepath{clip}%
\pgfsetrectcap%
\pgfsetroundjoin%
\pgfsetlinewidth{1.505625pt}%
\definecolor{currentstroke}{rgb}{1.000000,0.000000,0.000000}%
\pgfsetstrokecolor{currentstroke}%
\pgfsetdash{}{0pt}%
\pgfpathmoveto{\pgfqpoint{2.434947in}{1.931727in}}%
\pgfpathlineto{\pgfqpoint{2.308837in}{1.790901in}}%
\pgfusepath{stroke}%
\end{pgfscope}%
\begin{pgfscope}%
\pgfpathrectangle{\pgfqpoint{0.100000in}{0.212622in}}{\pgfqpoint{3.696000in}{3.696000in}}%
\pgfusepath{clip}%
\pgfsetrectcap%
\pgfsetroundjoin%
\pgfsetlinewidth{1.505625pt}%
\definecolor{currentstroke}{rgb}{1.000000,0.000000,0.000000}%
\pgfsetstrokecolor{currentstroke}%
\pgfsetdash{}{0pt}%
\pgfpathmoveto{\pgfqpoint{2.431677in}{1.932474in}}%
\pgfpathlineto{\pgfqpoint{2.308837in}{1.790901in}}%
\pgfusepath{stroke}%
\end{pgfscope}%
\begin{pgfscope}%
\pgfpathrectangle{\pgfqpoint{0.100000in}{0.212622in}}{\pgfqpoint{3.696000in}{3.696000in}}%
\pgfusepath{clip}%
\pgfsetrectcap%
\pgfsetroundjoin%
\pgfsetlinewidth{1.505625pt}%
\definecolor{currentstroke}{rgb}{1.000000,0.000000,0.000000}%
\pgfsetstrokecolor{currentstroke}%
\pgfsetdash{}{0pt}%
\pgfpathmoveto{\pgfqpoint{2.426943in}{1.933627in}}%
\pgfpathlineto{\pgfqpoint{2.308837in}{1.790901in}}%
\pgfusepath{stroke}%
\end{pgfscope}%
\begin{pgfscope}%
\pgfpathrectangle{\pgfqpoint{0.100000in}{0.212622in}}{\pgfqpoint{3.696000in}{3.696000in}}%
\pgfusepath{clip}%
\pgfsetrectcap%
\pgfsetroundjoin%
\pgfsetlinewidth{1.505625pt}%
\definecolor{currentstroke}{rgb}{1.000000,0.000000,0.000000}%
\pgfsetstrokecolor{currentstroke}%
\pgfsetdash{}{0pt}%
\pgfpathmoveto{\pgfqpoint{2.424420in}{1.934274in}}%
\pgfpathlineto{\pgfqpoint{2.308837in}{1.790901in}}%
\pgfusepath{stroke}%
\end{pgfscope}%
\begin{pgfscope}%
\pgfpathrectangle{\pgfqpoint{0.100000in}{0.212622in}}{\pgfqpoint{3.696000in}{3.696000in}}%
\pgfusepath{clip}%
\pgfsetrectcap%
\pgfsetroundjoin%
\pgfsetlinewidth{1.505625pt}%
\definecolor{currentstroke}{rgb}{1.000000,0.000000,0.000000}%
\pgfsetstrokecolor{currentstroke}%
\pgfsetdash{}{0pt}%
\pgfpathmoveto{\pgfqpoint{2.421670in}{1.935276in}}%
\pgfpathlineto{\pgfqpoint{2.308837in}{1.790901in}}%
\pgfusepath{stroke}%
\end{pgfscope}%
\begin{pgfscope}%
\pgfpathrectangle{\pgfqpoint{0.100000in}{0.212622in}}{\pgfqpoint{3.696000in}{3.696000in}}%
\pgfusepath{clip}%
\pgfsetrectcap%
\pgfsetroundjoin%
\pgfsetlinewidth{1.505625pt}%
\definecolor{currentstroke}{rgb}{1.000000,0.000000,0.000000}%
\pgfsetstrokecolor{currentstroke}%
\pgfsetdash{}{0pt}%
\pgfpathmoveto{\pgfqpoint{2.420386in}{1.935978in}}%
\pgfpathlineto{\pgfqpoint{2.308837in}{1.790901in}}%
\pgfusepath{stroke}%
\end{pgfscope}%
\begin{pgfscope}%
\pgfpathrectangle{\pgfqpoint{0.100000in}{0.212622in}}{\pgfqpoint{3.696000in}{3.696000in}}%
\pgfusepath{clip}%
\pgfsetrectcap%
\pgfsetroundjoin%
\pgfsetlinewidth{1.505625pt}%
\definecolor{currentstroke}{rgb}{1.000000,0.000000,0.000000}%
\pgfsetstrokecolor{currentstroke}%
\pgfsetdash{}{0pt}%
\pgfpathmoveto{\pgfqpoint{2.418645in}{1.936484in}}%
\pgfpathlineto{\pgfqpoint{2.308837in}{1.790901in}}%
\pgfusepath{stroke}%
\end{pgfscope}%
\begin{pgfscope}%
\pgfpathrectangle{\pgfqpoint{0.100000in}{0.212622in}}{\pgfqpoint{3.696000in}{3.696000in}}%
\pgfusepath{clip}%
\pgfsetrectcap%
\pgfsetroundjoin%
\pgfsetlinewidth{1.505625pt}%
\definecolor{currentstroke}{rgb}{1.000000,0.000000,0.000000}%
\pgfsetstrokecolor{currentstroke}%
\pgfsetdash{}{0pt}%
\pgfpathmoveto{\pgfqpoint{2.416539in}{1.937061in}}%
\pgfpathlineto{\pgfqpoint{2.308837in}{1.790901in}}%
\pgfusepath{stroke}%
\end{pgfscope}%
\begin{pgfscope}%
\pgfpathrectangle{\pgfqpoint{0.100000in}{0.212622in}}{\pgfqpoint{3.696000in}{3.696000in}}%
\pgfusepath{clip}%
\pgfsetrectcap%
\pgfsetroundjoin%
\pgfsetlinewidth{1.505625pt}%
\definecolor{currentstroke}{rgb}{1.000000,0.000000,0.000000}%
\pgfsetstrokecolor{currentstroke}%
\pgfsetdash{}{0pt}%
\pgfpathmoveto{\pgfqpoint{2.415332in}{1.937481in}}%
\pgfpathlineto{\pgfqpoint{2.308837in}{1.790901in}}%
\pgfusepath{stroke}%
\end{pgfscope}%
\begin{pgfscope}%
\pgfpathrectangle{\pgfqpoint{0.100000in}{0.212622in}}{\pgfqpoint{3.696000in}{3.696000in}}%
\pgfusepath{clip}%
\pgfsetrectcap%
\pgfsetroundjoin%
\pgfsetlinewidth{1.505625pt}%
\definecolor{currentstroke}{rgb}{1.000000,0.000000,0.000000}%
\pgfsetstrokecolor{currentstroke}%
\pgfsetdash{}{0pt}%
\pgfpathmoveto{\pgfqpoint{2.414730in}{1.937634in}}%
\pgfpathlineto{\pgfqpoint{2.308837in}{1.790901in}}%
\pgfusepath{stroke}%
\end{pgfscope}%
\begin{pgfscope}%
\pgfpathrectangle{\pgfqpoint{0.100000in}{0.212622in}}{\pgfqpoint{3.696000in}{3.696000in}}%
\pgfusepath{clip}%
\pgfsetrectcap%
\pgfsetroundjoin%
\pgfsetlinewidth{1.505625pt}%
\definecolor{currentstroke}{rgb}{1.000000,0.000000,0.000000}%
\pgfsetstrokecolor{currentstroke}%
\pgfsetdash{}{0pt}%
\pgfpathmoveto{\pgfqpoint{2.414392in}{1.937764in}}%
\pgfpathlineto{\pgfqpoint{2.308837in}{1.790901in}}%
\pgfusepath{stroke}%
\end{pgfscope}%
\begin{pgfscope}%
\pgfpathrectangle{\pgfqpoint{0.100000in}{0.212622in}}{\pgfqpoint{3.696000in}{3.696000in}}%
\pgfusepath{clip}%
\pgfsetrectcap%
\pgfsetroundjoin%
\pgfsetlinewidth{1.505625pt}%
\definecolor{currentstroke}{rgb}{1.000000,0.000000,0.000000}%
\pgfsetstrokecolor{currentstroke}%
\pgfsetdash{}{0pt}%
\pgfpathmoveto{\pgfqpoint{2.414214in}{1.937837in}}%
\pgfpathlineto{\pgfqpoint{2.308837in}{1.790901in}}%
\pgfusepath{stroke}%
\end{pgfscope}%
\begin{pgfscope}%
\pgfpathrectangle{\pgfqpoint{0.100000in}{0.212622in}}{\pgfqpoint{3.696000in}{3.696000in}}%
\pgfusepath{clip}%
\pgfsetrectcap%
\pgfsetroundjoin%
\pgfsetlinewidth{1.505625pt}%
\definecolor{currentstroke}{rgb}{1.000000,0.000000,0.000000}%
\pgfsetstrokecolor{currentstroke}%
\pgfsetdash{}{0pt}%
\pgfpathmoveto{\pgfqpoint{2.414117in}{1.937856in}}%
\pgfpathlineto{\pgfqpoint{2.308837in}{1.790901in}}%
\pgfusepath{stroke}%
\end{pgfscope}%
\begin{pgfscope}%
\pgfpathrectangle{\pgfqpoint{0.100000in}{0.212622in}}{\pgfqpoint{3.696000in}{3.696000in}}%
\pgfusepath{clip}%
\pgfsetrectcap%
\pgfsetroundjoin%
\pgfsetlinewidth{1.505625pt}%
\definecolor{currentstroke}{rgb}{1.000000,0.000000,0.000000}%
\pgfsetstrokecolor{currentstroke}%
\pgfsetdash{}{0pt}%
\pgfpathmoveto{\pgfqpoint{2.413691in}{1.937949in}}%
\pgfpathlineto{\pgfqpoint{2.308837in}{1.790901in}}%
\pgfusepath{stroke}%
\end{pgfscope}%
\begin{pgfscope}%
\pgfpathrectangle{\pgfqpoint{0.100000in}{0.212622in}}{\pgfqpoint{3.696000in}{3.696000in}}%
\pgfusepath{clip}%
\pgfsetrectcap%
\pgfsetroundjoin%
\pgfsetlinewidth{1.505625pt}%
\definecolor{currentstroke}{rgb}{1.000000,0.000000,0.000000}%
\pgfsetstrokecolor{currentstroke}%
\pgfsetdash{}{0pt}%
\pgfpathmoveto{\pgfqpoint{2.413427in}{1.937963in}}%
\pgfpathlineto{\pgfqpoint{2.308837in}{1.790901in}}%
\pgfusepath{stroke}%
\end{pgfscope}%
\begin{pgfscope}%
\pgfpathrectangle{\pgfqpoint{0.100000in}{0.212622in}}{\pgfqpoint{3.696000in}{3.696000in}}%
\pgfusepath{clip}%
\pgfsetrectcap%
\pgfsetroundjoin%
\pgfsetlinewidth{1.505625pt}%
\definecolor{currentstroke}{rgb}{1.000000,0.000000,0.000000}%
\pgfsetstrokecolor{currentstroke}%
\pgfsetdash{}{0pt}%
\pgfpathmoveto{\pgfqpoint{2.412207in}{1.938091in}}%
\pgfpathlineto{\pgfqpoint{2.308837in}{1.790901in}}%
\pgfusepath{stroke}%
\end{pgfscope}%
\begin{pgfscope}%
\pgfpathrectangle{\pgfqpoint{0.100000in}{0.212622in}}{\pgfqpoint{3.696000in}{3.696000in}}%
\pgfusepath{clip}%
\pgfsetrectcap%
\pgfsetroundjoin%
\pgfsetlinewidth{1.505625pt}%
\definecolor{currentstroke}{rgb}{1.000000,0.000000,0.000000}%
\pgfsetstrokecolor{currentstroke}%
\pgfsetdash{}{0pt}%
\pgfpathmoveto{\pgfqpoint{2.409450in}{1.939445in}}%
\pgfpathlineto{\pgfqpoint{2.308837in}{1.790901in}}%
\pgfusepath{stroke}%
\end{pgfscope}%
\begin{pgfscope}%
\pgfpathrectangle{\pgfqpoint{0.100000in}{0.212622in}}{\pgfqpoint{3.696000in}{3.696000in}}%
\pgfusepath{clip}%
\pgfsetrectcap%
\pgfsetroundjoin%
\pgfsetlinewidth{1.505625pt}%
\definecolor{currentstroke}{rgb}{1.000000,0.000000,0.000000}%
\pgfsetstrokecolor{currentstroke}%
\pgfsetdash{}{0pt}%
\pgfpathmoveto{\pgfqpoint{2.404719in}{1.941387in}}%
\pgfpathlineto{\pgfqpoint{2.308837in}{1.790901in}}%
\pgfusepath{stroke}%
\end{pgfscope}%
\begin{pgfscope}%
\pgfpathrectangle{\pgfqpoint{0.100000in}{0.212622in}}{\pgfqpoint{3.696000in}{3.696000in}}%
\pgfusepath{clip}%
\pgfsetrectcap%
\pgfsetroundjoin%
\pgfsetlinewidth{1.505625pt}%
\definecolor{currentstroke}{rgb}{1.000000,0.000000,0.000000}%
\pgfsetstrokecolor{currentstroke}%
\pgfsetdash{}{0pt}%
\pgfpathmoveto{\pgfqpoint{2.399186in}{1.943155in}}%
\pgfpathlineto{\pgfqpoint{2.308837in}{1.790901in}}%
\pgfusepath{stroke}%
\end{pgfscope}%
\begin{pgfscope}%
\pgfpathrectangle{\pgfqpoint{0.100000in}{0.212622in}}{\pgfqpoint{3.696000in}{3.696000in}}%
\pgfusepath{clip}%
\pgfsetrectcap%
\pgfsetroundjoin%
\pgfsetlinewidth{1.505625pt}%
\definecolor{currentstroke}{rgb}{1.000000,0.000000,0.000000}%
\pgfsetstrokecolor{currentstroke}%
\pgfsetdash{}{0pt}%
\pgfpathmoveto{\pgfqpoint{2.392994in}{1.944161in}}%
\pgfpathlineto{\pgfqpoint{2.308837in}{1.790901in}}%
\pgfusepath{stroke}%
\end{pgfscope}%
\begin{pgfscope}%
\pgfpathrectangle{\pgfqpoint{0.100000in}{0.212622in}}{\pgfqpoint{3.696000in}{3.696000in}}%
\pgfusepath{clip}%
\pgfsetrectcap%
\pgfsetroundjoin%
\pgfsetlinewidth{1.505625pt}%
\definecolor{currentstroke}{rgb}{1.000000,0.000000,0.000000}%
\pgfsetstrokecolor{currentstroke}%
\pgfsetdash{}{0pt}%
\pgfpathmoveto{\pgfqpoint{2.386668in}{1.945345in}}%
\pgfpathlineto{\pgfqpoint{2.308837in}{1.790901in}}%
\pgfusepath{stroke}%
\end{pgfscope}%
\begin{pgfscope}%
\pgfpathrectangle{\pgfqpoint{0.100000in}{0.212622in}}{\pgfqpoint{3.696000in}{3.696000in}}%
\pgfusepath{clip}%
\pgfsetrectcap%
\pgfsetroundjoin%
\pgfsetlinewidth{1.505625pt}%
\definecolor{currentstroke}{rgb}{1.000000,0.000000,0.000000}%
\pgfsetstrokecolor{currentstroke}%
\pgfsetdash{}{0pt}%
\pgfpathmoveto{\pgfqpoint{2.377091in}{1.949837in}}%
\pgfpathlineto{\pgfqpoint{2.308837in}{1.790901in}}%
\pgfusepath{stroke}%
\end{pgfscope}%
\begin{pgfscope}%
\pgfpathrectangle{\pgfqpoint{0.100000in}{0.212622in}}{\pgfqpoint{3.696000in}{3.696000in}}%
\pgfusepath{clip}%
\pgfsetrectcap%
\pgfsetroundjoin%
\pgfsetlinewidth{1.505625pt}%
\definecolor{currentstroke}{rgb}{1.000000,0.000000,0.000000}%
\pgfsetstrokecolor{currentstroke}%
\pgfsetdash{}{0pt}%
\pgfpathmoveto{\pgfqpoint{2.365506in}{1.953436in}}%
\pgfpathlineto{\pgfqpoint{2.308837in}{1.790901in}}%
\pgfusepath{stroke}%
\end{pgfscope}%
\begin{pgfscope}%
\pgfpathrectangle{\pgfqpoint{0.100000in}{0.212622in}}{\pgfqpoint{3.696000in}{3.696000in}}%
\pgfusepath{clip}%
\pgfsetrectcap%
\pgfsetroundjoin%
\pgfsetlinewidth{1.505625pt}%
\definecolor{currentstroke}{rgb}{1.000000,0.000000,0.000000}%
\pgfsetstrokecolor{currentstroke}%
\pgfsetdash{}{0pt}%
\pgfpathmoveto{\pgfqpoint{2.365795in}{1.963311in}}%
\pgfpathlineto{\pgfqpoint{2.317803in}{1.799071in}}%
\pgfusepath{stroke}%
\end{pgfscope}%
\begin{pgfscope}%
\pgfpathrectangle{\pgfqpoint{0.100000in}{0.212622in}}{\pgfqpoint{3.696000in}{3.696000in}}%
\pgfusepath{clip}%
\pgfsetrectcap%
\pgfsetroundjoin%
\pgfsetlinewidth{1.505625pt}%
\definecolor{currentstroke}{rgb}{1.000000,0.000000,0.000000}%
\pgfsetstrokecolor{currentstroke}%
\pgfsetdash{}{0pt}%
\pgfpathmoveto{\pgfqpoint{2.360279in}{1.966276in}}%
\pgfpathlineto{\pgfqpoint{2.317803in}{1.799071in}}%
\pgfusepath{stroke}%
\end{pgfscope}%
\begin{pgfscope}%
\pgfpathrectangle{\pgfqpoint{0.100000in}{0.212622in}}{\pgfqpoint{3.696000in}{3.696000in}}%
\pgfusepath{clip}%
\pgfsetrectcap%
\pgfsetroundjoin%
\pgfsetlinewidth{1.505625pt}%
\definecolor{currentstroke}{rgb}{1.000000,0.000000,0.000000}%
\pgfsetstrokecolor{currentstroke}%
\pgfsetdash{}{0pt}%
\pgfpathmoveto{\pgfqpoint{2.352984in}{1.968026in}}%
\pgfpathlineto{\pgfqpoint{2.317803in}{1.799071in}}%
\pgfusepath{stroke}%
\end{pgfscope}%
\begin{pgfscope}%
\pgfpathrectangle{\pgfqpoint{0.100000in}{0.212622in}}{\pgfqpoint{3.696000in}{3.696000in}}%
\pgfusepath{clip}%
\pgfsetrectcap%
\pgfsetroundjoin%
\pgfsetlinewidth{1.505625pt}%
\definecolor{currentstroke}{rgb}{1.000000,0.000000,0.000000}%
\pgfsetstrokecolor{currentstroke}%
\pgfsetdash{}{0pt}%
\pgfpathmoveto{\pgfqpoint{2.344505in}{1.970979in}}%
\pgfpathlineto{\pgfqpoint{2.317803in}{1.799071in}}%
\pgfusepath{stroke}%
\end{pgfscope}%
\begin{pgfscope}%
\pgfpathrectangle{\pgfqpoint{0.100000in}{0.212622in}}{\pgfqpoint{3.696000in}{3.696000in}}%
\pgfusepath{clip}%
\pgfsetrectcap%
\pgfsetroundjoin%
\pgfsetlinewidth{1.505625pt}%
\definecolor{currentstroke}{rgb}{1.000000,0.000000,0.000000}%
\pgfsetstrokecolor{currentstroke}%
\pgfsetdash{}{0pt}%
\pgfpathmoveto{\pgfqpoint{2.339702in}{1.971860in}}%
\pgfpathlineto{\pgfqpoint{2.317803in}{1.799071in}}%
\pgfusepath{stroke}%
\end{pgfscope}%
\begin{pgfscope}%
\pgfpathrectangle{\pgfqpoint{0.100000in}{0.212622in}}{\pgfqpoint{3.696000in}{3.696000in}}%
\pgfusepath{clip}%
\pgfsetrectcap%
\pgfsetroundjoin%
\pgfsetlinewidth{1.505625pt}%
\definecolor{currentstroke}{rgb}{1.000000,0.000000,0.000000}%
\pgfsetstrokecolor{currentstroke}%
\pgfsetdash{}{0pt}%
\pgfpathmoveto{\pgfqpoint{2.337109in}{1.972578in}}%
\pgfpathlineto{\pgfqpoint{2.317803in}{1.799071in}}%
\pgfusepath{stroke}%
\end{pgfscope}%
\begin{pgfscope}%
\pgfpathrectangle{\pgfqpoint{0.100000in}{0.212622in}}{\pgfqpoint{3.696000in}{3.696000in}}%
\pgfusepath{clip}%
\pgfsetrectcap%
\pgfsetroundjoin%
\pgfsetlinewidth{1.505625pt}%
\definecolor{currentstroke}{rgb}{1.000000,0.000000,0.000000}%
\pgfsetstrokecolor{currentstroke}%
\pgfsetdash{}{0pt}%
\pgfpathmoveto{\pgfqpoint{2.332987in}{1.974091in}}%
\pgfpathlineto{\pgfqpoint{2.317803in}{1.799071in}}%
\pgfusepath{stroke}%
\end{pgfscope}%
\begin{pgfscope}%
\pgfpathrectangle{\pgfqpoint{0.100000in}{0.212622in}}{\pgfqpoint{3.696000in}{3.696000in}}%
\pgfusepath{clip}%
\pgfsetrectcap%
\pgfsetroundjoin%
\pgfsetlinewidth{1.505625pt}%
\definecolor{currentstroke}{rgb}{1.000000,0.000000,0.000000}%
\pgfsetstrokecolor{currentstroke}%
\pgfsetdash{}{0pt}%
\pgfpathmoveto{\pgfqpoint{2.328605in}{1.975419in}}%
\pgfpathlineto{\pgfqpoint{2.317803in}{1.799071in}}%
\pgfusepath{stroke}%
\end{pgfscope}%
\begin{pgfscope}%
\pgfpathrectangle{\pgfqpoint{0.100000in}{0.212622in}}{\pgfqpoint{3.696000in}{3.696000in}}%
\pgfusepath{clip}%
\pgfsetrectcap%
\pgfsetroundjoin%
\pgfsetlinewidth{1.505625pt}%
\definecolor{currentstroke}{rgb}{1.000000,0.000000,0.000000}%
\pgfsetstrokecolor{currentstroke}%
\pgfsetdash{}{0pt}%
\pgfpathmoveto{\pgfqpoint{2.323757in}{1.976472in}}%
\pgfpathlineto{\pgfqpoint{2.317803in}{1.799071in}}%
\pgfusepath{stroke}%
\end{pgfscope}%
\begin{pgfscope}%
\pgfpathrectangle{\pgfqpoint{0.100000in}{0.212622in}}{\pgfqpoint{3.696000in}{3.696000in}}%
\pgfusepath{clip}%
\pgfsetrectcap%
\pgfsetroundjoin%
\pgfsetlinewidth{1.505625pt}%
\definecolor{currentstroke}{rgb}{1.000000,0.000000,0.000000}%
\pgfsetstrokecolor{currentstroke}%
\pgfsetdash{}{0pt}%
\pgfpathmoveto{\pgfqpoint{2.315767in}{1.979691in}}%
\pgfpathlineto{\pgfqpoint{2.317803in}{1.799071in}}%
\pgfusepath{stroke}%
\end{pgfscope}%
\begin{pgfscope}%
\pgfpathrectangle{\pgfqpoint{0.100000in}{0.212622in}}{\pgfqpoint{3.696000in}{3.696000in}}%
\pgfusepath{clip}%
\pgfsetrectcap%
\pgfsetroundjoin%
\pgfsetlinewidth{1.505625pt}%
\definecolor{currentstroke}{rgb}{1.000000,0.000000,0.000000}%
\pgfsetstrokecolor{currentstroke}%
\pgfsetdash{}{0pt}%
\pgfpathmoveto{\pgfqpoint{2.306893in}{1.982307in}}%
\pgfpathlineto{\pgfqpoint{2.317803in}{1.799071in}}%
\pgfusepath{stroke}%
\end{pgfscope}%
\begin{pgfscope}%
\pgfpathrectangle{\pgfqpoint{0.100000in}{0.212622in}}{\pgfqpoint{3.696000in}{3.696000in}}%
\pgfusepath{clip}%
\pgfsetrectcap%
\pgfsetroundjoin%
\pgfsetlinewidth{1.505625pt}%
\definecolor{currentstroke}{rgb}{1.000000,0.000000,0.000000}%
\pgfsetstrokecolor{currentstroke}%
\pgfsetdash{}{0pt}%
\pgfpathmoveto{\pgfqpoint{2.297696in}{1.984475in}}%
\pgfpathlineto{\pgfqpoint{2.308837in}{1.790901in}}%
\pgfusepath{stroke}%
\end{pgfscope}%
\begin{pgfscope}%
\pgfpathrectangle{\pgfqpoint{0.100000in}{0.212622in}}{\pgfqpoint{3.696000in}{3.696000in}}%
\pgfusepath{clip}%
\pgfsetrectcap%
\pgfsetroundjoin%
\pgfsetlinewidth{1.505625pt}%
\definecolor{currentstroke}{rgb}{1.000000,0.000000,0.000000}%
\pgfsetstrokecolor{currentstroke}%
\pgfsetdash{}{0pt}%
\pgfpathmoveto{\pgfqpoint{2.285882in}{1.991589in}}%
\pgfpathlineto{\pgfqpoint{2.308837in}{1.790901in}}%
\pgfusepath{stroke}%
\end{pgfscope}%
\begin{pgfscope}%
\pgfpathrectangle{\pgfqpoint{0.100000in}{0.212622in}}{\pgfqpoint{3.696000in}{3.696000in}}%
\pgfusepath{clip}%
\pgfsetrectcap%
\pgfsetroundjoin%
\pgfsetlinewidth{1.505625pt}%
\definecolor{currentstroke}{rgb}{1.000000,0.000000,0.000000}%
\pgfsetstrokecolor{currentstroke}%
\pgfsetdash{}{0pt}%
\pgfpathmoveto{\pgfqpoint{2.271590in}{1.995700in}}%
\pgfpathlineto{\pgfqpoint{2.308837in}{1.790901in}}%
\pgfusepath{stroke}%
\end{pgfscope}%
\begin{pgfscope}%
\pgfpathrectangle{\pgfqpoint{0.100000in}{0.212622in}}{\pgfqpoint{3.696000in}{3.696000in}}%
\pgfusepath{clip}%
\pgfsetrectcap%
\pgfsetroundjoin%
\pgfsetlinewidth{1.505625pt}%
\definecolor{currentstroke}{rgb}{1.000000,0.000000,0.000000}%
\pgfsetstrokecolor{currentstroke}%
\pgfsetdash{}{0pt}%
\pgfpathmoveto{\pgfqpoint{2.263413in}{1.997578in}}%
\pgfpathlineto{\pgfqpoint{2.308837in}{1.790901in}}%
\pgfusepath{stroke}%
\end{pgfscope}%
\begin{pgfscope}%
\pgfpathrectangle{\pgfqpoint{0.100000in}{0.212622in}}{\pgfqpoint{3.696000in}{3.696000in}}%
\pgfusepath{clip}%
\pgfsetrectcap%
\pgfsetroundjoin%
\pgfsetlinewidth{1.505625pt}%
\definecolor{currentstroke}{rgb}{1.000000,0.000000,0.000000}%
\pgfsetstrokecolor{currentstroke}%
\pgfsetdash{}{0pt}%
\pgfpathmoveto{\pgfqpoint{2.255176in}{2.004079in}}%
\pgfpathlineto{\pgfqpoint{2.308837in}{1.790901in}}%
\pgfusepath{stroke}%
\end{pgfscope}%
\begin{pgfscope}%
\pgfpathrectangle{\pgfqpoint{0.100000in}{0.212622in}}{\pgfqpoint{3.696000in}{3.696000in}}%
\pgfusepath{clip}%
\pgfsetrectcap%
\pgfsetroundjoin%
\pgfsetlinewidth{1.505625pt}%
\definecolor{currentstroke}{rgb}{1.000000,0.000000,0.000000}%
\pgfsetstrokecolor{currentstroke}%
\pgfsetdash{}{0pt}%
\pgfpathmoveto{\pgfqpoint{2.242841in}{2.007838in}}%
\pgfpathlineto{\pgfqpoint{2.209437in}{1.776842in}}%
\pgfusepath{stroke}%
\end{pgfscope}%
\begin{pgfscope}%
\pgfpathrectangle{\pgfqpoint{0.100000in}{0.212622in}}{\pgfqpoint{3.696000in}{3.696000in}}%
\pgfusepath{clip}%
\pgfsetrectcap%
\pgfsetroundjoin%
\pgfsetlinewidth{1.505625pt}%
\definecolor{currentstroke}{rgb}{1.000000,0.000000,0.000000}%
\pgfsetstrokecolor{currentstroke}%
\pgfsetdash{}{0pt}%
\pgfpathmoveto{\pgfqpoint{2.229616in}{2.012265in}}%
\pgfpathlineto{\pgfqpoint{2.193615in}{1.781517in}}%
\pgfusepath{stroke}%
\end{pgfscope}%
\begin{pgfscope}%
\pgfpathrectangle{\pgfqpoint{0.100000in}{0.212622in}}{\pgfqpoint{3.696000in}{3.696000in}}%
\pgfusepath{clip}%
\pgfsetrectcap%
\pgfsetroundjoin%
\pgfsetlinewidth{1.505625pt}%
\definecolor{currentstroke}{rgb}{1.000000,0.000000,0.000000}%
\pgfsetstrokecolor{currentstroke}%
\pgfsetdash{}{0pt}%
\pgfpathmoveto{\pgfqpoint{2.222606in}{2.013474in}}%
\pgfpathlineto{\pgfqpoint{2.177805in}{1.786188in}}%
\pgfusepath{stroke}%
\end{pgfscope}%
\begin{pgfscope}%
\pgfpathrectangle{\pgfqpoint{0.100000in}{0.212622in}}{\pgfqpoint{3.696000in}{3.696000in}}%
\pgfusepath{clip}%
\pgfsetrectcap%
\pgfsetroundjoin%
\pgfsetlinewidth{1.505625pt}%
\definecolor{currentstroke}{rgb}{1.000000,0.000000,0.000000}%
\pgfsetstrokecolor{currentstroke}%
\pgfsetdash{}{0pt}%
\pgfpathmoveto{\pgfqpoint{2.215018in}{2.015290in}}%
\pgfpathlineto{\pgfqpoint{2.177805in}{1.786188in}}%
\pgfusepath{stroke}%
\end{pgfscope}%
\begin{pgfscope}%
\pgfpathrectangle{\pgfqpoint{0.100000in}{0.212622in}}{\pgfqpoint{3.696000in}{3.696000in}}%
\pgfusepath{clip}%
\pgfsetrectcap%
\pgfsetroundjoin%
\pgfsetlinewidth{1.505625pt}%
\definecolor{currentstroke}{rgb}{1.000000,0.000000,0.000000}%
\pgfsetstrokecolor{currentstroke}%
\pgfsetdash{}{0pt}%
\pgfpathmoveto{\pgfqpoint{2.205305in}{2.018821in}}%
\pgfpathlineto{\pgfqpoint{2.162008in}{1.790856in}}%
\pgfusepath{stroke}%
\end{pgfscope}%
\begin{pgfscope}%
\pgfpathrectangle{\pgfqpoint{0.100000in}{0.212622in}}{\pgfqpoint{3.696000in}{3.696000in}}%
\pgfusepath{clip}%
\pgfsetrectcap%
\pgfsetroundjoin%
\pgfsetlinewidth{1.505625pt}%
\definecolor{currentstroke}{rgb}{1.000000,0.000000,0.000000}%
\pgfsetstrokecolor{currentstroke}%
\pgfsetdash{}{0pt}%
\pgfpathmoveto{\pgfqpoint{2.195510in}{2.022128in}}%
\pgfpathlineto{\pgfqpoint{2.162008in}{1.790856in}}%
\pgfusepath{stroke}%
\end{pgfscope}%
\begin{pgfscope}%
\pgfpathrectangle{\pgfqpoint{0.100000in}{0.212622in}}{\pgfqpoint{3.696000in}{3.696000in}}%
\pgfusepath{clip}%
\pgfsetrectcap%
\pgfsetroundjoin%
\pgfsetlinewidth{1.505625pt}%
\definecolor{currentstroke}{rgb}{1.000000,0.000000,0.000000}%
\pgfsetstrokecolor{currentstroke}%
\pgfsetdash{}{0pt}%
\pgfpathmoveto{\pgfqpoint{2.190279in}{2.023065in}}%
\pgfpathlineto{\pgfqpoint{2.146223in}{1.795520in}}%
\pgfusepath{stroke}%
\end{pgfscope}%
\begin{pgfscope}%
\pgfpathrectangle{\pgfqpoint{0.100000in}{0.212622in}}{\pgfqpoint{3.696000in}{3.696000in}}%
\pgfusepath{clip}%
\pgfsetrectcap%
\pgfsetroundjoin%
\pgfsetlinewidth{1.505625pt}%
\definecolor{currentstroke}{rgb}{1.000000,0.000000,0.000000}%
\pgfsetstrokecolor{currentstroke}%
\pgfsetdash{}{0pt}%
\pgfpathmoveto{\pgfqpoint{2.182057in}{2.026684in}}%
\pgfpathlineto{\pgfqpoint{2.146223in}{1.795520in}}%
\pgfusepath{stroke}%
\end{pgfscope}%
\begin{pgfscope}%
\pgfpathrectangle{\pgfqpoint{0.100000in}{0.212622in}}{\pgfqpoint{3.696000in}{3.696000in}}%
\pgfusepath{clip}%
\pgfsetrectcap%
\pgfsetroundjoin%
\pgfsetlinewidth{1.505625pt}%
\definecolor{currentstroke}{rgb}{1.000000,0.000000,0.000000}%
\pgfsetstrokecolor{currentstroke}%
\pgfsetdash{}{0pt}%
\pgfpathmoveto{\pgfqpoint{2.171683in}{2.029951in}}%
\pgfpathlineto{\pgfqpoint{2.130451in}{1.800180in}}%
\pgfusepath{stroke}%
\end{pgfscope}%
\begin{pgfscope}%
\pgfpathrectangle{\pgfqpoint{0.100000in}{0.212622in}}{\pgfqpoint{3.696000in}{3.696000in}}%
\pgfusepath{clip}%
\pgfsetrectcap%
\pgfsetroundjoin%
\pgfsetlinewidth{1.505625pt}%
\definecolor{currentstroke}{rgb}{1.000000,0.000000,0.000000}%
\pgfsetstrokecolor{currentstroke}%
\pgfsetdash{}{0pt}%
\pgfpathmoveto{\pgfqpoint{2.166291in}{2.031449in}}%
\pgfpathlineto{\pgfqpoint{2.130451in}{1.800180in}}%
\pgfusepath{stroke}%
\end{pgfscope}%
\begin{pgfscope}%
\pgfpathrectangle{\pgfqpoint{0.100000in}{0.212622in}}{\pgfqpoint{3.696000in}{3.696000in}}%
\pgfusepath{clip}%
\pgfsetrectcap%
\pgfsetroundjoin%
\pgfsetlinewidth{1.505625pt}%
\definecolor{currentstroke}{rgb}{1.000000,0.000000,0.000000}%
\pgfsetstrokecolor{currentstroke}%
\pgfsetdash{}{0pt}%
\pgfpathmoveto{\pgfqpoint{2.157685in}{2.036423in}}%
\pgfpathlineto{\pgfqpoint{2.114691in}{1.804836in}}%
\pgfusepath{stroke}%
\end{pgfscope}%
\begin{pgfscope}%
\pgfpathrectangle{\pgfqpoint{0.100000in}{0.212622in}}{\pgfqpoint{3.696000in}{3.696000in}}%
\pgfusepath{clip}%
\pgfsetrectcap%
\pgfsetroundjoin%
\pgfsetlinewidth{1.505625pt}%
\definecolor{currentstroke}{rgb}{1.000000,0.000000,0.000000}%
\pgfsetstrokecolor{currentstroke}%
\pgfsetdash{}{0pt}%
\pgfpathmoveto{\pgfqpoint{2.146763in}{2.040222in}}%
\pgfpathlineto{\pgfqpoint{2.114691in}{1.804836in}}%
\pgfusepath{stroke}%
\end{pgfscope}%
\begin{pgfscope}%
\pgfpathrectangle{\pgfqpoint{0.100000in}{0.212622in}}{\pgfqpoint{3.696000in}{3.696000in}}%
\pgfusepath{clip}%
\pgfsetrectcap%
\pgfsetroundjoin%
\pgfsetlinewidth{1.505625pt}%
\definecolor{currentstroke}{rgb}{1.000000,0.000000,0.000000}%
\pgfsetstrokecolor{currentstroke}%
\pgfsetdash{}{0pt}%
\pgfpathmoveto{\pgfqpoint{2.140784in}{2.041968in}}%
\pgfpathlineto{\pgfqpoint{2.098943in}{1.809489in}}%
\pgfusepath{stroke}%
\end{pgfscope}%
\begin{pgfscope}%
\pgfpathrectangle{\pgfqpoint{0.100000in}{0.212622in}}{\pgfqpoint{3.696000in}{3.696000in}}%
\pgfusepath{clip}%
\pgfsetrectcap%
\pgfsetroundjoin%
\pgfsetlinewidth{1.505625pt}%
\definecolor{currentstroke}{rgb}{1.000000,0.000000,0.000000}%
\pgfsetstrokecolor{currentstroke}%
\pgfsetdash{}{0pt}%
\pgfpathmoveto{\pgfqpoint{2.132725in}{2.046106in}}%
\pgfpathlineto{\pgfqpoint{2.098943in}{1.809489in}}%
\pgfusepath{stroke}%
\end{pgfscope}%
\begin{pgfscope}%
\pgfpathrectangle{\pgfqpoint{0.100000in}{0.212622in}}{\pgfqpoint{3.696000in}{3.696000in}}%
\pgfusepath{clip}%
\pgfsetrectcap%
\pgfsetroundjoin%
\pgfsetlinewidth{1.505625pt}%
\definecolor{currentstroke}{rgb}{1.000000,0.000000,0.000000}%
\pgfsetstrokecolor{currentstroke}%
\pgfsetdash{}{0pt}%
\pgfpathmoveto{\pgfqpoint{2.121529in}{2.049950in}}%
\pgfpathlineto{\pgfqpoint{2.083208in}{1.814138in}}%
\pgfusepath{stroke}%
\end{pgfscope}%
\begin{pgfscope}%
\pgfpathrectangle{\pgfqpoint{0.100000in}{0.212622in}}{\pgfqpoint{3.696000in}{3.696000in}}%
\pgfusepath{clip}%
\pgfsetrectcap%
\pgfsetroundjoin%
\pgfsetlinewidth{1.505625pt}%
\definecolor{currentstroke}{rgb}{1.000000,0.000000,0.000000}%
\pgfsetstrokecolor{currentstroke}%
\pgfsetdash{}{0pt}%
\pgfpathmoveto{\pgfqpoint{2.110432in}{2.053831in}}%
\pgfpathlineto{\pgfqpoint{2.083208in}{1.814138in}}%
\pgfusepath{stroke}%
\end{pgfscope}%
\begin{pgfscope}%
\pgfpathrectangle{\pgfqpoint{0.100000in}{0.212622in}}{\pgfqpoint{3.696000in}{3.696000in}}%
\pgfusepath{clip}%
\pgfsetrectcap%
\pgfsetroundjoin%
\pgfsetlinewidth{1.505625pt}%
\definecolor{currentstroke}{rgb}{1.000000,0.000000,0.000000}%
\pgfsetstrokecolor{currentstroke}%
\pgfsetdash{}{0pt}%
\pgfpathmoveto{\pgfqpoint{2.099233in}{2.061731in}}%
\pgfpathlineto{\pgfqpoint{2.067485in}{1.818784in}}%
\pgfusepath{stroke}%
\end{pgfscope}%
\begin{pgfscope}%
\pgfpathrectangle{\pgfqpoint{0.100000in}{0.212622in}}{\pgfqpoint{3.696000in}{3.696000in}}%
\pgfusepath{clip}%
\pgfsetrectcap%
\pgfsetroundjoin%
\pgfsetlinewidth{1.505625pt}%
\definecolor{currentstroke}{rgb}{1.000000,0.000000,0.000000}%
\pgfsetstrokecolor{currentstroke}%
\pgfsetdash{}{0pt}%
\pgfpathmoveto{\pgfqpoint{2.084499in}{2.065973in}}%
\pgfpathlineto{\pgfqpoint{2.051774in}{1.823426in}}%
\pgfusepath{stroke}%
\end{pgfscope}%
\begin{pgfscope}%
\pgfpathrectangle{\pgfqpoint{0.100000in}{0.212622in}}{\pgfqpoint{3.696000in}{3.696000in}}%
\pgfusepath{clip}%
\pgfsetrectcap%
\pgfsetroundjoin%
\pgfsetlinewidth{1.505625pt}%
\definecolor{currentstroke}{rgb}{1.000000,0.000000,0.000000}%
\pgfsetstrokecolor{currentstroke}%
\pgfsetdash{}{0pt}%
\pgfpathmoveto{\pgfqpoint{2.075786in}{2.069147in}}%
\pgfpathlineto{\pgfqpoint{2.036076in}{1.828064in}}%
\pgfusepath{stroke}%
\end{pgfscope}%
\begin{pgfscope}%
\pgfpathrectangle{\pgfqpoint{0.100000in}{0.212622in}}{\pgfqpoint{3.696000in}{3.696000in}}%
\pgfusepath{clip}%
\pgfsetrectcap%
\pgfsetroundjoin%
\pgfsetlinewidth{1.505625pt}%
\definecolor{currentstroke}{rgb}{1.000000,0.000000,0.000000}%
\pgfsetstrokecolor{currentstroke}%
\pgfsetdash{}{0pt}%
\pgfpathmoveto{\pgfqpoint{2.067552in}{2.072894in}}%
\pgfpathlineto{\pgfqpoint{2.036076in}{1.828064in}}%
\pgfusepath{stroke}%
\end{pgfscope}%
\begin{pgfscope}%
\pgfpathrectangle{\pgfqpoint{0.100000in}{0.212622in}}{\pgfqpoint{3.696000in}{3.696000in}}%
\pgfusepath{clip}%
\pgfsetrectcap%
\pgfsetroundjoin%
\pgfsetlinewidth{1.505625pt}%
\definecolor{currentstroke}{rgb}{1.000000,0.000000,0.000000}%
\pgfsetstrokecolor{currentstroke}%
\pgfsetdash{}{0pt}%
\pgfpathmoveto{\pgfqpoint{2.056857in}{2.075631in}}%
\pgfpathlineto{\pgfqpoint{2.020390in}{1.832699in}}%
\pgfusepath{stroke}%
\end{pgfscope}%
\begin{pgfscope}%
\pgfpathrectangle{\pgfqpoint{0.100000in}{0.212622in}}{\pgfqpoint{3.696000in}{3.696000in}}%
\pgfusepath{clip}%
\pgfsetrectcap%
\pgfsetroundjoin%
\pgfsetlinewidth{1.505625pt}%
\definecolor{currentstroke}{rgb}{1.000000,0.000000,0.000000}%
\pgfsetstrokecolor{currentstroke}%
\pgfsetdash{}{0pt}%
\pgfpathmoveto{\pgfqpoint{2.045456in}{2.079352in}}%
\pgfpathlineto{\pgfqpoint{2.020390in}{1.832699in}}%
\pgfusepath{stroke}%
\end{pgfscope}%
\begin{pgfscope}%
\pgfpathrectangle{\pgfqpoint{0.100000in}{0.212622in}}{\pgfqpoint{3.696000in}{3.696000in}}%
\pgfusepath{clip}%
\pgfsetrectcap%
\pgfsetroundjoin%
\pgfsetlinewidth{1.505625pt}%
\definecolor{currentstroke}{rgb}{1.000000,0.000000,0.000000}%
\pgfsetstrokecolor{currentstroke}%
\pgfsetdash{}{0pt}%
\pgfpathmoveto{\pgfqpoint{2.039305in}{2.081686in}}%
\pgfpathlineto{\pgfqpoint{2.004717in}{1.837330in}}%
\pgfusepath{stroke}%
\end{pgfscope}%
\begin{pgfscope}%
\pgfpathrectangle{\pgfqpoint{0.100000in}{0.212622in}}{\pgfqpoint{3.696000in}{3.696000in}}%
\pgfusepath{clip}%
\pgfsetrectcap%
\pgfsetroundjoin%
\pgfsetlinewidth{1.505625pt}%
\definecolor{currentstroke}{rgb}{1.000000,0.000000,0.000000}%
\pgfsetstrokecolor{currentstroke}%
\pgfsetdash{}{0pt}%
\pgfpathmoveto{\pgfqpoint{2.030942in}{2.084000in}}%
\pgfpathlineto{\pgfqpoint{2.004717in}{1.837330in}}%
\pgfusepath{stroke}%
\end{pgfscope}%
\begin{pgfscope}%
\pgfpathrectangle{\pgfqpoint{0.100000in}{0.212622in}}{\pgfqpoint{3.696000in}{3.696000in}}%
\pgfusepath{clip}%
\pgfsetrectcap%
\pgfsetroundjoin%
\pgfsetlinewidth{1.505625pt}%
\definecolor{currentstroke}{rgb}{1.000000,0.000000,0.000000}%
\pgfsetstrokecolor{currentstroke}%
\pgfsetdash{}{0pt}%
\pgfpathmoveto{\pgfqpoint{2.022028in}{2.086628in}}%
\pgfpathlineto{\pgfqpoint{1.989055in}{1.841957in}}%
\pgfusepath{stroke}%
\end{pgfscope}%
\begin{pgfscope}%
\pgfpathrectangle{\pgfqpoint{0.100000in}{0.212622in}}{\pgfqpoint{3.696000in}{3.696000in}}%
\pgfusepath{clip}%
\pgfsetrectcap%
\pgfsetroundjoin%
\pgfsetlinewidth{1.505625pt}%
\definecolor{currentstroke}{rgb}{1.000000,0.000000,0.000000}%
\pgfsetstrokecolor{currentstroke}%
\pgfsetdash{}{0pt}%
\pgfpathmoveto{\pgfqpoint{2.017158in}{2.087367in}}%
\pgfpathlineto{\pgfqpoint{1.989055in}{1.841957in}}%
\pgfusepath{stroke}%
\end{pgfscope}%
\begin{pgfscope}%
\pgfpathrectangle{\pgfqpoint{0.100000in}{0.212622in}}{\pgfqpoint{3.696000in}{3.696000in}}%
\pgfusepath{clip}%
\pgfsetrectcap%
\pgfsetroundjoin%
\pgfsetlinewidth{1.505625pt}%
\definecolor{currentstroke}{rgb}{1.000000,0.000000,0.000000}%
\pgfsetstrokecolor{currentstroke}%
\pgfsetdash{}{0pt}%
\pgfpathmoveto{\pgfqpoint{2.009780in}{2.089344in}}%
\pgfpathlineto{\pgfqpoint{1.973406in}{1.846581in}}%
\pgfusepath{stroke}%
\end{pgfscope}%
\begin{pgfscope}%
\pgfpathrectangle{\pgfqpoint{0.100000in}{0.212622in}}{\pgfqpoint{3.696000in}{3.696000in}}%
\pgfusepath{clip}%
\pgfsetrectcap%
\pgfsetroundjoin%
\pgfsetlinewidth{1.505625pt}%
\definecolor{currentstroke}{rgb}{1.000000,0.000000,0.000000}%
\pgfsetstrokecolor{currentstroke}%
\pgfsetdash{}{0pt}%
\pgfpathmoveto{\pgfqpoint{2.001760in}{2.091715in}}%
\pgfpathlineto{\pgfqpoint{1.973406in}{1.846581in}}%
\pgfusepath{stroke}%
\end{pgfscope}%
\begin{pgfscope}%
\pgfpathrectangle{\pgfqpoint{0.100000in}{0.212622in}}{\pgfqpoint{3.696000in}{3.696000in}}%
\pgfusepath{clip}%
\pgfsetrectcap%
\pgfsetroundjoin%
\pgfsetlinewidth{1.505625pt}%
\definecolor{currentstroke}{rgb}{1.000000,0.000000,0.000000}%
\pgfsetstrokecolor{currentstroke}%
\pgfsetdash{}{0pt}%
\pgfpathmoveto{\pgfqpoint{1.994442in}{2.096195in}}%
\pgfpathlineto{\pgfqpoint{1.957769in}{1.851201in}}%
\pgfusepath{stroke}%
\end{pgfscope}%
\begin{pgfscope}%
\pgfpathrectangle{\pgfqpoint{0.100000in}{0.212622in}}{\pgfqpoint{3.696000in}{3.696000in}}%
\pgfusepath{clip}%
\pgfsetrectcap%
\pgfsetroundjoin%
\pgfsetlinewidth{1.505625pt}%
\definecolor{currentstroke}{rgb}{1.000000,0.000000,0.000000}%
\pgfsetstrokecolor{currentstroke}%
\pgfsetdash{}{0pt}%
\pgfpathmoveto{\pgfqpoint{1.982555in}{2.100662in}}%
\pgfpathlineto{\pgfqpoint{1.957769in}{1.851201in}}%
\pgfusepath{stroke}%
\end{pgfscope}%
\begin{pgfscope}%
\pgfpathrectangle{\pgfqpoint{0.100000in}{0.212622in}}{\pgfqpoint{3.696000in}{3.696000in}}%
\pgfusepath{clip}%
\pgfsetrectcap%
\pgfsetroundjoin%
\pgfsetlinewidth{1.505625pt}%
\definecolor{currentstroke}{rgb}{1.000000,0.000000,0.000000}%
\pgfsetstrokecolor{currentstroke}%
\pgfsetdash{}{0pt}%
\pgfpathmoveto{\pgfqpoint{1.970514in}{2.104774in}}%
\pgfpathlineto{\pgfqpoint{1.942145in}{1.855817in}}%
\pgfusepath{stroke}%
\end{pgfscope}%
\begin{pgfscope}%
\pgfpathrectangle{\pgfqpoint{0.100000in}{0.212622in}}{\pgfqpoint{3.696000in}{3.696000in}}%
\pgfusepath{clip}%
\pgfsetrectcap%
\pgfsetroundjoin%
\pgfsetlinewidth{1.505625pt}%
\definecolor{currentstroke}{rgb}{1.000000,0.000000,0.000000}%
\pgfsetstrokecolor{currentstroke}%
\pgfsetdash{}{0pt}%
\pgfpathmoveto{\pgfqpoint{1.958461in}{2.110743in}}%
\pgfpathlineto{\pgfqpoint{1.926533in}{1.860430in}}%
\pgfusepath{stroke}%
\end{pgfscope}%
\begin{pgfscope}%
\pgfpathrectangle{\pgfqpoint{0.100000in}{0.212622in}}{\pgfqpoint{3.696000in}{3.696000in}}%
\pgfusepath{clip}%
\pgfsetrectcap%
\pgfsetroundjoin%
\pgfsetlinewidth{1.505625pt}%
\definecolor{currentstroke}{rgb}{1.000000,0.000000,0.000000}%
\pgfsetstrokecolor{currentstroke}%
\pgfsetdash{}{0pt}%
\pgfpathmoveto{\pgfqpoint{1.942859in}{2.115426in}}%
\pgfpathlineto{\pgfqpoint{1.910933in}{1.865039in}}%
\pgfusepath{stroke}%
\end{pgfscope}%
\begin{pgfscope}%
\pgfpathrectangle{\pgfqpoint{0.100000in}{0.212622in}}{\pgfqpoint{3.696000in}{3.696000in}}%
\pgfusepath{clip}%
\pgfsetrectcap%
\pgfsetroundjoin%
\pgfsetlinewidth{1.505625pt}%
\definecolor{currentstroke}{rgb}{1.000000,0.000000,0.000000}%
\pgfsetstrokecolor{currentstroke}%
\pgfsetdash{}{0pt}%
\pgfpathmoveto{\pgfqpoint{1.926940in}{2.120208in}}%
\pgfpathlineto{\pgfqpoint{1.895345in}{1.869645in}}%
\pgfusepath{stroke}%
\end{pgfscope}%
\begin{pgfscope}%
\pgfpathrectangle{\pgfqpoint{0.100000in}{0.212622in}}{\pgfqpoint{3.696000in}{3.696000in}}%
\pgfusepath{clip}%
\pgfsetrectcap%
\pgfsetroundjoin%
\pgfsetlinewidth{1.505625pt}%
\definecolor{currentstroke}{rgb}{1.000000,0.000000,0.000000}%
\pgfsetstrokecolor{currentstroke}%
\pgfsetdash{}{0pt}%
\pgfpathmoveto{\pgfqpoint{1.910357in}{2.126936in}}%
\pgfpathlineto{\pgfqpoint{1.879769in}{1.874247in}}%
\pgfusepath{stroke}%
\end{pgfscope}%
\begin{pgfscope}%
\pgfpathrectangle{\pgfqpoint{0.100000in}{0.212622in}}{\pgfqpoint{3.696000in}{3.696000in}}%
\pgfusepath{clip}%
\pgfsetrectcap%
\pgfsetroundjoin%
\pgfsetlinewidth{1.505625pt}%
\definecolor{currentstroke}{rgb}{1.000000,0.000000,0.000000}%
\pgfsetstrokecolor{currentstroke}%
\pgfsetdash{}{0pt}%
\pgfpathmoveto{\pgfqpoint{1.893622in}{2.129148in}}%
\pgfpathlineto{\pgfqpoint{1.864205in}{1.878845in}}%
\pgfusepath{stroke}%
\end{pgfscope}%
\begin{pgfscope}%
\pgfpathrectangle{\pgfqpoint{0.100000in}{0.212622in}}{\pgfqpoint{3.696000in}{3.696000in}}%
\pgfusepath{clip}%
\pgfsetrectcap%
\pgfsetroundjoin%
\pgfsetlinewidth{1.505625pt}%
\definecolor{currentstroke}{rgb}{1.000000,0.000000,0.000000}%
\pgfsetstrokecolor{currentstroke}%
\pgfsetdash{}{0pt}%
\pgfpathmoveto{\pgfqpoint{1.874765in}{2.135047in}}%
\pgfpathlineto{\pgfqpoint{1.848654in}{1.883440in}}%
\pgfusepath{stroke}%
\end{pgfscope}%
\begin{pgfscope}%
\pgfpathrectangle{\pgfqpoint{0.100000in}{0.212622in}}{\pgfqpoint{3.696000in}{3.696000in}}%
\pgfusepath{clip}%
\pgfsetrectcap%
\pgfsetroundjoin%
\pgfsetlinewidth{1.505625pt}%
\definecolor{currentstroke}{rgb}{1.000000,0.000000,0.000000}%
\pgfsetstrokecolor{currentstroke}%
\pgfsetdash{}{0pt}%
\pgfpathmoveto{\pgfqpoint{1.862921in}{2.133305in}}%
\pgfpathlineto{\pgfqpoint{1.833115in}{1.888032in}}%
\pgfusepath{stroke}%
\end{pgfscope}%
\begin{pgfscope}%
\pgfpathrectangle{\pgfqpoint{0.100000in}{0.212622in}}{\pgfqpoint{3.696000in}{3.696000in}}%
\pgfusepath{clip}%
\pgfsetrectcap%
\pgfsetroundjoin%
\pgfsetlinewidth{1.505625pt}%
\definecolor{currentstroke}{rgb}{1.000000,0.000000,0.000000}%
\pgfsetstrokecolor{currentstroke}%
\pgfsetdash{}{0pt}%
\pgfpathmoveto{\pgfqpoint{1.850363in}{2.136178in}}%
\pgfpathlineto{\pgfqpoint{1.817588in}{1.892619in}}%
\pgfusepath{stroke}%
\end{pgfscope}%
\begin{pgfscope}%
\pgfpathrectangle{\pgfqpoint{0.100000in}{0.212622in}}{\pgfqpoint{3.696000in}{3.696000in}}%
\pgfusepath{clip}%
\pgfsetrectcap%
\pgfsetroundjoin%
\pgfsetlinewidth{1.505625pt}%
\definecolor{currentstroke}{rgb}{1.000000,0.000000,0.000000}%
\pgfsetstrokecolor{currentstroke}%
\pgfsetdash{}{0pt}%
\pgfpathmoveto{\pgfqpoint{1.837404in}{2.140837in}}%
\pgfpathlineto{\pgfqpoint{1.817588in}{1.892619in}}%
\pgfusepath{stroke}%
\end{pgfscope}%
\begin{pgfscope}%
\pgfpathrectangle{\pgfqpoint{0.100000in}{0.212622in}}{\pgfqpoint{3.696000in}{3.696000in}}%
\pgfusepath{clip}%
\pgfsetrectcap%
\pgfsetroundjoin%
\pgfsetlinewidth{1.505625pt}%
\definecolor{currentstroke}{rgb}{1.000000,0.000000,0.000000}%
\pgfsetstrokecolor{currentstroke}%
\pgfsetdash{}{0pt}%
\pgfpathmoveto{\pgfqpoint{1.830100in}{2.142364in}}%
\pgfpathlineto{\pgfqpoint{1.802073in}{1.897203in}}%
\pgfusepath{stroke}%
\end{pgfscope}%
\begin{pgfscope}%
\pgfpathrectangle{\pgfqpoint{0.100000in}{0.212622in}}{\pgfqpoint{3.696000in}{3.696000in}}%
\pgfusepath{clip}%
\pgfsetrectcap%
\pgfsetroundjoin%
\pgfsetlinewidth{1.505625pt}%
\definecolor{currentstroke}{rgb}{1.000000,0.000000,0.000000}%
\pgfsetstrokecolor{currentstroke}%
\pgfsetdash{}{0pt}%
\pgfpathmoveto{\pgfqpoint{1.822488in}{2.143678in}}%
\pgfpathlineto{\pgfqpoint{1.802073in}{1.897203in}}%
\pgfusepath{stroke}%
\end{pgfscope}%
\begin{pgfscope}%
\pgfpathrectangle{\pgfqpoint{0.100000in}{0.212622in}}{\pgfqpoint{3.696000in}{3.696000in}}%
\pgfusepath{clip}%
\pgfsetrectcap%
\pgfsetroundjoin%
\pgfsetlinewidth{1.505625pt}%
\definecolor{currentstroke}{rgb}{1.000000,0.000000,0.000000}%
\pgfsetstrokecolor{currentstroke}%
\pgfsetdash{}{0pt}%
\pgfpathmoveto{\pgfqpoint{1.811405in}{2.148703in}}%
\pgfpathlineto{\pgfqpoint{1.786570in}{1.901784in}}%
\pgfusepath{stroke}%
\end{pgfscope}%
\begin{pgfscope}%
\pgfpathrectangle{\pgfqpoint{0.100000in}{0.212622in}}{\pgfqpoint{3.696000in}{3.696000in}}%
\pgfusepath{clip}%
\pgfsetrectcap%
\pgfsetroundjoin%
\pgfsetlinewidth{1.505625pt}%
\definecolor{currentstroke}{rgb}{1.000000,0.000000,0.000000}%
\pgfsetstrokecolor{currentstroke}%
\pgfsetdash{}{0pt}%
\pgfpathmoveto{\pgfqpoint{1.799494in}{2.152803in}}%
\pgfpathlineto{\pgfqpoint{1.771079in}{1.906361in}}%
\pgfusepath{stroke}%
\end{pgfscope}%
\begin{pgfscope}%
\pgfpathrectangle{\pgfqpoint{0.100000in}{0.212622in}}{\pgfqpoint{3.696000in}{3.696000in}}%
\pgfusepath{clip}%
\pgfsetrectcap%
\pgfsetroundjoin%
\pgfsetlinewidth{1.505625pt}%
\definecolor{currentstroke}{rgb}{1.000000,0.000000,0.000000}%
\pgfsetstrokecolor{currentstroke}%
\pgfsetdash{}{0pt}%
\pgfpathmoveto{\pgfqpoint{1.792906in}{2.154645in}}%
\pgfpathlineto{\pgfqpoint{1.771079in}{1.906361in}}%
\pgfusepath{stroke}%
\end{pgfscope}%
\begin{pgfscope}%
\pgfpathrectangle{\pgfqpoint{0.100000in}{0.212622in}}{\pgfqpoint{3.696000in}{3.696000in}}%
\pgfusepath{clip}%
\pgfsetrectcap%
\pgfsetroundjoin%
\pgfsetlinewidth{1.505625pt}%
\definecolor{currentstroke}{rgb}{1.000000,0.000000,0.000000}%
\pgfsetstrokecolor{currentstroke}%
\pgfsetdash{}{0pt}%
\pgfpathmoveto{\pgfqpoint{1.784358in}{2.159294in}}%
\pgfpathlineto{\pgfqpoint{1.755600in}{1.910934in}}%
\pgfusepath{stroke}%
\end{pgfscope}%
\begin{pgfscope}%
\pgfpathrectangle{\pgfqpoint{0.100000in}{0.212622in}}{\pgfqpoint{3.696000in}{3.696000in}}%
\pgfusepath{clip}%
\pgfsetrectcap%
\pgfsetroundjoin%
\pgfsetlinewidth{1.505625pt}%
\definecolor{currentstroke}{rgb}{1.000000,0.000000,0.000000}%
\pgfsetstrokecolor{currentstroke}%
\pgfsetdash{}{0pt}%
\pgfpathmoveto{\pgfqpoint{1.773688in}{2.162753in}}%
\pgfpathlineto{\pgfqpoint{1.755600in}{1.910934in}}%
\pgfusepath{stroke}%
\end{pgfscope}%
\begin{pgfscope}%
\pgfpathrectangle{\pgfqpoint{0.100000in}{0.212622in}}{\pgfqpoint{3.696000in}{3.696000in}}%
\pgfusepath{clip}%
\pgfsetrectcap%
\pgfsetroundjoin%
\pgfsetlinewidth{1.505625pt}%
\definecolor{currentstroke}{rgb}{1.000000,0.000000,0.000000}%
\pgfsetstrokecolor{currentstroke}%
\pgfsetdash{}{0pt}%
\pgfpathmoveto{\pgfqpoint{1.768062in}{2.164716in}}%
\pgfpathlineto{\pgfqpoint{1.740133in}{1.915504in}}%
\pgfusepath{stroke}%
\end{pgfscope}%
\begin{pgfscope}%
\pgfpathrectangle{\pgfqpoint{0.100000in}{0.212622in}}{\pgfqpoint{3.696000in}{3.696000in}}%
\pgfusepath{clip}%
\pgfsetrectcap%
\pgfsetroundjoin%
\pgfsetlinewidth{1.505625pt}%
\definecolor{currentstroke}{rgb}{1.000000,0.000000,0.000000}%
\pgfsetstrokecolor{currentstroke}%
\pgfsetdash{}{0pt}%
\pgfpathmoveto{\pgfqpoint{1.761390in}{2.168816in}}%
\pgfpathlineto{\pgfqpoint{1.740133in}{1.915504in}}%
\pgfusepath{stroke}%
\end{pgfscope}%
\begin{pgfscope}%
\pgfpathrectangle{\pgfqpoint{0.100000in}{0.212622in}}{\pgfqpoint{3.696000in}{3.696000in}}%
\pgfusepath{clip}%
\pgfsetrectcap%
\pgfsetroundjoin%
\pgfsetlinewidth{1.505625pt}%
\definecolor{currentstroke}{rgb}{1.000000,0.000000,0.000000}%
\pgfsetstrokecolor{currentstroke}%
\pgfsetdash{}{0pt}%
\pgfpathmoveto{\pgfqpoint{1.751119in}{2.172176in}}%
\pgfpathlineto{\pgfqpoint{1.724679in}{1.920071in}}%
\pgfusepath{stroke}%
\end{pgfscope}%
\begin{pgfscope}%
\pgfpathrectangle{\pgfqpoint{0.100000in}{0.212622in}}{\pgfqpoint{3.696000in}{3.696000in}}%
\pgfusepath{clip}%
\pgfsetrectcap%
\pgfsetroundjoin%
\pgfsetlinewidth{1.505625pt}%
\definecolor{currentstroke}{rgb}{1.000000,0.000000,0.000000}%
\pgfsetstrokecolor{currentstroke}%
\pgfsetdash{}{0pt}%
\pgfpathmoveto{\pgfqpoint{1.739472in}{2.176912in}}%
\pgfpathlineto{\pgfqpoint{1.709236in}{1.924633in}}%
\pgfusepath{stroke}%
\end{pgfscope}%
\begin{pgfscope}%
\pgfpathrectangle{\pgfqpoint{0.100000in}{0.212622in}}{\pgfqpoint{3.696000in}{3.696000in}}%
\pgfusepath{clip}%
\pgfsetrectcap%
\pgfsetroundjoin%
\pgfsetlinewidth{1.505625pt}%
\definecolor{currentstroke}{rgb}{1.000000,0.000000,0.000000}%
\pgfsetstrokecolor{currentstroke}%
\pgfsetdash{}{0pt}%
\pgfpathmoveto{\pgfqpoint{1.732479in}{2.177062in}}%
\pgfpathlineto{\pgfqpoint{1.709236in}{1.924633in}}%
\pgfusepath{stroke}%
\end{pgfscope}%
\begin{pgfscope}%
\pgfpathrectangle{\pgfqpoint{0.100000in}{0.212622in}}{\pgfqpoint{3.696000in}{3.696000in}}%
\pgfusepath{clip}%
\pgfsetrectcap%
\pgfsetroundjoin%
\pgfsetlinewidth{1.505625pt}%
\definecolor{currentstroke}{rgb}{1.000000,0.000000,0.000000}%
\pgfsetstrokecolor{currentstroke}%
\pgfsetdash{}{0pt}%
\pgfpathmoveto{\pgfqpoint{1.723746in}{2.179401in}}%
\pgfpathlineto{\pgfqpoint{1.693806in}{1.929192in}}%
\pgfusepath{stroke}%
\end{pgfscope}%
\begin{pgfscope}%
\pgfpathrectangle{\pgfqpoint{0.100000in}{0.212622in}}{\pgfqpoint{3.696000in}{3.696000in}}%
\pgfusepath{clip}%
\pgfsetrectcap%
\pgfsetroundjoin%
\pgfsetlinewidth{1.505625pt}%
\definecolor{currentstroke}{rgb}{1.000000,0.000000,0.000000}%
\pgfsetstrokecolor{currentstroke}%
\pgfsetdash{}{0pt}%
\pgfpathmoveto{\pgfqpoint{1.713796in}{2.183502in}}%
\pgfpathlineto{\pgfqpoint{1.693806in}{1.929192in}}%
\pgfusepath{stroke}%
\end{pgfscope}%
\begin{pgfscope}%
\pgfpathrectangle{\pgfqpoint{0.100000in}{0.212622in}}{\pgfqpoint{3.696000in}{3.696000in}}%
\pgfusepath{clip}%
\pgfsetrectcap%
\pgfsetroundjoin%
\pgfsetlinewidth{1.505625pt}%
\definecolor{currentstroke}{rgb}{1.000000,0.000000,0.000000}%
\pgfsetstrokecolor{currentstroke}%
\pgfsetdash{}{0pt}%
\pgfpathmoveto{\pgfqpoint{1.708336in}{2.185468in}}%
\pgfpathlineto{\pgfqpoint{1.678387in}{1.933748in}}%
\pgfusepath{stroke}%
\end{pgfscope}%
\begin{pgfscope}%
\pgfpathrectangle{\pgfqpoint{0.100000in}{0.212622in}}{\pgfqpoint{3.696000in}{3.696000in}}%
\pgfusepath{clip}%
\pgfsetrectcap%
\pgfsetroundjoin%
\pgfsetlinewidth{1.505625pt}%
\definecolor{currentstroke}{rgb}{1.000000,0.000000,0.000000}%
\pgfsetstrokecolor{currentstroke}%
\pgfsetdash{}{0pt}%
\pgfpathmoveto{\pgfqpoint{1.701268in}{2.187489in}}%
\pgfpathlineto{\pgfqpoint{1.678387in}{1.933748in}}%
\pgfusepath{stroke}%
\end{pgfscope}%
\begin{pgfscope}%
\pgfpathrectangle{\pgfqpoint{0.100000in}{0.212622in}}{\pgfqpoint{3.696000in}{3.696000in}}%
\pgfusepath{clip}%
\pgfsetrectcap%
\pgfsetroundjoin%
\pgfsetlinewidth{1.505625pt}%
\definecolor{currentstroke}{rgb}{1.000000,0.000000,0.000000}%
\pgfsetstrokecolor{currentstroke}%
\pgfsetdash{}{0pt}%
\pgfpathmoveto{\pgfqpoint{1.692054in}{2.193084in}}%
\pgfpathlineto{\pgfqpoint{1.662980in}{1.938300in}}%
\pgfusepath{stroke}%
\end{pgfscope}%
\begin{pgfscope}%
\pgfpathrectangle{\pgfqpoint{0.100000in}{0.212622in}}{\pgfqpoint{3.696000in}{3.696000in}}%
\pgfusepath{clip}%
\pgfsetrectcap%
\pgfsetroundjoin%
\pgfsetlinewidth{1.505625pt}%
\definecolor{currentstroke}{rgb}{1.000000,0.000000,0.000000}%
\pgfsetstrokecolor{currentstroke}%
\pgfsetdash{}{0pt}%
\pgfpathmoveto{\pgfqpoint{1.680513in}{2.197078in}}%
\pgfpathlineto{\pgfqpoint{1.662980in}{1.938300in}}%
\pgfusepath{stroke}%
\end{pgfscope}%
\begin{pgfscope}%
\pgfpathrectangle{\pgfqpoint{0.100000in}{0.212622in}}{\pgfqpoint{3.696000in}{3.696000in}}%
\pgfusepath{clip}%
\pgfsetrectcap%
\pgfsetroundjoin%
\pgfsetlinewidth{1.505625pt}%
\definecolor{currentstroke}{rgb}{1.000000,0.000000,0.000000}%
\pgfsetstrokecolor{currentstroke}%
\pgfsetdash{}{0pt}%
\pgfpathmoveto{\pgfqpoint{1.674131in}{2.198942in}}%
\pgfpathlineto{\pgfqpoint{1.647586in}{1.942849in}}%
\pgfusepath{stroke}%
\end{pgfscope}%
\begin{pgfscope}%
\pgfpathrectangle{\pgfqpoint{0.100000in}{0.212622in}}{\pgfqpoint{3.696000in}{3.696000in}}%
\pgfusepath{clip}%
\pgfsetrectcap%
\pgfsetroundjoin%
\pgfsetlinewidth{1.505625pt}%
\definecolor{currentstroke}{rgb}{1.000000,0.000000,0.000000}%
\pgfsetstrokecolor{currentstroke}%
\pgfsetdash{}{0pt}%
\pgfpathmoveto{\pgfqpoint{1.667197in}{2.204008in}}%
\pgfpathlineto{\pgfqpoint{1.647586in}{1.942849in}}%
\pgfusepath{stroke}%
\end{pgfscope}%
\begin{pgfscope}%
\pgfpathrectangle{\pgfqpoint{0.100000in}{0.212622in}}{\pgfqpoint{3.696000in}{3.696000in}}%
\pgfusepath{clip}%
\pgfsetrectcap%
\pgfsetroundjoin%
\pgfsetlinewidth{1.505625pt}%
\definecolor{currentstroke}{rgb}{1.000000,0.000000,0.000000}%
\pgfsetstrokecolor{currentstroke}%
\pgfsetdash{}{0pt}%
\pgfpathmoveto{\pgfqpoint{1.655474in}{2.208289in}}%
\pgfpathlineto{\pgfqpoint{1.632203in}{1.947394in}}%
\pgfusepath{stroke}%
\end{pgfscope}%
\begin{pgfscope}%
\pgfpathrectangle{\pgfqpoint{0.100000in}{0.212622in}}{\pgfqpoint{3.696000in}{3.696000in}}%
\pgfusepath{clip}%
\pgfsetrectcap%
\pgfsetroundjoin%
\pgfsetlinewidth{1.505625pt}%
\definecolor{currentstroke}{rgb}{1.000000,0.000000,0.000000}%
\pgfsetstrokecolor{currentstroke}%
\pgfsetdash{}{0pt}%
\pgfpathmoveto{\pgfqpoint{1.643810in}{2.212436in}}%
\pgfpathlineto{\pgfqpoint{1.616832in}{1.951935in}}%
\pgfusepath{stroke}%
\end{pgfscope}%
\begin{pgfscope}%
\pgfpathrectangle{\pgfqpoint{0.100000in}{0.212622in}}{\pgfqpoint{3.696000in}{3.696000in}}%
\pgfusepath{clip}%
\pgfsetrectcap%
\pgfsetroundjoin%
\pgfsetlinewidth{1.505625pt}%
\definecolor{currentstroke}{rgb}{1.000000,0.000000,0.000000}%
\pgfsetstrokecolor{currentstroke}%
\pgfsetdash{}{0pt}%
\pgfpathmoveto{\pgfqpoint{1.631876in}{2.218428in}}%
\pgfpathlineto{\pgfqpoint{1.601473in}{1.956473in}}%
\pgfusepath{stroke}%
\end{pgfscope}%
\begin{pgfscope}%
\pgfpathrectangle{\pgfqpoint{0.100000in}{0.212622in}}{\pgfqpoint{3.696000in}{3.696000in}}%
\pgfusepath{clip}%
\pgfsetrectcap%
\pgfsetroundjoin%
\pgfsetlinewidth{1.505625pt}%
\definecolor{currentstroke}{rgb}{1.000000,0.000000,0.000000}%
\pgfsetstrokecolor{currentstroke}%
\pgfsetdash{}{0pt}%
\pgfpathmoveto{\pgfqpoint{1.617504in}{2.221933in}}%
\pgfpathlineto{\pgfqpoint{1.601473in}{1.956473in}}%
\pgfusepath{stroke}%
\end{pgfscope}%
\begin{pgfscope}%
\pgfpathrectangle{\pgfqpoint{0.100000in}{0.212622in}}{\pgfqpoint{3.696000in}{3.696000in}}%
\pgfusepath{clip}%
\pgfsetrectcap%
\pgfsetroundjoin%
\pgfsetlinewidth{1.505625pt}%
\definecolor{currentstroke}{rgb}{1.000000,0.000000,0.000000}%
\pgfsetstrokecolor{currentstroke}%
\pgfsetdash{}{0pt}%
\pgfpathmoveto{\pgfqpoint{1.600914in}{2.227393in}}%
\pgfpathlineto{\pgfqpoint{1.586126in}{1.961008in}}%
\pgfusepath{stroke}%
\end{pgfscope}%
\begin{pgfscope}%
\pgfpathrectangle{\pgfqpoint{0.100000in}{0.212622in}}{\pgfqpoint{3.696000in}{3.696000in}}%
\pgfusepath{clip}%
\pgfsetrectcap%
\pgfsetroundjoin%
\pgfsetlinewidth{1.505625pt}%
\definecolor{currentstroke}{rgb}{1.000000,0.000000,0.000000}%
\pgfsetstrokecolor{currentstroke}%
\pgfsetdash{}{0pt}%
\pgfpathmoveto{\pgfqpoint{1.584615in}{2.233040in}}%
\pgfpathlineto{\pgfqpoint{1.555468in}{1.970066in}}%
\pgfusepath{stroke}%
\end{pgfscope}%
\begin{pgfscope}%
\pgfpathrectangle{\pgfqpoint{0.100000in}{0.212622in}}{\pgfqpoint{3.696000in}{3.696000in}}%
\pgfusepath{clip}%
\pgfsetrectcap%
\pgfsetroundjoin%
\pgfsetlinewidth{1.505625pt}%
\definecolor{currentstroke}{rgb}{1.000000,0.000000,0.000000}%
\pgfsetstrokecolor{currentstroke}%
\pgfsetdash{}{0pt}%
\pgfpathmoveto{\pgfqpoint{1.566303in}{2.238112in}}%
\pgfpathlineto{\pgfqpoint{1.540157in}{1.974590in}}%
\pgfusepath{stroke}%
\end{pgfscope}%
\begin{pgfscope}%
\pgfpathrectangle{\pgfqpoint{0.100000in}{0.212622in}}{\pgfqpoint{3.696000in}{3.696000in}}%
\pgfusepath{clip}%
\pgfsetrectcap%
\pgfsetroundjoin%
\pgfsetlinewidth{1.505625pt}%
\definecolor{currentstroke}{rgb}{1.000000,0.000000,0.000000}%
\pgfsetstrokecolor{currentstroke}%
\pgfsetdash{}{0pt}%
\pgfpathmoveto{\pgfqpoint{1.544840in}{2.246656in}}%
\pgfpathlineto{\pgfqpoint{1.524858in}{1.979110in}}%
\pgfusepath{stroke}%
\end{pgfscope}%
\begin{pgfscope}%
\pgfpathrectangle{\pgfqpoint{0.100000in}{0.212622in}}{\pgfqpoint{3.696000in}{3.696000in}}%
\pgfusepath{clip}%
\pgfsetrectcap%
\pgfsetroundjoin%
\pgfsetlinewidth{1.505625pt}%
\definecolor{currentstroke}{rgb}{1.000000,0.000000,0.000000}%
\pgfsetstrokecolor{currentstroke}%
\pgfsetdash{}{0pt}%
\pgfpathmoveto{\pgfqpoint{1.522965in}{2.255278in}}%
\pgfpathlineto{\pgfqpoint{1.509570in}{1.983627in}}%
\pgfusepath{stroke}%
\end{pgfscope}%
\begin{pgfscope}%
\pgfpathrectangle{\pgfqpoint{0.100000in}{0.212622in}}{\pgfqpoint{3.696000in}{3.696000in}}%
\pgfusepath{clip}%
\pgfsetrectcap%
\pgfsetroundjoin%
\pgfsetlinewidth{1.505625pt}%
\definecolor{currentstroke}{rgb}{1.000000,0.000000,0.000000}%
\pgfsetstrokecolor{currentstroke}%
\pgfsetdash{}{0pt}%
\pgfpathmoveto{\pgfqpoint{1.510734in}{2.258849in}}%
\pgfpathlineto{\pgfqpoint{1.494294in}{1.988141in}}%
\pgfusepath{stroke}%
\end{pgfscope}%
\begin{pgfscope}%
\pgfpathrectangle{\pgfqpoint{0.100000in}{0.212622in}}{\pgfqpoint{3.696000in}{3.696000in}}%
\pgfusepath{clip}%
\pgfsetrectcap%
\pgfsetroundjoin%
\pgfsetlinewidth{1.505625pt}%
\definecolor{currentstroke}{rgb}{1.000000,0.000000,0.000000}%
\pgfsetstrokecolor{currentstroke}%
\pgfsetdash{}{0pt}%
\pgfpathmoveto{\pgfqpoint{1.495962in}{2.265080in}}%
\pgfpathlineto{\pgfqpoint{1.479031in}{1.992651in}}%
\pgfusepath{stroke}%
\end{pgfscope}%
\begin{pgfscope}%
\pgfpathrectangle{\pgfqpoint{0.100000in}{0.212622in}}{\pgfqpoint{3.696000in}{3.696000in}}%
\pgfusepath{clip}%
\pgfsetrectcap%
\pgfsetroundjoin%
\pgfsetlinewidth{1.505625pt}%
\definecolor{currentstroke}{rgb}{1.000000,0.000000,0.000000}%
\pgfsetstrokecolor{currentstroke}%
\pgfsetdash{}{0pt}%
\pgfpathmoveto{\pgfqpoint{1.477950in}{2.272113in}}%
\pgfpathlineto{\pgfqpoint{1.463779in}{1.997157in}}%
\pgfusepath{stroke}%
\end{pgfscope}%
\begin{pgfscope}%
\pgfpathrectangle{\pgfqpoint{0.100000in}{0.212622in}}{\pgfqpoint{3.696000in}{3.696000in}}%
\pgfusepath{clip}%
\pgfsetrectcap%
\pgfsetroundjoin%
\pgfsetlinewidth{1.505625pt}%
\definecolor{currentstroke}{rgb}{1.000000,0.000000,0.000000}%
\pgfsetstrokecolor{currentstroke}%
\pgfsetdash{}{0pt}%
\pgfpathmoveto{\pgfqpoint{1.459420in}{2.278402in}}%
\pgfpathlineto{\pgfqpoint{1.448538in}{2.001660in}}%
\pgfusepath{stroke}%
\end{pgfscope}%
\begin{pgfscope}%
\pgfpathrectangle{\pgfqpoint{0.100000in}{0.212622in}}{\pgfqpoint{3.696000in}{3.696000in}}%
\pgfusepath{clip}%
\pgfsetrectcap%
\pgfsetroundjoin%
\pgfsetlinewidth{1.505625pt}%
\definecolor{currentstroke}{rgb}{1.000000,0.000000,0.000000}%
\pgfsetstrokecolor{currentstroke}%
\pgfsetdash{}{0pt}%
\pgfpathmoveto{\pgfqpoint{1.441097in}{2.288919in}}%
\pgfpathlineto{\pgfqpoint{1.433310in}{2.006159in}}%
\pgfusepath{stroke}%
\end{pgfscope}%
\begin{pgfscope}%
\pgfpathrectangle{\pgfqpoint{0.100000in}{0.212622in}}{\pgfqpoint{3.696000in}{3.696000in}}%
\pgfusepath{clip}%
\pgfsetrectcap%
\pgfsetroundjoin%
\pgfsetlinewidth{1.505625pt}%
\definecolor{currentstroke}{rgb}{1.000000,0.000000,0.000000}%
\pgfsetstrokecolor{currentstroke}%
\pgfsetdash{}{0pt}%
\pgfpathmoveto{\pgfqpoint{1.416905in}{2.297578in}}%
\pgfpathlineto{\pgfqpoint{1.402888in}{2.015148in}}%
\pgfusepath{stroke}%
\end{pgfscope}%
\begin{pgfscope}%
\pgfpathrectangle{\pgfqpoint{0.100000in}{0.212622in}}{\pgfqpoint{3.696000in}{3.696000in}}%
\pgfusepath{clip}%
\pgfsetrectcap%
\pgfsetroundjoin%
\pgfsetlinewidth{1.505625pt}%
\definecolor{currentstroke}{rgb}{1.000000,0.000000,0.000000}%
\pgfsetstrokecolor{currentstroke}%
\pgfsetdash{}{0pt}%
\pgfpathmoveto{\pgfqpoint{1.392141in}{2.304509in}}%
\pgfpathlineto{\pgfqpoint{1.387695in}{2.019637in}}%
\pgfusepath{stroke}%
\end{pgfscope}%
\begin{pgfscope}%
\pgfpathrectangle{\pgfqpoint{0.100000in}{0.212622in}}{\pgfqpoint{3.696000in}{3.696000in}}%
\pgfusepath{clip}%
\pgfsetrectcap%
\pgfsetroundjoin%
\pgfsetlinewidth{1.505625pt}%
\definecolor{currentstroke}{rgb}{1.000000,0.000000,0.000000}%
\pgfsetstrokecolor{currentstroke}%
\pgfsetdash{}{0pt}%
\pgfpathmoveto{\pgfqpoint{1.379545in}{2.306531in}}%
\pgfpathlineto{\pgfqpoint{1.372513in}{2.024123in}}%
\pgfusepath{stroke}%
\end{pgfscope}%
\begin{pgfscope}%
\pgfpathrectangle{\pgfqpoint{0.100000in}{0.212622in}}{\pgfqpoint{3.696000in}{3.696000in}}%
\pgfusepath{clip}%
\pgfsetrectcap%
\pgfsetroundjoin%
\pgfsetlinewidth{1.505625pt}%
\definecolor{currentstroke}{rgb}{1.000000,0.000000,0.000000}%
\pgfsetstrokecolor{currentstroke}%
\pgfsetdash{}{0pt}%
\pgfpathmoveto{\pgfqpoint{1.364870in}{2.311010in}}%
\pgfpathlineto{\pgfqpoint{1.357343in}{2.028605in}}%
\pgfusepath{stroke}%
\end{pgfscope}%
\begin{pgfscope}%
\pgfpathrectangle{\pgfqpoint{0.100000in}{0.212622in}}{\pgfqpoint{3.696000in}{3.696000in}}%
\pgfusepath{clip}%
\pgfsetrectcap%
\pgfsetroundjoin%
\pgfsetlinewidth{1.505625pt}%
\definecolor{currentstroke}{rgb}{1.000000,0.000000,0.000000}%
\pgfsetstrokecolor{currentstroke}%
\pgfsetdash{}{0pt}%
\pgfpathmoveto{\pgfqpoint{1.347874in}{2.317830in}}%
\pgfpathlineto{\pgfqpoint{1.342185in}{2.033083in}}%
\pgfusepath{stroke}%
\end{pgfscope}%
\begin{pgfscope}%
\pgfpathrectangle{\pgfqpoint{0.100000in}{0.212622in}}{\pgfqpoint{3.696000in}{3.696000in}}%
\pgfusepath{clip}%
\pgfsetrectcap%
\pgfsetroundjoin%
\pgfsetlinewidth{1.505625pt}%
\definecolor{currentstroke}{rgb}{1.000000,0.000000,0.000000}%
\pgfsetstrokecolor{currentstroke}%
\pgfsetdash{}{0pt}%
\pgfpathmoveto{\pgfqpoint{1.329717in}{2.324156in}}%
\pgfpathlineto{\pgfqpoint{1.327039in}{2.037559in}}%
\pgfusepath{stroke}%
\end{pgfscope}%
\begin{pgfscope}%
\pgfpathrectangle{\pgfqpoint{0.100000in}{0.212622in}}{\pgfqpoint{3.696000in}{3.696000in}}%
\pgfusepath{clip}%
\pgfsetrectcap%
\pgfsetroundjoin%
\pgfsetlinewidth{1.505625pt}%
\definecolor{currentstroke}{rgb}{1.000000,0.000000,0.000000}%
\pgfsetstrokecolor{currentstroke}%
\pgfsetdash{}{0pt}%
\pgfpathmoveto{\pgfqpoint{1.320368in}{2.326701in}}%
\pgfpathlineto{\pgfqpoint{1.327039in}{2.037559in}}%
\pgfusepath{stroke}%
\end{pgfscope}%
\begin{pgfscope}%
\pgfpathrectangle{\pgfqpoint{0.100000in}{0.212622in}}{\pgfqpoint{3.696000in}{3.696000in}}%
\pgfusepath{clip}%
\pgfsetrectcap%
\pgfsetroundjoin%
\pgfsetlinewidth{1.505625pt}%
\definecolor{currentstroke}{rgb}{1.000000,0.000000,0.000000}%
\pgfsetstrokecolor{currentstroke}%
\pgfsetdash{}{0pt}%
\pgfpathmoveto{\pgfqpoint{1.308687in}{2.333486in}}%
\pgfpathlineto{\pgfqpoint{1.311904in}{2.042030in}}%
\pgfusepath{stroke}%
\end{pgfscope}%
\begin{pgfscope}%
\pgfpathrectangle{\pgfqpoint{0.100000in}{0.212622in}}{\pgfqpoint{3.696000in}{3.696000in}}%
\pgfusepath{clip}%
\pgfsetrectcap%
\pgfsetroundjoin%
\pgfsetlinewidth{1.505625pt}%
\definecolor{currentstroke}{rgb}{1.000000,0.000000,0.000000}%
\pgfsetstrokecolor{currentstroke}%
\pgfsetdash{}{0pt}%
\pgfpathmoveto{\pgfqpoint{1.294607in}{2.338322in}}%
\pgfpathlineto{\pgfqpoint{1.296781in}{2.046499in}}%
\pgfusepath{stroke}%
\end{pgfscope}%
\begin{pgfscope}%
\pgfpathrectangle{\pgfqpoint{0.100000in}{0.212622in}}{\pgfqpoint{3.696000in}{3.696000in}}%
\pgfusepath{clip}%
\pgfsetrectcap%
\pgfsetroundjoin%
\pgfsetlinewidth{1.505625pt}%
\definecolor{currentstroke}{rgb}{1.000000,0.000000,0.000000}%
\pgfsetstrokecolor{currentstroke}%
\pgfsetdash{}{0pt}%
\pgfpathmoveto{\pgfqpoint{1.280356in}{2.341990in}}%
\pgfpathlineto{\pgfqpoint{1.281670in}{2.050964in}}%
\pgfusepath{stroke}%
\end{pgfscope}%
\begin{pgfscope}%
\pgfpathrectangle{\pgfqpoint{0.100000in}{0.212622in}}{\pgfqpoint{3.696000in}{3.696000in}}%
\pgfusepath{clip}%
\pgfsetrectcap%
\pgfsetroundjoin%
\pgfsetlinewidth{1.505625pt}%
\definecolor{currentstroke}{rgb}{1.000000,0.000000,0.000000}%
\pgfsetstrokecolor{currentstroke}%
\pgfsetdash{}{0pt}%
\pgfpathmoveto{\pgfqpoint{1.263887in}{2.350578in}}%
\pgfpathlineto{\pgfqpoint{1.266570in}{2.055425in}}%
\pgfusepath{stroke}%
\end{pgfscope}%
\begin{pgfscope}%
\pgfpathrectangle{\pgfqpoint{0.100000in}{0.212622in}}{\pgfqpoint{3.696000in}{3.696000in}}%
\pgfusepath{clip}%
\pgfsetrectcap%
\pgfsetroundjoin%
\pgfsetlinewidth{1.505625pt}%
\definecolor{currentstroke}{rgb}{1.000000,0.000000,0.000000}%
\pgfsetstrokecolor{currentstroke}%
\pgfsetdash{}{0pt}%
\pgfpathmoveto{\pgfqpoint{1.243662in}{2.357968in}}%
\pgfpathlineto{\pgfqpoint{1.251482in}{2.059883in}}%
\pgfusepath{stroke}%
\end{pgfscope}%
\begin{pgfscope}%
\pgfpathrectangle{\pgfqpoint{0.100000in}{0.212622in}}{\pgfqpoint{3.696000in}{3.696000in}}%
\pgfusepath{clip}%
\pgfsetrectcap%
\pgfsetroundjoin%
\pgfsetlinewidth{1.505625pt}%
\definecolor{currentstroke}{rgb}{1.000000,0.000000,0.000000}%
\pgfsetstrokecolor{currentstroke}%
\pgfsetdash{}{0pt}%
\pgfpathmoveto{\pgfqpoint{1.224339in}{2.362212in}}%
\pgfpathlineto{\pgfqpoint{1.236405in}{2.064338in}}%
\pgfusepath{stroke}%
\end{pgfscope}%
\begin{pgfscope}%
\pgfpathrectangle{\pgfqpoint{0.100000in}{0.212622in}}{\pgfqpoint{3.696000in}{3.696000in}}%
\pgfusepath{clip}%
\pgfsetrectcap%
\pgfsetroundjoin%
\pgfsetlinewidth{1.505625pt}%
\definecolor{currentstroke}{rgb}{1.000000,0.000000,0.000000}%
\pgfsetstrokecolor{currentstroke}%
\pgfsetdash{}{0pt}%
\pgfpathmoveto{\pgfqpoint{1.203270in}{2.371415in}}%
\pgfpathlineto{\pgfqpoint{1.206286in}{2.073236in}}%
\pgfusepath{stroke}%
\end{pgfscope}%
\begin{pgfscope}%
\pgfpathrectangle{\pgfqpoint{0.100000in}{0.212622in}}{\pgfqpoint{3.696000in}{3.696000in}}%
\pgfusepath{clip}%
\pgfsetrectcap%
\pgfsetroundjoin%
\pgfsetlinewidth{1.505625pt}%
\definecolor{currentstroke}{rgb}{1.000000,0.000000,0.000000}%
\pgfsetstrokecolor{currentstroke}%
\pgfsetdash{}{0pt}%
\pgfpathmoveto{\pgfqpoint{1.178934in}{2.379393in}}%
\pgfpathlineto{\pgfqpoint{1.191245in}{2.077681in}}%
\pgfusepath{stroke}%
\end{pgfscope}%
\begin{pgfscope}%
\pgfpathrectangle{\pgfqpoint{0.100000in}{0.212622in}}{\pgfqpoint{3.696000in}{3.696000in}}%
\pgfusepath{clip}%
\pgfsetrectcap%
\pgfsetroundjoin%
\pgfsetlinewidth{1.505625pt}%
\definecolor{currentstroke}{rgb}{1.000000,0.000000,0.000000}%
\pgfsetstrokecolor{currentstroke}%
\pgfsetdash{}{0pt}%
\pgfpathmoveto{\pgfqpoint{1.154808in}{2.386500in}}%
\pgfpathlineto{\pgfqpoint{1.161195in}{2.086559in}}%
\pgfusepath{stroke}%
\end{pgfscope}%
\begin{pgfscope}%
\pgfpathrectangle{\pgfqpoint{0.100000in}{0.212622in}}{\pgfqpoint{3.696000in}{3.696000in}}%
\pgfusepath{clip}%
\pgfsetrectcap%
\pgfsetroundjoin%
\pgfsetlinewidth{1.505625pt}%
\definecolor{currentstroke}{rgb}{1.000000,0.000000,0.000000}%
\pgfsetstrokecolor{currentstroke}%
\pgfsetdash{}{0pt}%
\pgfpathmoveto{\pgfqpoint{1.131086in}{2.395334in}}%
\pgfpathlineto{\pgfqpoint{1.146188in}{2.090993in}}%
\pgfusepath{stroke}%
\end{pgfscope}%
\begin{pgfscope}%
\pgfpathrectangle{\pgfqpoint{0.100000in}{0.212622in}}{\pgfqpoint{3.696000in}{3.696000in}}%
\pgfusepath{clip}%
\pgfsetrectcap%
\pgfsetroundjoin%
\pgfsetlinewidth{1.505625pt}%
\definecolor{currentstroke}{rgb}{1.000000,0.000000,0.000000}%
\pgfsetstrokecolor{currentstroke}%
\pgfsetdash{}{0pt}%
\pgfpathmoveto{\pgfqpoint{1.105529in}{2.404036in}}%
\pgfpathlineto{\pgfqpoint{1.116208in}{2.099851in}}%
\pgfusepath{stroke}%
\end{pgfscope}%
\begin{pgfscope}%
\pgfpathrectangle{\pgfqpoint{0.100000in}{0.212622in}}{\pgfqpoint{3.696000in}{3.696000in}}%
\pgfusepath{clip}%
\pgfsetrectcap%
\pgfsetroundjoin%
\pgfsetlinewidth{1.505625pt}%
\definecolor{currentstroke}{rgb}{1.000000,0.000000,0.000000}%
\pgfsetstrokecolor{currentstroke}%
\pgfsetdash{}{0pt}%
\pgfpathmoveto{\pgfqpoint{1.077496in}{2.414300in}}%
\pgfpathlineto{\pgfqpoint{1.086274in}{2.108696in}}%
\pgfusepath{stroke}%
\end{pgfscope}%
\begin{pgfscope}%
\pgfpathrectangle{\pgfqpoint{0.100000in}{0.212622in}}{\pgfqpoint{3.696000in}{3.696000in}}%
\pgfusepath{clip}%
\pgfsetrectcap%
\pgfsetroundjoin%
\pgfsetlinewidth{1.505625pt}%
\definecolor{currentstroke}{rgb}{1.000000,0.000000,0.000000}%
\pgfsetstrokecolor{currentstroke}%
\pgfsetdash{}{0pt}%
\pgfpathmoveto{\pgfqpoint{1.062844in}{2.420236in}}%
\pgfpathlineto{\pgfqpoint{1.071324in}{2.113113in}}%
\pgfusepath{stroke}%
\end{pgfscope}%
\begin{pgfscope}%
\pgfpathrectangle{\pgfqpoint{0.100000in}{0.212622in}}{\pgfqpoint{3.696000in}{3.696000in}}%
\pgfusepath{clip}%
\pgfsetrectcap%
\pgfsetroundjoin%
\pgfsetlinewidth{1.505625pt}%
\definecolor{currentstroke}{rgb}{1.000000,0.000000,0.000000}%
\pgfsetstrokecolor{currentstroke}%
\pgfsetdash{}{0pt}%
\pgfpathmoveto{\pgfqpoint{1.045522in}{2.425593in}}%
\pgfpathlineto{\pgfqpoint{1.056386in}{2.117527in}}%
\pgfusepath{stroke}%
\end{pgfscope}%
\begin{pgfscope}%
\pgfpathrectangle{\pgfqpoint{0.100000in}{0.212622in}}{\pgfqpoint{3.696000in}{3.696000in}}%
\pgfusepath{clip}%
\pgfsetrectcap%
\pgfsetroundjoin%
\pgfsetlinewidth{1.505625pt}%
\definecolor{currentstroke}{rgb}{1.000000,0.000000,0.000000}%
\pgfsetstrokecolor{currentstroke}%
\pgfsetdash{}{0pt}%
\pgfpathmoveto{\pgfqpoint{1.025707in}{2.433105in}}%
\pgfpathlineto{\pgfqpoint{1.041459in}{2.121937in}}%
\pgfusepath{stroke}%
\end{pgfscope}%
\begin{pgfscope}%
\pgfpathrectangle{\pgfqpoint{0.100000in}{0.212622in}}{\pgfqpoint{3.696000in}{3.696000in}}%
\pgfusepath{clip}%
\pgfsetrectcap%
\pgfsetroundjoin%
\pgfsetlinewidth{1.505625pt}%
\definecolor{currentstroke}{rgb}{1.000000,0.000000,0.000000}%
\pgfsetstrokecolor{currentstroke}%
\pgfsetdash{}{0pt}%
\pgfpathmoveto{\pgfqpoint{1.006933in}{2.439371in}}%
\pgfpathlineto{\pgfqpoint{1.026543in}{2.126344in}}%
\pgfusepath{stroke}%
\end{pgfscope}%
\begin{pgfscope}%
\pgfpathrectangle{\pgfqpoint{0.100000in}{0.212622in}}{\pgfqpoint{3.696000in}{3.696000in}}%
\pgfusepath{clip}%
\pgfsetrectcap%
\pgfsetroundjoin%
\pgfsetlinewidth{1.505625pt}%
\definecolor{currentstroke}{rgb}{1.000000,0.000000,0.000000}%
\pgfsetstrokecolor{currentstroke}%
\pgfsetdash{}{0pt}%
\pgfpathmoveto{\pgfqpoint{0.986320in}{2.445916in}}%
\pgfpathlineto{\pgfqpoint{0.996746in}{2.135148in}}%
\pgfusepath{stroke}%
\end{pgfscope}%
\begin{pgfscope}%
\pgfpathrectangle{\pgfqpoint{0.100000in}{0.212622in}}{\pgfqpoint{3.696000in}{3.696000in}}%
\pgfusepath{clip}%
\pgfsetrectcap%
\pgfsetroundjoin%
\pgfsetlinewidth{1.505625pt}%
\definecolor{currentstroke}{rgb}{1.000000,0.000000,0.000000}%
\pgfsetstrokecolor{currentstroke}%
\pgfsetdash{}{0pt}%
\pgfpathmoveto{\pgfqpoint{0.963944in}{2.457928in}}%
\pgfpathlineto{\pgfqpoint{0.981865in}{2.139545in}}%
\pgfusepath{stroke}%
\end{pgfscope}%
\begin{pgfscope}%
\pgfpathrectangle{\pgfqpoint{0.100000in}{0.212622in}}{\pgfqpoint{3.696000in}{3.696000in}}%
\pgfusepath{clip}%
\pgfsetrectcap%
\pgfsetroundjoin%
\pgfsetlinewidth{1.505625pt}%
\definecolor{currentstroke}{rgb}{1.000000,0.000000,0.000000}%
\pgfsetstrokecolor{currentstroke}%
\pgfsetdash{}{0pt}%
\pgfpathmoveto{\pgfqpoint{0.939396in}{2.466114in}}%
\pgfpathlineto{\pgfqpoint{0.952136in}{2.148329in}}%
\pgfusepath{stroke}%
\end{pgfscope}%
\begin{pgfscope}%
\pgfpathrectangle{\pgfqpoint{0.100000in}{0.212622in}}{\pgfqpoint{3.696000in}{3.696000in}}%
\pgfusepath{clip}%
\pgfsetrectcap%
\pgfsetroundjoin%
\pgfsetlinewidth{1.505625pt}%
\definecolor{currentstroke}{rgb}{1.000000,0.000000,0.000000}%
\pgfsetstrokecolor{currentstroke}%
\pgfsetdash{}{0pt}%
\pgfpathmoveto{\pgfqpoint{0.914068in}{2.474078in}}%
\pgfpathlineto{\pgfqpoint{0.937289in}{2.152715in}}%
\pgfusepath{stroke}%
\end{pgfscope}%
\begin{pgfscope}%
\pgfpathrectangle{\pgfqpoint{0.100000in}{0.212622in}}{\pgfqpoint{3.696000in}{3.696000in}}%
\pgfusepath{clip}%
\pgfsetrectcap%
\pgfsetroundjoin%
\pgfsetlinewidth{1.505625pt}%
\definecolor{currentstroke}{rgb}{1.000000,0.000000,0.000000}%
\pgfsetstrokecolor{currentstroke}%
\pgfsetdash{}{0pt}%
\pgfpathmoveto{\pgfqpoint{0.891290in}{2.488439in}}%
\pgfpathlineto{\pgfqpoint{0.907628in}{2.161479in}}%
\pgfusepath{stroke}%
\end{pgfscope}%
\begin{pgfscope}%
\pgfpathrectangle{\pgfqpoint{0.100000in}{0.212622in}}{\pgfqpoint{3.696000in}{3.696000in}}%
\pgfusepath{clip}%
\pgfsetrectcap%
\pgfsetroundjoin%
\pgfsetlinewidth{1.505625pt}%
\definecolor{currentstroke}{rgb}{1.000000,0.000000,0.000000}%
\pgfsetstrokecolor{currentstroke}%
\pgfsetdash{}{0pt}%
\pgfpathmoveto{\pgfqpoint{0.860373in}{2.500343in}}%
\pgfpathlineto{\pgfqpoint{0.878013in}{2.170229in}}%
\pgfusepath{stroke}%
\end{pgfscope}%
\begin{pgfscope}%
\pgfpathrectangle{\pgfqpoint{0.100000in}{0.212622in}}{\pgfqpoint{3.696000in}{3.696000in}}%
\pgfusepath{clip}%
\pgfsetrectcap%
\pgfsetroundjoin%
\pgfsetlinewidth{1.505625pt}%
\definecolor{currentstroke}{rgb}{1.000000,0.000000,0.000000}%
\pgfsetstrokecolor{currentstroke}%
\pgfsetdash{}{0pt}%
\pgfpathmoveto{\pgfqpoint{0.829935in}{2.510363in}}%
\pgfpathlineto{\pgfqpoint{0.848442in}{2.178966in}}%
\pgfusepath{stroke}%
\end{pgfscope}%
\begin{pgfscope}%
\pgfpathrectangle{\pgfqpoint{0.100000in}{0.212622in}}{\pgfqpoint{3.696000in}{3.696000in}}%
\pgfusepath{clip}%
\pgfsetrectcap%
\pgfsetroundjoin%
\pgfsetlinewidth{1.505625pt}%
\definecolor{currentstroke}{rgb}{1.000000,0.000000,0.000000}%
\pgfsetstrokecolor{currentstroke}%
\pgfsetdash{}{0pt}%
\pgfpathmoveto{\pgfqpoint{0.814351in}{2.516540in}}%
\pgfpathlineto{\pgfqpoint{0.833674in}{2.183330in}}%
\pgfusepath{stroke}%
\end{pgfscope}%
\begin{pgfscope}%
\pgfpathrectangle{\pgfqpoint{0.100000in}{0.212622in}}{\pgfqpoint{3.696000in}{3.696000in}}%
\pgfusepath{clip}%
\pgfsetrectcap%
\pgfsetroundjoin%
\pgfsetlinewidth{1.505625pt}%
\definecolor{currentstroke}{rgb}{1.000000,0.000000,0.000000}%
\pgfsetstrokecolor{currentstroke}%
\pgfsetdash{}{0pt}%
\pgfpathmoveto{\pgfqpoint{0.794943in}{2.522725in}}%
\pgfpathlineto{\pgfqpoint{0.818917in}{2.187690in}}%
\pgfusepath{stroke}%
\end{pgfscope}%
\begin{pgfscope}%
\pgfpathrectangle{\pgfqpoint{0.100000in}{0.212622in}}{\pgfqpoint{3.696000in}{3.696000in}}%
\pgfusepath{clip}%
\pgfsetrectcap%
\pgfsetroundjoin%
\pgfsetlinewidth{1.505625pt}%
\definecolor{currentstroke}{rgb}{1.000000,0.000000,0.000000}%
\pgfsetstrokecolor{currentstroke}%
\pgfsetdash{}{0pt}%
\pgfpathmoveto{\pgfqpoint{0.776540in}{2.529397in}}%
\pgfpathlineto{\pgfqpoint{0.804171in}{2.192047in}}%
\pgfusepath{stroke}%
\end{pgfscope}%
\begin{pgfscope}%
\pgfpathrectangle{\pgfqpoint{0.100000in}{0.212622in}}{\pgfqpoint{3.696000in}{3.696000in}}%
\pgfusepath{clip}%
\pgfsetrectcap%
\pgfsetroundjoin%
\pgfsetlinewidth{1.505625pt}%
\definecolor{currentstroke}{rgb}{1.000000,0.000000,0.000000}%
\pgfsetstrokecolor{currentstroke}%
\pgfsetdash{}{0pt}%
\pgfpathmoveto{\pgfqpoint{0.766236in}{2.531938in}}%
\pgfpathlineto{\pgfqpoint{0.789436in}{2.196400in}}%
\pgfusepath{stroke}%
\end{pgfscope}%
\begin{pgfscope}%
\pgfpathrectangle{\pgfqpoint{0.100000in}{0.212622in}}{\pgfqpoint{3.696000in}{3.696000in}}%
\pgfusepath{clip}%
\pgfsetrectcap%
\pgfsetroundjoin%
\pgfsetlinewidth{1.505625pt}%
\definecolor{currentstroke}{rgb}{1.000000,0.000000,0.000000}%
\pgfsetstrokecolor{currentstroke}%
\pgfsetdash{}{0pt}%
\pgfpathmoveto{\pgfqpoint{0.753596in}{2.535344in}}%
\pgfpathlineto{\pgfqpoint{0.774713in}{2.200751in}}%
\pgfusepath{stroke}%
\end{pgfscope}%
\begin{pgfscope}%
\pgfpathrectangle{\pgfqpoint{0.100000in}{0.212622in}}{\pgfqpoint{3.696000in}{3.696000in}}%
\pgfusepath{clip}%
\pgfsetrectcap%
\pgfsetroundjoin%
\pgfsetlinewidth{1.505625pt}%
\definecolor{currentstroke}{rgb}{1.000000,0.000000,0.000000}%
\pgfsetstrokecolor{currentstroke}%
\pgfsetdash{}{0pt}%
\pgfpathmoveto{\pgfqpoint{0.739097in}{2.540612in}}%
\pgfpathlineto{\pgfqpoint{0.760001in}{2.205098in}}%
\pgfusepath{stroke}%
\end{pgfscope}%
\begin{pgfscope}%
\pgfpathrectangle{\pgfqpoint{0.100000in}{0.212622in}}{\pgfqpoint{3.696000in}{3.696000in}}%
\pgfusepath{clip}%
\pgfsetrectcap%
\pgfsetroundjoin%
\pgfsetlinewidth{1.505625pt}%
\definecolor{currentstroke}{rgb}{1.000000,0.000000,0.000000}%
\pgfsetstrokecolor{currentstroke}%
\pgfsetdash{}{0pt}%
\pgfpathmoveto{\pgfqpoint{0.724952in}{2.545852in}}%
\pgfpathlineto{\pgfqpoint{0.760001in}{2.205098in}}%
\pgfusepath{stroke}%
\end{pgfscope}%
\begin{pgfscope}%
\pgfpathrectangle{\pgfqpoint{0.100000in}{0.212622in}}{\pgfqpoint{3.696000in}{3.696000in}}%
\pgfusepath{clip}%
\pgfsetrectcap%
\pgfsetroundjoin%
\pgfsetlinewidth{1.505625pt}%
\definecolor{currentstroke}{rgb}{1.000000,0.000000,0.000000}%
\pgfsetstrokecolor{currentstroke}%
\pgfsetdash{}{0pt}%
\pgfpathmoveto{\pgfqpoint{0.710052in}{2.549279in}}%
\pgfpathlineto{\pgfqpoint{0.760001in}{2.205098in}}%
\pgfusepath{stroke}%
\end{pgfscope}%
\begin{pgfscope}%
\pgfpathrectangle{\pgfqpoint{0.100000in}{0.212622in}}{\pgfqpoint{3.696000in}{3.696000in}}%
\pgfusepath{clip}%
\pgfsetrectcap%
\pgfsetroundjoin%
\pgfsetlinewidth{1.505625pt}%
\definecolor{currentstroke}{rgb}{1.000000,0.000000,0.000000}%
\pgfsetstrokecolor{currentstroke}%
\pgfsetdash{}{0pt}%
\pgfpathmoveto{\pgfqpoint{0.701619in}{2.551868in}}%
\pgfpathlineto{\pgfqpoint{0.760001in}{2.205098in}}%
\pgfusepath{stroke}%
\end{pgfscope}%
\begin{pgfscope}%
\pgfpathrectangle{\pgfqpoint{0.100000in}{0.212622in}}{\pgfqpoint{3.696000in}{3.696000in}}%
\pgfusepath{clip}%
\pgfsetrectcap%
\pgfsetroundjoin%
\pgfsetlinewidth{1.505625pt}%
\definecolor{currentstroke}{rgb}{1.000000,0.000000,0.000000}%
\pgfsetstrokecolor{currentstroke}%
\pgfsetdash{}{0pt}%
\pgfpathmoveto{\pgfqpoint{0.692602in}{2.555173in}}%
\pgfpathlineto{\pgfqpoint{0.760001in}{2.205098in}}%
\pgfusepath{stroke}%
\end{pgfscope}%
\begin{pgfscope}%
\pgfpathrectangle{\pgfqpoint{0.100000in}{0.212622in}}{\pgfqpoint{3.696000in}{3.696000in}}%
\pgfusepath{clip}%
\pgfsetrectcap%
\pgfsetroundjoin%
\pgfsetlinewidth{1.505625pt}%
\definecolor{currentstroke}{rgb}{1.000000,0.000000,0.000000}%
\pgfsetstrokecolor{currentstroke}%
\pgfsetdash{}{0pt}%
\pgfpathmoveto{\pgfqpoint{0.681152in}{2.558665in}}%
\pgfpathlineto{\pgfqpoint{0.760001in}{2.205098in}}%
\pgfusepath{stroke}%
\end{pgfscope}%
\begin{pgfscope}%
\pgfpathrectangle{\pgfqpoint{0.100000in}{0.212622in}}{\pgfqpoint{3.696000in}{3.696000in}}%
\pgfusepath{clip}%
\pgfsetrectcap%
\pgfsetroundjoin%
\pgfsetlinewidth{1.505625pt}%
\definecolor{currentstroke}{rgb}{1.000000,0.000000,0.000000}%
\pgfsetstrokecolor{currentstroke}%
\pgfsetdash{}{0pt}%
\pgfpathmoveto{\pgfqpoint{0.667970in}{2.564086in}}%
\pgfpathlineto{\pgfqpoint{0.760001in}{2.205098in}}%
\pgfusepath{stroke}%
\end{pgfscope}%
\begin{pgfscope}%
\pgfpathrectangle{\pgfqpoint{0.100000in}{0.212622in}}{\pgfqpoint{3.696000in}{3.696000in}}%
\pgfusepath{clip}%
\pgfsetrectcap%
\pgfsetroundjoin%
\pgfsetlinewidth{1.505625pt}%
\definecolor{currentstroke}{rgb}{1.000000,0.000000,0.000000}%
\pgfsetstrokecolor{currentstroke}%
\pgfsetdash{}{0pt}%
\pgfpathmoveto{\pgfqpoint{0.661011in}{2.566662in}}%
\pgfpathlineto{\pgfqpoint{0.760001in}{2.205098in}}%
\pgfusepath{stroke}%
\end{pgfscope}%
\begin{pgfscope}%
\pgfpathrectangle{\pgfqpoint{0.100000in}{0.212622in}}{\pgfqpoint{3.696000in}{3.696000in}}%
\pgfusepath{clip}%
\pgfsetrectcap%
\pgfsetroundjoin%
\pgfsetlinewidth{1.505625pt}%
\definecolor{currentstroke}{rgb}{1.000000,0.000000,0.000000}%
\pgfsetstrokecolor{currentstroke}%
\pgfsetdash{}{0pt}%
\pgfpathmoveto{\pgfqpoint{0.651644in}{2.569157in}}%
\pgfpathlineto{\pgfqpoint{0.760001in}{2.205098in}}%
\pgfusepath{stroke}%
\end{pgfscope}%
\begin{pgfscope}%
\pgfpathrectangle{\pgfqpoint{0.100000in}{0.212622in}}{\pgfqpoint{3.696000in}{3.696000in}}%
\pgfusepath{clip}%
\pgfsetrectcap%
\pgfsetroundjoin%
\pgfsetlinewidth{1.505625pt}%
\definecolor{currentstroke}{rgb}{1.000000,0.000000,0.000000}%
\pgfsetstrokecolor{currentstroke}%
\pgfsetdash{}{0pt}%
\pgfpathmoveto{\pgfqpoint{0.640098in}{2.573964in}}%
\pgfpathlineto{\pgfqpoint{0.760001in}{2.205098in}}%
\pgfusepath{stroke}%
\end{pgfscope}%
\begin{pgfscope}%
\pgfpathrectangle{\pgfqpoint{0.100000in}{0.212622in}}{\pgfqpoint{3.696000in}{3.696000in}}%
\pgfusepath{clip}%
\pgfsetrectcap%
\pgfsetroundjoin%
\pgfsetlinewidth{1.505625pt}%
\definecolor{currentstroke}{rgb}{1.000000,0.000000,0.000000}%
\pgfsetstrokecolor{currentstroke}%
\pgfsetdash{}{0pt}%
\pgfpathmoveto{\pgfqpoint{0.633985in}{2.576123in}}%
\pgfpathlineto{\pgfqpoint{0.760001in}{2.205098in}}%
\pgfusepath{stroke}%
\end{pgfscope}%
\begin{pgfscope}%
\pgfpathrectangle{\pgfqpoint{0.100000in}{0.212622in}}{\pgfqpoint{3.696000in}{3.696000in}}%
\pgfusepath{clip}%
\pgfsetrectcap%
\pgfsetroundjoin%
\pgfsetlinewidth{1.505625pt}%
\definecolor{currentstroke}{rgb}{1.000000,0.000000,0.000000}%
\pgfsetstrokecolor{currentstroke}%
\pgfsetdash{}{0pt}%
\pgfpathmoveto{\pgfqpoint{0.626515in}{2.577906in}}%
\pgfpathlineto{\pgfqpoint{0.760001in}{2.205098in}}%
\pgfusepath{stroke}%
\end{pgfscope}%
\begin{pgfscope}%
\pgfpathrectangle{\pgfqpoint{0.100000in}{0.212622in}}{\pgfqpoint{3.696000in}{3.696000in}}%
\pgfusepath{clip}%
\pgfsetrectcap%
\pgfsetroundjoin%
\pgfsetlinewidth{1.505625pt}%
\definecolor{currentstroke}{rgb}{1.000000,0.000000,0.000000}%
\pgfsetstrokecolor{currentstroke}%
\pgfsetdash{}{0pt}%
\pgfpathmoveto{\pgfqpoint{0.617214in}{2.583363in}}%
\pgfpathlineto{\pgfqpoint{0.760001in}{2.205098in}}%
\pgfusepath{stroke}%
\end{pgfscope}%
\begin{pgfscope}%
\pgfpathrectangle{\pgfqpoint{0.100000in}{0.212622in}}{\pgfqpoint{3.696000in}{3.696000in}}%
\pgfusepath{clip}%
\pgfsetrectcap%
\pgfsetroundjoin%
\pgfsetlinewidth{1.505625pt}%
\definecolor{currentstroke}{rgb}{1.000000,0.000000,0.000000}%
\pgfsetstrokecolor{currentstroke}%
\pgfsetdash{}{0pt}%
\pgfpathmoveto{\pgfqpoint{0.606017in}{2.587141in}}%
\pgfpathlineto{\pgfqpoint{0.760001in}{2.205098in}}%
\pgfusepath{stroke}%
\end{pgfscope}%
\begin{pgfscope}%
\pgfpathrectangle{\pgfqpoint{0.100000in}{0.212622in}}{\pgfqpoint{3.696000in}{3.696000in}}%
\pgfusepath{clip}%
\pgfsetrectcap%
\pgfsetroundjoin%
\pgfsetlinewidth{1.505625pt}%
\definecolor{currentstroke}{rgb}{1.000000,0.000000,0.000000}%
\pgfsetstrokecolor{currentstroke}%
\pgfsetdash{}{0pt}%
\pgfpathmoveto{\pgfqpoint{0.594556in}{2.590432in}}%
\pgfpathlineto{\pgfqpoint{0.760001in}{2.205098in}}%
\pgfusepath{stroke}%
\end{pgfscope}%
\begin{pgfscope}%
\pgfpathrectangle{\pgfqpoint{0.100000in}{0.212622in}}{\pgfqpoint{3.696000in}{3.696000in}}%
\pgfusepath{clip}%
\pgfsetrectcap%
\pgfsetroundjoin%
\pgfsetlinewidth{1.505625pt}%
\definecolor{currentstroke}{rgb}{1.000000,0.000000,0.000000}%
\pgfsetstrokecolor{currentstroke}%
\pgfsetdash{}{0pt}%
\pgfpathmoveto{\pgfqpoint{0.582912in}{2.594642in}}%
\pgfpathlineto{\pgfqpoint{0.760001in}{2.205098in}}%
\pgfusepath{stroke}%
\end{pgfscope}%
\begin{pgfscope}%
\pgfpathrectangle{\pgfqpoint{0.100000in}{0.212622in}}{\pgfqpoint{3.696000in}{3.696000in}}%
\pgfusepath{clip}%
\pgfsetrectcap%
\pgfsetroundjoin%
\pgfsetlinewidth{1.505625pt}%
\definecolor{currentstroke}{rgb}{1.000000,0.000000,0.000000}%
\pgfsetstrokecolor{currentstroke}%
\pgfsetdash{}{0pt}%
\pgfpathmoveto{\pgfqpoint{0.569641in}{2.598460in}}%
\pgfpathlineto{\pgfqpoint{0.760001in}{2.205098in}}%
\pgfusepath{stroke}%
\end{pgfscope}%
\begin{pgfscope}%
\pgfpathrectangle{\pgfqpoint{0.100000in}{0.212622in}}{\pgfqpoint{3.696000in}{3.696000in}}%
\pgfusepath{clip}%
\pgfsetrectcap%
\pgfsetroundjoin%
\pgfsetlinewidth{1.505625pt}%
\definecolor{currentstroke}{rgb}{1.000000,0.000000,0.000000}%
\pgfsetstrokecolor{currentstroke}%
\pgfsetdash{}{0pt}%
\pgfpathmoveto{\pgfqpoint{0.554858in}{2.603220in}}%
\pgfpathlineto{\pgfqpoint{0.760001in}{2.205098in}}%
\pgfusepath{stroke}%
\end{pgfscope}%
\begin{pgfscope}%
\pgfpathrectangle{\pgfqpoint{0.100000in}{0.212622in}}{\pgfqpoint{3.696000in}{3.696000in}}%
\pgfusepath{clip}%
\pgfsetrectcap%
\pgfsetroundjoin%
\pgfsetlinewidth{1.505625pt}%
\definecolor{currentstroke}{rgb}{1.000000,0.000000,0.000000}%
\pgfsetstrokecolor{currentstroke}%
\pgfsetdash{}{0pt}%
\pgfpathmoveto{\pgfqpoint{0.547064in}{2.606075in}}%
\pgfpathlineto{\pgfqpoint{0.760001in}{2.205098in}}%
\pgfusepath{stroke}%
\end{pgfscope}%
\begin{pgfscope}%
\pgfpathrectangle{\pgfqpoint{0.100000in}{0.212622in}}{\pgfqpoint{3.696000in}{3.696000in}}%
\pgfusepath{clip}%
\pgfsetrectcap%
\pgfsetroundjoin%
\pgfsetlinewidth{1.505625pt}%
\definecolor{currentstroke}{rgb}{1.000000,0.000000,0.000000}%
\pgfsetstrokecolor{currentstroke}%
\pgfsetdash{}{0pt}%
\pgfpathmoveto{\pgfqpoint{0.542513in}{2.607768in}}%
\pgfpathlineto{\pgfqpoint{0.760001in}{2.205098in}}%
\pgfusepath{stroke}%
\end{pgfscope}%
\begin{pgfscope}%
\pgfpathrectangle{\pgfqpoint{0.100000in}{0.212622in}}{\pgfqpoint{3.696000in}{3.696000in}}%
\pgfusepath{clip}%
\pgfsetrectcap%
\pgfsetroundjoin%
\pgfsetlinewidth{1.505625pt}%
\definecolor{currentstroke}{rgb}{1.000000,0.000000,0.000000}%
\pgfsetstrokecolor{currentstroke}%
\pgfsetdash{}{0pt}%
\pgfpathmoveto{\pgfqpoint{0.540140in}{2.608580in}}%
\pgfpathlineto{\pgfqpoint{0.760001in}{2.205098in}}%
\pgfusepath{stroke}%
\end{pgfscope}%
\begin{pgfscope}%
\pgfpathrectangle{\pgfqpoint{0.100000in}{0.212622in}}{\pgfqpoint{3.696000in}{3.696000in}}%
\pgfusepath{clip}%
\pgfsetrectcap%
\pgfsetroundjoin%
\pgfsetlinewidth{1.505625pt}%
\definecolor{currentstroke}{rgb}{1.000000,0.000000,0.000000}%
\pgfsetstrokecolor{currentstroke}%
\pgfsetdash{}{0pt}%
\pgfpathmoveto{\pgfqpoint{0.538843in}{2.609128in}}%
\pgfpathlineto{\pgfqpoint{0.760001in}{2.205098in}}%
\pgfusepath{stroke}%
\end{pgfscope}%
\begin{pgfscope}%
\pgfpathrectangle{\pgfqpoint{0.100000in}{0.212622in}}{\pgfqpoint{3.696000in}{3.696000in}}%
\pgfusepath{clip}%
\pgfsetrectcap%
\pgfsetroundjoin%
\pgfsetlinewidth{1.505625pt}%
\definecolor{currentstroke}{rgb}{1.000000,0.000000,0.000000}%
\pgfsetstrokecolor{currentstroke}%
\pgfsetdash{}{0pt}%
\pgfpathmoveto{\pgfqpoint{0.536311in}{2.610226in}}%
\pgfpathlineto{\pgfqpoint{0.760001in}{2.205098in}}%
\pgfusepath{stroke}%
\end{pgfscope}%
\begin{pgfscope}%
\pgfpathrectangle{\pgfqpoint{0.100000in}{0.212622in}}{\pgfqpoint{3.696000in}{3.696000in}}%
\pgfusepath{clip}%
\pgfsetrectcap%
\pgfsetroundjoin%
\pgfsetlinewidth{1.505625pt}%
\definecolor{currentstroke}{rgb}{1.000000,0.000000,0.000000}%
\pgfsetstrokecolor{currentstroke}%
\pgfsetdash{}{0pt}%
\pgfpathmoveto{\pgfqpoint{0.535145in}{2.610508in}}%
\pgfpathlineto{\pgfqpoint{0.760001in}{2.205098in}}%
\pgfusepath{stroke}%
\end{pgfscope}%
\begin{pgfscope}%
\pgfpathrectangle{\pgfqpoint{0.100000in}{0.212622in}}{\pgfqpoint{3.696000in}{3.696000in}}%
\pgfusepath{clip}%
\pgfsetrectcap%
\pgfsetroundjoin%
\pgfsetlinewidth{1.505625pt}%
\definecolor{currentstroke}{rgb}{1.000000,0.000000,0.000000}%
\pgfsetstrokecolor{currentstroke}%
\pgfsetdash{}{0pt}%
\pgfpathmoveto{\pgfqpoint{0.533877in}{2.610747in}}%
\pgfpathlineto{\pgfqpoint{0.760001in}{2.205098in}}%
\pgfusepath{stroke}%
\end{pgfscope}%
\begin{pgfscope}%
\pgfpathrectangle{\pgfqpoint{0.100000in}{0.212622in}}{\pgfqpoint{3.696000in}{3.696000in}}%
\pgfusepath{clip}%
\pgfsetrectcap%
\pgfsetroundjoin%
\pgfsetlinewidth{1.505625pt}%
\definecolor{currentstroke}{rgb}{1.000000,0.000000,0.000000}%
\pgfsetstrokecolor{currentstroke}%
\pgfsetdash{}{0pt}%
\pgfpathmoveto{\pgfqpoint{0.533088in}{2.611070in}}%
\pgfpathlineto{\pgfqpoint{0.760001in}{2.205098in}}%
\pgfusepath{stroke}%
\end{pgfscope}%
\begin{pgfscope}%
\pgfpathrectangle{\pgfqpoint{0.100000in}{0.212622in}}{\pgfqpoint{3.696000in}{3.696000in}}%
\pgfusepath{clip}%
\pgfsetrectcap%
\pgfsetroundjoin%
\pgfsetlinewidth{1.505625pt}%
\definecolor{currentstroke}{rgb}{1.000000,0.000000,0.000000}%
\pgfsetstrokecolor{currentstroke}%
\pgfsetdash{}{0pt}%
\pgfpathmoveto{\pgfqpoint{0.531780in}{2.611587in}}%
\pgfpathlineto{\pgfqpoint{0.760001in}{2.205098in}}%
\pgfusepath{stroke}%
\end{pgfscope}%
\begin{pgfscope}%
\pgfpathrectangle{\pgfqpoint{0.100000in}{0.212622in}}{\pgfqpoint{3.696000in}{3.696000in}}%
\pgfusepath{clip}%
\pgfsetrectcap%
\pgfsetroundjoin%
\pgfsetlinewidth{1.505625pt}%
\definecolor{currentstroke}{rgb}{1.000000,0.000000,0.000000}%
\pgfsetstrokecolor{currentstroke}%
\pgfsetdash{}{0pt}%
\pgfpathmoveto{\pgfqpoint{0.530113in}{2.611926in}}%
\pgfpathlineto{\pgfqpoint{0.760001in}{2.205098in}}%
\pgfusepath{stroke}%
\end{pgfscope}%
\begin{pgfscope}%
\pgfpathrectangle{\pgfqpoint{0.100000in}{0.212622in}}{\pgfqpoint{3.696000in}{3.696000in}}%
\pgfusepath{clip}%
\pgfsetbuttcap%
\pgfsetroundjoin%
\definecolor{currentfill}{rgb}{0.121569,0.466667,0.705882}%
\pgfsetfillcolor{currentfill}%
\pgfsetfillopacity{0.300000}%
\pgfsetlinewidth{1.003750pt}%
\definecolor{currentstroke}{rgb}{0.121569,0.466667,0.705882}%
\pgfsetstrokecolor{currentstroke}%
\pgfsetstrokeopacity{0.300000}%
\pgfsetdash{}{0pt}%
\pgfpathmoveto{\pgfqpoint{1.680525in}{2.094323in}}%
\pgfpathcurveto{\pgfqpoint{1.688762in}{2.094323in}}{\pgfqpoint{1.696662in}{2.097595in}}{\pgfqpoint{1.702486in}{2.103419in}}%
\pgfpathcurveto{\pgfqpoint{1.708310in}{2.109243in}}{\pgfqpoint{1.711582in}{2.117143in}}{\pgfqpoint{1.711582in}{2.125379in}}%
\pgfpathcurveto{\pgfqpoint{1.711582in}{2.133616in}}{\pgfqpoint{1.708310in}{2.141516in}}{\pgfqpoint{1.702486in}{2.147340in}}%
\pgfpathcurveto{\pgfqpoint{1.696662in}{2.153163in}}{\pgfqpoint{1.688762in}{2.156436in}}{\pgfqpoint{1.680525in}{2.156436in}}%
\pgfpathcurveto{\pgfqpoint{1.672289in}{2.156436in}}{\pgfqpoint{1.664389in}{2.153163in}}{\pgfqpoint{1.658565in}{2.147340in}}%
\pgfpathcurveto{\pgfqpoint{1.652741in}{2.141516in}}{\pgfqpoint{1.649469in}{2.133616in}}{\pgfqpoint{1.649469in}{2.125379in}}%
\pgfpathcurveto{\pgfqpoint{1.649469in}{2.117143in}}{\pgfqpoint{1.652741in}{2.109243in}}{\pgfqpoint{1.658565in}{2.103419in}}%
\pgfpathcurveto{\pgfqpoint{1.664389in}{2.097595in}}{\pgfqpoint{1.672289in}{2.094323in}}{\pgfqpoint{1.680525in}{2.094323in}}%
\pgfpathclose%
\pgfusepath{stroke,fill}%
\end{pgfscope}%
\begin{pgfscope}%
\pgfpathrectangle{\pgfqpoint{0.100000in}{0.212622in}}{\pgfqpoint{3.696000in}{3.696000in}}%
\pgfusepath{clip}%
\pgfsetbuttcap%
\pgfsetroundjoin%
\definecolor{currentfill}{rgb}{0.121569,0.466667,0.705882}%
\pgfsetfillcolor{currentfill}%
\pgfsetfillopacity{0.300000}%
\pgfsetlinewidth{1.003750pt}%
\definecolor{currentstroke}{rgb}{0.121569,0.466667,0.705882}%
\pgfsetstrokecolor{currentstroke}%
\pgfsetstrokeopacity{0.300000}%
\pgfsetdash{}{0pt}%
\pgfpathmoveto{\pgfqpoint{1.680974in}{2.094258in}}%
\pgfpathcurveto{\pgfqpoint{1.689210in}{2.094258in}}{\pgfqpoint{1.697110in}{2.097530in}}{\pgfqpoint{1.702934in}{2.103354in}}%
\pgfpathcurveto{\pgfqpoint{1.708758in}{2.109178in}}{\pgfqpoint{1.712030in}{2.117078in}}{\pgfqpoint{1.712030in}{2.125314in}}%
\pgfpathcurveto{\pgfqpoint{1.712030in}{2.133551in}}{\pgfqpoint{1.708758in}{2.141451in}}{\pgfqpoint{1.702934in}{2.147275in}}%
\pgfpathcurveto{\pgfqpoint{1.697110in}{2.153098in}}{\pgfqpoint{1.689210in}{2.156371in}}{\pgfqpoint{1.680974in}{2.156371in}}%
\pgfpathcurveto{\pgfqpoint{1.672738in}{2.156371in}}{\pgfqpoint{1.664837in}{2.153098in}}{\pgfqpoint{1.659014in}{2.147275in}}%
\pgfpathcurveto{\pgfqpoint{1.653190in}{2.141451in}}{\pgfqpoint{1.649917in}{2.133551in}}{\pgfqpoint{1.649917in}{2.125314in}}%
\pgfpathcurveto{\pgfqpoint{1.649917in}{2.117078in}}{\pgfqpoint{1.653190in}{2.109178in}}{\pgfqpoint{1.659014in}{2.103354in}}%
\pgfpathcurveto{\pgfqpoint{1.664837in}{2.097530in}}{\pgfqpoint{1.672738in}{2.094258in}}{\pgfqpoint{1.680974in}{2.094258in}}%
\pgfpathclose%
\pgfusepath{stroke,fill}%
\end{pgfscope}%
\begin{pgfscope}%
\pgfpathrectangle{\pgfqpoint{0.100000in}{0.212622in}}{\pgfqpoint{3.696000in}{3.696000in}}%
\pgfusepath{clip}%
\pgfsetbuttcap%
\pgfsetroundjoin%
\definecolor{currentfill}{rgb}{0.121569,0.466667,0.705882}%
\pgfsetfillcolor{currentfill}%
\pgfsetfillopacity{0.300010}%
\pgfsetlinewidth{1.003750pt}%
\definecolor{currentstroke}{rgb}{0.121569,0.466667,0.705882}%
\pgfsetstrokecolor{currentstroke}%
\pgfsetstrokeopacity{0.300010}%
\pgfsetdash{}{0pt}%
\pgfpathmoveto{\pgfqpoint{1.679706in}{2.094407in}}%
\pgfpathcurveto{\pgfqpoint{1.687943in}{2.094407in}}{\pgfqpoint{1.695843in}{2.097680in}}{\pgfqpoint{1.701667in}{2.103503in}}%
\pgfpathcurveto{\pgfqpoint{1.707491in}{2.109327in}}{\pgfqpoint{1.710763in}{2.117227in}}{\pgfqpoint{1.710763in}{2.125464in}}%
\pgfpathcurveto{\pgfqpoint{1.710763in}{2.133700in}}{\pgfqpoint{1.707491in}{2.141600in}}{\pgfqpoint{1.701667in}{2.147424in}}%
\pgfpathcurveto{\pgfqpoint{1.695843in}{2.153248in}}{\pgfqpoint{1.687943in}{2.156520in}}{\pgfqpoint{1.679706in}{2.156520in}}%
\pgfpathcurveto{\pgfqpoint{1.671470in}{2.156520in}}{\pgfqpoint{1.663570in}{2.153248in}}{\pgfqpoint{1.657746in}{2.147424in}}%
\pgfpathcurveto{\pgfqpoint{1.651922in}{2.141600in}}{\pgfqpoint{1.648650in}{2.133700in}}{\pgfqpoint{1.648650in}{2.125464in}}%
\pgfpathcurveto{\pgfqpoint{1.648650in}{2.117227in}}{\pgfqpoint{1.651922in}{2.109327in}}{\pgfqpoint{1.657746in}{2.103503in}}%
\pgfpathcurveto{\pgfqpoint{1.663570in}{2.097680in}}{\pgfqpoint{1.671470in}{2.094407in}}{\pgfqpoint{1.679706in}{2.094407in}}%
\pgfpathclose%
\pgfusepath{stroke,fill}%
\end{pgfscope}%
\begin{pgfscope}%
\pgfpathrectangle{\pgfqpoint{0.100000in}{0.212622in}}{\pgfqpoint{3.696000in}{3.696000in}}%
\pgfusepath{clip}%
\pgfsetbuttcap%
\pgfsetroundjoin%
\definecolor{currentfill}{rgb}{0.121569,0.466667,0.705882}%
\pgfsetfillcolor{currentfill}%
\pgfsetfillopacity{0.300053}%
\pgfsetlinewidth{1.003750pt}%
\definecolor{currentstroke}{rgb}{0.121569,0.466667,0.705882}%
\pgfsetstrokecolor{currentstroke}%
\pgfsetstrokeopacity{0.300053}%
\pgfsetdash{}{0pt}%
\pgfpathmoveto{\pgfqpoint{1.681790in}{2.094151in}}%
\pgfpathcurveto{\pgfqpoint{1.690026in}{2.094151in}}{\pgfqpoint{1.697926in}{2.097423in}}{\pgfqpoint{1.703750in}{2.103247in}}%
\pgfpathcurveto{\pgfqpoint{1.709574in}{2.109071in}}{\pgfqpoint{1.712846in}{2.116971in}}{\pgfqpoint{1.712846in}{2.125207in}}%
\pgfpathcurveto{\pgfqpoint{1.712846in}{2.133443in}}{\pgfqpoint{1.709574in}{2.141344in}}{\pgfqpoint{1.703750in}{2.147167in}}%
\pgfpathcurveto{\pgfqpoint{1.697926in}{2.152991in}}{\pgfqpoint{1.690026in}{2.156264in}}{\pgfqpoint{1.681790in}{2.156264in}}%
\pgfpathcurveto{\pgfqpoint{1.673554in}{2.156264in}}{\pgfqpoint{1.665653in}{2.152991in}}{\pgfqpoint{1.659830in}{2.147167in}}%
\pgfpathcurveto{\pgfqpoint{1.654006in}{2.141344in}}{\pgfqpoint{1.650733in}{2.133443in}}{\pgfqpoint{1.650733in}{2.125207in}}%
\pgfpathcurveto{\pgfqpoint{1.650733in}{2.116971in}}{\pgfqpoint{1.654006in}{2.109071in}}{\pgfqpoint{1.659830in}{2.103247in}}%
\pgfpathcurveto{\pgfqpoint{1.665653in}{2.097423in}}{\pgfqpoint{1.673554in}{2.094151in}}{\pgfqpoint{1.681790in}{2.094151in}}%
\pgfpathclose%
\pgfusepath{stroke,fill}%
\end{pgfscope}%
\begin{pgfscope}%
\pgfpathrectangle{\pgfqpoint{0.100000in}{0.212622in}}{\pgfqpoint{3.696000in}{3.696000in}}%
\pgfusepath{clip}%
\pgfsetbuttcap%
\pgfsetroundjoin%
\definecolor{currentfill}{rgb}{0.121569,0.466667,0.705882}%
\pgfsetfillcolor{currentfill}%
\pgfsetfillopacity{0.300146}%
\pgfsetlinewidth{1.003750pt}%
\definecolor{currentstroke}{rgb}{0.121569,0.466667,0.705882}%
\pgfsetstrokecolor{currentstroke}%
\pgfsetstrokeopacity{0.300146}%
\pgfsetdash{}{0pt}%
\pgfpathmoveto{\pgfqpoint{1.678267in}{2.094630in}}%
\pgfpathcurveto{\pgfqpoint{1.686504in}{2.094630in}}{\pgfqpoint{1.694404in}{2.097902in}}{\pgfqpoint{1.700228in}{2.103726in}}%
\pgfpathcurveto{\pgfqpoint{1.706052in}{2.109550in}}{\pgfqpoint{1.709324in}{2.117450in}}{\pgfqpoint{1.709324in}{2.125687in}}%
\pgfpathcurveto{\pgfqpoint{1.709324in}{2.133923in}}{\pgfqpoint{1.706052in}{2.141823in}}{\pgfqpoint{1.700228in}{2.147647in}}%
\pgfpathcurveto{\pgfqpoint{1.694404in}{2.153471in}}{\pgfqpoint{1.686504in}{2.156743in}}{\pgfqpoint{1.678267in}{2.156743in}}%
\pgfpathcurveto{\pgfqpoint{1.670031in}{2.156743in}}{\pgfqpoint{1.662131in}{2.153471in}}{\pgfqpoint{1.656307in}{2.147647in}}%
\pgfpathcurveto{\pgfqpoint{1.650483in}{2.141823in}}{\pgfqpoint{1.647211in}{2.133923in}}{\pgfqpoint{1.647211in}{2.125687in}}%
\pgfpathcurveto{\pgfqpoint{1.647211in}{2.117450in}}{\pgfqpoint{1.650483in}{2.109550in}}{\pgfqpoint{1.656307in}{2.103726in}}%
\pgfpathcurveto{\pgfqpoint{1.662131in}{2.097902in}}{\pgfqpoint{1.670031in}{2.094630in}}{\pgfqpoint{1.678267in}{2.094630in}}%
\pgfpathclose%
\pgfusepath{stroke,fill}%
\end{pgfscope}%
\begin{pgfscope}%
\pgfpathrectangle{\pgfqpoint{0.100000in}{0.212622in}}{\pgfqpoint{3.696000in}{3.696000in}}%
\pgfusepath{clip}%
\pgfsetbuttcap%
\pgfsetroundjoin%
\definecolor{currentfill}{rgb}{0.121569,0.466667,0.705882}%
\pgfsetfillcolor{currentfill}%
\pgfsetfillopacity{0.300190}%
\pgfsetlinewidth{1.003750pt}%
\definecolor{currentstroke}{rgb}{0.121569,0.466667,0.705882}%
\pgfsetstrokecolor{currentstroke}%
\pgfsetstrokeopacity{0.300190}%
\pgfsetdash{}{0pt}%
\pgfpathmoveto{\pgfqpoint{1.682906in}{2.093947in}}%
\pgfpathcurveto{\pgfqpoint{1.691143in}{2.093947in}}{\pgfqpoint{1.699043in}{2.097219in}}{\pgfqpoint{1.704867in}{2.103043in}}%
\pgfpathcurveto{\pgfqpoint{1.710691in}{2.108867in}}{\pgfqpoint{1.713963in}{2.116767in}}{\pgfqpoint{1.713963in}{2.125003in}}%
\pgfpathcurveto{\pgfqpoint{1.713963in}{2.133239in}}{\pgfqpoint{1.710691in}{2.141140in}}{\pgfqpoint{1.704867in}{2.146963in}}%
\pgfpathcurveto{\pgfqpoint{1.699043in}{2.152787in}}{\pgfqpoint{1.691143in}{2.156060in}}{\pgfqpoint{1.682906in}{2.156060in}}%
\pgfpathcurveto{\pgfqpoint{1.674670in}{2.156060in}}{\pgfqpoint{1.666770in}{2.152787in}}{\pgfqpoint{1.660946in}{2.146963in}}%
\pgfpathcurveto{\pgfqpoint{1.655122in}{2.141140in}}{\pgfqpoint{1.651850in}{2.133239in}}{\pgfqpoint{1.651850in}{2.125003in}}%
\pgfpathcurveto{\pgfqpoint{1.651850in}{2.116767in}}{\pgfqpoint{1.655122in}{2.108867in}}{\pgfqpoint{1.660946in}{2.103043in}}%
\pgfpathcurveto{\pgfqpoint{1.666770in}{2.097219in}}{\pgfqpoint{1.674670in}{2.093947in}}{\pgfqpoint{1.682906in}{2.093947in}}%
\pgfpathclose%
\pgfusepath{stroke,fill}%
\end{pgfscope}%
\begin{pgfscope}%
\pgfpathrectangle{\pgfqpoint{0.100000in}{0.212622in}}{\pgfqpoint{3.696000in}{3.696000in}}%
\pgfusepath{clip}%
\pgfsetbuttcap%
\pgfsetroundjoin%
\definecolor{currentfill}{rgb}{0.121569,0.466667,0.705882}%
\pgfsetfillcolor{currentfill}%
\pgfsetfillopacity{0.300240}%
\pgfsetlinewidth{1.003750pt}%
\definecolor{currentstroke}{rgb}{0.121569,0.466667,0.705882}%
\pgfsetstrokecolor{currentstroke}%
\pgfsetstrokeopacity{0.300240}%
\pgfsetdash{}{0pt}%
\pgfpathmoveto{\pgfqpoint{1.677634in}{2.094699in}}%
\pgfpathcurveto{\pgfqpoint{1.685870in}{2.094699in}}{\pgfqpoint{1.693770in}{2.097972in}}{\pgfqpoint{1.699594in}{2.103796in}}%
\pgfpathcurveto{\pgfqpoint{1.705418in}{2.109620in}}{\pgfqpoint{1.708691in}{2.117520in}}{\pgfqpoint{1.708691in}{2.125756in}}%
\pgfpathcurveto{\pgfqpoint{1.708691in}{2.133992in}}{\pgfqpoint{1.705418in}{2.141892in}}{\pgfqpoint{1.699594in}{2.147716in}}%
\pgfpathcurveto{\pgfqpoint{1.693770in}{2.153540in}}{\pgfqpoint{1.685870in}{2.156812in}}{\pgfqpoint{1.677634in}{2.156812in}}%
\pgfpathcurveto{\pgfqpoint{1.669398in}{2.156812in}}{\pgfqpoint{1.661498in}{2.153540in}}{\pgfqpoint{1.655674in}{2.147716in}}%
\pgfpathcurveto{\pgfqpoint{1.649850in}{2.141892in}}{\pgfqpoint{1.646578in}{2.133992in}}{\pgfqpoint{1.646578in}{2.125756in}}%
\pgfpathcurveto{\pgfqpoint{1.646578in}{2.117520in}}{\pgfqpoint{1.649850in}{2.109620in}}{\pgfqpoint{1.655674in}{2.103796in}}%
\pgfpathcurveto{\pgfqpoint{1.661498in}{2.097972in}}{\pgfqpoint{1.669398in}{2.094699in}}{\pgfqpoint{1.677634in}{2.094699in}}%
\pgfpathclose%
\pgfusepath{stroke,fill}%
\end{pgfscope}%
\begin{pgfscope}%
\pgfpathrectangle{\pgfqpoint{0.100000in}{0.212622in}}{\pgfqpoint{3.696000in}{3.696000in}}%
\pgfusepath{clip}%
\pgfsetbuttcap%
\pgfsetroundjoin%
\definecolor{currentfill}{rgb}{0.121569,0.466667,0.705882}%
\pgfsetfillcolor{currentfill}%
\pgfsetfillopacity{0.300262}%
\pgfsetlinewidth{1.003750pt}%
\definecolor{currentstroke}{rgb}{0.121569,0.466667,0.705882}%
\pgfsetstrokecolor{currentstroke}%
\pgfsetstrokeopacity{0.300262}%
\pgfsetdash{}{0pt}%
\pgfpathmoveto{\pgfqpoint{1.677533in}{2.094708in}}%
\pgfpathcurveto{\pgfqpoint{1.685769in}{2.094708in}}{\pgfqpoint{1.693669in}{2.097980in}}{\pgfqpoint{1.699493in}{2.103804in}}%
\pgfpathcurveto{\pgfqpoint{1.705317in}{2.109628in}}{\pgfqpoint{1.708589in}{2.117528in}}{\pgfqpoint{1.708589in}{2.125765in}}%
\pgfpathcurveto{\pgfqpoint{1.708589in}{2.134001in}}{\pgfqpoint{1.705317in}{2.141901in}}{\pgfqpoint{1.699493in}{2.147725in}}%
\pgfpathcurveto{\pgfqpoint{1.693669in}{2.153549in}}{\pgfqpoint{1.685769in}{2.156821in}}{\pgfqpoint{1.677533in}{2.156821in}}%
\pgfpathcurveto{\pgfqpoint{1.669297in}{2.156821in}}{\pgfqpoint{1.661396in}{2.153549in}}{\pgfqpoint{1.655573in}{2.147725in}}%
\pgfpathcurveto{\pgfqpoint{1.649749in}{2.141901in}}{\pgfqpoint{1.646476in}{2.134001in}}{\pgfqpoint{1.646476in}{2.125765in}}%
\pgfpathcurveto{\pgfqpoint{1.646476in}{2.117528in}}{\pgfqpoint{1.649749in}{2.109628in}}{\pgfqpoint{1.655573in}{2.103804in}}%
\pgfpathcurveto{\pgfqpoint{1.661396in}{2.097980in}}{\pgfqpoint{1.669297in}{2.094708in}}{\pgfqpoint{1.677533in}{2.094708in}}%
\pgfpathclose%
\pgfusepath{stroke,fill}%
\end{pgfscope}%
\begin{pgfscope}%
\pgfpathrectangle{\pgfqpoint{0.100000in}{0.212622in}}{\pgfqpoint{3.696000in}{3.696000in}}%
\pgfusepath{clip}%
\pgfsetbuttcap%
\pgfsetroundjoin%
\definecolor{currentfill}{rgb}{0.121569,0.466667,0.705882}%
\pgfsetfillcolor{currentfill}%
\pgfsetfillopacity{0.300279}%
\pgfsetlinewidth{1.003750pt}%
\definecolor{currentstroke}{rgb}{0.121569,0.466667,0.705882}%
\pgfsetstrokecolor{currentstroke}%
\pgfsetstrokeopacity{0.300279}%
\pgfsetdash{}{0pt}%
\pgfpathmoveto{\pgfqpoint{1.683482in}{2.093825in}}%
\pgfpathcurveto{\pgfqpoint{1.691718in}{2.093825in}}{\pgfqpoint{1.699618in}{2.097098in}}{\pgfqpoint{1.705442in}{2.102922in}}%
\pgfpathcurveto{\pgfqpoint{1.711266in}{2.108746in}}{\pgfqpoint{1.714538in}{2.116646in}}{\pgfqpoint{1.714538in}{2.124882in}}%
\pgfpathcurveto{\pgfqpoint{1.714538in}{2.133118in}}{\pgfqpoint{1.711266in}{2.141018in}}{\pgfqpoint{1.705442in}{2.146842in}}%
\pgfpathcurveto{\pgfqpoint{1.699618in}{2.152666in}}{\pgfqpoint{1.691718in}{2.155938in}}{\pgfqpoint{1.683482in}{2.155938in}}%
\pgfpathcurveto{\pgfqpoint{1.675246in}{2.155938in}}{\pgfqpoint{1.667346in}{2.152666in}}{\pgfqpoint{1.661522in}{2.146842in}}%
\pgfpathcurveto{\pgfqpoint{1.655698in}{2.141018in}}{\pgfqpoint{1.652425in}{2.133118in}}{\pgfqpoint{1.652425in}{2.124882in}}%
\pgfpathcurveto{\pgfqpoint{1.652425in}{2.116646in}}{\pgfqpoint{1.655698in}{2.108746in}}{\pgfqpoint{1.661522in}{2.102922in}}%
\pgfpathcurveto{\pgfqpoint{1.667346in}{2.097098in}}{\pgfqpoint{1.675246in}{2.093825in}}{\pgfqpoint{1.683482in}{2.093825in}}%
\pgfpathclose%
\pgfusepath{stroke,fill}%
\end{pgfscope}%
\begin{pgfscope}%
\pgfpathrectangle{\pgfqpoint{0.100000in}{0.212622in}}{\pgfqpoint{3.696000in}{3.696000in}}%
\pgfusepath{clip}%
\pgfsetbuttcap%
\pgfsetroundjoin%
\definecolor{currentfill}{rgb}{0.121569,0.466667,0.705882}%
\pgfsetfillcolor{currentfill}%
\pgfsetfillopacity{0.300294}%
\pgfsetlinewidth{1.003750pt}%
\definecolor{currentstroke}{rgb}{0.121569,0.466667,0.705882}%
\pgfsetstrokecolor{currentstroke}%
\pgfsetstrokeopacity{0.300294}%
\pgfsetdash{}{0pt}%
\pgfpathmoveto{\pgfqpoint{1.683856in}{2.093713in}}%
\pgfpathcurveto{\pgfqpoint{1.692093in}{2.093713in}}{\pgfqpoint{1.699993in}{2.096985in}}{\pgfqpoint{1.705817in}{2.102809in}}%
\pgfpathcurveto{\pgfqpoint{1.711641in}{2.108633in}}{\pgfqpoint{1.714913in}{2.116533in}}{\pgfqpoint{1.714913in}{2.124769in}}%
\pgfpathcurveto{\pgfqpoint{1.714913in}{2.133006in}}{\pgfqpoint{1.711641in}{2.140906in}}{\pgfqpoint{1.705817in}{2.146730in}}%
\pgfpathcurveto{\pgfqpoint{1.699993in}{2.152554in}}{\pgfqpoint{1.692093in}{2.155826in}}{\pgfqpoint{1.683856in}{2.155826in}}%
\pgfpathcurveto{\pgfqpoint{1.675620in}{2.155826in}}{\pgfqpoint{1.667720in}{2.152554in}}{\pgfqpoint{1.661896in}{2.146730in}}%
\pgfpathcurveto{\pgfqpoint{1.656072in}{2.140906in}}{\pgfqpoint{1.652800in}{2.133006in}}{\pgfqpoint{1.652800in}{2.124769in}}%
\pgfpathcurveto{\pgfqpoint{1.652800in}{2.116533in}}{\pgfqpoint{1.656072in}{2.108633in}}{\pgfqpoint{1.661896in}{2.102809in}}%
\pgfpathcurveto{\pgfqpoint{1.667720in}{2.096985in}}{\pgfqpoint{1.675620in}{2.093713in}}{\pgfqpoint{1.683856in}{2.093713in}}%
\pgfpathclose%
\pgfusepath{stroke,fill}%
\end{pgfscope}%
\begin{pgfscope}%
\pgfpathrectangle{\pgfqpoint{0.100000in}{0.212622in}}{\pgfqpoint{3.696000in}{3.696000in}}%
\pgfusepath{clip}%
\pgfsetbuttcap%
\pgfsetroundjoin%
\definecolor{currentfill}{rgb}{0.121569,0.466667,0.705882}%
\pgfsetfillcolor{currentfill}%
\pgfsetfillopacity{0.300306}%
\pgfsetlinewidth{1.003750pt}%
\definecolor{currentstroke}{rgb}{0.121569,0.466667,0.705882}%
\pgfsetstrokecolor{currentstroke}%
\pgfsetstrokeopacity{0.300306}%
\pgfsetdash{}{0pt}%
\pgfpathmoveto{\pgfqpoint{1.677363in}{2.094723in}}%
\pgfpathcurveto{\pgfqpoint{1.685600in}{2.094723in}}{\pgfqpoint{1.693500in}{2.097996in}}{\pgfqpoint{1.699324in}{2.103820in}}%
\pgfpathcurveto{\pgfqpoint{1.705148in}{2.109643in}}{\pgfqpoint{1.708420in}{2.117543in}}{\pgfqpoint{1.708420in}{2.125780in}}%
\pgfpathcurveto{\pgfqpoint{1.708420in}{2.134016in}}{\pgfqpoint{1.705148in}{2.141916in}}{\pgfqpoint{1.699324in}{2.147740in}}%
\pgfpathcurveto{\pgfqpoint{1.693500in}{2.153564in}}{\pgfqpoint{1.685600in}{2.156836in}}{\pgfqpoint{1.677363in}{2.156836in}}%
\pgfpathcurveto{\pgfqpoint{1.669127in}{2.156836in}}{\pgfqpoint{1.661227in}{2.153564in}}{\pgfqpoint{1.655403in}{2.147740in}}%
\pgfpathcurveto{\pgfqpoint{1.649579in}{2.141916in}}{\pgfqpoint{1.646307in}{2.134016in}}{\pgfqpoint{1.646307in}{2.125780in}}%
\pgfpathcurveto{\pgfqpoint{1.646307in}{2.117543in}}{\pgfqpoint{1.649579in}{2.109643in}}{\pgfqpoint{1.655403in}{2.103820in}}%
\pgfpathcurveto{\pgfqpoint{1.661227in}{2.097996in}}{\pgfqpoint{1.669127in}{2.094723in}}{\pgfqpoint{1.677363in}{2.094723in}}%
\pgfpathclose%
\pgfusepath{stroke,fill}%
\end{pgfscope}%
\begin{pgfscope}%
\pgfpathrectangle{\pgfqpoint{0.100000in}{0.212622in}}{\pgfqpoint{3.696000in}{3.696000in}}%
\pgfusepath{clip}%
\pgfsetbuttcap%
\pgfsetroundjoin%
\definecolor{currentfill}{rgb}{0.121569,0.466667,0.705882}%
\pgfsetfillcolor{currentfill}%
\pgfsetfillopacity{0.300320}%
\pgfsetlinewidth{1.003750pt}%
\definecolor{currentstroke}{rgb}{0.121569,0.466667,0.705882}%
\pgfsetstrokecolor{currentstroke}%
\pgfsetstrokeopacity{0.300320}%
\pgfsetdash{}{0pt}%
\pgfpathmoveto{\pgfqpoint{1.684036in}{2.093681in}}%
\pgfpathcurveto{\pgfqpoint{1.692272in}{2.093681in}}{\pgfqpoint{1.700172in}{2.096954in}}{\pgfqpoint{1.705996in}{2.102777in}}%
\pgfpathcurveto{\pgfqpoint{1.711820in}{2.108601in}}{\pgfqpoint{1.715092in}{2.116501in}}{\pgfqpoint{1.715092in}{2.124738in}}%
\pgfpathcurveto{\pgfqpoint{1.715092in}{2.132974in}}{\pgfqpoint{1.711820in}{2.140874in}}{\pgfqpoint{1.705996in}{2.146698in}}%
\pgfpathcurveto{\pgfqpoint{1.700172in}{2.152522in}}{\pgfqpoint{1.692272in}{2.155794in}}{\pgfqpoint{1.684036in}{2.155794in}}%
\pgfpathcurveto{\pgfqpoint{1.675799in}{2.155794in}}{\pgfqpoint{1.667899in}{2.152522in}}{\pgfqpoint{1.662075in}{2.146698in}}%
\pgfpathcurveto{\pgfqpoint{1.656251in}{2.140874in}}{\pgfqpoint{1.652979in}{2.132974in}}{\pgfqpoint{1.652979in}{2.124738in}}%
\pgfpathcurveto{\pgfqpoint{1.652979in}{2.116501in}}{\pgfqpoint{1.656251in}{2.108601in}}{\pgfqpoint{1.662075in}{2.102777in}}%
\pgfpathcurveto{\pgfqpoint{1.667899in}{2.096954in}}{\pgfqpoint{1.675799in}{2.093681in}}{\pgfqpoint{1.684036in}{2.093681in}}%
\pgfpathclose%
\pgfusepath{stroke,fill}%
\end{pgfscope}%
\begin{pgfscope}%
\pgfpathrectangle{\pgfqpoint{0.100000in}{0.212622in}}{\pgfqpoint{3.696000in}{3.696000in}}%
\pgfusepath{clip}%
\pgfsetbuttcap%
\pgfsetroundjoin%
\definecolor{currentfill}{rgb}{0.121569,0.466667,0.705882}%
\pgfsetfillcolor{currentfill}%
\pgfsetfillopacity{0.300327}%
\pgfsetlinewidth{1.003750pt}%
\definecolor{currentstroke}{rgb}{0.121569,0.466667,0.705882}%
\pgfsetstrokecolor{currentstroke}%
\pgfsetstrokeopacity{0.300327}%
\pgfsetdash{}{0pt}%
\pgfpathmoveto{\pgfqpoint{1.684147in}{2.093655in}}%
\pgfpathcurveto{\pgfqpoint{1.692383in}{2.093655in}}{\pgfqpoint{1.700283in}{2.096927in}}{\pgfqpoint{1.706107in}{2.102751in}}%
\pgfpathcurveto{\pgfqpoint{1.711931in}{2.108575in}}{\pgfqpoint{1.715204in}{2.116475in}}{\pgfqpoint{1.715204in}{2.124711in}}%
\pgfpathcurveto{\pgfqpoint{1.715204in}{2.132948in}}{\pgfqpoint{1.711931in}{2.140848in}}{\pgfqpoint{1.706107in}{2.146672in}}%
\pgfpathcurveto{\pgfqpoint{1.700283in}{2.152495in}}{\pgfqpoint{1.692383in}{2.155768in}}{\pgfqpoint{1.684147in}{2.155768in}}%
\pgfpathcurveto{\pgfqpoint{1.675911in}{2.155768in}}{\pgfqpoint{1.668011in}{2.152495in}}{\pgfqpoint{1.662187in}{2.146672in}}%
\pgfpathcurveto{\pgfqpoint{1.656363in}{2.140848in}}{\pgfqpoint{1.653091in}{2.132948in}}{\pgfqpoint{1.653091in}{2.124711in}}%
\pgfpathcurveto{\pgfqpoint{1.653091in}{2.116475in}}{\pgfqpoint{1.656363in}{2.108575in}}{\pgfqpoint{1.662187in}{2.102751in}}%
\pgfpathcurveto{\pgfqpoint{1.668011in}{2.096927in}}{\pgfqpoint{1.675911in}{2.093655in}}{\pgfqpoint{1.684147in}{2.093655in}}%
\pgfpathclose%
\pgfusepath{stroke,fill}%
\end{pgfscope}%
\begin{pgfscope}%
\pgfpathrectangle{\pgfqpoint{0.100000in}{0.212622in}}{\pgfqpoint{3.696000in}{3.696000in}}%
\pgfusepath{clip}%
\pgfsetbuttcap%
\pgfsetroundjoin%
\definecolor{currentfill}{rgb}{0.121569,0.466667,0.705882}%
\pgfsetfillcolor{currentfill}%
\pgfsetfillopacity{0.300335}%
\pgfsetlinewidth{1.003750pt}%
\definecolor{currentstroke}{rgb}{0.121569,0.466667,0.705882}%
\pgfsetstrokecolor{currentstroke}%
\pgfsetstrokeopacity{0.300335}%
\pgfsetdash{}{0pt}%
\pgfpathmoveto{\pgfqpoint{1.684201in}{2.093649in}}%
\pgfpathcurveto{\pgfqpoint{1.692438in}{2.093649in}}{\pgfqpoint{1.700338in}{2.096921in}}{\pgfqpoint{1.706162in}{2.102745in}}%
\pgfpathcurveto{\pgfqpoint{1.711986in}{2.108569in}}{\pgfqpoint{1.715258in}{2.116469in}}{\pgfqpoint{1.715258in}{2.124705in}}%
\pgfpathcurveto{\pgfqpoint{1.715258in}{2.132942in}}{\pgfqpoint{1.711986in}{2.140842in}}{\pgfqpoint{1.706162in}{2.146666in}}%
\pgfpathcurveto{\pgfqpoint{1.700338in}{2.152490in}}{\pgfqpoint{1.692438in}{2.155762in}}{\pgfqpoint{1.684201in}{2.155762in}}%
\pgfpathcurveto{\pgfqpoint{1.675965in}{2.155762in}}{\pgfqpoint{1.668065in}{2.152490in}}{\pgfqpoint{1.662241in}{2.146666in}}%
\pgfpathcurveto{\pgfqpoint{1.656417in}{2.140842in}}{\pgfqpoint{1.653145in}{2.132942in}}{\pgfqpoint{1.653145in}{2.124705in}}%
\pgfpathcurveto{\pgfqpoint{1.653145in}{2.116469in}}{\pgfqpoint{1.656417in}{2.108569in}}{\pgfqpoint{1.662241in}{2.102745in}}%
\pgfpathcurveto{\pgfqpoint{1.668065in}{2.096921in}}{\pgfqpoint{1.675965in}{2.093649in}}{\pgfqpoint{1.684201in}{2.093649in}}%
\pgfpathclose%
\pgfusepath{stroke,fill}%
\end{pgfscope}%
\begin{pgfscope}%
\pgfpathrectangle{\pgfqpoint{0.100000in}{0.212622in}}{\pgfqpoint{3.696000in}{3.696000in}}%
\pgfusepath{clip}%
\pgfsetbuttcap%
\pgfsetroundjoin%
\definecolor{currentfill}{rgb}{0.121569,0.466667,0.705882}%
\pgfsetfillcolor{currentfill}%
\pgfsetfillopacity{0.300393}%
\pgfsetlinewidth{1.003750pt}%
\definecolor{currentstroke}{rgb}{0.121569,0.466667,0.705882}%
\pgfsetstrokecolor{currentstroke}%
\pgfsetstrokeopacity{0.300393}%
\pgfsetdash{}{0pt}%
\pgfpathmoveto{\pgfqpoint{1.677090in}{2.094742in}}%
\pgfpathcurveto{\pgfqpoint{1.685326in}{2.094742in}}{\pgfqpoint{1.693226in}{2.098014in}}{\pgfqpoint{1.699050in}{2.103838in}}%
\pgfpathcurveto{\pgfqpoint{1.704874in}{2.109662in}}{\pgfqpoint{1.708146in}{2.117562in}}{\pgfqpoint{1.708146in}{2.125799in}}%
\pgfpathcurveto{\pgfqpoint{1.708146in}{2.134035in}}{\pgfqpoint{1.704874in}{2.141935in}}{\pgfqpoint{1.699050in}{2.147759in}}%
\pgfpathcurveto{\pgfqpoint{1.693226in}{2.153583in}}{\pgfqpoint{1.685326in}{2.156855in}}{\pgfqpoint{1.677090in}{2.156855in}}%
\pgfpathcurveto{\pgfqpoint{1.668854in}{2.156855in}}{\pgfqpoint{1.660953in}{2.153583in}}{\pgfqpoint{1.655130in}{2.147759in}}%
\pgfpathcurveto{\pgfqpoint{1.649306in}{2.141935in}}{\pgfqpoint{1.646033in}{2.134035in}}{\pgfqpoint{1.646033in}{2.125799in}}%
\pgfpathcurveto{\pgfqpoint{1.646033in}{2.117562in}}{\pgfqpoint{1.649306in}{2.109662in}}{\pgfqpoint{1.655130in}{2.103838in}}%
\pgfpathcurveto{\pgfqpoint{1.660953in}{2.098014in}}{\pgfqpoint{1.668854in}{2.094742in}}{\pgfqpoint{1.677090in}{2.094742in}}%
\pgfpathclose%
\pgfusepath{stroke,fill}%
\end{pgfscope}%
\begin{pgfscope}%
\pgfpathrectangle{\pgfqpoint{0.100000in}{0.212622in}}{\pgfqpoint{3.696000in}{3.696000in}}%
\pgfusepath{clip}%
\pgfsetbuttcap%
\pgfsetroundjoin%
\definecolor{currentfill}{rgb}{0.121569,0.466667,0.705882}%
\pgfsetfillcolor{currentfill}%
\pgfsetfillopacity{0.300446}%
\pgfsetlinewidth{1.003750pt}%
\definecolor{currentstroke}{rgb}{0.121569,0.466667,0.705882}%
\pgfsetstrokecolor{currentstroke}%
\pgfsetstrokeopacity{0.300446}%
\pgfsetdash{}{0pt}%
\pgfpathmoveto{\pgfqpoint{1.685205in}{2.093522in}}%
\pgfpathcurveto{\pgfqpoint{1.693441in}{2.093522in}}{\pgfqpoint{1.701341in}{2.096794in}}{\pgfqpoint{1.707165in}{2.102618in}}%
\pgfpathcurveto{\pgfqpoint{1.712989in}{2.108442in}}{\pgfqpoint{1.716261in}{2.116342in}}{\pgfqpoint{1.716261in}{2.124578in}}%
\pgfpathcurveto{\pgfqpoint{1.716261in}{2.132814in}}{\pgfqpoint{1.712989in}{2.140714in}}{\pgfqpoint{1.707165in}{2.146538in}}%
\pgfpathcurveto{\pgfqpoint{1.701341in}{2.152362in}}{\pgfqpoint{1.693441in}{2.155635in}}{\pgfqpoint{1.685205in}{2.155635in}}%
\pgfpathcurveto{\pgfqpoint{1.676969in}{2.155635in}}{\pgfqpoint{1.669069in}{2.152362in}}{\pgfqpoint{1.663245in}{2.146538in}}%
\pgfpathcurveto{\pgfqpoint{1.657421in}{2.140714in}}{\pgfqpoint{1.654148in}{2.132814in}}{\pgfqpoint{1.654148in}{2.124578in}}%
\pgfpathcurveto{\pgfqpoint{1.654148in}{2.116342in}}{\pgfqpoint{1.657421in}{2.108442in}}{\pgfqpoint{1.663245in}{2.102618in}}%
\pgfpathcurveto{\pgfqpoint{1.669069in}{2.096794in}}{\pgfqpoint{1.676969in}{2.093522in}}{\pgfqpoint{1.685205in}{2.093522in}}%
\pgfpathclose%
\pgfusepath{stroke,fill}%
\end{pgfscope}%
\begin{pgfscope}%
\pgfpathrectangle{\pgfqpoint{0.100000in}{0.212622in}}{\pgfqpoint{3.696000in}{3.696000in}}%
\pgfusepath{clip}%
\pgfsetbuttcap%
\pgfsetroundjoin%
\definecolor{currentfill}{rgb}{0.121569,0.466667,0.705882}%
\pgfsetfillcolor{currentfill}%
\pgfsetfillopacity{0.300486}%
\pgfsetlinewidth{1.003750pt}%
\definecolor{currentstroke}{rgb}{0.121569,0.466667,0.705882}%
\pgfsetstrokecolor{currentstroke}%
\pgfsetstrokeopacity{0.300486}%
\pgfsetdash{}{0pt}%
\pgfpathmoveto{\pgfqpoint{1.685789in}{2.093426in}}%
\pgfpathcurveto{\pgfqpoint{1.694025in}{2.093426in}}{\pgfqpoint{1.701925in}{2.096698in}}{\pgfqpoint{1.707749in}{2.102522in}}%
\pgfpathcurveto{\pgfqpoint{1.713573in}{2.108346in}}{\pgfqpoint{1.716845in}{2.116246in}}{\pgfqpoint{1.716845in}{2.124482in}}%
\pgfpathcurveto{\pgfqpoint{1.716845in}{2.132719in}}{\pgfqpoint{1.713573in}{2.140619in}}{\pgfqpoint{1.707749in}{2.146442in}}%
\pgfpathcurveto{\pgfqpoint{1.701925in}{2.152266in}}{\pgfqpoint{1.694025in}{2.155539in}}{\pgfqpoint{1.685789in}{2.155539in}}%
\pgfpathcurveto{\pgfqpoint{1.677552in}{2.155539in}}{\pgfqpoint{1.669652in}{2.152266in}}{\pgfqpoint{1.663828in}{2.146442in}}%
\pgfpathcurveto{\pgfqpoint{1.658004in}{2.140619in}}{\pgfqpoint{1.654732in}{2.132719in}}{\pgfqpoint{1.654732in}{2.124482in}}%
\pgfpathcurveto{\pgfqpoint{1.654732in}{2.116246in}}{\pgfqpoint{1.658004in}{2.108346in}}{\pgfqpoint{1.663828in}{2.102522in}}%
\pgfpathcurveto{\pgfqpoint{1.669652in}{2.096698in}}{\pgfqpoint{1.677552in}{2.093426in}}{\pgfqpoint{1.685789in}{2.093426in}}%
\pgfpathclose%
\pgfusepath{stroke,fill}%
\end{pgfscope}%
\begin{pgfscope}%
\pgfpathrectangle{\pgfqpoint{0.100000in}{0.212622in}}{\pgfqpoint{3.696000in}{3.696000in}}%
\pgfusepath{clip}%
\pgfsetbuttcap%
\pgfsetroundjoin%
\definecolor{currentfill}{rgb}{0.121569,0.466667,0.705882}%
\pgfsetfillcolor{currentfill}%
\pgfsetfillopacity{0.300551}%
\pgfsetlinewidth{1.003750pt}%
\definecolor{currentstroke}{rgb}{0.121569,0.466667,0.705882}%
\pgfsetstrokecolor{currentstroke}%
\pgfsetstrokeopacity{0.300551}%
\pgfsetdash{}{0pt}%
\pgfpathmoveto{\pgfqpoint{1.676602in}{2.094768in}}%
\pgfpathcurveto{\pgfqpoint{1.684838in}{2.094768in}}{\pgfqpoint{1.692738in}{2.098040in}}{\pgfqpoint{1.698562in}{2.103864in}}%
\pgfpathcurveto{\pgfqpoint{1.704386in}{2.109688in}}{\pgfqpoint{1.707658in}{2.117588in}}{\pgfqpoint{1.707658in}{2.125824in}}%
\pgfpathcurveto{\pgfqpoint{1.707658in}{2.134060in}}{\pgfqpoint{1.704386in}{2.141960in}}{\pgfqpoint{1.698562in}{2.147784in}}%
\pgfpathcurveto{\pgfqpoint{1.692738in}{2.153608in}}{\pgfqpoint{1.684838in}{2.156881in}}{\pgfqpoint{1.676602in}{2.156881in}}%
\pgfpathcurveto{\pgfqpoint{1.668365in}{2.156881in}}{\pgfqpoint{1.660465in}{2.153608in}}{\pgfqpoint{1.654642in}{2.147784in}}%
\pgfpathcurveto{\pgfqpoint{1.648818in}{2.141960in}}{\pgfqpoint{1.645545in}{2.134060in}}{\pgfqpoint{1.645545in}{2.125824in}}%
\pgfpathcurveto{\pgfqpoint{1.645545in}{2.117588in}}{\pgfqpoint{1.648818in}{2.109688in}}{\pgfqpoint{1.654642in}{2.103864in}}%
\pgfpathcurveto{\pgfqpoint{1.660465in}{2.098040in}}{\pgfqpoint{1.668365in}{2.094768in}}{\pgfqpoint{1.676602in}{2.094768in}}%
\pgfpathclose%
\pgfusepath{stroke,fill}%
\end{pgfscope}%
\begin{pgfscope}%
\pgfpathrectangle{\pgfqpoint{0.100000in}{0.212622in}}{\pgfqpoint{3.696000in}{3.696000in}}%
\pgfusepath{clip}%
\pgfsetbuttcap%
\pgfsetroundjoin%
\definecolor{currentfill}{rgb}{0.121569,0.466667,0.705882}%
\pgfsetfillcolor{currentfill}%
\pgfsetfillopacity{0.300552}%
\pgfsetlinewidth{1.003750pt}%
\definecolor{currentstroke}{rgb}{0.121569,0.466667,0.705882}%
\pgfsetstrokecolor{currentstroke}%
\pgfsetstrokeopacity{0.300552}%
\pgfsetdash{}{0pt}%
\pgfpathmoveto{\pgfqpoint{1.687658in}{2.093051in}}%
\pgfpathcurveto{\pgfqpoint{1.695894in}{2.093051in}}{\pgfqpoint{1.703795in}{2.096323in}}{\pgfqpoint{1.709618in}{2.102147in}}%
\pgfpathcurveto{\pgfqpoint{1.715442in}{2.107971in}}{\pgfqpoint{1.718715in}{2.115871in}}{\pgfqpoint{1.718715in}{2.124107in}}%
\pgfpathcurveto{\pgfqpoint{1.718715in}{2.132344in}}{\pgfqpoint{1.715442in}{2.140244in}}{\pgfqpoint{1.709618in}{2.146068in}}%
\pgfpathcurveto{\pgfqpoint{1.703795in}{2.151892in}}{\pgfqpoint{1.695894in}{2.155164in}}{\pgfqpoint{1.687658in}{2.155164in}}%
\pgfpathcurveto{\pgfqpoint{1.679422in}{2.155164in}}{\pgfqpoint{1.671522in}{2.151892in}}{\pgfqpoint{1.665698in}{2.146068in}}%
\pgfpathcurveto{\pgfqpoint{1.659874in}{2.140244in}}{\pgfqpoint{1.656602in}{2.132344in}}{\pgfqpoint{1.656602in}{2.124107in}}%
\pgfpathcurveto{\pgfqpoint{1.656602in}{2.115871in}}{\pgfqpoint{1.659874in}{2.107971in}}{\pgfqpoint{1.665698in}{2.102147in}}%
\pgfpathcurveto{\pgfqpoint{1.671522in}{2.096323in}}{\pgfqpoint{1.679422in}{2.093051in}}{\pgfqpoint{1.687658in}{2.093051in}}%
\pgfpathclose%
\pgfusepath{stroke,fill}%
\end{pgfscope}%
\begin{pgfscope}%
\pgfpathrectangle{\pgfqpoint{0.100000in}{0.212622in}}{\pgfqpoint{3.696000in}{3.696000in}}%
\pgfusepath{clip}%
\pgfsetbuttcap%
\pgfsetroundjoin%
\definecolor{currentfill}{rgb}{0.121569,0.466667,0.705882}%
\pgfsetfillcolor{currentfill}%
\pgfsetfillopacity{0.300552}%
\pgfsetlinewidth{1.003750pt}%
\definecolor{currentstroke}{rgb}{0.121569,0.466667,0.705882}%
\pgfsetstrokecolor{currentstroke}%
\pgfsetstrokeopacity{0.300552}%
\pgfsetdash{}{0pt}%
\pgfpathmoveto{\pgfqpoint{1.676599in}{2.094768in}}%
\pgfpathcurveto{\pgfqpoint{1.684836in}{2.094768in}}{\pgfqpoint{1.692736in}{2.098040in}}{\pgfqpoint{1.698560in}{2.103864in}}%
\pgfpathcurveto{\pgfqpoint{1.704384in}{2.109688in}}{\pgfqpoint{1.707656in}{2.117588in}}{\pgfqpoint{1.707656in}{2.125824in}}%
\pgfpathcurveto{\pgfqpoint{1.707656in}{2.134060in}}{\pgfqpoint{1.704384in}{2.141961in}}{\pgfqpoint{1.698560in}{2.147784in}}%
\pgfpathcurveto{\pgfqpoint{1.692736in}{2.153608in}}{\pgfqpoint{1.684836in}{2.156881in}}{\pgfqpoint{1.676599in}{2.156881in}}%
\pgfpathcurveto{\pgfqpoint{1.668363in}{2.156881in}}{\pgfqpoint{1.660463in}{2.153608in}}{\pgfqpoint{1.654639in}{2.147784in}}%
\pgfpathcurveto{\pgfqpoint{1.648815in}{2.141961in}}{\pgfqpoint{1.645543in}{2.134060in}}{\pgfqpoint{1.645543in}{2.125824in}}%
\pgfpathcurveto{\pgfqpoint{1.645543in}{2.117588in}}{\pgfqpoint{1.648815in}{2.109688in}}{\pgfqpoint{1.654639in}{2.103864in}}%
\pgfpathcurveto{\pgfqpoint{1.660463in}{2.098040in}}{\pgfqpoint{1.668363in}{2.094768in}}{\pgfqpoint{1.676599in}{2.094768in}}%
\pgfpathclose%
\pgfusepath{stroke,fill}%
\end{pgfscope}%
\begin{pgfscope}%
\pgfpathrectangle{\pgfqpoint{0.100000in}{0.212622in}}{\pgfqpoint{3.696000in}{3.696000in}}%
\pgfusepath{clip}%
\pgfsetbuttcap%
\pgfsetroundjoin%
\definecolor{currentfill}{rgb}{0.121569,0.466667,0.705882}%
\pgfsetfillcolor{currentfill}%
\pgfsetfillopacity{0.300555}%
\pgfsetlinewidth{1.003750pt}%
\definecolor{currentstroke}{rgb}{0.121569,0.466667,0.705882}%
\pgfsetstrokecolor{currentstroke}%
\pgfsetstrokeopacity{0.300555}%
\pgfsetdash{}{0pt}%
\pgfpathmoveto{\pgfqpoint{1.676595in}{2.094768in}}%
\pgfpathcurveto{\pgfqpoint{1.684831in}{2.094768in}}{\pgfqpoint{1.692731in}{2.098040in}}{\pgfqpoint{1.698555in}{2.103864in}}%
\pgfpathcurveto{\pgfqpoint{1.704379in}{2.109688in}}{\pgfqpoint{1.707651in}{2.117588in}}{\pgfqpoint{1.707651in}{2.125824in}}%
\pgfpathcurveto{\pgfqpoint{1.707651in}{2.134060in}}{\pgfqpoint{1.704379in}{2.141961in}}{\pgfqpoint{1.698555in}{2.147784in}}%
\pgfpathcurveto{\pgfqpoint{1.692731in}{2.153608in}}{\pgfqpoint{1.684831in}{2.156881in}}{\pgfqpoint{1.676595in}{2.156881in}}%
\pgfpathcurveto{\pgfqpoint{1.668359in}{2.156881in}}{\pgfqpoint{1.660459in}{2.153608in}}{\pgfqpoint{1.654635in}{2.147784in}}%
\pgfpathcurveto{\pgfqpoint{1.648811in}{2.141961in}}{\pgfqpoint{1.645538in}{2.134060in}}{\pgfqpoint{1.645538in}{2.125824in}}%
\pgfpathcurveto{\pgfqpoint{1.645538in}{2.117588in}}{\pgfqpoint{1.648811in}{2.109688in}}{\pgfqpoint{1.654635in}{2.103864in}}%
\pgfpathcurveto{\pgfqpoint{1.660459in}{2.098040in}}{\pgfqpoint{1.668359in}{2.094768in}}{\pgfqpoint{1.676595in}{2.094768in}}%
\pgfpathclose%
\pgfusepath{stroke,fill}%
\end{pgfscope}%
\begin{pgfscope}%
\pgfpathrectangle{\pgfqpoint{0.100000in}{0.212622in}}{\pgfqpoint{3.696000in}{3.696000in}}%
\pgfusepath{clip}%
\pgfsetbuttcap%
\pgfsetroundjoin%
\definecolor{currentfill}{rgb}{0.121569,0.466667,0.705882}%
\pgfsetfillcolor{currentfill}%
\pgfsetfillopacity{0.300559}%
\pgfsetlinewidth{1.003750pt}%
\definecolor{currentstroke}{rgb}{0.121569,0.466667,0.705882}%
\pgfsetstrokecolor{currentstroke}%
\pgfsetstrokeopacity{0.300559}%
\pgfsetdash{}{0pt}%
\pgfpathmoveto{\pgfqpoint{1.676587in}{2.094768in}}%
\pgfpathcurveto{\pgfqpoint{1.684823in}{2.094768in}}{\pgfqpoint{1.692724in}{2.098040in}}{\pgfqpoint{1.698547in}{2.103864in}}%
\pgfpathcurveto{\pgfqpoint{1.704371in}{2.109688in}}{\pgfqpoint{1.707644in}{2.117588in}}{\pgfqpoint{1.707644in}{2.125824in}}%
\pgfpathcurveto{\pgfqpoint{1.707644in}{2.134060in}}{\pgfqpoint{1.704371in}{2.141961in}}{\pgfqpoint{1.698547in}{2.147784in}}%
\pgfpathcurveto{\pgfqpoint{1.692724in}{2.153608in}}{\pgfqpoint{1.684823in}{2.156881in}}{\pgfqpoint{1.676587in}{2.156881in}}%
\pgfpathcurveto{\pgfqpoint{1.668351in}{2.156881in}}{\pgfqpoint{1.660451in}{2.153608in}}{\pgfqpoint{1.654627in}{2.147784in}}%
\pgfpathcurveto{\pgfqpoint{1.648803in}{2.141961in}}{\pgfqpoint{1.645531in}{2.134060in}}{\pgfqpoint{1.645531in}{2.125824in}}%
\pgfpathcurveto{\pgfqpoint{1.645531in}{2.117588in}}{\pgfqpoint{1.648803in}{2.109688in}}{\pgfqpoint{1.654627in}{2.103864in}}%
\pgfpathcurveto{\pgfqpoint{1.660451in}{2.098040in}}{\pgfqpoint{1.668351in}{2.094768in}}{\pgfqpoint{1.676587in}{2.094768in}}%
\pgfpathclose%
\pgfusepath{stroke,fill}%
\end{pgfscope}%
\begin{pgfscope}%
\pgfpathrectangle{\pgfqpoint{0.100000in}{0.212622in}}{\pgfqpoint{3.696000in}{3.696000in}}%
\pgfusepath{clip}%
\pgfsetbuttcap%
\pgfsetroundjoin%
\definecolor{currentfill}{rgb}{0.121569,0.466667,0.705882}%
\pgfsetfillcolor{currentfill}%
\pgfsetfillopacity{0.300566}%
\pgfsetlinewidth{1.003750pt}%
\definecolor{currentstroke}{rgb}{0.121569,0.466667,0.705882}%
\pgfsetstrokecolor{currentstroke}%
\pgfsetstrokeopacity{0.300566}%
\pgfsetdash{}{0pt}%
\pgfpathmoveto{\pgfqpoint{1.676575in}{2.094768in}}%
\pgfpathcurveto{\pgfqpoint{1.684811in}{2.094768in}}{\pgfqpoint{1.692711in}{2.098040in}}{\pgfqpoint{1.698535in}{2.103864in}}%
\pgfpathcurveto{\pgfqpoint{1.704359in}{2.109688in}}{\pgfqpoint{1.707631in}{2.117588in}}{\pgfqpoint{1.707631in}{2.125824in}}%
\pgfpathcurveto{\pgfqpoint{1.707631in}{2.134060in}}{\pgfqpoint{1.704359in}{2.141961in}}{\pgfqpoint{1.698535in}{2.147784in}}%
\pgfpathcurveto{\pgfqpoint{1.692711in}{2.153608in}}{\pgfqpoint{1.684811in}{2.156881in}}{\pgfqpoint{1.676575in}{2.156881in}}%
\pgfpathcurveto{\pgfqpoint{1.668338in}{2.156881in}}{\pgfqpoint{1.660438in}{2.153608in}}{\pgfqpoint{1.654614in}{2.147784in}}%
\pgfpathcurveto{\pgfqpoint{1.648790in}{2.141961in}}{\pgfqpoint{1.645518in}{2.134060in}}{\pgfqpoint{1.645518in}{2.125824in}}%
\pgfpathcurveto{\pgfqpoint{1.645518in}{2.117588in}}{\pgfqpoint{1.648790in}{2.109688in}}{\pgfqpoint{1.654614in}{2.103864in}}%
\pgfpathcurveto{\pgfqpoint{1.660438in}{2.098040in}}{\pgfqpoint{1.668338in}{2.094768in}}{\pgfqpoint{1.676575in}{2.094768in}}%
\pgfpathclose%
\pgfusepath{stroke,fill}%
\end{pgfscope}%
\begin{pgfscope}%
\pgfpathrectangle{\pgfqpoint{0.100000in}{0.212622in}}{\pgfqpoint{3.696000in}{3.696000in}}%
\pgfusepath{clip}%
\pgfsetbuttcap%
\pgfsetroundjoin%
\definecolor{currentfill}{rgb}{0.121569,0.466667,0.705882}%
\pgfsetfillcolor{currentfill}%
\pgfsetfillopacity{0.300580}%
\pgfsetlinewidth{1.003750pt}%
\definecolor{currentstroke}{rgb}{0.121569,0.466667,0.705882}%
\pgfsetstrokecolor{currentstroke}%
\pgfsetstrokeopacity{0.300580}%
\pgfsetdash{}{0pt}%
\pgfpathmoveto{\pgfqpoint{1.676548in}{2.094769in}}%
\pgfpathcurveto{\pgfqpoint{1.684785in}{2.094769in}}{\pgfqpoint{1.692685in}{2.098041in}}{\pgfqpoint{1.698509in}{2.103865in}}%
\pgfpathcurveto{\pgfqpoint{1.704333in}{2.109689in}}{\pgfqpoint{1.707605in}{2.117589in}}{\pgfqpoint{1.707605in}{2.125825in}}%
\pgfpathcurveto{\pgfqpoint{1.707605in}{2.134061in}}{\pgfqpoint{1.704333in}{2.141962in}}{\pgfqpoint{1.698509in}{2.147785in}}%
\pgfpathcurveto{\pgfqpoint{1.692685in}{2.153609in}}{\pgfqpoint{1.684785in}{2.156882in}}{\pgfqpoint{1.676548in}{2.156882in}}%
\pgfpathcurveto{\pgfqpoint{1.668312in}{2.156882in}}{\pgfqpoint{1.660412in}{2.153609in}}{\pgfqpoint{1.654588in}{2.147785in}}%
\pgfpathcurveto{\pgfqpoint{1.648764in}{2.141962in}}{\pgfqpoint{1.645492in}{2.134061in}}{\pgfqpoint{1.645492in}{2.125825in}}%
\pgfpathcurveto{\pgfqpoint{1.645492in}{2.117589in}}{\pgfqpoint{1.648764in}{2.109689in}}{\pgfqpoint{1.654588in}{2.103865in}}%
\pgfpathcurveto{\pgfqpoint{1.660412in}{2.098041in}}{\pgfqpoint{1.668312in}{2.094769in}}{\pgfqpoint{1.676548in}{2.094769in}}%
\pgfpathclose%
\pgfusepath{stroke,fill}%
\end{pgfscope}%
\begin{pgfscope}%
\pgfpathrectangle{\pgfqpoint{0.100000in}{0.212622in}}{\pgfqpoint{3.696000in}{3.696000in}}%
\pgfusepath{clip}%
\pgfsetbuttcap%
\pgfsetroundjoin%
\definecolor{currentfill}{rgb}{0.121569,0.466667,0.705882}%
\pgfsetfillcolor{currentfill}%
\pgfsetfillopacity{0.300605}%
\pgfsetlinewidth{1.003750pt}%
\definecolor{currentstroke}{rgb}{0.121569,0.466667,0.705882}%
\pgfsetstrokecolor{currentstroke}%
\pgfsetstrokeopacity{0.300605}%
\pgfsetdash{}{0pt}%
\pgfpathmoveto{\pgfqpoint{1.676505in}{2.094770in}}%
\pgfpathcurveto{\pgfqpoint{1.684742in}{2.094770in}}{\pgfqpoint{1.692642in}{2.098043in}}{\pgfqpoint{1.698466in}{2.103867in}}%
\pgfpathcurveto{\pgfqpoint{1.704290in}{2.109691in}}{\pgfqpoint{1.707562in}{2.117591in}}{\pgfqpoint{1.707562in}{2.125827in}}%
\pgfpathcurveto{\pgfqpoint{1.707562in}{2.134063in}}{\pgfqpoint{1.704290in}{2.141963in}}{\pgfqpoint{1.698466in}{2.147787in}}%
\pgfpathcurveto{\pgfqpoint{1.692642in}{2.153611in}}{\pgfqpoint{1.684742in}{2.156883in}}{\pgfqpoint{1.676505in}{2.156883in}}%
\pgfpathcurveto{\pgfqpoint{1.668269in}{2.156883in}}{\pgfqpoint{1.660369in}{2.153611in}}{\pgfqpoint{1.654545in}{2.147787in}}%
\pgfpathcurveto{\pgfqpoint{1.648721in}{2.141963in}}{\pgfqpoint{1.645449in}{2.134063in}}{\pgfqpoint{1.645449in}{2.125827in}}%
\pgfpathcurveto{\pgfqpoint{1.645449in}{2.117591in}}{\pgfqpoint{1.648721in}{2.109691in}}{\pgfqpoint{1.654545in}{2.103867in}}%
\pgfpathcurveto{\pgfqpoint{1.660369in}{2.098043in}}{\pgfqpoint{1.668269in}{2.094770in}}{\pgfqpoint{1.676505in}{2.094770in}}%
\pgfpathclose%
\pgfusepath{stroke,fill}%
\end{pgfscope}%
\begin{pgfscope}%
\pgfpathrectangle{\pgfqpoint{0.100000in}{0.212622in}}{\pgfqpoint{3.696000in}{3.696000in}}%
\pgfusepath{clip}%
\pgfsetbuttcap%
\pgfsetroundjoin%
\definecolor{currentfill}{rgb}{0.121569,0.466667,0.705882}%
\pgfsetfillcolor{currentfill}%
\pgfsetfillopacity{0.300651}%
\pgfsetlinewidth{1.003750pt}%
\definecolor{currentstroke}{rgb}{0.121569,0.466667,0.705882}%
\pgfsetstrokecolor{currentstroke}%
\pgfsetstrokeopacity{0.300651}%
\pgfsetdash{}{0pt}%
\pgfpathmoveto{\pgfqpoint{1.676432in}{2.094768in}}%
\pgfpathcurveto{\pgfqpoint{1.684668in}{2.094768in}}{\pgfqpoint{1.692568in}{2.098041in}}{\pgfqpoint{1.698392in}{2.103865in}}%
\pgfpathcurveto{\pgfqpoint{1.704216in}{2.109689in}}{\pgfqpoint{1.707488in}{2.117589in}}{\pgfqpoint{1.707488in}{2.125825in}}%
\pgfpathcurveto{\pgfqpoint{1.707488in}{2.134061in}}{\pgfqpoint{1.704216in}{2.141961in}}{\pgfqpoint{1.698392in}{2.147785in}}%
\pgfpathcurveto{\pgfqpoint{1.692568in}{2.153609in}}{\pgfqpoint{1.684668in}{2.156881in}}{\pgfqpoint{1.676432in}{2.156881in}}%
\pgfpathcurveto{\pgfqpoint{1.668196in}{2.156881in}}{\pgfqpoint{1.660296in}{2.153609in}}{\pgfqpoint{1.654472in}{2.147785in}}%
\pgfpathcurveto{\pgfqpoint{1.648648in}{2.141961in}}{\pgfqpoint{1.645375in}{2.134061in}}{\pgfqpoint{1.645375in}{2.125825in}}%
\pgfpathcurveto{\pgfqpoint{1.645375in}{2.117589in}}{\pgfqpoint{1.648648in}{2.109689in}}{\pgfqpoint{1.654472in}{2.103865in}}%
\pgfpathcurveto{\pgfqpoint{1.660296in}{2.098041in}}{\pgfqpoint{1.668196in}{2.094768in}}{\pgfqpoint{1.676432in}{2.094768in}}%
\pgfpathclose%
\pgfusepath{stroke,fill}%
\end{pgfscope}%
\begin{pgfscope}%
\pgfpathrectangle{\pgfqpoint{0.100000in}{0.212622in}}{\pgfqpoint{3.696000in}{3.696000in}}%
\pgfusepath{clip}%
\pgfsetbuttcap%
\pgfsetroundjoin%
\definecolor{currentfill}{rgb}{0.121569,0.466667,0.705882}%
\pgfsetfillcolor{currentfill}%
\pgfsetfillopacity{0.300735}%
\pgfsetlinewidth{1.003750pt}%
\definecolor{currentstroke}{rgb}{0.121569,0.466667,0.705882}%
\pgfsetstrokecolor{currentstroke}%
\pgfsetstrokeopacity{0.300735}%
\pgfsetdash{}{0pt}%
\pgfpathmoveto{\pgfqpoint{1.676306in}{2.094771in}}%
\pgfpathcurveto{\pgfqpoint{1.684542in}{2.094771in}}{\pgfqpoint{1.692442in}{2.098043in}}{\pgfqpoint{1.698266in}{2.103867in}}%
\pgfpathcurveto{\pgfqpoint{1.704090in}{2.109691in}}{\pgfqpoint{1.707362in}{2.117591in}}{\pgfqpoint{1.707362in}{2.125827in}}%
\pgfpathcurveto{\pgfqpoint{1.707362in}{2.134064in}}{\pgfqpoint{1.704090in}{2.141964in}}{\pgfqpoint{1.698266in}{2.147788in}}%
\pgfpathcurveto{\pgfqpoint{1.692442in}{2.153612in}}{\pgfqpoint{1.684542in}{2.156884in}}{\pgfqpoint{1.676306in}{2.156884in}}%
\pgfpathcurveto{\pgfqpoint{1.668069in}{2.156884in}}{\pgfqpoint{1.660169in}{2.153612in}}{\pgfqpoint{1.654345in}{2.147788in}}%
\pgfpathcurveto{\pgfqpoint{1.648521in}{2.141964in}}{\pgfqpoint{1.645249in}{2.134064in}}{\pgfqpoint{1.645249in}{2.125827in}}%
\pgfpathcurveto{\pgfqpoint{1.645249in}{2.117591in}}{\pgfqpoint{1.648521in}{2.109691in}}{\pgfqpoint{1.654345in}{2.103867in}}%
\pgfpathcurveto{\pgfqpoint{1.660169in}{2.098043in}}{\pgfqpoint{1.668069in}{2.094771in}}{\pgfqpoint{1.676306in}{2.094771in}}%
\pgfpathclose%
\pgfusepath{stroke,fill}%
\end{pgfscope}%
\begin{pgfscope}%
\pgfpathrectangle{\pgfqpoint{0.100000in}{0.212622in}}{\pgfqpoint{3.696000in}{3.696000in}}%
\pgfusepath{clip}%
\pgfsetbuttcap%
\pgfsetroundjoin%
\definecolor{currentfill}{rgb}{0.121569,0.466667,0.705882}%
\pgfsetfillcolor{currentfill}%
\pgfsetfillopacity{0.300778}%
\pgfsetlinewidth{1.003750pt}%
\definecolor{currentstroke}{rgb}{0.121569,0.466667,0.705882}%
\pgfsetstrokecolor{currentstroke}%
\pgfsetstrokeopacity{0.300778}%
\pgfsetdash{}{0pt}%
\pgfpathmoveto{\pgfqpoint{1.689905in}{2.092747in}}%
\pgfpathcurveto{\pgfqpoint{1.698141in}{2.092747in}}{\pgfqpoint{1.706041in}{2.096019in}}{\pgfqpoint{1.711865in}{2.101843in}}%
\pgfpathcurveto{\pgfqpoint{1.717689in}{2.107667in}}{\pgfqpoint{1.720961in}{2.115567in}}{\pgfqpoint{1.720961in}{2.123804in}}%
\pgfpathcurveto{\pgfqpoint{1.720961in}{2.132040in}}{\pgfqpoint{1.717689in}{2.139940in}}{\pgfqpoint{1.711865in}{2.145764in}}%
\pgfpathcurveto{\pgfqpoint{1.706041in}{2.151588in}}{\pgfqpoint{1.698141in}{2.154860in}}{\pgfqpoint{1.689905in}{2.154860in}}%
\pgfpathcurveto{\pgfqpoint{1.681668in}{2.154860in}}{\pgfqpoint{1.673768in}{2.151588in}}{\pgfqpoint{1.667944in}{2.145764in}}%
\pgfpathcurveto{\pgfqpoint{1.662120in}{2.139940in}}{\pgfqpoint{1.658848in}{2.132040in}}{\pgfqpoint{1.658848in}{2.123804in}}%
\pgfpathcurveto{\pgfqpoint{1.658848in}{2.115567in}}{\pgfqpoint{1.662120in}{2.107667in}}{\pgfqpoint{1.667944in}{2.101843in}}%
\pgfpathcurveto{\pgfqpoint{1.673768in}{2.096019in}}{\pgfqpoint{1.681668in}{2.092747in}}{\pgfqpoint{1.689905in}{2.092747in}}%
\pgfpathclose%
\pgfusepath{stroke,fill}%
\end{pgfscope}%
\begin{pgfscope}%
\pgfpathrectangle{\pgfqpoint{0.100000in}{0.212622in}}{\pgfqpoint{3.696000in}{3.696000in}}%
\pgfusepath{clip}%
\pgfsetbuttcap%
\pgfsetroundjoin%
\definecolor{currentfill}{rgb}{0.121569,0.466667,0.705882}%
\pgfsetfillcolor{currentfill}%
\pgfsetfillopacity{0.300885}%
\pgfsetlinewidth{1.003750pt}%
\definecolor{currentstroke}{rgb}{0.121569,0.466667,0.705882}%
\pgfsetstrokecolor{currentstroke}%
\pgfsetstrokeopacity{0.300885}%
\pgfsetdash{}{0pt}%
\pgfpathmoveto{\pgfqpoint{1.676028in}{2.094789in}}%
\pgfpathcurveto{\pgfqpoint{1.684264in}{2.094789in}}{\pgfqpoint{1.692164in}{2.098061in}}{\pgfqpoint{1.697988in}{2.103885in}}%
\pgfpathcurveto{\pgfqpoint{1.703812in}{2.109709in}}{\pgfqpoint{1.707084in}{2.117609in}}{\pgfqpoint{1.707084in}{2.125845in}}%
\pgfpathcurveto{\pgfqpoint{1.707084in}{2.134081in}}{\pgfqpoint{1.703812in}{2.141981in}}{\pgfqpoint{1.697988in}{2.147805in}}%
\pgfpathcurveto{\pgfqpoint{1.692164in}{2.153629in}}{\pgfqpoint{1.684264in}{2.156902in}}{\pgfqpoint{1.676028in}{2.156902in}}%
\pgfpathcurveto{\pgfqpoint{1.667791in}{2.156902in}}{\pgfqpoint{1.659891in}{2.153629in}}{\pgfqpoint{1.654067in}{2.147805in}}%
\pgfpathcurveto{\pgfqpoint{1.648244in}{2.141981in}}{\pgfqpoint{1.644971in}{2.134081in}}{\pgfqpoint{1.644971in}{2.125845in}}%
\pgfpathcurveto{\pgfqpoint{1.644971in}{2.117609in}}{\pgfqpoint{1.648244in}{2.109709in}}{\pgfqpoint{1.654067in}{2.103885in}}%
\pgfpathcurveto{\pgfqpoint{1.659891in}{2.098061in}}{\pgfqpoint{1.667791in}{2.094789in}}{\pgfqpoint{1.676028in}{2.094789in}}%
\pgfpathclose%
\pgfusepath{stroke,fill}%
\end{pgfscope}%
\begin{pgfscope}%
\pgfpathrectangle{\pgfqpoint{0.100000in}{0.212622in}}{\pgfqpoint{3.696000in}{3.696000in}}%
\pgfusepath{clip}%
\pgfsetbuttcap%
\pgfsetroundjoin%
\definecolor{currentfill}{rgb}{0.121569,0.466667,0.705882}%
\pgfsetfillcolor{currentfill}%
\pgfsetfillopacity{0.300975}%
\pgfsetlinewidth{1.003750pt}%
\definecolor{currentstroke}{rgb}{0.121569,0.466667,0.705882}%
\pgfsetstrokecolor{currentstroke}%
\pgfsetstrokeopacity{0.300975}%
\pgfsetdash{}{0pt}%
\pgfpathmoveto{\pgfqpoint{1.692824in}{2.092214in}}%
\pgfpathcurveto{\pgfqpoint{1.701060in}{2.092214in}}{\pgfqpoint{1.708960in}{2.095486in}}{\pgfqpoint{1.714784in}{2.101310in}}%
\pgfpathcurveto{\pgfqpoint{1.720608in}{2.107134in}}{\pgfqpoint{1.723881in}{2.115034in}}{\pgfqpoint{1.723881in}{2.123271in}}%
\pgfpathcurveto{\pgfqpoint{1.723881in}{2.131507in}}{\pgfqpoint{1.720608in}{2.139407in}}{\pgfqpoint{1.714784in}{2.145231in}}%
\pgfpathcurveto{\pgfqpoint{1.708960in}{2.151055in}}{\pgfqpoint{1.701060in}{2.154327in}}{\pgfqpoint{1.692824in}{2.154327in}}%
\pgfpathcurveto{\pgfqpoint{1.684588in}{2.154327in}}{\pgfqpoint{1.676688in}{2.151055in}}{\pgfqpoint{1.670864in}{2.145231in}}%
\pgfpathcurveto{\pgfqpoint{1.665040in}{2.139407in}}{\pgfqpoint{1.661768in}{2.131507in}}{\pgfqpoint{1.661768in}{2.123271in}}%
\pgfpathcurveto{\pgfqpoint{1.661768in}{2.115034in}}{\pgfqpoint{1.665040in}{2.107134in}}{\pgfqpoint{1.670864in}{2.101310in}}%
\pgfpathcurveto{\pgfqpoint{1.676688in}{2.095486in}}{\pgfqpoint{1.684588in}{2.092214in}}{\pgfqpoint{1.692824in}{2.092214in}}%
\pgfpathclose%
\pgfusepath{stroke,fill}%
\end{pgfscope}%
\begin{pgfscope}%
\pgfpathrectangle{\pgfqpoint{0.100000in}{0.212622in}}{\pgfqpoint{3.696000in}{3.696000in}}%
\pgfusepath{clip}%
\pgfsetbuttcap%
\pgfsetroundjoin%
\definecolor{currentfill}{rgb}{0.121569,0.466667,0.705882}%
\pgfsetfillcolor{currentfill}%
\pgfsetfillopacity{0.301096}%
\pgfsetlinewidth{1.003750pt}%
\definecolor{currentstroke}{rgb}{0.121569,0.466667,0.705882}%
\pgfsetstrokecolor{currentstroke}%
\pgfsetstrokeopacity{0.301096}%
\pgfsetdash{}{0pt}%
\pgfpathmoveto{\pgfqpoint{1.694417in}{2.091960in}}%
\pgfpathcurveto{\pgfqpoint{1.702653in}{2.091960in}}{\pgfqpoint{1.710553in}{2.095232in}}{\pgfqpoint{1.716377in}{2.101056in}}%
\pgfpathcurveto{\pgfqpoint{1.722201in}{2.106880in}}{\pgfqpoint{1.725474in}{2.114780in}}{\pgfqpoint{1.725474in}{2.123016in}}%
\pgfpathcurveto{\pgfqpoint{1.725474in}{2.131252in}}{\pgfqpoint{1.722201in}{2.139152in}}{\pgfqpoint{1.716377in}{2.144976in}}%
\pgfpathcurveto{\pgfqpoint{1.710553in}{2.150800in}}{\pgfqpoint{1.702653in}{2.154073in}}{\pgfqpoint{1.694417in}{2.154073in}}%
\pgfpathcurveto{\pgfqpoint{1.686181in}{2.154073in}}{\pgfqpoint{1.678281in}{2.150800in}}{\pgfqpoint{1.672457in}{2.144976in}}%
\pgfpathcurveto{\pgfqpoint{1.666633in}{2.139152in}}{\pgfqpoint{1.663361in}{2.131252in}}{\pgfqpoint{1.663361in}{2.123016in}}%
\pgfpathcurveto{\pgfqpoint{1.663361in}{2.114780in}}{\pgfqpoint{1.666633in}{2.106880in}}{\pgfqpoint{1.672457in}{2.101056in}}%
\pgfpathcurveto{\pgfqpoint{1.678281in}{2.095232in}}{\pgfqpoint{1.686181in}{2.091960in}}{\pgfqpoint{1.694417in}{2.091960in}}%
\pgfpathclose%
\pgfusepath{stroke,fill}%
\end{pgfscope}%
\begin{pgfscope}%
\pgfpathrectangle{\pgfqpoint{0.100000in}{0.212622in}}{\pgfqpoint{3.696000in}{3.696000in}}%
\pgfusepath{clip}%
\pgfsetbuttcap%
\pgfsetroundjoin%
\definecolor{currentfill}{rgb}{0.121569,0.466667,0.705882}%
\pgfsetfillcolor{currentfill}%
\pgfsetfillopacity{0.301156}%
\pgfsetlinewidth{1.003750pt}%
\definecolor{currentstroke}{rgb}{0.121569,0.466667,0.705882}%
\pgfsetstrokecolor{currentstroke}%
\pgfsetstrokeopacity{0.301156}%
\pgfsetdash{}{0pt}%
\pgfpathmoveto{\pgfqpoint{1.675540in}{2.094788in}}%
\pgfpathcurveto{\pgfqpoint{1.683777in}{2.094788in}}{\pgfqpoint{1.691677in}{2.098060in}}{\pgfqpoint{1.697501in}{2.103884in}}%
\pgfpathcurveto{\pgfqpoint{1.703325in}{2.109708in}}{\pgfqpoint{1.706597in}{2.117608in}}{\pgfqpoint{1.706597in}{2.125844in}}%
\pgfpathcurveto{\pgfqpoint{1.706597in}{2.134081in}}{\pgfqpoint{1.703325in}{2.141981in}}{\pgfqpoint{1.697501in}{2.147805in}}%
\pgfpathcurveto{\pgfqpoint{1.691677in}{2.153628in}}{\pgfqpoint{1.683777in}{2.156901in}}{\pgfqpoint{1.675540in}{2.156901in}}%
\pgfpathcurveto{\pgfqpoint{1.667304in}{2.156901in}}{\pgfqpoint{1.659404in}{2.153628in}}{\pgfqpoint{1.653580in}{2.147805in}}%
\pgfpathcurveto{\pgfqpoint{1.647756in}{2.141981in}}{\pgfqpoint{1.644484in}{2.134081in}}{\pgfqpoint{1.644484in}{2.125844in}}%
\pgfpathcurveto{\pgfqpoint{1.644484in}{2.117608in}}{\pgfqpoint{1.647756in}{2.109708in}}{\pgfqpoint{1.653580in}{2.103884in}}%
\pgfpathcurveto{\pgfqpoint{1.659404in}{2.098060in}}{\pgfqpoint{1.667304in}{2.094788in}}{\pgfqpoint{1.675540in}{2.094788in}}%
\pgfpathclose%
\pgfusepath{stroke,fill}%
\end{pgfscope}%
\begin{pgfscope}%
\pgfpathrectangle{\pgfqpoint{0.100000in}{0.212622in}}{\pgfqpoint{3.696000in}{3.696000in}}%
\pgfusepath{clip}%
\pgfsetbuttcap%
\pgfsetroundjoin%
\definecolor{currentfill}{rgb}{0.121569,0.466667,0.705882}%
\pgfsetfillcolor{currentfill}%
\pgfsetfillopacity{0.301300}%
\pgfsetlinewidth{1.003750pt}%
\definecolor{currentstroke}{rgb}{0.121569,0.466667,0.705882}%
\pgfsetstrokecolor{currentstroke}%
\pgfsetstrokeopacity{0.301300}%
\pgfsetdash{}{0pt}%
\pgfpathmoveto{\pgfqpoint{1.696697in}{2.091596in}}%
\pgfpathcurveto{\pgfqpoint{1.704934in}{2.091596in}}{\pgfqpoint{1.712834in}{2.094868in}}{\pgfqpoint{1.718658in}{2.100692in}}%
\pgfpathcurveto{\pgfqpoint{1.724481in}{2.106516in}}{\pgfqpoint{1.727754in}{2.114416in}}{\pgfqpoint{1.727754in}{2.122653in}}%
\pgfpathcurveto{\pgfqpoint{1.727754in}{2.130889in}}{\pgfqpoint{1.724481in}{2.138789in}}{\pgfqpoint{1.718658in}{2.144613in}}%
\pgfpathcurveto{\pgfqpoint{1.712834in}{2.150437in}}{\pgfqpoint{1.704934in}{2.153709in}}{\pgfqpoint{1.696697in}{2.153709in}}%
\pgfpathcurveto{\pgfqpoint{1.688461in}{2.153709in}}{\pgfqpoint{1.680561in}{2.150437in}}{\pgfqpoint{1.674737in}{2.144613in}}%
\pgfpathcurveto{\pgfqpoint{1.668913in}{2.138789in}}{\pgfqpoint{1.665641in}{2.130889in}}{\pgfqpoint{1.665641in}{2.122653in}}%
\pgfpathcurveto{\pgfqpoint{1.665641in}{2.114416in}}{\pgfqpoint{1.668913in}{2.106516in}}{\pgfqpoint{1.674737in}{2.100692in}}%
\pgfpathcurveto{\pgfqpoint{1.680561in}{2.094868in}}{\pgfqpoint{1.688461in}{2.091596in}}{\pgfqpoint{1.696697in}{2.091596in}}%
\pgfpathclose%
\pgfusepath{stroke,fill}%
\end{pgfscope}%
\begin{pgfscope}%
\pgfpathrectangle{\pgfqpoint{0.100000in}{0.212622in}}{\pgfqpoint{3.696000in}{3.696000in}}%
\pgfusepath{clip}%
\pgfsetbuttcap%
\pgfsetroundjoin%
\definecolor{currentfill}{rgb}{0.121569,0.466667,0.705882}%
\pgfsetfillcolor{currentfill}%
\pgfsetfillopacity{0.301470}%
\pgfsetlinewidth{1.003750pt}%
\definecolor{currentstroke}{rgb}{0.121569,0.466667,0.705882}%
\pgfsetstrokecolor{currentstroke}%
\pgfsetstrokeopacity{0.301470}%
\pgfsetdash{}{0pt}%
\pgfpathmoveto{\pgfqpoint{1.697843in}{2.091469in}}%
\pgfpathcurveto{\pgfqpoint{1.706079in}{2.091469in}}{\pgfqpoint{1.713979in}{2.094741in}}{\pgfqpoint{1.719803in}{2.100565in}}%
\pgfpathcurveto{\pgfqpoint{1.725627in}{2.106389in}}{\pgfqpoint{1.728900in}{2.114289in}}{\pgfqpoint{1.728900in}{2.122525in}}%
\pgfpathcurveto{\pgfqpoint{1.728900in}{2.130761in}}{\pgfqpoint{1.725627in}{2.138661in}}{\pgfqpoint{1.719803in}{2.144485in}}%
\pgfpathcurveto{\pgfqpoint{1.713979in}{2.150309in}}{\pgfqpoint{1.706079in}{2.153582in}}{\pgfqpoint{1.697843in}{2.153582in}}%
\pgfpathcurveto{\pgfqpoint{1.689607in}{2.153582in}}{\pgfqpoint{1.681707in}{2.150309in}}{\pgfqpoint{1.675883in}{2.144485in}}%
\pgfpathcurveto{\pgfqpoint{1.670059in}{2.138661in}}{\pgfqpoint{1.666787in}{2.130761in}}{\pgfqpoint{1.666787in}{2.122525in}}%
\pgfpathcurveto{\pgfqpoint{1.666787in}{2.114289in}}{\pgfqpoint{1.670059in}{2.106389in}}{\pgfqpoint{1.675883in}{2.100565in}}%
\pgfpathcurveto{\pgfqpoint{1.681707in}{2.094741in}}{\pgfqpoint{1.689607in}{2.091469in}}{\pgfqpoint{1.697843in}{2.091469in}}%
\pgfpathclose%
\pgfusepath{stroke,fill}%
\end{pgfscope}%
\begin{pgfscope}%
\pgfpathrectangle{\pgfqpoint{0.100000in}{0.212622in}}{\pgfqpoint{3.696000in}{3.696000in}}%
\pgfusepath{clip}%
\pgfsetbuttcap%
\pgfsetroundjoin%
\definecolor{currentfill}{rgb}{0.121569,0.466667,0.705882}%
\pgfsetfillcolor{currentfill}%
\pgfsetfillopacity{0.301631}%
\pgfsetlinewidth{1.003750pt}%
\definecolor{currentstroke}{rgb}{0.121569,0.466667,0.705882}%
\pgfsetstrokecolor{currentstroke}%
\pgfsetstrokeopacity{0.301631}%
\pgfsetdash{}{0pt}%
\pgfpathmoveto{\pgfqpoint{1.674490in}{2.094806in}}%
\pgfpathcurveto{\pgfqpoint{1.682726in}{2.094806in}}{\pgfqpoint{1.690627in}{2.098078in}}{\pgfqpoint{1.696450in}{2.103902in}}%
\pgfpathcurveto{\pgfqpoint{1.702274in}{2.109726in}}{\pgfqpoint{1.705547in}{2.117626in}}{\pgfqpoint{1.705547in}{2.125863in}}%
\pgfpathcurveto{\pgfqpoint{1.705547in}{2.134099in}}{\pgfqpoint{1.702274in}{2.141999in}}{\pgfqpoint{1.696450in}{2.147823in}}%
\pgfpathcurveto{\pgfqpoint{1.690627in}{2.153647in}}{\pgfqpoint{1.682726in}{2.156919in}}{\pgfqpoint{1.674490in}{2.156919in}}%
\pgfpathcurveto{\pgfqpoint{1.666254in}{2.156919in}}{\pgfqpoint{1.658354in}{2.153647in}}{\pgfqpoint{1.652530in}{2.147823in}}%
\pgfpathcurveto{\pgfqpoint{1.646706in}{2.141999in}}{\pgfqpoint{1.643434in}{2.134099in}}{\pgfqpoint{1.643434in}{2.125863in}}%
\pgfpathcurveto{\pgfqpoint{1.643434in}{2.117626in}}{\pgfqpoint{1.646706in}{2.109726in}}{\pgfqpoint{1.652530in}{2.103902in}}%
\pgfpathcurveto{\pgfqpoint{1.658354in}{2.098078in}}{\pgfqpoint{1.666254in}{2.094806in}}{\pgfqpoint{1.674490in}{2.094806in}}%
\pgfpathclose%
\pgfusepath{stroke,fill}%
\end{pgfscope}%
\begin{pgfscope}%
\pgfpathrectangle{\pgfqpoint{0.100000in}{0.212622in}}{\pgfqpoint{3.696000in}{3.696000in}}%
\pgfusepath{clip}%
\pgfsetbuttcap%
\pgfsetroundjoin%
\definecolor{currentfill}{rgb}{0.121569,0.466667,0.705882}%
\pgfsetfillcolor{currentfill}%
\pgfsetfillopacity{0.301648}%
\pgfsetlinewidth{1.003750pt}%
\definecolor{currentstroke}{rgb}{0.121569,0.466667,0.705882}%
\pgfsetstrokecolor{currentstroke}%
\pgfsetstrokeopacity{0.301648}%
\pgfsetdash{}{0pt}%
\pgfpathmoveto{\pgfqpoint{1.699636in}{2.091253in}}%
\pgfpathcurveto{\pgfqpoint{1.707872in}{2.091253in}}{\pgfqpoint{1.715772in}{2.094525in}}{\pgfqpoint{1.721596in}{2.100349in}}%
\pgfpathcurveto{\pgfqpoint{1.727420in}{2.106173in}}{\pgfqpoint{1.730692in}{2.114073in}}{\pgfqpoint{1.730692in}{2.122309in}}%
\pgfpathcurveto{\pgfqpoint{1.730692in}{2.130545in}}{\pgfqpoint{1.727420in}{2.138445in}}{\pgfqpoint{1.721596in}{2.144269in}}%
\pgfpathcurveto{\pgfqpoint{1.715772in}{2.150093in}}{\pgfqpoint{1.707872in}{2.153366in}}{\pgfqpoint{1.699636in}{2.153366in}}%
\pgfpathcurveto{\pgfqpoint{1.691400in}{2.153366in}}{\pgfqpoint{1.683500in}{2.150093in}}{\pgfqpoint{1.677676in}{2.144269in}}%
\pgfpathcurveto{\pgfqpoint{1.671852in}{2.138445in}}{\pgfqpoint{1.668579in}{2.130545in}}{\pgfqpoint{1.668579in}{2.122309in}}%
\pgfpathcurveto{\pgfqpoint{1.668579in}{2.114073in}}{\pgfqpoint{1.671852in}{2.106173in}}{\pgfqpoint{1.677676in}{2.100349in}}%
\pgfpathcurveto{\pgfqpoint{1.683500in}{2.094525in}}{\pgfqpoint{1.691400in}{2.091253in}}{\pgfqpoint{1.699636in}{2.091253in}}%
\pgfpathclose%
\pgfusepath{stroke,fill}%
\end{pgfscope}%
\begin{pgfscope}%
\pgfpathrectangle{\pgfqpoint{0.100000in}{0.212622in}}{\pgfqpoint{3.696000in}{3.696000in}}%
\pgfusepath{clip}%
\pgfsetbuttcap%
\pgfsetroundjoin%
\definecolor{currentfill}{rgb}{0.121569,0.466667,0.705882}%
\pgfsetfillcolor{currentfill}%
\pgfsetfillopacity{0.301828}%
\pgfsetlinewidth{1.003750pt}%
\definecolor{currentstroke}{rgb}{0.121569,0.466667,0.705882}%
\pgfsetstrokecolor{currentstroke}%
\pgfsetstrokeopacity{0.301828}%
\pgfsetdash{}{0pt}%
\pgfpathmoveto{\pgfqpoint{1.701767in}{2.090884in}}%
\pgfpathcurveto{\pgfqpoint{1.710003in}{2.090884in}}{\pgfqpoint{1.717903in}{2.094156in}}{\pgfqpoint{1.723727in}{2.099980in}}%
\pgfpathcurveto{\pgfqpoint{1.729551in}{2.105804in}}{\pgfqpoint{1.732823in}{2.113704in}}{\pgfqpoint{1.732823in}{2.121940in}}%
\pgfpathcurveto{\pgfqpoint{1.732823in}{2.130177in}}{\pgfqpoint{1.729551in}{2.138077in}}{\pgfqpoint{1.723727in}{2.143901in}}%
\pgfpathcurveto{\pgfqpoint{1.717903in}{2.149725in}}{\pgfqpoint{1.710003in}{2.152997in}}{\pgfqpoint{1.701767in}{2.152997in}}%
\pgfpathcurveto{\pgfqpoint{1.693531in}{2.152997in}}{\pgfqpoint{1.685631in}{2.149725in}}{\pgfqpoint{1.679807in}{2.143901in}}%
\pgfpathcurveto{\pgfqpoint{1.673983in}{2.138077in}}{\pgfqpoint{1.670710in}{2.130177in}}{\pgfqpoint{1.670710in}{2.121940in}}%
\pgfpathcurveto{\pgfqpoint{1.670710in}{2.113704in}}{\pgfqpoint{1.673983in}{2.105804in}}{\pgfqpoint{1.679807in}{2.099980in}}%
\pgfpathcurveto{\pgfqpoint{1.685631in}{2.094156in}}{\pgfqpoint{1.693531in}{2.090884in}}{\pgfqpoint{1.701767in}{2.090884in}}%
\pgfpathclose%
\pgfusepath{stroke,fill}%
\end{pgfscope}%
\begin{pgfscope}%
\pgfpathrectangle{\pgfqpoint{0.100000in}{0.212622in}}{\pgfqpoint{3.696000in}{3.696000in}}%
\pgfusepath{clip}%
\pgfsetbuttcap%
\pgfsetroundjoin%
\definecolor{currentfill}{rgb}{0.121569,0.466667,0.705882}%
\pgfsetfillcolor{currentfill}%
\pgfsetfillopacity{0.302042}%
\pgfsetlinewidth{1.003750pt}%
\definecolor{currentstroke}{rgb}{0.121569,0.466667,0.705882}%
\pgfsetstrokecolor{currentstroke}%
\pgfsetstrokeopacity{0.302042}%
\pgfsetdash{}{0pt}%
\pgfpathmoveto{\pgfqpoint{1.673748in}{2.094846in}}%
\pgfpathcurveto{\pgfqpoint{1.681984in}{2.094846in}}{\pgfqpoint{1.689884in}{2.098118in}}{\pgfqpoint{1.695708in}{2.103942in}}%
\pgfpathcurveto{\pgfqpoint{1.701532in}{2.109766in}}{\pgfqpoint{1.704805in}{2.117666in}}{\pgfqpoint{1.704805in}{2.125902in}}%
\pgfpathcurveto{\pgfqpoint{1.704805in}{2.134139in}}{\pgfqpoint{1.701532in}{2.142039in}}{\pgfqpoint{1.695708in}{2.147863in}}%
\pgfpathcurveto{\pgfqpoint{1.689884in}{2.153686in}}{\pgfqpoint{1.681984in}{2.156959in}}{\pgfqpoint{1.673748in}{2.156959in}}%
\pgfpathcurveto{\pgfqpoint{1.665512in}{2.156959in}}{\pgfqpoint{1.657612in}{2.153686in}}{\pgfqpoint{1.651788in}{2.147863in}}%
\pgfpathcurveto{\pgfqpoint{1.645964in}{2.142039in}}{\pgfqpoint{1.642692in}{2.134139in}}{\pgfqpoint{1.642692in}{2.125902in}}%
\pgfpathcurveto{\pgfqpoint{1.642692in}{2.117666in}}{\pgfqpoint{1.645964in}{2.109766in}}{\pgfqpoint{1.651788in}{2.103942in}}%
\pgfpathcurveto{\pgfqpoint{1.657612in}{2.098118in}}{\pgfqpoint{1.665512in}{2.094846in}}{\pgfqpoint{1.673748in}{2.094846in}}%
\pgfpathclose%
\pgfusepath{stroke,fill}%
\end{pgfscope}%
\begin{pgfscope}%
\pgfpathrectangle{\pgfqpoint{0.100000in}{0.212622in}}{\pgfqpoint{3.696000in}{3.696000in}}%
\pgfusepath{clip}%
\pgfsetbuttcap%
\pgfsetroundjoin%
\definecolor{currentfill}{rgb}{0.121569,0.466667,0.705882}%
\pgfsetfillcolor{currentfill}%
\pgfsetfillopacity{0.302052}%
\pgfsetlinewidth{1.003750pt}%
\definecolor{currentstroke}{rgb}{0.121569,0.466667,0.705882}%
\pgfsetstrokecolor{currentstroke}%
\pgfsetstrokeopacity{0.302052}%
\pgfsetdash{}{0pt}%
\pgfpathmoveto{\pgfqpoint{1.704573in}{2.090290in}}%
\pgfpathcurveto{\pgfqpoint{1.712809in}{2.090290in}}{\pgfqpoint{1.720709in}{2.093563in}}{\pgfqpoint{1.726533in}{2.099387in}}%
\pgfpathcurveto{\pgfqpoint{1.732357in}{2.105210in}}{\pgfqpoint{1.735629in}{2.113110in}}{\pgfqpoint{1.735629in}{2.121347in}}%
\pgfpathcurveto{\pgfqpoint{1.735629in}{2.129583in}}{\pgfqpoint{1.732357in}{2.137483in}}{\pgfqpoint{1.726533in}{2.143307in}}%
\pgfpathcurveto{\pgfqpoint{1.720709in}{2.149131in}}{\pgfqpoint{1.712809in}{2.152403in}}{\pgfqpoint{1.704573in}{2.152403in}}%
\pgfpathcurveto{\pgfqpoint{1.696337in}{2.152403in}}{\pgfqpoint{1.688437in}{2.149131in}}{\pgfqpoint{1.682613in}{2.143307in}}%
\pgfpathcurveto{\pgfqpoint{1.676789in}{2.137483in}}{\pgfqpoint{1.673516in}{2.129583in}}{\pgfqpoint{1.673516in}{2.121347in}}%
\pgfpathcurveto{\pgfqpoint{1.673516in}{2.113110in}}{\pgfqpoint{1.676789in}{2.105210in}}{\pgfqpoint{1.682613in}{2.099387in}}%
\pgfpathcurveto{\pgfqpoint{1.688437in}{2.093563in}}{\pgfqpoint{1.696337in}{2.090290in}}{\pgfqpoint{1.704573in}{2.090290in}}%
\pgfpathclose%
\pgfusepath{stroke,fill}%
\end{pgfscope}%
\begin{pgfscope}%
\pgfpathrectangle{\pgfqpoint{0.100000in}{0.212622in}}{\pgfqpoint{3.696000in}{3.696000in}}%
\pgfusepath{clip}%
\pgfsetbuttcap%
\pgfsetroundjoin%
\definecolor{currentfill}{rgb}{0.121569,0.466667,0.705882}%
\pgfsetfillcolor{currentfill}%
\pgfsetfillopacity{0.302315}%
\pgfsetlinewidth{1.003750pt}%
\definecolor{currentstroke}{rgb}{0.121569,0.466667,0.705882}%
\pgfsetstrokecolor{currentstroke}%
\pgfsetstrokeopacity{0.302315}%
\pgfsetdash{}{0pt}%
\pgfpathmoveto{\pgfqpoint{1.673118in}{2.094883in}}%
\pgfpathcurveto{\pgfqpoint{1.681354in}{2.094883in}}{\pgfqpoint{1.689254in}{2.098155in}}{\pgfqpoint{1.695078in}{2.103979in}}%
\pgfpathcurveto{\pgfqpoint{1.700902in}{2.109803in}}{\pgfqpoint{1.704174in}{2.117703in}}{\pgfqpoint{1.704174in}{2.125940in}}%
\pgfpathcurveto{\pgfqpoint{1.704174in}{2.134176in}}{\pgfqpoint{1.700902in}{2.142076in}}{\pgfqpoint{1.695078in}{2.147900in}}%
\pgfpathcurveto{\pgfqpoint{1.689254in}{2.153724in}}{\pgfqpoint{1.681354in}{2.156996in}}{\pgfqpoint{1.673118in}{2.156996in}}%
\pgfpathcurveto{\pgfqpoint{1.664881in}{2.156996in}}{\pgfqpoint{1.656981in}{2.153724in}}{\pgfqpoint{1.651157in}{2.147900in}}%
\pgfpathcurveto{\pgfqpoint{1.645334in}{2.142076in}}{\pgfqpoint{1.642061in}{2.134176in}}{\pgfqpoint{1.642061in}{2.125940in}}%
\pgfpathcurveto{\pgfqpoint{1.642061in}{2.117703in}}{\pgfqpoint{1.645334in}{2.109803in}}{\pgfqpoint{1.651157in}{2.103979in}}%
\pgfpathcurveto{\pgfqpoint{1.656981in}{2.098155in}}{\pgfqpoint{1.664881in}{2.094883in}}{\pgfqpoint{1.673118in}{2.094883in}}%
\pgfpathclose%
\pgfusepath{stroke,fill}%
\end{pgfscope}%
\begin{pgfscope}%
\pgfpathrectangle{\pgfqpoint{0.100000in}{0.212622in}}{\pgfqpoint{3.696000in}{3.696000in}}%
\pgfusepath{clip}%
\pgfsetbuttcap%
\pgfsetroundjoin%
\definecolor{currentfill}{rgb}{0.121569,0.466667,0.705882}%
\pgfsetfillcolor{currentfill}%
\pgfsetfillopacity{0.302453}%
\pgfsetlinewidth{1.003750pt}%
\definecolor{currentstroke}{rgb}{0.121569,0.466667,0.705882}%
\pgfsetstrokecolor{currentstroke}%
\pgfsetstrokeopacity{0.302453}%
\pgfsetdash{}{0pt}%
\pgfpathmoveto{\pgfqpoint{1.707838in}{2.089814in}}%
\pgfpathcurveto{\pgfqpoint{1.716074in}{2.089814in}}{\pgfqpoint{1.723974in}{2.093086in}}{\pgfqpoint{1.729798in}{2.098910in}}%
\pgfpathcurveto{\pgfqpoint{1.735622in}{2.104734in}}{\pgfqpoint{1.738894in}{2.112634in}}{\pgfqpoint{1.738894in}{2.120871in}}%
\pgfpathcurveto{\pgfqpoint{1.738894in}{2.129107in}}{\pgfqpoint{1.735622in}{2.137007in}}{\pgfqpoint{1.729798in}{2.142831in}}%
\pgfpathcurveto{\pgfqpoint{1.723974in}{2.148655in}}{\pgfqpoint{1.716074in}{2.151927in}}{\pgfqpoint{1.707838in}{2.151927in}}%
\pgfpathcurveto{\pgfqpoint{1.699601in}{2.151927in}}{\pgfqpoint{1.691701in}{2.148655in}}{\pgfqpoint{1.685877in}{2.142831in}}%
\pgfpathcurveto{\pgfqpoint{1.680053in}{2.137007in}}{\pgfqpoint{1.676781in}{2.129107in}}{\pgfqpoint{1.676781in}{2.120871in}}%
\pgfpathcurveto{\pgfqpoint{1.676781in}{2.112634in}}{\pgfqpoint{1.680053in}{2.104734in}}{\pgfqpoint{1.685877in}{2.098910in}}%
\pgfpathcurveto{\pgfqpoint{1.691701in}{2.093086in}}{\pgfqpoint{1.699601in}{2.089814in}}{\pgfqpoint{1.707838in}{2.089814in}}%
\pgfpathclose%
\pgfusepath{stroke,fill}%
\end{pgfscope}%
\begin{pgfscope}%
\pgfpathrectangle{\pgfqpoint{0.100000in}{0.212622in}}{\pgfqpoint{3.696000in}{3.696000in}}%
\pgfusepath{clip}%
\pgfsetbuttcap%
\pgfsetroundjoin%
\definecolor{currentfill}{rgb}{0.121569,0.466667,0.705882}%
\pgfsetfillcolor{currentfill}%
\pgfsetfillopacity{0.302672}%
\pgfsetlinewidth{1.003750pt}%
\definecolor{currentstroke}{rgb}{0.121569,0.466667,0.705882}%
\pgfsetstrokecolor{currentstroke}%
\pgfsetstrokeopacity{0.302672}%
\pgfsetdash{}{0pt}%
\pgfpathmoveto{\pgfqpoint{1.709643in}{2.089568in}}%
\pgfpathcurveto{\pgfqpoint{1.717880in}{2.089568in}}{\pgfqpoint{1.725780in}{2.092840in}}{\pgfqpoint{1.731604in}{2.098664in}}%
\pgfpathcurveto{\pgfqpoint{1.737428in}{2.104488in}}{\pgfqpoint{1.740700in}{2.112388in}}{\pgfqpoint{1.740700in}{2.120625in}}%
\pgfpathcurveto{\pgfqpoint{1.740700in}{2.128861in}}{\pgfqpoint{1.737428in}{2.136761in}}{\pgfqpoint{1.731604in}{2.142585in}}%
\pgfpathcurveto{\pgfqpoint{1.725780in}{2.148409in}}{\pgfqpoint{1.717880in}{2.151681in}}{\pgfqpoint{1.709643in}{2.151681in}}%
\pgfpathcurveto{\pgfqpoint{1.701407in}{2.151681in}}{\pgfqpoint{1.693507in}{2.148409in}}{\pgfqpoint{1.687683in}{2.142585in}}%
\pgfpathcurveto{\pgfqpoint{1.681859in}{2.136761in}}{\pgfqpoint{1.678587in}{2.128861in}}{\pgfqpoint{1.678587in}{2.120625in}}%
\pgfpathcurveto{\pgfqpoint{1.678587in}{2.112388in}}{\pgfqpoint{1.681859in}{2.104488in}}{\pgfqpoint{1.687683in}{2.098664in}}%
\pgfpathcurveto{\pgfqpoint{1.693507in}{2.092840in}}{\pgfqpoint{1.701407in}{2.089568in}}{\pgfqpoint{1.709643in}{2.089568in}}%
\pgfpathclose%
\pgfusepath{stroke,fill}%
\end{pgfscope}%
\begin{pgfscope}%
\pgfpathrectangle{\pgfqpoint{0.100000in}{0.212622in}}{\pgfqpoint{3.696000in}{3.696000in}}%
\pgfusepath{clip}%
\pgfsetbuttcap%
\pgfsetroundjoin%
\definecolor{currentfill}{rgb}{0.121569,0.466667,0.705882}%
\pgfsetfillcolor{currentfill}%
\pgfsetfillopacity{0.302854}%
\pgfsetlinewidth{1.003750pt}%
\definecolor{currentstroke}{rgb}{0.121569,0.466667,0.705882}%
\pgfsetstrokecolor{currentstroke}%
\pgfsetstrokeopacity{0.302854}%
\pgfsetdash{}{0pt}%
\pgfpathmoveto{\pgfqpoint{1.712716in}{2.088901in}}%
\pgfpathcurveto{\pgfqpoint{1.720952in}{2.088901in}}{\pgfqpoint{1.728852in}{2.092174in}}{\pgfqpoint{1.734676in}{2.097998in}}%
\pgfpathcurveto{\pgfqpoint{1.740500in}{2.103822in}}{\pgfqpoint{1.743773in}{2.111722in}}{\pgfqpoint{1.743773in}{2.119958in}}%
\pgfpathcurveto{\pgfqpoint{1.743773in}{2.128194in}}{\pgfqpoint{1.740500in}{2.136094in}}{\pgfqpoint{1.734676in}{2.141918in}}%
\pgfpathcurveto{\pgfqpoint{1.728852in}{2.147742in}}{\pgfqpoint{1.720952in}{2.151014in}}{\pgfqpoint{1.712716in}{2.151014in}}%
\pgfpathcurveto{\pgfqpoint{1.704480in}{2.151014in}}{\pgfqpoint{1.696580in}{2.147742in}}{\pgfqpoint{1.690756in}{2.141918in}}%
\pgfpathcurveto{\pgfqpoint{1.684932in}{2.136094in}}{\pgfqpoint{1.681660in}{2.128194in}}{\pgfqpoint{1.681660in}{2.119958in}}%
\pgfpathcurveto{\pgfqpoint{1.681660in}{2.111722in}}{\pgfqpoint{1.684932in}{2.103822in}}{\pgfqpoint{1.690756in}{2.097998in}}%
\pgfpathcurveto{\pgfqpoint{1.696580in}{2.092174in}}{\pgfqpoint{1.704480in}{2.088901in}}{\pgfqpoint{1.712716in}{2.088901in}}%
\pgfpathclose%
\pgfusepath{stroke,fill}%
\end{pgfscope}%
\begin{pgfscope}%
\pgfpathrectangle{\pgfqpoint{0.100000in}{0.212622in}}{\pgfqpoint{3.696000in}{3.696000in}}%
\pgfusepath{clip}%
\pgfsetbuttcap%
\pgfsetroundjoin%
\definecolor{currentfill}{rgb}{0.121569,0.466667,0.705882}%
\pgfsetfillcolor{currentfill}%
\pgfsetfillopacity{0.302856}%
\pgfsetlinewidth{1.003750pt}%
\definecolor{currentstroke}{rgb}{0.121569,0.466667,0.705882}%
\pgfsetstrokecolor{currentstroke}%
\pgfsetstrokeopacity{0.302856}%
\pgfsetdash{}{0pt}%
\pgfpathmoveto{\pgfqpoint{1.672441in}{2.094915in}}%
\pgfpathcurveto{\pgfqpoint{1.680677in}{2.094915in}}{\pgfqpoint{1.688577in}{2.098188in}}{\pgfqpoint{1.694401in}{2.104012in}}%
\pgfpathcurveto{\pgfqpoint{1.700225in}{2.109835in}}{\pgfqpoint{1.703497in}{2.117736in}}{\pgfqpoint{1.703497in}{2.125972in}}%
\pgfpathcurveto{\pgfqpoint{1.703497in}{2.134208in}}{\pgfqpoint{1.700225in}{2.142108in}}{\pgfqpoint{1.694401in}{2.147932in}}%
\pgfpathcurveto{\pgfqpoint{1.688577in}{2.153756in}}{\pgfqpoint{1.680677in}{2.157028in}}{\pgfqpoint{1.672441in}{2.157028in}}%
\pgfpathcurveto{\pgfqpoint{1.664204in}{2.157028in}}{\pgfqpoint{1.656304in}{2.153756in}}{\pgfqpoint{1.650480in}{2.147932in}}%
\pgfpathcurveto{\pgfqpoint{1.644656in}{2.142108in}}{\pgfqpoint{1.641384in}{2.134208in}}{\pgfqpoint{1.641384in}{2.125972in}}%
\pgfpathcurveto{\pgfqpoint{1.641384in}{2.117736in}}{\pgfqpoint{1.644656in}{2.109835in}}{\pgfqpoint{1.650480in}{2.104012in}}%
\pgfpathcurveto{\pgfqpoint{1.656304in}{2.098188in}}{\pgfqpoint{1.664204in}{2.094915in}}{\pgfqpoint{1.672441in}{2.094915in}}%
\pgfpathclose%
\pgfusepath{stroke,fill}%
\end{pgfscope}%
\begin{pgfscope}%
\pgfpathrectangle{\pgfqpoint{0.100000in}{0.212622in}}{\pgfqpoint{3.696000in}{3.696000in}}%
\pgfusepath{clip}%
\pgfsetbuttcap%
\pgfsetroundjoin%
\definecolor{currentfill}{rgb}{0.121569,0.466667,0.705882}%
\pgfsetfillcolor{currentfill}%
\pgfsetfillopacity{0.303298}%
\pgfsetlinewidth{1.003750pt}%
\definecolor{currentstroke}{rgb}{0.121569,0.466667,0.705882}%
\pgfsetstrokecolor{currentstroke}%
\pgfsetstrokeopacity{0.303298}%
\pgfsetdash{}{0pt}%
\pgfpathmoveto{\pgfqpoint{1.671644in}{2.094919in}}%
\pgfpathcurveto{\pgfqpoint{1.679880in}{2.094919in}}{\pgfqpoint{1.687780in}{2.098192in}}{\pgfqpoint{1.693604in}{2.104016in}}%
\pgfpathcurveto{\pgfqpoint{1.699428in}{2.109840in}}{\pgfqpoint{1.702700in}{2.117740in}}{\pgfqpoint{1.702700in}{2.125976in}}%
\pgfpathcurveto{\pgfqpoint{1.702700in}{2.134212in}}{\pgfqpoint{1.699428in}{2.142112in}}{\pgfqpoint{1.693604in}{2.147936in}}%
\pgfpathcurveto{\pgfqpoint{1.687780in}{2.153760in}}{\pgfqpoint{1.679880in}{2.157032in}}{\pgfqpoint{1.671644in}{2.157032in}}%
\pgfpathcurveto{\pgfqpoint{1.663407in}{2.157032in}}{\pgfqpoint{1.655507in}{2.153760in}}{\pgfqpoint{1.649683in}{2.147936in}}%
\pgfpathcurveto{\pgfqpoint{1.643859in}{2.142112in}}{\pgfqpoint{1.640587in}{2.134212in}}{\pgfqpoint{1.640587in}{2.125976in}}%
\pgfpathcurveto{\pgfqpoint{1.640587in}{2.117740in}}{\pgfqpoint{1.643859in}{2.109840in}}{\pgfqpoint{1.649683in}{2.104016in}}%
\pgfpathcurveto{\pgfqpoint{1.655507in}{2.098192in}}{\pgfqpoint{1.663407in}{2.094919in}}{\pgfqpoint{1.671644in}{2.094919in}}%
\pgfpathclose%
\pgfusepath{stroke,fill}%
\end{pgfscope}%
\begin{pgfscope}%
\pgfpathrectangle{\pgfqpoint{0.100000in}{0.212622in}}{\pgfqpoint{3.696000in}{3.696000in}}%
\pgfusepath{clip}%
\pgfsetbuttcap%
\pgfsetroundjoin%
\definecolor{currentfill}{rgb}{0.121569,0.466667,0.705882}%
\pgfsetfillcolor{currentfill}%
\pgfsetfillopacity{0.303313}%
\pgfsetlinewidth{1.003750pt}%
\definecolor{currentstroke}{rgb}{0.121569,0.466667,0.705882}%
\pgfsetstrokecolor{currentstroke}%
\pgfsetstrokeopacity{0.303313}%
\pgfsetdash{}{0pt}%
\pgfpathmoveto{\pgfqpoint{1.717703in}{2.088192in}}%
\pgfpathcurveto{\pgfqpoint{1.725940in}{2.088192in}}{\pgfqpoint{1.733840in}{2.091464in}}{\pgfqpoint{1.739664in}{2.097288in}}%
\pgfpathcurveto{\pgfqpoint{1.745488in}{2.103112in}}{\pgfqpoint{1.748760in}{2.111012in}}{\pgfqpoint{1.748760in}{2.119248in}}%
\pgfpathcurveto{\pgfqpoint{1.748760in}{2.127485in}}{\pgfqpoint{1.745488in}{2.135385in}}{\pgfqpoint{1.739664in}{2.141209in}}%
\pgfpathcurveto{\pgfqpoint{1.733840in}{2.147033in}}{\pgfqpoint{1.725940in}{2.150305in}}{\pgfqpoint{1.717703in}{2.150305in}}%
\pgfpathcurveto{\pgfqpoint{1.709467in}{2.150305in}}{\pgfqpoint{1.701567in}{2.147033in}}{\pgfqpoint{1.695743in}{2.141209in}}%
\pgfpathcurveto{\pgfqpoint{1.689919in}{2.135385in}}{\pgfqpoint{1.686647in}{2.127485in}}{\pgfqpoint{1.686647in}{2.119248in}}%
\pgfpathcurveto{\pgfqpoint{1.686647in}{2.111012in}}{\pgfqpoint{1.689919in}{2.103112in}}{\pgfqpoint{1.695743in}{2.097288in}}%
\pgfpathcurveto{\pgfqpoint{1.701567in}{2.091464in}}{\pgfqpoint{1.709467in}{2.088192in}}{\pgfqpoint{1.717703in}{2.088192in}}%
\pgfpathclose%
\pgfusepath{stroke,fill}%
\end{pgfscope}%
\begin{pgfscope}%
\pgfpathrectangle{\pgfqpoint{0.100000in}{0.212622in}}{\pgfqpoint{3.696000in}{3.696000in}}%
\pgfusepath{clip}%
\pgfsetbuttcap%
\pgfsetroundjoin%
\definecolor{currentfill}{rgb}{0.121569,0.466667,0.705882}%
\pgfsetfillcolor{currentfill}%
\pgfsetfillopacity{0.303582}%
\pgfsetlinewidth{1.003750pt}%
\definecolor{currentstroke}{rgb}{0.121569,0.466667,0.705882}%
\pgfsetstrokecolor{currentstroke}%
\pgfsetstrokeopacity{0.303582}%
\pgfsetdash{}{0pt}%
\pgfpathmoveto{\pgfqpoint{1.720421in}{2.087817in}}%
\pgfpathcurveto{\pgfqpoint{1.728658in}{2.087817in}}{\pgfqpoint{1.736558in}{2.091089in}}{\pgfqpoint{1.742382in}{2.096913in}}%
\pgfpathcurveto{\pgfqpoint{1.748206in}{2.102737in}}{\pgfqpoint{1.751478in}{2.110637in}}{\pgfqpoint{1.751478in}{2.118874in}}%
\pgfpathcurveto{\pgfqpoint{1.751478in}{2.127110in}}{\pgfqpoint{1.748206in}{2.135010in}}{\pgfqpoint{1.742382in}{2.140834in}}%
\pgfpathcurveto{\pgfqpoint{1.736558in}{2.146658in}}{\pgfqpoint{1.728658in}{2.149930in}}{\pgfqpoint{1.720421in}{2.149930in}}%
\pgfpathcurveto{\pgfqpoint{1.712185in}{2.149930in}}{\pgfqpoint{1.704285in}{2.146658in}}{\pgfqpoint{1.698461in}{2.140834in}}%
\pgfpathcurveto{\pgfqpoint{1.692637in}{2.135010in}}{\pgfqpoint{1.689365in}{2.127110in}}{\pgfqpoint{1.689365in}{2.118874in}}%
\pgfpathcurveto{\pgfqpoint{1.689365in}{2.110637in}}{\pgfqpoint{1.692637in}{2.102737in}}{\pgfqpoint{1.698461in}{2.096913in}}%
\pgfpathcurveto{\pgfqpoint{1.704285in}{2.091089in}}{\pgfqpoint{1.712185in}{2.087817in}}{\pgfqpoint{1.720421in}{2.087817in}}%
\pgfpathclose%
\pgfusepath{stroke,fill}%
\end{pgfscope}%
\begin{pgfscope}%
\pgfpathrectangle{\pgfqpoint{0.100000in}{0.212622in}}{\pgfqpoint{3.696000in}{3.696000in}}%
\pgfusepath{clip}%
\pgfsetbuttcap%
\pgfsetroundjoin%
\definecolor{currentfill}{rgb}{0.121569,0.466667,0.705882}%
\pgfsetfillcolor{currentfill}%
\pgfsetfillopacity{0.303643}%
\pgfsetlinewidth{1.003750pt}%
\definecolor{currentstroke}{rgb}{0.121569,0.466667,0.705882}%
\pgfsetstrokecolor{currentstroke}%
\pgfsetstrokeopacity{0.303643}%
\pgfsetdash{}{0pt}%
\pgfpathmoveto{\pgfqpoint{1.670823in}{2.094986in}}%
\pgfpathcurveto{\pgfqpoint{1.679059in}{2.094986in}}{\pgfqpoint{1.686959in}{2.098258in}}{\pgfqpoint{1.692783in}{2.104082in}}%
\pgfpathcurveto{\pgfqpoint{1.698607in}{2.109906in}}{\pgfqpoint{1.701879in}{2.117806in}}{\pgfqpoint{1.701879in}{2.126043in}}%
\pgfpathcurveto{\pgfqpoint{1.701879in}{2.134279in}}{\pgfqpoint{1.698607in}{2.142179in}}{\pgfqpoint{1.692783in}{2.148003in}}%
\pgfpathcurveto{\pgfqpoint{1.686959in}{2.153827in}}{\pgfqpoint{1.679059in}{2.157099in}}{\pgfqpoint{1.670823in}{2.157099in}}%
\pgfpathcurveto{\pgfqpoint{1.662587in}{2.157099in}}{\pgfqpoint{1.654687in}{2.153827in}}{\pgfqpoint{1.648863in}{2.148003in}}%
\pgfpathcurveto{\pgfqpoint{1.643039in}{2.142179in}}{\pgfqpoint{1.639766in}{2.134279in}}{\pgfqpoint{1.639766in}{2.126043in}}%
\pgfpathcurveto{\pgfqpoint{1.639766in}{2.117806in}}{\pgfqpoint{1.643039in}{2.109906in}}{\pgfqpoint{1.648863in}{2.104082in}}%
\pgfpathcurveto{\pgfqpoint{1.654687in}{2.098258in}}{\pgfqpoint{1.662587in}{2.094986in}}{\pgfqpoint{1.670823in}{2.094986in}}%
\pgfpathclose%
\pgfusepath{stroke,fill}%
\end{pgfscope}%
\begin{pgfscope}%
\pgfpathrectangle{\pgfqpoint{0.100000in}{0.212622in}}{\pgfqpoint{3.696000in}{3.696000in}}%
\pgfusepath{clip}%
\pgfsetbuttcap%
\pgfsetroundjoin%
\definecolor{currentfill}{rgb}{0.121569,0.466667,0.705882}%
\pgfsetfillcolor{currentfill}%
\pgfsetfillopacity{0.303794}%
\pgfsetlinewidth{1.003750pt}%
\definecolor{currentstroke}{rgb}{0.121569,0.466667,0.705882}%
\pgfsetstrokecolor{currentstroke}%
\pgfsetstrokeopacity{0.303794}%
\pgfsetdash{}{0pt}%
\pgfpathmoveto{\pgfqpoint{1.670598in}{2.094993in}}%
\pgfpathcurveto{\pgfqpoint{1.678834in}{2.094993in}}{\pgfqpoint{1.686734in}{2.098266in}}{\pgfqpoint{1.692558in}{2.104090in}}%
\pgfpathcurveto{\pgfqpoint{1.698382in}{2.109913in}}{\pgfqpoint{1.701654in}{2.117814in}}{\pgfqpoint{1.701654in}{2.126050in}}%
\pgfpathcurveto{\pgfqpoint{1.701654in}{2.134286in}}{\pgfqpoint{1.698382in}{2.142186in}}{\pgfqpoint{1.692558in}{2.148010in}}%
\pgfpathcurveto{\pgfqpoint{1.686734in}{2.153834in}}{\pgfqpoint{1.678834in}{2.157106in}}{\pgfqpoint{1.670598in}{2.157106in}}%
\pgfpathcurveto{\pgfqpoint{1.662361in}{2.157106in}}{\pgfqpoint{1.654461in}{2.153834in}}{\pgfqpoint{1.648637in}{2.148010in}}%
\pgfpathcurveto{\pgfqpoint{1.642814in}{2.142186in}}{\pgfqpoint{1.639541in}{2.134286in}}{\pgfqpoint{1.639541in}{2.126050in}}%
\pgfpathcurveto{\pgfqpoint{1.639541in}{2.117814in}}{\pgfqpoint{1.642814in}{2.109913in}}{\pgfqpoint{1.648637in}{2.104090in}}%
\pgfpathcurveto{\pgfqpoint{1.654461in}{2.098266in}}{\pgfqpoint{1.662361in}{2.094993in}}{\pgfqpoint{1.670598in}{2.094993in}}%
\pgfpathclose%
\pgfusepath{stroke,fill}%
\end{pgfscope}%
\begin{pgfscope}%
\pgfpathrectangle{\pgfqpoint{0.100000in}{0.212622in}}{\pgfqpoint{3.696000in}{3.696000in}}%
\pgfusepath{clip}%
\pgfsetbuttcap%
\pgfsetroundjoin%
\definecolor{currentfill}{rgb}{0.121569,0.466667,0.705882}%
\pgfsetfillcolor{currentfill}%
\pgfsetfillopacity{0.304050}%
\pgfsetlinewidth{1.003750pt}%
\definecolor{currentstroke}{rgb}{0.121569,0.466667,0.705882}%
\pgfsetstrokecolor{currentstroke}%
\pgfsetstrokeopacity{0.304050}%
\pgfsetdash{}{0pt}%
\pgfpathmoveto{\pgfqpoint{1.670023in}{2.095003in}}%
\pgfpathcurveto{\pgfqpoint{1.678260in}{2.095003in}}{\pgfqpoint{1.686160in}{2.098275in}}{\pgfqpoint{1.691984in}{2.104099in}}%
\pgfpathcurveto{\pgfqpoint{1.697807in}{2.109923in}}{\pgfqpoint{1.701080in}{2.117823in}}{\pgfqpoint{1.701080in}{2.126059in}}%
\pgfpathcurveto{\pgfqpoint{1.701080in}{2.134296in}}{\pgfqpoint{1.697807in}{2.142196in}}{\pgfqpoint{1.691984in}{2.148020in}}%
\pgfpathcurveto{\pgfqpoint{1.686160in}{2.153843in}}{\pgfqpoint{1.678260in}{2.157116in}}{\pgfqpoint{1.670023in}{2.157116in}}%
\pgfpathcurveto{\pgfqpoint{1.661787in}{2.157116in}}{\pgfqpoint{1.653887in}{2.153843in}}{\pgfqpoint{1.648063in}{2.148020in}}%
\pgfpathcurveto{\pgfqpoint{1.642239in}{2.142196in}}{\pgfqpoint{1.638967in}{2.134296in}}{\pgfqpoint{1.638967in}{2.126059in}}%
\pgfpathcurveto{\pgfqpoint{1.638967in}{2.117823in}}{\pgfqpoint{1.642239in}{2.109923in}}{\pgfqpoint{1.648063in}{2.104099in}}%
\pgfpathcurveto{\pgfqpoint{1.653887in}{2.098275in}}{\pgfqpoint{1.661787in}{2.095003in}}{\pgfqpoint{1.670023in}{2.095003in}}%
\pgfpathclose%
\pgfusepath{stroke,fill}%
\end{pgfscope}%
\begin{pgfscope}%
\pgfpathrectangle{\pgfqpoint{0.100000in}{0.212622in}}{\pgfqpoint{3.696000in}{3.696000in}}%
\pgfusepath{clip}%
\pgfsetbuttcap%
\pgfsetroundjoin%
\definecolor{currentfill}{rgb}{0.121569,0.466667,0.705882}%
\pgfsetfillcolor{currentfill}%
\pgfsetfillopacity{0.304090}%
\pgfsetlinewidth{1.003750pt}%
\definecolor{currentstroke}{rgb}{0.121569,0.466667,0.705882}%
\pgfsetstrokecolor{currentstroke}%
\pgfsetstrokeopacity{0.304090}%
\pgfsetdash{}{0pt}%
\pgfpathmoveto{\pgfqpoint{1.723742in}{2.087433in}}%
\pgfpathcurveto{\pgfqpoint{1.731979in}{2.087433in}}{\pgfqpoint{1.739879in}{2.090705in}}{\pgfqpoint{1.745703in}{2.096529in}}%
\pgfpathcurveto{\pgfqpoint{1.751527in}{2.102353in}}{\pgfqpoint{1.754799in}{2.110253in}}{\pgfqpoint{1.754799in}{2.118490in}}%
\pgfpathcurveto{\pgfqpoint{1.754799in}{2.126726in}}{\pgfqpoint{1.751527in}{2.134626in}}{\pgfqpoint{1.745703in}{2.140450in}}%
\pgfpathcurveto{\pgfqpoint{1.739879in}{2.146274in}}{\pgfqpoint{1.731979in}{2.149546in}}{\pgfqpoint{1.723742in}{2.149546in}}%
\pgfpathcurveto{\pgfqpoint{1.715506in}{2.149546in}}{\pgfqpoint{1.707606in}{2.146274in}}{\pgfqpoint{1.701782in}{2.140450in}}%
\pgfpathcurveto{\pgfqpoint{1.695958in}{2.134626in}}{\pgfqpoint{1.692686in}{2.126726in}}{\pgfqpoint{1.692686in}{2.118490in}}%
\pgfpathcurveto{\pgfqpoint{1.692686in}{2.110253in}}{\pgfqpoint{1.695958in}{2.102353in}}{\pgfqpoint{1.701782in}{2.096529in}}%
\pgfpathcurveto{\pgfqpoint{1.707606in}{2.090705in}}{\pgfqpoint{1.715506in}{2.087433in}}{\pgfqpoint{1.723742in}{2.087433in}}%
\pgfpathclose%
\pgfusepath{stroke,fill}%
\end{pgfscope}%
\begin{pgfscope}%
\pgfpathrectangle{\pgfqpoint{0.100000in}{0.212622in}}{\pgfqpoint{3.696000in}{3.696000in}}%
\pgfusepath{clip}%
\pgfsetbuttcap%
\pgfsetroundjoin%
\definecolor{currentfill}{rgb}{0.121569,0.466667,0.705882}%
\pgfsetfillcolor{currentfill}%
\pgfsetfillopacity{0.304399}%
\pgfsetlinewidth{1.003750pt}%
\definecolor{currentstroke}{rgb}{0.121569,0.466667,0.705882}%
\pgfsetstrokecolor{currentstroke}%
\pgfsetstrokeopacity{0.304399}%
\pgfsetdash{}{0pt}%
\pgfpathmoveto{\pgfqpoint{1.728017in}{2.086600in}}%
\pgfpathcurveto{\pgfqpoint{1.736254in}{2.086600in}}{\pgfqpoint{1.744154in}{2.089873in}}{\pgfqpoint{1.749978in}{2.095697in}}%
\pgfpathcurveto{\pgfqpoint{1.755802in}{2.101520in}}{\pgfqpoint{1.759074in}{2.109421in}}{\pgfqpoint{1.759074in}{2.117657in}}%
\pgfpathcurveto{\pgfqpoint{1.759074in}{2.125893in}}{\pgfqpoint{1.755802in}{2.133793in}}{\pgfqpoint{1.749978in}{2.139617in}}%
\pgfpathcurveto{\pgfqpoint{1.744154in}{2.145441in}}{\pgfqpoint{1.736254in}{2.148713in}}{\pgfqpoint{1.728017in}{2.148713in}}%
\pgfpathcurveto{\pgfqpoint{1.719781in}{2.148713in}}{\pgfqpoint{1.711881in}{2.145441in}}{\pgfqpoint{1.706057in}{2.139617in}}%
\pgfpathcurveto{\pgfqpoint{1.700233in}{2.133793in}}{\pgfqpoint{1.696961in}{2.125893in}}{\pgfqpoint{1.696961in}{2.117657in}}%
\pgfpathcurveto{\pgfqpoint{1.696961in}{2.109421in}}{\pgfqpoint{1.700233in}{2.101520in}}{\pgfqpoint{1.706057in}{2.095697in}}%
\pgfpathcurveto{\pgfqpoint{1.711881in}{2.089873in}}{\pgfqpoint{1.719781in}{2.086600in}}{\pgfqpoint{1.728017in}{2.086600in}}%
\pgfpathclose%
\pgfusepath{stroke,fill}%
\end{pgfscope}%
\begin{pgfscope}%
\pgfpathrectangle{\pgfqpoint{0.100000in}{0.212622in}}{\pgfqpoint{3.696000in}{3.696000in}}%
\pgfusepath{clip}%
\pgfsetbuttcap%
\pgfsetroundjoin%
\definecolor{currentfill}{rgb}{0.121569,0.466667,0.705882}%
\pgfsetfillcolor{currentfill}%
\pgfsetfillopacity{0.304517}%
\pgfsetlinewidth{1.003750pt}%
\definecolor{currentstroke}{rgb}{0.121569,0.466667,0.705882}%
\pgfsetstrokecolor{currentstroke}%
\pgfsetstrokeopacity{0.304517}%
\pgfsetdash{}{0pt}%
\pgfpathmoveto{\pgfqpoint{1.669004in}{2.095004in}}%
\pgfpathcurveto{\pgfqpoint{1.677240in}{2.095004in}}{\pgfqpoint{1.685140in}{2.098276in}}{\pgfqpoint{1.690964in}{2.104100in}}%
\pgfpathcurveto{\pgfqpoint{1.696788in}{2.109924in}}{\pgfqpoint{1.700060in}{2.117824in}}{\pgfqpoint{1.700060in}{2.126061in}}%
\pgfpathcurveto{\pgfqpoint{1.700060in}{2.134297in}}{\pgfqpoint{1.696788in}{2.142197in}}{\pgfqpoint{1.690964in}{2.148021in}}%
\pgfpathcurveto{\pgfqpoint{1.685140in}{2.153845in}}{\pgfqpoint{1.677240in}{2.157117in}}{\pgfqpoint{1.669004in}{2.157117in}}%
\pgfpathcurveto{\pgfqpoint{1.660767in}{2.157117in}}{\pgfqpoint{1.652867in}{2.153845in}}{\pgfqpoint{1.647043in}{2.148021in}}%
\pgfpathcurveto{\pgfqpoint{1.641219in}{2.142197in}}{\pgfqpoint{1.637947in}{2.134297in}}{\pgfqpoint{1.637947in}{2.126061in}}%
\pgfpathcurveto{\pgfqpoint{1.637947in}{2.117824in}}{\pgfqpoint{1.641219in}{2.109924in}}{\pgfqpoint{1.647043in}{2.104100in}}%
\pgfpathcurveto{\pgfqpoint{1.652867in}{2.098276in}}{\pgfqpoint{1.660767in}{2.095004in}}{\pgfqpoint{1.669004in}{2.095004in}}%
\pgfpathclose%
\pgfusepath{stroke,fill}%
\end{pgfscope}%
\begin{pgfscope}%
\pgfpathrectangle{\pgfqpoint{0.100000in}{0.212622in}}{\pgfqpoint{3.696000in}{3.696000in}}%
\pgfusepath{clip}%
\pgfsetbuttcap%
\pgfsetroundjoin%
\definecolor{currentfill}{rgb}{0.121569,0.466667,0.705882}%
\pgfsetfillcolor{currentfill}%
\pgfsetfillopacity{0.304678}%
\pgfsetlinewidth{1.003750pt}%
\definecolor{currentstroke}{rgb}{0.121569,0.466667,0.705882}%
\pgfsetstrokecolor{currentstroke}%
\pgfsetstrokeopacity{0.304678}%
\pgfsetdash{}{0pt}%
\pgfpathmoveto{\pgfqpoint{1.733217in}{2.085732in}}%
\pgfpathcurveto{\pgfqpoint{1.741454in}{2.085732in}}{\pgfqpoint{1.749354in}{2.089005in}}{\pgfqpoint{1.755178in}{2.094828in}}%
\pgfpathcurveto{\pgfqpoint{1.761001in}{2.100652in}}{\pgfqpoint{1.764274in}{2.108552in}}{\pgfqpoint{1.764274in}{2.116789in}}%
\pgfpathcurveto{\pgfqpoint{1.764274in}{2.125025in}}{\pgfqpoint{1.761001in}{2.132925in}}{\pgfqpoint{1.755178in}{2.138749in}}%
\pgfpathcurveto{\pgfqpoint{1.749354in}{2.144573in}}{\pgfqpoint{1.741454in}{2.147845in}}{\pgfqpoint{1.733217in}{2.147845in}}%
\pgfpathcurveto{\pgfqpoint{1.724981in}{2.147845in}}{\pgfqpoint{1.717081in}{2.144573in}}{\pgfqpoint{1.711257in}{2.138749in}}%
\pgfpathcurveto{\pgfqpoint{1.705433in}{2.132925in}}{\pgfqpoint{1.702161in}{2.125025in}}{\pgfqpoint{1.702161in}{2.116789in}}%
\pgfpathcurveto{\pgfqpoint{1.702161in}{2.108552in}}{\pgfqpoint{1.705433in}{2.100652in}}{\pgfqpoint{1.711257in}{2.094828in}}%
\pgfpathcurveto{\pgfqpoint{1.717081in}{2.089005in}}{\pgfqpoint{1.724981in}{2.085732in}}{\pgfqpoint{1.733217in}{2.085732in}}%
\pgfpathclose%
\pgfusepath{stroke,fill}%
\end{pgfscope}%
\begin{pgfscope}%
\pgfpathrectangle{\pgfqpoint{0.100000in}{0.212622in}}{\pgfqpoint{3.696000in}{3.696000in}}%
\pgfusepath{clip}%
\pgfsetbuttcap%
\pgfsetroundjoin%
\definecolor{currentfill}{rgb}{0.121569,0.466667,0.705882}%
\pgfsetfillcolor{currentfill}%
\pgfsetfillopacity{0.304738}%
\pgfsetlinewidth{1.003750pt}%
\definecolor{currentstroke}{rgb}{0.121569,0.466667,0.705882}%
\pgfsetstrokecolor{currentstroke}%
\pgfsetstrokeopacity{0.304738}%
\pgfsetdash{}{0pt}%
\pgfpathmoveto{\pgfqpoint{1.668672in}{2.095014in}}%
\pgfpathcurveto{\pgfqpoint{1.676908in}{2.095014in}}{\pgfqpoint{1.684808in}{2.098287in}}{\pgfqpoint{1.690632in}{2.104111in}}%
\pgfpathcurveto{\pgfqpoint{1.696456in}{2.109935in}}{\pgfqpoint{1.699728in}{2.117835in}}{\pgfqpoint{1.699728in}{2.126071in}}%
\pgfpathcurveto{\pgfqpoint{1.699728in}{2.134307in}}{\pgfqpoint{1.696456in}{2.142207in}}{\pgfqpoint{1.690632in}{2.148031in}}%
\pgfpathcurveto{\pgfqpoint{1.684808in}{2.153855in}}{\pgfqpoint{1.676908in}{2.157127in}}{\pgfqpoint{1.668672in}{2.157127in}}%
\pgfpathcurveto{\pgfqpoint{1.660436in}{2.157127in}}{\pgfqpoint{1.652536in}{2.153855in}}{\pgfqpoint{1.646712in}{2.148031in}}%
\pgfpathcurveto{\pgfqpoint{1.640888in}{2.142207in}}{\pgfqpoint{1.637615in}{2.134307in}}{\pgfqpoint{1.637615in}{2.126071in}}%
\pgfpathcurveto{\pgfqpoint{1.637615in}{2.117835in}}{\pgfqpoint{1.640888in}{2.109935in}}{\pgfqpoint{1.646712in}{2.104111in}}%
\pgfpathcurveto{\pgfqpoint{1.652536in}{2.098287in}}{\pgfqpoint{1.660436in}{2.095014in}}{\pgfqpoint{1.668672in}{2.095014in}}%
\pgfpathclose%
\pgfusepath{stroke,fill}%
\end{pgfscope}%
\begin{pgfscope}%
\pgfpathrectangle{\pgfqpoint{0.100000in}{0.212622in}}{\pgfqpoint{3.696000in}{3.696000in}}%
\pgfusepath{clip}%
\pgfsetbuttcap%
\pgfsetroundjoin%
\definecolor{currentfill}{rgb}{0.121569,0.466667,0.705882}%
\pgfsetfillcolor{currentfill}%
\pgfsetfillopacity{0.304881}%
\pgfsetlinewidth{1.003750pt}%
\definecolor{currentstroke}{rgb}{0.121569,0.466667,0.705882}%
\pgfsetstrokecolor{currentstroke}%
\pgfsetstrokeopacity{0.304881}%
\pgfsetdash{}{0pt}%
\pgfpathmoveto{\pgfqpoint{1.736015in}{2.085292in}}%
\pgfpathcurveto{\pgfqpoint{1.744251in}{2.085292in}}{\pgfqpoint{1.752151in}{2.088565in}}{\pgfqpoint{1.757975in}{2.094389in}}%
\pgfpathcurveto{\pgfqpoint{1.763799in}{2.100213in}}{\pgfqpoint{1.767072in}{2.108113in}}{\pgfqpoint{1.767072in}{2.116349in}}%
\pgfpathcurveto{\pgfqpoint{1.767072in}{2.124585in}}{\pgfqpoint{1.763799in}{2.132485in}}{\pgfqpoint{1.757975in}{2.138309in}}%
\pgfpathcurveto{\pgfqpoint{1.752151in}{2.144133in}}{\pgfqpoint{1.744251in}{2.147405in}}{\pgfqpoint{1.736015in}{2.147405in}}%
\pgfpathcurveto{\pgfqpoint{1.727779in}{2.147405in}}{\pgfqpoint{1.719879in}{2.144133in}}{\pgfqpoint{1.714055in}{2.138309in}}%
\pgfpathcurveto{\pgfqpoint{1.708231in}{2.132485in}}{\pgfqpoint{1.704959in}{2.124585in}}{\pgfqpoint{1.704959in}{2.116349in}}%
\pgfpathcurveto{\pgfqpoint{1.704959in}{2.108113in}}{\pgfqpoint{1.708231in}{2.100213in}}{\pgfqpoint{1.714055in}{2.094389in}}%
\pgfpathcurveto{\pgfqpoint{1.719879in}{2.088565in}}{\pgfqpoint{1.727779in}{2.085292in}}{\pgfqpoint{1.736015in}{2.085292in}}%
\pgfpathclose%
\pgfusepath{stroke,fill}%
\end{pgfscope}%
\begin{pgfscope}%
\pgfpathrectangle{\pgfqpoint{0.100000in}{0.212622in}}{\pgfqpoint{3.696000in}{3.696000in}}%
\pgfusepath{clip}%
\pgfsetbuttcap%
\pgfsetroundjoin%
\definecolor{currentfill}{rgb}{0.121569,0.466667,0.705882}%
\pgfsetfillcolor{currentfill}%
\pgfsetfillopacity{0.305112}%
\pgfsetlinewidth{1.003750pt}%
\definecolor{currentstroke}{rgb}{0.121569,0.466667,0.705882}%
\pgfsetstrokecolor{currentstroke}%
\pgfsetstrokeopacity{0.305112}%
\pgfsetdash{}{0pt}%
\pgfpathmoveto{\pgfqpoint{1.667813in}{2.095053in}}%
\pgfpathcurveto{\pgfqpoint{1.676049in}{2.095053in}}{\pgfqpoint{1.683949in}{2.098325in}}{\pgfqpoint{1.689773in}{2.104149in}}%
\pgfpathcurveto{\pgfqpoint{1.695597in}{2.109973in}}{\pgfqpoint{1.698869in}{2.117873in}}{\pgfqpoint{1.698869in}{2.126109in}}%
\pgfpathcurveto{\pgfqpoint{1.698869in}{2.134345in}}{\pgfqpoint{1.695597in}{2.142245in}}{\pgfqpoint{1.689773in}{2.148069in}}%
\pgfpathcurveto{\pgfqpoint{1.683949in}{2.153893in}}{\pgfqpoint{1.676049in}{2.157166in}}{\pgfqpoint{1.667813in}{2.157166in}}%
\pgfpathcurveto{\pgfqpoint{1.659576in}{2.157166in}}{\pgfqpoint{1.651676in}{2.153893in}}{\pgfqpoint{1.645852in}{2.148069in}}%
\pgfpathcurveto{\pgfqpoint{1.640028in}{2.142245in}}{\pgfqpoint{1.636756in}{2.134345in}}{\pgfqpoint{1.636756in}{2.126109in}}%
\pgfpathcurveto{\pgfqpoint{1.636756in}{2.117873in}}{\pgfqpoint{1.640028in}{2.109973in}}{\pgfqpoint{1.645852in}{2.104149in}}%
\pgfpathcurveto{\pgfqpoint{1.651676in}{2.098325in}}{\pgfqpoint{1.659576in}{2.095053in}}{\pgfqpoint{1.667813in}{2.095053in}}%
\pgfpathclose%
\pgfusepath{stroke,fill}%
\end{pgfscope}%
\begin{pgfscope}%
\pgfpathrectangle{\pgfqpoint{0.100000in}{0.212622in}}{\pgfqpoint{3.696000in}{3.696000in}}%
\pgfusepath{clip}%
\pgfsetbuttcap%
\pgfsetroundjoin%
\definecolor{currentfill}{rgb}{0.121569,0.466667,0.705882}%
\pgfsetfillcolor{currentfill}%
\pgfsetfillopacity{0.305223}%
\pgfsetlinewidth{1.003750pt}%
\definecolor{currentstroke}{rgb}{0.121569,0.466667,0.705882}%
\pgfsetstrokecolor{currentstroke}%
\pgfsetstrokeopacity{0.305223}%
\pgfsetdash{}{0pt}%
\pgfpathmoveto{\pgfqpoint{1.738937in}{2.084896in}}%
\pgfpathcurveto{\pgfqpoint{1.747174in}{2.084896in}}{\pgfqpoint{1.755074in}{2.088168in}}{\pgfqpoint{1.760898in}{2.093992in}}%
\pgfpathcurveto{\pgfqpoint{1.766722in}{2.099816in}}{\pgfqpoint{1.769994in}{2.107716in}}{\pgfqpoint{1.769994in}{2.115953in}}%
\pgfpathcurveto{\pgfqpoint{1.769994in}{2.124189in}}{\pgfqpoint{1.766722in}{2.132089in}}{\pgfqpoint{1.760898in}{2.137913in}}%
\pgfpathcurveto{\pgfqpoint{1.755074in}{2.143737in}}{\pgfqpoint{1.747174in}{2.147009in}}{\pgfqpoint{1.738937in}{2.147009in}}%
\pgfpathcurveto{\pgfqpoint{1.730701in}{2.147009in}}{\pgfqpoint{1.722801in}{2.143737in}}{\pgfqpoint{1.716977in}{2.137913in}}%
\pgfpathcurveto{\pgfqpoint{1.711153in}{2.132089in}}{\pgfqpoint{1.707881in}{2.124189in}}{\pgfqpoint{1.707881in}{2.115953in}}%
\pgfpathcurveto{\pgfqpoint{1.707881in}{2.107716in}}{\pgfqpoint{1.711153in}{2.099816in}}{\pgfqpoint{1.716977in}{2.093992in}}%
\pgfpathcurveto{\pgfqpoint{1.722801in}{2.088168in}}{\pgfqpoint{1.730701in}{2.084896in}}{\pgfqpoint{1.738937in}{2.084896in}}%
\pgfpathclose%
\pgfusepath{stroke,fill}%
\end{pgfscope}%
\begin{pgfscope}%
\pgfpathrectangle{\pgfqpoint{0.100000in}{0.212622in}}{\pgfqpoint{3.696000in}{3.696000in}}%
\pgfusepath{clip}%
\pgfsetbuttcap%
\pgfsetroundjoin%
\definecolor{currentfill}{rgb}{0.121569,0.466667,0.705882}%
\pgfsetfillcolor{currentfill}%
\pgfsetfillopacity{0.305469}%
\pgfsetlinewidth{1.003750pt}%
\definecolor{currentstroke}{rgb}{0.121569,0.466667,0.705882}%
\pgfsetstrokecolor{currentstroke}%
\pgfsetstrokeopacity{0.305469}%
\pgfsetdash{}{0pt}%
\pgfpathmoveto{\pgfqpoint{1.742582in}{2.084294in}}%
\pgfpathcurveto{\pgfqpoint{1.750818in}{2.084294in}}{\pgfqpoint{1.758718in}{2.087567in}}{\pgfqpoint{1.764542in}{2.093391in}}%
\pgfpathcurveto{\pgfqpoint{1.770366in}{2.099214in}}{\pgfqpoint{1.773638in}{2.107114in}}{\pgfqpoint{1.773638in}{2.115351in}}%
\pgfpathcurveto{\pgfqpoint{1.773638in}{2.123587in}}{\pgfqpoint{1.770366in}{2.131487in}}{\pgfqpoint{1.764542in}{2.137311in}}%
\pgfpathcurveto{\pgfqpoint{1.758718in}{2.143135in}}{\pgfqpoint{1.750818in}{2.146407in}}{\pgfqpoint{1.742582in}{2.146407in}}%
\pgfpathcurveto{\pgfqpoint{1.734346in}{2.146407in}}{\pgfqpoint{1.726446in}{2.143135in}}{\pgfqpoint{1.720622in}{2.137311in}}%
\pgfpathcurveto{\pgfqpoint{1.714798in}{2.131487in}}{\pgfqpoint{1.711525in}{2.123587in}}{\pgfqpoint{1.711525in}{2.115351in}}%
\pgfpathcurveto{\pgfqpoint{1.711525in}{2.107114in}}{\pgfqpoint{1.714798in}{2.099214in}}{\pgfqpoint{1.720622in}{2.093391in}}%
\pgfpathcurveto{\pgfqpoint{1.726446in}{2.087567in}}{\pgfqpoint{1.734346in}{2.084294in}}{\pgfqpoint{1.742582in}{2.084294in}}%
\pgfpathclose%
\pgfusepath{stroke,fill}%
\end{pgfscope}%
\begin{pgfscope}%
\pgfpathrectangle{\pgfqpoint{0.100000in}{0.212622in}}{\pgfqpoint{3.696000in}{3.696000in}}%
\pgfusepath{clip}%
\pgfsetbuttcap%
\pgfsetroundjoin%
\definecolor{currentfill}{rgb}{0.121569,0.466667,0.705882}%
\pgfsetfillcolor{currentfill}%
\pgfsetfillopacity{0.305808}%
\pgfsetlinewidth{1.003750pt}%
\definecolor{currentstroke}{rgb}{0.121569,0.466667,0.705882}%
\pgfsetstrokecolor{currentstroke}%
\pgfsetstrokeopacity{0.305808}%
\pgfsetdash{}{0pt}%
\pgfpathmoveto{\pgfqpoint{1.666413in}{2.095072in}}%
\pgfpathcurveto{\pgfqpoint{1.674650in}{2.095072in}}{\pgfqpoint{1.682550in}{2.098344in}}{\pgfqpoint{1.688374in}{2.104168in}}%
\pgfpathcurveto{\pgfqpoint{1.694197in}{2.109992in}}{\pgfqpoint{1.697470in}{2.117892in}}{\pgfqpoint{1.697470in}{2.126128in}}%
\pgfpathcurveto{\pgfqpoint{1.697470in}{2.134364in}}{\pgfqpoint{1.694197in}{2.142264in}}{\pgfqpoint{1.688374in}{2.148088in}}%
\pgfpathcurveto{\pgfqpoint{1.682550in}{2.153912in}}{\pgfqpoint{1.674650in}{2.157185in}}{\pgfqpoint{1.666413in}{2.157185in}}%
\pgfpathcurveto{\pgfqpoint{1.658177in}{2.157185in}}{\pgfqpoint{1.650277in}{2.153912in}}{\pgfqpoint{1.644453in}{2.148088in}}%
\pgfpathcurveto{\pgfqpoint{1.638629in}{2.142264in}}{\pgfqpoint{1.635357in}{2.134364in}}{\pgfqpoint{1.635357in}{2.126128in}}%
\pgfpathcurveto{\pgfqpoint{1.635357in}{2.117892in}}{\pgfqpoint{1.638629in}{2.109992in}}{\pgfqpoint{1.644453in}{2.104168in}}%
\pgfpathcurveto{\pgfqpoint{1.650277in}{2.098344in}}{\pgfqpoint{1.658177in}{2.095072in}}{\pgfqpoint{1.666413in}{2.095072in}}%
\pgfpathclose%
\pgfusepath{stroke,fill}%
\end{pgfscope}%
\begin{pgfscope}%
\pgfpathrectangle{\pgfqpoint{0.100000in}{0.212622in}}{\pgfqpoint{3.696000in}{3.696000in}}%
\pgfusepath{clip}%
\pgfsetbuttcap%
\pgfsetroundjoin%
\definecolor{currentfill}{rgb}{0.121569,0.466667,0.705882}%
\pgfsetfillcolor{currentfill}%
\pgfsetfillopacity{0.306023}%
\pgfsetlinewidth{1.003750pt}%
\definecolor{currentstroke}{rgb}{0.121569,0.466667,0.705882}%
\pgfsetstrokecolor{currentstroke}%
\pgfsetstrokeopacity{0.306023}%
\pgfsetdash{}{0pt}%
\pgfpathmoveto{\pgfqpoint{1.747533in}{2.083665in}}%
\pgfpathcurveto{\pgfqpoint{1.755769in}{2.083665in}}{\pgfqpoint{1.763669in}{2.086937in}}{\pgfqpoint{1.769493in}{2.092761in}}%
\pgfpathcurveto{\pgfqpoint{1.775317in}{2.098585in}}{\pgfqpoint{1.778590in}{2.106485in}}{\pgfqpoint{1.778590in}{2.114721in}}%
\pgfpathcurveto{\pgfqpoint{1.778590in}{2.122958in}}{\pgfqpoint{1.775317in}{2.130858in}}{\pgfqpoint{1.769493in}{2.136682in}}%
\pgfpathcurveto{\pgfqpoint{1.763669in}{2.142506in}}{\pgfqpoint{1.755769in}{2.145778in}}{\pgfqpoint{1.747533in}{2.145778in}}%
\pgfpathcurveto{\pgfqpoint{1.739297in}{2.145778in}}{\pgfqpoint{1.731397in}{2.142506in}}{\pgfqpoint{1.725573in}{2.136682in}}%
\pgfpathcurveto{\pgfqpoint{1.719749in}{2.130858in}}{\pgfqpoint{1.716477in}{2.122958in}}{\pgfqpoint{1.716477in}{2.114721in}}%
\pgfpathcurveto{\pgfqpoint{1.716477in}{2.106485in}}{\pgfqpoint{1.719749in}{2.098585in}}{\pgfqpoint{1.725573in}{2.092761in}}%
\pgfpathcurveto{\pgfqpoint{1.731397in}{2.086937in}}{\pgfqpoint{1.739297in}{2.083665in}}{\pgfqpoint{1.747533in}{2.083665in}}%
\pgfpathclose%
\pgfusepath{stroke,fill}%
\end{pgfscope}%
\begin{pgfscope}%
\pgfpathrectangle{\pgfqpoint{0.100000in}{0.212622in}}{\pgfqpoint{3.696000in}{3.696000in}}%
\pgfusepath{clip}%
\pgfsetbuttcap%
\pgfsetroundjoin%
\definecolor{currentfill}{rgb}{0.121569,0.466667,0.705882}%
\pgfsetfillcolor{currentfill}%
\pgfsetfillopacity{0.306673}%
\pgfsetlinewidth{1.003750pt}%
\definecolor{currentstroke}{rgb}{0.121569,0.466667,0.705882}%
\pgfsetstrokecolor{currentstroke}%
\pgfsetstrokeopacity{0.306673}%
\pgfsetdash{}{0pt}%
\pgfpathmoveto{\pgfqpoint{1.752799in}{2.083002in}}%
\pgfpathcurveto{\pgfqpoint{1.761036in}{2.083002in}}{\pgfqpoint{1.768936in}{2.086274in}}{\pgfqpoint{1.774760in}{2.092098in}}%
\pgfpathcurveto{\pgfqpoint{1.780583in}{2.097922in}}{\pgfqpoint{1.783856in}{2.105822in}}{\pgfqpoint{1.783856in}{2.114058in}}%
\pgfpathcurveto{\pgfqpoint{1.783856in}{2.122294in}}{\pgfqpoint{1.780583in}{2.130194in}}{\pgfqpoint{1.774760in}{2.136018in}}%
\pgfpathcurveto{\pgfqpoint{1.768936in}{2.141842in}}{\pgfqpoint{1.761036in}{2.145115in}}{\pgfqpoint{1.752799in}{2.145115in}}%
\pgfpathcurveto{\pgfqpoint{1.744563in}{2.145115in}}{\pgfqpoint{1.736663in}{2.141842in}}{\pgfqpoint{1.730839in}{2.136018in}}%
\pgfpathcurveto{\pgfqpoint{1.725015in}{2.130194in}}{\pgfqpoint{1.721743in}{2.122294in}}{\pgfqpoint{1.721743in}{2.114058in}}%
\pgfpathcurveto{\pgfqpoint{1.721743in}{2.105822in}}{\pgfqpoint{1.725015in}{2.097922in}}{\pgfqpoint{1.730839in}{2.092098in}}%
\pgfpathcurveto{\pgfqpoint{1.736663in}{2.086274in}}{\pgfqpoint{1.744563in}{2.083002in}}{\pgfqpoint{1.752799in}{2.083002in}}%
\pgfpathclose%
\pgfusepath{stroke,fill}%
\end{pgfscope}%
\begin{pgfscope}%
\pgfpathrectangle{\pgfqpoint{0.100000in}{0.212622in}}{\pgfqpoint{3.696000in}{3.696000in}}%
\pgfusepath{clip}%
\pgfsetbuttcap%
\pgfsetroundjoin%
\definecolor{currentfill}{rgb}{0.121569,0.466667,0.705882}%
\pgfsetfillcolor{currentfill}%
\pgfsetfillopacity{0.307121}%
\pgfsetlinewidth{1.003750pt}%
\definecolor{currentstroke}{rgb}{0.121569,0.466667,0.705882}%
\pgfsetstrokecolor{currentstroke}%
\pgfsetstrokeopacity{0.307121}%
\pgfsetdash{}{0pt}%
\pgfpathmoveto{\pgfqpoint{1.664347in}{2.095066in}}%
\pgfpathcurveto{\pgfqpoint{1.672584in}{2.095066in}}{\pgfqpoint{1.680484in}{2.098339in}}{\pgfqpoint{1.686308in}{2.104163in}}%
\pgfpathcurveto{\pgfqpoint{1.692131in}{2.109987in}}{\pgfqpoint{1.695404in}{2.117887in}}{\pgfqpoint{1.695404in}{2.126123in}}%
\pgfpathcurveto{\pgfqpoint{1.695404in}{2.134359in}}{\pgfqpoint{1.692131in}{2.142259in}}{\pgfqpoint{1.686308in}{2.148083in}}%
\pgfpathcurveto{\pgfqpoint{1.680484in}{2.153907in}}{\pgfqpoint{1.672584in}{2.157179in}}{\pgfqpoint{1.664347in}{2.157179in}}%
\pgfpathcurveto{\pgfqpoint{1.656111in}{2.157179in}}{\pgfqpoint{1.648211in}{2.153907in}}{\pgfqpoint{1.642387in}{2.148083in}}%
\pgfpathcurveto{\pgfqpoint{1.636563in}{2.142259in}}{\pgfqpoint{1.633291in}{2.134359in}}{\pgfqpoint{1.633291in}{2.126123in}}%
\pgfpathcurveto{\pgfqpoint{1.633291in}{2.117887in}}{\pgfqpoint{1.636563in}{2.109987in}}{\pgfqpoint{1.642387in}{2.104163in}}%
\pgfpathcurveto{\pgfqpoint{1.648211in}{2.098339in}}{\pgfqpoint{1.656111in}{2.095066in}}{\pgfqpoint{1.664347in}{2.095066in}}%
\pgfpathclose%
\pgfusepath{stroke,fill}%
\end{pgfscope}%
\begin{pgfscope}%
\pgfpathrectangle{\pgfqpoint{0.100000in}{0.212622in}}{\pgfqpoint{3.696000in}{3.696000in}}%
\pgfusepath{clip}%
\pgfsetbuttcap%
\pgfsetroundjoin%
\definecolor{currentfill}{rgb}{0.121569,0.466667,0.705882}%
\pgfsetfillcolor{currentfill}%
\pgfsetfillopacity{0.307556}%
\pgfsetlinewidth{1.003750pt}%
\definecolor{currentstroke}{rgb}{0.121569,0.466667,0.705882}%
\pgfsetstrokecolor{currentstroke}%
\pgfsetstrokeopacity{0.307556}%
\pgfsetdash{}{0pt}%
\pgfpathmoveto{\pgfqpoint{1.758865in}{2.082528in}}%
\pgfpathcurveto{\pgfqpoint{1.767101in}{2.082528in}}{\pgfqpoint{1.775001in}{2.085800in}}{\pgfqpoint{1.780825in}{2.091624in}}%
\pgfpathcurveto{\pgfqpoint{1.786649in}{2.097448in}}{\pgfqpoint{1.789921in}{2.105348in}}{\pgfqpoint{1.789921in}{2.113584in}}%
\pgfpathcurveto{\pgfqpoint{1.789921in}{2.121820in}}{\pgfqpoint{1.786649in}{2.129720in}}{\pgfqpoint{1.780825in}{2.135544in}}%
\pgfpathcurveto{\pgfqpoint{1.775001in}{2.141368in}}{\pgfqpoint{1.767101in}{2.144641in}}{\pgfqpoint{1.758865in}{2.144641in}}%
\pgfpathcurveto{\pgfqpoint{1.750628in}{2.144641in}}{\pgfqpoint{1.742728in}{2.141368in}}{\pgfqpoint{1.736904in}{2.135544in}}%
\pgfpathcurveto{\pgfqpoint{1.731080in}{2.129720in}}{\pgfqpoint{1.727808in}{2.121820in}}{\pgfqpoint{1.727808in}{2.113584in}}%
\pgfpathcurveto{\pgfqpoint{1.727808in}{2.105348in}}{\pgfqpoint{1.731080in}{2.097448in}}{\pgfqpoint{1.736904in}{2.091624in}}%
\pgfpathcurveto{\pgfqpoint{1.742728in}{2.085800in}}{\pgfqpoint{1.750628in}{2.082528in}}{\pgfqpoint{1.758865in}{2.082528in}}%
\pgfpathclose%
\pgfusepath{stroke,fill}%
\end{pgfscope}%
\begin{pgfscope}%
\pgfpathrectangle{\pgfqpoint{0.100000in}{0.212622in}}{\pgfqpoint{3.696000in}{3.696000in}}%
\pgfusepath{clip}%
\pgfsetbuttcap%
\pgfsetroundjoin%
\definecolor{currentfill}{rgb}{0.121569,0.466667,0.705882}%
\pgfsetfillcolor{currentfill}%
\pgfsetfillopacity{0.307785}%
\pgfsetlinewidth{1.003750pt}%
\definecolor{currentstroke}{rgb}{0.121569,0.466667,0.705882}%
\pgfsetstrokecolor{currentstroke}%
\pgfsetstrokeopacity{0.307785}%
\pgfsetdash{}{0pt}%
\pgfpathmoveto{\pgfqpoint{1.766165in}{2.080986in}}%
\pgfpathcurveto{\pgfqpoint{1.774401in}{2.080986in}}{\pgfqpoint{1.782301in}{2.084258in}}{\pgfqpoint{1.788125in}{2.090082in}}%
\pgfpathcurveto{\pgfqpoint{1.793949in}{2.095906in}}{\pgfqpoint{1.797221in}{2.103806in}}{\pgfqpoint{1.797221in}{2.112042in}}%
\pgfpathcurveto{\pgfqpoint{1.797221in}{2.120279in}}{\pgfqpoint{1.793949in}{2.128179in}}{\pgfqpoint{1.788125in}{2.134003in}}%
\pgfpathcurveto{\pgfqpoint{1.782301in}{2.139826in}}{\pgfqpoint{1.774401in}{2.143099in}}{\pgfqpoint{1.766165in}{2.143099in}}%
\pgfpathcurveto{\pgfqpoint{1.757928in}{2.143099in}}{\pgfqpoint{1.750028in}{2.139826in}}{\pgfqpoint{1.744204in}{2.134003in}}%
\pgfpathcurveto{\pgfqpoint{1.738380in}{2.128179in}}{\pgfqpoint{1.735108in}{2.120279in}}{\pgfqpoint{1.735108in}{2.112042in}}%
\pgfpathcurveto{\pgfqpoint{1.735108in}{2.103806in}}{\pgfqpoint{1.738380in}{2.095906in}}{\pgfqpoint{1.744204in}{2.090082in}}%
\pgfpathcurveto{\pgfqpoint{1.750028in}{2.084258in}}{\pgfqpoint{1.757928in}{2.080986in}}{\pgfqpoint{1.766165in}{2.080986in}}%
\pgfpathclose%
\pgfusepath{stroke,fill}%
\end{pgfscope}%
\begin{pgfscope}%
\pgfpathrectangle{\pgfqpoint{0.100000in}{0.212622in}}{\pgfqpoint{3.696000in}{3.696000in}}%
\pgfusepath{clip}%
\pgfsetbuttcap%
\pgfsetroundjoin%
\definecolor{currentfill}{rgb}{0.121569,0.466667,0.705882}%
\pgfsetfillcolor{currentfill}%
\pgfsetfillopacity{0.308052}%
\pgfsetlinewidth{1.003750pt}%
\definecolor{currentstroke}{rgb}{0.121569,0.466667,0.705882}%
\pgfsetstrokecolor{currentstroke}%
\pgfsetstrokeopacity{0.308052}%
\pgfsetdash{}{0pt}%
\pgfpathmoveto{\pgfqpoint{1.662056in}{2.095242in}}%
\pgfpathcurveto{\pgfqpoint{1.670292in}{2.095242in}}{\pgfqpoint{1.678192in}{2.098514in}}{\pgfqpoint{1.684016in}{2.104338in}}%
\pgfpathcurveto{\pgfqpoint{1.689840in}{2.110162in}}{\pgfqpoint{1.693112in}{2.118062in}}{\pgfqpoint{1.693112in}{2.126299in}}%
\pgfpathcurveto{\pgfqpoint{1.693112in}{2.134535in}}{\pgfqpoint{1.689840in}{2.142435in}}{\pgfqpoint{1.684016in}{2.148259in}}%
\pgfpathcurveto{\pgfqpoint{1.678192in}{2.154083in}}{\pgfqpoint{1.670292in}{2.157355in}}{\pgfqpoint{1.662056in}{2.157355in}}%
\pgfpathcurveto{\pgfqpoint{1.653819in}{2.157355in}}{\pgfqpoint{1.645919in}{2.154083in}}{\pgfqpoint{1.640095in}{2.148259in}}%
\pgfpathcurveto{\pgfqpoint{1.634271in}{2.142435in}}{\pgfqpoint{1.630999in}{2.134535in}}{\pgfqpoint{1.630999in}{2.126299in}}%
\pgfpathcurveto{\pgfqpoint{1.630999in}{2.118062in}}{\pgfqpoint{1.634271in}{2.110162in}}{\pgfqpoint{1.640095in}{2.104338in}}%
\pgfpathcurveto{\pgfqpoint{1.645919in}{2.098514in}}{\pgfqpoint{1.653819in}{2.095242in}}{\pgfqpoint{1.662056in}{2.095242in}}%
\pgfpathclose%
\pgfusepath{stroke,fill}%
\end{pgfscope}%
\begin{pgfscope}%
\pgfpathrectangle{\pgfqpoint{0.100000in}{0.212622in}}{\pgfqpoint{3.696000in}{3.696000in}}%
\pgfusepath{clip}%
\pgfsetbuttcap%
\pgfsetroundjoin%
\definecolor{currentfill}{rgb}{0.121569,0.466667,0.705882}%
\pgfsetfillcolor{currentfill}%
\pgfsetfillopacity{0.308346}%
\pgfsetlinewidth{1.003750pt}%
\definecolor{currentstroke}{rgb}{0.121569,0.466667,0.705882}%
\pgfsetstrokecolor{currentstroke}%
\pgfsetstrokeopacity{0.308346}%
\pgfsetdash{}{0pt}%
\pgfpathmoveto{\pgfqpoint{1.774176in}{2.079469in}}%
\pgfpathcurveto{\pgfqpoint{1.782412in}{2.079469in}}{\pgfqpoint{1.790312in}{2.082742in}}{\pgfqpoint{1.796136in}{2.088566in}}%
\pgfpathcurveto{\pgfqpoint{1.801960in}{2.094390in}}{\pgfqpoint{1.805233in}{2.102290in}}{\pgfqpoint{1.805233in}{2.110526in}}%
\pgfpathcurveto{\pgfqpoint{1.805233in}{2.118762in}}{\pgfqpoint{1.801960in}{2.126662in}}{\pgfqpoint{1.796136in}{2.132486in}}%
\pgfpathcurveto{\pgfqpoint{1.790312in}{2.138310in}}{\pgfqpoint{1.782412in}{2.141582in}}{\pgfqpoint{1.774176in}{2.141582in}}%
\pgfpathcurveto{\pgfqpoint{1.765940in}{2.141582in}}{\pgfqpoint{1.758040in}{2.138310in}}{\pgfqpoint{1.752216in}{2.132486in}}%
\pgfpathcurveto{\pgfqpoint{1.746392in}{2.126662in}}{\pgfqpoint{1.743120in}{2.118762in}}{\pgfqpoint{1.743120in}{2.110526in}}%
\pgfpathcurveto{\pgfqpoint{1.743120in}{2.102290in}}{\pgfqpoint{1.746392in}{2.094390in}}{\pgfqpoint{1.752216in}{2.088566in}}%
\pgfpathcurveto{\pgfqpoint{1.758040in}{2.082742in}}{\pgfqpoint{1.765940in}{2.079469in}}{\pgfqpoint{1.774176in}{2.079469in}}%
\pgfpathclose%
\pgfusepath{stroke,fill}%
\end{pgfscope}%
\begin{pgfscope}%
\pgfpathrectangle{\pgfqpoint{0.100000in}{0.212622in}}{\pgfqpoint{3.696000in}{3.696000in}}%
\pgfusepath{clip}%
\pgfsetbuttcap%
\pgfsetroundjoin%
\definecolor{currentfill}{rgb}{0.121569,0.466667,0.705882}%
\pgfsetfillcolor{currentfill}%
\pgfsetfillopacity{0.309067}%
\pgfsetlinewidth{1.003750pt}%
\definecolor{currentstroke}{rgb}{0.121569,0.466667,0.705882}%
\pgfsetstrokecolor{currentstroke}%
\pgfsetstrokeopacity{0.309067}%
\pgfsetdash{}{0pt}%
\pgfpathmoveto{\pgfqpoint{1.782570in}{2.078037in}}%
\pgfpathcurveto{\pgfqpoint{1.790807in}{2.078037in}}{\pgfqpoint{1.798707in}{2.081309in}}{\pgfqpoint{1.804531in}{2.087133in}}%
\pgfpathcurveto{\pgfqpoint{1.810355in}{2.092957in}}{\pgfqpoint{1.813627in}{2.100857in}}{\pgfqpoint{1.813627in}{2.109093in}}%
\pgfpathcurveto{\pgfqpoint{1.813627in}{2.117330in}}{\pgfqpoint{1.810355in}{2.125230in}}{\pgfqpoint{1.804531in}{2.131054in}}%
\pgfpathcurveto{\pgfqpoint{1.798707in}{2.136878in}}{\pgfqpoint{1.790807in}{2.140150in}}{\pgfqpoint{1.782570in}{2.140150in}}%
\pgfpathcurveto{\pgfqpoint{1.774334in}{2.140150in}}{\pgfqpoint{1.766434in}{2.136878in}}{\pgfqpoint{1.760610in}{2.131054in}}%
\pgfpathcurveto{\pgfqpoint{1.754786in}{2.125230in}}{\pgfqpoint{1.751514in}{2.117330in}}{\pgfqpoint{1.751514in}{2.109093in}}%
\pgfpathcurveto{\pgfqpoint{1.751514in}{2.100857in}}{\pgfqpoint{1.754786in}{2.092957in}}{\pgfqpoint{1.760610in}{2.087133in}}%
\pgfpathcurveto{\pgfqpoint{1.766434in}{2.081309in}}{\pgfqpoint{1.774334in}{2.078037in}}{\pgfqpoint{1.782570in}{2.078037in}}%
\pgfpathclose%
\pgfusepath{stroke,fill}%
\end{pgfscope}%
\begin{pgfscope}%
\pgfpathrectangle{\pgfqpoint{0.100000in}{0.212622in}}{\pgfqpoint{3.696000in}{3.696000in}}%
\pgfusepath{clip}%
\pgfsetbuttcap%
\pgfsetroundjoin%
\definecolor{currentfill}{rgb}{0.121569,0.466667,0.705882}%
\pgfsetfillcolor{currentfill}%
\pgfsetfillopacity{0.309812}%
\pgfsetlinewidth{1.003750pt}%
\definecolor{currentstroke}{rgb}{0.121569,0.466667,0.705882}%
\pgfsetstrokecolor{currentstroke}%
\pgfsetstrokeopacity{0.309812}%
\pgfsetdash{}{0pt}%
\pgfpathmoveto{\pgfqpoint{1.666571in}{2.097107in}}%
\pgfpathcurveto{\pgfqpoint{1.674808in}{2.097107in}}{\pgfqpoint{1.682708in}{2.100379in}}{\pgfqpoint{1.688532in}{2.106203in}}%
\pgfpathcurveto{\pgfqpoint{1.694356in}{2.112027in}}{\pgfqpoint{1.697628in}{2.119927in}}{\pgfqpoint{1.697628in}{2.128163in}}%
\pgfpathcurveto{\pgfqpoint{1.697628in}{2.136400in}}{\pgfqpoint{1.694356in}{2.144300in}}{\pgfqpoint{1.688532in}{2.150124in}}%
\pgfpathcurveto{\pgfqpoint{1.682708in}{2.155948in}}{\pgfqpoint{1.674808in}{2.159220in}}{\pgfqpoint{1.666571in}{2.159220in}}%
\pgfpathcurveto{\pgfqpoint{1.658335in}{2.159220in}}{\pgfqpoint{1.650435in}{2.155948in}}{\pgfqpoint{1.644611in}{2.150124in}}%
\pgfpathcurveto{\pgfqpoint{1.638787in}{2.144300in}}{\pgfqpoint{1.635515in}{2.136400in}}{\pgfqpoint{1.635515in}{2.128163in}}%
\pgfpathcurveto{\pgfqpoint{1.635515in}{2.119927in}}{\pgfqpoint{1.638787in}{2.112027in}}{\pgfqpoint{1.644611in}{2.106203in}}%
\pgfpathcurveto{\pgfqpoint{1.650435in}{2.100379in}}{\pgfqpoint{1.658335in}{2.097107in}}{\pgfqpoint{1.666571in}{2.097107in}}%
\pgfpathclose%
\pgfusepath{stroke,fill}%
\end{pgfscope}%
\begin{pgfscope}%
\pgfpathrectangle{\pgfqpoint{0.100000in}{0.212622in}}{\pgfqpoint{3.696000in}{3.696000in}}%
\pgfusepath{clip}%
\pgfsetbuttcap%
\pgfsetroundjoin%
\definecolor{currentfill}{rgb}{0.121569,0.466667,0.705882}%
\pgfsetfillcolor{currentfill}%
\pgfsetfillopacity{0.310202}%
\pgfsetlinewidth{1.003750pt}%
\definecolor{currentstroke}{rgb}{0.121569,0.466667,0.705882}%
\pgfsetstrokecolor{currentstroke}%
\pgfsetstrokeopacity{0.310202}%
\pgfsetdash{}{0pt}%
\pgfpathmoveto{\pgfqpoint{1.791001in}{2.076694in}}%
\pgfpathcurveto{\pgfqpoint{1.799237in}{2.076694in}}{\pgfqpoint{1.807137in}{2.079966in}}{\pgfqpoint{1.812961in}{2.085790in}}%
\pgfpathcurveto{\pgfqpoint{1.818785in}{2.091614in}}{\pgfqpoint{1.822057in}{2.099514in}}{\pgfqpoint{1.822057in}{2.107750in}}%
\pgfpathcurveto{\pgfqpoint{1.822057in}{2.115987in}}{\pgfqpoint{1.818785in}{2.123887in}}{\pgfqpoint{1.812961in}{2.129711in}}%
\pgfpathcurveto{\pgfqpoint{1.807137in}{2.135535in}}{\pgfqpoint{1.799237in}{2.138807in}}{\pgfqpoint{1.791001in}{2.138807in}}%
\pgfpathcurveto{\pgfqpoint{1.782765in}{2.138807in}}{\pgfqpoint{1.774865in}{2.135535in}}{\pgfqpoint{1.769041in}{2.129711in}}%
\pgfpathcurveto{\pgfqpoint{1.763217in}{2.123887in}}{\pgfqpoint{1.759944in}{2.115987in}}{\pgfqpoint{1.759944in}{2.107750in}}%
\pgfpathcurveto{\pgfqpoint{1.759944in}{2.099514in}}{\pgfqpoint{1.763217in}{2.091614in}}{\pgfqpoint{1.769041in}{2.085790in}}%
\pgfpathcurveto{\pgfqpoint{1.774865in}{2.079966in}}{\pgfqpoint{1.782765in}{2.076694in}}{\pgfqpoint{1.791001in}{2.076694in}}%
\pgfpathclose%
\pgfusepath{stroke,fill}%
\end{pgfscope}%
\begin{pgfscope}%
\pgfpathrectangle{\pgfqpoint{0.100000in}{0.212622in}}{\pgfqpoint{3.696000in}{3.696000in}}%
\pgfusepath{clip}%
\pgfsetbuttcap%
\pgfsetroundjoin%
\definecolor{currentfill}{rgb}{0.121569,0.466667,0.705882}%
\pgfsetfillcolor{currentfill}%
\pgfsetfillopacity{0.311385}%
\pgfsetlinewidth{1.003750pt}%
\definecolor{currentstroke}{rgb}{0.121569,0.466667,0.705882}%
\pgfsetstrokecolor{currentstroke}%
\pgfsetstrokeopacity{0.311385}%
\pgfsetdash{}{0pt}%
\pgfpathmoveto{\pgfqpoint{1.799919in}{2.075398in}}%
\pgfpathcurveto{\pgfqpoint{1.808155in}{2.075398in}}{\pgfqpoint{1.816055in}{2.078671in}}{\pgfqpoint{1.821879in}{2.084495in}}%
\pgfpathcurveto{\pgfqpoint{1.827703in}{2.090319in}}{\pgfqpoint{1.830975in}{2.098219in}}{\pgfqpoint{1.830975in}{2.106455in}}%
\pgfpathcurveto{\pgfqpoint{1.830975in}{2.114691in}}{\pgfqpoint{1.827703in}{2.122591in}}{\pgfqpoint{1.821879in}{2.128415in}}%
\pgfpathcurveto{\pgfqpoint{1.816055in}{2.134239in}}{\pgfqpoint{1.808155in}{2.137511in}}{\pgfqpoint{1.799919in}{2.137511in}}%
\pgfpathcurveto{\pgfqpoint{1.791682in}{2.137511in}}{\pgfqpoint{1.783782in}{2.134239in}}{\pgfqpoint{1.777958in}{2.128415in}}%
\pgfpathcurveto{\pgfqpoint{1.772135in}{2.122591in}}{\pgfqpoint{1.768862in}{2.114691in}}{\pgfqpoint{1.768862in}{2.106455in}}%
\pgfpathcurveto{\pgfqpoint{1.768862in}{2.098219in}}{\pgfqpoint{1.772135in}{2.090319in}}{\pgfqpoint{1.777958in}{2.084495in}}%
\pgfpathcurveto{\pgfqpoint{1.783782in}{2.078671in}}{\pgfqpoint{1.791682in}{2.075398in}}{\pgfqpoint{1.799919in}{2.075398in}}%
\pgfpathclose%
\pgfusepath{stroke,fill}%
\end{pgfscope}%
\begin{pgfscope}%
\pgfpathrectangle{\pgfqpoint{0.100000in}{0.212622in}}{\pgfqpoint{3.696000in}{3.696000in}}%
\pgfusepath{clip}%
\pgfsetbuttcap%
\pgfsetroundjoin%
\definecolor{currentfill}{rgb}{0.121569,0.466667,0.705882}%
\pgfsetfillcolor{currentfill}%
\pgfsetfillopacity{0.311483}%
\pgfsetlinewidth{1.003750pt}%
\definecolor{currentstroke}{rgb}{0.121569,0.466667,0.705882}%
\pgfsetstrokecolor{currentstroke}%
\pgfsetstrokeopacity{0.311483}%
\pgfsetdash{}{0pt}%
\pgfpathmoveto{\pgfqpoint{1.662959in}{2.097395in}}%
\pgfpathcurveto{\pgfqpoint{1.671195in}{2.097395in}}{\pgfqpoint{1.679095in}{2.100667in}}{\pgfqpoint{1.684919in}{2.106491in}}%
\pgfpathcurveto{\pgfqpoint{1.690743in}{2.112315in}}{\pgfqpoint{1.694016in}{2.120215in}}{\pgfqpoint{1.694016in}{2.128452in}}%
\pgfpathcurveto{\pgfqpoint{1.694016in}{2.136688in}}{\pgfqpoint{1.690743in}{2.144588in}}{\pgfqpoint{1.684919in}{2.150412in}}%
\pgfpathcurveto{\pgfqpoint{1.679095in}{2.156236in}}{\pgfqpoint{1.671195in}{2.159508in}}{\pgfqpoint{1.662959in}{2.159508in}}%
\pgfpathcurveto{\pgfqpoint{1.654723in}{2.159508in}}{\pgfqpoint{1.646823in}{2.156236in}}{\pgfqpoint{1.640999in}{2.150412in}}%
\pgfpathcurveto{\pgfqpoint{1.635175in}{2.144588in}}{\pgfqpoint{1.631903in}{2.136688in}}{\pgfqpoint{1.631903in}{2.128452in}}%
\pgfpathcurveto{\pgfqpoint{1.631903in}{2.120215in}}{\pgfqpoint{1.635175in}{2.112315in}}{\pgfqpoint{1.640999in}{2.106491in}}%
\pgfpathcurveto{\pgfqpoint{1.646823in}{2.100667in}}{\pgfqpoint{1.654723in}{2.097395in}}{\pgfqpoint{1.662959in}{2.097395in}}%
\pgfpathclose%
\pgfusepath{stroke,fill}%
\end{pgfscope}%
\begin{pgfscope}%
\pgfpathrectangle{\pgfqpoint{0.100000in}{0.212622in}}{\pgfqpoint{3.696000in}{3.696000in}}%
\pgfusepath{clip}%
\pgfsetbuttcap%
\pgfsetroundjoin%
\definecolor{currentfill}{rgb}{0.121569,0.466667,0.705882}%
\pgfsetfillcolor{currentfill}%
\pgfsetfillopacity{0.312991}%
\pgfsetlinewidth{1.003750pt}%
\definecolor{currentstroke}{rgb}{0.121569,0.466667,0.705882}%
\pgfsetstrokecolor{currentstroke}%
\pgfsetstrokeopacity{0.312991}%
\pgfsetdash{}{0pt}%
\pgfpathmoveto{\pgfqpoint{1.809630in}{2.074500in}}%
\pgfpathcurveto{\pgfqpoint{1.817866in}{2.074500in}}{\pgfqpoint{1.825766in}{2.077773in}}{\pgfqpoint{1.831590in}{2.083596in}}%
\pgfpathcurveto{\pgfqpoint{1.837414in}{2.089420in}}{\pgfqpoint{1.840686in}{2.097320in}}{\pgfqpoint{1.840686in}{2.105557in}}%
\pgfpathcurveto{\pgfqpoint{1.840686in}{2.113793in}}{\pgfqpoint{1.837414in}{2.121693in}}{\pgfqpoint{1.831590in}{2.127517in}}%
\pgfpathcurveto{\pgfqpoint{1.825766in}{2.133341in}}{\pgfqpoint{1.817866in}{2.136613in}}{\pgfqpoint{1.809630in}{2.136613in}}%
\pgfpathcurveto{\pgfqpoint{1.801393in}{2.136613in}}{\pgfqpoint{1.793493in}{2.133341in}}{\pgfqpoint{1.787669in}{2.127517in}}%
\pgfpathcurveto{\pgfqpoint{1.781845in}{2.121693in}}{\pgfqpoint{1.778573in}{2.113793in}}{\pgfqpoint{1.778573in}{2.105557in}}%
\pgfpathcurveto{\pgfqpoint{1.778573in}{2.097320in}}{\pgfqpoint{1.781845in}{2.089420in}}{\pgfqpoint{1.787669in}{2.083596in}}%
\pgfpathcurveto{\pgfqpoint{1.793493in}{2.077773in}}{\pgfqpoint{1.801393in}{2.074500in}}{\pgfqpoint{1.809630in}{2.074500in}}%
\pgfpathclose%
\pgfusepath{stroke,fill}%
\end{pgfscope}%
\begin{pgfscope}%
\pgfpathrectangle{\pgfqpoint{0.100000in}{0.212622in}}{\pgfqpoint{3.696000in}{3.696000in}}%
\pgfusepath{clip}%
\pgfsetbuttcap%
\pgfsetroundjoin%
\definecolor{currentfill}{rgb}{0.121569,0.466667,0.705882}%
\pgfsetfillcolor{currentfill}%
\pgfsetfillopacity{0.313019}%
\pgfsetlinewidth{1.003750pt}%
\definecolor{currentstroke}{rgb}{0.121569,0.466667,0.705882}%
\pgfsetstrokecolor{currentstroke}%
\pgfsetstrokeopacity{0.313019}%
\pgfsetdash{}{0pt}%
\pgfpathmoveto{\pgfqpoint{1.659283in}{2.097569in}}%
\pgfpathcurveto{\pgfqpoint{1.667519in}{2.097569in}}{\pgfqpoint{1.675419in}{2.100841in}}{\pgfqpoint{1.681243in}{2.106665in}}%
\pgfpathcurveto{\pgfqpoint{1.687067in}{2.112489in}}{\pgfqpoint{1.690339in}{2.120389in}}{\pgfqpoint{1.690339in}{2.128625in}}%
\pgfpathcurveto{\pgfqpoint{1.690339in}{2.136862in}}{\pgfqpoint{1.687067in}{2.144762in}}{\pgfqpoint{1.681243in}{2.150586in}}%
\pgfpathcurveto{\pgfqpoint{1.675419in}{2.156409in}}{\pgfqpoint{1.667519in}{2.159682in}}{\pgfqpoint{1.659283in}{2.159682in}}%
\pgfpathcurveto{\pgfqpoint{1.651047in}{2.159682in}}{\pgfqpoint{1.643147in}{2.156409in}}{\pgfqpoint{1.637323in}{2.150586in}}%
\pgfpathcurveto{\pgfqpoint{1.631499in}{2.144762in}}{\pgfqpoint{1.628226in}{2.136862in}}{\pgfqpoint{1.628226in}{2.128625in}}%
\pgfpathcurveto{\pgfqpoint{1.628226in}{2.120389in}}{\pgfqpoint{1.631499in}{2.112489in}}{\pgfqpoint{1.637323in}{2.106665in}}%
\pgfpathcurveto{\pgfqpoint{1.643147in}{2.100841in}}{\pgfqpoint{1.651047in}{2.097569in}}{\pgfqpoint{1.659283in}{2.097569in}}%
\pgfpathclose%
\pgfusepath{stroke,fill}%
\end{pgfscope}%
\begin{pgfscope}%
\pgfpathrectangle{\pgfqpoint{0.100000in}{0.212622in}}{\pgfqpoint{3.696000in}{3.696000in}}%
\pgfusepath{clip}%
\pgfsetbuttcap%
\pgfsetroundjoin%
\definecolor{currentfill}{rgb}{0.121569,0.466667,0.705882}%
\pgfsetfillcolor{currentfill}%
\pgfsetfillopacity{0.314094}%
\pgfsetlinewidth{1.003750pt}%
\definecolor{currentstroke}{rgb}{0.121569,0.466667,0.705882}%
\pgfsetstrokecolor{currentstroke}%
\pgfsetstrokeopacity{0.314094}%
\pgfsetdash{}{0pt}%
\pgfpathmoveto{\pgfqpoint{1.814209in}{2.073947in}}%
\pgfpathcurveto{\pgfqpoint{1.822445in}{2.073947in}}{\pgfqpoint{1.830345in}{2.077220in}}{\pgfqpoint{1.836169in}{2.083044in}}%
\pgfpathcurveto{\pgfqpoint{1.841993in}{2.088867in}}{\pgfqpoint{1.845266in}{2.096768in}}{\pgfqpoint{1.845266in}{2.105004in}}%
\pgfpathcurveto{\pgfqpoint{1.845266in}{2.113240in}}{\pgfqpoint{1.841993in}{2.121140in}}{\pgfqpoint{1.836169in}{2.126964in}}%
\pgfpathcurveto{\pgfqpoint{1.830345in}{2.132788in}}{\pgfqpoint{1.822445in}{2.136060in}}{\pgfqpoint{1.814209in}{2.136060in}}%
\pgfpathcurveto{\pgfqpoint{1.805973in}{2.136060in}}{\pgfqpoint{1.798073in}{2.132788in}}{\pgfqpoint{1.792249in}{2.126964in}}%
\pgfpathcurveto{\pgfqpoint{1.786425in}{2.121140in}}{\pgfqpoint{1.783153in}{2.113240in}}{\pgfqpoint{1.783153in}{2.105004in}}%
\pgfpathcurveto{\pgfqpoint{1.783153in}{2.096768in}}{\pgfqpoint{1.786425in}{2.088867in}}{\pgfqpoint{1.792249in}{2.083044in}}%
\pgfpathcurveto{\pgfqpoint{1.798073in}{2.077220in}}{\pgfqpoint{1.805973in}{2.073947in}}{\pgfqpoint{1.814209in}{2.073947in}}%
\pgfpathclose%
\pgfusepath{stroke,fill}%
\end{pgfscope}%
\begin{pgfscope}%
\pgfpathrectangle{\pgfqpoint{0.100000in}{0.212622in}}{\pgfqpoint{3.696000in}{3.696000in}}%
\pgfusepath{clip}%
\pgfsetbuttcap%
\pgfsetroundjoin%
\definecolor{currentfill}{rgb}{0.121569,0.466667,0.705882}%
\pgfsetfillcolor{currentfill}%
\pgfsetfillopacity{0.314303}%
\pgfsetlinewidth{1.003750pt}%
\definecolor{currentstroke}{rgb}{0.121569,0.466667,0.705882}%
\pgfsetstrokecolor{currentstroke}%
\pgfsetstrokeopacity{0.314303}%
\pgfsetdash{}{0pt}%
\pgfpathmoveto{\pgfqpoint{1.656517in}{2.097545in}}%
\pgfpathcurveto{\pgfqpoint{1.664753in}{2.097545in}}{\pgfqpoint{1.672653in}{2.100818in}}{\pgfqpoint{1.678477in}{2.106642in}}%
\pgfpathcurveto{\pgfqpoint{1.684301in}{2.112466in}}{\pgfqpoint{1.687573in}{2.120366in}}{\pgfqpoint{1.687573in}{2.128602in}}%
\pgfpathcurveto{\pgfqpoint{1.687573in}{2.136838in}}{\pgfqpoint{1.684301in}{2.144738in}}{\pgfqpoint{1.678477in}{2.150562in}}%
\pgfpathcurveto{\pgfqpoint{1.672653in}{2.156386in}}{\pgfqpoint{1.664753in}{2.159658in}}{\pgfqpoint{1.656517in}{2.159658in}}%
\pgfpathcurveto{\pgfqpoint{1.648281in}{2.159658in}}{\pgfqpoint{1.640381in}{2.156386in}}{\pgfqpoint{1.634557in}{2.150562in}}%
\pgfpathcurveto{\pgfqpoint{1.628733in}{2.144738in}}{\pgfqpoint{1.625460in}{2.136838in}}{\pgfqpoint{1.625460in}{2.128602in}}%
\pgfpathcurveto{\pgfqpoint{1.625460in}{2.120366in}}{\pgfqpoint{1.628733in}{2.112466in}}{\pgfqpoint{1.634557in}{2.106642in}}%
\pgfpathcurveto{\pgfqpoint{1.640381in}{2.100818in}}{\pgfqpoint{1.648281in}{2.097545in}}{\pgfqpoint{1.656517in}{2.097545in}}%
\pgfpathclose%
\pgfusepath{stroke,fill}%
\end{pgfscope}%
\begin{pgfscope}%
\pgfpathrectangle{\pgfqpoint{0.100000in}{0.212622in}}{\pgfqpoint{3.696000in}{3.696000in}}%
\pgfusepath{clip}%
\pgfsetbuttcap%
\pgfsetroundjoin%
\definecolor{currentfill}{rgb}{0.121569,0.466667,0.705882}%
\pgfsetfillcolor{currentfill}%
\pgfsetfillopacity{0.314570}%
\pgfsetlinewidth{1.003750pt}%
\definecolor{currentstroke}{rgb}{0.121569,0.466667,0.705882}%
\pgfsetstrokecolor{currentstroke}%
\pgfsetstrokeopacity{0.314570}%
\pgfsetdash{}{0pt}%
\pgfpathmoveto{\pgfqpoint{1.817129in}{2.073540in}}%
\pgfpathcurveto{\pgfqpoint{1.825366in}{2.073540in}}{\pgfqpoint{1.833266in}{2.076812in}}{\pgfqpoint{1.839090in}{2.082636in}}%
\pgfpathcurveto{\pgfqpoint{1.844914in}{2.088460in}}{\pgfqpoint{1.848186in}{2.096360in}}{\pgfqpoint{1.848186in}{2.104596in}}%
\pgfpathcurveto{\pgfqpoint{1.848186in}{2.112832in}}{\pgfqpoint{1.844914in}{2.120732in}}{\pgfqpoint{1.839090in}{2.126556in}}%
\pgfpathcurveto{\pgfqpoint{1.833266in}{2.132380in}}{\pgfqpoint{1.825366in}{2.135653in}}{\pgfqpoint{1.817129in}{2.135653in}}%
\pgfpathcurveto{\pgfqpoint{1.808893in}{2.135653in}}{\pgfqpoint{1.800993in}{2.132380in}}{\pgfqpoint{1.795169in}{2.126556in}}%
\pgfpathcurveto{\pgfqpoint{1.789345in}{2.120732in}}{\pgfqpoint{1.786073in}{2.112832in}}{\pgfqpoint{1.786073in}{2.104596in}}%
\pgfpathcurveto{\pgfqpoint{1.786073in}{2.096360in}}{\pgfqpoint{1.789345in}{2.088460in}}{\pgfqpoint{1.795169in}{2.082636in}}%
\pgfpathcurveto{\pgfqpoint{1.800993in}{2.076812in}}{\pgfqpoint{1.808893in}{2.073540in}}{\pgfqpoint{1.817129in}{2.073540in}}%
\pgfpathclose%
\pgfusepath{stroke,fill}%
\end{pgfscope}%
\begin{pgfscope}%
\pgfpathrectangle{\pgfqpoint{0.100000in}{0.212622in}}{\pgfqpoint{3.696000in}{3.696000in}}%
\pgfusepath{clip}%
\pgfsetbuttcap%
\pgfsetroundjoin%
\definecolor{currentfill}{rgb}{0.121569,0.466667,0.705882}%
\pgfsetfillcolor{currentfill}%
\pgfsetfillopacity{0.314784}%
\pgfsetlinewidth{1.003750pt}%
\definecolor{currentstroke}{rgb}{0.121569,0.466667,0.705882}%
\pgfsetstrokecolor{currentstroke}%
\pgfsetstrokeopacity{0.314784}%
\pgfsetdash{}{0pt}%
\pgfpathmoveto{\pgfqpoint{1.821677in}{2.072441in}}%
\pgfpathcurveto{\pgfqpoint{1.829913in}{2.072441in}}{\pgfqpoint{1.837813in}{2.075713in}}{\pgfqpoint{1.843637in}{2.081537in}}%
\pgfpathcurveto{\pgfqpoint{1.849461in}{2.087361in}}{\pgfqpoint{1.852734in}{2.095261in}}{\pgfqpoint{1.852734in}{2.103497in}}%
\pgfpathcurveto{\pgfqpoint{1.852734in}{2.111734in}}{\pgfqpoint{1.849461in}{2.119634in}}{\pgfqpoint{1.843637in}{2.125458in}}%
\pgfpathcurveto{\pgfqpoint{1.837813in}{2.131281in}}{\pgfqpoint{1.829913in}{2.134554in}}{\pgfqpoint{1.821677in}{2.134554in}}%
\pgfpathcurveto{\pgfqpoint{1.813441in}{2.134554in}}{\pgfqpoint{1.805541in}{2.131281in}}{\pgfqpoint{1.799717in}{2.125458in}}%
\pgfpathcurveto{\pgfqpoint{1.793893in}{2.119634in}}{\pgfqpoint{1.790621in}{2.111734in}}{\pgfqpoint{1.790621in}{2.103497in}}%
\pgfpathcurveto{\pgfqpoint{1.790621in}{2.095261in}}{\pgfqpoint{1.793893in}{2.087361in}}{\pgfqpoint{1.799717in}{2.081537in}}%
\pgfpathcurveto{\pgfqpoint{1.805541in}{2.075713in}}{\pgfqpoint{1.813441in}{2.072441in}}{\pgfqpoint{1.821677in}{2.072441in}}%
\pgfpathclose%
\pgfusepath{stroke,fill}%
\end{pgfscope}%
\begin{pgfscope}%
\pgfpathrectangle{\pgfqpoint{0.100000in}{0.212622in}}{\pgfqpoint{3.696000in}{3.696000in}}%
\pgfusepath{clip}%
\pgfsetbuttcap%
\pgfsetroundjoin%
\definecolor{currentfill}{rgb}{0.121569,0.466667,0.705882}%
\pgfsetfillcolor{currentfill}%
\pgfsetfillopacity{0.315121}%
\pgfsetlinewidth{1.003750pt}%
\definecolor{currentstroke}{rgb}{0.121569,0.466667,0.705882}%
\pgfsetstrokecolor{currentstroke}%
\pgfsetstrokeopacity{0.315121}%
\pgfsetdash{}{0pt}%
\pgfpathmoveto{\pgfqpoint{1.654963in}{2.097622in}}%
\pgfpathcurveto{\pgfqpoint{1.663199in}{2.097622in}}{\pgfqpoint{1.671099in}{2.100895in}}{\pgfqpoint{1.676923in}{2.106719in}}%
\pgfpathcurveto{\pgfqpoint{1.682747in}{2.112543in}}{\pgfqpoint{1.686019in}{2.120443in}}{\pgfqpoint{1.686019in}{2.128679in}}%
\pgfpathcurveto{\pgfqpoint{1.686019in}{2.136915in}}{\pgfqpoint{1.682747in}{2.144815in}}{\pgfqpoint{1.676923in}{2.150639in}}%
\pgfpathcurveto{\pgfqpoint{1.671099in}{2.156463in}}{\pgfqpoint{1.663199in}{2.159735in}}{\pgfqpoint{1.654963in}{2.159735in}}%
\pgfpathcurveto{\pgfqpoint{1.646727in}{2.159735in}}{\pgfqpoint{1.638827in}{2.156463in}}{\pgfqpoint{1.633003in}{2.150639in}}%
\pgfpathcurveto{\pgfqpoint{1.627179in}{2.144815in}}{\pgfqpoint{1.623906in}{2.136915in}}{\pgfqpoint{1.623906in}{2.128679in}}%
\pgfpathcurveto{\pgfqpoint{1.623906in}{2.120443in}}{\pgfqpoint{1.627179in}{2.112543in}}{\pgfqpoint{1.633003in}{2.106719in}}%
\pgfpathcurveto{\pgfqpoint{1.638827in}{2.100895in}}{\pgfqpoint{1.646727in}{2.097622in}}{\pgfqpoint{1.654963in}{2.097622in}}%
\pgfpathclose%
\pgfusepath{stroke,fill}%
\end{pgfscope}%
\begin{pgfscope}%
\pgfpathrectangle{\pgfqpoint{0.100000in}{0.212622in}}{\pgfqpoint{3.696000in}{3.696000in}}%
\pgfusepath{clip}%
\pgfsetbuttcap%
\pgfsetroundjoin%
\definecolor{currentfill}{rgb}{0.121569,0.466667,0.705882}%
\pgfsetfillcolor{currentfill}%
\pgfsetfillopacity{0.315417}%
\pgfsetlinewidth{1.003750pt}%
\definecolor{currentstroke}{rgb}{0.121569,0.466667,0.705882}%
\pgfsetstrokecolor{currentstroke}%
\pgfsetstrokeopacity{0.315417}%
\pgfsetdash{}{0pt}%
\pgfpathmoveto{\pgfqpoint{1.827567in}{2.071446in}}%
\pgfpathcurveto{\pgfqpoint{1.835803in}{2.071446in}}{\pgfqpoint{1.843703in}{2.074718in}}{\pgfqpoint{1.849527in}{2.080542in}}%
\pgfpathcurveto{\pgfqpoint{1.855351in}{2.086366in}}{\pgfqpoint{1.858623in}{2.094266in}}{\pgfqpoint{1.858623in}{2.102502in}}%
\pgfpathcurveto{\pgfqpoint{1.858623in}{2.110738in}}{\pgfqpoint{1.855351in}{2.118638in}}{\pgfqpoint{1.849527in}{2.124462in}}%
\pgfpathcurveto{\pgfqpoint{1.843703in}{2.130286in}}{\pgfqpoint{1.835803in}{2.133559in}}{\pgfqpoint{1.827567in}{2.133559in}}%
\pgfpathcurveto{\pgfqpoint{1.819330in}{2.133559in}}{\pgfqpoint{1.811430in}{2.130286in}}{\pgfqpoint{1.805606in}{2.124462in}}%
\pgfpathcurveto{\pgfqpoint{1.799783in}{2.118638in}}{\pgfqpoint{1.796510in}{2.110738in}}{\pgfqpoint{1.796510in}{2.102502in}}%
\pgfpathcurveto{\pgfqpoint{1.796510in}{2.094266in}}{\pgfqpoint{1.799783in}{2.086366in}}{\pgfqpoint{1.805606in}{2.080542in}}%
\pgfpathcurveto{\pgfqpoint{1.811430in}{2.074718in}}{\pgfqpoint{1.819330in}{2.071446in}}{\pgfqpoint{1.827567in}{2.071446in}}%
\pgfpathclose%
\pgfusepath{stroke,fill}%
\end{pgfscope}%
\begin{pgfscope}%
\pgfpathrectangle{\pgfqpoint{0.100000in}{0.212622in}}{\pgfqpoint{3.696000in}{3.696000in}}%
\pgfusepath{clip}%
\pgfsetbuttcap%
\pgfsetroundjoin%
\definecolor{currentfill}{rgb}{0.121569,0.466667,0.705882}%
\pgfsetfillcolor{currentfill}%
\pgfsetfillopacity{0.315856}%
\pgfsetlinewidth{1.003750pt}%
\definecolor{currentstroke}{rgb}{0.121569,0.466667,0.705882}%
\pgfsetstrokecolor{currentstroke}%
\pgfsetstrokeopacity{0.315856}%
\pgfsetdash{}{0pt}%
\pgfpathmoveto{\pgfqpoint{1.830614in}{2.070963in}}%
\pgfpathcurveto{\pgfqpoint{1.838851in}{2.070963in}}{\pgfqpoint{1.846751in}{2.074236in}}{\pgfqpoint{1.852575in}{2.080060in}}%
\pgfpathcurveto{\pgfqpoint{1.858398in}{2.085884in}}{\pgfqpoint{1.861671in}{2.093784in}}{\pgfqpoint{1.861671in}{2.102020in}}%
\pgfpathcurveto{\pgfqpoint{1.861671in}{2.110256in}}{\pgfqpoint{1.858398in}{2.118156in}}{\pgfqpoint{1.852575in}{2.123980in}}%
\pgfpathcurveto{\pgfqpoint{1.846751in}{2.129804in}}{\pgfqpoint{1.838851in}{2.133076in}}{\pgfqpoint{1.830614in}{2.133076in}}%
\pgfpathcurveto{\pgfqpoint{1.822378in}{2.133076in}}{\pgfqpoint{1.814478in}{2.129804in}}{\pgfqpoint{1.808654in}{2.123980in}}%
\pgfpathcurveto{\pgfqpoint{1.802830in}{2.118156in}}{\pgfqpoint{1.799558in}{2.110256in}}{\pgfqpoint{1.799558in}{2.102020in}}%
\pgfpathcurveto{\pgfqpoint{1.799558in}{2.093784in}}{\pgfqpoint{1.802830in}{2.085884in}}{\pgfqpoint{1.808654in}{2.080060in}}%
\pgfpathcurveto{\pgfqpoint{1.814478in}{2.074236in}}{\pgfqpoint{1.822378in}{2.070963in}}{\pgfqpoint{1.830614in}{2.070963in}}%
\pgfpathclose%
\pgfusepath{stroke,fill}%
\end{pgfscope}%
\begin{pgfscope}%
\pgfpathrectangle{\pgfqpoint{0.100000in}{0.212622in}}{\pgfqpoint{3.696000in}{3.696000in}}%
\pgfusepath{clip}%
\pgfsetbuttcap%
\pgfsetroundjoin%
\definecolor{currentfill}{rgb}{0.121569,0.466667,0.705882}%
\pgfsetfillcolor{currentfill}%
\pgfsetfillopacity{0.316361}%
\pgfsetlinewidth{1.003750pt}%
\definecolor{currentstroke}{rgb}{0.121569,0.466667,0.705882}%
\pgfsetstrokecolor{currentstroke}%
\pgfsetstrokeopacity{0.316361}%
\pgfsetdash{}{0pt}%
\pgfpathmoveto{\pgfqpoint{1.834084in}{2.070394in}}%
\pgfpathcurveto{\pgfqpoint{1.842320in}{2.070394in}}{\pgfqpoint{1.850220in}{2.073667in}}{\pgfqpoint{1.856044in}{2.079491in}}%
\pgfpathcurveto{\pgfqpoint{1.861868in}{2.085314in}}{\pgfqpoint{1.865140in}{2.093215in}}{\pgfqpoint{1.865140in}{2.101451in}}%
\pgfpathcurveto{\pgfqpoint{1.865140in}{2.109687in}}{\pgfqpoint{1.861868in}{2.117587in}}{\pgfqpoint{1.856044in}{2.123411in}}%
\pgfpathcurveto{\pgfqpoint{1.850220in}{2.129235in}}{\pgfqpoint{1.842320in}{2.132507in}}{\pgfqpoint{1.834084in}{2.132507in}}%
\pgfpathcurveto{\pgfqpoint{1.825847in}{2.132507in}}{\pgfqpoint{1.817947in}{2.129235in}}{\pgfqpoint{1.812124in}{2.123411in}}%
\pgfpathcurveto{\pgfqpoint{1.806300in}{2.117587in}}{\pgfqpoint{1.803027in}{2.109687in}}{\pgfqpoint{1.803027in}{2.101451in}}%
\pgfpathcurveto{\pgfqpoint{1.803027in}{2.093215in}}{\pgfqpoint{1.806300in}{2.085314in}}{\pgfqpoint{1.812124in}{2.079491in}}%
\pgfpathcurveto{\pgfqpoint{1.817947in}{2.073667in}}{\pgfqpoint{1.825847in}{2.070394in}}{\pgfqpoint{1.834084in}{2.070394in}}%
\pgfpathclose%
\pgfusepath{stroke,fill}%
\end{pgfscope}%
\begin{pgfscope}%
\pgfpathrectangle{\pgfqpoint{0.100000in}{0.212622in}}{\pgfqpoint{3.696000in}{3.696000in}}%
\pgfusepath{clip}%
\pgfsetbuttcap%
\pgfsetroundjoin%
\definecolor{currentfill}{rgb}{0.121569,0.466667,0.705882}%
\pgfsetfillcolor{currentfill}%
\pgfsetfillopacity{0.316546}%
\pgfsetlinewidth{1.003750pt}%
\definecolor{currentstroke}{rgb}{0.121569,0.466667,0.705882}%
\pgfsetstrokecolor{currentstroke}%
\pgfsetstrokeopacity{0.316546}%
\pgfsetdash{}{0pt}%
\pgfpathmoveto{\pgfqpoint{1.651485in}{2.097965in}}%
\pgfpathcurveto{\pgfqpoint{1.659721in}{2.097965in}}{\pgfqpoint{1.667621in}{2.101238in}}{\pgfqpoint{1.673445in}{2.107061in}}%
\pgfpathcurveto{\pgfqpoint{1.679269in}{2.112885in}}{\pgfqpoint{1.682542in}{2.120785in}}{\pgfqpoint{1.682542in}{2.129022in}}%
\pgfpathcurveto{\pgfqpoint{1.682542in}{2.137258in}}{\pgfqpoint{1.679269in}{2.145158in}}{\pgfqpoint{1.673445in}{2.150982in}}%
\pgfpathcurveto{\pgfqpoint{1.667621in}{2.156806in}}{\pgfqpoint{1.659721in}{2.160078in}}{\pgfqpoint{1.651485in}{2.160078in}}%
\pgfpathcurveto{\pgfqpoint{1.643249in}{2.160078in}}{\pgfqpoint{1.635349in}{2.156806in}}{\pgfqpoint{1.629525in}{2.150982in}}%
\pgfpathcurveto{\pgfqpoint{1.623701in}{2.145158in}}{\pgfqpoint{1.620429in}{2.137258in}}{\pgfqpoint{1.620429in}{2.129022in}}%
\pgfpathcurveto{\pgfqpoint{1.620429in}{2.120785in}}{\pgfqpoint{1.623701in}{2.112885in}}{\pgfqpoint{1.629525in}{2.107061in}}%
\pgfpathcurveto{\pgfqpoint{1.635349in}{2.101238in}}{\pgfqpoint{1.643249in}{2.097965in}}{\pgfqpoint{1.651485in}{2.097965in}}%
\pgfpathclose%
\pgfusepath{stroke,fill}%
\end{pgfscope}%
\begin{pgfscope}%
\pgfpathrectangle{\pgfqpoint{0.100000in}{0.212622in}}{\pgfqpoint{3.696000in}{3.696000in}}%
\pgfusepath{clip}%
\pgfsetbuttcap%
\pgfsetroundjoin%
\definecolor{currentfill}{rgb}{0.121569,0.466667,0.705882}%
\pgfsetfillcolor{currentfill}%
\pgfsetfillopacity{0.317132}%
\pgfsetlinewidth{1.003750pt}%
\definecolor{currentstroke}{rgb}{0.121569,0.466667,0.705882}%
\pgfsetstrokecolor{currentstroke}%
\pgfsetstrokeopacity{0.317132}%
\pgfsetdash{}{0pt}%
\pgfpathmoveto{\pgfqpoint{1.838239in}{2.069740in}}%
\pgfpathcurveto{\pgfqpoint{1.846475in}{2.069740in}}{\pgfqpoint{1.854375in}{2.073013in}}{\pgfqpoint{1.860199in}{2.078837in}}%
\pgfpathcurveto{\pgfqpoint{1.866023in}{2.084661in}}{\pgfqpoint{1.869295in}{2.092561in}}{\pgfqpoint{1.869295in}{2.100797in}}%
\pgfpathcurveto{\pgfqpoint{1.869295in}{2.109033in}}{\pgfqpoint{1.866023in}{2.116933in}}{\pgfqpoint{1.860199in}{2.122757in}}%
\pgfpathcurveto{\pgfqpoint{1.854375in}{2.128581in}}{\pgfqpoint{1.846475in}{2.131853in}}{\pgfqpoint{1.838239in}{2.131853in}}%
\pgfpathcurveto{\pgfqpoint{1.830003in}{2.131853in}}{\pgfqpoint{1.822103in}{2.128581in}}{\pgfqpoint{1.816279in}{2.122757in}}%
\pgfpathcurveto{\pgfqpoint{1.810455in}{2.116933in}}{\pgfqpoint{1.807182in}{2.109033in}}{\pgfqpoint{1.807182in}{2.100797in}}%
\pgfpathcurveto{\pgfqpoint{1.807182in}{2.092561in}}{\pgfqpoint{1.810455in}{2.084661in}}{\pgfqpoint{1.816279in}{2.078837in}}%
\pgfpathcurveto{\pgfqpoint{1.822103in}{2.073013in}}{\pgfqpoint{1.830003in}{2.069740in}}{\pgfqpoint{1.838239in}{2.069740in}}%
\pgfpathclose%
\pgfusepath{stroke,fill}%
\end{pgfscope}%
\begin{pgfscope}%
\pgfpathrectangle{\pgfqpoint{0.100000in}{0.212622in}}{\pgfqpoint{3.696000in}{3.696000in}}%
\pgfusepath{clip}%
\pgfsetbuttcap%
\pgfsetroundjoin%
\definecolor{currentfill}{rgb}{0.121569,0.466667,0.705882}%
\pgfsetfillcolor{currentfill}%
\pgfsetfillopacity{0.317699}%
\pgfsetlinewidth{1.003750pt}%
\definecolor{currentstroke}{rgb}{0.121569,0.466667,0.705882}%
\pgfsetstrokecolor{currentstroke}%
\pgfsetstrokeopacity{0.317699}%
\pgfsetdash{}{0pt}%
\pgfpathmoveto{\pgfqpoint{1.650383in}{2.098019in}}%
\pgfpathcurveto{\pgfqpoint{1.658619in}{2.098019in}}{\pgfqpoint{1.666519in}{2.101291in}}{\pgfqpoint{1.672343in}{2.107115in}}%
\pgfpathcurveto{\pgfqpoint{1.678167in}{2.112939in}}{\pgfqpoint{1.681440in}{2.120839in}}{\pgfqpoint{1.681440in}{2.129076in}}%
\pgfpathcurveto{\pgfqpoint{1.681440in}{2.137312in}}{\pgfqpoint{1.678167in}{2.145212in}}{\pgfqpoint{1.672343in}{2.151036in}}%
\pgfpathcurveto{\pgfqpoint{1.666519in}{2.156860in}}{\pgfqpoint{1.658619in}{2.160132in}}{\pgfqpoint{1.650383in}{2.160132in}}%
\pgfpathcurveto{\pgfqpoint{1.642147in}{2.160132in}}{\pgfqpoint{1.634247in}{2.156860in}}{\pgfqpoint{1.628423in}{2.151036in}}%
\pgfpathcurveto{\pgfqpoint{1.622599in}{2.145212in}}{\pgfqpoint{1.619327in}{2.137312in}}{\pgfqpoint{1.619327in}{2.129076in}}%
\pgfpathcurveto{\pgfqpoint{1.619327in}{2.120839in}}{\pgfqpoint{1.622599in}{2.112939in}}{\pgfqpoint{1.628423in}{2.107115in}}%
\pgfpathcurveto{\pgfqpoint{1.634247in}{2.101291in}}{\pgfqpoint{1.642147in}{2.098019in}}{\pgfqpoint{1.650383in}{2.098019in}}%
\pgfpathclose%
\pgfusepath{stroke,fill}%
\end{pgfscope}%
\begin{pgfscope}%
\pgfpathrectangle{\pgfqpoint{0.100000in}{0.212622in}}{\pgfqpoint{3.696000in}{3.696000in}}%
\pgfusepath{clip}%
\pgfsetbuttcap%
\pgfsetroundjoin%
\definecolor{currentfill}{rgb}{0.121569,0.466667,0.705882}%
\pgfsetfillcolor{currentfill}%
\pgfsetfillopacity{0.317994}%
\pgfsetlinewidth{1.003750pt}%
\definecolor{currentstroke}{rgb}{0.121569,0.466667,0.705882}%
\pgfsetstrokecolor{currentstroke}%
\pgfsetstrokeopacity{0.317994}%
\pgfsetdash{}{0pt}%
\pgfpathmoveto{\pgfqpoint{1.843930in}{2.068988in}}%
\pgfpathcurveto{\pgfqpoint{1.852166in}{2.068988in}}{\pgfqpoint{1.860066in}{2.072260in}}{\pgfqpoint{1.865890in}{2.078084in}}%
\pgfpathcurveto{\pgfqpoint{1.871714in}{2.083908in}}{\pgfqpoint{1.874986in}{2.091808in}}{\pgfqpoint{1.874986in}{2.100044in}}%
\pgfpathcurveto{\pgfqpoint{1.874986in}{2.108281in}}{\pgfqpoint{1.871714in}{2.116181in}}{\pgfqpoint{1.865890in}{2.122005in}}%
\pgfpathcurveto{\pgfqpoint{1.860066in}{2.127829in}}{\pgfqpoint{1.852166in}{2.131101in}}{\pgfqpoint{1.843930in}{2.131101in}}%
\pgfpathcurveto{\pgfqpoint{1.835694in}{2.131101in}}{\pgfqpoint{1.827793in}{2.127829in}}{\pgfqpoint{1.821970in}{2.122005in}}%
\pgfpathcurveto{\pgfqpoint{1.816146in}{2.116181in}}{\pgfqpoint{1.812873in}{2.108281in}}{\pgfqpoint{1.812873in}{2.100044in}}%
\pgfpathcurveto{\pgfqpoint{1.812873in}{2.091808in}}{\pgfqpoint{1.816146in}{2.083908in}}{\pgfqpoint{1.821970in}{2.078084in}}%
\pgfpathcurveto{\pgfqpoint{1.827793in}{2.072260in}}{\pgfqpoint{1.835694in}{2.068988in}}{\pgfqpoint{1.843930in}{2.068988in}}%
\pgfpathclose%
\pgfusepath{stroke,fill}%
\end{pgfscope}%
\begin{pgfscope}%
\pgfpathrectangle{\pgfqpoint{0.100000in}{0.212622in}}{\pgfqpoint{3.696000in}{3.696000in}}%
\pgfusepath{clip}%
\pgfsetbuttcap%
\pgfsetroundjoin%
\definecolor{currentfill}{rgb}{0.121569,0.466667,0.705882}%
\pgfsetfillcolor{currentfill}%
\pgfsetfillopacity{0.318018}%
\pgfsetlinewidth{1.003750pt}%
\definecolor{currentstroke}{rgb}{0.121569,0.466667,0.705882}%
\pgfsetstrokecolor{currentstroke}%
\pgfsetstrokeopacity{0.318018}%
\pgfsetdash{}{0pt}%
\pgfpathmoveto{\pgfqpoint{1.649778in}{2.098041in}}%
\pgfpathcurveto{\pgfqpoint{1.658014in}{2.098041in}}{\pgfqpoint{1.665914in}{2.101314in}}{\pgfqpoint{1.671738in}{2.107138in}}%
\pgfpathcurveto{\pgfqpoint{1.677562in}{2.112962in}}{\pgfqpoint{1.680835in}{2.120862in}}{\pgfqpoint{1.680835in}{2.129098in}}%
\pgfpathcurveto{\pgfqpoint{1.680835in}{2.137334in}}{\pgfqpoint{1.677562in}{2.145234in}}{\pgfqpoint{1.671738in}{2.151058in}}%
\pgfpathcurveto{\pgfqpoint{1.665914in}{2.156882in}}{\pgfqpoint{1.658014in}{2.160154in}}{\pgfqpoint{1.649778in}{2.160154in}}%
\pgfpathcurveto{\pgfqpoint{1.641542in}{2.160154in}}{\pgfqpoint{1.633642in}{2.156882in}}{\pgfqpoint{1.627818in}{2.151058in}}%
\pgfpathcurveto{\pgfqpoint{1.621994in}{2.145234in}}{\pgfqpoint{1.618722in}{2.137334in}}{\pgfqpoint{1.618722in}{2.129098in}}%
\pgfpathcurveto{\pgfqpoint{1.618722in}{2.120862in}}{\pgfqpoint{1.621994in}{2.112962in}}{\pgfqpoint{1.627818in}{2.107138in}}%
\pgfpathcurveto{\pgfqpoint{1.633642in}{2.101314in}}{\pgfqpoint{1.641542in}{2.098041in}}{\pgfqpoint{1.649778in}{2.098041in}}%
\pgfpathclose%
\pgfusepath{stroke,fill}%
\end{pgfscope}%
\begin{pgfscope}%
\pgfpathrectangle{\pgfqpoint{0.100000in}{0.212622in}}{\pgfqpoint{3.696000in}{3.696000in}}%
\pgfusepath{clip}%
\pgfsetbuttcap%
\pgfsetroundjoin%
\definecolor{currentfill}{rgb}{0.121569,0.466667,0.705882}%
\pgfsetfillcolor{currentfill}%
\pgfsetfillopacity{0.318578}%
\pgfsetlinewidth{1.003750pt}%
\definecolor{currentstroke}{rgb}{0.121569,0.466667,0.705882}%
\pgfsetstrokecolor{currentstroke}%
\pgfsetstrokeopacity{0.318578}%
\pgfsetdash{}{0pt}%
\pgfpathmoveto{\pgfqpoint{1.648534in}{2.098055in}}%
\pgfpathcurveto{\pgfqpoint{1.656771in}{2.098055in}}{\pgfqpoint{1.664671in}{2.101327in}}{\pgfqpoint{1.670494in}{2.107151in}}%
\pgfpathcurveto{\pgfqpoint{1.676318in}{2.112975in}}{\pgfqpoint{1.679591in}{2.120875in}}{\pgfqpoint{1.679591in}{2.129111in}}%
\pgfpathcurveto{\pgfqpoint{1.679591in}{2.137347in}}{\pgfqpoint{1.676318in}{2.145247in}}{\pgfqpoint{1.670494in}{2.151071in}}%
\pgfpathcurveto{\pgfqpoint{1.664671in}{2.156895in}}{\pgfqpoint{1.656771in}{2.160168in}}{\pgfqpoint{1.648534in}{2.160168in}}%
\pgfpathcurveto{\pgfqpoint{1.640298in}{2.160168in}}{\pgfqpoint{1.632398in}{2.156895in}}{\pgfqpoint{1.626574in}{2.151071in}}%
\pgfpathcurveto{\pgfqpoint{1.620750in}{2.145247in}}{\pgfqpoint{1.617478in}{2.137347in}}{\pgfqpoint{1.617478in}{2.129111in}}%
\pgfpathcurveto{\pgfqpoint{1.617478in}{2.120875in}}{\pgfqpoint{1.620750in}{2.112975in}}{\pgfqpoint{1.626574in}{2.107151in}}%
\pgfpathcurveto{\pgfqpoint{1.632398in}{2.101327in}}{\pgfqpoint{1.640298in}{2.098055in}}{\pgfqpoint{1.648534in}{2.098055in}}%
\pgfpathclose%
\pgfusepath{stroke,fill}%
\end{pgfscope}%
\begin{pgfscope}%
\pgfpathrectangle{\pgfqpoint{0.100000in}{0.212622in}}{\pgfqpoint{3.696000in}{3.696000in}}%
\pgfusepath{clip}%
\pgfsetbuttcap%
\pgfsetroundjoin%
\definecolor{currentfill}{rgb}{0.121569,0.466667,0.705882}%
\pgfsetfillcolor{currentfill}%
\pgfsetfillopacity{0.318820}%
\pgfsetlinewidth{1.003750pt}%
\definecolor{currentstroke}{rgb}{0.121569,0.466667,0.705882}%
\pgfsetstrokecolor{currentstroke}%
\pgfsetstrokeopacity{0.318820}%
\pgfsetdash{}{0pt}%
\pgfpathmoveto{\pgfqpoint{1.850217in}{2.067831in}}%
\pgfpathcurveto{\pgfqpoint{1.858453in}{2.067831in}}{\pgfqpoint{1.866353in}{2.071103in}}{\pgfqpoint{1.872177in}{2.076927in}}%
\pgfpathcurveto{\pgfqpoint{1.878001in}{2.082751in}}{\pgfqpoint{1.881273in}{2.090651in}}{\pgfqpoint{1.881273in}{2.098888in}}%
\pgfpathcurveto{\pgfqpoint{1.881273in}{2.107124in}}{\pgfqpoint{1.878001in}{2.115024in}}{\pgfqpoint{1.872177in}{2.120848in}}%
\pgfpathcurveto{\pgfqpoint{1.866353in}{2.126672in}}{\pgfqpoint{1.858453in}{2.129944in}}{\pgfqpoint{1.850217in}{2.129944in}}%
\pgfpathcurveto{\pgfqpoint{1.841980in}{2.129944in}}{\pgfqpoint{1.834080in}{2.126672in}}{\pgfqpoint{1.828257in}{2.120848in}}%
\pgfpathcurveto{\pgfqpoint{1.822433in}{2.115024in}}{\pgfqpoint{1.819160in}{2.107124in}}{\pgfqpoint{1.819160in}{2.098888in}}%
\pgfpathcurveto{\pgfqpoint{1.819160in}{2.090651in}}{\pgfqpoint{1.822433in}{2.082751in}}{\pgfqpoint{1.828257in}{2.076927in}}%
\pgfpathcurveto{\pgfqpoint{1.834080in}{2.071103in}}{\pgfqpoint{1.841980in}{2.067831in}}{\pgfqpoint{1.850217in}{2.067831in}}%
\pgfpathclose%
\pgfusepath{stroke,fill}%
\end{pgfscope}%
\begin{pgfscope}%
\pgfpathrectangle{\pgfqpoint{0.100000in}{0.212622in}}{\pgfqpoint{3.696000in}{3.696000in}}%
\pgfusepath{clip}%
\pgfsetbuttcap%
\pgfsetroundjoin%
\definecolor{currentfill}{rgb}{0.121569,0.466667,0.705882}%
\pgfsetfillcolor{currentfill}%
\pgfsetfillopacity{0.319101}%
\pgfsetlinewidth{1.003750pt}%
\definecolor{currentstroke}{rgb}{0.121569,0.466667,0.705882}%
\pgfsetstrokecolor{currentstroke}%
\pgfsetstrokeopacity{0.319101}%
\pgfsetdash{}{0pt}%
\pgfpathmoveto{\pgfqpoint{1.647845in}{2.098026in}}%
\pgfpathcurveto{\pgfqpoint{1.656081in}{2.098026in}}{\pgfqpoint{1.663981in}{2.101298in}}{\pgfqpoint{1.669805in}{2.107122in}}%
\pgfpathcurveto{\pgfqpoint{1.675629in}{2.112946in}}{\pgfqpoint{1.678901in}{2.120846in}}{\pgfqpoint{1.678901in}{2.129082in}}%
\pgfpathcurveto{\pgfqpoint{1.678901in}{2.137319in}}{\pgfqpoint{1.675629in}{2.145219in}}{\pgfqpoint{1.669805in}{2.151043in}}%
\pgfpathcurveto{\pgfqpoint{1.663981in}{2.156867in}}{\pgfqpoint{1.656081in}{2.160139in}}{\pgfqpoint{1.647845in}{2.160139in}}%
\pgfpathcurveto{\pgfqpoint{1.639608in}{2.160139in}}{\pgfqpoint{1.631708in}{2.156867in}}{\pgfqpoint{1.625884in}{2.151043in}}%
\pgfpathcurveto{\pgfqpoint{1.620060in}{2.145219in}}{\pgfqpoint{1.616788in}{2.137319in}}{\pgfqpoint{1.616788in}{2.129082in}}%
\pgfpathcurveto{\pgfqpoint{1.616788in}{2.120846in}}{\pgfqpoint{1.620060in}{2.112946in}}{\pgfqpoint{1.625884in}{2.107122in}}%
\pgfpathcurveto{\pgfqpoint{1.631708in}{2.101298in}}{\pgfqpoint{1.639608in}{2.098026in}}{\pgfqpoint{1.647845in}{2.098026in}}%
\pgfpathclose%
\pgfusepath{stroke,fill}%
\end{pgfscope}%
\begin{pgfscope}%
\pgfpathrectangle{\pgfqpoint{0.100000in}{0.212622in}}{\pgfqpoint{3.696000in}{3.696000in}}%
\pgfusepath{clip}%
\pgfsetbuttcap%
\pgfsetroundjoin%
\definecolor{currentfill}{rgb}{0.121569,0.466667,0.705882}%
\pgfsetfillcolor{currentfill}%
\pgfsetfillopacity{0.319370}%
\pgfsetlinewidth{1.003750pt}%
\definecolor{currentstroke}{rgb}{0.121569,0.466667,0.705882}%
\pgfsetstrokecolor{currentstroke}%
\pgfsetstrokeopacity{0.319370}%
\pgfsetdash{}{0pt}%
\pgfpathmoveto{\pgfqpoint{1.853477in}{2.067363in}}%
\pgfpathcurveto{\pgfqpoint{1.861713in}{2.067363in}}{\pgfqpoint{1.869613in}{2.070635in}}{\pgfqpoint{1.875437in}{2.076459in}}%
\pgfpathcurveto{\pgfqpoint{1.881261in}{2.082283in}}{\pgfqpoint{1.884533in}{2.090183in}}{\pgfqpoint{1.884533in}{2.098419in}}%
\pgfpathcurveto{\pgfqpoint{1.884533in}{2.106655in}}{\pgfqpoint{1.881261in}{2.114555in}}{\pgfqpoint{1.875437in}{2.120379in}}%
\pgfpathcurveto{\pgfqpoint{1.869613in}{2.126203in}}{\pgfqpoint{1.861713in}{2.129476in}}{\pgfqpoint{1.853477in}{2.129476in}}%
\pgfpathcurveto{\pgfqpoint{1.845241in}{2.129476in}}{\pgfqpoint{1.837340in}{2.126203in}}{\pgfqpoint{1.831517in}{2.120379in}}%
\pgfpathcurveto{\pgfqpoint{1.825693in}{2.114555in}}{\pgfqpoint{1.822420in}{2.106655in}}{\pgfqpoint{1.822420in}{2.098419in}}%
\pgfpathcurveto{\pgfqpoint{1.822420in}{2.090183in}}{\pgfqpoint{1.825693in}{2.082283in}}{\pgfqpoint{1.831517in}{2.076459in}}%
\pgfpathcurveto{\pgfqpoint{1.837340in}{2.070635in}}{\pgfqpoint{1.845241in}{2.067363in}}{\pgfqpoint{1.853477in}{2.067363in}}%
\pgfpathclose%
\pgfusepath{stroke,fill}%
\end{pgfscope}%
\begin{pgfscope}%
\pgfpathrectangle{\pgfqpoint{0.100000in}{0.212622in}}{\pgfqpoint{3.696000in}{3.696000in}}%
\pgfusepath{clip}%
\pgfsetbuttcap%
\pgfsetroundjoin%
\definecolor{currentfill}{rgb}{0.121569,0.466667,0.705882}%
\pgfsetfillcolor{currentfill}%
\pgfsetfillopacity{0.320012}%
\pgfsetlinewidth{1.003750pt}%
\definecolor{currentstroke}{rgb}{0.121569,0.466667,0.705882}%
\pgfsetstrokecolor{currentstroke}%
\pgfsetstrokeopacity{0.320012}%
\pgfsetdash{}{0pt}%
\pgfpathmoveto{\pgfqpoint{1.857041in}{2.066796in}}%
\pgfpathcurveto{\pgfqpoint{1.865277in}{2.066796in}}{\pgfqpoint{1.873177in}{2.070068in}}{\pgfqpoint{1.879001in}{2.075892in}}%
\pgfpathcurveto{\pgfqpoint{1.884825in}{2.081716in}}{\pgfqpoint{1.888098in}{2.089616in}}{\pgfqpoint{1.888098in}{2.097852in}}%
\pgfpathcurveto{\pgfqpoint{1.888098in}{2.106089in}}{\pgfqpoint{1.884825in}{2.113989in}}{\pgfqpoint{1.879001in}{2.119813in}}%
\pgfpathcurveto{\pgfqpoint{1.873177in}{2.125636in}}{\pgfqpoint{1.865277in}{2.128909in}}{\pgfqpoint{1.857041in}{2.128909in}}%
\pgfpathcurveto{\pgfqpoint{1.848805in}{2.128909in}}{\pgfqpoint{1.840905in}{2.125636in}}{\pgfqpoint{1.835081in}{2.119813in}}%
\pgfpathcurveto{\pgfqpoint{1.829257in}{2.113989in}}{\pgfqpoint{1.825985in}{2.106089in}}{\pgfqpoint{1.825985in}{2.097852in}}%
\pgfpathcurveto{\pgfqpoint{1.825985in}{2.089616in}}{\pgfqpoint{1.829257in}{2.081716in}}{\pgfqpoint{1.835081in}{2.075892in}}%
\pgfpathcurveto{\pgfqpoint{1.840905in}{2.070068in}}{\pgfqpoint{1.848805in}{2.066796in}}{\pgfqpoint{1.857041in}{2.066796in}}%
\pgfpathclose%
\pgfusepath{stroke,fill}%
\end{pgfscope}%
\begin{pgfscope}%
\pgfpathrectangle{\pgfqpoint{0.100000in}{0.212622in}}{\pgfqpoint{3.696000in}{3.696000in}}%
\pgfusepath{clip}%
\pgfsetbuttcap%
\pgfsetroundjoin%
\definecolor{currentfill}{rgb}{0.121569,0.466667,0.705882}%
\pgfsetfillcolor{currentfill}%
\pgfsetfillopacity{0.320022}%
\pgfsetlinewidth{1.003750pt}%
\definecolor{currentstroke}{rgb}{0.121569,0.466667,0.705882}%
\pgfsetstrokecolor{currentstroke}%
\pgfsetstrokeopacity{0.320022}%
\pgfsetdash{}{0pt}%
\pgfpathmoveto{\pgfqpoint{1.646047in}{2.098147in}}%
\pgfpathcurveto{\pgfqpoint{1.654283in}{2.098147in}}{\pgfqpoint{1.662183in}{2.101420in}}{\pgfqpoint{1.668007in}{2.107244in}}%
\pgfpathcurveto{\pgfqpoint{1.673831in}{2.113068in}}{\pgfqpoint{1.677103in}{2.120968in}}{\pgfqpoint{1.677103in}{2.129204in}}%
\pgfpathcurveto{\pgfqpoint{1.677103in}{2.137440in}}{\pgfqpoint{1.673831in}{2.145340in}}{\pgfqpoint{1.668007in}{2.151164in}}%
\pgfpathcurveto{\pgfqpoint{1.662183in}{2.156988in}}{\pgfqpoint{1.654283in}{2.160260in}}{\pgfqpoint{1.646047in}{2.160260in}}%
\pgfpathcurveto{\pgfqpoint{1.637811in}{2.160260in}}{\pgfqpoint{1.629911in}{2.156988in}}{\pgfqpoint{1.624087in}{2.151164in}}%
\pgfpathcurveto{\pgfqpoint{1.618263in}{2.145340in}}{\pgfqpoint{1.614990in}{2.137440in}}{\pgfqpoint{1.614990in}{2.129204in}}%
\pgfpathcurveto{\pgfqpoint{1.614990in}{2.120968in}}{\pgfqpoint{1.618263in}{2.113068in}}{\pgfqpoint{1.624087in}{2.107244in}}%
\pgfpathcurveto{\pgfqpoint{1.629911in}{2.101420in}}{\pgfqpoint{1.637811in}{2.098147in}}{\pgfqpoint{1.646047in}{2.098147in}}%
\pgfpathclose%
\pgfusepath{stroke,fill}%
\end{pgfscope}%
\begin{pgfscope}%
\pgfpathrectangle{\pgfqpoint{0.100000in}{0.212622in}}{\pgfqpoint{3.696000in}{3.696000in}}%
\pgfusepath{clip}%
\pgfsetbuttcap%
\pgfsetroundjoin%
\definecolor{currentfill}{rgb}{0.121569,0.466667,0.705882}%
\pgfsetfillcolor{currentfill}%
\pgfsetfillopacity{0.320730}%
\pgfsetlinewidth{1.003750pt}%
\definecolor{currentstroke}{rgb}{0.121569,0.466667,0.705882}%
\pgfsetstrokecolor{currentstroke}%
\pgfsetstrokeopacity{0.320730}%
\pgfsetdash{}{0pt}%
\pgfpathmoveto{\pgfqpoint{1.862417in}{2.065777in}}%
\pgfpathcurveto{\pgfqpoint{1.870653in}{2.065777in}}{\pgfqpoint{1.878553in}{2.069049in}}{\pgfqpoint{1.884377in}{2.074873in}}%
\pgfpathcurveto{\pgfqpoint{1.890201in}{2.080697in}}{\pgfqpoint{1.893473in}{2.088597in}}{\pgfqpoint{1.893473in}{2.096833in}}%
\pgfpathcurveto{\pgfqpoint{1.893473in}{2.105069in}}{\pgfqpoint{1.890201in}{2.112970in}}{\pgfqpoint{1.884377in}{2.118793in}}%
\pgfpathcurveto{\pgfqpoint{1.878553in}{2.124617in}}{\pgfqpoint{1.870653in}{2.127890in}}{\pgfqpoint{1.862417in}{2.127890in}}%
\pgfpathcurveto{\pgfqpoint{1.854180in}{2.127890in}}{\pgfqpoint{1.846280in}{2.124617in}}{\pgfqpoint{1.840456in}{2.118793in}}%
\pgfpathcurveto{\pgfqpoint{1.834632in}{2.112970in}}{\pgfqpoint{1.831360in}{2.105069in}}{\pgfqpoint{1.831360in}{2.096833in}}%
\pgfpathcurveto{\pgfqpoint{1.831360in}{2.088597in}}{\pgfqpoint{1.834632in}{2.080697in}}{\pgfqpoint{1.840456in}{2.074873in}}%
\pgfpathcurveto{\pgfqpoint{1.846280in}{2.069049in}}{\pgfqpoint{1.854180in}{2.065777in}}{\pgfqpoint{1.862417in}{2.065777in}}%
\pgfpathclose%
\pgfusepath{stroke,fill}%
\end{pgfscope}%
\begin{pgfscope}%
\pgfpathrectangle{\pgfqpoint{0.100000in}{0.212622in}}{\pgfqpoint{3.696000in}{3.696000in}}%
\pgfusepath{clip}%
\pgfsetbuttcap%
\pgfsetroundjoin%
\definecolor{currentfill}{rgb}{0.121569,0.466667,0.705882}%
\pgfsetfillcolor{currentfill}%
\pgfsetfillopacity{0.320823}%
\pgfsetlinewidth{1.003750pt}%
\definecolor{currentstroke}{rgb}{0.121569,0.466667,0.705882}%
\pgfsetstrokecolor{currentstroke}%
\pgfsetstrokeopacity{0.320823}%
\pgfsetdash{}{0pt}%
\pgfpathmoveto{\pgfqpoint{1.644135in}{2.098294in}}%
\pgfpathcurveto{\pgfqpoint{1.652371in}{2.098294in}}{\pgfqpoint{1.660271in}{2.101567in}}{\pgfqpoint{1.666095in}{2.107390in}}%
\pgfpathcurveto{\pgfqpoint{1.671919in}{2.113214in}}{\pgfqpoint{1.675191in}{2.121114in}}{\pgfqpoint{1.675191in}{2.129351in}}%
\pgfpathcurveto{\pgfqpoint{1.675191in}{2.137587in}}{\pgfqpoint{1.671919in}{2.145487in}}{\pgfqpoint{1.666095in}{2.151311in}}%
\pgfpathcurveto{\pgfqpoint{1.660271in}{2.157135in}}{\pgfqpoint{1.652371in}{2.160407in}}{\pgfqpoint{1.644135in}{2.160407in}}%
\pgfpathcurveto{\pgfqpoint{1.635899in}{2.160407in}}{\pgfqpoint{1.627999in}{2.157135in}}{\pgfqpoint{1.622175in}{2.151311in}}%
\pgfpathcurveto{\pgfqpoint{1.616351in}{2.145487in}}{\pgfqpoint{1.613078in}{2.137587in}}{\pgfqpoint{1.613078in}{2.129351in}}%
\pgfpathcurveto{\pgfqpoint{1.613078in}{2.121114in}}{\pgfqpoint{1.616351in}{2.113214in}}{\pgfqpoint{1.622175in}{2.107390in}}%
\pgfpathcurveto{\pgfqpoint{1.627999in}{2.101567in}}{\pgfqpoint{1.635899in}{2.098294in}}{\pgfqpoint{1.644135in}{2.098294in}}%
\pgfpathclose%
\pgfusepath{stroke,fill}%
\end{pgfscope}%
\begin{pgfscope}%
\pgfpathrectangle{\pgfqpoint{0.100000in}{0.212622in}}{\pgfqpoint{3.696000in}{3.696000in}}%
\pgfusepath{clip}%
\pgfsetbuttcap%
\pgfsetroundjoin%
\definecolor{currentfill}{rgb}{0.121569,0.466667,0.705882}%
\pgfsetfillcolor{currentfill}%
\pgfsetfillopacity{0.321458}%
\pgfsetlinewidth{1.003750pt}%
\definecolor{currentstroke}{rgb}{0.121569,0.466667,0.705882}%
\pgfsetstrokecolor{currentstroke}%
\pgfsetstrokeopacity{0.321458}%
\pgfsetdash{}{0pt}%
\pgfpathmoveto{\pgfqpoint{1.643392in}{2.098415in}}%
\pgfpathcurveto{\pgfqpoint{1.651629in}{2.098415in}}{\pgfqpoint{1.659529in}{2.101687in}}{\pgfqpoint{1.665353in}{2.107511in}}%
\pgfpathcurveto{\pgfqpoint{1.671177in}{2.113335in}}{\pgfqpoint{1.674449in}{2.121235in}}{\pgfqpoint{1.674449in}{2.129471in}}%
\pgfpathcurveto{\pgfqpoint{1.674449in}{2.137707in}}{\pgfqpoint{1.671177in}{2.145607in}}{\pgfqpoint{1.665353in}{2.151431in}}%
\pgfpathcurveto{\pgfqpoint{1.659529in}{2.157255in}}{\pgfqpoint{1.651629in}{2.160528in}}{\pgfqpoint{1.643392in}{2.160528in}}%
\pgfpathcurveto{\pgfqpoint{1.635156in}{2.160528in}}{\pgfqpoint{1.627256in}{2.157255in}}{\pgfqpoint{1.621432in}{2.151431in}}%
\pgfpathcurveto{\pgfqpoint{1.615608in}{2.145607in}}{\pgfqpoint{1.612336in}{2.137707in}}{\pgfqpoint{1.612336in}{2.129471in}}%
\pgfpathcurveto{\pgfqpoint{1.612336in}{2.121235in}}{\pgfqpoint{1.615608in}{2.113335in}}{\pgfqpoint{1.621432in}{2.107511in}}%
\pgfpathcurveto{\pgfqpoint{1.627256in}{2.101687in}}{\pgfqpoint{1.635156in}{2.098415in}}{\pgfqpoint{1.643392in}{2.098415in}}%
\pgfpathclose%
\pgfusepath{stroke,fill}%
\end{pgfscope}%
\begin{pgfscope}%
\pgfpathrectangle{\pgfqpoint{0.100000in}{0.212622in}}{\pgfqpoint{3.696000in}{3.696000in}}%
\pgfusepath{clip}%
\pgfsetbuttcap%
\pgfsetroundjoin%
\definecolor{currentfill}{rgb}{0.121569,0.466667,0.705882}%
\pgfsetfillcolor{currentfill}%
\pgfsetfillopacity{0.321503}%
\pgfsetlinewidth{1.003750pt}%
\definecolor{currentstroke}{rgb}{0.121569,0.466667,0.705882}%
\pgfsetstrokecolor{currentstroke}%
\pgfsetstrokeopacity{0.321503}%
\pgfsetdash{}{0pt}%
\pgfpathmoveto{\pgfqpoint{1.868129in}{2.064817in}}%
\pgfpathcurveto{\pgfqpoint{1.876366in}{2.064817in}}{\pgfqpoint{1.884266in}{2.068089in}}{\pgfqpoint{1.890090in}{2.073913in}}%
\pgfpathcurveto{\pgfqpoint{1.895914in}{2.079737in}}{\pgfqpoint{1.899186in}{2.087637in}}{\pgfqpoint{1.899186in}{2.095873in}}%
\pgfpathcurveto{\pgfqpoint{1.899186in}{2.104110in}}{\pgfqpoint{1.895914in}{2.112010in}}{\pgfqpoint{1.890090in}{2.117834in}}%
\pgfpathcurveto{\pgfqpoint{1.884266in}{2.123658in}}{\pgfqpoint{1.876366in}{2.126930in}}{\pgfqpoint{1.868129in}{2.126930in}}%
\pgfpathcurveto{\pgfqpoint{1.859893in}{2.126930in}}{\pgfqpoint{1.851993in}{2.123658in}}{\pgfqpoint{1.846169in}{2.117834in}}%
\pgfpathcurveto{\pgfqpoint{1.840345in}{2.112010in}}{\pgfqpoint{1.837073in}{2.104110in}}{\pgfqpoint{1.837073in}{2.095873in}}%
\pgfpathcurveto{\pgfqpoint{1.837073in}{2.087637in}}{\pgfqpoint{1.840345in}{2.079737in}}{\pgfqpoint{1.846169in}{2.073913in}}%
\pgfpathcurveto{\pgfqpoint{1.851993in}{2.068089in}}{\pgfqpoint{1.859893in}{2.064817in}}{\pgfqpoint{1.868129in}{2.064817in}}%
\pgfpathclose%
\pgfusepath{stroke,fill}%
\end{pgfscope}%
\begin{pgfscope}%
\pgfpathrectangle{\pgfqpoint{0.100000in}{0.212622in}}{\pgfqpoint{3.696000in}{3.696000in}}%
\pgfusepath{clip}%
\pgfsetbuttcap%
\pgfsetroundjoin%
\definecolor{currentfill}{rgb}{0.121569,0.466667,0.705882}%
\pgfsetfillcolor{currentfill}%
\pgfsetfillopacity{0.321781}%
\pgfsetlinewidth{1.003750pt}%
\definecolor{currentstroke}{rgb}{0.121569,0.466667,0.705882}%
\pgfsetstrokecolor{currentstroke}%
\pgfsetstrokeopacity{0.321781}%
\pgfsetdash{}{0pt}%
\pgfpathmoveto{\pgfqpoint{1.642530in}{2.098496in}}%
\pgfpathcurveto{\pgfqpoint{1.650766in}{2.098496in}}{\pgfqpoint{1.658667in}{2.101768in}}{\pgfqpoint{1.664490in}{2.107592in}}%
\pgfpathcurveto{\pgfqpoint{1.670314in}{2.113416in}}{\pgfqpoint{1.673587in}{2.121316in}}{\pgfqpoint{1.673587in}{2.129553in}}%
\pgfpathcurveto{\pgfqpoint{1.673587in}{2.137789in}}{\pgfqpoint{1.670314in}{2.145689in}}{\pgfqpoint{1.664490in}{2.151513in}}%
\pgfpathcurveto{\pgfqpoint{1.658667in}{2.157337in}}{\pgfqpoint{1.650766in}{2.160609in}}{\pgfqpoint{1.642530in}{2.160609in}}%
\pgfpathcurveto{\pgfqpoint{1.634294in}{2.160609in}}{\pgfqpoint{1.626394in}{2.157337in}}{\pgfqpoint{1.620570in}{2.151513in}}%
\pgfpathcurveto{\pgfqpoint{1.614746in}{2.145689in}}{\pgfqpoint{1.611474in}{2.137789in}}{\pgfqpoint{1.611474in}{2.129553in}}%
\pgfpathcurveto{\pgfqpoint{1.611474in}{2.121316in}}{\pgfqpoint{1.614746in}{2.113416in}}{\pgfqpoint{1.620570in}{2.107592in}}%
\pgfpathcurveto{\pgfqpoint{1.626394in}{2.101768in}}{\pgfqpoint{1.634294in}{2.098496in}}{\pgfqpoint{1.642530in}{2.098496in}}%
\pgfpathclose%
\pgfusepath{stroke,fill}%
\end{pgfscope}%
\begin{pgfscope}%
\pgfpathrectangle{\pgfqpoint{0.100000in}{0.212622in}}{\pgfqpoint{3.696000in}{3.696000in}}%
\pgfusepath{clip}%
\pgfsetbuttcap%
\pgfsetroundjoin%
\definecolor{currentfill}{rgb}{0.121569,0.466667,0.705882}%
\pgfsetfillcolor{currentfill}%
\pgfsetfillopacity{0.321980}%
\pgfsetlinewidth{1.003750pt}%
\definecolor{currentstroke}{rgb}{0.121569,0.466667,0.705882}%
\pgfsetstrokecolor{currentstroke}%
\pgfsetstrokeopacity{0.321980}%
\pgfsetdash{}{0pt}%
\pgfpathmoveto{\pgfqpoint{1.874654in}{2.063332in}}%
\pgfpathcurveto{\pgfqpoint{1.882890in}{2.063332in}}{\pgfqpoint{1.890790in}{2.066605in}}{\pgfqpoint{1.896614in}{2.072429in}}%
\pgfpathcurveto{\pgfqpoint{1.902438in}{2.078253in}}{\pgfqpoint{1.905710in}{2.086153in}}{\pgfqpoint{1.905710in}{2.094389in}}%
\pgfpathcurveto{\pgfqpoint{1.905710in}{2.102625in}}{\pgfqpoint{1.902438in}{2.110525in}}{\pgfqpoint{1.896614in}{2.116349in}}%
\pgfpathcurveto{\pgfqpoint{1.890790in}{2.122173in}}{\pgfqpoint{1.882890in}{2.125445in}}{\pgfqpoint{1.874654in}{2.125445in}}%
\pgfpathcurveto{\pgfqpoint{1.866418in}{2.125445in}}{\pgfqpoint{1.858517in}{2.122173in}}{\pgfqpoint{1.852694in}{2.116349in}}%
\pgfpathcurveto{\pgfqpoint{1.846870in}{2.110525in}}{\pgfqpoint{1.843597in}{2.102625in}}{\pgfqpoint{1.843597in}{2.094389in}}%
\pgfpathcurveto{\pgfqpoint{1.843597in}{2.086153in}}{\pgfqpoint{1.846870in}{2.078253in}}{\pgfqpoint{1.852694in}{2.072429in}}%
\pgfpathcurveto{\pgfqpoint{1.858517in}{2.066605in}}{\pgfqpoint{1.866418in}{2.063332in}}{\pgfqpoint{1.874654in}{2.063332in}}%
\pgfpathclose%
\pgfusepath{stroke,fill}%
\end{pgfscope}%
\begin{pgfscope}%
\pgfpathrectangle{\pgfqpoint{0.100000in}{0.212622in}}{\pgfqpoint{3.696000in}{3.696000in}}%
\pgfusepath{clip}%
\pgfsetbuttcap%
\pgfsetroundjoin%
\definecolor{currentfill}{rgb}{0.121569,0.466667,0.705882}%
\pgfsetfillcolor{currentfill}%
\pgfsetfillopacity{0.322396}%
\pgfsetlinewidth{1.003750pt}%
\definecolor{currentstroke}{rgb}{0.121569,0.466667,0.705882}%
\pgfsetstrokecolor{currentstroke}%
\pgfsetstrokeopacity{0.322396}%
\pgfsetdash{}{0pt}%
\pgfpathmoveto{\pgfqpoint{1.641260in}{2.098520in}}%
\pgfpathcurveto{\pgfqpoint{1.649497in}{2.098520in}}{\pgfqpoint{1.657397in}{2.101792in}}{\pgfqpoint{1.663221in}{2.107616in}}%
\pgfpathcurveto{\pgfqpoint{1.669045in}{2.113440in}}{\pgfqpoint{1.672317in}{2.121340in}}{\pgfqpoint{1.672317in}{2.129576in}}%
\pgfpathcurveto{\pgfqpoint{1.672317in}{2.137812in}}{\pgfqpoint{1.669045in}{2.145712in}}{\pgfqpoint{1.663221in}{2.151536in}}%
\pgfpathcurveto{\pgfqpoint{1.657397in}{2.157360in}}{\pgfqpoint{1.649497in}{2.160633in}}{\pgfqpoint{1.641260in}{2.160633in}}%
\pgfpathcurveto{\pgfqpoint{1.633024in}{2.160633in}}{\pgfqpoint{1.625124in}{2.157360in}}{\pgfqpoint{1.619300in}{2.151536in}}%
\pgfpathcurveto{\pgfqpoint{1.613476in}{2.145712in}}{\pgfqpoint{1.610204in}{2.137812in}}{\pgfqpoint{1.610204in}{2.129576in}}%
\pgfpathcurveto{\pgfqpoint{1.610204in}{2.121340in}}{\pgfqpoint{1.613476in}{2.113440in}}{\pgfqpoint{1.619300in}{2.107616in}}%
\pgfpathcurveto{\pgfqpoint{1.625124in}{2.101792in}}{\pgfqpoint{1.633024in}{2.098520in}}{\pgfqpoint{1.641260in}{2.098520in}}%
\pgfpathclose%
\pgfusepath{stroke,fill}%
\end{pgfscope}%
\begin{pgfscope}%
\pgfpathrectangle{\pgfqpoint{0.100000in}{0.212622in}}{\pgfqpoint{3.696000in}{3.696000in}}%
\pgfusepath{clip}%
\pgfsetbuttcap%
\pgfsetroundjoin%
\definecolor{currentfill}{rgb}{0.121569,0.466667,0.705882}%
\pgfsetfillcolor{currentfill}%
\pgfsetfillopacity{0.323217}%
\pgfsetlinewidth{1.003750pt}%
\definecolor{currentstroke}{rgb}{0.121569,0.466667,0.705882}%
\pgfsetstrokecolor{currentstroke}%
\pgfsetstrokeopacity{0.323217}%
\pgfsetdash{}{0pt}%
\pgfpathmoveto{\pgfqpoint{1.879898in}{2.062605in}}%
\pgfpathcurveto{\pgfqpoint{1.888134in}{2.062605in}}{\pgfqpoint{1.896034in}{2.065877in}}{\pgfqpoint{1.901858in}{2.071701in}}%
\pgfpathcurveto{\pgfqpoint{1.907682in}{2.077525in}}{\pgfqpoint{1.910954in}{2.085425in}}{\pgfqpoint{1.910954in}{2.093661in}}%
\pgfpathcurveto{\pgfqpoint{1.910954in}{2.101898in}}{\pgfqpoint{1.907682in}{2.109798in}}{\pgfqpoint{1.901858in}{2.115622in}}%
\pgfpathcurveto{\pgfqpoint{1.896034in}{2.121446in}}{\pgfqpoint{1.888134in}{2.124718in}}{\pgfqpoint{1.879898in}{2.124718in}}%
\pgfpathcurveto{\pgfqpoint{1.871661in}{2.124718in}}{\pgfqpoint{1.863761in}{2.121446in}}{\pgfqpoint{1.857937in}{2.115622in}}%
\pgfpathcurveto{\pgfqpoint{1.852113in}{2.109798in}}{\pgfqpoint{1.848841in}{2.101898in}}{\pgfqpoint{1.848841in}{2.093661in}}%
\pgfpathcurveto{\pgfqpoint{1.848841in}{2.085425in}}{\pgfqpoint{1.852113in}{2.077525in}}{\pgfqpoint{1.857937in}{2.071701in}}%
\pgfpathcurveto{\pgfqpoint{1.863761in}{2.065877in}}{\pgfqpoint{1.871661in}{2.062605in}}{\pgfqpoint{1.879898in}{2.062605in}}%
\pgfpathclose%
\pgfusepath{stroke,fill}%
\end{pgfscope}%
\begin{pgfscope}%
\pgfpathrectangle{\pgfqpoint{0.100000in}{0.212622in}}{\pgfqpoint{3.696000in}{3.696000in}}%
\pgfusepath{clip}%
\pgfsetbuttcap%
\pgfsetroundjoin%
\definecolor{currentfill}{rgb}{0.121569,0.466667,0.705882}%
\pgfsetfillcolor{currentfill}%
\pgfsetfillopacity{0.323575}%
\pgfsetlinewidth{1.003750pt}%
\definecolor{currentstroke}{rgb}{0.121569,0.466667,0.705882}%
\pgfsetstrokecolor{currentstroke}%
\pgfsetstrokeopacity{0.323575}%
\pgfsetdash{}{0pt}%
\pgfpathmoveto{\pgfqpoint{1.639460in}{2.098595in}}%
\pgfpathcurveto{\pgfqpoint{1.647696in}{2.098595in}}{\pgfqpoint{1.655596in}{2.101868in}}{\pgfqpoint{1.661420in}{2.107692in}}%
\pgfpathcurveto{\pgfqpoint{1.667244in}{2.113516in}}{\pgfqpoint{1.670517in}{2.121416in}}{\pgfqpoint{1.670517in}{2.129652in}}%
\pgfpathcurveto{\pgfqpoint{1.670517in}{2.137888in}}{\pgfqpoint{1.667244in}{2.145788in}}{\pgfqpoint{1.661420in}{2.151612in}}%
\pgfpathcurveto{\pgfqpoint{1.655596in}{2.157436in}}{\pgfqpoint{1.647696in}{2.160708in}}{\pgfqpoint{1.639460in}{2.160708in}}%
\pgfpathcurveto{\pgfqpoint{1.631224in}{2.160708in}}{\pgfqpoint{1.623324in}{2.157436in}}{\pgfqpoint{1.617500in}{2.151612in}}%
\pgfpathcurveto{\pgfqpoint{1.611676in}{2.145788in}}{\pgfqpoint{1.608404in}{2.137888in}}{\pgfqpoint{1.608404in}{2.129652in}}%
\pgfpathcurveto{\pgfqpoint{1.608404in}{2.121416in}}{\pgfqpoint{1.611676in}{2.113516in}}{\pgfqpoint{1.617500in}{2.107692in}}%
\pgfpathcurveto{\pgfqpoint{1.623324in}{2.101868in}}{\pgfqpoint{1.631224in}{2.098595in}}{\pgfqpoint{1.639460in}{2.098595in}}%
\pgfpathclose%
\pgfusepath{stroke,fill}%
\end{pgfscope}%
\begin{pgfscope}%
\pgfpathrectangle{\pgfqpoint{0.100000in}{0.212622in}}{\pgfqpoint{3.696000in}{3.696000in}}%
\pgfusepath{clip}%
\pgfsetbuttcap%
\pgfsetroundjoin%
\definecolor{currentfill}{rgb}{0.121569,0.466667,0.705882}%
\pgfsetfillcolor{currentfill}%
\pgfsetfillopacity{0.324306}%
\pgfsetlinewidth{1.003750pt}%
\definecolor{currentstroke}{rgb}{0.121569,0.466667,0.705882}%
\pgfsetstrokecolor{currentstroke}%
\pgfsetstrokeopacity{0.324306}%
\pgfsetdash{}{0pt}%
\pgfpathmoveto{\pgfqpoint{1.887305in}{2.060911in}}%
\pgfpathcurveto{\pgfqpoint{1.895541in}{2.060911in}}{\pgfqpoint{1.903441in}{2.064183in}}{\pgfqpoint{1.909265in}{2.070007in}}%
\pgfpathcurveto{\pgfqpoint{1.915089in}{2.075831in}}{\pgfqpoint{1.918361in}{2.083731in}}{\pgfqpoint{1.918361in}{2.091967in}}%
\pgfpathcurveto{\pgfqpoint{1.918361in}{2.100204in}}{\pgfqpoint{1.915089in}{2.108104in}}{\pgfqpoint{1.909265in}{2.113928in}}%
\pgfpathcurveto{\pgfqpoint{1.903441in}{2.119752in}}{\pgfqpoint{1.895541in}{2.123024in}}{\pgfqpoint{1.887305in}{2.123024in}}%
\pgfpathcurveto{\pgfqpoint{1.879069in}{2.123024in}}{\pgfqpoint{1.871169in}{2.119752in}}{\pgfqpoint{1.865345in}{2.113928in}}%
\pgfpathcurveto{\pgfqpoint{1.859521in}{2.108104in}}{\pgfqpoint{1.856248in}{2.100204in}}{\pgfqpoint{1.856248in}{2.091967in}}%
\pgfpathcurveto{\pgfqpoint{1.856248in}{2.083731in}}{\pgfqpoint{1.859521in}{2.075831in}}{\pgfqpoint{1.865345in}{2.070007in}}%
\pgfpathcurveto{\pgfqpoint{1.871169in}{2.064183in}}{\pgfqpoint{1.879069in}{2.060911in}}{\pgfqpoint{1.887305in}{2.060911in}}%
\pgfpathclose%
\pgfusepath{stroke,fill}%
\end{pgfscope}%
\begin{pgfscope}%
\pgfpathrectangle{\pgfqpoint{0.100000in}{0.212622in}}{\pgfqpoint{3.696000in}{3.696000in}}%
\pgfusepath{clip}%
\pgfsetbuttcap%
\pgfsetroundjoin%
\definecolor{currentfill}{rgb}{0.121569,0.466667,0.705882}%
\pgfsetfillcolor{currentfill}%
\pgfsetfillopacity{0.324338}%
\pgfsetlinewidth{1.003750pt}%
\definecolor{currentstroke}{rgb}{0.121569,0.466667,0.705882}%
\pgfsetstrokecolor{currentstroke}%
\pgfsetstrokeopacity{0.324338}%
\pgfsetdash{}{0pt}%
\pgfpathmoveto{\pgfqpoint{1.637459in}{2.098783in}}%
\pgfpathcurveto{\pgfqpoint{1.645696in}{2.098783in}}{\pgfqpoint{1.653596in}{2.102055in}}{\pgfqpoint{1.659420in}{2.107879in}}%
\pgfpathcurveto{\pgfqpoint{1.665244in}{2.113703in}}{\pgfqpoint{1.668516in}{2.121603in}}{\pgfqpoint{1.668516in}{2.129840in}}%
\pgfpathcurveto{\pgfqpoint{1.668516in}{2.138076in}}{\pgfqpoint{1.665244in}{2.145976in}}{\pgfqpoint{1.659420in}{2.151800in}}%
\pgfpathcurveto{\pgfqpoint{1.653596in}{2.157624in}}{\pgfqpoint{1.645696in}{2.160896in}}{\pgfqpoint{1.637459in}{2.160896in}}%
\pgfpathcurveto{\pgfqpoint{1.629223in}{2.160896in}}{\pgfqpoint{1.621323in}{2.157624in}}{\pgfqpoint{1.615499in}{2.151800in}}%
\pgfpathcurveto{\pgfqpoint{1.609675in}{2.145976in}}{\pgfqpoint{1.606403in}{2.138076in}}{\pgfqpoint{1.606403in}{2.129840in}}%
\pgfpathcurveto{\pgfqpoint{1.606403in}{2.121603in}}{\pgfqpoint{1.609675in}{2.113703in}}{\pgfqpoint{1.615499in}{2.107879in}}%
\pgfpathcurveto{\pgfqpoint{1.621323in}{2.102055in}}{\pgfqpoint{1.629223in}{2.098783in}}{\pgfqpoint{1.637459in}{2.098783in}}%
\pgfpathclose%
\pgfusepath{stroke,fill}%
\end{pgfscope}%
\begin{pgfscope}%
\pgfpathrectangle{\pgfqpoint{0.100000in}{0.212622in}}{\pgfqpoint{3.696000in}{3.696000in}}%
\pgfusepath{clip}%
\pgfsetbuttcap%
\pgfsetroundjoin%
\definecolor{currentfill}{rgb}{0.121569,0.466667,0.705882}%
\pgfsetfillcolor{currentfill}%
\pgfsetfillopacity{0.325685}%
\pgfsetlinewidth{1.003750pt}%
\definecolor{currentstroke}{rgb}{0.121569,0.466667,0.705882}%
\pgfsetstrokecolor{currentstroke}%
\pgfsetstrokeopacity{0.325685}%
\pgfsetdash{}{0pt}%
\pgfpathmoveto{\pgfqpoint{1.895105in}{2.060083in}}%
\pgfpathcurveto{\pgfqpoint{1.903341in}{2.060083in}}{\pgfqpoint{1.911241in}{2.063355in}}{\pgfqpoint{1.917065in}{2.069179in}}%
\pgfpathcurveto{\pgfqpoint{1.922889in}{2.075003in}}{\pgfqpoint{1.926161in}{2.082903in}}{\pgfqpoint{1.926161in}{2.091140in}}%
\pgfpathcurveto{\pgfqpoint{1.926161in}{2.099376in}}{\pgfqpoint{1.922889in}{2.107276in}}{\pgfqpoint{1.917065in}{2.113100in}}%
\pgfpathcurveto{\pgfqpoint{1.911241in}{2.118924in}}{\pgfqpoint{1.903341in}{2.122196in}}{\pgfqpoint{1.895105in}{2.122196in}}%
\pgfpathcurveto{\pgfqpoint{1.886868in}{2.122196in}}{\pgfqpoint{1.878968in}{2.118924in}}{\pgfqpoint{1.873144in}{2.113100in}}%
\pgfpathcurveto{\pgfqpoint{1.867320in}{2.107276in}}{\pgfqpoint{1.864048in}{2.099376in}}{\pgfqpoint{1.864048in}{2.091140in}}%
\pgfpathcurveto{\pgfqpoint{1.864048in}{2.082903in}}{\pgfqpoint{1.867320in}{2.075003in}}{\pgfqpoint{1.873144in}{2.069179in}}%
\pgfpathcurveto{\pgfqpoint{1.878968in}{2.063355in}}{\pgfqpoint{1.886868in}{2.060083in}}{\pgfqpoint{1.895105in}{2.060083in}}%
\pgfpathclose%
\pgfusepath{stroke,fill}%
\end{pgfscope}%
\begin{pgfscope}%
\pgfpathrectangle{\pgfqpoint{0.100000in}{0.212622in}}{\pgfqpoint{3.696000in}{3.696000in}}%
\pgfusepath{clip}%
\pgfsetbuttcap%
\pgfsetroundjoin%
\definecolor{currentfill}{rgb}{0.121569,0.466667,0.705882}%
\pgfsetfillcolor{currentfill}%
\pgfsetfillopacity{0.325839}%
\pgfsetlinewidth{1.003750pt}%
\definecolor{currentstroke}{rgb}{0.121569,0.466667,0.705882}%
\pgfsetstrokecolor{currentstroke}%
\pgfsetstrokeopacity{0.325839}%
\pgfsetdash{}{0pt}%
\pgfpathmoveto{\pgfqpoint{1.634731in}{2.099012in}}%
\pgfpathcurveto{\pgfqpoint{1.642967in}{2.099012in}}{\pgfqpoint{1.650868in}{2.102285in}}{\pgfqpoint{1.656691in}{2.108108in}}%
\pgfpathcurveto{\pgfqpoint{1.662515in}{2.113932in}}{\pgfqpoint{1.665788in}{2.121832in}}{\pgfqpoint{1.665788in}{2.130069in}}%
\pgfpathcurveto{\pgfqpoint{1.665788in}{2.138305in}}{\pgfqpoint{1.662515in}{2.146205in}}{\pgfqpoint{1.656691in}{2.152029in}}%
\pgfpathcurveto{\pgfqpoint{1.650868in}{2.157853in}}{\pgfqpoint{1.642967in}{2.161125in}}{\pgfqpoint{1.634731in}{2.161125in}}%
\pgfpathcurveto{\pgfqpoint{1.626495in}{2.161125in}}{\pgfqpoint{1.618595in}{2.157853in}}{\pgfqpoint{1.612771in}{2.152029in}}%
\pgfpathcurveto{\pgfqpoint{1.606947in}{2.146205in}}{\pgfqpoint{1.603675in}{2.138305in}}{\pgfqpoint{1.603675in}{2.130069in}}%
\pgfpathcurveto{\pgfqpoint{1.603675in}{2.121832in}}{\pgfqpoint{1.606947in}{2.113932in}}{\pgfqpoint{1.612771in}{2.108108in}}%
\pgfpathcurveto{\pgfqpoint{1.618595in}{2.102285in}}{\pgfqpoint{1.626495in}{2.099012in}}{\pgfqpoint{1.634731in}{2.099012in}}%
\pgfpathclose%
\pgfusepath{stroke,fill}%
\end{pgfscope}%
\begin{pgfscope}%
\pgfpathrectangle{\pgfqpoint{0.100000in}{0.212622in}}{\pgfqpoint{3.696000in}{3.696000in}}%
\pgfusepath{clip}%
\pgfsetbuttcap%
\pgfsetroundjoin%
\definecolor{currentfill}{rgb}{0.121569,0.466667,0.705882}%
\pgfsetfillcolor{currentfill}%
\pgfsetfillopacity{0.326464}%
\pgfsetlinewidth{1.003750pt}%
\definecolor{currentstroke}{rgb}{0.121569,0.466667,0.705882}%
\pgfsetstrokecolor{currentstroke}%
\pgfsetstrokeopacity{0.326464}%
\pgfsetdash{}{0pt}%
\pgfpathmoveto{\pgfqpoint{1.904674in}{2.057966in}}%
\pgfpathcurveto{\pgfqpoint{1.912911in}{2.057966in}}{\pgfqpoint{1.920811in}{2.061239in}}{\pgfqpoint{1.926635in}{2.067062in}}%
\pgfpathcurveto{\pgfqpoint{1.932459in}{2.072886in}}{\pgfqpoint{1.935731in}{2.080786in}}{\pgfqpoint{1.935731in}{2.089023in}}%
\pgfpathcurveto{\pgfqpoint{1.935731in}{2.097259in}}{\pgfqpoint{1.932459in}{2.105159in}}{\pgfqpoint{1.926635in}{2.110983in}}%
\pgfpathcurveto{\pgfqpoint{1.920811in}{2.116807in}}{\pgfqpoint{1.912911in}{2.120079in}}{\pgfqpoint{1.904674in}{2.120079in}}%
\pgfpathcurveto{\pgfqpoint{1.896438in}{2.120079in}}{\pgfqpoint{1.888538in}{2.116807in}}{\pgfqpoint{1.882714in}{2.110983in}}%
\pgfpathcurveto{\pgfqpoint{1.876890in}{2.105159in}}{\pgfqpoint{1.873618in}{2.097259in}}{\pgfqpoint{1.873618in}{2.089023in}}%
\pgfpathcurveto{\pgfqpoint{1.873618in}{2.080786in}}{\pgfqpoint{1.876890in}{2.072886in}}{\pgfqpoint{1.882714in}{2.067062in}}%
\pgfpathcurveto{\pgfqpoint{1.888538in}{2.061239in}}{\pgfqpoint{1.896438in}{2.057966in}}{\pgfqpoint{1.904674in}{2.057966in}}%
\pgfpathclose%
\pgfusepath{stroke,fill}%
\end{pgfscope}%
\begin{pgfscope}%
\pgfpathrectangle{\pgfqpoint{0.100000in}{0.212622in}}{\pgfqpoint{3.696000in}{3.696000in}}%
\pgfusepath{clip}%
\pgfsetbuttcap%
\pgfsetroundjoin%
\definecolor{currentfill}{rgb}{0.121569,0.466667,0.705882}%
\pgfsetfillcolor{currentfill}%
\pgfsetfillopacity{0.327254}%
\pgfsetlinewidth{1.003750pt}%
\definecolor{currentstroke}{rgb}{0.121569,0.466667,0.705882}%
\pgfsetstrokecolor{currentstroke}%
\pgfsetstrokeopacity{0.327254}%
\pgfsetdash{}{0pt}%
\pgfpathmoveto{\pgfqpoint{1.632329in}{2.099089in}}%
\pgfpathcurveto{\pgfqpoint{1.640565in}{2.099089in}}{\pgfqpoint{1.648466in}{2.102361in}}{\pgfqpoint{1.654289in}{2.108185in}}%
\pgfpathcurveto{\pgfqpoint{1.660113in}{2.114009in}}{\pgfqpoint{1.663386in}{2.121909in}}{\pgfqpoint{1.663386in}{2.130145in}}%
\pgfpathcurveto{\pgfqpoint{1.663386in}{2.138382in}}{\pgfqpoint{1.660113in}{2.146282in}}{\pgfqpoint{1.654289in}{2.152106in}}%
\pgfpathcurveto{\pgfqpoint{1.648466in}{2.157930in}}{\pgfqpoint{1.640565in}{2.161202in}}{\pgfqpoint{1.632329in}{2.161202in}}%
\pgfpathcurveto{\pgfqpoint{1.624093in}{2.161202in}}{\pgfqpoint{1.616193in}{2.157930in}}{\pgfqpoint{1.610369in}{2.152106in}}%
\pgfpathcurveto{\pgfqpoint{1.604545in}{2.146282in}}{\pgfqpoint{1.601273in}{2.138382in}}{\pgfqpoint{1.601273in}{2.130145in}}%
\pgfpathcurveto{\pgfqpoint{1.601273in}{2.121909in}}{\pgfqpoint{1.604545in}{2.114009in}}{\pgfqpoint{1.610369in}{2.108185in}}%
\pgfpathcurveto{\pgfqpoint{1.616193in}{2.102361in}}{\pgfqpoint{1.624093in}{2.099089in}}{\pgfqpoint{1.632329in}{2.099089in}}%
\pgfpathclose%
\pgfusepath{stroke,fill}%
\end{pgfscope}%
\begin{pgfscope}%
\pgfpathrectangle{\pgfqpoint{0.100000in}{0.212622in}}{\pgfqpoint{3.696000in}{3.696000in}}%
\pgfusepath{clip}%
\pgfsetbuttcap%
\pgfsetroundjoin%
\definecolor{currentfill}{rgb}{0.121569,0.466667,0.705882}%
\pgfsetfillcolor{currentfill}%
\pgfsetfillopacity{0.327901}%
\pgfsetlinewidth{1.003750pt}%
\definecolor{currentstroke}{rgb}{0.121569,0.466667,0.705882}%
\pgfsetstrokecolor{currentstroke}%
\pgfsetstrokeopacity{0.327901}%
\pgfsetdash{}{0pt}%
\pgfpathmoveto{\pgfqpoint{1.915292in}{2.055701in}}%
\pgfpathcurveto{\pgfqpoint{1.923528in}{2.055701in}}{\pgfqpoint{1.931428in}{2.058973in}}{\pgfqpoint{1.937252in}{2.064797in}}%
\pgfpathcurveto{\pgfqpoint{1.943076in}{2.070621in}}{\pgfqpoint{1.946348in}{2.078521in}}{\pgfqpoint{1.946348in}{2.086757in}}%
\pgfpathcurveto{\pgfqpoint{1.946348in}{2.094994in}}{\pgfqpoint{1.943076in}{2.102894in}}{\pgfqpoint{1.937252in}{2.108718in}}%
\pgfpathcurveto{\pgfqpoint{1.931428in}{2.114541in}}{\pgfqpoint{1.923528in}{2.117814in}}{\pgfqpoint{1.915292in}{2.117814in}}%
\pgfpathcurveto{\pgfqpoint{1.907055in}{2.117814in}}{\pgfqpoint{1.899155in}{2.114541in}}{\pgfqpoint{1.893331in}{2.108718in}}%
\pgfpathcurveto{\pgfqpoint{1.887507in}{2.102894in}}{\pgfqpoint{1.884235in}{2.094994in}}{\pgfqpoint{1.884235in}{2.086757in}}%
\pgfpathcurveto{\pgfqpoint{1.884235in}{2.078521in}}{\pgfqpoint{1.887507in}{2.070621in}}{\pgfqpoint{1.893331in}{2.064797in}}%
\pgfpathcurveto{\pgfqpoint{1.899155in}{2.058973in}}{\pgfqpoint{1.907055in}{2.055701in}}{\pgfqpoint{1.915292in}{2.055701in}}%
\pgfpathclose%
\pgfusepath{stroke,fill}%
\end{pgfscope}%
\begin{pgfscope}%
\pgfpathrectangle{\pgfqpoint{0.100000in}{0.212622in}}{\pgfqpoint{3.696000in}{3.696000in}}%
\pgfusepath{clip}%
\pgfsetbuttcap%
\pgfsetroundjoin%
\definecolor{currentfill}{rgb}{0.121569,0.466667,0.705882}%
\pgfsetfillcolor{currentfill}%
\pgfsetfillopacity{0.328134}%
\pgfsetlinewidth{1.003750pt}%
\definecolor{currentstroke}{rgb}{0.121569,0.466667,0.705882}%
\pgfsetstrokecolor{currentstroke}%
\pgfsetstrokeopacity{0.328134}%
\pgfsetdash{}{0pt}%
\pgfpathmoveto{\pgfqpoint{1.630049in}{2.099357in}}%
\pgfpathcurveto{\pgfqpoint{1.638286in}{2.099357in}}{\pgfqpoint{1.646186in}{2.102630in}}{\pgfqpoint{1.652010in}{2.108454in}}%
\pgfpathcurveto{\pgfqpoint{1.657834in}{2.114278in}}{\pgfqpoint{1.661106in}{2.122178in}}{\pgfqpoint{1.661106in}{2.130414in}}%
\pgfpathcurveto{\pgfqpoint{1.661106in}{2.138650in}}{\pgfqpoint{1.657834in}{2.146550in}}{\pgfqpoint{1.652010in}{2.152374in}}%
\pgfpathcurveto{\pgfqpoint{1.646186in}{2.158198in}}{\pgfqpoint{1.638286in}{2.161470in}}{\pgfqpoint{1.630049in}{2.161470in}}%
\pgfpathcurveto{\pgfqpoint{1.621813in}{2.161470in}}{\pgfqpoint{1.613913in}{2.158198in}}{\pgfqpoint{1.608089in}{2.152374in}}%
\pgfpathcurveto{\pgfqpoint{1.602265in}{2.146550in}}{\pgfqpoint{1.598993in}{2.138650in}}{\pgfqpoint{1.598993in}{2.130414in}}%
\pgfpathcurveto{\pgfqpoint{1.598993in}{2.122178in}}{\pgfqpoint{1.602265in}{2.114278in}}{\pgfqpoint{1.608089in}{2.108454in}}%
\pgfpathcurveto{\pgfqpoint{1.613913in}{2.102630in}}{\pgfqpoint{1.621813in}{2.099357in}}{\pgfqpoint{1.630049in}{2.099357in}}%
\pgfpathclose%
\pgfusepath{stroke,fill}%
\end{pgfscope}%
\begin{pgfscope}%
\pgfpathrectangle{\pgfqpoint{0.100000in}{0.212622in}}{\pgfqpoint{3.696000in}{3.696000in}}%
\pgfusepath{clip}%
\pgfsetbuttcap%
\pgfsetroundjoin%
\definecolor{currentfill}{rgb}{0.121569,0.466667,0.705882}%
\pgfsetfillcolor{currentfill}%
\pgfsetfillopacity{0.328586}%
\pgfsetlinewidth{1.003750pt}%
\definecolor{currentstroke}{rgb}{0.121569,0.466667,0.705882}%
\pgfsetstrokecolor{currentstroke}%
\pgfsetstrokeopacity{0.328586}%
\pgfsetdash{}{0pt}%
\pgfpathmoveto{\pgfqpoint{1.921308in}{2.054182in}}%
\pgfpathcurveto{\pgfqpoint{1.929544in}{2.054182in}}{\pgfqpoint{1.937444in}{2.057454in}}{\pgfqpoint{1.943268in}{2.063278in}}%
\pgfpathcurveto{\pgfqpoint{1.949092in}{2.069102in}}{\pgfqpoint{1.952364in}{2.077002in}}{\pgfqpoint{1.952364in}{2.085239in}}%
\pgfpathcurveto{\pgfqpoint{1.952364in}{2.093475in}}{\pgfqpoint{1.949092in}{2.101375in}}{\pgfqpoint{1.943268in}{2.107199in}}%
\pgfpathcurveto{\pgfqpoint{1.937444in}{2.113023in}}{\pgfqpoint{1.929544in}{2.116295in}}{\pgfqpoint{1.921308in}{2.116295in}}%
\pgfpathcurveto{\pgfqpoint{1.913072in}{2.116295in}}{\pgfqpoint{1.905171in}{2.113023in}}{\pgfqpoint{1.899348in}{2.107199in}}%
\pgfpathcurveto{\pgfqpoint{1.893524in}{2.101375in}}{\pgfqpoint{1.890251in}{2.093475in}}{\pgfqpoint{1.890251in}{2.085239in}}%
\pgfpathcurveto{\pgfqpoint{1.890251in}{2.077002in}}{\pgfqpoint{1.893524in}{2.069102in}}{\pgfqpoint{1.899348in}{2.063278in}}%
\pgfpathcurveto{\pgfqpoint{1.905171in}{2.057454in}}{\pgfqpoint{1.913072in}{2.054182in}}{\pgfqpoint{1.921308in}{2.054182in}}%
\pgfpathclose%
\pgfusepath{stroke,fill}%
\end{pgfscope}%
\begin{pgfscope}%
\pgfpathrectangle{\pgfqpoint{0.100000in}{0.212622in}}{\pgfqpoint{3.696000in}{3.696000in}}%
\pgfusepath{clip}%
\pgfsetbuttcap%
\pgfsetroundjoin%
\definecolor{currentfill}{rgb}{0.121569,0.466667,0.705882}%
\pgfsetfillcolor{currentfill}%
\pgfsetfillopacity{0.328970}%
\pgfsetlinewidth{1.003750pt}%
\definecolor{currentstroke}{rgb}{0.121569,0.466667,0.705882}%
\pgfsetstrokecolor{currentstroke}%
\pgfsetstrokeopacity{0.328970}%
\pgfsetdash{}{0pt}%
\pgfpathmoveto{\pgfqpoint{1.924635in}{2.053461in}}%
\pgfpathcurveto{\pgfqpoint{1.932871in}{2.053461in}}{\pgfqpoint{1.940771in}{2.056734in}}{\pgfqpoint{1.946595in}{2.062558in}}%
\pgfpathcurveto{\pgfqpoint{1.952419in}{2.068382in}}{\pgfqpoint{1.955691in}{2.076282in}}{\pgfqpoint{1.955691in}{2.084518in}}%
\pgfpathcurveto{\pgfqpoint{1.955691in}{2.092754in}}{\pgfqpoint{1.952419in}{2.100654in}}{\pgfqpoint{1.946595in}{2.106478in}}%
\pgfpathcurveto{\pgfqpoint{1.940771in}{2.112302in}}{\pgfqpoint{1.932871in}{2.115574in}}{\pgfqpoint{1.924635in}{2.115574in}}%
\pgfpathcurveto{\pgfqpoint{1.916398in}{2.115574in}}{\pgfqpoint{1.908498in}{2.112302in}}{\pgfqpoint{1.902674in}{2.106478in}}%
\pgfpathcurveto{\pgfqpoint{1.896851in}{2.100654in}}{\pgfqpoint{1.893578in}{2.092754in}}{\pgfqpoint{1.893578in}{2.084518in}}%
\pgfpathcurveto{\pgfqpoint{1.893578in}{2.076282in}}{\pgfqpoint{1.896851in}{2.068382in}}{\pgfqpoint{1.902674in}{2.062558in}}%
\pgfpathcurveto{\pgfqpoint{1.908498in}{2.056734in}}{\pgfqpoint{1.916398in}{2.053461in}}{\pgfqpoint{1.924635in}{2.053461in}}%
\pgfpathclose%
\pgfusepath{stroke,fill}%
\end{pgfscope}%
\begin{pgfscope}%
\pgfpathrectangle{\pgfqpoint{0.100000in}{0.212622in}}{\pgfqpoint{3.696000in}{3.696000in}}%
\pgfusepath{clip}%
\pgfsetbuttcap%
\pgfsetroundjoin%
\definecolor{currentfill}{rgb}{0.121569,0.466667,0.705882}%
\pgfsetfillcolor{currentfill}%
\pgfsetfillopacity{0.329667}%
\pgfsetlinewidth{1.003750pt}%
\definecolor{currentstroke}{rgb}{0.121569,0.466667,0.705882}%
\pgfsetstrokecolor{currentstroke}%
\pgfsetstrokeopacity{0.329667}%
\pgfsetdash{}{0pt}%
\pgfpathmoveto{\pgfqpoint{1.928552in}{2.052864in}}%
\pgfpathcurveto{\pgfqpoint{1.936789in}{2.052864in}}{\pgfqpoint{1.944689in}{2.056136in}}{\pgfqpoint{1.950513in}{2.061960in}}%
\pgfpathcurveto{\pgfqpoint{1.956337in}{2.067784in}}{\pgfqpoint{1.959609in}{2.075684in}}{\pgfqpoint{1.959609in}{2.083920in}}%
\pgfpathcurveto{\pgfqpoint{1.959609in}{2.092157in}}{\pgfqpoint{1.956337in}{2.100057in}}{\pgfqpoint{1.950513in}{2.105881in}}%
\pgfpathcurveto{\pgfqpoint{1.944689in}{2.111705in}}{\pgfqpoint{1.936789in}{2.114977in}}{\pgfqpoint{1.928552in}{2.114977in}}%
\pgfpathcurveto{\pgfqpoint{1.920316in}{2.114977in}}{\pgfqpoint{1.912416in}{2.111705in}}{\pgfqpoint{1.906592in}{2.105881in}}%
\pgfpathcurveto{\pgfqpoint{1.900768in}{2.100057in}}{\pgfqpoint{1.897496in}{2.092157in}}{\pgfqpoint{1.897496in}{2.083920in}}%
\pgfpathcurveto{\pgfqpoint{1.897496in}{2.075684in}}{\pgfqpoint{1.900768in}{2.067784in}}{\pgfqpoint{1.906592in}{2.061960in}}%
\pgfpathcurveto{\pgfqpoint{1.912416in}{2.056136in}}{\pgfqpoint{1.920316in}{2.052864in}}{\pgfqpoint{1.928552in}{2.052864in}}%
\pgfpathclose%
\pgfusepath{stroke,fill}%
\end{pgfscope}%
\begin{pgfscope}%
\pgfpathrectangle{\pgfqpoint{0.100000in}{0.212622in}}{\pgfqpoint{3.696000in}{3.696000in}}%
\pgfusepath{clip}%
\pgfsetbuttcap%
\pgfsetroundjoin%
\definecolor{currentfill}{rgb}{0.121569,0.466667,0.705882}%
\pgfsetfillcolor{currentfill}%
\pgfsetfillopacity{0.329886}%
\pgfsetlinewidth{1.003750pt}%
\definecolor{currentstroke}{rgb}{0.121569,0.466667,0.705882}%
\pgfsetstrokecolor{currentstroke}%
\pgfsetstrokeopacity{0.329886}%
\pgfsetdash{}{0pt}%
\pgfpathmoveto{\pgfqpoint{1.627503in}{2.099462in}}%
\pgfpathcurveto{\pgfqpoint{1.635740in}{2.099462in}}{\pgfqpoint{1.643640in}{2.102734in}}{\pgfqpoint{1.649464in}{2.108558in}}%
\pgfpathcurveto{\pgfqpoint{1.655288in}{2.114382in}}{\pgfqpoint{1.658560in}{2.122282in}}{\pgfqpoint{1.658560in}{2.130519in}}%
\pgfpathcurveto{\pgfqpoint{1.658560in}{2.138755in}}{\pgfqpoint{1.655288in}{2.146655in}}{\pgfqpoint{1.649464in}{2.152479in}}%
\pgfpathcurveto{\pgfqpoint{1.643640in}{2.158303in}}{\pgfqpoint{1.635740in}{2.161575in}}{\pgfqpoint{1.627503in}{2.161575in}}%
\pgfpathcurveto{\pgfqpoint{1.619267in}{2.161575in}}{\pgfqpoint{1.611367in}{2.158303in}}{\pgfqpoint{1.605543in}{2.152479in}}%
\pgfpathcurveto{\pgfqpoint{1.599719in}{2.146655in}}{\pgfqpoint{1.596447in}{2.138755in}}{\pgfqpoint{1.596447in}{2.130519in}}%
\pgfpathcurveto{\pgfqpoint{1.596447in}{2.122282in}}{\pgfqpoint{1.599719in}{2.114382in}}{\pgfqpoint{1.605543in}{2.108558in}}%
\pgfpathcurveto{\pgfqpoint{1.611367in}{2.102734in}}{\pgfqpoint{1.619267in}{2.099462in}}{\pgfqpoint{1.627503in}{2.099462in}}%
\pgfpathclose%
\pgfusepath{stroke,fill}%
\end{pgfscope}%
\begin{pgfscope}%
\pgfpathrectangle{\pgfqpoint{0.100000in}{0.212622in}}{\pgfqpoint{3.696000in}{3.696000in}}%
\pgfusepath{clip}%
\pgfsetbuttcap%
\pgfsetroundjoin%
\definecolor{currentfill}{rgb}{0.121569,0.466667,0.705882}%
\pgfsetfillcolor{currentfill}%
\pgfsetfillopacity{0.330588}%
\pgfsetlinewidth{1.003750pt}%
\definecolor{currentstroke}{rgb}{0.121569,0.466667,0.705882}%
\pgfsetstrokecolor{currentstroke}%
\pgfsetstrokeopacity{0.330588}%
\pgfsetdash{}{0pt}%
\pgfpathmoveto{\pgfqpoint{1.932760in}{2.052522in}}%
\pgfpathcurveto{\pgfqpoint{1.940996in}{2.052522in}}{\pgfqpoint{1.948896in}{2.055794in}}{\pgfqpoint{1.954720in}{2.061618in}}%
\pgfpathcurveto{\pgfqpoint{1.960544in}{2.067442in}}{\pgfqpoint{1.963816in}{2.075342in}}{\pgfqpoint{1.963816in}{2.083578in}}%
\pgfpathcurveto{\pgfqpoint{1.963816in}{2.091815in}}{\pgfqpoint{1.960544in}{2.099715in}}{\pgfqpoint{1.954720in}{2.105539in}}%
\pgfpathcurveto{\pgfqpoint{1.948896in}{2.111362in}}{\pgfqpoint{1.940996in}{2.114635in}}{\pgfqpoint{1.932760in}{2.114635in}}%
\pgfpathcurveto{\pgfqpoint{1.924523in}{2.114635in}}{\pgfqpoint{1.916623in}{2.111362in}}{\pgfqpoint{1.910799in}{2.105539in}}%
\pgfpathcurveto{\pgfqpoint{1.904975in}{2.099715in}}{\pgfqpoint{1.901703in}{2.091815in}}{\pgfqpoint{1.901703in}{2.083578in}}%
\pgfpathcurveto{\pgfqpoint{1.901703in}{2.075342in}}{\pgfqpoint{1.904975in}{2.067442in}}{\pgfqpoint{1.910799in}{2.061618in}}%
\pgfpathcurveto{\pgfqpoint{1.916623in}{2.055794in}}{\pgfqpoint{1.924523in}{2.052522in}}{\pgfqpoint{1.932760in}{2.052522in}}%
\pgfpathclose%
\pgfusepath{stroke,fill}%
\end{pgfscope}%
\begin{pgfscope}%
\pgfpathrectangle{\pgfqpoint{0.100000in}{0.212622in}}{\pgfqpoint{3.696000in}{3.696000in}}%
\pgfusepath{clip}%
\pgfsetbuttcap%
\pgfsetroundjoin%
\definecolor{currentfill}{rgb}{0.121569,0.466667,0.705882}%
\pgfsetfillcolor{currentfill}%
\pgfsetfillopacity{0.331081}%
\pgfsetlinewidth{1.003750pt}%
\definecolor{currentstroke}{rgb}{0.121569,0.466667,0.705882}%
\pgfsetstrokecolor{currentstroke}%
\pgfsetstrokeopacity{0.331081}%
\pgfsetdash{}{0pt}%
\pgfpathmoveto{\pgfqpoint{1.938515in}{2.051366in}}%
\pgfpathcurveto{\pgfqpoint{1.946751in}{2.051366in}}{\pgfqpoint{1.954651in}{2.054638in}}{\pgfqpoint{1.960475in}{2.060462in}}%
\pgfpathcurveto{\pgfqpoint{1.966299in}{2.066286in}}{\pgfqpoint{1.969572in}{2.074186in}}{\pgfqpoint{1.969572in}{2.082422in}}%
\pgfpathcurveto{\pgfqpoint{1.969572in}{2.090659in}}{\pgfqpoint{1.966299in}{2.098559in}}{\pgfqpoint{1.960475in}{2.104383in}}%
\pgfpathcurveto{\pgfqpoint{1.954651in}{2.110207in}}{\pgfqpoint{1.946751in}{2.113479in}}{\pgfqpoint{1.938515in}{2.113479in}}%
\pgfpathcurveto{\pgfqpoint{1.930279in}{2.113479in}}{\pgfqpoint{1.922379in}{2.110207in}}{\pgfqpoint{1.916555in}{2.104383in}}%
\pgfpathcurveto{\pgfqpoint{1.910731in}{2.098559in}}{\pgfqpoint{1.907459in}{2.090659in}}{\pgfqpoint{1.907459in}{2.082422in}}%
\pgfpathcurveto{\pgfqpoint{1.907459in}{2.074186in}}{\pgfqpoint{1.910731in}{2.066286in}}{\pgfqpoint{1.916555in}{2.060462in}}%
\pgfpathcurveto{\pgfqpoint{1.922379in}{2.054638in}}{\pgfqpoint{1.930279in}{2.051366in}}{\pgfqpoint{1.938515in}{2.051366in}}%
\pgfpathclose%
\pgfusepath{stroke,fill}%
\end{pgfscope}%
\begin{pgfscope}%
\pgfpathrectangle{\pgfqpoint{0.100000in}{0.212622in}}{\pgfqpoint{3.696000in}{3.696000in}}%
\pgfusepath{clip}%
\pgfsetbuttcap%
\pgfsetroundjoin%
\definecolor{currentfill}{rgb}{0.121569,0.466667,0.705882}%
\pgfsetfillcolor{currentfill}%
\pgfsetfillopacity{0.331221}%
\pgfsetlinewidth{1.003750pt}%
\definecolor{currentstroke}{rgb}{0.121569,0.466667,0.705882}%
\pgfsetstrokecolor{currentstroke}%
\pgfsetstrokeopacity{0.331221}%
\pgfsetdash{}{0pt}%
\pgfpathmoveto{\pgfqpoint{1.941847in}{2.050596in}}%
\pgfpathcurveto{\pgfqpoint{1.950084in}{2.050596in}}{\pgfqpoint{1.957984in}{2.053868in}}{\pgfqpoint{1.963808in}{2.059692in}}%
\pgfpathcurveto{\pgfqpoint{1.969632in}{2.065516in}}{\pgfqpoint{1.972904in}{2.073416in}}{\pgfqpoint{1.972904in}{2.081652in}}%
\pgfpathcurveto{\pgfqpoint{1.972904in}{2.089888in}}{\pgfqpoint{1.969632in}{2.097788in}}{\pgfqpoint{1.963808in}{2.103612in}}%
\pgfpathcurveto{\pgfqpoint{1.957984in}{2.109436in}}{\pgfqpoint{1.950084in}{2.112709in}}{\pgfqpoint{1.941847in}{2.112709in}}%
\pgfpathcurveto{\pgfqpoint{1.933611in}{2.112709in}}{\pgfqpoint{1.925711in}{2.109436in}}{\pgfqpoint{1.919887in}{2.103612in}}%
\pgfpathcurveto{\pgfqpoint{1.914063in}{2.097788in}}{\pgfqpoint{1.910791in}{2.089888in}}{\pgfqpoint{1.910791in}{2.081652in}}%
\pgfpathcurveto{\pgfqpoint{1.910791in}{2.073416in}}{\pgfqpoint{1.914063in}{2.065516in}}{\pgfqpoint{1.919887in}{2.059692in}}%
\pgfpathcurveto{\pgfqpoint{1.925711in}{2.053868in}}{\pgfqpoint{1.933611in}{2.050596in}}{\pgfqpoint{1.941847in}{2.050596in}}%
\pgfpathclose%
\pgfusepath{stroke,fill}%
\end{pgfscope}%
\begin{pgfscope}%
\pgfpathrectangle{\pgfqpoint{0.100000in}{0.212622in}}{\pgfqpoint{3.696000in}{3.696000in}}%
\pgfusepath{clip}%
\pgfsetbuttcap%
\pgfsetroundjoin%
\definecolor{currentfill}{rgb}{0.121569,0.466667,0.705882}%
\pgfsetfillcolor{currentfill}%
\pgfsetfillopacity{0.331444}%
\pgfsetlinewidth{1.003750pt}%
\definecolor{currentstroke}{rgb}{0.121569,0.466667,0.705882}%
\pgfsetstrokecolor{currentstroke}%
\pgfsetstrokeopacity{0.331444}%
\pgfsetdash{}{0pt}%
\pgfpathmoveto{\pgfqpoint{1.624249in}{2.099545in}}%
\pgfpathcurveto{\pgfqpoint{1.632486in}{2.099545in}}{\pgfqpoint{1.640386in}{2.102818in}}{\pgfqpoint{1.646210in}{2.108641in}}%
\pgfpathcurveto{\pgfqpoint{1.652034in}{2.114465in}}{\pgfqpoint{1.655306in}{2.122365in}}{\pgfqpoint{1.655306in}{2.130602in}}%
\pgfpathcurveto{\pgfqpoint{1.655306in}{2.138838in}}{\pgfqpoint{1.652034in}{2.146738in}}{\pgfqpoint{1.646210in}{2.152562in}}%
\pgfpathcurveto{\pgfqpoint{1.640386in}{2.158386in}}{\pgfqpoint{1.632486in}{2.161658in}}{\pgfqpoint{1.624249in}{2.161658in}}%
\pgfpathcurveto{\pgfqpoint{1.616013in}{2.161658in}}{\pgfqpoint{1.608113in}{2.158386in}}{\pgfqpoint{1.602289in}{2.152562in}}%
\pgfpathcurveto{\pgfqpoint{1.596465in}{2.146738in}}{\pgfqpoint{1.593193in}{2.138838in}}{\pgfqpoint{1.593193in}{2.130602in}}%
\pgfpathcurveto{\pgfqpoint{1.593193in}{2.122365in}}{\pgfqpoint{1.596465in}{2.114465in}}{\pgfqpoint{1.602289in}{2.108641in}}%
\pgfpathcurveto{\pgfqpoint{1.608113in}{2.102818in}}{\pgfqpoint{1.616013in}{2.099545in}}{\pgfqpoint{1.624249in}{2.099545in}}%
\pgfpathclose%
\pgfusepath{stroke,fill}%
\end{pgfscope}%
\begin{pgfscope}%
\pgfpathrectangle{\pgfqpoint{0.100000in}{0.212622in}}{\pgfqpoint{3.696000in}{3.696000in}}%
\pgfusepath{clip}%
\pgfsetbuttcap%
\pgfsetroundjoin%
\definecolor{currentfill}{rgb}{0.121569,0.466667,0.705882}%
\pgfsetfillcolor{currentfill}%
\pgfsetfillopacity{0.331761}%
\pgfsetlinewidth{1.003750pt}%
\definecolor{currentstroke}{rgb}{0.121569,0.466667,0.705882}%
\pgfsetstrokecolor{currentstroke}%
\pgfsetstrokeopacity{0.331761}%
\pgfsetdash{}{0pt}%
\pgfpathmoveto{\pgfqpoint{1.945277in}{2.050012in}}%
\pgfpathcurveto{\pgfqpoint{1.953513in}{2.050012in}}{\pgfqpoint{1.961413in}{2.053285in}}{\pgfqpoint{1.967237in}{2.059109in}}%
\pgfpathcurveto{\pgfqpoint{1.973061in}{2.064932in}}{\pgfqpoint{1.976333in}{2.072833in}}{\pgfqpoint{1.976333in}{2.081069in}}%
\pgfpathcurveto{\pgfqpoint{1.976333in}{2.089305in}}{\pgfqpoint{1.973061in}{2.097205in}}{\pgfqpoint{1.967237in}{2.103029in}}%
\pgfpathcurveto{\pgfqpoint{1.961413in}{2.108853in}}{\pgfqpoint{1.953513in}{2.112125in}}{\pgfqpoint{1.945277in}{2.112125in}}%
\pgfpathcurveto{\pgfqpoint{1.937040in}{2.112125in}}{\pgfqpoint{1.929140in}{2.108853in}}{\pgfqpoint{1.923317in}{2.103029in}}%
\pgfpathcurveto{\pgfqpoint{1.917493in}{2.097205in}}{\pgfqpoint{1.914220in}{2.089305in}}{\pgfqpoint{1.914220in}{2.081069in}}%
\pgfpathcurveto{\pgfqpoint{1.914220in}{2.072833in}}{\pgfqpoint{1.917493in}{2.064932in}}{\pgfqpoint{1.923317in}{2.059109in}}%
\pgfpathcurveto{\pgfqpoint{1.929140in}{2.053285in}}{\pgfqpoint{1.937040in}{2.050012in}}{\pgfqpoint{1.945277in}{2.050012in}}%
\pgfpathclose%
\pgfusepath{stroke,fill}%
\end{pgfscope}%
\begin{pgfscope}%
\pgfpathrectangle{\pgfqpoint{0.100000in}{0.212622in}}{\pgfqpoint{3.696000in}{3.696000in}}%
\pgfusepath{clip}%
\pgfsetbuttcap%
\pgfsetroundjoin%
\definecolor{currentfill}{rgb}{0.121569,0.466667,0.705882}%
\pgfsetfillcolor{currentfill}%
\pgfsetfillopacity{0.332147}%
\pgfsetlinewidth{1.003750pt}%
\definecolor{currentstroke}{rgb}{0.121569,0.466667,0.705882}%
\pgfsetstrokecolor{currentstroke}%
\pgfsetstrokeopacity{0.332147}%
\pgfsetdash{}{0pt}%
\pgfpathmoveto{\pgfqpoint{1.949444in}{2.049115in}}%
\pgfpathcurveto{\pgfqpoint{1.957680in}{2.049115in}}{\pgfqpoint{1.965580in}{2.052388in}}{\pgfqpoint{1.971404in}{2.058212in}}%
\pgfpathcurveto{\pgfqpoint{1.977228in}{2.064036in}}{\pgfqpoint{1.980500in}{2.071936in}}{\pgfqpoint{1.980500in}{2.080172in}}%
\pgfpathcurveto{\pgfqpoint{1.980500in}{2.088408in}}{\pgfqpoint{1.977228in}{2.096308in}}{\pgfqpoint{1.971404in}{2.102132in}}%
\pgfpathcurveto{\pgfqpoint{1.965580in}{2.107956in}}{\pgfqpoint{1.957680in}{2.111228in}}{\pgfqpoint{1.949444in}{2.111228in}}%
\pgfpathcurveto{\pgfqpoint{1.941208in}{2.111228in}}{\pgfqpoint{1.933308in}{2.107956in}}{\pgfqpoint{1.927484in}{2.102132in}}%
\pgfpathcurveto{\pgfqpoint{1.921660in}{2.096308in}}{\pgfqpoint{1.918387in}{2.088408in}}{\pgfqpoint{1.918387in}{2.080172in}}%
\pgfpathcurveto{\pgfqpoint{1.918387in}{2.071936in}}{\pgfqpoint{1.921660in}{2.064036in}}{\pgfqpoint{1.927484in}{2.058212in}}%
\pgfpathcurveto{\pgfqpoint{1.933308in}{2.052388in}}{\pgfqpoint{1.941208in}{2.049115in}}{\pgfqpoint{1.949444in}{2.049115in}}%
\pgfpathclose%
\pgfusepath{stroke,fill}%
\end{pgfscope}%
\begin{pgfscope}%
\pgfpathrectangle{\pgfqpoint{0.100000in}{0.212622in}}{\pgfqpoint{3.696000in}{3.696000in}}%
\pgfusepath{clip}%
\pgfsetbuttcap%
\pgfsetroundjoin%
\definecolor{currentfill}{rgb}{0.121569,0.466667,0.705882}%
\pgfsetfillcolor{currentfill}%
\pgfsetfillopacity{0.332431}%
\pgfsetlinewidth{1.003750pt}%
\definecolor{currentstroke}{rgb}{0.121569,0.466667,0.705882}%
\pgfsetstrokecolor{currentstroke}%
\pgfsetstrokeopacity{0.332431}%
\pgfsetdash{}{0pt}%
\pgfpathmoveto{\pgfqpoint{1.951579in}{2.048628in}}%
\pgfpathcurveto{\pgfqpoint{1.959816in}{2.048628in}}{\pgfqpoint{1.967716in}{2.051900in}}{\pgfqpoint{1.973540in}{2.057724in}}%
\pgfpathcurveto{\pgfqpoint{1.979364in}{2.063548in}}{\pgfqpoint{1.982636in}{2.071448in}}{\pgfqpoint{1.982636in}{2.079684in}}%
\pgfpathcurveto{\pgfqpoint{1.982636in}{2.087920in}}{\pgfqpoint{1.979364in}{2.095820in}}{\pgfqpoint{1.973540in}{2.101644in}}%
\pgfpathcurveto{\pgfqpoint{1.967716in}{2.107468in}}{\pgfqpoint{1.959816in}{2.110741in}}{\pgfqpoint{1.951579in}{2.110741in}}%
\pgfpathcurveto{\pgfqpoint{1.943343in}{2.110741in}}{\pgfqpoint{1.935443in}{2.107468in}}{\pgfqpoint{1.929619in}{2.101644in}}%
\pgfpathcurveto{\pgfqpoint{1.923795in}{2.095820in}}{\pgfqpoint{1.920523in}{2.087920in}}{\pgfqpoint{1.920523in}{2.079684in}}%
\pgfpathcurveto{\pgfqpoint{1.920523in}{2.071448in}}{\pgfqpoint{1.923795in}{2.063548in}}{\pgfqpoint{1.929619in}{2.057724in}}%
\pgfpathcurveto{\pgfqpoint{1.935443in}{2.051900in}}{\pgfqpoint{1.943343in}{2.048628in}}{\pgfqpoint{1.951579in}{2.048628in}}%
\pgfpathclose%
\pgfusepath{stroke,fill}%
\end{pgfscope}%
\begin{pgfscope}%
\pgfpathrectangle{\pgfqpoint{0.100000in}{0.212622in}}{\pgfqpoint{3.696000in}{3.696000in}}%
\pgfusepath{clip}%
\pgfsetbuttcap%
\pgfsetroundjoin%
\definecolor{currentfill}{rgb}{0.121569,0.466667,0.705882}%
\pgfsetfillcolor{currentfill}%
\pgfsetfillopacity{0.332609}%
\pgfsetlinewidth{1.003750pt}%
\definecolor{currentstroke}{rgb}{0.121569,0.466667,0.705882}%
\pgfsetstrokecolor{currentstroke}%
\pgfsetstrokeopacity{0.332609}%
\pgfsetdash{}{0pt}%
\pgfpathmoveto{\pgfqpoint{1.952722in}{2.048434in}}%
\pgfpathcurveto{\pgfqpoint{1.960959in}{2.048434in}}{\pgfqpoint{1.968859in}{2.051706in}}{\pgfqpoint{1.974683in}{2.057530in}}%
\pgfpathcurveto{\pgfqpoint{1.980506in}{2.063354in}}{\pgfqpoint{1.983779in}{2.071254in}}{\pgfqpoint{1.983779in}{2.079490in}}%
\pgfpathcurveto{\pgfqpoint{1.983779in}{2.087727in}}{\pgfqpoint{1.980506in}{2.095627in}}{\pgfqpoint{1.974683in}{2.101451in}}%
\pgfpathcurveto{\pgfqpoint{1.968859in}{2.107275in}}{\pgfqpoint{1.960959in}{2.110547in}}{\pgfqpoint{1.952722in}{2.110547in}}%
\pgfpathcurveto{\pgfqpoint{1.944486in}{2.110547in}}{\pgfqpoint{1.936586in}{2.107275in}}{\pgfqpoint{1.930762in}{2.101451in}}%
\pgfpathcurveto{\pgfqpoint{1.924938in}{2.095627in}}{\pgfqpoint{1.921666in}{2.087727in}}{\pgfqpoint{1.921666in}{2.079490in}}%
\pgfpathcurveto{\pgfqpoint{1.921666in}{2.071254in}}{\pgfqpoint{1.924938in}{2.063354in}}{\pgfqpoint{1.930762in}{2.057530in}}%
\pgfpathcurveto{\pgfqpoint{1.936586in}{2.051706in}}{\pgfqpoint{1.944486in}{2.048434in}}{\pgfqpoint{1.952722in}{2.048434in}}%
\pgfpathclose%
\pgfusepath{stroke,fill}%
\end{pgfscope}%
\begin{pgfscope}%
\pgfpathrectangle{\pgfqpoint{0.100000in}{0.212622in}}{\pgfqpoint{3.696000in}{3.696000in}}%
\pgfusepath{clip}%
\pgfsetbuttcap%
\pgfsetroundjoin%
\definecolor{currentfill}{rgb}{0.121569,0.466667,0.705882}%
\pgfsetfillcolor{currentfill}%
\pgfsetfillopacity{0.332754}%
\pgfsetlinewidth{1.003750pt}%
\definecolor{currentstroke}{rgb}{0.121569,0.466667,0.705882}%
\pgfsetstrokecolor{currentstroke}%
\pgfsetstrokeopacity{0.332754}%
\pgfsetdash{}{0pt}%
\pgfpathmoveto{\pgfqpoint{1.955712in}{2.047525in}}%
\pgfpathcurveto{\pgfqpoint{1.963949in}{2.047525in}}{\pgfqpoint{1.971849in}{2.050797in}}{\pgfqpoint{1.977673in}{2.056621in}}%
\pgfpathcurveto{\pgfqpoint{1.983497in}{2.062445in}}{\pgfqpoint{1.986769in}{2.070345in}}{\pgfqpoint{1.986769in}{2.078582in}}%
\pgfpathcurveto{\pgfqpoint{1.986769in}{2.086818in}}{\pgfqpoint{1.983497in}{2.094718in}}{\pgfqpoint{1.977673in}{2.100542in}}%
\pgfpathcurveto{\pgfqpoint{1.971849in}{2.106366in}}{\pgfqpoint{1.963949in}{2.109638in}}{\pgfqpoint{1.955712in}{2.109638in}}%
\pgfpathcurveto{\pgfqpoint{1.947476in}{2.109638in}}{\pgfqpoint{1.939576in}{2.106366in}}{\pgfqpoint{1.933752in}{2.100542in}}%
\pgfpathcurveto{\pgfqpoint{1.927928in}{2.094718in}}{\pgfqpoint{1.924656in}{2.086818in}}{\pgfqpoint{1.924656in}{2.078582in}}%
\pgfpathcurveto{\pgfqpoint{1.924656in}{2.070345in}}{\pgfqpoint{1.927928in}{2.062445in}}{\pgfqpoint{1.933752in}{2.056621in}}%
\pgfpathcurveto{\pgfqpoint{1.939576in}{2.050797in}}{\pgfqpoint{1.947476in}{2.047525in}}{\pgfqpoint{1.955712in}{2.047525in}}%
\pgfpathclose%
\pgfusepath{stroke,fill}%
\end{pgfscope}%
\begin{pgfscope}%
\pgfpathrectangle{\pgfqpoint{0.100000in}{0.212622in}}{\pgfqpoint{3.696000in}{3.696000in}}%
\pgfusepath{clip}%
\pgfsetbuttcap%
\pgfsetroundjoin%
\definecolor{currentfill}{rgb}{0.121569,0.466667,0.705882}%
\pgfsetfillcolor{currentfill}%
\pgfsetfillopacity{0.332779}%
\pgfsetlinewidth{1.003750pt}%
\definecolor{currentstroke}{rgb}{0.121569,0.466667,0.705882}%
\pgfsetstrokecolor{currentstroke}%
\pgfsetstrokeopacity{0.332779}%
\pgfsetdash{}{0pt}%
\pgfpathmoveto{\pgfqpoint{1.620931in}{2.099849in}}%
\pgfpathcurveto{\pgfqpoint{1.629167in}{2.099849in}}{\pgfqpoint{1.637067in}{2.103121in}}{\pgfqpoint{1.642891in}{2.108945in}}%
\pgfpathcurveto{\pgfqpoint{1.648715in}{2.114769in}}{\pgfqpoint{1.651987in}{2.122669in}}{\pgfqpoint{1.651987in}{2.130905in}}%
\pgfpathcurveto{\pgfqpoint{1.651987in}{2.139141in}}{\pgfqpoint{1.648715in}{2.147041in}}{\pgfqpoint{1.642891in}{2.152865in}}%
\pgfpathcurveto{\pgfqpoint{1.637067in}{2.158689in}}{\pgfqpoint{1.629167in}{2.161962in}}{\pgfqpoint{1.620931in}{2.161962in}}%
\pgfpathcurveto{\pgfqpoint{1.612694in}{2.161962in}}{\pgfqpoint{1.604794in}{2.158689in}}{\pgfqpoint{1.598970in}{2.152865in}}%
\pgfpathcurveto{\pgfqpoint{1.593146in}{2.147041in}}{\pgfqpoint{1.589874in}{2.139141in}}{\pgfqpoint{1.589874in}{2.130905in}}%
\pgfpathcurveto{\pgfqpoint{1.589874in}{2.122669in}}{\pgfqpoint{1.593146in}{2.114769in}}{\pgfqpoint{1.598970in}{2.108945in}}%
\pgfpathcurveto{\pgfqpoint{1.604794in}{2.103121in}}{\pgfqpoint{1.612694in}{2.099849in}}{\pgfqpoint{1.620931in}{2.099849in}}%
\pgfpathclose%
\pgfusepath{stroke,fill}%
\end{pgfscope}%
\begin{pgfscope}%
\pgfpathrectangle{\pgfqpoint{0.100000in}{0.212622in}}{\pgfqpoint{3.696000in}{3.696000in}}%
\pgfusepath{clip}%
\pgfsetbuttcap%
\pgfsetroundjoin%
\definecolor{currentfill}{rgb}{0.121569,0.466667,0.705882}%
\pgfsetfillcolor{currentfill}%
\pgfsetfillopacity{0.333359}%
\pgfsetlinewidth{1.003750pt}%
\definecolor{currentstroke}{rgb}{0.121569,0.466667,0.705882}%
\pgfsetstrokecolor{currentstroke}%
\pgfsetstrokeopacity{0.333359}%
\pgfsetdash{}{0pt}%
\pgfpathmoveto{\pgfqpoint{1.958493in}{2.047129in}}%
\pgfpathcurveto{\pgfqpoint{1.966729in}{2.047129in}}{\pgfqpoint{1.974629in}{2.050401in}}{\pgfqpoint{1.980453in}{2.056225in}}%
\pgfpathcurveto{\pgfqpoint{1.986277in}{2.062049in}}{\pgfqpoint{1.989549in}{2.069949in}}{\pgfqpoint{1.989549in}{2.078186in}}%
\pgfpathcurveto{\pgfqpoint{1.989549in}{2.086422in}}{\pgfqpoint{1.986277in}{2.094322in}}{\pgfqpoint{1.980453in}{2.100146in}}%
\pgfpathcurveto{\pgfqpoint{1.974629in}{2.105970in}}{\pgfqpoint{1.966729in}{2.109242in}}{\pgfqpoint{1.958493in}{2.109242in}}%
\pgfpathcurveto{\pgfqpoint{1.950256in}{2.109242in}}{\pgfqpoint{1.942356in}{2.105970in}}{\pgfqpoint{1.936532in}{2.100146in}}%
\pgfpathcurveto{\pgfqpoint{1.930709in}{2.094322in}}{\pgfqpoint{1.927436in}{2.086422in}}{\pgfqpoint{1.927436in}{2.078186in}}%
\pgfpathcurveto{\pgfqpoint{1.927436in}{2.069949in}}{\pgfqpoint{1.930709in}{2.062049in}}{\pgfqpoint{1.936532in}{2.056225in}}%
\pgfpathcurveto{\pgfqpoint{1.942356in}{2.050401in}}{\pgfqpoint{1.950256in}{2.047129in}}{\pgfqpoint{1.958493in}{2.047129in}}%
\pgfpathclose%
\pgfusepath{stroke,fill}%
\end{pgfscope}%
\begin{pgfscope}%
\pgfpathrectangle{\pgfqpoint{0.100000in}{0.212622in}}{\pgfqpoint{3.696000in}{3.696000in}}%
\pgfusepath{clip}%
\pgfsetbuttcap%
\pgfsetroundjoin%
\definecolor{currentfill}{rgb}{0.121569,0.466667,0.705882}%
\pgfsetfillcolor{currentfill}%
\pgfsetfillopacity{0.333953}%
\pgfsetlinewidth{1.003750pt}%
\definecolor{currentstroke}{rgb}{0.121569,0.466667,0.705882}%
\pgfsetstrokecolor{currentstroke}%
\pgfsetstrokeopacity{0.333953}%
\pgfsetdash{}{0pt}%
\pgfpathmoveto{\pgfqpoint{1.618743in}{2.099675in}}%
\pgfpathcurveto{\pgfqpoint{1.626979in}{2.099675in}}{\pgfqpoint{1.634879in}{2.102947in}}{\pgfqpoint{1.640703in}{2.108771in}}%
\pgfpathcurveto{\pgfqpoint{1.646527in}{2.114595in}}{\pgfqpoint{1.649799in}{2.122495in}}{\pgfqpoint{1.649799in}{2.130731in}}%
\pgfpathcurveto{\pgfqpoint{1.649799in}{2.138968in}}{\pgfqpoint{1.646527in}{2.146868in}}{\pgfqpoint{1.640703in}{2.152692in}}%
\pgfpathcurveto{\pgfqpoint{1.634879in}{2.158516in}}{\pgfqpoint{1.626979in}{2.161788in}}{\pgfqpoint{1.618743in}{2.161788in}}%
\pgfpathcurveto{\pgfqpoint{1.610507in}{2.161788in}}{\pgfqpoint{1.602606in}{2.158516in}}{\pgfqpoint{1.596783in}{2.152692in}}%
\pgfpathcurveto{\pgfqpoint{1.590959in}{2.146868in}}{\pgfqpoint{1.587686in}{2.138968in}}{\pgfqpoint{1.587686in}{2.130731in}}%
\pgfpathcurveto{\pgfqpoint{1.587686in}{2.122495in}}{\pgfqpoint{1.590959in}{2.114595in}}{\pgfqpoint{1.596783in}{2.108771in}}%
\pgfpathcurveto{\pgfqpoint{1.602606in}{2.102947in}}{\pgfqpoint{1.610507in}{2.099675in}}{\pgfqpoint{1.618743in}{2.099675in}}%
\pgfpathclose%
\pgfusepath{stroke,fill}%
\end{pgfscope}%
\begin{pgfscope}%
\pgfpathrectangle{\pgfqpoint{0.100000in}{0.212622in}}{\pgfqpoint{3.696000in}{3.696000in}}%
\pgfusepath{clip}%
\pgfsetbuttcap%
\pgfsetroundjoin%
\definecolor{currentfill}{rgb}{0.121569,0.466667,0.705882}%
\pgfsetfillcolor{currentfill}%
\pgfsetfillopacity{0.334239}%
\pgfsetlinewidth{1.003750pt}%
\definecolor{currentstroke}{rgb}{0.121569,0.466667,0.705882}%
\pgfsetstrokecolor{currentstroke}%
\pgfsetstrokeopacity{0.334239}%
\pgfsetdash{}{0pt}%
\pgfpathmoveto{\pgfqpoint{1.962050in}{2.046784in}}%
\pgfpathcurveto{\pgfqpoint{1.970287in}{2.046784in}}{\pgfqpoint{1.978187in}{2.050056in}}{\pgfqpoint{1.984010in}{2.055880in}}%
\pgfpathcurveto{\pgfqpoint{1.989834in}{2.061704in}}{\pgfqpoint{1.993107in}{2.069604in}}{\pgfqpoint{1.993107in}{2.077841in}}%
\pgfpathcurveto{\pgfqpoint{1.993107in}{2.086077in}}{\pgfqpoint{1.989834in}{2.093977in}}{\pgfqpoint{1.984010in}{2.099801in}}%
\pgfpathcurveto{\pgfqpoint{1.978187in}{2.105625in}}{\pgfqpoint{1.970287in}{2.108897in}}{\pgfqpoint{1.962050in}{2.108897in}}%
\pgfpathcurveto{\pgfqpoint{1.953814in}{2.108897in}}{\pgfqpoint{1.945914in}{2.105625in}}{\pgfqpoint{1.940090in}{2.099801in}}%
\pgfpathcurveto{\pgfqpoint{1.934266in}{2.093977in}}{\pgfqpoint{1.930994in}{2.086077in}}{\pgfqpoint{1.930994in}{2.077841in}}%
\pgfpathcurveto{\pgfqpoint{1.930994in}{2.069604in}}{\pgfqpoint{1.934266in}{2.061704in}}{\pgfqpoint{1.940090in}{2.055880in}}%
\pgfpathcurveto{\pgfqpoint{1.945914in}{2.050056in}}{\pgfqpoint{1.953814in}{2.046784in}}{\pgfqpoint{1.962050in}{2.046784in}}%
\pgfpathclose%
\pgfusepath{stroke,fill}%
\end{pgfscope}%
\begin{pgfscope}%
\pgfpathrectangle{\pgfqpoint{0.100000in}{0.212622in}}{\pgfqpoint{3.696000in}{3.696000in}}%
\pgfusepath{clip}%
\pgfsetbuttcap%
\pgfsetroundjoin%
\definecolor{currentfill}{rgb}{0.121569,0.466667,0.705882}%
\pgfsetfillcolor{currentfill}%
\pgfsetfillopacity{0.335375}%
\pgfsetlinewidth{1.003750pt}%
\definecolor{currentstroke}{rgb}{0.121569,0.466667,0.705882}%
\pgfsetstrokecolor{currentstroke}%
\pgfsetstrokeopacity{0.335375}%
\pgfsetdash{}{0pt}%
\pgfpathmoveto{\pgfqpoint{1.968682in}{2.045535in}}%
\pgfpathcurveto{\pgfqpoint{1.976918in}{2.045535in}}{\pgfqpoint{1.984818in}{2.048807in}}{\pgfqpoint{1.990642in}{2.054631in}}%
\pgfpathcurveto{\pgfqpoint{1.996466in}{2.060455in}}{\pgfqpoint{1.999738in}{2.068355in}}{\pgfqpoint{1.999738in}{2.076592in}}%
\pgfpathcurveto{\pgfqpoint{1.999738in}{2.084828in}}{\pgfqpoint{1.996466in}{2.092728in}}{\pgfqpoint{1.990642in}{2.098552in}}%
\pgfpathcurveto{\pgfqpoint{1.984818in}{2.104376in}}{\pgfqpoint{1.976918in}{2.107648in}}{\pgfqpoint{1.968682in}{2.107648in}}%
\pgfpathcurveto{\pgfqpoint{1.960445in}{2.107648in}}{\pgfqpoint{1.952545in}{2.104376in}}{\pgfqpoint{1.946721in}{2.098552in}}%
\pgfpathcurveto{\pgfqpoint{1.940897in}{2.092728in}}{\pgfqpoint{1.937625in}{2.084828in}}{\pgfqpoint{1.937625in}{2.076592in}}%
\pgfpathcurveto{\pgfqpoint{1.937625in}{2.068355in}}{\pgfqpoint{1.940897in}{2.060455in}}{\pgfqpoint{1.946721in}{2.054631in}}%
\pgfpathcurveto{\pgfqpoint{1.952545in}{2.048807in}}{\pgfqpoint{1.960445in}{2.045535in}}{\pgfqpoint{1.968682in}{2.045535in}}%
\pgfpathclose%
\pgfusepath{stroke,fill}%
\end{pgfscope}%
\begin{pgfscope}%
\pgfpathrectangle{\pgfqpoint{0.100000in}{0.212622in}}{\pgfqpoint{3.696000in}{3.696000in}}%
\pgfusepath{clip}%
\pgfsetbuttcap%
\pgfsetroundjoin%
\definecolor{currentfill}{rgb}{0.121569,0.466667,0.705882}%
\pgfsetfillcolor{currentfill}%
\pgfsetfillopacity{0.336101}%
\pgfsetlinewidth{1.003750pt}%
\definecolor{currentstroke}{rgb}{0.121569,0.466667,0.705882}%
\pgfsetstrokecolor{currentstroke}%
\pgfsetstrokeopacity{0.336101}%
\pgfsetdash{}{0pt}%
\pgfpathmoveto{\pgfqpoint{1.972105in}{2.045107in}}%
\pgfpathcurveto{\pgfqpoint{1.980342in}{2.045107in}}{\pgfqpoint{1.988242in}{2.048379in}}{\pgfqpoint{1.994066in}{2.054203in}}%
\pgfpathcurveto{\pgfqpoint{1.999890in}{2.060027in}}{\pgfqpoint{2.003162in}{2.067927in}}{\pgfqpoint{2.003162in}{2.076163in}}%
\pgfpathcurveto{\pgfqpoint{2.003162in}{2.084400in}}{\pgfqpoint{1.999890in}{2.092300in}}{\pgfqpoint{1.994066in}{2.098124in}}%
\pgfpathcurveto{\pgfqpoint{1.988242in}{2.103947in}}{\pgfqpoint{1.980342in}{2.107220in}}{\pgfqpoint{1.972105in}{2.107220in}}%
\pgfpathcurveto{\pgfqpoint{1.963869in}{2.107220in}}{\pgfqpoint{1.955969in}{2.103947in}}{\pgfqpoint{1.950145in}{2.098124in}}%
\pgfpathcurveto{\pgfqpoint{1.944321in}{2.092300in}}{\pgfqpoint{1.941049in}{2.084400in}}{\pgfqpoint{1.941049in}{2.076163in}}%
\pgfpathcurveto{\pgfqpoint{1.941049in}{2.067927in}}{\pgfqpoint{1.944321in}{2.060027in}}{\pgfqpoint{1.950145in}{2.054203in}}%
\pgfpathcurveto{\pgfqpoint{1.955969in}{2.048379in}}{\pgfqpoint{1.963869in}{2.045107in}}{\pgfqpoint{1.972105in}{2.045107in}}%
\pgfpathclose%
\pgfusepath{stroke,fill}%
\end{pgfscope}%
\begin{pgfscope}%
\pgfpathrectangle{\pgfqpoint{0.100000in}{0.212622in}}{\pgfqpoint{3.696000in}{3.696000in}}%
\pgfusepath{clip}%
\pgfsetbuttcap%
\pgfsetroundjoin%
\definecolor{currentfill}{rgb}{0.121569,0.466667,0.705882}%
\pgfsetfillcolor{currentfill}%
\pgfsetfillopacity{0.336106}%
\pgfsetlinewidth{1.003750pt}%
\definecolor{currentstroke}{rgb}{0.121569,0.466667,0.705882}%
\pgfsetstrokecolor{currentstroke}%
\pgfsetstrokeopacity{0.336106}%
\pgfsetdash{}{0pt}%
\pgfpathmoveto{\pgfqpoint{1.614287in}{2.099929in}}%
\pgfpathcurveto{\pgfqpoint{1.622523in}{2.099929in}}{\pgfqpoint{1.630423in}{2.103201in}}{\pgfqpoint{1.636247in}{2.109025in}}%
\pgfpathcurveto{\pgfqpoint{1.642071in}{2.114849in}}{\pgfqpoint{1.645343in}{2.122749in}}{\pgfqpoint{1.645343in}{2.130985in}}%
\pgfpathcurveto{\pgfqpoint{1.645343in}{2.139221in}}{\pgfqpoint{1.642071in}{2.147122in}}{\pgfqpoint{1.636247in}{2.152945in}}%
\pgfpathcurveto{\pgfqpoint{1.630423in}{2.158769in}}{\pgfqpoint{1.622523in}{2.162042in}}{\pgfqpoint{1.614287in}{2.162042in}}%
\pgfpathcurveto{\pgfqpoint{1.606050in}{2.162042in}}{\pgfqpoint{1.598150in}{2.158769in}}{\pgfqpoint{1.592326in}{2.152945in}}%
\pgfpathcurveto{\pgfqpoint{1.586502in}{2.147122in}}{\pgfqpoint{1.583230in}{2.139221in}}{\pgfqpoint{1.583230in}{2.130985in}}%
\pgfpathcurveto{\pgfqpoint{1.583230in}{2.122749in}}{\pgfqpoint{1.586502in}{2.114849in}}{\pgfqpoint{1.592326in}{2.109025in}}%
\pgfpathcurveto{\pgfqpoint{1.598150in}{2.103201in}}{\pgfqpoint{1.606050in}{2.099929in}}{\pgfqpoint{1.614287in}{2.099929in}}%
\pgfpathclose%
\pgfusepath{stroke,fill}%
\end{pgfscope}%
\begin{pgfscope}%
\pgfpathrectangle{\pgfqpoint{0.100000in}{0.212622in}}{\pgfqpoint{3.696000in}{3.696000in}}%
\pgfusepath{clip}%
\pgfsetbuttcap%
\pgfsetroundjoin%
\definecolor{currentfill}{rgb}{0.121569,0.466667,0.705882}%
\pgfsetfillcolor{currentfill}%
\pgfsetfillopacity{0.336407}%
\pgfsetlinewidth{1.003750pt}%
\definecolor{currentstroke}{rgb}{0.121569,0.466667,0.705882}%
\pgfsetstrokecolor{currentstroke}%
\pgfsetstrokeopacity{0.336407}%
\pgfsetdash{}{0pt}%
\pgfpathmoveto{\pgfqpoint{1.974208in}{2.044697in}}%
\pgfpathcurveto{\pgfqpoint{1.982444in}{2.044697in}}{\pgfqpoint{1.990344in}{2.047970in}}{\pgfqpoint{1.996168in}{2.053794in}}%
\pgfpathcurveto{\pgfqpoint{2.001992in}{2.059618in}}{\pgfqpoint{2.005264in}{2.067518in}}{\pgfqpoint{2.005264in}{2.075754in}}%
\pgfpathcurveto{\pgfqpoint{2.005264in}{2.083990in}}{\pgfqpoint{2.001992in}{2.091890in}}{\pgfqpoint{1.996168in}{2.097714in}}%
\pgfpathcurveto{\pgfqpoint{1.990344in}{2.103538in}}{\pgfqpoint{1.982444in}{2.106810in}}{\pgfqpoint{1.974208in}{2.106810in}}%
\pgfpathcurveto{\pgfqpoint{1.965972in}{2.106810in}}{\pgfqpoint{1.958072in}{2.103538in}}{\pgfqpoint{1.952248in}{2.097714in}}%
\pgfpathcurveto{\pgfqpoint{1.946424in}{2.091890in}}{\pgfqpoint{1.943151in}{2.083990in}}{\pgfqpoint{1.943151in}{2.075754in}}%
\pgfpathcurveto{\pgfqpoint{1.943151in}{2.067518in}}{\pgfqpoint{1.946424in}{2.059618in}}{\pgfqpoint{1.952248in}{2.053794in}}%
\pgfpathcurveto{\pgfqpoint{1.958072in}{2.047970in}}{\pgfqpoint{1.965972in}{2.044697in}}{\pgfqpoint{1.974208in}{2.044697in}}%
\pgfpathclose%
\pgfusepath{stroke,fill}%
\end{pgfscope}%
\begin{pgfscope}%
\pgfpathrectangle{\pgfqpoint{0.100000in}{0.212622in}}{\pgfqpoint{3.696000in}{3.696000in}}%
\pgfusepath{clip}%
\pgfsetbuttcap%
\pgfsetroundjoin%
\definecolor{currentfill}{rgb}{0.121569,0.466667,0.705882}%
\pgfsetfillcolor{currentfill}%
\pgfsetfillopacity{0.336846}%
\pgfsetlinewidth{1.003750pt}%
\definecolor{currentstroke}{rgb}{0.121569,0.466667,0.705882}%
\pgfsetstrokecolor{currentstroke}%
\pgfsetstrokeopacity{0.336846}%
\pgfsetdash{}{0pt}%
\pgfpathmoveto{\pgfqpoint{1.976616in}{2.044226in}}%
\pgfpathcurveto{\pgfqpoint{1.984852in}{2.044226in}}{\pgfqpoint{1.992753in}{2.047498in}}{\pgfqpoint{1.998576in}{2.053322in}}%
\pgfpathcurveto{\pgfqpoint{2.004400in}{2.059146in}}{\pgfqpoint{2.007673in}{2.067046in}}{\pgfqpoint{2.007673in}{2.075283in}}%
\pgfpathcurveto{\pgfqpoint{2.007673in}{2.083519in}}{\pgfqpoint{2.004400in}{2.091419in}}{\pgfqpoint{1.998576in}{2.097243in}}%
\pgfpathcurveto{\pgfqpoint{1.992753in}{2.103067in}}{\pgfqpoint{1.984852in}{2.106339in}}{\pgfqpoint{1.976616in}{2.106339in}}%
\pgfpathcurveto{\pgfqpoint{1.968380in}{2.106339in}}{\pgfqpoint{1.960480in}{2.103067in}}{\pgfqpoint{1.954656in}{2.097243in}}%
\pgfpathcurveto{\pgfqpoint{1.948832in}{2.091419in}}{\pgfqpoint{1.945560in}{2.083519in}}{\pgfqpoint{1.945560in}{2.075283in}}%
\pgfpathcurveto{\pgfqpoint{1.945560in}{2.067046in}}{\pgfqpoint{1.948832in}{2.059146in}}{\pgfqpoint{1.954656in}{2.053322in}}%
\pgfpathcurveto{\pgfqpoint{1.960480in}{2.047498in}}{\pgfqpoint{1.968380in}{2.044226in}}{\pgfqpoint{1.976616in}{2.044226in}}%
\pgfpathclose%
\pgfusepath{stroke,fill}%
\end{pgfscope}%
\begin{pgfscope}%
\pgfpathrectangle{\pgfqpoint{0.100000in}{0.212622in}}{\pgfqpoint{3.696000in}{3.696000in}}%
\pgfusepath{clip}%
\pgfsetbuttcap%
\pgfsetroundjoin%
\definecolor{currentfill}{rgb}{0.121569,0.466667,0.705882}%
\pgfsetfillcolor{currentfill}%
\pgfsetfillopacity{0.337297}%
\pgfsetlinewidth{1.003750pt}%
\definecolor{currentstroke}{rgb}{0.121569,0.466667,0.705882}%
\pgfsetstrokecolor{currentstroke}%
\pgfsetstrokeopacity{0.337297}%
\pgfsetdash{}{0pt}%
\pgfpathmoveto{\pgfqpoint{1.979814in}{2.043708in}}%
\pgfpathcurveto{\pgfqpoint{1.988051in}{2.043708in}}{\pgfqpoint{1.995951in}{2.046980in}}{\pgfqpoint{2.001775in}{2.052804in}}%
\pgfpathcurveto{\pgfqpoint{2.007599in}{2.058628in}}{\pgfqpoint{2.010871in}{2.066528in}}{\pgfqpoint{2.010871in}{2.074765in}}%
\pgfpathcurveto{\pgfqpoint{2.010871in}{2.083001in}}{\pgfqpoint{2.007599in}{2.090901in}}{\pgfqpoint{2.001775in}{2.096725in}}%
\pgfpathcurveto{\pgfqpoint{1.995951in}{2.102549in}}{\pgfqpoint{1.988051in}{2.105821in}}{\pgfqpoint{1.979814in}{2.105821in}}%
\pgfpathcurveto{\pgfqpoint{1.971578in}{2.105821in}}{\pgfqpoint{1.963678in}{2.102549in}}{\pgfqpoint{1.957854in}{2.096725in}}%
\pgfpathcurveto{\pgfqpoint{1.952030in}{2.090901in}}{\pgfqpoint{1.948758in}{2.083001in}}{\pgfqpoint{1.948758in}{2.074765in}}%
\pgfpathcurveto{\pgfqpoint{1.948758in}{2.066528in}}{\pgfqpoint{1.952030in}{2.058628in}}{\pgfqpoint{1.957854in}{2.052804in}}%
\pgfpathcurveto{\pgfqpoint{1.963678in}{2.046980in}}{\pgfqpoint{1.971578in}{2.043708in}}{\pgfqpoint{1.979814in}{2.043708in}}%
\pgfpathclose%
\pgfusepath{stroke,fill}%
\end{pgfscope}%
\begin{pgfscope}%
\pgfpathrectangle{\pgfqpoint{0.100000in}{0.212622in}}{\pgfqpoint{3.696000in}{3.696000in}}%
\pgfusepath{clip}%
\pgfsetbuttcap%
\pgfsetroundjoin%
\definecolor{currentfill}{rgb}{0.121569,0.466667,0.705882}%
\pgfsetfillcolor{currentfill}%
\pgfsetfillopacity{0.337373}%
\pgfsetlinewidth{1.003750pt}%
\definecolor{currentstroke}{rgb}{0.121569,0.466667,0.705882}%
\pgfsetstrokecolor{currentstroke}%
\pgfsetstrokeopacity{0.337373}%
\pgfsetdash{}{0pt}%
\pgfpathmoveto{\pgfqpoint{1.981820in}{2.043141in}}%
\pgfpathcurveto{\pgfqpoint{1.990057in}{2.043141in}}{\pgfqpoint{1.997957in}{2.046413in}}{\pgfqpoint{2.003781in}{2.052237in}}%
\pgfpathcurveto{\pgfqpoint{2.009605in}{2.058061in}}{\pgfqpoint{2.012877in}{2.065961in}}{\pgfqpoint{2.012877in}{2.074198in}}%
\pgfpathcurveto{\pgfqpoint{2.012877in}{2.082434in}}{\pgfqpoint{2.009605in}{2.090334in}}{\pgfqpoint{2.003781in}{2.096158in}}%
\pgfpathcurveto{\pgfqpoint{1.997957in}{2.101982in}}{\pgfqpoint{1.990057in}{2.105254in}}{\pgfqpoint{1.981820in}{2.105254in}}%
\pgfpathcurveto{\pgfqpoint{1.973584in}{2.105254in}}{\pgfqpoint{1.965684in}{2.101982in}}{\pgfqpoint{1.959860in}{2.096158in}}%
\pgfpathcurveto{\pgfqpoint{1.954036in}{2.090334in}}{\pgfqpoint{1.950764in}{2.082434in}}{\pgfqpoint{1.950764in}{2.074198in}}%
\pgfpathcurveto{\pgfqpoint{1.950764in}{2.065961in}}{\pgfqpoint{1.954036in}{2.058061in}}{\pgfqpoint{1.959860in}{2.052237in}}%
\pgfpathcurveto{\pgfqpoint{1.965684in}{2.046413in}}{\pgfqpoint{1.973584in}{2.043141in}}{\pgfqpoint{1.981820in}{2.043141in}}%
\pgfpathclose%
\pgfusepath{stroke,fill}%
\end{pgfscope}%
\begin{pgfscope}%
\pgfpathrectangle{\pgfqpoint{0.100000in}{0.212622in}}{\pgfqpoint{3.696000in}{3.696000in}}%
\pgfusepath{clip}%
\pgfsetbuttcap%
\pgfsetroundjoin%
\definecolor{currentfill}{rgb}{0.121569,0.466667,0.705882}%
\pgfsetfillcolor{currentfill}%
\pgfsetfillopacity{0.337721}%
\pgfsetlinewidth{1.003750pt}%
\definecolor{currentstroke}{rgb}{0.121569,0.466667,0.705882}%
\pgfsetstrokecolor{currentstroke}%
\pgfsetstrokeopacity{0.337721}%
\pgfsetdash{}{0pt}%
\pgfpathmoveto{\pgfqpoint{1.984289in}{2.042663in}}%
\pgfpathcurveto{\pgfqpoint{1.992526in}{2.042663in}}{\pgfqpoint{2.000426in}{2.045935in}}{\pgfqpoint{2.006250in}{2.051759in}}%
\pgfpathcurveto{\pgfqpoint{2.012074in}{2.057583in}}{\pgfqpoint{2.015346in}{2.065483in}}{\pgfqpoint{2.015346in}{2.073719in}}%
\pgfpathcurveto{\pgfqpoint{2.015346in}{2.081956in}}{\pgfqpoint{2.012074in}{2.089856in}}{\pgfqpoint{2.006250in}{2.095680in}}%
\pgfpathcurveto{\pgfqpoint{2.000426in}{2.101504in}}{\pgfqpoint{1.992526in}{2.104776in}}{\pgfqpoint{1.984289in}{2.104776in}}%
\pgfpathcurveto{\pgfqpoint{1.976053in}{2.104776in}}{\pgfqpoint{1.968153in}{2.101504in}}{\pgfqpoint{1.962329in}{2.095680in}}%
\pgfpathcurveto{\pgfqpoint{1.956505in}{2.089856in}}{\pgfqpoint{1.953233in}{2.081956in}}{\pgfqpoint{1.953233in}{2.073719in}}%
\pgfpathcurveto{\pgfqpoint{1.953233in}{2.065483in}}{\pgfqpoint{1.956505in}{2.057583in}}{\pgfqpoint{1.962329in}{2.051759in}}%
\pgfpathcurveto{\pgfqpoint{1.968153in}{2.045935in}}{\pgfqpoint{1.976053in}{2.042663in}}{\pgfqpoint{1.984289in}{2.042663in}}%
\pgfpathclose%
\pgfusepath{stroke,fill}%
\end{pgfscope}%
\begin{pgfscope}%
\pgfpathrectangle{\pgfqpoint{0.100000in}{0.212622in}}{\pgfqpoint{3.696000in}{3.696000in}}%
\pgfusepath{clip}%
\pgfsetbuttcap%
\pgfsetroundjoin%
\definecolor{currentfill}{rgb}{0.121569,0.466667,0.705882}%
\pgfsetfillcolor{currentfill}%
\pgfsetfillopacity{0.338088}%
\pgfsetlinewidth{1.003750pt}%
\definecolor{currentstroke}{rgb}{0.121569,0.466667,0.705882}%
\pgfsetstrokecolor{currentstroke}%
\pgfsetstrokeopacity{0.338088}%
\pgfsetdash{}{0pt}%
\pgfpathmoveto{\pgfqpoint{1.609121in}{2.100502in}}%
\pgfpathcurveto{\pgfqpoint{1.617357in}{2.100502in}}{\pgfqpoint{1.625257in}{2.103774in}}{\pgfqpoint{1.631081in}{2.109598in}}%
\pgfpathcurveto{\pgfqpoint{1.636905in}{2.115422in}}{\pgfqpoint{1.640177in}{2.123322in}}{\pgfqpoint{1.640177in}{2.131559in}}%
\pgfpathcurveto{\pgfqpoint{1.640177in}{2.139795in}}{\pgfqpoint{1.636905in}{2.147695in}}{\pgfqpoint{1.631081in}{2.153519in}}%
\pgfpathcurveto{\pgfqpoint{1.625257in}{2.159343in}}{\pgfqpoint{1.617357in}{2.162615in}}{\pgfqpoint{1.609121in}{2.162615in}}%
\pgfpathcurveto{\pgfqpoint{1.600884in}{2.162615in}}{\pgfqpoint{1.592984in}{2.159343in}}{\pgfqpoint{1.587160in}{2.153519in}}%
\pgfpathcurveto{\pgfqpoint{1.581336in}{2.147695in}}{\pgfqpoint{1.578064in}{2.139795in}}{\pgfqpoint{1.578064in}{2.131559in}}%
\pgfpathcurveto{\pgfqpoint{1.578064in}{2.123322in}}{\pgfqpoint{1.581336in}{2.115422in}}{\pgfqpoint{1.587160in}{2.109598in}}%
\pgfpathcurveto{\pgfqpoint{1.592984in}{2.103774in}}{\pgfqpoint{1.600884in}{2.100502in}}{\pgfqpoint{1.609121in}{2.100502in}}%
\pgfpathclose%
\pgfusepath{stroke,fill}%
\end{pgfscope}%
\begin{pgfscope}%
\pgfpathrectangle{\pgfqpoint{0.100000in}{0.212622in}}{\pgfqpoint{3.696000in}{3.696000in}}%
\pgfusepath{clip}%
\pgfsetbuttcap%
\pgfsetroundjoin%
\definecolor{currentfill}{rgb}{0.121569,0.466667,0.705882}%
\pgfsetfillcolor{currentfill}%
\pgfsetfillopacity{0.338100}%
\pgfsetlinewidth{1.003750pt}%
\definecolor{currentstroke}{rgb}{0.121569,0.466667,0.705882}%
\pgfsetstrokecolor{currentstroke}%
\pgfsetstrokeopacity{0.338100}%
\pgfsetdash{}{0pt}%
\pgfpathmoveto{\pgfqpoint{1.987371in}{2.042076in}}%
\pgfpathcurveto{\pgfqpoint{1.995608in}{2.042076in}}{\pgfqpoint{2.003508in}{2.045348in}}{\pgfqpoint{2.009332in}{2.051172in}}%
\pgfpathcurveto{\pgfqpoint{2.015156in}{2.056996in}}{\pgfqpoint{2.018428in}{2.064896in}}{\pgfqpoint{2.018428in}{2.073132in}}%
\pgfpathcurveto{\pgfqpoint{2.018428in}{2.081369in}}{\pgfqpoint{2.015156in}{2.089269in}}{\pgfqpoint{2.009332in}{2.095093in}}%
\pgfpathcurveto{\pgfqpoint{2.003508in}{2.100917in}}{\pgfqpoint{1.995608in}{2.104189in}}{\pgfqpoint{1.987371in}{2.104189in}}%
\pgfpathcurveto{\pgfqpoint{1.979135in}{2.104189in}}{\pgfqpoint{1.971235in}{2.100917in}}{\pgfqpoint{1.965411in}{2.095093in}}%
\pgfpathcurveto{\pgfqpoint{1.959587in}{2.089269in}}{\pgfqpoint{1.956315in}{2.081369in}}{\pgfqpoint{1.956315in}{2.073132in}}%
\pgfpathcurveto{\pgfqpoint{1.956315in}{2.064896in}}{\pgfqpoint{1.959587in}{2.056996in}}{\pgfqpoint{1.965411in}{2.051172in}}%
\pgfpathcurveto{\pgfqpoint{1.971235in}{2.045348in}}{\pgfqpoint{1.979135in}{2.042076in}}{\pgfqpoint{1.987371in}{2.042076in}}%
\pgfpathclose%
\pgfusepath{stroke,fill}%
\end{pgfscope}%
\begin{pgfscope}%
\pgfpathrectangle{\pgfqpoint{0.100000in}{0.212622in}}{\pgfqpoint{3.696000in}{3.696000in}}%
\pgfusepath{clip}%
\pgfsetbuttcap%
\pgfsetroundjoin%
\definecolor{currentfill}{rgb}{0.121569,0.466667,0.705882}%
\pgfsetfillcolor{currentfill}%
\pgfsetfillopacity{0.338875}%
\pgfsetlinewidth{1.003750pt}%
\definecolor{currentstroke}{rgb}{0.121569,0.466667,0.705882}%
\pgfsetstrokecolor{currentstroke}%
\pgfsetstrokeopacity{0.338875}%
\pgfsetdash{}{0pt}%
\pgfpathmoveto{\pgfqpoint{1.991880in}{2.041062in}}%
\pgfpathcurveto{\pgfqpoint{2.000116in}{2.041062in}}{\pgfqpoint{2.008016in}{2.044334in}}{\pgfqpoint{2.013840in}{2.050158in}}%
\pgfpathcurveto{\pgfqpoint{2.019664in}{2.055982in}}{\pgfqpoint{2.022937in}{2.063882in}}{\pgfqpoint{2.022937in}{2.072119in}}%
\pgfpathcurveto{\pgfqpoint{2.022937in}{2.080355in}}{\pgfqpoint{2.019664in}{2.088255in}}{\pgfqpoint{2.013840in}{2.094079in}}%
\pgfpathcurveto{\pgfqpoint{2.008016in}{2.099903in}}{\pgfqpoint{2.000116in}{2.103175in}}{\pgfqpoint{1.991880in}{2.103175in}}%
\pgfpathcurveto{\pgfqpoint{1.983644in}{2.103175in}}{\pgfqpoint{1.975744in}{2.099903in}}{\pgfqpoint{1.969920in}{2.094079in}}%
\pgfpathcurveto{\pgfqpoint{1.964096in}{2.088255in}}{\pgfqpoint{1.960824in}{2.080355in}}{\pgfqpoint{1.960824in}{2.072119in}}%
\pgfpathcurveto{\pgfqpoint{1.960824in}{2.063882in}}{\pgfqpoint{1.964096in}{2.055982in}}{\pgfqpoint{1.969920in}{2.050158in}}%
\pgfpathcurveto{\pgfqpoint{1.975744in}{2.044334in}}{\pgfqpoint{1.983644in}{2.041062in}}{\pgfqpoint{1.991880in}{2.041062in}}%
\pgfpathclose%
\pgfusepath{stroke,fill}%
\end{pgfscope}%
\begin{pgfscope}%
\pgfpathrectangle{\pgfqpoint{0.100000in}{0.212622in}}{\pgfqpoint{3.696000in}{3.696000in}}%
\pgfusepath{clip}%
\pgfsetbuttcap%
\pgfsetroundjoin%
\definecolor{currentfill}{rgb}{0.121569,0.466667,0.705882}%
\pgfsetfillcolor{currentfill}%
\pgfsetfillopacity{0.339867}%
\pgfsetlinewidth{1.003750pt}%
\definecolor{currentstroke}{rgb}{0.121569,0.466667,0.705882}%
\pgfsetstrokecolor{currentstroke}%
\pgfsetstrokeopacity{0.339867}%
\pgfsetdash{}{0pt}%
\pgfpathmoveto{\pgfqpoint{1.996486in}{2.040294in}}%
\pgfpathcurveto{\pgfqpoint{2.004723in}{2.040294in}}{\pgfqpoint{2.012623in}{2.043567in}}{\pgfqpoint{2.018447in}{2.049391in}}%
\pgfpathcurveto{\pgfqpoint{2.024270in}{2.055215in}}{\pgfqpoint{2.027543in}{2.063115in}}{\pgfqpoint{2.027543in}{2.071351in}}%
\pgfpathcurveto{\pgfqpoint{2.027543in}{2.079587in}}{\pgfqpoint{2.024270in}{2.087487in}}{\pgfqpoint{2.018447in}{2.093311in}}%
\pgfpathcurveto{\pgfqpoint{2.012623in}{2.099135in}}{\pgfqpoint{2.004723in}{2.102407in}}{\pgfqpoint{1.996486in}{2.102407in}}%
\pgfpathcurveto{\pgfqpoint{1.988250in}{2.102407in}}{\pgfqpoint{1.980350in}{2.099135in}}{\pgfqpoint{1.974526in}{2.093311in}}%
\pgfpathcurveto{\pgfqpoint{1.968702in}{2.087487in}}{\pgfqpoint{1.965430in}{2.079587in}}{\pgfqpoint{1.965430in}{2.071351in}}%
\pgfpathcurveto{\pgfqpoint{1.965430in}{2.063115in}}{\pgfqpoint{1.968702in}{2.055215in}}{\pgfqpoint{1.974526in}{2.049391in}}%
\pgfpathcurveto{\pgfqpoint{1.980350in}{2.043567in}}{\pgfqpoint{1.988250in}{2.040294in}}{\pgfqpoint{1.996486in}{2.040294in}}%
\pgfpathclose%
\pgfusepath{stroke,fill}%
\end{pgfscope}%
\begin{pgfscope}%
\pgfpathrectangle{\pgfqpoint{0.100000in}{0.212622in}}{\pgfqpoint{3.696000in}{3.696000in}}%
\pgfusepath{clip}%
\pgfsetbuttcap%
\pgfsetroundjoin%
\definecolor{currentfill}{rgb}{0.121569,0.466667,0.705882}%
\pgfsetfillcolor{currentfill}%
\pgfsetfillopacity{0.340033}%
\pgfsetlinewidth{1.003750pt}%
\definecolor{currentstroke}{rgb}{0.121569,0.466667,0.705882}%
\pgfsetstrokecolor{currentstroke}%
\pgfsetstrokeopacity{0.340033}%
\pgfsetdash{}{0pt}%
\pgfpathmoveto{\pgfqpoint{1.607866in}{2.100513in}}%
\pgfpathcurveto{\pgfqpoint{1.616102in}{2.100513in}}{\pgfqpoint{1.624002in}{2.103786in}}{\pgfqpoint{1.629826in}{2.109610in}}%
\pgfpathcurveto{\pgfqpoint{1.635650in}{2.115433in}}{\pgfqpoint{1.638922in}{2.123333in}}{\pgfqpoint{1.638922in}{2.131570in}}%
\pgfpathcurveto{\pgfqpoint{1.638922in}{2.139806in}}{\pgfqpoint{1.635650in}{2.147706in}}{\pgfqpoint{1.629826in}{2.153530in}}%
\pgfpathcurveto{\pgfqpoint{1.624002in}{2.159354in}}{\pgfqpoint{1.616102in}{2.162626in}}{\pgfqpoint{1.607866in}{2.162626in}}%
\pgfpathcurveto{\pgfqpoint{1.599630in}{2.162626in}}{\pgfqpoint{1.591729in}{2.159354in}}{\pgfqpoint{1.585906in}{2.153530in}}%
\pgfpathcurveto{\pgfqpoint{1.580082in}{2.147706in}}{\pgfqpoint{1.576809in}{2.139806in}}{\pgfqpoint{1.576809in}{2.131570in}}%
\pgfpathcurveto{\pgfqpoint{1.576809in}{2.123333in}}{\pgfqpoint{1.580082in}{2.115433in}}{\pgfqpoint{1.585906in}{2.109610in}}%
\pgfpathcurveto{\pgfqpoint{1.591729in}{2.103786in}}{\pgfqpoint{1.599630in}{2.100513in}}{\pgfqpoint{1.607866in}{2.100513in}}%
\pgfpathclose%
\pgfusepath{stroke,fill}%
\end{pgfscope}%
\begin{pgfscope}%
\pgfpathrectangle{\pgfqpoint{0.100000in}{0.212622in}}{\pgfqpoint{3.696000in}{3.696000in}}%
\pgfusepath{clip}%
\pgfsetbuttcap%
\pgfsetroundjoin%
\definecolor{currentfill}{rgb}{0.121569,0.466667,0.705882}%
\pgfsetfillcolor{currentfill}%
\pgfsetfillopacity{0.340742}%
\pgfsetlinewidth{1.003750pt}%
\definecolor{currentstroke}{rgb}{0.121569,0.466667,0.705882}%
\pgfsetstrokecolor{currentstroke}%
\pgfsetstrokeopacity{0.340742}%
\pgfsetdash{}{0pt}%
\pgfpathmoveto{\pgfqpoint{2.002080in}{2.039306in}}%
\pgfpathcurveto{\pgfqpoint{2.010317in}{2.039306in}}{\pgfqpoint{2.018217in}{2.042578in}}{\pgfqpoint{2.024041in}{2.048402in}}%
\pgfpathcurveto{\pgfqpoint{2.029865in}{2.054226in}}{\pgfqpoint{2.033137in}{2.062126in}}{\pgfqpoint{2.033137in}{2.070362in}}%
\pgfpathcurveto{\pgfqpoint{2.033137in}{2.078599in}}{\pgfqpoint{2.029865in}{2.086499in}}{\pgfqpoint{2.024041in}{2.092322in}}%
\pgfpathcurveto{\pgfqpoint{2.018217in}{2.098146in}}{\pgfqpoint{2.010317in}{2.101419in}}{\pgfqpoint{2.002080in}{2.101419in}}%
\pgfpathcurveto{\pgfqpoint{1.993844in}{2.101419in}}{\pgfqpoint{1.985944in}{2.098146in}}{\pgfqpoint{1.980120in}{2.092322in}}%
\pgfpathcurveto{\pgfqpoint{1.974296in}{2.086499in}}{\pgfqpoint{1.971024in}{2.078599in}}{\pgfqpoint{1.971024in}{2.070362in}}%
\pgfpathcurveto{\pgfqpoint{1.971024in}{2.062126in}}{\pgfqpoint{1.974296in}{2.054226in}}{\pgfqpoint{1.980120in}{2.048402in}}%
\pgfpathcurveto{\pgfqpoint{1.985944in}{2.042578in}}{\pgfqpoint{1.993844in}{2.039306in}}{\pgfqpoint{2.002080in}{2.039306in}}%
\pgfpathclose%
\pgfusepath{stroke,fill}%
\end{pgfscope}%
\begin{pgfscope}%
\pgfpathrectangle{\pgfqpoint{0.100000in}{0.212622in}}{\pgfqpoint{3.696000in}{3.696000in}}%
\pgfusepath{clip}%
\pgfsetbuttcap%
\pgfsetroundjoin%
\definecolor{currentfill}{rgb}{0.121569,0.466667,0.705882}%
\pgfsetfillcolor{currentfill}%
\pgfsetfillopacity{0.341749}%
\pgfsetlinewidth{1.003750pt}%
\definecolor{currentstroke}{rgb}{0.121569,0.466667,0.705882}%
\pgfsetstrokecolor{currentstroke}%
\pgfsetstrokeopacity{0.341749}%
\pgfsetdash{}{0pt}%
\pgfpathmoveto{\pgfqpoint{1.604615in}{2.100670in}}%
\pgfpathcurveto{\pgfqpoint{1.612851in}{2.100670in}}{\pgfqpoint{1.620751in}{2.103942in}}{\pgfqpoint{1.626575in}{2.109766in}}%
\pgfpathcurveto{\pgfqpoint{1.632399in}{2.115590in}}{\pgfqpoint{1.635672in}{2.123490in}}{\pgfqpoint{1.635672in}{2.131727in}}%
\pgfpathcurveto{\pgfqpoint{1.635672in}{2.139963in}}{\pgfqpoint{1.632399in}{2.147863in}}{\pgfqpoint{1.626575in}{2.153687in}}%
\pgfpathcurveto{\pgfqpoint{1.620751in}{2.159511in}}{\pgfqpoint{1.612851in}{2.162783in}}{\pgfqpoint{1.604615in}{2.162783in}}%
\pgfpathcurveto{\pgfqpoint{1.596379in}{2.162783in}}{\pgfqpoint{1.588479in}{2.159511in}}{\pgfqpoint{1.582655in}{2.153687in}}%
\pgfpathcurveto{\pgfqpoint{1.576831in}{2.147863in}}{\pgfqpoint{1.573559in}{2.139963in}}{\pgfqpoint{1.573559in}{2.131727in}}%
\pgfpathcurveto{\pgfqpoint{1.573559in}{2.123490in}}{\pgfqpoint{1.576831in}{2.115590in}}{\pgfqpoint{1.582655in}{2.109766in}}%
\pgfpathcurveto{\pgfqpoint{1.588479in}{2.103942in}}{\pgfqpoint{1.596379in}{2.100670in}}{\pgfqpoint{1.604615in}{2.100670in}}%
\pgfpathclose%
\pgfusepath{stroke,fill}%
\end{pgfscope}%
\begin{pgfscope}%
\pgfpathrectangle{\pgfqpoint{0.100000in}{0.212622in}}{\pgfqpoint{3.696000in}{3.696000in}}%
\pgfusepath{clip}%
\pgfsetbuttcap%
\pgfsetroundjoin%
\definecolor{currentfill}{rgb}{0.121569,0.466667,0.705882}%
\pgfsetfillcolor{currentfill}%
\pgfsetfillopacity{0.342169}%
\pgfsetlinewidth{1.003750pt}%
\definecolor{currentstroke}{rgb}{0.121569,0.466667,0.705882}%
\pgfsetstrokecolor{currentstroke}%
\pgfsetstrokeopacity{0.342169}%
\pgfsetdash{}{0pt}%
\pgfpathmoveto{\pgfqpoint{2.007632in}{2.038443in}}%
\pgfpathcurveto{\pgfqpoint{2.015868in}{2.038443in}}{\pgfqpoint{2.023768in}{2.041715in}}{\pgfqpoint{2.029592in}{2.047539in}}%
\pgfpathcurveto{\pgfqpoint{2.035416in}{2.053363in}}{\pgfqpoint{2.038688in}{2.061263in}}{\pgfqpoint{2.038688in}{2.069499in}}%
\pgfpathcurveto{\pgfqpoint{2.038688in}{2.077736in}}{\pgfqpoint{2.035416in}{2.085636in}}{\pgfqpoint{2.029592in}{2.091460in}}%
\pgfpathcurveto{\pgfqpoint{2.023768in}{2.097284in}}{\pgfqpoint{2.015868in}{2.100556in}}{\pgfqpoint{2.007632in}{2.100556in}}%
\pgfpathcurveto{\pgfqpoint{1.999395in}{2.100556in}}{\pgfqpoint{1.991495in}{2.097284in}}{\pgfqpoint{1.985671in}{2.091460in}}%
\pgfpathcurveto{\pgfqpoint{1.979847in}{2.085636in}}{\pgfqpoint{1.976575in}{2.077736in}}{\pgfqpoint{1.976575in}{2.069499in}}%
\pgfpathcurveto{\pgfqpoint{1.976575in}{2.061263in}}{\pgfqpoint{1.979847in}{2.053363in}}{\pgfqpoint{1.985671in}{2.047539in}}%
\pgfpathcurveto{\pgfqpoint{1.991495in}{2.041715in}}{\pgfqpoint{1.999395in}{2.038443in}}{\pgfqpoint{2.007632in}{2.038443in}}%
\pgfpathclose%
\pgfusepath{stroke,fill}%
\end{pgfscope}%
\begin{pgfscope}%
\pgfpathrectangle{\pgfqpoint{0.100000in}{0.212622in}}{\pgfqpoint{3.696000in}{3.696000in}}%
\pgfusepath{clip}%
\pgfsetbuttcap%
\pgfsetroundjoin%
\definecolor{currentfill}{rgb}{0.121569,0.466667,0.705882}%
\pgfsetfillcolor{currentfill}%
\pgfsetfillopacity{0.343266}%
\pgfsetlinewidth{1.003750pt}%
\definecolor{currentstroke}{rgb}{0.121569,0.466667,0.705882}%
\pgfsetstrokecolor{currentstroke}%
\pgfsetstrokeopacity{0.343266}%
\pgfsetdash{}{0pt}%
\pgfpathmoveto{\pgfqpoint{1.600565in}{2.101022in}}%
\pgfpathcurveto{\pgfqpoint{1.608802in}{2.101022in}}{\pgfqpoint{1.616702in}{2.104294in}}{\pgfqpoint{1.622526in}{2.110118in}}%
\pgfpathcurveto{\pgfqpoint{1.628350in}{2.115942in}}{\pgfqpoint{1.631622in}{2.123842in}}{\pgfqpoint{1.631622in}{2.132078in}}%
\pgfpathcurveto{\pgfqpoint{1.631622in}{2.140314in}}{\pgfqpoint{1.628350in}{2.148214in}}{\pgfqpoint{1.622526in}{2.154038in}}%
\pgfpathcurveto{\pgfqpoint{1.616702in}{2.159862in}}{\pgfqpoint{1.608802in}{2.163135in}}{\pgfqpoint{1.600565in}{2.163135in}}%
\pgfpathcurveto{\pgfqpoint{1.592329in}{2.163135in}}{\pgfqpoint{1.584429in}{2.159862in}}{\pgfqpoint{1.578605in}{2.154038in}}%
\pgfpathcurveto{\pgfqpoint{1.572781in}{2.148214in}}{\pgfqpoint{1.569509in}{2.140314in}}{\pgfqpoint{1.569509in}{2.132078in}}%
\pgfpathcurveto{\pgfqpoint{1.569509in}{2.123842in}}{\pgfqpoint{1.572781in}{2.115942in}}{\pgfqpoint{1.578605in}{2.110118in}}%
\pgfpathcurveto{\pgfqpoint{1.584429in}{2.104294in}}{\pgfqpoint{1.592329in}{2.101022in}}{\pgfqpoint{1.600565in}{2.101022in}}%
\pgfpathclose%
\pgfusepath{stroke,fill}%
\end{pgfscope}%
\begin{pgfscope}%
\pgfpathrectangle{\pgfqpoint{0.100000in}{0.212622in}}{\pgfqpoint{3.696000in}{3.696000in}}%
\pgfusepath{clip}%
\pgfsetbuttcap%
\pgfsetroundjoin%
\definecolor{currentfill}{rgb}{0.121569,0.466667,0.705882}%
\pgfsetfillcolor{currentfill}%
\pgfsetfillopacity{0.343960}%
\pgfsetlinewidth{1.003750pt}%
\definecolor{currentstroke}{rgb}{0.121569,0.466667,0.705882}%
\pgfsetstrokecolor{currentstroke}%
\pgfsetstrokeopacity{0.343960}%
\pgfsetdash{}{0pt}%
\pgfpathmoveto{\pgfqpoint{2.013261in}{2.038014in}}%
\pgfpathcurveto{\pgfqpoint{2.021498in}{2.038014in}}{\pgfqpoint{2.029398in}{2.041287in}}{\pgfqpoint{2.035222in}{2.047111in}}%
\pgfpathcurveto{\pgfqpoint{2.041045in}{2.052935in}}{\pgfqpoint{2.044318in}{2.060835in}}{\pgfqpoint{2.044318in}{2.069071in}}%
\pgfpathcurveto{\pgfqpoint{2.044318in}{2.077307in}}{\pgfqpoint{2.041045in}{2.085207in}}{\pgfqpoint{2.035222in}{2.091031in}}%
\pgfpathcurveto{\pgfqpoint{2.029398in}{2.096855in}}{\pgfqpoint{2.021498in}{2.100127in}}{\pgfqpoint{2.013261in}{2.100127in}}%
\pgfpathcurveto{\pgfqpoint{2.005025in}{2.100127in}}{\pgfqpoint{1.997125in}{2.096855in}}{\pgfqpoint{1.991301in}{2.091031in}}%
\pgfpathcurveto{\pgfqpoint{1.985477in}{2.085207in}}{\pgfqpoint{1.982205in}{2.077307in}}{\pgfqpoint{1.982205in}{2.069071in}}%
\pgfpathcurveto{\pgfqpoint{1.982205in}{2.060835in}}{\pgfqpoint{1.985477in}{2.052935in}}{\pgfqpoint{1.991301in}{2.047111in}}%
\pgfpathcurveto{\pgfqpoint{1.997125in}{2.041287in}}{\pgfqpoint{2.005025in}{2.038014in}}{\pgfqpoint{2.013261in}{2.038014in}}%
\pgfpathclose%
\pgfusepath{stroke,fill}%
\end{pgfscope}%
\begin{pgfscope}%
\pgfpathrectangle{\pgfqpoint{0.100000in}{0.212622in}}{\pgfqpoint{3.696000in}{3.696000in}}%
\pgfusepath{clip}%
\pgfsetbuttcap%
\pgfsetroundjoin%
\definecolor{currentfill}{rgb}{0.121569,0.466667,0.705882}%
\pgfsetfillcolor{currentfill}%
\pgfsetfillopacity{0.344410}%
\pgfsetlinewidth{1.003750pt}%
\definecolor{currentstroke}{rgb}{0.121569,0.466667,0.705882}%
\pgfsetstrokecolor{currentstroke}%
\pgfsetstrokeopacity{0.344410}%
\pgfsetdash{}{0pt}%
\pgfpathmoveto{\pgfqpoint{1.598352in}{2.100982in}}%
\pgfpathcurveto{\pgfqpoint{1.606589in}{2.100982in}}{\pgfqpoint{1.614489in}{2.104254in}}{\pgfqpoint{1.620313in}{2.110078in}}%
\pgfpathcurveto{\pgfqpoint{1.626137in}{2.115902in}}{\pgfqpoint{1.629409in}{2.123802in}}{\pgfqpoint{1.629409in}{2.132038in}}%
\pgfpathcurveto{\pgfqpoint{1.629409in}{2.140275in}}{\pgfqpoint{1.626137in}{2.148175in}}{\pgfqpoint{1.620313in}{2.153999in}}%
\pgfpathcurveto{\pgfqpoint{1.614489in}{2.159823in}}{\pgfqpoint{1.606589in}{2.163095in}}{\pgfqpoint{1.598352in}{2.163095in}}%
\pgfpathcurveto{\pgfqpoint{1.590116in}{2.163095in}}{\pgfqpoint{1.582216in}{2.159823in}}{\pgfqpoint{1.576392in}{2.153999in}}%
\pgfpathcurveto{\pgfqpoint{1.570568in}{2.148175in}}{\pgfqpoint{1.567296in}{2.140275in}}{\pgfqpoint{1.567296in}{2.132038in}}%
\pgfpathcurveto{\pgfqpoint{1.567296in}{2.123802in}}{\pgfqpoint{1.570568in}{2.115902in}}{\pgfqpoint{1.576392in}{2.110078in}}%
\pgfpathcurveto{\pgfqpoint{1.582216in}{2.104254in}}{\pgfqpoint{1.590116in}{2.100982in}}{\pgfqpoint{1.598352in}{2.100982in}}%
\pgfpathclose%
\pgfusepath{stroke,fill}%
\end{pgfscope}%
\begin{pgfscope}%
\pgfpathrectangle{\pgfqpoint{0.100000in}{0.212622in}}{\pgfqpoint{3.696000in}{3.696000in}}%
\pgfusepath{clip}%
\pgfsetbuttcap%
\pgfsetroundjoin%
\definecolor{currentfill}{rgb}{0.121569,0.466667,0.705882}%
\pgfsetfillcolor{currentfill}%
\pgfsetfillopacity{0.345033}%
\pgfsetlinewidth{1.003750pt}%
\definecolor{currentstroke}{rgb}{0.121569,0.466667,0.705882}%
\pgfsetstrokecolor{currentstroke}%
\pgfsetstrokeopacity{0.345033}%
\pgfsetdash{}{0pt}%
\pgfpathmoveto{\pgfqpoint{1.597175in}{2.101038in}}%
\pgfpathcurveto{\pgfqpoint{1.605411in}{2.101038in}}{\pgfqpoint{1.613311in}{2.104310in}}{\pgfqpoint{1.619135in}{2.110134in}}%
\pgfpathcurveto{\pgfqpoint{1.624959in}{2.115958in}}{\pgfqpoint{1.628231in}{2.123858in}}{\pgfqpoint{1.628231in}{2.132094in}}%
\pgfpathcurveto{\pgfqpoint{1.628231in}{2.140331in}}{\pgfqpoint{1.624959in}{2.148231in}}{\pgfqpoint{1.619135in}{2.154055in}}%
\pgfpathcurveto{\pgfqpoint{1.613311in}{2.159878in}}{\pgfqpoint{1.605411in}{2.163151in}}{\pgfqpoint{1.597175in}{2.163151in}}%
\pgfpathcurveto{\pgfqpoint{1.588938in}{2.163151in}}{\pgfqpoint{1.581038in}{2.159878in}}{\pgfqpoint{1.575214in}{2.154055in}}%
\pgfpathcurveto{\pgfqpoint{1.569391in}{2.148231in}}{\pgfqpoint{1.566118in}{2.140331in}}{\pgfqpoint{1.566118in}{2.132094in}}%
\pgfpathcurveto{\pgfqpoint{1.566118in}{2.123858in}}{\pgfqpoint{1.569391in}{2.115958in}}{\pgfqpoint{1.575214in}{2.110134in}}%
\pgfpathcurveto{\pgfqpoint{1.581038in}{2.104310in}}{\pgfqpoint{1.588938in}{2.101038in}}{\pgfqpoint{1.597175in}{2.101038in}}%
\pgfpathclose%
\pgfusepath{stroke,fill}%
\end{pgfscope}%
\begin{pgfscope}%
\pgfpathrectangle{\pgfqpoint{0.100000in}{0.212622in}}{\pgfqpoint{3.696000in}{3.696000in}}%
\pgfusepath{clip}%
\pgfsetbuttcap%
\pgfsetroundjoin%
\definecolor{currentfill}{rgb}{0.121569,0.466667,0.705882}%
\pgfsetfillcolor{currentfill}%
\pgfsetfillopacity{0.345466}%
\pgfsetlinewidth{1.003750pt}%
\definecolor{currentstroke}{rgb}{0.121569,0.466667,0.705882}%
\pgfsetstrokecolor{currentstroke}%
\pgfsetstrokeopacity{0.345466}%
\pgfsetdash{}{0pt}%
\pgfpathmoveto{\pgfqpoint{2.020618in}{2.036514in}}%
\pgfpathcurveto{\pgfqpoint{2.028854in}{2.036514in}}{\pgfqpoint{2.036754in}{2.039786in}}{\pgfqpoint{2.042578in}{2.045610in}}%
\pgfpathcurveto{\pgfqpoint{2.048402in}{2.051434in}}{\pgfqpoint{2.051675in}{2.059334in}}{\pgfqpoint{2.051675in}{2.067571in}}%
\pgfpathcurveto{\pgfqpoint{2.051675in}{2.075807in}}{\pgfqpoint{2.048402in}{2.083707in}}{\pgfqpoint{2.042578in}{2.089531in}}%
\pgfpathcurveto{\pgfqpoint{2.036754in}{2.095355in}}{\pgfqpoint{2.028854in}{2.098627in}}{\pgfqpoint{2.020618in}{2.098627in}}%
\pgfpathcurveto{\pgfqpoint{2.012382in}{2.098627in}}{\pgfqpoint{2.004482in}{2.095355in}}{\pgfqpoint{1.998658in}{2.089531in}}%
\pgfpathcurveto{\pgfqpoint{1.992834in}{2.083707in}}{\pgfqpoint{1.989562in}{2.075807in}}{\pgfqpoint{1.989562in}{2.067571in}}%
\pgfpathcurveto{\pgfqpoint{1.989562in}{2.059334in}}{\pgfqpoint{1.992834in}{2.051434in}}{\pgfqpoint{1.998658in}{2.045610in}}%
\pgfpathcurveto{\pgfqpoint{2.004482in}{2.039786in}}{\pgfqpoint{2.012382in}{2.036514in}}{\pgfqpoint{2.020618in}{2.036514in}}%
\pgfpathclose%
\pgfusepath{stroke,fill}%
\end{pgfscope}%
\begin{pgfscope}%
\pgfpathrectangle{\pgfqpoint{0.100000in}{0.212622in}}{\pgfqpoint{3.696000in}{3.696000in}}%
\pgfusepath{clip}%
\pgfsetbuttcap%
\pgfsetroundjoin%
\definecolor{currentfill}{rgb}{0.121569,0.466667,0.705882}%
\pgfsetfillcolor{currentfill}%
\pgfsetfillopacity{0.346101}%
\pgfsetlinewidth{1.003750pt}%
\definecolor{currentstroke}{rgb}{0.121569,0.466667,0.705882}%
\pgfsetstrokecolor{currentstroke}%
\pgfsetstrokeopacity{0.346101}%
\pgfsetdash{}{0pt}%
\pgfpathmoveto{\pgfqpoint{1.594325in}{2.101442in}}%
\pgfpathcurveto{\pgfqpoint{1.602561in}{2.101442in}}{\pgfqpoint{1.610461in}{2.104714in}}{\pgfqpoint{1.616285in}{2.110538in}}%
\pgfpathcurveto{\pgfqpoint{1.622109in}{2.116362in}}{\pgfqpoint{1.625381in}{2.124262in}}{\pgfqpoint{1.625381in}{2.132498in}}%
\pgfpathcurveto{\pgfqpoint{1.625381in}{2.140734in}}{\pgfqpoint{1.622109in}{2.148634in}}{\pgfqpoint{1.616285in}{2.154458in}}%
\pgfpathcurveto{\pgfqpoint{1.610461in}{2.160282in}}{\pgfqpoint{1.602561in}{2.163555in}}{\pgfqpoint{1.594325in}{2.163555in}}%
\pgfpathcurveto{\pgfqpoint{1.586089in}{2.163555in}}{\pgfqpoint{1.578189in}{2.160282in}}{\pgfqpoint{1.572365in}{2.154458in}}%
\pgfpathcurveto{\pgfqpoint{1.566541in}{2.148634in}}{\pgfqpoint{1.563268in}{2.140734in}}{\pgfqpoint{1.563268in}{2.132498in}}%
\pgfpathcurveto{\pgfqpoint{1.563268in}{2.124262in}}{\pgfqpoint{1.566541in}{2.116362in}}{\pgfqpoint{1.572365in}{2.110538in}}%
\pgfpathcurveto{\pgfqpoint{1.578189in}{2.104714in}}{\pgfqpoint{1.586089in}{2.101442in}}{\pgfqpoint{1.594325in}{2.101442in}}%
\pgfpathclose%
\pgfusepath{stroke,fill}%
\end{pgfscope}%
\begin{pgfscope}%
\pgfpathrectangle{\pgfqpoint{0.100000in}{0.212622in}}{\pgfqpoint{3.696000in}{3.696000in}}%
\pgfusepath{clip}%
\pgfsetbuttcap%
\pgfsetroundjoin%
\definecolor{currentfill}{rgb}{0.121569,0.466667,0.705882}%
\pgfsetfillcolor{currentfill}%
\pgfsetfillopacity{0.346776}%
\pgfsetlinewidth{1.003750pt}%
\definecolor{currentstroke}{rgb}{0.121569,0.466667,0.705882}%
\pgfsetstrokecolor{currentstroke}%
\pgfsetstrokeopacity{0.346776}%
\pgfsetdash{}{0pt}%
\pgfpathmoveto{\pgfqpoint{1.593383in}{2.101526in}}%
\pgfpathcurveto{\pgfqpoint{1.601620in}{2.101526in}}{\pgfqpoint{1.609520in}{2.104798in}}{\pgfqpoint{1.615344in}{2.110622in}}%
\pgfpathcurveto{\pgfqpoint{1.621168in}{2.116446in}}{\pgfqpoint{1.624440in}{2.124346in}}{\pgfqpoint{1.624440in}{2.132582in}}%
\pgfpathcurveto{\pgfqpoint{1.624440in}{2.140818in}}{\pgfqpoint{1.621168in}{2.148719in}}{\pgfqpoint{1.615344in}{2.154542in}}%
\pgfpathcurveto{\pgfqpoint{1.609520in}{2.160366in}}{\pgfqpoint{1.601620in}{2.163639in}}{\pgfqpoint{1.593383in}{2.163639in}}%
\pgfpathcurveto{\pgfqpoint{1.585147in}{2.163639in}}{\pgfqpoint{1.577247in}{2.160366in}}{\pgfqpoint{1.571423in}{2.154542in}}%
\pgfpathcurveto{\pgfqpoint{1.565599in}{2.148719in}}{\pgfqpoint{1.562327in}{2.140818in}}{\pgfqpoint{1.562327in}{2.132582in}}%
\pgfpathcurveto{\pgfqpoint{1.562327in}{2.124346in}}{\pgfqpoint{1.565599in}{2.116446in}}{\pgfqpoint{1.571423in}{2.110622in}}%
\pgfpathcurveto{\pgfqpoint{1.577247in}{2.104798in}}{\pgfqpoint{1.585147in}{2.101526in}}{\pgfqpoint{1.593383in}{2.101526in}}%
\pgfpathclose%
\pgfusepath{stroke,fill}%
\end{pgfscope}%
\begin{pgfscope}%
\pgfpathrectangle{\pgfqpoint{0.100000in}{0.212622in}}{\pgfqpoint{3.696000in}{3.696000in}}%
\pgfusepath{clip}%
\pgfsetbuttcap%
\pgfsetroundjoin%
\definecolor{currentfill}{rgb}{0.121569,0.466667,0.705882}%
\pgfsetfillcolor{currentfill}%
\pgfsetfillopacity{0.346820}%
\pgfsetlinewidth{1.003750pt}%
\definecolor{currentstroke}{rgb}{0.121569,0.466667,0.705882}%
\pgfsetstrokecolor{currentstroke}%
\pgfsetstrokeopacity{0.346820}%
\pgfsetdash{}{0pt}%
\pgfpathmoveto{\pgfqpoint{1.593299in}{2.101530in}}%
\pgfpathcurveto{\pgfqpoint{1.601536in}{2.101530in}}{\pgfqpoint{1.609436in}{2.104803in}}{\pgfqpoint{1.615260in}{2.110627in}}%
\pgfpathcurveto{\pgfqpoint{1.621084in}{2.116450in}}{\pgfqpoint{1.624356in}{2.124351in}}{\pgfqpoint{1.624356in}{2.132587in}}%
\pgfpathcurveto{\pgfqpoint{1.624356in}{2.140823in}}{\pgfqpoint{1.621084in}{2.148723in}}{\pgfqpoint{1.615260in}{2.154547in}}%
\pgfpathcurveto{\pgfqpoint{1.609436in}{2.160371in}}{\pgfqpoint{1.601536in}{2.163643in}}{\pgfqpoint{1.593299in}{2.163643in}}%
\pgfpathcurveto{\pgfqpoint{1.585063in}{2.163643in}}{\pgfqpoint{1.577163in}{2.160371in}}{\pgfqpoint{1.571339in}{2.154547in}}%
\pgfpathcurveto{\pgfqpoint{1.565515in}{2.148723in}}{\pgfqpoint{1.562243in}{2.140823in}}{\pgfqpoint{1.562243in}{2.132587in}}%
\pgfpathcurveto{\pgfqpoint{1.562243in}{2.124351in}}{\pgfqpoint{1.565515in}{2.116450in}}{\pgfqpoint{1.571339in}{2.110627in}}%
\pgfpathcurveto{\pgfqpoint{1.577163in}{2.104803in}}{\pgfqpoint{1.585063in}{2.101530in}}{\pgfqpoint{1.593299in}{2.101530in}}%
\pgfpathclose%
\pgfusepath{stroke,fill}%
\end{pgfscope}%
\begin{pgfscope}%
\pgfpathrectangle{\pgfqpoint{0.100000in}{0.212622in}}{\pgfqpoint{3.696000in}{3.696000in}}%
\pgfusepath{clip}%
\pgfsetbuttcap%
\pgfsetroundjoin%
\definecolor{currentfill}{rgb}{0.121569,0.466667,0.705882}%
\pgfsetfillcolor{currentfill}%
\pgfsetfillopacity{0.346895}%
\pgfsetlinewidth{1.003750pt}%
\definecolor{currentstroke}{rgb}{0.121569,0.466667,0.705882}%
\pgfsetstrokecolor{currentstroke}%
\pgfsetstrokeopacity{0.346895}%
\pgfsetdash{}{0pt}%
\pgfpathmoveto{\pgfqpoint{1.593118in}{2.101537in}}%
\pgfpathcurveto{\pgfqpoint{1.601354in}{2.101537in}}{\pgfqpoint{1.609254in}{2.104810in}}{\pgfqpoint{1.615078in}{2.110634in}}%
\pgfpathcurveto{\pgfqpoint{1.620902in}{2.116458in}}{\pgfqpoint{1.624174in}{2.124358in}}{\pgfqpoint{1.624174in}{2.132594in}}%
\pgfpathcurveto{\pgfqpoint{1.624174in}{2.140830in}}{\pgfqpoint{1.620902in}{2.148730in}}{\pgfqpoint{1.615078in}{2.154554in}}%
\pgfpathcurveto{\pgfqpoint{1.609254in}{2.160378in}}{\pgfqpoint{1.601354in}{2.163650in}}{\pgfqpoint{1.593118in}{2.163650in}}%
\pgfpathcurveto{\pgfqpoint{1.584882in}{2.163650in}}{\pgfqpoint{1.576981in}{2.160378in}}{\pgfqpoint{1.571158in}{2.154554in}}%
\pgfpathcurveto{\pgfqpoint{1.565334in}{2.148730in}}{\pgfqpoint{1.562061in}{2.140830in}}{\pgfqpoint{1.562061in}{2.132594in}}%
\pgfpathcurveto{\pgfqpoint{1.562061in}{2.124358in}}{\pgfqpoint{1.565334in}{2.116458in}}{\pgfqpoint{1.571158in}{2.110634in}}%
\pgfpathcurveto{\pgfqpoint{1.576981in}{2.104810in}}{\pgfqpoint{1.584882in}{2.101537in}}{\pgfqpoint{1.593118in}{2.101537in}}%
\pgfpathclose%
\pgfusepath{stroke,fill}%
\end{pgfscope}%
\begin{pgfscope}%
\pgfpathrectangle{\pgfqpoint{0.100000in}{0.212622in}}{\pgfqpoint{3.696000in}{3.696000in}}%
\pgfusepath{clip}%
\pgfsetbuttcap%
\pgfsetroundjoin%
\definecolor{currentfill}{rgb}{0.121569,0.466667,0.705882}%
\pgfsetfillcolor{currentfill}%
\pgfsetfillopacity{0.346937}%
\pgfsetlinewidth{1.003750pt}%
\definecolor{currentstroke}{rgb}{0.121569,0.466667,0.705882}%
\pgfsetstrokecolor{currentstroke}%
\pgfsetstrokeopacity{0.346937}%
\pgfsetdash{}{0pt}%
\pgfpathmoveto{\pgfqpoint{2.028732in}{2.034776in}}%
\pgfpathcurveto{\pgfqpoint{2.036968in}{2.034776in}}{\pgfqpoint{2.044868in}{2.038049in}}{\pgfqpoint{2.050692in}{2.043873in}}%
\pgfpathcurveto{\pgfqpoint{2.056516in}{2.049696in}}{\pgfqpoint{2.059789in}{2.057596in}}{\pgfqpoint{2.059789in}{2.065833in}}%
\pgfpathcurveto{\pgfqpoint{2.059789in}{2.074069in}}{\pgfqpoint{2.056516in}{2.081969in}}{\pgfqpoint{2.050692in}{2.087793in}}%
\pgfpathcurveto{\pgfqpoint{2.044868in}{2.093617in}}{\pgfqpoint{2.036968in}{2.096889in}}{\pgfqpoint{2.028732in}{2.096889in}}%
\pgfpathcurveto{\pgfqpoint{2.020496in}{2.096889in}}{\pgfqpoint{2.012596in}{2.093617in}}{\pgfqpoint{2.006772in}{2.087793in}}%
\pgfpathcurveto{\pgfqpoint{2.000948in}{2.081969in}}{\pgfqpoint{1.997676in}{2.074069in}}{\pgfqpoint{1.997676in}{2.065833in}}%
\pgfpathcurveto{\pgfqpoint{1.997676in}{2.057596in}}{\pgfqpoint{2.000948in}{2.049696in}}{\pgfqpoint{2.006772in}{2.043873in}}%
\pgfpathcurveto{\pgfqpoint{2.012596in}{2.038049in}}{\pgfqpoint{2.020496in}{2.034776in}}{\pgfqpoint{2.028732in}{2.034776in}}%
\pgfpathclose%
\pgfusepath{stroke,fill}%
\end{pgfscope}%
\begin{pgfscope}%
\pgfpathrectangle{\pgfqpoint{0.100000in}{0.212622in}}{\pgfqpoint{3.696000in}{3.696000in}}%
\pgfusepath{clip}%
\pgfsetbuttcap%
\pgfsetroundjoin%
\definecolor{currentfill}{rgb}{0.121569,0.466667,0.705882}%
\pgfsetfillcolor{currentfill}%
\pgfsetfillopacity{0.347044}%
\pgfsetlinewidth{1.003750pt}%
\definecolor{currentstroke}{rgb}{0.121569,0.466667,0.705882}%
\pgfsetstrokecolor{currentstroke}%
\pgfsetstrokeopacity{0.347044}%
\pgfsetdash{}{0pt}%
\pgfpathmoveto{\pgfqpoint{1.592900in}{2.101543in}}%
\pgfpathcurveto{\pgfqpoint{1.601136in}{2.101543in}}{\pgfqpoint{1.609036in}{2.104815in}}{\pgfqpoint{1.614860in}{2.110639in}}%
\pgfpathcurveto{\pgfqpoint{1.620684in}{2.116463in}}{\pgfqpoint{1.623956in}{2.124363in}}{\pgfqpoint{1.623956in}{2.132599in}}%
\pgfpathcurveto{\pgfqpoint{1.623956in}{2.140836in}}{\pgfqpoint{1.620684in}{2.148736in}}{\pgfqpoint{1.614860in}{2.154560in}}%
\pgfpathcurveto{\pgfqpoint{1.609036in}{2.160384in}}{\pgfqpoint{1.601136in}{2.163656in}}{\pgfqpoint{1.592900in}{2.163656in}}%
\pgfpathcurveto{\pgfqpoint{1.584664in}{2.163656in}}{\pgfqpoint{1.576764in}{2.160384in}}{\pgfqpoint{1.570940in}{2.154560in}}%
\pgfpathcurveto{\pgfqpoint{1.565116in}{2.148736in}}{\pgfqpoint{1.561843in}{2.140836in}}{\pgfqpoint{1.561843in}{2.132599in}}%
\pgfpathcurveto{\pgfqpoint{1.561843in}{2.124363in}}{\pgfqpoint{1.565116in}{2.116463in}}{\pgfqpoint{1.570940in}{2.110639in}}%
\pgfpathcurveto{\pgfqpoint{1.576764in}{2.104815in}}{\pgfqpoint{1.584664in}{2.101543in}}{\pgfqpoint{1.592900in}{2.101543in}}%
\pgfpathclose%
\pgfusepath{stroke,fill}%
\end{pgfscope}%
\begin{pgfscope}%
\pgfpathrectangle{\pgfqpoint{0.100000in}{0.212622in}}{\pgfqpoint{3.696000in}{3.696000in}}%
\pgfusepath{clip}%
\pgfsetbuttcap%
\pgfsetroundjoin%
\definecolor{currentfill}{rgb}{0.121569,0.466667,0.705882}%
\pgfsetfillcolor{currentfill}%
\pgfsetfillopacity{0.347303}%
\pgfsetlinewidth{1.003750pt}%
\definecolor{currentstroke}{rgb}{0.121569,0.466667,0.705882}%
\pgfsetstrokecolor{currentstroke}%
\pgfsetstrokeopacity{0.347303}%
\pgfsetdash{}{0pt}%
\pgfpathmoveto{\pgfqpoint{1.592389in}{2.101553in}}%
\pgfpathcurveto{\pgfqpoint{1.600625in}{2.101553in}}{\pgfqpoint{1.608525in}{2.104825in}}{\pgfqpoint{1.614349in}{2.110649in}}%
\pgfpathcurveto{\pgfqpoint{1.620173in}{2.116473in}}{\pgfqpoint{1.623445in}{2.124373in}}{\pgfqpoint{1.623445in}{2.132609in}}%
\pgfpathcurveto{\pgfqpoint{1.623445in}{2.140845in}}{\pgfqpoint{1.620173in}{2.148745in}}{\pgfqpoint{1.614349in}{2.154569in}}%
\pgfpathcurveto{\pgfqpoint{1.608525in}{2.160393in}}{\pgfqpoint{1.600625in}{2.163666in}}{\pgfqpoint{1.592389in}{2.163666in}}%
\pgfpathcurveto{\pgfqpoint{1.584152in}{2.163666in}}{\pgfqpoint{1.576252in}{2.160393in}}{\pgfqpoint{1.570428in}{2.154569in}}%
\pgfpathcurveto{\pgfqpoint{1.564604in}{2.148745in}}{\pgfqpoint{1.561332in}{2.140845in}}{\pgfqpoint{1.561332in}{2.132609in}}%
\pgfpathcurveto{\pgfqpoint{1.561332in}{2.124373in}}{\pgfqpoint{1.564604in}{2.116473in}}{\pgfqpoint{1.570428in}{2.110649in}}%
\pgfpathcurveto{\pgfqpoint{1.576252in}{2.104825in}}{\pgfqpoint{1.584152in}{2.101553in}}{\pgfqpoint{1.592389in}{2.101553in}}%
\pgfpathclose%
\pgfusepath{stroke,fill}%
\end{pgfscope}%
\begin{pgfscope}%
\pgfpathrectangle{\pgfqpoint{0.100000in}{0.212622in}}{\pgfqpoint{3.696000in}{3.696000in}}%
\pgfusepath{clip}%
\pgfsetbuttcap%
\pgfsetroundjoin%
\definecolor{currentfill}{rgb}{0.121569,0.466667,0.705882}%
\pgfsetfillcolor{currentfill}%
\pgfsetfillopacity{0.347759}%
\pgfsetlinewidth{1.003750pt}%
\definecolor{currentstroke}{rgb}{0.121569,0.466667,0.705882}%
\pgfsetstrokecolor{currentstroke}%
\pgfsetstrokeopacity{0.347759}%
\pgfsetdash{}{0pt}%
\pgfpathmoveto{\pgfqpoint{1.591348in}{2.101565in}}%
\pgfpathcurveto{\pgfqpoint{1.599584in}{2.101565in}}{\pgfqpoint{1.607484in}{2.104838in}}{\pgfqpoint{1.613308in}{2.110661in}}%
\pgfpathcurveto{\pgfqpoint{1.619132in}{2.116485in}}{\pgfqpoint{1.622404in}{2.124385in}}{\pgfqpoint{1.622404in}{2.132622in}}%
\pgfpathcurveto{\pgfqpoint{1.622404in}{2.140858in}}{\pgfqpoint{1.619132in}{2.148758in}}{\pgfqpoint{1.613308in}{2.154582in}}%
\pgfpathcurveto{\pgfqpoint{1.607484in}{2.160406in}}{\pgfqpoint{1.599584in}{2.163678in}}{\pgfqpoint{1.591348in}{2.163678in}}%
\pgfpathcurveto{\pgfqpoint{1.583111in}{2.163678in}}{\pgfqpoint{1.575211in}{2.160406in}}{\pgfqpoint{1.569388in}{2.154582in}}%
\pgfpathcurveto{\pgfqpoint{1.563564in}{2.148758in}}{\pgfqpoint{1.560291in}{2.140858in}}{\pgfqpoint{1.560291in}{2.132622in}}%
\pgfpathcurveto{\pgfqpoint{1.560291in}{2.124385in}}{\pgfqpoint{1.563564in}{2.116485in}}{\pgfqpoint{1.569388in}{2.110661in}}%
\pgfpathcurveto{\pgfqpoint{1.575211in}{2.104838in}}{\pgfqpoint{1.583111in}{2.101565in}}{\pgfqpoint{1.591348in}{2.101565in}}%
\pgfpathclose%
\pgfusepath{stroke,fill}%
\end{pgfscope}%
\begin{pgfscope}%
\pgfpathrectangle{\pgfqpoint{0.100000in}{0.212622in}}{\pgfqpoint{3.696000in}{3.696000in}}%
\pgfusepath{clip}%
\pgfsetbuttcap%
\pgfsetroundjoin%
\definecolor{currentfill}{rgb}{0.121569,0.466667,0.705882}%
\pgfsetfillcolor{currentfill}%
\pgfsetfillopacity{0.347839}%
\pgfsetlinewidth{1.003750pt}%
\definecolor{currentstroke}{rgb}{0.121569,0.466667,0.705882}%
\pgfsetstrokecolor{currentstroke}%
\pgfsetstrokeopacity{0.347839}%
\pgfsetdash{}{0pt}%
\pgfpathmoveto{\pgfqpoint{1.591218in}{2.101566in}}%
\pgfpathcurveto{\pgfqpoint{1.599454in}{2.101566in}}{\pgfqpoint{1.607354in}{2.104838in}}{\pgfqpoint{1.613178in}{2.110662in}}%
\pgfpathcurveto{\pgfqpoint{1.619002in}{2.116486in}}{\pgfqpoint{1.622274in}{2.124386in}}{\pgfqpoint{1.622274in}{2.132623in}}%
\pgfpathcurveto{\pgfqpoint{1.622274in}{2.140859in}}{\pgfqpoint{1.619002in}{2.148759in}}{\pgfqpoint{1.613178in}{2.154583in}}%
\pgfpathcurveto{\pgfqpoint{1.607354in}{2.160407in}}{\pgfqpoint{1.599454in}{2.163679in}}{\pgfqpoint{1.591218in}{2.163679in}}%
\pgfpathcurveto{\pgfqpoint{1.582981in}{2.163679in}}{\pgfqpoint{1.575081in}{2.160407in}}{\pgfqpoint{1.569257in}{2.154583in}}%
\pgfpathcurveto{\pgfqpoint{1.563433in}{2.148759in}}{\pgfqpoint{1.560161in}{2.140859in}}{\pgfqpoint{1.560161in}{2.132623in}}%
\pgfpathcurveto{\pgfqpoint{1.560161in}{2.124386in}}{\pgfqpoint{1.563433in}{2.116486in}}{\pgfqpoint{1.569257in}{2.110662in}}%
\pgfpathcurveto{\pgfqpoint{1.575081in}{2.104838in}}{\pgfqpoint{1.582981in}{2.101566in}}{\pgfqpoint{1.591218in}{2.101566in}}%
\pgfpathclose%
\pgfusepath{stroke,fill}%
\end{pgfscope}%
\begin{pgfscope}%
\pgfpathrectangle{\pgfqpoint{0.100000in}{0.212622in}}{\pgfqpoint{3.696000in}{3.696000in}}%
\pgfusepath{clip}%
\pgfsetbuttcap%
\pgfsetroundjoin%
\definecolor{currentfill}{rgb}{0.121569,0.466667,0.705882}%
\pgfsetfillcolor{currentfill}%
\pgfsetfillopacity{0.347977}%
\pgfsetlinewidth{1.003750pt}%
\definecolor{currentstroke}{rgb}{0.121569,0.466667,0.705882}%
\pgfsetstrokecolor{currentstroke}%
\pgfsetstrokeopacity{0.347977}%
\pgfsetdash{}{0pt}%
\pgfpathmoveto{\pgfqpoint{1.590901in}{2.101578in}}%
\pgfpathcurveto{\pgfqpoint{1.599137in}{2.101578in}}{\pgfqpoint{1.607037in}{2.104851in}}{\pgfqpoint{1.612861in}{2.110675in}}%
\pgfpathcurveto{\pgfqpoint{1.618685in}{2.116499in}}{\pgfqpoint{1.621957in}{2.124399in}}{\pgfqpoint{1.621957in}{2.132635in}}%
\pgfpathcurveto{\pgfqpoint{1.621957in}{2.140871in}}{\pgfqpoint{1.618685in}{2.148771in}}{\pgfqpoint{1.612861in}{2.154595in}}%
\pgfpathcurveto{\pgfqpoint{1.607037in}{2.160419in}}{\pgfqpoint{1.599137in}{2.163691in}}{\pgfqpoint{1.590901in}{2.163691in}}%
\pgfpathcurveto{\pgfqpoint{1.582665in}{2.163691in}}{\pgfqpoint{1.574764in}{2.160419in}}{\pgfqpoint{1.568941in}{2.154595in}}%
\pgfpathcurveto{\pgfqpoint{1.563117in}{2.148771in}}{\pgfqpoint{1.559844in}{2.140871in}}{\pgfqpoint{1.559844in}{2.132635in}}%
\pgfpathcurveto{\pgfqpoint{1.559844in}{2.124399in}}{\pgfqpoint{1.563117in}{2.116499in}}{\pgfqpoint{1.568941in}{2.110675in}}%
\pgfpathcurveto{\pgfqpoint{1.574764in}{2.104851in}}{\pgfqpoint{1.582665in}{2.101578in}}{\pgfqpoint{1.590901in}{2.101578in}}%
\pgfpathclose%
\pgfusepath{stroke,fill}%
\end{pgfscope}%
\begin{pgfscope}%
\pgfpathrectangle{\pgfqpoint{0.100000in}{0.212622in}}{\pgfqpoint{3.696000in}{3.696000in}}%
\pgfusepath{clip}%
\pgfsetbuttcap%
\pgfsetroundjoin%
\definecolor{currentfill}{rgb}{0.121569,0.466667,0.705882}%
\pgfsetfillcolor{currentfill}%
\pgfsetfillopacity{0.348228}%
\pgfsetlinewidth{1.003750pt}%
\definecolor{currentstroke}{rgb}{0.121569,0.466667,0.705882}%
\pgfsetstrokecolor{currentstroke}%
\pgfsetstrokeopacity{0.348228}%
\pgfsetdash{}{0pt}%
\pgfpathmoveto{\pgfqpoint{1.590327in}{2.101606in}}%
\pgfpathcurveto{\pgfqpoint{1.598564in}{2.101606in}}{\pgfqpoint{1.606464in}{2.104879in}}{\pgfqpoint{1.612288in}{2.110703in}}%
\pgfpathcurveto{\pgfqpoint{1.618111in}{2.116527in}}{\pgfqpoint{1.621384in}{2.124427in}}{\pgfqpoint{1.621384in}{2.132663in}}%
\pgfpathcurveto{\pgfqpoint{1.621384in}{2.140899in}}{\pgfqpoint{1.618111in}{2.148799in}}{\pgfqpoint{1.612288in}{2.154623in}}%
\pgfpathcurveto{\pgfqpoint{1.606464in}{2.160447in}}{\pgfqpoint{1.598564in}{2.163719in}}{\pgfqpoint{1.590327in}{2.163719in}}%
\pgfpathcurveto{\pgfqpoint{1.582091in}{2.163719in}}{\pgfqpoint{1.574191in}{2.160447in}}{\pgfqpoint{1.568367in}{2.154623in}}%
\pgfpathcurveto{\pgfqpoint{1.562543in}{2.148799in}}{\pgfqpoint{1.559271in}{2.140899in}}{\pgfqpoint{1.559271in}{2.132663in}}%
\pgfpathcurveto{\pgfqpoint{1.559271in}{2.124427in}}{\pgfqpoint{1.562543in}{2.116527in}}{\pgfqpoint{1.568367in}{2.110703in}}%
\pgfpathcurveto{\pgfqpoint{1.574191in}{2.104879in}}{\pgfqpoint{1.582091in}{2.101606in}}{\pgfqpoint{1.590327in}{2.101606in}}%
\pgfpathclose%
\pgfusepath{stroke,fill}%
\end{pgfscope}%
\begin{pgfscope}%
\pgfpathrectangle{\pgfqpoint{0.100000in}{0.212622in}}{\pgfqpoint{3.696000in}{3.696000in}}%
\pgfusepath{clip}%
\pgfsetbuttcap%
\pgfsetroundjoin%
\definecolor{currentfill}{rgb}{0.121569,0.466667,0.705882}%
\pgfsetfillcolor{currentfill}%
\pgfsetfillopacity{0.348314}%
\pgfsetlinewidth{1.003750pt}%
\definecolor{currentstroke}{rgb}{0.121569,0.466667,0.705882}%
\pgfsetstrokecolor{currentstroke}%
\pgfsetstrokeopacity{0.348314}%
\pgfsetdash{}{0pt}%
\pgfpathmoveto{\pgfqpoint{1.590213in}{2.101615in}}%
\pgfpathcurveto{\pgfqpoint{1.598449in}{2.101615in}}{\pgfqpoint{1.606349in}{2.104888in}}{\pgfqpoint{1.612173in}{2.110712in}}%
\pgfpathcurveto{\pgfqpoint{1.617997in}{2.116536in}}{\pgfqpoint{1.621269in}{2.124436in}}{\pgfqpoint{1.621269in}{2.132672in}}%
\pgfpathcurveto{\pgfqpoint{1.621269in}{2.140908in}}{\pgfqpoint{1.617997in}{2.148808in}}{\pgfqpoint{1.612173in}{2.154632in}}%
\pgfpathcurveto{\pgfqpoint{1.606349in}{2.160456in}}{\pgfqpoint{1.598449in}{2.163728in}}{\pgfqpoint{1.590213in}{2.163728in}}%
\pgfpathcurveto{\pgfqpoint{1.581976in}{2.163728in}}{\pgfqpoint{1.574076in}{2.160456in}}{\pgfqpoint{1.568252in}{2.154632in}}%
\pgfpathcurveto{\pgfqpoint{1.562428in}{2.148808in}}{\pgfqpoint{1.559156in}{2.140908in}}{\pgfqpoint{1.559156in}{2.132672in}}%
\pgfpathcurveto{\pgfqpoint{1.559156in}{2.124436in}}{\pgfqpoint{1.562428in}{2.116536in}}{\pgfqpoint{1.568252in}{2.110712in}}%
\pgfpathcurveto{\pgfqpoint{1.574076in}{2.104888in}}{\pgfqpoint{1.581976in}{2.101615in}}{\pgfqpoint{1.590213in}{2.101615in}}%
\pgfpathclose%
\pgfusepath{stroke,fill}%
\end{pgfscope}%
\begin{pgfscope}%
\pgfpathrectangle{\pgfqpoint{0.100000in}{0.212622in}}{\pgfqpoint{3.696000in}{3.696000in}}%
\pgfusepath{clip}%
\pgfsetbuttcap%
\pgfsetroundjoin%
\definecolor{currentfill}{rgb}{0.121569,0.466667,0.705882}%
\pgfsetfillcolor{currentfill}%
\pgfsetfillopacity{0.348456}%
\pgfsetlinewidth{1.003750pt}%
\definecolor{currentstroke}{rgb}{0.121569,0.466667,0.705882}%
\pgfsetstrokecolor{currentstroke}%
\pgfsetstrokeopacity{0.348456}%
\pgfsetdash{}{0pt}%
\pgfpathmoveto{\pgfqpoint{1.589887in}{2.101628in}}%
\pgfpathcurveto{\pgfqpoint{1.598123in}{2.101628in}}{\pgfqpoint{1.606023in}{2.104900in}}{\pgfqpoint{1.611847in}{2.110724in}}%
\pgfpathcurveto{\pgfqpoint{1.617671in}{2.116548in}}{\pgfqpoint{1.620943in}{2.124448in}}{\pgfqpoint{1.620943in}{2.132684in}}%
\pgfpathcurveto{\pgfqpoint{1.620943in}{2.140920in}}{\pgfqpoint{1.617671in}{2.148820in}}{\pgfqpoint{1.611847in}{2.154644in}}%
\pgfpathcurveto{\pgfqpoint{1.606023in}{2.160468in}}{\pgfqpoint{1.598123in}{2.163741in}}{\pgfqpoint{1.589887in}{2.163741in}}%
\pgfpathcurveto{\pgfqpoint{1.581651in}{2.163741in}}{\pgfqpoint{1.573751in}{2.160468in}}{\pgfqpoint{1.567927in}{2.154644in}}%
\pgfpathcurveto{\pgfqpoint{1.562103in}{2.148820in}}{\pgfqpoint{1.558830in}{2.140920in}}{\pgfqpoint{1.558830in}{2.132684in}}%
\pgfpathcurveto{\pgfqpoint{1.558830in}{2.124448in}}{\pgfqpoint{1.562103in}{2.116548in}}{\pgfqpoint{1.567927in}{2.110724in}}%
\pgfpathcurveto{\pgfqpoint{1.573751in}{2.104900in}}{\pgfqpoint{1.581651in}{2.101628in}}{\pgfqpoint{1.589887in}{2.101628in}}%
\pgfpathclose%
\pgfusepath{stroke,fill}%
\end{pgfscope}%
\begin{pgfscope}%
\pgfpathrectangle{\pgfqpoint{0.100000in}{0.212622in}}{\pgfqpoint{3.696000in}{3.696000in}}%
\pgfusepath{clip}%
\pgfsetbuttcap%
\pgfsetroundjoin%
\definecolor{currentfill}{rgb}{0.121569,0.466667,0.705882}%
\pgfsetfillcolor{currentfill}%
\pgfsetfillopacity{0.348670}%
\pgfsetlinewidth{1.003750pt}%
\definecolor{currentstroke}{rgb}{0.121569,0.466667,0.705882}%
\pgfsetstrokecolor{currentstroke}%
\pgfsetstrokeopacity{0.348670}%
\pgfsetdash{}{0pt}%
\pgfpathmoveto{\pgfqpoint{2.038565in}{2.032825in}}%
\pgfpathcurveto{\pgfqpoint{2.046801in}{2.032825in}}{\pgfqpoint{2.054701in}{2.036097in}}{\pgfqpoint{2.060525in}{2.041921in}}%
\pgfpathcurveto{\pgfqpoint{2.066349in}{2.047745in}}{\pgfqpoint{2.069621in}{2.055645in}}{\pgfqpoint{2.069621in}{2.063881in}}%
\pgfpathcurveto{\pgfqpoint{2.069621in}{2.072118in}}{\pgfqpoint{2.066349in}{2.080018in}}{\pgfqpoint{2.060525in}{2.085842in}}%
\pgfpathcurveto{\pgfqpoint{2.054701in}{2.091666in}}{\pgfqpoint{2.046801in}{2.094938in}}{\pgfqpoint{2.038565in}{2.094938in}}%
\pgfpathcurveto{\pgfqpoint{2.030328in}{2.094938in}}{\pgfqpoint{2.022428in}{2.091666in}}{\pgfqpoint{2.016604in}{2.085842in}}%
\pgfpathcurveto{\pgfqpoint{2.010781in}{2.080018in}}{\pgfqpoint{2.007508in}{2.072118in}}{\pgfqpoint{2.007508in}{2.063881in}}%
\pgfpathcurveto{\pgfqpoint{2.007508in}{2.055645in}}{\pgfqpoint{2.010781in}{2.047745in}}{\pgfqpoint{2.016604in}{2.041921in}}%
\pgfpathcurveto{\pgfqpoint{2.022428in}{2.036097in}}{\pgfqpoint{2.030328in}{2.032825in}}{\pgfqpoint{2.038565in}{2.032825in}}%
\pgfpathclose%
\pgfusepath{stroke,fill}%
\end{pgfscope}%
\begin{pgfscope}%
\pgfpathrectangle{\pgfqpoint{0.100000in}{0.212622in}}{\pgfqpoint{3.696000in}{3.696000in}}%
\pgfusepath{clip}%
\pgfsetbuttcap%
\pgfsetroundjoin%
\definecolor{currentfill}{rgb}{0.121569,0.466667,0.705882}%
\pgfsetfillcolor{currentfill}%
\pgfsetfillopacity{0.348723}%
\pgfsetlinewidth{1.003750pt}%
\definecolor{currentstroke}{rgb}{0.121569,0.466667,0.705882}%
\pgfsetstrokecolor{currentstroke}%
\pgfsetstrokeopacity{0.348723}%
\pgfsetdash{}{0pt}%
\pgfpathmoveto{\pgfqpoint{1.589400in}{2.101621in}}%
\pgfpathcurveto{\pgfqpoint{1.597636in}{2.101621in}}{\pgfqpoint{1.605536in}{2.104893in}}{\pgfqpoint{1.611360in}{2.110717in}}%
\pgfpathcurveto{\pgfqpoint{1.617184in}{2.116541in}}{\pgfqpoint{1.620457in}{2.124441in}}{\pgfqpoint{1.620457in}{2.132677in}}%
\pgfpathcurveto{\pgfqpoint{1.620457in}{2.140913in}}{\pgfqpoint{1.617184in}{2.148813in}}{\pgfqpoint{1.611360in}{2.154637in}}%
\pgfpathcurveto{\pgfqpoint{1.605536in}{2.160461in}}{\pgfqpoint{1.597636in}{2.163734in}}{\pgfqpoint{1.589400in}{2.163734in}}%
\pgfpathcurveto{\pgfqpoint{1.581164in}{2.163734in}}{\pgfqpoint{1.573264in}{2.160461in}}{\pgfqpoint{1.567440in}{2.154637in}}%
\pgfpathcurveto{\pgfqpoint{1.561616in}{2.148813in}}{\pgfqpoint{1.558344in}{2.140913in}}{\pgfqpoint{1.558344in}{2.132677in}}%
\pgfpathcurveto{\pgfqpoint{1.558344in}{2.124441in}}{\pgfqpoint{1.561616in}{2.116541in}}{\pgfqpoint{1.567440in}{2.110717in}}%
\pgfpathcurveto{\pgfqpoint{1.573264in}{2.104893in}}{\pgfqpoint{1.581164in}{2.101621in}}{\pgfqpoint{1.589400in}{2.101621in}}%
\pgfpathclose%
\pgfusepath{stroke,fill}%
\end{pgfscope}%
\begin{pgfscope}%
\pgfpathrectangle{\pgfqpoint{0.100000in}{0.212622in}}{\pgfqpoint{3.696000in}{3.696000in}}%
\pgfusepath{clip}%
\pgfsetbuttcap%
\pgfsetroundjoin%
\definecolor{currentfill}{rgb}{0.121569,0.466667,0.705882}%
\pgfsetfillcolor{currentfill}%
\pgfsetfillopacity{0.348860}%
\pgfsetlinewidth{1.003750pt}%
\definecolor{currentstroke}{rgb}{0.121569,0.466667,0.705882}%
\pgfsetstrokecolor{currentstroke}%
\pgfsetstrokeopacity{0.348860}%
\pgfsetdash{}{0pt}%
\pgfpathmoveto{\pgfqpoint{1.589214in}{2.101642in}}%
\pgfpathcurveto{\pgfqpoint{1.597451in}{2.101642in}}{\pgfqpoint{1.605351in}{2.104914in}}{\pgfqpoint{1.611175in}{2.110738in}}%
\pgfpathcurveto{\pgfqpoint{1.616998in}{2.116562in}}{\pgfqpoint{1.620271in}{2.124462in}}{\pgfqpoint{1.620271in}{2.132698in}}%
\pgfpathcurveto{\pgfqpoint{1.620271in}{2.140935in}}{\pgfqpoint{1.616998in}{2.148835in}}{\pgfqpoint{1.611175in}{2.154659in}}%
\pgfpathcurveto{\pgfqpoint{1.605351in}{2.160482in}}{\pgfqpoint{1.597451in}{2.163755in}}{\pgfqpoint{1.589214in}{2.163755in}}%
\pgfpathcurveto{\pgfqpoint{1.580978in}{2.163755in}}{\pgfqpoint{1.573078in}{2.160482in}}{\pgfqpoint{1.567254in}{2.154659in}}%
\pgfpathcurveto{\pgfqpoint{1.561430in}{2.148835in}}{\pgfqpoint{1.558158in}{2.140935in}}{\pgfqpoint{1.558158in}{2.132698in}}%
\pgfpathcurveto{\pgfqpoint{1.558158in}{2.124462in}}{\pgfqpoint{1.561430in}{2.116562in}}{\pgfqpoint{1.567254in}{2.110738in}}%
\pgfpathcurveto{\pgfqpoint{1.573078in}{2.104914in}}{\pgfqpoint{1.580978in}{2.101642in}}{\pgfqpoint{1.589214in}{2.101642in}}%
\pgfpathclose%
\pgfusepath{stroke,fill}%
\end{pgfscope}%
\begin{pgfscope}%
\pgfpathrectangle{\pgfqpoint{0.100000in}{0.212622in}}{\pgfqpoint{3.696000in}{3.696000in}}%
\pgfusepath{clip}%
\pgfsetbuttcap%
\pgfsetroundjoin%
\definecolor{currentfill}{rgb}{0.121569,0.466667,0.705882}%
\pgfsetfillcolor{currentfill}%
\pgfsetfillopacity{0.349083}%
\pgfsetlinewidth{1.003750pt}%
\definecolor{currentstroke}{rgb}{0.121569,0.466667,0.705882}%
\pgfsetstrokecolor{currentstroke}%
\pgfsetstrokeopacity{0.349083}%
\pgfsetdash{}{0pt}%
\pgfpathmoveto{\pgfqpoint{1.588661in}{2.101668in}}%
\pgfpathcurveto{\pgfqpoint{1.596897in}{2.101668in}}{\pgfqpoint{1.604797in}{2.104941in}}{\pgfqpoint{1.610621in}{2.110765in}}%
\pgfpathcurveto{\pgfqpoint{1.616445in}{2.116589in}}{\pgfqpoint{1.619717in}{2.124489in}}{\pgfqpoint{1.619717in}{2.132725in}}%
\pgfpathcurveto{\pgfqpoint{1.619717in}{2.140961in}}{\pgfqpoint{1.616445in}{2.148861in}}{\pgfqpoint{1.610621in}{2.154685in}}%
\pgfpathcurveto{\pgfqpoint{1.604797in}{2.160509in}}{\pgfqpoint{1.596897in}{2.163781in}}{\pgfqpoint{1.588661in}{2.163781in}}%
\pgfpathcurveto{\pgfqpoint{1.580424in}{2.163781in}}{\pgfqpoint{1.572524in}{2.160509in}}{\pgfqpoint{1.566700in}{2.154685in}}%
\pgfpathcurveto{\pgfqpoint{1.560876in}{2.148861in}}{\pgfqpoint{1.557604in}{2.140961in}}{\pgfqpoint{1.557604in}{2.132725in}}%
\pgfpathcurveto{\pgfqpoint{1.557604in}{2.124489in}}{\pgfqpoint{1.560876in}{2.116589in}}{\pgfqpoint{1.566700in}{2.110765in}}%
\pgfpathcurveto{\pgfqpoint{1.572524in}{2.104941in}}{\pgfqpoint{1.580424in}{2.101668in}}{\pgfqpoint{1.588661in}{2.101668in}}%
\pgfpathclose%
\pgfusepath{stroke,fill}%
\end{pgfscope}%
\begin{pgfscope}%
\pgfpathrectangle{\pgfqpoint{0.100000in}{0.212622in}}{\pgfqpoint{3.696000in}{3.696000in}}%
\pgfusepath{clip}%
\pgfsetbuttcap%
\pgfsetroundjoin%
\definecolor{currentfill}{rgb}{0.121569,0.466667,0.705882}%
\pgfsetfillcolor{currentfill}%
\pgfsetfillopacity{0.349527}%
\pgfsetlinewidth{1.003750pt}%
\definecolor{currentstroke}{rgb}{0.121569,0.466667,0.705882}%
\pgfsetstrokecolor{currentstroke}%
\pgfsetstrokeopacity{0.349527}%
\pgfsetdash{}{0pt}%
\pgfpathmoveto{\pgfqpoint{1.587956in}{2.101719in}}%
\pgfpathcurveto{\pgfqpoint{1.596192in}{2.101719in}}{\pgfqpoint{1.604092in}{2.104991in}}{\pgfqpoint{1.609916in}{2.110815in}}%
\pgfpathcurveto{\pgfqpoint{1.615740in}{2.116639in}}{\pgfqpoint{1.619013in}{2.124539in}}{\pgfqpoint{1.619013in}{2.132775in}}%
\pgfpathcurveto{\pgfqpoint{1.619013in}{2.141011in}}{\pgfqpoint{1.615740in}{2.148911in}}{\pgfqpoint{1.609916in}{2.154735in}}%
\pgfpathcurveto{\pgfqpoint{1.604092in}{2.160559in}}{\pgfqpoint{1.596192in}{2.163832in}}{\pgfqpoint{1.587956in}{2.163832in}}%
\pgfpathcurveto{\pgfqpoint{1.579720in}{2.163832in}}{\pgfqpoint{1.571820in}{2.160559in}}{\pgfqpoint{1.565996in}{2.154735in}}%
\pgfpathcurveto{\pgfqpoint{1.560172in}{2.148911in}}{\pgfqpoint{1.556900in}{2.141011in}}{\pgfqpoint{1.556900in}{2.132775in}}%
\pgfpathcurveto{\pgfqpoint{1.556900in}{2.124539in}}{\pgfqpoint{1.560172in}{2.116639in}}{\pgfqpoint{1.565996in}{2.110815in}}%
\pgfpathcurveto{\pgfqpoint{1.571820in}{2.104991in}}{\pgfqpoint{1.579720in}{2.101719in}}{\pgfqpoint{1.587956in}{2.101719in}}%
\pgfpathclose%
\pgfusepath{stroke,fill}%
\end{pgfscope}%
\begin{pgfscope}%
\pgfpathrectangle{\pgfqpoint{0.100000in}{0.212622in}}{\pgfqpoint{3.696000in}{3.696000in}}%
\pgfusepath{clip}%
\pgfsetbuttcap%
\pgfsetroundjoin%
\definecolor{currentfill}{rgb}{0.121569,0.466667,0.705882}%
\pgfsetfillcolor{currentfill}%
\pgfsetfillopacity{0.350207}%
\pgfsetlinewidth{1.003750pt}%
\definecolor{currentstroke}{rgb}{0.121569,0.466667,0.705882}%
\pgfsetstrokecolor{currentstroke}%
\pgfsetstrokeopacity{0.350207}%
\pgfsetdash{}{0pt}%
\pgfpathmoveto{\pgfqpoint{2.049465in}{2.030259in}}%
\pgfpathcurveto{\pgfqpoint{2.057702in}{2.030259in}}{\pgfqpoint{2.065602in}{2.033532in}}{\pgfqpoint{2.071426in}{2.039356in}}%
\pgfpathcurveto{\pgfqpoint{2.077250in}{2.045179in}}{\pgfqpoint{2.080522in}{2.053079in}}{\pgfqpoint{2.080522in}{2.061316in}}%
\pgfpathcurveto{\pgfqpoint{2.080522in}{2.069552in}}{\pgfqpoint{2.077250in}{2.077452in}}{\pgfqpoint{2.071426in}{2.083276in}}%
\pgfpathcurveto{\pgfqpoint{2.065602in}{2.089100in}}{\pgfqpoint{2.057702in}{2.092372in}}{\pgfqpoint{2.049465in}{2.092372in}}%
\pgfpathcurveto{\pgfqpoint{2.041229in}{2.092372in}}{\pgfqpoint{2.033329in}{2.089100in}}{\pgfqpoint{2.027505in}{2.083276in}}%
\pgfpathcurveto{\pgfqpoint{2.021681in}{2.077452in}}{\pgfqpoint{2.018409in}{2.069552in}}{\pgfqpoint{2.018409in}{2.061316in}}%
\pgfpathcurveto{\pgfqpoint{2.018409in}{2.053079in}}{\pgfqpoint{2.021681in}{2.045179in}}{\pgfqpoint{2.027505in}{2.039356in}}%
\pgfpathcurveto{\pgfqpoint{2.033329in}{2.033532in}}{\pgfqpoint{2.041229in}{2.030259in}}{\pgfqpoint{2.049465in}{2.030259in}}%
\pgfpathclose%
\pgfusepath{stroke,fill}%
\end{pgfscope}%
\begin{pgfscope}%
\pgfpathrectangle{\pgfqpoint{0.100000in}{0.212622in}}{\pgfqpoint{3.696000in}{3.696000in}}%
\pgfusepath{clip}%
\pgfsetbuttcap%
\pgfsetroundjoin%
\definecolor{currentfill}{rgb}{0.121569,0.466667,0.705882}%
\pgfsetfillcolor{currentfill}%
\pgfsetfillopacity{0.350316}%
\pgfsetlinewidth{1.003750pt}%
\definecolor{currentstroke}{rgb}{0.121569,0.466667,0.705882}%
\pgfsetstrokecolor{currentstroke}%
\pgfsetstrokeopacity{0.350316}%
\pgfsetdash{}{0pt}%
\pgfpathmoveto{\pgfqpoint{1.586542in}{2.101776in}}%
\pgfpathcurveto{\pgfqpoint{1.594778in}{2.101776in}}{\pgfqpoint{1.602678in}{2.105049in}}{\pgfqpoint{1.608502in}{2.110873in}}%
\pgfpathcurveto{\pgfqpoint{1.614326in}{2.116696in}}{\pgfqpoint{1.617598in}{2.124597in}}{\pgfqpoint{1.617598in}{2.132833in}}%
\pgfpathcurveto{\pgfqpoint{1.617598in}{2.141069in}}{\pgfqpoint{1.614326in}{2.148969in}}{\pgfqpoint{1.608502in}{2.154793in}}%
\pgfpathcurveto{\pgfqpoint{1.602678in}{2.160617in}}{\pgfqpoint{1.594778in}{2.163889in}}{\pgfqpoint{1.586542in}{2.163889in}}%
\pgfpathcurveto{\pgfqpoint{1.578305in}{2.163889in}}{\pgfqpoint{1.570405in}{2.160617in}}{\pgfqpoint{1.564581in}{2.154793in}}%
\pgfpathcurveto{\pgfqpoint{1.558757in}{2.148969in}}{\pgfqpoint{1.555485in}{2.141069in}}{\pgfqpoint{1.555485in}{2.132833in}}%
\pgfpathcurveto{\pgfqpoint{1.555485in}{2.124597in}}{\pgfqpoint{1.558757in}{2.116696in}}{\pgfqpoint{1.564581in}{2.110873in}}%
\pgfpathcurveto{\pgfqpoint{1.570405in}{2.105049in}}{\pgfqpoint{1.578305in}{2.101776in}}{\pgfqpoint{1.586542in}{2.101776in}}%
\pgfpathclose%
\pgfusepath{stroke,fill}%
\end{pgfscope}%
\begin{pgfscope}%
\pgfpathrectangle{\pgfqpoint{0.100000in}{0.212622in}}{\pgfqpoint{3.696000in}{3.696000in}}%
\pgfusepath{clip}%
\pgfsetbuttcap%
\pgfsetroundjoin%
\definecolor{currentfill}{rgb}{0.121569,0.466667,0.705882}%
\pgfsetfillcolor{currentfill}%
\pgfsetfillopacity{0.350745}%
\pgfsetlinewidth{1.003750pt}%
\definecolor{currentstroke}{rgb}{0.121569,0.466667,0.705882}%
\pgfsetstrokecolor{currentstroke}%
\pgfsetstrokeopacity{0.350745}%
\pgfsetdash{}{0pt}%
\pgfpathmoveto{\pgfqpoint{1.585495in}{2.101832in}}%
\pgfpathcurveto{\pgfqpoint{1.593731in}{2.101832in}}{\pgfqpoint{1.601632in}{2.105104in}}{\pgfqpoint{1.607455in}{2.110928in}}%
\pgfpathcurveto{\pgfqpoint{1.613279in}{2.116752in}}{\pgfqpoint{1.616552in}{2.124652in}}{\pgfqpoint{1.616552in}{2.132889in}}%
\pgfpathcurveto{\pgfqpoint{1.616552in}{2.141125in}}{\pgfqpoint{1.613279in}{2.149025in}}{\pgfqpoint{1.607455in}{2.154849in}}%
\pgfpathcurveto{\pgfqpoint{1.601632in}{2.160673in}}{\pgfqpoint{1.593731in}{2.163945in}}{\pgfqpoint{1.585495in}{2.163945in}}%
\pgfpathcurveto{\pgfqpoint{1.577259in}{2.163945in}}{\pgfqpoint{1.569359in}{2.160673in}}{\pgfqpoint{1.563535in}{2.154849in}}%
\pgfpathcurveto{\pgfqpoint{1.557711in}{2.149025in}}{\pgfqpoint{1.554439in}{2.141125in}}{\pgfqpoint{1.554439in}{2.132889in}}%
\pgfpathcurveto{\pgfqpoint{1.554439in}{2.124652in}}{\pgfqpoint{1.557711in}{2.116752in}}{\pgfqpoint{1.563535in}{2.110928in}}%
\pgfpathcurveto{\pgfqpoint{1.569359in}{2.105104in}}{\pgfqpoint{1.577259in}{2.101832in}}{\pgfqpoint{1.585495in}{2.101832in}}%
\pgfpathclose%
\pgfusepath{stroke,fill}%
\end{pgfscope}%
\begin{pgfscope}%
\pgfpathrectangle{\pgfqpoint{0.100000in}{0.212622in}}{\pgfqpoint{3.696000in}{3.696000in}}%
\pgfusepath{clip}%
\pgfsetbuttcap%
\pgfsetroundjoin%
\definecolor{currentfill}{rgb}{0.121569,0.466667,0.705882}%
\pgfsetfillcolor{currentfill}%
\pgfsetfillopacity{0.351174}%
\pgfsetlinewidth{1.003750pt}%
\definecolor{currentstroke}{rgb}{0.121569,0.466667,0.705882}%
\pgfsetstrokecolor{currentstroke}%
\pgfsetstrokeopacity{0.351174}%
\pgfsetdash{}{0pt}%
\pgfpathmoveto{\pgfqpoint{1.584943in}{2.102062in}}%
\pgfpathcurveto{\pgfqpoint{1.593179in}{2.102062in}}{\pgfqpoint{1.601079in}{2.105335in}}{\pgfqpoint{1.606903in}{2.111159in}}%
\pgfpathcurveto{\pgfqpoint{1.612727in}{2.116983in}}{\pgfqpoint{1.615999in}{2.124883in}}{\pgfqpoint{1.615999in}{2.133119in}}%
\pgfpathcurveto{\pgfqpoint{1.615999in}{2.141355in}}{\pgfqpoint{1.612727in}{2.149255in}}{\pgfqpoint{1.606903in}{2.155079in}}%
\pgfpathcurveto{\pgfqpoint{1.601079in}{2.160903in}}{\pgfqpoint{1.593179in}{2.164175in}}{\pgfqpoint{1.584943in}{2.164175in}}%
\pgfpathcurveto{\pgfqpoint{1.576706in}{2.164175in}}{\pgfqpoint{1.568806in}{2.160903in}}{\pgfqpoint{1.562982in}{2.155079in}}%
\pgfpathcurveto{\pgfqpoint{1.557158in}{2.149255in}}{\pgfqpoint{1.553886in}{2.141355in}}{\pgfqpoint{1.553886in}{2.133119in}}%
\pgfpathcurveto{\pgfqpoint{1.553886in}{2.124883in}}{\pgfqpoint{1.557158in}{2.116983in}}{\pgfqpoint{1.562982in}{2.111159in}}%
\pgfpathcurveto{\pgfqpoint{1.568806in}{2.105335in}}{\pgfqpoint{1.576706in}{2.102062in}}{\pgfqpoint{1.584943in}{2.102062in}}%
\pgfpathclose%
\pgfusepath{stroke,fill}%
\end{pgfscope}%
\begin{pgfscope}%
\pgfpathrectangle{\pgfqpoint{0.100000in}{0.212622in}}{\pgfqpoint{3.696000in}{3.696000in}}%
\pgfusepath{clip}%
\pgfsetbuttcap%
\pgfsetroundjoin%
\definecolor{currentfill}{rgb}{0.121569,0.466667,0.705882}%
\pgfsetfillcolor{currentfill}%
\pgfsetfillopacity{0.351888}%
\pgfsetlinewidth{1.003750pt}%
\definecolor{currentstroke}{rgb}{0.121569,0.466667,0.705882}%
\pgfsetstrokecolor{currentstroke}%
\pgfsetstrokeopacity{0.351888}%
\pgfsetdash{}{0pt}%
\pgfpathmoveto{\pgfqpoint{1.583802in}{2.102066in}}%
\pgfpathcurveto{\pgfqpoint{1.592038in}{2.102066in}}{\pgfqpoint{1.599938in}{2.105338in}}{\pgfqpoint{1.605762in}{2.111162in}}%
\pgfpathcurveto{\pgfqpoint{1.611586in}{2.116986in}}{\pgfqpoint{1.614858in}{2.124886in}}{\pgfqpoint{1.614858in}{2.133122in}}%
\pgfpathcurveto{\pgfqpoint{1.614858in}{2.141358in}}{\pgfqpoint{1.611586in}{2.149258in}}{\pgfqpoint{1.605762in}{2.155082in}}%
\pgfpathcurveto{\pgfqpoint{1.599938in}{2.160906in}}{\pgfqpoint{1.592038in}{2.164179in}}{\pgfqpoint{1.583802in}{2.164179in}}%
\pgfpathcurveto{\pgfqpoint{1.575566in}{2.164179in}}{\pgfqpoint{1.567666in}{2.160906in}}{\pgfqpoint{1.561842in}{2.155082in}}%
\pgfpathcurveto{\pgfqpoint{1.556018in}{2.149258in}}{\pgfqpoint{1.552745in}{2.141358in}}{\pgfqpoint{1.552745in}{2.133122in}}%
\pgfpathcurveto{\pgfqpoint{1.552745in}{2.124886in}}{\pgfqpoint{1.556018in}{2.116986in}}{\pgfqpoint{1.561842in}{2.111162in}}%
\pgfpathcurveto{\pgfqpoint{1.567666in}{2.105338in}}{\pgfqpoint{1.575566in}{2.102066in}}{\pgfqpoint{1.583802in}{2.102066in}}%
\pgfpathclose%
\pgfusepath{stroke,fill}%
\end{pgfscope}%
\begin{pgfscope}%
\pgfpathrectangle{\pgfqpoint{0.100000in}{0.212622in}}{\pgfqpoint{3.696000in}{3.696000in}}%
\pgfusepath{clip}%
\pgfsetbuttcap%
\pgfsetroundjoin%
\definecolor{currentfill}{rgb}{0.121569,0.466667,0.705882}%
\pgfsetfillcolor{currentfill}%
\pgfsetfillopacity{0.352036}%
\pgfsetlinewidth{1.003750pt}%
\definecolor{currentstroke}{rgb}{0.121569,0.466667,0.705882}%
\pgfsetstrokecolor{currentstroke}%
\pgfsetstrokeopacity{0.352036}%
\pgfsetdash{}{0pt}%
\pgfpathmoveto{\pgfqpoint{2.061282in}{2.027713in}}%
\pgfpathcurveto{\pgfqpoint{2.069518in}{2.027713in}}{\pgfqpoint{2.077418in}{2.030985in}}{\pgfqpoint{2.083242in}{2.036809in}}%
\pgfpathcurveto{\pgfqpoint{2.089066in}{2.042633in}}{\pgfqpoint{2.092338in}{2.050533in}}{\pgfqpoint{2.092338in}{2.058769in}}%
\pgfpathcurveto{\pgfqpoint{2.092338in}{2.067006in}}{\pgfqpoint{2.089066in}{2.074906in}}{\pgfqpoint{2.083242in}{2.080730in}}%
\pgfpathcurveto{\pgfqpoint{2.077418in}{2.086554in}}{\pgfqpoint{2.069518in}{2.089826in}}{\pgfqpoint{2.061282in}{2.089826in}}%
\pgfpathcurveto{\pgfqpoint{2.053045in}{2.089826in}}{\pgfqpoint{2.045145in}{2.086554in}}{\pgfqpoint{2.039321in}{2.080730in}}%
\pgfpathcurveto{\pgfqpoint{2.033498in}{2.074906in}}{\pgfqpoint{2.030225in}{2.067006in}}{\pgfqpoint{2.030225in}{2.058769in}}%
\pgfpathcurveto{\pgfqpoint{2.030225in}{2.050533in}}{\pgfqpoint{2.033498in}{2.042633in}}{\pgfqpoint{2.039321in}{2.036809in}}%
\pgfpathcurveto{\pgfqpoint{2.045145in}{2.030985in}}{\pgfqpoint{2.053045in}{2.027713in}}{\pgfqpoint{2.061282in}{2.027713in}}%
\pgfpathclose%
\pgfusepath{stroke,fill}%
\end{pgfscope}%
\begin{pgfscope}%
\pgfpathrectangle{\pgfqpoint{0.100000in}{0.212622in}}{\pgfqpoint{3.696000in}{3.696000in}}%
\pgfusepath{clip}%
\pgfsetbuttcap%
\pgfsetroundjoin%
\definecolor{currentfill}{rgb}{0.121569,0.466667,0.705882}%
\pgfsetfillcolor{currentfill}%
\pgfsetfillopacity{0.353084}%
\pgfsetlinewidth{1.003750pt}%
\definecolor{currentstroke}{rgb}{0.121569,0.466667,0.705882}%
\pgfsetstrokecolor{currentstroke}%
\pgfsetstrokeopacity{0.353084}%
\pgfsetdash{}{0pt}%
\pgfpathmoveto{\pgfqpoint{1.580625in}{2.102420in}}%
\pgfpathcurveto{\pgfqpoint{1.588861in}{2.102420in}}{\pgfqpoint{1.596761in}{2.105692in}}{\pgfqpoint{1.602585in}{2.111516in}}%
\pgfpathcurveto{\pgfqpoint{1.608409in}{2.117340in}}{\pgfqpoint{1.611681in}{2.125240in}}{\pgfqpoint{1.611681in}{2.133477in}}%
\pgfpathcurveto{\pgfqpoint{1.611681in}{2.141713in}}{\pgfqpoint{1.608409in}{2.149613in}}{\pgfqpoint{1.602585in}{2.155437in}}%
\pgfpathcurveto{\pgfqpoint{1.596761in}{2.161261in}}{\pgfqpoint{1.588861in}{2.164533in}}{\pgfqpoint{1.580625in}{2.164533in}}%
\pgfpathcurveto{\pgfqpoint{1.572389in}{2.164533in}}{\pgfqpoint{1.564489in}{2.161261in}}{\pgfqpoint{1.558665in}{2.155437in}}%
\pgfpathcurveto{\pgfqpoint{1.552841in}{2.149613in}}{\pgfqpoint{1.549568in}{2.141713in}}{\pgfqpoint{1.549568in}{2.133477in}}%
\pgfpathcurveto{\pgfqpoint{1.549568in}{2.125240in}}{\pgfqpoint{1.552841in}{2.117340in}}{\pgfqpoint{1.558665in}{2.111516in}}%
\pgfpathcurveto{\pgfqpoint{1.564489in}{2.105692in}}{\pgfqpoint{1.572389in}{2.102420in}}{\pgfqpoint{1.580625in}{2.102420in}}%
\pgfpathclose%
\pgfusepath{stroke,fill}%
\end{pgfscope}%
\begin{pgfscope}%
\pgfpathrectangle{\pgfqpoint{0.100000in}{0.212622in}}{\pgfqpoint{3.696000in}{3.696000in}}%
\pgfusepath{clip}%
\pgfsetbuttcap%
\pgfsetroundjoin%
\definecolor{currentfill}{rgb}{0.121569,0.466667,0.705882}%
\pgfsetfillcolor{currentfill}%
\pgfsetfillopacity{0.353559}%
\pgfsetlinewidth{1.003750pt}%
\definecolor{currentstroke}{rgb}{0.121569,0.466667,0.705882}%
\pgfsetstrokecolor{currentstroke}%
\pgfsetstrokeopacity{0.353559}%
\pgfsetdash{}{0pt}%
\pgfpathmoveto{\pgfqpoint{2.074451in}{2.024579in}}%
\pgfpathcurveto{\pgfqpoint{2.082687in}{2.024579in}}{\pgfqpoint{2.090587in}{2.027851in}}{\pgfqpoint{2.096411in}{2.033675in}}%
\pgfpathcurveto{\pgfqpoint{2.102235in}{2.039499in}}{\pgfqpoint{2.105508in}{2.047399in}}{\pgfqpoint{2.105508in}{2.055635in}}%
\pgfpathcurveto{\pgfqpoint{2.105508in}{2.063871in}}{\pgfqpoint{2.102235in}{2.071771in}}{\pgfqpoint{2.096411in}{2.077595in}}%
\pgfpathcurveto{\pgfqpoint{2.090587in}{2.083419in}}{\pgfqpoint{2.082687in}{2.086692in}}{\pgfqpoint{2.074451in}{2.086692in}}%
\pgfpathcurveto{\pgfqpoint{2.066215in}{2.086692in}}{\pgfqpoint{2.058315in}{2.083419in}}{\pgfqpoint{2.052491in}{2.077595in}}%
\pgfpathcurveto{\pgfqpoint{2.046667in}{2.071771in}}{\pgfqpoint{2.043395in}{2.063871in}}{\pgfqpoint{2.043395in}{2.055635in}}%
\pgfpathcurveto{\pgfqpoint{2.043395in}{2.047399in}}{\pgfqpoint{2.046667in}{2.039499in}}{\pgfqpoint{2.052491in}{2.033675in}}%
\pgfpathcurveto{\pgfqpoint{2.058315in}{2.027851in}}{\pgfqpoint{2.066215in}{2.024579in}}{\pgfqpoint{2.074451in}{2.024579in}}%
\pgfpathclose%
\pgfusepath{stroke,fill}%
\end{pgfscope}%
\begin{pgfscope}%
\pgfpathrectangle{\pgfqpoint{0.100000in}{0.212622in}}{\pgfqpoint{3.696000in}{3.696000in}}%
\pgfusepath{clip}%
\pgfsetbuttcap%
\pgfsetroundjoin%
\definecolor{currentfill}{rgb}{0.121569,0.466667,0.705882}%
\pgfsetfillcolor{currentfill}%
\pgfsetfillopacity{0.353886}%
\pgfsetlinewidth{1.003750pt}%
\definecolor{currentstroke}{rgb}{0.121569,0.466667,0.705882}%
\pgfsetstrokecolor{currentstroke}%
\pgfsetstrokeopacity{0.353886}%
\pgfsetdash{}{0pt}%
\pgfpathmoveto{\pgfqpoint{1.579008in}{2.102322in}}%
\pgfpathcurveto{\pgfqpoint{1.587244in}{2.102322in}}{\pgfqpoint{1.595144in}{2.105594in}}{\pgfqpoint{1.600968in}{2.111418in}}%
\pgfpathcurveto{\pgfqpoint{1.606792in}{2.117242in}}{\pgfqpoint{1.610064in}{2.125142in}}{\pgfqpoint{1.610064in}{2.133378in}}%
\pgfpathcurveto{\pgfqpoint{1.610064in}{2.141615in}}{\pgfqpoint{1.606792in}{2.149515in}}{\pgfqpoint{1.600968in}{2.155339in}}%
\pgfpathcurveto{\pgfqpoint{1.595144in}{2.161163in}}{\pgfqpoint{1.587244in}{2.164435in}}{\pgfqpoint{1.579008in}{2.164435in}}%
\pgfpathcurveto{\pgfqpoint{1.570771in}{2.164435in}}{\pgfqpoint{1.562871in}{2.161163in}}{\pgfqpoint{1.557047in}{2.155339in}}%
\pgfpathcurveto{\pgfqpoint{1.551224in}{2.149515in}}{\pgfqpoint{1.547951in}{2.141615in}}{\pgfqpoint{1.547951in}{2.133378in}}%
\pgfpathcurveto{\pgfqpoint{1.547951in}{2.125142in}}{\pgfqpoint{1.551224in}{2.117242in}}{\pgfqpoint{1.557047in}{2.111418in}}%
\pgfpathcurveto{\pgfqpoint{1.562871in}{2.105594in}}{\pgfqpoint{1.570771in}{2.102322in}}{\pgfqpoint{1.579008in}{2.102322in}}%
\pgfpathclose%
\pgfusepath{stroke,fill}%
\end{pgfscope}%
\begin{pgfscope}%
\pgfpathrectangle{\pgfqpoint{0.100000in}{0.212622in}}{\pgfqpoint{3.696000in}{3.696000in}}%
\pgfusepath{clip}%
\pgfsetbuttcap%
\pgfsetroundjoin%
\definecolor{currentfill}{rgb}{0.121569,0.466667,0.705882}%
\pgfsetfillcolor{currentfill}%
\pgfsetfillopacity{0.354300}%
\pgfsetlinewidth{1.003750pt}%
\definecolor{currentstroke}{rgb}{0.121569,0.466667,0.705882}%
\pgfsetstrokecolor{currentstroke}%
\pgfsetstrokeopacity{0.354300}%
\pgfsetdash{}{0pt}%
\pgfpathmoveto{\pgfqpoint{1.578228in}{2.102364in}}%
\pgfpathcurveto{\pgfqpoint{1.586464in}{2.102364in}}{\pgfqpoint{1.594365in}{2.105636in}}{\pgfqpoint{1.600188in}{2.111460in}}%
\pgfpathcurveto{\pgfqpoint{1.606012in}{2.117284in}}{\pgfqpoint{1.609285in}{2.125184in}}{\pgfqpoint{1.609285in}{2.133420in}}%
\pgfpathcurveto{\pgfqpoint{1.609285in}{2.141657in}}{\pgfqpoint{1.606012in}{2.149557in}}{\pgfqpoint{1.600188in}{2.155381in}}%
\pgfpathcurveto{\pgfqpoint{1.594365in}{2.161204in}}{\pgfqpoint{1.586464in}{2.164477in}}{\pgfqpoint{1.578228in}{2.164477in}}%
\pgfpathcurveto{\pgfqpoint{1.569992in}{2.164477in}}{\pgfqpoint{1.562092in}{2.161204in}}{\pgfqpoint{1.556268in}{2.155381in}}%
\pgfpathcurveto{\pgfqpoint{1.550444in}{2.149557in}}{\pgfqpoint{1.547172in}{2.141657in}}{\pgfqpoint{1.547172in}{2.133420in}}%
\pgfpathcurveto{\pgfqpoint{1.547172in}{2.125184in}}{\pgfqpoint{1.550444in}{2.117284in}}{\pgfqpoint{1.556268in}{2.111460in}}%
\pgfpathcurveto{\pgfqpoint{1.562092in}{2.105636in}}{\pgfqpoint{1.569992in}{2.102364in}}{\pgfqpoint{1.578228in}{2.102364in}}%
\pgfpathclose%
\pgfusepath{stroke,fill}%
\end{pgfscope}%
\begin{pgfscope}%
\pgfpathrectangle{\pgfqpoint{0.100000in}{0.212622in}}{\pgfqpoint{3.696000in}{3.696000in}}%
\pgfusepath{clip}%
\pgfsetbuttcap%
\pgfsetroundjoin%
\definecolor{currentfill}{rgb}{0.121569,0.466667,0.705882}%
\pgfsetfillcolor{currentfill}%
\pgfsetfillopacity{0.355014}%
\pgfsetlinewidth{1.003750pt}%
\definecolor{currentstroke}{rgb}{0.121569,0.466667,0.705882}%
\pgfsetstrokecolor{currentstroke}%
\pgfsetstrokeopacity{0.355014}%
\pgfsetdash{}{0pt}%
\pgfpathmoveto{\pgfqpoint{1.576511in}{2.102446in}}%
\pgfpathcurveto{\pgfqpoint{1.584747in}{2.102446in}}{\pgfqpoint{1.592647in}{2.105718in}}{\pgfqpoint{1.598471in}{2.111542in}}%
\pgfpathcurveto{\pgfqpoint{1.604295in}{2.117366in}}{\pgfqpoint{1.607567in}{2.125266in}}{\pgfqpoint{1.607567in}{2.133502in}}%
\pgfpathcurveto{\pgfqpoint{1.607567in}{2.141738in}}{\pgfqpoint{1.604295in}{2.149638in}}{\pgfqpoint{1.598471in}{2.155462in}}%
\pgfpathcurveto{\pgfqpoint{1.592647in}{2.161286in}}{\pgfqpoint{1.584747in}{2.164559in}}{\pgfqpoint{1.576511in}{2.164559in}}%
\pgfpathcurveto{\pgfqpoint{1.568274in}{2.164559in}}{\pgfqpoint{1.560374in}{2.161286in}}{\pgfqpoint{1.554550in}{2.155462in}}%
\pgfpathcurveto{\pgfqpoint{1.548726in}{2.149638in}}{\pgfqpoint{1.545454in}{2.141738in}}{\pgfqpoint{1.545454in}{2.133502in}}%
\pgfpathcurveto{\pgfqpoint{1.545454in}{2.125266in}}{\pgfqpoint{1.548726in}{2.117366in}}{\pgfqpoint{1.554550in}{2.111542in}}%
\pgfpathcurveto{\pgfqpoint{1.560374in}{2.105718in}}{\pgfqpoint{1.568274in}{2.102446in}}{\pgfqpoint{1.576511in}{2.102446in}}%
\pgfpathclose%
\pgfusepath{stroke,fill}%
\end{pgfscope}%
\begin{pgfscope}%
\pgfpathrectangle{\pgfqpoint{0.100000in}{0.212622in}}{\pgfqpoint{3.696000in}{3.696000in}}%
\pgfusepath{clip}%
\pgfsetbuttcap%
\pgfsetroundjoin%
\definecolor{currentfill}{rgb}{0.121569,0.466667,0.705882}%
\pgfsetfillcolor{currentfill}%
\pgfsetfillopacity{0.355416}%
\pgfsetlinewidth{1.003750pt}%
\definecolor{currentstroke}{rgb}{0.121569,0.466667,0.705882}%
\pgfsetstrokecolor{currentstroke}%
\pgfsetstrokeopacity{0.355416}%
\pgfsetdash{}{0pt}%
\pgfpathmoveto{\pgfqpoint{1.575842in}{2.102459in}}%
\pgfpathcurveto{\pgfqpoint{1.584078in}{2.102459in}}{\pgfqpoint{1.591978in}{2.105732in}}{\pgfqpoint{1.597802in}{2.111555in}}%
\pgfpathcurveto{\pgfqpoint{1.603626in}{2.117379in}}{\pgfqpoint{1.606899in}{2.125279in}}{\pgfqpoint{1.606899in}{2.133516in}}%
\pgfpathcurveto{\pgfqpoint{1.606899in}{2.141752in}}{\pgfqpoint{1.603626in}{2.149652in}}{\pgfqpoint{1.597802in}{2.155476in}}%
\pgfpathcurveto{\pgfqpoint{1.591978in}{2.161300in}}{\pgfqpoint{1.584078in}{2.164572in}}{\pgfqpoint{1.575842in}{2.164572in}}%
\pgfpathcurveto{\pgfqpoint{1.567606in}{2.164572in}}{\pgfqpoint{1.559706in}{2.161300in}}{\pgfqpoint{1.553882in}{2.155476in}}%
\pgfpathcurveto{\pgfqpoint{1.548058in}{2.149652in}}{\pgfqpoint{1.544786in}{2.141752in}}{\pgfqpoint{1.544786in}{2.133516in}}%
\pgfpathcurveto{\pgfqpoint{1.544786in}{2.125279in}}{\pgfqpoint{1.548058in}{2.117379in}}{\pgfqpoint{1.553882in}{2.111555in}}%
\pgfpathcurveto{\pgfqpoint{1.559706in}{2.105732in}}{\pgfqpoint{1.567606in}{2.102459in}}{\pgfqpoint{1.575842in}{2.102459in}}%
\pgfpathclose%
\pgfusepath{stroke,fill}%
\end{pgfscope}%
\begin{pgfscope}%
\pgfpathrectangle{\pgfqpoint{0.100000in}{0.212622in}}{\pgfqpoint{3.696000in}{3.696000in}}%
\pgfusepath{clip}%
\pgfsetbuttcap%
\pgfsetroundjoin%
\definecolor{currentfill}{rgb}{0.121569,0.466667,0.705882}%
\pgfsetfillcolor{currentfill}%
\pgfsetfillopacity{0.355431}%
\pgfsetlinewidth{1.003750pt}%
\definecolor{currentstroke}{rgb}{0.121569,0.466667,0.705882}%
\pgfsetstrokecolor{currentstroke}%
\pgfsetstrokeopacity{0.355431}%
\pgfsetdash{}{0pt}%
\pgfpathmoveto{\pgfqpoint{2.087634in}{2.022121in}}%
\pgfpathcurveto{\pgfqpoint{2.095870in}{2.022121in}}{\pgfqpoint{2.103770in}{2.025393in}}{\pgfqpoint{2.109594in}{2.031217in}}%
\pgfpathcurveto{\pgfqpoint{2.115418in}{2.037041in}}{\pgfqpoint{2.118690in}{2.044941in}}{\pgfqpoint{2.118690in}{2.053178in}}%
\pgfpathcurveto{\pgfqpoint{2.118690in}{2.061414in}}{\pgfqpoint{2.115418in}{2.069314in}}{\pgfqpoint{2.109594in}{2.075138in}}%
\pgfpathcurveto{\pgfqpoint{2.103770in}{2.080962in}}{\pgfqpoint{2.095870in}{2.084234in}}{\pgfqpoint{2.087634in}{2.084234in}}%
\pgfpathcurveto{\pgfqpoint{2.079397in}{2.084234in}}{\pgfqpoint{2.071497in}{2.080962in}}{\pgfqpoint{2.065673in}{2.075138in}}%
\pgfpathcurveto{\pgfqpoint{2.059850in}{2.069314in}}{\pgfqpoint{2.056577in}{2.061414in}}{\pgfqpoint{2.056577in}{2.053178in}}%
\pgfpathcurveto{\pgfqpoint{2.056577in}{2.044941in}}{\pgfqpoint{2.059850in}{2.037041in}}{\pgfqpoint{2.065673in}{2.031217in}}%
\pgfpathcurveto{\pgfqpoint{2.071497in}{2.025393in}}{\pgfqpoint{2.079397in}{2.022121in}}{\pgfqpoint{2.087634in}{2.022121in}}%
\pgfpathclose%
\pgfusepath{stroke,fill}%
\end{pgfscope}%
\begin{pgfscope}%
\pgfpathrectangle{\pgfqpoint{0.100000in}{0.212622in}}{\pgfqpoint{3.696000in}{3.696000in}}%
\pgfusepath{clip}%
\pgfsetbuttcap%
\pgfsetroundjoin%
\definecolor{currentfill}{rgb}{0.121569,0.466667,0.705882}%
\pgfsetfillcolor{currentfill}%
\pgfsetfillopacity{0.356114}%
\pgfsetlinewidth{1.003750pt}%
\definecolor{currentstroke}{rgb}{0.121569,0.466667,0.705882}%
\pgfsetstrokecolor{currentstroke}%
\pgfsetstrokeopacity{0.356114}%
\pgfsetdash{}{0pt}%
\pgfpathmoveto{\pgfqpoint{1.574292in}{2.102550in}}%
\pgfpathcurveto{\pgfqpoint{1.582528in}{2.102550in}}{\pgfqpoint{1.590428in}{2.105822in}}{\pgfqpoint{1.596252in}{2.111646in}}%
\pgfpathcurveto{\pgfqpoint{1.602076in}{2.117470in}}{\pgfqpoint{1.605348in}{2.125370in}}{\pgfqpoint{1.605348in}{2.133607in}}%
\pgfpathcurveto{\pgfqpoint{1.605348in}{2.141843in}}{\pgfqpoint{1.602076in}{2.149743in}}{\pgfqpoint{1.596252in}{2.155567in}}%
\pgfpathcurveto{\pgfqpoint{1.590428in}{2.161391in}}{\pgfqpoint{1.582528in}{2.164663in}}{\pgfqpoint{1.574292in}{2.164663in}}%
\pgfpathcurveto{\pgfqpoint{1.566055in}{2.164663in}}{\pgfqpoint{1.558155in}{2.161391in}}{\pgfqpoint{1.552331in}{2.155567in}}%
\pgfpathcurveto{\pgfqpoint{1.546507in}{2.149743in}}{\pgfqpoint{1.543235in}{2.141843in}}{\pgfqpoint{1.543235in}{2.133607in}}%
\pgfpathcurveto{\pgfqpoint{1.543235in}{2.125370in}}{\pgfqpoint{1.546507in}{2.117470in}}{\pgfqpoint{1.552331in}{2.111646in}}%
\pgfpathcurveto{\pgfqpoint{1.558155in}{2.105822in}}{\pgfqpoint{1.566055in}{2.102550in}}{\pgfqpoint{1.574292in}{2.102550in}}%
\pgfpathclose%
\pgfusepath{stroke,fill}%
\end{pgfscope}%
\begin{pgfscope}%
\pgfpathrectangle{\pgfqpoint{0.100000in}{0.212622in}}{\pgfqpoint{3.696000in}{3.696000in}}%
\pgfusepath{clip}%
\pgfsetbuttcap%
\pgfsetroundjoin%
\definecolor{currentfill}{rgb}{0.121569,0.466667,0.705882}%
\pgfsetfillcolor{currentfill}%
\pgfsetfillopacity{0.356726}%
\pgfsetlinewidth{1.003750pt}%
\definecolor{currentstroke}{rgb}{0.121569,0.466667,0.705882}%
\pgfsetstrokecolor{currentstroke}%
\pgfsetstrokeopacity{0.356726}%
\pgfsetdash{}{0pt}%
\pgfpathmoveto{\pgfqpoint{1.572909in}{2.102584in}}%
\pgfpathcurveto{\pgfqpoint{1.581145in}{2.102584in}}{\pgfqpoint{1.589045in}{2.105856in}}{\pgfqpoint{1.594869in}{2.111680in}}%
\pgfpathcurveto{\pgfqpoint{1.600693in}{2.117504in}}{\pgfqpoint{1.603966in}{2.125404in}}{\pgfqpoint{1.603966in}{2.133640in}}%
\pgfpathcurveto{\pgfqpoint{1.603966in}{2.141877in}}{\pgfqpoint{1.600693in}{2.149777in}}{\pgfqpoint{1.594869in}{2.155601in}}%
\pgfpathcurveto{\pgfqpoint{1.589045in}{2.161425in}}{\pgfqpoint{1.581145in}{2.164697in}}{\pgfqpoint{1.572909in}{2.164697in}}%
\pgfpathcurveto{\pgfqpoint{1.564673in}{2.164697in}}{\pgfqpoint{1.556773in}{2.161425in}}{\pgfqpoint{1.550949in}{2.155601in}}%
\pgfpathcurveto{\pgfqpoint{1.545125in}{2.149777in}}{\pgfqpoint{1.541853in}{2.141877in}}{\pgfqpoint{1.541853in}{2.133640in}}%
\pgfpathcurveto{\pgfqpoint{1.541853in}{2.125404in}}{\pgfqpoint{1.545125in}{2.117504in}}{\pgfqpoint{1.550949in}{2.111680in}}%
\pgfpathcurveto{\pgfqpoint{1.556773in}{2.105856in}}{\pgfqpoint{1.564673in}{2.102584in}}{\pgfqpoint{1.572909in}{2.102584in}}%
\pgfpathclose%
\pgfusepath{stroke,fill}%
\end{pgfscope}%
\begin{pgfscope}%
\pgfpathrectangle{\pgfqpoint{0.100000in}{0.212622in}}{\pgfqpoint{3.696000in}{3.696000in}}%
\pgfusepath{clip}%
\pgfsetbuttcap%
\pgfsetroundjoin%
\definecolor{currentfill}{rgb}{0.121569,0.466667,0.705882}%
\pgfsetfillcolor{currentfill}%
\pgfsetfillopacity{0.357052}%
\pgfsetlinewidth{1.003750pt}%
\definecolor{currentstroke}{rgb}{0.121569,0.466667,0.705882}%
\pgfsetstrokecolor{currentstroke}%
\pgfsetstrokeopacity{0.357052}%
\pgfsetdash{}{0pt}%
\pgfpathmoveto{\pgfqpoint{2.102638in}{2.018830in}}%
\pgfpathcurveto{\pgfqpoint{2.110875in}{2.018830in}}{\pgfqpoint{2.118775in}{2.022102in}}{\pgfqpoint{2.124599in}{2.027926in}}%
\pgfpathcurveto{\pgfqpoint{2.130423in}{2.033750in}}{\pgfqpoint{2.133695in}{2.041650in}}{\pgfqpoint{2.133695in}{2.049886in}}%
\pgfpathcurveto{\pgfqpoint{2.133695in}{2.058123in}}{\pgfqpoint{2.130423in}{2.066023in}}{\pgfqpoint{2.124599in}{2.071847in}}%
\pgfpathcurveto{\pgfqpoint{2.118775in}{2.077671in}}{\pgfqpoint{2.110875in}{2.080943in}}{\pgfqpoint{2.102638in}{2.080943in}}%
\pgfpathcurveto{\pgfqpoint{2.094402in}{2.080943in}}{\pgfqpoint{2.086502in}{2.077671in}}{\pgfqpoint{2.080678in}{2.071847in}}%
\pgfpathcurveto{\pgfqpoint{2.074854in}{2.066023in}}{\pgfqpoint{2.071582in}{2.058123in}}{\pgfqpoint{2.071582in}{2.049886in}}%
\pgfpathcurveto{\pgfqpoint{2.071582in}{2.041650in}}{\pgfqpoint{2.074854in}{2.033750in}}{\pgfqpoint{2.080678in}{2.027926in}}%
\pgfpathcurveto{\pgfqpoint{2.086502in}{2.022102in}}{\pgfqpoint{2.094402in}{2.018830in}}{\pgfqpoint{2.102638in}{2.018830in}}%
\pgfpathclose%
\pgfusepath{stroke,fill}%
\end{pgfscope}%
\begin{pgfscope}%
\pgfpathrectangle{\pgfqpoint{0.100000in}{0.212622in}}{\pgfqpoint{3.696000in}{3.696000in}}%
\pgfusepath{clip}%
\pgfsetbuttcap%
\pgfsetroundjoin%
\definecolor{currentfill}{rgb}{0.121569,0.466667,0.705882}%
\pgfsetfillcolor{currentfill}%
\pgfsetfillopacity{0.357290}%
\pgfsetlinewidth{1.003750pt}%
\definecolor{currentstroke}{rgb}{0.121569,0.466667,0.705882}%
\pgfsetstrokecolor{currentstroke}%
\pgfsetstrokeopacity{0.357290}%
\pgfsetdash{}{0pt}%
\pgfpathmoveto{\pgfqpoint{1.572131in}{2.102626in}}%
\pgfpathcurveto{\pgfqpoint{1.580367in}{2.102626in}}{\pgfqpoint{1.588267in}{2.105898in}}{\pgfqpoint{1.594091in}{2.111722in}}%
\pgfpathcurveto{\pgfqpoint{1.599915in}{2.117546in}}{\pgfqpoint{1.603187in}{2.125446in}}{\pgfqpoint{1.603187in}{2.133682in}}%
\pgfpathcurveto{\pgfqpoint{1.603187in}{2.141919in}}{\pgfqpoint{1.599915in}{2.149819in}}{\pgfqpoint{1.594091in}{2.155643in}}%
\pgfpathcurveto{\pgfqpoint{1.588267in}{2.161466in}}{\pgfqpoint{1.580367in}{2.164739in}}{\pgfqpoint{1.572131in}{2.164739in}}%
\pgfpathcurveto{\pgfqpoint{1.563895in}{2.164739in}}{\pgfqpoint{1.555995in}{2.161466in}}{\pgfqpoint{1.550171in}{2.155643in}}%
\pgfpathcurveto{\pgfqpoint{1.544347in}{2.149819in}}{\pgfqpoint{1.541074in}{2.141919in}}{\pgfqpoint{1.541074in}{2.133682in}}%
\pgfpathcurveto{\pgfqpoint{1.541074in}{2.125446in}}{\pgfqpoint{1.544347in}{2.117546in}}{\pgfqpoint{1.550171in}{2.111722in}}%
\pgfpathcurveto{\pgfqpoint{1.555995in}{2.105898in}}{\pgfqpoint{1.563895in}{2.102626in}}{\pgfqpoint{1.572131in}{2.102626in}}%
\pgfpathclose%
\pgfusepath{stroke,fill}%
\end{pgfscope}%
\begin{pgfscope}%
\pgfpathrectangle{\pgfqpoint{0.100000in}{0.212622in}}{\pgfqpoint{3.696000in}{3.696000in}}%
\pgfusepath{clip}%
\pgfsetbuttcap%
\pgfsetroundjoin%
\definecolor{currentfill}{rgb}{0.121569,0.466667,0.705882}%
\pgfsetfillcolor{currentfill}%
\pgfsetfillopacity{0.357629}%
\pgfsetlinewidth{1.003750pt}%
\definecolor{currentstroke}{rgb}{0.121569,0.466667,0.705882}%
\pgfsetstrokecolor{currentstroke}%
\pgfsetstrokeopacity{0.357629}%
\pgfsetdash{}{0pt}%
\pgfpathmoveto{\pgfqpoint{1.571305in}{2.102697in}}%
\pgfpathcurveto{\pgfqpoint{1.579541in}{2.102697in}}{\pgfqpoint{1.587441in}{2.105969in}}{\pgfqpoint{1.593265in}{2.111793in}}%
\pgfpathcurveto{\pgfqpoint{1.599089in}{2.117617in}}{\pgfqpoint{1.602361in}{2.125517in}}{\pgfqpoint{1.602361in}{2.133753in}}%
\pgfpathcurveto{\pgfqpoint{1.602361in}{2.141990in}}{\pgfqpoint{1.599089in}{2.149890in}}{\pgfqpoint{1.593265in}{2.155714in}}%
\pgfpathcurveto{\pgfqpoint{1.587441in}{2.161538in}}{\pgfqpoint{1.579541in}{2.164810in}}{\pgfqpoint{1.571305in}{2.164810in}}%
\pgfpathcurveto{\pgfqpoint{1.563069in}{2.164810in}}{\pgfqpoint{1.555169in}{2.161538in}}{\pgfqpoint{1.549345in}{2.155714in}}%
\pgfpathcurveto{\pgfqpoint{1.543521in}{2.149890in}}{\pgfqpoint{1.540248in}{2.141990in}}{\pgfqpoint{1.540248in}{2.133753in}}%
\pgfpathcurveto{\pgfqpoint{1.540248in}{2.125517in}}{\pgfqpoint{1.543521in}{2.117617in}}{\pgfqpoint{1.549345in}{2.111793in}}%
\pgfpathcurveto{\pgfqpoint{1.555169in}{2.105969in}}{\pgfqpoint{1.563069in}{2.102697in}}{\pgfqpoint{1.571305in}{2.102697in}}%
\pgfpathclose%
\pgfusepath{stroke,fill}%
\end{pgfscope}%
\begin{pgfscope}%
\pgfpathrectangle{\pgfqpoint{0.100000in}{0.212622in}}{\pgfqpoint{3.696000in}{3.696000in}}%
\pgfusepath{clip}%
\pgfsetbuttcap%
\pgfsetroundjoin%
\definecolor{currentfill}{rgb}{0.121569,0.466667,0.705882}%
\pgfsetfillcolor{currentfill}%
\pgfsetfillopacity{0.358283}%
\pgfsetlinewidth{1.003750pt}%
\definecolor{currentstroke}{rgb}{0.121569,0.466667,0.705882}%
\pgfsetstrokecolor{currentstroke}%
\pgfsetstrokeopacity{0.358283}%
\pgfsetdash{}{0pt}%
\pgfpathmoveto{\pgfqpoint{1.570201in}{2.102711in}}%
\pgfpathcurveto{\pgfqpoint{1.578437in}{2.102711in}}{\pgfqpoint{1.586337in}{2.105983in}}{\pgfqpoint{1.592161in}{2.111807in}}%
\pgfpathcurveto{\pgfqpoint{1.597985in}{2.117631in}}{\pgfqpoint{1.601258in}{2.125531in}}{\pgfqpoint{1.601258in}{2.133768in}}%
\pgfpathcurveto{\pgfqpoint{1.601258in}{2.142004in}}{\pgfqpoint{1.597985in}{2.149904in}}{\pgfqpoint{1.592161in}{2.155728in}}%
\pgfpathcurveto{\pgfqpoint{1.586337in}{2.161552in}}{\pgfqpoint{1.578437in}{2.164824in}}{\pgfqpoint{1.570201in}{2.164824in}}%
\pgfpathcurveto{\pgfqpoint{1.561965in}{2.164824in}}{\pgfqpoint{1.554065in}{2.161552in}}{\pgfqpoint{1.548241in}{2.155728in}}%
\pgfpathcurveto{\pgfqpoint{1.542417in}{2.149904in}}{\pgfqpoint{1.539145in}{2.142004in}}{\pgfqpoint{1.539145in}{2.133768in}}%
\pgfpathcurveto{\pgfqpoint{1.539145in}{2.125531in}}{\pgfqpoint{1.542417in}{2.117631in}}{\pgfqpoint{1.548241in}{2.111807in}}%
\pgfpathcurveto{\pgfqpoint{1.554065in}{2.105983in}}{\pgfqpoint{1.561965in}{2.102711in}}{\pgfqpoint{1.570201in}{2.102711in}}%
\pgfpathclose%
\pgfusepath{stroke,fill}%
\end{pgfscope}%
\begin{pgfscope}%
\pgfpathrectangle{\pgfqpoint{0.100000in}{0.212622in}}{\pgfqpoint{3.696000in}{3.696000in}}%
\pgfusepath{clip}%
\pgfsetbuttcap%
\pgfsetroundjoin%
\definecolor{currentfill}{rgb}{0.121569,0.466667,0.705882}%
\pgfsetfillcolor{currentfill}%
\pgfsetfillopacity{0.358667}%
\pgfsetlinewidth{1.003750pt}%
\definecolor{currentstroke}{rgb}{0.121569,0.466667,0.705882}%
\pgfsetstrokecolor{currentstroke}%
\pgfsetstrokeopacity{0.358667}%
\pgfsetdash{}{0pt}%
\pgfpathmoveto{\pgfqpoint{2.109392in}{2.018513in}}%
\pgfpathcurveto{\pgfqpoint{2.117628in}{2.018513in}}{\pgfqpoint{2.125528in}{2.021785in}}{\pgfqpoint{2.131352in}{2.027609in}}%
\pgfpathcurveto{\pgfqpoint{2.137176in}{2.033433in}}{\pgfqpoint{2.140449in}{2.041333in}}{\pgfqpoint{2.140449in}{2.049570in}}%
\pgfpathcurveto{\pgfqpoint{2.140449in}{2.057806in}}{\pgfqpoint{2.137176in}{2.065706in}}{\pgfqpoint{2.131352in}{2.071530in}}%
\pgfpathcurveto{\pgfqpoint{2.125528in}{2.077354in}}{\pgfqpoint{2.117628in}{2.080626in}}{\pgfqpoint{2.109392in}{2.080626in}}%
\pgfpathcurveto{\pgfqpoint{2.101156in}{2.080626in}}{\pgfqpoint{2.093256in}{2.077354in}}{\pgfqpoint{2.087432in}{2.071530in}}%
\pgfpathcurveto{\pgfqpoint{2.081608in}{2.065706in}}{\pgfqpoint{2.078336in}{2.057806in}}{\pgfqpoint{2.078336in}{2.049570in}}%
\pgfpathcurveto{\pgfqpoint{2.078336in}{2.041333in}}{\pgfqpoint{2.081608in}{2.033433in}}{\pgfqpoint{2.087432in}{2.027609in}}%
\pgfpathcurveto{\pgfqpoint{2.093256in}{2.021785in}}{\pgfqpoint{2.101156in}{2.018513in}}{\pgfqpoint{2.109392in}{2.018513in}}%
\pgfpathclose%
\pgfusepath{stroke,fill}%
\end{pgfscope}%
\begin{pgfscope}%
\pgfpathrectangle{\pgfqpoint{0.100000in}{0.212622in}}{\pgfqpoint{3.696000in}{3.696000in}}%
\pgfusepath{clip}%
\pgfsetbuttcap%
\pgfsetroundjoin%
\definecolor{currentfill}{rgb}{0.121569,0.466667,0.705882}%
\pgfsetfillcolor{currentfill}%
\pgfsetfillopacity{0.359481}%
\pgfsetlinewidth{1.003750pt}%
\definecolor{currentstroke}{rgb}{0.121569,0.466667,0.705882}%
\pgfsetstrokecolor{currentstroke}%
\pgfsetstrokeopacity{0.359481}%
\pgfsetdash{}{0pt}%
\pgfpathmoveto{\pgfqpoint{1.568245in}{2.102761in}}%
\pgfpathcurveto{\pgfqpoint{1.576481in}{2.102761in}}{\pgfqpoint{1.584381in}{2.106033in}}{\pgfqpoint{1.590205in}{2.111857in}}%
\pgfpathcurveto{\pgfqpoint{1.596029in}{2.117681in}}{\pgfqpoint{1.599301in}{2.125581in}}{\pgfqpoint{1.599301in}{2.133818in}}%
\pgfpathcurveto{\pgfqpoint{1.599301in}{2.142054in}}{\pgfqpoint{1.596029in}{2.149954in}}{\pgfqpoint{1.590205in}{2.155778in}}%
\pgfpathcurveto{\pgfqpoint{1.584381in}{2.161602in}}{\pgfqpoint{1.576481in}{2.164874in}}{\pgfqpoint{1.568245in}{2.164874in}}%
\pgfpathcurveto{\pgfqpoint{1.560008in}{2.164874in}}{\pgfqpoint{1.552108in}{2.161602in}}{\pgfqpoint{1.546285in}{2.155778in}}%
\pgfpathcurveto{\pgfqpoint{1.540461in}{2.149954in}}{\pgfqpoint{1.537188in}{2.142054in}}{\pgfqpoint{1.537188in}{2.133818in}}%
\pgfpathcurveto{\pgfqpoint{1.537188in}{2.125581in}}{\pgfqpoint{1.540461in}{2.117681in}}{\pgfqpoint{1.546285in}{2.111857in}}%
\pgfpathcurveto{\pgfqpoint{1.552108in}{2.106033in}}{\pgfqpoint{1.560008in}{2.102761in}}{\pgfqpoint{1.568245in}{2.102761in}}%
\pgfpathclose%
\pgfusepath{stroke,fill}%
\end{pgfscope}%
\begin{pgfscope}%
\pgfpathrectangle{\pgfqpoint{0.100000in}{0.212622in}}{\pgfqpoint{3.696000in}{3.696000in}}%
\pgfusepath{clip}%
\pgfsetbuttcap%
\pgfsetroundjoin%
\definecolor{currentfill}{rgb}{0.121569,0.466667,0.705882}%
\pgfsetfillcolor{currentfill}%
\pgfsetfillopacity{0.359823}%
\pgfsetlinewidth{1.003750pt}%
\definecolor{currentstroke}{rgb}{0.121569,0.466667,0.705882}%
\pgfsetstrokecolor{currentstroke}%
\pgfsetstrokeopacity{0.359823}%
\pgfsetdash{}{0pt}%
\pgfpathmoveto{\pgfqpoint{2.118397in}{2.016484in}}%
\pgfpathcurveto{\pgfqpoint{2.126633in}{2.016484in}}{\pgfqpoint{2.134533in}{2.019756in}}{\pgfqpoint{2.140357in}{2.025580in}}%
\pgfpathcurveto{\pgfqpoint{2.146181in}{2.031404in}}{\pgfqpoint{2.149453in}{2.039304in}}{\pgfqpoint{2.149453in}{2.047541in}}%
\pgfpathcurveto{\pgfqpoint{2.149453in}{2.055777in}}{\pgfqpoint{2.146181in}{2.063677in}}{\pgfqpoint{2.140357in}{2.069501in}}%
\pgfpathcurveto{\pgfqpoint{2.134533in}{2.075325in}}{\pgfqpoint{2.126633in}{2.078597in}}{\pgfqpoint{2.118397in}{2.078597in}}%
\pgfpathcurveto{\pgfqpoint{2.110161in}{2.078597in}}{\pgfqpoint{2.102261in}{2.075325in}}{\pgfqpoint{2.096437in}{2.069501in}}%
\pgfpathcurveto{\pgfqpoint{2.090613in}{2.063677in}}{\pgfqpoint{2.087340in}{2.055777in}}{\pgfqpoint{2.087340in}{2.047541in}}%
\pgfpathcurveto{\pgfqpoint{2.087340in}{2.039304in}}{\pgfqpoint{2.090613in}{2.031404in}}{\pgfqpoint{2.096437in}{2.025580in}}%
\pgfpathcurveto{\pgfqpoint{2.102261in}{2.019756in}}{\pgfqpoint{2.110161in}{2.016484in}}{\pgfqpoint{2.118397in}{2.016484in}}%
\pgfpathclose%
\pgfusepath{stroke,fill}%
\end{pgfscope}%
\begin{pgfscope}%
\pgfpathrectangle{\pgfqpoint{0.100000in}{0.212622in}}{\pgfqpoint{3.696000in}{3.696000in}}%
\pgfusepath{clip}%
\pgfsetbuttcap%
\pgfsetroundjoin%
\definecolor{currentfill}{rgb}{0.121569,0.466667,0.705882}%
\pgfsetfillcolor{currentfill}%
\pgfsetfillopacity{0.360395}%
\pgfsetlinewidth{1.003750pt}%
\definecolor{currentstroke}{rgb}{0.121569,0.466667,0.705882}%
\pgfsetstrokecolor{currentstroke}%
\pgfsetstrokeopacity{0.360395}%
\pgfsetdash{}{0pt}%
\pgfpathmoveto{\pgfqpoint{1.565819in}{2.103033in}}%
\pgfpathcurveto{\pgfqpoint{1.574055in}{2.103033in}}{\pgfqpoint{1.581955in}{2.106305in}}{\pgfqpoint{1.587779in}{2.112129in}}%
\pgfpathcurveto{\pgfqpoint{1.593603in}{2.117953in}}{\pgfqpoint{1.596875in}{2.125853in}}{\pgfqpoint{1.596875in}{2.134090in}}%
\pgfpathcurveto{\pgfqpoint{1.596875in}{2.142326in}}{\pgfqpoint{1.593603in}{2.150226in}}{\pgfqpoint{1.587779in}{2.156050in}}%
\pgfpathcurveto{\pgfqpoint{1.581955in}{2.161874in}}{\pgfqpoint{1.574055in}{2.165146in}}{\pgfqpoint{1.565819in}{2.165146in}}%
\pgfpathcurveto{\pgfqpoint{1.557583in}{2.165146in}}{\pgfqpoint{1.549683in}{2.161874in}}{\pgfqpoint{1.543859in}{2.156050in}}%
\pgfpathcurveto{\pgfqpoint{1.538035in}{2.150226in}}{\pgfqpoint{1.534762in}{2.142326in}}{\pgfqpoint{1.534762in}{2.134090in}}%
\pgfpathcurveto{\pgfqpoint{1.534762in}{2.125853in}}{\pgfqpoint{1.538035in}{2.117953in}}{\pgfqpoint{1.543859in}{2.112129in}}%
\pgfpathcurveto{\pgfqpoint{1.549683in}{2.106305in}}{\pgfqpoint{1.557583in}{2.103033in}}{\pgfqpoint{1.565819in}{2.103033in}}%
\pgfpathclose%
\pgfusepath{stroke,fill}%
\end{pgfscope}%
\begin{pgfscope}%
\pgfpathrectangle{\pgfqpoint{0.100000in}{0.212622in}}{\pgfqpoint{3.696000in}{3.696000in}}%
\pgfusepath{clip}%
\pgfsetbuttcap%
\pgfsetroundjoin%
\definecolor{currentfill}{rgb}{0.121569,0.466667,0.705882}%
\pgfsetfillcolor{currentfill}%
\pgfsetfillopacity{0.360926}%
\pgfsetlinewidth{1.003750pt}%
\definecolor{currentstroke}{rgb}{0.121569,0.466667,0.705882}%
\pgfsetstrokecolor{currentstroke}%
\pgfsetstrokeopacity{0.360926}%
\pgfsetdash{}{0pt}%
\pgfpathmoveto{\pgfqpoint{2.128236in}{2.013492in}}%
\pgfpathcurveto{\pgfqpoint{2.136473in}{2.013492in}}{\pgfqpoint{2.144373in}{2.016764in}}{\pgfqpoint{2.150197in}{2.022588in}}%
\pgfpathcurveto{\pgfqpoint{2.156021in}{2.028412in}}{\pgfqpoint{2.159293in}{2.036312in}}{\pgfqpoint{2.159293in}{2.044549in}}%
\pgfpathcurveto{\pgfqpoint{2.159293in}{2.052785in}}{\pgfqpoint{2.156021in}{2.060685in}}{\pgfqpoint{2.150197in}{2.066509in}}%
\pgfpathcurveto{\pgfqpoint{2.144373in}{2.072333in}}{\pgfqpoint{2.136473in}{2.075605in}}{\pgfqpoint{2.128236in}{2.075605in}}%
\pgfpathcurveto{\pgfqpoint{2.120000in}{2.075605in}}{\pgfqpoint{2.112100in}{2.072333in}}{\pgfqpoint{2.106276in}{2.066509in}}%
\pgfpathcurveto{\pgfqpoint{2.100452in}{2.060685in}}{\pgfqpoint{2.097180in}{2.052785in}}{\pgfqpoint{2.097180in}{2.044549in}}%
\pgfpathcurveto{\pgfqpoint{2.097180in}{2.036312in}}{\pgfqpoint{2.100452in}{2.028412in}}{\pgfqpoint{2.106276in}{2.022588in}}%
\pgfpathcurveto{\pgfqpoint{2.112100in}{2.016764in}}{\pgfqpoint{2.120000in}{2.013492in}}{\pgfqpoint{2.128236in}{2.013492in}}%
\pgfpathclose%
\pgfusepath{stroke,fill}%
\end{pgfscope}%
\begin{pgfscope}%
\pgfpathrectangle{\pgfqpoint{0.100000in}{0.212622in}}{\pgfqpoint{3.696000in}{3.696000in}}%
\pgfusepath{clip}%
\pgfsetbuttcap%
\pgfsetroundjoin%
\definecolor{currentfill}{rgb}{0.121569,0.466667,0.705882}%
\pgfsetfillcolor{currentfill}%
\pgfsetfillopacity{0.361135}%
\pgfsetlinewidth{1.003750pt}%
\definecolor{currentstroke}{rgb}{0.121569,0.466667,0.705882}%
\pgfsetstrokecolor{currentstroke}%
\pgfsetstrokeopacity{0.361135}%
\pgfsetdash{}{0pt}%
\pgfpathmoveto{\pgfqpoint{1.564791in}{2.102932in}}%
\pgfpathcurveto{\pgfqpoint{1.573028in}{2.102932in}}{\pgfqpoint{1.580928in}{2.106205in}}{\pgfqpoint{1.586752in}{2.112029in}}%
\pgfpathcurveto{\pgfqpoint{1.592576in}{2.117853in}}{\pgfqpoint{1.595848in}{2.125753in}}{\pgfqpoint{1.595848in}{2.133989in}}%
\pgfpathcurveto{\pgfqpoint{1.595848in}{2.142225in}}{\pgfqpoint{1.592576in}{2.150125in}}{\pgfqpoint{1.586752in}{2.155949in}}%
\pgfpathcurveto{\pgfqpoint{1.580928in}{2.161773in}}{\pgfqpoint{1.573028in}{2.165045in}}{\pgfqpoint{1.564791in}{2.165045in}}%
\pgfpathcurveto{\pgfqpoint{1.556555in}{2.165045in}}{\pgfqpoint{1.548655in}{2.161773in}}{\pgfqpoint{1.542831in}{2.155949in}}%
\pgfpathcurveto{\pgfqpoint{1.537007in}{2.150125in}}{\pgfqpoint{1.533735in}{2.142225in}}{\pgfqpoint{1.533735in}{2.133989in}}%
\pgfpathcurveto{\pgfqpoint{1.533735in}{2.125753in}}{\pgfqpoint{1.537007in}{2.117853in}}{\pgfqpoint{1.542831in}{2.112029in}}%
\pgfpathcurveto{\pgfqpoint{1.548655in}{2.106205in}}{\pgfqpoint{1.556555in}{2.102932in}}{\pgfqpoint{1.564791in}{2.102932in}}%
\pgfpathclose%
\pgfusepath{stroke,fill}%
\end{pgfscope}%
\begin{pgfscope}%
\pgfpathrectangle{\pgfqpoint{0.100000in}{0.212622in}}{\pgfqpoint{3.696000in}{3.696000in}}%
\pgfusepath{clip}%
\pgfsetbuttcap%
\pgfsetroundjoin%
\definecolor{currentfill}{rgb}{0.121569,0.466667,0.705882}%
\pgfsetfillcolor{currentfill}%
\pgfsetfillopacity{0.361766}%
\pgfsetlinewidth{1.003750pt}%
\definecolor{currentstroke}{rgb}{0.121569,0.466667,0.705882}%
\pgfsetstrokecolor{currentstroke}%
\pgfsetstrokeopacity{0.361766}%
\pgfsetdash{}{0pt}%
\pgfpathmoveto{\pgfqpoint{1.563462in}{2.102999in}}%
\pgfpathcurveto{\pgfqpoint{1.571698in}{2.102999in}}{\pgfqpoint{1.579598in}{2.106272in}}{\pgfqpoint{1.585422in}{2.112095in}}%
\pgfpathcurveto{\pgfqpoint{1.591246in}{2.117919in}}{\pgfqpoint{1.594519in}{2.125819in}}{\pgfqpoint{1.594519in}{2.134056in}}%
\pgfpathcurveto{\pgfqpoint{1.594519in}{2.142292in}}{\pgfqpoint{1.591246in}{2.150192in}}{\pgfqpoint{1.585422in}{2.156016in}}%
\pgfpathcurveto{\pgfqpoint{1.579598in}{2.161840in}}{\pgfqpoint{1.571698in}{2.165112in}}{\pgfqpoint{1.563462in}{2.165112in}}%
\pgfpathcurveto{\pgfqpoint{1.555226in}{2.165112in}}{\pgfqpoint{1.547326in}{2.161840in}}{\pgfqpoint{1.541502in}{2.156016in}}%
\pgfpathcurveto{\pgfqpoint{1.535678in}{2.150192in}}{\pgfqpoint{1.532406in}{2.142292in}}{\pgfqpoint{1.532406in}{2.134056in}}%
\pgfpathcurveto{\pgfqpoint{1.532406in}{2.125819in}}{\pgfqpoint{1.535678in}{2.117919in}}{\pgfqpoint{1.541502in}{2.112095in}}%
\pgfpathcurveto{\pgfqpoint{1.547326in}{2.106272in}}{\pgfqpoint{1.555226in}{2.102999in}}{\pgfqpoint{1.563462in}{2.102999in}}%
\pgfpathclose%
\pgfusepath{stroke,fill}%
\end{pgfscope}%
\begin{pgfscope}%
\pgfpathrectangle{\pgfqpoint{0.100000in}{0.212622in}}{\pgfqpoint{3.696000in}{3.696000in}}%
\pgfusepath{clip}%
\pgfsetbuttcap%
\pgfsetroundjoin%
\definecolor{currentfill}{rgb}{0.121569,0.466667,0.705882}%
\pgfsetfillcolor{currentfill}%
\pgfsetfillopacity{0.362291}%
\pgfsetlinewidth{1.003750pt}%
\definecolor{currentstroke}{rgb}{0.121569,0.466667,0.705882}%
\pgfsetstrokecolor{currentstroke}%
\pgfsetstrokeopacity{0.362291}%
\pgfsetdash{}{0pt}%
\pgfpathmoveto{\pgfqpoint{1.562280in}{2.103060in}}%
\pgfpathcurveto{\pgfqpoint{1.570516in}{2.103060in}}{\pgfqpoint{1.578416in}{2.106332in}}{\pgfqpoint{1.584240in}{2.112156in}}%
\pgfpathcurveto{\pgfqpoint{1.590064in}{2.117980in}}{\pgfqpoint{1.593336in}{2.125880in}}{\pgfqpoint{1.593336in}{2.134117in}}%
\pgfpathcurveto{\pgfqpoint{1.593336in}{2.142353in}}{\pgfqpoint{1.590064in}{2.150253in}}{\pgfqpoint{1.584240in}{2.156077in}}%
\pgfpathcurveto{\pgfqpoint{1.578416in}{2.161901in}}{\pgfqpoint{1.570516in}{2.165173in}}{\pgfqpoint{1.562280in}{2.165173in}}%
\pgfpathcurveto{\pgfqpoint{1.554043in}{2.165173in}}{\pgfqpoint{1.546143in}{2.161901in}}{\pgfqpoint{1.540319in}{2.156077in}}%
\pgfpathcurveto{\pgfqpoint{1.534496in}{2.150253in}}{\pgfqpoint{1.531223in}{2.142353in}}{\pgfqpoint{1.531223in}{2.134117in}}%
\pgfpathcurveto{\pgfqpoint{1.531223in}{2.125880in}}{\pgfqpoint{1.534496in}{2.117980in}}{\pgfqpoint{1.540319in}{2.112156in}}%
\pgfpathcurveto{\pgfqpoint{1.546143in}{2.106332in}}{\pgfqpoint{1.554043in}{2.103060in}}{\pgfqpoint{1.562280in}{2.103060in}}%
\pgfpathclose%
\pgfusepath{stroke,fill}%
\end{pgfscope}%
\begin{pgfscope}%
\pgfpathrectangle{\pgfqpoint{0.100000in}{0.212622in}}{\pgfqpoint{3.696000in}{3.696000in}}%
\pgfusepath{clip}%
\pgfsetbuttcap%
\pgfsetroundjoin%
\definecolor{currentfill}{rgb}{0.121569,0.466667,0.705882}%
\pgfsetfillcolor{currentfill}%
\pgfsetfillopacity{0.362510}%
\pgfsetlinewidth{1.003750pt}%
\definecolor{currentstroke}{rgb}{0.121569,0.466667,0.705882}%
\pgfsetstrokecolor{currentstroke}%
\pgfsetstrokeopacity{0.362510}%
\pgfsetdash{}{0pt}%
\pgfpathmoveto{\pgfqpoint{2.138280in}{2.011265in}}%
\pgfpathcurveto{\pgfqpoint{2.146516in}{2.011265in}}{\pgfqpoint{2.154416in}{2.014537in}}{\pgfqpoint{2.160240in}{2.020361in}}%
\pgfpathcurveto{\pgfqpoint{2.166064in}{2.026185in}}{\pgfqpoint{2.169337in}{2.034085in}}{\pgfqpoint{2.169337in}{2.042321in}}%
\pgfpathcurveto{\pgfqpoint{2.169337in}{2.050557in}}{\pgfqpoint{2.166064in}{2.058457in}}{\pgfqpoint{2.160240in}{2.064281in}}%
\pgfpathcurveto{\pgfqpoint{2.154416in}{2.070105in}}{\pgfqpoint{2.146516in}{2.073378in}}{\pgfqpoint{2.138280in}{2.073378in}}%
\pgfpathcurveto{\pgfqpoint{2.130044in}{2.073378in}}{\pgfqpoint{2.122144in}{2.070105in}}{\pgfqpoint{2.116320in}{2.064281in}}%
\pgfpathcurveto{\pgfqpoint{2.110496in}{2.058457in}}{\pgfqpoint{2.107224in}{2.050557in}}{\pgfqpoint{2.107224in}{2.042321in}}%
\pgfpathcurveto{\pgfqpoint{2.107224in}{2.034085in}}{\pgfqpoint{2.110496in}{2.026185in}}{\pgfqpoint{2.116320in}{2.020361in}}%
\pgfpathcurveto{\pgfqpoint{2.122144in}{2.014537in}}{\pgfqpoint{2.130044in}{2.011265in}}{\pgfqpoint{2.138280in}{2.011265in}}%
\pgfpathclose%
\pgfusepath{stroke,fill}%
\end{pgfscope}%
\begin{pgfscope}%
\pgfpathrectangle{\pgfqpoint{0.100000in}{0.212622in}}{\pgfqpoint{3.696000in}{3.696000in}}%
\pgfusepath{clip}%
\pgfsetbuttcap%
\pgfsetroundjoin%
\definecolor{currentfill}{rgb}{0.121569,0.466667,0.705882}%
\pgfsetfillcolor{currentfill}%
\pgfsetfillopacity{0.362764}%
\pgfsetlinewidth{1.003750pt}%
\definecolor{currentstroke}{rgb}{0.121569,0.466667,0.705882}%
\pgfsetstrokecolor{currentstroke}%
\pgfsetstrokeopacity{0.362764}%
\pgfsetdash{}{0pt}%
\pgfpathmoveto{\pgfqpoint{1.561684in}{2.103113in}}%
\pgfpathcurveto{\pgfqpoint{1.569921in}{2.103113in}}{\pgfqpoint{1.577821in}{2.106385in}}{\pgfqpoint{1.583645in}{2.112209in}}%
\pgfpathcurveto{\pgfqpoint{1.589469in}{2.118033in}}{\pgfqpoint{1.592741in}{2.125933in}}{\pgfqpoint{1.592741in}{2.134169in}}%
\pgfpathcurveto{\pgfqpoint{1.592741in}{2.142405in}}{\pgfqpoint{1.589469in}{2.150306in}}{\pgfqpoint{1.583645in}{2.156129in}}%
\pgfpathcurveto{\pgfqpoint{1.577821in}{2.161953in}}{\pgfqpoint{1.569921in}{2.165226in}}{\pgfqpoint{1.561684in}{2.165226in}}%
\pgfpathcurveto{\pgfqpoint{1.553448in}{2.165226in}}{\pgfqpoint{1.545548in}{2.161953in}}{\pgfqpoint{1.539724in}{2.156129in}}%
\pgfpathcurveto{\pgfqpoint{1.533900in}{2.150306in}}{\pgfqpoint{1.530628in}{2.142405in}}{\pgfqpoint{1.530628in}{2.134169in}}%
\pgfpathcurveto{\pgfqpoint{1.530628in}{2.125933in}}{\pgfqpoint{1.533900in}{2.118033in}}{\pgfqpoint{1.539724in}{2.112209in}}%
\pgfpathcurveto{\pgfqpoint{1.545548in}{2.106385in}}{\pgfqpoint{1.553448in}{2.103113in}}{\pgfqpoint{1.561684in}{2.103113in}}%
\pgfpathclose%
\pgfusepath{stroke,fill}%
\end{pgfscope}%
\begin{pgfscope}%
\pgfpathrectangle{\pgfqpoint{0.100000in}{0.212622in}}{\pgfqpoint{3.696000in}{3.696000in}}%
\pgfusepath{clip}%
\pgfsetbuttcap%
\pgfsetroundjoin%
\definecolor{currentfill}{rgb}{0.121569,0.466667,0.705882}%
\pgfsetfillcolor{currentfill}%
\pgfsetfillopacity{0.363006}%
\pgfsetlinewidth{1.003750pt}%
\definecolor{currentstroke}{rgb}{0.121569,0.466667,0.705882}%
\pgfsetstrokecolor{currentstroke}%
\pgfsetstrokeopacity{0.363006}%
\pgfsetdash{}{0pt}%
\pgfpathmoveto{\pgfqpoint{1.561184in}{2.103152in}}%
\pgfpathcurveto{\pgfqpoint{1.569421in}{2.103152in}}{\pgfqpoint{1.577321in}{2.106424in}}{\pgfqpoint{1.583145in}{2.112248in}}%
\pgfpathcurveto{\pgfqpoint{1.588969in}{2.118072in}}{\pgfqpoint{1.592241in}{2.125972in}}{\pgfqpoint{1.592241in}{2.134208in}}%
\pgfpathcurveto{\pgfqpoint{1.592241in}{2.142444in}}{\pgfqpoint{1.588969in}{2.150344in}}{\pgfqpoint{1.583145in}{2.156168in}}%
\pgfpathcurveto{\pgfqpoint{1.577321in}{2.161992in}}{\pgfqpoint{1.569421in}{2.165265in}}{\pgfqpoint{1.561184in}{2.165265in}}%
\pgfpathcurveto{\pgfqpoint{1.552948in}{2.165265in}}{\pgfqpoint{1.545048in}{2.161992in}}{\pgfqpoint{1.539224in}{2.156168in}}%
\pgfpathcurveto{\pgfqpoint{1.533400in}{2.150344in}}{\pgfqpoint{1.530128in}{2.142444in}}{\pgfqpoint{1.530128in}{2.134208in}}%
\pgfpathcurveto{\pgfqpoint{1.530128in}{2.125972in}}{\pgfqpoint{1.533400in}{2.118072in}}{\pgfqpoint{1.539224in}{2.112248in}}%
\pgfpathcurveto{\pgfqpoint{1.545048in}{2.106424in}}{\pgfqpoint{1.552948in}{2.103152in}}{\pgfqpoint{1.561184in}{2.103152in}}%
\pgfpathclose%
\pgfusepath{stroke,fill}%
\end{pgfscope}%
\begin{pgfscope}%
\pgfpathrectangle{\pgfqpoint{0.100000in}{0.212622in}}{\pgfqpoint{3.696000in}{3.696000in}}%
\pgfusepath{clip}%
\pgfsetbuttcap%
\pgfsetroundjoin%
\definecolor{currentfill}{rgb}{0.121569,0.466667,0.705882}%
\pgfsetfillcolor{currentfill}%
\pgfsetfillopacity{0.363431}%
\pgfsetlinewidth{1.003750pt}%
\definecolor{currentstroke}{rgb}{0.121569,0.466667,0.705882}%
\pgfsetstrokecolor{currentstroke}%
\pgfsetstrokeopacity{0.363431}%
\pgfsetdash{}{0pt}%
\pgfpathmoveto{\pgfqpoint{1.560212in}{2.103177in}}%
\pgfpathcurveto{\pgfqpoint{1.568448in}{2.103177in}}{\pgfqpoint{1.576348in}{2.106449in}}{\pgfqpoint{1.582172in}{2.112273in}}%
\pgfpathcurveto{\pgfqpoint{1.587996in}{2.118097in}}{\pgfqpoint{1.591268in}{2.125997in}}{\pgfqpoint{1.591268in}{2.134234in}}%
\pgfpathcurveto{\pgfqpoint{1.591268in}{2.142470in}}{\pgfqpoint{1.587996in}{2.150370in}}{\pgfqpoint{1.582172in}{2.156194in}}%
\pgfpathcurveto{\pgfqpoint{1.576348in}{2.162018in}}{\pgfqpoint{1.568448in}{2.165290in}}{\pgfqpoint{1.560212in}{2.165290in}}%
\pgfpathcurveto{\pgfqpoint{1.551976in}{2.165290in}}{\pgfqpoint{1.544076in}{2.162018in}}{\pgfqpoint{1.538252in}{2.156194in}}%
\pgfpathcurveto{\pgfqpoint{1.532428in}{2.150370in}}{\pgfqpoint{1.529155in}{2.142470in}}{\pgfqpoint{1.529155in}{2.134234in}}%
\pgfpathcurveto{\pgfqpoint{1.529155in}{2.125997in}}{\pgfqpoint{1.532428in}{2.118097in}}{\pgfqpoint{1.538252in}{2.112273in}}%
\pgfpathcurveto{\pgfqpoint{1.544076in}{2.106449in}}{\pgfqpoint{1.551976in}{2.103177in}}{\pgfqpoint{1.560212in}{2.103177in}}%
\pgfpathclose%
\pgfusepath{stroke,fill}%
\end{pgfscope}%
\begin{pgfscope}%
\pgfpathrectangle{\pgfqpoint{0.100000in}{0.212622in}}{\pgfqpoint{3.696000in}{3.696000in}}%
\pgfusepath{clip}%
\pgfsetbuttcap%
\pgfsetroundjoin%
\definecolor{currentfill}{rgb}{0.121569,0.466667,0.705882}%
\pgfsetfillcolor{currentfill}%
\pgfsetfillopacity{0.363589}%
\pgfsetlinewidth{1.003750pt}%
\definecolor{currentstroke}{rgb}{0.121569,0.466667,0.705882}%
\pgfsetstrokecolor{currentstroke}%
\pgfsetstrokeopacity{0.363589}%
\pgfsetdash{}{0pt}%
\pgfpathmoveto{\pgfqpoint{1.559951in}{2.103174in}}%
\pgfpathcurveto{\pgfqpoint{1.568187in}{2.103174in}}{\pgfqpoint{1.576087in}{2.106446in}}{\pgfqpoint{1.581911in}{2.112270in}}%
\pgfpathcurveto{\pgfqpoint{1.587735in}{2.118094in}}{\pgfqpoint{1.591008in}{2.125994in}}{\pgfqpoint{1.591008in}{2.134230in}}%
\pgfpathcurveto{\pgfqpoint{1.591008in}{2.142467in}}{\pgfqpoint{1.587735in}{2.150367in}}{\pgfqpoint{1.581911in}{2.156191in}}%
\pgfpathcurveto{\pgfqpoint{1.576087in}{2.162015in}}{\pgfqpoint{1.568187in}{2.165287in}}{\pgfqpoint{1.559951in}{2.165287in}}%
\pgfpathcurveto{\pgfqpoint{1.551715in}{2.165287in}}{\pgfqpoint{1.543815in}{2.162015in}}{\pgfqpoint{1.537991in}{2.156191in}}%
\pgfpathcurveto{\pgfqpoint{1.532167in}{2.150367in}}{\pgfqpoint{1.528895in}{2.142467in}}{\pgfqpoint{1.528895in}{2.134230in}}%
\pgfpathcurveto{\pgfqpoint{1.528895in}{2.125994in}}{\pgfqpoint{1.532167in}{2.118094in}}{\pgfqpoint{1.537991in}{2.112270in}}%
\pgfpathcurveto{\pgfqpoint{1.543815in}{2.106446in}}{\pgfqpoint{1.551715in}{2.103174in}}{\pgfqpoint{1.559951in}{2.103174in}}%
\pgfpathclose%
\pgfusepath{stroke,fill}%
\end{pgfscope}%
\begin{pgfscope}%
\pgfpathrectangle{\pgfqpoint{0.100000in}{0.212622in}}{\pgfqpoint{3.696000in}{3.696000in}}%
\pgfusepath{clip}%
\pgfsetbuttcap%
\pgfsetroundjoin%
\definecolor{currentfill}{rgb}{0.121569,0.466667,0.705882}%
\pgfsetfillcolor{currentfill}%
\pgfsetfillopacity{0.363872}%
\pgfsetlinewidth{1.003750pt}%
\definecolor{currentstroke}{rgb}{0.121569,0.466667,0.705882}%
\pgfsetstrokecolor{currentstroke}%
\pgfsetstrokeopacity{0.363872}%
\pgfsetdash{}{0pt}%
\pgfpathmoveto{\pgfqpoint{1.559379in}{2.103211in}}%
\pgfpathcurveto{\pgfqpoint{1.567615in}{2.103211in}}{\pgfqpoint{1.575515in}{2.106484in}}{\pgfqpoint{1.581339in}{2.112308in}}%
\pgfpathcurveto{\pgfqpoint{1.587163in}{2.118132in}}{\pgfqpoint{1.590435in}{2.126032in}}{\pgfqpoint{1.590435in}{2.134268in}}%
\pgfpathcurveto{\pgfqpoint{1.590435in}{2.142504in}}{\pgfqpoint{1.587163in}{2.150404in}}{\pgfqpoint{1.581339in}{2.156228in}}%
\pgfpathcurveto{\pgfqpoint{1.575515in}{2.162052in}}{\pgfqpoint{1.567615in}{2.165324in}}{\pgfqpoint{1.559379in}{2.165324in}}%
\pgfpathcurveto{\pgfqpoint{1.551142in}{2.165324in}}{\pgfqpoint{1.543242in}{2.162052in}}{\pgfqpoint{1.537419in}{2.156228in}}%
\pgfpathcurveto{\pgfqpoint{1.531595in}{2.150404in}}{\pgfqpoint{1.528322in}{2.142504in}}{\pgfqpoint{1.528322in}{2.134268in}}%
\pgfpathcurveto{\pgfqpoint{1.528322in}{2.126032in}}{\pgfqpoint{1.531595in}{2.118132in}}{\pgfqpoint{1.537419in}{2.112308in}}%
\pgfpathcurveto{\pgfqpoint{1.543242in}{2.106484in}}{\pgfqpoint{1.551142in}{2.103211in}}{\pgfqpoint{1.559379in}{2.103211in}}%
\pgfpathclose%
\pgfusepath{stroke,fill}%
\end{pgfscope}%
\begin{pgfscope}%
\pgfpathrectangle{\pgfqpoint{0.100000in}{0.212622in}}{\pgfqpoint{3.696000in}{3.696000in}}%
\pgfusepath{clip}%
\pgfsetbuttcap%
\pgfsetroundjoin%
\definecolor{currentfill}{rgb}{0.121569,0.466667,0.705882}%
\pgfsetfillcolor{currentfill}%
\pgfsetfillopacity{0.364076}%
\pgfsetlinewidth{1.003750pt}%
\definecolor{currentstroke}{rgb}{0.121569,0.466667,0.705882}%
\pgfsetstrokecolor{currentstroke}%
\pgfsetstrokeopacity{0.364076}%
\pgfsetdash{}{0pt}%
\pgfpathmoveto{\pgfqpoint{2.149027in}{2.009048in}}%
\pgfpathcurveto{\pgfqpoint{2.157264in}{2.009048in}}{\pgfqpoint{2.165164in}{2.012320in}}{\pgfqpoint{2.170987in}{2.018144in}}%
\pgfpathcurveto{\pgfqpoint{2.176811in}{2.023968in}}{\pgfqpoint{2.180084in}{2.031868in}}{\pgfqpoint{2.180084in}{2.040104in}}%
\pgfpathcurveto{\pgfqpoint{2.180084in}{2.048340in}}{\pgfqpoint{2.176811in}{2.056240in}}{\pgfqpoint{2.170987in}{2.062064in}}%
\pgfpathcurveto{\pgfqpoint{2.165164in}{2.067888in}}{\pgfqpoint{2.157264in}{2.071161in}}{\pgfqpoint{2.149027in}{2.071161in}}%
\pgfpathcurveto{\pgfqpoint{2.140791in}{2.071161in}}{\pgfqpoint{2.132891in}{2.067888in}}{\pgfqpoint{2.127067in}{2.062064in}}%
\pgfpathcurveto{\pgfqpoint{2.121243in}{2.056240in}}{\pgfqpoint{2.117971in}{2.048340in}}{\pgfqpoint{2.117971in}{2.040104in}}%
\pgfpathcurveto{\pgfqpoint{2.117971in}{2.031868in}}{\pgfqpoint{2.121243in}{2.023968in}}{\pgfqpoint{2.127067in}{2.018144in}}%
\pgfpathcurveto{\pgfqpoint{2.132891in}{2.012320in}}{\pgfqpoint{2.140791in}{2.009048in}}{\pgfqpoint{2.149027in}{2.009048in}}%
\pgfpathclose%
\pgfusepath{stroke,fill}%
\end{pgfscope}%
\begin{pgfscope}%
\pgfpathrectangle{\pgfqpoint{0.100000in}{0.212622in}}{\pgfqpoint{3.696000in}{3.696000in}}%
\pgfusepath{clip}%
\pgfsetbuttcap%
\pgfsetroundjoin%
\definecolor{currentfill}{rgb}{0.121569,0.466667,0.705882}%
\pgfsetfillcolor{currentfill}%
\pgfsetfillopacity{0.364371}%
\pgfsetlinewidth{1.003750pt}%
\definecolor{currentstroke}{rgb}{0.121569,0.466667,0.705882}%
\pgfsetstrokecolor{currentstroke}%
\pgfsetstrokeopacity{0.364371}%
\pgfsetdash{}{0pt}%
\pgfpathmoveto{\pgfqpoint{1.558245in}{2.103261in}}%
\pgfpathcurveto{\pgfqpoint{1.566481in}{2.103261in}}{\pgfqpoint{1.574381in}{2.106534in}}{\pgfqpoint{1.580205in}{2.112358in}}%
\pgfpathcurveto{\pgfqpoint{1.586029in}{2.118182in}}{\pgfqpoint{1.589301in}{2.126082in}}{\pgfqpoint{1.589301in}{2.134318in}}%
\pgfpathcurveto{\pgfqpoint{1.589301in}{2.142554in}}{\pgfqpoint{1.586029in}{2.150454in}}{\pgfqpoint{1.580205in}{2.156278in}}%
\pgfpathcurveto{\pgfqpoint{1.574381in}{2.162102in}}{\pgfqpoint{1.566481in}{2.165374in}}{\pgfqpoint{1.558245in}{2.165374in}}%
\pgfpathcurveto{\pgfqpoint{1.550008in}{2.165374in}}{\pgfqpoint{1.542108in}{2.162102in}}{\pgfqpoint{1.536284in}{2.156278in}}%
\pgfpathcurveto{\pgfqpoint{1.530460in}{2.150454in}}{\pgfqpoint{1.527188in}{2.142554in}}{\pgfqpoint{1.527188in}{2.134318in}}%
\pgfpathcurveto{\pgfqpoint{1.527188in}{2.126082in}}{\pgfqpoint{1.530460in}{2.118182in}}{\pgfqpoint{1.536284in}{2.112358in}}%
\pgfpathcurveto{\pgfqpoint{1.542108in}{2.106534in}}{\pgfqpoint{1.550008in}{2.103261in}}{\pgfqpoint{1.558245in}{2.103261in}}%
\pgfpathclose%
\pgfusepath{stroke,fill}%
\end{pgfscope}%
\begin{pgfscope}%
\pgfpathrectangle{\pgfqpoint{0.100000in}{0.212622in}}{\pgfqpoint{3.696000in}{3.696000in}}%
\pgfusepath{clip}%
\pgfsetbuttcap%
\pgfsetroundjoin%
\definecolor{currentfill}{rgb}{0.121569,0.466667,0.705882}%
\pgfsetfillcolor{currentfill}%
\pgfsetfillopacity{0.364572}%
\pgfsetlinewidth{1.003750pt}%
\definecolor{currentstroke}{rgb}{0.121569,0.466667,0.705882}%
\pgfsetstrokecolor{currentstroke}%
\pgfsetstrokeopacity{0.364572}%
\pgfsetdash{}{0pt}%
\pgfpathmoveto{\pgfqpoint{1.557957in}{2.103289in}}%
\pgfpathcurveto{\pgfqpoint{1.566193in}{2.103289in}}{\pgfqpoint{1.574093in}{2.106562in}}{\pgfqpoint{1.579917in}{2.112385in}}%
\pgfpathcurveto{\pgfqpoint{1.585741in}{2.118209in}}{\pgfqpoint{1.589013in}{2.126109in}}{\pgfqpoint{1.589013in}{2.134346in}}%
\pgfpathcurveto{\pgfqpoint{1.589013in}{2.142582in}}{\pgfqpoint{1.585741in}{2.150482in}}{\pgfqpoint{1.579917in}{2.156306in}}%
\pgfpathcurveto{\pgfqpoint{1.574093in}{2.162130in}}{\pgfqpoint{1.566193in}{2.165402in}}{\pgfqpoint{1.557957in}{2.165402in}}%
\pgfpathcurveto{\pgfqpoint{1.549721in}{2.165402in}}{\pgfqpoint{1.541821in}{2.162130in}}{\pgfqpoint{1.535997in}{2.156306in}}%
\pgfpathcurveto{\pgfqpoint{1.530173in}{2.150482in}}{\pgfqpoint{1.526900in}{2.142582in}}{\pgfqpoint{1.526900in}{2.134346in}}%
\pgfpathcurveto{\pgfqpoint{1.526900in}{2.126109in}}{\pgfqpoint{1.530173in}{2.118209in}}{\pgfqpoint{1.535997in}{2.112385in}}%
\pgfpathcurveto{\pgfqpoint{1.541821in}{2.106562in}}{\pgfqpoint{1.549721in}{2.103289in}}{\pgfqpoint{1.557957in}{2.103289in}}%
\pgfpathclose%
\pgfusepath{stroke,fill}%
\end{pgfscope}%
\begin{pgfscope}%
\pgfpathrectangle{\pgfqpoint{0.100000in}{0.212622in}}{\pgfqpoint{3.696000in}{3.696000in}}%
\pgfusepath{clip}%
\pgfsetbuttcap%
\pgfsetroundjoin%
\definecolor{currentfill}{rgb}{0.121569,0.466667,0.705882}%
\pgfsetfillcolor{currentfill}%
\pgfsetfillopacity{0.364910}%
\pgfsetlinewidth{1.003750pt}%
\definecolor{currentstroke}{rgb}{0.121569,0.466667,0.705882}%
\pgfsetstrokecolor{currentstroke}%
\pgfsetstrokeopacity{0.364910}%
\pgfsetdash{}{0pt}%
\pgfpathmoveto{\pgfqpoint{1.557187in}{2.103353in}}%
\pgfpathcurveto{\pgfqpoint{1.565423in}{2.103353in}}{\pgfqpoint{1.573323in}{2.106625in}}{\pgfqpoint{1.579147in}{2.112449in}}%
\pgfpathcurveto{\pgfqpoint{1.584971in}{2.118273in}}{\pgfqpoint{1.588243in}{2.126173in}}{\pgfqpoint{1.588243in}{2.134410in}}%
\pgfpathcurveto{\pgfqpoint{1.588243in}{2.142646in}}{\pgfqpoint{1.584971in}{2.150546in}}{\pgfqpoint{1.579147in}{2.156370in}}%
\pgfpathcurveto{\pgfqpoint{1.573323in}{2.162194in}}{\pgfqpoint{1.565423in}{2.165466in}}{\pgfqpoint{1.557187in}{2.165466in}}%
\pgfpathcurveto{\pgfqpoint{1.548951in}{2.165466in}}{\pgfqpoint{1.541050in}{2.162194in}}{\pgfqpoint{1.535227in}{2.156370in}}%
\pgfpathcurveto{\pgfqpoint{1.529403in}{2.150546in}}{\pgfqpoint{1.526130in}{2.142646in}}{\pgfqpoint{1.526130in}{2.134410in}}%
\pgfpathcurveto{\pgfqpoint{1.526130in}{2.126173in}}{\pgfqpoint{1.529403in}{2.118273in}}{\pgfqpoint{1.535227in}{2.112449in}}%
\pgfpathcurveto{\pgfqpoint{1.541050in}{2.106625in}}{\pgfqpoint{1.548951in}{2.103353in}}{\pgfqpoint{1.557187in}{2.103353in}}%
\pgfpathclose%
\pgfusepath{stroke,fill}%
\end{pgfscope}%
\begin{pgfscope}%
\pgfpathrectangle{\pgfqpoint{0.100000in}{0.212622in}}{\pgfqpoint{3.696000in}{3.696000in}}%
\pgfusepath{clip}%
\pgfsetbuttcap%
\pgfsetroundjoin%
\definecolor{currentfill}{rgb}{0.121569,0.466667,0.705882}%
\pgfsetfillcolor{currentfill}%
\pgfsetfillopacity{0.365526}%
\pgfsetlinewidth{1.003750pt}%
\definecolor{currentstroke}{rgb}{0.121569,0.466667,0.705882}%
\pgfsetstrokecolor{currentstroke}%
\pgfsetstrokeopacity{0.365526}%
\pgfsetdash{}{0pt}%
\pgfpathmoveto{\pgfqpoint{1.555852in}{2.103403in}}%
\pgfpathcurveto{\pgfqpoint{1.564088in}{2.103403in}}{\pgfqpoint{1.571988in}{2.106675in}}{\pgfqpoint{1.577812in}{2.112499in}}%
\pgfpathcurveto{\pgfqpoint{1.583636in}{2.118323in}}{\pgfqpoint{1.586908in}{2.126223in}}{\pgfqpoint{1.586908in}{2.134459in}}%
\pgfpathcurveto{\pgfqpoint{1.586908in}{2.142695in}}{\pgfqpoint{1.583636in}{2.150595in}}{\pgfqpoint{1.577812in}{2.156419in}}%
\pgfpathcurveto{\pgfqpoint{1.571988in}{2.162243in}}{\pgfqpoint{1.564088in}{2.165516in}}{\pgfqpoint{1.555852in}{2.165516in}}%
\pgfpathcurveto{\pgfqpoint{1.547615in}{2.165516in}}{\pgfqpoint{1.539715in}{2.162243in}}{\pgfqpoint{1.533891in}{2.156419in}}%
\pgfpathcurveto{\pgfqpoint{1.528068in}{2.150595in}}{\pgfqpoint{1.524795in}{2.142695in}}{\pgfqpoint{1.524795in}{2.134459in}}%
\pgfpathcurveto{\pgfqpoint{1.524795in}{2.126223in}}{\pgfqpoint{1.528068in}{2.118323in}}{\pgfqpoint{1.533891in}{2.112499in}}%
\pgfpathcurveto{\pgfqpoint{1.539715in}{2.106675in}}{\pgfqpoint{1.547615in}{2.103403in}}{\pgfqpoint{1.555852in}{2.103403in}}%
\pgfpathclose%
\pgfusepath{stroke,fill}%
\end{pgfscope}%
\begin{pgfscope}%
\pgfpathrectangle{\pgfqpoint{0.100000in}{0.212622in}}{\pgfqpoint{3.696000in}{3.696000in}}%
\pgfusepath{clip}%
\pgfsetbuttcap%
\pgfsetroundjoin%
\definecolor{currentfill}{rgb}{0.121569,0.466667,0.705882}%
\pgfsetfillcolor{currentfill}%
\pgfsetfillopacity{0.366020}%
\pgfsetlinewidth{1.003750pt}%
\definecolor{currentstroke}{rgb}{0.121569,0.466667,0.705882}%
\pgfsetstrokecolor{currentstroke}%
\pgfsetstrokeopacity{0.366020}%
\pgfsetdash{}{0pt}%
\pgfpathmoveto{\pgfqpoint{1.555256in}{2.103493in}}%
\pgfpathcurveto{\pgfqpoint{1.563493in}{2.103493in}}{\pgfqpoint{1.571393in}{2.106765in}}{\pgfqpoint{1.577217in}{2.112589in}}%
\pgfpathcurveto{\pgfqpoint{1.583041in}{2.118413in}}{\pgfqpoint{1.586313in}{2.126313in}}{\pgfqpoint{1.586313in}{2.134549in}}%
\pgfpathcurveto{\pgfqpoint{1.586313in}{2.142785in}}{\pgfqpoint{1.583041in}{2.150685in}}{\pgfqpoint{1.577217in}{2.156509in}}%
\pgfpathcurveto{\pgfqpoint{1.571393in}{2.162333in}}{\pgfqpoint{1.563493in}{2.165606in}}{\pgfqpoint{1.555256in}{2.165606in}}%
\pgfpathcurveto{\pgfqpoint{1.547020in}{2.165606in}}{\pgfqpoint{1.539120in}{2.162333in}}{\pgfqpoint{1.533296in}{2.156509in}}%
\pgfpathcurveto{\pgfqpoint{1.527472in}{2.150685in}}{\pgfqpoint{1.524200in}{2.142785in}}{\pgfqpoint{1.524200in}{2.134549in}}%
\pgfpathcurveto{\pgfqpoint{1.524200in}{2.126313in}}{\pgfqpoint{1.527472in}{2.118413in}}{\pgfqpoint{1.533296in}{2.112589in}}%
\pgfpathcurveto{\pgfqpoint{1.539120in}{2.106765in}}{\pgfqpoint{1.547020in}{2.103493in}}{\pgfqpoint{1.555256in}{2.103493in}}%
\pgfpathclose%
\pgfusepath{stroke,fill}%
\end{pgfscope}%
\begin{pgfscope}%
\pgfpathrectangle{\pgfqpoint{0.100000in}{0.212622in}}{\pgfqpoint{3.696000in}{3.696000in}}%
\pgfusepath{clip}%
\pgfsetbuttcap%
\pgfsetroundjoin%
\definecolor{currentfill}{rgb}{0.121569,0.466667,0.705882}%
\pgfsetfillcolor{currentfill}%
\pgfsetfillopacity{0.366102}%
\pgfsetlinewidth{1.003750pt}%
\definecolor{currentstroke}{rgb}{0.121569,0.466667,0.705882}%
\pgfsetstrokecolor{currentstroke}%
\pgfsetstrokeopacity{0.366102}%
\pgfsetdash{}{0pt}%
\pgfpathmoveto{\pgfqpoint{2.160389in}{2.007427in}}%
\pgfpathcurveto{\pgfqpoint{2.168625in}{2.007427in}}{\pgfqpoint{2.176525in}{2.010699in}}{\pgfqpoint{2.182349in}{2.016523in}}%
\pgfpathcurveto{\pgfqpoint{2.188173in}{2.022347in}}{\pgfqpoint{2.191445in}{2.030247in}}{\pgfqpoint{2.191445in}{2.038483in}}%
\pgfpathcurveto{\pgfqpoint{2.191445in}{2.046719in}}{\pgfqpoint{2.188173in}{2.054620in}}{\pgfqpoint{2.182349in}{2.060443in}}%
\pgfpathcurveto{\pgfqpoint{2.176525in}{2.066267in}}{\pgfqpoint{2.168625in}{2.069540in}}{\pgfqpoint{2.160389in}{2.069540in}}%
\pgfpathcurveto{\pgfqpoint{2.152152in}{2.069540in}}{\pgfqpoint{2.144252in}{2.066267in}}{\pgfqpoint{2.138429in}{2.060443in}}%
\pgfpathcurveto{\pgfqpoint{2.132605in}{2.054620in}}{\pgfqpoint{2.129332in}{2.046719in}}{\pgfqpoint{2.129332in}{2.038483in}}%
\pgfpathcurveto{\pgfqpoint{2.129332in}{2.030247in}}{\pgfqpoint{2.132605in}{2.022347in}}{\pgfqpoint{2.138429in}{2.016523in}}%
\pgfpathcurveto{\pgfqpoint{2.144252in}{2.010699in}}{\pgfqpoint{2.152152in}{2.007427in}}{\pgfqpoint{2.160389in}{2.007427in}}%
\pgfpathclose%
\pgfusepath{stroke,fill}%
\end{pgfscope}%
\begin{pgfscope}%
\pgfpathrectangle{\pgfqpoint{0.100000in}{0.212622in}}{\pgfqpoint{3.696000in}{3.696000in}}%
\pgfusepath{clip}%
\pgfsetbuttcap%
\pgfsetroundjoin%
\definecolor{currentfill}{rgb}{0.121569,0.466667,0.705882}%
\pgfsetfillcolor{currentfill}%
\pgfsetfillopacity{0.366848}%
\pgfsetlinewidth{1.003750pt}%
\definecolor{currentstroke}{rgb}{0.121569,0.466667,0.705882}%
\pgfsetstrokecolor{currentstroke}%
\pgfsetstrokeopacity{0.366848}%
\pgfsetdash{}{0pt}%
\pgfpathmoveto{\pgfqpoint{1.553548in}{2.103596in}}%
\pgfpathcurveto{\pgfqpoint{1.561784in}{2.103596in}}{\pgfqpoint{1.569684in}{2.106868in}}{\pgfqpoint{1.575508in}{2.112692in}}%
\pgfpathcurveto{\pgfqpoint{1.581332in}{2.118516in}}{\pgfqpoint{1.584605in}{2.126416in}}{\pgfqpoint{1.584605in}{2.134653in}}%
\pgfpathcurveto{\pgfqpoint{1.584605in}{2.142889in}}{\pgfqpoint{1.581332in}{2.150789in}}{\pgfqpoint{1.575508in}{2.156613in}}%
\pgfpathcurveto{\pgfqpoint{1.569684in}{2.162437in}}{\pgfqpoint{1.561784in}{2.165709in}}{\pgfqpoint{1.553548in}{2.165709in}}%
\pgfpathcurveto{\pgfqpoint{1.545312in}{2.165709in}}{\pgfqpoint{1.537412in}{2.162437in}}{\pgfqpoint{1.531588in}{2.156613in}}%
\pgfpathcurveto{\pgfqpoint{1.525764in}{2.150789in}}{\pgfqpoint{1.522492in}{2.142889in}}{\pgfqpoint{1.522492in}{2.134653in}}%
\pgfpathcurveto{\pgfqpoint{1.522492in}{2.126416in}}{\pgfqpoint{1.525764in}{2.118516in}}{\pgfqpoint{1.531588in}{2.112692in}}%
\pgfpathcurveto{\pgfqpoint{1.537412in}{2.106868in}}{\pgfqpoint{1.545312in}{2.103596in}}{\pgfqpoint{1.553548in}{2.103596in}}%
\pgfpathclose%
\pgfusepath{stroke,fill}%
\end{pgfscope}%
\begin{pgfscope}%
\pgfpathrectangle{\pgfqpoint{0.100000in}{0.212622in}}{\pgfqpoint{3.696000in}{3.696000in}}%
\pgfusepath{clip}%
\pgfsetbuttcap%
\pgfsetroundjoin%
\definecolor{currentfill}{rgb}{0.121569,0.466667,0.705882}%
\pgfsetfillcolor{currentfill}%
\pgfsetfillopacity{0.367573}%
\pgfsetlinewidth{1.003750pt}%
\definecolor{currentstroke}{rgb}{0.121569,0.466667,0.705882}%
\pgfsetstrokecolor{currentstroke}%
\pgfsetstrokeopacity{0.367573}%
\pgfsetdash{}{0pt}%
\pgfpathmoveto{\pgfqpoint{2.173267in}{2.004423in}}%
\pgfpathcurveto{\pgfqpoint{2.181503in}{2.004423in}}{\pgfqpoint{2.189403in}{2.007696in}}{\pgfqpoint{2.195227in}{2.013519in}}%
\pgfpathcurveto{\pgfqpoint{2.201051in}{2.019343in}}{\pgfqpoint{2.204323in}{2.027243in}}{\pgfqpoint{2.204323in}{2.035480in}}%
\pgfpathcurveto{\pgfqpoint{2.204323in}{2.043716in}}{\pgfqpoint{2.201051in}{2.051616in}}{\pgfqpoint{2.195227in}{2.057440in}}%
\pgfpathcurveto{\pgfqpoint{2.189403in}{2.063264in}}{\pgfqpoint{2.181503in}{2.066536in}}{\pgfqpoint{2.173267in}{2.066536in}}%
\pgfpathcurveto{\pgfqpoint{2.165030in}{2.066536in}}{\pgfqpoint{2.157130in}{2.063264in}}{\pgfqpoint{2.151306in}{2.057440in}}%
\pgfpathcurveto{\pgfqpoint{2.145483in}{2.051616in}}{\pgfqpoint{2.142210in}{2.043716in}}{\pgfqpoint{2.142210in}{2.035480in}}%
\pgfpathcurveto{\pgfqpoint{2.142210in}{2.027243in}}{\pgfqpoint{2.145483in}{2.019343in}}{\pgfqpoint{2.151306in}{2.013519in}}%
\pgfpathcurveto{\pgfqpoint{2.157130in}{2.007696in}}{\pgfqpoint{2.165030in}{2.004423in}}{\pgfqpoint{2.173267in}{2.004423in}}%
\pgfpathclose%
\pgfusepath{stroke,fill}%
\end{pgfscope}%
\begin{pgfscope}%
\pgfpathrectangle{\pgfqpoint{0.100000in}{0.212622in}}{\pgfqpoint{3.696000in}{3.696000in}}%
\pgfusepath{clip}%
\pgfsetbuttcap%
\pgfsetroundjoin%
\definecolor{currentfill}{rgb}{0.121569,0.466667,0.705882}%
\pgfsetfillcolor{currentfill}%
\pgfsetfillopacity{0.367589}%
\pgfsetlinewidth{1.003750pt}%
\definecolor{currentstroke}{rgb}{0.121569,0.466667,0.705882}%
\pgfsetstrokecolor{currentstroke}%
\pgfsetstrokeopacity{0.367589}%
\pgfsetdash{}{0pt}%
\pgfpathmoveto{\pgfqpoint{1.552002in}{2.103653in}}%
\pgfpathcurveto{\pgfqpoint{1.560238in}{2.103653in}}{\pgfqpoint{1.568139in}{2.106925in}}{\pgfqpoint{1.573962in}{2.112749in}}%
\pgfpathcurveto{\pgfqpoint{1.579786in}{2.118573in}}{\pgfqpoint{1.583059in}{2.126473in}}{\pgfqpoint{1.583059in}{2.134709in}}%
\pgfpathcurveto{\pgfqpoint{1.583059in}{2.142945in}}{\pgfqpoint{1.579786in}{2.150845in}}{\pgfqpoint{1.573962in}{2.156669in}}%
\pgfpathcurveto{\pgfqpoint{1.568139in}{2.162493in}}{\pgfqpoint{1.560238in}{2.165766in}}{\pgfqpoint{1.552002in}{2.165766in}}%
\pgfpathcurveto{\pgfqpoint{1.543766in}{2.165766in}}{\pgfqpoint{1.535866in}{2.162493in}}{\pgfqpoint{1.530042in}{2.156669in}}%
\pgfpathcurveto{\pgfqpoint{1.524218in}{2.150845in}}{\pgfqpoint{1.520946in}{2.142945in}}{\pgfqpoint{1.520946in}{2.134709in}}%
\pgfpathcurveto{\pgfqpoint{1.520946in}{2.126473in}}{\pgfqpoint{1.524218in}{2.118573in}}{\pgfqpoint{1.530042in}{2.112749in}}%
\pgfpathcurveto{\pgfqpoint{1.535866in}{2.106925in}}{\pgfqpoint{1.543766in}{2.103653in}}{\pgfqpoint{1.552002in}{2.103653in}}%
\pgfpathclose%
\pgfusepath{stroke,fill}%
\end{pgfscope}%
\begin{pgfscope}%
\pgfpathrectangle{\pgfqpoint{0.100000in}{0.212622in}}{\pgfqpoint{3.696000in}{3.696000in}}%
\pgfusepath{clip}%
\pgfsetbuttcap%
\pgfsetroundjoin%
\definecolor{currentfill}{rgb}{0.121569,0.466667,0.705882}%
\pgfsetfillcolor{currentfill}%
\pgfsetfillopacity{0.369039}%
\pgfsetlinewidth{1.003750pt}%
\definecolor{currentstroke}{rgb}{0.121569,0.466667,0.705882}%
\pgfsetstrokecolor{currentstroke}%
\pgfsetstrokeopacity{0.369039}%
\pgfsetdash{}{0pt}%
\pgfpathmoveto{\pgfqpoint{1.549983in}{2.103881in}}%
\pgfpathcurveto{\pgfqpoint{1.558219in}{2.103881in}}{\pgfqpoint{1.566119in}{2.107153in}}{\pgfqpoint{1.571943in}{2.112977in}}%
\pgfpathcurveto{\pgfqpoint{1.577767in}{2.118801in}}{\pgfqpoint{1.581040in}{2.126701in}}{\pgfqpoint{1.581040in}{2.134937in}}%
\pgfpathcurveto{\pgfqpoint{1.581040in}{2.143173in}}{\pgfqpoint{1.577767in}{2.151073in}}{\pgfqpoint{1.571943in}{2.156897in}}%
\pgfpathcurveto{\pgfqpoint{1.566119in}{2.162721in}}{\pgfqpoint{1.558219in}{2.165994in}}{\pgfqpoint{1.549983in}{2.165994in}}%
\pgfpathcurveto{\pgfqpoint{1.541747in}{2.165994in}}{\pgfqpoint{1.533847in}{2.162721in}}{\pgfqpoint{1.528023in}{2.156897in}}%
\pgfpathcurveto{\pgfqpoint{1.522199in}{2.151073in}}{\pgfqpoint{1.518927in}{2.143173in}}{\pgfqpoint{1.518927in}{2.134937in}}%
\pgfpathcurveto{\pgfqpoint{1.518927in}{2.126701in}}{\pgfqpoint{1.522199in}{2.118801in}}{\pgfqpoint{1.528023in}{2.112977in}}%
\pgfpathcurveto{\pgfqpoint{1.533847in}{2.107153in}}{\pgfqpoint{1.541747in}{2.103881in}}{\pgfqpoint{1.549983in}{2.103881in}}%
\pgfpathclose%
\pgfusepath{stroke,fill}%
\end{pgfscope}%
\begin{pgfscope}%
\pgfpathrectangle{\pgfqpoint{0.100000in}{0.212622in}}{\pgfqpoint{3.696000in}{3.696000in}}%
\pgfusepath{clip}%
\pgfsetbuttcap%
\pgfsetroundjoin%
\definecolor{currentfill}{rgb}{0.121569,0.466667,0.705882}%
\pgfsetfillcolor{currentfill}%
\pgfsetfillopacity{0.369976}%
\pgfsetlinewidth{1.003750pt}%
\definecolor{currentstroke}{rgb}{0.121569,0.466667,0.705882}%
\pgfsetstrokecolor{currentstroke}%
\pgfsetstrokeopacity{0.369976}%
\pgfsetdash{}{0pt}%
\pgfpathmoveto{\pgfqpoint{2.185978in}{2.001686in}}%
\pgfpathcurveto{\pgfqpoint{2.194215in}{2.001686in}}{\pgfqpoint{2.202115in}{2.004959in}}{\pgfqpoint{2.207938in}{2.010783in}}%
\pgfpathcurveto{\pgfqpoint{2.213762in}{2.016606in}}{\pgfqpoint{2.217035in}{2.024507in}}{\pgfqpoint{2.217035in}{2.032743in}}%
\pgfpathcurveto{\pgfqpoint{2.217035in}{2.040979in}}{\pgfqpoint{2.213762in}{2.048879in}}{\pgfqpoint{2.207938in}{2.054703in}}%
\pgfpathcurveto{\pgfqpoint{2.202115in}{2.060527in}}{\pgfqpoint{2.194215in}{2.063799in}}{\pgfqpoint{2.185978in}{2.063799in}}%
\pgfpathcurveto{\pgfqpoint{2.177742in}{2.063799in}}{\pgfqpoint{2.169842in}{2.060527in}}{\pgfqpoint{2.164018in}{2.054703in}}%
\pgfpathcurveto{\pgfqpoint{2.158194in}{2.048879in}}{\pgfqpoint{2.154922in}{2.040979in}}{\pgfqpoint{2.154922in}{2.032743in}}%
\pgfpathcurveto{\pgfqpoint{2.154922in}{2.024507in}}{\pgfqpoint{2.158194in}{2.016606in}}{\pgfqpoint{2.164018in}{2.010783in}}%
\pgfpathcurveto{\pgfqpoint{2.169842in}{2.004959in}}{\pgfqpoint{2.177742in}{2.001686in}}{\pgfqpoint{2.185978in}{2.001686in}}%
\pgfpathclose%
\pgfusepath{stroke,fill}%
\end{pgfscope}%
\begin{pgfscope}%
\pgfpathrectangle{\pgfqpoint{0.100000in}{0.212622in}}{\pgfqpoint{3.696000in}{3.696000in}}%
\pgfusepath{clip}%
\pgfsetbuttcap%
\pgfsetroundjoin%
\definecolor{currentfill}{rgb}{0.121569,0.466667,0.705882}%
\pgfsetfillcolor{currentfill}%
\pgfsetfillopacity{0.370175}%
\pgfsetlinewidth{1.003750pt}%
\definecolor{currentstroke}{rgb}{0.121569,0.466667,0.705882}%
\pgfsetstrokecolor{currentstroke}%
\pgfsetstrokeopacity{0.370175}%
\pgfsetdash{}{0pt}%
\pgfpathmoveto{\pgfqpoint{1.547446in}{2.103934in}}%
\pgfpathcurveto{\pgfqpoint{1.555683in}{2.103934in}}{\pgfqpoint{1.563583in}{2.107206in}}{\pgfqpoint{1.569407in}{2.113030in}}%
\pgfpathcurveto{\pgfqpoint{1.575231in}{2.118854in}}{\pgfqpoint{1.578503in}{2.126754in}}{\pgfqpoint{1.578503in}{2.134990in}}%
\pgfpathcurveto{\pgfqpoint{1.578503in}{2.143226in}}{\pgfqpoint{1.575231in}{2.151127in}}{\pgfqpoint{1.569407in}{2.156950in}}%
\pgfpathcurveto{\pgfqpoint{1.563583in}{2.162774in}}{\pgfqpoint{1.555683in}{2.166047in}}{\pgfqpoint{1.547446in}{2.166047in}}%
\pgfpathcurveto{\pgfqpoint{1.539210in}{2.166047in}}{\pgfqpoint{1.531310in}{2.162774in}}{\pgfqpoint{1.525486in}{2.156950in}}%
\pgfpathcurveto{\pgfqpoint{1.519662in}{2.151127in}}{\pgfqpoint{1.516390in}{2.143226in}}{\pgfqpoint{1.516390in}{2.134990in}}%
\pgfpathcurveto{\pgfqpoint{1.516390in}{2.126754in}}{\pgfqpoint{1.519662in}{2.118854in}}{\pgfqpoint{1.525486in}{2.113030in}}%
\pgfpathcurveto{\pgfqpoint{1.531310in}{2.107206in}}{\pgfqpoint{1.539210in}{2.103934in}}{\pgfqpoint{1.547446in}{2.103934in}}%
\pgfpathclose%
\pgfusepath{stroke,fill}%
\end{pgfscope}%
\begin{pgfscope}%
\pgfpathrectangle{\pgfqpoint{0.100000in}{0.212622in}}{\pgfqpoint{3.696000in}{3.696000in}}%
\pgfusepath{clip}%
\pgfsetbuttcap%
\pgfsetroundjoin%
\definecolor{currentfill}{rgb}{0.121569,0.466667,0.705882}%
\pgfsetfillcolor{currentfill}%
\pgfsetfillopacity{0.371409}%
\pgfsetlinewidth{1.003750pt}%
\definecolor{currentstroke}{rgb}{0.121569,0.466667,0.705882}%
\pgfsetstrokecolor{currentstroke}%
\pgfsetstrokeopacity{0.371409}%
\pgfsetdash{}{0pt}%
\pgfpathmoveto{\pgfqpoint{1.548583in}{2.104652in}}%
\pgfpathcurveto{\pgfqpoint{1.556819in}{2.104652in}}{\pgfqpoint{1.564719in}{2.107924in}}{\pgfqpoint{1.570543in}{2.113748in}}%
\pgfpathcurveto{\pgfqpoint{1.576367in}{2.119572in}}{\pgfqpoint{1.579640in}{2.127472in}}{\pgfqpoint{1.579640in}{2.135708in}}%
\pgfpathcurveto{\pgfqpoint{1.579640in}{2.143945in}}{\pgfqpoint{1.576367in}{2.151845in}}{\pgfqpoint{1.570543in}{2.157669in}}%
\pgfpathcurveto{\pgfqpoint{1.564719in}{2.163492in}}{\pgfqpoint{1.556819in}{2.166765in}}{\pgfqpoint{1.548583in}{2.166765in}}%
\pgfpathcurveto{\pgfqpoint{1.540347in}{2.166765in}}{\pgfqpoint{1.532447in}{2.163492in}}{\pgfqpoint{1.526623in}{2.157669in}}%
\pgfpathcurveto{\pgfqpoint{1.520799in}{2.151845in}}{\pgfqpoint{1.517527in}{2.143945in}}{\pgfqpoint{1.517527in}{2.135708in}}%
\pgfpathcurveto{\pgfqpoint{1.517527in}{2.127472in}}{\pgfqpoint{1.520799in}{2.119572in}}{\pgfqpoint{1.526623in}{2.113748in}}%
\pgfpathcurveto{\pgfqpoint{1.532447in}{2.107924in}}{\pgfqpoint{1.540347in}{2.104652in}}{\pgfqpoint{1.548583in}{2.104652in}}%
\pgfpathclose%
\pgfusepath{stroke,fill}%
\end{pgfscope}%
\begin{pgfscope}%
\pgfpathrectangle{\pgfqpoint{0.100000in}{0.212622in}}{\pgfqpoint{3.696000in}{3.696000in}}%
\pgfusepath{clip}%
\pgfsetbuttcap%
\pgfsetroundjoin%
\definecolor{currentfill}{rgb}{0.121569,0.466667,0.705882}%
\pgfsetfillcolor{currentfill}%
\pgfsetfillopacity{0.372401}%
\pgfsetlinewidth{1.003750pt}%
\definecolor{currentstroke}{rgb}{0.121569,0.466667,0.705882}%
\pgfsetstrokecolor{currentstroke}%
\pgfsetstrokeopacity{0.372401}%
\pgfsetdash{}{0pt}%
\pgfpathmoveto{\pgfqpoint{1.546642in}{2.104737in}}%
\pgfpathcurveto{\pgfqpoint{1.554879in}{2.104737in}}{\pgfqpoint{1.562779in}{2.108009in}}{\pgfqpoint{1.568602in}{2.113833in}}%
\pgfpathcurveto{\pgfqpoint{1.574426in}{2.119657in}}{\pgfqpoint{1.577699in}{2.127557in}}{\pgfqpoint{1.577699in}{2.135793in}}%
\pgfpathcurveto{\pgfqpoint{1.577699in}{2.144029in}}{\pgfqpoint{1.574426in}{2.151929in}}{\pgfqpoint{1.568602in}{2.157753in}}%
\pgfpathcurveto{\pgfqpoint{1.562779in}{2.163577in}}{\pgfqpoint{1.554879in}{2.166850in}}{\pgfqpoint{1.546642in}{2.166850in}}%
\pgfpathcurveto{\pgfqpoint{1.538406in}{2.166850in}}{\pgfqpoint{1.530506in}{2.163577in}}{\pgfqpoint{1.524682in}{2.157753in}}%
\pgfpathcurveto{\pgfqpoint{1.518858in}{2.151929in}}{\pgfqpoint{1.515586in}{2.144029in}}{\pgfqpoint{1.515586in}{2.135793in}}%
\pgfpathcurveto{\pgfqpoint{1.515586in}{2.127557in}}{\pgfqpoint{1.518858in}{2.119657in}}{\pgfqpoint{1.524682in}{2.113833in}}%
\pgfpathcurveto{\pgfqpoint{1.530506in}{2.108009in}}{\pgfqpoint{1.538406in}{2.104737in}}{\pgfqpoint{1.546642in}{2.104737in}}%
\pgfpathclose%
\pgfusepath{stroke,fill}%
\end{pgfscope}%
\begin{pgfscope}%
\pgfpathrectangle{\pgfqpoint{0.100000in}{0.212622in}}{\pgfqpoint{3.696000in}{3.696000in}}%
\pgfusepath{clip}%
\pgfsetbuttcap%
\pgfsetroundjoin%
\definecolor{currentfill}{rgb}{0.121569,0.466667,0.705882}%
\pgfsetfillcolor{currentfill}%
\pgfsetfillopacity{0.373233}%
\pgfsetlinewidth{1.003750pt}%
\definecolor{currentstroke}{rgb}{0.121569,0.466667,0.705882}%
\pgfsetstrokecolor{currentstroke}%
\pgfsetstrokeopacity{0.373233}%
\pgfsetdash{}{0pt}%
\pgfpathmoveto{\pgfqpoint{1.544690in}{2.104880in}}%
\pgfpathcurveto{\pgfqpoint{1.552926in}{2.104880in}}{\pgfqpoint{1.560826in}{2.108152in}}{\pgfqpoint{1.566650in}{2.113976in}}%
\pgfpathcurveto{\pgfqpoint{1.572474in}{2.119800in}}{\pgfqpoint{1.575746in}{2.127700in}}{\pgfqpoint{1.575746in}{2.135936in}}%
\pgfpathcurveto{\pgfqpoint{1.575746in}{2.144173in}}{\pgfqpoint{1.572474in}{2.152073in}}{\pgfqpoint{1.566650in}{2.157897in}}%
\pgfpathcurveto{\pgfqpoint{1.560826in}{2.163721in}}{\pgfqpoint{1.552926in}{2.166993in}}{\pgfqpoint{1.544690in}{2.166993in}}%
\pgfpathcurveto{\pgfqpoint{1.536453in}{2.166993in}}{\pgfqpoint{1.528553in}{2.163721in}}{\pgfqpoint{1.522729in}{2.157897in}}%
\pgfpathcurveto{\pgfqpoint{1.516905in}{2.152073in}}{\pgfqpoint{1.513633in}{2.144173in}}{\pgfqpoint{1.513633in}{2.135936in}}%
\pgfpathcurveto{\pgfqpoint{1.513633in}{2.127700in}}{\pgfqpoint{1.516905in}{2.119800in}}{\pgfqpoint{1.522729in}{2.113976in}}%
\pgfpathcurveto{\pgfqpoint{1.528553in}{2.108152in}}{\pgfqpoint{1.536453in}{2.104880in}}{\pgfqpoint{1.544690in}{2.104880in}}%
\pgfpathclose%
\pgfusepath{stroke,fill}%
\end{pgfscope}%
\begin{pgfscope}%
\pgfpathrectangle{\pgfqpoint{0.100000in}{0.212622in}}{\pgfqpoint{3.696000in}{3.696000in}}%
\pgfusepath{clip}%
\pgfsetbuttcap%
\pgfsetroundjoin%
\definecolor{currentfill}{rgb}{0.121569,0.466667,0.705882}%
\pgfsetfillcolor{currentfill}%
\pgfsetfillopacity{0.373298}%
\pgfsetlinewidth{1.003750pt}%
\definecolor{currentstroke}{rgb}{0.121569,0.466667,0.705882}%
\pgfsetstrokecolor{currentstroke}%
\pgfsetstrokeopacity{0.373298}%
\pgfsetdash{}{0pt}%
\pgfpathmoveto{\pgfqpoint{2.197226in}{2.000666in}}%
\pgfpathcurveto{\pgfqpoint{2.205463in}{2.000666in}}{\pgfqpoint{2.213363in}{2.003938in}}{\pgfqpoint{2.219187in}{2.009762in}}%
\pgfpathcurveto{\pgfqpoint{2.225011in}{2.015586in}}{\pgfqpoint{2.228283in}{2.023486in}}{\pgfqpoint{2.228283in}{2.031722in}}%
\pgfpathcurveto{\pgfqpoint{2.228283in}{2.039959in}}{\pgfqpoint{2.225011in}{2.047859in}}{\pgfqpoint{2.219187in}{2.053683in}}%
\pgfpathcurveto{\pgfqpoint{2.213363in}{2.059506in}}{\pgfqpoint{2.205463in}{2.062779in}}{\pgfqpoint{2.197226in}{2.062779in}}%
\pgfpathcurveto{\pgfqpoint{2.188990in}{2.062779in}}{\pgfqpoint{2.181090in}{2.059506in}}{\pgfqpoint{2.175266in}{2.053683in}}%
\pgfpathcurveto{\pgfqpoint{2.169442in}{2.047859in}}{\pgfqpoint{2.166170in}{2.039959in}}{\pgfqpoint{2.166170in}{2.031722in}}%
\pgfpathcurveto{\pgfqpoint{2.166170in}{2.023486in}}{\pgfqpoint{2.169442in}{2.015586in}}{\pgfqpoint{2.175266in}{2.009762in}}%
\pgfpathcurveto{\pgfqpoint{2.181090in}{2.003938in}}{\pgfqpoint{2.188990in}{2.000666in}}{\pgfqpoint{2.197226in}{2.000666in}}%
\pgfpathclose%
\pgfusepath{stroke,fill}%
\end{pgfscope}%
\begin{pgfscope}%
\pgfpathrectangle{\pgfqpoint{0.100000in}{0.212622in}}{\pgfqpoint{3.696000in}{3.696000in}}%
\pgfusepath{clip}%
\pgfsetbuttcap%
\pgfsetroundjoin%
\definecolor{currentfill}{rgb}{0.121569,0.466667,0.705882}%
\pgfsetfillcolor{currentfill}%
\pgfsetfillopacity{0.373975}%
\pgfsetlinewidth{1.003750pt}%
\definecolor{currentstroke}{rgb}{0.121569,0.466667,0.705882}%
\pgfsetstrokecolor{currentstroke}%
\pgfsetstrokeopacity{0.373975}%
\pgfsetdash{}{0pt}%
\pgfpathmoveto{\pgfqpoint{1.543771in}{2.104843in}}%
\pgfpathcurveto{\pgfqpoint{1.552008in}{2.104843in}}{\pgfqpoint{1.559908in}{2.108115in}}{\pgfqpoint{1.565732in}{2.113939in}}%
\pgfpathcurveto{\pgfqpoint{1.571555in}{2.119763in}}{\pgfqpoint{1.574828in}{2.127663in}}{\pgfqpoint{1.574828in}{2.135899in}}%
\pgfpathcurveto{\pgfqpoint{1.574828in}{2.144135in}}{\pgfqpoint{1.571555in}{2.152036in}}{\pgfqpoint{1.565732in}{2.157859in}}%
\pgfpathcurveto{\pgfqpoint{1.559908in}{2.163683in}}{\pgfqpoint{1.552008in}{2.166956in}}{\pgfqpoint{1.543771in}{2.166956in}}%
\pgfpathcurveto{\pgfqpoint{1.535535in}{2.166956in}}{\pgfqpoint{1.527635in}{2.163683in}}{\pgfqpoint{1.521811in}{2.157859in}}%
\pgfpathcurveto{\pgfqpoint{1.515987in}{2.152036in}}{\pgfqpoint{1.512715in}{2.144135in}}{\pgfqpoint{1.512715in}{2.135899in}}%
\pgfpathcurveto{\pgfqpoint{1.512715in}{2.127663in}}{\pgfqpoint{1.515987in}{2.119763in}}{\pgfqpoint{1.521811in}{2.113939in}}%
\pgfpathcurveto{\pgfqpoint{1.527635in}{2.108115in}}{\pgfqpoint{1.535535in}{2.104843in}}{\pgfqpoint{1.543771in}{2.104843in}}%
\pgfpathclose%
\pgfusepath{stroke,fill}%
\end{pgfscope}%
\begin{pgfscope}%
\pgfpathrectangle{\pgfqpoint{0.100000in}{0.212622in}}{\pgfqpoint{3.696000in}{3.696000in}}%
\pgfusepath{clip}%
\pgfsetbuttcap%
\pgfsetroundjoin%
\definecolor{currentfill}{rgb}{0.121569,0.466667,0.705882}%
\pgfsetfillcolor{currentfill}%
\pgfsetfillopacity{0.375269}%
\pgfsetlinewidth{1.003750pt}%
\definecolor{currentstroke}{rgb}{0.121569,0.466667,0.705882}%
\pgfsetstrokecolor{currentstroke}%
\pgfsetstrokeopacity{0.375269}%
\pgfsetdash{}{0pt}%
\pgfpathmoveto{\pgfqpoint{1.541316in}{2.104913in}}%
\pgfpathcurveto{\pgfqpoint{1.549553in}{2.104913in}}{\pgfqpoint{1.557453in}{2.108185in}}{\pgfqpoint{1.563277in}{2.114009in}}%
\pgfpathcurveto{\pgfqpoint{1.569101in}{2.119833in}}{\pgfqpoint{1.572373in}{2.127733in}}{\pgfqpoint{1.572373in}{2.135969in}}%
\pgfpathcurveto{\pgfqpoint{1.572373in}{2.144205in}}{\pgfqpoint{1.569101in}{2.152105in}}{\pgfqpoint{1.563277in}{2.157929in}}%
\pgfpathcurveto{\pgfqpoint{1.557453in}{2.163753in}}{\pgfqpoint{1.549553in}{2.167026in}}{\pgfqpoint{1.541316in}{2.167026in}}%
\pgfpathcurveto{\pgfqpoint{1.533080in}{2.167026in}}{\pgfqpoint{1.525180in}{2.163753in}}{\pgfqpoint{1.519356in}{2.157929in}}%
\pgfpathcurveto{\pgfqpoint{1.513532in}{2.152105in}}{\pgfqpoint{1.510260in}{2.144205in}}{\pgfqpoint{1.510260in}{2.135969in}}%
\pgfpathcurveto{\pgfqpoint{1.510260in}{2.127733in}}{\pgfqpoint{1.513532in}{2.119833in}}{\pgfqpoint{1.519356in}{2.114009in}}%
\pgfpathcurveto{\pgfqpoint{1.525180in}{2.108185in}}{\pgfqpoint{1.533080in}{2.104913in}}{\pgfqpoint{1.541316in}{2.104913in}}%
\pgfpathclose%
\pgfusepath{stroke,fill}%
\end{pgfscope}%
\begin{pgfscope}%
\pgfpathrectangle{\pgfqpoint{0.100000in}{0.212622in}}{\pgfqpoint{3.696000in}{3.696000in}}%
\pgfusepath{clip}%
\pgfsetbuttcap%
\pgfsetroundjoin%
\definecolor{currentfill}{rgb}{0.121569,0.466667,0.705882}%
\pgfsetfillcolor{currentfill}%
\pgfsetfillopacity{0.375502}%
\pgfsetlinewidth{1.003750pt}%
\definecolor{currentstroke}{rgb}{0.121569,0.466667,0.705882}%
\pgfsetstrokecolor{currentstroke}%
\pgfsetstrokeopacity{0.375502}%
\pgfsetdash{}{0pt}%
\pgfpathmoveto{\pgfqpoint{2.211906in}{1.997420in}}%
\pgfpathcurveto{\pgfqpoint{2.220143in}{1.997420in}}{\pgfqpoint{2.228043in}{2.000692in}}{\pgfqpoint{2.233866in}{2.006516in}}%
\pgfpathcurveto{\pgfqpoint{2.239690in}{2.012340in}}{\pgfqpoint{2.242963in}{2.020240in}}{\pgfqpoint{2.242963in}{2.028476in}}%
\pgfpathcurveto{\pgfqpoint{2.242963in}{2.036713in}}{\pgfqpoint{2.239690in}{2.044613in}}{\pgfqpoint{2.233866in}{2.050437in}}%
\pgfpathcurveto{\pgfqpoint{2.228043in}{2.056261in}}{\pgfqpoint{2.220143in}{2.059533in}}{\pgfqpoint{2.211906in}{2.059533in}}%
\pgfpathcurveto{\pgfqpoint{2.203670in}{2.059533in}}{\pgfqpoint{2.195770in}{2.056261in}}{\pgfqpoint{2.189946in}{2.050437in}}%
\pgfpathcurveto{\pgfqpoint{2.184122in}{2.044613in}}{\pgfqpoint{2.180850in}{2.036713in}}{\pgfqpoint{2.180850in}{2.028476in}}%
\pgfpathcurveto{\pgfqpoint{2.180850in}{2.020240in}}{\pgfqpoint{2.184122in}{2.012340in}}{\pgfqpoint{2.189946in}{2.006516in}}%
\pgfpathcurveto{\pgfqpoint{2.195770in}{2.000692in}}{\pgfqpoint{2.203670in}{1.997420in}}{\pgfqpoint{2.211906in}{1.997420in}}%
\pgfpathclose%
\pgfusepath{stroke,fill}%
\end{pgfscope}%
\begin{pgfscope}%
\pgfpathrectangle{\pgfqpoint{0.100000in}{0.212622in}}{\pgfqpoint{3.696000in}{3.696000in}}%
\pgfusepath{clip}%
\pgfsetbuttcap%
\pgfsetroundjoin%
\definecolor{currentfill}{rgb}{0.121569,0.466667,0.705882}%
\pgfsetfillcolor{currentfill}%
\pgfsetfillopacity{0.376450}%
\pgfsetlinewidth{1.003750pt}%
\definecolor{currentstroke}{rgb}{0.121569,0.466667,0.705882}%
\pgfsetstrokecolor{currentstroke}%
\pgfsetstrokeopacity{0.376450}%
\pgfsetdash{}{0pt}%
\pgfpathmoveto{\pgfqpoint{1.538494in}{2.105198in}}%
\pgfpathcurveto{\pgfqpoint{1.546731in}{2.105198in}}{\pgfqpoint{1.554631in}{2.108470in}}{\pgfqpoint{1.560455in}{2.114294in}}%
\pgfpathcurveto{\pgfqpoint{1.566278in}{2.120118in}}{\pgfqpoint{1.569551in}{2.128018in}}{\pgfqpoint{1.569551in}{2.136255in}}%
\pgfpathcurveto{\pgfqpoint{1.569551in}{2.144491in}}{\pgfqpoint{1.566278in}{2.152391in}}{\pgfqpoint{1.560455in}{2.158215in}}%
\pgfpathcurveto{\pgfqpoint{1.554631in}{2.164039in}}{\pgfqpoint{1.546731in}{2.167311in}}{\pgfqpoint{1.538494in}{2.167311in}}%
\pgfpathcurveto{\pgfqpoint{1.530258in}{2.167311in}}{\pgfqpoint{1.522358in}{2.164039in}}{\pgfqpoint{1.516534in}{2.158215in}}%
\pgfpathcurveto{\pgfqpoint{1.510710in}{2.152391in}}{\pgfqpoint{1.507438in}{2.144491in}}{\pgfqpoint{1.507438in}{2.136255in}}%
\pgfpathcurveto{\pgfqpoint{1.507438in}{2.128018in}}{\pgfqpoint{1.510710in}{2.120118in}}{\pgfqpoint{1.516534in}{2.114294in}}%
\pgfpathcurveto{\pgfqpoint{1.522358in}{2.108470in}}{\pgfqpoint{1.530258in}{2.105198in}}{\pgfqpoint{1.538494in}{2.105198in}}%
\pgfpathclose%
\pgfusepath{stroke,fill}%
\end{pgfscope}%
\begin{pgfscope}%
\pgfpathrectangle{\pgfqpoint{0.100000in}{0.212622in}}{\pgfqpoint{3.696000in}{3.696000in}}%
\pgfusepath{clip}%
\pgfsetbuttcap%
\pgfsetroundjoin%
\definecolor{currentfill}{rgb}{0.121569,0.466667,0.705882}%
\pgfsetfillcolor{currentfill}%
\pgfsetfillopacity{0.377039}%
\pgfsetlinewidth{1.003750pt}%
\definecolor{currentstroke}{rgb}{0.121569,0.466667,0.705882}%
\pgfsetstrokecolor{currentstroke}%
\pgfsetstrokeopacity{0.377039}%
\pgfsetdash{}{0pt}%
\pgfpathmoveto{\pgfqpoint{2.219209in}{1.996181in}}%
\pgfpathcurveto{\pgfqpoint{2.227445in}{1.996181in}}{\pgfqpoint{2.235345in}{1.999453in}}{\pgfqpoint{2.241169in}{2.005277in}}%
\pgfpathcurveto{\pgfqpoint{2.246993in}{2.011101in}}{\pgfqpoint{2.250265in}{2.019001in}}{\pgfqpoint{2.250265in}{2.027237in}}%
\pgfpathcurveto{\pgfqpoint{2.250265in}{2.035474in}}{\pgfqpoint{2.246993in}{2.043374in}}{\pgfqpoint{2.241169in}{2.049198in}}%
\pgfpathcurveto{\pgfqpoint{2.235345in}{2.055022in}}{\pgfqpoint{2.227445in}{2.058294in}}{\pgfqpoint{2.219209in}{2.058294in}}%
\pgfpathcurveto{\pgfqpoint{2.210973in}{2.058294in}}{\pgfqpoint{2.203073in}{2.055022in}}{\pgfqpoint{2.197249in}{2.049198in}}%
\pgfpathcurveto{\pgfqpoint{2.191425in}{2.043374in}}{\pgfqpoint{2.188152in}{2.035474in}}{\pgfqpoint{2.188152in}{2.027237in}}%
\pgfpathcurveto{\pgfqpoint{2.188152in}{2.019001in}}{\pgfqpoint{2.191425in}{2.011101in}}{\pgfqpoint{2.197249in}{2.005277in}}%
\pgfpathcurveto{\pgfqpoint{2.203073in}{1.999453in}}{\pgfqpoint{2.210973in}{1.996181in}}{\pgfqpoint{2.219209in}{1.996181in}}%
\pgfpathclose%
\pgfusepath{stroke,fill}%
\end{pgfscope}%
\begin{pgfscope}%
\pgfpathrectangle{\pgfqpoint{0.100000in}{0.212622in}}{\pgfqpoint{3.696000in}{3.696000in}}%
\pgfusepath{clip}%
\pgfsetbuttcap%
\pgfsetroundjoin%
\definecolor{currentfill}{rgb}{0.121569,0.466667,0.705882}%
\pgfsetfillcolor{currentfill}%
\pgfsetfillopacity{0.377341}%
\pgfsetlinewidth{1.003750pt}%
\definecolor{currentstroke}{rgb}{0.121569,0.466667,0.705882}%
\pgfsetstrokecolor{currentstroke}%
\pgfsetstrokeopacity{0.377341}%
\pgfsetdash{}{0pt}%
\pgfpathmoveto{\pgfqpoint{1.536823in}{2.105235in}}%
\pgfpathcurveto{\pgfqpoint{1.545060in}{2.105235in}}{\pgfqpoint{1.552960in}{2.108508in}}{\pgfqpoint{1.558784in}{2.114332in}}%
\pgfpathcurveto{\pgfqpoint{1.564608in}{2.120155in}}{\pgfqpoint{1.567880in}{2.128056in}}{\pgfqpoint{1.567880in}{2.136292in}}%
\pgfpathcurveto{\pgfqpoint{1.567880in}{2.144528in}}{\pgfqpoint{1.564608in}{2.152428in}}{\pgfqpoint{1.558784in}{2.158252in}}%
\pgfpathcurveto{\pgfqpoint{1.552960in}{2.164076in}}{\pgfqpoint{1.545060in}{2.167348in}}{\pgfqpoint{1.536823in}{2.167348in}}%
\pgfpathcurveto{\pgfqpoint{1.528587in}{2.167348in}}{\pgfqpoint{1.520687in}{2.164076in}}{\pgfqpoint{1.514863in}{2.158252in}}%
\pgfpathcurveto{\pgfqpoint{1.509039in}{2.152428in}}{\pgfqpoint{1.505767in}{2.144528in}}{\pgfqpoint{1.505767in}{2.136292in}}%
\pgfpathcurveto{\pgfqpoint{1.505767in}{2.128056in}}{\pgfqpoint{1.509039in}{2.120155in}}{\pgfqpoint{1.514863in}{2.114332in}}%
\pgfpathcurveto{\pgfqpoint{1.520687in}{2.108508in}}{\pgfqpoint{1.528587in}{2.105235in}}{\pgfqpoint{1.536823in}{2.105235in}}%
\pgfpathclose%
\pgfusepath{stroke,fill}%
\end{pgfscope}%
\begin{pgfscope}%
\pgfpathrectangle{\pgfqpoint{0.100000in}{0.212622in}}{\pgfqpoint{3.696000in}{3.696000in}}%
\pgfusepath{clip}%
\pgfsetbuttcap%
\pgfsetroundjoin%
\definecolor{currentfill}{rgb}{0.121569,0.466667,0.705882}%
\pgfsetfillcolor{currentfill}%
\pgfsetfillopacity{0.377641}%
\pgfsetlinewidth{1.003750pt}%
\definecolor{currentstroke}{rgb}{0.121569,0.466667,0.705882}%
\pgfsetstrokecolor{currentstroke}%
\pgfsetstrokeopacity{0.377641}%
\pgfsetdash{}{0pt}%
\pgfpathmoveto{\pgfqpoint{1.536224in}{2.105271in}}%
\pgfpathcurveto{\pgfqpoint{1.544460in}{2.105271in}}{\pgfqpoint{1.552360in}{2.108544in}}{\pgfqpoint{1.558184in}{2.114368in}}%
\pgfpathcurveto{\pgfqpoint{1.564008in}{2.120192in}}{\pgfqpoint{1.567280in}{2.128092in}}{\pgfqpoint{1.567280in}{2.136328in}}%
\pgfpathcurveto{\pgfqpoint{1.567280in}{2.144564in}}{\pgfqpoint{1.564008in}{2.152464in}}{\pgfqpoint{1.558184in}{2.158288in}}%
\pgfpathcurveto{\pgfqpoint{1.552360in}{2.164112in}}{\pgfqpoint{1.544460in}{2.167384in}}{\pgfqpoint{1.536224in}{2.167384in}}%
\pgfpathcurveto{\pgfqpoint{1.527988in}{2.167384in}}{\pgfqpoint{1.520088in}{2.164112in}}{\pgfqpoint{1.514264in}{2.158288in}}%
\pgfpathcurveto{\pgfqpoint{1.508440in}{2.152464in}}{\pgfqpoint{1.505167in}{2.144564in}}{\pgfqpoint{1.505167in}{2.136328in}}%
\pgfpathcurveto{\pgfqpoint{1.505167in}{2.128092in}}{\pgfqpoint{1.508440in}{2.120192in}}{\pgfqpoint{1.514264in}{2.114368in}}%
\pgfpathcurveto{\pgfqpoint{1.520088in}{2.108544in}}{\pgfqpoint{1.527988in}{2.105271in}}{\pgfqpoint{1.536224in}{2.105271in}}%
\pgfpathclose%
\pgfusepath{stroke,fill}%
\end{pgfscope}%
\begin{pgfscope}%
\pgfpathrectangle{\pgfqpoint{0.100000in}{0.212622in}}{\pgfqpoint{3.696000in}{3.696000in}}%
\pgfusepath{clip}%
\pgfsetbuttcap%
\pgfsetroundjoin%
\definecolor{currentfill}{rgb}{0.121569,0.466667,0.705882}%
\pgfsetfillcolor{currentfill}%
\pgfsetfillopacity{0.378165}%
\pgfsetlinewidth{1.003750pt}%
\definecolor{currentstroke}{rgb}{0.121569,0.466667,0.705882}%
\pgfsetstrokecolor{currentstroke}%
\pgfsetstrokeopacity{0.378165}%
\pgfsetdash{}{0pt}%
\pgfpathmoveto{\pgfqpoint{1.534978in}{2.105330in}}%
\pgfpathcurveto{\pgfqpoint{1.543214in}{2.105330in}}{\pgfqpoint{1.551114in}{2.108602in}}{\pgfqpoint{1.556938in}{2.114426in}}%
\pgfpathcurveto{\pgfqpoint{1.562762in}{2.120250in}}{\pgfqpoint{1.566034in}{2.128150in}}{\pgfqpoint{1.566034in}{2.136386in}}%
\pgfpathcurveto{\pgfqpoint{1.566034in}{2.144623in}}{\pgfqpoint{1.562762in}{2.152523in}}{\pgfqpoint{1.556938in}{2.158346in}}%
\pgfpathcurveto{\pgfqpoint{1.551114in}{2.164170in}}{\pgfqpoint{1.543214in}{2.167443in}}{\pgfqpoint{1.534978in}{2.167443in}}%
\pgfpathcurveto{\pgfqpoint{1.526741in}{2.167443in}}{\pgfqpoint{1.518841in}{2.164170in}}{\pgfqpoint{1.513017in}{2.158346in}}%
\pgfpathcurveto{\pgfqpoint{1.507194in}{2.152523in}}{\pgfqpoint{1.503921in}{2.144623in}}{\pgfqpoint{1.503921in}{2.136386in}}%
\pgfpathcurveto{\pgfqpoint{1.503921in}{2.128150in}}{\pgfqpoint{1.507194in}{2.120250in}}{\pgfqpoint{1.513017in}{2.114426in}}%
\pgfpathcurveto{\pgfqpoint{1.518841in}{2.108602in}}{\pgfqpoint{1.526741in}{2.105330in}}{\pgfqpoint{1.534978in}{2.105330in}}%
\pgfpathclose%
\pgfusepath{stroke,fill}%
\end{pgfscope}%
\begin{pgfscope}%
\pgfpathrectangle{\pgfqpoint{0.100000in}{0.212622in}}{\pgfqpoint{3.696000in}{3.696000in}}%
\pgfusepath{clip}%
\pgfsetbuttcap%
\pgfsetroundjoin%
\definecolor{currentfill}{rgb}{0.121569,0.466667,0.705882}%
\pgfsetfillcolor{currentfill}%
\pgfsetfillopacity{0.378617}%
\pgfsetlinewidth{1.003750pt}%
\definecolor{currentstroke}{rgb}{0.121569,0.466667,0.705882}%
\pgfsetstrokecolor{currentstroke}%
\pgfsetstrokeopacity{0.378617}%
\pgfsetdash{}{0pt}%
\pgfpathmoveto{\pgfqpoint{1.534336in}{2.105355in}}%
\pgfpathcurveto{\pgfqpoint{1.542572in}{2.105355in}}{\pgfqpoint{1.550472in}{2.108628in}}{\pgfqpoint{1.556296in}{2.114452in}}%
\pgfpathcurveto{\pgfqpoint{1.562120in}{2.120275in}}{\pgfqpoint{1.565392in}{2.128176in}}{\pgfqpoint{1.565392in}{2.136412in}}%
\pgfpathcurveto{\pgfqpoint{1.565392in}{2.144648in}}{\pgfqpoint{1.562120in}{2.152548in}}{\pgfqpoint{1.556296in}{2.158372in}}%
\pgfpathcurveto{\pgfqpoint{1.550472in}{2.164196in}}{\pgfqpoint{1.542572in}{2.167468in}}{\pgfqpoint{1.534336in}{2.167468in}}%
\pgfpathcurveto{\pgfqpoint{1.526099in}{2.167468in}}{\pgfqpoint{1.518199in}{2.164196in}}{\pgfqpoint{1.512375in}{2.158372in}}%
\pgfpathcurveto{\pgfqpoint{1.506551in}{2.152548in}}{\pgfqpoint{1.503279in}{2.144648in}}{\pgfqpoint{1.503279in}{2.136412in}}%
\pgfpathcurveto{\pgfqpoint{1.503279in}{2.128176in}}{\pgfqpoint{1.506551in}{2.120275in}}{\pgfqpoint{1.512375in}{2.114452in}}%
\pgfpathcurveto{\pgfqpoint{1.518199in}{2.108628in}}{\pgfqpoint{1.526099in}{2.105355in}}{\pgfqpoint{1.534336in}{2.105355in}}%
\pgfpathclose%
\pgfusepath{stroke,fill}%
\end{pgfscope}%
\begin{pgfscope}%
\pgfpathrectangle{\pgfqpoint{0.100000in}{0.212622in}}{\pgfqpoint{3.696000in}{3.696000in}}%
\pgfusepath{clip}%
\pgfsetbuttcap%
\pgfsetroundjoin%
\definecolor{currentfill}{rgb}{0.121569,0.466667,0.705882}%
\pgfsetfillcolor{currentfill}%
\pgfsetfillopacity{0.378988}%
\pgfsetlinewidth{1.003750pt}%
\definecolor{currentstroke}{rgb}{0.121569,0.466667,0.705882}%
\pgfsetstrokecolor{currentstroke}%
\pgfsetstrokeopacity{0.378988}%
\pgfsetdash{}{0pt}%
\pgfpathmoveto{\pgfqpoint{1.533858in}{2.105387in}}%
\pgfpathcurveto{\pgfqpoint{1.542095in}{2.105387in}}{\pgfqpoint{1.549995in}{2.108659in}}{\pgfqpoint{1.555819in}{2.114483in}}%
\pgfpathcurveto{\pgfqpoint{1.561643in}{2.120307in}}{\pgfqpoint{1.564915in}{2.128207in}}{\pgfqpoint{1.564915in}{2.136444in}}%
\pgfpathcurveto{\pgfqpoint{1.564915in}{2.144680in}}{\pgfqpoint{1.561643in}{2.152580in}}{\pgfqpoint{1.555819in}{2.158404in}}%
\pgfpathcurveto{\pgfqpoint{1.549995in}{2.164228in}}{\pgfqpoint{1.542095in}{2.167500in}}{\pgfqpoint{1.533858in}{2.167500in}}%
\pgfpathcurveto{\pgfqpoint{1.525622in}{2.167500in}}{\pgfqpoint{1.517722in}{2.164228in}}{\pgfqpoint{1.511898in}{2.158404in}}%
\pgfpathcurveto{\pgfqpoint{1.506074in}{2.152580in}}{\pgfqpoint{1.502802in}{2.144680in}}{\pgfqpoint{1.502802in}{2.136444in}}%
\pgfpathcurveto{\pgfqpoint{1.502802in}{2.128207in}}{\pgfqpoint{1.506074in}{2.120307in}}{\pgfqpoint{1.511898in}{2.114483in}}%
\pgfpathcurveto{\pgfqpoint{1.517722in}{2.108659in}}{\pgfqpoint{1.525622in}{2.105387in}}{\pgfqpoint{1.533858in}{2.105387in}}%
\pgfpathclose%
\pgfusepath{stroke,fill}%
\end{pgfscope}%
\begin{pgfscope}%
\pgfpathrectangle{\pgfqpoint{0.100000in}{0.212622in}}{\pgfqpoint{3.696000in}{3.696000in}}%
\pgfusepath{clip}%
\pgfsetbuttcap%
\pgfsetroundjoin%
\definecolor{currentfill}{rgb}{0.121569,0.466667,0.705882}%
\pgfsetfillcolor{currentfill}%
\pgfsetfillopacity{0.379115}%
\pgfsetlinewidth{1.003750pt}%
\definecolor{currentstroke}{rgb}{0.121569,0.466667,0.705882}%
\pgfsetstrokecolor{currentstroke}%
\pgfsetstrokeopacity{0.379115}%
\pgfsetdash{}{0pt}%
\pgfpathmoveto{\pgfqpoint{2.228392in}{1.994867in}}%
\pgfpathcurveto{\pgfqpoint{2.236628in}{1.994867in}}{\pgfqpoint{2.244528in}{1.998140in}}{\pgfqpoint{2.250352in}{2.003964in}}%
\pgfpathcurveto{\pgfqpoint{2.256176in}{2.009787in}}{\pgfqpoint{2.259449in}{2.017688in}}{\pgfqpoint{2.259449in}{2.025924in}}%
\pgfpathcurveto{\pgfqpoint{2.259449in}{2.034160in}}{\pgfqpoint{2.256176in}{2.042060in}}{\pgfqpoint{2.250352in}{2.047884in}}%
\pgfpathcurveto{\pgfqpoint{2.244528in}{2.053708in}}{\pgfqpoint{2.236628in}{2.056980in}}{\pgfqpoint{2.228392in}{2.056980in}}%
\pgfpathcurveto{\pgfqpoint{2.220156in}{2.056980in}}{\pgfqpoint{2.212256in}{2.053708in}}{\pgfqpoint{2.206432in}{2.047884in}}%
\pgfpathcurveto{\pgfqpoint{2.200608in}{2.042060in}}{\pgfqpoint{2.197336in}{2.034160in}}{\pgfqpoint{2.197336in}{2.025924in}}%
\pgfpathcurveto{\pgfqpoint{2.197336in}{2.017688in}}{\pgfqpoint{2.200608in}{2.009787in}}{\pgfqpoint{2.206432in}{2.003964in}}%
\pgfpathcurveto{\pgfqpoint{2.212256in}{1.998140in}}{\pgfqpoint{2.220156in}{1.994867in}}{\pgfqpoint{2.228392in}{1.994867in}}%
\pgfpathclose%
\pgfusepath{stroke,fill}%
\end{pgfscope}%
\begin{pgfscope}%
\pgfpathrectangle{\pgfqpoint{0.100000in}{0.212622in}}{\pgfqpoint{3.696000in}{3.696000in}}%
\pgfusepath{clip}%
\pgfsetbuttcap%
\pgfsetroundjoin%
\definecolor{currentfill}{rgb}{0.121569,0.466667,0.705882}%
\pgfsetfillcolor{currentfill}%
\pgfsetfillopacity{0.379574}%
\pgfsetlinewidth{1.003750pt}%
\definecolor{currentstroke}{rgb}{0.121569,0.466667,0.705882}%
\pgfsetstrokecolor{currentstroke}%
\pgfsetstrokeopacity{0.379574}%
\pgfsetdash{}{0pt}%
\pgfpathmoveto{\pgfqpoint{1.532213in}{2.105533in}}%
\pgfpathcurveto{\pgfqpoint{1.540450in}{2.105533in}}{\pgfqpoint{1.548350in}{2.108806in}}{\pgfqpoint{1.554174in}{2.114630in}}%
\pgfpathcurveto{\pgfqpoint{1.559998in}{2.120454in}}{\pgfqpoint{1.563270in}{2.128354in}}{\pgfqpoint{1.563270in}{2.136590in}}%
\pgfpathcurveto{\pgfqpoint{1.563270in}{2.144826in}}{\pgfqpoint{1.559998in}{2.152726in}}{\pgfqpoint{1.554174in}{2.158550in}}%
\pgfpathcurveto{\pgfqpoint{1.548350in}{2.164374in}}{\pgfqpoint{1.540450in}{2.167646in}}{\pgfqpoint{1.532213in}{2.167646in}}%
\pgfpathcurveto{\pgfqpoint{1.523977in}{2.167646in}}{\pgfqpoint{1.516077in}{2.164374in}}{\pgfqpoint{1.510253in}{2.158550in}}%
\pgfpathcurveto{\pgfqpoint{1.504429in}{2.152726in}}{\pgfqpoint{1.501157in}{2.144826in}}{\pgfqpoint{1.501157in}{2.136590in}}%
\pgfpathcurveto{\pgfqpoint{1.501157in}{2.128354in}}{\pgfqpoint{1.504429in}{2.120454in}}{\pgfqpoint{1.510253in}{2.114630in}}%
\pgfpathcurveto{\pgfqpoint{1.516077in}{2.108806in}}{\pgfqpoint{1.523977in}{2.105533in}}{\pgfqpoint{1.532213in}{2.105533in}}%
\pgfpathclose%
\pgfusepath{stroke,fill}%
\end{pgfscope}%
\begin{pgfscope}%
\pgfpathrectangle{\pgfqpoint{0.100000in}{0.212622in}}{\pgfqpoint{3.696000in}{3.696000in}}%
\pgfusepath{clip}%
\pgfsetbuttcap%
\pgfsetroundjoin%
\definecolor{currentfill}{rgb}{0.121569,0.466667,0.705882}%
\pgfsetfillcolor{currentfill}%
\pgfsetfillopacity{0.380758}%
\pgfsetlinewidth{1.003750pt}%
\definecolor{currentstroke}{rgb}{0.121569,0.466667,0.705882}%
\pgfsetstrokecolor{currentstroke}%
\pgfsetstrokeopacity{0.380758}%
\pgfsetdash{}{0pt}%
\pgfpathmoveto{\pgfqpoint{1.530259in}{2.105573in}}%
\pgfpathcurveto{\pgfqpoint{1.538495in}{2.105573in}}{\pgfqpoint{1.546395in}{2.108846in}}{\pgfqpoint{1.552219in}{2.114670in}}%
\pgfpathcurveto{\pgfqpoint{1.558043in}{2.120493in}}{\pgfqpoint{1.561316in}{2.128393in}}{\pgfqpoint{1.561316in}{2.136630in}}%
\pgfpathcurveto{\pgfqpoint{1.561316in}{2.144866in}}{\pgfqpoint{1.558043in}{2.152766in}}{\pgfqpoint{1.552219in}{2.158590in}}%
\pgfpathcurveto{\pgfqpoint{1.546395in}{2.164414in}}{\pgfqpoint{1.538495in}{2.167686in}}{\pgfqpoint{1.530259in}{2.167686in}}%
\pgfpathcurveto{\pgfqpoint{1.522023in}{2.167686in}}{\pgfqpoint{1.514123in}{2.164414in}}{\pgfqpoint{1.508299in}{2.158590in}}%
\pgfpathcurveto{\pgfqpoint{1.502475in}{2.152766in}}{\pgfqpoint{1.499203in}{2.144866in}}{\pgfqpoint{1.499203in}{2.136630in}}%
\pgfpathcurveto{\pgfqpoint{1.499203in}{2.128393in}}{\pgfqpoint{1.502475in}{2.120493in}}{\pgfqpoint{1.508299in}{2.114670in}}%
\pgfpathcurveto{\pgfqpoint{1.514123in}{2.108846in}}{\pgfqpoint{1.522023in}{2.105573in}}{\pgfqpoint{1.530259in}{2.105573in}}%
\pgfpathclose%
\pgfusepath{stroke,fill}%
\end{pgfscope}%
\begin{pgfscope}%
\pgfpathrectangle{\pgfqpoint{0.100000in}{0.212622in}}{\pgfqpoint{3.696000in}{3.696000in}}%
\pgfusepath{clip}%
\pgfsetbuttcap%
\pgfsetroundjoin%
\definecolor{currentfill}{rgb}{0.121569,0.466667,0.705882}%
\pgfsetfillcolor{currentfill}%
\pgfsetfillopacity{0.381176}%
\pgfsetlinewidth{1.003750pt}%
\definecolor{currentstroke}{rgb}{0.121569,0.466667,0.705882}%
\pgfsetstrokecolor{currentstroke}%
\pgfsetstrokeopacity{0.381176}%
\pgfsetdash{}{0pt}%
\pgfpathmoveto{\pgfqpoint{2.238481in}{1.993309in}}%
\pgfpathcurveto{\pgfqpoint{2.246717in}{1.993309in}}{\pgfqpoint{2.254617in}{1.996581in}}{\pgfqpoint{2.260441in}{2.002405in}}%
\pgfpathcurveto{\pgfqpoint{2.266265in}{2.008229in}}{\pgfqpoint{2.269537in}{2.016129in}}{\pgfqpoint{2.269537in}{2.024365in}}%
\pgfpathcurveto{\pgfqpoint{2.269537in}{2.032602in}}{\pgfqpoint{2.266265in}{2.040502in}}{\pgfqpoint{2.260441in}{2.046326in}}%
\pgfpathcurveto{\pgfqpoint{2.254617in}{2.052149in}}{\pgfqpoint{2.246717in}{2.055422in}}{\pgfqpoint{2.238481in}{2.055422in}}%
\pgfpathcurveto{\pgfqpoint{2.230244in}{2.055422in}}{\pgfqpoint{2.222344in}{2.052149in}}{\pgfqpoint{2.216520in}{2.046326in}}%
\pgfpathcurveto{\pgfqpoint{2.210696in}{2.040502in}}{\pgfqpoint{2.207424in}{2.032602in}}{\pgfqpoint{2.207424in}{2.024365in}}%
\pgfpathcurveto{\pgfqpoint{2.207424in}{2.016129in}}{\pgfqpoint{2.210696in}{2.008229in}}{\pgfqpoint{2.216520in}{2.002405in}}%
\pgfpathcurveto{\pgfqpoint{2.222344in}{1.996581in}}{\pgfqpoint{2.230244in}{1.993309in}}{\pgfqpoint{2.238481in}{1.993309in}}%
\pgfpathclose%
\pgfusepath{stroke,fill}%
\end{pgfscope}%
\begin{pgfscope}%
\pgfpathrectangle{\pgfqpoint{0.100000in}{0.212622in}}{\pgfqpoint{3.696000in}{3.696000in}}%
\pgfusepath{clip}%
\pgfsetbuttcap%
\pgfsetroundjoin%
\definecolor{currentfill}{rgb}{0.121569,0.466667,0.705882}%
\pgfsetfillcolor{currentfill}%
\pgfsetfillopacity{0.381824}%
\pgfsetlinewidth{1.003750pt}%
\definecolor{currentstroke}{rgb}{0.121569,0.466667,0.705882}%
\pgfsetstrokecolor{currentstroke}%
\pgfsetstrokeopacity{0.381824}%
\pgfsetdash{}{0pt}%
\pgfpathmoveto{\pgfqpoint{1.528196in}{2.105619in}}%
\pgfpathcurveto{\pgfqpoint{1.536432in}{2.105619in}}{\pgfqpoint{1.544332in}{2.108892in}}{\pgfqpoint{1.550156in}{2.114716in}}%
\pgfpathcurveto{\pgfqpoint{1.555980in}{2.120539in}}{\pgfqpoint{1.559252in}{2.128439in}}{\pgfqpoint{1.559252in}{2.136676in}}%
\pgfpathcurveto{\pgfqpoint{1.559252in}{2.144912in}}{\pgfqpoint{1.555980in}{2.152812in}}{\pgfqpoint{1.550156in}{2.158636in}}%
\pgfpathcurveto{\pgfqpoint{1.544332in}{2.164460in}}{\pgfqpoint{1.536432in}{2.167732in}}{\pgfqpoint{1.528196in}{2.167732in}}%
\pgfpathcurveto{\pgfqpoint{1.519959in}{2.167732in}}{\pgfqpoint{1.512059in}{2.164460in}}{\pgfqpoint{1.506235in}{2.158636in}}%
\pgfpathcurveto{\pgfqpoint{1.500411in}{2.152812in}}{\pgfqpoint{1.497139in}{2.144912in}}{\pgfqpoint{1.497139in}{2.136676in}}%
\pgfpathcurveto{\pgfqpoint{1.497139in}{2.128439in}}{\pgfqpoint{1.500411in}{2.120539in}}{\pgfqpoint{1.506235in}{2.114716in}}%
\pgfpathcurveto{\pgfqpoint{1.512059in}{2.108892in}}{\pgfqpoint{1.519959in}{2.105619in}}{\pgfqpoint{1.528196in}{2.105619in}}%
\pgfpathclose%
\pgfusepath{stroke,fill}%
\end{pgfscope}%
\begin{pgfscope}%
\pgfpathrectangle{\pgfqpoint{0.100000in}{0.212622in}}{\pgfqpoint{3.696000in}{3.696000in}}%
\pgfusepath{clip}%
\pgfsetbuttcap%
\pgfsetroundjoin%
\definecolor{currentfill}{rgb}{0.121569,0.466667,0.705882}%
\pgfsetfillcolor{currentfill}%
\pgfsetfillopacity{0.382605}%
\pgfsetlinewidth{1.003750pt}%
\definecolor{currentstroke}{rgb}{0.121569,0.466667,0.705882}%
\pgfsetstrokecolor{currentstroke}%
\pgfsetstrokeopacity{0.382605}%
\pgfsetdash{}{0pt}%
\pgfpathmoveto{\pgfqpoint{1.526305in}{2.105759in}}%
\pgfpathcurveto{\pgfqpoint{1.534541in}{2.105759in}}{\pgfqpoint{1.542442in}{2.109031in}}{\pgfqpoint{1.548265in}{2.114855in}}%
\pgfpathcurveto{\pgfqpoint{1.554089in}{2.120679in}}{\pgfqpoint{1.557362in}{2.128579in}}{\pgfqpoint{1.557362in}{2.136815in}}%
\pgfpathcurveto{\pgfqpoint{1.557362in}{2.145051in}}{\pgfqpoint{1.554089in}{2.152951in}}{\pgfqpoint{1.548265in}{2.158775in}}%
\pgfpathcurveto{\pgfqpoint{1.542442in}{2.164599in}}{\pgfqpoint{1.534541in}{2.167872in}}{\pgfqpoint{1.526305in}{2.167872in}}%
\pgfpathcurveto{\pgfqpoint{1.518069in}{2.167872in}}{\pgfqpoint{1.510169in}{2.164599in}}{\pgfqpoint{1.504345in}{2.158775in}}%
\pgfpathcurveto{\pgfqpoint{1.498521in}{2.152951in}}{\pgfqpoint{1.495249in}{2.145051in}}{\pgfqpoint{1.495249in}{2.136815in}}%
\pgfpathcurveto{\pgfqpoint{1.495249in}{2.128579in}}{\pgfqpoint{1.498521in}{2.120679in}}{\pgfqpoint{1.504345in}{2.114855in}}%
\pgfpathcurveto{\pgfqpoint{1.510169in}{2.109031in}}{\pgfqpoint{1.518069in}{2.105759in}}{\pgfqpoint{1.526305in}{2.105759in}}%
\pgfpathclose%
\pgfusepath{stroke,fill}%
\end{pgfscope}%
\begin{pgfscope}%
\pgfpathrectangle{\pgfqpoint{0.100000in}{0.212622in}}{\pgfqpoint{3.696000in}{3.696000in}}%
\pgfusepath{clip}%
\pgfsetbuttcap%
\pgfsetroundjoin%
\definecolor{currentfill}{rgb}{0.121569,0.466667,0.705882}%
\pgfsetfillcolor{currentfill}%
\pgfsetfillopacity{0.383324}%
\pgfsetlinewidth{1.003750pt}%
\definecolor{currentstroke}{rgb}{0.121569,0.466667,0.705882}%
\pgfsetstrokecolor{currentstroke}%
\pgfsetstrokeopacity{0.383324}%
\pgfsetdash{}{0pt}%
\pgfpathmoveto{\pgfqpoint{1.525118in}{2.105603in}}%
\pgfpathcurveto{\pgfqpoint{1.533354in}{2.105603in}}{\pgfqpoint{1.541254in}{2.108875in}}{\pgfqpoint{1.547078in}{2.114699in}}%
\pgfpathcurveto{\pgfqpoint{1.552902in}{2.120523in}}{\pgfqpoint{1.556174in}{2.128423in}}{\pgfqpoint{1.556174in}{2.136659in}}%
\pgfpathcurveto{\pgfqpoint{1.556174in}{2.144896in}}{\pgfqpoint{1.552902in}{2.152796in}}{\pgfqpoint{1.547078in}{2.158620in}}%
\pgfpathcurveto{\pgfqpoint{1.541254in}{2.164444in}}{\pgfqpoint{1.533354in}{2.167716in}}{\pgfqpoint{1.525118in}{2.167716in}}%
\pgfpathcurveto{\pgfqpoint{1.516881in}{2.167716in}}{\pgfqpoint{1.508981in}{2.164444in}}{\pgfqpoint{1.503157in}{2.158620in}}%
\pgfpathcurveto{\pgfqpoint{1.497333in}{2.152796in}}{\pgfqpoint{1.494061in}{2.144896in}}{\pgfqpoint{1.494061in}{2.136659in}}%
\pgfpathcurveto{\pgfqpoint{1.494061in}{2.128423in}}{\pgfqpoint{1.497333in}{2.120523in}}{\pgfqpoint{1.503157in}{2.114699in}}%
\pgfpathcurveto{\pgfqpoint{1.508981in}{2.108875in}}{\pgfqpoint{1.516881in}{2.105603in}}{\pgfqpoint{1.525118in}{2.105603in}}%
\pgfpathclose%
\pgfusepath{stroke,fill}%
\end{pgfscope}%
\begin{pgfscope}%
\pgfpathrectangle{\pgfqpoint{0.100000in}{0.212622in}}{\pgfqpoint{3.696000in}{3.696000in}}%
\pgfusepath{clip}%
\pgfsetbuttcap%
\pgfsetroundjoin%
\definecolor{currentfill}{rgb}{0.121569,0.466667,0.705882}%
\pgfsetfillcolor{currentfill}%
\pgfsetfillopacity{0.384080}%
\pgfsetlinewidth{1.003750pt}%
\definecolor{currentstroke}{rgb}{0.121569,0.466667,0.705882}%
\pgfsetstrokecolor{currentstroke}%
\pgfsetstrokeopacity{0.384080}%
\pgfsetdash{}{0pt}%
\pgfpathmoveto{\pgfqpoint{2.247897in}{1.992264in}}%
\pgfpathcurveto{\pgfqpoint{2.256133in}{1.992264in}}{\pgfqpoint{2.264033in}{1.995536in}}{\pgfqpoint{2.269857in}{2.001360in}}%
\pgfpathcurveto{\pgfqpoint{2.275681in}{2.007184in}}{\pgfqpoint{2.278953in}{2.015084in}}{\pgfqpoint{2.278953in}{2.023320in}}%
\pgfpathcurveto{\pgfqpoint{2.278953in}{2.031556in}}{\pgfqpoint{2.275681in}{2.039457in}}{\pgfqpoint{2.269857in}{2.045280in}}%
\pgfpathcurveto{\pgfqpoint{2.264033in}{2.051104in}}{\pgfqpoint{2.256133in}{2.054377in}}{\pgfqpoint{2.247897in}{2.054377in}}%
\pgfpathcurveto{\pgfqpoint{2.239660in}{2.054377in}}{\pgfqpoint{2.231760in}{2.051104in}}{\pgfqpoint{2.225936in}{2.045280in}}%
\pgfpathcurveto{\pgfqpoint{2.220112in}{2.039457in}}{\pgfqpoint{2.216840in}{2.031556in}}{\pgfqpoint{2.216840in}{2.023320in}}%
\pgfpathcurveto{\pgfqpoint{2.216840in}{2.015084in}}{\pgfqpoint{2.220112in}{2.007184in}}{\pgfqpoint{2.225936in}{2.001360in}}%
\pgfpathcurveto{\pgfqpoint{2.231760in}{1.995536in}}{\pgfqpoint{2.239660in}{1.992264in}}{\pgfqpoint{2.247897in}{1.992264in}}%
\pgfpathclose%
\pgfusepath{stroke,fill}%
\end{pgfscope}%
\begin{pgfscope}%
\pgfpathrectangle{\pgfqpoint{0.100000in}{0.212622in}}{\pgfqpoint{3.696000in}{3.696000in}}%
\pgfusepath{clip}%
\pgfsetbuttcap%
\pgfsetroundjoin%
\definecolor{currentfill}{rgb}{0.121569,0.466667,0.705882}%
\pgfsetfillcolor{currentfill}%
\pgfsetfillopacity{0.384635}%
\pgfsetlinewidth{1.003750pt}%
\definecolor{currentstroke}{rgb}{0.121569,0.466667,0.705882}%
\pgfsetstrokecolor{currentstroke}%
\pgfsetstrokeopacity{0.384635}%
\pgfsetdash{}{0pt}%
\pgfpathmoveto{\pgfqpoint{1.522552in}{2.105677in}}%
\pgfpathcurveto{\pgfqpoint{1.530789in}{2.105677in}}{\pgfqpoint{1.538689in}{2.108949in}}{\pgfqpoint{1.544513in}{2.114773in}}%
\pgfpathcurveto{\pgfqpoint{1.550336in}{2.120597in}}{\pgfqpoint{1.553609in}{2.128497in}}{\pgfqpoint{1.553609in}{2.136733in}}%
\pgfpathcurveto{\pgfqpoint{1.553609in}{2.144969in}}{\pgfqpoint{1.550336in}{2.152869in}}{\pgfqpoint{1.544513in}{2.158693in}}%
\pgfpathcurveto{\pgfqpoint{1.538689in}{2.164517in}}{\pgfqpoint{1.530789in}{2.167790in}}{\pgfqpoint{1.522552in}{2.167790in}}%
\pgfpathcurveto{\pgfqpoint{1.514316in}{2.167790in}}{\pgfqpoint{1.506416in}{2.164517in}}{\pgfqpoint{1.500592in}{2.158693in}}%
\pgfpathcurveto{\pgfqpoint{1.494768in}{2.152869in}}{\pgfqpoint{1.491496in}{2.144969in}}{\pgfqpoint{1.491496in}{2.136733in}}%
\pgfpathcurveto{\pgfqpoint{1.491496in}{2.128497in}}{\pgfqpoint{1.494768in}{2.120597in}}{\pgfqpoint{1.500592in}{2.114773in}}%
\pgfpathcurveto{\pgfqpoint{1.506416in}{2.108949in}}{\pgfqpoint{1.514316in}{2.105677in}}{\pgfqpoint{1.522552in}{2.105677in}}%
\pgfpathclose%
\pgfusepath{stroke,fill}%
\end{pgfscope}%
\begin{pgfscope}%
\pgfpathrectangle{\pgfqpoint{0.100000in}{0.212622in}}{\pgfqpoint{3.696000in}{3.696000in}}%
\pgfusepath{clip}%
\pgfsetbuttcap%
\pgfsetroundjoin%
\definecolor{currentfill}{rgb}{0.121569,0.466667,0.705882}%
\pgfsetfillcolor{currentfill}%
\pgfsetfillopacity{0.385820}%
\pgfsetlinewidth{1.003750pt}%
\definecolor{currentstroke}{rgb}{0.121569,0.466667,0.705882}%
\pgfsetstrokecolor{currentstroke}%
\pgfsetstrokeopacity{0.385820}%
\pgfsetdash{}{0pt}%
\pgfpathmoveto{\pgfqpoint{1.519602in}{2.105926in}}%
\pgfpathcurveto{\pgfqpoint{1.527838in}{2.105926in}}{\pgfqpoint{1.535738in}{2.109199in}}{\pgfqpoint{1.541562in}{2.115023in}}%
\pgfpathcurveto{\pgfqpoint{1.547386in}{2.120846in}}{\pgfqpoint{1.550658in}{2.128747in}}{\pgfqpoint{1.550658in}{2.136983in}}%
\pgfpathcurveto{\pgfqpoint{1.550658in}{2.145219in}}{\pgfqpoint{1.547386in}{2.153119in}}{\pgfqpoint{1.541562in}{2.158943in}}%
\pgfpathcurveto{\pgfqpoint{1.535738in}{2.164767in}}{\pgfqpoint{1.527838in}{2.168039in}}{\pgfqpoint{1.519602in}{2.168039in}}%
\pgfpathcurveto{\pgfqpoint{1.511366in}{2.168039in}}{\pgfqpoint{1.503465in}{2.164767in}}{\pgfqpoint{1.497642in}{2.158943in}}%
\pgfpathcurveto{\pgfqpoint{1.491818in}{2.153119in}}{\pgfqpoint{1.488545in}{2.145219in}}{\pgfqpoint{1.488545in}{2.136983in}}%
\pgfpathcurveto{\pgfqpoint{1.488545in}{2.128747in}}{\pgfqpoint{1.491818in}{2.120846in}}{\pgfqpoint{1.497642in}{2.115023in}}%
\pgfpathcurveto{\pgfqpoint{1.503465in}{2.109199in}}{\pgfqpoint{1.511366in}{2.105926in}}{\pgfqpoint{1.519602in}{2.105926in}}%
\pgfpathclose%
\pgfusepath{stroke,fill}%
\end{pgfscope}%
\begin{pgfscope}%
\pgfpathrectangle{\pgfqpoint{0.100000in}{0.212622in}}{\pgfqpoint{3.696000in}{3.696000in}}%
\pgfusepath{clip}%
\pgfsetbuttcap%
\pgfsetroundjoin%
\definecolor{currentfill}{rgb}{0.121569,0.466667,0.705882}%
\pgfsetfillcolor{currentfill}%
\pgfsetfillopacity{0.386664}%
\pgfsetlinewidth{1.003750pt}%
\definecolor{currentstroke}{rgb}{0.121569,0.466667,0.705882}%
\pgfsetstrokecolor{currentstroke}%
\pgfsetstrokeopacity{0.386664}%
\pgfsetdash{}{0pt}%
\pgfpathmoveto{\pgfqpoint{1.517966in}{2.105940in}}%
\pgfpathcurveto{\pgfqpoint{1.526202in}{2.105940in}}{\pgfqpoint{1.534102in}{2.109212in}}{\pgfqpoint{1.539926in}{2.115036in}}%
\pgfpathcurveto{\pgfqpoint{1.545750in}{2.120860in}}{\pgfqpoint{1.549023in}{2.128760in}}{\pgfqpoint{1.549023in}{2.136996in}}%
\pgfpathcurveto{\pgfqpoint{1.549023in}{2.145233in}}{\pgfqpoint{1.545750in}{2.153133in}}{\pgfqpoint{1.539926in}{2.158957in}}%
\pgfpathcurveto{\pgfqpoint{1.534102in}{2.164781in}}{\pgfqpoint{1.526202in}{2.168053in}}{\pgfqpoint{1.517966in}{2.168053in}}%
\pgfpathcurveto{\pgfqpoint{1.509730in}{2.168053in}}{\pgfqpoint{1.501830in}{2.164781in}}{\pgfqpoint{1.496006in}{2.158957in}}%
\pgfpathcurveto{\pgfqpoint{1.490182in}{2.153133in}}{\pgfqpoint{1.486910in}{2.145233in}}{\pgfqpoint{1.486910in}{2.136996in}}%
\pgfpathcurveto{\pgfqpoint{1.486910in}{2.128760in}}{\pgfqpoint{1.490182in}{2.120860in}}{\pgfqpoint{1.496006in}{2.115036in}}%
\pgfpathcurveto{\pgfqpoint{1.501830in}{2.109212in}}{\pgfqpoint{1.509730in}{2.105940in}}{\pgfqpoint{1.517966in}{2.105940in}}%
\pgfpathclose%
\pgfusepath{stroke,fill}%
\end{pgfscope}%
\begin{pgfscope}%
\pgfpathrectangle{\pgfqpoint{0.100000in}{0.212622in}}{\pgfqpoint{3.696000in}{3.696000in}}%
\pgfusepath{clip}%
\pgfsetbuttcap%
\pgfsetroundjoin%
\definecolor{currentfill}{rgb}{0.121569,0.466667,0.705882}%
\pgfsetfillcolor{currentfill}%
\pgfsetfillopacity{0.386863}%
\pgfsetlinewidth{1.003750pt}%
\definecolor{currentstroke}{rgb}{0.121569,0.466667,0.705882}%
\pgfsetstrokecolor{currentstroke}%
\pgfsetstrokeopacity{0.386863}%
\pgfsetdash{}{0pt}%
\pgfpathmoveto{\pgfqpoint{2.258788in}{1.990490in}}%
\pgfpathcurveto{\pgfqpoint{2.267024in}{1.990490in}}{\pgfqpoint{2.274924in}{1.993762in}}{\pgfqpoint{2.280748in}{1.999586in}}%
\pgfpathcurveto{\pgfqpoint{2.286572in}{2.005410in}}{\pgfqpoint{2.289845in}{2.013310in}}{\pgfqpoint{2.289845in}{2.021546in}}%
\pgfpathcurveto{\pgfqpoint{2.289845in}{2.029782in}}{\pgfqpoint{2.286572in}{2.037682in}}{\pgfqpoint{2.280748in}{2.043506in}}%
\pgfpathcurveto{\pgfqpoint{2.274924in}{2.049330in}}{\pgfqpoint{2.267024in}{2.052603in}}{\pgfqpoint{2.258788in}{2.052603in}}%
\pgfpathcurveto{\pgfqpoint{2.250552in}{2.052603in}}{\pgfqpoint{2.242652in}{2.049330in}}{\pgfqpoint{2.236828in}{2.043506in}}%
\pgfpathcurveto{\pgfqpoint{2.231004in}{2.037682in}}{\pgfqpoint{2.227732in}{2.029782in}}{\pgfqpoint{2.227732in}{2.021546in}}%
\pgfpathcurveto{\pgfqpoint{2.227732in}{2.013310in}}{\pgfqpoint{2.231004in}{2.005410in}}{\pgfqpoint{2.236828in}{1.999586in}}%
\pgfpathcurveto{\pgfqpoint{2.242652in}{1.993762in}}{\pgfqpoint{2.250552in}{1.990490in}}{\pgfqpoint{2.258788in}{1.990490in}}%
\pgfpathclose%
\pgfusepath{stroke,fill}%
\end{pgfscope}%
\begin{pgfscope}%
\pgfpathrectangle{\pgfqpoint{0.100000in}{0.212622in}}{\pgfqpoint{3.696000in}{3.696000in}}%
\pgfusepath{clip}%
\pgfsetbuttcap%
\pgfsetroundjoin%
\definecolor{currentfill}{rgb}{0.121569,0.466667,0.705882}%
\pgfsetfillcolor{currentfill}%
\pgfsetfillopacity{0.386996}%
\pgfsetlinewidth{1.003750pt}%
\definecolor{currentstroke}{rgb}{0.121569,0.466667,0.705882}%
\pgfsetstrokecolor{currentstroke}%
\pgfsetstrokeopacity{0.386996}%
\pgfsetdash{}{0pt}%
\pgfpathmoveto{\pgfqpoint{1.517246in}{2.105957in}}%
\pgfpathcurveto{\pgfqpoint{1.525482in}{2.105957in}}{\pgfqpoint{1.533382in}{2.109230in}}{\pgfqpoint{1.539206in}{2.115053in}}%
\pgfpathcurveto{\pgfqpoint{1.545030in}{2.120877in}}{\pgfqpoint{1.548302in}{2.128777in}}{\pgfqpoint{1.548302in}{2.137014in}}%
\pgfpathcurveto{\pgfqpoint{1.548302in}{2.145250in}}{\pgfqpoint{1.545030in}{2.153150in}}{\pgfqpoint{1.539206in}{2.158974in}}%
\pgfpathcurveto{\pgfqpoint{1.533382in}{2.164798in}}{\pgfqpoint{1.525482in}{2.168070in}}{\pgfqpoint{1.517246in}{2.168070in}}%
\pgfpathcurveto{\pgfqpoint{1.509010in}{2.168070in}}{\pgfqpoint{1.501110in}{2.164798in}}{\pgfqpoint{1.495286in}{2.158974in}}%
\pgfpathcurveto{\pgfqpoint{1.489462in}{2.153150in}}{\pgfqpoint{1.486189in}{2.145250in}}{\pgfqpoint{1.486189in}{2.137014in}}%
\pgfpathcurveto{\pgfqpoint{1.486189in}{2.128777in}}{\pgfqpoint{1.489462in}{2.120877in}}{\pgfqpoint{1.495286in}{2.115053in}}%
\pgfpathcurveto{\pgfqpoint{1.501110in}{2.109230in}}{\pgfqpoint{1.509010in}{2.105957in}}{\pgfqpoint{1.517246in}{2.105957in}}%
\pgfpathclose%
\pgfusepath{stroke,fill}%
\end{pgfscope}%
\begin{pgfscope}%
\pgfpathrectangle{\pgfqpoint{0.100000in}{0.212622in}}{\pgfqpoint{3.696000in}{3.696000in}}%
\pgfusepath{clip}%
\pgfsetbuttcap%
\pgfsetroundjoin%
\definecolor{currentfill}{rgb}{0.121569,0.466667,0.705882}%
\pgfsetfillcolor{currentfill}%
\pgfsetfillopacity{0.387257}%
\pgfsetlinewidth{1.003750pt}%
\definecolor{currentstroke}{rgb}{0.121569,0.466667,0.705882}%
\pgfsetstrokecolor{currentstroke}%
\pgfsetstrokeopacity{0.387257}%
\pgfsetdash{}{0pt}%
\pgfpathmoveto{\pgfqpoint{1.516658in}{2.105995in}}%
\pgfpathcurveto{\pgfqpoint{1.524894in}{2.105995in}}{\pgfqpoint{1.532794in}{2.109267in}}{\pgfqpoint{1.538618in}{2.115091in}}%
\pgfpathcurveto{\pgfqpoint{1.544442in}{2.120915in}}{\pgfqpoint{1.547715in}{2.128815in}}{\pgfqpoint{1.547715in}{2.137051in}}%
\pgfpathcurveto{\pgfqpoint{1.547715in}{2.145287in}}{\pgfqpoint{1.544442in}{2.153187in}}{\pgfqpoint{1.538618in}{2.159011in}}%
\pgfpathcurveto{\pgfqpoint{1.532794in}{2.164835in}}{\pgfqpoint{1.524894in}{2.168108in}}{\pgfqpoint{1.516658in}{2.168108in}}%
\pgfpathcurveto{\pgfqpoint{1.508422in}{2.168108in}}{\pgfqpoint{1.500522in}{2.164835in}}{\pgfqpoint{1.494698in}{2.159011in}}%
\pgfpathcurveto{\pgfqpoint{1.488874in}{2.153187in}}{\pgfqpoint{1.485602in}{2.145287in}}{\pgfqpoint{1.485602in}{2.137051in}}%
\pgfpathcurveto{\pgfqpoint{1.485602in}{2.128815in}}{\pgfqpoint{1.488874in}{2.120915in}}{\pgfqpoint{1.494698in}{2.115091in}}%
\pgfpathcurveto{\pgfqpoint{1.500522in}{2.109267in}}{\pgfqpoint{1.508422in}{2.105995in}}{\pgfqpoint{1.516658in}{2.105995in}}%
\pgfpathclose%
\pgfusepath{stroke,fill}%
\end{pgfscope}%
\begin{pgfscope}%
\pgfpathrectangle{\pgfqpoint{0.100000in}{0.212622in}}{\pgfqpoint{3.696000in}{3.696000in}}%
\pgfusepath{clip}%
\pgfsetbuttcap%
\pgfsetroundjoin%
\definecolor{currentfill}{rgb}{0.121569,0.466667,0.705882}%
\pgfsetfillcolor{currentfill}%
\pgfsetfillopacity{0.387756}%
\pgfsetlinewidth{1.003750pt}%
\definecolor{currentstroke}{rgb}{0.121569,0.466667,0.705882}%
\pgfsetstrokecolor{currentstroke}%
\pgfsetstrokeopacity{0.387756}%
\pgfsetdash{}{0pt}%
\pgfpathmoveto{\pgfqpoint{1.515878in}{2.105986in}}%
\pgfpathcurveto{\pgfqpoint{1.524114in}{2.105986in}}{\pgfqpoint{1.532014in}{2.109258in}}{\pgfqpoint{1.537838in}{2.115082in}}%
\pgfpathcurveto{\pgfqpoint{1.543662in}{2.120906in}}{\pgfqpoint{1.546934in}{2.128806in}}{\pgfqpoint{1.546934in}{2.137042in}}%
\pgfpathcurveto{\pgfqpoint{1.546934in}{2.145278in}}{\pgfqpoint{1.543662in}{2.153178in}}{\pgfqpoint{1.537838in}{2.159002in}}%
\pgfpathcurveto{\pgfqpoint{1.532014in}{2.164826in}}{\pgfqpoint{1.524114in}{2.168099in}}{\pgfqpoint{1.515878in}{2.168099in}}%
\pgfpathcurveto{\pgfqpoint{1.507641in}{2.168099in}}{\pgfqpoint{1.499741in}{2.164826in}}{\pgfqpoint{1.493917in}{2.159002in}}%
\pgfpathcurveto{\pgfqpoint{1.488093in}{2.153178in}}{\pgfqpoint{1.484821in}{2.145278in}}{\pgfqpoint{1.484821in}{2.137042in}}%
\pgfpathcurveto{\pgfqpoint{1.484821in}{2.128806in}}{\pgfqpoint{1.488093in}{2.120906in}}{\pgfqpoint{1.493917in}{2.115082in}}%
\pgfpathcurveto{\pgfqpoint{1.499741in}{2.109258in}}{\pgfqpoint{1.507641in}{2.105986in}}{\pgfqpoint{1.515878in}{2.105986in}}%
\pgfpathclose%
\pgfusepath{stroke,fill}%
\end{pgfscope}%
\begin{pgfscope}%
\pgfpathrectangle{\pgfqpoint{0.100000in}{0.212622in}}{\pgfqpoint{3.696000in}{3.696000in}}%
\pgfusepath{clip}%
\pgfsetbuttcap%
\pgfsetroundjoin%
\definecolor{currentfill}{rgb}{0.121569,0.466667,0.705882}%
\pgfsetfillcolor{currentfill}%
\pgfsetfillopacity{0.387986}%
\pgfsetlinewidth{1.003750pt}%
\definecolor{currentstroke}{rgb}{0.121569,0.466667,0.705882}%
\pgfsetstrokecolor{currentstroke}%
\pgfsetstrokeopacity{0.387986}%
\pgfsetdash{}{0pt}%
\pgfpathmoveto{\pgfqpoint{2.265901in}{1.988893in}}%
\pgfpathcurveto{\pgfqpoint{2.274137in}{1.988893in}}{\pgfqpoint{2.282037in}{1.992165in}}{\pgfqpoint{2.287861in}{1.997989in}}%
\pgfpathcurveto{\pgfqpoint{2.293685in}{2.003813in}}{\pgfqpoint{2.296958in}{2.011713in}}{\pgfqpoint{2.296958in}{2.019949in}}%
\pgfpathcurveto{\pgfqpoint{2.296958in}{2.028185in}}{\pgfqpoint{2.293685in}{2.036085in}}{\pgfqpoint{2.287861in}{2.041909in}}%
\pgfpathcurveto{\pgfqpoint{2.282037in}{2.047733in}}{\pgfqpoint{2.274137in}{2.051006in}}{\pgfqpoint{2.265901in}{2.051006in}}%
\pgfpathcurveto{\pgfqpoint{2.257665in}{2.051006in}}{\pgfqpoint{2.249765in}{2.047733in}}{\pgfqpoint{2.243941in}{2.041909in}}%
\pgfpathcurveto{\pgfqpoint{2.238117in}{2.036085in}}{\pgfqpoint{2.234845in}{2.028185in}}{\pgfqpoint{2.234845in}{2.019949in}}%
\pgfpathcurveto{\pgfqpoint{2.234845in}{2.011713in}}{\pgfqpoint{2.238117in}{2.003813in}}{\pgfqpoint{2.243941in}{1.997989in}}%
\pgfpathcurveto{\pgfqpoint{2.249765in}{1.992165in}}{\pgfqpoint{2.257665in}{1.988893in}}{\pgfqpoint{2.265901in}{1.988893in}}%
\pgfpathclose%
\pgfusepath{stroke,fill}%
\end{pgfscope}%
\begin{pgfscope}%
\pgfpathrectangle{\pgfqpoint{0.100000in}{0.212622in}}{\pgfqpoint{3.696000in}{3.696000in}}%
\pgfusepath{clip}%
\pgfsetbuttcap%
\pgfsetroundjoin%
\definecolor{currentfill}{rgb}{0.121569,0.466667,0.705882}%
\pgfsetfillcolor{currentfill}%
\pgfsetfillopacity{0.388615}%
\pgfsetlinewidth{1.003750pt}%
\definecolor{currentstroke}{rgb}{0.121569,0.466667,0.705882}%
\pgfsetstrokecolor{currentstroke}%
\pgfsetstrokeopacity{0.388615}%
\pgfsetdash{}{0pt}%
\pgfpathmoveto{\pgfqpoint{1.513969in}{2.106034in}}%
\pgfpathcurveto{\pgfqpoint{1.522205in}{2.106034in}}{\pgfqpoint{1.530105in}{2.109306in}}{\pgfqpoint{1.535929in}{2.115130in}}%
\pgfpathcurveto{\pgfqpoint{1.541753in}{2.120954in}}{\pgfqpoint{1.545026in}{2.128854in}}{\pgfqpoint{1.545026in}{2.137091in}}%
\pgfpathcurveto{\pgfqpoint{1.545026in}{2.145327in}}{\pgfqpoint{1.541753in}{2.153227in}}{\pgfqpoint{1.535929in}{2.159051in}}%
\pgfpathcurveto{\pgfqpoint{1.530105in}{2.164875in}}{\pgfqpoint{1.522205in}{2.168147in}}{\pgfqpoint{1.513969in}{2.168147in}}%
\pgfpathcurveto{\pgfqpoint{1.505733in}{2.168147in}}{\pgfqpoint{1.497833in}{2.164875in}}{\pgfqpoint{1.492009in}{2.159051in}}%
\pgfpathcurveto{\pgfqpoint{1.486185in}{2.153227in}}{\pgfqpoint{1.482913in}{2.145327in}}{\pgfqpoint{1.482913in}{2.137091in}}%
\pgfpathcurveto{\pgfqpoint{1.482913in}{2.128854in}}{\pgfqpoint{1.486185in}{2.120954in}}{\pgfqpoint{1.492009in}{2.115130in}}%
\pgfpathcurveto{\pgfqpoint{1.497833in}{2.109306in}}{\pgfqpoint{1.505733in}{2.106034in}}{\pgfqpoint{1.513969in}{2.106034in}}%
\pgfpathclose%
\pgfusepath{stroke,fill}%
\end{pgfscope}%
\begin{pgfscope}%
\pgfpathrectangle{\pgfqpoint{0.100000in}{0.212622in}}{\pgfqpoint{3.696000in}{3.696000in}}%
\pgfusepath{clip}%
\pgfsetbuttcap%
\pgfsetroundjoin%
\definecolor{currentfill}{rgb}{0.121569,0.466667,0.705882}%
\pgfsetfillcolor{currentfill}%
\pgfsetfillopacity{0.389411}%
\pgfsetlinewidth{1.003750pt}%
\definecolor{currentstroke}{rgb}{0.121569,0.466667,0.705882}%
\pgfsetstrokecolor{currentstroke}%
\pgfsetstrokeopacity{0.389411}%
\pgfsetdash{}{0pt}%
\pgfpathmoveto{\pgfqpoint{1.512255in}{2.106126in}}%
\pgfpathcurveto{\pgfqpoint{1.520491in}{2.106126in}}{\pgfqpoint{1.528392in}{2.109398in}}{\pgfqpoint{1.534215in}{2.115222in}}%
\pgfpathcurveto{\pgfqpoint{1.540039in}{2.121046in}}{\pgfqpoint{1.543312in}{2.128946in}}{\pgfqpoint{1.543312in}{2.137183in}}%
\pgfpathcurveto{\pgfqpoint{1.543312in}{2.145419in}}{\pgfqpoint{1.540039in}{2.153319in}}{\pgfqpoint{1.534215in}{2.159143in}}%
\pgfpathcurveto{\pgfqpoint{1.528392in}{2.164967in}}{\pgfqpoint{1.520491in}{2.168239in}}{\pgfqpoint{1.512255in}{2.168239in}}%
\pgfpathcurveto{\pgfqpoint{1.504019in}{2.168239in}}{\pgfqpoint{1.496119in}{2.164967in}}{\pgfqpoint{1.490295in}{2.159143in}}%
\pgfpathcurveto{\pgfqpoint{1.484471in}{2.153319in}}{\pgfqpoint{1.481199in}{2.145419in}}{\pgfqpoint{1.481199in}{2.137183in}}%
\pgfpathcurveto{\pgfqpoint{1.481199in}{2.128946in}}{\pgfqpoint{1.484471in}{2.121046in}}{\pgfqpoint{1.490295in}{2.115222in}}%
\pgfpathcurveto{\pgfqpoint{1.496119in}{2.109398in}}{\pgfqpoint{1.504019in}{2.106126in}}{\pgfqpoint{1.512255in}{2.106126in}}%
\pgfpathclose%
\pgfusepath{stroke,fill}%
\end{pgfscope}%
\begin{pgfscope}%
\pgfpathrectangle{\pgfqpoint{0.100000in}{0.212622in}}{\pgfqpoint{3.696000in}{3.696000in}}%
\pgfusepath{clip}%
\pgfsetbuttcap%
\pgfsetroundjoin%
\definecolor{currentfill}{rgb}{0.121569,0.466667,0.705882}%
\pgfsetfillcolor{currentfill}%
\pgfsetfillopacity{0.389642}%
\pgfsetlinewidth{1.003750pt}%
\definecolor{currentstroke}{rgb}{0.121569,0.466667,0.705882}%
\pgfsetstrokecolor{currentstroke}%
\pgfsetstrokeopacity{0.389642}%
\pgfsetdash{}{0pt}%
\pgfpathmoveto{\pgfqpoint{2.272676in}{1.987942in}}%
\pgfpathcurveto{\pgfqpoint{2.280912in}{1.987942in}}{\pgfqpoint{2.288812in}{1.991214in}}{\pgfqpoint{2.294636in}{1.997038in}}%
\pgfpathcurveto{\pgfqpoint{2.300460in}{2.002862in}}{\pgfqpoint{2.303733in}{2.010762in}}{\pgfqpoint{2.303733in}{2.018998in}}%
\pgfpathcurveto{\pgfqpoint{2.303733in}{2.027235in}}{\pgfqpoint{2.300460in}{2.035135in}}{\pgfqpoint{2.294636in}{2.040959in}}%
\pgfpathcurveto{\pgfqpoint{2.288812in}{2.046783in}}{\pgfqpoint{2.280912in}{2.050055in}}{\pgfqpoint{2.272676in}{2.050055in}}%
\pgfpathcurveto{\pgfqpoint{2.264440in}{2.050055in}}{\pgfqpoint{2.256540in}{2.046783in}}{\pgfqpoint{2.250716in}{2.040959in}}%
\pgfpathcurveto{\pgfqpoint{2.244892in}{2.035135in}}{\pgfqpoint{2.241620in}{2.027235in}}{\pgfqpoint{2.241620in}{2.018998in}}%
\pgfpathcurveto{\pgfqpoint{2.241620in}{2.010762in}}{\pgfqpoint{2.244892in}{2.002862in}}{\pgfqpoint{2.250716in}{1.997038in}}%
\pgfpathcurveto{\pgfqpoint{2.256540in}{1.991214in}}{\pgfqpoint{2.264440in}{1.987942in}}{\pgfqpoint{2.272676in}{1.987942in}}%
\pgfpathclose%
\pgfusepath{stroke,fill}%
\end{pgfscope}%
\begin{pgfscope}%
\pgfpathrectangle{\pgfqpoint{0.100000in}{0.212622in}}{\pgfqpoint{3.696000in}{3.696000in}}%
\pgfusepath{clip}%
\pgfsetbuttcap%
\pgfsetroundjoin%
\definecolor{currentfill}{rgb}{0.121569,0.466667,0.705882}%
\pgfsetfillcolor{currentfill}%
\pgfsetfillopacity{0.390977}%
\pgfsetlinewidth{1.003750pt}%
\definecolor{currentstroke}{rgb}{0.121569,0.466667,0.705882}%
\pgfsetstrokecolor{currentstroke}%
\pgfsetstrokeopacity{0.390977}%
\pgfsetdash{}{0pt}%
\pgfpathmoveto{\pgfqpoint{1.510234in}{2.106297in}}%
\pgfpathcurveto{\pgfqpoint{1.518470in}{2.106297in}}{\pgfqpoint{1.526370in}{2.109569in}}{\pgfqpoint{1.532194in}{2.115393in}}%
\pgfpathcurveto{\pgfqpoint{1.538018in}{2.121217in}}{\pgfqpoint{1.541291in}{2.129117in}}{\pgfqpoint{1.541291in}{2.137353in}}%
\pgfpathcurveto{\pgfqpoint{1.541291in}{2.145590in}}{\pgfqpoint{1.538018in}{2.153490in}}{\pgfqpoint{1.532194in}{2.159314in}}%
\pgfpathcurveto{\pgfqpoint{1.526370in}{2.165138in}}{\pgfqpoint{1.518470in}{2.168410in}}{\pgfqpoint{1.510234in}{2.168410in}}%
\pgfpathcurveto{\pgfqpoint{1.501998in}{2.168410in}}{\pgfqpoint{1.494098in}{2.165138in}}{\pgfqpoint{1.488274in}{2.159314in}}%
\pgfpathcurveto{\pgfqpoint{1.482450in}{2.153490in}}{\pgfqpoint{1.479178in}{2.145590in}}{\pgfqpoint{1.479178in}{2.137353in}}%
\pgfpathcurveto{\pgfqpoint{1.479178in}{2.129117in}}{\pgfqpoint{1.482450in}{2.121217in}}{\pgfqpoint{1.488274in}{2.115393in}}%
\pgfpathcurveto{\pgfqpoint{1.494098in}{2.109569in}}{\pgfqpoint{1.501998in}{2.106297in}}{\pgfqpoint{1.510234in}{2.106297in}}%
\pgfpathclose%
\pgfusepath{stroke,fill}%
\end{pgfscope}%
\begin{pgfscope}%
\pgfpathrectangle{\pgfqpoint{0.100000in}{0.212622in}}{\pgfqpoint{3.696000in}{3.696000in}}%
\pgfusepath{clip}%
\pgfsetbuttcap%
\pgfsetroundjoin%
\definecolor{currentfill}{rgb}{0.121569,0.466667,0.705882}%
\pgfsetfillcolor{currentfill}%
\pgfsetfillopacity{0.391375}%
\pgfsetlinewidth{1.003750pt}%
\definecolor{currentstroke}{rgb}{0.121569,0.466667,0.705882}%
\pgfsetstrokecolor{currentstroke}%
\pgfsetstrokeopacity{0.391375}%
\pgfsetdash{}{0pt}%
\pgfpathmoveto{\pgfqpoint{2.280019in}{1.986611in}}%
\pgfpathcurveto{\pgfqpoint{2.288256in}{1.986611in}}{\pgfqpoint{2.296156in}{1.989883in}}{\pgfqpoint{2.301979in}{1.995707in}}%
\pgfpathcurveto{\pgfqpoint{2.307803in}{2.001531in}}{\pgfqpoint{2.311076in}{2.009431in}}{\pgfqpoint{2.311076in}{2.017667in}}%
\pgfpathcurveto{\pgfqpoint{2.311076in}{2.025904in}}{\pgfqpoint{2.307803in}{2.033804in}}{\pgfqpoint{2.301979in}{2.039628in}}%
\pgfpathcurveto{\pgfqpoint{2.296156in}{2.045452in}}{\pgfqpoint{2.288256in}{2.048724in}}{\pgfqpoint{2.280019in}{2.048724in}}%
\pgfpathcurveto{\pgfqpoint{2.271783in}{2.048724in}}{\pgfqpoint{2.263883in}{2.045452in}}{\pgfqpoint{2.258059in}{2.039628in}}%
\pgfpathcurveto{\pgfqpoint{2.252235in}{2.033804in}}{\pgfqpoint{2.248963in}{2.025904in}}{\pgfqpoint{2.248963in}{2.017667in}}%
\pgfpathcurveto{\pgfqpoint{2.248963in}{2.009431in}}{\pgfqpoint{2.252235in}{2.001531in}}{\pgfqpoint{2.258059in}{1.995707in}}%
\pgfpathcurveto{\pgfqpoint{2.263883in}{1.989883in}}{\pgfqpoint{2.271783in}{1.986611in}}{\pgfqpoint{2.280019in}{1.986611in}}%
\pgfpathclose%
\pgfusepath{stroke,fill}%
\end{pgfscope}%
\begin{pgfscope}%
\pgfpathrectangle{\pgfqpoint{0.100000in}{0.212622in}}{\pgfqpoint{3.696000in}{3.696000in}}%
\pgfusepath{clip}%
\pgfsetbuttcap%
\pgfsetroundjoin%
\definecolor{currentfill}{rgb}{0.121569,0.466667,0.705882}%
\pgfsetfillcolor{currentfill}%
\pgfsetfillopacity{0.391890}%
\pgfsetlinewidth{1.003750pt}%
\definecolor{currentstroke}{rgb}{0.121569,0.466667,0.705882}%
\pgfsetstrokecolor{currentstroke}%
\pgfsetstrokeopacity{0.391890}%
\pgfsetdash{}{0pt}%
\pgfpathmoveto{\pgfqpoint{2.285043in}{1.985137in}}%
\pgfpathcurveto{\pgfqpoint{2.293279in}{1.985137in}}{\pgfqpoint{2.301179in}{1.988410in}}{\pgfqpoint{2.307003in}{1.994233in}}%
\pgfpathcurveto{\pgfqpoint{2.312827in}{2.000057in}}{\pgfqpoint{2.316099in}{2.007957in}}{\pgfqpoint{2.316099in}{2.016194in}}%
\pgfpathcurveto{\pgfqpoint{2.316099in}{2.024430in}}{\pgfqpoint{2.312827in}{2.032330in}}{\pgfqpoint{2.307003in}{2.038154in}}%
\pgfpathcurveto{\pgfqpoint{2.301179in}{2.043978in}}{\pgfqpoint{2.293279in}{2.047250in}}{\pgfqpoint{2.285043in}{2.047250in}}%
\pgfpathcurveto{\pgfqpoint{2.276806in}{2.047250in}}{\pgfqpoint{2.268906in}{2.043978in}}{\pgfqpoint{2.263082in}{2.038154in}}%
\pgfpathcurveto{\pgfqpoint{2.257258in}{2.032330in}}{\pgfqpoint{2.253986in}{2.024430in}}{\pgfqpoint{2.253986in}{2.016194in}}%
\pgfpathcurveto{\pgfqpoint{2.253986in}{2.007957in}}{\pgfqpoint{2.257258in}{2.000057in}}{\pgfqpoint{2.263082in}{1.994233in}}%
\pgfpathcurveto{\pgfqpoint{2.268906in}{1.988410in}}{\pgfqpoint{2.276806in}{1.985137in}}{\pgfqpoint{2.285043in}{1.985137in}}%
\pgfpathclose%
\pgfusepath{stroke,fill}%
\end{pgfscope}%
\begin{pgfscope}%
\pgfpathrectangle{\pgfqpoint{0.100000in}{0.212622in}}{\pgfqpoint{3.696000in}{3.696000in}}%
\pgfusepath{clip}%
\pgfsetbuttcap%
\pgfsetroundjoin%
\definecolor{currentfill}{rgb}{0.121569,0.466667,0.705882}%
\pgfsetfillcolor{currentfill}%
\pgfsetfillopacity{0.392092}%
\pgfsetlinewidth{1.003750pt}%
\definecolor{currentstroke}{rgb}{0.121569,0.466667,0.705882}%
\pgfsetstrokecolor{currentstroke}%
\pgfsetstrokeopacity{0.392092}%
\pgfsetdash{}{0pt}%
\pgfpathmoveto{\pgfqpoint{1.507514in}{2.106498in}}%
\pgfpathcurveto{\pgfqpoint{1.515750in}{2.106498in}}{\pgfqpoint{1.523650in}{2.109771in}}{\pgfqpoint{1.529474in}{2.115595in}}%
\pgfpathcurveto{\pgfqpoint{1.535298in}{2.121419in}}{\pgfqpoint{1.538570in}{2.129319in}}{\pgfqpoint{1.538570in}{2.137555in}}%
\pgfpathcurveto{\pgfqpoint{1.538570in}{2.145791in}}{\pgfqpoint{1.535298in}{2.153691in}}{\pgfqpoint{1.529474in}{2.159515in}}%
\pgfpathcurveto{\pgfqpoint{1.523650in}{2.165339in}}{\pgfqpoint{1.515750in}{2.168611in}}{\pgfqpoint{1.507514in}{2.168611in}}%
\pgfpathcurveto{\pgfqpoint{1.499278in}{2.168611in}}{\pgfqpoint{1.491378in}{2.165339in}}{\pgfqpoint{1.485554in}{2.159515in}}%
\pgfpathcurveto{\pgfqpoint{1.479730in}{2.153691in}}{\pgfqpoint{1.476457in}{2.145791in}}{\pgfqpoint{1.476457in}{2.137555in}}%
\pgfpathcurveto{\pgfqpoint{1.476457in}{2.129319in}}{\pgfqpoint{1.479730in}{2.121419in}}{\pgfqpoint{1.485554in}{2.115595in}}%
\pgfpathcurveto{\pgfqpoint{1.491378in}{2.109771in}}{\pgfqpoint{1.499278in}{2.106498in}}{\pgfqpoint{1.507514in}{2.106498in}}%
\pgfpathclose%
\pgfusepath{stroke,fill}%
\end{pgfscope}%
\begin{pgfscope}%
\pgfpathrectangle{\pgfqpoint{0.100000in}{0.212622in}}{\pgfqpoint{3.696000in}{3.696000in}}%
\pgfusepath{clip}%
\pgfsetbuttcap%
\pgfsetroundjoin%
\definecolor{currentfill}{rgb}{0.121569,0.466667,0.705882}%
\pgfsetfillcolor{currentfill}%
\pgfsetfillopacity{0.392922}%
\pgfsetlinewidth{1.003750pt}%
\definecolor{currentstroke}{rgb}{0.121569,0.466667,0.705882}%
\pgfsetstrokecolor{currentstroke}%
\pgfsetstrokeopacity{0.392922}%
\pgfsetdash{}{0pt}%
\pgfpathmoveto{\pgfqpoint{2.291032in}{1.983794in}}%
\pgfpathcurveto{\pgfqpoint{2.299268in}{1.983794in}}{\pgfqpoint{2.307168in}{1.987066in}}{\pgfqpoint{2.312992in}{1.992890in}}%
\pgfpathcurveto{\pgfqpoint{2.318816in}{1.998714in}}{\pgfqpoint{2.322088in}{2.006614in}}{\pgfqpoint{2.322088in}{2.014850in}}%
\pgfpathcurveto{\pgfqpoint{2.322088in}{2.023086in}}{\pgfqpoint{2.318816in}{2.030986in}}{\pgfqpoint{2.312992in}{2.036810in}}%
\pgfpathcurveto{\pgfqpoint{2.307168in}{2.042634in}}{\pgfqpoint{2.299268in}{2.045907in}}{\pgfqpoint{2.291032in}{2.045907in}}%
\pgfpathcurveto{\pgfqpoint{2.282795in}{2.045907in}}{\pgfqpoint{2.274895in}{2.042634in}}{\pgfqpoint{2.269071in}{2.036810in}}%
\pgfpathcurveto{\pgfqpoint{2.263247in}{2.030986in}}{\pgfqpoint{2.259975in}{2.023086in}}{\pgfqpoint{2.259975in}{2.014850in}}%
\pgfpathcurveto{\pgfqpoint{2.259975in}{2.006614in}}{\pgfqpoint{2.263247in}{1.998714in}}{\pgfqpoint{2.269071in}{1.992890in}}%
\pgfpathcurveto{\pgfqpoint{2.274895in}{1.987066in}}{\pgfqpoint{2.282795in}{1.983794in}}{\pgfqpoint{2.291032in}{1.983794in}}%
\pgfpathclose%
\pgfusepath{stroke,fill}%
\end{pgfscope}%
\begin{pgfscope}%
\pgfpathrectangle{\pgfqpoint{0.100000in}{0.212622in}}{\pgfqpoint{3.696000in}{3.696000in}}%
\pgfusepath{clip}%
\pgfsetbuttcap%
\pgfsetroundjoin%
\definecolor{currentfill}{rgb}{0.121569,0.466667,0.705882}%
\pgfsetfillcolor{currentfill}%
\pgfsetfillopacity{0.393335}%
\pgfsetlinewidth{1.003750pt}%
\definecolor{currentstroke}{rgb}{0.121569,0.466667,0.705882}%
\pgfsetstrokecolor{currentstroke}%
\pgfsetstrokeopacity{0.393335}%
\pgfsetdash{}{0pt}%
\pgfpathmoveto{\pgfqpoint{1.506879in}{2.107143in}}%
\pgfpathcurveto{\pgfqpoint{1.515115in}{2.107143in}}{\pgfqpoint{1.523015in}{2.110416in}}{\pgfqpoint{1.528839in}{2.116240in}}%
\pgfpathcurveto{\pgfqpoint{1.534663in}{2.122063in}}{\pgfqpoint{1.537935in}{2.129964in}}{\pgfqpoint{1.537935in}{2.138200in}}%
\pgfpathcurveto{\pgfqpoint{1.537935in}{2.146436in}}{\pgfqpoint{1.534663in}{2.154336in}}{\pgfqpoint{1.528839in}{2.160160in}}%
\pgfpathcurveto{\pgfqpoint{1.523015in}{2.165984in}}{\pgfqpoint{1.515115in}{2.169256in}}{\pgfqpoint{1.506879in}{2.169256in}}%
\pgfpathcurveto{\pgfqpoint{1.498642in}{2.169256in}}{\pgfqpoint{1.490742in}{2.165984in}}{\pgfqpoint{1.484918in}{2.160160in}}%
\pgfpathcurveto{\pgfqpoint{1.479094in}{2.154336in}}{\pgfqpoint{1.475822in}{2.146436in}}{\pgfqpoint{1.475822in}{2.138200in}}%
\pgfpathcurveto{\pgfqpoint{1.475822in}{2.129964in}}{\pgfqpoint{1.479094in}{2.122063in}}{\pgfqpoint{1.484918in}{2.116240in}}%
\pgfpathcurveto{\pgfqpoint{1.490742in}{2.110416in}}{\pgfqpoint{1.498642in}{2.107143in}}{\pgfqpoint{1.506879in}{2.107143in}}%
\pgfpathclose%
\pgfusepath{stroke,fill}%
\end{pgfscope}%
\begin{pgfscope}%
\pgfpathrectangle{\pgfqpoint{0.100000in}{0.212622in}}{\pgfqpoint{3.696000in}{3.696000in}}%
\pgfusepath{clip}%
\pgfsetbuttcap%
\pgfsetroundjoin%
\definecolor{currentfill}{rgb}{0.121569,0.466667,0.705882}%
\pgfsetfillcolor{currentfill}%
\pgfsetfillopacity{0.393571}%
\pgfsetlinewidth{1.003750pt}%
\definecolor{currentstroke}{rgb}{0.121569,0.466667,0.705882}%
\pgfsetstrokecolor{currentstroke}%
\pgfsetstrokeopacity{0.393571}%
\pgfsetdash{}{0pt}%
\pgfpathmoveto{\pgfqpoint{2.294161in}{1.983273in}}%
\pgfpathcurveto{\pgfqpoint{2.302397in}{1.983273in}}{\pgfqpoint{2.310297in}{1.986545in}}{\pgfqpoint{2.316121in}{1.992369in}}%
\pgfpathcurveto{\pgfqpoint{2.321945in}{1.998193in}}{\pgfqpoint{2.325217in}{2.006093in}}{\pgfqpoint{2.325217in}{2.014330in}}%
\pgfpathcurveto{\pgfqpoint{2.325217in}{2.022566in}}{\pgfqpoint{2.321945in}{2.030466in}}{\pgfqpoint{2.316121in}{2.036290in}}%
\pgfpathcurveto{\pgfqpoint{2.310297in}{2.042114in}}{\pgfqpoint{2.302397in}{2.045386in}}{\pgfqpoint{2.294161in}{2.045386in}}%
\pgfpathcurveto{\pgfqpoint{2.285924in}{2.045386in}}{\pgfqpoint{2.278024in}{2.042114in}}{\pgfqpoint{2.272200in}{2.036290in}}%
\pgfpathcurveto{\pgfqpoint{2.266376in}{2.030466in}}{\pgfqpoint{2.263104in}{2.022566in}}{\pgfqpoint{2.263104in}{2.014330in}}%
\pgfpathcurveto{\pgfqpoint{2.263104in}{2.006093in}}{\pgfqpoint{2.266376in}{1.998193in}}{\pgfqpoint{2.272200in}{1.992369in}}%
\pgfpathcurveto{\pgfqpoint{2.278024in}{1.986545in}}{\pgfqpoint{2.285924in}{1.983273in}}{\pgfqpoint{2.294161in}{1.983273in}}%
\pgfpathclose%
\pgfusepath{stroke,fill}%
\end{pgfscope}%
\begin{pgfscope}%
\pgfpathrectangle{\pgfqpoint{0.100000in}{0.212622in}}{\pgfqpoint{3.696000in}{3.696000in}}%
\pgfusepath{clip}%
\pgfsetbuttcap%
\pgfsetroundjoin%
\definecolor{currentfill}{rgb}{0.121569,0.466667,0.705882}%
\pgfsetfillcolor{currentfill}%
\pgfsetfillopacity{0.394130}%
\pgfsetlinewidth{1.003750pt}%
\definecolor{currentstroke}{rgb}{0.121569,0.466667,0.705882}%
\pgfsetstrokecolor{currentstroke}%
\pgfsetstrokeopacity{0.394130}%
\pgfsetdash{}{0pt}%
\pgfpathmoveto{\pgfqpoint{2.297790in}{1.982360in}}%
\pgfpathcurveto{\pgfqpoint{2.306026in}{1.982360in}}{\pgfqpoint{2.313926in}{1.985633in}}{\pgfqpoint{2.319750in}{1.991457in}}%
\pgfpathcurveto{\pgfqpoint{2.325574in}{1.997280in}}{\pgfqpoint{2.328846in}{2.005181in}}{\pgfqpoint{2.328846in}{2.013417in}}%
\pgfpathcurveto{\pgfqpoint{2.328846in}{2.021653in}}{\pgfqpoint{2.325574in}{2.029553in}}{\pgfqpoint{2.319750in}{2.035377in}}%
\pgfpathcurveto{\pgfqpoint{2.313926in}{2.041201in}}{\pgfqpoint{2.306026in}{2.044473in}}{\pgfqpoint{2.297790in}{2.044473in}}%
\pgfpathcurveto{\pgfqpoint{2.289554in}{2.044473in}}{\pgfqpoint{2.281653in}{2.041201in}}{\pgfqpoint{2.275830in}{2.035377in}}%
\pgfpathcurveto{\pgfqpoint{2.270006in}{2.029553in}}{\pgfqpoint{2.266733in}{2.021653in}}{\pgfqpoint{2.266733in}{2.013417in}}%
\pgfpathcurveto{\pgfqpoint{2.266733in}{2.005181in}}{\pgfqpoint{2.270006in}{1.997280in}}{\pgfqpoint{2.275830in}{1.991457in}}%
\pgfpathcurveto{\pgfqpoint{2.281653in}{1.985633in}}{\pgfqpoint{2.289554in}{1.982360in}}{\pgfqpoint{2.297790in}{1.982360in}}%
\pgfpathclose%
\pgfusepath{stroke,fill}%
\end{pgfscope}%
\begin{pgfscope}%
\pgfpathrectangle{\pgfqpoint{0.100000in}{0.212622in}}{\pgfqpoint{3.696000in}{3.696000in}}%
\pgfusepath{clip}%
\pgfsetbuttcap%
\pgfsetroundjoin%
\definecolor{currentfill}{rgb}{0.121569,0.466667,0.705882}%
\pgfsetfillcolor{currentfill}%
\pgfsetfillopacity{0.394804}%
\pgfsetlinewidth{1.003750pt}%
\definecolor{currentstroke}{rgb}{0.121569,0.466667,0.705882}%
\pgfsetstrokecolor{currentstroke}%
\pgfsetstrokeopacity{0.394804}%
\pgfsetdash{}{0pt}%
\pgfpathmoveto{\pgfqpoint{2.301563in}{1.981463in}}%
\pgfpathcurveto{\pgfqpoint{2.309800in}{1.981463in}}{\pgfqpoint{2.317700in}{1.984735in}}{\pgfqpoint{2.323524in}{1.990559in}}%
\pgfpathcurveto{\pgfqpoint{2.329348in}{1.996383in}}{\pgfqpoint{2.332620in}{2.004283in}}{\pgfqpoint{2.332620in}{2.012519in}}%
\pgfpathcurveto{\pgfqpoint{2.332620in}{2.020756in}}{\pgfqpoint{2.329348in}{2.028656in}}{\pgfqpoint{2.323524in}{2.034480in}}%
\pgfpathcurveto{\pgfqpoint{2.317700in}{2.040304in}}{\pgfqpoint{2.309800in}{2.043576in}}{\pgfqpoint{2.301563in}{2.043576in}}%
\pgfpathcurveto{\pgfqpoint{2.293327in}{2.043576in}}{\pgfqpoint{2.285427in}{2.040304in}}{\pgfqpoint{2.279603in}{2.034480in}}%
\pgfpathcurveto{\pgfqpoint{2.273779in}{2.028656in}}{\pgfqpoint{2.270507in}{2.020756in}}{\pgfqpoint{2.270507in}{2.012519in}}%
\pgfpathcurveto{\pgfqpoint{2.270507in}{2.004283in}}{\pgfqpoint{2.273779in}{1.996383in}}{\pgfqpoint{2.279603in}{1.990559in}}%
\pgfpathcurveto{\pgfqpoint{2.285427in}{1.984735in}}{\pgfqpoint{2.293327in}{1.981463in}}{\pgfqpoint{2.301563in}{1.981463in}}%
\pgfpathclose%
\pgfusepath{stroke,fill}%
\end{pgfscope}%
\begin{pgfscope}%
\pgfpathrectangle{\pgfqpoint{0.100000in}{0.212622in}}{\pgfqpoint{3.696000in}{3.696000in}}%
\pgfusepath{clip}%
\pgfsetbuttcap%
\pgfsetroundjoin%
\definecolor{currentfill}{rgb}{0.121569,0.466667,0.705882}%
\pgfsetfillcolor{currentfill}%
\pgfsetfillopacity{0.395289}%
\pgfsetlinewidth{1.003750pt}%
\definecolor{currentstroke}{rgb}{0.121569,0.466667,0.705882}%
\pgfsetstrokecolor{currentstroke}%
\pgfsetstrokeopacity{0.395289}%
\pgfsetdash{}{0pt}%
\pgfpathmoveto{\pgfqpoint{1.503282in}{2.107224in}}%
\pgfpathcurveto{\pgfqpoint{1.511518in}{2.107224in}}{\pgfqpoint{1.519418in}{2.110496in}}{\pgfqpoint{1.525242in}{2.116320in}}%
\pgfpathcurveto{\pgfqpoint{1.531066in}{2.122144in}}{\pgfqpoint{1.534338in}{2.130044in}}{\pgfqpoint{1.534338in}{2.138281in}}%
\pgfpathcurveto{\pgfqpoint{1.534338in}{2.146517in}}{\pgfqpoint{1.531066in}{2.154417in}}{\pgfqpoint{1.525242in}{2.160241in}}%
\pgfpathcurveto{\pgfqpoint{1.519418in}{2.166065in}}{\pgfqpoint{1.511518in}{2.169337in}}{\pgfqpoint{1.503282in}{2.169337in}}%
\pgfpathcurveto{\pgfqpoint{1.495045in}{2.169337in}}{\pgfqpoint{1.487145in}{2.166065in}}{\pgfqpoint{1.481322in}{2.160241in}}%
\pgfpathcurveto{\pgfqpoint{1.475498in}{2.154417in}}{\pgfqpoint{1.472225in}{2.146517in}}{\pgfqpoint{1.472225in}{2.138281in}}%
\pgfpathcurveto{\pgfqpoint{1.472225in}{2.130044in}}{\pgfqpoint{1.475498in}{2.122144in}}{\pgfqpoint{1.481322in}{2.116320in}}%
\pgfpathcurveto{\pgfqpoint{1.487145in}{2.110496in}}{\pgfqpoint{1.495045in}{2.107224in}}{\pgfqpoint{1.503282in}{2.107224in}}%
\pgfpathclose%
\pgfusepath{stroke,fill}%
\end{pgfscope}%
\begin{pgfscope}%
\pgfpathrectangle{\pgfqpoint{0.100000in}{0.212622in}}{\pgfqpoint{3.696000in}{3.696000in}}%
\pgfusepath{clip}%
\pgfsetbuttcap%
\pgfsetroundjoin%
\definecolor{currentfill}{rgb}{0.121569,0.466667,0.705882}%
\pgfsetfillcolor{currentfill}%
\pgfsetfillopacity{0.395545}%
\pgfsetlinewidth{1.003750pt}%
\definecolor{currentstroke}{rgb}{0.121569,0.466667,0.705882}%
\pgfsetstrokecolor{currentstroke}%
\pgfsetstrokeopacity{0.395545}%
\pgfsetdash{}{0pt}%
\pgfpathmoveto{\pgfqpoint{2.306998in}{1.980300in}}%
\pgfpathcurveto{\pgfqpoint{2.315234in}{1.980300in}}{\pgfqpoint{2.323134in}{1.983572in}}{\pgfqpoint{2.328958in}{1.989396in}}%
\pgfpathcurveto{\pgfqpoint{2.334782in}{1.995220in}}{\pgfqpoint{2.338054in}{2.003120in}}{\pgfqpoint{2.338054in}{2.011357in}}%
\pgfpathcurveto{\pgfqpoint{2.338054in}{2.019593in}}{\pgfqpoint{2.334782in}{2.027493in}}{\pgfqpoint{2.328958in}{2.033317in}}%
\pgfpathcurveto{\pgfqpoint{2.323134in}{2.039141in}}{\pgfqpoint{2.315234in}{2.042413in}}{\pgfqpoint{2.306998in}{2.042413in}}%
\pgfpathcurveto{\pgfqpoint{2.298761in}{2.042413in}}{\pgfqpoint{2.290861in}{2.039141in}}{\pgfqpoint{2.285037in}{2.033317in}}%
\pgfpathcurveto{\pgfqpoint{2.279213in}{2.027493in}}{\pgfqpoint{2.275941in}{2.019593in}}{\pgfqpoint{2.275941in}{2.011357in}}%
\pgfpathcurveto{\pgfqpoint{2.275941in}{2.003120in}}{\pgfqpoint{2.279213in}{1.995220in}}{\pgfqpoint{2.285037in}{1.989396in}}%
\pgfpathcurveto{\pgfqpoint{2.290861in}{1.983572in}}{\pgfqpoint{2.298761in}{1.980300in}}{\pgfqpoint{2.306998in}{1.980300in}}%
\pgfpathclose%
\pgfusepath{stroke,fill}%
\end{pgfscope}%
\begin{pgfscope}%
\pgfpathrectangle{\pgfqpoint{0.100000in}{0.212622in}}{\pgfqpoint{3.696000in}{3.696000in}}%
\pgfusepath{clip}%
\pgfsetbuttcap%
\pgfsetroundjoin%
\definecolor{currentfill}{rgb}{0.121569,0.466667,0.705882}%
\pgfsetfillcolor{currentfill}%
\pgfsetfillopacity{0.395988}%
\pgfsetlinewidth{1.003750pt}%
\definecolor{currentstroke}{rgb}{0.121569,0.466667,0.705882}%
\pgfsetstrokecolor{currentstroke}%
\pgfsetstrokeopacity{0.395988}%
\pgfsetdash{}{0pt}%
\pgfpathmoveto{\pgfqpoint{2.313344in}{1.978425in}}%
\pgfpathcurveto{\pgfqpoint{2.321580in}{1.978425in}}{\pgfqpoint{2.329480in}{1.981697in}}{\pgfqpoint{2.335304in}{1.987521in}}%
\pgfpathcurveto{\pgfqpoint{2.341128in}{1.993345in}}{\pgfqpoint{2.344400in}{2.001245in}}{\pgfqpoint{2.344400in}{2.009481in}}%
\pgfpathcurveto{\pgfqpoint{2.344400in}{2.017717in}}{\pgfqpoint{2.341128in}{2.025618in}}{\pgfqpoint{2.335304in}{2.031441in}}%
\pgfpathcurveto{\pgfqpoint{2.329480in}{2.037265in}}{\pgfqpoint{2.321580in}{2.040538in}}{\pgfqpoint{2.313344in}{2.040538in}}%
\pgfpathcurveto{\pgfqpoint{2.305108in}{2.040538in}}{\pgfqpoint{2.297207in}{2.037265in}}{\pgfqpoint{2.291384in}{2.031441in}}%
\pgfpathcurveto{\pgfqpoint{2.285560in}{2.025618in}}{\pgfqpoint{2.282287in}{2.017717in}}{\pgfqpoint{2.282287in}{2.009481in}}%
\pgfpathcurveto{\pgfqpoint{2.282287in}{2.001245in}}{\pgfqpoint{2.285560in}{1.993345in}}{\pgfqpoint{2.291384in}{1.987521in}}%
\pgfpathcurveto{\pgfqpoint{2.297207in}{1.981697in}}{\pgfqpoint{2.305108in}{1.978425in}}{\pgfqpoint{2.313344in}{1.978425in}}%
\pgfpathclose%
\pgfusepath{stroke,fill}%
\end{pgfscope}%
\begin{pgfscope}%
\pgfpathrectangle{\pgfqpoint{0.100000in}{0.212622in}}{\pgfqpoint{3.696000in}{3.696000in}}%
\pgfusepath{clip}%
\pgfsetbuttcap%
\pgfsetroundjoin%
\definecolor{currentfill}{rgb}{0.121569,0.466667,0.705882}%
\pgfsetfillcolor{currentfill}%
\pgfsetfillopacity{0.396946}%
\pgfsetlinewidth{1.003750pt}%
\definecolor{currentstroke}{rgb}{0.121569,0.466667,0.705882}%
\pgfsetstrokecolor{currentstroke}%
\pgfsetstrokeopacity{0.396946}%
\pgfsetdash{}{0pt}%
\pgfpathmoveto{\pgfqpoint{1.499273in}{2.107513in}}%
\pgfpathcurveto{\pgfqpoint{1.507509in}{2.107513in}}{\pgfqpoint{1.515409in}{2.110785in}}{\pgfqpoint{1.521233in}{2.116609in}}%
\pgfpathcurveto{\pgfqpoint{1.527057in}{2.122433in}}{\pgfqpoint{1.530329in}{2.130333in}}{\pgfqpoint{1.530329in}{2.138569in}}%
\pgfpathcurveto{\pgfqpoint{1.530329in}{2.146805in}}{\pgfqpoint{1.527057in}{2.154705in}}{\pgfqpoint{1.521233in}{2.160529in}}%
\pgfpathcurveto{\pgfqpoint{1.515409in}{2.166353in}}{\pgfqpoint{1.507509in}{2.169626in}}{\pgfqpoint{1.499273in}{2.169626in}}%
\pgfpathcurveto{\pgfqpoint{1.491037in}{2.169626in}}{\pgfqpoint{1.483137in}{2.166353in}}{\pgfqpoint{1.477313in}{2.160529in}}%
\pgfpathcurveto{\pgfqpoint{1.471489in}{2.154705in}}{\pgfqpoint{1.468216in}{2.146805in}}{\pgfqpoint{1.468216in}{2.138569in}}%
\pgfpathcurveto{\pgfqpoint{1.468216in}{2.130333in}}{\pgfqpoint{1.471489in}{2.122433in}}{\pgfqpoint{1.477313in}{2.116609in}}%
\pgfpathcurveto{\pgfqpoint{1.483137in}{2.110785in}}{\pgfqpoint{1.491037in}{2.107513in}}{\pgfqpoint{1.499273in}{2.107513in}}%
\pgfpathclose%
\pgfusepath{stroke,fill}%
\end{pgfscope}%
\begin{pgfscope}%
\pgfpathrectangle{\pgfqpoint{0.100000in}{0.212622in}}{\pgfqpoint{3.696000in}{3.696000in}}%
\pgfusepath{clip}%
\pgfsetbuttcap%
\pgfsetroundjoin%
\definecolor{currentfill}{rgb}{0.121569,0.466667,0.705882}%
\pgfsetfillcolor{currentfill}%
\pgfsetfillopacity{0.396971}%
\pgfsetlinewidth{1.003750pt}%
\definecolor{currentstroke}{rgb}{0.121569,0.466667,0.705882}%
\pgfsetstrokecolor{currentstroke}%
\pgfsetstrokeopacity{0.396971}%
\pgfsetdash{}{0pt}%
\pgfpathmoveto{\pgfqpoint{2.319449in}{1.976920in}}%
\pgfpathcurveto{\pgfqpoint{2.327685in}{1.976920in}}{\pgfqpoint{2.335586in}{1.980193in}}{\pgfqpoint{2.341409in}{1.986017in}}%
\pgfpathcurveto{\pgfqpoint{2.347233in}{1.991840in}}{\pgfqpoint{2.350506in}{1.999740in}}{\pgfqpoint{2.350506in}{2.007977in}}%
\pgfpathcurveto{\pgfqpoint{2.350506in}{2.016213in}}{\pgfqpoint{2.347233in}{2.024113in}}{\pgfqpoint{2.341409in}{2.029937in}}%
\pgfpathcurveto{\pgfqpoint{2.335586in}{2.035761in}}{\pgfqpoint{2.327685in}{2.039033in}}{\pgfqpoint{2.319449in}{2.039033in}}%
\pgfpathcurveto{\pgfqpoint{2.311213in}{2.039033in}}{\pgfqpoint{2.303313in}{2.035761in}}{\pgfqpoint{2.297489in}{2.029937in}}%
\pgfpathcurveto{\pgfqpoint{2.291665in}{2.024113in}}{\pgfqpoint{2.288393in}{2.016213in}}{\pgfqpoint{2.288393in}{2.007977in}}%
\pgfpathcurveto{\pgfqpoint{2.288393in}{1.999740in}}{\pgfqpoint{2.291665in}{1.991840in}}{\pgfqpoint{2.297489in}{1.986017in}}%
\pgfpathcurveto{\pgfqpoint{2.303313in}{1.980193in}}{\pgfqpoint{2.311213in}{1.976920in}}{\pgfqpoint{2.319449in}{1.976920in}}%
\pgfpathclose%
\pgfusepath{stroke,fill}%
\end{pgfscope}%
\begin{pgfscope}%
\pgfpathrectangle{\pgfqpoint{0.100000in}{0.212622in}}{\pgfqpoint{3.696000in}{3.696000in}}%
\pgfusepath{clip}%
\pgfsetbuttcap%
\pgfsetroundjoin%
\definecolor{currentfill}{rgb}{0.121569,0.466667,0.705882}%
\pgfsetfillcolor{currentfill}%
\pgfsetfillopacity{0.397785}%
\pgfsetlinewidth{1.003750pt}%
\definecolor{currentstroke}{rgb}{0.121569,0.466667,0.705882}%
\pgfsetstrokecolor{currentstroke}%
\pgfsetstrokeopacity{0.397785}%
\pgfsetdash{}{0pt}%
\pgfpathmoveto{\pgfqpoint{2.322138in}{1.976768in}}%
\pgfpathcurveto{\pgfqpoint{2.330374in}{1.976768in}}{\pgfqpoint{2.338274in}{1.980040in}}{\pgfqpoint{2.344098in}{1.985864in}}%
\pgfpathcurveto{\pgfqpoint{2.349922in}{1.991688in}}{\pgfqpoint{2.353194in}{1.999588in}}{\pgfqpoint{2.353194in}{2.007824in}}%
\pgfpathcurveto{\pgfqpoint{2.353194in}{2.016060in}}{\pgfqpoint{2.349922in}{2.023960in}}{\pgfqpoint{2.344098in}{2.029784in}}%
\pgfpathcurveto{\pgfqpoint{2.338274in}{2.035608in}}{\pgfqpoint{2.330374in}{2.038881in}}{\pgfqpoint{2.322138in}{2.038881in}}%
\pgfpathcurveto{\pgfqpoint{2.313901in}{2.038881in}}{\pgfqpoint{2.306001in}{2.035608in}}{\pgfqpoint{2.300177in}{2.029784in}}%
\pgfpathcurveto{\pgfqpoint{2.294353in}{2.023960in}}{\pgfqpoint{2.291081in}{2.016060in}}{\pgfqpoint{2.291081in}{2.007824in}}%
\pgfpathcurveto{\pgfqpoint{2.291081in}{1.999588in}}{\pgfqpoint{2.294353in}{1.991688in}}{\pgfqpoint{2.300177in}{1.985864in}}%
\pgfpathcurveto{\pgfqpoint{2.306001in}{1.980040in}}{\pgfqpoint{2.313901in}{1.976768in}}{\pgfqpoint{2.322138in}{1.976768in}}%
\pgfpathclose%
\pgfusepath{stroke,fill}%
\end{pgfscope}%
\begin{pgfscope}%
\pgfpathrectangle{\pgfqpoint{0.100000in}{0.212622in}}{\pgfqpoint{3.696000in}{3.696000in}}%
\pgfusepath{clip}%
\pgfsetbuttcap%
\pgfsetroundjoin%
\definecolor{currentfill}{rgb}{0.121569,0.466667,0.705882}%
\pgfsetfillcolor{currentfill}%
\pgfsetfillopacity{0.398295}%
\pgfsetlinewidth{1.003750pt}%
\definecolor{currentstroke}{rgb}{0.121569,0.466667,0.705882}%
\pgfsetstrokecolor{currentstroke}%
\pgfsetstrokeopacity{0.398295}%
\pgfsetdash{}{0pt}%
\pgfpathmoveto{\pgfqpoint{2.326743in}{1.975612in}}%
\pgfpathcurveto{\pgfqpoint{2.334980in}{1.975612in}}{\pgfqpoint{2.342880in}{1.978885in}}{\pgfqpoint{2.348704in}{1.984709in}}%
\pgfpathcurveto{\pgfqpoint{2.354528in}{1.990533in}}{\pgfqpoint{2.357800in}{1.998433in}}{\pgfqpoint{2.357800in}{2.006669in}}%
\pgfpathcurveto{\pgfqpoint{2.357800in}{2.014905in}}{\pgfqpoint{2.354528in}{2.022805in}}{\pgfqpoint{2.348704in}{2.028629in}}%
\pgfpathcurveto{\pgfqpoint{2.342880in}{2.034453in}}{\pgfqpoint{2.334980in}{2.037725in}}{\pgfqpoint{2.326743in}{2.037725in}}%
\pgfpathcurveto{\pgfqpoint{2.318507in}{2.037725in}}{\pgfqpoint{2.310607in}{2.034453in}}{\pgfqpoint{2.304783in}{2.028629in}}%
\pgfpathcurveto{\pgfqpoint{2.298959in}{2.022805in}}{\pgfqpoint{2.295687in}{2.014905in}}{\pgfqpoint{2.295687in}{2.006669in}}%
\pgfpathcurveto{\pgfqpoint{2.295687in}{1.998433in}}{\pgfqpoint{2.298959in}{1.990533in}}{\pgfqpoint{2.304783in}{1.984709in}}%
\pgfpathcurveto{\pgfqpoint{2.310607in}{1.978885in}}{\pgfqpoint{2.318507in}{1.975612in}}{\pgfqpoint{2.326743in}{1.975612in}}%
\pgfpathclose%
\pgfusepath{stroke,fill}%
\end{pgfscope}%
\begin{pgfscope}%
\pgfpathrectangle{\pgfqpoint{0.100000in}{0.212622in}}{\pgfqpoint{3.696000in}{3.696000in}}%
\pgfusepath{clip}%
\pgfsetbuttcap%
\pgfsetroundjoin%
\definecolor{currentfill}{rgb}{0.121569,0.466667,0.705882}%
\pgfsetfillcolor{currentfill}%
\pgfsetfillopacity{0.398753}%
\pgfsetlinewidth{1.003750pt}%
\definecolor{currentstroke}{rgb}{0.121569,0.466667,0.705882}%
\pgfsetstrokecolor{currentstroke}%
\pgfsetstrokeopacity{0.398753}%
\pgfsetdash{}{0pt}%
\pgfpathmoveto{\pgfqpoint{1.501940in}{2.108835in}}%
\pgfpathcurveto{\pgfqpoint{1.510177in}{2.108835in}}{\pgfqpoint{1.518077in}{2.112108in}}{\pgfqpoint{1.523901in}{2.117932in}}%
\pgfpathcurveto{\pgfqpoint{1.529725in}{2.123756in}}{\pgfqpoint{1.532997in}{2.131656in}}{\pgfqpoint{1.532997in}{2.139892in}}%
\pgfpathcurveto{\pgfqpoint{1.532997in}{2.148128in}}{\pgfqpoint{1.529725in}{2.156028in}}{\pgfqpoint{1.523901in}{2.161852in}}%
\pgfpathcurveto{\pgfqpoint{1.518077in}{2.167676in}}{\pgfqpoint{1.510177in}{2.170948in}}{\pgfqpoint{1.501940in}{2.170948in}}%
\pgfpathcurveto{\pgfqpoint{1.493704in}{2.170948in}}{\pgfqpoint{1.485804in}{2.167676in}}{\pgfqpoint{1.479980in}{2.161852in}}%
\pgfpathcurveto{\pgfqpoint{1.474156in}{2.156028in}}{\pgfqpoint{1.470884in}{2.148128in}}{\pgfqpoint{1.470884in}{2.139892in}}%
\pgfpathcurveto{\pgfqpoint{1.470884in}{2.131656in}}{\pgfqpoint{1.474156in}{2.123756in}}{\pgfqpoint{1.479980in}{2.117932in}}%
\pgfpathcurveto{\pgfqpoint{1.485804in}{2.112108in}}{\pgfqpoint{1.493704in}{2.108835in}}{\pgfqpoint{1.501940in}{2.108835in}}%
\pgfpathclose%
\pgfusepath{stroke,fill}%
\end{pgfscope}%
\begin{pgfscope}%
\pgfpathrectangle{\pgfqpoint{0.100000in}{0.212622in}}{\pgfqpoint{3.696000in}{3.696000in}}%
\pgfusepath{clip}%
\pgfsetbuttcap%
\pgfsetroundjoin%
\definecolor{currentfill}{rgb}{0.121569,0.466667,0.705882}%
\pgfsetfillcolor{currentfill}%
\pgfsetfillopacity{0.399017}%
\pgfsetlinewidth{1.003750pt}%
\definecolor{currentstroke}{rgb}{0.121569,0.466667,0.705882}%
\pgfsetstrokecolor{currentstroke}%
\pgfsetstrokeopacity{0.399017}%
\pgfsetdash{}{0pt}%
\pgfpathmoveto{\pgfqpoint{2.331740in}{1.974272in}}%
\pgfpathcurveto{\pgfqpoint{2.339976in}{1.974272in}}{\pgfqpoint{2.347876in}{1.977544in}}{\pgfqpoint{2.353700in}{1.983368in}}%
\pgfpathcurveto{\pgfqpoint{2.359524in}{1.989192in}}{\pgfqpoint{2.362796in}{1.997092in}}{\pgfqpoint{2.362796in}{2.005329in}}%
\pgfpathcurveto{\pgfqpoint{2.362796in}{2.013565in}}{\pgfqpoint{2.359524in}{2.021465in}}{\pgfqpoint{2.353700in}{2.027289in}}%
\pgfpathcurveto{\pgfqpoint{2.347876in}{2.033113in}}{\pgfqpoint{2.339976in}{2.036385in}}{\pgfqpoint{2.331740in}{2.036385in}}%
\pgfpathcurveto{\pgfqpoint{2.323504in}{2.036385in}}{\pgfqpoint{2.315603in}{2.033113in}}{\pgfqpoint{2.309780in}{2.027289in}}%
\pgfpathcurveto{\pgfqpoint{2.303956in}{2.021465in}}{\pgfqpoint{2.300683in}{2.013565in}}{\pgfqpoint{2.300683in}{2.005329in}}%
\pgfpathcurveto{\pgfqpoint{2.300683in}{1.997092in}}{\pgfqpoint{2.303956in}{1.989192in}}{\pgfqpoint{2.309780in}{1.983368in}}%
\pgfpathcurveto{\pgfqpoint{2.315603in}{1.977544in}}{\pgfqpoint{2.323504in}{1.974272in}}{\pgfqpoint{2.331740in}{1.974272in}}%
\pgfpathclose%
\pgfusepath{stroke,fill}%
\end{pgfscope}%
\begin{pgfscope}%
\pgfpathrectangle{\pgfqpoint{0.100000in}{0.212622in}}{\pgfqpoint{3.696000in}{3.696000in}}%
\pgfusepath{clip}%
\pgfsetbuttcap%
\pgfsetroundjoin%
\definecolor{currentfill}{rgb}{0.121569,0.466667,0.705882}%
\pgfsetfillcolor{currentfill}%
\pgfsetfillopacity{0.399403}%
\pgfsetlinewidth{1.003750pt}%
\definecolor{currentstroke}{rgb}{0.121569,0.466667,0.705882}%
\pgfsetstrokecolor{currentstroke}%
\pgfsetstrokeopacity{0.399403}%
\pgfsetdash{}{0pt}%
\pgfpathmoveto{\pgfqpoint{2.334567in}{1.973689in}}%
\pgfpathcurveto{\pgfqpoint{2.342804in}{1.973689in}}{\pgfqpoint{2.350704in}{1.976961in}}{\pgfqpoint{2.356528in}{1.982785in}}%
\pgfpathcurveto{\pgfqpoint{2.362352in}{1.988609in}}{\pgfqpoint{2.365624in}{1.996509in}}{\pgfqpoint{2.365624in}{2.004746in}}%
\pgfpathcurveto{\pgfqpoint{2.365624in}{2.012982in}}{\pgfqpoint{2.362352in}{2.020882in}}{\pgfqpoint{2.356528in}{2.026706in}}%
\pgfpathcurveto{\pgfqpoint{2.350704in}{2.032530in}}{\pgfqpoint{2.342804in}{2.035802in}}{\pgfqpoint{2.334567in}{2.035802in}}%
\pgfpathcurveto{\pgfqpoint{2.326331in}{2.035802in}}{\pgfqpoint{2.318431in}{2.032530in}}{\pgfqpoint{2.312607in}{2.026706in}}%
\pgfpathcurveto{\pgfqpoint{2.306783in}{2.020882in}}{\pgfqpoint{2.303511in}{2.012982in}}{\pgfqpoint{2.303511in}{2.004746in}}%
\pgfpathcurveto{\pgfqpoint{2.303511in}{1.996509in}}{\pgfqpoint{2.306783in}{1.988609in}}{\pgfqpoint{2.312607in}{1.982785in}}%
\pgfpathcurveto{\pgfqpoint{2.318431in}{1.976961in}}{\pgfqpoint{2.326331in}{1.973689in}}{\pgfqpoint{2.334567in}{1.973689in}}%
\pgfpathclose%
\pgfusepath{stroke,fill}%
\end{pgfscope}%
\begin{pgfscope}%
\pgfpathrectangle{\pgfqpoint{0.100000in}{0.212622in}}{\pgfqpoint{3.696000in}{3.696000in}}%
\pgfusepath{clip}%
\pgfsetbuttcap%
\pgfsetroundjoin%
\definecolor{currentfill}{rgb}{0.121569,0.466667,0.705882}%
\pgfsetfillcolor{currentfill}%
\pgfsetfillopacity{0.399851}%
\pgfsetlinewidth{1.003750pt}%
\definecolor{currentstroke}{rgb}{0.121569,0.466667,0.705882}%
\pgfsetstrokecolor{currentstroke}%
\pgfsetstrokeopacity{0.399851}%
\pgfsetdash{}{0pt}%
\pgfpathmoveto{\pgfqpoint{2.337818in}{1.972977in}}%
\pgfpathcurveto{\pgfqpoint{2.346054in}{1.972977in}}{\pgfqpoint{2.353954in}{1.976250in}}{\pgfqpoint{2.359778in}{1.982074in}}%
\pgfpathcurveto{\pgfqpoint{2.365602in}{1.987898in}}{\pgfqpoint{2.368874in}{1.995798in}}{\pgfqpoint{2.368874in}{2.004034in}}%
\pgfpathcurveto{\pgfqpoint{2.368874in}{2.012270in}}{\pgfqpoint{2.365602in}{2.020170in}}{\pgfqpoint{2.359778in}{2.025994in}}%
\pgfpathcurveto{\pgfqpoint{2.353954in}{2.031818in}}{\pgfqpoint{2.346054in}{2.035090in}}{\pgfqpoint{2.337818in}{2.035090in}}%
\pgfpathcurveto{\pgfqpoint{2.329582in}{2.035090in}}{\pgfqpoint{2.321681in}{2.031818in}}{\pgfqpoint{2.315858in}{2.025994in}}%
\pgfpathcurveto{\pgfqpoint{2.310034in}{2.020170in}}{\pgfqpoint{2.306761in}{2.012270in}}{\pgfqpoint{2.306761in}{2.004034in}}%
\pgfpathcurveto{\pgfqpoint{2.306761in}{1.995798in}}{\pgfqpoint{2.310034in}{1.987898in}}{\pgfqpoint{2.315858in}{1.982074in}}%
\pgfpathcurveto{\pgfqpoint{2.321681in}{1.976250in}}{\pgfqpoint{2.329582in}{1.972977in}}{\pgfqpoint{2.337818in}{1.972977in}}%
\pgfpathclose%
\pgfusepath{stroke,fill}%
\end{pgfscope}%
\begin{pgfscope}%
\pgfpathrectangle{\pgfqpoint{0.100000in}{0.212622in}}{\pgfqpoint{3.696000in}{3.696000in}}%
\pgfusepath{clip}%
\pgfsetbuttcap%
\pgfsetroundjoin%
\definecolor{currentfill}{rgb}{0.121569,0.466667,0.705882}%
\pgfsetfillcolor{currentfill}%
\pgfsetfillopacity{0.400066}%
\pgfsetlinewidth{1.003750pt}%
\definecolor{currentstroke}{rgb}{0.121569,0.466667,0.705882}%
\pgfsetstrokecolor{currentstroke}%
\pgfsetstrokeopacity{0.400066}%
\pgfsetdash{}{0pt}%
\pgfpathmoveto{\pgfqpoint{2.339673in}{1.972571in}}%
\pgfpathcurveto{\pgfqpoint{2.347909in}{1.972571in}}{\pgfqpoint{2.355809in}{1.975843in}}{\pgfqpoint{2.361633in}{1.981667in}}%
\pgfpathcurveto{\pgfqpoint{2.367457in}{1.987491in}}{\pgfqpoint{2.370729in}{1.995391in}}{\pgfqpoint{2.370729in}{2.003628in}}%
\pgfpathcurveto{\pgfqpoint{2.370729in}{2.011864in}}{\pgfqpoint{2.367457in}{2.019764in}}{\pgfqpoint{2.361633in}{2.025588in}}%
\pgfpathcurveto{\pgfqpoint{2.355809in}{2.031412in}}{\pgfqpoint{2.347909in}{2.034684in}}{\pgfqpoint{2.339673in}{2.034684in}}%
\pgfpathcurveto{\pgfqpoint{2.331436in}{2.034684in}}{\pgfqpoint{2.323536in}{2.031412in}}{\pgfqpoint{2.317712in}{2.025588in}}%
\pgfpathcurveto{\pgfqpoint{2.311888in}{2.019764in}}{\pgfqpoint{2.308616in}{2.011864in}}{\pgfqpoint{2.308616in}{2.003628in}}%
\pgfpathcurveto{\pgfqpoint{2.308616in}{1.995391in}}{\pgfqpoint{2.311888in}{1.987491in}}{\pgfqpoint{2.317712in}{1.981667in}}%
\pgfpathcurveto{\pgfqpoint{2.323536in}{1.975843in}}{\pgfqpoint{2.331436in}{1.972571in}}{\pgfqpoint{2.339673in}{1.972571in}}%
\pgfpathclose%
\pgfusepath{stroke,fill}%
\end{pgfscope}%
\begin{pgfscope}%
\pgfpathrectangle{\pgfqpoint{0.100000in}{0.212622in}}{\pgfqpoint{3.696000in}{3.696000in}}%
\pgfusepath{clip}%
\pgfsetbuttcap%
\pgfsetroundjoin%
\definecolor{currentfill}{rgb}{0.121569,0.466667,0.705882}%
\pgfsetfillcolor{currentfill}%
\pgfsetfillopacity{0.400318}%
\pgfsetlinewidth{1.003750pt}%
\definecolor{currentstroke}{rgb}{0.121569,0.466667,0.705882}%
\pgfsetstrokecolor{currentstroke}%
\pgfsetstrokeopacity{0.400318}%
\pgfsetdash{}{0pt}%
\pgfpathmoveto{\pgfqpoint{1.499094in}{2.108906in}}%
\pgfpathcurveto{\pgfqpoint{1.507330in}{2.108906in}}{\pgfqpoint{1.515230in}{2.112178in}}{\pgfqpoint{1.521054in}{2.118002in}}%
\pgfpathcurveto{\pgfqpoint{1.526878in}{2.123826in}}{\pgfqpoint{1.530151in}{2.131726in}}{\pgfqpoint{1.530151in}{2.139962in}}%
\pgfpathcurveto{\pgfqpoint{1.530151in}{2.148199in}}{\pgfqpoint{1.526878in}{2.156099in}}{\pgfqpoint{1.521054in}{2.161923in}}%
\pgfpathcurveto{\pgfqpoint{1.515230in}{2.167747in}}{\pgfqpoint{1.507330in}{2.171019in}}{\pgfqpoint{1.499094in}{2.171019in}}%
\pgfpathcurveto{\pgfqpoint{1.490858in}{2.171019in}}{\pgfqpoint{1.482958in}{2.167747in}}{\pgfqpoint{1.477134in}{2.161923in}}%
\pgfpathcurveto{\pgfqpoint{1.471310in}{2.156099in}}{\pgfqpoint{1.468038in}{2.148199in}}{\pgfqpoint{1.468038in}{2.139962in}}%
\pgfpathcurveto{\pgfqpoint{1.468038in}{2.131726in}}{\pgfqpoint{1.471310in}{2.123826in}}{\pgfqpoint{1.477134in}{2.118002in}}%
\pgfpathcurveto{\pgfqpoint{1.482958in}{2.112178in}}{\pgfqpoint{1.490858in}{2.108906in}}{\pgfqpoint{1.499094in}{2.108906in}}%
\pgfpathclose%
\pgfusepath{stroke,fill}%
\end{pgfscope}%
\begin{pgfscope}%
\pgfpathrectangle{\pgfqpoint{0.100000in}{0.212622in}}{\pgfqpoint{3.696000in}{3.696000in}}%
\pgfusepath{clip}%
\pgfsetbuttcap%
\pgfsetroundjoin%
\definecolor{currentfill}{rgb}{0.121569,0.466667,0.705882}%
\pgfsetfillcolor{currentfill}%
\pgfsetfillopacity{0.400444}%
\pgfsetlinewidth{1.003750pt}%
\definecolor{currentstroke}{rgb}{0.121569,0.466667,0.705882}%
\pgfsetstrokecolor{currentstroke}%
\pgfsetstrokeopacity{0.400444}%
\pgfsetdash{}{0pt}%
\pgfpathmoveto{\pgfqpoint{2.341914in}{1.972197in}}%
\pgfpathcurveto{\pgfqpoint{2.350151in}{1.972197in}}{\pgfqpoint{2.358051in}{1.975470in}}{\pgfqpoint{2.363875in}{1.981293in}}%
\pgfpathcurveto{\pgfqpoint{2.369699in}{1.987117in}}{\pgfqpoint{2.372971in}{1.995017in}}{\pgfqpoint{2.372971in}{2.003254in}}%
\pgfpathcurveto{\pgfqpoint{2.372971in}{2.011490in}}{\pgfqpoint{2.369699in}{2.019390in}}{\pgfqpoint{2.363875in}{2.025214in}}%
\pgfpathcurveto{\pgfqpoint{2.358051in}{2.031038in}}{\pgfqpoint{2.350151in}{2.034310in}}{\pgfqpoint{2.341914in}{2.034310in}}%
\pgfpathcurveto{\pgfqpoint{2.333678in}{2.034310in}}{\pgfqpoint{2.325778in}{2.031038in}}{\pgfqpoint{2.319954in}{2.025214in}}%
\pgfpathcurveto{\pgfqpoint{2.314130in}{2.019390in}}{\pgfqpoint{2.310858in}{2.011490in}}{\pgfqpoint{2.310858in}{2.003254in}}%
\pgfpathcurveto{\pgfqpoint{2.310858in}{1.995017in}}{\pgfqpoint{2.314130in}{1.987117in}}{\pgfqpoint{2.319954in}{1.981293in}}%
\pgfpathcurveto{\pgfqpoint{2.325778in}{1.975470in}}{\pgfqpoint{2.333678in}{1.972197in}}{\pgfqpoint{2.341914in}{1.972197in}}%
\pgfpathclose%
\pgfusepath{stroke,fill}%
\end{pgfscope}%
\begin{pgfscope}%
\pgfpathrectangle{\pgfqpoint{0.100000in}{0.212622in}}{\pgfqpoint{3.696000in}{3.696000in}}%
\pgfusepath{clip}%
\pgfsetbuttcap%
\pgfsetroundjoin%
\definecolor{currentfill}{rgb}{0.121569,0.466667,0.705882}%
\pgfsetfillcolor{currentfill}%
\pgfsetfillopacity{0.400923}%
\pgfsetlinewidth{1.003750pt}%
\definecolor{currentstroke}{rgb}{0.121569,0.466667,0.705882}%
\pgfsetstrokecolor{currentstroke}%
\pgfsetstrokeopacity{0.400923}%
\pgfsetdash{}{0pt}%
\pgfpathmoveto{\pgfqpoint{2.344786in}{1.971572in}}%
\pgfpathcurveto{\pgfqpoint{2.353023in}{1.971572in}}{\pgfqpoint{2.360923in}{1.974844in}}{\pgfqpoint{2.366747in}{1.980668in}}%
\pgfpathcurveto{\pgfqpoint{2.372570in}{1.986492in}}{\pgfqpoint{2.375843in}{1.994392in}}{\pgfqpoint{2.375843in}{2.002628in}}%
\pgfpathcurveto{\pgfqpoint{2.375843in}{2.010864in}}{\pgfqpoint{2.372570in}{2.018764in}}{\pgfqpoint{2.366747in}{2.024588in}}%
\pgfpathcurveto{\pgfqpoint{2.360923in}{2.030412in}}{\pgfqpoint{2.353023in}{2.033685in}}{\pgfqpoint{2.344786in}{2.033685in}}%
\pgfpathcurveto{\pgfqpoint{2.336550in}{2.033685in}}{\pgfqpoint{2.328650in}{2.030412in}}{\pgfqpoint{2.322826in}{2.024588in}}%
\pgfpathcurveto{\pgfqpoint{2.317002in}{2.018764in}}{\pgfqpoint{2.313730in}{2.010864in}}{\pgfqpoint{2.313730in}{2.002628in}}%
\pgfpathcurveto{\pgfqpoint{2.313730in}{1.994392in}}{\pgfqpoint{2.317002in}{1.986492in}}{\pgfqpoint{2.322826in}{1.980668in}}%
\pgfpathcurveto{\pgfqpoint{2.328650in}{1.974844in}}{\pgfqpoint{2.336550in}{1.971572in}}{\pgfqpoint{2.344786in}{1.971572in}}%
\pgfpathclose%
\pgfusepath{stroke,fill}%
\end{pgfscope}%
\begin{pgfscope}%
\pgfpathrectangle{\pgfqpoint{0.100000in}{0.212622in}}{\pgfqpoint{3.696000in}{3.696000in}}%
\pgfusepath{clip}%
\pgfsetbuttcap%
\pgfsetroundjoin%
\definecolor{currentfill}{rgb}{0.121569,0.466667,0.705882}%
\pgfsetfillcolor{currentfill}%
\pgfsetfillopacity{0.401500}%
\pgfsetlinewidth{1.003750pt}%
\definecolor{currentstroke}{rgb}{0.121569,0.466667,0.705882}%
\pgfsetstrokecolor{currentstroke}%
\pgfsetstrokeopacity{0.401500}%
\pgfsetdash{}{0pt}%
\pgfpathmoveto{\pgfqpoint{2.348732in}{1.970811in}}%
\pgfpathcurveto{\pgfqpoint{2.356969in}{1.970811in}}{\pgfqpoint{2.364869in}{1.974084in}}{\pgfqpoint{2.370693in}{1.979908in}}%
\pgfpathcurveto{\pgfqpoint{2.376517in}{1.985732in}}{\pgfqpoint{2.379789in}{1.993632in}}{\pgfqpoint{2.379789in}{2.001868in}}%
\pgfpathcurveto{\pgfqpoint{2.379789in}{2.010104in}}{\pgfqpoint{2.376517in}{2.018004in}}{\pgfqpoint{2.370693in}{2.023828in}}%
\pgfpathcurveto{\pgfqpoint{2.364869in}{2.029652in}}{\pgfqpoint{2.356969in}{2.032924in}}{\pgfqpoint{2.348732in}{2.032924in}}%
\pgfpathcurveto{\pgfqpoint{2.340496in}{2.032924in}}{\pgfqpoint{2.332596in}{2.029652in}}{\pgfqpoint{2.326772in}{2.023828in}}%
\pgfpathcurveto{\pgfqpoint{2.320948in}{2.018004in}}{\pgfqpoint{2.317676in}{2.010104in}}{\pgfqpoint{2.317676in}{2.001868in}}%
\pgfpathcurveto{\pgfqpoint{2.317676in}{1.993632in}}{\pgfqpoint{2.320948in}{1.985732in}}{\pgfqpoint{2.326772in}{1.979908in}}%
\pgfpathcurveto{\pgfqpoint{2.332596in}{1.974084in}}{\pgfqpoint{2.340496in}{1.970811in}}{\pgfqpoint{2.348732in}{1.970811in}}%
\pgfpathclose%
\pgfusepath{stroke,fill}%
\end{pgfscope}%
\begin{pgfscope}%
\pgfpathrectangle{\pgfqpoint{0.100000in}{0.212622in}}{\pgfqpoint{3.696000in}{3.696000in}}%
\pgfusepath{clip}%
\pgfsetbuttcap%
\pgfsetroundjoin%
\definecolor{currentfill}{rgb}{0.121569,0.466667,0.705882}%
\pgfsetfillcolor{currentfill}%
\pgfsetfillopacity{0.401708}%
\pgfsetlinewidth{1.003750pt}%
\definecolor{currentstroke}{rgb}{0.121569,0.466667,0.705882}%
\pgfsetstrokecolor{currentstroke}%
\pgfsetstrokeopacity{0.401708}%
\pgfsetdash{}{0pt}%
\pgfpathmoveto{\pgfqpoint{1.495658in}{2.109195in}}%
\pgfpathcurveto{\pgfqpoint{1.503894in}{2.109195in}}{\pgfqpoint{1.511794in}{2.112467in}}{\pgfqpoint{1.517618in}{2.118291in}}%
\pgfpathcurveto{\pgfqpoint{1.523442in}{2.124115in}}{\pgfqpoint{1.526714in}{2.132015in}}{\pgfqpoint{1.526714in}{2.140252in}}%
\pgfpathcurveto{\pgfqpoint{1.526714in}{2.148488in}}{\pgfqpoint{1.523442in}{2.156388in}}{\pgfqpoint{1.517618in}{2.162212in}}%
\pgfpathcurveto{\pgfqpoint{1.511794in}{2.168036in}}{\pgfqpoint{1.503894in}{2.171308in}}{\pgfqpoint{1.495658in}{2.171308in}}%
\pgfpathcurveto{\pgfqpoint{1.487422in}{2.171308in}}{\pgfqpoint{1.479522in}{2.168036in}}{\pgfqpoint{1.473698in}{2.162212in}}%
\pgfpathcurveto{\pgfqpoint{1.467874in}{2.156388in}}{\pgfqpoint{1.464601in}{2.148488in}}{\pgfqpoint{1.464601in}{2.140252in}}%
\pgfpathcurveto{\pgfqpoint{1.464601in}{2.132015in}}{\pgfqpoint{1.467874in}{2.124115in}}{\pgfqpoint{1.473698in}{2.118291in}}%
\pgfpathcurveto{\pgfqpoint{1.479522in}{2.112467in}}{\pgfqpoint{1.487422in}{2.109195in}}{\pgfqpoint{1.495658in}{2.109195in}}%
\pgfpathclose%
\pgfusepath{stroke,fill}%
\end{pgfscope}%
\begin{pgfscope}%
\pgfpathrectangle{\pgfqpoint{0.100000in}{0.212622in}}{\pgfqpoint{3.696000in}{3.696000in}}%
\pgfusepath{clip}%
\pgfsetbuttcap%
\pgfsetroundjoin%
\definecolor{currentfill}{rgb}{0.121569,0.466667,0.705882}%
\pgfsetfillcolor{currentfill}%
\pgfsetfillopacity{0.402153}%
\pgfsetlinewidth{1.003750pt}%
\definecolor{currentstroke}{rgb}{0.121569,0.466667,0.705882}%
\pgfsetstrokecolor{currentstroke}%
\pgfsetstrokeopacity{0.402153}%
\pgfsetdash{}{0pt}%
\pgfpathmoveto{\pgfqpoint{2.353311in}{1.969926in}}%
\pgfpathcurveto{\pgfqpoint{2.361547in}{1.969926in}}{\pgfqpoint{2.369447in}{1.973198in}}{\pgfqpoint{2.375271in}{1.979022in}}%
\pgfpathcurveto{\pgfqpoint{2.381095in}{1.984846in}}{\pgfqpoint{2.384367in}{1.992746in}}{\pgfqpoint{2.384367in}{2.000982in}}%
\pgfpathcurveto{\pgfqpoint{2.384367in}{2.009218in}}{\pgfqpoint{2.381095in}{2.017118in}}{\pgfqpoint{2.375271in}{2.022942in}}%
\pgfpathcurveto{\pgfqpoint{2.369447in}{2.028766in}}{\pgfqpoint{2.361547in}{2.032039in}}{\pgfqpoint{2.353311in}{2.032039in}}%
\pgfpathcurveto{\pgfqpoint{2.345074in}{2.032039in}}{\pgfqpoint{2.337174in}{2.028766in}}{\pgfqpoint{2.331350in}{2.022942in}}%
\pgfpathcurveto{\pgfqpoint{2.325526in}{2.017118in}}{\pgfqpoint{2.322254in}{2.009218in}}{\pgfqpoint{2.322254in}{2.000982in}}%
\pgfpathcurveto{\pgfqpoint{2.322254in}{1.992746in}}{\pgfqpoint{2.325526in}{1.984846in}}{\pgfqpoint{2.331350in}{1.979022in}}%
\pgfpathcurveto{\pgfqpoint{2.337174in}{1.973198in}}{\pgfqpoint{2.345074in}{1.969926in}}{\pgfqpoint{2.353311in}{1.969926in}}%
\pgfpathclose%
\pgfusepath{stroke,fill}%
\end{pgfscope}%
\begin{pgfscope}%
\pgfpathrectangle{\pgfqpoint{0.100000in}{0.212622in}}{\pgfqpoint{3.696000in}{3.696000in}}%
\pgfusepath{clip}%
\pgfsetbuttcap%
\pgfsetroundjoin%
\definecolor{currentfill}{rgb}{0.121569,0.466667,0.705882}%
\pgfsetfillcolor{currentfill}%
\pgfsetfillopacity{0.402771}%
\pgfsetlinewidth{1.003750pt}%
\definecolor{currentstroke}{rgb}{0.121569,0.466667,0.705882}%
\pgfsetstrokecolor{currentstroke}%
\pgfsetstrokeopacity{0.402771}%
\pgfsetdash{}{0pt}%
\pgfpathmoveto{\pgfqpoint{1.494248in}{2.109248in}}%
\pgfpathcurveto{\pgfqpoint{1.502484in}{2.109248in}}{\pgfqpoint{1.510384in}{2.112520in}}{\pgfqpoint{1.516208in}{2.118344in}}%
\pgfpathcurveto{\pgfqpoint{1.522032in}{2.124168in}}{\pgfqpoint{1.525304in}{2.132068in}}{\pgfqpoint{1.525304in}{2.140304in}}%
\pgfpathcurveto{\pgfqpoint{1.525304in}{2.148541in}}{\pgfqpoint{1.522032in}{2.156441in}}{\pgfqpoint{1.516208in}{2.162265in}}%
\pgfpathcurveto{\pgfqpoint{1.510384in}{2.168089in}}{\pgfqpoint{1.502484in}{2.171361in}}{\pgfqpoint{1.494248in}{2.171361in}}%
\pgfpathcurveto{\pgfqpoint{1.486011in}{2.171361in}}{\pgfqpoint{1.478111in}{2.168089in}}{\pgfqpoint{1.472287in}{2.162265in}}%
\pgfpathcurveto{\pgfqpoint{1.466463in}{2.156441in}}{\pgfqpoint{1.463191in}{2.148541in}}{\pgfqpoint{1.463191in}{2.140304in}}%
\pgfpathcurveto{\pgfqpoint{1.463191in}{2.132068in}}{\pgfqpoint{1.466463in}{2.124168in}}{\pgfqpoint{1.472287in}{2.118344in}}%
\pgfpathcurveto{\pgfqpoint{1.478111in}{2.112520in}}{\pgfqpoint{1.486011in}{2.109248in}}{\pgfqpoint{1.494248in}{2.109248in}}%
\pgfpathclose%
\pgfusepath{stroke,fill}%
\end{pgfscope}%
\begin{pgfscope}%
\pgfpathrectangle{\pgfqpoint{0.100000in}{0.212622in}}{\pgfqpoint{3.696000in}{3.696000in}}%
\pgfusepath{clip}%
\pgfsetbuttcap%
\pgfsetroundjoin%
\definecolor{currentfill}{rgb}{0.121569,0.466667,0.705882}%
\pgfsetfillcolor{currentfill}%
\pgfsetfillopacity{0.402945}%
\pgfsetlinewidth{1.003750pt}%
\definecolor{currentstroke}{rgb}{0.121569,0.466667,0.705882}%
\pgfsetstrokecolor{currentstroke}%
\pgfsetstrokeopacity{0.402945}%
\pgfsetdash{}{0pt}%
\pgfpathmoveto{\pgfqpoint{2.358253in}{1.968703in}}%
\pgfpathcurveto{\pgfqpoint{2.366489in}{1.968703in}}{\pgfqpoint{2.374389in}{1.971975in}}{\pgfqpoint{2.380213in}{1.977799in}}%
\pgfpathcurveto{\pgfqpoint{2.386037in}{1.983623in}}{\pgfqpoint{2.389310in}{1.991523in}}{\pgfqpoint{2.389310in}{1.999759in}}%
\pgfpathcurveto{\pgfqpoint{2.389310in}{2.007996in}}{\pgfqpoint{2.386037in}{2.015896in}}{\pgfqpoint{2.380213in}{2.021720in}}%
\pgfpathcurveto{\pgfqpoint{2.374389in}{2.027544in}}{\pgfqpoint{2.366489in}{2.030816in}}{\pgfqpoint{2.358253in}{2.030816in}}%
\pgfpathcurveto{\pgfqpoint{2.350017in}{2.030816in}}{\pgfqpoint{2.342117in}{2.027544in}}{\pgfqpoint{2.336293in}{2.021720in}}%
\pgfpathcurveto{\pgfqpoint{2.330469in}{2.015896in}}{\pgfqpoint{2.327197in}{2.007996in}}{\pgfqpoint{2.327197in}{1.999759in}}%
\pgfpathcurveto{\pgfqpoint{2.327197in}{1.991523in}}{\pgfqpoint{2.330469in}{1.983623in}}{\pgfqpoint{2.336293in}{1.977799in}}%
\pgfpathcurveto{\pgfqpoint{2.342117in}{1.971975in}}{\pgfqpoint{2.350017in}{1.968703in}}{\pgfqpoint{2.358253in}{1.968703in}}%
\pgfpathclose%
\pgfusepath{stroke,fill}%
\end{pgfscope}%
\begin{pgfscope}%
\pgfpathrectangle{\pgfqpoint{0.100000in}{0.212622in}}{\pgfqpoint{3.696000in}{3.696000in}}%
\pgfusepath{clip}%
\pgfsetbuttcap%
\pgfsetroundjoin%
\definecolor{currentfill}{rgb}{0.121569,0.466667,0.705882}%
\pgfsetfillcolor{currentfill}%
\pgfsetfillopacity{0.403287}%
\pgfsetlinewidth{1.003750pt}%
\definecolor{currentstroke}{rgb}{0.121569,0.466667,0.705882}%
\pgfsetstrokecolor{currentstroke}%
\pgfsetstrokeopacity{0.403287}%
\pgfsetdash{}{0pt}%
\pgfpathmoveto{\pgfqpoint{1.493134in}{2.109288in}}%
\pgfpathcurveto{\pgfqpoint{1.501370in}{2.109288in}}{\pgfqpoint{1.509270in}{2.112561in}}{\pgfqpoint{1.515094in}{2.118385in}}%
\pgfpathcurveto{\pgfqpoint{1.520918in}{2.124209in}}{\pgfqpoint{1.524190in}{2.132109in}}{\pgfqpoint{1.524190in}{2.140345in}}%
\pgfpathcurveto{\pgfqpoint{1.524190in}{2.148581in}}{\pgfqpoint{1.520918in}{2.156481in}}{\pgfqpoint{1.515094in}{2.162305in}}%
\pgfpathcurveto{\pgfqpoint{1.509270in}{2.168129in}}{\pgfqpoint{1.501370in}{2.171401in}}{\pgfqpoint{1.493134in}{2.171401in}}%
\pgfpathcurveto{\pgfqpoint{1.484897in}{2.171401in}}{\pgfqpoint{1.476997in}{2.168129in}}{\pgfqpoint{1.471173in}{2.162305in}}%
\pgfpathcurveto{\pgfqpoint{1.465349in}{2.156481in}}{\pgfqpoint{1.462077in}{2.148581in}}{\pgfqpoint{1.462077in}{2.140345in}}%
\pgfpathcurveto{\pgfqpoint{1.462077in}{2.132109in}}{\pgfqpoint{1.465349in}{2.124209in}}{\pgfqpoint{1.471173in}{2.118385in}}%
\pgfpathcurveto{\pgfqpoint{1.476997in}{2.112561in}}{\pgfqpoint{1.484897in}{2.109288in}}{\pgfqpoint{1.493134in}{2.109288in}}%
\pgfpathclose%
\pgfusepath{stroke,fill}%
\end{pgfscope}%
\begin{pgfscope}%
\pgfpathrectangle{\pgfqpoint{0.100000in}{0.212622in}}{\pgfqpoint{3.696000in}{3.696000in}}%
\pgfusepath{clip}%
\pgfsetbuttcap%
\pgfsetroundjoin%
\definecolor{currentfill}{rgb}{0.121569,0.466667,0.705882}%
\pgfsetfillcolor{currentfill}%
\pgfsetfillopacity{0.404229}%
\pgfsetlinewidth{1.003750pt}%
\definecolor{currentstroke}{rgb}{0.121569,0.466667,0.705882}%
\pgfsetstrokecolor{currentstroke}%
\pgfsetstrokeopacity{0.404229}%
\pgfsetdash{}{0pt}%
\pgfpathmoveto{\pgfqpoint{1.491032in}{2.109460in}}%
\pgfpathcurveto{\pgfqpoint{1.499268in}{2.109460in}}{\pgfqpoint{1.507169in}{2.112732in}}{\pgfqpoint{1.512992in}{2.118556in}}%
\pgfpathcurveto{\pgfqpoint{1.518816in}{2.124380in}}{\pgfqpoint{1.522089in}{2.132280in}}{\pgfqpoint{1.522089in}{2.140516in}}%
\pgfpathcurveto{\pgfqpoint{1.522089in}{2.148753in}}{\pgfqpoint{1.518816in}{2.156653in}}{\pgfqpoint{1.512992in}{2.162477in}}%
\pgfpathcurveto{\pgfqpoint{1.507169in}{2.168301in}}{\pgfqpoint{1.499268in}{2.171573in}}{\pgfqpoint{1.491032in}{2.171573in}}%
\pgfpathcurveto{\pgfqpoint{1.482796in}{2.171573in}}{\pgfqpoint{1.474896in}{2.168301in}}{\pgfqpoint{1.469072in}{2.162477in}}%
\pgfpathcurveto{\pgfqpoint{1.463248in}{2.156653in}}{\pgfqpoint{1.459976in}{2.148753in}}{\pgfqpoint{1.459976in}{2.140516in}}%
\pgfpathcurveto{\pgfqpoint{1.459976in}{2.132280in}}{\pgfqpoint{1.463248in}{2.124380in}}{\pgfqpoint{1.469072in}{2.118556in}}%
\pgfpathcurveto{\pgfqpoint{1.474896in}{2.112732in}}{\pgfqpoint{1.482796in}{2.109460in}}{\pgfqpoint{1.491032in}{2.109460in}}%
\pgfpathclose%
\pgfusepath{stroke,fill}%
\end{pgfscope}%
\begin{pgfscope}%
\pgfpathrectangle{\pgfqpoint{0.100000in}{0.212622in}}{\pgfqpoint{3.696000in}{3.696000in}}%
\pgfusepath{clip}%
\pgfsetbuttcap%
\pgfsetroundjoin%
\definecolor{currentfill}{rgb}{0.121569,0.466667,0.705882}%
\pgfsetfillcolor{currentfill}%
\pgfsetfillopacity{0.404315}%
\pgfsetlinewidth{1.003750pt}%
\definecolor{currentstroke}{rgb}{0.121569,0.466667,0.705882}%
\pgfsetstrokecolor{currentstroke}%
\pgfsetstrokeopacity{0.404315}%
\pgfsetdash{}{0pt}%
\pgfpathmoveto{\pgfqpoint{2.365487in}{1.967284in}}%
\pgfpathcurveto{\pgfqpoint{2.373724in}{1.967284in}}{\pgfqpoint{2.381624in}{1.970556in}}{\pgfqpoint{2.387448in}{1.976380in}}%
\pgfpathcurveto{\pgfqpoint{2.393272in}{1.982204in}}{\pgfqpoint{2.396544in}{1.990104in}}{\pgfqpoint{2.396544in}{1.998340in}}%
\pgfpathcurveto{\pgfqpoint{2.396544in}{2.006576in}}{\pgfqpoint{2.393272in}{2.014476in}}{\pgfqpoint{2.387448in}{2.020300in}}%
\pgfpathcurveto{\pgfqpoint{2.381624in}{2.026124in}}{\pgfqpoint{2.373724in}{2.029397in}}{\pgfqpoint{2.365487in}{2.029397in}}%
\pgfpathcurveto{\pgfqpoint{2.357251in}{2.029397in}}{\pgfqpoint{2.349351in}{2.026124in}}{\pgfqpoint{2.343527in}{2.020300in}}%
\pgfpathcurveto{\pgfqpoint{2.337703in}{2.014476in}}{\pgfqpoint{2.334431in}{2.006576in}}{\pgfqpoint{2.334431in}{1.998340in}}%
\pgfpathcurveto{\pgfqpoint{2.334431in}{1.990104in}}{\pgfqpoint{2.337703in}{1.982204in}}{\pgfqpoint{2.343527in}{1.976380in}}%
\pgfpathcurveto{\pgfqpoint{2.349351in}{1.970556in}}{\pgfqpoint{2.357251in}{1.967284in}}{\pgfqpoint{2.365487in}{1.967284in}}%
\pgfpathclose%
\pgfusepath{stroke,fill}%
\end{pgfscope}%
\begin{pgfscope}%
\pgfpathrectangle{\pgfqpoint{0.100000in}{0.212622in}}{\pgfqpoint{3.696000in}{3.696000in}}%
\pgfusepath{clip}%
\pgfsetbuttcap%
\pgfsetroundjoin%
\definecolor{currentfill}{rgb}{0.121569,0.466667,0.705882}%
\pgfsetfillcolor{currentfill}%
\pgfsetfillopacity{0.404794}%
\pgfsetlinewidth{1.003750pt}%
\definecolor{currentstroke}{rgb}{0.121569,0.466667,0.705882}%
\pgfsetstrokecolor{currentstroke}%
\pgfsetstrokeopacity{0.404794}%
\pgfsetdash{}{0pt}%
\pgfpathmoveto{\pgfqpoint{2.375041in}{1.964319in}}%
\pgfpathcurveto{\pgfqpoint{2.383278in}{1.964319in}}{\pgfqpoint{2.391178in}{1.967591in}}{\pgfqpoint{2.397002in}{1.973415in}}%
\pgfpathcurveto{\pgfqpoint{2.402826in}{1.979239in}}{\pgfqpoint{2.406098in}{1.987139in}}{\pgfqpoint{2.406098in}{1.995375in}}%
\pgfpathcurveto{\pgfqpoint{2.406098in}{2.003612in}}{\pgfqpoint{2.402826in}{2.011512in}}{\pgfqpoint{2.397002in}{2.017336in}}%
\pgfpathcurveto{\pgfqpoint{2.391178in}{2.023160in}}{\pgfqpoint{2.383278in}{2.026432in}}{\pgfqpoint{2.375041in}{2.026432in}}%
\pgfpathcurveto{\pgfqpoint{2.366805in}{2.026432in}}{\pgfqpoint{2.358905in}{2.023160in}}{\pgfqpoint{2.353081in}{2.017336in}}%
\pgfpathcurveto{\pgfqpoint{2.347257in}{2.011512in}}{\pgfqpoint{2.343985in}{2.003612in}}{\pgfqpoint{2.343985in}{1.995375in}}%
\pgfpathcurveto{\pgfqpoint{2.343985in}{1.987139in}}{\pgfqpoint{2.347257in}{1.979239in}}{\pgfqpoint{2.353081in}{1.973415in}}%
\pgfpathcurveto{\pgfqpoint{2.358905in}{1.967591in}}{\pgfqpoint{2.366805in}{1.964319in}}{\pgfqpoint{2.375041in}{1.964319in}}%
\pgfpathclose%
\pgfusepath{stroke,fill}%
\end{pgfscope}%
\begin{pgfscope}%
\pgfpathrectangle{\pgfqpoint{0.100000in}{0.212622in}}{\pgfqpoint{3.696000in}{3.696000in}}%
\pgfusepath{clip}%
\pgfsetbuttcap%
\pgfsetroundjoin%
\definecolor{currentfill}{rgb}{0.121569,0.466667,0.705882}%
\pgfsetfillcolor{currentfill}%
\pgfsetfillopacity{0.404965}%
\pgfsetlinewidth{1.003750pt}%
\definecolor{currentstroke}{rgb}{0.121569,0.466667,0.705882}%
\pgfsetstrokecolor{currentstroke}%
\pgfsetstrokeopacity{0.404965}%
\pgfsetdash{}{0pt}%
\pgfpathmoveto{\pgfqpoint{1.489830in}{2.109532in}}%
\pgfpathcurveto{\pgfqpoint{1.498066in}{2.109532in}}{\pgfqpoint{1.505966in}{2.112805in}}{\pgfqpoint{1.511790in}{2.118629in}}%
\pgfpathcurveto{\pgfqpoint{1.517614in}{2.124452in}}{\pgfqpoint{1.520886in}{2.132353in}}{\pgfqpoint{1.520886in}{2.140589in}}%
\pgfpathcurveto{\pgfqpoint{1.520886in}{2.148825in}}{\pgfqpoint{1.517614in}{2.156725in}}{\pgfqpoint{1.511790in}{2.162549in}}%
\pgfpathcurveto{\pgfqpoint{1.505966in}{2.168373in}}{\pgfqpoint{1.498066in}{2.171645in}}{\pgfqpoint{1.489830in}{2.171645in}}%
\pgfpathcurveto{\pgfqpoint{1.481593in}{2.171645in}}{\pgfqpoint{1.473693in}{2.168373in}}{\pgfqpoint{1.467869in}{2.162549in}}%
\pgfpathcurveto{\pgfqpoint{1.462045in}{2.156725in}}{\pgfqpoint{1.458773in}{2.148825in}}{\pgfqpoint{1.458773in}{2.140589in}}%
\pgfpathcurveto{\pgfqpoint{1.458773in}{2.132353in}}{\pgfqpoint{1.462045in}{2.124452in}}{\pgfqpoint{1.467869in}{2.118629in}}%
\pgfpathcurveto{\pgfqpoint{1.473693in}{2.112805in}}{\pgfqpoint{1.481593in}{2.109532in}}{\pgfqpoint{1.489830in}{2.109532in}}%
\pgfpathclose%
\pgfusepath{stroke,fill}%
\end{pgfscope}%
\begin{pgfscope}%
\pgfpathrectangle{\pgfqpoint{0.100000in}{0.212622in}}{\pgfqpoint{3.696000in}{3.696000in}}%
\pgfusepath{clip}%
\pgfsetbuttcap%
\pgfsetroundjoin%
\definecolor{currentfill}{rgb}{0.121569,0.466667,0.705882}%
\pgfsetfillcolor{currentfill}%
\pgfsetfillopacity{0.405210}%
\pgfsetlinewidth{1.003750pt}%
\definecolor{currentstroke}{rgb}{0.121569,0.466667,0.705882}%
\pgfsetstrokecolor{currentstroke}%
\pgfsetstrokeopacity{0.405210}%
\pgfsetdash{}{0pt}%
\pgfpathmoveto{\pgfqpoint{2.380120in}{1.962943in}}%
\pgfpathcurveto{\pgfqpoint{2.388356in}{1.962943in}}{\pgfqpoint{2.396256in}{1.966215in}}{\pgfqpoint{2.402080in}{1.972039in}}%
\pgfpathcurveto{\pgfqpoint{2.407904in}{1.977863in}}{\pgfqpoint{2.411176in}{1.985763in}}{\pgfqpoint{2.411176in}{1.994000in}}%
\pgfpathcurveto{\pgfqpoint{2.411176in}{2.002236in}}{\pgfqpoint{2.407904in}{2.010136in}}{\pgfqpoint{2.402080in}{2.015960in}}%
\pgfpathcurveto{\pgfqpoint{2.396256in}{2.021784in}}{\pgfqpoint{2.388356in}{2.025056in}}{\pgfqpoint{2.380120in}{2.025056in}}%
\pgfpathcurveto{\pgfqpoint{2.371883in}{2.025056in}}{\pgfqpoint{2.363983in}{2.021784in}}{\pgfqpoint{2.358159in}{2.015960in}}%
\pgfpathcurveto{\pgfqpoint{2.352336in}{2.010136in}}{\pgfqpoint{2.349063in}{2.002236in}}{\pgfqpoint{2.349063in}{1.994000in}}%
\pgfpathcurveto{\pgfqpoint{2.349063in}{1.985763in}}{\pgfqpoint{2.352336in}{1.977863in}}{\pgfqpoint{2.358159in}{1.972039in}}%
\pgfpathcurveto{\pgfqpoint{2.363983in}{1.966215in}}{\pgfqpoint{2.371883in}{1.962943in}}{\pgfqpoint{2.380120in}{1.962943in}}%
\pgfpathclose%
\pgfusepath{stroke,fill}%
\end{pgfscope}%
\begin{pgfscope}%
\pgfpathrectangle{\pgfqpoint{0.100000in}{0.212622in}}{\pgfqpoint{3.696000in}{3.696000in}}%
\pgfusepath{clip}%
\pgfsetbuttcap%
\pgfsetroundjoin%
\definecolor{currentfill}{rgb}{0.121569,0.466667,0.705882}%
\pgfsetfillcolor{currentfill}%
\pgfsetfillopacity{0.405436}%
\pgfsetlinewidth{1.003750pt}%
\definecolor{currentstroke}{rgb}{0.121569,0.466667,0.705882}%
\pgfsetstrokecolor{currentstroke}%
\pgfsetstrokeopacity{0.405436}%
\pgfsetdash{}{0pt}%
\pgfpathmoveto{\pgfqpoint{1.488690in}{2.109638in}}%
\pgfpathcurveto{\pgfqpoint{1.496926in}{2.109638in}}{\pgfqpoint{1.504826in}{2.112910in}}{\pgfqpoint{1.510650in}{2.118734in}}%
\pgfpathcurveto{\pgfqpoint{1.516474in}{2.124558in}}{\pgfqpoint{1.519746in}{2.132458in}}{\pgfqpoint{1.519746in}{2.140694in}}%
\pgfpathcurveto{\pgfqpoint{1.519746in}{2.148930in}}{\pgfqpoint{1.516474in}{2.156830in}}{\pgfqpoint{1.510650in}{2.162654in}}%
\pgfpathcurveto{\pgfqpoint{1.504826in}{2.168478in}}{\pgfqpoint{1.496926in}{2.171751in}}{\pgfqpoint{1.488690in}{2.171751in}}%
\pgfpathcurveto{\pgfqpoint{1.480453in}{2.171751in}}{\pgfqpoint{1.472553in}{2.168478in}}{\pgfqpoint{1.466729in}{2.162654in}}%
\pgfpathcurveto{\pgfqpoint{1.460906in}{2.156830in}}{\pgfqpoint{1.457633in}{2.148930in}}{\pgfqpoint{1.457633in}{2.140694in}}%
\pgfpathcurveto{\pgfqpoint{1.457633in}{2.132458in}}{\pgfqpoint{1.460906in}{2.124558in}}{\pgfqpoint{1.466729in}{2.118734in}}%
\pgfpathcurveto{\pgfqpoint{1.472553in}{2.112910in}}{\pgfqpoint{1.480453in}{2.109638in}}{\pgfqpoint{1.488690in}{2.109638in}}%
\pgfpathclose%
\pgfusepath{stroke,fill}%
\end{pgfscope}%
\begin{pgfscope}%
\pgfpathrectangle{\pgfqpoint{0.100000in}{0.212622in}}{\pgfqpoint{3.696000in}{3.696000in}}%
\pgfusepath{clip}%
\pgfsetbuttcap%
\pgfsetroundjoin%
\definecolor{currentfill}{rgb}{0.121569,0.466667,0.705882}%
\pgfsetfillcolor{currentfill}%
\pgfsetfillopacity{0.405551}%
\pgfsetlinewidth{1.003750pt}%
\definecolor{currentstroke}{rgb}{0.121569,0.466667,0.705882}%
\pgfsetstrokecolor{currentstroke}%
\pgfsetstrokeopacity{0.405551}%
\pgfsetdash{}{0pt}%
\pgfpathmoveto{\pgfqpoint{2.382747in}{1.962390in}}%
\pgfpathcurveto{\pgfqpoint{2.390983in}{1.962390in}}{\pgfqpoint{2.398883in}{1.965662in}}{\pgfqpoint{2.404707in}{1.971486in}}%
\pgfpathcurveto{\pgfqpoint{2.410531in}{1.977310in}}{\pgfqpoint{2.413803in}{1.985210in}}{\pgfqpoint{2.413803in}{1.993446in}}%
\pgfpathcurveto{\pgfqpoint{2.413803in}{2.001682in}}{\pgfqpoint{2.410531in}{2.009582in}}{\pgfqpoint{2.404707in}{2.015406in}}%
\pgfpathcurveto{\pgfqpoint{2.398883in}{2.021230in}}{\pgfqpoint{2.390983in}{2.024503in}}{\pgfqpoint{2.382747in}{2.024503in}}%
\pgfpathcurveto{\pgfqpoint{2.374510in}{2.024503in}}{\pgfqpoint{2.366610in}{2.021230in}}{\pgfqpoint{2.360787in}{2.015406in}}%
\pgfpathcurveto{\pgfqpoint{2.354963in}{2.009582in}}{\pgfqpoint{2.351690in}{2.001682in}}{\pgfqpoint{2.351690in}{1.993446in}}%
\pgfpathcurveto{\pgfqpoint{2.351690in}{1.985210in}}{\pgfqpoint{2.354963in}{1.977310in}}{\pgfqpoint{2.360787in}{1.971486in}}%
\pgfpathcurveto{\pgfqpoint{2.366610in}{1.965662in}}{\pgfqpoint{2.374510in}{1.962390in}}{\pgfqpoint{2.382747in}{1.962390in}}%
\pgfpathclose%
\pgfusepath{stroke,fill}%
\end{pgfscope}%
\begin{pgfscope}%
\pgfpathrectangle{\pgfqpoint{0.100000in}{0.212622in}}{\pgfqpoint{3.696000in}{3.696000in}}%
\pgfusepath{clip}%
\pgfsetbuttcap%
\pgfsetroundjoin%
\definecolor{currentfill}{rgb}{0.121569,0.466667,0.705882}%
\pgfsetfillcolor{currentfill}%
\pgfsetfillopacity{0.405930}%
\pgfsetlinewidth{1.003750pt}%
\definecolor{currentstroke}{rgb}{0.121569,0.466667,0.705882}%
\pgfsetstrokecolor{currentstroke}%
\pgfsetstrokeopacity{0.405930}%
\pgfsetdash{}{0pt}%
\pgfpathmoveto{\pgfqpoint{2.386486in}{1.961453in}}%
\pgfpathcurveto{\pgfqpoint{2.394723in}{1.961453in}}{\pgfqpoint{2.402623in}{1.964725in}}{\pgfqpoint{2.408447in}{1.970549in}}%
\pgfpathcurveto{\pgfqpoint{2.414271in}{1.976373in}}{\pgfqpoint{2.417543in}{1.984273in}}{\pgfqpoint{2.417543in}{1.992509in}}%
\pgfpathcurveto{\pgfqpoint{2.417543in}{2.000745in}}{\pgfqpoint{2.414271in}{2.008645in}}{\pgfqpoint{2.408447in}{2.014469in}}%
\pgfpathcurveto{\pgfqpoint{2.402623in}{2.020293in}}{\pgfqpoint{2.394723in}{2.023566in}}{\pgfqpoint{2.386486in}{2.023566in}}%
\pgfpathcurveto{\pgfqpoint{2.378250in}{2.023566in}}{\pgfqpoint{2.370350in}{2.020293in}}{\pgfqpoint{2.364526in}{2.014469in}}%
\pgfpathcurveto{\pgfqpoint{2.358702in}{2.008645in}}{\pgfqpoint{2.355430in}{2.000745in}}{\pgfqpoint{2.355430in}{1.992509in}}%
\pgfpathcurveto{\pgfqpoint{2.355430in}{1.984273in}}{\pgfqpoint{2.358702in}{1.976373in}}{\pgfqpoint{2.364526in}{1.970549in}}%
\pgfpathcurveto{\pgfqpoint{2.370350in}{1.964725in}}{\pgfqpoint{2.378250in}{1.961453in}}{\pgfqpoint{2.386486in}{1.961453in}}%
\pgfpathclose%
\pgfusepath{stroke,fill}%
\end{pgfscope}%
\begin{pgfscope}%
\pgfpathrectangle{\pgfqpoint{0.100000in}{0.212622in}}{\pgfqpoint{3.696000in}{3.696000in}}%
\pgfusepath{clip}%
\pgfsetbuttcap%
\pgfsetroundjoin%
\definecolor{currentfill}{rgb}{0.121569,0.466667,0.705882}%
\pgfsetfillcolor{currentfill}%
\pgfsetfillopacity{0.406331}%
\pgfsetlinewidth{1.003750pt}%
\definecolor{currentstroke}{rgb}{0.121569,0.466667,0.705882}%
\pgfsetstrokecolor{currentstroke}%
\pgfsetstrokeopacity{0.406331}%
\pgfsetdash{}{0pt}%
\pgfpathmoveto{\pgfqpoint{1.487035in}{2.109683in}}%
\pgfpathcurveto{\pgfqpoint{1.495271in}{2.109683in}}{\pgfqpoint{1.503171in}{2.112955in}}{\pgfqpoint{1.508995in}{2.118779in}}%
\pgfpathcurveto{\pgfqpoint{1.514819in}{2.124603in}}{\pgfqpoint{1.518091in}{2.132503in}}{\pgfqpoint{1.518091in}{2.140739in}}%
\pgfpathcurveto{\pgfqpoint{1.518091in}{2.148976in}}{\pgfqpoint{1.514819in}{2.156876in}}{\pgfqpoint{1.508995in}{2.162700in}}%
\pgfpathcurveto{\pgfqpoint{1.503171in}{2.168524in}}{\pgfqpoint{1.495271in}{2.171796in}}{\pgfqpoint{1.487035in}{2.171796in}}%
\pgfpathcurveto{\pgfqpoint{1.478799in}{2.171796in}}{\pgfqpoint{1.470899in}{2.168524in}}{\pgfqpoint{1.465075in}{2.162700in}}%
\pgfpathcurveto{\pgfqpoint{1.459251in}{2.156876in}}{\pgfqpoint{1.455978in}{2.148976in}}{\pgfqpoint{1.455978in}{2.140739in}}%
\pgfpathcurveto{\pgfqpoint{1.455978in}{2.132503in}}{\pgfqpoint{1.459251in}{2.124603in}}{\pgfqpoint{1.465075in}{2.118779in}}%
\pgfpathcurveto{\pgfqpoint{1.470899in}{2.112955in}}{\pgfqpoint{1.478799in}{2.109683in}}{\pgfqpoint{1.487035in}{2.109683in}}%
\pgfpathclose%
\pgfusepath{stroke,fill}%
\end{pgfscope}%
\begin{pgfscope}%
\pgfpathrectangle{\pgfqpoint{0.100000in}{0.212622in}}{\pgfqpoint{3.696000in}{3.696000in}}%
\pgfusepath{clip}%
\pgfsetbuttcap%
\pgfsetroundjoin%
\definecolor{currentfill}{rgb}{0.121569,0.466667,0.705882}%
\pgfsetfillcolor{currentfill}%
\pgfsetfillopacity{0.406445}%
\pgfsetlinewidth{1.003750pt}%
\definecolor{currentstroke}{rgb}{0.121569,0.466667,0.705882}%
\pgfsetstrokecolor{currentstroke}%
\pgfsetstrokeopacity{0.406445}%
\pgfsetdash{}{0pt}%
\pgfpathmoveto{\pgfqpoint{2.391040in}{1.959795in}}%
\pgfpathcurveto{\pgfqpoint{2.399276in}{1.959795in}}{\pgfqpoint{2.407176in}{1.963067in}}{\pgfqpoint{2.413000in}{1.968891in}}%
\pgfpathcurveto{\pgfqpoint{2.418824in}{1.974715in}}{\pgfqpoint{2.422096in}{1.982615in}}{\pgfqpoint{2.422096in}{1.990851in}}%
\pgfpathcurveto{\pgfqpoint{2.422096in}{1.999087in}}{\pgfqpoint{2.418824in}{2.006987in}}{\pgfqpoint{2.413000in}{2.012811in}}%
\pgfpathcurveto{\pgfqpoint{2.407176in}{2.018635in}}{\pgfqpoint{2.399276in}{2.021908in}}{\pgfqpoint{2.391040in}{2.021908in}}%
\pgfpathcurveto{\pgfqpoint{2.382803in}{2.021908in}}{\pgfqpoint{2.374903in}{2.018635in}}{\pgfqpoint{2.369079in}{2.012811in}}%
\pgfpathcurveto{\pgfqpoint{2.363256in}{2.006987in}}{\pgfqpoint{2.359983in}{1.999087in}}{\pgfqpoint{2.359983in}{1.990851in}}%
\pgfpathcurveto{\pgfqpoint{2.359983in}{1.982615in}}{\pgfqpoint{2.363256in}{1.974715in}}{\pgfqpoint{2.369079in}{1.968891in}}%
\pgfpathcurveto{\pgfqpoint{2.374903in}{1.963067in}}{\pgfqpoint{2.382803in}{1.959795in}}{\pgfqpoint{2.391040in}{1.959795in}}%
\pgfpathclose%
\pgfusepath{stroke,fill}%
\end{pgfscope}%
\begin{pgfscope}%
\pgfpathrectangle{\pgfqpoint{0.100000in}{0.212622in}}{\pgfqpoint{3.696000in}{3.696000in}}%
\pgfusepath{clip}%
\pgfsetbuttcap%
\pgfsetroundjoin%
\definecolor{currentfill}{rgb}{0.121569,0.466667,0.705882}%
\pgfsetfillcolor{currentfill}%
\pgfsetfillopacity{0.407167}%
\pgfsetlinewidth{1.003750pt}%
\definecolor{currentstroke}{rgb}{0.121569,0.466667,0.705882}%
\pgfsetstrokecolor{currentstroke}%
\pgfsetstrokeopacity{0.407167}%
\pgfsetdash{}{0pt}%
\pgfpathmoveto{\pgfqpoint{1.485722in}{2.109782in}}%
\pgfpathcurveto{\pgfqpoint{1.493958in}{2.109782in}}{\pgfqpoint{1.501858in}{2.113054in}}{\pgfqpoint{1.507682in}{2.118878in}}%
\pgfpathcurveto{\pgfqpoint{1.513506in}{2.124702in}}{\pgfqpoint{1.516778in}{2.132602in}}{\pgfqpoint{1.516778in}{2.140838in}}%
\pgfpathcurveto{\pgfqpoint{1.516778in}{2.149075in}}{\pgfqpoint{1.513506in}{2.156975in}}{\pgfqpoint{1.507682in}{2.162799in}}%
\pgfpathcurveto{\pgfqpoint{1.501858in}{2.168623in}}{\pgfqpoint{1.493958in}{2.171895in}}{\pgfqpoint{1.485722in}{2.171895in}}%
\pgfpathcurveto{\pgfqpoint{1.477486in}{2.171895in}}{\pgfqpoint{1.469585in}{2.168623in}}{\pgfqpoint{1.463762in}{2.162799in}}%
\pgfpathcurveto{\pgfqpoint{1.457938in}{2.156975in}}{\pgfqpoint{1.454665in}{2.149075in}}{\pgfqpoint{1.454665in}{2.140838in}}%
\pgfpathcurveto{\pgfqpoint{1.454665in}{2.132602in}}{\pgfqpoint{1.457938in}{2.124702in}}{\pgfqpoint{1.463762in}{2.118878in}}%
\pgfpathcurveto{\pgfqpoint{1.469585in}{2.113054in}}{\pgfqpoint{1.477486in}{2.109782in}}{\pgfqpoint{1.485722in}{2.109782in}}%
\pgfpathclose%
\pgfusepath{stroke,fill}%
\end{pgfscope}%
\begin{pgfscope}%
\pgfpathrectangle{\pgfqpoint{0.100000in}{0.212622in}}{\pgfqpoint{3.696000in}{3.696000in}}%
\pgfusepath{clip}%
\pgfsetbuttcap%
\pgfsetroundjoin%
\definecolor{currentfill}{rgb}{0.121569,0.466667,0.705882}%
\pgfsetfillcolor{currentfill}%
\pgfsetfillopacity{0.407207}%
\pgfsetlinewidth{1.003750pt}%
\definecolor{currentstroke}{rgb}{0.121569,0.466667,0.705882}%
\pgfsetstrokecolor{currentstroke}%
\pgfsetstrokeopacity{0.407207}%
\pgfsetdash{}{0pt}%
\pgfpathmoveto{\pgfqpoint{2.396619in}{1.958467in}}%
\pgfpathcurveto{\pgfqpoint{2.404856in}{1.958467in}}{\pgfqpoint{2.412756in}{1.961739in}}{\pgfqpoint{2.418580in}{1.967563in}}%
\pgfpathcurveto{\pgfqpoint{2.424404in}{1.973387in}}{\pgfqpoint{2.427676in}{1.981287in}}{\pgfqpoint{2.427676in}{1.989523in}}%
\pgfpathcurveto{\pgfqpoint{2.427676in}{1.997760in}}{\pgfqpoint{2.424404in}{2.005660in}}{\pgfqpoint{2.418580in}{2.011484in}}%
\pgfpathcurveto{\pgfqpoint{2.412756in}{2.017308in}}{\pgfqpoint{2.404856in}{2.020580in}}{\pgfqpoint{2.396619in}{2.020580in}}%
\pgfpathcurveto{\pgfqpoint{2.388383in}{2.020580in}}{\pgfqpoint{2.380483in}{2.017308in}}{\pgfqpoint{2.374659in}{2.011484in}}%
\pgfpathcurveto{\pgfqpoint{2.368835in}{2.005660in}}{\pgfqpoint{2.365563in}{1.997760in}}{\pgfqpoint{2.365563in}{1.989523in}}%
\pgfpathcurveto{\pgfqpoint{2.365563in}{1.981287in}}{\pgfqpoint{2.368835in}{1.973387in}}{\pgfqpoint{2.374659in}{1.967563in}}%
\pgfpathcurveto{\pgfqpoint{2.380483in}{1.961739in}}{\pgfqpoint{2.388383in}{1.958467in}}{\pgfqpoint{2.396619in}{1.958467in}}%
\pgfpathclose%
\pgfusepath{stroke,fill}%
\end{pgfscope}%
\begin{pgfscope}%
\pgfpathrectangle{\pgfqpoint{0.100000in}{0.212622in}}{\pgfqpoint{3.696000in}{3.696000in}}%
\pgfusepath{clip}%
\pgfsetbuttcap%
\pgfsetroundjoin%
\definecolor{currentfill}{rgb}{0.121569,0.466667,0.705882}%
\pgfsetfillcolor{currentfill}%
\pgfsetfillopacity{0.407503}%
\pgfsetlinewidth{1.003750pt}%
\definecolor{currentstroke}{rgb}{0.121569,0.466667,0.705882}%
\pgfsetstrokecolor{currentstroke}%
\pgfsetstrokeopacity{0.407503}%
\pgfsetdash{}{0pt}%
\pgfpathmoveto{\pgfqpoint{1.484843in}{2.109880in}}%
\pgfpathcurveto{\pgfqpoint{1.493079in}{2.109880in}}{\pgfqpoint{1.500979in}{2.113153in}}{\pgfqpoint{1.506803in}{2.118977in}}%
\pgfpathcurveto{\pgfqpoint{1.512627in}{2.124801in}}{\pgfqpoint{1.515899in}{2.132701in}}{\pgfqpoint{1.515899in}{2.140937in}}%
\pgfpathcurveto{\pgfqpoint{1.515899in}{2.149173in}}{\pgfqpoint{1.512627in}{2.157073in}}{\pgfqpoint{1.506803in}{2.162897in}}%
\pgfpathcurveto{\pgfqpoint{1.500979in}{2.168721in}}{\pgfqpoint{1.493079in}{2.171993in}}{\pgfqpoint{1.484843in}{2.171993in}}%
\pgfpathcurveto{\pgfqpoint{1.476607in}{2.171993in}}{\pgfqpoint{1.468706in}{2.168721in}}{\pgfqpoint{1.462883in}{2.162897in}}%
\pgfpathcurveto{\pgfqpoint{1.457059in}{2.157073in}}{\pgfqpoint{1.453786in}{2.149173in}}{\pgfqpoint{1.453786in}{2.140937in}}%
\pgfpathcurveto{\pgfqpoint{1.453786in}{2.132701in}}{\pgfqpoint{1.457059in}{2.124801in}}{\pgfqpoint{1.462883in}{2.118977in}}%
\pgfpathcurveto{\pgfqpoint{1.468706in}{2.113153in}}{\pgfqpoint{1.476607in}{2.109880in}}{\pgfqpoint{1.484843in}{2.109880in}}%
\pgfpathclose%
\pgfusepath{stroke,fill}%
\end{pgfscope}%
\begin{pgfscope}%
\pgfpathrectangle{\pgfqpoint{0.100000in}{0.212622in}}{\pgfqpoint{3.696000in}{3.696000in}}%
\pgfusepath{clip}%
\pgfsetbuttcap%
\pgfsetroundjoin%
\definecolor{currentfill}{rgb}{0.121569,0.466667,0.705882}%
\pgfsetfillcolor{currentfill}%
\pgfsetfillopacity{0.407525}%
\pgfsetlinewidth{1.003750pt}%
\definecolor{currentstroke}{rgb}{0.121569,0.466667,0.705882}%
\pgfsetstrokecolor{currentstroke}%
\pgfsetstrokeopacity{0.407525}%
\pgfsetdash{}{0pt}%
\pgfpathmoveto{\pgfqpoint{2.399854in}{1.957547in}}%
\pgfpathcurveto{\pgfqpoint{2.408090in}{1.957547in}}{\pgfqpoint{2.415990in}{1.960819in}}{\pgfqpoint{2.421814in}{1.966643in}}%
\pgfpathcurveto{\pgfqpoint{2.427638in}{1.972467in}}{\pgfqpoint{2.430911in}{1.980367in}}{\pgfqpoint{2.430911in}{1.988603in}}%
\pgfpathcurveto{\pgfqpoint{2.430911in}{1.996839in}}{\pgfqpoint{2.427638in}{2.004740in}}{\pgfqpoint{2.421814in}{2.010563in}}%
\pgfpathcurveto{\pgfqpoint{2.415990in}{2.016387in}}{\pgfqpoint{2.408090in}{2.019660in}}{\pgfqpoint{2.399854in}{2.019660in}}%
\pgfpathcurveto{\pgfqpoint{2.391618in}{2.019660in}}{\pgfqpoint{2.383718in}{2.016387in}}{\pgfqpoint{2.377894in}{2.010563in}}%
\pgfpathcurveto{\pgfqpoint{2.372070in}{2.004740in}}{\pgfqpoint{2.368798in}{1.996839in}}{\pgfqpoint{2.368798in}{1.988603in}}%
\pgfpathcurveto{\pgfqpoint{2.368798in}{1.980367in}}{\pgfqpoint{2.372070in}{1.972467in}}{\pgfqpoint{2.377894in}{1.966643in}}%
\pgfpathcurveto{\pgfqpoint{2.383718in}{1.960819in}}{\pgfqpoint{2.391618in}{1.957547in}}{\pgfqpoint{2.399854in}{1.957547in}}%
\pgfpathclose%
\pgfusepath{stroke,fill}%
\end{pgfscope}%
\begin{pgfscope}%
\pgfpathrectangle{\pgfqpoint{0.100000in}{0.212622in}}{\pgfqpoint{3.696000in}{3.696000in}}%
\pgfusepath{clip}%
\pgfsetbuttcap%
\pgfsetroundjoin%
\definecolor{currentfill}{rgb}{0.121569,0.466667,0.705882}%
\pgfsetfillcolor{currentfill}%
\pgfsetfillopacity{0.407879}%
\pgfsetlinewidth{1.003750pt}%
\definecolor{currentstroke}{rgb}{0.121569,0.466667,0.705882}%
\pgfsetstrokecolor{currentstroke}%
\pgfsetstrokeopacity{0.407879}%
\pgfsetdash{}{0pt}%
\pgfpathmoveto{\pgfqpoint{2.403667in}{1.956521in}}%
\pgfpathcurveto{\pgfqpoint{2.411903in}{1.956521in}}{\pgfqpoint{2.419803in}{1.959794in}}{\pgfqpoint{2.425627in}{1.965618in}}%
\pgfpathcurveto{\pgfqpoint{2.431451in}{1.971441in}}{\pgfqpoint{2.434723in}{1.979342in}}{\pgfqpoint{2.434723in}{1.987578in}}%
\pgfpathcurveto{\pgfqpoint{2.434723in}{1.995814in}}{\pgfqpoint{2.431451in}{2.003714in}}{\pgfqpoint{2.425627in}{2.009538in}}%
\pgfpathcurveto{\pgfqpoint{2.419803in}{2.015362in}}{\pgfqpoint{2.411903in}{2.018634in}}{\pgfqpoint{2.403667in}{2.018634in}}%
\pgfpathcurveto{\pgfqpoint{2.395430in}{2.018634in}}{\pgfqpoint{2.387530in}{2.015362in}}{\pgfqpoint{2.381706in}{2.009538in}}%
\pgfpathcurveto{\pgfqpoint{2.375882in}{2.003714in}}{\pgfqpoint{2.372610in}{1.995814in}}{\pgfqpoint{2.372610in}{1.987578in}}%
\pgfpathcurveto{\pgfqpoint{2.372610in}{1.979342in}}{\pgfqpoint{2.375882in}{1.971441in}}{\pgfqpoint{2.381706in}{1.965618in}}%
\pgfpathcurveto{\pgfqpoint{2.387530in}{1.959794in}}{\pgfqpoint{2.395430in}{1.956521in}}{\pgfqpoint{2.403667in}{1.956521in}}%
\pgfpathclose%
\pgfusepath{stroke,fill}%
\end{pgfscope}%
\begin{pgfscope}%
\pgfpathrectangle{\pgfqpoint{0.100000in}{0.212622in}}{\pgfqpoint{3.696000in}{3.696000in}}%
\pgfusepath{clip}%
\pgfsetbuttcap%
\pgfsetroundjoin%
\definecolor{currentfill}{rgb}{0.121569,0.466667,0.705882}%
\pgfsetfillcolor{currentfill}%
\pgfsetfillopacity{0.408165}%
\pgfsetlinewidth{1.003750pt}%
\definecolor{currentstroke}{rgb}{0.121569,0.466667,0.705882}%
\pgfsetstrokecolor{currentstroke}%
\pgfsetstrokeopacity{0.408165}%
\pgfsetdash{}{0pt}%
\pgfpathmoveto{\pgfqpoint{2.405630in}{1.956175in}}%
\pgfpathcurveto{\pgfqpoint{2.413866in}{1.956175in}}{\pgfqpoint{2.421767in}{1.959447in}}{\pgfqpoint{2.427590in}{1.965271in}}%
\pgfpathcurveto{\pgfqpoint{2.433414in}{1.971095in}}{\pgfqpoint{2.436687in}{1.978995in}}{\pgfqpoint{2.436687in}{1.987232in}}%
\pgfpathcurveto{\pgfqpoint{2.436687in}{1.995468in}}{\pgfqpoint{2.433414in}{2.003368in}}{\pgfqpoint{2.427590in}{2.009192in}}%
\pgfpathcurveto{\pgfqpoint{2.421767in}{2.015016in}}{\pgfqpoint{2.413866in}{2.018288in}}{\pgfqpoint{2.405630in}{2.018288in}}%
\pgfpathcurveto{\pgfqpoint{2.397394in}{2.018288in}}{\pgfqpoint{2.389494in}{2.015016in}}{\pgfqpoint{2.383670in}{2.009192in}}%
\pgfpathcurveto{\pgfqpoint{2.377846in}{2.003368in}}{\pgfqpoint{2.374574in}{1.995468in}}{\pgfqpoint{2.374574in}{1.987232in}}%
\pgfpathcurveto{\pgfqpoint{2.374574in}{1.978995in}}{\pgfqpoint{2.377846in}{1.971095in}}{\pgfqpoint{2.383670in}{1.965271in}}%
\pgfpathcurveto{\pgfqpoint{2.389494in}{1.959447in}}{\pgfqpoint{2.397394in}{1.956175in}}{\pgfqpoint{2.405630in}{1.956175in}}%
\pgfpathclose%
\pgfusepath{stroke,fill}%
\end{pgfscope}%
\begin{pgfscope}%
\pgfpathrectangle{\pgfqpoint{0.100000in}{0.212622in}}{\pgfqpoint{3.696000in}{3.696000in}}%
\pgfusepath{clip}%
\pgfsetbuttcap%
\pgfsetroundjoin%
\definecolor{currentfill}{rgb}{0.121569,0.466667,0.705882}%
\pgfsetfillcolor{currentfill}%
\pgfsetfillopacity{0.408231}%
\pgfsetlinewidth{1.003750pt}%
\definecolor{currentstroke}{rgb}{0.121569,0.466667,0.705882}%
\pgfsetstrokecolor{currentstroke}%
\pgfsetstrokeopacity{0.408231}%
\pgfsetdash{}{0pt}%
\pgfpathmoveto{\pgfqpoint{1.484030in}{2.110260in}}%
\pgfpathcurveto{\pgfqpoint{1.492266in}{2.110260in}}{\pgfqpoint{1.500166in}{2.113533in}}{\pgfqpoint{1.505990in}{2.119357in}}%
\pgfpathcurveto{\pgfqpoint{1.511814in}{2.125181in}}{\pgfqpoint{1.515086in}{2.133081in}}{\pgfqpoint{1.515086in}{2.141317in}}%
\pgfpathcurveto{\pgfqpoint{1.515086in}{2.149553in}}{\pgfqpoint{1.511814in}{2.157453in}}{\pgfqpoint{1.505990in}{2.163277in}}%
\pgfpathcurveto{\pgfqpoint{1.500166in}{2.169101in}}{\pgfqpoint{1.492266in}{2.172373in}}{\pgfqpoint{1.484030in}{2.172373in}}%
\pgfpathcurveto{\pgfqpoint{1.475793in}{2.172373in}}{\pgfqpoint{1.467893in}{2.169101in}}{\pgfqpoint{1.462069in}{2.163277in}}%
\pgfpathcurveto{\pgfqpoint{1.456245in}{2.157453in}}{\pgfqpoint{1.452973in}{2.149553in}}{\pgfqpoint{1.452973in}{2.141317in}}%
\pgfpathcurveto{\pgfqpoint{1.452973in}{2.133081in}}{\pgfqpoint{1.456245in}{2.125181in}}{\pgfqpoint{1.462069in}{2.119357in}}%
\pgfpathcurveto{\pgfqpoint{1.467893in}{2.113533in}}{\pgfqpoint{1.475793in}{2.110260in}}{\pgfqpoint{1.484030in}{2.110260in}}%
\pgfpathclose%
\pgfusepath{stroke,fill}%
\end{pgfscope}%
\begin{pgfscope}%
\pgfpathrectangle{\pgfqpoint{0.100000in}{0.212622in}}{\pgfqpoint{3.696000in}{3.696000in}}%
\pgfusepath{clip}%
\pgfsetbuttcap%
\pgfsetroundjoin%
\definecolor{currentfill}{rgb}{0.121569,0.466667,0.705882}%
\pgfsetfillcolor{currentfill}%
\pgfsetfillopacity{0.408255}%
\pgfsetlinewidth{1.003750pt}%
\definecolor{currentstroke}{rgb}{0.121569,0.466667,0.705882}%
\pgfsetstrokecolor{currentstroke}%
\pgfsetstrokeopacity{0.408255}%
\pgfsetdash{}{0pt}%
\pgfpathmoveto{\pgfqpoint{2.409132in}{1.954942in}}%
\pgfpathcurveto{\pgfqpoint{2.417369in}{1.954942in}}{\pgfqpoint{2.425269in}{1.958214in}}{\pgfqpoint{2.431093in}{1.964038in}}%
\pgfpathcurveto{\pgfqpoint{2.436917in}{1.969862in}}{\pgfqpoint{2.440189in}{1.977762in}}{\pgfqpoint{2.440189in}{1.985998in}}%
\pgfpathcurveto{\pgfqpoint{2.440189in}{1.994234in}}{\pgfqpoint{2.436917in}{2.002134in}}{\pgfqpoint{2.431093in}{2.007958in}}%
\pgfpathcurveto{\pgfqpoint{2.425269in}{2.013782in}}{\pgfqpoint{2.417369in}{2.017055in}}{\pgfqpoint{2.409132in}{2.017055in}}%
\pgfpathcurveto{\pgfqpoint{2.400896in}{2.017055in}}{\pgfqpoint{2.392996in}{2.013782in}}{\pgfqpoint{2.387172in}{2.007958in}}%
\pgfpathcurveto{\pgfqpoint{2.381348in}{2.002134in}}{\pgfqpoint{2.378076in}{1.994234in}}{\pgfqpoint{2.378076in}{1.985998in}}%
\pgfpathcurveto{\pgfqpoint{2.378076in}{1.977762in}}{\pgfqpoint{2.381348in}{1.969862in}}{\pgfqpoint{2.387172in}{1.964038in}}%
\pgfpathcurveto{\pgfqpoint{2.392996in}{1.958214in}}{\pgfqpoint{2.400896in}{1.954942in}}{\pgfqpoint{2.409132in}{1.954942in}}%
\pgfpathclose%
\pgfusepath{stroke,fill}%
\end{pgfscope}%
\begin{pgfscope}%
\pgfpathrectangle{\pgfqpoint{0.100000in}{0.212622in}}{\pgfqpoint{3.696000in}{3.696000in}}%
\pgfusepath{clip}%
\pgfsetbuttcap%
\pgfsetroundjoin%
\definecolor{currentfill}{rgb}{0.121569,0.466667,0.705882}%
\pgfsetfillcolor{currentfill}%
\pgfsetfillopacity{0.408379}%
\pgfsetlinewidth{1.003750pt}%
\definecolor{currentstroke}{rgb}{0.121569,0.466667,0.705882}%
\pgfsetstrokecolor{currentstroke}%
\pgfsetstrokeopacity{0.408379}%
\pgfsetdash{}{0pt}%
\pgfpathmoveto{\pgfqpoint{2.410999in}{1.954446in}}%
\pgfpathcurveto{\pgfqpoint{2.419235in}{1.954446in}}{\pgfqpoint{2.427135in}{1.957719in}}{\pgfqpoint{2.432959in}{1.963543in}}%
\pgfpathcurveto{\pgfqpoint{2.438783in}{1.969367in}}{\pgfqpoint{2.442055in}{1.977267in}}{\pgfqpoint{2.442055in}{1.985503in}}%
\pgfpathcurveto{\pgfqpoint{2.442055in}{1.993739in}}{\pgfqpoint{2.438783in}{2.001639in}}{\pgfqpoint{2.432959in}{2.007463in}}%
\pgfpathcurveto{\pgfqpoint{2.427135in}{2.013287in}}{\pgfqpoint{2.419235in}{2.016559in}}{\pgfqpoint{2.410999in}{2.016559in}}%
\pgfpathcurveto{\pgfqpoint{2.402762in}{2.016559in}}{\pgfqpoint{2.394862in}{2.013287in}}{\pgfqpoint{2.389038in}{2.007463in}}%
\pgfpathcurveto{\pgfqpoint{2.383214in}{2.001639in}}{\pgfqpoint{2.379942in}{1.993739in}}{\pgfqpoint{2.379942in}{1.985503in}}%
\pgfpathcurveto{\pgfqpoint{2.379942in}{1.977267in}}{\pgfqpoint{2.383214in}{1.969367in}}{\pgfqpoint{2.389038in}{1.963543in}}%
\pgfpathcurveto{\pgfqpoint{2.394862in}{1.957719in}}{\pgfqpoint{2.402762in}{1.954446in}}{\pgfqpoint{2.410999in}{1.954446in}}%
\pgfpathclose%
\pgfusepath{stroke,fill}%
\end{pgfscope}%
\begin{pgfscope}%
\pgfpathrectangle{\pgfqpoint{0.100000in}{0.212622in}}{\pgfqpoint{3.696000in}{3.696000in}}%
\pgfusepath{clip}%
\pgfsetbuttcap%
\pgfsetroundjoin%
\definecolor{currentfill}{rgb}{0.121569,0.466667,0.705882}%
\pgfsetfillcolor{currentfill}%
\pgfsetfillopacity{0.408552}%
\pgfsetlinewidth{1.003750pt}%
\definecolor{currentstroke}{rgb}{0.121569,0.466667,0.705882}%
\pgfsetstrokecolor{currentstroke}%
\pgfsetstrokeopacity{0.408552}%
\pgfsetdash{}{0pt}%
\pgfpathmoveto{\pgfqpoint{2.411840in}{1.954343in}}%
\pgfpathcurveto{\pgfqpoint{2.420076in}{1.954343in}}{\pgfqpoint{2.427976in}{1.957615in}}{\pgfqpoint{2.433800in}{1.963439in}}%
\pgfpathcurveto{\pgfqpoint{2.439624in}{1.969263in}}{\pgfqpoint{2.442897in}{1.977163in}}{\pgfqpoint{2.442897in}{1.985399in}}%
\pgfpathcurveto{\pgfqpoint{2.442897in}{1.993636in}}{\pgfqpoint{2.439624in}{2.001536in}}{\pgfqpoint{2.433800in}{2.007360in}}%
\pgfpathcurveto{\pgfqpoint{2.427976in}{2.013184in}}{\pgfqpoint{2.420076in}{2.016456in}}{\pgfqpoint{2.411840in}{2.016456in}}%
\pgfpathcurveto{\pgfqpoint{2.403604in}{2.016456in}}{\pgfqpoint{2.395704in}{2.013184in}}{\pgfqpoint{2.389880in}{2.007360in}}%
\pgfpathcurveto{\pgfqpoint{2.384056in}{2.001536in}}{\pgfqpoint{2.380784in}{1.993636in}}{\pgfqpoint{2.380784in}{1.985399in}}%
\pgfpathcurveto{\pgfqpoint{2.380784in}{1.977163in}}{\pgfqpoint{2.384056in}{1.969263in}}{\pgfqpoint{2.389880in}{1.963439in}}%
\pgfpathcurveto{\pgfqpoint{2.395704in}{1.957615in}}{\pgfqpoint{2.403604in}{1.954343in}}{\pgfqpoint{2.411840in}{1.954343in}}%
\pgfpathclose%
\pgfusepath{stroke,fill}%
\end{pgfscope}%
\begin{pgfscope}%
\pgfpathrectangle{\pgfqpoint{0.100000in}{0.212622in}}{\pgfqpoint{3.696000in}{3.696000in}}%
\pgfusepath{clip}%
\pgfsetbuttcap%
\pgfsetroundjoin%
\definecolor{currentfill}{rgb}{0.121569,0.466667,0.705882}%
\pgfsetfillcolor{currentfill}%
\pgfsetfillopacity{0.408812}%
\pgfsetlinewidth{1.003750pt}%
\definecolor{currentstroke}{rgb}{0.121569,0.466667,0.705882}%
\pgfsetstrokecolor{currentstroke}%
\pgfsetstrokeopacity{0.408812}%
\pgfsetdash{}{0pt}%
\pgfpathmoveto{\pgfqpoint{2.413939in}{1.953836in}}%
\pgfpathcurveto{\pgfqpoint{2.422176in}{1.953836in}}{\pgfqpoint{2.430076in}{1.957108in}}{\pgfqpoint{2.435900in}{1.962932in}}%
\pgfpathcurveto{\pgfqpoint{2.441724in}{1.968756in}}{\pgfqpoint{2.444996in}{1.976656in}}{\pgfqpoint{2.444996in}{1.984892in}}%
\pgfpathcurveto{\pgfqpoint{2.444996in}{1.993129in}}{\pgfqpoint{2.441724in}{2.001029in}}{\pgfqpoint{2.435900in}{2.006853in}}%
\pgfpathcurveto{\pgfqpoint{2.430076in}{2.012677in}}{\pgfqpoint{2.422176in}{2.015949in}}{\pgfqpoint{2.413939in}{2.015949in}}%
\pgfpathcurveto{\pgfqpoint{2.405703in}{2.015949in}}{\pgfqpoint{2.397803in}{2.012677in}}{\pgfqpoint{2.391979in}{2.006853in}}%
\pgfpathcurveto{\pgfqpoint{2.386155in}{2.001029in}}{\pgfqpoint{2.382883in}{1.993129in}}{\pgfqpoint{2.382883in}{1.984892in}}%
\pgfpathcurveto{\pgfqpoint{2.382883in}{1.976656in}}{\pgfqpoint{2.386155in}{1.968756in}}{\pgfqpoint{2.391979in}{1.962932in}}%
\pgfpathcurveto{\pgfqpoint{2.397803in}{1.957108in}}{\pgfqpoint{2.405703in}{1.953836in}}{\pgfqpoint{2.413939in}{1.953836in}}%
\pgfpathclose%
\pgfusepath{stroke,fill}%
\end{pgfscope}%
\begin{pgfscope}%
\pgfpathrectangle{\pgfqpoint{0.100000in}{0.212622in}}{\pgfqpoint{3.696000in}{3.696000in}}%
\pgfusepath{clip}%
\pgfsetbuttcap%
\pgfsetroundjoin%
\definecolor{currentfill}{rgb}{0.121569,0.466667,0.705882}%
\pgfsetfillcolor{currentfill}%
\pgfsetfillopacity{0.409003}%
\pgfsetlinewidth{1.003750pt}%
\definecolor{currentstroke}{rgb}{0.121569,0.466667,0.705882}%
\pgfsetstrokecolor{currentstroke}%
\pgfsetstrokeopacity{0.409003}%
\pgfsetdash{}{0pt}%
\pgfpathmoveto{\pgfqpoint{2.416571in}{1.953067in}}%
\pgfpathcurveto{\pgfqpoint{2.424807in}{1.953067in}}{\pgfqpoint{2.432707in}{1.956340in}}{\pgfqpoint{2.438531in}{1.962164in}}%
\pgfpathcurveto{\pgfqpoint{2.444355in}{1.967987in}}{\pgfqpoint{2.447627in}{1.975888in}}{\pgfqpoint{2.447627in}{1.984124in}}%
\pgfpathcurveto{\pgfqpoint{2.447627in}{1.992360in}}{\pgfqpoint{2.444355in}{2.000260in}}{\pgfqpoint{2.438531in}{2.006084in}}%
\pgfpathcurveto{\pgfqpoint{2.432707in}{2.011908in}}{\pgfqpoint{2.424807in}{2.015180in}}{\pgfqpoint{2.416571in}{2.015180in}}%
\pgfpathcurveto{\pgfqpoint{2.408334in}{2.015180in}}{\pgfqpoint{2.400434in}{2.011908in}}{\pgfqpoint{2.394610in}{2.006084in}}%
\pgfpathcurveto{\pgfqpoint{2.388786in}{2.000260in}}{\pgfqpoint{2.385514in}{1.992360in}}{\pgfqpoint{2.385514in}{1.984124in}}%
\pgfpathcurveto{\pgfqpoint{2.385514in}{1.975888in}}{\pgfqpoint{2.388786in}{1.967987in}}{\pgfqpoint{2.394610in}{1.962164in}}%
\pgfpathcurveto{\pgfqpoint{2.400434in}{1.956340in}}{\pgfqpoint{2.408334in}{1.953067in}}{\pgfqpoint{2.416571in}{1.953067in}}%
\pgfpathclose%
\pgfusepath{stroke,fill}%
\end{pgfscope}%
\begin{pgfscope}%
\pgfpathrectangle{\pgfqpoint{0.100000in}{0.212622in}}{\pgfqpoint{3.696000in}{3.696000in}}%
\pgfusepath{clip}%
\pgfsetbuttcap%
\pgfsetroundjoin%
\definecolor{currentfill}{rgb}{0.121569,0.466667,0.705882}%
\pgfsetfillcolor{currentfill}%
\pgfsetfillopacity{0.409182}%
\pgfsetlinewidth{1.003750pt}%
\definecolor{currentstroke}{rgb}{0.121569,0.466667,0.705882}%
\pgfsetstrokecolor{currentstroke}%
\pgfsetstrokeopacity{0.409182}%
\pgfsetdash{}{0pt}%
\pgfpathmoveto{\pgfqpoint{2.417911in}{1.952788in}}%
\pgfpathcurveto{\pgfqpoint{2.426148in}{1.952788in}}{\pgfqpoint{2.434048in}{1.956060in}}{\pgfqpoint{2.439872in}{1.961884in}}%
\pgfpathcurveto{\pgfqpoint{2.445696in}{1.967708in}}{\pgfqpoint{2.448968in}{1.975608in}}{\pgfqpoint{2.448968in}{1.983844in}}%
\pgfpathcurveto{\pgfqpoint{2.448968in}{1.992080in}}{\pgfqpoint{2.445696in}{1.999980in}}{\pgfqpoint{2.439872in}{2.005804in}}%
\pgfpathcurveto{\pgfqpoint{2.434048in}{2.011628in}}{\pgfqpoint{2.426148in}{2.014901in}}{\pgfqpoint{2.417911in}{2.014901in}}%
\pgfpathcurveto{\pgfqpoint{2.409675in}{2.014901in}}{\pgfqpoint{2.401775in}{2.011628in}}{\pgfqpoint{2.395951in}{2.005804in}}%
\pgfpathcurveto{\pgfqpoint{2.390127in}{1.999980in}}{\pgfqpoint{2.386855in}{1.992080in}}{\pgfqpoint{2.386855in}{1.983844in}}%
\pgfpathcurveto{\pgfqpoint{2.386855in}{1.975608in}}{\pgfqpoint{2.390127in}{1.967708in}}{\pgfqpoint{2.395951in}{1.961884in}}%
\pgfpathcurveto{\pgfqpoint{2.401775in}{1.956060in}}{\pgfqpoint{2.409675in}{1.952788in}}{\pgfqpoint{2.417911in}{1.952788in}}%
\pgfpathclose%
\pgfusepath{stroke,fill}%
\end{pgfscope}%
\begin{pgfscope}%
\pgfpathrectangle{\pgfqpoint{0.100000in}{0.212622in}}{\pgfqpoint{3.696000in}{3.696000in}}%
\pgfusepath{clip}%
\pgfsetbuttcap%
\pgfsetroundjoin%
\definecolor{currentfill}{rgb}{0.121569,0.466667,0.705882}%
\pgfsetfillcolor{currentfill}%
\pgfsetfillopacity{0.409409}%
\pgfsetlinewidth{1.003750pt}%
\definecolor{currentstroke}{rgb}{0.121569,0.466667,0.705882}%
\pgfsetstrokecolor{currentstroke}%
\pgfsetstrokeopacity{0.409409}%
\pgfsetdash{}{0pt}%
\pgfpathmoveto{\pgfqpoint{1.481797in}{2.110351in}}%
\pgfpathcurveto{\pgfqpoint{1.490034in}{2.110351in}}{\pgfqpoint{1.497934in}{2.113623in}}{\pgfqpoint{1.503758in}{2.119447in}}%
\pgfpathcurveto{\pgfqpoint{1.509582in}{2.125271in}}{\pgfqpoint{1.512854in}{2.133171in}}{\pgfqpoint{1.512854in}{2.141407in}}%
\pgfpathcurveto{\pgfqpoint{1.512854in}{2.149643in}}{\pgfqpoint{1.509582in}{2.157543in}}{\pgfqpoint{1.503758in}{2.163367in}}%
\pgfpathcurveto{\pgfqpoint{1.497934in}{2.169191in}}{\pgfqpoint{1.490034in}{2.172464in}}{\pgfqpoint{1.481797in}{2.172464in}}%
\pgfpathcurveto{\pgfqpoint{1.473561in}{2.172464in}}{\pgfqpoint{1.465661in}{2.169191in}}{\pgfqpoint{1.459837in}{2.163367in}}%
\pgfpathcurveto{\pgfqpoint{1.454013in}{2.157543in}}{\pgfqpoint{1.450741in}{2.149643in}}{\pgfqpoint{1.450741in}{2.141407in}}%
\pgfpathcurveto{\pgfqpoint{1.450741in}{2.133171in}}{\pgfqpoint{1.454013in}{2.125271in}}{\pgfqpoint{1.459837in}{2.119447in}}%
\pgfpathcurveto{\pgfqpoint{1.465661in}{2.113623in}}{\pgfqpoint{1.473561in}{2.110351in}}{\pgfqpoint{1.481797in}{2.110351in}}%
\pgfpathclose%
\pgfusepath{stroke,fill}%
\end{pgfscope}%
\begin{pgfscope}%
\pgfpathrectangle{\pgfqpoint{0.100000in}{0.212622in}}{\pgfqpoint{3.696000in}{3.696000in}}%
\pgfusepath{clip}%
\pgfsetbuttcap%
\pgfsetroundjoin%
\definecolor{currentfill}{rgb}{0.121569,0.466667,0.705882}%
\pgfsetfillcolor{currentfill}%
\pgfsetfillopacity{0.409501}%
\pgfsetlinewidth{1.003750pt}%
\definecolor{currentstroke}{rgb}{0.121569,0.466667,0.705882}%
\pgfsetstrokecolor{currentstroke}%
\pgfsetstrokeopacity{0.409501}%
\pgfsetdash{}{0pt}%
\pgfpathmoveto{\pgfqpoint{2.420471in}{1.952190in}}%
\pgfpathcurveto{\pgfqpoint{2.428708in}{1.952190in}}{\pgfqpoint{2.436608in}{1.955462in}}{\pgfqpoint{2.442432in}{1.961286in}}%
\pgfpathcurveto{\pgfqpoint{2.448256in}{1.967110in}}{\pgfqpoint{2.451528in}{1.975010in}}{\pgfqpoint{2.451528in}{1.983246in}}%
\pgfpathcurveto{\pgfqpoint{2.451528in}{1.991483in}}{\pgfqpoint{2.448256in}{1.999383in}}{\pgfqpoint{2.442432in}{2.005207in}}%
\pgfpathcurveto{\pgfqpoint{2.436608in}{2.011031in}}{\pgfqpoint{2.428708in}{2.014303in}}{\pgfqpoint{2.420471in}{2.014303in}}%
\pgfpathcurveto{\pgfqpoint{2.412235in}{2.014303in}}{\pgfqpoint{2.404335in}{2.011031in}}{\pgfqpoint{2.398511in}{2.005207in}}%
\pgfpathcurveto{\pgfqpoint{2.392687in}{1.999383in}}{\pgfqpoint{2.389415in}{1.991483in}}{\pgfqpoint{2.389415in}{1.983246in}}%
\pgfpathcurveto{\pgfqpoint{2.389415in}{1.975010in}}{\pgfqpoint{2.392687in}{1.967110in}}{\pgfqpoint{2.398511in}{1.961286in}}%
\pgfpathcurveto{\pgfqpoint{2.404335in}{1.955462in}}{\pgfqpoint{2.412235in}{1.952190in}}{\pgfqpoint{2.420471in}{1.952190in}}%
\pgfpathclose%
\pgfusepath{stroke,fill}%
\end{pgfscope}%
\begin{pgfscope}%
\pgfpathrectangle{\pgfqpoint{0.100000in}{0.212622in}}{\pgfqpoint{3.696000in}{3.696000in}}%
\pgfusepath{clip}%
\pgfsetbuttcap%
\pgfsetroundjoin%
\definecolor{currentfill}{rgb}{0.121569,0.466667,0.705882}%
\pgfsetfillcolor{currentfill}%
\pgfsetfillopacity{0.409694}%
\pgfsetlinewidth{1.003750pt}%
\definecolor{currentstroke}{rgb}{0.121569,0.466667,0.705882}%
\pgfsetstrokecolor{currentstroke}%
\pgfsetstrokeopacity{0.409694}%
\pgfsetdash{}{0pt}%
\pgfpathmoveto{\pgfqpoint{2.421844in}{1.951884in}}%
\pgfpathcurveto{\pgfqpoint{2.430080in}{1.951884in}}{\pgfqpoint{2.437981in}{1.955156in}}{\pgfqpoint{2.443804in}{1.960980in}}%
\pgfpathcurveto{\pgfqpoint{2.449628in}{1.966804in}}{\pgfqpoint{2.452901in}{1.974704in}}{\pgfqpoint{2.452901in}{1.982940in}}%
\pgfpathcurveto{\pgfqpoint{2.452901in}{1.991177in}}{\pgfqpoint{2.449628in}{1.999077in}}{\pgfqpoint{2.443804in}{2.004901in}}%
\pgfpathcurveto{\pgfqpoint{2.437981in}{2.010725in}}{\pgfqpoint{2.430080in}{2.013997in}}{\pgfqpoint{2.421844in}{2.013997in}}%
\pgfpathcurveto{\pgfqpoint{2.413608in}{2.013997in}}{\pgfqpoint{2.405708in}{2.010725in}}{\pgfqpoint{2.399884in}{2.004901in}}%
\pgfpathcurveto{\pgfqpoint{2.394060in}{1.999077in}}{\pgfqpoint{2.390788in}{1.991177in}}{\pgfqpoint{2.390788in}{1.982940in}}%
\pgfpathcurveto{\pgfqpoint{2.390788in}{1.974704in}}{\pgfqpoint{2.394060in}{1.966804in}}{\pgfqpoint{2.399884in}{1.960980in}}%
\pgfpathcurveto{\pgfqpoint{2.405708in}{1.955156in}}{\pgfqpoint{2.413608in}{1.951884in}}{\pgfqpoint{2.421844in}{1.951884in}}%
\pgfpathclose%
\pgfusepath{stroke,fill}%
\end{pgfscope}%
\begin{pgfscope}%
\pgfpathrectangle{\pgfqpoint{0.100000in}{0.212622in}}{\pgfqpoint{3.696000in}{3.696000in}}%
\pgfusepath{clip}%
\pgfsetbuttcap%
\pgfsetroundjoin%
\definecolor{currentfill}{rgb}{0.121569,0.466667,0.705882}%
\pgfsetfillcolor{currentfill}%
\pgfsetfillopacity{0.409759}%
\pgfsetlinewidth{1.003750pt}%
\definecolor{currentstroke}{rgb}{0.121569,0.466667,0.705882}%
\pgfsetstrokecolor{currentstroke}%
\pgfsetstrokeopacity{0.409759}%
\pgfsetdash{}{0pt}%
\pgfpathmoveto{\pgfqpoint{2.422668in}{1.951661in}}%
\pgfpathcurveto{\pgfqpoint{2.430904in}{1.951661in}}{\pgfqpoint{2.438805in}{1.954933in}}{\pgfqpoint{2.444628in}{1.960757in}}%
\pgfpathcurveto{\pgfqpoint{2.450452in}{1.966581in}}{\pgfqpoint{2.453725in}{1.974481in}}{\pgfqpoint{2.453725in}{1.982717in}}%
\pgfpathcurveto{\pgfqpoint{2.453725in}{1.990954in}}{\pgfqpoint{2.450452in}{1.998854in}}{\pgfqpoint{2.444628in}{2.004678in}}%
\pgfpathcurveto{\pgfqpoint{2.438805in}{2.010501in}}{\pgfqpoint{2.430904in}{2.013774in}}{\pgfqpoint{2.422668in}{2.013774in}}%
\pgfpathcurveto{\pgfqpoint{2.414432in}{2.013774in}}{\pgfqpoint{2.406532in}{2.010501in}}{\pgfqpoint{2.400708in}{2.004678in}}%
\pgfpathcurveto{\pgfqpoint{2.394884in}{1.998854in}}{\pgfqpoint{2.391612in}{1.990954in}}{\pgfqpoint{2.391612in}{1.982717in}}%
\pgfpathcurveto{\pgfqpoint{2.391612in}{1.974481in}}{\pgfqpoint{2.394884in}{1.966581in}}{\pgfqpoint{2.400708in}{1.960757in}}%
\pgfpathcurveto{\pgfqpoint{2.406532in}{1.954933in}}{\pgfqpoint{2.414432in}{1.951661in}}{\pgfqpoint{2.422668in}{1.951661in}}%
\pgfpathclose%
\pgfusepath{stroke,fill}%
\end{pgfscope}%
\begin{pgfscope}%
\pgfpathrectangle{\pgfqpoint{0.100000in}{0.212622in}}{\pgfqpoint{3.696000in}{3.696000in}}%
\pgfusepath{clip}%
\pgfsetbuttcap%
\pgfsetroundjoin%
\definecolor{currentfill}{rgb}{0.121569,0.466667,0.705882}%
\pgfsetfillcolor{currentfill}%
\pgfsetfillopacity{0.410019}%
\pgfsetlinewidth{1.003750pt}%
\definecolor{currentstroke}{rgb}{0.121569,0.466667,0.705882}%
\pgfsetstrokecolor{currentstroke}%
\pgfsetstrokeopacity{0.410019}%
\pgfsetdash{}{0pt}%
\pgfpathmoveto{\pgfqpoint{2.424468in}{1.951264in}}%
\pgfpathcurveto{\pgfqpoint{2.432704in}{1.951264in}}{\pgfqpoint{2.440604in}{1.954536in}}{\pgfqpoint{2.446428in}{1.960360in}}%
\pgfpathcurveto{\pgfqpoint{2.452252in}{1.966184in}}{\pgfqpoint{2.455525in}{1.974084in}}{\pgfqpoint{2.455525in}{1.982320in}}%
\pgfpathcurveto{\pgfqpoint{2.455525in}{1.990556in}}{\pgfqpoint{2.452252in}{1.998456in}}{\pgfqpoint{2.446428in}{2.004280in}}%
\pgfpathcurveto{\pgfqpoint{2.440604in}{2.010104in}}{\pgfqpoint{2.432704in}{2.013377in}}{\pgfqpoint{2.424468in}{2.013377in}}%
\pgfpathcurveto{\pgfqpoint{2.416232in}{2.013377in}}{\pgfqpoint{2.408332in}{2.010104in}}{\pgfqpoint{2.402508in}{2.004280in}}%
\pgfpathcurveto{\pgfqpoint{2.396684in}{1.998456in}}{\pgfqpoint{2.393412in}{1.990556in}}{\pgfqpoint{2.393412in}{1.982320in}}%
\pgfpathcurveto{\pgfqpoint{2.393412in}{1.974084in}}{\pgfqpoint{2.396684in}{1.966184in}}{\pgfqpoint{2.402508in}{1.960360in}}%
\pgfpathcurveto{\pgfqpoint{2.408332in}{1.954536in}}{\pgfqpoint{2.416232in}{1.951264in}}{\pgfqpoint{2.424468in}{1.951264in}}%
\pgfpathclose%
\pgfusepath{stroke,fill}%
\end{pgfscope}%
\begin{pgfscope}%
\pgfpathrectangle{\pgfqpoint{0.100000in}{0.212622in}}{\pgfqpoint{3.696000in}{3.696000in}}%
\pgfusepath{clip}%
\pgfsetbuttcap%
\pgfsetroundjoin%
\definecolor{currentfill}{rgb}{0.121569,0.466667,0.705882}%
\pgfsetfillcolor{currentfill}%
\pgfsetfillopacity{0.410186}%
\pgfsetlinewidth{1.003750pt}%
\definecolor{currentstroke}{rgb}{0.121569,0.466667,0.705882}%
\pgfsetstrokecolor{currentstroke}%
\pgfsetstrokeopacity{0.410186}%
\pgfsetdash{}{0pt}%
\pgfpathmoveto{\pgfqpoint{2.425399in}{1.951064in}}%
\pgfpathcurveto{\pgfqpoint{2.433635in}{1.951064in}}{\pgfqpoint{2.441535in}{1.954336in}}{\pgfqpoint{2.447359in}{1.960160in}}%
\pgfpathcurveto{\pgfqpoint{2.453183in}{1.965984in}}{\pgfqpoint{2.456455in}{1.973884in}}{\pgfqpoint{2.456455in}{1.982120in}}%
\pgfpathcurveto{\pgfqpoint{2.456455in}{1.990356in}}{\pgfqpoint{2.453183in}{1.998256in}}{\pgfqpoint{2.447359in}{2.004080in}}%
\pgfpathcurveto{\pgfqpoint{2.441535in}{2.009904in}}{\pgfqpoint{2.433635in}{2.013177in}}{\pgfqpoint{2.425399in}{2.013177in}}%
\pgfpathcurveto{\pgfqpoint{2.417163in}{2.013177in}}{\pgfqpoint{2.409263in}{2.009904in}}{\pgfqpoint{2.403439in}{2.004080in}}%
\pgfpathcurveto{\pgfqpoint{2.397615in}{1.998256in}}{\pgfqpoint{2.394342in}{1.990356in}}{\pgfqpoint{2.394342in}{1.982120in}}%
\pgfpathcurveto{\pgfqpoint{2.394342in}{1.973884in}}{\pgfqpoint{2.397615in}{1.965984in}}{\pgfqpoint{2.403439in}{1.960160in}}%
\pgfpathcurveto{\pgfqpoint{2.409263in}{1.954336in}}{\pgfqpoint{2.417163in}{1.951064in}}{\pgfqpoint{2.425399in}{1.951064in}}%
\pgfpathclose%
\pgfusepath{stroke,fill}%
\end{pgfscope}%
\begin{pgfscope}%
\pgfpathrectangle{\pgfqpoint{0.100000in}{0.212622in}}{\pgfqpoint{3.696000in}{3.696000in}}%
\pgfusepath{clip}%
\pgfsetbuttcap%
\pgfsetroundjoin%
\definecolor{currentfill}{rgb}{0.121569,0.466667,0.705882}%
\pgfsetfillcolor{currentfill}%
\pgfsetfillopacity{0.410241}%
\pgfsetlinewidth{1.003750pt}%
\definecolor{currentstroke}{rgb}{0.121569,0.466667,0.705882}%
\pgfsetstrokecolor{currentstroke}%
\pgfsetstrokeopacity{0.410241}%
\pgfsetdash{}{0pt}%
\pgfpathmoveto{\pgfqpoint{2.426904in}{1.950583in}}%
\pgfpathcurveto{\pgfqpoint{2.435141in}{1.950583in}}{\pgfqpoint{2.443041in}{1.953855in}}{\pgfqpoint{2.448865in}{1.959679in}}%
\pgfpathcurveto{\pgfqpoint{2.454689in}{1.965503in}}{\pgfqpoint{2.457961in}{1.973403in}}{\pgfqpoint{2.457961in}{1.981639in}}%
\pgfpathcurveto{\pgfqpoint{2.457961in}{1.989875in}}{\pgfqpoint{2.454689in}{1.997776in}}{\pgfqpoint{2.448865in}{2.003599in}}%
\pgfpathcurveto{\pgfqpoint{2.443041in}{2.009423in}}{\pgfqpoint{2.435141in}{2.012696in}}{\pgfqpoint{2.426904in}{2.012696in}}%
\pgfpathcurveto{\pgfqpoint{2.418668in}{2.012696in}}{\pgfqpoint{2.410768in}{2.009423in}}{\pgfqpoint{2.404944in}{2.003599in}}%
\pgfpathcurveto{\pgfqpoint{2.399120in}{1.997776in}}{\pgfqpoint{2.395848in}{1.989875in}}{\pgfqpoint{2.395848in}{1.981639in}}%
\pgfpathcurveto{\pgfqpoint{2.395848in}{1.973403in}}{\pgfqpoint{2.399120in}{1.965503in}}{\pgfqpoint{2.404944in}{1.959679in}}%
\pgfpathcurveto{\pgfqpoint{2.410768in}{1.953855in}}{\pgfqpoint{2.418668in}{1.950583in}}{\pgfqpoint{2.426904in}{1.950583in}}%
\pgfpathclose%
\pgfusepath{stroke,fill}%
\end{pgfscope}%
\begin{pgfscope}%
\pgfpathrectangle{\pgfqpoint{0.100000in}{0.212622in}}{\pgfqpoint{3.696000in}{3.696000in}}%
\pgfusepath{clip}%
\pgfsetbuttcap%
\pgfsetroundjoin%
\definecolor{currentfill}{rgb}{0.121569,0.466667,0.705882}%
\pgfsetfillcolor{currentfill}%
\pgfsetfillopacity{0.410452}%
\pgfsetlinewidth{1.003750pt}%
\definecolor{currentstroke}{rgb}{0.121569,0.466667,0.705882}%
\pgfsetstrokecolor{currentstroke}%
\pgfsetstrokeopacity{0.410452}%
\pgfsetdash{}{0pt}%
\pgfpathmoveto{\pgfqpoint{1.479149in}{2.110544in}}%
\pgfpathcurveto{\pgfqpoint{1.487385in}{2.110544in}}{\pgfqpoint{1.495285in}{2.113817in}}{\pgfqpoint{1.501109in}{2.119641in}}%
\pgfpathcurveto{\pgfqpoint{1.506933in}{2.125465in}}{\pgfqpoint{1.510205in}{2.133365in}}{\pgfqpoint{1.510205in}{2.141601in}}%
\pgfpathcurveto{\pgfqpoint{1.510205in}{2.149837in}}{\pgfqpoint{1.506933in}{2.157737in}}{\pgfqpoint{1.501109in}{2.163561in}}%
\pgfpathcurveto{\pgfqpoint{1.495285in}{2.169385in}}{\pgfqpoint{1.487385in}{2.172657in}}{\pgfqpoint{1.479149in}{2.172657in}}%
\pgfpathcurveto{\pgfqpoint{1.470912in}{2.172657in}}{\pgfqpoint{1.463012in}{2.169385in}}{\pgfqpoint{1.457188in}{2.163561in}}%
\pgfpathcurveto{\pgfqpoint{1.451364in}{2.157737in}}{\pgfqpoint{1.448092in}{2.149837in}}{\pgfqpoint{1.448092in}{2.141601in}}%
\pgfpathcurveto{\pgfqpoint{1.448092in}{2.133365in}}{\pgfqpoint{1.451364in}{2.125465in}}{\pgfqpoint{1.457188in}{2.119641in}}%
\pgfpathcurveto{\pgfqpoint{1.463012in}{2.113817in}}{\pgfqpoint{1.470912in}{2.110544in}}{\pgfqpoint{1.479149in}{2.110544in}}%
\pgfpathclose%
\pgfusepath{stroke,fill}%
\end{pgfscope}%
\begin{pgfscope}%
\pgfpathrectangle{\pgfqpoint{0.100000in}{0.212622in}}{\pgfqpoint{3.696000in}{3.696000in}}%
\pgfusepath{clip}%
\pgfsetbuttcap%
\pgfsetroundjoin%
\definecolor{currentfill}{rgb}{0.121569,0.466667,0.705882}%
\pgfsetfillcolor{currentfill}%
\pgfsetfillopacity{0.410528}%
\pgfsetlinewidth{1.003750pt}%
\definecolor{currentstroke}{rgb}{0.121569,0.466667,0.705882}%
\pgfsetstrokecolor{currentstroke}%
\pgfsetstrokeopacity{0.410528}%
\pgfsetdash{}{0pt}%
\pgfpathmoveto{\pgfqpoint{2.429070in}{1.950072in}}%
\pgfpathcurveto{\pgfqpoint{2.437306in}{1.950072in}}{\pgfqpoint{2.445206in}{1.953344in}}{\pgfqpoint{2.451030in}{1.959168in}}%
\pgfpathcurveto{\pgfqpoint{2.456854in}{1.964992in}}{\pgfqpoint{2.460126in}{1.972892in}}{\pgfqpoint{2.460126in}{1.981129in}}%
\pgfpathcurveto{\pgfqpoint{2.460126in}{1.989365in}}{\pgfqpoint{2.456854in}{1.997265in}}{\pgfqpoint{2.451030in}{2.003089in}}%
\pgfpathcurveto{\pgfqpoint{2.445206in}{2.008913in}}{\pgfqpoint{2.437306in}{2.012185in}}{\pgfqpoint{2.429070in}{2.012185in}}%
\pgfpathcurveto{\pgfqpoint{2.420833in}{2.012185in}}{\pgfqpoint{2.412933in}{2.008913in}}{\pgfqpoint{2.407109in}{2.003089in}}%
\pgfpathcurveto{\pgfqpoint{2.401285in}{1.997265in}}{\pgfqpoint{2.398013in}{1.989365in}}{\pgfqpoint{2.398013in}{1.981129in}}%
\pgfpathcurveto{\pgfqpoint{2.398013in}{1.972892in}}{\pgfqpoint{2.401285in}{1.964992in}}{\pgfqpoint{2.407109in}{1.959168in}}%
\pgfpathcurveto{\pgfqpoint{2.412933in}{1.953344in}}{\pgfqpoint{2.420833in}{1.950072in}}{\pgfqpoint{2.429070in}{1.950072in}}%
\pgfpathclose%
\pgfusepath{stroke,fill}%
\end{pgfscope}%
\begin{pgfscope}%
\pgfpathrectangle{\pgfqpoint{0.100000in}{0.212622in}}{\pgfqpoint{3.696000in}{3.696000in}}%
\pgfusepath{clip}%
\pgfsetbuttcap%
\pgfsetroundjoin%
\definecolor{currentfill}{rgb}{0.121569,0.466667,0.705882}%
\pgfsetfillcolor{currentfill}%
\pgfsetfillopacity{0.410654}%
\pgfsetlinewidth{1.003750pt}%
\definecolor{currentstroke}{rgb}{0.121569,0.466667,0.705882}%
\pgfsetstrokecolor{currentstroke}%
\pgfsetstrokeopacity{0.410654}%
\pgfsetdash{}{0pt}%
\pgfpathmoveto{\pgfqpoint{2.430328in}{1.949788in}}%
\pgfpathcurveto{\pgfqpoint{2.438564in}{1.949788in}}{\pgfqpoint{2.446464in}{1.953060in}}{\pgfqpoint{2.452288in}{1.958884in}}%
\pgfpathcurveto{\pgfqpoint{2.458112in}{1.964708in}}{\pgfqpoint{2.461385in}{1.972608in}}{\pgfqpoint{2.461385in}{1.980845in}}%
\pgfpathcurveto{\pgfqpoint{2.461385in}{1.989081in}}{\pgfqpoint{2.458112in}{1.996981in}}{\pgfqpoint{2.452288in}{2.002805in}}%
\pgfpathcurveto{\pgfqpoint{2.446464in}{2.008629in}}{\pgfqpoint{2.438564in}{2.011901in}}{\pgfqpoint{2.430328in}{2.011901in}}%
\pgfpathcurveto{\pgfqpoint{2.422092in}{2.011901in}}{\pgfqpoint{2.414192in}{2.008629in}}{\pgfqpoint{2.408368in}{2.002805in}}%
\pgfpathcurveto{\pgfqpoint{2.402544in}{1.996981in}}{\pgfqpoint{2.399272in}{1.989081in}}{\pgfqpoint{2.399272in}{1.980845in}}%
\pgfpathcurveto{\pgfqpoint{2.399272in}{1.972608in}}{\pgfqpoint{2.402544in}{1.964708in}}{\pgfqpoint{2.408368in}{1.958884in}}%
\pgfpathcurveto{\pgfqpoint{2.414192in}{1.953060in}}{\pgfqpoint{2.422092in}{1.949788in}}{\pgfqpoint{2.430328in}{1.949788in}}%
\pgfpathclose%
\pgfusepath{stroke,fill}%
\end{pgfscope}%
\begin{pgfscope}%
\pgfpathrectangle{\pgfqpoint{0.100000in}{0.212622in}}{\pgfqpoint{3.696000in}{3.696000in}}%
\pgfusepath{clip}%
\pgfsetbuttcap%
\pgfsetroundjoin%
\definecolor{currentfill}{rgb}{0.121569,0.466667,0.705882}%
\pgfsetfillcolor{currentfill}%
\pgfsetfillopacity{0.410835}%
\pgfsetlinewidth{1.003750pt}%
\definecolor{currentstroke}{rgb}{0.121569,0.466667,0.705882}%
\pgfsetstrokecolor{currentstroke}%
\pgfsetstrokeopacity{0.410835}%
\pgfsetdash{}{0pt}%
\pgfpathmoveto{\pgfqpoint{2.432477in}{1.949165in}}%
\pgfpathcurveto{\pgfqpoint{2.440714in}{1.949165in}}{\pgfqpoint{2.448614in}{1.952437in}}{\pgfqpoint{2.454438in}{1.958261in}}%
\pgfpathcurveto{\pgfqpoint{2.460262in}{1.964085in}}{\pgfqpoint{2.463534in}{1.971985in}}{\pgfqpoint{2.463534in}{1.980221in}}%
\pgfpathcurveto{\pgfqpoint{2.463534in}{1.988457in}}{\pgfqpoint{2.460262in}{1.996357in}}{\pgfqpoint{2.454438in}{2.002181in}}%
\pgfpathcurveto{\pgfqpoint{2.448614in}{2.008005in}}{\pgfqpoint{2.440714in}{2.011278in}}{\pgfqpoint{2.432477in}{2.011278in}}%
\pgfpathcurveto{\pgfqpoint{2.424241in}{2.011278in}}{\pgfqpoint{2.416341in}{2.008005in}}{\pgfqpoint{2.410517in}{2.002181in}}%
\pgfpathcurveto{\pgfqpoint{2.404693in}{1.996357in}}{\pgfqpoint{2.401421in}{1.988457in}}{\pgfqpoint{2.401421in}{1.980221in}}%
\pgfpathcurveto{\pgfqpoint{2.401421in}{1.971985in}}{\pgfqpoint{2.404693in}{1.964085in}}{\pgfqpoint{2.410517in}{1.958261in}}%
\pgfpathcurveto{\pgfqpoint{2.416341in}{1.952437in}}{\pgfqpoint{2.424241in}{1.949165in}}{\pgfqpoint{2.432477in}{1.949165in}}%
\pgfpathclose%
\pgfusepath{stroke,fill}%
\end{pgfscope}%
\begin{pgfscope}%
\pgfpathrectangle{\pgfqpoint{0.100000in}{0.212622in}}{\pgfqpoint{3.696000in}{3.696000in}}%
\pgfusepath{clip}%
\pgfsetbuttcap%
\pgfsetroundjoin%
\definecolor{currentfill}{rgb}{0.121569,0.466667,0.705882}%
\pgfsetfillcolor{currentfill}%
\pgfsetfillopacity{0.411120}%
\pgfsetlinewidth{1.003750pt}%
\definecolor{currentstroke}{rgb}{0.121569,0.466667,0.705882}%
\pgfsetstrokecolor{currentstroke}%
\pgfsetstrokeopacity{0.411120}%
\pgfsetdash{}{0pt}%
\pgfpathmoveto{\pgfqpoint{1.478031in}{2.110595in}}%
\pgfpathcurveto{\pgfqpoint{1.486267in}{2.110595in}}{\pgfqpoint{1.494167in}{2.113867in}}{\pgfqpoint{1.499991in}{2.119691in}}%
\pgfpathcurveto{\pgfqpoint{1.505815in}{2.125515in}}{\pgfqpoint{1.509087in}{2.133415in}}{\pgfqpoint{1.509087in}{2.141652in}}%
\pgfpathcurveto{\pgfqpoint{1.509087in}{2.149888in}}{\pgfqpoint{1.505815in}{2.157788in}}{\pgfqpoint{1.499991in}{2.163612in}}%
\pgfpathcurveto{\pgfqpoint{1.494167in}{2.169436in}}{\pgfqpoint{1.486267in}{2.172708in}}{\pgfqpoint{1.478031in}{2.172708in}}%
\pgfpathcurveto{\pgfqpoint{1.469795in}{2.172708in}}{\pgfqpoint{1.461894in}{2.169436in}}{\pgfqpoint{1.456071in}{2.163612in}}%
\pgfpathcurveto{\pgfqpoint{1.450247in}{2.157788in}}{\pgfqpoint{1.446974in}{2.149888in}}{\pgfqpoint{1.446974in}{2.141652in}}%
\pgfpathcurveto{\pgfqpoint{1.446974in}{2.133415in}}{\pgfqpoint{1.450247in}{2.125515in}}{\pgfqpoint{1.456071in}{2.119691in}}%
\pgfpathcurveto{\pgfqpoint{1.461894in}{2.113867in}}{\pgfqpoint{1.469795in}{2.110595in}}{\pgfqpoint{1.478031in}{2.110595in}}%
\pgfpathclose%
\pgfusepath{stroke,fill}%
\end{pgfscope}%
\begin{pgfscope}%
\pgfpathrectangle{\pgfqpoint{0.100000in}{0.212622in}}{\pgfqpoint{3.696000in}{3.696000in}}%
\pgfusepath{clip}%
\pgfsetbuttcap%
\pgfsetroundjoin%
\definecolor{currentfill}{rgb}{0.121569,0.466667,0.705882}%
\pgfsetfillcolor{currentfill}%
\pgfsetfillopacity{0.411197}%
\pgfsetlinewidth{1.003750pt}%
\definecolor{currentstroke}{rgb}{0.121569,0.466667,0.705882}%
\pgfsetstrokecolor{currentstroke}%
\pgfsetstrokeopacity{0.411197}%
\pgfsetdash{}{0pt}%
\pgfpathmoveto{\pgfqpoint{1.477854in}{2.110605in}}%
\pgfpathcurveto{\pgfqpoint{1.486090in}{2.110605in}}{\pgfqpoint{1.493990in}{2.113877in}}{\pgfqpoint{1.499814in}{2.119701in}}%
\pgfpathcurveto{\pgfqpoint{1.505638in}{2.125525in}}{\pgfqpoint{1.508910in}{2.133425in}}{\pgfqpoint{1.508910in}{2.141661in}}%
\pgfpathcurveto{\pgfqpoint{1.508910in}{2.149898in}}{\pgfqpoint{1.505638in}{2.157798in}}{\pgfqpoint{1.499814in}{2.163622in}}%
\pgfpathcurveto{\pgfqpoint{1.493990in}{2.169446in}}{\pgfqpoint{1.486090in}{2.172718in}}{\pgfqpoint{1.477854in}{2.172718in}}%
\pgfpathcurveto{\pgfqpoint{1.469618in}{2.172718in}}{\pgfqpoint{1.461717in}{2.169446in}}{\pgfqpoint{1.455894in}{2.163622in}}%
\pgfpathcurveto{\pgfqpoint{1.450070in}{2.157798in}}{\pgfqpoint{1.446797in}{2.149898in}}{\pgfqpoint{1.446797in}{2.141661in}}%
\pgfpathcurveto{\pgfqpoint{1.446797in}{2.133425in}}{\pgfqpoint{1.450070in}{2.125525in}}{\pgfqpoint{1.455894in}{2.119701in}}%
\pgfpathcurveto{\pgfqpoint{1.461717in}{2.113877in}}{\pgfqpoint{1.469618in}{2.110605in}}{\pgfqpoint{1.477854in}{2.110605in}}%
\pgfpathclose%
\pgfusepath{stroke,fill}%
\end{pgfscope}%
\begin{pgfscope}%
\pgfpathrectangle{\pgfqpoint{0.100000in}{0.212622in}}{\pgfqpoint{3.696000in}{3.696000in}}%
\pgfusepath{clip}%
\pgfsetbuttcap%
\pgfsetroundjoin%
\definecolor{currentfill}{rgb}{0.121569,0.466667,0.705882}%
\pgfsetfillcolor{currentfill}%
\pgfsetfillopacity{0.411339}%
\pgfsetlinewidth{1.003750pt}%
\definecolor{currentstroke}{rgb}{0.121569,0.466667,0.705882}%
\pgfsetstrokecolor{currentstroke}%
\pgfsetstrokeopacity{0.411339}%
\pgfsetdash{}{0pt}%
\pgfpathmoveto{\pgfqpoint{1.477542in}{2.110622in}}%
\pgfpathcurveto{\pgfqpoint{1.485779in}{2.110622in}}{\pgfqpoint{1.493679in}{2.113894in}}{\pgfqpoint{1.499503in}{2.119718in}}%
\pgfpathcurveto{\pgfqpoint{1.505327in}{2.125542in}}{\pgfqpoint{1.508599in}{2.133442in}}{\pgfqpoint{1.508599in}{2.141678in}}%
\pgfpathcurveto{\pgfqpoint{1.508599in}{2.149914in}}{\pgfqpoint{1.505327in}{2.157815in}}{\pgfqpoint{1.499503in}{2.163638in}}%
\pgfpathcurveto{\pgfqpoint{1.493679in}{2.169462in}}{\pgfqpoint{1.485779in}{2.172735in}}{\pgfqpoint{1.477542in}{2.172735in}}%
\pgfpathcurveto{\pgfqpoint{1.469306in}{2.172735in}}{\pgfqpoint{1.461406in}{2.169462in}}{\pgfqpoint{1.455582in}{2.163638in}}%
\pgfpathcurveto{\pgfqpoint{1.449758in}{2.157815in}}{\pgfqpoint{1.446486in}{2.149914in}}{\pgfqpoint{1.446486in}{2.141678in}}%
\pgfpathcurveto{\pgfqpoint{1.446486in}{2.133442in}}{\pgfqpoint{1.449758in}{2.125542in}}{\pgfqpoint{1.455582in}{2.119718in}}%
\pgfpathcurveto{\pgfqpoint{1.461406in}{2.113894in}}{\pgfqpoint{1.469306in}{2.110622in}}{\pgfqpoint{1.477542in}{2.110622in}}%
\pgfpathclose%
\pgfusepath{stroke,fill}%
\end{pgfscope}%
\begin{pgfscope}%
\pgfpathrectangle{\pgfqpoint{0.100000in}{0.212622in}}{\pgfqpoint{3.696000in}{3.696000in}}%
\pgfusepath{clip}%
\pgfsetbuttcap%
\pgfsetroundjoin%
\definecolor{currentfill}{rgb}{0.121569,0.466667,0.705882}%
\pgfsetfillcolor{currentfill}%
\pgfsetfillopacity{0.411363}%
\pgfsetlinewidth{1.003750pt}%
\definecolor{currentstroke}{rgb}{0.121569,0.466667,0.705882}%
\pgfsetstrokecolor{currentstroke}%
\pgfsetstrokeopacity{0.411363}%
\pgfsetdash{}{0pt}%
\pgfpathmoveto{\pgfqpoint{2.436834in}{1.948211in}}%
\pgfpathcurveto{\pgfqpoint{2.445070in}{1.948211in}}{\pgfqpoint{2.452970in}{1.951483in}}{\pgfqpoint{2.458794in}{1.957307in}}%
\pgfpathcurveto{\pgfqpoint{2.464618in}{1.963131in}}{\pgfqpoint{2.467890in}{1.971031in}}{\pgfqpoint{2.467890in}{1.979268in}}%
\pgfpathcurveto{\pgfqpoint{2.467890in}{1.987504in}}{\pgfqpoint{2.464618in}{1.995404in}}{\pgfqpoint{2.458794in}{2.001228in}}%
\pgfpathcurveto{\pgfqpoint{2.452970in}{2.007052in}}{\pgfqpoint{2.445070in}{2.010324in}}{\pgfqpoint{2.436834in}{2.010324in}}%
\pgfpathcurveto{\pgfqpoint{2.428597in}{2.010324in}}{\pgfqpoint{2.420697in}{2.007052in}}{\pgfqpoint{2.414874in}{2.001228in}}%
\pgfpathcurveto{\pgfqpoint{2.409050in}{1.995404in}}{\pgfqpoint{2.405777in}{1.987504in}}{\pgfqpoint{2.405777in}{1.979268in}}%
\pgfpathcurveto{\pgfqpoint{2.405777in}{1.971031in}}{\pgfqpoint{2.409050in}{1.963131in}}{\pgfqpoint{2.414874in}{1.957307in}}%
\pgfpathcurveto{\pgfqpoint{2.420697in}{1.951483in}}{\pgfqpoint{2.428597in}{1.948211in}}{\pgfqpoint{2.436834in}{1.948211in}}%
\pgfpathclose%
\pgfusepath{stroke,fill}%
\end{pgfscope}%
\begin{pgfscope}%
\pgfpathrectangle{\pgfqpoint{0.100000in}{0.212622in}}{\pgfqpoint{3.696000in}{3.696000in}}%
\pgfusepath{clip}%
\pgfsetbuttcap%
\pgfsetroundjoin%
\definecolor{currentfill}{rgb}{0.121569,0.466667,0.705882}%
\pgfsetfillcolor{currentfill}%
\pgfsetfillopacity{0.411624}%
\pgfsetlinewidth{1.003750pt}%
\definecolor{currentstroke}{rgb}{0.121569,0.466667,0.705882}%
\pgfsetstrokecolor{currentstroke}%
\pgfsetstrokeopacity{0.411624}%
\pgfsetdash{}{0pt}%
\pgfpathmoveto{\pgfqpoint{1.477197in}{2.110673in}}%
\pgfpathcurveto{\pgfqpoint{1.485433in}{2.110673in}}{\pgfqpoint{1.493333in}{2.113945in}}{\pgfqpoint{1.499157in}{2.119769in}}%
\pgfpathcurveto{\pgfqpoint{1.504981in}{2.125593in}}{\pgfqpoint{1.508253in}{2.133493in}}{\pgfqpoint{1.508253in}{2.141729in}}%
\pgfpathcurveto{\pgfqpoint{1.508253in}{2.149966in}}{\pgfqpoint{1.504981in}{2.157866in}}{\pgfqpoint{1.499157in}{2.163690in}}%
\pgfpathcurveto{\pgfqpoint{1.493333in}{2.169514in}}{\pgfqpoint{1.485433in}{2.172786in}}{\pgfqpoint{1.477197in}{2.172786in}}%
\pgfpathcurveto{\pgfqpoint{1.468960in}{2.172786in}}{\pgfqpoint{1.461060in}{2.169514in}}{\pgfqpoint{1.455236in}{2.163690in}}%
\pgfpathcurveto{\pgfqpoint{1.449412in}{2.157866in}}{\pgfqpoint{1.446140in}{2.149966in}}{\pgfqpoint{1.446140in}{2.141729in}}%
\pgfpathcurveto{\pgfqpoint{1.446140in}{2.133493in}}{\pgfqpoint{1.449412in}{2.125593in}}{\pgfqpoint{1.455236in}{2.119769in}}%
\pgfpathcurveto{\pgfqpoint{1.461060in}{2.113945in}}{\pgfqpoint{1.468960in}{2.110673in}}{\pgfqpoint{1.477197in}{2.110673in}}%
\pgfpathclose%
\pgfusepath{stroke,fill}%
\end{pgfscope}%
\begin{pgfscope}%
\pgfpathrectangle{\pgfqpoint{0.100000in}{0.212622in}}{\pgfqpoint{3.696000in}{3.696000in}}%
\pgfusepath{clip}%
\pgfsetbuttcap%
\pgfsetroundjoin%
\definecolor{currentfill}{rgb}{0.121569,0.466667,0.705882}%
\pgfsetfillcolor{currentfill}%
\pgfsetfillopacity{0.412018}%
\pgfsetlinewidth{1.003750pt}%
\definecolor{currentstroke}{rgb}{0.121569,0.466667,0.705882}%
\pgfsetstrokecolor{currentstroke}%
\pgfsetstrokeopacity{0.412018}%
\pgfsetdash{}{0pt}%
\pgfpathmoveto{\pgfqpoint{2.441587in}{1.947058in}}%
\pgfpathcurveto{\pgfqpoint{2.449823in}{1.947058in}}{\pgfqpoint{2.457723in}{1.950331in}}{\pgfqpoint{2.463547in}{1.956155in}}%
\pgfpathcurveto{\pgfqpoint{2.469371in}{1.961978in}}{\pgfqpoint{2.472643in}{1.969879in}}{\pgfqpoint{2.472643in}{1.978115in}}%
\pgfpathcurveto{\pgfqpoint{2.472643in}{1.986351in}}{\pgfqpoint{2.469371in}{1.994251in}}{\pgfqpoint{2.463547in}{2.000075in}}%
\pgfpathcurveto{\pgfqpoint{2.457723in}{2.005899in}}{\pgfqpoint{2.449823in}{2.009171in}}{\pgfqpoint{2.441587in}{2.009171in}}%
\pgfpathcurveto{\pgfqpoint{2.433350in}{2.009171in}}{\pgfqpoint{2.425450in}{2.005899in}}{\pgfqpoint{2.419626in}{2.000075in}}%
\pgfpathcurveto{\pgfqpoint{2.413802in}{1.994251in}}{\pgfqpoint{2.410530in}{1.986351in}}{\pgfqpoint{2.410530in}{1.978115in}}%
\pgfpathcurveto{\pgfqpoint{2.410530in}{1.969879in}}{\pgfqpoint{2.413802in}{1.961978in}}{\pgfqpoint{2.419626in}{1.956155in}}%
\pgfpathcurveto{\pgfqpoint{2.425450in}{1.950331in}}{\pgfqpoint{2.433350in}{1.947058in}}{\pgfqpoint{2.441587in}{1.947058in}}%
\pgfpathclose%
\pgfusepath{stroke,fill}%
\end{pgfscope}%
\begin{pgfscope}%
\pgfpathrectangle{\pgfqpoint{0.100000in}{0.212622in}}{\pgfqpoint{3.696000in}{3.696000in}}%
\pgfusepath{clip}%
\pgfsetbuttcap%
\pgfsetroundjoin%
\definecolor{currentfill}{rgb}{0.121569,0.466667,0.705882}%
\pgfsetfillcolor{currentfill}%
\pgfsetfillopacity{0.412093}%
\pgfsetlinewidth{1.003750pt}%
\definecolor{currentstroke}{rgb}{0.121569,0.466667,0.705882}%
\pgfsetstrokecolor{currentstroke}%
\pgfsetstrokeopacity{0.412093}%
\pgfsetdash{}{0pt}%
\pgfpathmoveto{\pgfqpoint{1.476183in}{2.110705in}}%
\pgfpathcurveto{\pgfqpoint{1.484420in}{2.110705in}}{\pgfqpoint{1.492320in}{2.113977in}}{\pgfqpoint{1.498144in}{2.119801in}}%
\pgfpathcurveto{\pgfqpoint{1.503968in}{2.125625in}}{\pgfqpoint{1.507240in}{2.133525in}}{\pgfqpoint{1.507240in}{2.141761in}}%
\pgfpathcurveto{\pgfqpoint{1.507240in}{2.149998in}}{\pgfqpoint{1.503968in}{2.157898in}}{\pgfqpoint{1.498144in}{2.163722in}}%
\pgfpathcurveto{\pgfqpoint{1.492320in}{2.169545in}}{\pgfqpoint{1.484420in}{2.172818in}}{\pgfqpoint{1.476183in}{2.172818in}}%
\pgfpathcurveto{\pgfqpoint{1.467947in}{2.172818in}}{\pgfqpoint{1.460047in}{2.169545in}}{\pgfqpoint{1.454223in}{2.163722in}}%
\pgfpathcurveto{\pgfqpoint{1.448399in}{2.157898in}}{\pgfqpoint{1.445127in}{2.149998in}}{\pgfqpoint{1.445127in}{2.141761in}}%
\pgfpathcurveto{\pgfqpoint{1.445127in}{2.133525in}}{\pgfqpoint{1.448399in}{2.125625in}}{\pgfqpoint{1.454223in}{2.119801in}}%
\pgfpathcurveto{\pgfqpoint{1.460047in}{2.113977in}}{\pgfqpoint{1.467947in}{2.110705in}}{\pgfqpoint{1.476183in}{2.110705in}}%
\pgfpathclose%
\pgfusepath{stroke,fill}%
\end{pgfscope}%
\begin{pgfscope}%
\pgfpathrectangle{\pgfqpoint{0.100000in}{0.212622in}}{\pgfqpoint{3.696000in}{3.696000in}}%
\pgfusepath{clip}%
\pgfsetbuttcap%
\pgfsetroundjoin%
\definecolor{currentfill}{rgb}{0.121569,0.466667,0.705882}%
\pgfsetfillcolor{currentfill}%
\pgfsetfillopacity{0.412270}%
\pgfsetlinewidth{1.003750pt}%
\definecolor{currentstroke}{rgb}{0.121569,0.466667,0.705882}%
\pgfsetstrokecolor{currentstroke}%
\pgfsetstrokeopacity{0.412270}%
\pgfsetdash{}{0pt}%
\pgfpathmoveto{\pgfqpoint{2.444416in}{1.946391in}}%
\pgfpathcurveto{\pgfqpoint{2.452652in}{1.946391in}}{\pgfqpoint{2.460552in}{1.949663in}}{\pgfqpoint{2.466376in}{1.955487in}}%
\pgfpathcurveto{\pgfqpoint{2.472200in}{1.961311in}}{\pgfqpoint{2.475472in}{1.969211in}}{\pgfqpoint{2.475472in}{1.977447in}}%
\pgfpathcurveto{\pgfqpoint{2.475472in}{1.985683in}}{\pgfqpoint{2.472200in}{1.993583in}}{\pgfqpoint{2.466376in}{1.999407in}}%
\pgfpathcurveto{\pgfqpoint{2.460552in}{2.005231in}}{\pgfqpoint{2.452652in}{2.008504in}}{\pgfqpoint{2.444416in}{2.008504in}}%
\pgfpathcurveto{\pgfqpoint{2.436179in}{2.008504in}}{\pgfqpoint{2.428279in}{2.005231in}}{\pgfqpoint{2.422455in}{1.999407in}}%
\pgfpathcurveto{\pgfqpoint{2.416631in}{1.993583in}}{\pgfqpoint{2.413359in}{1.985683in}}{\pgfqpoint{2.413359in}{1.977447in}}%
\pgfpathcurveto{\pgfqpoint{2.413359in}{1.969211in}}{\pgfqpoint{2.416631in}{1.961311in}}{\pgfqpoint{2.422455in}{1.955487in}}%
\pgfpathcurveto{\pgfqpoint{2.428279in}{1.949663in}}{\pgfqpoint{2.436179in}{1.946391in}}{\pgfqpoint{2.444416in}{1.946391in}}%
\pgfpathclose%
\pgfusepath{stroke,fill}%
\end{pgfscope}%
\begin{pgfscope}%
\pgfpathrectangle{\pgfqpoint{0.100000in}{0.212622in}}{\pgfqpoint{3.696000in}{3.696000in}}%
\pgfusepath{clip}%
\pgfsetbuttcap%
\pgfsetroundjoin%
\definecolor{currentfill}{rgb}{0.121569,0.466667,0.705882}%
\pgfsetfillcolor{currentfill}%
\pgfsetfillopacity{0.412691}%
\pgfsetlinewidth{1.003750pt}%
\definecolor{currentstroke}{rgb}{0.121569,0.466667,0.705882}%
\pgfsetstrokecolor{currentstroke}%
\pgfsetstrokeopacity{0.412691}%
\pgfsetdash{}{0pt}%
\pgfpathmoveto{\pgfqpoint{2.448066in}{1.945511in}}%
\pgfpathcurveto{\pgfqpoint{2.456302in}{1.945511in}}{\pgfqpoint{2.464203in}{1.948783in}}{\pgfqpoint{2.470026in}{1.954607in}}%
\pgfpathcurveto{\pgfqpoint{2.475850in}{1.960431in}}{\pgfqpoint{2.479123in}{1.968331in}}{\pgfqpoint{2.479123in}{1.976568in}}%
\pgfpathcurveto{\pgfqpoint{2.479123in}{1.984804in}}{\pgfqpoint{2.475850in}{1.992704in}}{\pgfqpoint{2.470026in}{1.998528in}}%
\pgfpathcurveto{\pgfqpoint{2.464203in}{2.004352in}}{\pgfqpoint{2.456302in}{2.007624in}}{\pgfqpoint{2.448066in}{2.007624in}}%
\pgfpathcurveto{\pgfqpoint{2.439830in}{2.007624in}}{\pgfqpoint{2.431930in}{2.004352in}}{\pgfqpoint{2.426106in}{1.998528in}}%
\pgfpathcurveto{\pgfqpoint{2.420282in}{1.992704in}}{\pgfqpoint{2.417010in}{1.984804in}}{\pgfqpoint{2.417010in}{1.976568in}}%
\pgfpathcurveto{\pgfqpoint{2.417010in}{1.968331in}}{\pgfqpoint{2.420282in}{1.960431in}}{\pgfqpoint{2.426106in}{1.954607in}}%
\pgfpathcurveto{\pgfqpoint{2.431930in}{1.948783in}}{\pgfqpoint{2.439830in}{1.945511in}}{\pgfqpoint{2.448066in}{1.945511in}}%
\pgfpathclose%
\pgfusepath{stroke,fill}%
\end{pgfscope}%
\begin{pgfscope}%
\pgfpathrectangle{\pgfqpoint{0.100000in}{0.212622in}}{\pgfqpoint{3.696000in}{3.696000in}}%
\pgfusepath{clip}%
\pgfsetbuttcap%
\pgfsetroundjoin%
\definecolor{currentfill}{rgb}{0.121569,0.466667,0.705882}%
\pgfsetfillcolor{currentfill}%
\pgfsetfillopacity{0.412957}%
\pgfsetlinewidth{1.003750pt}%
\definecolor{currentstroke}{rgb}{0.121569,0.466667,0.705882}%
\pgfsetstrokecolor{currentstroke}%
\pgfsetstrokeopacity{0.412957}%
\pgfsetdash{}{0pt}%
\pgfpathmoveto{\pgfqpoint{1.474423in}{2.110763in}}%
\pgfpathcurveto{\pgfqpoint{1.482659in}{2.110763in}}{\pgfqpoint{1.490559in}{2.114036in}}{\pgfqpoint{1.496383in}{2.119860in}}%
\pgfpathcurveto{\pgfqpoint{1.502207in}{2.125684in}}{\pgfqpoint{1.505480in}{2.133584in}}{\pgfqpoint{1.505480in}{2.141820in}}%
\pgfpathcurveto{\pgfqpoint{1.505480in}{2.150056in}}{\pgfqpoint{1.502207in}{2.157956in}}{\pgfqpoint{1.496383in}{2.163780in}}%
\pgfpathcurveto{\pgfqpoint{1.490559in}{2.169604in}}{\pgfqpoint{1.482659in}{2.172876in}}{\pgfqpoint{1.474423in}{2.172876in}}%
\pgfpathcurveto{\pgfqpoint{1.466187in}{2.172876in}}{\pgfqpoint{1.458287in}{2.169604in}}{\pgfqpoint{1.452463in}{2.163780in}}%
\pgfpathcurveto{\pgfqpoint{1.446639in}{2.157956in}}{\pgfqpoint{1.443367in}{2.150056in}}{\pgfqpoint{1.443367in}{2.141820in}}%
\pgfpathcurveto{\pgfqpoint{1.443367in}{2.133584in}}{\pgfqpoint{1.446639in}{2.125684in}}{\pgfqpoint{1.452463in}{2.119860in}}%
\pgfpathcurveto{\pgfqpoint{1.458287in}{2.114036in}}{\pgfqpoint{1.466187in}{2.110763in}}{\pgfqpoint{1.474423in}{2.110763in}}%
\pgfpathclose%
\pgfusepath{stroke,fill}%
\end{pgfscope}%
\begin{pgfscope}%
\pgfpathrectangle{\pgfqpoint{0.100000in}{0.212622in}}{\pgfqpoint{3.696000in}{3.696000in}}%
\pgfusepath{clip}%
\pgfsetbuttcap%
\pgfsetroundjoin%
\definecolor{currentfill}{rgb}{0.121569,0.466667,0.705882}%
\pgfsetfillcolor{currentfill}%
\pgfsetfillopacity{0.412976}%
\pgfsetlinewidth{1.003750pt}%
\definecolor{currentstroke}{rgb}{0.121569,0.466667,0.705882}%
\pgfsetstrokecolor{currentstroke}%
\pgfsetstrokeopacity{0.412976}%
\pgfsetdash{}{0pt}%
\pgfpathmoveto{\pgfqpoint{2.449928in}{1.944984in}}%
\pgfpathcurveto{\pgfqpoint{2.458165in}{1.944984in}}{\pgfqpoint{2.466065in}{1.948256in}}{\pgfqpoint{2.471889in}{1.954080in}}%
\pgfpathcurveto{\pgfqpoint{2.477712in}{1.959904in}}{\pgfqpoint{2.480985in}{1.967804in}}{\pgfqpoint{2.480985in}{1.976040in}}%
\pgfpathcurveto{\pgfqpoint{2.480985in}{1.984276in}}{\pgfqpoint{2.477712in}{1.992177in}}{\pgfqpoint{2.471889in}{1.998000in}}%
\pgfpathcurveto{\pgfqpoint{2.466065in}{2.003824in}}{\pgfqpoint{2.458165in}{2.007097in}}{\pgfqpoint{2.449928in}{2.007097in}}%
\pgfpathcurveto{\pgfqpoint{2.441692in}{2.007097in}}{\pgfqpoint{2.433792in}{2.003824in}}{\pgfqpoint{2.427968in}{1.998000in}}%
\pgfpathcurveto{\pgfqpoint{2.422144in}{1.992177in}}{\pgfqpoint{2.418872in}{1.984276in}}{\pgfqpoint{2.418872in}{1.976040in}}%
\pgfpathcurveto{\pgfqpoint{2.418872in}{1.967804in}}{\pgfqpoint{2.422144in}{1.959904in}}{\pgfqpoint{2.427968in}{1.954080in}}%
\pgfpathcurveto{\pgfqpoint{2.433792in}{1.948256in}}{\pgfqpoint{2.441692in}{1.944984in}}{\pgfqpoint{2.449928in}{1.944984in}}%
\pgfpathclose%
\pgfusepath{stroke,fill}%
\end{pgfscope}%
\begin{pgfscope}%
\pgfpathrectangle{\pgfqpoint{0.100000in}{0.212622in}}{\pgfqpoint{3.696000in}{3.696000in}}%
\pgfusepath{clip}%
\pgfsetbuttcap%
\pgfsetroundjoin%
\definecolor{currentfill}{rgb}{0.121569,0.466667,0.705882}%
\pgfsetfillcolor{currentfill}%
\pgfsetfillopacity{0.413191}%
\pgfsetlinewidth{1.003750pt}%
\definecolor{currentstroke}{rgb}{0.121569,0.466667,0.705882}%
\pgfsetstrokecolor{currentstroke}%
\pgfsetstrokeopacity{0.413191}%
\pgfsetdash{}{0pt}%
\pgfpathmoveto{\pgfqpoint{2.452870in}{1.944059in}}%
\pgfpathcurveto{\pgfqpoint{2.461107in}{1.944059in}}{\pgfqpoint{2.469007in}{1.947332in}}{\pgfqpoint{2.474831in}{1.953156in}}%
\pgfpathcurveto{\pgfqpoint{2.480654in}{1.958980in}}{\pgfqpoint{2.483927in}{1.966880in}}{\pgfqpoint{2.483927in}{1.975116in}}%
\pgfpathcurveto{\pgfqpoint{2.483927in}{1.983352in}}{\pgfqpoint{2.480654in}{1.991252in}}{\pgfqpoint{2.474831in}{1.997076in}}%
\pgfpathcurveto{\pgfqpoint{2.469007in}{2.002900in}}{\pgfqpoint{2.461107in}{2.006172in}}{\pgfqpoint{2.452870in}{2.006172in}}%
\pgfpathcurveto{\pgfqpoint{2.444634in}{2.006172in}}{\pgfqpoint{2.436734in}{2.002900in}}{\pgfqpoint{2.430910in}{1.997076in}}%
\pgfpathcurveto{\pgfqpoint{2.425086in}{1.991252in}}{\pgfqpoint{2.421814in}{1.983352in}}{\pgfqpoint{2.421814in}{1.975116in}}%
\pgfpathcurveto{\pgfqpoint{2.421814in}{1.966880in}}{\pgfqpoint{2.425086in}{1.958980in}}{\pgfqpoint{2.430910in}{1.953156in}}%
\pgfpathcurveto{\pgfqpoint{2.436734in}{1.947332in}}{\pgfqpoint{2.444634in}{1.944059in}}{\pgfqpoint{2.452870in}{1.944059in}}%
\pgfpathclose%
\pgfusepath{stroke,fill}%
\end{pgfscope}%
\begin{pgfscope}%
\pgfpathrectangle{\pgfqpoint{0.100000in}{0.212622in}}{\pgfqpoint{3.696000in}{3.696000in}}%
\pgfusepath{clip}%
\pgfsetbuttcap%
\pgfsetroundjoin%
\definecolor{currentfill}{rgb}{0.121569,0.466667,0.705882}%
\pgfsetfillcolor{currentfill}%
\pgfsetfillopacity{0.413630}%
\pgfsetlinewidth{1.003750pt}%
\definecolor{currentstroke}{rgb}{0.121569,0.466667,0.705882}%
\pgfsetstrokecolor{currentstroke}%
\pgfsetstrokeopacity{0.413630}%
\pgfsetdash{}{0pt}%
\pgfpathmoveto{\pgfqpoint{2.457843in}{1.942578in}}%
\pgfpathcurveto{\pgfqpoint{2.466079in}{1.942578in}}{\pgfqpoint{2.473979in}{1.945850in}}{\pgfqpoint{2.479803in}{1.951674in}}%
\pgfpathcurveto{\pgfqpoint{2.485627in}{1.957498in}}{\pgfqpoint{2.488900in}{1.965398in}}{\pgfqpoint{2.488900in}{1.973634in}}%
\pgfpathcurveto{\pgfqpoint{2.488900in}{1.981871in}}{\pgfqpoint{2.485627in}{1.989771in}}{\pgfqpoint{2.479803in}{1.995595in}}%
\pgfpathcurveto{\pgfqpoint{2.473979in}{2.001419in}}{\pgfqpoint{2.466079in}{2.004691in}}{\pgfqpoint{2.457843in}{2.004691in}}%
\pgfpathcurveto{\pgfqpoint{2.449607in}{2.004691in}}{\pgfqpoint{2.441707in}{2.001419in}}{\pgfqpoint{2.435883in}{1.995595in}}%
\pgfpathcurveto{\pgfqpoint{2.430059in}{1.989771in}}{\pgfqpoint{2.426787in}{1.981871in}}{\pgfqpoint{2.426787in}{1.973634in}}%
\pgfpathcurveto{\pgfqpoint{2.426787in}{1.965398in}}{\pgfqpoint{2.430059in}{1.957498in}}{\pgfqpoint{2.435883in}{1.951674in}}%
\pgfpathcurveto{\pgfqpoint{2.441707in}{1.945850in}}{\pgfqpoint{2.449607in}{1.942578in}}{\pgfqpoint{2.457843in}{1.942578in}}%
\pgfpathclose%
\pgfusepath{stroke,fill}%
\end{pgfscope}%
\begin{pgfscope}%
\pgfpathrectangle{\pgfqpoint{0.100000in}{0.212622in}}{\pgfqpoint{3.696000in}{3.696000in}}%
\pgfusepath{clip}%
\pgfsetbuttcap%
\pgfsetroundjoin%
\definecolor{currentfill}{rgb}{0.121569,0.466667,0.705882}%
\pgfsetfillcolor{currentfill}%
\pgfsetfillopacity{0.413811}%
\pgfsetlinewidth{1.003750pt}%
\definecolor{currentstroke}{rgb}{0.121569,0.466667,0.705882}%
\pgfsetstrokecolor{currentstroke}%
\pgfsetstrokeopacity{0.413811}%
\pgfsetdash{}{0pt}%
\pgfpathmoveto{\pgfqpoint{1.473382in}{2.110828in}}%
\pgfpathcurveto{\pgfqpoint{1.481619in}{2.110828in}}{\pgfqpoint{1.489519in}{2.114100in}}{\pgfqpoint{1.495343in}{2.119924in}}%
\pgfpathcurveto{\pgfqpoint{1.501167in}{2.125748in}}{\pgfqpoint{1.504439in}{2.133648in}}{\pgfqpoint{1.504439in}{2.141885in}}%
\pgfpathcurveto{\pgfqpoint{1.504439in}{2.150121in}}{\pgfqpoint{1.501167in}{2.158021in}}{\pgfqpoint{1.495343in}{2.163845in}}%
\pgfpathcurveto{\pgfqpoint{1.489519in}{2.169669in}}{\pgfqpoint{1.481619in}{2.172941in}}{\pgfqpoint{1.473382in}{2.172941in}}%
\pgfpathcurveto{\pgfqpoint{1.465146in}{2.172941in}}{\pgfqpoint{1.457246in}{2.169669in}}{\pgfqpoint{1.451422in}{2.163845in}}%
\pgfpathcurveto{\pgfqpoint{1.445598in}{2.158021in}}{\pgfqpoint{1.442326in}{2.150121in}}{\pgfqpoint{1.442326in}{2.141885in}}%
\pgfpathcurveto{\pgfqpoint{1.442326in}{2.133648in}}{\pgfqpoint{1.445598in}{2.125748in}}{\pgfqpoint{1.451422in}{2.119924in}}%
\pgfpathcurveto{\pgfqpoint{1.457246in}{2.114100in}}{\pgfqpoint{1.465146in}{2.110828in}}{\pgfqpoint{1.473382in}{2.110828in}}%
\pgfpathclose%
\pgfusepath{stroke,fill}%
\end{pgfscope}%
\begin{pgfscope}%
\pgfpathrectangle{\pgfqpoint{0.100000in}{0.212622in}}{\pgfqpoint{3.696000in}{3.696000in}}%
\pgfusepath{clip}%
\pgfsetbuttcap%
\pgfsetroundjoin%
\definecolor{currentfill}{rgb}{0.121569,0.466667,0.705882}%
\pgfsetfillcolor{currentfill}%
\pgfsetfillopacity{0.413926}%
\pgfsetlinewidth{1.003750pt}%
\definecolor{currentstroke}{rgb}{0.121569,0.466667,0.705882}%
\pgfsetstrokecolor{currentstroke}%
\pgfsetstrokeopacity{0.413926}%
\pgfsetdash{}{0pt}%
\pgfpathmoveto{\pgfqpoint{2.464302in}{1.940152in}}%
\pgfpathcurveto{\pgfqpoint{2.472538in}{1.940152in}}{\pgfqpoint{2.480438in}{1.943425in}}{\pgfqpoint{2.486262in}{1.949249in}}%
\pgfpathcurveto{\pgfqpoint{2.492086in}{1.955073in}}{\pgfqpoint{2.495358in}{1.962973in}}{\pgfqpoint{2.495358in}{1.971209in}}%
\pgfpathcurveto{\pgfqpoint{2.495358in}{1.979445in}}{\pgfqpoint{2.492086in}{1.987345in}}{\pgfqpoint{2.486262in}{1.993169in}}%
\pgfpathcurveto{\pgfqpoint{2.480438in}{1.998993in}}{\pgfqpoint{2.472538in}{2.002265in}}{\pgfqpoint{2.464302in}{2.002265in}}%
\pgfpathcurveto{\pgfqpoint{2.456065in}{2.002265in}}{\pgfqpoint{2.448165in}{1.998993in}}{\pgfqpoint{2.442341in}{1.993169in}}%
\pgfpathcurveto{\pgfqpoint{2.436517in}{1.987345in}}{\pgfqpoint{2.433245in}{1.979445in}}{\pgfqpoint{2.433245in}{1.971209in}}%
\pgfpathcurveto{\pgfqpoint{2.433245in}{1.962973in}}{\pgfqpoint{2.436517in}{1.955073in}}{\pgfqpoint{2.442341in}{1.949249in}}%
\pgfpathcurveto{\pgfqpoint{2.448165in}{1.943425in}}{\pgfqpoint{2.456065in}{1.940152in}}{\pgfqpoint{2.464302in}{1.940152in}}%
\pgfpathclose%
\pgfusepath{stroke,fill}%
\end{pgfscope}%
\begin{pgfscope}%
\pgfpathrectangle{\pgfqpoint{0.100000in}{0.212622in}}{\pgfqpoint{3.696000in}{3.696000in}}%
\pgfusepath{clip}%
\pgfsetbuttcap%
\pgfsetroundjoin%
\definecolor{currentfill}{rgb}{0.121569,0.466667,0.705882}%
\pgfsetfillcolor{currentfill}%
\pgfsetfillopacity{0.414342}%
\pgfsetlinewidth{1.003750pt}%
\definecolor{currentstroke}{rgb}{0.121569,0.466667,0.705882}%
\pgfsetstrokecolor{currentstroke}%
\pgfsetstrokeopacity{0.414342}%
\pgfsetdash{}{0pt}%
\pgfpathmoveto{\pgfqpoint{1.472042in}{2.110946in}}%
\pgfpathcurveto{\pgfqpoint{1.480278in}{2.110946in}}{\pgfqpoint{1.488178in}{2.114218in}}{\pgfqpoint{1.494002in}{2.120042in}}%
\pgfpathcurveto{\pgfqpoint{1.499826in}{2.125866in}}{\pgfqpoint{1.503098in}{2.133766in}}{\pgfqpoint{1.503098in}{2.142002in}}%
\pgfpathcurveto{\pgfqpoint{1.503098in}{2.150239in}}{\pgfqpoint{1.499826in}{2.158139in}}{\pgfqpoint{1.494002in}{2.163963in}}%
\pgfpathcurveto{\pgfqpoint{1.488178in}{2.169787in}}{\pgfqpoint{1.480278in}{2.173059in}}{\pgfqpoint{1.472042in}{2.173059in}}%
\pgfpathcurveto{\pgfqpoint{1.463806in}{2.173059in}}{\pgfqpoint{1.455905in}{2.169787in}}{\pgfqpoint{1.450082in}{2.163963in}}%
\pgfpathcurveto{\pgfqpoint{1.444258in}{2.158139in}}{\pgfqpoint{1.440985in}{2.150239in}}{\pgfqpoint{1.440985in}{2.142002in}}%
\pgfpathcurveto{\pgfqpoint{1.440985in}{2.133766in}}{\pgfqpoint{1.444258in}{2.125866in}}{\pgfqpoint{1.450082in}{2.120042in}}%
\pgfpathcurveto{\pgfqpoint{1.455905in}{2.114218in}}{\pgfqpoint{1.463806in}{2.110946in}}{\pgfqpoint{1.472042in}{2.110946in}}%
\pgfpathclose%
\pgfusepath{stroke,fill}%
\end{pgfscope}%
\begin{pgfscope}%
\pgfpathrectangle{\pgfqpoint{0.100000in}{0.212622in}}{\pgfqpoint{3.696000in}{3.696000in}}%
\pgfusepath{clip}%
\pgfsetbuttcap%
\pgfsetroundjoin%
\definecolor{currentfill}{rgb}{0.121569,0.466667,0.705882}%
\pgfsetfillcolor{currentfill}%
\pgfsetfillopacity{0.414658}%
\pgfsetlinewidth{1.003750pt}%
\definecolor{currentstroke}{rgb}{0.121569,0.466667,0.705882}%
\pgfsetstrokecolor{currentstroke}%
\pgfsetstrokeopacity{0.414658}%
\pgfsetdash{}{0pt}%
\pgfpathmoveto{\pgfqpoint{2.471184in}{1.938246in}}%
\pgfpathcurveto{\pgfqpoint{2.479421in}{1.938246in}}{\pgfqpoint{2.487321in}{1.941518in}}{\pgfqpoint{2.493145in}{1.947342in}}%
\pgfpathcurveto{\pgfqpoint{2.498969in}{1.953166in}}{\pgfqpoint{2.502241in}{1.961066in}}{\pgfqpoint{2.502241in}{1.969303in}}%
\pgfpathcurveto{\pgfqpoint{2.502241in}{1.977539in}}{\pgfqpoint{2.498969in}{1.985439in}}{\pgfqpoint{2.493145in}{1.991263in}}%
\pgfpathcurveto{\pgfqpoint{2.487321in}{1.997087in}}{\pgfqpoint{2.479421in}{2.000359in}}{\pgfqpoint{2.471184in}{2.000359in}}%
\pgfpathcurveto{\pgfqpoint{2.462948in}{2.000359in}}{\pgfqpoint{2.455048in}{1.997087in}}{\pgfqpoint{2.449224in}{1.991263in}}%
\pgfpathcurveto{\pgfqpoint{2.443400in}{1.985439in}}{\pgfqpoint{2.440128in}{1.977539in}}{\pgfqpoint{2.440128in}{1.969303in}}%
\pgfpathcurveto{\pgfqpoint{2.440128in}{1.961066in}}{\pgfqpoint{2.443400in}{1.953166in}}{\pgfqpoint{2.449224in}{1.947342in}}%
\pgfpathcurveto{\pgfqpoint{2.455048in}{1.941518in}}{\pgfqpoint{2.462948in}{1.938246in}}{\pgfqpoint{2.471184in}{1.938246in}}%
\pgfpathclose%
\pgfusepath{stroke,fill}%
\end{pgfscope}%
\begin{pgfscope}%
\pgfpathrectangle{\pgfqpoint{0.100000in}{0.212622in}}{\pgfqpoint{3.696000in}{3.696000in}}%
\pgfusepath{clip}%
\pgfsetbuttcap%
\pgfsetroundjoin%
\definecolor{currentfill}{rgb}{0.121569,0.466667,0.705882}%
\pgfsetfillcolor{currentfill}%
\pgfsetfillopacity{0.414796}%
\pgfsetlinewidth{1.003750pt}%
\definecolor{currentstroke}{rgb}{0.121569,0.466667,0.705882}%
\pgfsetstrokecolor{currentstroke}%
\pgfsetstrokeopacity{0.414796}%
\pgfsetdash{}{0pt}%
\pgfpathmoveto{\pgfqpoint{1.471496in}{2.111147in}}%
\pgfpathcurveto{\pgfqpoint{1.479732in}{2.111147in}}{\pgfqpoint{1.487632in}{2.114419in}}{\pgfqpoint{1.493456in}{2.120243in}}%
\pgfpathcurveto{\pgfqpoint{1.499280in}{2.126067in}}{\pgfqpoint{1.502552in}{2.133967in}}{\pgfqpoint{1.502552in}{2.142203in}}%
\pgfpathcurveto{\pgfqpoint{1.502552in}{2.150439in}}{\pgfqpoint{1.499280in}{2.158339in}}{\pgfqpoint{1.493456in}{2.164163in}}%
\pgfpathcurveto{\pgfqpoint{1.487632in}{2.169987in}}{\pgfqpoint{1.479732in}{2.173260in}}{\pgfqpoint{1.471496in}{2.173260in}}%
\pgfpathcurveto{\pgfqpoint{1.463259in}{2.173260in}}{\pgfqpoint{1.455359in}{2.169987in}}{\pgfqpoint{1.449535in}{2.164163in}}%
\pgfpathcurveto{\pgfqpoint{1.443711in}{2.158339in}}{\pgfqpoint{1.440439in}{2.150439in}}{\pgfqpoint{1.440439in}{2.142203in}}%
\pgfpathcurveto{\pgfqpoint{1.440439in}{2.133967in}}{\pgfqpoint{1.443711in}{2.126067in}}{\pgfqpoint{1.449535in}{2.120243in}}%
\pgfpathcurveto{\pgfqpoint{1.455359in}{2.114419in}}{\pgfqpoint{1.463259in}{2.111147in}}{\pgfqpoint{1.471496in}{2.111147in}}%
\pgfpathclose%
\pgfusepath{stroke,fill}%
\end{pgfscope}%
\begin{pgfscope}%
\pgfpathrectangle{\pgfqpoint{0.100000in}{0.212622in}}{\pgfqpoint{3.696000in}{3.696000in}}%
\pgfusepath{clip}%
\pgfsetbuttcap%
\pgfsetroundjoin%
\definecolor{currentfill}{rgb}{0.121569,0.466667,0.705882}%
\pgfsetfillcolor{currentfill}%
\pgfsetfillopacity{0.415124}%
\pgfsetlinewidth{1.003750pt}%
\definecolor{currentstroke}{rgb}{0.121569,0.466667,0.705882}%
\pgfsetstrokecolor{currentstroke}%
\pgfsetstrokeopacity{0.415124}%
\pgfsetdash{}{0pt}%
\pgfpathmoveto{\pgfqpoint{1.470888in}{2.111165in}}%
\pgfpathcurveto{\pgfqpoint{1.479125in}{2.111165in}}{\pgfqpoint{1.487025in}{2.114437in}}{\pgfqpoint{1.492848in}{2.120261in}}%
\pgfpathcurveto{\pgfqpoint{1.498672in}{2.126085in}}{\pgfqpoint{1.501945in}{2.133985in}}{\pgfqpoint{1.501945in}{2.142221in}}%
\pgfpathcurveto{\pgfqpoint{1.501945in}{2.150457in}}{\pgfqpoint{1.498672in}{2.158357in}}{\pgfqpoint{1.492848in}{2.164181in}}%
\pgfpathcurveto{\pgfqpoint{1.487025in}{2.170005in}}{\pgfqpoint{1.479125in}{2.173278in}}{\pgfqpoint{1.470888in}{2.173278in}}%
\pgfpathcurveto{\pgfqpoint{1.462652in}{2.173278in}}{\pgfqpoint{1.454752in}{2.170005in}}{\pgfqpoint{1.448928in}{2.164181in}}%
\pgfpathcurveto{\pgfqpoint{1.443104in}{2.158357in}}{\pgfqpoint{1.439832in}{2.150457in}}{\pgfqpoint{1.439832in}{2.142221in}}%
\pgfpathcurveto{\pgfqpoint{1.439832in}{2.133985in}}{\pgfqpoint{1.443104in}{2.126085in}}{\pgfqpoint{1.448928in}{2.120261in}}%
\pgfpathcurveto{\pgfqpoint{1.454752in}{2.114437in}}{\pgfqpoint{1.462652in}{2.111165in}}{\pgfqpoint{1.470888in}{2.111165in}}%
\pgfpathclose%
\pgfusepath{stroke,fill}%
\end{pgfscope}%
\begin{pgfscope}%
\pgfpathrectangle{\pgfqpoint{0.100000in}{0.212622in}}{\pgfqpoint{3.696000in}{3.696000in}}%
\pgfusepath{clip}%
\pgfsetbuttcap%
\pgfsetroundjoin%
\definecolor{currentfill}{rgb}{0.121569,0.466667,0.705882}%
\pgfsetfillcolor{currentfill}%
\pgfsetfillopacity{0.415282}%
\pgfsetlinewidth{1.003750pt}%
\definecolor{currentstroke}{rgb}{0.121569,0.466667,0.705882}%
\pgfsetstrokecolor{currentstroke}%
\pgfsetstrokeopacity{0.415282}%
\pgfsetdash{}{0pt}%
\pgfpathmoveto{\pgfqpoint{1.470499in}{2.111189in}}%
\pgfpathcurveto{\pgfqpoint{1.478735in}{2.111189in}}{\pgfqpoint{1.486636in}{2.114461in}}{\pgfqpoint{1.492459in}{2.120285in}}%
\pgfpathcurveto{\pgfqpoint{1.498283in}{2.126109in}}{\pgfqpoint{1.501556in}{2.134009in}}{\pgfqpoint{1.501556in}{2.142246in}}%
\pgfpathcurveto{\pgfqpoint{1.501556in}{2.150482in}}{\pgfqpoint{1.498283in}{2.158382in}}{\pgfqpoint{1.492459in}{2.164206in}}%
\pgfpathcurveto{\pgfqpoint{1.486636in}{2.170030in}}{\pgfqpoint{1.478735in}{2.173302in}}{\pgfqpoint{1.470499in}{2.173302in}}%
\pgfpathcurveto{\pgfqpoint{1.462263in}{2.173302in}}{\pgfqpoint{1.454363in}{2.170030in}}{\pgfqpoint{1.448539in}{2.164206in}}%
\pgfpathcurveto{\pgfqpoint{1.442715in}{2.158382in}}{\pgfqpoint{1.439443in}{2.150482in}}{\pgfqpoint{1.439443in}{2.142246in}}%
\pgfpathcurveto{\pgfqpoint{1.439443in}{2.134009in}}{\pgfqpoint{1.442715in}{2.126109in}}{\pgfqpoint{1.448539in}{2.120285in}}%
\pgfpathcurveto{\pgfqpoint{1.454363in}{2.114461in}}{\pgfqpoint{1.462263in}{2.111189in}}{\pgfqpoint{1.470499in}{2.111189in}}%
\pgfpathclose%
\pgfusepath{stroke,fill}%
\end{pgfscope}%
\begin{pgfscope}%
\pgfpathrectangle{\pgfqpoint{0.100000in}{0.212622in}}{\pgfqpoint{3.696000in}{3.696000in}}%
\pgfusepath{clip}%
\pgfsetbuttcap%
\pgfsetroundjoin%
\definecolor{currentfill}{rgb}{0.121569,0.466667,0.705882}%
\pgfsetfillcolor{currentfill}%
\pgfsetfillopacity{0.415417}%
\pgfsetlinewidth{1.003750pt}%
\definecolor{currentstroke}{rgb}{0.121569,0.466667,0.705882}%
\pgfsetstrokecolor{currentstroke}%
\pgfsetstrokeopacity{0.415417}%
\pgfsetdash{}{0pt}%
\pgfpathmoveto{\pgfqpoint{2.478809in}{1.935980in}}%
\pgfpathcurveto{\pgfqpoint{2.487046in}{1.935980in}}{\pgfqpoint{2.494946in}{1.939253in}}{\pgfqpoint{2.500770in}{1.945077in}}%
\pgfpathcurveto{\pgfqpoint{2.506594in}{1.950901in}}{\pgfqpoint{2.509866in}{1.958801in}}{\pgfqpoint{2.509866in}{1.967037in}}%
\pgfpathcurveto{\pgfqpoint{2.509866in}{1.975273in}}{\pgfqpoint{2.506594in}{1.983173in}}{\pgfqpoint{2.500770in}{1.988997in}}%
\pgfpathcurveto{\pgfqpoint{2.494946in}{1.994821in}}{\pgfqpoint{2.487046in}{1.998093in}}{\pgfqpoint{2.478809in}{1.998093in}}%
\pgfpathcurveto{\pgfqpoint{2.470573in}{1.998093in}}{\pgfqpoint{2.462673in}{1.994821in}}{\pgfqpoint{2.456849in}{1.988997in}}%
\pgfpathcurveto{\pgfqpoint{2.451025in}{1.983173in}}{\pgfqpoint{2.447753in}{1.975273in}}{\pgfqpoint{2.447753in}{1.967037in}}%
\pgfpathcurveto{\pgfqpoint{2.447753in}{1.958801in}}{\pgfqpoint{2.451025in}{1.950901in}}{\pgfqpoint{2.456849in}{1.945077in}}%
\pgfpathcurveto{\pgfqpoint{2.462673in}{1.939253in}}{\pgfqpoint{2.470573in}{1.935980in}}{\pgfqpoint{2.478809in}{1.935980in}}%
\pgfpathclose%
\pgfusepath{stroke,fill}%
\end{pgfscope}%
\begin{pgfscope}%
\pgfpathrectangle{\pgfqpoint{0.100000in}{0.212622in}}{\pgfqpoint{3.696000in}{3.696000in}}%
\pgfusepath{clip}%
\pgfsetbuttcap%
\pgfsetroundjoin%
\definecolor{currentfill}{rgb}{0.121569,0.466667,0.705882}%
\pgfsetfillcolor{currentfill}%
\pgfsetfillopacity{0.415602}%
\pgfsetlinewidth{1.003750pt}%
\definecolor{currentstroke}{rgb}{0.121569,0.466667,0.705882}%
\pgfsetstrokecolor{currentstroke}%
\pgfsetstrokeopacity{0.415602}%
\pgfsetdash{}{0pt}%
\pgfpathmoveto{\pgfqpoint{1.470084in}{2.111223in}}%
\pgfpathcurveto{\pgfqpoint{1.478320in}{2.111223in}}{\pgfqpoint{1.486220in}{2.114495in}}{\pgfqpoint{1.492044in}{2.120319in}}%
\pgfpathcurveto{\pgfqpoint{1.497868in}{2.126143in}}{\pgfqpoint{1.501140in}{2.134043in}}{\pgfqpoint{1.501140in}{2.142279in}}%
\pgfpathcurveto{\pgfqpoint{1.501140in}{2.150516in}}{\pgfqpoint{1.497868in}{2.158416in}}{\pgfqpoint{1.492044in}{2.164240in}}%
\pgfpathcurveto{\pgfqpoint{1.486220in}{2.170064in}}{\pgfqpoint{1.478320in}{2.173336in}}{\pgfqpoint{1.470084in}{2.173336in}}%
\pgfpathcurveto{\pgfqpoint{1.461847in}{2.173336in}}{\pgfqpoint{1.453947in}{2.170064in}}{\pgfqpoint{1.448123in}{2.164240in}}%
\pgfpathcurveto{\pgfqpoint{1.442300in}{2.158416in}}{\pgfqpoint{1.439027in}{2.150516in}}{\pgfqpoint{1.439027in}{2.142279in}}%
\pgfpathcurveto{\pgfqpoint{1.439027in}{2.134043in}}{\pgfqpoint{1.442300in}{2.126143in}}{\pgfqpoint{1.448123in}{2.120319in}}%
\pgfpathcurveto{\pgfqpoint{1.453947in}{2.114495in}}{\pgfqpoint{1.461847in}{2.111223in}}{\pgfqpoint{1.470084in}{2.111223in}}%
\pgfpathclose%
\pgfusepath{stroke,fill}%
\end{pgfscope}%
\begin{pgfscope}%
\pgfpathrectangle{\pgfqpoint{0.100000in}{0.212622in}}{\pgfqpoint{3.696000in}{3.696000in}}%
\pgfusepath{clip}%
\pgfsetbuttcap%
\pgfsetroundjoin%
\definecolor{currentfill}{rgb}{0.121569,0.466667,0.705882}%
\pgfsetfillcolor{currentfill}%
\pgfsetfillopacity{0.415917}%
\pgfsetlinewidth{1.003750pt}%
\definecolor{currentstroke}{rgb}{0.121569,0.466667,0.705882}%
\pgfsetstrokecolor{currentstroke}%
\pgfsetstrokeopacity{0.415917}%
\pgfsetdash{}{0pt}%
\pgfpathmoveto{\pgfqpoint{2.487349in}{1.933170in}}%
\pgfpathcurveto{\pgfqpoint{2.495585in}{1.933170in}}{\pgfqpoint{2.503485in}{1.936442in}}{\pgfqpoint{2.509309in}{1.942266in}}%
\pgfpathcurveto{\pgfqpoint{2.515133in}{1.948090in}}{\pgfqpoint{2.518405in}{1.955990in}}{\pgfqpoint{2.518405in}{1.964226in}}%
\pgfpathcurveto{\pgfqpoint{2.518405in}{1.972462in}}{\pgfqpoint{2.515133in}{1.980362in}}{\pgfqpoint{2.509309in}{1.986186in}}%
\pgfpathcurveto{\pgfqpoint{2.503485in}{1.992010in}}{\pgfqpoint{2.495585in}{1.995283in}}{\pgfqpoint{2.487349in}{1.995283in}}%
\pgfpathcurveto{\pgfqpoint{2.479112in}{1.995283in}}{\pgfqpoint{2.471212in}{1.992010in}}{\pgfqpoint{2.465388in}{1.986186in}}%
\pgfpathcurveto{\pgfqpoint{2.459564in}{1.980362in}}{\pgfqpoint{2.456292in}{1.972462in}}{\pgfqpoint{2.456292in}{1.964226in}}%
\pgfpathcurveto{\pgfqpoint{2.456292in}{1.955990in}}{\pgfqpoint{2.459564in}{1.948090in}}{\pgfqpoint{2.465388in}{1.942266in}}%
\pgfpathcurveto{\pgfqpoint{2.471212in}{1.936442in}}{\pgfqpoint{2.479112in}{1.933170in}}{\pgfqpoint{2.487349in}{1.933170in}}%
\pgfpathclose%
\pgfusepath{stroke,fill}%
\end{pgfscope}%
\begin{pgfscope}%
\pgfpathrectangle{\pgfqpoint{0.100000in}{0.212622in}}{\pgfqpoint{3.696000in}{3.696000in}}%
\pgfusepath{clip}%
\pgfsetbuttcap%
\pgfsetroundjoin%
\definecolor{currentfill}{rgb}{0.121569,0.466667,0.705882}%
\pgfsetfillcolor{currentfill}%
\pgfsetfillopacity{0.416153}%
\pgfsetlinewidth{1.003750pt}%
\definecolor{currentstroke}{rgb}{0.121569,0.466667,0.705882}%
\pgfsetstrokecolor{currentstroke}%
\pgfsetstrokeopacity{0.416153}%
\pgfsetdash{}{0pt}%
\pgfpathmoveto{\pgfqpoint{1.469043in}{2.111259in}}%
\pgfpathcurveto{\pgfqpoint{1.477279in}{2.111259in}}{\pgfqpoint{1.485179in}{2.114531in}}{\pgfqpoint{1.491003in}{2.120355in}}%
\pgfpathcurveto{\pgfqpoint{1.496827in}{2.126179in}}{\pgfqpoint{1.500099in}{2.134079in}}{\pgfqpoint{1.500099in}{2.142316in}}%
\pgfpathcurveto{\pgfqpoint{1.500099in}{2.150552in}}{\pgfqpoint{1.496827in}{2.158452in}}{\pgfqpoint{1.491003in}{2.164276in}}%
\pgfpathcurveto{\pgfqpoint{1.485179in}{2.170100in}}{\pgfqpoint{1.477279in}{2.173372in}}{\pgfqpoint{1.469043in}{2.173372in}}%
\pgfpathcurveto{\pgfqpoint{1.460807in}{2.173372in}}{\pgfqpoint{1.452907in}{2.170100in}}{\pgfqpoint{1.447083in}{2.164276in}}%
\pgfpathcurveto{\pgfqpoint{1.441259in}{2.158452in}}{\pgfqpoint{1.437986in}{2.150552in}}{\pgfqpoint{1.437986in}{2.142316in}}%
\pgfpathcurveto{\pgfqpoint{1.437986in}{2.134079in}}{\pgfqpoint{1.441259in}{2.126179in}}{\pgfqpoint{1.447083in}{2.120355in}}%
\pgfpathcurveto{\pgfqpoint{1.452907in}{2.114531in}}{\pgfqpoint{1.460807in}{2.111259in}}{\pgfqpoint{1.469043in}{2.111259in}}%
\pgfpathclose%
\pgfusepath{stroke,fill}%
\end{pgfscope}%
\begin{pgfscope}%
\pgfpathrectangle{\pgfqpoint{0.100000in}{0.212622in}}{\pgfqpoint{3.696000in}{3.696000in}}%
\pgfusepath{clip}%
\pgfsetbuttcap%
\pgfsetroundjoin%
\definecolor{currentfill}{rgb}{0.121569,0.466667,0.705882}%
\pgfsetfillcolor{currentfill}%
\pgfsetfillopacity{0.416609}%
\pgfsetlinewidth{1.003750pt}%
\definecolor{currentstroke}{rgb}{0.121569,0.466667,0.705882}%
\pgfsetstrokecolor{currentstroke}%
\pgfsetstrokeopacity{0.416609}%
\pgfsetdash{}{0pt}%
\pgfpathmoveto{\pgfqpoint{1.468073in}{2.111279in}}%
\pgfpathcurveto{\pgfqpoint{1.476309in}{2.111279in}}{\pgfqpoint{1.484209in}{2.114551in}}{\pgfqpoint{1.490033in}{2.120375in}}%
\pgfpathcurveto{\pgfqpoint{1.495857in}{2.126199in}}{\pgfqpoint{1.499129in}{2.134099in}}{\pgfqpoint{1.499129in}{2.142335in}}%
\pgfpathcurveto{\pgfqpoint{1.499129in}{2.150571in}}{\pgfqpoint{1.495857in}{2.158471in}}{\pgfqpoint{1.490033in}{2.164295in}}%
\pgfpathcurveto{\pgfqpoint{1.484209in}{2.170119in}}{\pgfqpoint{1.476309in}{2.173392in}}{\pgfqpoint{1.468073in}{2.173392in}}%
\pgfpathcurveto{\pgfqpoint{1.459836in}{2.173392in}}{\pgfqpoint{1.451936in}{2.170119in}}{\pgfqpoint{1.446112in}{2.164295in}}%
\pgfpathcurveto{\pgfqpoint{1.440288in}{2.158471in}}{\pgfqpoint{1.437016in}{2.150571in}}{\pgfqpoint{1.437016in}{2.142335in}}%
\pgfpathcurveto{\pgfqpoint{1.437016in}{2.134099in}}{\pgfqpoint{1.440288in}{2.126199in}}{\pgfqpoint{1.446112in}{2.120375in}}%
\pgfpathcurveto{\pgfqpoint{1.451936in}{2.114551in}}{\pgfqpoint{1.459836in}{2.111279in}}{\pgfqpoint{1.468073in}{2.111279in}}%
\pgfpathclose%
\pgfusepath{stroke,fill}%
\end{pgfscope}%
\begin{pgfscope}%
\pgfpathrectangle{\pgfqpoint{0.100000in}{0.212622in}}{\pgfqpoint{3.696000in}{3.696000in}}%
\pgfusepath{clip}%
\pgfsetbuttcap%
\pgfsetroundjoin%
\definecolor{currentfill}{rgb}{0.121569,0.466667,0.705882}%
\pgfsetfillcolor{currentfill}%
\pgfsetfillopacity{0.416889}%
\pgfsetlinewidth{1.003750pt}%
\definecolor{currentstroke}{rgb}{0.121569,0.466667,0.705882}%
\pgfsetstrokecolor{currentstroke}%
\pgfsetstrokeopacity{0.416889}%
\pgfsetdash{}{0pt}%
\pgfpathmoveto{\pgfqpoint{2.495773in}{1.931026in}}%
\pgfpathcurveto{\pgfqpoint{2.504009in}{1.931026in}}{\pgfqpoint{2.511909in}{1.934298in}}{\pgfqpoint{2.517733in}{1.940122in}}%
\pgfpathcurveto{\pgfqpoint{2.523557in}{1.945946in}}{\pgfqpoint{2.526829in}{1.953846in}}{\pgfqpoint{2.526829in}{1.962082in}}%
\pgfpathcurveto{\pgfqpoint{2.526829in}{1.970318in}}{\pgfqpoint{2.523557in}{1.978218in}}{\pgfqpoint{2.517733in}{1.984042in}}%
\pgfpathcurveto{\pgfqpoint{2.511909in}{1.989866in}}{\pgfqpoint{2.504009in}{1.993139in}}{\pgfqpoint{2.495773in}{1.993139in}}%
\pgfpathcurveto{\pgfqpoint{2.487537in}{1.993139in}}{\pgfqpoint{2.479637in}{1.989866in}}{\pgfqpoint{2.473813in}{1.984042in}}%
\pgfpathcurveto{\pgfqpoint{2.467989in}{1.978218in}}{\pgfqpoint{2.464716in}{1.970318in}}{\pgfqpoint{2.464716in}{1.962082in}}%
\pgfpathcurveto{\pgfqpoint{2.464716in}{1.953846in}}{\pgfqpoint{2.467989in}{1.945946in}}{\pgfqpoint{2.473813in}{1.940122in}}%
\pgfpathcurveto{\pgfqpoint{2.479637in}{1.934298in}}{\pgfqpoint{2.487537in}{1.931026in}}{\pgfqpoint{2.495773in}{1.931026in}}%
\pgfpathclose%
\pgfusepath{stroke,fill}%
\end{pgfscope}%
\begin{pgfscope}%
\pgfpathrectangle{\pgfqpoint{0.100000in}{0.212622in}}{\pgfqpoint{3.696000in}{3.696000in}}%
\pgfusepath{clip}%
\pgfsetbuttcap%
\pgfsetroundjoin%
\definecolor{currentfill}{rgb}{0.121569,0.466667,0.705882}%
\pgfsetfillcolor{currentfill}%
\pgfsetfillopacity{0.416997}%
\pgfsetlinewidth{1.003750pt}%
\definecolor{currentstroke}{rgb}{0.121569,0.466667,0.705882}%
\pgfsetstrokecolor{currentstroke}%
\pgfsetstrokeopacity{0.416997}%
\pgfsetdash{}{0pt}%
\pgfpathmoveto{\pgfqpoint{1.467689in}{2.111346in}}%
\pgfpathcurveto{\pgfqpoint{1.475925in}{2.111346in}}{\pgfqpoint{1.483825in}{2.114618in}}{\pgfqpoint{1.489649in}{2.120442in}}%
\pgfpathcurveto{\pgfqpoint{1.495473in}{2.126266in}}{\pgfqpoint{1.498745in}{2.134166in}}{\pgfqpoint{1.498745in}{2.142402in}}%
\pgfpathcurveto{\pgfqpoint{1.498745in}{2.150639in}}{\pgfqpoint{1.495473in}{2.158539in}}{\pgfqpoint{1.489649in}{2.164363in}}%
\pgfpathcurveto{\pgfqpoint{1.483825in}{2.170186in}}{\pgfqpoint{1.475925in}{2.173459in}}{\pgfqpoint{1.467689in}{2.173459in}}%
\pgfpathcurveto{\pgfqpoint{1.459452in}{2.173459in}}{\pgfqpoint{1.451552in}{2.170186in}}{\pgfqpoint{1.445728in}{2.164363in}}%
\pgfpathcurveto{\pgfqpoint{1.439904in}{2.158539in}}{\pgfqpoint{1.436632in}{2.150639in}}{\pgfqpoint{1.436632in}{2.142402in}}%
\pgfpathcurveto{\pgfqpoint{1.436632in}{2.134166in}}{\pgfqpoint{1.439904in}{2.126266in}}{\pgfqpoint{1.445728in}{2.120442in}}%
\pgfpathcurveto{\pgfqpoint{1.451552in}{2.114618in}}{\pgfqpoint{1.459452in}{2.111346in}}{\pgfqpoint{1.467689in}{2.111346in}}%
\pgfpathclose%
\pgfusepath{stroke,fill}%
\end{pgfscope}%
\begin{pgfscope}%
\pgfpathrectangle{\pgfqpoint{0.100000in}{0.212622in}}{\pgfqpoint{3.696000in}{3.696000in}}%
\pgfusepath{clip}%
\pgfsetbuttcap%
\pgfsetroundjoin%
\definecolor{currentfill}{rgb}{0.121569,0.466667,0.705882}%
\pgfsetfillcolor{currentfill}%
\pgfsetfillopacity{0.417248}%
\pgfsetlinewidth{1.003750pt}%
\definecolor{currentstroke}{rgb}{0.121569,0.466667,0.705882}%
\pgfsetstrokecolor{currentstroke}%
\pgfsetstrokeopacity{0.417248}%
\pgfsetdash{}{0pt}%
\pgfpathmoveto{\pgfqpoint{1.467081in}{2.111374in}}%
\pgfpathcurveto{\pgfqpoint{1.475318in}{2.111374in}}{\pgfqpoint{1.483218in}{2.114646in}}{\pgfqpoint{1.489042in}{2.120470in}}%
\pgfpathcurveto{\pgfqpoint{1.494866in}{2.126294in}}{\pgfqpoint{1.498138in}{2.134194in}}{\pgfqpoint{1.498138in}{2.142430in}}%
\pgfpathcurveto{\pgfqpoint{1.498138in}{2.150667in}}{\pgfqpoint{1.494866in}{2.158567in}}{\pgfqpoint{1.489042in}{2.164391in}}%
\pgfpathcurveto{\pgfqpoint{1.483218in}{2.170214in}}{\pgfqpoint{1.475318in}{2.173487in}}{\pgfqpoint{1.467081in}{2.173487in}}%
\pgfpathcurveto{\pgfqpoint{1.458845in}{2.173487in}}{\pgfqpoint{1.450945in}{2.170214in}}{\pgfqpoint{1.445121in}{2.164391in}}%
\pgfpathcurveto{\pgfqpoint{1.439297in}{2.158567in}}{\pgfqpoint{1.436025in}{2.150667in}}{\pgfqpoint{1.436025in}{2.142430in}}%
\pgfpathcurveto{\pgfqpoint{1.436025in}{2.134194in}}{\pgfqpoint{1.439297in}{2.126294in}}{\pgfqpoint{1.445121in}{2.120470in}}%
\pgfpathcurveto{\pgfqpoint{1.450945in}{2.114646in}}{\pgfqpoint{1.458845in}{2.111374in}}{\pgfqpoint{1.467081in}{2.111374in}}%
\pgfpathclose%
\pgfusepath{stroke,fill}%
\end{pgfscope}%
\begin{pgfscope}%
\pgfpathrectangle{\pgfqpoint{0.100000in}{0.212622in}}{\pgfqpoint{3.696000in}{3.696000in}}%
\pgfusepath{clip}%
\pgfsetbuttcap%
\pgfsetroundjoin%
\definecolor{currentfill}{rgb}{0.121569,0.466667,0.705882}%
\pgfsetfillcolor{currentfill}%
\pgfsetfillopacity{0.417663}%
\pgfsetlinewidth{1.003750pt}%
\definecolor{currentstroke}{rgb}{0.121569,0.466667,0.705882}%
\pgfsetstrokecolor{currentstroke}%
\pgfsetstrokeopacity{0.417663}%
\pgfsetdash{}{0pt}%
\pgfpathmoveto{\pgfqpoint{2.506893in}{1.927462in}}%
\pgfpathcurveto{\pgfqpoint{2.515129in}{1.927462in}}{\pgfqpoint{2.523029in}{1.930734in}}{\pgfqpoint{2.528853in}{1.936558in}}%
\pgfpathcurveto{\pgfqpoint{2.534677in}{1.942382in}}{\pgfqpoint{2.537949in}{1.950282in}}{\pgfqpoint{2.537949in}{1.958518in}}%
\pgfpathcurveto{\pgfqpoint{2.537949in}{1.966754in}}{\pgfqpoint{2.534677in}{1.974655in}}{\pgfqpoint{2.528853in}{1.980478in}}%
\pgfpathcurveto{\pgfqpoint{2.523029in}{1.986302in}}{\pgfqpoint{2.515129in}{1.989575in}}{\pgfqpoint{2.506893in}{1.989575in}}%
\pgfpathcurveto{\pgfqpoint{2.498657in}{1.989575in}}{\pgfqpoint{2.490757in}{1.986302in}}{\pgfqpoint{2.484933in}{1.980478in}}%
\pgfpathcurveto{\pgfqpoint{2.479109in}{1.974655in}}{\pgfqpoint{2.475836in}{1.966754in}}{\pgfqpoint{2.475836in}{1.958518in}}%
\pgfpathcurveto{\pgfqpoint{2.475836in}{1.950282in}}{\pgfqpoint{2.479109in}{1.942382in}}{\pgfqpoint{2.484933in}{1.936558in}}%
\pgfpathcurveto{\pgfqpoint{2.490757in}{1.930734in}}{\pgfqpoint{2.498657in}{1.927462in}}{\pgfqpoint{2.506893in}{1.927462in}}%
\pgfpathclose%
\pgfusepath{stroke,fill}%
\end{pgfscope}%
\begin{pgfscope}%
\pgfpathrectangle{\pgfqpoint{0.100000in}{0.212622in}}{\pgfqpoint{3.696000in}{3.696000in}}%
\pgfusepath{clip}%
\pgfsetbuttcap%
\pgfsetroundjoin%
\definecolor{currentfill}{rgb}{0.121569,0.466667,0.705882}%
\pgfsetfillcolor{currentfill}%
\pgfsetfillopacity{0.417733}%
\pgfsetlinewidth{1.003750pt}%
\definecolor{currentstroke}{rgb}{0.121569,0.466667,0.705882}%
\pgfsetstrokecolor{currentstroke}%
\pgfsetstrokeopacity{0.417733}%
\pgfsetdash{}{0pt}%
\pgfpathmoveto{\pgfqpoint{1.466207in}{2.111405in}}%
\pgfpathcurveto{\pgfqpoint{1.474443in}{2.111405in}}{\pgfqpoint{1.482343in}{2.114677in}}{\pgfqpoint{1.488167in}{2.120501in}}%
\pgfpathcurveto{\pgfqpoint{1.493991in}{2.126325in}}{\pgfqpoint{1.497263in}{2.134225in}}{\pgfqpoint{1.497263in}{2.142461in}}%
\pgfpathcurveto{\pgfqpoint{1.497263in}{2.150698in}}{\pgfqpoint{1.493991in}{2.158598in}}{\pgfqpoint{1.488167in}{2.164422in}}%
\pgfpathcurveto{\pgfqpoint{1.482343in}{2.170246in}}{\pgfqpoint{1.474443in}{2.173518in}}{\pgfqpoint{1.466207in}{2.173518in}}%
\pgfpathcurveto{\pgfqpoint{1.457970in}{2.173518in}}{\pgfqpoint{1.450070in}{2.170246in}}{\pgfqpoint{1.444246in}{2.164422in}}%
\pgfpathcurveto{\pgfqpoint{1.438422in}{2.158598in}}{\pgfqpoint{1.435150in}{2.150698in}}{\pgfqpoint{1.435150in}{2.142461in}}%
\pgfpathcurveto{\pgfqpoint{1.435150in}{2.134225in}}{\pgfqpoint{1.438422in}{2.126325in}}{\pgfqpoint{1.444246in}{2.120501in}}%
\pgfpathcurveto{\pgfqpoint{1.450070in}{2.114677in}}{\pgfqpoint{1.457970in}{2.111405in}}{\pgfqpoint{1.466207in}{2.111405in}}%
\pgfpathclose%
\pgfusepath{stroke,fill}%
\end{pgfscope}%
\begin{pgfscope}%
\pgfpathrectangle{\pgfqpoint{0.100000in}{0.212622in}}{\pgfqpoint{3.696000in}{3.696000in}}%
\pgfusepath{clip}%
\pgfsetbuttcap%
\pgfsetroundjoin%
\definecolor{currentfill}{rgb}{0.121569,0.466667,0.705882}%
\pgfsetfillcolor{currentfill}%
\pgfsetfillopacity{0.418617}%
\pgfsetlinewidth{1.003750pt}%
\definecolor{currentstroke}{rgb}{0.121569,0.466667,0.705882}%
\pgfsetstrokecolor{currentstroke}%
\pgfsetstrokeopacity{0.418617}%
\pgfsetdash{}{0pt}%
\pgfpathmoveto{\pgfqpoint{1.464770in}{2.111344in}}%
\pgfpathcurveto{\pgfqpoint{1.473006in}{2.111344in}}{\pgfqpoint{1.480906in}{2.114616in}}{\pgfqpoint{1.486730in}{2.120440in}}%
\pgfpathcurveto{\pgfqpoint{1.492554in}{2.126264in}}{\pgfqpoint{1.495826in}{2.134164in}}{\pgfqpoint{1.495826in}{2.142400in}}%
\pgfpathcurveto{\pgfqpoint{1.495826in}{2.150636in}}{\pgfqpoint{1.492554in}{2.158536in}}{\pgfqpoint{1.486730in}{2.164360in}}%
\pgfpathcurveto{\pgfqpoint{1.480906in}{2.170184in}}{\pgfqpoint{1.473006in}{2.173457in}}{\pgfqpoint{1.464770in}{2.173457in}}%
\pgfpathcurveto{\pgfqpoint{1.456533in}{2.173457in}}{\pgfqpoint{1.448633in}{2.170184in}}{\pgfqpoint{1.442809in}{2.164360in}}%
\pgfpathcurveto{\pgfqpoint{1.436986in}{2.158536in}}{\pgfqpoint{1.433713in}{2.150636in}}{\pgfqpoint{1.433713in}{2.142400in}}%
\pgfpathcurveto{\pgfqpoint{1.433713in}{2.134164in}}{\pgfqpoint{1.436986in}{2.126264in}}{\pgfqpoint{1.442809in}{2.120440in}}%
\pgfpathcurveto{\pgfqpoint{1.448633in}{2.114616in}}{\pgfqpoint{1.456533in}{2.111344in}}{\pgfqpoint{1.464770in}{2.111344in}}%
\pgfpathclose%
\pgfusepath{stroke,fill}%
\end{pgfscope}%
\begin{pgfscope}%
\pgfpathrectangle{\pgfqpoint{0.100000in}{0.212622in}}{\pgfqpoint{3.696000in}{3.696000in}}%
\pgfusepath{clip}%
\pgfsetbuttcap%
\pgfsetroundjoin%
\definecolor{currentfill}{rgb}{0.121569,0.466667,0.705882}%
\pgfsetfillcolor{currentfill}%
\pgfsetfillopacity{0.418831}%
\pgfsetlinewidth{1.003750pt}%
\definecolor{currentstroke}{rgb}{0.121569,0.466667,0.705882}%
\pgfsetstrokecolor{currentstroke}%
\pgfsetstrokeopacity{0.418831}%
\pgfsetdash{}{0pt}%
\pgfpathmoveto{\pgfqpoint{2.518536in}{1.923617in}}%
\pgfpathcurveto{\pgfqpoint{2.526772in}{1.923617in}}{\pgfqpoint{2.534672in}{1.926889in}}{\pgfqpoint{2.540496in}{1.932713in}}%
\pgfpathcurveto{\pgfqpoint{2.546320in}{1.938537in}}{\pgfqpoint{2.549592in}{1.946437in}}{\pgfqpoint{2.549592in}{1.954674in}}%
\pgfpathcurveto{\pgfqpoint{2.549592in}{1.962910in}}{\pgfqpoint{2.546320in}{1.970810in}}{\pgfqpoint{2.540496in}{1.976634in}}%
\pgfpathcurveto{\pgfqpoint{2.534672in}{1.982458in}}{\pgfqpoint{2.526772in}{1.985730in}}{\pgfqpoint{2.518536in}{1.985730in}}%
\pgfpathcurveto{\pgfqpoint{2.510299in}{1.985730in}}{\pgfqpoint{2.502399in}{1.982458in}}{\pgfqpoint{2.496575in}{1.976634in}}%
\pgfpathcurveto{\pgfqpoint{2.490751in}{1.970810in}}{\pgfqpoint{2.487479in}{1.962910in}}{\pgfqpoint{2.487479in}{1.954674in}}%
\pgfpathcurveto{\pgfqpoint{2.487479in}{1.946437in}}{\pgfqpoint{2.490751in}{1.938537in}}{\pgfqpoint{2.496575in}{1.932713in}}%
\pgfpathcurveto{\pgfqpoint{2.502399in}{1.926889in}}{\pgfqpoint{2.510299in}{1.923617in}}{\pgfqpoint{2.518536in}{1.923617in}}%
\pgfpathclose%
\pgfusepath{stroke,fill}%
\end{pgfscope}%
\begin{pgfscope}%
\pgfpathrectangle{\pgfqpoint{0.100000in}{0.212622in}}{\pgfqpoint{3.696000in}{3.696000in}}%
\pgfusepath{clip}%
\pgfsetbuttcap%
\pgfsetroundjoin%
\definecolor{currentfill}{rgb}{0.121569,0.466667,0.705882}%
\pgfsetfillcolor{currentfill}%
\pgfsetfillopacity{0.419337}%
\pgfsetlinewidth{1.003750pt}%
\definecolor{currentstroke}{rgb}{0.121569,0.466667,0.705882}%
\pgfsetstrokecolor{currentstroke}%
\pgfsetstrokeopacity{0.419337}%
\pgfsetdash{}{0pt}%
\pgfpathmoveto{\pgfqpoint{1.462884in}{2.111455in}}%
\pgfpathcurveto{\pgfqpoint{1.471121in}{2.111455in}}{\pgfqpoint{1.479021in}{2.114727in}}{\pgfqpoint{1.484845in}{2.120551in}}%
\pgfpathcurveto{\pgfqpoint{1.490669in}{2.126375in}}{\pgfqpoint{1.493941in}{2.134275in}}{\pgfqpoint{1.493941in}{2.142512in}}%
\pgfpathcurveto{\pgfqpoint{1.493941in}{2.150748in}}{\pgfqpoint{1.490669in}{2.158648in}}{\pgfqpoint{1.484845in}{2.164472in}}%
\pgfpathcurveto{\pgfqpoint{1.479021in}{2.170296in}}{\pgfqpoint{1.471121in}{2.173568in}}{\pgfqpoint{1.462884in}{2.173568in}}%
\pgfpathcurveto{\pgfqpoint{1.454648in}{2.173568in}}{\pgfqpoint{1.446748in}{2.170296in}}{\pgfqpoint{1.440924in}{2.164472in}}%
\pgfpathcurveto{\pgfqpoint{1.435100in}{2.158648in}}{\pgfqpoint{1.431828in}{2.150748in}}{\pgfqpoint{1.431828in}{2.142512in}}%
\pgfpathcurveto{\pgfqpoint{1.431828in}{2.134275in}}{\pgfqpoint{1.435100in}{2.126375in}}{\pgfqpoint{1.440924in}{2.120551in}}%
\pgfpathcurveto{\pgfqpoint{1.446748in}{2.114727in}}{\pgfqpoint{1.454648in}{2.111455in}}{\pgfqpoint{1.462884in}{2.111455in}}%
\pgfpathclose%
\pgfusepath{stroke,fill}%
\end{pgfscope}%
\begin{pgfscope}%
\pgfpathrectangle{\pgfqpoint{0.100000in}{0.212622in}}{\pgfqpoint{3.696000in}{3.696000in}}%
\pgfusepath{clip}%
\pgfsetbuttcap%
\pgfsetroundjoin%
\definecolor{currentfill}{rgb}{0.121569,0.466667,0.705882}%
\pgfsetfillcolor{currentfill}%
\pgfsetfillopacity{0.419380}%
\pgfsetlinewidth{1.003750pt}%
\definecolor{currentstroke}{rgb}{0.121569,0.466667,0.705882}%
\pgfsetstrokecolor{currentstroke}%
\pgfsetstrokeopacity{0.419380}%
\pgfsetdash{}{0pt}%
\pgfpathmoveto{\pgfqpoint{2.532498in}{1.919668in}}%
\pgfpathcurveto{\pgfqpoint{2.540735in}{1.919668in}}{\pgfqpoint{2.548635in}{1.922940in}}{\pgfqpoint{2.554459in}{1.928764in}}%
\pgfpathcurveto{\pgfqpoint{2.560283in}{1.934588in}}{\pgfqpoint{2.563555in}{1.942488in}}{\pgfqpoint{2.563555in}{1.950724in}}%
\pgfpathcurveto{\pgfqpoint{2.563555in}{1.958961in}}{\pgfqpoint{2.560283in}{1.966861in}}{\pgfqpoint{2.554459in}{1.972685in}}%
\pgfpathcurveto{\pgfqpoint{2.548635in}{1.978509in}}{\pgfqpoint{2.540735in}{1.981781in}}{\pgfqpoint{2.532498in}{1.981781in}}%
\pgfpathcurveto{\pgfqpoint{2.524262in}{1.981781in}}{\pgfqpoint{2.516362in}{1.978509in}}{\pgfqpoint{2.510538in}{1.972685in}}%
\pgfpathcurveto{\pgfqpoint{2.504714in}{1.966861in}}{\pgfqpoint{2.501442in}{1.958961in}}{\pgfqpoint{2.501442in}{1.950724in}}%
\pgfpathcurveto{\pgfqpoint{2.501442in}{1.942488in}}{\pgfqpoint{2.504714in}{1.934588in}}{\pgfqpoint{2.510538in}{1.928764in}}%
\pgfpathcurveto{\pgfqpoint{2.516362in}{1.922940in}}{\pgfqpoint{2.524262in}{1.919668in}}{\pgfqpoint{2.532498in}{1.919668in}}%
\pgfpathclose%
\pgfusepath{stroke,fill}%
\end{pgfscope}%
\begin{pgfscope}%
\pgfpathrectangle{\pgfqpoint{0.100000in}{0.212622in}}{\pgfqpoint{3.696000in}{3.696000in}}%
\pgfusepath{clip}%
\pgfsetbuttcap%
\pgfsetroundjoin%
\definecolor{currentfill}{rgb}{0.121569,0.466667,0.705882}%
\pgfsetfillcolor{currentfill}%
\pgfsetfillopacity{0.419718}%
\pgfsetlinewidth{1.003750pt}%
\definecolor{currentstroke}{rgb}{0.121569,0.466667,0.705882}%
\pgfsetstrokecolor{currentstroke}%
\pgfsetstrokeopacity{0.419718}%
\pgfsetdash{}{0pt}%
\pgfpathmoveto{\pgfqpoint{1.462157in}{2.111410in}}%
\pgfpathcurveto{\pgfqpoint{1.470394in}{2.111410in}}{\pgfqpoint{1.478294in}{2.114683in}}{\pgfqpoint{1.484118in}{2.120507in}}%
\pgfpathcurveto{\pgfqpoint{1.489942in}{2.126331in}}{\pgfqpoint{1.493214in}{2.134231in}}{\pgfqpoint{1.493214in}{2.142467in}}%
\pgfpathcurveto{\pgfqpoint{1.493214in}{2.150703in}}{\pgfqpoint{1.489942in}{2.158603in}}{\pgfqpoint{1.484118in}{2.164427in}}%
\pgfpathcurveto{\pgfqpoint{1.478294in}{2.170251in}}{\pgfqpoint{1.470394in}{2.173523in}}{\pgfqpoint{1.462157in}{2.173523in}}%
\pgfpathcurveto{\pgfqpoint{1.453921in}{2.173523in}}{\pgfqpoint{1.446021in}{2.170251in}}{\pgfqpoint{1.440197in}{2.164427in}}%
\pgfpathcurveto{\pgfqpoint{1.434373in}{2.158603in}}{\pgfqpoint{1.431101in}{2.150703in}}{\pgfqpoint{1.431101in}{2.142467in}}%
\pgfpathcurveto{\pgfqpoint{1.431101in}{2.134231in}}{\pgfqpoint{1.434373in}{2.126331in}}{\pgfqpoint{1.440197in}{2.120507in}}%
\pgfpathcurveto{\pgfqpoint{1.446021in}{2.114683in}}{\pgfqpoint{1.453921in}{2.111410in}}{\pgfqpoint{1.462157in}{2.111410in}}%
\pgfpathclose%
\pgfusepath{stroke,fill}%
\end{pgfscope}%
\begin{pgfscope}%
\pgfpathrectangle{\pgfqpoint{0.100000in}{0.212622in}}{\pgfqpoint{3.696000in}{3.696000in}}%
\pgfusepath{clip}%
\pgfsetbuttcap%
\pgfsetroundjoin%
\definecolor{currentfill}{rgb}{0.121569,0.466667,0.705882}%
\pgfsetfillcolor{currentfill}%
\pgfsetfillopacity{0.420213}%
\pgfsetlinewidth{1.003750pt}%
\definecolor{currentstroke}{rgb}{0.121569,0.466667,0.705882}%
\pgfsetstrokecolor{currentstroke}%
\pgfsetstrokeopacity{0.420213}%
\pgfsetdash{}{0pt}%
\pgfpathmoveto{\pgfqpoint{2.539406in}{1.917827in}}%
\pgfpathcurveto{\pgfqpoint{2.547642in}{1.917827in}}{\pgfqpoint{2.555542in}{1.921100in}}{\pgfqpoint{2.561366in}{1.926924in}}%
\pgfpathcurveto{\pgfqpoint{2.567190in}{1.932747in}}{\pgfqpoint{2.570462in}{1.940647in}}{\pgfqpoint{2.570462in}{1.948884in}}%
\pgfpathcurveto{\pgfqpoint{2.570462in}{1.957120in}}{\pgfqpoint{2.567190in}{1.965020in}}{\pgfqpoint{2.561366in}{1.970844in}}%
\pgfpathcurveto{\pgfqpoint{2.555542in}{1.976668in}}{\pgfqpoint{2.547642in}{1.979940in}}{\pgfqpoint{2.539406in}{1.979940in}}%
\pgfpathcurveto{\pgfqpoint{2.531169in}{1.979940in}}{\pgfqpoint{2.523269in}{1.976668in}}{\pgfqpoint{2.517445in}{1.970844in}}%
\pgfpathcurveto{\pgfqpoint{2.511621in}{1.965020in}}{\pgfqpoint{2.508349in}{1.957120in}}{\pgfqpoint{2.508349in}{1.948884in}}%
\pgfpathcurveto{\pgfqpoint{2.508349in}{1.940647in}}{\pgfqpoint{2.511621in}{1.932747in}}{\pgfqpoint{2.517445in}{1.926924in}}%
\pgfpathcurveto{\pgfqpoint{2.523269in}{1.921100in}}{\pgfqpoint{2.531169in}{1.917827in}}{\pgfqpoint{2.539406in}{1.917827in}}%
\pgfpathclose%
\pgfusepath{stroke,fill}%
\end{pgfscope}%
\begin{pgfscope}%
\pgfpathrectangle{\pgfqpoint{0.100000in}{0.212622in}}{\pgfqpoint{3.696000in}{3.696000in}}%
\pgfusepath{clip}%
\pgfsetbuttcap%
\pgfsetroundjoin%
\definecolor{currentfill}{rgb}{0.121569,0.466667,0.705882}%
\pgfsetfillcolor{currentfill}%
\pgfsetfillopacity{0.420416}%
\pgfsetlinewidth{1.003750pt}%
\definecolor{currentstroke}{rgb}{0.121569,0.466667,0.705882}%
\pgfsetstrokecolor{currentstroke}%
\pgfsetstrokeopacity{0.420416}%
\pgfsetdash{}{0pt}%
\pgfpathmoveto{\pgfqpoint{1.460715in}{2.111472in}}%
\pgfpathcurveto{\pgfqpoint{1.468951in}{2.111472in}}{\pgfqpoint{1.476851in}{2.114744in}}{\pgfqpoint{1.482675in}{2.120568in}}%
\pgfpathcurveto{\pgfqpoint{1.488499in}{2.126392in}}{\pgfqpoint{1.491771in}{2.134292in}}{\pgfqpoint{1.491771in}{2.142528in}}%
\pgfpathcurveto{\pgfqpoint{1.491771in}{2.150764in}}{\pgfqpoint{1.488499in}{2.158664in}}{\pgfqpoint{1.482675in}{2.164488in}}%
\pgfpathcurveto{\pgfqpoint{1.476851in}{2.170312in}}{\pgfqpoint{1.468951in}{2.173585in}}{\pgfqpoint{1.460715in}{2.173585in}}%
\pgfpathcurveto{\pgfqpoint{1.452478in}{2.173585in}}{\pgfqpoint{1.444578in}{2.170312in}}{\pgfqpoint{1.438754in}{2.164488in}}%
\pgfpathcurveto{\pgfqpoint{1.432931in}{2.158664in}}{\pgfqpoint{1.429658in}{2.150764in}}{\pgfqpoint{1.429658in}{2.142528in}}%
\pgfpathcurveto{\pgfqpoint{1.429658in}{2.134292in}}{\pgfqpoint{1.432931in}{2.126392in}}{\pgfqpoint{1.438754in}{2.120568in}}%
\pgfpathcurveto{\pgfqpoint{1.444578in}{2.114744in}}{\pgfqpoint{1.452478in}{2.111472in}}{\pgfqpoint{1.460715in}{2.111472in}}%
\pgfpathclose%
\pgfusepath{stroke,fill}%
\end{pgfscope}%
\begin{pgfscope}%
\pgfpathrectangle{\pgfqpoint{0.100000in}{0.212622in}}{\pgfqpoint{3.696000in}{3.696000in}}%
\pgfusepath{clip}%
\pgfsetbuttcap%
\pgfsetroundjoin%
\definecolor{currentfill}{rgb}{0.121569,0.466667,0.705882}%
\pgfsetfillcolor{currentfill}%
\pgfsetfillopacity{0.420884}%
\pgfsetlinewidth{1.003750pt}%
\definecolor{currentstroke}{rgb}{0.121569,0.466667,0.705882}%
\pgfsetstrokecolor{currentstroke}%
\pgfsetstrokeopacity{0.420884}%
\pgfsetdash{}{0pt}%
\pgfpathmoveto{\pgfqpoint{2.547104in}{1.915753in}}%
\pgfpathcurveto{\pgfqpoint{2.555340in}{1.915753in}}{\pgfqpoint{2.563241in}{1.919026in}}{\pgfqpoint{2.569064in}{1.924850in}}%
\pgfpathcurveto{\pgfqpoint{2.574888in}{1.930674in}}{\pgfqpoint{2.578161in}{1.938574in}}{\pgfqpoint{2.578161in}{1.946810in}}%
\pgfpathcurveto{\pgfqpoint{2.578161in}{1.955046in}}{\pgfqpoint{2.574888in}{1.962946in}}{\pgfqpoint{2.569064in}{1.968770in}}%
\pgfpathcurveto{\pgfqpoint{2.563241in}{1.974594in}}{\pgfqpoint{2.555340in}{1.977866in}}{\pgfqpoint{2.547104in}{1.977866in}}%
\pgfpathcurveto{\pgfqpoint{2.538868in}{1.977866in}}{\pgfqpoint{2.530968in}{1.974594in}}{\pgfqpoint{2.525144in}{1.968770in}}%
\pgfpathcurveto{\pgfqpoint{2.519320in}{1.962946in}}{\pgfqpoint{2.516048in}{1.955046in}}{\pgfqpoint{2.516048in}{1.946810in}}%
\pgfpathcurveto{\pgfqpoint{2.516048in}{1.938574in}}{\pgfqpoint{2.519320in}{1.930674in}}{\pgfqpoint{2.525144in}{1.924850in}}%
\pgfpathcurveto{\pgfqpoint{2.530968in}{1.919026in}}{\pgfqpoint{2.538868in}{1.915753in}}{\pgfqpoint{2.547104in}{1.915753in}}%
\pgfpathclose%
\pgfusepath{stroke,fill}%
\end{pgfscope}%
\begin{pgfscope}%
\pgfpathrectangle{\pgfqpoint{0.100000in}{0.212622in}}{\pgfqpoint{3.696000in}{3.696000in}}%
\pgfusepath{clip}%
\pgfsetbuttcap%
\pgfsetroundjoin%
\definecolor{currentfill}{rgb}{0.121569,0.466667,0.705882}%
\pgfsetfillcolor{currentfill}%
\pgfsetfillopacity{0.421035}%
\pgfsetlinewidth{1.003750pt}%
\definecolor{currentstroke}{rgb}{0.121569,0.466667,0.705882}%
\pgfsetstrokecolor{currentstroke}%
\pgfsetstrokeopacity{0.421035}%
\pgfsetdash{}{0pt}%
\pgfpathmoveto{\pgfqpoint{1.459298in}{2.111569in}}%
\pgfpathcurveto{\pgfqpoint{1.467534in}{2.111569in}}{\pgfqpoint{1.475434in}{2.114841in}}{\pgfqpoint{1.481258in}{2.120665in}}%
\pgfpathcurveto{\pgfqpoint{1.487082in}{2.126489in}}{\pgfqpoint{1.490355in}{2.134389in}}{\pgfqpoint{1.490355in}{2.142625in}}%
\pgfpathcurveto{\pgfqpoint{1.490355in}{2.150862in}}{\pgfqpoint{1.487082in}{2.158762in}}{\pgfqpoint{1.481258in}{2.164586in}}%
\pgfpathcurveto{\pgfqpoint{1.475434in}{2.170410in}}{\pgfqpoint{1.467534in}{2.173682in}}{\pgfqpoint{1.459298in}{2.173682in}}%
\pgfpathcurveto{\pgfqpoint{1.451062in}{2.173682in}}{\pgfqpoint{1.443162in}{2.170410in}}{\pgfqpoint{1.437338in}{2.164586in}}%
\pgfpathcurveto{\pgfqpoint{1.431514in}{2.158762in}}{\pgfqpoint{1.428242in}{2.150862in}}{\pgfqpoint{1.428242in}{2.142625in}}%
\pgfpathcurveto{\pgfqpoint{1.428242in}{2.134389in}}{\pgfqpoint{1.431514in}{2.126489in}}{\pgfqpoint{1.437338in}{2.120665in}}%
\pgfpathcurveto{\pgfqpoint{1.443162in}{2.114841in}}{\pgfqpoint{1.451062in}{2.111569in}}{\pgfqpoint{1.459298in}{2.111569in}}%
\pgfpathclose%
\pgfusepath{stroke,fill}%
\end{pgfscope}%
\begin{pgfscope}%
\pgfpathrectangle{\pgfqpoint{0.100000in}{0.212622in}}{\pgfqpoint{3.696000in}{3.696000in}}%
\pgfusepath{clip}%
\pgfsetbuttcap%
\pgfsetroundjoin%
\definecolor{currentfill}{rgb}{0.121569,0.466667,0.705882}%
\pgfsetfillcolor{currentfill}%
\pgfsetfillopacity{0.421469}%
\pgfsetlinewidth{1.003750pt}%
\definecolor{currentstroke}{rgb}{0.121569,0.466667,0.705882}%
\pgfsetstrokecolor{currentstroke}%
\pgfsetstrokeopacity{0.421469}%
\pgfsetdash{}{0pt}%
\pgfpathmoveto{\pgfqpoint{1.458606in}{2.111600in}}%
\pgfpathcurveto{\pgfqpoint{1.466843in}{2.111600in}}{\pgfqpoint{1.474743in}{2.114872in}}{\pgfqpoint{1.480567in}{2.120696in}}%
\pgfpathcurveto{\pgfqpoint{1.486391in}{2.126520in}}{\pgfqpoint{1.489663in}{2.134420in}}{\pgfqpoint{1.489663in}{2.142656in}}%
\pgfpathcurveto{\pgfqpoint{1.489663in}{2.150892in}}{\pgfqpoint{1.486391in}{2.158792in}}{\pgfqpoint{1.480567in}{2.164616in}}%
\pgfpathcurveto{\pgfqpoint{1.474743in}{2.170440in}}{\pgfqpoint{1.466843in}{2.173713in}}{\pgfqpoint{1.458606in}{2.173713in}}%
\pgfpathcurveto{\pgfqpoint{1.450370in}{2.173713in}}{\pgfqpoint{1.442470in}{2.170440in}}{\pgfqpoint{1.436646in}{2.164616in}}%
\pgfpathcurveto{\pgfqpoint{1.430822in}{2.158792in}}{\pgfqpoint{1.427550in}{2.150892in}}{\pgfqpoint{1.427550in}{2.142656in}}%
\pgfpathcurveto{\pgfqpoint{1.427550in}{2.134420in}}{\pgfqpoint{1.430822in}{2.126520in}}{\pgfqpoint{1.436646in}{2.120696in}}%
\pgfpathcurveto{\pgfqpoint{1.442470in}{2.114872in}}{\pgfqpoint{1.450370in}{2.111600in}}{\pgfqpoint{1.458606in}{2.111600in}}%
\pgfpathclose%
\pgfusepath{stroke,fill}%
\end{pgfscope}%
\begin{pgfscope}%
\pgfpathrectangle{\pgfqpoint{0.100000in}{0.212622in}}{\pgfqpoint{3.696000in}{3.696000in}}%
\pgfusepath{clip}%
\pgfsetbuttcap%
\pgfsetroundjoin%
\definecolor{currentfill}{rgb}{0.121569,0.466667,0.705882}%
\pgfsetfillcolor{currentfill}%
\pgfsetfillopacity{0.421516}%
\pgfsetlinewidth{1.003750pt}%
\definecolor{currentstroke}{rgb}{0.121569,0.466667,0.705882}%
\pgfsetstrokecolor{currentstroke}%
\pgfsetstrokeopacity{0.421516}%
\pgfsetdash{}{0pt}%
\pgfpathmoveto{\pgfqpoint{2.550885in}{1.915039in}}%
\pgfpathcurveto{\pgfqpoint{2.559122in}{1.915039in}}{\pgfqpoint{2.567022in}{1.918312in}}{\pgfqpoint{2.572846in}{1.924135in}}%
\pgfpathcurveto{\pgfqpoint{2.578670in}{1.929959in}}{\pgfqpoint{2.581942in}{1.937859in}}{\pgfqpoint{2.581942in}{1.946096in}}%
\pgfpathcurveto{\pgfqpoint{2.581942in}{1.954332in}}{\pgfqpoint{2.578670in}{1.962232in}}{\pgfqpoint{2.572846in}{1.968056in}}%
\pgfpathcurveto{\pgfqpoint{2.567022in}{1.973880in}}{\pgfqpoint{2.559122in}{1.977152in}}{\pgfqpoint{2.550885in}{1.977152in}}%
\pgfpathcurveto{\pgfqpoint{2.542649in}{1.977152in}}{\pgfqpoint{2.534749in}{1.973880in}}{\pgfqpoint{2.528925in}{1.968056in}}%
\pgfpathcurveto{\pgfqpoint{2.523101in}{1.962232in}}{\pgfqpoint{2.519829in}{1.954332in}}{\pgfqpoint{2.519829in}{1.946096in}}%
\pgfpathcurveto{\pgfqpoint{2.519829in}{1.937859in}}{\pgfqpoint{2.523101in}{1.929959in}}{\pgfqpoint{2.528925in}{1.924135in}}%
\pgfpathcurveto{\pgfqpoint{2.534749in}{1.918312in}}{\pgfqpoint{2.542649in}{1.915039in}}{\pgfqpoint{2.550885in}{1.915039in}}%
\pgfpathclose%
\pgfusepath{stroke,fill}%
\end{pgfscope}%
\begin{pgfscope}%
\pgfpathrectangle{\pgfqpoint{0.100000in}{0.212622in}}{\pgfqpoint{3.696000in}{3.696000in}}%
\pgfusepath{clip}%
\pgfsetbuttcap%
\pgfsetroundjoin%
\definecolor{currentfill}{rgb}{0.121569,0.466667,0.705882}%
\pgfsetfillcolor{currentfill}%
\pgfsetfillopacity{0.421679}%
\pgfsetlinewidth{1.003750pt}%
\definecolor{currentstroke}{rgb}{0.121569,0.466667,0.705882}%
\pgfsetstrokecolor{currentstroke}%
\pgfsetstrokeopacity{0.421679}%
\pgfsetdash{}{0pt}%
\pgfpathmoveto{\pgfqpoint{1.458089in}{2.111637in}}%
\pgfpathcurveto{\pgfqpoint{1.466325in}{2.111637in}}{\pgfqpoint{1.474225in}{2.114909in}}{\pgfqpoint{1.480049in}{2.120733in}}%
\pgfpathcurveto{\pgfqpoint{1.485873in}{2.126557in}}{\pgfqpoint{1.489145in}{2.134457in}}{\pgfqpoint{1.489145in}{2.142693in}}%
\pgfpathcurveto{\pgfqpoint{1.489145in}{2.150930in}}{\pgfqpoint{1.485873in}{2.158830in}}{\pgfqpoint{1.480049in}{2.164654in}}%
\pgfpathcurveto{\pgfqpoint{1.474225in}{2.170478in}}{\pgfqpoint{1.466325in}{2.173750in}}{\pgfqpoint{1.458089in}{2.173750in}}%
\pgfpathcurveto{\pgfqpoint{1.449853in}{2.173750in}}{\pgfqpoint{1.441953in}{2.170478in}}{\pgfqpoint{1.436129in}{2.164654in}}%
\pgfpathcurveto{\pgfqpoint{1.430305in}{2.158830in}}{\pgfqpoint{1.427032in}{2.150930in}}{\pgfqpoint{1.427032in}{2.142693in}}%
\pgfpathcurveto{\pgfqpoint{1.427032in}{2.134457in}}{\pgfqpoint{1.430305in}{2.126557in}}{\pgfqpoint{1.436129in}{2.120733in}}%
\pgfpathcurveto{\pgfqpoint{1.441953in}{2.114909in}}{\pgfqpoint{1.449853in}{2.111637in}}{\pgfqpoint{1.458089in}{2.111637in}}%
\pgfpathclose%
\pgfusepath{stroke,fill}%
\end{pgfscope}%
\begin{pgfscope}%
\pgfpathrectangle{\pgfqpoint{0.100000in}{0.212622in}}{\pgfqpoint{3.696000in}{3.696000in}}%
\pgfusepath{clip}%
\pgfsetbuttcap%
\pgfsetroundjoin%
\definecolor{currentfill}{rgb}{0.121569,0.466667,0.705882}%
\pgfsetfillcolor{currentfill}%
\pgfsetfillopacity{0.422089}%
\pgfsetlinewidth{1.003750pt}%
\definecolor{currentstroke}{rgb}{0.121569,0.466667,0.705882}%
\pgfsetstrokecolor{currentstroke}%
\pgfsetstrokeopacity{0.422089}%
\pgfsetdash{}{0pt}%
\pgfpathmoveto{\pgfqpoint{1.457420in}{2.111650in}}%
\pgfpathcurveto{\pgfqpoint{1.465656in}{2.111650in}}{\pgfqpoint{1.473556in}{2.114922in}}{\pgfqpoint{1.479380in}{2.120746in}}%
\pgfpathcurveto{\pgfqpoint{1.485204in}{2.126570in}}{\pgfqpoint{1.488476in}{2.134470in}}{\pgfqpoint{1.488476in}{2.142706in}}%
\pgfpathcurveto{\pgfqpoint{1.488476in}{2.150943in}}{\pgfqpoint{1.485204in}{2.158843in}}{\pgfqpoint{1.479380in}{2.164667in}}%
\pgfpathcurveto{\pgfqpoint{1.473556in}{2.170490in}}{\pgfqpoint{1.465656in}{2.173763in}}{\pgfqpoint{1.457420in}{2.173763in}}%
\pgfpathcurveto{\pgfqpoint{1.449183in}{2.173763in}}{\pgfqpoint{1.441283in}{2.170490in}}{\pgfqpoint{1.435459in}{2.164667in}}%
\pgfpathcurveto{\pgfqpoint{1.429635in}{2.158843in}}{\pgfqpoint{1.426363in}{2.150943in}}{\pgfqpoint{1.426363in}{2.142706in}}%
\pgfpathcurveto{\pgfqpoint{1.426363in}{2.134470in}}{\pgfqpoint{1.429635in}{2.126570in}}{\pgfqpoint{1.435459in}{2.120746in}}%
\pgfpathcurveto{\pgfqpoint{1.441283in}{2.114922in}}{\pgfqpoint{1.449183in}{2.111650in}}{\pgfqpoint{1.457420in}{2.111650in}}%
\pgfpathclose%
\pgfusepath{stroke,fill}%
\end{pgfscope}%
\begin{pgfscope}%
\pgfpathrectangle{\pgfqpoint{0.100000in}{0.212622in}}{\pgfqpoint{3.696000in}{3.696000in}}%
\pgfusepath{clip}%
\pgfsetbuttcap%
\pgfsetroundjoin%
\definecolor{currentfill}{rgb}{0.121569,0.466667,0.705882}%
\pgfsetfillcolor{currentfill}%
\pgfsetfillopacity{0.422201}%
\pgfsetlinewidth{1.003750pt}%
\definecolor{currentstroke}{rgb}{0.121569,0.466667,0.705882}%
\pgfsetstrokecolor{currentstroke}%
\pgfsetstrokeopacity{0.422201}%
\pgfsetdash{}{0pt}%
\pgfpathmoveto{\pgfqpoint{2.556724in}{1.913468in}}%
\pgfpathcurveto{\pgfqpoint{2.564961in}{1.913468in}}{\pgfqpoint{2.572861in}{1.916740in}}{\pgfqpoint{2.578685in}{1.922564in}}%
\pgfpathcurveto{\pgfqpoint{2.584509in}{1.928388in}}{\pgfqpoint{2.587781in}{1.936288in}}{\pgfqpoint{2.587781in}{1.944525in}}%
\pgfpathcurveto{\pgfqpoint{2.587781in}{1.952761in}}{\pgfqpoint{2.584509in}{1.960661in}}{\pgfqpoint{2.578685in}{1.966485in}}%
\pgfpathcurveto{\pgfqpoint{2.572861in}{1.972309in}}{\pgfqpoint{2.564961in}{1.975581in}}{\pgfqpoint{2.556724in}{1.975581in}}%
\pgfpathcurveto{\pgfqpoint{2.548488in}{1.975581in}}{\pgfqpoint{2.540588in}{1.972309in}}{\pgfqpoint{2.534764in}{1.966485in}}%
\pgfpathcurveto{\pgfqpoint{2.528940in}{1.960661in}}{\pgfqpoint{2.525668in}{1.952761in}}{\pgfqpoint{2.525668in}{1.944525in}}%
\pgfpathcurveto{\pgfqpoint{2.525668in}{1.936288in}}{\pgfqpoint{2.528940in}{1.928388in}}{\pgfqpoint{2.534764in}{1.922564in}}%
\pgfpathcurveto{\pgfqpoint{2.540588in}{1.916740in}}{\pgfqpoint{2.548488in}{1.913468in}}{\pgfqpoint{2.556724in}{1.913468in}}%
\pgfpathclose%
\pgfusepath{stroke,fill}%
\end{pgfscope}%
\begin{pgfscope}%
\pgfpathrectangle{\pgfqpoint{0.100000in}{0.212622in}}{\pgfqpoint{3.696000in}{3.696000in}}%
\pgfusepath{clip}%
\pgfsetbuttcap%
\pgfsetroundjoin%
\definecolor{currentfill}{rgb}{0.121569,0.466667,0.705882}%
\pgfsetfillcolor{currentfill}%
\pgfsetfillopacity{0.422553}%
\pgfsetlinewidth{1.003750pt}%
\definecolor{currentstroke}{rgb}{0.121569,0.466667,0.705882}%
\pgfsetstrokecolor{currentstroke}%
\pgfsetstrokeopacity{0.422553}%
\pgfsetdash{}{0pt}%
\pgfpathmoveto{\pgfqpoint{2.559957in}{1.912501in}}%
\pgfpathcurveto{\pgfqpoint{2.568193in}{1.912501in}}{\pgfqpoint{2.576094in}{1.915773in}}{\pgfqpoint{2.581917in}{1.921597in}}%
\pgfpathcurveto{\pgfqpoint{2.587741in}{1.927421in}}{\pgfqpoint{2.591014in}{1.935321in}}{\pgfqpoint{2.591014in}{1.943557in}}%
\pgfpathcurveto{\pgfqpoint{2.591014in}{1.951794in}}{\pgfqpoint{2.587741in}{1.959694in}}{\pgfqpoint{2.581917in}{1.965518in}}%
\pgfpathcurveto{\pgfqpoint{2.576094in}{1.971342in}}{\pgfqpoint{2.568193in}{1.974614in}}{\pgfqpoint{2.559957in}{1.974614in}}%
\pgfpathcurveto{\pgfqpoint{2.551721in}{1.974614in}}{\pgfqpoint{2.543821in}{1.971342in}}{\pgfqpoint{2.537997in}{1.965518in}}%
\pgfpathcurveto{\pgfqpoint{2.532173in}{1.959694in}}{\pgfqpoint{2.528901in}{1.951794in}}{\pgfqpoint{2.528901in}{1.943557in}}%
\pgfpathcurveto{\pgfqpoint{2.528901in}{1.935321in}}{\pgfqpoint{2.532173in}{1.927421in}}{\pgfqpoint{2.537997in}{1.921597in}}%
\pgfpathcurveto{\pgfqpoint{2.543821in}{1.915773in}}{\pgfqpoint{2.551721in}{1.912501in}}{\pgfqpoint{2.559957in}{1.912501in}}%
\pgfpathclose%
\pgfusepath{stroke,fill}%
\end{pgfscope}%
\begin{pgfscope}%
\pgfpathrectangle{\pgfqpoint{0.100000in}{0.212622in}}{\pgfqpoint{3.696000in}{3.696000in}}%
\pgfusepath{clip}%
\pgfsetbuttcap%
\pgfsetroundjoin%
\definecolor{currentfill}{rgb}{0.121569,0.466667,0.705882}%
\pgfsetfillcolor{currentfill}%
\pgfsetfillopacity{0.422815}%
\pgfsetlinewidth{1.003750pt}%
\definecolor{currentstroke}{rgb}{0.121569,0.466667,0.705882}%
\pgfsetstrokecolor{currentstroke}%
\pgfsetstrokeopacity{0.422815}%
\pgfsetdash{}{0pt}%
\pgfpathmoveto{\pgfqpoint{1.456069in}{2.111626in}}%
\pgfpathcurveto{\pgfqpoint{1.464305in}{2.111626in}}{\pgfqpoint{1.472205in}{2.114898in}}{\pgfqpoint{1.478029in}{2.120722in}}%
\pgfpathcurveto{\pgfqpoint{1.483853in}{2.126546in}}{\pgfqpoint{1.487125in}{2.134446in}}{\pgfqpoint{1.487125in}{2.142682in}}%
\pgfpathcurveto{\pgfqpoint{1.487125in}{2.150919in}}{\pgfqpoint{1.483853in}{2.158819in}}{\pgfqpoint{1.478029in}{2.164643in}}%
\pgfpathcurveto{\pgfqpoint{1.472205in}{2.170466in}}{\pgfqpoint{1.464305in}{2.173739in}}{\pgfqpoint{1.456069in}{2.173739in}}%
\pgfpathcurveto{\pgfqpoint{1.447832in}{2.173739in}}{\pgfqpoint{1.439932in}{2.170466in}}{\pgfqpoint{1.434108in}{2.164643in}}%
\pgfpathcurveto{\pgfqpoint{1.428284in}{2.158819in}}{\pgfqpoint{1.425012in}{2.150919in}}{\pgfqpoint{1.425012in}{2.142682in}}%
\pgfpathcurveto{\pgfqpoint{1.425012in}{2.134446in}}{\pgfqpoint{1.428284in}{2.126546in}}{\pgfqpoint{1.434108in}{2.120722in}}%
\pgfpathcurveto{\pgfqpoint{1.439932in}{2.114898in}}{\pgfqpoint{1.447832in}{2.111626in}}{\pgfqpoint{1.456069in}{2.111626in}}%
\pgfpathclose%
\pgfusepath{stroke,fill}%
\end{pgfscope}%
\begin{pgfscope}%
\pgfpathrectangle{\pgfqpoint{0.100000in}{0.212622in}}{\pgfqpoint{3.696000in}{3.696000in}}%
\pgfusepath{clip}%
\pgfsetbuttcap%
\pgfsetroundjoin%
\definecolor{currentfill}{rgb}{0.121569,0.466667,0.705882}%
\pgfsetfillcolor{currentfill}%
\pgfsetfillopacity{0.423244}%
\pgfsetlinewidth{1.003750pt}%
\definecolor{currentstroke}{rgb}{0.121569,0.466667,0.705882}%
\pgfsetstrokecolor{currentstroke}%
\pgfsetstrokeopacity{0.423244}%
\pgfsetdash{}{0pt}%
\pgfpathmoveto{\pgfqpoint{2.564120in}{1.911537in}}%
\pgfpathcurveto{\pgfqpoint{2.572356in}{1.911537in}}{\pgfqpoint{2.580256in}{1.914809in}}{\pgfqpoint{2.586080in}{1.920633in}}%
\pgfpathcurveto{\pgfqpoint{2.591904in}{1.926457in}}{\pgfqpoint{2.595177in}{1.934357in}}{\pgfqpoint{2.595177in}{1.942593in}}%
\pgfpathcurveto{\pgfqpoint{2.595177in}{1.950829in}}{\pgfqpoint{2.591904in}{1.958729in}}{\pgfqpoint{2.586080in}{1.964553in}}%
\pgfpathcurveto{\pgfqpoint{2.580256in}{1.970377in}}{\pgfqpoint{2.572356in}{1.973650in}}{\pgfqpoint{2.564120in}{1.973650in}}%
\pgfpathcurveto{\pgfqpoint{2.555884in}{1.973650in}}{\pgfqpoint{2.547984in}{1.970377in}}{\pgfqpoint{2.542160in}{1.964553in}}%
\pgfpathcurveto{\pgfqpoint{2.536336in}{1.958729in}}{\pgfqpoint{2.533064in}{1.950829in}}{\pgfqpoint{2.533064in}{1.942593in}}%
\pgfpathcurveto{\pgfqpoint{2.533064in}{1.934357in}}{\pgfqpoint{2.536336in}{1.926457in}}{\pgfqpoint{2.542160in}{1.920633in}}%
\pgfpathcurveto{\pgfqpoint{2.547984in}{1.914809in}}{\pgfqpoint{2.555884in}{1.911537in}}{\pgfqpoint{2.564120in}{1.911537in}}%
\pgfpathclose%
\pgfusepath{stroke,fill}%
\end{pgfscope}%
\begin{pgfscope}%
\pgfpathrectangle{\pgfqpoint{0.100000in}{0.212622in}}{\pgfqpoint{3.696000in}{3.696000in}}%
\pgfusepath{clip}%
\pgfsetbuttcap%
\pgfsetroundjoin%
\definecolor{currentfill}{rgb}{0.121569,0.466667,0.705882}%
\pgfsetfillcolor{currentfill}%
\pgfsetfillopacity{0.423303}%
\pgfsetlinewidth{1.003750pt}%
\definecolor{currentstroke}{rgb}{0.121569,0.466667,0.705882}%
\pgfsetstrokecolor{currentstroke}%
\pgfsetstrokeopacity{0.423303}%
\pgfsetdash{}{0pt}%
\pgfpathmoveto{\pgfqpoint{1.454852in}{2.111731in}}%
\pgfpathcurveto{\pgfqpoint{1.463089in}{2.111731in}}{\pgfqpoint{1.470989in}{2.115004in}}{\pgfqpoint{1.476813in}{2.120827in}}%
\pgfpathcurveto{\pgfqpoint{1.482637in}{2.126651in}}{\pgfqpoint{1.485909in}{2.134551in}}{\pgfqpoint{1.485909in}{2.142788in}}%
\pgfpathcurveto{\pgfqpoint{1.485909in}{2.151024in}}{\pgfqpoint{1.482637in}{2.158924in}}{\pgfqpoint{1.476813in}{2.164748in}}%
\pgfpathcurveto{\pgfqpoint{1.470989in}{2.170572in}}{\pgfqpoint{1.463089in}{2.173844in}}{\pgfqpoint{1.454852in}{2.173844in}}%
\pgfpathcurveto{\pgfqpoint{1.446616in}{2.173844in}}{\pgfqpoint{1.438716in}{2.170572in}}{\pgfqpoint{1.432892in}{2.164748in}}%
\pgfpathcurveto{\pgfqpoint{1.427068in}{2.158924in}}{\pgfqpoint{1.423796in}{2.151024in}}{\pgfqpoint{1.423796in}{2.142788in}}%
\pgfpathcurveto{\pgfqpoint{1.423796in}{2.134551in}}{\pgfqpoint{1.427068in}{2.126651in}}{\pgfqpoint{1.432892in}{2.120827in}}%
\pgfpathcurveto{\pgfqpoint{1.438716in}{2.115004in}}{\pgfqpoint{1.446616in}{2.111731in}}{\pgfqpoint{1.454852in}{2.111731in}}%
\pgfpathclose%
\pgfusepath{stroke,fill}%
\end{pgfscope}%
\begin{pgfscope}%
\pgfpathrectangle{\pgfqpoint{0.100000in}{0.212622in}}{\pgfqpoint{3.696000in}{3.696000in}}%
\pgfusepath{clip}%
\pgfsetbuttcap%
\pgfsetroundjoin%
\definecolor{currentfill}{rgb}{0.121569,0.466667,0.705882}%
\pgfsetfillcolor{currentfill}%
\pgfsetfillopacity{0.423539}%
\pgfsetlinewidth{1.003750pt}%
\definecolor{currentstroke}{rgb}{0.121569,0.466667,0.705882}%
\pgfsetstrokecolor{currentstroke}%
\pgfsetstrokeopacity{0.423539}%
\pgfsetdash{}{0pt}%
\pgfpathmoveto{\pgfqpoint{2.566589in}{1.910906in}}%
\pgfpathcurveto{\pgfqpoint{2.574825in}{1.910906in}}{\pgfqpoint{2.582725in}{1.914179in}}{\pgfqpoint{2.588549in}{1.920003in}}%
\pgfpathcurveto{\pgfqpoint{2.594373in}{1.925827in}}{\pgfqpoint{2.597645in}{1.933727in}}{\pgfqpoint{2.597645in}{1.941963in}}%
\pgfpathcurveto{\pgfqpoint{2.597645in}{1.950199in}}{\pgfqpoint{2.594373in}{1.958099in}}{\pgfqpoint{2.588549in}{1.963923in}}%
\pgfpathcurveto{\pgfqpoint{2.582725in}{1.969747in}}{\pgfqpoint{2.574825in}{1.973019in}}{\pgfqpoint{2.566589in}{1.973019in}}%
\pgfpathcurveto{\pgfqpoint{2.558352in}{1.973019in}}{\pgfqpoint{2.550452in}{1.969747in}}{\pgfqpoint{2.544628in}{1.963923in}}%
\pgfpathcurveto{\pgfqpoint{2.538805in}{1.958099in}}{\pgfqpoint{2.535532in}{1.950199in}}{\pgfqpoint{2.535532in}{1.941963in}}%
\pgfpathcurveto{\pgfqpoint{2.535532in}{1.933727in}}{\pgfqpoint{2.538805in}{1.925827in}}{\pgfqpoint{2.544628in}{1.920003in}}%
\pgfpathcurveto{\pgfqpoint{2.550452in}{1.914179in}}{\pgfqpoint{2.558352in}{1.910906in}}{\pgfqpoint{2.566589in}{1.910906in}}%
\pgfpathclose%
\pgfusepath{stroke,fill}%
\end{pgfscope}%
\begin{pgfscope}%
\pgfpathrectangle{\pgfqpoint{0.100000in}{0.212622in}}{\pgfqpoint{3.696000in}{3.696000in}}%
\pgfusepath{clip}%
\pgfsetbuttcap%
\pgfsetroundjoin%
\definecolor{currentfill}{rgb}{0.121569,0.466667,0.705882}%
\pgfsetfillcolor{currentfill}%
\pgfsetfillopacity{0.423597}%
\pgfsetlinewidth{1.003750pt}%
\definecolor{currentstroke}{rgb}{0.121569,0.466667,0.705882}%
\pgfsetstrokecolor{currentstroke}%
\pgfsetstrokeopacity{0.423597}%
\pgfsetdash{}{0pt}%
\pgfpathmoveto{\pgfqpoint{1.454904in}{2.111835in}}%
\pgfpathcurveto{\pgfqpoint{1.463141in}{2.111835in}}{\pgfqpoint{1.471041in}{2.115108in}}{\pgfqpoint{1.476864in}{2.120931in}}%
\pgfpathcurveto{\pgfqpoint{1.482688in}{2.126755in}}{\pgfqpoint{1.485961in}{2.134655in}}{\pgfqpoint{1.485961in}{2.142892in}}%
\pgfpathcurveto{\pgfqpoint{1.485961in}{2.151128in}}{\pgfqpoint{1.482688in}{2.159028in}}{\pgfqpoint{1.476864in}{2.164852in}}%
\pgfpathcurveto{\pgfqpoint{1.471041in}{2.170676in}}{\pgfqpoint{1.463141in}{2.173948in}}{\pgfqpoint{1.454904in}{2.173948in}}%
\pgfpathcurveto{\pgfqpoint{1.446668in}{2.173948in}}{\pgfqpoint{1.438768in}{2.170676in}}{\pgfqpoint{1.432944in}{2.164852in}}%
\pgfpathcurveto{\pgfqpoint{1.427120in}{2.159028in}}{\pgfqpoint{1.423848in}{2.151128in}}{\pgfqpoint{1.423848in}{2.142892in}}%
\pgfpathcurveto{\pgfqpoint{1.423848in}{2.134655in}}{\pgfqpoint{1.427120in}{2.126755in}}{\pgfqpoint{1.432944in}{2.120931in}}%
\pgfpathcurveto{\pgfqpoint{1.438768in}{2.115108in}}{\pgfqpoint{1.446668in}{2.111835in}}{\pgfqpoint{1.454904in}{2.111835in}}%
\pgfpathclose%
\pgfusepath{stroke,fill}%
\end{pgfscope}%
\begin{pgfscope}%
\pgfpathrectangle{\pgfqpoint{0.100000in}{0.212622in}}{\pgfqpoint{3.696000in}{3.696000in}}%
\pgfusepath{clip}%
\pgfsetbuttcap%
\pgfsetroundjoin%
\definecolor{currentfill}{rgb}{0.121569,0.466667,0.705882}%
\pgfsetfillcolor{currentfill}%
\pgfsetfillopacity{0.423768}%
\pgfsetlinewidth{1.003750pt}%
\definecolor{currentstroke}{rgb}{0.121569,0.466667,0.705882}%
\pgfsetstrokecolor{currentstroke}%
\pgfsetstrokeopacity{0.423768}%
\pgfsetdash{}{0pt}%
\pgfpathmoveto{\pgfqpoint{1.454559in}{2.111840in}}%
\pgfpathcurveto{\pgfqpoint{1.462795in}{2.111840in}}{\pgfqpoint{1.470695in}{2.115112in}}{\pgfqpoint{1.476519in}{2.120936in}}%
\pgfpathcurveto{\pgfqpoint{1.482343in}{2.126760in}}{\pgfqpoint{1.485615in}{2.134660in}}{\pgfqpoint{1.485615in}{2.142896in}}%
\pgfpathcurveto{\pgfqpoint{1.485615in}{2.151133in}}{\pgfqpoint{1.482343in}{2.159033in}}{\pgfqpoint{1.476519in}{2.164857in}}%
\pgfpathcurveto{\pgfqpoint{1.470695in}{2.170681in}}{\pgfqpoint{1.462795in}{2.173953in}}{\pgfqpoint{1.454559in}{2.173953in}}%
\pgfpathcurveto{\pgfqpoint{1.446323in}{2.173953in}}{\pgfqpoint{1.438423in}{2.170681in}}{\pgfqpoint{1.432599in}{2.164857in}}%
\pgfpathcurveto{\pgfqpoint{1.426775in}{2.159033in}}{\pgfqpoint{1.423502in}{2.151133in}}{\pgfqpoint{1.423502in}{2.142896in}}%
\pgfpathcurveto{\pgfqpoint{1.423502in}{2.134660in}}{\pgfqpoint{1.426775in}{2.126760in}}{\pgfqpoint{1.432599in}{2.120936in}}%
\pgfpathcurveto{\pgfqpoint{1.438423in}{2.115112in}}{\pgfqpoint{1.446323in}{2.111840in}}{\pgfqpoint{1.454559in}{2.111840in}}%
\pgfpathclose%
\pgfusepath{stroke,fill}%
\end{pgfscope}%
\begin{pgfscope}%
\pgfpathrectangle{\pgfqpoint{0.100000in}{0.212622in}}{\pgfqpoint{3.696000in}{3.696000in}}%
\pgfusepath{clip}%
\pgfsetbuttcap%
\pgfsetroundjoin%
\definecolor{currentfill}{rgb}{0.121569,0.466667,0.705882}%
\pgfsetfillcolor{currentfill}%
\pgfsetfillopacity{0.424042}%
\pgfsetlinewidth{1.003750pt}%
\definecolor{currentstroke}{rgb}{0.121569,0.466667,0.705882}%
\pgfsetstrokecolor{currentstroke}%
\pgfsetstrokeopacity{0.424042}%
\pgfsetdash{}{0pt}%
\pgfpathmoveto{\pgfqpoint{2.569838in}{1.910189in}}%
\pgfpathcurveto{\pgfqpoint{2.578074in}{1.910189in}}{\pgfqpoint{2.585974in}{1.913461in}}{\pgfqpoint{2.591798in}{1.919285in}}%
\pgfpathcurveto{\pgfqpoint{2.597622in}{1.925109in}}{\pgfqpoint{2.600894in}{1.933009in}}{\pgfqpoint{2.600894in}{1.941245in}}%
\pgfpathcurveto{\pgfqpoint{2.600894in}{1.949481in}}{\pgfqpoint{2.597622in}{1.957381in}}{\pgfqpoint{2.591798in}{1.963205in}}%
\pgfpathcurveto{\pgfqpoint{2.585974in}{1.969029in}}{\pgfqpoint{2.578074in}{1.972302in}}{\pgfqpoint{2.569838in}{1.972302in}}%
\pgfpathcurveto{\pgfqpoint{2.561602in}{1.972302in}}{\pgfqpoint{2.553701in}{1.969029in}}{\pgfqpoint{2.547878in}{1.963205in}}%
\pgfpathcurveto{\pgfqpoint{2.542054in}{1.957381in}}{\pgfqpoint{2.538781in}{1.949481in}}{\pgfqpoint{2.538781in}{1.941245in}}%
\pgfpathcurveto{\pgfqpoint{2.538781in}{1.933009in}}{\pgfqpoint{2.542054in}{1.925109in}}{\pgfqpoint{2.547878in}{1.919285in}}%
\pgfpathcurveto{\pgfqpoint{2.553701in}{1.913461in}}{\pgfqpoint{2.561602in}{1.910189in}}{\pgfqpoint{2.569838in}{1.910189in}}%
\pgfpathclose%
\pgfusepath{stroke,fill}%
\end{pgfscope}%
\begin{pgfscope}%
\pgfpathrectangle{\pgfqpoint{0.100000in}{0.212622in}}{\pgfqpoint{3.696000in}{3.696000in}}%
\pgfusepath{clip}%
\pgfsetbuttcap%
\pgfsetroundjoin%
\definecolor{currentfill}{rgb}{0.121569,0.466667,0.705882}%
\pgfsetfillcolor{currentfill}%
\pgfsetfillopacity{0.424061}%
\pgfsetlinewidth{1.003750pt}%
\definecolor{currentstroke}{rgb}{0.121569,0.466667,0.705882}%
\pgfsetstrokecolor{currentstroke}%
\pgfsetstrokeopacity{0.424061}%
\pgfsetdash{}{0pt}%
\pgfpathmoveto{\pgfqpoint{1.453802in}{2.111859in}}%
\pgfpathcurveto{\pgfqpoint{1.462038in}{2.111859in}}{\pgfqpoint{1.469938in}{2.115131in}}{\pgfqpoint{1.475762in}{2.120955in}}%
\pgfpathcurveto{\pgfqpoint{1.481586in}{2.126779in}}{\pgfqpoint{1.484858in}{2.134679in}}{\pgfqpoint{1.484858in}{2.142915in}}%
\pgfpathcurveto{\pgfqpoint{1.484858in}{2.151151in}}{\pgfqpoint{1.481586in}{2.159051in}}{\pgfqpoint{1.475762in}{2.164875in}}%
\pgfpathcurveto{\pgfqpoint{1.469938in}{2.170699in}}{\pgfqpoint{1.462038in}{2.173972in}}{\pgfqpoint{1.453802in}{2.173972in}}%
\pgfpathcurveto{\pgfqpoint{1.445565in}{2.173972in}}{\pgfqpoint{1.437665in}{2.170699in}}{\pgfqpoint{1.431841in}{2.164875in}}%
\pgfpathcurveto{\pgfqpoint{1.426017in}{2.159051in}}{\pgfqpoint{1.422745in}{2.151151in}}{\pgfqpoint{1.422745in}{2.142915in}}%
\pgfpathcurveto{\pgfqpoint{1.422745in}{2.134679in}}{\pgfqpoint{1.426017in}{2.126779in}}{\pgfqpoint{1.431841in}{2.120955in}}%
\pgfpathcurveto{\pgfqpoint{1.437665in}{2.115131in}}{\pgfqpoint{1.445565in}{2.111859in}}{\pgfqpoint{1.453802in}{2.111859in}}%
\pgfpathclose%
\pgfusepath{stroke,fill}%
\end{pgfscope}%
\begin{pgfscope}%
\pgfpathrectangle{\pgfqpoint{0.100000in}{0.212622in}}{\pgfqpoint{3.696000in}{3.696000in}}%
\pgfusepath{clip}%
\pgfsetbuttcap%
\pgfsetroundjoin%
\definecolor{currentfill}{rgb}{0.121569,0.466667,0.705882}%
\pgfsetfillcolor{currentfill}%
\pgfsetfillopacity{0.424672}%
\pgfsetlinewidth{1.003750pt}%
\definecolor{currentstroke}{rgb}{0.121569,0.466667,0.705882}%
\pgfsetstrokecolor{currentstroke}%
\pgfsetstrokeopacity{0.424672}%
\pgfsetdash{}{0pt}%
\pgfpathmoveto{\pgfqpoint{1.453091in}{2.111898in}}%
\pgfpathcurveto{\pgfqpoint{1.461327in}{2.111898in}}{\pgfqpoint{1.469227in}{2.115171in}}{\pgfqpoint{1.475051in}{2.120994in}}%
\pgfpathcurveto{\pgfqpoint{1.480875in}{2.126818in}}{\pgfqpoint{1.484147in}{2.134718in}}{\pgfqpoint{1.484147in}{2.142955in}}%
\pgfpathcurveto{\pgfqpoint{1.484147in}{2.151191in}}{\pgfqpoint{1.480875in}{2.159091in}}{\pgfqpoint{1.475051in}{2.164915in}}%
\pgfpathcurveto{\pgfqpoint{1.469227in}{2.170739in}}{\pgfqpoint{1.461327in}{2.174011in}}{\pgfqpoint{1.453091in}{2.174011in}}%
\pgfpathcurveto{\pgfqpoint{1.444855in}{2.174011in}}{\pgfqpoint{1.436955in}{2.170739in}}{\pgfqpoint{1.431131in}{2.164915in}}%
\pgfpathcurveto{\pgfqpoint{1.425307in}{2.159091in}}{\pgfqpoint{1.422034in}{2.151191in}}{\pgfqpoint{1.422034in}{2.142955in}}%
\pgfpathcurveto{\pgfqpoint{1.422034in}{2.134718in}}{\pgfqpoint{1.425307in}{2.126818in}}{\pgfqpoint{1.431131in}{2.120994in}}%
\pgfpathcurveto{\pgfqpoint{1.436955in}{2.115171in}}{\pgfqpoint{1.444855in}{2.111898in}}{\pgfqpoint{1.453091in}{2.111898in}}%
\pgfpathclose%
\pgfusepath{stroke,fill}%
\end{pgfscope}%
\begin{pgfscope}%
\pgfpathrectangle{\pgfqpoint{0.100000in}{0.212622in}}{\pgfqpoint{3.696000in}{3.696000in}}%
\pgfusepath{clip}%
\pgfsetbuttcap%
\pgfsetroundjoin%
\definecolor{currentfill}{rgb}{0.121569,0.466667,0.705882}%
\pgfsetfillcolor{currentfill}%
\pgfsetfillopacity{0.425000}%
\pgfsetlinewidth{1.003750pt}%
\definecolor{currentstroke}{rgb}{0.121569,0.466667,0.705882}%
\pgfsetstrokecolor{currentstroke}%
\pgfsetstrokeopacity{0.425000}%
\pgfsetdash{}{0pt}%
\pgfpathmoveto{\pgfqpoint{2.574662in}{1.909238in}}%
\pgfpathcurveto{\pgfqpoint{2.582899in}{1.909238in}}{\pgfqpoint{2.590799in}{1.912510in}}{\pgfqpoint{2.596623in}{1.918334in}}%
\pgfpathcurveto{\pgfqpoint{2.602447in}{1.924158in}}{\pgfqpoint{2.605719in}{1.932058in}}{\pgfqpoint{2.605719in}{1.940294in}}%
\pgfpathcurveto{\pgfqpoint{2.605719in}{1.948530in}}{\pgfqpoint{2.602447in}{1.956431in}}{\pgfqpoint{2.596623in}{1.962254in}}%
\pgfpathcurveto{\pgfqpoint{2.590799in}{1.968078in}}{\pgfqpoint{2.582899in}{1.971351in}}{\pgfqpoint{2.574662in}{1.971351in}}%
\pgfpathcurveto{\pgfqpoint{2.566426in}{1.971351in}}{\pgfqpoint{2.558526in}{1.968078in}}{\pgfqpoint{2.552702in}{1.962254in}}%
\pgfpathcurveto{\pgfqpoint{2.546878in}{1.956431in}}{\pgfqpoint{2.543606in}{1.948530in}}{\pgfqpoint{2.543606in}{1.940294in}}%
\pgfpathcurveto{\pgfqpoint{2.543606in}{1.932058in}}{\pgfqpoint{2.546878in}{1.924158in}}{\pgfqpoint{2.552702in}{1.918334in}}%
\pgfpathcurveto{\pgfqpoint{2.558526in}{1.912510in}}{\pgfqpoint{2.566426in}{1.909238in}}{\pgfqpoint{2.574662in}{1.909238in}}%
\pgfpathclose%
\pgfusepath{stroke,fill}%
\end{pgfscope}%
\begin{pgfscope}%
\pgfpathrectangle{\pgfqpoint{0.100000in}{0.212622in}}{\pgfqpoint{3.696000in}{3.696000in}}%
\pgfusepath{clip}%
\pgfsetbuttcap%
\pgfsetroundjoin%
\definecolor{currentfill}{rgb}{0.121569,0.466667,0.705882}%
\pgfsetfillcolor{currentfill}%
\pgfsetfillopacity{0.425695}%
\pgfsetlinewidth{1.003750pt}%
\definecolor{currentstroke}{rgb}{0.121569,0.466667,0.705882}%
\pgfsetstrokecolor{currentstroke}%
\pgfsetstrokeopacity{0.425695}%
\pgfsetdash{}{0pt}%
\pgfpathmoveto{\pgfqpoint{1.450905in}{2.111986in}}%
\pgfpathcurveto{\pgfqpoint{1.459142in}{2.111986in}}{\pgfqpoint{1.467042in}{2.115258in}}{\pgfqpoint{1.472866in}{2.121082in}}%
\pgfpathcurveto{\pgfqpoint{1.478690in}{2.126906in}}{\pgfqpoint{1.481962in}{2.134806in}}{\pgfqpoint{1.481962in}{2.143042in}}%
\pgfpathcurveto{\pgfqpoint{1.481962in}{2.151279in}}{\pgfqpoint{1.478690in}{2.159179in}}{\pgfqpoint{1.472866in}{2.165002in}}%
\pgfpathcurveto{\pgfqpoint{1.467042in}{2.170826in}}{\pgfqpoint{1.459142in}{2.174099in}}{\pgfqpoint{1.450905in}{2.174099in}}%
\pgfpathcurveto{\pgfqpoint{1.442669in}{2.174099in}}{\pgfqpoint{1.434769in}{2.170826in}}{\pgfqpoint{1.428945in}{2.165002in}}%
\pgfpathcurveto{\pgfqpoint{1.423121in}{2.159179in}}{\pgfqpoint{1.419849in}{2.151279in}}{\pgfqpoint{1.419849in}{2.143042in}}%
\pgfpathcurveto{\pgfqpoint{1.419849in}{2.134806in}}{\pgfqpoint{1.423121in}{2.126906in}}{\pgfqpoint{1.428945in}{2.121082in}}%
\pgfpathcurveto{\pgfqpoint{1.434769in}{2.115258in}}{\pgfqpoint{1.442669in}{2.111986in}}{\pgfqpoint{1.450905in}{2.111986in}}%
\pgfpathclose%
\pgfusepath{stroke,fill}%
\end{pgfscope}%
\begin{pgfscope}%
\pgfpathrectangle{\pgfqpoint{0.100000in}{0.212622in}}{\pgfqpoint{3.696000in}{3.696000in}}%
\pgfusepath{clip}%
\pgfsetbuttcap%
\pgfsetroundjoin%
\definecolor{currentfill}{rgb}{0.121569,0.466667,0.705882}%
\pgfsetfillcolor{currentfill}%
\pgfsetfillopacity{0.425808}%
\pgfsetlinewidth{1.003750pt}%
\definecolor{currentstroke}{rgb}{0.121569,0.466667,0.705882}%
\pgfsetstrokecolor{currentstroke}%
\pgfsetstrokeopacity{0.425808}%
\pgfsetdash{}{0pt}%
\pgfpathmoveto{\pgfqpoint{2.580428in}{1.907792in}}%
\pgfpathcurveto{\pgfqpoint{2.588664in}{1.907792in}}{\pgfqpoint{2.596564in}{1.911064in}}{\pgfqpoint{2.602388in}{1.916888in}}%
\pgfpathcurveto{\pgfqpoint{2.608212in}{1.922712in}}{\pgfqpoint{2.611484in}{1.930612in}}{\pgfqpoint{2.611484in}{1.938849in}}%
\pgfpathcurveto{\pgfqpoint{2.611484in}{1.947085in}}{\pgfqpoint{2.608212in}{1.954985in}}{\pgfqpoint{2.602388in}{1.960809in}}%
\pgfpathcurveto{\pgfqpoint{2.596564in}{1.966633in}}{\pgfqpoint{2.588664in}{1.969905in}}{\pgfqpoint{2.580428in}{1.969905in}}%
\pgfpathcurveto{\pgfqpoint{2.572192in}{1.969905in}}{\pgfqpoint{2.564291in}{1.966633in}}{\pgfqpoint{2.558468in}{1.960809in}}%
\pgfpathcurveto{\pgfqpoint{2.552644in}{1.954985in}}{\pgfqpoint{2.549371in}{1.947085in}}{\pgfqpoint{2.549371in}{1.938849in}}%
\pgfpathcurveto{\pgfqpoint{2.549371in}{1.930612in}}{\pgfqpoint{2.552644in}{1.922712in}}{\pgfqpoint{2.558468in}{1.916888in}}%
\pgfpathcurveto{\pgfqpoint{2.564291in}{1.911064in}}{\pgfqpoint{2.572192in}{1.907792in}}{\pgfqpoint{2.580428in}{1.907792in}}%
\pgfpathclose%
\pgfusepath{stroke,fill}%
\end{pgfscope}%
\begin{pgfscope}%
\pgfpathrectangle{\pgfqpoint{0.100000in}{0.212622in}}{\pgfqpoint{3.696000in}{3.696000in}}%
\pgfusepath{clip}%
\pgfsetbuttcap%
\pgfsetroundjoin%
\definecolor{currentfill}{rgb}{0.121569,0.466667,0.705882}%
\pgfsetfillcolor{currentfill}%
\pgfsetfillopacity{0.427009}%
\pgfsetlinewidth{1.003750pt}%
\definecolor{currentstroke}{rgb}{0.121569,0.466667,0.705882}%
\pgfsetstrokecolor{currentstroke}%
\pgfsetstrokeopacity{0.427009}%
\pgfsetdash{}{0pt}%
\pgfpathmoveto{\pgfqpoint{2.586209in}{1.906951in}}%
\pgfpathcurveto{\pgfqpoint{2.594445in}{1.906951in}}{\pgfqpoint{2.602345in}{1.910223in}}{\pgfqpoint{2.608169in}{1.916047in}}%
\pgfpathcurveto{\pgfqpoint{2.613993in}{1.921871in}}{\pgfqpoint{2.617265in}{1.929771in}}{\pgfqpoint{2.617265in}{1.938007in}}%
\pgfpathcurveto{\pgfqpoint{2.617265in}{1.946244in}}{\pgfqpoint{2.613993in}{1.954144in}}{\pgfqpoint{2.608169in}{1.959968in}}%
\pgfpathcurveto{\pgfqpoint{2.602345in}{1.965791in}}{\pgfqpoint{2.594445in}{1.969064in}}{\pgfqpoint{2.586209in}{1.969064in}}%
\pgfpathcurveto{\pgfqpoint{2.577973in}{1.969064in}}{\pgfqpoint{2.570073in}{1.965791in}}{\pgfqpoint{2.564249in}{1.959968in}}%
\pgfpathcurveto{\pgfqpoint{2.558425in}{1.954144in}}{\pgfqpoint{2.555152in}{1.946244in}}{\pgfqpoint{2.555152in}{1.938007in}}%
\pgfpathcurveto{\pgfqpoint{2.555152in}{1.929771in}}{\pgfqpoint{2.558425in}{1.921871in}}{\pgfqpoint{2.564249in}{1.916047in}}%
\pgfpathcurveto{\pgfqpoint{2.570073in}{1.910223in}}{\pgfqpoint{2.577973in}{1.906951in}}{\pgfqpoint{2.586209in}{1.906951in}}%
\pgfpathclose%
\pgfusepath{stroke,fill}%
\end{pgfscope}%
\begin{pgfscope}%
\pgfpathrectangle{\pgfqpoint{0.100000in}{0.212622in}}{\pgfqpoint{3.696000in}{3.696000in}}%
\pgfusepath{clip}%
\pgfsetbuttcap%
\pgfsetroundjoin%
\definecolor{currentfill}{rgb}{0.121569,0.466667,0.705882}%
\pgfsetfillcolor{currentfill}%
\pgfsetfillopacity{0.427561}%
\pgfsetlinewidth{1.003750pt}%
\definecolor{currentstroke}{rgb}{0.121569,0.466667,0.705882}%
\pgfsetstrokecolor{currentstroke}%
\pgfsetstrokeopacity{0.427561}%
\pgfsetdash{}{0pt}%
\pgfpathmoveto{\pgfqpoint{1.446979in}{2.112137in}}%
\pgfpathcurveto{\pgfqpoint{1.455215in}{2.112137in}}{\pgfqpoint{1.463115in}{2.115409in}}{\pgfqpoint{1.468939in}{2.121233in}}%
\pgfpathcurveto{\pgfqpoint{1.474763in}{2.127057in}}{\pgfqpoint{1.478035in}{2.134957in}}{\pgfqpoint{1.478035in}{2.143193in}}%
\pgfpathcurveto{\pgfqpoint{1.478035in}{2.151429in}}{\pgfqpoint{1.474763in}{2.159330in}}{\pgfqpoint{1.468939in}{2.165153in}}%
\pgfpathcurveto{\pgfqpoint{1.463115in}{2.170977in}}{\pgfqpoint{1.455215in}{2.174250in}}{\pgfqpoint{1.446979in}{2.174250in}}%
\pgfpathcurveto{\pgfqpoint{1.438742in}{2.174250in}}{\pgfqpoint{1.430842in}{2.170977in}}{\pgfqpoint{1.425018in}{2.165153in}}%
\pgfpathcurveto{\pgfqpoint{1.419194in}{2.159330in}}{\pgfqpoint{1.415922in}{2.151429in}}{\pgfqpoint{1.415922in}{2.143193in}}%
\pgfpathcurveto{\pgfqpoint{1.415922in}{2.134957in}}{\pgfqpoint{1.419194in}{2.127057in}}{\pgfqpoint{1.425018in}{2.121233in}}%
\pgfpathcurveto{\pgfqpoint{1.430842in}{2.115409in}}{\pgfqpoint{1.438742in}{2.112137in}}{\pgfqpoint{1.446979in}{2.112137in}}%
\pgfpathclose%
\pgfusepath{stroke,fill}%
\end{pgfscope}%
\begin{pgfscope}%
\pgfpathrectangle{\pgfqpoint{0.100000in}{0.212622in}}{\pgfqpoint{3.696000in}{3.696000in}}%
\pgfusepath{clip}%
\pgfsetbuttcap%
\pgfsetroundjoin%
\definecolor{currentfill}{rgb}{0.121569,0.466667,0.705882}%
\pgfsetfillcolor{currentfill}%
\pgfsetfillopacity{0.428139}%
\pgfsetlinewidth{1.003750pt}%
\definecolor{currentstroke}{rgb}{0.121569,0.466667,0.705882}%
\pgfsetstrokecolor{currentstroke}%
\pgfsetstrokeopacity{0.428139}%
\pgfsetdash{}{0pt}%
\pgfpathmoveto{\pgfqpoint{2.593351in}{1.905426in}}%
\pgfpathcurveto{\pgfqpoint{2.601587in}{1.905426in}}{\pgfqpoint{2.609487in}{1.908698in}}{\pgfqpoint{2.615311in}{1.914522in}}%
\pgfpathcurveto{\pgfqpoint{2.621135in}{1.920346in}}{\pgfqpoint{2.624407in}{1.928246in}}{\pgfqpoint{2.624407in}{1.936482in}}%
\pgfpathcurveto{\pgfqpoint{2.624407in}{1.944718in}}{\pgfqpoint{2.621135in}{1.952618in}}{\pgfqpoint{2.615311in}{1.958442in}}%
\pgfpathcurveto{\pgfqpoint{2.609487in}{1.964266in}}{\pgfqpoint{2.601587in}{1.967539in}}{\pgfqpoint{2.593351in}{1.967539in}}%
\pgfpathcurveto{\pgfqpoint{2.585115in}{1.967539in}}{\pgfqpoint{2.577215in}{1.964266in}}{\pgfqpoint{2.571391in}{1.958442in}}%
\pgfpathcurveto{\pgfqpoint{2.565567in}{1.952618in}}{\pgfqpoint{2.562294in}{1.944718in}}{\pgfqpoint{2.562294in}{1.936482in}}%
\pgfpathcurveto{\pgfqpoint{2.562294in}{1.928246in}}{\pgfqpoint{2.565567in}{1.920346in}}{\pgfqpoint{2.571391in}{1.914522in}}%
\pgfpathcurveto{\pgfqpoint{2.577215in}{1.908698in}}{\pgfqpoint{2.585115in}{1.905426in}}{\pgfqpoint{2.593351in}{1.905426in}}%
\pgfpathclose%
\pgfusepath{stroke,fill}%
\end{pgfscope}%
\begin{pgfscope}%
\pgfpathrectangle{\pgfqpoint{0.100000in}{0.212622in}}{\pgfqpoint{3.696000in}{3.696000in}}%
\pgfusepath{clip}%
\pgfsetbuttcap%
\pgfsetroundjoin%
\definecolor{currentfill}{rgb}{0.121569,0.466667,0.705882}%
\pgfsetfillcolor{currentfill}%
\pgfsetfillopacity{0.428945}%
\pgfsetlinewidth{1.003750pt}%
\definecolor{currentstroke}{rgb}{0.121569,0.466667,0.705882}%
\pgfsetstrokecolor{currentstroke}%
\pgfsetstrokeopacity{0.428945}%
\pgfsetdash{}{0pt}%
\pgfpathmoveto{\pgfqpoint{2.602817in}{1.902982in}}%
\pgfpathcurveto{\pgfqpoint{2.611054in}{1.902982in}}{\pgfqpoint{2.618954in}{1.906255in}}{\pgfqpoint{2.624778in}{1.912079in}}%
\pgfpathcurveto{\pgfqpoint{2.630601in}{1.917902in}}{\pgfqpoint{2.633874in}{1.925803in}}{\pgfqpoint{2.633874in}{1.934039in}}%
\pgfpathcurveto{\pgfqpoint{2.633874in}{1.942275in}}{\pgfqpoint{2.630601in}{1.950175in}}{\pgfqpoint{2.624778in}{1.955999in}}%
\pgfpathcurveto{\pgfqpoint{2.618954in}{1.961823in}}{\pgfqpoint{2.611054in}{1.965095in}}{\pgfqpoint{2.602817in}{1.965095in}}%
\pgfpathcurveto{\pgfqpoint{2.594581in}{1.965095in}}{\pgfqpoint{2.586681in}{1.961823in}}{\pgfqpoint{2.580857in}{1.955999in}}%
\pgfpathcurveto{\pgfqpoint{2.575033in}{1.950175in}}{\pgfqpoint{2.571761in}{1.942275in}}{\pgfqpoint{2.571761in}{1.934039in}}%
\pgfpathcurveto{\pgfqpoint{2.571761in}{1.925803in}}{\pgfqpoint{2.575033in}{1.917902in}}{\pgfqpoint{2.580857in}{1.912079in}}%
\pgfpathcurveto{\pgfqpoint{2.586681in}{1.906255in}}{\pgfqpoint{2.594581in}{1.902982in}}{\pgfqpoint{2.602817in}{1.902982in}}%
\pgfpathclose%
\pgfusepath{stroke,fill}%
\end{pgfscope}%
\begin{pgfscope}%
\pgfpathrectangle{\pgfqpoint{0.100000in}{0.212622in}}{\pgfqpoint{3.696000in}{3.696000in}}%
\pgfusepath{clip}%
\pgfsetbuttcap%
\pgfsetroundjoin%
\definecolor{currentfill}{rgb}{0.121569,0.466667,0.705882}%
\pgfsetfillcolor{currentfill}%
\pgfsetfillopacity{0.429362}%
\pgfsetlinewidth{1.003750pt}%
\definecolor{currentstroke}{rgb}{0.121569,0.466667,0.705882}%
\pgfsetstrokecolor{currentstroke}%
\pgfsetstrokeopacity{0.429362}%
\pgfsetdash{}{0pt}%
\pgfpathmoveto{\pgfqpoint{1.444015in}{2.112147in}}%
\pgfpathcurveto{\pgfqpoint{1.452251in}{2.112147in}}{\pgfqpoint{1.460151in}{2.115419in}}{\pgfqpoint{1.465975in}{2.121243in}}%
\pgfpathcurveto{\pgfqpoint{1.471799in}{2.127067in}}{\pgfqpoint{1.475072in}{2.134967in}}{\pgfqpoint{1.475072in}{2.143203in}}%
\pgfpathcurveto{\pgfqpoint{1.475072in}{2.151440in}}{\pgfqpoint{1.471799in}{2.159340in}}{\pgfqpoint{1.465975in}{2.165164in}}%
\pgfpathcurveto{\pgfqpoint{1.460151in}{2.170988in}}{\pgfqpoint{1.452251in}{2.174260in}}{\pgfqpoint{1.444015in}{2.174260in}}%
\pgfpathcurveto{\pgfqpoint{1.435779in}{2.174260in}}{\pgfqpoint{1.427879in}{2.170988in}}{\pgfqpoint{1.422055in}{2.165164in}}%
\pgfpathcurveto{\pgfqpoint{1.416231in}{2.159340in}}{\pgfqpoint{1.412959in}{2.151440in}}{\pgfqpoint{1.412959in}{2.143203in}}%
\pgfpathcurveto{\pgfqpoint{1.412959in}{2.134967in}}{\pgfqpoint{1.416231in}{2.127067in}}{\pgfqpoint{1.422055in}{2.121243in}}%
\pgfpathcurveto{\pgfqpoint{1.427879in}{2.115419in}}{\pgfqpoint{1.435779in}{2.112147in}}{\pgfqpoint{1.444015in}{2.112147in}}%
\pgfpathclose%
\pgfusepath{stroke,fill}%
\end{pgfscope}%
\begin{pgfscope}%
\pgfpathrectangle{\pgfqpoint{0.100000in}{0.212622in}}{\pgfqpoint{3.696000in}{3.696000in}}%
\pgfusepath{clip}%
\pgfsetbuttcap%
\pgfsetroundjoin%
\definecolor{currentfill}{rgb}{0.121569,0.466667,0.705882}%
\pgfsetfillcolor{currentfill}%
\pgfsetfillopacity{0.430515}%
\pgfsetlinewidth{1.003750pt}%
\definecolor{currentstroke}{rgb}{0.121569,0.466667,0.705882}%
\pgfsetstrokecolor{currentstroke}%
\pgfsetstrokeopacity{0.430515}%
\pgfsetdash{}{0pt}%
\pgfpathmoveto{\pgfqpoint{2.613692in}{1.900993in}}%
\pgfpathcurveto{\pgfqpoint{2.621929in}{1.900993in}}{\pgfqpoint{2.629829in}{1.904266in}}{\pgfqpoint{2.635653in}{1.910090in}}%
\pgfpathcurveto{\pgfqpoint{2.641476in}{1.915914in}}{\pgfqpoint{2.644749in}{1.923814in}}{\pgfqpoint{2.644749in}{1.932050in}}%
\pgfpathcurveto{\pgfqpoint{2.644749in}{1.940286in}}{\pgfqpoint{2.641476in}{1.948186in}}{\pgfqpoint{2.635653in}{1.954010in}}%
\pgfpathcurveto{\pgfqpoint{2.629829in}{1.959834in}}{\pgfqpoint{2.621929in}{1.963106in}}{\pgfqpoint{2.613692in}{1.963106in}}%
\pgfpathcurveto{\pgfqpoint{2.605456in}{1.963106in}}{\pgfqpoint{2.597556in}{1.959834in}}{\pgfqpoint{2.591732in}{1.954010in}}%
\pgfpathcurveto{\pgfqpoint{2.585908in}{1.948186in}}{\pgfqpoint{2.582636in}{1.940286in}}{\pgfqpoint{2.582636in}{1.932050in}}%
\pgfpathcurveto{\pgfqpoint{2.582636in}{1.923814in}}{\pgfqpoint{2.585908in}{1.915914in}}{\pgfqpoint{2.591732in}{1.910090in}}%
\pgfpathcurveto{\pgfqpoint{2.597556in}{1.904266in}}{\pgfqpoint{2.605456in}{1.900993in}}{\pgfqpoint{2.613692in}{1.900993in}}%
\pgfpathclose%
\pgfusepath{stroke,fill}%
\end{pgfscope}%
\begin{pgfscope}%
\pgfpathrectangle{\pgfqpoint{0.100000in}{0.212622in}}{\pgfqpoint{3.696000in}{3.696000in}}%
\pgfusepath{clip}%
\pgfsetbuttcap%
\pgfsetroundjoin%
\definecolor{currentfill}{rgb}{0.121569,0.466667,0.705882}%
\pgfsetfillcolor{currentfill}%
\pgfsetfillopacity{0.430862}%
\pgfsetlinewidth{1.003750pt}%
\definecolor{currentstroke}{rgb}{0.121569,0.466667,0.705882}%
\pgfsetstrokecolor{currentstroke}%
\pgfsetstrokeopacity{0.430862}%
\pgfsetdash{}{0pt}%
\pgfpathmoveto{\pgfqpoint{1.439576in}{2.112624in}}%
\pgfpathcurveto{\pgfqpoint{1.447813in}{2.112624in}}{\pgfqpoint{1.455713in}{2.115896in}}{\pgfqpoint{1.461537in}{2.121720in}}%
\pgfpathcurveto{\pgfqpoint{1.467361in}{2.127544in}}{\pgfqpoint{1.470633in}{2.135444in}}{\pgfqpoint{1.470633in}{2.143680in}}%
\pgfpathcurveto{\pgfqpoint{1.470633in}{2.151916in}}{\pgfqpoint{1.467361in}{2.159816in}}{\pgfqpoint{1.461537in}{2.165640in}}%
\pgfpathcurveto{\pgfqpoint{1.455713in}{2.171464in}}{\pgfqpoint{1.447813in}{2.174737in}}{\pgfqpoint{1.439576in}{2.174737in}}%
\pgfpathcurveto{\pgfqpoint{1.431340in}{2.174737in}}{\pgfqpoint{1.423440in}{2.171464in}}{\pgfqpoint{1.417616in}{2.165640in}}%
\pgfpathcurveto{\pgfqpoint{1.411792in}{2.159816in}}{\pgfqpoint{1.408520in}{2.151916in}}{\pgfqpoint{1.408520in}{2.143680in}}%
\pgfpathcurveto{\pgfqpoint{1.408520in}{2.135444in}}{\pgfqpoint{1.411792in}{2.127544in}}{\pgfqpoint{1.417616in}{2.121720in}}%
\pgfpathcurveto{\pgfqpoint{1.423440in}{2.115896in}}{\pgfqpoint{1.431340in}{2.112624in}}{\pgfqpoint{1.439576in}{2.112624in}}%
\pgfpathclose%
\pgfusepath{stroke,fill}%
\end{pgfscope}%
\begin{pgfscope}%
\pgfpathrectangle{\pgfqpoint{0.100000in}{0.212622in}}{\pgfqpoint{3.696000in}{3.696000in}}%
\pgfusepath{clip}%
\pgfsetbuttcap%
\pgfsetroundjoin%
\definecolor{currentfill}{rgb}{0.121569,0.466667,0.705882}%
\pgfsetfillcolor{currentfill}%
\pgfsetfillopacity{0.432030}%
\pgfsetlinewidth{1.003750pt}%
\definecolor{currentstroke}{rgb}{0.121569,0.466667,0.705882}%
\pgfsetstrokecolor{currentstroke}%
\pgfsetstrokeopacity{0.432030}%
\pgfsetdash{}{0pt}%
\pgfpathmoveto{\pgfqpoint{1.437235in}{2.112689in}}%
\pgfpathcurveto{\pgfqpoint{1.445471in}{2.112689in}}{\pgfqpoint{1.453371in}{2.115961in}}{\pgfqpoint{1.459195in}{2.121785in}}%
\pgfpathcurveto{\pgfqpoint{1.465019in}{2.127609in}}{\pgfqpoint{1.468291in}{2.135509in}}{\pgfqpoint{1.468291in}{2.143745in}}%
\pgfpathcurveto{\pgfqpoint{1.468291in}{2.151982in}}{\pgfqpoint{1.465019in}{2.159882in}}{\pgfqpoint{1.459195in}{2.165706in}}%
\pgfpathcurveto{\pgfqpoint{1.453371in}{2.171530in}}{\pgfqpoint{1.445471in}{2.174802in}}{\pgfqpoint{1.437235in}{2.174802in}}%
\pgfpathcurveto{\pgfqpoint{1.428999in}{2.174802in}}{\pgfqpoint{1.421099in}{2.171530in}}{\pgfqpoint{1.415275in}{2.165706in}}%
\pgfpathcurveto{\pgfqpoint{1.409451in}{2.159882in}}{\pgfqpoint{1.406178in}{2.151982in}}{\pgfqpoint{1.406178in}{2.143745in}}%
\pgfpathcurveto{\pgfqpoint{1.406178in}{2.135509in}}{\pgfqpoint{1.409451in}{2.127609in}}{\pgfqpoint{1.415275in}{2.121785in}}%
\pgfpathcurveto{\pgfqpoint{1.421099in}{2.115961in}}{\pgfqpoint{1.428999in}{2.112689in}}{\pgfqpoint{1.437235in}{2.112689in}}%
\pgfpathclose%
\pgfusepath{stroke,fill}%
\end{pgfscope}%
\begin{pgfscope}%
\pgfpathrectangle{\pgfqpoint{0.100000in}{0.212622in}}{\pgfqpoint{3.696000in}{3.696000in}}%
\pgfusepath{clip}%
\pgfsetbuttcap%
\pgfsetroundjoin%
\definecolor{currentfill}{rgb}{0.121569,0.466667,0.705882}%
\pgfsetfillcolor{currentfill}%
\pgfsetfillopacity{0.432245}%
\pgfsetlinewidth{1.003750pt}%
\definecolor{currentstroke}{rgb}{0.121569,0.466667,0.705882}%
\pgfsetstrokecolor{currentstroke}%
\pgfsetstrokeopacity{0.432245}%
\pgfsetdash{}{0pt}%
\pgfpathmoveto{\pgfqpoint{2.625613in}{1.898974in}}%
\pgfpathcurveto{\pgfqpoint{2.633849in}{1.898974in}}{\pgfqpoint{2.641749in}{1.902246in}}{\pgfqpoint{2.647573in}{1.908070in}}%
\pgfpathcurveto{\pgfqpoint{2.653397in}{1.913894in}}{\pgfqpoint{2.656669in}{1.921794in}}{\pgfqpoint{2.656669in}{1.930030in}}%
\pgfpathcurveto{\pgfqpoint{2.656669in}{1.938266in}}{\pgfqpoint{2.653397in}{1.946167in}}{\pgfqpoint{2.647573in}{1.951990in}}%
\pgfpathcurveto{\pgfqpoint{2.641749in}{1.957814in}}{\pgfqpoint{2.633849in}{1.961087in}}{\pgfqpoint{2.625613in}{1.961087in}}%
\pgfpathcurveto{\pgfqpoint{2.617377in}{1.961087in}}{\pgfqpoint{2.609477in}{1.957814in}}{\pgfqpoint{2.603653in}{1.951990in}}%
\pgfpathcurveto{\pgfqpoint{2.597829in}{1.946167in}}{\pgfqpoint{2.594556in}{1.938266in}}{\pgfqpoint{2.594556in}{1.930030in}}%
\pgfpathcurveto{\pgfqpoint{2.594556in}{1.921794in}}{\pgfqpoint{2.597829in}{1.913894in}}{\pgfqpoint{2.603653in}{1.908070in}}%
\pgfpathcurveto{\pgfqpoint{2.609477in}{1.902246in}}{\pgfqpoint{2.617377in}{1.898974in}}{\pgfqpoint{2.625613in}{1.898974in}}%
\pgfpathclose%
\pgfusepath{stroke,fill}%
\end{pgfscope}%
\begin{pgfscope}%
\pgfpathrectangle{\pgfqpoint{0.100000in}{0.212622in}}{\pgfqpoint{3.696000in}{3.696000in}}%
\pgfusepath{clip}%
\pgfsetbuttcap%
\pgfsetroundjoin%
\definecolor{currentfill}{rgb}{0.121569,0.466667,0.705882}%
\pgfsetfillcolor{currentfill}%
\pgfsetfillopacity{0.432679}%
\pgfsetlinewidth{1.003750pt}%
\definecolor{currentstroke}{rgb}{0.121569,0.466667,0.705882}%
\pgfsetstrokecolor{currentstroke}%
\pgfsetstrokeopacity{0.432679}%
\pgfsetdash{}{0pt}%
\pgfpathmoveto{\pgfqpoint{1.436005in}{2.112765in}}%
\pgfpathcurveto{\pgfqpoint{1.444241in}{2.112765in}}{\pgfqpoint{1.452141in}{2.116038in}}{\pgfqpoint{1.457965in}{2.121862in}}%
\pgfpathcurveto{\pgfqpoint{1.463789in}{2.127686in}}{\pgfqpoint{1.467061in}{2.135586in}}{\pgfqpoint{1.467061in}{2.143822in}}%
\pgfpathcurveto{\pgfqpoint{1.467061in}{2.152058in}}{\pgfqpoint{1.463789in}{2.159958in}}{\pgfqpoint{1.457965in}{2.165782in}}%
\pgfpathcurveto{\pgfqpoint{1.452141in}{2.171606in}}{\pgfqpoint{1.444241in}{2.174878in}}{\pgfqpoint{1.436005in}{2.174878in}}%
\pgfpathcurveto{\pgfqpoint{1.427768in}{2.174878in}}{\pgfqpoint{1.419868in}{2.171606in}}{\pgfqpoint{1.414044in}{2.165782in}}%
\pgfpathcurveto{\pgfqpoint{1.408221in}{2.159958in}}{\pgfqpoint{1.404948in}{2.152058in}}{\pgfqpoint{1.404948in}{2.143822in}}%
\pgfpathcurveto{\pgfqpoint{1.404948in}{2.135586in}}{\pgfqpoint{1.408221in}{2.127686in}}{\pgfqpoint{1.414044in}{2.121862in}}%
\pgfpathcurveto{\pgfqpoint{1.419868in}{2.116038in}}{\pgfqpoint{1.427768in}{2.112765in}}{\pgfqpoint{1.436005in}{2.112765in}}%
\pgfpathclose%
\pgfusepath{stroke,fill}%
\end{pgfscope}%
\begin{pgfscope}%
\pgfpathrectangle{\pgfqpoint{0.100000in}{0.212622in}}{\pgfqpoint{3.696000in}{3.696000in}}%
\pgfusepath{clip}%
\pgfsetbuttcap%
\pgfsetroundjoin%
\definecolor{currentfill}{rgb}{0.121569,0.466667,0.705882}%
\pgfsetfillcolor{currentfill}%
\pgfsetfillopacity{0.433818}%
\pgfsetlinewidth{1.003750pt}%
\definecolor{currentstroke}{rgb}{0.121569,0.466667,0.705882}%
\pgfsetstrokecolor{currentstroke}%
\pgfsetstrokeopacity{0.433818}%
\pgfsetdash{}{0pt}%
\pgfpathmoveto{\pgfqpoint{1.433470in}{2.112881in}}%
\pgfpathcurveto{\pgfqpoint{1.441706in}{2.112881in}}{\pgfqpoint{1.449606in}{2.116153in}}{\pgfqpoint{1.455430in}{2.121977in}}%
\pgfpathcurveto{\pgfqpoint{1.461254in}{2.127801in}}{\pgfqpoint{1.464526in}{2.135701in}}{\pgfqpoint{1.464526in}{2.143938in}}%
\pgfpathcurveto{\pgfqpoint{1.464526in}{2.152174in}}{\pgfqpoint{1.461254in}{2.160074in}}{\pgfqpoint{1.455430in}{2.165898in}}%
\pgfpathcurveto{\pgfqpoint{1.449606in}{2.171722in}}{\pgfqpoint{1.441706in}{2.174994in}}{\pgfqpoint{1.433470in}{2.174994in}}%
\pgfpathcurveto{\pgfqpoint{1.425234in}{2.174994in}}{\pgfqpoint{1.417333in}{2.171722in}}{\pgfqpoint{1.411510in}{2.165898in}}%
\pgfpathcurveto{\pgfqpoint{1.405686in}{2.160074in}}{\pgfqpoint{1.402413in}{2.152174in}}{\pgfqpoint{1.402413in}{2.143938in}}%
\pgfpathcurveto{\pgfqpoint{1.402413in}{2.135701in}}{\pgfqpoint{1.405686in}{2.127801in}}{\pgfqpoint{1.411510in}{2.121977in}}%
\pgfpathcurveto{\pgfqpoint{1.417333in}{2.116153in}}{\pgfqpoint{1.425234in}{2.112881in}}{\pgfqpoint{1.433470in}{2.112881in}}%
\pgfpathclose%
\pgfusepath{stroke,fill}%
\end{pgfscope}%
\begin{pgfscope}%
\pgfpathrectangle{\pgfqpoint{0.100000in}{0.212622in}}{\pgfqpoint{3.696000in}{3.696000in}}%
\pgfusepath{clip}%
\pgfsetbuttcap%
\pgfsetroundjoin%
\definecolor{currentfill}{rgb}{0.121569,0.466667,0.705882}%
\pgfsetfillcolor{currentfill}%
\pgfsetfillopacity{0.433970}%
\pgfsetlinewidth{1.003750pt}%
\definecolor{currentstroke}{rgb}{0.121569,0.466667,0.705882}%
\pgfsetstrokecolor{currentstroke}%
\pgfsetstrokeopacity{0.433970}%
\pgfsetdash{}{0pt}%
\pgfpathmoveto{\pgfqpoint{2.638088in}{1.896089in}}%
\pgfpathcurveto{\pgfqpoint{2.646325in}{1.896089in}}{\pgfqpoint{2.654225in}{1.899361in}}{\pgfqpoint{2.660049in}{1.905185in}}%
\pgfpathcurveto{\pgfqpoint{2.665872in}{1.911009in}}{\pgfqpoint{2.669145in}{1.918909in}}{\pgfqpoint{2.669145in}{1.927146in}}%
\pgfpathcurveto{\pgfqpoint{2.669145in}{1.935382in}}{\pgfqpoint{2.665872in}{1.943282in}}{\pgfqpoint{2.660049in}{1.949106in}}%
\pgfpathcurveto{\pgfqpoint{2.654225in}{1.954930in}}{\pgfqpoint{2.646325in}{1.958202in}}{\pgfqpoint{2.638088in}{1.958202in}}%
\pgfpathcurveto{\pgfqpoint{2.629852in}{1.958202in}}{\pgfqpoint{2.621952in}{1.954930in}}{\pgfqpoint{2.616128in}{1.949106in}}%
\pgfpathcurveto{\pgfqpoint{2.610304in}{1.943282in}}{\pgfqpoint{2.607032in}{1.935382in}}{\pgfqpoint{2.607032in}{1.927146in}}%
\pgfpathcurveto{\pgfqpoint{2.607032in}{1.918909in}}{\pgfqpoint{2.610304in}{1.911009in}}{\pgfqpoint{2.616128in}{1.905185in}}%
\pgfpathcurveto{\pgfqpoint{2.621952in}{1.899361in}}{\pgfqpoint{2.629852in}{1.896089in}}{\pgfqpoint{2.638088in}{1.896089in}}%
\pgfpathclose%
\pgfusepath{stroke,fill}%
\end{pgfscope}%
\begin{pgfscope}%
\pgfpathrectangle{\pgfqpoint{0.100000in}{0.212622in}}{\pgfqpoint{3.696000in}{3.696000in}}%
\pgfusepath{clip}%
\pgfsetbuttcap%
\pgfsetroundjoin%
\definecolor{currentfill}{rgb}{0.121569,0.466667,0.705882}%
\pgfsetfillcolor{currentfill}%
\pgfsetfillopacity{0.434861}%
\pgfsetlinewidth{1.003750pt}%
\definecolor{currentstroke}{rgb}{0.121569,0.466667,0.705882}%
\pgfsetstrokecolor{currentstroke}%
\pgfsetstrokeopacity{0.434861}%
\pgfsetdash{}{0pt}%
\pgfpathmoveto{\pgfqpoint{1.431900in}{2.112967in}}%
\pgfpathcurveto{\pgfqpoint{1.440136in}{2.112967in}}{\pgfqpoint{1.448036in}{2.116239in}}{\pgfqpoint{1.453860in}{2.122063in}}%
\pgfpathcurveto{\pgfqpoint{1.459684in}{2.127887in}}{\pgfqpoint{1.462956in}{2.135787in}}{\pgfqpoint{1.462956in}{2.144023in}}%
\pgfpathcurveto{\pgfqpoint{1.462956in}{2.152259in}}{\pgfqpoint{1.459684in}{2.160159in}}{\pgfqpoint{1.453860in}{2.165983in}}%
\pgfpathcurveto{\pgfqpoint{1.448036in}{2.171807in}}{\pgfqpoint{1.440136in}{2.175080in}}{\pgfqpoint{1.431900in}{2.175080in}}%
\pgfpathcurveto{\pgfqpoint{1.423663in}{2.175080in}}{\pgfqpoint{1.415763in}{2.171807in}}{\pgfqpoint{1.409939in}{2.165983in}}%
\pgfpathcurveto{\pgfqpoint{1.404115in}{2.160159in}}{\pgfqpoint{1.400843in}{2.152259in}}{\pgfqpoint{1.400843in}{2.144023in}}%
\pgfpathcurveto{\pgfqpoint{1.400843in}{2.135787in}}{\pgfqpoint{1.404115in}{2.127887in}}{\pgfqpoint{1.409939in}{2.122063in}}%
\pgfpathcurveto{\pgfqpoint{1.415763in}{2.116239in}}{\pgfqpoint{1.423663in}{2.112967in}}{\pgfqpoint{1.431900in}{2.112967in}}%
\pgfpathclose%
\pgfusepath{stroke,fill}%
\end{pgfscope}%
\begin{pgfscope}%
\pgfpathrectangle{\pgfqpoint{0.100000in}{0.212622in}}{\pgfqpoint{3.696000in}{3.696000in}}%
\pgfusepath{clip}%
\pgfsetbuttcap%
\pgfsetroundjoin%
\definecolor{currentfill}{rgb}{0.121569,0.466667,0.705882}%
\pgfsetfillcolor{currentfill}%
\pgfsetfillopacity{0.435152}%
\pgfsetlinewidth{1.003750pt}%
\definecolor{currentstroke}{rgb}{0.121569,0.466667,0.705882}%
\pgfsetstrokecolor{currentstroke}%
\pgfsetstrokeopacity{0.435152}%
\pgfsetdash{}{0pt}%
\pgfpathmoveto{\pgfqpoint{2.651853in}{1.892032in}}%
\pgfpathcurveto{\pgfqpoint{2.660090in}{1.892032in}}{\pgfqpoint{2.667990in}{1.895304in}}{\pgfqpoint{2.673814in}{1.901128in}}%
\pgfpathcurveto{\pgfqpoint{2.679637in}{1.906952in}}{\pgfqpoint{2.682910in}{1.914852in}}{\pgfqpoint{2.682910in}{1.923088in}}%
\pgfpathcurveto{\pgfqpoint{2.682910in}{1.931324in}}{\pgfqpoint{2.679637in}{1.939224in}}{\pgfqpoint{2.673814in}{1.945048in}}%
\pgfpathcurveto{\pgfqpoint{2.667990in}{1.950872in}}{\pgfqpoint{2.660090in}{1.954145in}}{\pgfqpoint{2.651853in}{1.954145in}}%
\pgfpathcurveto{\pgfqpoint{2.643617in}{1.954145in}}{\pgfqpoint{2.635717in}{1.950872in}}{\pgfqpoint{2.629893in}{1.945048in}}%
\pgfpathcurveto{\pgfqpoint{2.624069in}{1.939224in}}{\pgfqpoint{2.620797in}{1.931324in}}{\pgfqpoint{2.620797in}{1.923088in}}%
\pgfpathcurveto{\pgfqpoint{2.620797in}{1.914852in}}{\pgfqpoint{2.624069in}{1.906952in}}{\pgfqpoint{2.629893in}{1.901128in}}%
\pgfpathcurveto{\pgfqpoint{2.635717in}{1.895304in}}{\pgfqpoint{2.643617in}{1.892032in}}{\pgfqpoint{2.651853in}{1.892032in}}%
\pgfpathclose%
\pgfusepath{stroke,fill}%
\end{pgfscope}%
\begin{pgfscope}%
\pgfpathrectangle{\pgfqpoint{0.100000in}{0.212622in}}{\pgfqpoint{3.696000in}{3.696000in}}%
\pgfusepath{clip}%
\pgfsetbuttcap%
\pgfsetroundjoin%
\definecolor{currentfill}{rgb}{0.121569,0.466667,0.705882}%
\pgfsetfillcolor{currentfill}%
\pgfsetfillopacity{0.435621}%
\pgfsetlinewidth{1.003750pt}%
\definecolor{currentstroke}{rgb}{0.121569,0.466667,0.705882}%
\pgfsetstrokecolor{currentstroke}%
\pgfsetstrokeopacity{0.435621}%
\pgfsetdash{}{0pt}%
\pgfpathmoveto{\pgfqpoint{1.430162in}{2.113057in}}%
\pgfpathcurveto{\pgfqpoint{1.438398in}{2.113057in}}{\pgfqpoint{1.446299in}{2.116329in}}{\pgfqpoint{1.452122in}{2.122153in}}%
\pgfpathcurveto{\pgfqpoint{1.457946in}{2.127977in}}{\pgfqpoint{1.461219in}{2.135877in}}{\pgfqpoint{1.461219in}{2.144113in}}%
\pgfpathcurveto{\pgfqpoint{1.461219in}{2.152349in}}{\pgfqpoint{1.457946in}{2.160249in}}{\pgfqpoint{1.452122in}{2.166073in}}%
\pgfpathcurveto{\pgfqpoint{1.446299in}{2.171897in}}{\pgfqpoint{1.438398in}{2.175170in}}{\pgfqpoint{1.430162in}{2.175170in}}%
\pgfpathcurveto{\pgfqpoint{1.421926in}{2.175170in}}{\pgfqpoint{1.414026in}{2.171897in}}{\pgfqpoint{1.408202in}{2.166073in}}%
\pgfpathcurveto{\pgfqpoint{1.402378in}{2.160249in}}{\pgfqpoint{1.399106in}{2.152349in}}{\pgfqpoint{1.399106in}{2.144113in}}%
\pgfpathcurveto{\pgfqpoint{1.399106in}{2.135877in}}{\pgfqpoint{1.402378in}{2.127977in}}{\pgfqpoint{1.408202in}{2.122153in}}%
\pgfpathcurveto{\pgfqpoint{1.414026in}{2.116329in}}{\pgfqpoint{1.421926in}{2.113057in}}{\pgfqpoint{1.430162in}{2.113057in}}%
\pgfpathclose%
\pgfusepath{stroke,fill}%
\end{pgfscope}%
\begin{pgfscope}%
\pgfpathrectangle{\pgfqpoint{0.100000in}{0.212622in}}{\pgfqpoint{3.696000in}{3.696000in}}%
\pgfusepath{clip}%
\pgfsetbuttcap%
\pgfsetroundjoin%
\definecolor{currentfill}{rgb}{0.121569,0.466667,0.705882}%
\pgfsetfillcolor{currentfill}%
\pgfsetfillopacity{0.436970}%
\pgfsetlinewidth{1.003750pt}%
\definecolor{currentstroke}{rgb}{0.121569,0.466667,0.705882}%
\pgfsetstrokecolor{currentstroke}%
\pgfsetstrokeopacity{0.436970}%
\pgfsetdash{}{0pt}%
\pgfpathmoveto{\pgfqpoint{2.664933in}{1.888827in}}%
\pgfpathcurveto{\pgfqpoint{2.673169in}{1.888827in}}{\pgfqpoint{2.681069in}{1.892100in}}{\pgfqpoint{2.686893in}{1.897924in}}%
\pgfpathcurveto{\pgfqpoint{2.692717in}{1.903748in}}{\pgfqpoint{2.695989in}{1.911648in}}{\pgfqpoint{2.695989in}{1.919884in}}%
\pgfpathcurveto{\pgfqpoint{2.695989in}{1.928120in}}{\pgfqpoint{2.692717in}{1.936020in}}{\pgfqpoint{2.686893in}{1.941844in}}%
\pgfpathcurveto{\pgfqpoint{2.681069in}{1.947668in}}{\pgfqpoint{2.673169in}{1.950940in}}{\pgfqpoint{2.664933in}{1.950940in}}%
\pgfpathcurveto{\pgfqpoint{2.656696in}{1.950940in}}{\pgfqpoint{2.648796in}{1.947668in}}{\pgfqpoint{2.642973in}{1.941844in}}%
\pgfpathcurveto{\pgfqpoint{2.637149in}{1.936020in}}{\pgfqpoint{2.633876in}{1.928120in}}{\pgfqpoint{2.633876in}{1.919884in}}%
\pgfpathcurveto{\pgfqpoint{2.633876in}{1.911648in}}{\pgfqpoint{2.637149in}{1.903748in}}{\pgfqpoint{2.642973in}{1.897924in}}%
\pgfpathcurveto{\pgfqpoint{2.648796in}{1.892100in}}{\pgfqpoint{2.656696in}{1.888827in}}{\pgfqpoint{2.664933in}{1.888827in}}%
\pgfpathclose%
\pgfusepath{stroke,fill}%
\end{pgfscope}%
\begin{pgfscope}%
\pgfpathrectangle{\pgfqpoint{0.100000in}{0.212622in}}{\pgfqpoint{3.696000in}{3.696000in}}%
\pgfusepath{clip}%
\pgfsetbuttcap%
\pgfsetroundjoin%
\definecolor{currentfill}{rgb}{0.121569,0.466667,0.705882}%
\pgfsetfillcolor{currentfill}%
\pgfsetfillopacity{0.437055}%
\pgfsetlinewidth{1.003750pt}%
\definecolor{currentstroke}{rgb}{0.121569,0.466667,0.705882}%
\pgfsetstrokecolor{currentstroke}%
\pgfsetstrokeopacity{0.437055}%
\pgfsetdash{}{0pt}%
\pgfpathmoveto{\pgfqpoint{1.427443in}{2.113165in}}%
\pgfpathcurveto{\pgfqpoint{1.435680in}{2.113165in}}{\pgfqpoint{1.443580in}{2.116437in}}{\pgfqpoint{1.449404in}{2.122261in}}%
\pgfpathcurveto{\pgfqpoint{1.455228in}{2.128085in}}{\pgfqpoint{1.458500in}{2.135985in}}{\pgfqpoint{1.458500in}{2.144221in}}%
\pgfpathcurveto{\pgfqpoint{1.458500in}{2.152457in}}{\pgfqpoint{1.455228in}{2.160357in}}{\pgfqpoint{1.449404in}{2.166181in}}%
\pgfpathcurveto{\pgfqpoint{1.443580in}{2.172005in}}{\pgfqpoint{1.435680in}{2.175278in}}{\pgfqpoint{1.427443in}{2.175278in}}%
\pgfpathcurveto{\pgfqpoint{1.419207in}{2.175278in}}{\pgfqpoint{1.411307in}{2.172005in}}{\pgfqpoint{1.405483in}{2.166181in}}%
\pgfpathcurveto{\pgfqpoint{1.399659in}{2.160357in}}{\pgfqpoint{1.396387in}{2.152457in}}{\pgfqpoint{1.396387in}{2.144221in}}%
\pgfpathcurveto{\pgfqpoint{1.396387in}{2.135985in}}{\pgfqpoint{1.399659in}{2.128085in}}{\pgfqpoint{1.405483in}{2.122261in}}%
\pgfpathcurveto{\pgfqpoint{1.411307in}{2.116437in}}{\pgfqpoint{1.419207in}{2.113165in}}{\pgfqpoint{1.427443in}{2.113165in}}%
\pgfpathclose%
\pgfusepath{stroke,fill}%
\end{pgfscope}%
\begin{pgfscope}%
\pgfpathrectangle{\pgfqpoint{0.100000in}{0.212622in}}{\pgfqpoint{3.696000in}{3.696000in}}%
\pgfusepath{clip}%
\pgfsetbuttcap%
\pgfsetroundjoin%
\definecolor{currentfill}{rgb}{0.121569,0.466667,0.705882}%
\pgfsetfillcolor{currentfill}%
\pgfsetfillopacity{0.437711}%
\pgfsetlinewidth{1.003750pt}%
\definecolor{currentstroke}{rgb}{0.121569,0.466667,0.705882}%
\pgfsetstrokecolor{currentstroke}%
\pgfsetstrokeopacity{0.437711}%
\pgfsetdash{}{0pt}%
\pgfpathmoveto{\pgfqpoint{2.672603in}{1.886810in}}%
\pgfpathcurveto{\pgfqpoint{2.680839in}{1.886810in}}{\pgfqpoint{2.688739in}{1.890082in}}{\pgfqpoint{2.694563in}{1.895906in}}%
\pgfpathcurveto{\pgfqpoint{2.700387in}{1.901730in}}{\pgfqpoint{2.703659in}{1.909630in}}{\pgfqpoint{2.703659in}{1.917866in}}%
\pgfpathcurveto{\pgfqpoint{2.703659in}{1.926103in}}{\pgfqpoint{2.700387in}{1.934003in}}{\pgfqpoint{2.694563in}{1.939827in}}%
\pgfpathcurveto{\pgfqpoint{2.688739in}{1.945651in}}{\pgfqpoint{2.680839in}{1.948923in}}{\pgfqpoint{2.672603in}{1.948923in}}%
\pgfpathcurveto{\pgfqpoint{2.664366in}{1.948923in}}{\pgfqpoint{2.656466in}{1.945651in}}{\pgfqpoint{2.650642in}{1.939827in}}%
\pgfpathcurveto{\pgfqpoint{2.644819in}{1.934003in}}{\pgfqpoint{2.641546in}{1.926103in}}{\pgfqpoint{2.641546in}{1.917866in}}%
\pgfpathcurveto{\pgfqpoint{2.641546in}{1.909630in}}{\pgfqpoint{2.644819in}{1.901730in}}{\pgfqpoint{2.650642in}{1.895906in}}%
\pgfpathcurveto{\pgfqpoint{2.656466in}{1.890082in}}{\pgfqpoint{2.664366in}{1.886810in}}{\pgfqpoint{2.672603in}{1.886810in}}%
\pgfpathclose%
\pgfusepath{stroke,fill}%
\end{pgfscope}%
\begin{pgfscope}%
\pgfpathrectangle{\pgfqpoint{0.100000in}{0.212622in}}{\pgfqpoint{3.696000in}{3.696000in}}%
\pgfusepath{clip}%
\pgfsetbuttcap%
\pgfsetroundjoin%
\definecolor{currentfill}{rgb}{0.121569,0.466667,0.705882}%
\pgfsetfillcolor{currentfill}%
\pgfsetfillopacity{0.438026}%
\pgfsetlinewidth{1.003750pt}%
\definecolor{currentstroke}{rgb}{0.121569,0.466667,0.705882}%
\pgfsetstrokecolor{currentstroke}%
\pgfsetstrokeopacity{0.438026}%
\pgfsetdash{}{0pt}%
\pgfpathmoveto{\pgfqpoint{2.676959in}{1.885635in}}%
\pgfpathcurveto{\pgfqpoint{2.685195in}{1.885635in}}{\pgfqpoint{2.693095in}{1.888908in}}{\pgfqpoint{2.698919in}{1.894731in}}%
\pgfpathcurveto{\pgfqpoint{2.704743in}{1.900555in}}{\pgfqpoint{2.708015in}{1.908455in}}{\pgfqpoint{2.708015in}{1.916692in}}%
\pgfpathcurveto{\pgfqpoint{2.708015in}{1.924928in}}{\pgfqpoint{2.704743in}{1.932828in}}{\pgfqpoint{2.698919in}{1.938652in}}%
\pgfpathcurveto{\pgfqpoint{2.693095in}{1.944476in}}{\pgfqpoint{2.685195in}{1.947748in}}{\pgfqpoint{2.676959in}{1.947748in}}%
\pgfpathcurveto{\pgfqpoint{2.668723in}{1.947748in}}{\pgfqpoint{2.660822in}{1.944476in}}{\pgfqpoint{2.654999in}{1.938652in}}%
\pgfpathcurveto{\pgfqpoint{2.649175in}{1.932828in}}{\pgfqpoint{2.645902in}{1.924928in}}{\pgfqpoint{2.645902in}{1.916692in}}%
\pgfpathcurveto{\pgfqpoint{2.645902in}{1.908455in}}{\pgfqpoint{2.649175in}{1.900555in}}{\pgfqpoint{2.654999in}{1.894731in}}%
\pgfpathcurveto{\pgfqpoint{2.660822in}{1.888908in}}{\pgfqpoint{2.668723in}{1.885635in}}{\pgfqpoint{2.676959in}{1.885635in}}%
\pgfpathclose%
\pgfusepath{stroke,fill}%
\end{pgfscope}%
\begin{pgfscope}%
\pgfpathrectangle{\pgfqpoint{0.100000in}{0.212622in}}{\pgfqpoint{3.696000in}{3.696000in}}%
\pgfusepath{clip}%
\pgfsetbuttcap%
\pgfsetroundjoin%
\definecolor{currentfill}{rgb}{0.121569,0.466667,0.705882}%
\pgfsetfillcolor{currentfill}%
\pgfsetfillopacity{0.438344}%
\pgfsetlinewidth{1.003750pt}%
\definecolor{currentstroke}{rgb}{0.121569,0.466667,0.705882}%
\pgfsetstrokecolor{currentstroke}%
\pgfsetstrokeopacity{0.438344}%
\pgfsetdash{}{0pt}%
\pgfpathmoveto{\pgfqpoint{1.425238in}{2.113259in}}%
\pgfpathcurveto{\pgfqpoint{1.433474in}{2.113259in}}{\pgfqpoint{1.441374in}{2.116531in}}{\pgfqpoint{1.447198in}{2.122355in}}%
\pgfpathcurveto{\pgfqpoint{1.453022in}{2.128179in}}{\pgfqpoint{1.456294in}{2.136079in}}{\pgfqpoint{1.456294in}{2.144316in}}%
\pgfpathcurveto{\pgfqpoint{1.456294in}{2.152552in}}{\pgfqpoint{1.453022in}{2.160452in}}{\pgfqpoint{1.447198in}{2.166276in}}%
\pgfpathcurveto{\pgfqpoint{1.441374in}{2.172100in}}{\pgfqpoint{1.433474in}{2.175372in}}{\pgfqpoint{1.425238in}{2.175372in}}%
\pgfpathcurveto{\pgfqpoint{1.417002in}{2.175372in}}{\pgfqpoint{1.409102in}{2.172100in}}{\pgfqpoint{1.403278in}{2.166276in}}%
\pgfpathcurveto{\pgfqpoint{1.397454in}{2.160452in}}{\pgfqpoint{1.394181in}{2.152552in}}{\pgfqpoint{1.394181in}{2.144316in}}%
\pgfpathcurveto{\pgfqpoint{1.394181in}{2.136079in}}{\pgfqpoint{1.397454in}{2.128179in}}{\pgfqpoint{1.403278in}{2.122355in}}%
\pgfpathcurveto{\pgfqpoint{1.409102in}{2.116531in}}{\pgfqpoint{1.417002in}{2.113259in}}{\pgfqpoint{1.425238in}{2.113259in}}%
\pgfpathclose%
\pgfusepath{stroke,fill}%
\end{pgfscope}%
\begin{pgfscope}%
\pgfpathrectangle{\pgfqpoint{0.100000in}{0.212622in}}{\pgfqpoint{3.696000in}{3.696000in}}%
\pgfusepath{clip}%
\pgfsetbuttcap%
\pgfsetroundjoin%
\definecolor{currentfill}{rgb}{0.121569,0.466667,0.705882}%
\pgfsetfillcolor{currentfill}%
\pgfsetfillopacity{0.438572}%
\pgfsetlinewidth{1.003750pt}%
\definecolor{currentstroke}{rgb}{0.121569,0.466667,0.705882}%
\pgfsetstrokecolor{currentstroke}%
\pgfsetstrokeopacity{0.438572}%
\pgfsetdash{}{0pt}%
\pgfpathmoveto{\pgfqpoint{2.681534in}{1.884348in}}%
\pgfpathcurveto{\pgfqpoint{2.689770in}{1.884348in}}{\pgfqpoint{2.697670in}{1.887620in}}{\pgfqpoint{2.703494in}{1.893444in}}%
\pgfpathcurveto{\pgfqpoint{2.709318in}{1.899268in}}{\pgfqpoint{2.712590in}{1.907168in}}{\pgfqpoint{2.712590in}{1.915404in}}%
\pgfpathcurveto{\pgfqpoint{2.712590in}{1.923641in}}{\pgfqpoint{2.709318in}{1.931541in}}{\pgfqpoint{2.703494in}{1.937365in}}%
\pgfpathcurveto{\pgfqpoint{2.697670in}{1.943189in}}{\pgfqpoint{2.689770in}{1.946461in}}{\pgfqpoint{2.681534in}{1.946461in}}%
\pgfpathcurveto{\pgfqpoint{2.673298in}{1.946461in}}{\pgfqpoint{2.665398in}{1.943189in}}{\pgfqpoint{2.659574in}{1.937365in}}%
\pgfpathcurveto{\pgfqpoint{2.653750in}{1.931541in}}{\pgfqpoint{2.650477in}{1.923641in}}{\pgfqpoint{2.650477in}{1.915404in}}%
\pgfpathcurveto{\pgfqpoint{2.650477in}{1.907168in}}{\pgfqpoint{2.653750in}{1.899268in}}{\pgfqpoint{2.659574in}{1.893444in}}%
\pgfpathcurveto{\pgfqpoint{2.665398in}{1.887620in}}{\pgfqpoint{2.673298in}{1.884348in}}{\pgfqpoint{2.681534in}{1.884348in}}%
\pgfpathclose%
\pgfusepath{stroke,fill}%
\end{pgfscope}%
\begin{pgfscope}%
\pgfpathrectangle{\pgfqpoint{0.100000in}{0.212622in}}{\pgfqpoint{3.696000in}{3.696000in}}%
\pgfusepath{clip}%
\pgfsetbuttcap%
\pgfsetroundjoin%
\definecolor{currentfill}{rgb}{0.121569,0.466667,0.705882}%
\pgfsetfillcolor{currentfill}%
\pgfsetfillopacity{0.439164}%
\pgfsetlinewidth{1.003750pt}%
\definecolor{currentstroke}{rgb}{0.121569,0.466667,0.705882}%
\pgfsetstrokecolor{currentstroke}%
\pgfsetstrokeopacity{0.439164}%
\pgfsetdash{}{0pt}%
\pgfpathmoveto{\pgfqpoint{2.686560in}{1.883279in}}%
\pgfpathcurveto{\pgfqpoint{2.694797in}{1.883279in}}{\pgfqpoint{2.702697in}{1.886551in}}{\pgfqpoint{2.708521in}{1.892375in}}%
\pgfpathcurveto{\pgfqpoint{2.714345in}{1.898199in}}{\pgfqpoint{2.717617in}{1.906099in}}{\pgfqpoint{2.717617in}{1.914336in}}%
\pgfpathcurveto{\pgfqpoint{2.717617in}{1.922572in}}{\pgfqpoint{2.714345in}{1.930472in}}{\pgfqpoint{2.708521in}{1.936296in}}%
\pgfpathcurveto{\pgfqpoint{2.702697in}{1.942120in}}{\pgfqpoint{2.694797in}{1.945392in}}{\pgfqpoint{2.686560in}{1.945392in}}%
\pgfpathcurveto{\pgfqpoint{2.678324in}{1.945392in}}{\pgfqpoint{2.670424in}{1.942120in}}{\pgfqpoint{2.664600in}{1.936296in}}%
\pgfpathcurveto{\pgfqpoint{2.658776in}{1.930472in}}{\pgfqpoint{2.655504in}{1.922572in}}{\pgfqpoint{2.655504in}{1.914336in}}%
\pgfpathcurveto{\pgfqpoint{2.655504in}{1.906099in}}{\pgfqpoint{2.658776in}{1.898199in}}{\pgfqpoint{2.664600in}{1.892375in}}%
\pgfpathcurveto{\pgfqpoint{2.670424in}{1.886551in}}{\pgfqpoint{2.678324in}{1.883279in}}{\pgfqpoint{2.686560in}{1.883279in}}%
\pgfpathclose%
\pgfusepath{stroke,fill}%
\end{pgfscope}%
\begin{pgfscope}%
\pgfpathrectangle{\pgfqpoint{0.100000in}{0.212622in}}{\pgfqpoint{3.696000in}{3.696000in}}%
\pgfusepath{clip}%
\pgfsetbuttcap%
\pgfsetroundjoin%
\definecolor{currentfill}{rgb}{0.121569,0.466667,0.705882}%
\pgfsetfillcolor{currentfill}%
\pgfsetfillopacity{0.439365}%
\pgfsetlinewidth{1.003750pt}%
\definecolor{currentstroke}{rgb}{0.121569,0.466667,0.705882}%
\pgfsetstrokecolor{currentstroke}%
\pgfsetstrokeopacity{0.439365}%
\pgfsetdash{}{0pt}%
\pgfpathmoveto{\pgfqpoint{1.422786in}{2.113386in}}%
\pgfpathcurveto{\pgfqpoint{1.431022in}{2.113386in}}{\pgfqpoint{1.438923in}{2.116658in}}{\pgfqpoint{1.444746in}{2.122482in}}%
\pgfpathcurveto{\pgfqpoint{1.450570in}{2.128306in}}{\pgfqpoint{1.453843in}{2.136206in}}{\pgfqpoint{1.453843in}{2.144442in}}%
\pgfpathcurveto{\pgfqpoint{1.453843in}{2.152679in}}{\pgfqpoint{1.450570in}{2.160579in}}{\pgfqpoint{1.444746in}{2.166402in}}%
\pgfpathcurveto{\pgfqpoint{1.438923in}{2.172226in}}{\pgfqpoint{1.431022in}{2.175499in}}{\pgfqpoint{1.422786in}{2.175499in}}%
\pgfpathcurveto{\pgfqpoint{1.414550in}{2.175499in}}{\pgfqpoint{1.406650in}{2.172226in}}{\pgfqpoint{1.400826in}{2.166402in}}%
\pgfpathcurveto{\pgfqpoint{1.395002in}{2.160579in}}{\pgfqpoint{1.391730in}{2.152679in}}{\pgfqpoint{1.391730in}{2.144442in}}%
\pgfpathcurveto{\pgfqpoint{1.391730in}{2.136206in}}{\pgfqpoint{1.395002in}{2.128306in}}{\pgfqpoint{1.400826in}{2.122482in}}%
\pgfpathcurveto{\pgfqpoint{1.406650in}{2.116658in}}{\pgfqpoint{1.414550in}{2.113386in}}{\pgfqpoint{1.422786in}{2.113386in}}%
\pgfpathclose%
\pgfusepath{stroke,fill}%
\end{pgfscope}%
\begin{pgfscope}%
\pgfpathrectangle{\pgfqpoint{0.100000in}{0.212622in}}{\pgfqpoint{3.696000in}{3.696000in}}%
\pgfusepath{clip}%
\pgfsetbuttcap%
\pgfsetroundjoin%
\definecolor{currentfill}{rgb}{0.121569,0.466667,0.705882}%
\pgfsetfillcolor{currentfill}%
\pgfsetfillopacity{0.439675}%
\pgfsetlinewidth{1.003750pt}%
\definecolor{currentstroke}{rgb}{0.121569,0.466667,0.705882}%
\pgfsetstrokecolor{currentstroke}%
\pgfsetstrokeopacity{0.439675}%
\pgfsetdash{}{0pt}%
\pgfpathmoveto{\pgfqpoint{2.692047in}{1.882051in}}%
\pgfpathcurveto{\pgfqpoint{2.700283in}{1.882051in}}{\pgfqpoint{2.708183in}{1.885324in}}{\pgfqpoint{2.714007in}{1.891147in}}%
\pgfpathcurveto{\pgfqpoint{2.719831in}{1.896971in}}{\pgfqpoint{2.723103in}{1.904871in}}{\pgfqpoint{2.723103in}{1.913108in}}%
\pgfpathcurveto{\pgfqpoint{2.723103in}{1.921344in}}{\pgfqpoint{2.719831in}{1.929244in}}{\pgfqpoint{2.714007in}{1.935068in}}%
\pgfpathcurveto{\pgfqpoint{2.708183in}{1.940892in}}{\pgfqpoint{2.700283in}{1.944164in}}{\pgfqpoint{2.692047in}{1.944164in}}%
\pgfpathcurveto{\pgfqpoint{2.683810in}{1.944164in}}{\pgfqpoint{2.675910in}{1.940892in}}{\pgfqpoint{2.670087in}{1.935068in}}%
\pgfpathcurveto{\pgfqpoint{2.664263in}{1.929244in}}{\pgfqpoint{2.660990in}{1.921344in}}{\pgfqpoint{2.660990in}{1.913108in}}%
\pgfpathcurveto{\pgfqpoint{2.660990in}{1.904871in}}{\pgfqpoint{2.664263in}{1.896971in}}{\pgfqpoint{2.670087in}{1.891147in}}%
\pgfpathcurveto{\pgfqpoint{2.675910in}{1.885324in}}{\pgfqpoint{2.683810in}{1.882051in}}{\pgfqpoint{2.692047in}{1.882051in}}%
\pgfpathclose%
\pgfusepath{stroke,fill}%
\end{pgfscope}%
\begin{pgfscope}%
\pgfpathrectangle{\pgfqpoint{0.100000in}{0.212622in}}{\pgfqpoint{3.696000in}{3.696000in}}%
\pgfusepath{clip}%
\pgfsetbuttcap%
\pgfsetroundjoin%
\definecolor{currentfill}{rgb}{0.121569,0.466667,0.705882}%
\pgfsetfillcolor{currentfill}%
\pgfsetfillopacity{0.440112}%
\pgfsetlinewidth{1.003750pt}%
\definecolor{currentstroke}{rgb}{0.121569,0.466667,0.705882}%
\pgfsetstrokecolor{currentstroke}%
\pgfsetstrokeopacity{0.440112}%
\pgfsetdash{}{0pt}%
\pgfpathmoveto{\pgfqpoint{1.421300in}{2.113303in}}%
\pgfpathcurveto{\pgfqpoint{1.429537in}{2.113303in}}{\pgfqpoint{1.437437in}{2.116575in}}{\pgfqpoint{1.443261in}{2.122399in}}%
\pgfpathcurveto{\pgfqpoint{1.449085in}{2.128223in}}{\pgfqpoint{1.452357in}{2.136123in}}{\pgfqpoint{1.452357in}{2.144359in}}%
\pgfpathcurveto{\pgfqpoint{1.452357in}{2.152595in}}{\pgfqpoint{1.449085in}{2.160495in}}{\pgfqpoint{1.443261in}{2.166319in}}%
\pgfpathcurveto{\pgfqpoint{1.437437in}{2.172143in}}{\pgfqpoint{1.429537in}{2.175416in}}{\pgfqpoint{1.421300in}{2.175416in}}%
\pgfpathcurveto{\pgfqpoint{1.413064in}{2.175416in}}{\pgfqpoint{1.405164in}{2.172143in}}{\pgfqpoint{1.399340in}{2.166319in}}%
\pgfpathcurveto{\pgfqpoint{1.393516in}{2.160495in}}{\pgfqpoint{1.390244in}{2.152595in}}{\pgfqpoint{1.390244in}{2.144359in}}%
\pgfpathcurveto{\pgfqpoint{1.390244in}{2.136123in}}{\pgfqpoint{1.393516in}{2.128223in}}{\pgfqpoint{1.399340in}{2.122399in}}%
\pgfpathcurveto{\pgfqpoint{1.405164in}{2.116575in}}{\pgfqpoint{1.413064in}{2.113303in}}{\pgfqpoint{1.421300in}{2.113303in}}%
\pgfpathclose%
\pgfusepath{stroke,fill}%
\end{pgfscope}%
\begin{pgfscope}%
\pgfpathrectangle{\pgfqpoint{0.100000in}{0.212622in}}{\pgfqpoint{3.696000in}{3.696000in}}%
\pgfusepath{clip}%
\pgfsetbuttcap%
\pgfsetroundjoin%
\definecolor{currentfill}{rgb}{0.121569,0.466667,0.705882}%
\pgfsetfillcolor{currentfill}%
\pgfsetfillopacity{0.440188}%
\pgfsetlinewidth{1.003750pt}%
\definecolor{currentstroke}{rgb}{0.121569,0.466667,0.705882}%
\pgfsetstrokecolor{currentstroke}%
\pgfsetstrokeopacity{0.440188}%
\pgfsetdash{}{0pt}%
\pgfpathmoveto{\pgfqpoint{2.694607in}{1.881649in}}%
\pgfpathcurveto{\pgfqpoint{2.702843in}{1.881649in}}{\pgfqpoint{2.710743in}{1.884921in}}{\pgfqpoint{2.716567in}{1.890745in}}%
\pgfpathcurveto{\pgfqpoint{2.722391in}{1.896569in}}{\pgfqpoint{2.725663in}{1.904469in}}{\pgfqpoint{2.725663in}{1.912706in}}%
\pgfpathcurveto{\pgfqpoint{2.725663in}{1.920942in}}{\pgfqpoint{2.722391in}{1.928842in}}{\pgfqpoint{2.716567in}{1.934666in}}%
\pgfpathcurveto{\pgfqpoint{2.710743in}{1.940490in}}{\pgfqpoint{2.702843in}{1.943762in}}{\pgfqpoint{2.694607in}{1.943762in}}%
\pgfpathcurveto{\pgfqpoint{2.686371in}{1.943762in}}{\pgfqpoint{2.678471in}{1.940490in}}{\pgfqpoint{2.672647in}{1.934666in}}%
\pgfpathcurveto{\pgfqpoint{2.666823in}{1.928842in}}{\pgfqpoint{2.663550in}{1.920942in}}{\pgfqpoint{2.663550in}{1.912706in}}%
\pgfpathcurveto{\pgfqpoint{2.663550in}{1.904469in}}{\pgfqpoint{2.666823in}{1.896569in}}{\pgfqpoint{2.672647in}{1.890745in}}%
\pgfpathcurveto{\pgfqpoint{2.678471in}{1.884921in}}{\pgfqpoint{2.686371in}{1.881649in}}{\pgfqpoint{2.694607in}{1.881649in}}%
\pgfpathclose%
\pgfusepath{stroke,fill}%
\end{pgfscope}%
\begin{pgfscope}%
\pgfpathrectangle{\pgfqpoint{0.100000in}{0.212622in}}{\pgfqpoint{3.696000in}{3.696000in}}%
\pgfusepath{clip}%
\pgfsetbuttcap%
\pgfsetroundjoin%
\definecolor{currentfill}{rgb}{0.121569,0.466667,0.705882}%
\pgfsetfillcolor{currentfill}%
\pgfsetfillopacity{0.440479}%
\pgfsetlinewidth{1.003750pt}%
\definecolor{currentstroke}{rgb}{0.121569,0.466667,0.705882}%
\pgfsetstrokecolor{currentstroke}%
\pgfsetstrokeopacity{0.440479}%
\pgfsetdash{}{0pt}%
\pgfpathmoveto{\pgfqpoint{1.420538in}{2.113361in}}%
\pgfpathcurveto{\pgfqpoint{1.428774in}{2.113361in}}{\pgfqpoint{1.436674in}{2.116634in}}{\pgfqpoint{1.442498in}{2.122458in}}%
\pgfpathcurveto{\pgfqpoint{1.448322in}{2.128282in}}{\pgfqpoint{1.451594in}{2.136182in}}{\pgfqpoint{1.451594in}{2.144418in}}%
\pgfpathcurveto{\pgfqpoint{1.451594in}{2.152654in}}{\pgfqpoint{1.448322in}{2.160554in}}{\pgfqpoint{1.442498in}{2.166378in}}%
\pgfpathcurveto{\pgfqpoint{1.436674in}{2.172202in}}{\pgfqpoint{1.428774in}{2.175474in}}{\pgfqpoint{1.420538in}{2.175474in}}%
\pgfpathcurveto{\pgfqpoint{1.412301in}{2.175474in}}{\pgfqpoint{1.404401in}{2.172202in}}{\pgfqpoint{1.398577in}{2.166378in}}%
\pgfpathcurveto{\pgfqpoint{1.392753in}{2.160554in}}{\pgfqpoint{1.389481in}{2.152654in}}{\pgfqpoint{1.389481in}{2.144418in}}%
\pgfpathcurveto{\pgfqpoint{1.389481in}{2.136182in}}{\pgfqpoint{1.392753in}{2.128282in}}{\pgfqpoint{1.398577in}{2.122458in}}%
\pgfpathcurveto{\pgfqpoint{1.404401in}{2.116634in}}{\pgfqpoint{1.412301in}{2.113361in}}{\pgfqpoint{1.420538in}{2.113361in}}%
\pgfpathclose%
\pgfusepath{stroke,fill}%
\end{pgfscope}%
\begin{pgfscope}%
\pgfpathrectangle{\pgfqpoint{0.100000in}{0.212622in}}{\pgfqpoint{3.696000in}{3.696000in}}%
\pgfusepath{clip}%
\pgfsetbuttcap%
\pgfsetroundjoin%
\definecolor{currentfill}{rgb}{0.121569,0.466667,0.705882}%
\pgfsetfillcolor{currentfill}%
\pgfsetfillopacity{0.440914}%
\pgfsetlinewidth{1.003750pt}%
\definecolor{currentstroke}{rgb}{0.121569,0.466667,0.705882}%
\pgfsetstrokecolor{currentstroke}%
\pgfsetstrokeopacity{0.440914}%
\pgfsetdash{}{0pt}%
\pgfpathmoveto{\pgfqpoint{2.699650in}{1.880293in}}%
\pgfpathcurveto{\pgfqpoint{2.707886in}{1.880293in}}{\pgfqpoint{2.715786in}{1.883565in}}{\pgfqpoint{2.721610in}{1.889389in}}%
\pgfpathcurveto{\pgfqpoint{2.727434in}{1.895213in}}{\pgfqpoint{2.730706in}{1.903113in}}{\pgfqpoint{2.730706in}{1.911349in}}%
\pgfpathcurveto{\pgfqpoint{2.730706in}{1.919586in}}{\pgfqpoint{2.727434in}{1.927486in}}{\pgfqpoint{2.721610in}{1.933310in}}%
\pgfpathcurveto{\pgfqpoint{2.715786in}{1.939133in}}{\pgfqpoint{2.707886in}{1.942406in}}{\pgfqpoint{2.699650in}{1.942406in}}%
\pgfpathcurveto{\pgfqpoint{2.691414in}{1.942406in}}{\pgfqpoint{2.683514in}{1.939133in}}{\pgfqpoint{2.677690in}{1.933310in}}%
\pgfpathcurveto{\pgfqpoint{2.671866in}{1.927486in}}{\pgfqpoint{2.668593in}{1.919586in}}{\pgfqpoint{2.668593in}{1.911349in}}%
\pgfpathcurveto{\pgfqpoint{2.668593in}{1.903113in}}{\pgfqpoint{2.671866in}{1.895213in}}{\pgfqpoint{2.677690in}{1.889389in}}%
\pgfpathcurveto{\pgfqpoint{2.683514in}{1.883565in}}{\pgfqpoint{2.691414in}{1.880293in}}{\pgfqpoint{2.699650in}{1.880293in}}%
\pgfpathclose%
\pgfusepath{stroke,fill}%
\end{pgfscope}%
\begin{pgfscope}%
\pgfpathrectangle{\pgfqpoint{0.100000in}{0.212622in}}{\pgfqpoint{3.696000in}{3.696000in}}%
\pgfusepath{clip}%
\pgfsetbuttcap%
\pgfsetroundjoin%
\definecolor{currentfill}{rgb}{0.121569,0.466667,0.705882}%
\pgfsetfillcolor{currentfill}%
\pgfsetfillopacity{0.441121}%
\pgfsetlinewidth{1.003750pt}%
\definecolor{currentstroke}{rgb}{0.121569,0.466667,0.705882}%
\pgfsetstrokecolor{currentstroke}%
\pgfsetstrokeopacity{0.441121}%
\pgfsetdash{}{0pt}%
\pgfpathmoveto{\pgfqpoint{1.419032in}{2.113419in}}%
\pgfpathcurveto{\pgfqpoint{1.427268in}{2.113419in}}{\pgfqpoint{1.435168in}{2.116691in}}{\pgfqpoint{1.440992in}{2.122515in}}%
\pgfpathcurveto{\pgfqpoint{1.446816in}{2.128339in}}{\pgfqpoint{1.450089in}{2.136239in}}{\pgfqpoint{1.450089in}{2.144475in}}%
\pgfpathcurveto{\pgfqpoint{1.450089in}{2.152712in}}{\pgfqpoint{1.446816in}{2.160612in}}{\pgfqpoint{1.440992in}{2.166436in}}%
\pgfpathcurveto{\pgfqpoint{1.435168in}{2.172260in}}{\pgfqpoint{1.427268in}{2.175532in}}{\pgfqpoint{1.419032in}{2.175532in}}%
\pgfpathcurveto{\pgfqpoint{1.410796in}{2.175532in}}{\pgfqpoint{1.402896in}{2.172260in}}{\pgfqpoint{1.397072in}{2.166436in}}%
\pgfpathcurveto{\pgfqpoint{1.391248in}{2.160612in}}{\pgfqpoint{1.387976in}{2.152712in}}{\pgfqpoint{1.387976in}{2.144475in}}%
\pgfpathcurveto{\pgfqpoint{1.387976in}{2.136239in}}{\pgfqpoint{1.391248in}{2.128339in}}{\pgfqpoint{1.397072in}{2.122515in}}%
\pgfpathcurveto{\pgfqpoint{1.402896in}{2.116691in}}{\pgfqpoint{1.410796in}{2.113419in}}{\pgfqpoint{1.419032in}{2.113419in}}%
\pgfpathclose%
\pgfusepath{stroke,fill}%
\end{pgfscope}%
\begin{pgfscope}%
\pgfpathrectangle{\pgfqpoint{0.100000in}{0.212622in}}{\pgfqpoint{3.696000in}{3.696000in}}%
\pgfusepath{clip}%
\pgfsetbuttcap%
\pgfsetroundjoin%
\definecolor{currentfill}{rgb}{0.121569,0.466667,0.705882}%
\pgfsetfillcolor{currentfill}%
\pgfsetfillopacity{0.441349}%
\pgfsetlinewidth{1.003750pt}%
\definecolor{currentstroke}{rgb}{0.121569,0.466667,0.705882}%
\pgfsetstrokecolor{currentstroke}%
\pgfsetstrokeopacity{0.441349}%
\pgfsetdash{}{0pt}%
\pgfpathmoveto{\pgfqpoint{2.702400in}{1.879733in}}%
\pgfpathcurveto{\pgfqpoint{2.710636in}{1.879733in}}{\pgfqpoint{2.718536in}{1.883005in}}{\pgfqpoint{2.724360in}{1.888829in}}%
\pgfpathcurveto{\pgfqpoint{2.730184in}{1.894653in}}{\pgfqpoint{2.733457in}{1.902553in}}{\pgfqpoint{2.733457in}{1.910789in}}%
\pgfpathcurveto{\pgfqpoint{2.733457in}{1.919025in}}{\pgfqpoint{2.730184in}{1.926925in}}{\pgfqpoint{2.724360in}{1.932749in}}%
\pgfpathcurveto{\pgfqpoint{2.718536in}{1.938573in}}{\pgfqpoint{2.710636in}{1.941846in}}{\pgfqpoint{2.702400in}{1.941846in}}%
\pgfpathcurveto{\pgfqpoint{2.694164in}{1.941846in}}{\pgfqpoint{2.686264in}{1.938573in}}{\pgfqpoint{2.680440in}{1.932749in}}%
\pgfpathcurveto{\pgfqpoint{2.674616in}{1.926925in}}{\pgfqpoint{2.671344in}{1.919025in}}{\pgfqpoint{2.671344in}{1.910789in}}%
\pgfpathcurveto{\pgfqpoint{2.671344in}{1.902553in}}{\pgfqpoint{2.674616in}{1.894653in}}{\pgfqpoint{2.680440in}{1.888829in}}%
\pgfpathcurveto{\pgfqpoint{2.686264in}{1.883005in}}{\pgfqpoint{2.694164in}{1.879733in}}{\pgfqpoint{2.702400in}{1.879733in}}%
\pgfpathclose%
\pgfusepath{stroke,fill}%
\end{pgfscope}%
\begin{pgfscope}%
\pgfpathrectangle{\pgfqpoint{0.100000in}{0.212622in}}{\pgfqpoint{3.696000in}{3.696000in}}%
\pgfusepath{clip}%
\pgfsetbuttcap%
\pgfsetroundjoin%
\definecolor{currentfill}{rgb}{0.121569,0.466667,0.705882}%
\pgfsetfillcolor{currentfill}%
\pgfsetfillopacity{0.441746}%
\pgfsetlinewidth{1.003750pt}%
\definecolor{currentstroke}{rgb}{0.121569,0.466667,0.705882}%
\pgfsetstrokecolor{currentstroke}%
\pgfsetstrokeopacity{0.441746}%
\pgfsetdash{}{0pt}%
\pgfpathmoveto{\pgfqpoint{1.418192in}{2.113527in}}%
\pgfpathcurveto{\pgfqpoint{1.426429in}{2.113527in}}{\pgfqpoint{1.434329in}{2.116799in}}{\pgfqpoint{1.440153in}{2.122623in}}%
\pgfpathcurveto{\pgfqpoint{1.445977in}{2.128447in}}{\pgfqpoint{1.449249in}{2.136347in}}{\pgfqpoint{1.449249in}{2.144583in}}%
\pgfpathcurveto{\pgfqpoint{1.449249in}{2.152819in}}{\pgfqpoint{1.445977in}{2.160719in}}{\pgfqpoint{1.440153in}{2.166543in}}%
\pgfpathcurveto{\pgfqpoint{1.434329in}{2.172367in}}{\pgfqpoint{1.426429in}{2.175640in}}{\pgfqpoint{1.418192in}{2.175640in}}%
\pgfpathcurveto{\pgfqpoint{1.409956in}{2.175640in}}{\pgfqpoint{1.402056in}{2.172367in}}{\pgfqpoint{1.396232in}{2.166543in}}%
\pgfpathcurveto{\pgfqpoint{1.390408in}{2.160719in}}{\pgfqpoint{1.387136in}{2.152819in}}{\pgfqpoint{1.387136in}{2.144583in}}%
\pgfpathcurveto{\pgfqpoint{1.387136in}{2.136347in}}{\pgfqpoint{1.390408in}{2.128447in}}{\pgfqpoint{1.396232in}{2.122623in}}%
\pgfpathcurveto{\pgfqpoint{1.402056in}{2.116799in}}{\pgfqpoint{1.409956in}{2.113527in}}{\pgfqpoint{1.418192in}{2.113527in}}%
\pgfpathclose%
\pgfusepath{stroke,fill}%
\end{pgfscope}%
\begin{pgfscope}%
\pgfpathrectangle{\pgfqpoint{0.100000in}{0.212622in}}{\pgfqpoint{3.696000in}{3.696000in}}%
\pgfusepath{clip}%
\pgfsetbuttcap%
\pgfsetroundjoin%
\definecolor{currentfill}{rgb}{0.121569,0.466667,0.705882}%
\pgfsetfillcolor{currentfill}%
\pgfsetfillopacity{0.441900}%
\pgfsetlinewidth{1.003750pt}%
\definecolor{currentstroke}{rgb}{0.121569,0.466667,0.705882}%
\pgfsetstrokecolor{currentstroke}%
\pgfsetstrokeopacity{0.441900}%
\pgfsetdash{}{0pt}%
\pgfpathmoveto{\pgfqpoint{2.705998in}{1.878911in}}%
\pgfpathcurveto{\pgfqpoint{2.714234in}{1.878911in}}{\pgfqpoint{2.722134in}{1.882183in}}{\pgfqpoint{2.727958in}{1.888007in}}%
\pgfpathcurveto{\pgfqpoint{2.733782in}{1.893831in}}{\pgfqpoint{2.737055in}{1.901731in}}{\pgfqpoint{2.737055in}{1.909967in}}%
\pgfpathcurveto{\pgfqpoint{2.737055in}{1.918203in}}{\pgfqpoint{2.733782in}{1.926103in}}{\pgfqpoint{2.727958in}{1.931927in}}%
\pgfpathcurveto{\pgfqpoint{2.722134in}{1.937751in}}{\pgfqpoint{2.714234in}{1.941024in}}{\pgfqpoint{2.705998in}{1.941024in}}%
\pgfpathcurveto{\pgfqpoint{2.697762in}{1.941024in}}{\pgfqpoint{2.689862in}{1.937751in}}{\pgfqpoint{2.684038in}{1.931927in}}%
\pgfpathcurveto{\pgfqpoint{2.678214in}{1.926103in}}{\pgfqpoint{2.674942in}{1.918203in}}{\pgfqpoint{2.674942in}{1.909967in}}%
\pgfpathcurveto{\pgfqpoint{2.674942in}{1.901731in}}{\pgfqpoint{2.678214in}{1.893831in}}{\pgfqpoint{2.684038in}{1.888007in}}%
\pgfpathcurveto{\pgfqpoint{2.689862in}{1.882183in}}{\pgfqpoint{2.697762in}{1.878911in}}{\pgfqpoint{2.705998in}{1.878911in}}%
\pgfpathclose%
\pgfusepath{stroke,fill}%
\end{pgfscope}%
\begin{pgfscope}%
\pgfpathrectangle{\pgfqpoint{0.100000in}{0.212622in}}{\pgfqpoint{3.696000in}{3.696000in}}%
\pgfusepath{clip}%
\pgfsetbuttcap%
\pgfsetroundjoin%
\definecolor{currentfill}{rgb}{0.121569,0.466667,0.705882}%
\pgfsetfillcolor{currentfill}%
\pgfsetfillopacity{0.442080}%
\pgfsetlinewidth{1.003750pt}%
\definecolor{currentstroke}{rgb}{0.121569,0.466667,0.705882}%
\pgfsetstrokecolor{currentstroke}%
\pgfsetstrokeopacity{0.442080}%
\pgfsetdash{}{0pt}%
\pgfpathmoveto{\pgfqpoint{1.417320in}{2.113575in}}%
\pgfpathcurveto{\pgfqpoint{1.425557in}{2.113575in}}{\pgfqpoint{1.433457in}{2.116847in}}{\pgfqpoint{1.439281in}{2.122671in}}%
\pgfpathcurveto{\pgfqpoint{1.445105in}{2.128495in}}{\pgfqpoint{1.448377in}{2.136395in}}{\pgfqpoint{1.448377in}{2.144632in}}%
\pgfpathcurveto{\pgfqpoint{1.448377in}{2.152868in}}{\pgfqpoint{1.445105in}{2.160768in}}{\pgfqpoint{1.439281in}{2.166592in}}%
\pgfpathcurveto{\pgfqpoint{1.433457in}{2.172416in}}{\pgfqpoint{1.425557in}{2.175688in}}{\pgfqpoint{1.417320in}{2.175688in}}%
\pgfpathcurveto{\pgfqpoint{1.409084in}{2.175688in}}{\pgfqpoint{1.401184in}{2.172416in}}{\pgfqpoint{1.395360in}{2.166592in}}%
\pgfpathcurveto{\pgfqpoint{1.389536in}{2.160768in}}{\pgfqpoint{1.386264in}{2.152868in}}{\pgfqpoint{1.386264in}{2.144632in}}%
\pgfpathcurveto{\pgfqpoint{1.386264in}{2.136395in}}{\pgfqpoint{1.389536in}{2.128495in}}{\pgfqpoint{1.395360in}{2.122671in}}%
\pgfpathcurveto{\pgfqpoint{1.401184in}{2.116847in}}{\pgfqpoint{1.409084in}{2.113575in}}{\pgfqpoint{1.417320in}{2.113575in}}%
\pgfpathclose%
\pgfusepath{stroke,fill}%
\end{pgfscope}%
\begin{pgfscope}%
\pgfpathrectangle{\pgfqpoint{0.100000in}{0.212622in}}{\pgfqpoint{3.696000in}{3.696000in}}%
\pgfusepath{clip}%
\pgfsetbuttcap%
\pgfsetroundjoin%
\definecolor{currentfill}{rgb}{0.121569,0.466667,0.705882}%
\pgfsetfillcolor{currentfill}%
\pgfsetfillopacity{0.442297}%
\pgfsetlinewidth{1.003750pt}%
\definecolor{currentstroke}{rgb}{0.121569,0.466667,0.705882}%
\pgfsetstrokecolor{currentstroke}%
\pgfsetstrokeopacity{0.442297}%
\pgfsetdash{}{0pt}%
\pgfpathmoveto{\pgfqpoint{2.710487in}{1.877917in}}%
\pgfpathcurveto{\pgfqpoint{2.718724in}{1.877917in}}{\pgfqpoint{2.726624in}{1.881189in}}{\pgfqpoint{2.732448in}{1.887013in}}%
\pgfpathcurveto{\pgfqpoint{2.738272in}{1.892837in}}{\pgfqpoint{2.741544in}{1.900737in}}{\pgfqpoint{2.741544in}{1.908974in}}%
\pgfpathcurveto{\pgfqpoint{2.741544in}{1.917210in}}{\pgfqpoint{2.738272in}{1.925110in}}{\pgfqpoint{2.732448in}{1.930934in}}%
\pgfpathcurveto{\pgfqpoint{2.726624in}{1.936758in}}{\pgfqpoint{2.718724in}{1.940030in}}{\pgfqpoint{2.710487in}{1.940030in}}%
\pgfpathcurveto{\pgfqpoint{2.702251in}{1.940030in}}{\pgfqpoint{2.694351in}{1.936758in}}{\pgfqpoint{2.688527in}{1.930934in}}%
\pgfpathcurveto{\pgfqpoint{2.682703in}{1.925110in}}{\pgfqpoint{2.679431in}{1.917210in}}{\pgfqpoint{2.679431in}{1.908974in}}%
\pgfpathcurveto{\pgfqpoint{2.679431in}{1.900737in}}{\pgfqpoint{2.682703in}{1.892837in}}{\pgfqpoint{2.688527in}{1.887013in}}%
\pgfpathcurveto{\pgfqpoint{2.694351in}{1.881189in}}{\pgfqpoint{2.702251in}{1.877917in}}{\pgfqpoint{2.710487in}{1.877917in}}%
\pgfpathclose%
\pgfusepath{stroke,fill}%
\end{pgfscope}%
\begin{pgfscope}%
\pgfpathrectangle{\pgfqpoint{0.100000in}{0.212622in}}{\pgfqpoint{3.696000in}{3.696000in}}%
\pgfusepath{clip}%
\pgfsetbuttcap%
\pgfsetroundjoin%
\definecolor{currentfill}{rgb}{0.121569,0.466667,0.705882}%
\pgfsetfillcolor{currentfill}%
\pgfsetfillopacity{0.442368}%
\pgfsetlinewidth{1.003750pt}%
\definecolor{currentstroke}{rgb}{0.121569,0.466667,0.705882}%
\pgfsetstrokecolor{currentstroke}%
\pgfsetstrokeopacity{0.442368}%
\pgfsetdash{}{0pt}%
\pgfpathmoveto{\pgfqpoint{1.417872in}{2.113817in}}%
\pgfpathcurveto{\pgfqpoint{1.426108in}{2.113817in}}{\pgfqpoint{1.434008in}{2.117089in}}{\pgfqpoint{1.439832in}{2.122913in}}%
\pgfpathcurveto{\pgfqpoint{1.445656in}{2.128737in}}{\pgfqpoint{1.448928in}{2.136637in}}{\pgfqpoint{1.448928in}{2.144873in}}%
\pgfpathcurveto{\pgfqpoint{1.448928in}{2.153109in}}{\pgfqpoint{1.445656in}{2.161009in}}{\pgfqpoint{1.439832in}{2.166833in}}%
\pgfpathcurveto{\pgfqpoint{1.434008in}{2.172657in}}{\pgfqpoint{1.426108in}{2.175930in}}{\pgfqpoint{1.417872in}{2.175930in}}%
\pgfpathcurveto{\pgfqpoint{1.409635in}{2.175930in}}{\pgfqpoint{1.401735in}{2.172657in}}{\pgfqpoint{1.395911in}{2.166833in}}%
\pgfpathcurveto{\pgfqpoint{1.390088in}{2.161009in}}{\pgfqpoint{1.386815in}{2.153109in}}{\pgfqpoint{1.386815in}{2.144873in}}%
\pgfpathcurveto{\pgfqpoint{1.386815in}{2.136637in}}{\pgfqpoint{1.390088in}{2.128737in}}{\pgfqpoint{1.395911in}{2.122913in}}%
\pgfpathcurveto{\pgfqpoint{1.401735in}{2.117089in}}{\pgfqpoint{1.409635in}{2.113817in}}{\pgfqpoint{1.417872in}{2.113817in}}%
\pgfpathclose%
\pgfusepath{stroke,fill}%
\end{pgfscope}%
\begin{pgfscope}%
\pgfpathrectangle{\pgfqpoint{0.100000in}{0.212622in}}{\pgfqpoint{3.696000in}{3.696000in}}%
\pgfusepath{clip}%
\pgfsetbuttcap%
\pgfsetroundjoin%
\definecolor{currentfill}{rgb}{0.121569,0.466667,0.705882}%
\pgfsetfillcolor{currentfill}%
\pgfsetfillopacity{0.442868}%
\pgfsetlinewidth{1.003750pt}%
\definecolor{currentstroke}{rgb}{0.121569,0.466667,0.705882}%
\pgfsetstrokecolor{currentstroke}%
\pgfsetstrokeopacity{0.442868}%
\pgfsetdash{}{0pt}%
\pgfpathmoveto{\pgfqpoint{1.416923in}{2.113832in}}%
\pgfpathcurveto{\pgfqpoint{1.425159in}{2.113832in}}{\pgfqpoint{1.433059in}{2.117104in}}{\pgfqpoint{1.438883in}{2.122928in}}%
\pgfpathcurveto{\pgfqpoint{1.444707in}{2.128752in}}{\pgfqpoint{1.447980in}{2.136652in}}{\pgfqpoint{1.447980in}{2.144888in}}%
\pgfpathcurveto{\pgfqpoint{1.447980in}{2.153124in}}{\pgfqpoint{1.444707in}{2.161025in}}{\pgfqpoint{1.438883in}{2.166848in}}%
\pgfpathcurveto{\pgfqpoint{1.433059in}{2.172672in}}{\pgfqpoint{1.425159in}{2.175945in}}{\pgfqpoint{1.416923in}{2.175945in}}%
\pgfpathcurveto{\pgfqpoint{1.408687in}{2.175945in}}{\pgfqpoint{1.400787in}{2.172672in}}{\pgfqpoint{1.394963in}{2.166848in}}%
\pgfpathcurveto{\pgfqpoint{1.389139in}{2.161025in}}{\pgfqpoint{1.385867in}{2.153124in}}{\pgfqpoint{1.385867in}{2.144888in}}%
\pgfpathcurveto{\pgfqpoint{1.385867in}{2.136652in}}{\pgfqpoint{1.389139in}{2.128752in}}{\pgfqpoint{1.394963in}{2.122928in}}%
\pgfpathcurveto{\pgfqpoint{1.400787in}{2.117104in}}{\pgfqpoint{1.408687in}{2.113832in}}{\pgfqpoint{1.416923in}{2.113832in}}%
\pgfpathclose%
\pgfusepath{stroke,fill}%
\end{pgfscope}%
\begin{pgfscope}%
\pgfpathrectangle{\pgfqpoint{0.100000in}{0.212622in}}{\pgfqpoint{3.696000in}{3.696000in}}%
\pgfusepath{clip}%
\pgfsetbuttcap%
\pgfsetroundjoin%
\definecolor{currentfill}{rgb}{0.121569,0.466667,0.705882}%
\pgfsetfillcolor{currentfill}%
\pgfsetfillopacity{0.443151}%
\pgfsetlinewidth{1.003750pt}%
\definecolor{currentstroke}{rgb}{0.121569,0.466667,0.705882}%
\pgfsetstrokecolor{currentstroke}%
\pgfsetstrokeopacity{0.443151}%
\pgfsetdash{}{0pt}%
\pgfpathmoveto{\pgfqpoint{2.715153in}{1.877278in}}%
\pgfpathcurveto{\pgfqpoint{2.723389in}{1.877278in}}{\pgfqpoint{2.731289in}{1.880551in}}{\pgfqpoint{2.737113in}{1.886375in}}%
\pgfpathcurveto{\pgfqpoint{2.742937in}{1.892198in}}{\pgfqpoint{2.746209in}{1.900099in}}{\pgfqpoint{2.746209in}{1.908335in}}%
\pgfpathcurveto{\pgfqpoint{2.746209in}{1.916571in}}{\pgfqpoint{2.742937in}{1.924471in}}{\pgfqpoint{2.737113in}{1.930295in}}%
\pgfpathcurveto{\pgfqpoint{2.731289in}{1.936119in}}{\pgfqpoint{2.723389in}{1.939391in}}{\pgfqpoint{2.715153in}{1.939391in}}%
\pgfpathcurveto{\pgfqpoint{2.706916in}{1.939391in}}{\pgfqpoint{2.699016in}{1.936119in}}{\pgfqpoint{2.693192in}{1.930295in}}%
\pgfpathcurveto{\pgfqpoint{2.687368in}{1.924471in}}{\pgfqpoint{2.684096in}{1.916571in}}{\pgfqpoint{2.684096in}{1.908335in}}%
\pgfpathcurveto{\pgfqpoint{2.684096in}{1.900099in}}{\pgfqpoint{2.687368in}{1.892198in}}{\pgfqpoint{2.693192in}{1.886375in}}%
\pgfpathcurveto{\pgfqpoint{2.699016in}{1.880551in}}{\pgfqpoint{2.706916in}{1.877278in}}{\pgfqpoint{2.715153in}{1.877278in}}%
\pgfpathclose%
\pgfusepath{stroke,fill}%
\end{pgfscope}%
\begin{pgfscope}%
\pgfpathrectangle{\pgfqpoint{0.100000in}{0.212622in}}{\pgfqpoint{3.696000in}{3.696000in}}%
\pgfusepath{clip}%
\pgfsetbuttcap%
\pgfsetroundjoin%
\definecolor{currentfill}{rgb}{0.121569,0.466667,0.705882}%
\pgfsetfillcolor{currentfill}%
\pgfsetfillopacity{0.443739}%
\pgfsetlinewidth{1.003750pt}%
\definecolor{currentstroke}{rgb}{0.121569,0.466667,0.705882}%
\pgfsetstrokecolor{currentstroke}%
\pgfsetstrokeopacity{0.443739}%
\pgfsetdash{}{0pt}%
\pgfpathmoveto{\pgfqpoint{1.414846in}{2.113940in}}%
\pgfpathcurveto{\pgfqpoint{1.423083in}{2.113940in}}{\pgfqpoint{1.430983in}{2.117212in}}{\pgfqpoint{1.436807in}{2.123036in}}%
\pgfpathcurveto{\pgfqpoint{1.442630in}{2.128860in}}{\pgfqpoint{1.445903in}{2.136760in}}{\pgfqpoint{1.445903in}{2.144996in}}%
\pgfpathcurveto{\pgfqpoint{1.445903in}{2.153233in}}{\pgfqpoint{1.442630in}{2.161133in}}{\pgfqpoint{1.436807in}{2.166957in}}%
\pgfpathcurveto{\pgfqpoint{1.430983in}{2.172781in}}{\pgfqpoint{1.423083in}{2.176053in}}{\pgfqpoint{1.414846in}{2.176053in}}%
\pgfpathcurveto{\pgfqpoint{1.406610in}{2.176053in}}{\pgfqpoint{1.398710in}{2.172781in}}{\pgfqpoint{1.392886in}{2.166957in}}%
\pgfpathcurveto{\pgfqpoint{1.387062in}{2.161133in}}{\pgfqpoint{1.383790in}{2.153233in}}{\pgfqpoint{1.383790in}{2.144996in}}%
\pgfpathcurveto{\pgfqpoint{1.383790in}{2.136760in}}{\pgfqpoint{1.387062in}{2.128860in}}{\pgfqpoint{1.392886in}{2.123036in}}%
\pgfpathcurveto{\pgfqpoint{1.398710in}{2.117212in}}{\pgfqpoint{1.406610in}{2.113940in}}{\pgfqpoint{1.414846in}{2.113940in}}%
\pgfpathclose%
\pgfusepath{stroke,fill}%
\end{pgfscope}%
\begin{pgfscope}%
\pgfpathrectangle{\pgfqpoint{0.100000in}{0.212622in}}{\pgfqpoint{3.696000in}{3.696000in}}%
\pgfusepath{clip}%
\pgfsetbuttcap%
\pgfsetroundjoin%
\definecolor{currentfill}{rgb}{0.121569,0.466667,0.705882}%
\pgfsetfillcolor{currentfill}%
\pgfsetfillopacity{0.443797}%
\pgfsetlinewidth{1.003750pt}%
\definecolor{currentstroke}{rgb}{0.121569,0.466667,0.705882}%
\pgfsetstrokecolor{currentstroke}%
\pgfsetstrokeopacity{0.443797}%
\pgfsetdash{}{0pt}%
\pgfpathmoveto{\pgfqpoint{2.720523in}{1.876186in}}%
\pgfpathcurveto{\pgfqpoint{2.728759in}{1.876186in}}{\pgfqpoint{2.736659in}{1.879459in}}{\pgfqpoint{2.742483in}{1.885282in}}%
\pgfpathcurveto{\pgfqpoint{2.748307in}{1.891106in}}{\pgfqpoint{2.751580in}{1.899006in}}{\pgfqpoint{2.751580in}{1.907243in}}%
\pgfpathcurveto{\pgfqpoint{2.751580in}{1.915479in}}{\pgfqpoint{2.748307in}{1.923379in}}{\pgfqpoint{2.742483in}{1.929203in}}%
\pgfpathcurveto{\pgfqpoint{2.736659in}{1.935027in}}{\pgfqpoint{2.728759in}{1.938299in}}{\pgfqpoint{2.720523in}{1.938299in}}%
\pgfpathcurveto{\pgfqpoint{2.712287in}{1.938299in}}{\pgfqpoint{2.704387in}{1.935027in}}{\pgfqpoint{2.698563in}{1.929203in}}%
\pgfpathcurveto{\pgfqpoint{2.692739in}{1.923379in}}{\pgfqpoint{2.689467in}{1.915479in}}{\pgfqpoint{2.689467in}{1.907243in}}%
\pgfpathcurveto{\pgfqpoint{2.689467in}{1.899006in}}{\pgfqpoint{2.692739in}{1.891106in}}{\pgfqpoint{2.698563in}{1.885282in}}%
\pgfpathcurveto{\pgfqpoint{2.704387in}{1.879459in}}{\pgfqpoint{2.712287in}{1.876186in}}{\pgfqpoint{2.720523in}{1.876186in}}%
\pgfpathclose%
\pgfusepath{stroke,fill}%
\end{pgfscope}%
\begin{pgfscope}%
\pgfpathrectangle{\pgfqpoint{0.100000in}{0.212622in}}{\pgfqpoint{3.696000in}{3.696000in}}%
\pgfusepath{clip}%
\pgfsetbuttcap%
\pgfsetroundjoin%
\definecolor{currentfill}{rgb}{0.121569,0.466667,0.705882}%
\pgfsetfillcolor{currentfill}%
\pgfsetfillopacity{0.444525}%
\pgfsetlinewidth{1.003750pt}%
\definecolor{currentstroke}{rgb}{0.121569,0.466667,0.705882}%
\pgfsetstrokecolor{currentstroke}%
\pgfsetstrokeopacity{0.444525}%
\pgfsetdash{}{0pt}%
\pgfpathmoveto{\pgfqpoint{1.413595in}{2.114023in}}%
\pgfpathcurveto{\pgfqpoint{1.421831in}{2.114023in}}{\pgfqpoint{1.429731in}{2.117295in}}{\pgfqpoint{1.435555in}{2.123119in}}%
\pgfpathcurveto{\pgfqpoint{1.441379in}{2.128943in}}{\pgfqpoint{1.444651in}{2.136843in}}{\pgfqpoint{1.444651in}{2.145080in}}%
\pgfpathcurveto{\pgfqpoint{1.444651in}{2.153316in}}{\pgfqpoint{1.441379in}{2.161216in}}{\pgfqpoint{1.435555in}{2.167040in}}%
\pgfpathcurveto{\pgfqpoint{1.429731in}{2.172864in}}{\pgfqpoint{1.421831in}{2.176136in}}{\pgfqpoint{1.413595in}{2.176136in}}%
\pgfpathcurveto{\pgfqpoint{1.405358in}{2.176136in}}{\pgfqpoint{1.397458in}{2.172864in}}{\pgfqpoint{1.391634in}{2.167040in}}%
\pgfpathcurveto{\pgfqpoint{1.385811in}{2.161216in}}{\pgfqpoint{1.382538in}{2.153316in}}{\pgfqpoint{1.382538in}{2.145080in}}%
\pgfpathcurveto{\pgfqpoint{1.382538in}{2.136843in}}{\pgfqpoint{1.385811in}{2.128943in}}{\pgfqpoint{1.391634in}{2.123119in}}%
\pgfpathcurveto{\pgfqpoint{1.397458in}{2.117295in}}{\pgfqpoint{1.405358in}{2.114023in}}{\pgfqpoint{1.413595in}{2.114023in}}%
\pgfpathclose%
\pgfusepath{stroke,fill}%
\end{pgfscope}%
\begin{pgfscope}%
\pgfpathrectangle{\pgfqpoint{0.100000in}{0.212622in}}{\pgfqpoint{3.696000in}{3.696000in}}%
\pgfusepath{clip}%
\pgfsetbuttcap%
\pgfsetroundjoin%
\definecolor{currentfill}{rgb}{0.121569,0.466667,0.705882}%
\pgfsetfillcolor{currentfill}%
\pgfsetfillopacity{0.445068}%
\pgfsetlinewidth{1.003750pt}%
\definecolor{currentstroke}{rgb}{0.121569,0.466667,0.705882}%
\pgfsetstrokecolor{currentstroke}%
\pgfsetstrokeopacity{0.445068}%
\pgfsetdash{}{0pt}%
\pgfpathmoveto{\pgfqpoint{2.725782in}{1.875700in}}%
\pgfpathcurveto{\pgfqpoint{2.734019in}{1.875700in}}{\pgfqpoint{2.741919in}{1.878972in}}{\pgfqpoint{2.747743in}{1.884796in}}%
\pgfpathcurveto{\pgfqpoint{2.753567in}{1.890620in}}{\pgfqpoint{2.756839in}{1.898520in}}{\pgfqpoint{2.756839in}{1.906756in}}%
\pgfpathcurveto{\pgfqpoint{2.756839in}{1.914992in}}{\pgfqpoint{2.753567in}{1.922893in}}{\pgfqpoint{2.747743in}{1.928716in}}%
\pgfpathcurveto{\pgfqpoint{2.741919in}{1.934540in}}{\pgfqpoint{2.734019in}{1.937813in}}{\pgfqpoint{2.725782in}{1.937813in}}%
\pgfpathcurveto{\pgfqpoint{2.717546in}{1.937813in}}{\pgfqpoint{2.709646in}{1.934540in}}{\pgfqpoint{2.703822in}{1.928716in}}%
\pgfpathcurveto{\pgfqpoint{2.697998in}{1.922893in}}{\pgfqpoint{2.694726in}{1.914992in}}{\pgfqpoint{2.694726in}{1.906756in}}%
\pgfpathcurveto{\pgfqpoint{2.694726in}{1.898520in}}{\pgfqpoint{2.697998in}{1.890620in}}{\pgfqpoint{2.703822in}{1.884796in}}%
\pgfpathcurveto{\pgfqpoint{2.709646in}{1.878972in}}{\pgfqpoint{2.717546in}{1.875700in}}{\pgfqpoint{2.725782in}{1.875700in}}%
\pgfpathclose%
\pgfusepath{stroke,fill}%
\end{pgfscope}%
\begin{pgfscope}%
\pgfpathrectangle{\pgfqpoint{0.100000in}{0.212622in}}{\pgfqpoint{3.696000in}{3.696000in}}%
\pgfusepath{clip}%
\pgfsetbuttcap%
\pgfsetroundjoin%
\definecolor{currentfill}{rgb}{0.121569,0.466667,0.705882}%
\pgfsetfillcolor{currentfill}%
\pgfsetfillopacity{0.445848}%
\pgfsetlinewidth{1.003750pt}%
\definecolor{currentstroke}{rgb}{0.121569,0.466667,0.705882}%
\pgfsetstrokecolor{currentstroke}%
\pgfsetstrokeopacity{0.445848}%
\pgfsetdash{}{0pt}%
\pgfpathmoveto{\pgfqpoint{1.410364in}{2.114302in}}%
\pgfpathcurveto{\pgfqpoint{1.418601in}{2.114302in}}{\pgfqpoint{1.426501in}{2.117575in}}{\pgfqpoint{1.432325in}{2.123399in}}%
\pgfpathcurveto{\pgfqpoint{1.438149in}{2.129223in}}{\pgfqpoint{1.441421in}{2.137123in}}{\pgfqpoint{1.441421in}{2.145359in}}%
\pgfpathcurveto{\pgfqpoint{1.441421in}{2.153595in}}{\pgfqpoint{1.438149in}{2.161495in}}{\pgfqpoint{1.432325in}{2.167319in}}%
\pgfpathcurveto{\pgfqpoint{1.426501in}{2.173143in}}{\pgfqpoint{1.418601in}{2.176415in}}{\pgfqpoint{1.410364in}{2.176415in}}%
\pgfpathcurveto{\pgfqpoint{1.402128in}{2.176415in}}{\pgfqpoint{1.394228in}{2.173143in}}{\pgfqpoint{1.388404in}{2.167319in}}%
\pgfpathcurveto{\pgfqpoint{1.382580in}{2.161495in}}{\pgfqpoint{1.379308in}{2.153595in}}{\pgfqpoint{1.379308in}{2.145359in}}%
\pgfpathcurveto{\pgfqpoint{1.379308in}{2.137123in}}{\pgfqpoint{1.382580in}{2.129223in}}{\pgfqpoint{1.388404in}{2.123399in}}%
\pgfpathcurveto{\pgfqpoint{1.394228in}{2.117575in}}{\pgfqpoint{1.402128in}{2.114302in}}{\pgfqpoint{1.410364in}{2.114302in}}%
\pgfpathclose%
\pgfusepath{stroke,fill}%
\end{pgfscope}%
\begin{pgfscope}%
\pgfpathrectangle{\pgfqpoint{0.100000in}{0.212622in}}{\pgfqpoint{3.696000in}{3.696000in}}%
\pgfusepath{clip}%
\pgfsetbuttcap%
\pgfsetroundjoin%
\definecolor{currentfill}{rgb}{0.121569,0.466667,0.705882}%
\pgfsetfillcolor{currentfill}%
\pgfsetfillopacity{0.445979}%
\pgfsetlinewidth{1.003750pt}%
\definecolor{currentstroke}{rgb}{0.121569,0.466667,0.705882}%
\pgfsetstrokecolor{currentstroke}%
\pgfsetstrokeopacity{0.445979}%
\pgfsetdash{}{0pt}%
\pgfpathmoveto{\pgfqpoint{2.732348in}{1.874321in}}%
\pgfpathcurveto{\pgfqpoint{2.740584in}{1.874321in}}{\pgfqpoint{2.748484in}{1.877594in}}{\pgfqpoint{2.754308in}{1.883418in}}%
\pgfpathcurveto{\pgfqpoint{2.760132in}{1.889242in}}{\pgfqpoint{2.763405in}{1.897142in}}{\pgfqpoint{2.763405in}{1.905378in}}%
\pgfpathcurveto{\pgfqpoint{2.763405in}{1.913614in}}{\pgfqpoint{2.760132in}{1.921514in}}{\pgfqpoint{2.754308in}{1.927338in}}%
\pgfpathcurveto{\pgfqpoint{2.748484in}{1.933162in}}{\pgfqpoint{2.740584in}{1.936434in}}{\pgfqpoint{2.732348in}{1.936434in}}%
\pgfpathcurveto{\pgfqpoint{2.724112in}{1.936434in}}{\pgfqpoint{2.716212in}{1.933162in}}{\pgfqpoint{2.710388in}{1.927338in}}%
\pgfpathcurveto{\pgfqpoint{2.704564in}{1.921514in}}{\pgfqpoint{2.701292in}{1.913614in}}{\pgfqpoint{2.701292in}{1.905378in}}%
\pgfpathcurveto{\pgfqpoint{2.701292in}{1.897142in}}{\pgfqpoint{2.704564in}{1.889242in}}{\pgfqpoint{2.710388in}{1.883418in}}%
\pgfpathcurveto{\pgfqpoint{2.716212in}{1.877594in}}{\pgfqpoint{2.724112in}{1.874321in}}{\pgfqpoint{2.732348in}{1.874321in}}%
\pgfpathclose%
\pgfusepath{stroke,fill}%
\end{pgfscope}%
\begin{pgfscope}%
\pgfpathrectangle{\pgfqpoint{0.100000in}{0.212622in}}{\pgfqpoint{3.696000in}{3.696000in}}%
\pgfusepath{clip}%
\pgfsetbuttcap%
\pgfsetroundjoin%
\definecolor{currentfill}{rgb}{0.121569,0.466667,0.705882}%
\pgfsetfillcolor{currentfill}%
\pgfsetfillopacity{0.447135}%
\pgfsetlinewidth{1.003750pt}%
\definecolor{currentstroke}{rgb}{0.121569,0.466667,0.705882}%
\pgfsetstrokecolor{currentstroke}%
\pgfsetstrokeopacity{0.447135}%
\pgfsetdash{}{0pt}%
\pgfpathmoveto{\pgfqpoint{2.740084in}{1.872751in}}%
\pgfpathcurveto{\pgfqpoint{2.748320in}{1.872751in}}{\pgfqpoint{2.756220in}{1.876023in}}{\pgfqpoint{2.762044in}{1.881847in}}%
\pgfpathcurveto{\pgfqpoint{2.767868in}{1.887671in}}{\pgfqpoint{2.771140in}{1.895571in}}{\pgfqpoint{2.771140in}{1.903807in}}%
\pgfpathcurveto{\pgfqpoint{2.771140in}{1.912043in}}{\pgfqpoint{2.767868in}{1.919944in}}{\pgfqpoint{2.762044in}{1.925767in}}%
\pgfpathcurveto{\pgfqpoint{2.756220in}{1.931591in}}{\pgfqpoint{2.748320in}{1.934864in}}{\pgfqpoint{2.740084in}{1.934864in}}%
\pgfpathcurveto{\pgfqpoint{2.731848in}{1.934864in}}{\pgfqpoint{2.723947in}{1.931591in}}{\pgfqpoint{2.718124in}{1.925767in}}%
\pgfpathcurveto{\pgfqpoint{2.712300in}{1.919944in}}{\pgfqpoint{2.709027in}{1.912043in}}{\pgfqpoint{2.709027in}{1.903807in}}%
\pgfpathcurveto{\pgfqpoint{2.709027in}{1.895571in}}{\pgfqpoint{2.712300in}{1.887671in}}{\pgfqpoint{2.718124in}{1.881847in}}%
\pgfpathcurveto{\pgfqpoint{2.723947in}{1.876023in}}{\pgfqpoint{2.731848in}{1.872751in}}{\pgfqpoint{2.740084in}{1.872751in}}%
\pgfpathclose%
\pgfusepath{stroke,fill}%
\end{pgfscope}%
\begin{pgfscope}%
\pgfpathrectangle{\pgfqpoint{0.100000in}{0.212622in}}{\pgfqpoint{3.696000in}{3.696000in}}%
\pgfusepath{clip}%
\pgfsetbuttcap%
\pgfsetroundjoin%
\definecolor{currentfill}{rgb}{0.121569,0.466667,0.705882}%
\pgfsetfillcolor{currentfill}%
\pgfsetfillopacity{0.447989}%
\pgfsetlinewidth{1.003750pt}%
\definecolor{currentstroke}{rgb}{0.121569,0.466667,0.705882}%
\pgfsetstrokecolor{currentstroke}%
\pgfsetstrokeopacity{0.447989}%
\pgfsetdash{}{0pt}%
\pgfpathmoveto{\pgfqpoint{2.748692in}{1.870470in}}%
\pgfpathcurveto{\pgfqpoint{2.756928in}{1.870470in}}{\pgfqpoint{2.764828in}{1.873743in}}{\pgfqpoint{2.770652in}{1.879567in}}%
\pgfpathcurveto{\pgfqpoint{2.776476in}{1.885391in}}{\pgfqpoint{2.779749in}{1.893291in}}{\pgfqpoint{2.779749in}{1.901527in}}%
\pgfpathcurveto{\pgfqpoint{2.779749in}{1.909763in}}{\pgfqpoint{2.776476in}{1.917663in}}{\pgfqpoint{2.770652in}{1.923487in}}%
\pgfpathcurveto{\pgfqpoint{2.764828in}{1.929311in}}{\pgfqpoint{2.756928in}{1.932583in}}{\pgfqpoint{2.748692in}{1.932583in}}%
\pgfpathcurveto{\pgfqpoint{2.740456in}{1.932583in}}{\pgfqpoint{2.732556in}{1.929311in}}{\pgfqpoint{2.726732in}{1.923487in}}%
\pgfpathcurveto{\pgfqpoint{2.720908in}{1.917663in}}{\pgfqpoint{2.717636in}{1.909763in}}{\pgfqpoint{2.717636in}{1.901527in}}%
\pgfpathcurveto{\pgfqpoint{2.717636in}{1.893291in}}{\pgfqpoint{2.720908in}{1.885391in}}{\pgfqpoint{2.726732in}{1.879567in}}%
\pgfpathcurveto{\pgfqpoint{2.732556in}{1.873743in}}{\pgfqpoint{2.740456in}{1.870470in}}{\pgfqpoint{2.748692in}{1.870470in}}%
\pgfpathclose%
\pgfusepath{stroke,fill}%
\end{pgfscope}%
\begin{pgfscope}%
\pgfpathrectangle{\pgfqpoint{0.100000in}{0.212622in}}{\pgfqpoint{3.696000in}{3.696000in}}%
\pgfusepath{clip}%
\pgfsetbuttcap%
\pgfsetroundjoin%
\definecolor{currentfill}{rgb}{0.121569,0.466667,0.705882}%
\pgfsetfillcolor{currentfill}%
\pgfsetfillopacity{0.448415}%
\pgfsetlinewidth{1.003750pt}%
\definecolor{currentstroke}{rgb}{0.121569,0.466667,0.705882}%
\pgfsetstrokecolor{currentstroke}%
\pgfsetstrokeopacity{0.448415}%
\pgfsetdash{}{0pt}%
\pgfpathmoveto{\pgfqpoint{1.406191in}{2.114342in}}%
\pgfpathcurveto{\pgfqpoint{1.414427in}{2.114342in}}{\pgfqpoint{1.422327in}{2.117614in}}{\pgfqpoint{1.428151in}{2.123438in}}%
\pgfpathcurveto{\pgfqpoint{1.433975in}{2.129262in}}{\pgfqpoint{1.437247in}{2.137162in}}{\pgfqpoint{1.437247in}{2.145398in}}%
\pgfpathcurveto{\pgfqpoint{1.437247in}{2.153635in}}{\pgfqpoint{1.433975in}{2.161535in}}{\pgfqpoint{1.428151in}{2.167359in}}%
\pgfpathcurveto{\pgfqpoint{1.422327in}{2.173183in}}{\pgfqpoint{1.414427in}{2.176455in}}{\pgfqpoint{1.406191in}{2.176455in}}%
\pgfpathcurveto{\pgfqpoint{1.397954in}{2.176455in}}{\pgfqpoint{1.390054in}{2.173183in}}{\pgfqpoint{1.384230in}{2.167359in}}%
\pgfpathcurveto{\pgfqpoint{1.378406in}{2.161535in}}{\pgfqpoint{1.375134in}{2.153635in}}{\pgfqpoint{1.375134in}{2.145398in}}%
\pgfpathcurveto{\pgfqpoint{1.375134in}{2.137162in}}{\pgfqpoint{1.378406in}{2.129262in}}{\pgfqpoint{1.384230in}{2.123438in}}%
\pgfpathcurveto{\pgfqpoint{1.390054in}{2.117614in}}{\pgfqpoint{1.397954in}{2.114342in}}{\pgfqpoint{1.406191in}{2.114342in}}%
\pgfpathclose%
\pgfusepath{stroke,fill}%
\end{pgfscope}%
\begin{pgfscope}%
\pgfpathrectangle{\pgfqpoint{0.100000in}{0.212622in}}{\pgfqpoint{3.696000in}{3.696000in}}%
\pgfusepath{clip}%
\pgfsetbuttcap%
\pgfsetroundjoin%
\definecolor{currentfill}{rgb}{0.121569,0.466667,0.705882}%
\pgfsetfillcolor{currentfill}%
\pgfsetfillopacity{0.449301}%
\pgfsetlinewidth{1.003750pt}%
\definecolor{currentstroke}{rgb}{0.121569,0.466667,0.705882}%
\pgfsetstrokecolor{currentstroke}%
\pgfsetstrokeopacity{0.449301}%
\pgfsetdash{}{0pt}%
\pgfpathmoveto{\pgfqpoint{2.757020in}{1.869190in}}%
\pgfpathcurveto{\pgfqpoint{2.765257in}{1.869190in}}{\pgfqpoint{2.773157in}{1.872462in}}{\pgfqpoint{2.778981in}{1.878286in}}%
\pgfpathcurveto{\pgfqpoint{2.784805in}{1.884110in}}{\pgfqpoint{2.788077in}{1.892010in}}{\pgfqpoint{2.788077in}{1.900246in}}%
\pgfpathcurveto{\pgfqpoint{2.788077in}{1.908483in}}{\pgfqpoint{2.784805in}{1.916383in}}{\pgfqpoint{2.778981in}{1.922207in}}%
\pgfpathcurveto{\pgfqpoint{2.773157in}{1.928030in}}{\pgfqpoint{2.765257in}{1.931303in}}{\pgfqpoint{2.757020in}{1.931303in}}%
\pgfpathcurveto{\pgfqpoint{2.748784in}{1.931303in}}{\pgfqpoint{2.740884in}{1.928030in}}{\pgfqpoint{2.735060in}{1.922207in}}%
\pgfpathcurveto{\pgfqpoint{2.729236in}{1.916383in}}{\pgfqpoint{2.725964in}{1.908483in}}{\pgfqpoint{2.725964in}{1.900246in}}%
\pgfpathcurveto{\pgfqpoint{2.725964in}{1.892010in}}{\pgfqpoint{2.729236in}{1.884110in}}{\pgfqpoint{2.735060in}{1.878286in}}%
\pgfpathcurveto{\pgfqpoint{2.740884in}{1.872462in}}{\pgfqpoint{2.748784in}{1.869190in}}{\pgfqpoint{2.757020in}{1.869190in}}%
\pgfpathclose%
\pgfusepath{stroke,fill}%
\end{pgfscope}%
\begin{pgfscope}%
\pgfpathrectangle{\pgfqpoint{0.100000in}{0.212622in}}{\pgfqpoint{3.696000in}{3.696000in}}%
\pgfusepath{clip}%
\pgfsetbuttcap%
\pgfsetroundjoin%
\definecolor{currentfill}{rgb}{0.121569,0.466667,0.705882}%
\pgfsetfillcolor{currentfill}%
\pgfsetfillopacity{0.450323}%
\pgfsetlinewidth{1.003750pt}%
\definecolor{currentstroke}{rgb}{0.121569,0.466667,0.705882}%
\pgfsetstrokecolor{currentstroke}%
\pgfsetstrokeopacity{0.450323}%
\pgfsetdash{}{0pt}%
\pgfpathmoveto{\pgfqpoint{2.766382in}{1.866618in}}%
\pgfpathcurveto{\pgfqpoint{2.774618in}{1.866618in}}{\pgfqpoint{2.782518in}{1.869891in}}{\pgfqpoint{2.788342in}{1.875714in}}%
\pgfpathcurveto{\pgfqpoint{2.794166in}{1.881538in}}{\pgfqpoint{2.797438in}{1.889438in}}{\pgfqpoint{2.797438in}{1.897675in}}%
\pgfpathcurveto{\pgfqpoint{2.797438in}{1.905911in}}{\pgfqpoint{2.794166in}{1.913811in}}{\pgfqpoint{2.788342in}{1.919635in}}%
\pgfpathcurveto{\pgfqpoint{2.782518in}{1.925459in}}{\pgfqpoint{2.774618in}{1.928731in}}{\pgfqpoint{2.766382in}{1.928731in}}%
\pgfpathcurveto{\pgfqpoint{2.758146in}{1.928731in}}{\pgfqpoint{2.750246in}{1.925459in}}{\pgfqpoint{2.744422in}{1.919635in}}%
\pgfpathcurveto{\pgfqpoint{2.738598in}{1.913811in}}{\pgfqpoint{2.735325in}{1.905911in}}{\pgfqpoint{2.735325in}{1.897675in}}%
\pgfpathcurveto{\pgfqpoint{2.735325in}{1.889438in}}{\pgfqpoint{2.738598in}{1.881538in}}{\pgfqpoint{2.744422in}{1.875714in}}%
\pgfpathcurveto{\pgfqpoint{2.750246in}{1.869891in}}{\pgfqpoint{2.758146in}{1.866618in}}{\pgfqpoint{2.766382in}{1.866618in}}%
\pgfpathclose%
\pgfusepath{stroke,fill}%
\end{pgfscope}%
\begin{pgfscope}%
\pgfpathrectangle{\pgfqpoint{0.100000in}{0.212622in}}{\pgfqpoint{3.696000in}{3.696000in}}%
\pgfusepath{clip}%
\pgfsetbuttcap%
\pgfsetroundjoin%
\definecolor{currentfill}{rgb}{0.121569,0.466667,0.705882}%
\pgfsetfillcolor{currentfill}%
\pgfsetfillopacity{0.450823}%
\pgfsetlinewidth{1.003750pt}%
\definecolor{currentstroke}{rgb}{0.121569,0.466667,0.705882}%
\pgfsetstrokecolor{currentstroke}%
\pgfsetstrokeopacity{0.450823}%
\pgfsetdash{}{0pt}%
\pgfpathmoveto{\pgfqpoint{2.771675in}{1.865297in}}%
\pgfpathcurveto{\pgfqpoint{2.779911in}{1.865297in}}{\pgfqpoint{2.787812in}{1.868569in}}{\pgfqpoint{2.793635in}{1.874393in}}%
\pgfpathcurveto{\pgfqpoint{2.799459in}{1.880217in}}{\pgfqpoint{2.802732in}{1.888117in}}{\pgfqpoint{2.802732in}{1.896353in}}%
\pgfpathcurveto{\pgfqpoint{2.802732in}{1.904589in}}{\pgfqpoint{2.799459in}{1.912489in}}{\pgfqpoint{2.793635in}{1.918313in}}%
\pgfpathcurveto{\pgfqpoint{2.787812in}{1.924137in}}{\pgfqpoint{2.779911in}{1.927410in}}{\pgfqpoint{2.771675in}{1.927410in}}%
\pgfpathcurveto{\pgfqpoint{2.763439in}{1.927410in}}{\pgfqpoint{2.755539in}{1.924137in}}{\pgfqpoint{2.749715in}{1.918313in}}%
\pgfpathcurveto{\pgfqpoint{2.743891in}{1.912489in}}{\pgfqpoint{2.740619in}{1.904589in}}{\pgfqpoint{2.740619in}{1.896353in}}%
\pgfpathcurveto{\pgfqpoint{2.740619in}{1.888117in}}{\pgfqpoint{2.743891in}{1.880217in}}{\pgfqpoint{2.749715in}{1.874393in}}%
\pgfpathcurveto{\pgfqpoint{2.755539in}{1.868569in}}{\pgfqpoint{2.763439in}{1.865297in}}{\pgfqpoint{2.771675in}{1.865297in}}%
\pgfpathclose%
\pgfusepath{stroke,fill}%
\end{pgfscope}%
\begin{pgfscope}%
\pgfpathrectangle{\pgfqpoint{0.100000in}{0.212622in}}{\pgfqpoint{3.696000in}{3.696000in}}%
\pgfusepath{clip}%
\pgfsetbuttcap%
\pgfsetroundjoin%
\definecolor{currentfill}{rgb}{0.121569,0.466667,0.705882}%
\pgfsetfillcolor{currentfill}%
\pgfsetfillopacity{0.450828}%
\pgfsetlinewidth{1.003750pt}%
\definecolor{currentstroke}{rgb}{0.121569,0.466667,0.705882}%
\pgfsetstrokecolor{currentstroke}%
\pgfsetstrokeopacity{0.450828}%
\pgfsetdash{}{0pt}%
\pgfpathmoveto{\pgfqpoint{1.401711in}{2.114418in}}%
\pgfpathcurveto{\pgfqpoint{1.409948in}{2.114418in}}{\pgfqpoint{1.417848in}{2.117691in}}{\pgfqpoint{1.423672in}{2.123515in}}%
\pgfpathcurveto{\pgfqpoint{1.429496in}{2.129339in}}{\pgfqpoint{1.432768in}{2.137239in}}{\pgfqpoint{1.432768in}{2.145475in}}%
\pgfpathcurveto{\pgfqpoint{1.432768in}{2.153711in}}{\pgfqpoint{1.429496in}{2.161611in}}{\pgfqpoint{1.423672in}{2.167435in}}%
\pgfpathcurveto{\pgfqpoint{1.417848in}{2.173259in}}{\pgfqpoint{1.409948in}{2.176531in}}{\pgfqpoint{1.401711in}{2.176531in}}%
\pgfpathcurveto{\pgfqpoint{1.393475in}{2.176531in}}{\pgfqpoint{1.385575in}{2.173259in}}{\pgfqpoint{1.379751in}{2.167435in}}%
\pgfpathcurveto{\pgfqpoint{1.373927in}{2.161611in}}{\pgfqpoint{1.370655in}{2.153711in}}{\pgfqpoint{1.370655in}{2.145475in}}%
\pgfpathcurveto{\pgfqpoint{1.370655in}{2.137239in}}{\pgfqpoint{1.373927in}{2.129339in}}{\pgfqpoint{1.379751in}{2.123515in}}%
\pgfpathcurveto{\pgfqpoint{1.385575in}{2.117691in}}{\pgfqpoint{1.393475in}{2.114418in}}{\pgfqpoint{1.401711in}{2.114418in}}%
\pgfpathclose%
\pgfusepath{stroke,fill}%
\end{pgfscope}%
\begin{pgfscope}%
\pgfpathrectangle{\pgfqpoint{0.100000in}{0.212622in}}{\pgfqpoint{3.696000in}{3.696000in}}%
\pgfusepath{clip}%
\pgfsetbuttcap%
\pgfsetroundjoin%
\definecolor{currentfill}{rgb}{0.121569,0.466667,0.705882}%
\pgfsetfillcolor{currentfill}%
\pgfsetfillopacity{0.451626}%
\pgfsetlinewidth{1.003750pt}%
\definecolor{currentstroke}{rgb}{0.121569,0.466667,0.705882}%
\pgfsetstrokecolor{currentstroke}%
\pgfsetstrokeopacity{0.451626}%
\pgfsetdash{}{0pt}%
\pgfpathmoveto{\pgfqpoint{2.777058in}{1.864128in}}%
\pgfpathcurveto{\pgfqpoint{2.785294in}{1.864128in}}{\pgfqpoint{2.793194in}{1.867400in}}{\pgfqpoint{2.799018in}{1.873224in}}%
\pgfpathcurveto{\pgfqpoint{2.804842in}{1.879048in}}{\pgfqpoint{2.808114in}{1.886948in}}{\pgfqpoint{2.808114in}{1.895184in}}%
\pgfpathcurveto{\pgfqpoint{2.808114in}{1.903421in}}{\pgfqpoint{2.804842in}{1.911321in}}{\pgfqpoint{2.799018in}{1.917144in}}%
\pgfpathcurveto{\pgfqpoint{2.793194in}{1.922968in}}{\pgfqpoint{2.785294in}{1.926241in}}{\pgfqpoint{2.777058in}{1.926241in}}%
\pgfpathcurveto{\pgfqpoint{2.768821in}{1.926241in}}{\pgfqpoint{2.760921in}{1.922968in}}{\pgfqpoint{2.755097in}{1.917144in}}%
\pgfpathcurveto{\pgfqpoint{2.749274in}{1.911321in}}{\pgfqpoint{2.746001in}{1.903421in}}{\pgfqpoint{2.746001in}{1.895184in}}%
\pgfpathcurveto{\pgfqpoint{2.746001in}{1.886948in}}{\pgfqpoint{2.749274in}{1.879048in}}{\pgfqpoint{2.755097in}{1.873224in}}%
\pgfpathcurveto{\pgfqpoint{2.760921in}{1.867400in}}{\pgfqpoint{2.768821in}{1.864128in}}{\pgfqpoint{2.777058in}{1.864128in}}%
\pgfpathclose%
\pgfusepath{stroke,fill}%
\end{pgfscope}%
\begin{pgfscope}%
\pgfpathrectangle{\pgfqpoint{0.100000in}{0.212622in}}{\pgfqpoint{3.696000in}{3.696000in}}%
\pgfusepath{clip}%
\pgfsetbuttcap%
\pgfsetroundjoin%
\definecolor{currentfill}{rgb}{0.121569,0.466667,0.705882}%
\pgfsetfillcolor{currentfill}%
\pgfsetfillopacity{0.452303}%
\pgfsetlinewidth{1.003750pt}%
\definecolor{currentstroke}{rgb}{0.121569,0.466667,0.705882}%
\pgfsetstrokecolor{currentstroke}%
\pgfsetstrokeopacity{0.452303}%
\pgfsetdash{}{0pt}%
\pgfpathmoveto{\pgfqpoint{2.784234in}{1.862227in}}%
\pgfpathcurveto{\pgfqpoint{2.792470in}{1.862227in}}{\pgfqpoint{2.800371in}{1.865500in}}{\pgfqpoint{2.806194in}{1.871324in}}%
\pgfpathcurveto{\pgfqpoint{2.812018in}{1.877148in}}{\pgfqpoint{2.815291in}{1.885048in}}{\pgfqpoint{2.815291in}{1.893284in}}%
\pgfpathcurveto{\pgfqpoint{2.815291in}{1.901520in}}{\pgfqpoint{2.812018in}{1.909420in}}{\pgfqpoint{2.806194in}{1.915244in}}%
\pgfpathcurveto{\pgfqpoint{2.800371in}{1.921068in}}{\pgfqpoint{2.792470in}{1.924340in}}{\pgfqpoint{2.784234in}{1.924340in}}%
\pgfpathcurveto{\pgfqpoint{2.775998in}{1.924340in}}{\pgfqpoint{2.768098in}{1.921068in}}{\pgfqpoint{2.762274in}{1.915244in}}%
\pgfpathcurveto{\pgfqpoint{2.756450in}{1.909420in}}{\pgfqpoint{2.753178in}{1.901520in}}{\pgfqpoint{2.753178in}{1.893284in}}%
\pgfpathcurveto{\pgfqpoint{2.753178in}{1.885048in}}{\pgfqpoint{2.756450in}{1.877148in}}{\pgfqpoint{2.762274in}{1.871324in}}%
\pgfpathcurveto{\pgfqpoint{2.768098in}{1.865500in}}{\pgfqpoint{2.775998in}{1.862227in}}{\pgfqpoint{2.784234in}{1.862227in}}%
\pgfpathclose%
\pgfusepath{stroke,fill}%
\end{pgfscope}%
\begin{pgfscope}%
\pgfpathrectangle{\pgfqpoint{0.100000in}{0.212622in}}{\pgfqpoint{3.696000in}{3.696000in}}%
\pgfusepath{clip}%
\pgfsetbuttcap%
\pgfsetroundjoin%
\definecolor{currentfill}{rgb}{0.121569,0.466667,0.705882}%
\pgfsetfillcolor{currentfill}%
\pgfsetfillopacity{0.453034}%
\pgfsetlinewidth{1.003750pt}%
\definecolor{currentstroke}{rgb}{0.121569,0.466667,0.705882}%
\pgfsetstrokecolor{currentstroke}%
\pgfsetstrokeopacity{0.453034}%
\pgfsetdash{}{0pt}%
\pgfpathmoveto{\pgfqpoint{1.396390in}{2.114798in}}%
\pgfpathcurveto{\pgfqpoint{1.404627in}{2.114798in}}{\pgfqpoint{1.412527in}{2.118070in}}{\pgfqpoint{1.418351in}{2.123894in}}%
\pgfpathcurveto{\pgfqpoint{1.424175in}{2.129718in}}{\pgfqpoint{1.427447in}{2.137618in}}{\pgfqpoint{1.427447in}{2.145854in}}%
\pgfpathcurveto{\pgfqpoint{1.427447in}{2.154090in}}{\pgfqpoint{1.424175in}{2.161990in}}{\pgfqpoint{1.418351in}{2.167814in}}%
\pgfpathcurveto{\pgfqpoint{1.412527in}{2.173638in}}{\pgfqpoint{1.404627in}{2.176911in}}{\pgfqpoint{1.396390in}{2.176911in}}%
\pgfpathcurveto{\pgfqpoint{1.388154in}{2.176911in}}{\pgfqpoint{1.380254in}{2.173638in}}{\pgfqpoint{1.374430in}{2.167814in}}%
\pgfpathcurveto{\pgfqpoint{1.368606in}{2.161990in}}{\pgfqpoint{1.365334in}{2.154090in}}{\pgfqpoint{1.365334in}{2.145854in}}%
\pgfpathcurveto{\pgfqpoint{1.365334in}{2.137618in}}{\pgfqpoint{1.368606in}{2.129718in}}{\pgfqpoint{1.374430in}{2.123894in}}%
\pgfpathcurveto{\pgfqpoint{1.380254in}{2.118070in}}{\pgfqpoint{1.388154in}{2.114798in}}{\pgfqpoint{1.396390in}{2.114798in}}%
\pgfpathclose%
\pgfusepath{stroke,fill}%
\end{pgfscope}%
\begin{pgfscope}%
\pgfpathrectangle{\pgfqpoint{0.100000in}{0.212622in}}{\pgfqpoint{3.696000in}{3.696000in}}%
\pgfusepath{clip}%
\pgfsetbuttcap%
\pgfsetroundjoin%
\definecolor{currentfill}{rgb}{0.121569,0.466667,0.705882}%
\pgfsetfillcolor{currentfill}%
\pgfsetfillopacity{0.453264}%
\pgfsetlinewidth{1.003750pt}%
\definecolor{currentstroke}{rgb}{0.121569,0.466667,0.705882}%
\pgfsetstrokecolor{currentstroke}%
\pgfsetstrokeopacity{0.453264}%
\pgfsetdash{}{0pt}%
\pgfpathmoveto{\pgfqpoint{2.791536in}{1.860681in}}%
\pgfpathcurveto{\pgfqpoint{2.799772in}{1.860681in}}{\pgfqpoint{2.807673in}{1.863954in}}{\pgfqpoint{2.813496in}{1.869777in}}%
\pgfpathcurveto{\pgfqpoint{2.819320in}{1.875601in}}{\pgfqpoint{2.822593in}{1.883501in}}{\pgfqpoint{2.822593in}{1.891738in}}%
\pgfpathcurveto{\pgfqpoint{2.822593in}{1.899974in}}{\pgfqpoint{2.819320in}{1.907874in}}{\pgfqpoint{2.813496in}{1.913698in}}%
\pgfpathcurveto{\pgfqpoint{2.807673in}{1.919522in}}{\pgfqpoint{2.799772in}{1.922794in}}{\pgfqpoint{2.791536in}{1.922794in}}%
\pgfpathcurveto{\pgfqpoint{2.783300in}{1.922794in}}{\pgfqpoint{2.775400in}{1.919522in}}{\pgfqpoint{2.769576in}{1.913698in}}%
\pgfpathcurveto{\pgfqpoint{2.763752in}{1.907874in}}{\pgfqpoint{2.760480in}{1.899974in}}{\pgfqpoint{2.760480in}{1.891738in}}%
\pgfpathcurveto{\pgfqpoint{2.760480in}{1.883501in}}{\pgfqpoint{2.763752in}{1.875601in}}{\pgfqpoint{2.769576in}{1.869777in}}%
\pgfpathcurveto{\pgfqpoint{2.775400in}{1.863954in}}{\pgfqpoint{2.783300in}{1.860681in}}{\pgfqpoint{2.791536in}{1.860681in}}%
\pgfpathclose%
\pgfusepath{stroke,fill}%
\end{pgfscope}%
\begin{pgfscope}%
\pgfpathrectangle{\pgfqpoint{0.100000in}{0.212622in}}{\pgfqpoint{3.696000in}{3.696000in}}%
\pgfusepath{clip}%
\pgfsetbuttcap%
\pgfsetroundjoin%
\definecolor{currentfill}{rgb}{0.121569,0.466667,0.705882}%
\pgfsetfillcolor{currentfill}%
\pgfsetfillopacity{0.453871}%
\pgfsetlinewidth{1.003750pt}%
\definecolor{currentstroke}{rgb}{0.121569,0.466667,0.705882}%
\pgfsetstrokecolor{currentstroke}%
\pgfsetstrokeopacity{0.453871}%
\pgfsetdash{}{0pt}%
\pgfpathmoveto{\pgfqpoint{2.795363in}{1.859822in}}%
\pgfpathcurveto{\pgfqpoint{2.803600in}{1.859822in}}{\pgfqpoint{2.811500in}{1.863094in}}{\pgfqpoint{2.817324in}{1.868918in}}%
\pgfpathcurveto{\pgfqpoint{2.823148in}{1.874742in}}{\pgfqpoint{2.826420in}{1.882642in}}{\pgfqpoint{2.826420in}{1.890879in}}%
\pgfpathcurveto{\pgfqpoint{2.826420in}{1.899115in}}{\pgfqpoint{2.823148in}{1.907015in}}{\pgfqpoint{2.817324in}{1.912839in}}%
\pgfpathcurveto{\pgfqpoint{2.811500in}{1.918663in}}{\pgfqpoint{2.803600in}{1.921935in}}{\pgfqpoint{2.795363in}{1.921935in}}%
\pgfpathcurveto{\pgfqpoint{2.787127in}{1.921935in}}{\pgfqpoint{2.779227in}{1.918663in}}{\pgfqpoint{2.773403in}{1.912839in}}%
\pgfpathcurveto{\pgfqpoint{2.767579in}{1.907015in}}{\pgfqpoint{2.764307in}{1.899115in}}{\pgfqpoint{2.764307in}{1.890879in}}%
\pgfpathcurveto{\pgfqpoint{2.764307in}{1.882642in}}{\pgfqpoint{2.767579in}{1.874742in}}{\pgfqpoint{2.773403in}{1.868918in}}%
\pgfpathcurveto{\pgfqpoint{2.779227in}{1.863094in}}{\pgfqpoint{2.787127in}{1.859822in}}{\pgfqpoint{2.795363in}{1.859822in}}%
\pgfpathclose%
\pgfusepath{stroke,fill}%
\end{pgfscope}%
\begin{pgfscope}%
\pgfpathrectangle{\pgfqpoint{0.100000in}{0.212622in}}{\pgfqpoint{3.696000in}{3.696000in}}%
\pgfusepath{clip}%
\pgfsetbuttcap%
\pgfsetroundjoin%
\definecolor{currentfill}{rgb}{0.121569,0.466667,0.705882}%
\pgfsetfillcolor{currentfill}%
\pgfsetfillopacity{0.454661}%
\pgfsetlinewidth{1.003750pt}%
\definecolor{currentstroke}{rgb}{0.121569,0.466667,0.705882}%
\pgfsetstrokecolor{currentstroke}%
\pgfsetstrokeopacity{0.454661}%
\pgfsetdash{}{0pt}%
\pgfpathmoveto{\pgfqpoint{2.800248in}{1.858897in}}%
\pgfpathcurveto{\pgfqpoint{2.808485in}{1.858897in}}{\pgfqpoint{2.816385in}{1.862169in}}{\pgfqpoint{2.822209in}{1.867993in}}%
\pgfpathcurveto{\pgfqpoint{2.828033in}{1.873817in}}{\pgfqpoint{2.831305in}{1.881717in}}{\pgfqpoint{2.831305in}{1.889953in}}%
\pgfpathcurveto{\pgfqpoint{2.831305in}{1.898190in}}{\pgfqpoint{2.828033in}{1.906090in}}{\pgfqpoint{2.822209in}{1.911914in}}%
\pgfpathcurveto{\pgfqpoint{2.816385in}{1.917738in}}{\pgfqpoint{2.808485in}{1.921010in}}{\pgfqpoint{2.800248in}{1.921010in}}%
\pgfpathcurveto{\pgfqpoint{2.792012in}{1.921010in}}{\pgfqpoint{2.784112in}{1.917738in}}{\pgfqpoint{2.778288in}{1.911914in}}%
\pgfpathcurveto{\pgfqpoint{2.772464in}{1.906090in}}{\pgfqpoint{2.769192in}{1.898190in}}{\pgfqpoint{2.769192in}{1.889953in}}%
\pgfpathcurveto{\pgfqpoint{2.769192in}{1.881717in}}{\pgfqpoint{2.772464in}{1.873817in}}{\pgfqpoint{2.778288in}{1.867993in}}%
\pgfpathcurveto{\pgfqpoint{2.784112in}{1.862169in}}{\pgfqpoint{2.792012in}{1.858897in}}{\pgfqpoint{2.800248in}{1.858897in}}%
\pgfpathclose%
\pgfusepath{stroke,fill}%
\end{pgfscope}%
\begin{pgfscope}%
\pgfpathrectangle{\pgfqpoint{0.100000in}{0.212622in}}{\pgfqpoint{3.696000in}{3.696000in}}%
\pgfusepath{clip}%
\pgfsetbuttcap%
\pgfsetroundjoin%
\definecolor{currentfill}{rgb}{0.121569,0.466667,0.705882}%
\pgfsetfillcolor{currentfill}%
\pgfsetfillopacity{0.455006}%
\pgfsetlinewidth{1.003750pt}%
\definecolor{currentstroke}{rgb}{0.121569,0.466667,0.705882}%
\pgfsetstrokecolor{currentstroke}%
\pgfsetstrokeopacity{0.455006}%
\pgfsetdash{}{0pt}%
\pgfpathmoveto{\pgfqpoint{1.394079in}{2.114990in}}%
\pgfpathcurveto{\pgfqpoint{1.402316in}{2.114990in}}{\pgfqpoint{1.410216in}{2.118262in}}{\pgfqpoint{1.416040in}{2.124086in}}%
\pgfpathcurveto{\pgfqpoint{1.421864in}{2.129910in}}{\pgfqpoint{1.425136in}{2.137810in}}{\pgfqpoint{1.425136in}{2.146046in}}%
\pgfpathcurveto{\pgfqpoint{1.425136in}{2.154283in}}{\pgfqpoint{1.421864in}{2.162183in}}{\pgfqpoint{1.416040in}{2.168007in}}%
\pgfpathcurveto{\pgfqpoint{1.410216in}{2.173831in}}{\pgfqpoint{1.402316in}{2.177103in}}{\pgfqpoint{1.394079in}{2.177103in}}%
\pgfpathcurveto{\pgfqpoint{1.385843in}{2.177103in}}{\pgfqpoint{1.377943in}{2.173831in}}{\pgfqpoint{1.372119in}{2.168007in}}%
\pgfpathcurveto{\pgfqpoint{1.366295in}{2.162183in}}{\pgfqpoint{1.363023in}{2.154283in}}{\pgfqpoint{1.363023in}{2.146046in}}%
\pgfpathcurveto{\pgfqpoint{1.363023in}{2.137810in}}{\pgfqpoint{1.366295in}{2.129910in}}{\pgfqpoint{1.372119in}{2.124086in}}%
\pgfpathcurveto{\pgfqpoint{1.377943in}{2.118262in}}{\pgfqpoint{1.385843in}{2.114990in}}{\pgfqpoint{1.394079in}{2.114990in}}%
\pgfpathclose%
\pgfusepath{stroke,fill}%
\end{pgfscope}%
\begin{pgfscope}%
\pgfpathrectangle{\pgfqpoint{0.100000in}{0.212622in}}{\pgfqpoint{3.696000in}{3.696000in}}%
\pgfusepath{clip}%
\pgfsetbuttcap%
\pgfsetroundjoin%
\definecolor{currentfill}{rgb}{0.121569,0.466667,0.705882}%
\pgfsetfillcolor{currentfill}%
\pgfsetfillopacity{0.455684}%
\pgfsetlinewidth{1.003750pt}%
\definecolor{currentstroke}{rgb}{0.121569,0.466667,0.705882}%
\pgfsetstrokecolor{currentstroke}%
\pgfsetstrokeopacity{0.455684}%
\pgfsetdash{}{0pt}%
\pgfpathmoveto{\pgfqpoint{2.805122in}{1.858149in}}%
\pgfpathcurveto{\pgfqpoint{2.813358in}{1.858149in}}{\pgfqpoint{2.821258in}{1.861422in}}{\pgfqpoint{2.827082in}{1.867246in}}%
\pgfpathcurveto{\pgfqpoint{2.832906in}{1.873070in}}{\pgfqpoint{2.836178in}{1.880970in}}{\pgfqpoint{2.836178in}{1.889206in}}%
\pgfpathcurveto{\pgfqpoint{2.836178in}{1.897442in}}{\pgfqpoint{2.832906in}{1.905342in}}{\pgfqpoint{2.827082in}{1.911166in}}%
\pgfpathcurveto{\pgfqpoint{2.821258in}{1.916990in}}{\pgfqpoint{2.813358in}{1.920262in}}{\pgfqpoint{2.805122in}{1.920262in}}%
\pgfpathcurveto{\pgfqpoint{2.796886in}{1.920262in}}{\pgfqpoint{2.788986in}{1.916990in}}{\pgfqpoint{2.783162in}{1.911166in}}%
\pgfpathcurveto{\pgfqpoint{2.777338in}{1.905342in}}{\pgfqpoint{2.774065in}{1.897442in}}{\pgfqpoint{2.774065in}{1.889206in}}%
\pgfpathcurveto{\pgfqpoint{2.774065in}{1.880970in}}{\pgfqpoint{2.777338in}{1.873070in}}{\pgfqpoint{2.783162in}{1.867246in}}%
\pgfpathcurveto{\pgfqpoint{2.788986in}{1.861422in}}{\pgfqpoint{2.796886in}{1.858149in}}{\pgfqpoint{2.805122in}{1.858149in}}%
\pgfpathclose%
\pgfusepath{stroke,fill}%
\end{pgfscope}%
\begin{pgfscope}%
\pgfpathrectangle{\pgfqpoint{0.100000in}{0.212622in}}{\pgfqpoint{3.696000in}{3.696000in}}%
\pgfusepath{clip}%
\pgfsetbuttcap%
\pgfsetroundjoin%
\definecolor{currentfill}{rgb}{0.121569,0.466667,0.705882}%
\pgfsetfillcolor{currentfill}%
\pgfsetfillopacity{0.456288}%
\pgfsetlinewidth{1.003750pt}%
\definecolor{currentstroke}{rgb}{0.121569,0.466667,0.705882}%
\pgfsetstrokecolor{currentstroke}%
\pgfsetstrokeopacity{0.456288}%
\pgfsetdash{}{0pt}%
\pgfpathmoveto{\pgfqpoint{2.807682in}{1.857786in}}%
\pgfpathcurveto{\pgfqpoint{2.815918in}{1.857786in}}{\pgfqpoint{2.823818in}{1.861058in}}{\pgfqpoint{2.829642in}{1.866882in}}%
\pgfpathcurveto{\pgfqpoint{2.835466in}{1.872706in}}{\pgfqpoint{2.838739in}{1.880606in}}{\pgfqpoint{2.838739in}{1.888842in}}%
\pgfpathcurveto{\pgfqpoint{2.838739in}{1.897079in}}{\pgfqpoint{2.835466in}{1.904979in}}{\pgfqpoint{2.829642in}{1.910803in}}%
\pgfpathcurveto{\pgfqpoint{2.823818in}{1.916627in}}{\pgfqpoint{2.815918in}{1.919899in}}{\pgfqpoint{2.807682in}{1.919899in}}%
\pgfpathcurveto{\pgfqpoint{2.799446in}{1.919899in}}{\pgfqpoint{2.791546in}{1.916627in}}{\pgfqpoint{2.785722in}{1.910803in}}%
\pgfpathcurveto{\pgfqpoint{2.779898in}{1.904979in}}{\pgfqpoint{2.776626in}{1.897079in}}{\pgfqpoint{2.776626in}{1.888842in}}%
\pgfpathcurveto{\pgfqpoint{2.776626in}{1.880606in}}{\pgfqpoint{2.779898in}{1.872706in}}{\pgfqpoint{2.785722in}{1.866882in}}%
\pgfpathcurveto{\pgfqpoint{2.791546in}{1.861058in}}{\pgfqpoint{2.799446in}{1.857786in}}{\pgfqpoint{2.807682in}{1.857786in}}%
\pgfpathclose%
\pgfusepath{stroke,fill}%
\end{pgfscope}%
\begin{pgfscope}%
\pgfpathrectangle{\pgfqpoint{0.100000in}{0.212622in}}{\pgfqpoint{3.696000in}{3.696000in}}%
\pgfusepath{clip}%
\pgfsetbuttcap%
\pgfsetroundjoin%
\definecolor{currentfill}{rgb}{0.121569,0.466667,0.705882}%
\pgfsetfillcolor{currentfill}%
\pgfsetfillopacity{0.456530}%
\pgfsetlinewidth{1.003750pt}%
\definecolor{currentstroke}{rgb}{0.121569,0.466667,0.705882}%
\pgfsetstrokecolor{currentstroke}%
\pgfsetstrokeopacity{0.456530}%
\pgfsetdash{}{0pt}%
\pgfpathmoveto{\pgfqpoint{1.390917in}{2.115105in}}%
\pgfpathcurveto{\pgfqpoint{1.399153in}{2.115105in}}{\pgfqpoint{1.407053in}{2.118378in}}{\pgfqpoint{1.412877in}{2.124202in}}%
\pgfpathcurveto{\pgfqpoint{1.418701in}{2.130026in}}{\pgfqpoint{1.421973in}{2.137926in}}{\pgfqpoint{1.421973in}{2.146162in}}%
\pgfpathcurveto{\pgfqpoint{1.421973in}{2.154398in}}{\pgfqpoint{1.418701in}{2.162298in}}{\pgfqpoint{1.412877in}{2.168122in}}%
\pgfpathcurveto{\pgfqpoint{1.407053in}{2.173946in}}{\pgfqpoint{1.399153in}{2.177218in}}{\pgfqpoint{1.390917in}{2.177218in}}%
\pgfpathcurveto{\pgfqpoint{1.382680in}{2.177218in}}{\pgfqpoint{1.374780in}{2.173946in}}{\pgfqpoint{1.368956in}{2.168122in}}%
\pgfpathcurveto{\pgfqpoint{1.363132in}{2.162298in}}{\pgfqpoint{1.359860in}{2.154398in}}{\pgfqpoint{1.359860in}{2.146162in}}%
\pgfpathcurveto{\pgfqpoint{1.359860in}{2.137926in}}{\pgfqpoint{1.363132in}{2.130026in}}{\pgfqpoint{1.368956in}{2.124202in}}%
\pgfpathcurveto{\pgfqpoint{1.374780in}{2.118378in}}{\pgfqpoint{1.382680in}{2.115105in}}{\pgfqpoint{1.390917in}{2.115105in}}%
\pgfpathclose%
\pgfusepath{stroke,fill}%
\end{pgfscope}%
\begin{pgfscope}%
\pgfpathrectangle{\pgfqpoint{0.100000in}{0.212622in}}{\pgfqpoint{3.696000in}{3.696000in}}%
\pgfusepath{clip}%
\pgfsetbuttcap%
\pgfsetroundjoin%
\definecolor{currentfill}{rgb}{0.121569,0.466667,0.705882}%
\pgfsetfillcolor{currentfill}%
\pgfsetfillopacity{0.457100}%
\pgfsetlinewidth{1.003750pt}%
\definecolor{currentstroke}{rgb}{0.121569,0.466667,0.705882}%
\pgfsetstrokecolor{currentstroke}%
\pgfsetstrokeopacity{0.457100}%
\pgfsetdash{}{0pt}%
\pgfpathmoveto{\pgfqpoint{2.811121in}{1.857073in}}%
\pgfpathcurveto{\pgfqpoint{2.819357in}{1.857073in}}{\pgfqpoint{2.827257in}{1.860345in}}{\pgfqpoint{2.833081in}{1.866169in}}%
\pgfpathcurveto{\pgfqpoint{2.838905in}{1.871993in}}{\pgfqpoint{2.842177in}{1.879893in}}{\pgfqpoint{2.842177in}{1.888129in}}%
\pgfpathcurveto{\pgfqpoint{2.842177in}{1.896365in}}{\pgfqpoint{2.838905in}{1.904265in}}{\pgfqpoint{2.833081in}{1.910089in}}%
\pgfpathcurveto{\pgfqpoint{2.827257in}{1.915913in}}{\pgfqpoint{2.819357in}{1.919186in}}{\pgfqpoint{2.811121in}{1.919186in}}%
\pgfpathcurveto{\pgfqpoint{2.802885in}{1.919186in}}{\pgfqpoint{2.794985in}{1.915913in}}{\pgfqpoint{2.789161in}{1.910089in}}%
\pgfpathcurveto{\pgfqpoint{2.783337in}{1.904265in}}{\pgfqpoint{2.780064in}{1.896365in}}{\pgfqpoint{2.780064in}{1.888129in}}%
\pgfpathcurveto{\pgfqpoint{2.780064in}{1.879893in}}{\pgfqpoint{2.783337in}{1.871993in}}{\pgfqpoint{2.789161in}{1.866169in}}%
\pgfpathcurveto{\pgfqpoint{2.794985in}{1.860345in}}{\pgfqpoint{2.802885in}{1.857073in}}{\pgfqpoint{2.811121in}{1.857073in}}%
\pgfpathclose%
\pgfusepath{stroke,fill}%
\end{pgfscope}%
\begin{pgfscope}%
\pgfpathrectangle{\pgfqpoint{0.100000in}{0.212622in}}{\pgfqpoint{3.696000in}{3.696000in}}%
\pgfusepath{clip}%
\pgfsetbuttcap%
\pgfsetroundjoin%
\definecolor{currentfill}{rgb}{0.121569,0.466667,0.705882}%
\pgfsetfillcolor{currentfill}%
\pgfsetfillopacity{0.457604}%
\pgfsetlinewidth{1.003750pt}%
\definecolor{currentstroke}{rgb}{0.121569,0.466667,0.705882}%
\pgfsetstrokecolor{currentstroke}%
\pgfsetstrokeopacity{0.457604}%
\pgfsetdash{}{0pt}%
\pgfpathmoveto{\pgfqpoint{2.812928in}{1.856938in}}%
\pgfpathcurveto{\pgfqpoint{2.821164in}{1.856938in}}{\pgfqpoint{2.829064in}{1.860210in}}{\pgfqpoint{2.834888in}{1.866034in}}%
\pgfpathcurveto{\pgfqpoint{2.840712in}{1.871858in}}{\pgfqpoint{2.843984in}{1.879758in}}{\pgfqpoint{2.843984in}{1.887994in}}%
\pgfpathcurveto{\pgfqpoint{2.843984in}{1.896230in}}{\pgfqpoint{2.840712in}{1.904130in}}{\pgfqpoint{2.834888in}{1.909954in}}%
\pgfpathcurveto{\pgfqpoint{2.829064in}{1.915778in}}{\pgfqpoint{2.821164in}{1.919051in}}{\pgfqpoint{2.812928in}{1.919051in}}%
\pgfpathcurveto{\pgfqpoint{2.804692in}{1.919051in}}{\pgfqpoint{2.796792in}{1.915778in}}{\pgfqpoint{2.790968in}{1.909954in}}%
\pgfpathcurveto{\pgfqpoint{2.785144in}{1.904130in}}{\pgfqpoint{2.781871in}{1.896230in}}{\pgfqpoint{2.781871in}{1.887994in}}%
\pgfpathcurveto{\pgfqpoint{2.781871in}{1.879758in}}{\pgfqpoint{2.785144in}{1.871858in}}{\pgfqpoint{2.790968in}{1.866034in}}%
\pgfpathcurveto{\pgfqpoint{2.796792in}{1.860210in}}{\pgfqpoint{2.804692in}{1.856938in}}{\pgfqpoint{2.812928in}{1.856938in}}%
\pgfpathclose%
\pgfusepath{stroke,fill}%
\end{pgfscope}%
\begin{pgfscope}%
\pgfpathrectangle{\pgfqpoint{0.100000in}{0.212622in}}{\pgfqpoint{3.696000in}{3.696000in}}%
\pgfusepath{clip}%
\pgfsetbuttcap%
\pgfsetroundjoin%
\definecolor{currentfill}{rgb}{0.121569,0.466667,0.705882}%
\pgfsetfillcolor{currentfill}%
\pgfsetfillopacity{0.457814}%
\pgfsetlinewidth{1.003750pt}%
\definecolor{currentstroke}{rgb}{0.121569,0.466667,0.705882}%
\pgfsetstrokecolor{currentstroke}%
\pgfsetstrokeopacity{0.457814}%
\pgfsetdash{}{0pt}%
\pgfpathmoveto{\pgfqpoint{2.814110in}{1.856782in}}%
\pgfpathcurveto{\pgfqpoint{2.822346in}{1.856782in}}{\pgfqpoint{2.830246in}{1.860054in}}{\pgfqpoint{2.836070in}{1.865878in}}%
\pgfpathcurveto{\pgfqpoint{2.841894in}{1.871702in}}{\pgfqpoint{2.845167in}{1.879602in}}{\pgfqpoint{2.845167in}{1.887839in}}%
\pgfpathcurveto{\pgfqpoint{2.845167in}{1.896075in}}{\pgfqpoint{2.841894in}{1.903975in}}{\pgfqpoint{2.836070in}{1.909799in}}%
\pgfpathcurveto{\pgfqpoint{2.830246in}{1.915623in}}{\pgfqpoint{2.822346in}{1.918895in}}{\pgfqpoint{2.814110in}{1.918895in}}%
\pgfpathcurveto{\pgfqpoint{2.805874in}{1.918895in}}{\pgfqpoint{2.797974in}{1.915623in}}{\pgfqpoint{2.792150in}{1.909799in}}%
\pgfpathcurveto{\pgfqpoint{2.786326in}{1.903975in}}{\pgfqpoint{2.783054in}{1.896075in}}{\pgfqpoint{2.783054in}{1.887839in}}%
\pgfpathcurveto{\pgfqpoint{2.783054in}{1.879602in}}{\pgfqpoint{2.786326in}{1.871702in}}{\pgfqpoint{2.792150in}{1.865878in}}%
\pgfpathcurveto{\pgfqpoint{2.797974in}{1.860054in}}{\pgfqpoint{2.805874in}{1.856782in}}{\pgfqpoint{2.814110in}{1.856782in}}%
\pgfpathclose%
\pgfusepath{stroke,fill}%
\end{pgfscope}%
\begin{pgfscope}%
\pgfpathrectangle{\pgfqpoint{0.100000in}{0.212622in}}{\pgfqpoint{3.696000in}{3.696000in}}%
\pgfusepath{clip}%
\pgfsetbuttcap%
\pgfsetroundjoin%
\definecolor{currentfill}{rgb}{0.121569,0.466667,0.705882}%
\pgfsetfillcolor{currentfill}%
\pgfsetfillopacity{0.457944}%
\pgfsetlinewidth{1.003750pt}%
\definecolor{currentstroke}{rgb}{0.121569,0.466667,0.705882}%
\pgfsetstrokecolor{currentstroke}%
\pgfsetstrokeopacity{0.457944}%
\pgfsetdash{}{0pt}%
\pgfpathmoveto{\pgfqpoint{1.387624in}{2.115253in}}%
\pgfpathcurveto{\pgfqpoint{1.395861in}{2.115253in}}{\pgfqpoint{1.403761in}{2.118525in}}{\pgfqpoint{1.409585in}{2.124349in}}%
\pgfpathcurveto{\pgfqpoint{1.415409in}{2.130173in}}{\pgfqpoint{1.418681in}{2.138073in}}{\pgfqpoint{1.418681in}{2.146309in}}%
\pgfpathcurveto{\pgfqpoint{1.418681in}{2.154545in}}{\pgfqpoint{1.415409in}{2.162445in}}{\pgfqpoint{1.409585in}{2.168269in}}%
\pgfpathcurveto{\pgfqpoint{1.403761in}{2.174093in}}{\pgfqpoint{1.395861in}{2.177366in}}{\pgfqpoint{1.387624in}{2.177366in}}%
\pgfpathcurveto{\pgfqpoint{1.379388in}{2.177366in}}{\pgfqpoint{1.371488in}{2.174093in}}{\pgfqpoint{1.365664in}{2.168269in}}%
\pgfpathcurveto{\pgfqpoint{1.359840in}{2.162445in}}{\pgfqpoint{1.356568in}{2.154545in}}{\pgfqpoint{1.356568in}{2.146309in}}%
\pgfpathcurveto{\pgfqpoint{1.356568in}{2.138073in}}{\pgfqpoint{1.359840in}{2.130173in}}{\pgfqpoint{1.365664in}{2.124349in}}%
\pgfpathcurveto{\pgfqpoint{1.371488in}{2.118525in}}{\pgfqpoint{1.379388in}{2.115253in}}{\pgfqpoint{1.387624in}{2.115253in}}%
\pgfpathclose%
\pgfusepath{stroke,fill}%
\end{pgfscope}%
\begin{pgfscope}%
\pgfpathrectangle{\pgfqpoint{0.100000in}{0.212622in}}{\pgfqpoint{3.696000in}{3.696000in}}%
\pgfusepath{clip}%
\pgfsetbuttcap%
\pgfsetroundjoin%
\definecolor{currentfill}{rgb}{0.121569,0.466667,0.705882}%
\pgfsetfillcolor{currentfill}%
\pgfsetfillopacity{0.458036}%
\pgfsetlinewidth{1.003750pt}%
\definecolor{currentstroke}{rgb}{0.121569,0.466667,0.705882}%
\pgfsetstrokecolor{currentstroke}%
\pgfsetstrokeopacity{0.458036}%
\pgfsetdash{}{0pt}%
\pgfpathmoveto{\pgfqpoint{2.815779in}{1.856481in}}%
\pgfpathcurveto{\pgfqpoint{2.824015in}{1.856481in}}{\pgfqpoint{2.831915in}{1.859753in}}{\pgfqpoint{2.837739in}{1.865577in}}%
\pgfpathcurveto{\pgfqpoint{2.843563in}{1.871401in}}{\pgfqpoint{2.846836in}{1.879301in}}{\pgfqpoint{2.846836in}{1.887538in}}%
\pgfpathcurveto{\pgfqpoint{2.846836in}{1.895774in}}{\pgfqpoint{2.843563in}{1.903674in}}{\pgfqpoint{2.837739in}{1.909498in}}%
\pgfpathcurveto{\pgfqpoint{2.831915in}{1.915322in}}{\pgfqpoint{2.824015in}{1.918594in}}{\pgfqpoint{2.815779in}{1.918594in}}%
\pgfpathcurveto{\pgfqpoint{2.807543in}{1.918594in}}{\pgfqpoint{2.799643in}{1.915322in}}{\pgfqpoint{2.793819in}{1.909498in}}%
\pgfpathcurveto{\pgfqpoint{2.787995in}{1.903674in}}{\pgfqpoint{2.784723in}{1.895774in}}{\pgfqpoint{2.784723in}{1.887538in}}%
\pgfpathcurveto{\pgfqpoint{2.784723in}{1.879301in}}{\pgfqpoint{2.787995in}{1.871401in}}{\pgfqpoint{2.793819in}{1.865577in}}%
\pgfpathcurveto{\pgfqpoint{2.799643in}{1.859753in}}{\pgfqpoint{2.807543in}{1.856481in}}{\pgfqpoint{2.815779in}{1.856481in}}%
\pgfpathclose%
\pgfusepath{stroke,fill}%
\end{pgfscope}%
\begin{pgfscope}%
\pgfpathrectangle{\pgfqpoint{0.100000in}{0.212622in}}{\pgfqpoint{3.696000in}{3.696000in}}%
\pgfusepath{clip}%
\pgfsetbuttcap%
\pgfsetroundjoin%
\definecolor{currentfill}{rgb}{0.121569,0.466667,0.705882}%
\pgfsetfillcolor{currentfill}%
\pgfsetfillopacity{0.458264}%
\pgfsetlinewidth{1.003750pt}%
\definecolor{currentstroke}{rgb}{0.121569,0.466667,0.705882}%
\pgfsetstrokecolor{currentstroke}%
\pgfsetstrokeopacity{0.458264}%
\pgfsetdash{}{0pt}%
\pgfpathmoveto{\pgfqpoint{2.817915in}{1.856100in}}%
\pgfpathcurveto{\pgfqpoint{2.826151in}{1.856100in}}{\pgfqpoint{2.834051in}{1.859373in}}{\pgfqpoint{2.839875in}{1.865197in}}%
\pgfpathcurveto{\pgfqpoint{2.845699in}{1.871020in}}{\pgfqpoint{2.848971in}{1.878921in}}{\pgfqpoint{2.848971in}{1.887157in}}%
\pgfpathcurveto{\pgfqpoint{2.848971in}{1.895393in}}{\pgfqpoint{2.845699in}{1.903293in}}{\pgfqpoint{2.839875in}{1.909117in}}%
\pgfpathcurveto{\pgfqpoint{2.834051in}{1.914941in}}{\pgfqpoint{2.826151in}{1.918213in}}{\pgfqpoint{2.817915in}{1.918213in}}%
\pgfpathcurveto{\pgfqpoint{2.809679in}{1.918213in}}{\pgfqpoint{2.801779in}{1.914941in}}{\pgfqpoint{2.795955in}{1.909117in}}%
\pgfpathcurveto{\pgfqpoint{2.790131in}{1.903293in}}{\pgfqpoint{2.786858in}{1.895393in}}{\pgfqpoint{2.786858in}{1.887157in}}%
\pgfpathcurveto{\pgfqpoint{2.786858in}{1.878921in}}{\pgfqpoint{2.790131in}{1.871020in}}{\pgfqpoint{2.795955in}{1.865197in}}%
\pgfpathcurveto{\pgfqpoint{2.801779in}{1.859373in}}{\pgfqpoint{2.809679in}{1.856100in}}{\pgfqpoint{2.817915in}{1.856100in}}%
\pgfpathclose%
\pgfusepath{stroke,fill}%
\end{pgfscope}%
\begin{pgfscope}%
\pgfpathrectangle{\pgfqpoint{0.100000in}{0.212622in}}{\pgfqpoint{3.696000in}{3.696000in}}%
\pgfusepath{clip}%
\pgfsetbuttcap%
\pgfsetroundjoin%
\definecolor{currentfill}{rgb}{0.121569,0.466667,0.705882}%
\pgfsetfillcolor{currentfill}%
\pgfsetfillopacity{0.458630}%
\pgfsetlinewidth{1.003750pt}%
\definecolor{currentstroke}{rgb}{0.121569,0.466667,0.705882}%
\pgfsetstrokecolor{currentstroke}%
\pgfsetstrokeopacity{0.458630}%
\pgfsetdash{}{0pt}%
\pgfpathmoveto{\pgfqpoint{2.821177in}{1.855622in}}%
\pgfpathcurveto{\pgfqpoint{2.829414in}{1.855622in}}{\pgfqpoint{2.837314in}{1.858895in}}{\pgfqpoint{2.843138in}{1.864719in}}%
\pgfpathcurveto{\pgfqpoint{2.848961in}{1.870543in}}{\pgfqpoint{2.852234in}{1.878443in}}{\pgfqpoint{2.852234in}{1.886679in}}%
\pgfpathcurveto{\pgfqpoint{2.852234in}{1.894915in}}{\pgfqpoint{2.848961in}{1.902815in}}{\pgfqpoint{2.843138in}{1.908639in}}%
\pgfpathcurveto{\pgfqpoint{2.837314in}{1.914463in}}{\pgfqpoint{2.829414in}{1.917735in}}{\pgfqpoint{2.821177in}{1.917735in}}%
\pgfpathcurveto{\pgfqpoint{2.812941in}{1.917735in}}{\pgfqpoint{2.805041in}{1.914463in}}{\pgfqpoint{2.799217in}{1.908639in}}%
\pgfpathcurveto{\pgfqpoint{2.793393in}{1.902815in}}{\pgfqpoint{2.790121in}{1.894915in}}{\pgfqpoint{2.790121in}{1.886679in}}%
\pgfpathcurveto{\pgfqpoint{2.790121in}{1.878443in}}{\pgfqpoint{2.793393in}{1.870543in}}{\pgfqpoint{2.799217in}{1.864719in}}%
\pgfpathcurveto{\pgfqpoint{2.805041in}{1.858895in}}{\pgfqpoint{2.812941in}{1.855622in}}{\pgfqpoint{2.821177in}{1.855622in}}%
\pgfpathclose%
\pgfusepath{stroke,fill}%
\end{pgfscope}%
\begin{pgfscope}%
\pgfpathrectangle{\pgfqpoint{0.100000in}{0.212622in}}{\pgfqpoint{3.696000in}{3.696000in}}%
\pgfusepath{clip}%
\pgfsetbuttcap%
\pgfsetroundjoin%
\definecolor{currentfill}{rgb}{0.121569,0.466667,0.705882}%
\pgfsetfillcolor{currentfill}%
\pgfsetfillopacity{0.458819}%
\pgfsetlinewidth{1.003750pt}%
\definecolor{currentstroke}{rgb}{0.121569,0.466667,0.705882}%
\pgfsetstrokecolor{currentstroke}%
\pgfsetstrokeopacity{0.458819}%
\pgfsetdash{}{0pt}%
\pgfpathmoveto{\pgfqpoint{2.826105in}{1.854546in}}%
\pgfpathcurveto{\pgfqpoint{2.834342in}{1.854546in}}{\pgfqpoint{2.842242in}{1.857818in}}{\pgfqpoint{2.848066in}{1.863642in}}%
\pgfpathcurveto{\pgfqpoint{2.853890in}{1.869466in}}{\pgfqpoint{2.857162in}{1.877366in}}{\pgfqpoint{2.857162in}{1.885603in}}%
\pgfpathcurveto{\pgfqpoint{2.857162in}{1.893839in}}{\pgfqpoint{2.853890in}{1.901739in}}{\pgfqpoint{2.848066in}{1.907563in}}%
\pgfpathcurveto{\pgfqpoint{2.842242in}{1.913387in}}{\pgfqpoint{2.834342in}{1.916659in}}{\pgfqpoint{2.826105in}{1.916659in}}%
\pgfpathcurveto{\pgfqpoint{2.817869in}{1.916659in}}{\pgfqpoint{2.809969in}{1.913387in}}{\pgfqpoint{2.804145in}{1.907563in}}%
\pgfpathcurveto{\pgfqpoint{2.798321in}{1.901739in}}{\pgfqpoint{2.795049in}{1.893839in}}{\pgfqpoint{2.795049in}{1.885603in}}%
\pgfpathcurveto{\pgfqpoint{2.795049in}{1.877366in}}{\pgfqpoint{2.798321in}{1.869466in}}{\pgfqpoint{2.804145in}{1.863642in}}%
\pgfpathcurveto{\pgfqpoint{2.809969in}{1.857818in}}{\pgfqpoint{2.817869in}{1.854546in}}{\pgfqpoint{2.826105in}{1.854546in}}%
\pgfpathclose%
\pgfusepath{stroke,fill}%
\end{pgfscope}%
\begin{pgfscope}%
\pgfpathrectangle{\pgfqpoint{0.100000in}{0.212622in}}{\pgfqpoint{3.696000in}{3.696000in}}%
\pgfusepath{clip}%
\pgfsetbuttcap%
\pgfsetroundjoin%
\definecolor{currentfill}{rgb}{0.121569,0.466667,0.705882}%
\pgfsetfillcolor{currentfill}%
\pgfsetfillopacity{0.459333}%
\pgfsetlinewidth{1.003750pt}%
\definecolor{currentstroke}{rgb}{0.121569,0.466667,0.705882}%
\pgfsetstrokecolor{currentstroke}%
\pgfsetstrokeopacity{0.459333}%
\pgfsetdash{}{0pt}%
\pgfpathmoveto{\pgfqpoint{1.385843in}{2.115435in}}%
\pgfpathcurveto{\pgfqpoint{1.394080in}{2.115435in}}{\pgfqpoint{1.401980in}{2.118707in}}{\pgfqpoint{1.407804in}{2.124531in}}%
\pgfpathcurveto{\pgfqpoint{1.413628in}{2.130355in}}{\pgfqpoint{1.416900in}{2.138255in}}{\pgfqpoint{1.416900in}{2.146492in}}%
\pgfpathcurveto{\pgfqpoint{1.416900in}{2.154728in}}{\pgfqpoint{1.413628in}{2.162628in}}{\pgfqpoint{1.407804in}{2.168452in}}%
\pgfpathcurveto{\pgfqpoint{1.401980in}{2.174276in}}{\pgfqpoint{1.394080in}{2.177548in}}{\pgfqpoint{1.385843in}{2.177548in}}%
\pgfpathcurveto{\pgfqpoint{1.377607in}{2.177548in}}{\pgfqpoint{1.369707in}{2.174276in}}{\pgfqpoint{1.363883in}{2.168452in}}%
\pgfpathcurveto{\pgfqpoint{1.358059in}{2.162628in}}{\pgfqpoint{1.354787in}{2.154728in}}{\pgfqpoint{1.354787in}{2.146492in}}%
\pgfpathcurveto{\pgfqpoint{1.354787in}{2.138255in}}{\pgfqpoint{1.358059in}{2.130355in}}{\pgfqpoint{1.363883in}{2.124531in}}%
\pgfpathcurveto{\pgfqpoint{1.369707in}{2.118707in}}{\pgfqpoint{1.377607in}{2.115435in}}{\pgfqpoint{1.385843in}{2.115435in}}%
\pgfpathclose%
\pgfusepath{stroke,fill}%
\end{pgfscope}%
\begin{pgfscope}%
\pgfpathrectangle{\pgfqpoint{0.100000in}{0.212622in}}{\pgfqpoint{3.696000in}{3.696000in}}%
\pgfusepath{clip}%
\pgfsetbuttcap%
\pgfsetroundjoin%
\definecolor{currentfill}{rgb}{0.121569,0.466667,0.705882}%
\pgfsetfillcolor{currentfill}%
\pgfsetfillopacity{0.459373}%
\pgfsetlinewidth{1.003750pt}%
\definecolor{currentstroke}{rgb}{0.121569,0.466667,0.705882}%
\pgfsetstrokecolor{currentstroke}%
\pgfsetstrokeopacity{0.459373}%
\pgfsetdash{}{0pt}%
\pgfpathmoveto{\pgfqpoint{2.832649in}{1.853299in}}%
\pgfpathcurveto{\pgfqpoint{2.840885in}{1.853299in}}{\pgfqpoint{2.848786in}{1.856571in}}{\pgfqpoint{2.854609in}{1.862395in}}%
\pgfpathcurveto{\pgfqpoint{2.860433in}{1.868219in}}{\pgfqpoint{2.863706in}{1.876119in}}{\pgfqpoint{2.863706in}{1.884355in}}%
\pgfpathcurveto{\pgfqpoint{2.863706in}{1.892591in}}{\pgfqpoint{2.860433in}{1.900491in}}{\pgfqpoint{2.854609in}{1.906315in}}%
\pgfpathcurveto{\pgfqpoint{2.848786in}{1.912139in}}{\pgfqpoint{2.840885in}{1.915412in}}{\pgfqpoint{2.832649in}{1.915412in}}%
\pgfpathcurveto{\pgfqpoint{2.824413in}{1.915412in}}{\pgfqpoint{2.816513in}{1.912139in}}{\pgfqpoint{2.810689in}{1.906315in}}%
\pgfpathcurveto{\pgfqpoint{2.804865in}{1.900491in}}{\pgfqpoint{2.801593in}{1.892591in}}{\pgfqpoint{2.801593in}{1.884355in}}%
\pgfpathcurveto{\pgfqpoint{2.801593in}{1.876119in}}{\pgfqpoint{2.804865in}{1.868219in}}{\pgfqpoint{2.810689in}{1.862395in}}%
\pgfpathcurveto{\pgfqpoint{2.816513in}{1.856571in}}{\pgfqpoint{2.824413in}{1.853299in}}{\pgfqpoint{2.832649in}{1.853299in}}%
\pgfpathclose%
\pgfusepath{stroke,fill}%
\end{pgfscope}%
\begin{pgfscope}%
\pgfpathrectangle{\pgfqpoint{0.100000in}{0.212622in}}{\pgfqpoint{3.696000in}{3.696000in}}%
\pgfusepath{clip}%
\pgfsetbuttcap%
\pgfsetroundjoin%
\definecolor{currentfill}{rgb}{0.121569,0.466667,0.705882}%
\pgfsetfillcolor{currentfill}%
\pgfsetfillopacity{0.459993}%
\pgfsetlinewidth{1.003750pt}%
\definecolor{currentstroke}{rgb}{0.121569,0.466667,0.705882}%
\pgfsetstrokecolor{currentstroke}%
\pgfsetstrokeopacity{0.459993}%
\pgfsetdash{}{0pt}%
\pgfpathmoveto{\pgfqpoint{2.840069in}{1.851813in}}%
\pgfpathcurveto{\pgfqpoint{2.848306in}{1.851813in}}{\pgfqpoint{2.856206in}{1.855085in}}{\pgfqpoint{2.862030in}{1.860909in}}%
\pgfpathcurveto{\pgfqpoint{2.867854in}{1.866733in}}{\pgfqpoint{2.871126in}{1.874633in}}{\pgfqpoint{2.871126in}{1.882869in}}%
\pgfpathcurveto{\pgfqpoint{2.871126in}{1.891105in}}{\pgfqpoint{2.867854in}{1.899006in}}{\pgfqpoint{2.862030in}{1.904829in}}%
\pgfpathcurveto{\pgfqpoint{2.856206in}{1.910653in}}{\pgfqpoint{2.848306in}{1.913926in}}{\pgfqpoint{2.840069in}{1.913926in}}%
\pgfpathcurveto{\pgfqpoint{2.831833in}{1.913926in}}{\pgfqpoint{2.823933in}{1.910653in}}{\pgfqpoint{2.818109in}{1.904829in}}%
\pgfpathcurveto{\pgfqpoint{2.812285in}{1.899006in}}{\pgfqpoint{2.809013in}{1.891105in}}{\pgfqpoint{2.809013in}{1.882869in}}%
\pgfpathcurveto{\pgfqpoint{2.809013in}{1.874633in}}{\pgfqpoint{2.812285in}{1.866733in}}{\pgfqpoint{2.818109in}{1.860909in}}%
\pgfpathcurveto{\pgfqpoint{2.823933in}{1.855085in}}{\pgfqpoint{2.831833in}{1.851813in}}{\pgfqpoint{2.840069in}{1.851813in}}%
\pgfpathclose%
\pgfusepath{stroke,fill}%
\end{pgfscope}%
\begin{pgfscope}%
\pgfpathrectangle{\pgfqpoint{0.100000in}{0.212622in}}{\pgfqpoint{3.696000in}{3.696000in}}%
\pgfusepath{clip}%
\pgfsetbuttcap%
\pgfsetroundjoin%
\definecolor{currentfill}{rgb}{0.121569,0.466667,0.705882}%
\pgfsetfillcolor{currentfill}%
\pgfsetfillopacity{0.460382}%
\pgfsetlinewidth{1.003750pt}%
\definecolor{currentstroke}{rgb}{0.121569,0.466667,0.705882}%
\pgfsetstrokecolor{currentstroke}%
\pgfsetstrokeopacity{0.460382}%
\pgfsetdash{}{0pt}%
\pgfpathmoveto{\pgfqpoint{1.383189in}{2.115630in}}%
\pgfpathcurveto{\pgfqpoint{1.391425in}{2.115630in}}{\pgfqpoint{1.399325in}{2.118902in}}{\pgfqpoint{1.405149in}{2.124726in}}%
\pgfpathcurveto{\pgfqpoint{1.410973in}{2.130550in}}{\pgfqpoint{1.414245in}{2.138450in}}{\pgfqpoint{1.414245in}{2.146686in}}%
\pgfpathcurveto{\pgfqpoint{1.414245in}{2.154923in}}{\pgfqpoint{1.410973in}{2.162823in}}{\pgfqpoint{1.405149in}{2.168647in}}%
\pgfpathcurveto{\pgfqpoint{1.399325in}{2.174470in}}{\pgfqpoint{1.391425in}{2.177743in}}{\pgfqpoint{1.383189in}{2.177743in}}%
\pgfpathcurveto{\pgfqpoint{1.374952in}{2.177743in}}{\pgfqpoint{1.367052in}{2.174470in}}{\pgfqpoint{1.361228in}{2.168647in}}%
\pgfpathcurveto{\pgfqpoint{1.355405in}{2.162823in}}{\pgfqpoint{1.352132in}{2.154923in}}{\pgfqpoint{1.352132in}{2.146686in}}%
\pgfpathcurveto{\pgfqpoint{1.352132in}{2.138450in}}{\pgfqpoint{1.355405in}{2.130550in}}{\pgfqpoint{1.361228in}{2.124726in}}%
\pgfpathcurveto{\pgfqpoint{1.367052in}{2.118902in}}{\pgfqpoint{1.374952in}{2.115630in}}{\pgfqpoint{1.383189in}{2.115630in}}%
\pgfpathclose%
\pgfusepath{stroke,fill}%
\end{pgfscope}%
\begin{pgfscope}%
\pgfpathrectangle{\pgfqpoint{0.100000in}{0.212622in}}{\pgfqpoint{3.696000in}{3.696000in}}%
\pgfusepath{clip}%
\pgfsetbuttcap%
\pgfsetroundjoin%
\definecolor{currentfill}{rgb}{0.121569,0.466667,0.705882}%
\pgfsetfillcolor{currentfill}%
\pgfsetfillopacity{0.460446}%
\pgfsetlinewidth{1.003750pt}%
\definecolor{currentstroke}{rgb}{0.121569,0.466667,0.705882}%
\pgfsetstrokecolor{currentstroke}%
\pgfsetstrokeopacity{0.460446}%
\pgfsetdash{}{0pt}%
\pgfpathmoveto{\pgfqpoint{2.853985in}{1.848086in}}%
\pgfpathcurveto{\pgfqpoint{2.862221in}{1.848086in}}{\pgfqpoint{2.870121in}{1.851358in}}{\pgfqpoint{2.875945in}{1.857182in}}%
\pgfpathcurveto{\pgfqpoint{2.881769in}{1.863006in}}{\pgfqpoint{2.885041in}{1.870906in}}{\pgfqpoint{2.885041in}{1.879142in}}%
\pgfpathcurveto{\pgfqpoint{2.885041in}{1.887378in}}{\pgfqpoint{2.881769in}{1.895278in}}{\pgfqpoint{2.875945in}{1.901102in}}%
\pgfpathcurveto{\pgfqpoint{2.870121in}{1.906926in}}{\pgfqpoint{2.862221in}{1.910199in}}{\pgfqpoint{2.853985in}{1.910199in}}%
\pgfpathcurveto{\pgfqpoint{2.845749in}{1.910199in}}{\pgfqpoint{2.837848in}{1.906926in}}{\pgfqpoint{2.832025in}{1.901102in}}%
\pgfpathcurveto{\pgfqpoint{2.826201in}{1.895278in}}{\pgfqpoint{2.822928in}{1.887378in}}{\pgfqpoint{2.822928in}{1.879142in}}%
\pgfpathcurveto{\pgfqpoint{2.822928in}{1.870906in}}{\pgfqpoint{2.826201in}{1.863006in}}{\pgfqpoint{2.832025in}{1.857182in}}%
\pgfpathcurveto{\pgfqpoint{2.837848in}{1.851358in}}{\pgfqpoint{2.845749in}{1.848086in}}{\pgfqpoint{2.853985in}{1.848086in}}%
\pgfpathclose%
\pgfusepath{stroke,fill}%
\end{pgfscope}%
\begin{pgfscope}%
\pgfpathrectangle{\pgfqpoint{0.100000in}{0.212622in}}{\pgfqpoint{3.696000in}{3.696000in}}%
\pgfusepath{clip}%
\pgfsetbuttcap%
\pgfsetroundjoin%
\definecolor{currentfill}{rgb}{0.121569,0.466667,0.705882}%
\pgfsetfillcolor{currentfill}%
\pgfsetfillopacity{0.460496}%
\pgfsetlinewidth{1.003750pt}%
\definecolor{currentstroke}{rgb}{0.121569,0.466667,0.705882}%
\pgfsetstrokecolor{currentstroke}%
\pgfsetstrokeopacity{0.460496}%
\pgfsetdash{}{0pt}%
\pgfpathmoveto{\pgfqpoint{2.848754in}{1.849804in}}%
\pgfpathcurveto{\pgfqpoint{2.856990in}{1.849804in}}{\pgfqpoint{2.864890in}{1.853076in}}{\pgfqpoint{2.870714in}{1.858900in}}%
\pgfpathcurveto{\pgfqpoint{2.876538in}{1.864724in}}{\pgfqpoint{2.879810in}{1.872624in}}{\pgfqpoint{2.879810in}{1.880860in}}%
\pgfpathcurveto{\pgfqpoint{2.879810in}{1.889096in}}{\pgfqpoint{2.876538in}{1.896996in}}{\pgfqpoint{2.870714in}{1.902820in}}%
\pgfpathcurveto{\pgfqpoint{2.864890in}{1.908644in}}{\pgfqpoint{2.856990in}{1.911917in}}{\pgfqpoint{2.848754in}{1.911917in}}%
\pgfpathcurveto{\pgfqpoint{2.840517in}{1.911917in}}{\pgfqpoint{2.832617in}{1.908644in}}{\pgfqpoint{2.826793in}{1.902820in}}%
\pgfpathcurveto{\pgfqpoint{2.820970in}{1.896996in}}{\pgfqpoint{2.817697in}{1.889096in}}{\pgfqpoint{2.817697in}{1.880860in}}%
\pgfpathcurveto{\pgfqpoint{2.817697in}{1.872624in}}{\pgfqpoint{2.820970in}{1.864724in}}{\pgfqpoint{2.826793in}{1.858900in}}%
\pgfpathcurveto{\pgfqpoint{2.832617in}{1.853076in}}{\pgfqpoint{2.840517in}{1.849804in}}{\pgfqpoint{2.848754in}{1.849804in}}%
\pgfpathclose%
\pgfusepath{stroke,fill}%
\end{pgfscope}%
\begin{pgfscope}%
\pgfpathrectangle{\pgfqpoint{0.100000in}{0.212622in}}{\pgfqpoint{3.696000in}{3.696000in}}%
\pgfusepath{clip}%
\pgfsetbuttcap%
\pgfsetroundjoin%
\definecolor{currentfill}{rgb}{0.121569,0.466667,0.705882}%
\pgfsetfillcolor{currentfill}%
\pgfsetfillopacity{0.460500}%
\pgfsetlinewidth{1.003750pt}%
\definecolor{currentstroke}{rgb}{0.121569,0.466667,0.705882}%
\pgfsetstrokecolor{currentstroke}%
\pgfsetstrokeopacity{0.460500}%
\pgfsetdash{}{0pt}%
\pgfpathmoveto{\pgfqpoint{2.843919in}{1.851233in}}%
\pgfpathcurveto{\pgfqpoint{2.852155in}{1.851233in}}{\pgfqpoint{2.860055in}{1.854505in}}{\pgfqpoint{2.865879in}{1.860329in}}%
\pgfpathcurveto{\pgfqpoint{2.871703in}{1.866153in}}{\pgfqpoint{2.874976in}{1.874053in}}{\pgfqpoint{2.874976in}{1.882289in}}%
\pgfpathcurveto{\pgfqpoint{2.874976in}{1.890526in}}{\pgfqpoint{2.871703in}{1.898426in}}{\pgfqpoint{2.865879in}{1.904250in}}%
\pgfpathcurveto{\pgfqpoint{2.860055in}{1.910073in}}{\pgfqpoint{2.852155in}{1.913346in}}{\pgfqpoint{2.843919in}{1.913346in}}%
\pgfpathcurveto{\pgfqpoint{2.835683in}{1.913346in}}{\pgfqpoint{2.827783in}{1.910073in}}{\pgfqpoint{2.821959in}{1.904250in}}%
\pgfpathcurveto{\pgfqpoint{2.816135in}{1.898426in}}{\pgfqpoint{2.812863in}{1.890526in}}{\pgfqpoint{2.812863in}{1.882289in}}%
\pgfpathcurveto{\pgfqpoint{2.812863in}{1.874053in}}{\pgfqpoint{2.816135in}{1.866153in}}{\pgfqpoint{2.821959in}{1.860329in}}%
\pgfpathcurveto{\pgfqpoint{2.827783in}{1.854505in}}{\pgfqpoint{2.835683in}{1.851233in}}{\pgfqpoint{2.843919in}{1.851233in}}%
\pgfpathclose%
\pgfusepath{stroke,fill}%
\end{pgfscope}%
\begin{pgfscope}%
\pgfpathrectangle{\pgfqpoint{0.100000in}{0.212622in}}{\pgfqpoint{3.696000in}{3.696000in}}%
\pgfusepath{clip}%
\pgfsetbuttcap%
\pgfsetroundjoin%
\definecolor{currentfill}{rgb}{0.121569,0.466667,0.705882}%
\pgfsetfillcolor{currentfill}%
\pgfsetfillopacity{0.461050}%
\pgfsetlinewidth{1.003750pt}%
\definecolor{currentstroke}{rgb}{0.121569,0.466667,0.705882}%
\pgfsetstrokecolor{currentstroke}%
\pgfsetstrokeopacity{0.461050}%
\pgfsetdash{}{0pt}%
\pgfpathmoveto{\pgfqpoint{2.858959in}{1.847160in}}%
\pgfpathcurveto{\pgfqpoint{2.867196in}{1.847160in}}{\pgfqpoint{2.875096in}{1.850432in}}{\pgfqpoint{2.880920in}{1.856256in}}%
\pgfpathcurveto{\pgfqpoint{2.886744in}{1.862080in}}{\pgfqpoint{2.890016in}{1.869980in}}{\pgfqpoint{2.890016in}{1.878217in}}%
\pgfpathcurveto{\pgfqpoint{2.890016in}{1.886453in}}{\pgfqpoint{2.886744in}{1.894353in}}{\pgfqpoint{2.880920in}{1.900177in}}%
\pgfpathcurveto{\pgfqpoint{2.875096in}{1.906001in}}{\pgfqpoint{2.867196in}{1.909273in}}{\pgfqpoint{2.858959in}{1.909273in}}%
\pgfpathcurveto{\pgfqpoint{2.850723in}{1.909273in}}{\pgfqpoint{2.842823in}{1.906001in}}{\pgfqpoint{2.836999in}{1.900177in}}%
\pgfpathcurveto{\pgfqpoint{2.831175in}{1.894353in}}{\pgfqpoint{2.827903in}{1.886453in}}{\pgfqpoint{2.827903in}{1.878217in}}%
\pgfpathcurveto{\pgfqpoint{2.827903in}{1.869980in}}{\pgfqpoint{2.831175in}{1.862080in}}{\pgfqpoint{2.836999in}{1.856256in}}%
\pgfpathcurveto{\pgfqpoint{2.842823in}{1.850432in}}{\pgfqpoint{2.850723in}{1.847160in}}{\pgfqpoint{2.858959in}{1.847160in}}%
\pgfpathclose%
\pgfusepath{stroke,fill}%
\end{pgfscope}%
\begin{pgfscope}%
\pgfpathrectangle{\pgfqpoint{0.100000in}{0.212622in}}{\pgfqpoint{3.696000in}{3.696000in}}%
\pgfusepath{clip}%
\pgfsetbuttcap%
\pgfsetroundjoin%
\definecolor{currentfill}{rgb}{0.121569,0.466667,0.705882}%
\pgfsetfillcolor{currentfill}%
\pgfsetfillopacity{0.461282}%
\pgfsetlinewidth{1.003750pt}%
\definecolor{currentstroke}{rgb}{0.121569,0.466667,0.705882}%
\pgfsetstrokecolor{currentstroke}%
\pgfsetstrokeopacity{0.461282}%
\pgfsetdash{}{0pt}%
\pgfpathmoveto{\pgfqpoint{1.384870in}{2.116398in}}%
\pgfpathcurveto{\pgfqpoint{1.393106in}{2.116398in}}{\pgfqpoint{1.401006in}{2.119670in}}{\pgfqpoint{1.406830in}{2.125494in}}%
\pgfpathcurveto{\pgfqpoint{1.412654in}{2.131318in}}{\pgfqpoint{1.415926in}{2.139218in}}{\pgfqpoint{1.415926in}{2.147454in}}%
\pgfpathcurveto{\pgfqpoint{1.415926in}{2.155690in}}{\pgfqpoint{1.412654in}{2.163591in}}{\pgfqpoint{1.406830in}{2.169414in}}%
\pgfpathcurveto{\pgfqpoint{1.401006in}{2.175238in}}{\pgfqpoint{1.393106in}{2.178511in}}{\pgfqpoint{1.384870in}{2.178511in}}%
\pgfpathcurveto{\pgfqpoint{1.376634in}{2.178511in}}{\pgfqpoint{1.368733in}{2.175238in}}{\pgfqpoint{1.362910in}{2.169414in}}%
\pgfpathcurveto{\pgfqpoint{1.357086in}{2.163591in}}{\pgfqpoint{1.353813in}{2.155690in}}{\pgfqpoint{1.353813in}{2.147454in}}%
\pgfpathcurveto{\pgfqpoint{1.353813in}{2.139218in}}{\pgfqpoint{1.357086in}{2.131318in}}{\pgfqpoint{1.362910in}{2.125494in}}%
\pgfpathcurveto{\pgfqpoint{1.368733in}{2.119670in}}{\pgfqpoint{1.376634in}{2.116398in}}{\pgfqpoint{1.384870in}{2.116398in}}%
\pgfpathclose%
\pgfusepath{stroke,fill}%
\end{pgfscope}%
\begin{pgfscope}%
\pgfpathrectangle{\pgfqpoint{0.100000in}{0.212622in}}{\pgfqpoint{3.696000in}{3.696000in}}%
\pgfusepath{clip}%
\pgfsetbuttcap%
\pgfsetroundjoin%
\definecolor{currentfill}{rgb}{0.121569,0.466667,0.705882}%
\pgfsetfillcolor{currentfill}%
\pgfsetfillopacity{0.461597}%
\pgfsetlinewidth{1.003750pt}%
\definecolor{currentstroke}{rgb}{0.121569,0.466667,0.705882}%
\pgfsetstrokecolor{currentstroke}%
\pgfsetstrokeopacity{0.461597}%
\pgfsetdash{}{0pt}%
\pgfpathmoveto{\pgfqpoint{2.864801in}{1.845904in}}%
\pgfpathcurveto{\pgfqpoint{2.873037in}{1.845904in}}{\pgfqpoint{2.880938in}{1.849176in}}{\pgfqpoint{2.886761in}{1.855000in}}%
\pgfpathcurveto{\pgfqpoint{2.892585in}{1.860824in}}{\pgfqpoint{2.895858in}{1.868724in}}{\pgfqpoint{2.895858in}{1.876960in}}%
\pgfpathcurveto{\pgfqpoint{2.895858in}{1.885197in}}{\pgfqpoint{2.892585in}{1.893097in}}{\pgfqpoint{2.886761in}{1.898921in}}%
\pgfpathcurveto{\pgfqpoint{2.880938in}{1.904745in}}{\pgfqpoint{2.873037in}{1.908017in}}{\pgfqpoint{2.864801in}{1.908017in}}%
\pgfpathcurveto{\pgfqpoint{2.856565in}{1.908017in}}{\pgfqpoint{2.848665in}{1.904745in}}{\pgfqpoint{2.842841in}{1.898921in}}%
\pgfpathcurveto{\pgfqpoint{2.837017in}{1.893097in}}{\pgfqpoint{2.833745in}{1.885197in}}{\pgfqpoint{2.833745in}{1.876960in}}%
\pgfpathcurveto{\pgfqpoint{2.833745in}{1.868724in}}{\pgfqpoint{2.837017in}{1.860824in}}{\pgfqpoint{2.842841in}{1.855000in}}%
\pgfpathcurveto{\pgfqpoint{2.848665in}{1.849176in}}{\pgfqpoint{2.856565in}{1.845904in}}{\pgfqpoint{2.864801in}{1.845904in}}%
\pgfpathclose%
\pgfusepath{stroke,fill}%
\end{pgfscope}%
\begin{pgfscope}%
\pgfpathrectangle{\pgfqpoint{0.100000in}{0.212622in}}{\pgfqpoint{3.696000in}{3.696000in}}%
\pgfusepath{clip}%
\pgfsetbuttcap%
\pgfsetroundjoin%
\definecolor{currentfill}{rgb}{0.121569,0.466667,0.705882}%
\pgfsetfillcolor{currentfill}%
\pgfsetfillopacity{0.462029}%
\pgfsetlinewidth{1.003750pt}%
\definecolor{currentstroke}{rgb}{0.121569,0.466667,0.705882}%
\pgfsetstrokecolor{currentstroke}%
\pgfsetstrokeopacity{0.462029}%
\pgfsetdash{}{0pt}%
\pgfpathmoveto{\pgfqpoint{1.383281in}{2.116457in}}%
\pgfpathcurveto{\pgfqpoint{1.391517in}{2.116457in}}{\pgfqpoint{1.399417in}{2.119730in}}{\pgfqpoint{1.405241in}{2.125554in}}%
\pgfpathcurveto{\pgfqpoint{1.411065in}{2.131378in}}{\pgfqpoint{1.414338in}{2.139278in}}{\pgfqpoint{1.414338in}{2.147514in}}%
\pgfpathcurveto{\pgfqpoint{1.414338in}{2.155750in}}{\pgfqpoint{1.411065in}{2.163650in}}{\pgfqpoint{1.405241in}{2.169474in}}%
\pgfpathcurveto{\pgfqpoint{1.399417in}{2.175298in}}{\pgfqpoint{1.391517in}{2.178570in}}{\pgfqpoint{1.383281in}{2.178570in}}%
\pgfpathcurveto{\pgfqpoint{1.375045in}{2.178570in}}{\pgfqpoint{1.367145in}{2.175298in}}{\pgfqpoint{1.361321in}{2.169474in}}%
\pgfpathcurveto{\pgfqpoint{1.355497in}{2.163650in}}{\pgfqpoint{1.352225in}{2.155750in}}{\pgfqpoint{1.352225in}{2.147514in}}%
\pgfpathcurveto{\pgfqpoint{1.352225in}{2.139278in}}{\pgfqpoint{1.355497in}{2.131378in}}{\pgfqpoint{1.361321in}{2.125554in}}%
\pgfpathcurveto{\pgfqpoint{1.367145in}{2.119730in}}{\pgfqpoint{1.375045in}{2.116457in}}{\pgfqpoint{1.383281in}{2.116457in}}%
\pgfpathclose%
\pgfusepath{stroke,fill}%
\end{pgfscope}%
\begin{pgfscope}%
\pgfpathrectangle{\pgfqpoint{0.100000in}{0.212622in}}{\pgfqpoint{3.696000in}{3.696000in}}%
\pgfusepath{clip}%
\pgfsetbuttcap%
\pgfsetroundjoin%
\definecolor{currentfill}{rgb}{0.121569,0.466667,0.705882}%
\pgfsetfillcolor{currentfill}%
\pgfsetfillopacity{0.462226}%
\pgfsetlinewidth{1.003750pt}%
\definecolor{currentstroke}{rgb}{0.121569,0.466667,0.705882}%
\pgfsetstrokecolor{currentstroke}%
\pgfsetstrokeopacity{0.462226}%
\pgfsetdash{}{0pt}%
\pgfpathmoveto{\pgfqpoint{2.870967in}{1.844721in}}%
\pgfpathcurveto{\pgfqpoint{2.879203in}{1.844721in}}{\pgfqpoint{2.887103in}{1.847993in}}{\pgfqpoint{2.892927in}{1.853817in}}%
\pgfpathcurveto{\pgfqpoint{2.898751in}{1.859641in}}{\pgfqpoint{2.902024in}{1.867541in}}{\pgfqpoint{2.902024in}{1.875777in}}%
\pgfpathcurveto{\pgfqpoint{2.902024in}{1.884014in}}{\pgfqpoint{2.898751in}{1.891914in}}{\pgfqpoint{2.892927in}{1.897738in}}%
\pgfpathcurveto{\pgfqpoint{2.887103in}{1.903561in}}{\pgfqpoint{2.879203in}{1.906834in}}{\pgfqpoint{2.870967in}{1.906834in}}%
\pgfpathcurveto{\pgfqpoint{2.862731in}{1.906834in}}{\pgfqpoint{2.854831in}{1.903561in}}{\pgfqpoint{2.849007in}{1.897738in}}%
\pgfpathcurveto{\pgfqpoint{2.843183in}{1.891914in}}{\pgfqpoint{2.839911in}{1.884014in}}{\pgfqpoint{2.839911in}{1.875777in}}%
\pgfpathcurveto{\pgfqpoint{2.839911in}{1.867541in}}{\pgfqpoint{2.843183in}{1.859641in}}{\pgfqpoint{2.849007in}{1.853817in}}%
\pgfpathcurveto{\pgfqpoint{2.854831in}{1.847993in}}{\pgfqpoint{2.862731in}{1.844721in}}{\pgfqpoint{2.870967in}{1.844721in}}%
\pgfpathclose%
\pgfusepath{stroke,fill}%
\end{pgfscope}%
\begin{pgfscope}%
\pgfpathrectangle{\pgfqpoint{0.100000in}{0.212622in}}{\pgfqpoint{3.696000in}{3.696000in}}%
\pgfusepath{clip}%
\pgfsetbuttcap%
\pgfsetroundjoin%
\definecolor{currentfill}{rgb}{0.121569,0.466667,0.705882}%
\pgfsetfillcolor{currentfill}%
\pgfsetfillopacity{0.463011}%
\pgfsetlinewidth{1.003750pt}%
\definecolor{currentstroke}{rgb}{0.121569,0.466667,0.705882}%
\pgfsetstrokecolor{currentstroke}%
\pgfsetstrokeopacity{0.463011}%
\pgfsetdash{}{0pt}%
\pgfpathmoveto{\pgfqpoint{2.877272in}{1.843503in}}%
\pgfpathcurveto{\pgfqpoint{2.885509in}{1.843503in}}{\pgfqpoint{2.893409in}{1.846776in}}{\pgfqpoint{2.899233in}{1.852600in}}%
\pgfpathcurveto{\pgfqpoint{2.905057in}{1.858423in}}{\pgfqpoint{2.908329in}{1.866324in}}{\pgfqpoint{2.908329in}{1.874560in}}%
\pgfpathcurveto{\pgfqpoint{2.908329in}{1.882796in}}{\pgfqpoint{2.905057in}{1.890696in}}{\pgfqpoint{2.899233in}{1.896520in}}%
\pgfpathcurveto{\pgfqpoint{2.893409in}{1.902344in}}{\pgfqpoint{2.885509in}{1.905616in}}{\pgfqpoint{2.877272in}{1.905616in}}%
\pgfpathcurveto{\pgfqpoint{2.869036in}{1.905616in}}{\pgfqpoint{2.861136in}{1.902344in}}{\pgfqpoint{2.855312in}{1.896520in}}%
\pgfpathcurveto{\pgfqpoint{2.849488in}{1.890696in}}{\pgfqpoint{2.846216in}{1.882796in}}{\pgfqpoint{2.846216in}{1.874560in}}%
\pgfpathcurveto{\pgfqpoint{2.846216in}{1.866324in}}{\pgfqpoint{2.849488in}{1.858423in}}{\pgfqpoint{2.855312in}{1.852600in}}%
\pgfpathcurveto{\pgfqpoint{2.861136in}{1.846776in}}{\pgfqpoint{2.869036in}{1.843503in}}{\pgfqpoint{2.877272in}{1.843503in}}%
\pgfpathclose%
\pgfusepath{stroke,fill}%
\end{pgfscope}%
\begin{pgfscope}%
\pgfpathrectangle{\pgfqpoint{0.100000in}{0.212622in}}{\pgfqpoint{3.696000in}{3.696000in}}%
\pgfusepath{clip}%
\pgfsetbuttcap%
\pgfsetroundjoin%
\definecolor{currentfill}{rgb}{0.121569,0.466667,0.705882}%
\pgfsetfillcolor{currentfill}%
\pgfsetfillopacity{0.463205}%
\pgfsetlinewidth{1.003750pt}%
\definecolor{currentstroke}{rgb}{0.121569,0.466667,0.705882}%
\pgfsetstrokecolor{currentstroke}%
\pgfsetstrokeopacity{0.463205}%
\pgfsetdash{}{0pt}%
\pgfpathmoveto{\pgfqpoint{2.881030in}{1.842488in}}%
\pgfpathcurveto{\pgfqpoint{2.889266in}{1.842488in}}{\pgfqpoint{2.897166in}{1.845760in}}{\pgfqpoint{2.902990in}{1.851584in}}%
\pgfpathcurveto{\pgfqpoint{2.908814in}{1.857408in}}{\pgfqpoint{2.912087in}{1.865308in}}{\pgfqpoint{2.912087in}{1.873544in}}%
\pgfpathcurveto{\pgfqpoint{2.912087in}{1.881781in}}{\pgfqpoint{2.908814in}{1.889681in}}{\pgfqpoint{2.902990in}{1.895505in}}%
\pgfpathcurveto{\pgfqpoint{2.897166in}{1.901329in}}{\pgfqpoint{2.889266in}{1.904601in}}{\pgfqpoint{2.881030in}{1.904601in}}%
\pgfpathcurveto{\pgfqpoint{2.872794in}{1.904601in}}{\pgfqpoint{2.864894in}{1.901329in}}{\pgfqpoint{2.859070in}{1.895505in}}%
\pgfpathcurveto{\pgfqpoint{2.853246in}{1.889681in}}{\pgfqpoint{2.849974in}{1.881781in}}{\pgfqpoint{2.849974in}{1.873544in}}%
\pgfpathcurveto{\pgfqpoint{2.849974in}{1.865308in}}{\pgfqpoint{2.853246in}{1.857408in}}{\pgfqpoint{2.859070in}{1.851584in}}%
\pgfpathcurveto{\pgfqpoint{2.864894in}{1.845760in}}{\pgfqpoint{2.872794in}{1.842488in}}{\pgfqpoint{2.881030in}{1.842488in}}%
\pgfpathclose%
\pgfusepath{stroke,fill}%
\end{pgfscope}%
\begin{pgfscope}%
\pgfpathrectangle{\pgfqpoint{0.100000in}{0.212622in}}{\pgfqpoint{3.696000in}{3.696000in}}%
\pgfusepath{clip}%
\pgfsetbuttcap%
\pgfsetroundjoin%
\definecolor{currentfill}{rgb}{0.121569,0.466667,0.705882}%
\pgfsetfillcolor{currentfill}%
\pgfsetfillopacity{0.463351}%
\pgfsetlinewidth{1.003750pt}%
\definecolor{currentstroke}{rgb}{0.121569,0.466667,0.705882}%
\pgfsetstrokecolor{currentstroke}%
\pgfsetstrokeopacity{0.463351}%
\pgfsetdash{}{0pt}%
\pgfpathmoveto{\pgfqpoint{1.380171in}{2.116543in}}%
\pgfpathcurveto{\pgfqpoint{1.388407in}{2.116543in}}{\pgfqpoint{1.396307in}{2.119815in}}{\pgfqpoint{1.402131in}{2.125639in}}%
\pgfpathcurveto{\pgfqpoint{1.407955in}{2.131463in}}{\pgfqpoint{1.411228in}{2.139363in}}{\pgfqpoint{1.411228in}{2.147599in}}%
\pgfpathcurveto{\pgfqpoint{1.411228in}{2.155835in}}{\pgfqpoint{1.407955in}{2.163736in}}{\pgfqpoint{1.402131in}{2.169559in}}%
\pgfpathcurveto{\pgfqpoint{1.396307in}{2.175383in}}{\pgfqpoint{1.388407in}{2.178656in}}{\pgfqpoint{1.380171in}{2.178656in}}%
\pgfpathcurveto{\pgfqpoint{1.371935in}{2.178656in}}{\pgfqpoint{1.364035in}{2.175383in}}{\pgfqpoint{1.358211in}{2.169559in}}%
\pgfpathcurveto{\pgfqpoint{1.352387in}{2.163736in}}{\pgfqpoint{1.349115in}{2.155835in}}{\pgfqpoint{1.349115in}{2.147599in}}%
\pgfpathcurveto{\pgfqpoint{1.349115in}{2.139363in}}{\pgfqpoint{1.352387in}{2.131463in}}{\pgfqpoint{1.358211in}{2.125639in}}%
\pgfpathcurveto{\pgfqpoint{1.364035in}{2.119815in}}{\pgfqpoint{1.371935in}{2.116543in}}{\pgfqpoint{1.380171in}{2.116543in}}%
\pgfpathclose%
\pgfusepath{stroke,fill}%
\end{pgfscope}%
\begin{pgfscope}%
\pgfpathrectangle{\pgfqpoint{0.100000in}{0.212622in}}{\pgfqpoint{3.696000in}{3.696000in}}%
\pgfusepath{clip}%
\pgfsetbuttcap%
\pgfsetroundjoin%
\definecolor{currentfill}{rgb}{0.121569,0.466667,0.705882}%
\pgfsetfillcolor{currentfill}%
\pgfsetfillopacity{0.463622}%
\pgfsetlinewidth{1.003750pt}%
\definecolor{currentstroke}{rgb}{0.121569,0.466667,0.705882}%
\pgfsetstrokecolor{currentstroke}%
\pgfsetstrokeopacity{0.463622}%
\pgfsetdash{}{0pt}%
\pgfpathmoveto{\pgfqpoint{2.885078in}{1.841497in}}%
\pgfpathcurveto{\pgfqpoint{2.893314in}{1.841497in}}{\pgfqpoint{2.901214in}{1.844769in}}{\pgfqpoint{2.907038in}{1.850593in}}%
\pgfpathcurveto{\pgfqpoint{2.912862in}{1.856417in}}{\pgfqpoint{2.916134in}{1.864317in}}{\pgfqpoint{2.916134in}{1.872554in}}%
\pgfpathcurveto{\pgfqpoint{2.916134in}{1.880790in}}{\pgfqpoint{2.912862in}{1.888690in}}{\pgfqpoint{2.907038in}{1.894514in}}%
\pgfpathcurveto{\pgfqpoint{2.901214in}{1.900338in}}{\pgfqpoint{2.893314in}{1.903610in}}{\pgfqpoint{2.885078in}{1.903610in}}%
\pgfpathcurveto{\pgfqpoint{2.876841in}{1.903610in}}{\pgfqpoint{2.868941in}{1.900338in}}{\pgfqpoint{2.863117in}{1.894514in}}%
\pgfpathcurveto{\pgfqpoint{2.857293in}{1.888690in}}{\pgfqpoint{2.854021in}{1.880790in}}{\pgfqpoint{2.854021in}{1.872554in}}%
\pgfpathcurveto{\pgfqpoint{2.854021in}{1.864317in}}{\pgfqpoint{2.857293in}{1.856417in}}{\pgfqpoint{2.863117in}{1.850593in}}%
\pgfpathcurveto{\pgfqpoint{2.868941in}{1.844769in}}{\pgfqpoint{2.876841in}{1.841497in}}{\pgfqpoint{2.885078in}{1.841497in}}%
\pgfpathclose%
\pgfusepath{stroke,fill}%
\end{pgfscope}%
\begin{pgfscope}%
\pgfpathrectangle{\pgfqpoint{0.100000in}{0.212622in}}{\pgfqpoint{3.696000in}{3.696000in}}%
\pgfusepath{clip}%
\pgfsetbuttcap%
\pgfsetroundjoin%
\definecolor{currentfill}{rgb}{0.121569,0.466667,0.705882}%
\pgfsetfillcolor{currentfill}%
\pgfsetfillopacity{0.464179}%
\pgfsetlinewidth{1.003750pt}%
\definecolor{currentstroke}{rgb}{0.121569,0.466667,0.705882}%
\pgfsetstrokecolor{currentstroke}%
\pgfsetstrokeopacity{0.464179}%
\pgfsetdash{}{0pt}%
\pgfpathmoveto{\pgfqpoint{2.889523in}{1.840441in}}%
\pgfpathcurveto{\pgfqpoint{2.897759in}{1.840441in}}{\pgfqpoint{2.905659in}{1.843713in}}{\pgfqpoint{2.911483in}{1.849537in}}%
\pgfpathcurveto{\pgfqpoint{2.917307in}{1.855361in}}{\pgfqpoint{2.920579in}{1.863261in}}{\pgfqpoint{2.920579in}{1.871498in}}%
\pgfpathcurveto{\pgfqpoint{2.920579in}{1.879734in}}{\pgfqpoint{2.917307in}{1.887634in}}{\pgfqpoint{2.911483in}{1.893458in}}%
\pgfpathcurveto{\pgfqpoint{2.905659in}{1.899282in}}{\pgfqpoint{2.897759in}{1.902554in}}{\pgfqpoint{2.889523in}{1.902554in}}%
\pgfpathcurveto{\pgfqpoint{2.881286in}{1.902554in}}{\pgfqpoint{2.873386in}{1.899282in}}{\pgfqpoint{2.867562in}{1.893458in}}%
\pgfpathcurveto{\pgfqpoint{2.861738in}{1.887634in}}{\pgfqpoint{2.858466in}{1.879734in}}{\pgfqpoint{2.858466in}{1.871498in}}%
\pgfpathcurveto{\pgfqpoint{2.858466in}{1.863261in}}{\pgfqpoint{2.861738in}{1.855361in}}{\pgfqpoint{2.867562in}{1.849537in}}%
\pgfpathcurveto{\pgfqpoint{2.873386in}{1.843713in}}{\pgfqpoint{2.881286in}{1.840441in}}{\pgfqpoint{2.889523in}{1.840441in}}%
\pgfpathclose%
\pgfusepath{stroke,fill}%
\end{pgfscope}%
\begin{pgfscope}%
\pgfpathrectangle{\pgfqpoint{0.100000in}{0.212622in}}{\pgfqpoint{3.696000in}{3.696000in}}%
\pgfusepath{clip}%
\pgfsetbuttcap%
\pgfsetroundjoin%
\definecolor{currentfill}{rgb}{0.121569,0.466667,0.705882}%
\pgfsetfillcolor{currentfill}%
\pgfsetfillopacity{0.464580}%
\pgfsetlinewidth{1.003750pt}%
\definecolor{currentstroke}{rgb}{0.121569,0.466667,0.705882}%
\pgfsetstrokecolor{currentstroke}%
\pgfsetstrokeopacity{0.464580}%
\pgfsetdash{}{0pt}%
\pgfpathmoveto{\pgfqpoint{2.891802in}{1.840048in}}%
\pgfpathcurveto{\pgfqpoint{2.900038in}{1.840048in}}{\pgfqpoint{2.907938in}{1.843320in}}{\pgfqpoint{2.913762in}{1.849144in}}%
\pgfpathcurveto{\pgfqpoint{2.919586in}{1.854968in}}{\pgfqpoint{2.922859in}{1.862868in}}{\pgfqpoint{2.922859in}{1.871104in}}%
\pgfpathcurveto{\pgfqpoint{2.922859in}{1.879340in}}{\pgfqpoint{2.919586in}{1.887240in}}{\pgfqpoint{2.913762in}{1.893064in}}%
\pgfpathcurveto{\pgfqpoint{2.907938in}{1.898888in}}{\pgfqpoint{2.900038in}{1.902161in}}{\pgfqpoint{2.891802in}{1.902161in}}%
\pgfpathcurveto{\pgfqpoint{2.883566in}{1.902161in}}{\pgfqpoint{2.875666in}{1.898888in}}{\pgfqpoint{2.869842in}{1.893064in}}%
\pgfpathcurveto{\pgfqpoint{2.864018in}{1.887240in}}{\pgfqpoint{2.860746in}{1.879340in}}{\pgfqpoint{2.860746in}{1.871104in}}%
\pgfpathcurveto{\pgfqpoint{2.860746in}{1.862868in}}{\pgfqpoint{2.864018in}{1.854968in}}{\pgfqpoint{2.869842in}{1.849144in}}%
\pgfpathcurveto{\pgfqpoint{2.875666in}{1.843320in}}{\pgfqpoint{2.883566in}{1.840048in}}{\pgfqpoint{2.891802in}{1.840048in}}%
\pgfpathclose%
\pgfusepath{stroke,fill}%
\end{pgfscope}%
\begin{pgfscope}%
\pgfpathrectangle{\pgfqpoint{0.100000in}{0.212622in}}{\pgfqpoint{3.696000in}{3.696000in}}%
\pgfusepath{clip}%
\pgfsetbuttcap%
\pgfsetroundjoin%
\definecolor{currentfill}{rgb}{0.121569,0.466667,0.705882}%
\pgfsetfillcolor{currentfill}%
\pgfsetfillopacity{0.464581}%
\pgfsetlinewidth{1.003750pt}%
\definecolor{currentstroke}{rgb}{0.121569,0.466667,0.705882}%
\pgfsetstrokecolor{currentstroke}%
\pgfsetstrokeopacity{0.464581}%
\pgfsetdash{}{0pt}%
\pgfpathmoveto{\pgfqpoint{1.378631in}{2.116724in}}%
\pgfpathcurveto{\pgfqpoint{1.386867in}{2.116724in}}{\pgfqpoint{1.394767in}{2.119996in}}{\pgfqpoint{1.400591in}{2.125820in}}%
\pgfpathcurveto{\pgfqpoint{1.406415in}{2.131644in}}{\pgfqpoint{1.409687in}{2.139544in}}{\pgfqpoint{1.409687in}{2.147780in}}%
\pgfpathcurveto{\pgfqpoint{1.409687in}{2.156017in}}{\pgfqpoint{1.406415in}{2.163917in}}{\pgfqpoint{1.400591in}{2.169741in}}%
\pgfpathcurveto{\pgfqpoint{1.394767in}{2.175565in}}{\pgfqpoint{1.386867in}{2.178837in}}{\pgfqpoint{1.378631in}{2.178837in}}%
\pgfpathcurveto{\pgfqpoint{1.370395in}{2.178837in}}{\pgfqpoint{1.362495in}{2.175565in}}{\pgfqpoint{1.356671in}{2.169741in}}%
\pgfpathcurveto{\pgfqpoint{1.350847in}{2.163917in}}{\pgfqpoint{1.347574in}{2.156017in}}{\pgfqpoint{1.347574in}{2.147780in}}%
\pgfpathcurveto{\pgfqpoint{1.347574in}{2.139544in}}{\pgfqpoint{1.350847in}{2.131644in}}{\pgfqpoint{1.356671in}{2.125820in}}%
\pgfpathcurveto{\pgfqpoint{1.362495in}{2.119996in}}{\pgfqpoint{1.370395in}{2.116724in}}{\pgfqpoint{1.378631in}{2.116724in}}%
\pgfpathclose%
\pgfusepath{stroke,fill}%
\end{pgfscope}%
\begin{pgfscope}%
\pgfpathrectangle{\pgfqpoint{0.100000in}{0.212622in}}{\pgfqpoint{3.696000in}{3.696000in}}%
\pgfusepath{clip}%
\pgfsetbuttcap%
\pgfsetroundjoin%
\definecolor{currentfill}{rgb}{0.121569,0.466667,0.705882}%
\pgfsetfillcolor{currentfill}%
\pgfsetfillopacity{0.465077}%
\pgfsetlinewidth{1.003750pt}%
\definecolor{currentstroke}{rgb}{0.121569,0.466667,0.705882}%
\pgfsetstrokecolor{currentstroke}%
\pgfsetstrokeopacity{0.465077}%
\pgfsetdash{}{0pt}%
\pgfpathmoveto{\pgfqpoint{2.894539in}{1.839664in}}%
\pgfpathcurveto{\pgfqpoint{2.902775in}{1.839664in}}{\pgfqpoint{2.910675in}{1.842936in}}{\pgfqpoint{2.916499in}{1.848760in}}%
\pgfpathcurveto{\pgfqpoint{2.922323in}{1.854584in}}{\pgfqpoint{2.925596in}{1.862484in}}{\pgfqpoint{2.925596in}{1.870720in}}%
\pgfpathcurveto{\pgfqpoint{2.925596in}{1.878956in}}{\pgfqpoint{2.922323in}{1.886856in}}{\pgfqpoint{2.916499in}{1.892680in}}%
\pgfpathcurveto{\pgfqpoint{2.910675in}{1.898504in}}{\pgfqpoint{2.902775in}{1.901777in}}{\pgfqpoint{2.894539in}{1.901777in}}%
\pgfpathcurveto{\pgfqpoint{2.886303in}{1.901777in}}{\pgfqpoint{2.878403in}{1.898504in}}{\pgfqpoint{2.872579in}{1.892680in}}%
\pgfpathcurveto{\pgfqpoint{2.866755in}{1.886856in}}{\pgfqpoint{2.863483in}{1.878956in}}{\pgfqpoint{2.863483in}{1.870720in}}%
\pgfpathcurveto{\pgfqpoint{2.863483in}{1.862484in}}{\pgfqpoint{2.866755in}{1.854584in}}{\pgfqpoint{2.872579in}{1.848760in}}%
\pgfpathcurveto{\pgfqpoint{2.878403in}{1.842936in}}{\pgfqpoint{2.886303in}{1.839664in}}{\pgfqpoint{2.894539in}{1.839664in}}%
\pgfpathclose%
\pgfusepath{stroke,fill}%
\end{pgfscope}%
\begin{pgfscope}%
\pgfpathrectangle{\pgfqpoint{0.100000in}{0.212622in}}{\pgfqpoint{3.696000in}{3.696000in}}%
\pgfusepath{clip}%
\pgfsetbuttcap%
\pgfsetroundjoin%
\definecolor{currentfill}{rgb}{0.121569,0.466667,0.705882}%
\pgfsetfillcolor{currentfill}%
\pgfsetfillopacity{0.465289}%
\pgfsetlinewidth{1.003750pt}%
\definecolor{currentstroke}{rgb}{0.121569,0.466667,0.705882}%
\pgfsetstrokecolor{currentstroke}%
\pgfsetstrokeopacity{0.465289}%
\pgfsetdash{}{0pt}%
\pgfpathmoveto{\pgfqpoint{2.896158in}{1.839345in}}%
\pgfpathcurveto{\pgfqpoint{2.904395in}{1.839345in}}{\pgfqpoint{2.912295in}{1.842618in}}{\pgfqpoint{2.918119in}{1.848442in}}%
\pgfpathcurveto{\pgfqpoint{2.923943in}{1.854266in}}{\pgfqpoint{2.927215in}{1.862166in}}{\pgfqpoint{2.927215in}{1.870402in}}%
\pgfpathcurveto{\pgfqpoint{2.927215in}{1.878638in}}{\pgfqpoint{2.923943in}{1.886538in}}{\pgfqpoint{2.918119in}{1.892362in}}%
\pgfpathcurveto{\pgfqpoint{2.912295in}{1.898186in}}{\pgfqpoint{2.904395in}{1.901458in}}{\pgfqpoint{2.896158in}{1.901458in}}%
\pgfpathcurveto{\pgfqpoint{2.887922in}{1.901458in}}{\pgfqpoint{2.880022in}{1.898186in}}{\pgfqpoint{2.874198in}{1.892362in}}%
\pgfpathcurveto{\pgfqpoint{2.868374in}{1.886538in}}{\pgfqpoint{2.865102in}{1.878638in}}{\pgfqpoint{2.865102in}{1.870402in}}%
\pgfpathcurveto{\pgfqpoint{2.865102in}{1.862166in}}{\pgfqpoint{2.868374in}{1.854266in}}{\pgfqpoint{2.874198in}{1.848442in}}%
\pgfpathcurveto{\pgfqpoint{2.880022in}{1.842618in}}{\pgfqpoint{2.887922in}{1.839345in}}{\pgfqpoint{2.896158in}{1.839345in}}%
\pgfpathclose%
\pgfusepath{stroke,fill}%
\end{pgfscope}%
\begin{pgfscope}%
\pgfpathrectangle{\pgfqpoint{0.100000in}{0.212622in}}{\pgfqpoint{3.696000in}{3.696000in}}%
\pgfusepath{clip}%
\pgfsetbuttcap%
\pgfsetroundjoin%
\definecolor{currentfill}{rgb}{0.121569,0.466667,0.705882}%
\pgfsetfillcolor{currentfill}%
\pgfsetfillopacity{0.465554}%
\pgfsetlinewidth{1.003750pt}%
\definecolor{currentstroke}{rgb}{0.121569,0.466667,0.705882}%
\pgfsetstrokecolor{currentstroke}%
\pgfsetstrokeopacity{0.465554}%
\pgfsetdash{}{0pt}%
\pgfpathmoveto{\pgfqpoint{1.376257in}{2.116885in}}%
\pgfpathcurveto{\pgfqpoint{1.384493in}{2.116885in}}{\pgfqpoint{1.392393in}{2.120157in}}{\pgfqpoint{1.398217in}{2.125981in}}%
\pgfpathcurveto{\pgfqpoint{1.404041in}{2.131805in}}{\pgfqpoint{1.407313in}{2.139705in}}{\pgfqpoint{1.407313in}{2.147942in}}%
\pgfpathcurveto{\pgfqpoint{1.407313in}{2.156178in}}{\pgfqpoint{1.404041in}{2.164078in}}{\pgfqpoint{1.398217in}{2.169902in}}%
\pgfpathcurveto{\pgfqpoint{1.392393in}{2.175726in}}{\pgfqpoint{1.384493in}{2.178998in}}{\pgfqpoint{1.376257in}{2.178998in}}%
\pgfpathcurveto{\pgfqpoint{1.368020in}{2.178998in}}{\pgfqpoint{1.360120in}{2.175726in}}{\pgfqpoint{1.354296in}{2.169902in}}%
\pgfpathcurveto{\pgfqpoint{1.348472in}{2.164078in}}{\pgfqpoint{1.345200in}{2.156178in}}{\pgfqpoint{1.345200in}{2.147942in}}%
\pgfpathcurveto{\pgfqpoint{1.345200in}{2.139705in}}{\pgfqpoint{1.348472in}{2.131805in}}{\pgfqpoint{1.354296in}{2.125981in}}%
\pgfpathcurveto{\pgfqpoint{1.360120in}{2.120157in}}{\pgfqpoint{1.368020in}{2.116885in}}{\pgfqpoint{1.376257in}{2.116885in}}%
\pgfpathclose%
\pgfusepath{stroke,fill}%
\end{pgfscope}%
\begin{pgfscope}%
\pgfpathrectangle{\pgfqpoint{0.100000in}{0.212622in}}{\pgfqpoint{3.696000in}{3.696000in}}%
\pgfusepath{clip}%
\pgfsetbuttcap%
\pgfsetroundjoin%
\definecolor{currentfill}{rgb}{0.121569,0.466667,0.705882}%
\pgfsetfillcolor{currentfill}%
\pgfsetfillopacity{0.465712}%
\pgfsetlinewidth{1.003750pt}%
\definecolor{currentstroke}{rgb}{0.121569,0.466667,0.705882}%
\pgfsetstrokecolor{currentstroke}%
\pgfsetstrokeopacity{0.465712}%
\pgfsetdash{}{0pt}%
\pgfpathmoveto{\pgfqpoint{2.898903in}{1.838902in}}%
\pgfpathcurveto{\pgfqpoint{2.907139in}{1.838902in}}{\pgfqpoint{2.915039in}{1.842175in}}{\pgfqpoint{2.920863in}{1.847999in}}%
\pgfpathcurveto{\pgfqpoint{2.926687in}{1.853823in}}{\pgfqpoint{2.929960in}{1.861723in}}{\pgfqpoint{2.929960in}{1.869959in}}%
\pgfpathcurveto{\pgfqpoint{2.929960in}{1.878195in}}{\pgfqpoint{2.926687in}{1.886095in}}{\pgfqpoint{2.920863in}{1.891919in}}%
\pgfpathcurveto{\pgfqpoint{2.915039in}{1.897743in}}{\pgfqpoint{2.907139in}{1.901015in}}{\pgfqpoint{2.898903in}{1.901015in}}%
\pgfpathcurveto{\pgfqpoint{2.890667in}{1.901015in}}{\pgfqpoint{2.882767in}{1.897743in}}{\pgfqpoint{2.876943in}{1.891919in}}%
\pgfpathcurveto{\pgfqpoint{2.871119in}{1.886095in}}{\pgfqpoint{2.867847in}{1.878195in}}{\pgfqpoint{2.867847in}{1.869959in}}%
\pgfpathcurveto{\pgfqpoint{2.867847in}{1.861723in}}{\pgfqpoint{2.871119in}{1.853823in}}{\pgfqpoint{2.876943in}{1.847999in}}%
\pgfpathcurveto{\pgfqpoint{2.882767in}{1.842175in}}{\pgfqpoint{2.890667in}{1.838902in}}{\pgfqpoint{2.898903in}{1.838902in}}%
\pgfpathclose%
\pgfusepath{stroke,fill}%
\end{pgfscope}%
\begin{pgfscope}%
\pgfpathrectangle{\pgfqpoint{0.100000in}{0.212622in}}{\pgfqpoint{3.696000in}{3.696000in}}%
\pgfusepath{clip}%
\pgfsetbuttcap%
\pgfsetroundjoin%
\definecolor{currentfill}{rgb}{0.121569,0.466667,0.705882}%
\pgfsetfillcolor{currentfill}%
\pgfsetfillopacity{0.466104}%
\pgfsetlinewidth{1.003750pt}%
\definecolor{currentstroke}{rgb}{0.121569,0.466667,0.705882}%
\pgfsetstrokecolor{currentstroke}%
\pgfsetstrokeopacity{0.466104}%
\pgfsetdash{}{0pt}%
\pgfpathmoveto{\pgfqpoint{2.902163in}{1.838385in}}%
\pgfpathcurveto{\pgfqpoint{2.910399in}{1.838385in}}{\pgfqpoint{2.918299in}{1.841658in}}{\pgfqpoint{2.924123in}{1.847481in}}%
\pgfpathcurveto{\pgfqpoint{2.929947in}{1.853305in}}{\pgfqpoint{2.933219in}{1.861205in}}{\pgfqpoint{2.933219in}{1.869442in}}%
\pgfpathcurveto{\pgfqpoint{2.933219in}{1.877678in}}{\pgfqpoint{2.929947in}{1.885578in}}{\pgfqpoint{2.924123in}{1.891402in}}%
\pgfpathcurveto{\pgfqpoint{2.918299in}{1.897226in}}{\pgfqpoint{2.910399in}{1.900498in}}{\pgfqpoint{2.902163in}{1.900498in}}%
\pgfpathcurveto{\pgfqpoint{2.893926in}{1.900498in}}{\pgfqpoint{2.886026in}{1.897226in}}{\pgfqpoint{2.880202in}{1.891402in}}%
\pgfpathcurveto{\pgfqpoint{2.874379in}{1.885578in}}{\pgfqpoint{2.871106in}{1.877678in}}{\pgfqpoint{2.871106in}{1.869442in}}%
\pgfpathcurveto{\pgfqpoint{2.871106in}{1.861205in}}{\pgfqpoint{2.874379in}{1.853305in}}{\pgfqpoint{2.880202in}{1.847481in}}%
\pgfpathcurveto{\pgfqpoint{2.886026in}{1.841658in}}{\pgfqpoint{2.893926in}{1.838385in}}{\pgfqpoint{2.902163in}{1.838385in}}%
\pgfpathclose%
\pgfusepath{stroke,fill}%
\end{pgfscope}%
\begin{pgfscope}%
\pgfpathrectangle{\pgfqpoint{0.100000in}{0.212622in}}{\pgfqpoint{3.696000in}{3.696000in}}%
\pgfusepath{clip}%
\pgfsetbuttcap%
\pgfsetroundjoin%
\definecolor{currentfill}{rgb}{0.121569,0.466667,0.705882}%
\pgfsetfillcolor{currentfill}%
\pgfsetfillopacity{0.466246}%
\pgfsetlinewidth{1.003750pt}%
\definecolor{currentstroke}{rgb}{0.121569,0.466667,0.705882}%
\pgfsetstrokecolor{currentstroke}%
\pgfsetstrokeopacity{0.466246}%
\pgfsetdash{}{0pt}%
\pgfpathmoveto{\pgfqpoint{2.904048in}{1.837987in}}%
\pgfpathcurveto{\pgfqpoint{2.912284in}{1.837987in}}{\pgfqpoint{2.920185in}{1.841259in}}{\pgfqpoint{2.926008in}{1.847083in}}%
\pgfpathcurveto{\pgfqpoint{2.931832in}{1.852907in}}{\pgfqpoint{2.935105in}{1.860807in}}{\pgfqpoint{2.935105in}{1.869044in}}%
\pgfpathcurveto{\pgfqpoint{2.935105in}{1.877280in}}{\pgfqpoint{2.931832in}{1.885180in}}{\pgfqpoint{2.926008in}{1.891004in}}%
\pgfpathcurveto{\pgfqpoint{2.920185in}{1.896828in}}{\pgfqpoint{2.912284in}{1.900100in}}{\pgfqpoint{2.904048in}{1.900100in}}%
\pgfpathcurveto{\pgfqpoint{2.895812in}{1.900100in}}{\pgfqpoint{2.887912in}{1.896828in}}{\pgfqpoint{2.882088in}{1.891004in}}%
\pgfpathcurveto{\pgfqpoint{2.876264in}{1.885180in}}{\pgfqpoint{2.872992in}{1.877280in}}{\pgfqpoint{2.872992in}{1.869044in}}%
\pgfpathcurveto{\pgfqpoint{2.872992in}{1.860807in}}{\pgfqpoint{2.876264in}{1.852907in}}{\pgfqpoint{2.882088in}{1.847083in}}%
\pgfpathcurveto{\pgfqpoint{2.887912in}{1.841259in}}{\pgfqpoint{2.895812in}{1.837987in}}{\pgfqpoint{2.904048in}{1.837987in}}%
\pgfpathclose%
\pgfusepath{stroke,fill}%
\end{pgfscope}%
\begin{pgfscope}%
\pgfpathrectangle{\pgfqpoint{0.100000in}{0.212622in}}{\pgfqpoint{3.696000in}{3.696000in}}%
\pgfusepath{clip}%
\pgfsetbuttcap%
\pgfsetroundjoin%
\definecolor{currentfill}{rgb}{0.121569,0.466667,0.705882}%
\pgfsetfillcolor{currentfill}%
\pgfsetfillopacity{0.466533}%
\pgfsetlinewidth{1.003750pt}%
\definecolor{currentstroke}{rgb}{0.121569,0.466667,0.705882}%
\pgfsetstrokecolor{currentstroke}%
\pgfsetstrokeopacity{0.466533}%
\pgfsetdash{}{0pt}%
\pgfpathmoveto{\pgfqpoint{2.906168in}{1.837660in}}%
\pgfpathcurveto{\pgfqpoint{2.914404in}{1.837660in}}{\pgfqpoint{2.922304in}{1.840932in}}{\pgfqpoint{2.928128in}{1.846756in}}%
\pgfpathcurveto{\pgfqpoint{2.933952in}{1.852580in}}{\pgfqpoint{2.937224in}{1.860480in}}{\pgfqpoint{2.937224in}{1.868716in}}%
\pgfpathcurveto{\pgfqpoint{2.937224in}{1.876953in}}{\pgfqpoint{2.933952in}{1.884853in}}{\pgfqpoint{2.928128in}{1.890677in}}%
\pgfpathcurveto{\pgfqpoint{2.922304in}{1.896501in}}{\pgfqpoint{2.914404in}{1.899773in}}{\pgfqpoint{2.906168in}{1.899773in}}%
\pgfpathcurveto{\pgfqpoint{2.897931in}{1.899773in}}{\pgfqpoint{2.890031in}{1.896501in}}{\pgfqpoint{2.884207in}{1.890677in}}%
\pgfpathcurveto{\pgfqpoint{2.878383in}{1.884853in}}{\pgfqpoint{2.875111in}{1.876953in}}{\pgfqpoint{2.875111in}{1.868716in}}%
\pgfpathcurveto{\pgfqpoint{2.875111in}{1.860480in}}{\pgfqpoint{2.878383in}{1.852580in}}{\pgfqpoint{2.884207in}{1.846756in}}%
\pgfpathcurveto{\pgfqpoint{2.890031in}{1.840932in}}{\pgfqpoint{2.897931in}{1.837660in}}{\pgfqpoint{2.906168in}{1.837660in}}%
\pgfpathclose%
\pgfusepath{stroke,fill}%
\end{pgfscope}%
\begin{pgfscope}%
\pgfpathrectangle{\pgfqpoint{0.100000in}{0.212622in}}{\pgfqpoint{3.696000in}{3.696000in}}%
\pgfusepath{clip}%
\pgfsetbuttcap%
\pgfsetroundjoin%
\definecolor{currentfill}{rgb}{0.121569,0.466667,0.705882}%
\pgfsetfillcolor{currentfill}%
\pgfsetfillopacity{0.466767}%
\pgfsetlinewidth{1.003750pt}%
\definecolor{currentstroke}{rgb}{0.121569,0.466667,0.705882}%
\pgfsetstrokecolor{currentstroke}%
\pgfsetstrokeopacity{0.466767}%
\pgfsetdash{}{0pt}%
\pgfpathmoveto{\pgfqpoint{2.908860in}{1.837144in}}%
\pgfpathcurveto{\pgfqpoint{2.917096in}{1.837144in}}{\pgfqpoint{2.924996in}{1.840416in}}{\pgfqpoint{2.930820in}{1.846240in}}%
\pgfpathcurveto{\pgfqpoint{2.936644in}{1.852064in}}{\pgfqpoint{2.939917in}{1.859964in}}{\pgfqpoint{2.939917in}{1.868200in}}%
\pgfpathcurveto{\pgfqpoint{2.939917in}{1.876437in}}{\pgfqpoint{2.936644in}{1.884337in}}{\pgfqpoint{2.930820in}{1.890161in}}%
\pgfpathcurveto{\pgfqpoint{2.924996in}{1.895984in}}{\pgfqpoint{2.917096in}{1.899257in}}{\pgfqpoint{2.908860in}{1.899257in}}%
\pgfpathcurveto{\pgfqpoint{2.900624in}{1.899257in}}{\pgfqpoint{2.892724in}{1.895984in}}{\pgfqpoint{2.886900in}{1.890161in}}%
\pgfpathcurveto{\pgfqpoint{2.881076in}{1.884337in}}{\pgfqpoint{2.877804in}{1.876437in}}{\pgfqpoint{2.877804in}{1.868200in}}%
\pgfpathcurveto{\pgfqpoint{2.877804in}{1.859964in}}{\pgfqpoint{2.881076in}{1.852064in}}{\pgfqpoint{2.886900in}{1.846240in}}%
\pgfpathcurveto{\pgfqpoint{2.892724in}{1.840416in}}{\pgfqpoint{2.900624in}{1.837144in}}{\pgfqpoint{2.908860in}{1.837144in}}%
\pgfpathclose%
\pgfusepath{stroke,fill}%
\end{pgfscope}%
\begin{pgfscope}%
\pgfpathrectangle{\pgfqpoint{0.100000in}{0.212622in}}{\pgfqpoint{3.696000in}{3.696000in}}%
\pgfusepath{clip}%
\pgfsetbuttcap%
\pgfsetroundjoin%
\definecolor{currentfill}{rgb}{0.121569,0.466667,0.705882}%
\pgfsetfillcolor{currentfill}%
\pgfsetfillopacity{0.467241}%
\pgfsetlinewidth{1.003750pt}%
\definecolor{currentstroke}{rgb}{0.121569,0.466667,0.705882}%
\pgfsetstrokecolor{currentstroke}%
\pgfsetstrokeopacity{0.467241}%
\pgfsetdash{}{0pt}%
\pgfpathmoveto{\pgfqpoint{2.912071in}{1.836487in}}%
\pgfpathcurveto{\pgfqpoint{2.920307in}{1.836487in}}{\pgfqpoint{2.928207in}{1.839759in}}{\pgfqpoint{2.934031in}{1.845583in}}%
\pgfpathcurveto{\pgfqpoint{2.939855in}{1.851407in}}{\pgfqpoint{2.943127in}{1.859307in}}{\pgfqpoint{2.943127in}{1.867543in}}%
\pgfpathcurveto{\pgfqpoint{2.943127in}{1.875780in}}{\pgfqpoint{2.939855in}{1.883680in}}{\pgfqpoint{2.934031in}{1.889504in}}%
\pgfpathcurveto{\pgfqpoint{2.928207in}{1.895328in}}{\pgfqpoint{2.920307in}{1.898600in}}{\pgfqpoint{2.912071in}{1.898600in}}%
\pgfpathcurveto{\pgfqpoint{2.903835in}{1.898600in}}{\pgfqpoint{2.895935in}{1.895328in}}{\pgfqpoint{2.890111in}{1.889504in}}%
\pgfpathcurveto{\pgfqpoint{2.884287in}{1.883680in}}{\pgfqpoint{2.881014in}{1.875780in}}{\pgfqpoint{2.881014in}{1.867543in}}%
\pgfpathcurveto{\pgfqpoint{2.881014in}{1.859307in}}{\pgfqpoint{2.884287in}{1.851407in}}{\pgfqpoint{2.890111in}{1.845583in}}%
\pgfpathcurveto{\pgfqpoint{2.895935in}{1.839759in}}{\pgfqpoint{2.903835in}{1.836487in}}{\pgfqpoint{2.912071in}{1.836487in}}%
\pgfpathclose%
\pgfusepath{stroke,fill}%
\end{pgfscope}%
\begin{pgfscope}%
\pgfpathrectangle{\pgfqpoint{0.100000in}{0.212622in}}{\pgfqpoint{3.696000in}{3.696000in}}%
\pgfusepath{clip}%
\pgfsetbuttcap%
\pgfsetroundjoin%
\definecolor{currentfill}{rgb}{0.121569,0.466667,0.705882}%
\pgfsetfillcolor{currentfill}%
\pgfsetfillopacity{0.467429}%
\pgfsetlinewidth{1.003750pt}%
\definecolor{currentstroke}{rgb}{0.121569,0.466667,0.705882}%
\pgfsetstrokecolor{currentstroke}%
\pgfsetstrokeopacity{0.467429}%
\pgfsetdash{}{0pt}%
\pgfpathmoveto{\pgfqpoint{1.372702in}{2.117155in}}%
\pgfpathcurveto{\pgfqpoint{1.380938in}{2.117155in}}{\pgfqpoint{1.388838in}{2.120427in}}{\pgfqpoint{1.394662in}{2.126251in}}%
\pgfpathcurveto{\pgfqpoint{1.400486in}{2.132075in}}{\pgfqpoint{1.403758in}{2.139975in}}{\pgfqpoint{1.403758in}{2.148212in}}%
\pgfpathcurveto{\pgfqpoint{1.403758in}{2.156448in}}{\pgfqpoint{1.400486in}{2.164348in}}{\pgfqpoint{1.394662in}{2.170172in}}%
\pgfpathcurveto{\pgfqpoint{1.388838in}{2.175996in}}{\pgfqpoint{1.380938in}{2.179268in}}{\pgfqpoint{1.372702in}{2.179268in}}%
\pgfpathcurveto{\pgfqpoint{1.364465in}{2.179268in}}{\pgfqpoint{1.356565in}{2.175996in}}{\pgfqpoint{1.350742in}{2.170172in}}%
\pgfpathcurveto{\pgfqpoint{1.344918in}{2.164348in}}{\pgfqpoint{1.341645in}{2.156448in}}{\pgfqpoint{1.341645in}{2.148212in}}%
\pgfpathcurveto{\pgfqpoint{1.341645in}{2.139975in}}{\pgfqpoint{1.344918in}{2.132075in}}{\pgfqpoint{1.350742in}{2.126251in}}%
\pgfpathcurveto{\pgfqpoint{1.356565in}{2.120427in}}{\pgfqpoint{1.364465in}{2.117155in}}{\pgfqpoint{1.372702in}{2.117155in}}%
\pgfpathclose%
\pgfusepath{stroke,fill}%
\end{pgfscope}%
\begin{pgfscope}%
\pgfpathrectangle{\pgfqpoint{0.100000in}{0.212622in}}{\pgfqpoint{3.696000in}{3.696000in}}%
\pgfusepath{clip}%
\pgfsetbuttcap%
\pgfsetroundjoin%
\definecolor{currentfill}{rgb}{0.121569,0.466667,0.705882}%
\pgfsetfillcolor{currentfill}%
\pgfsetfillopacity{0.467589}%
\pgfsetlinewidth{1.003750pt}%
\definecolor{currentstroke}{rgb}{0.121569,0.466667,0.705882}%
\pgfsetstrokecolor{currentstroke}%
\pgfsetstrokeopacity{0.467589}%
\pgfsetdash{}{0pt}%
\pgfpathmoveto{\pgfqpoint{2.916338in}{1.835564in}}%
\pgfpathcurveto{\pgfqpoint{2.924574in}{1.835564in}}{\pgfqpoint{2.932475in}{1.838837in}}{\pgfqpoint{2.938298in}{1.844661in}}%
\pgfpathcurveto{\pgfqpoint{2.944122in}{1.850485in}}{\pgfqpoint{2.947395in}{1.858385in}}{\pgfqpoint{2.947395in}{1.866621in}}%
\pgfpathcurveto{\pgfqpoint{2.947395in}{1.874857in}}{\pgfqpoint{2.944122in}{1.882757in}}{\pgfqpoint{2.938298in}{1.888581in}}%
\pgfpathcurveto{\pgfqpoint{2.932475in}{1.894405in}}{\pgfqpoint{2.924574in}{1.897677in}}{\pgfqpoint{2.916338in}{1.897677in}}%
\pgfpathcurveto{\pgfqpoint{2.908102in}{1.897677in}}{\pgfqpoint{2.900202in}{1.894405in}}{\pgfqpoint{2.894378in}{1.888581in}}%
\pgfpathcurveto{\pgfqpoint{2.888554in}{1.882757in}}{\pgfqpoint{2.885282in}{1.874857in}}{\pgfqpoint{2.885282in}{1.866621in}}%
\pgfpathcurveto{\pgfqpoint{2.885282in}{1.858385in}}{\pgfqpoint{2.888554in}{1.850485in}}{\pgfqpoint{2.894378in}{1.844661in}}%
\pgfpathcurveto{\pgfqpoint{2.900202in}{1.838837in}}{\pgfqpoint{2.908102in}{1.835564in}}{\pgfqpoint{2.916338in}{1.835564in}}%
\pgfpathclose%
\pgfusepath{stroke,fill}%
\end{pgfscope}%
\begin{pgfscope}%
\pgfpathrectangle{\pgfqpoint{0.100000in}{0.212622in}}{\pgfqpoint{3.696000in}{3.696000in}}%
\pgfusepath{clip}%
\pgfsetbuttcap%
\pgfsetroundjoin%
\definecolor{currentfill}{rgb}{0.121569,0.466667,0.705882}%
\pgfsetfillcolor{currentfill}%
\pgfsetfillopacity{0.468034}%
\pgfsetlinewidth{1.003750pt}%
\definecolor{currentstroke}{rgb}{0.121569,0.466667,0.705882}%
\pgfsetstrokecolor{currentstroke}%
\pgfsetstrokeopacity{0.468034}%
\pgfsetdash{}{0pt}%
\pgfpathmoveto{\pgfqpoint{2.922215in}{1.834232in}}%
\pgfpathcurveto{\pgfqpoint{2.930452in}{1.834232in}}{\pgfqpoint{2.938352in}{1.837504in}}{\pgfqpoint{2.944176in}{1.843328in}}%
\pgfpathcurveto{\pgfqpoint{2.950000in}{1.849152in}}{\pgfqpoint{2.953272in}{1.857052in}}{\pgfqpoint{2.953272in}{1.865288in}}%
\pgfpathcurveto{\pgfqpoint{2.953272in}{1.873525in}}{\pgfqpoint{2.950000in}{1.881425in}}{\pgfqpoint{2.944176in}{1.887249in}}%
\pgfpathcurveto{\pgfqpoint{2.938352in}{1.893072in}}{\pgfqpoint{2.930452in}{1.896345in}}{\pgfqpoint{2.922215in}{1.896345in}}%
\pgfpathcurveto{\pgfqpoint{2.913979in}{1.896345in}}{\pgfqpoint{2.906079in}{1.893072in}}{\pgfqpoint{2.900255in}{1.887249in}}%
\pgfpathcurveto{\pgfqpoint{2.894431in}{1.881425in}}{\pgfqpoint{2.891159in}{1.873525in}}{\pgfqpoint{2.891159in}{1.865288in}}%
\pgfpathcurveto{\pgfqpoint{2.891159in}{1.857052in}}{\pgfqpoint{2.894431in}{1.849152in}}{\pgfqpoint{2.900255in}{1.843328in}}%
\pgfpathcurveto{\pgfqpoint{2.906079in}{1.837504in}}{\pgfqpoint{2.913979in}{1.834232in}}{\pgfqpoint{2.922215in}{1.834232in}}%
\pgfpathclose%
\pgfusepath{stroke,fill}%
\end{pgfscope}%
\begin{pgfscope}%
\pgfpathrectangle{\pgfqpoint{0.100000in}{0.212622in}}{\pgfqpoint{3.696000in}{3.696000in}}%
\pgfusepath{clip}%
\pgfsetbuttcap%
\pgfsetroundjoin%
\definecolor{currentfill}{rgb}{0.121569,0.466667,0.705882}%
\pgfsetfillcolor{currentfill}%
\pgfsetfillopacity{0.468470}%
\pgfsetlinewidth{1.003750pt}%
\definecolor{currentstroke}{rgb}{0.121569,0.466667,0.705882}%
\pgfsetstrokecolor{currentstroke}%
\pgfsetstrokeopacity{0.468470}%
\pgfsetdash{}{0pt}%
\pgfpathmoveto{\pgfqpoint{2.928450in}{1.832689in}}%
\pgfpathcurveto{\pgfqpoint{2.936687in}{1.832689in}}{\pgfqpoint{2.944587in}{1.835961in}}{\pgfqpoint{2.950411in}{1.841785in}}%
\pgfpathcurveto{\pgfqpoint{2.956234in}{1.847609in}}{\pgfqpoint{2.959507in}{1.855509in}}{\pgfqpoint{2.959507in}{1.863745in}}%
\pgfpathcurveto{\pgfqpoint{2.959507in}{1.871982in}}{\pgfqpoint{2.956234in}{1.879882in}}{\pgfqpoint{2.950411in}{1.885706in}}%
\pgfpathcurveto{\pgfqpoint{2.944587in}{1.891530in}}{\pgfqpoint{2.936687in}{1.894802in}}{\pgfqpoint{2.928450in}{1.894802in}}%
\pgfpathcurveto{\pgfqpoint{2.920214in}{1.894802in}}{\pgfqpoint{2.912314in}{1.891530in}}{\pgfqpoint{2.906490in}{1.885706in}}%
\pgfpathcurveto{\pgfqpoint{2.900666in}{1.879882in}}{\pgfqpoint{2.897394in}{1.871982in}}{\pgfqpoint{2.897394in}{1.863745in}}%
\pgfpathcurveto{\pgfqpoint{2.897394in}{1.855509in}}{\pgfqpoint{2.900666in}{1.847609in}}{\pgfqpoint{2.906490in}{1.841785in}}%
\pgfpathcurveto{\pgfqpoint{2.912314in}{1.835961in}}{\pgfqpoint{2.920214in}{1.832689in}}{\pgfqpoint{2.928450in}{1.832689in}}%
\pgfpathclose%
\pgfusepath{stroke,fill}%
\end{pgfscope}%
\begin{pgfscope}%
\pgfpathrectangle{\pgfqpoint{0.100000in}{0.212622in}}{\pgfqpoint{3.696000in}{3.696000in}}%
\pgfusepath{clip}%
\pgfsetbuttcap%
\pgfsetroundjoin%
\definecolor{currentfill}{rgb}{0.121569,0.466667,0.705882}%
\pgfsetfillcolor{currentfill}%
\pgfsetfillopacity{0.469086}%
\pgfsetlinewidth{1.003750pt}%
\definecolor{currentstroke}{rgb}{0.121569,0.466667,0.705882}%
\pgfsetstrokecolor{currentstroke}%
\pgfsetstrokeopacity{0.469086}%
\pgfsetdash{}{0pt}%
\pgfpathmoveto{\pgfqpoint{1.369794in}{2.117127in}}%
\pgfpathcurveto{\pgfqpoint{1.378030in}{2.117127in}}{\pgfqpoint{1.385930in}{2.120399in}}{\pgfqpoint{1.391754in}{2.126223in}}%
\pgfpathcurveto{\pgfqpoint{1.397578in}{2.132047in}}{\pgfqpoint{1.400851in}{2.139947in}}{\pgfqpoint{1.400851in}{2.148183in}}%
\pgfpathcurveto{\pgfqpoint{1.400851in}{2.156419in}}{\pgfqpoint{1.397578in}{2.164319in}}{\pgfqpoint{1.391754in}{2.170143in}}%
\pgfpathcurveto{\pgfqpoint{1.385930in}{2.175967in}}{\pgfqpoint{1.378030in}{2.179240in}}{\pgfqpoint{1.369794in}{2.179240in}}%
\pgfpathcurveto{\pgfqpoint{1.361558in}{2.179240in}}{\pgfqpoint{1.353658in}{2.175967in}}{\pgfqpoint{1.347834in}{2.170143in}}%
\pgfpathcurveto{\pgfqpoint{1.342010in}{2.164319in}}{\pgfqpoint{1.338738in}{2.156419in}}{\pgfqpoint{1.338738in}{2.148183in}}%
\pgfpathcurveto{\pgfqpoint{1.338738in}{2.139947in}}{\pgfqpoint{1.342010in}{2.132047in}}{\pgfqpoint{1.347834in}{2.126223in}}%
\pgfpathcurveto{\pgfqpoint{1.353658in}{2.120399in}}{\pgfqpoint{1.361558in}{2.117127in}}{\pgfqpoint{1.369794in}{2.117127in}}%
\pgfpathclose%
\pgfusepath{stroke,fill}%
\end{pgfscope}%
\begin{pgfscope}%
\pgfpathrectangle{\pgfqpoint{0.100000in}{0.212622in}}{\pgfqpoint{3.696000in}{3.696000in}}%
\pgfusepath{clip}%
\pgfsetbuttcap%
\pgfsetroundjoin%
\definecolor{currentfill}{rgb}{0.121569,0.466667,0.705882}%
\pgfsetfillcolor{currentfill}%
\pgfsetfillopacity{0.469104}%
\pgfsetlinewidth{1.003750pt}%
\definecolor{currentstroke}{rgb}{0.121569,0.466667,0.705882}%
\pgfsetstrokecolor{currentstroke}%
\pgfsetstrokeopacity{0.469104}%
\pgfsetdash{}{0pt}%
\pgfpathmoveto{\pgfqpoint{2.934935in}{1.831446in}}%
\pgfpathcurveto{\pgfqpoint{2.943171in}{1.831446in}}{\pgfqpoint{2.951071in}{1.834718in}}{\pgfqpoint{2.956895in}{1.840542in}}%
\pgfpathcurveto{\pgfqpoint{2.962719in}{1.846366in}}{\pgfqpoint{2.965992in}{1.854266in}}{\pgfqpoint{2.965992in}{1.862502in}}%
\pgfpathcurveto{\pgfqpoint{2.965992in}{1.870739in}}{\pgfqpoint{2.962719in}{1.878639in}}{\pgfqpoint{2.956895in}{1.884463in}}%
\pgfpathcurveto{\pgfqpoint{2.951071in}{1.890287in}}{\pgfqpoint{2.943171in}{1.893559in}}{\pgfqpoint{2.934935in}{1.893559in}}%
\pgfpathcurveto{\pgfqpoint{2.926699in}{1.893559in}}{\pgfqpoint{2.918799in}{1.890287in}}{\pgfqpoint{2.912975in}{1.884463in}}%
\pgfpathcurveto{\pgfqpoint{2.907151in}{1.878639in}}{\pgfqpoint{2.903879in}{1.870739in}}{\pgfqpoint{2.903879in}{1.862502in}}%
\pgfpathcurveto{\pgfqpoint{2.903879in}{1.854266in}}{\pgfqpoint{2.907151in}{1.846366in}}{\pgfqpoint{2.912975in}{1.840542in}}%
\pgfpathcurveto{\pgfqpoint{2.918799in}{1.834718in}}{\pgfqpoint{2.926699in}{1.831446in}}{\pgfqpoint{2.934935in}{1.831446in}}%
\pgfpathclose%
\pgfusepath{stroke,fill}%
\end{pgfscope}%
\begin{pgfscope}%
\pgfpathrectangle{\pgfqpoint{0.100000in}{0.212622in}}{\pgfqpoint{3.696000in}{3.696000in}}%
\pgfusepath{clip}%
\pgfsetbuttcap%
\pgfsetroundjoin%
\definecolor{currentfill}{rgb}{0.121569,0.466667,0.705882}%
\pgfsetfillcolor{currentfill}%
\pgfsetfillopacity{0.469595}%
\pgfsetlinewidth{1.003750pt}%
\definecolor{currentstroke}{rgb}{0.121569,0.466667,0.705882}%
\pgfsetstrokecolor{currentstroke}%
\pgfsetstrokeopacity{0.469595}%
\pgfsetdash{}{0pt}%
\pgfpathmoveto{\pgfqpoint{2.942203in}{1.829702in}}%
\pgfpathcurveto{\pgfqpoint{2.950439in}{1.829702in}}{\pgfqpoint{2.958339in}{1.832974in}}{\pgfqpoint{2.964163in}{1.838798in}}%
\pgfpathcurveto{\pgfqpoint{2.969987in}{1.844622in}}{\pgfqpoint{2.973259in}{1.852522in}}{\pgfqpoint{2.973259in}{1.860758in}}%
\pgfpathcurveto{\pgfqpoint{2.973259in}{1.868994in}}{\pgfqpoint{2.969987in}{1.876894in}}{\pgfqpoint{2.964163in}{1.882718in}}%
\pgfpathcurveto{\pgfqpoint{2.958339in}{1.888542in}}{\pgfqpoint{2.950439in}{1.891815in}}{\pgfqpoint{2.942203in}{1.891815in}}%
\pgfpathcurveto{\pgfqpoint{2.933966in}{1.891815in}}{\pgfqpoint{2.926066in}{1.888542in}}{\pgfqpoint{2.920242in}{1.882718in}}%
\pgfpathcurveto{\pgfqpoint{2.914418in}{1.876894in}}{\pgfqpoint{2.911146in}{1.868994in}}{\pgfqpoint{2.911146in}{1.860758in}}%
\pgfpathcurveto{\pgfqpoint{2.911146in}{1.852522in}}{\pgfqpoint{2.914418in}{1.844622in}}{\pgfqpoint{2.920242in}{1.838798in}}%
\pgfpathcurveto{\pgfqpoint{2.926066in}{1.832974in}}{\pgfqpoint{2.933966in}{1.829702in}}{\pgfqpoint{2.942203in}{1.829702in}}%
\pgfpathclose%
\pgfusepath{stroke,fill}%
\end{pgfscope}%
\begin{pgfscope}%
\pgfpathrectangle{\pgfqpoint{0.100000in}{0.212622in}}{\pgfqpoint{3.696000in}{3.696000in}}%
\pgfusepath{clip}%
\pgfsetbuttcap%
\pgfsetroundjoin%
\definecolor{currentfill}{rgb}{0.121569,0.466667,0.705882}%
\pgfsetfillcolor{currentfill}%
\pgfsetfillopacity{0.470129}%
\pgfsetlinewidth{1.003750pt}%
\definecolor{currentstroke}{rgb}{0.121569,0.466667,0.705882}%
\pgfsetstrokecolor{currentstroke}%
\pgfsetstrokeopacity{0.470129}%
\pgfsetdash{}{0pt}%
\pgfpathmoveto{\pgfqpoint{2.945816in}{1.829063in}}%
\pgfpathcurveto{\pgfqpoint{2.954053in}{1.829063in}}{\pgfqpoint{2.961953in}{1.832336in}}{\pgfqpoint{2.967777in}{1.838160in}}%
\pgfpathcurveto{\pgfqpoint{2.973601in}{1.843984in}}{\pgfqpoint{2.976873in}{1.851884in}}{\pgfqpoint{2.976873in}{1.860120in}}%
\pgfpathcurveto{\pgfqpoint{2.976873in}{1.868356in}}{\pgfqpoint{2.973601in}{1.876256in}}{\pgfqpoint{2.967777in}{1.882080in}}%
\pgfpathcurveto{\pgfqpoint{2.961953in}{1.887904in}}{\pgfqpoint{2.954053in}{1.891176in}}{\pgfqpoint{2.945816in}{1.891176in}}%
\pgfpathcurveto{\pgfqpoint{2.937580in}{1.891176in}}{\pgfqpoint{2.929680in}{1.887904in}}{\pgfqpoint{2.923856in}{1.882080in}}%
\pgfpathcurveto{\pgfqpoint{2.918032in}{1.876256in}}{\pgfqpoint{2.914760in}{1.868356in}}{\pgfqpoint{2.914760in}{1.860120in}}%
\pgfpathcurveto{\pgfqpoint{2.914760in}{1.851884in}}{\pgfqpoint{2.918032in}{1.843984in}}{\pgfqpoint{2.923856in}{1.838160in}}%
\pgfpathcurveto{\pgfqpoint{2.929680in}{1.832336in}}{\pgfqpoint{2.937580in}{1.829063in}}{\pgfqpoint{2.945816in}{1.829063in}}%
\pgfpathclose%
\pgfusepath{stroke,fill}%
\end{pgfscope}%
\begin{pgfscope}%
\pgfpathrectangle{\pgfqpoint{0.100000in}{0.212622in}}{\pgfqpoint{3.696000in}{3.696000in}}%
\pgfusepath{clip}%
\pgfsetbuttcap%
\pgfsetroundjoin%
\definecolor{currentfill}{rgb}{0.121569,0.466667,0.705882}%
\pgfsetfillcolor{currentfill}%
\pgfsetfillopacity{0.470428}%
\pgfsetlinewidth{1.003750pt}%
\definecolor{currentstroke}{rgb}{0.121569,0.466667,0.705882}%
\pgfsetstrokecolor{currentstroke}%
\pgfsetstrokeopacity{0.470428}%
\pgfsetdash{}{0pt}%
\pgfpathmoveto{\pgfqpoint{2.950948in}{1.828004in}}%
\pgfpathcurveto{\pgfqpoint{2.959184in}{1.828004in}}{\pgfqpoint{2.967084in}{1.831277in}}{\pgfqpoint{2.972908in}{1.837101in}}%
\pgfpathcurveto{\pgfqpoint{2.978732in}{1.842924in}}{\pgfqpoint{2.982004in}{1.850825in}}{\pgfqpoint{2.982004in}{1.859061in}}%
\pgfpathcurveto{\pgfqpoint{2.982004in}{1.867297in}}{\pgfqpoint{2.978732in}{1.875197in}}{\pgfqpoint{2.972908in}{1.881021in}}%
\pgfpathcurveto{\pgfqpoint{2.967084in}{1.886845in}}{\pgfqpoint{2.959184in}{1.890117in}}{\pgfqpoint{2.950948in}{1.890117in}}%
\pgfpathcurveto{\pgfqpoint{2.942711in}{1.890117in}}{\pgfqpoint{2.934811in}{1.886845in}}{\pgfqpoint{2.928987in}{1.881021in}}%
\pgfpathcurveto{\pgfqpoint{2.923163in}{1.875197in}}{\pgfqpoint{2.919891in}{1.867297in}}{\pgfqpoint{2.919891in}{1.859061in}}%
\pgfpathcurveto{\pgfqpoint{2.919891in}{1.850825in}}{\pgfqpoint{2.923163in}{1.842924in}}{\pgfqpoint{2.928987in}{1.837101in}}%
\pgfpathcurveto{\pgfqpoint{2.934811in}{1.831277in}}{\pgfqpoint{2.942711in}{1.828004in}}{\pgfqpoint{2.950948in}{1.828004in}}%
\pgfpathclose%
\pgfusepath{stroke,fill}%
\end{pgfscope}%
\begin{pgfscope}%
\pgfpathrectangle{\pgfqpoint{0.100000in}{0.212622in}}{\pgfqpoint{3.696000in}{3.696000in}}%
\pgfusepath{clip}%
\pgfsetbuttcap%
\pgfsetroundjoin%
\definecolor{currentfill}{rgb}{0.121569,0.466667,0.705882}%
\pgfsetfillcolor{currentfill}%
\pgfsetfillopacity{0.470505}%
\pgfsetlinewidth{1.003750pt}%
\definecolor{currentstroke}{rgb}{0.121569,0.466667,0.705882}%
\pgfsetstrokecolor{currentstroke}%
\pgfsetstrokeopacity{0.470505}%
\pgfsetdash{}{0pt}%
\pgfpathmoveto{\pgfqpoint{1.366164in}{2.117444in}}%
\pgfpathcurveto{\pgfqpoint{1.374400in}{2.117444in}}{\pgfqpoint{1.382301in}{2.120716in}}{\pgfqpoint{1.388124in}{2.126540in}}%
\pgfpathcurveto{\pgfqpoint{1.393948in}{2.132364in}}{\pgfqpoint{1.397221in}{2.140264in}}{\pgfqpoint{1.397221in}{2.148500in}}%
\pgfpathcurveto{\pgfqpoint{1.397221in}{2.156736in}}{\pgfqpoint{1.393948in}{2.164637in}}{\pgfqpoint{1.388124in}{2.170460in}}%
\pgfpathcurveto{\pgfqpoint{1.382301in}{2.176284in}}{\pgfqpoint{1.374400in}{2.179557in}}{\pgfqpoint{1.366164in}{2.179557in}}%
\pgfpathcurveto{\pgfqpoint{1.357928in}{2.179557in}}{\pgfqpoint{1.350028in}{2.176284in}}{\pgfqpoint{1.344204in}{2.170460in}}%
\pgfpathcurveto{\pgfqpoint{1.338380in}{2.164637in}}{\pgfqpoint{1.335108in}{2.156736in}}{\pgfqpoint{1.335108in}{2.148500in}}%
\pgfpathcurveto{\pgfqpoint{1.335108in}{2.140264in}}{\pgfqpoint{1.338380in}{2.132364in}}{\pgfqpoint{1.344204in}{2.126540in}}%
\pgfpathcurveto{\pgfqpoint{1.350028in}{2.120716in}}{\pgfqpoint{1.357928in}{2.117444in}}{\pgfqpoint{1.366164in}{2.117444in}}%
\pgfpathclose%
\pgfusepath{stroke,fill}%
\end{pgfscope}%
\begin{pgfscope}%
\pgfpathrectangle{\pgfqpoint{0.100000in}{0.212622in}}{\pgfqpoint{3.696000in}{3.696000in}}%
\pgfusepath{clip}%
\pgfsetbuttcap%
\pgfsetroundjoin%
\definecolor{currentfill}{rgb}{0.121569,0.466667,0.705882}%
\pgfsetfillcolor{currentfill}%
\pgfsetfillopacity{0.470886}%
\pgfsetlinewidth{1.003750pt}%
\definecolor{currentstroke}{rgb}{0.121569,0.466667,0.705882}%
\pgfsetstrokecolor{currentstroke}%
\pgfsetstrokeopacity{0.470886}%
\pgfsetdash{}{0pt}%
\pgfpathmoveto{\pgfqpoint{2.958054in}{1.826211in}}%
\pgfpathcurveto{\pgfqpoint{2.966290in}{1.826211in}}{\pgfqpoint{2.974190in}{1.829483in}}{\pgfqpoint{2.980014in}{1.835307in}}%
\pgfpathcurveto{\pgfqpoint{2.985838in}{1.841131in}}{\pgfqpoint{2.989110in}{1.849031in}}{\pgfqpoint{2.989110in}{1.857267in}}%
\pgfpathcurveto{\pgfqpoint{2.989110in}{1.865504in}}{\pgfqpoint{2.985838in}{1.873404in}}{\pgfqpoint{2.980014in}{1.879228in}}%
\pgfpathcurveto{\pgfqpoint{2.974190in}{1.885052in}}{\pgfqpoint{2.966290in}{1.888324in}}{\pgfqpoint{2.958054in}{1.888324in}}%
\pgfpathcurveto{\pgfqpoint{2.949818in}{1.888324in}}{\pgfqpoint{2.941918in}{1.885052in}}{\pgfqpoint{2.936094in}{1.879228in}}%
\pgfpathcurveto{\pgfqpoint{2.930270in}{1.873404in}}{\pgfqpoint{2.926997in}{1.865504in}}{\pgfqpoint{2.926997in}{1.857267in}}%
\pgfpathcurveto{\pgfqpoint{2.926997in}{1.849031in}}{\pgfqpoint{2.930270in}{1.841131in}}{\pgfqpoint{2.936094in}{1.835307in}}%
\pgfpathcurveto{\pgfqpoint{2.941918in}{1.829483in}}{\pgfqpoint{2.949818in}{1.826211in}}{\pgfqpoint{2.958054in}{1.826211in}}%
\pgfpathclose%
\pgfusepath{stroke,fill}%
\end{pgfscope}%
\begin{pgfscope}%
\pgfpathrectangle{\pgfqpoint{0.100000in}{0.212622in}}{\pgfqpoint{3.696000in}{3.696000in}}%
\pgfusepath{clip}%
\pgfsetbuttcap%
\pgfsetroundjoin%
\definecolor{currentfill}{rgb}{0.121569,0.466667,0.705882}%
\pgfsetfillcolor{currentfill}%
\pgfsetfillopacity{0.471682}%
\pgfsetlinewidth{1.003750pt}%
\definecolor{currentstroke}{rgb}{0.121569,0.466667,0.705882}%
\pgfsetstrokecolor{currentstroke}%
\pgfsetstrokeopacity{0.471682}%
\pgfsetdash{}{0pt}%
\pgfpathmoveto{\pgfqpoint{2.965541in}{1.824865in}}%
\pgfpathcurveto{\pgfqpoint{2.973777in}{1.824865in}}{\pgfqpoint{2.981677in}{1.828137in}}{\pgfqpoint{2.987501in}{1.833961in}}%
\pgfpathcurveto{\pgfqpoint{2.993325in}{1.839785in}}{\pgfqpoint{2.996597in}{1.847685in}}{\pgfqpoint{2.996597in}{1.855921in}}%
\pgfpathcurveto{\pgfqpoint{2.996597in}{1.864158in}}{\pgfqpoint{2.993325in}{1.872058in}}{\pgfqpoint{2.987501in}{1.877882in}}%
\pgfpathcurveto{\pgfqpoint{2.981677in}{1.883706in}}{\pgfqpoint{2.973777in}{1.886978in}}{\pgfqpoint{2.965541in}{1.886978in}}%
\pgfpathcurveto{\pgfqpoint{2.957304in}{1.886978in}}{\pgfqpoint{2.949404in}{1.883706in}}{\pgfqpoint{2.943580in}{1.877882in}}%
\pgfpathcurveto{\pgfqpoint{2.937756in}{1.872058in}}{\pgfqpoint{2.934484in}{1.864158in}}{\pgfqpoint{2.934484in}{1.855921in}}%
\pgfpathcurveto{\pgfqpoint{2.934484in}{1.847685in}}{\pgfqpoint{2.937756in}{1.839785in}}{\pgfqpoint{2.943580in}{1.833961in}}%
\pgfpathcurveto{\pgfqpoint{2.949404in}{1.828137in}}{\pgfqpoint{2.957304in}{1.824865in}}{\pgfqpoint{2.965541in}{1.824865in}}%
\pgfpathclose%
\pgfusepath{stroke,fill}%
\end{pgfscope}%
\begin{pgfscope}%
\pgfpathrectangle{\pgfqpoint{0.100000in}{0.212622in}}{\pgfqpoint{3.696000in}{3.696000in}}%
\pgfusepath{clip}%
\pgfsetbuttcap%
\pgfsetroundjoin%
\definecolor{currentfill}{rgb}{0.121569,0.466667,0.705882}%
\pgfsetfillcolor{currentfill}%
\pgfsetfillopacity{0.471990}%
\pgfsetlinewidth{1.003750pt}%
\definecolor{currentstroke}{rgb}{0.121569,0.466667,0.705882}%
\pgfsetstrokecolor{currentstroke}%
\pgfsetstrokeopacity{0.471990}%
\pgfsetdash{}{0pt}%
\pgfpathmoveto{\pgfqpoint{1.363994in}{2.117536in}}%
\pgfpathcurveto{\pgfqpoint{1.372230in}{2.117536in}}{\pgfqpoint{1.380130in}{2.120808in}}{\pgfqpoint{1.385954in}{2.126632in}}%
\pgfpathcurveto{\pgfqpoint{1.391778in}{2.132456in}}{\pgfqpoint{1.395050in}{2.140356in}}{\pgfqpoint{1.395050in}{2.148593in}}%
\pgfpathcurveto{\pgfqpoint{1.395050in}{2.156829in}}{\pgfqpoint{1.391778in}{2.164729in}}{\pgfqpoint{1.385954in}{2.170553in}}%
\pgfpathcurveto{\pgfqpoint{1.380130in}{2.176377in}}{\pgfqpoint{1.372230in}{2.179649in}}{\pgfqpoint{1.363994in}{2.179649in}}%
\pgfpathcurveto{\pgfqpoint{1.355758in}{2.179649in}}{\pgfqpoint{1.347858in}{2.176377in}}{\pgfqpoint{1.342034in}{2.170553in}}%
\pgfpathcurveto{\pgfqpoint{1.336210in}{2.164729in}}{\pgfqpoint{1.332937in}{2.156829in}}{\pgfqpoint{1.332937in}{2.148593in}}%
\pgfpathcurveto{\pgfqpoint{1.332937in}{2.140356in}}{\pgfqpoint{1.336210in}{2.132456in}}{\pgfqpoint{1.342034in}{2.126632in}}%
\pgfpathcurveto{\pgfqpoint{1.347858in}{2.120808in}}{\pgfqpoint{1.355758in}{2.117536in}}{\pgfqpoint{1.363994in}{2.117536in}}%
\pgfpathclose%
\pgfusepath{stroke,fill}%
\end{pgfscope}%
\begin{pgfscope}%
\pgfpathrectangle{\pgfqpoint{0.100000in}{0.212622in}}{\pgfqpoint{3.696000in}{3.696000in}}%
\pgfusepath{clip}%
\pgfsetbuttcap%
\pgfsetroundjoin%
\definecolor{currentfill}{rgb}{0.121569,0.466667,0.705882}%
\pgfsetfillcolor{currentfill}%
\pgfsetfillopacity{0.472151}%
\pgfsetlinewidth{1.003750pt}%
\definecolor{currentstroke}{rgb}{0.121569,0.466667,0.705882}%
\pgfsetstrokecolor{currentstroke}%
\pgfsetstrokeopacity{0.472151}%
\pgfsetdash{}{0pt}%
\pgfpathmoveto{\pgfqpoint{2.969615in}{1.824171in}}%
\pgfpathcurveto{\pgfqpoint{2.977852in}{1.824171in}}{\pgfqpoint{2.985752in}{1.827443in}}{\pgfqpoint{2.991576in}{1.833267in}}%
\pgfpathcurveto{\pgfqpoint{2.997400in}{1.839091in}}{\pgfqpoint{3.000672in}{1.846991in}}{\pgfqpoint{3.000672in}{1.855227in}}%
\pgfpathcurveto{\pgfqpoint{3.000672in}{1.863464in}}{\pgfqpoint{2.997400in}{1.871364in}}{\pgfqpoint{2.991576in}{1.877188in}}%
\pgfpathcurveto{\pgfqpoint{2.985752in}{1.883012in}}{\pgfqpoint{2.977852in}{1.886284in}}{\pgfqpoint{2.969615in}{1.886284in}}%
\pgfpathcurveto{\pgfqpoint{2.961379in}{1.886284in}}{\pgfqpoint{2.953479in}{1.883012in}}{\pgfqpoint{2.947655in}{1.877188in}}%
\pgfpathcurveto{\pgfqpoint{2.941831in}{1.871364in}}{\pgfqpoint{2.938559in}{1.863464in}}{\pgfqpoint{2.938559in}{1.855227in}}%
\pgfpathcurveto{\pgfqpoint{2.938559in}{1.846991in}}{\pgfqpoint{2.941831in}{1.839091in}}{\pgfqpoint{2.947655in}{1.833267in}}%
\pgfpathcurveto{\pgfqpoint{2.953479in}{1.827443in}}{\pgfqpoint{2.961379in}{1.824171in}}{\pgfqpoint{2.969615in}{1.824171in}}%
\pgfpathclose%
\pgfusepath{stroke,fill}%
\end{pgfscope}%
\begin{pgfscope}%
\pgfpathrectangle{\pgfqpoint{0.100000in}{0.212622in}}{\pgfqpoint{3.696000in}{3.696000in}}%
\pgfusepath{clip}%
\pgfsetbuttcap%
\pgfsetroundjoin%
\definecolor{currentfill}{rgb}{0.121569,0.466667,0.705882}%
\pgfsetfillcolor{currentfill}%
\pgfsetfillopacity{0.472689}%
\pgfsetlinewidth{1.003750pt}%
\definecolor{currentstroke}{rgb}{0.121569,0.466667,0.705882}%
\pgfsetstrokecolor{currentstroke}%
\pgfsetstrokeopacity{0.472689}%
\pgfsetdash{}{0pt}%
\pgfpathmoveto{\pgfqpoint{2.974164in}{1.823139in}}%
\pgfpathcurveto{\pgfqpoint{2.982400in}{1.823139in}}{\pgfqpoint{2.990300in}{1.826411in}}{\pgfqpoint{2.996124in}{1.832235in}}%
\pgfpathcurveto{\pgfqpoint{3.001948in}{1.838059in}}{\pgfqpoint{3.005220in}{1.845959in}}{\pgfqpoint{3.005220in}{1.854195in}}%
\pgfpathcurveto{\pgfqpoint{3.005220in}{1.862432in}}{\pgfqpoint{3.001948in}{1.870332in}}{\pgfqpoint{2.996124in}{1.876156in}}%
\pgfpathcurveto{\pgfqpoint{2.990300in}{1.881980in}}{\pgfqpoint{2.982400in}{1.885252in}}{\pgfqpoint{2.974164in}{1.885252in}}%
\pgfpathcurveto{\pgfqpoint{2.965927in}{1.885252in}}{\pgfqpoint{2.958027in}{1.881980in}}{\pgfqpoint{2.952203in}{1.876156in}}%
\pgfpathcurveto{\pgfqpoint{2.946379in}{1.870332in}}{\pgfqpoint{2.943107in}{1.862432in}}{\pgfqpoint{2.943107in}{1.854195in}}%
\pgfpathcurveto{\pgfqpoint{2.943107in}{1.845959in}}{\pgfqpoint{2.946379in}{1.838059in}}{\pgfqpoint{2.952203in}{1.832235in}}%
\pgfpathcurveto{\pgfqpoint{2.958027in}{1.826411in}}{\pgfqpoint{2.965927in}{1.823139in}}{\pgfqpoint{2.974164in}{1.823139in}}%
\pgfpathclose%
\pgfusepath{stroke,fill}%
\end{pgfscope}%
\begin{pgfscope}%
\pgfpathrectangle{\pgfqpoint{0.100000in}{0.212622in}}{\pgfqpoint{3.696000in}{3.696000in}}%
\pgfusepath{clip}%
\pgfsetbuttcap%
\pgfsetroundjoin%
\definecolor{currentfill}{rgb}{0.121569,0.466667,0.705882}%
\pgfsetfillcolor{currentfill}%
\pgfsetfillopacity{0.473082}%
\pgfsetlinewidth{1.003750pt}%
\definecolor{currentstroke}{rgb}{0.121569,0.466667,0.705882}%
\pgfsetstrokecolor{currentstroke}%
\pgfsetstrokeopacity{0.473082}%
\pgfsetdash{}{0pt}%
\pgfpathmoveto{\pgfqpoint{2.976491in}{1.822724in}}%
\pgfpathcurveto{\pgfqpoint{2.984727in}{1.822724in}}{\pgfqpoint{2.992627in}{1.825997in}}{\pgfqpoint{2.998451in}{1.831820in}}%
\pgfpathcurveto{\pgfqpoint{3.004275in}{1.837644in}}{\pgfqpoint{3.007547in}{1.845544in}}{\pgfqpoint{3.007547in}{1.853781in}}%
\pgfpathcurveto{\pgfqpoint{3.007547in}{1.862017in}}{\pgfqpoint{3.004275in}{1.869917in}}{\pgfqpoint{2.998451in}{1.875741in}}%
\pgfpathcurveto{\pgfqpoint{2.992627in}{1.881565in}}{\pgfqpoint{2.984727in}{1.884837in}}{\pgfqpoint{2.976491in}{1.884837in}}%
\pgfpathcurveto{\pgfqpoint{2.968255in}{1.884837in}}{\pgfqpoint{2.960355in}{1.881565in}}{\pgfqpoint{2.954531in}{1.875741in}}%
\pgfpathcurveto{\pgfqpoint{2.948707in}{1.869917in}}{\pgfqpoint{2.945434in}{1.862017in}}{\pgfqpoint{2.945434in}{1.853781in}}%
\pgfpathcurveto{\pgfqpoint{2.945434in}{1.845544in}}{\pgfqpoint{2.948707in}{1.837644in}}{\pgfqpoint{2.954531in}{1.831820in}}%
\pgfpathcurveto{\pgfqpoint{2.960355in}{1.825997in}}{\pgfqpoint{2.968255in}{1.822724in}}{\pgfqpoint{2.976491in}{1.822724in}}%
\pgfpathclose%
\pgfusepath{stroke,fill}%
\end{pgfscope}%
\begin{pgfscope}%
\pgfpathrectangle{\pgfqpoint{0.100000in}{0.212622in}}{\pgfqpoint{3.696000in}{3.696000in}}%
\pgfusepath{clip}%
\pgfsetbuttcap%
\pgfsetroundjoin%
\definecolor{currentfill}{rgb}{0.121569,0.466667,0.705882}%
\pgfsetfillcolor{currentfill}%
\pgfsetfillopacity{0.473253}%
\pgfsetlinewidth{1.003750pt}%
\definecolor{currentstroke}{rgb}{0.121569,0.466667,0.705882}%
\pgfsetstrokecolor{currentstroke}%
\pgfsetstrokeopacity{0.473253}%
\pgfsetdash{}{0pt}%
\pgfpathmoveto{\pgfqpoint{1.361017in}{2.117498in}}%
\pgfpathcurveto{\pgfqpoint{1.369253in}{2.117498in}}{\pgfqpoint{1.377153in}{2.120770in}}{\pgfqpoint{1.382977in}{2.126594in}}%
\pgfpathcurveto{\pgfqpoint{1.388801in}{2.132418in}}{\pgfqpoint{1.392073in}{2.140318in}}{\pgfqpoint{1.392073in}{2.148554in}}%
\pgfpathcurveto{\pgfqpoint{1.392073in}{2.156791in}}{\pgfqpoint{1.388801in}{2.164691in}}{\pgfqpoint{1.382977in}{2.170515in}}%
\pgfpathcurveto{\pgfqpoint{1.377153in}{2.176339in}}{\pgfqpoint{1.369253in}{2.179611in}}{\pgfqpoint{1.361017in}{2.179611in}}%
\pgfpathcurveto{\pgfqpoint{1.352781in}{2.179611in}}{\pgfqpoint{1.344881in}{2.176339in}}{\pgfqpoint{1.339057in}{2.170515in}}%
\pgfpathcurveto{\pgfqpoint{1.333233in}{2.164691in}}{\pgfqpoint{1.329960in}{2.156791in}}{\pgfqpoint{1.329960in}{2.148554in}}%
\pgfpathcurveto{\pgfqpoint{1.329960in}{2.140318in}}{\pgfqpoint{1.333233in}{2.132418in}}{\pgfqpoint{1.339057in}{2.126594in}}%
\pgfpathcurveto{\pgfqpoint{1.344881in}{2.120770in}}{\pgfqpoint{1.352781in}{2.117498in}}{\pgfqpoint{1.361017in}{2.117498in}}%
\pgfpathclose%
\pgfusepath{stroke,fill}%
\end{pgfscope}%
\begin{pgfscope}%
\pgfpathrectangle{\pgfqpoint{0.100000in}{0.212622in}}{\pgfqpoint{3.696000in}{3.696000in}}%
\pgfusepath{clip}%
\pgfsetbuttcap%
\pgfsetroundjoin%
\definecolor{currentfill}{rgb}{0.121569,0.466667,0.705882}%
\pgfsetfillcolor{currentfill}%
\pgfsetfillopacity{0.473485}%
\pgfsetlinewidth{1.003750pt}%
\definecolor{currentstroke}{rgb}{0.121569,0.466667,0.705882}%
\pgfsetstrokecolor{currentstroke}%
\pgfsetstrokeopacity{0.473485}%
\pgfsetdash{}{0pt}%
\pgfpathmoveto{\pgfqpoint{2.980033in}{1.822113in}}%
\pgfpathcurveto{\pgfqpoint{2.988269in}{1.822113in}}{\pgfqpoint{2.996169in}{1.825385in}}{\pgfqpoint{3.001993in}{1.831209in}}%
\pgfpathcurveto{\pgfqpoint{3.007817in}{1.837033in}}{\pgfqpoint{3.011089in}{1.844933in}}{\pgfqpoint{3.011089in}{1.853169in}}%
\pgfpathcurveto{\pgfqpoint{3.011089in}{1.861406in}}{\pgfqpoint{3.007817in}{1.869306in}}{\pgfqpoint{3.001993in}{1.875130in}}%
\pgfpathcurveto{\pgfqpoint{2.996169in}{1.880954in}}{\pgfqpoint{2.988269in}{1.884226in}}{\pgfqpoint{2.980033in}{1.884226in}}%
\pgfpathcurveto{\pgfqpoint{2.971796in}{1.884226in}}{\pgfqpoint{2.963896in}{1.880954in}}{\pgfqpoint{2.958072in}{1.875130in}}%
\pgfpathcurveto{\pgfqpoint{2.952249in}{1.869306in}}{\pgfqpoint{2.948976in}{1.861406in}}{\pgfqpoint{2.948976in}{1.853169in}}%
\pgfpathcurveto{\pgfqpoint{2.948976in}{1.844933in}}{\pgfqpoint{2.952249in}{1.837033in}}{\pgfqpoint{2.958072in}{1.831209in}}%
\pgfpathcurveto{\pgfqpoint{2.963896in}{1.825385in}}{\pgfqpoint{2.971796in}{1.822113in}}{\pgfqpoint{2.980033in}{1.822113in}}%
\pgfpathclose%
\pgfusepath{stroke,fill}%
\end{pgfscope}%
\begin{pgfscope}%
\pgfpathrectangle{\pgfqpoint{0.100000in}{0.212622in}}{\pgfqpoint{3.696000in}{3.696000in}}%
\pgfusepath{clip}%
\pgfsetbuttcap%
\pgfsetroundjoin%
\definecolor{currentfill}{rgb}{0.121569,0.466667,0.705882}%
\pgfsetfillcolor{currentfill}%
\pgfsetfillopacity{0.473993}%
\pgfsetlinewidth{1.003750pt}%
\definecolor{currentstroke}{rgb}{0.121569,0.466667,0.705882}%
\pgfsetstrokecolor{currentstroke}%
\pgfsetstrokeopacity{0.473993}%
\pgfsetdash{}{0pt}%
\pgfpathmoveto{\pgfqpoint{2.985357in}{1.820882in}}%
\pgfpathcurveto{\pgfqpoint{2.993594in}{1.820882in}}{\pgfqpoint{3.001494in}{1.824155in}}{\pgfqpoint{3.007318in}{1.829979in}}%
\pgfpathcurveto{\pgfqpoint{3.013142in}{1.835803in}}{\pgfqpoint{3.016414in}{1.843703in}}{\pgfqpoint{3.016414in}{1.851939in}}%
\pgfpathcurveto{\pgfqpoint{3.016414in}{1.860175in}}{\pgfqpoint{3.013142in}{1.868075in}}{\pgfqpoint{3.007318in}{1.873899in}}%
\pgfpathcurveto{\pgfqpoint{3.001494in}{1.879723in}}{\pgfqpoint{2.993594in}{1.882995in}}{\pgfqpoint{2.985357in}{1.882995in}}%
\pgfpathcurveto{\pgfqpoint{2.977121in}{1.882995in}}{\pgfqpoint{2.969221in}{1.879723in}}{\pgfqpoint{2.963397in}{1.873899in}}%
\pgfpathcurveto{\pgfqpoint{2.957573in}{1.868075in}}{\pgfqpoint{2.954301in}{1.860175in}}{\pgfqpoint{2.954301in}{1.851939in}}%
\pgfpathcurveto{\pgfqpoint{2.954301in}{1.843703in}}{\pgfqpoint{2.957573in}{1.835803in}}{\pgfqpoint{2.963397in}{1.829979in}}%
\pgfpathcurveto{\pgfqpoint{2.969221in}{1.824155in}}{\pgfqpoint{2.977121in}{1.820882in}}{\pgfqpoint{2.985357in}{1.820882in}}%
\pgfpathclose%
\pgfusepath{stroke,fill}%
\end{pgfscope}%
\begin{pgfscope}%
\pgfpathrectangle{\pgfqpoint{0.100000in}{0.212622in}}{\pgfqpoint{3.696000in}{3.696000in}}%
\pgfusepath{clip}%
\pgfsetbuttcap%
\pgfsetroundjoin%
\definecolor{currentfill}{rgb}{0.121569,0.466667,0.705882}%
\pgfsetfillcolor{currentfill}%
\pgfsetfillopacity{0.474546}%
\pgfsetlinewidth{1.003750pt}%
\definecolor{currentstroke}{rgb}{0.121569,0.466667,0.705882}%
\pgfsetstrokecolor{currentstroke}%
\pgfsetstrokeopacity{0.474546}%
\pgfsetdash{}{0pt}%
\pgfpathmoveto{\pgfqpoint{2.991384in}{1.819484in}}%
\pgfpathcurveto{\pgfqpoint{2.999620in}{1.819484in}}{\pgfqpoint{3.007520in}{1.822757in}}{\pgfqpoint{3.013344in}{1.828581in}}%
\pgfpathcurveto{\pgfqpoint{3.019168in}{1.834405in}}{\pgfqpoint{3.022441in}{1.842305in}}{\pgfqpoint{3.022441in}{1.850541in}}%
\pgfpathcurveto{\pgfqpoint{3.022441in}{1.858777in}}{\pgfqpoint{3.019168in}{1.866677in}}{\pgfqpoint{3.013344in}{1.872501in}}%
\pgfpathcurveto{\pgfqpoint{3.007520in}{1.878325in}}{\pgfqpoint{2.999620in}{1.881597in}}{\pgfqpoint{2.991384in}{1.881597in}}%
\pgfpathcurveto{\pgfqpoint{2.983148in}{1.881597in}}{\pgfqpoint{2.975248in}{1.878325in}}{\pgfqpoint{2.969424in}{1.872501in}}%
\pgfpathcurveto{\pgfqpoint{2.963600in}{1.866677in}}{\pgfqpoint{2.960328in}{1.858777in}}{\pgfqpoint{2.960328in}{1.850541in}}%
\pgfpathcurveto{\pgfqpoint{2.960328in}{1.842305in}}{\pgfqpoint{2.963600in}{1.834405in}}{\pgfqpoint{2.969424in}{1.828581in}}%
\pgfpathcurveto{\pgfqpoint{2.975248in}{1.822757in}}{\pgfqpoint{2.983148in}{1.819484in}}{\pgfqpoint{2.991384in}{1.819484in}}%
\pgfpathclose%
\pgfusepath{stroke,fill}%
\end{pgfscope}%
\begin{pgfscope}%
\pgfpathrectangle{\pgfqpoint{0.100000in}{0.212622in}}{\pgfqpoint{3.696000in}{3.696000in}}%
\pgfusepath{clip}%
\pgfsetbuttcap%
\pgfsetroundjoin%
\definecolor{currentfill}{rgb}{0.121569,0.466667,0.705882}%
\pgfsetfillcolor{currentfill}%
\pgfsetfillopacity{0.475267}%
\pgfsetlinewidth{1.003750pt}%
\definecolor{currentstroke}{rgb}{0.121569,0.466667,0.705882}%
\pgfsetstrokecolor{currentstroke}%
\pgfsetstrokeopacity{0.475267}%
\pgfsetdash{}{0pt}%
\pgfpathmoveto{\pgfqpoint{2.997593in}{1.818248in}}%
\pgfpathcurveto{\pgfqpoint{3.005829in}{1.818248in}}{\pgfqpoint{3.013729in}{1.821521in}}{\pgfqpoint{3.019553in}{1.827344in}}%
\pgfpathcurveto{\pgfqpoint{3.025377in}{1.833168in}}{\pgfqpoint{3.028649in}{1.841068in}}{\pgfqpoint{3.028649in}{1.849305in}}%
\pgfpathcurveto{\pgfqpoint{3.028649in}{1.857541in}}{\pgfqpoint{3.025377in}{1.865441in}}{\pgfqpoint{3.019553in}{1.871265in}}%
\pgfpathcurveto{\pgfqpoint{3.013729in}{1.877089in}}{\pgfqpoint{3.005829in}{1.880361in}}{\pgfqpoint{2.997593in}{1.880361in}}%
\pgfpathcurveto{\pgfqpoint{2.989356in}{1.880361in}}{\pgfqpoint{2.981456in}{1.877089in}}{\pgfqpoint{2.975632in}{1.871265in}}%
\pgfpathcurveto{\pgfqpoint{2.969808in}{1.865441in}}{\pgfqpoint{2.966536in}{1.857541in}}{\pgfqpoint{2.966536in}{1.849305in}}%
\pgfpathcurveto{\pgfqpoint{2.966536in}{1.841068in}}{\pgfqpoint{2.969808in}{1.833168in}}{\pgfqpoint{2.975632in}{1.827344in}}%
\pgfpathcurveto{\pgfqpoint{2.981456in}{1.821521in}}{\pgfqpoint{2.989356in}{1.818248in}}{\pgfqpoint{2.997593in}{1.818248in}}%
\pgfpathclose%
\pgfusepath{stroke,fill}%
\end{pgfscope}%
\begin{pgfscope}%
\pgfpathrectangle{\pgfqpoint{0.100000in}{0.212622in}}{\pgfqpoint{3.696000in}{3.696000in}}%
\pgfusepath{clip}%
\pgfsetbuttcap%
\pgfsetroundjoin%
\definecolor{currentfill}{rgb}{0.121569,0.466667,0.705882}%
\pgfsetfillcolor{currentfill}%
\pgfsetfillopacity{0.475690}%
\pgfsetlinewidth{1.003750pt}%
\definecolor{currentstroke}{rgb}{0.121569,0.466667,0.705882}%
\pgfsetstrokecolor{currentstroke}%
\pgfsetstrokeopacity{0.475690}%
\pgfsetdash{}{0pt}%
\pgfpathmoveto{\pgfqpoint{1.356471in}{2.117581in}}%
\pgfpathcurveto{\pgfqpoint{1.364708in}{2.117581in}}{\pgfqpoint{1.372608in}{2.120853in}}{\pgfqpoint{1.378432in}{2.126677in}}%
\pgfpathcurveto{\pgfqpoint{1.384256in}{2.132501in}}{\pgfqpoint{1.387528in}{2.140401in}}{\pgfqpoint{1.387528in}{2.148637in}}%
\pgfpathcurveto{\pgfqpoint{1.387528in}{2.156874in}}{\pgfqpoint{1.384256in}{2.164774in}}{\pgfqpoint{1.378432in}{2.170598in}}%
\pgfpathcurveto{\pgfqpoint{1.372608in}{2.176422in}}{\pgfqpoint{1.364708in}{2.179694in}}{\pgfqpoint{1.356471in}{2.179694in}}%
\pgfpathcurveto{\pgfqpoint{1.348235in}{2.179694in}}{\pgfqpoint{1.340335in}{2.176422in}}{\pgfqpoint{1.334511in}{2.170598in}}%
\pgfpathcurveto{\pgfqpoint{1.328687in}{2.164774in}}{\pgfqpoint{1.325415in}{2.156874in}}{\pgfqpoint{1.325415in}{2.148637in}}%
\pgfpathcurveto{\pgfqpoint{1.325415in}{2.140401in}}{\pgfqpoint{1.328687in}{2.132501in}}{\pgfqpoint{1.334511in}{2.126677in}}%
\pgfpathcurveto{\pgfqpoint{1.340335in}{2.120853in}}{\pgfqpoint{1.348235in}{2.117581in}}{\pgfqpoint{1.356471in}{2.117581in}}%
\pgfpathclose%
\pgfusepath{stroke,fill}%
\end{pgfscope}%
\begin{pgfscope}%
\pgfpathrectangle{\pgfqpoint{0.100000in}{0.212622in}}{\pgfqpoint{3.696000in}{3.696000in}}%
\pgfusepath{clip}%
\pgfsetbuttcap%
\pgfsetroundjoin%
\definecolor{currentfill}{rgb}{0.121569,0.466667,0.705882}%
\pgfsetfillcolor{currentfill}%
\pgfsetfillopacity{0.475965}%
\pgfsetlinewidth{1.003750pt}%
\definecolor{currentstroke}{rgb}{0.121569,0.466667,0.705882}%
\pgfsetstrokecolor{currentstroke}%
\pgfsetstrokeopacity{0.475965}%
\pgfsetdash{}{0pt}%
\pgfpathmoveto{\pgfqpoint{3.004160in}{1.816804in}}%
\pgfpathcurveto{\pgfqpoint{3.012396in}{1.816804in}}{\pgfqpoint{3.020296in}{1.820076in}}{\pgfqpoint{3.026120in}{1.825900in}}%
\pgfpathcurveto{\pgfqpoint{3.031944in}{1.831724in}}{\pgfqpoint{3.035217in}{1.839624in}}{\pgfqpoint{3.035217in}{1.847860in}}%
\pgfpathcurveto{\pgfqpoint{3.035217in}{1.856097in}}{\pgfqpoint{3.031944in}{1.863997in}}{\pgfqpoint{3.026120in}{1.869821in}}%
\pgfpathcurveto{\pgfqpoint{3.020296in}{1.875645in}}{\pgfqpoint{3.012396in}{1.878917in}}{\pgfqpoint{3.004160in}{1.878917in}}%
\pgfpathcurveto{\pgfqpoint{2.995924in}{1.878917in}}{\pgfqpoint{2.988024in}{1.875645in}}{\pgfqpoint{2.982200in}{1.869821in}}%
\pgfpathcurveto{\pgfqpoint{2.976376in}{1.863997in}}{\pgfqpoint{2.973104in}{1.856097in}}{\pgfqpoint{2.973104in}{1.847860in}}%
\pgfpathcurveto{\pgfqpoint{2.973104in}{1.839624in}}{\pgfqpoint{2.976376in}{1.831724in}}{\pgfqpoint{2.982200in}{1.825900in}}%
\pgfpathcurveto{\pgfqpoint{2.988024in}{1.820076in}}{\pgfqpoint{2.995924in}{1.816804in}}{\pgfqpoint{3.004160in}{1.816804in}}%
\pgfpathclose%
\pgfusepath{stroke,fill}%
\end{pgfscope}%
\begin{pgfscope}%
\pgfpathrectangle{\pgfqpoint{0.100000in}{0.212622in}}{\pgfqpoint{3.696000in}{3.696000in}}%
\pgfusepath{clip}%
\pgfsetbuttcap%
\pgfsetroundjoin%
\definecolor{currentfill}{rgb}{0.121569,0.466667,0.705882}%
\pgfsetfillcolor{currentfill}%
\pgfsetfillopacity{0.476229}%
\pgfsetlinewidth{1.003750pt}%
\definecolor{currentstroke}{rgb}{0.121569,0.466667,0.705882}%
\pgfsetstrokecolor{currentstroke}%
\pgfsetstrokeopacity{0.476229}%
\pgfsetdash{}{0pt}%
\pgfpathmoveto{\pgfqpoint{3.007949in}{1.815990in}}%
\pgfpathcurveto{\pgfqpoint{3.016185in}{1.815990in}}{\pgfqpoint{3.024085in}{1.819262in}}{\pgfqpoint{3.029909in}{1.825086in}}%
\pgfpathcurveto{\pgfqpoint{3.035733in}{1.830910in}}{\pgfqpoint{3.039005in}{1.838810in}}{\pgfqpoint{3.039005in}{1.847046in}}%
\pgfpathcurveto{\pgfqpoint{3.039005in}{1.855282in}}{\pgfqpoint{3.035733in}{1.863183in}}{\pgfqpoint{3.029909in}{1.869006in}}%
\pgfpathcurveto{\pgfqpoint{3.024085in}{1.874830in}}{\pgfqpoint{3.016185in}{1.878103in}}{\pgfqpoint{3.007949in}{1.878103in}}%
\pgfpathcurveto{\pgfqpoint{2.999712in}{1.878103in}}{\pgfqpoint{2.991812in}{1.874830in}}{\pgfqpoint{2.985989in}{1.869006in}}%
\pgfpathcurveto{\pgfqpoint{2.980165in}{1.863183in}}{\pgfqpoint{2.976892in}{1.855282in}}{\pgfqpoint{2.976892in}{1.847046in}}%
\pgfpathcurveto{\pgfqpoint{2.976892in}{1.838810in}}{\pgfqpoint{2.980165in}{1.830910in}}{\pgfqpoint{2.985989in}{1.825086in}}%
\pgfpathcurveto{\pgfqpoint{2.991812in}{1.819262in}}{\pgfqpoint{2.999712in}{1.815990in}}{\pgfqpoint{3.007949in}{1.815990in}}%
\pgfpathclose%
\pgfusepath{stroke,fill}%
\end{pgfscope}%
\begin{pgfscope}%
\pgfpathrectangle{\pgfqpoint{0.100000in}{0.212622in}}{\pgfqpoint{3.696000in}{3.696000in}}%
\pgfusepath{clip}%
\pgfsetbuttcap%
\pgfsetroundjoin%
\definecolor{currentfill}{rgb}{0.121569,0.466667,0.705882}%
\pgfsetfillcolor{currentfill}%
\pgfsetfillopacity{0.476782}%
\pgfsetlinewidth{1.003750pt}%
\definecolor{currentstroke}{rgb}{0.121569,0.466667,0.705882}%
\pgfsetstrokecolor{currentstroke}%
\pgfsetstrokeopacity{0.476782}%
\pgfsetdash{}{0pt}%
\pgfpathmoveto{\pgfqpoint{3.013092in}{1.815039in}}%
\pgfpathcurveto{\pgfqpoint{3.021328in}{1.815039in}}{\pgfqpoint{3.029228in}{1.818311in}}{\pgfqpoint{3.035052in}{1.824135in}}%
\pgfpathcurveto{\pgfqpoint{3.040876in}{1.829959in}}{\pgfqpoint{3.044149in}{1.837859in}}{\pgfqpoint{3.044149in}{1.846096in}}%
\pgfpathcurveto{\pgfqpoint{3.044149in}{1.854332in}}{\pgfqpoint{3.040876in}{1.862232in}}{\pgfqpoint{3.035052in}{1.868056in}}%
\pgfpathcurveto{\pgfqpoint{3.029228in}{1.873880in}}{\pgfqpoint{3.021328in}{1.877152in}}{\pgfqpoint{3.013092in}{1.877152in}}%
\pgfpathcurveto{\pgfqpoint{3.004856in}{1.877152in}}{\pgfqpoint{2.996956in}{1.873880in}}{\pgfqpoint{2.991132in}{1.868056in}}%
\pgfpathcurveto{\pgfqpoint{2.985308in}{1.862232in}}{\pgfqpoint{2.982036in}{1.854332in}}{\pgfqpoint{2.982036in}{1.846096in}}%
\pgfpathcurveto{\pgfqpoint{2.982036in}{1.837859in}}{\pgfqpoint{2.985308in}{1.829959in}}{\pgfqpoint{2.991132in}{1.824135in}}%
\pgfpathcurveto{\pgfqpoint{2.996956in}{1.818311in}}{\pgfqpoint{3.004856in}{1.815039in}}{\pgfqpoint{3.013092in}{1.815039in}}%
\pgfpathclose%
\pgfusepath{stroke,fill}%
\end{pgfscope}%
\begin{pgfscope}%
\pgfpathrectangle{\pgfqpoint{0.100000in}{0.212622in}}{\pgfqpoint{3.696000in}{3.696000in}}%
\pgfusepath{clip}%
\pgfsetbuttcap%
\pgfsetroundjoin%
\definecolor{currentfill}{rgb}{0.121569,0.466667,0.705882}%
\pgfsetfillcolor{currentfill}%
\pgfsetfillopacity{0.477673}%
\pgfsetlinewidth{1.003750pt}%
\definecolor{currentstroke}{rgb}{0.121569,0.466667,0.705882}%
\pgfsetstrokecolor{currentstroke}%
\pgfsetstrokeopacity{0.477673}%
\pgfsetdash{}{0pt}%
\pgfpathmoveto{\pgfqpoint{3.020410in}{1.813649in}}%
\pgfpathcurveto{\pgfqpoint{3.028647in}{1.813649in}}{\pgfqpoint{3.036547in}{1.816921in}}{\pgfqpoint{3.042371in}{1.822745in}}%
\pgfpathcurveto{\pgfqpoint{3.048195in}{1.828569in}}{\pgfqpoint{3.051467in}{1.836469in}}{\pgfqpoint{3.051467in}{1.844705in}}%
\pgfpathcurveto{\pgfqpoint{3.051467in}{1.852941in}}{\pgfqpoint{3.048195in}{1.860842in}}{\pgfqpoint{3.042371in}{1.866665in}}%
\pgfpathcurveto{\pgfqpoint{3.036547in}{1.872489in}}{\pgfqpoint{3.028647in}{1.875762in}}{\pgfqpoint{3.020410in}{1.875762in}}%
\pgfpathcurveto{\pgfqpoint{3.012174in}{1.875762in}}{\pgfqpoint{3.004274in}{1.872489in}}{\pgfqpoint{2.998450in}{1.866665in}}%
\pgfpathcurveto{\pgfqpoint{2.992626in}{1.860842in}}{\pgfqpoint{2.989354in}{1.852941in}}{\pgfqpoint{2.989354in}{1.844705in}}%
\pgfpathcurveto{\pgfqpoint{2.989354in}{1.836469in}}{\pgfqpoint{2.992626in}{1.828569in}}{\pgfqpoint{2.998450in}{1.822745in}}%
\pgfpathcurveto{\pgfqpoint{3.004274in}{1.816921in}}{\pgfqpoint{3.012174in}{1.813649in}}{\pgfqpoint{3.020410in}{1.813649in}}%
\pgfpathclose%
\pgfusepath{stroke,fill}%
\end{pgfscope}%
\begin{pgfscope}%
\pgfpathrectangle{\pgfqpoint{0.100000in}{0.212622in}}{\pgfqpoint{3.696000in}{3.696000in}}%
\pgfusepath{clip}%
\pgfsetbuttcap%
\pgfsetroundjoin%
\definecolor{currentfill}{rgb}{0.121569,0.466667,0.705882}%
\pgfsetfillcolor{currentfill}%
\pgfsetfillopacity{0.477893}%
\pgfsetlinewidth{1.003750pt}%
\definecolor{currentstroke}{rgb}{0.121569,0.466667,0.705882}%
\pgfsetstrokecolor{currentstroke}%
\pgfsetstrokeopacity{0.477893}%
\pgfsetdash{}{0pt}%
\pgfpathmoveto{\pgfqpoint{1.352733in}{2.117483in}}%
\pgfpathcurveto{\pgfqpoint{1.360969in}{2.117483in}}{\pgfqpoint{1.368869in}{2.120755in}}{\pgfqpoint{1.374693in}{2.126579in}}%
\pgfpathcurveto{\pgfqpoint{1.380517in}{2.132403in}}{\pgfqpoint{1.383789in}{2.140303in}}{\pgfqpoint{1.383789in}{2.148539in}}%
\pgfpathcurveto{\pgfqpoint{1.383789in}{2.156775in}}{\pgfqpoint{1.380517in}{2.164675in}}{\pgfqpoint{1.374693in}{2.170499in}}%
\pgfpathcurveto{\pgfqpoint{1.368869in}{2.176323in}}{\pgfqpoint{1.360969in}{2.179596in}}{\pgfqpoint{1.352733in}{2.179596in}}%
\pgfpathcurveto{\pgfqpoint{1.344497in}{2.179596in}}{\pgfqpoint{1.336597in}{2.176323in}}{\pgfqpoint{1.330773in}{2.170499in}}%
\pgfpathcurveto{\pgfqpoint{1.324949in}{2.164675in}}{\pgfqpoint{1.321676in}{2.156775in}}{\pgfqpoint{1.321676in}{2.148539in}}%
\pgfpathcurveto{\pgfqpoint{1.321676in}{2.140303in}}{\pgfqpoint{1.324949in}{2.132403in}}{\pgfqpoint{1.330773in}{2.126579in}}%
\pgfpathcurveto{\pgfqpoint{1.336597in}{2.120755in}}{\pgfqpoint{1.344497in}{2.117483in}}{\pgfqpoint{1.352733in}{2.117483in}}%
\pgfpathclose%
\pgfusepath{stroke,fill}%
\end{pgfscope}%
\begin{pgfscope}%
\pgfpathrectangle{\pgfqpoint{0.100000in}{0.212622in}}{\pgfqpoint{3.696000in}{3.696000in}}%
\pgfusepath{clip}%
\pgfsetbuttcap%
\pgfsetroundjoin%
\definecolor{currentfill}{rgb}{0.121569,0.466667,0.705882}%
\pgfsetfillcolor{currentfill}%
\pgfsetfillopacity{0.478491}%
\pgfsetlinewidth{1.003750pt}%
\definecolor{currentstroke}{rgb}{0.121569,0.466667,0.705882}%
\pgfsetstrokecolor{currentstroke}%
\pgfsetstrokeopacity{0.478491}%
\pgfsetdash{}{0pt}%
\pgfpathmoveto{\pgfqpoint{3.029405in}{1.811622in}}%
\pgfpathcurveto{\pgfqpoint{3.037641in}{1.811622in}}{\pgfqpoint{3.045541in}{1.814894in}}{\pgfqpoint{3.051365in}{1.820718in}}%
\pgfpathcurveto{\pgfqpoint{3.057189in}{1.826542in}}{\pgfqpoint{3.060462in}{1.834442in}}{\pgfqpoint{3.060462in}{1.842679in}}%
\pgfpathcurveto{\pgfqpoint{3.060462in}{1.850915in}}{\pgfqpoint{3.057189in}{1.858815in}}{\pgfqpoint{3.051365in}{1.864639in}}%
\pgfpathcurveto{\pgfqpoint{3.045541in}{1.870463in}}{\pgfqpoint{3.037641in}{1.873735in}}{\pgfqpoint{3.029405in}{1.873735in}}%
\pgfpathcurveto{\pgfqpoint{3.021169in}{1.873735in}}{\pgfqpoint{3.013269in}{1.870463in}}{\pgfqpoint{3.007445in}{1.864639in}}%
\pgfpathcurveto{\pgfqpoint{3.001621in}{1.858815in}}{\pgfqpoint{2.998349in}{1.850915in}}{\pgfqpoint{2.998349in}{1.842679in}}%
\pgfpathcurveto{\pgfqpoint{2.998349in}{1.834442in}}{\pgfqpoint{3.001621in}{1.826542in}}{\pgfqpoint{3.007445in}{1.820718in}}%
\pgfpathcurveto{\pgfqpoint{3.013269in}{1.814894in}}{\pgfqpoint{3.021169in}{1.811622in}}{\pgfqpoint{3.029405in}{1.811622in}}%
\pgfpathclose%
\pgfusepath{stroke,fill}%
\end{pgfscope}%
\begin{pgfscope}%
\pgfpathrectangle{\pgfqpoint{0.100000in}{0.212622in}}{\pgfqpoint{3.696000in}{3.696000in}}%
\pgfusepath{clip}%
\pgfsetbuttcap%
\pgfsetroundjoin%
\definecolor{currentfill}{rgb}{0.121569,0.466667,0.705882}%
\pgfsetfillcolor{currentfill}%
\pgfsetfillopacity{0.479798}%
\pgfsetlinewidth{1.003750pt}%
\definecolor{currentstroke}{rgb}{0.121569,0.466667,0.705882}%
\pgfsetstrokecolor{currentstroke}%
\pgfsetstrokeopacity{0.479798}%
\pgfsetdash{}{0pt}%
\pgfpathmoveto{\pgfqpoint{1.347963in}{2.118023in}}%
\pgfpathcurveto{\pgfqpoint{1.356199in}{2.118023in}}{\pgfqpoint{1.364099in}{2.121296in}}{\pgfqpoint{1.369923in}{2.127120in}}%
\pgfpathcurveto{\pgfqpoint{1.375747in}{2.132943in}}{\pgfqpoint{1.379019in}{2.140844in}}{\pgfqpoint{1.379019in}{2.149080in}}%
\pgfpathcurveto{\pgfqpoint{1.379019in}{2.157316in}}{\pgfqpoint{1.375747in}{2.165216in}}{\pgfqpoint{1.369923in}{2.171040in}}%
\pgfpathcurveto{\pgfqpoint{1.364099in}{2.176864in}}{\pgfqpoint{1.356199in}{2.180136in}}{\pgfqpoint{1.347963in}{2.180136in}}%
\pgfpathcurveto{\pgfqpoint{1.339726in}{2.180136in}}{\pgfqpoint{1.331826in}{2.176864in}}{\pgfqpoint{1.326002in}{2.171040in}}%
\pgfpathcurveto{\pgfqpoint{1.320178in}{2.165216in}}{\pgfqpoint{1.316906in}{2.157316in}}{\pgfqpoint{1.316906in}{2.149080in}}%
\pgfpathcurveto{\pgfqpoint{1.316906in}{2.140844in}}{\pgfqpoint{1.320178in}{2.132943in}}{\pgfqpoint{1.326002in}{2.127120in}}%
\pgfpathcurveto{\pgfqpoint{1.331826in}{2.121296in}}{\pgfqpoint{1.339726in}{2.118023in}}{\pgfqpoint{1.347963in}{2.118023in}}%
\pgfpathclose%
\pgfusepath{stroke,fill}%
\end{pgfscope}%
\begin{pgfscope}%
\pgfpathrectangle{\pgfqpoint{0.100000in}{0.212622in}}{\pgfqpoint{3.696000in}{3.696000in}}%
\pgfusepath{clip}%
\pgfsetbuttcap%
\pgfsetroundjoin%
\definecolor{currentfill}{rgb}{0.121569,0.466667,0.705882}%
\pgfsetfillcolor{currentfill}%
\pgfsetfillopacity{0.479896}%
\pgfsetlinewidth{1.003750pt}%
\definecolor{currentstroke}{rgb}{0.121569,0.466667,0.705882}%
\pgfsetstrokecolor{currentstroke}%
\pgfsetstrokeopacity{0.479896}%
\pgfsetdash{}{0pt}%
\pgfpathmoveto{\pgfqpoint{3.038063in}{1.810556in}}%
\pgfpathcurveto{\pgfqpoint{3.046299in}{1.810556in}}{\pgfqpoint{3.054199in}{1.813828in}}{\pgfqpoint{3.060023in}{1.819652in}}%
\pgfpathcurveto{\pgfqpoint{3.065847in}{1.825476in}}{\pgfqpoint{3.069119in}{1.833376in}}{\pgfqpoint{3.069119in}{1.841612in}}%
\pgfpathcurveto{\pgfqpoint{3.069119in}{1.849848in}}{\pgfqpoint{3.065847in}{1.857748in}}{\pgfqpoint{3.060023in}{1.863572in}}%
\pgfpathcurveto{\pgfqpoint{3.054199in}{1.869396in}}{\pgfqpoint{3.046299in}{1.872669in}}{\pgfqpoint{3.038063in}{1.872669in}}%
\pgfpathcurveto{\pgfqpoint{3.029827in}{1.872669in}}{\pgfqpoint{3.021926in}{1.869396in}}{\pgfqpoint{3.016103in}{1.863572in}}%
\pgfpathcurveto{\pgfqpoint{3.010279in}{1.857748in}}{\pgfqpoint{3.007006in}{1.849848in}}{\pgfqpoint{3.007006in}{1.841612in}}%
\pgfpathcurveto{\pgfqpoint{3.007006in}{1.833376in}}{\pgfqpoint{3.010279in}{1.825476in}}{\pgfqpoint{3.016103in}{1.819652in}}%
\pgfpathcurveto{\pgfqpoint{3.021926in}{1.813828in}}{\pgfqpoint{3.029827in}{1.810556in}}{\pgfqpoint{3.038063in}{1.810556in}}%
\pgfpathclose%
\pgfusepath{stroke,fill}%
\end{pgfscope}%
\begin{pgfscope}%
\pgfpathrectangle{\pgfqpoint{0.100000in}{0.212622in}}{\pgfqpoint{3.696000in}{3.696000in}}%
\pgfusepath{clip}%
\pgfsetbuttcap%
\pgfsetroundjoin%
\definecolor{currentfill}{rgb}{0.121569,0.466667,0.705882}%
\pgfsetfillcolor{currentfill}%
\pgfsetfillopacity{0.480845}%
\pgfsetlinewidth{1.003750pt}%
\definecolor{currentstroke}{rgb}{0.121569,0.466667,0.705882}%
\pgfsetstrokecolor{currentstroke}%
\pgfsetstrokeopacity{0.480845}%
\pgfsetdash{}{0pt}%
\pgfpathmoveto{\pgfqpoint{3.047858in}{1.808347in}}%
\pgfpathcurveto{\pgfqpoint{3.056095in}{1.808347in}}{\pgfqpoint{3.063995in}{1.811620in}}{\pgfqpoint{3.069819in}{1.817444in}}%
\pgfpathcurveto{\pgfqpoint{3.075643in}{1.823268in}}{\pgfqpoint{3.078915in}{1.831168in}}{\pgfqpoint{3.078915in}{1.839404in}}%
\pgfpathcurveto{\pgfqpoint{3.078915in}{1.847640in}}{\pgfqpoint{3.075643in}{1.855540in}}{\pgfqpoint{3.069819in}{1.861364in}}%
\pgfpathcurveto{\pgfqpoint{3.063995in}{1.867188in}}{\pgfqpoint{3.056095in}{1.870460in}}{\pgfqpoint{3.047858in}{1.870460in}}%
\pgfpathcurveto{\pgfqpoint{3.039622in}{1.870460in}}{\pgfqpoint{3.031722in}{1.867188in}}{\pgfqpoint{3.025898in}{1.861364in}}%
\pgfpathcurveto{\pgfqpoint{3.020074in}{1.855540in}}{\pgfqpoint{3.016802in}{1.847640in}}{\pgfqpoint{3.016802in}{1.839404in}}%
\pgfpathcurveto{\pgfqpoint{3.016802in}{1.831168in}}{\pgfqpoint{3.020074in}{1.823268in}}{\pgfqpoint{3.025898in}{1.817444in}}%
\pgfpathcurveto{\pgfqpoint{3.031722in}{1.811620in}}{\pgfqpoint{3.039622in}{1.808347in}}{\pgfqpoint{3.047858in}{1.808347in}}%
\pgfpathclose%
\pgfusepath{stroke,fill}%
\end{pgfscope}%
\begin{pgfscope}%
\pgfpathrectangle{\pgfqpoint{0.100000in}{0.212622in}}{\pgfqpoint{3.696000in}{3.696000in}}%
\pgfusepath{clip}%
\pgfsetbuttcap%
\pgfsetroundjoin%
\definecolor{currentfill}{rgb}{0.121569,0.466667,0.705882}%
\pgfsetfillcolor{currentfill}%
\pgfsetfillopacity{0.481255}%
\pgfsetlinewidth{1.003750pt}%
\definecolor{currentstroke}{rgb}{0.121569,0.466667,0.705882}%
\pgfsetstrokecolor{currentstroke}%
\pgfsetstrokeopacity{0.481255}%
\pgfsetdash{}{0pt}%
\pgfpathmoveto{\pgfqpoint{1.345148in}{2.117995in}}%
\pgfpathcurveto{\pgfqpoint{1.353384in}{2.117995in}}{\pgfqpoint{1.361284in}{2.121267in}}{\pgfqpoint{1.367108in}{2.127091in}}%
\pgfpathcurveto{\pgfqpoint{1.372932in}{2.132915in}}{\pgfqpoint{1.376205in}{2.140815in}}{\pgfqpoint{1.376205in}{2.149051in}}%
\pgfpathcurveto{\pgfqpoint{1.376205in}{2.157288in}}{\pgfqpoint{1.372932in}{2.165188in}}{\pgfqpoint{1.367108in}{2.171012in}}%
\pgfpathcurveto{\pgfqpoint{1.361284in}{2.176835in}}{\pgfqpoint{1.353384in}{2.180108in}}{\pgfqpoint{1.345148in}{2.180108in}}%
\pgfpathcurveto{\pgfqpoint{1.336912in}{2.180108in}}{\pgfqpoint{1.329012in}{2.176835in}}{\pgfqpoint{1.323188in}{2.171012in}}%
\pgfpathcurveto{\pgfqpoint{1.317364in}{2.165188in}}{\pgfqpoint{1.314092in}{2.157288in}}{\pgfqpoint{1.314092in}{2.149051in}}%
\pgfpathcurveto{\pgfqpoint{1.314092in}{2.140815in}}{\pgfqpoint{1.317364in}{2.132915in}}{\pgfqpoint{1.323188in}{2.127091in}}%
\pgfpathcurveto{\pgfqpoint{1.329012in}{2.121267in}}{\pgfqpoint{1.336912in}{2.117995in}}{\pgfqpoint{1.345148in}{2.117995in}}%
\pgfpathclose%
\pgfusepath{stroke,fill}%
\end{pgfscope}%
\begin{pgfscope}%
\pgfpathrectangle{\pgfqpoint{0.100000in}{0.212622in}}{\pgfqpoint{3.696000in}{3.696000in}}%
\pgfusepath{clip}%
\pgfsetbuttcap%
\pgfsetroundjoin%
\definecolor{currentfill}{rgb}{0.121569,0.466667,0.705882}%
\pgfsetfillcolor{currentfill}%
\pgfsetfillopacity{0.481486}%
\pgfsetlinewidth{1.003750pt}%
\definecolor{currentstroke}{rgb}{0.121569,0.466667,0.705882}%
\pgfsetstrokecolor{currentstroke}%
\pgfsetstrokeopacity{0.481486}%
\pgfsetdash{}{0pt}%
\pgfpathmoveto{\pgfqpoint{3.053083in}{1.807319in}}%
\pgfpathcurveto{\pgfqpoint{3.061320in}{1.807319in}}{\pgfqpoint{3.069220in}{1.810592in}}{\pgfqpoint{3.075044in}{1.816416in}}%
\pgfpathcurveto{\pgfqpoint{3.080867in}{1.822240in}}{\pgfqpoint{3.084140in}{1.830140in}}{\pgfqpoint{3.084140in}{1.838376in}}%
\pgfpathcurveto{\pgfqpoint{3.084140in}{1.846612in}}{\pgfqpoint{3.080867in}{1.854512in}}{\pgfqpoint{3.075044in}{1.860336in}}%
\pgfpathcurveto{\pgfqpoint{3.069220in}{1.866160in}}{\pgfqpoint{3.061320in}{1.869432in}}{\pgfqpoint{3.053083in}{1.869432in}}%
\pgfpathcurveto{\pgfqpoint{3.044847in}{1.869432in}}{\pgfqpoint{3.036947in}{1.866160in}}{\pgfqpoint{3.031123in}{1.860336in}}%
\pgfpathcurveto{\pgfqpoint{3.025299in}{1.854512in}}{\pgfqpoint{3.022027in}{1.846612in}}{\pgfqpoint{3.022027in}{1.838376in}}%
\pgfpathcurveto{\pgfqpoint{3.022027in}{1.830140in}}{\pgfqpoint{3.025299in}{1.822240in}}{\pgfqpoint{3.031123in}{1.816416in}}%
\pgfpathcurveto{\pgfqpoint{3.036947in}{1.810592in}}{\pgfqpoint{3.044847in}{1.807319in}}{\pgfqpoint{3.053083in}{1.807319in}}%
\pgfpathclose%
\pgfusepath{stroke,fill}%
\end{pgfscope}%
\begin{pgfscope}%
\pgfpathrectangle{\pgfqpoint{0.100000in}{0.212622in}}{\pgfqpoint{3.696000in}{3.696000in}}%
\pgfusepath{clip}%
\pgfsetbuttcap%
\pgfsetroundjoin%
\definecolor{currentfill}{rgb}{0.121569,0.466667,0.705882}%
\pgfsetfillcolor{currentfill}%
\pgfsetfillopacity{0.482093}%
\pgfsetlinewidth{1.003750pt}%
\definecolor{currentstroke}{rgb}{0.121569,0.466667,0.705882}%
\pgfsetstrokecolor{currentstroke}%
\pgfsetstrokeopacity{0.482093}%
\pgfsetdash{}{0pt}%
\pgfpathmoveto{\pgfqpoint{3.059297in}{1.806133in}}%
\pgfpathcurveto{\pgfqpoint{3.067533in}{1.806133in}}{\pgfqpoint{3.075433in}{1.809405in}}{\pgfqpoint{3.081257in}{1.815229in}}%
\pgfpathcurveto{\pgfqpoint{3.087081in}{1.821053in}}{\pgfqpoint{3.090353in}{1.828953in}}{\pgfqpoint{3.090353in}{1.837189in}}%
\pgfpathcurveto{\pgfqpoint{3.090353in}{1.845425in}}{\pgfqpoint{3.087081in}{1.853325in}}{\pgfqpoint{3.081257in}{1.859149in}}%
\pgfpathcurveto{\pgfqpoint{3.075433in}{1.864973in}}{\pgfqpoint{3.067533in}{1.868246in}}{\pgfqpoint{3.059297in}{1.868246in}}%
\pgfpathcurveto{\pgfqpoint{3.051061in}{1.868246in}}{\pgfqpoint{3.043161in}{1.864973in}}{\pgfqpoint{3.037337in}{1.859149in}}%
\pgfpathcurveto{\pgfqpoint{3.031513in}{1.853325in}}{\pgfqpoint{3.028240in}{1.845425in}}{\pgfqpoint{3.028240in}{1.837189in}}%
\pgfpathcurveto{\pgfqpoint{3.028240in}{1.828953in}}{\pgfqpoint{3.031513in}{1.821053in}}{\pgfqpoint{3.037337in}{1.815229in}}%
\pgfpathcurveto{\pgfqpoint{3.043161in}{1.809405in}}{\pgfqpoint{3.051061in}{1.806133in}}{\pgfqpoint{3.059297in}{1.806133in}}%
\pgfpathclose%
\pgfusepath{stroke,fill}%
\end{pgfscope}%
\begin{pgfscope}%
\pgfpathrectangle{\pgfqpoint{0.100000in}{0.212622in}}{\pgfqpoint{3.696000in}{3.696000in}}%
\pgfusepath{clip}%
\pgfsetbuttcap%
\pgfsetroundjoin%
\definecolor{currentfill}{rgb}{0.121569,0.466667,0.705882}%
\pgfsetfillcolor{currentfill}%
\pgfsetfillopacity{0.482099}%
\pgfsetlinewidth{1.003750pt}%
\definecolor{currentstroke}{rgb}{0.121569,0.466667,0.705882}%
\pgfsetstrokecolor{currentstroke}%
\pgfsetstrokeopacity{0.482099}%
\pgfsetdash{}{0pt}%
\pgfpathmoveto{\pgfqpoint{1.343387in}{2.118032in}}%
\pgfpathcurveto{\pgfqpoint{1.351623in}{2.118032in}}{\pgfqpoint{1.359523in}{2.121304in}}{\pgfqpoint{1.365347in}{2.127128in}}%
\pgfpathcurveto{\pgfqpoint{1.371171in}{2.132952in}}{\pgfqpoint{1.374443in}{2.140852in}}{\pgfqpoint{1.374443in}{2.149089in}}%
\pgfpathcurveto{\pgfqpoint{1.374443in}{2.157325in}}{\pgfqpoint{1.371171in}{2.165225in}}{\pgfqpoint{1.365347in}{2.171049in}}%
\pgfpathcurveto{\pgfqpoint{1.359523in}{2.176873in}}{\pgfqpoint{1.351623in}{2.180145in}}{\pgfqpoint{1.343387in}{2.180145in}}%
\pgfpathcurveto{\pgfqpoint{1.335150in}{2.180145in}}{\pgfqpoint{1.327250in}{2.176873in}}{\pgfqpoint{1.321426in}{2.171049in}}%
\pgfpathcurveto{\pgfqpoint{1.315603in}{2.165225in}}{\pgfqpoint{1.312330in}{2.157325in}}{\pgfqpoint{1.312330in}{2.149089in}}%
\pgfpathcurveto{\pgfqpoint{1.312330in}{2.140852in}}{\pgfqpoint{1.315603in}{2.132952in}}{\pgfqpoint{1.321426in}{2.127128in}}%
\pgfpathcurveto{\pgfqpoint{1.327250in}{2.121304in}}{\pgfqpoint{1.335150in}{2.118032in}}{\pgfqpoint{1.343387in}{2.118032in}}%
\pgfpathclose%
\pgfusepath{stroke,fill}%
\end{pgfscope}%
\begin{pgfscope}%
\pgfpathrectangle{\pgfqpoint{0.100000in}{0.212622in}}{\pgfqpoint{3.696000in}{3.696000in}}%
\pgfusepath{clip}%
\pgfsetbuttcap%
\pgfsetroundjoin%
\definecolor{currentfill}{rgb}{0.121569,0.466667,0.705882}%
\pgfsetfillcolor{currentfill}%
\pgfsetfillopacity{0.483058}%
\pgfsetlinewidth{1.003750pt}%
\definecolor{currentstroke}{rgb}{0.121569,0.466667,0.705882}%
\pgfsetstrokecolor{currentstroke}%
\pgfsetstrokeopacity{0.483058}%
\pgfsetdash{}{0pt}%
\pgfpathmoveto{\pgfqpoint{3.067014in}{1.804365in}}%
\pgfpathcurveto{\pgfqpoint{3.075250in}{1.804365in}}{\pgfqpoint{3.083150in}{1.807637in}}{\pgfqpoint{3.088974in}{1.813461in}}%
\pgfpathcurveto{\pgfqpoint{3.094798in}{1.819285in}}{\pgfqpoint{3.098070in}{1.827185in}}{\pgfqpoint{3.098070in}{1.835421in}}%
\pgfpathcurveto{\pgfqpoint{3.098070in}{1.843658in}}{\pgfqpoint{3.094798in}{1.851558in}}{\pgfqpoint{3.088974in}{1.857382in}}%
\pgfpathcurveto{\pgfqpoint{3.083150in}{1.863205in}}{\pgfqpoint{3.075250in}{1.866478in}}{\pgfqpoint{3.067014in}{1.866478in}}%
\pgfpathcurveto{\pgfqpoint{3.058777in}{1.866478in}}{\pgfqpoint{3.050877in}{1.863205in}}{\pgfqpoint{3.045053in}{1.857382in}}%
\pgfpathcurveto{\pgfqpoint{3.039230in}{1.851558in}}{\pgfqpoint{3.035957in}{1.843658in}}{\pgfqpoint{3.035957in}{1.835421in}}%
\pgfpathcurveto{\pgfqpoint{3.035957in}{1.827185in}}{\pgfqpoint{3.039230in}{1.819285in}}{\pgfqpoint{3.045053in}{1.813461in}}%
\pgfpathcurveto{\pgfqpoint{3.050877in}{1.807637in}}{\pgfqpoint{3.058777in}{1.804365in}}{\pgfqpoint{3.067014in}{1.804365in}}%
\pgfpathclose%
\pgfusepath{stroke,fill}%
\end{pgfscope}%
\begin{pgfscope}%
\pgfpathrectangle{\pgfqpoint{0.100000in}{0.212622in}}{\pgfqpoint{3.696000in}{3.696000in}}%
\pgfusepath{clip}%
\pgfsetbuttcap%
\pgfsetroundjoin%
\definecolor{currentfill}{rgb}{0.121569,0.466667,0.705882}%
\pgfsetfillcolor{currentfill}%
\pgfsetfillopacity{0.483616}%
\pgfsetlinewidth{1.003750pt}%
\definecolor{currentstroke}{rgb}{0.121569,0.466667,0.705882}%
\pgfsetstrokecolor{currentstroke}%
\pgfsetstrokeopacity{0.483616}%
\pgfsetdash{}{0pt}%
\pgfpathmoveto{\pgfqpoint{1.340030in}{2.118114in}}%
\pgfpathcurveto{\pgfqpoint{1.348266in}{2.118114in}}{\pgfqpoint{1.356166in}{2.121386in}}{\pgfqpoint{1.361990in}{2.127210in}}%
\pgfpathcurveto{\pgfqpoint{1.367814in}{2.133034in}}{\pgfqpoint{1.371086in}{2.140934in}}{\pgfqpoint{1.371086in}{2.149170in}}%
\pgfpathcurveto{\pgfqpoint{1.371086in}{2.157407in}}{\pgfqpoint{1.367814in}{2.165307in}}{\pgfqpoint{1.361990in}{2.171131in}}%
\pgfpathcurveto{\pgfqpoint{1.356166in}{2.176954in}}{\pgfqpoint{1.348266in}{2.180227in}}{\pgfqpoint{1.340030in}{2.180227in}}%
\pgfpathcurveto{\pgfqpoint{1.331794in}{2.180227in}}{\pgfqpoint{1.323894in}{2.176954in}}{\pgfqpoint{1.318070in}{2.171131in}}%
\pgfpathcurveto{\pgfqpoint{1.312246in}{2.165307in}}{\pgfqpoint{1.308973in}{2.157407in}}{\pgfqpoint{1.308973in}{2.149170in}}%
\pgfpathcurveto{\pgfqpoint{1.308973in}{2.140934in}}{\pgfqpoint{1.312246in}{2.133034in}}{\pgfqpoint{1.318070in}{2.127210in}}%
\pgfpathcurveto{\pgfqpoint{1.323894in}{2.121386in}}{\pgfqpoint{1.331794in}{2.118114in}}{\pgfqpoint{1.340030in}{2.118114in}}%
\pgfpathclose%
\pgfusepath{stroke,fill}%
\end{pgfscope}%
\begin{pgfscope}%
\pgfpathrectangle{\pgfqpoint{0.100000in}{0.212622in}}{\pgfqpoint{3.696000in}{3.696000in}}%
\pgfusepath{clip}%
\pgfsetbuttcap%
\pgfsetroundjoin%
\definecolor{currentfill}{rgb}{0.121569,0.466667,0.705882}%
\pgfsetfillcolor{currentfill}%
\pgfsetfillopacity{0.483676}%
\pgfsetlinewidth{1.003750pt}%
\definecolor{currentstroke}{rgb}{0.121569,0.466667,0.705882}%
\pgfsetstrokecolor{currentstroke}%
\pgfsetstrokeopacity{0.483676}%
\pgfsetdash{}{0pt}%
\pgfpathmoveto{\pgfqpoint{3.075679in}{1.802021in}}%
\pgfpathcurveto{\pgfqpoint{3.083915in}{1.802021in}}{\pgfqpoint{3.091815in}{1.805293in}}{\pgfqpoint{3.097639in}{1.811117in}}%
\pgfpathcurveto{\pgfqpoint{3.103463in}{1.816941in}}{\pgfqpoint{3.106736in}{1.824841in}}{\pgfqpoint{3.106736in}{1.833077in}}%
\pgfpathcurveto{\pgfqpoint{3.106736in}{1.841314in}}{\pgfqpoint{3.103463in}{1.849214in}}{\pgfqpoint{3.097639in}{1.855038in}}%
\pgfpathcurveto{\pgfqpoint{3.091815in}{1.860861in}}{\pgfqpoint{3.083915in}{1.864134in}}{\pgfqpoint{3.075679in}{1.864134in}}%
\pgfpathcurveto{\pgfqpoint{3.067443in}{1.864134in}}{\pgfqpoint{3.059543in}{1.860861in}}{\pgfqpoint{3.053719in}{1.855038in}}%
\pgfpathcurveto{\pgfqpoint{3.047895in}{1.849214in}}{\pgfqpoint{3.044623in}{1.841314in}}{\pgfqpoint{3.044623in}{1.833077in}}%
\pgfpathcurveto{\pgfqpoint{3.044623in}{1.824841in}}{\pgfqpoint{3.047895in}{1.816941in}}{\pgfqpoint{3.053719in}{1.811117in}}%
\pgfpathcurveto{\pgfqpoint{3.059543in}{1.805293in}}{\pgfqpoint{3.067443in}{1.802021in}}{\pgfqpoint{3.075679in}{1.802021in}}%
\pgfpathclose%
\pgfusepath{stroke,fill}%
\end{pgfscope}%
\begin{pgfscope}%
\pgfpathrectangle{\pgfqpoint{0.100000in}{0.212622in}}{\pgfqpoint{3.696000in}{3.696000in}}%
\pgfusepath{clip}%
\pgfsetbuttcap%
\pgfsetroundjoin%
\definecolor{currentfill}{rgb}{0.121569,0.466667,0.705882}%
\pgfsetfillcolor{currentfill}%
\pgfsetfillopacity{0.484231}%
\pgfsetlinewidth{1.003750pt}%
\definecolor{currentstroke}{rgb}{0.121569,0.466667,0.705882}%
\pgfsetstrokecolor{currentstroke}%
\pgfsetstrokeopacity{0.484231}%
\pgfsetdash{}{0pt}%
\pgfpathmoveto{\pgfqpoint{3.080251in}{1.801347in}}%
\pgfpathcurveto{\pgfqpoint{3.088487in}{1.801347in}}{\pgfqpoint{3.096387in}{1.804619in}}{\pgfqpoint{3.102211in}{1.810443in}}%
\pgfpathcurveto{\pgfqpoint{3.108035in}{1.816267in}}{\pgfqpoint{3.111308in}{1.824167in}}{\pgfqpoint{3.111308in}{1.832404in}}%
\pgfpathcurveto{\pgfqpoint{3.111308in}{1.840640in}}{\pgfqpoint{3.108035in}{1.848540in}}{\pgfqpoint{3.102211in}{1.854364in}}%
\pgfpathcurveto{\pgfqpoint{3.096387in}{1.860188in}}{\pgfqpoint{3.088487in}{1.863460in}}{\pgfqpoint{3.080251in}{1.863460in}}%
\pgfpathcurveto{\pgfqpoint{3.072015in}{1.863460in}}{\pgfqpoint{3.064115in}{1.860188in}}{\pgfqpoint{3.058291in}{1.854364in}}%
\pgfpathcurveto{\pgfqpoint{3.052467in}{1.848540in}}{\pgfqpoint{3.049195in}{1.840640in}}{\pgfqpoint{3.049195in}{1.832404in}}%
\pgfpathcurveto{\pgfqpoint{3.049195in}{1.824167in}}{\pgfqpoint{3.052467in}{1.816267in}}{\pgfqpoint{3.058291in}{1.810443in}}%
\pgfpathcurveto{\pgfqpoint{3.064115in}{1.804619in}}{\pgfqpoint{3.072015in}{1.801347in}}{\pgfqpoint{3.080251in}{1.801347in}}%
\pgfpathclose%
\pgfusepath{stroke,fill}%
\end{pgfscope}%
\begin{pgfscope}%
\pgfpathrectangle{\pgfqpoint{0.100000in}{0.212622in}}{\pgfqpoint{3.696000in}{3.696000in}}%
\pgfusepath{clip}%
\pgfsetbuttcap%
\pgfsetroundjoin%
\definecolor{currentfill}{rgb}{0.121569,0.466667,0.705882}%
\pgfsetfillcolor{currentfill}%
\pgfsetfillopacity{0.484984}%
\pgfsetlinewidth{1.003750pt}%
\definecolor{currentstroke}{rgb}{0.121569,0.466667,0.705882}%
\pgfsetstrokecolor{currentstroke}%
\pgfsetstrokeopacity{0.484984}%
\pgfsetdash{}{0pt}%
\pgfpathmoveto{\pgfqpoint{3.085393in}{1.800493in}}%
\pgfpathcurveto{\pgfqpoint{3.093629in}{1.800493in}}{\pgfqpoint{3.101529in}{1.803765in}}{\pgfqpoint{3.107353in}{1.809589in}}%
\pgfpathcurveto{\pgfqpoint{3.113177in}{1.815413in}}{\pgfqpoint{3.116449in}{1.823313in}}{\pgfqpoint{3.116449in}{1.831549in}}%
\pgfpathcurveto{\pgfqpoint{3.116449in}{1.839786in}}{\pgfqpoint{3.113177in}{1.847686in}}{\pgfqpoint{3.107353in}{1.853509in}}%
\pgfpathcurveto{\pgfqpoint{3.101529in}{1.859333in}}{\pgfqpoint{3.093629in}{1.862606in}}{\pgfqpoint{3.085393in}{1.862606in}}%
\pgfpathcurveto{\pgfqpoint{3.077156in}{1.862606in}}{\pgfqpoint{3.069256in}{1.859333in}}{\pgfqpoint{3.063432in}{1.853509in}}%
\pgfpathcurveto{\pgfqpoint{3.057608in}{1.847686in}}{\pgfqpoint{3.054336in}{1.839786in}}{\pgfqpoint{3.054336in}{1.831549in}}%
\pgfpathcurveto{\pgfqpoint{3.054336in}{1.823313in}}{\pgfqpoint{3.057608in}{1.815413in}}{\pgfqpoint{3.063432in}{1.809589in}}%
\pgfpathcurveto{\pgfqpoint{3.069256in}{1.803765in}}{\pgfqpoint{3.077156in}{1.800493in}}{\pgfqpoint{3.085393in}{1.800493in}}%
\pgfpathclose%
\pgfusepath{stroke,fill}%
\end{pgfscope}%
\begin{pgfscope}%
\pgfpathrectangle{\pgfqpoint{0.100000in}{0.212622in}}{\pgfqpoint{3.696000in}{3.696000in}}%
\pgfusepath{clip}%
\pgfsetbuttcap%
\pgfsetroundjoin%
\definecolor{currentfill}{rgb}{0.121569,0.466667,0.705882}%
\pgfsetfillcolor{currentfill}%
\pgfsetfillopacity{0.485067}%
\pgfsetlinewidth{1.003750pt}%
\definecolor{currentstroke}{rgb}{0.121569,0.466667,0.705882}%
\pgfsetstrokecolor{currentstroke}%
\pgfsetstrokeopacity{0.485067}%
\pgfsetdash{}{0pt}%
\pgfpathmoveto{\pgfqpoint{1.338049in}{2.118245in}}%
\pgfpathcurveto{\pgfqpoint{1.346285in}{2.118245in}}{\pgfqpoint{1.354185in}{2.121517in}}{\pgfqpoint{1.360009in}{2.127341in}}%
\pgfpathcurveto{\pgfqpoint{1.365833in}{2.133165in}}{\pgfqpoint{1.369105in}{2.141065in}}{\pgfqpoint{1.369105in}{2.149301in}}%
\pgfpathcurveto{\pgfqpoint{1.369105in}{2.157538in}}{\pgfqpoint{1.365833in}{2.165438in}}{\pgfqpoint{1.360009in}{2.171262in}}%
\pgfpathcurveto{\pgfqpoint{1.354185in}{2.177085in}}{\pgfqpoint{1.346285in}{2.180358in}}{\pgfqpoint{1.338049in}{2.180358in}}%
\pgfpathcurveto{\pgfqpoint{1.329812in}{2.180358in}}{\pgfqpoint{1.321912in}{2.177085in}}{\pgfqpoint{1.316088in}{2.171262in}}%
\pgfpathcurveto{\pgfqpoint{1.310265in}{2.165438in}}{\pgfqpoint{1.306992in}{2.157538in}}{\pgfqpoint{1.306992in}{2.149301in}}%
\pgfpathcurveto{\pgfqpoint{1.306992in}{2.141065in}}{\pgfqpoint{1.310265in}{2.133165in}}{\pgfqpoint{1.316088in}{2.127341in}}%
\pgfpathcurveto{\pgfqpoint{1.321912in}{2.121517in}}{\pgfqpoint{1.329812in}{2.118245in}}{\pgfqpoint{1.338049in}{2.118245in}}%
\pgfpathclose%
\pgfusepath{stroke,fill}%
\end{pgfscope}%
\begin{pgfscope}%
\pgfpathrectangle{\pgfqpoint{0.100000in}{0.212622in}}{\pgfqpoint{3.696000in}{3.696000in}}%
\pgfusepath{clip}%
\pgfsetbuttcap%
\pgfsetroundjoin%
\definecolor{currentfill}{rgb}{0.121569,0.466667,0.705882}%
\pgfsetfillcolor{currentfill}%
\pgfsetfillopacity{0.485519}%
\pgfsetlinewidth{1.003750pt}%
\definecolor{currentstroke}{rgb}{0.121569,0.466667,0.705882}%
\pgfsetstrokecolor{currentstroke}%
\pgfsetstrokeopacity{0.485519}%
\pgfsetdash{}{0pt}%
\pgfpathmoveto{\pgfqpoint{3.091504in}{1.799128in}}%
\pgfpathcurveto{\pgfqpoint{3.099741in}{1.799128in}}{\pgfqpoint{3.107641in}{1.802400in}}{\pgfqpoint{3.113465in}{1.808224in}}%
\pgfpathcurveto{\pgfqpoint{3.119289in}{1.814048in}}{\pgfqpoint{3.122561in}{1.821948in}}{\pgfqpoint{3.122561in}{1.830185in}}%
\pgfpathcurveto{\pgfqpoint{3.122561in}{1.838421in}}{\pgfqpoint{3.119289in}{1.846321in}}{\pgfqpoint{3.113465in}{1.852145in}}%
\pgfpathcurveto{\pgfqpoint{3.107641in}{1.857969in}}{\pgfqpoint{3.099741in}{1.861241in}}{\pgfqpoint{3.091504in}{1.861241in}}%
\pgfpathcurveto{\pgfqpoint{3.083268in}{1.861241in}}{\pgfqpoint{3.075368in}{1.857969in}}{\pgfqpoint{3.069544in}{1.852145in}}%
\pgfpathcurveto{\pgfqpoint{3.063720in}{1.846321in}}{\pgfqpoint{3.060448in}{1.838421in}}{\pgfqpoint{3.060448in}{1.830185in}}%
\pgfpathcurveto{\pgfqpoint{3.060448in}{1.821948in}}{\pgfqpoint{3.063720in}{1.814048in}}{\pgfqpoint{3.069544in}{1.808224in}}%
\pgfpathcurveto{\pgfqpoint{3.075368in}{1.802400in}}{\pgfqpoint{3.083268in}{1.799128in}}{\pgfqpoint{3.091504in}{1.799128in}}%
\pgfpathclose%
\pgfusepath{stroke,fill}%
\end{pgfscope}%
\begin{pgfscope}%
\pgfpathrectangle{\pgfqpoint{0.100000in}{0.212622in}}{\pgfqpoint{3.696000in}{3.696000in}}%
\pgfusepath{clip}%
\pgfsetbuttcap%
\pgfsetroundjoin%
\definecolor{currentfill}{rgb}{0.121569,0.466667,0.705882}%
\pgfsetfillcolor{currentfill}%
\pgfsetfillopacity{0.486196}%
\pgfsetlinewidth{1.003750pt}%
\definecolor{currentstroke}{rgb}{0.121569,0.466667,0.705882}%
\pgfsetstrokecolor{currentstroke}%
\pgfsetstrokeopacity{0.486196}%
\pgfsetdash{}{0pt}%
\pgfpathmoveto{\pgfqpoint{1.335115in}{2.118440in}}%
\pgfpathcurveto{\pgfqpoint{1.343351in}{2.118440in}}{\pgfqpoint{1.351251in}{2.121713in}}{\pgfqpoint{1.357075in}{2.127537in}}%
\pgfpathcurveto{\pgfqpoint{1.362899in}{2.133360in}}{\pgfqpoint{1.366171in}{2.141260in}}{\pgfqpoint{1.366171in}{2.149497in}}%
\pgfpathcurveto{\pgfqpoint{1.366171in}{2.157733in}}{\pgfqpoint{1.362899in}{2.165633in}}{\pgfqpoint{1.357075in}{2.171457in}}%
\pgfpathcurveto{\pgfqpoint{1.351251in}{2.177281in}}{\pgfqpoint{1.343351in}{2.180553in}}{\pgfqpoint{1.335115in}{2.180553in}}%
\pgfpathcurveto{\pgfqpoint{1.326878in}{2.180553in}}{\pgfqpoint{1.318978in}{2.177281in}}{\pgfqpoint{1.313154in}{2.171457in}}%
\pgfpathcurveto{\pgfqpoint{1.307330in}{2.165633in}}{\pgfqpoint{1.304058in}{2.157733in}}{\pgfqpoint{1.304058in}{2.149497in}}%
\pgfpathcurveto{\pgfqpoint{1.304058in}{2.141260in}}{\pgfqpoint{1.307330in}{2.133360in}}{\pgfqpoint{1.313154in}{2.127537in}}%
\pgfpathcurveto{\pgfqpoint{1.318978in}{2.121713in}}{\pgfqpoint{1.326878in}{2.118440in}}{\pgfqpoint{1.335115in}{2.118440in}}%
\pgfpathclose%
\pgfusepath{stroke,fill}%
\end{pgfscope}%
\begin{pgfscope}%
\pgfpathrectangle{\pgfqpoint{0.100000in}{0.212622in}}{\pgfqpoint{3.696000in}{3.696000in}}%
\pgfusepath{clip}%
\pgfsetbuttcap%
\pgfsetroundjoin%
\definecolor{currentfill}{rgb}{0.121569,0.466667,0.705882}%
\pgfsetfillcolor{currentfill}%
\pgfsetfillopacity{0.486448}%
\pgfsetlinewidth{1.003750pt}%
\definecolor{currentstroke}{rgb}{0.121569,0.466667,0.705882}%
\pgfsetstrokecolor{currentstroke}%
\pgfsetstrokeopacity{0.486448}%
\pgfsetdash{}{0pt}%
\pgfpathmoveto{\pgfqpoint{3.097591in}{1.798009in}}%
\pgfpathcurveto{\pgfqpoint{3.105827in}{1.798009in}}{\pgfqpoint{3.113727in}{1.801281in}}{\pgfqpoint{3.119551in}{1.807105in}}%
\pgfpathcurveto{\pgfqpoint{3.125375in}{1.812929in}}{\pgfqpoint{3.128648in}{1.820829in}}{\pgfqpoint{3.128648in}{1.829065in}}%
\pgfpathcurveto{\pgfqpoint{3.128648in}{1.837302in}}{\pgfqpoint{3.125375in}{1.845202in}}{\pgfqpoint{3.119551in}{1.851026in}}%
\pgfpathcurveto{\pgfqpoint{3.113727in}{1.856850in}}{\pgfqpoint{3.105827in}{1.860122in}}{\pgfqpoint{3.097591in}{1.860122in}}%
\pgfpathcurveto{\pgfqpoint{3.089355in}{1.860122in}}{\pgfqpoint{3.081455in}{1.856850in}}{\pgfqpoint{3.075631in}{1.851026in}}%
\pgfpathcurveto{\pgfqpoint{3.069807in}{1.845202in}}{\pgfqpoint{3.066535in}{1.837302in}}{\pgfqpoint{3.066535in}{1.829065in}}%
\pgfpathcurveto{\pgfqpoint{3.066535in}{1.820829in}}{\pgfqpoint{3.069807in}{1.812929in}}{\pgfqpoint{3.075631in}{1.807105in}}%
\pgfpathcurveto{\pgfqpoint{3.081455in}{1.801281in}}{\pgfqpoint{3.089355in}{1.798009in}}{\pgfqpoint{3.097591in}{1.798009in}}%
\pgfpathclose%
\pgfusepath{stroke,fill}%
\end{pgfscope}%
\begin{pgfscope}%
\pgfpathrectangle{\pgfqpoint{0.100000in}{0.212622in}}{\pgfqpoint{3.696000in}{3.696000in}}%
\pgfusepath{clip}%
\pgfsetbuttcap%
\pgfsetroundjoin%
\definecolor{currentfill}{rgb}{0.121569,0.466667,0.705882}%
\pgfsetfillcolor{currentfill}%
\pgfsetfillopacity{0.487215}%
\pgfsetlinewidth{1.003750pt}%
\definecolor{currentstroke}{rgb}{0.121569,0.466667,0.705882}%
\pgfsetstrokecolor{currentstroke}%
\pgfsetstrokeopacity{0.487215}%
\pgfsetdash{}{0pt}%
\pgfpathmoveto{\pgfqpoint{1.336649in}{2.119193in}}%
\pgfpathcurveto{\pgfqpoint{1.344885in}{2.119193in}}{\pgfqpoint{1.352785in}{2.122465in}}{\pgfqpoint{1.358609in}{2.128289in}}%
\pgfpathcurveto{\pgfqpoint{1.364433in}{2.134113in}}{\pgfqpoint{1.367706in}{2.142013in}}{\pgfqpoint{1.367706in}{2.150249in}}%
\pgfpathcurveto{\pgfqpoint{1.367706in}{2.158486in}}{\pgfqpoint{1.364433in}{2.166386in}}{\pgfqpoint{1.358609in}{2.172210in}}%
\pgfpathcurveto{\pgfqpoint{1.352785in}{2.178034in}}{\pgfqpoint{1.344885in}{2.181306in}}{\pgfqpoint{1.336649in}{2.181306in}}%
\pgfpathcurveto{\pgfqpoint{1.328413in}{2.181306in}}{\pgfqpoint{1.320513in}{2.178034in}}{\pgfqpoint{1.314689in}{2.172210in}}%
\pgfpathcurveto{\pgfqpoint{1.308865in}{2.166386in}}{\pgfqpoint{1.305593in}{2.158486in}}{\pgfqpoint{1.305593in}{2.150249in}}%
\pgfpathcurveto{\pgfqpoint{1.305593in}{2.142013in}}{\pgfqpoint{1.308865in}{2.134113in}}{\pgfqpoint{1.314689in}{2.128289in}}%
\pgfpathcurveto{\pgfqpoint{1.320513in}{2.122465in}}{\pgfqpoint{1.328413in}{2.119193in}}{\pgfqpoint{1.336649in}{2.119193in}}%
\pgfpathclose%
\pgfusepath{stroke,fill}%
\end{pgfscope}%
\begin{pgfscope}%
\pgfpathrectangle{\pgfqpoint{0.100000in}{0.212622in}}{\pgfqpoint{3.696000in}{3.696000in}}%
\pgfusepath{clip}%
\pgfsetbuttcap%
\pgfsetroundjoin%
\definecolor{currentfill}{rgb}{0.121569,0.466667,0.705882}%
\pgfsetfillcolor{currentfill}%
\pgfsetfillopacity{0.487443}%
\pgfsetlinewidth{1.003750pt}%
\definecolor{currentstroke}{rgb}{0.121569,0.466667,0.705882}%
\pgfsetstrokecolor{currentstroke}%
\pgfsetstrokeopacity{0.487443}%
\pgfsetdash{}{0pt}%
\pgfpathmoveto{\pgfqpoint{3.104346in}{1.796899in}}%
\pgfpathcurveto{\pgfqpoint{3.112583in}{1.796899in}}{\pgfqpoint{3.120483in}{1.800171in}}{\pgfqpoint{3.126307in}{1.805995in}}%
\pgfpathcurveto{\pgfqpoint{3.132131in}{1.811819in}}{\pgfqpoint{3.135403in}{1.819719in}}{\pgfqpoint{3.135403in}{1.827955in}}%
\pgfpathcurveto{\pgfqpoint{3.135403in}{1.836192in}}{\pgfqpoint{3.132131in}{1.844092in}}{\pgfqpoint{3.126307in}{1.849916in}}%
\pgfpathcurveto{\pgfqpoint{3.120483in}{1.855739in}}{\pgfqpoint{3.112583in}{1.859012in}}{\pgfqpoint{3.104346in}{1.859012in}}%
\pgfpathcurveto{\pgfqpoint{3.096110in}{1.859012in}}{\pgfqpoint{3.088210in}{1.855739in}}{\pgfqpoint{3.082386in}{1.849916in}}%
\pgfpathcurveto{\pgfqpoint{3.076562in}{1.844092in}}{\pgfqpoint{3.073290in}{1.836192in}}{\pgfqpoint{3.073290in}{1.827955in}}%
\pgfpathcurveto{\pgfqpoint{3.073290in}{1.819719in}}{\pgfqpoint{3.076562in}{1.811819in}}{\pgfqpoint{3.082386in}{1.805995in}}%
\pgfpathcurveto{\pgfqpoint{3.088210in}{1.800171in}}{\pgfqpoint{3.096110in}{1.796899in}}{\pgfqpoint{3.104346in}{1.796899in}}%
\pgfpathclose%
\pgfusepath{stroke,fill}%
\end{pgfscope}%
\begin{pgfscope}%
\pgfpathrectangle{\pgfqpoint{0.100000in}{0.212622in}}{\pgfqpoint{3.696000in}{3.696000in}}%
\pgfusepath{clip}%
\pgfsetbuttcap%
\pgfsetroundjoin%
\definecolor{currentfill}{rgb}{0.121569,0.466667,0.705882}%
\pgfsetfillcolor{currentfill}%
\pgfsetfillopacity{0.488057}%
\pgfsetlinewidth{1.003750pt}%
\definecolor{currentstroke}{rgb}{0.121569,0.466667,0.705882}%
\pgfsetstrokecolor{currentstroke}%
\pgfsetstrokeopacity{0.488057}%
\pgfsetdash{}{0pt}%
\pgfpathmoveto{\pgfqpoint{1.334849in}{2.119291in}}%
\pgfpathcurveto{\pgfqpoint{1.343085in}{2.119291in}}{\pgfqpoint{1.350985in}{2.122563in}}{\pgfqpoint{1.356809in}{2.128387in}}%
\pgfpathcurveto{\pgfqpoint{1.362633in}{2.134211in}}{\pgfqpoint{1.365905in}{2.142111in}}{\pgfqpoint{1.365905in}{2.150347in}}%
\pgfpathcurveto{\pgfqpoint{1.365905in}{2.158584in}}{\pgfqpoint{1.362633in}{2.166484in}}{\pgfqpoint{1.356809in}{2.172308in}}%
\pgfpathcurveto{\pgfqpoint{1.350985in}{2.178132in}}{\pgfqpoint{1.343085in}{2.181404in}}{\pgfqpoint{1.334849in}{2.181404in}}%
\pgfpathcurveto{\pgfqpoint{1.326613in}{2.181404in}}{\pgfqpoint{1.318713in}{2.178132in}}{\pgfqpoint{1.312889in}{2.172308in}}%
\pgfpathcurveto{\pgfqpoint{1.307065in}{2.166484in}}{\pgfqpoint{1.303792in}{2.158584in}}{\pgfqpoint{1.303792in}{2.150347in}}%
\pgfpathcurveto{\pgfqpoint{1.303792in}{2.142111in}}{\pgfqpoint{1.307065in}{2.134211in}}{\pgfqpoint{1.312889in}{2.128387in}}%
\pgfpathcurveto{\pgfqpoint{1.318713in}{2.122563in}}{\pgfqpoint{1.326613in}{2.119291in}}{\pgfqpoint{1.334849in}{2.119291in}}%
\pgfpathclose%
\pgfusepath{stroke,fill}%
\end{pgfscope}%
\begin{pgfscope}%
\pgfpathrectangle{\pgfqpoint{0.100000in}{0.212622in}}{\pgfqpoint{3.696000in}{3.696000in}}%
\pgfusepath{clip}%
\pgfsetbuttcap%
\pgfsetroundjoin%
\definecolor{currentfill}{rgb}{0.121569,0.466667,0.705882}%
\pgfsetfillcolor{currentfill}%
\pgfsetfillopacity{0.488480}%
\pgfsetlinewidth{1.003750pt}%
\definecolor{currentstroke}{rgb}{0.121569,0.466667,0.705882}%
\pgfsetstrokecolor{currentstroke}%
\pgfsetstrokeopacity{0.488480}%
\pgfsetdash{}{0pt}%
\pgfpathmoveto{\pgfqpoint{3.112048in}{1.795573in}}%
\pgfpathcurveto{\pgfqpoint{3.120284in}{1.795573in}}{\pgfqpoint{3.128184in}{1.798846in}}{\pgfqpoint{3.134008in}{1.804670in}}%
\pgfpathcurveto{\pgfqpoint{3.139832in}{1.810494in}}{\pgfqpoint{3.143105in}{1.818394in}}{\pgfqpoint{3.143105in}{1.826630in}}%
\pgfpathcurveto{\pgfqpoint{3.143105in}{1.834866in}}{\pgfqpoint{3.139832in}{1.842766in}}{\pgfqpoint{3.134008in}{1.848590in}}%
\pgfpathcurveto{\pgfqpoint{3.128184in}{1.854414in}}{\pgfqpoint{3.120284in}{1.857686in}}{\pgfqpoint{3.112048in}{1.857686in}}%
\pgfpathcurveto{\pgfqpoint{3.103812in}{1.857686in}}{\pgfqpoint{3.095912in}{1.854414in}}{\pgfqpoint{3.090088in}{1.848590in}}%
\pgfpathcurveto{\pgfqpoint{3.084264in}{1.842766in}}{\pgfqpoint{3.080992in}{1.834866in}}{\pgfqpoint{3.080992in}{1.826630in}}%
\pgfpathcurveto{\pgfqpoint{3.080992in}{1.818394in}}{\pgfqpoint{3.084264in}{1.810494in}}{\pgfqpoint{3.090088in}{1.804670in}}%
\pgfpathcurveto{\pgfqpoint{3.095912in}{1.798846in}}{\pgfqpoint{3.103812in}{1.795573in}}{\pgfqpoint{3.112048in}{1.795573in}}%
\pgfpathclose%
\pgfusepath{stroke,fill}%
\end{pgfscope}%
\begin{pgfscope}%
\pgfpathrectangle{\pgfqpoint{0.100000in}{0.212622in}}{\pgfqpoint{3.696000in}{3.696000in}}%
\pgfusepath{clip}%
\pgfsetbuttcap%
\pgfsetroundjoin%
\definecolor{currentfill}{rgb}{0.121569,0.466667,0.705882}%
\pgfsetfillcolor{currentfill}%
\pgfsetfillopacity{0.489554}%
\pgfsetlinewidth{1.003750pt}%
\definecolor{currentstroke}{rgb}{0.121569,0.466667,0.705882}%
\pgfsetstrokecolor{currentstroke}%
\pgfsetstrokeopacity{0.489554}%
\pgfsetdash{}{0pt}%
\pgfpathmoveto{\pgfqpoint{1.331473in}{2.119310in}}%
\pgfpathcurveto{\pgfqpoint{1.339709in}{2.119310in}}{\pgfqpoint{1.347609in}{2.122582in}}{\pgfqpoint{1.353433in}{2.128406in}}%
\pgfpathcurveto{\pgfqpoint{1.359257in}{2.134230in}}{\pgfqpoint{1.362530in}{2.142130in}}{\pgfqpoint{1.362530in}{2.150366in}}%
\pgfpathcurveto{\pgfqpoint{1.362530in}{2.158603in}}{\pgfqpoint{1.359257in}{2.166503in}}{\pgfqpoint{1.353433in}{2.172327in}}%
\pgfpathcurveto{\pgfqpoint{1.347609in}{2.178151in}}{\pgfqpoint{1.339709in}{2.181423in}}{\pgfqpoint{1.331473in}{2.181423in}}%
\pgfpathcurveto{\pgfqpoint{1.323237in}{2.181423in}}{\pgfqpoint{1.315337in}{2.178151in}}{\pgfqpoint{1.309513in}{2.172327in}}%
\pgfpathcurveto{\pgfqpoint{1.303689in}{2.166503in}}{\pgfqpoint{1.300417in}{2.158603in}}{\pgfqpoint{1.300417in}{2.150366in}}%
\pgfpathcurveto{\pgfqpoint{1.300417in}{2.142130in}}{\pgfqpoint{1.303689in}{2.134230in}}{\pgfqpoint{1.309513in}{2.128406in}}%
\pgfpathcurveto{\pgfqpoint{1.315337in}{2.122582in}}{\pgfqpoint{1.323237in}{2.119310in}}{\pgfqpoint{1.331473in}{2.119310in}}%
\pgfpathclose%
\pgfusepath{stroke,fill}%
\end{pgfscope}%
\begin{pgfscope}%
\pgfpathrectangle{\pgfqpoint{0.100000in}{0.212622in}}{\pgfqpoint{3.696000in}{3.696000in}}%
\pgfusepath{clip}%
\pgfsetbuttcap%
\pgfsetroundjoin%
\definecolor{currentfill}{rgb}{0.121569,0.466667,0.705882}%
\pgfsetfillcolor{currentfill}%
\pgfsetfillopacity{0.489644}%
\pgfsetlinewidth{1.003750pt}%
\definecolor{currentstroke}{rgb}{0.121569,0.466667,0.705882}%
\pgfsetstrokecolor{currentstroke}%
\pgfsetstrokeopacity{0.489644}%
\pgfsetdash{}{0pt}%
\pgfpathmoveto{\pgfqpoint{3.120582in}{1.794195in}}%
\pgfpathcurveto{\pgfqpoint{3.128818in}{1.794195in}}{\pgfqpoint{3.136718in}{1.797468in}}{\pgfqpoint{3.142542in}{1.803292in}}%
\pgfpathcurveto{\pgfqpoint{3.148366in}{1.809116in}}{\pgfqpoint{3.151639in}{1.817016in}}{\pgfqpoint{3.151639in}{1.825252in}}%
\pgfpathcurveto{\pgfqpoint{3.151639in}{1.833488in}}{\pgfqpoint{3.148366in}{1.841388in}}{\pgfqpoint{3.142542in}{1.847212in}}%
\pgfpathcurveto{\pgfqpoint{3.136718in}{1.853036in}}{\pgfqpoint{3.128818in}{1.856308in}}{\pgfqpoint{3.120582in}{1.856308in}}%
\pgfpathcurveto{\pgfqpoint{3.112346in}{1.856308in}}{\pgfqpoint{3.104446in}{1.853036in}}{\pgfqpoint{3.098622in}{1.847212in}}%
\pgfpathcurveto{\pgfqpoint{3.092798in}{1.841388in}}{\pgfqpoint{3.089526in}{1.833488in}}{\pgfqpoint{3.089526in}{1.825252in}}%
\pgfpathcurveto{\pgfqpoint{3.089526in}{1.817016in}}{\pgfqpoint{3.092798in}{1.809116in}}{\pgfqpoint{3.098622in}{1.803292in}}%
\pgfpathcurveto{\pgfqpoint{3.104446in}{1.797468in}}{\pgfqpoint{3.112346in}{1.794195in}}{\pgfqpoint{3.120582in}{1.794195in}}%
\pgfpathclose%
\pgfusepath{stroke,fill}%
\end{pgfscope}%
\begin{pgfscope}%
\pgfpathrectangle{\pgfqpoint{0.100000in}{0.212622in}}{\pgfqpoint{3.696000in}{3.696000in}}%
\pgfusepath{clip}%
\pgfsetbuttcap%
\pgfsetroundjoin%
\definecolor{currentfill}{rgb}{0.121569,0.466667,0.705882}%
\pgfsetfillcolor{currentfill}%
\pgfsetfillopacity{0.490271}%
\pgfsetlinewidth{1.003750pt}%
\definecolor{currentstroke}{rgb}{0.121569,0.466667,0.705882}%
\pgfsetstrokecolor{currentstroke}%
\pgfsetstrokeopacity{0.490271}%
\pgfsetdash{}{0pt}%
\pgfpathmoveto{\pgfqpoint{3.125255in}{1.793272in}}%
\pgfpathcurveto{\pgfqpoint{3.133491in}{1.793272in}}{\pgfqpoint{3.141391in}{1.796544in}}{\pgfqpoint{3.147215in}{1.802368in}}%
\pgfpathcurveto{\pgfqpoint{3.153039in}{1.808192in}}{\pgfqpoint{3.156311in}{1.816092in}}{\pgfqpoint{3.156311in}{1.824329in}}%
\pgfpathcurveto{\pgfqpoint{3.156311in}{1.832565in}}{\pgfqpoint{3.153039in}{1.840465in}}{\pgfqpoint{3.147215in}{1.846289in}}%
\pgfpathcurveto{\pgfqpoint{3.141391in}{1.852113in}}{\pgfqpoint{3.133491in}{1.855385in}}{\pgfqpoint{3.125255in}{1.855385in}}%
\pgfpathcurveto{\pgfqpoint{3.117018in}{1.855385in}}{\pgfqpoint{3.109118in}{1.852113in}}{\pgfqpoint{3.103294in}{1.846289in}}%
\pgfpathcurveto{\pgfqpoint{3.097470in}{1.840465in}}{\pgfqpoint{3.094198in}{1.832565in}}{\pgfqpoint{3.094198in}{1.824329in}}%
\pgfpathcurveto{\pgfqpoint{3.094198in}{1.816092in}}{\pgfqpoint{3.097470in}{1.808192in}}{\pgfqpoint{3.103294in}{1.802368in}}%
\pgfpathcurveto{\pgfqpoint{3.109118in}{1.796544in}}{\pgfqpoint{3.117018in}{1.793272in}}{\pgfqpoint{3.125255in}{1.793272in}}%
\pgfpathclose%
\pgfusepath{stroke,fill}%
\end{pgfscope}%
\begin{pgfscope}%
\pgfpathrectangle{\pgfqpoint{0.100000in}{0.212622in}}{\pgfqpoint{3.696000in}{3.696000in}}%
\pgfusepath{clip}%
\pgfsetbuttcap%
\pgfsetroundjoin%
\definecolor{currentfill}{rgb}{0.121569,0.466667,0.705882}%
\pgfsetfillcolor{currentfill}%
\pgfsetfillopacity{0.491060}%
\pgfsetlinewidth{1.003750pt}%
\definecolor{currentstroke}{rgb}{0.121569,0.466667,0.705882}%
\pgfsetstrokecolor{currentstroke}%
\pgfsetstrokeopacity{0.491060}%
\pgfsetdash{}{0pt}%
\pgfpathmoveto{\pgfqpoint{3.130904in}{1.792339in}}%
\pgfpathcurveto{\pgfqpoint{3.139140in}{1.792339in}}{\pgfqpoint{3.147040in}{1.795611in}}{\pgfqpoint{3.152864in}{1.801435in}}%
\pgfpathcurveto{\pgfqpoint{3.158688in}{1.807259in}}{\pgfqpoint{3.161960in}{1.815159in}}{\pgfqpoint{3.161960in}{1.823395in}}%
\pgfpathcurveto{\pgfqpoint{3.161960in}{1.831631in}}{\pgfqpoint{3.158688in}{1.839532in}}{\pgfqpoint{3.152864in}{1.845355in}}%
\pgfpathcurveto{\pgfqpoint{3.147040in}{1.851179in}}{\pgfqpoint{3.139140in}{1.854452in}}{\pgfqpoint{3.130904in}{1.854452in}}%
\pgfpathcurveto{\pgfqpoint{3.122668in}{1.854452in}}{\pgfqpoint{3.114768in}{1.851179in}}{\pgfqpoint{3.108944in}{1.845355in}}%
\pgfpathcurveto{\pgfqpoint{3.103120in}{1.839532in}}{\pgfqpoint{3.099847in}{1.831631in}}{\pgfqpoint{3.099847in}{1.823395in}}%
\pgfpathcurveto{\pgfqpoint{3.099847in}{1.815159in}}{\pgfqpoint{3.103120in}{1.807259in}}{\pgfqpoint{3.108944in}{1.801435in}}%
\pgfpathcurveto{\pgfqpoint{3.114768in}{1.795611in}}{\pgfqpoint{3.122668in}{1.792339in}}{\pgfqpoint{3.130904in}{1.792339in}}%
\pgfpathclose%
\pgfusepath{stroke,fill}%
\end{pgfscope}%
\begin{pgfscope}%
\pgfpathrectangle{\pgfqpoint{0.100000in}{0.212622in}}{\pgfqpoint{3.696000in}{3.696000in}}%
\pgfusepath{clip}%
\pgfsetbuttcap%
\pgfsetroundjoin%
\definecolor{currentfill}{rgb}{0.121569,0.466667,0.705882}%
\pgfsetfillcolor{currentfill}%
\pgfsetfillopacity{0.491076}%
\pgfsetlinewidth{1.003750pt}%
\definecolor{currentstroke}{rgb}{0.121569,0.466667,0.705882}%
\pgfsetstrokecolor{currentstroke}%
\pgfsetstrokeopacity{0.491076}%
\pgfsetdash{}{0pt}%
\pgfpathmoveto{\pgfqpoint{1.329111in}{2.119505in}}%
\pgfpathcurveto{\pgfqpoint{1.337347in}{2.119505in}}{\pgfqpoint{1.345247in}{2.122777in}}{\pgfqpoint{1.351071in}{2.128601in}}%
\pgfpathcurveto{\pgfqpoint{1.356895in}{2.134425in}}{\pgfqpoint{1.360167in}{2.142325in}}{\pgfqpoint{1.360167in}{2.150561in}}%
\pgfpathcurveto{\pgfqpoint{1.360167in}{2.158797in}}{\pgfqpoint{1.356895in}{2.166697in}}{\pgfqpoint{1.351071in}{2.172521in}}%
\pgfpathcurveto{\pgfqpoint{1.345247in}{2.178345in}}{\pgfqpoint{1.337347in}{2.181618in}}{\pgfqpoint{1.329111in}{2.181618in}}%
\pgfpathcurveto{\pgfqpoint{1.320874in}{2.181618in}}{\pgfqpoint{1.312974in}{2.178345in}}{\pgfqpoint{1.307150in}{2.172521in}}%
\pgfpathcurveto{\pgfqpoint{1.301326in}{2.166697in}}{\pgfqpoint{1.298054in}{2.158797in}}{\pgfqpoint{1.298054in}{2.150561in}}%
\pgfpathcurveto{\pgfqpoint{1.298054in}{2.142325in}}{\pgfqpoint{1.301326in}{2.134425in}}{\pgfqpoint{1.307150in}{2.128601in}}%
\pgfpathcurveto{\pgfqpoint{1.312974in}{2.122777in}}{\pgfqpoint{1.320874in}{2.119505in}}{\pgfqpoint{1.329111in}{2.119505in}}%
\pgfpathclose%
\pgfusepath{stroke,fill}%
\end{pgfscope}%
\begin{pgfscope}%
\pgfpathrectangle{\pgfqpoint{0.100000in}{0.212622in}}{\pgfqpoint{3.696000in}{3.696000in}}%
\pgfusepath{clip}%
\pgfsetbuttcap%
\pgfsetroundjoin%
\definecolor{currentfill}{rgb}{0.121569,0.466667,0.705882}%
\pgfsetfillcolor{currentfill}%
\pgfsetfillopacity{0.491626}%
\pgfsetlinewidth{1.003750pt}%
\definecolor{currentstroke}{rgb}{0.121569,0.466667,0.705882}%
\pgfsetstrokecolor{currentstroke}%
\pgfsetstrokeopacity{0.491626}%
\pgfsetdash{}{0pt}%
\pgfpathmoveto{\pgfqpoint{3.137639in}{1.790699in}}%
\pgfpathcurveto{\pgfqpoint{3.145875in}{1.790699in}}{\pgfqpoint{3.153775in}{1.793971in}}{\pgfqpoint{3.159599in}{1.799795in}}%
\pgfpathcurveto{\pgfqpoint{3.165423in}{1.805619in}}{\pgfqpoint{3.168695in}{1.813519in}}{\pgfqpoint{3.168695in}{1.821755in}}%
\pgfpathcurveto{\pgfqpoint{3.168695in}{1.829992in}}{\pgfqpoint{3.165423in}{1.837892in}}{\pgfqpoint{3.159599in}{1.843716in}}%
\pgfpathcurveto{\pgfqpoint{3.153775in}{1.849540in}}{\pgfqpoint{3.145875in}{1.852812in}}{\pgfqpoint{3.137639in}{1.852812in}}%
\pgfpathcurveto{\pgfqpoint{3.129403in}{1.852812in}}{\pgfqpoint{3.121503in}{1.849540in}}{\pgfqpoint{3.115679in}{1.843716in}}%
\pgfpathcurveto{\pgfqpoint{3.109855in}{1.837892in}}{\pgfqpoint{3.106582in}{1.829992in}}{\pgfqpoint{3.106582in}{1.821755in}}%
\pgfpathcurveto{\pgfqpoint{3.106582in}{1.813519in}}{\pgfqpoint{3.109855in}{1.805619in}}{\pgfqpoint{3.115679in}{1.799795in}}%
\pgfpathcurveto{\pgfqpoint{3.121503in}{1.793971in}}{\pgfqpoint{3.129403in}{1.790699in}}{\pgfqpoint{3.137639in}{1.790699in}}%
\pgfpathclose%
\pgfusepath{stroke,fill}%
\end{pgfscope}%
\begin{pgfscope}%
\pgfpathrectangle{\pgfqpoint{0.100000in}{0.212622in}}{\pgfqpoint{3.696000in}{3.696000in}}%
\pgfusepath{clip}%
\pgfsetbuttcap%
\pgfsetroundjoin%
\definecolor{currentfill}{rgb}{0.121569,0.466667,0.705882}%
\pgfsetfillcolor{currentfill}%
\pgfsetfillopacity{0.492088}%
\pgfsetlinewidth{1.003750pt}%
\definecolor{currentstroke}{rgb}{0.121569,0.466667,0.705882}%
\pgfsetstrokecolor{currentstroke}%
\pgfsetstrokeopacity{0.492088}%
\pgfsetdash{}{0pt}%
\pgfpathmoveto{\pgfqpoint{1.326468in}{2.119746in}}%
\pgfpathcurveto{\pgfqpoint{1.334704in}{2.119746in}}{\pgfqpoint{1.342604in}{2.123018in}}{\pgfqpoint{1.348428in}{2.128842in}}%
\pgfpathcurveto{\pgfqpoint{1.354252in}{2.134666in}}{\pgfqpoint{1.357525in}{2.142566in}}{\pgfqpoint{1.357525in}{2.150802in}}%
\pgfpathcurveto{\pgfqpoint{1.357525in}{2.159039in}}{\pgfqpoint{1.354252in}{2.166939in}}{\pgfqpoint{1.348428in}{2.172763in}}%
\pgfpathcurveto{\pgfqpoint{1.342604in}{2.178586in}}{\pgfqpoint{1.334704in}{2.181859in}}{\pgfqpoint{1.326468in}{2.181859in}}%
\pgfpathcurveto{\pgfqpoint{1.318232in}{2.181859in}}{\pgfqpoint{1.310332in}{2.178586in}}{\pgfqpoint{1.304508in}{2.172763in}}%
\pgfpathcurveto{\pgfqpoint{1.298684in}{2.166939in}}{\pgfqpoint{1.295412in}{2.159039in}}{\pgfqpoint{1.295412in}{2.150802in}}%
\pgfpathcurveto{\pgfqpoint{1.295412in}{2.142566in}}{\pgfqpoint{1.298684in}{2.134666in}}{\pgfqpoint{1.304508in}{2.128842in}}%
\pgfpathcurveto{\pgfqpoint{1.310332in}{2.123018in}}{\pgfqpoint{1.318232in}{2.119746in}}{\pgfqpoint{1.326468in}{2.119746in}}%
\pgfpathclose%
\pgfusepath{stroke,fill}%
\end{pgfscope}%
\begin{pgfscope}%
\pgfpathrectangle{\pgfqpoint{0.100000in}{0.212622in}}{\pgfqpoint{3.696000in}{3.696000in}}%
\pgfusepath{clip}%
\pgfsetbuttcap%
\pgfsetroundjoin%
\definecolor{currentfill}{rgb}{0.121569,0.466667,0.705882}%
\pgfsetfillcolor{currentfill}%
\pgfsetfillopacity{0.492200}%
\pgfsetlinewidth{1.003750pt}%
\definecolor{currentstroke}{rgb}{0.121569,0.466667,0.705882}%
\pgfsetstrokecolor{currentstroke}%
\pgfsetstrokeopacity{0.492200}%
\pgfsetdash{}{0pt}%
\pgfpathmoveto{\pgfqpoint{3.140904in}{1.790115in}}%
\pgfpathcurveto{\pgfqpoint{3.149140in}{1.790115in}}{\pgfqpoint{3.157040in}{1.793388in}}{\pgfqpoint{3.162864in}{1.799211in}}%
\pgfpathcurveto{\pgfqpoint{3.168688in}{1.805035in}}{\pgfqpoint{3.171960in}{1.812935in}}{\pgfqpoint{3.171960in}{1.821172in}}%
\pgfpathcurveto{\pgfqpoint{3.171960in}{1.829408in}}{\pgfqpoint{3.168688in}{1.837308in}}{\pgfqpoint{3.162864in}{1.843132in}}%
\pgfpathcurveto{\pgfqpoint{3.157040in}{1.848956in}}{\pgfqpoint{3.149140in}{1.852228in}}{\pgfqpoint{3.140904in}{1.852228in}}%
\pgfpathcurveto{\pgfqpoint{3.132667in}{1.852228in}}{\pgfqpoint{3.124767in}{1.848956in}}{\pgfqpoint{3.118943in}{1.843132in}}%
\pgfpathcurveto{\pgfqpoint{3.113119in}{1.837308in}}{\pgfqpoint{3.109847in}{1.829408in}}{\pgfqpoint{3.109847in}{1.821172in}}%
\pgfpathcurveto{\pgfqpoint{3.109847in}{1.812935in}}{\pgfqpoint{3.113119in}{1.805035in}}{\pgfqpoint{3.118943in}{1.799211in}}%
\pgfpathcurveto{\pgfqpoint{3.124767in}{1.793388in}}{\pgfqpoint{3.132667in}{1.790115in}}{\pgfqpoint{3.140904in}{1.790115in}}%
\pgfpathclose%
\pgfusepath{stroke,fill}%
\end{pgfscope}%
\begin{pgfscope}%
\pgfpathrectangle{\pgfqpoint{0.100000in}{0.212622in}}{\pgfqpoint{3.696000in}{3.696000in}}%
\pgfusepath{clip}%
\pgfsetbuttcap%
\pgfsetroundjoin%
\definecolor{currentfill}{rgb}{0.121569,0.466667,0.705882}%
\pgfsetfillcolor{currentfill}%
\pgfsetfillopacity{0.493051}%
\pgfsetlinewidth{1.003750pt}%
\definecolor{currentstroke}{rgb}{0.121569,0.466667,0.705882}%
\pgfsetstrokecolor{currentstroke}%
\pgfsetstrokeopacity{0.493051}%
\pgfsetdash{}{0pt}%
\pgfpathmoveto{\pgfqpoint{3.145275in}{1.789546in}}%
\pgfpathcurveto{\pgfqpoint{3.153511in}{1.789546in}}{\pgfqpoint{3.161411in}{1.792819in}}{\pgfqpoint{3.167235in}{1.798643in}}%
\pgfpathcurveto{\pgfqpoint{3.173059in}{1.804467in}}{\pgfqpoint{3.176331in}{1.812367in}}{\pgfqpoint{3.176331in}{1.820603in}}%
\pgfpathcurveto{\pgfqpoint{3.176331in}{1.828839in}}{\pgfqpoint{3.173059in}{1.836739in}}{\pgfqpoint{3.167235in}{1.842563in}}%
\pgfpathcurveto{\pgfqpoint{3.161411in}{1.848387in}}{\pgfqpoint{3.153511in}{1.851659in}}{\pgfqpoint{3.145275in}{1.851659in}}%
\pgfpathcurveto{\pgfqpoint{3.137038in}{1.851659in}}{\pgfqpoint{3.129138in}{1.848387in}}{\pgfqpoint{3.123314in}{1.842563in}}%
\pgfpathcurveto{\pgfqpoint{3.117491in}{1.836739in}}{\pgfqpoint{3.114218in}{1.828839in}}{\pgfqpoint{3.114218in}{1.820603in}}%
\pgfpathcurveto{\pgfqpoint{3.114218in}{1.812367in}}{\pgfqpoint{3.117491in}{1.804467in}}{\pgfqpoint{3.123314in}{1.798643in}}%
\pgfpathcurveto{\pgfqpoint{3.129138in}{1.792819in}}{\pgfqpoint{3.137038in}{1.789546in}}{\pgfqpoint{3.145275in}{1.789546in}}%
\pgfpathclose%
\pgfusepath{stroke,fill}%
\end{pgfscope}%
\begin{pgfscope}%
\pgfpathrectangle{\pgfqpoint{0.100000in}{0.212622in}}{\pgfqpoint{3.696000in}{3.696000in}}%
\pgfusepath{clip}%
\pgfsetbuttcap%
\pgfsetroundjoin%
\definecolor{currentfill}{rgb}{0.121569,0.466667,0.705882}%
\pgfsetfillcolor{currentfill}%
\pgfsetfillopacity{0.493440}%
\pgfsetlinewidth{1.003750pt}%
\definecolor{currentstroke}{rgb}{0.121569,0.466667,0.705882}%
\pgfsetstrokecolor{currentstroke}%
\pgfsetstrokeopacity{0.493440}%
\pgfsetdash{}{0pt}%
\pgfpathmoveto{\pgfqpoint{3.150878in}{1.788254in}}%
\pgfpathcurveto{\pgfqpoint{3.159114in}{1.788254in}}{\pgfqpoint{3.167014in}{1.791526in}}{\pgfqpoint{3.172838in}{1.797350in}}%
\pgfpathcurveto{\pgfqpoint{3.178662in}{1.803174in}}{\pgfqpoint{3.181934in}{1.811074in}}{\pgfqpoint{3.181934in}{1.819310in}}%
\pgfpathcurveto{\pgfqpoint{3.181934in}{1.827547in}}{\pgfqpoint{3.178662in}{1.835447in}}{\pgfqpoint{3.172838in}{1.841270in}}%
\pgfpathcurveto{\pgfqpoint{3.167014in}{1.847094in}}{\pgfqpoint{3.159114in}{1.850367in}}{\pgfqpoint{3.150878in}{1.850367in}}%
\pgfpathcurveto{\pgfqpoint{3.142641in}{1.850367in}}{\pgfqpoint{3.134741in}{1.847094in}}{\pgfqpoint{3.128917in}{1.841270in}}%
\pgfpathcurveto{\pgfqpoint{3.123093in}{1.835447in}}{\pgfqpoint{3.119821in}{1.827547in}}{\pgfqpoint{3.119821in}{1.819310in}}%
\pgfpathcurveto{\pgfqpoint{3.119821in}{1.811074in}}{\pgfqpoint{3.123093in}{1.803174in}}{\pgfqpoint{3.128917in}{1.797350in}}%
\pgfpathcurveto{\pgfqpoint{3.134741in}{1.791526in}}{\pgfqpoint{3.142641in}{1.788254in}}{\pgfqpoint{3.150878in}{1.788254in}}%
\pgfpathclose%
\pgfusepath{stroke,fill}%
\end{pgfscope}%
\begin{pgfscope}%
\pgfpathrectangle{\pgfqpoint{0.100000in}{0.212622in}}{\pgfqpoint{3.696000in}{3.696000in}}%
\pgfusepath{clip}%
\pgfsetbuttcap%
\pgfsetroundjoin%
\definecolor{currentfill}{rgb}{0.121569,0.466667,0.705882}%
\pgfsetfillcolor{currentfill}%
\pgfsetfillopacity{0.493583}%
\pgfsetlinewidth{1.003750pt}%
\definecolor{currentstroke}{rgb}{0.121569,0.466667,0.705882}%
\pgfsetstrokecolor{currentstroke}%
\pgfsetstrokeopacity{0.493583}%
\pgfsetdash{}{0pt}%
\pgfpathmoveto{\pgfqpoint{3.153984in}{1.787231in}}%
\pgfpathcurveto{\pgfqpoint{3.162220in}{1.787231in}}{\pgfqpoint{3.170120in}{1.790503in}}{\pgfqpoint{3.175944in}{1.796327in}}%
\pgfpathcurveto{\pgfqpoint{3.181768in}{1.802151in}}{\pgfqpoint{3.185040in}{1.810051in}}{\pgfqpoint{3.185040in}{1.818287in}}%
\pgfpathcurveto{\pgfqpoint{3.185040in}{1.826524in}}{\pgfqpoint{3.181768in}{1.834424in}}{\pgfqpoint{3.175944in}{1.840248in}}%
\pgfpathcurveto{\pgfqpoint{3.170120in}{1.846072in}}{\pgfqpoint{3.162220in}{1.849344in}}{\pgfqpoint{3.153984in}{1.849344in}}%
\pgfpathcurveto{\pgfqpoint{3.145747in}{1.849344in}}{\pgfqpoint{3.137847in}{1.846072in}}{\pgfqpoint{3.132023in}{1.840248in}}%
\pgfpathcurveto{\pgfqpoint{3.126199in}{1.834424in}}{\pgfqpoint{3.122927in}{1.826524in}}{\pgfqpoint{3.122927in}{1.818287in}}%
\pgfpathcurveto{\pgfqpoint{3.122927in}{1.810051in}}{\pgfqpoint{3.126199in}{1.802151in}}{\pgfqpoint{3.132023in}{1.796327in}}%
\pgfpathcurveto{\pgfqpoint{3.137847in}{1.790503in}}{\pgfqpoint{3.145747in}{1.787231in}}{\pgfqpoint{3.153984in}{1.787231in}}%
\pgfpathclose%
\pgfusepath{stroke,fill}%
\end{pgfscope}%
\begin{pgfscope}%
\pgfpathrectangle{\pgfqpoint{0.100000in}{0.212622in}}{\pgfqpoint{3.696000in}{3.696000in}}%
\pgfusepath{clip}%
\pgfsetbuttcap%
\pgfsetroundjoin%
\definecolor{currentfill}{rgb}{0.121569,0.466667,0.705882}%
\pgfsetfillcolor{currentfill}%
\pgfsetfillopacity{0.494097}%
\pgfsetlinewidth{1.003750pt}%
\definecolor{currentstroke}{rgb}{0.121569,0.466667,0.705882}%
\pgfsetstrokecolor{currentstroke}%
\pgfsetstrokeopacity{0.494097}%
\pgfsetdash{}{0pt}%
\pgfpathmoveto{\pgfqpoint{1.322987in}{2.120028in}}%
\pgfpathcurveto{\pgfqpoint{1.331223in}{2.120028in}}{\pgfqpoint{1.339123in}{2.123301in}}{\pgfqpoint{1.344947in}{2.129125in}}%
\pgfpathcurveto{\pgfqpoint{1.350771in}{2.134948in}}{\pgfqpoint{1.354043in}{2.142849in}}{\pgfqpoint{1.354043in}{2.151085in}}%
\pgfpathcurveto{\pgfqpoint{1.354043in}{2.159321in}}{\pgfqpoint{1.350771in}{2.167221in}}{\pgfqpoint{1.344947in}{2.173045in}}%
\pgfpathcurveto{\pgfqpoint{1.339123in}{2.178869in}}{\pgfqpoint{1.331223in}{2.182141in}}{\pgfqpoint{1.322987in}{2.182141in}}%
\pgfpathcurveto{\pgfqpoint{1.314751in}{2.182141in}}{\pgfqpoint{1.306851in}{2.178869in}}{\pgfqpoint{1.301027in}{2.173045in}}%
\pgfpathcurveto{\pgfqpoint{1.295203in}{2.167221in}}{\pgfqpoint{1.291930in}{2.159321in}}{\pgfqpoint{1.291930in}{2.151085in}}%
\pgfpathcurveto{\pgfqpoint{1.291930in}{2.142849in}}{\pgfqpoint{1.295203in}{2.134948in}}{\pgfqpoint{1.301027in}{2.129125in}}%
\pgfpathcurveto{\pgfqpoint{1.306851in}{2.123301in}}{\pgfqpoint{1.314751in}{2.120028in}}{\pgfqpoint{1.322987in}{2.120028in}}%
\pgfpathclose%
\pgfusepath{stroke,fill}%
\end{pgfscope}%
\begin{pgfscope}%
\pgfpathrectangle{\pgfqpoint{0.100000in}{0.212622in}}{\pgfqpoint{3.696000in}{3.696000in}}%
\pgfusepath{clip}%
\pgfsetbuttcap%
\pgfsetroundjoin%
\definecolor{currentfill}{rgb}{0.121569,0.466667,0.705882}%
\pgfsetfillcolor{currentfill}%
\pgfsetfillopacity{0.494267}%
\pgfsetlinewidth{1.003750pt}%
\definecolor{currentstroke}{rgb}{0.121569,0.466667,0.705882}%
\pgfsetstrokecolor{currentstroke}%
\pgfsetstrokeopacity{0.494267}%
\pgfsetdash{}{0pt}%
\pgfpathmoveto{\pgfqpoint{3.156755in}{1.787094in}}%
\pgfpathcurveto{\pgfqpoint{3.164991in}{1.787094in}}{\pgfqpoint{3.172891in}{1.790367in}}{\pgfqpoint{3.178715in}{1.796191in}}%
\pgfpathcurveto{\pgfqpoint{3.184539in}{1.802014in}}{\pgfqpoint{3.187812in}{1.809915in}}{\pgfqpoint{3.187812in}{1.818151in}}%
\pgfpathcurveto{\pgfqpoint{3.187812in}{1.826387in}}{\pgfqpoint{3.184539in}{1.834287in}}{\pgfqpoint{3.178715in}{1.840111in}}%
\pgfpathcurveto{\pgfqpoint{3.172891in}{1.845935in}}{\pgfqpoint{3.164991in}{1.849207in}}{\pgfqpoint{3.156755in}{1.849207in}}%
\pgfpathcurveto{\pgfqpoint{3.148519in}{1.849207in}}{\pgfqpoint{3.140619in}{1.845935in}}{\pgfqpoint{3.134795in}{1.840111in}}%
\pgfpathcurveto{\pgfqpoint{3.128971in}{1.834287in}}{\pgfqpoint{3.125699in}{1.826387in}}{\pgfqpoint{3.125699in}{1.818151in}}%
\pgfpathcurveto{\pgfqpoint{3.125699in}{1.809915in}}{\pgfqpoint{3.128971in}{1.802014in}}{\pgfqpoint{3.134795in}{1.796191in}}%
\pgfpathcurveto{\pgfqpoint{3.140619in}{1.790367in}}{\pgfqpoint{3.148519in}{1.787094in}}{\pgfqpoint{3.156755in}{1.787094in}}%
\pgfpathclose%
\pgfusepath{stroke,fill}%
\end{pgfscope}%
\begin{pgfscope}%
\pgfpathrectangle{\pgfqpoint{0.100000in}{0.212622in}}{\pgfqpoint{3.696000in}{3.696000in}}%
\pgfusepath{clip}%
\pgfsetbuttcap%
\pgfsetroundjoin%
\definecolor{currentfill}{rgb}{0.121569,0.466667,0.705882}%
\pgfsetfillcolor{currentfill}%
\pgfsetfillopacity{0.494418}%
\pgfsetlinewidth{1.003750pt}%
\definecolor{currentstroke}{rgb}{0.121569,0.466667,0.705882}%
\pgfsetstrokecolor{currentstroke}%
\pgfsetstrokeopacity{0.494418}%
\pgfsetdash{}{0pt}%
\pgfpathmoveto{\pgfqpoint{3.161725in}{1.785684in}}%
\pgfpathcurveto{\pgfqpoint{3.169961in}{1.785684in}}{\pgfqpoint{3.177861in}{1.788956in}}{\pgfqpoint{3.183685in}{1.794780in}}%
\pgfpathcurveto{\pgfqpoint{3.189509in}{1.800604in}}{\pgfqpoint{3.192781in}{1.808504in}}{\pgfqpoint{3.192781in}{1.816740in}}%
\pgfpathcurveto{\pgfqpoint{3.192781in}{1.824976in}}{\pgfqpoint{3.189509in}{1.832876in}}{\pgfqpoint{3.183685in}{1.838700in}}%
\pgfpathcurveto{\pgfqpoint{3.177861in}{1.844524in}}{\pgfqpoint{3.169961in}{1.847797in}}{\pgfqpoint{3.161725in}{1.847797in}}%
\pgfpathcurveto{\pgfqpoint{3.153489in}{1.847797in}}{\pgfqpoint{3.145589in}{1.844524in}}{\pgfqpoint{3.139765in}{1.838700in}}%
\pgfpathcurveto{\pgfqpoint{3.133941in}{1.832876in}}{\pgfqpoint{3.130668in}{1.824976in}}{\pgfqpoint{3.130668in}{1.816740in}}%
\pgfpathcurveto{\pgfqpoint{3.130668in}{1.808504in}}{\pgfqpoint{3.133941in}{1.800604in}}{\pgfqpoint{3.139765in}{1.794780in}}%
\pgfpathcurveto{\pgfqpoint{3.145589in}{1.788956in}}{\pgfqpoint{3.153489in}{1.785684in}}{\pgfqpoint{3.161725in}{1.785684in}}%
\pgfpathclose%
\pgfusepath{stroke,fill}%
\end{pgfscope}%
\begin{pgfscope}%
\pgfpathrectangle{\pgfqpoint{0.100000in}{0.212622in}}{\pgfqpoint{3.696000in}{3.696000in}}%
\pgfusepath{clip}%
\pgfsetbuttcap%
\pgfsetroundjoin%
\definecolor{currentfill}{rgb}{0.121569,0.466667,0.705882}%
\pgfsetfillcolor{currentfill}%
\pgfsetfillopacity{0.495002}%
\pgfsetlinewidth{1.003750pt}%
\definecolor{currentstroke}{rgb}{0.121569,0.466667,0.705882}%
\pgfsetstrokecolor{currentstroke}%
\pgfsetstrokeopacity{0.495002}%
\pgfsetdash{}{0pt}%
\pgfpathmoveto{\pgfqpoint{3.167285in}{1.784445in}}%
\pgfpathcurveto{\pgfqpoint{3.175521in}{1.784445in}}{\pgfqpoint{3.183422in}{1.787717in}}{\pgfqpoint{3.189245in}{1.793541in}}%
\pgfpathcurveto{\pgfqpoint{3.195069in}{1.799365in}}{\pgfqpoint{3.198342in}{1.807265in}}{\pgfqpoint{3.198342in}{1.815502in}}%
\pgfpathcurveto{\pgfqpoint{3.198342in}{1.823738in}}{\pgfqpoint{3.195069in}{1.831638in}}{\pgfqpoint{3.189245in}{1.837462in}}%
\pgfpathcurveto{\pgfqpoint{3.183422in}{1.843286in}}{\pgfqpoint{3.175521in}{1.846558in}}{\pgfqpoint{3.167285in}{1.846558in}}%
\pgfpathcurveto{\pgfqpoint{3.159049in}{1.846558in}}{\pgfqpoint{3.151149in}{1.843286in}}{\pgfqpoint{3.145325in}{1.837462in}}%
\pgfpathcurveto{\pgfqpoint{3.139501in}{1.831638in}}{\pgfqpoint{3.136229in}{1.823738in}}{\pgfqpoint{3.136229in}{1.815502in}}%
\pgfpathcurveto{\pgfqpoint{3.136229in}{1.807265in}}{\pgfqpoint{3.139501in}{1.799365in}}{\pgfqpoint{3.145325in}{1.793541in}}%
\pgfpathcurveto{\pgfqpoint{3.151149in}{1.787717in}}{\pgfqpoint{3.159049in}{1.784445in}}{\pgfqpoint{3.167285in}{1.784445in}}%
\pgfpathclose%
\pgfusepath{stroke,fill}%
\end{pgfscope}%
\begin{pgfscope}%
\pgfpathrectangle{\pgfqpoint{0.100000in}{0.212622in}}{\pgfqpoint{3.696000in}{3.696000in}}%
\pgfusepath{clip}%
\pgfsetbuttcap%
\pgfsetroundjoin%
\definecolor{currentfill}{rgb}{0.121569,0.466667,0.705882}%
\pgfsetfillcolor{currentfill}%
\pgfsetfillopacity{0.495342}%
\pgfsetlinewidth{1.003750pt}%
\definecolor{currentstroke}{rgb}{0.121569,0.466667,0.705882}%
\pgfsetstrokecolor{currentstroke}%
\pgfsetstrokeopacity{0.495342}%
\pgfsetdash{}{0pt}%
\pgfpathmoveto{\pgfqpoint{3.170298in}{1.783722in}}%
\pgfpathcurveto{\pgfqpoint{3.178534in}{1.783722in}}{\pgfqpoint{3.186435in}{1.786994in}}{\pgfqpoint{3.192258in}{1.792818in}}%
\pgfpathcurveto{\pgfqpoint{3.198082in}{1.798642in}}{\pgfqpoint{3.201355in}{1.806542in}}{\pgfqpoint{3.201355in}{1.814779in}}%
\pgfpathcurveto{\pgfqpoint{3.201355in}{1.823015in}}{\pgfqpoint{3.198082in}{1.830915in}}{\pgfqpoint{3.192258in}{1.836739in}}%
\pgfpathcurveto{\pgfqpoint{3.186435in}{1.842563in}}{\pgfqpoint{3.178534in}{1.845835in}}{\pgfqpoint{3.170298in}{1.845835in}}%
\pgfpathcurveto{\pgfqpoint{3.162062in}{1.845835in}}{\pgfqpoint{3.154162in}{1.842563in}}{\pgfqpoint{3.148338in}{1.836739in}}%
\pgfpathcurveto{\pgfqpoint{3.142514in}{1.830915in}}{\pgfqpoint{3.139242in}{1.823015in}}{\pgfqpoint{3.139242in}{1.814779in}}%
\pgfpathcurveto{\pgfqpoint{3.139242in}{1.806542in}}{\pgfqpoint{3.142514in}{1.798642in}}{\pgfqpoint{3.148338in}{1.792818in}}%
\pgfpathcurveto{\pgfqpoint{3.154162in}{1.786994in}}{\pgfqpoint{3.162062in}{1.783722in}}{\pgfqpoint{3.170298in}{1.783722in}}%
\pgfpathclose%
\pgfusepath{stroke,fill}%
\end{pgfscope}%
\begin{pgfscope}%
\pgfpathrectangle{\pgfqpoint{0.100000in}{0.212622in}}{\pgfqpoint{3.696000in}{3.696000in}}%
\pgfusepath{clip}%
\pgfsetbuttcap%
\pgfsetroundjoin%
\definecolor{currentfill}{rgb}{0.121569,0.466667,0.705882}%
\pgfsetfillcolor{currentfill}%
\pgfsetfillopacity{0.495605}%
\pgfsetlinewidth{1.003750pt}%
\definecolor{currentstroke}{rgb}{0.121569,0.466667,0.705882}%
\pgfsetstrokecolor{currentstroke}%
\pgfsetstrokeopacity{0.495605}%
\pgfsetdash{}{0pt}%
\pgfpathmoveto{\pgfqpoint{3.174682in}{1.782603in}}%
\pgfpathcurveto{\pgfqpoint{3.182918in}{1.782603in}}{\pgfqpoint{3.190819in}{1.785875in}}{\pgfqpoint{3.196642in}{1.791699in}}%
\pgfpathcurveto{\pgfqpoint{3.202466in}{1.797523in}}{\pgfqpoint{3.205739in}{1.805423in}}{\pgfqpoint{3.205739in}{1.813659in}}%
\pgfpathcurveto{\pgfqpoint{3.205739in}{1.821895in}}{\pgfqpoint{3.202466in}{1.829795in}}{\pgfqpoint{3.196642in}{1.835619in}}%
\pgfpathcurveto{\pgfqpoint{3.190819in}{1.841443in}}{\pgfqpoint{3.182918in}{1.844716in}}{\pgfqpoint{3.174682in}{1.844716in}}%
\pgfpathcurveto{\pgfqpoint{3.166446in}{1.844716in}}{\pgfqpoint{3.158546in}{1.841443in}}{\pgfqpoint{3.152722in}{1.835619in}}%
\pgfpathcurveto{\pgfqpoint{3.146898in}{1.829795in}}{\pgfqpoint{3.143626in}{1.821895in}}{\pgfqpoint{3.143626in}{1.813659in}}%
\pgfpathcurveto{\pgfqpoint{3.143626in}{1.805423in}}{\pgfqpoint{3.146898in}{1.797523in}}{\pgfqpoint{3.152722in}{1.791699in}}%
\pgfpathcurveto{\pgfqpoint{3.158546in}{1.785875in}}{\pgfqpoint{3.166446in}{1.782603in}}{\pgfqpoint{3.174682in}{1.782603in}}%
\pgfpathclose%
\pgfusepath{stroke,fill}%
\end{pgfscope}%
\begin{pgfscope}%
\pgfpathrectangle{\pgfqpoint{0.100000in}{0.212622in}}{\pgfqpoint{3.696000in}{3.696000in}}%
\pgfusepath{clip}%
\pgfsetbuttcap%
\pgfsetroundjoin%
\definecolor{currentfill}{rgb}{0.121569,0.466667,0.705882}%
\pgfsetfillcolor{currentfill}%
\pgfsetfillopacity{0.495913}%
\pgfsetlinewidth{1.003750pt}%
\definecolor{currentstroke}{rgb}{0.121569,0.466667,0.705882}%
\pgfsetstrokecolor{currentstroke}%
\pgfsetstrokeopacity{0.495913}%
\pgfsetdash{}{0pt}%
\pgfpathmoveto{\pgfqpoint{3.179744in}{1.781303in}}%
\pgfpathcurveto{\pgfqpoint{3.187981in}{1.781303in}}{\pgfqpoint{3.195881in}{1.784576in}}{\pgfqpoint{3.201705in}{1.790399in}}%
\pgfpathcurveto{\pgfqpoint{3.207529in}{1.796223in}}{\pgfqpoint{3.210801in}{1.804123in}}{\pgfqpoint{3.210801in}{1.812360in}}%
\pgfpathcurveto{\pgfqpoint{3.210801in}{1.820596in}}{\pgfqpoint{3.207529in}{1.828496in}}{\pgfqpoint{3.201705in}{1.834320in}}%
\pgfpathcurveto{\pgfqpoint{3.195881in}{1.840144in}}{\pgfqpoint{3.187981in}{1.843416in}}{\pgfqpoint{3.179744in}{1.843416in}}%
\pgfpathcurveto{\pgfqpoint{3.171508in}{1.843416in}}{\pgfqpoint{3.163608in}{1.840144in}}{\pgfqpoint{3.157784in}{1.834320in}}%
\pgfpathcurveto{\pgfqpoint{3.151960in}{1.828496in}}{\pgfqpoint{3.148688in}{1.820596in}}{\pgfqpoint{3.148688in}{1.812360in}}%
\pgfpathcurveto{\pgfqpoint{3.148688in}{1.804123in}}{\pgfqpoint{3.151960in}{1.796223in}}{\pgfqpoint{3.157784in}{1.790399in}}%
\pgfpathcurveto{\pgfqpoint{3.163608in}{1.784576in}}{\pgfqpoint{3.171508in}{1.781303in}}{\pgfqpoint{3.179744in}{1.781303in}}%
\pgfpathclose%
\pgfusepath{stroke,fill}%
\end{pgfscope}%
\begin{pgfscope}%
\pgfpathrectangle{\pgfqpoint{0.100000in}{0.212622in}}{\pgfqpoint{3.696000in}{3.696000in}}%
\pgfusepath{clip}%
\pgfsetbuttcap%
\pgfsetroundjoin%
\definecolor{currentfill}{rgb}{0.121569,0.466667,0.705882}%
\pgfsetfillcolor{currentfill}%
\pgfsetfillopacity{0.496239}%
\pgfsetlinewidth{1.003750pt}%
\definecolor{currentstroke}{rgb}{0.121569,0.466667,0.705882}%
\pgfsetstrokecolor{currentstroke}%
\pgfsetstrokeopacity{0.496239}%
\pgfsetdash{}{0pt}%
\pgfpathmoveto{\pgfqpoint{3.182335in}{1.780790in}}%
\pgfpathcurveto{\pgfqpoint{3.190571in}{1.780790in}}{\pgfqpoint{3.198471in}{1.784062in}}{\pgfqpoint{3.204295in}{1.789886in}}%
\pgfpathcurveto{\pgfqpoint{3.210119in}{1.795710in}}{\pgfqpoint{3.213391in}{1.803610in}}{\pgfqpoint{3.213391in}{1.811846in}}%
\pgfpathcurveto{\pgfqpoint{3.213391in}{1.820082in}}{\pgfqpoint{3.210119in}{1.827982in}}{\pgfqpoint{3.204295in}{1.833806in}}%
\pgfpathcurveto{\pgfqpoint{3.198471in}{1.839630in}}{\pgfqpoint{3.190571in}{1.842903in}}{\pgfqpoint{3.182335in}{1.842903in}}%
\pgfpathcurveto{\pgfqpoint{3.174098in}{1.842903in}}{\pgfqpoint{3.166198in}{1.839630in}}{\pgfqpoint{3.160374in}{1.833806in}}%
\pgfpathcurveto{\pgfqpoint{3.154550in}{1.827982in}}{\pgfqpoint{3.151278in}{1.820082in}}{\pgfqpoint{3.151278in}{1.811846in}}%
\pgfpathcurveto{\pgfqpoint{3.151278in}{1.803610in}}{\pgfqpoint{3.154550in}{1.795710in}}{\pgfqpoint{3.160374in}{1.789886in}}%
\pgfpathcurveto{\pgfqpoint{3.166198in}{1.784062in}}{\pgfqpoint{3.174098in}{1.780790in}}{\pgfqpoint{3.182335in}{1.780790in}}%
\pgfpathclose%
\pgfusepath{stroke,fill}%
\end{pgfscope}%
\begin{pgfscope}%
\pgfpathrectangle{\pgfqpoint{0.100000in}{0.212622in}}{\pgfqpoint{3.696000in}{3.696000in}}%
\pgfusepath{clip}%
\pgfsetbuttcap%
\pgfsetroundjoin%
\definecolor{currentfill}{rgb}{0.121569,0.466667,0.705882}%
\pgfsetfillcolor{currentfill}%
\pgfsetfillopacity{0.496521}%
\pgfsetlinewidth{1.003750pt}%
\definecolor{currentstroke}{rgb}{0.121569,0.466667,0.705882}%
\pgfsetstrokecolor{currentstroke}%
\pgfsetstrokeopacity{0.496521}%
\pgfsetdash{}{0pt}%
\pgfpathmoveto{\pgfqpoint{3.186556in}{1.779817in}}%
\pgfpathcurveto{\pgfqpoint{3.194792in}{1.779817in}}{\pgfqpoint{3.202692in}{1.783090in}}{\pgfqpoint{3.208516in}{1.788914in}}%
\pgfpathcurveto{\pgfqpoint{3.214340in}{1.794738in}}{\pgfqpoint{3.217612in}{1.802638in}}{\pgfqpoint{3.217612in}{1.810874in}}%
\pgfpathcurveto{\pgfqpoint{3.217612in}{1.819110in}}{\pgfqpoint{3.214340in}{1.827010in}}{\pgfqpoint{3.208516in}{1.832834in}}%
\pgfpathcurveto{\pgfqpoint{3.202692in}{1.838658in}}{\pgfqpoint{3.194792in}{1.841930in}}{\pgfqpoint{3.186556in}{1.841930in}}%
\pgfpathcurveto{\pgfqpoint{3.178320in}{1.841930in}}{\pgfqpoint{3.170420in}{1.838658in}}{\pgfqpoint{3.164596in}{1.832834in}}%
\pgfpathcurveto{\pgfqpoint{3.158772in}{1.827010in}}{\pgfqpoint{3.155499in}{1.819110in}}{\pgfqpoint{3.155499in}{1.810874in}}%
\pgfpathcurveto{\pgfqpoint{3.155499in}{1.802638in}}{\pgfqpoint{3.158772in}{1.794738in}}{\pgfqpoint{3.164596in}{1.788914in}}%
\pgfpathcurveto{\pgfqpoint{3.170420in}{1.783090in}}{\pgfqpoint{3.178320in}{1.779817in}}{\pgfqpoint{3.186556in}{1.779817in}}%
\pgfpathclose%
\pgfusepath{stroke,fill}%
\end{pgfscope}%
\begin{pgfscope}%
\pgfpathrectangle{\pgfqpoint{0.100000in}{0.212622in}}{\pgfqpoint{3.696000in}{3.696000in}}%
\pgfusepath{clip}%
\pgfsetbuttcap%
\pgfsetroundjoin%
\definecolor{currentfill}{rgb}{0.121569,0.466667,0.705882}%
\pgfsetfillcolor{currentfill}%
\pgfsetfillopacity{0.496802}%
\pgfsetlinewidth{1.003750pt}%
\definecolor{currentstroke}{rgb}{0.121569,0.466667,0.705882}%
\pgfsetstrokecolor{currentstroke}%
\pgfsetstrokeopacity{0.496802}%
\pgfsetdash{}{0pt}%
\pgfpathmoveto{\pgfqpoint{3.188717in}{1.779430in}}%
\pgfpathcurveto{\pgfqpoint{3.196953in}{1.779430in}}{\pgfqpoint{3.204853in}{1.782702in}}{\pgfqpoint{3.210677in}{1.788526in}}%
\pgfpathcurveto{\pgfqpoint{3.216501in}{1.794350in}}{\pgfqpoint{3.219773in}{1.802250in}}{\pgfqpoint{3.219773in}{1.810486in}}%
\pgfpathcurveto{\pgfqpoint{3.219773in}{1.818722in}}{\pgfqpoint{3.216501in}{1.826622in}}{\pgfqpoint{3.210677in}{1.832446in}}%
\pgfpathcurveto{\pgfqpoint{3.204853in}{1.838270in}}{\pgfqpoint{3.196953in}{1.841543in}}{\pgfqpoint{3.188717in}{1.841543in}}%
\pgfpathcurveto{\pgfqpoint{3.180480in}{1.841543in}}{\pgfqpoint{3.172580in}{1.838270in}}{\pgfqpoint{3.166756in}{1.832446in}}%
\pgfpathcurveto{\pgfqpoint{3.160932in}{1.826622in}}{\pgfqpoint{3.157660in}{1.818722in}}{\pgfqpoint{3.157660in}{1.810486in}}%
\pgfpathcurveto{\pgfqpoint{3.157660in}{1.802250in}}{\pgfqpoint{3.160932in}{1.794350in}}{\pgfqpoint{3.166756in}{1.788526in}}%
\pgfpathcurveto{\pgfqpoint{3.172580in}{1.782702in}}{\pgfqpoint{3.180480in}{1.779430in}}{\pgfqpoint{3.188717in}{1.779430in}}%
\pgfpathclose%
\pgfusepath{stroke,fill}%
\end{pgfscope}%
\begin{pgfscope}%
\pgfpathrectangle{\pgfqpoint{0.100000in}{0.212622in}}{\pgfqpoint{3.696000in}{3.696000in}}%
\pgfusepath{clip}%
\pgfsetbuttcap%
\pgfsetroundjoin%
\definecolor{currentfill}{rgb}{0.121569,0.466667,0.705882}%
\pgfsetfillcolor{currentfill}%
\pgfsetfillopacity{0.497204}%
\pgfsetlinewidth{1.003750pt}%
\definecolor{currentstroke}{rgb}{0.121569,0.466667,0.705882}%
\pgfsetstrokecolor{currentstroke}%
\pgfsetstrokeopacity{0.497204}%
\pgfsetdash{}{0pt}%
\pgfpathmoveto{\pgfqpoint{3.191285in}{1.779077in}}%
\pgfpathcurveto{\pgfqpoint{3.199522in}{1.779077in}}{\pgfqpoint{3.207422in}{1.782350in}}{\pgfqpoint{3.213246in}{1.788174in}}%
\pgfpathcurveto{\pgfqpoint{3.219070in}{1.793998in}}{\pgfqpoint{3.222342in}{1.801898in}}{\pgfqpoint{3.222342in}{1.810134in}}%
\pgfpathcurveto{\pgfqpoint{3.222342in}{1.818370in}}{\pgfqpoint{3.219070in}{1.826270in}}{\pgfqpoint{3.213246in}{1.832094in}}%
\pgfpathcurveto{\pgfqpoint{3.207422in}{1.837918in}}{\pgfqpoint{3.199522in}{1.841190in}}{\pgfqpoint{3.191285in}{1.841190in}}%
\pgfpathcurveto{\pgfqpoint{3.183049in}{1.841190in}}{\pgfqpoint{3.175149in}{1.837918in}}{\pgfqpoint{3.169325in}{1.832094in}}%
\pgfpathcurveto{\pgfqpoint{3.163501in}{1.826270in}}{\pgfqpoint{3.160229in}{1.818370in}}{\pgfqpoint{3.160229in}{1.810134in}}%
\pgfpathcurveto{\pgfqpoint{3.160229in}{1.801898in}}{\pgfqpoint{3.163501in}{1.793998in}}{\pgfqpoint{3.169325in}{1.788174in}}%
\pgfpathcurveto{\pgfqpoint{3.175149in}{1.782350in}}{\pgfqpoint{3.183049in}{1.779077in}}{\pgfqpoint{3.191285in}{1.779077in}}%
\pgfpathclose%
\pgfusepath{stroke,fill}%
\end{pgfscope}%
\begin{pgfscope}%
\pgfpathrectangle{\pgfqpoint{0.100000in}{0.212622in}}{\pgfqpoint{3.696000in}{3.696000in}}%
\pgfusepath{clip}%
\pgfsetbuttcap%
\pgfsetroundjoin%
\definecolor{currentfill}{rgb}{0.121569,0.466667,0.705882}%
\pgfsetfillcolor{currentfill}%
\pgfsetfillopacity{0.497623}%
\pgfsetlinewidth{1.003750pt}%
\definecolor{currentstroke}{rgb}{0.121569,0.466667,0.705882}%
\pgfsetstrokecolor{currentstroke}%
\pgfsetstrokeopacity{0.497623}%
\pgfsetdash{}{0pt}%
\pgfpathmoveto{\pgfqpoint{3.194606in}{1.778343in}}%
\pgfpathcurveto{\pgfqpoint{3.202843in}{1.778343in}}{\pgfqpoint{3.210743in}{1.781615in}}{\pgfqpoint{3.216566in}{1.787439in}}%
\pgfpathcurveto{\pgfqpoint{3.222390in}{1.793263in}}{\pgfqpoint{3.225663in}{1.801163in}}{\pgfqpoint{3.225663in}{1.809399in}}%
\pgfpathcurveto{\pgfqpoint{3.225663in}{1.817635in}}{\pgfqpoint{3.222390in}{1.825535in}}{\pgfqpoint{3.216566in}{1.831359in}}%
\pgfpathcurveto{\pgfqpoint{3.210743in}{1.837183in}}{\pgfqpoint{3.202843in}{1.840456in}}{\pgfqpoint{3.194606in}{1.840456in}}%
\pgfpathcurveto{\pgfqpoint{3.186370in}{1.840456in}}{\pgfqpoint{3.178470in}{1.837183in}}{\pgfqpoint{3.172646in}{1.831359in}}%
\pgfpathcurveto{\pgfqpoint{3.166822in}{1.825535in}}{\pgfqpoint{3.163550in}{1.817635in}}{\pgfqpoint{3.163550in}{1.809399in}}%
\pgfpathcurveto{\pgfqpoint{3.163550in}{1.801163in}}{\pgfqpoint{3.166822in}{1.793263in}}{\pgfqpoint{3.172646in}{1.787439in}}%
\pgfpathcurveto{\pgfqpoint{3.178470in}{1.781615in}}{\pgfqpoint{3.186370in}{1.778343in}}{\pgfqpoint{3.194606in}{1.778343in}}%
\pgfpathclose%
\pgfusepath{stroke,fill}%
\end{pgfscope}%
\begin{pgfscope}%
\pgfpathrectangle{\pgfqpoint{0.100000in}{0.212622in}}{\pgfqpoint{3.696000in}{3.696000in}}%
\pgfusepath{clip}%
\pgfsetbuttcap%
\pgfsetroundjoin%
\definecolor{currentfill}{rgb}{0.121569,0.466667,0.705882}%
\pgfsetfillcolor{currentfill}%
\pgfsetfillopacity{0.497662}%
\pgfsetlinewidth{1.003750pt}%
\definecolor{currentstroke}{rgb}{0.121569,0.466667,0.705882}%
\pgfsetstrokecolor{currentstroke}%
\pgfsetstrokeopacity{0.497662}%
\pgfsetdash{}{0pt}%
\pgfpathmoveto{\pgfqpoint{1.316517in}{2.119952in}}%
\pgfpathcurveto{\pgfqpoint{1.324754in}{2.119952in}}{\pgfqpoint{1.332654in}{2.123224in}}{\pgfqpoint{1.338477in}{2.129048in}}%
\pgfpathcurveto{\pgfqpoint{1.344301in}{2.134872in}}{\pgfqpoint{1.347574in}{2.142772in}}{\pgfqpoint{1.347574in}{2.151009in}}%
\pgfpathcurveto{\pgfqpoint{1.347574in}{2.159245in}}{\pgfqpoint{1.344301in}{2.167145in}}{\pgfqpoint{1.338477in}{2.172969in}}%
\pgfpathcurveto{\pgfqpoint{1.332654in}{2.178793in}}{\pgfqpoint{1.324754in}{2.182065in}}{\pgfqpoint{1.316517in}{2.182065in}}%
\pgfpathcurveto{\pgfqpoint{1.308281in}{2.182065in}}{\pgfqpoint{1.300381in}{2.178793in}}{\pgfqpoint{1.294557in}{2.172969in}}%
\pgfpathcurveto{\pgfqpoint{1.288733in}{2.167145in}}{\pgfqpoint{1.285461in}{2.159245in}}{\pgfqpoint{1.285461in}{2.151009in}}%
\pgfpathcurveto{\pgfqpoint{1.285461in}{2.142772in}}{\pgfqpoint{1.288733in}{2.134872in}}{\pgfqpoint{1.294557in}{2.129048in}}%
\pgfpathcurveto{\pgfqpoint{1.300381in}{2.123224in}}{\pgfqpoint{1.308281in}{2.119952in}}{\pgfqpoint{1.316517in}{2.119952in}}%
\pgfpathclose%
\pgfusepath{stroke,fill}%
\end{pgfscope}%
\begin{pgfscope}%
\pgfpathrectangle{\pgfqpoint{0.100000in}{0.212622in}}{\pgfqpoint{3.696000in}{3.696000in}}%
\pgfusepath{clip}%
\pgfsetbuttcap%
\pgfsetroundjoin%
\definecolor{currentfill}{rgb}{0.121569,0.466667,0.705882}%
\pgfsetfillcolor{currentfill}%
\pgfsetfillopacity{0.498155}%
\pgfsetlinewidth{1.003750pt}%
\definecolor{currentstroke}{rgb}{0.121569,0.466667,0.705882}%
\pgfsetstrokecolor{currentstroke}%
\pgfsetstrokeopacity{0.498155}%
\pgfsetdash{}{0pt}%
\pgfpathmoveto{\pgfqpoint{3.198294in}{1.777783in}}%
\pgfpathcurveto{\pgfqpoint{3.206530in}{1.777783in}}{\pgfqpoint{3.214430in}{1.781055in}}{\pgfqpoint{3.220254in}{1.786879in}}%
\pgfpathcurveto{\pgfqpoint{3.226078in}{1.792703in}}{\pgfqpoint{3.229350in}{1.800603in}}{\pgfqpoint{3.229350in}{1.808840in}}%
\pgfpathcurveto{\pgfqpoint{3.229350in}{1.817076in}}{\pgfqpoint{3.226078in}{1.824976in}}{\pgfqpoint{3.220254in}{1.830800in}}%
\pgfpathcurveto{\pgfqpoint{3.214430in}{1.836624in}}{\pgfqpoint{3.206530in}{1.839896in}}{\pgfqpoint{3.198294in}{1.839896in}}%
\pgfpathcurveto{\pgfqpoint{3.190057in}{1.839896in}}{\pgfqpoint{3.182157in}{1.836624in}}{\pgfqpoint{3.176333in}{1.830800in}}%
\pgfpathcurveto{\pgfqpoint{3.170509in}{1.824976in}}{\pgfqpoint{3.167237in}{1.817076in}}{\pgfqpoint{3.167237in}{1.808840in}}%
\pgfpathcurveto{\pgfqpoint{3.167237in}{1.800603in}}{\pgfqpoint{3.170509in}{1.792703in}}{\pgfqpoint{3.176333in}{1.786879in}}%
\pgfpathcurveto{\pgfqpoint{3.182157in}{1.781055in}}{\pgfqpoint{3.190057in}{1.777783in}}{\pgfqpoint{3.198294in}{1.777783in}}%
\pgfpathclose%
\pgfusepath{stroke,fill}%
\end{pgfscope}%
\begin{pgfscope}%
\pgfpathrectangle{\pgfqpoint{0.100000in}{0.212622in}}{\pgfqpoint{3.696000in}{3.696000in}}%
\pgfusepath{clip}%
\pgfsetbuttcap%
\pgfsetroundjoin%
\definecolor{currentfill}{rgb}{0.121569,0.466667,0.705882}%
\pgfsetfillcolor{currentfill}%
\pgfsetfillopacity{0.498423}%
\pgfsetlinewidth{1.003750pt}%
\definecolor{currentstroke}{rgb}{0.121569,0.466667,0.705882}%
\pgfsetstrokecolor{currentstroke}%
\pgfsetstrokeopacity{0.498423}%
\pgfsetdash{}{0pt}%
\pgfpathmoveto{\pgfqpoint{3.200359in}{1.777432in}}%
\pgfpathcurveto{\pgfqpoint{3.208595in}{1.777432in}}{\pgfqpoint{3.216495in}{1.780705in}}{\pgfqpoint{3.222319in}{1.786529in}}%
\pgfpathcurveto{\pgfqpoint{3.228143in}{1.792352in}}{\pgfqpoint{3.231415in}{1.800252in}}{\pgfqpoint{3.231415in}{1.808489in}}%
\pgfpathcurveto{\pgfqpoint{3.231415in}{1.816725in}}{\pgfqpoint{3.228143in}{1.824625in}}{\pgfqpoint{3.222319in}{1.830449in}}%
\pgfpathcurveto{\pgfqpoint{3.216495in}{1.836273in}}{\pgfqpoint{3.208595in}{1.839545in}}{\pgfqpoint{3.200359in}{1.839545in}}%
\pgfpathcurveto{\pgfqpoint{3.192123in}{1.839545in}}{\pgfqpoint{3.184223in}{1.836273in}}{\pgfqpoint{3.178399in}{1.830449in}}%
\pgfpathcurveto{\pgfqpoint{3.172575in}{1.824625in}}{\pgfqpoint{3.169302in}{1.816725in}}{\pgfqpoint{3.169302in}{1.808489in}}%
\pgfpathcurveto{\pgfqpoint{3.169302in}{1.800252in}}{\pgfqpoint{3.172575in}{1.792352in}}{\pgfqpoint{3.178399in}{1.786529in}}%
\pgfpathcurveto{\pgfqpoint{3.184223in}{1.780705in}}{\pgfqpoint{3.192123in}{1.777432in}}{\pgfqpoint{3.200359in}{1.777432in}}%
\pgfpathclose%
\pgfusepath{stroke,fill}%
\end{pgfscope}%
\begin{pgfscope}%
\pgfpathrectangle{\pgfqpoint{0.100000in}{0.212622in}}{\pgfqpoint{3.696000in}{3.696000in}}%
\pgfusepath{clip}%
\pgfsetbuttcap%
\pgfsetroundjoin%
\definecolor{currentfill}{rgb}{0.121569,0.466667,0.705882}%
\pgfsetfillcolor{currentfill}%
\pgfsetfillopacity{0.498654}%
\pgfsetlinewidth{1.003750pt}%
\definecolor{currentstroke}{rgb}{0.121569,0.466667,0.705882}%
\pgfsetstrokecolor{currentstroke}%
\pgfsetstrokeopacity{0.498654}%
\pgfsetdash{}{0pt}%
\pgfpathmoveto{\pgfqpoint{3.203123in}{1.776764in}}%
\pgfpathcurveto{\pgfqpoint{3.211359in}{1.776764in}}{\pgfqpoint{3.219259in}{1.780037in}}{\pgfqpoint{3.225083in}{1.785861in}}%
\pgfpathcurveto{\pgfqpoint{3.230907in}{1.791685in}}{\pgfqpoint{3.234180in}{1.799585in}}{\pgfqpoint{3.234180in}{1.807821in}}%
\pgfpathcurveto{\pgfqpoint{3.234180in}{1.816057in}}{\pgfqpoint{3.230907in}{1.823957in}}{\pgfqpoint{3.225083in}{1.829781in}}%
\pgfpathcurveto{\pgfqpoint{3.219259in}{1.835605in}}{\pgfqpoint{3.211359in}{1.838877in}}{\pgfqpoint{3.203123in}{1.838877in}}%
\pgfpathcurveto{\pgfqpoint{3.194887in}{1.838877in}}{\pgfqpoint{3.186987in}{1.835605in}}{\pgfqpoint{3.181163in}{1.829781in}}%
\pgfpathcurveto{\pgfqpoint{3.175339in}{1.823957in}}{\pgfqpoint{3.172067in}{1.816057in}}{\pgfqpoint{3.172067in}{1.807821in}}%
\pgfpathcurveto{\pgfqpoint{3.172067in}{1.799585in}}{\pgfqpoint{3.175339in}{1.791685in}}{\pgfqpoint{3.181163in}{1.785861in}}%
\pgfpathcurveto{\pgfqpoint{3.186987in}{1.780037in}}{\pgfqpoint{3.194887in}{1.776764in}}{\pgfqpoint{3.203123in}{1.776764in}}%
\pgfpathclose%
\pgfusepath{stroke,fill}%
\end{pgfscope}%
\begin{pgfscope}%
\pgfpathrectangle{\pgfqpoint{0.100000in}{0.212622in}}{\pgfqpoint{3.696000in}{3.696000in}}%
\pgfusepath{clip}%
\pgfsetbuttcap%
\pgfsetroundjoin%
\definecolor{currentfill}{rgb}{0.121569,0.466667,0.705882}%
\pgfsetfillcolor{currentfill}%
\pgfsetfillopacity{0.499129}%
\pgfsetlinewidth{1.003750pt}%
\definecolor{currentstroke}{rgb}{0.121569,0.466667,0.705882}%
\pgfsetstrokecolor{currentstroke}%
\pgfsetstrokeopacity{0.499129}%
\pgfsetdash{}{0pt}%
\pgfpathmoveto{\pgfqpoint{3.205904in}{1.776439in}}%
\pgfpathcurveto{\pgfqpoint{3.214141in}{1.776439in}}{\pgfqpoint{3.222041in}{1.779711in}}{\pgfqpoint{3.227865in}{1.785535in}}%
\pgfpathcurveto{\pgfqpoint{3.233688in}{1.791359in}}{\pgfqpoint{3.236961in}{1.799259in}}{\pgfqpoint{3.236961in}{1.807495in}}%
\pgfpathcurveto{\pgfqpoint{3.236961in}{1.815732in}}{\pgfqpoint{3.233688in}{1.823632in}}{\pgfqpoint{3.227865in}{1.829456in}}%
\pgfpathcurveto{\pgfqpoint{3.222041in}{1.835280in}}{\pgfqpoint{3.214141in}{1.838552in}}{\pgfqpoint{3.205904in}{1.838552in}}%
\pgfpathcurveto{\pgfqpoint{3.197668in}{1.838552in}}{\pgfqpoint{3.189768in}{1.835280in}}{\pgfqpoint{3.183944in}{1.829456in}}%
\pgfpathcurveto{\pgfqpoint{3.178120in}{1.823632in}}{\pgfqpoint{3.174848in}{1.815732in}}{\pgfqpoint{3.174848in}{1.807495in}}%
\pgfpathcurveto{\pgfqpoint{3.174848in}{1.799259in}}{\pgfqpoint{3.178120in}{1.791359in}}{\pgfqpoint{3.183944in}{1.785535in}}%
\pgfpathcurveto{\pgfqpoint{3.189768in}{1.779711in}}{\pgfqpoint{3.197668in}{1.776439in}}{\pgfqpoint{3.205904in}{1.776439in}}%
\pgfpathclose%
\pgfusepath{stroke,fill}%
\end{pgfscope}%
\begin{pgfscope}%
\pgfpathrectangle{\pgfqpoint{0.100000in}{0.212622in}}{\pgfqpoint{3.696000in}{3.696000in}}%
\pgfusepath{clip}%
\pgfsetbuttcap%
\pgfsetroundjoin%
\definecolor{currentfill}{rgb}{0.121569,0.466667,0.705882}%
\pgfsetfillcolor{currentfill}%
\pgfsetfillopacity{0.499293}%
\pgfsetlinewidth{1.003750pt}%
\definecolor{currentstroke}{rgb}{0.121569,0.466667,0.705882}%
\pgfsetstrokecolor{currentstroke}%
\pgfsetstrokeopacity{0.499293}%
\pgfsetdash{}{0pt}%
\pgfpathmoveto{\pgfqpoint{3.209984in}{1.775421in}}%
\pgfpathcurveto{\pgfqpoint{3.218221in}{1.775421in}}{\pgfqpoint{3.226121in}{1.778693in}}{\pgfqpoint{3.231945in}{1.784517in}}%
\pgfpathcurveto{\pgfqpoint{3.237769in}{1.790341in}}{\pgfqpoint{3.241041in}{1.798241in}}{\pgfqpoint{3.241041in}{1.806477in}}%
\pgfpathcurveto{\pgfqpoint{3.241041in}{1.814713in}}{\pgfqpoint{3.237769in}{1.822613in}}{\pgfqpoint{3.231945in}{1.828437in}}%
\pgfpathcurveto{\pgfqpoint{3.226121in}{1.834261in}}{\pgfqpoint{3.218221in}{1.837534in}}{\pgfqpoint{3.209984in}{1.837534in}}%
\pgfpathcurveto{\pgfqpoint{3.201748in}{1.837534in}}{\pgfqpoint{3.193848in}{1.834261in}}{\pgfqpoint{3.188024in}{1.828437in}}%
\pgfpathcurveto{\pgfqpoint{3.182200in}{1.822613in}}{\pgfqpoint{3.178928in}{1.814713in}}{\pgfqpoint{3.178928in}{1.806477in}}%
\pgfpathcurveto{\pgfqpoint{3.178928in}{1.798241in}}{\pgfqpoint{3.182200in}{1.790341in}}{\pgfqpoint{3.188024in}{1.784517in}}%
\pgfpathcurveto{\pgfqpoint{3.193848in}{1.778693in}}{\pgfqpoint{3.201748in}{1.775421in}}{\pgfqpoint{3.209984in}{1.775421in}}%
\pgfpathclose%
\pgfusepath{stroke,fill}%
\end{pgfscope}%
\begin{pgfscope}%
\pgfpathrectangle{\pgfqpoint{0.100000in}{0.212622in}}{\pgfqpoint{3.696000in}{3.696000in}}%
\pgfusepath{clip}%
\pgfsetbuttcap%
\pgfsetroundjoin%
\definecolor{currentfill}{rgb}{0.121569,0.466667,0.705882}%
\pgfsetfillcolor{currentfill}%
\pgfsetfillopacity{0.499726}%
\pgfsetlinewidth{1.003750pt}%
\definecolor{currentstroke}{rgb}{0.121569,0.466667,0.705882}%
\pgfsetstrokecolor{currentstroke}%
\pgfsetstrokeopacity{0.499726}%
\pgfsetdash{}{0pt}%
\pgfpathmoveto{\pgfqpoint{3.214681in}{1.774337in}}%
\pgfpathcurveto{\pgfqpoint{3.222918in}{1.774337in}}{\pgfqpoint{3.230818in}{1.777610in}}{\pgfqpoint{3.236642in}{1.783434in}}%
\pgfpathcurveto{\pgfqpoint{3.242466in}{1.789257in}}{\pgfqpoint{3.245738in}{1.797158in}}{\pgfqpoint{3.245738in}{1.805394in}}%
\pgfpathcurveto{\pgfqpoint{3.245738in}{1.813630in}}{\pgfqpoint{3.242466in}{1.821530in}}{\pgfqpoint{3.236642in}{1.827354in}}%
\pgfpathcurveto{\pgfqpoint{3.230818in}{1.833178in}}{\pgfqpoint{3.222918in}{1.836450in}}{\pgfqpoint{3.214681in}{1.836450in}}%
\pgfpathcurveto{\pgfqpoint{3.206445in}{1.836450in}}{\pgfqpoint{3.198545in}{1.833178in}}{\pgfqpoint{3.192721in}{1.827354in}}%
\pgfpathcurveto{\pgfqpoint{3.186897in}{1.821530in}}{\pgfqpoint{3.183625in}{1.813630in}}{\pgfqpoint{3.183625in}{1.805394in}}%
\pgfpathcurveto{\pgfqpoint{3.183625in}{1.797158in}}{\pgfqpoint{3.186897in}{1.789257in}}{\pgfqpoint{3.192721in}{1.783434in}}%
\pgfpathcurveto{\pgfqpoint{3.198545in}{1.777610in}}{\pgfqpoint{3.206445in}{1.774337in}}{\pgfqpoint{3.214681in}{1.774337in}}%
\pgfpathclose%
\pgfusepath{stroke,fill}%
\end{pgfscope}%
\begin{pgfscope}%
\pgfpathrectangle{\pgfqpoint{0.100000in}{0.212622in}}{\pgfqpoint{3.696000in}{3.696000in}}%
\pgfusepath{clip}%
\pgfsetbuttcap%
\pgfsetroundjoin%
\definecolor{currentfill}{rgb}{0.121569,0.466667,0.705882}%
\pgfsetfillcolor{currentfill}%
\pgfsetfillopacity{0.500567}%
\pgfsetlinewidth{1.003750pt}%
\definecolor{currentstroke}{rgb}{0.121569,0.466667,0.705882}%
\pgfsetstrokecolor{currentstroke}%
\pgfsetstrokeopacity{0.500567}%
\pgfsetdash{}{0pt}%
\pgfpathmoveto{\pgfqpoint{3.220105in}{1.773507in}}%
\pgfpathcurveto{\pgfqpoint{3.228341in}{1.773507in}}{\pgfqpoint{3.236241in}{1.776780in}}{\pgfqpoint{3.242065in}{1.782603in}}%
\pgfpathcurveto{\pgfqpoint{3.247889in}{1.788427in}}{\pgfqpoint{3.251161in}{1.796327in}}{\pgfqpoint{3.251161in}{1.804564in}}%
\pgfpathcurveto{\pgfqpoint{3.251161in}{1.812800in}}{\pgfqpoint{3.247889in}{1.820700in}}{\pgfqpoint{3.242065in}{1.826524in}}%
\pgfpathcurveto{\pgfqpoint{3.236241in}{1.832348in}}{\pgfqpoint{3.228341in}{1.835620in}}{\pgfqpoint{3.220105in}{1.835620in}}%
\pgfpathcurveto{\pgfqpoint{3.211868in}{1.835620in}}{\pgfqpoint{3.203968in}{1.832348in}}{\pgfqpoint{3.198144in}{1.826524in}}%
\pgfpathcurveto{\pgfqpoint{3.192320in}{1.820700in}}{\pgfqpoint{3.189048in}{1.812800in}}{\pgfqpoint{3.189048in}{1.804564in}}%
\pgfpathcurveto{\pgfqpoint{3.189048in}{1.796327in}}{\pgfqpoint{3.192320in}{1.788427in}}{\pgfqpoint{3.198144in}{1.782603in}}%
\pgfpathcurveto{\pgfqpoint{3.203968in}{1.776780in}}{\pgfqpoint{3.211868in}{1.773507in}}{\pgfqpoint{3.220105in}{1.773507in}}%
\pgfpathclose%
\pgfusepath{stroke,fill}%
\end{pgfscope}%
\begin{pgfscope}%
\pgfpathrectangle{\pgfqpoint{0.100000in}{0.212622in}}{\pgfqpoint{3.696000in}{3.696000in}}%
\pgfusepath{clip}%
\pgfsetbuttcap%
\pgfsetroundjoin%
\definecolor{currentfill}{rgb}{0.121569,0.466667,0.705882}%
\pgfsetfillcolor{currentfill}%
\pgfsetfillopacity{0.500907}%
\pgfsetlinewidth{1.003750pt}%
\definecolor{currentstroke}{rgb}{0.121569,0.466667,0.705882}%
\pgfsetstrokecolor{currentstroke}%
\pgfsetstrokeopacity{0.500907}%
\pgfsetdash{}{0pt}%
\pgfpathmoveto{\pgfqpoint{1.307807in}{2.120598in}}%
\pgfpathcurveto{\pgfqpoint{1.316044in}{2.120598in}}{\pgfqpoint{1.323944in}{2.123870in}}{\pgfqpoint{1.329767in}{2.129694in}}%
\pgfpathcurveto{\pgfqpoint{1.335591in}{2.135518in}}{\pgfqpoint{1.338864in}{2.143418in}}{\pgfqpoint{1.338864in}{2.151654in}}%
\pgfpathcurveto{\pgfqpoint{1.338864in}{2.159891in}}{\pgfqpoint{1.335591in}{2.167791in}}{\pgfqpoint{1.329767in}{2.173615in}}%
\pgfpathcurveto{\pgfqpoint{1.323944in}{2.179439in}}{\pgfqpoint{1.316044in}{2.182711in}}{\pgfqpoint{1.307807in}{2.182711in}}%
\pgfpathcurveto{\pgfqpoint{1.299571in}{2.182711in}}{\pgfqpoint{1.291671in}{2.179439in}}{\pgfqpoint{1.285847in}{2.173615in}}%
\pgfpathcurveto{\pgfqpoint{1.280023in}{2.167791in}}{\pgfqpoint{1.276751in}{2.159891in}}{\pgfqpoint{1.276751in}{2.151654in}}%
\pgfpathcurveto{\pgfqpoint{1.276751in}{2.143418in}}{\pgfqpoint{1.280023in}{2.135518in}}{\pgfqpoint{1.285847in}{2.129694in}}%
\pgfpathcurveto{\pgfqpoint{1.291671in}{2.123870in}}{\pgfqpoint{1.299571in}{2.120598in}}{\pgfqpoint{1.307807in}{2.120598in}}%
\pgfpathclose%
\pgfusepath{stroke,fill}%
\end{pgfscope}%
\begin{pgfscope}%
\pgfpathrectangle{\pgfqpoint{0.100000in}{0.212622in}}{\pgfqpoint{3.696000in}{3.696000in}}%
\pgfusepath{clip}%
\pgfsetbuttcap%
\pgfsetroundjoin%
\definecolor{currentfill}{rgb}{0.121569,0.466667,0.705882}%
\pgfsetfillcolor{currentfill}%
\pgfsetfillopacity{0.501134}%
\pgfsetlinewidth{1.003750pt}%
\definecolor{currentstroke}{rgb}{0.121569,0.466667,0.705882}%
\pgfsetstrokecolor{currentstroke}%
\pgfsetstrokeopacity{0.501134}%
\pgfsetdash{}{0pt}%
\pgfpathmoveto{\pgfqpoint{3.228185in}{1.771213in}}%
\pgfpathcurveto{\pgfqpoint{3.236422in}{1.771213in}}{\pgfqpoint{3.244322in}{1.774485in}}{\pgfqpoint{3.250146in}{1.780309in}}%
\pgfpathcurveto{\pgfqpoint{3.255970in}{1.786133in}}{\pgfqpoint{3.259242in}{1.794033in}}{\pgfqpoint{3.259242in}{1.802270in}}%
\pgfpathcurveto{\pgfqpoint{3.259242in}{1.810506in}}{\pgfqpoint{3.255970in}{1.818406in}}{\pgfqpoint{3.250146in}{1.824230in}}%
\pgfpathcurveto{\pgfqpoint{3.244322in}{1.830054in}}{\pgfqpoint{3.236422in}{1.833326in}}{\pgfqpoint{3.228185in}{1.833326in}}%
\pgfpathcurveto{\pgfqpoint{3.219949in}{1.833326in}}{\pgfqpoint{3.212049in}{1.830054in}}{\pgfqpoint{3.206225in}{1.824230in}}%
\pgfpathcurveto{\pgfqpoint{3.200401in}{1.818406in}}{\pgfqpoint{3.197129in}{1.810506in}}{\pgfqpoint{3.197129in}{1.802270in}}%
\pgfpathcurveto{\pgfqpoint{3.197129in}{1.794033in}}{\pgfqpoint{3.200401in}{1.786133in}}{\pgfqpoint{3.206225in}{1.780309in}}%
\pgfpathcurveto{\pgfqpoint{3.212049in}{1.774485in}}{\pgfqpoint{3.219949in}{1.771213in}}{\pgfqpoint{3.228185in}{1.771213in}}%
\pgfpathclose%
\pgfusepath{stroke,fill}%
\end{pgfscope}%
\begin{pgfscope}%
\pgfpathrectangle{\pgfqpoint{0.100000in}{0.212622in}}{\pgfqpoint{3.696000in}{3.696000in}}%
\pgfusepath{clip}%
\pgfsetbuttcap%
\pgfsetroundjoin%
\definecolor{currentfill}{rgb}{0.121569,0.466667,0.705882}%
\pgfsetfillcolor{currentfill}%
\pgfsetfillopacity{0.501458}%
\pgfsetlinewidth{1.003750pt}%
\definecolor{currentstroke}{rgb}{0.121569,0.466667,0.705882}%
\pgfsetstrokecolor{currentstroke}%
\pgfsetstrokeopacity{0.501458}%
\pgfsetdash{}{0pt}%
\pgfpathmoveto{\pgfqpoint{3.237507in}{1.768901in}}%
\pgfpathcurveto{\pgfqpoint{3.245743in}{1.768901in}}{\pgfqpoint{3.253643in}{1.772173in}}{\pgfqpoint{3.259467in}{1.777997in}}%
\pgfpathcurveto{\pgfqpoint{3.265291in}{1.783821in}}{\pgfqpoint{3.268563in}{1.791721in}}{\pgfqpoint{3.268563in}{1.799957in}}%
\pgfpathcurveto{\pgfqpoint{3.268563in}{1.808193in}}{\pgfqpoint{3.265291in}{1.816094in}}{\pgfqpoint{3.259467in}{1.821917in}}%
\pgfpathcurveto{\pgfqpoint{3.253643in}{1.827741in}}{\pgfqpoint{3.245743in}{1.831014in}}{\pgfqpoint{3.237507in}{1.831014in}}%
\pgfpathcurveto{\pgfqpoint{3.229271in}{1.831014in}}{\pgfqpoint{3.221371in}{1.827741in}}{\pgfqpoint{3.215547in}{1.821917in}}%
\pgfpathcurveto{\pgfqpoint{3.209723in}{1.816094in}}{\pgfqpoint{3.206450in}{1.808193in}}{\pgfqpoint{3.206450in}{1.799957in}}%
\pgfpathcurveto{\pgfqpoint{3.206450in}{1.791721in}}{\pgfqpoint{3.209723in}{1.783821in}}{\pgfqpoint{3.215547in}{1.777997in}}%
\pgfpathcurveto{\pgfqpoint{3.221371in}{1.772173in}}{\pgfqpoint{3.229271in}{1.768901in}}{\pgfqpoint{3.237507in}{1.768901in}}%
\pgfpathclose%
\pgfusepath{stroke,fill}%
\end{pgfscope}%
\begin{pgfscope}%
\pgfpathrectangle{\pgfqpoint{0.100000in}{0.212622in}}{\pgfqpoint{3.696000in}{3.696000in}}%
\pgfusepath{clip}%
\pgfsetbuttcap%
\pgfsetroundjoin%
\definecolor{currentfill}{rgb}{0.121569,0.466667,0.705882}%
\pgfsetfillcolor{currentfill}%
\pgfsetfillopacity{0.502456}%
\pgfsetlinewidth{1.003750pt}%
\definecolor{currentstroke}{rgb}{0.121569,0.466667,0.705882}%
\pgfsetstrokecolor{currentstroke}%
\pgfsetstrokeopacity{0.502456}%
\pgfsetdash{}{0pt}%
\pgfpathmoveto{\pgfqpoint{3.246604in}{1.767413in}}%
\pgfpathcurveto{\pgfqpoint{3.254840in}{1.767413in}}{\pgfqpoint{3.262740in}{1.770686in}}{\pgfqpoint{3.268564in}{1.776510in}}%
\pgfpathcurveto{\pgfqpoint{3.274388in}{1.782334in}}{\pgfqpoint{3.277660in}{1.790234in}}{\pgfqpoint{3.277660in}{1.798470in}}%
\pgfpathcurveto{\pgfqpoint{3.277660in}{1.806706in}}{\pgfqpoint{3.274388in}{1.814606in}}{\pgfqpoint{3.268564in}{1.820430in}}%
\pgfpathcurveto{\pgfqpoint{3.262740in}{1.826254in}}{\pgfqpoint{3.254840in}{1.829526in}}{\pgfqpoint{3.246604in}{1.829526in}}%
\pgfpathcurveto{\pgfqpoint{3.238367in}{1.829526in}}{\pgfqpoint{3.230467in}{1.826254in}}{\pgfqpoint{3.224643in}{1.820430in}}%
\pgfpathcurveto{\pgfqpoint{3.218819in}{1.814606in}}{\pgfqpoint{3.215547in}{1.806706in}}{\pgfqpoint{3.215547in}{1.798470in}}%
\pgfpathcurveto{\pgfqpoint{3.215547in}{1.790234in}}{\pgfqpoint{3.218819in}{1.782334in}}{\pgfqpoint{3.224643in}{1.776510in}}%
\pgfpathcurveto{\pgfqpoint{3.230467in}{1.770686in}}{\pgfqpoint{3.238367in}{1.767413in}}{\pgfqpoint{3.246604in}{1.767413in}}%
\pgfpathclose%
\pgfusepath{stroke,fill}%
\end{pgfscope}%
\begin{pgfscope}%
\pgfpathrectangle{\pgfqpoint{0.100000in}{0.212622in}}{\pgfqpoint{3.696000in}{3.696000in}}%
\pgfusepath{clip}%
\pgfsetbuttcap%
\pgfsetroundjoin%
\definecolor{currentfill}{rgb}{0.121569,0.466667,0.705882}%
\pgfsetfillcolor{currentfill}%
\pgfsetfillopacity{0.503593}%
\pgfsetlinewidth{1.003750pt}%
\definecolor{currentstroke}{rgb}{0.121569,0.466667,0.705882}%
\pgfsetstrokecolor{currentstroke}%
\pgfsetstrokeopacity{0.503593}%
\pgfsetdash{}{0pt}%
\pgfpathmoveto{\pgfqpoint{3.256623in}{1.765347in}}%
\pgfpathcurveto{\pgfqpoint{3.264860in}{1.765347in}}{\pgfqpoint{3.272760in}{1.768620in}}{\pgfqpoint{3.278584in}{1.774444in}}%
\pgfpathcurveto{\pgfqpoint{3.284407in}{1.780267in}}{\pgfqpoint{3.287680in}{1.788168in}}{\pgfqpoint{3.287680in}{1.796404in}}%
\pgfpathcurveto{\pgfqpoint{3.287680in}{1.804640in}}{\pgfqpoint{3.284407in}{1.812540in}}{\pgfqpoint{3.278584in}{1.818364in}}%
\pgfpathcurveto{\pgfqpoint{3.272760in}{1.824188in}}{\pgfqpoint{3.264860in}{1.827460in}}{\pgfqpoint{3.256623in}{1.827460in}}%
\pgfpathcurveto{\pgfqpoint{3.248387in}{1.827460in}}{\pgfqpoint{3.240487in}{1.824188in}}{\pgfqpoint{3.234663in}{1.818364in}}%
\pgfpathcurveto{\pgfqpoint{3.228839in}{1.812540in}}{\pgfqpoint{3.225567in}{1.804640in}}{\pgfqpoint{3.225567in}{1.796404in}}%
\pgfpathcurveto{\pgfqpoint{3.225567in}{1.788168in}}{\pgfqpoint{3.228839in}{1.780267in}}{\pgfqpoint{3.234663in}{1.774444in}}%
\pgfpathcurveto{\pgfqpoint{3.240487in}{1.768620in}}{\pgfqpoint{3.248387in}{1.765347in}}{\pgfqpoint{3.256623in}{1.765347in}}%
\pgfpathclose%
\pgfusepath{stroke,fill}%
\end{pgfscope}%
\begin{pgfscope}%
\pgfpathrectangle{\pgfqpoint{0.100000in}{0.212622in}}{\pgfqpoint{3.696000in}{3.696000in}}%
\pgfusepath{clip}%
\pgfsetbuttcap%
\pgfsetroundjoin%
\definecolor{currentfill}{rgb}{0.121569,0.466667,0.705882}%
\pgfsetfillcolor{currentfill}%
\pgfsetfillopacity{0.504421}%
\pgfsetlinewidth{1.003750pt}%
\definecolor{currentstroke}{rgb}{0.121569,0.466667,0.705882}%
\pgfsetstrokecolor{currentstroke}%
\pgfsetstrokeopacity{0.504421}%
\pgfsetdash{}{0pt}%
\pgfpathmoveto{\pgfqpoint{1.303583in}{2.121100in}}%
\pgfpathcurveto{\pgfqpoint{1.311819in}{2.121100in}}{\pgfqpoint{1.319720in}{2.124373in}}{\pgfqpoint{1.325543in}{2.130196in}}%
\pgfpathcurveto{\pgfqpoint{1.331367in}{2.136020in}}{\pgfqpoint{1.334640in}{2.143920in}}{\pgfqpoint{1.334640in}{2.152157in}}%
\pgfpathcurveto{\pgfqpoint{1.334640in}{2.160393in}}{\pgfqpoint{1.331367in}{2.168293in}}{\pgfqpoint{1.325543in}{2.174117in}}%
\pgfpathcurveto{\pgfqpoint{1.319720in}{2.179941in}}{\pgfqpoint{1.311819in}{2.183213in}}{\pgfqpoint{1.303583in}{2.183213in}}%
\pgfpathcurveto{\pgfqpoint{1.295347in}{2.183213in}}{\pgfqpoint{1.287447in}{2.179941in}}{\pgfqpoint{1.281623in}{2.174117in}}%
\pgfpathcurveto{\pgfqpoint{1.275799in}{2.168293in}}{\pgfqpoint{1.272527in}{2.160393in}}{\pgfqpoint{1.272527in}{2.152157in}}%
\pgfpathcurveto{\pgfqpoint{1.272527in}{2.143920in}}{\pgfqpoint{1.275799in}{2.136020in}}{\pgfqpoint{1.281623in}{2.130196in}}%
\pgfpathcurveto{\pgfqpoint{1.287447in}{2.124373in}}{\pgfqpoint{1.295347in}{2.121100in}}{\pgfqpoint{1.303583in}{2.121100in}}%
\pgfpathclose%
\pgfusepath{stroke,fill}%
\end{pgfscope}%
\begin{pgfscope}%
\pgfpathrectangle{\pgfqpoint{0.100000in}{0.212622in}}{\pgfqpoint{3.696000in}{3.696000in}}%
\pgfusepath{clip}%
\pgfsetbuttcap%
\pgfsetroundjoin%
\definecolor{currentfill}{rgb}{0.121569,0.466667,0.705882}%
\pgfsetfillcolor{currentfill}%
\pgfsetfillopacity{0.505051}%
\pgfsetlinewidth{1.003750pt}%
\definecolor{currentstroke}{rgb}{0.121569,0.466667,0.705882}%
\pgfsetstrokecolor{currentstroke}%
\pgfsetstrokeopacity{0.505051}%
\pgfsetdash{}{0pt}%
\pgfpathmoveto{\pgfqpoint{3.266805in}{1.763631in}}%
\pgfpathcurveto{\pgfqpoint{3.275041in}{1.763631in}}{\pgfqpoint{3.282941in}{1.766903in}}{\pgfqpoint{3.288765in}{1.772727in}}%
\pgfpathcurveto{\pgfqpoint{3.294589in}{1.778551in}}{\pgfqpoint{3.297861in}{1.786451in}}{\pgfqpoint{3.297861in}{1.794687in}}%
\pgfpathcurveto{\pgfqpoint{3.297861in}{1.802923in}}{\pgfqpoint{3.294589in}{1.810823in}}{\pgfqpoint{3.288765in}{1.816647in}}%
\pgfpathcurveto{\pgfqpoint{3.282941in}{1.822471in}}{\pgfqpoint{3.275041in}{1.825744in}}{\pgfqpoint{3.266805in}{1.825744in}}%
\pgfpathcurveto{\pgfqpoint{3.258568in}{1.825744in}}{\pgfqpoint{3.250668in}{1.822471in}}{\pgfqpoint{3.244845in}{1.816647in}}%
\pgfpathcurveto{\pgfqpoint{3.239021in}{1.810823in}}{\pgfqpoint{3.235748in}{1.802923in}}{\pgfqpoint{3.235748in}{1.794687in}}%
\pgfpathcurveto{\pgfqpoint{3.235748in}{1.786451in}}{\pgfqpoint{3.239021in}{1.778551in}}{\pgfqpoint{3.244845in}{1.772727in}}%
\pgfpathcurveto{\pgfqpoint{3.250668in}{1.766903in}}{\pgfqpoint{3.258568in}{1.763631in}}{\pgfqpoint{3.266805in}{1.763631in}}%
\pgfpathclose%
\pgfusepath{stroke,fill}%
\end{pgfscope}%
\begin{pgfscope}%
\pgfpathrectangle{\pgfqpoint{0.100000in}{0.212622in}}{\pgfqpoint{3.696000in}{3.696000in}}%
\pgfusepath{clip}%
\pgfsetbuttcap%
\pgfsetroundjoin%
\definecolor{currentfill}{rgb}{0.121569,0.466667,0.705882}%
\pgfsetfillcolor{currentfill}%
\pgfsetfillopacity{0.506198}%
\pgfsetlinewidth{1.003750pt}%
\definecolor{currentstroke}{rgb}{0.121569,0.466667,0.705882}%
\pgfsetstrokecolor{currentstroke}%
\pgfsetstrokeopacity{0.506198}%
\pgfsetdash{}{0pt}%
\pgfpathmoveto{\pgfqpoint{3.278143in}{1.761711in}}%
\pgfpathcurveto{\pgfqpoint{3.286379in}{1.761711in}}{\pgfqpoint{3.294279in}{1.764983in}}{\pgfqpoint{3.300103in}{1.770807in}}%
\pgfpathcurveto{\pgfqpoint{3.305927in}{1.776631in}}{\pgfqpoint{3.309199in}{1.784531in}}{\pgfqpoint{3.309199in}{1.792767in}}%
\pgfpathcurveto{\pgfqpoint{3.309199in}{1.801003in}}{\pgfqpoint{3.305927in}{1.808904in}}{\pgfqpoint{3.300103in}{1.814727in}}%
\pgfpathcurveto{\pgfqpoint{3.294279in}{1.820551in}}{\pgfqpoint{3.286379in}{1.823824in}}{\pgfqpoint{3.278143in}{1.823824in}}%
\pgfpathcurveto{\pgfqpoint{3.269906in}{1.823824in}}{\pgfqpoint{3.262006in}{1.820551in}}{\pgfqpoint{3.256182in}{1.814727in}}%
\pgfpathcurveto{\pgfqpoint{3.250358in}{1.808904in}}{\pgfqpoint{3.247086in}{1.801003in}}{\pgfqpoint{3.247086in}{1.792767in}}%
\pgfpathcurveto{\pgfqpoint{3.247086in}{1.784531in}}{\pgfqpoint{3.250358in}{1.776631in}}{\pgfqpoint{3.256182in}{1.770807in}}%
\pgfpathcurveto{\pgfqpoint{3.262006in}{1.764983in}}{\pgfqpoint{3.269906in}{1.761711in}}{\pgfqpoint{3.278143in}{1.761711in}}%
\pgfpathclose%
\pgfusepath{stroke,fill}%
\end{pgfscope}%
\begin{pgfscope}%
\pgfpathrectangle{\pgfqpoint{0.100000in}{0.212622in}}{\pgfqpoint{3.696000in}{3.696000in}}%
\pgfusepath{clip}%
\pgfsetbuttcap%
\pgfsetroundjoin%
\definecolor{currentfill}{rgb}{0.121569,0.466667,0.705882}%
\pgfsetfillcolor{currentfill}%
\pgfsetfillopacity{0.506794}%
\pgfsetlinewidth{1.003750pt}%
\definecolor{currentstroke}{rgb}{0.121569,0.466667,0.705882}%
\pgfsetstrokecolor{currentstroke}%
\pgfsetstrokeopacity{0.506794}%
\pgfsetdash{}{0pt}%
\pgfpathmoveto{\pgfqpoint{3.284378in}{1.760415in}}%
\pgfpathcurveto{\pgfqpoint{3.292614in}{1.760415in}}{\pgfqpoint{3.300514in}{1.763687in}}{\pgfqpoint{3.306338in}{1.769511in}}%
\pgfpathcurveto{\pgfqpoint{3.312162in}{1.775335in}}{\pgfqpoint{3.315434in}{1.783235in}}{\pgfqpoint{3.315434in}{1.791471in}}%
\pgfpathcurveto{\pgfqpoint{3.315434in}{1.799708in}}{\pgfqpoint{3.312162in}{1.807608in}}{\pgfqpoint{3.306338in}{1.813432in}}%
\pgfpathcurveto{\pgfqpoint{3.300514in}{1.819255in}}{\pgfqpoint{3.292614in}{1.822528in}}{\pgfqpoint{3.284378in}{1.822528in}}%
\pgfpathcurveto{\pgfqpoint{3.276142in}{1.822528in}}{\pgfqpoint{3.268241in}{1.819255in}}{\pgfqpoint{3.262418in}{1.813432in}}%
\pgfpathcurveto{\pgfqpoint{3.256594in}{1.807608in}}{\pgfqpoint{3.253321in}{1.799708in}}{\pgfqpoint{3.253321in}{1.791471in}}%
\pgfpathcurveto{\pgfqpoint{3.253321in}{1.783235in}}{\pgfqpoint{3.256594in}{1.775335in}}{\pgfqpoint{3.262418in}{1.769511in}}%
\pgfpathcurveto{\pgfqpoint{3.268241in}{1.763687in}}{\pgfqpoint{3.276142in}{1.760415in}}{\pgfqpoint{3.284378in}{1.760415in}}%
\pgfpathclose%
\pgfusepath{stroke,fill}%
\end{pgfscope}%
\begin{pgfscope}%
\pgfpathrectangle{\pgfqpoint{0.100000in}{0.212622in}}{\pgfqpoint{3.696000in}{3.696000in}}%
\pgfusepath{clip}%
\pgfsetbuttcap%
\pgfsetroundjoin%
\definecolor{currentfill}{rgb}{0.121569,0.466667,0.705882}%
\pgfsetfillcolor{currentfill}%
\pgfsetfillopacity{0.507289}%
\pgfsetlinewidth{1.003750pt}%
\definecolor{currentstroke}{rgb}{0.121569,0.466667,0.705882}%
\pgfsetstrokecolor{currentstroke}%
\pgfsetstrokeopacity{0.507289}%
\pgfsetdash{}{0pt}%
\pgfpathmoveto{\pgfqpoint{1.296113in}{2.121856in}}%
\pgfpathcurveto{\pgfqpoint{1.304349in}{2.121856in}}{\pgfqpoint{1.312249in}{2.125129in}}{\pgfqpoint{1.318073in}{2.130953in}}%
\pgfpathcurveto{\pgfqpoint{1.323897in}{2.136776in}}{\pgfqpoint{1.327169in}{2.144677in}}{\pgfqpoint{1.327169in}{2.152913in}}%
\pgfpathcurveto{\pgfqpoint{1.327169in}{2.161149in}}{\pgfqpoint{1.323897in}{2.169049in}}{\pgfqpoint{1.318073in}{2.174873in}}%
\pgfpathcurveto{\pgfqpoint{1.312249in}{2.180697in}}{\pgfqpoint{1.304349in}{2.183969in}}{\pgfqpoint{1.296113in}{2.183969in}}%
\pgfpathcurveto{\pgfqpoint{1.287876in}{2.183969in}}{\pgfqpoint{1.279976in}{2.180697in}}{\pgfqpoint{1.274152in}{2.174873in}}%
\pgfpathcurveto{\pgfqpoint{1.268328in}{2.169049in}}{\pgfqpoint{1.265056in}{2.161149in}}{\pgfqpoint{1.265056in}{2.152913in}}%
\pgfpathcurveto{\pgfqpoint{1.265056in}{2.144677in}}{\pgfqpoint{1.268328in}{2.136776in}}{\pgfqpoint{1.274152in}{2.130953in}}%
\pgfpathcurveto{\pgfqpoint{1.279976in}{2.125129in}}{\pgfqpoint{1.287876in}{2.121856in}}{\pgfqpoint{1.296113in}{2.121856in}}%
\pgfpathclose%
\pgfusepath{stroke,fill}%
\end{pgfscope}%
\begin{pgfscope}%
\pgfpathrectangle{\pgfqpoint{0.100000in}{0.212622in}}{\pgfqpoint{3.696000in}{3.696000in}}%
\pgfusepath{clip}%
\pgfsetbuttcap%
\pgfsetroundjoin%
\definecolor{currentfill}{rgb}{0.121569,0.466667,0.705882}%
\pgfsetfillcolor{currentfill}%
\pgfsetfillopacity{0.507620}%
\pgfsetlinewidth{1.003750pt}%
\definecolor{currentstroke}{rgb}{0.121569,0.466667,0.705882}%
\pgfsetstrokecolor{currentstroke}%
\pgfsetstrokeopacity{0.507620}%
\pgfsetdash{}{0pt}%
\pgfpathmoveto{\pgfqpoint{3.291017in}{1.759087in}}%
\pgfpathcurveto{\pgfqpoint{3.299253in}{1.759087in}}{\pgfqpoint{3.307153in}{1.762359in}}{\pgfqpoint{3.312977in}{1.768183in}}%
\pgfpathcurveto{\pgfqpoint{3.318801in}{1.774007in}}{\pgfqpoint{3.322073in}{1.781907in}}{\pgfqpoint{3.322073in}{1.790143in}}%
\pgfpathcurveto{\pgfqpoint{3.322073in}{1.798380in}}{\pgfqpoint{3.318801in}{1.806280in}}{\pgfqpoint{3.312977in}{1.812104in}}%
\pgfpathcurveto{\pgfqpoint{3.307153in}{1.817928in}}{\pgfqpoint{3.299253in}{1.821200in}}{\pgfqpoint{3.291017in}{1.821200in}}%
\pgfpathcurveto{\pgfqpoint{3.282781in}{1.821200in}}{\pgfqpoint{3.274881in}{1.817928in}}{\pgfqpoint{3.269057in}{1.812104in}}%
\pgfpathcurveto{\pgfqpoint{3.263233in}{1.806280in}}{\pgfqpoint{3.259960in}{1.798380in}}{\pgfqpoint{3.259960in}{1.790143in}}%
\pgfpathcurveto{\pgfqpoint{3.259960in}{1.781907in}}{\pgfqpoint{3.263233in}{1.774007in}}{\pgfqpoint{3.269057in}{1.768183in}}%
\pgfpathcurveto{\pgfqpoint{3.274881in}{1.762359in}}{\pgfqpoint{3.282781in}{1.759087in}}{\pgfqpoint{3.291017in}{1.759087in}}%
\pgfpathclose%
\pgfusepath{stroke,fill}%
\end{pgfscope}%
\begin{pgfscope}%
\pgfpathrectangle{\pgfqpoint{0.100000in}{0.212622in}}{\pgfqpoint{3.696000in}{3.696000in}}%
\pgfusepath{clip}%
\pgfsetbuttcap%
\pgfsetroundjoin%
\definecolor{currentfill}{rgb}{0.121569,0.466667,0.705882}%
\pgfsetfillcolor{currentfill}%
\pgfsetfillopacity{0.508425}%
\pgfsetlinewidth{1.003750pt}%
\definecolor{currentstroke}{rgb}{0.121569,0.466667,0.705882}%
\pgfsetstrokecolor{currentstroke}%
\pgfsetstrokeopacity{0.508425}%
\pgfsetdash{}{0pt}%
\pgfpathmoveto{\pgfqpoint{3.298265in}{1.757801in}}%
\pgfpathcurveto{\pgfqpoint{3.306501in}{1.757801in}}{\pgfqpoint{3.314401in}{1.761074in}}{\pgfqpoint{3.320225in}{1.766898in}}%
\pgfpathcurveto{\pgfqpoint{3.326049in}{1.772721in}}{\pgfqpoint{3.329321in}{1.780621in}}{\pgfqpoint{3.329321in}{1.788858in}}%
\pgfpathcurveto{\pgfqpoint{3.329321in}{1.797094in}}{\pgfqpoint{3.326049in}{1.804994in}}{\pgfqpoint{3.320225in}{1.810818in}}%
\pgfpathcurveto{\pgfqpoint{3.314401in}{1.816642in}}{\pgfqpoint{3.306501in}{1.819914in}}{\pgfqpoint{3.298265in}{1.819914in}}%
\pgfpathcurveto{\pgfqpoint{3.290029in}{1.819914in}}{\pgfqpoint{3.282128in}{1.816642in}}{\pgfqpoint{3.276305in}{1.810818in}}%
\pgfpathcurveto{\pgfqpoint{3.270481in}{1.804994in}}{\pgfqpoint{3.267208in}{1.797094in}}{\pgfqpoint{3.267208in}{1.788858in}}%
\pgfpathcurveto{\pgfqpoint{3.267208in}{1.780621in}}{\pgfqpoint{3.270481in}{1.772721in}}{\pgfqpoint{3.276305in}{1.766898in}}%
\pgfpathcurveto{\pgfqpoint{3.282128in}{1.761074in}}{\pgfqpoint{3.290029in}{1.757801in}}{\pgfqpoint{3.298265in}{1.757801in}}%
\pgfpathclose%
\pgfusepath{stroke,fill}%
\end{pgfscope}%
\begin{pgfscope}%
\pgfpathrectangle{\pgfqpoint{0.100000in}{0.212622in}}{\pgfqpoint{3.696000in}{3.696000in}}%
\pgfusepath{clip}%
\pgfsetbuttcap%
\pgfsetroundjoin%
\definecolor{currentfill}{rgb}{0.121569,0.466667,0.705882}%
\pgfsetfillcolor{currentfill}%
\pgfsetfillopacity{0.509509}%
\pgfsetlinewidth{1.003750pt}%
\definecolor{currentstroke}{rgb}{0.121569,0.466667,0.705882}%
\pgfsetstrokecolor{currentstroke}%
\pgfsetstrokeopacity{0.509509}%
\pgfsetdash{}{0pt}%
\pgfpathmoveto{\pgfqpoint{3.305615in}{1.756784in}}%
\pgfpathcurveto{\pgfqpoint{3.313852in}{1.756784in}}{\pgfqpoint{3.321752in}{1.760057in}}{\pgfqpoint{3.327576in}{1.765881in}}%
\pgfpathcurveto{\pgfqpoint{3.333399in}{1.771705in}}{\pgfqpoint{3.336672in}{1.779605in}}{\pgfqpoint{3.336672in}{1.787841in}}%
\pgfpathcurveto{\pgfqpoint{3.336672in}{1.796077in}}{\pgfqpoint{3.333399in}{1.803977in}}{\pgfqpoint{3.327576in}{1.809801in}}%
\pgfpathcurveto{\pgfqpoint{3.321752in}{1.815625in}}{\pgfqpoint{3.313852in}{1.818897in}}{\pgfqpoint{3.305615in}{1.818897in}}%
\pgfpathcurveto{\pgfqpoint{3.297379in}{1.818897in}}{\pgfqpoint{3.289479in}{1.815625in}}{\pgfqpoint{3.283655in}{1.809801in}}%
\pgfpathcurveto{\pgfqpoint{3.277831in}{1.803977in}}{\pgfqpoint{3.274559in}{1.796077in}}{\pgfqpoint{3.274559in}{1.787841in}}%
\pgfpathcurveto{\pgfqpoint{3.274559in}{1.779605in}}{\pgfqpoint{3.277831in}{1.771705in}}{\pgfqpoint{3.283655in}{1.765881in}}%
\pgfpathcurveto{\pgfqpoint{3.289479in}{1.760057in}}{\pgfqpoint{3.297379in}{1.756784in}}{\pgfqpoint{3.305615in}{1.756784in}}%
\pgfpathclose%
\pgfusepath{stroke,fill}%
\end{pgfscope}%
\begin{pgfscope}%
\pgfpathrectangle{\pgfqpoint{0.100000in}{0.212622in}}{\pgfqpoint{3.696000in}{3.696000in}}%
\pgfusepath{clip}%
\pgfsetbuttcap%
\pgfsetroundjoin%
\definecolor{currentfill}{rgb}{0.121569,0.466667,0.705882}%
\pgfsetfillcolor{currentfill}%
\pgfsetfillopacity{0.510101}%
\pgfsetlinewidth{1.003750pt}%
\definecolor{currentstroke}{rgb}{0.121569,0.466667,0.705882}%
\pgfsetstrokecolor{currentstroke}%
\pgfsetstrokeopacity{0.510101}%
\pgfsetdash{}{0pt}%
\pgfpathmoveto{\pgfqpoint{3.309660in}{1.756200in}}%
\pgfpathcurveto{\pgfqpoint{3.317896in}{1.756200in}}{\pgfqpoint{3.325796in}{1.759473in}}{\pgfqpoint{3.331620in}{1.765297in}}%
\pgfpathcurveto{\pgfqpoint{3.337444in}{1.771121in}}{\pgfqpoint{3.340716in}{1.779021in}}{\pgfqpoint{3.340716in}{1.787257in}}%
\pgfpathcurveto{\pgfqpoint{3.340716in}{1.795493in}}{\pgfqpoint{3.337444in}{1.803393in}}{\pgfqpoint{3.331620in}{1.809217in}}%
\pgfpathcurveto{\pgfqpoint{3.325796in}{1.815041in}}{\pgfqpoint{3.317896in}{1.818313in}}{\pgfqpoint{3.309660in}{1.818313in}}%
\pgfpathcurveto{\pgfqpoint{3.301423in}{1.818313in}}{\pgfqpoint{3.293523in}{1.815041in}}{\pgfqpoint{3.287699in}{1.809217in}}%
\pgfpathcurveto{\pgfqpoint{3.281875in}{1.803393in}}{\pgfqpoint{3.278603in}{1.795493in}}{\pgfqpoint{3.278603in}{1.787257in}}%
\pgfpathcurveto{\pgfqpoint{3.278603in}{1.779021in}}{\pgfqpoint{3.281875in}{1.771121in}}{\pgfqpoint{3.287699in}{1.765297in}}%
\pgfpathcurveto{\pgfqpoint{3.293523in}{1.759473in}}{\pgfqpoint{3.301423in}{1.756200in}}{\pgfqpoint{3.309660in}{1.756200in}}%
\pgfpathclose%
\pgfusepath{stroke,fill}%
\end{pgfscope}%
\begin{pgfscope}%
\pgfpathrectangle{\pgfqpoint{0.100000in}{0.212622in}}{\pgfqpoint{3.696000in}{3.696000in}}%
\pgfusepath{clip}%
\pgfsetbuttcap%
\pgfsetroundjoin%
\definecolor{currentfill}{rgb}{0.121569,0.466667,0.705882}%
\pgfsetfillcolor{currentfill}%
\pgfsetfillopacity{0.510706}%
\pgfsetlinewidth{1.003750pt}%
\definecolor{currentstroke}{rgb}{0.121569,0.466667,0.705882}%
\pgfsetstrokecolor{currentstroke}%
\pgfsetstrokeopacity{0.510706}%
\pgfsetdash{}{0pt}%
\pgfpathmoveto{\pgfqpoint{1.294186in}{2.123590in}}%
\pgfpathcurveto{\pgfqpoint{1.302422in}{2.123590in}}{\pgfqpoint{1.310322in}{2.126862in}}{\pgfqpoint{1.316146in}{2.132686in}}%
\pgfpathcurveto{\pgfqpoint{1.321970in}{2.138510in}}{\pgfqpoint{1.325242in}{2.146410in}}{\pgfqpoint{1.325242in}{2.154646in}}%
\pgfpathcurveto{\pgfqpoint{1.325242in}{2.162883in}}{\pgfqpoint{1.321970in}{2.170783in}}{\pgfqpoint{1.316146in}{2.176607in}}%
\pgfpathcurveto{\pgfqpoint{1.310322in}{2.182431in}}{\pgfqpoint{1.302422in}{2.185703in}}{\pgfqpoint{1.294186in}{2.185703in}}%
\pgfpathcurveto{\pgfqpoint{1.285949in}{2.185703in}}{\pgfqpoint{1.278049in}{2.182431in}}{\pgfqpoint{1.272225in}{2.176607in}}%
\pgfpathcurveto{\pgfqpoint{1.266401in}{2.170783in}}{\pgfqpoint{1.263129in}{2.162883in}}{\pgfqpoint{1.263129in}{2.154646in}}%
\pgfpathcurveto{\pgfqpoint{1.263129in}{2.146410in}}{\pgfqpoint{1.266401in}{2.138510in}}{\pgfqpoint{1.272225in}{2.132686in}}%
\pgfpathcurveto{\pgfqpoint{1.278049in}{2.126862in}}{\pgfqpoint{1.285949in}{2.123590in}}{\pgfqpoint{1.294186in}{2.123590in}}%
\pgfpathclose%
\pgfusepath{stroke,fill}%
\end{pgfscope}%
\begin{pgfscope}%
\pgfpathrectangle{\pgfqpoint{0.100000in}{0.212622in}}{\pgfqpoint{3.696000in}{3.696000in}}%
\pgfusepath{clip}%
\pgfsetbuttcap%
\pgfsetroundjoin%
\definecolor{currentfill}{rgb}{0.121569,0.466667,0.705882}%
\pgfsetfillcolor{currentfill}%
\pgfsetfillopacity{0.510743}%
\pgfsetlinewidth{1.003750pt}%
\definecolor{currentstroke}{rgb}{0.121569,0.466667,0.705882}%
\pgfsetstrokecolor{currentstroke}%
\pgfsetstrokeopacity{0.510743}%
\pgfsetdash{}{0pt}%
\pgfpathmoveto{\pgfqpoint{3.314090in}{1.755433in}}%
\pgfpathcurveto{\pgfqpoint{3.322326in}{1.755433in}}{\pgfqpoint{3.330226in}{1.758706in}}{\pgfqpoint{3.336050in}{1.764530in}}%
\pgfpathcurveto{\pgfqpoint{3.341874in}{1.770354in}}{\pgfqpoint{3.345146in}{1.778254in}}{\pgfqpoint{3.345146in}{1.786490in}}%
\pgfpathcurveto{\pgfqpoint{3.345146in}{1.794726in}}{\pgfqpoint{3.341874in}{1.802626in}}{\pgfqpoint{3.336050in}{1.808450in}}%
\pgfpathcurveto{\pgfqpoint{3.330226in}{1.814274in}}{\pgfqpoint{3.322326in}{1.817546in}}{\pgfqpoint{3.314090in}{1.817546in}}%
\pgfpathcurveto{\pgfqpoint{3.305854in}{1.817546in}}{\pgfqpoint{3.297954in}{1.814274in}}{\pgfqpoint{3.292130in}{1.808450in}}%
\pgfpathcurveto{\pgfqpoint{3.286306in}{1.802626in}}{\pgfqpoint{3.283033in}{1.794726in}}{\pgfqpoint{3.283033in}{1.786490in}}%
\pgfpathcurveto{\pgfqpoint{3.283033in}{1.778254in}}{\pgfqpoint{3.286306in}{1.770354in}}{\pgfqpoint{3.292130in}{1.764530in}}%
\pgfpathcurveto{\pgfqpoint{3.297954in}{1.758706in}}{\pgfqpoint{3.305854in}{1.755433in}}{\pgfqpoint{3.314090in}{1.755433in}}%
\pgfpathclose%
\pgfusepath{stroke,fill}%
\end{pgfscope}%
\begin{pgfscope}%
\pgfpathrectangle{\pgfqpoint{0.100000in}{0.212622in}}{\pgfqpoint{3.696000in}{3.696000in}}%
\pgfusepath{clip}%
\pgfsetbuttcap%
\pgfsetroundjoin%
\definecolor{currentfill}{rgb}{0.121569,0.466667,0.705882}%
\pgfsetfillcolor{currentfill}%
\pgfsetfillopacity{0.510924}%
\pgfsetlinewidth{1.003750pt}%
\definecolor{currentstroke}{rgb}{0.121569,0.466667,0.705882}%
\pgfsetstrokecolor{currentstroke}%
\pgfsetstrokeopacity{0.510924}%
\pgfsetdash{}{0pt}%
\pgfpathmoveto{\pgfqpoint{3.316770in}{1.754873in}}%
\pgfpathcurveto{\pgfqpoint{3.325006in}{1.754873in}}{\pgfqpoint{3.332906in}{1.758145in}}{\pgfqpoint{3.338730in}{1.763969in}}%
\pgfpathcurveto{\pgfqpoint{3.344554in}{1.769793in}}{\pgfqpoint{3.347827in}{1.777693in}}{\pgfqpoint{3.347827in}{1.785929in}}%
\pgfpathcurveto{\pgfqpoint{3.347827in}{1.794166in}}{\pgfqpoint{3.344554in}{1.802066in}}{\pgfqpoint{3.338730in}{1.807890in}}%
\pgfpathcurveto{\pgfqpoint{3.332906in}{1.813714in}}{\pgfqpoint{3.325006in}{1.816986in}}{\pgfqpoint{3.316770in}{1.816986in}}%
\pgfpathcurveto{\pgfqpoint{3.308534in}{1.816986in}}{\pgfqpoint{3.300634in}{1.813714in}}{\pgfqpoint{3.294810in}{1.807890in}}%
\pgfpathcurveto{\pgfqpoint{3.288986in}{1.802066in}}{\pgfqpoint{3.285714in}{1.794166in}}{\pgfqpoint{3.285714in}{1.785929in}}%
\pgfpathcurveto{\pgfqpoint{3.285714in}{1.777693in}}{\pgfqpoint{3.288986in}{1.769793in}}{\pgfqpoint{3.294810in}{1.763969in}}%
\pgfpathcurveto{\pgfqpoint{3.300634in}{1.758145in}}{\pgfqpoint{3.308534in}{1.754873in}}{\pgfqpoint{3.316770in}{1.754873in}}%
\pgfpathclose%
\pgfusepath{stroke,fill}%
\end{pgfscope}%
\begin{pgfscope}%
\pgfpathrectangle{\pgfqpoint{0.100000in}{0.212622in}}{\pgfqpoint{3.696000in}{3.696000in}}%
\pgfusepath{clip}%
\pgfsetbuttcap%
\pgfsetroundjoin%
\definecolor{currentfill}{rgb}{0.121569,0.466667,0.705882}%
\pgfsetfillcolor{currentfill}%
\pgfsetfillopacity{0.511100}%
\pgfsetlinewidth{1.003750pt}%
\definecolor{currentstroke}{rgb}{0.121569,0.466667,0.705882}%
\pgfsetstrokecolor{currentstroke}%
\pgfsetstrokeopacity{0.511100}%
\pgfsetdash{}{0pt}%
\pgfpathmoveto{\pgfqpoint{3.318134in}{1.754593in}}%
\pgfpathcurveto{\pgfqpoint{3.326370in}{1.754593in}}{\pgfqpoint{3.334270in}{1.757865in}}{\pgfqpoint{3.340094in}{1.763689in}}%
\pgfpathcurveto{\pgfqpoint{3.345918in}{1.769513in}}{\pgfqpoint{3.349191in}{1.777413in}}{\pgfqpoint{3.349191in}{1.785649in}}%
\pgfpathcurveto{\pgfqpoint{3.349191in}{1.793885in}}{\pgfqpoint{3.345918in}{1.801785in}}{\pgfqpoint{3.340094in}{1.807609in}}%
\pgfpathcurveto{\pgfqpoint{3.334270in}{1.813433in}}{\pgfqpoint{3.326370in}{1.816706in}}{\pgfqpoint{3.318134in}{1.816706in}}%
\pgfpathcurveto{\pgfqpoint{3.309898in}{1.816706in}}{\pgfqpoint{3.301998in}{1.813433in}}{\pgfqpoint{3.296174in}{1.807609in}}%
\pgfpathcurveto{\pgfqpoint{3.290350in}{1.801785in}}{\pgfqpoint{3.287078in}{1.793885in}}{\pgfqpoint{3.287078in}{1.785649in}}%
\pgfpathcurveto{\pgfqpoint{3.287078in}{1.777413in}}{\pgfqpoint{3.290350in}{1.769513in}}{\pgfqpoint{3.296174in}{1.763689in}}%
\pgfpathcurveto{\pgfqpoint{3.301998in}{1.757865in}}{\pgfqpoint{3.309898in}{1.754593in}}{\pgfqpoint{3.318134in}{1.754593in}}%
\pgfpathclose%
\pgfusepath{stroke,fill}%
\end{pgfscope}%
\begin{pgfscope}%
\pgfpathrectangle{\pgfqpoint{0.100000in}{0.212622in}}{\pgfqpoint{3.696000in}{3.696000in}}%
\pgfusepath{clip}%
\pgfsetbuttcap%
\pgfsetroundjoin%
\definecolor{currentfill}{rgb}{0.121569,0.466667,0.705882}%
\pgfsetfillcolor{currentfill}%
\pgfsetfillopacity{0.511215}%
\pgfsetlinewidth{1.003750pt}%
\definecolor{currentstroke}{rgb}{0.121569,0.466667,0.705882}%
\pgfsetstrokecolor{currentstroke}%
\pgfsetstrokeopacity{0.511215}%
\pgfsetdash{}{0pt}%
\pgfpathmoveto{\pgfqpoint{3.318846in}{1.754451in}}%
\pgfpathcurveto{\pgfqpoint{3.327083in}{1.754451in}}{\pgfqpoint{3.334983in}{1.757723in}}{\pgfqpoint{3.340807in}{1.763547in}}%
\pgfpathcurveto{\pgfqpoint{3.346631in}{1.769371in}}{\pgfqpoint{3.349903in}{1.777271in}}{\pgfqpoint{3.349903in}{1.785507in}}%
\pgfpathcurveto{\pgfqpoint{3.349903in}{1.793744in}}{\pgfqpoint{3.346631in}{1.801644in}}{\pgfqpoint{3.340807in}{1.807468in}}%
\pgfpathcurveto{\pgfqpoint{3.334983in}{1.813292in}}{\pgfqpoint{3.327083in}{1.816564in}}{\pgfqpoint{3.318846in}{1.816564in}}%
\pgfpathcurveto{\pgfqpoint{3.310610in}{1.816564in}}{\pgfqpoint{3.302710in}{1.813292in}}{\pgfqpoint{3.296886in}{1.807468in}}%
\pgfpathcurveto{\pgfqpoint{3.291062in}{1.801644in}}{\pgfqpoint{3.287790in}{1.793744in}}{\pgfqpoint{3.287790in}{1.785507in}}%
\pgfpathcurveto{\pgfqpoint{3.287790in}{1.777271in}}{\pgfqpoint{3.291062in}{1.769371in}}{\pgfqpoint{3.296886in}{1.763547in}}%
\pgfpathcurveto{\pgfqpoint{3.302710in}{1.757723in}}{\pgfqpoint{3.310610in}{1.754451in}}{\pgfqpoint{3.318846in}{1.754451in}}%
\pgfpathclose%
\pgfusepath{stroke,fill}%
\end{pgfscope}%
\begin{pgfscope}%
\pgfpathrectangle{\pgfqpoint{0.100000in}{0.212622in}}{\pgfqpoint{3.696000in}{3.696000in}}%
\pgfusepath{clip}%
\pgfsetbuttcap%
\pgfsetroundjoin%
\definecolor{currentfill}{rgb}{0.121569,0.466667,0.705882}%
\pgfsetfillcolor{currentfill}%
\pgfsetfillopacity{0.511272}%
\pgfsetlinewidth{1.003750pt}%
\definecolor{currentstroke}{rgb}{0.121569,0.466667,0.705882}%
\pgfsetstrokecolor{currentstroke}%
\pgfsetstrokeopacity{0.511272}%
\pgfsetdash{}{0pt}%
\pgfpathmoveto{\pgfqpoint{3.319254in}{1.754375in}}%
\pgfpathcurveto{\pgfqpoint{3.327491in}{1.754375in}}{\pgfqpoint{3.335391in}{1.757647in}}{\pgfqpoint{3.341215in}{1.763471in}}%
\pgfpathcurveto{\pgfqpoint{3.347039in}{1.769295in}}{\pgfqpoint{3.350311in}{1.777195in}}{\pgfqpoint{3.350311in}{1.785431in}}%
\pgfpathcurveto{\pgfqpoint{3.350311in}{1.793667in}}{\pgfqpoint{3.347039in}{1.801567in}}{\pgfqpoint{3.341215in}{1.807391in}}%
\pgfpathcurveto{\pgfqpoint{3.335391in}{1.813215in}}{\pgfqpoint{3.327491in}{1.816488in}}{\pgfqpoint{3.319254in}{1.816488in}}%
\pgfpathcurveto{\pgfqpoint{3.311018in}{1.816488in}}{\pgfqpoint{3.303118in}{1.813215in}}{\pgfqpoint{3.297294in}{1.807391in}}%
\pgfpathcurveto{\pgfqpoint{3.291470in}{1.801567in}}{\pgfqpoint{3.288198in}{1.793667in}}{\pgfqpoint{3.288198in}{1.785431in}}%
\pgfpathcurveto{\pgfqpoint{3.288198in}{1.777195in}}{\pgfqpoint{3.291470in}{1.769295in}}{\pgfqpoint{3.297294in}{1.763471in}}%
\pgfpathcurveto{\pgfqpoint{3.303118in}{1.757647in}}{\pgfqpoint{3.311018in}{1.754375in}}{\pgfqpoint{3.319254in}{1.754375in}}%
\pgfpathclose%
\pgfusepath{stroke,fill}%
\end{pgfscope}%
\begin{pgfscope}%
\pgfpathrectangle{\pgfqpoint{0.100000in}{0.212622in}}{\pgfqpoint{3.696000in}{3.696000in}}%
\pgfusepath{clip}%
\pgfsetbuttcap%
\pgfsetroundjoin%
\definecolor{currentfill}{rgb}{0.121569,0.466667,0.705882}%
\pgfsetfillcolor{currentfill}%
\pgfsetfillopacity{0.511306}%
\pgfsetlinewidth{1.003750pt}%
\definecolor{currentstroke}{rgb}{0.121569,0.466667,0.705882}%
\pgfsetstrokecolor{currentstroke}%
\pgfsetstrokeopacity{0.511306}%
\pgfsetdash{}{0pt}%
\pgfpathmoveto{\pgfqpoint{3.319473in}{1.754335in}}%
\pgfpathcurveto{\pgfqpoint{3.327709in}{1.754335in}}{\pgfqpoint{3.335609in}{1.757607in}}{\pgfqpoint{3.341433in}{1.763431in}}%
\pgfpathcurveto{\pgfqpoint{3.347257in}{1.769255in}}{\pgfqpoint{3.350529in}{1.777155in}}{\pgfqpoint{3.350529in}{1.785391in}}%
\pgfpathcurveto{\pgfqpoint{3.350529in}{1.793627in}}{\pgfqpoint{3.347257in}{1.801528in}}{\pgfqpoint{3.341433in}{1.807351in}}%
\pgfpathcurveto{\pgfqpoint{3.335609in}{1.813175in}}{\pgfqpoint{3.327709in}{1.816448in}}{\pgfqpoint{3.319473in}{1.816448in}}%
\pgfpathcurveto{\pgfqpoint{3.311237in}{1.816448in}}{\pgfqpoint{3.303337in}{1.813175in}}{\pgfqpoint{3.297513in}{1.807351in}}%
\pgfpathcurveto{\pgfqpoint{3.291689in}{1.801528in}}{\pgfqpoint{3.288416in}{1.793627in}}{\pgfqpoint{3.288416in}{1.785391in}}%
\pgfpathcurveto{\pgfqpoint{3.288416in}{1.777155in}}{\pgfqpoint{3.291689in}{1.769255in}}{\pgfqpoint{3.297513in}{1.763431in}}%
\pgfpathcurveto{\pgfqpoint{3.303337in}{1.757607in}}{\pgfqpoint{3.311237in}{1.754335in}}{\pgfqpoint{3.319473in}{1.754335in}}%
\pgfpathclose%
\pgfusepath{stroke,fill}%
\end{pgfscope}%
\begin{pgfscope}%
\pgfpathrectangle{\pgfqpoint{0.100000in}{0.212622in}}{\pgfqpoint{3.696000in}{3.696000in}}%
\pgfusepath{clip}%
\pgfsetbuttcap%
\pgfsetroundjoin%
\definecolor{currentfill}{rgb}{0.121569,0.466667,0.705882}%
\pgfsetfillcolor{currentfill}%
\pgfsetfillopacity{0.511324}%
\pgfsetlinewidth{1.003750pt}%
\definecolor{currentstroke}{rgb}{0.121569,0.466667,0.705882}%
\pgfsetstrokecolor{currentstroke}%
\pgfsetstrokeopacity{0.511324}%
\pgfsetdash{}{0pt}%
\pgfpathmoveto{\pgfqpoint{3.319596in}{1.754313in}}%
\pgfpathcurveto{\pgfqpoint{3.327832in}{1.754313in}}{\pgfqpoint{3.335732in}{1.757586in}}{\pgfqpoint{3.341556in}{1.763410in}}%
\pgfpathcurveto{\pgfqpoint{3.347380in}{1.769234in}}{\pgfqpoint{3.350652in}{1.777134in}}{\pgfqpoint{3.350652in}{1.785370in}}%
\pgfpathcurveto{\pgfqpoint{3.350652in}{1.793606in}}{\pgfqpoint{3.347380in}{1.801506in}}{\pgfqpoint{3.341556in}{1.807330in}}%
\pgfpathcurveto{\pgfqpoint{3.335732in}{1.813154in}}{\pgfqpoint{3.327832in}{1.816426in}}{\pgfqpoint{3.319596in}{1.816426in}}%
\pgfpathcurveto{\pgfqpoint{3.311359in}{1.816426in}}{\pgfqpoint{3.303459in}{1.813154in}}{\pgfqpoint{3.297635in}{1.807330in}}%
\pgfpathcurveto{\pgfqpoint{3.291812in}{1.801506in}}{\pgfqpoint{3.288539in}{1.793606in}}{\pgfqpoint{3.288539in}{1.785370in}}%
\pgfpathcurveto{\pgfqpoint{3.288539in}{1.777134in}}{\pgfqpoint{3.291812in}{1.769234in}}{\pgfqpoint{3.297635in}{1.763410in}}%
\pgfpathcurveto{\pgfqpoint{3.303459in}{1.757586in}}{\pgfqpoint{3.311359in}{1.754313in}}{\pgfqpoint{3.319596in}{1.754313in}}%
\pgfpathclose%
\pgfusepath{stroke,fill}%
\end{pgfscope}%
\begin{pgfscope}%
\pgfpathrectangle{\pgfqpoint{0.100000in}{0.212622in}}{\pgfqpoint{3.696000in}{3.696000in}}%
\pgfusepath{clip}%
\pgfsetbuttcap%
\pgfsetroundjoin%
\definecolor{currentfill}{rgb}{0.121569,0.466667,0.705882}%
\pgfsetfillcolor{currentfill}%
\pgfsetfillopacity{0.511331}%
\pgfsetlinewidth{1.003750pt}%
\definecolor{currentstroke}{rgb}{0.121569,0.466667,0.705882}%
\pgfsetstrokecolor{currentstroke}%
\pgfsetstrokeopacity{0.511331}%
\pgfsetdash{}{0pt}%
\pgfpathmoveto{\pgfqpoint{3.319668in}{1.754299in}}%
\pgfpathcurveto{\pgfqpoint{3.327904in}{1.754299in}}{\pgfqpoint{3.335804in}{1.757572in}}{\pgfqpoint{3.341628in}{1.763395in}}%
\pgfpathcurveto{\pgfqpoint{3.347452in}{1.769219in}}{\pgfqpoint{3.350724in}{1.777119in}}{\pgfqpoint{3.350724in}{1.785356in}}%
\pgfpathcurveto{\pgfqpoint{3.350724in}{1.793592in}}{\pgfqpoint{3.347452in}{1.801492in}}{\pgfqpoint{3.341628in}{1.807316in}}%
\pgfpathcurveto{\pgfqpoint{3.335804in}{1.813140in}}{\pgfqpoint{3.327904in}{1.816412in}}{\pgfqpoint{3.319668in}{1.816412in}}%
\pgfpathcurveto{\pgfqpoint{3.311432in}{1.816412in}}{\pgfqpoint{3.303532in}{1.813140in}}{\pgfqpoint{3.297708in}{1.807316in}}%
\pgfpathcurveto{\pgfqpoint{3.291884in}{1.801492in}}{\pgfqpoint{3.288611in}{1.793592in}}{\pgfqpoint{3.288611in}{1.785356in}}%
\pgfpathcurveto{\pgfqpoint{3.288611in}{1.777119in}}{\pgfqpoint{3.291884in}{1.769219in}}{\pgfqpoint{3.297708in}{1.763395in}}%
\pgfpathcurveto{\pgfqpoint{3.303532in}{1.757572in}}{\pgfqpoint{3.311432in}{1.754299in}}{\pgfqpoint{3.319668in}{1.754299in}}%
\pgfpathclose%
\pgfusepath{stroke,fill}%
\end{pgfscope}%
\begin{pgfscope}%
\pgfpathrectangle{\pgfqpoint{0.100000in}{0.212622in}}{\pgfqpoint{3.696000in}{3.696000in}}%
\pgfusepath{clip}%
\pgfsetbuttcap%
\pgfsetroundjoin%
\definecolor{currentfill}{rgb}{0.121569,0.466667,0.705882}%
\pgfsetfillcolor{currentfill}%
\pgfsetfillopacity{0.511335}%
\pgfsetlinewidth{1.003750pt}%
\definecolor{currentstroke}{rgb}{0.121569,0.466667,0.705882}%
\pgfsetstrokecolor{currentstroke}%
\pgfsetstrokeopacity{0.511335}%
\pgfsetdash{}{0pt}%
\pgfpathmoveto{\pgfqpoint{3.319706in}{1.754292in}}%
\pgfpathcurveto{\pgfqpoint{3.327943in}{1.754292in}}{\pgfqpoint{3.335843in}{1.757564in}}{\pgfqpoint{3.341667in}{1.763388in}}%
\pgfpathcurveto{\pgfqpoint{3.347491in}{1.769212in}}{\pgfqpoint{3.350763in}{1.777112in}}{\pgfqpoint{3.350763in}{1.785349in}}%
\pgfpathcurveto{\pgfqpoint{3.350763in}{1.793585in}}{\pgfqpoint{3.347491in}{1.801485in}}{\pgfqpoint{3.341667in}{1.807309in}}%
\pgfpathcurveto{\pgfqpoint{3.335843in}{1.813133in}}{\pgfqpoint{3.327943in}{1.816405in}}{\pgfqpoint{3.319706in}{1.816405in}}%
\pgfpathcurveto{\pgfqpoint{3.311470in}{1.816405in}}{\pgfqpoint{3.303570in}{1.813133in}}{\pgfqpoint{3.297746in}{1.807309in}}%
\pgfpathcurveto{\pgfqpoint{3.291922in}{1.801485in}}{\pgfqpoint{3.288650in}{1.793585in}}{\pgfqpoint{3.288650in}{1.785349in}}%
\pgfpathcurveto{\pgfqpoint{3.288650in}{1.777112in}}{\pgfqpoint{3.291922in}{1.769212in}}{\pgfqpoint{3.297746in}{1.763388in}}%
\pgfpathcurveto{\pgfqpoint{3.303570in}{1.757564in}}{\pgfqpoint{3.311470in}{1.754292in}}{\pgfqpoint{3.319706in}{1.754292in}}%
\pgfpathclose%
\pgfusepath{stroke,fill}%
\end{pgfscope}%
\begin{pgfscope}%
\pgfpathrectangle{\pgfqpoint{0.100000in}{0.212622in}}{\pgfqpoint{3.696000in}{3.696000in}}%
\pgfusepath{clip}%
\pgfsetbuttcap%
\pgfsetroundjoin%
\definecolor{currentfill}{rgb}{0.121569,0.466667,0.705882}%
\pgfsetfillcolor{currentfill}%
\pgfsetfillopacity{0.511339}%
\pgfsetlinewidth{1.003750pt}%
\definecolor{currentstroke}{rgb}{0.121569,0.466667,0.705882}%
\pgfsetstrokecolor{currentstroke}%
\pgfsetstrokeopacity{0.511339}%
\pgfsetdash{}{0pt}%
\pgfpathmoveto{\pgfqpoint{3.319726in}{1.754289in}}%
\pgfpathcurveto{\pgfqpoint{3.327962in}{1.754289in}}{\pgfqpoint{3.335862in}{1.757561in}}{\pgfqpoint{3.341686in}{1.763385in}}%
\pgfpathcurveto{\pgfqpoint{3.347510in}{1.769209in}}{\pgfqpoint{3.350782in}{1.777109in}}{\pgfqpoint{3.350782in}{1.785345in}}%
\pgfpathcurveto{\pgfqpoint{3.350782in}{1.793582in}}{\pgfqpoint{3.347510in}{1.801482in}}{\pgfqpoint{3.341686in}{1.807306in}}%
\pgfpathcurveto{\pgfqpoint{3.335862in}{1.813129in}}{\pgfqpoint{3.327962in}{1.816402in}}{\pgfqpoint{3.319726in}{1.816402in}}%
\pgfpathcurveto{\pgfqpoint{3.311489in}{1.816402in}}{\pgfqpoint{3.303589in}{1.813129in}}{\pgfqpoint{3.297765in}{1.807306in}}%
\pgfpathcurveto{\pgfqpoint{3.291941in}{1.801482in}}{\pgfqpoint{3.288669in}{1.793582in}}{\pgfqpoint{3.288669in}{1.785345in}}%
\pgfpathcurveto{\pgfqpoint{3.288669in}{1.777109in}}{\pgfqpoint{3.291941in}{1.769209in}}{\pgfqpoint{3.297765in}{1.763385in}}%
\pgfpathcurveto{\pgfqpoint{3.303589in}{1.757561in}}{\pgfqpoint{3.311489in}{1.754289in}}{\pgfqpoint{3.319726in}{1.754289in}}%
\pgfpathclose%
\pgfusepath{stroke,fill}%
\end{pgfscope}%
\begin{pgfscope}%
\pgfpathrectangle{\pgfqpoint{0.100000in}{0.212622in}}{\pgfqpoint{3.696000in}{3.696000in}}%
\pgfusepath{clip}%
\pgfsetbuttcap%
\pgfsetroundjoin%
\definecolor{currentfill}{rgb}{0.121569,0.466667,0.705882}%
\pgfsetfillcolor{currentfill}%
\pgfsetfillopacity{0.511341}%
\pgfsetlinewidth{1.003750pt}%
\definecolor{currentstroke}{rgb}{0.121569,0.466667,0.705882}%
\pgfsetstrokecolor{currentstroke}%
\pgfsetstrokeopacity{0.511341}%
\pgfsetdash{}{0pt}%
\pgfpathmoveto{\pgfqpoint{3.319735in}{1.754287in}}%
\pgfpathcurveto{\pgfqpoint{3.327971in}{1.754287in}}{\pgfqpoint{3.335871in}{1.757559in}}{\pgfqpoint{3.341695in}{1.763383in}}%
\pgfpathcurveto{\pgfqpoint{3.347519in}{1.769207in}}{\pgfqpoint{3.350791in}{1.777107in}}{\pgfqpoint{3.350791in}{1.785343in}}%
\pgfpathcurveto{\pgfqpoint{3.350791in}{1.793580in}}{\pgfqpoint{3.347519in}{1.801480in}}{\pgfqpoint{3.341695in}{1.807304in}}%
\pgfpathcurveto{\pgfqpoint{3.335871in}{1.813128in}}{\pgfqpoint{3.327971in}{1.816400in}}{\pgfqpoint{3.319735in}{1.816400in}}%
\pgfpathcurveto{\pgfqpoint{3.311498in}{1.816400in}}{\pgfqpoint{3.303598in}{1.813128in}}{\pgfqpoint{3.297774in}{1.807304in}}%
\pgfpathcurveto{\pgfqpoint{3.291950in}{1.801480in}}{\pgfqpoint{3.288678in}{1.793580in}}{\pgfqpoint{3.288678in}{1.785343in}}%
\pgfpathcurveto{\pgfqpoint{3.288678in}{1.777107in}}{\pgfqpoint{3.291950in}{1.769207in}}{\pgfqpoint{3.297774in}{1.763383in}}%
\pgfpathcurveto{\pgfqpoint{3.303598in}{1.757559in}}{\pgfqpoint{3.311498in}{1.754287in}}{\pgfqpoint{3.319735in}{1.754287in}}%
\pgfpathclose%
\pgfusepath{stroke,fill}%
\end{pgfscope}%
\begin{pgfscope}%
\pgfpathrectangle{\pgfqpoint{0.100000in}{0.212622in}}{\pgfqpoint{3.696000in}{3.696000in}}%
\pgfusepath{clip}%
\pgfsetbuttcap%
\pgfsetroundjoin%
\definecolor{currentfill}{rgb}{0.121569,0.466667,0.705882}%
\pgfsetfillcolor{currentfill}%
\pgfsetfillopacity{0.511343}%
\pgfsetlinewidth{1.003750pt}%
\definecolor{currentstroke}{rgb}{0.121569,0.466667,0.705882}%
\pgfsetstrokecolor{currentstroke}%
\pgfsetstrokeopacity{0.511343}%
\pgfsetdash{}{0pt}%
\pgfpathmoveto{\pgfqpoint{3.319738in}{1.754287in}}%
\pgfpathcurveto{\pgfqpoint{3.327974in}{1.754287in}}{\pgfqpoint{3.335874in}{1.757559in}}{\pgfqpoint{3.341698in}{1.763383in}}%
\pgfpathcurveto{\pgfqpoint{3.347522in}{1.769207in}}{\pgfqpoint{3.350794in}{1.777107in}}{\pgfqpoint{3.350794in}{1.785343in}}%
\pgfpathcurveto{\pgfqpoint{3.350794in}{1.793579in}}{\pgfqpoint{3.347522in}{1.801479in}}{\pgfqpoint{3.341698in}{1.807303in}}%
\pgfpathcurveto{\pgfqpoint{3.335874in}{1.813127in}}{\pgfqpoint{3.327974in}{1.816400in}}{\pgfqpoint{3.319738in}{1.816400in}}%
\pgfpathcurveto{\pgfqpoint{3.311502in}{1.816400in}}{\pgfqpoint{3.303602in}{1.813127in}}{\pgfqpoint{3.297778in}{1.807303in}}%
\pgfpathcurveto{\pgfqpoint{3.291954in}{1.801479in}}{\pgfqpoint{3.288681in}{1.793579in}}{\pgfqpoint{3.288681in}{1.785343in}}%
\pgfpathcurveto{\pgfqpoint{3.288681in}{1.777107in}}{\pgfqpoint{3.291954in}{1.769207in}}{\pgfqpoint{3.297778in}{1.763383in}}%
\pgfpathcurveto{\pgfqpoint{3.303602in}{1.757559in}}{\pgfqpoint{3.311502in}{1.754287in}}{\pgfqpoint{3.319738in}{1.754287in}}%
\pgfpathclose%
\pgfusepath{stroke,fill}%
\end{pgfscope}%
\begin{pgfscope}%
\pgfpathrectangle{\pgfqpoint{0.100000in}{0.212622in}}{\pgfqpoint{3.696000in}{3.696000in}}%
\pgfusepath{clip}%
\pgfsetbuttcap%
\pgfsetroundjoin%
\definecolor{currentfill}{rgb}{0.121569,0.466667,0.705882}%
\pgfsetfillcolor{currentfill}%
\pgfsetfillopacity{0.511441}%
\pgfsetlinewidth{1.003750pt}%
\definecolor{currentstroke}{rgb}{0.121569,0.466667,0.705882}%
\pgfsetstrokecolor{currentstroke}%
\pgfsetstrokeopacity{0.511441}%
\pgfsetdash{}{0pt}%
\pgfpathmoveto{\pgfqpoint{3.319819in}{1.754283in}}%
\pgfpathcurveto{\pgfqpoint{3.328055in}{1.754283in}}{\pgfqpoint{3.335955in}{1.757556in}}{\pgfqpoint{3.341779in}{1.763380in}}%
\pgfpathcurveto{\pgfqpoint{3.347603in}{1.769204in}}{\pgfqpoint{3.350876in}{1.777104in}}{\pgfqpoint{3.350876in}{1.785340in}}%
\pgfpathcurveto{\pgfqpoint{3.350876in}{1.793576in}}{\pgfqpoint{3.347603in}{1.801476in}}{\pgfqpoint{3.341779in}{1.807300in}}%
\pgfpathcurveto{\pgfqpoint{3.335955in}{1.813124in}}{\pgfqpoint{3.328055in}{1.816396in}}{\pgfqpoint{3.319819in}{1.816396in}}%
\pgfpathcurveto{\pgfqpoint{3.311583in}{1.816396in}}{\pgfqpoint{3.303683in}{1.813124in}}{\pgfqpoint{3.297859in}{1.807300in}}%
\pgfpathcurveto{\pgfqpoint{3.292035in}{1.801476in}}{\pgfqpoint{3.288763in}{1.793576in}}{\pgfqpoint{3.288763in}{1.785340in}}%
\pgfpathcurveto{\pgfqpoint{3.288763in}{1.777104in}}{\pgfqpoint{3.292035in}{1.769204in}}{\pgfqpoint{3.297859in}{1.763380in}}%
\pgfpathcurveto{\pgfqpoint{3.303683in}{1.757556in}}{\pgfqpoint{3.311583in}{1.754283in}}{\pgfqpoint{3.319819in}{1.754283in}}%
\pgfpathclose%
\pgfusepath{stroke,fill}%
\end{pgfscope}%
\begin{pgfscope}%
\pgfpathrectangle{\pgfqpoint{0.100000in}{0.212622in}}{\pgfqpoint{3.696000in}{3.696000in}}%
\pgfusepath{clip}%
\pgfsetbuttcap%
\pgfsetroundjoin%
\definecolor{currentfill}{rgb}{0.121569,0.466667,0.705882}%
\pgfsetfillcolor{currentfill}%
\pgfsetfillopacity{0.511495}%
\pgfsetlinewidth{1.003750pt}%
\definecolor{currentstroke}{rgb}{0.121569,0.466667,0.705882}%
\pgfsetstrokecolor{currentstroke}%
\pgfsetstrokeopacity{0.511495}%
\pgfsetdash{}{0pt}%
\pgfpathmoveto{\pgfqpoint{3.319852in}{1.754278in}}%
\pgfpathcurveto{\pgfqpoint{3.328088in}{1.754278in}}{\pgfqpoint{3.335988in}{1.757551in}}{\pgfqpoint{3.341812in}{1.763374in}}%
\pgfpathcurveto{\pgfqpoint{3.347636in}{1.769198in}}{\pgfqpoint{3.350909in}{1.777098in}}{\pgfqpoint{3.350909in}{1.785335in}}%
\pgfpathcurveto{\pgfqpoint{3.350909in}{1.793571in}}{\pgfqpoint{3.347636in}{1.801471in}}{\pgfqpoint{3.341812in}{1.807295in}}%
\pgfpathcurveto{\pgfqpoint{3.335988in}{1.813119in}}{\pgfqpoint{3.328088in}{1.816391in}}{\pgfqpoint{3.319852in}{1.816391in}}%
\pgfpathcurveto{\pgfqpoint{3.311616in}{1.816391in}}{\pgfqpoint{3.303716in}{1.813119in}}{\pgfqpoint{3.297892in}{1.807295in}}%
\pgfpathcurveto{\pgfqpoint{3.292068in}{1.801471in}}{\pgfqpoint{3.288796in}{1.793571in}}{\pgfqpoint{3.288796in}{1.785335in}}%
\pgfpathcurveto{\pgfqpoint{3.288796in}{1.777098in}}{\pgfqpoint{3.292068in}{1.769198in}}{\pgfqpoint{3.297892in}{1.763374in}}%
\pgfpathcurveto{\pgfqpoint{3.303716in}{1.757551in}}{\pgfqpoint{3.311616in}{1.754278in}}{\pgfqpoint{3.319852in}{1.754278in}}%
\pgfpathclose%
\pgfusepath{stroke,fill}%
\end{pgfscope}%
\begin{pgfscope}%
\pgfpathrectangle{\pgfqpoint{0.100000in}{0.212622in}}{\pgfqpoint{3.696000in}{3.696000in}}%
\pgfusepath{clip}%
\pgfsetbuttcap%
\pgfsetroundjoin%
\definecolor{currentfill}{rgb}{0.121569,0.466667,0.705882}%
\pgfsetfillcolor{currentfill}%
\pgfsetfillopacity{0.511715}%
\pgfsetlinewidth{1.003750pt}%
\definecolor{currentstroke}{rgb}{0.121569,0.466667,0.705882}%
\pgfsetstrokecolor{currentstroke}%
\pgfsetstrokeopacity{0.511715}%
\pgfsetdash{}{0pt}%
\pgfpathmoveto{\pgfqpoint{3.319895in}{1.754251in}}%
\pgfpathcurveto{\pgfqpoint{3.328131in}{1.754251in}}{\pgfqpoint{3.336031in}{1.757523in}}{\pgfqpoint{3.341855in}{1.763347in}}%
\pgfpathcurveto{\pgfqpoint{3.347679in}{1.769171in}}{\pgfqpoint{3.350952in}{1.777071in}}{\pgfqpoint{3.350952in}{1.785307in}}%
\pgfpathcurveto{\pgfqpoint{3.350952in}{1.793543in}}{\pgfqpoint{3.347679in}{1.801443in}}{\pgfqpoint{3.341855in}{1.807267in}}%
\pgfpathcurveto{\pgfqpoint{3.336031in}{1.813091in}}{\pgfqpoint{3.328131in}{1.816364in}}{\pgfqpoint{3.319895in}{1.816364in}}%
\pgfpathcurveto{\pgfqpoint{3.311659in}{1.816364in}}{\pgfqpoint{3.303759in}{1.813091in}}{\pgfqpoint{3.297935in}{1.807267in}}%
\pgfpathcurveto{\pgfqpoint{3.292111in}{1.801443in}}{\pgfqpoint{3.288839in}{1.793543in}}{\pgfqpoint{3.288839in}{1.785307in}}%
\pgfpathcurveto{\pgfqpoint{3.288839in}{1.777071in}}{\pgfqpoint{3.292111in}{1.769171in}}{\pgfqpoint{3.297935in}{1.763347in}}%
\pgfpathcurveto{\pgfqpoint{3.303759in}{1.757523in}}{\pgfqpoint{3.311659in}{1.754251in}}{\pgfqpoint{3.319895in}{1.754251in}}%
\pgfpathclose%
\pgfusepath{stroke,fill}%
\end{pgfscope}%
\begin{pgfscope}%
\pgfpathrectangle{\pgfqpoint{0.100000in}{0.212622in}}{\pgfqpoint{3.696000in}{3.696000in}}%
\pgfusepath{clip}%
\pgfsetbuttcap%
\pgfsetroundjoin%
\definecolor{currentfill}{rgb}{0.121569,0.466667,0.705882}%
\pgfsetfillcolor{currentfill}%
\pgfsetfillopacity{0.512202}%
\pgfsetlinewidth{1.003750pt}%
\definecolor{currentstroke}{rgb}{0.121569,0.466667,0.705882}%
\pgfsetstrokecolor{currentstroke}%
\pgfsetstrokeopacity{0.512202}%
\pgfsetdash{}{0pt}%
\pgfpathmoveto{\pgfqpoint{3.319891in}{1.754126in}}%
\pgfpathcurveto{\pgfqpoint{3.328127in}{1.754126in}}{\pgfqpoint{3.336027in}{1.757398in}}{\pgfqpoint{3.341851in}{1.763222in}}%
\pgfpathcurveto{\pgfqpoint{3.347675in}{1.769046in}}{\pgfqpoint{3.350948in}{1.776946in}}{\pgfqpoint{3.350948in}{1.785182in}}%
\pgfpathcurveto{\pgfqpoint{3.350948in}{1.793418in}}{\pgfqpoint{3.347675in}{1.801318in}}{\pgfqpoint{3.341851in}{1.807142in}}%
\pgfpathcurveto{\pgfqpoint{3.336027in}{1.812966in}}{\pgfqpoint{3.328127in}{1.816239in}}{\pgfqpoint{3.319891in}{1.816239in}}%
\pgfpathcurveto{\pgfqpoint{3.311655in}{1.816239in}}{\pgfqpoint{3.303755in}{1.812966in}}{\pgfqpoint{3.297931in}{1.807142in}}%
\pgfpathcurveto{\pgfqpoint{3.292107in}{1.801318in}}{\pgfqpoint{3.288835in}{1.793418in}}{\pgfqpoint{3.288835in}{1.785182in}}%
\pgfpathcurveto{\pgfqpoint{3.288835in}{1.776946in}}{\pgfqpoint{3.292107in}{1.769046in}}{\pgfqpoint{3.297931in}{1.763222in}}%
\pgfpathcurveto{\pgfqpoint{3.303755in}{1.757398in}}{\pgfqpoint{3.311655in}{1.754126in}}{\pgfqpoint{3.319891in}{1.754126in}}%
\pgfpathclose%
\pgfusepath{stroke,fill}%
\end{pgfscope}%
\begin{pgfscope}%
\pgfpathrectangle{\pgfqpoint{0.100000in}{0.212622in}}{\pgfqpoint{3.696000in}{3.696000in}}%
\pgfusepath{clip}%
\pgfsetbuttcap%
\pgfsetroundjoin%
\definecolor{currentfill}{rgb}{0.121569,0.466667,0.705882}%
\pgfsetfillcolor{currentfill}%
\pgfsetfillopacity{0.512789}%
\pgfsetlinewidth{1.003750pt}%
\definecolor{currentstroke}{rgb}{0.121569,0.466667,0.705882}%
\pgfsetstrokecolor{currentstroke}%
\pgfsetstrokeopacity{0.512789}%
\pgfsetdash{}{0pt}%
\pgfpathmoveto{\pgfqpoint{3.319505in}{1.754046in}}%
\pgfpathcurveto{\pgfqpoint{3.327742in}{1.754046in}}{\pgfqpoint{3.335642in}{1.757318in}}{\pgfqpoint{3.341466in}{1.763142in}}%
\pgfpathcurveto{\pgfqpoint{3.347290in}{1.768966in}}{\pgfqpoint{3.350562in}{1.776866in}}{\pgfqpoint{3.350562in}{1.785102in}}%
\pgfpathcurveto{\pgfqpoint{3.350562in}{1.793338in}}{\pgfqpoint{3.347290in}{1.801238in}}{\pgfqpoint{3.341466in}{1.807062in}}%
\pgfpathcurveto{\pgfqpoint{3.335642in}{1.812886in}}{\pgfqpoint{3.327742in}{1.816159in}}{\pgfqpoint{3.319505in}{1.816159in}}%
\pgfpathcurveto{\pgfqpoint{3.311269in}{1.816159in}}{\pgfqpoint{3.303369in}{1.812886in}}{\pgfqpoint{3.297545in}{1.807062in}}%
\pgfpathcurveto{\pgfqpoint{3.291721in}{1.801238in}}{\pgfqpoint{3.288449in}{1.793338in}}{\pgfqpoint{3.288449in}{1.785102in}}%
\pgfpathcurveto{\pgfqpoint{3.288449in}{1.776866in}}{\pgfqpoint{3.291721in}{1.768966in}}{\pgfqpoint{3.297545in}{1.763142in}}%
\pgfpathcurveto{\pgfqpoint{3.303369in}{1.757318in}}{\pgfqpoint{3.311269in}{1.754046in}}{\pgfqpoint{3.319505in}{1.754046in}}%
\pgfpathclose%
\pgfusepath{stroke,fill}%
\end{pgfscope}%
\begin{pgfscope}%
\pgfpathrectangle{\pgfqpoint{0.100000in}{0.212622in}}{\pgfqpoint{3.696000in}{3.696000in}}%
\pgfusepath{clip}%
\pgfsetbuttcap%
\pgfsetroundjoin%
\definecolor{currentfill}{rgb}{0.121569,0.466667,0.705882}%
\pgfsetfillcolor{currentfill}%
\pgfsetfillopacity{0.513436}%
\pgfsetlinewidth{1.003750pt}%
\definecolor{currentstroke}{rgb}{0.121569,0.466667,0.705882}%
\pgfsetstrokecolor{currentstroke}%
\pgfsetstrokeopacity{0.513436}%
\pgfsetdash{}{0pt}%
\pgfpathmoveto{\pgfqpoint{3.318606in}{1.754077in}}%
\pgfpathcurveto{\pgfqpoint{3.326842in}{1.754077in}}{\pgfqpoint{3.334742in}{1.757349in}}{\pgfqpoint{3.340566in}{1.763173in}}%
\pgfpathcurveto{\pgfqpoint{3.346390in}{1.768997in}}{\pgfqpoint{3.349662in}{1.776897in}}{\pgfqpoint{3.349662in}{1.785133in}}%
\pgfpathcurveto{\pgfqpoint{3.349662in}{1.793369in}}{\pgfqpoint{3.346390in}{1.801269in}}{\pgfqpoint{3.340566in}{1.807093in}}%
\pgfpathcurveto{\pgfqpoint{3.334742in}{1.812917in}}{\pgfqpoint{3.326842in}{1.816190in}}{\pgfqpoint{3.318606in}{1.816190in}}%
\pgfpathcurveto{\pgfqpoint{3.310370in}{1.816190in}}{\pgfqpoint{3.302470in}{1.812917in}}{\pgfqpoint{3.296646in}{1.807093in}}%
\pgfpathcurveto{\pgfqpoint{3.290822in}{1.801269in}}{\pgfqpoint{3.287549in}{1.793369in}}{\pgfqpoint{3.287549in}{1.785133in}}%
\pgfpathcurveto{\pgfqpoint{3.287549in}{1.776897in}}{\pgfqpoint{3.290822in}{1.768997in}}{\pgfqpoint{3.296646in}{1.763173in}}%
\pgfpathcurveto{\pgfqpoint{3.302470in}{1.757349in}}{\pgfqpoint{3.310370in}{1.754077in}}{\pgfqpoint{3.318606in}{1.754077in}}%
\pgfpathclose%
\pgfusepath{stroke,fill}%
\end{pgfscope}%
\begin{pgfscope}%
\pgfpathrectangle{\pgfqpoint{0.100000in}{0.212622in}}{\pgfqpoint{3.696000in}{3.696000in}}%
\pgfusepath{clip}%
\pgfsetbuttcap%
\pgfsetroundjoin%
\definecolor{currentfill}{rgb}{0.121569,0.466667,0.705882}%
\pgfsetfillcolor{currentfill}%
\pgfsetfillopacity{0.513511}%
\pgfsetlinewidth{1.003750pt}%
\definecolor{currentstroke}{rgb}{0.121569,0.466667,0.705882}%
\pgfsetstrokecolor{currentstroke}%
\pgfsetstrokeopacity{0.513511}%
\pgfsetdash{}{0pt}%
\pgfpathmoveto{\pgfqpoint{1.288605in}{2.123740in}}%
\pgfpathcurveto{\pgfqpoint{1.296842in}{2.123740in}}{\pgfqpoint{1.304742in}{2.127012in}}{\pgfqpoint{1.310566in}{2.132836in}}%
\pgfpathcurveto{\pgfqpoint{1.316390in}{2.138660in}}{\pgfqpoint{1.319662in}{2.146560in}}{\pgfqpoint{1.319662in}{2.154796in}}%
\pgfpathcurveto{\pgfqpoint{1.319662in}{2.163033in}}{\pgfqpoint{1.316390in}{2.170933in}}{\pgfqpoint{1.310566in}{2.176757in}}%
\pgfpathcurveto{\pgfqpoint{1.304742in}{2.182581in}}{\pgfqpoint{1.296842in}{2.185853in}}{\pgfqpoint{1.288605in}{2.185853in}}%
\pgfpathcurveto{\pgfqpoint{1.280369in}{2.185853in}}{\pgfqpoint{1.272469in}{2.182581in}}{\pgfqpoint{1.266645in}{2.176757in}}%
\pgfpathcurveto{\pgfqpoint{1.260821in}{2.170933in}}{\pgfqpoint{1.257549in}{2.163033in}}{\pgfqpoint{1.257549in}{2.154796in}}%
\pgfpathcurveto{\pgfqpoint{1.257549in}{2.146560in}}{\pgfqpoint{1.260821in}{2.138660in}}{\pgfqpoint{1.266645in}{2.132836in}}%
\pgfpathcurveto{\pgfqpoint{1.272469in}{2.127012in}}{\pgfqpoint{1.280369in}{2.123740in}}{\pgfqpoint{1.288605in}{2.123740in}}%
\pgfpathclose%
\pgfusepath{stroke,fill}%
\end{pgfscope}%
\begin{pgfscope}%
\pgfpathrectangle{\pgfqpoint{0.100000in}{0.212622in}}{\pgfqpoint{3.696000in}{3.696000in}}%
\pgfusepath{clip}%
\pgfsetbuttcap%
\pgfsetroundjoin%
\definecolor{currentfill}{rgb}{0.121569,0.466667,0.705882}%
\pgfsetfillcolor{currentfill}%
\pgfsetfillopacity{0.514210}%
\pgfsetlinewidth{1.003750pt}%
\definecolor{currentstroke}{rgb}{0.121569,0.466667,0.705882}%
\pgfsetstrokecolor{currentstroke}%
\pgfsetstrokeopacity{0.514210}%
\pgfsetdash{}{0pt}%
\pgfpathmoveto{\pgfqpoint{3.317302in}{1.754195in}}%
\pgfpathcurveto{\pgfqpoint{3.325538in}{1.754195in}}{\pgfqpoint{3.333438in}{1.757467in}}{\pgfqpoint{3.339262in}{1.763291in}}%
\pgfpathcurveto{\pgfqpoint{3.345086in}{1.769115in}}{\pgfqpoint{3.348359in}{1.777015in}}{\pgfqpoint{3.348359in}{1.785251in}}%
\pgfpathcurveto{\pgfqpoint{3.348359in}{1.793487in}}{\pgfqpoint{3.345086in}{1.801387in}}{\pgfqpoint{3.339262in}{1.807211in}}%
\pgfpathcurveto{\pgfqpoint{3.333438in}{1.813035in}}{\pgfqpoint{3.325538in}{1.816308in}}{\pgfqpoint{3.317302in}{1.816308in}}%
\pgfpathcurveto{\pgfqpoint{3.309066in}{1.816308in}}{\pgfqpoint{3.301166in}{1.813035in}}{\pgfqpoint{3.295342in}{1.807211in}}%
\pgfpathcurveto{\pgfqpoint{3.289518in}{1.801387in}}{\pgfqpoint{3.286246in}{1.793487in}}{\pgfqpoint{3.286246in}{1.785251in}}%
\pgfpathcurveto{\pgfqpoint{3.286246in}{1.777015in}}{\pgfqpoint{3.289518in}{1.769115in}}{\pgfqpoint{3.295342in}{1.763291in}}%
\pgfpathcurveto{\pgfqpoint{3.301166in}{1.757467in}}{\pgfqpoint{3.309066in}{1.754195in}}{\pgfqpoint{3.317302in}{1.754195in}}%
\pgfpathclose%
\pgfusepath{stroke,fill}%
\end{pgfscope}%
\begin{pgfscope}%
\pgfpathrectangle{\pgfqpoint{0.100000in}{0.212622in}}{\pgfqpoint{3.696000in}{3.696000in}}%
\pgfusepath{clip}%
\pgfsetbuttcap%
\pgfsetroundjoin%
\definecolor{currentfill}{rgb}{0.121569,0.466667,0.705882}%
\pgfsetfillcolor{currentfill}%
\pgfsetfillopacity{0.515097}%
\pgfsetlinewidth{1.003750pt}%
\definecolor{currentstroke}{rgb}{0.121569,0.466667,0.705882}%
\pgfsetstrokecolor{currentstroke}%
\pgfsetstrokeopacity{0.515097}%
\pgfsetdash{}{0pt}%
\pgfpathmoveto{\pgfqpoint{3.315603in}{1.754480in}}%
\pgfpathcurveto{\pgfqpoint{3.323839in}{1.754480in}}{\pgfqpoint{3.331739in}{1.757752in}}{\pgfqpoint{3.337563in}{1.763576in}}%
\pgfpathcurveto{\pgfqpoint{3.343387in}{1.769400in}}{\pgfqpoint{3.346659in}{1.777300in}}{\pgfqpoint{3.346659in}{1.785536in}}%
\pgfpathcurveto{\pgfqpoint{3.346659in}{1.793772in}}{\pgfqpoint{3.343387in}{1.801672in}}{\pgfqpoint{3.337563in}{1.807496in}}%
\pgfpathcurveto{\pgfqpoint{3.331739in}{1.813320in}}{\pgfqpoint{3.323839in}{1.816593in}}{\pgfqpoint{3.315603in}{1.816593in}}%
\pgfpathcurveto{\pgfqpoint{3.307367in}{1.816593in}}{\pgfqpoint{3.299466in}{1.813320in}}{\pgfqpoint{3.293643in}{1.807496in}}%
\pgfpathcurveto{\pgfqpoint{3.287819in}{1.801672in}}{\pgfqpoint{3.284546in}{1.793772in}}{\pgfqpoint{3.284546in}{1.785536in}}%
\pgfpathcurveto{\pgfqpoint{3.284546in}{1.777300in}}{\pgfqpoint{3.287819in}{1.769400in}}{\pgfqpoint{3.293643in}{1.763576in}}%
\pgfpathcurveto{\pgfqpoint{3.299466in}{1.757752in}}{\pgfqpoint{3.307367in}{1.754480in}}{\pgfqpoint{3.315603in}{1.754480in}}%
\pgfpathclose%
\pgfusepath{stroke,fill}%
\end{pgfscope}%
\begin{pgfscope}%
\pgfpathrectangle{\pgfqpoint{0.100000in}{0.212622in}}{\pgfqpoint{3.696000in}{3.696000in}}%
\pgfusepath{clip}%
\pgfsetbuttcap%
\pgfsetroundjoin%
\definecolor{currentfill}{rgb}{0.121569,0.466667,0.705882}%
\pgfsetfillcolor{currentfill}%
\pgfsetfillopacity{0.515630}%
\pgfsetlinewidth{1.003750pt}%
\definecolor{currentstroke}{rgb}{0.121569,0.466667,0.705882}%
\pgfsetstrokecolor{currentstroke}%
\pgfsetstrokeopacity{0.515630}%
\pgfsetdash{}{0pt}%
\pgfpathmoveto{\pgfqpoint{3.314972in}{1.754683in}}%
\pgfpathcurveto{\pgfqpoint{3.323208in}{1.754683in}}{\pgfqpoint{3.331108in}{1.757955in}}{\pgfqpoint{3.336932in}{1.763779in}}%
\pgfpathcurveto{\pgfqpoint{3.342756in}{1.769603in}}{\pgfqpoint{3.346028in}{1.777503in}}{\pgfqpoint{3.346028in}{1.785740in}}%
\pgfpathcurveto{\pgfqpoint{3.346028in}{1.793976in}}{\pgfqpoint{3.342756in}{1.801876in}}{\pgfqpoint{3.336932in}{1.807700in}}%
\pgfpathcurveto{\pgfqpoint{3.331108in}{1.813524in}}{\pgfqpoint{3.323208in}{1.816796in}}{\pgfqpoint{3.314972in}{1.816796in}}%
\pgfpathcurveto{\pgfqpoint{3.306736in}{1.816796in}}{\pgfqpoint{3.298835in}{1.813524in}}{\pgfqpoint{3.293012in}{1.807700in}}%
\pgfpathcurveto{\pgfqpoint{3.287188in}{1.801876in}}{\pgfqpoint{3.283915in}{1.793976in}}{\pgfqpoint{3.283915in}{1.785740in}}%
\pgfpathcurveto{\pgfqpoint{3.283915in}{1.777503in}}{\pgfqpoint{3.287188in}{1.769603in}}{\pgfqpoint{3.293012in}{1.763779in}}%
\pgfpathcurveto{\pgfqpoint{3.298835in}{1.757955in}}{\pgfqpoint{3.306736in}{1.754683in}}{\pgfqpoint{3.314972in}{1.754683in}}%
\pgfpathclose%
\pgfusepath{stroke,fill}%
\end{pgfscope}%
\begin{pgfscope}%
\pgfpathrectangle{\pgfqpoint{0.100000in}{0.212622in}}{\pgfqpoint{3.696000in}{3.696000in}}%
\pgfusepath{clip}%
\pgfsetbuttcap%
\pgfsetroundjoin%
\definecolor{currentfill}{rgb}{0.121569,0.466667,0.705882}%
\pgfsetfillcolor{currentfill}%
\pgfsetfillopacity{0.515895}%
\pgfsetlinewidth{1.003750pt}%
\definecolor{currentstroke}{rgb}{0.121569,0.466667,0.705882}%
\pgfsetstrokecolor{currentstroke}%
\pgfsetstrokeopacity{0.515895}%
\pgfsetdash{}{0pt}%
\pgfpathmoveto{\pgfqpoint{3.314458in}{1.754742in}}%
\pgfpathcurveto{\pgfqpoint{3.322694in}{1.754742in}}{\pgfqpoint{3.330594in}{1.758014in}}{\pgfqpoint{3.336418in}{1.763838in}}%
\pgfpathcurveto{\pgfqpoint{3.342242in}{1.769662in}}{\pgfqpoint{3.345514in}{1.777562in}}{\pgfqpoint{3.345514in}{1.785798in}}%
\pgfpathcurveto{\pgfqpoint{3.345514in}{1.794034in}}{\pgfqpoint{3.342242in}{1.801934in}}{\pgfqpoint{3.336418in}{1.807758in}}%
\pgfpathcurveto{\pgfqpoint{3.330594in}{1.813582in}}{\pgfqpoint{3.322694in}{1.816855in}}{\pgfqpoint{3.314458in}{1.816855in}}%
\pgfpathcurveto{\pgfqpoint{3.306221in}{1.816855in}}{\pgfqpoint{3.298321in}{1.813582in}}{\pgfqpoint{3.292497in}{1.807758in}}%
\pgfpathcurveto{\pgfqpoint{3.286673in}{1.801934in}}{\pgfqpoint{3.283401in}{1.794034in}}{\pgfqpoint{3.283401in}{1.785798in}}%
\pgfpathcurveto{\pgfqpoint{3.283401in}{1.777562in}}{\pgfqpoint{3.286673in}{1.769662in}}{\pgfqpoint{3.292497in}{1.763838in}}%
\pgfpathcurveto{\pgfqpoint{3.298321in}{1.758014in}}{\pgfqpoint{3.306221in}{1.754742in}}{\pgfqpoint{3.314458in}{1.754742in}}%
\pgfpathclose%
\pgfusepath{stroke,fill}%
\end{pgfscope}%
\begin{pgfscope}%
\pgfpathrectangle{\pgfqpoint{0.100000in}{0.212622in}}{\pgfqpoint{3.696000in}{3.696000in}}%
\pgfusepath{clip}%
\pgfsetbuttcap%
\pgfsetroundjoin%
\definecolor{currentfill}{rgb}{0.121569,0.466667,0.705882}%
\pgfsetfillcolor{currentfill}%
\pgfsetfillopacity{0.516116}%
\pgfsetlinewidth{1.003750pt}%
\definecolor{currentstroke}{rgb}{0.121569,0.466667,0.705882}%
\pgfsetstrokecolor{currentstroke}%
\pgfsetstrokeopacity{0.516116}%
\pgfsetdash{}{0pt}%
\pgfpathmoveto{\pgfqpoint{1.282371in}{2.123732in}}%
\pgfpathcurveto{\pgfqpoint{1.290607in}{2.123732in}}{\pgfqpoint{1.298507in}{2.127004in}}{\pgfqpoint{1.304331in}{2.132828in}}%
\pgfpathcurveto{\pgfqpoint{1.310155in}{2.138652in}}{\pgfqpoint{1.313427in}{2.146552in}}{\pgfqpoint{1.313427in}{2.154788in}}%
\pgfpathcurveto{\pgfqpoint{1.313427in}{2.163025in}}{\pgfqpoint{1.310155in}{2.170925in}}{\pgfqpoint{1.304331in}{2.176749in}}%
\pgfpathcurveto{\pgfqpoint{1.298507in}{2.182573in}}{\pgfqpoint{1.290607in}{2.185845in}}{\pgfqpoint{1.282371in}{2.185845in}}%
\pgfpathcurveto{\pgfqpoint{1.274134in}{2.185845in}}{\pgfqpoint{1.266234in}{2.182573in}}{\pgfqpoint{1.260410in}{2.176749in}}%
\pgfpathcurveto{\pgfqpoint{1.254586in}{2.170925in}}{\pgfqpoint{1.251314in}{2.163025in}}{\pgfqpoint{1.251314in}{2.154788in}}%
\pgfpathcurveto{\pgfqpoint{1.251314in}{2.146552in}}{\pgfqpoint{1.254586in}{2.138652in}}{\pgfqpoint{1.260410in}{2.132828in}}%
\pgfpathcurveto{\pgfqpoint{1.266234in}{2.127004in}}{\pgfqpoint{1.274134in}{2.123732in}}{\pgfqpoint{1.282371in}{2.123732in}}%
\pgfpathclose%
\pgfusepath{stroke,fill}%
\end{pgfscope}%
\begin{pgfscope}%
\pgfpathrectangle{\pgfqpoint{0.100000in}{0.212622in}}{\pgfqpoint{3.696000in}{3.696000in}}%
\pgfusepath{clip}%
\pgfsetbuttcap%
\pgfsetroundjoin%
\definecolor{currentfill}{rgb}{0.121569,0.466667,0.705882}%
\pgfsetfillcolor{currentfill}%
\pgfsetfillopacity{0.516228}%
\pgfsetlinewidth{1.003750pt}%
\definecolor{currentstroke}{rgb}{0.121569,0.466667,0.705882}%
\pgfsetstrokecolor{currentstroke}%
\pgfsetstrokeopacity{0.516228}%
\pgfsetdash{}{0pt}%
\pgfpathmoveto{\pgfqpoint{3.313711in}{1.754888in}}%
\pgfpathcurveto{\pgfqpoint{3.321947in}{1.754888in}}{\pgfqpoint{3.329847in}{1.758160in}}{\pgfqpoint{3.335671in}{1.763984in}}%
\pgfpathcurveto{\pgfqpoint{3.341495in}{1.769808in}}{\pgfqpoint{3.344768in}{1.777708in}}{\pgfqpoint{3.344768in}{1.785944in}}%
\pgfpathcurveto{\pgfqpoint{3.344768in}{1.794180in}}{\pgfqpoint{3.341495in}{1.802081in}}{\pgfqpoint{3.335671in}{1.807904in}}%
\pgfpathcurveto{\pgfqpoint{3.329847in}{1.813728in}}{\pgfqpoint{3.321947in}{1.817001in}}{\pgfqpoint{3.313711in}{1.817001in}}%
\pgfpathcurveto{\pgfqpoint{3.305475in}{1.817001in}}{\pgfqpoint{3.297575in}{1.813728in}}{\pgfqpoint{3.291751in}{1.807904in}}%
\pgfpathcurveto{\pgfqpoint{3.285927in}{1.802081in}}{\pgfqpoint{3.282655in}{1.794180in}}{\pgfqpoint{3.282655in}{1.785944in}}%
\pgfpathcurveto{\pgfqpoint{3.282655in}{1.777708in}}{\pgfqpoint{3.285927in}{1.769808in}}{\pgfqpoint{3.291751in}{1.763984in}}%
\pgfpathcurveto{\pgfqpoint{3.297575in}{1.758160in}}{\pgfqpoint{3.305475in}{1.754888in}}{\pgfqpoint{3.313711in}{1.754888in}}%
\pgfpathclose%
\pgfusepath{stroke,fill}%
\end{pgfscope}%
\begin{pgfscope}%
\pgfpathrectangle{\pgfqpoint{0.100000in}{0.212622in}}{\pgfqpoint{3.696000in}{3.696000in}}%
\pgfusepath{clip}%
\pgfsetbuttcap%
\pgfsetroundjoin%
\definecolor{currentfill}{rgb}{0.121569,0.466667,0.705882}%
\pgfsetfillcolor{currentfill}%
\pgfsetfillopacity{0.516412}%
\pgfsetlinewidth{1.003750pt}%
\definecolor{currentstroke}{rgb}{0.121569,0.466667,0.705882}%
\pgfsetstrokecolor{currentstroke}%
\pgfsetstrokeopacity{0.516412}%
\pgfsetdash{}{0pt}%
\pgfpathmoveto{\pgfqpoint{3.313311in}{1.754954in}}%
\pgfpathcurveto{\pgfqpoint{3.321548in}{1.754954in}}{\pgfqpoint{3.329448in}{1.758227in}}{\pgfqpoint{3.335272in}{1.764051in}}%
\pgfpathcurveto{\pgfqpoint{3.341096in}{1.769875in}}{\pgfqpoint{3.344368in}{1.777775in}}{\pgfqpoint{3.344368in}{1.786011in}}%
\pgfpathcurveto{\pgfqpoint{3.344368in}{1.794247in}}{\pgfqpoint{3.341096in}{1.802147in}}{\pgfqpoint{3.335272in}{1.807971in}}%
\pgfpathcurveto{\pgfqpoint{3.329448in}{1.813795in}}{\pgfqpoint{3.321548in}{1.817067in}}{\pgfqpoint{3.313311in}{1.817067in}}%
\pgfpathcurveto{\pgfqpoint{3.305075in}{1.817067in}}{\pgfqpoint{3.297175in}{1.813795in}}{\pgfqpoint{3.291351in}{1.807971in}}%
\pgfpathcurveto{\pgfqpoint{3.285527in}{1.802147in}}{\pgfqpoint{3.282255in}{1.794247in}}{\pgfqpoint{3.282255in}{1.786011in}}%
\pgfpathcurveto{\pgfqpoint{3.282255in}{1.777775in}}{\pgfqpoint{3.285527in}{1.769875in}}{\pgfqpoint{3.291351in}{1.764051in}}%
\pgfpathcurveto{\pgfqpoint{3.297175in}{1.758227in}}{\pgfqpoint{3.305075in}{1.754954in}}{\pgfqpoint{3.313311in}{1.754954in}}%
\pgfpathclose%
\pgfusepath{stroke,fill}%
\end{pgfscope}%
\begin{pgfscope}%
\pgfpathrectangle{\pgfqpoint{0.100000in}{0.212622in}}{\pgfqpoint{3.696000in}{3.696000in}}%
\pgfusepath{clip}%
\pgfsetbuttcap%
\pgfsetroundjoin%
\definecolor{currentfill}{rgb}{0.121569,0.466667,0.705882}%
\pgfsetfillcolor{currentfill}%
\pgfsetfillopacity{0.516772}%
\pgfsetlinewidth{1.003750pt}%
\definecolor{currentstroke}{rgb}{0.121569,0.466667,0.705882}%
\pgfsetstrokecolor{currentstroke}%
\pgfsetstrokeopacity{0.516772}%
\pgfsetdash{}{0pt}%
\pgfpathmoveto{\pgfqpoint{3.312378in}{1.755176in}}%
\pgfpathcurveto{\pgfqpoint{3.320614in}{1.755176in}}{\pgfqpoint{3.328514in}{1.758448in}}{\pgfqpoint{3.334338in}{1.764272in}}%
\pgfpathcurveto{\pgfqpoint{3.340162in}{1.770096in}}{\pgfqpoint{3.343434in}{1.777996in}}{\pgfqpoint{3.343434in}{1.786233in}}%
\pgfpathcurveto{\pgfqpoint{3.343434in}{1.794469in}}{\pgfqpoint{3.340162in}{1.802369in}}{\pgfqpoint{3.334338in}{1.808193in}}%
\pgfpathcurveto{\pgfqpoint{3.328514in}{1.814017in}}{\pgfqpoint{3.320614in}{1.817289in}}{\pgfqpoint{3.312378in}{1.817289in}}%
\pgfpathcurveto{\pgfqpoint{3.304141in}{1.817289in}}{\pgfqpoint{3.296241in}{1.814017in}}{\pgfqpoint{3.290417in}{1.808193in}}%
\pgfpathcurveto{\pgfqpoint{3.284593in}{1.802369in}}{\pgfqpoint{3.281321in}{1.794469in}}{\pgfqpoint{3.281321in}{1.786233in}}%
\pgfpathcurveto{\pgfqpoint{3.281321in}{1.777996in}}{\pgfqpoint{3.284593in}{1.770096in}}{\pgfqpoint{3.290417in}{1.764272in}}%
\pgfpathcurveto{\pgfqpoint{3.296241in}{1.758448in}}{\pgfqpoint{3.304141in}{1.755176in}}{\pgfqpoint{3.312378in}{1.755176in}}%
\pgfpathclose%
\pgfusepath{stroke,fill}%
\end{pgfscope}%
\begin{pgfscope}%
\pgfpathrectangle{\pgfqpoint{0.100000in}{0.212622in}}{\pgfqpoint{3.696000in}{3.696000in}}%
\pgfusepath{clip}%
\pgfsetbuttcap%
\pgfsetroundjoin%
\definecolor{currentfill}{rgb}{0.121569,0.466667,0.705882}%
\pgfsetfillcolor{currentfill}%
\pgfsetfillopacity{0.516972}%
\pgfsetlinewidth{1.003750pt}%
\definecolor{currentstroke}{rgb}{0.121569,0.466667,0.705882}%
\pgfsetstrokecolor{currentstroke}%
\pgfsetstrokeopacity{0.516972}%
\pgfsetdash{}{0pt}%
\pgfpathmoveto{\pgfqpoint{3.311911in}{1.755241in}}%
\pgfpathcurveto{\pgfqpoint{3.320147in}{1.755241in}}{\pgfqpoint{3.328047in}{1.758514in}}{\pgfqpoint{3.333871in}{1.764337in}}%
\pgfpathcurveto{\pgfqpoint{3.339695in}{1.770161in}}{\pgfqpoint{3.342967in}{1.778061in}}{\pgfqpoint{3.342967in}{1.786298in}}%
\pgfpathcurveto{\pgfqpoint{3.342967in}{1.794534in}}{\pgfqpoint{3.339695in}{1.802434in}}{\pgfqpoint{3.333871in}{1.808258in}}%
\pgfpathcurveto{\pgfqpoint{3.328047in}{1.814082in}}{\pgfqpoint{3.320147in}{1.817354in}}{\pgfqpoint{3.311911in}{1.817354in}}%
\pgfpathcurveto{\pgfqpoint{3.303674in}{1.817354in}}{\pgfqpoint{3.295774in}{1.814082in}}{\pgfqpoint{3.289950in}{1.808258in}}%
\pgfpathcurveto{\pgfqpoint{3.284126in}{1.802434in}}{\pgfqpoint{3.280854in}{1.794534in}}{\pgfqpoint{3.280854in}{1.786298in}}%
\pgfpathcurveto{\pgfqpoint{3.280854in}{1.778061in}}{\pgfqpoint{3.284126in}{1.770161in}}{\pgfqpoint{3.289950in}{1.764337in}}%
\pgfpathcurveto{\pgfqpoint{3.295774in}{1.758514in}}{\pgfqpoint{3.303674in}{1.755241in}}{\pgfqpoint{3.311911in}{1.755241in}}%
\pgfpathclose%
\pgfusepath{stroke,fill}%
\end{pgfscope}%
\begin{pgfscope}%
\pgfpathrectangle{\pgfqpoint{0.100000in}{0.212622in}}{\pgfqpoint{3.696000in}{3.696000in}}%
\pgfusepath{clip}%
\pgfsetbuttcap%
\pgfsetroundjoin%
\definecolor{currentfill}{rgb}{0.121569,0.466667,0.705882}%
\pgfsetfillcolor{currentfill}%
\pgfsetfillopacity{0.517085}%
\pgfsetlinewidth{1.003750pt}%
\definecolor{currentstroke}{rgb}{0.121569,0.466667,0.705882}%
\pgfsetstrokecolor{currentstroke}%
\pgfsetstrokeopacity{0.517085}%
\pgfsetdash{}{0pt}%
\pgfpathmoveto{\pgfqpoint{3.311664in}{1.755282in}}%
\pgfpathcurveto{\pgfqpoint{3.319900in}{1.755282in}}{\pgfqpoint{3.327800in}{1.758555in}}{\pgfqpoint{3.333624in}{1.764379in}}%
\pgfpathcurveto{\pgfqpoint{3.339448in}{1.770202in}}{\pgfqpoint{3.342720in}{1.778102in}}{\pgfqpoint{3.342720in}{1.786339in}}%
\pgfpathcurveto{\pgfqpoint{3.342720in}{1.794575in}}{\pgfqpoint{3.339448in}{1.802475in}}{\pgfqpoint{3.333624in}{1.808299in}}%
\pgfpathcurveto{\pgfqpoint{3.327800in}{1.814123in}}{\pgfqpoint{3.319900in}{1.817395in}}{\pgfqpoint{3.311664in}{1.817395in}}%
\pgfpathcurveto{\pgfqpoint{3.303428in}{1.817395in}}{\pgfqpoint{3.295528in}{1.814123in}}{\pgfqpoint{3.289704in}{1.808299in}}%
\pgfpathcurveto{\pgfqpoint{3.283880in}{1.802475in}}{\pgfqpoint{3.280607in}{1.794575in}}{\pgfqpoint{3.280607in}{1.786339in}}%
\pgfpathcurveto{\pgfqpoint{3.280607in}{1.778102in}}{\pgfqpoint{3.283880in}{1.770202in}}{\pgfqpoint{3.289704in}{1.764379in}}%
\pgfpathcurveto{\pgfqpoint{3.295528in}{1.758555in}}{\pgfqpoint{3.303428in}{1.755282in}}{\pgfqpoint{3.311664in}{1.755282in}}%
\pgfpathclose%
\pgfusepath{stroke,fill}%
\end{pgfscope}%
\begin{pgfscope}%
\pgfpathrectangle{\pgfqpoint{0.100000in}{0.212622in}}{\pgfqpoint{3.696000in}{3.696000in}}%
\pgfusepath{clip}%
\pgfsetbuttcap%
\pgfsetroundjoin%
\definecolor{currentfill}{rgb}{0.121569,0.466667,0.705882}%
\pgfsetfillcolor{currentfill}%
\pgfsetfillopacity{0.517423}%
\pgfsetlinewidth{1.003750pt}%
\definecolor{currentstroke}{rgb}{0.121569,0.466667,0.705882}%
\pgfsetstrokecolor{currentstroke}%
\pgfsetstrokeopacity{0.517423}%
\pgfsetdash{}{0pt}%
\pgfpathmoveto{\pgfqpoint{3.310705in}{1.755520in}}%
\pgfpathcurveto{\pgfqpoint{3.318941in}{1.755520in}}{\pgfqpoint{3.326841in}{1.758792in}}{\pgfqpoint{3.332665in}{1.764616in}}%
\pgfpathcurveto{\pgfqpoint{3.338489in}{1.770440in}}{\pgfqpoint{3.341762in}{1.778340in}}{\pgfqpoint{3.341762in}{1.786577in}}%
\pgfpathcurveto{\pgfqpoint{3.341762in}{1.794813in}}{\pgfqpoint{3.338489in}{1.802713in}}{\pgfqpoint{3.332665in}{1.808537in}}%
\pgfpathcurveto{\pgfqpoint{3.326841in}{1.814361in}}{\pgfqpoint{3.318941in}{1.817633in}}{\pgfqpoint{3.310705in}{1.817633in}}%
\pgfpathcurveto{\pgfqpoint{3.302469in}{1.817633in}}{\pgfqpoint{3.294569in}{1.814361in}}{\pgfqpoint{3.288745in}{1.808537in}}%
\pgfpathcurveto{\pgfqpoint{3.282921in}{1.802713in}}{\pgfqpoint{3.279649in}{1.794813in}}{\pgfqpoint{3.279649in}{1.786577in}}%
\pgfpathcurveto{\pgfqpoint{3.279649in}{1.778340in}}{\pgfqpoint{3.282921in}{1.770440in}}{\pgfqpoint{3.288745in}{1.764616in}}%
\pgfpathcurveto{\pgfqpoint{3.294569in}{1.758792in}}{\pgfqpoint{3.302469in}{1.755520in}}{\pgfqpoint{3.310705in}{1.755520in}}%
\pgfpathclose%
\pgfusepath{stroke,fill}%
\end{pgfscope}%
\begin{pgfscope}%
\pgfpathrectangle{\pgfqpoint{0.100000in}{0.212622in}}{\pgfqpoint{3.696000in}{3.696000in}}%
\pgfusepath{clip}%
\pgfsetbuttcap%
\pgfsetroundjoin%
\definecolor{currentfill}{rgb}{0.121569,0.466667,0.705882}%
\pgfsetfillcolor{currentfill}%
\pgfsetfillopacity{0.517857}%
\pgfsetlinewidth{1.003750pt}%
\definecolor{currentstroke}{rgb}{0.121569,0.466667,0.705882}%
\pgfsetstrokecolor{currentstroke}%
\pgfsetstrokeopacity{0.517857}%
\pgfsetdash{}{0pt}%
\pgfpathmoveto{\pgfqpoint{3.309715in}{1.755667in}}%
\pgfpathcurveto{\pgfqpoint{3.317951in}{1.755667in}}{\pgfqpoint{3.325851in}{1.758939in}}{\pgfqpoint{3.331675in}{1.764763in}}%
\pgfpathcurveto{\pgfqpoint{3.337499in}{1.770587in}}{\pgfqpoint{3.340772in}{1.778487in}}{\pgfqpoint{3.340772in}{1.786723in}}%
\pgfpathcurveto{\pgfqpoint{3.340772in}{1.794959in}}{\pgfqpoint{3.337499in}{1.802859in}}{\pgfqpoint{3.331675in}{1.808683in}}%
\pgfpathcurveto{\pgfqpoint{3.325851in}{1.814507in}}{\pgfqpoint{3.317951in}{1.817780in}}{\pgfqpoint{3.309715in}{1.817780in}}%
\pgfpathcurveto{\pgfqpoint{3.301479in}{1.817780in}}{\pgfqpoint{3.293579in}{1.814507in}}{\pgfqpoint{3.287755in}{1.808683in}}%
\pgfpathcurveto{\pgfqpoint{3.281931in}{1.802859in}}{\pgfqpoint{3.278659in}{1.794959in}}{\pgfqpoint{3.278659in}{1.786723in}}%
\pgfpathcurveto{\pgfqpoint{3.278659in}{1.778487in}}{\pgfqpoint{3.281931in}{1.770587in}}{\pgfqpoint{3.287755in}{1.764763in}}%
\pgfpathcurveto{\pgfqpoint{3.293579in}{1.758939in}}{\pgfqpoint{3.301479in}{1.755667in}}{\pgfqpoint{3.309715in}{1.755667in}}%
\pgfpathclose%
\pgfusepath{stroke,fill}%
\end{pgfscope}%
\begin{pgfscope}%
\pgfpathrectangle{\pgfqpoint{0.100000in}{0.212622in}}{\pgfqpoint{3.696000in}{3.696000in}}%
\pgfusepath{clip}%
\pgfsetbuttcap%
\pgfsetroundjoin%
\definecolor{currentfill}{rgb}{0.121569,0.466667,0.705882}%
\pgfsetfillcolor{currentfill}%
\pgfsetfillopacity{0.518104}%
\pgfsetlinewidth{1.003750pt}%
\definecolor{currentstroke}{rgb}{0.121569,0.466667,0.705882}%
\pgfsetstrokecolor{currentstroke}%
\pgfsetstrokeopacity{0.518104}%
\pgfsetdash{}{0pt}%
\pgfpathmoveto{\pgfqpoint{3.309229in}{1.755736in}}%
\pgfpathcurveto{\pgfqpoint{3.317465in}{1.755736in}}{\pgfqpoint{3.325365in}{1.759008in}}{\pgfqpoint{3.331189in}{1.764832in}}%
\pgfpathcurveto{\pgfqpoint{3.337013in}{1.770656in}}{\pgfqpoint{3.340286in}{1.778556in}}{\pgfqpoint{3.340286in}{1.786792in}}%
\pgfpathcurveto{\pgfqpoint{3.340286in}{1.795028in}}{\pgfqpoint{3.337013in}{1.802928in}}{\pgfqpoint{3.331189in}{1.808752in}}%
\pgfpathcurveto{\pgfqpoint{3.325365in}{1.814576in}}{\pgfqpoint{3.317465in}{1.817849in}}{\pgfqpoint{3.309229in}{1.817849in}}%
\pgfpathcurveto{\pgfqpoint{3.300993in}{1.817849in}}{\pgfqpoint{3.293093in}{1.814576in}}{\pgfqpoint{3.287269in}{1.808752in}}%
\pgfpathcurveto{\pgfqpoint{3.281445in}{1.802928in}}{\pgfqpoint{3.278173in}{1.795028in}}{\pgfqpoint{3.278173in}{1.786792in}}%
\pgfpathcurveto{\pgfqpoint{3.278173in}{1.778556in}}{\pgfqpoint{3.281445in}{1.770656in}}{\pgfqpoint{3.287269in}{1.764832in}}%
\pgfpathcurveto{\pgfqpoint{3.293093in}{1.759008in}}{\pgfqpoint{3.300993in}{1.755736in}}{\pgfqpoint{3.309229in}{1.755736in}}%
\pgfpathclose%
\pgfusepath{stroke,fill}%
\end{pgfscope}%
\begin{pgfscope}%
\pgfpathrectangle{\pgfqpoint{0.100000in}{0.212622in}}{\pgfqpoint{3.696000in}{3.696000in}}%
\pgfusepath{clip}%
\pgfsetbuttcap%
\pgfsetroundjoin%
\definecolor{currentfill}{rgb}{0.121569,0.466667,0.705882}%
\pgfsetfillcolor{currentfill}%
\pgfsetfillopacity{0.518723}%
\pgfsetlinewidth{1.003750pt}%
\definecolor{currentstroke}{rgb}{0.121569,0.466667,0.705882}%
\pgfsetstrokecolor{currentstroke}%
\pgfsetstrokeopacity{0.518723}%
\pgfsetdash{}{0pt}%
\pgfpathmoveto{\pgfqpoint{3.307286in}{1.756315in}}%
\pgfpathcurveto{\pgfqpoint{3.315523in}{1.756315in}}{\pgfqpoint{3.323423in}{1.759587in}}{\pgfqpoint{3.329247in}{1.765411in}}%
\pgfpathcurveto{\pgfqpoint{3.335071in}{1.771235in}}{\pgfqpoint{3.338343in}{1.779135in}}{\pgfqpoint{3.338343in}{1.787372in}}%
\pgfpathcurveto{\pgfqpoint{3.338343in}{1.795608in}}{\pgfqpoint{3.335071in}{1.803508in}}{\pgfqpoint{3.329247in}{1.809332in}}%
\pgfpathcurveto{\pgfqpoint{3.323423in}{1.815156in}}{\pgfqpoint{3.315523in}{1.818428in}}{\pgfqpoint{3.307286in}{1.818428in}}%
\pgfpathcurveto{\pgfqpoint{3.299050in}{1.818428in}}{\pgfqpoint{3.291150in}{1.815156in}}{\pgfqpoint{3.285326in}{1.809332in}}%
\pgfpathcurveto{\pgfqpoint{3.279502in}{1.803508in}}{\pgfqpoint{3.276230in}{1.795608in}}{\pgfqpoint{3.276230in}{1.787372in}}%
\pgfpathcurveto{\pgfqpoint{3.276230in}{1.779135in}}{\pgfqpoint{3.279502in}{1.771235in}}{\pgfqpoint{3.285326in}{1.765411in}}%
\pgfpathcurveto{\pgfqpoint{3.291150in}{1.759587in}}{\pgfqpoint{3.299050in}{1.756315in}}{\pgfqpoint{3.307286in}{1.756315in}}%
\pgfpathclose%
\pgfusepath{stroke,fill}%
\end{pgfscope}%
\begin{pgfscope}%
\pgfpathrectangle{\pgfqpoint{0.100000in}{0.212622in}}{\pgfqpoint{3.696000in}{3.696000in}}%
\pgfusepath{clip}%
\pgfsetbuttcap%
\pgfsetroundjoin%
\definecolor{currentfill}{rgb}{0.121569,0.466667,0.705882}%
\pgfsetfillcolor{currentfill}%
\pgfsetfillopacity{0.518833}%
\pgfsetlinewidth{1.003750pt}%
\definecolor{currentstroke}{rgb}{0.121569,0.466667,0.705882}%
\pgfsetstrokecolor{currentstroke}%
\pgfsetstrokeopacity{0.518833}%
\pgfsetdash{}{0pt}%
\pgfpathmoveto{\pgfqpoint{1.279051in}{2.124226in}}%
\pgfpathcurveto{\pgfqpoint{1.287287in}{2.124226in}}{\pgfqpoint{1.295187in}{2.127498in}}{\pgfqpoint{1.301011in}{2.133322in}}%
\pgfpathcurveto{\pgfqpoint{1.306835in}{2.139146in}}{\pgfqpoint{1.310107in}{2.147046in}}{\pgfqpoint{1.310107in}{2.155282in}}%
\pgfpathcurveto{\pgfqpoint{1.310107in}{2.163518in}}{\pgfqpoint{1.306835in}{2.171418in}}{\pgfqpoint{1.301011in}{2.177242in}}%
\pgfpathcurveto{\pgfqpoint{1.295187in}{2.183066in}}{\pgfqpoint{1.287287in}{2.186339in}}{\pgfqpoint{1.279051in}{2.186339in}}%
\pgfpathcurveto{\pgfqpoint{1.270814in}{2.186339in}}{\pgfqpoint{1.262914in}{2.183066in}}{\pgfqpoint{1.257090in}{2.177242in}}%
\pgfpathcurveto{\pgfqpoint{1.251266in}{2.171418in}}{\pgfqpoint{1.247994in}{2.163518in}}{\pgfqpoint{1.247994in}{2.155282in}}%
\pgfpathcurveto{\pgfqpoint{1.247994in}{2.147046in}}{\pgfqpoint{1.251266in}{2.139146in}}{\pgfqpoint{1.257090in}{2.133322in}}%
\pgfpathcurveto{\pgfqpoint{1.262914in}{2.127498in}}{\pgfqpoint{1.270814in}{2.124226in}}{\pgfqpoint{1.279051in}{2.124226in}}%
\pgfpathclose%
\pgfusepath{stroke,fill}%
\end{pgfscope}%
\begin{pgfscope}%
\pgfpathrectangle{\pgfqpoint{0.100000in}{0.212622in}}{\pgfqpoint{3.696000in}{3.696000in}}%
\pgfusepath{clip}%
\pgfsetbuttcap%
\pgfsetroundjoin%
\definecolor{currentfill}{rgb}{0.121569,0.466667,0.705882}%
\pgfsetfillcolor{currentfill}%
\pgfsetfillopacity{0.519075}%
\pgfsetlinewidth{1.003750pt}%
\definecolor{currentstroke}{rgb}{0.121569,0.466667,0.705882}%
\pgfsetstrokecolor{currentstroke}%
\pgfsetstrokeopacity{0.519075}%
\pgfsetdash{}{0pt}%
\pgfpathmoveto{\pgfqpoint{3.306343in}{1.756495in}}%
\pgfpathcurveto{\pgfqpoint{3.314579in}{1.756495in}}{\pgfqpoint{3.322479in}{1.759768in}}{\pgfqpoint{3.328303in}{1.765592in}}%
\pgfpathcurveto{\pgfqpoint{3.334127in}{1.771416in}}{\pgfqpoint{3.337399in}{1.779316in}}{\pgfqpoint{3.337399in}{1.787552in}}%
\pgfpathcurveto{\pgfqpoint{3.337399in}{1.795788in}}{\pgfqpoint{3.334127in}{1.803688in}}{\pgfqpoint{3.328303in}{1.809512in}}%
\pgfpathcurveto{\pgfqpoint{3.322479in}{1.815336in}}{\pgfqpoint{3.314579in}{1.818608in}}{\pgfqpoint{3.306343in}{1.818608in}}%
\pgfpathcurveto{\pgfqpoint{3.298106in}{1.818608in}}{\pgfqpoint{3.290206in}{1.815336in}}{\pgfqpoint{3.284382in}{1.809512in}}%
\pgfpathcurveto{\pgfqpoint{3.278559in}{1.803688in}}{\pgfqpoint{3.275286in}{1.795788in}}{\pgfqpoint{3.275286in}{1.787552in}}%
\pgfpathcurveto{\pgfqpoint{3.275286in}{1.779316in}}{\pgfqpoint{3.278559in}{1.771416in}}{\pgfqpoint{3.284382in}{1.765592in}}%
\pgfpathcurveto{\pgfqpoint{3.290206in}{1.759768in}}{\pgfqpoint{3.298106in}{1.756495in}}{\pgfqpoint{3.306343in}{1.756495in}}%
\pgfpathclose%
\pgfusepath{stroke,fill}%
\end{pgfscope}%
\begin{pgfscope}%
\pgfpathrectangle{\pgfqpoint{0.100000in}{0.212622in}}{\pgfqpoint{3.696000in}{3.696000in}}%
\pgfusepath{clip}%
\pgfsetbuttcap%
\pgfsetroundjoin%
\definecolor{currentfill}{rgb}{0.121569,0.466667,0.705882}%
\pgfsetfillcolor{currentfill}%
\pgfsetfillopacity{0.519284}%
\pgfsetlinewidth{1.003750pt}%
\definecolor{currentstroke}{rgb}{0.121569,0.466667,0.705882}%
\pgfsetstrokecolor{currentstroke}%
\pgfsetstrokeopacity{0.519284}%
\pgfsetdash{}{0pt}%
\pgfpathmoveto{\pgfqpoint{3.305957in}{1.756521in}}%
\pgfpathcurveto{\pgfqpoint{3.314193in}{1.756521in}}{\pgfqpoint{3.322093in}{1.759793in}}{\pgfqpoint{3.327917in}{1.765617in}}%
\pgfpathcurveto{\pgfqpoint{3.333741in}{1.771441in}}{\pgfqpoint{3.337013in}{1.779341in}}{\pgfqpoint{3.337013in}{1.787577in}}%
\pgfpathcurveto{\pgfqpoint{3.337013in}{1.795813in}}{\pgfqpoint{3.333741in}{1.803713in}}{\pgfqpoint{3.327917in}{1.809537in}}%
\pgfpathcurveto{\pgfqpoint{3.322093in}{1.815361in}}{\pgfqpoint{3.314193in}{1.818634in}}{\pgfqpoint{3.305957in}{1.818634in}}%
\pgfpathcurveto{\pgfqpoint{3.297720in}{1.818634in}}{\pgfqpoint{3.289820in}{1.815361in}}{\pgfqpoint{3.283996in}{1.809537in}}%
\pgfpathcurveto{\pgfqpoint{3.278172in}{1.803713in}}{\pgfqpoint{3.274900in}{1.795813in}}{\pgfqpoint{3.274900in}{1.787577in}}%
\pgfpathcurveto{\pgfqpoint{3.274900in}{1.779341in}}{\pgfqpoint{3.278172in}{1.771441in}}{\pgfqpoint{3.283996in}{1.765617in}}%
\pgfpathcurveto{\pgfqpoint{3.289820in}{1.759793in}}{\pgfqpoint{3.297720in}{1.756521in}}{\pgfqpoint{3.305957in}{1.756521in}}%
\pgfpathclose%
\pgfusepath{stroke,fill}%
\end{pgfscope}%
\begin{pgfscope}%
\pgfpathrectangle{\pgfqpoint{0.100000in}{0.212622in}}{\pgfqpoint{3.696000in}{3.696000in}}%
\pgfusepath{clip}%
\pgfsetbuttcap%
\pgfsetroundjoin%
\definecolor{currentfill}{rgb}{0.121569,0.466667,0.705882}%
\pgfsetfillcolor{currentfill}%
\pgfsetfillopacity{0.519980}%
\pgfsetlinewidth{1.003750pt}%
\definecolor{currentstroke}{rgb}{0.121569,0.466667,0.705882}%
\pgfsetstrokecolor{currentstroke}%
\pgfsetstrokeopacity{0.519980}%
\pgfsetdash{}{0pt}%
\pgfpathmoveto{\pgfqpoint{3.304000in}{1.756970in}}%
\pgfpathcurveto{\pgfqpoint{3.312236in}{1.756970in}}{\pgfqpoint{3.320137in}{1.760242in}}{\pgfqpoint{3.325960in}{1.766066in}}%
\pgfpathcurveto{\pgfqpoint{3.331784in}{1.771890in}}{\pgfqpoint{3.335057in}{1.779790in}}{\pgfqpoint{3.335057in}{1.788027in}}%
\pgfpathcurveto{\pgfqpoint{3.335057in}{1.796263in}}{\pgfqpoint{3.331784in}{1.804163in}}{\pgfqpoint{3.325960in}{1.809987in}}%
\pgfpathcurveto{\pgfqpoint{3.320137in}{1.815811in}}{\pgfqpoint{3.312236in}{1.819083in}}{\pgfqpoint{3.304000in}{1.819083in}}%
\pgfpathcurveto{\pgfqpoint{3.295764in}{1.819083in}}{\pgfqpoint{3.287864in}{1.815811in}}{\pgfqpoint{3.282040in}{1.809987in}}%
\pgfpathcurveto{\pgfqpoint{3.276216in}{1.804163in}}{\pgfqpoint{3.272944in}{1.796263in}}{\pgfqpoint{3.272944in}{1.788027in}}%
\pgfpathcurveto{\pgfqpoint{3.272944in}{1.779790in}}{\pgfqpoint{3.276216in}{1.771890in}}{\pgfqpoint{3.282040in}{1.766066in}}%
\pgfpathcurveto{\pgfqpoint{3.287864in}{1.760242in}}{\pgfqpoint{3.295764in}{1.756970in}}{\pgfqpoint{3.304000in}{1.756970in}}%
\pgfpathclose%
\pgfusepath{stroke,fill}%
\end{pgfscope}%
\begin{pgfscope}%
\pgfpathrectangle{\pgfqpoint{0.100000in}{0.212622in}}{\pgfqpoint{3.696000in}{3.696000in}}%
\pgfusepath{clip}%
\pgfsetbuttcap%
\pgfsetroundjoin%
\definecolor{currentfill}{rgb}{0.121569,0.466667,0.705882}%
\pgfsetfillcolor{currentfill}%
\pgfsetfillopacity{0.520360}%
\pgfsetlinewidth{1.003750pt}%
\definecolor{currentstroke}{rgb}{0.121569,0.466667,0.705882}%
\pgfsetstrokecolor{currentstroke}%
\pgfsetstrokeopacity{0.520360}%
\pgfsetdash{}{0pt}%
\pgfpathmoveto{\pgfqpoint{3.302901in}{1.757236in}}%
\pgfpathcurveto{\pgfqpoint{3.311138in}{1.757236in}}{\pgfqpoint{3.319038in}{1.760508in}}{\pgfqpoint{3.324862in}{1.766332in}}%
\pgfpathcurveto{\pgfqpoint{3.330685in}{1.772156in}}{\pgfqpoint{3.333958in}{1.780056in}}{\pgfqpoint{3.333958in}{1.788292in}}%
\pgfpathcurveto{\pgfqpoint{3.333958in}{1.796529in}}{\pgfqpoint{3.330685in}{1.804429in}}{\pgfqpoint{3.324862in}{1.810253in}}%
\pgfpathcurveto{\pgfqpoint{3.319038in}{1.816077in}}{\pgfqpoint{3.311138in}{1.819349in}}{\pgfqpoint{3.302901in}{1.819349in}}%
\pgfpathcurveto{\pgfqpoint{3.294665in}{1.819349in}}{\pgfqpoint{3.286765in}{1.816077in}}{\pgfqpoint{3.280941in}{1.810253in}}%
\pgfpathcurveto{\pgfqpoint{3.275117in}{1.804429in}}{\pgfqpoint{3.271845in}{1.796529in}}{\pgfqpoint{3.271845in}{1.788292in}}%
\pgfpathcurveto{\pgfqpoint{3.271845in}{1.780056in}}{\pgfqpoint{3.275117in}{1.772156in}}{\pgfqpoint{3.280941in}{1.766332in}}%
\pgfpathcurveto{\pgfqpoint{3.286765in}{1.760508in}}{\pgfqpoint{3.294665in}{1.757236in}}{\pgfqpoint{3.302901in}{1.757236in}}%
\pgfpathclose%
\pgfusepath{stroke,fill}%
\end{pgfscope}%
\begin{pgfscope}%
\pgfpathrectangle{\pgfqpoint{0.100000in}{0.212622in}}{\pgfqpoint{3.696000in}{3.696000in}}%
\pgfusepath{clip}%
\pgfsetbuttcap%
\pgfsetroundjoin%
\definecolor{currentfill}{rgb}{0.121569,0.466667,0.705882}%
\pgfsetfillcolor{currentfill}%
\pgfsetfillopacity{0.521012}%
\pgfsetlinewidth{1.003750pt}%
\definecolor{currentstroke}{rgb}{0.121569,0.466667,0.705882}%
\pgfsetstrokecolor{currentstroke}%
\pgfsetstrokeopacity{0.521012}%
\pgfsetdash{}{0pt}%
\pgfpathmoveto{\pgfqpoint{1.273279in}{2.124683in}}%
\pgfpathcurveto{\pgfqpoint{1.281515in}{2.124683in}}{\pgfqpoint{1.289415in}{2.127955in}}{\pgfqpoint{1.295239in}{2.133779in}}%
\pgfpathcurveto{\pgfqpoint{1.301063in}{2.139603in}}{\pgfqpoint{1.304336in}{2.147503in}}{\pgfqpoint{1.304336in}{2.155739in}}%
\pgfpathcurveto{\pgfqpoint{1.304336in}{2.163975in}}{\pgfqpoint{1.301063in}{2.171875in}}{\pgfqpoint{1.295239in}{2.177699in}}%
\pgfpathcurveto{\pgfqpoint{1.289415in}{2.183523in}}{\pgfqpoint{1.281515in}{2.186796in}}{\pgfqpoint{1.273279in}{2.186796in}}%
\pgfpathcurveto{\pgfqpoint{1.265043in}{2.186796in}}{\pgfqpoint{1.257143in}{2.183523in}}{\pgfqpoint{1.251319in}{2.177699in}}%
\pgfpathcurveto{\pgfqpoint{1.245495in}{2.171875in}}{\pgfqpoint{1.242223in}{2.163975in}}{\pgfqpoint{1.242223in}{2.155739in}}%
\pgfpathcurveto{\pgfqpoint{1.242223in}{2.147503in}}{\pgfqpoint{1.245495in}{2.139603in}}{\pgfqpoint{1.251319in}{2.133779in}}%
\pgfpathcurveto{\pgfqpoint{1.257143in}{2.127955in}}{\pgfqpoint{1.265043in}{2.124683in}}{\pgfqpoint{1.273279in}{2.124683in}}%
\pgfpathclose%
\pgfusepath{stroke,fill}%
\end{pgfscope}%
\begin{pgfscope}%
\pgfpathrectangle{\pgfqpoint{0.100000in}{0.212622in}}{\pgfqpoint{3.696000in}{3.696000in}}%
\pgfusepath{clip}%
\pgfsetbuttcap%
\pgfsetroundjoin%
\definecolor{currentfill}{rgb}{0.121569,0.466667,0.705882}%
\pgfsetfillcolor{currentfill}%
\pgfsetfillopacity{0.521222}%
\pgfsetlinewidth{1.003750pt}%
\definecolor{currentstroke}{rgb}{0.121569,0.466667,0.705882}%
\pgfsetstrokecolor{currentstroke}%
\pgfsetstrokeopacity{0.521222}%
\pgfsetdash{}{0pt}%
\pgfpathmoveto{\pgfqpoint{3.301556in}{1.757377in}}%
\pgfpathcurveto{\pgfqpoint{3.309792in}{1.757377in}}{\pgfqpoint{3.317692in}{1.760649in}}{\pgfqpoint{3.323516in}{1.766473in}}%
\pgfpathcurveto{\pgfqpoint{3.329340in}{1.772297in}}{\pgfqpoint{3.332613in}{1.780197in}}{\pgfqpoint{3.332613in}{1.788433in}}%
\pgfpathcurveto{\pgfqpoint{3.332613in}{1.796669in}}{\pgfqpoint{3.329340in}{1.804569in}}{\pgfqpoint{3.323516in}{1.810393in}}%
\pgfpathcurveto{\pgfqpoint{3.317692in}{1.816217in}}{\pgfqpoint{3.309792in}{1.819490in}}{\pgfqpoint{3.301556in}{1.819490in}}%
\pgfpathcurveto{\pgfqpoint{3.293320in}{1.819490in}}{\pgfqpoint{3.285420in}{1.816217in}}{\pgfqpoint{3.279596in}{1.810393in}}%
\pgfpathcurveto{\pgfqpoint{3.273772in}{1.804569in}}{\pgfqpoint{3.270500in}{1.796669in}}{\pgfqpoint{3.270500in}{1.788433in}}%
\pgfpathcurveto{\pgfqpoint{3.270500in}{1.780197in}}{\pgfqpoint{3.273772in}{1.772297in}}{\pgfqpoint{3.279596in}{1.766473in}}%
\pgfpathcurveto{\pgfqpoint{3.285420in}{1.760649in}}{\pgfqpoint{3.293320in}{1.757377in}}{\pgfqpoint{3.301556in}{1.757377in}}%
\pgfpathclose%
\pgfusepath{stroke,fill}%
\end{pgfscope}%
\begin{pgfscope}%
\pgfpathrectangle{\pgfqpoint{0.100000in}{0.212622in}}{\pgfqpoint{3.696000in}{3.696000in}}%
\pgfusepath{clip}%
\pgfsetbuttcap%
\pgfsetroundjoin%
\definecolor{currentfill}{rgb}{0.121569,0.466667,0.705882}%
\pgfsetfillcolor{currentfill}%
\pgfsetfillopacity{0.521676}%
\pgfsetlinewidth{1.003750pt}%
\definecolor{currentstroke}{rgb}{0.121569,0.466667,0.705882}%
\pgfsetstrokecolor{currentstroke}%
\pgfsetstrokeopacity{0.521676}%
\pgfsetdash{}{0pt}%
\pgfpathmoveto{\pgfqpoint{3.300646in}{1.757483in}}%
\pgfpathcurveto{\pgfqpoint{3.308883in}{1.757483in}}{\pgfqpoint{3.316783in}{1.760755in}}{\pgfqpoint{3.322607in}{1.766579in}}%
\pgfpathcurveto{\pgfqpoint{3.328431in}{1.772403in}}{\pgfqpoint{3.331703in}{1.780303in}}{\pgfqpoint{3.331703in}{1.788539in}}%
\pgfpathcurveto{\pgfqpoint{3.331703in}{1.796776in}}{\pgfqpoint{3.328431in}{1.804676in}}{\pgfqpoint{3.322607in}{1.810500in}}%
\pgfpathcurveto{\pgfqpoint{3.316783in}{1.816323in}}{\pgfqpoint{3.308883in}{1.819596in}}{\pgfqpoint{3.300646in}{1.819596in}}%
\pgfpathcurveto{\pgfqpoint{3.292410in}{1.819596in}}{\pgfqpoint{3.284510in}{1.816323in}}{\pgfqpoint{3.278686in}{1.810500in}}%
\pgfpathcurveto{\pgfqpoint{3.272862in}{1.804676in}}{\pgfqpoint{3.269590in}{1.796776in}}{\pgfqpoint{3.269590in}{1.788539in}}%
\pgfpathcurveto{\pgfqpoint{3.269590in}{1.780303in}}{\pgfqpoint{3.272862in}{1.772403in}}{\pgfqpoint{3.278686in}{1.766579in}}%
\pgfpathcurveto{\pgfqpoint{3.284510in}{1.760755in}}{\pgfqpoint{3.292410in}{1.757483in}}{\pgfqpoint{3.300646in}{1.757483in}}%
\pgfpathclose%
\pgfusepath{stroke,fill}%
\end{pgfscope}%
\begin{pgfscope}%
\pgfpathrectangle{\pgfqpoint{0.100000in}{0.212622in}}{\pgfqpoint{3.696000in}{3.696000in}}%
\pgfusepath{clip}%
\pgfsetbuttcap%
\pgfsetroundjoin%
\definecolor{currentfill}{rgb}{0.121569,0.466667,0.705882}%
\pgfsetfillcolor{currentfill}%
\pgfsetfillopacity{0.522203}%
\pgfsetlinewidth{1.003750pt}%
\definecolor{currentstroke}{rgb}{0.121569,0.466667,0.705882}%
\pgfsetstrokecolor{currentstroke}%
\pgfsetstrokeopacity{0.522203}%
\pgfsetdash{}{0pt}%
\pgfpathmoveto{\pgfqpoint{3.299171in}{1.757958in}}%
\pgfpathcurveto{\pgfqpoint{3.307407in}{1.757958in}}{\pgfqpoint{3.315307in}{1.761231in}}{\pgfqpoint{3.321131in}{1.767055in}}%
\pgfpathcurveto{\pgfqpoint{3.326955in}{1.772879in}}{\pgfqpoint{3.330227in}{1.780779in}}{\pgfqpoint{3.330227in}{1.789015in}}%
\pgfpathcurveto{\pgfqpoint{3.330227in}{1.797251in}}{\pgfqpoint{3.326955in}{1.805151in}}{\pgfqpoint{3.321131in}{1.810975in}}%
\pgfpathcurveto{\pgfqpoint{3.315307in}{1.816799in}}{\pgfqpoint{3.307407in}{1.820071in}}{\pgfqpoint{3.299171in}{1.820071in}}%
\pgfpathcurveto{\pgfqpoint{3.290935in}{1.820071in}}{\pgfqpoint{3.283034in}{1.816799in}}{\pgfqpoint{3.277211in}{1.810975in}}%
\pgfpathcurveto{\pgfqpoint{3.271387in}{1.805151in}}{\pgfqpoint{3.268114in}{1.797251in}}{\pgfqpoint{3.268114in}{1.789015in}}%
\pgfpathcurveto{\pgfqpoint{3.268114in}{1.780779in}}{\pgfqpoint{3.271387in}{1.772879in}}{\pgfqpoint{3.277211in}{1.767055in}}%
\pgfpathcurveto{\pgfqpoint{3.283034in}{1.761231in}}{\pgfqpoint{3.290935in}{1.757958in}}{\pgfqpoint{3.299171in}{1.757958in}}%
\pgfpathclose%
\pgfusepath{stroke,fill}%
\end{pgfscope}%
\begin{pgfscope}%
\pgfpathrectangle{\pgfqpoint{0.100000in}{0.212622in}}{\pgfqpoint{3.696000in}{3.696000in}}%
\pgfusepath{clip}%
\pgfsetbuttcap%
\pgfsetroundjoin%
\definecolor{currentfill}{rgb}{0.121569,0.466667,0.705882}%
\pgfsetfillcolor{currentfill}%
\pgfsetfillopacity{0.522519}%
\pgfsetlinewidth{1.003750pt}%
\definecolor{currentstroke}{rgb}{0.121569,0.466667,0.705882}%
\pgfsetstrokecolor{currentstroke}%
\pgfsetstrokeopacity{0.522519}%
\pgfsetdash{}{0pt}%
\pgfpathmoveto{\pgfqpoint{3.298676in}{1.757990in}}%
\pgfpathcurveto{\pgfqpoint{3.306913in}{1.757990in}}{\pgfqpoint{3.314813in}{1.761262in}}{\pgfqpoint{3.320637in}{1.767086in}}%
\pgfpathcurveto{\pgfqpoint{3.326461in}{1.772910in}}{\pgfqpoint{3.329733in}{1.780810in}}{\pgfqpoint{3.329733in}{1.789046in}}%
\pgfpathcurveto{\pgfqpoint{3.329733in}{1.797282in}}{\pgfqpoint{3.326461in}{1.805182in}}{\pgfqpoint{3.320637in}{1.811006in}}%
\pgfpathcurveto{\pgfqpoint{3.314813in}{1.816830in}}{\pgfqpoint{3.306913in}{1.820103in}}{\pgfqpoint{3.298676in}{1.820103in}}%
\pgfpathcurveto{\pgfqpoint{3.290440in}{1.820103in}}{\pgfqpoint{3.282540in}{1.816830in}}{\pgfqpoint{3.276716in}{1.811006in}}%
\pgfpathcurveto{\pgfqpoint{3.270892in}{1.805182in}}{\pgfqpoint{3.267620in}{1.797282in}}{\pgfqpoint{3.267620in}{1.789046in}}%
\pgfpathcurveto{\pgfqpoint{3.267620in}{1.780810in}}{\pgfqpoint{3.270892in}{1.772910in}}{\pgfqpoint{3.276716in}{1.767086in}}%
\pgfpathcurveto{\pgfqpoint{3.282540in}{1.761262in}}{\pgfqpoint{3.290440in}{1.757990in}}{\pgfqpoint{3.298676in}{1.757990in}}%
\pgfpathclose%
\pgfusepath{stroke,fill}%
\end{pgfscope}%
\begin{pgfscope}%
\pgfpathrectangle{\pgfqpoint{0.100000in}{0.212622in}}{\pgfqpoint{3.696000in}{3.696000in}}%
\pgfusepath{clip}%
\pgfsetbuttcap%
\pgfsetroundjoin%
\definecolor{currentfill}{rgb}{0.121569,0.466667,0.705882}%
\pgfsetfillcolor{currentfill}%
\pgfsetfillopacity{0.522902}%
\pgfsetlinewidth{1.003750pt}%
\definecolor{currentstroke}{rgb}{0.121569,0.466667,0.705882}%
\pgfsetstrokecolor{currentstroke}%
\pgfsetstrokeopacity{0.522902}%
\pgfsetdash{}{0pt}%
\pgfpathmoveto{\pgfqpoint{1.269324in}{2.124499in}}%
\pgfpathcurveto{\pgfqpoint{1.277560in}{2.124499in}}{\pgfqpoint{1.285460in}{2.127771in}}{\pgfqpoint{1.291284in}{2.133595in}}%
\pgfpathcurveto{\pgfqpoint{1.297108in}{2.139419in}}{\pgfqpoint{1.300380in}{2.147319in}}{\pgfqpoint{1.300380in}{2.155555in}}%
\pgfpathcurveto{\pgfqpoint{1.300380in}{2.163791in}}{\pgfqpoint{1.297108in}{2.171691in}}{\pgfqpoint{1.291284in}{2.177515in}}%
\pgfpathcurveto{\pgfqpoint{1.285460in}{2.183339in}}{\pgfqpoint{1.277560in}{2.186612in}}{\pgfqpoint{1.269324in}{2.186612in}}%
\pgfpathcurveto{\pgfqpoint{1.261088in}{2.186612in}}{\pgfqpoint{1.253188in}{2.183339in}}{\pgfqpoint{1.247364in}{2.177515in}}%
\pgfpathcurveto{\pgfqpoint{1.241540in}{2.171691in}}{\pgfqpoint{1.238267in}{2.163791in}}{\pgfqpoint{1.238267in}{2.155555in}}%
\pgfpathcurveto{\pgfqpoint{1.238267in}{2.147319in}}{\pgfqpoint{1.241540in}{2.139419in}}{\pgfqpoint{1.247364in}{2.133595in}}%
\pgfpathcurveto{\pgfqpoint{1.253188in}{2.127771in}}{\pgfqpoint{1.261088in}{2.124499in}}{\pgfqpoint{1.269324in}{2.124499in}}%
\pgfpathclose%
\pgfusepath{stroke,fill}%
\end{pgfscope}%
\begin{pgfscope}%
\pgfpathrectangle{\pgfqpoint{0.100000in}{0.212622in}}{\pgfqpoint{3.696000in}{3.696000in}}%
\pgfusepath{clip}%
\pgfsetbuttcap%
\pgfsetroundjoin%
\definecolor{currentfill}{rgb}{0.121569,0.466667,0.705882}%
\pgfsetfillcolor{currentfill}%
\pgfsetfillopacity{0.522912}%
\pgfsetlinewidth{1.003750pt}%
\definecolor{currentstroke}{rgb}{0.121569,0.466667,0.705882}%
\pgfsetstrokecolor{currentstroke}%
\pgfsetstrokeopacity{0.522912}%
\pgfsetdash{}{0pt}%
\pgfpathmoveto{\pgfqpoint{3.298059in}{1.758062in}}%
\pgfpathcurveto{\pgfqpoint{3.306296in}{1.758062in}}{\pgfqpoint{3.314196in}{1.761334in}}{\pgfqpoint{3.320020in}{1.767158in}}%
\pgfpathcurveto{\pgfqpoint{3.325843in}{1.772982in}}{\pgfqpoint{3.329116in}{1.780882in}}{\pgfqpoint{3.329116in}{1.789118in}}%
\pgfpathcurveto{\pgfqpoint{3.329116in}{1.797355in}}{\pgfqpoint{3.325843in}{1.805255in}}{\pgfqpoint{3.320020in}{1.811079in}}%
\pgfpathcurveto{\pgfqpoint{3.314196in}{1.816902in}}{\pgfqpoint{3.306296in}{1.820175in}}{\pgfqpoint{3.298059in}{1.820175in}}%
\pgfpathcurveto{\pgfqpoint{3.289823in}{1.820175in}}{\pgfqpoint{3.281923in}{1.816902in}}{\pgfqpoint{3.276099in}{1.811079in}}%
\pgfpathcurveto{\pgfqpoint{3.270275in}{1.805255in}}{\pgfqpoint{3.267003in}{1.797355in}}{\pgfqpoint{3.267003in}{1.789118in}}%
\pgfpathcurveto{\pgfqpoint{3.267003in}{1.780882in}}{\pgfqpoint{3.270275in}{1.772982in}}{\pgfqpoint{3.276099in}{1.767158in}}%
\pgfpathcurveto{\pgfqpoint{3.281923in}{1.761334in}}{\pgfqpoint{3.289823in}{1.758062in}}{\pgfqpoint{3.298059in}{1.758062in}}%
\pgfpathclose%
\pgfusepath{stroke,fill}%
\end{pgfscope}%
\begin{pgfscope}%
\pgfpathrectangle{\pgfqpoint{0.100000in}{0.212622in}}{\pgfqpoint{3.696000in}{3.696000in}}%
\pgfusepath{clip}%
\pgfsetbuttcap%
\pgfsetroundjoin%
\definecolor{currentfill}{rgb}{0.121569,0.466667,0.705882}%
\pgfsetfillcolor{currentfill}%
\pgfsetfillopacity{0.523372}%
\pgfsetlinewidth{1.003750pt}%
\definecolor{currentstroke}{rgb}{0.121569,0.466667,0.705882}%
\pgfsetstrokecolor{currentstroke}%
\pgfsetstrokeopacity{0.523372}%
\pgfsetdash{}{0pt}%
\pgfpathmoveto{\pgfqpoint{3.296973in}{1.758370in}}%
\pgfpathcurveto{\pgfqpoint{3.305210in}{1.758370in}}{\pgfqpoint{3.313110in}{1.761642in}}{\pgfqpoint{3.318934in}{1.767466in}}%
\pgfpathcurveto{\pgfqpoint{3.324758in}{1.773290in}}{\pgfqpoint{3.328030in}{1.781190in}}{\pgfqpoint{3.328030in}{1.789426in}}%
\pgfpathcurveto{\pgfqpoint{3.328030in}{1.797663in}}{\pgfqpoint{3.324758in}{1.805563in}}{\pgfqpoint{3.318934in}{1.811387in}}%
\pgfpathcurveto{\pgfqpoint{3.313110in}{1.817211in}}{\pgfqpoint{3.305210in}{1.820483in}}{\pgfqpoint{3.296973in}{1.820483in}}%
\pgfpathcurveto{\pgfqpoint{3.288737in}{1.820483in}}{\pgfqpoint{3.280837in}{1.817211in}}{\pgfqpoint{3.275013in}{1.811387in}}%
\pgfpathcurveto{\pgfqpoint{3.269189in}{1.805563in}}{\pgfqpoint{3.265917in}{1.797663in}}{\pgfqpoint{3.265917in}{1.789426in}}%
\pgfpathcurveto{\pgfqpoint{3.265917in}{1.781190in}}{\pgfqpoint{3.269189in}{1.773290in}}{\pgfqpoint{3.275013in}{1.767466in}}%
\pgfpathcurveto{\pgfqpoint{3.280837in}{1.761642in}}{\pgfqpoint{3.288737in}{1.758370in}}{\pgfqpoint{3.296973in}{1.758370in}}%
\pgfpathclose%
\pgfusepath{stroke,fill}%
\end{pgfscope}%
\begin{pgfscope}%
\pgfpathrectangle{\pgfqpoint{0.100000in}{0.212622in}}{\pgfqpoint{3.696000in}{3.696000in}}%
\pgfusepath{clip}%
\pgfsetbuttcap%
\pgfsetroundjoin%
\definecolor{currentfill}{rgb}{0.121569,0.466667,0.705882}%
\pgfsetfillcolor{currentfill}%
\pgfsetfillopacity{0.524089}%
\pgfsetlinewidth{1.003750pt}%
\definecolor{currentstroke}{rgb}{0.121569,0.466667,0.705882}%
\pgfsetstrokecolor{currentstroke}%
\pgfsetstrokeopacity{0.524089}%
\pgfsetdash{}{0pt}%
\pgfpathmoveto{\pgfqpoint{3.295539in}{1.758619in}}%
\pgfpathcurveto{\pgfqpoint{3.303775in}{1.758619in}}{\pgfqpoint{3.311675in}{1.761892in}}{\pgfqpoint{3.317499in}{1.767715in}}%
\pgfpathcurveto{\pgfqpoint{3.323323in}{1.773539in}}{\pgfqpoint{3.326595in}{1.781439in}}{\pgfqpoint{3.326595in}{1.789676in}}%
\pgfpathcurveto{\pgfqpoint{3.326595in}{1.797912in}}{\pgfqpoint{3.323323in}{1.805812in}}{\pgfqpoint{3.317499in}{1.811636in}}%
\pgfpathcurveto{\pgfqpoint{3.311675in}{1.817460in}}{\pgfqpoint{3.303775in}{1.820732in}}{\pgfqpoint{3.295539in}{1.820732in}}%
\pgfpathcurveto{\pgfqpoint{3.287303in}{1.820732in}}{\pgfqpoint{3.279403in}{1.817460in}}{\pgfqpoint{3.273579in}{1.811636in}}%
\pgfpathcurveto{\pgfqpoint{3.267755in}{1.805812in}}{\pgfqpoint{3.264482in}{1.797912in}}{\pgfqpoint{3.264482in}{1.789676in}}%
\pgfpathcurveto{\pgfqpoint{3.264482in}{1.781439in}}{\pgfqpoint{3.267755in}{1.773539in}}{\pgfqpoint{3.273579in}{1.767715in}}%
\pgfpathcurveto{\pgfqpoint{3.279403in}{1.761892in}}{\pgfqpoint{3.287303in}{1.758619in}}{\pgfqpoint{3.295539in}{1.758619in}}%
\pgfpathclose%
\pgfusepath{stroke,fill}%
\end{pgfscope}%
\begin{pgfscope}%
\pgfpathrectangle{\pgfqpoint{0.100000in}{0.212622in}}{\pgfqpoint{3.696000in}{3.696000in}}%
\pgfusepath{clip}%
\pgfsetbuttcap%
\pgfsetroundjoin%
\definecolor{currentfill}{rgb}{0.121569,0.466667,0.705882}%
\pgfsetfillcolor{currentfill}%
\pgfsetfillopacity{0.524091}%
\pgfsetlinewidth{1.003750pt}%
\definecolor{currentstroke}{rgb}{0.121569,0.466667,0.705882}%
\pgfsetstrokecolor{currentstroke}%
\pgfsetstrokeopacity{0.524091}%
\pgfsetdash{}{0pt}%
\pgfpathmoveto{\pgfqpoint{1.266772in}{2.124572in}}%
\pgfpathcurveto{\pgfqpoint{1.275008in}{2.124572in}}{\pgfqpoint{1.282908in}{2.127845in}}{\pgfqpoint{1.288732in}{2.133668in}}%
\pgfpathcurveto{\pgfqpoint{1.294556in}{2.139492in}}{\pgfqpoint{1.297829in}{2.147392in}}{\pgfqpoint{1.297829in}{2.155629in}}%
\pgfpathcurveto{\pgfqpoint{1.297829in}{2.163865in}}{\pgfqpoint{1.294556in}{2.171765in}}{\pgfqpoint{1.288732in}{2.177589in}}%
\pgfpathcurveto{\pgfqpoint{1.282908in}{2.183413in}}{\pgfqpoint{1.275008in}{2.186685in}}{\pgfqpoint{1.266772in}{2.186685in}}%
\pgfpathcurveto{\pgfqpoint{1.258536in}{2.186685in}}{\pgfqpoint{1.250636in}{2.183413in}}{\pgfqpoint{1.244812in}{2.177589in}}%
\pgfpathcurveto{\pgfqpoint{1.238988in}{2.171765in}}{\pgfqpoint{1.235716in}{2.163865in}}{\pgfqpoint{1.235716in}{2.155629in}}%
\pgfpathcurveto{\pgfqpoint{1.235716in}{2.147392in}}{\pgfqpoint{1.238988in}{2.139492in}}{\pgfqpoint{1.244812in}{2.133668in}}%
\pgfpathcurveto{\pgfqpoint{1.250636in}{2.127845in}}{\pgfqpoint{1.258536in}{2.124572in}}{\pgfqpoint{1.266772in}{2.124572in}}%
\pgfpathclose%
\pgfusepath{stroke,fill}%
\end{pgfscope}%
\begin{pgfscope}%
\pgfpathrectangle{\pgfqpoint{0.100000in}{0.212622in}}{\pgfqpoint{3.696000in}{3.696000in}}%
\pgfusepath{clip}%
\pgfsetbuttcap%
\pgfsetroundjoin%
\definecolor{currentfill}{rgb}{0.121569,0.466667,0.705882}%
\pgfsetfillcolor{currentfill}%
\pgfsetfillopacity{0.524511}%
\pgfsetlinewidth{1.003750pt}%
\definecolor{currentstroke}{rgb}{0.121569,0.466667,0.705882}%
\pgfsetstrokecolor{currentstroke}%
\pgfsetstrokeopacity{0.524511}%
\pgfsetdash{}{0pt}%
\pgfpathmoveto{\pgfqpoint{3.294994in}{1.758707in}}%
\pgfpathcurveto{\pgfqpoint{3.303230in}{1.758707in}}{\pgfqpoint{3.311130in}{1.761979in}}{\pgfqpoint{3.316954in}{1.767803in}}%
\pgfpathcurveto{\pgfqpoint{3.322778in}{1.773627in}}{\pgfqpoint{3.326050in}{1.781527in}}{\pgfqpoint{3.326050in}{1.789764in}}%
\pgfpathcurveto{\pgfqpoint{3.326050in}{1.798000in}}{\pgfqpoint{3.322778in}{1.805900in}}{\pgfqpoint{3.316954in}{1.811724in}}%
\pgfpathcurveto{\pgfqpoint{3.311130in}{1.817548in}}{\pgfqpoint{3.303230in}{1.820820in}}{\pgfqpoint{3.294994in}{1.820820in}}%
\pgfpathcurveto{\pgfqpoint{3.286757in}{1.820820in}}{\pgfqpoint{3.278857in}{1.817548in}}{\pgfqpoint{3.273033in}{1.811724in}}%
\pgfpathcurveto{\pgfqpoint{3.267210in}{1.805900in}}{\pgfqpoint{3.263937in}{1.798000in}}{\pgfqpoint{3.263937in}{1.789764in}}%
\pgfpathcurveto{\pgfqpoint{3.263937in}{1.781527in}}{\pgfqpoint{3.267210in}{1.773627in}}{\pgfqpoint{3.273033in}{1.767803in}}%
\pgfpathcurveto{\pgfqpoint{3.278857in}{1.761979in}}{\pgfqpoint{3.286757in}{1.758707in}}{\pgfqpoint{3.294994in}{1.758707in}}%
\pgfpathclose%
\pgfusepath{stroke,fill}%
\end{pgfscope}%
\begin{pgfscope}%
\pgfpathrectangle{\pgfqpoint{0.100000in}{0.212622in}}{\pgfqpoint{3.696000in}{3.696000in}}%
\pgfusepath{clip}%
\pgfsetbuttcap%
\pgfsetroundjoin%
\definecolor{currentfill}{rgb}{0.121569,0.466667,0.705882}%
\pgfsetfillcolor{currentfill}%
\pgfsetfillopacity{0.524721}%
\pgfsetlinewidth{1.003750pt}%
\definecolor{currentstroke}{rgb}{0.121569,0.466667,0.705882}%
\pgfsetstrokecolor{currentstroke}%
\pgfsetstrokeopacity{0.524721}%
\pgfsetdash{}{0pt}%
\pgfpathmoveto{\pgfqpoint{3.294485in}{1.758837in}}%
\pgfpathcurveto{\pgfqpoint{3.302722in}{1.758837in}}{\pgfqpoint{3.310622in}{1.762109in}}{\pgfqpoint{3.316446in}{1.767933in}}%
\pgfpathcurveto{\pgfqpoint{3.322270in}{1.773757in}}{\pgfqpoint{3.325542in}{1.781657in}}{\pgfqpoint{3.325542in}{1.789893in}}%
\pgfpathcurveto{\pgfqpoint{3.325542in}{1.798129in}}{\pgfqpoint{3.322270in}{1.806029in}}{\pgfqpoint{3.316446in}{1.811853in}}%
\pgfpathcurveto{\pgfqpoint{3.310622in}{1.817677in}}{\pgfqpoint{3.302722in}{1.820950in}}{\pgfqpoint{3.294485in}{1.820950in}}%
\pgfpathcurveto{\pgfqpoint{3.286249in}{1.820950in}}{\pgfqpoint{3.278349in}{1.817677in}}{\pgfqpoint{3.272525in}{1.811853in}}%
\pgfpathcurveto{\pgfqpoint{3.266701in}{1.806029in}}{\pgfqpoint{3.263429in}{1.798129in}}{\pgfqpoint{3.263429in}{1.789893in}}%
\pgfpathcurveto{\pgfqpoint{3.263429in}{1.781657in}}{\pgfqpoint{3.266701in}{1.773757in}}{\pgfqpoint{3.272525in}{1.767933in}}%
\pgfpathcurveto{\pgfqpoint{3.278349in}{1.762109in}}{\pgfqpoint{3.286249in}{1.758837in}}{\pgfqpoint{3.294485in}{1.758837in}}%
\pgfpathclose%
\pgfusepath{stroke,fill}%
\end{pgfscope}%
\begin{pgfscope}%
\pgfpathrectangle{\pgfqpoint{0.100000in}{0.212622in}}{\pgfqpoint{3.696000in}{3.696000in}}%
\pgfusepath{clip}%
\pgfsetbuttcap%
\pgfsetroundjoin%
\definecolor{currentfill}{rgb}{0.121569,0.466667,0.705882}%
\pgfsetfillcolor{currentfill}%
\pgfsetfillopacity{0.524839}%
\pgfsetlinewidth{1.003750pt}%
\definecolor{currentstroke}{rgb}{0.121569,0.466667,0.705882}%
\pgfsetstrokecolor{currentstroke}%
\pgfsetstrokeopacity{0.524839}%
\pgfsetdash{}{0pt}%
\pgfpathmoveto{\pgfqpoint{3.294236in}{1.758880in}}%
\pgfpathcurveto{\pgfqpoint{3.302472in}{1.758880in}}{\pgfqpoint{3.310372in}{1.762152in}}{\pgfqpoint{3.316196in}{1.767976in}}%
\pgfpathcurveto{\pgfqpoint{3.322020in}{1.773800in}}{\pgfqpoint{3.325292in}{1.781700in}}{\pgfqpoint{3.325292in}{1.789936in}}%
\pgfpathcurveto{\pgfqpoint{3.325292in}{1.798172in}}{\pgfqpoint{3.322020in}{1.806072in}}{\pgfqpoint{3.316196in}{1.811896in}}%
\pgfpathcurveto{\pgfqpoint{3.310372in}{1.817720in}}{\pgfqpoint{3.302472in}{1.820993in}}{\pgfqpoint{3.294236in}{1.820993in}}%
\pgfpathcurveto{\pgfqpoint{3.285999in}{1.820993in}}{\pgfqpoint{3.278099in}{1.817720in}}{\pgfqpoint{3.272276in}{1.811896in}}%
\pgfpathcurveto{\pgfqpoint{3.266452in}{1.806072in}}{\pgfqpoint{3.263179in}{1.798172in}}{\pgfqpoint{3.263179in}{1.789936in}}%
\pgfpathcurveto{\pgfqpoint{3.263179in}{1.781700in}}{\pgfqpoint{3.266452in}{1.773800in}}{\pgfqpoint{3.272276in}{1.767976in}}%
\pgfpathcurveto{\pgfqpoint{3.278099in}{1.762152in}}{\pgfqpoint{3.285999in}{1.758880in}}{\pgfqpoint{3.294236in}{1.758880in}}%
\pgfpathclose%
\pgfusepath{stroke,fill}%
\end{pgfscope}%
\begin{pgfscope}%
\pgfpathrectangle{\pgfqpoint{0.100000in}{0.212622in}}{\pgfqpoint{3.696000in}{3.696000in}}%
\pgfusepath{clip}%
\pgfsetbuttcap%
\pgfsetroundjoin%
\definecolor{currentfill}{rgb}{0.121569,0.466667,0.705882}%
\pgfsetfillcolor{currentfill}%
\pgfsetfillopacity{0.525135}%
\pgfsetlinewidth{1.003750pt}%
\definecolor{currentstroke}{rgb}{0.121569,0.466667,0.705882}%
\pgfsetstrokecolor{currentstroke}%
\pgfsetstrokeopacity{0.525135}%
\pgfsetdash{}{0pt}%
\pgfpathmoveto{\pgfqpoint{3.293926in}{1.758909in}}%
\pgfpathcurveto{\pgfqpoint{3.302162in}{1.758909in}}{\pgfqpoint{3.310062in}{1.762181in}}{\pgfqpoint{3.315886in}{1.768005in}}%
\pgfpathcurveto{\pgfqpoint{3.321710in}{1.773829in}}{\pgfqpoint{3.324983in}{1.781729in}}{\pgfqpoint{3.324983in}{1.789965in}}%
\pgfpathcurveto{\pgfqpoint{3.324983in}{1.798202in}}{\pgfqpoint{3.321710in}{1.806102in}}{\pgfqpoint{3.315886in}{1.811926in}}%
\pgfpathcurveto{\pgfqpoint{3.310062in}{1.817749in}}{\pgfqpoint{3.302162in}{1.821022in}}{\pgfqpoint{3.293926in}{1.821022in}}%
\pgfpathcurveto{\pgfqpoint{3.285690in}{1.821022in}}{\pgfqpoint{3.277790in}{1.817749in}}{\pgfqpoint{3.271966in}{1.811926in}}%
\pgfpathcurveto{\pgfqpoint{3.266142in}{1.806102in}}{\pgfqpoint{3.262870in}{1.798202in}}{\pgfqpoint{3.262870in}{1.789965in}}%
\pgfpathcurveto{\pgfqpoint{3.262870in}{1.781729in}}{\pgfqpoint{3.266142in}{1.773829in}}{\pgfqpoint{3.271966in}{1.768005in}}%
\pgfpathcurveto{\pgfqpoint{3.277790in}{1.762181in}}{\pgfqpoint{3.285690in}{1.758909in}}{\pgfqpoint{3.293926in}{1.758909in}}%
\pgfpathclose%
\pgfusepath{stroke,fill}%
\end{pgfscope}%
\begin{pgfscope}%
\pgfpathrectangle{\pgfqpoint{0.100000in}{0.212622in}}{\pgfqpoint{3.696000in}{3.696000in}}%
\pgfusepath{clip}%
\pgfsetbuttcap%
\pgfsetroundjoin%
\definecolor{currentfill}{rgb}{0.121569,0.466667,0.705882}%
\pgfsetfillcolor{currentfill}%
\pgfsetfillopacity{0.525942}%
\pgfsetlinewidth{1.003750pt}%
\definecolor{currentstroke}{rgb}{0.121569,0.466667,0.705882}%
\pgfsetstrokecolor{currentstroke}%
\pgfsetstrokeopacity{0.525942}%
\pgfsetdash{}{0pt}%
\pgfpathmoveto{\pgfqpoint{3.292008in}{1.759340in}}%
\pgfpathcurveto{\pgfqpoint{3.300244in}{1.759340in}}{\pgfqpoint{3.308144in}{1.762613in}}{\pgfqpoint{3.313968in}{1.768437in}}%
\pgfpathcurveto{\pgfqpoint{3.319792in}{1.774261in}}{\pgfqpoint{3.323065in}{1.782161in}}{\pgfqpoint{3.323065in}{1.790397in}}%
\pgfpathcurveto{\pgfqpoint{3.323065in}{1.798633in}}{\pgfqpoint{3.319792in}{1.806533in}}{\pgfqpoint{3.313968in}{1.812357in}}%
\pgfpathcurveto{\pgfqpoint{3.308144in}{1.818181in}}{\pgfqpoint{3.300244in}{1.821453in}}{\pgfqpoint{3.292008in}{1.821453in}}%
\pgfpathcurveto{\pgfqpoint{3.283772in}{1.821453in}}{\pgfqpoint{3.275872in}{1.818181in}}{\pgfqpoint{3.270048in}{1.812357in}}%
\pgfpathcurveto{\pgfqpoint{3.264224in}{1.806533in}}{\pgfqpoint{3.260952in}{1.798633in}}{\pgfqpoint{3.260952in}{1.790397in}}%
\pgfpathcurveto{\pgfqpoint{3.260952in}{1.782161in}}{\pgfqpoint{3.264224in}{1.774261in}}{\pgfqpoint{3.270048in}{1.768437in}}%
\pgfpathcurveto{\pgfqpoint{3.275872in}{1.762613in}}{\pgfqpoint{3.283772in}{1.759340in}}{\pgfqpoint{3.292008in}{1.759340in}}%
\pgfpathclose%
\pgfusepath{stroke,fill}%
\end{pgfscope}%
\begin{pgfscope}%
\pgfpathrectangle{\pgfqpoint{0.100000in}{0.212622in}}{\pgfqpoint{3.696000in}{3.696000in}}%
\pgfusepath{clip}%
\pgfsetbuttcap%
\pgfsetroundjoin%
\definecolor{currentfill}{rgb}{0.121569,0.466667,0.705882}%
\pgfsetfillcolor{currentfill}%
\pgfsetfillopacity{0.526202}%
\pgfsetlinewidth{1.003750pt}%
\definecolor{currentstroke}{rgb}{0.121569,0.466667,0.705882}%
\pgfsetstrokecolor{currentstroke}%
\pgfsetstrokeopacity{0.526202}%
\pgfsetdash{}{0pt}%
\pgfpathmoveto{\pgfqpoint{1.261662in}{2.124847in}}%
\pgfpathcurveto{\pgfqpoint{1.269898in}{2.124847in}}{\pgfqpoint{1.277798in}{2.128120in}}{\pgfqpoint{1.283622in}{2.133944in}}%
\pgfpathcurveto{\pgfqpoint{1.289446in}{2.139768in}}{\pgfqpoint{1.292718in}{2.147668in}}{\pgfqpoint{1.292718in}{2.155904in}}%
\pgfpathcurveto{\pgfqpoint{1.292718in}{2.164140in}}{\pgfqpoint{1.289446in}{2.172040in}}{\pgfqpoint{1.283622in}{2.177864in}}%
\pgfpathcurveto{\pgfqpoint{1.277798in}{2.183688in}}{\pgfqpoint{1.269898in}{2.186960in}}{\pgfqpoint{1.261662in}{2.186960in}}%
\pgfpathcurveto{\pgfqpoint{1.253425in}{2.186960in}}{\pgfqpoint{1.245525in}{2.183688in}}{\pgfqpoint{1.239701in}{2.177864in}}%
\pgfpathcurveto{\pgfqpoint{1.233877in}{2.172040in}}{\pgfqpoint{1.230605in}{2.164140in}}{\pgfqpoint{1.230605in}{2.155904in}}%
\pgfpathcurveto{\pgfqpoint{1.230605in}{2.147668in}}{\pgfqpoint{1.233877in}{2.139768in}}{\pgfqpoint{1.239701in}{2.133944in}}%
\pgfpathcurveto{\pgfqpoint{1.245525in}{2.128120in}}{\pgfqpoint{1.253425in}{2.124847in}}{\pgfqpoint{1.261662in}{2.124847in}}%
\pgfpathclose%
\pgfusepath{stroke,fill}%
\end{pgfscope}%
\begin{pgfscope}%
\pgfpathrectangle{\pgfqpoint{0.100000in}{0.212622in}}{\pgfqpoint{3.696000in}{3.696000in}}%
\pgfusepath{clip}%
\pgfsetbuttcap%
\pgfsetroundjoin%
\definecolor{currentfill}{rgb}{0.121569,0.466667,0.705882}%
\pgfsetfillcolor{currentfill}%
\pgfsetfillopacity{0.526824}%
\pgfsetlinewidth{1.003750pt}%
\definecolor{currentstroke}{rgb}{0.121569,0.466667,0.705882}%
\pgfsetstrokecolor{currentstroke}%
\pgfsetstrokeopacity{0.526824}%
\pgfsetdash{}{0pt}%
\pgfpathmoveto{\pgfqpoint{3.289962in}{1.759755in}}%
\pgfpathcurveto{\pgfqpoint{3.298198in}{1.759755in}}{\pgfqpoint{3.306098in}{1.763027in}}{\pgfqpoint{3.311922in}{1.768851in}}%
\pgfpathcurveto{\pgfqpoint{3.317746in}{1.774675in}}{\pgfqpoint{3.321018in}{1.782575in}}{\pgfqpoint{3.321018in}{1.790811in}}%
\pgfpathcurveto{\pgfqpoint{3.321018in}{1.799048in}}{\pgfqpoint{3.317746in}{1.806948in}}{\pgfqpoint{3.311922in}{1.812772in}}%
\pgfpathcurveto{\pgfqpoint{3.306098in}{1.818595in}}{\pgfqpoint{3.298198in}{1.821868in}}{\pgfqpoint{3.289962in}{1.821868in}}%
\pgfpathcurveto{\pgfqpoint{3.281725in}{1.821868in}}{\pgfqpoint{3.273825in}{1.818595in}}{\pgfqpoint{3.268001in}{1.812772in}}%
\pgfpathcurveto{\pgfqpoint{3.262177in}{1.806948in}}{\pgfqpoint{3.258905in}{1.799048in}}{\pgfqpoint{3.258905in}{1.790811in}}%
\pgfpathcurveto{\pgfqpoint{3.258905in}{1.782575in}}{\pgfqpoint{3.262177in}{1.774675in}}{\pgfqpoint{3.268001in}{1.768851in}}%
\pgfpathcurveto{\pgfqpoint{3.273825in}{1.763027in}}{\pgfqpoint{3.281725in}{1.759755in}}{\pgfqpoint{3.289962in}{1.759755in}}%
\pgfpathclose%
\pgfusepath{stroke,fill}%
\end{pgfscope}%
\begin{pgfscope}%
\pgfpathrectangle{\pgfqpoint{0.100000in}{0.212622in}}{\pgfqpoint{3.696000in}{3.696000in}}%
\pgfusepath{clip}%
\pgfsetbuttcap%
\pgfsetroundjoin%
\definecolor{currentfill}{rgb}{0.121569,0.466667,0.705882}%
\pgfsetfillcolor{currentfill}%
\pgfsetfillopacity{0.528108}%
\pgfsetlinewidth{1.003750pt}%
\definecolor{currentstroke}{rgb}{0.121569,0.466667,0.705882}%
\pgfsetstrokecolor{currentstroke}%
\pgfsetstrokeopacity{0.528108}%
\pgfsetdash{}{0pt}%
\pgfpathmoveto{\pgfqpoint{1.258651in}{2.124821in}}%
\pgfpathcurveto{\pgfqpoint{1.266887in}{2.124821in}}{\pgfqpoint{1.274787in}{2.128093in}}{\pgfqpoint{1.280611in}{2.133917in}}%
\pgfpathcurveto{\pgfqpoint{1.286435in}{2.139741in}}{\pgfqpoint{1.289707in}{2.147641in}}{\pgfqpoint{1.289707in}{2.155877in}}%
\pgfpathcurveto{\pgfqpoint{1.289707in}{2.164114in}}{\pgfqpoint{1.286435in}{2.172014in}}{\pgfqpoint{1.280611in}{2.177838in}}%
\pgfpathcurveto{\pgfqpoint{1.274787in}{2.183662in}}{\pgfqpoint{1.266887in}{2.186934in}}{\pgfqpoint{1.258651in}{2.186934in}}%
\pgfpathcurveto{\pgfqpoint{1.250414in}{2.186934in}}{\pgfqpoint{1.242514in}{2.183662in}}{\pgfqpoint{1.236690in}{2.177838in}}%
\pgfpathcurveto{\pgfqpoint{1.230866in}{2.172014in}}{\pgfqpoint{1.227594in}{2.164114in}}{\pgfqpoint{1.227594in}{2.155877in}}%
\pgfpathcurveto{\pgfqpoint{1.227594in}{2.147641in}}{\pgfqpoint{1.230866in}{2.139741in}}{\pgfqpoint{1.236690in}{2.133917in}}%
\pgfpathcurveto{\pgfqpoint{1.242514in}{2.128093in}}{\pgfqpoint{1.250414in}{2.124821in}}{\pgfqpoint{1.258651in}{2.124821in}}%
\pgfpathclose%
\pgfusepath{stroke,fill}%
\end{pgfscope}%
\begin{pgfscope}%
\pgfpathrectangle{\pgfqpoint{0.100000in}{0.212622in}}{\pgfqpoint{3.696000in}{3.696000in}}%
\pgfusepath{clip}%
\pgfsetbuttcap%
\pgfsetroundjoin%
\definecolor{currentfill}{rgb}{0.121569,0.466667,0.705882}%
\pgfsetfillcolor{currentfill}%
\pgfsetfillopacity{0.528139}%
\pgfsetlinewidth{1.003750pt}%
\definecolor{currentstroke}{rgb}{0.121569,0.466667,0.705882}%
\pgfsetstrokecolor{currentstroke}%
\pgfsetstrokeopacity{0.528139}%
\pgfsetdash{}{0pt}%
\pgfpathmoveto{\pgfqpoint{3.288221in}{1.759955in}}%
\pgfpathcurveto{\pgfqpoint{3.296457in}{1.759955in}}{\pgfqpoint{3.304358in}{1.763228in}}{\pgfqpoint{3.310181in}{1.769052in}}%
\pgfpathcurveto{\pgfqpoint{3.316005in}{1.774876in}}{\pgfqpoint{3.319278in}{1.782776in}}{\pgfqpoint{3.319278in}{1.791012in}}%
\pgfpathcurveto{\pgfqpoint{3.319278in}{1.799248in}}{\pgfqpoint{3.316005in}{1.807148in}}{\pgfqpoint{3.310181in}{1.812972in}}%
\pgfpathcurveto{\pgfqpoint{3.304358in}{1.818796in}}{\pgfqpoint{3.296457in}{1.822068in}}{\pgfqpoint{3.288221in}{1.822068in}}%
\pgfpathcurveto{\pgfqpoint{3.279985in}{1.822068in}}{\pgfqpoint{3.272085in}{1.818796in}}{\pgfqpoint{3.266261in}{1.812972in}}%
\pgfpathcurveto{\pgfqpoint{3.260437in}{1.807148in}}{\pgfqpoint{3.257165in}{1.799248in}}{\pgfqpoint{3.257165in}{1.791012in}}%
\pgfpathcurveto{\pgfqpoint{3.257165in}{1.782776in}}{\pgfqpoint{3.260437in}{1.774876in}}{\pgfqpoint{3.266261in}{1.769052in}}%
\pgfpathcurveto{\pgfqpoint{3.272085in}{1.763228in}}{\pgfqpoint{3.279985in}{1.759955in}}{\pgfqpoint{3.288221in}{1.759955in}}%
\pgfpathclose%
\pgfusepath{stroke,fill}%
\end{pgfscope}%
\begin{pgfscope}%
\pgfpathrectangle{\pgfqpoint{0.100000in}{0.212622in}}{\pgfqpoint{3.696000in}{3.696000in}}%
\pgfusepath{clip}%
\pgfsetbuttcap%
\pgfsetroundjoin%
\definecolor{currentfill}{rgb}{0.121569,0.466667,0.705882}%
\pgfsetfillcolor{currentfill}%
\pgfsetfillopacity{0.528863}%
\pgfsetlinewidth{1.003750pt}%
\definecolor{currentstroke}{rgb}{0.121569,0.466667,0.705882}%
\pgfsetstrokecolor{currentstroke}%
\pgfsetstrokeopacity{0.528863}%
\pgfsetdash{}{0pt}%
\pgfpathmoveto{\pgfqpoint{3.287187in}{1.760135in}}%
\pgfpathcurveto{\pgfqpoint{3.295423in}{1.760135in}}{\pgfqpoint{3.303323in}{1.763407in}}{\pgfqpoint{3.309147in}{1.769231in}}%
\pgfpathcurveto{\pgfqpoint{3.314971in}{1.775055in}}{\pgfqpoint{3.318244in}{1.782955in}}{\pgfqpoint{3.318244in}{1.791191in}}%
\pgfpathcurveto{\pgfqpoint{3.318244in}{1.799427in}}{\pgfqpoint{3.314971in}{1.807327in}}{\pgfqpoint{3.309147in}{1.813151in}}%
\pgfpathcurveto{\pgfqpoint{3.303323in}{1.818975in}}{\pgfqpoint{3.295423in}{1.822248in}}{\pgfqpoint{3.287187in}{1.822248in}}%
\pgfpathcurveto{\pgfqpoint{3.278951in}{1.822248in}}{\pgfqpoint{3.271051in}{1.818975in}}{\pgfqpoint{3.265227in}{1.813151in}}%
\pgfpathcurveto{\pgfqpoint{3.259403in}{1.807327in}}{\pgfqpoint{3.256131in}{1.799427in}}{\pgfqpoint{3.256131in}{1.791191in}}%
\pgfpathcurveto{\pgfqpoint{3.256131in}{1.782955in}}{\pgfqpoint{3.259403in}{1.775055in}}{\pgfqpoint{3.265227in}{1.769231in}}%
\pgfpathcurveto{\pgfqpoint{3.271051in}{1.763407in}}{\pgfqpoint{3.278951in}{1.760135in}}{\pgfqpoint{3.287187in}{1.760135in}}%
\pgfpathclose%
\pgfusepath{stroke,fill}%
\end{pgfscope}%
\begin{pgfscope}%
\pgfpathrectangle{\pgfqpoint{0.100000in}{0.212622in}}{\pgfqpoint{3.696000in}{3.696000in}}%
\pgfusepath{clip}%
\pgfsetbuttcap%
\pgfsetroundjoin%
\definecolor{currentfill}{rgb}{0.121569,0.466667,0.705882}%
\pgfsetfillcolor{currentfill}%
\pgfsetfillopacity{0.529567}%
\pgfsetlinewidth{1.003750pt}%
\definecolor{currentstroke}{rgb}{0.121569,0.466667,0.705882}%
\pgfsetstrokecolor{currentstroke}%
\pgfsetstrokeopacity{0.529567}%
\pgfsetdash{}{0pt}%
\pgfpathmoveto{\pgfqpoint{3.285215in}{1.760639in}}%
\pgfpathcurveto{\pgfqpoint{3.293451in}{1.760639in}}{\pgfqpoint{3.301351in}{1.763911in}}{\pgfqpoint{3.307175in}{1.769735in}}%
\pgfpathcurveto{\pgfqpoint{3.312999in}{1.775559in}}{\pgfqpoint{3.316271in}{1.783459in}}{\pgfqpoint{3.316271in}{1.791696in}}%
\pgfpathcurveto{\pgfqpoint{3.316271in}{1.799932in}}{\pgfqpoint{3.312999in}{1.807832in}}{\pgfqpoint{3.307175in}{1.813656in}}%
\pgfpathcurveto{\pgfqpoint{3.301351in}{1.819480in}}{\pgfqpoint{3.293451in}{1.822752in}}{\pgfqpoint{3.285215in}{1.822752in}}%
\pgfpathcurveto{\pgfqpoint{3.276978in}{1.822752in}}{\pgfqpoint{3.269078in}{1.819480in}}{\pgfqpoint{3.263254in}{1.813656in}}%
\pgfpathcurveto{\pgfqpoint{3.257430in}{1.807832in}}{\pgfqpoint{3.254158in}{1.799932in}}{\pgfqpoint{3.254158in}{1.791696in}}%
\pgfpathcurveto{\pgfqpoint{3.254158in}{1.783459in}}{\pgfqpoint{3.257430in}{1.775559in}}{\pgfqpoint{3.263254in}{1.769735in}}%
\pgfpathcurveto{\pgfqpoint{3.269078in}{1.763911in}}{\pgfqpoint{3.276978in}{1.760639in}}{\pgfqpoint{3.285215in}{1.760639in}}%
\pgfpathclose%
\pgfusepath{stroke,fill}%
\end{pgfscope}%
\begin{pgfscope}%
\pgfpathrectangle{\pgfqpoint{0.100000in}{0.212622in}}{\pgfqpoint{3.696000in}{3.696000in}}%
\pgfusepath{clip}%
\pgfsetbuttcap%
\pgfsetroundjoin%
\definecolor{currentfill}{rgb}{0.121569,0.466667,0.705882}%
\pgfsetfillcolor{currentfill}%
\pgfsetfillopacity{0.529639}%
\pgfsetlinewidth{1.003750pt}%
\definecolor{currentstroke}{rgb}{0.121569,0.466667,0.705882}%
\pgfsetstrokecolor{currentstroke}%
\pgfsetstrokeopacity{0.529639}%
\pgfsetdash{}{0pt}%
\pgfpathmoveto{\pgfqpoint{1.254706in}{2.125125in}}%
\pgfpathcurveto{\pgfqpoint{1.262942in}{2.125125in}}{\pgfqpoint{1.270842in}{2.128397in}}{\pgfqpoint{1.276666in}{2.134221in}}%
\pgfpathcurveto{\pgfqpoint{1.282490in}{2.140045in}}{\pgfqpoint{1.285762in}{2.147945in}}{\pgfqpoint{1.285762in}{2.156181in}}%
\pgfpathcurveto{\pgfqpoint{1.285762in}{2.164418in}}{\pgfqpoint{1.282490in}{2.172318in}}{\pgfqpoint{1.276666in}{2.178142in}}%
\pgfpathcurveto{\pgfqpoint{1.270842in}{2.183965in}}{\pgfqpoint{1.262942in}{2.187238in}}{\pgfqpoint{1.254706in}{2.187238in}}%
\pgfpathcurveto{\pgfqpoint{1.246469in}{2.187238in}}{\pgfqpoint{1.238569in}{2.183965in}}{\pgfqpoint{1.232745in}{2.178142in}}%
\pgfpathcurveto{\pgfqpoint{1.226922in}{2.172318in}}{\pgfqpoint{1.223649in}{2.164418in}}{\pgfqpoint{1.223649in}{2.156181in}}%
\pgfpathcurveto{\pgfqpoint{1.223649in}{2.147945in}}{\pgfqpoint{1.226922in}{2.140045in}}{\pgfqpoint{1.232745in}{2.134221in}}%
\pgfpathcurveto{\pgfqpoint{1.238569in}{2.128397in}}{\pgfqpoint{1.246469in}{2.125125in}}{\pgfqpoint{1.254706in}{2.125125in}}%
\pgfpathclose%
\pgfusepath{stroke,fill}%
\end{pgfscope}%
\begin{pgfscope}%
\pgfpathrectangle{\pgfqpoint{0.100000in}{0.212622in}}{\pgfqpoint{3.696000in}{3.696000in}}%
\pgfusepath{clip}%
\pgfsetbuttcap%
\pgfsetroundjoin%
\definecolor{currentfill}{rgb}{0.121569,0.466667,0.705882}%
\pgfsetfillcolor{currentfill}%
\pgfsetfillopacity{0.530580}%
\pgfsetlinewidth{1.003750pt}%
\definecolor{currentstroke}{rgb}{0.121569,0.466667,0.705882}%
\pgfsetstrokecolor{currentstroke}%
\pgfsetstrokeopacity{0.530580}%
\pgfsetdash{}{0pt}%
\pgfpathmoveto{\pgfqpoint{3.283360in}{1.760822in}}%
\pgfpathcurveto{\pgfqpoint{3.291596in}{1.760822in}}{\pgfqpoint{3.299496in}{1.764094in}}{\pgfqpoint{3.305320in}{1.769918in}}%
\pgfpathcurveto{\pgfqpoint{3.311144in}{1.775742in}}{\pgfqpoint{3.314416in}{1.783642in}}{\pgfqpoint{3.314416in}{1.791878in}}%
\pgfpathcurveto{\pgfqpoint{3.314416in}{1.800114in}}{\pgfqpoint{3.311144in}{1.808015in}}{\pgfqpoint{3.305320in}{1.813838in}}%
\pgfpathcurveto{\pgfqpoint{3.299496in}{1.819662in}}{\pgfqpoint{3.291596in}{1.822935in}}{\pgfqpoint{3.283360in}{1.822935in}}%
\pgfpathcurveto{\pgfqpoint{3.275124in}{1.822935in}}{\pgfqpoint{3.267224in}{1.819662in}}{\pgfqpoint{3.261400in}{1.813838in}}%
\pgfpathcurveto{\pgfqpoint{3.255576in}{1.808015in}}{\pgfqpoint{3.252303in}{1.800114in}}{\pgfqpoint{3.252303in}{1.791878in}}%
\pgfpathcurveto{\pgfqpoint{3.252303in}{1.783642in}}{\pgfqpoint{3.255576in}{1.775742in}}{\pgfqpoint{3.261400in}{1.769918in}}%
\pgfpathcurveto{\pgfqpoint{3.267224in}{1.764094in}}{\pgfqpoint{3.275124in}{1.760822in}}{\pgfqpoint{3.283360in}{1.760822in}}%
\pgfpathclose%
\pgfusepath{stroke,fill}%
\end{pgfscope}%
\begin{pgfscope}%
\pgfpathrectangle{\pgfqpoint{0.100000in}{0.212622in}}{\pgfqpoint{3.696000in}{3.696000in}}%
\pgfusepath{clip}%
\pgfsetbuttcap%
\pgfsetroundjoin%
\definecolor{currentfill}{rgb}{0.121569,0.466667,0.705882}%
\pgfsetfillcolor{currentfill}%
\pgfsetfillopacity{0.530784}%
\pgfsetlinewidth{1.003750pt}%
\definecolor{currentstroke}{rgb}{0.121569,0.466667,0.705882}%
\pgfsetstrokecolor{currentstroke}%
\pgfsetstrokeopacity{0.530784}%
\pgfsetdash{}{0pt}%
\pgfpathmoveto{\pgfqpoint{1.252799in}{2.125095in}}%
\pgfpathcurveto{\pgfqpoint{1.261035in}{2.125095in}}{\pgfqpoint{1.268935in}{2.128368in}}{\pgfqpoint{1.274759in}{2.134192in}}%
\pgfpathcurveto{\pgfqpoint{1.280583in}{2.140016in}}{\pgfqpoint{1.283855in}{2.147916in}}{\pgfqpoint{1.283855in}{2.156152in}}%
\pgfpathcurveto{\pgfqpoint{1.283855in}{2.164388in}}{\pgfqpoint{1.280583in}{2.172288in}}{\pgfqpoint{1.274759in}{2.178112in}}%
\pgfpathcurveto{\pgfqpoint{1.268935in}{2.183936in}}{\pgfqpoint{1.261035in}{2.187208in}}{\pgfqpoint{1.252799in}{2.187208in}}%
\pgfpathcurveto{\pgfqpoint{1.244563in}{2.187208in}}{\pgfqpoint{1.236663in}{2.183936in}}{\pgfqpoint{1.230839in}{2.178112in}}%
\pgfpathcurveto{\pgfqpoint{1.225015in}{2.172288in}}{\pgfqpoint{1.221742in}{2.164388in}}{\pgfqpoint{1.221742in}{2.156152in}}%
\pgfpathcurveto{\pgfqpoint{1.221742in}{2.147916in}}{\pgfqpoint{1.225015in}{2.140016in}}{\pgfqpoint{1.230839in}{2.134192in}}%
\pgfpathcurveto{\pgfqpoint{1.236663in}{2.128368in}}{\pgfqpoint{1.244563in}{2.125095in}}{\pgfqpoint{1.252799in}{2.125095in}}%
\pgfpathclose%
\pgfusepath{stroke,fill}%
\end{pgfscope}%
\begin{pgfscope}%
\pgfpathrectangle{\pgfqpoint{0.100000in}{0.212622in}}{\pgfqpoint{3.696000in}{3.696000in}}%
\pgfusepath{clip}%
\pgfsetbuttcap%
\pgfsetroundjoin%
\definecolor{currentfill}{rgb}{0.121569,0.466667,0.705882}%
\pgfsetfillcolor{currentfill}%
\pgfsetfillopacity{0.531176}%
\pgfsetlinewidth{1.003750pt}%
\definecolor{currentstroke}{rgb}{0.121569,0.466667,0.705882}%
\pgfsetstrokecolor{currentstroke}%
\pgfsetstrokeopacity{0.531176}%
\pgfsetdash{}{0pt}%
\pgfpathmoveto{\pgfqpoint{3.282675in}{1.760899in}}%
\pgfpathcurveto{\pgfqpoint{3.290912in}{1.760899in}}{\pgfqpoint{3.298812in}{1.764172in}}{\pgfqpoint{3.304636in}{1.769996in}}%
\pgfpathcurveto{\pgfqpoint{3.310460in}{1.775819in}}{\pgfqpoint{3.313732in}{1.783720in}}{\pgfqpoint{3.313732in}{1.791956in}}%
\pgfpathcurveto{\pgfqpoint{3.313732in}{1.800192in}}{\pgfqpoint{3.310460in}{1.808092in}}{\pgfqpoint{3.304636in}{1.813916in}}%
\pgfpathcurveto{\pgfqpoint{3.298812in}{1.819740in}}{\pgfqpoint{3.290912in}{1.823012in}}{\pgfqpoint{3.282675in}{1.823012in}}%
\pgfpathcurveto{\pgfqpoint{3.274439in}{1.823012in}}{\pgfqpoint{3.266539in}{1.819740in}}{\pgfqpoint{3.260715in}{1.813916in}}%
\pgfpathcurveto{\pgfqpoint{3.254891in}{1.808092in}}{\pgfqpoint{3.251619in}{1.800192in}}{\pgfqpoint{3.251619in}{1.791956in}}%
\pgfpathcurveto{\pgfqpoint{3.251619in}{1.783720in}}{\pgfqpoint{3.254891in}{1.775819in}}{\pgfqpoint{3.260715in}{1.769996in}}%
\pgfpathcurveto{\pgfqpoint{3.266539in}{1.764172in}}{\pgfqpoint{3.274439in}{1.760899in}}{\pgfqpoint{3.282675in}{1.760899in}}%
\pgfpathclose%
\pgfusepath{stroke,fill}%
\end{pgfscope}%
\begin{pgfscope}%
\pgfpathrectangle{\pgfqpoint{0.100000in}{0.212622in}}{\pgfqpoint{3.696000in}{3.696000in}}%
\pgfusepath{clip}%
\pgfsetbuttcap%
\pgfsetroundjoin%
\definecolor{currentfill}{rgb}{0.121569,0.466667,0.705882}%
\pgfsetfillcolor{currentfill}%
\pgfsetfillopacity{0.531301}%
\pgfsetlinewidth{1.003750pt}%
\definecolor{currentstroke}{rgb}{0.121569,0.466667,0.705882}%
\pgfsetstrokecolor{currentstroke}%
\pgfsetstrokeopacity{0.531301}%
\pgfsetdash{}{0pt}%
\pgfpathmoveto{\pgfqpoint{1.251534in}{2.125145in}}%
\pgfpathcurveto{\pgfqpoint{1.259770in}{2.125145in}}{\pgfqpoint{1.267670in}{2.128417in}}{\pgfqpoint{1.273494in}{2.134241in}}%
\pgfpathcurveto{\pgfqpoint{1.279318in}{2.140065in}}{\pgfqpoint{1.282590in}{2.147965in}}{\pgfqpoint{1.282590in}{2.156202in}}%
\pgfpathcurveto{\pgfqpoint{1.282590in}{2.164438in}}{\pgfqpoint{1.279318in}{2.172338in}}{\pgfqpoint{1.273494in}{2.178162in}}%
\pgfpathcurveto{\pgfqpoint{1.267670in}{2.183986in}}{\pgfqpoint{1.259770in}{2.187258in}}{\pgfqpoint{1.251534in}{2.187258in}}%
\pgfpathcurveto{\pgfqpoint{1.243298in}{2.187258in}}{\pgfqpoint{1.235397in}{2.183986in}}{\pgfqpoint{1.229574in}{2.178162in}}%
\pgfpathcurveto{\pgfqpoint{1.223750in}{2.172338in}}{\pgfqpoint{1.220477in}{2.164438in}}{\pgfqpoint{1.220477in}{2.156202in}}%
\pgfpathcurveto{\pgfqpoint{1.220477in}{2.147965in}}{\pgfqpoint{1.223750in}{2.140065in}}{\pgfqpoint{1.229574in}{2.134241in}}%
\pgfpathcurveto{\pgfqpoint{1.235397in}{2.128417in}}{\pgfqpoint{1.243298in}{2.125145in}}{\pgfqpoint{1.251534in}{2.125145in}}%
\pgfpathclose%
\pgfusepath{stroke,fill}%
\end{pgfscope}%
\begin{pgfscope}%
\pgfpathrectangle{\pgfqpoint{0.100000in}{0.212622in}}{\pgfqpoint{3.696000in}{3.696000in}}%
\pgfusepath{clip}%
\pgfsetbuttcap%
\pgfsetroundjoin%
\definecolor{currentfill}{rgb}{0.121569,0.466667,0.705882}%
\pgfsetfillcolor{currentfill}%
\pgfsetfillopacity{0.531928}%
\pgfsetlinewidth{1.003750pt}%
\definecolor{currentstroke}{rgb}{0.121569,0.466667,0.705882}%
\pgfsetstrokecolor{currentstroke}%
\pgfsetstrokeopacity{0.531928}%
\pgfsetdash{}{0pt}%
\pgfpathmoveto{\pgfqpoint{3.280678in}{1.761427in}}%
\pgfpathcurveto{\pgfqpoint{3.288914in}{1.761427in}}{\pgfqpoint{3.296814in}{1.764699in}}{\pgfqpoint{3.302638in}{1.770523in}}%
\pgfpathcurveto{\pgfqpoint{3.308462in}{1.776347in}}{\pgfqpoint{3.311735in}{1.784247in}}{\pgfqpoint{3.311735in}{1.792484in}}%
\pgfpathcurveto{\pgfqpoint{3.311735in}{1.800720in}}{\pgfqpoint{3.308462in}{1.808620in}}{\pgfqpoint{3.302638in}{1.814444in}}%
\pgfpathcurveto{\pgfqpoint{3.296814in}{1.820268in}}{\pgfqpoint{3.288914in}{1.823540in}}{\pgfqpoint{3.280678in}{1.823540in}}%
\pgfpathcurveto{\pgfqpoint{3.272442in}{1.823540in}}{\pgfqpoint{3.264542in}{1.820268in}}{\pgfqpoint{3.258718in}{1.814444in}}%
\pgfpathcurveto{\pgfqpoint{3.252894in}{1.808620in}}{\pgfqpoint{3.249622in}{1.800720in}}{\pgfqpoint{3.249622in}{1.792484in}}%
\pgfpathcurveto{\pgfqpoint{3.249622in}{1.784247in}}{\pgfqpoint{3.252894in}{1.776347in}}{\pgfqpoint{3.258718in}{1.770523in}}%
\pgfpathcurveto{\pgfqpoint{3.264542in}{1.764699in}}{\pgfqpoint{3.272442in}{1.761427in}}{\pgfqpoint{3.280678in}{1.761427in}}%
\pgfpathclose%
\pgfusepath{stroke,fill}%
\end{pgfscope}%
\begin{pgfscope}%
\pgfpathrectangle{\pgfqpoint{0.100000in}{0.212622in}}{\pgfqpoint{3.696000in}{3.696000in}}%
\pgfusepath{clip}%
\pgfsetbuttcap%
\pgfsetroundjoin%
\definecolor{currentfill}{rgb}{0.121569,0.466667,0.705882}%
\pgfsetfillcolor{currentfill}%
\pgfsetfillopacity{0.532293}%
\pgfsetlinewidth{1.003750pt}%
\definecolor{currentstroke}{rgb}{0.121569,0.466667,0.705882}%
\pgfsetstrokecolor{currentstroke}%
\pgfsetstrokeopacity{0.532293}%
\pgfsetdash{}{0pt}%
\pgfpathmoveto{\pgfqpoint{1.249626in}{2.125209in}}%
\pgfpathcurveto{\pgfqpoint{1.257862in}{2.125209in}}{\pgfqpoint{1.265762in}{2.128481in}}{\pgfqpoint{1.271586in}{2.134305in}}%
\pgfpathcurveto{\pgfqpoint{1.277410in}{2.140129in}}{\pgfqpoint{1.280682in}{2.148029in}}{\pgfqpoint{1.280682in}{2.156265in}}%
\pgfpathcurveto{\pgfqpoint{1.280682in}{2.164502in}}{\pgfqpoint{1.277410in}{2.172402in}}{\pgfqpoint{1.271586in}{2.178226in}}%
\pgfpathcurveto{\pgfqpoint{1.265762in}{2.184049in}}{\pgfqpoint{1.257862in}{2.187322in}}{\pgfqpoint{1.249626in}{2.187322in}}%
\pgfpathcurveto{\pgfqpoint{1.241390in}{2.187322in}}{\pgfqpoint{1.233490in}{2.184049in}}{\pgfqpoint{1.227666in}{2.178226in}}%
\pgfpathcurveto{\pgfqpoint{1.221842in}{2.172402in}}{\pgfqpoint{1.218569in}{2.164502in}}{\pgfqpoint{1.218569in}{2.156265in}}%
\pgfpathcurveto{\pgfqpoint{1.218569in}{2.148029in}}{\pgfqpoint{1.221842in}{2.140129in}}{\pgfqpoint{1.227666in}{2.134305in}}%
\pgfpathcurveto{\pgfqpoint{1.233490in}{2.128481in}}{\pgfqpoint{1.241390in}{2.125209in}}{\pgfqpoint{1.249626in}{2.125209in}}%
\pgfpathclose%
\pgfusepath{stroke,fill}%
\end{pgfscope}%
\begin{pgfscope}%
\pgfpathrectangle{\pgfqpoint{0.100000in}{0.212622in}}{\pgfqpoint{3.696000in}{3.696000in}}%
\pgfusepath{clip}%
\pgfsetbuttcap%
\pgfsetroundjoin%
\definecolor{currentfill}{rgb}{0.121569,0.466667,0.705882}%
\pgfsetfillcolor{currentfill}%
\pgfsetfillopacity{0.532351}%
\pgfsetlinewidth{1.003750pt}%
\definecolor{currentstroke}{rgb}{0.121569,0.466667,0.705882}%
\pgfsetstrokecolor{currentstroke}%
\pgfsetstrokeopacity{0.532351}%
\pgfsetdash{}{0pt}%
\pgfpathmoveto{\pgfqpoint{3.279677in}{1.761630in}}%
\pgfpathcurveto{\pgfqpoint{3.287913in}{1.761630in}}{\pgfqpoint{3.295813in}{1.764903in}}{\pgfqpoint{3.301637in}{1.770727in}}%
\pgfpathcurveto{\pgfqpoint{3.307461in}{1.776551in}}{\pgfqpoint{3.310733in}{1.784451in}}{\pgfqpoint{3.310733in}{1.792687in}}%
\pgfpathcurveto{\pgfqpoint{3.310733in}{1.800923in}}{\pgfqpoint{3.307461in}{1.808823in}}{\pgfqpoint{3.301637in}{1.814647in}}%
\pgfpathcurveto{\pgfqpoint{3.295813in}{1.820471in}}{\pgfqpoint{3.287913in}{1.823743in}}{\pgfqpoint{3.279677in}{1.823743in}}%
\pgfpathcurveto{\pgfqpoint{3.271440in}{1.823743in}}{\pgfqpoint{3.263540in}{1.820471in}}{\pgfqpoint{3.257716in}{1.814647in}}%
\pgfpathcurveto{\pgfqpoint{3.251893in}{1.808823in}}{\pgfqpoint{3.248620in}{1.800923in}}{\pgfqpoint{3.248620in}{1.792687in}}%
\pgfpathcurveto{\pgfqpoint{3.248620in}{1.784451in}}{\pgfqpoint{3.251893in}{1.776551in}}{\pgfqpoint{3.257716in}{1.770727in}}%
\pgfpathcurveto{\pgfqpoint{3.263540in}{1.764903in}}{\pgfqpoint{3.271440in}{1.761630in}}{\pgfqpoint{3.279677in}{1.761630in}}%
\pgfpathclose%
\pgfusepath{stroke,fill}%
\end{pgfscope}%
\begin{pgfscope}%
\pgfpathrectangle{\pgfqpoint{0.100000in}{0.212622in}}{\pgfqpoint{3.696000in}{3.696000in}}%
\pgfusepath{clip}%
\pgfsetbuttcap%
\pgfsetroundjoin%
\definecolor{currentfill}{rgb}{0.121569,0.466667,0.705882}%
\pgfsetfillcolor{currentfill}%
\pgfsetfillopacity{0.532954}%
\pgfsetlinewidth{1.003750pt}%
\definecolor{currentstroke}{rgb}{0.121569,0.466667,0.705882}%
\pgfsetstrokecolor{currentstroke}%
\pgfsetstrokeopacity{0.532954}%
\pgfsetdash{}{0pt}%
\pgfpathmoveto{\pgfqpoint{3.278942in}{1.761727in}}%
\pgfpathcurveto{\pgfqpoint{3.287178in}{1.761727in}}{\pgfqpoint{3.295078in}{1.764999in}}{\pgfqpoint{3.300902in}{1.770823in}}%
\pgfpathcurveto{\pgfqpoint{3.306726in}{1.776647in}}{\pgfqpoint{3.309999in}{1.784547in}}{\pgfqpoint{3.309999in}{1.792783in}}%
\pgfpathcurveto{\pgfqpoint{3.309999in}{1.801020in}}{\pgfqpoint{3.306726in}{1.808920in}}{\pgfqpoint{3.300902in}{1.814744in}}%
\pgfpathcurveto{\pgfqpoint{3.295078in}{1.820568in}}{\pgfqpoint{3.287178in}{1.823840in}}{\pgfqpoint{3.278942in}{1.823840in}}%
\pgfpathcurveto{\pgfqpoint{3.270706in}{1.823840in}}{\pgfqpoint{3.262806in}{1.820568in}}{\pgfqpoint{3.256982in}{1.814744in}}%
\pgfpathcurveto{\pgfqpoint{3.251158in}{1.808920in}}{\pgfqpoint{3.247886in}{1.801020in}}{\pgfqpoint{3.247886in}{1.792783in}}%
\pgfpathcurveto{\pgfqpoint{3.247886in}{1.784547in}}{\pgfqpoint{3.251158in}{1.776647in}}{\pgfqpoint{3.256982in}{1.770823in}}%
\pgfpathcurveto{\pgfqpoint{3.262806in}{1.764999in}}{\pgfqpoint{3.270706in}{1.761727in}}{\pgfqpoint{3.278942in}{1.761727in}}%
\pgfpathclose%
\pgfusepath{stroke,fill}%
\end{pgfscope}%
\begin{pgfscope}%
\pgfpathrectangle{\pgfqpoint{0.100000in}{0.212622in}}{\pgfqpoint{3.696000in}{3.696000in}}%
\pgfusepath{clip}%
\pgfsetbuttcap%
\pgfsetroundjoin%
\definecolor{currentfill}{rgb}{0.121569,0.466667,0.705882}%
\pgfsetfillcolor{currentfill}%
\pgfsetfillopacity{0.533197}%
\pgfsetlinewidth{1.003750pt}%
\definecolor{currentstroke}{rgb}{0.121569,0.466667,0.705882}%
\pgfsetstrokecolor{currentstroke}%
\pgfsetstrokeopacity{0.533197}%
\pgfsetdash{}{0pt}%
\pgfpathmoveto{\pgfqpoint{1.247845in}{2.125190in}}%
\pgfpathcurveto{\pgfqpoint{1.256081in}{2.125190in}}{\pgfqpoint{1.263981in}{2.128462in}}{\pgfqpoint{1.269805in}{2.134286in}}%
\pgfpathcurveto{\pgfqpoint{1.275629in}{2.140110in}}{\pgfqpoint{1.278901in}{2.148010in}}{\pgfqpoint{1.278901in}{2.156246in}}%
\pgfpathcurveto{\pgfqpoint{1.278901in}{2.164483in}}{\pgfqpoint{1.275629in}{2.172383in}}{\pgfqpoint{1.269805in}{2.178207in}}%
\pgfpathcurveto{\pgfqpoint{1.263981in}{2.184031in}}{\pgfqpoint{1.256081in}{2.187303in}}{\pgfqpoint{1.247845in}{2.187303in}}%
\pgfpathcurveto{\pgfqpoint{1.239609in}{2.187303in}}{\pgfqpoint{1.231709in}{2.184031in}}{\pgfqpoint{1.225885in}{2.178207in}}%
\pgfpathcurveto{\pgfqpoint{1.220061in}{2.172383in}}{\pgfqpoint{1.216788in}{2.164483in}}{\pgfqpoint{1.216788in}{2.156246in}}%
\pgfpathcurveto{\pgfqpoint{1.216788in}{2.148010in}}{\pgfqpoint{1.220061in}{2.140110in}}{\pgfqpoint{1.225885in}{2.134286in}}%
\pgfpathcurveto{\pgfqpoint{1.231709in}{2.128462in}}{\pgfqpoint{1.239609in}{2.125190in}}{\pgfqpoint{1.247845in}{2.125190in}}%
\pgfpathclose%
\pgfusepath{stroke,fill}%
\end{pgfscope}%
\begin{pgfscope}%
\pgfpathrectangle{\pgfqpoint{0.100000in}{0.212622in}}{\pgfqpoint{3.696000in}{3.696000in}}%
\pgfusepath{clip}%
\pgfsetbuttcap%
\pgfsetroundjoin%
\definecolor{currentfill}{rgb}{0.121569,0.466667,0.705882}%
\pgfsetfillcolor{currentfill}%
\pgfsetfillopacity{0.533586}%
\pgfsetlinewidth{1.003750pt}%
\definecolor{currentstroke}{rgb}{0.121569,0.466667,0.705882}%
\pgfsetstrokecolor{currentstroke}%
\pgfsetstrokeopacity{0.533586}%
\pgfsetdash{}{0pt}%
\pgfpathmoveto{\pgfqpoint{1.246902in}{2.125248in}}%
\pgfpathcurveto{\pgfqpoint{1.255138in}{2.125248in}}{\pgfqpoint{1.263038in}{2.128520in}}{\pgfqpoint{1.268862in}{2.134344in}}%
\pgfpathcurveto{\pgfqpoint{1.274686in}{2.140168in}}{\pgfqpoint{1.277958in}{2.148068in}}{\pgfqpoint{1.277958in}{2.156305in}}%
\pgfpathcurveto{\pgfqpoint{1.277958in}{2.164541in}}{\pgfqpoint{1.274686in}{2.172441in}}{\pgfqpoint{1.268862in}{2.178265in}}%
\pgfpathcurveto{\pgfqpoint{1.263038in}{2.184089in}}{\pgfqpoint{1.255138in}{2.187361in}}{\pgfqpoint{1.246902in}{2.187361in}}%
\pgfpathcurveto{\pgfqpoint{1.238665in}{2.187361in}}{\pgfqpoint{1.230765in}{2.184089in}}{\pgfqpoint{1.224941in}{2.178265in}}%
\pgfpathcurveto{\pgfqpoint{1.219118in}{2.172441in}}{\pgfqpoint{1.215845in}{2.164541in}}{\pgfqpoint{1.215845in}{2.156305in}}%
\pgfpathcurveto{\pgfqpoint{1.215845in}{2.148068in}}{\pgfqpoint{1.219118in}{2.140168in}}{\pgfqpoint{1.224941in}{2.134344in}}%
\pgfpathcurveto{\pgfqpoint{1.230765in}{2.128520in}}{\pgfqpoint{1.238665in}{2.125248in}}{\pgfqpoint{1.246902in}{2.125248in}}%
\pgfpathclose%
\pgfusepath{stroke,fill}%
\end{pgfscope}%
\begin{pgfscope}%
\pgfpathrectangle{\pgfqpoint{0.100000in}{0.212622in}}{\pgfqpoint{3.696000in}{3.696000in}}%
\pgfusepath{clip}%
\pgfsetbuttcap%
\pgfsetroundjoin%
\definecolor{currentfill}{rgb}{0.121569,0.466667,0.705882}%
\pgfsetfillcolor{currentfill}%
\pgfsetfillopacity{0.533767}%
\pgfsetlinewidth{1.003750pt}%
\definecolor{currentstroke}{rgb}{0.121569,0.466667,0.705882}%
\pgfsetstrokecolor{currentstroke}%
\pgfsetstrokeopacity{0.533767}%
\pgfsetdash{}{0pt}%
\pgfpathmoveto{\pgfqpoint{3.277237in}{1.762035in}}%
\pgfpathcurveto{\pgfqpoint{3.285473in}{1.762035in}}{\pgfqpoint{3.293373in}{1.765307in}}{\pgfqpoint{3.299197in}{1.771131in}}%
\pgfpathcurveto{\pgfqpoint{3.305021in}{1.776955in}}{\pgfqpoint{3.308293in}{1.784855in}}{\pgfqpoint{3.308293in}{1.793091in}}%
\pgfpathcurveto{\pgfqpoint{3.308293in}{1.801328in}}{\pgfqpoint{3.305021in}{1.809228in}}{\pgfqpoint{3.299197in}{1.815052in}}%
\pgfpathcurveto{\pgfqpoint{3.293373in}{1.820875in}}{\pgfqpoint{3.285473in}{1.824148in}}{\pgfqpoint{3.277237in}{1.824148in}}%
\pgfpathcurveto{\pgfqpoint{3.269000in}{1.824148in}}{\pgfqpoint{3.261100in}{1.820875in}}{\pgfqpoint{3.255276in}{1.815052in}}%
\pgfpathcurveto{\pgfqpoint{3.249453in}{1.809228in}}{\pgfqpoint{3.246180in}{1.801328in}}{\pgfqpoint{3.246180in}{1.793091in}}%
\pgfpathcurveto{\pgfqpoint{3.246180in}{1.784855in}}{\pgfqpoint{3.249453in}{1.776955in}}{\pgfqpoint{3.255276in}{1.771131in}}%
\pgfpathcurveto{\pgfqpoint{3.261100in}{1.765307in}}{\pgfqpoint{3.269000in}{1.762035in}}{\pgfqpoint{3.277237in}{1.762035in}}%
\pgfpathclose%
\pgfusepath{stroke,fill}%
\end{pgfscope}%
\begin{pgfscope}%
\pgfpathrectangle{\pgfqpoint{0.100000in}{0.212622in}}{\pgfqpoint{3.696000in}{3.696000in}}%
\pgfusepath{clip}%
\pgfsetbuttcap%
\pgfsetroundjoin%
\definecolor{currentfill}{rgb}{0.121569,0.466667,0.705882}%
\pgfsetfillcolor{currentfill}%
\pgfsetfillopacity{0.533911}%
\pgfsetlinewidth{1.003750pt}%
\definecolor{currentstroke}{rgb}{0.121569,0.466667,0.705882}%
\pgfsetstrokecolor{currentstroke}%
\pgfsetstrokeopacity{0.533911}%
\pgfsetdash{}{0pt}%
\pgfpathmoveto{\pgfqpoint{1.246503in}{2.125253in}}%
\pgfpathcurveto{\pgfqpoint{1.254740in}{2.125253in}}{\pgfqpoint{1.262640in}{2.128526in}}{\pgfqpoint{1.268464in}{2.134349in}}%
\pgfpathcurveto{\pgfqpoint{1.274288in}{2.140173in}}{\pgfqpoint{1.277560in}{2.148073in}}{\pgfqpoint{1.277560in}{2.156310in}}%
\pgfpathcurveto{\pgfqpoint{1.277560in}{2.164546in}}{\pgfqpoint{1.274288in}{2.172446in}}{\pgfqpoint{1.268464in}{2.178270in}}%
\pgfpathcurveto{\pgfqpoint{1.262640in}{2.184094in}}{\pgfqpoint{1.254740in}{2.187366in}}{\pgfqpoint{1.246503in}{2.187366in}}%
\pgfpathcurveto{\pgfqpoint{1.238267in}{2.187366in}}{\pgfqpoint{1.230367in}{2.184094in}}{\pgfqpoint{1.224543in}{2.178270in}}%
\pgfpathcurveto{\pgfqpoint{1.218719in}{2.172446in}}{\pgfqpoint{1.215447in}{2.164546in}}{\pgfqpoint{1.215447in}{2.156310in}}%
\pgfpathcurveto{\pgfqpoint{1.215447in}{2.148073in}}{\pgfqpoint{1.218719in}{2.140173in}}{\pgfqpoint{1.224543in}{2.134349in}}%
\pgfpathcurveto{\pgfqpoint{1.230367in}{2.128526in}}{\pgfqpoint{1.238267in}{2.125253in}}{\pgfqpoint{1.246503in}{2.125253in}}%
\pgfpathclose%
\pgfusepath{stroke,fill}%
\end{pgfscope}%
\begin{pgfscope}%
\pgfpathrectangle{\pgfqpoint{0.100000in}{0.212622in}}{\pgfqpoint{3.696000in}{3.696000in}}%
\pgfusepath{clip}%
\pgfsetbuttcap%
\pgfsetroundjoin%
\definecolor{currentfill}{rgb}{0.121569,0.466667,0.705882}%
\pgfsetfillcolor{currentfill}%
\pgfsetfillopacity{0.534432}%
\pgfsetlinewidth{1.003750pt}%
\definecolor{currentstroke}{rgb}{0.121569,0.466667,0.705882}%
\pgfsetstrokecolor{currentstroke}%
\pgfsetstrokeopacity{0.534432}%
\pgfsetdash{}{0pt}%
\pgfpathmoveto{\pgfqpoint{1.245080in}{2.125392in}}%
\pgfpathcurveto{\pgfqpoint{1.253317in}{2.125392in}}{\pgfqpoint{1.261217in}{2.128664in}}{\pgfqpoint{1.267041in}{2.134488in}}%
\pgfpathcurveto{\pgfqpoint{1.272865in}{2.140312in}}{\pgfqpoint{1.276137in}{2.148212in}}{\pgfqpoint{1.276137in}{2.156448in}}%
\pgfpathcurveto{\pgfqpoint{1.276137in}{2.164685in}}{\pgfqpoint{1.272865in}{2.172585in}}{\pgfqpoint{1.267041in}{2.178409in}}%
\pgfpathcurveto{\pgfqpoint{1.261217in}{2.184232in}}{\pgfqpoint{1.253317in}{2.187505in}}{\pgfqpoint{1.245080in}{2.187505in}}%
\pgfpathcurveto{\pgfqpoint{1.236844in}{2.187505in}}{\pgfqpoint{1.228944in}{2.184232in}}{\pgfqpoint{1.223120in}{2.178409in}}%
\pgfpathcurveto{\pgfqpoint{1.217296in}{2.172585in}}{\pgfqpoint{1.214024in}{2.164685in}}{\pgfqpoint{1.214024in}{2.156448in}}%
\pgfpathcurveto{\pgfqpoint{1.214024in}{2.148212in}}{\pgfqpoint{1.217296in}{2.140312in}}{\pgfqpoint{1.223120in}{2.134488in}}%
\pgfpathcurveto{\pgfqpoint{1.228944in}{2.128664in}}{\pgfqpoint{1.236844in}{2.125392in}}{\pgfqpoint{1.245080in}{2.125392in}}%
\pgfpathclose%
\pgfusepath{stroke,fill}%
\end{pgfscope}%
\begin{pgfscope}%
\pgfpathrectangle{\pgfqpoint{0.100000in}{0.212622in}}{\pgfqpoint{3.696000in}{3.696000in}}%
\pgfusepath{clip}%
\pgfsetbuttcap%
\pgfsetroundjoin%
\definecolor{currentfill}{rgb}{0.121569,0.466667,0.705882}%
\pgfsetfillcolor{currentfill}%
\pgfsetfillopacity{0.534625}%
\pgfsetlinewidth{1.003750pt}%
\definecolor{currentstroke}{rgb}{0.121569,0.466667,0.705882}%
\pgfsetstrokecolor{currentstroke}%
\pgfsetstrokeopacity{0.534625}%
\pgfsetdash{}{0pt}%
\pgfpathmoveto{\pgfqpoint{3.275123in}{1.762504in}}%
\pgfpathcurveto{\pgfqpoint{3.283359in}{1.762504in}}{\pgfqpoint{3.291259in}{1.765776in}}{\pgfqpoint{3.297083in}{1.771600in}}%
\pgfpathcurveto{\pgfqpoint{3.302907in}{1.777424in}}{\pgfqpoint{3.306179in}{1.785324in}}{\pgfqpoint{3.306179in}{1.793560in}}%
\pgfpathcurveto{\pgfqpoint{3.306179in}{1.801796in}}{\pgfqpoint{3.302907in}{1.809697in}}{\pgfqpoint{3.297083in}{1.815520in}}%
\pgfpathcurveto{\pgfqpoint{3.291259in}{1.821344in}}{\pgfqpoint{3.283359in}{1.824617in}}{\pgfqpoint{3.275123in}{1.824617in}}%
\pgfpathcurveto{\pgfqpoint{3.266887in}{1.824617in}}{\pgfqpoint{3.258987in}{1.821344in}}{\pgfqpoint{3.253163in}{1.815520in}}%
\pgfpathcurveto{\pgfqpoint{3.247339in}{1.809697in}}{\pgfqpoint{3.244066in}{1.801796in}}{\pgfqpoint{3.244066in}{1.793560in}}%
\pgfpathcurveto{\pgfqpoint{3.244066in}{1.785324in}}{\pgfqpoint{3.247339in}{1.777424in}}{\pgfqpoint{3.253163in}{1.771600in}}%
\pgfpathcurveto{\pgfqpoint{3.258987in}{1.765776in}}{\pgfqpoint{3.266887in}{1.762504in}}{\pgfqpoint{3.275123in}{1.762504in}}%
\pgfpathclose%
\pgfusepath{stroke,fill}%
\end{pgfscope}%
\begin{pgfscope}%
\pgfpathrectangle{\pgfqpoint{0.100000in}{0.212622in}}{\pgfqpoint{3.696000in}{3.696000in}}%
\pgfusepath{clip}%
\pgfsetbuttcap%
\pgfsetroundjoin%
\definecolor{currentfill}{rgb}{0.121569,0.466667,0.705882}%
\pgfsetfillcolor{currentfill}%
\pgfsetfillopacity{0.535470}%
\pgfsetlinewidth{1.003750pt}%
\definecolor{currentstroke}{rgb}{0.121569,0.466667,0.705882}%
\pgfsetstrokecolor{currentstroke}%
\pgfsetstrokeopacity{0.535470}%
\pgfsetdash{}{0pt}%
\pgfpathmoveto{\pgfqpoint{1.243206in}{2.125506in}}%
\pgfpathcurveto{\pgfqpoint{1.251442in}{2.125506in}}{\pgfqpoint{1.259342in}{2.128778in}}{\pgfqpoint{1.265166in}{2.134602in}}%
\pgfpathcurveto{\pgfqpoint{1.270990in}{2.140426in}}{\pgfqpoint{1.274262in}{2.148326in}}{\pgfqpoint{1.274262in}{2.156562in}}%
\pgfpathcurveto{\pgfqpoint{1.274262in}{2.164799in}}{\pgfqpoint{1.270990in}{2.172699in}}{\pgfqpoint{1.265166in}{2.178523in}}%
\pgfpathcurveto{\pgfqpoint{1.259342in}{2.184347in}}{\pgfqpoint{1.251442in}{2.187619in}}{\pgfqpoint{1.243206in}{2.187619in}}%
\pgfpathcurveto{\pgfqpoint{1.234969in}{2.187619in}}{\pgfqpoint{1.227069in}{2.184347in}}{\pgfqpoint{1.221245in}{2.178523in}}%
\pgfpathcurveto{\pgfqpoint{1.215421in}{2.172699in}}{\pgfqpoint{1.212149in}{2.164799in}}{\pgfqpoint{1.212149in}{2.156562in}}%
\pgfpathcurveto{\pgfqpoint{1.212149in}{2.148326in}}{\pgfqpoint{1.215421in}{2.140426in}}{\pgfqpoint{1.221245in}{2.134602in}}%
\pgfpathcurveto{\pgfqpoint{1.227069in}{2.128778in}}{\pgfqpoint{1.234969in}{2.125506in}}{\pgfqpoint{1.243206in}{2.125506in}}%
\pgfpathclose%
\pgfusepath{stroke,fill}%
\end{pgfscope}%
\begin{pgfscope}%
\pgfpathrectangle{\pgfqpoint{0.100000in}{0.212622in}}{\pgfqpoint{3.696000in}{3.696000in}}%
\pgfusepath{clip}%
\pgfsetbuttcap%
\pgfsetroundjoin%
\definecolor{currentfill}{rgb}{0.121569,0.466667,0.705882}%
\pgfsetfillcolor{currentfill}%
\pgfsetfillopacity{0.536152}%
\pgfsetlinewidth{1.003750pt}%
\definecolor{currentstroke}{rgb}{0.121569,0.466667,0.705882}%
\pgfsetstrokecolor{currentstroke}%
\pgfsetstrokeopacity{0.536152}%
\pgfsetdash{}{0pt}%
\pgfpathmoveto{\pgfqpoint{3.273050in}{1.762834in}}%
\pgfpathcurveto{\pgfqpoint{3.281286in}{1.762834in}}{\pgfqpoint{3.289186in}{1.766106in}}{\pgfqpoint{3.295010in}{1.771930in}}%
\pgfpathcurveto{\pgfqpoint{3.300834in}{1.777754in}}{\pgfqpoint{3.304106in}{1.785654in}}{\pgfqpoint{3.304106in}{1.793891in}}%
\pgfpathcurveto{\pgfqpoint{3.304106in}{1.802127in}}{\pgfqpoint{3.300834in}{1.810027in}}{\pgfqpoint{3.295010in}{1.815851in}}%
\pgfpathcurveto{\pgfqpoint{3.289186in}{1.821675in}}{\pgfqpoint{3.281286in}{1.824947in}}{\pgfqpoint{3.273050in}{1.824947in}}%
\pgfpathcurveto{\pgfqpoint{3.264814in}{1.824947in}}{\pgfqpoint{3.256914in}{1.821675in}}{\pgfqpoint{3.251090in}{1.815851in}}%
\pgfpathcurveto{\pgfqpoint{3.245266in}{1.810027in}}{\pgfqpoint{3.241993in}{1.802127in}}{\pgfqpoint{3.241993in}{1.793891in}}%
\pgfpathcurveto{\pgfqpoint{3.241993in}{1.785654in}}{\pgfqpoint{3.245266in}{1.777754in}}{\pgfqpoint{3.251090in}{1.771930in}}%
\pgfpathcurveto{\pgfqpoint{3.256914in}{1.766106in}}{\pgfqpoint{3.264814in}{1.762834in}}{\pgfqpoint{3.273050in}{1.762834in}}%
\pgfpathclose%
\pgfusepath{stroke,fill}%
\end{pgfscope}%
\begin{pgfscope}%
\pgfpathrectangle{\pgfqpoint{0.100000in}{0.212622in}}{\pgfqpoint{3.696000in}{3.696000in}}%
\pgfusepath{clip}%
\pgfsetbuttcap%
\pgfsetroundjoin%
\definecolor{currentfill}{rgb}{0.121569,0.466667,0.705882}%
\pgfsetfillcolor{currentfill}%
\pgfsetfillopacity{0.537235}%
\pgfsetlinewidth{1.003750pt}%
\definecolor{currentstroke}{rgb}{0.121569,0.466667,0.705882}%
\pgfsetstrokecolor{currentstroke}%
\pgfsetstrokeopacity{0.537235}%
\pgfsetdash{}{0pt}%
\pgfpathmoveto{\pgfqpoint{1.238972in}{2.125654in}}%
\pgfpathcurveto{\pgfqpoint{1.247208in}{2.125654in}}{\pgfqpoint{1.255108in}{2.128927in}}{\pgfqpoint{1.260932in}{2.134751in}}%
\pgfpathcurveto{\pgfqpoint{1.266756in}{2.140574in}}{\pgfqpoint{1.270028in}{2.148474in}}{\pgfqpoint{1.270028in}{2.156711in}}%
\pgfpathcurveto{\pgfqpoint{1.270028in}{2.164947in}}{\pgfqpoint{1.266756in}{2.172847in}}{\pgfqpoint{1.260932in}{2.178671in}}%
\pgfpathcurveto{\pgfqpoint{1.255108in}{2.184495in}}{\pgfqpoint{1.247208in}{2.187767in}}{\pgfqpoint{1.238972in}{2.187767in}}%
\pgfpathcurveto{\pgfqpoint{1.230735in}{2.187767in}}{\pgfqpoint{1.222835in}{2.184495in}}{\pgfqpoint{1.217011in}{2.178671in}}%
\pgfpathcurveto{\pgfqpoint{1.211187in}{2.172847in}}{\pgfqpoint{1.207915in}{2.164947in}}{\pgfqpoint{1.207915in}{2.156711in}}%
\pgfpathcurveto{\pgfqpoint{1.207915in}{2.148474in}}{\pgfqpoint{1.211187in}{2.140574in}}{\pgfqpoint{1.217011in}{2.134751in}}%
\pgfpathcurveto{\pgfqpoint{1.222835in}{2.128927in}}{\pgfqpoint{1.230735in}{2.125654in}}{\pgfqpoint{1.238972in}{2.125654in}}%
\pgfpathclose%
\pgfusepath{stroke,fill}%
\end{pgfscope}%
\begin{pgfscope}%
\pgfpathrectangle{\pgfqpoint{0.100000in}{0.212622in}}{\pgfqpoint{3.696000in}{3.696000in}}%
\pgfusepath{clip}%
\pgfsetbuttcap%
\pgfsetroundjoin%
\definecolor{currentfill}{rgb}{0.121569,0.466667,0.705882}%
\pgfsetfillcolor{currentfill}%
\pgfsetfillopacity{0.537750}%
\pgfsetlinewidth{1.003750pt}%
\definecolor{currentstroke}{rgb}{0.121569,0.466667,0.705882}%
\pgfsetstrokecolor{currentstroke}%
\pgfsetstrokeopacity{0.537750}%
\pgfsetdash{}{0pt}%
\pgfpathmoveto{\pgfqpoint{3.270822in}{1.763062in}}%
\pgfpathcurveto{\pgfqpoint{3.279058in}{1.763062in}}{\pgfqpoint{3.286958in}{1.766334in}}{\pgfqpoint{3.292782in}{1.772158in}}%
\pgfpathcurveto{\pgfqpoint{3.298606in}{1.777982in}}{\pgfqpoint{3.301879in}{1.785882in}}{\pgfqpoint{3.301879in}{1.794119in}}%
\pgfpathcurveto{\pgfqpoint{3.301879in}{1.802355in}}{\pgfqpoint{3.298606in}{1.810255in}}{\pgfqpoint{3.292782in}{1.816079in}}%
\pgfpathcurveto{\pgfqpoint{3.286958in}{1.821903in}}{\pgfqpoint{3.279058in}{1.825175in}}{\pgfqpoint{3.270822in}{1.825175in}}%
\pgfpathcurveto{\pgfqpoint{3.262586in}{1.825175in}}{\pgfqpoint{3.254686in}{1.821903in}}{\pgfqpoint{3.248862in}{1.816079in}}%
\pgfpathcurveto{\pgfqpoint{3.243038in}{1.810255in}}{\pgfqpoint{3.239766in}{1.802355in}}{\pgfqpoint{3.239766in}{1.794119in}}%
\pgfpathcurveto{\pgfqpoint{3.239766in}{1.785882in}}{\pgfqpoint{3.243038in}{1.777982in}}{\pgfqpoint{3.248862in}{1.772158in}}%
\pgfpathcurveto{\pgfqpoint{3.254686in}{1.766334in}}{\pgfqpoint{3.262586in}{1.763062in}}{\pgfqpoint{3.270822in}{1.763062in}}%
\pgfpathclose%
\pgfusepath{stroke,fill}%
\end{pgfscope}%
\begin{pgfscope}%
\pgfpathrectangle{\pgfqpoint{0.100000in}{0.212622in}}{\pgfqpoint{3.696000in}{3.696000in}}%
\pgfusepath{clip}%
\pgfsetbuttcap%
\pgfsetroundjoin%
\definecolor{currentfill}{rgb}{0.121569,0.466667,0.705882}%
\pgfsetfillcolor{currentfill}%
\pgfsetfillopacity{0.538550}%
\pgfsetlinewidth{1.003750pt}%
\definecolor{currentstroke}{rgb}{0.121569,0.466667,0.705882}%
\pgfsetstrokecolor{currentstroke}%
\pgfsetstrokeopacity{0.538550}%
\pgfsetdash{}{0pt}%
\pgfpathmoveto{\pgfqpoint{3.268814in}{1.763539in}}%
\pgfpathcurveto{\pgfqpoint{3.277050in}{1.763539in}}{\pgfqpoint{3.284950in}{1.766812in}}{\pgfqpoint{3.290774in}{1.772636in}}%
\pgfpathcurveto{\pgfqpoint{3.296598in}{1.778460in}}{\pgfqpoint{3.299870in}{1.786360in}}{\pgfqpoint{3.299870in}{1.794596in}}%
\pgfpathcurveto{\pgfqpoint{3.299870in}{1.802832in}}{\pgfqpoint{3.296598in}{1.810732in}}{\pgfqpoint{3.290774in}{1.816556in}}%
\pgfpathcurveto{\pgfqpoint{3.284950in}{1.822380in}}{\pgfqpoint{3.277050in}{1.825652in}}{\pgfqpoint{3.268814in}{1.825652in}}%
\pgfpathcurveto{\pgfqpoint{3.260577in}{1.825652in}}{\pgfqpoint{3.252677in}{1.822380in}}{\pgfqpoint{3.246853in}{1.816556in}}%
\pgfpathcurveto{\pgfqpoint{3.241029in}{1.810732in}}{\pgfqpoint{3.237757in}{1.802832in}}{\pgfqpoint{3.237757in}{1.794596in}}%
\pgfpathcurveto{\pgfqpoint{3.237757in}{1.786360in}}{\pgfqpoint{3.241029in}{1.778460in}}{\pgfqpoint{3.246853in}{1.772636in}}%
\pgfpathcurveto{\pgfqpoint{3.252677in}{1.766812in}}{\pgfqpoint{3.260577in}{1.763539in}}{\pgfqpoint{3.268814in}{1.763539in}}%
\pgfpathclose%
\pgfusepath{stroke,fill}%
\end{pgfscope}%
\begin{pgfscope}%
\pgfpathrectangle{\pgfqpoint{0.100000in}{0.212622in}}{\pgfqpoint{3.696000in}{3.696000in}}%
\pgfusepath{clip}%
\pgfsetbuttcap%
\pgfsetroundjoin%
\definecolor{currentfill}{rgb}{0.121569,0.466667,0.705882}%
\pgfsetfillcolor{currentfill}%
\pgfsetfillopacity{0.539645}%
\pgfsetlinewidth{1.003750pt}%
\definecolor{currentstroke}{rgb}{0.121569,0.466667,0.705882}%
\pgfsetstrokecolor{currentstroke}%
\pgfsetstrokeopacity{0.539645}%
\pgfsetdash{}{0pt}%
\pgfpathmoveto{\pgfqpoint{3.266403in}{1.764012in}}%
\pgfpathcurveto{\pgfqpoint{3.274640in}{1.764012in}}{\pgfqpoint{3.282540in}{1.767284in}}{\pgfqpoint{3.288364in}{1.773108in}}%
\pgfpathcurveto{\pgfqpoint{3.294187in}{1.778932in}}{\pgfqpoint{3.297460in}{1.786832in}}{\pgfqpoint{3.297460in}{1.795069in}}%
\pgfpathcurveto{\pgfqpoint{3.297460in}{1.803305in}}{\pgfqpoint{3.294187in}{1.811205in}}{\pgfqpoint{3.288364in}{1.817029in}}%
\pgfpathcurveto{\pgfqpoint{3.282540in}{1.822853in}}{\pgfqpoint{3.274640in}{1.826125in}}{\pgfqpoint{3.266403in}{1.826125in}}%
\pgfpathcurveto{\pgfqpoint{3.258167in}{1.826125in}}{\pgfqpoint{3.250267in}{1.822853in}}{\pgfqpoint{3.244443in}{1.817029in}}%
\pgfpathcurveto{\pgfqpoint{3.238619in}{1.811205in}}{\pgfqpoint{3.235347in}{1.803305in}}{\pgfqpoint{3.235347in}{1.795069in}}%
\pgfpathcurveto{\pgfqpoint{3.235347in}{1.786832in}}{\pgfqpoint{3.238619in}{1.778932in}}{\pgfqpoint{3.244443in}{1.773108in}}%
\pgfpathcurveto{\pgfqpoint{3.250267in}{1.767284in}}{\pgfqpoint{3.258167in}{1.764012in}}{\pgfqpoint{3.266403in}{1.764012in}}%
\pgfpathclose%
\pgfusepath{stroke,fill}%
\end{pgfscope}%
\begin{pgfscope}%
\pgfpathrectangle{\pgfqpoint{0.100000in}{0.212622in}}{\pgfqpoint{3.696000in}{3.696000in}}%
\pgfusepath{clip}%
\pgfsetbuttcap%
\pgfsetroundjoin%
\definecolor{currentfill}{rgb}{0.121569,0.466667,0.705882}%
\pgfsetfillcolor{currentfill}%
\pgfsetfillopacity{0.540451}%
\pgfsetlinewidth{1.003750pt}%
\definecolor{currentstroke}{rgb}{0.121569,0.466667,0.705882}%
\pgfsetstrokecolor{currentstroke}%
\pgfsetstrokeopacity{0.540451}%
\pgfsetdash{}{0pt}%
\pgfpathmoveto{\pgfqpoint{1.231388in}{2.125802in}}%
\pgfpathcurveto{\pgfqpoint{1.239624in}{2.125802in}}{\pgfqpoint{1.247524in}{2.129075in}}{\pgfqpoint{1.253348in}{2.134898in}}%
\pgfpathcurveto{\pgfqpoint{1.259172in}{2.140722in}}{\pgfqpoint{1.262444in}{2.148622in}}{\pgfqpoint{1.262444in}{2.156859in}}%
\pgfpathcurveto{\pgfqpoint{1.262444in}{2.165095in}}{\pgfqpoint{1.259172in}{2.172995in}}{\pgfqpoint{1.253348in}{2.178819in}}%
\pgfpathcurveto{\pgfqpoint{1.247524in}{2.184643in}}{\pgfqpoint{1.239624in}{2.187915in}}{\pgfqpoint{1.231388in}{2.187915in}}%
\pgfpathcurveto{\pgfqpoint{1.223151in}{2.187915in}}{\pgfqpoint{1.215251in}{2.184643in}}{\pgfqpoint{1.209427in}{2.178819in}}%
\pgfpathcurveto{\pgfqpoint{1.203603in}{2.172995in}}{\pgfqpoint{1.200331in}{2.165095in}}{\pgfqpoint{1.200331in}{2.156859in}}%
\pgfpathcurveto{\pgfqpoint{1.200331in}{2.148622in}}{\pgfqpoint{1.203603in}{2.140722in}}{\pgfqpoint{1.209427in}{2.134898in}}%
\pgfpathcurveto{\pgfqpoint{1.215251in}{2.129075in}}{\pgfqpoint{1.223151in}{2.125802in}}{\pgfqpoint{1.231388in}{2.125802in}}%
\pgfpathclose%
\pgfusepath{stroke,fill}%
\end{pgfscope}%
\begin{pgfscope}%
\pgfpathrectangle{\pgfqpoint{0.100000in}{0.212622in}}{\pgfqpoint{3.696000in}{3.696000in}}%
\pgfusepath{clip}%
\pgfsetbuttcap%
\pgfsetroundjoin%
\definecolor{currentfill}{rgb}{0.121569,0.466667,0.705882}%
\pgfsetfillcolor{currentfill}%
\pgfsetfillopacity{0.541158}%
\pgfsetlinewidth{1.003750pt}%
\definecolor{currentstroke}{rgb}{0.121569,0.466667,0.705882}%
\pgfsetstrokecolor{currentstroke}%
\pgfsetstrokeopacity{0.541158}%
\pgfsetdash{}{0pt}%
\pgfpathmoveto{\pgfqpoint{3.264960in}{1.764138in}}%
\pgfpathcurveto{\pgfqpoint{3.273197in}{1.764138in}}{\pgfqpoint{3.281097in}{1.767411in}}{\pgfqpoint{3.286921in}{1.773235in}}%
\pgfpathcurveto{\pgfqpoint{3.292744in}{1.779059in}}{\pgfqpoint{3.296017in}{1.786959in}}{\pgfqpoint{3.296017in}{1.795195in}}%
\pgfpathcurveto{\pgfqpoint{3.296017in}{1.803431in}}{\pgfqpoint{3.292744in}{1.811331in}}{\pgfqpoint{3.286921in}{1.817155in}}%
\pgfpathcurveto{\pgfqpoint{3.281097in}{1.822979in}}{\pgfqpoint{3.273197in}{1.826251in}}{\pgfqpoint{3.264960in}{1.826251in}}%
\pgfpathcurveto{\pgfqpoint{3.256724in}{1.826251in}}{\pgfqpoint{3.248824in}{1.822979in}}{\pgfqpoint{3.243000in}{1.817155in}}%
\pgfpathcurveto{\pgfqpoint{3.237176in}{1.811331in}}{\pgfqpoint{3.233904in}{1.803431in}}{\pgfqpoint{3.233904in}{1.795195in}}%
\pgfpathcurveto{\pgfqpoint{3.233904in}{1.786959in}}{\pgfqpoint{3.237176in}{1.779059in}}{\pgfqpoint{3.243000in}{1.773235in}}%
\pgfpathcurveto{\pgfqpoint{3.248824in}{1.767411in}}{\pgfqpoint{3.256724in}{1.764138in}}{\pgfqpoint{3.264960in}{1.764138in}}%
\pgfpathclose%
\pgfusepath{stroke,fill}%
\end{pgfscope}%
\begin{pgfscope}%
\pgfpathrectangle{\pgfqpoint{0.100000in}{0.212622in}}{\pgfqpoint{3.696000in}{3.696000in}}%
\pgfusepath{clip}%
\pgfsetbuttcap%
\pgfsetroundjoin%
\definecolor{currentfill}{rgb}{0.121569,0.466667,0.705882}%
\pgfsetfillcolor{currentfill}%
\pgfsetfillopacity{0.542886}%
\pgfsetlinewidth{1.003750pt}%
\definecolor{currentstroke}{rgb}{0.121569,0.466667,0.705882}%
\pgfsetstrokecolor{currentstroke}%
\pgfsetstrokeopacity{0.542886}%
\pgfsetdash{}{0pt}%
\pgfpathmoveto{\pgfqpoint{3.261317in}{1.764904in}}%
\pgfpathcurveto{\pgfqpoint{3.269554in}{1.764904in}}{\pgfqpoint{3.277454in}{1.768177in}}{\pgfqpoint{3.283278in}{1.774001in}}%
\pgfpathcurveto{\pgfqpoint{3.289102in}{1.779825in}}{\pgfqpoint{3.292374in}{1.787725in}}{\pgfqpoint{3.292374in}{1.795961in}}%
\pgfpathcurveto{\pgfqpoint{3.292374in}{1.804197in}}{\pgfqpoint{3.289102in}{1.812097in}}{\pgfqpoint{3.283278in}{1.817921in}}%
\pgfpathcurveto{\pgfqpoint{3.277454in}{1.823745in}}{\pgfqpoint{3.269554in}{1.827017in}}{\pgfqpoint{3.261317in}{1.827017in}}%
\pgfpathcurveto{\pgfqpoint{3.253081in}{1.827017in}}{\pgfqpoint{3.245181in}{1.823745in}}{\pgfqpoint{3.239357in}{1.817921in}}%
\pgfpathcurveto{\pgfqpoint{3.233533in}{1.812097in}}{\pgfqpoint{3.230261in}{1.804197in}}{\pgfqpoint{3.230261in}{1.795961in}}%
\pgfpathcurveto{\pgfqpoint{3.230261in}{1.787725in}}{\pgfqpoint{3.233533in}{1.779825in}}{\pgfqpoint{3.239357in}{1.774001in}}%
\pgfpathcurveto{\pgfqpoint{3.245181in}{1.768177in}}{\pgfqpoint{3.253081in}{1.764904in}}{\pgfqpoint{3.261317in}{1.764904in}}%
\pgfpathclose%
\pgfusepath{stroke,fill}%
\end{pgfscope}%
\begin{pgfscope}%
\pgfpathrectangle{\pgfqpoint{0.100000in}{0.212622in}}{\pgfqpoint{3.696000in}{3.696000in}}%
\pgfusepath{clip}%
\pgfsetbuttcap%
\pgfsetroundjoin%
\definecolor{currentfill}{rgb}{0.121569,0.466667,0.705882}%
\pgfsetfillcolor{currentfill}%
\pgfsetfillopacity{0.543671}%
\pgfsetlinewidth{1.003750pt}%
\definecolor{currentstroke}{rgb}{0.121569,0.466667,0.705882}%
\pgfsetstrokecolor{currentstroke}%
\pgfsetstrokeopacity{0.543671}%
\pgfsetdash{}{0pt}%
\pgfpathmoveto{\pgfqpoint{1.227444in}{2.126191in}}%
\pgfpathcurveto{\pgfqpoint{1.235680in}{2.126191in}}{\pgfqpoint{1.243580in}{2.129463in}}{\pgfqpoint{1.249404in}{2.135287in}}%
\pgfpathcurveto{\pgfqpoint{1.255228in}{2.141111in}}{\pgfqpoint{1.258500in}{2.149011in}}{\pgfqpoint{1.258500in}{2.157247in}}%
\pgfpathcurveto{\pgfqpoint{1.258500in}{2.165483in}}{\pgfqpoint{1.255228in}{2.173383in}}{\pgfqpoint{1.249404in}{2.179207in}}%
\pgfpathcurveto{\pgfqpoint{1.243580in}{2.185031in}}{\pgfqpoint{1.235680in}{2.188304in}}{\pgfqpoint{1.227444in}{2.188304in}}%
\pgfpathcurveto{\pgfqpoint{1.219207in}{2.188304in}}{\pgfqpoint{1.211307in}{2.185031in}}{\pgfqpoint{1.205483in}{2.179207in}}%
\pgfpathcurveto{\pgfqpoint{1.199660in}{2.173383in}}{\pgfqpoint{1.196387in}{2.165483in}}{\pgfqpoint{1.196387in}{2.157247in}}%
\pgfpathcurveto{\pgfqpoint{1.196387in}{2.149011in}}{\pgfqpoint{1.199660in}{2.141111in}}{\pgfqpoint{1.205483in}{2.135287in}}%
\pgfpathcurveto{\pgfqpoint{1.211307in}{2.129463in}}{\pgfqpoint{1.219207in}{2.126191in}}{\pgfqpoint{1.227444in}{2.126191in}}%
\pgfpathclose%
\pgfusepath{stroke,fill}%
\end{pgfscope}%
\begin{pgfscope}%
\pgfpathrectangle{\pgfqpoint{0.100000in}{0.212622in}}{\pgfqpoint{3.696000in}{3.696000in}}%
\pgfusepath{clip}%
\pgfsetbuttcap%
\pgfsetroundjoin%
\definecolor{currentfill}{rgb}{0.121569,0.466667,0.705882}%
\pgfsetfillcolor{currentfill}%
\pgfsetfillopacity{0.544602}%
\pgfsetlinewidth{1.003750pt}%
\definecolor{currentstroke}{rgb}{0.121569,0.466667,0.705882}%
\pgfsetstrokecolor{currentstroke}%
\pgfsetstrokeopacity{0.544602}%
\pgfsetdash{}{0pt}%
\pgfpathmoveto{\pgfqpoint{3.256840in}{1.765988in}}%
\pgfpathcurveto{\pgfqpoint{3.265076in}{1.765988in}}{\pgfqpoint{3.272976in}{1.769260in}}{\pgfqpoint{3.278800in}{1.775084in}}%
\pgfpathcurveto{\pgfqpoint{3.284624in}{1.780908in}}{\pgfqpoint{3.287897in}{1.788808in}}{\pgfqpoint{3.287897in}{1.797044in}}%
\pgfpathcurveto{\pgfqpoint{3.287897in}{1.805280in}}{\pgfqpoint{3.284624in}{1.813180in}}{\pgfqpoint{3.278800in}{1.819004in}}%
\pgfpathcurveto{\pgfqpoint{3.272976in}{1.824828in}}{\pgfqpoint{3.265076in}{1.828101in}}{\pgfqpoint{3.256840in}{1.828101in}}%
\pgfpathcurveto{\pgfqpoint{3.248604in}{1.828101in}}{\pgfqpoint{3.240704in}{1.824828in}}{\pgfqpoint{3.234880in}{1.819004in}}%
\pgfpathcurveto{\pgfqpoint{3.229056in}{1.813180in}}{\pgfqpoint{3.225784in}{1.805280in}}{\pgfqpoint{3.225784in}{1.797044in}}%
\pgfpathcurveto{\pgfqpoint{3.225784in}{1.788808in}}{\pgfqpoint{3.229056in}{1.780908in}}{\pgfqpoint{3.234880in}{1.775084in}}%
\pgfpathcurveto{\pgfqpoint{3.240704in}{1.769260in}}{\pgfqpoint{3.248604in}{1.765988in}}{\pgfqpoint{3.256840in}{1.765988in}}%
\pgfpathclose%
\pgfusepath{stroke,fill}%
\end{pgfscope}%
\begin{pgfscope}%
\pgfpathrectangle{\pgfqpoint{0.100000in}{0.212622in}}{\pgfqpoint{3.696000in}{3.696000in}}%
\pgfusepath{clip}%
\pgfsetbuttcap%
\pgfsetroundjoin%
\definecolor{currentfill}{rgb}{0.121569,0.466667,0.705882}%
\pgfsetfillcolor{currentfill}%
\pgfsetfillopacity{0.546173}%
\pgfsetlinewidth{1.003750pt}%
\definecolor{currentstroke}{rgb}{0.121569,0.466667,0.705882}%
\pgfsetstrokecolor{currentstroke}%
\pgfsetstrokeopacity{0.546173}%
\pgfsetdash{}{0pt}%
\pgfpathmoveto{\pgfqpoint{1.220149in}{2.127075in}}%
\pgfpathcurveto{\pgfqpoint{1.228385in}{2.127075in}}{\pgfqpoint{1.236285in}{2.130347in}}{\pgfqpoint{1.242109in}{2.136171in}}%
\pgfpathcurveto{\pgfqpoint{1.247933in}{2.141995in}}{\pgfqpoint{1.251205in}{2.149895in}}{\pgfqpoint{1.251205in}{2.158131in}}%
\pgfpathcurveto{\pgfqpoint{1.251205in}{2.166367in}}{\pgfqpoint{1.247933in}{2.174267in}}{\pgfqpoint{1.242109in}{2.180091in}}%
\pgfpathcurveto{\pgfqpoint{1.236285in}{2.185915in}}{\pgfqpoint{1.228385in}{2.189188in}}{\pgfqpoint{1.220149in}{2.189188in}}%
\pgfpathcurveto{\pgfqpoint{1.211913in}{2.189188in}}{\pgfqpoint{1.204013in}{2.185915in}}{\pgfqpoint{1.198189in}{2.180091in}}%
\pgfpathcurveto{\pgfqpoint{1.192365in}{2.174267in}}{\pgfqpoint{1.189092in}{2.166367in}}{\pgfqpoint{1.189092in}{2.158131in}}%
\pgfpathcurveto{\pgfqpoint{1.189092in}{2.149895in}}{\pgfqpoint{1.192365in}{2.141995in}}{\pgfqpoint{1.198189in}{2.136171in}}%
\pgfpathcurveto{\pgfqpoint{1.204013in}{2.130347in}}{\pgfqpoint{1.211913in}{2.127075in}}{\pgfqpoint{1.220149in}{2.127075in}}%
\pgfpathclose%
\pgfusepath{stroke,fill}%
\end{pgfscope}%
\begin{pgfscope}%
\pgfpathrectangle{\pgfqpoint{0.100000in}{0.212622in}}{\pgfqpoint{3.696000in}{3.696000in}}%
\pgfusepath{clip}%
\pgfsetbuttcap%
\pgfsetroundjoin%
\definecolor{currentfill}{rgb}{0.121569,0.466667,0.705882}%
\pgfsetfillcolor{currentfill}%
\pgfsetfillopacity{0.546983}%
\pgfsetlinewidth{1.003750pt}%
\definecolor{currentstroke}{rgb}{0.121569,0.466667,0.705882}%
\pgfsetstrokecolor{currentstroke}%
\pgfsetstrokeopacity{0.546983}%
\pgfsetdash{}{0pt}%
\pgfpathmoveto{\pgfqpoint{3.253831in}{1.766349in}}%
\pgfpathcurveto{\pgfqpoint{3.262067in}{1.766349in}}{\pgfqpoint{3.269967in}{1.769621in}}{\pgfqpoint{3.275791in}{1.775445in}}%
\pgfpathcurveto{\pgfqpoint{3.281615in}{1.781269in}}{\pgfqpoint{3.284887in}{1.789169in}}{\pgfqpoint{3.284887in}{1.797406in}}%
\pgfpathcurveto{\pgfqpoint{3.284887in}{1.805642in}}{\pgfqpoint{3.281615in}{1.813542in}}{\pgfqpoint{3.275791in}{1.819366in}}%
\pgfpathcurveto{\pgfqpoint{3.269967in}{1.825190in}}{\pgfqpoint{3.262067in}{1.828462in}}{\pgfqpoint{3.253831in}{1.828462in}}%
\pgfpathcurveto{\pgfqpoint{3.245595in}{1.828462in}}{\pgfqpoint{3.237694in}{1.825190in}}{\pgfqpoint{3.231871in}{1.819366in}}%
\pgfpathcurveto{\pgfqpoint{3.226047in}{1.813542in}}{\pgfqpoint{3.222774in}{1.805642in}}{\pgfqpoint{3.222774in}{1.797406in}}%
\pgfpathcurveto{\pgfqpoint{3.222774in}{1.789169in}}{\pgfqpoint{3.226047in}{1.781269in}}{\pgfqpoint{3.231871in}{1.775445in}}%
\pgfpathcurveto{\pgfqpoint{3.237694in}{1.769621in}}{\pgfqpoint{3.245595in}{1.766349in}}{\pgfqpoint{3.253831in}{1.766349in}}%
\pgfpathclose%
\pgfusepath{stroke,fill}%
\end{pgfscope}%
\begin{pgfscope}%
\pgfpathrectangle{\pgfqpoint{0.100000in}{0.212622in}}{\pgfqpoint{3.696000in}{3.696000in}}%
\pgfusepath{clip}%
\pgfsetbuttcap%
\pgfsetroundjoin%
\definecolor{currentfill}{rgb}{0.121569,0.466667,0.705882}%
\pgfsetfillcolor{currentfill}%
\pgfsetfillopacity{0.548292}%
\pgfsetlinewidth{1.003750pt}%
\definecolor{currentstroke}{rgb}{0.121569,0.466667,0.705882}%
\pgfsetstrokecolor{currentstroke}%
\pgfsetstrokeopacity{0.548292}%
\pgfsetdash{}{0pt}%
\pgfpathmoveto{\pgfqpoint{1.214925in}{2.127461in}}%
\pgfpathcurveto{\pgfqpoint{1.223162in}{2.127461in}}{\pgfqpoint{1.231062in}{2.130734in}}{\pgfqpoint{1.236886in}{2.136558in}}%
\pgfpathcurveto{\pgfqpoint{1.242710in}{2.142382in}}{\pgfqpoint{1.245982in}{2.150282in}}{\pgfqpoint{1.245982in}{2.158518in}}%
\pgfpathcurveto{\pgfqpoint{1.245982in}{2.166754in}}{\pgfqpoint{1.242710in}{2.174654in}}{\pgfqpoint{1.236886in}{2.180478in}}%
\pgfpathcurveto{\pgfqpoint{1.231062in}{2.186302in}}{\pgfqpoint{1.223162in}{2.189574in}}{\pgfqpoint{1.214925in}{2.189574in}}%
\pgfpathcurveto{\pgfqpoint{1.206689in}{2.189574in}}{\pgfqpoint{1.198789in}{2.186302in}}{\pgfqpoint{1.192965in}{2.180478in}}%
\pgfpathcurveto{\pgfqpoint{1.187141in}{2.174654in}}{\pgfqpoint{1.183869in}{2.166754in}}{\pgfqpoint{1.183869in}{2.158518in}}%
\pgfpathcurveto{\pgfqpoint{1.183869in}{2.150282in}}{\pgfqpoint{1.187141in}{2.142382in}}{\pgfqpoint{1.192965in}{2.136558in}}%
\pgfpathcurveto{\pgfqpoint{1.198789in}{2.130734in}}{\pgfqpoint{1.206689in}{2.127461in}}{\pgfqpoint{1.214925in}{2.127461in}}%
\pgfpathclose%
\pgfusepath{stroke,fill}%
\end{pgfscope}%
\begin{pgfscope}%
\pgfpathrectangle{\pgfqpoint{0.100000in}{0.212622in}}{\pgfqpoint{3.696000in}{3.696000in}}%
\pgfusepath{clip}%
\pgfsetbuttcap%
\pgfsetroundjoin%
\definecolor{currentfill}{rgb}{0.121569,0.466667,0.705882}%
\pgfsetfillcolor{currentfill}%
\pgfsetfillopacity{0.549384}%
\pgfsetlinewidth{1.003750pt}%
\definecolor{currentstroke}{rgb}{0.121569,0.466667,0.705882}%
\pgfsetstrokecolor{currentstroke}%
\pgfsetstrokeopacity{0.549384}%
\pgfsetdash{}{0pt}%
\pgfpathmoveto{\pgfqpoint{3.249567in}{1.766807in}}%
\pgfpathcurveto{\pgfqpoint{3.257803in}{1.766807in}}{\pgfqpoint{3.265703in}{1.770079in}}{\pgfqpoint{3.271527in}{1.775903in}}%
\pgfpathcurveto{\pgfqpoint{3.277351in}{1.781727in}}{\pgfqpoint{3.280624in}{1.789627in}}{\pgfqpoint{3.280624in}{1.797863in}}%
\pgfpathcurveto{\pgfqpoint{3.280624in}{1.806099in}}{\pgfqpoint{3.277351in}{1.814000in}}{\pgfqpoint{3.271527in}{1.819823in}}%
\pgfpathcurveto{\pgfqpoint{3.265703in}{1.825647in}}{\pgfqpoint{3.257803in}{1.828920in}}{\pgfqpoint{3.249567in}{1.828920in}}%
\pgfpathcurveto{\pgfqpoint{3.241331in}{1.828920in}}{\pgfqpoint{3.233431in}{1.825647in}}{\pgfqpoint{3.227607in}{1.819823in}}%
\pgfpathcurveto{\pgfqpoint{3.221783in}{1.814000in}}{\pgfqpoint{3.218511in}{1.806099in}}{\pgfqpoint{3.218511in}{1.797863in}}%
\pgfpathcurveto{\pgfqpoint{3.218511in}{1.789627in}}{\pgfqpoint{3.221783in}{1.781727in}}{\pgfqpoint{3.227607in}{1.775903in}}%
\pgfpathcurveto{\pgfqpoint{3.233431in}{1.770079in}}{\pgfqpoint{3.241331in}{1.766807in}}{\pgfqpoint{3.249567in}{1.766807in}}%
\pgfpathclose%
\pgfusepath{stroke,fill}%
\end{pgfscope}%
\begin{pgfscope}%
\pgfpathrectangle{\pgfqpoint{0.100000in}{0.212622in}}{\pgfqpoint{3.696000in}{3.696000in}}%
\pgfusepath{clip}%
\pgfsetbuttcap%
\pgfsetroundjoin%
\definecolor{currentfill}{rgb}{0.121569,0.466667,0.705882}%
\pgfsetfillcolor{currentfill}%
\pgfsetfillopacity{0.550077}%
\pgfsetlinewidth{1.003750pt}%
\definecolor{currentstroke}{rgb}{0.121569,0.466667,0.705882}%
\pgfsetstrokecolor{currentstroke}%
\pgfsetstrokeopacity{0.550077}%
\pgfsetdash{}{0pt}%
\pgfpathmoveto{\pgfqpoint{1.211066in}{2.127637in}}%
\pgfpathcurveto{\pgfqpoint{1.219302in}{2.127637in}}{\pgfqpoint{1.227202in}{2.130909in}}{\pgfqpoint{1.233026in}{2.136733in}}%
\pgfpathcurveto{\pgfqpoint{1.238850in}{2.142557in}}{\pgfqpoint{1.242122in}{2.150457in}}{\pgfqpoint{1.242122in}{2.158693in}}%
\pgfpathcurveto{\pgfqpoint{1.242122in}{2.166930in}}{\pgfqpoint{1.238850in}{2.174830in}}{\pgfqpoint{1.233026in}{2.180654in}}%
\pgfpathcurveto{\pgfqpoint{1.227202in}{2.186477in}}{\pgfqpoint{1.219302in}{2.189750in}}{\pgfqpoint{1.211066in}{2.189750in}}%
\pgfpathcurveto{\pgfqpoint{1.202830in}{2.189750in}}{\pgfqpoint{1.194930in}{2.186477in}}{\pgfqpoint{1.189106in}{2.180654in}}%
\pgfpathcurveto{\pgfqpoint{1.183282in}{2.174830in}}{\pgfqpoint{1.180009in}{2.166930in}}{\pgfqpoint{1.180009in}{2.158693in}}%
\pgfpathcurveto{\pgfqpoint{1.180009in}{2.150457in}}{\pgfqpoint{1.183282in}{2.142557in}}{\pgfqpoint{1.189106in}{2.136733in}}%
\pgfpathcurveto{\pgfqpoint{1.194930in}{2.130909in}}{\pgfqpoint{1.202830in}{2.127637in}}{\pgfqpoint{1.211066in}{2.127637in}}%
\pgfpathclose%
\pgfusepath{stroke,fill}%
\end{pgfscope}%
\begin{pgfscope}%
\pgfpathrectangle{\pgfqpoint{0.100000in}{0.212622in}}{\pgfqpoint{3.696000in}{3.696000in}}%
\pgfusepath{clip}%
\pgfsetbuttcap%
\pgfsetroundjoin%
\definecolor{currentfill}{rgb}{0.121569,0.466667,0.705882}%
\pgfsetfillcolor{currentfill}%
\pgfsetfillopacity{0.551737}%
\pgfsetlinewidth{1.003750pt}%
\definecolor{currentstroke}{rgb}{0.121569,0.466667,0.705882}%
\pgfsetstrokecolor{currentstroke}%
\pgfsetstrokeopacity{0.551737}%
\pgfsetdash{}{0pt}%
\pgfpathmoveto{\pgfqpoint{3.243682in}{1.768169in}}%
\pgfpathcurveto{\pgfqpoint{3.251918in}{1.768169in}}{\pgfqpoint{3.259818in}{1.771441in}}{\pgfqpoint{3.265642in}{1.777265in}}%
\pgfpathcurveto{\pgfqpoint{3.271466in}{1.783089in}}{\pgfqpoint{3.274738in}{1.790989in}}{\pgfqpoint{3.274738in}{1.799225in}}%
\pgfpathcurveto{\pgfqpoint{3.274738in}{1.807462in}}{\pgfqpoint{3.271466in}{1.815362in}}{\pgfqpoint{3.265642in}{1.821186in}}%
\pgfpathcurveto{\pgfqpoint{3.259818in}{1.827010in}}{\pgfqpoint{3.251918in}{1.830282in}}{\pgfqpoint{3.243682in}{1.830282in}}%
\pgfpathcurveto{\pgfqpoint{3.235445in}{1.830282in}}{\pgfqpoint{3.227545in}{1.827010in}}{\pgfqpoint{3.221721in}{1.821186in}}%
\pgfpathcurveto{\pgfqpoint{3.215897in}{1.815362in}}{\pgfqpoint{3.212625in}{1.807462in}}{\pgfqpoint{3.212625in}{1.799225in}}%
\pgfpathcurveto{\pgfqpoint{3.212625in}{1.790989in}}{\pgfqpoint{3.215897in}{1.783089in}}{\pgfqpoint{3.221721in}{1.777265in}}%
\pgfpathcurveto{\pgfqpoint{3.227545in}{1.771441in}}{\pgfqpoint{3.235445in}{1.768169in}}{\pgfqpoint{3.243682in}{1.768169in}}%
\pgfpathclose%
\pgfusepath{stroke,fill}%
\end{pgfscope}%
\begin{pgfscope}%
\pgfpathrectangle{\pgfqpoint{0.100000in}{0.212622in}}{\pgfqpoint{3.696000in}{3.696000in}}%
\pgfusepath{clip}%
\pgfsetbuttcap%
\pgfsetroundjoin%
\definecolor{currentfill}{rgb}{0.121569,0.466667,0.705882}%
\pgfsetfillcolor{currentfill}%
\pgfsetfillopacity{0.551742}%
\pgfsetlinewidth{1.003750pt}%
\definecolor{currentstroke}{rgb}{0.121569,0.466667,0.705882}%
\pgfsetstrokecolor{currentstroke}%
\pgfsetstrokeopacity{0.551742}%
\pgfsetdash{}{0pt}%
\pgfpathmoveto{\pgfqpoint{1.207274in}{2.127750in}}%
\pgfpathcurveto{\pgfqpoint{1.215510in}{2.127750in}}{\pgfqpoint{1.223410in}{2.131022in}}{\pgfqpoint{1.229234in}{2.136846in}}%
\pgfpathcurveto{\pgfqpoint{1.235058in}{2.142670in}}{\pgfqpoint{1.238330in}{2.150570in}}{\pgfqpoint{1.238330in}{2.158806in}}%
\pgfpathcurveto{\pgfqpoint{1.238330in}{2.167043in}}{\pgfqpoint{1.235058in}{2.174943in}}{\pgfqpoint{1.229234in}{2.180767in}}%
\pgfpathcurveto{\pgfqpoint{1.223410in}{2.186591in}}{\pgfqpoint{1.215510in}{2.189863in}}{\pgfqpoint{1.207274in}{2.189863in}}%
\pgfpathcurveto{\pgfqpoint{1.199037in}{2.189863in}}{\pgfqpoint{1.191137in}{2.186591in}}{\pgfqpoint{1.185313in}{2.180767in}}%
\pgfpathcurveto{\pgfqpoint{1.179490in}{2.174943in}}{\pgfqpoint{1.176217in}{2.167043in}}{\pgfqpoint{1.176217in}{2.158806in}}%
\pgfpathcurveto{\pgfqpoint{1.176217in}{2.150570in}}{\pgfqpoint{1.179490in}{2.142670in}}{\pgfqpoint{1.185313in}{2.136846in}}%
\pgfpathcurveto{\pgfqpoint{1.191137in}{2.131022in}}{\pgfqpoint{1.199037in}{2.127750in}}{\pgfqpoint{1.207274in}{2.127750in}}%
\pgfpathclose%
\pgfusepath{stroke,fill}%
\end{pgfscope}%
\begin{pgfscope}%
\pgfpathrectangle{\pgfqpoint{0.100000in}{0.212622in}}{\pgfqpoint{3.696000in}{3.696000in}}%
\pgfusepath{clip}%
\pgfsetbuttcap%
\pgfsetroundjoin%
\definecolor{currentfill}{rgb}{0.121569,0.466667,0.705882}%
\pgfsetfillcolor{currentfill}%
\pgfsetfillopacity{0.553335}%
\pgfsetlinewidth{1.003750pt}%
\definecolor{currentstroke}{rgb}{0.121569,0.466667,0.705882}%
\pgfsetstrokecolor{currentstroke}%
\pgfsetstrokeopacity{0.553335}%
\pgfsetdash{}{0pt}%
\pgfpathmoveto{\pgfqpoint{1.204318in}{2.127583in}}%
\pgfpathcurveto{\pgfqpoint{1.212554in}{2.127583in}}{\pgfqpoint{1.220454in}{2.130855in}}{\pgfqpoint{1.226278in}{2.136679in}}%
\pgfpathcurveto{\pgfqpoint{1.232102in}{2.142503in}}{\pgfqpoint{1.235374in}{2.150403in}}{\pgfqpoint{1.235374in}{2.158639in}}%
\pgfpathcurveto{\pgfqpoint{1.235374in}{2.166875in}}{\pgfqpoint{1.232102in}{2.174776in}}{\pgfqpoint{1.226278in}{2.180599in}}%
\pgfpathcurveto{\pgfqpoint{1.220454in}{2.186423in}}{\pgfqpoint{1.212554in}{2.189696in}}{\pgfqpoint{1.204318in}{2.189696in}}%
\pgfpathcurveto{\pgfqpoint{1.196082in}{2.189696in}}{\pgfqpoint{1.188182in}{2.186423in}}{\pgfqpoint{1.182358in}{2.180599in}}%
\pgfpathcurveto{\pgfqpoint{1.176534in}{2.174776in}}{\pgfqpoint{1.173261in}{2.166875in}}{\pgfqpoint{1.173261in}{2.158639in}}%
\pgfpathcurveto{\pgfqpoint{1.173261in}{2.150403in}}{\pgfqpoint{1.176534in}{2.142503in}}{\pgfqpoint{1.182358in}{2.136679in}}%
\pgfpathcurveto{\pgfqpoint{1.188182in}{2.130855in}}{\pgfqpoint{1.196082in}{2.127583in}}{\pgfqpoint{1.204318in}{2.127583in}}%
\pgfpathclose%
\pgfusepath{stroke,fill}%
\end{pgfscope}%
\begin{pgfscope}%
\pgfpathrectangle{\pgfqpoint{0.100000in}{0.212622in}}{\pgfqpoint{3.696000in}{3.696000in}}%
\pgfusepath{clip}%
\pgfsetbuttcap%
\pgfsetroundjoin%
\definecolor{currentfill}{rgb}{0.121569,0.466667,0.705882}%
\pgfsetfillcolor{currentfill}%
\pgfsetfillopacity{0.554622}%
\pgfsetlinewidth{1.003750pt}%
\definecolor{currentstroke}{rgb}{0.121569,0.466667,0.705882}%
\pgfsetstrokecolor{currentstroke}%
\pgfsetstrokeopacity{0.554622}%
\pgfsetdash{}{0pt}%
\pgfpathmoveto{\pgfqpoint{1.200504in}{2.127886in}}%
\pgfpathcurveto{\pgfqpoint{1.208740in}{2.127886in}}{\pgfqpoint{1.216640in}{2.131158in}}{\pgfqpoint{1.222464in}{2.136982in}}%
\pgfpathcurveto{\pgfqpoint{1.228288in}{2.142806in}}{\pgfqpoint{1.231560in}{2.150706in}}{\pgfqpoint{1.231560in}{2.158942in}}%
\pgfpathcurveto{\pgfqpoint{1.231560in}{2.167178in}}{\pgfqpoint{1.228288in}{2.175079in}}{\pgfqpoint{1.222464in}{2.180902in}}%
\pgfpathcurveto{\pgfqpoint{1.216640in}{2.186726in}}{\pgfqpoint{1.208740in}{2.189999in}}{\pgfqpoint{1.200504in}{2.189999in}}%
\pgfpathcurveto{\pgfqpoint{1.192267in}{2.189999in}}{\pgfqpoint{1.184367in}{2.186726in}}{\pgfqpoint{1.178543in}{2.180902in}}%
\pgfpathcurveto{\pgfqpoint{1.172720in}{2.175079in}}{\pgfqpoint{1.169447in}{2.167178in}}{\pgfqpoint{1.169447in}{2.158942in}}%
\pgfpathcurveto{\pgfqpoint{1.169447in}{2.150706in}}{\pgfqpoint{1.172720in}{2.142806in}}{\pgfqpoint{1.178543in}{2.136982in}}%
\pgfpathcurveto{\pgfqpoint{1.184367in}{2.131158in}}{\pgfqpoint{1.192267in}{2.127886in}}{\pgfqpoint{1.200504in}{2.127886in}}%
\pgfpathclose%
\pgfusepath{stroke,fill}%
\end{pgfscope}%
\begin{pgfscope}%
\pgfpathrectangle{\pgfqpoint{0.100000in}{0.212622in}}{\pgfqpoint{3.696000in}{3.696000in}}%
\pgfusepath{clip}%
\pgfsetbuttcap%
\pgfsetroundjoin%
\definecolor{currentfill}{rgb}{0.121569,0.466667,0.705882}%
\pgfsetfillcolor{currentfill}%
\pgfsetfillopacity{0.554957}%
\pgfsetlinewidth{1.003750pt}%
\definecolor{currentstroke}{rgb}{0.121569,0.466667,0.705882}%
\pgfsetstrokecolor{currentstroke}%
\pgfsetstrokeopacity{0.554957}%
\pgfsetdash{}{0pt}%
\pgfpathmoveto{\pgfqpoint{3.238720in}{1.768648in}}%
\pgfpathcurveto{\pgfqpoint{3.246956in}{1.768648in}}{\pgfqpoint{3.254856in}{1.771921in}}{\pgfqpoint{3.260680in}{1.777744in}}%
\pgfpathcurveto{\pgfqpoint{3.266504in}{1.783568in}}{\pgfqpoint{3.269777in}{1.791468in}}{\pgfqpoint{3.269777in}{1.799705in}}%
\pgfpathcurveto{\pgfqpoint{3.269777in}{1.807941in}}{\pgfqpoint{3.266504in}{1.815841in}}{\pgfqpoint{3.260680in}{1.821665in}}%
\pgfpathcurveto{\pgfqpoint{3.254856in}{1.827489in}}{\pgfqpoint{3.246956in}{1.830761in}}{\pgfqpoint{3.238720in}{1.830761in}}%
\pgfpathcurveto{\pgfqpoint{3.230484in}{1.830761in}}{\pgfqpoint{3.222584in}{1.827489in}}{\pgfqpoint{3.216760in}{1.821665in}}%
\pgfpathcurveto{\pgfqpoint{3.210936in}{1.815841in}}{\pgfqpoint{3.207664in}{1.807941in}}{\pgfqpoint{3.207664in}{1.799705in}}%
\pgfpathcurveto{\pgfqpoint{3.207664in}{1.791468in}}{\pgfqpoint{3.210936in}{1.783568in}}{\pgfqpoint{3.216760in}{1.777744in}}%
\pgfpathcurveto{\pgfqpoint{3.222584in}{1.771921in}}{\pgfqpoint{3.230484in}{1.768648in}}{\pgfqpoint{3.238720in}{1.768648in}}%
\pgfpathclose%
\pgfusepath{stroke,fill}%
\end{pgfscope}%
\begin{pgfscope}%
\pgfpathrectangle{\pgfqpoint{0.100000in}{0.212622in}}{\pgfqpoint{3.696000in}{3.696000in}}%
\pgfusepath{clip}%
\pgfsetbuttcap%
\pgfsetroundjoin%
\definecolor{currentfill}{rgb}{0.121569,0.466667,0.705882}%
\pgfsetfillcolor{currentfill}%
\pgfsetfillopacity{0.555646}%
\pgfsetlinewidth{1.003750pt}%
\definecolor{currentstroke}{rgb}{0.121569,0.466667,0.705882}%
\pgfsetstrokecolor{currentstroke}%
\pgfsetstrokeopacity{0.555646}%
\pgfsetdash{}{0pt}%
\pgfpathmoveto{\pgfqpoint{1.199207in}{2.128049in}}%
\pgfpathcurveto{\pgfqpoint{1.207444in}{2.128049in}}{\pgfqpoint{1.215344in}{2.131321in}}{\pgfqpoint{1.221168in}{2.137145in}}%
\pgfpathcurveto{\pgfqpoint{1.226992in}{2.142969in}}{\pgfqpoint{1.230264in}{2.150869in}}{\pgfqpoint{1.230264in}{2.159105in}}%
\pgfpathcurveto{\pgfqpoint{1.230264in}{2.167342in}}{\pgfqpoint{1.226992in}{2.175242in}}{\pgfqpoint{1.221168in}{2.181066in}}%
\pgfpathcurveto{\pgfqpoint{1.215344in}{2.186889in}}{\pgfqpoint{1.207444in}{2.190162in}}{\pgfqpoint{1.199207in}{2.190162in}}%
\pgfpathcurveto{\pgfqpoint{1.190971in}{2.190162in}}{\pgfqpoint{1.183071in}{2.186889in}}{\pgfqpoint{1.177247in}{2.181066in}}%
\pgfpathcurveto{\pgfqpoint{1.171423in}{2.175242in}}{\pgfqpoint{1.168151in}{2.167342in}}{\pgfqpoint{1.168151in}{2.159105in}}%
\pgfpathcurveto{\pgfqpoint{1.168151in}{2.150869in}}{\pgfqpoint{1.171423in}{2.142969in}}{\pgfqpoint{1.177247in}{2.137145in}}%
\pgfpathcurveto{\pgfqpoint{1.183071in}{2.131321in}}{\pgfqpoint{1.190971in}{2.128049in}}{\pgfqpoint{1.199207in}{2.128049in}}%
\pgfpathclose%
\pgfusepath{stroke,fill}%
\end{pgfscope}%
\begin{pgfscope}%
\pgfpathrectangle{\pgfqpoint{0.100000in}{0.212622in}}{\pgfqpoint{3.696000in}{3.696000in}}%
\pgfusepath{clip}%
\pgfsetbuttcap%
\pgfsetroundjoin%
\definecolor{currentfill}{rgb}{0.121569,0.466667,0.705882}%
\pgfsetfillcolor{currentfill}%
\pgfsetfillopacity{0.556287}%
\pgfsetlinewidth{1.003750pt}%
\definecolor{currentstroke}{rgb}{0.121569,0.466667,0.705882}%
\pgfsetstrokecolor{currentstroke}%
\pgfsetstrokeopacity{0.556287}%
\pgfsetdash{}{0pt}%
\pgfpathmoveto{\pgfqpoint{1.197233in}{2.128276in}}%
\pgfpathcurveto{\pgfqpoint{1.205470in}{2.128276in}}{\pgfqpoint{1.213370in}{2.131548in}}{\pgfqpoint{1.219194in}{2.137372in}}%
\pgfpathcurveto{\pgfqpoint{1.225018in}{2.143196in}}{\pgfqpoint{1.228290in}{2.151096in}}{\pgfqpoint{1.228290in}{2.159332in}}%
\pgfpathcurveto{\pgfqpoint{1.228290in}{2.167569in}}{\pgfqpoint{1.225018in}{2.175469in}}{\pgfqpoint{1.219194in}{2.181293in}}%
\pgfpathcurveto{\pgfqpoint{1.213370in}{2.187116in}}{\pgfqpoint{1.205470in}{2.190389in}}{\pgfqpoint{1.197233in}{2.190389in}}%
\pgfpathcurveto{\pgfqpoint{1.188997in}{2.190389in}}{\pgfqpoint{1.181097in}{2.187116in}}{\pgfqpoint{1.175273in}{2.181293in}}%
\pgfpathcurveto{\pgfqpoint{1.169449in}{2.175469in}}{\pgfqpoint{1.166177in}{2.167569in}}{\pgfqpoint{1.166177in}{2.159332in}}%
\pgfpathcurveto{\pgfqpoint{1.166177in}{2.151096in}}{\pgfqpoint{1.169449in}{2.143196in}}{\pgfqpoint{1.175273in}{2.137372in}}%
\pgfpathcurveto{\pgfqpoint{1.181097in}{2.131548in}}{\pgfqpoint{1.188997in}{2.128276in}}{\pgfqpoint{1.197233in}{2.128276in}}%
\pgfpathclose%
\pgfusepath{stroke,fill}%
\end{pgfscope}%
\begin{pgfscope}%
\pgfpathrectangle{\pgfqpoint{0.100000in}{0.212622in}}{\pgfqpoint{3.696000in}{3.696000in}}%
\pgfusepath{clip}%
\pgfsetbuttcap%
\pgfsetroundjoin%
\definecolor{currentfill}{rgb}{0.121569,0.466667,0.705882}%
\pgfsetfillcolor{currentfill}%
\pgfsetfillopacity{0.556527}%
\pgfsetlinewidth{1.003750pt}%
\definecolor{currentstroke}{rgb}{0.121569,0.466667,0.705882}%
\pgfsetstrokecolor{currentstroke}%
\pgfsetstrokeopacity{0.556527}%
\pgfsetdash{}{0pt}%
\pgfpathmoveto{\pgfqpoint{1.198042in}{2.128590in}}%
\pgfpathcurveto{\pgfqpoint{1.206279in}{2.128590in}}{\pgfqpoint{1.214179in}{2.131862in}}{\pgfqpoint{1.220003in}{2.137686in}}%
\pgfpathcurveto{\pgfqpoint{1.225827in}{2.143510in}}{\pgfqpoint{1.229099in}{2.151410in}}{\pgfqpoint{1.229099in}{2.159647in}}%
\pgfpathcurveto{\pgfqpoint{1.229099in}{2.167883in}}{\pgfqpoint{1.225827in}{2.175783in}}{\pgfqpoint{1.220003in}{2.181607in}}%
\pgfpathcurveto{\pgfqpoint{1.214179in}{2.187431in}}{\pgfqpoint{1.206279in}{2.190703in}}{\pgfqpoint{1.198042in}{2.190703in}}%
\pgfpathcurveto{\pgfqpoint{1.189806in}{2.190703in}}{\pgfqpoint{1.181906in}{2.187431in}}{\pgfqpoint{1.176082in}{2.181607in}}%
\pgfpathcurveto{\pgfqpoint{1.170258in}{2.175783in}}{\pgfqpoint{1.166986in}{2.167883in}}{\pgfqpoint{1.166986in}{2.159647in}}%
\pgfpathcurveto{\pgfqpoint{1.166986in}{2.151410in}}{\pgfqpoint{1.170258in}{2.143510in}}{\pgfqpoint{1.176082in}{2.137686in}}%
\pgfpathcurveto{\pgfqpoint{1.181906in}{2.131862in}}{\pgfqpoint{1.189806in}{2.128590in}}{\pgfqpoint{1.198042in}{2.128590in}}%
\pgfpathclose%
\pgfusepath{stroke,fill}%
\end{pgfscope}%
\begin{pgfscope}%
\pgfpathrectangle{\pgfqpoint{0.100000in}{0.212622in}}{\pgfqpoint{3.696000in}{3.696000in}}%
\pgfusepath{clip}%
\pgfsetbuttcap%
\pgfsetroundjoin%
\definecolor{currentfill}{rgb}{0.121569,0.466667,0.705882}%
\pgfsetfillcolor{currentfill}%
\pgfsetfillopacity{0.557006}%
\pgfsetlinewidth{1.003750pt}%
\definecolor{currentstroke}{rgb}{0.121569,0.466667,0.705882}%
\pgfsetstrokecolor{currentstroke}%
\pgfsetstrokeopacity{0.557006}%
\pgfsetdash{}{0pt}%
\pgfpathmoveto{\pgfqpoint{1.196855in}{2.128666in}}%
\pgfpathcurveto{\pgfqpoint{1.205091in}{2.128666in}}{\pgfqpoint{1.212991in}{2.131939in}}{\pgfqpoint{1.218815in}{2.137762in}}%
\pgfpathcurveto{\pgfqpoint{1.224639in}{2.143586in}}{\pgfqpoint{1.227912in}{2.151486in}}{\pgfqpoint{1.227912in}{2.159723in}}%
\pgfpathcurveto{\pgfqpoint{1.227912in}{2.167959in}}{\pgfqpoint{1.224639in}{2.175859in}}{\pgfqpoint{1.218815in}{2.181683in}}%
\pgfpathcurveto{\pgfqpoint{1.212991in}{2.187507in}}{\pgfqpoint{1.205091in}{2.190779in}}{\pgfqpoint{1.196855in}{2.190779in}}%
\pgfpathcurveto{\pgfqpoint{1.188619in}{2.190779in}}{\pgfqpoint{1.180719in}{2.187507in}}{\pgfqpoint{1.174895in}{2.181683in}}%
\pgfpathcurveto{\pgfqpoint{1.169071in}{2.175859in}}{\pgfqpoint{1.165799in}{2.167959in}}{\pgfqpoint{1.165799in}{2.159723in}}%
\pgfpathcurveto{\pgfqpoint{1.165799in}{2.151486in}}{\pgfqpoint{1.169071in}{2.143586in}}{\pgfqpoint{1.174895in}{2.137762in}}%
\pgfpathcurveto{\pgfqpoint{1.180719in}{2.131939in}}{\pgfqpoint{1.188619in}{2.128666in}}{\pgfqpoint{1.196855in}{2.128666in}}%
\pgfpathclose%
\pgfusepath{stroke,fill}%
\end{pgfscope}%
\begin{pgfscope}%
\pgfpathrectangle{\pgfqpoint{0.100000in}{0.212622in}}{\pgfqpoint{3.696000in}{3.696000in}}%
\pgfusepath{clip}%
\pgfsetbuttcap%
\pgfsetroundjoin%
\definecolor{currentfill}{rgb}{0.121569,0.466667,0.705882}%
\pgfsetfillcolor{currentfill}%
\pgfsetfillopacity{0.557871}%
\pgfsetlinewidth{1.003750pt}%
\definecolor{currentstroke}{rgb}{0.121569,0.466667,0.705882}%
\pgfsetstrokecolor{currentstroke}%
\pgfsetstrokeopacity{0.557871}%
\pgfsetdash{}{0pt}%
\pgfpathmoveto{\pgfqpoint{1.194690in}{2.128769in}}%
\pgfpathcurveto{\pgfqpoint{1.202927in}{2.128769in}}{\pgfqpoint{1.210827in}{2.132041in}}{\pgfqpoint{1.216650in}{2.137865in}}%
\pgfpathcurveto{\pgfqpoint{1.222474in}{2.143689in}}{\pgfqpoint{1.225747in}{2.151589in}}{\pgfqpoint{1.225747in}{2.159825in}}%
\pgfpathcurveto{\pgfqpoint{1.225747in}{2.168061in}}{\pgfqpoint{1.222474in}{2.175961in}}{\pgfqpoint{1.216650in}{2.181785in}}%
\pgfpathcurveto{\pgfqpoint{1.210827in}{2.187609in}}{\pgfqpoint{1.202927in}{2.190882in}}{\pgfqpoint{1.194690in}{2.190882in}}%
\pgfpathcurveto{\pgfqpoint{1.186454in}{2.190882in}}{\pgfqpoint{1.178554in}{2.187609in}}{\pgfqpoint{1.172730in}{2.181785in}}%
\pgfpathcurveto{\pgfqpoint{1.166906in}{2.175961in}}{\pgfqpoint{1.163634in}{2.168061in}}{\pgfqpoint{1.163634in}{2.159825in}}%
\pgfpathcurveto{\pgfqpoint{1.163634in}{2.151589in}}{\pgfqpoint{1.166906in}{2.143689in}}{\pgfqpoint{1.172730in}{2.137865in}}%
\pgfpathcurveto{\pgfqpoint{1.178554in}{2.132041in}}{\pgfqpoint{1.186454in}{2.128769in}}{\pgfqpoint{1.194690in}{2.128769in}}%
\pgfpathclose%
\pgfusepath{stroke,fill}%
\end{pgfscope}%
\begin{pgfscope}%
\pgfpathrectangle{\pgfqpoint{0.100000in}{0.212622in}}{\pgfqpoint{3.696000in}{3.696000in}}%
\pgfusepath{clip}%
\pgfsetbuttcap%
\pgfsetroundjoin%
\definecolor{currentfill}{rgb}{0.121569,0.466667,0.705882}%
\pgfsetfillcolor{currentfill}%
\pgfsetfillopacity{0.558423}%
\pgfsetlinewidth{1.003750pt}%
\definecolor{currentstroke}{rgb}{0.121569,0.466667,0.705882}%
\pgfsetstrokecolor{currentstroke}%
\pgfsetstrokeopacity{0.558423}%
\pgfsetdash{}{0pt}%
\pgfpathmoveto{\pgfqpoint{3.235272in}{1.769009in}}%
\pgfpathcurveto{\pgfqpoint{3.243508in}{1.769009in}}{\pgfqpoint{3.251408in}{1.772281in}}{\pgfqpoint{3.257232in}{1.778105in}}%
\pgfpathcurveto{\pgfqpoint{3.263056in}{1.783929in}}{\pgfqpoint{3.266329in}{1.791829in}}{\pgfqpoint{3.266329in}{1.800066in}}%
\pgfpathcurveto{\pgfqpoint{3.266329in}{1.808302in}}{\pgfqpoint{3.263056in}{1.816202in}}{\pgfqpoint{3.257232in}{1.822026in}}%
\pgfpathcurveto{\pgfqpoint{3.251408in}{1.827850in}}{\pgfqpoint{3.243508in}{1.831122in}}{\pgfqpoint{3.235272in}{1.831122in}}%
\pgfpathcurveto{\pgfqpoint{3.227036in}{1.831122in}}{\pgfqpoint{3.219136in}{1.827850in}}{\pgfqpoint{3.213312in}{1.822026in}}%
\pgfpathcurveto{\pgfqpoint{3.207488in}{1.816202in}}{\pgfqpoint{3.204216in}{1.808302in}}{\pgfqpoint{3.204216in}{1.800066in}}%
\pgfpathcurveto{\pgfqpoint{3.204216in}{1.791829in}}{\pgfqpoint{3.207488in}{1.783929in}}{\pgfqpoint{3.213312in}{1.778105in}}%
\pgfpathcurveto{\pgfqpoint{3.219136in}{1.772281in}}{\pgfqpoint{3.227036in}{1.769009in}}{\pgfqpoint{3.235272in}{1.769009in}}%
\pgfpathclose%
\pgfusepath{stroke,fill}%
\end{pgfscope}%
\begin{pgfscope}%
\pgfpathrectangle{\pgfqpoint{0.100000in}{0.212622in}}{\pgfqpoint{3.696000in}{3.696000in}}%
\pgfusepath{clip}%
\pgfsetbuttcap%
\pgfsetroundjoin%
\definecolor{currentfill}{rgb}{0.121569,0.466667,0.705882}%
\pgfsetfillcolor{currentfill}%
\pgfsetfillopacity{0.559530}%
\pgfsetlinewidth{1.003750pt}%
\definecolor{currentstroke}{rgb}{0.121569,0.466667,0.705882}%
\pgfsetstrokecolor{currentstroke}%
\pgfsetstrokeopacity{0.559530}%
\pgfsetdash{}{0pt}%
\pgfpathmoveto{\pgfqpoint{1.191393in}{2.128880in}}%
\pgfpathcurveto{\pgfqpoint{1.199629in}{2.128880in}}{\pgfqpoint{1.207529in}{2.132153in}}{\pgfqpoint{1.213353in}{2.137977in}}%
\pgfpathcurveto{\pgfqpoint{1.219177in}{2.143801in}}{\pgfqpoint{1.222449in}{2.151701in}}{\pgfqpoint{1.222449in}{2.159937in}}%
\pgfpathcurveto{\pgfqpoint{1.222449in}{2.168173in}}{\pgfqpoint{1.219177in}{2.176073in}}{\pgfqpoint{1.213353in}{2.181897in}}%
\pgfpathcurveto{\pgfqpoint{1.207529in}{2.187721in}}{\pgfqpoint{1.199629in}{2.190993in}}{\pgfqpoint{1.191393in}{2.190993in}}%
\pgfpathcurveto{\pgfqpoint{1.183156in}{2.190993in}}{\pgfqpoint{1.175256in}{2.187721in}}{\pgfqpoint{1.169432in}{2.181897in}}%
\pgfpathcurveto{\pgfqpoint{1.163608in}{2.176073in}}{\pgfqpoint{1.160336in}{2.168173in}}{\pgfqpoint{1.160336in}{2.159937in}}%
\pgfpathcurveto{\pgfqpoint{1.160336in}{2.151701in}}{\pgfqpoint{1.163608in}{2.143801in}}{\pgfqpoint{1.169432in}{2.137977in}}%
\pgfpathcurveto{\pgfqpoint{1.175256in}{2.132153in}}{\pgfqpoint{1.183156in}{2.128880in}}{\pgfqpoint{1.191393in}{2.128880in}}%
\pgfpathclose%
\pgfusepath{stroke,fill}%
\end{pgfscope}%
\begin{pgfscope}%
\pgfpathrectangle{\pgfqpoint{0.100000in}{0.212622in}}{\pgfqpoint{3.696000in}{3.696000in}}%
\pgfusepath{clip}%
\pgfsetbuttcap%
\pgfsetroundjoin%
\definecolor{currentfill}{rgb}{0.121569,0.466667,0.705882}%
\pgfsetfillcolor{currentfill}%
\pgfsetfillopacity{0.560773}%
\pgfsetlinewidth{1.003750pt}%
\definecolor{currentstroke}{rgb}{0.121569,0.466667,0.705882}%
\pgfsetstrokecolor{currentstroke}%
\pgfsetstrokeopacity{0.560773}%
\pgfsetdash{}{0pt}%
\pgfpathmoveto{\pgfqpoint{1.187636in}{2.129373in}}%
\pgfpathcurveto{\pgfqpoint{1.195872in}{2.129373in}}{\pgfqpoint{1.203772in}{2.132645in}}{\pgfqpoint{1.209596in}{2.138469in}}%
\pgfpathcurveto{\pgfqpoint{1.215420in}{2.144293in}}{\pgfqpoint{1.218692in}{2.152193in}}{\pgfqpoint{1.218692in}{2.160429in}}%
\pgfpathcurveto{\pgfqpoint{1.218692in}{2.168666in}}{\pgfqpoint{1.215420in}{2.176566in}}{\pgfqpoint{1.209596in}{2.182390in}}%
\pgfpathcurveto{\pgfqpoint{1.203772in}{2.188214in}}{\pgfqpoint{1.195872in}{2.191486in}}{\pgfqpoint{1.187636in}{2.191486in}}%
\pgfpathcurveto{\pgfqpoint{1.179400in}{2.191486in}}{\pgfqpoint{1.171500in}{2.188214in}}{\pgfqpoint{1.165676in}{2.182390in}}%
\pgfpathcurveto{\pgfqpoint{1.159852in}{2.176566in}}{\pgfqpoint{1.156579in}{2.168666in}}{\pgfqpoint{1.156579in}{2.160429in}}%
\pgfpathcurveto{\pgfqpoint{1.156579in}{2.152193in}}{\pgfqpoint{1.159852in}{2.144293in}}{\pgfqpoint{1.165676in}{2.138469in}}%
\pgfpathcurveto{\pgfqpoint{1.171500in}{2.132645in}}{\pgfqpoint{1.179400in}{2.129373in}}{\pgfqpoint{1.187636in}{2.129373in}}%
\pgfpathclose%
\pgfusepath{stroke,fill}%
\end{pgfscope}%
\begin{pgfscope}%
\pgfpathrectangle{\pgfqpoint{0.100000in}{0.212622in}}{\pgfqpoint{3.696000in}{3.696000in}}%
\pgfusepath{clip}%
\pgfsetbuttcap%
\pgfsetroundjoin%
\definecolor{currentfill}{rgb}{0.121569,0.466667,0.705882}%
\pgfsetfillcolor{currentfill}%
\pgfsetfillopacity{0.561823}%
\pgfsetlinewidth{1.003750pt}%
\definecolor{currentstroke}{rgb}{0.121569,0.466667,0.705882}%
\pgfsetstrokecolor{currentstroke}%
\pgfsetstrokeopacity{0.561823}%
\pgfsetdash{}{0pt}%
\pgfpathmoveto{\pgfqpoint{3.227773in}{1.770565in}}%
\pgfpathcurveto{\pgfqpoint{3.236009in}{1.770565in}}{\pgfqpoint{3.243909in}{1.773837in}}{\pgfqpoint{3.249733in}{1.779661in}}%
\pgfpathcurveto{\pgfqpoint{3.255557in}{1.785485in}}{\pgfqpoint{3.258830in}{1.793385in}}{\pgfqpoint{3.258830in}{1.801621in}}%
\pgfpathcurveto{\pgfqpoint{3.258830in}{1.809857in}}{\pgfqpoint{3.255557in}{1.817757in}}{\pgfqpoint{3.249733in}{1.823581in}}%
\pgfpathcurveto{\pgfqpoint{3.243909in}{1.829405in}}{\pgfqpoint{3.236009in}{1.832678in}}{\pgfqpoint{3.227773in}{1.832678in}}%
\pgfpathcurveto{\pgfqpoint{3.219537in}{1.832678in}}{\pgfqpoint{3.211637in}{1.829405in}}{\pgfqpoint{3.205813in}{1.823581in}}%
\pgfpathcurveto{\pgfqpoint{3.199989in}{1.817757in}}{\pgfqpoint{3.196717in}{1.809857in}}{\pgfqpoint{3.196717in}{1.801621in}}%
\pgfpathcurveto{\pgfqpoint{3.196717in}{1.793385in}}{\pgfqpoint{3.199989in}{1.785485in}}{\pgfqpoint{3.205813in}{1.779661in}}%
\pgfpathcurveto{\pgfqpoint{3.211637in}{1.773837in}}{\pgfqpoint{3.219537in}{1.770565in}}{\pgfqpoint{3.227773in}{1.770565in}}%
\pgfpathclose%
\pgfusepath{stroke,fill}%
\end{pgfscope}%
\begin{pgfscope}%
\pgfpathrectangle{\pgfqpoint{0.100000in}{0.212622in}}{\pgfqpoint{3.696000in}{3.696000in}}%
\pgfusepath{clip}%
\pgfsetbuttcap%
\pgfsetroundjoin%
\definecolor{currentfill}{rgb}{0.121569,0.466667,0.705882}%
\pgfsetfillcolor{currentfill}%
\pgfsetfillopacity{0.562031}%
\pgfsetlinewidth{1.003750pt}%
\definecolor{currentstroke}{rgb}{0.121569,0.466667,0.705882}%
\pgfsetstrokecolor{currentstroke}%
\pgfsetstrokeopacity{0.562031}%
\pgfsetdash{}{0pt}%
\pgfpathmoveto{\pgfqpoint{1.186146in}{2.129523in}}%
\pgfpathcurveto{\pgfqpoint{1.194382in}{2.129523in}}{\pgfqpoint{1.202282in}{2.132795in}}{\pgfqpoint{1.208106in}{2.138619in}}%
\pgfpathcurveto{\pgfqpoint{1.213930in}{2.144443in}}{\pgfqpoint{1.217202in}{2.152343in}}{\pgfqpoint{1.217202in}{2.160580in}}%
\pgfpathcurveto{\pgfqpoint{1.217202in}{2.168816in}}{\pgfqpoint{1.213930in}{2.176716in}}{\pgfqpoint{1.208106in}{2.182540in}}%
\pgfpathcurveto{\pgfqpoint{1.202282in}{2.188364in}}{\pgfqpoint{1.194382in}{2.191636in}}{\pgfqpoint{1.186146in}{2.191636in}}%
\pgfpathcurveto{\pgfqpoint{1.177910in}{2.191636in}}{\pgfqpoint{1.170010in}{2.188364in}}{\pgfqpoint{1.164186in}{2.182540in}}%
\pgfpathcurveto{\pgfqpoint{1.158362in}{2.176716in}}{\pgfqpoint{1.155089in}{2.168816in}}{\pgfqpoint{1.155089in}{2.160580in}}%
\pgfpathcurveto{\pgfqpoint{1.155089in}{2.152343in}}{\pgfqpoint{1.158362in}{2.144443in}}{\pgfqpoint{1.164186in}{2.138619in}}%
\pgfpathcurveto{\pgfqpoint{1.170010in}{2.132795in}}{\pgfqpoint{1.177910in}{2.129523in}}{\pgfqpoint{1.186146in}{2.129523in}}%
\pgfpathclose%
\pgfusepath{stroke,fill}%
\end{pgfscope}%
\begin{pgfscope}%
\pgfpathrectangle{\pgfqpoint{0.100000in}{0.212622in}}{\pgfqpoint{3.696000in}{3.696000in}}%
\pgfusepath{clip}%
\pgfsetbuttcap%
\pgfsetroundjoin%
\definecolor{currentfill}{rgb}{0.121569,0.466667,0.705882}%
\pgfsetfillcolor{currentfill}%
\pgfsetfillopacity{0.562621}%
\pgfsetlinewidth{1.003750pt}%
\definecolor{currentstroke}{rgb}{0.121569,0.466667,0.705882}%
\pgfsetstrokecolor{currentstroke}%
\pgfsetstrokeopacity{0.562621}%
\pgfsetdash{}{0pt}%
\pgfpathmoveto{\pgfqpoint{1.184292in}{2.129791in}}%
\pgfpathcurveto{\pgfqpoint{1.192528in}{2.129791in}}{\pgfqpoint{1.200428in}{2.133064in}}{\pgfqpoint{1.206252in}{2.138888in}}%
\pgfpathcurveto{\pgfqpoint{1.212076in}{2.144712in}}{\pgfqpoint{1.215349in}{2.152612in}}{\pgfqpoint{1.215349in}{2.160848in}}%
\pgfpathcurveto{\pgfqpoint{1.215349in}{2.169084in}}{\pgfqpoint{1.212076in}{2.176984in}}{\pgfqpoint{1.206252in}{2.182808in}}%
\pgfpathcurveto{\pgfqpoint{1.200428in}{2.188632in}}{\pgfqpoint{1.192528in}{2.191904in}}{\pgfqpoint{1.184292in}{2.191904in}}%
\pgfpathcurveto{\pgfqpoint{1.176056in}{2.191904in}}{\pgfqpoint{1.168156in}{2.188632in}}{\pgfqpoint{1.162332in}{2.182808in}}%
\pgfpathcurveto{\pgfqpoint{1.156508in}{2.176984in}}{\pgfqpoint{1.153236in}{2.169084in}}{\pgfqpoint{1.153236in}{2.160848in}}%
\pgfpathcurveto{\pgfqpoint{1.153236in}{2.152612in}}{\pgfqpoint{1.156508in}{2.144712in}}{\pgfqpoint{1.162332in}{2.138888in}}%
\pgfpathcurveto{\pgfqpoint{1.168156in}{2.133064in}}{\pgfqpoint{1.176056in}{2.129791in}}{\pgfqpoint{1.184292in}{2.129791in}}%
\pgfpathclose%
\pgfusepath{stroke,fill}%
\end{pgfscope}%
\begin{pgfscope}%
\pgfpathrectangle{\pgfqpoint{0.100000in}{0.212622in}}{\pgfqpoint{3.696000in}{3.696000in}}%
\pgfusepath{clip}%
\pgfsetbuttcap%
\pgfsetroundjoin%
\definecolor{currentfill}{rgb}{0.121569,0.466667,0.705882}%
\pgfsetfillcolor{currentfill}%
\pgfsetfillopacity{0.563711}%
\pgfsetlinewidth{1.003750pt}%
\definecolor{currentstroke}{rgb}{0.121569,0.466667,0.705882}%
\pgfsetstrokecolor{currentstroke}%
\pgfsetstrokeopacity{0.563711}%
\pgfsetdash{}{0pt}%
\pgfpathmoveto{\pgfqpoint{3.223839in}{1.771279in}}%
\pgfpathcurveto{\pgfqpoint{3.232076in}{1.771279in}}{\pgfqpoint{3.239976in}{1.774552in}}{\pgfqpoint{3.245800in}{1.780376in}}%
\pgfpathcurveto{\pgfqpoint{3.251624in}{1.786200in}}{\pgfqpoint{3.254896in}{1.794100in}}{\pgfqpoint{3.254896in}{1.802336in}}%
\pgfpathcurveto{\pgfqpoint{3.254896in}{1.810572in}}{\pgfqpoint{3.251624in}{1.818472in}}{\pgfqpoint{3.245800in}{1.824296in}}%
\pgfpathcurveto{\pgfqpoint{3.239976in}{1.830120in}}{\pgfqpoint{3.232076in}{1.833392in}}{\pgfqpoint{3.223839in}{1.833392in}}%
\pgfpathcurveto{\pgfqpoint{3.215603in}{1.833392in}}{\pgfqpoint{3.207703in}{1.830120in}}{\pgfqpoint{3.201879in}{1.824296in}}%
\pgfpathcurveto{\pgfqpoint{3.196055in}{1.818472in}}{\pgfqpoint{3.192783in}{1.810572in}}{\pgfqpoint{3.192783in}{1.802336in}}%
\pgfpathcurveto{\pgfqpoint{3.192783in}{1.794100in}}{\pgfqpoint{3.196055in}{1.786200in}}{\pgfqpoint{3.201879in}{1.780376in}}%
\pgfpathcurveto{\pgfqpoint{3.207703in}{1.774552in}}{\pgfqpoint{3.215603in}{1.771279in}}{\pgfqpoint{3.223839in}{1.771279in}}%
\pgfpathclose%
\pgfusepath{stroke,fill}%
\end{pgfscope}%
\begin{pgfscope}%
\pgfpathrectangle{\pgfqpoint{0.100000in}{0.212622in}}{\pgfqpoint{3.696000in}{3.696000in}}%
\pgfusepath{clip}%
\pgfsetbuttcap%
\pgfsetroundjoin%
\definecolor{currentfill}{rgb}{0.121569,0.466667,0.705882}%
\pgfsetfillcolor{currentfill}%
\pgfsetfillopacity{0.563842}%
\pgfsetlinewidth{1.003750pt}%
\definecolor{currentstroke}{rgb}{0.121569,0.466667,0.705882}%
\pgfsetstrokecolor{currentstroke}%
\pgfsetstrokeopacity{0.563842}%
\pgfsetdash{}{0pt}%
\pgfpathmoveto{\pgfqpoint{1.182108in}{2.129883in}}%
\pgfpathcurveto{\pgfqpoint{1.190344in}{2.129883in}}{\pgfqpoint{1.198244in}{2.133155in}}{\pgfqpoint{1.204068in}{2.138979in}}%
\pgfpathcurveto{\pgfqpoint{1.209892in}{2.144803in}}{\pgfqpoint{1.213165in}{2.152703in}}{\pgfqpoint{1.213165in}{2.160939in}}%
\pgfpathcurveto{\pgfqpoint{1.213165in}{2.169175in}}{\pgfqpoint{1.209892in}{2.177075in}}{\pgfqpoint{1.204068in}{2.182899in}}%
\pgfpathcurveto{\pgfqpoint{1.198244in}{2.188723in}}{\pgfqpoint{1.190344in}{2.191996in}}{\pgfqpoint{1.182108in}{2.191996in}}%
\pgfpathcurveto{\pgfqpoint{1.173872in}{2.191996in}}{\pgfqpoint{1.165972in}{2.188723in}}{\pgfqpoint{1.160148in}{2.182899in}}%
\pgfpathcurveto{\pgfqpoint{1.154324in}{2.177075in}}{\pgfqpoint{1.151052in}{2.169175in}}{\pgfqpoint{1.151052in}{2.160939in}}%
\pgfpathcurveto{\pgfqpoint{1.151052in}{2.152703in}}{\pgfqpoint{1.154324in}{2.144803in}}{\pgfqpoint{1.160148in}{2.138979in}}%
\pgfpathcurveto{\pgfqpoint{1.165972in}{2.133155in}}{\pgfqpoint{1.173872in}{2.129883in}}{\pgfqpoint{1.182108in}{2.129883in}}%
\pgfpathclose%
\pgfusepath{stroke,fill}%
\end{pgfscope}%
\begin{pgfscope}%
\pgfpathrectangle{\pgfqpoint{0.100000in}{0.212622in}}{\pgfqpoint{3.696000in}{3.696000in}}%
\pgfusepath{clip}%
\pgfsetbuttcap%
\pgfsetroundjoin%
\definecolor{currentfill}{rgb}{0.121569,0.466667,0.705882}%
\pgfsetfillcolor{currentfill}%
\pgfsetfillopacity{0.564830}%
\pgfsetlinewidth{1.003750pt}%
\definecolor{currentstroke}{rgb}{0.121569,0.466667,0.705882}%
\pgfsetstrokecolor{currentstroke}%
\pgfsetstrokeopacity{0.564830}%
\pgfsetdash{}{0pt}%
\pgfpathmoveto{\pgfqpoint{1.180140in}{2.129928in}}%
\pgfpathcurveto{\pgfqpoint{1.188376in}{2.129928in}}{\pgfqpoint{1.196276in}{2.133200in}}{\pgfqpoint{1.202100in}{2.139024in}}%
\pgfpathcurveto{\pgfqpoint{1.207924in}{2.144848in}}{\pgfqpoint{1.211196in}{2.152748in}}{\pgfqpoint{1.211196in}{2.160984in}}%
\pgfpathcurveto{\pgfqpoint{1.211196in}{2.169221in}}{\pgfqpoint{1.207924in}{2.177121in}}{\pgfqpoint{1.202100in}{2.182944in}}%
\pgfpathcurveto{\pgfqpoint{1.196276in}{2.188768in}}{\pgfqpoint{1.188376in}{2.192041in}}{\pgfqpoint{1.180140in}{2.192041in}}%
\pgfpathcurveto{\pgfqpoint{1.171904in}{2.192041in}}{\pgfqpoint{1.164004in}{2.188768in}}{\pgfqpoint{1.158180in}{2.182944in}}%
\pgfpathcurveto{\pgfqpoint{1.152356in}{2.177121in}}{\pgfqpoint{1.149083in}{2.169221in}}{\pgfqpoint{1.149083in}{2.160984in}}%
\pgfpathcurveto{\pgfqpoint{1.149083in}{2.152748in}}{\pgfqpoint{1.152356in}{2.144848in}}{\pgfqpoint{1.158180in}{2.139024in}}%
\pgfpathcurveto{\pgfqpoint{1.164004in}{2.133200in}}{\pgfqpoint{1.171904in}{2.129928in}}{\pgfqpoint{1.180140in}{2.129928in}}%
\pgfpathclose%
\pgfusepath{stroke,fill}%
\end{pgfscope}%
\begin{pgfscope}%
\pgfpathrectangle{\pgfqpoint{0.100000in}{0.212622in}}{\pgfqpoint{3.696000in}{3.696000in}}%
\pgfusepath{clip}%
\pgfsetbuttcap%
\pgfsetroundjoin%
\definecolor{currentfill}{rgb}{0.121569,0.466667,0.705882}%
\pgfsetfillcolor{currentfill}%
\pgfsetfillopacity{0.564838}%
\pgfsetlinewidth{1.003750pt}%
\definecolor{currentstroke}{rgb}{0.121569,0.466667,0.705882}%
\pgfsetstrokecolor{currentstroke}%
\pgfsetstrokeopacity{0.564838}%
\pgfsetdash{}{0pt}%
\pgfpathmoveto{\pgfqpoint{3.222550in}{1.771452in}}%
\pgfpathcurveto{\pgfqpoint{3.230787in}{1.771452in}}{\pgfqpoint{3.238687in}{1.774725in}}{\pgfqpoint{3.244511in}{1.780548in}}%
\pgfpathcurveto{\pgfqpoint{3.250335in}{1.786372in}}{\pgfqpoint{3.253607in}{1.794272in}}{\pgfqpoint{3.253607in}{1.802509in}}%
\pgfpathcurveto{\pgfqpoint{3.253607in}{1.810745in}}{\pgfqpoint{3.250335in}{1.818645in}}{\pgfqpoint{3.244511in}{1.824469in}}%
\pgfpathcurveto{\pgfqpoint{3.238687in}{1.830293in}}{\pgfqpoint{3.230787in}{1.833565in}}{\pgfqpoint{3.222550in}{1.833565in}}%
\pgfpathcurveto{\pgfqpoint{3.214314in}{1.833565in}}{\pgfqpoint{3.206414in}{1.830293in}}{\pgfqpoint{3.200590in}{1.824469in}}%
\pgfpathcurveto{\pgfqpoint{3.194766in}{1.818645in}}{\pgfqpoint{3.191494in}{1.810745in}}{\pgfqpoint{3.191494in}{1.802509in}}%
\pgfpathcurveto{\pgfqpoint{3.191494in}{1.794272in}}{\pgfqpoint{3.194766in}{1.786372in}}{\pgfqpoint{3.200590in}{1.780548in}}%
\pgfpathcurveto{\pgfqpoint{3.206414in}{1.774725in}}{\pgfqpoint{3.214314in}{1.771452in}}{\pgfqpoint{3.222550in}{1.771452in}}%
\pgfpathclose%
\pgfusepath{stroke,fill}%
\end{pgfscope}%
\begin{pgfscope}%
\pgfpathrectangle{\pgfqpoint{0.100000in}{0.212622in}}{\pgfqpoint{3.696000in}{3.696000in}}%
\pgfusepath{clip}%
\pgfsetbuttcap%
\pgfsetroundjoin%
\definecolor{currentfill}{rgb}{0.121569,0.466667,0.705882}%
\pgfsetfillcolor{currentfill}%
\pgfsetfillopacity{0.566380}%
\pgfsetlinewidth{1.003750pt}%
\definecolor{currentstroke}{rgb}{0.121569,0.466667,0.705882}%
\pgfsetstrokecolor{currentstroke}%
\pgfsetstrokeopacity{0.566380}%
\pgfsetdash{}{0pt}%
\pgfpathmoveto{\pgfqpoint{3.219198in}{1.772122in}}%
\pgfpathcurveto{\pgfqpoint{3.227434in}{1.772122in}}{\pgfqpoint{3.235334in}{1.775394in}}{\pgfqpoint{3.241158in}{1.781218in}}%
\pgfpathcurveto{\pgfqpoint{3.246982in}{1.787042in}}{\pgfqpoint{3.250254in}{1.794942in}}{\pgfqpoint{3.250254in}{1.803179in}}%
\pgfpathcurveto{\pgfqpoint{3.250254in}{1.811415in}}{\pgfqpoint{3.246982in}{1.819315in}}{\pgfqpoint{3.241158in}{1.825139in}}%
\pgfpathcurveto{\pgfqpoint{3.235334in}{1.830963in}}{\pgfqpoint{3.227434in}{1.834235in}}{\pgfqpoint{3.219198in}{1.834235in}}%
\pgfpathcurveto{\pgfqpoint{3.210962in}{1.834235in}}{\pgfqpoint{3.203061in}{1.830963in}}{\pgfqpoint{3.197238in}{1.825139in}}%
\pgfpathcurveto{\pgfqpoint{3.191414in}{1.819315in}}{\pgfqpoint{3.188141in}{1.811415in}}{\pgfqpoint{3.188141in}{1.803179in}}%
\pgfpathcurveto{\pgfqpoint{3.188141in}{1.794942in}}{\pgfqpoint{3.191414in}{1.787042in}}{\pgfqpoint{3.197238in}{1.781218in}}%
\pgfpathcurveto{\pgfqpoint{3.203061in}{1.775394in}}{\pgfqpoint{3.210962in}{1.772122in}}{\pgfqpoint{3.219198in}{1.772122in}}%
\pgfpathclose%
\pgfusepath{stroke,fill}%
\end{pgfscope}%
\begin{pgfscope}%
\pgfpathrectangle{\pgfqpoint{0.100000in}{0.212622in}}{\pgfqpoint{3.696000in}{3.696000in}}%
\pgfusepath{clip}%
\pgfsetbuttcap%
\pgfsetroundjoin%
\definecolor{currentfill}{rgb}{0.121569,0.466667,0.705882}%
\pgfsetfillcolor{currentfill}%
\pgfsetfillopacity{0.566492}%
\pgfsetlinewidth{1.003750pt}%
\definecolor{currentstroke}{rgb}{0.121569,0.466667,0.705882}%
\pgfsetstrokecolor{currentstroke}%
\pgfsetstrokeopacity{0.566492}%
\pgfsetdash{}{0pt}%
\pgfpathmoveto{\pgfqpoint{1.175445in}{2.130348in}}%
\pgfpathcurveto{\pgfqpoint{1.183682in}{2.130348in}}{\pgfqpoint{1.191582in}{2.133620in}}{\pgfqpoint{1.197405in}{2.139444in}}%
\pgfpathcurveto{\pgfqpoint{1.203229in}{2.145268in}}{\pgfqpoint{1.206502in}{2.153168in}}{\pgfqpoint{1.206502in}{2.161404in}}%
\pgfpathcurveto{\pgfqpoint{1.206502in}{2.169641in}}{\pgfqpoint{1.203229in}{2.177541in}}{\pgfqpoint{1.197405in}{2.183365in}}%
\pgfpathcurveto{\pgfqpoint{1.191582in}{2.189189in}}{\pgfqpoint{1.183682in}{2.192461in}}{\pgfqpoint{1.175445in}{2.192461in}}%
\pgfpathcurveto{\pgfqpoint{1.167209in}{2.192461in}}{\pgfqpoint{1.159309in}{2.189189in}}{\pgfqpoint{1.153485in}{2.183365in}}%
\pgfpathcurveto{\pgfqpoint{1.147661in}{2.177541in}}{\pgfqpoint{1.144389in}{2.169641in}}{\pgfqpoint{1.144389in}{2.161404in}}%
\pgfpathcurveto{\pgfqpoint{1.144389in}{2.153168in}}{\pgfqpoint{1.147661in}{2.145268in}}{\pgfqpoint{1.153485in}{2.139444in}}%
\pgfpathcurveto{\pgfqpoint{1.159309in}{2.133620in}}{\pgfqpoint{1.167209in}{2.130348in}}{\pgfqpoint{1.175445in}{2.130348in}}%
\pgfpathclose%
\pgfusepath{stroke,fill}%
\end{pgfscope}%
\begin{pgfscope}%
\pgfpathrectangle{\pgfqpoint{0.100000in}{0.212622in}}{\pgfqpoint{3.696000in}{3.696000in}}%
\pgfusepath{clip}%
\pgfsetbuttcap%
\pgfsetroundjoin%
\definecolor{currentfill}{rgb}{0.121569,0.466667,0.705882}%
\pgfsetfillcolor{currentfill}%
\pgfsetfillopacity{0.567195}%
\pgfsetlinewidth{1.003750pt}%
\definecolor{currentstroke}{rgb}{0.121569,0.466667,0.705882}%
\pgfsetstrokecolor{currentstroke}%
\pgfsetstrokeopacity{0.567195}%
\pgfsetdash{}{0pt}%
\pgfpathmoveto{\pgfqpoint{3.217019in}{1.772753in}}%
\pgfpathcurveto{\pgfqpoint{3.225255in}{1.772753in}}{\pgfqpoint{3.233155in}{1.776026in}}{\pgfqpoint{3.238979in}{1.781850in}}%
\pgfpathcurveto{\pgfqpoint{3.244803in}{1.787674in}}{\pgfqpoint{3.248076in}{1.795574in}}{\pgfqpoint{3.248076in}{1.803810in}}%
\pgfpathcurveto{\pgfqpoint{3.248076in}{1.812046in}}{\pgfqpoint{3.244803in}{1.819946in}}{\pgfqpoint{3.238979in}{1.825770in}}%
\pgfpathcurveto{\pgfqpoint{3.233155in}{1.831594in}}{\pgfqpoint{3.225255in}{1.834866in}}{\pgfqpoint{3.217019in}{1.834866in}}%
\pgfpathcurveto{\pgfqpoint{3.208783in}{1.834866in}}{\pgfqpoint{3.200883in}{1.831594in}}{\pgfqpoint{3.195059in}{1.825770in}}%
\pgfpathcurveto{\pgfqpoint{3.189235in}{1.819946in}}{\pgfqpoint{3.185963in}{1.812046in}}{\pgfqpoint{3.185963in}{1.803810in}}%
\pgfpathcurveto{\pgfqpoint{3.185963in}{1.795574in}}{\pgfqpoint{3.189235in}{1.787674in}}{\pgfqpoint{3.195059in}{1.781850in}}%
\pgfpathcurveto{\pgfqpoint{3.200883in}{1.776026in}}{\pgfqpoint{3.208783in}{1.772753in}}{\pgfqpoint{3.217019in}{1.772753in}}%
\pgfpathclose%
\pgfusepath{stroke,fill}%
\end{pgfscope}%
\begin{pgfscope}%
\pgfpathrectangle{\pgfqpoint{0.100000in}{0.212622in}}{\pgfqpoint{3.696000in}{3.696000in}}%
\pgfusepath{clip}%
\pgfsetbuttcap%
\pgfsetroundjoin%
\definecolor{currentfill}{rgb}{0.121569,0.466667,0.705882}%
\pgfsetfillcolor{currentfill}%
\pgfsetfillopacity{0.568093}%
\pgfsetlinewidth{1.003750pt}%
\definecolor{currentstroke}{rgb}{0.121569,0.466667,0.705882}%
\pgfsetstrokecolor{currentstroke}%
\pgfsetstrokeopacity{0.568093}%
\pgfsetdash{}{0pt}%
\pgfpathmoveto{\pgfqpoint{1.174005in}{2.130648in}}%
\pgfpathcurveto{\pgfqpoint{1.182242in}{2.130648in}}{\pgfqpoint{1.190142in}{2.133920in}}{\pgfqpoint{1.195966in}{2.139744in}}%
\pgfpathcurveto{\pgfqpoint{1.201790in}{2.145568in}}{\pgfqpoint{1.205062in}{2.153468in}}{\pgfqpoint{1.205062in}{2.161704in}}%
\pgfpathcurveto{\pgfqpoint{1.205062in}{2.169941in}}{\pgfqpoint{1.201790in}{2.177841in}}{\pgfqpoint{1.195966in}{2.183665in}}%
\pgfpathcurveto{\pgfqpoint{1.190142in}{2.189489in}}{\pgfqpoint{1.182242in}{2.192761in}}{\pgfqpoint{1.174005in}{2.192761in}}%
\pgfpathcurveto{\pgfqpoint{1.165769in}{2.192761in}}{\pgfqpoint{1.157869in}{2.189489in}}{\pgfqpoint{1.152045in}{2.183665in}}%
\pgfpathcurveto{\pgfqpoint{1.146221in}{2.177841in}}{\pgfqpoint{1.142949in}{2.169941in}}{\pgfqpoint{1.142949in}{2.161704in}}%
\pgfpathcurveto{\pgfqpoint{1.142949in}{2.153468in}}{\pgfqpoint{1.146221in}{2.145568in}}{\pgfqpoint{1.152045in}{2.139744in}}%
\pgfpathcurveto{\pgfqpoint{1.157869in}{2.133920in}}{\pgfqpoint{1.165769in}{2.130648in}}{\pgfqpoint{1.174005in}{2.130648in}}%
\pgfpathclose%
\pgfusepath{stroke,fill}%
\end{pgfscope}%
\begin{pgfscope}%
\pgfpathrectangle{\pgfqpoint{0.100000in}{0.212622in}}{\pgfqpoint{3.696000in}{3.696000in}}%
\pgfusepath{clip}%
\pgfsetbuttcap%
\pgfsetroundjoin%
\definecolor{currentfill}{rgb}{0.121569,0.466667,0.705882}%
\pgfsetfillcolor{currentfill}%
\pgfsetfillopacity{0.568714}%
\pgfsetlinewidth{1.003750pt}%
\definecolor{currentstroke}{rgb}{0.121569,0.466667,0.705882}%
\pgfsetstrokecolor{currentstroke}%
\pgfsetstrokeopacity{0.568714}%
\pgfsetdash{}{0pt}%
\pgfpathmoveto{\pgfqpoint{3.215695in}{1.772935in}}%
\pgfpathcurveto{\pgfqpoint{3.223931in}{1.772935in}}{\pgfqpoint{3.231831in}{1.776208in}}{\pgfqpoint{3.237655in}{1.782032in}}%
\pgfpathcurveto{\pgfqpoint{3.243479in}{1.787856in}}{\pgfqpoint{3.246752in}{1.795756in}}{\pgfqpoint{3.246752in}{1.803992in}}%
\pgfpathcurveto{\pgfqpoint{3.246752in}{1.812228in}}{\pgfqpoint{3.243479in}{1.820128in}}{\pgfqpoint{3.237655in}{1.825952in}}%
\pgfpathcurveto{\pgfqpoint{3.231831in}{1.831776in}}{\pgfqpoint{3.223931in}{1.835048in}}{\pgfqpoint{3.215695in}{1.835048in}}%
\pgfpathcurveto{\pgfqpoint{3.207459in}{1.835048in}}{\pgfqpoint{3.199559in}{1.831776in}}{\pgfqpoint{3.193735in}{1.825952in}}%
\pgfpathcurveto{\pgfqpoint{3.187911in}{1.820128in}}{\pgfqpoint{3.184639in}{1.812228in}}{\pgfqpoint{3.184639in}{1.803992in}}%
\pgfpathcurveto{\pgfqpoint{3.184639in}{1.795756in}}{\pgfqpoint{3.187911in}{1.787856in}}{\pgfqpoint{3.193735in}{1.782032in}}%
\pgfpathcurveto{\pgfqpoint{3.199559in}{1.776208in}}{\pgfqpoint{3.207459in}{1.772935in}}{\pgfqpoint{3.215695in}{1.772935in}}%
\pgfpathclose%
\pgfusepath{stroke,fill}%
\end{pgfscope}%
\begin{pgfscope}%
\pgfpathrectangle{\pgfqpoint{0.100000in}{0.212622in}}{\pgfqpoint{3.696000in}{3.696000in}}%
\pgfusepath{clip}%
\pgfsetbuttcap%
\pgfsetroundjoin%
\definecolor{currentfill}{rgb}{0.121569,0.466667,0.705882}%
\pgfsetfillcolor{currentfill}%
\pgfsetfillopacity{0.569174}%
\pgfsetlinewidth{1.003750pt}%
\definecolor{currentstroke}{rgb}{0.121569,0.466667,0.705882}%
\pgfsetstrokecolor{currentstroke}%
\pgfsetstrokeopacity{0.569174}%
\pgfsetdash{}{0pt}%
\pgfpathmoveto{\pgfqpoint{1.170733in}{2.131121in}}%
\pgfpathcurveto{\pgfqpoint{1.178969in}{2.131121in}}{\pgfqpoint{1.186869in}{2.134394in}}{\pgfqpoint{1.192693in}{2.140218in}}%
\pgfpathcurveto{\pgfqpoint{1.198517in}{2.146042in}}{\pgfqpoint{1.201789in}{2.153942in}}{\pgfqpoint{1.201789in}{2.162178in}}%
\pgfpathcurveto{\pgfqpoint{1.201789in}{2.170414in}}{\pgfqpoint{1.198517in}{2.178314in}}{\pgfqpoint{1.192693in}{2.184138in}}%
\pgfpathcurveto{\pgfqpoint{1.186869in}{2.189962in}}{\pgfqpoint{1.178969in}{2.193234in}}{\pgfqpoint{1.170733in}{2.193234in}}%
\pgfpathcurveto{\pgfqpoint{1.162496in}{2.193234in}}{\pgfqpoint{1.154596in}{2.189962in}}{\pgfqpoint{1.148772in}{2.184138in}}%
\pgfpathcurveto{\pgfqpoint{1.142948in}{2.178314in}}{\pgfqpoint{1.139676in}{2.170414in}}{\pgfqpoint{1.139676in}{2.162178in}}%
\pgfpathcurveto{\pgfqpoint{1.139676in}{2.153942in}}{\pgfqpoint{1.142948in}{2.146042in}}{\pgfqpoint{1.148772in}{2.140218in}}%
\pgfpathcurveto{\pgfqpoint{1.154596in}{2.134394in}}{\pgfqpoint{1.162496in}{2.131121in}}{\pgfqpoint{1.170733in}{2.131121in}}%
\pgfpathclose%
\pgfusepath{stroke,fill}%
\end{pgfscope}%
\begin{pgfscope}%
\pgfpathrectangle{\pgfqpoint{0.100000in}{0.212622in}}{\pgfqpoint{3.696000in}{3.696000in}}%
\pgfusepath{clip}%
\pgfsetbuttcap%
\pgfsetroundjoin%
\definecolor{currentfill}{rgb}{0.121569,0.466667,0.705882}%
\pgfsetfillcolor{currentfill}%
\pgfsetfillopacity{0.569944}%
\pgfsetlinewidth{1.003750pt}%
\definecolor{currentstroke}{rgb}{0.121569,0.466667,0.705882}%
\pgfsetstrokecolor{currentstroke}%
\pgfsetstrokeopacity{0.569944}%
\pgfsetdash{}{0pt}%
\pgfpathmoveto{\pgfqpoint{1.172918in}{2.132033in}}%
\pgfpathcurveto{\pgfqpoint{1.181154in}{2.132033in}}{\pgfqpoint{1.189054in}{2.135305in}}{\pgfqpoint{1.194878in}{2.141129in}}%
\pgfpathcurveto{\pgfqpoint{1.200702in}{2.146953in}}{\pgfqpoint{1.203974in}{2.154853in}}{\pgfqpoint{1.203974in}{2.163090in}}%
\pgfpathcurveto{\pgfqpoint{1.203974in}{2.171326in}}{\pgfqpoint{1.200702in}{2.179226in}}{\pgfqpoint{1.194878in}{2.185050in}}%
\pgfpathcurveto{\pgfqpoint{1.189054in}{2.190874in}}{\pgfqpoint{1.181154in}{2.194146in}}{\pgfqpoint{1.172918in}{2.194146in}}%
\pgfpathcurveto{\pgfqpoint{1.164682in}{2.194146in}}{\pgfqpoint{1.156781in}{2.190874in}}{\pgfqpoint{1.150958in}{2.185050in}}%
\pgfpathcurveto{\pgfqpoint{1.145134in}{2.179226in}}{\pgfqpoint{1.141861in}{2.171326in}}{\pgfqpoint{1.141861in}{2.163090in}}%
\pgfpathcurveto{\pgfqpoint{1.141861in}{2.154853in}}{\pgfqpoint{1.145134in}{2.146953in}}{\pgfqpoint{1.150958in}{2.141129in}}%
\pgfpathcurveto{\pgfqpoint{1.156781in}{2.135305in}}{\pgfqpoint{1.164682in}{2.132033in}}{\pgfqpoint{1.172918in}{2.132033in}}%
\pgfpathclose%
\pgfusepath{stroke,fill}%
\end{pgfscope}%
\begin{pgfscope}%
\pgfpathrectangle{\pgfqpoint{0.100000in}{0.212622in}}{\pgfqpoint{3.696000in}{3.696000in}}%
\pgfusepath{clip}%
\pgfsetbuttcap%
\pgfsetroundjoin%
\definecolor{currentfill}{rgb}{0.121569,0.466667,0.705882}%
\pgfsetfillcolor{currentfill}%
\pgfsetfillopacity{0.570205}%
\pgfsetlinewidth{1.003750pt}%
\definecolor{currentstroke}{rgb}{0.121569,0.466667,0.705882}%
\pgfsetstrokecolor{currentstroke}%
\pgfsetstrokeopacity{0.570205}%
\pgfsetdash{}{0pt}%
\pgfpathmoveto{\pgfqpoint{3.212590in}{1.773564in}}%
\pgfpathcurveto{\pgfqpoint{3.220826in}{1.773564in}}{\pgfqpoint{3.228727in}{1.776836in}}{\pgfqpoint{3.234550in}{1.782660in}}%
\pgfpathcurveto{\pgfqpoint{3.240374in}{1.788484in}}{\pgfqpoint{3.243647in}{1.796384in}}{\pgfqpoint{3.243647in}{1.804620in}}%
\pgfpathcurveto{\pgfqpoint{3.243647in}{1.812857in}}{\pgfqpoint{3.240374in}{1.820757in}}{\pgfqpoint{3.234550in}{1.826581in}}%
\pgfpathcurveto{\pgfqpoint{3.228727in}{1.832404in}}{\pgfqpoint{3.220826in}{1.835677in}}{\pgfqpoint{3.212590in}{1.835677in}}%
\pgfpathcurveto{\pgfqpoint{3.204354in}{1.835677in}}{\pgfqpoint{3.196454in}{1.832404in}}{\pgfqpoint{3.190630in}{1.826581in}}%
\pgfpathcurveto{\pgfqpoint{3.184806in}{1.820757in}}{\pgfqpoint{3.181534in}{1.812857in}}{\pgfqpoint{3.181534in}{1.804620in}}%
\pgfpathcurveto{\pgfqpoint{3.181534in}{1.796384in}}{\pgfqpoint{3.184806in}{1.788484in}}{\pgfqpoint{3.190630in}{1.782660in}}%
\pgfpathcurveto{\pgfqpoint{3.196454in}{1.776836in}}{\pgfqpoint{3.204354in}{1.773564in}}{\pgfqpoint{3.212590in}{1.773564in}}%
\pgfpathclose%
\pgfusepath{stroke,fill}%
\end{pgfscope}%
\begin{pgfscope}%
\pgfpathrectangle{\pgfqpoint{0.100000in}{0.212622in}}{\pgfqpoint{3.696000in}{3.696000in}}%
\pgfusepath{clip}%
\pgfsetbuttcap%
\pgfsetroundjoin%
\definecolor{currentfill}{rgb}{0.121569,0.466667,0.705882}%
\pgfsetfillcolor{currentfill}%
\pgfsetfillopacity{0.571380}%
\pgfsetlinewidth{1.003750pt}%
\definecolor{currentstroke}{rgb}{0.121569,0.466667,0.705882}%
\pgfsetstrokecolor{currentstroke}%
\pgfsetstrokeopacity{0.571380}%
\pgfsetdash{}{0pt}%
\pgfpathmoveto{\pgfqpoint{1.169360in}{2.132345in}}%
\pgfpathcurveto{\pgfqpoint{1.177596in}{2.132345in}}{\pgfqpoint{1.185496in}{2.135617in}}{\pgfqpoint{1.191320in}{2.141441in}}%
\pgfpathcurveto{\pgfqpoint{1.197144in}{2.147265in}}{\pgfqpoint{1.200417in}{2.155165in}}{\pgfqpoint{1.200417in}{2.163401in}}%
\pgfpathcurveto{\pgfqpoint{1.200417in}{2.171637in}}{\pgfqpoint{1.197144in}{2.179538in}}{\pgfqpoint{1.191320in}{2.185361in}}%
\pgfpathcurveto{\pgfqpoint{1.185496in}{2.191185in}}{\pgfqpoint{1.177596in}{2.194458in}}{\pgfqpoint{1.169360in}{2.194458in}}%
\pgfpathcurveto{\pgfqpoint{1.161124in}{2.194458in}}{\pgfqpoint{1.153224in}{2.191185in}}{\pgfqpoint{1.147400in}{2.185361in}}%
\pgfpathcurveto{\pgfqpoint{1.141576in}{2.179538in}}{\pgfqpoint{1.138304in}{2.171637in}}{\pgfqpoint{1.138304in}{2.163401in}}%
\pgfpathcurveto{\pgfqpoint{1.138304in}{2.155165in}}{\pgfqpoint{1.141576in}{2.147265in}}{\pgfqpoint{1.147400in}{2.141441in}}%
\pgfpathcurveto{\pgfqpoint{1.153224in}{2.135617in}}{\pgfqpoint{1.161124in}{2.132345in}}{\pgfqpoint{1.169360in}{2.132345in}}%
\pgfpathclose%
\pgfusepath{stroke,fill}%
\end{pgfscope}%
\begin{pgfscope}%
\pgfpathrectangle{\pgfqpoint{0.100000in}{0.212622in}}{\pgfqpoint{3.696000in}{3.696000in}}%
\pgfusepath{clip}%
\pgfsetbuttcap%
\pgfsetroundjoin%
\definecolor{currentfill}{rgb}{0.121569,0.466667,0.705882}%
\pgfsetfillcolor{currentfill}%
\pgfsetfillopacity{0.571716}%
\pgfsetlinewidth{1.003750pt}%
\definecolor{currentstroke}{rgb}{0.121569,0.466667,0.705882}%
\pgfsetstrokecolor{currentstroke}%
\pgfsetstrokeopacity{0.571716}%
\pgfsetdash{}{0pt}%
\pgfpathmoveto{\pgfqpoint{3.208879in}{1.774318in}}%
\pgfpathcurveto{\pgfqpoint{3.217115in}{1.774318in}}{\pgfqpoint{3.225015in}{1.777590in}}{\pgfqpoint{3.230839in}{1.783414in}}%
\pgfpathcurveto{\pgfqpoint{3.236663in}{1.789238in}}{\pgfqpoint{3.239935in}{1.797138in}}{\pgfqpoint{3.239935in}{1.805375in}}%
\pgfpathcurveto{\pgfqpoint{3.239935in}{1.813611in}}{\pgfqpoint{3.236663in}{1.821511in}}{\pgfqpoint{3.230839in}{1.827335in}}%
\pgfpathcurveto{\pgfqpoint{3.225015in}{1.833159in}}{\pgfqpoint{3.217115in}{1.836431in}}{\pgfqpoint{3.208879in}{1.836431in}}%
\pgfpathcurveto{\pgfqpoint{3.200643in}{1.836431in}}{\pgfqpoint{3.192743in}{1.833159in}}{\pgfqpoint{3.186919in}{1.827335in}}%
\pgfpathcurveto{\pgfqpoint{3.181095in}{1.821511in}}{\pgfqpoint{3.177822in}{1.813611in}}{\pgfqpoint{3.177822in}{1.805375in}}%
\pgfpathcurveto{\pgfqpoint{3.177822in}{1.797138in}}{\pgfqpoint{3.181095in}{1.789238in}}{\pgfqpoint{3.186919in}{1.783414in}}%
\pgfpathcurveto{\pgfqpoint{3.192743in}{1.777590in}}{\pgfqpoint{3.200643in}{1.774318in}}{\pgfqpoint{3.208879in}{1.774318in}}%
\pgfpathclose%
\pgfusepath{stroke,fill}%
\end{pgfscope}%
\begin{pgfscope}%
\pgfpathrectangle{\pgfqpoint{0.100000in}{0.212622in}}{\pgfqpoint{3.696000in}{3.696000in}}%
\pgfusepath{clip}%
\pgfsetbuttcap%
\pgfsetroundjoin%
\definecolor{currentfill}{rgb}{0.121569,0.466667,0.705882}%
\pgfsetfillcolor{currentfill}%
\pgfsetfillopacity{0.573928}%
\pgfsetlinewidth{1.003750pt}%
\definecolor{currentstroke}{rgb}{0.121569,0.466667,0.705882}%
\pgfsetstrokecolor{currentstroke}%
\pgfsetstrokeopacity{0.573928}%
\pgfsetdash{}{0pt}%
\pgfpathmoveto{\pgfqpoint{3.205486in}{1.774585in}}%
\pgfpathcurveto{\pgfqpoint{3.213722in}{1.774585in}}{\pgfqpoint{3.221622in}{1.777857in}}{\pgfqpoint{3.227446in}{1.783681in}}%
\pgfpathcurveto{\pgfqpoint{3.233270in}{1.789505in}}{\pgfqpoint{3.236542in}{1.797405in}}{\pgfqpoint{3.236542in}{1.805641in}}%
\pgfpathcurveto{\pgfqpoint{3.236542in}{1.813878in}}{\pgfqpoint{3.233270in}{1.821778in}}{\pgfqpoint{3.227446in}{1.827602in}}%
\pgfpathcurveto{\pgfqpoint{3.221622in}{1.833425in}}{\pgfqpoint{3.213722in}{1.836698in}}{\pgfqpoint{3.205486in}{1.836698in}}%
\pgfpathcurveto{\pgfqpoint{3.197250in}{1.836698in}}{\pgfqpoint{3.189349in}{1.833425in}}{\pgfqpoint{3.183526in}{1.827602in}}%
\pgfpathcurveto{\pgfqpoint{3.177702in}{1.821778in}}{\pgfqpoint{3.174429in}{1.813878in}}{\pgfqpoint{3.174429in}{1.805641in}}%
\pgfpathcurveto{\pgfqpoint{3.174429in}{1.797405in}}{\pgfqpoint{3.177702in}{1.789505in}}{\pgfqpoint{3.183526in}{1.783681in}}%
\pgfpathcurveto{\pgfqpoint{3.189349in}{1.777857in}}{\pgfqpoint{3.197250in}{1.774585in}}{\pgfqpoint{3.205486in}{1.774585in}}%
\pgfpathclose%
\pgfusepath{stroke,fill}%
\end{pgfscope}%
\begin{pgfscope}%
\pgfpathrectangle{\pgfqpoint{0.100000in}{0.212622in}}{\pgfqpoint{3.696000in}{3.696000in}}%
\pgfusepath{clip}%
\pgfsetbuttcap%
\pgfsetroundjoin%
\definecolor{currentfill}{rgb}{0.121569,0.466667,0.705882}%
\pgfsetfillcolor{currentfill}%
\pgfsetfillopacity{0.573967}%
\pgfsetlinewidth{1.003750pt}%
\definecolor{currentstroke}{rgb}{0.121569,0.466667,0.705882}%
\pgfsetstrokecolor{currentstroke}%
\pgfsetstrokeopacity{0.573967}%
\pgfsetdash{}{0pt}%
\pgfpathmoveto{\pgfqpoint{1.162814in}{2.132807in}}%
\pgfpathcurveto{\pgfqpoint{1.171051in}{2.132807in}}{\pgfqpoint{1.178951in}{2.136080in}}{\pgfqpoint{1.184775in}{2.141904in}}%
\pgfpathcurveto{\pgfqpoint{1.190599in}{2.147728in}}{\pgfqpoint{1.193871in}{2.155628in}}{\pgfqpoint{1.193871in}{2.163864in}}%
\pgfpathcurveto{\pgfqpoint{1.193871in}{2.172100in}}{\pgfqpoint{1.190599in}{2.180000in}}{\pgfqpoint{1.184775in}{2.185824in}}%
\pgfpathcurveto{\pgfqpoint{1.178951in}{2.191648in}}{\pgfqpoint{1.171051in}{2.194920in}}{\pgfqpoint{1.162814in}{2.194920in}}%
\pgfpathcurveto{\pgfqpoint{1.154578in}{2.194920in}}{\pgfqpoint{1.146678in}{2.191648in}}{\pgfqpoint{1.140854in}{2.185824in}}%
\pgfpathcurveto{\pgfqpoint{1.135030in}{2.180000in}}{\pgfqpoint{1.131758in}{2.172100in}}{\pgfqpoint{1.131758in}{2.163864in}}%
\pgfpathcurveto{\pgfqpoint{1.131758in}{2.155628in}}{\pgfqpoint{1.135030in}{2.147728in}}{\pgfqpoint{1.140854in}{2.141904in}}%
\pgfpathcurveto{\pgfqpoint{1.146678in}{2.136080in}}{\pgfqpoint{1.154578in}{2.132807in}}{\pgfqpoint{1.162814in}{2.132807in}}%
\pgfpathclose%
\pgfusepath{stroke,fill}%
\end{pgfscope}%
\begin{pgfscope}%
\pgfpathrectangle{\pgfqpoint{0.100000in}{0.212622in}}{\pgfqpoint{3.696000in}{3.696000in}}%
\pgfusepath{clip}%
\pgfsetbuttcap%
\pgfsetroundjoin%
\definecolor{currentfill}{rgb}{0.121569,0.466667,0.705882}%
\pgfsetfillcolor{currentfill}%
\pgfsetfillopacity{0.575154}%
\pgfsetlinewidth{1.003750pt}%
\definecolor{currentstroke}{rgb}{0.121569,0.466667,0.705882}%
\pgfsetstrokecolor{currentstroke}%
\pgfsetstrokeopacity{0.575154}%
\pgfsetdash{}{0pt}%
\pgfpathmoveto{\pgfqpoint{3.203650in}{1.774764in}}%
\pgfpathcurveto{\pgfqpoint{3.211886in}{1.774764in}}{\pgfqpoint{3.219786in}{1.778037in}}{\pgfqpoint{3.225610in}{1.783861in}}%
\pgfpathcurveto{\pgfqpoint{3.231434in}{1.789685in}}{\pgfqpoint{3.234707in}{1.797585in}}{\pgfqpoint{3.234707in}{1.805821in}}%
\pgfpathcurveto{\pgfqpoint{3.234707in}{1.814057in}}{\pgfqpoint{3.231434in}{1.821957in}}{\pgfqpoint{3.225610in}{1.827781in}}%
\pgfpathcurveto{\pgfqpoint{3.219786in}{1.833605in}}{\pgfqpoint{3.211886in}{1.836877in}}{\pgfqpoint{3.203650in}{1.836877in}}%
\pgfpathcurveto{\pgfqpoint{3.195414in}{1.836877in}}{\pgfqpoint{3.187514in}{1.833605in}}{\pgfqpoint{3.181690in}{1.827781in}}%
\pgfpathcurveto{\pgfqpoint{3.175866in}{1.821957in}}{\pgfqpoint{3.172594in}{1.814057in}}{\pgfqpoint{3.172594in}{1.805821in}}%
\pgfpathcurveto{\pgfqpoint{3.172594in}{1.797585in}}{\pgfqpoint{3.175866in}{1.789685in}}{\pgfqpoint{3.181690in}{1.783861in}}%
\pgfpathcurveto{\pgfqpoint{3.187514in}{1.778037in}}{\pgfqpoint{3.195414in}{1.774764in}}{\pgfqpoint{3.203650in}{1.774764in}}%
\pgfpathclose%
\pgfusepath{stroke,fill}%
\end{pgfscope}%
\begin{pgfscope}%
\pgfpathrectangle{\pgfqpoint{0.100000in}{0.212622in}}{\pgfqpoint{3.696000in}{3.696000in}}%
\pgfusepath{clip}%
\pgfsetbuttcap%
\pgfsetroundjoin%
\definecolor{currentfill}{rgb}{0.121569,0.466667,0.705882}%
\pgfsetfillcolor{currentfill}%
\pgfsetfillopacity{0.576402}%
\pgfsetlinewidth{1.003750pt}%
\definecolor{currentstroke}{rgb}{0.121569,0.466667,0.705882}%
\pgfsetstrokecolor{currentstroke}%
\pgfsetstrokeopacity{0.576402}%
\pgfsetdash{}{0pt}%
\pgfpathmoveto{\pgfqpoint{3.200864in}{1.775347in}}%
\pgfpathcurveto{\pgfqpoint{3.209101in}{1.775347in}}{\pgfqpoint{3.217001in}{1.778620in}}{\pgfqpoint{3.222825in}{1.784444in}}%
\pgfpathcurveto{\pgfqpoint{3.228648in}{1.790268in}}{\pgfqpoint{3.231921in}{1.798168in}}{\pgfqpoint{3.231921in}{1.806404in}}%
\pgfpathcurveto{\pgfqpoint{3.231921in}{1.814640in}}{\pgfqpoint{3.228648in}{1.822540in}}{\pgfqpoint{3.222825in}{1.828364in}}%
\pgfpathcurveto{\pgfqpoint{3.217001in}{1.834188in}}{\pgfqpoint{3.209101in}{1.837460in}}{\pgfqpoint{3.200864in}{1.837460in}}%
\pgfpathcurveto{\pgfqpoint{3.192628in}{1.837460in}}{\pgfqpoint{3.184728in}{1.834188in}}{\pgfqpoint{3.178904in}{1.828364in}}%
\pgfpathcurveto{\pgfqpoint{3.173080in}{1.822540in}}{\pgfqpoint{3.169808in}{1.814640in}}{\pgfqpoint{3.169808in}{1.806404in}}%
\pgfpathcurveto{\pgfqpoint{3.169808in}{1.798168in}}{\pgfqpoint{3.173080in}{1.790268in}}{\pgfqpoint{3.178904in}{1.784444in}}%
\pgfpathcurveto{\pgfqpoint{3.184728in}{1.778620in}}{\pgfqpoint{3.192628in}{1.775347in}}{\pgfqpoint{3.200864in}{1.775347in}}%
\pgfpathclose%
\pgfusepath{stroke,fill}%
\end{pgfscope}%
\begin{pgfscope}%
\pgfpathrectangle{\pgfqpoint{0.100000in}{0.212622in}}{\pgfqpoint{3.696000in}{3.696000in}}%
\pgfusepath{clip}%
\pgfsetbuttcap%
\pgfsetroundjoin%
\definecolor{currentfill}{rgb}{0.121569,0.466667,0.705882}%
\pgfsetfillcolor{currentfill}%
\pgfsetfillopacity{0.576733}%
\pgfsetlinewidth{1.003750pt}%
\definecolor{currentstroke}{rgb}{0.121569,0.466667,0.705882}%
\pgfsetstrokecolor{currentstroke}%
\pgfsetstrokeopacity{0.576733}%
\pgfsetdash{}{0pt}%
\pgfpathmoveto{\pgfqpoint{1.158982in}{2.132918in}}%
\pgfpathcurveto{\pgfqpoint{1.167218in}{2.132918in}}{\pgfqpoint{1.175118in}{2.136191in}}{\pgfqpoint{1.180942in}{2.142015in}}%
\pgfpathcurveto{\pgfqpoint{1.186766in}{2.147839in}}{\pgfqpoint{1.190038in}{2.155739in}}{\pgfqpoint{1.190038in}{2.163975in}}%
\pgfpathcurveto{\pgfqpoint{1.190038in}{2.172211in}}{\pgfqpoint{1.186766in}{2.180111in}}{\pgfqpoint{1.180942in}{2.185935in}}%
\pgfpathcurveto{\pgfqpoint{1.175118in}{2.191759in}}{\pgfqpoint{1.167218in}{2.195031in}}{\pgfqpoint{1.158982in}{2.195031in}}%
\pgfpathcurveto{\pgfqpoint{1.150746in}{2.195031in}}{\pgfqpoint{1.142845in}{2.191759in}}{\pgfqpoint{1.137022in}{2.185935in}}%
\pgfpathcurveto{\pgfqpoint{1.131198in}{2.180111in}}{\pgfqpoint{1.127925in}{2.172211in}}{\pgfqpoint{1.127925in}{2.163975in}}%
\pgfpathcurveto{\pgfqpoint{1.127925in}{2.155739in}}{\pgfqpoint{1.131198in}{2.147839in}}{\pgfqpoint{1.137022in}{2.142015in}}%
\pgfpathcurveto{\pgfqpoint{1.142845in}{2.136191in}}{\pgfqpoint{1.150746in}{2.132918in}}{\pgfqpoint{1.158982in}{2.132918in}}%
\pgfpathclose%
\pgfusepath{stroke,fill}%
\end{pgfscope}%
\begin{pgfscope}%
\pgfpathrectangle{\pgfqpoint{0.100000in}{0.212622in}}{\pgfqpoint{3.696000in}{3.696000in}}%
\pgfusepath{clip}%
\pgfsetbuttcap%
\pgfsetroundjoin%
\definecolor{currentfill}{rgb}{0.121569,0.466667,0.705882}%
\pgfsetfillcolor{currentfill}%
\pgfsetfillopacity{0.577117}%
\pgfsetlinewidth{1.003750pt}%
\definecolor{currentstroke}{rgb}{0.121569,0.466667,0.705882}%
\pgfsetstrokecolor{currentstroke}%
\pgfsetstrokeopacity{0.577117}%
\pgfsetdash{}{0pt}%
\pgfpathmoveto{\pgfqpoint{3.199633in}{1.775507in}}%
\pgfpathcurveto{\pgfqpoint{3.207869in}{1.775507in}}{\pgfqpoint{3.215769in}{1.778779in}}{\pgfqpoint{3.221593in}{1.784603in}}%
\pgfpathcurveto{\pgfqpoint{3.227417in}{1.790427in}}{\pgfqpoint{3.230690in}{1.798327in}}{\pgfqpoint{3.230690in}{1.806563in}}%
\pgfpathcurveto{\pgfqpoint{3.230690in}{1.814800in}}{\pgfqpoint{3.227417in}{1.822700in}}{\pgfqpoint{3.221593in}{1.828524in}}%
\pgfpathcurveto{\pgfqpoint{3.215769in}{1.834347in}}{\pgfqpoint{3.207869in}{1.837620in}}{\pgfqpoint{3.199633in}{1.837620in}}%
\pgfpathcurveto{\pgfqpoint{3.191397in}{1.837620in}}{\pgfqpoint{3.183497in}{1.834347in}}{\pgfqpoint{3.177673in}{1.828524in}}%
\pgfpathcurveto{\pgfqpoint{3.171849in}{1.822700in}}{\pgfqpoint{3.168577in}{1.814800in}}{\pgfqpoint{3.168577in}{1.806563in}}%
\pgfpathcurveto{\pgfqpoint{3.168577in}{1.798327in}}{\pgfqpoint{3.171849in}{1.790427in}}{\pgfqpoint{3.177673in}{1.784603in}}%
\pgfpathcurveto{\pgfqpoint{3.183497in}{1.778779in}}{\pgfqpoint{3.191397in}{1.775507in}}{\pgfqpoint{3.199633in}{1.775507in}}%
\pgfpathclose%
\pgfusepath{stroke,fill}%
\end{pgfscope}%
\begin{pgfscope}%
\pgfpathrectangle{\pgfqpoint{0.100000in}{0.212622in}}{\pgfqpoint{3.696000in}{3.696000in}}%
\pgfusepath{clip}%
\pgfsetbuttcap%
\pgfsetroundjoin%
\definecolor{currentfill}{rgb}{0.121569,0.466667,0.705882}%
\pgfsetfillcolor{currentfill}%
\pgfsetfillopacity{0.577526}%
\pgfsetlinewidth{1.003750pt}%
\definecolor{currentstroke}{rgb}{0.121569,0.466667,0.705882}%
\pgfsetstrokecolor{currentstroke}%
\pgfsetstrokeopacity{0.577526}%
\pgfsetdash{}{0pt}%
\pgfpathmoveto{\pgfqpoint{3.199085in}{1.775582in}}%
\pgfpathcurveto{\pgfqpoint{3.207321in}{1.775582in}}{\pgfqpoint{3.215221in}{1.778854in}}{\pgfqpoint{3.221045in}{1.784678in}}%
\pgfpathcurveto{\pgfqpoint{3.226869in}{1.790502in}}{\pgfqpoint{3.230141in}{1.798402in}}{\pgfqpoint{3.230141in}{1.806638in}}%
\pgfpathcurveto{\pgfqpoint{3.230141in}{1.814875in}}{\pgfqpoint{3.226869in}{1.822775in}}{\pgfqpoint{3.221045in}{1.828599in}}%
\pgfpathcurveto{\pgfqpoint{3.215221in}{1.834422in}}{\pgfqpoint{3.207321in}{1.837695in}}{\pgfqpoint{3.199085in}{1.837695in}}%
\pgfpathcurveto{\pgfqpoint{3.190849in}{1.837695in}}{\pgfqpoint{3.182949in}{1.834422in}}{\pgfqpoint{3.177125in}{1.828599in}}%
\pgfpathcurveto{\pgfqpoint{3.171301in}{1.822775in}}{\pgfqpoint{3.168028in}{1.814875in}}{\pgfqpoint{3.168028in}{1.806638in}}%
\pgfpathcurveto{\pgfqpoint{3.168028in}{1.798402in}}{\pgfqpoint{3.171301in}{1.790502in}}{\pgfqpoint{3.177125in}{1.784678in}}%
\pgfpathcurveto{\pgfqpoint{3.182949in}{1.778854in}}{\pgfqpoint{3.190849in}{1.775582in}}{\pgfqpoint{3.199085in}{1.775582in}}%
\pgfpathclose%
\pgfusepath{stroke,fill}%
\end{pgfscope}%
\begin{pgfscope}%
\pgfpathrectangle{\pgfqpoint{0.100000in}{0.212622in}}{\pgfqpoint{3.696000in}{3.696000in}}%
\pgfusepath{clip}%
\pgfsetbuttcap%
\pgfsetroundjoin%
\definecolor{currentfill}{rgb}{0.121569,0.466667,0.705882}%
\pgfsetfillcolor{currentfill}%
\pgfsetfillopacity{0.578290}%
\pgfsetlinewidth{1.003750pt}%
\definecolor{currentstroke}{rgb}{0.121569,0.466667,0.705882}%
\pgfsetstrokecolor{currentstroke}%
\pgfsetstrokeopacity{0.578290}%
\pgfsetdash{}{0pt}%
\pgfpathmoveto{\pgfqpoint{3.197447in}{1.775894in}}%
\pgfpathcurveto{\pgfqpoint{3.205684in}{1.775894in}}{\pgfqpoint{3.213584in}{1.779166in}}{\pgfqpoint{3.219408in}{1.784990in}}%
\pgfpathcurveto{\pgfqpoint{3.225231in}{1.790814in}}{\pgfqpoint{3.228504in}{1.798714in}}{\pgfqpoint{3.228504in}{1.806950in}}%
\pgfpathcurveto{\pgfqpoint{3.228504in}{1.815186in}}{\pgfqpoint{3.225231in}{1.823087in}}{\pgfqpoint{3.219408in}{1.828910in}}%
\pgfpathcurveto{\pgfqpoint{3.213584in}{1.834734in}}{\pgfqpoint{3.205684in}{1.838007in}}{\pgfqpoint{3.197447in}{1.838007in}}%
\pgfpathcurveto{\pgfqpoint{3.189211in}{1.838007in}}{\pgfqpoint{3.181311in}{1.834734in}}{\pgfqpoint{3.175487in}{1.828910in}}%
\pgfpathcurveto{\pgfqpoint{3.169663in}{1.823087in}}{\pgfqpoint{3.166391in}{1.815186in}}{\pgfqpoint{3.166391in}{1.806950in}}%
\pgfpathcurveto{\pgfqpoint{3.166391in}{1.798714in}}{\pgfqpoint{3.169663in}{1.790814in}}{\pgfqpoint{3.175487in}{1.784990in}}%
\pgfpathcurveto{\pgfqpoint{3.181311in}{1.779166in}}{\pgfqpoint{3.189211in}{1.775894in}}{\pgfqpoint{3.197447in}{1.775894in}}%
\pgfpathclose%
\pgfusepath{stroke,fill}%
\end{pgfscope}%
\begin{pgfscope}%
\pgfpathrectangle{\pgfqpoint{0.100000in}{0.212622in}}{\pgfqpoint{3.696000in}{3.696000in}}%
\pgfusepath{clip}%
\pgfsetbuttcap%
\pgfsetroundjoin%
\definecolor{currentfill}{rgb}{0.121569,0.466667,0.705882}%
\pgfsetfillcolor{currentfill}%
\pgfsetfillopacity{0.578425}%
\pgfsetlinewidth{1.003750pt}%
\definecolor{currentstroke}{rgb}{0.121569,0.466667,0.705882}%
\pgfsetstrokecolor{currentstroke}%
\pgfsetstrokeopacity{0.578425}%
\pgfsetdash{}{0pt}%
\pgfpathmoveto{\pgfqpoint{1.154482in}{2.133284in}}%
\pgfpathcurveto{\pgfqpoint{1.162718in}{2.133284in}}{\pgfqpoint{1.170618in}{2.136556in}}{\pgfqpoint{1.176442in}{2.142380in}}%
\pgfpathcurveto{\pgfqpoint{1.182266in}{2.148204in}}{\pgfqpoint{1.185538in}{2.156104in}}{\pgfqpoint{1.185538in}{2.164341in}}%
\pgfpathcurveto{\pgfqpoint{1.185538in}{2.172577in}}{\pgfqpoint{1.182266in}{2.180477in}}{\pgfqpoint{1.176442in}{2.186301in}}%
\pgfpathcurveto{\pgfqpoint{1.170618in}{2.192125in}}{\pgfqpoint{1.162718in}{2.195397in}}{\pgfqpoint{1.154482in}{2.195397in}}%
\pgfpathcurveto{\pgfqpoint{1.146246in}{2.195397in}}{\pgfqpoint{1.138346in}{2.192125in}}{\pgfqpoint{1.132522in}{2.186301in}}%
\pgfpathcurveto{\pgfqpoint{1.126698in}{2.180477in}}{\pgfqpoint{1.123425in}{2.172577in}}{\pgfqpoint{1.123425in}{2.164341in}}%
\pgfpathcurveto{\pgfqpoint{1.123425in}{2.156104in}}{\pgfqpoint{1.126698in}{2.148204in}}{\pgfqpoint{1.132522in}{2.142380in}}%
\pgfpathcurveto{\pgfqpoint{1.138346in}{2.136556in}}{\pgfqpoint{1.146246in}{2.133284in}}{\pgfqpoint{1.154482in}{2.133284in}}%
\pgfpathclose%
\pgfusepath{stroke,fill}%
\end{pgfscope}%
\begin{pgfscope}%
\pgfpathrectangle{\pgfqpoint{0.100000in}{0.212622in}}{\pgfqpoint{3.696000in}{3.696000in}}%
\pgfusepath{clip}%
\pgfsetbuttcap%
\pgfsetroundjoin%
\definecolor{currentfill}{rgb}{0.121569,0.466667,0.705882}%
\pgfsetfillcolor{currentfill}%
\pgfsetfillopacity{0.578701}%
\pgfsetlinewidth{1.003750pt}%
\definecolor{currentstroke}{rgb}{0.121569,0.466667,0.705882}%
\pgfsetstrokecolor{currentstroke}%
\pgfsetstrokeopacity{0.578701}%
\pgfsetdash{}{0pt}%
\pgfpathmoveto{\pgfqpoint{3.196500in}{1.776060in}}%
\pgfpathcurveto{\pgfqpoint{3.204736in}{1.776060in}}{\pgfqpoint{3.212636in}{1.779332in}}{\pgfqpoint{3.218460in}{1.785156in}}%
\pgfpathcurveto{\pgfqpoint{3.224284in}{1.790980in}}{\pgfqpoint{3.227556in}{1.798880in}}{\pgfqpoint{3.227556in}{1.807116in}}%
\pgfpathcurveto{\pgfqpoint{3.227556in}{1.815352in}}{\pgfqpoint{3.224284in}{1.823252in}}{\pgfqpoint{3.218460in}{1.829076in}}%
\pgfpathcurveto{\pgfqpoint{3.212636in}{1.834900in}}{\pgfqpoint{3.204736in}{1.838173in}}{\pgfqpoint{3.196500in}{1.838173in}}%
\pgfpathcurveto{\pgfqpoint{3.188263in}{1.838173in}}{\pgfqpoint{3.180363in}{1.834900in}}{\pgfqpoint{3.174539in}{1.829076in}}%
\pgfpathcurveto{\pgfqpoint{3.168715in}{1.823252in}}{\pgfqpoint{3.165443in}{1.815352in}}{\pgfqpoint{3.165443in}{1.807116in}}%
\pgfpathcurveto{\pgfqpoint{3.165443in}{1.798880in}}{\pgfqpoint{3.168715in}{1.790980in}}{\pgfqpoint{3.174539in}{1.785156in}}%
\pgfpathcurveto{\pgfqpoint{3.180363in}{1.779332in}}{\pgfqpoint{3.188263in}{1.776060in}}{\pgfqpoint{3.196500in}{1.776060in}}%
\pgfpathclose%
\pgfusepath{stroke,fill}%
\end{pgfscope}%
\begin{pgfscope}%
\pgfpathrectangle{\pgfqpoint{0.100000in}{0.212622in}}{\pgfqpoint{3.696000in}{3.696000in}}%
\pgfusepath{clip}%
\pgfsetbuttcap%
\pgfsetroundjoin%
\definecolor{currentfill}{rgb}{0.121569,0.466667,0.705882}%
\pgfsetfillcolor{currentfill}%
\pgfsetfillopacity{0.579244}%
\pgfsetlinewidth{1.003750pt}%
\definecolor{currentstroke}{rgb}{0.121569,0.466667,0.705882}%
\pgfsetstrokecolor{currentstroke}%
\pgfsetstrokeopacity{0.579244}%
\pgfsetdash{}{0pt}%
\pgfpathmoveto{\pgfqpoint{3.195879in}{1.776118in}}%
\pgfpathcurveto{\pgfqpoint{3.204116in}{1.776118in}}{\pgfqpoint{3.212016in}{1.779390in}}{\pgfqpoint{3.217840in}{1.785214in}}%
\pgfpathcurveto{\pgfqpoint{3.223663in}{1.791038in}}{\pgfqpoint{3.226936in}{1.798938in}}{\pgfqpoint{3.226936in}{1.807174in}}%
\pgfpathcurveto{\pgfqpoint{3.226936in}{1.815411in}}{\pgfqpoint{3.223663in}{1.823311in}}{\pgfqpoint{3.217840in}{1.829135in}}%
\pgfpathcurveto{\pgfqpoint{3.212016in}{1.834959in}}{\pgfqpoint{3.204116in}{1.838231in}}{\pgfqpoint{3.195879in}{1.838231in}}%
\pgfpathcurveto{\pgfqpoint{3.187643in}{1.838231in}}{\pgfqpoint{3.179743in}{1.834959in}}{\pgfqpoint{3.173919in}{1.829135in}}%
\pgfpathcurveto{\pgfqpoint{3.168095in}{1.823311in}}{\pgfqpoint{3.164823in}{1.815411in}}{\pgfqpoint{3.164823in}{1.807174in}}%
\pgfpathcurveto{\pgfqpoint{3.164823in}{1.798938in}}{\pgfqpoint{3.168095in}{1.791038in}}{\pgfqpoint{3.173919in}{1.785214in}}%
\pgfpathcurveto{\pgfqpoint{3.179743in}{1.779390in}}{\pgfqpoint{3.187643in}{1.776118in}}{\pgfqpoint{3.195879in}{1.776118in}}%
\pgfpathclose%
\pgfusepath{stroke,fill}%
\end{pgfscope}%
\begin{pgfscope}%
\pgfpathrectangle{\pgfqpoint{0.100000in}{0.212622in}}{\pgfqpoint{3.696000in}{3.696000in}}%
\pgfusepath{clip}%
\pgfsetbuttcap%
\pgfsetroundjoin%
\definecolor{currentfill}{rgb}{0.121569,0.466667,0.705882}%
\pgfsetfillcolor{currentfill}%
\pgfsetfillopacity{0.579938}%
\pgfsetlinewidth{1.003750pt}%
\definecolor{currentstroke}{rgb}{0.121569,0.466667,0.705882}%
\pgfsetstrokecolor{currentstroke}%
\pgfsetstrokeopacity{0.579938}%
\pgfsetdash{}{0pt}%
\pgfpathmoveto{\pgfqpoint{3.194752in}{1.776250in}}%
\pgfpathcurveto{\pgfqpoint{3.202989in}{1.776250in}}{\pgfqpoint{3.210889in}{1.779522in}}{\pgfqpoint{3.216713in}{1.785346in}}%
\pgfpathcurveto{\pgfqpoint{3.222537in}{1.791170in}}{\pgfqpoint{3.225809in}{1.799070in}}{\pgfqpoint{3.225809in}{1.807306in}}%
\pgfpathcurveto{\pgfqpoint{3.225809in}{1.815543in}}{\pgfqpoint{3.222537in}{1.823443in}}{\pgfqpoint{3.216713in}{1.829267in}}%
\pgfpathcurveto{\pgfqpoint{3.210889in}{1.835091in}}{\pgfqpoint{3.202989in}{1.838363in}}{\pgfqpoint{3.194752in}{1.838363in}}%
\pgfpathcurveto{\pgfqpoint{3.186516in}{1.838363in}}{\pgfqpoint{3.178616in}{1.835091in}}{\pgfqpoint{3.172792in}{1.829267in}}%
\pgfpathcurveto{\pgfqpoint{3.166968in}{1.823443in}}{\pgfqpoint{3.163696in}{1.815543in}}{\pgfqpoint{3.163696in}{1.807306in}}%
\pgfpathcurveto{\pgfqpoint{3.163696in}{1.799070in}}{\pgfqpoint{3.166968in}{1.791170in}}{\pgfqpoint{3.172792in}{1.785346in}}%
\pgfpathcurveto{\pgfqpoint{3.178616in}{1.779522in}}{\pgfqpoint{3.186516in}{1.776250in}}{\pgfqpoint{3.194752in}{1.776250in}}%
\pgfpathclose%
\pgfusepath{stroke,fill}%
\end{pgfscope}%
\begin{pgfscope}%
\pgfpathrectangle{\pgfqpoint{0.100000in}{0.212622in}}{\pgfqpoint{3.696000in}{3.696000in}}%
\pgfusepath{clip}%
\pgfsetbuttcap%
\pgfsetroundjoin%
\definecolor{currentfill}{rgb}{0.121569,0.466667,0.705882}%
\pgfsetfillcolor{currentfill}%
\pgfsetfillopacity{0.580181}%
\pgfsetlinewidth{1.003750pt}%
\definecolor{currentstroke}{rgb}{0.121569,0.466667,0.705882}%
\pgfsetstrokecolor{currentstroke}%
\pgfsetstrokeopacity{0.580181}%
\pgfsetdash{}{0pt}%
\pgfpathmoveto{\pgfqpoint{1.151732in}{2.133427in}}%
\pgfpathcurveto{\pgfqpoint{1.159968in}{2.133427in}}{\pgfqpoint{1.167868in}{2.136699in}}{\pgfqpoint{1.173692in}{2.142523in}}%
\pgfpathcurveto{\pgfqpoint{1.179516in}{2.148347in}}{\pgfqpoint{1.182788in}{2.156247in}}{\pgfqpoint{1.182788in}{2.164483in}}%
\pgfpathcurveto{\pgfqpoint{1.182788in}{2.172719in}}{\pgfqpoint{1.179516in}{2.180619in}}{\pgfqpoint{1.173692in}{2.186443in}}%
\pgfpathcurveto{\pgfqpoint{1.167868in}{2.192267in}}{\pgfqpoint{1.159968in}{2.195540in}}{\pgfqpoint{1.151732in}{2.195540in}}%
\pgfpathcurveto{\pgfqpoint{1.143495in}{2.195540in}}{\pgfqpoint{1.135595in}{2.192267in}}{\pgfqpoint{1.129771in}{2.186443in}}%
\pgfpathcurveto{\pgfqpoint{1.123948in}{2.180619in}}{\pgfqpoint{1.120675in}{2.172719in}}{\pgfqpoint{1.120675in}{2.164483in}}%
\pgfpathcurveto{\pgfqpoint{1.120675in}{2.156247in}}{\pgfqpoint{1.123948in}{2.148347in}}{\pgfqpoint{1.129771in}{2.142523in}}%
\pgfpathcurveto{\pgfqpoint{1.135595in}{2.136699in}}{\pgfqpoint{1.143495in}{2.133427in}}{\pgfqpoint{1.151732in}{2.133427in}}%
\pgfpathclose%
\pgfusepath{stroke,fill}%
\end{pgfscope}%
\begin{pgfscope}%
\pgfpathrectangle{\pgfqpoint{0.100000in}{0.212622in}}{\pgfqpoint{3.696000in}{3.696000in}}%
\pgfusepath{clip}%
\pgfsetbuttcap%
\pgfsetroundjoin%
\definecolor{currentfill}{rgb}{0.121569,0.466667,0.705882}%
\pgfsetfillcolor{currentfill}%
\pgfsetfillopacity{0.580301}%
\pgfsetlinewidth{1.003750pt}%
\definecolor{currentstroke}{rgb}{0.121569,0.466667,0.705882}%
\pgfsetstrokecolor{currentstroke}%
\pgfsetstrokeopacity{0.580301}%
\pgfsetdash{}{0pt}%
\pgfpathmoveto{\pgfqpoint{3.193943in}{1.776413in}}%
\pgfpathcurveto{\pgfqpoint{3.202179in}{1.776413in}}{\pgfqpoint{3.210079in}{1.779686in}}{\pgfqpoint{3.215903in}{1.785510in}}%
\pgfpathcurveto{\pgfqpoint{3.221727in}{1.791334in}}{\pgfqpoint{3.224999in}{1.799234in}}{\pgfqpoint{3.224999in}{1.807470in}}%
\pgfpathcurveto{\pgfqpoint{3.224999in}{1.815706in}}{\pgfqpoint{3.221727in}{1.823606in}}{\pgfqpoint{3.215903in}{1.829430in}}%
\pgfpathcurveto{\pgfqpoint{3.210079in}{1.835254in}}{\pgfqpoint{3.202179in}{1.838526in}}{\pgfqpoint{3.193943in}{1.838526in}}%
\pgfpathcurveto{\pgfqpoint{3.185706in}{1.838526in}}{\pgfqpoint{3.177806in}{1.835254in}}{\pgfqpoint{3.171982in}{1.829430in}}%
\pgfpathcurveto{\pgfqpoint{3.166158in}{1.823606in}}{\pgfqpoint{3.162886in}{1.815706in}}{\pgfqpoint{3.162886in}{1.807470in}}%
\pgfpathcurveto{\pgfqpoint{3.162886in}{1.799234in}}{\pgfqpoint{3.166158in}{1.791334in}}{\pgfqpoint{3.171982in}{1.785510in}}%
\pgfpathcurveto{\pgfqpoint{3.177806in}{1.779686in}}{\pgfqpoint{3.185706in}{1.776413in}}{\pgfqpoint{3.193943in}{1.776413in}}%
\pgfpathclose%
\pgfusepath{stroke,fill}%
\end{pgfscope}%
\begin{pgfscope}%
\pgfpathrectangle{\pgfqpoint{0.100000in}{0.212622in}}{\pgfqpoint{3.696000in}{3.696000in}}%
\pgfusepath{clip}%
\pgfsetbuttcap%
\pgfsetroundjoin%
\definecolor{currentfill}{rgb}{0.121569,0.466667,0.705882}%
\pgfsetfillcolor{currentfill}%
\pgfsetfillopacity{0.581132}%
\pgfsetlinewidth{1.003750pt}%
\definecolor{currentstroke}{rgb}{0.121569,0.466667,0.705882}%
\pgfsetstrokecolor{currentstroke}%
\pgfsetstrokeopacity{0.581132}%
\pgfsetdash{}{0pt}%
\pgfpathmoveto{\pgfqpoint{3.192624in}{1.776568in}}%
\pgfpathcurveto{\pgfqpoint{3.200860in}{1.776568in}}{\pgfqpoint{3.208761in}{1.779840in}}{\pgfqpoint{3.214584in}{1.785664in}}%
\pgfpathcurveto{\pgfqpoint{3.220408in}{1.791488in}}{\pgfqpoint{3.223681in}{1.799388in}}{\pgfqpoint{3.223681in}{1.807625in}}%
\pgfpathcurveto{\pgfqpoint{3.223681in}{1.815861in}}{\pgfqpoint{3.220408in}{1.823761in}}{\pgfqpoint{3.214584in}{1.829585in}}%
\pgfpathcurveto{\pgfqpoint{3.208761in}{1.835409in}}{\pgfqpoint{3.200860in}{1.838681in}}{\pgfqpoint{3.192624in}{1.838681in}}%
\pgfpathcurveto{\pgfqpoint{3.184388in}{1.838681in}}{\pgfqpoint{3.176488in}{1.835409in}}{\pgfqpoint{3.170664in}{1.829585in}}%
\pgfpathcurveto{\pgfqpoint{3.164840in}{1.823761in}}{\pgfqpoint{3.161568in}{1.815861in}}{\pgfqpoint{3.161568in}{1.807625in}}%
\pgfpathcurveto{\pgfqpoint{3.161568in}{1.799388in}}{\pgfqpoint{3.164840in}{1.791488in}}{\pgfqpoint{3.170664in}{1.785664in}}%
\pgfpathcurveto{\pgfqpoint{3.176488in}{1.779840in}}{\pgfqpoint{3.184388in}{1.776568in}}{\pgfqpoint{3.192624in}{1.776568in}}%
\pgfpathclose%
\pgfusepath{stroke,fill}%
\end{pgfscope}%
\begin{pgfscope}%
\pgfpathrectangle{\pgfqpoint{0.100000in}{0.212622in}}{\pgfqpoint{3.696000in}{3.696000in}}%
\pgfusepath{clip}%
\pgfsetbuttcap%
\pgfsetroundjoin%
\definecolor{currentfill}{rgb}{0.121569,0.466667,0.705882}%
\pgfsetfillcolor{currentfill}%
\pgfsetfillopacity{0.581223}%
\pgfsetlinewidth{1.003750pt}%
\definecolor{currentstroke}{rgb}{0.121569,0.466667,0.705882}%
\pgfsetstrokecolor{currentstroke}%
\pgfsetstrokeopacity{0.581223}%
\pgfsetdash{}{0pt}%
\pgfpathmoveto{\pgfqpoint{1.149272in}{2.133549in}}%
\pgfpathcurveto{\pgfqpoint{1.157508in}{2.133549in}}{\pgfqpoint{1.165408in}{2.136821in}}{\pgfqpoint{1.171232in}{2.142645in}}%
\pgfpathcurveto{\pgfqpoint{1.177056in}{2.148469in}}{\pgfqpoint{1.180328in}{2.156369in}}{\pgfqpoint{1.180328in}{2.164605in}}%
\pgfpathcurveto{\pgfqpoint{1.180328in}{2.172841in}}{\pgfqpoint{1.177056in}{2.180741in}}{\pgfqpoint{1.171232in}{2.186565in}}%
\pgfpathcurveto{\pgfqpoint{1.165408in}{2.192389in}}{\pgfqpoint{1.157508in}{2.195662in}}{\pgfqpoint{1.149272in}{2.195662in}}%
\pgfpathcurveto{\pgfqpoint{1.141035in}{2.195662in}}{\pgfqpoint{1.133135in}{2.192389in}}{\pgfqpoint{1.127311in}{2.186565in}}%
\pgfpathcurveto{\pgfqpoint{1.121487in}{2.180741in}}{\pgfqpoint{1.118215in}{2.172841in}}{\pgfqpoint{1.118215in}{2.164605in}}%
\pgfpathcurveto{\pgfqpoint{1.118215in}{2.156369in}}{\pgfqpoint{1.121487in}{2.148469in}}{\pgfqpoint{1.127311in}{2.142645in}}%
\pgfpathcurveto{\pgfqpoint{1.133135in}{2.136821in}}{\pgfqpoint{1.141035in}{2.133549in}}{\pgfqpoint{1.149272in}{2.133549in}}%
\pgfpathclose%
\pgfusepath{stroke,fill}%
\end{pgfscope}%
\begin{pgfscope}%
\pgfpathrectangle{\pgfqpoint{0.100000in}{0.212622in}}{\pgfqpoint{3.696000in}{3.696000in}}%
\pgfusepath{clip}%
\pgfsetbuttcap%
\pgfsetroundjoin%
\definecolor{currentfill}{rgb}{0.121569,0.466667,0.705882}%
\pgfsetfillcolor{currentfill}%
\pgfsetfillopacity{0.582135}%
\pgfsetlinewidth{1.003750pt}%
\definecolor{currentstroke}{rgb}{0.121569,0.466667,0.705882}%
\pgfsetstrokecolor{currentstroke}%
\pgfsetstrokeopacity{0.582135}%
\pgfsetdash{}{0pt}%
\pgfpathmoveto{\pgfqpoint{3.191229in}{1.776755in}}%
\pgfpathcurveto{\pgfqpoint{3.199466in}{1.776755in}}{\pgfqpoint{3.207366in}{1.780027in}}{\pgfqpoint{3.213190in}{1.785851in}}%
\pgfpathcurveto{\pgfqpoint{3.219013in}{1.791675in}}{\pgfqpoint{3.222286in}{1.799575in}}{\pgfqpoint{3.222286in}{1.807812in}}%
\pgfpathcurveto{\pgfqpoint{3.222286in}{1.816048in}}{\pgfqpoint{3.219013in}{1.823948in}}{\pgfqpoint{3.213190in}{1.829772in}}%
\pgfpathcurveto{\pgfqpoint{3.207366in}{1.835596in}}{\pgfqpoint{3.199466in}{1.838868in}}{\pgfqpoint{3.191229in}{1.838868in}}%
\pgfpathcurveto{\pgfqpoint{3.182993in}{1.838868in}}{\pgfqpoint{3.175093in}{1.835596in}}{\pgfqpoint{3.169269in}{1.829772in}}%
\pgfpathcurveto{\pgfqpoint{3.163445in}{1.823948in}}{\pgfqpoint{3.160173in}{1.816048in}}{\pgfqpoint{3.160173in}{1.807812in}}%
\pgfpathcurveto{\pgfqpoint{3.160173in}{1.799575in}}{\pgfqpoint{3.163445in}{1.791675in}}{\pgfqpoint{3.169269in}{1.785851in}}%
\pgfpathcurveto{\pgfqpoint{3.175093in}{1.780027in}}{\pgfqpoint{3.182993in}{1.776755in}}{\pgfqpoint{3.191229in}{1.776755in}}%
\pgfpathclose%
\pgfusepath{stroke,fill}%
\end{pgfscope}%
\begin{pgfscope}%
\pgfpathrectangle{\pgfqpoint{0.100000in}{0.212622in}}{\pgfqpoint{3.696000in}{3.696000in}}%
\pgfusepath{clip}%
\pgfsetbuttcap%
\pgfsetroundjoin%
\definecolor{currentfill}{rgb}{0.121569,0.466667,0.705882}%
\pgfsetfillcolor{currentfill}%
\pgfsetfillopacity{0.582641}%
\pgfsetlinewidth{1.003750pt}%
\definecolor{currentstroke}{rgb}{0.121569,0.466667,0.705882}%
\pgfsetstrokecolor{currentstroke}%
\pgfsetstrokeopacity{0.582641}%
\pgfsetdash{}{0pt}%
\pgfpathmoveto{\pgfqpoint{3.190027in}{1.777017in}}%
\pgfpathcurveto{\pgfqpoint{3.198264in}{1.777017in}}{\pgfqpoint{3.206164in}{1.780289in}}{\pgfqpoint{3.211988in}{1.786113in}}%
\pgfpathcurveto{\pgfqpoint{3.217812in}{1.791937in}}{\pgfqpoint{3.221084in}{1.799837in}}{\pgfqpoint{3.221084in}{1.808073in}}%
\pgfpathcurveto{\pgfqpoint{3.221084in}{1.816309in}}{\pgfqpoint{3.217812in}{1.824209in}}{\pgfqpoint{3.211988in}{1.830033in}}%
\pgfpathcurveto{\pgfqpoint{3.206164in}{1.835857in}}{\pgfqpoint{3.198264in}{1.839130in}}{\pgfqpoint{3.190027in}{1.839130in}}%
\pgfpathcurveto{\pgfqpoint{3.181791in}{1.839130in}}{\pgfqpoint{3.173891in}{1.835857in}}{\pgfqpoint{3.168067in}{1.830033in}}%
\pgfpathcurveto{\pgfqpoint{3.162243in}{1.824209in}}{\pgfqpoint{3.158971in}{1.816309in}}{\pgfqpoint{3.158971in}{1.808073in}}%
\pgfpathcurveto{\pgfqpoint{3.158971in}{1.799837in}}{\pgfqpoint{3.162243in}{1.791937in}}{\pgfqpoint{3.168067in}{1.786113in}}%
\pgfpathcurveto{\pgfqpoint{3.173891in}{1.780289in}}{\pgfqpoint{3.181791in}{1.777017in}}{\pgfqpoint{3.190027in}{1.777017in}}%
\pgfpathclose%
\pgfusepath{stroke,fill}%
\end{pgfscope}%
\begin{pgfscope}%
\pgfpathrectangle{\pgfqpoint{0.100000in}{0.212622in}}{\pgfqpoint{3.696000in}{3.696000in}}%
\pgfusepath{clip}%
\pgfsetbuttcap%
\pgfsetroundjoin%
\definecolor{currentfill}{rgb}{0.121569,0.466667,0.705882}%
\pgfsetfillcolor{currentfill}%
\pgfsetfillopacity{0.583111}%
\pgfsetlinewidth{1.003750pt}%
\definecolor{currentstroke}{rgb}{0.121569,0.466667,0.705882}%
\pgfsetstrokecolor{currentstroke}%
\pgfsetstrokeopacity{0.583111}%
\pgfsetdash{}{0pt}%
\pgfpathmoveto{\pgfqpoint{1.144831in}{2.133668in}}%
\pgfpathcurveto{\pgfqpoint{1.153067in}{2.133668in}}{\pgfqpoint{1.160967in}{2.136940in}}{\pgfqpoint{1.166791in}{2.142764in}}%
\pgfpathcurveto{\pgfqpoint{1.172615in}{2.148588in}}{\pgfqpoint{1.175887in}{2.156488in}}{\pgfqpoint{1.175887in}{2.164724in}}%
\pgfpathcurveto{\pgfqpoint{1.175887in}{2.172960in}}{\pgfqpoint{1.172615in}{2.180860in}}{\pgfqpoint{1.166791in}{2.186684in}}%
\pgfpathcurveto{\pgfqpoint{1.160967in}{2.192508in}}{\pgfqpoint{1.153067in}{2.195781in}}{\pgfqpoint{1.144831in}{2.195781in}}%
\pgfpathcurveto{\pgfqpoint{1.136594in}{2.195781in}}{\pgfqpoint{1.128694in}{2.192508in}}{\pgfqpoint{1.122870in}{2.186684in}}%
\pgfpathcurveto{\pgfqpoint{1.117046in}{2.180860in}}{\pgfqpoint{1.113774in}{2.172960in}}{\pgfqpoint{1.113774in}{2.164724in}}%
\pgfpathcurveto{\pgfqpoint{1.113774in}{2.156488in}}{\pgfqpoint{1.117046in}{2.148588in}}{\pgfqpoint{1.122870in}{2.142764in}}%
\pgfpathcurveto{\pgfqpoint{1.128694in}{2.136940in}}{\pgfqpoint{1.136594in}{2.133668in}}{\pgfqpoint{1.144831in}{2.133668in}}%
\pgfpathclose%
\pgfusepath{stroke,fill}%
\end{pgfscope}%
\begin{pgfscope}%
\pgfpathrectangle{\pgfqpoint{0.100000in}{0.212622in}}{\pgfqpoint{3.696000in}{3.696000in}}%
\pgfusepath{clip}%
\pgfsetbuttcap%
\pgfsetroundjoin%
\definecolor{currentfill}{rgb}{0.121569,0.466667,0.705882}%
\pgfsetfillcolor{currentfill}%
\pgfsetfillopacity{0.583618}%
\pgfsetlinewidth{1.003750pt}%
\definecolor{currentstroke}{rgb}{0.121569,0.466667,0.705882}%
\pgfsetstrokecolor{currentstroke}%
\pgfsetstrokeopacity{0.583618}%
\pgfsetdash{}{0pt}%
\pgfpathmoveto{\pgfqpoint{3.188481in}{1.777144in}}%
\pgfpathcurveto{\pgfqpoint{3.196717in}{1.777144in}}{\pgfqpoint{3.204617in}{1.780416in}}{\pgfqpoint{3.210441in}{1.786240in}}%
\pgfpathcurveto{\pgfqpoint{3.216265in}{1.792064in}}{\pgfqpoint{3.219538in}{1.799964in}}{\pgfqpoint{3.219538in}{1.808201in}}%
\pgfpathcurveto{\pgfqpoint{3.219538in}{1.816437in}}{\pgfqpoint{3.216265in}{1.824337in}}{\pgfqpoint{3.210441in}{1.830161in}}%
\pgfpathcurveto{\pgfqpoint{3.204617in}{1.835985in}}{\pgfqpoint{3.196717in}{1.839257in}}{\pgfqpoint{3.188481in}{1.839257in}}%
\pgfpathcurveto{\pgfqpoint{3.180245in}{1.839257in}}{\pgfqpoint{3.172345in}{1.835985in}}{\pgfqpoint{3.166521in}{1.830161in}}%
\pgfpathcurveto{\pgfqpoint{3.160697in}{1.824337in}}{\pgfqpoint{3.157425in}{1.816437in}}{\pgfqpoint{3.157425in}{1.808201in}}%
\pgfpathcurveto{\pgfqpoint{3.157425in}{1.799964in}}{\pgfqpoint{3.160697in}{1.792064in}}{\pgfqpoint{3.166521in}{1.786240in}}%
\pgfpathcurveto{\pgfqpoint{3.172345in}{1.780416in}}{\pgfqpoint{3.180245in}{1.777144in}}{\pgfqpoint{3.188481in}{1.777144in}}%
\pgfpathclose%
\pgfusepath{stroke,fill}%
\end{pgfscope}%
\begin{pgfscope}%
\pgfpathrectangle{\pgfqpoint{0.100000in}{0.212622in}}{\pgfqpoint{3.696000in}{3.696000in}}%
\pgfusepath{clip}%
\pgfsetbuttcap%
\pgfsetroundjoin%
\definecolor{currentfill}{rgb}{0.121569,0.466667,0.705882}%
\pgfsetfillcolor{currentfill}%
\pgfsetfillopacity{0.584746}%
\pgfsetlinewidth{1.003750pt}%
\definecolor{currentstroke}{rgb}{0.121569,0.466667,0.705882}%
\pgfsetstrokecolor{currentstroke}%
\pgfsetstrokeopacity{0.584746}%
\pgfsetdash{}{0pt}%
\pgfpathmoveto{\pgfqpoint{3.186989in}{1.777339in}}%
\pgfpathcurveto{\pgfqpoint{3.195226in}{1.777339in}}{\pgfqpoint{3.203126in}{1.780611in}}{\pgfqpoint{3.208950in}{1.786435in}}%
\pgfpathcurveto{\pgfqpoint{3.214774in}{1.792259in}}{\pgfqpoint{3.218046in}{1.800159in}}{\pgfqpoint{3.218046in}{1.808396in}}%
\pgfpathcurveto{\pgfqpoint{3.218046in}{1.816632in}}{\pgfqpoint{3.214774in}{1.824532in}}{\pgfqpoint{3.208950in}{1.830356in}}%
\pgfpathcurveto{\pgfqpoint{3.203126in}{1.836180in}}{\pgfqpoint{3.195226in}{1.839452in}}{\pgfqpoint{3.186989in}{1.839452in}}%
\pgfpathcurveto{\pgfqpoint{3.178753in}{1.839452in}}{\pgfqpoint{3.170853in}{1.836180in}}{\pgfqpoint{3.165029in}{1.830356in}}%
\pgfpathcurveto{\pgfqpoint{3.159205in}{1.824532in}}{\pgfqpoint{3.155933in}{1.816632in}}{\pgfqpoint{3.155933in}{1.808396in}}%
\pgfpathcurveto{\pgfqpoint{3.155933in}{1.800159in}}{\pgfqpoint{3.159205in}{1.792259in}}{\pgfqpoint{3.165029in}{1.786435in}}%
\pgfpathcurveto{\pgfqpoint{3.170853in}{1.780611in}}{\pgfqpoint{3.178753in}{1.777339in}}{\pgfqpoint{3.186989in}{1.777339in}}%
\pgfpathclose%
\pgfusepath{stroke,fill}%
\end{pgfscope}%
\begin{pgfscope}%
\pgfpathrectangle{\pgfqpoint{0.100000in}{0.212622in}}{\pgfqpoint{3.696000in}{3.696000in}}%
\pgfusepath{clip}%
\pgfsetbuttcap%
\pgfsetroundjoin%
\definecolor{currentfill}{rgb}{0.121569,0.466667,0.705882}%
\pgfsetfillcolor{currentfill}%
\pgfsetfillopacity{0.584890}%
\pgfsetlinewidth{1.003750pt}%
\definecolor{currentstroke}{rgb}{0.121569,0.466667,0.705882}%
\pgfsetstrokecolor{currentstroke}%
\pgfsetstrokeopacity{0.584890}%
\pgfsetdash{}{0pt}%
\pgfpathmoveto{\pgfqpoint{1.141362in}{2.133672in}}%
\pgfpathcurveto{\pgfqpoint{1.149599in}{2.133672in}}{\pgfqpoint{1.157499in}{2.136944in}}{\pgfqpoint{1.163323in}{2.142768in}}%
\pgfpathcurveto{\pgfqpoint{1.169146in}{2.148592in}}{\pgfqpoint{1.172419in}{2.156492in}}{\pgfqpoint{1.172419in}{2.164728in}}%
\pgfpathcurveto{\pgfqpoint{1.172419in}{2.172964in}}{\pgfqpoint{1.169146in}{2.180864in}}{\pgfqpoint{1.163323in}{2.186688in}}%
\pgfpathcurveto{\pgfqpoint{1.157499in}{2.192512in}}{\pgfqpoint{1.149599in}{2.195785in}}{\pgfqpoint{1.141362in}{2.195785in}}%
\pgfpathcurveto{\pgfqpoint{1.133126in}{2.195785in}}{\pgfqpoint{1.125226in}{2.192512in}}{\pgfqpoint{1.119402in}{2.186688in}}%
\pgfpathcurveto{\pgfqpoint{1.113578in}{2.180864in}}{\pgfqpoint{1.110306in}{2.172964in}}{\pgfqpoint{1.110306in}{2.164728in}}%
\pgfpathcurveto{\pgfqpoint{1.110306in}{2.156492in}}{\pgfqpoint{1.113578in}{2.148592in}}{\pgfqpoint{1.119402in}{2.142768in}}%
\pgfpathcurveto{\pgfqpoint{1.125226in}{2.136944in}}{\pgfqpoint{1.133126in}{2.133672in}}{\pgfqpoint{1.141362in}{2.133672in}}%
\pgfpathclose%
\pgfusepath{stroke,fill}%
\end{pgfscope}%
\begin{pgfscope}%
\pgfpathrectangle{\pgfqpoint{0.100000in}{0.212622in}}{\pgfqpoint{3.696000in}{3.696000in}}%
\pgfusepath{clip}%
\pgfsetbuttcap%
\pgfsetroundjoin%
\definecolor{currentfill}{rgb}{0.121569,0.466667,0.705882}%
\pgfsetfillcolor{currentfill}%
\pgfsetfillopacity{0.585882}%
\pgfsetlinewidth{1.003750pt}%
\definecolor{currentstroke}{rgb}{0.121569,0.466667,0.705882}%
\pgfsetstrokecolor{currentstroke}%
\pgfsetstrokeopacity{0.585882}%
\pgfsetdash{}{0pt}%
\pgfpathmoveto{\pgfqpoint{3.184542in}{1.777764in}}%
\pgfpathcurveto{\pgfqpoint{3.192778in}{1.777764in}}{\pgfqpoint{3.200678in}{1.781037in}}{\pgfqpoint{3.206502in}{1.786861in}}%
\pgfpathcurveto{\pgfqpoint{3.212326in}{1.792684in}}{\pgfqpoint{3.215598in}{1.800585in}}{\pgfqpoint{3.215598in}{1.808821in}}%
\pgfpathcurveto{\pgfqpoint{3.215598in}{1.817057in}}{\pgfqpoint{3.212326in}{1.824957in}}{\pgfqpoint{3.206502in}{1.830781in}}%
\pgfpathcurveto{\pgfqpoint{3.200678in}{1.836605in}}{\pgfqpoint{3.192778in}{1.839877in}}{\pgfqpoint{3.184542in}{1.839877in}}%
\pgfpathcurveto{\pgfqpoint{3.176306in}{1.839877in}}{\pgfqpoint{3.168406in}{1.836605in}}{\pgfqpoint{3.162582in}{1.830781in}}%
\pgfpathcurveto{\pgfqpoint{3.156758in}{1.824957in}}{\pgfqpoint{3.153485in}{1.817057in}}{\pgfqpoint{3.153485in}{1.808821in}}%
\pgfpathcurveto{\pgfqpoint{3.153485in}{1.800585in}}{\pgfqpoint{3.156758in}{1.792684in}}{\pgfqpoint{3.162582in}{1.786861in}}%
\pgfpathcurveto{\pgfqpoint{3.168406in}{1.781037in}}{\pgfqpoint{3.176306in}{1.777764in}}{\pgfqpoint{3.184542in}{1.777764in}}%
\pgfpathclose%
\pgfusepath{stroke,fill}%
\end{pgfscope}%
\begin{pgfscope}%
\pgfpathrectangle{\pgfqpoint{0.100000in}{0.212622in}}{\pgfqpoint{3.696000in}{3.696000in}}%
\pgfusepath{clip}%
\pgfsetbuttcap%
\pgfsetroundjoin%
\definecolor{currentfill}{rgb}{0.121569,0.466667,0.705882}%
\pgfsetfillcolor{currentfill}%
\pgfsetfillopacity{0.586098}%
\pgfsetlinewidth{1.003750pt}%
\definecolor{currentstroke}{rgb}{0.121569,0.466667,0.705882}%
\pgfsetstrokecolor{currentstroke}%
\pgfsetstrokeopacity{0.586098}%
\pgfsetdash{}{0pt}%
\pgfpathmoveto{\pgfqpoint{1.138133in}{2.134035in}}%
\pgfpathcurveto{\pgfqpoint{1.146369in}{2.134035in}}{\pgfqpoint{1.154269in}{2.137308in}}{\pgfqpoint{1.160093in}{2.143131in}}%
\pgfpathcurveto{\pgfqpoint{1.165917in}{2.148955in}}{\pgfqpoint{1.169189in}{2.156855in}}{\pgfqpoint{1.169189in}{2.165092in}}%
\pgfpathcurveto{\pgfqpoint{1.169189in}{2.173328in}}{\pgfqpoint{1.165917in}{2.181228in}}{\pgfqpoint{1.160093in}{2.187052in}}%
\pgfpathcurveto{\pgfqpoint{1.154269in}{2.192876in}}{\pgfqpoint{1.146369in}{2.196148in}}{\pgfqpoint{1.138133in}{2.196148in}}%
\pgfpathcurveto{\pgfqpoint{1.129897in}{2.196148in}}{\pgfqpoint{1.121997in}{2.192876in}}{\pgfqpoint{1.116173in}{2.187052in}}%
\pgfpathcurveto{\pgfqpoint{1.110349in}{2.181228in}}{\pgfqpoint{1.107076in}{2.173328in}}{\pgfqpoint{1.107076in}{2.165092in}}%
\pgfpathcurveto{\pgfqpoint{1.107076in}{2.156855in}}{\pgfqpoint{1.110349in}{2.148955in}}{\pgfqpoint{1.116173in}{2.143131in}}%
\pgfpathcurveto{\pgfqpoint{1.121997in}{2.137308in}}{\pgfqpoint{1.129897in}{2.134035in}}{\pgfqpoint{1.138133in}{2.134035in}}%
\pgfpathclose%
\pgfusepath{stroke,fill}%
\end{pgfscope}%
\begin{pgfscope}%
\pgfpathrectangle{\pgfqpoint{0.100000in}{0.212622in}}{\pgfqpoint{3.696000in}{3.696000in}}%
\pgfusepath{clip}%
\pgfsetbuttcap%
\pgfsetroundjoin%
\definecolor{currentfill}{rgb}{0.121569,0.466667,0.705882}%
\pgfsetfillcolor{currentfill}%
\pgfsetfillopacity{0.587223}%
\pgfsetlinewidth{1.003750pt}%
\definecolor{currentstroke}{rgb}{0.121569,0.466667,0.705882}%
\pgfsetstrokecolor{currentstroke}%
\pgfsetstrokeopacity{0.587223}%
\pgfsetdash{}{0pt}%
\pgfpathmoveto{\pgfqpoint{1.136430in}{2.134157in}}%
\pgfpathcurveto{\pgfqpoint{1.144667in}{2.134157in}}{\pgfqpoint{1.152567in}{2.137429in}}{\pgfqpoint{1.158391in}{2.143253in}}%
\pgfpathcurveto{\pgfqpoint{1.164215in}{2.149077in}}{\pgfqpoint{1.167487in}{2.156977in}}{\pgfqpoint{1.167487in}{2.165213in}}%
\pgfpathcurveto{\pgfqpoint{1.167487in}{2.173449in}}{\pgfqpoint{1.164215in}{2.181349in}}{\pgfqpoint{1.158391in}{2.187173in}}%
\pgfpathcurveto{\pgfqpoint{1.152567in}{2.192997in}}{\pgfqpoint{1.144667in}{2.196270in}}{\pgfqpoint{1.136430in}{2.196270in}}%
\pgfpathcurveto{\pgfqpoint{1.128194in}{2.196270in}}{\pgfqpoint{1.120294in}{2.192997in}}{\pgfqpoint{1.114470in}{2.187173in}}%
\pgfpathcurveto{\pgfqpoint{1.108646in}{2.181349in}}{\pgfqpoint{1.105374in}{2.173449in}}{\pgfqpoint{1.105374in}{2.165213in}}%
\pgfpathcurveto{\pgfqpoint{1.105374in}{2.156977in}}{\pgfqpoint{1.108646in}{2.149077in}}{\pgfqpoint{1.114470in}{2.143253in}}%
\pgfpathcurveto{\pgfqpoint{1.120294in}{2.137429in}}{\pgfqpoint{1.128194in}{2.134157in}}{\pgfqpoint{1.136430in}{2.134157in}}%
\pgfpathclose%
\pgfusepath{stroke,fill}%
\end{pgfscope}%
\begin{pgfscope}%
\pgfpathrectangle{\pgfqpoint{0.100000in}{0.212622in}}{\pgfqpoint{3.696000in}{3.696000in}}%
\pgfusepath{clip}%
\pgfsetbuttcap%
\pgfsetroundjoin%
\definecolor{currentfill}{rgb}{0.121569,0.466667,0.705882}%
\pgfsetfillcolor{currentfill}%
\pgfsetfillopacity{0.587257}%
\pgfsetlinewidth{1.003750pt}%
\definecolor{currentstroke}{rgb}{0.121569,0.466667,0.705882}%
\pgfsetstrokecolor{currentstroke}%
\pgfsetstrokeopacity{0.587257}%
\pgfsetdash{}{0pt}%
\pgfpathmoveto{\pgfqpoint{3.182086in}{1.778075in}}%
\pgfpathcurveto{\pgfqpoint{3.190322in}{1.778075in}}{\pgfqpoint{3.198222in}{1.781348in}}{\pgfqpoint{3.204046in}{1.787172in}}%
\pgfpathcurveto{\pgfqpoint{3.209870in}{1.792996in}}{\pgfqpoint{3.213143in}{1.800896in}}{\pgfqpoint{3.213143in}{1.809132in}}%
\pgfpathcurveto{\pgfqpoint{3.213143in}{1.817368in}}{\pgfqpoint{3.209870in}{1.825268in}}{\pgfqpoint{3.204046in}{1.831092in}}%
\pgfpathcurveto{\pgfqpoint{3.198222in}{1.836916in}}{\pgfqpoint{3.190322in}{1.840188in}}{\pgfqpoint{3.182086in}{1.840188in}}%
\pgfpathcurveto{\pgfqpoint{3.173850in}{1.840188in}}{\pgfqpoint{3.165950in}{1.836916in}}{\pgfqpoint{3.160126in}{1.831092in}}%
\pgfpathcurveto{\pgfqpoint{3.154302in}{1.825268in}}{\pgfqpoint{3.151030in}{1.817368in}}{\pgfqpoint{3.151030in}{1.809132in}}%
\pgfpathcurveto{\pgfqpoint{3.151030in}{1.800896in}}{\pgfqpoint{3.154302in}{1.792996in}}{\pgfqpoint{3.160126in}{1.787172in}}%
\pgfpathcurveto{\pgfqpoint{3.165950in}{1.781348in}}{\pgfqpoint{3.173850in}{1.778075in}}{\pgfqpoint{3.182086in}{1.778075in}}%
\pgfpathclose%
\pgfusepath{stroke,fill}%
\end{pgfscope}%
\begin{pgfscope}%
\pgfpathrectangle{\pgfqpoint{0.100000in}{0.212622in}}{\pgfqpoint{3.696000in}{3.696000in}}%
\pgfusepath{clip}%
\pgfsetbuttcap%
\pgfsetroundjoin%
\definecolor{currentfill}{rgb}{0.121569,0.466667,0.705882}%
\pgfsetfillcolor{currentfill}%
\pgfsetfillopacity{0.587852}%
\pgfsetlinewidth{1.003750pt}%
\definecolor{currentstroke}{rgb}{0.121569,0.466667,0.705882}%
\pgfsetstrokecolor{currentstroke}%
\pgfsetstrokeopacity{0.587852}%
\pgfsetdash{}{0pt}%
\pgfpathmoveto{\pgfqpoint{1.134708in}{2.134314in}}%
\pgfpathcurveto{\pgfqpoint{1.142945in}{2.134314in}}{\pgfqpoint{1.150845in}{2.137586in}}{\pgfqpoint{1.156669in}{2.143410in}}%
\pgfpathcurveto{\pgfqpoint{1.162493in}{2.149234in}}{\pgfqpoint{1.165765in}{2.157134in}}{\pgfqpoint{1.165765in}{2.165370in}}%
\pgfpathcurveto{\pgfqpoint{1.165765in}{2.173606in}}{\pgfqpoint{1.162493in}{2.181507in}}{\pgfqpoint{1.156669in}{2.187330in}}%
\pgfpathcurveto{\pgfqpoint{1.150845in}{2.193154in}}{\pgfqpoint{1.142945in}{2.196427in}}{\pgfqpoint{1.134708in}{2.196427in}}%
\pgfpathcurveto{\pgfqpoint{1.126472in}{2.196427in}}{\pgfqpoint{1.118572in}{2.193154in}}{\pgfqpoint{1.112748in}{2.187330in}}%
\pgfpathcurveto{\pgfqpoint{1.106924in}{2.181507in}}{\pgfqpoint{1.103652in}{2.173606in}}{\pgfqpoint{1.103652in}{2.165370in}}%
\pgfpathcurveto{\pgfqpoint{1.103652in}{2.157134in}}{\pgfqpoint{1.106924in}{2.149234in}}{\pgfqpoint{1.112748in}{2.143410in}}%
\pgfpathcurveto{\pgfqpoint{1.118572in}{2.137586in}}{\pgfqpoint{1.126472in}{2.134314in}}{\pgfqpoint{1.134708in}{2.134314in}}%
\pgfpathclose%
\pgfusepath{stroke,fill}%
\end{pgfscope}%
\begin{pgfscope}%
\pgfpathrectangle{\pgfqpoint{0.100000in}{0.212622in}}{\pgfqpoint{3.696000in}{3.696000in}}%
\pgfusepath{clip}%
\pgfsetbuttcap%
\pgfsetroundjoin%
\definecolor{currentfill}{rgb}{0.121569,0.466667,0.705882}%
\pgfsetfillcolor{currentfill}%
\pgfsetfillopacity{0.588398}%
\pgfsetlinewidth{1.003750pt}%
\definecolor{currentstroke}{rgb}{0.121569,0.466667,0.705882}%
\pgfsetstrokecolor{currentstroke}%
\pgfsetstrokeopacity{0.588398}%
\pgfsetdash{}{0pt}%
\pgfpathmoveto{\pgfqpoint{1.133596in}{2.134255in}}%
\pgfpathcurveto{\pgfqpoint{1.141832in}{2.134255in}}{\pgfqpoint{1.149733in}{2.137527in}}{\pgfqpoint{1.155556in}{2.143351in}}%
\pgfpathcurveto{\pgfqpoint{1.161380in}{2.149175in}}{\pgfqpoint{1.164653in}{2.157075in}}{\pgfqpoint{1.164653in}{2.165311in}}%
\pgfpathcurveto{\pgfqpoint{1.164653in}{2.173548in}}{\pgfqpoint{1.161380in}{2.181448in}}{\pgfqpoint{1.155556in}{2.187272in}}%
\pgfpathcurveto{\pgfqpoint{1.149733in}{2.193095in}}{\pgfqpoint{1.141832in}{2.196368in}}{\pgfqpoint{1.133596in}{2.196368in}}%
\pgfpathcurveto{\pgfqpoint{1.125360in}{2.196368in}}{\pgfqpoint{1.117460in}{2.193095in}}{\pgfqpoint{1.111636in}{2.187272in}}%
\pgfpathcurveto{\pgfqpoint{1.105812in}{2.181448in}}{\pgfqpoint{1.102540in}{2.173548in}}{\pgfqpoint{1.102540in}{2.165311in}}%
\pgfpathcurveto{\pgfqpoint{1.102540in}{2.157075in}}{\pgfqpoint{1.105812in}{2.149175in}}{\pgfqpoint{1.111636in}{2.143351in}}%
\pgfpathcurveto{\pgfqpoint{1.117460in}{2.137527in}}{\pgfqpoint{1.125360in}{2.134255in}}{\pgfqpoint{1.133596in}{2.134255in}}%
\pgfpathclose%
\pgfusepath{stroke,fill}%
\end{pgfscope}%
\begin{pgfscope}%
\pgfpathrectangle{\pgfqpoint{0.100000in}{0.212622in}}{\pgfqpoint{3.696000in}{3.696000in}}%
\pgfusepath{clip}%
\pgfsetbuttcap%
\pgfsetroundjoin%
\definecolor{currentfill}{rgb}{0.121569,0.466667,0.705882}%
\pgfsetfillcolor{currentfill}%
\pgfsetfillopacity{0.588836}%
\pgfsetlinewidth{1.003750pt}%
\definecolor{currentstroke}{rgb}{0.121569,0.466667,0.705882}%
\pgfsetstrokecolor{currentstroke}%
\pgfsetstrokeopacity{0.588836}%
\pgfsetdash{}{0pt}%
\pgfpathmoveto{\pgfqpoint{3.180474in}{1.778282in}}%
\pgfpathcurveto{\pgfqpoint{3.188710in}{1.778282in}}{\pgfqpoint{3.196610in}{1.781555in}}{\pgfqpoint{3.202434in}{1.787378in}}%
\pgfpathcurveto{\pgfqpoint{3.208258in}{1.793202in}}{\pgfqpoint{3.211530in}{1.801102in}}{\pgfqpoint{3.211530in}{1.809339in}}%
\pgfpathcurveto{\pgfqpoint{3.211530in}{1.817575in}}{\pgfqpoint{3.208258in}{1.825475in}}{\pgfqpoint{3.202434in}{1.831299in}}%
\pgfpathcurveto{\pgfqpoint{3.196610in}{1.837123in}}{\pgfqpoint{3.188710in}{1.840395in}}{\pgfqpoint{3.180474in}{1.840395in}}%
\pgfpathcurveto{\pgfqpoint{3.172238in}{1.840395in}}{\pgfqpoint{3.164338in}{1.837123in}}{\pgfqpoint{3.158514in}{1.831299in}}%
\pgfpathcurveto{\pgfqpoint{3.152690in}{1.825475in}}{\pgfqpoint{3.149417in}{1.817575in}}{\pgfqpoint{3.149417in}{1.809339in}}%
\pgfpathcurveto{\pgfqpoint{3.149417in}{1.801102in}}{\pgfqpoint{3.152690in}{1.793202in}}{\pgfqpoint{3.158514in}{1.787378in}}%
\pgfpathcurveto{\pgfqpoint{3.164338in}{1.781555in}}{\pgfqpoint{3.172238in}{1.778282in}}{\pgfqpoint{3.180474in}{1.778282in}}%
\pgfpathclose%
\pgfusepath{stroke,fill}%
\end{pgfscope}%
\begin{pgfscope}%
\pgfpathrectangle{\pgfqpoint{0.100000in}{0.212622in}}{\pgfqpoint{3.696000in}{3.696000in}}%
\pgfusepath{clip}%
\pgfsetbuttcap%
\pgfsetroundjoin%
\definecolor{currentfill}{rgb}{0.121569,0.466667,0.705882}%
\pgfsetfillcolor{currentfill}%
\pgfsetfillopacity{0.589420}%
\pgfsetlinewidth{1.003750pt}%
\definecolor{currentstroke}{rgb}{0.121569,0.466667,0.705882}%
\pgfsetstrokecolor{currentstroke}%
\pgfsetstrokeopacity{0.589420}%
\pgfsetdash{}{0pt}%
\pgfpathmoveto{\pgfqpoint{1.131491in}{2.134445in}}%
\pgfpathcurveto{\pgfqpoint{1.139727in}{2.134445in}}{\pgfqpoint{1.147627in}{2.137718in}}{\pgfqpoint{1.153451in}{2.143542in}}%
\pgfpathcurveto{\pgfqpoint{1.159275in}{2.149366in}}{\pgfqpoint{1.162547in}{2.157266in}}{\pgfqpoint{1.162547in}{2.165502in}}%
\pgfpathcurveto{\pgfqpoint{1.162547in}{2.173738in}}{\pgfqpoint{1.159275in}{2.181638in}}{\pgfqpoint{1.153451in}{2.187462in}}%
\pgfpathcurveto{\pgfqpoint{1.147627in}{2.193286in}}{\pgfqpoint{1.139727in}{2.196558in}}{\pgfqpoint{1.131491in}{2.196558in}}%
\pgfpathcurveto{\pgfqpoint{1.123255in}{2.196558in}}{\pgfqpoint{1.115354in}{2.193286in}}{\pgfqpoint{1.109531in}{2.187462in}}%
\pgfpathcurveto{\pgfqpoint{1.103707in}{2.181638in}}{\pgfqpoint{1.100434in}{2.173738in}}{\pgfqpoint{1.100434in}{2.165502in}}%
\pgfpathcurveto{\pgfqpoint{1.100434in}{2.157266in}}{\pgfqpoint{1.103707in}{2.149366in}}{\pgfqpoint{1.109531in}{2.143542in}}%
\pgfpathcurveto{\pgfqpoint{1.115354in}{2.137718in}}{\pgfqpoint{1.123255in}{2.134445in}}{\pgfqpoint{1.131491in}{2.134445in}}%
\pgfpathclose%
\pgfusepath{stroke,fill}%
\end{pgfscope}%
\begin{pgfscope}%
\pgfpathrectangle{\pgfqpoint{0.100000in}{0.212622in}}{\pgfqpoint{3.696000in}{3.696000in}}%
\pgfusepath{clip}%
\pgfsetbuttcap%
\pgfsetroundjoin%
\definecolor{currentfill}{rgb}{0.121569,0.466667,0.705882}%
\pgfsetfillcolor{currentfill}%
\pgfsetfillopacity{0.590220}%
\pgfsetlinewidth{1.003750pt}%
\definecolor{currentstroke}{rgb}{0.121569,0.466667,0.705882}%
\pgfsetstrokecolor{currentstroke}%
\pgfsetstrokeopacity{0.590220}%
\pgfsetdash{}{0pt}%
\pgfpathmoveto{\pgfqpoint{1.129443in}{2.134586in}}%
\pgfpathcurveto{\pgfqpoint{1.137679in}{2.134586in}}{\pgfqpoint{1.145579in}{2.137859in}}{\pgfqpoint{1.151403in}{2.143683in}}%
\pgfpathcurveto{\pgfqpoint{1.157227in}{2.149507in}}{\pgfqpoint{1.160499in}{2.157407in}}{\pgfqpoint{1.160499in}{2.165643in}}%
\pgfpathcurveto{\pgfqpoint{1.160499in}{2.173879in}}{\pgfqpoint{1.157227in}{2.181779in}}{\pgfqpoint{1.151403in}{2.187603in}}%
\pgfpathcurveto{\pgfqpoint{1.145579in}{2.193427in}}{\pgfqpoint{1.137679in}{2.196699in}}{\pgfqpoint{1.129443in}{2.196699in}}%
\pgfpathcurveto{\pgfqpoint{1.121206in}{2.196699in}}{\pgfqpoint{1.113306in}{2.193427in}}{\pgfqpoint{1.107482in}{2.187603in}}%
\pgfpathcurveto{\pgfqpoint{1.101658in}{2.181779in}}{\pgfqpoint{1.098386in}{2.173879in}}{\pgfqpoint{1.098386in}{2.165643in}}%
\pgfpathcurveto{\pgfqpoint{1.098386in}{2.157407in}}{\pgfqpoint{1.101658in}{2.149507in}}{\pgfqpoint{1.107482in}{2.143683in}}%
\pgfpathcurveto{\pgfqpoint{1.113306in}{2.137859in}}{\pgfqpoint{1.121206in}{2.134586in}}{\pgfqpoint{1.129443in}{2.134586in}}%
\pgfpathclose%
\pgfusepath{stroke,fill}%
\end{pgfscope}%
\begin{pgfscope}%
\pgfpathrectangle{\pgfqpoint{0.100000in}{0.212622in}}{\pgfqpoint{3.696000in}{3.696000in}}%
\pgfusepath{clip}%
\pgfsetbuttcap%
\pgfsetroundjoin%
\definecolor{currentfill}{rgb}{0.121569,0.466667,0.705882}%
\pgfsetfillcolor{currentfill}%
\pgfsetfillopacity{0.590750}%
\pgfsetlinewidth{1.003750pt}%
\definecolor{currentstroke}{rgb}{0.121569,0.466667,0.705882}%
\pgfsetstrokecolor{currentstroke}%
\pgfsetstrokeopacity{0.590750}%
\pgfsetdash{}{0pt}%
\pgfpathmoveto{\pgfqpoint{3.176668in}{1.778885in}}%
\pgfpathcurveto{\pgfqpoint{3.184904in}{1.778885in}}{\pgfqpoint{3.192804in}{1.782157in}}{\pgfqpoint{3.198628in}{1.787981in}}%
\pgfpathcurveto{\pgfqpoint{3.204452in}{1.793805in}}{\pgfqpoint{3.207724in}{1.801705in}}{\pgfqpoint{3.207724in}{1.809941in}}%
\pgfpathcurveto{\pgfqpoint{3.207724in}{1.818177in}}{\pgfqpoint{3.204452in}{1.826077in}}{\pgfqpoint{3.198628in}{1.831901in}}%
\pgfpathcurveto{\pgfqpoint{3.192804in}{1.837725in}}{\pgfqpoint{3.184904in}{1.840998in}}{\pgfqpoint{3.176668in}{1.840998in}}%
\pgfpathcurveto{\pgfqpoint{3.168431in}{1.840998in}}{\pgfqpoint{3.160531in}{1.837725in}}{\pgfqpoint{3.154707in}{1.831901in}}%
\pgfpathcurveto{\pgfqpoint{3.148884in}{1.826077in}}{\pgfqpoint{3.145611in}{1.818177in}}{\pgfqpoint{3.145611in}{1.809941in}}%
\pgfpathcurveto{\pgfqpoint{3.145611in}{1.801705in}}{\pgfqpoint{3.148884in}{1.793805in}}{\pgfqpoint{3.154707in}{1.787981in}}%
\pgfpathcurveto{\pgfqpoint{3.160531in}{1.782157in}}{\pgfqpoint{3.168431in}{1.778885in}}{\pgfqpoint{3.176668in}{1.778885in}}%
\pgfpathclose%
\pgfusepath{stroke,fill}%
\end{pgfscope}%
\begin{pgfscope}%
\pgfpathrectangle{\pgfqpoint{0.100000in}{0.212622in}}{\pgfqpoint{3.696000in}{3.696000in}}%
\pgfusepath{clip}%
\pgfsetbuttcap%
\pgfsetroundjoin%
\definecolor{currentfill}{rgb}{0.121569,0.466667,0.705882}%
\pgfsetfillcolor{currentfill}%
\pgfsetfillopacity{0.591747}%
\pgfsetlinewidth{1.003750pt}%
\definecolor{currentstroke}{rgb}{0.121569,0.466667,0.705882}%
\pgfsetstrokecolor{currentstroke}%
\pgfsetstrokeopacity{0.591747}%
\pgfsetdash{}{0pt}%
\pgfpathmoveto{\pgfqpoint{3.174120in}{1.779430in}}%
\pgfpathcurveto{\pgfqpoint{3.182356in}{1.779430in}}{\pgfqpoint{3.190256in}{1.782703in}}{\pgfqpoint{3.196080in}{1.788527in}}%
\pgfpathcurveto{\pgfqpoint{3.201904in}{1.794350in}}{\pgfqpoint{3.205176in}{1.802251in}}{\pgfqpoint{3.205176in}{1.810487in}}%
\pgfpathcurveto{\pgfqpoint{3.205176in}{1.818723in}}{\pgfqpoint{3.201904in}{1.826623in}}{\pgfqpoint{3.196080in}{1.832447in}}%
\pgfpathcurveto{\pgfqpoint{3.190256in}{1.838271in}}{\pgfqpoint{3.182356in}{1.841543in}}{\pgfqpoint{3.174120in}{1.841543in}}%
\pgfpathcurveto{\pgfqpoint{3.165883in}{1.841543in}}{\pgfqpoint{3.157983in}{1.838271in}}{\pgfqpoint{3.152159in}{1.832447in}}%
\pgfpathcurveto{\pgfqpoint{3.146335in}{1.826623in}}{\pgfqpoint{3.143063in}{1.818723in}}{\pgfqpoint{3.143063in}{1.810487in}}%
\pgfpathcurveto{\pgfqpoint{3.143063in}{1.802251in}}{\pgfqpoint{3.146335in}{1.794350in}}{\pgfqpoint{3.152159in}{1.788527in}}%
\pgfpathcurveto{\pgfqpoint{3.157983in}{1.782703in}}{\pgfqpoint{3.165883in}{1.779430in}}{\pgfqpoint{3.174120in}{1.779430in}}%
\pgfpathclose%
\pgfusepath{stroke,fill}%
\end{pgfscope}%
\begin{pgfscope}%
\pgfpathrectangle{\pgfqpoint{0.100000in}{0.212622in}}{\pgfqpoint{3.696000in}{3.696000in}}%
\pgfusepath{clip}%
\pgfsetbuttcap%
\pgfsetroundjoin%
\definecolor{currentfill}{rgb}{0.121569,0.466667,0.705882}%
\pgfsetfillcolor{currentfill}%
\pgfsetfillopacity{0.591864}%
\pgfsetlinewidth{1.003750pt}%
\definecolor{currentstroke}{rgb}{0.121569,0.466667,0.705882}%
\pgfsetstrokecolor{currentstroke}%
\pgfsetstrokeopacity{0.591864}%
\pgfsetdash{}{0pt}%
\pgfpathmoveto{\pgfqpoint{1.127129in}{2.134871in}}%
\pgfpathcurveto{\pgfqpoint{1.135365in}{2.134871in}}{\pgfqpoint{1.143265in}{2.138143in}}{\pgfqpoint{1.149089in}{2.143967in}}%
\pgfpathcurveto{\pgfqpoint{1.154913in}{2.149791in}}{\pgfqpoint{1.158185in}{2.157691in}}{\pgfqpoint{1.158185in}{2.165927in}}%
\pgfpathcurveto{\pgfqpoint{1.158185in}{2.174164in}}{\pgfqpoint{1.154913in}{2.182064in}}{\pgfqpoint{1.149089in}{2.187888in}}%
\pgfpathcurveto{\pgfqpoint{1.143265in}{2.193712in}}{\pgfqpoint{1.135365in}{2.196984in}}{\pgfqpoint{1.127129in}{2.196984in}}%
\pgfpathcurveto{\pgfqpoint{1.118893in}{2.196984in}}{\pgfqpoint{1.110993in}{2.193712in}}{\pgfqpoint{1.105169in}{2.187888in}}%
\pgfpathcurveto{\pgfqpoint{1.099345in}{2.182064in}}{\pgfqpoint{1.096072in}{2.174164in}}{\pgfqpoint{1.096072in}{2.165927in}}%
\pgfpathcurveto{\pgfqpoint{1.096072in}{2.157691in}}{\pgfqpoint{1.099345in}{2.149791in}}{\pgfqpoint{1.105169in}{2.143967in}}%
\pgfpathcurveto{\pgfqpoint{1.110993in}{2.138143in}}{\pgfqpoint{1.118893in}{2.134871in}}{\pgfqpoint{1.127129in}{2.134871in}}%
\pgfpathclose%
\pgfusepath{stroke,fill}%
\end{pgfscope}%
\begin{pgfscope}%
\pgfpathrectangle{\pgfqpoint{0.100000in}{0.212622in}}{\pgfqpoint{3.696000in}{3.696000in}}%
\pgfusepath{clip}%
\pgfsetbuttcap%
\pgfsetroundjoin%
\definecolor{currentfill}{rgb}{0.121569,0.466667,0.705882}%
\pgfsetfillcolor{currentfill}%
\pgfsetfillopacity{0.593068}%
\pgfsetlinewidth{1.003750pt}%
\definecolor{currentstroke}{rgb}{0.121569,0.466667,0.705882}%
\pgfsetstrokecolor{currentstroke}%
\pgfsetstrokeopacity{0.593068}%
\pgfsetdash{}{0pt}%
\pgfpathmoveto{\pgfqpoint{1.123768in}{2.135190in}}%
\pgfpathcurveto{\pgfqpoint{1.132005in}{2.135190in}}{\pgfqpoint{1.139905in}{2.138463in}}{\pgfqpoint{1.145729in}{2.144287in}}%
\pgfpathcurveto{\pgfqpoint{1.151553in}{2.150111in}}{\pgfqpoint{1.154825in}{2.158011in}}{\pgfqpoint{1.154825in}{2.166247in}}%
\pgfpathcurveto{\pgfqpoint{1.154825in}{2.174483in}}{\pgfqpoint{1.151553in}{2.182383in}}{\pgfqpoint{1.145729in}{2.188207in}}%
\pgfpathcurveto{\pgfqpoint{1.139905in}{2.194031in}}{\pgfqpoint{1.132005in}{2.197303in}}{\pgfqpoint{1.123768in}{2.197303in}}%
\pgfpathcurveto{\pgfqpoint{1.115532in}{2.197303in}}{\pgfqpoint{1.107632in}{2.194031in}}{\pgfqpoint{1.101808in}{2.188207in}}%
\pgfpathcurveto{\pgfqpoint{1.095984in}{2.182383in}}{\pgfqpoint{1.092712in}{2.174483in}}{\pgfqpoint{1.092712in}{2.166247in}}%
\pgfpathcurveto{\pgfqpoint{1.092712in}{2.158011in}}{\pgfqpoint{1.095984in}{2.150111in}}{\pgfqpoint{1.101808in}{2.144287in}}%
\pgfpathcurveto{\pgfqpoint{1.107632in}{2.138463in}}{\pgfqpoint{1.115532in}{2.135190in}}{\pgfqpoint{1.123768in}{2.135190in}}%
\pgfpathclose%
\pgfusepath{stroke,fill}%
\end{pgfscope}%
\begin{pgfscope}%
\pgfpathrectangle{\pgfqpoint{0.100000in}{0.212622in}}{\pgfqpoint{3.696000in}{3.696000in}}%
\pgfusepath{clip}%
\pgfsetbuttcap%
\pgfsetroundjoin%
\definecolor{currentfill}{rgb}{0.121569,0.466667,0.705882}%
\pgfsetfillcolor{currentfill}%
\pgfsetfillopacity{0.593077}%
\pgfsetlinewidth{1.003750pt}%
\definecolor{currentstroke}{rgb}{0.121569,0.466667,0.705882}%
\pgfsetstrokecolor{currentstroke}%
\pgfsetstrokeopacity{0.593077}%
\pgfsetdash{}{0pt}%
\pgfpathmoveto{\pgfqpoint{3.171339in}{1.779840in}}%
\pgfpathcurveto{\pgfqpoint{3.179575in}{1.779840in}}{\pgfqpoint{3.187475in}{1.783112in}}{\pgfqpoint{3.193299in}{1.788936in}}%
\pgfpathcurveto{\pgfqpoint{3.199123in}{1.794760in}}{\pgfqpoint{3.202396in}{1.802660in}}{\pgfqpoint{3.202396in}{1.810896in}}%
\pgfpathcurveto{\pgfqpoint{3.202396in}{1.819132in}}{\pgfqpoint{3.199123in}{1.827033in}}{\pgfqpoint{3.193299in}{1.832856in}}%
\pgfpathcurveto{\pgfqpoint{3.187475in}{1.838680in}}{\pgfqpoint{3.179575in}{1.841953in}}{\pgfqpoint{3.171339in}{1.841953in}}%
\pgfpathcurveto{\pgfqpoint{3.163103in}{1.841953in}}{\pgfqpoint{3.155203in}{1.838680in}}{\pgfqpoint{3.149379in}{1.832856in}}%
\pgfpathcurveto{\pgfqpoint{3.143555in}{1.827033in}}{\pgfqpoint{3.140283in}{1.819132in}}{\pgfqpoint{3.140283in}{1.810896in}}%
\pgfpathcurveto{\pgfqpoint{3.140283in}{1.802660in}}{\pgfqpoint{3.143555in}{1.794760in}}{\pgfqpoint{3.149379in}{1.788936in}}%
\pgfpathcurveto{\pgfqpoint{3.155203in}{1.783112in}}{\pgfqpoint{3.163103in}{1.779840in}}{\pgfqpoint{3.171339in}{1.779840in}}%
\pgfpathclose%
\pgfusepath{stroke,fill}%
\end{pgfscope}%
\begin{pgfscope}%
\pgfpathrectangle{\pgfqpoint{0.100000in}{0.212622in}}{\pgfqpoint{3.696000in}{3.696000in}}%
\pgfusepath{clip}%
\pgfsetbuttcap%
\pgfsetroundjoin%
\definecolor{currentfill}{rgb}{0.121569,0.466667,0.705882}%
\pgfsetfillcolor{currentfill}%
\pgfsetfillopacity{0.593886}%
\pgfsetlinewidth{1.003750pt}%
\definecolor{currentstroke}{rgb}{0.121569,0.466667,0.705882}%
\pgfsetstrokecolor{currentstroke}%
\pgfsetstrokeopacity{0.593886}%
\pgfsetdash{}{0pt}%
\pgfpathmoveto{\pgfqpoint{3.170608in}{1.779899in}}%
\pgfpathcurveto{\pgfqpoint{3.178844in}{1.779899in}}{\pgfqpoint{3.186744in}{1.783171in}}{\pgfqpoint{3.192568in}{1.788995in}}%
\pgfpathcurveto{\pgfqpoint{3.198392in}{1.794819in}}{\pgfqpoint{3.201665in}{1.802719in}}{\pgfqpoint{3.201665in}{1.810955in}}%
\pgfpathcurveto{\pgfqpoint{3.201665in}{1.819192in}}{\pgfqpoint{3.198392in}{1.827092in}}{\pgfqpoint{3.192568in}{1.832915in}}%
\pgfpathcurveto{\pgfqpoint{3.186744in}{1.838739in}}{\pgfqpoint{3.178844in}{1.842012in}}{\pgfqpoint{3.170608in}{1.842012in}}%
\pgfpathcurveto{\pgfqpoint{3.162372in}{1.842012in}}{\pgfqpoint{3.154472in}{1.838739in}}{\pgfqpoint{3.148648in}{1.832915in}}%
\pgfpathcurveto{\pgfqpoint{3.142824in}{1.827092in}}{\pgfqpoint{3.139552in}{1.819192in}}{\pgfqpoint{3.139552in}{1.810955in}}%
\pgfpathcurveto{\pgfqpoint{3.139552in}{1.802719in}}{\pgfqpoint{3.142824in}{1.794819in}}{\pgfqpoint{3.148648in}{1.788995in}}%
\pgfpathcurveto{\pgfqpoint{3.154472in}{1.783171in}}{\pgfqpoint{3.162372in}{1.779899in}}{\pgfqpoint{3.170608in}{1.779899in}}%
\pgfpathclose%
\pgfusepath{stroke,fill}%
\end{pgfscope}%
\begin{pgfscope}%
\pgfpathrectangle{\pgfqpoint{0.100000in}{0.212622in}}{\pgfqpoint{3.696000in}{3.696000in}}%
\pgfusepath{clip}%
\pgfsetbuttcap%
\pgfsetroundjoin%
\definecolor{currentfill}{rgb}{0.121569,0.466667,0.705882}%
\pgfsetfillcolor{currentfill}%
\pgfsetfillopacity{0.594044}%
\pgfsetlinewidth{1.003750pt}%
\definecolor{currentstroke}{rgb}{0.121569,0.466667,0.705882}%
\pgfsetstrokecolor{currentstroke}%
\pgfsetstrokeopacity{0.594044}%
\pgfsetdash{}{0pt}%
\pgfpathmoveto{\pgfqpoint{1.127465in}{2.136556in}}%
\pgfpathcurveto{\pgfqpoint{1.135702in}{2.136556in}}{\pgfqpoint{1.143602in}{2.139828in}}{\pgfqpoint{1.149426in}{2.145652in}}%
\pgfpathcurveto{\pgfqpoint{1.155249in}{2.151476in}}{\pgfqpoint{1.158522in}{2.159376in}}{\pgfqpoint{1.158522in}{2.167612in}}%
\pgfpathcurveto{\pgfqpoint{1.158522in}{2.175848in}}{\pgfqpoint{1.155249in}{2.183748in}}{\pgfqpoint{1.149426in}{2.189572in}}%
\pgfpathcurveto{\pgfqpoint{1.143602in}{2.195396in}}{\pgfqpoint{1.135702in}{2.198669in}}{\pgfqpoint{1.127465in}{2.198669in}}%
\pgfpathcurveto{\pgfqpoint{1.119229in}{2.198669in}}{\pgfqpoint{1.111329in}{2.195396in}}{\pgfqpoint{1.105505in}{2.189572in}}%
\pgfpathcurveto{\pgfqpoint{1.099681in}{2.183748in}}{\pgfqpoint{1.096409in}{2.175848in}}{\pgfqpoint{1.096409in}{2.167612in}}%
\pgfpathcurveto{\pgfqpoint{1.096409in}{2.159376in}}{\pgfqpoint{1.099681in}{2.151476in}}{\pgfqpoint{1.105505in}{2.145652in}}%
\pgfpathcurveto{\pgfqpoint{1.111329in}{2.139828in}}{\pgfqpoint{1.119229in}{2.136556in}}{\pgfqpoint{1.127465in}{2.136556in}}%
\pgfpathclose%
\pgfusepath{stroke,fill}%
\end{pgfscope}%
\begin{pgfscope}%
\pgfpathrectangle{\pgfqpoint{0.100000in}{0.212622in}}{\pgfqpoint{3.696000in}{3.696000in}}%
\pgfusepath{clip}%
\pgfsetbuttcap%
\pgfsetroundjoin%
\definecolor{currentfill}{rgb}{0.121569,0.466667,0.705882}%
\pgfsetfillcolor{currentfill}%
\pgfsetfillopacity{0.595026}%
\pgfsetlinewidth{1.003750pt}%
\definecolor{currentstroke}{rgb}{0.121569,0.466667,0.705882}%
\pgfsetstrokecolor{currentstroke}%
\pgfsetstrokeopacity{0.595026}%
\pgfsetdash{}{0pt}%
\pgfpathmoveto{\pgfqpoint{3.168439in}{1.780281in}}%
\pgfpathcurveto{\pgfqpoint{3.176675in}{1.780281in}}{\pgfqpoint{3.184575in}{1.783554in}}{\pgfqpoint{3.190399in}{1.789377in}}%
\pgfpathcurveto{\pgfqpoint{3.196223in}{1.795201in}}{\pgfqpoint{3.199496in}{1.803101in}}{\pgfqpoint{3.199496in}{1.811338in}}%
\pgfpathcurveto{\pgfqpoint{3.199496in}{1.819574in}}{\pgfqpoint{3.196223in}{1.827474in}}{\pgfqpoint{3.190399in}{1.833298in}}%
\pgfpathcurveto{\pgfqpoint{3.184575in}{1.839122in}}{\pgfqpoint{3.176675in}{1.842394in}}{\pgfqpoint{3.168439in}{1.842394in}}%
\pgfpathcurveto{\pgfqpoint{3.160203in}{1.842394in}}{\pgfqpoint{3.152303in}{1.839122in}}{\pgfqpoint{3.146479in}{1.833298in}}%
\pgfpathcurveto{\pgfqpoint{3.140655in}{1.827474in}}{\pgfqpoint{3.137383in}{1.819574in}}{\pgfqpoint{3.137383in}{1.811338in}}%
\pgfpathcurveto{\pgfqpoint{3.137383in}{1.803101in}}{\pgfqpoint{3.140655in}{1.795201in}}{\pgfqpoint{3.146479in}{1.789377in}}%
\pgfpathcurveto{\pgfqpoint{3.152303in}{1.783554in}}{\pgfqpoint{3.160203in}{1.780281in}}{\pgfqpoint{3.168439in}{1.780281in}}%
\pgfpathclose%
\pgfusepath{stroke,fill}%
\end{pgfscope}%
\begin{pgfscope}%
\pgfpathrectangle{\pgfqpoint{0.100000in}{0.212622in}}{\pgfqpoint{3.696000in}{3.696000in}}%
\pgfusepath{clip}%
\pgfsetbuttcap%
\pgfsetroundjoin%
\definecolor{currentfill}{rgb}{0.121569,0.466667,0.705882}%
\pgfsetfillcolor{currentfill}%
\pgfsetfillopacity{0.595112}%
\pgfsetlinewidth{1.003750pt}%
\definecolor{currentstroke}{rgb}{0.121569,0.466667,0.705882}%
\pgfsetstrokecolor{currentstroke}%
\pgfsetstrokeopacity{0.595112}%
\pgfsetdash{}{0pt}%
\pgfpathmoveto{\pgfqpoint{1.124994in}{2.136755in}}%
\pgfpathcurveto{\pgfqpoint{1.133230in}{2.136755in}}{\pgfqpoint{1.141130in}{2.140028in}}{\pgfqpoint{1.146954in}{2.145851in}}%
\pgfpathcurveto{\pgfqpoint{1.152778in}{2.151675in}}{\pgfqpoint{1.156051in}{2.159575in}}{\pgfqpoint{1.156051in}{2.167812in}}%
\pgfpathcurveto{\pgfqpoint{1.156051in}{2.176048in}}{\pgfqpoint{1.152778in}{2.183948in}}{\pgfqpoint{1.146954in}{2.189772in}}%
\pgfpathcurveto{\pgfqpoint{1.141130in}{2.195596in}}{\pgfqpoint{1.133230in}{2.198868in}}{\pgfqpoint{1.124994in}{2.198868in}}%
\pgfpathcurveto{\pgfqpoint{1.116758in}{2.198868in}}{\pgfqpoint{1.108858in}{2.195596in}}{\pgfqpoint{1.103034in}{2.189772in}}%
\pgfpathcurveto{\pgfqpoint{1.097210in}{2.183948in}}{\pgfqpoint{1.093938in}{2.176048in}}{\pgfqpoint{1.093938in}{2.167812in}}%
\pgfpathcurveto{\pgfqpoint{1.093938in}{2.159575in}}{\pgfqpoint{1.097210in}{2.151675in}}{\pgfqpoint{1.103034in}{2.145851in}}%
\pgfpathcurveto{\pgfqpoint{1.108858in}{2.140028in}}{\pgfqpoint{1.116758in}{2.136755in}}{\pgfqpoint{1.124994in}{2.136755in}}%
\pgfpathclose%
\pgfusepath{stroke,fill}%
\end{pgfscope}%
\begin{pgfscope}%
\pgfpathrectangle{\pgfqpoint{0.100000in}{0.212622in}}{\pgfqpoint{3.696000in}{3.696000in}}%
\pgfusepath{clip}%
\pgfsetbuttcap%
\pgfsetroundjoin%
\definecolor{currentfill}{rgb}{0.121569,0.466667,0.705882}%
\pgfsetfillcolor{currentfill}%
\pgfsetfillopacity{0.595620}%
\pgfsetlinewidth{1.003750pt}%
\definecolor{currentstroke}{rgb}{0.121569,0.466667,0.705882}%
\pgfsetstrokecolor{currentstroke}%
\pgfsetstrokeopacity{0.595620}%
\pgfsetdash{}{0pt}%
\pgfpathmoveto{\pgfqpoint{3.167023in}{1.780537in}}%
\pgfpathcurveto{\pgfqpoint{3.175259in}{1.780537in}}{\pgfqpoint{3.183159in}{1.783810in}}{\pgfqpoint{3.188983in}{1.789634in}}%
\pgfpathcurveto{\pgfqpoint{3.194807in}{1.795458in}}{\pgfqpoint{3.198079in}{1.803358in}}{\pgfqpoint{3.198079in}{1.811594in}}%
\pgfpathcurveto{\pgfqpoint{3.198079in}{1.819830in}}{\pgfqpoint{3.194807in}{1.827730in}}{\pgfqpoint{3.188983in}{1.833554in}}%
\pgfpathcurveto{\pgfqpoint{3.183159in}{1.839378in}}{\pgfqpoint{3.175259in}{1.842650in}}{\pgfqpoint{3.167023in}{1.842650in}}%
\pgfpathcurveto{\pgfqpoint{3.158787in}{1.842650in}}{\pgfqpoint{3.150886in}{1.839378in}}{\pgfqpoint{3.145063in}{1.833554in}}%
\pgfpathcurveto{\pgfqpoint{3.139239in}{1.827730in}}{\pgfqpoint{3.135966in}{1.819830in}}{\pgfqpoint{3.135966in}{1.811594in}}%
\pgfpathcurveto{\pgfqpoint{3.135966in}{1.803358in}}{\pgfqpoint{3.139239in}{1.795458in}}{\pgfqpoint{3.145063in}{1.789634in}}%
\pgfpathcurveto{\pgfqpoint{3.150886in}{1.783810in}}{\pgfqpoint{3.158787in}{1.780537in}}{\pgfqpoint{3.167023in}{1.780537in}}%
\pgfpathclose%
\pgfusepath{stroke,fill}%
\end{pgfscope}%
\begin{pgfscope}%
\pgfpathrectangle{\pgfqpoint{0.100000in}{0.212622in}}{\pgfqpoint{3.696000in}{3.696000in}}%
\pgfusepath{clip}%
\pgfsetbuttcap%
\pgfsetroundjoin%
\definecolor{currentfill}{rgb}{0.121569,0.466667,0.705882}%
\pgfsetfillcolor{currentfill}%
\pgfsetfillopacity{0.596607}%
\pgfsetlinewidth{1.003750pt}%
\definecolor{currentstroke}{rgb}{0.121569,0.466667,0.705882}%
\pgfsetstrokecolor{currentstroke}%
\pgfsetstrokeopacity{0.596607}%
\pgfsetdash{}{0pt}%
\pgfpathmoveto{\pgfqpoint{3.165722in}{1.780621in}}%
\pgfpathcurveto{\pgfqpoint{3.173958in}{1.780621in}}{\pgfqpoint{3.181858in}{1.783894in}}{\pgfqpoint{3.187682in}{1.789718in}}%
\pgfpathcurveto{\pgfqpoint{3.193506in}{1.795541in}}{\pgfqpoint{3.196779in}{1.803442in}}{\pgfqpoint{3.196779in}{1.811678in}}%
\pgfpathcurveto{\pgfqpoint{3.196779in}{1.819914in}}{\pgfqpoint{3.193506in}{1.827814in}}{\pgfqpoint{3.187682in}{1.833638in}}%
\pgfpathcurveto{\pgfqpoint{3.181858in}{1.839462in}}{\pgfqpoint{3.173958in}{1.842734in}}{\pgfqpoint{3.165722in}{1.842734in}}%
\pgfpathcurveto{\pgfqpoint{3.157486in}{1.842734in}}{\pgfqpoint{3.149586in}{1.839462in}}{\pgfqpoint{3.143762in}{1.833638in}}%
\pgfpathcurveto{\pgfqpoint{3.137938in}{1.827814in}}{\pgfqpoint{3.134666in}{1.819914in}}{\pgfqpoint{3.134666in}{1.811678in}}%
\pgfpathcurveto{\pgfqpoint{3.134666in}{1.803442in}}{\pgfqpoint{3.137938in}{1.795541in}}{\pgfqpoint{3.143762in}{1.789718in}}%
\pgfpathcurveto{\pgfqpoint{3.149586in}{1.783894in}}{\pgfqpoint{3.157486in}{1.780621in}}{\pgfqpoint{3.165722in}{1.780621in}}%
\pgfpathclose%
\pgfusepath{stroke,fill}%
\end{pgfscope}%
\begin{pgfscope}%
\pgfpathrectangle{\pgfqpoint{0.100000in}{0.212622in}}{\pgfqpoint{3.696000in}{3.696000in}}%
\pgfusepath{clip}%
\pgfsetbuttcap%
\pgfsetroundjoin%
\definecolor{currentfill}{rgb}{0.121569,0.466667,0.705882}%
\pgfsetfillcolor{currentfill}%
\pgfsetfillopacity{0.597027}%
\pgfsetlinewidth{1.003750pt}%
\definecolor{currentstroke}{rgb}{0.121569,0.466667,0.705882}%
\pgfsetstrokecolor{currentstroke}%
\pgfsetstrokeopacity{0.597027}%
\pgfsetdash{}{0pt}%
\pgfpathmoveto{\pgfqpoint{1.120418in}{2.136995in}}%
\pgfpathcurveto{\pgfqpoint{1.128655in}{2.136995in}}{\pgfqpoint{1.136555in}{2.140267in}}{\pgfqpoint{1.142379in}{2.146091in}}%
\pgfpathcurveto{\pgfqpoint{1.148203in}{2.151915in}}{\pgfqpoint{1.151475in}{2.159815in}}{\pgfqpoint{1.151475in}{2.168052in}}%
\pgfpathcurveto{\pgfqpoint{1.151475in}{2.176288in}}{\pgfqpoint{1.148203in}{2.184188in}}{\pgfqpoint{1.142379in}{2.190012in}}%
\pgfpathcurveto{\pgfqpoint{1.136555in}{2.195836in}}{\pgfqpoint{1.128655in}{2.199108in}}{\pgfqpoint{1.120418in}{2.199108in}}%
\pgfpathcurveto{\pgfqpoint{1.112182in}{2.199108in}}{\pgfqpoint{1.104282in}{2.195836in}}{\pgfqpoint{1.098458in}{2.190012in}}%
\pgfpathcurveto{\pgfqpoint{1.092634in}{2.184188in}}{\pgfqpoint{1.089362in}{2.176288in}}{\pgfqpoint{1.089362in}{2.168052in}}%
\pgfpathcurveto{\pgfqpoint{1.089362in}{2.159815in}}{\pgfqpoint{1.092634in}{2.151915in}}{\pgfqpoint{1.098458in}{2.146091in}}%
\pgfpathcurveto{\pgfqpoint{1.104282in}{2.140267in}}{\pgfqpoint{1.112182in}{2.136995in}}{\pgfqpoint{1.120418in}{2.136995in}}%
\pgfpathclose%
\pgfusepath{stroke,fill}%
\end{pgfscope}%
\begin{pgfscope}%
\pgfpathrectangle{\pgfqpoint{0.100000in}{0.212622in}}{\pgfqpoint{3.696000in}{3.696000in}}%
\pgfusepath{clip}%
\pgfsetbuttcap%
\pgfsetroundjoin%
\definecolor{currentfill}{rgb}{0.121569,0.466667,0.705882}%
\pgfsetfillcolor{currentfill}%
\pgfsetfillopacity{0.597158}%
\pgfsetlinewidth{1.003750pt}%
\definecolor{currentstroke}{rgb}{0.121569,0.466667,0.705882}%
\pgfsetstrokecolor{currentstroke}%
\pgfsetstrokeopacity{0.597158}%
\pgfsetdash{}{0pt}%
\pgfpathmoveto{\pgfqpoint{3.165052in}{1.780687in}}%
\pgfpathcurveto{\pgfqpoint{3.173288in}{1.780687in}}{\pgfqpoint{3.181188in}{1.783959in}}{\pgfqpoint{3.187012in}{1.789783in}}%
\pgfpathcurveto{\pgfqpoint{3.192836in}{1.795607in}}{\pgfqpoint{3.196108in}{1.803507in}}{\pgfqpoint{3.196108in}{1.811743in}}%
\pgfpathcurveto{\pgfqpoint{3.196108in}{1.819979in}}{\pgfqpoint{3.192836in}{1.827880in}}{\pgfqpoint{3.187012in}{1.833703in}}%
\pgfpathcurveto{\pgfqpoint{3.181188in}{1.839527in}}{\pgfqpoint{3.173288in}{1.842800in}}{\pgfqpoint{3.165052in}{1.842800in}}%
\pgfpathcurveto{\pgfqpoint{3.156815in}{1.842800in}}{\pgfqpoint{3.148915in}{1.839527in}}{\pgfqpoint{3.143091in}{1.833703in}}%
\pgfpathcurveto{\pgfqpoint{3.137267in}{1.827880in}}{\pgfqpoint{3.133995in}{1.819979in}}{\pgfqpoint{3.133995in}{1.811743in}}%
\pgfpathcurveto{\pgfqpoint{3.133995in}{1.803507in}}{\pgfqpoint{3.137267in}{1.795607in}}{\pgfqpoint{3.143091in}{1.789783in}}%
\pgfpathcurveto{\pgfqpoint{3.148915in}{1.783959in}}{\pgfqpoint{3.156815in}{1.780687in}}{\pgfqpoint{3.165052in}{1.780687in}}%
\pgfpathclose%
\pgfusepath{stroke,fill}%
\end{pgfscope}%
\begin{pgfscope}%
\pgfpathrectangle{\pgfqpoint{0.100000in}{0.212622in}}{\pgfqpoint{3.696000in}{3.696000in}}%
\pgfusepath{clip}%
\pgfsetbuttcap%
\pgfsetroundjoin%
\definecolor{currentfill}{rgb}{0.121569,0.466667,0.705882}%
\pgfsetfillcolor{currentfill}%
\pgfsetfillopacity{0.597800}%
\pgfsetlinewidth{1.003750pt}%
\definecolor{currentstroke}{rgb}{0.121569,0.466667,0.705882}%
\pgfsetstrokecolor{currentstroke}%
\pgfsetstrokeopacity{0.597800}%
\pgfsetdash{}{0pt}%
\pgfpathmoveto{\pgfqpoint{3.163667in}{1.780887in}}%
\pgfpathcurveto{\pgfqpoint{3.171903in}{1.780887in}}{\pgfqpoint{3.179803in}{1.784159in}}{\pgfqpoint{3.185627in}{1.789983in}}%
\pgfpathcurveto{\pgfqpoint{3.191451in}{1.795807in}}{\pgfqpoint{3.194723in}{1.803707in}}{\pgfqpoint{3.194723in}{1.811943in}}%
\pgfpathcurveto{\pgfqpoint{3.194723in}{1.820179in}}{\pgfqpoint{3.191451in}{1.828079in}}{\pgfqpoint{3.185627in}{1.833903in}}%
\pgfpathcurveto{\pgfqpoint{3.179803in}{1.839727in}}{\pgfqpoint{3.171903in}{1.843000in}}{\pgfqpoint{3.163667in}{1.843000in}}%
\pgfpathcurveto{\pgfqpoint{3.155431in}{1.843000in}}{\pgfqpoint{3.147531in}{1.839727in}}{\pgfqpoint{3.141707in}{1.833903in}}%
\pgfpathcurveto{\pgfqpoint{3.135883in}{1.828079in}}{\pgfqpoint{3.132610in}{1.820179in}}{\pgfqpoint{3.132610in}{1.811943in}}%
\pgfpathcurveto{\pgfqpoint{3.132610in}{1.803707in}}{\pgfqpoint{3.135883in}{1.795807in}}{\pgfqpoint{3.141707in}{1.789983in}}%
\pgfpathcurveto{\pgfqpoint{3.147531in}{1.784159in}}{\pgfqpoint{3.155431in}{1.780887in}}{\pgfqpoint{3.163667in}{1.780887in}}%
\pgfpathclose%
\pgfusepath{stroke,fill}%
\end{pgfscope}%
\begin{pgfscope}%
\pgfpathrectangle{\pgfqpoint{0.100000in}{0.212622in}}{\pgfqpoint{3.696000in}{3.696000in}}%
\pgfusepath{clip}%
\pgfsetbuttcap%
\pgfsetroundjoin%
\definecolor{currentfill}{rgb}{0.121569,0.466667,0.705882}%
\pgfsetfillcolor{currentfill}%
\pgfsetfillopacity{0.598156}%
\pgfsetlinewidth{1.003750pt}%
\definecolor{currentstroke}{rgb}{0.121569,0.466667,0.705882}%
\pgfsetstrokecolor{currentstroke}%
\pgfsetstrokeopacity{0.598156}%
\pgfsetdash{}{0pt}%
\pgfpathmoveto{\pgfqpoint{3.162936in}{1.780976in}}%
\pgfpathcurveto{\pgfqpoint{3.171173in}{1.780976in}}{\pgfqpoint{3.179073in}{1.784248in}}{\pgfqpoint{3.184897in}{1.790072in}}%
\pgfpathcurveto{\pgfqpoint{3.190721in}{1.795896in}}{\pgfqpoint{3.193993in}{1.803796in}}{\pgfqpoint{3.193993in}{1.812032in}}%
\pgfpathcurveto{\pgfqpoint{3.193993in}{1.820269in}}{\pgfqpoint{3.190721in}{1.828169in}}{\pgfqpoint{3.184897in}{1.833993in}}%
\pgfpathcurveto{\pgfqpoint{3.179073in}{1.839816in}}{\pgfqpoint{3.171173in}{1.843089in}}{\pgfqpoint{3.162936in}{1.843089in}}%
\pgfpathcurveto{\pgfqpoint{3.154700in}{1.843089in}}{\pgfqpoint{3.146800in}{1.839816in}}{\pgfqpoint{3.140976in}{1.833993in}}%
\pgfpathcurveto{\pgfqpoint{3.135152in}{1.828169in}}{\pgfqpoint{3.131880in}{1.820269in}}{\pgfqpoint{3.131880in}{1.812032in}}%
\pgfpathcurveto{\pgfqpoint{3.131880in}{1.803796in}}{\pgfqpoint{3.135152in}{1.795896in}}{\pgfqpoint{3.140976in}{1.790072in}}%
\pgfpathcurveto{\pgfqpoint{3.146800in}{1.784248in}}{\pgfqpoint{3.154700in}{1.780976in}}{\pgfqpoint{3.162936in}{1.780976in}}%
\pgfpathclose%
\pgfusepath{stroke,fill}%
\end{pgfscope}%
\begin{pgfscope}%
\pgfpathrectangle{\pgfqpoint{0.100000in}{0.212622in}}{\pgfqpoint{3.696000in}{3.696000in}}%
\pgfusepath{clip}%
\pgfsetbuttcap%
\pgfsetroundjoin%
\definecolor{currentfill}{rgb}{0.121569,0.466667,0.705882}%
\pgfsetfillcolor{currentfill}%
\pgfsetfillopacity{0.598629}%
\pgfsetlinewidth{1.003750pt}%
\definecolor{currentstroke}{rgb}{0.121569,0.466667,0.705882}%
\pgfsetstrokecolor{currentstroke}%
\pgfsetstrokeopacity{0.598629}%
\pgfsetdash{}{0pt}%
\pgfpathmoveto{\pgfqpoint{3.162483in}{1.781028in}}%
\pgfpathcurveto{\pgfqpoint{3.170719in}{1.781028in}}{\pgfqpoint{3.178619in}{1.784300in}}{\pgfqpoint{3.184443in}{1.790124in}}%
\pgfpathcurveto{\pgfqpoint{3.190267in}{1.795948in}}{\pgfqpoint{3.193539in}{1.803848in}}{\pgfqpoint{3.193539in}{1.812084in}}%
\pgfpathcurveto{\pgfqpoint{3.193539in}{1.820320in}}{\pgfqpoint{3.190267in}{1.828220in}}{\pgfqpoint{3.184443in}{1.834044in}}%
\pgfpathcurveto{\pgfqpoint{3.178619in}{1.839868in}}{\pgfqpoint{3.170719in}{1.843141in}}{\pgfqpoint{3.162483in}{1.843141in}}%
\pgfpathcurveto{\pgfqpoint{3.154247in}{1.843141in}}{\pgfqpoint{3.146346in}{1.839868in}}{\pgfqpoint{3.140523in}{1.834044in}}%
\pgfpathcurveto{\pgfqpoint{3.134699in}{1.828220in}}{\pgfqpoint{3.131426in}{1.820320in}}{\pgfqpoint{3.131426in}{1.812084in}}%
\pgfpathcurveto{\pgfqpoint{3.131426in}{1.803848in}}{\pgfqpoint{3.134699in}{1.795948in}}{\pgfqpoint{3.140523in}{1.790124in}}%
\pgfpathcurveto{\pgfqpoint{3.146346in}{1.784300in}}{\pgfqpoint{3.154247in}{1.781028in}}{\pgfqpoint{3.162483in}{1.781028in}}%
\pgfpathclose%
\pgfusepath{stroke,fill}%
\end{pgfscope}%
\begin{pgfscope}%
\pgfpathrectangle{\pgfqpoint{0.100000in}{0.212622in}}{\pgfqpoint{3.696000in}{3.696000in}}%
\pgfusepath{clip}%
\pgfsetbuttcap%
\pgfsetroundjoin%
\definecolor{currentfill}{rgb}{0.121569,0.466667,0.705882}%
\pgfsetfillcolor{currentfill}%
\pgfsetfillopacity{0.598778}%
\pgfsetlinewidth{1.003750pt}%
\definecolor{currentstroke}{rgb}{0.121569,0.466667,0.705882}%
\pgfsetstrokecolor{currentstroke}%
\pgfsetstrokeopacity{0.598778}%
\pgfsetdash{}{0pt}%
\pgfpathmoveto{\pgfqpoint{1.117951in}{2.137250in}}%
\pgfpathcurveto{\pgfqpoint{1.126187in}{2.137250in}}{\pgfqpoint{1.134087in}{2.140522in}}{\pgfqpoint{1.139911in}{2.146346in}}%
\pgfpathcurveto{\pgfqpoint{1.145735in}{2.152170in}}{\pgfqpoint{1.149007in}{2.160070in}}{\pgfqpoint{1.149007in}{2.168307in}}%
\pgfpathcurveto{\pgfqpoint{1.149007in}{2.176543in}}{\pgfqpoint{1.145735in}{2.184443in}}{\pgfqpoint{1.139911in}{2.190267in}}%
\pgfpathcurveto{\pgfqpoint{1.134087in}{2.196091in}}{\pgfqpoint{1.126187in}{2.199363in}}{\pgfqpoint{1.117951in}{2.199363in}}%
\pgfpathcurveto{\pgfqpoint{1.109714in}{2.199363in}}{\pgfqpoint{1.101814in}{2.196091in}}{\pgfqpoint{1.095990in}{2.190267in}}%
\pgfpathcurveto{\pgfqpoint{1.090166in}{2.184443in}}{\pgfqpoint{1.086894in}{2.176543in}}{\pgfqpoint{1.086894in}{2.168307in}}%
\pgfpathcurveto{\pgfqpoint{1.086894in}{2.160070in}}{\pgfqpoint{1.090166in}{2.152170in}}{\pgfqpoint{1.095990in}{2.146346in}}%
\pgfpathcurveto{\pgfqpoint{1.101814in}{2.140522in}}{\pgfqpoint{1.109714in}{2.137250in}}{\pgfqpoint{1.117951in}{2.137250in}}%
\pgfpathclose%
\pgfusepath{stroke,fill}%
\end{pgfscope}%
\begin{pgfscope}%
\pgfpathrectangle{\pgfqpoint{0.100000in}{0.212622in}}{\pgfqpoint{3.696000in}{3.696000in}}%
\pgfusepath{clip}%
\pgfsetbuttcap%
\pgfsetroundjoin%
\definecolor{currentfill}{rgb}{0.121569,0.466667,0.705882}%
\pgfsetfillcolor{currentfill}%
\pgfsetfillopacity{0.599477}%
\pgfsetlinewidth{1.003750pt}%
\definecolor{currentstroke}{rgb}{0.121569,0.466667,0.705882}%
\pgfsetstrokecolor{currentstroke}%
\pgfsetstrokeopacity{0.599477}%
\pgfsetdash{}{0pt}%
\pgfpathmoveto{\pgfqpoint{3.160763in}{1.781343in}}%
\pgfpathcurveto{\pgfqpoint{3.169000in}{1.781343in}}{\pgfqpoint{3.176900in}{1.784616in}}{\pgfqpoint{3.182724in}{1.790440in}}%
\pgfpathcurveto{\pgfqpoint{3.188548in}{1.796263in}}{\pgfqpoint{3.191820in}{1.804164in}}{\pgfqpoint{3.191820in}{1.812400in}}%
\pgfpathcurveto{\pgfqpoint{3.191820in}{1.820636in}}{\pgfqpoint{3.188548in}{1.828536in}}{\pgfqpoint{3.182724in}{1.834360in}}%
\pgfpathcurveto{\pgfqpoint{3.176900in}{1.840184in}}{\pgfqpoint{3.169000in}{1.843456in}}{\pgfqpoint{3.160763in}{1.843456in}}%
\pgfpathcurveto{\pgfqpoint{3.152527in}{1.843456in}}{\pgfqpoint{3.144627in}{1.840184in}}{\pgfqpoint{3.138803in}{1.834360in}}%
\pgfpathcurveto{\pgfqpoint{3.132979in}{1.828536in}}{\pgfqpoint{3.129707in}{1.820636in}}{\pgfqpoint{3.129707in}{1.812400in}}%
\pgfpathcurveto{\pgfqpoint{3.129707in}{1.804164in}}{\pgfqpoint{3.132979in}{1.796263in}}{\pgfqpoint{3.138803in}{1.790440in}}%
\pgfpathcurveto{\pgfqpoint{3.144627in}{1.784616in}}{\pgfqpoint{3.152527in}{1.781343in}}{\pgfqpoint{3.160763in}{1.781343in}}%
\pgfpathclose%
\pgfusepath{stroke,fill}%
\end{pgfscope}%
\begin{pgfscope}%
\pgfpathrectangle{\pgfqpoint{0.100000in}{0.212622in}}{\pgfqpoint{3.696000in}{3.696000in}}%
\pgfusepath{clip}%
\pgfsetbuttcap%
\pgfsetroundjoin%
\definecolor{currentfill}{rgb}{0.121569,0.466667,0.705882}%
\pgfsetfillcolor{currentfill}%
\pgfsetfillopacity{0.600210}%
\pgfsetlinewidth{1.003750pt}%
\definecolor{currentstroke}{rgb}{0.121569,0.466667,0.705882}%
\pgfsetstrokecolor{currentstroke}%
\pgfsetstrokeopacity{0.600210}%
\pgfsetdash{}{0pt}%
\pgfpathmoveto{\pgfqpoint{1.113874in}{2.137633in}}%
\pgfpathcurveto{\pgfqpoint{1.122111in}{2.137633in}}{\pgfqpoint{1.130011in}{2.140905in}}{\pgfqpoint{1.135835in}{2.146729in}}%
\pgfpathcurveto{\pgfqpoint{1.141659in}{2.152553in}}{\pgfqpoint{1.144931in}{2.160453in}}{\pgfqpoint{1.144931in}{2.168689in}}%
\pgfpathcurveto{\pgfqpoint{1.144931in}{2.176926in}}{\pgfqpoint{1.141659in}{2.184826in}}{\pgfqpoint{1.135835in}{2.190650in}}%
\pgfpathcurveto{\pgfqpoint{1.130011in}{2.196473in}}{\pgfqpoint{1.122111in}{2.199746in}}{\pgfqpoint{1.113874in}{2.199746in}}%
\pgfpathcurveto{\pgfqpoint{1.105638in}{2.199746in}}{\pgfqpoint{1.097738in}{2.196473in}}{\pgfqpoint{1.091914in}{2.190650in}}%
\pgfpathcurveto{\pgfqpoint{1.086090in}{2.184826in}}{\pgfqpoint{1.082818in}{2.176926in}}{\pgfqpoint{1.082818in}{2.168689in}}%
\pgfpathcurveto{\pgfqpoint{1.082818in}{2.160453in}}{\pgfqpoint{1.086090in}{2.152553in}}{\pgfqpoint{1.091914in}{2.146729in}}%
\pgfpathcurveto{\pgfqpoint{1.097738in}{2.140905in}}{\pgfqpoint{1.105638in}{2.137633in}}{\pgfqpoint{1.113874in}{2.137633in}}%
\pgfpathclose%
\pgfusepath{stroke,fill}%
\end{pgfscope}%
\begin{pgfscope}%
\pgfpathrectangle{\pgfqpoint{0.100000in}{0.212622in}}{\pgfqpoint{3.696000in}{3.696000in}}%
\pgfusepath{clip}%
\pgfsetbuttcap%
\pgfsetroundjoin%
\definecolor{currentfill}{rgb}{0.121569,0.466667,0.705882}%
\pgfsetfillcolor{currentfill}%
\pgfsetfillopacity{0.600419}%
\pgfsetlinewidth{1.003750pt}%
\definecolor{currentstroke}{rgb}{0.121569,0.466667,0.705882}%
\pgfsetstrokecolor{currentstroke}%
\pgfsetstrokeopacity{0.600419}%
\pgfsetdash{}{0pt}%
\pgfpathmoveto{\pgfqpoint{3.158331in}{1.781960in}}%
\pgfpathcurveto{\pgfqpoint{3.166568in}{1.781960in}}{\pgfqpoint{3.174468in}{1.785232in}}{\pgfqpoint{3.180292in}{1.791056in}}%
\pgfpathcurveto{\pgfqpoint{3.186116in}{1.796880in}}{\pgfqpoint{3.189388in}{1.804780in}}{\pgfqpoint{3.189388in}{1.813016in}}%
\pgfpathcurveto{\pgfqpoint{3.189388in}{1.821252in}}{\pgfqpoint{3.186116in}{1.829152in}}{\pgfqpoint{3.180292in}{1.834976in}}%
\pgfpathcurveto{\pgfqpoint{3.174468in}{1.840800in}}{\pgfqpoint{3.166568in}{1.844073in}}{\pgfqpoint{3.158331in}{1.844073in}}%
\pgfpathcurveto{\pgfqpoint{3.150095in}{1.844073in}}{\pgfqpoint{3.142195in}{1.840800in}}{\pgfqpoint{3.136371in}{1.834976in}}%
\pgfpathcurveto{\pgfqpoint{3.130547in}{1.829152in}}{\pgfqpoint{3.127275in}{1.821252in}}{\pgfqpoint{3.127275in}{1.813016in}}%
\pgfpathcurveto{\pgfqpoint{3.127275in}{1.804780in}}{\pgfqpoint{3.130547in}{1.796880in}}{\pgfqpoint{3.136371in}{1.791056in}}%
\pgfpathcurveto{\pgfqpoint{3.142195in}{1.785232in}}{\pgfqpoint{3.150095in}{1.781960in}}{\pgfqpoint{3.158331in}{1.781960in}}%
\pgfpathclose%
\pgfusepath{stroke,fill}%
\end{pgfscope}%
\begin{pgfscope}%
\pgfpathrectangle{\pgfqpoint{0.100000in}{0.212622in}}{\pgfqpoint{3.696000in}{3.696000in}}%
\pgfusepath{clip}%
\pgfsetbuttcap%
\pgfsetroundjoin%
\definecolor{currentfill}{rgb}{0.121569,0.466667,0.705882}%
\pgfsetfillcolor{currentfill}%
\pgfsetfillopacity{0.601224}%
\pgfsetlinewidth{1.003750pt}%
\definecolor{currentstroke}{rgb}{0.121569,0.466667,0.705882}%
\pgfsetstrokecolor{currentstroke}%
\pgfsetstrokeopacity{0.601224}%
\pgfsetdash{}{0pt}%
\pgfpathmoveto{\pgfqpoint{1.111490in}{2.137663in}}%
\pgfpathcurveto{\pgfqpoint{1.119727in}{2.137663in}}{\pgfqpoint{1.127627in}{2.140935in}}{\pgfqpoint{1.133451in}{2.146759in}}%
\pgfpathcurveto{\pgfqpoint{1.139274in}{2.152583in}}{\pgfqpoint{1.142547in}{2.160483in}}{\pgfqpoint{1.142547in}{2.168719in}}%
\pgfpathcurveto{\pgfqpoint{1.142547in}{2.176955in}}{\pgfqpoint{1.139274in}{2.184856in}}{\pgfqpoint{1.133451in}{2.190679in}}%
\pgfpathcurveto{\pgfqpoint{1.127627in}{2.196503in}}{\pgfqpoint{1.119727in}{2.199776in}}{\pgfqpoint{1.111490in}{2.199776in}}%
\pgfpathcurveto{\pgfqpoint{1.103254in}{2.199776in}}{\pgfqpoint{1.095354in}{2.196503in}}{\pgfqpoint{1.089530in}{2.190679in}}%
\pgfpathcurveto{\pgfqpoint{1.083706in}{2.184856in}}{\pgfqpoint{1.080434in}{2.176955in}}{\pgfqpoint{1.080434in}{2.168719in}}%
\pgfpathcurveto{\pgfqpoint{1.080434in}{2.160483in}}{\pgfqpoint{1.083706in}{2.152583in}}{\pgfqpoint{1.089530in}{2.146759in}}%
\pgfpathcurveto{\pgfqpoint{1.095354in}{2.140935in}}{\pgfqpoint{1.103254in}{2.137663in}}{\pgfqpoint{1.111490in}{2.137663in}}%
\pgfpathclose%
\pgfusepath{stroke,fill}%
\end{pgfscope}%
\begin{pgfscope}%
\pgfpathrectangle{\pgfqpoint{0.100000in}{0.212622in}}{\pgfqpoint{3.696000in}{3.696000in}}%
\pgfusepath{clip}%
\pgfsetbuttcap%
\pgfsetroundjoin%
\definecolor{currentfill}{rgb}{0.121569,0.466667,0.705882}%
\pgfsetfillcolor{currentfill}%
\pgfsetfillopacity{0.601644}%
\pgfsetlinewidth{1.003750pt}%
\definecolor{currentstroke}{rgb}{0.121569,0.466667,0.705882}%
\pgfsetstrokecolor{currentstroke}%
\pgfsetstrokeopacity{0.601644}%
\pgfsetdash{}{0pt}%
\pgfpathmoveto{\pgfqpoint{1.110576in}{2.137709in}}%
\pgfpathcurveto{\pgfqpoint{1.118812in}{2.137709in}}{\pgfqpoint{1.126712in}{2.140981in}}{\pgfqpoint{1.132536in}{2.146805in}}%
\pgfpathcurveto{\pgfqpoint{1.138360in}{2.152629in}}{\pgfqpoint{1.141632in}{2.160529in}}{\pgfqpoint{1.141632in}{2.168765in}}%
\pgfpathcurveto{\pgfqpoint{1.141632in}{2.177002in}}{\pgfqpoint{1.138360in}{2.184902in}}{\pgfqpoint{1.132536in}{2.190726in}}%
\pgfpathcurveto{\pgfqpoint{1.126712in}{2.196550in}}{\pgfqpoint{1.118812in}{2.199822in}}{\pgfqpoint{1.110576in}{2.199822in}}%
\pgfpathcurveto{\pgfqpoint{1.102339in}{2.199822in}}{\pgfqpoint{1.094439in}{2.196550in}}{\pgfqpoint{1.088615in}{2.190726in}}%
\pgfpathcurveto{\pgfqpoint{1.082791in}{2.184902in}}{\pgfqpoint{1.079519in}{2.177002in}}{\pgfqpoint{1.079519in}{2.168765in}}%
\pgfpathcurveto{\pgfqpoint{1.079519in}{2.160529in}}{\pgfqpoint{1.082791in}{2.152629in}}{\pgfqpoint{1.088615in}{2.146805in}}%
\pgfpathcurveto{\pgfqpoint{1.094439in}{2.140981in}}{\pgfqpoint{1.102339in}{2.137709in}}{\pgfqpoint{1.110576in}{2.137709in}}%
\pgfpathclose%
\pgfusepath{stroke,fill}%
\end{pgfscope}%
\begin{pgfscope}%
\pgfpathrectangle{\pgfqpoint{0.100000in}{0.212622in}}{\pgfqpoint{3.696000in}{3.696000in}}%
\pgfusepath{clip}%
\pgfsetbuttcap%
\pgfsetroundjoin%
\definecolor{currentfill}{rgb}{0.121569,0.466667,0.705882}%
\pgfsetfillcolor{currentfill}%
\pgfsetfillopacity{0.601963}%
\pgfsetlinewidth{1.003750pt}%
\definecolor{currentstroke}{rgb}{0.121569,0.466667,0.705882}%
\pgfsetstrokecolor{currentstroke}%
\pgfsetstrokeopacity{0.601963}%
\pgfsetdash{}{0pt}%
\pgfpathmoveto{\pgfqpoint{3.156106in}{1.782115in}}%
\pgfpathcurveto{\pgfqpoint{3.164343in}{1.782115in}}{\pgfqpoint{3.172243in}{1.785387in}}{\pgfqpoint{3.178067in}{1.791211in}}%
\pgfpathcurveto{\pgfqpoint{3.183891in}{1.797035in}}{\pgfqpoint{3.187163in}{1.804935in}}{\pgfqpoint{3.187163in}{1.813171in}}%
\pgfpathcurveto{\pgfqpoint{3.187163in}{1.821407in}}{\pgfqpoint{3.183891in}{1.829308in}}{\pgfqpoint{3.178067in}{1.835131in}}%
\pgfpathcurveto{\pgfqpoint{3.172243in}{1.840955in}}{\pgfqpoint{3.164343in}{1.844228in}}{\pgfqpoint{3.156106in}{1.844228in}}%
\pgfpathcurveto{\pgfqpoint{3.147870in}{1.844228in}}{\pgfqpoint{3.139970in}{1.840955in}}{\pgfqpoint{3.134146in}{1.835131in}}%
\pgfpathcurveto{\pgfqpoint{3.128322in}{1.829308in}}{\pgfqpoint{3.125050in}{1.821407in}}{\pgfqpoint{3.125050in}{1.813171in}}%
\pgfpathcurveto{\pgfqpoint{3.125050in}{1.804935in}}{\pgfqpoint{3.128322in}{1.797035in}}{\pgfqpoint{3.134146in}{1.791211in}}%
\pgfpathcurveto{\pgfqpoint{3.139970in}{1.785387in}}{\pgfqpoint{3.147870in}{1.782115in}}{\pgfqpoint{3.156106in}{1.782115in}}%
\pgfpathclose%
\pgfusepath{stroke,fill}%
\end{pgfscope}%
\begin{pgfscope}%
\pgfpathrectangle{\pgfqpoint{0.100000in}{0.212622in}}{\pgfqpoint{3.696000in}{3.696000in}}%
\pgfusepath{clip}%
\pgfsetbuttcap%
\pgfsetroundjoin%
\definecolor{currentfill}{rgb}{0.121569,0.466667,0.705882}%
\pgfsetfillcolor{currentfill}%
\pgfsetfillopacity{0.602369}%
\pgfsetlinewidth{1.003750pt}%
\definecolor{currentstroke}{rgb}{0.121569,0.466667,0.705882}%
\pgfsetstrokecolor{currentstroke}%
\pgfsetstrokeopacity{0.602369}%
\pgfsetdash{}{0pt}%
\pgfpathmoveto{\pgfqpoint{1.108669in}{2.137795in}}%
\pgfpathcurveto{\pgfqpoint{1.116905in}{2.137795in}}{\pgfqpoint{1.124805in}{2.141068in}}{\pgfqpoint{1.130629in}{2.146892in}}%
\pgfpathcurveto{\pgfqpoint{1.136453in}{2.152716in}}{\pgfqpoint{1.139725in}{2.160616in}}{\pgfqpoint{1.139725in}{2.168852in}}%
\pgfpathcurveto{\pgfqpoint{1.139725in}{2.177088in}}{\pgfqpoint{1.136453in}{2.184988in}}{\pgfqpoint{1.130629in}{2.190812in}}%
\pgfpathcurveto{\pgfqpoint{1.124805in}{2.196636in}}{\pgfqpoint{1.116905in}{2.199908in}}{\pgfqpoint{1.108669in}{2.199908in}}%
\pgfpathcurveto{\pgfqpoint{1.100432in}{2.199908in}}{\pgfqpoint{1.092532in}{2.196636in}}{\pgfqpoint{1.086708in}{2.190812in}}%
\pgfpathcurveto{\pgfqpoint{1.080885in}{2.184988in}}{\pgfqpoint{1.077612in}{2.177088in}}{\pgfqpoint{1.077612in}{2.168852in}}%
\pgfpathcurveto{\pgfqpoint{1.077612in}{2.160616in}}{\pgfqpoint{1.080885in}{2.152716in}}{\pgfqpoint{1.086708in}{2.146892in}}%
\pgfpathcurveto{\pgfqpoint{1.092532in}{2.141068in}}{\pgfqpoint{1.100432in}{2.137795in}}{\pgfqpoint{1.108669in}{2.137795in}}%
\pgfpathclose%
\pgfusepath{stroke,fill}%
\end{pgfscope}%
\begin{pgfscope}%
\pgfpathrectangle{\pgfqpoint{0.100000in}{0.212622in}}{\pgfqpoint{3.696000in}{3.696000in}}%
\pgfusepath{clip}%
\pgfsetbuttcap%
\pgfsetroundjoin%
\definecolor{currentfill}{rgb}{0.121569,0.466667,0.705882}%
\pgfsetfillcolor{currentfill}%
\pgfsetfillopacity{0.602834}%
\pgfsetlinewidth{1.003750pt}%
\definecolor{currentstroke}{rgb}{0.121569,0.466667,0.705882}%
\pgfsetstrokecolor{currentstroke}%
\pgfsetstrokeopacity{0.602834}%
\pgfsetdash{}{0pt}%
\pgfpathmoveto{\pgfqpoint{3.155162in}{1.782134in}}%
\pgfpathcurveto{\pgfqpoint{3.163398in}{1.782134in}}{\pgfqpoint{3.171298in}{1.785406in}}{\pgfqpoint{3.177122in}{1.791230in}}%
\pgfpathcurveto{\pgfqpoint{3.182946in}{1.797054in}}{\pgfqpoint{3.186218in}{1.804954in}}{\pgfqpoint{3.186218in}{1.813190in}}%
\pgfpathcurveto{\pgfqpoint{3.186218in}{1.821427in}}{\pgfqpoint{3.182946in}{1.829327in}}{\pgfqpoint{3.177122in}{1.835151in}}%
\pgfpathcurveto{\pgfqpoint{3.171298in}{1.840975in}}{\pgfqpoint{3.163398in}{1.844247in}}{\pgfqpoint{3.155162in}{1.844247in}}%
\pgfpathcurveto{\pgfqpoint{3.146926in}{1.844247in}}{\pgfqpoint{3.139026in}{1.840975in}}{\pgfqpoint{3.133202in}{1.835151in}}%
\pgfpathcurveto{\pgfqpoint{3.127378in}{1.829327in}}{\pgfqpoint{3.124105in}{1.821427in}}{\pgfqpoint{3.124105in}{1.813190in}}%
\pgfpathcurveto{\pgfqpoint{3.124105in}{1.804954in}}{\pgfqpoint{3.127378in}{1.797054in}}{\pgfqpoint{3.133202in}{1.791230in}}%
\pgfpathcurveto{\pgfqpoint{3.139026in}{1.785406in}}{\pgfqpoint{3.146926in}{1.782134in}}{\pgfqpoint{3.155162in}{1.782134in}}%
\pgfpathclose%
\pgfusepath{stroke,fill}%
\end{pgfscope}%
\begin{pgfscope}%
\pgfpathrectangle{\pgfqpoint{0.100000in}{0.212622in}}{\pgfqpoint{3.696000in}{3.696000in}}%
\pgfusepath{clip}%
\pgfsetbuttcap%
\pgfsetroundjoin%
\definecolor{currentfill}{rgb}{0.121569,0.466667,0.705882}%
\pgfsetfillcolor{currentfill}%
\pgfsetfillopacity{0.603803}%
\pgfsetlinewidth{1.003750pt}%
\definecolor{currentstroke}{rgb}{0.121569,0.466667,0.705882}%
\pgfsetstrokecolor{currentstroke}%
\pgfsetstrokeopacity{0.603803}%
\pgfsetdash{}{0pt}%
\pgfpathmoveto{\pgfqpoint{1.106080in}{2.137836in}}%
\pgfpathcurveto{\pgfqpoint{1.114316in}{2.137836in}}{\pgfqpoint{1.122216in}{2.141108in}}{\pgfqpoint{1.128040in}{2.146932in}}%
\pgfpathcurveto{\pgfqpoint{1.133864in}{2.152756in}}{\pgfqpoint{1.137136in}{2.160656in}}{\pgfqpoint{1.137136in}{2.168892in}}%
\pgfpathcurveto{\pgfqpoint{1.137136in}{2.177129in}}{\pgfqpoint{1.133864in}{2.185029in}}{\pgfqpoint{1.128040in}{2.190853in}}%
\pgfpathcurveto{\pgfqpoint{1.122216in}{2.196677in}}{\pgfqpoint{1.114316in}{2.199949in}}{\pgfqpoint{1.106080in}{2.199949in}}%
\pgfpathcurveto{\pgfqpoint{1.097843in}{2.199949in}}{\pgfqpoint{1.089943in}{2.196677in}}{\pgfqpoint{1.084119in}{2.190853in}}%
\pgfpathcurveto{\pgfqpoint{1.078296in}{2.185029in}}{\pgfqpoint{1.075023in}{2.177129in}}{\pgfqpoint{1.075023in}{2.168892in}}%
\pgfpathcurveto{\pgfqpoint{1.075023in}{2.160656in}}{\pgfqpoint{1.078296in}{2.152756in}}{\pgfqpoint{1.084119in}{2.146932in}}%
\pgfpathcurveto{\pgfqpoint{1.089943in}{2.141108in}}{\pgfqpoint{1.097843in}{2.137836in}}{\pgfqpoint{1.106080in}{2.137836in}}%
\pgfpathclose%
\pgfusepath{stroke,fill}%
\end{pgfscope}%
\begin{pgfscope}%
\pgfpathrectangle{\pgfqpoint{0.100000in}{0.212622in}}{\pgfqpoint{3.696000in}{3.696000in}}%
\pgfusepath{clip}%
\pgfsetbuttcap%
\pgfsetroundjoin%
\definecolor{currentfill}{rgb}{0.121569,0.466667,0.705882}%
\pgfsetfillcolor{currentfill}%
\pgfsetfillopacity{0.603897}%
\pgfsetlinewidth{1.003750pt}%
\definecolor{currentstroke}{rgb}{0.121569,0.466667,0.705882}%
\pgfsetstrokecolor{currentstroke}%
\pgfsetstrokeopacity{0.603897}%
\pgfsetdash{}{0pt}%
\pgfpathmoveto{\pgfqpoint{3.152499in}{1.782615in}}%
\pgfpathcurveto{\pgfqpoint{3.160736in}{1.782615in}}{\pgfqpoint{3.168636in}{1.785888in}}{\pgfqpoint{3.174460in}{1.791712in}}%
\pgfpathcurveto{\pgfqpoint{3.180284in}{1.797535in}}{\pgfqpoint{3.183556in}{1.805436in}}{\pgfqpoint{3.183556in}{1.813672in}}%
\pgfpathcurveto{\pgfqpoint{3.183556in}{1.821908in}}{\pgfqpoint{3.180284in}{1.829808in}}{\pgfqpoint{3.174460in}{1.835632in}}%
\pgfpathcurveto{\pgfqpoint{3.168636in}{1.841456in}}{\pgfqpoint{3.160736in}{1.844728in}}{\pgfqpoint{3.152499in}{1.844728in}}%
\pgfpathcurveto{\pgfqpoint{3.144263in}{1.844728in}}{\pgfqpoint{3.136363in}{1.841456in}}{\pgfqpoint{3.130539in}{1.835632in}}%
\pgfpathcurveto{\pgfqpoint{3.124715in}{1.829808in}}{\pgfqpoint{3.121443in}{1.821908in}}{\pgfqpoint{3.121443in}{1.813672in}}%
\pgfpathcurveto{\pgfqpoint{3.121443in}{1.805436in}}{\pgfqpoint{3.124715in}{1.797535in}}{\pgfqpoint{3.130539in}{1.791712in}}%
\pgfpathcurveto{\pgfqpoint{3.136363in}{1.785888in}}{\pgfqpoint{3.144263in}{1.782615in}}{\pgfqpoint{3.152499in}{1.782615in}}%
\pgfpathclose%
\pgfusepath{stroke,fill}%
\end{pgfscope}%
\begin{pgfscope}%
\pgfpathrectangle{\pgfqpoint{0.100000in}{0.212622in}}{\pgfqpoint{3.696000in}{3.696000in}}%
\pgfusepath{clip}%
\pgfsetbuttcap%
\pgfsetroundjoin%
\definecolor{currentfill}{rgb}{0.121569,0.466667,0.705882}%
\pgfsetfillcolor{currentfill}%
\pgfsetfillopacity{0.604503}%
\pgfsetlinewidth{1.003750pt}%
\definecolor{currentstroke}{rgb}{0.121569,0.466667,0.705882}%
\pgfsetstrokecolor{currentstroke}%
\pgfsetstrokeopacity{0.604503}%
\pgfsetdash{}{0pt}%
\pgfpathmoveto{\pgfqpoint{3.151172in}{1.782833in}}%
\pgfpathcurveto{\pgfqpoint{3.159408in}{1.782833in}}{\pgfqpoint{3.167308in}{1.786105in}}{\pgfqpoint{3.173132in}{1.791929in}}%
\pgfpathcurveto{\pgfqpoint{3.178956in}{1.797753in}}{\pgfqpoint{3.182228in}{1.805653in}}{\pgfqpoint{3.182228in}{1.813889in}}%
\pgfpathcurveto{\pgfqpoint{3.182228in}{1.822126in}}{\pgfqpoint{3.178956in}{1.830026in}}{\pgfqpoint{3.173132in}{1.835850in}}%
\pgfpathcurveto{\pgfqpoint{3.167308in}{1.841674in}}{\pgfqpoint{3.159408in}{1.844946in}}{\pgfqpoint{3.151172in}{1.844946in}}%
\pgfpathcurveto{\pgfqpoint{3.142935in}{1.844946in}}{\pgfqpoint{3.135035in}{1.841674in}}{\pgfqpoint{3.129212in}{1.835850in}}%
\pgfpathcurveto{\pgfqpoint{3.123388in}{1.830026in}}{\pgfqpoint{3.120115in}{1.822126in}}{\pgfqpoint{3.120115in}{1.813889in}}%
\pgfpathcurveto{\pgfqpoint{3.120115in}{1.805653in}}{\pgfqpoint{3.123388in}{1.797753in}}{\pgfqpoint{3.129212in}{1.791929in}}%
\pgfpathcurveto{\pgfqpoint{3.135035in}{1.786105in}}{\pgfqpoint{3.142935in}{1.782833in}}{\pgfqpoint{3.151172in}{1.782833in}}%
\pgfpathclose%
\pgfusepath{stroke,fill}%
\end{pgfscope}%
\begin{pgfscope}%
\pgfpathrectangle{\pgfqpoint{0.100000in}{0.212622in}}{\pgfqpoint{3.696000in}{3.696000in}}%
\pgfusepath{clip}%
\pgfsetbuttcap%
\pgfsetroundjoin%
\definecolor{currentfill}{rgb}{0.121569,0.466667,0.705882}%
\pgfsetfillcolor{currentfill}%
\pgfsetfillopacity{0.604637}%
\pgfsetlinewidth{1.003750pt}%
\definecolor{currentstroke}{rgb}{0.121569,0.466667,0.705882}%
\pgfsetstrokecolor{currentstroke}%
\pgfsetstrokeopacity{0.604637}%
\pgfsetdash{}{0pt}%
\pgfpathmoveto{\pgfqpoint{1.103784in}{2.138128in}}%
\pgfpathcurveto{\pgfqpoint{1.112020in}{2.138128in}}{\pgfqpoint{1.119920in}{2.141400in}}{\pgfqpoint{1.125744in}{2.147224in}}%
\pgfpathcurveto{\pgfqpoint{1.131568in}{2.153048in}}{\pgfqpoint{1.134841in}{2.160948in}}{\pgfqpoint{1.134841in}{2.169185in}}%
\pgfpathcurveto{\pgfqpoint{1.134841in}{2.177421in}}{\pgfqpoint{1.131568in}{2.185321in}}{\pgfqpoint{1.125744in}{2.191145in}}%
\pgfpathcurveto{\pgfqpoint{1.119920in}{2.196969in}}{\pgfqpoint{1.112020in}{2.200241in}}{\pgfqpoint{1.103784in}{2.200241in}}%
\pgfpathcurveto{\pgfqpoint{1.095548in}{2.200241in}}{\pgfqpoint{1.087648in}{2.196969in}}{\pgfqpoint{1.081824in}{2.191145in}}%
\pgfpathcurveto{\pgfqpoint{1.076000in}{2.185321in}}{\pgfqpoint{1.072728in}{2.177421in}}{\pgfqpoint{1.072728in}{2.169185in}}%
\pgfpathcurveto{\pgfqpoint{1.072728in}{2.160948in}}{\pgfqpoint{1.076000in}{2.153048in}}{\pgfqpoint{1.081824in}{2.147224in}}%
\pgfpathcurveto{\pgfqpoint{1.087648in}{2.141400in}}{\pgfqpoint{1.095548in}{2.138128in}}{\pgfqpoint{1.103784in}{2.138128in}}%
\pgfpathclose%
\pgfusepath{stroke,fill}%
\end{pgfscope}%
\begin{pgfscope}%
\pgfpathrectangle{\pgfqpoint{0.100000in}{0.212622in}}{\pgfqpoint{3.696000in}{3.696000in}}%
\pgfusepath{clip}%
\pgfsetbuttcap%
\pgfsetroundjoin%
\definecolor{currentfill}{rgb}{0.121569,0.466667,0.705882}%
\pgfsetfillcolor{currentfill}%
\pgfsetfillopacity{0.605316}%
\pgfsetlinewidth{1.003750pt}%
\definecolor{currentstroke}{rgb}{0.121569,0.466667,0.705882}%
\pgfsetstrokecolor{currentstroke}%
\pgfsetstrokeopacity{0.605316}%
\pgfsetdash{}{0pt}%
\pgfpathmoveto{\pgfqpoint{3.150229in}{1.782965in}}%
\pgfpathcurveto{\pgfqpoint{3.158465in}{1.782965in}}{\pgfqpoint{3.166365in}{1.786237in}}{\pgfqpoint{3.172189in}{1.792061in}}%
\pgfpathcurveto{\pgfqpoint{3.178013in}{1.797885in}}{\pgfqpoint{3.181285in}{1.805785in}}{\pgfqpoint{3.181285in}{1.814021in}}%
\pgfpathcurveto{\pgfqpoint{3.181285in}{1.822258in}}{\pgfqpoint{3.178013in}{1.830158in}}{\pgfqpoint{3.172189in}{1.835982in}}%
\pgfpathcurveto{\pgfqpoint{3.166365in}{1.841805in}}{\pgfqpoint{3.158465in}{1.845078in}}{\pgfqpoint{3.150229in}{1.845078in}}%
\pgfpathcurveto{\pgfqpoint{3.141993in}{1.845078in}}{\pgfqpoint{3.134092in}{1.841805in}}{\pgfqpoint{3.128269in}{1.835982in}}%
\pgfpathcurveto{\pgfqpoint{3.122445in}{1.830158in}}{\pgfqpoint{3.119172in}{1.822258in}}{\pgfqpoint{3.119172in}{1.814021in}}%
\pgfpathcurveto{\pgfqpoint{3.119172in}{1.805785in}}{\pgfqpoint{3.122445in}{1.797885in}}{\pgfqpoint{3.128269in}{1.792061in}}%
\pgfpathcurveto{\pgfqpoint{3.134092in}{1.786237in}}{\pgfqpoint{3.141993in}{1.782965in}}{\pgfqpoint{3.150229in}{1.782965in}}%
\pgfpathclose%
\pgfusepath{stroke,fill}%
\end{pgfscope}%
\begin{pgfscope}%
\pgfpathrectangle{\pgfqpoint{0.100000in}{0.212622in}}{\pgfqpoint{3.696000in}{3.696000in}}%
\pgfusepath{clip}%
\pgfsetbuttcap%
\pgfsetroundjoin%
\definecolor{currentfill}{rgb}{0.121569,0.466667,0.705882}%
\pgfsetfillcolor{currentfill}%
\pgfsetfillopacity{0.605344}%
\pgfsetlinewidth{1.003750pt}%
\definecolor{currentstroke}{rgb}{0.121569,0.466667,0.705882}%
\pgfsetstrokecolor{currentstroke}%
\pgfsetstrokeopacity{0.605344}%
\pgfsetdash{}{0pt}%
\pgfpathmoveto{\pgfqpoint{1.102620in}{2.138125in}}%
\pgfpathcurveto{\pgfqpoint{1.110856in}{2.138125in}}{\pgfqpoint{1.118756in}{2.141398in}}{\pgfqpoint{1.124580in}{2.147222in}}%
\pgfpathcurveto{\pgfqpoint{1.130404in}{2.153046in}}{\pgfqpoint{1.133677in}{2.160946in}}{\pgfqpoint{1.133677in}{2.169182in}}%
\pgfpathcurveto{\pgfqpoint{1.133677in}{2.177418in}}{\pgfqpoint{1.130404in}{2.185318in}}{\pgfqpoint{1.124580in}{2.191142in}}%
\pgfpathcurveto{\pgfqpoint{1.118756in}{2.196966in}}{\pgfqpoint{1.110856in}{2.200238in}}{\pgfqpoint{1.102620in}{2.200238in}}%
\pgfpathcurveto{\pgfqpoint{1.094384in}{2.200238in}}{\pgfqpoint{1.086484in}{2.196966in}}{\pgfqpoint{1.080660in}{2.191142in}}%
\pgfpathcurveto{\pgfqpoint{1.074836in}{2.185318in}}{\pgfqpoint{1.071564in}{2.177418in}}{\pgfqpoint{1.071564in}{2.169182in}}%
\pgfpathcurveto{\pgfqpoint{1.071564in}{2.160946in}}{\pgfqpoint{1.074836in}{2.153046in}}{\pgfqpoint{1.080660in}{2.147222in}}%
\pgfpathcurveto{\pgfqpoint{1.086484in}{2.141398in}}{\pgfqpoint{1.094384in}{2.138125in}}{\pgfqpoint{1.102620in}{2.138125in}}%
\pgfpathclose%
\pgfusepath{stroke,fill}%
\end{pgfscope}%
\begin{pgfscope}%
\pgfpathrectangle{\pgfqpoint{0.100000in}{0.212622in}}{\pgfqpoint{3.696000in}{3.696000in}}%
\pgfusepath{clip}%
\pgfsetbuttcap%
\pgfsetroundjoin%
\definecolor{currentfill}{rgb}{0.121569,0.466667,0.705882}%
\pgfsetfillcolor{currentfill}%
\pgfsetfillopacity{0.605743}%
\pgfsetlinewidth{1.003750pt}%
\definecolor{currentstroke}{rgb}{0.121569,0.466667,0.705882}%
\pgfsetstrokecolor{currentstroke}%
\pgfsetstrokeopacity{0.605743}%
\pgfsetdash{}{0pt}%
\pgfpathmoveto{\pgfqpoint{1.101503in}{2.138261in}}%
\pgfpathcurveto{\pgfqpoint{1.109739in}{2.138261in}}{\pgfqpoint{1.117639in}{2.141533in}}{\pgfqpoint{1.123463in}{2.147357in}}%
\pgfpathcurveto{\pgfqpoint{1.129287in}{2.153181in}}{\pgfqpoint{1.132559in}{2.161081in}}{\pgfqpoint{1.132559in}{2.169318in}}%
\pgfpathcurveto{\pgfqpoint{1.132559in}{2.177554in}}{\pgfqpoint{1.129287in}{2.185454in}}{\pgfqpoint{1.123463in}{2.191278in}}%
\pgfpathcurveto{\pgfqpoint{1.117639in}{2.197102in}}{\pgfqpoint{1.109739in}{2.200374in}}{\pgfqpoint{1.101503in}{2.200374in}}%
\pgfpathcurveto{\pgfqpoint{1.093266in}{2.200374in}}{\pgfqpoint{1.085366in}{2.197102in}}{\pgfqpoint{1.079542in}{2.191278in}}%
\pgfpathcurveto{\pgfqpoint{1.073718in}{2.185454in}}{\pgfqpoint{1.070446in}{2.177554in}}{\pgfqpoint{1.070446in}{2.169318in}}%
\pgfpathcurveto{\pgfqpoint{1.070446in}{2.161081in}}{\pgfqpoint{1.073718in}{2.153181in}}{\pgfqpoint{1.079542in}{2.147357in}}%
\pgfpathcurveto{\pgfqpoint{1.085366in}{2.141533in}}{\pgfqpoint{1.093266in}{2.138261in}}{\pgfqpoint{1.101503in}{2.138261in}}%
\pgfpathclose%
\pgfusepath{stroke,fill}%
\end{pgfscope}%
\begin{pgfscope}%
\pgfpathrectangle{\pgfqpoint{0.100000in}{0.212622in}}{\pgfqpoint{3.696000in}{3.696000in}}%
\pgfusepath{clip}%
\pgfsetbuttcap%
\pgfsetroundjoin%
\definecolor{currentfill}{rgb}{0.121569,0.466667,0.705882}%
\pgfsetfillcolor{currentfill}%
\pgfsetfillopacity{0.606073}%
\pgfsetlinewidth{1.003750pt}%
\definecolor{currentstroke}{rgb}{0.121569,0.466667,0.705882}%
\pgfsetstrokecolor{currentstroke}%
\pgfsetstrokeopacity{0.606073}%
\pgfsetdash{}{0pt}%
\pgfpathmoveto{\pgfqpoint{1.100827in}{2.138301in}}%
\pgfpathcurveto{\pgfqpoint{1.109063in}{2.138301in}}{\pgfqpoint{1.116963in}{2.141574in}}{\pgfqpoint{1.122787in}{2.147398in}}%
\pgfpathcurveto{\pgfqpoint{1.128611in}{2.153221in}}{\pgfqpoint{1.131883in}{2.161122in}}{\pgfqpoint{1.131883in}{2.169358in}}%
\pgfpathcurveto{\pgfqpoint{1.131883in}{2.177594in}}{\pgfqpoint{1.128611in}{2.185494in}}{\pgfqpoint{1.122787in}{2.191318in}}%
\pgfpathcurveto{\pgfqpoint{1.116963in}{2.197142in}}{\pgfqpoint{1.109063in}{2.200414in}}{\pgfqpoint{1.100827in}{2.200414in}}%
\pgfpathcurveto{\pgfqpoint{1.092590in}{2.200414in}}{\pgfqpoint{1.084690in}{2.197142in}}{\pgfqpoint{1.078866in}{2.191318in}}%
\pgfpathcurveto{\pgfqpoint{1.073042in}{2.185494in}}{\pgfqpoint{1.069770in}{2.177594in}}{\pgfqpoint{1.069770in}{2.169358in}}%
\pgfpathcurveto{\pgfqpoint{1.069770in}{2.161122in}}{\pgfqpoint{1.073042in}{2.153221in}}{\pgfqpoint{1.078866in}{2.147398in}}%
\pgfpathcurveto{\pgfqpoint{1.084690in}{2.141574in}}{\pgfqpoint{1.092590in}{2.138301in}}{\pgfqpoint{1.100827in}{2.138301in}}%
\pgfpathclose%
\pgfusepath{stroke,fill}%
\end{pgfscope}%
\begin{pgfscope}%
\pgfpathrectangle{\pgfqpoint{0.100000in}{0.212622in}}{\pgfqpoint{3.696000in}{3.696000in}}%
\pgfusepath{clip}%
\pgfsetbuttcap%
\pgfsetroundjoin%
\definecolor{currentfill}{rgb}{0.121569,0.466667,0.705882}%
\pgfsetfillcolor{currentfill}%
\pgfsetfillopacity{0.606308}%
\pgfsetlinewidth{1.003750pt}%
\definecolor{currentstroke}{rgb}{0.121569,0.466667,0.705882}%
\pgfsetstrokecolor{currentstroke}%
\pgfsetstrokeopacity{0.606308}%
\pgfsetdash{}{0pt}%
\pgfpathmoveto{\pgfqpoint{1.100266in}{2.138335in}}%
\pgfpathcurveto{\pgfqpoint{1.108502in}{2.138335in}}{\pgfqpoint{1.116402in}{2.141607in}}{\pgfqpoint{1.122226in}{2.147431in}}%
\pgfpathcurveto{\pgfqpoint{1.128050in}{2.153255in}}{\pgfqpoint{1.131322in}{2.161155in}}{\pgfqpoint{1.131322in}{2.169391in}}%
\pgfpathcurveto{\pgfqpoint{1.131322in}{2.177627in}}{\pgfqpoint{1.128050in}{2.185527in}}{\pgfqpoint{1.122226in}{2.191351in}}%
\pgfpathcurveto{\pgfqpoint{1.116402in}{2.197175in}}{\pgfqpoint{1.108502in}{2.200448in}}{\pgfqpoint{1.100266in}{2.200448in}}%
\pgfpathcurveto{\pgfqpoint{1.092029in}{2.200448in}}{\pgfqpoint{1.084129in}{2.197175in}}{\pgfqpoint{1.078306in}{2.191351in}}%
\pgfpathcurveto{\pgfqpoint{1.072482in}{2.185527in}}{\pgfqpoint{1.069209in}{2.177627in}}{\pgfqpoint{1.069209in}{2.169391in}}%
\pgfpathcurveto{\pgfqpoint{1.069209in}{2.161155in}}{\pgfqpoint{1.072482in}{2.153255in}}{\pgfqpoint{1.078306in}{2.147431in}}%
\pgfpathcurveto{\pgfqpoint{1.084129in}{2.141607in}}{\pgfqpoint{1.092029in}{2.138335in}}{\pgfqpoint{1.100266in}{2.138335in}}%
\pgfpathclose%
\pgfusepath{stroke,fill}%
\end{pgfscope}%
\begin{pgfscope}%
\pgfpathrectangle{\pgfqpoint{0.100000in}{0.212622in}}{\pgfqpoint{3.696000in}{3.696000in}}%
\pgfusepath{clip}%
\pgfsetbuttcap%
\pgfsetroundjoin%
\definecolor{currentfill}{rgb}{0.121569,0.466667,0.705882}%
\pgfsetfillcolor{currentfill}%
\pgfsetfillopacity{0.606590}%
\pgfsetlinewidth{1.003750pt}%
\definecolor{currentstroke}{rgb}{0.121569,0.466667,0.705882}%
\pgfsetstrokecolor{currentstroke}%
\pgfsetstrokeopacity{0.606590}%
\pgfsetdash{}{0pt}%
\pgfpathmoveto{\pgfqpoint{3.147697in}{1.783226in}}%
\pgfpathcurveto{\pgfqpoint{3.155934in}{1.783226in}}{\pgfqpoint{3.163834in}{1.786499in}}{\pgfqpoint{3.169658in}{1.792323in}}%
\pgfpathcurveto{\pgfqpoint{3.175482in}{1.798147in}}{\pgfqpoint{3.178754in}{1.806047in}}{\pgfqpoint{3.178754in}{1.814283in}}%
\pgfpathcurveto{\pgfqpoint{3.178754in}{1.822519in}}{\pgfqpoint{3.175482in}{1.830419in}}{\pgfqpoint{3.169658in}{1.836243in}}%
\pgfpathcurveto{\pgfqpoint{3.163834in}{1.842067in}}{\pgfqpoint{3.155934in}{1.845339in}}{\pgfqpoint{3.147697in}{1.845339in}}%
\pgfpathcurveto{\pgfqpoint{3.139461in}{1.845339in}}{\pgfqpoint{3.131561in}{1.842067in}}{\pgfqpoint{3.125737in}{1.836243in}}%
\pgfpathcurveto{\pgfqpoint{3.119913in}{1.830419in}}{\pgfqpoint{3.116641in}{1.822519in}}{\pgfqpoint{3.116641in}{1.814283in}}%
\pgfpathcurveto{\pgfqpoint{3.116641in}{1.806047in}}{\pgfqpoint{3.119913in}{1.798147in}}{\pgfqpoint{3.125737in}{1.792323in}}%
\pgfpathcurveto{\pgfqpoint{3.131561in}{1.786499in}}{\pgfqpoint{3.139461in}{1.783226in}}{\pgfqpoint{3.147697in}{1.783226in}}%
\pgfpathclose%
\pgfusepath{stroke,fill}%
\end{pgfscope}%
\begin{pgfscope}%
\pgfpathrectangle{\pgfqpoint{0.100000in}{0.212622in}}{\pgfqpoint{3.696000in}{3.696000in}}%
\pgfusepath{clip}%
\pgfsetbuttcap%
\pgfsetroundjoin%
\definecolor{currentfill}{rgb}{0.121569,0.466667,0.705882}%
\pgfsetfillcolor{currentfill}%
\pgfsetfillopacity{0.606721}%
\pgfsetlinewidth{1.003750pt}%
\definecolor{currentstroke}{rgb}{0.121569,0.466667,0.705882}%
\pgfsetstrokecolor{currentstroke}%
\pgfsetstrokeopacity{0.606721}%
\pgfsetdash{}{0pt}%
\pgfpathmoveto{\pgfqpoint{1.099124in}{2.138456in}}%
\pgfpathcurveto{\pgfqpoint{1.107360in}{2.138456in}}{\pgfqpoint{1.115260in}{2.141728in}}{\pgfqpoint{1.121084in}{2.147552in}}%
\pgfpathcurveto{\pgfqpoint{1.126908in}{2.153376in}}{\pgfqpoint{1.130180in}{2.161276in}}{\pgfqpoint{1.130180in}{2.169512in}}%
\pgfpathcurveto{\pgfqpoint{1.130180in}{2.177749in}}{\pgfqpoint{1.126908in}{2.185649in}}{\pgfqpoint{1.121084in}{2.191473in}}%
\pgfpathcurveto{\pgfqpoint{1.115260in}{2.197296in}}{\pgfqpoint{1.107360in}{2.200569in}}{\pgfqpoint{1.099124in}{2.200569in}}%
\pgfpathcurveto{\pgfqpoint{1.090888in}{2.200569in}}{\pgfqpoint{1.082988in}{2.197296in}}{\pgfqpoint{1.077164in}{2.191473in}}%
\pgfpathcurveto{\pgfqpoint{1.071340in}{2.185649in}}{\pgfqpoint{1.068067in}{2.177749in}}{\pgfqpoint{1.068067in}{2.169512in}}%
\pgfpathcurveto{\pgfqpoint{1.068067in}{2.161276in}}{\pgfqpoint{1.071340in}{2.153376in}}{\pgfqpoint{1.077164in}{2.147552in}}%
\pgfpathcurveto{\pgfqpoint{1.082988in}{2.141728in}}{\pgfqpoint{1.090888in}{2.138456in}}{\pgfqpoint{1.099124in}{2.138456in}}%
\pgfpathclose%
\pgfusepath{stroke,fill}%
\end{pgfscope}%
\begin{pgfscope}%
\pgfpathrectangle{\pgfqpoint{0.100000in}{0.212622in}}{\pgfqpoint{3.696000in}{3.696000in}}%
\pgfusepath{clip}%
\pgfsetbuttcap%
\pgfsetroundjoin%
\definecolor{currentfill}{rgb}{0.121569,0.466667,0.705882}%
\pgfsetfillcolor{currentfill}%
\pgfsetfillopacity{0.607594}%
\pgfsetlinewidth{1.003750pt}%
\definecolor{currentstroke}{rgb}{0.121569,0.466667,0.705882}%
\pgfsetstrokecolor{currentstroke}%
\pgfsetstrokeopacity{0.607594}%
\pgfsetdash{}{0pt}%
\pgfpathmoveto{\pgfqpoint{1.097998in}{2.138623in}}%
\pgfpathcurveto{\pgfqpoint{1.106234in}{2.138623in}}{\pgfqpoint{1.114134in}{2.141895in}}{\pgfqpoint{1.119958in}{2.147719in}}%
\pgfpathcurveto{\pgfqpoint{1.125782in}{2.153543in}}{\pgfqpoint{1.129054in}{2.161443in}}{\pgfqpoint{1.129054in}{2.169680in}}%
\pgfpathcurveto{\pgfqpoint{1.129054in}{2.177916in}}{\pgfqpoint{1.125782in}{2.185816in}}{\pgfqpoint{1.119958in}{2.191640in}}%
\pgfpathcurveto{\pgfqpoint{1.114134in}{2.197464in}}{\pgfqpoint{1.106234in}{2.200736in}}{\pgfqpoint{1.097998in}{2.200736in}}%
\pgfpathcurveto{\pgfqpoint{1.089762in}{2.200736in}}{\pgfqpoint{1.081861in}{2.197464in}}{\pgfqpoint{1.076038in}{2.191640in}}%
\pgfpathcurveto{\pgfqpoint{1.070214in}{2.185816in}}{\pgfqpoint{1.066941in}{2.177916in}}{\pgfqpoint{1.066941in}{2.169680in}}%
\pgfpathcurveto{\pgfqpoint{1.066941in}{2.161443in}}{\pgfqpoint{1.070214in}{2.153543in}}{\pgfqpoint{1.076038in}{2.147719in}}%
\pgfpathcurveto{\pgfqpoint{1.081861in}{2.141895in}}{\pgfqpoint{1.089762in}{2.138623in}}{\pgfqpoint{1.097998in}{2.138623in}}%
\pgfpathclose%
\pgfusepath{stroke,fill}%
\end{pgfscope}%
\begin{pgfscope}%
\pgfpathrectangle{\pgfqpoint{0.100000in}{0.212622in}}{\pgfqpoint{3.696000in}{3.696000in}}%
\pgfusepath{clip}%
\pgfsetbuttcap%
\pgfsetroundjoin%
\definecolor{currentfill}{rgb}{0.121569,0.466667,0.705882}%
\pgfsetfillcolor{currentfill}%
\pgfsetfillopacity{0.607833}%
\pgfsetlinewidth{1.003750pt}%
\definecolor{currentstroke}{rgb}{0.121569,0.466667,0.705882}%
\pgfsetstrokecolor{currentstroke}%
\pgfsetstrokeopacity{0.607833}%
\pgfsetdash{}{0pt}%
\pgfpathmoveto{\pgfqpoint{1.097258in}{2.138736in}}%
\pgfpathcurveto{\pgfqpoint{1.105495in}{2.138736in}}{\pgfqpoint{1.113395in}{2.142008in}}{\pgfqpoint{1.119219in}{2.147832in}}%
\pgfpathcurveto{\pgfqpoint{1.125043in}{2.153656in}}{\pgfqpoint{1.128315in}{2.161556in}}{\pgfqpoint{1.128315in}{2.169792in}}%
\pgfpathcurveto{\pgfqpoint{1.128315in}{2.178029in}}{\pgfqpoint{1.125043in}{2.185929in}}{\pgfqpoint{1.119219in}{2.191753in}}%
\pgfpathcurveto{\pgfqpoint{1.113395in}{2.197577in}}{\pgfqpoint{1.105495in}{2.200849in}}{\pgfqpoint{1.097258in}{2.200849in}}%
\pgfpathcurveto{\pgfqpoint{1.089022in}{2.200849in}}{\pgfqpoint{1.081122in}{2.197577in}}{\pgfqpoint{1.075298in}{2.191753in}}%
\pgfpathcurveto{\pgfqpoint{1.069474in}{2.185929in}}{\pgfqpoint{1.066202in}{2.178029in}}{\pgfqpoint{1.066202in}{2.169792in}}%
\pgfpathcurveto{\pgfqpoint{1.066202in}{2.161556in}}{\pgfqpoint{1.069474in}{2.153656in}}{\pgfqpoint{1.075298in}{2.147832in}}%
\pgfpathcurveto{\pgfqpoint{1.081122in}{2.142008in}}{\pgfqpoint{1.089022in}{2.138736in}}{\pgfqpoint{1.097258in}{2.138736in}}%
\pgfpathclose%
\pgfusepath{stroke,fill}%
\end{pgfscope}%
\begin{pgfscope}%
\pgfpathrectangle{\pgfqpoint{0.100000in}{0.212622in}}{\pgfqpoint{3.696000in}{3.696000in}}%
\pgfusepath{clip}%
\pgfsetbuttcap%
\pgfsetroundjoin%
\definecolor{currentfill}{rgb}{0.121569,0.466667,0.705882}%
\pgfsetfillcolor{currentfill}%
\pgfsetfillopacity{0.608031}%
\pgfsetlinewidth{1.003750pt}%
\definecolor{currentstroke}{rgb}{0.121569,0.466667,0.705882}%
\pgfsetstrokecolor{currentstroke}%
\pgfsetstrokeopacity{0.608031}%
\pgfsetdash{}{0pt}%
\pgfpathmoveto{\pgfqpoint{3.144242in}{1.784079in}}%
\pgfpathcurveto{\pgfqpoint{3.152478in}{1.784079in}}{\pgfqpoint{3.160378in}{1.787352in}}{\pgfqpoint{3.166202in}{1.793176in}}%
\pgfpathcurveto{\pgfqpoint{3.172026in}{1.799000in}}{\pgfqpoint{3.175298in}{1.806900in}}{\pgfqpoint{3.175298in}{1.815136in}}%
\pgfpathcurveto{\pgfqpoint{3.175298in}{1.823372in}}{\pgfqpoint{3.172026in}{1.831272in}}{\pgfqpoint{3.166202in}{1.837096in}}%
\pgfpathcurveto{\pgfqpoint{3.160378in}{1.842920in}}{\pgfqpoint{3.152478in}{1.846192in}}{\pgfqpoint{3.144242in}{1.846192in}}%
\pgfpathcurveto{\pgfqpoint{3.136006in}{1.846192in}}{\pgfqpoint{3.128106in}{1.842920in}}{\pgfqpoint{3.122282in}{1.837096in}}%
\pgfpathcurveto{\pgfqpoint{3.116458in}{1.831272in}}{\pgfqpoint{3.113185in}{1.823372in}}{\pgfqpoint{3.113185in}{1.815136in}}%
\pgfpathcurveto{\pgfqpoint{3.113185in}{1.806900in}}{\pgfqpoint{3.116458in}{1.799000in}}{\pgfqpoint{3.122282in}{1.793176in}}%
\pgfpathcurveto{\pgfqpoint{3.128106in}{1.787352in}}{\pgfqpoint{3.136006in}{1.784079in}}{\pgfqpoint{3.144242in}{1.784079in}}%
\pgfpathclose%
\pgfusepath{stroke,fill}%
\end{pgfscope}%
\begin{pgfscope}%
\pgfpathrectangle{\pgfqpoint{0.100000in}{0.212622in}}{\pgfqpoint{3.696000in}{3.696000in}}%
\pgfusepath{clip}%
\pgfsetbuttcap%
\pgfsetroundjoin%
\definecolor{currentfill}{rgb}{0.121569,0.466667,0.705882}%
\pgfsetfillcolor{currentfill}%
\pgfsetfillopacity{0.608282}%
\pgfsetlinewidth{1.003750pt}%
\definecolor{currentstroke}{rgb}{0.121569,0.466667,0.705882}%
\pgfsetstrokecolor{currentstroke}%
\pgfsetstrokeopacity{0.608282}%
\pgfsetdash{}{0pt}%
\pgfpathmoveto{\pgfqpoint{1.098529in}{2.139272in}}%
\pgfpathcurveto{\pgfqpoint{1.106766in}{2.139272in}}{\pgfqpoint{1.114666in}{2.142545in}}{\pgfqpoint{1.120490in}{2.148369in}}%
\pgfpathcurveto{\pgfqpoint{1.126314in}{2.154192in}}{\pgfqpoint{1.129586in}{2.162093in}}{\pgfqpoint{1.129586in}{2.170329in}}%
\pgfpathcurveto{\pgfqpoint{1.129586in}{2.178565in}}{\pgfqpoint{1.126314in}{2.186465in}}{\pgfqpoint{1.120490in}{2.192289in}}%
\pgfpathcurveto{\pgfqpoint{1.114666in}{2.198113in}}{\pgfqpoint{1.106766in}{2.201385in}}{\pgfqpoint{1.098529in}{2.201385in}}%
\pgfpathcurveto{\pgfqpoint{1.090293in}{2.201385in}}{\pgfqpoint{1.082393in}{2.198113in}}{\pgfqpoint{1.076569in}{2.192289in}}%
\pgfpathcurveto{\pgfqpoint{1.070745in}{2.186465in}}{\pgfqpoint{1.067473in}{2.178565in}}{\pgfqpoint{1.067473in}{2.170329in}}%
\pgfpathcurveto{\pgfqpoint{1.067473in}{2.162093in}}{\pgfqpoint{1.070745in}{2.154192in}}{\pgfqpoint{1.076569in}{2.148369in}}%
\pgfpathcurveto{\pgfqpoint{1.082393in}{2.142545in}}{\pgfqpoint{1.090293in}{2.139272in}}{\pgfqpoint{1.098529in}{2.139272in}}%
\pgfpathclose%
\pgfusepath{stroke,fill}%
\end{pgfscope}%
\begin{pgfscope}%
\pgfpathrectangle{\pgfqpoint{0.100000in}{0.212622in}}{\pgfqpoint{3.696000in}{3.696000in}}%
\pgfusepath{clip}%
\pgfsetbuttcap%
\pgfsetroundjoin%
\definecolor{currentfill}{rgb}{0.121569,0.466667,0.705882}%
\pgfsetfillcolor{currentfill}%
\pgfsetfillopacity{0.608630}%
\pgfsetlinewidth{1.003750pt}%
\definecolor{currentstroke}{rgb}{0.121569,0.466667,0.705882}%
\pgfsetstrokecolor{currentstroke}%
\pgfsetstrokeopacity{0.608630}%
\pgfsetdash{}{0pt}%
\pgfpathmoveto{\pgfqpoint{1.097669in}{2.139331in}}%
\pgfpathcurveto{\pgfqpoint{1.105905in}{2.139331in}}{\pgfqpoint{1.113805in}{2.142604in}}{\pgfqpoint{1.119629in}{2.148428in}}%
\pgfpathcurveto{\pgfqpoint{1.125453in}{2.154251in}}{\pgfqpoint{1.128725in}{2.162151in}}{\pgfqpoint{1.128725in}{2.170388in}}%
\pgfpathcurveto{\pgfqpoint{1.128725in}{2.178624in}}{\pgfqpoint{1.125453in}{2.186524in}}{\pgfqpoint{1.119629in}{2.192348in}}%
\pgfpathcurveto{\pgfqpoint{1.113805in}{2.198172in}}{\pgfqpoint{1.105905in}{2.201444in}}{\pgfqpoint{1.097669in}{2.201444in}}%
\pgfpathcurveto{\pgfqpoint{1.089433in}{2.201444in}}{\pgfqpoint{1.081532in}{2.198172in}}{\pgfqpoint{1.075709in}{2.192348in}}%
\pgfpathcurveto{\pgfqpoint{1.069885in}{2.186524in}}{\pgfqpoint{1.066612in}{2.178624in}}{\pgfqpoint{1.066612in}{2.170388in}}%
\pgfpathcurveto{\pgfqpoint{1.066612in}{2.162151in}}{\pgfqpoint{1.069885in}{2.154251in}}{\pgfqpoint{1.075709in}{2.148428in}}%
\pgfpathcurveto{\pgfqpoint{1.081532in}{2.142604in}}{\pgfqpoint{1.089433in}{2.139331in}}{\pgfqpoint{1.097669in}{2.139331in}}%
\pgfpathclose%
\pgfusepath{stroke,fill}%
\end{pgfscope}%
\begin{pgfscope}%
\pgfpathrectangle{\pgfqpoint{0.100000in}{0.212622in}}{\pgfqpoint{3.696000in}{3.696000in}}%
\pgfusepath{clip}%
\pgfsetbuttcap%
\pgfsetroundjoin%
\definecolor{currentfill}{rgb}{0.121569,0.466667,0.705882}%
\pgfsetfillcolor{currentfill}%
\pgfsetfillopacity{0.609264}%
\pgfsetlinewidth{1.003750pt}%
\definecolor{currentstroke}{rgb}{0.121569,0.466667,0.705882}%
\pgfsetstrokecolor{currentstroke}%
\pgfsetstrokeopacity{0.609264}%
\pgfsetdash{}{0pt}%
\pgfpathmoveto{\pgfqpoint{1.096131in}{2.139413in}}%
\pgfpathcurveto{\pgfqpoint{1.104367in}{2.139413in}}{\pgfqpoint{1.112267in}{2.142685in}}{\pgfqpoint{1.118091in}{2.148509in}}%
\pgfpathcurveto{\pgfqpoint{1.123915in}{2.154333in}}{\pgfqpoint{1.127187in}{2.162233in}}{\pgfqpoint{1.127187in}{2.170469in}}%
\pgfpathcurveto{\pgfqpoint{1.127187in}{2.178706in}}{\pgfqpoint{1.123915in}{2.186606in}}{\pgfqpoint{1.118091in}{2.192430in}}%
\pgfpathcurveto{\pgfqpoint{1.112267in}{2.198253in}}{\pgfqpoint{1.104367in}{2.201526in}}{\pgfqpoint{1.096131in}{2.201526in}}%
\pgfpathcurveto{\pgfqpoint{1.087894in}{2.201526in}}{\pgfqpoint{1.079994in}{2.198253in}}{\pgfqpoint{1.074170in}{2.192430in}}%
\pgfpathcurveto{\pgfqpoint{1.068346in}{2.186606in}}{\pgfqpoint{1.065074in}{2.178706in}}{\pgfqpoint{1.065074in}{2.170469in}}%
\pgfpathcurveto{\pgfqpoint{1.065074in}{2.162233in}}{\pgfqpoint{1.068346in}{2.154333in}}{\pgfqpoint{1.074170in}{2.148509in}}%
\pgfpathcurveto{\pgfqpoint{1.079994in}{2.142685in}}{\pgfqpoint{1.087894in}{2.139413in}}{\pgfqpoint{1.096131in}{2.139413in}}%
\pgfpathclose%
\pgfusepath{stroke,fill}%
\end{pgfscope}%
\begin{pgfscope}%
\pgfpathrectangle{\pgfqpoint{0.100000in}{0.212622in}}{\pgfqpoint{3.696000in}{3.696000in}}%
\pgfusepath{clip}%
\pgfsetbuttcap%
\pgfsetroundjoin%
\definecolor{currentfill}{rgb}{0.121569,0.466667,0.705882}%
\pgfsetfillcolor{currentfill}%
\pgfsetfillopacity{0.610051}%
\pgfsetlinewidth{1.003750pt}%
\definecolor{currentstroke}{rgb}{0.121569,0.466667,0.705882}%
\pgfsetstrokecolor{currentstroke}%
\pgfsetstrokeopacity{0.610051}%
\pgfsetdash{}{0pt}%
\pgfpathmoveto{\pgfqpoint{3.141286in}{1.784385in}}%
\pgfpathcurveto{\pgfqpoint{3.149523in}{1.784385in}}{\pgfqpoint{3.157423in}{1.787658in}}{\pgfqpoint{3.163247in}{1.793482in}}%
\pgfpathcurveto{\pgfqpoint{3.169071in}{1.799305in}}{\pgfqpoint{3.172343in}{1.807206in}}{\pgfqpoint{3.172343in}{1.815442in}}%
\pgfpathcurveto{\pgfqpoint{3.172343in}{1.823678in}}{\pgfqpoint{3.169071in}{1.831578in}}{\pgfqpoint{3.163247in}{1.837402in}}%
\pgfpathcurveto{\pgfqpoint{3.157423in}{1.843226in}}{\pgfqpoint{3.149523in}{1.846498in}}{\pgfqpoint{3.141286in}{1.846498in}}%
\pgfpathcurveto{\pgfqpoint{3.133050in}{1.846498in}}{\pgfqpoint{3.125150in}{1.843226in}}{\pgfqpoint{3.119326in}{1.837402in}}%
\pgfpathcurveto{\pgfqpoint{3.113502in}{1.831578in}}{\pgfqpoint{3.110230in}{1.823678in}}{\pgfqpoint{3.110230in}{1.815442in}}%
\pgfpathcurveto{\pgfqpoint{3.110230in}{1.807206in}}{\pgfqpoint{3.113502in}{1.799305in}}{\pgfqpoint{3.119326in}{1.793482in}}%
\pgfpathcurveto{\pgfqpoint{3.125150in}{1.787658in}}{\pgfqpoint{3.133050in}{1.784385in}}{\pgfqpoint{3.141286in}{1.784385in}}%
\pgfpathclose%
\pgfusepath{stroke,fill}%
\end{pgfscope}%
\begin{pgfscope}%
\pgfpathrectangle{\pgfqpoint{0.100000in}{0.212622in}}{\pgfqpoint{3.696000in}{3.696000in}}%
\pgfusepath{clip}%
\pgfsetbuttcap%
\pgfsetroundjoin%
\definecolor{currentfill}{rgb}{0.121569,0.466667,0.705882}%
\pgfsetfillcolor{currentfill}%
\pgfsetfillopacity{0.610458}%
\pgfsetlinewidth{1.003750pt}%
\definecolor{currentstroke}{rgb}{0.121569,0.466667,0.705882}%
\pgfsetstrokecolor{currentstroke}%
\pgfsetstrokeopacity{0.610458}%
\pgfsetdash{}{0pt}%
\pgfpathmoveto{\pgfqpoint{1.093871in}{2.139299in}}%
\pgfpathcurveto{\pgfqpoint{1.102107in}{2.139299in}}{\pgfqpoint{1.110007in}{2.142572in}}{\pgfqpoint{1.115831in}{2.148396in}}%
\pgfpathcurveto{\pgfqpoint{1.121655in}{2.154220in}}{\pgfqpoint{1.124927in}{2.162120in}}{\pgfqpoint{1.124927in}{2.170356in}}%
\pgfpathcurveto{\pgfqpoint{1.124927in}{2.178592in}}{\pgfqpoint{1.121655in}{2.186492in}}{\pgfqpoint{1.115831in}{2.192316in}}%
\pgfpathcurveto{\pgfqpoint{1.110007in}{2.198140in}}{\pgfqpoint{1.102107in}{2.201412in}}{\pgfqpoint{1.093871in}{2.201412in}}%
\pgfpathcurveto{\pgfqpoint{1.085634in}{2.201412in}}{\pgfqpoint{1.077734in}{2.198140in}}{\pgfqpoint{1.071910in}{2.192316in}}%
\pgfpathcurveto{\pgfqpoint{1.066086in}{2.186492in}}{\pgfqpoint{1.062814in}{2.178592in}}{\pgfqpoint{1.062814in}{2.170356in}}%
\pgfpathcurveto{\pgfqpoint{1.062814in}{2.162120in}}{\pgfqpoint{1.066086in}{2.154220in}}{\pgfqpoint{1.071910in}{2.148396in}}%
\pgfpathcurveto{\pgfqpoint{1.077734in}{2.142572in}}{\pgfqpoint{1.085634in}{2.139299in}}{\pgfqpoint{1.093871in}{2.139299in}}%
\pgfpathclose%
\pgfusepath{stroke,fill}%
\end{pgfscope}%
\begin{pgfscope}%
\pgfpathrectangle{\pgfqpoint{0.100000in}{0.212622in}}{\pgfqpoint{3.696000in}{3.696000in}}%
\pgfusepath{clip}%
\pgfsetbuttcap%
\pgfsetroundjoin%
\definecolor{currentfill}{rgb}{0.121569,0.466667,0.705882}%
\pgfsetfillcolor{currentfill}%
\pgfsetfillopacity{0.610971}%
\pgfsetlinewidth{1.003750pt}%
\definecolor{currentstroke}{rgb}{0.121569,0.466667,0.705882}%
\pgfsetstrokecolor{currentstroke}%
\pgfsetstrokeopacity{0.610971}%
\pgfsetdash{}{0pt}%
\pgfpathmoveto{\pgfqpoint{1.092490in}{2.139463in}}%
\pgfpathcurveto{\pgfqpoint{1.100727in}{2.139463in}}{\pgfqpoint{1.108627in}{2.142736in}}{\pgfqpoint{1.114451in}{2.148560in}}%
\pgfpathcurveto{\pgfqpoint{1.120275in}{2.154384in}}{\pgfqpoint{1.123547in}{2.162284in}}{\pgfqpoint{1.123547in}{2.170520in}}%
\pgfpathcurveto{\pgfqpoint{1.123547in}{2.178756in}}{\pgfqpoint{1.120275in}{2.186656in}}{\pgfqpoint{1.114451in}{2.192480in}}%
\pgfpathcurveto{\pgfqpoint{1.108627in}{2.198304in}}{\pgfqpoint{1.100727in}{2.201576in}}{\pgfqpoint{1.092490in}{2.201576in}}%
\pgfpathcurveto{\pgfqpoint{1.084254in}{2.201576in}}{\pgfqpoint{1.076354in}{2.198304in}}{\pgfqpoint{1.070530in}{2.192480in}}%
\pgfpathcurveto{\pgfqpoint{1.064706in}{2.186656in}}{\pgfqpoint{1.061434in}{2.178756in}}{\pgfqpoint{1.061434in}{2.170520in}}%
\pgfpathcurveto{\pgfqpoint{1.061434in}{2.162284in}}{\pgfqpoint{1.064706in}{2.154384in}}{\pgfqpoint{1.070530in}{2.148560in}}%
\pgfpathcurveto{\pgfqpoint{1.076354in}{2.142736in}}{\pgfqpoint{1.084254in}{2.139463in}}{\pgfqpoint{1.092490in}{2.139463in}}%
\pgfpathclose%
\pgfusepath{stroke,fill}%
\end{pgfscope}%
\begin{pgfscope}%
\pgfpathrectangle{\pgfqpoint{0.100000in}{0.212622in}}{\pgfqpoint{3.696000in}{3.696000in}}%
\pgfusepath{clip}%
\pgfsetbuttcap%
\pgfsetroundjoin%
\definecolor{currentfill}{rgb}{0.121569,0.466667,0.705882}%
\pgfsetfillcolor{currentfill}%
\pgfsetfillopacity{0.612021}%
\pgfsetlinewidth{1.003750pt}%
\definecolor{currentstroke}{rgb}{0.121569,0.466667,0.705882}%
\pgfsetstrokecolor{currentstroke}%
\pgfsetstrokeopacity{0.612021}%
\pgfsetdash{}{0pt}%
\pgfpathmoveto{\pgfqpoint{1.090916in}{2.139636in}}%
\pgfpathcurveto{\pgfqpoint{1.099152in}{2.139636in}}{\pgfqpoint{1.107052in}{2.142908in}}{\pgfqpoint{1.112876in}{2.148732in}}%
\pgfpathcurveto{\pgfqpoint{1.118700in}{2.154556in}}{\pgfqpoint{1.121972in}{2.162456in}}{\pgfqpoint{1.121972in}{2.170692in}}%
\pgfpathcurveto{\pgfqpoint{1.121972in}{2.178928in}}{\pgfqpoint{1.118700in}{2.186828in}}{\pgfqpoint{1.112876in}{2.192652in}}%
\pgfpathcurveto{\pgfqpoint{1.107052in}{2.198476in}}{\pgfqpoint{1.099152in}{2.201749in}}{\pgfqpoint{1.090916in}{2.201749in}}%
\pgfpathcurveto{\pgfqpoint{1.082680in}{2.201749in}}{\pgfqpoint{1.074780in}{2.198476in}}{\pgfqpoint{1.068956in}{2.192652in}}%
\pgfpathcurveto{\pgfqpoint{1.063132in}{2.186828in}}{\pgfqpoint{1.059859in}{2.178928in}}{\pgfqpoint{1.059859in}{2.170692in}}%
\pgfpathcurveto{\pgfqpoint{1.059859in}{2.162456in}}{\pgfqpoint{1.063132in}{2.154556in}}{\pgfqpoint{1.068956in}{2.148732in}}%
\pgfpathcurveto{\pgfqpoint{1.074780in}{2.142908in}}{\pgfqpoint{1.082680in}{2.139636in}}{\pgfqpoint{1.090916in}{2.139636in}}%
\pgfpathclose%
\pgfusepath{stroke,fill}%
\end{pgfscope}%
\begin{pgfscope}%
\pgfpathrectangle{\pgfqpoint{0.100000in}{0.212622in}}{\pgfqpoint{3.696000in}{3.696000in}}%
\pgfusepath{clip}%
\pgfsetbuttcap%
\pgfsetroundjoin%
\definecolor{currentfill}{rgb}{0.121569,0.466667,0.705882}%
\pgfsetfillcolor{currentfill}%
\pgfsetfillopacity{0.612214}%
\pgfsetlinewidth{1.003750pt}%
\definecolor{currentstroke}{rgb}{0.121569,0.466667,0.705882}%
\pgfsetstrokecolor{currentstroke}%
\pgfsetstrokeopacity{0.612214}%
\pgfsetdash{}{0pt}%
\pgfpathmoveto{\pgfqpoint{3.138470in}{1.784550in}}%
\pgfpathcurveto{\pgfqpoint{3.146706in}{1.784550in}}{\pgfqpoint{3.154606in}{1.787822in}}{\pgfqpoint{3.160430in}{1.793646in}}%
\pgfpathcurveto{\pgfqpoint{3.166254in}{1.799470in}}{\pgfqpoint{3.169526in}{1.807370in}}{\pgfqpoint{3.169526in}{1.815607in}}%
\pgfpathcurveto{\pgfqpoint{3.169526in}{1.823843in}}{\pgfqpoint{3.166254in}{1.831743in}}{\pgfqpoint{3.160430in}{1.837567in}}%
\pgfpathcurveto{\pgfqpoint{3.154606in}{1.843391in}}{\pgfqpoint{3.146706in}{1.846663in}}{\pgfqpoint{3.138470in}{1.846663in}}%
\pgfpathcurveto{\pgfqpoint{3.130233in}{1.846663in}}{\pgfqpoint{3.122333in}{1.843391in}}{\pgfqpoint{3.116509in}{1.837567in}}%
\pgfpathcurveto{\pgfqpoint{3.110685in}{1.831743in}}{\pgfqpoint{3.107413in}{1.823843in}}{\pgfqpoint{3.107413in}{1.815607in}}%
\pgfpathcurveto{\pgfqpoint{3.107413in}{1.807370in}}{\pgfqpoint{3.110685in}{1.799470in}}{\pgfqpoint{3.116509in}{1.793646in}}%
\pgfpathcurveto{\pgfqpoint{3.122333in}{1.787822in}}{\pgfqpoint{3.130233in}{1.784550in}}{\pgfqpoint{3.138470in}{1.784550in}}%
\pgfpathclose%
\pgfusepath{stroke,fill}%
\end{pgfscope}%
\begin{pgfscope}%
\pgfpathrectangle{\pgfqpoint{0.100000in}{0.212622in}}{\pgfqpoint{3.696000in}{3.696000in}}%
\pgfusepath{clip}%
\pgfsetbuttcap%
\pgfsetroundjoin%
\definecolor{currentfill}{rgb}{0.121569,0.466667,0.705882}%
\pgfsetfillcolor{currentfill}%
\pgfsetfillopacity{0.613316}%
\pgfsetlinewidth{1.003750pt}%
\definecolor{currentstroke}{rgb}{0.121569,0.466667,0.705882}%
\pgfsetstrokecolor{currentstroke}%
\pgfsetstrokeopacity{0.613316}%
\pgfsetdash{}{0pt}%
\pgfpathmoveto{\pgfqpoint{3.135953in}{1.785055in}}%
\pgfpathcurveto{\pgfqpoint{3.144189in}{1.785055in}}{\pgfqpoint{3.152089in}{1.788327in}}{\pgfqpoint{3.157913in}{1.794151in}}%
\pgfpathcurveto{\pgfqpoint{3.163737in}{1.799975in}}{\pgfqpoint{3.167009in}{1.807875in}}{\pgfqpoint{3.167009in}{1.816112in}}%
\pgfpathcurveto{\pgfqpoint{3.167009in}{1.824348in}}{\pgfqpoint{3.163737in}{1.832248in}}{\pgfqpoint{3.157913in}{1.838072in}}%
\pgfpathcurveto{\pgfqpoint{3.152089in}{1.843896in}}{\pgfqpoint{3.144189in}{1.847168in}}{\pgfqpoint{3.135953in}{1.847168in}}%
\pgfpathcurveto{\pgfqpoint{3.127717in}{1.847168in}}{\pgfqpoint{3.119816in}{1.843896in}}{\pgfqpoint{3.113993in}{1.838072in}}%
\pgfpathcurveto{\pgfqpoint{3.108169in}{1.832248in}}{\pgfqpoint{3.104896in}{1.824348in}}{\pgfqpoint{3.104896in}{1.816112in}}%
\pgfpathcurveto{\pgfqpoint{3.104896in}{1.807875in}}{\pgfqpoint{3.108169in}{1.799975in}}{\pgfqpoint{3.113993in}{1.794151in}}%
\pgfpathcurveto{\pgfqpoint{3.119816in}{1.788327in}}{\pgfqpoint{3.127717in}{1.785055in}}{\pgfqpoint{3.135953in}{1.785055in}}%
\pgfpathclose%
\pgfusepath{stroke,fill}%
\end{pgfscope}%
\begin{pgfscope}%
\pgfpathrectangle{\pgfqpoint{0.100000in}{0.212622in}}{\pgfqpoint{3.696000in}{3.696000in}}%
\pgfusepath{clip}%
\pgfsetbuttcap%
\pgfsetroundjoin%
\definecolor{currentfill}{rgb}{0.121569,0.466667,0.705882}%
\pgfsetfillcolor{currentfill}%
\pgfsetfillopacity{0.613662}%
\pgfsetlinewidth{1.003750pt}%
\definecolor{currentstroke}{rgb}{0.121569,0.466667,0.705882}%
\pgfsetstrokecolor{currentstroke}%
\pgfsetstrokeopacity{0.613662}%
\pgfsetdash{}{0pt}%
\pgfpathmoveto{\pgfqpoint{1.086005in}{2.140284in}}%
\pgfpathcurveto{\pgfqpoint{1.094242in}{2.140284in}}{\pgfqpoint{1.102142in}{2.143556in}}{\pgfqpoint{1.107966in}{2.149380in}}%
\pgfpathcurveto{\pgfqpoint{1.113789in}{2.155204in}}{\pgfqpoint{1.117062in}{2.163104in}}{\pgfqpoint{1.117062in}{2.171341in}}%
\pgfpathcurveto{\pgfqpoint{1.117062in}{2.179577in}}{\pgfqpoint{1.113789in}{2.187477in}}{\pgfqpoint{1.107966in}{2.193301in}}%
\pgfpathcurveto{\pgfqpoint{1.102142in}{2.199125in}}{\pgfqpoint{1.094242in}{2.202397in}}{\pgfqpoint{1.086005in}{2.202397in}}%
\pgfpathcurveto{\pgfqpoint{1.077769in}{2.202397in}}{\pgfqpoint{1.069869in}{2.199125in}}{\pgfqpoint{1.064045in}{2.193301in}}%
\pgfpathcurveto{\pgfqpoint{1.058221in}{2.187477in}}{\pgfqpoint{1.054949in}{2.179577in}}{\pgfqpoint{1.054949in}{2.171341in}}%
\pgfpathcurveto{\pgfqpoint{1.054949in}{2.163104in}}{\pgfqpoint{1.058221in}{2.155204in}}{\pgfqpoint{1.064045in}{2.149380in}}%
\pgfpathcurveto{\pgfqpoint{1.069869in}{2.143556in}}{\pgfqpoint{1.077769in}{2.140284in}}{\pgfqpoint{1.086005in}{2.140284in}}%
\pgfpathclose%
\pgfusepath{stroke,fill}%
\end{pgfscope}%
\begin{pgfscope}%
\pgfpathrectangle{\pgfqpoint{0.100000in}{0.212622in}}{\pgfqpoint{3.696000in}{3.696000in}}%
\pgfusepath{clip}%
\pgfsetbuttcap%
\pgfsetroundjoin%
\definecolor{currentfill}{rgb}{0.121569,0.466667,0.705882}%
\pgfsetfillcolor{currentfill}%
\pgfsetfillopacity{0.614510}%
\pgfsetlinewidth{1.003750pt}%
\definecolor{currentstroke}{rgb}{0.121569,0.466667,0.705882}%
\pgfsetstrokecolor{currentstroke}%
\pgfsetstrokeopacity{0.614510}%
\pgfsetdash{}{0pt}%
\pgfpathmoveto{\pgfqpoint{3.133158in}{1.785630in}}%
\pgfpathcurveto{\pgfqpoint{3.141395in}{1.785630in}}{\pgfqpoint{3.149295in}{1.788902in}}{\pgfqpoint{3.155119in}{1.794726in}}%
\pgfpathcurveto{\pgfqpoint{3.160943in}{1.800550in}}{\pgfqpoint{3.164215in}{1.808450in}}{\pgfqpoint{3.164215in}{1.816686in}}%
\pgfpathcurveto{\pgfqpoint{3.164215in}{1.824923in}}{\pgfqpoint{3.160943in}{1.832823in}}{\pgfqpoint{3.155119in}{1.838647in}}%
\pgfpathcurveto{\pgfqpoint{3.149295in}{1.844471in}}{\pgfqpoint{3.141395in}{1.847743in}}{\pgfqpoint{3.133158in}{1.847743in}}%
\pgfpathcurveto{\pgfqpoint{3.124922in}{1.847743in}}{\pgfqpoint{3.117022in}{1.844471in}}{\pgfqpoint{3.111198in}{1.838647in}}%
\pgfpathcurveto{\pgfqpoint{3.105374in}{1.832823in}}{\pgfqpoint{3.102102in}{1.824923in}}{\pgfqpoint{3.102102in}{1.816686in}}%
\pgfpathcurveto{\pgfqpoint{3.102102in}{1.808450in}}{\pgfqpoint{3.105374in}{1.800550in}}{\pgfqpoint{3.111198in}{1.794726in}}%
\pgfpathcurveto{\pgfqpoint{3.117022in}{1.788902in}}{\pgfqpoint{3.124922in}{1.785630in}}{\pgfqpoint{3.133158in}{1.785630in}}%
\pgfpathclose%
\pgfusepath{stroke,fill}%
\end{pgfscope}%
\begin{pgfscope}%
\pgfpathrectangle{\pgfqpoint{0.100000in}{0.212622in}}{\pgfqpoint{3.696000in}{3.696000in}}%
\pgfusepath{clip}%
\pgfsetbuttcap%
\pgfsetroundjoin%
\definecolor{currentfill}{rgb}{0.121569,0.466667,0.705882}%
\pgfsetfillcolor{currentfill}%
\pgfsetfillopacity{0.616084}%
\pgfsetlinewidth{1.003750pt}%
\definecolor{currentstroke}{rgb}{0.121569,0.466667,0.705882}%
\pgfsetstrokecolor{currentstroke}%
\pgfsetstrokeopacity{0.616084}%
\pgfsetdash{}{0pt}%
\pgfpathmoveto{\pgfqpoint{3.131209in}{1.785755in}}%
\pgfpathcurveto{\pgfqpoint{3.139446in}{1.785755in}}{\pgfqpoint{3.147346in}{1.789028in}}{\pgfqpoint{3.153169in}{1.794852in}}%
\pgfpathcurveto{\pgfqpoint{3.158993in}{1.800676in}}{\pgfqpoint{3.162266in}{1.808576in}}{\pgfqpoint{3.162266in}{1.816812in}}%
\pgfpathcurveto{\pgfqpoint{3.162266in}{1.825048in}}{\pgfqpoint{3.158993in}{1.832948in}}{\pgfqpoint{3.153169in}{1.838772in}}%
\pgfpathcurveto{\pgfqpoint{3.147346in}{1.844596in}}{\pgfqpoint{3.139446in}{1.847868in}}{\pgfqpoint{3.131209in}{1.847868in}}%
\pgfpathcurveto{\pgfqpoint{3.122973in}{1.847868in}}{\pgfqpoint{3.115073in}{1.844596in}}{\pgfqpoint{3.109249in}{1.838772in}}%
\pgfpathcurveto{\pgfqpoint{3.103425in}{1.832948in}}{\pgfqpoint{3.100153in}{1.825048in}}{\pgfqpoint{3.100153in}{1.816812in}}%
\pgfpathcurveto{\pgfqpoint{3.100153in}{1.808576in}}{\pgfqpoint{3.103425in}{1.800676in}}{\pgfqpoint{3.109249in}{1.794852in}}%
\pgfpathcurveto{\pgfqpoint{3.115073in}{1.789028in}}{\pgfqpoint{3.122973in}{1.785755in}}{\pgfqpoint{3.131209in}{1.785755in}}%
\pgfpathclose%
\pgfusepath{stroke,fill}%
\end{pgfscope}%
\begin{pgfscope}%
\pgfpathrectangle{\pgfqpoint{0.100000in}{0.212622in}}{\pgfqpoint{3.696000in}{3.696000in}}%
\pgfusepath{clip}%
\pgfsetbuttcap%
\pgfsetroundjoin%
\definecolor{currentfill}{rgb}{0.121569,0.466667,0.705882}%
\pgfsetfillcolor{currentfill}%
\pgfsetfillopacity{0.617478}%
\pgfsetlinewidth{1.003750pt}%
\definecolor{currentstroke}{rgb}{0.121569,0.466667,0.705882}%
\pgfsetstrokecolor{currentstroke}%
\pgfsetstrokeopacity{0.617478}%
\pgfsetdash{}{0pt}%
\pgfpathmoveto{\pgfqpoint{1.082747in}{2.142325in}}%
\pgfpathcurveto{\pgfqpoint{1.090984in}{2.142325in}}{\pgfqpoint{1.098884in}{2.145597in}}{\pgfqpoint{1.104708in}{2.151421in}}%
\pgfpathcurveto{\pgfqpoint{1.110532in}{2.157245in}}{\pgfqpoint{1.113804in}{2.165145in}}{\pgfqpoint{1.113804in}{2.173381in}}%
\pgfpathcurveto{\pgfqpoint{1.113804in}{2.181618in}}{\pgfqpoint{1.110532in}{2.189518in}}{\pgfqpoint{1.104708in}{2.195342in}}%
\pgfpathcurveto{\pgfqpoint{1.098884in}{2.201166in}}{\pgfqpoint{1.090984in}{2.204438in}}{\pgfqpoint{1.082747in}{2.204438in}}%
\pgfpathcurveto{\pgfqpoint{1.074511in}{2.204438in}}{\pgfqpoint{1.066611in}{2.201166in}}{\pgfqpoint{1.060787in}{2.195342in}}%
\pgfpathcurveto{\pgfqpoint{1.054963in}{2.189518in}}{\pgfqpoint{1.051691in}{2.181618in}}{\pgfqpoint{1.051691in}{2.173381in}}%
\pgfpathcurveto{\pgfqpoint{1.051691in}{2.165145in}}{\pgfqpoint{1.054963in}{2.157245in}}{\pgfqpoint{1.060787in}{2.151421in}}%
\pgfpathcurveto{\pgfqpoint{1.066611in}{2.145597in}}{\pgfqpoint{1.074511in}{2.142325in}}{\pgfqpoint{1.082747in}{2.142325in}}%
\pgfpathclose%
\pgfusepath{stroke,fill}%
\end{pgfscope}%
\begin{pgfscope}%
\pgfpathrectangle{\pgfqpoint{0.100000in}{0.212622in}}{\pgfqpoint{3.696000in}{3.696000in}}%
\pgfusepath{clip}%
\pgfsetbuttcap%
\pgfsetroundjoin%
\definecolor{currentfill}{rgb}{0.121569,0.466667,0.705882}%
\pgfsetfillcolor{currentfill}%
\pgfsetfillopacity{0.617945}%
\pgfsetlinewidth{1.003750pt}%
\definecolor{currentstroke}{rgb}{0.121569,0.466667,0.705882}%
\pgfsetstrokecolor{currentstroke}%
\pgfsetstrokeopacity{0.617945}%
\pgfsetdash{}{0pt}%
\pgfpathmoveto{\pgfqpoint{3.127384in}{1.786480in}}%
\pgfpathcurveto{\pgfqpoint{3.135620in}{1.786480in}}{\pgfqpoint{3.143520in}{1.789753in}}{\pgfqpoint{3.149344in}{1.795577in}}%
\pgfpathcurveto{\pgfqpoint{3.155168in}{1.801401in}}{\pgfqpoint{3.158440in}{1.809301in}}{\pgfqpoint{3.158440in}{1.817537in}}%
\pgfpathcurveto{\pgfqpoint{3.158440in}{1.825773in}}{\pgfqpoint{3.155168in}{1.833673in}}{\pgfqpoint{3.149344in}{1.839497in}}%
\pgfpathcurveto{\pgfqpoint{3.143520in}{1.845321in}}{\pgfqpoint{3.135620in}{1.848593in}}{\pgfqpoint{3.127384in}{1.848593in}}%
\pgfpathcurveto{\pgfqpoint{3.119147in}{1.848593in}}{\pgfqpoint{3.111247in}{1.845321in}}{\pgfqpoint{3.105424in}{1.839497in}}%
\pgfpathcurveto{\pgfqpoint{3.099600in}{1.833673in}}{\pgfqpoint{3.096327in}{1.825773in}}{\pgfqpoint{3.096327in}{1.817537in}}%
\pgfpathcurveto{\pgfqpoint{3.096327in}{1.809301in}}{\pgfqpoint{3.099600in}{1.801401in}}{\pgfqpoint{3.105424in}{1.795577in}}%
\pgfpathcurveto{\pgfqpoint{3.111247in}{1.789753in}}{\pgfqpoint{3.119147in}{1.786480in}}{\pgfqpoint{3.127384in}{1.786480in}}%
\pgfpathclose%
\pgfusepath{stroke,fill}%
\end{pgfscope}%
\begin{pgfscope}%
\pgfpathrectangle{\pgfqpoint{0.100000in}{0.212622in}}{\pgfqpoint{3.696000in}{3.696000in}}%
\pgfusepath{clip}%
\pgfsetbuttcap%
\pgfsetroundjoin%
\definecolor{currentfill}{rgb}{0.121569,0.466667,0.705882}%
\pgfsetfillcolor{currentfill}%
\pgfsetfillopacity{0.619795}%
\pgfsetlinewidth{1.003750pt}%
\definecolor{currentstroke}{rgb}{0.121569,0.466667,0.705882}%
\pgfsetstrokecolor{currentstroke}%
\pgfsetstrokeopacity{0.619795}%
\pgfsetdash{}{0pt}%
\pgfpathmoveto{\pgfqpoint{3.121993in}{1.787949in}}%
\pgfpathcurveto{\pgfqpoint{3.130230in}{1.787949in}}{\pgfqpoint{3.138130in}{1.791222in}}{\pgfqpoint{3.143954in}{1.797046in}}%
\pgfpathcurveto{\pgfqpoint{3.149778in}{1.802869in}}{\pgfqpoint{3.153050in}{1.810769in}}{\pgfqpoint{3.153050in}{1.819006in}}%
\pgfpathcurveto{\pgfqpoint{3.153050in}{1.827242in}}{\pgfqpoint{3.149778in}{1.835142in}}{\pgfqpoint{3.143954in}{1.840966in}}%
\pgfpathcurveto{\pgfqpoint{3.138130in}{1.846790in}}{\pgfqpoint{3.130230in}{1.850062in}}{\pgfqpoint{3.121993in}{1.850062in}}%
\pgfpathcurveto{\pgfqpoint{3.113757in}{1.850062in}}{\pgfqpoint{3.105857in}{1.846790in}}{\pgfqpoint{3.100033in}{1.840966in}}%
\pgfpathcurveto{\pgfqpoint{3.094209in}{1.835142in}}{\pgfqpoint{3.090937in}{1.827242in}}{\pgfqpoint{3.090937in}{1.819006in}}%
\pgfpathcurveto{\pgfqpoint{3.090937in}{1.810769in}}{\pgfqpoint{3.094209in}{1.802869in}}{\pgfqpoint{3.100033in}{1.797046in}}%
\pgfpathcurveto{\pgfqpoint{3.105857in}{1.791222in}}{\pgfqpoint{3.113757in}{1.787949in}}{\pgfqpoint{3.121993in}{1.787949in}}%
\pgfpathclose%
\pgfusepath{stroke,fill}%
\end{pgfscope}%
\begin{pgfscope}%
\pgfpathrectangle{\pgfqpoint{0.100000in}{0.212622in}}{\pgfqpoint{3.696000in}{3.696000in}}%
\pgfusepath{clip}%
\pgfsetbuttcap%
\pgfsetroundjoin%
\definecolor{currentfill}{rgb}{0.121569,0.466667,0.705882}%
\pgfsetfillcolor{currentfill}%
\pgfsetfillopacity{0.620572}%
\pgfsetlinewidth{1.003750pt}%
\definecolor{currentstroke}{rgb}{0.121569,0.466667,0.705882}%
\pgfsetstrokecolor{currentstroke}%
\pgfsetstrokeopacity{0.620572}%
\pgfsetdash{}{0pt}%
\pgfpathmoveto{\pgfqpoint{1.075637in}{2.142906in}}%
\pgfpathcurveto{\pgfqpoint{1.083873in}{2.142906in}}{\pgfqpoint{1.091773in}{2.146179in}}{\pgfqpoint{1.097597in}{2.152002in}}%
\pgfpathcurveto{\pgfqpoint{1.103421in}{2.157826in}}{\pgfqpoint{1.106693in}{2.165726in}}{\pgfqpoint{1.106693in}{2.173963in}}%
\pgfpathcurveto{\pgfqpoint{1.106693in}{2.182199in}}{\pgfqpoint{1.103421in}{2.190099in}}{\pgfqpoint{1.097597in}{2.195923in}}%
\pgfpathcurveto{\pgfqpoint{1.091773in}{2.201747in}}{\pgfqpoint{1.083873in}{2.205019in}}{\pgfqpoint{1.075637in}{2.205019in}}%
\pgfpathcurveto{\pgfqpoint{1.067401in}{2.205019in}}{\pgfqpoint{1.059501in}{2.201747in}}{\pgfqpoint{1.053677in}{2.195923in}}%
\pgfpathcurveto{\pgfqpoint{1.047853in}{2.190099in}}{\pgfqpoint{1.044580in}{2.182199in}}{\pgfqpoint{1.044580in}{2.173963in}}%
\pgfpathcurveto{\pgfqpoint{1.044580in}{2.165726in}}{\pgfqpoint{1.047853in}{2.157826in}}{\pgfqpoint{1.053677in}{2.152002in}}%
\pgfpathcurveto{\pgfqpoint{1.059501in}{2.146179in}}{\pgfqpoint{1.067401in}{2.142906in}}{\pgfqpoint{1.075637in}{2.142906in}}%
\pgfpathclose%
\pgfusepath{stroke,fill}%
\end{pgfscope}%
\begin{pgfscope}%
\pgfpathrectangle{\pgfqpoint{0.100000in}{0.212622in}}{\pgfqpoint{3.696000in}{3.696000in}}%
\pgfusepath{clip}%
\pgfsetbuttcap%
\pgfsetroundjoin%
\definecolor{currentfill}{rgb}{0.121569,0.466667,0.705882}%
\pgfsetfillcolor{currentfill}%
\pgfsetfillopacity{0.622569}%
\pgfsetlinewidth{1.003750pt}%
\definecolor{currentstroke}{rgb}{0.121569,0.466667,0.705882}%
\pgfsetstrokecolor{currentstroke}%
\pgfsetstrokeopacity{0.622569}%
\pgfsetdash{}{0pt}%
\pgfpathmoveto{\pgfqpoint{3.118903in}{1.788046in}}%
\pgfpathcurveto{\pgfqpoint{3.127139in}{1.788046in}}{\pgfqpoint{3.135039in}{1.791318in}}{\pgfqpoint{3.140863in}{1.797142in}}%
\pgfpathcurveto{\pgfqpoint{3.146687in}{1.802966in}}{\pgfqpoint{3.149960in}{1.810866in}}{\pgfqpoint{3.149960in}{1.819102in}}%
\pgfpathcurveto{\pgfqpoint{3.149960in}{1.827338in}}{\pgfqpoint{3.146687in}{1.835239in}}{\pgfqpoint{3.140863in}{1.841062in}}%
\pgfpathcurveto{\pgfqpoint{3.135039in}{1.846886in}}{\pgfqpoint{3.127139in}{1.850159in}}{\pgfqpoint{3.118903in}{1.850159in}}%
\pgfpathcurveto{\pgfqpoint{3.110667in}{1.850159in}}{\pgfqpoint{3.102767in}{1.846886in}}{\pgfqpoint{3.096943in}{1.841062in}}%
\pgfpathcurveto{\pgfqpoint{3.091119in}{1.835239in}}{\pgfqpoint{3.087847in}{1.827338in}}{\pgfqpoint{3.087847in}{1.819102in}}%
\pgfpathcurveto{\pgfqpoint{3.087847in}{1.810866in}}{\pgfqpoint{3.091119in}{1.802966in}}{\pgfqpoint{3.096943in}{1.797142in}}%
\pgfpathcurveto{\pgfqpoint{3.102767in}{1.791318in}}{\pgfqpoint{3.110667in}{1.788046in}}{\pgfqpoint{3.118903in}{1.788046in}}%
\pgfpathclose%
\pgfusepath{stroke,fill}%
\end{pgfscope}%
\begin{pgfscope}%
\pgfpathrectangle{\pgfqpoint{0.100000in}{0.212622in}}{\pgfqpoint{3.696000in}{3.696000in}}%
\pgfusepath{clip}%
\pgfsetbuttcap%
\pgfsetroundjoin%
\definecolor{currentfill}{rgb}{0.121569,0.466667,0.705882}%
\pgfsetfillcolor{currentfill}%
\pgfsetfillopacity{0.623451}%
\pgfsetlinewidth{1.003750pt}%
\definecolor{currentstroke}{rgb}{0.121569,0.466667,0.705882}%
\pgfsetstrokecolor{currentstroke}%
\pgfsetstrokeopacity{0.623451}%
\pgfsetdash{}{0pt}%
\pgfpathmoveto{\pgfqpoint{1.068144in}{2.143513in}}%
\pgfpathcurveto{\pgfqpoint{1.076380in}{2.143513in}}{\pgfqpoint{1.084280in}{2.146785in}}{\pgfqpoint{1.090104in}{2.152609in}}%
\pgfpathcurveto{\pgfqpoint{1.095928in}{2.158433in}}{\pgfqpoint{1.099200in}{2.166333in}}{\pgfqpoint{1.099200in}{2.174569in}}%
\pgfpathcurveto{\pgfqpoint{1.099200in}{2.182806in}}{\pgfqpoint{1.095928in}{2.190706in}}{\pgfqpoint{1.090104in}{2.196530in}}%
\pgfpathcurveto{\pgfqpoint{1.084280in}{2.202354in}}{\pgfqpoint{1.076380in}{2.205626in}}{\pgfqpoint{1.068144in}{2.205626in}}%
\pgfpathcurveto{\pgfqpoint{1.059908in}{2.205626in}}{\pgfqpoint{1.052008in}{2.202354in}}{\pgfqpoint{1.046184in}{2.196530in}}%
\pgfpathcurveto{\pgfqpoint{1.040360in}{2.190706in}}{\pgfqpoint{1.037087in}{2.182806in}}{\pgfqpoint{1.037087in}{2.174569in}}%
\pgfpathcurveto{\pgfqpoint{1.037087in}{2.166333in}}{\pgfqpoint{1.040360in}{2.158433in}}{\pgfqpoint{1.046184in}{2.152609in}}%
\pgfpathcurveto{\pgfqpoint{1.052008in}{2.146785in}}{\pgfqpoint{1.059908in}{2.143513in}}{\pgfqpoint{1.068144in}{2.143513in}}%
\pgfpathclose%
\pgfusepath{stroke,fill}%
\end{pgfscope}%
\begin{pgfscope}%
\pgfpathrectangle{\pgfqpoint{0.100000in}{0.212622in}}{\pgfqpoint{3.696000in}{3.696000in}}%
\pgfusepath{clip}%
\pgfsetbuttcap%
\pgfsetroundjoin%
\definecolor{currentfill}{rgb}{0.121569,0.466667,0.705882}%
\pgfsetfillcolor{currentfill}%
\pgfsetfillopacity{0.625378}%
\pgfsetlinewidth{1.003750pt}%
\definecolor{currentstroke}{rgb}{0.121569,0.466667,0.705882}%
\pgfsetstrokecolor{currentstroke}%
\pgfsetstrokeopacity{0.625378}%
\pgfsetdash{}{0pt}%
\pgfpathmoveto{\pgfqpoint{3.114910in}{1.788228in}}%
\pgfpathcurveto{\pgfqpoint{3.123146in}{1.788228in}}{\pgfqpoint{3.131046in}{1.791500in}}{\pgfqpoint{3.136870in}{1.797324in}}%
\pgfpathcurveto{\pgfqpoint{3.142694in}{1.803148in}}{\pgfqpoint{3.145966in}{1.811048in}}{\pgfqpoint{3.145966in}{1.819284in}}%
\pgfpathcurveto{\pgfqpoint{3.145966in}{1.827520in}}{\pgfqpoint{3.142694in}{1.835420in}}{\pgfqpoint{3.136870in}{1.841244in}}%
\pgfpathcurveto{\pgfqpoint{3.131046in}{1.847068in}}{\pgfqpoint{3.123146in}{1.850341in}}{\pgfqpoint{3.114910in}{1.850341in}}%
\pgfpathcurveto{\pgfqpoint{3.106674in}{1.850341in}}{\pgfqpoint{3.098774in}{1.847068in}}{\pgfqpoint{3.092950in}{1.841244in}}%
\pgfpathcurveto{\pgfqpoint{3.087126in}{1.835420in}}{\pgfqpoint{3.083853in}{1.827520in}}{\pgfqpoint{3.083853in}{1.819284in}}%
\pgfpathcurveto{\pgfqpoint{3.083853in}{1.811048in}}{\pgfqpoint{3.087126in}{1.803148in}}{\pgfqpoint{3.092950in}{1.797324in}}%
\pgfpathcurveto{\pgfqpoint{3.098774in}{1.791500in}}{\pgfqpoint{3.106674in}{1.788228in}}{\pgfqpoint{3.114910in}{1.788228in}}%
\pgfpathclose%
\pgfusepath{stroke,fill}%
\end{pgfscope}%
\begin{pgfscope}%
\pgfpathrectangle{\pgfqpoint{0.100000in}{0.212622in}}{\pgfqpoint{3.696000in}{3.696000in}}%
\pgfusepath{clip}%
\pgfsetbuttcap%
\pgfsetroundjoin%
\definecolor{currentfill}{rgb}{0.121569,0.466667,0.705882}%
\pgfsetfillcolor{currentfill}%
\pgfsetfillopacity{0.626227}%
\pgfsetlinewidth{1.003750pt}%
\definecolor{currentstroke}{rgb}{0.121569,0.466667,0.705882}%
\pgfsetstrokecolor{currentstroke}%
\pgfsetstrokeopacity{0.626227}%
\pgfsetdash{}{0pt}%
\pgfpathmoveto{\pgfqpoint{1.063767in}{2.143993in}}%
\pgfpathcurveto{\pgfqpoint{1.072003in}{2.143993in}}{\pgfqpoint{1.079903in}{2.147266in}}{\pgfqpoint{1.085727in}{2.153090in}}%
\pgfpathcurveto{\pgfqpoint{1.091551in}{2.158914in}}{\pgfqpoint{1.094823in}{2.166814in}}{\pgfqpoint{1.094823in}{2.175050in}}%
\pgfpathcurveto{\pgfqpoint{1.094823in}{2.183286in}}{\pgfqpoint{1.091551in}{2.191186in}}{\pgfqpoint{1.085727in}{2.197010in}}%
\pgfpathcurveto{\pgfqpoint{1.079903in}{2.202834in}}{\pgfqpoint{1.072003in}{2.206106in}}{\pgfqpoint{1.063767in}{2.206106in}}%
\pgfpathcurveto{\pgfqpoint{1.055531in}{2.206106in}}{\pgfqpoint{1.047631in}{2.202834in}}{\pgfqpoint{1.041807in}{2.197010in}}%
\pgfpathcurveto{\pgfqpoint{1.035983in}{2.191186in}}{\pgfqpoint{1.032710in}{2.183286in}}{\pgfqpoint{1.032710in}{2.175050in}}%
\pgfpathcurveto{\pgfqpoint{1.032710in}{2.166814in}}{\pgfqpoint{1.035983in}{2.158914in}}{\pgfqpoint{1.041807in}{2.153090in}}%
\pgfpathcurveto{\pgfqpoint{1.047631in}{2.147266in}}{\pgfqpoint{1.055531in}{2.143993in}}{\pgfqpoint{1.063767in}{2.143993in}}%
\pgfpathclose%
\pgfusepath{stroke,fill}%
\end{pgfscope}%
\begin{pgfscope}%
\pgfpathrectangle{\pgfqpoint{0.100000in}{0.212622in}}{\pgfqpoint{3.696000in}{3.696000in}}%
\pgfusepath{clip}%
\pgfsetbuttcap%
\pgfsetroundjoin%
\definecolor{currentfill}{rgb}{0.121569,0.466667,0.705882}%
\pgfsetfillcolor{currentfill}%
\pgfsetfillopacity{0.628181}%
\pgfsetlinewidth{1.003750pt}%
\definecolor{currentstroke}{rgb}{0.121569,0.466667,0.705882}%
\pgfsetstrokecolor{currentstroke}%
\pgfsetstrokeopacity{0.628181}%
\pgfsetdash{}{0pt}%
\pgfpathmoveto{\pgfqpoint{3.108364in}{1.789370in}}%
\pgfpathcurveto{\pgfqpoint{3.116601in}{1.789370in}}{\pgfqpoint{3.124501in}{1.792642in}}{\pgfqpoint{3.130325in}{1.798466in}}%
\pgfpathcurveto{\pgfqpoint{3.136149in}{1.804290in}}{\pgfqpoint{3.139421in}{1.812190in}}{\pgfqpoint{3.139421in}{1.820427in}}%
\pgfpathcurveto{\pgfqpoint{3.139421in}{1.828663in}}{\pgfqpoint{3.136149in}{1.836563in}}{\pgfqpoint{3.130325in}{1.842387in}}%
\pgfpathcurveto{\pgfqpoint{3.124501in}{1.848211in}}{\pgfqpoint{3.116601in}{1.851483in}}{\pgfqpoint{3.108364in}{1.851483in}}%
\pgfpathcurveto{\pgfqpoint{3.100128in}{1.851483in}}{\pgfqpoint{3.092228in}{1.848211in}}{\pgfqpoint{3.086404in}{1.842387in}}%
\pgfpathcurveto{\pgfqpoint{3.080580in}{1.836563in}}{\pgfqpoint{3.077308in}{1.828663in}}{\pgfqpoint{3.077308in}{1.820427in}}%
\pgfpathcurveto{\pgfqpoint{3.077308in}{1.812190in}}{\pgfqpoint{3.080580in}{1.804290in}}{\pgfqpoint{3.086404in}{1.798466in}}%
\pgfpathcurveto{\pgfqpoint{3.092228in}{1.792642in}}{\pgfqpoint{3.100128in}{1.789370in}}{\pgfqpoint{3.108364in}{1.789370in}}%
\pgfpathclose%
\pgfusepath{stroke,fill}%
\end{pgfscope}%
\begin{pgfscope}%
\pgfpathrectangle{\pgfqpoint{0.100000in}{0.212622in}}{\pgfqpoint{3.696000in}{3.696000in}}%
\pgfusepath{clip}%
\pgfsetbuttcap%
\pgfsetroundjoin%
\definecolor{currentfill}{rgb}{0.121569,0.466667,0.705882}%
\pgfsetfillcolor{currentfill}%
\pgfsetfillopacity{0.628369}%
\pgfsetlinewidth{1.003750pt}%
\definecolor{currentstroke}{rgb}{0.121569,0.466667,0.705882}%
\pgfsetstrokecolor{currentstroke}%
\pgfsetstrokeopacity{0.628369}%
\pgfsetdash{}{0pt}%
\pgfpathmoveto{\pgfqpoint{1.057496in}{2.144688in}}%
\pgfpathcurveto{\pgfqpoint{1.065732in}{2.144688in}}{\pgfqpoint{1.073632in}{2.147961in}}{\pgfqpoint{1.079456in}{2.153785in}}%
\pgfpathcurveto{\pgfqpoint{1.085280in}{2.159609in}}{\pgfqpoint{1.088552in}{2.167509in}}{\pgfqpoint{1.088552in}{2.175745in}}%
\pgfpathcurveto{\pgfqpoint{1.088552in}{2.183981in}}{\pgfqpoint{1.085280in}{2.191881in}}{\pgfqpoint{1.079456in}{2.197705in}}%
\pgfpathcurveto{\pgfqpoint{1.073632in}{2.203529in}}{\pgfqpoint{1.065732in}{2.206801in}}{\pgfqpoint{1.057496in}{2.206801in}}%
\pgfpathcurveto{\pgfqpoint{1.049259in}{2.206801in}}{\pgfqpoint{1.041359in}{2.203529in}}{\pgfqpoint{1.035535in}{2.197705in}}%
\pgfpathcurveto{\pgfqpoint{1.029711in}{2.191881in}}{\pgfqpoint{1.026439in}{2.183981in}}{\pgfqpoint{1.026439in}{2.175745in}}%
\pgfpathcurveto{\pgfqpoint{1.026439in}{2.167509in}}{\pgfqpoint{1.029711in}{2.159609in}}{\pgfqpoint{1.035535in}{2.153785in}}%
\pgfpathcurveto{\pgfqpoint{1.041359in}{2.147961in}}{\pgfqpoint{1.049259in}{2.144688in}}{\pgfqpoint{1.057496in}{2.144688in}}%
\pgfpathclose%
\pgfusepath{stroke,fill}%
\end{pgfscope}%
\begin{pgfscope}%
\pgfpathrectangle{\pgfqpoint{0.100000in}{0.212622in}}{\pgfqpoint{3.696000in}{3.696000in}}%
\pgfusepath{clip}%
\pgfsetbuttcap%
\pgfsetroundjoin%
\definecolor{currentfill}{rgb}{0.121569,0.466667,0.705882}%
\pgfsetfillcolor{currentfill}%
\pgfsetfillopacity{0.630108}%
\pgfsetlinewidth{1.003750pt}%
\definecolor{currentstroke}{rgb}{0.121569,0.466667,0.705882}%
\pgfsetstrokecolor{currentstroke}%
\pgfsetstrokeopacity{0.630108}%
\pgfsetdash{}{0pt}%
\pgfpathmoveto{\pgfqpoint{1.053519in}{2.144845in}}%
\pgfpathcurveto{\pgfqpoint{1.061755in}{2.144845in}}{\pgfqpoint{1.069655in}{2.148117in}}{\pgfqpoint{1.075479in}{2.153941in}}%
\pgfpathcurveto{\pgfqpoint{1.081303in}{2.159765in}}{\pgfqpoint{1.084576in}{2.167665in}}{\pgfqpoint{1.084576in}{2.175901in}}%
\pgfpathcurveto{\pgfqpoint{1.084576in}{2.184137in}}{\pgfqpoint{1.081303in}{2.192037in}}{\pgfqpoint{1.075479in}{2.197861in}}%
\pgfpathcurveto{\pgfqpoint{1.069655in}{2.203685in}}{\pgfqpoint{1.061755in}{2.206958in}}{\pgfqpoint{1.053519in}{2.206958in}}%
\pgfpathcurveto{\pgfqpoint{1.045283in}{2.206958in}}{\pgfqpoint{1.037383in}{2.203685in}}{\pgfqpoint{1.031559in}{2.197861in}}%
\pgfpathcurveto{\pgfqpoint{1.025735in}{2.192037in}}{\pgfqpoint{1.022463in}{2.184137in}}{\pgfqpoint{1.022463in}{2.175901in}}%
\pgfpathcurveto{\pgfqpoint{1.022463in}{2.167665in}}{\pgfqpoint{1.025735in}{2.159765in}}{\pgfqpoint{1.031559in}{2.153941in}}%
\pgfpathcurveto{\pgfqpoint{1.037383in}{2.148117in}}{\pgfqpoint{1.045283in}{2.144845in}}{\pgfqpoint{1.053519in}{2.144845in}}%
\pgfpathclose%
\pgfusepath{stroke,fill}%
\end{pgfscope}%
\begin{pgfscope}%
\pgfpathrectangle{\pgfqpoint{0.100000in}{0.212622in}}{\pgfqpoint{3.696000in}{3.696000in}}%
\pgfusepath{clip}%
\pgfsetbuttcap%
\pgfsetroundjoin%
\definecolor{currentfill}{rgb}{0.121569,0.466667,0.705882}%
\pgfsetfillcolor{currentfill}%
\pgfsetfillopacity{0.631205}%
\pgfsetlinewidth{1.003750pt}%
\definecolor{currentstroke}{rgb}{0.121569,0.466667,0.705882}%
\pgfsetstrokecolor{currentstroke}%
\pgfsetstrokeopacity{0.631205}%
\pgfsetdash{}{0pt}%
\pgfpathmoveto{\pgfqpoint{3.102251in}{1.790036in}}%
\pgfpathcurveto{\pgfqpoint{3.110487in}{1.790036in}}{\pgfqpoint{3.118387in}{1.793308in}}{\pgfqpoint{3.124211in}{1.799132in}}%
\pgfpathcurveto{\pgfqpoint{3.130035in}{1.804956in}}{\pgfqpoint{3.133307in}{1.812856in}}{\pgfqpoint{3.133307in}{1.821093in}}%
\pgfpathcurveto{\pgfqpoint{3.133307in}{1.829329in}}{\pgfqpoint{3.130035in}{1.837229in}}{\pgfqpoint{3.124211in}{1.843053in}}%
\pgfpathcurveto{\pgfqpoint{3.118387in}{1.848877in}}{\pgfqpoint{3.110487in}{1.852149in}}{\pgfqpoint{3.102251in}{1.852149in}}%
\pgfpathcurveto{\pgfqpoint{3.094015in}{1.852149in}}{\pgfqpoint{3.086114in}{1.848877in}}{\pgfqpoint{3.080291in}{1.843053in}}%
\pgfpathcurveto{\pgfqpoint{3.074467in}{1.837229in}}{\pgfqpoint{3.071194in}{1.829329in}}{\pgfqpoint{3.071194in}{1.821093in}}%
\pgfpathcurveto{\pgfqpoint{3.071194in}{1.812856in}}{\pgfqpoint{3.074467in}{1.804956in}}{\pgfqpoint{3.080291in}{1.799132in}}%
\pgfpathcurveto{\pgfqpoint{3.086114in}{1.793308in}}{\pgfqpoint{3.094015in}{1.790036in}}{\pgfqpoint{3.102251in}{1.790036in}}%
\pgfpathclose%
\pgfusepath{stroke,fill}%
\end{pgfscope}%
\begin{pgfscope}%
\pgfpathrectangle{\pgfqpoint{0.100000in}{0.212622in}}{\pgfqpoint{3.696000in}{3.696000in}}%
\pgfusepath{clip}%
\pgfsetbuttcap%
\pgfsetroundjoin%
\definecolor{currentfill}{rgb}{0.121569,0.466667,0.705882}%
\pgfsetfillcolor{currentfill}%
\pgfsetfillopacity{0.631391}%
\pgfsetlinewidth{1.003750pt}%
\definecolor{currentstroke}{rgb}{0.121569,0.466667,0.705882}%
\pgfsetstrokecolor{currentstroke}%
\pgfsetstrokeopacity{0.631391}%
\pgfsetdash{}{0pt}%
\pgfpathmoveto{\pgfqpoint{1.050087in}{2.145097in}}%
\pgfpathcurveto{\pgfqpoint{1.058323in}{2.145097in}}{\pgfqpoint{1.066224in}{2.148369in}}{\pgfqpoint{1.072047in}{2.154193in}}%
\pgfpathcurveto{\pgfqpoint{1.077871in}{2.160017in}}{\pgfqpoint{1.081144in}{2.167917in}}{\pgfqpoint{1.081144in}{2.176153in}}%
\pgfpathcurveto{\pgfqpoint{1.081144in}{2.184390in}}{\pgfqpoint{1.077871in}{2.192290in}}{\pgfqpoint{1.072047in}{2.198114in}}%
\pgfpathcurveto{\pgfqpoint{1.066224in}{2.203938in}}{\pgfqpoint{1.058323in}{2.207210in}}{\pgfqpoint{1.050087in}{2.207210in}}%
\pgfpathcurveto{\pgfqpoint{1.041851in}{2.207210in}}{\pgfqpoint{1.033951in}{2.203938in}}{\pgfqpoint{1.028127in}{2.198114in}}%
\pgfpathcurveto{\pgfqpoint{1.022303in}{2.192290in}}{\pgfqpoint{1.019031in}{2.184390in}}{\pgfqpoint{1.019031in}{2.176153in}}%
\pgfpathcurveto{\pgfqpoint{1.019031in}{2.167917in}}{\pgfqpoint{1.022303in}{2.160017in}}{\pgfqpoint{1.028127in}{2.154193in}}%
\pgfpathcurveto{\pgfqpoint{1.033951in}{2.148369in}}{\pgfqpoint{1.041851in}{2.145097in}}{\pgfqpoint{1.050087in}{2.145097in}}%
\pgfpathclose%
\pgfusepath{stroke,fill}%
\end{pgfscope}%
\begin{pgfscope}%
\pgfpathrectangle{\pgfqpoint{0.100000in}{0.212622in}}{\pgfqpoint{3.696000in}{3.696000in}}%
\pgfusepath{clip}%
\pgfsetbuttcap%
\pgfsetroundjoin%
\definecolor{currentfill}{rgb}{0.121569,0.466667,0.705882}%
\pgfsetfillcolor{currentfill}%
\pgfsetfillopacity{0.632641}%
\pgfsetlinewidth{1.003750pt}%
\definecolor{currentstroke}{rgb}{0.121569,0.466667,0.705882}%
\pgfsetstrokecolor{currentstroke}%
\pgfsetstrokeopacity{0.632641}%
\pgfsetdash{}{0pt}%
\pgfpathmoveto{\pgfqpoint{1.047554in}{2.145089in}}%
\pgfpathcurveto{\pgfqpoint{1.055791in}{2.145089in}}{\pgfqpoint{1.063691in}{2.148361in}}{\pgfqpoint{1.069515in}{2.154185in}}%
\pgfpathcurveto{\pgfqpoint{1.075339in}{2.160009in}}{\pgfqpoint{1.078611in}{2.167909in}}{\pgfqpoint{1.078611in}{2.176146in}}%
\pgfpathcurveto{\pgfqpoint{1.078611in}{2.184382in}}{\pgfqpoint{1.075339in}{2.192282in}}{\pgfqpoint{1.069515in}{2.198106in}}%
\pgfpathcurveto{\pgfqpoint{1.063691in}{2.203930in}}{\pgfqpoint{1.055791in}{2.207202in}}{\pgfqpoint{1.047554in}{2.207202in}}%
\pgfpathcurveto{\pgfqpoint{1.039318in}{2.207202in}}{\pgfqpoint{1.031418in}{2.203930in}}{\pgfqpoint{1.025594in}{2.198106in}}%
\pgfpathcurveto{\pgfqpoint{1.019770in}{2.192282in}}{\pgfqpoint{1.016498in}{2.184382in}}{\pgfqpoint{1.016498in}{2.176146in}}%
\pgfpathcurveto{\pgfqpoint{1.016498in}{2.167909in}}{\pgfqpoint{1.019770in}{2.160009in}}{\pgfqpoint{1.025594in}{2.154185in}}%
\pgfpathcurveto{\pgfqpoint{1.031418in}{2.148361in}}{\pgfqpoint{1.039318in}{2.145089in}}{\pgfqpoint{1.047554in}{2.145089in}}%
\pgfpathclose%
\pgfusepath{stroke,fill}%
\end{pgfscope}%
\begin{pgfscope}%
\pgfpathrectangle{\pgfqpoint{0.100000in}{0.212622in}}{\pgfqpoint{3.696000in}{3.696000in}}%
\pgfusepath{clip}%
\pgfsetbuttcap%
\pgfsetroundjoin%
\definecolor{currentfill}{rgb}{0.121569,0.466667,0.705882}%
\pgfsetfillcolor{currentfill}%
\pgfsetfillopacity{0.634742}%
\pgfsetlinewidth{1.003750pt}%
\definecolor{currentstroke}{rgb}{0.121569,0.466667,0.705882}%
\pgfsetstrokecolor{currentstroke}%
\pgfsetstrokeopacity{0.634742}%
\pgfsetdash{}{0pt}%
\pgfpathmoveto{\pgfqpoint{3.099496in}{1.790399in}}%
\pgfpathcurveto{\pgfqpoint{3.107732in}{1.790399in}}{\pgfqpoint{3.115632in}{1.793672in}}{\pgfqpoint{3.121456in}{1.799496in}}%
\pgfpathcurveto{\pgfqpoint{3.127280in}{1.805319in}}{\pgfqpoint{3.130553in}{1.813220in}}{\pgfqpoint{3.130553in}{1.821456in}}%
\pgfpathcurveto{\pgfqpoint{3.130553in}{1.829692in}}{\pgfqpoint{3.127280in}{1.837592in}}{\pgfqpoint{3.121456in}{1.843416in}}%
\pgfpathcurveto{\pgfqpoint{3.115632in}{1.849240in}}{\pgfqpoint{3.107732in}{1.852512in}}{\pgfqpoint{3.099496in}{1.852512in}}%
\pgfpathcurveto{\pgfqpoint{3.091260in}{1.852512in}}{\pgfqpoint{3.083360in}{1.849240in}}{\pgfqpoint{3.077536in}{1.843416in}}%
\pgfpathcurveto{\pgfqpoint{3.071712in}{1.837592in}}{\pgfqpoint{3.068440in}{1.829692in}}{\pgfqpoint{3.068440in}{1.821456in}}%
\pgfpathcurveto{\pgfqpoint{3.068440in}{1.813220in}}{\pgfqpoint{3.071712in}{1.805319in}}{\pgfqpoint{3.077536in}{1.799496in}}%
\pgfpathcurveto{\pgfqpoint{3.083360in}{1.793672in}}{\pgfqpoint{3.091260in}{1.790399in}}{\pgfqpoint{3.099496in}{1.790399in}}%
\pgfpathclose%
\pgfusepath{stroke,fill}%
\end{pgfscope}%
\begin{pgfscope}%
\pgfpathrectangle{\pgfqpoint{0.100000in}{0.212622in}}{\pgfqpoint{3.696000in}{3.696000in}}%
\pgfusepath{clip}%
\pgfsetbuttcap%
\pgfsetroundjoin%
\definecolor{currentfill}{rgb}{0.121569,0.466667,0.705882}%
\pgfsetfillcolor{currentfill}%
\pgfsetfillopacity{0.634877}%
\pgfsetlinewidth{1.003750pt}%
\definecolor{currentstroke}{rgb}{0.121569,0.466667,0.705882}%
\pgfsetstrokecolor{currentstroke}%
\pgfsetstrokeopacity{0.634877}%
\pgfsetdash{}{0pt}%
\pgfpathmoveto{\pgfqpoint{1.042345in}{2.145432in}}%
\pgfpathcurveto{\pgfqpoint{1.050582in}{2.145432in}}{\pgfqpoint{1.058482in}{2.148705in}}{\pgfqpoint{1.064306in}{2.154529in}}%
\pgfpathcurveto{\pgfqpoint{1.070130in}{2.160353in}}{\pgfqpoint{1.073402in}{2.168253in}}{\pgfqpoint{1.073402in}{2.176489in}}%
\pgfpathcurveto{\pgfqpoint{1.073402in}{2.184725in}}{\pgfqpoint{1.070130in}{2.192625in}}{\pgfqpoint{1.064306in}{2.198449in}}%
\pgfpathcurveto{\pgfqpoint{1.058482in}{2.204273in}}{\pgfqpoint{1.050582in}{2.207545in}}{\pgfqpoint{1.042345in}{2.207545in}}%
\pgfpathcurveto{\pgfqpoint{1.034109in}{2.207545in}}{\pgfqpoint{1.026209in}{2.204273in}}{\pgfqpoint{1.020385in}{2.198449in}}%
\pgfpathcurveto{\pgfqpoint{1.014561in}{2.192625in}}{\pgfqpoint{1.011289in}{2.184725in}}{\pgfqpoint{1.011289in}{2.176489in}}%
\pgfpathcurveto{\pgfqpoint{1.011289in}{2.168253in}}{\pgfqpoint{1.014561in}{2.160353in}}{\pgfqpoint{1.020385in}{2.154529in}}%
\pgfpathcurveto{\pgfqpoint{1.026209in}{2.148705in}}{\pgfqpoint{1.034109in}{2.145432in}}{\pgfqpoint{1.042345in}{2.145432in}}%
\pgfpathclose%
\pgfusepath{stroke,fill}%
\end{pgfscope}%
\begin{pgfscope}%
\pgfpathrectangle{\pgfqpoint{0.100000in}{0.212622in}}{\pgfqpoint{3.696000in}{3.696000in}}%
\pgfusepath{clip}%
\pgfsetbuttcap%
\pgfsetroundjoin%
\definecolor{currentfill}{rgb}{0.121569,0.466667,0.705882}%
\pgfsetfillcolor{currentfill}%
\pgfsetfillopacity{0.636886}%
\pgfsetlinewidth{1.003750pt}%
\definecolor{currentstroke}{rgb}{0.121569,0.466667,0.705882}%
\pgfsetstrokecolor{currentstroke}%
\pgfsetstrokeopacity{0.636886}%
\pgfsetdash{}{0pt}%
\pgfpathmoveto{\pgfqpoint{1.037406in}{2.145801in}}%
\pgfpathcurveto{\pgfqpoint{1.045642in}{2.145801in}}{\pgfqpoint{1.053542in}{2.149073in}}{\pgfqpoint{1.059366in}{2.154897in}}%
\pgfpathcurveto{\pgfqpoint{1.065190in}{2.160721in}}{\pgfqpoint{1.068462in}{2.168621in}}{\pgfqpoint{1.068462in}{2.176857in}}%
\pgfpathcurveto{\pgfqpoint{1.068462in}{2.185093in}}{\pgfqpoint{1.065190in}{2.192993in}}{\pgfqpoint{1.059366in}{2.198817in}}%
\pgfpathcurveto{\pgfqpoint{1.053542in}{2.204641in}}{\pgfqpoint{1.045642in}{2.207914in}}{\pgfqpoint{1.037406in}{2.207914in}}%
\pgfpathcurveto{\pgfqpoint{1.029170in}{2.207914in}}{\pgfqpoint{1.021269in}{2.204641in}}{\pgfqpoint{1.015446in}{2.198817in}}%
\pgfpathcurveto{\pgfqpoint{1.009622in}{2.192993in}}{\pgfqpoint{1.006349in}{2.185093in}}{\pgfqpoint{1.006349in}{2.176857in}}%
\pgfpathcurveto{\pgfqpoint{1.006349in}{2.168621in}}{\pgfqpoint{1.009622in}{2.160721in}}{\pgfqpoint{1.015446in}{2.154897in}}%
\pgfpathcurveto{\pgfqpoint{1.021269in}{2.149073in}}{\pgfqpoint{1.029170in}{2.145801in}}{\pgfqpoint{1.037406in}{2.145801in}}%
\pgfpathclose%
\pgfusepath{stroke,fill}%
\end{pgfscope}%
\begin{pgfscope}%
\pgfpathrectangle{\pgfqpoint{0.100000in}{0.212622in}}{\pgfqpoint{3.696000in}{3.696000in}}%
\pgfusepath{clip}%
\pgfsetbuttcap%
\pgfsetroundjoin%
\definecolor{currentfill}{rgb}{0.121569,0.466667,0.705882}%
\pgfsetfillcolor{currentfill}%
\pgfsetfillopacity{0.638182}%
\pgfsetlinewidth{1.003750pt}%
\definecolor{currentstroke}{rgb}{0.121569,0.466667,0.705882}%
\pgfsetstrokecolor{currentstroke}%
\pgfsetstrokeopacity{0.638182}%
\pgfsetdash{}{0pt}%
\pgfpathmoveto{\pgfqpoint{3.092247in}{1.791656in}}%
\pgfpathcurveto{\pgfqpoint{3.100484in}{1.791656in}}{\pgfqpoint{3.108384in}{1.794928in}}{\pgfqpoint{3.114208in}{1.800752in}}%
\pgfpathcurveto{\pgfqpoint{3.120032in}{1.806576in}}{\pgfqpoint{3.123304in}{1.814476in}}{\pgfqpoint{3.123304in}{1.822712in}}%
\pgfpathcurveto{\pgfqpoint{3.123304in}{1.830948in}}{\pgfqpoint{3.120032in}{1.838848in}}{\pgfqpoint{3.114208in}{1.844672in}}%
\pgfpathcurveto{\pgfqpoint{3.108384in}{1.850496in}}{\pgfqpoint{3.100484in}{1.853769in}}{\pgfqpoint{3.092247in}{1.853769in}}%
\pgfpathcurveto{\pgfqpoint{3.084011in}{1.853769in}}{\pgfqpoint{3.076111in}{1.850496in}}{\pgfqpoint{3.070287in}{1.844672in}}%
\pgfpathcurveto{\pgfqpoint{3.064463in}{1.838848in}}{\pgfqpoint{3.061191in}{1.830948in}}{\pgfqpoint{3.061191in}{1.822712in}}%
\pgfpathcurveto{\pgfqpoint{3.061191in}{1.814476in}}{\pgfqpoint{3.064463in}{1.806576in}}{\pgfqpoint{3.070287in}{1.800752in}}%
\pgfpathcurveto{\pgfqpoint{3.076111in}{1.794928in}}{\pgfqpoint{3.084011in}{1.791656in}}{\pgfqpoint{3.092247in}{1.791656in}}%
\pgfpathclose%
\pgfusepath{stroke,fill}%
\end{pgfscope}%
\begin{pgfscope}%
\pgfpathrectangle{\pgfqpoint{0.100000in}{0.212622in}}{\pgfqpoint{3.696000in}{3.696000in}}%
\pgfusepath{clip}%
\pgfsetbuttcap%
\pgfsetroundjoin%
\definecolor{currentfill}{rgb}{0.121569,0.466667,0.705882}%
\pgfsetfillcolor{currentfill}%
\pgfsetfillopacity{0.640945}%
\pgfsetlinewidth{1.003750pt}%
\definecolor{currentstroke}{rgb}{0.121569,0.466667,0.705882}%
\pgfsetstrokecolor{currentstroke}%
\pgfsetstrokeopacity{0.640945}%
\pgfsetdash{}{0pt}%
\pgfpathmoveto{\pgfqpoint{1.031069in}{2.146761in}}%
\pgfpathcurveto{\pgfqpoint{1.039306in}{2.146761in}}{\pgfqpoint{1.047206in}{2.150033in}}{\pgfqpoint{1.053030in}{2.155857in}}%
\pgfpathcurveto{\pgfqpoint{1.058854in}{2.161681in}}{\pgfqpoint{1.062126in}{2.169581in}}{\pgfqpoint{1.062126in}{2.177818in}}%
\pgfpathcurveto{\pgfqpoint{1.062126in}{2.186054in}}{\pgfqpoint{1.058854in}{2.193954in}}{\pgfqpoint{1.053030in}{2.199778in}}%
\pgfpathcurveto{\pgfqpoint{1.047206in}{2.205602in}}{\pgfqpoint{1.039306in}{2.208874in}}{\pgfqpoint{1.031069in}{2.208874in}}%
\pgfpathcurveto{\pgfqpoint{1.022833in}{2.208874in}}{\pgfqpoint{1.014933in}{2.205602in}}{\pgfqpoint{1.009109in}{2.199778in}}%
\pgfpathcurveto{\pgfqpoint{1.003285in}{2.193954in}}{\pgfqpoint{1.000013in}{2.186054in}}{\pgfqpoint{1.000013in}{2.177818in}}%
\pgfpathcurveto{\pgfqpoint{1.000013in}{2.169581in}}{\pgfqpoint{1.003285in}{2.161681in}}{\pgfqpoint{1.009109in}{2.155857in}}%
\pgfpathcurveto{\pgfqpoint{1.014933in}{2.150033in}}{\pgfqpoint{1.022833in}{2.146761in}}{\pgfqpoint{1.031069in}{2.146761in}}%
\pgfpathclose%
\pgfusepath{stroke,fill}%
\end{pgfscope}%
\begin{pgfscope}%
\pgfpathrectangle{\pgfqpoint{0.100000in}{0.212622in}}{\pgfqpoint{3.696000in}{3.696000in}}%
\pgfusepath{clip}%
\pgfsetbuttcap%
\pgfsetroundjoin%
\definecolor{currentfill}{rgb}{0.121569,0.466667,0.705882}%
\pgfsetfillcolor{currentfill}%
\pgfsetfillopacity{0.641720}%
\pgfsetlinewidth{1.003750pt}%
\definecolor{currentstroke}{rgb}{0.121569,0.466667,0.705882}%
\pgfsetstrokecolor{currentstroke}%
\pgfsetstrokeopacity{0.641720}%
\pgfsetdash{}{0pt}%
\pgfpathmoveto{\pgfqpoint{3.083118in}{1.793812in}}%
\pgfpathcurveto{\pgfqpoint{3.091355in}{1.793812in}}{\pgfqpoint{3.099255in}{1.797084in}}{\pgfqpoint{3.105079in}{1.802908in}}%
\pgfpathcurveto{\pgfqpoint{3.110903in}{1.808732in}}{\pgfqpoint{3.114175in}{1.816632in}}{\pgfqpoint{3.114175in}{1.824868in}}%
\pgfpathcurveto{\pgfqpoint{3.114175in}{1.833105in}}{\pgfqpoint{3.110903in}{1.841005in}}{\pgfqpoint{3.105079in}{1.846829in}}%
\pgfpathcurveto{\pgfqpoint{3.099255in}{1.852653in}}{\pgfqpoint{3.091355in}{1.855925in}}{\pgfqpoint{3.083118in}{1.855925in}}%
\pgfpathcurveto{\pgfqpoint{3.074882in}{1.855925in}}{\pgfqpoint{3.066982in}{1.852653in}}{\pgfqpoint{3.061158in}{1.846829in}}%
\pgfpathcurveto{\pgfqpoint{3.055334in}{1.841005in}}{\pgfqpoint{3.052062in}{1.833105in}}{\pgfqpoint{3.052062in}{1.824868in}}%
\pgfpathcurveto{\pgfqpoint{3.052062in}{1.816632in}}{\pgfqpoint{3.055334in}{1.808732in}}{\pgfqpoint{3.061158in}{1.802908in}}%
\pgfpathcurveto{\pgfqpoint{3.066982in}{1.797084in}}{\pgfqpoint{3.074882in}{1.793812in}}{\pgfqpoint{3.083118in}{1.793812in}}%
\pgfpathclose%
\pgfusepath{stroke,fill}%
\end{pgfscope}%
\begin{pgfscope}%
\pgfpathrectangle{\pgfqpoint{0.100000in}{0.212622in}}{\pgfqpoint{3.696000in}{3.696000in}}%
\pgfusepath{clip}%
\pgfsetbuttcap%
\pgfsetroundjoin%
\definecolor{currentfill}{rgb}{0.121569,0.466667,0.705882}%
\pgfsetfillcolor{currentfill}%
\pgfsetfillopacity{0.644066}%
\pgfsetlinewidth{1.003750pt}%
\definecolor{currentstroke}{rgb}{0.121569,0.466667,0.705882}%
\pgfsetstrokecolor{currentstroke}%
\pgfsetstrokeopacity{0.644066}%
\pgfsetdash{}{0pt}%
\pgfpathmoveto{\pgfqpoint{1.022542in}{2.147770in}}%
\pgfpathcurveto{\pgfqpoint{1.030779in}{2.147770in}}{\pgfqpoint{1.038679in}{2.151043in}}{\pgfqpoint{1.044503in}{2.156867in}}%
\pgfpathcurveto{\pgfqpoint{1.050327in}{2.162691in}}{\pgfqpoint{1.053599in}{2.170591in}}{\pgfqpoint{1.053599in}{2.178827in}}%
\pgfpathcurveto{\pgfqpoint{1.053599in}{2.187063in}}{\pgfqpoint{1.050327in}{2.194963in}}{\pgfqpoint{1.044503in}{2.200787in}}%
\pgfpathcurveto{\pgfqpoint{1.038679in}{2.206611in}}{\pgfqpoint{1.030779in}{2.209883in}}{\pgfqpoint{1.022542in}{2.209883in}}%
\pgfpathcurveto{\pgfqpoint{1.014306in}{2.209883in}}{\pgfqpoint{1.006406in}{2.206611in}}{\pgfqpoint{1.000582in}{2.200787in}}%
\pgfpathcurveto{\pgfqpoint{0.994758in}{2.194963in}}{\pgfqpoint{0.991486in}{2.187063in}}{\pgfqpoint{0.991486in}{2.178827in}}%
\pgfpathcurveto{\pgfqpoint{0.991486in}{2.170591in}}{\pgfqpoint{0.994758in}{2.162691in}}{\pgfqpoint{1.000582in}{2.156867in}}%
\pgfpathcurveto{\pgfqpoint{1.006406in}{2.151043in}}{\pgfqpoint{1.014306in}{2.147770in}}{\pgfqpoint{1.022542in}{2.147770in}}%
\pgfpathclose%
\pgfusepath{stroke,fill}%
\end{pgfscope}%
\begin{pgfscope}%
\pgfpathrectangle{\pgfqpoint{0.100000in}{0.212622in}}{\pgfqpoint{3.696000in}{3.696000in}}%
\pgfusepath{clip}%
\pgfsetbuttcap%
\pgfsetroundjoin%
\definecolor{currentfill}{rgb}{0.121569,0.466667,0.705882}%
\pgfsetfillcolor{currentfill}%
\pgfsetfillopacity{0.646180}%
\pgfsetlinewidth{1.003750pt}%
\definecolor{currentstroke}{rgb}{0.121569,0.466667,0.705882}%
\pgfsetstrokecolor{currentstroke}%
\pgfsetstrokeopacity{0.646180}%
\pgfsetdash{}{0pt}%
\pgfpathmoveto{\pgfqpoint{3.076842in}{1.793777in}}%
\pgfpathcurveto{\pgfqpoint{3.085078in}{1.793777in}}{\pgfqpoint{3.092979in}{1.797050in}}{\pgfqpoint{3.098802in}{1.802874in}}%
\pgfpathcurveto{\pgfqpoint{3.104626in}{1.808698in}}{\pgfqpoint{3.107899in}{1.816598in}}{\pgfqpoint{3.107899in}{1.824834in}}%
\pgfpathcurveto{\pgfqpoint{3.107899in}{1.833070in}}{\pgfqpoint{3.104626in}{1.840970in}}{\pgfqpoint{3.098802in}{1.846794in}}%
\pgfpathcurveto{\pgfqpoint{3.092979in}{1.852618in}}{\pgfqpoint{3.085078in}{1.855890in}}{\pgfqpoint{3.076842in}{1.855890in}}%
\pgfpathcurveto{\pgfqpoint{3.068606in}{1.855890in}}{\pgfqpoint{3.060706in}{1.852618in}}{\pgfqpoint{3.054882in}{1.846794in}}%
\pgfpathcurveto{\pgfqpoint{3.049058in}{1.840970in}}{\pgfqpoint{3.045786in}{1.833070in}}{\pgfqpoint{3.045786in}{1.824834in}}%
\pgfpathcurveto{\pgfqpoint{3.045786in}{1.816598in}}{\pgfqpoint{3.049058in}{1.808698in}}{\pgfqpoint{3.054882in}{1.802874in}}%
\pgfpathcurveto{\pgfqpoint{3.060706in}{1.797050in}}{\pgfqpoint{3.068606in}{1.793777in}}{\pgfqpoint{3.076842in}{1.793777in}}%
\pgfpathclose%
\pgfusepath{stroke,fill}%
\end{pgfscope}%
\begin{pgfscope}%
\pgfpathrectangle{\pgfqpoint{0.100000in}{0.212622in}}{\pgfqpoint{3.696000in}{3.696000in}}%
\pgfusepath{clip}%
\pgfsetbuttcap%
\pgfsetroundjoin%
\definecolor{currentfill}{rgb}{0.121569,0.466667,0.705882}%
\pgfsetfillcolor{currentfill}%
\pgfsetfillopacity{0.647852}%
\pgfsetlinewidth{1.003750pt}%
\definecolor{currentstroke}{rgb}{0.121569,0.466667,0.705882}%
\pgfsetstrokecolor{currentstroke}%
\pgfsetstrokeopacity{0.647852}%
\pgfsetdash{}{0pt}%
\pgfpathmoveto{\pgfqpoint{1.021347in}{2.149672in}}%
\pgfpathcurveto{\pgfqpoint{1.029584in}{2.149672in}}{\pgfqpoint{1.037484in}{2.152945in}}{\pgfqpoint{1.043308in}{2.158769in}}%
\pgfpathcurveto{\pgfqpoint{1.049132in}{2.164593in}}{\pgfqpoint{1.052404in}{2.172493in}}{\pgfqpoint{1.052404in}{2.180729in}}%
\pgfpathcurveto{\pgfqpoint{1.052404in}{2.188965in}}{\pgfqpoint{1.049132in}{2.196865in}}{\pgfqpoint{1.043308in}{2.202689in}}%
\pgfpathcurveto{\pgfqpoint{1.037484in}{2.208513in}}{\pgfqpoint{1.029584in}{2.211785in}}{\pgfqpoint{1.021347in}{2.211785in}}%
\pgfpathcurveto{\pgfqpoint{1.013111in}{2.211785in}}{\pgfqpoint{1.005211in}{2.208513in}}{\pgfqpoint{0.999387in}{2.202689in}}%
\pgfpathcurveto{\pgfqpoint{0.993563in}{2.196865in}}{\pgfqpoint{0.990291in}{2.188965in}}{\pgfqpoint{0.990291in}{2.180729in}}%
\pgfpathcurveto{\pgfqpoint{0.990291in}{2.172493in}}{\pgfqpoint{0.993563in}{2.164593in}}{\pgfqpoint{0.999387in}{2.158769in}}%
\pgfpathcurveto{\pgfqpoint{1.005211in}{2.152945in}}{\pgfqpoint{1.013111in}{2.149672in}}{\pgfqpoint{1.021347in}{2.149672in}}%
\pgfpathclose%
\pgfusepath{stroke,fill}%
\end{pgfscope}%
\begin{pgfscope}%
\pgfpathrectangle{\pgfqpoint{0.100000in}{0.212622in}}{\pgfqpoint{3.696000in}{3.696000in}}%
\pgfusepath{clip}%
\pgfsetbuttcap%
\pgfsetroundjoin%
\definecolor{currentfill}{rgb}{0.121569,0.466667,0.705882}%
\pgfsetfillcolor{currentfill}%
\pgfsetfillopacity{0.650979}%
\pgfsetlinewidth{1.003750pt}%
\definecolor{currentstroke}{rgb}{0.121569,0.466667,0.705882}%
\pgfsetstrokecolor{currentstroke}%
\pgfsetstrokeopacity{0.650979}%
\pgfsetdash{}{0pt}%
\pgfpathmoveto{\pgfqpoint{3.072270in}{1.793820in}}%
\pgfpathcurveto{\pgfqpoint{3.080506in}{1.793820in}}{\pgfqpoint{3.088406in}{1.797092in}}{\pgfqpoint{3.094230in}{1.802916in}}%
\pgfpathcurveto{\pgfqpoint{3.100054in}{1.808740in}}{\pgfqpoint{3.103327in}{1.816640in}}{\pgfqpoint{3.103327in}{1.824877in}}%
\pgfpathcurveto{\pgfqpoint{3.103327in}{1.833113in}}{\pgfqpoint{3.100054in}{1.841013in}}{\pgfqpoint{3.094230in}{1.846837in}}%
\pgfpathcurveto{\pgfqpoint{3.088406in}{1.852661in}}{\pgfqpoint{3.080506in}{1.855933in}}{\pgfqpoint{3.072270in}{1.855933in}}%
\pgfpathcurveto{\pgfqpoint{3.064034in}{1.855933in}}{\pgfqpoint{3.056134in}{1.852661in}}{\pgfqpoint{3.050310in}{1.846837in}}%
\pgfpathcurveto{\pgfqpoint{3.044486in}{1.841013in}}{\pgfqpoint{3.041214in}{1.833113in}}{\pgfqpoint{3.041214in}{1.824877in}}%
\pgfpathcurveto{\pgfqpoint{3.041214in}{1.816640in}}{\pgfqpoint{3.044486in}{1.808740in}}{\pgfqpoint{3.050310in}{1.802916in}}%
\pgfpathcurveto{\pgfqpoint{3.056134in}{1.797092in}}{\pgfqpoint{3.064034in}{1.793820in}}{\pgfqpoint{3.072270in}{1.793820in}}%
\pgfpathclose%
\pgfusepath{stroke,fill}%
\end{pgfscope}%
\begin{pgfscope}%
\pgfpathrectangle{\pgfqpoint{0.100000in}{0.212622in}}{\pgfqpoint{3.696000in}{3.696000in}}%
\pgfusepath{clip}%
\pgfsetbuttcap%
\pgfsetroundjoin%
\definecolor{currentfill}{rgb}{0.121569,0.466667,0.705882}%
\pgfsetfillcolor{currentfill}%
\pgfsetfillopacity{0.650986}%
\pgfsetlinewidth{1.003750pt}%
\definecolor{currentstroke}{rgb}{0.121569,0.466667,0.705882}%
\pgfsetstrokecolor{currentstroke}%
\pgfsetstrokeopacity{0.650986}%
\pgfsetdash{}{0pt}%
\pgfpathmoveto{\pgfqpoint{1.014990in}{2.150001in}}%
\pgfpathcurveto{\pgfqpoint{1.023226in}{2.150001in}}{\pgfqpoint{1.031127in}{2.153273in}}{\pgfqpoint{1.036950in}{2.159097in}}%
\pgfpathcurveto{\pgfqpoint{1.042774in}{2.164921in}}{\pgfqpoint{1.046047in}{2.172821in}}{\pgfqpoint{1.046047in}{2.181057in}}%
\pgfpathcurveto{\pgfqpoint{1.046047in}{2.189293in}}{\pgfqpoint{1.042774in}{2.197193in}}{\pgfqpoint{1.036950in}{2.203017in}}%
\pgfpathcurveto{\pgfqpoint{1.031127in}{2.208841in}}{\pgfqpoint{1.023226in}{2.212114in}}{\pgfqpoint{1.014990in}{2.212114in}}%
\pgfpathcurveto{\pgfqpoint{1.006754in}{2.212114in}}{\pgfqpoint{0.998854in}{2.208841in}}{\pgfqpoint{0.993030in}{2.203017in}}%
\pgfpathcurveto{\pgfqpoint{0.987206in}{2.197193in}}{\pgfqpoint{0.983934in}{2.189293in}}{\pgfqpoint{0.983934in}{2.181057in}}%
\pgfpathcurveto{\pgfqpoint{0.983934in}{2.172821in}}{\pgfqpoint{0.987206in}{2.164921in}}{\pgfqpoint{0.993030in}{2.159097in}}%
\pgfpathcurveto{\pgfqpoint{0.998854in}{2.153273in}}{\pgfqpoint{1.006754in}{2.150001in}}{\pgfqpoint{1.014990in}{2.150001in}}%
\pgfpathclose%
\pgfusepath{stroke,fill}%
\end{pgfscope}%
\begin{pgfscope}%
\pgfpathrectangle{\pgfqpoint{0.100000in}{0.212622in}}{\pgfqpoint{3.696000in}{3.696000in}}%
\pgfusepath{clip}%
\pgfsetbuttcap%
\pgfsetroundjoin%
\definecolor{currentfill}{rgb}{0.121569,0.466667,0.705882}%
\pgfsetfillcolor{currentfill}%
\pgfsetfillopacity{0.653758}%
\pgfsetlinewidth{1.003750pt}%
\definecolor{currentstroke}{rgb}{0.121569,0.466667,0.705882}%
\pgfsetstrokecolor{currentstroke}%
\pgfsetstrokeopacity{0.653758}%
\pgfsetdash{}{0pt}%
\pgfpathmoveto{\pgfqpoint{1.007144in}{2.150751in}}%
\pgfpathcurveto{\pgfqpoint{1.015380in}{2.150751in}}{\pgfqpoint{1.023280in}{2.154023in}}{\pgfqpoint{1.029104in}{2.159847in}}%
\pgfpathcurveto{\pgfqpoint{1.034928in}{2.165671in}}{\pgfqpoint{1.038200in}{2.173571in}}{\pgfqpoint{1.038200in}{2.181807in}}%
\pgfpathcurveto{\pgfqpoint{1.038200in}{2.190044in}}{\pgfqpoint{1.034928in}{2.197944in}}{\pgfqpoint{1.029104in}{2.203768in}}%
\pgfpathcurveto{\pgfqpoint{1.023280in}{2.209591in}}{\pgfqpoint{1.015380in}{2.212864in}}{\pgfqpoint{1.007144in}{2.212864in}}%
\pgfpathcurveto{\pgfqpoint{0.998908in}{2.212864in}}{\pgfqpoint{0.991008in}{2.209591in}}{\pgfqpoint{0.985184in}{2.203768in}}%
\pgfpathcurveto{\pgfqpoint{0.979360in}{2.197944in}}{\pgfqpoint{0.976087in}{2.190044in}}{\pgfqpoint{0.976087in}{2.181807in}}%
\pgfpathcurveto{\pgfqpoint{0.976087in}{2.173571in}}{\pgfqpoint{0.979360in}{2.165671in}}{\pgfqpoint{0.985184in}{2.159847in}}%
\pgfpathcurveto{\pgfqpoint{0.991008in}{2.154023in}}{\pgfqpoint{0.998908in}{2.150751in}}{\pgfqpoint{1.007144in}{2.150751in}}%
\pgfpathclose%
\pgfusepath{stroke,fill}%
\end{pgfscope}%
\begin{pgfscope}%
\pgfpathrectangle{\pgfqpoint{0.100000in}{0.212622in}}{\pgfqpoint{3.696000in}{3.696000in}}%
\pgfusepath{clip}%
\pgfsetbuttcap%
\pgfsetroundjoin%
\definecolor{currentfill}{rgb}{0.121569,0.466667,0.705882}%
\pgfsetfillcolor{currentfill}%
\pgfsetfillopacity{0.655697}%
\pgfsetlinewidth{1.003750pt}%
\definecolor{currentstroke}{rgb}{0.121569,0.466667,0.705882}%
\pgfsetstrokecolor{currentstroke}%
\pgfsetstrokeopacity{0.655697}%
\pgfsetdash{}{0pt}%
\pgfpathmoveto{\pgfqpoint{3.062324in}{1.795491in}}%
\pgfpathcurveto{\pgfqpoint{3.070560in}{1.795491in}}{\pgfqpoint{3.078460in}{1.798764in}}{\pgfqpoint{3.084284in}{1.804588in}}%
\pgfpathcurveto{\pgfqpoint{3.090108in}{1.810412in}}{\pgfqpoint{3.093381in}{1.818312in}}{\pgfqpoint{3.093381in}{1.826548in}}%
\pgfpathcurveto{\pgfqpoint{3.093381in}{1.834784in}}{\pgfqpoint{3.090108in}{1.842684in}}{\pgfqpoint{3.084284in}{1.848508in}}%
\pgfpathcurveto{\pgfqpoint{3.078460in}{1.854332in}}{\pgfqpoint{3.070560in}{1.857604in}}{\pgfqpoint{3.062324in}{1.857604in}}%
\pgfpathcurveto{\pgfqpoint{3.054088in}{1.857604in}}{\pgfqpoint{3.046188in}{1.854332in}}{\pgfqpoint{3.040364in}{1.848508in}}%
\pgfpathcurveto{\pgfqpoint{3.034540in}{1.842684in}}{\pgfqpoint{3.031268in}{1.834784in}}{\pgfqpoint{3.031268in}{1.826548in}}%
\pgfpathcurveto{\pgfqpoint{3.031268in}{1.818312in}}{\pgfqpoint{3.034540in}{1.810412in}}{\pgfqpoint{3.040364in}{1.804588in}}%
\pgfpathcurveto{\pgfqpoint{3.046188in}{1.798764in}}{\pgfqpoint{3.054088in}{1.795491in}}{\pgfqpoint{3.062324in}{1.795491in}}%
\pgfpathclose%
\pgfusepath{stroke,fill}%
\end{pgfscope}%
\begin{pgfscope}%
\pgfpathrectangle{\pgfqpoint{0.100000in}{0.212622in}}{\pgfqpoint{3.696000in}{3.696000in}}%
\pgfusepath{clip}%
\pgfsetbuttcap%
\pgfsetroundjoin%
\definecolor{currentfill}{rgb}{0.121569,0.466667,0.705882}%
\pgfsetfillcolor{currentfill}%
\pgfsetfillopacity{0.656639}%
\pgfsetlinewidth{1.003750pt}%
\definecolor{currentstroke}{rgb}{0.121569,0.466667,0.705882}%
\pgfsetstrokecolor{currentstroke}%
\pgfsetstrokeopacity{0.656639}%
\pgfsetdash{}{0pt}%
\pgfpathmoveto{\pgfqpoint{1.003853in}{2.151496in}}%
\pgfpathcurveto{\pgfqpoint{1.012089in}{2.151496in}}{\pgfqpoint{1.019989in}{2.154768in}}{\pgfqpoint{1.025813in}{2.160592in}}%
\pgfpathcurveto{\pgfqpoint{1.031637in}{2.166416in}}{\pgfqpoint{1.034909in}{2.174316in}}{\pgfqpoint{1.034909in}{2.182553in}}%
\pgfpathcurveto{\pgfqpoint{1.034909in}{2.190789in}}{\pgfqpoint{1.031637in}{2.198689in}}{\pgfqpoint{1.025813in}{2.204513in}}%
\pgfpathcurveto{\pgfqpoint{1.019989in}{2.210337in}}{\pgfqpoint{1.012089in}{2.213609in}}{\pgfqpoint{1.003853in}{2.213609in}}%
\pgfpathcurveto{\pgfqpoint{0.995616in}{2.213609in}}{\pgfqpoint{0.987716in}{2.210337in}}{\pgfqpoint{0.981892in}{2.204513in}}%
\pgfpathcurveto{\pgfqpoint{0.976069in}{2.198689in}}{\pgfqpoint{0.972796in}{2.190789in}}{\pgfqpoint{0.972796in}{2.182553in}}%
\pgfpathcurveto{\pgfqpoint{0.972796in}{2.174316in}}{\pgfqpoint{0.976069in}{2.166416in}}{\pgfqpoint{0.981892in}{2.160592in}}%
\pgfpathcurveto{\pgfqpoint{0.987716in}{2.154768in}}{\pgfqpoint{0.995616in}{2.151496in}}{\pgfqpoint{1.003853in}{2.151496in}}%
\pgfpathclose%
\pgfusepath{stroke,fill}%
\end{pgfscope}%
\begin{pgfscope}%
\pgfpathrectangle{\pgfqpoint{0.100000in}{0.212622in}}{\pgfqpoint{3.696000in}{3.696000in}}%
\pgfusepath{clip}%
\pgfsetbuttcap%
\pgfsetroundjoin%
\definecolor{currentfill}{rgb}{0.121569,0.466667,0.705882}%
\pgfsetfillcolor{currentfill}%
\pgfsetfillopacity{0.658866}%
\pgfsetlinewidth{1.003750pt}%
\definecolor{currentstroke}{rgb}{0.121569,0.466667,0.705882}%
\pgfsetstrokecolor{currentstroke}%
\pgfsetstrokeopacity{0.658866}%
\pgfsetdash{}{0pt}%
\pgfpathmoveto{\pgfqpoint{0.996803in}{2.152503in}}%
\pgfpathcurveto{\pgfqpoint{1.005039in}{2.152503in}}{\pgfqpoint{1.012939in}{2.155775in}}{\pgfqpoint{1.018763in}{2.161599in}}%
\pgfpathcurveto{\pgfqpoint{1.024587in}{2.167423in}}{\pgfqpoint{1.027859in}{2.175323in}}{\pgfqpoint{1.027859in}{2.183560in}}%
\pgfpathcurveto{\pgfqpoint{1.027859in}{2.191796in}}{\pgfqpoint{1.024587in}{2.199696in}}{\pgfqpoint{1.018763in}{2.205520in}}%
\pgfpathcurveto{\pgfqpoint{1.012939in}{2.211344in}}{\pgfqpoint{1.005039in}{2.214616in}}{\pgfqpoint{0.996803in}{2.214616in}}%
\pgfpathcurveto{\pgfqpoint{0.988566in}{2.214616in}}{\pgfqpoint{0.980666in}{2.211344in}}{\pgfqpoint{0.974842in}{2.205520in}}%
\pgfpathcurveto{\pgfqpoint{0.969019in}{2.199696in}}{\pgfqpoint{0.965746in}{2.191796in}}{\pgfqpoint{0.965746in}{2.183560in}}%
\pgfpathcurveto{\pgfqpoint{0.965746in}{2.175323in}}{\pgfqpoint{0.969019in}{2.167423in}}{\pgfqpoint{0.974842in}{2.161599in}}%
\pgfpathcurveto{\pgfqpoint{0.980666in}{2.155775in}}{\pgfqpoint{0.988566in}{2.152503in}}{\pgfqpoint{0.996803in}{2.152503in}}%
\pgfpathclose%
\pgfusepath{stroke,fill}%
\end{pgfscope}%
\begin{pgfscope}%
\pgfpathrectangle{\pgfqpoint{0.100000in}{0.212622in}}{\pgfqpoint{3.696000in}{3.696000in}}%
\pgfusepath{clip}%
\pgfsetbuttcap%
\pgfsetroundjoin%
\definecolor{currentfill}{rgb}{0.121569,0.466667,0.705882}%
\pgfsetfillcolor{currentfill}%
\pgfsetfillopacity{0.660497}%
\pgfsetlinewidth{1.003750pt}%
\definecolor{currentstroke}{rgb}{0.121569,0.466667,0.705882}%
\pgfsetstrokecolor{currentstroke}%
\pgfsetstrokeopacity{0.660497}%
\pgfsetdash{}{0pt}%
\pgfpathmoveto{\pgfqpoint{3.051188in}{1.797985in}}%
\pgfpathcurveto{\pgfqpoint{3.059425in}{1.797985in}}{\pgfqpoint{3.067325in}{1.801258in}}{\pgfqpoint{3.073149in}{1.807082in}}%
\pgfpathcurveto{\pgfqpoint{3.078973in}{1.812906in}}{\pgfqpoint{3.082245in}{1.820806in}}{\pgfqpoint{3.082245in}{1.829042in}}%
\pgfpathcurveto{\pgfqpoint{3.082245in}{1.837278in}}{\pgfqpoint{3.078973in}{1.845178in}}{\pgfqpoint{3.073149in}{1.851002in}}%
\pgfpathcurveto{\pgfqpoint{3.067325in}{1.856826in}}{\pgfqpoint{3.059425in}{1.860098in}}{\pgfqpoint{3.051188in}{1.860098in}}%
\pgfpathcurveto{\pgfqpoint{3.042952in}{1.860098in}}{\pgfqpoint{3.035052in}{1.856826in}}{\pgfqpoint{3.029228in}{1.851002in}}%
\pgfpathcurveto{\pgfqpoint{3.023404in}{1.845178in}}{\pgfqpoint{3.020132in}{1.837278in}}{\pgfqpoint{3.020132in}{1.829042in}}%
\pgfpathcurveto{\pgfqpoint{3.020132in}{1.820806in}}{\pgfqpoint{3.023404in}{1.812906in}}{\pgfqpoint{3.029228in}{1.807082in}}%
\pgfpathcurveto{\pgfqpoint{3.035052in}{1.801258in}}{\pgfqpoint{3.042952in}{1.797985in}}{\pgfqpoint{3.051188in}{1.797985in}}%
\pgfpathclose%
\pgfusepath{stroke,fill}%
\end{pgfscope}%
\begin{pgfscope}%
\pgfpathrectangle{\pgfqpoint{0.100000in}{0.212622in}}{\pgfqpoint{3.696000in}{3.696000in}}%
\pgfusepath{clip}%
\pgfsetbuttcap%
\pgfsetroundjoin%
\definecolor{currentfill}{rgb}{0.121569,0.466667,0.705882}%
\pgfsetfillcolor{currentfill}%
\pgfsetfillopacity{0.660846}%
\pgfsetlinewidth{1.003750pt}%
\definecolor{currentstroke}{rgb}{0.121569,0.466667,0.705882}%
\pgfsetstrokecolor{currentstroke}%
\pgfsetstrokeopacity{0.660846}%
\pgfsetdash{}{0pt}%
\pgfpathmoveto{\pgfqpoint{0.993173in}{2.152526in}}%
\pgfpathcurveto{\pgfqpoint{1.001409in}{2.152526in}}{\pgfqpoint{1.009309in}{2.155798in}}{\pgfqpoint{1.015133in}{2.161622in}}%
\pgfpathcurveto{\pgfqpoint{1.020957in}{2.167446in}}{\pgfqpoint{1.024229in}{2.175346in}}{\pgfqpoint{1.024229in}{2.183582in}}%
\pgfpathcurveto{\pgfqpoint{1.024229in}{2.191818in}}{\pgfqpoint{1.020957in}{2.199719in}}{\pgfqpoint{1.015133in}{2.205542in}}%
\pgfpathcurveto{\pgfqpoint{1.009309in}{2.211366in}}{\pgfqpoint{1.001409in}{2.214639in}}{\pgfqpoint{0.993173in}{2.214639in}}%
\pgfpathcurveto{\pgfqpoint{0.984936in}{2.214639in}}{\pgfqpoint{0.977036in}{2.211366in}}{\pgfqpoint{0.971212in}{2.205542in}}%
\pgfpathcurveto{\pgfqpoint{0.965388in}{2.199719in}}{\pgfqpoint{0.962116in}{2.191818in}}{\pgfqpoint{0.962116in}{2.183582in}}%
\pgfpathcurveto{\pgfqpoint{0.962116in}{2.175346in}}{\pgfqpoint{0.965388in}{2.167446in}}{\pgfqpoint{0.971212in}{2.161622in}}%
\pgfpathcurveto{\pgfqpoint{0.977036in}{2.155798in}}{\pgfqpoint{0.984936in}{2.152526in}}{\pgfqpoint{0.993173in}{2.152526in}}%
\pgfpathclose%
\pgfusepath{stroke,fill}%
\end{pgfscope}%
\begin{pgfscope}%
\pgfpathrectangle{\pgfqpoint{0.100000in}{0.212622in}}{\pgfqpoint{3.696000in}{3.696000in}}%
\pgfusepath{clip}%
\pgfsetbuttcap%
\pgfsetroundjoin%
\definecolor{currentfill}{rgb}{0.121569,0.466667,0.705882}%
\pgfsetfillcolor{currentfill}%
\pgfsetfillopacity{0.661942}%
\pgfsetlinewidth{1.003750pt}%
\definecolor{currentstroke}{rgb}{0.121569,0.466667,0.705882}%
\pgfsetstrokecolor{currentstroke}%
\pgfsetstrokeopacity{0.661942}%
\pgfsetdash{}{0pt}%
\pgfpathmoveto{\pgfqpoint{0.990144in}{2.152841in}}%
\pgfpathcurveto{\pgfqpoint{0.998380in}{2.152841in}}{\pgfqpoint{1.006280in}{2.156113in}}{\pgfqpoint{1.012104in}{2.161937in}}%
\pgfpathcurveto{\pgfqpoint{1.017928in}{2.167761in}}{\pgfqpoint{1.021200in}{2.175661in}}{\pgfqpoint{1.021200in}{2.183897in}}%
\pgfpathcurveto{\pgfqpoint{1.021200in}{2.192133in}}{\pgfqpoint{1.017928in}{2.200033in}}{\pgfqpoint{1.012104in}{2.205857in}}%
\pgfpathcurveto{\pgfqpoint{1.006280in}{2.211681in}}{\pgfqpoint{0.998380in}{2.214954in}}{\pgfqpoint{0.990144in}{2.214954in}}%
\pgfpathcurveto{\pgfqpoint{0.981907in}{2.214954in}}{\pgfqpoint{0.974007in}{2.211681in}}{\pgfqpoint{0.968183in}{2.205857in}}%
\pgfpathcurveto{\pgfqpoint{0.962359in}{2.200033in}}{\pgfqpoint{0.959087in}{2.192133in}}{\pgfqpoint{0.959087in}{2.183897in}}%
\pgfpathcurveto{\pgfqpoint{0.959087in}{2.175661in}}{\pgfqpoint{0.962359in}{2.167761in}}{\pgfqpoint{0.968183in}{2.161937in}}%
\pgfpathcurveto{\pgfqpoint{0.974007in}{2.156113in}}{\pgfqpoint{0.981907in}{2.152841in}}{\pgfqpoint{0.990144in}{2.152841in}}%
\pgfpathclose%
\pgfusepath{stroke,fill}%
\end{pgfscope}%
\begin{pgfscope}%
\pgfpathrectangle{\pgfqpoint{0.100000in}{0.212622in}}{\pgfqpoint{3.696000in}{3.696000in}}%
\pgfusepath{clip}%
\pgfsetbuttcap%
\pgfsetroundjoin%
\definecolor{currentfill}{rgb}{0.121569,0.466667,0.705882}%
\pgfsetfillcolor{currentfill}%
\pgfsetfillopacity{0.664086}%
\pgfsetlinewidth{1.003750pt}%
\definecolor{currentstroke}{rgb}{0.121569,0.466667,0.705882}%
\pgfsetstrokecolor{currentstroke}%
\pgfsetstrokeopacity{0.664086}%
\pgfsetdash{}{0pt}%
\pgfpathmoveto{\pgfqpoint{0.985927in}{2.152948in}}%
\pgfpathcurveto{\pgfqpoint{0.994163in}{2.152948in}}{\pgfqpoint{1.002063in}{2.156220in}}{\pgfqpoint{1.007887in}{2.162044in}}%
\pgfpathcurveto{\pgfqpoint{1.013711in}{2.167868in}}{\pgfqpoint{1.016983in}{2.175768in}}{\pgfqpoint{1.016983in}{2.184004in}}%
\pgfpathcurveto{\pgfqpoint{1.016983in}{2.192241in}}{\pgfqpoint{1.013711in}{2.200141in}}{\pgfqpoint{1.007887in}{2.205965in}}%
\pgfpathcurveto{\pgfqpoint{1.002063in}{2.211789in}}{\pgfqpoint{0.994163in}{2.215061in}}{\pgfqpoint{0.985927in}{2.215061in}}%
\pgfpathcurveto{\pgfqpoint{0.977691in}{2.215061in}}{\pgfqpoint{0.969791in}{2.211789in}}{\pgfqpoint{0.963967in}{2.205965in}}%
\pgfpathcurveto{\pgfqpoint{0.958143in}{2.200141in}}{\pgfqpoint{0.954870in}{2.192241in}}{\pgfqpoint{0.954870in}{2.184004in}}%
\pgfpathcurveto{\pgfqpoint{0.954870in}{2.175768in}}{\pgfqpoint{0.958143in}{2.167868in}}{\pgfqpoint{0.963967in}{2.162044in}}%
\pgfpathcurveto{\pgfqpoint{0.969791in}{2.156220in}}{\pgfqpoint{0.977691in}{2.152948in}}{\pgfqpoint{0.985927in}{2.152948in}}%
\pgfpathclose%
\pgfusepath{stroke,fill}%
\end{pgfscope}%
\begin{pgfscope}%
\pgfpathrectangle{\pgfqpoint{0.100000in}{0.212622in}}{\pgfqpoint{3.696000in}{3.696000in}}%
\pgfusepath{clip}%
\pgfsetbuttcap%
\pgfsetroundjoin%
\definecolor{currentfill}{rgb}{0.121569,0.466667,0.705882}%
\pgfsetfillcolor{currentfill}%
\pgfsetfillopacity{0.666129}%
\pgfsetlinewidth{1.003750pt}%
\definecolor{currentstroke}{rgb}{0.121569,0.466667,0.705882}%
\pgfsetstrokecolor{currentstroke}%
\pgfsetstrokeopacity{0.666129}%
\pgfsetdash{}{0pt}%
\pgfpathmoveto{\pgfqpoint{0.981789in}{2.152920in}}%
\pgfpathcurveto{\pgfqpoint{0.990026in}{2.152920in}}{\pgfqpoint{0.997926in}{2.156193in}}{\pgfqpoint{1.003749in}{2.162017in}}%
\pgfpathcurveto{\pgfqpoint{1.009573in}{2.167841in}}{\pgfqpoint{1.012846in}{2.175741in}}{\pgfqpoint{1.012846in}{2.183977in}}%
\pgfpathcurveto{\pgfqpoint{1.012846in}{2.192213in}}{\pgfqpoint{1.009573in}{2.200113in}}{\pgfqpoint{1.003749in}{2.205937in}}%
\pgfpathcurveto{\pgfqpoint{0.997926in}{2.211761in}}{\pgfqpoint{0.990026in}{2.215033in}}{\pgfqpoint{0.981789in}{2.215033in}}%
\pgfpathcurveto{\pgfqpoint{0.973553in}{2.215033in}}{\pgfqpoint{0.965653in}{2.211761in}}{\pgfqpoint{0.959829in}{2.205937in}}%
\pgfpathcurveto{\pgfqpoint{0.954005in}{2.200113in}}{\pgfqpoint{0.950733in}{2.192213in}}{\pgfqpoint{0.950733in}{2.183977in}}%
\pgfpathcurveto{\pgfqpoint{0.950733in}{2.175741in}}{\pgfqpoint{0.954005in}{2.167841in}}{\pgfqpoint{0.959829in}{2.162017in}}%
\pgfpathcurveto{\pgfqpoint{0.965653in}{2.156193in}}{\pgfqpoint{0.973553in}{2.152920in}}{\pgfqpoint{0.981789in}{2.152920in}}%
\pgfpathclose%
\pgfusepath{stroke,fill}%
\end{pgfscope}%
\begin{pgfscope}%
\pgfpathrectangle{\pgfqpoint{0.100000in}{0.212622in}}{\pgfqpoint{3.696000in}{3.696000in}}%
\pgfusepath{clip}%
\pgfsetbuttcap%
\pgfsetroundjoin%
\definecolor{currentfill}{rgb}{0.121569,0.466667,0.705882}%
\pgfsetfillcolor{currentfill}%
\pgfsetfillopacity{0.666379}%
\pgfsetlinewidth{1.003750pt}%
\definecolor{currentstroke}{rgb}{0.121569,0.466667,0.705882}%
\pgfsetstrokecolor{currentstroke}%
\pgfsetstrokeopacity{0.666379}%
\pgfsetdash{}{0pt}%
\pgfpathmoveto{\pgfqpoint{3.046154in}{1.798463in}}%
\pgfpathcurveto{\pgfqpoint{3.054390in}{1.798463in}}{\pgfqpoint{3.062291in}{1.801735in}}{\pgfqpoint{3.068114in}{1.807559in}}%
\pgfpathcurveto{\pgfqpoint{3.073938in}{1.813383in}}{\pgfqpoint{3.077211in}{1.821283in}}{\pgfqpoint{3.077211in}{1.829519in}}%
\pgfpathcurveto{\pgfqpoint{3.077211in}{1.837756in}}{\pgfqpoint{3.073938in}{1.845656in}}{\pgfqpoint{3.068114in}{1.851480in}}%
\pgfpathcurveto{\pgfqpoint{3.062291in}{1.857303in}}{\pgfqpoint{3.054390in}{1.860576in}}{\pgfqpoint{3.046154in}{1.860576in}}%
\pgfpathcurveto{\pgfqpoint{3.037918in}{1.860576in}}{\pgfqpoint{3.030018in}{1.857303in}}{\pgfqpoint{3.024194in}{1.851480in}}%
\pgfpathcurveto{\pgfqpoint{3.018370in}{1.845656in}}{\pgfqpoint{3.015098in}{1.837756in}}{\pgfqpoint{3.015098in}{1.829519in}}%
\pgfpathcurveto{\pgfqpoint{3.015098in}{1.821283in}}{\pgfqpoint{3.018370in}{1.813383in}}{\pgfqpoint{3.024194in}{1.807559in}}%
\pgfpathcurveto{\pgfqpoint{3.030018in}{1.801735in}}{\pgfqpoint{3.037918in}{1.798463in}}{\pgfqpoint{3.046154in}{1.798463in}}%
\pgfpathclose%
\pgfusepath{stroke,fill}%
\end{pgfscope}%
\begin{pgfscope}%
\pgfpathrectangle{\pgfqpoint{0.100000in}{0.212622in}}{\pgfqpoint{3.696000in}{3.696000in}}%
\pgfusepath{clip}%
\pgfsetbuttcap%
\pgfsetroundjoin%
\definecolor{currentfill}{rgb}{0.121569,0.466667,0.705882}%
\pgfsetfillcolor{currentfill}%
\pgfsetfillopacity{0.667545}%
\pgfsetlinewidth{1.003750pt}%
\definecolor{currentstroke}{rgb}{0.121569,0.466667,0.705882}%
\pgfsetstrokecolor{currentstroke}%
\pgfsetstrokeopacity{0.667545}%
\pgfsetdash{}{0pt}%
\pgfpathmoveto{\pgfqpoint{0.977823in}{2.153337in}}%
\pgfpathcurveto{\pgfqpoint{0.986059in}{2.153337in}}{\pgfqpoint{0.993959in}{2.156609in}}{\pgfqpoint{0.999783in}{2.162433in}}%
\pgfpathcurveto{\pgfqpoint{1.005607in}{2.168257in}}{\pgfqpoint{1.008879in}{2.176157in}}{\pgfqpoint{1.008879in}{2.184393in}}%
\pgfpathcurveto{\pgfqpoint{1.008879in}{2.192630in}}{\pgfqpoint{1.005607in}{2.200530in}}{\pgfqpoint{0.999783in}{2.206354in}}%
\pgfpathcurveto{\pgfqpoint{0.993959in}{2.212178in}}{\pgfqpoint{0.986059in}{2.215450in}}{\pgfqpoint{0.977823in}{2.215450in}}%
\pgfpathcurveto{\pgfqpoint{0.969586in}{2.215450in}}{\pgfqpoint{0.961686in}{2.212178in}}{\pgfqpoint{0.955862in}{2.206354in}}%
\pgfpathcurveto{\pgfqpoint{0.950039in}{2.200530in}}{\pgfqpoint{0.946766in}{2.192630in}}{\pgfqpoint{0.946766in}{2.184393in}}%
\pgfpathcurveto{\pgfqpoint{0.946766in}{2.176157in}}{\pgfqpoint{0.950039in}{2.168257in}}{\pgfqpoint{0.955862in}{2.162433in}}%
\pgfpathcurveto{\pgfqpoint{0.961686in}{2.156609in}}{\pgfqpoint{0.969586in}{2.153337in}}{\pgfqpoint{0.977823in}{2.153337in}}%
\pgfpathclose%
\pgfusepath{stroke,fill}%
\end{pgfscope}%
\begin{pgfscope}%
\pgfpathrectangle{\pgfqpoint{0.100000in}{0.212622in}}{\pgfqpoint{3.696000in}{3.696000in}}%
\pgfusepath{clip}%
\pgfsetbuttcap%
\pgfsetroundjoin%
\definecolor{currentfill}{rgb}{0.121569,0.466667,0.705882}%
\pgfsetfillcolor{currentfill}%
\pgfsetfillopacity{0.669036}%
\pgfsetlinewidth{1.003750pt}%
\definecolor{currentstroke}{rgb}{0.121569,0.466667,0.705882}%
\pgfsetstrokecolor{currentstroke}%
\pgfsetstrokeopacity{0.669036}%
\pgfsetdash{}{0pt}%
\pgfpathmoveto{\pgfqpoint{0.975432in}{2.153602in}}%
\pgfpathcurveto{\pgfqpoint{0.983668in}{2.153602in}}{\pgfqpoint{0.991568in}{2.156874in}}{\pgfqpoint{0.997392in}{2.162698in}}%
\pgfpathcurveto{\pgfqpoint{1.003216in}{2.168522in}}{\pgfqpoint{1.006489in}{2.176422in}}{\pgfqpoint{1.006489in}{2.184659in}}%
\pgfpathcurveto{\pgfqpoint{1.006489in}{2.192895in}}{\pgfqpoint{1.003216in}{2.200795in}}{\pgfqpoint{0.997392in}{2.206619in}}%
\pgfpathcurveto{\pgfqpoint{0.991568in}{2.212443in}}{\pgfqpoint{0.983668in}{2.215715in}}{\pgfqpoint{0.975432in}{2.215715in}}%
\pgfpathcurveto{\pgfqpoint{0.967196in}{2.215715in}}{\pgfqpoint{0.959296in}{2.212443in}}{\pgfqpoint{0.953472in}{2.206619in}}%
\pgfpathcurveto{\pgfqpoint{0.947648in}{2.200795in}}{\pgfqpoint{0.944376in}{2.192895in}}{\pgfqpoint{0.944376in}{2.184659in}}%
\pgfpathcurveto{\pgfqpoint{0.944376in}{2.176422in}}{\pgfqpoint{0.947648in}{2.168522in}}{\pgfqpoint{0.953472in}{2.162698in}}%
\pgfpathcurveto{\pgfqpoint{0.959296in}{2.156874in}}{\pgfqpoint{0.967196in}{2.153602in}}{\pgfqpoint{0.975432in}{2.153602in}}%
\pgfpathclose%
\pgfusepath{stroke,fill}%
\end{pgfscope}%
\begin{pgfscope}%
\pgfpathrectangle{\pgfqpoint{0.100000in}{0.212622in}}{\pgfqpoint{3.696000in}{3.696000in}}%
\pgfusepath{clip}%
\pgfsetbuttcap%
\pgfsetroundjoin%
\definecolor{currentfill}{rgb}{0.121569,0.466667,0.705882}%
\pgfsetfillcolor{currentfill}%
\pgfsetfillopacity{0.669417}%
\pgfsetlinewidth{1.003750pt}%
\definecolor{currentstroke}{rgb}{0.121569,0.466667,0.705882}%
\pgfsetstrokecolor{currentstroke}%
\pgfsetstrokeopacity{0.669417}%
\pgfsetdash{}{0pt}%
\pgfpathmoveto{\pgfqpoint{3.041474in}{1.798688in}}%
\pgfpathcurveto{\pgfqpoint{3.049710in}{1.798688in}}{\pgfqpoint{3.057610in}{1.801960in}}{\pgfqpoint{3.063434in}{1.807784in}}%
\pgfpathcurveto{\pgfqpoint{3.069258in}{1.813608in}}{\pgfqpoint{3.072530in}{1.821508in}}{\pgfqpoint{3.072530in}{1.829744in}}%
\pgfpathcurveto{\pgfqpoint{3.072530in}{1.837981in}}{\pgfqpoint{3.069258in}{1.845881in}}{\pgfqpoint{3.063434in}{1.851704in}}%
\pgfpathcurveto{\pgfqpoint{3.057610in}{1.857528in}}{\pgfqpoint{3.049710in}{1.860801in}}{\pgfqpoint{3.041474in}{1.860801in}}%
\pgfpathcurveto{\pgfqpoint{3.033237in}{1.860801in}}{\pgfqpoint{3.025337in}{1.857528in}}{\pgfqpoint{3.019513in}{1.851704in}}%
\pgfpathcurveto{\pgfqpoint{3.013689in}{1.845881in}}{\pgfqpoint{3.010417in}{1.837981in}}{\pgfqpoint{3.010417in}{1.829744in}}%
\pgfpathcurveto{\pgfqpoint{3.010417in}{1.821508in}}{\pgfqpoint{3.013689in}{1.813608in}}{\pgfqpoint{3.019513in}{1.807784in}}%
\pgfpathcurveto{\pgfqpoint{3.025337in}{1.801960in}}{\pgfqpoint{3.033237in}{1.798688in}}{\pgfqpoint{3.041474in}{1.798688in}}%
\pgfpathclose%
\pgfusepath{stroke,fill}%
\end{pgfscope}%
\begin{pgfscope}%
\pgfpathrectangle{\pgfqpoint{0.100000in}{0.212622in}}{\pgfqpoint{3.696000in}{3.696000in}}%
\pgfusepath{clip}%
\pgfsetbuttcap%
\pgfsetroundjoin%
\definecolor{currentfill}{rgb}{0.121569,0.466667,0.705882}%
\pgfsetfillcolor{currentfill}%
\pgfsetfillopacity{0.669932}%
\pgfsetlinewidth{1.003750pt}%
\definecolor{currentstroke}{rgb}{0.121569,0.466667,0.705882}%
\pgfsetstrokecolor{currentstroke}%
\pgfsetstrokeopacity{0.669932}%
\pgfsetdash{}{0pt}%
\pgfpathmoveto{\pgfqpoint{0.972725in}{2.153933in}}%
\pgfpathcurveto{\pgfqpoint{0.980961in}{2.153933in}}{\pgfqpoint{0.988861in}{2.157205in}}{\pgfqpoint{0.994685in}{2.163029in}}%
\pgfpathcurveto{\pgfqpoint{1.000509in}{2.168853in}}{\pgfqpoint{1.003782in}{2.176753in}}{\pgfqpoint{1.003782in}{2.184990in}}%
\pgfpathcurveto{\pgfqpoint{1.003782in}{2.193226in}}{\pgfqpoint{1.000509in}{2.201126in}}{\pgfqpoint{0.994685in}{2.206950in}}%
\pgfpathcurveto{\pgfqpoint{0.988861in}{2.212774in}}{\pgfqpoint{0.980961in}{2.216046in}}{\pgfqpoint{0.972725in}{2.216046in}}%
\pgfpathcurveto{\pgfqpoint{0.964489in}{2.216046in}}{\pgfqpoint{0.956589in}{2.212774in}}{\pgfqpoint{0.950765in}{2.206950in}}%
\pgfpathcurveto{\pgfqpoint{0.944941in}{2.201126in}}{\pgfqpoint{0.941669in}{2.193226in}}{\pgfqpoint{0.941669in}{2.184990in}}%
\pgfpathcurveto{\pgfqpoint{0.941669in}{2.176753in}}{\pgfqpoint{0.944941in}{2.168853in}}{\pgfqpoint{0.950765in}{2.163029in}}%
\pgfpathcurveto{\pgfqpoint{0.956589in}{2.157205in}}{\pgfqpoint{0.964489in}{2.153933in}}{\pgfqpoint{0.972725in}{2.153933in}}%
\pgfpathclose%
\pgfusepath{stroke,fill}%
\end{pgfscope}%
\begin{pgfscope}%
\pgfpathrectangle{\pgfqpoint{0.100000in}{0.212622in}}{\pgfqpoint{3.696000in}{3.696000in}}%
\pgfusepath{clip}%
\pgfsetbuttcap%
\pgfsetroundjoin%
\definecolor{currentfill}{rgb}{0.121569,0.466667,0.705882}%
\pgfsetfillcolor{currentfill}%
\pgfsetfillopacity{0.671015}%
\pgfsetlinewidth{1.003750pt}%
\definecolor{currentstroke}{rgb}{0.121569,0.466667,0.705882}%
\pgfsetstrokecolor{currentstroke}%
\pgfsetstrokeopacity{0.671015}%
\pgfsetdash{}{0pt}%
\pgfpathmoveto{\pgfqpoint{3.038121in}{1.799131in}}%
\pgfpathcurveto{\pgfqpoint{3.046358in}{1.799131in}}{\pgfqpoint{3.054258in}{1.802404in}}{\pgfqpoint{3.060082in}{1.808227in}}%
\pgfpathcurveto{\pgfqpoint{3.065906in}{1.814051in}}{\pgfqpoint{3.069178in}{1.821951in}}{\pgfqpoint{3.069178in}{1.830188in}}%
\pgfpathcurveto{\pgfqpoint{3.069178in}{1.838424in}}{\pgfqpoint{3.065906in}{1.846324in}}{\pgfqpoint{3.060082in}{1.852148in}}%
\pgfpathcurveto{\pgfqpoint{3.054258in}{1.857972in}}{\pgfqpoint{3.046358in}{1.861244in}}{\pgfqpoint{3.038121in}{1.861244in}}%
\pgfpathcurveto{\pgfqpoint{3.029885in}{1.861244in}}{\pgfqpoint{3.021985in}{1.857972in}}{\pgfqpoint{3.016161in}{1.852148in}}%
\pgfpathcurveto{\pgfqpoint{3.010337in}{1.846324in}}{\pgfqpoint{3.007065in}{1.838424in}}{\pgfqpoint{3.007065in}{1.830188in}}%
\pgfpathcurveto{\pgfqpoint{3.007065in}{1.821951in}}{\pgfqpoint{3.010337in}{1.814051in}}{\pgfqpoint{3.016161in}{1.808227in}}%
\pgfpathcurveto{\pgfqpoint{3.021985in}{1.802404in}}{\pgfqpoint{3.029885in}{1.799131in}}{\pgfqpoint{3.038121in}{1.799131in}}%
\pgfpathclose%
\pgfusepath{stroke,fill}%
\end{pgfscope}%
\begin{pgfscope}%
\pgfpathrectangle{\pgfqpoint{0.100000in}{0.212622in}}{\pgfqpoint{3.696000in}{3.696000in}}%
\pgfusepath{clip}%
\pgfsetbuttcap%
\pgfsetroundjoin%
\definecolor{currentfill}{rgb}{0.121569,0.466667,0.705882}%
\pgfsetfillcolor{currentfill}%
\pgfsetfillopacity{0.671999}%
\pgfsetlinewidth{1.003750pt}%
\definecolor{currentstroke}{rgb}{0.121569,0.466667,0.705882}%
\pgfsetstrokecolor{currentstroke}%
\pgfsetstrokeopacity{0.671999}%
\pgfsetdash{}{0pt}%
\pgfpathmoveto{\pgfqpoint{0.974127in}{2.155119in}}%
\pgfpathcurveto{\pgfqpoint{0.982364in}{2.155119in}}{\pgfqpoint{0.990264in}{2.158391in}}{\pgfqpoint{0.996088in}{2.164215in}}%
\pgfpathcurveto{\pgfqpoint{1.001912in}{2.170039in}}{\pgfqpoint{1.005184in}{2.177939in}}{\pgfqpoint{1.005184in}{2.186176in}}%
\pgfpathcurveto{\pgfqpoint{1.005184in}{2.194412in}}{\pgfqpoint{1.001912in}{2.202312in}}{\pgfqpoint{0.996088in}{2.208136in}}%
\pgfpathcurveto{\pgfqpoint{0.990264in}{2.213960in}}{\pgfqpoint{0.982364in}{2.217232in}}{\pgfqpoint{0.974127in}{2.217232in}}%
\pgfpathcurveto{\pgfqpoint{0.965891in}{2.217232in}}{\pgfqpoint{0.957991in}{2.213960in}}{\pgfqpoint{0.952167in}{2.208136in}}%
\pgfpathcurveto{\pgfqpoint{0.946343in}{2.202312in}}{\pgfqpoint{0.943071in}{2.194412in}}{\pgfqpoint{0.943071in}{2.186176in}}%
\pgfpathcurveto{\pgfqpoint{0.943071in}{2.177939in}}{\pgfqpoint{0.946343in}{2.170039in}}{\pgfqpoint{0.952167in}{2.164215in}}%
\pgfpathcurveto{\pgfqpoint{0.957991in}{2.158391in}}{\pgfqpoint{0.965891in}{2.155119in}}{\pgfqpoint{0.974127in}{2.155119in}}%
\pgfpathclose%
\pgfusepath{stroke,fill}%
\end{pgfscope}%
\begin{pgfscope}%
\pgfpathrectangle{\pgfqpoint{0.100000in}{0.212622in}}{\pgfqpoint{3.696000in}{3.696000in}}%
\pgfusepath{clip}%
\pgfsetbuttcap%
\pgfsetroundjoin%
\definecolor{currentfill}{rgb}{0.121569,0.466667,0.705882}%
\pgfsetfillcolor{currentfill}%
\pgfsetfillopacity{0.673314}%
\pgfsetlinewidth{1.003750pt}%
\definecolor{currentstroke}{rgb}{0.121569,0.466667,0.705882}%
\pgfsetstrokecolor{currentstroke}%
\pgfsetstrokeopacity{0.673314}%
\pgfsetdash{}{0pt}%
\pgfpathmoveto{\pgfqpoint{3.034342in}{1.799469in}}%
\pgfpathcurveto{\pgfqpoint{3.042578in}{1.799469in}}{\pgfqpoint{3.050478in}{1.802742in}}{\pgfqpoint{3.056302in}{1.808566in}}%
\pgfpathcurveto{\pgfqpoint{3.062126in}{1.814389in}}{\pgfqpoint{3.065398in}{1.822290in}}{\pgfqpoint{3.065398in}{1.830526in}}%
\pgfpathcurveto{\pgfqpoint{3.065398in}{1.838762in}}{\pgfqpoint{3.062126in}{1.846662in}}{\pgfqpoint{3.056302in}{1.852486in}}%
\pgfpathcurveto{\pgfqpoint{3.050478in}{1.858310in}}{\pgfqpoint{3.042578in}{1.861582in}}{\pgfqpoint{3.034342in}{1.861582in}}%
\pgfpathcurveto{\pgfqpoint{3.026106in}{1.861582in}}{\pgfqpoint{3.018206in}{1.858310in}}{\pgfqpoint{3.012382in}{1.852486in}}%
\pgfpathcurveto{\pgfqpoint{3.006558in}{1.846662in}}{\pgfqpoint{3.003285in}{1.838762in}}{\pgfqpoint{3.003285in}{1.830526in}}%
\pgfpathcurveto{\pgfqpoint{3.003285in}{1.822290in}}{\pgfqpoint{3.006558in}{1.814389in}}{\pgfqpoint{3.012382in}{1.808566in}}%
\pgfpathcurveto{\pgfqpoint{3.018206in}{1.802742in}}{\pgfqpoint{3.026106in}{1.799469in}}{\pgfqpoint{3.034342in}{1.799469in}}%
\pgfpathclose%
\pgfusepath{stroke,fill}%
\end{pgfscope}%
\begin{pgfscope}%
\pgfpathrectangle{\pgfqpoint{0.100000in}{0.212622in}}{\pgfqpoint{3.696000in}{3.696000in}}%
\pgfusepath{clip}%
\pgfsetbuttcap%
\pgfsetroundjoin%
\definecolor{currentfill}{rgb}{0.121569,0.466667,0.705882}%
\pgfsetfillcolor{currentfill}%
\pgfsetfillopacity{0.673546}%
\pgfsetlinewidth{1.003750pt}%
\definecolor{currentstroke}{rgb}{0.121569,0.466667,0.705882}%
\pgfsetstrokecolor{currentstroke}%
\pgfsetstrokeopacity{0.673546}%
\pgfsetdash{}{0pt}%
\pgfpathmoveto{\pgfqpoint{0.970674in}{2.155378in}}%
\pgfpathcurveto{\pgfqpoint{0.978910in}{2.155378in}}{\pgfqpoint{0.986810in}{2.158650in}}{\pgfqpoint{0.992634in}{2.164474in}}%
\pgfpathcurveto{\pgfqpoint{0.998458in}{2.170298in}}{\pgfqpoint{1.001730in}{2.178198in}}{\pgfqpoint{1.001730in}{2.186434in}}%
\pgfpathcurveto{\pgfqpoint{1.001730in}{2.194671in}}{\pgfqpoint{0.998458in}{2.202571in}}{\pgfqpoint{0.992634in}{2.208395in}}%
\pgfpathcurveto{\pgfqpoint{0.986810in}{2.214218in}}{\pgfqpoint{0.978910in}{2.217491in}}{\pgfqpoint{0.970674in}{2.217491in}}%
\pgfpathcurveto{\pgfqpoint{0.962438in}{2.217491in}}{\pgfqpoint{0.954538in}{2.214218in}}{\pgfqpoint{0.948714in}{2.208395in}}%
\pgfpathcurveto{\pgfqpoint{0.942890in}{2.202571in}}{\pgfqpoint{0.939617in}{2.194671in}}{\pgfqpoint{0.939617in}{2.186434in}}%
\pgfpathcurveto{\pgfqpoint{0.939617in}{2.178198in}}{\pgfqpoint{0.942890in}{2.170298in}}{\pgfqpoint{0.948714in}{2.164474in}}%
\pgfpathcurveto{\pgfqpoint{0.954538in}{2.158650in}}{\pgfqpoint{0.962438in}{2.155378in}}{\pgfqpoint{0.970674in}{2.155378in}}%
\pgfpathclose%
\pgfusepath{stroke,fill}%
\end{pgfscope}%
\begin{pgfscope}%
\pgfpathrectangle{\pgfqpoint{0.100000in}{0.212622in}}{\pgfqpoint{3.696000in}{3.696000in}}%
\pgfusepath{clip}%
\pgfsetbuttcap%
\pgfsetroundjoin%
\definecolor{currentfill}{rgb}{0.121569,0.466667,0.705882}%
\pgfsetfillcolor{currentfill}%
\pgfsetfillopacity{0.674938}%
\pgfsetlinewidth{1.003750pt}%
\definecolor{currentstroke}{rgb}{0.121569,0.466667,0.705882}%
\pgfsetstrokecolor{currentstroke}%
\pgfsetstrokeopacity{0.674938}%
\pgfsetdash{}{0pt}%
\pgfpathmoveto{\pgfqpoint{0.967065in}{2.155742in}}%
\pgfpathcurveto{\pgfqpoint{0.975301in}{2.155742in}}{\pgfqpoint{0.983201in}{2.159015in}}{\pgfqpoint{0.989025in}{2.164839in}}%
\pgfpathcurveto{\pgfqpoint{0.994849in}{2.170663in}}{\pgfqpoint{0.998121in}{2.178563in}}{\pgfqpoint{0.998121in}{2.186799in}}%
\pgfpathcurveto{\pgfqpoint{0.998121in}{2.195035in}}{\pgfqpoint{0.994849in}{2.202935in}}{\pgfqpoint{0.989025in}{2.208759in}}%
\pgfpathcurveto{\pgfqpoint{0.983201in}{2.214583in}}{\pgfqpoint{0.975301in}{2.217855in}}{\pgfqpoint{0.967065in}{2.217855in}}%
\pgfpathcurveto{\pgfqpoint{0.958829in}{2.217855in}}{\pgfqpoint{0.950929in}{2.214583in}}{\pgfqpoint{0.945105in}{2.208759in}}%
\pgfpathcurveto{\pgfqpoint{0.939281in}{2.202935in}}{\pgfqpoint{0.936008in}{2.195035in}}{\pgfqpoint{0.936008in}{2.186799in}}%
\pgfpathcurveto{\pgfqpoint{0.936008in}{2.178563in}}{\pgfqpoint{0.939281in}{2.170663in}}{\pgfqpoint{0.945105in}{2.164839in}}%
\pgfpathcurveto{\pgfqpoint{0.950929in}{2.159015in}}{\pgfqpoint{0.958829in}{2.155742in}}{\pgfqpoint{0.967065in}{2.155742in}}%
\pgfpathclose%
\pgfusepath{stroke,fill}%
\end{pgfscope}%
\begin{pgfscope}%
\pgfpathrectangle{\pgfqpoint{0.100000in}{0.212622in}}{\pgfqpoint{3.696000in}{3.696000in}}%
\pgfusepath{clip}%
\pgfsetbuttcap%
\pgfsetroundjoin%
\definecolor{currentfill}{rgb}{0.121569,0.466667,0.705882}%
\pgfsetfillcolor{currentfill}%
\pgfsetfillopacity{0.675909}%
\pgfsetlinewidth{1.003750pt}%
\definecolor{currentstroke}{rgb}{0.121569,0.466667,0.705882}%
\pgfsetstrokecolor{currentstroke}%
\pgfsetstrokeopacity{0.675909}%
\pgfsetdash{}{0pt}%
\pgfpathmoveto{\pgfqpoint{3.031916in}{1.799590in}}%
\pgfpathcurveto{\pgfqpoint{3.040152in}{1.799590in}}{\pgfqpoint{3.048052in}{1.802862in}}{\pgfqpoint{3.053876in}{1.808686in}}%
\pgfpathcurveto{\pgfqpoint{3.059700in}{1.814510in}}{\pgfqpoint{3.062972in}{1.822410in}}{\pgfqpoint{3.062972in}{1.830646in}}%
\pgfpathcurveto{\pgfqpoint{3.062972in}{1.838882in}}{\pgfqpoint{3.059700in}{1.846782in}}{\pgfqpoint{3.053876in}{1.852606in}}%
\pgfpathcurveto{\pgfqpoint{3.048052in}{1.858430in}}{\pgfqpoint{3.040152in}{1.861703in}}{\pgfqpoint{3.031916in}{1.861703in}}%
\pgfpathcurveto{\pgfqpoint{3.023680in}{1.861703in}}{\pgfqpoint{3.015780in}{1.858430in}}{\pgfqpoint{3.009956in}{1.852606in}}%
\pgfpathcurveto{\pgfqpoint{3.004132in}{1.846782in}}{\pgfqpoint{3.000859in}{1.838882in}}{\pgfqpoint{3.000859in}{1.830646in}}%
\pgfpathcurveto{\pgfqpoint{3.000859in}{1.822410in}}{\pgfqpoint{3.004132in}{1.814510in}}{\pgfqpoint{3.009956in}{1.808686in}}%
\pgfpathcurveto{\pgfqpoint{3.015780in}{1.802862in}}{\pgfqpoint{3.023680in}{1.799590in}}{\pgfqpoint{3.031916in}{1.799590in}}%
\pgfpathclose%
\pgfusepath{stroke,fill}%
\end{pgfscope}%
\begin{pgfscope}%
\pgfpathrectangle{\pgfqpoint{0.100000in}{0.212622in}}{\pgfqpoint{3.696000in}{3.696000in}}%
\pgfusepath{clip}%
\pgfsetbuttcap%
\pgfsetroundjoin%
\definecolor{currentfill}{rgb}{0.121569,0.466667,0.705882}%
\pgfsetfillcolor{currentfill}%
\pgfsetfillopacity{0.676272}%
\pgfsetlinewidth{1.003750pt}%
\definecolor{currentstroke}{rgb}{0.121569,0.466667,0.705882}%
\pgfsetstrokecolor{currentstroke}%
\pgfsetstrokeopacity{0.676272}%
\pgfsetdash{}{0pt}%
\pgfpathmoveto{\pgfqpoint{0.965825in}{2.156007in}}%
\pgfpathcurveto{\pgfqpoint{0.974061in}{2.156007in}}{\pgfqpoint{0.981961in}{2.159280in}}{\pgfqpoint{0.987785in}{2.165104in}}%
\pgfpathcurveto{\pgfqpoint{0.993609in}{2.170928in}}{\pgfqpoint{0.996881in}{2.178828in}}{\pgfqpoint{0.996881in}{2.187064in}}%
\pgfpathcurveto{\pgfqpoint{0.996881in}{2.195300in}}{\pgfqpoint{0.993609in}{2.203200in}}{\pgfqpoint{0.987785in}{2.209024in}}%
\pgfpathcurveto{\pgfqpoint{0.981961in}{2.214848in}}{\pgfqpoint{0.974061in}{2.218120in}}{\pgfqpoint{0.965825in}{2.218120in}}%
\pgfpathcurveto{\pgfqpoint{0.957588in}{2.218120in}}{\pgfqpoint{0.949688in}{2.214848in}}{\pgfqpoint{0.943864in}{2.209024in}}%
\pgfpathcurveto{\pgfqpoint{0.938040in}{2.203200in}}{\pgfqpoint{0.934768in}{2.195300in}}{\pgfqpoint{0.934768in}{2.187064in}}%
\pgfpathcurveto{\pgfqpoint{0.934768in}{2.178828in}}{\pgfqpoint{0.938040in}{2.170928in}}{\pgfqpoint{0.943864in}{2.165104in}}%
\pgfpathcurveto{\pgfqpoint{0.949688in}{2.159280in}}{\pgfqpoint{0.957588in}{2.156007in}}{\pgfqpoint{0.965825in}{2.156007in}}%
\pgfpathclose%
\pgfusepath{stroke,fill}%
\end{pgfscope}%
\begin{pgfscope}%
\pgfpathrectangle{\pgfqpoint{0.100000in}{0.212622in}}{\pgfqpoint{3.696000in}{3.696000in}}%
\pgfusepath{clip}%
\pgfsetbuttcap%
\pgfsetroundjoin%
\definecolor{currentfill}{rgb}{0.121569,0.466667,0.705882}%
\pgfsetfillcolor{currentfill}%
\pgfsetfillopacity{0.677146}%
\pgfsetlinewidth{1.003750pt}%
\definecolor{currentstroke}{rgb}{0.121569,0.466667,0.705882}%
\pgfsetstrokecolor{currentstroke}%
\pgfsetstrokeopacity{0.677146}%
\pgfsetdash{}{0pt}%
\pgfpathmoveto{\pgfqpoint{0.963149in}{2.156385in}}%
\pgfpathcurveto{\pgfqpoint{0.971385in}{2.156385in}}{\pgfqpoint{0.979285in}{2.159658in}}{\pgfqpoint{0.985109in}{2.165482in}}%
\pgfpathcurveto{\pgfqpoint{0.990933in}{2.171306in}}{\pgfqpoint{0.994205in}{2.179206in}}{\pgfqpoint{0.994205in}{2.187442in}}%
\pgfpathcurveto{\pgfqpoint{0.994205in}{2.195678in}}{\pgfqpoint{0.990933in}{2.203578in}}{\pgfqpoint{0.985109in}{2.209402in}}%
\pgfpathcurveto{\pgfqpoint{0.979285in}{2.215226in}}{\pgfqpoint{0.971385in}{2.218498in}}{\pgfqpoint{0.963149in}{2.218498in}}%
\pgfpathcurveto{\pgfqpoint{0.954913in}{2.218498in}}{\pgfqpoint{0.947013in}{2.215226in}}{\pgfqpoint{0.941189in}{2.209402in}}%
\pgfpathcurveto{\pgfqpoint{0.935365in}{2.203578in}}{\pgfqpoint{0.932092in}{2.195678in}}{\pgfqpoint{0.932092in}{2.187442in}}%
\pgfpathcurveto{\pgfqpoint{0.932092in}{2.179206in}}{\pgfqpoint{0.935365in}{2.171306in}}{\pgfqpoint{0.941189in}{2.165482in}}%
\pgfpathcurveto{\pgfqpoint{0.947013in}{2.159658in}}{\pgfqpoint{0.954913in}{2.156385in}}{\pgfqpoint{0.963149in}{2.156385in}}%
\pgfpathclose%
\pgfusepath{stroke,fill}%
\end{pgfscope}%
\begin{pgfscope}%
\pgfpathrectangle{\pgfqpoint{0.100000in}{0.212622in}}{\pgfqpoint{3.696000in}{3.696000in}}%
\pgfusepath{clip}%
\pgfsetbuttcap%
\pgfsetroundjoin%
\definecolor{currentfill}{rgb}{0.121569,0.466667,0.705882}%
\pgfsetfillcolor{currentfill}%
\pgfsetfillopacity{0.677538}%
\pgfsetlinewidth{1.003750pt}%
\definecolor{currentstroke}{rgb}{0.121569,0.466667,0.705882}%
\pgfsetstrokecolor{currentstroke}%
\pgfsetstrokeopacity{0.677538}%
\pgfsetdash{}{0pt}%
\pgfpathmoveto{\pgfqpoint{0.963954in}{2.156735in}}%
\pgfpathcurveto{\pgfqpoint{0.972190in}{2.156735in}}{\pgfqpoint{0.980090in}{2.160008in}}{\pgfqpoint{0.985914in}{2.165831in}}%
\pgfpathcurveto{\pgfqpoint{0.991738in}{2.171655in}}{\pgfqpoint{0.995010in}{2.179555in}}{\pgfqpoint{0.995010in}{2.187792in}}%
\pgfpathcurveto{\pgfqpoint{0.995010in}{2.196028in}}{\pgfqpoint{0.991738in}{2.203928in}}{\pgfqpoint{0.985914in}{2.209752in}}%
\pgfpathcurveto{\pgfqpoint{0.980090in}{2.215576in}}{\pgfqpoint{0.972190in}{2.218848in}}{\pgfqpoint{0.963954in}{2.218848in}}%
\pgfpathcurveto{\pgfqpoint{0.955717in}{2.218848in}}{\pgfqpoint{0.947817in}{2.215576in}}{\pgfqpoint{0.941993in}{2.209752in}}%
\pgfpathcurveto{\pgfqpoint{0.936170in}{2.203928in}}{\pgfqpoint{0.932897in}{2.196028in}}{\pgfqpoint{0.932897in}{2.187792in}}%
\pgfpathcurveto{\pgfqpoint{0.932897in}{2.179555in}}{\pgfqpoint{0.936170in}{2.171655in}}{\pgfqpoint{0.941993in}{2.165831in}}%
\pgfpathcurveto{\pgfqpoint{0.947817in}{2.160008in}}{\pgfqpoint{0.955717in}{2.156735in}}{\pgfqpoint{0.963954in}{2.156735in}}%
\pgfpathclose%
\pgfusepath{stroke,fill}%
\end{pgfscope}%
\begin{pgfscope}%
\pgfpathrectangle{\pgfqpoint{0.100000in}{0.212622in}}{\pgfqpoint{3.696000in}{3.696000in}}%
\pgfusepath{clip}%
\pgfsetbuttcap%
\pgfsetroundjoin%
\definecolor{currentfill}{rgb}{0.121569,0.466667,0.705882}%
\pgfsetfillcolor{currentfill}%
\pgfsetfillopacity{0.678215}%
\pgfsetlinewidth{1.003750pt}%
\definecolor{currentstroke}{rgb}{0.121569,0.466667,0.705882}%
\pgfsetstrokecolor{currentstroke}%
\pgfsetstrokeopacity{0.678215}%
\pgfsetdash{}{0pt}%
\pgfpathmoveto{\pgfqpoint{0.962338in}{2.156846in}}%
\pgfpathcurveto{\pgfqpoint{0.970575in}{2.156846in}}{\pgfqpoint{0.978475in}{2.160118in}}{\pgfqpoint{0.984299in}{2.165942in}}%
\pgfpathcurveto{\pgfqpoint{0.990123in}{2.171766in}}{\pgfqpoint{0.993395in}{2.179666in}}{\pgfqpoint{0.993395in}{2.187903in}}%
\pgfpathcurveto{\pgfqpoint{0.993395in}{2.196139in}}{\pgfqpoint{0.990123in}{2.204039in}}{\pgfqpoint{0.984299in}{2.209863in}}%
\pgfpathcurveto{\pgfqpoint{0.978475in}{2.215687in}}{\pgfqpoint{0.970575in}{2.218959in}}{\pgfqpoint{0.962338in}{2.218959in}}%
\pgfpathcurveto{\pgfqpoint{0.954102in}{2.218959in}}{\pgfqpoint{0.946202in}{2.215687in}}{\pgfqpoint{0.940378in}{2.209863in}}%
\pgfpathcurveto{\pgfqpoint{0.934554in}{2.204039in}}{\pgfqpoint{0.931282in}{2.196139in}}{\pgfqpoint{0.931282in}{2.187903in}}%
\pgfpathcurveto{\pgfqpoint{0.931282in}{2.179666in}}{\pgfqpoint{0.934554in}{2.171766in}}{\pgfqpoint{0.940378in}{2.165942in}}%
\pgfpathcurveto{\pgfqpoint{0.946202in}{2.160118in}}{\pgfqpoint{0.954102in}{2.156846in}}{\pgfqpoint{0.962338in}{2.156846in}}%
\pgfpathclose%
\pgfusepath{stroke,fill}%
\end{pgfscope}%
\begin{pgfscope}%
\pgfpathrectangle{\pgfqpoint{0.100000in}{0.212622in}}{\pgfqpoint{3.696000in}{3.696000in}}%
\pgfusepath{clip}%
\pgfsetbuttcap%
\pgfsetroundjoin%
\definecolor{currentfill}{rgb}{0.121569,0.466667,0.705882}%
\pgfsetfillcolor{currentfill}%
\pgfsetfillopacity{0.678692}%
\pgfsetlinewidth{1.003750pt}%
\definecolor{currentstroke}{rgb}{0.121569,0.466667,0.705882}%
\pgfsetstrokecolor{currentstroke}%
\pgfsetstrokeopacity{0.678692}%
\pgfsetdash{}{0pt}%
\pgfpathmoveto{\pgfqpoint{3.026544in}{1.800220in}}%
\pgfpathcurveto{\pgfqpoint{3.034780in}{1.800220in}}{\pgfqpoint{3.042680in}{1.803492in}}{\pgfqpoint{3.048504in}{1.809316in}}%
\pgfpathcurveto{\pgfqpoint{3.054328in}{1.815140in}}{\pgfqpoint{3.057600in}{1.823040in}}{\pgfqpoint{3.057600in}{1.831277in}}%
\pgfpathcurveto{\pgfqpoint{3.057600in}{1.839513in}}{\pgfqpoint{3.054328in}{1.847413in}}{\pgfqpoint{3.048504in}{1.853237in}}%
\pgfpathcurveto{\pgfqpoint{3.042680in}{1.859061in}}{\pgfqpoint{3.034780in}{1.862333in}}{\pgfqpoint{3.026544in}{1.862333in}}%
\pgfpathcurveto{\pgfqpoint{3.018307in}{1.862333in}}{\pgfqpoint{3.010407in}{1.859061in}}{\pgfqpoint{3.004583in}{1.853237in}}%
\pgfpathcurveto{\pgfqpoint{2.998760in}{1.847413in}}{\pgfqpoint{2.995487in}{1.839513in}}{\pgfqpoint{2.995487in}{1.831277in}}%
\pgfpathcurveto{\pgfqpoint{2.995487in}{1.823040in}}{\pgfqpoint{2.998760in}{1.815140in}}{\pgfqpoint{3.004583in}{1.809316in}}%
\pgfpathcurveto{\pgfqpoint{3.010407in}{1.803492in}}{\pgfqpoint{3.018307in}{1.800220in}}{\pgfqpoint{3.026544in}{1.800220in}}%
\pgfpathclose%
\pgfusepath{stroke,fill}%
\end{pgfscope}%
\begin{pgfscope}%
\pgfpathrectangle{\pgfqpoint{0.100000in}{0.212622in}}{\pgfqpoint{3.696000in}{3.696000in}}%
\pgfusepath{clip}%
\pgfsetbuttcap%
\pgfsetroundjoin%
\definecolor{currentfill}{rgb}{0.121569,0.466667,0.705882}%
\pgfsetfillcolor{currentfill}%
\pgfsetfillopacity{0.679456}%
\pgfsetlinewidth{1.003750pt}%
\definecolor{currentstroke}{rgb}{0.121569,0.466667,0.705882}%
\pgfsetstrokecolor{currentstroke}%
\pgfsetstrokeopacity{0.679456}%
\pgfsetdash{}{0pt}%
\pgfpathmoveto{\pgfqpoint{0.959536in}{2.156956in}}%
\pgfpathcurveto{\pgfqpoint{0.967772in}{2.156956in}}{\pgfqpoint{0.975672in}{2.160228in}}{\pgfqpoint{0.981496in}{2.166052in}}%
\pgfpathcurveto{\pgfqpoint{0.987320in}{2.171876in}}{\pgfqpoint{0.990593in}{2.179776in}}{\pgfqpoint{0.990593in}{2.188012in}}%
\pgfpathcurveto{\pgfqpoint{0.990593in}{2.196248in}}{\pgfqpoint{0.987320in}{2.204148in}}{\pgfqpoint{0.981496in}{2.209972in}}%
\pgfpathcurveto{\pgfqpoint{0.975672in}{2.215796in}}{\pgfqpoint{0.967772in}{2.219069in}}{\pgfqpoint{0.959536in}{2.219069in}}%
\pgfpathcurveto{\pgfqpoint{0.951300in}{2.219069in}}{\pgfqpoint{0.943400in}{2.215796in}}{\pgfqpoint{0.937576in}{2.209972in}}%
\pgfpathcurveto{\pgfqpoint{0.931752in}{2.204148in}}{\pgfqpoint{0.928480in}{2.196248in}}{\pgfqpoint{0.928480in}{2.188012in}}%
\pgfpathcurveto{\pgfqpoint{0.928480in}{2.179776in}}{\pgfqpoint{0.931752in}{2.171876in}}{\pgfqpoint{0.937576in}{2.166052in}}%
\pgfpathcurveto{\pgfqpoint{0.943400in}{2.160228in}}{\pgfqpoint{0.951300in}{2.156956in}}{\pgfqpoint{0.959536in}{2.156956in}}%
\pgfpathclose%
\pgfusepath{stroke,fill}%
\end{pgfscope}%
\begin{pgfscope}%
\pgfpathrectangle{\pgfqpoint{0.100000in}{0.212622in}}{\pgfqpoint{3.696000in}{3.696000in}}%
\pgfusepath{clip}%
\pgfsetbuttcap%
\pgfsetroundjoin%
\definecolor{currentfill}{rgb}{0.121569,0.466667,0.705882}%
\pgfsetfillcolor{currentfill}%
\pgfsetfillopacity{0.680533}%
\pgfsetlinewidth{1.003750pt}%
\definecolor{currentstroke}{rgb}{0.121569,0.466667,0.705882}%
\pgfsetstrokecolor{currentstroke}%
\pgfsetstrokeopacity{0.680533}%
\pgfsetdash{}{0pt}%
\pgfpathmoveto{\pgfqpoint{0.957085in}{2.157081in}}%
\pgfpathcurveto{\pgfqpoint{0.965321in}{2.157081in}}{\pgfqpoint{0.973221in}{2.160353in}}{\pgfqpoint{0.979045in}{2.166177in}}%
\pgfpathcurveto{\pgfqpoint{0.984869in}{2.172001in}}{\pgfqpoint{0.988142in}{2.179901in}}{\pgfqpoint{0.988142in}{2.188137in}}%
\pgfpathcurveto{\pgfqpoint{0.988142in}{2.196374in}}{\pgfqpoint{0.984869in}{2.204274in}}{\pgfqpoint{0.979045in}{2.210098in}}%
\pgfpathcurveto{\pgfqpoint{0.973221in}{2.215922in}}{\pgfqpoint{0.965321in}{2.219194in}}{\pgfqpoint{0.957085in}{2.219194in}}%
\pgfpathcurveto{\pgfqpoint{0.948849in}{2.219194in}}{\pgfqpoint{0.940949in}{2.215922in}}{\pgfqpoint{0.935125in}{2.210098in}}%
\pgfpathcurveto{\pgfqpoint{0.929301in}{2.204274in}}{\pgfqpoint{0.926029in}{2.196374in}}{\pgfqpoint{0.926029in}{2.188137in}}%
\pgfpathcurveto{\pgfqpoint{0.926029in}{2.179901in}}{\pgfqpoint{0.929301in}{2.172001in}}{\pgfqpoint{0.935125in}{2.166177in}}%
\pgfpathcurveto{\pgfqpoint{0.940949in}{2.160353in}}{\pgfqpoint{0.948849in}{2.157081in}}{\pgfqpoint{0.957085in}{2.157081in}}%
\pgfpathclose%
\pgfusepath{stroke,fill}%
\end{pgfscope}%
\begin{pgfscope}%
\pgfpathrectangle{\pgfqpoint{0.100000in}{0.212622in}}{\pgfqpoint{3.696000in}{3.696000in}}%
\pgfusepath{clip}%
\pgfsetbuttcap%
\pgfsetroundjoin%
\definecolor{currentfill}{rgb}{0.121569,0.466667,0.705882}%
\pgfsetfillcolor{currentfill}%
\pgfsetfillopacity{0.681244}%
\pgfsetlinewidth{1.003750pt}%
\definecolor{currentstroke}{rgb}{0.121569,0.466667,0.705882}%
\pgfsetstrokecolor{currentstroke}%
\pgfsetstrokeopacity{0.681244}%
\pgfsetdash{}{0pt}%
\pgfpathmoveto{\pgfqpoint{0.955358in}{2.157203in}}%
\pgfpathcurveto{\pgfqpoint{0.963594in}{2.157203in}}{\pgfqpoint{0.971494in}{2.160475in}}{\pgfqpoint{0.977318in}{2.166299in}}%
\pgfpathcurveto{\pgfqpoint{0.983142in}{2.172123in}}{\pgfqpoint{0.986414in}{2.180023in}}{\pgfqpoint{0.986414in}{2.188259in}}%
\pgfpathcurveto{\pgfqpoint{0.986414in}{2.196496in}}{\pgfqpoint{0.983142in}{2.204396in}}{\pgfqpoint{0.977318in}{2.210220in}}%
\pgfpathcurveto{\pgfqpoint{0.971494in}{2.216044in}}{\pgfqpoint{0.963594in}{2.219316in}}{\pgfqpoint{0.955358in}{2.219316in}}%
\pgfpathcurveto{\pgfqpoint{0.947121in}{2.219316in}}{\pgfqpoint{0.939221in}{2.216044in}}{\pgfqpoint{0.933397in}{2.210220in}}%
\pgfpathcurveto{\pgfqpoint{0.927573in}{2.204396in}}{\pgfqpoint{0.924301in}{2.196496in}}{\pgfqpoint{0.924301in}{2.188259in}}%
\pgfpathcurveto{\pgfqpoint{0.924301in}{2.180023in}}{\pgfqpoint{0.927573in}{2.172123in}}{\pgfqpoint{0.933397in}{2.166299in}}%
\pgfpathcurveto{\pgfqpoint{0.939221in}{2.160475in}}{\pgfqpoint{0.947121in}{2.157203in}}{\pgfqpoint{0.955358in}{2.157203in}}%
\pgfpathclose%
\pgfusepath{stroke,fill}%
\end{pgfscope}%
\begin{pgfscope}%
\pgfpathrectangle{\pgfqpoint{0.100000in}{0.212622in}}{\pgfqpoint{3.696000in}{3.696000in}}%
\pgfusepath{clip}%
\pgfsetbuttcap%
\pgfsetroundjoin%
\definecolor{currentfill}{rgb}{0.121569,0.466667,0.705882}%
\pgfsetfillcolor{currentfill}%
\pgfsetfillopacity{0.681535}%
\pgfsetlinewidth{1.003750pt}%
\definecolor{currentstroke}{rgb}{0.121569,0.466667,0.705882}%
\pgfsetstrokecolor{currentstroke}%
\pgfsetstrokeopacity{0.681535}%
\pgfsetdash{}{0pt}%
\pgfpathmoveto{\pgfqpoint{3.019661in}{1.801855in}}%
\pgfpathcurveto{\pgfqpoint{3.027897in}{1.801855in}}{\pgfqpoint{3.035797in}{1.805127in}}{\pgfqpoint{3.041621in}{1.810951in}}%
\pgfpathcurveto{\pgfqpoint{3.047445in}{1.816775in}}{\pgfqpoint{3.050717in}{1.824675in}}{\pgfqpoint{3.050717in}{1.832912in}}%
\pgfpathcurveto{\pgfqpoint{3.050717in}{1.841148in}}{\pgfqpoint{3.047445in}{1.849048in}}{\pgfqpoint{3.041621in}{1.854872in}}%
\pgfpathcurveto{\pgfqpoint{3.035797in}{1.860696in}}{\pgfqpoint{3.027897in}{1.863968in}}{\pgfqpoint{3.019661in}{1.863968in}}%
\pgfpathcurveto{\pgfqpoint{3.011424in}{1.863968in}}{\pgfqpoint{3.003524in}{1.860696in}}{\pgfqpoint{2.997700in}{1.854872in}}%
\pgfpathcurveto{\pgfqpoint{2.991876in}{1.849048in}}{\pgfqpoint{2.988604in}{1.841148in}}{\pgfqpoint{2.988604in}{1.832912in}}%
\pgfpathcurveto{\pgfqpoint{2.988604in}{1.824675in}}{\pgfqpoint{2.991876in}{1.816775in}}{\pgfqpoint{2.997700in}{1.810951in}}%
\pgfpathcurveto{\pgfqpoint{3.003524in}{1.805127in}}{\pgfqpoint{3.011424in}{1.801855in}}{\pgfqpoint{3.019661in}{1.801855in}}%
\pgfpathclose%
\pgfusepath{stroke,fill}%
\end{pgfscope}%
\begin{pgfscope}%
\pgfpathrectangle{\pgfqpoint{0.100000in}{0.212622in}}{\pgfqpoint{3.696000in}{3.696000in}}%
\pgfusepath{clip}%
\pgfsetbuttcap%
\pgfsetroundjoin%
\definecolor{currentfill}{rgb}{0.121569,0.466667,0.705882}%
\pgfsetfillcolor{currentfill}%
\pgfsetfillopacity{0.681881}%
\pgfsetlinewidth{1.003750pt}%
\definecolor{currentstroke}{rgb}{0.121569,0.466667,0.705882}%
\pgfsetstrokecolor{currentstroke}%
\pgfsetstrokeopacity{0.681881}%
\pgfsetdash{}{0pt}%
\pgfpathmoveto{\pgfqpoint{0.954566in}{2.157304in}}%
\pgfpathcurveto{\pgfqpoint{0.962803in}{2.157304in}}{\pgfqpoint{0.970703in}{2.160576in}}{\pgfqpoint{0.976527in}{2.166400in}}%
\pgfpathcurveto{\pgfqpoint{0.982351in}{2.172224in}}{\pgfqpoint{0.985623in}{2.180124in}}{\pgfqpoint{0.985623in}{2.188361in}}%
\pgfpathcurveto{\pgfqpoint{0.985623in}{2.196597in}}{\pgfqpoint{0.982351in}{2.204497in}}{\pgfqpoint{0.976527in}{2.210321in}}%
\pgfpathcurveto{\pgfqpoint{0.970703in}{2.216145in}}{\pgfqpoint{0.962803in}{2.219417in}}{\pgfqpoint{0.954566in}{2.219417in}}%
\pgfpathcurveto{\pgfqpoint{0.946330in}{2.219417in}}{\pgfqpoint{0.938430in}{2.216145in}}{\pgfqpoint{0.932606in}{2.210321in}}%
\pgfpathcurveto{\pgfqpoint{0.926782in}{2.204497in}}{\pgfqpoint{0.923510in}{2.196597in}}{\pgfqpoint{0.923510in}{2.188361in}}%
\pgfpathcurveto{\pgfqpoint{0.923510in}{2.180124in}}{\pgfqpoint{0.926782in}{2.172224in}}{\pgfqpoint{0.932606in}{2.166400in}}%
\pgfpathcurveto{\pgfqpoint{0.938430in}{2.160576in}}{\pgfqpoint{0.946330in}{2.157304in}}{\pgfqpoint{0.954566in}{2.157304in}}%
\pgfpathclose%
\pgfusepath{stroke,fill}%
\end{pgfscope}%
\begin{pgfscope}%
\pgfpathrectangle{\pgfqpoint{0.100000in}{0.212622in}}{\pgfqpoint{3.696000in}{3.696000in}}%
\pgfusepath{clip}%
\pgfsetbuttcap%
\pgfsetroundjoin%
\definecolor{currentfill}{rgb}{0.121569,0.466667,0.705882}%
\pgfsetfillcolor{currentfill}%
\pgfsetfillopacity{0.682291}%
\pgfsetlinewidth{1.003750pt}%
\definecolor{currentstroke}{rgb}{0.121569,0.466667,0.705882}%
\pgfsetstrokecolor{currentstroke}%
\pgfsetstrokeopacity{0.682291}%
\pgfsetdash{}{0pt}%
\pgfpathmoveto{\pgfqpoint{0.953524in}{2.157373in}}%
\pgfpathcurveto{\pgfqpoint{0.961760in}{2.157373in}}{\pgfqpoint{0.969660in}{2.160645in}}{\pgfqpoint{0.975484in}{2.166469in}}%
\pgfpathcurveto{\pgfqpoint{0.981308in}{2.172293in}}{\pgfqpoint{0.984581in}{2.180193in}}{\pgfqpoint{0.984581in}{2.188430in}}%
\pgfpathcurveto{\pgfqpoint{0.984581in}{2.196666in}}{\pgfqpoint{0.981308in}{2.204566in}}{\pgfqpoint{0.975484in}{2.210390in}}%
\pgfpathcurveto{\pgfqpoint{0.969660in}{2.216214in}}{\pgfqpoint{0.961760in}{2.219486in}}{\pgfqpoint{0.953524in}{2.219486in}}%
\pgfpathcurveto{\pgfqpoint{0.945288in}{2.219486in}}{\pgfqpoint{0.937388in}{2.216214in}}{\pgfqpoint{0.931564in}{2.210390in}}%
\pgfpathcurveto{\pgfqpoint{0.925740in}{2.204566in}}{\pgfqpoint{0.922468in}{2.196666in}}{\pgfqpoint{0.922468in}{2.188430in}}%
\pgfpathcurveto{\pgfqpoint{0.922468in}{2.180193in}}{\pgfqpoint{0.925740in}{2.172293in}}{\pgfqpoint{0.931564in}{2.166469in}}%
\pgfpathcurveto{\pgfqpoint{0.937388in}{2.160645in}}{\pgfqpoint{0.945288in}{2.157373in}}{\pgfqpoint{0.953524in}{2.157373in}}%
\pgfpathclose%
\pgfusepath{stroke,fill}%
\end{pgfscope}%
\begin{pgfscope}%
\pgfpathrectangle{\pgfqpoint{0.100000in}{0.212622in}}{\pgfqpoint{3.696000in}{3.696000in}}%
\pgfusepath{clip}%
\pgfsetbuttcap%
\pgfsetroundjoin%
\definecolor{currentfill}{rgb}{0.121569,0.466667,0.705882}%
\pgfsetfillcolor{currentfill}%
\pgfsetfillopacity{0.683086}%
\pgfsetlinewidth{1.003750pt}%
\definecolor{currentstroke}{rgb}{0.121569,0.466667,0.705882}%
\pgfsetstrokecolor{currentstroke}%
\pgfsetstrokeopacity{0.683086}%
\pgfsetdash{}{0pt}%
\pgfpathmoveto{\pgfqpoint{0.951965in}{2.157466in}}%
\pgfpathcurveto{\pgfqpoint{0.960202in}{2.157466in}}{\pgfqpoint{0.968102in}{2.160738in}}{\pgfqpoint{0.973926in}{2.166562in}}%
\pgfpathcurveto{\pgfqpoint{0.979749in}{2.172386in}}{\pgfqpoint{0.983022in}{2.180286in}}{\pgfqpoint{0.983022in}{2.188522in}}%
\pgfpathcurveto{\pgfqpoint{0.983022in}{2.196759in}}{\pgfqpoint{0.979749in}{2.204659in}}{\pgfqpoint{0.973926in}{2.210483in}}%
\pgfpathcurveto{\pgfqpoint{0.968102in}{2.216306in}}{\pgfqpoint{0.960202in}{2.219579in}}{\pgfqpoint{0.951965in}{2.219579in}}%
\pgfpathcurveto{\pgfqpoint{0.943729in}{2.219579in}}{\pgfqpoint{0.935829in}{2.216306in}}{\pgfqpoint{0.930005in}{2.210483in}}%
\pgfpathcurveto{\pgfqpoint{0.924181in}{2.204659in}}{\pgfqpoint{0.920909in}{2.196759in}}{\pgfqpoint{0.920909in}{2.188522in}}%
\pgfpathcurveto{\pgfqpoint{0.920909in}{2.180286in}}{\pgfqpoint{0.924181in}{2.172386in}}{\pgfqpoint{0.930005in}{2.166562in}}%
\pgfpathcurveto{\pgfqpoint{0.935829in}{2.160738in}}{\pgfqpoint{0.943729in}{2.157466in}}{\pgfqpoint{0.951965in}{2.157466in}}%
\pgfpathclose%
\pgfusepath{stroke,fill}%
\end{pgfscope}%
\begin{pgfscope}%
\pgfpathrectangle{\pgfqpoint{0.100000in}{0.212622in}}{\pgfqpoint{3.696000in}{3.696000in}}%
\pgfusepath{clip}%
\pgfsetbuttcap%
\pgfsetroundjoin%
\definecolor{currentfill}{rgb}{0.121569,0.466667,0.705882}%
\pgfsetfillcolor{currentfill}%
\pgfsetfillopacity{0.683682}%
\pgfsetlinewidth{1.003750pt}%
\definecolor{currentstroke}{rgb}{0.121569,0.466667,0.705882}%
\pgfsetstrokecolor{currentstroke}%
\pgfsetstrokeopacity{0.683682}%
\pgfsetdash{}{0pt}%
\pgfpathmoveto{\pgfqpoint{0.950599in}{2.157546in}}%
\pgfpathcurveto{\pgfqpoint{0.958836in}{2.157546in}}{\pgfqpoint{0.966736in}{2.160819in}}{\pgfqpoint{0.972560in}{2.166643in}}%
\pgfpathcurveto{\pgfqpoint{0.978383in}{2.172467in}}{\pgfqpoint{0.981656in}{2.180367in}}{\pgfqpoint{0.981656in}{2.188603in}}%
\pgfpathcurveto{\pgfqpoint{0.981656in}{2.196839in}}{\pgfqpoint{0.978383in}{2.204739in}}{\pgfqpoint{0.972560in}{2.210563in}}%
\pgfpathcurveto{\pgfqpoint{0.966736in}{2.216387in}}{\pgfqpoint{0.958836in}{2.219659in}}{\pgfqpoint{0.950599in}{2.219659in}}%
\pgfpathcurveto{\pgfqpoint{0.942363in}{2.219659in}}{\pgfqpoint{0.934463in}{2.216387in}}{\pgfqpoint{0.928639in}{2.210563in}}%
\pgfpathcurveto{\pgfqpoint{0.922815in}{2.204739in}}{\pgfqpoint{0.919543in}{2.196839in}}{\pgfqpoint{0.919543in}{2.188603in}}%
\pgfpathcurveto{\pgfqpoint{0.919543in}{2.180367in}}{\pgfqpoint{0.922815in}{2.172467in}}{\pgfqpoint{0.928639in}{2.166643in}}%
\pgfpathcurveto{\pgfqpoint{0.934463in}{2.160819in}}{\pgfqpoint{0.942363in}{2.157546in}}{\pgfqpoint{0.950599in}{2.157546in}}%
\pgfpathclose%
\pgfusepath{stroke,fill}%
\end{pgfscope}%
\begin{pgfscope}%
\pgfpathrectangle{\pgfqpoint{0.100000in}{0.212622in}}{\pgfqpoint{3.696000in}{3.696000in}}%
\pgfusepath{clip}%
\pgfsetbuttcap%
\pgfsetroundjoin%
\definecolor{currentfill}{rgb}{0.121569,0.466667,0.705882}%
\pgfsetfillcolor{currentfill}%
\pgfsetfillopacity{0.684729}%
\pgfsetlinewidth{1.003750pt}%
\definecolor{currentstroke}{rgb}{0.121569,0.466667,0.705882}%
\pgfsetstrokecolor{currentstroke}%
\pgfsetstrokeopacity{0.684729}%
\pgfsetdash{}{0pt}%
\pgfpathmoveto{\pgfqpoint{0.947868in}{2.157743in}}%
\pgfpathcurveto{\pgfqpoint{0.956104in}{2.157743in}}{\pgfqpoint{0.964004in}{2.161015in}}{\pgfqpoint{0.969828in}{2.166839in}}%
\pgfpathcurveto{\pgfqpoint{0.975652in}{2.172663in}}{\pgfqpoint{0.978924in}{2.180563in}}{\pgfqpoint{0.978924in}{2.188799in}}%
\pgfpathcurveto{\pgfqpoint{0.978924in}{2.197036in}}{\pgfqpoint{0.975652in}{2.204936in}}{\pgfqpoint{0.969828in}{2.210760in}}%
\pgfpathcurveto{\pgfqpoint{0.964004in}{2.216584in}}{\pgfqpoint{0.956104in}{2.219856in}}{\pgfqpoint{0.947868in}{2.219856in}}%
\pgfpathcurveto{\pgfqpoint{0.939632in}{2.219856in}}{\pgfqpoint{0.931732in}{2.216584in}}{\pgfqpoint{0.925908in}{2.210760in}}%
\pgfpathcurveto{\pgfqpoint{0.920084in}{2.204936in}}{\pgfqpoint{0.916811in}{2.197036in}}{\pgfqpoint{0.916811in}{2.188799in}}%
\pgfpathcurveto{\pgfqpoint{0.916811in}{2.180563in}}{\pgfqpoint{0.920084in}{2.172663in}}{\pgfqpoint{0.925908in}{2.166839in}}%
\pgfpathcurveto{\pgfqpoint{0.931732in}{2.161015in}}{\pgfqpoint{0.939632in}{2.157743in}}{\pgfqpoint{0.947868in}{2.157743in}}%
\pgfpathclose%
\pgfusepath{stroke,fill}%
\end{pgfscope}%
\begin{pgfscope}%
\pgfpathrectangle{\pgfqpoint{0.100000in}{0.212622in}}{\pgfqpoint{3.696000in}{3.696000in}}%
\pgfusepath{clip}%
\pgfsetbuttcap%
\pgfsetroundjoin%
\definecolor{currentfill}{rgb}{0.121569,0.466667,0.705882}%
\pgfsetfillcolor{currentfill}%
\pgfsetfillopacity{0.685187}%
\pgfsetlinewidth{1.003750pt}%
\definecolor{currentstroke}{rgb}{0.121569,0.466667,0.705882}%
\pgfsetstrokecolor{currentstroke}%
\pgfsetstrokeopacity{0.685187}%
\pgfsetdash{}{0pt}%
\pgfpathmoveto{\pgfqpoint{3.016158in}{1.802161in}}%
\pgfpathcurveto{\pgfqpoint{3.024395in}{1.802161in}}{\pgfqpoint{3.032295in}{1.805433in}}{\pgfqpoint{3.038119in}{1.811257in}}%
\pgfpathcurveto{\pgfqpoint{3.043943in}{1.817081in}}{\pgfqpoint{3.047215in}{1.824981in}}{\pgfqpoint{3.047215in}{1.833218in}}%
\pgfpathcurveto{\pgfqpoint{3.047215in}{1.841454in}}{\pgfqpoint{3.043943in}{1.849354in}}{\pgfqpoint{3.038119in}{1.855178in}}%
\pgfpathcurveto{\pgfqpoint{3.032295in}{1.861002in}}{\pgfqpoint{3.024395in}{1.864274in}}{\pgfqpoint{3.016158in}{1.864274in}}%
\pgfpathcurveto{\pgfqpoint{3.007922in}{1.864274in}}{\pgfqpoint{3.000022in}{1.861002in}}{\pgfqpoint{2.994198in}{1.855178in}}%
\pgfpathcurveto{\pgfqpoint{2.988374in}{1.849354in}}{\pgfqpoint{2.985102in}{1.841454in}}{\pgfqpoint{2.985102in}{1.833218in}}%
\pgfpathcurveto{\pgfqpoint{2.985102in}{1.824981in}}{\pgfqpoint{2.988374in}{1.817081in}}{\pgfqpoint{2.994198in}{1.811257in}}%
\pgfpathcurveto{\pgfqpoint{3.000022in}{1.805433in}}{\pgfqpoint{3.007922in}{1.802161in}}{\pgfqpoint{3.016158in}{1.802161in}}%
\pgfpathclose%
\pgfusepath{stroke,fill}%
\end{pgfscope}%
\begin{pgfscope}%
\pgfpathrectangle{\pgfqpoint{0.100000in}{0.212622in}}{\pgfqpoint{3.696000in}{3.696000in}}%
\pgfusepath{clip}%
\pgfsetbuttcap%
\pgfsetroundjoin%
\definecolor{currentfill}{rgb}{0.121569,0.466667,0.705882}%
\pgfsetfillcolor{currentfill}%
\pgfsetfillopacity{0.685744}%
\pgfsetlinewidth{1.003750pt}%
\definecolor{currentstroke}{rgb}{0.121569,0.466667,0.705882}%
\pgfsetstrokecolor{currentstroke}%
\pgfsetstrokeopacity{0.685744}%
\pgfsetdash{}{0pt}%
\pgfpathmoveto{\pgfqpoint{0.946509in}{2.157946in}}%
\pgfpathcurveto{\pgfqpoint{0.954745in}{2.157946in}}{\pgfqpoint{0.962645in}{2.161218in}}{\pgfqpoint{0.968469in}{2.167042in}}%
\pgfpathcurveto{\pgfqpoint{0.974293in}{2.172866in}}{\pgfqpoint{0.977566in}{2.180766in}}{\pgfqpoint{0.977566in}{2.189002in}}%
\pgfpathcurveto{\pgfqpoint{0.977566in}{2.197239in}}{\pgfqpoint{0.974293in}{2.205139in}}{\pgfqpoint{0.968469in}{2.210963in}}%
\pgfpathcurveto{\pgfqpoint{0.962645in}{2.216787in}}{\pgfqpoint{0.954745in}{2.220059in}}{\pgfqpoint{0.946509in}{2.220059in}}%
\pgfpathcurveto{\pgfqpoint{0.938273in}{2.220059in}}{\pgfqpoint{0.930373in}{2.216787in}}{\pgfqpoint{0.924549in}{2.210963in}}%
\pgfpathcurveto{\pgfqpoint{0.918725in}{2.205139in}}{\pgfqpoint{0.915453in}{2.197239in}}{\pgfqpoint{0.915453in}{2.189002in}}%
\pgfpathcurveto{\pgfqpoint{0.915453in}{2.180766in}}{\pgfqpoint{0.918725in}{2.172866in}}{\pgfqpoint{0.924549in}{2.167042in}}%
\pgfpathcurveto{\pgfqpoint{0.930373in}{2.161218in}}{\pgfqpoint{0.938273in}{2.157946in}}{\pgfqpoint{0.946509in}{2.157946in}}%
\pgfpathclose%
\pgfusepath{stroke,fill}%
\end{pgfscope}%
\begin{pgfscope}%
\pgfpathrectangle{\pgfqpoint{0.100000in}{0.212622in}}{\pgfqpoint{3.696000in}{3.696000in}}%
\pgfusepath{clip}%
\pgfsetbuttcap%
\pgfsetroundjoin%
\definecolor{currentfill}{rgb}{0.121569,0.466667,0.705882}%
\pgfsetfillcolor{currentfill}%
\pgfsetfillopacity{0.686296}%
\pgfsetlinewidth{1.003750pt}%
\definecolor{currentstroke}{rgb}{0.121569,0.466667,0.705882}%
\pgfsetstrokecolor{currentstroke}%
\pgfsetstrokeopacity{0.686296}%
\pgfsetdash{}{0pt}%
\pgfpathmoveto{\pgfqpoint{0.944919in}{2.158109in}}%
\pgfpathcurveto{\pgfqpoint{0.953155in}{2.158109in}}{\pgfqpoint{0.961055in}{2.161382in}}{\pgfqpoint{0.966879in}{2.167205in}}%
\pgfpathcurveto{\pgfqpoint{0.972703in}{2.173029in}}{\pgfqpoint{0.975975in}{2.180929in}}{\pgfqpoint{0.975975in}{2.189166in}}%
\pgfpathcurveto{\pgfqpoint{0.975975in}{2.197402in}}{\pgfqpoint{0.972703in}{2.205302in}}{\pgfqpoint{0.966879in}{2.211126in}}%
\pgfpathcurveto{\pgfqpoint{0.961055in}{2.216950in}}{\pgfqpoint{0.953155in}{2.220222in}}{\pgfqpoint{0.944919in}{2.220222in}}%
\pgfpathcurveto{\pgfqpoint{0.936683in}{2.220222in}}{\pgfqpoint{0.928783in}{2.216950in}}{\pgfqpoint{0.922959in}{2.211126in}}%
\pgfpathcurveto{\pgfqpoint{0.917135in}{2.205302in}}{\pgfqpoint{0.913862in}{2.197402in}}{\pgfqpoint{0.913862in}{2.189166in}}%
\pgfpathcurveto{\pgfqpoint{0.913862in}{2.180929in}}{\pgfqpoint{0.917135in}{2.173029in}}{\pgfqpoint{0.922959in}{2.167205in}}%
\pgfpathcurveto{\pgfqpoint{0.928783in}{2.161382in}}{\pgfqpoint{0.936683in}{2.158109in}}{\pgfqpoint{0.944919in}{2.158109in}}%
\pgfpathclose%
\pgfusepath{stroke,fill}%
\end{pgfscope}%
\begin{pgfscope}%
\pgfpathrectangle{\pgfqpoint{0.100000in}{0.212622in}}{\pgfqpoint{3.696000in}{3.696000in}}%
\pgfusepath{clip}%
\pgfsetbuttcap%
\pgfsetroundjoin%
\definecolor{currentfill}{rgb}{0.121569,0.466667,0.705882}%
\pgfsetfillcolor{currentfill}%
\pgfsetfillopacity{0.686744}%
\pgfsetlinewidth{1.003750pt}%
\definecolor{currentstroke}{rgb}{0.121569,0.466667,0.705882}%
\pgfsetstrokecolor{currentstroke}%
\pgfsetstrokeopacity{0.686744}%
\pgfsetdash{}{0pt}%
\pgfpathmoveto{\pgfqpoint{0.945968in}{2.158558in}}%
\pgfpathcurveto{\pgfqpoint{0.954205in}{2.158558in}}{\pgfqpoint{0.962105in}{2.161830in}}{\pgfqpoint{0.967929in}{2.167654in}}%
\pgfpathcurveto{\pgfqpoint{0.973753in}{2.173478in}}{\pgfqpoint{0.977025in}{2.181378in}}{\pgfqpoint{0.977025in}{2.189614in}}%
\pgfpathcurveto{\pgfqpoint{0.977025in}{2.197851in}}{\pgfqpoint{0.973753in}{2.205751in}}{\pgfqpoint{0.967929in}{2.211575in}}%
\pgfpathcurveto{\pgfqpoint{0.962105in}{2.217398in}}{\pgfqpoint{0.954205in}{2.220671in}}{\pgfqpoint{0.945968in}{2.220671in}}%
\pgfpathcurveto{\pgfqpoint{0.937732in}{2.220671in}}{\pgfqpoint{0.929832in}{2.217398in}}{\pgfqpoint{0.924008in}{2.211575in}}%
\pgfpathcurveto{\pgfqpoint{0.918184in}{2.205751in}}{\pgfqpoint{0.914912in}{2.197851in}}{\pgfqpoint{0.914912in}{2.189614in}}%
\pgfpathcurveto{\pgfqpoint{0.914912in}{2.181378in}}{\pgfqpoint{0.918184in}{2.173478in}}{\pgfqpoint{0.924008in}{2.167654in}}%
\pgfpathcurveto{\pgfqpoint{0.929832in}{2.161830in}}{\pgfqpoint{0.937732in}{2.158558in}}{\pgfqpoint{0.945968in}{2.158558in}}%
\pgfpathclose%
\pgfusepath{stroke,fill}%
\end{pgfscope}%
\begin{pgfscope}%
\pgfpathrectangle{\pgfqpoint{0.100000in}{0.212622in}}{\pgfqpoint{3.696000in}{3.696000in}}%
\pgfusepath{clip}%
\pgfsetbuttcap%
\pgfsetroundjoin%
\definecolor{currentfill}{rgb}{0.121569,0.466667,0.705882}%
\pgfsetfillcolor{currentfill}%
\pgfsetfillopacity{0.687071}%
\pgfsetlinewidth{1.003750pt}%
\definecolor{currentstroke}{rgb}{0.121569,0.466667,0.705882}%
\pgfsetstrokecolor{currentstroke}%
\pgfsetstrokeopacity{0.687071}%
\pgfsetdash{}{0pt}%
\pgfpathmoveto{\pgfqpoint{3.012958in}{1.802470in}}%
\pgfpathcurveto{\pgfqpoint{3.021194in}{1.802470in}}{\pgfqpoint{3.029094in}{1.805742in}}{\pgfqpoint{3.034918in}{1.811566in}}%
\pgfpathcurveto{\pgfqpoint{3.040742in}{1.817390in}}{\pgfqpoint{3.044014in}{1.825290in}}{\pgfqpoint{3.044014in}{1.833526in}}%
\pgfpathcurveto{\pgfqpoint{3.044014in}{1.841762in}}{\pgfqpoint{3.040742in}{1.849662in}}{\pgfqpoint{3.034918in}{1.855486in}}%
\pgfpathcurveto{\pgfqpoint{3.029094in}{1.861310in}}{\pgfqpoint{3.021194in}{1.864583in}}{\pgfqpoint{3.012958in}{1.864583in}}%
\pgfpathcurveto{\pgfqpoint{3.004722in}{1.864583in}}{\pgfqpoint{2.996822in}{1.861310in}}{\pgfqpoint{2.990998in}{1.855486in}}%
\pgfpathcurveto{\pgfqpoint{2.985174in}{1.849662in}}{\pgfqpoint{2.981901in}{1.841762in}}{\pgfqpoint{2.981901in}{1.833526in}}%
\pgfpathcurveto{\pgfqpoint{2.981901in}{1.825290in}}{\pgfqpoint{2.985174in}{1.817390in}}{\pgfqpoint{2.990998in}{1.811566in}}%
\pgfpathcurveto{\pgfqpoint{2.996822in}{1.805742in}}{\pgfqpoint{3.004722in}{1.802470in}}{\pgfqpoint{3.012958in}{1.802470in}}%
\pgfpathclose%
\pgfusepath{stroke,fill}%
\end{pgfscope}%
\begin{pgfscope}%
\pgfpathrectangle{\pgfqpoint{0.100000in}{0.212622in}}{\pgfqpoint{3.696000in}{3.696000in}}%
\pgfusepath{clip}%
\pgfsetbuttcap%
\pgfsetroundjoin%
\definecolor{currentfill}{rgb}{0.121569,0.466667,0.705882}%
\pgfsetfillcolor{currentfill}%
\pgfsetfillopacity{0.687561}%
\pgfsetlinewidth{1.003750pt}%
\definecolor{currentstroke}{rgb}{0.121569,0.466667,0.705882}%
\pgfsetstrokecolor{currentstroke}%
\pgfsetstrokeopacity{0.687561}%
\pgfsetdash{}{0pt}%
\pgfpathmoveto{\pgfqpoint{0.944230in}{2.158613in}}%
\pgfpathcurveto{\pgfqpoint{0.952466in}{2.158613in}}{\pgfqpoint{0.960366in}{2.161885in}}{\pgfqpoint{0.966190in}{2.167709in}}%
\pgfpathcurveto{\pgfqpoint{0.972014in}{2.173533in}}{\pgfqpoint{0.975286in}{2.181433in}}{\pgfqpoint{0.975286in}{2.189669in}}%
\pgfpathcurveto{\pgfqpoint{0.975286in}{2.197905in}}{\pgfqpoint{0.972014in}{2.205805in}}{\pgfqpoint{0.966190in}{2.211629in}}%
\pgfpathcurveto{\pgfqpoint{0.960366in}{2.217453in}}{\pgfqpoint{0.952466in}{2.220726in}}{\pgfqpoint{0.944230in}{2.220726in}}%
\pgfpathcurveto{\pgfqpoint{0.935993in}{2.220726in}}{\pgfqpoint{0.928093in}{2.217453in}}{\pgfqpoint{0.922269in}{2.211629in}}%
\pgfpathcurveto{\pgfqpoint{0.916445in}{2.205805in}}{\pgfqpoint{0.913173in}{2.197905in}}{\pgfqpoint{0.913173in}{2.189669in}}%
\pgfpathcurveto{\pgfqpoint{0.913173in}{2.181433in}}{\pgfqpoint{0.916445in}{2.173533in}}{\pgfqpoint{0.922269in}{2.167709in}}%
\pgfpathcurveto{\pgfqpoint{0.928093in}{2.161885in}}{\pgfqpoint{0.935993in}{2.158613in}}{\pgfqpoint{0.944230in}{2.158613in}}%
\pgfpathclose%
\pgfusepath{stroke,fill}%
\end{pgfscope}%
\begin{pgfscope}%
\pgfpathrectangle{\pgfqpoint{0.100000in}{0.212622in}}{\pgfqpoint{3.696000in}{3.696000in}}%
\pgfusepath{clip}%
\pgfsetbuttcap%
\pgfsetroundjoin%
\definecolor{currentfill}{rgb}{0.121569,0.466667,0.705882}%
\pgfsetfillcolor{currentfill}%
\pgfsetfillopacity{0.688041}%
\pgfsetlinewidth{1.003750pt}%
\definecolor{currentstroke}{rgb}{0.121569,0.466667,0.705882}%
\pgfsetstrokecolor{currentstroke}%
\pgfsetstrokeopacity{0.688041}%
\pgfsetdash{}{0pt}%
\pgfpathmoveto{\pgfqpoint{3.010553in}{1.802985in}}%
\pgfpathcurveto{\pgfqpoint{3.018789in}{1.802985in}}{\pgfqpoint{3.026689in}{1.806257in}}{\pgfqpoint{3.032513in}{1.812081in}}%
\pgfpathcurveto{\pgfqpoint{3.038337in}{1.817905in}}{\pgfqpoint{3.041609in}{1.825805in}}{\pgfqpoint{3.041609in}{1.834042in}}%
\pgfpathcurveto{\pgfqpoint{3.041609in}{1.842278in}}{\pgfqpoint{3.038337in}{1.850178in}}{\pgfqpoint{3.032513in}{1.856002in}}%
\pgfpathcurveto{\pgfqpoint{3.026689in}{1.861826in}}{\pgfqpoint{3.018789in}{1.865098in}}{\pgfqpoint{3.010553in}{1.865098in}}%
\pgfpathcurveto{\pgfqpoint{3.002317in}{1.865098in}}{\pgfqpoint{2.994416in}{1.861826in}}{\pgfqpoint{2.988593in}{1.856002in}}%
\pgfpathcurveto{\pgfqpoint{2.982769in}{1.850178in}}{\pgfqpoint{2.979496in}{1.842278in}}{\pgfqpoint{2.979496in}{1.834042in}}%
\pgfpathcurveto{\pgfqpoint{2.979496in}{1.825805in}}{\pgfqpoint{2.982769in}{1.817905in}}{\pgfqpoint{2.988593in}{1.812081in}}%
\pgfpathcurveto{\pgfqpoint{2.994416in}{1.806257in}}{\pgfqpoint{3.002317in}{1.802985in}}{\pgfqpoint{3.010553in}{1.802985in}}%
\pgfpathclose%
\pgfusepath{stroke,fill}%
\end{pgfscope}%
\begin{pgfscope}%
\pgfpathrectangle{\pgfqpoint{0.100000in}{0.212622in}}{\pgfqpoint{3.696000in}{3.696000in}}%
\pgfusepath{clip}%
\pgfsetbuttcap%
\pgfsetroundjoin%
\definecolor{currentfill}{rgb}{0.121569,0.466667,0.705882}%
\pgfsetfillcolor{currentfill}%
\pgfsetfillopacity{0.688304}%
\pgfsetlinewidth{1.003750pt}%
\definecolor{currentstroke}{rgb}{0.121569,0.466667,0.705882}%
\pgfsetstrokecolor{currentstroke}%
\pgfsetstrokeopacity{0.688304}%
\pgfsetdash{}{0pt}%
\pgfpathmoveto{\pgfqpoint{0.942624in}{2.158720in}}%
\pgfpathcurveto{\pgfqpoint{0.950860in}{2.158720in}}{\pgfqpoint{0.958760in}{2.161993in}}{\pgfqpoint{0.964584in}{2.167817in}}%
\pgfpathcurveto{\pgfqpoint{0.970408in}{2.173641in}}{\pgfqpoint{0.973680in}{2.181541in}}{\pgfqpoint{0.973680in}{2.189777in}}%
\pgfpathcurveto{\pgfqpoint{0.973680in}{2.198013in}}{\pgfqpoint{0.970408in}{2.205913in}}{\pgfqpoint{0.964584in}{2.211737in}}%
\pgfpathcurveto{\pgfqpoint{0.958760in}{2.217561in}}{\pgfqpoint{0.950860in}{2.220833in}}{\pgfqpoint{0.942624in}{2.220833in}}%
\pgfpathcurveto{\pgfqpoint{0.934388in}{2.220833in}}{\pgfqpoint{0.926488in}{2.217561in}}{\pgfqpoint{0.920664in}{2.211737in}}%
\pgfpathcurveto{\pgfqpoint{0.914840in}{2.205913in}}{\pgfqpoint{0.911567in}{2.198013in}}{\pgfqpoint{0.911567in}{2.189777in}}%
\pgfpathcurveto{\pgfqpoint{0.911567in}{2.181541in}}{\pgfqpoint{0.914840in}{2.173641in}}{\pgfqpoint{0.920664in}{2.167817in}}%
\pgfpathcurveto{\pgfqpoint{0.926488in}{2.161993in}}{\pgfqpoint{0.934388in}{2.158720in}}{\pgfqpoint{0.942624in}{2.158720in}}%
\pgfpathclose%
\pgfusepath{stroke,fill}%
\end{pgfscope}%
\begin{pgfscope}%
\pgfpathrectangle{\pgfqpoint{0.100000in}{0.212622in}}{\pgfqpoint{3.696000in}{3.696000in}}%
\pgfusepath{clip}%
\pgfsetbuttcap%
\pgfsetroundjoin%
\definecolor{currentfill}{rgb}{0.121569,0.466667,0.705882}%
\pgfsetfillcolor{currentfill}%
\pgfsetfillopacity{0.689671}%
\pgfsetlinewidth{1.003750pt}%
\definecolor{currentstroke}{rgb}{0.121569,0.466667,0.705882}%
\pgfsetstrokecolor{currentstroke}%
\pgfsetstrokeopacity{0.689671}%
\pgfsetdash{}{0pt}%
\pgfpathmoveto{\pgfqpoint{0.939929in}{2.158807in}}%
\pgfpathcurveto{\pgfqpoint{0.948165in}{2.158807in}}{\pgfqpoint{0.956065in}{2.162080in}}{\pgfqpoint{0.961889in}{2.167904in}}%
\pgfpathcurveto{\pgfqpoint{0.967713in}{2.173728in}}{\pgfqpoint{0.970986in}{2.181628in}}{\pgfqpoint{0.970986in}{2.189864in}}%
\pgfpathcurveto{\pgfqpoint{0.970986in}{2.198100in}}{\pgfqpoint{0.967713in}{2.206000in}}{\pgfqpoint{0.961889in}{2.211824in}}%
\pgfpathcurveto{\pgfqpoint{0.956065in}{2.217648in}}{\pgfqpoint{0.948165in}{2.220920in}}{\pgfqpoint{0.939929in}{2.220920in}}%
\pgfpathcurveto{\pgfqpoint{0.931693in}{2.220920in}}{\pgfqpoint{0.923793in}{2.217648in}}{\pgfqpoint{0.917969in}{2.211824in}}%
\pgfpathcurveto{\pgfqpoint{0.912145in}{2.206000in}}{\pgfqpoint{0.908873in}{2.198100in}}{\pgfqpoint{0.908873in}{2.189864in}}%
\pgfpathcurveto{\pgfqpoint{0.908873in}{2.181628in}}{\pgfqpoint{0.912145in}{2.173728in}}{\pgfqpoint{0.917969in}{2.167904in}}%
\pgfpathcurveto{\pgfqpoint{0.923793in}{2.162080in}}{\pgfqpoint{0.931693in}{2.158807in}}{\pgfqpoint{0.939929in}{2.158807in}}%
\pgfpathclose%
\pgfusepath{stroke,fill}%
\end{pgfscope}%
\begin{pgfscope}%
\pgfpathrectangle{\pgfqpoint{0.100000in}{0.212622in}}{\pgfqpoint{3.696000in}{3.696000in}}%
\pgfusepath{clip}%
\pgfsetbuttcap%
\pgfsetroundjoin%
\definecolor{currentfill}{rgb}{0.121569,0.466667,0.705882}%
\pgfsetfillcolor{currentfill}%
\pgfsetfillopacity{0.689741}%
\pgfsetlinewidth{1.003750pt}%
\definecolor{currentstroke}{rgb}{0.121569,0.466667,0.705882}%
\pgfsetstrokecolor{currentstroke}%
\pgfsetstrokeopacity{0.689741}%
\pgfsetdash{}{0pt}%
\pgfpathmoveto{\pgfqpoint{3.007786in}{1.803166in}}%
\pgfpathcurveto{\pgfqpoint{3.016022in}{1.803166in}}{\pgfqpoint{3.023922in}{1.806438in}}{\pgfqpoint{3.029746in}{1.812262in}}%
\pgfpathcurveto{\pgfqpoint{3.035570in}{1.818086in}}{\pgfqpoint{3.038842in}{1.825986in}}{\pgfqpoint{3.038842in}{1.834223in}}%
\pgfpathcurveto{\pgfqpoint{3.038842in}{1.842459in}}{\pgfqpoint{3.035570in}{1.850359in}}{\pgfqpoint{3.029746in}{1.856183in}}%
\pgfpathcurveto{\pgfqpoint{3.023922in}{1.862007in}}{\pgfqpoint{3.016022in}{1.865279in}}{\pgfqpoint{3.007786in}{1.865279in}}%
\pgfpathcurveto{\pgfqpoint{2.999549in}{1.865279in}}{\pgfqpoint{2.991649in}{1.862007in}}{\pgfqpoint{2.985825in}{1.856183in}}%
\pgfpathcurveto{\pgfqpoint{2.980002in}{1.850359in}}{\pgfqpoint{2.976729in}{1.842459in}}{\pgfqpoint{2.976729in}{1.834223in}}%
\pgfpathcurveto{\pgfqpoint{2.976729in}{1.825986in}}{\pgfqpoint{2.980002in}{1.818086in}}{\pgfqpoint{2.985825in}{1.812262in}}%
\pgfpathcurveto{\pgfqpoint{2.991649in}{1.806438in}}{\pgfqpoint{2.999549in}{1.803166in}}{\pgfqpoint{3.007786in}{1.803166in}}%
\pgfpathclose%
\pgfusepath{stroke,fill}%
\end{pgfscope}%
\begin{pgfscope}%
\pgfpathrectangle{\pgfqpoint{0.100000in}{0.212622in}}{\pgfqpoint{3.696000in}{3.696000in}}%
\pgfusepath{clip}%
\pgfsetbuttcap%
\pgfsetroundjoin%
\definecolor{currentfill}{rgb}{0.121569,0.466667,0.705882}%
\pgfsetfillcolor{currentfill}%
\pgfsetfillopacity{0.690416}%
\pgfsetlinewidth{1.003750pt}%
\definecolor{currentstroke}{rgb}{0.121569,0.466667,0.705882}%
\pgfsetstrokecolor{currentstroke}%
\pgfsetstrokeopacity{0.690416}%
\pgfsetdash{}{0pt}%
\pgfpathmoveto{\pgfqpoint{0.937944in}{2.158966in}}%
\pgfpathcurveto{\pgfqpoint{0.946180in}{2.158966in}}{\pgfqpoint{0.954080in}{2.162238in}}{\pgfqpoint{0.959904in}{2.168062in}}%
\pgfpathcurveto{\pgfqpoint{0.965728in}{2.173886in}}{\pgfqpoint{0.969001in}{2.181786in}}{\pgfqpoint{0.969001in}{2.190022in}}%
\pgfpathcurveto{\pgfqpoint{0.969001in}{2.198258in}}{\pgfqpoint{0.965728in}{2.206158in}}{\pgfqpoint{0.959904in}{2.211982in}}%
\pgfpathcurveto{\pgfqpoint{0.954080in}{2.217806in}}{\pgfqpoint{0.946180in}{2.221079in}}{\pgfqpoint{0.937944in}{2.221079in}}%
\pgfpathcurveto{\pgfqpoint{0.929708in}{2.221079in}}{\pgfqpoint{0.921808in}{2.217806in}}{\pgfqpoint{0.915984in}{2.211982in}}%
\pgfpathcurveto{\pgfqpoint{0.910160in}{2.206158in}}{\pgfqpoint{0.906888in}{2.198258in}}{\pgfqpoint{0.906888in}{2.190022in}}%
\pgfpathcurveto{\pgfqpoint{0.906888in}{2.181786in}}{\pgfqpoint{0.910160in}{2.173886in}}{\pgfqpoint{0.915984in}{2.168062in}}%
\pgfpathcurveto{\pgfqpoint{0.921808in}{2.162238in}}{\pgfqpoint{0.929708in}{2.158966in}}{\pgfqpoint{0.937944in}{2.158966in}}%
\pgfpathclose%
\pgfusepath{stroke,fill}%
\end{pgfscope}%
\begin{pgfscope}%
\pgfpathrectangle{\pgfqpoint{0.100000in}{0.212622in}}{\pgfqpoint{3.696000in}{3.696000in}}%
\pgfusepath{clip}%
\pgfsetbuttcap%
\pgfsetroundjoin%
\definecolor{currentfill}{rgb}{0.121569,0.466667,0.705882}%
\pgfsetfillcolor{currentfill}%
\pgfsetfillopacity{0.690749}%
\pgfsetlinewidth{1.003750pt}%
\definecolor{currentstroke}{rgb}{0.121569,0.466667,0.705882}%
\pgfsetstrokecolor{currentstroke}%
\pgfsetstrokeopacity{0.690749}%
\pgfsetdash{}{0pt}%
\pgfpathmoveto{\pgfqpoint{3.006961in}{1.803256in}}%
\pgfpathcurveto{\pgfqpoint{3.015197in}{1.803256in}}{\pgfqpoint{3.023097in}{1.806528in}}{\pgfqpoint{3.028921in}{1.812352in}}%
\pgfpathcurveto{\pgfqpoint{3.034745in}{1.818176in}}{\pgfqpoint{3.038017in}{1.826076in}}{\pgfqpoint{3.038017in}{1.834312in}}%
\pgfpathcurveto{\pgfqpoint{3.038017in}{1.842549in}}{\pgfqpoint{3.034745in}{1.850449in}}{\pgfqpoint{3.028921in}{1.856272in}}%
\pgfpathcurveto{\pgfqpoint{3.023097in}{1.862096in}}{\pgfqpoint{3.015197in}{1.865369in}}{\pgfqpoint{3.006961in}{1.865369in}}%
\pgfpathcurveto{\pgfqpoint{2.998724in}{1.865369in}}{\pgfqpoint{2.990824in}{1.862096in}}{\pgfqpoint{2.985000in}{1.856272in}}%
\pgfpathcurveto{\pgfqpoint{2.979176in}{1.850449in}}{\pgfqpoint{2.975904in}{1.842549in}}{\pgfqpoint{2.975904in}{1.834312in}}%
\pgfpathcurveto{\pgfqpoint{2.975904in}{1.826076in}}{\pgfqpoint{2.979176in}{1.818176in}}{\pgfqpoint{2.985000in}{1.812352in}}%
\pgfpathcurveto{\pgfqpoint{2.990824in}{1.806528in}}{\pgfqpoint{2.998724in}{1.803256in}}{\pgfqpoint{3.006961in}{1.803256in}}%
\pgfpathclose%
\pgfusepath{stroke,fill}%
\end{pgfscope}%
\begin{pgfscope}%
\pgfpathrectangle{\pgfqpoint{0.100000in}{0.212622in}}{\pgfqpoint{3.696000in}{3.696000in}}%
\pgfusepath{clip}%
\pgfsetbuttcap%
\pgfsetroundjoin%
\definecolor{currentfill}{rgb}{0.121569,0.466667,0.705882}%
\pgfsetfillcolor{currentfill}%
\pgfsetfillopacity{0.691145}%
\pgfsetlinewidth{1.003750pt}%
\definecolor{currentstroke}{rgb}{0.121569,0.466667,0.705882}%
\pgfsetstrokecolor{currentstroke}%
\pgfsetstrokeopacity{0.691145}%
\pgfsetdash{}{0pt}%
\pgfpathmoveto{\pgfqpoint{0.936945in}{2.159053in}}%
\pgfpathcurveto{\pgfqpoint{0.945181in}{2.159053in}}{\pgfqpoint{0.953081in}{2.162326in}}{\pgfqpoint{0.958905in}{2.168149in}}%
\pgfpathcurveto{\pgfqpoint{0.964729in}{2.173973in}}{\pgfqpoint{0.968001in}{2.181873in}}{\pgfqpoint{0.968001in}{2.190110in}}%
\pgfpathcurveto{\pgfqpoint{0.968001in}{2.198346in}}{\pgfqpoint{0.964729in}{2.206246in}}{\pgfqpoint{0.958905in}{2.212070in}}%
\pgfpathcurveto{\pgfqpoint{0.953081in}{2.217894in}}{\pgfqpoint{0.945181in}{2.221166in}}{\pgfqpoint{0.936945in}{2.221166in}}%
\pgfpathcurveto{\pgfqpoint{0.928709in}{2.221166in}}{\pgfqpoint{0.920809in}{2.217894in}}{\pgfqpoint{0.914985in}{2.212070in}}%
\pgfpathcurveto{\pgfqpoint{0.909161in}{2.206246in}}{\pgfqpoint{0.905888in}{2.198346in}}{\pgfqpoint{0.905888in}{2.190110in}}%
\pgfpathcurveto{\pgfqpoint{0.905888in}{2.181873in}}{\pgfqpoint{0.909161in}{2.173973in}}{\pgfqpoint{0.914985in}{2.168149in}}%
\pgfpathcurveto{\pgfqpoint{0.920809in}{2.162326in}}{\pgfqpoint{0.928709in}{2.159053in}}{\pgfqpoint{0.936945in}{2.159053in}}%
\pgfpathclose%
\pgfusepath{stroke,fill}%
\end{pgfscope}%
\begin{pgfscope}%
\pgfpathrectangle{\pgfqpoint{0.100000in}{0.212622in}}{\pgfqpoint{3.696000in}{3.696000in}}%
\pgfusepath{clip}%
\pgfsetbuttcap%
\pgfsetroundjoin%
\definecolor{currentfill}{rgb}{0.121569,0.466667,0.705882}%
\pgfsetfillcolor{currentfill}%
\pgfsetfillopacity{0.691404}%
\pgfsetlinewidth{1.003750pt}%
\definecolor{currentstroke}{rgb}{0.121569,0.466667,0.705882}%
\pgfsetstrokecolor{currentstroke}%
\pgfsetstrokeopacity{0.691404}%
\pgfsetdash{}{0pt}%
\pgfpathmoveto{\pgfqpoint{0.936215in}{2.159136in}}%
\pgfpathcurveto{\pgfqpoint{0.944451in}{2.159136in}}{\pgfqpoint{0.952352in}{2.162408in}}{\pgfqpoint{0.958175in}{2.168232in}}%
\pgfpathcurveto{\pgfqpoint{0.963999in}{2.174056in}}{\pgfqpoint{0.967272in}{2.181956in}}{\pgfqpoint{0.967272in}{2.190192in}}%
\pgfpathcurveto{\pgfqpoint{0.967272in}{2.198428in}}{\pgfqpoint{0.963999in}{2.206328in}}{\pgfqpoint{0.958175in}{2.212152in}}%
\pgfpathcurveto{\pgfqpoint{0.952352in}{2.217976in}}{\pgfqpoint{0.944451in}{2.221249in}}{\pgfqpoint{0.936215in}{2.221249in}}%
\pgfpathcurveto{\pgfqpoint{0.927979in}{2.221249in}}{\pgfqpoint{0.920079in}{2.217976in}}{\pgfqpoint{0.914255in}{2.212152in}}%
\pgfpathcurveto{\pgfqpoint{0.908431in}{2.206328in}}{\pgfqpoint{0.905159in}{2.198428in}}{\pgfqpoint{0.905159in}{2.190192in}}%
\pgfpathcurveto{\pgfqpoint{0.905159in}{2.181956in}}{\pgfqpoint{0.908431in}{2.174056in}}{\pgfqpoint{0.914255in}{2.168232in}}%
\pgfpathcurveto{\pgfqpoint{0.920079in}{2.162408in}}{\pgfqpoint{0.927979in}{2.159136in}}{\pgfqpoint{0.936215in}{2.159136in}}%
\pgfpathclose%
\pgfusepath{stroke,fill}%
\end{pgfscope}%
\begin{pgfscope}%
\pgfpathrectangle{\pgfqpoint{0.100000in}{0.212622in}}{\pgfqpoint{3.696000in}{3.696000in}}%
\pgfusepath{clip}%
\pgfsetbuttcap%
\pgfsetroundjoin%
\definecolor{currentfill}{rgb}{0.121569,0.466667,0.705882}%
\pgfsetfillcolor{currentfill}%
\pgfsetfillopacity{0.691838}%
\pgfsetlinewidth{1.003750pt}%
\definecolor{currentstroke}{rgb}{0.121569,0.466667,0.705882}%
\pgfsetstrokecolor{currentstroke}%
\pgfsetstrokeopacity{0.691838}%
\pgfsetdash{}{0pt}%
\pgfpathmoveto{\pgfqpoint{0.937687in}{2.159737in}}%
\pgfpathcurveto{\pgfqpoint{0.945923in}{2.159737in}}{\pgfqpoint{0.953823in}{2.163010in}}{\pgfqpoint{0.959647in}{2.168834in}}%
\pgfpathcurveto{\pgfqpoint{0.965471in}{2.174657in}}{\pgfqpoint{0.968743in}{2.182558in}}{\pgfqpoint{0.968743in}{2.190794in}}%
\pgfpathcurveto{\pgfqpoint{0.968743in}{2.199030in}}{\pgfqpoint{0.965471in}{2.206930in}}{\pgfqpoint{0.959647in}{2.212754in}}%
\pgfpathcurveto{\pgfqpoint{0.953823in}{2.218578in}}{\pgfqpoint{0.945923in}{2.221850in}}{\pgfqpoint{0.937687in}{2.221850in}}%
\pgfpathcurveto{\pgfqpoint{0.929450in}{2.221850in}}{\pgfqpoint{0.921550in}{2.218578in}}{\pgfqpoint{0.915727in}{2.212754in}}%
\pgfpathcurveto{\pgfqpoint{0.909903in}{2.206930in}}{\pgfqpoint{0.906630in}{2.199030in}}{\pgfqpoint{0.906630in}{2.190794in}}%
\pgfpathcurveto{\pgfqpoint{0.906630in}{2.182558in}}{\pgfqpoint{0.909903in}{2.174657in}}{\pgfqpoint{0.915727in}{2.168834in}}%
\pgfpathcurveto{\pgfqpoint{0.921550in}{2.163010in}}{\pgfqpoint{0.929450in}{2.159737in}}{\pgfqpoint{0.937687in}{2.159737in}}%
\pgfpathclose%
\pgfusepath{stroke,fill}%
\end{pgfscope}%
\begin{pgfscope}%
\pgfpathrectangle{\pgfqpoint{0.100000in}{0.212622in}}{\pgfqpoint{3.696000in}{3.696000in}}%
\pgfusepath{clip}%
\pgfsetbuttcap%
\pgfsetroundjoin%
\definecolor{currentfill}{rgb}{0.121569,0.466667,0.705882}%
\pgfsetfillcolor{currentfill}%
\pgfsetfillopacity{0.691952}%
\pgfsetlinewidth{1.003750pt}%
\definecolor{currentstroke}{rgb}{0.121569,0.466667,0.705882}%
\pgfsetstrokecolor{currentstroke}%
\pgfsetstrokeopacity{0.691952}%
\pgfsetdash{}{0pt}%
\pgfpathmoveto{\pgfqpoint{3.004469in}{1.803676in}}%
\pgfpathcurveto{\pgfqpoint{3.012706in}{1.803676in}}{\pgfqpoint{3.020606in}{1.806949in}}{\pgfqpoint{3.026430in}{1.812773in}}%
\pgfpathcurveto{\pgfqpoint{3.032254in}{1.818597in}}{\pgfqpoint{3.035526in}{1.826497in}}{\pgfqpoint{3.035526in}{1.834733in}}%
\pgfpathcurveto{\pgfqpoint{3.035526in}{1.842969in}}{\pgfqpoint{3.032254in}{1.850869in}}{\pgfqpoint{3.026430in}{1.856693in}}%
\pgfpathcurveto{\pgfqpoint{3.020606in}{1.862517in}}{\pgfqpoint{3.012706in}{1.865789in}}{\pgfqpoint{3.004469in}{1.865789in}}%
\pgfpathcurveto{\pgfqpoint{2.996233in}{1.865789in}}{\pgfqpoint{2.988333in}{1.862517in}}{\pgfqpoint{2.982509in}{1.856693in}}%
\pgfpathcurveto{\pgfqpoint{2.976685in}{1.850869in}}{\pgfqpoint{2.973413in}{1.842969in}}{\pgfqpoint{2.973413in}{1.834733in}}%
\pgfpathcurveto{\pgfqpoint{2.973413in}{1.826497in}}{\pgfqpoint{2.976685in}{1.818597in}}{\pgfqpoint{2.982509in}{1.812773in}}%
\pgfpathcurveto{\pgfqpoint{2.988333in}{1.806949in}}{\pgfqpoint{2.996233in}{1.803676in}}{\pgfqpoint{3.004469in}{1.803676in}}%
\pgfpathclose%
\pgfusepath{stroke,fill}%
\end{pgfscope}%
\begin{pgfscope}%
\pgfpathrectangle{\pgfqpoint{0.100000in}{0.212622in}}{\pgfqpoint{3.696000in}{3.696000in}}%
\pgfusepath{clip}%
\pgfsetbuttcap%
\pgfsetroundjoin%
\definecolor{currentfill}{rgb}{0.121569,0.466667,0.705882}%
\pgfsetfillcolor{currentfill}%
\pgfsetfillopacity{0.692219}%
\pgfsetlinewidth{1.003750pt}%
\definecolor{currentstroke}{rgb}{0.121569,0.466667,0.705882}%
\pgfsetstrokecolor{currentstroke}%
\pgfsetstrokeopacity{0.692219}%
\pgfsetdash{}{0pt}%
\pgfpathmoveto{\pgfqpoint{0.936757in}{2.159796in}}%
\pgfpathcurveto{\pgfqpoint{0.944993in}{2.159796in}}{\pgfqpoint{0.952893in}{2.163068in}}{\pgfqpoint{0.958717in}{2.168892in}}%
\pgfpathcurveto{\pgfqpoint{0.964541in}{2.174716in}}{\pgfqpoint{0.967814in}{2.182616in}}{\pgfqpoint{0.967814in}{2.190852in}}%
\pgfpathcurveto{\pgfqpoint{0.967814in}{2.199088in}}{\pgfqpoint{0.964541in}{2.206988in}}{\pgfqpoint{0.958717in}{2.212812in}}%
\pgfpathcurveto{\pgfqpoint{0.952893in}{2.218636in}}{\pgfqpoint{0.944993in}{2.221909in}}{\pgfqpoint{0.936757in}{2.221909in}}%
\pgfpathcurveto{\pgfqpoint{0.928521in}{2.221909in}}{\pgfqpoint{0.920621in}{2.218636in}}{\pgfqpoint{0.914797in}{2.212812in}}%
\pgfpathcurveto{\pgfqpoint{0.908973in}{2.206988in}}{\pgfqpoint{0.905701in}{2.199088in}}{\pgfqpoint{0.905701in}{2.190852in}}%
\pgfpathcurveto{\pgfqpoint{0.905701in}{2.182616in}}{\pgfqpoint{0.908973in}{2.174716in}}{\pgfqpoint{0.914797in}{2.168892in}}%
\pgfpathcurveto{\pgfqpoint{0.920621in}{2.163068in}}{\pgfqpoint{0.928521in}{2.159796in}}{\pgfqpoint{0.936757in}{2.159796in}}%
\pgfpathclose%
\pgfusepath{stroke,fill}%
\end{pgfscope}%
\begin{pgfscope}%
\pgfpathrectangle{\pgfqpoint{0.100000in}{0.212622in}}{\pgfqpoint{3.696000in}{3.696000in}}%
\pgfusepath{clip}%
\pgfsetbuttcap%
\pgfsetroundjoin%
\definecolor{currentfill}{rgb}{0.121569,0.466667,0.705882}%
\pgfsetfillcolor{currentfill}%
\pgfsetfillopacity{0.692926}%
\pgfsetlinewidth{1.003750pt}%
\definecolor{currentstroke}{rgb}{0.121569,0.466667,0.705882}%
\pgfsetstrokecolor{currentstroke}%
\pgfsetstrokeopacity{0.692926}%
\pgfsetdash{}{0pt}%
\pgfpathmoveto{\pgfqpoint{0.935142in}{2.159912in}}%
\pgfpathcurveto{\pgfqpoint{0.943378in}{2.159912in}}{\pgfqpoint{0.951278in}{2.163184in}}{\pgfqpoint{0.957102in}{2.169008in}}%
\pgfpathcurveto{\pgfqpoint{0.962926in}{2.174832in}}{\pgfqpoint{0.966198in}{2.182732in}}{\pgfqpoint{0.966198in}{2.190968in}}%
\pgfpathcurveto{\pgfqpoint{0.966198in}{2.199204in}}{\pgfqpoint{0.962926in}{2.207105in}}{\pgfqpoint{0.957102in}{2.212928in}}%
\pgfpathcurveto{\pgfqpoint{0.951278in}{2.218752in}}{\pgfqpoint{0.943378in}{2.222025in}}{\pgfqpoint{0.935142in}{2.222025in}}%
\pgfpathcurveto{\pgfqpoint{0.926906in}{2.222025in}}{\pgfqpoint{0.919006in}{2.218752in}}{\pgfqpoint{0.913182in}{2.212928in}}%
\pgfpathcurveto{\pgfqpoint{0.907358in}{2.207105in}}{\pgfqpoint{0.904085in}{2.199204in}}{\pgfqpoint{0.904085in}{2.190968in}}%
\pgfpathcurveto{\pgfqpoint{0.904085in}{2.182732in}}{\pgfqpoint{0.907358in}{2.174832in}}{\pgfqpoint{0.913182in}{2.169008in}}%
\pgfpathcurveto{\pgfqpoint{0.919006in}{2.163184in}}{\pgfqpoint{0.926906in}{2.159912in}}{\pgfqpoint{0.935142in}{2.159912in}}%
\pgfpathclose%
\pgfusepath{stroke,fill}%
\end{pgfscope}%
\begin{pgfscope}%
\pgfpathrectangle{\pgfqpoint{0.100000in}{0.212622in}}{\pgfqpoint{3.696000in}{3.696000in}}%
\pgfusepath{clip}%
\pgfsetbuttcap%
\pgfsetroundjoin%
\definecolor{currentfill}{rgb}{0.121569,0.466667,0.705882}%
\pgfsetfillcolor{currentfill}%
\pgfsetfillopacity{0.693227}%
\pgfsetlinewidth{1.003750pt}%
\definecolor{currentstroke}{rgb}{0.121569,0.466667,0.705882}%
\pgfsetstrokecolor{currentstroke}%
\pgfsetstrokeopacity{0.693227}%
\pgfsetdash{}{0pt}%
\pgfpathmoveto{\pgfqpoint{3.001230in}{1.804272in}}%
\pgfpathcurveto{\pgfqpoint{3.009466in}{1.804272in}}{\pgfqpoint{3.017366in}{1.807544in}}{\pgfqpoint{3.023190in}{1.813368in}}%
\pgfpathcurveto{\pgfqpoint{3.029014in}{1.819192in}}{\pgfqpoint{3.032286in}{1.827092in}}{\pgfqpoint{3.032286in}{1.835328in}}%
\pgfpathcurveto{\pgfqpoint{3.032286in}{1.843565in}}{\pgfqpoint{3.029014in}{1.851465in}}{\pgfqpoint{3.023190in}{1.857289in}}%
\pgfpathcurveto{\pgfqpoint{3.017366in}{1.863113in}}{\pgfqpoint{3.009466in}{1.866385in}}{\pgfqpoint{3.001230in}{1.866385in}}%
\pgfpathcurveto{\pgfqpoint{2.992993in}{1.866385in}}{\pgfqpoint{2.985093in}{1.863113in}}{\pgfqpoint{2.979269in}{1.857289in}}%
\pgfpathcurveto{\pgfqpoint{2.973445in}{1.851465in}}{\pgfqpoint{2.970173in}{1.843565in}}{\pgfqpoint{2.970173in}{1.835328in}}%
\pgfpathcurveto{\pgfqpoint{2.970173in}{1.827092in}}{\pgfqpoint{2.973445in}{1.819192in}}{\pgfqpoint{2.979269in}{1.813368in}}%
\pgfpathcurveto{\pgfqpoint{2.985093in}{1.807544in}}{\pgfqpoint{2.992993in}{1.804272in}}{\pgfqpoint{3.001230in}{1.804272in}}%
\pgfpathclose%
\pgfusepath{stroke,fill}%
\end{pgfscope}%
\begin{pgfscope}%
\pgfpathrectangle{\pgfqpoint{0.100000in}{0.212622in}}{\pgfqpoint{3.696000in}{3.696000in}}%
\pgfusepath{clip}%
\pgfsetbuttcap%
\pgfsetroundjoin%
\definecolor{currentfill}{rgb}{0.121569,0.466667,0.705882}%
\pgfsetfillcolor{currentfill}%
\pgfsetfillopacity{0.694316}%
\pgfsetlinewidth{1.003750pt}%
\definecolor{currentstroke}{rgb}{0.121569,0.466667,0.705882}%
\pgfsetstrokecolor{currentstroke}%
\pgfsetstrokeopacity{0.694316}%
\pgfsetdash{}{0pt}%
\pgfpathmoveto{\pgfqpoint{0.933032in}{2.160092in}}%
\pgfpathcurveto{\pgfqpoint{0.941268in}{2.160092in}}{\pgfqpoint{0.949168in}{2.163364in}}{\pgfqpoint{0.954992in}{2.169188in}}%
\pgfpathcurveto{\pgfqpoint{0.960816in}{2.175012in}}{\pgfqpoint{0.964089in}{2.182912in}}{\pgfqpoint{0.964089in}{2.191149in}}%
\pgfpathcurveto{\pgfqpoint{0.964089in}{2.199385in}}{\pgfqpoint{0.960816in}{2.207285in}}{\pgfqpoint{0.954992in}{2.213109in}}%
\pgfpathcurveto{\pgfqpoint{0.949168in}{2.218933in}}{\pgfqpoint{0.941268in}{2.222205in}}{\pgfqpoint{0.933032in}{2.222205in}}%
\pgfpathcurveto{\pgfqpoint{0.924796in}{2.222205in}}{\pgfqpoint{0.916896in}{2.218933in}}{\pgfqpoint{0.911072in}{2.213109in}}%
\pgfpathcurveto{\pgfqpoint{0.905248in}{2.207285in}}{\pgfqpoint{0.901976in}{2.199385in}}{\pgfqpoint{0.901976in}{2.191149in}}%
\pgfpathcurveto{\pgfqpoint{0.901976in}{2.182912in}}{\pgfqpoint{0.905248in}{2.175012in}}{\pgfqpoint{0.911072in}{2.169188in}}%
\pgfpathcurveto{\pgfqpoint{0.916896in}{2.163364in}}{\pgfqpoint{0.924796in}{2.160092in}}{\pgfqpoint{0.933032in}{2.160092in}}%
\pgfpathclose%
\pgfusepath{stroke,fill}%
\end{pgfscope}%
\begin{pgfscope}%
\pgfpathrectangle{\pgfqpoint{0.100000in}{0.212622in}}{\pgfqpoint{3.696000in}{3.696000in}}%
\pgfusepath{clip}%
\pgfsetbuttcap%
\pgfsetroundjoin%
\definecolor{currentfill}{rgb}{0.121569,0.466667,0.705882}%
\pgfsetfillcolor{currentfill}%
\pgfsetfillopacity{0.695003}%
\pgfsetlinewidth{1.003750pt}%
\definecolor{currentstroke}{rgb}{0.121569,0.466667,0.705882}%
\pgfsetstrokecolor{currentstroke}%
\pgfsetstrokeopacity{0.695003}%
\pgfsetdash{}{0pt}%
\pgfpathmoveto{\pgfqpoint{0.931012in}{2.160448in}}%
\pgfpathcurveto{\pgfqpoint{0.939249in}{2.160448in}}{\pgfqpoint{0.947149in}{2.163720in}}{\pgfqpoint{0.952973in}{2.169544in}}%
\pgfpathcurveto{\pgfqpoint{0.958797in}{2.175368in}}{\pgfqpoint{0.962069in}{2.183268in}}{\pgfqpoint{0.962069in}{2.191504in}}%
\pgfpathcurveto{\pgfqpoint{0.962069in}{2.199740in}}{\pgfqpoint{0.958797in}{2.207641in}}{\pgfqpoint{0.952973in}{2.213464in}}%
\pgfpathcurveto{\pgfqpoint{0.947149in}{2.219288in}}{\pgfqpoint{0.939249in}{2.222561in}}{\pgfqpoint{0.931012in}{2.222561in}}%
\pgfpathcurveto{\pgfqpoint{0.922776in}{2.222561in}}{\pgfqpoint{0.914876in}{2.219288in}}{\pgfqpoint{0.909052in}{2.213464in}}%
\pgfpathcurveto{\pgfqpoint{0.903228in}{2.207641in}}{\pgfqpoint{0.899956in}{2.199740in}}{\pgfqpoint{0.899956in}{2.191504in}}%
\pgfpathcurveto{\pgfqpoint{0.899956in}{2.183268in}}{\pgfqpoint{0.903228in}{2.175368in}}{\pgfqpoint{0.909052in}{2.169544in}}%
\pgfpathcurveto{\pgfqpoint{0.914876in}{2.163720in}}{\pgfqpoint{0.922776in}{2.160448in}}{\pgfqpoint{0.931012in}{2.160448in}}%
\pgfpathclose%
\pgfusepath{stroke,fill}%
\end{pgfscope}%
\begin{pgfscope}%
\pgfpathrectangle{\pgfqpoint{0.100000in}{0.212622in}}{\pgfqpoint{3.696000in}{3.696000in}}%
\pgfusepath{clip}%
\pgfsetbuttcap%
\pgfsetroundjoin%
\definecolor{currentfill}{rgb}{0.121569,0.466667,0.705882}%
\pgfsetfillcolor{currentfill}%
\pgfsetfillopacity{0.695606}%
\pgfsetlinewidth{1.003750pt}%
\definecolor{currentstroke}{rgb}{0.121569,0.466667,0.705882}%
\pgfsetstrokecolor{currentstroke}%
\pgfsetstrokeopacity{0.695606}%
\pgfsetdash{}{0pt}%
\pgfpathmoveto{\pgfqpoint{2.999330in}{1.804417in}}%
\pgfpathcurveto{\pgfqpoint{3.007566in}{1.804417in}}{\pgfqpoint{3.015466in}{1.807689in}}{\pgfqpoint{3.021290in}{1.813513in}}%
\pgfpathcurveto{\pgfqpoint{3.027114in}{1.819337in}}{\pgfqpoint{3.030386in}{1.827237in}}{\pgfqpoint{3.030386in}{1.835474in}}%
\pgfpathcurveto{\pgfqpoint{3.030386in}{1.843710in}}{\pgfqpoint{3.027114in}{1.851610in}}{\pgfqpoint{3.021290in}{1.857434in}}%
\pgfpathcurveto{\pgfqpoint{3.015466in}{1.863258in}}{\pgfqpoint{3.007566in}{1.866530in}}{\pgfqpoint{2.999330in}{1.866530in}}%
\pgfpathcurveto{\pgfqpoint{2.991093in}{1.866530in}}{\pgfqpoint{2.983193in}{1.863258in}}{\pgfqpoint{2.977369in}{1.857434in}}%
\pgfpathcurveto{\pgfqpoint{2.971545in}{1.851610in}}{\pgfqpoint{2.968273in}{1.843710in}}{\pgfqpoint{2.968273in}{1.835474in}}%
\pgfpathcurveto{\pgfqpoint{2.968273in}{1.827237in}}{\pgfqpoint{2.971545in}{1.819337in}}{\pgfqpoint{2.977369in}{1.813513in}}%
\pgfpathcurveto{\pgfqpoint{2.983193in}{1.807689in}}{\pgfqpoint{2.991093in}{1.804417in}}{\pgfqpoint{2.999330in}{1.804417in}}%
\pgfpathclose%
\pgfusepath{stroke,fill}%
\end{pgfscope}%
\begin{pgfscope}%
\pgfpathrectangle{\pgfqpoint{0.100000in}{0.212622in}}{\pgfqpoint{3.696000in}{3.696000in}}%
\pgfusepath{clip}%
\pgfsetbuttcap%
\pgfsetroundjoin%
\definecolor{currentfill}{rgb}{0.121569,0.466667,0.705882}%
\pgfsetfillcolor{currentfill}%
\pgfsetfillopacity{0.695759}%
\pgfsetlinewidth{1.003750pt}%
\definecolor{currentstroke}{rgb}{0.121569,0.466667,0.705882}%
\pgfsetstrokecolor{currentstroke}%
\pgfsetstrokeopacity{0.695759}%
\pgfsetdash{}{0pt}%
\pgfpathmoveto{\pgfqpoint{0.930633in}{2.160826in}}%
\pgfpathcurveto{\pgfqpoint{0.938869in}{2.160826in}}{\pgfqpoint{0.946769in}{2.164099in}}{\pgfqpoint{0.952593in}{2.169923in}}%
\pgfpathcurveto{\pgfqpoint{0.958417in}{2.175747in}}{\pgfqpoint{0.961689in}{2.183647in}}{\pgfqpoint{0.961689in}{2.191883in}}%
\pgfpathcurveto{\pgfqpoint{0.961689in}{2.200119in}}{\pgfqpoint{0.958417in}{2.208019in}}{\pgfqpoint{0.952593in}{2.213843in}}%
\pgfpathcurveto{\pgfqpoint{0.946769in}{2.219667in}}{\pgfqpoint{0.938869in}{2.222939in}}{\pgfqpoint{0.930633in}{2.222939in}}%
\pgfpathcurveto{\pgfqpoint{0.922397in}{2.222939in}}{\pgfqpoint{0.914497in}{2.219667in}}{\pgfqpoint{0.908673in}{2.213843in}}%
\pgfpathcurveto{\pgfqpoint{0.902849in}{2.208019in}}{\pgfqpoint{0.899576in}{2.200119in}}{\pgfqpoint{0.899576in}{2.191883in}}%
\pgfpathcurveto{\pgfqpoint{0.899576in}{2.183647in}}{\pgfqpoint{0.902849in}{2.175747in}}{\pgfqpoint{0.908673in}{2.169923in}}%
\pgfpathcurveto{\pgfqpoint{0.914497in}{2.164099in}}{\pgfqpoint{0.922397in}{2.160826in}}{\pgfqpoint{0.930633in}{2.160826in}}%
\pgfpathclose%
\pgfusepath{stroke,fill}%
\end{pgfscope}%
\begin{pgfscope}%
\pgfpathrectangle{\pgfqpoint{0.100000in}{0.212622in}}{\pgfqpoint{3.696000in}{3.696000in}}%
\pgfusepath{clip}%
\pgfsetbuttcap%
\pgfsetroundjoin%
\definecolor{currentfill}{rgb}{0.121569,0.466667,0.705882}%
\pgfsetfillcolor{currentfill}%
\pgfsetfillopacity{0.696316}%
\pgfsetlinewidth{1.003750pt}%
\definecolor{currentstroke}{rgb}{0.121569,0.466667,0.705882}%
\pgfsetstrokecolor{currentstroke}%
\pgfsetstrokeopacity{0.696316}%
\pgfsetdash{}{0pt}%
\pgfpathmoveto{\pgfqpoint{0.929450in}{2.160868in}}%
\pgfpathcurveto{\pgfqpoint{0.937686in}{2.160868in}}{\pgfqpoint{0.945586in}{2.164140in}}{\pgfqpoint{0.951410in}{2.169964in}}%
\pgfpathcurveto{\pgfqpoint{0.957234in}{2.175788in}}{\pgfqpoint{0.960506in}{2.183688in}}{\pgfqpoint{0.960506in}{2.191925in}}%
\pgfpathcurveto{\pgfqpoint{0.960506in}{2.200161in}}{\pgfqpoint{0.957234in}{2.208061in}}{\pgfqpoint{0.951410in}{2.213885in}}%
\pgfpathcurveto{\pgfqpoint{0.945586in}{2.219709in}}{\pgfqpoint{0.937686in}{2.222981in}}{\pgfqpoint{0.929450in}{2.222981in}}%
\pgfpathcurveto{\pgfqpoint{0.921214in}{2.222981in}}{\pgfqpoint{0.913314in}{2.219709in}}{\pgfqpoint{0.907490in}{2.213885in}}%
\pgfpathcurveto{\pgfqpoint{0.901666in}{2.208061in}}{\pgfqpoint{0.898393in}{2.200161in}}{\pgfqpoint{0.898393in}{2.191925in}}%
\pgfpathcurveto{\pgfqpoint{0.898393in}{2.183688in}}{\pgfqpoint{0.901666in}{2.175788in}}{\pgfqpoint{0.907490in}{2.169964in}}%
\pgfpathcurveto{\pgfqpoint{0.913314in}{2.164140in}}{\pgfqpoint{0.921214in}{2.160868in}}{\pgfqpoint{0.929450in}{2.160868in}}%
\pgfpathclose%
\pgfusepath{stroke,fill}%
\end{pgfscope}%
\begin{pgfscope}%
\pgfpathrectangle{\pgfqpoint{0.100000in}{0.212622in}}{\pgfqpoint{3.696000in}{3.696000in}}%
\pgfusepath{clip}%
\pgfsetbuttcap%
\pgfsetroundjoin%
\definecolor{currentfill}{rgb}{0.121569,0.466667,0.705882}%
\pgfsetfillcolor{currentfill}%
\pgfsetfillopacity{0.696816}%
\pgfsetlinewidth{1.003750pt}%
\definecolor{currentstroke}{rgb}{0.121569,0.466667,0.705882}%
\pgfsetstrokecolor{currentstroke}%
\pgfsetstrokeopacity{0.696816}%
\pgfsetdash{}{0pt}%
\pgfpathmoveto{\pgfqpoint{2.997049in}{1.804822in}}%
\pgfpathcurveto{\pgfqpoint{3.005286in}{1.804822in}}{\pgfqpoint{3.013186in}{1.808094in}}{\pgfqpoint{3.019010in}{1.813918in}}%
\pgfpathcurveto{\pgfqpoint{3.024834in}{1.819742in}}{\pgfqpoint{3.028106in}{1.827642in}}{\pgfqpoint{3.028106in}{1.835879in}}%
\pgfpathcurveto{\pgfqpoint{3.028106in}{1.844115in}}{\pgfqpoint{3.024834in}{1.852015in}}{\pgfqpoint{3.019010in}{1.857839in}}%
\pgfpathcurveto{\pgfqpoint{3.013186in}{1.863663in}}{\pgfqpoint{3.005286in}{1.866935in}}{\pgfqpoint{2.997049in}{1.866935in}}%
\pgfpathcurveto{\pgfqpoint{2.988813in}{1.866935in}}{\pgfqpoint{2.980913in}{1.863663in}}{\pgfqpoint{2.975089in}{1.857839in}}%
\pgfpathcurveto{\pgfqpoint{2.969265in}{1.852015in}}{\pgfqpoint{2.965993in}{1.844115in}}{\pgfqpoint{2.965993in}{1.835879in}}%
\pgfpathcurveto{\pgfqpoint{2.965993in}{1.827642in}}{\pgfqpoint{2.969265in}{1.819742in}}{\pgfqpoint{2.975089in}{1.813918in}}%
\pgfpathcurveto{\pgfqpoint{2.980913in}{1.808094in}}{\pgfqpoint{2.988813in}{1.804822in}}{\pgfqpoint{2.997049in}{1.804822in}}%
\pgfpathclose%
\pgfusepath{stroke,fill}%
\end{pgfscope}%
\begin{pgfscope}%
\pgfpathrectangle{\pgfqpoint{0.100000in}{0.212622in}}{\pgfqpoint{3.696000in}{3.696000in}}%
\pgfusepath{clip}%
\pgfsetbuttcap%
\pgfsetroundjoin%
\definecolor{currentfill}{rgb}{0.121569,0.466667,0.705882}%
\pgfsetfillcolor{currentfill}%
\pgfsetfillopacity{0.697315}%
\pgfsetlinewidth{1.003750pt}%
\definecolor{currentstroke}{rgb}{0.121569,0.466667,0.705882}%
\pgfsetstrokecolor{currentstroke}%
\pgfsetstrokeopacity{0.697315}%
\pgfsetdash{}{0pt}%
\pgfpathmoveto{\pgfqpoint{0.927068in}{2.161096in}}%
\pgfpathcurveto{\pgfqpoint{0.935305in}{2.161096in}}{\pgfqpoint{0.943205in}{2.164369in}}{\pgfqpoint{0.949029in}{2.170193in}}%
\pgfpathcurveto{\pgfqpoint{0.954853in}{2.176017in}}{\pgfqpoint{0.958125in}{2.183917in}}{\pgfqpoint{0.958125in}{2.192153in}}%
\pgfpathcurveto{\pgfqpoint{0.958125in}{2.200389in}}{\pgfqpoint{0.954853in}{2.208289in}}{\pgfqpoint{0.949029in}{2.214113in}}%
\pgfpathcurveto{\pgfqpoint{0.943205in}{2.219937in}}{\pgfqpoint{0.935305in}{2.223209in}}{\pgfqpoint{0.927068in}{2.223209in}}%
\pgfpathcurveto{\pgfqpoint{0.918832in}{2.223209in}}{\pgfqpoint{0.910932in}{2.219937in}}{\pgfqpoint{0.905108in}{2.214113in}}%
\pgfpathcurveto{\pgfqpoint{0.899284in}{2.208289in}}{\pgfqpoint{0.896012in}{2.200389in}}{\pgfqpoint{0.896012in}{2.192153in}}%
\pgfpathcurveto{\pgfqpoint{0.896012in}{2.183917in}}{\pgfqpoint{0.899284in}{2.176017in}}{\pgfqpoint{0.905108in}{2.170193in}}%
\pgfpathcurveto{\pgfqpoint{0.910932in}{2.164369in}}{\pgfqpoint{0.918832in}{2.161096in}}{\pgfqpoint{0.927068in}{2.161096in}}%
\pgfpathclose%
\pgfusepath{stroke,fill}%
\end{pgfscope}%
\begin{pgfscope}%
\pgfpathrectangle{\pgfqpoint{0.100000in}{0.212622in}}{\pgfqpoint{3.696000in}{3.696000in}}%
\pgfusepath{clip}%
\pgfsetbuttcap%
\pgfsetroundjoin%
\definecolor{currentfill}{rgb}{0.121569,0.466667,0.705882}%
\pgfsetfillcolor{currentfill}%
\pgfsetfillopacity{0.697439}%
\pgfsetlinewidth{1.003750pt}%
\definecolor{currentstroke}{rgb}{0.121569,0.466667,0.705882}%
\pgfsetstrokecolor{currentstroke}%
\pgfsetstrokeopacity{0.697439}%
\pgfsetdash{}{0pt}%
\pgfpathmoveto{\pgfqpoint{2.995477in}{1.805176in}}%
\pgfpathcurveto{\pgfqpoint{3.003713in}{1.805176in}}{\pgfqpoint{3.011613in}{1.808448in}}{\pgfqpoint{3.017437in}{1.814272in}}%
\pgfpathcurveto{\pgfqpoint{3.023261in}{1.820096in}}{\pgfqpoint{3.026533in}{1.827996in}}{\pgfqpoint{3.026533in}{1.836232in}}%
\pgfpathcurveto{\pgfqpoint{3.026533in}{1.844468in}}{\pgfqpoint{3.023261in}{1.852368in}}{\pgfqpoint{3.017437in}{1.858192in}}%
\pgfpathcurveto{\pgfqpoint{3.011613in}{1.864016in}}{\pgfqpoint{3.003713in}{1.867289in}}{\pgfqpoint{2.995477in}{1.867289in}}%
\pgfpathcurveto{\pgfqpoint{2.987241in}{1.867289in}}{\pgfqpoint{2.979341in}{1.864016in}}{\pgfqpoint{2.973517in}{1.858192in}}%
\pgfpathcurveto{\pgfqpoint{2.967693in}{1.852368in}}{\pgfqpoint{2.964420in}{1.844468in}}{\pgfqpoint{2.964420in}{1.836232in}}%
\pgfpathcurveto{\pgfqpoint{2.964420in}{1.827996in}}{\pgfqpoint{2.967693in}{1.820096in}}{\pgfqpoint{2.973517in}{1.814272in}}%
\pgfpathcurveto{\pgfqpoint{2.979341in}{1.808448in}}{\pgfqpoint{2.987241in}{1.805176in}}{\pgfqpoint{2.995477in}{1.805176in}}%
\pgfpathclose%
\pgfusepath{stroke,fill}%
\end{pgfscope}%
\begin{pgfscope}%
\pgfpathrectangle{\pgfqpoint{0.100000in}{0.212622in}}{\pgfqpoint{3.696000in}{3.696000in}}%
\pgfusepath{clip}%
\pgfsetbuttcap%
\pgfsetroundjoin%
\definecolor{currentfill}{rgb}{0.121569,0.466667,0.705882}%
\pgfsetfillcolor{currentfill}%
\pgfsetfillopacity{0.698281}%
\pgfsetlinewidth{1.003750pt}%
\definecolor{currentstroke}{rgb}{0.121569,0.466667,0.705882}%
\pgfsetstrokecolor{currentstroke}%
\pgfsetstrokeopacity{0.698281}%
\pgfsetdash{}{0pt}%
\pgfpathmoveto{\pgfqpoint{0.926110in}{2.161296in}}%
\pgfpathcurveto{\pgfqpoint{0.934346in}{2.161296in}}{\pgfqpoint{0.942247in}{2.164568in}}{\pgfqpoint{0.948070in}{2.170392in}}%
\pgfpathcurveto{\pgfqpoint{0.953894in}{2.176216in}}{\pgfqpoint{0.957167in}{2.184116in}}{\pgfqpoint{0.957167in}{2.192353in}}%
\pgfpathcurveto{\pgfqpoint{0.957167in}{2.200589in}}{\pgfqpoint{0.953894in}{2.208489in}}{\pgfqpoint{0.948070in}{2.214313in}}%
\pgfpathcurveto{\pgfqpoint{0.942247in}{2.220137in}}{\pgfqpoint{0.934346in}{2.223409in}}{\pgfqpoint{0.926110in}{2.223409in}}%
\pgfpathcurveto{\pgfqpoint{0.917874in}{2.223409in}}{\pgfqpoint{0.909974in}{2.220137in}}{\pgfqpoint{0.904150in}{2.214313in}}%
\pgfpathcurveto{\pgfqpoint{0.898326in}{2.208489in}}{\pgfqpoint{0.895054in}{2.200589in}}{\pgfqpoint{0.895054in}{2.192353in}}%
\pgfpathcurveto{\pgfqpoint{0.895054in}{2.184116in}}{\pgfqpoint{0.898326in}{2.176216in}}{\pgfqpoint{0.904150in}{2.170392in}}%
\pgfpathcurveto{\pgfqpoint{0.909974in}{2.164568in}}{\pgfqpoint{0.917874in}{2.161296in}}{\pgfqpoint{0.926110in}{2.161296in}}%
\pgfpathclose%
\pgfusepath{stroke,fill}%
\end{pgfscope}%
\begin{pgfscope}%
\pgfpathrectangle{\pgfqpoint{0.100000in}{0.212622in}}{\pgfqpoint{3.696000in}{3.696000in}}%
\pgfusepath{clip}%
\pgfsetbuttcap%
\pgfsetroundjoin%
\definecolor{currentfill}{rgb}{0.121569,0.466667,0.705882}%
\pgfsetfillcolor{currentfill}%
\pgfsetfillopacity{0.698733}%
\pgfsetlinewidth{1.003750pt}%
\definecolor{currentstroke}{rgb}{0.121569,0.466667,0.705882}%
\pgfsetstrokecolor{currentstroke}%
\pgfsetstrokeopacity{0.698733}%
\pgfsetdash{}{0pt}%
\pgfpathmoveto{\pgfqpoint{0.924795in}{2.161485in}}%
\pgfpathcurveto{\pgfqpoint{0.933031in}{2.161485in}}{\pgfqpoint{0.940931in}{2.164757in}}{\pgfqpoint{0.946755in}{2.170581in}}%
\pgfpathcurveto{\pgfqpoint{0.952579in}{2.176405in}}{\pgfqpoint{0.955852in}{2.184305in}}{\pgfqpoint{0.955852in}{2.192541in}}%
\pgfpathcurveto{\pgfqpoint{0.955852in}{2.200778in}}{\pgfqpoint{0.952579in}{2.208678in}}{\pgfqpoint{0.946755in}{2.214502in}}%
\pgfpathcurveto{\pgfqpoint{0.940931in}{2.220326in}}{\pgfqpoint{0.933031in}{2.223598in}}{\pgfqpoint{0.924795in}{2.223598in}}%
\pgfpathcurveto{\pgfqpoint{0.916559in}{2.223598in}}{\pgfqpoint{0.908659in}{2.220326in}}{\pgfqpoint{0.902835in}{2.214502in}}%
\pgfpathcurveto{\pgfqpoint{0.897011in}{2.208678in}}{\pgfqpoint{0.893739in}{2.200778in}}{\pgfqpoint{0.893739in}{2.192541in}}%
\pgfpathcurveto{\pgfqpoint{0.893739in}{2.184305in}}{\pgfqpoint{0.897011in}{2.176405in}}{\pgfqpoint{0.902835in}{2.170581in}}%
\pgfpathcurveto{\pgfqpoint{0.908659in}{2.164757in}}{\pgfqpoint{0.916559in}{2.161485in}}{\pgfqpoint{0.924795in}{2.161485in}}%
\pgfpathclose%
\pgfusepath{stroke,fill}%
\end{pgfscope}%
\begin{pgfscope}%
\pgfpathrectangle{\pgfqpoint{0.100000in}{0.212622in}}{\pgfqpoint{3.696000in}{3.696000in}}%
\pgfusepath{clip}%
\pgfsetbuttcap%
\pgfsetroundjoin%
\definecolor{currentfill}{rgb}{0.121569,0.466667,0.705882}%
\pgfsetfillcolor{currentfill}%
\pgfsetfillopacity{0.699074}%
\pgfsetlinewidth{1.003750pt}%
\definecolor{currentstroke}{rgb}{0.121569,0.466667,0.705882}%
\pgfsetstrokecolor{currentstroke}%
\pgfsetstrokeopacity{0.699074}%
\pgfsetdash{}{0pt}%
\pgfpathmoveto{\pgfqpoint{0.924194in}{2.161537in}}%
\pgfpathcurveto{\pgfqpoint{0.932430in}{2.161537in}}{\pgfqpoint{0.940330in}{2.164810in}}{\pgfqpoint{0.946154in}{2.170634in}}%
\pgfpathcurveto{\pgfqpoint{0.951978in}{2.176458in}}{\pgfqpoint{0.955250in}{2.184358in}}{\pgfqpoint{0.955250in}{2.192594in}}%
\pgfpathcurveto{\pgfqpoint{0.955250in}{2.200830in}}{\pgfqpoint{0.951978in}{2.208730in}}{\pgfqpoint{0.946154in}{2.214554in}}%
\pgfpathcurveto{\pgfqpoint{0.940330in}{2.220378in}}{\pgfqpoint{0.932430in}{2.223650in}}{\pgfqpoint{0.924194in}{2.223650in}}%
\pgfpathcurveto{\pgfqpoint{0.915957in}{2.223650in}}{\pgfqpoint{0.908057in}{2.220378in}}{\pgfqpoint{0.902233in}{2.214554in}}%
\pgfpathcurveto{\pgfqpoint{0.896409in}{2.208730in}}{\pgfqpoint{0.893137in}{2.200830in}}{\pgfqpoint{0.893137in}{2.192594in}}%
\pgfpathcurveto{\pgfqpoint{0.893137in}{2.184358in}}{\pgfqpoint{0.896409in}{2.176458in}}{\pgfqpoint{0.902233in}{2.170634in}}%
\pgfpathcurveto{\pgfqpoint{0.908057in}{2.164810in}}{\pgfqpoint{0.915957in}{2.161537in}}{\pgfqpoint{0.924194in}{2.161537in}}%
\pgfpathclose%
\pgfusepath{stroke,fill}%
\end{pgfscope}%
\begin{pgfscope}%
\pgfpathrectangle{\pgfqpoint{0.100000in}{0.212622in}}{\pgfqpoint{3.696000in}{3.696000in}}%
\pgfusepath{clip}%
\pgfsetbuttcap%
\pgfsetroundjoin%
\definecolor{currentfill}{rgb}{0.121569,0.466667,0.705882}%
\pgfsetfillcolor{currentfill}%
\pgfsetfillopacity{0.699104}%
\pgfsetlinewidth{1.003750pt}%
\definecolor{currentstroke}{rgb}{0.121569,0.466667,0.705882}%
\pgfsetstrokecolor{currentstroke}%
\pgfsetstrokeopacity{0.699104}%
\pgfsetdash{}{0pt}%
\pgfpathmoveto{\pgfqpoint{2.993790in}{1.805236in}}%
\pgfpathcurveto{\pgfqpoint{3.002026in}{1.805236in}}{\pgfqpoint{3.009927in}{1.808509in}}{\pgfqpoint{3.015750in}{1.814332in}}%
\pgfpathcurveto{\pgfqpoint{3.021574in}{1.820156in}}{\pgfqpoint{3.024847in}{1.828056in}}{\pgfqpoint{3.024847in}{1.836293in}}%
\pgfpathcurveto{\pgfqpoint{3.024847in}{1.844529in}}{\pgfqpoint{3.021574in}{1.852429in}}{\pgfqpoint{3.015750in}{1.858253in}}%
\pgfpathcurveto{\pgfqpoint{3.009927in}{1.864077in}}{\pgfqpoint{3.002026in}{1.867349in}}{\pgfqpoint{2.993790in}{1.867349in}}%
\pgfpathcurveto{\pgfqpoint{2.985554in}{1.867349in}}{\pgfqpoint{2.977654in}{1.864077in}}{\pgfqpoint{2.971830in}{1.858253in}}%
\pgfpathcurveto{\pgfqpoint{2.966006in}{1.852429in}}{\pgfqpoint{2.962734in}{1.844529in}}{\pgfqpoint{2.962734in}{1.836293in}}%
\pgfpathcurveto{\pgfqpoint{2.962734in}{1.828056in}}{\pgfqpoint{2.966006in}{1.820156in}}{\pgfqpoint{2.971830in}{1.814332in}}%
\pgfpathcurveto{\pgfqpoint{2.977654in}{1.808509in}}{\pgfqpoint{2.985554in}{1.805236in}}{\pgfqpoint{2.993790in}{1.805236in}}%
\pgfpathclose%
\pgfusepath{stroke,fill}%
\end{pgfscope}%
\begin{pgfscope}%
\pgfpathrectangle{\pgfqpoint{0.100000in}{0.212622in}}{\pgfqpoint{3.696000in}{3.696000in}}%
\pgfusepath{clip}%
\pgfsetbuttcap%
\pgfsetroundjoin%
\definecolor{currentfill}{rgb}{0.121569,0.466667,0.705882}%
\pgfsetfillcolor{currentfill}%
\pgfsetfillopacity{0.699653}%
\pgfsetlinewidth{1.003750pt}%
\definecolor{currentstroke}{rgb}{0.121569,0.466667,0.705882}%
\pgfsetstrokecolor{currentstroke}%
\pgfsetstrokeopacity{0.699653}%
\pgfsetdash{}{0pt}%
\pgfpathmoveto{\pgfqpoint{0.922811in}{2.161642in}}%
\pgfpathcurveto{\pgfqpoint{0.931047in}{2.161642in}}{\pgfqpoint{0.938948in}{2.164915in}}{\pgfqpoint{0.944771in}{2.170739in}}%
\pgfpathcurveto{\pgfqpoint{0.950595in}{2.176563in}}{\pgfqpoint{0.953868in}{2.184463in}}{\pgfqpoint{0.953868in}{2.192699in}}%
\pgfpathcurveto{\pgfqpoint{0.953868in}{2.200935in}}{\pgfqpoint{0.950595in}{2.208835in}}{\pgfqpoint{0.944771in}{2.214659in}}%
\pgfpathcurveto{\pgfqpoint{0.938948in}{2.220483in}}{\pgfqpoint{0.931047in}{2.223755in}}{\pgfqpoint{0.922811in}{2.223755in}}%
\pgfpathcurveto{\pgfqpoint{0.914575in}{2.223755in}}{\pgfqpoint{0.906675in}{2.220483in}}{\pgfqpoint{0.900851in}{2.214659in}}%
\pgfpathcurveto{\pgfqpoint{0.895027in}{2.208835in}}{\pgfqpoint{0.891755in}{2.200935in}}{\pgfqpoint{0.891755in}{2.192699in}}%
\pgfpathcurveto{\pgfqpoint{0.891755in}{2.184463in}}{\pgfqpoint{0.895027in}{2.176563in}}{\pgfqpoint{0.900851in}{2.170739in}}%
\pgfpathcurveto{\pgfqpoint{0.906675in}{2.164915in}}{\pgfqpoint{0.914575in}{2.161642in}}{\pgfqpoint{0.922811in}{2.161642in}}%
\pgfpathclose%
\pgfusepath{stroke,fill}%
\end{pgfscope}%
\begin{pgfscope}%
\pgfpathrectangle{\pgfqpoint{0.100000in}{0.212622in}}{\pgfqpoint{3.696000in}{3.696000in}}%
\pgfusepath{clip}%
\pgfsetbuttcap%
\pgfsetroundjoin%
\definecolor{currentfill}{rgb}{0.121569,0.466667,0.705882}%
\pgfsetfillcolor{currentfill}%
\pgfsetfillopacity{0.700718}%
\pgfsetlinewidth{1.003750pt}%
\definecolor{currentstroke}{rgb}{0.121569,0.466667,0.705882}%
\pgfsetstrokecolor{currentstroke}%
\pgfsetstrokeopacity{0.700718}%
\pgfsetdash{}{0pt}%
\pgfpathmoveto{\pgfqpoint{0.920379in}{2.161814in}}%
\pgfpathcurveto{\pgfqpoint{0.928616in}{2.161814in}}{\pgfqpoint{0.936516in}{2.165087in}}{\pgfqpoint{0.942340in}{2.170911in}}%
\pgfpathcurveto{\pgfqpoint{0.948163in}{2.176735in}}{\pgfqpoint{0.951436in}{2.184635in}}{\pgfqpoint{0.951436in}{2.192871in}}%
\pgfpathcurveto{\pgfqpoint{0.951436in}{2.201107in}}{\pgfqpoint{0.948163in}{2.209007in}}{\pgfqpoint{0.942340in}{2.214831in}}%
\pgfpathcurveto{\pgfqpoint{0.936516in}{2.220655in}}{\pgfqpoint{0.928616in}{2.223927in}}{\pgfqpoint{0.920379in}{2.223927in}}%
\pgfpathcurveto{\pgfqpoint{0.912143in}{2.223927in}}{\pgfqpoint{0.904243in}{2.220655in}}{\pgfqpoint{0.898419in}{2.214831in}}%
\pgfpathcurveto{\pgfqpoint{0.892595in}{2.209007in}}{\pgfqpoint{0.889323in}{2.201107in}}{\pgfqpoint{0.889323in}{2.192871in}}%
\pgfpathcurveto{\pgfqpoint{0.889323in}{2.184635in}}{\pgfqpoint{0.892595in}{2.176735in}}{\pgfqpoint{0.898419in}{2.170911in}}%
\pgfpathcurveto{\pgfqpoint{0.904243in}{2.165087in}}{\pgfqpoint{0.912143in}{2.161814in}}{\pgfqpoint{0.920379in}{2.161814in}}%
\pgfpathclose%
\pgfusepath{stroke,fill}%
\end{pgfscope}%
\begin{pgfscope}%
\pgfpathrectangle{\pgfqpoint{0.100000in}{0.212622in}}{\pgfqpoint{3.696000in}{3.696000in}}%
\pgfusepath{clip}%
\pgfsetbuttcap%
\pgfsetroundjoin%
\definecolor{currentfill}{rgb}{0.121569,0.466667,0.705882}%
\pgfsetfillcolor{currentfill}%
\pgfsetfillopacity{0.700841}%
\pgfsetlinewidth{1.003750pt}%
\definecolor{currentstroke}{rgb}{0.121569,0.466667,0.705882}%
\pgfsetstrokecolor{currentstroke}%
\pgfsetstrokeopacity{0.700841}%
\pgfsetdash{}{0pt}%
\pgfpathmoveto{\pgfqpoint{2.990969in}{1.805557in}}%
\pgfpathcurveto{\pgfqpoint{2.999205in}{1.805557in}}{\pgfqpoint{3.007105in}{1.808829in}}{\pgfqpoint{3.012929in}{1.814653in}}%
\pgfpathcurveto{\pgfqpoint{3.018753in}{1.820477in}}{\pgfqpoint{3.022025in}{1.828377in}}{\pgfqpoint{3.022025in}{1.836613in}}%
\pgfpathcurveto{\pgfqpoint{3.022025in}{1.844850in}}{\pgfqpoint{3.018753in}{1.852750in}}{\pgfqpoint{3.012929in}{1.858574in}}%
\pgfpathcurveto{\pgfqpoint{3.007105in}{1.864397in}}{\pgfqpoint{2.999205in}{1.867670in}}{\pgfqpoint{2.990969in}{1.867670in}}%
\pgfpathcurveto{\pgfqpoint{2.982733in}{1.867670in}}{\pgfqpoint{2.974833in}{1.864397in}}{\pgfqpoint{2.969009in}{1.858574in}}%
\pgfpathcurveto{\pgfqpoint{2.963185in}{1.852750in}}{\pgfqpoint{2.959912in}{1.844850in}}{\pgfqpoint{2.959912in}{1.836613in}}%
\pgfpathcurveto{\pgfqpoint{2.959912in}{1.828377in}}{\pgfqpoint{2.963185in}{1.820477in}}{\pgfqpoint{2.969009in}{1.814653in}}%
\pgfpathcurveto{\pgfqpoint{2.974833in}{1.808829in}}{\pgfqpoint{2.982733in}{1.805557in}}{\pgfqpoint{2.990969in}{1.805557in}}%
\pgfpathclose%
\pgfusepath{stroke,fill}%
\end{pgfscope}%
\begin{pgfscope}%
\pgfpathrectangle{\pgfqpoint{0.100000in}{0.212622in}}{\pgfqpoint{3.696000in}{3.696000in}}%
\pgfusepath{clip}%
\pgfsetbuttcap%
\pgfsetroundjoin%
\definecolor{currentfill}{rgb}{0.121569,0.466667,0.705882}%
\pgfsetfillcolor{currentfill}%
\pgfsetfillopacity{0.701718}%
\pgfsetlinewidth{1.003750pt}%
\definecolor{currentstroke}{rgb}{0.121569,0.466667,0.705882}%
\pgfsetstrokecolor{currentstroke}%
\pgfsetstrokeopacity{0.701718}%
\pgfsetdash{}{0pt}%
\pgfpathmoveto{\pgfqpoint{2.988783in}{1.805972in}}%
\pgfpathcurveto{\pgfqpoint{2.997019in}{1.805972in}}{\pgfqpoint{3.004919in}{1.809244in}}{\pgfqpoint{3.010743in}{1.815068in}}%
\pgfpathcurveto{\pgfqpoint{3.016567in}{1.820892in}}{\pgfqpoint{3.019839in}{1.828792in}}{\pgfqpoint{3.019839in}{1.837029in}}%
\pgfpathcurveto{\pgfqpoint{3.019839in}{1.845265in}}{\pgfqpoint{3.016567in}{1.853165in}}{\pgfqpoint{3.010743in}{1.858989in}}%
\pgfpathcurveto{\pgfqpoint{3.004919in}{1.864813in}}{\pgfqpoint{2.997019in}{1.868085in}}{\pgfqpoint{2.988783in}{1.868085in}}%
\pgfpathcurveto{\pgfqpoint{2.980546in}{1.868085in}}{\pgfqpoint{2.972646in}{1.864813in}}{\pgfqpoint{2.966822in}{1.858989in}}%
\pgfpathcurveto{\pgfqpoint{2.960999in}{1.853165in}}{\pgfqpoint{2.957726in}{1.845265in}}{\pgfqpoint{2.957726in}{1.837029in}}%
\pgfpathcurveto{\pgfqpoint{2.957726in}{1.828792in}}{\pgfqpoint{2.960999in}{1.820892in}}{\pgfqpoint{2.966822in}{1.815068in}}%
\pgfpathcurveto{\pgfqpoint{2.972646in}{1.809244in}}{\pgfqpoint{2.980546in}{1.805972in}}{\pgfqpoint{2.988783in}{1.805972in}}%
\pgfpathclose%
\pgfusepath{stroke,fill}%
\end{pgfscope}%
\begin{pgfscope}%
\pgfpathrectangle{\pgfqpoint{0.100000in}{0.212622in}}{\pgfqpoint{3.696000in}{3.696000in}}%
\pgfusepath{clip}%
\pgfsetbuttcap%
\pgfsetroundjoin%
\definecolor{currentfill}{rgb}{0.121569,0.466667,0.705882}%
\pgfsetfillcolor{currentfill}%
\pgfsetfillopacity{0.702750}%
\pgfsetlinewidth{1.003750pt}%
\definecolor{currentstroke}{rgb}{0.121569,0.466667,0.705882}%
\pgfsetstrokecolor{currentstroke}%
\pgfsetstrokeopacity{0.702750}%
\pgfsetdash{}{0pt}%
\pgfpathmoveto{\pgfqpoint{0.916687in}{2.162063in}}%
\pgfpathcurveto{\pgfqpoint{0.924924in}{2.162063in}}{\pgfqpoint{0.932824in}{2.165335in}}{\pgfqpoint{0.938648in}{2.171159in}}%
\pgfpathcurveto{\pgfqpoint{0.944471in}{2.176983in}}{\pgfqpoint{0.947744in}{2.184883in}}{\pgfqpoint{0.947744in}{2.193119in}}%
\pgfpathcurveto{\pgfqpoint{0.947744in}{2.201356in}}{\pgfqpoint{0.944471in}{2.209256in}}{\pgfqpoint{0.938648in}{2.215080in}}%
\pgfpathcurveto{\pgfqpoint{0.932824in}{2.220904in}}{\pgfqpoint{0.924924in}{2.224176in}}{\pgfqpoint{0.916687in}{2.224176in}}%
\pgfpathcurveto{\pgfqpoint{0.908451in}{2.224176in}}{\pgfqpoint{0.900551in}{2.220904in}}{\pgfqpoint{0.894727in}{2.215080in}}%
\pgfpathcurveto{\pgfqpoint{0.888903in}{2.209256in}}{\pgfqpoint{0.885631in}{2.201356in}}{\pgfqpoint{0.885631in}{2.193119in}}%
\pgfpathcurveto{\pgfqpoint{0.885631in}{2.184883in}}{\pgfqpoint{0.888903in}{2.176983in}}{\pgfqpoint{0.894727in}{2.171159in}}%
\pgfpathcurveto{\pgfqpoint{0.900551in}{2.165335in}}{\pgfqpoint{0.908451in}{2.162063in}}{\pgfqpoint{0.916687in}{2.162063in}}%
\pgfpathclose%
\pgfusepath{stroke,fill}%
\end{pgfscope}%
\begin{pgfscope}%
\pgfpathrectangle{\pgfqpoint{0.100000in}{0.212622in}}{\pgfqpoint{3.696000in}{3.696000in}}%
\pgfusepath{clip}%
\pgfsetbuttcap%
\pgfsetroundjoin%
\definecolor{currentfill}{rgb}{0.121569,0.466667,0.705882}%
\pgfsetfillcolor{currentfill}%
\pgfsetfillopacity{0.703307}%
\pgfsetlinewidth{1.003750pt}%
\definecolor{currentstroke}{rgb}{0.121569,0.466667,0.705882}%
\pgfsetstrokecolor{currentstroke}%
\pgfsetstrokeopacity{0.703307}%
\pgfsetdash{}{0pt}%
\pgfpathmoveto{\pgfqpoint{2.985729in}{1.806256in}}%
\pgfpathcurveto{\pgfqpoint{2.993965in}{1.806256in}}{\pgfqpoint{3.001865in}{1.809528in}}{\pgfqpoint{3.007689in}{1.815352in}}%
\pgfpathcurveto{\pgfqpoint{3.013513in}{1.821176in}}{\pgfqpoint{3.016785in}{1.829076in}}{\pgfqpoint{3.016785in}{1.837312in}}%
\pgfpathcurveto{\pgfqpoint{3.016785in}{1.845548in}}{\pgfqpoint{3.013513in}{1.853449in}}{\pgfqpoint{3.007689in}{1.859272in}}%
\pgfpathcurveto{\pgfqpoint{3.001865in}{1.865096in}}{\pgfqpoint{2.993965in}{1.868369in}}{\pgfqpoint{2.985729in}{1.868369in}}%
\pgfpathcurveto{\pgfqpoint{2.977493in}{1.868369in}}{\pgfqpoint{2.969593in}{1.865096in}}{\pgfqpoint{2.963769in}{1.859272in}}%
\pgfpathcurveto{\pgfqpoint{2.957945in}{1.853449in}}{\pgfqpoint{2.954672in}{1.845548in}}{\pgfqpoint{2.954672in}{1.837312in}}%
\pgfpathcurveto{\pgfqpoint{2.954672in}{1.829076in}}{\pgfqpoint{2.957945in}{1.821176in}}{\pgfqpoint{2.963769in}{1.815352in}}%
\pgfpathcurveto{\pgfqpoint{2.969593in}{1.809528in}}{\pgfqpoint{2.977493in}{1.806256in}}{\pgfqpoint{2.985729in}{1.806256in}}%
\pgfpathclose%
\pgfusepath{stroke,fill}%
\end{pgfscope}%
\begin{pgfscope}%
\pgfpathrectangle{\pgfqpoint{0.100000in}{0.212622in}}{\pgfqpoint{3.696000in}{3.696000in}}%
\pgfusepath{clip}%
\pgfsetbuttcap%
\pgfsetroundjoin%
\definecolor{currentfill}{rgb}{0.121569,0.466667,0.705882}%
\pgfsetfillcolor{currentfill}%
\pgfsetfillopacity{0.703997}%
\pgfsetlinewidth{1.003750pt}%
\definecolor{currentstroke}{rgb}{0.121569,0.466667,0.705882}%
\pgfsetstrokecolor{currentstroke}%
\pgfsetstrokeopacity{0.703997}%
\pgfsetdash{}{0pt}%
\pgfpathmoveto{\pgfqpoint{0.913261in}{2.162416in}}%
\pgfpathcurveto{\pgfqpoint{0.921498in}{2.162416in}}{\pgfqpoint{0.929398in}{2.165688in}}{\pgfqpoint{0.935222in}{2.171512in}}%
\pgfpathcurveto{\pgfqpoint{0.941046in}{2.177336in}}{\pgfqpoint{0.944318in}{2.185236in}}{\pgfqpoint{0.944318in}{2.193473in}}%
\pgfpathcurveto{\pgfqpoint{0.944318in}{2.201709in}}{\pgfqpoint{0.941046in}{2.209609in}}{\pgfqpoint{0.935222in}{2.215433in}}%
\pgfpathcurveto{\pgfqpoint{0.929398in}{2.221257in}}{\pgfqpoint{0.921498in}{2.224529in}}{\pgfqpoint{0.913261in}{2.224529in}}%
\pgfpathcurveto{\pgfqpoint{0.905025in}{2.224529in}}{\pgfqpoint{0.897125in}{2.221257in}}{\pgfqpoint{0.891301in}{2.215433in}}%
\pgfpathcurveto{\pgfqpoint{0.885477in}{2.209609in}}{\pgfqpoint{0.882205in}{2.201709in}}{\pgfqpoint{0.882205in}{2.193473in}}%
\pgfpathcurveto{\pgfqpoint{0.882205in}{2.185236in}}{\pgfqpoint{0.885477in}{2.177336in}}{\pgfqpoint{0.891301in}{2.171512in}}%
\pgfpathcurveto{\pgfqpoint{0.897125in}{2.165688in}}{\pgfqpoint{0.905025in}{2.162416in}}{\pgfqpoint{0.913261in}{2.162416in}}%
\pgfpathclose%
\pgfusepath{stroke,fill}%
\end{pgfscope}%
\begin{pgfscope}%
\pgfpathrectangle{\pgfqpoint{0.100000in}{0.212622in}}{\pgfqpoint{3.696000in}{3.696000in}}%
\pgfusepath{clip}%
\pgfsetbuttcap%
\pgfsetroundjoin%
\definecolor{currentfill}{rgb}{0.121569,0.466667,0.705882}%
\pgfsetfillcolor{currentfill}%
\pgfsetfillopacity{0.704249}%
\pgfsetlinewidth{1.003750pt}%
\definecolor{currentstroke}{rgb}{0.121569,0.466667,0.705882}%
\pgfsetstrokecolor{currentstroke}%
\pgfsetstrokeopacity{0.704249}%
\pgfsetdash{}{0pt}%
\pgfpathmoveto{\pgfqpoint{2.984697in}{1.806298in}}%
\pgfpathcurveto{\pgfqpoint{2.992933in}{1.806298in}}{\pgfqpoint{3.000833in}{1.809570in}}{\pgfqpoint{3.006657in}{1.815394in}}%
\pgfpathcurveto{\pgfqpoint{3.012481in}{1.821218in}}{\pgfqpoint{3.015753in}{1.829118in}}{\pgfqpoint{3.015753in}{1.837355in}}%
\pgfpathcurveto{\pgfqpoint{3.015753in}{1.845591in}}{\pgfqpoint{3.012481in}{1.853491in}}{\pgfqpoint{3.006657in}{1.859315in}}%
\pgfpathcurveto{\pgfqpoint{3.000833in}{1.865139in}}{\pgfqpoint{2.992933in}{1.868411in}}{\pgfqpoint{2.984697in}{1.868411in}}%
\pgfpathcurveto{\pgfqpoint{2.976460in}{1.868411in}}{\pgfqpoint{2.968560in}{1.865139in}}{\pgfqpoint{2.962736in}{1.859315in}}%
\pgfpathcurveto{\pgfqpoint{2.956913in}{1.853491in}}{\pgfqpoint{2.953640in}{1.845591in}}{\pgfqpoint{2.953640in}{1.837355in}}%
\pgfpathcurveto{\pgfqpoint{2.953640in}{1.829118in}}{\pgfqpoint{2.956913in}{1.821218in}}{\pgfqpoint{2.962736in}{1.815394in}}%
\pgfpathcurveto{\pgfqpoint{2.968560in}{1.809570in}}{\pgfqpoint{2.976460in}{1.806298in}}{\pgfqpoint{2.984697in}{1.806298in}}%
\pgfpathclose%
\pgfusepath{stroke,fill}%
\end{pgfscope}%
\begin{pgfscope}%
\pgfpathrectangle{\pgfqpoint{0.100000in}{0.212622in}}{\pgfqpoint{3.696000in}{3.696000in}}%
\pgfusepath{clip}%
\pgfsetbuttcap%
\pgfsetroundjoin%
\definecolor{currentfill}{rgb}{0.121569,0.466667,0.705882}%
\pgfsetfillcolor{currentfill}%
\pgfsetfillopacity{0.705301}%
\pgfsetlinewidth{1.003750pt}%
\definecolor{currentstroke}{rgb}{0.121569,0.466667,0.705882}%
\pgfsetstrokecolor{currentstroke}%
\pgfsetstrokeopacity{0.705301}%
\pgfsetdash{}{0pt}%
\pgfpathmoveto{\pgfqpoint{0.911414in}{2.162591in}}%
\pgfpathcurveto{\pgfqpoint{0.919651in}{2.162591in}}{\pgfqpoint{0.927551in}{2.165863in}}{\pgfqpoint{0.933374in}{2.171687in}}%
\pgfpathcurveto{\pgfqpoint{0.939198in}{2.177511in}}{\pgfqpoint{0.942471in}{2.185411in}}{\pgfqpoint{0.942471in}{2.193647in}}%
\pgfpathcurveto{\pgfqpoint{0.942471in}{2.201883in}}{\pgfqpoint{0.939198in}{2.209784in}}{\pgfqpoint{0.933374in}{2.215607in}}%
\pgfpathcurveto{\pgfqpoint{0.927551in}{2.221431in}}{\pgfqpoint{0.919651in}{2.224704in}}{\pgfqpoint{0.911414in}{2.224704in}}%
\pgfpathcurveto{\pgfqpoint{0.903178in}{2.224704in}}{\pgfqpoint{0.895278in}{2.221431in}}{\pgfqpoint{0.889454in}{2.215607in}}%
\pgfpathcurveto{\pgfqpoint{0.883630in}{2.209784in}}{\pgfqpoint{0.880358in}{2.201883in}}{\pgfqpoint{0.880358in}{2.193647in}}%
\pgfpathcurveto{\pgfqpoint{0.880358in}{2.185411in}}{\pgfqpoint{0.883630in}{2.177511in}}{\pgfqpoint{0.889454in}{2.171687in}}%
\pgfpathcurveto{\pgfqpoint{0.895278in}{2.165863in}}{\pgfqpoint{0.903178in}{2.162591in}}{\pgfqpoint{0.911414in}{2.162591in}}%
\pgfpathclose%
\pgfusepath{stroke,fill}%
\end{pgfscope}%
\begin{pgfscope}%
\pgfpathrectangle{\pgfqpoint{0.100000in}{0.212622in}}{\pgfqpoint{3.696000in}{3.696000in}}%
\pgfusepath{clip}%
\pgfsetbuttcap%
\pgfsetroundjoin%
\definecolor{currentfill}{rgb}{0.121569,0.466667,0.705882}%
\pgfsetfillcolor{currentfill}%
\pgfsetfillopacity{0.705358}%
\pgfsetlinewidth{1.003750pt}%
\definecolor{currentstroke}{rgb}{0.121569,0.466667,0.705882}%
\pgfsetstrokecolor{currentstroke}%
\pgfsetstrokeopacity{0.705358}%
\pgfsetdash{}{0pt}%
\pgfpathmoveto{\pgfqpoint{2.982187in}{1.806707in}}%
\pgfpathcurveto{\pgfqpoint{2.990424in}{1.806707in}}{\pgfqpoint{2.998324in}{1.809980in}}{\pgfqpoint{3.004148in}{1.815803in}}%
\pgfpathcurveto{\pgfqpoint{3.009971in}{1.821627in}}{\pgfqpoint{3.013244in}{1.829527in}}{\pgfqpoint{3.013244in}{1.837764in}}%
\pgfpathcurveto{\pgfqpoint{3.013244in}{1.846000in}}{\pgfqpoint{3.009971in}{1.853900in}}{\pgfqpoint{3.004148in}{1.859724in}}%
\pgfpathcurveto{\pgfqpoint{2.998324in}{1.865548in}}{\pgfqpoint{2.990424in}{1.868820in}}{\pgfqpoint{2.982187in}{1.868820in}}%
\pgfpathcurveto{\pgfqpoint{2.973951in}{1.868820in}}{\pgfqpoint{2.966051in}{1.865548in}}{\pgfqpoint{2.960227in}{1.859724in}}%
\pgfpathcurveto{\pgfqpoint{2.954403in}{1.853900in}}{\pgfqpoint{2.951131in}{1.846000in}}{\pgfqpoint{2.951131in}{1.837764in}}%
\pgfpathcurveto{\pgfqpoint{2.951131in}{1.829527in}}{\pgfqpoint{2.954403in}{1.821627in}}{\pgfqpoint{2.960227in}{1.815803in}}%
\pgfpathcurveto{\pgfqpoint{2.966051in}{1.809980in}}{\pgfqpoint{2.973951in}{1.806707in}}{\pgfqpoint{2.982187in}{1.806707in}}%
\pgfpathclose%
\pgfusepath{stroke,fill}%
\end{pgfscope}%
\begin{pgfscope}%
\pgfpathrectangle{\pgfqpoint{0.100000in}{0.212622in}}{\pgfqpoint{3.696000in}{3.696000in}}%
\pgfusepath{clip}%
\pgfsetbuttcap%
\pgfsetroundjoin%
\definecolor{currentfill}{rgb}{0.121569,0.466667,0.705882}%
\pgfsetfillcolor{currentfill}%
\pgfsetfillopacity{0.705962}%
\pgfsetlinewidth{1.003750pt}%
\definecolor{currentstroke}{rgb}{0.121569,0.466667,0.705882}%
\pgfsetstrokecolor{currentstroke}%
\pgfsetstrokeopacity{0.705962}%
\pgfsetdash{}{0pt}%
\pgfpathmoveto{\pgfqpoint{2.980706in}{1.807031in}}%
\pgfpathcurveto{\pgfqpoint{2.988942in}{1.807031in}}{\pgfqpoint{2.996842in}{1.810303in}}{\pgfqpoint{3.002666in}{1.816127in}}%
\pgfpathcurveto{\pgfqpoint{3.008490in}{1.821951in}}{\pgfqpoint{3.011762in}{1.829851in}}{\pgfqpoint{3.011762in}{1.838088in}}%
\pgfpathcurveto{\pgfqpoint{3.011762in}{1.846324in}}{\pgfqpoint{3.008490in}{1.854224in}}{\pgfqpoint{3.002666in}{1.860048in}}%
\pgfpathcurveto{\pgfqpoint{2.996842in}{1.865872in}}{\pgfqpoint{2.988942in}{1.869144in}}{\pgfqpoint{2.980706in}{1.869144in}}%
\pgfpathcurveto{\pgfqpoint{2.972469in}{1.869144in}}{\pgfqpoint{2.964569in}{1.865872in}}{\pgfqpoint{2.958745in}{1.860048in}}%
\pgfpathcurveto{\pgfqpoint{2.952922in}{1.854224in}}{\pgfqpoint{2.949649in}{1.846324in}}{\pgfqpoint{2.949649in}{1.838088in}}%
\pgfpathcurveto{\pgfqpoint{2.949649in}{1.829851in}}{\pgfqpoint{2.952922in}{1.821951in}}{\pgfqpoint{2.958745in}{1.816127in}}%
\pgfpathcurveto{\pgfqpoint{2.964569in}{1.810303in}}{\pgfqpoint{2.972469in}{1.807031in}}{\pgfqpoint{2.980706in}{1.807031in}}%
\pgfpathclose%
\pgfusepath{stroke,fill}%
\end{pgfscope}%
\begin{pgfscope}%
\pgfpathrectangle{\pgfqpoint{0.100000in}{0.212622in}}{\pgfqpoint{3.696000in}{3.696000in}}%
\pgfusepath{clip}%
\pgfsetbuttcap%
\pgfsetroundjoin%
\definecolor{currentfill}{rgb}{0.121569,0.466667,0.705882}%
\pgfsetfillcolor{currentfill}%
\pgfsetfillopacity{0.705989}%
\pgfsetlinewidth{1.003750pt}%
\definecolor{currentstroke}{rgb}{0.121569,0.466667,0.705882}%
\pgfsetstrokecolor{currentstroke}%
\pgfsetstrokeopacity{0.705989}%
\pgfsetdash{}{0pt}%
\pgfpathmoveto{\pgfqpoint{0.909318in}{2.162882in}}%
\pgfpathcurveto{\pgfqpoint{0.917555in}{2.162882in}}{\pgfqpoint{0.925455in}{2.166154in}}{\pgfqpoint{0.931279in}{2.171978in}}%
\pgfpathcurveto{\pgfqpoint{0.937102in}{2.177802in}}{\pgfqpoint{0.940375in}{2.185702in}}{\pgfqpoint{0.940375in}{2.193938in}}%
\pgfpathcurveto{\pgfqpoint{0.940375in}{2.202174in}}{\pgfqpoint{0.937102in}{2.210074in}}{\pgfqpoint{0.931279in}{2.215898in}}%
\pgfpathcurveto{\pgfqpoint{0.925455in}{2.221722in}}{\pgfqpoint{0.917555in}{2.224995in}}{\pgfqpoint{0.909318in}{2.224995in}}%
\pgfpathcurveto{\pgfqpoint{0.901082in}{2.224995in}}{\pgfqpoint{0.893182in}{2.221722in}}{\pgfqpoint{0.887358in}{2.215898in}}%
\pgfpathcurveto{\pgfqpoint{0.881534in}{2.210074in}}{\pgfqpoint{0.878262in}{2.202174in}}{\pgfqpoint{0.878262in}{2.193938in}}%
\pgfpathcurveto{\pgfqpoint{0.878262in}{2.185702in}}{\pgfqpoint{0.881534in}{2.177802in}}{\pgfqpoint{0.887358in}{2.171978in}}%
\pgfpathcurveto{\pgfqpoint{0.893182in}{2.166154in}}{\pgfqpoint{0.901082in}{2.162882in}}{\pgfqpoint{0.909318in}{2.162882in}}%
\pgfpathclose%
\pgfusepath{stroke,fill}%
\end{pgfscope}%
\begin{pgfscope}%
\pgfpathrectangle{\pgfqpoint{0.100000in}{0.212622in}}{\pgfqpoint{3.696000in}{3.696000in}}%
\pgfusepath{clip}%
\pgfsetbuttcap%
\pgfsetroundjoin%
\definecolor{currentfill}{rgb}{0.121569,0.466667,0.705882}%
\pgfsetfillcolor{currentfill}%
\pgfsetfillopacity{0.706635}%
\pgfsetlinewidth{1.003750pt}%
\definecolor{currentstroke}{rgb}{0.121569,0.466667,0.705882}%
\pgfsetstrokecolor{currentstroke}%
\pgfsetstrokeopacity{0.706635}%
\pgfsetdash{}{0pt}%
\pgfpathmoveto{\pgfqpoint{0.908504in}{2.162876in}}%
\pgfpathcurveto{\pgfqpoint{0.916741in}{2.162876in}}{\pgfqpoint{0.924641in}{2.166148in}}{\pgfqpoint{0.930465in}{2.171972in}}%
\pgfpathcurveto{\pgfqpoint{0.936289in}{2.177796in}}{\pgfqpoint{0.939561in}{2.185696in}}{\pgfqpoint{0.939561in}{2.193932in}}%
\pgfpathcurveto{\pgfqpoint{0.939561in}{2.202168in}}{\pgfqpoint{0.936289in}{2.210069in}}{\pgfqpoint{0.930465in}{2.215892in}}%
\pgfpathcurveto{\pgfqpoint{0.924641in}{2.221716in}}{\pgfqpoint{0.916741in}{2.224989in}}{\pgfqpoint{0.908504in}{2.224989in}}%
\pgfpathcurveto{\pgfqpoint{0.900268in}{2.224989in}}{\pgfqpoint{0.892368in}{2.221716in}}{\pgfqpoint{0.886544in}{2.215892in}}%
\pgfpathcurveto{\pgfqpoint{0.880720in}{2.210069in}}{\pgfqpoint{0.877448in}{2.202168in}}{\pgfqpoint{0.877448in}{2.193932in}}%
\pgfpathcurveto{\pgfqpoint{0.877448in}{2.185696in}}{\pgfqpoint{0.880720in}{2.177796in}}{\pgfqpoint{0.886544in}{2.171972in}}%
\pgfpathcurveto{\pgfqpoint{0.892368in}{2.166148in}}{\pgfqpoint{0.900268in}{2.162876in}}{\pgfqpoint{0.908504in}{2.162876in}}%
\pgfpathclose%
\pgfusepath{stroke,fill}%
\end{pgfscope}%
\begin{pgfscope}%
\pgfpathrectangle{\pgfqpoint{0.100000in}{0.212622in}}{\pgfqpoint{3.696000in}{3.696000in}}%
\pgfusepath{clip}%
\pgfsetbuttcap%
\pgfsetroundjoin%
\definecolor{currentfill}{rgb}{0.121569,0.466667,0.705882}%
\pgfsetfillcolor{currentfill}%
\pgfsetfillopacity{0.707022}%
\pgfsetlinewidth{1.003750pt}%
\definecolor{currentstroke}{rgb}{0.121569,0.466667,0.705882}%
\pgfsetstrokecolor{currentstroke}%
\pgfsetstrokeopacity{0.707022}%
\pgfsetdash{}{0pt}%
\pgfpathmoveto{\pgfqpoint{2.979716in}{1.807149in}}%
\pgfpathcurveto{\pgfqpoint{2.987952in}{1.807149in}}{\pgfqpoint{2.995852in}{1.810422in}}{\pgfqpoint{3.001676in}{1.816245in}}%
\pgfpathcurveto{\pgfqpoint{3.007500in}{1.822069in}}{\pgfqpoint{3.010773in}{1.829969in}}{\pgfqpoint{3.010773in}{1.838206in}}%
\pgfpathcurveto{\pgfqpoint{3.010773in}{1.846442in}}{\pgfqpoint{3.007500in}{1.854342in}}{\pgfqpoint{3.001676in}{1.860166in}}%
\pgfpathcurveto{\pgfqpoint{2.995852in}{1.865990in}}{\pgfqpoint{2.987952in}{1.869262in}}{\pgfqpoint{2.979716in}{1.869262in}}%
\pgfpathcurveto{\pgfqpoint{2.971480in}{1.869262in}}{\pgfqpoint{2.963580in}{1.865990in}}{\pgfqpoint{2.957756in}{1.860166in}}%
\pgfpathcurveto{\pgfqpoint{2.951932in}{1.854342in}}{\pgfqpoint{2.948660in}{1.846442in}}{\pgfqpoint{2.948660in}{1.838206in}}%
\pgfpathcurveto{\pgfqpoint{2.948660in}{1.829969in}}{\pgfqpoint{2.951932in}{1.822069in}}{\pgfqpoint{2.957756in}{1.816245in}}%
\pgfpathcurveto{\pgfqpoint{2.963580in}{1.810422in}}{\pgfqpoint{2.971480in}{1.807149in}}{\pgfqpoint{2.979716in}{1.807149in}}%
\pgfpathclose%
\pgfusepath{stroke,fill}%
\end{pgfscope}%
\begin{pgfscope}%
\pgfpathrectangle{\pgfqpoint{0.100000in}{0.212622in}}{\pgfqpoint{3.696000in}{3.696000in}}%
\pgfusepath{clip}%
\pgfsetbuttcap%
\pgfsetroundjoin%
\definecolor{currentfill}{rgb}{0.121569,0.466667,0.705882}%
\pgfsetfillcolor{currentfill}%
\pgfsetfillopacity{0.707727}%
\pgfsetlinewidth{1.003750pt}%
\definecolor{currentstroke}{rgb}{0.121569,0.466667,0.705882}%
\pgfsetstrokecolor{currentstroke}%
\pgfsetstrokeopacity{0.707727}%
\pgfsetdash{}{0pt}%
\pgfpathmoveto{\pgfqpoint{0.906037in}{2.163083in}}%
\pgfpathcurveto{\pgfqpoint{0.914273in}{2.163083in}}{\pgfqpoint{0.922173in}{2.166356in}}{\pgfqpoint{0.927997in}{2.172180in}}%
\pgfpathcurveto{\pgfqpoint{0.933821in}{2.178003in}}{\pgfqpoint{0.937093in}{2.185904in}}{\pgfqpoint{0.937093in}{2.194140in}}%
\pgfpathcurveto{\pgfqpoint{0.937093in}{2.202376in}}{\pgfqpoint{0.933821in}{2.210276in}}{\pgfqpoint{0.927997in}{2.216100in}}%
\pgfpathcurveto{\pgfqpoint{0.922173in}{2.221924in}}{\pgfqpoint{0.914273in}{2.225196in}}{\pgfqpoint{0.906037in}{2.225196in}}%
\pgfpathcurveto{\pgfqpoint{0.897800in}{2.225196in}}{\pgfqpoint{0.889900in}{2.221924in}}{\pgfqpoint{0.884076in}{2.216100in}}%
\pgfpathcurveto{\pgfqpoint{0.878252in}{2.210276in}}{\pgfqpoint{0.874980in}{2.202376in}}{\pgfqpoint{0.874980in}{2.194140in}}%
\pgfpathcurveto{\pgfqpoint{0.874980in}{2.185904in}}{\pgfqpoint{0.878252in}{2.178003in}}{\pgfqpoint{0.884076in}{2.172180in}}%
\pgfpathcurveto{\pgfqpoint{0.889900in}{2.166356in}}{\pgfqpoint{0.897800in}{2.163083in}}{\pgfqpoint{0.906037in}{2.163083in}}%
\pgfpathclose%
\pgfusepath{stroke,fill}%
\end{pgfscope}%
\begin{pgfscope}%
\pgfpathrectangle{\pgfqpoint{0.100000in}{0.212622in}}{\pgfqpoint{3.696000in}{3.696000in}}%
\pgfusepath{clip}%
\pgfsetbuttcap%
\pgfsetroundjoin%
\definecolor{currentfill}{rgb}{0.121569,0.466667,0.705882}%
\pgfsetfillcolor{currentfill}%
\pgfsetfillopacity{0.708322}%
\pgfsetlinewidth{1.003750pt}%
\definecolor{currentstroke}{rgb}{0.121569,0.466667,0.705882}%
\pgfsetstrokecolor{currentstroke}%
\pgfsetstrokeopacity{0.708322}%
\pgfsetdash{}{0pt}%
\pgfpathmoveto{\pgfqpoint{2.977292in}{1.807421in}}%
\pgfpathcurveto{\pgfqpoint{2.985529in}{1.807421in}}{\pgfqpoint{2.993429in}{1.810694in}}{\pgfqpoint{2.999252in}{1.816518in}}%
\pgfpathcurveto{\pgfqpoint{3.005076in}{1.822342in}}{\pgfqpoint{3.008349in}{1.830242in}}{\pgfqpoint{3.008349in}{1.838478in}}%
\pgfpathcurveto{\pgfqpoint{3.008349in}{1.846714in}}{\pgfqpoint{3.005076in}{1.854614in}}{\pgfqpoint{2.999252in}{1.860438in}}%
\pgfpathcurveto{\pgfqpoint{2.993429in}{1.866262in}}{\pgfqpoint{2.985529in}{1.869534in}}{\pgfqpoint{2.977292in}{1.869534in}}%
\pgfpathcurveto{\pgfqpoint{2.969056in}{1.869534in}}{\pgfqpoint{2.961156in}{1.866262in}}{\pgfqpoint{2.955332in}{1.860438in}}%
\pgfpathcurveto{\pgfqpoint{2.949508in}{1.854614in}}{\pgfqpoint{2.946236in}{1.846714in}}{\pgfqpoint{2.946236in}{1.838478in}}%
\pgfpathcurveto{\pgfqpoint{2.946236in}{1.830242in}}{\pgfqpoint{2.949508in}{1.822342in}}{\pgfqpoint{2.955332in}{1.816518in}}%
\pgfpathcurveto{\pgfqpoint{2.961156in}{1.810694in}}{\pgfqpoint{2.969056in}{1.807421in}}{\pgfqpoint{2.977292in}{1.807421in}}%
\pgfpathclose%
\pgfusepath{stroke,fill}%
\end{pgfscope}%
\begin{pgfscope}%
\pgfpathrectangle{\pgfqpoint{0.100000in}{0.212622in}}{\pgfqpoint{3.696000in}{3.696000in}}%
\pgfusepath{clip}%
\pgfsetbuttcap%
\pgfsetroundjoin%
\definecolor{currentfill}{rgb}{0.121569,0.466667,0.705882}%
\pgfsetfillcolor{currentfill}%
\pgfsetfillopacity{0.708719}%
\pgfsetlinewidth{1.003750pt}%
\definecolor{currentstroke}{rgb}{0.121569,0.466667,0.705882}%
\pgfsetstrokecolor{currentstroke}%
\pgfsetstrokeopacity{0.708719}%
\pgfsetdash{}{0pt}%
\pgfpathmoveto{\pgfqpoint{0.903832in}{2.163179in}}%
\pgfpathcurveto{\pgfqpoint{0.912069in}{2.163179in}}{\pgfqpoint{0.919969in}{2.166451in}}{\pgfqpoint{0.925793in}{2.172275in}}%
\pgfpathcurveto{\pgfqpoint{0.931617in}{2.178099in}}{\pgfqpoint{0.934889in}{2.185999in}}{\pgfqpoint{0.934889in}{2.194235in}}%
\pgfpathcurveto{\pgfqpoint{0.934889in}{2.202472in}}{\pgfqpoint{0.931617in}{2.210372in}}{\pgfqpoint{0.925793in}{2.216196in}}%
\pgfpathcurveto{\pgfqpoint{0.919969in}{2.222020in}}{\pgfqpoint{0.912069in}{2.225292in}}{\pgfqpoint{0.903832in}{2.225292in}}%
\pgfpathcurveto{\pgfqpoint{0.895596in}{2.225292in}}{\pgfqpoint{0.887696in}{2.222020in}}{\pgfqpoint{0.881872in}{2.216196in}}%
\pgfpathcurveto{\pgfqpoint{0.876048in}{2.210372in}}{\pgfqpoint{0.872776in}{2.202472in}}{\pgfqpoint{0.872776in}{2.194235in}}%
\pgfpathcurveto{\pgfqpoint{0.872776in}{2.185999in}}{\pgfqpoint{0.876048in}{2.178099in}}{\pgfqpoint{0.881872in}{2.172275in}}%
\pgfpathcurveto{\pgfqpoint{0.887696in}{2.166451in}}{\pgfqpoint{0.895596in}{2.163179in}}{\pgfqpoint{0.903832in}{2.163179in}}%
\pgfpathclose%
\pgfusepath{stroke,fill}%
\end{pgfscope}%
\begin{pgfscope}%
\pgfpathrectangle{\pgfqpoint{0.100000in}{0.212622in}}{\pgfqpoint{3.696000in}{3.696000in}}%
\pgfusepath{clip}%
\pgfsetbuttcap%
\pgfsetroundjoin%
\definecolor{currentfill}{rgb}{0.121569,0.466667,0.705882}%
\pgfsetfillcolor{currentfill}%
\pgfsetfillopacity{0.709672}%
\pgfsetlinewidth{1.003750pt}%
\definecolor{currentstroke}{rgb}{0.121569,0.466667,0.705882}%
\pgfsetstrokecolor{currentstroke}%
\pgfsetstrokeopacity{0.709672}%
\pgfsetdash{}{0pt}%
\pgfpathmoveto{\pgfqpoint{2.973882in}{1.808149in}}%
\pgfpathcurveto{\pgfqpoint{2.982118in}{1.808149in}}{\pgfqpoint{2.990018in}{1.811422in}}{\pgfqpoint{2.995842in}{1.817246in}}%
\pgfpathcurveto{\pgfqpoint{3.001666in}{1.823070in}}{\pgfqpoint{3.004938in}{1.830970in}}{\pgfqpoint{3.004938in}{1.839206in}}%
\pgfpathcurveto{\pgfqpoint{3.004938in}{1.847442in}}{\pgfqpoint{3.001666in}{1.855342in}}{\pgfqpoint{2.995842in}{1.861166in}}%
\pgfpathcurveto{\pgfqpoint{2.990018in}{1.866990in}}{\pgfqpoint{2.982118in}{1.870262in}}{\pgfqpoint{2.973882in}{1.870262in}}%
\pgfpathcurveto{\pgfqpoint{2.965645in}{1.870262in}}{\pgfqpoint{2.957745in}{1.866990in}}{\pgfqpoint{2.951921in}{1.861166in}}%
\pgfpathcurveto{\pgfqpoint{2.946097in}{1.855342in}}{\pgfqpoint{2.942825in}{1.847442in}}{\pgfqpoint{2.942825in}{1.839206in}}%
\pgfpathcurveto{\pgfqpoint{2.942825in}{1.830970in}}{\pgfqpoint{2.946097in}{1.823070in}}{\pgfqpoint{2.951921in}{1.817246in}}%
\pgfpathcurveto{\pgfqpoint{2.957745in}{1.811422in}}{\pgfqpoint{2.965645in}{1.808149in}}{\pgfqpoint{2.973882in}{1.808149in}}%
\pgfpathclose%
\pgfusepath{stroke,fill}%
\end{pgfscope}%
\begin{pgfscope}%
\pgfpathrectangle{\pgfqpoint{0.100000in}{0.212622in}}{\pgfqpoint{3.696000in}{3.696000in}}%
\pgfusepath{clip}%
\pgfsetbuttcap%
\pgfsetroundjoin%
\definecolor{currentfill}{rgb}{0.121569,0.466667,0.705882}%
\pgfsetfillcolor{currentfill}%
\pgfsetfillopacity{0.710696}%
\pgfsetlinewidth{1.003750pt}%
\definecolor{currentstroke}{rgb}{0.121569,0.466667,0.705882}%
\pgfsetstrokecolor{currentstroke}%
\pgfsetstrokeopacity{0.710696}%
\pgfsetdash{}{0pt}%
\pgfpathmoveto{\pgfqpoint{0.901249in}{2.163402in}}%
\pgfpathcurveto{\pgfqpoint{0.909485in}{2.163402in}}{\pgfqpoint{0.917385in}{2.166675in}}{\pgfqpoint{0.923209in}{2.172499in}}%
\pgfpathcurveto{\pgfqpoint{0.929033in}{2.178323in}}{\pgfqpoint{0.932305in}{2.186223in}}{\pgfqpoint{0.932305in}{2.194459in}}%
\pgfpathcurveto{\pgfqpoint{0.932305in}{2.202695in}}{\pgfqpoint{0.929033in}{2.210595in}}{\pgfqpoint{0.923209in}{2.216419in}}%
\pgfpathcurveto{\pgfqpoint{0.917385in}{2.222243in}}{\pgfqpoint{0.909485in}{2.225515in}}{\pgfqpoint{0.901249in}{2.225515in}}%
\pgfpathcurveto{\pgfqpoint{0.893012in}{2.225515in}}{\pgfqpoint{0.885112in}{2.222243in}}{\pgfqpoint{0.879288in}{2.216419in}}%
\pgfpathcurveto{\pgfqpoint{0.873464in}{2.210595in}}{\pgfqpoint{0.870192in}{2.202695in}}{\pgfqpoint{0.870192in}{2.194459in}}%
\pgfpathcurveto{\pgfqpoint{0.870192in}{2.186223in}}{\pgfqpoint{0.873464in}{2.178323in}}{\pgfqpoint{0.879288in}{2.172499in}}%
\pgfpathcurveto{\pgfqpoint{0.885112in}{2.166675in}}{\pgfqpoint{0.893012in}{2.163402in}}{\pgfqpoint{0.901249in}{2.163402in}}%
\pgfpathclose%
\pgfusepath{stroke,fill}%
\end{pgfscope}%
\begin{pgfscope}%
\pgfpathrectangle{\pgfqpoint{0.100000in}{0.212622in}}{\pgfqpoint{3.696000in}{3.696000in}}%
\pgfusepath{clip}%
\pgfsetbuttcap%
\pgfsetroundjoin%
\definecolor{currentfill}{rgb}{0.121569,0.466667,0.705882}%
\pgfsetfillcolor{currentfill}%
\pgfsetfillopacity{0.711931}%
\pgfsetlinewidth{1.003750pt}%
\definecolor{currentstroke}{rgb}{0.121569,0.466667,0.705882}%
\pgfsetstrokecolor{currentstroke}%
\pgfsetstrokeopacity{0.711931}%
\pgfsetdash{}{0pt}%
\pgfpathmoveto{\pgfqpoint{0.897911in}{2.163801in}}%
\pgfpathcurveto{\pgfqpoint{0.906148in}{2.163801in}}{\pgfqpoint{0.914048in}{2.167073in}}{\pgfqpoint{0.919872in}{2.172897in}}%
\pgfpathcurveto{\pgfqpoint{0.925696in}{2.178721in}}{\pgfqpoint{0.928968in}{2.186621in}}{\pgfqpoint{0.928968in}{2.194857in}}%
\pgfpathcurveto{\pgfqpoint{0.928968in}{2.203094in}}{\pgfqpoint{0.925696in}{2.210994in}}{\pgfqpoint{0.919872in}{2.216818in}}%
\pgfpathcurveto{\pgfqpoint{0.914048in}{2.222642in}}{\pgfqpoint{0.906148in}{2.225914in}}{\pgfqpoint{0.897911in}{2.225914in}}%
\pgfpathcurveto{\pgfqpoint{0.889675in}{2.225914in}}{\pgfqpoint{0.881775in}{2.222642in}}{\pgfqpoint{0.875951in}{2.216818in}}%
\pgfpathcurveto{\pgfqpoint{0.870127in}{2.210994in}}{\pgfqpoint{0.866855in}{2.203094in}}{\pgfqpoint{0.866855in}{2.194857in}}%
\pgfpathcurveto{\pgfqpoint{0.866855in}{2.186621in}}{\pgfqpoint{0.870127in}{2.178721in}}{\pgfqpoint{0.875951in}{2.172897in}}%
\pgfpathcurveto{\pgfqpoint{0.881775in}{2.167073in}}{\pgfqpoint{0.889675in}{2.163801in}}{\pgfqpoint{0.897911in}{2.163801in}}%
\pgfpathclose%
\pgfusepath{stroke,fill}%
\end{pgfscope}%
\begin{pgfscope}%
\pgfpathrectangle{\pgfqpoint{0.100000in}{0.212622in}}{\pgfqpoint{3.696000in}{3.696000in}}%
\pgfusepath{clip}%
\pgfsetbuttcap%
\pgfsetroundjoin%
\definecolor{currentfill}{rgb}{0.121569,0.466667,0.705882}%
\pgfsetfillcolor{currentfill}%
\pgfsetfillopacity{0.712035}%
\pgfsetlinewidth{1.003750pt}%
\definecolor{currentstroke}{rgb}{0.121569,0.466667,0.705882}%
\pgfsetstrokecolor{currentstroke}%
\pgfsetstrokeopacity{0.712035}%
\pgfsetdash{}{0pt}%
\pgfpathmoveto{\pgfqpoint{2.971649in}{1.808351in}}%
\pgfpathcurveto{\pgfqpoint{2.979885in}{1.808351in}}{\pgfqpoint{2.987785in}{1.811623in}}{\pgfqpoint{2.993609in}{1.817447in}}%
\pgfpathcurveto{\pgfqpoint{2.999433in}{1.823271in}}{\pgfqpoint{3.002705in}{1.831171in}}{\pgfqpoint{3.002705in}{1.839407in}}%
\pgfpathcurveto{\pgfqpoint{3.002705in}{1.847643in}}{\pgfqpoint{2.999433in}{1.855544in}}{\pgfqpoint{2.993609in}{1.861367in}}%
\pgfpathcurveto{\pgfqpoint{2.987785in}{1.867191in}}{\pgfqpoint{2.979885in}{1.870464in}}{\pgfqpoint{2.971649in}{1.870464in}}%
\pgfpathcurveto{\pgfqpoint{2.963413in}{1.870464in}}{\pgfqpoint{2.955513in}{1.867191in}}{\pgfqpoint{2.949689in}{1.861367in}}%
\pgfpathcurveto{\pgfqpoint{2.943865in}{1.855544in}}{\pgfqpoint{2.940592in}{1.847643in}}{\pgfqpoint{2.940592in}{1.839407in}}%
\pgfpathcurveto{\pgfqpoint{2.940592in}{1.831171in}}{\pgfqpoint{2.943865in}{1.823271in}}{\pgfqpoint{2.949689in}{1.817447in}}%
\pgfpathcurveto{\pgfqpoint{2.955513in}{1.811623in}}{\pgfqpoint{2.963413in}{1.808351in}}{\pgfqpoint{2.971649in}{1.808351in}}%
\pgfpathclose%
\pgfusepath{stroke,fill}%
\end{pgfscope}%
\begin{pgfscope}%
\pgfpathrectangle{\pgfqpoint{0.100000in}{0.212622in}}{\pgfqpoint{3.696000in}{3.696000in}}%
\pgfusepath{clip}%
\pgfsetbuttcap%
\pgfsetroundjoin%
\definecolor{currentfill}{rgb}{0.121569,0.466667,0.705882}%
\pgfsetfillcolor{currentfill}%
\pgfsetfillopacity{0.713097}%
\pgfsetlinewidth{1.003750pt}%
\definecolor{currentstroke}{rgb}{0.121569,0.466667,0.705882}%
\pgfsetstrokecolor{currentstroke}%
\pgfsetstrokeopacity{0.713097}%
\pgfsetdash{}{0pt}%
\pgfpathmoveto{\pgfqpoint{0.896019in}{2.163928in}}%
\pgfpathcurveto{\pgfqpoint{0.904255in}{2.163928in}}{\pgfqpoint{0.912156in}{2.167201in}}{\pgfqpoint{0.917979in}{2.173025in}}%
\pgfpathcurveto{\pgfqpoint{0.923803in}{2.178849in}}{\pgfqpoint{0.927076in}{2.186749in}}{\pgfqpoint{0.927076in}{2.194985in}}%
\pgfpathcurveto{\pgfqpoint{0.927076in}{2.203221in}}{\pgfqpoint{0.923803in}{2.211121in}}{\pgfqpoint{0.917979in}{2.216945in}}%
\pgfpathcurveto{\pgfqpoint{0.912156in}{2.222769in}}{\pgfqpoint{0.904255in}{2.226041in}}{\pgfqpoint{0.896019in}{2.226041in}}%
\pgfpathcurveto{\pgfqpoint{0.887783in}{2.226041in}}{\pgfqpoint{0.879883in}{2.222769in}}{\pgfqpoint{0.874059in}{2.216945in}}%
\pgfpathcurveto{\pgfqpoint{0.868235in}{2.211121in}}{\pgfqpoint{0.864963in}{2.203221in}}{\pgfqpoint{0.864963in}{2.194985in}}%
\pgfpathcurveto{\pgfqpoint{0.864963in}{2.186749in}}{\pgfqpoint{0.868235in}{2.178849in}}{\pgfqpoint{0.874059in}{2.173025in}}%
\pgfpathcurveto{\pgfqpoint{0.879883in}{2.167201in}}{\pgfqpoint{0.887783in}{2.163928in}}{\pgfqpoint{0.896019in}{2.163928in}}%
\pgfpathclose%
\pgfusepath{stroke,fill}%
\end{pgfscope}%
\begin{pgfscope}%
\pgfpathrectangle{\pgfqpoint{0.100000in}{0.212622in}}{\pgfqpoint{3.696000in}{3.696000in}}%
\pgfusepath{clip}%
\pgfsetbuttcap%
\pgfsetroundjoin%
\definecolor{currentfill}{rgb}{0.121569,0.466667,0.705882}%
\pgfsetfillcolor{currentfill}%
\pgfsetfillopacity{0.713757}%
\pgfsetlinewidth{1.003750pt}%
\definecolor{currentstroke}{rgb}{0.121569,0.466667,0.705882}%
\pgfsetstrokecolor{currentstroke}%
\pgfsetstrokeopacity{0.713757}%
\pgfsetdash{}{0pt}%
\pgfpathmoveto{\pgfqpoint{0.894229in}{2.164135in}}%
\pgfpathcurveto{\pgfqpoint{0.902466in}{2.164135in}}{\pgfqpoint{0.910366in}{2.167408in}}{\pgfqpoint{0.916190in}{2.173231in}}%
\pgfpathcurveto{\pgfqpoint{0.922014in}{2.179055in}}{\pgfqpoint{0.925286in}{2.186955in}}{\pgfqpoint{0.925286in}{2.195192in}}%
\pgfpathcurveto{\pgfqpoint{0.925286in}{2.203428in}}{\pgfqpoint{0.922014in}{2.211328in}}{\pgfqpoint{0.916190in}{2.217152in}}%
\pgfpathcurveto{\pgfqpoint{0.910366in}{2.222976in}}{\pgfqpoint{0.902466in}{2.226248in}}{\pgfqpoint{0.894229in}{2.226248in}}%
\pgfpathcurveto{\pgfqpoint{0.885993in}{2.226248in}}{\pgfqpoint{0.878093in}{2.222976in}}{\pgfqpoint{0.872269in}{2.217152in}}%
\pgfpathcurveto{\pgfqpoint{0.866445in}{2.211328in}}{\pgfqpoint{0.863173in}{2.203428in}}{\pgfqpoint{0.863173in}{2.195192in}}%
\pgfpathcurveto{\pgfqpoint{0.863173in}{2.186955in}}{\pgfqpoint{0.866445in}{2.179055in}}{\pgfqpoint{0.872269in}{2.173231in}}%
\pgfpathcurveto{\pgfqpoint{0.878093in}{2.167408in}}{\pgfqpoint{0.885993in}{2.164135in}}{\pgfqpoint{0.894229in}{2.164135in}}%
\pgfpathclose%
\pgfusepath{stroke,fill}%
\end{pgfscope}%
\begin{pgfscope}%
\pgfpathrectangle{\pgfqpoint{0.100000in}{0.212622in}}{\pgfqpoint{3.696000in}{3.696000in}}%
\pgfusepath{clip}%
\pgfsetbuttcap%
\pgfsetroundjoin%
\definecolor{currentfill}{rgb}{0.121569,0.466667,0.705882}%
\pgfsetfillcolor{currentfill}%
\pgfsetfillopacity{0.714415}%
\pgfsetlinewidth{1.003750pt}%
\definecolor{currentstroke}{rgb}{0.121569,0.466667,0.705882}%
\pgfsetstrokecolor{currentstroke}%
\pgfsetstrokeopacity{0.714415}%
\pgfsetdash{}{0pt}%
\pgfpathmoveto{\pgfqpoint{2.968038in}{1.808554in}}%
\pgfpathcurveto{\pgfqpoint{2.976274in}{1.808554in}}{\pgfqpoint{2.984174in}{1.811826in}}{\pgfqpoint{2.989998in}{1.817650in}}%
\pgfpathcurveto{\pgfqpoint{2.995822in}{1.823474in}}{\pgfqpoint{2.999094in}{1.831374in}}{\pgfqpoint{2.999094in}{1.839610in}}%
\pgfpathcurveto{\pgfqpoint{2.999094in}{1.847847in}}{\pgfqpoint{2.995822in}{1.855747in}}{\pgfqpoint{2.989998in}{1.861570in}}%
\pgfpathcurveto{\pgfqpoint{2.984174in}{1.867394in}}{\pgfqpoint{2.976274in}{1.870667in}}{\pgfqpoint{2.968038in}{1.870667in}}%
\pgfpathcurveto{\pgfqpoint{2.959801in}{1.870667in}}{\pgfqpoint{2.951901in}{1.867394in}}{\pgfqpoint{2.946077in}{1.861570in}}%
\pgfpathcurveto{\pgfqpoint{2.940254in}{1.855747in}}{\pgfqpoint{2.936981in}{1.847847in}}{\pgfqpoint{2.936981in}{1.839610in}}%
\pgfpathcurveto{\pgfqpoint{2.936981in}{1.831374in}}{\pgfqpoint{2.940254in}{1.823474in}}{\pgfqpoint{2.946077in}{1.817650in}}%
\pgfpathcurveto{\pgfqpoint{2.951901in}{1.811826in}}{\pgfqpoint{2.959801in}{1.808554in}}{\pgfqpoint{2.968038in}{1.808554in}}%
\pgfpathclose%
\pgfusepath{stroke,fill}%
\end{pgfscope}%
\begin{pgfscope}%
\pgfpathrectangle{\pgfqpoint{0.100000in}{0.212622in}}{\pgfqpoint{3.696000in}{3.696000in}}%
\pgfusepath{clip}%
\pgfsetbuttcap%
\pgfsetroundjoin%
\definecolor{currentfill}{rgb}{0.121569,0.466667,0.705882}%
\pgfsetfillcolor{currentfill}%
\pgfsetfillopacity{0.715047}%
\pgfsetlinewidth{1.003750pt}%
\definecolor{currentstroke}{rgb}{0.121569,0.466667,0.705882}%
\pgfsetstrokecolor{currentstroke}%
\pgfsetstrokeopacity{0.715047}%
\pgfsetdash{}{0pt}%
\pgfpathmoveto{\pgfqpoint{0.891647in}{2.164340in}}%
\pgfpathcurveto{\pgfqpoint{0.899883in}{2.164340in}}{\pgfqpoint{0.907783in}{2.167612in}}{\pgfqpoint{0.913607in}{2.173436in}}%
\pgfpathcurveto{\pgfqpoint{0.919431in}{2.179260in}}{\pgfqpoint{0.922703in}{2.187160in}}{\pgfqpoint{0.922703in}{2.195397in}}%
\pgfpathcurveto{\pgfqpoint{0.922703in}{2.203633in}}{\pgfqpoint{0.919431in}{2.211533in}}{\pgfqpoint{0.913607in}{2.217357in}}%
\pgfpathcurveto{\pgfqpoint{0.907783in}{2.223181in}}{\pgfqpoint{0.899883in}{2.226453in}}{\pgfqpoint{0.891647in}{2.226453in}}%
\pgfpathcurveto{\pgfqpoint{0.883410in}{2.226453in}}{\pgfqpoint{0.875510in}{2.223181in}}{\pgfqpoint{0.869686in}{2.217357in}}%
\pgfpathcurveto{\pgfqpoint{0.863862in}{2.211533in}}{\pgfqpoint{0.860590in}{2.203633in}}{\pgfqpoint{0.860590in}{2.195397in}}%
\pgfpathcurveto{\pgfqpoint{0.860590in}{2.187160in}}{\pgfqpoint{0.863862in}{2.179260in}}{\pgfqpoint{0.869686in}{2.173436in}}%
\pgfpathcurveto{\pgfqpoint{0.875510in}{2.167612in}}{\pgfqpoint{0.883410in}{2.164340in}}{\pgfqpoint{0.891647in}{2.164340in}}%
\pgfpathclose%
\pgfusepath{stroke,fill}%
\end{pgfscope}%
\begin{pgfscope}%
\pgfpathrectangle{\pgfqpoint{0.100000in}{0.212622in}}{\pgfqpoint{3.696000in}{3.696000in}}%
\pgfusepath{clip}%
\pgfsetbuttcap%
\pgfsetroundjoin%
\definecolor{currentfill}{rgb}{0.121569,0.466667,0.705882}%
\pgfsetfillcolor{currentfill}%
\pgfsetfillopacity{0.715644}%
\pgfsetlinewidth{1.003750pt}%
\definecolor{currentstroke}{rgb}{0.121569,0.466667,0.705882}%
\pgfsetstrokecolor{currentstroke}%
\pgfsetstrokeopacity{0.715644}%
\pgfsetdash{}{0pt}%
\pgfpathmoveto{\pgfqpoint{2.965309in}{1.808930in}}%
\pgfpathcurveto{\pgfqpoint{2.973545in}{1.808930in}}{\pgfqpoint{2.981445in}{1.812202in}}{\pgfqpoint{2.987269in}{1.818026in}}%
\pgfpathcurveto{\pgfqpoint{2.993093in}{1.823850in}}{\pgfqpoint{2.996365in}{1.831750in}}{\pgfqpoint{2.996365in}{1.839986in}}%
\pgfpathcurveto{\pgfqpoint{2.996365in}{1.848222in}}{\pgfqpoint{2.993093in}{1.856122in}}{\pgfqpoint{2.987269in}{1.861946in}}%
\pgfpathcurveto{\pgfqpoint{2.981445in}{1.867770in}}{\pgfqpoint{2.973545in}{1.871043in}}{\pgfqpoint{2.965309in}{1.871043in}}%
\pgfpathcurveto{\pgfqpoint{2.957072in}{1.871043in}}{\pgfqpoint{2.949172in}{1.867770in}}{\pgfqpoint{2.943348in}{1.861946in}}%
\pgfpathcurveto{\pgfqpoint{2.937525in}{1.856122in}}{\pgfqpoint{2.934252in}{1.848222in}}{\pgfqpoint{2.934252in}{1.839986in}}%
\pgfpathcurveto{\pgfqpoint{2.934252in}{1.831750in}}{\pgfqpoint{2.937525in}{1.823850in}}{\pgfqpoint{2.943348in}{1.818026in}}%
\pgfpathcurveto{\pgfqpoint{2.949172in}{1.812202in}}{\pgfqpoint{2.957072in}{1.808930in}}{\pgfqpoint{2.965309in}{1.808930in}}%
\pgfpathclose%
\pgfusepath{stroke,fill}%
\end{pgfscope}%
\begin{pgfscope}%
\pgfpathrectangle{\pgfqpoint{0.100000in}{0.212622in}}{\pgfqpoint{3.696000in}{3.696000in}}%
\pgfusepath{clip}%
\pgfsetbuttcap%
\pgfsetroundjoin%
\definecolor{currentfill}{rgb}{0.121569,0.466667,0.705882}%
\pgfsetfillcolor{currentfill}%
\pgfsetfillopacity{0.716099}%
\pgfsetlinewidth{1.003750pt}%
\definecolor{currentstroke}{rgb}{0.121569,0.466667,0.705882}%
\pgfsetstrokecolor{currentstroke}%
\pgfsetstrokeopacity{0.716099}%
\pgfsetdash{}{0pt}%
\pgfpathmoveto{\pgfqpoint{0.889231in}{2.164440in}}%
\pgfpathcurveto{\pgfqpoint{0.897468in}{2.164440in}}{\pgfqpoint{0.905368in}{2.167713in}}{\pgfqpoint{0.911192in}{2.173537in}}%
\pgfpathcurveto{\pgfqpoint{0.917016in}{2.179360in}}{\pgfqpoint{0.920288in}{2.187261in}}{\pgfqpoint{0.920288in}{2.195497in}}%
\pgfpathcurveto{\pgfqpoint{0.920288in}{2.203733in}}{\pgfqpoint{0.917016in}{2.211633in}}{\pgfqpoint{0.911192in}{2.217457in}}%
\pgfpathcurveto{\pgfqpoint{0.905368in}{2.223281in}}{\pgfqpoint{0.897468in}{2.226553in}}{\pgfqpoint{0.889231in}{2.226553in}}%
\pgfpathcurveto{\pgfqpoint{0.880995in}{2.226553in}}{\pgfqpoint{0.873095in}{2.223281in}}{\pgfqpoint{0.867271in}{2.217457in}}%
\pgfpathcurveto{\pgfqpoint{0.861447in}{2.211633in}}{\pgfqpoint{0.858175in}{2.203733in}}{\pgfqpoint{0.858175in}{2.195497in}}%
\pgfpathcurveto{\pgfqpoint{0.858175in}{2.187261in}}{\pgfqpoint{0.861447in}{2.179360in}}{\pgfqpoint{0.867271in}{2.173537in}}%
\pgfpathcurveto{\pgfqpoint{0.873095in}{2.167713in}}{\pgfqpoint{0.880995in}{2.164440in}}{\pgfqpoint{0.889231in}{2.164440in}}%
\pgfpathclose%
\pgfusepath{stroke,fill}%
\end{pgfscope}%
\begin{pgfscope}%
\pgfpathrectangle{\pgfqpoint{0.100000in}{0.212622in}}{\pgfqpoint{3.696000in}{3.696000in}}%
\pgfusepath{clip}%
\pgfsetbuttcap%
\pgfsetroundjoin%
\definecolor{currentfill}{rgb}{0.121569,0.466667,0.705882}%
\pgfsetfillcolor{currentfill}%
\pgfsetfillopacity{0.717531}%
\pgfsetlinewidth{1.003750pt}%
\definecolor{currentstroke}{rgb}{0.121569,0.466667,0.705882}%
\pgfsetstrokecolor{currentstroke}%
\pgfsetstrokeopacity{0.717531}%
\pgfsetdash{}{0pt}%
\pgfpathmoveto{\pgfqpoint{2.962011in}{1.809094in}}%
\pgfpathcurveto{\pgfqpoint{2.970247in}{1.809094in}}{\pgfqpoint{2.978147in}{1.812366in}}{\pgfqpoint{2.983971in}{1.818190in}}%
\pgfpathcurveto{\pgfqpoint{2.989795in}{1.824014in}}{\pgfqpoint{2.993067in}{1.831914in}}{\pgfqpoint{2.993067in}{1.840150in}}%
\pgfpathcurveto{\pgfqpoint{2.993067in}{1.848386in}}{\pgfqpoint{2.989795in}{1.856286in}}{\pgfqpoint{2.983971in}{1.862110in}}%
\pgfpathcurveto{\pgfqpoint{2.978147in}{1.867934in}}{\pgfqpoint{2.970247in}{1.871207in}}{\pgfqpoint{2.962011in}{1.871207in}}%
\pgfpathcurveto{\pgfqpoint{2.953775in}{1.871207in}}{\pgfqpoint{2.945875in}{1.867934in}}{\pgfqpoint{2.940051in}{1.862110in}}%
\pgfpathcurveto{\pgfqpoint{2.934227in}{1.856286in}}{\pgfqpoint{2.930954in}{1.848386in}}{\pgfqpoint{2.930954in}{1.840150in}}%
\pgfpathcurveto{\pgfqpoint{2.930954in}{1.831914in}}{\pgfqpoint{2.934227in}{1.824014in}}{\pgfqpoint{2.940051in}{1.818190in}}%
\pgfpathcurveto{\pgfqpoint{2.945875in}{1.812366in}}{\pgfqpoint{2.953775in}{1.809094in}}{\pgfqpoint{2.962011in}{1.809094in}}%
\pgfpathclose%
\pgfusepath{stroke,fill}%
\end{pgfscope}%
\begin{pgfscope}%
\pgfpathrectangle{\pgfqpoint{0.100000in}{0.212622in}}{\pgfqpoint{3.696000in}{3.696000in}}%
\pgfusepath{clip}%
\pgfsetbuttcap%
\pgfsetroundjoin%
\definecolor{currentfill}{rgb}{0.121569,0.466667,0.705882}%
\pgfsetfillcolor{currentfill}%
\pgfsetfillopacity{0.717992}%
\pgfsetlinewidth{1.003750pt}%
\definecolor{currentstroke}{rgb}{0.121569,0.466667,0.705882}%
\pgfsetstrokecolor{currentstroke}%
\pgfsetstrokeopacity{0.717992}%
\pgfsetdash{}{0pt}%
\pgfpathmoveto{\pgfqpoint{0.884462in}{2.164931in}}%
\pgfpathcurveto{\pgfqpoint{0.892698in}{2.164931in}}{\pgfqpoint{0.900598in}{2.168203in}}{\pgfqpoint{0.906422in}{2.174027in}}%
\pgfpathcurveto{\pgfqpoint{0.912246in}{2.179851in}}{\pgfqpoint{0.915518in}{2.187751in}}{\pgfqpoint{0.915518in}{2.195987in}}%
\pgfpathcurveto{\pgfqpoint{0.915518in}{2.204224in}}{\pgfqpoint{0.912246in}{2.212124in}}{\pgfqpoint{0.906422in}{2.217948in}}%
\pgfpathcurveto{\pgfqpoint{0.900598in}{2.223772in}}{\pgfqpoint{0.892698in}{2.227044in}}{\pgfqpoint{0.884462in}{2.227044in}}%
\pgfpathcurveto{\pgfqpoint{0.876225in}{2.227044in}}{\pgfqpoint{0.868325in}{2.223772in}}{\pgfqpoint{0.862501in}{2.217948in}}%
\pgfpathcurveto{\pgfqpoint{0.856678in}{2.212124in}}{\pgfqpoint{0.853405in}{2.204224in}}{\pgfqpoint{0.853405in}{2.195987in}}%
\pgfpathcurveto{\pgfqpoint{0.853405in}{2.187751in}}{\pgfqpoint{0.856678in}{2.179851in}}{\pgfqpoint{0.862501in}{2.174027in}}%
\pgfpathcurveto{\pgfqpoint{0.868325in}{2.168203in}}{\pgfqpoint{0.876225in}{2.164931in}}{\pgfqpoint{0.884462in}{2.164931in}}%
\pgfpathclose%
\pgfusepath{stroke,fill}%
\end{pgfscope}%
\begin{pgfscope}%
\pgfpathrectangle{\pgfqpoint{0.100000in}{0.212622in}}{\pgfqpoint{3.696000in}{3.696000in}}%
\pgfusepath{clip}%
\pgfsetbuttcap%
\pgfsetroundjoin%
\definecolor{currentfill}{rgb}{0.121569,0.466667,0.705882}%
\pgfsetfillcolor{currentfill}%
\pgfsetfillopacity{0.719733}%
\pgfsetlinewidth{1.003750pt}%
\definecolor{currentstroke}{rgb}{0.121569,0.466667,0.705882}%
\pgfsetstrokecolor{currentstroke}%
\pgfsetstrokeopacity{0.719733}%
\pgfsetdash{}{0pt}%
\pgfpathmoveto{\pgfqpoint{2.960142in}{1.809173in}}%
\pgfpathcurveto{\pgfqpoint{2.968379in}{1.809173in}}{\pgfqpoint{2.976279in}{1.812445in}}{\pgfqpoint{2.982103in}{1.818269in}}%
\pgfpathcurveto{\pgfqpoint{2.987926in}{1.824093in}}{\pgfqpoint{2.991199in}{1.831993in}}{\pgfqpoint{2.991199in}{1.840229in}}%
\pgfpathcurveto{\pgfqpoint{2.991199in}{1.848465in}}{\pgfqpoint{2.987926in}{1.856365in}}{\pgfqpoint{2.982103in}{1.862189in}}%
\pgfpathcurveto{\pgfqpoint{2.976279in}{1.868013in}}{\pgfqpoint{2.968379in}{1.871286in}}{\pgfqpoint{2.960142in}{1.871286in}}%
\pgfpathcurveto{\pgfqpoint{2.951906in}{1.871286in}}{\pgfqpoint{2.944006in}{1.868013in}}{\pgfqpoint{2.938182in}{1.862189in}}%
\pgfpathcurveto{\pgfqpoint{2.932358in}{1.856365in}}{\pgfqpoint{2.929086in}{1.848465in}}{\pgfqpoint{2.929086in}{1.840229in}}%
\pgfpathcurveto{\pgfqpoint{2.929086in}{1.831993in}}{\pgfqpoint{2.932358in}{1.824093in}}{\pgfqpoint{2.938182in}{1.818269in}}%
\pgfpathcurveto{\pgfqpoint{2.944006in}{1.812445in}}{\pgfqpoint{2.951906in}{1.809173in}}{\pgfqpoint{2.960142in}{1.809173in}}%
\pgfpathclose%
\pgfusepath{stroke,fill}%
\end{pgfscope}%
\begin{pgfscope}%
\pgfpathrectangle{\pgfqpoint{0.100000in}{0.212622in}}{\pgfqpoint{3.696000in}{3.696000in}}%
\pgfusepath{clip}%
\pgfsetbuttcap%
\pgfsetroundjoin%
\definecolor{currentfill}{rgb}{0.121569,0.466667,0.705882}%
\pgfsetfillcolor{currentfill}%
\pgfsetfillopacity{0.719869}%
\pgfsetlinewidth{1.003750pt}%
\definecolor{currentstroke}{rgb}{0.121569,0.466667,0.705882}%
\pgfsetstrokecolor{currentstroke}%
\pgfsetstrokeopacity{0.719869}%
\pgfsetdash{}{0pt}%
\pgfpathmoveto{\pgfqpoint{0.881719in}{2.165252in}}%
\pgfpathcurveto{\pgfqpoint{0.889955in}{2.165252in}}{\pgfqpoint{0.897855in}{2.168524in}}{\pgfqpoint{0.903679in}{2.174348in}}%
\pgfpathcurveto{\pgfqpoint{0.909503in}{2.180172in}}{\pgfqpoint{0.912776in}{2.188072in}}{\pgfqpoint{0.912776in}{2.196309in}}%
\pgfpathcurveto{\pgfqpoint{0.912776in}{2.204545in}}{\pgfqpoint{0.909503in}{2.212445in}}{\pgfqpoint{0.903679in}{2.218269in}}%
\pgfpathcurveto{\pgfqpoint{0.897855in}{2.224093in}}{\pgfqpoint{0.889955in}{2.227365in}}{\pgfqpoint{0.881719in}{2.227365in}}%
\pgfpathcurveto{\pgfqpoint{0.873483in}{2.227365in}}{\pgfqpoint{0.865583in}{2.224093in}}{\pgfqpoint{0.859759in}{2.218269in}}%
\pgfpathcurveto{\pgfqpoint{0.853935in}{2.212445in}}{\pgfqpoint{0.850663in}{2.204545in}}{\pgfqpoint{0.850663in}{2.196309in}}%
\pgfpathcurveto{\pgfqpoint{0.850663in}{2.188072in}}{\pgfqpoint{0.853935in}{2.180172in}}{\pgfqpoint{0.859759in}{2.174348in}}%
\pgfpathcurveto{\pgfqpoint{0.865583in}{2.168524in}}{\pgfqpoint{0.873483in}{2.165252in}}{\pgfqpoint{0.881719in}{2.165252in}}%
\pgfpathclose%
\pgfusepath{stroke,fill}%
\end{pgfscope}%
\begin{pgfscope}%
\pgfpathrectangle{\pgfqpoint{0.100000in}{0.212622in}}{\pgfqpoint{3.696000in}{3.696000in}}%
\pgfusepath{clip}%
\pgfsetbuttcap%
\pgfsetroundjoin%
\definecolor{currentfill}{rgb}{0.121569,0.466667,0.705882}%
\pgfsetfillcolor{currentfill}%
\pgfsetfillopacity{0.721137}%
\pgfsetlinewidth{1.003750pt}%
\definecolor{currentstroke}{rgb}{0.121569,0.466667,0.705882}%
\pgfsetstrokecolor{currentstroke}%
\pgfsetstrokeopacity{0.721137}%
\pgfsetdash{}{0pt}%
\pgfpathmoveto{\pgfqpoint{0.877916in}{2.165778in}}%
\pgfpathcurveto{\pgfqpoint{0.886153in}{2.165778in}}{\pgfqpoint{0.894053in}{2.169050in}}{\pgfqpoint{0.899877in}{2.174874in}}%
\pgfpathcurveto{\pgfqpoint{0.905700in}{2.180698in}}{\pgfqpoint{0.908973in}{2.188598in}}{\pgfqpoint{0.908973in}{2.196835in}}%
\pgfpathcurveto{\pgfqpoint{0.908973in}{2.205071in}}{\pgfqpoint{0.905700in}{2.212971in}}{\pgfqpoint{0.899877in}{2.218795in}}%
\pgfpathcurveto{\pgfqpoint{0.894053in}{2.224619in}}{\pgfqpoint{0.886153in}{2.227891in}}{\pgfqpoint{0.877916in}{2.227891in}}%
\pgfpathcurveto{\pgfqpoint{0.869680in}{2.227891in}}{\pgfqpoint{0.861780in}{2.224619in}}{\pgfqpoint{0.855956in}{2.218795in}}%
\pgfpathcurveto{\pgfqpoint{0.850132in}{2.212971in}}{\pgfqpoint{0.846860in}{2.205071in}}{\pgfqpoint{0.846860in}{2.196835in}}%
\pgfpathcurveto{\pgfqpoint{0.846860in}{2.188598in}}{\pgfqpoint{0.850132in}{2.180698in}}{\pgfqpoint{0.855956in}{2.174874in}}%
\pgfpathcurveto{\pgfqpoint{0.861780in}{2.169050in}}{\pgfqpoint{0.869680in}{2.165778in}}{\pgfqpoint{0.877916in}{2.165778in}}%
\pgfpathclose%
\pgfusepath{stroke,fill}%
\end{pgfscope}%
\begin{pgfscope}%
\pgfpathrectangle{\pgfqpoint{0.100000in}{0.212622in}}{\pgfqpoint{3.696000in}{3.696000in}}%
\pgfusepath{clip}%
\pgfsetbuttcap%
\pgfsetroundjoin%
\definecolor{currentfill}{rgb}{0.121569,0.466667,0.705882}%
\pgfsetfillcolor{currentfill}%
\pgfsetfillopacity{0.722020}%
\pgfsetlinewidth{1.003750pt}%
\definecolor{currentstroke}{rgb}{0.121569,0.466667,0.705882}%
\pgfsetstrokecolor{currentstroke}%
\pgfsetstrokeopacity{0.722020}%
\pgfsetdash{}{0pt}%
\pgfpathmoveto{\pgfqpoint{2.955350in}{1.809839in}}%
\pgfpathcurveto{\pgfqpoint{2.963586in}{1.809839in}}{\pgfqpoint{2.971486in}{1.813112in}}{\pgfqpoint{2.977310in}{1.818936in}}%
\pgfpathcurveto{\pgfqpoint{2.983134in}{1.824760in}}{\pgfqpoint{2.986406in}{1.832660in}}{\pgfqpoint{2.986406in}{1.840896in}}%
\pgfpathcurveto{\pgfqpoint{2.986406in}{1.849132in}}{\pgfqpoint{2.983134in}{1.857032in}}{\pgfqpoint{2.977310in}{1.862856in}}%
\pgfpathcurveto{\pgfqpoint{2.971486in}{1.868680in}}{\pgfqpoint{2.963586in}{1.871952in}}{\pgfqpoint{2.955350in}{1.871952in}}%
\pgfpathcurveto{\pgfqpoint{2.947114in}{1.871952in}}{\pgfqpoint{2.939213in}{1.868680in}}{\pgfqpoint{2.933390in}{1.862856in}}%
\pgfpathcurveto{\pgfqpoint{2.927566in}{1.857032in}}{\pgfqpoint{2.924293in}{1.849132in}}{\pgfqpoint{2.924293in}{1.840896in}}%
\pgfpathcurveto{\pgfqpoint{2.924293in}{1.832660in}}{\pgfqpoint{2.927566in}{1.824760in}}{\pgfqpoint{2.933390in}{1.818936in}}%
\pgfpathcurveto{\pgfqpoint{2.939213in}{1.813112in}}{\pgfqpoint{2.947114in}{1.809839in}}{\pgfqpoint{2.955350in}{1.809839in}}%
\pgfpathclose%
\pgfusepath{stroke,fill}%
\end{pgfscope}%
\begin{pgfscope}%
\pgfpathrectangle{\pgfqpoint{0.100000in}{0.212622in}}{\pgfqpoint{3.696000in}{3.696000in}}%
\pgfusepath{clip}%
\pgfsetbuttcap%
\pgfsetroundjoin%
\definecolor{currentfill}{rgb}{0.121569,0.466667,0.705882}%
\pgfsetfillcolor{currentfill}%
\pgfsetfillopacity{0.722164}%
\pgfsetlinewidth{1.003750pt}%
\definecolor{currentstroke}{rgb}{0.121569,0.466667,0.705882}%
\pgfsetstrokecolor{currentstroke}%
\pgfsetstrokeopacity{0.722164}%
\pgfsetdash{}{0pt}%
\pgfpathmoveto{\pgfqpoint{0.879586in}{2.166605in}}%
\pgfpathcurveto{\pgfqpoint{0.887822in}{2.166605in}}{\pgfqpoint{0.895723in}{2.169877in}}{\pgfqpoint{0.901546in}{2.175701in}}%
\pgfpathcurveto{\pgfqpoint{0.907370in}{2.181525in}}{\pgfqpoint{0.910643in}{2.189425in}}{\pgfqpoint{0.910643in}{2.197661in}}%
\pgfpathcurveto{\pgfqpoint{0.910643in}{2.205898in}}{\pgfqpoint{0.907370in}{2.213798in}}{\pgfqpoint{0.901546in}{2.219622in}}%
\pgfpathcurveto{\pgfqpoint{0.895723in}{2.225446in}}{\pgfqpoint{0.887822in}{2.228718in}}{\pgfqpoint{0.879586in}{2.228718in}}%
\pgfpathcurveto{\pgfqpoint{0.871350in}{2.228718in}}{\pgfqpoint{0.863450in}{2.225446in}}{\pgfqpoint{0.857626in}{2.219622in}}%
\pgfpathcurveto{\pgfqpoint{0.851802in}{2.213798in}}{\pgfqpoint{0.848530in}{2.205898in}}{\pgfqpoint{0.848530in}{2.197661in}}%
\pgfpathcurveto{\pgfqpoint{0.848530in}{2.189425in}}{\pgfqpoint{0.851802in}{2.181525in}}{\pgfqpoint{0.857626in}{2.175701in}}%
\pgfpathcurveto{\pgfqpoint{0.863450in}{2.169877in}}{\pgfqpoint{0.871350in}{2.166605in}}{\pgfqpoint{0.879586in}{2.166605in}}%
\pgfpathclose%
\pgfusepath{stroke,fill}%
\end{pgfscope}%
\begin{pgfscope}%
\pgfpathrectangle{\pgfqpoint{0.100000in}{0.212622in}}{\pgfqpoint{3.696000in}{3.696000in}}%
\pgfusepath{clip}%
\pgfsetbuttcap%
\pgfsetroundjoin%
\definecolor{currentfill}{rgb}{0.121569,0.466667,0.705882}%
\pgfsetfillcolor{currentfill}%
\pgfsetfillopacity{0.722846}%
\pgfsetlinewidth{1.003750pt}%
\definecolor{currentstroke}{rgb}{0.121569,0.466667,0.705882}%
\pgfsetstrokecolor{currentstroke}%
\pgfsetstrokeopacity{0.722846}%
\pgfsetdash{}{0pt}%
\pgfpathmoveto{\pgfqpoint{0.877896in}{2.166744in}}%
\pgfpathcurveto{\pgfqpoint{0.886132in}{2.166744in}}{\pgfqpoint{0.894032in}{2.170017in}}{\pgfqpoint{0.899856in}{2.175840in}}%
\pgfpathcurveto{\pgfqpoint{0.905680in}{2.181664in}}{\pgfqpoint{0.908952in}{2.189564in}}{\pgfqpoint{0.908952in}{2.197801in}}%
\pgfpathcurveto{\pgfqpoint{0.908952in}{2.206037in}}{\pgfqpoint{0.905680in}{2.213937in}}{\pgfqpoint{0.899856in}{2.219761in}}%
\pgfpathcurveto{\pgfqpoint{0.894032in}{2.225585in}}{\pgfqpoint{0.886132in}{2.228857in}}{\pgfqpoint{0.877896in}{2.228857in}}%
\pgfpathcurveto{\pgfqpoint{0.869659in}{2.228857in}}{\pgfqpoint{0.861759in}{2.225585in}}{\pgfqpoint{0.855936in}{2.219761in}}%
\pgfpathcurveto{\pgfqpoint{0.850112in}{2.213937in}}{\pgfqpoint{0.846839in}{2.206037in}}{\pgfqpoint{0.846839in}{2.197801in}}%
\pgfpathcurveto{\pgfqpoint{0.846839in}{2.189564in}}{\pgfqpoint{0.850112in}{2.181664in}}{\pgfqpoint{0.855936in}{2.175840in}}%
\pgfpathcurveto{\pgfqpoint{0.861759in}{2.170017in}}{\pgfqpoint{0.869659in}{2.166744in}}{\pgfqpoint{0.877896in}{2.166744in}}%
\pgfpathclose%
\pgfusepath{stroke,fill}%
\end{pgfscope}%
\begin{pgfscope}%
\pgfpathrectangle{\pgfqpoint{0.100000in}{0.212622in}}{\pgfqpoint{3.696000in}{3.696000in}}%
\pgfusepath{clip}%
\pgfsetbuttcap%
\pgfsetroundjoin%
\definecolor{currentfill}{rgb}{0.121569,0.466667,0.705882}%
\pgfsetfillcolor{currentfill}%
\pgfsetfillopacity{0.724131}%
\pgfsetlinewidth{1.003750pt}%
\definecolor{currentstroke}{rgb}{0.121569,0.466667,0.705882}%
\pgfsetstrokecolor{currentstroke}%
\pgfsetstrokeopacity{0.724131}%
\pgfsetdash{}{0pt}%
\pgfpathmoveto{\pgfqpoint{0.875353in}{2.166729in}}%
\pgfpathcurveto{\pgfqpoint{0.883589in}{2.166729in}}{\pgfqpoint{0.891489in}{2.170001in}}{\pgfqpoint{0.897313in}{2.175825in}}%
\pgfpathcurveto{\pgfqpoint{0.903137in}{2.181649in}}{\pgfqpoint{0.906410in}{2.189549in}}{\pgfqpoint{0.906410in}{2.197786in}}%
\pgfpathcurveto{\pgfqpoint{0.906410in}{2.206022in}}{\pgfqpoint{0.903137in}{2.213922in}}{\pgfqpoint{0.897313in}{2.219746in}}%
\pgfpathcurveto{\pgfqpoint{0.891489in}{2.225570in}}{\pgfqpoint{0.883589in}{2.228842in}}{\pgfqpoint{0.875353in}{2.228842in}}%
\pgfpathcurveto{\pgfqpoint{0.867117in}{2.228842in}}{\pgfqpoint{0.859217in}{2.225570in}}{\pgfqpoint{0.853393in}{2.219746in}}%
\pgfpathcurveto{\pgfqpoint{0.847569in}{2.213922in}}{\pgfqpoint{0.844297in}{2.206022in}}{\pgfqpoint{0.844297in}{2.197786in}}%
\pgfpathcurveto{\pgfqpoint{0.844297in}{2.189549in}}{\pgfqpoint{0.847569in}{2.181649in}}{\pgfqpoint{0.853393in}{2.175825in}}%
\pgfpathcurveto{\pgfqpoint{0.859217in}{2.170001in}}{\pgfqpoint{0.867117in}{2.166729in}}{\pgfqpoint{0.875353in}{2.166729in}}%
\pgfpathclose%
\pgfusepath{stroke,fill}%
\end{pgfscope}%
\begin{pgfscope}%
\pgfpathrectangle{\pgfqpoint{0.100000in}{0.212622in}}{\pgfqpoint{3.696000in}{3.696000in}}%
\pgfusepath{clip}%
\pgfsetbuttcap%
\pgfsetroundjoin%
\definecolor{currentfill}{rgb}{0.121569,0.466667,0.705882}%
\pgfsetfillcolor{currentfill}%
\pgfsetfillopacity{0.724326}%
\pgfsetlinewidth{1.003750pt}%
\definecolor{currentstroke}{rgb}{0.121569,0.466667,0.705882}%
\pgfsetstrokecolor{currentstroke}%
\pgfsetstrokeopacity{0.724326}%
\pgfsetdash{}{0pt}%
\pgfpathmoveto{\pgfqpoint{2.949565in}{1.810789in}}%
\pgfpathcurveto{\pgfqpoint{2.957801in}{1.810789in}}{\pgfqpoint{2.965701in}{1.814061in}}{\pgfqpoint{2.971525in}{1.819885in}}%
\pgfpathcurveto{\pgfqpoint{2.977349in}{1.825709in}}{\pgfqpoint{2.980621in}{1.833609in}}{\pgfqpoint{2.980621in}{1.841845in}}%
\pgfpathcurveto{\pgfqpoint{2.980621in}{1.850082in}}{\pgfqpoint{2.977349in}{1.857982in}}{\pgfqpoint{2.971525in}{1.863806in}}%
\pgfpathcurveto{\pgfqpoint{2.965701in}{1.869630in}}{\pgfqpoint{2.957801in}{1.872902in}}{\pgfqpoint{2.949565in}{1.872902in}}%
\pgfpathcurveto{\pgfqpoint{2.941329in}{1.872902in}}{\pgfqpoint{2.933428in}{1.869630in}}{\pgfqpoint{2.927605in}{1.863806in}}%
\pgfpathcurveto{\pgfqpoint{2.921781in}{1.857982in}}{\pgfqpoint{2.918508in}{1.850082in}}{\pgfqpoint{2.918508in}{1.841845in}}%
\pgfpathcurveto{\pgfqpoint{2.918508in}{1.833609in}}{\pgfqpoint{2.921781in}{1.825709in}}{\pgfqpoint{2.927605in}{1.819885in}}%
\pgfpathcurveto{\pgfqpoint{2.933428in}{1.814061in}}{\pgfqpoint{2.941329in}{1.810789in}}{\pgfqpoint{2.949565in}{1.810789in}}%
\pgfpathclose%
\pgfusepath{stroke,fill}%
\end{pgfscope}%
\begin{pgfscope}%
\pgfpathrectangle{\pgfqpoint{0.100000in}{0.212622in}}{\pgfqpoint{3.696000in}{3.696000in}}%
\pgfusepath{clip}%
\pgfsetbuttcap%
\pgfsetroundjoin%
\definecolor{currentfill}{rgb}{0.121569,0.466667,0.705882}%
\pgfsetfillcolor{currentfill}%
\pgfsetfillopacity{0.725300}%
\pgfsetlinewidth{1.003750pt}%
\definecolor{currentstroke}{rgb}{0.121569,0.466667,0.705882}%
\pgfsetstrokecolor{currentstroke}%
\pgfsetstrokeopacity{0.725300}%
\pgfsetdash{}{0pt}%
\pgfpathmoveto{\pgfqpoint{0.873378in}{2.166779in}}%
\pgfpathcurveto{\pgfqpoint{0.881615in}{2.166779in}}{\pgfqpoint{0.889515in}{2.170051in}}{\pgfqpoint{0.895339in}{2.175875in}}%
\pgfpathcurveto{\pgfqpoint{0.901163in}{2.181699in}}{\pgfqpoint{0.904435in}{2.189599in}}{\pgfqpoint{0.904435in}{2.197835in}}%
\pgfpathcurveto{\pgfqpoint{0.904435in}{2.206071in}}{\pgfqpoint{0.901163in}{2.213971in}}{\pgfqpoint{0.895339in}{2.219795in}}%
\pgfpathcurveto{\pgfqpoint{0.889515in}{2.225619in}}{\pgfqpoint{0.881615in}{2.228892in}}{\pgfqpoint{0.873378in}{2.228892in}}%
\pgfpathcurveto{\pgfqpoint{0.865142in}{2.228892in}}{\pgfqpoint{0.857242in}{2.225619in}}{\pgfqpoint{0.851418in}{2.219795in}}%
\pgfpathcurveto{\pgfqpoint{0.845594in}{2.213971in}}{\pgfqpoint{0.842322in}{2.206071in}}{\pgfqpoint{0.842322in}{2.197835in}}%
\pgfpathcurveto{\pgfqpoint{0.842322in}{2.189599in}}{\pgfqpoint{0.845594in}{2.181699in}}{\pgfqpoint{0.851418in}{2.175875in}}%
\pgfpathcurveto{\pgfqpoint{0.857242in}{2.170051in}}{\pgfqpoint{0.865142in}{2.166779in}}{\pgfqpoint{0.873378in}{2.166779in}}%
\pgfpathclose%
\pgfusepath{stroke,fill}%
\end{pgfscope}%
\begin{pgfscope}%
\pgfpathrectangle{\pgfqpoint{0.100000in}{0.212622in}}{\pgfqpoint{3.696000in}{3.696000in}}%
\pgfusepath{clip}%
\pgfsetbuttcap%
\pgfsetroundjoin%
\definecolor{currentfill}{rgb}{0.121569,0.466667,0.705882}%
\pgfsetfillcolor{currentfill}%
\pgfsetfillopacity{0.725856}%
\pgfsetlinewidth{1.003750pt}%
\definecolor{currentstroke}{rgb}{0.121569,0.466667,0.705882}%
\pgfsetstrokecolor{currentstroke}%
\pgfsetstrokeopacity{0.725856}%
\pgfsetdash{}{0pt}%
\pgfpathmoveto{\pgfqpoint{0.871961in}{2.166907in}}%
\pgfpathcurveto{\pgfqpoint{0.880197in}{2.166907in}}{\pgfqpoint{0.888097in}{2.170180in}}{\pgfqpoint{0.893921in}{2.176004in}}%
\pgfpathcurveto{\pgfqpoint{0.899745in}{2.181828in}}{\pgfqpoint{0.903017in}{2.189728in}}{\pgfqpoint{0.903017in}{2.197964in}}%
\pgfpathcurveto{\pgfqpoint{0.903017in}{2.206200in}}{\pgfqpoint{0.899745in}{2.214100in}}{\pgfqpoint{0.893921in}{2.219924in}}%
\pgfpathcurveto{\pgfqpoint{0.888097in}{2.225748in}}{\pgfqpoint{0.880197in}{2.229020in}}{\pgfqpoint{0.871961in}{2.229020in}}%
\pgfpathcurveto{\pgfqpoint{0.863725in}{2.229020in}}{\pgfqpoint{0.855825in}{2.225748in}}{\pgfqpoint{0.850001in}{2.219924in}}%
\pgfpathcurveto{\pgfqpoint{0.844177in}{2.214100in}}{\pgfqpoint{0.840904in}{2.206200in}}{\pgfqpoint{0.840904in}{2.197964in}}%
\pgfpathcurveto{\pgfqpoint{0.840904in}{2.189728in}}{\pgfqpoint{0.844177in}{2.181828in}}{\pgfqpoint{0.850001in}{2.176004in}}%
\pgfpathcurveto{\pgfqpoint{0.855825in}{2.170180in}}{\pgfqpoint{0.863725in}{2.166907in}}{\pgfqpoint{0.871961in}{2.166907in}}%
\pgfpathclose%
\pgfusepath{stroke,fill}%
\end{pgfscope}%
\begin{pgfscope}%
\pgfpathrectangle{\pgfqpoint{0.100000in}{0.212622in}}{\pgfqpoint{3.696000in}{3.696000in}}%
\pgfusepath{clip}%
\pgfsetbuttcap%
\pgfsetroundjoin%
\definecolor{currentfill}{rgb}{0.121569,0.466667,0.705882}%
\pgfsetfillcolor{currentfill}%
\pgfsetfillopacity{0.726257}%
\pgfsetlinewidth{1.003750pt}%
\definecolor{currentstroke}{rgb}{0.121569,0.466667,0.705882}%
\pgfsetstrokecolor{currentstroke}%
\pgfsetstrokeopacity{0.726257}%
\pgfsetdash{}{0pt}%
\pgfpathmoveto{\pgfqpoint{0.871411in}{2.166935in}}%
\pgfpathcurveto{\pgfqpoint{0.879648in}{2.166935in}}{\pgfqpoint{0.887548in}{2.170207in}}{\pgfqpoint{0.893372in}{2.176031in}}%
\pgfpathcurveto{\pgfqpoint{0.899195in}{2.181855in}}{\pgfqpoint{0.902468in}{2.189755in}}{\pgfqpoint{0.902468in}{2.197991in}}%
\pgfpathcurveto{\pgfqpoint{0.902468in}{2.206228in}}{\pgfqpoint{0.899195in}{2.214128in}}{\pgfqpoint{0.893372in}{2.219952in}}%
\pgfpathcurveto{\pgfqpoint{0.887548in}{2.225776in}}{\pgfqpoint{0.879648in}{2.229048in}}{\pgfqpoint{0.871411in}{2.229048in}}%
\pgfpathcurveto{\pgfqpoint{0.863175in}{2.229048in}}{\pgfqpoint{0.855275in}{2.225776in}}{\pgfqpoint{0.849451in}{2.219952in}}%
\pgfpathcurveto{\pgfqpoint{0.843627in}{2.214128in}}{\pgfqpoint{0.840355in}{2.206228in}}{\pgfqpoint{0.840355in}{2.197991in}}%
\pgfpathcurveto{\pgfqpoint{0.840355in}{2.189755in}}{\pgfqpoint{0.843627in}{2.181855in}}{\pgfqpoint{0.849451in}{2.176031in}}%
\pgfpathcurveto{\pgfqpoint{0.855275in}{2.170207in}}{\pgfqpoint{0.863175in}{2.166935in}}{\pgfqpoint{0.871411in}{2.166935in}}%
\pgfpathclose%
\pgfusepath{stroke,fill}%
\end{pgfscope}%
\begin{pgfscope}%
\pgfpathrectangle{\pgfqpoint{0.100000in}{0.212622in}}{\pgfqpoint{3.696000in}{3.696000in}}%
\pgfusepath{clip}%
\pgfsetbuttcap%
\pgfsetroundjoin%
\definecolor{currentfill}{rgb}{0.121569,0.466667,0.705882}%
\pgfsetfillcolor{currentfill}%
\pgfsetfillopacity{0.726905}%
\pgfsetlinewidth{1.003750pt}%
\definecolor{currentstroke}{rgb}{0.121569,0.466667,0.705882}%
\pgfsetstrokecolor{currentstroke}%
\pgfsetstrokeopacity{0.726905}%
\pgfsetdash{}{0pt}%
\pgfpathmoveto{\pgfqpoint{0.869665in}{2.167148in}}%
\pgfpathcurveto{\pgfqpoint{0.877901in}{2.167148in}}{\pgfqpoint{0.885801in}{2.170420in}}{\pgfqpoint{0.891625in}{2.176244in}}%
\pgfpathcurveto{\pgfqpoint{0.897449in}{2.182068in}}{\pgfqpoint{0.900721in}{2.189968in}}{\pgfqpoint{0.900721in}{2.198205in}}%
\pgfpathcurveto{\pgfqpoint{0.900721in}{2.206441in}}{\pgfqpoint{0.897449in}{2.214341in}}{\pgfqpoint{0.891625in}{2.220165in}}%
\pgfpathcurveto{\pgfqpoint{0.885801in}{2.225989in}}{\pgfqpoint{0.877901in}{2.229261in}}{\pgfqpoint{0.869665in}{2.229261in}}%
\pgfpathcurveto{\pgfqpoint{0.861429in}{2.229261in}}{\pgfqpoint{0.853529in}{2.225989in}}{\pgfqpoint{0.847705in}{2.220165in}}%
\pgfpathcurveto{\pgfqpoint{0.841881in}{2.214341in}}{\pgfqpoint{0.838608in}{2.206441in}}{\pgfqpoint{0.838608in}{2.198205in}}%
\pgfpathcurveto{\pgfqpoint{0.838608in}{2.189968in}}{\pgfqpoint{0.841881in}{2.182068in}}{\pgfqpoint{0.847705in}{2.176244in}}%
\pgfpathcurveto{\pgfqpoint{0.853529in}{2.170420in}}{\pgfqpoint{0.861429in}{2.167148in}}{\pgfqpoint{0.869665in}{2.167148in}}%
\pgfpathclose%
\pgfusepath{stroke,fill}%
\end{pgfscope}%
\begin{pgfscope}%
\pgfpathrectangle{\pgfqpoint{0.100000in}{0.212622in}}{\pgfqpoint{3.696000in}{3.696000in}}%
\pgfusepath{clip}%
\pgfsetbuttcap%
\pgfsetroundjoin%
\definecolor{currentfill}{rgb}{0.121569,0.466667,0.705882}%
\pgfsetfillcolor{currentfill}%
\pgfsetfillopacity{0.727481}%
\pgfsetlinewidth{1.003750pt}%
\definecolor{currentstroke}{rgb}{0.121569,0.466667,0.705882}%
\pgfsetstrokecolor{currentstroke}%
\pgfsetstrokeopacity{0.727481}%
\pgfsetdash{}{0pt}%
\pgfpathmoveto{\pgfqpoint{2.946878in}{1.810761in}}%
\pgfpathcurveto{\pgfqpoint{2.955114in}{1.810761in}}{\pgfqpoint{2.963014in}{1.814034in}}{\pgfqpoint{2.968838in}{1.819857in}}%
\pgfpathcurveto{\pgfqpoint{2.974662in}{1.825681in}}{\pgfqpoint{2.977934in}{1.833581in}}{\pgfqpoint{2.977934in}{1.841818in}}%
\pgfpathcurveto{\pgfqpoint{2.977934in}{1.850054in}}{\pgfqpoint{2.974662in}{1.857954in}}{\pgfqpoint{2.968838in}{1.863778in}}%
\pgfpathcurveto{\pgfqpoint{2.963014in}{1.869602in}}{\pgfqpoint{2.955114in}{1.872874in}}{\pgfqpoint{2.946878in}{1.872874in}}%
\pgfpathcurveto{\pgfqpoint{2.938641in}{1.872874in}}{\pgfqpoint{2.930741in}{1.869602in}}{\pgfqpoint{2.924917in}{1.863778in}}%
\pgfpathcurveto{\pgfqpoint{2.919093in}{1.857954in}}{\pgfqpoint{2.915821in}{1.850054in}}{\pgfqpoint{2.915821in}{1.841818in}}%
\pgfpathcurveto{\pgfqpoint{2.915821in}{1.833581in}}{\pgfqpoint{2.919093in}{1.825681in}}{\pgfqpoint{2.924917in}{1.819857in}}%
\pgfpathcurveto{\pgfqpoint{2.930741in}{1.814034in}}{\pgfqpoint{2.938641in}{1.810761in}}{\pgfqpoint{2.946878in}{1.810761in}}%
\pgfpathclose%
\pgfusepath{stroke,fill}%
\end{pgfscope}%
\begin{pgfscope}%
\pgfpathrectangle{\pgfqpoint{0.100000in}{0.212622in}}{\pgfqpoint{3.696000in}{3.696000in}}%
\pgfusepath{clip}%
\pgfsetbuttcap%
\pgfsetroundjoin%
\definecolor{currentfill}{rgb}{0.121569,0.466667,0.705882}%
\pgfsetfillcolor{currentfill}%
\pgfsetfillopacity{0.728207}%
\pgfsetlinewidth{1.003750pt}%
\definecolor{currentstroke}{rgb}{0.121569,0.466667,0.705882}%
\pgfsetstrokecolor{currentstroke}%
\pgfsetstrokeopacity{0.728207}%
\pgfsetdash{}{0pt}%
\pgfpathmoveto{\pgfqpoint{0.867546in}{2.167275in}}%
\pgfpathcurveto{\pgfqpoint{0.875782in}{2.167275in}}{\pgfqpoint{0.883682in}{2.170547in}}{\pgfqpoint{0.889506in}{2.176371in}}%
\pgfpathcurveto{\pgfqpoint{0.895330in}{2.182195in}}{\pgfqpoint{0.898602in}{2.190095in}}{\pgfqpoint{0.898602in}{2.198332in}}%
\pgfpathcurveto{\pgfqpoint{0.898602in}{2.206568in}}{\pgfqpoint{0.895330in}{2.214468in}}{\pgfqpoint{0.889506in}{2.220292in}}%
\pgfpathcurveto{\pgfqpoint{0.883682in}{2.226116in}}{\pgfqpoint{0.875782in}{2.229388in}}{\pgfqpoint{0.867546in}{2.229388in}}%
\pgfpathcurveto{\pgfqpoint{0.859310in}{2.229388in}}{\pgfqpoint{0.851409in}{2.226116in}}{\pgfqpoint{0.845586in}{2.220292in}}%
\pgfpathcurveto{\pgfqpoint{0.839762in}{2.214468in}}{\pgfqpoint{0.836489in}{2.206568in}}{\pgfqpoint{0.836489in}{2.198332in}}%
\pgfpathcurveto{\pgfqpoint{0.836489in}{2.190095in}}{\pgfqpoint{0.839762in}{2.182195in}}{\pgfqpoint{0.845586in}{2.176371in}}%
\pgfpathcurveto{\pgfqpoint{0.851409in}{2.170547in}}{\pgfqpoint{0.859310in}{2.167275in}}{\pgfqpoint{0.867546in}{2.167275in}}%
\pgfpathclose%
\pgfusepath{stroke,fill}%
\end{pgfscope}%
\begin{pgfscope}%
\pgfpathrectangle{\pgfqpoint{0.100000in}{0.212622in}}{\pgfqpoint{3.696000in}{3.696000in}}%
\pgfusepath{clip}%
\pgfsetbuttcap%
\pgfsetroundjoin%
\definecolor{currentfill}{rgb}{0.121569,0.466667,0.705882}%
\pgfsetfillcolor{currentfill}%
\pgfsetfillopacity{0.729344}%
\pgfsetlinewidth{1.003750pt}%
\definecolor{currentstroke}{rgb}{0.121569,0.466667,0.705882}%
\pgfsetstrokecolor{currentstroke}%
\pgfsetstrokeopacity{0.729344}%
\pgfsetdash{}{0pt}%
\pgfpathmoveto{\pgfqpoint{0.865055in}{2.167446in}}%
\pgfpathcurveto{\pgfqpoint{0.873292in}{2.167446in}}{\pgfqpoint{0.881192in}{2.170718in}}{\pgfqpoint{0.887016in}{2.176542in}}%
\pgfpathcurveto{\pgfqpoint{0.892840in}{2.182366in}}{\pgfqpoint{0.896112in}{2.190266in}}{\pgfqpoint{0.896112in}{2.198502in}}%
\pgfpathcurveto{\pgfqpoint{0.896112in}{2.206738in}}{\pgfqpoint{0.892840in}{2.214638in}}{\pgfqpoint{0.887016in}{2.220462in}}%
\pgfpathcurveto{\pgfqpoint{0.881192in}{2.226286in}}{\pgfqpoint{0.873292in}{2.229559in}}{\pgfqpoint{0.865055in}{2.229559in}}%
\pgfpathcurveto{\pgfqpoint{0.856819in}{2.229559in}}{\pgfqpoint{0.848919in}{2.226286in}}{\pgfqpoint{0.843095in}{2.220462in}}%
\pgfpathcurveto{\pgfqpoint{0.837271in}{2.214638in}}{\pgfqpoint{0.833999in}{2.206738in}}{\pgfqpoint{0.833999in}{2.198502in}}%
\pgfpathcurveto{\pgfqpoint{0.833999in}{2.190266in}}{\pgfqpoint{0.837271in}{2.182366in}}{\pgfqpoint{0.843095in}{2.176542in}}%
\pgfpathcurveto{\pgfqpoint{0.848919in}{2.170718in}}{\pgfqpoint{0.856819in}{2.167446in}}{\pgfqpoint{0.865055in}{2.167446in}}%
\pgfpathclose%
\pgfusepath{stroke,fill}%
\end{pgfscope}%
\begin{pgfscope}%
\pgfpathrectangle{\pgfqpoint{0.100000in}{0.212622in}}{\pgfqpoint{3.696000in}{3.696000in}}%
\pgfusepath{clip}%
\pgfsetbuttcap%
\pgfsetroundjoin%
\definecolor{currentfill}{rgb}{0.121569,0.466667,0.705882}%
\pgfsetfillcolor{currentfill}%
\pgfsetfillopacity{0.730633}%
\pgfsetlinewidth{1.003750pt}%
\definecolor{currentstroke}{rgb}{0.121569,0.466667,0.705882}%
\pgfsetstrokecolor{currentstroke}%
\pgfsetstrokeopacity{0.730633}%
\pgfsetdash{}{0pt}%
\pgfpathmoveto{\pgfqpoint{2.941350in}{1.811532in}}%
\pgfpathcurveto{\pgfqpoint{2.949586in}{1.811532in}}{\pgfqpoint{2.957486in}{1.814804in}}{\pgfqpoint{2.963310in}{1.820628in}}%
\pgfpathcurveto{\pgfqpoint{2.969134in}{1.826452in}}{\pgfqpoint{2.972407in}{1.834352in}}{\pgfqpoint{2.972407in}{1.842588in}}%
\pgfpathcurveto{\pgfqpoint{2.972407in}{1.850825in}}{\pgfqpoint{2.969134in}{1.858725in}}{\pgfqpoint{2.963310in}{1.864549in}}%
\pgfpathcurveto{\pgfqpoint{2.957486in}{1.870373in}}{\pgfqpoint{2.949586in}{1.873645in}}{\pgfqpoint{2.941350in}{1.873645in}}%
\pgfpathcurveto{\pgfqpoint{2.933114in}{1.873645in}}{\pgfqpoint{2.925214in}{1.870373in}}{\pgfqpoint{2.919390in}{1.864549in}}%
\pgfpathcurveto{\pgfqpoint{2.913566in}{1.858725in}}{\pgfqpoint{2.910294in}{1.850825in}}{\pgfqpoint{2.910294in}{1.842588in}}%
\pgfpathcurveto{\pgfqpoint{2.910294in}{1.834352in}}{\pgfqpoint{2.913566in}{1.826452in}}{\pgfqpoint{2.919390in}{1.820628in}}%
\pgfpathcurveto{\pgfqpoint{2.925214in}{1.814804in}}{\pgfqpoint{2.933114in}{1.811532in}}{\pgfqpoint{2.941350in}{1.811532in}}%
\pgfpathclose%
\pgfusepath{stroke,fill}%
\end{pgfscope}%
\begin{pgfscope}%
\pgfpathrectangle{\pgfqpoint{0.100000in}{0.212622in}}{\pgfqpoint{3.696000in}{3.696000in}}%
\pgfusepath{clip}%
\pgfsetbuttcap%
\pgfsetroundjoin%
\definecolor{currentfill}{rgb}{0.121569,0.466667,0.705882}%
\pgfsetfillcolor{currentfill}%
\pgfsetfillopacity{0.731366}%
\pgfsetlinewidth{1.003750pt}%
\definecolor{currentstroke}{rgb}{0.121569,0.466667,0.705882}%
\pgfsetstrokecolor{currentstroke}%
\pgfsetstrokeopacity{0.731366}%
\pgfsetdash{}{0pt}%
\pgfpathmoveto{\pgfqpoint{0.860095in}{2.167932in}}%
\pgfpathcurveto{\pgfqpoint{0.868331in}{2.167932in}}{\pgfqpoint{0.876231in}{2.171204in}}{\pgfqpoint{0.882055in}{2.177028in}}%
\pgfpathcurveto{\pgfqpoint{0.887879in}{2.182852in}}{\pgfqpoint{0.891151in}{2.190752in}}{\pgfqpoint{0.891151in}{2.198989in}}%
\pgfpathcurveto{\pgfqpoint{0.891151in}{2.207225in}}{\pgfqpoint{0.887879in}{2.215125in}}{\pgfqpoint{0.882055in}{2.220949in}}%
\pgfpathcurveto{\pgfqpoint{0.876231in}{2.226773in}}{\pgfqpoint{0.868331in}{2.230045in}}{\pgfqpoint{0.860095in}{2.230045in}}%
\pgfpathcurveto{\pgfqpoint{0.851859in}{2.230045in}}{\pgfqpoint{0.843959in}{2.226773in}}{\pgfqpoint{0.838135in}{2.220949in}}%
\pgfpathcurveto{\pgfqpoint{0.832311in}{2.215125in}}{\pgfqpoint{0.829038in}{2.207225in}}{\pgfqpoint{0.829038in}{2.198989in}}%
\pgfpathcurveto{\pgfqpoint{0.829038in}{2.190752in}}{\pgfqpoint{0.832311in}{2.182852in}}{\pgfqpoint{0.838135in}{2.177028in}}%
\pgfpathcurveto{\pgfqpoint{0.843959in}{2.171204in}}{\pgfqpoint{0.851859in}{2.167932in}}{\pgfqpoint{0.860095in}{2.167932in}}%
\pgfpathclose%
\pgfusepath{stroke,fill}%
\end{pgfscope}%
\begin{pgfscope}%
\pgfpathrectangle{\pgfqpoint{0.100000in}{0.212622in}}{\pgfqpoint{3.696000in}{3.696000in}}%
\pgfusepath{clip}%
\pgfsetbuttcap%
\pgfsetroundjoin%
\definecolor{currentfill}{rgb}{0.121569,0.466667,0.705882}%
\pgfsetfillcolor{currentfill}%
\pgfsetfillopacity{0.733270}%
\pgfsetlinewidth{1.003750pt}%
\definecolor{currentstroke}{rgb}{0.121569,0.466667,0.705882}%
\pgfsetstrokecolor{currentstroke}%
\pgfsetstrokeopacity{0.733270}%
\pgfsetdash{}{0pt}%
\pgfpathmoveto{\pgfqpoint{0.857202in}{2.168180in}}%
\pgfpathcurveto{\pgfqpoint{0.865438in}{2.168180in}}{\pgfqpoint{0.873338in}{2.171452in}}{\pgfqpoint{0.879162in}{2.177276in}}%
\pgfpathcurveto{\pgfqpoint{0.884986in}{2.183100in}}{\pgfqpoint{0.888259in}{2.191000in}}{\pgfqpoint{0.888259in}{2.199236in}}%
\pgfpathcurveto{\pgfqpoint{0.888259in}{2.207473in}}{\pgfqpoint{0.884986in}{2.215373in}}{\pgfqpoint{0.879162in}{2.221197in}}%
\pgfpathcurveto{\pgfqpoint{0.873338in}{2.227021in}}{\pgfqpoint{0.865438in}{2.230293in}}{\pgfqpoint{0.857202in}{2.230293in}}%
\pgfpathcurveto{\pgfqpoint{0.848966in}{2.230293in}}{\pgfqpoint{0.841066in}{2.227021in}}{\pgfqpoint{0.835242in}{2.221197in}}%
\pgfpathcurveto{\pgfqpoint{0.829418in}{2.215373in}}{\pgfqpoint{0.826146in}{2.207473in}}{\pgfqpoint{0.826146in}{2.199236in}}%
\pgfpathcurveto{\pgfqpoint{0.826146in}{2.191000in}}{\pgfqpoint{0.829418in}{2.183100in}}{\pgfqpoint{0.835242in}{2.177276in}}%
\pgfpathcurveto{\pgfqpoint{0.841066in}{2.171452in}}{\pgfqpoint{0.848966in}{2.168180in}}{\pgfqpoint{0.857202in}{2.168180in}}%
\pgfpathclose%
\pgfusepath{stroke,fill}%
\end{pgfscope}%
\begin{pgfscope}%
\pgfpathrectangle{\pgfqpoint{0.100000in}{0.212622in}}{\pgfqpoint{3.696000in}{3.696000in}}%
\pgfusepath{clip}%
\pgfsetbuttcap%
\pgfsetroundjoin%
\definecolor{currentfill}{rgb}{0.121569,0.466667,0.705882}%
\pgfsetfillcolor{currentfill}%
\pgfsetfillopacity{0.733702}%
\pgfsetlinewidth{1.003750pt}%
\definecolor{currentstroke}{rgb}{0.121569,0.466667,0.705882}%
\pgfsetstrokecolor{currentstroke}%
\pgfsetstrokeopacity{0.733702}%
\pgfsetdash{}{0pt}%
\pgfpathmoveto{\pgfqpoint{2.934237in}{1.812883in}}%
\pgfpathcurveto{\pgfqpoint{2.942473in}{1.812883in}}{\pgfqpoint{2.950373in}{1.816155in}}{\pgfqpoint{2.956197in}{1.821979in}}%
\pgfpathcurveto{\pgfqpoint{2.962021in}{1.827803in}}{\pgfqpoint{2.965293in}{1.835703in}}{\pgfqpoint{2.965293in}{1.843939in}}%
\pgfpathcurveto{\pgfqpoint{2.965293in}{1.852176in}}{\pgfqpoint{2.962021in}{1.860076in}}{\pgfqpoint{2.956197in}{1.865900in}}%
\pgfpathcurveto{\pgfqpoint{2.950373in}{1.871724in}}{\pgfqpoint{2.942473in}{1.874996in}}{\pgfqpoint{2.934237in}{1.874996in}}%
\pgfpathcurveto{\pgfqpoint{2.926000in}{1.874996in}}{\pgfqpoint{2.918100in}{1.871724in}}{\pgfqpoint{2.912276in}{1.865900in}}%
\pgfpathcurveto{\pgfqpoint{2.906452in}{1.860076in}}{\pgfqpoint{2.903180in}{1.852176in}}{\pgfqpoint{2.903180in}{1.843939in}}%
\pgfpathcurveto{\pgfqpoint{2.903180in}{1.835703in}}{\pgfqpoint{2.906452in}{1.827803in}}{\pgfqpoint{2.912276in}{1.821979in}}%
\pgfpathcurveto{\pgfqpoint{2.918100in}{1.816155in}}{\pgfqpoint{2.926000in}{1.812883in}}{\pgfqpoint{2.934237in}{1.812883in}}%
\pgfpathclose%
\pgfusepath{stroke,fill}%
\end{pgfscope}%
\begin{pgfscope}%
\pgfpathrectangle{\pgfqpoint{0.100000in}{0.212622in}}{\pgfqpoint{3.696000in}{3.696000in}}%
\pgfusepath{clip}%
\pgfsetbuttcap%
\pgfsetroundjoin%
\definecolor{currentfill}{rgb}{0.121569,0.466667,0.705882}%
\pgfsetfillcolor{currentfill}%
\pgfsetfillopacity{0.734826}%
\pgfsetlinewidth{1.003750pt}%
\definecolor{currentstroke}{rgb}{0.121569,0.466667,0.705882}%
\pgfsetstrokecolor{currentstroke}%
\pgfsetstrokeopacity{0.734826}%
\pgfsetdash{}{0pt}%
\pgfpathmoveto{\pgfqpoint{0.853097in}{2.168752in}}%
\pgfpathcurveto{\pgfqpoint{0.861334in}{2.168752in}}{\pgfqpoint{0.869234in}{2.172024in}}{\pgfqpoint{0.875058in}{2.177848in}}%
\pgfpathcurveto{\pgfqpoint{0.880882in}{2.183672in}}{\pgfqpoint{0.884154in}{2.191572in}}{\pgfqpoint{0.884154in}{2.199808in}}%
\pgfpathcurveto{\pgfqpoint{0.884154in}{2.208044in}}{\pgfqpoint{0.880882in}{2.215944in}}{\pgfqpoint{0.875058in}{2.221768in}}%
\pgfpathcurveto{\pgfqpoint{0.869234in}{2.227592in}}{\pgfqpoint{0.861334in}{2.230865in}}{\pgfqpoint{0.853097in}{2.230865in}}%
\pgfpathcurveto{\pgfqpoint{0.844861in}{2.230865in}}{\pgfqpoint{0.836961in}{2.227592in}}{\pgfqpoint{0.831137in}{2.221768in}}%
\pgfpathcurveto{\pgfqpoint{0.825313in}{2.215944in}}{\pgfqpoint{0.822041in}{2.208044in}}{\pgfqpoint{0.822041in}{2.199808in}}%
\pgfpathcurveto{\pgfqpoint{0.822041in}{2.191572in}}{\pgfqpoint{0.825313in}{2.183672in}}{\pgfqpoint{0.831137in}{2.177848in}}%
\pgfpathcurveto{\pgfqpoint{0.836961in}{2.172024in}}{\pgfqpoint{0.844861in}{2.168752in}}{\pgfqpoint{0.853097in}{2.168752in}}%
\pgfpathclose%
\pgfusepath{stroke,fill}%
\end{pgfscope}%
\begin{pgfscope}%
\pgfpathrectangle{\pgfqpoint{0.100000in}{0.212622in}}{\pgfqpoint{3.696000in}{3.696000in}}%
\pgfusepath{clip}%
\pgfsetbuttcap%
\pgfsetroundjoin%
\definecolor{currentfill}{rgb}{0.121569,0.466667,0.705882}%
\pgfsetfillcolor{currentfill}%
\pgfsetfillopacity{0.736126}%
\pgfsetlinewidth{1.003750pt}%
\definecolor{currentstroke}{rgb}{0.121569,0.466667,0.705882}%
\pgfsetstrokecolor{currentstroke}%
\pgfsetstrokeopacity{0.736126}%
\pgfsetdash{}{0pt}%
\pgfpathmoveto{\pgfqpoint{0.852005in}{2.169040in}}%
\pgfpathcurveto{\pgfqpoint{0.860241in}{2.169040in}}{\pgfqpoint{0.868141in}{2.172312in}}{\pgfqpoint{0.873965in}{2.178136in}}%
\pgfpathcurveto{\pgfqpoint{0.879789in}{2.183960in}}{\pgfqpoint{0.883061in}{2.191860in}}{\pgfqpoint{0.883061in}{2.200096in}}%
\pgfpathcurveto{\pgfqpoint{0.883061in}{2.208332in}}{\pgfqpoint{0.879789in}{2.216232in}}{\pgfqpoint{0.873965in}{2.222056in}}%
\pgfpathcurveto{\pgfqpoint{0.868141in}{2.227880in}}{\pgfqpoint{0.860241in}{2.231153in}}{\pgfqpoint{0.852005in}{2.231153in}}%
\pgfpathcurveto{\pgfqpoint{0.843768in}{2.231153in}}{\pgfqpoint{0.835868in}{2.227880in}}{\pgfqpoint{0.830044in}{2.222056in}}%
\pgfpathcurveto{\pgfqpoint{0.824220in}{2.216232in}}{\pgfqpoint{0.820948in}{2.208332in}}{\pgfqpoint{0.820948in}{2.200096in}}%
\pgfpathcurveto{\pgfqpoint{0.820948in}{2.191860in}}{\pgfqpoint{0.824220in}{2.183960in}}{\pgfqpoint{0.830044in}{2.178136in}}%
\pgfpathcurveto{\pgfqpoint{0.835868in}{2.172312in}}{\pgfqpoint{0.843768in}{2.169040in}}{\pgfqpoint{0.852005in}{2.169040in}}%
\pgfpathclose%
\pgfusepath{stroke,fill}%
\end{pgfscope}%
\begin{pgfscope}%
\pgfpathrectangle{\pgfqpoint{0.100000in}{0.212622in}}{\pgfqpoint{3.696000in}{3.696000in}}%
\pgfusepath{clip}%
\pgfsetbuttcap%
\pgfsetroundjoin%
\definecolor{currentfill}{rgb}{0.121569,0.466667,0.705882}%
\pgfsetfillcolor{currentfill}%
\pgfsetfillopacity{0.737181}%
\pgfsetlinewidth{1.003750pt}%
\definecolor{currentstroke}{rgb}{0.121569,0.466667,0.705882}%
\pgfsetstrokecolor{currentstroke}%
\pgfsetstrokeopacity{0.737181}%
\pgfsetdash{}{0pt}%
\pgfpathmoveto{\pgfqpoint{0.849687in}{2.169209in}}%
\pgfpathcurveto{\pgfqpoint{0.857923in}{2.169209in}}{\pgfqpoint{0.865823in}{2.172481in}}{\pgfqpoint{0.871647in}{2.178305in}}%
\pgfpathcurveto{\pgfqpoint{0.877471in}{2.184129in}}{\pgfqpoint{0.880743in}{2.192029in}}{\pgfqpoint{0.880743in}{2.200265in}}%
\pgfpathcurveto{\pgfqpoint{0.880743in}{2.208502in}}{\pgfqpoint{0.877471in}{2.216402in}}{\pgfqpoint{0.871647in}{2.222226in}}%
\pgfpathcurveto{\pgfqpoint{0.865823in}{2.228050in}}{\pgfqpoint{0.857923in}{2.231322in}}{\pgfqpoint{0.849687in}{2.231322in}}%
\pgfpathcurveto{\pgfqpoint{0.841450in}{2.231322in}}{\pgfqpoint{0.833550in}{2.228050in}}{\pgfqpoint{0.827726in}{2.222226in}}%
\pgfpathcurveto{\pgfqpoint{0.821903in}{2.216402in}}{\pgfqpoint{0.818630in}{2.208502in}}{\pgfqpoint{0.818630in}{2.200265in}}%
\pgfpathcurveto{\pgfqpoint{0.818630in}{2.192029in}}{\pgfqpoint{0.821903in}{2.184129in}}{\pgfqpoint{0.827726in}{2.178305in}}%
\pgfpathcurveto{\pgfqpoint{0.833550in}{2.172481in}}{\pgfqpoint{0.841450in}{2.169209in}}{\pgfqpoint{0.849687in}{2.169209in}}%
\pgfpathclose%
\pgfusepath{stroke,fill}%
\end{pgfscope}%
\begin{pgfscope}%
\pgfpathrectangle{\pgfqpoint{0.100000in}{0.212622in}}{\pgfqpoint{3.696000in}{3.696000in}}%
\pgfusepath{clip}%
\pgfsetbuttcap%
\pgfsetroundjoin%
\definecolor{currentfill}{rgb}{0.121569,0.466667,0.705882}%
\pgfsetfillcolor{currentfill}%
\pgfsetfillopacity{0.737773}%
\pgfsetlinewidth{1.003750pt}%
\definecolor{currentstroke}{rgb}{0.121569,0.466667,0.705882}%
\pgfsetstrokecolor{currentstroke}%
\pgfsetstrokeopacity{0.737773}%
\pgfsetdash{}{0pt}%
\pgfpathmoveto{\pgfqpoint{2.929615in}{1.812885in}}%
\pgfpathcurveto{\pgfqpoint{2.937851in}{1.812885in}}{\pgfqpoint{2.945751in}{1.816157in}}{\pgfqpoint{2.951575in}{1.821981in}}%
\pgfpathcurveto{\pgfqpoint{2.957399in}{1.827805in}}{\pgfqpoint{2.960671in}{1.835705in}}{\pgfqpoint{2.960671in}{1.843941in}}%
\pgfpathcurveto{\pgfqpoint{2.960671in}{1.852177in}}{\pgfqpoint{2.957399in}{1.860078in}}{\pgfqpoint{2.951575in}{1.865901in}}%
\pgfpathcurveto{\pgfqpoint{2.945751in}{1.871725in}}{\pgfqpoint{2.937851in}{1.874998in}}{\pgfqpoint{2.929615in}{1.874998in}}%
\pgfpathcurveto{\pgfqpoint{2.921378in}{1.874998in}}{\pgfqpoint{2.913478in}{1.871725in}}{\pgfqpoint{2.907654in}{1.865901in}}%
\pgfpathcurveto{\pgfqpoint{2.901831in}{1.860078in}}{\pgfqpoint{2.898558in}{1.852177in}}{\pgfqpoint{2.898558in}{1.843941in}}%
\pgfpathcurveto{\pgfqpoint{2.898558in}{1.835705in}}{\pgfqpoint{2.901831in}{1.827805in}}{\pgfqpoint{2.907654in}{1.821981in}}%
\pgfpathcurveto{\pgfqpoint{2.913478in}{1.816157in}}{\pgfqpoint{2.921378in}{1.812885in}}{\pgfqpoint{2.929615in}{1.812885in}}%
\pgfpathclose%
\pgfusepath{stroke,fill}%
\end{pgfscope}%
\begin{pgfscope}%
\pgfpathrectangle{\pgfqpoint{0.100000in}{0.212622in}}{\pgfqpoint{3.696000in}{3.696000in}}%
\pgfusepath{clip}%
\pgfsetbuttcap%
\pgfsetroundjoin%
\definecolor{currentfill}{rgb}{0.121569,0.466667,0.705882}%
\pgfsetfillcolor{currentfill}%
\pgfsetfillopacity{0.739087}%
\pgfsetlinewidth{1.003750pt}%
\definecolor{currentstroke}{rgb}{0.121569,0.466667,0.705882}%
\pgfsetstrokecolor{currentstroke}%
\pgfsetstrokeopacity{0.739087}%
\pgfsetdash{}{0pt}%
\pgfpathmoveto{\pgfqpoint{0.845670in}{2.169209in}}%
\pgfpathcurveto{\pgfqpoint{0.853906in}{2.169209in}}{\pgfqpoint{0.861807in}{2.172481in}}{\pgfqpoint{0.867630in}{2.178305in}}%
\pgfpathcurveto{\pgfqpoint{0.873454in}{2.184129in}}{\pgfqpoint{0.876727in}{2.192029in}}{\pgfqpoint{0.876727in}{2.200266in}}%
\pgfpathcurveto{\pgfqpoint{0.876727in}{2.208502in}}{\pgfqpoint{0.873454in}{2.216402in}}{\pgfqpoint{0.867630in}{2.222226in}}%
\pgfpathcurveto{\pgfqpoint{0.861807in}{2.228050in}}{\pgfqpoint{0.853906in}{2.231322in}}{\pgfqpoint{0.845670in}{2.231322in}}%
\pgfpathcurveto{\pgfqpoint{0.837434in}{2.231322in}}{\pgfqpoint{0.829534in}{2.228050in}}{\pgfqpoint{0.823710in}{2.222226in}}%
\pgfpathcurveto{\pgfqpoint{0.817886in}{2.216402in}}{\pgfqpoint{0.814614in}{2.208502in}}{\pgfqpoint{0.814614in}{2.200266in}}%
\pgfpathcurveto{\pgfqpoint{0.814614in}{2.192029in}}{\pgfqpoint{0.817886in}{2.184129in}}{\pgfqpoint{0.823710in}{2.178305in}}%
\pgfpathcurveto{\pgfqpoint{0.829534in}{2.172481in}}{\pgfqpoint{0.837434in}{2.169209in}}{\pgfqpoint{0.845670in}{2.169209in}}%
\pgfpathclose%
\pgfusepath{stroke,fill}%
\end{pgfscope}%
\begin{pgfscope}%
\pgfpathrectangle{\pgfqpoint{0.100000in}{0.212622in}}{\pgfqpoint{3.696000in}{3.696000in}}%
\pgfusepath{clip}%
\pgfsetbuttcap%
\pgfsetroundjoin%
\definecolor{currentfill}{rgb}{0.121569,0.466667,0.705882}%
\pgfsetfillcolor{currentfill}%
\pgfsetfillopacity{0.740038}%
\pgfsetlinewidth{1.003750pt}%
\definecolor{currentstroke}{rgb}{0.121569,0.466667,0.705882}%
\pgfsetstrokecolor{currentstroke}%
\pgfsetstrokeopacity{0.740038}%
\pgfsetdash{}{0pt}%
\pgfpathmoveto{\pgfqpoint{2.927454in}{1.812793in}}%
\pgfpathcurveto{\pgfqpoint{2.935690in}{1.812793in}}{\pgfqpoint{2.943590in}{1.816066in}}{\pgfqpoint{2.949414in}{1.821890in}}%
\pgfpathcurveto{\pgfqpoint{2.955238in}{1.827714in}}{\pgfqpoint{2.958511in}{1.835614in}}{\pgfqpoint{2.958511in}{1.843850in}}%
\pgfpathcurveto{\pgfqpoint{2.958511in}{1.852086in}}{\pgfqpoint{2.955238in}{1.859986in}}{\pgfqpoint{2.949414in}{1.865810in}}%
\pgfpathcurveto{\pgfqpoint{2.943590in}{1.871634in}}{\pgfqpoint{2.935690in}{1.874906in}}{\pgfqpoint{2.927454in}{1.874906in}}%
\pgfpathcurveto{\pgfqpoint{2.919218in}{1.874906in}}{\pgfqpoint{2.911318in}{1.871634in}}{\pgfqpoint{2.905494in}{1.865810in}}%
\pgfpathcurveto{\pgfqpoint{2.899670in}{1.859986in}}{\pgfqpoint{2.896398in}{1.852086in}}{\pgfqpoint{2.896398in}{1.843850in}}%
\pgfpathcurveto{\pgfqpoint{2.896398in}{1.835614in}}{\pgfqpoint{2.899670in}{1.827714in}}{\pgfqpoint{2.905494in}{1.821890in}}%
\pgfpathcurveto{\pgfqpoint{2.911318in}{1.816066in}}{\pgfqpoint{2.919218in}{1.812793in}}{\pgfqpoint{2.927454in}{1.812793in}}%
\pgfpathclose%
\pgfusepath{stroke,fill}%
\end{pgfscope}%
\begin{pgfscope}%
\pgfpathrectangle{\pgfqpoint{0.100000in}{0.212622in}}{\pgfqpoint{3.696000in}{3.696000in}}%
\pgfusepath{clip}%
\pgfsetbuttcap%
\pgfsetroundjoin%
\definecolor{currentfill}{rgb}{0.121569,0.466667,0.705882}%
\pgfsetfillcolor{currentfill}%
\pgfsetfillopacity{0.740894}%
\pgfsetlinewidth{1.003750pt}%
\definecolor{currentstroke}{rgb}{0.121569,0.466667,0.705882}%
\pgfsetstrokecolor{currentstroke}%
\pgfsetstrokeopacity{0.740894}%
\pgfsetdash{}{0pt}%
\pgfpathmoveto{\pgfqpoint{0.842755in}{2.169345in}}%
\pgfpathcurveto{\pgfqpoint{0.850991in}{2.169345in}}{\pgfqpoint{0.858891in}{2.172618in}}{\pgfqpoint{0.864715in}{2.178442in}}%
\pgfpathcurveto{\pgfqpoint{0.870539in}{2.184265in}}{\pgfqpoint{0.873812in}{2.192165in}}{\pgfqpoint{0.873812in}{2.200402in}}%
\pgfpathcurveto{\pgfqpoint{0.873812in}{2.208638in}}{\pgfqpoint{0.870539in}{2.216538in}}{\pgfqpoint{0.864715in}{2.222362in}}%
\pgfpathcurveto{\pgfqpoint{0.858891in}{2.228186in}}{\pgfqpoint{0.850991in}{2.231458in}}{\pgfqpoint{0.842755in}{2.231458in}}%
\pgfpathcurveto{\pgfqpoint{0.834519in}{2.231458in}}{\pgfqpoint{0.826619in}{2.228186in}}{\pgfqpoint{0.820795in}{2.222362in}}%
\pgfpathcurveto{\pgfqpoint{0.814971in}{2.216538in}}{\pgfqpoint{0.811699in}{2.208638in}}{\pgfqpoint{0.811699in}{2.200402in}}%
\pgfpathcurveto{\pgfqpoint{0.811699in}{2.192165in}}{\pgfqpoint{0.814971in}{2.184265in}}{\pgfqpoint{0.820795in}{2.178442in}}%
\pgfpathcurveto{\pgfqpoint{0.826619in}{2.172618in}}{\pgfqpoint{0.834519in}{2.169345in}}{\pgfqpoint{0.842755in}{2.169345in}}%
\pgfpathclose%
\pgfusepath{stroke,fill}%
\end{pgfscope}%
\begin{pgfscope}%
\pgfpathrectangle{\pgfqpoint{0.100000in}{0.212622in}}{\pgfqpoint{3.696000in}{3.696000in}}%
\pgfusepath{clip}%
\pgfsetbuttcap%
\pgfsetroundjoin%
\definecolor{currentfill}{rgb}{0.121569,0.466667,0.705882}%
\pgfsetfillcolor{currentfill}%
\pgfsetfillopacity{0.742121}%
\pgfsetlinewidth{1.003750pt}%
\definecolor{currentstroke}{rgb}{0.121569,0.466667,0.705882}%
\pgfsetstrokecolor{currentstroke}%
\pgfsetstrokeopacity{0.742121}%
\pgfsetdash{}{0pt}%
\pgfpathmoveto{\pgfqpoint{0.839521in}{2.169616in}}%
\pgfpathcurveto{\pgfqpoint{0.847757in}{2.169616in}}{\pgfqpoint{0.855657in}{2.172889in}}{\pgfqpoint{0.861481in}{2.178712in}}%
\pgfpathcurveto{\pgfqpoint{0.867305in}{2.184536in}}{\pgfqpoint{0.870577in}{2.192436in}}{\pgfqpoint{0.870577in}{2.200673in}}%
\pgfpathcurveto{\pgfqpoint{0.870577in}{2.208909in}}{\pgfqpoint{0.867305in}{2.216809in}}{\pgfqpoint{0.861481in}{2.222633in}}%
\pgfpathcurveto{\pgfqpoint{0.855657in}{2.228457in}}{\pgfqpoint{0.847757in}{2.231729in}}{\pgfqpoint{0.839521in}{2.231729in}}%
\pgfpathcurveto{\pgfqpoint{0.831284in}{2.231729in}}{\pgfqpoint{0.823384in}{2.228457in}}{\pgfqpoint{0.817560in}{2.222633in}}%
\pgfpathcurveto{\pgfqpoint{0.811737in}{2.216809in}}{\pgfqpoint{0.808464in}{2.208909in}}{\pgfqpoint{0.808464in}{2.200673in}}%
\pgfpathcurveto{\pgfqpoint{0.808464in}{2.192436in}}{\pgfqpoint{0.811737in}{2.184536in}}{\pgfqpoint{0.817560in}{2.178712in}}%
\pgfpathcurveto{\pgfqpoint{0.823384in}{2.172889in}}{\pgfqpoint{0.831284in}{2.169616in}}{\pgfqpoint{0.839521in}{2.169616in}}%
\pgfpathclose%
\pgfusepath{stroke,fill}%
\end{pgfscope}%
\begin{pgfscope}%
\pgfpathrectangle{\pgfqpoint{0.100000in}{0.212622in}}{\pgfqpoint{3.696000in}{3.696000in}}%
\pgfusepath{clip}%
\pgfsetbuttcap%
\pgfsetroundjoin%
\definecolor{currentfill}{rgb}{0.121569,0.466667,0.705882}%
\pgfsetfillcolor{currentfill}%
\pgfsetfillopacity{0.742382}%
\pgfsetlinewidth{1.003750pt}%
\definecolor{currentstroke}{rgb}{0.121569,0.466667,0.705882}%
\pgfsetstrokecolor{currentstroke}%
\pgfsetstrokeopacity{0.742382}%
\pgfsetdash{}{0pt}%
\pgfpathmoveto{\pgfqpoint{2.922363in}{1.813675in}}%
\pgfpathcurveto{\pgfqpoint{2.930599in}{1.813675in}}{\pgfqpoint{2.938499in}{1.816948in}}{\pgfqpoint{2.944323in}{1.822772in}}%
\pgfpathcurveto{\pgfqpoint{2.950147in}{1.828596in}}{\pgfqpoint{2.953419in}{1.836496in}}{\pgfqpoint{2.953419in}{1.844732in}}%
\pgfpathcurveto{\pgfqpoint{2.953419in}{1.852968in}}{\pgfqpoint{2.950147in}{1.860868in}}{\pgfqpoint{2.944323in}{1.866692in}}%
\pgfpathcurveto{\pgfqpoint{2.938499in}{1.872516in}}{\pgfqpoint{2.930599in}{1.875788in}}{\pgfqpoint{2.922363in}{1.875788in}}%
\pgfpathcurveto{\pgfqpoint{2.914127in}{1.875788in}}{\pgfqpoint{2.906227in}{1.872516in}}{\pgfqpoint{2.900403in}{1.866692in}}%
\pgfpathcurveto{\pgfqpoint{2.894579in}{1.860868in}}{\pgfqpoint{2.891306in}{1.852968in}}{\pgfqpoint{2.891306in}{1.844732in}}%
\pgfpathcurveto{\pgfqpoint{2.891306in}{1.836496in}}{\pgfqpoint{2.894579in}{1.828596in}}{\pgfqpoint{2.900403in}{1.822772in}}%
\pgfpathcurveto{\pgfqpoint{2.906227in}{1.816948in}}{\pgfqpoint{2.914127in}{1.813675in}}{\pgfqpoint{2.922363in}{1.813675in}}%
\pgfpathclose%
\pgfusepath{stroke,fill}%
\end{pgfscope}%
\begin{pgfscope}%
\pgfpathrectangle{\pgfqpoint{0.100000in}{0.212622in}}{\pgfqpoint{3.696000in}{3.696000in}}%
\pgfusepath{clip}%
\pgfsetbuttcap%
\pgfsetroundjoin%
\definecolor{currentfill}{rgb}{0.121569,0.466667,0.705882}%
\pgfsetfillcolor{currentfill}%
\pgfsetfillopacity{0.743136}%
\pgfsetlinewidth{1.003750pt}%
\definecolor{currentstroke}{rgb}{0.121569,0.466667,0.705882}%
\pgfsetstrokecolor{currentstroke}%
\pgfsetstrokeopacity{0.743136}%
\pgfsetdash{}{0pt}%
\pgfpathmoveto{\pgfqpoint{0.839242in}{2.169887in}}%
\pgfpathcurveto{\pgfqpoint{0.847478in}{2.169887in}}{\pgfqpoint{0.855379in}{2.173159in}}{\pgfqpoint{0.861202in}{2.178983in}}%
\pgfpathcurveto{\pgfqpoint{0.867026in}{2.184807in}}{\pgfqpoint{0.870299in}{2.192707in}}{\pgfqpoint{0.870299in}{2.200943in}}%
\pgfpathcurveto{\pgfqpoint{0.870299in}{2.209180in}}{\pgfqpoint{0.867026in}{2.217080in}}{\pgfqpoint{0.861202in}{2.222904in}}%
\pgfpathcurveto{\pgfqpoint{0.855379in}{2.228727in}}{\pgfqpoint{0.847478in}{2.232000in}}{\pgfqpoint{0.839242in}{2.232000in}}%
\pgfpathcurveto{\pgfqpoint{0.831006in}{2.232000in}}{\pgfqpoint{0.823106in}{2.228727in}}{\pgfqpoint{0.817282in}{2.222904in}}%
\pgfpathcurveto{\pgfqpoint{0.811458in}{2.217080in}}{\pgfqpoint{0.808186in}{2.209180in}}{\pgfqpoint{0.808186in}{2.200943in}}%
\pgfpathcurveto{\pgfqpoint{0.808186in}{2.192707in}}{\pgfqpoint{0.811458in}{2.184807in}}{\pgfqpoint{0.817282in}{2.178983in}}%
\pgfpathcurveto{\pgfqpoint{0.823106in}{2.173159in}}{\pgfqpoint{0.831006in}{2.169887in}}{\pgfqpoint{0.839242in}{2.169887in}}%
\pgfpathclose%
\pgfusepath{stroke,fill}%
\end{pgfscope}%
\begin{pgfscope}%
\pgfpathrectangle{\pgfqpoint{0.100000in}{0.212622in}}{\pgfqpoint{3.696000in}{3.696000in}}%
\pgfusepath{clip}%
\pgfsetbuttcap%
\pgfsetroundjoin%
\definecolor{currentfill}{rgb}{0.121569,0.466667,0.705882}%
\pgfsetfillcolor{currentfill}%
\pgfsetfillopacity{0.743708}%
\pgfsetlinewidth{1.003750pt}%
\definecolor{currentstroke}{rgb}{0.121569,0.466667,0.705882}%
\pgfsetstrokecolor{currentstroke}%
\pgfsetstrokeopacity{0.743708}%
\pgfsetdash{}{0pt}%
\pgfpathmoveto{\pgfqpoint{2.919838in}{1.814056in}}%
\pgfpathcurveto{\pgfqpoint{2.928075in}{1.814056in}}{\pgfqpoint{2.935975in}{1.817329in}}{\pgfqpoint{2.941799in}{1.823153in}}%
\pgfpathcurveto{\pgfqpoint{2.947623in}{1.828976in}}{\pgfqpoint{2.950895in}{1.836877in}}{\pgfqpoint{2.950895in}{1.845113in}}%
\pgfpathcurveto{\pgfqpoint{2.950895in}{1.853349in}}{\pgfqpoint{2.947623in}{1.861249in}}{\pgfqpoint{2.941799in}{1.867073in}}%
\pgfpathcurveto{\pgfqpoint{2.935975in}{1.872897in}}{\pgfqpoint{2.928075in}{1.876169in}}{\pgfqpoint{2.919838in}{1.876169in}}%
\pgfpathcurveto{\pgfqpoint{2.911602in}{1.876169in}}{\pgfqpoint{2.903702in}{1.872897in}}{\pgfqpoint{2.897878in}{1.867073in}}%
\pgfpathcurveto{\pgfqpoint{2.892054in}{1.861249in}}{\pgfqpoint{2.888782in}{1.853349in}}{\pgfqpoint{2.888782in}{1.845113in}}%
\pgfpathcurveto{\pgfqpoint{2.888782in}{1.836877in}}{\pgfqpoint{2.892054in}{1.828976in}}{\pgfqpoint{2.897878in}{1.823153in}}%
\pgfpathcurveto{\pgfqpoint{2.903702in}{1.817329in}}{\pgfqpoint{2.911602in}{1.814056in}}{\pgfqpoint{2.919838in}{1.814056in}}%
\pgfpathclose%
\pgfusepath{stroke,fill}%
\end{pgfscope}%
\begin{pgfscope}%
\pgfpathrectangle{\pgfqpoint{0.100000in}{0.212622in}}{\pgfqpoint{3.696000in}{3.696000in}}%
\pgfusepath{clip}%
\pgfsetbuttcap%
\pgfsetroundjoin%
\definecolor{currentfill}{rgb}{0.121569,0.466667,0.705882}%
\pgfsetfillcolor{currentfill}%
\pgfsetfillopacity{0.743828}%
\pgfsetlinewidth{1.003750pt}%
\definecolor{currentstroke}{rgb}{0.121569,0.466667,0.705882}%
\pgfsetstrokecolor{currentstroke}%
\pgfsetstrokeopacity{0.743828}%
\pgfsetdash{}{0pt}%
\pgfpathmoveto{\pgfqpoint{0.837742in}{2.169950in}}%
\pgfpathcurveto{\pgfqpoint{0.845978in}{2.169950in}}{\pgfqpoint{0.853878in}{2.173222in}}{\pgfqpoint{0.859702in}{2.179046in}}%
\pgfpathcurveto{\pgfqpoint{0.865526in}{2.184870in}}{\pgfqpoint{0.868798in}{2.192770in}}{\pgfqpoint{0.868798in}{2.201006in}}%
\pgfpathcurveto{\pgfqpoint{0.868798in}{2.209242in}}{\pgfqpoint{0.865526in}{2.217142in}}{\pgfqpoint{0.859702in}{2.222966in}}%
\pgfpathcurveto{\pgfqpoint{0.853878in}{2.228790in}}{\pgfqpoint{0.845978in}{2.232063in}}{\pgfqpoint{0.837742in}{2.232063in}}%
\pgfpathcurveto{\pgfqpoint{0.829506in}{2.232063in}}{\pgfqpoint{0.821606in}{2.228790in}}{\pgfqpoint{0.815782in}{2.222966in}}%
\pgfpathcurveto{\pgfqpoint{0.809958in}{2.217142in}}{\pgfqpoint{0.806685in}{2.209242in}}{\pgfqpoint{0.806685in}{2.201006in}}%
\pgfpathcurveto{\pgfqpoint{0.806685in}{2.192770in}}{\pgfqpoint{0.809958in}{2.184870in}}{\pgfqpoint{0.815782in}{2.179046in}}%
\pgfpathcurveto{\pgfqpoint{0.821606in}{2.173222in}}{\pgfqpoint{0.829506in}{2.169950in}}{\pgfqpoint{0.837742in}{2.169950in}}%
\pgfpathclose%
\pgfusepath{stroke,fill}%
\end{pgfscope}%
\begin{pgfscope}%
\pgfpathrectangle{\pgfqpoint{0.100000in}{0.212622in}}{\pgfqpoint{3.696000in}{3.696000in}}%
\pgfusepath{clip}%
\pgfsetbuttcap%
\pgfsetroundjoin%
\definecolor{currentfill}{rgb}{0.121569,0.466667,0.705882}%
\pgfsetfillcolor{currentfill}%
\pgfsetfillopacity{0.745075}%
\pgfsetlinewidth{1.003750pt}%
\definecolor{currentstroke}{rgb}{0.121569,0.466667,0.705882}%
\pgfsetstrokecolor{currentstroke}%
\pgfsetstrokeopacity{0.745075}%
\pgfsetdash{}{0pt}%
\pgfpathmoveto{\pgfqpoint{0.835044in}{2.169946in}}%
\pgfpathcurveto{\pgfqpoint{0.843280in}{2.169946in}}{\pgfqpoint{0.851180in}{2.173219in}}{\pgfqpoint{0.857004in}{2.179043in}}%
\pgfpathcurveto{\pgfqpoint{0.862828in}{2.184867in}}{\pgfqpoint{0.866100in}{2.192767in}}{\pgfqpoint{0.866100in}{2.201003in}}%
\pgfpathcurveto{\pgfqpoint{0.866100in}{2.209239in}}{\pgfqpoint{0.862828in}{2.217139in}}{\pgfqpoint{0.857004in}{2.222963in}}%
\pgfpathcurveto{\pgfqpoint{0.851180in}{2.228787in}}{\pgfqpoint{0.843280in}{2.232059in}}{\pgfqpoint{0.835044in}{2.232059in}}%
\pgfpathcurveto{\pgfqpoint{0.826807in}{2.232059in}}{\pgfqpoint{0.818907in}{2.228787in}}{\pgfqpoint{0.813083in}{2.222963in}}%
\pgfpathcurveto{\pgfqpoint{0.807260in}{2.217139in}}{\pgfqpoint{0.803987in}{2.209239in}}{\pgfqpoint{0.803987in}{2.201003in}}%
\pgfpathcurveto{\pgfqpoint{0.803987in}{2.192767in}}{\pgfqpoint{0.807260in}{2.184867in}}{\pgfqpoint{0.813083in}{2.179043in}}%
\pgfpathcurveto{\pgfqpoint{0.818907in}{2.173219in}}{\pgfqpoint{0.826807in}{2.169946in}}{\pgfqpoint{0.835044in}{2.169946in}}%
\pgfpathclose%
\pgfusepath{stroke,fill}%
\end{pgfscope}%
\begin{pgfscope}%
\pgfpathrectangle{\pgfqpoint{0.100000in}{0.212622in}}{\pgfqpoint{3.696000in}{3.696000in}}%
\pgfusepath{clip}%
\pgfsetbuttcap%
\pgfsetroundjoin%
\definecolor{currentfill}{rgb}{0.121569,0.466667,0.705882}%
\pgfsetfillcolor{currentfill}%
\pgfsetfillopacity{0.745489}%
\pgfsetlinewidth{1.003750pt}%
\definecolor{currentstroke}{rgb}{0.121569,0.466667,0.705882}%
\pgfsetstrokecolor{currentstroke}%
\pgfsetstrokeopacity{0.745489}%
\pgfsetdash{}{0pt}%
\pgfpathmoveto{\pgfqpoint{2.918333in}{1.814152in}}%
\pgfpathcurveto{\pgfqpoint{2.926569in}{1.814152in}}{\pgfqpoint{2.934469in}{1.817425in}}{\pgfqpoint{2.940293in}{1.823248in}}%
\pgfpathcurveto{\pgfqpoint{2.946117in}{1.829072in}}{\pgfqpoint{2.949389in}{1.836972in}}{\pgfqpoint{2.949389in}{1.845209in}}%
\pgfpathcurveto{\pgfqpoint{2.949389in}{1.853445in}}{\pgfqpoint{2.946117in}{1.861345in}}{\pgfqpoint{2.940293in}{1.867169in}}%
\pgfpathcurveto{\pgfqpoint{2.934469in}{1.872993in}}{\pgfqpoint{2.926569in}{1.876265in}}{\pgfqpoint{2.918333in}{1.876265in}}%
\pgfpathcurveto{\pgfqpoint{2.910096in}{1.876265in}}{\pgfqpoint{2.902196in}{1.872993in}}{\pgfqpoint{2.896372in}{1.867169in}}%
\pgfpathcurveto{\pgfqpoint{2.890549in}{1.861345in}}{\pgfqpoint{2.887276in}{1.853445in}}{\pgfqpoint{2.887276in}{1.845209in}}%
\pgfpathcurveto{\pgfqpoint{2.887276in}{1.836972in}}{\pgfqpoint{2.890549in}{1.829072in}}{\pgfqpoint{2.896372in}{1.823248in}}%
\pgfpathcurveto{\pgfqpoint{2.902196in}{1.817425in}}{\pgfqpoint{2.910096in}{1.814152in}}{\pgfqpoint{2.918333in}{1.814152in}}%
\pgfpathclose%
\pgfusepath{stroke,fill}%
\end{pgfscope}%
\begin{pgfscope}%
\pgfpathrectangle{\pgfqpoint{0.100000in}{0.212622in}}{\pgfqpoint{3.696000in}{3.696000in}}%
\pgfusepath{clip}%
\pgfsetbuttcap%
\pgfsetroundjoin%
\definecolor{currentfill}{rgb}{0.121569,0.466667,0.705882}%
\pgfsetfillcolor{currentfill}%
\pgfsetfillopacity{0.746188}%
\pgfsetlinewidth{1.003750pt}%
\definecolor{currentstroke}{rgb}{0.121569,0.466667,0.705882}%
\pgfsetstrokecolor{currentstroke}%
\pgfsetstrokeopacity{0.746188}%
\pgfsetdash{}{0pt}%
\pgfpathmoveto{\pgfqpoint{0.832921in}{2.169913in}}%
\pgfpathcurveto{\pgfqpoint{0.841157in}{2.169913in}}{\pgfqpoint{0.849057in}{2.173186in}}{\pgfqpoint{0.854881in}{2.179009in}}%
\pgfpathcurveto{\pgfqpoint{0.860705in}{2.184833in}}{\pgfqpoint{0.863977in}{2.192733in}}{\pgfqpoint{0.863977in}{2.200970in}}%
\pgfpathcurveto{\pgfqpoint{0.863977in}{2.209206in}}{\pgfqpoint{0.860705in}{2.217106in}}{\pgfqpoint{0.854881in}{2.222930in}}%
\pgfpathcurveto{\pgfqpoint{0.849057in}{2.228754in}}{\pgfqpoint{0.841157in}{2.232026in}}{\pgfqpoint{0.832921in}{2.232026in}}%
\pgfpathcurveto{\pgfqpoint{0.824685in}{2.232026in}}{\pgfqpoint{0.816785in}{2.228754in}}{\pgfqpoint{0.810961in}{2.222930in}}%
\pgfpathcurveto{\pgfqpoint{0.805137in}{2.217106in}}{\pgfqpoint{0.801864in}{2.209206in}}{\pgfqpoint{0.801864in}{2.200970in}}%
\pgfpathcurveto{\pgfqpoint{0.801864in}{2.192733in}}{\pgfqpoint{0.805137in}{2.184833in}}{\pgfqpoint{0.810961in}{2.179009in}}%
\pgfpathcurveto{\pgfqpoint{0.816785in}{2.173186in}}{\pgfqpoint{0.824685in}{2.169913in}}{\pgfqpoint{0.832921in}{2.169913in}}%
\pgfpathclose%
\pgfusepath{stroke,fill}%
\end{pgfscope}%
\begin{pgfscope}%
\pgfpathrectangle{\pgfqpoint{0.100000in}{0.212622in}}{\pgfqpoint{3.696000in}{3.696000in}}%
\pgfusepath{clip}%
\pgfsetbuttcap%
\pgfsetroundjoin%
\definecolor{currentfill}{rgb}{0.121569,0.466667,0.705882}%
\pgfsetfillcolor{currentfill}%
\pgfsetfillopacity{0.746849}%
\pgfsetlinewidth{1.003750pt}%
\definecolor{currentstroke}{rgb}{0.121569,0.466667,0.705882}%
\pgfsetstrokecolor{currentstroke}%
\pgfsetstrokeopacity{0.746849}%
\pgfsetdash{}{0pt}%
\pgfpathmoveto{\pgfqpoint{0.831368in}{2.170061in}}%
\pgfpathcurveto{\pgfqpoint{0.839605in}{2.170061in}}{\pgfqpoint{0.847505in}{2.173333in}}{\pgfqpoint{0.853329in}{2.179157in}}%
\pgfpathcurveto{\pgfqpoint{0.859153in}{2.184981in}}{\pgfqpoint{0.862425in}{2.192881in}}{\pgfqpoint{0.862425in}{2.201117in}}%
\pgfpathcurveto{\pgfqpoint{0.862425in}{2.209354in}}{\pgfqpoint{0.859153in}{2.217254in}}{\pgfqpoint{0.853329in}{2.223078in}}%
\pgfpathcurveto{\pgfqpoint{0.847505in}{2.228902in}}{\pgfqpoint{0.839605in}{2.232174in}}{\pgfqpoint{0.831368in}{2.232174in}}%
\pgfpathcurveto{\pgfqpoint{0.823132in}{2.232174in}}{\pgfqpoint{0.815232in}{2.228902in}}{\pgfqpoint{0.809408in}{2.223078in}}%
\pgfpathcurveto{\pgfqpoint{0.803584in}{2.217254in}}{\pgfqpoint{0.800312in}{2.209354in}}{\pgfqpoint{0.800312in}{2.201117in}}%
\pgfpathcurveto{\pgfqpoint{0.800312in}{2.192881in}}{\pgfqpoint{0.803584in}{2.184981in}}{\pgfqpoint{0.809408in}{2.179157in}}%
\pgfpathcurveto{\pgfqpoint{0.815232in}{2.173333in}}{\pgfqpoint{0.823132in}{2.170061in}}{\pgfqpoint{0.831368in}{2.170061in}}%
\pgfpathclose%
\pgfusepath{stroke,fill}%
\end{pgfscope}%
\begin{pgfscope}%
\pgfpathrectangle{\pgfqpoint{0.100000in}{0.212622in}}{\pgfqpoint{3.696000in}{3.696000in}}%
\pgfusepath{clip}%
\pgfsetbuttcap%
\pgfsetroundjoin%
\definecolor{currentfill}{rgb}{0.121569,0.466667,0.705882}%
\pgfsetfillcolor{currentfill}%
\pgfsetfillopacity{0.747567}%
\pgfsetlinewidth{1.003750pt}%
\definecolor{currentstroke}{rgb}{0.121569,0.466667,0.705882}%
\pgfsetstrokecolor{currentstroke}%
\pgfsetstrokeopacity{0.747567}%
\pgfsetdash{}{0pt}%
\pgfpathmoveto{\pgfqpoint{2.914403in}{1.814769in}}%
\pgfpathcurveto{\pgfqpoint{2.922639in}{1.814769in}}{\pgfqpoint{2.930540in}{1.818042in}}{\pgfqpoint{2.936363in}{1.823866in}}%
\pgfpathcurveto{\pgfqpoint{2.942187in}{1.829690in}}{\pgfqpoint{2.945460in}{1.837590in}}{\pgfqpoint{2.945460in}{1.845826in}}%
\pgfpathcurveto{\pgfqpoint{2.945460in}{1.854062in}}{\pgfqpoint{2.942187in}{1.861962in}}{\pgfqpoint{2.936363in}{1.867786in}}%
\pgfpathcurveto{\pgfqpoint{2.930540in}{1.873610in}}{\pgfqpoint{2.922639in}{1.876882in}}{\pgfqpoint{2.914403in}{1.876882in}}%
\pgfpathcurveto{\pgfqpoint{2.906167in}{1.876882in}}{\pgfqpoint{2.898267in}{1.873610in}}{\pgfqpoint{2.892443in}{1.867786in}}%
\pgfpathcurveto{\pgfqpoint{2.886619in}{1.861962in}}{\pgfqpoint{2.883347in}{1.854062in}}{\pgfqpoint{2.883347in}{1.845826in}}%
\pgfpathcurveto{\pgfqpoint{2.883347in}{1.837590in}}{\pgfqpoint{2.886619in}{1.829690in}}{\pgfqpoint{2.892443in}{1.823866in}}%
\pgfpathcurveto{\pgfqpoint{2.898267in}{1.818042in}}{\pgfqpoint{2.906167in}{1.814769in}}{\pgfqpoint{2.914403in}{1.814769in}}%
\pgfpathclose%
\pgfusepath{stroke,fill}%
\end{pgfscope}%
\begin{pgfscope}%
\pgfpathrectangle{\pgfqpoint{0.100000in}{0.212622in}}{\pgfqpoint{3.696000in}{3.696000in}}%
\pgfusepath{clip}%
\pgfsetbuttcap%
\pgfsetroundjoin%
\definecolor{currentfill}{rgb}{0.121569,0.466667,0.705882}%
\pgfsetfillcolor{currentfill}%
\pgfsetfillopacity{0.748168}%
\pgfsetlinewidth{1.003750pt}%
\definecolor{currentstroke}{rgb}{0.121569,0.466667,0.705882}%
\pgfsetstrokecolor{currentstroke}%
\pgfsetstrokeopacity{0.748168}%
\pgfsetdash{}{0pt}%
\pgfpathmoveto{\pgfqpoint{0.829603in}{2.170213in}}%
\pgfpathcurveto{\pgfqpoint{0.837840in}{2.170213in}}{\pgfqpoint{0.845740in}{2.173485in}}{\pgfqpoint{0.851564in}{2.179309in}}%
\pgfpathcurveto{\pgfqpoint{0.857388in}{2.185133in}}{\pgfqpoint{0.860660in}{2.193033in}}{\pgfqpoint{0.860660in}{2.201270in}}%
\pgfpathcurveto{\pgfqpoint{0.860660in}{2.209506in}}{\pgfqpoint{0.857388in}{2.217406in}}{\pgfqpoint{0.851564in}{2.223230in}}%
\pgfpathcurveto{\pgfqpoint{0.845740in}{2.229054in}}{\pgfqpoint{0.837840in}{2.232326in}}{\pgfqpoint{0.829603in}{2.232326in}}%
\pgfpathcurveto{\pgfqpoint{0.821367in}{2.232326in}}{\pgfqpoint{0.813467in}{2.229054in}}{\pgfqpoint{0.807643in}{2.223230in}}%
\pgfpathcurveto{\pgfqpoint{0.801819in}{2.217406in}}{\pgfqpoint{0.798547in}{2.209506in}}{\pgfqpoint{0.798547in}{2.201270in}}%
\pgfpathcurveto{\pgfqpoint{0.798547in}{2.193033in}}{\pgfqpoint{0.801819in}{2.185133in}}{\pgfqpoint{0.807643in}{2.179309in}}%
\pgfpathcurveto{\pgfqpoint{0.813467in}{2.173485in}}{\pgfqpoint{0.821367in}{2.170213in}}{\pgfqpoint{0.829603in}{2.170213in}}%
\pgfpathclose%
\pgfusepath{stroke,fill}%
\end{pgfscope}%
\begin{pgfscope}%
\pgfpathrectangle{\pgfqpoint{0.100000in}{0.212622in}}{\pgfqpoint{3.696000in}{3.696000in}}%
\pgfusepath{clip}%
\pgfsetbuttcap%
\pgfsetroundjoin%
\definecolor{currentfill}{rgb}{0.121569,0.466667,0.705882}%
\pgfsetfillcolor{currentfill}%
\pgfsetfillopacity{0.748655}%
\pgfsetlinewidth{1.003750pt}%
\definecolor{currentstroke}{rgb}{0.121569,0.466667,0.705882}%
\pgfsetstrokecolor{currentstroke}%
\pgfsetstrokeopacity{0.748655}%
\pgfsetdash{}{0pt}%
\pgfpathmoveto{\pgfqpoint{2.911789in}{1.815314in}}%
\pgfpathcurveto{\pgfqpoint{2.920026in}{1.815314in}}{\pgfqpoint{2.927926in}{1.818587in}}{\pgfqpoint{2.933750in}{1.824411in}}%
\pgfpathcurveto{\pgfqpoint{2.939573in}{1.830234in}}{\pgfqpoint{2.942846in}{1.838135in}}{\pgfqpoint{2.942846in}{1.846371in}}%
\pgfpathcurveto{\pgfqpoint{2.942846in}{1.854607in}}{\pgfqpoint{2.939573in}{1.862507in}}{\pgfqpoint{2.933750in}{1.868331in}}%
\pgfpathcurveto{\pgfqpoint{2.927926in}{1.874155in}}{\pgfqpoint{2.920026in}{1.877427in}}{\pgfqpoint{2.911789in}{1.877427in}}%
\pgfpathcurveto{\pgfqpoint{2.903553in}{1.877427in}}{\pgfqpoint{2.895653in}{1.874155in}}{\pgfqpoint{2.889829in}{1.868331in}}%
\pgfpathcurveto{\pgfqpoint{2.884005in}{1.862507in}}{\pgfqpoint{2.880733in}{1.854607in}}{\pgfqpoint{2.880733in}{1.846371in}}%
\pgfpathcurveto{\pgfqpoint{2.880733in}{1.838135in}}{\pgfqpoint{2.884005in}{1.830234in}}{\pgfqpoint{2.889829in}{1.824411in}}%
\pgfpathcurveto{\pgfqpoint{2.895653in}{1.818587in}}{\pgfqpoint{2.903553in}{1.815314in}}{\pgfqpoint{2.911789in}{1.815314in}}%
\pgfpathclose%
\pgfusepath{stroke,fill}%
\end{pgfscope}%
\begin{pgfscope}%
\pgfpathrectangle{\pgfqpoint{0.100000in}{0.212622in}}{\pgfqpoint{3.696000in}{3.696000in}}%
\pgfusepath{clip}%
\pgfsetbuttcap%
\pgfsetroundjoin%
\definecolor{currentfill}{rgb}{0.121569,0.466667,0.705882}%
\pgfsetfillcolor{currentfill}%
\pgfsetfillopacity{0.750279}%
\pgfsetlinewidth{1.003750pt}%
\definecolor{currentstroke}{rgb}{0.121569,0.466667,0.705882}%
\pgfsetstrokecolor{currentstroke}%
\pgfsetstrokeopacity{0.750279}%
\pgfsetdash{}{0pt}%
\pgfpathmoveto{\pgfqpoint{0.823984in}{2.170792in}}%
\pgfpathcurveto{\pgfqpoint{0.832221in}{2.170792in}}{\pgfqpoint{0.840121in}{2.174065in}}{\pgfqpoint{0.845945in}{2.179888in}}%
\pgfpathcurveto{\pgfqpoint{0.851769in}{2.185712in}}{\pgfqpoint{0.855041in}{2.193612in}}{\pgfqpoint{0.855041in}{2.201849in}}%
\pgfpathcurveto{\pgfqpoint{0.855041in}{2.210085in}}{\pgfqpoint{0.851769in}{2.217985in}}{\pgfqpoint{0.845945in}{2.223809in}}%
\pgfpathcurveto{\pgfqpoint{0.840121in}{2.229633in}}{\pgfqpoint{0.832221in}{2.232905in}}{\pgfqpoint{0.823984in}{2.232905in}}%
\pgfpathcurveto{\pgfqpoint{0.815748in}{2.232905in}}{\pgfqpoint{0.807848in}{2.229633in}}{\pgfqpoint{0.802024in}{2.223809in}}%
\pgfpathcurveto{\pgfqpoint{0.796200in}{2.217985in}}{\pgfqpoint{0.792928in}{2.210085in}}{\pgfqpoint{0.792928in}{2.201849in}}%
\pgfpathcurveto{\pgfqpoint{0.792928in}{2.193612in}}{\pgfqpoint{0.796200in}{2.185712in}}{\pgfqpoint{0.802024in}{2.179888in}}%
\pgfpathcurveto{\pgfqpoint{0.807848in}{2.174065in}}{\pgfqpoint{0.815748in}{2.170792in}}{\pgfqpoint{0.823984in}{2.170792in}}%
\pgfpathclose%
\pgfusepath{stroke,fill}%
\end{pgfscope}%
\begin{pgfscope}%
\pgfpathrectangle{\pgfqpoint{0.100000in}{0.212622in}}{\pgfqpoint{3.696000in}{3.696000in}}%
\pgfusepath{clip}%
\pgfsetbuttcap%
\pgfsetroundjoin%
\definecolor{currentfill}{rgb}{0.121569,0.466667,0.705882}%
\pgfsetfillcolor{currentfill}%
\pgfsetfillopacity{0.750495}%
\pgfsetlinewidth{1.003750pt}%
\definecolor{currentstroke}{rgb}{0.121569,0.466667,0.705882}%
\pgfsetstrokecolor{currentstroke}%
\pgfsetstrokeopacity{0.750495}%
\pgfsetdash{}{0pt}%
\pgfpathmoveto{\pgfqpoint{2.910291in}{1.815477in}}%
\pgfpathcurveto{\pgfqpoint{2.918528in}{1.815477in}}{\pgfqpoint{2.926428in}{1.818750in}}{\pgfqpoint{2.932252in}{1.824574in}}%
\pgfpathcurveto{\pgfqpoint{2.938076in}{1.830398in}}{\pgfqpoint{2.941348in}{1.838298in}}{\pgfqpoint{2.941348in}{1.846534in}}%
\pgfpathcurveto{\pgfqpoint{2.941348in}{1.854770in}}{\pgfqpoint{2.938076in}{1.862670in}}{\pgfqpoint{2.932252in}{1.868494in}}%
\pgfpathcurveto{\pgfqpoint{2.926428in}{1.874318in}}{\pgfqpoint{2.918528in}{1.877590in}}{\pgfqpoint{2.910291in}{1.877590in}}%
\pgfpathcurveto{\pgfqpoint{2.902055in}{1.877590in}}{\pgfqpoint{2.894155in}{1.874318in}}{\pgfqpoint{2.888331in}{1.868494in}}%
\pgfpathcurveto{\pgfqpoint{2.882507in}{1.862670in}}{\pgfqpoint{2.879235in}{1.854770in}}{\pgfqpoint{2.879235in}{1.846534in}}%
\pgfpathcurveto{\pgfqpoint{2.879235in}{1.838298in}}{\pgfqpoint{2.882507in}{1.830398in}}{\pgfqpoint{2.888331in}{1.824574in}}%
\pgfpathcurveto{\pgfqpoint{2.894155in}{1.818750in}}{\pgfqpoint{2.902055in}{1.815477in}}{\pgfqpoint{2.910291in}{1.815477in}}%
\pgfpathclose%
\pgfusepath{stroke,fill}%
\end{pgfscope}%
\begin{pgfscope}%
\pgfpathrectangle{\pgfqpoint{0.100000in}{0.212622in}}{\pgfqpoint{3.696000in}{3.696000in}}%
\pgfusepath{clip}%
\pgfsetbuttcap%
\pgfsetroundjoin%
\definecolor{currentfill}{rgb}{0.121569,0.466667,0.705882}%
\pgfsetfillcolor{currentfill}%
\pgfsetfillopacity{0.751444}%
\pgfsetlinewidth{1.003750pt}%
\definecolor{currentstroke}{rgb}{0.121569,0.466667,0.705882}%
\pgfsetstrokecolor{currentstroke}%
\pgfsetstrokeopacity{0.751444}%
\pgfsetdash{}{0pt}%
\pgfpathmoveto{\pgfqpoint{2.908929in}{1.815485in}}%
\pgfpathcurveto{\pgfqpoint{2.917165in}{1.815485in}}{\pgfqpoint{2.925065in}{1.818757in}}{\pgfqpoint{2.930889in}{1.824581in}}%
\pgfpathcurveto{\pgfqpoint{2.936713in}{1.830405in}}{\pgfqpoint{2.939985in}{1.838305in}}{\pgfqpoint{2.939985in}{1.846542in}}%
\pgfpathcurveto{\pgfqpoint{2.939985in}{1.854778in}}{\pgfqpoint{2.936713in}{1.862678in}}{\pgfqpoint{2.930889in}{1.868502in}}%
\pgfpathcurveto{\pgfqpoint{2.925065in}{1.874326in}}{\pgfqpoint{2.917165in}{1.877598in}}{\pgfqpoint{2.908929in}{1.877598in}}%
\pgfpathcurveto{\pgfqpoint{2.900692in}{1.877598in}}{\pgfqpoint{2.892792in}{1.874326in}}{\pgfqpoint{2.886968in}{1.868502in}}%
\pgfpathcurveto{\pgfqpoint{2.881144in}{1.862678in}}{\pgfqpoint{2.877872in}{1.854778in}}{\pgfqpoint{2.877872in}{1.846542in}}%
\pgfpathcurveto{\pgfqpoint{2.877872in}{1.838305in}}{\pgfqpoint{2.881144in}{1.830405in}}{\pgfqpoint{2.886968in}{1.824581in}}%
\pgfpathcurveto{\pgfqpoint{2.892792in}{1.818757in}}{\pgfqpoint{2.900692in}{1.815485in}}{\pgfqpoint{2.908929in}{1.815485in}}%
\pgfpathclose%
\pgfusepath{stroke,fill}%
\end{pgfscope}%
\begin{pgfscope}%
\pgfpathrectangle{\pgfqpoint{0.100000in}{0.212622in}}{\pgfqpoint{3.696000in}{3.696000in}}%
\pgfusepath{clip}%
\pgfsetbuttcap%
\pgfsetroundjoin%
\definecolor{currentfill}{rgb}{0.121569,0.466667,0.705882}%
\pgfsetfillcolor{currentfill}%
\pgfsetfillopacity{0.751918}%
\pgfsetlinewidth{1.003750pt}%
\definecolor{currentstroke}{rgb}{0.121569,0.466667,0.705882}%
\pgfsetstrokecolor{currentstroke}%
\pgfsetstrokeopacity{0.751918}%
\pgfsetdash{}{0pt}%
\pgfpathmoveto{\pgfqpoint{2.907652in}{1.815823in}}%
\pgfpathcurveto{\pgfqpoint{2.915888in}{1.815823in}}{\pgfqpoint{2.923789in}{1.819095in}}{\pgfqpoint{2.929612in}{1.824919in}}%
\pgfpathcurveto{\pgfqpoint{2.935436in}{1.830743in}}{\pgfqpoint{2.938709in}{1.838643in}}{\pgfqpoint{2.938709in}{1.846880in}}%
\pgfpathcurveto{\pgfqpoint{2.938709in}{1.855116in}}{\pgfqpoint{2.935436in}{1.863016in}}{\pgfqpoint{2.929612in}{1.868840in}}%
\pgfpathcurveto{\pgfqpoint{2.923789in}{1.874664in}}{\pgfqpoint{2.915888in}{1.877936in}}{\pgfqpoint{2.907652in}{1.877936in}}%
\pgfpathcurveto{\pgfqpoint{2.899416in}{1.877936in}}{\pgfqpoint{2.891516in}{1.874664in}}{\pgfqpoint{2.885692in}{1.868840in}}%
\pgfpathcurveto{\pgfqpoint{2.879868in}{1.863016in}}{\pgfqpoint{2.876596in}{1.855116in}}{\pgfqpoint{2.876596in}{1.846880in}}%
\pgfpathcurveto{\pgfqpoint{2.876596in}{1.838643in}}{\pgfqpoint{2.879868in}{1.830743in}}{\pgfqpoint{2.885692in}{1.824919in}}%
\pgfpathcurveto{\pgfqpoint{2.891516in}{1.819095in}}{\pgfqpoint{2.899416in}{1.815823in}}{\pgfqpoint{2.907652in}{1.815823in}}%
\pgfpathclose%
\pgfusepath{stroke,fill}%
\end{pgfscope}%
\begin{pgfscope}%
\pgfpathrectangle{\pgfqpoint{0.100000in}{0.212622in}}{\pgfqpoint{3.696000in}{3.696000in}}%
\pgfusepath{clip}%
\pgfsetbuttcap%
\pgfsetroundjoin%
\definecolor{currentfill}{rgb}{0.121569,0.466667,0.705882}%
\pgfsetfillcolor{currentfill}%
\pgfsetfillopacity{0.752388}%
\pgfsetlinewidth{1.003750pt}%
\definecolor{currentstroke}{rgb}{0.121569,0.466667,0.705882}%
\pgfsetstrokecolor{currentstroke}%
\pgfsetstrokeopacity{0.752388}%
\pgfsetdash{}{0pt}%
\pgfpathmoveto{\pgfqpoint{0.820669in}{2.170893in}}%
\pgfpathcurveto{\pgfqpoint{0.828905in}{2.170893in}}{\pgfqpoint{0.836806in}{2.174165in}}{\pgfqpoint{0.842629in}{2.179989in}}%
\pgfpathcurveto{\pgfqpoint{0.848453in}{2.185813in}}{\pgfqpoint{0.851726in}{2.193713in}}{\pgfqpoint{0.851726in}{2.201949in}}%
\pgfpathcurveto{\pgfqpoint{0.851726in}{2.210186in}}{\pgfqpoint{0.848453in}{2.218086in}}{\pgfqpoint{0.842629in}{2.223910in}}%
\pgfpathcurveto{\pgfqpoint{0.836806in}{2.229734in}}{\pgfqpoint{0.828905in}{2.233006in}}{\pgfqpoint{0.820669in}{2.233006in}}%
\pgfpathcurveto{\pgfqpoint{0.812433in}{2.233006in}}{\pgfqpoint{0.804533in}{2.229734in}}{\pgfqpoint{0.798709in}{2.223910in}}%
\pgfpathcurveto{\pgfqpoint{0.792885in}{2.218086in}}{\pgfqpoint{0.789613in}{2.210186in}}{\pgfqpoint{0.789613in}{2.201949in}}%
\pgfpathcurveto{\pgfqpoint{0.789613in}{2.193713in}}{\pgfqpoint{0.792885in}{2.185813in}}{\pgfqpoint{0.798709in}{2.179989in}}%
\pgfpathcurveto{\pgfqpoint{0.804533in}{2.174165in}}{\pgfqpoint{0.812433in}{2.170893in}}{\pgfqpoint{0.820669in}{2.170893in}}%
\pgfpathclose%
\pgfusepath{stroke,fill}%
\end{pgfscope}%
\begin{pgfscope}%
\pgfpathrectangle{\pgfqpoint{0.100000in}{0.212622in}}{\pgfqpoint{3.696000in}{3.696000in}}%
\pgfusepath{clip}%
\pgfsetbuttcap%
\pgfsetroundjoin%
\definecolor{currentfill}{rgb}{0.121569,0.466667,0.705882}%
\pgfsetfillcolor{currentfill}%
\pgfsetfillopacity{0.753212}%
\pgfsetlinewidth{1.003750pt}%
\definecolor{currentstroke}{rgb}{0.121569,0.466667,0.705882}%
\pgfsetstrokecolor{currentstroke}%
\pgfsetstrokeopacity{0.753212}%
\pgfsetdash{}{0pt}%
\pgfpathmoveto{\pgfqpoint{2.906008in}{1.815893in}}%
\pgfpathcurveto{\pgfqpoint{2.914244in}{1.815893in}}{\pgfqpoint{2.922144in}{1.819165in}}{\pgfqpoint{2.927968in}{1.824989in}}%
\pgfpathcurveto{\pgfqpoint{2.933792in}{1.830813in}}{\pgfqpoint{2.937064in}{1.838713in}}{\pgfqpoint{2.937064in}{1.846949in}}%
\pgfpathcurveto{\pgfqpoint{2.937064in}{1.855186in}}{\pgfqpoint{2.933792in}{1.863086in}}{\pgfqpoint{2.927968in}{1.868910in}}%
\pgfpathcurveto{\pgfqpoint{2.922144in}{1.874734in}}{\pgfqpoint{2.914244in}{1.878006in}}{\pgfqpoint{2.906008in}{1.878006in}}%
\pgfpathcurveto{\pgfqpoint{2.897772in}{1.878006in}}{\pgfqpoint{2.889872in}{1.874734in}}{\pgfqpoint{2.884048in}{1.868910in}}%
\pgfpathcurveto{\pgfqpoint{2.878224in}{1.863086in}}{\pgfqpoint{2.874951in}{1.855186in}}{\pgfqpoint{2.874951in}{1.846949in}}%
\pgfpathcurveto{\pgfqpoint{2.874951in}{1.838713in}}{\pgfqpoint{2.878224in}{1.830813in}}{\pgfqpoint{2.884048in}{1.824989in}}%
\pgfpathcurveto{\pgfqpoint{2.889872in}{1.819165in}}{\pgfqpoint{2.897772in}{1.815893in}}{\pgfqpoint{2.906008in}{1.815893in}}%
\pgfpathclose%
\pgfusepath{stroke,fill}%
\end{pgfscope}%
\begin{pgfscope}%
\pgfpathrectangle{\pgfqpoint{0.100000in}{0.212622in}}{\pgfqpoint{3.696000in}{3.696000in}}%
\pgfusepath{clip}%
\pgfsetbuttcap%
\pgfsetroundjoin%
\definecolor{currentfill}{rgb}{0.121569,0.466667,0.705882}%
\pgfsetfillcolor{currentfill}%
\pgfsetfillopacity{0.753949}%
\pgfsetlinewidth{1.003750pt}%
\definecolor{currentstroke}{rgb}{0.121569,0.466667,0.705882}%
\pgfsetstrokecolor{currentstroke}%
\pgfsetstrokeopacity{0.753949}%
\pgfsetdash{}{0pt}%
\pgfpathmoveto{\pgfqpoint{2.905396in}{1.815914in}}%
\pgfpathcurveto{\pgfqpoint{2.913632in}{1.815914in}}{\pgfqpoint{2.921532in}{1.819186in}}{\pgfqpoint{2.927356in}{1.825010in}}%
\pgfpathcurveto{\pgfqpoint{2.933180in}{1.830834in}}{\pgfqpoint{2.936453in}{1.838734in}}{\pgfqpoint{2.936453in}{1.846970in}}%
\pgfpathcurveto{\pgfqpoint{2.936453in}{1.855207in}}{\pgfqpoint{2.933180in}{1.863107in}}{\pgfqpoint{2.927356in}{1.868930in}}%
\pgfpathcurveto{\pgfqpoint{2.921532in}{1.874754in}}{\pgfqpoint{2.913632in}{1.878027in}}{\pgfqpoint{2.905396in}{1.878027in}}%
\pgfpathcurveto{\pgfqpoint{2.897160in}{1.878027in}}{\pgfqpoint{2.889260in}{1.874754in}}{\pgfqpoint{2.883436in}{1.868930in}}%
\pgfpathcurveto{\pgfqpoint{2.877612in}{1.863107in}}{\pgfqpoint{2.874340in}{1.855207in}}{\pgfqpoint{2.874340in}{1.846970in}}%
\pgfpathcurveto{\pgfqpoint{2.874340in}{1.838734in}}{\pgfqpoint{2.877612in}{1.830834in}}{\pgfqpoint{2.883436in}{1.825010in}}%
\pgfpathcurveto{\pgfqpoint{2.889260in}{1.819186in}}{\pgfqpoint{2.897160in}{1.815914in}}{\pgfqpoint{2.905396in}{1.815914in}}%
\pgfpathclose%
\pgfusepath{stroke,fill}%
\end{pgfscope}%
\begin{pgfscope}%
\pgfpathrectangle{\pgfqpoint{0.100000in}{0.212622in}}{\pgfqpoint{3.696000in}{3.696000in}}%
\pgfusepath{clip}%
\pgfsetbuttcap%
\pgfsetroundjoin%
\definecolor{currentfill}{rgb}{0.121569,0.466667,0.705882}%
\pgfsetfillcolor{currentfill}%
\pgfsetfillopacity{0.753980}%
\pgfsetlinewidth{1.003750pt}%
\definecolor{currentstroke}{rgb}{0.121569,0.466667,0.705882}%
\pgfsetstrokecolor{currentstroke}%
\pgfsetstrokeopacity{0.753980}%
\pgfsetdash{}{0pt}%
\pgfpathmoveto{\pgfqpoint{0.816369in}{2.171378in}}%
\pgfpathcurveto{\pgfqpoint{0.824605in}{2.171378in}}{\pgfqpoint{0.832506in}{2.174651in}}{\pgfqpoint{0.838329in}{2.180475in}}%
\pgfpathcurveto{\pgfqpoint{0.844153in}{2.186298in}}{\pgfqpoint{0.847426in}{2.194199in}}{\pgfqpoint{0.847426in}{2.202435in}}%
\pgfpathcurveto{\pgfqpoint{0.847426in}{2.210671in}}{\pgfqpoint{0.844153in}{2.218571in}}{\pgfqpoint{0.838329in}{2.224395in}}%
\pgfpathcurveto{\pgfqpoint{0.832506in}{2.230219in}}{\pgfqpoint{0.824605in}{2.233491in}}{\pgfqpoint{0.816369in}{2.233491in}}%
\pgfpathcurveto{\pgfqpoint{0.808133in}{2.233491in}}{\pgfqpoint{0.800233in}{2.230219in}}{\pgfqpoint{0.794409in}{2.224395in}}%
\pgfpathcurveto{\pgfqpoint{0.788585in}{2.218571in}}{\pgfqpoint{0.785313in}{2.210671in}}{\pgfqpoint{0.785313in}{2.202435in}}%
\pgfpathcurveto{\pgfqpoint{0.785313in}{2.194199in}}{\pgfqpoint{0.788585in}{2.186298in}}{\pgfqpoint{0.794409in}{2.180475in}}%
\pgfpathcurveto{\pgfqpoint{0.800233in}{2.174651in}}{\pgfqpoint{0.808133in}{2.171378in}}{\pgfqpoint{0.816369in}{2.171378in}}%
\pgfpathclose%
\pgfusepath{stroke,fill}%
\end{pgfscope}%
\begin{pgfscope}%
\pgfpathrectangle{\pgfqpoint{0.100000in}{0.212622in}}{\pgfqpoint{3.696000in}{3.696000in}}%
\pgfusepath{clip}%
\pgfsetbuttcap%
\pgfsetroundjoin%
\definecolor{currentfill}{rgb}{0.121569,0.466667,0.705882}%
\pgfsetfillcolor{currentfill}%
\pgfsetfillopacity{0.755184}%
\pgfsetlinewidth{1.003750pt}%
\definecolor{currentstroke}{rgb}{0.121569,0.466667,0.705882}%
\pgfsetstrokecolor{currentstroke}%
\pgfsetstrokeopacity{0.755184}%
\pgfsetdash{}{0pt}%
\pgfpathmoveto{\pgfqpoint{2.903083in}{1.816212in}}%
\pgfpathcurveto{\pgfqpoint{2.911319in}{1.816212in}}{\pgfqpoint{2.919219in}{1.819484in}}{\pgfqpoint{2.925043in}{1.825308in}}%
\pgfpathcurveto{\pgfqpoint{2.930867in}{1.831132in}}{\pgfqpoint{2.934139in}{1.839032in}}{\pgfqpoint{2.934139in}{1.847268in}}%
\pgfpathcurveto{\pgfqpoint{2.934139in}{1.855504in}}{\pgfqpoint{2.930867in}{1.863405in}}{\pgfqpoint{2.925043in}{1.869228in}}%
\pgfpathcurveto{\pgfqpoint{2.919219in}{1.875052in}}{\pgfqpoint{2.911319in}{1.878325in}}{\pgfqpoint{2.903083in}{1.878325in}}%
\pgfpathcurveto{\pgfqpoint{2.894846in}{1.878325in}}{\pgfqpoint{2.886946in}{1.875052in}}{\pgfqpoint{2.881122in}{1.869228in}}%
\pgfpathcurveto{\pgfqpoint{2.875299in}{1.863405in}}{\pgfqpoint{2.872026in}{1.855504in}}{\pgfqpoint{2.872026in}{1.847268in}}%
\pgfpathcurveto{\pgfqpoint{2.872026in}{1.839032in}}{\pgfqpoint{2.875299in}{1.831132in}}{\pgfqpoint{2.881122in}{1.825308in}}%
\pgfpathcurveto{\pgfqpoint{2.886946in}{1.819484in}}{\pgfqpoint{2.894846in}{1.816212in}}{\pgfqpoint{2.903083in}{1.816212in}}%
\pgfpathclose%
\pgfusepath{stroke,fill}%
\end{pgfscope}%
\begin{pgfscope}%
\pgfpathrectangle{\pgfqpoint{0.100000in}{0.212622in}}{\pgfqpoint{3.696000in}{3.696000in}}%
\pgfusepath{clip}%
\pgfsetbuttcap%
\pgfsetroundjoin%
\definecolor{currentfill}{rgb}{0.121569,0.466667,0.705882}%
\pgfsetfillcolor{currentfill}%
\pgfsetfillopacity{0.755859}%
\pgfsetlinewidth{1.003750pt}%
\definecolor{currentstroke}{rgb}{0.121569,0.466667,0.705882}%
\pgfsetstrokecolor{currentstroke}%
\pgfsetstrokeopacity{0.755859}%
\pgfsetdash{}{0pt}%
\pgfpathmoveto{\pgfqpoint{2.901749in}{1.816408in}}%
\pgfpathcurveto{\pgfqpoint{2.909985in}{1.816408in}}{\pgfqpoint{2.917885in}{1.819681in}}{\pgfqpoint{2.923709in}{1.825505in}}%
\pgfpathcurveto{\pgfqpoint{2.929533in}{1.831329in}}{\pgfqpoint{2.932806in}{1.839229in}}{\pgfqpoint{2.932806in}{1.847465in}}%
\pgfpathcurveto{\pgfqpoint{2.932806in}{1.855701in}}{\pgfqpoint{2.929533in}{1.863601in}}{\pgfqpoint{2.923709in}{1.869425in}}%
\pgfpathcurveto{\pgfqpoint{2.917885in}{1.875249in}}{\pgfqpoint{2.909985in}{1.878521in}}{\pgfqpoint{2.901749in}{1.878521in}}%
\pgfpathcurveto{\pgfqpoint{2.893513in}{1.878521in}}{\pgfqpoint{2.885613in}{1.875249in}}{\pgfqpoint{2.879789in}{1.869425in}}%
\pgfpathcurveto{\pgfqpoint{2.873965in}{1.863601in}}{\pgfqpoint{2.870693in}{1.855701in}}{\pgfqpoint{2.870693in}{1.847465in}}%
\pgfpathcurveto{\pgfqpoint{2.870693in}{1.839229in}}{\pgfqpoint{2.873965in}{1.831329in}}{\pgfqpoint{2.879789in}{1.825505in}}%
\pgfpathcurveto{\pgfqpoint{2.885613in}{1.819681in}}{\pgfqpoint{2.893513in}{1.816408in}}{\pgfqpoint{2.901749in}{1.816408in}}%
\pgfpathclose%
\pgfusepath{stroke,fill}%
\end{pgfscope}%
\begin{pgfscope}%
\pgfpathrectangle{\pgfqpoint{0.100000in}{0.212622in}}{\pgfqpoint{3.696000in}{3.696000in}}%
\pgfusepath{clip}%
\pgfsetbuttcap%
\pgfsetroundjoin%
\definecolor{currentfill}{rgb}{0.121569,0.466667,0.705882}%
\pgfsetfillcolor{currentfill}%
\pgfsetfillopacity{0.756700}%
\pgfsetlinewidth{1.003750pt}%
\definecolor{currentstroke}{rgb}{0.121569,0.466667,0.705882}%
\pgfsetstrokecolor{currentstroke}%
\pgfsetstrokeopacity{0.756700}%
\pgfsetdash{}{0pt}%
\pgfpathmoveto{\pgfqpoint{2.901173in}{1.816405in}}%
\pgfpathcurveto{\pgfqpoint{2.909409in}{1.816405in}}{\pgfqpoint{2.917309in}{1.819677in}}{\pgfqpoint{2.923133in}{1.825501in}}%
\pgfpathcurveto{\pgfqpoint{2.928957in}{1.831325in}}{\pgfqpoint{2.932230in}{1.839225in}}{\pgfqpoint{2.932230in}{1.847461in}}%
\pgfpathcurveto{\pgfqpoint{2.932230in}{1.855697in}}{\pgfqpoint{2.928957in}{1.863598in}}{\pgfqpoint{2.923133in}{1.869421in}}%
\pgfpathcurveto{\pgfqpoint{2.917309in}{1.875245in}}{\pgfqpoint{2.909409in}{1.878518in}}{\pgfqpoint{2.901173in}{1.878518in}}%
\pgfpathcurveto{\pgfqpoint{2.892937in}{1.878518in}}{\pgfqpoint{2.885037in}{1.875245in}}{\pgfqpoint{2.879213in}{1.869421in}}%
\pgfpathcurveto{\pgfqpoint{2.873389in}{1.863598in}}{\pgfqpoint{2.870117in}{1.855697in}}{\pgfqpoint{2.870117in}{1.847461in}}%
\pgfpathcurveto{\pgfqpoint{2.870117in}{1.839225in}}{\pgfqpoint{2.873389in}{1.831325in}}{\pgfqpoint{2.879213in}{1.825501in}}%
\pgfpathcurveto{\pgfqpoint{2.885037in}{1.819677in}}{\pgfqpoint{2.892937in}{1.816405in}}{\pgfqpoint{2.901173in}{1.816405in}}%
\pgfpathclose%
\pgfusepath{stroke,fill}%
\end{pgfscope}%
\begin{pgfscope}%
\pgfpathrectangle{\pgfqpoint{0.100000in}{0.212622in}}{\pgfqpoint{3.696000in}{3.696000in}}%
\pgfusepath{clip}%
\pgfsetbuttcap%
\pgfsetroundjoin%
\definecolor{currentfill}{rgb}{0.121569,0.466667,0.705882}%
\pgfsetfillcolor{currentfill}%
\pgfsetfillopacity{0.757247}%
\pgfsetlinewidth{1.003750pt}%
\definecolor{currentstroke}{rgb}{0.121569,0.466667,0.705882}%
\pgfsetstrokecolor{currentstroke}%
\pgfsetstrokeopacity{0.757247}%
\pgfsetdash{}{0pt}%
\pgfpathmoveto{\pgfqpoint{0.811481in}{2.171807in}}%
\pgfpathcurveto{\pgfqpoint{0.819717in}{2.171807in}}{\pgfqpoint{0.827617in}{2.175080in}}{\pgfqpoint{0.833441in}{2.180904in}}%
\pgfpathcurveto{\pgfqpoint{0.839265in}{2.186728in}}{\pgfqpoint{0.842538in}{2.194628in}}{\pgfqpoint{0.842538in}{2.202864in}}%
\pgfpathcurveto{\pgfqpoint{0.842538in}{2.211100in}}{\pgfqpoint{0.839265in}{2.219000in}}{\pgfqpoint{0.833441in}{2.224824in}}%
\pgfpathcurveto{\pgfqpoint{0.827617in}{2.230648in}}{\pgfqpoint{0.819717in}{2.233920in}}{\pgfqpoint{0.811481in}{2.233920in}}%
\pgfpathcurveto{\pgfqpoint{0.803245in}{2.233920in}}{\pgfqpoint{0.795345in}{2.230648in}}{\pgfqpoint{0.789521in}{2.224824in}}%
\pgfpathcurveto{\pgfqpoint{0.783697in}{2.219000in}}{\pgfqpoint{0.780425in}{2.211100in}}{\pgfqpoint{0.780425in}{2.202864in}}%
\pgfpathcurveto{\pgfqpoint{0.780425in}{2.194628in}}{\pgfqpoint{0.783697in}{2.186728in}}{\pgfqpoint{0.789521in}{2.180904in}}%
\pgfpathcurveto{\pgfqpoint{0.795345in}{2.175080in}}{\pgfqpoint{0.803245in}{2.171807in}}{\pgfqpoint{0.811481in}{2.171807in}}%
\pgfpathclose%
\pgfusepath{stroke,fill}%
\end{pgfscope}%
\begin{pgfscope}%
\pgfpathrectangle{\pgfqpoint{0.100000in}{0.212622in}}{\pgfqpoint{3.696000in}{3.696000in}}%
\pgfusepath{clip}%
\pgfsetbuttcap%
\pgfsetroundjoin%
\definecolor{currentfill}{rgb}{0.121569,0.466667,0.705882}%
\pgfsetfillcolor{currentfill}%
\pgfsetfillopacity{0.757940}%
\pgfsetlinewidth{1.003750pt}%
\definecolor{currentstroke}{rgb}{0.121569,0.466667,0.705882}%
\pgfsetstrokecolor{currentstroke}%
\pgfsetstrokeopacity{0.757940}%
\pgfsetdash{}{0pt}%
\pgfpathmoveto{\pgfqpoint{2.898836in}{1.816765in}}%
\pgfpathcurveto{\pgfqpoint{2.907072in}{1.816765in}}{\pgfqpoint{2.914973in}{1.820037in}}{\pgfqpoint{2.920796in}{1.825861in}}%
\pgfpathcurveto{\pgfqpoint{2.926620in}{1.831685in}}{\pgfqpoint{2.929893in}{1.839585in}}{\pgfqpoint{2.929893in}{1.847821in}}%
\pgfpathcurveto{\pgfqpoint{2.929893in}{1.856058in}}{\pgfqpoint{2.926620in}{1.863958in}}{\pgfqpoint{2.920796in}{1.869782in}}%
\pgfpathcurveto{\pgfqpoint{2.914973in}{1.875606in}}{\pgfqpoint{2.907072in}{1.878878in}}{\pgfqpoint{2.898836in}{1.878878in}}%
\pgfpathcurveto{\pgfqpoint{2.890600in}{1.878878in}}{\pgfqpoint{2.882700in}{1.875606in}}{\pgfqpoint{2.876876in}{1.869782in}}%
\pgfpathcurveto{\pgfqpoint{2.871052in}{1.863958in}}{\pgfqpoint{2.867780in}{1.856058in}}{\pgfqpoint{2.867780in}{1.847821in}}%
\pgfpathcurveto{\pgfqpoint{2.867780in}{1.839585in}}{\pgfqpoint{2.871052in}{1.831685in}}{\pgfqpoint{2.876876in}{1.825861in}}%
\pgfpathcurveto{\pgfqpoint{2.882700in}{1.820037in}}{\pgfqpoint{2.890600in}{1.816765in}}{\pgfqpoint{2.898836in}{1.816765in}}%
\pgfpathclose%
\pgfusepath{stroke,fill}%
\end{pgfscope}%
\begin{pgfscope}%
\pgfpathrectangle{\pgfqpoint{0.100000in}{0.212622in}}{\pgfqpoint{3.696000in}{3.696000in}}%
\pgfusepath{clip}%
\pgfsetbuttcap%
\pgfsetroundjoin%
\definecolor{currentfill}{rgb}{0.121569,0.466667,0.705882}%
\pgfsetfillcolor{currentfill}%
\pgfsetfillopacity{0.759271}%
\pgfsetlinewidth{1.003750pt}%
\definecolor{currentstroke}{rgb}{0.121569,0.466667,0.705882}%
\pgfsetstrokecolor{currentstroke}%
\pgfsetstrokeopacity{0.759271}%
\pgfsetdash{}{0pt}%
\pgfpathmoveto{\pgfqpoint{2.895513in}{1.817716in}}%
\pgfpathcurveto{\pgfqpoint{2.903749in}{1.817716in}}{\pgfqpoint{2.911649in}{1.820988in}}{\pgfqpoint{2.917473in}{1.826812in}}%
\pgfpathcurveto{\pgfqpoint{2.923297in}{1.832636in}}{\pgfqpoint{2.926569in}{1.840536in}}{\pgfqpoint{2.926569in}{1.848772in}}%
\pgfpathcurveto{\pgfqpoint{2.926569in}{1.857009in}}{\pgfqpoint{2.923297in}{1.864909in}}{\pgfqpoint{2.917473in}{1.870733in}}%
\pgfpathcurveto{\pgfqpoint{2.911649in}{1.876556in}}{\pgfqpoint{2.903749in}{1.879829in}}{\pgfqpoint{2.895513in}{1.879829in}}%
\pgfpathcurveto{\pgfqpoint{2.887276in}{1.879829in}}{\pgfqpoint{2.879376in}{1.876556in}}{\pgfqpoint{2.873552in}{1.870733in}}%
\pgfpathcurveto{\pgfqpoint{2.867728in}{1.864909in}}{\pgfqpoint{2.864456in}{1.857009in}}{\pgfqpoint{2.864456in}{1.848772in}}%
\pgfpathcurveto{\pgfqpoint{2.864456in}{1.840536in}}{\pgfqpoint{2.867728in}{1.832636in}}{\pgfqpoint{2.873552in}{1.826812in}}%
\pgfpathcurveto{\pgfqpoint{2.879376in}{1.820988in}}{\pgfqpoint{2.887276in}{1.817716in}}{\pgfqpoint{2.895513in}{1.817716in}}%
\pgfpathclose%
\pgfusepath{stroke,fill}%
\end{pgfscope}%
\begin{pgfscope}%
\pgfpathrectangle{\pgfqpoint{0.100000in}{0.212622in}}{\pgfqpoint{3.696000in}{3.696000in}}%
\pgfusepath{clip}%
\pgfsetbuttcap%
\pgfsetroundjoin%
\definecolor{currentfill}{rgb}{0.121569,0.466667,0.705882}%
\pgfsetfillcolor{currentfill}%
\pgfsetfillopacity{0.760086}%
\pgfsetlinewidth{1.003750pt}%
\definecolor{currentstroke}{rgb}{0.121569,0.466667,0.705882}%
\pgfsetstrokecolor{currentstroke}%
\pgfsetstrokeopacity{0.760086}%
\pgfsetdash{}{0pt}%
\pgfpathmoveto{\pgfqpoint{0.804117in}{2.172471in}}%
\pgfpathcurveto{\pgfqpoint{0.812353in}{2.172471in}}{\pgfqpoint{0.820253in}{2.175743in}}{\pgfqpoint{0.826077in}{2.181567in}}%
\pgfpathcurveto{\pgfqpoint{0.831901in}{2.187391in}}{\pgfqpoint{0.835174in}{2.195291in}}{\pgfqpoint{0.835174in}{2.203527in}}%
\pgfpathcurveto{\pgfqpoint{0.835174in}{2.211763in}}{\pgfqpoint{0.831901in}{2.219663in}}{\pgfqpoint{0.826077in}{2.225487in}}%
\pgfpathcurveto{\pgfqpoint{0.820253in}{2.231311in}}{\pgfqpoint{0.812353in}{2.234584in}}{\pgfqpoint{0.804117in}{2.234584in}}%
\pgfpathcurveto{\pgfqpoint{0.795881in}{2.234584in}}{\pgfqpoint{0.787981in}{2.231311in}}{\pgfqpoint{0.782157in}{2.225487in}}%
\pgfpathcurveto{\pgfqpoint{0.776333in}{2.219663in}}{\pgfqpoint{0.773061in}{2.211763in}}{\pgfqpoint{0.773061in}{2.203527in}}%
\pgfpathcurveto{\pgfqpoint{0.773061in}{2.195291in}}{\pgfqpoint{0.776333in}{2.187391in}}{\pgfqpoint{0.782157in}{2.181567in}}%
\pgfpathcurveto{\pgfqpoint{0.787981in}{2.175743in}}{\pgfqpoint{0.795881in}{2.172471in}}{\pgfqpoint{0.804117in}{2.172471in}}%
\pgfpathclose%
\pgfusepath{stroke,fill}%
\end{pgfscope}%
\begin{pgfscope}%
\pgfpathrectangle{\pgfqpoint{0.100000in}{0.212622in}}{\pgfqpoint{3.696000in}{3.696000in}}%
\pgfusepath{clip}%
\pgfsetbuttcap%
\pgfsetroundjoin%
\definecolor{currentfill}{rgb}{0.121569,0.466667,0.705882}%
\pgfsetfillcolor{currentfill}%
\pgfsetfillopacity{0.761426}%
\pgfsetlinewidth{1.003750pt}%
\definecolor{currentstroke}{rgb}{0.121569,0.466667,0.705882}%
\pgfsetstrokecolor{currentstroke}%
\pgfsetstrokeopacity{0.761426}%
\pgfsetdash{}{0pt}%
\pgfpathmoveto{\pgfqpoint{2.893699in}{1.817739in}}%
\pgfpathcurveto{\pgfqpoint{2.901935in}{1.817739in}}{\pgfqpoint{2.909835in}{1.821011in}}{\pgfqpoint{2.915659in}{1.826835in}}%
\pgfpathcurveto{\pgfqpoint{2.921483in}{1.832659in}}{\pgfqpoint{2.924756in}{1.840559in}}{\pgfqpoint{2.924756in}{1.848795in}}%
\pgfpathcurveto{\pgfqpoint{2.924756in}{1.857031in}}{\pgfqpoint{2.921483in}{1.864931in}}{\pgfqpoint{2.915659in}{1.870755in}}%
\pgfpathcurveto{\pgfqpoint{2.909835in}{1.876579in}}{\pgfqpoint{2.901935in}{1.879852in}}{\pgfqpoint{2.893699in}{1.879852in}}%
\pgfpathcurveto{\pgfqpoint{2.885463in}{1.879852in}}{\pgfqpoint{2.877563in}{1.876579in}}{\pgfqpoint{2.871739in}{1.870755in}}%
\pgfpathcurveto{\pgfqpoint{2.865915in}{1.864931in}}{\pgfqpoint{2.862643in}{1.857031in}}{\pgfqpoint{2.862643in}{1.848795in}}%
\pgfpathcurveto{\pgfqpoint{2.862643in}{1.840559in}}{\pgfqpoint{2.865915in}{1.832659in}}{\pgfqpoint{2.871739in}{1.826835in}}%
\pgfpathcurveto{\pgfqpoint{2.877563in}{1.821011in}}{\pgfqpoint{2.885463in}{1.817739in}}{\pgfqpoint{2.893699in}{1.817739in}}%
\pgfpathclose%
\pgfusepath{stroke,fill}%
\end{pgfscope}%
\begin{pgfscope}%
\pgfpathrectangle{\pgfqpoint{0.100000in}{0.212622in}}{\pgfqpoint{3.696000in}{3.696000in}}%
\pgfusepath{clip}%
\pgfsetbuttcap%
\pgfsetroundjoin%
\definecolor{currentfill}{rgb}{0.121569,0.466667,0.705882}%
\pgfsetfillcolor{currentfill}%
\pgfsetfillopacity{0.762534}%
\pgfsetlinewidth{1.003750pt}%
\definecolor{currentstroke}{rgb}{0.121569,0.466667,0.705882}%
\pgfsetstrokecolor{currentstroke}%
\pgfsetstrokeopacity{0.762534}%
\pgfsetdash{}{0pt}%
\pgfpathmoveto{\pgfqpoint{2.891738in}{1.817987in}}%
\pgfpathcurveto{\pgfqpoint{2.899974in}{1.817987in}}{\pgfqpoint{2.907874in}{1.821260in}}{\pgfqpoint{2.913698in}{1.827084in}}%
\pgfpathcurveto{\pgfqpoint{2.919522in}{1.832907in}}{\pgfqpoint{2.922795in}{1.840808in}}{\pgfqpoint{2.922795in}{1.849044in}}%
\pgfpathcurveto{\pgfqpoint{2.922795in}{1.857280in}}{\pgfqpoint{2.919522in}{1.865180in}}{\pgfqpoint{2.913698in}{1.871004in}}%
\pgfpathcurveto{\pgfqpoint{2.907874in}{1.876828in}}{\pgfqpoint{2.899974in}{1.880100in}}{\pgfqpoint{2.891738in}{1.880100in}}%
\pgfpathcurveto{\pgfqpoint{2.883502in}{1.880100in}}{\pgfqpoint{2.875602in}{1.876828in}}{\pgfqpoint{2.869778in}{1.871004in}}%
\pgfpathcurveto{\pgfqpoint{2.863954in}{1.865180in}}{\pgfqpoint{2.860682in}{1.857280in}}{\pgfqpoint{2.860682in}{1.849044in}}%
\pgfpathcurveto{\pgfqpoint{2.860682in}{1.840808in}}{\pgfqpoint{2.863954in}{1.832907in}}{\pgfqpoint{2.869778in}{1.827084in}}%
\pgfpathcurveto{\pgfqpoint{2.875602in}{1.821260in}}{\pgfqpoint{2.883502in}{1.817987in}}{\pgfqpoint{2.891738in}{1.817987in}}%
\pgfpathclose%
\pgfusepath{stroke,fill}%
\end{pgfscope}%
\begin{pgfscope}%
\pgfpathrectangle{\pgfqpoint{0.100000in}{0.212622in}}{\pgfqpoint{3.696000in}{3.696000in}}%
\pgfusepath{clip}%
\pgfsetbuttcap%
\pgfsetroundjoin%
\definecolor{currentfill}{rgb}{0.121569,0.466667,0.705882}%
\pgfsetfillcolor{currentfill}%
\pgfsetfillopacity{0.763040}%
\pgfsetlinewidth{1.003750pt}%
\definecolor{currentstroke}{rgb}{0.121569,0.466667,0.705882}%
\pgfsetstrokecolor{currentstroke}%
\pgfsetstrokeopacity{0.763040}%
\pgfsetdash{}{0pt}%
\pgfpathmoveto{\pgfqpoint{0.798360in}{2.172833in}}%
\pgfpathcurveto{\pgfqpoint{0.806596in}{2.172833in}}{\pgfqpoint{0.814496in}{2.176106in}}{\pgfqpoint{0.820320in}{2.181930in}}%
\pgfpathcurveto{\pgfqpoint{0.826144in}{2.187753in}}{\pgfqpoint{0.829416in}{2.195654in}}{\pgfqpoint{0.829416in}{2.203890in}}%
\pgfpathcurveto{\pgfqpoint{0.829416in}{2.212126in}}{\pgfqpoint{0.826144in}{2.220026in}}{\pgfqpoint{0.820320in}{2.225850in}}%
\pgfpathcurveto{\pgfqpoint{0.814496in}{2.231674in}}{\pgfqpoint{0.806596in}{2.234946in}}{\pgfqpoint{0.798360in}{2.234946in}}%
\pgfpathcurveto{\pgfqpoint{0.790123in}{2.234946in}}{\pgfqpoint{0.782223in}{2.231674in}}{\pgfqpoint{0.776399in}{2.225850in}}%
\pgfpathcurveto{\pgfqpoint{0.770575in}{2.220026in}}{\pgfqpoint{0.767303in}{2.212126in}}{\pgfqpoint{0.767303in}{2.203890in}}%
\pgfpathcurveto{\pgfqpoint{0.767303in}{2.195654in}}{\pgfqpoint{0.770575in}{2.187753in}}{\pgfqpoint{0.776399in}{2.181930in}}%
\pgfpathcurveto{\pgfqpoint{0.782223in}{2.176106in}}{\pgfqpoint{0.790123in}{2.172833in}}{\pgfqpoint{0.798360in}{2.172833in}}%
\pgfpathclose%
\pgfusepath{stroke,fill}%
\end{pgfscope}%
\begin{pgfscope}%
\pgfpathrectangle{\pgfqpoint{0.100000in}{0.212622in}}{\pgfqpoint{3.696000in}{3.696000in}}%
\pgfusepath{clip}%
\pgfsetbuttcap%
\pgfsetroundjoin%
\definecolor{currentfill}{rgb}{0.121569,0.466667,0.705882}%
\pgfsetfillcolor{currentfill}%
\pgfsetfillopacity{0.763757}%
\pgfsetlinewidth{1.003750pt}%
\definecolor{currentstroke}{rgb}{0.121569,0.466667,0.705882}%
\pgfsetstrokecolor{currentstroke}%
\pgfsetstrokeopacity{0.763757}%
\pgfsetdash{}{0pt}%
\pgfpathmoveto{\pgfqpoint{2.888912in}{1.818600in}}%
\pgfpathcurveto{\pgfqpoint{2.897148in}{1.818600in}}{\pgfqpoint{2.905048in}{1.821872in}}{\pgfqpoint{2.910872in}{1.827696in}}%
\pgfpathcurveto{\pgfqpoint{2.916696in}{1.833520in}}{\pgfqpoint{2.919969in}{1.841420in}}{\pgfqpoint{2.919969in}{1.849656in}}%
\pgfpathcurveto{\pgfqpoint{2.919969in}{1.857893in}}{\pgfqpoint{2.916696in}{1.865793in}}{\pgfqpoint{2.910872in}{1.871617in}}%
\pgfpathcurveto{\pgfqpoint{2.905048in}{1.877441in}}{\pgfqpoint{2.897148in}{1.880713in}}{\pgfqpoint{2.888912in}{1.880713in}}%
\pgfpathcurveto{\pgfqpoint{2.880676in}{1.880713in}}{\pgfqpoint{2.872776in}{1.877441in}}{\pgfqpoint{2.866952in}{1.871617in}}%
\pgfpathcurveto{\pgfqpoint{2.861128in}{1.865793in}}{\pgfqpoint{2.857856in}{1.857893in}}{\pgfqpoint{2.857856in}{1.849656in}}%
\pgfpathcurveto{\pgfqpoint{2.857856in}{1.841420in}}{\pgfqpoint{2.861128in}{1.833520in}}{\pgfqpoint{2.866952in}{1.827696in}}%
\pgfpathcurveto{\pgfqpoint{2.872776in}{1.821872in}}{\pgfqpoint{2.880676in}{1.818600in}}{\pgfqpoint{2.888912in}{1.818600in}}%
\pgfpathclose%
\pgfusepath{stroke,fill}%
\end{pgfscope}%
\begin{pgfscope}%
\pgfpathrectangle{\pgfqpoint{0.100000in}{0.212622in}}{\pgfqpoint{3.696000in}{3.696000in}}%
\pgfusepath{clip}%
\pgfsetbuttcap%
\pgfsetroundjoin%
\definecolor{currentfill}{rgb}{0.121569,0.466667,0.705882}%
\pgfsetfillcolor{currentfill}%
\pgfsetfillopacity{0.765906}%
\pgfsetlinewidth{1.003750pt}%
\definecolor{currentstroke}{rgb}{0.121569,0.466667,0.705882}%
\pgfsetstrokecolor{currentstroke}%
\pgfsetstrokeopacity{0.765906}%
\pgfsetdash{}{0pt}%
\pgfpathmoveto{\pgfqpoint{0.793059in}{2.172969in}}%
\pgfpathcurveto{\pgfqpoint{0.801296in}{2.172969in}}{\pgfqpoint{0.809196in}{2.176241in}}{\pgfqpoint{0.815020in}{2.182065in}}%
\pgfpathcurveto{\pgfqpoint{0.820844in}{2.187889in}}{\pgfqpoint{0.824116in}{2.195789in}}{\pgfqpoint{0.824116in}{2.204026in}}%
\pgfpathcurveto{\pgfqpoint{0.824116in}{2.212262in}}{\pgfqpoint{0.820844in}{2.220162in}}{\pgfqpoint{0.815020in}{2.225986in}}%
\pgfpathcurveto{\pgfqpoint{0.809196in}{2.231810in}}{\pgfqpoint{0.801296in}{2.235082in}}{\pgfqpoint{0.793059in}{2.235082in}}%
\pgfpathcurveto{\pgfqpoint{0.784823in}{2.235082in}}{\pgfqpoint{0.776923in}{2.231810in}}{\pgfqpoint{0.771099in}{2.225986in}}%
\pgfpathcurveto{\pgfqpoint{0.765275in}{2.220162in}}{\pgfqpoint{0.762003in}{2.212262in}}{\pgfqpoint{0.762003in}{2.204026in}}%
\pgfpathcurveto{\pgfqpoint{0.762003in}{2.195789in}}{\pgfqpoint{0.765275in}{2.187889in}}{\pgfqpoint{0.771099in}{2.182065in}}%
\pgfpathcurveto{\pgfqpoint{0.776923in}{2.176241in}}{\pgfqpoint{0.784823in}{2.172969in}}{\pgfqpoint{0.793059in}{2.172969in}}%
\pgfpathclose%
\pgfusepath{stroke,fill}%
\end{pgfscope}%
\begin{pgfscope}%
\pgfpathrectangle{\pgfqpoint{0.100000in}{0.212622in}}{\pgfqpoint{3.696000in}{3.696000in}}%
\pgfusepath{clip}%
\pgfsetbuttcap%
\pgfsetroundjoin%
\definecolor{currentfill}{rgb}{0.121569,0.466667,0.705882}%
\pgfsetfillcolor{currentfill}%
\pgfsetfillopacity{0.766193}%
\pgfsetlinewidth{1.003750pt}%
\definecolor{currentstroke}{rgb}{0.121569,0.466667,0.705882}%
\pgfsetstrokecolor{currentstroke}%
\pgfsetstrokeopacity{0.766193}%
\pgfsetdash{}{0pt}%
\pgfpathmoveto{\pgfqpoint{2.887180in}{1.818797in}}%
\pgfpathcurveto{\pgfqpoint{2.895416in}{1.818797in}}{\pgfqpoint{2.903316in}{1.822070in}}{\pgfqpoint{2.909140in}{1.827893in}}%
\pgfpathcurveto{\pgfqpoint{2.914964in}{1.833717in}}{\pgfqpoint{2.918237in}{1.841617in}}{\pgfqpoint{2.918237in}{1.849854in}}%
\pgfpathcurveto{\pgfqpoint{2.918237in}{1.858090in}}{\pgfqpoint{2.914964in}{1.865990in}}{\pgfqpoint{2.909140in}{1.871814in}}%
\pgfpathcurveto{\pgfqpoint{2.903316in}{1.877638in}}{\pgfqpoint{2.895416in}{1.880910in}}{\pgfqpoint{2.887180in}{1.880910in}}%
\pgfpathcurveto{\pgfqpoint{2.878944in}{1.880910in}}{\pgfqpoint{2.871044in}{1.877638in}}{\pgfqpoint{2.865220in}{1.871814in}}%
\pgfpathcurveto{\pgfqpoint{2.859396in}{1.865990in}}{\pgfqpoint{2.856124in}{1.858090in}}{\pgfqpoint{2.856124in}{1.849854in}}%
\pgfpathcurveto{\pgfqpoint{2.856124in}{1.841617in}}{\pgfqpoint{2.859396in}{1.833717in}}{\pgfqpoint{2.865220in}{1.827893in}}%
\pgfpathcurveto{\pgfqpoint{2.871044in}{1.822070in}}{\pgfqpoint{2.878944in}{1.818797in}}{\pgfqpoint{2.887180in}{1.818797in}}%
\pgfpathclose%
\pgfusepath{stroke,fill}%
\end{pgfscope}%
\begin{pgfscope}%
\pgfpathrectangle{\pgfqpoint{0.100000in}{0.212622in}}{\pgfqpoint{3.696000in}{3.696000in}}%
\pgfusepath{clip}%
\pgfsetbuttcap%
\pgfsetroundjoin%
\definecolor{currentfill}{rgb}{0.121569,0.466667,0.705882}%
\pgfsetfillcolor{currentfill}%
\pgfsetfillopacity{0.767434}%
\pgfsetlinewidth{1.003750pt}%
\definecolor{currentstroke}{rgb}{0.121569,0.466667,0.705882}%
\pgfsetstrokecolor{currentstroke}%
\pgfsetstrokeopacity{0.767434}%
\pgfsetdash{}{0pt}%
\pgfpathmoveto{\pgfqpoint{2.885049in}{1.819095in}}%
\pgfpathcurveto{\pgfqpoint{2.893286in}{1.819095in}}{\pgfqpoint{2.901186in}{1.822368in}}{\pgfqpoint{2.907010in}{1.828191in}}%
\pgfpathcurveto{\pgfqpoint{2.912834in}{1.834015in}}{\pgfqpoint{2.916106in}{1.841915in}}{\pgfqpoint{2.916106in}{1.850152in}}%
\pgfpathcurveto{\pgfqpoint{2.916106in}{1.858388in}}{\pgfqpoint{2.912834in}{1.866288in}}{\pgfqpoint{2.907010in}{1.872112in}}%
\pgfpathcurveto{\pgfqpoint{2.901186in}{1.877936in}}{\pgfqpoint{2.893286in}{1.881208in}}{\pgfqpoint{2.885049in}{1.881208in}}%
\pgfpathcurveto{\pgfqpoint{2.876813in}{1.881208in}}{\pgfqpoint{2.868913in}{1.877936in}}{\pgfqpoint{2.863089in}{1.872112in}}%
\pgfpathcurveto{\pgfqpoint{2.857265in}{1.866288in}}{\pgfqpoint{2.853993in}{1.858388in}}{\pgfqpoint{2.853993in}{1.850152in}}%
\pgfpathcurveto{\pgfqpoint{2.853993in}{1.841915in}}{\pgfqpoint{2.857265in}{1.834015in}}{\pgfqpoint{2.863089in}{1.828191in}}%
\pgfpathcurveto{\pgfqpoint{2.868913in}{1.822368in}}{\pgfqpoint{2.876813in}{1.819095in}}{\pgfqpoint{2.885049in}{1.819095in}}%
\pgfpathclose%
\pgfusepath{stroke,fill}%
\end{pgfscope}%
\begin{pgfscope}%
\pgfpathrectangle{\pgfqpoint{0.100000in}{0.212622in}}{\pgfqpoint{3.696000in}{3.696000in}}%
\pgfusepath{clip}%
\pgfsetbuttcap%
\pgfsetroundjoin%
\definecolor{currentfill}{rgb}{0.121569,0.466667,0.705882}%
\pgfsetfillcolor{currentfill}%
\pgfsetfillopacity{0.768063}%
\pgfsetlinewidth{1.003750pt}%
\definecolor{currentstroke}{rgb}{0.121569,0.466667,0.705882}%
\pgfsetstrokecolor{currentstroke}%
\pgfsetstrokeopacity{0.768063}%
\pgfsetdash{}{0pt}%
\pgfpathmoveto{\pgfqpoint{2.883390in}{1.819548in}}%
\pgfpathcurveto{\pgfqpoint{2.891626in}{1.819548in}}{\pgfqpoint{2.899526in}{1.822820in}}{\pgfqpoint{2.905350in}{1.828644in}}%
\pgfpathcurveto{\pgfqpoint{2.911174in}{1.834468in}}{\pgfqpoint{2.914447in}{1.842368in}}{\pgfqpoint{2.914447in}{1.850604in}}%
\pgfpathcurveto{\pgfqpoint{2.914447in}{1.858841in}}{\pgfqpoint{2.911174in}{1.866741in}}{\pgfqpoint{2.905350in}{1.872565in}}%
\pgfpathcurveto{\pgfqpoint{2.899526in}{1.878389in}}{\pgfqpoint{2.891626in}{1.881661in}}{\pgfqpoint{2.883390in}{1.881661in}}%
\pgfpathcurveto{\pgfqpoint{2.875154in}{1.881661in}}{\pgfqpoint{2.867254in}{1.878389in}}{\pgfqpoint{2.861430in}{1.872565in}}%
\pgfpathcurveto{\pgfqpoint{2.855606in}{1.866741in}}{\pgfqpoint{2.852334in}{1.858841in}}{\pgfqpoint{2.852334in}{1.850604in}}%
\pgfpathcurveto{\pgfqpoint{2.852334in}{1.842368in}}{\pgfqpoint{2.855606in}{1.834468in}}{\pgfqpoint{2.861430in}{1.828644in}}%
\pgfpathcurveto{\pgfqpoint{2.867254in}{1.822820in}}{\pgfqpoint{2.875154in}{1.819548in}}{\pgfqpoint{2.883390in}{1.819548in}}%
\pgfpathclose%
\pgfusepath{stroke,fill}%
\end{pgfscope}%
\begin{pgfscope}%
\pgfpathrectangle{\pgfqpoint{0.100000in}{0.212622in}}{\pgfqpoint{3.696000in}{3.696000in}}%
\pgfusepath{clip}%
\pgfsetbuttcap%
\pgfsetroundjoin%
\definecolor{currentfill}{rgb}{0.121569,0.466667,0.705882}%
\pgfsetfillcolor{currentfill}%
\pgfsetfillopacity{0.768376}%
\pgfsetlinewidth{1.003750pt}%
\definecolor{currentstroke}{rgb}{0.121569,0.466667,0.705882}%
\pgfsetstrokecolor{currentstroke}%
\pgfsetstrokeopacity{0.768376}%
\pgfsetdash{}{0pt}%
\pgfpathmoveto{\pgfqpoint{0.786550in}{2.173821in}}%
\pgfpathcurveto{\pgfqpoint{0.794787in}{2.173821in}}{\pgfqpoint{0.802687in}{2.177093in}}{\pgfqpoint{0.808511in}{2.182917in}}%
\pgfpathcurveto{\pgfqpoint{0.814334in}{2.188741in}}{\pgfqpoint{0.817607in}{2.196641in}}{\pgfqpoint{0.817607in}{2.204878in}}%
\pgfpathcurveto{\pgfqpoint{0.817607in}{2.213114in}}{\pgfqpoint{0.814334in}{2.221014in}}{\pgfqpoint{0.808511in}{2.226838in}}%
\pgfpathcurveto{\pgfqpoint{0.802687in}{2.232662in}}{\pgfqpoint{0.794787in}{2.235934in}}{\pgfqpoint{0.786550in}{2.235934in}}%
\pgfpathcurveto{\pgfqpoint{0.778314in}{2.235934in}}{\pgfqpoint{0.770414in}{2.232662in}}{\pgfqpoint{0.764590in}{2.226838in}}%
\pgfpathcurveto{\pgfqpoint{0.758766in}{2.221014in}}{\pgfqpoint{0.755494in}{2.213114in}}{\pgfqpoint{0.755494in}{2.204878in}}%
\pgfpathcurveto{\pgfqpoint{0.755494in}{2.196641in}}{\pgfqpoint{0.758766in}{2.188741in}}{\pgfqpoint{0.764590in}{2.182917in}}%
\pgfpathcurveto{\pgfqpoint{0.770414in}{2.177093in}}{\pgfqpoint{0.778314in}{2.173821in}}{\pgfqpoint{0.786550in}{2.173821in}}%
\pgfpathclose%
\pgfusepath{stroke,fill}%
\end{pgfscope}%
\begin{pgfscope}%
\pgfpathrectangle{\pgfqpoint{0.100000in}{0.212622in}}{\pgfqpoint{3.696000in}{3.696000in}}%
\pgfusepath{clip}%
\pgfsetbuttcap%
\pgfsetroundjoin%
\definecolor{currentfill}{rgb}{0.121569,0.466667,0.705882}%
\pgfsetfillcolor{currentfill}%
\pgfsetfillopacity{0.770042}%
\pgfsetlinewidth{1.003750pt}%
\definecolor{currentstroke}{rgb}{0.121569,0.466667,0.705882}%
\pgfsetstrokecolor{currentstroke}%
\pgfsetstrokeopacity{0.770042}%
\pgfsetdash{}{0pt}%
\pgfpathmoveto{\pgfqpoint{2.882060in}{1.819562in}}%
\pgfpathcurveto{\pgfqpoint{2.890296in}{1.819562in}}{\pgfqpoint{2.898196in}{1.822834in}}{\pgfqpoint{2.904020in}{1.828658in}}%
\pgfpathcurveto{\pgfqpoint{2.909844in}{1.834482in}}{\pgfqpoint{2.913116in}{1.842382in}}{\pgfqpoint{2.913116in}{1.850618in}}%
\pgfpathcurveto{\pgfqpoint{2.913116in}{1.858855in}}{\pgfqpoint{2.909844in}{1.866755in}}{\pgfqpoint{2.904020in}{1.872578in}}%
\pgfpathcurveto{\pgfqpoint{2.898196in}{1.878402in}}{\pgfqpoint{2.890296in}{1.881675in}}{\pgfqpoint{2.882060in}{1.881675in}}%
\pgfpathcurveto{\pgfqpoint{2.873824in}{1.881675in}}{\pgfqpoint{2.865924in}{1.878402in}}{\pgfqpoint{2.860100in}{1.872578in}}%
\pgfpathcurveto{\pgfqpoint{2.854276in}{1.866755in}}{\pgfqpoint{2.851003in}{1.858855in}}{\pgfqpoint{2.851003in}{1.850618in}}%
\pgfpathcurveto{\pgfqpoint{2.851003in}{1.842382in}}{\pgfqpoint{2.854276in}{1.834482in}}{\pgfqpoint{2.860100in}{1.828658in}}%
\pgfpathcurveto{\pgfqpoint{2.865924in}{1.822834in}}{\pgfqpoint{2.873824in}{1.819562in}}{\pgfqpoint{2.882060in}{1.819562in}}%
\pgfpathclose%
\pgfusepath{stroke,fill}%
\end{pgfscope}%
\begin{pgfscope}%
\pgfpathrectangle{\pgfqpoint{0.100000in}{0.212622in}}{\pgfqpoint{3.696000in}{3.696000in}}%
\pgfusepath{clip}%
\pgfsetbuttcap%
\pgfsetroundjoin%
\definecolor{currentfill}{rgb}{0.121569,0.466667,0.705882}%
\pgfsetfillcolor{currentfill}%
\pgfsetfillopacity{0.770729}%
\pgfsetlinewidth{1.003750pt}%
\definecolor{currentstroke}{rgb}{0.121569,0.466667,0.705882}%
\pgfsetstrokecolor{currentstroke}%
\pgfsetstrokeopacity{0.770729}%
\pgfsetdash{}{0pt}%
\pgfpathmoveto{\pgfqpoint{0.783582in}{2.174147in}}%
\pgfpathcurveto{\pgfqpoint{0.791818in}{2.174147in}}{\pgfqpoint{0.799718in}{2.177419in}}{\pgfqpoint{0.805542in}{2.183243in}}%
\pgfpathcurveto{\pgfqpoint{0.811366in}{2.189067in}}{\pgfqpoint{0.814638in}{2.196967in}}{\pgfqpoint{0.814638in}{2.205203in}}%
\pgfpathcurveto{\pgfqpoint{0.814638in}{2.213440in}}{\pgfqpoint{0.811366in}{2.221340in}}{\pgfqpoint{0.805542in}{2.227164in}}%
\pgfpathcurveto{\pgfqpoint{0.799718in}{2.232988in}}{\pgfqpoint{0.791818in}{2.236260in}}{\pgfqpoint{0.783582in}{2.236260in}}%
\pgfpathcurveto{\pgfqpoint{0.775346in}{2.236260in}}{\pgfqpoint{0.767446in}{2.232988in}}{\pgfqpoint{0.761622in}{2.227164in}}%
\pgfpathcurveto{\pgfqpoint{0.755798in}{2.221340in}}{\pgfqpoint{0.752525in}{2.213440in}}{\pgfqpoint{0.752525in}{2.205203in}}%
\pgfpathcurveto{\pgfqpoint{0.752525in}{2.196967in}}{\pgfqpoint{0.755798in}{2.189067in}}{\pgfqpoint{0.761622in}{2.183243in}}%
\pgfpathcurveto{\pgfqpoint{0.767446in}{2.177419in}}{\pgfqpoint{0.775346in}{2.174147in}}{\pgfqpoint{0.783582in}{2.174147in}}%
\pgfpathclose%
\pgfusepath{stroke,fill}%
\end{pgfscope}%
\begin{pgfscope}%
\pgfpathrectangle{\pgfqpoint{0.100000in}{0.212622in}}{\pgfqpoint{3.696000in}{3.696000in}}%
\pgfusepath{clip}%
\pgfsetbuttcap%
\pgfsetroundjoin%
\definecolor{currentfill}{rgb}{0.121569,0.466667,0.705882}%
\pgfsetfillcolor{currentfill}%
\pgfsetfillopacity{0.771073}%
\pgfsetlinewidth{1.003750pt}%
\definecolor{currentstroke}{rgb}{0.121569,0.466667,0.705882}%
\pgfsetstrokecolor{currentstroke}%
\pgfsetstrokeopacity{0.771073}%
\pgfsetdash{}{0pt}%
\pgfpathmoveto{\pgfqpoint{2.880529in}{1.819712in}}%
\pgfpathcurveto{\pgfqpoint{2.888765in}{1.819712in}}{\pgfqpoint{2.896666in}{1.822985in}}{\pgfqpoint{2.902489in}{1.828809in}}%
\pgfpathcurveto{\pgfqpoint{2.908313in}{1.834632in}}{\pgfqpoint{2.911586in}{1.842533in}}{\pgfqpoint{2.911586in}{1.850769in}}%
\pgfpathcurveto{\pgfqpoint{2.911586in}{1.859005in}}{\pgfqpoint{2.908313in}{1.866905in}}{\pgfqpoint{2.902489in}{1.872729in}}%
\pgfpathcurveto{\pgfqpoint{2.896666in}{1.878553in}}{\pgfqpoint{2.888765in}{1.881825in}}{\pgfqpoint{2.880529in}{1.881825in}}%
\pgfpathcurveto{\pgfqpoint{2.872293in}{1.881825in}}{\pgfqpoint{2.864393in}{1.878553in}}{\pgfqpoint{2.858569in}{1.872729in}}%
\pgfpathcurveto{\pgfqpoint{2.852745in}{1.866905in}}{\pgfqpoint{2.849473in}{1.859005in}}{\pgfqpoint{2.849473in}{1.850769in}}%
\pgfpathcurveto{\pgfqpoint{2.849473in}{1.842533in}}{\pgfqpoint{2.852745in}{1.834632in}}{\pgfqpoint{2.858569in}{1.828809in}}%
\pgfpathcurveto{\pgfqpoint{2.864393in}{1.822985in}}{\pgfqpoint{2.872293in}{1.819712in}}{\pgfqpoint{2.880529in}{1.819712in}}%
\pgfpathclose%
\pgfusepath{stroke,fill}%
\end{pgfscope}%
\begin{pgfscope}%
\pgfpathrectangle{\pgfqpoint{0.100000in}{0.212622in}}{\pgfqpoint{3.696000in}{3.696000in}}%
\pgfusepath{clip}%
\pgfsetbuttcap%
\pgfsetroundjoin%
\definecolor{currentfill}{rgb}{0.121569,0.466667,0.705882}%
\pgfsetfillcolor{currentfill}%
\pgfsetfillopacity{0.772150}%
\pgfsetlinewidth{1.003750pt}%
\definecolor{currentstroke}{rgb}{0.121569,0.466667,0.705882}%
\pgfsetstrokecolor{currentstroke}%
\pgfsetstrokeopacity{0.772150}%
\pgfsetdash{}{0pt}%
\pgfpathmoveto{\pgfqpoint{2.877837in}{1.820406in}}%
\pgfpathcurveto{\pgfqpoint{2.886073in}{1.820406in}}{\pgfqpoint{2.893973in}{1.823678in}}{\pgfqpoint{2.899797in}{1.829502in}}%
\pgfpathcurveto{\pgfqpoint{2.905621in}{1.835326in}}{\pgfqpoint{2.908893in}{1.843226in}}{\pgfqpoint{2.908893in}{1.851462in}}%
\pgfpathcurveto{\pgfqpoint{2.908893in}{1.859699in}}{\pgfqpoint{2.905621in}{1.867599in}}{\pgfqpoint{2.899797in}{1.873423in}}%
\pgfpathcurveto{\pgfqpoint{2.893973in}{1.879247in}}{\pgfqpoint{2.886073in}{1.882519in}}{\pgfqpoint{2.877837in}{1.882519in}}%
\pgfpathcurveto{\pgfqpoint{2.869601in}{1.882519in}}{\pgfqpoint{2.861701in}{1.879247in}}{\pgfqpoint{2.855877in}{1.873423in}}%
\pgfpathcurveto{\pgfqpoint{2.850053in}{1.867599in}}{\pgfqpoint{2.846780in}{1.859699in}}{\pgfqpoint{2.846780in}{1.851462in}}%
\pgfpathcurveto{\pgfqpoint{2.846780in}{1.843226in}}{\pgfqpoint{2.850053in}{1.835326in}}{\pgfqpoint{2.855877in}{1.829502in}}%
\pgfpathcurveto{\pgfqpoint{2.861701in}{1.823678in}}{\pgfqpoint{2.869601in}{1.820406in}}{\pgfqpoint{2.877837in}{1.820406in}}%
\pgfpathclose%
\pgfusepath{stroke,fill}%
\end{pgfscope}%
\begin{pgfscope}%
\pgfpathrectangle{\pgfqpoint{0.100000in}{0.212622in}}{\pgfqpoint{3.696000in}{3.696000in}}%
\pgfusepath{clip}%
\pgfsetbuttcap%
\pgfsetroundjoin%
\definecolor{currentfill}{rgb}{0.121569,0.466667,0.705882}%
\pgfsetfillcolor{currentfill}%
\pgfsetfillopacity{0.772307}%
\pgfsetlinewidth{1.003750pt}%
\definecolor{currentstroke}{rgb}{0.121569,0.466667,0.705882}%
\pgfsetstrokecolor{currentstroke}%
\pgfsetstrokeopacity{0.772307}%
\pgfsetdash{}{0pt}%
\pgfpathmoveto{\pgfqpoint{0.778853in}{2.175198in}}%
\pgfpathcurveto{\pgfqpoint{0.787089in}{2.175198in}}{\pgfqpoint{0.794989in}{2.178470in}}{\pgfqpoint{0.800813in}{2.184294in}}%
\pgfpathcurveto{\pgfqpoint{0.806637in}{2.190118in}}{\pgfqpoint{0.809909in}{2.198018in}}{\pgfqpoint{0.809909in}{2.206254in}}%
\pgfpathcurveto{\pgfqpoint{0.809909in}{2.214491in}}{\pgfqpoint{0.806637in}{2.222391in}}{\pgfqpoint{0.800813in}{2.228215in}}%
\pgfpathcurveto{\pgfqpoint{0.794989in}{2.234039in}}{\pgfqpoint{0.787089in}{2.237311in}}{\pgfqpoint{0.778853in}{2.237311in}}%
\pgfpathcurveto{\pgfqpoint{0.770617in}{2.237311in}}{\pgfqpoint{0.762716in}{2.234039in}}{\pgfqpoint{0.756893in}{2.228215in}}%
\pgfpathcurveto{\pgfqpoint{0.751069in}{2.222391in}}{\pgfqpoint{0.747796in}{2.214491in}}{\pgfqpoint{0.747796in}{2.206254in}}%
\pgfpathcurveto{\pgfqpoint{0.747796in}{2.198018in}}{\pgfqpoint{0.751069in}{2.190118in}}{\pgfqpoint{0.756893in}{2.184294in}}%
\pgfpathcurveto{\pgfqpoint{0.762716in}{2.178470in}}{\pgfqpoint{0.770617in}{2.175198in}}{\pgfqpoint{0.778853in}{2.175198in}}%
\pgfpathclose%
\pgfusepath{stroke,fill}%
\end{pgfscope}%
\begin{pgfscope}%
\pgfpathrectangle{\pgfqpoint{0.100000in}{0.212622in}}{\pgfqpoint{3.696000in}{3.696000in}}%
\pgfusepath{clip}%
\pgfsetbuttcap%
\pgfsetroundjoin%
\definecolor{currentfill}{rgb}{0.121569,0.466667,0.705882}%
\pgfsetfillcolor{currentfill}%
\pgfsetfillopacity{0.773535}%
\pgfsetlinewidth{1.003750pt}%
\definecolor{currentstroke}{rgb}{0.121569,0.466667,0.705882}%
\pgfsetstrokecolor{currentstroke}%
\pgfsetstrokeopacity{0.773535}%
\pgfsetdash{}{0pt}%
\pgfpathmoveto{\pgfqpoint{0.776976in}{2.175341in}}%
\pgfpathcurveto{\pgfqpoint{0.785212in}{2.175341in}}{\pgfqpoint{0.793112in}{2.178613in}}{\pgfqpoint{0.798936in}{2.184437in}}%
\pgfpathcurveto{\pgfqpoint{0.804760in}{2.190261in}}{\pgfqpoint{0.808032in}{2.198161in}}{\pgfqpoint{0.808032in}{2.206398in}}%
\pgfpathcurveto{\pgfqpoint{0.808032in}{2.214634in}}{\pgfqpoint{0.804760in}{2.222534in}}{\pgfqpoint{0.798936in}{2.228358in}}%
\pgfpathcurveto{\pgfqpoint{0.793112in}{2.234182in}}{\pgfqpoint{0.785212in}{2.237454in}}{\pgfqpoint{0.776976in}{2.237454in}}%
\pgfpathcurveto{\pgfqpoint{0.768739in}{2.237454in}}{\pgfqpoint{0.760839in}{2.234182in}}{\pgfqpoint{0.755015in}{2.228358in}}%
\pgfpathcurveto{\pgfqpoint{0.749191in}{2.222534in}}{\pgfqpoint{0.745919in}{2.214634in}}{\pgfqpoint{0.745919in}{2.206398in}}%
\pgfpathcurveto{\pgfqpoint{0.745919in}{2.198161in}}{\pgfqpoint{0.749191in}{2.190261in}}{\pgfqpoint{0.755015in}{2.184437in}}%
\pgfpathcurveto{\pgfqpoint{0.760839in}{2.178613in}}{\pgfqpoint{0.768739in}{2.175341in}}{\pgfqpoint{0.776976in}{2.175341in}}%
\pgfpathclose%
\pgfusepath{stroke,fill}%
\end{pgfscope}%
\begin{pgfscope}%
\pgfpathrectangle{\pgfqpoint{0.100000in}{0.212622in}}{\pgfqpoint{3.696000in}{3.696000in}}%
\pgfusepath{clip}%
\pgfsetbuttcap%
\pgfsetroundjoin%
\definecolor{currentfill}{rgb}{0.121569,0.466667,0.705882}%
\pgfsetfillcolor{currentfill}%
\pgfsetfillopacity{0.774054}%
\pgfsetlinewidth{1.003750pt}%
\definecolor{currentstroke}{rgb}{0.121569,0.466667,0.705882}%
\pgfsetstrokecolor{currentstroke}%
\pgfsetstrokeopacity{0.774054}%
\pgfsetdash{}{0pt}%
\pgfpathmoveto{\pgfqpoint{2.875504in}{1.820362in}}%
\pgfpathcurveto{\pgfqpoint{2.883740in}{1.820362in}}{\pgfqpoint{2.891640in}{1.823635in}}{\pgfqpoint{2.897464in}{1.829459in}}%
\pgfpathcurveto{\pgfqpoint{2.903288in}{1.835282in}}{\pgfqpoint{2.906560in}{1.843183in}}{\pgfqpoint{2.906560in}{1.851419in}}%
\pgfpathcurveto{\pgfqpoint{2.906560in}{1.859655in}}{\pgfqpoint{2.903288in}{1.867555in}}{\pgfqpoint{2.897464in}{1.873379in}}%
\pgfpathcurveto{\pgfqpoint{2.891640in}{1.879203in}}{\pgfqpoint{2.883740in}{1.882475in}}{\pgfqpoint{2.875504in}{1.882475in}}%
\pgfpathcurveto{\pgfqpoint{2.867267in}{1.882475in}}{\pgfqpoint{2.859367in}{1.879203in}}{\pgfqpoint{2.853543in}{1.873379in}}%
\pgfpathcurveto{\pgfqpoint{2.847719in}{1.867555in}}{\pgfqpoint{2.844447in}{1.859655in}}{\pgfqpoint{2.844447in}{1.851419in}}%
\pgfpathcurveto{\pgfqpoint{2.844447in}{1.843183in}}{\pgfqpoint{2.847719in}{1.835282in}}{\pgfqpoint{2.853543in}{1.829459in}}%
\pgfpathcurveto{\pgfqpoint{2.859367in}{1.823635in}}{\pgfqpoint{2.867267in}{1.820362in}}{\pgfqpoint{2.875504in}{1.820362in}}%
\pgfpathclose%
\pgfusepath{stroke,fill}%
\end{pgfscope}%
\begin{pgfscope}%
\pgfpathrectangle{\pgfqpoint{0.100000in}{0.212622in}}{\pgfqpoint{3.696000in}{3.696000in}}%
\pgfusepath{clip}%
\pgfsetbuttcap%
\pgfsetroundjoin%
\definecolor{currentfill}{rgb}{0.121569,0.466667,0.705882}%
\pgfsetfillcolor{currentfill}%
\pgfsetfillopacity{0.774259}%
\pgfsetlinewidth{1.003750pt}%
\definecolor{currentstroke}{rgb}{0.121569,0.466667,0.705882}%
\pgfsetstrokecolor{currentstroke}%
\pgfsetstrokeopacity{0.774259}%
\pgfsetdash{}{0pt}%
\pgfpathmoveto{\pgfqpoint{0.774758in}{2.175764in}}%
\pgfpathcurveto{\pgfqpoint{0.782994in}{2.175764in}}{\pgfqpoint{0.790894in}{2.179036in}}{\pgfqpoint{0.796718in}{2.184860in}}%
\pgfpathcurveto{\pgfqpoint{0.802542in}{2.190684in}}{\pgfqpoint{0.805815in}{2.198584in}}{\pgfqpoint{0.805815in}{2.206821in}}%
\pgfpathcurveto{\pgfqpoint{0.805815in}{2.215057in}}{\pgfqpoint{0.802542in}{2.222957in}}{\pgfqpoint{0.796718in}{2.228781in}}%
\pgfpathcurveto{\pgfqpoint{0.790894in}{2.234605in}}{\pgfqpoint{0.782994in}{2.237877in}}{\pgfqpoint{0.774758in}{2.237877in}}%
\pgfpathcurveto{\pgfqpoint{0.766522in}{2.237877in}}{\pgfqpoint{0.758622in}{2.234605in}}{\pgfqpoint{0.752798in}{2.228781in}}%
\pgfpathcurveto{\pgfqpoint{0.746974in}{2.222957in}}{\pgfqpoint{0.743702in}{2.215057in}}{\pgfqpoint{0.743702in}{2.206821in}}%
\pgfpathcurveto{\pgfqpoint{0.743702in}{2.198584in}}{\pgfqpoint{0.746974in}{2.190684in}}{\pgfqpoint{0.752798in}{2.184860in}}%
\pgfpathcurveto{\pgfqpoint{0.758622in}{2.179036in}}{\pgfqpoint{0.766522in}{2.175764in}}{\pgfqpoint{0.774758in}{2.175764in}}%
\pgfpathclose%
\pgfusepath{stroke,fill}%
\end{pgfscope}%
\begin{pgfscope}%
\pgfpathrectangle{\pgfqpoint{0.100000in}{0.212622in}}{\pgfqpoint{3.696000in}{3.696000in}}%
\pgfusepath{clip}%
\pgfsetbuttcap%
\pgfsetroundjoin%
\definecolor{currentfill}{rgb}{0.121569,0.466667,0.705882}%
\pgfsetfillcolor{currentfill}%
\pgfsetfillopacity{0.774384}%
\pgfsetlinewidth{1.003750pt}%
\definecolor{currentstroke}{rgb}{0.121569,0.466667,0.705882}%
\pgfsetstrokecolor{currentstroke}%
\pgfsetstrokeopacity{0.774384}%
\pgfsetdash{}{0pt}%
\pgfpathmoveto{\pgfqpoint{0.774636in}{2.175788in}}%
\pgfpathcurveto{\pgfqpoint{0.782872in}{2.175788in}}{\pgfqpoint{0.790772in}{2.179060in}}{\pgfqpoint{0.796596in}{2.184884in}}%
\pgfpathcurveto{\pgfqpoint{0.802420in}{2.190708in}}{\pgfqpoint{0.805693in}{2.198608in}}{\pgfqpoint{0.805693in}{2.206844in}}%
\pgfpathcurveto{\pgfqpoint{0.805693in}{2.215081in}}{\pgfqpoint{0.802420in}{2.222981in}}{\pgfqpoint{0.796596in}{2.228805in}}%
\pgfpathcurveto{\pgfqpoint{0.790772in}{2.234629in}}{\pgfqpoint{0.782872in}{2.237901in}}{\pgfqpoint{0.774636in}{2.237901in}}%
\pgfpathcurveto{\pgfqpoint{0.766400in}{2.237901in}}{\pgfqpoint{0.758500in}{2.234629in}}{\pgfqpoint{0.752676in}{2.228805in}}%
\pgfpathcurveto{\pgfqpoint{0.746852in}{2.222981in}}{\pgfqpoint{0.743580in}{2.215081in}}{\pgfqpoint{0.743580in}{2.206844in}}%
\pgfpathcurveto{\pgfqpoint{0.743580in}{2.198608in}}{\pgfqpoint{0.746852in}{2.190708in}}{\pgfqpoint{0.752676in}{2.184884in}}%
\pgfpathcurveto{\pgfqpoint{0.758500in}{2.179060in}}{\pgfqpoint{0.766400in}{2.175788in}}{\pgfqpoint{0.774636in}{2.175788in}}%
\pgfpathclose%
\pgfusepath{stroke,fill}%
\end{pgfscope}%
\begin{pgfscope}%
\pgfpathrectangle{\pgfqpoint{0.100000in}{0.212622in}}{\pgfqpoint{3.696000in}{3.696000in}}%
\pgfusepath{clip}%
\pgfsetbuttcap%
\pgfsetroundjoin%
\definecolor{currentfill}{rgb}{0.121569,0.466667,0.705882}%
\pgfsetfillcolor{currentfill}%
\pgfsetfillopacity{0.774384}%
\pgfsetlinewidth{1.003750pt}%
\definecolor{currentstroke}{rgb}{0.121569,0.466667,0.705882}%
\pgfsetstrokecolor{currentstroke}%
\pgfsetstrokeopacity{0.774384}%
\pgfsetdash{}{0pt}%
\pgfpathmoveto{\pgfqpoint{0.774636in}{2.175788in}}%
\pgfpathcurveto{\pgfqpoint{0.782872in}{2.175788in}}{\pgfqpoint{0.790772in}{2.179060in}}{\pgfqpoint{0.796596in}{2.184884in}}%
\pgfpathcurveto{\pgfqpoint{0.802420in}{2.190708in}}{\pgfqpoint{0.805693in}{2.198608in}}{\pgfqpoint{0.805693in}{2.206844in}}%
\pgfpathcurveto{\pgfqpoint{0.805693in}{2.215081in}}{\pgfqpoint{0.802420in}{2.222981in}}{\pgfqpoint{0.796596in}{2.228805in}}%
\pgfpathcurveto{\pgfqpoint{0.790772in}{2.234629in}}{\pgfqpoint{0.782872in}{2.237901in}}{\pgfqpoint{0.774636in}{2.237901in}}%
\pgfpathcurveto{\pgfqpoint{0.766400in}{2.237901in}}{\pgfqpoint{0.758500in}{2.234629in}}{\pgfqpoint{0.752676in}{2.228805in}}%
\pgfpathcurveto{\pgfqpoint{0.746852in}{2.222981in}}{\pgfqpoint{0.743580in}{2.215081in}}{\pgfqpoint{0.743580in}{2.206844in}}%
\pgfpathcurveto{\pgfqpoint{0.743580in}{2.198608in}}{\pgfqpoint{0.746852in}{2.190708in}}{\pgfqpoint{0.752676in}{2.184884in}}%
\pgfpathcurveto{\pgfqpoint{0.758500in}{2.179060in}}{\pgfqpoint{0.766400in}{2.175788in}}{\pgfqpoint{0.774636in}{2.175788in}}%
\pgfpathclose%
\pgfusepath{stroke,fill}%
\end{pgfscope}%
\begin{pgfscope}%
\pgfpathrectangle{\pgfqpoint{0.100000in}{0.212622in}}{\pgfqpoint{3.696000in}{3.696000in}}%
\pgfusepath{clip}%
\pgfsetbuttcap%
\pgfsetroundjoin%
\definecolor{currentfill}{rgb}{0.121569,0.466667,0.705882}%
\pgfsetfillcolor{currentfill}%
\pgfsetfillopacity{0.774384}%
\pgfsetlinewidth{1.003750pt}%
\definecolor{currentstroke}{rgb}{0.121569,0.466667,0.705882}%
\pgfsetstrokecolor{currentstroke}%
\pgfsetstrokeopacity{0.774384}%
\pgfsetdash{}{0pt}%
\pgfpathmoveto{\pgfqpoint{0.774636in}{2.175788in}}%
\pgfpathcurveto{\pgfqpoint{0.782872in}{2.175788in}}{\pgfqpoint{0.790772in}{2.179060in}}{\pgfqpoint{0.796596in}{2.184884in}}%
\pgfpathcurveto{\pgfqpoint{0.802420in}{2.190708in}}{\pgfqpoint{0.805692in}{2.198608in}}{\pgfqpoint{0.805692in}{2.206845in}}%
\pgfpathcurveto{\pgfqpoint{0.805692in}{2.215081in}}{\pgfqpoint{0.802420in}{2.222981in}}{\pgfqpoint{0.796596in}{2.228805in}}%
\pgfpathcurveto{\pgfqpoint{0.790772in}{2.234629in}}{\pgfqpoint{0.782872in}{2.237901in}}{\pgfqpoint{0.774636in}{2.237901in}}%
\pgfpathcurveto{\pgfqpoint{0.766399in}{2.237901in}}{\pgfqpoint{0.758499in}{2.234629in}}{\pgfqpoint{0.752675in}{2.228805in}}%
\pgfpathcurveto{\pgfqpoint{0.746852in}{2.222981in}}{\pgfqpoint{0.743579in}{2.215081in}}{\pgfqpoint{0.743579in}{2.206845in}}%
\pgfpathcurveto{\pgfqpoint{0.743579in}{2.198608in}}{\pgfqpoint{0.746852in}{2.190708in}}{\pgfqpoint{0.752675in}{2.184884in}}%
\pgfpathcurveto{\pgfqpoint{0.758499in}{2.179060in}}{\pgfqpoint{0.766399in}{2.175788in}}{\pgfqpoint{0.774636in}{2.175788in}}%
\pgfpathclose%
\pgfusepath{stroke,fill}%
\end{pgfscope}%
\begin{pgfscope}%
\pgfpathrectangle{\pgfqpoint{0.100000in}{0.212622in}}{\pgfqpoint{3.696000in}{3.696000in}}%
\pgfusepath{clip}%
\pgfsetbuttcap%
\pgfsetroundjoin%
\definecolor{currentfill}{rgb}{0.121569,0.466667,0.705882}%
\pgfsetfillcolor{currentfill}%
\pgfsetfillopacity{0.774384}%
\pgfsetlinewidth{1.003750pt}%
\definecolor{currentstroke}{rgb}{0.121569,0.466667,0.705882}%
\pgfsetstrokecolor{currentstroke}%
\pgfsetstrokeopacity{0.774384}%
\pgfsetdash{}{0pt}%
\pgfpathmoveto{\pgfqpoint{0.774635in}{2.175788in}}%
\pgfpathcurveto{\pgfqpoint{0.782872in}{2.175788in}}{\pgfqpoint{0.790772in}{2.179060in}}{\pgfqpoint{0.796596in}{2.184884in}}%
\pgfpathcurveto{\pgfqpoint{0.802420in}{2.190708in}}{\pgfqpoint{0.805692in}{2.198608in}}{\pgfqpoint{0.805692in}{2.206845in}}%
\pgfpathcurveto{\pgfqpoint{0.805692in}{2.215081in}}{\pgfqpoint{0.802420in}{2.222981in}}{\pgfqpoint{0.796596in}{2.228805in}}%
\pgfpathcurveto{\pgfqpoint{0.790772in}{2.234629in}}{\pgfqpoint{0.782872in}{2.237901in}}{\pgfqpoint{0.774635in}{2.237901in}}%
\pgfpathcurveto{\pgfqpoint{0.766399in}{2.237901in}}{\pgfqpoint{0.758499in}{2.234629in}}{\pgfqpoint{0.752675in}{2.228805in}}%
\pgfpathcurveto{\pgfqpoint{0.746851in}{2.222981in}}{\pgfqpoint{0.743579in}{2.215081in}}{\pgfqpoint{0.743579in}{2.206845in}}%
\pgfpathcurveto{\pgfqpoint{0.743579in}{2.198608in}}{\pgfqpoint{0.746851in}{2.190708in}}{\pgfqpoint{0.752675in}{2.184884in}}%
\pgfpathcurveto{\pgfqpoint{0.758499in}{2.179060in}}{\pgfqpoint{0.766399in}{2.175788in}}{\pgfqpoint{0.774635in}{2.175788in}}%
\pgfpathclose%
\pgfusepath{stroke,fill}%
\end{pgfscope}%
\begin{pgfscope}%
\pgfpathrectangle{\pgfqpoint{0.100000in}{0.212622in}}{\pgfqpoint{3.696000in}{3.696000in}}%
\pgfusepath{clip}%
\pgfsetbuttcap%
\pgfsetroundjoin%
\definecolor{currentfill}{rgb}{0.121569,0.466667,0.705882}%
\pgfsetfillcolor{currentfill}%
\pgfsetfillopacity{0.774385}%
\pgfsetlinewidth{1.003750pt}%
\definecolor{currentstroke}{rgb}{0.121569,0.466667,0.705882}%
\pgfsetstrokecolor{currentstroke}%
\pgfsetstrokeopacity{0.774385}%
\pgfsetdash{}{0pt}%
\pgfpathmoveto{\pgfqpoint{0.774634in}{2.175788in}}%
\pgfpathcurveto{\pgfqpoint{0.782871in}{2.175788in}}{\pgfqpoint{0.790771in}{2.179060in}}{\pgfqpoint{0.796595in}{2.184884in}}%
\pgfpathcurveto{\pgfqpoint{0.802419in}{2.190708in}}{\pgfqpoint{0.805691in}{2.198608in}}{\pgfqpoint{0.805691in}{2.206845in}}%
\pgfpathcurveto{\pgfqpoint{0.805691in}{2.215081in}}{\pgfqpoint{0.802419in}{2.222981in}}{\pgfqpoint{0.796595in}{2.228805in}}%
\pgfpathcurveto{\pgfqpoint{0.790771in}{2.234629in}}{\pgfqpoint{0.782871in}{2.237901in}}{\pgfqpoint{0.774634in}{2.237901in}}%
\pgfpathcurveto{\pgfqpoint{0.766398in}{2.237901in}}{\pgfqpoint{0.758498in}{2.234629in}}{\pgfqpoint{0.752674in}{2.228805in}}%
\pgfpathcurveto{\pgfqpoint{0.746850in}{2.222981in}}{\pgfqpoint{0.743578in}{2.215081in}}{\pgfqpoint{0.743578in}{2.206845in}}%
\pgfpathcurveto{\pgfqpoint{0.743578in}{2.198608in}}{\pgfqpoint{0.746850in}{2.190708in}}{\pgfqpoint{0.752674in}{2.184884in}}%
\pgfpathcurveto{\pgfqpoint{0.758498in}{2.179060in}}{\pgfqpoint{0.766398in}{2.175788in}}{\pgfqpoint{0.774634in}{2.175788in}}%
\pgfpathclose%
\pgfusepath{stroke,fill}%
\end{pgfscope}%
\begin{pgfscope}%
\pgfpathrectangle{\pgfqpoint{0.100000in}{0.212622in}}{\pgfqpoint{3.696000in}{3.696000in}}%
\pgfusepath{clip}%
\pgfsetbuttcap%
\pgfsetroundjoin%
\definecolor{currentfill}{rgb}{0.121569,0.466667,0.705882}%
\pgfsetfillcolor{currentfill}%
\pgfsetfillopacity{0.774386}%
\pgfsetlinewidth{1.003750pt}%
\definecolor{currentstroke}{rgb}{0.121569,0.466667,0.705882}%
\pgfsetstrokecolor{currentstroke}%
\pgfsetstrokeopacity{0.774386}%
\pgfsetdash{}{0pt}%
\pgfpathmoveto{\pgfqpoint{0.774633in}{2.175788in}}%
\pgfpathcurveto{\pgfqpoint{0.782869in}{2.175788in}}{\pgfqpoint{0.790769in}{2.179060in}}{\pgfqpoint{0.796593in}{2.184884in}}%
\pgfpathcurveto{\pgfqpoint{0.802417in}{2.190708in}}{\pgfqpoint{0.805689in}{2.198608in}}{\pgfqpoint{0.805689in}{2.206845in}}%
\pgfpathcurveto{\pgfqpoint{0.805689in}{2.215081in}}{\pgfqpoint{0.802417in}{2.222981in}}{\pgfqpoint{0.796593in}{2.228805in}}%
\pgfpathcurveto{\pgfqpoint{0.790769in}{2.234629in}}{\pgfqpoint{0.782869in}{2.237901in}}{\pgfqpoint{0.774633in}{2.237901in}}%
\pgfpathcurveto{\pgfqpoint{0.766396in}{2.237901in}}{\pgfqpoint{0.758496in}{2.234629in}}{\pgfqpoint{0.752672in}{2.228805in}}%
\pgfpathcurveto{\pgfqpoint{0.746849in}{2.222981in}}{\pgfqpoint{0.743576in}{2.215081in}}{\pgfqpoint{0.743576in}{2.206845in}}%
\pgfpathcurveto{\pgfqpoint{0.743576in}{2.198608in}}{\pgfqpoint{0.746849in}{2.190708in}}{\pgfqpoint{0.752672in}{2.184884in}}%
\pgfpathcurveto{\pgfqpoint{0.758496in}{2.179060in}}{\pgfqpoint{0.766396in}{2.175788in}}{\pgfqpoint{0.774633in}{2.175788in}}%
\pgfpathclose%
\pgfusepath{stroke,fill}%
\end{pgfscope}%
\begin{pgfscope}%
\pgfpathrectangle{\pgfqpoint{0.100000in}{0.212622in}}{\pgfqpoint{3.696000in}{3.696000in}}%
\pgfusepath{clip}%
\pgfsetbuttcap%
\pgfsetroundjoin%
\definecolor{currentfill}{rgb}{0.121569,0.466667,0.705882}%
\pgfsetfillcolor{currentfill}%
\pgfsetfillopacity{0.774387}%
\pgfsetlinewidth{1.003750pt}%
\definecolor{currentstroke}{rgb}{0.121569,0.466667,0.705882}%
\pgfsetstrokecolor{currentstroke}%
\pgfsetstrokeopacity{0.774387}%
\pgfsetdash{}{0pt}%
\pgfpathmoveto{\pgfqpoint{0.774630in}{2.175788in}}%
\pgfpathcurveto{\pgfqpoint{0.782866in}{2.175788in}}{\pgfqpoint{0.790766in}{2.179061in}}{\pgfqpoint{0.796590in}{2.184884in}}%
\pgfpathcurveto{\pgfqpoint{0.802414in}{2.190708in}}{\pgfqpoint{0.805686in}{2.198608in}}{\pgfqpoint{0.805686in}{2.206845in}}%
\pgfpathcurveto{\pgfqpoint{0.805686in}{2.215081in}}{\pgfqpoint{0.802414in}{2.222981in}}{\pgfqpoint{0.796590in}{2.228805in}}%
\pgfpathcurveto{\pgfqpoint{0.790766in}{2.234629in}}{\pgfqpoint{0.782866in}{2.237901in}}{\pgfqpoint{0.774630in}{2.237901in}}%
\pgfpathcurveto{\pgfqpoint{0.766394in}{2.237901in}}{\pgfqpoint{0.758494in}{2.234629in}}{\pgfqpoint{0.752670in}{2.228805in}}%
\pgfpathcurveto{\pgfqpoint{0.746846in}{2.222981in}}{\pgfqpoint{0.743573in}{2.215081in}}{\pgfqpoint{0.743573in}{2.206845in}}%
\pgfpathcurveto{\pgfqpoint{0.743573in}{2.198608in}}{\pgfqpoint{0.746846in}{2.190708in}}{\pgfqpoint{0.752670in}{2.184884in}}%
\pgfpathcurveto{\pgfqpoint{0.758494in}{2.179061in}}{\pgfqpoint{0.766394in}{2.175788in}}{\pgfqpoint{0.774630in}{2.175788in}}%
\pgfpathclose%
\pgfusepath{stroke,fill}%
\end{pgfscope}%
\begin{pgfscope}%
\pgfpathrectangle{\pgfqpoint{0.100000in}{0.212622in}}{\pgfqpoint{3.696000in}{3.696000in}}%
\pgfusepath{clip}%
\pgfsetbuttcap%
\pgfsetroundjoin%
\definecolor{currentfill}{rgb}{0.121569,0.466667,0.705882}%
\pgfsetfillcolor{currentfill}%
\pgfsetfillopacity{0.774390}%
\pgfsetlinewidth{1.003750pt}%
\definecolor{currentstroke}{rgb}{0.121569,0.466667,0.705882}%
\pgfsetstrokecolor{currentstroke}%
\pgfsetstrokeopacity{0.774390}%
\pgfsetdash{}{0pt}%
\pgfpathmoveto{\pgfqpoint{0.774625in}{2.175788in}}%
\pgfpathcurveto{\pgfqpoint{0.782861in}{2.175788in}}{\pgfqpoint{0.790761in}{2.179061in}}{\pgfqpoint{0.796585in}{2.184885in}}%
\pgfpathcurveto{\pgfqpoint{0.802409in}{2.190709in}}{\pgfqpoint{0.805681in}{2.198609in}}{\pgfqpoint{0.805681in}{2.206845in}}%
\pgfpathcurveto{\pgfqpoint{0.805681in}{2.215081in}}{\pgfqpoint{0.802409in}{2.222981in}}{\pgfqpoint{0.796585in}{2.228805in}}%
\pgfpathcurveto{\pgfqpoint{0.790761in}{2.234629in}}{\pgfqpoint{0.782861in}{2.237901in}}{\pgfqpoint{0.774625in}{2.237901in}}%
\pgfpathcurveto{\pgfqpoint{0.766389in}{2.237901in}}{\pgfqpoint{0.758488in}{2.234629in}}{\pgfqpoint{0.752665in}{2.228805in}}%
\pgfpathcurveto{\pgfqpoint{0.746841in}{2.222981in}}{\pgfqpoint{0.743568in}{2.215081in}}{\pgfqpoint{0.743568in}{2.206845in}}%
\pgfpathcurveto{\pgfqpoint{0.743568in}{2.198609in}}{\pgfqpoint{0.746841in}{2.190709in}}{\pgfqpoint{0.752665in}{2.184885in}}%
\pgfpathcurveto{\pgfqpoint{0.758488in}{2.179061in}}{\pgfqpoint{0.766389in}{2.175788in}}{\pgfqpoint{0.774625in}{2.175788in}}%
\pgfpathclose%
\pgfusepath{stroke,fill}%
\end{pgfscope}%
\begin{pgfscope}%
\pgfpathrectangle{\pgfqpoint{0.100000in}{0.212622in}}{\pgfqpoint{3.696000in}{3.696000in}}%
\pgfusepath{clip}%
\pgfsetbuttcap%
\pgfsetroundjoin%
\definecolor{currentfill}{rgb}{0.121569,0.466667,0.705882}%
\pgfsetfillcolor{currentfill}%
\pgfsetfillopacity{0.774395}%
\pgfsetlinewidth{1.003750pt}%
\definecolor{currentstroke}{rgb}{0.121569,0.466667,0.705882}%
\pgfsetstrokecolor{currentstroke}%
\pgfsetstrokeopacity{0.774395}%
\pgfsetdash{}{0pt}%
\pgfpathmoveto{\pgfqpoint{0.774615in}{2.175789in}}%
\pgfpathcurveto{\pgfqpoint{0.782851in}{2.175789in}}{\pgfqpoint{0.790751in}{2.179061in}}{\pgfqpoint{0.796575in}{2.184885in}}%
\pgfpathcurveto{\pgfqpoint{0.802399in}{2.190709in}}{\pgfqpoint{0.805671in}{2.198609in}}{\pgfqpoint{0.805671in}{2.206845in}}%
\pgfpathcurveto{\pgfqpoint{0.805671in}{2.215082in}}{\pgfqpoint{0.802399in}{2.222982in}}{\pgfqpoint{0.796575in}{2.228806in}}%
\pgfpathcurveto{\pgfqpoint{0.790751in}{2.234630in}}{\pgfqpoint{0.782851in}{2.237902in}}{\pgfqpoint{0.774615in}{2.237902in}}%
\pgfpathcurveto{\pgfqpoint{0.766378in}{2.237902in}}{\pgfqpoint{0.758478in}{2.234630in}}{\pgfqpoint{0.752654in}{2.228806in}}%
\pgfpathcurveto{\pgfqpoint{0.746830in}{2.222982in}}{\pgfqpoint{0.743558in}{2.215082in}}{\pgfqpoint{0.743558in}{2.206845in}}%
\pgfpathcurveto{\pgfqpoint{0.743558in}{2.198609in}}{\pgfqpoint{0.746830in}{2.190709in}}{\pgfqpoint{0.752654in}{2.184885in}}%
\pgfpathcurveto{\pgfqpoint{0.758478in}{2.179061in}}{\pgfqpoint{0.766378in}{2.175789in}}{\pgfqpoint{0.774615in}{2.175789in}}%
\pgfpathclose%
\pgfusepath{stroke,fill}%
\end{pgfscope}%
\begin{pgfscope}%
\pgfpathrectangle{\pgfqpoint{0.100000in}{0.212622in}}{\pgfqpoint{3.696000in}{3.696000in}}%
\pgfusepath{clip}%
\pgfsetbuttcap%
\pgfsetroundjoin%
\definecolor{currentfill}{rgb}{0.121569,0.466667,0.705882}%
\pgfsetfillcolor{currentfill}%
\pgfsetfillopacity{0.774405}%
\pgfsetlinewidth{1.003750pt}%
\definecolor{currentstroke}{rgb}{0.121569,0.466667,0.705882}%
\pgfsetstrokecolor{currentstroke}%
\pgfsetstrokeopacity{0.774405}%
\pgfsetdash{}{0pt}%
\pgfpathmoveto{\pgfqpoint{0.774597in}{2.175789in}}%
\pgfpathcurveto{\pgfqpoint{0.782833in}{2.175789in}}{\pgfqpoint{0.790734in}{2.179061in}}{\pgfqpoint{0.796557in}{2.184885in}}%
\pgfpathcurveto{\pgfqpoint{0.802381in}{2.190709in}}{\pgfqpoint{0.805654in}{2.198609in}}{\pgfqpoint{0.805654in}{2.206846in}}%
\pgfpathcurveto{\pgfqpoint{0.805654in}{2.215082in}}{\pgfqpoint{0.802381in}{2.222982in}}{\pgfqpoint{0.796557in}{2.228806in}}%
\pgfpathcurveto{\pgfqpoint{0.790734in}{2.234630in}}{\pgfqpoint{0.782833in}{2.237902in}}{\pgfqpoint{0.774597in}{2.237902in}}%
\pgfpathcurveto{\pgfqpoint{0.766361in}{2.237902in}}{\pgfqpoint{0.758461in}{2.234630in}}{\pgfqpoint{0.752637in}{2.228806in}}%
\pgfpathcurveto{\pgfqpoint{0.746813in}{2.222982in}}{\pgfqpoint{0.743541in}{2.215082in}}{\pgfqpoint{0.743541in}{2.206846in}}%
\pgfpathcurveto{\pgfqpoint{0.743541in}{2.198609in}}{\pgfqpoint{0.746813in}{2.190709in}}{\pgfqpoint{0.752637in}{2.184885in}}%
\pgfpathcurveto{\pgfqpoint{0.758461in}{2.179061in}}{\pgfqpoint{0.766361in}{2.175789in}}{\pgfqpoint{0.774597in}{2.175789in}}%
\pgfpathclose%
\pgfusepath{stroke,fill}%
\end{pgfscope}%
\begin{pgfscope}%
\pgfpathrectangle{\pgfqpoint{0.100000in}{0.212622in}}{\pgfqpoint{3.696000in}{3.696000in}}%
\pgfusepath{clip}%
\pgfsetbuttcap%
\pgfsetroundjoin%
\definecolor{currentfill}{rgb}{0.121569,0.466667,0.705882}%
\pgfsetfillcolor{currentfill}%
\pgfsetfillopacity{0.774423}%
\pgfsetlinewidth{1.003750pt}%
\definecolor{currentstroke}{rgb}{0.121569,0.466667,0.705882}%
\pgfsetstrokecolor{currentstroke}%
\pgfsetstrokeopacity{0.774423}%
\pgfsetdash{}{0pt}%
\pgfpathmoveto{\pgfqpoint{0.774565in}{2.175791in}}%
\pgfpathcurveto{\pgfqpoint{0.782802in}{2.175791in}}{\pgfqpoint{0.790702in}{2.179064in}}{\pgfqpoint{0.796526in}{2.184887in}}%
\pgfpathcurveto{\pgfqpoint{0.802350in}{2.190711in}}{\pgfqpoint{0.805622in}{2.198611in}}{\pgfqpoint{0.805622in}{2.206848in}}%
\pgfpathcurveto{\pgfqpoint{0.805622in}{2.215084in}}{\pgfqpoint{0.802350in}{2.222984in}}{\pgfqpoint{0.796526in}{2.228808in}}%
\pgfpathcurveto{\pgfqpoint{0.790702in}{2.234632in}}{\pgfqpoint{0.782802in}{2.237904in}}{\pgfqpoint{0.774565in}{2.237904in}}%
\pgfpathcurveto{\pgfqpoint{0.766329in}{2.237904in}}{\pgfqpoint{0.758429in}{2.234632in}}{\pgfqpoint{0.752605in}{2.228808in}}%
\pgfpathcurveto{\pgfqpoint{0.746781in}{2.222984in}}{\pgfqpoint{0.743509in}{2.215084in}}{\pgfqpoint{0.743509in}{2.206848in}}%
\pgfpathcurveto{\pgfqpoint{0.743509in}{2.198611in}}{\pgfqpoint{0.746781in}{2.190711in}}{\pgfqpoint{0.752605in}{2.184887in}}%
\pgfpathcurveto{\pgfqpoint{0.758429in}{2.179064in}}{\pgfqpoint{0.766329in}{2.175791in}}{\pgfqpoint{0.774565in}{2.175791in}}%
\pgfpathclose%
\pgfusepath{stroke,fill}%
\end{pgfscope}%
\begin{pgfscope}%
\pgfpathrectangle{\pgfqpoint{0.100000in}{0.212622in}}{\pgfqpoint{3.696000in}{3.696000in}}%
\pgfusepath{clip}%
\pgfsetbuttcap%
\pgfsetroundjoin%
\definecolor{currentfill}{rgb}{0.121569,0.466667,0.705882}%
\pgfsetfillcolor{currentfill}%
\pgfsetfillopacity{0.774455}%
\pgfsetlinewidth{1.003750pt}%
\definecolor{currentstroke}{rgb}{0.121569,0.466667,0.705882}%
\pgfsetstrokecolor{currentstroke}%
\pgfsetstrokeopacity{0.774455}%
\pgfsetdash{}{0pt}%
\pgfpathmoveto{\pgfqpoint{0.774514in}{2.175791in}}%
\pgfpathcurveto{\pgfqpoint{0.782750in}{2.175791in}}{\pgfqpoint{0.790650in}{2.179063in}}{\pgfqpoint{0.796474in}{2.184887in}}%
\pgfpathcurveto{\pgfqpoint{0.802298in}{2.190711in}}{\pgfqpoint{0.805570in}{2.198611in}}{\pgfqpoint{0.805570in}{2.206847in}}%
\pgfpathcurveto{\pgfqpoint{0.805570in}{2.215084in}}{\pgfqpoint{0.802298in}{2.222984in}}{\pgfqpoint{0.796474in}{2.228808in}}%
\pgfpathcurveto{\pgfqpoint{0.790650in}{2.234632in}}{\pgfqpoint{0.782750in}{2.237904in}}{\pgfqpoint{0.774514in}{2.237904in}}%
\pgfpathcurveto{\pgfqpoint{0.766278in}{2.237904in}}{\pgfqpoint{0.758378in}{2.234632in}}{\pgfqpoint{0.752554in}{2.228808in}}%
\pgfpathcurveto{\pgfqpoint{0.746730in}{2.222984in}}{\pgfqpoint{0.743457in}{2.215084in}}{\pgfqpoint{0.743457in}{2.206847in}}%
\pgfpathcurveto{\pgfqpoint{0.743457in}{2.198611in}}{\pgfqpoint{0.746730in}{2.190711in}}{\pgfqpoint{0.752554in}{2.184887in}}%
\pgfpathcurveto{\pgfqpoint{0.758378in}{2.179063in}}{\pgfqpoint{0.766278in}{2.175791in}}{\pgfqpoint{0.774514in}{2.175791in}}%
\pgfpathclose%
\pgfusepath{stroke,fill}%
\end{pgfscope}%
\begin{pgfscope}%
\pgfpathrectangle{\pgfqpoint{0.100000in}{0.212622in}}{\pgfqpoint{3.696000in}{3.696000in}}%
\pgfusepath{clip}%
\pgfsetbuttcap%
\pgfsetroundjoin%
\definecolor{currentfill}{rgb}{0.121569,0.466667,0.705882}%
\pgfsetfillcolor{currentfill}%
\pgfsetfillopacity{0.774513}%
\pgfsetlinewidth{1.003750pt}%
\definecolor{currentstroke}{rgb}{0.121569,0.466667,0.705882}%
\pgfsetstrokecolor{currentstroke}%
\pgfsetstrokeopacity{0.774513}%
\pgfsetdash{}{0pt}%
\pgfpathmoveto{\pgfqpoint{0.774405in}{2.175795in}}%
\pgfpathcurveto{\pgfqpoint{0.782641in}{2.175795in}}{\pgfqpoint{0.790541in}{2.179067in}}{\pgfqpoint{0.796365in}{2.184891in}}%
\pgfpathcurveto{\pgfqpoint{0.802189in}{2.190715in}}{\pgfqpoint{0.805461in}{2.198615in}}{\pgfqpoint{0.805461in}{2.206852in}}%
\pgfpathcurveto{\pgfqpoint{0.805461in}{2.215088in}}{\pgfqpoint{0.802189in}{2.222988in}}{\pgfqpoint{0.796365in}{2.228812in}}%
\pgfpathcurveto{\pgfqpoint{0.790541in}{2.234636in}}{\pgfqpoint{0.782641in}{2.237908in}}{\pgfqpoint{0.774405in}{2.237908in}}%
\pgfpathcurveto{\pgfqpoint{0.766168in}{2.237908in}}{\pgfqpoint{0.758268in}{2.234636in}}{\pgfqpoint{0.752444in}{2.228812in}}%
\pgfpathcurveto{\pgfqpoint{0.746620in}{2.222988in}}{\pgfqpoint{0.743348in}{2.215088in}}{\pgfqpoint{0.743348in}{2.206852in}}%
\pgfpathcurveto{\pgfqpoint{0.743348in}{2.198615in}}{\pgfqpoint{0.746620in}{2.190715in}}{\pgfqpoint{0.752444in}{2.184891in}}%
\pgfpathcurveto{\pgfqpoint{0.758268in}{2.179067in}}{\pgfqpoint{0.766168in}{2.175795in}}{\pgfqpoint{0.774405in}{2.175795in}}%
\pgfpathclose%
\pgfusepath{stroke,fill}%
\end{pgfscope}%
\begin{pgfscope}%
\pgfpathrectangle{\pgfqpoint{0.100000in}{0.212622in}}{\pgfqpoint{3.696000in}{3.696000in}}%
\pgfusepath{clip}%
\pgfsetbuttcap%
\pgfsetroundjoin%
\definecolor{currentfill}{rgb}{0.121569,0.466667,0.705882}%
\pgfsetfillcolor{currentfill}%
\pgfsetfillopacity{0.774618}%
\pgfsetlinewidth{1.003750pt}%
\definecolor{currentstroke}{rgb}{0.121569,0.466667,0.705882}%
\pgfsetstrokecolor{currentstroke}%
\pgfsetstrokeopacity{0.774618}%
\pgfsetdash{}{0pt}%
\pgfpathmoveto{\pgfqpoint{0.774216in}{2.175801in}}%
\pgfpathcurveto{\pgfqpoint{0.782452in}{2.175801in}}{\pgfqpoint{0.790352in}{2.179073in}}{\pgfqpoint{0.796176in}{2.184897in}}%
\pgfpathcurveto{\pgfqpoint{0.802000in}{2.190721in}}{\pgfqpoint{0.805272in}{2.198621in}}{\pgfqpoint{0.805272in}{2.206857in}}%
\pgfpathcurveto{\pgfqpoint{0.805272in}{2.215094in}}{\pgfqpoint{0.802000in}{2.222994in}}{\pgfqpoint{0.796176in}{2.228818in}}%
\pgfpathcurveto{\pgfqpoint{0.790352in}{2.234642in}}{\pgfqpoint{0.782452in}{2.237914in}}{\pgfqpoint{0.774216in}{2.237914in}}%
\pgfpathcurveto{\pgfqpoint{0.765980in}{2.237914in}}{\pgfqpoint{0.758079in}{2.234642in}}{\pgfqpoint{0.752256in}{2.228818in}}%
\pgfpathcurveto{\pgfqpoint{0.746432in}{2.222994in}}{\pgfqpoint{0.743159in}{2.215094in}}{\pgfqpoint{0.743159in}{2.206857in}}%
\pgfpathcurveto{\pgfqpoint{0.743159in}{2.198621in}}{\pgfqpoint{0.746432in}{2.190721in}}{\pgfqpoint{0.752256in}{2.184897in}}%
\pgfpathcurveto{\pgfqpoint{0.758079in}{2.179073in}}{\pgfqpoint{0.765980in}{2.175801in}}{\pgfqpoint{0.774216in}{2.175801in}}%
\pgfpathclose%
\pgfusepath{stroke,fill}%
\end{pgfscope}%
\begin{pgfscope}%
\pgfpathrectangle{\pgfqpoint{0.100000in}{0.212622in}}{\pgfqpoint{3.696000in}{3.696000in}}%
\pgfusepath{clip}%
\pgfsetbuttcap%
\pgfsetroundjoin%
\definecolor{currentfill}{rgb}{0.121569,0.466667,0.705882}%
\pgfsetfillcolor{currentfill}%
\pgfsetfillopacity{0.774809}%
\pgfsetlinewidth{1.003750pt}%
\definecolor{currentstroke}{rgb}{0.121569,0.466667,0.705882}%
\pgfsetstrokecolor{currentstroke}%
\pgfsetstrokeopacity{0.774809}%
\pgfsetdash{}{0pt}%
\pgfpathmoveto{\pgfqpoint{0.773843in}{2.175822in}}%
\pgfpathcurveto{\pgfqpoint{0.782080in}{2.175822in}}{\pgfqpoint{0.789980in}{2.179094in}}{\pgfqpoint{0.795804in}{2.184918in}}%
\pgfpathcurveto{\pgfqpoint{0.801628in}{2.190742in}}{\pgfqpoint{0.804900in}{2.198642in}}{\pgfqpoint{0.804900in}{2.206878in}}%
\pgfpathcurveto{\pgfqpoint{0.804900in}{2.215115in}}{\pgfqpoint{0.801628in}{2.223015in}}{\pgfqpoint{0.795804in}{2.228839in}}%
\pgfpathcurveto{\pgfqpoint{0.789980in}{2.234663in}}{\pgfqpoint{0.782080in}{2.237935in}}{\pgfqpoint{0.773843in}{2.237935in}}%
\pgfpathcurveto{\pgfqpoint{0.765607in}{2.237935in}}{\pgfqpoint{0.757707in}{2.234663in}}{\pgfqpoint{0.751883in}{2.228839in}}%
\pgfpathcurveto{\pgfqpoint{0.746059in}{2.223015in}}{\pgfqpoint{0.742787in}{2.215115in}}{\pgfqpoint{0.742787in}{2.206878in}}%
\pgfpathcurveto{\pgfqpoint{0.742787in}{2.198642in}}{\pgfqpoint{0.746059in}{2.190742in}}{\pgfqpoint{0.751883in}{2.184918in}}%
\pgfpathcurveto{\pgfqpoint{0.757707in}{2.179094in}}{\pgfqpoint{0.765607in}{2.175822in}}{\pgfqpoint{0.773843in}{2.175822in}}%
\pgfpathclose%
\pgfusepath{stroke,fill}%
\end{pgfscope}%
\begin{pgfscope}%
\pgfpathrectangle{\pgfqpoint{0.100000in}{0.212622in}}{\pgfqpoint{3.696000in}{3.696000in}}%
\pgfusepath{clip}%
\pgfsetbuttcap%
\pgfsetroundjoin%
\definecolor{currentfill}{rgb}{0.121569,0.466667,0.705882}%
\pgfsetfillcolor{currentfill}%
\pgfsetfillopacity{0.774892}%
\pgfsetlinewidth{1.003750pt}%
\definecolor{currentstroke}{rgb}{0.121569,0.466667,0.705882}%
\pgfsetstrokecolor{currentstroke}%
\pgfsetstrokeopacity{0.774892}%
\pgfsetdash{}{0pt}%
\pgfpathmoveto{\pgfqpoint{0.773669in}{2.175836in}}%
\pgfpathcurveto{\pgfqpoint{0.781906in}{2.175836in}}{\pgfqpoint{0.789806in}{2.179109in}}{\pgfqpoint{0.795630in}{2.184933in}}%
\pgfpathcurveto{\pgfqpoint{0.801454in}{2.190757in}}{\pgfqpoint{0.804726in}{2.198657in}}{\pgfqpoint{0.804726in}{2.206893in}}%
\pgfpathcurveto{\pgfqpoint{0.804726in}{2.215129in}}{\pgfqpoint{0.801454in}{2.223029in}}{\pgfqpoint{0.795630in}{2.228853in}}%
\pgfpathcurveto{\pgfqpoint{0.789806in}{2.234677in}}{\pgfqpoint{0.781906in}{2.237949in}}{\pgfqpoint{0.773669in}{2.237949in}}%
\pgfpathcurveto{\pgfqpoint{0.765433in}{2.237949in}}{\pgfqpoint{0.757533in}{2.234677in}}{\pgfqpoint{0.751709in}{2.228853in}}%
\pgfpathcurveto{\pgfqpoint{0.745885in}{2.223029in}}{\pgfqpoint{0.742613in}{2.215129in}}{\pgfqpoint{0.742613in}{2.206893in}}%
\pgfpathcurveto{\pgfqpoint{0.742613in}{2.198657in}}{\pgfqpoint{0.745885in}{2.190757in}}{\pgfqpoint{0.751709in}{2.184933in}}%
\pgfpathcurveto{\pgfqpoint{0.757533in}{2.179109in}}{\pgfqpoint{0.765433in}{2.175836in}}{\pgfqpoint{0.773669in}{2.175836in}}%
\pgfpathclose%
\pgfusepath{stroke,fill}%
\end{pgfscope}%
\begin{pgfscope}%
\pgfpathrectangle{\pgfqpoint{0.100000in}{0.212622in}}{\pgfqpoint{3.696000in}{3.696000in}}%
\pgfusepath{clip}%
\pgfsetbuttcap%
\pgfsetroundjoin%
\definecolor{currentfill}{rgb}{0.121569,0.466667,0.705882}%
\pgfsetfillcolor{currentfill}%
\pgfsetfillopacity{0.775044}%
\pgfsetlinewidth{1.003750pt}%
\definecolor{currentstroke}{rgb}{0.121569,0.466667,0.705882}%
\pgfsetstrokecolor{currentstroke}%
\pgfsetstrokeopacity{0.775044}%
\pgfsetdash{}{0pt}%
\pgfpathmoveto{\pgfqpoint{0.773372in}{2.175851in}}%
\pgfpathcurveto{\pgfqpoint{0.781608in}{2.175851in}}{\pgfqpoint{0.789508in}{2.179124in}}{\pgfqpoint{0.795332in}{2.184948in}}%
\pgfpathcurveto{\pgfqpoint{0.801156in}{2.190772in}}{\pgfqpoint{0.804429in}{2.198672in}}{\pgfqpoint{0.804429in}{2.206908in}}%
\pgfpathcurveto{\pgfqpoint{0.804429in}{2.215144in}}{\pgfqpoint{0.801156in}{2.223044in}}{\pgfqpoint{0.795332in}{2.228868in}}%
\pgfpathcurveto{\pgfqpoint{0.789508in}{2.234692in}}{\pgfqpoint{0.781608in}{2.237964in}}{\pgfqpoint{0.773372in}{2.237964in}}%
\pgfpathcurveto{\pgfqpoint{0.765136in}{2.237964in}}{\pgfqpoint{0.757236in}{2.234692in}}{\pgfqpoint{0.751412in}{2.228868in}}%
\pgfpathcurveto{\pgfqpoint{0.745588in}{2.223044in}}{\pgfqpoint{0.742316in}{2.215144in}}{\pgfqpoint{0.742316in}{2.206908in}}%
\pgfpathcurveto{\pgfqpoint{0.742316in}{2.198672in}}{\pgfqpoint{0.745588in}{2.190772in}}{\pgfqpoint{0.751412in}{2.184948in}}%
\pgfpathcurveto{\pgfqpoint{0.757236in}{2.179124in}}{\pgfqpoint{0.765136in}{2.175851in}}{\pgfqpoint{0.773372in}{2.175851in}}%
\pgfpathclose%
\pgfusepath{stroke,fill}%
\end{pgfscope}%
\begin{pgfscope}%
\pgfpathrectangle{\pgfqpoint{0.100000in}{0.212622in}}{\pgfqpoint{3.696000in}{3.696000in}}%
\pgfusepath{clip}%
\pgfsetbuttcap%
\pgfsetroundjoin%
\definecolor{currentfill}{rgb}{0.121569,0.466667,0.705882}%
\pgfsetfillcolor{currentfill}%
\pgfsetfillopacity{0.775122}%
\pgfsetlinewidth{1.003750pt}%
\definecolor{currentstroke}{rgb}{0.121569,0.466667,0.705882}%
\pgfsetstrokecolor{currentstroke}%
\pgfsetstrokeopacity{0.775122}%
\pgfsetdash{}{0pt}%
\pgfpathmoveto{\pgfqpoint{2.874356in}{1.820381in}}%
\pgfpathcurveto{\pgfqpoint{2.882592in}{1.820381in}}{\pgfqpoint{2.890492in}{1.823653in}}{\pgfqpoint{2.896316in}{1.829477in}}%
\pgfpathcurveto{\pgfqpoint{2.902140in}{1.835301in}}{\pgfqpoint{2.905412in}{1.843201in}}{\pgfqpoint{2.905412in}{1.851437in}}%
\pgfpathcurveto{\pgfqpoint{2.905412in}{1.859674in}}{\pgfqpoint{2.902140in}{1.867574in}}{\pgfqpoint{2.896316in}{1.873398in}}%
\pgfpathcurveto{\pgfqpoint{2.890492in}{1.879221in}}{\pgfqpoint{2.882592in}{1.882494in}}{\pgfqpoint{2.874356in}{1.882494in}}%
\pgfpathcurveto{\pgfqpoint{2.866119in}{1.882494in}}{\pgfqpoint{2.858219in}{1.879221in}}{\pgfqpoint{2.852395in}{1.873398in}}%
\pgfpathcurveto{\pgfqpoint{2.846571in}{1.867574in}}{\pgfqpoint{2.843299in}{1.859674in}}{\pgfqpoint{2.843299in}{1.851437in}}%
\pgfpathcurveto{\pgfqpoint{2.843299in}{1.843201in}}{\pgfqpoint{2.846571in}{1.835301in}}{\pgfqpoint{2.852395in}{1.829477in}}%
\pgfpathcurveto{\pgfqpoint{2.858219in}{1.823653in}}{\pgfqpoint{2.866119in}{1.820381in}}{\pgfqpoint{2.874356in}{1.820381in}}%
\pgfpathclose%
\pgfusepath{stroke,fill}%
\end{pgfscope}%
\begin{pgfscope}%
\pgfpathrectangle{\pgfqpoint{0.100000in}{0.212622in}}{\pgfqpoint{3.696000in}{3.696000in}}%
\pgfusepath{clip}%
\pgfsetbuttcap%
\pgfsetroundjoin%
\definecolor{currentfill}{rgb}{0.121569,0.466667,0.705882}%
\pgfsetfillcolor{currentfill}%
\pgfsetfillopacity{0.775324}%
\pgfsetlinewidth{1.003750pt}%
\definecolor{currentstroke}{rgb}{0.121569,0.466667,0.705882}%
\pgfsetstrokecolor{currentstroke}%
\pgfsetstrokeopacity{0.775324}%
\pgfsetdash{}{0pt}%
\pgfpathmoveto{\pgfqpoint{0.772843in}{2.175884in}}%
\pgfpathcurveto{\pgfqpoint{0.781079in}{2.175884in}}{\pgfqpoint{0.788979in}{2.179157in}}{\pgfqpoint{0.794803in}{2.184981in}}%
\pgfpathcurveto{\pgfqpoint{0.800627in}{2.190805in}}{\pgfqpoint{0.803899in}{2.198705in}}{\pgfqpoint{0.803899in}{2.206941in}}%
\pgfpathcurveto{\pgfqpoint{0.803899in}{2.215177in}}{\pgfqpoint{0.800627in}{2.223077in}}{\pgfqpoint{0.794803in}{2.228901in}}%
\pgfpathcurveto{\pgfqpoint{0.788979in}{2.234725in}}{\pgfqpoint{0.781079in}{2.237997in}}{\pgfqpoint{0.772843in}{2.237997in}}%
\pgfpathcurveto{\pgfqpoint{0.764607in}{2.237997in}}{\pgfqpoint{0.756707in}{2.234725in}}{\pgfqpoint{0.750883in}{2.228901in}}%
\pgfpathcurveto{\pgfqpoint{0.745059in}{2.223077in}}{\pgfqpoint{0.741786in}{2.215177in}}{\pgfqpoint{0.741786in}{2.206941in}}%
\pgfpathcurveto{\pgfqpoint{0.741786in}{2.198705in}}{\pgfqpoint{0.745059in}{2.190805in}}{\pgfqpoint{0.750883in}{2.184981in}}%
\pgfpathcurveto{\pgfqpoint{0.756707in}{2.179157in}}{\pgfqpoint{0.764607in}{2.175884in}}{\pgfqpoint{0.772843in}{2.175884in}}%
\pgfpathclose%
\pgfusepath{stroke,fill}%
\end{pgfscope}%
\begin{pgfscope}%
\pgfpathrectangle{\pgfqpoint{0.100000in}{0.212622in}}{\pgfqpoint{3.696000in}{3.696000in}}%
\pgfusepath{clip}%
\pgfsetbuttcap%
\pgfsetroundjoin%
\definecolor{currentfill}{rgb}{0.121569,0.466667,0.705882}%
\pgfsetfillcolor{currentfill}%
\pgfsetfillopacity{0.775828}%
\pgfsetlinewidth{1.003750pt}%
\definecolor{currentstroke}{rgb}{0.121569,0.466667,0.705882}%
\pgfsetstrokecolor{currentstroke}%
\pgfsetstrokeopacity{0.775828}%
\pgfsetdash{}{0pt}%
\pgfpathmoveto{\pgfqpoint{0.771843in}{2.175946in}}%
\pgfpathcurveto{\pgfqpoint{0.780079in}{2.175946in}}{\pgfqpoint{0.787979in}{2.179218in}}{\pgfqpoint{0.793803in}{2.185042in}}%
\pgfpathcurveto{\pgfqpoint{0.799627in}{2.190866in}}{\pgfqpoint{0.802899in}{2.198766in}}{\pgfqpoint{0.802899in}{2.207002in}}%
\pgfpathcurveto{\pgfqpoint{0.802899in}{2.215239in}}{\pgfqpoint{0.799627in}{2.223139in}}{\pgfqpoint{0.793803in}{2.228963in}}%
\pgfpathcurveto{\pgfqpoint{0.787979in}{2.234786in}}{\pgfqpoint{0.780079in}{2.238059in}}{\pgfqpoint{0.771843in}{2.238059in}}%
\pgfpathcurveto{\pgfqpoint{0.763606in}{2.238059in}}{\pgfqpoint{0.755706in}{2.234786in}}{\pgfqpoint{0.749882in}{2.228963in}}%
\pgfpathcurveto{\pgfqpoint{0.744059in}{2.223139in}}{\pgfqpoint{0.740786in}{2.215239in}}{\pgfqpoint{0.740786in}{2.207002in}}%
\pgfpathcurveto{\pgfqpoint{0.740786in}{2.198766in}}{\pgfqpoint{0.744059in}{2.190866in}}{\pgfqpoint{0.749882in}{2.185042in}}%
\pgfpathcurveto{\pgfqpoint{0.755706in}{2.179218in}}{\pgfqpoint{0.763606in}{2.175946in}}{\pgfqpoint{0.771843in}{2.175946in}}%
\pgfpathclose%
\pgfusepath{stroke,fill}%
\end{pgfscope}%
\begin{pgfscope}%
\pgfpathrectangle{\pgfqpoint{0.100000in}{0.212622in}}{\pgfqpoint{3.696000in}{3.696000in}}%
\pgfusepath{clip}%
\pgfsetbuttcap%
\pgfsetroundjoin%
\definecolor{currentfill}{rgb}{0.121569,0.466667,0.705882}%
\pgfsetfillcolor{currentfill}%
\pgfsetfillopacity{0.776247}%
\pgfsetlinewidth{1.003750pt}%
\definecolor{currentstroke}{rgb}{0.121569,0.466667,0.705882}%
\pgfsetstrokecolor{currentstroke}%
\pgfsetstrokeopacity{0.776247}%
\pgfsetdash{}{0pt}%
\pgfpathmoveto{\pgfqpoint{0.771078in}{2.175980in}}%
\pgfpathcurveto{\pgfqpoint{0.779315in}{2.175980in}}{\pgfqpoint{0.787215in}{2.179252in}}{\pgfqpoint{0.793039in}{2.185076in}}%
\pgfpathcurveto{\pgfqpoint{0.798863in}{2.190900in}}{\pgfqpoint{0.802135in}{2.198800in}}{\pgfqpoint{0.802135in}{2.207036in}}%
\pgfpathcurveto{\pgfqpoint{0.802135in}{2.215272in}}{\pgfqpoint{0.798863in}{2.223172in}}{\pgfqpoint{0.793039in}{2.228996in}}%
\pgfpathcurveto{\pgfqpoint{0.787215in}{2.234820in}}{\pgfqpoint{0.779315in}{2.238093in}}{\pgfqpoint{0.771078in}{2.238093in}}%
\pgfpathcurveto{\pgfqpoint{0.762842in}{2.238093in}}{\pgfqpoint{0.754942in}{2.234820in}}{\pgfqpoint{0.749118in}{2.228996in}}%
\pgfpathcurveto{\pgfqpoint{0.743294in}{2.223172in}}{\pgfqpoint{0.740022in}{2.215272in}}{\pgfqpoint{0.740022in}{2.207036in}}%
\pgfpathcurveto{\pgfqpoint{0.740022in}{2.198800in}}{\pgfqpoint{0.743294in}{2.190900in}}{\pgfqpoint{0.749118in}{2.185076in}}%
\pgfpathcurveto{\pgfqpoint{0.754942in}{2.179252in}}{\pgfqpoint{0.762842in}{2.175980in}}{\pgfqpoint{0.771078in}{2.175980in}}%
\pgfpathclose%
\pgfusepath{stroke,fill}%
\end{pgfscope}%
\begin{pgfscope}%
\pgfpathrectangle{\pgfqpoint{0.100000in}{0.212622in}}{\pgfqpoint{3.696000in}{3.696000in}}%
\pgfusepath{clip}%
\pgfsetbuttcap%
\pgfsetroundjoin%
\definecolor{currentfill}{rgb}{0.121569,0.466667,0.705882}%
\pgfsetfillcolor{currentfill}%
\pgfsetfillopacity{0.776462}%
\pgfsetlinewidth{1.003750pt}%
\definecolor{currentstroke}{rgb}{0.121569,0.466667,0.705882}%
\pgfsetstrokecolor{currentstroke}%
\pgfsetstrokeopacity{0.776462}%
\pgfsetdash{}{0pt}%
\pgfpathmoveto{\pgfqpoint{2.871678in}{1.820795in}}%
\pgfpathcurveto{\pgfqpoint{2.879914in}{1.820795in}}{\pgfqpoint{2.887814in}{1.824067in}}{\pgfqpoint{2.893638in}{1.829891in}}%
\pgfpathcurveto{\pgfqpoint{2.899462in}{1.835715in}}{\pgfqpoint{2.902734in}{1.843615in}}{\pgfqpoint{2.902734in}{1.851852in}}%
\pgfpathcurveto{\pgfqpoint{2.902734in}{1.860088in}}{\pgfqpoint{2.899462in}{1.867988in}}{\pgfqpoint{2.893638in}{1.873812in}}%
\pgfpathcurveto{\pgfqpoint{2.887814in}{1.879636in}}{\pgfqpoint{2.879914in}{1.882908in}}{\pgfqpoint{2.871678in}{1.882908in}}%
\pgfpathcurveto{\pgfqpoint{2.863441in}{1.882908in}}{\pgfqpoint{2.855541in}{1.879636in}}{\pgfqpoint{2.849717in}{1.873812in}}%
\pgfpathcurveto{\pgfqpoint{2.843893in}{1.867988in}}{\pgfqpoint{2.840621in}{1.860088in}}{\pgfqpoint{2.840621in}{1.851852in}}%
\pgfpathcurveto{\pgfqpoint{2.840621in}{1.843615in}}{\pgfqpoint{2.843893in}{1.835715in}}{\pgfqpoint{2.849717in}{1.829891in}}%
\pgfpathcurveto{\pgfqpoint{2.855541in}{1.824067in}}{\pgfqpoint{2.863441in}{1.820795in}}{\pgfqpoint{2.871678in}{1.820795in}}%
\pgfpathclose%
\pgfusepath{stroke,fill}%
\end{pgfscope}%
\begin{pgfscope}%
\pgfpathrectangle{\pgfqpoint{0.100000in}{0.212622in}}{\pgfqpoint{3.696000in}{3.696000in}}%
\pgfusepath{clip}%
\pgfsetbuttcap%
\pgfsetroundjoin%
\definecolor{currentfill}{rgb}{0.121569,0.466667,0.705882}%
\pgfsetfillcolor{currentfill}%
\pgfsetfillopacity{0.776542}%
\pgfsetlinewidth{1.003750pt}%
\definecolor{currentstroke}{rgb}{0.121569,0.466667,0.705882}%
\pgfsetstrokecolor{currentstroke}%
\pgfsetstrokeopacity{0.776542}%
\pgfsetdash{}{0pt}%
\pgfpathmoveto{\pgfqpoint{0.770601in}{2.175985in}}%
\pgfpathcurveto{\pgfqpoint{0.778837in}{2.175985in}}{\pgfqpoint{0.786737in}{2.179257in}}{\pgfqpoint{0.792561in}{2.185081in}}%
\pgfpathcurveto{\pgfqpoint{0.798385in}{2.190905in}}{\pgfqpoint{0.801657in}{2.198805in}}{\pgfqpoint{0.801657in}{2.207041in}}%
\pgfpathcurveto{\pgfqpoint{0.801657in}{2.215277in}}{\pgfqpoint{0.798385in}{2.223177in}}{\pgfqpoint{0.792561in}{2.229001in}}%
\pgfpathcurveto{\pgfqpoint{0.786737in}{2.234825in}}{\pgfqpoint{0.778837in}{2.238098in}}{\pgfqpoint{0.770601in}{2.238098in}}%
\pgfpathcurveto{\pgfqpoint{0.762365in}{2.238098in}}{\pgfqpoint{0.754464in}{2.234825in}}{\pgfqpoint{0.748641in}{2.229001in}}%
\pgfpathcurveto{\pgfqpoint{0.742817in}{2.223177in}}{\pgfqpoint{0.739544in}{2.215277in}}{\pgfqpoint{0.739544in}{2.207041in}}%
\pgfpathcurveto{\pgfqpoint{0.739544in}{2.198805in}}{\pgfqpoint{0.742817in}{2.190905in}}{\pgfqpoint{0.748641in}{2.185081in}}%
\pgfpathcurveto{\pgfqpoint{0.754464in}{2.179257in}}{\pgfqpoint{0.762365in}{2.175985in}}{\pgfqpoint{0.770601in}{2.175985in}}%
\pgfpathclose%
\pgfusepath{stroke,fill}%
\end{pgfscope}%
\begin{pgfscope}%
\pgfpathrectangle{\pgfqpoint{0.100000in}{0.212622in}}{\pgfqpoint{3.696000in}{3.696000in}}%
\pgfusepath{clip}%
\pgfsetbuttcap%
\pgfsetroundjoin%
\definecolor{currentfill}{rgb}{0.121569,0.466667,0.705882}%
\pgfsetfillcolor{currentfill}%
\pgfsetfillopacity{0.777075}%
\pgfsetlinewidth{1.003750pt}%
\definecolor{currentstroke}{rgb}{0.121569,0.466667,0.705882}%
\pgfsetstrokecolor{currentstroke}%
\pgfsetstrokeopacity{0.777075}%
\pgfsetdash{}{0pt}%
\pgfpathmoveto{\pgfqpoint{0.769784in}{2.175946in}}%
\pgfpathcurveto{\pgfqpoint{0.778020in}{2.175946in}}{\pgfqpoint{0.785920in}{2.179219in}}{\pgfqpoint{0.791744in}{2.185043in}}%
\pgfpathcurveto{\pgfqpoint{0.797568in}{2.190866in}}{\pgfqpoint{0.800840in}{2.198766in}}{\pgfqpoint{0.800840in}{2.207003in}}%
\pgfpathcurveto{\pgfqpoint{0.800840in}{2.215239in}}{\pgfqpoint{0.797568in}{2.223139in}}{\pgfqpoint{0.791744in}{2.228963in}}%
\pgfpathcurveto{\pgfqpoint{0.785920in}{2.234787in}}{\pgfqpoint{0.778020in}{2.238059in}}{\pgfqpoint{0.769784in}{2.238059in}}%
\pgfpathcurveto{\pgfqpoint{0.761547in}{2.238059in}}{\pgfqpoint{0.753647in}{2.234787in}}{\pgfqpoint{0.747823in}{2.228963in}}%
\pgfpathcurveto{\pgfqpoint{0.741999in}{2.223139in}}{\pgfqpoint{0.738727in}{2.215239in}}{\pgfqpoint{0.738727in}{2.207003in}}%
\pgfpathcurveto{\pgfqpoint{0.738727in}{2.198766in}}{\pgfqpoint{0.741999in}{2.190866in}}{\pgfqpoint{0.747823in}{2.185043in}}%
\pgfpathcurveto{\pgfqpoint{0.753647in}{2.179219in}}{\pgfqpoint{0.761547in}{2.175946in}}{\pgfqpoint{0.769784in}{2.175946in}}%
\pgfpathclose%
\pgfusepath{stroke,fill}%
\end{pgfscope}%
\begin{pgfscope}%
\pgfpathrectangle{\pgfqpoint{0.100000in}{0.212622in}}{\pgfqpoint{3.696000in}{3.696000in}}%
\pgfusepath{clip}%
\pgfsetbuttcap%
\pgfsetroundjoin%
\definecolor{currentfill}{rgb}{0.121569,0.466667,0.705882}%
\pgfsetfillcolor{currentfill}%
\pgfsetfillopacity{0.778029}%
\pgfsetlinewidth{1.003750pt}%
\definecolor{currentstroke}{rgb}{0.121569,0.466667,0.705882}%
\pgfsetstrokecolor{currentstroke}%
\pgfsetstrokeopacity{0.778029}%
\pgfsetdash{}{0pt}%
\pgfpathmoveto{\pgfqpoint{0.768091in}{2.175914in}}%
\pgfpathcurveto{\pgfqpoint{0.776327in}{2.175914in}}{\pgfqpoint{0.784227in}{2.179186in}}{\pgfqpoint{0.790051in}{2.185010in}}%
\pgfpathcurveto{\pgfqpoint{0.795875in}{2.190834in}}{\pgfqpoint{0.799147in}{2.198734in}}{\pgfqpoint{0.799147in}{2.206970in}}%
\pgfpathcurveto{\pgfqpoint{0.799147in}{2.215207in}}{\pgfqpoint{0.795875in}{2.223107in}}{\pgfqpoint{0.790051in}{2.228931in}}%
\pgfpathcurveto{\pgfqpoint{0.784227in}{2.234755in}}{\pgfqpoint{0.776327in}{2.238027in}}{\pgfqpoint{0.768091in}{2.238027in}}%
\pgfpathcurveto{\pgfqpoint{0.759854in}{2.238027in}}{\pgfqpoint{0.751954in}{2.234755in}}{\pgfqpoint{0.746130in}{2.228931in}}%
\pgfpathcurveto{\pgfqpoint{0.740306in}{2.223107in}}{\pgfqpoint{0.737034in}{2.215207in}}{\pgfqpoint{0.737034in}{2.206970in}}%
\pgfpathcurveto{\pgfqpoint{0.737034in}{2.198734in}}{\pgfqpoint{0.740306in}{2.190834in}}{\pgfqpoint{0.746130in}{2.185010in}}%
\pgfpathcurveto{\pgfqpoint{0.751954in}{2.179186in}}{\pgfqpoint{0.759854in}{2.175914in}}{\pgfqpoint{0.768091in}{2.175914in}}%
\pgfpathclose%
\pgfusepath{stroke,fill}%
\end{pgfscope}%
\begin{pgfscope}%
\pgfpathrectangle{\pgfqpoint{0.100000in}{0.212622in}}{\pgfqpoint{3.696000in}{3.696000in}}%
\pgfusepath{clip}%
\pgfsetbuttcap%
\pgfsetroundjoin%
\definecolor{currentfill}{rgb}{0.121569,0.466667,0.705882}%
\pgfsetfillcolor{currentfill}%
\pgfsetfillopacity{0.778070}%
\pgfsetlinewidth{1.003750pt}%
\definecolor{currentstroke}{rgb}{0.121569,0.466667,0.705882}%
\pgfsetstrokecolor{currentstroke}%
\pgfsetstrokeopacity{0.778070}%
\pgfsetdash{}{0pt}%
\pgfpathmoveto{\pgfqpoint{2.868423in}{1.821221in}}%
\pgfpathcurveto{\pgfqpoint{2.876659in}{1.821221in}}{\pgfqpoint{2.884560in}{1.824493in}}{\pgfqpoint{2.890383in}{1.830317in}}%
\pgfpathcurveto{\pgfqpoint{2.896207in}{1.836141in}}{\pgfqpoint{2.899480in}{1.844041in}}{\pgfqpoint{2.899480in}{1.852277in}}%
\pgfpathcurveto{\pgfqpoint{2.899480in}{1.860514in}}{\pgfqpoint{2.896207in}{1.868414in}}{\pgfqpoint{2.890383in}{1.874238in}}%
\pgfpathcurveto{\pgfqpoint{2.884560in}{1.880062in}}{\pgfqpoint{2.876659in}{1.883334in}}{\pgfqpoint{2.868423in}{1.883334in}}%
\pgfpathcurveto{\pgfqpoint{2.860187in}{1.883334in}}{\pgfqpoint{2.852287in}{1.880062in}}{\pgfqpoint{2.846463in}{1.874238in}}%
\pgfpathcurveto{\pgfqpoint{2.840639in}{1.868414in}}{\pgfqpoint{2.837367in}{1.860514in}}{\pgfqpoint{2.837367in}{1.852277in}}%
\pgfpathcurveto{\pgfqpoint{2.837367in}{1.844041in}}{\pgfqpoint{2.840639in}{1.836141in}}{\pgfqpoint{2.846463in}{1.830317in}}%
\pgfpathcurveto{\pgfqpoint{2.852287in}{1.824493in}}{\pgfqpoint{2.860187in}{1.821221in}}{\pgfqpoint{2.868423in}{1.821221in}}%
\pgfpathclose%
\pgfusepath{stroke,fill}%
\end{pgfscope}%
\begin{pgfscope}%
\pgfpathrectangle{\pgfqpoint{0.100000in}{0.212622in}}{\pgfqpoint{3.696000in}{3.696000in}}%
\pgfusepath{clip}%
\pgfsetbuttcap%
\pgfsetroundjoin%
\definecolor{currentfill}{rgb}{0.121569,0.466667,0.705882}%
\pgfsetfillcolor{currentfill}%
\pgfsetfillopacity{0.779813}%
\pgfsetlinewidth{1.003750pt}%
\definecolor{currentstroke}{rgb}{0.121569,0.466667,0.705882}%
\pgfsetstrokecolor{currentstroke}%
\pgfsetstrokeopacity{0.779813}%
\pgfsetdash{}{0pt}%
\pgfpathmoveto{\pgfqpoint{0.765461in}{2.175868in}}%
\pgfpathcurveto{\pgfqpoint{0.773698in}{2.175868in}}{\pgfqpoint{0.781598in}{2.179140in}}{\pgfqpoint{0.787422in}{2.184964in}}%
\pgfpathcurveto{\pgfqpoint{0.793246in}{2.190788in}}{\pgfqpoint{0.796518in}{2.198688in}}{\pgfqpoint{0.796518in}{2.206924in}}%
\pgfpathcurveto{\pgfqpoint{0.796518in}{2.215160in}}{\pgfqpoint{0.793246in}{2.223061in}}{\pgfqpoint{0.787422in}{2.228884in}}%
\pgfpathcurveto{\pgfqpoint{0.781598in}{2.234708in}}{\pgfqpoint{0.773698in}{2.237981in}}{\pgfqpoint{0.765461in}{2.237981in}}%
\pgfpathcurveto{\pgfqpoint{0.757225in}{2.237981in}}{\pgfqpoint{0.749325in}{2.234708in}}{\pgfqpoint{0.743501in}{2.228884in}}%
\pgfpathcurveto{\pgfqpoint{0.737677in}{2.223061in}}{\pgfqpoint{0.734405in}{2.215160in}}{\pgfqpoint{0.734405in}{2.206924in}}%
\pgfpathcurveto{\pgfqpoint{0.734405in}{2.198688in}}{\pgfqpoint{0.737677in}{2.190788in}}{\pgfqpoint{0.743501in}{2.184964in}}%
\pgfpathcurveto{\pgfqpoint{0.749325in}{2.179140in}}{\pgfqpoint{0.757225in}{2.175868in}}{\pgfqpoint{0.765461in}{2.175868in}}%
\pgfpathclose%
\pgfusepath{stroke,fill}%
\end{pgfscope}%
\begin{pgfscope}%
\pgfpathrectangle{\pgfqpoint{0.100000in}{0.212622in}}{\pgfqpoint{3.696000in}{3.696000in}}%
\pgfusepath{clip}%
\pgfsetbuttcap%
\pgfsetroundjoin%
\definecolor{currentfill}{rgb}{0.121569,0.466667,0.705882}%
\pgfsetfillcolor{currentfill}%
\pgfsetfillopacity{0.779980}%
\pgfsetlinewidth{1.003750pt}%
\definecolor{currentstroke}{rgb}{0.121569,0.466667,0.705882}%
\pgfsetstrokecolor{currentstroke}%
\pgfsetstrokeopacity{0.779980}%
\pgfsetdash{}{0pt}%
\pgfpathmoveto{\pgfqpoint{2.866493in}{1.821382in}}%
\pgfpathcurveto{\pgfqpoint{2.874729in}{1.821382in}}{\pgfqpoint{2.882629in}{1.824655in}}{\pgfqpoint{2.888453in}{1.830479in}}%
\pgfpathcurveto{\pgfqpoint{2.894277in}{1.836303in}}{\pgfqpoint{2.897550in}{1.844203in}}{\pgfqpoint{2.897550in}{1.852439in}}%
\pgfpathcurveto{\pgfqpoint{2.897550in}{1.860675in}}{\pgfqpoint{2.894277in}{1.868575in}}{\pgfqpoint{2.888453in}{1.874399in}}%
\pgfpathcurveto{\pgfqpoint{2.882629in}{1.880223in}}{\pgfqpoint{2.874729in}{1.883495in}}{\pgfqpoint{2.866493in}{1.883495in}}%
\pgfpathcurveto{\pgfqpoint{2.858257in}{1.883495in}}{\pgfqpoint{2.850357in}{1.880223in}}{\pgfqpoint{2.844533in}{1.874399in}}%
\pgfpathcurveto{\pgfqpoint{2.838709in}{1.868575in}}{\pgfqpoint{2.835437in}{1.860675in}}{\pgfqpoint{2.835437in}{1.852439in}}%
\pgfpathcurveto{\pgfqpoint{2.835437in}{1.844203in}}{\pgfqpoint{2.838709in}{1.836303in}}{\pgfqpoint{2.844533in}{1.830479in}}%
\pgfpathcurveto{\pgfqpoint{2.850357in}{1.824655in}}{\pgfqpoint{2.858257in}{1.821382in}}{\pgfqpoint{2.866493in}{1.821382in}}%
\pgfpathclose%
\pgfusepath{stroke,fill}%
\end{pgfscope}%
\begin{pgfscope}%
\pgfpathrectangle{\pgfqpoint{0.100000in}{0.212622in}}{\pgfqpoint{3.696000in}{3.696000in}}%
\pgfusepath{clip}%
\pgfsetbuttcap%
\pgfsetroundjoin%
\definecolor{currentfill}{rgb}{0.121569,0.466667,0.705882}%
\pgfsetfillcolor{currentfill}%
\pgfsetfillopacity{0.781306}%
\pgfsetlinewidth{1.003750pt}%
\definecolor{currentstroke}{rgb}{0.121569,0.466667,0.705882}%
\pgfsetstrokecolor{currentstroke}%
\pgfsetstrokeopacity{0.781306}%
\pgfsetdash{}{0pt}%
\pgfpathmoveto{\pgfqpoint{0.762785in}{2.175645in}}%
\pgfpathcurveto{\pgfqpoint{0.771021in}{2.175645in}}{\pgfqpoint{0.778921in}{2.178918in}}{\pgfqpoint{0.784745in}{2.184741in}}%
\pgfpathcurveto{\pgfqpoint{0.790569in}{2.190565in}}{\pgfqpoint{0.793841in}{2.198465in}}{\pgfqpoint{0.793841in}{2.206702in}}%
\pgfpathcurveto{\pgfqpoint{0.793841in}{2.214938in}}{\pgfqpoint{0.790569in}{2.222838in}}{\pgfqpoint{0.784745in}{2.228662in}}%
\pgfpathcurveto{\pgfqpoint{0.778921in}{2.234486in}}{\pgfqpoint{0.771021in}{2.237758in}}{\pgfqpoint{0.762785in}{2.237758in}}%
\pgfpathcurveto{\pgfqpoint{0.754548in}{2.237758in}}{\pgfqpoint{0.746648in}{2.234486in}}{\pgfqpoint{0.740824in}{2.228662in}}%
\pgfpathcurveto{\pgfqpoint{0.735000in}{2.222838in}}{\pgfqpoint{0.731728in}{2.214938in}}{\pgfqpoint{0.731728in}{2.206702in}}%
\pgfpathcurveto{\pgfqpoint{0.731728in}{2.198465in}}{\pgfqpoint{0.735000in}{2.190565in}}{\pgfqpoint{0.740824in}{2.184741in}}%
\pgfpathcurveto{\pgfqpoint{0.746648in}{2.178918in}}{\pgfqpoint{0.754548in}{2.175645in}}{\pgfqpoint{0.762785in}{2.175645in}}%
\pgfpathclose%
\pgfusepath{stroke,fill}%
\end{pgfscope}%
\begin{pgfscope}%
\pgfpathrectangle{\pgfqpoint{0.100000in}{0.212622in}}{\pgfqpoint{3.696000in}{3.696000in}}%
\pgfusepath{clip}%
\pgfsetbuttcap%
\pgfsetroundjoin%
\definecolor{currentfill}{rgb}{0.121569,0.466667,0.705882}%
\pgfsetfillcolor{currentfill}%
\pgfsetfillopacity{0.782443}%
\pgfsetlinewidth{1.003750pt}%
\definecolor{currentstroke}{rgb}{0.121569,0.466667,0.705882}%
\pgfsetstrokecolor{currentstroke}%
\pgfsetstrokeopacity{0.782443}%
\pgfsetdash{}{0pt}%
\pgfpathmoveto{\pgfqpoint{2.860569in}{1.822894in}}%
\pgfpathcurveto{\pgfqpoint{2.868805in}{1.822894in}}{\pgfqpoint{2.876705in}{1.826166in}}{\pgfqpoint{2.882529in}{1.831990in}}%
\pgfpathcurveto{\pgfqpoint{2.888353in}{1.837814in}}{\pgfqpoint{2.891626in}{1.845714in}}{\pgfqpoint{2.891626in}{1.853951in}}%
\pgfpathcurveto{\pgfqpoint{2.891626in}{1.862187in}}{\pgfqpoint{2.888353in}{1.870087in}}{\pgfqpoint{2.882529in}{1.875911in}}%
\pgfpathcurveto{\pgfqpoint{2.876705in}{1.881735in}}{\pgfqpoint{2.868805in}{1.885007in}}{\pgfqpoint{2.860569in}{1.885007in}}%
\pgfpathcurveto{\pgfqpoint{2.852333in}{1.885007in}}{\pgfqpoint{2.844433in}{1.881735in}}{\pgfqpoint{2.838609in}{1.875911in}}%
\pgfpathcurveto{\pgfqpoint{2.832785in}{1.870087in}}{\pgfqpoint{2.829513in}{1.862187in}}{\pgfqpoint{2.829513in}{1.853951in}}%
\pgfpathcurveto{\pgfqpoint{2.829513in}{1.845714in}}{\pgfqpoint{2.832785in}{1.837814in}}{\pgfqpoint{2.838609in}{1.831990in}}%
\pgfpathcurveto{\pgfqpoint{2.844433in}{1.826166in}}{\pgfqpoint{2.852333in}{1.822894in}}{\pgfqpoint{2.860569in}{1.822894in}}%
\pgfpathclose%
\pgfusepath{stroke,fill}%
\end{pgfscope}%
\begin{pgfscope}%
\pgfpathrectangle{\pgfqpoint{0.100000in}{0.212622in}}{\pgfqpoint{3.696000in}{3.696000in}}%
\pgfusepath{clip}%
\pgfsetbuttcap%
\pgfsetroundjoin%
\definecolor{currentfill}{rgb}{0.121569,0.466667,0.705882}%
\pgfsetfillcolor{currentfill}%
\pgfsetfillopacity{0.782606}%
\pgfsetlinewidth{1.003750pt}%
\definecolor{currentstroke}{rgb}{0.121569,0.466667,0.705882}%
\pgfsetstrokecolor{currentstroke}%
\pgfsetstrokeopacity{0.782606}%
\pgfsetdash{}{0pt}%
\pgfpathmoveto{\pgfqpoint{0.760589in}{2.175611in}}%
\pgfpathcurveto{\pgfqpoint{0.768825in}{2.175611in}}{\pgfqpoint{0.776725in}{2.178883in}}{\pgfqpoint{0.782549in}{2.184707in}}%
\pgfpathcurveto{\pgfqpoint{0.788373in}{2.190531in}}{\pgfqpoint{0.791645in}{2.198431in}}{\pgfqpoint{0.791645in}{2.206667in}}%
\pgfpathcurveto{\pgfqpoint{0.791645in}{2.214903in}}{\pgfqpoint{0.788373in}{2.222803in}}{\pgfqpoint{0.782549in}{2.228627in}}%
\pgfpathcurveto{\pgfqpoint{0.776725in}{2.234451in}}{\pgfqpoint{0.768825in}{2.237724in}}{\pgfqpoint{0.760589in}{2.237724in}}%
\pgfpathcurveto{\pgfqpoint{0.752352in}{2.237724in}}{\pgfqpoint{0.744452in}{2.234451in}}{\pgfqpoint{0.738628in}{2.228627in}}%
\pgfpathcurveto{\pgfqpoint{0.732804in}{2.222803in}}{\pgfqpoint{0.729532in}{2.214903in}}{\pgfqpoint{0.729532in}{2.206667in}}%
\pgfpathcurveto{\pgfqpoint{0.729532in}{2.198431in}}{\pgfqpoint{0.732804in}{2.190531in}}{\pgfqpoint{0.738628in}{2.184707in}}%
\pgfpathcurveto{\pgfqpoint{0.744452in}{2.178883in}}{\pgfqpoint{0.752352in}{2.175611in}}{\pgfqpoint{0.760589in}{2.175611in}}%
\pgfpathclose%
\pgfusepath{stroke,fill}%
\end{pgfscope}%
\begin{pgfscope}%
\pgfpathrectangle{\pgfqpoint{0.100000in}{0.212622in}}{\pgfqpoint{3.696000in}{3.696000in}}%
\pgfusepath{clip}%
\pgfsetbuttcap%
\pgfsetroundjoin%
\definecolor{currentfill}{rgb}{0.121569,0.466667,0.705882}%
\pgfsetfillcolor{currentfill}%
\pgfsetfillopacity{0.783688}%
\pgfsetlinewidth{1.003750pt}%
\definecolor{currentstroke}{rgb}{0.121569,0.466667,0.705882}%
\pgfsetstrokecolor{currentstroke}%
\pgfsetstrokeopacity{0.783688}%
\pgfsetdash{}{0pt}%
\pgfpathmoveto{\pgfqpoint{0.758774in}{2.175545in}}%
\pgfpathcurveto{\pgfqpoint{0.767010in}{2.175545in}}{\pgfqpoint{0.774910in}{2.178817in}}{\pgfqpoint{0.780734in}{2.184641in}}%
\pgfpathcurveto{\pgfqpoint{0.786558in}{2.190465in}}{\pgfqpoint{0.789831in}{2.198365in}}{\pgfqpoint{0.789831in}{2.206601in}}%
\pgfpathcurveto{\pgfqpoint{0.789831in}{2.214837in}}{\pgfqpoint{0.786558in}{2.222737in}}{\pgfqpoint{0.780734in}{2.228561in}}%
\pgfpathcurveto{\pgfqpoint{0.774910in}{2.234385in}}{\pgfqpoint{0.767010in}{2.237658in}}{\pgfqpoint{0.758774in}{2.237658in}}%
\pgfpathcurveto{\pgfqpoint{0.750538in}{2.237658in}}{\pgfqpoint{0.742638in}{2.234385in}}{\pgfqpoint{0.736814in}{2.228561in}}%
\pgfpathcurveto{\pgfqpoint{0.730990in}{2.222737in}}{\pgfqpoint{0.727718in}{2.214837in}}{\pgfqpoint{0.727718in}{2.206601in}}%
\pgfpathcurveto{\pgfqpoint{0.727718in}{2.198365in}}{\pgfqpoint{0.730990in}{2.190465in}}{\pgfqpoint{0.736814in}{2.184641in}}%
\pgfpathcurveto{\pgfqpoint{0.742638in}{2.178817in}}{\pgfqpoint{0.750538in}{2.175545in}}{\pgfqpoint{0.758774in}{2.175545in}}%
\pgfpathclose%
\pgfusepath{stroke,fill}%
\end{pgfscope}%
\begin{pgfscope}%
\pgfpathrectangle{\pgfqpoint{0.100000in}{0.212622in}}{\pgfqpoint{3.696000in}{3.696000in}}%
\pgfusepath{clip}%
\pgfsetbuttcap%
\pgfsetroundjoin%
\definecolor{currentfill}{rgb}{0.121569,0.466667,0.705882}%
\pgfsetfillcolor{currentfill}%
\pgfsetfillopacity{0.784474}%
\pgfsetlinewidth{1.003750pt}%
\definecolor{currentstroke}{rgb}{0.121569,0.466667,0.705882}%
\pgfsetstrokecolor{currentstroke}%
\pgfsetstrokeopacity{0.784474}%
\pgfsetdash{}{0pt}%
\pgfpathmoveto{\pgfqpoint{0.757589in}{2.175376in}}%
\pgfpathcurveto{\pgfqpoint{0.765825in}{2.175376in}}{\pgfqpoint{0.773725in}{2.178649in}}{\pgfqpoint{0.779549in}{2.184473in}}%
\pgfpathcurveto{\pgfqpoint{0.785373in}{2.190297in}}{\pgfqpoint{0.788645in}{2.198197in}}{\pgfqpoint{0.788645in}{2.206433in}}%
\pgfpathcurveto{\pgfqpoint{0.788645in}{2.214669in}}{\pgfqpoint{0.785373in}{2.222569in}}{\pgfqpoint{0.779549in}{2.228393in}}%
\pgfpathcurveto{\pgfqpoint{0.773725in}{2.234217in}}{\pgfqpoint{0.765825in}{2.237489in}}{\pgfqpoint{0.757589in}{2.237489in}}%
\pgfpathcurveto{\pgfqpoint{0.749352in}{2.237489in}}{\pgfqpoint{0.741452in}{2.234217in}}{\pgfqpoint{0.735628in}{2.228393in}}%
\pgfpathcurveto{\pgfqpoint{0.729804in}{2.222569in}}{\pgfqpoint{0.726532in}{2.214669in}}{\pgfqpoint{0.726532in}{2.206433in}}%
\pgfpathcurveto{\pgfqpoint{0.726532in}{2.198197in}}{\pgfqpoint{0.729804in}{2.190297in}}{\pgfqpoint{0.735628in}{2.184473in}}%
\pgfpathcurveto{\pgfqpoint{0.741452in}{2.178649in}}{\pgfqpoint{0.749352in}{2.175376in}}{\pgfqpoint{0.757589in}{2.175376in}}%
\pgfpathclose%
\pgfusepath{stroke,fill}%
\end{pgfscope}%
\begin{pgfscope}%
\pgfpathrectangle{\pgfqpoint{0.100000in}{0.212622in}}{\pgfqpoint{3.696000in}{3.696000in}}%
\pgfusepath{clip}%
\pgfsetbuttcap%
\pgfsetroundjoin%
\definecolor{currentfill}{rgb}{0.121569,0.466667,0.705882}%
\pgfsetfillcolor{currentfill}%
\pgfsetfillopacity{0.785031}%
\pgfsetlinewidth{1.003750pt}%
\definecolor{currentstroke}{rgb}{0.121569,0.466667,0.705882}%
\pgfsetstrokecolor{currentstroke}%
\pgfsetstrokeopacity{0.785031}%
\pgfsetdash{}{0pt}%
\pgfpathmoveto{\pgfqpoint{0.756943in}{2.175337in}}%
\pgfpathcurveto{\pgfqpoint{0.765179in}{2.175337in}}{\pgfqpoint{0.773079in}{2.178609in}}{\pgfqpoint{0.778903in}{2.184433in}}%
\pgfpathcurveto{\pgfqpoint{0.784727in}{2.190257in}}{\pgfqpoint{0.787999in}{2.198157in}}{\pgfqpoint{0.787999in}{2.206393in}}%
\pgfpathcurveto{\pgfqpoint{0.787999in}{2.214629in}}{\pgfqpoint{0.784727in}{2.222529in}}{\pgfqpoint{0.778903in}{2.228353in}}%
\pgfpathcurveto{\pgfqpoint{0.773079in}{2.234177in}}{\pgfqpoint{0.765179in}{2.237450in}}{\pgfqpoint{0.756943in}{2.237450in}}%
\pgfpathcurveto{\pgfqpoint{0.748706in}{2.237450in}}{\pgfqpoint{0.740806in}{2.234177in}}{\pgfqpoint{0.734982in}{2.228353in}}%
\pgfpathcurveto{\pgfqpoint{0.729158in}{2.222529in}}{\pgfqpoint{0.725886in}{2.214629in}}{\pgfqpoint{0.725886in}{2.206393in}}%
\pgfpathcurveto{\pgfqpoint{0.725886in}{2.198157in}}{\pgfqpoint{0.729158in}{2.190257in}}{\pgfqpoint{0.734982in}{2.184433in}}%
\pgfpathcurveto{\pgfqpoint{0.740806in}{2.178609in}}{\pgfqpoint{0.748706in}{2.175337in}}{\pgfqpoint{0.756943in}{2.175337in}}%
\pgfpathclose%
\pgfusepath{stroke,fill}%
\end{pgfscope}%
\begin{pgfscope}%
\pgfpathrectangle{\pgfqpoint{0.100000in}{0.212622in}}{\pgfqpoint{3.696000in}{3.696000in}}%
\pgfusepath{clip}%
\pgfsetbuttcap%
\pgfsetroundjoin%
\definecolor{currentfill}{rgb}{0.121569,0.466667,0.705882}%
\pgfsetfillcolor{currentfill}%
\pgfsetfillopacity{0.785070}%
\pgfsetlinewidth{1.003750pt}%
\definecolor{currentstroke}{rgb}{0.121569,0.466667,0.705882}%
\pgfsetstrokecolor{currentstroke}%
\pgfsetstrokeopacity{0.785070}%
\pgfsetdash{}{0pt}%
\pgfpathmoveto{\pgfqpoint{2.854886in}{1.823949in}}%
\pgfpathcurveto{\pgfqpoint{2.863122in}{1.823949in}}{\pgfqpoint{2.871023in}{1.827221in}}{\pgfqpoint{2.876846in}{1.833045in}}%
\pgfpathcurveto{\pgfqpoint{2.882670in}{1.838869in}}{\pgfqpoint{2.885943in}{1.846769in}}{\pgfqpoint{2.885943in}{1.855005in}}%
\pgfpathcurveto{\pgfqpoint{2.885943in}{1.863242in}}{\pgfqpoint{2.882670in}{1.871142in}}{\pgfqpoint{2.876846in}{1.876966in}}%
\pgfpathcurveto{\pgfqpoint{2.871023in}{1.882790in}}{\pgfqpoint{2.863122in}{1.886062in}}{\pgfqpoint{2.854886in}{1.886062in}}%
\pgfpathcurveto{\pgfqpoint{2.846650in}{1.886062in}}{\pgfqpoint{2.838750in}{1.882790in}}{\pgfqpoint{2.832926in}{1.876966in}}%
\pgfpathcurveto{\pgfqpoint{2.827102in}{1.871142in}}{\pgfqpoint{2.823830in}{1.863242in}}{\pgfqpoint{2.823830in}{1.855005in}}%
\pgfpathcurveto{\pgfqpoint{2.823830in}{1.846769in}}{\pgfqpoint{2.827102in}{1.838869in}}{\pgfqpoint{2.832926in}{1.833045in}}%
\pgfpathcurveto{\pgfqpoint{2.838750in}{1.827221in}}{\pgfqpoint{2.846650in}{1.823949in}}{\pgfqpoint{2.854886in}{1.823949in}}%
\pgfpathclose%
\pgfusepath{stroke,fill}%
\end{pgfscope}%
\begin{pgfscope}%
\pgfpathrectangle{\pgfqpoint{0.100000in}{0.212622in}}{\pgfqpoint{3.696000in}{3.696000in}}%
\pgfusepath{clip}%
\pgfsetbuttcap%
\pgfsetroundjoin%
\definecolor{currentfill}{rgb}{0.121569,0.466667,0.705882}%
\pgfsetfillcolor{currentfill}%
\pgfsetfillopacity{0.785376}%
\pgfsetlinewidth{1.003750pt}%
\definecolor{currentstroke}{rgb}{0.121569,0.466667,0.705882}%
\pgfsetstrokecolor{currentstroke}%
\pgfsetstrokeopacity{0.785376}%
\pgfsetdash{}{0pt}%
\pgfpathmoveto{\pgfqpoint{0.756877in}{2.175247in}}%
\pgfpathcurveto{\pgfqpoint{0.765114in}{2.175247in}}{\pgfqpoint{0.773014in}{2.178519in}}{\pgfqpoint{0.778838in}{2.184343in}}%
\pgfpathcurveto{\pgfqpoint{0.784662in}{2.190167in}}{\pgfqpoint{0.787934in}{2.198067in}}{\pgfqpoint{0.787934in}{2.206303in}}%
\pgfpathcurveto{\pgfqpoint{0.787934in}{2.214539in}}{\pgfqpoint{0.784662in}{2.222439in}}{\pgfqpoint{0.778838in}{2.228263in}}%
\pgfpathcurveto{\pgfqpoint{0.773014in}{2.234087in}}{\pgfqpoint{0.765114in}{2.237360in}}{\pgfqpoint{0.756877in}{2.237360in}}%
\pgfpathcurveto{\pgfqpoint{0.748641in}{2.237360in}}{\pgfqpoint{0.740741in}{2.234087in}}{\pgfqpoint{0.734917in}{2.228263in}}%
\pgfpathcurveto{\pgfqpoint{0.729093in}{2.222439in}}{\pgfqpoint{0.725821in}{2.214539in}}{\pgfqpoint{0.725821in}{2.206303in}}%
\pgfpathcurveto{\pgfqpoint{0.725821in}{2.198067in}}{\pgfqpoint{0.729093in}{2.190167in}}{\pgfqpoint{0.734917in}{2.184343in}}%
\pgfpathcurveto{\pgfqpoint{0.740741in}{2.178519in}}{\pgfqpoint{0.748641in}{2.175247in}}{\pgfqpoint{0.756877in}{2.175247in}}%
\pgfpathclose%
\pgfusepath{stroke,fill}%
\end{pgfscope}%
\begin{pgfscope}%
\pgfpathrectangle{\pgfqpoint{0.100000in}{0.212622in}}{\pgfqpoint{3.696000in}{3.696000in}}%
\pgfusepath{clip}%
\pgfsetbuttcap%
\pgfsetroundjoin%
\definecolor{currentfill}{rgb}{0.121569,0.466667,0.705882}%
\pgfsetfillcolor{currentfill}%
\pgfsetfillopacity{0.785908}%
\pgfsetlinewidth{1.003750pt}%
\definecolor{currentstroke}{rgb}{0.121569,0.466667,0.705882}%
\pgfsetstrokecolor{currentstroke}%
\pgfsetstrokeopacity{0.785908}%
\pgfsetdash{}{0pt}%
\pgfpathmoveto{\pgfqpoint{0.757702in}{2.174826in}}%
\pgfpathcurveto{\pgfqpoint{0.765938in}{2.174826in}}{\pgfqpoint{0.773838in}{2.178098in}}{\pgfqpoint{0.779662in}{2.183922in}}%
\pgfpathcurveto{\pgfqpoint{0.785486in}{2.189746in}}{\pgfqpoint{0.788758in}{2.197646in}}{\pgfqpoint{0.788758in}{2.205882in}}%
\pgfpathcurveto{\pgfqpoint{0.788758in}{2.214119in}}{\pgfqpoint{0.785486in}{2.222019in}}{\pgfqpoint{0.779662in}{2.227843in}}%
\pgfpathcurveto{\pgfqpoint{0.773838in}{2.233667in}}{\pgfqpoint{0.765938in}{2.236939in}}{\pgfqpoint{0.757702in}{2.236939in}}%
\pgfpathcurveto{\pgfqpoint{0.749466in}{2.236939in}}{\pgfqpoint{0.741566in}{2.233667in}}{\pgfqpoint{0.735742in}{2.227843in}}%
\pgfpathcurveto{\pgfqpoint{0.729918in}{2.222019in}}{\pgfqpoint{0.726645in}{2.214119in}}{\pgfqpoint{0.726645in}{2.205882in}}%
\pgfpathcurveto{\pgfqpoint{0.726645in}{2.197646in}}{\pgfqpoint{0.729918in}{2.189746in}}{\pgfqpoint{0.735742in}{2.183922in}}%
\pgfpathcurveto{\pgfqpoint{0.741566in}{2.178098in}}{\pgfqpoint{0.749466in}{2.174826in}}{\pgfqpoint{0.757702in}{2.174826in}}%
\pgfpathclose%
\pgfusepath{stroke,fill}%
\end{pgfscope}%
\begin{pgfscope}%
\pgfpathrectangle{\pgfqpoint{0.100000in}{0.212622in}}{\pgfqpoint{3.696000in}{3.696000in}}%
\pgfusepath{clip}%
\pgfsetbuttcap%
\pgfsetroundjoin%
\definecolor{currentfill}{rgb}{0.121569,0.466667,0.705882}%
\pgfsetfillcolor{currentfill}%
\pgfsetfillopacity{0.786057}%
\pgfsetlinewidth{1.003750pt}%
\definecolor{currentstroke}{rgb}{0.121569,0.466667,0.705882}%
\pgfsetstrokecolor{currentstroke}%
\pgfsetstrokeopacity{0.786057}%
\pgfsetdash{}{0pt}%
\pgfpathmoveto{\pgfqpoint{0.760001in}{2.174041in}}%
\pgfpathcurveto{\pgfqpoint{0.768237in}{2.174041in}}{\pgfqpoint{0.776137in}{2.177313in}}{\pgfqpoint{0.781961in}{2.183137in}}%
\pgfpathcurveto{\pgfqpoint{0.787785in}{2.188961in}}{\pgfqpoint{0.791057in}{2.196861in}}{\pgfqpoint{0.791057in}{2.205098in}}%
\pgfpathcurveto{\pgfqpoint{0.791057in}{2.213334in}}{\pgfqpoint{0.787785in}{2.221234in}}{\pgfqpoint{0.781961in}{2.227058in}}%
\pgfpathcurveto{\pgfqpoint{0.776137in}{2.232882in}}{\pgfqpoint{0.768237in}{2.236154in}}{\pgfqpoint{0.760001in}{2.236154in}}%
\pgfpathcurveto{\pgfqpoint{0.751764in}{2.236154in}}{\pgfqpoint{0.743864in}{2.232882in}}{\pgfqpoint{0.738040in}{2.227058in}}%
\pgfpathcurveto{\pgfqpoint{0.732217in}{2.221234in}}{\pgfqpoint{0.728944in}{2.213334in}}{\pgfqpoint{0.728944in}{2.205098in}}%
\pgfpathcurveto{\pgfqpoint{0.728944in}{2.196861in}}{\pgfqpoint{0.732217in}{2.188961in}}{\pgfqpoint{0.738040in}{2.183137in}}%
\pgfpathcurveto{\pgfqpoint{0.743864in}{2.177313in}}{\pgfqpoint{0.751764in}{2.174041in}}{\pgfqpoint{0.760001in}{2.174041in}}%
\pgfpathclose%
\pgfusepath{stroke,fill}%
\end{pgfscope}%
\begin{pgfscope}%
\pgfpathrectangle{\pgfqpoint{0.100000in}{0.212622in}}{\pgfqpoint{3.696000in}{3.696000in}}%
\pgfusepath{clip}%
\pgfsetbuttcap%
\pgfsetroundjoin%
\definecolor{currentfill}{rgb}{0.121569,0.466667,0.705882}%
\pgfsetfillcolor{currentfill}%
\pgfsetfillopacity{0.786057}%
\pgfsetlinewidth{1.003750pt}%
\definecolor{currentstroke}{rgb}{0.121569,0.466667,0.705882}%
\pgfsetstrokecolor{currentstroke}%
\pgfsetstrokeopacity{0.786057}%
\pgfsetdash{}{0pt}%
\pgfpathmoveto{\pgfqpoint{0.760001in}{2.174041in}}%
\pgfpathcurveto{\pgfqpoint{0.768237in}{2.174041in}}{\pgfqpoint{0.776137in}{2.177313in}}{\pgfqpoint{0.781961in}{2.183137in}}%
\pgfpathcurveto{\pgfqpoint{0.787785in}{2.188961in}}{\pgfqpoint{0.791057in}{2.196861in}}{\pgfqpoint{0.791057in}{2.205098in}}%
\pgfpathcurveto{\pgfqpoint{0.791057in}{2.213334in}}{\pgfqpoint{0.787785in}{2.221234in}}{\pgfqpoint{0.781961in}{2.227058in}}%
\pgfpathcurveto{\pgfqpoint{0.776137in}{2.232882in}}{\pgfqpoint{0.768237in}{2.236154in}}{\pgfqpoint{0.760001in}{2.236154in}}%
\pgfpathcurveto{\pgfqpoint{0.751764in}{2.236154in}}{\pgfqpoint{0.743864in}{2.232882in}}{\pgfqpoint{0.738040in}{2.227058in}}%
\pgfpathcurveto{\pgfqpoint{0.732217in}{2.221234in}}{\pgfqpoint{0.728944in}{2.213334in}}{\pgfqpoint{0.728944in}{2.205098in}}%
\pgfpathcurveto{\pgfqpoint{0.728944in}{2.196861in}}{\pgfqpoint{0.732217in}{2.188961in}}{\pgfqpoint{0.738040in}{2.183137in}}%
\pgfpathcurveto{\pgfqpoint{0.743864in}{2.177313in}}{\pgfqpoint{0.751764in}{2.174041in}}{\pgfqpoint{0.760001in}{2.174041in}}%
\pgfpathclose%
\pgfusepath{stroke,fill}%
\end{pgfscope}%
\begin{pgfscope}%
\pgfpathrectangle{\pgfqpoint{0.100000in}{0.212622in}}{\pgfqpoint{3.696000in}{3.696000in}}%
\pgfusepath{clip}%
\pgfsetbuttcap%
\pgfsetroundjoin%
\definecolor{currentfill}{rgb}{0.121569,0.466667,0.705882}%
\pgfsetfillcolor{currentfill}%
\pgfsetfillopacity{0.786057}%
\pgfsetlinewidth{1.003750pt}%
\definecolor{currentstroke}{rgb}{0.121569,0.466667,0.705882}%
\pgfsetstrokecolor{currentstroke}%
\pgfsetstrokeopacity{0.786057}%
\pgfsetdash{}{0pt}%
\pgfpathmoveto{\pgfqpoint{0.760001in}{2.174041in}}%
\pgfpathcurveto{\pgfqpoint{0.768237in}{2.174041in}}{\pgfqpoint{0.776137in}{2.177313in}}{\pgfqpoint{0.781961in}{2.183137in}}%
\pgfpathcurveto{\pgfqpoint{0.787785in}{2.188961in}}{\pgfqpoint{0.791057in}{2.196861in}}{\pgfqpoint{0.791057in}{2.205098in}}%
\pgfpathcurveto{\pgfqpoint{0.791057in}{2.213334in}}{\pgfqpoint{0.787785in}{2.221234in}}{\pgfqpoint{0.781961in}{2.227058in}}%
\pgfpathcurveto{\pgfqpoint{0.776137in}{2.232882in}}{\pgfqpoint{0.768237in}{2.236154in}}{\pgfqpoint{0.760001in}{2.236154in}}%
\pgfpathcurveto{\pgfqpoint{0.751764in}{2.236154in}}{\pgfqpoint{0.743864in}{2.232882in}}{\pgfqpoint{0.738040in}{2.227058in}}%
\pgfpathcurveto{\pgfqpoint{0.732217in}{2.221234in}}{\pgfqpoint{0.728944in}{2.213334in}}{\pgfqpoint{0.728944in}{2.205098in}}%
\pgfpathcurveto{\pgfqpoint{0.728944in}{2.196861in}}{\pgfqpoint{0.732217in}{2.188961in}}{\pgfqpoint{0.738040in}{2.183137in}}%
\pgfpathcurveto{\pgfqpoint{0.743864in}{2.177313in}}{\pgfqpoint{0.751764in}{2.174041in}}{\pgfqpoint{0.760001in}{2.174041in}}%
\pgfpathclose%
\pgfusepath{stroke,fill}%
\end{pgfscope}%
\begin{pgfscope}%
\pgfpathrectangle{\pgfqpoint{0.100000in}{0.212622in}}{\pgfqpoint{3.696000in}{3.696000in}}%
\pgfusepath{clip}%
\pgfsetbuttcap%
\pgfsetroundjoin%
\definecolor{currentfill}{rgb}{0.121569,0.466667,0.705882}%
\pgfsetfillcolor{currentfill}%
\pgfsetfillopacity{0.786057}%
\pgfsetlinewidth{1.003750pt}%
\definecolor{currentstroke}{rgb}{0.121569,0.466667,0.705882}%
\pgfsetstrokecolor{currentstroke}%
\pgfsetstrokeopacity{0.786057}%
\pgfsetdash{}{0pt}%
\pgfpathmoveto{\pgfqpoint{0.760001in}{2.174041in}}%
\pgfpathcurveto{\pgfqpoint{0.768237in}{2.174041in}}{\pgfqpoint{0.776137in}{2.177313in}}{\pgfqpoint{0.781961in}{2.183137in}}%
\pgfpathcurveto{\pgfqpoint{0.787785in}{2.188961in}}{\pgfqpoint{0.791057in}{2.196861in}}{\pgfqpoint{0.791057in}{2.205098in}}%
\pgfpathcurveto{\pgfqpoint{0.791057in}{2.213334in}}{\pgfqpoint{0.787785in}{2.221234in}}{\pgfqpoint{0.781961in}{2.227058in}}%
\pgfpathcurveto{\pgfqpoint{0.776137in}{2.232882in}}{\pgfqpoint{0.768237in}{2.236154in}}{\pgfqpoint{0.760001in}{2.236154in}}%
\pgfpathcurveto{\pgfqpoint{0.751764in}{2.236154in}}{\pgfqpoint{0.743864in}{2.232882in}}{\pgfqpoint{0.738040in}{2.227058in}}%
\pgfpathcurveto{\pgfqpoint{0.732217in}{2.221234in}}{\pgfqpoint{0.728944in}{2.213334in}}{\pgfqpoint{0.728944in}{2.205098in}}%
\pgfpathcurveto{\pgfqpoint{0.728944in}{2.196861in}}{\pgfqpoint{0.732217in}{2.188961in}}{\pgfqpoint{0.738040in}{2.183137in}}%
\pgfpathcurveto{\pgfqpoint{0.743864in}{2.177313in}}{\pgfqpoint{0.751764in}{2.174041in}}{\pgfqpoint{0.760001in}{2.174041in}}%
\pgfpathclose%
\pgfusepath{stroke,fill}%
\end{pgfscope}%
\begin{pgfscope}%
\pgfpathrectangle{\pgfqpoint{0.100000in}{0.212622in}}{\pgfqpoint{3.696000in}{3.696000in}}%
\pgfusepath{clip}%
\pgfsetbuttcap%
\pgfsetroundjoin%
\definecolor{currentfill}{rgb}{0.121569,0.466667,0.705882}%
\pgfsetfillcolor{currentfill}%
\pgfsetfillopacity{0.786088}%
\pgfsetlinewidth{1.003750pt}%
\definecolor{currentstroke}{rgb}{0.121569,0.466667,0.705882}%
\pgfsetstrokecolor{currentstroke}%
\pgfsetstrokeopacity{0.786088}%
\pgfsetdash{}{0pt}%
\pgfpathmoveto{\pgfqpoint{0.758913in}{2.174396in}}%
\pgfpathcurveto{\pgfqpoint{0.767149in}{2.174396in}}{\pgfqpoint{0.775049in}{2.177668in}}{\pgfqpoint{0.780873in}{2.183492in}}%
\pgfpathcurveto{\pgfqpoint{0.786697in}{2.189316in}}{\pgfqpoint{0.789969in}{2.197216in}}{\pgfqpoint{0.789969in}{2.205452in}}%
\pgfpathcurveto{\pgfqpoint{0.789969in}{2.213688in}}{\pgfqpoint{0.786697in}{2.221588in}}{\pgfqpoint{0.780873in}{2.227412in}}%
\pgfpathcurveto{\pgfqpoint{0.775049in}{2.233236in}}{\pgfqpoint{0.767149in}{2.236509in}}{\pgfqpoint{0.758913in}{2.236509in}}%
\pgfpathcurveto{\pgfqpoint{0.750676in}{2.236509in}}{\pgfqpoint{0.742776in}{2.233236in}}{\pgfqpoint{0.736952in}{2.227412in}}%
\pgfpathcurveto{\pgfqpoint{0.731129in}{2.221588in}}{\pgfqpoint{0.727856in}{2.213688in}}{\pgfqpoint{0.727856in}{2.205452in}}%
\pgfpathcurveto{\pgfqpoint{0.727856in}{2.197216in}}{\pgfqpoint{0.731129in}{2.189316in}}{\pgfqpoint{0.736952in}{2.183492in}}%
\pgfpathcurveto{\pgfqpoint{0.742776in}{2.177668in}}{\pgfqpoint{0.750676in}{2.174396in}}{\pgfqpoint{0.758913in}{2.174396in}}%
\pgfpathclose%
\pgfusepath{stroke,fill}%
\end{pgfscope}%
\begin{pgfscope}%
\pgfpathrectangle{\pgfqpoint{0.100000in}{0.212622in}}{\pgfqpoint{3.696000in}{3.696000in}}%
\pgfusepath{clip}%
\pgfsetbuttcap%
\pgfsetroundjoin%
\definecolor{currentfill}{rgb}{0.121569,0.466667,0.705882}%
\pgfsetfillcolor{currentfill}%
\pgfsetfillopacity{0.788159}%
\pgfsetlinewidth{1.003750pt}%
\definecolor{currentstroke}{rgb}{0.121569,0.466667,0.705882}%
\pgfsetstrokecolor{currentstroke}%
\pgfsetstrokeopacity{0.788159}%
\pgfsetdash{}{0pt}%
\pgfpathmoveto{\pgfqpoint{2.851218in}{1.824175in}}%
\pgfpathcurveto{\pgfqpoint{2.859454in}{1.824175in}}{\pgfqpoint{2.867354in}{1.827448in}}{\pgfqpoint{2.873178in}{1.833272in}}%
\pgfpathcurveto{\pgfqpoint{2.879002in}{1.839096in}}{\pgfqpoint{2.882274in}{1.846996in}}{\pgfqpoint{2.882274in}{1.855232in}}%
\pgfpathcurveto{\pgfqpoint{2.882274in}{1.863468in}}{\pgfqpoint{2.879002in}{1.871368in}}{\pgfqpoint{2.873178in}{1.877192in}}%
\pgfpathcurveto{\pgfqpoint{2.867354in}{1.883016in}}{\pgfqpoint{2.859454in}{1.886288in}}{\pgfqpoint{2.851218in}{1.886288in}}%
\pgfpathcurveto{\pgfqpoint{2.842981in}{1.886288in}}{\pgfqpoint{2.835081in}{1.883016in}}{\pgfqpoint{2.829257in}{1.877192in}}%
\pgfpathcurveto{\pgfqpoint{2.823433in}{1.871368in}}{\pgfqpoint{2.820161in}{1.863468in}}{\pgfqpoint{2.820161in}{1.855232in}}%
\pgfpathcurveto{\pgfqpoint{2.820161in}{1.846996in}}{\pgfqpoint{2.823433in}{1.839096in}}{\pgfqpoint{2.829257in}{1.833272in}}%
\pgfpathcurveto{\pgfqpoint{2.835081in}{1.827448in}}{\pgfqpoint{2.842981in}{1.824175in}}{\pgfqpoint{2.851218in}{1.824175in}}%
\pgfpathclose%
\pgfusepath{stroke,fill}%
\end{pgfscope}%
\begin{pgfscope}%
\pgfpathrectangle{\pgfqpoint{0.100000in}{0.212622in}}{\pgfqpoint{3.696000in}{3.696000in}}%
\pgfusepath{clip}%
\pgfsetbuttcap%
\pgfsetroundjoin%
\definecolor{currentfill}{rgb}{0.121569,0.466667,0.705882}%
\pgfsetfillcolor{currentfill}%
\pgfsetfillopacity{0.791694}%
\pgfsetlinewidth{1.003750pt}%
\definecolor{currentstroke}{rgb}{0.121569,0.466667,0.705882}%
\pgfsetstrokecolor{currentstroke}%
\pgfsetstrokeopacity{0.791694}%
\pgfsetdash{}{0pt}%
\pgfpathmoveto{\pgfqpoint{2.842574in}{1.826401in}}%
\pgfpathcurveto{\pgfqpoint{2.850810in}{1.826401in}}{\pgfqpoint{2.858710in}{1.829674in}}{\pgfqpoint{2.864534in}{1.835498in}}%
\pgfpathcurveto{\pgfqpoint{2.870358in}{1.841322in}}{\pgfqpoint{2.873630in}{1.849222in}}{\pgfqpoint{2.873630in}{1.857458in}}%
\pgfpathcurveto{\pgfqpoint{2.873630in}{1.865694in}}{\pgfqpoint{2.870358in}{1.873594in}}{\pgfqpoint{2.864534in}{1.879418in}}%
\pgfpathcurveto{\pgfqpoint{2.858710in}{1.885242in}}{\pgfqpoint{2.850810in}{1.888514in}}{\pgfqpoint{2.842574in}{1.888514in}}%
\pgfpathcurveto{\pgfqpoint{2.834337in}{1.888514in}}{\pgfqpoint{2.826437in}{1.885242in}}{\pgfqpoint{2.820613in}{1.879418in}}%
\pgfpathcurveto{\pgfqpoint{2.814789in}{1.873594in}}{\pgfqpoint{2.811517in}{1.865694in}}{\pgfqpoint{2.811517in}{1.857458in}}%
\pgfpathcurveto{\pgfqpoint{2.811517in}{1.849222in}}{\pgfqpoint{2.814789in}{1.841322in}}{\pgfqpoint{2.820613in}{1.835498in}}%
\pgfpathcurveto{\pgfqpoint{2.826437in}{1.829674in}}{\pgfqpoint{2.834337in}{1.826401in}}{\pgfqpoint{2.842574in}{1.826401in}}%
\pgfpathclose%
\pgfusepath{stroke,fill}%
\end{pgfscope}%
\begin{pgfscope}%
\pgfpathrectangle{\pgfqpoint{0.100000in}{0.212622in}}{\pgfqpoint{3.696000in}{3.696000in}}%
\pgfusepath{clip}%
\pgfsetbuttcap%
\pgfsetroundjoin%
\definecolor{currentfill}{rgb}{0.121569,0.466667,0.705882}%
\pgfsetfillcolor{currentfill}%
\pgfsetfillopacity{0.793705}%
\pgfsetlinewidth{1.003750pt}%
\definecolor{currentstroke}{rgb}{0.121569,0.466667,0.705882}%
\pgfsetstrokecolor{currentstroke}%
\pgfsetstrokeopacity{0.793705}%
\pgfsetdash{}{0pt}%
\pgfpathmoveto{\pgfqpoint{2.838349in}{1.827306in}}%
\pgfpathcurveto{\pgfqpoint{2.846585in}{1.827306in}}{\pgfqpoint{2.854485in}{1.830578in}}{\pgfqpoint{2.860309in}{1.836402in}}%
\pgfpathcurveto{\pgfqpoint{2.866133in}{1.842226in}}{\pgfqpoint{2.869405in}{1.850126in}}{\pgfqpoint{2.869405in}{1.858363in}}%
\pgfpathcurveto{\pgfqpoint{2.869405in}{1.866599in}}{\pgfqpoint{2.866133in}{1.874499in}}{\pgfqpoint{2.860309in}{1.880323in}}%
\pgfpathcurveto{\pgfqpoint{2.854485in}{1.886147in}}{\pgfqpoint{2.846585in}{1.889419in}}{\pgfqpoint{2.838349in}{1.889419in}}%
\pgfpathcurveto{\pgfqpoint{2.830113in}{1.889419in}}{\pgfqpoint{2.822213in}{1.886147in}}{\pgfqpoint{2.816389in}{1.880323in}}%
\pgfpathcurveto{\pgfqpoint{2.810565in}{1.874499in}}{\pgfqpoint{2.807292in}{1.866599in}}{\pgfqpoint{2.807292in}{1.858363in}}%
\pgfpathcurveto{\pgfqpoint{2.807292in}{1.850126in}}{\pgfqpoint{2.810565in}{1.842226in}}{\pgfqpoint{2.816389in}{1.836402in}}%
\pgfpathcurveto{\pgfqpoint{2.822213in}{1.830578in}}{\pgfqpoint{2.830113in}{1.827306in}}{\pgfqpoint{2.838349in}{1.827306in}}%
\pgfpathclose%
\pgfusepath{stroke,fill}%
\end{pgfscope}%
\begin{pgfscope}%
\pgfpathrectangle{\pgfqpoint{0.100000in}{0.212622in}}{\pgfqpoint{3.696000in}{3.696000in}}%
\pgfusepath{clip}%
\pgfsetbuttcap%
\pgfsetroundjoin%
\definecolor{currentfill}{rgb}{0.121569,0.466667,0.705882}%
\pgfsetfillcolor{currentfill}%
\pgfsetfillopacity{0.796208}%
\pgfsetlinewidth{1.003750pt}%
\definecolor{currentstroke}{rgb}{0.121569,0.466667,0.705882}%
\pgfsetstrokecolor{currentstroke}%
\pgfsetstrokeopacity{0.796208}%
\pgfsetdash{}{0pt}%
\pgfpathmoveto{\pgfqpoint{2.835554in}{1.827619in}}%
\pgfpathcurveto{\pgfqpoint{2.843791in}{1.827619in}}{\pgfqpoint{2.851691in}{1.830892in}}{\pgfqpoint{2.857515in}{1.836716in}}%
\pgfpathcurveto{\pgfqpoint{2.863338in}{1.842540in}}{\pgfqpoint{2.866611in}{1.850440in}}{\pgfqpoint{2.866611in}{1.858676in}}%
\pgfpathcurveto{\pgfqpoint{2.866611in}{1.866912in}}{\pgfqpoint{2.863338in}{1.874812in}}{\pgfqpoint{2.857515in}{1.880636in}}%
\pgfpathcurveto{\pgfqpoint{2.851691in}{1.886460in}}{\pgfqpoint{2.843791in}{1.889732in}}{\pgfqpoint{2.835554in}{1.889732in}}%
\pgfpathcurveto{\pgfqpoint{2.827318in}{1.889732in}}{\pgfqpoint{2.819418in}{1.886460in}}{\pgfqpoint{2.813594in}{1.880636in}}%
\pgfpathcurveto{\pgfqpoint{2.807770in}{1.874812in}}{\pgfqpoint{2.804498in}{1.866912in}}{\pgfqpoint{2.804498in}{1.858676in}}%
\pgfpathcurveto{\pgfqpoint{2.804498in}{1.850440in}}{\pgfqpoint{2.807770in}{1.842540in}}{\pgfqpoint{2.813594in}{1.836716in}}%
\pgfpathcurveto{\pgfqpoint{2.819418in}{1.830892in}}{\pgfqpoint{2.827318in}{1.827619in}}{\pgfqpoint{2.835554in}{1.827619in}}%
\pgfpathclose%
\pgfusepath{stroke,fill}%
\end{pgfscope}%
\begin{pgfscope}%
\pgfpathrectangle{\pgfqpoint{0.100000in}{0.212622in}}{\pgfqpoint{3.696000in}{3.696000in}}%
\pgfusepath{clip}%
\pgfsetbuttcap%
\pgfsetroundjoin%
\definecolor{currentfill}{rgb}{0.121569,0.466667,0.705882}%
\pgfsetfillcolor{currentfill}%
\pgfsetfillopacity{0.799249}%
\pgfsetlinewidth{1.003750pt}%
\definecolor{currentstroke}{rgb}{0.121569,0.466667,0.705882}%
\pgfsetstrokecolor{currentstroke}%
\pgfsetstrokeopacity{0.799249}%
\pgfsetdash{}{0pt}%
\pgfpathmoveto{\pgfqpoint{2.829348in}{1.829056in}}%
\pgfpathcurveto{\pgfqpoint{2.837584in}{1.829056in}}{\pgfqpoint{2.845484in}{1.832329in}}{\pgfqpoint{2.851308in}{1.838153in}}%
\pgfpathcurveto{\pgfqpoint{2.857132in}{1.843977in}}{\pgfqpoint{2.860404in}{1.851877in}}{\pgfqpoint{2.860404in}{1.860113in}}%
\pgfpathcurveto{\pgfqpoint{2.860404in}{1.868349in}}{\pgfqpoint{2.857132in}{1.876249in}}{\pgfqpoint{2.851308in}{1.882073in}}%
\pgfpathcurveto{\pgfqpoint{2.845484in}{1.887897in}}{\pgfqpoint{2.837584in}{1.891169in}}{\pgfqpoint{2.829348in}{1.891169in}}%
\pgfpathcurveto{\pgfqpoint{2.821112in}{1.891169in}}{\pgfqpoint{2.813212in}{1.887897in}}{\pgfqpoint{2.807388in}{1.882073in}}%
\pgfpathcurveto{\pgfqpoint{2.801564in}{1.876249in}}{\pgfqpoint{2.798291in}{1.868349in}}{\pgfqpoint{2.798291in}{1.860113in}}%
\pgfpathcurveto{\pgfqpoint{2.798291in}{1.851877in}}{\pgfqpoint{2.801564in}{1.843977in}}{\pgfqpoint{2.807388in}{1.838153in}}%
\pgfpathcurveto{\pgfqpoint{2.813212in}{1.832329in}}{\pgfqpoint{2.821112in}{1.829056in}}{\pgfqpoint{2.829348in}{1.829056in}}%
\pgfpathclose%
\pgfusepath{stroke,fill}%
\end{pgfscope}%
\begin{pgfscope}%
\pgfpathrectangle{\pgfqpoint{0.100000in}{0.212622in}}{\pgfqpoint{3.696000in}{3.696000in}}%
\pgfusepath{clip}%
\pgfsetbuttcap%
\pgfsetroundjoin%
\definecolor{currentfill}{rgb}{0.121569,0.466667,0.705882}%
\pgfsetfillcolor{currentfill}%
\pgfsetfillopacity{0.802281}%
\pgfsetlinewidth{1.003750pt}%
\definecolor{currentstroke}{rgb}{0.121569,0.466667,0.705882}%
\pgfsetstrokecolor{currentstroke}%
\pgfsetstrokeopacity{0.802281}%
\pgfsetdash{}{0pt}%
\pgfpathmoveto{\pgfqpoint{2.821866in}{1.831085in}}%
\pgfpathcurveto{\pgfqpoint{2.830103in}{1.831085in}}{\pgfqpoint{2.838003in}{1.834357in}}{\pgfqpoint{2.843827in}{1.840181in}}%
\pgfpathcurveto{\pgfqpoint{2.849650in}{1.846005in}}{\pgfqpoint{2.852923in}{1.853905in}}{\pgfqpoint{2.852923in}{1.862141in}}%
\pgfpathcurveto{\pgfqpoint{2.852923in}{1.870377in}}{\pgfqpoint{2.849650in}{1.878277in}}{\pgfqpoint{2.843827in}{1.884101in}}%
\pgfpathcurveto{\pgfqpoint{2.838003in}{1.889925in}}{\pgfqpoint{2.830103in}{1.893198in}}{\pgfqpoint{2.821866in}{1.893198in}}%
\pgfpathcurveto{\pgfqpoint{2.813630in}{1.893198in}}{\pgfqpoint{2.805730in}{1.889925in}}{\pgfqpoint{2.799906in}{1.884101in}}%
\pgfpathcurveto{\pgfqpoint{2.794082in}{1.878277in}}{\pgfqpoint{2.790810in}{1.870377in}}{\pgfqpoint{2.790810in}{1.862141in}}%
\pgfpathcurveto{\pgfqpoint{2.790810in}{1.853905in}}{\pgfqpoint{2.794082in}{1.846005in}}{\pgfqpoint{2.799906in}{1.840181in}}%
\pgfpathcurveto{\pgfqpoint{2.805730in}{1.834357in}}{\pgfqpoint{2.813630in}{1.831085in}}{\pgfqpoint{2.821866in}{1.831085in}}%
\pgfpathclose%
\pgfusepath{stroke,fill}%
\end{pgfscope}%
\begin{pgfscope}%
\pgfpathrectangle{\pgfqpoint{0.100000in}{0.212622in}}{\pgfqpoint{3.696000in}{3.696000in}}%
\pgfusepath{clip}%
\pgfsetbuttcap%
\pgfsetroundjoin%
\definecolor{currentfill}{rgb}{0.121569,0.466667,0.705882}%
\pgfsetfillcolor{currentfill}%
\pgfsetfillopacity{0.806233}%
\pgfsetlinewidth{1.003750pt}%
\definecolor{currentstroke}{rgb}{0.121569,0.466667,0.705882}%
\pgfsetstrokecolor{currentstroke}%
\pgfsetstrokeopacity{0.806233}%
\pgfsetdash{}{0pt}%
\pgfpathmoveto{\pgfqpoint{2.817250in}{1.831728in}}%
\pgfpathcurveto{\pgfqpoint{2.825486in}{1.831728in}}{\pgfqpoint{2.833386in}{1.835000in}}{\pgfqpoint{2.839210in}{1.840824in}}%
\pgfpathcurveto{\pgfqpoint{2.845034in}{1.846648in}}{\pgfqpoint{2.848306in}{1.854548in}}{\pgfqpoint{2.848306in}{1.862784in}}%
\pgfpathcurveto{\pgfqpoint{2.848306in}{1.871021in}}{\pgfqpoint{2.845034in}{1.878921in}}{\pgfqpoint{2.839210in}{1.884745in}}%
\pgfpathcurveto{\pgfqpoint{2.833386in}{1.890569in}}{\pgfqpoint{2.825486in}{1.893841in}}{\pgfqpoint{2.817250in}{1.893841in}}%
\pgfpathcurveto{\pgfqpoint{2.809013in}{1.893841in}}{\pgfqpoint{2.801113in}{1.890569in}}{\pgfqpoint{2.795290in}{1.884745in}}%
\pgfpathcurveto{\pgfqpoint{2.789466in}{1.878921in}}{\pgfqpoint{2.786193in}{1.871021in}}{\pgfqpoint{2.786193in}{1.862784in}}%
\pgfpathcurveto{\pgfqpoint{2.786193in}{1.854548in}}{\pgfqpoint{2.789466in}{1.846648in}}{\pgfqpoint{2.795290in}{1.840824in}}%
\pgfpathcurveto{\pgfqpoint{2.801113in}{1.835000in}}{\pgfqpoint{2.809013in}{1.831728in}}{\pgfqpoint{2.817250in}{1.831728in}}%
\pgfpathclose%
\pgfusepath{stroke,fill}%
\end{pgfscope}%
\begin{pgfscope}%
\pgfpathrectangle{\pgfqpoint{0.100000in}{0.212622in}}{\pgfqpoint{3.696000in}{3.696000in}}%
\pgfusepath{clip}%
\pgfsetbuttcap%
\pgfsetroundjoin%
\definecolor{currentfill}{rgb}{0.121569,0.466667,0.705882}%
\pgfsetfillcolor{currentfill}%
\pgfsetfillopacity{0.810111}%
\pgfsetlinewidth{1.003750pt}%
\definecolor{currentstroke}{rgb}{0.121569,0.466667,0.705882}%
\pgfsetstrokecolor{currentstroke}%
\pgfsetstrokeopacity{0.810111}%
\pgfsetdash{}{0pt}%
\pgfpathmoveto{\pgfqpoint{2.809822in}{1.833220in}}%
\pgfpathcurveto{\pgfqpoint{2.818058in}{1.833220in}}{\pgfqpoint{2.825958in}{1.836492in}}{\pgfqpoint{2.831782in}{1.842316in}}%
\pgfpathcurveto{\pgfqpoint{2.837606in}{1.848140in}}{\pgfqpoint{2.840878in}{1.856040in}}{\pgfqpoint{2.840878in}{1.864276in}}%
\pgfpathcurveto{\pgfqpoint{2.840878in}{1.872513in}}{\pgfqpoint{2.837606in}{1.880413in}}{\pgfqpoint{2.831782in}{1.886237in}}%
\pgfpathcurveto{\pgfqpoint{2.825958in}{1.892061in}}{\pgfqpoint{2.818058in}{1.895333in}}{\pgfqpoint{2.809822in}{1.895333in}}%
\pgfpathcurveto{\pgfqpoint{2.801585in}{1.895333in}}{\pgfqpoint{2.793685in}{1.892061in}}{\pgfqpoint{2.787861in}{1.886237in}}%
\pgfpathcurveto{\pgfqpoint{2.782037in}{1.880413in}}{\pgfqpoint{2.778765in}{1.872513in}}{\pgfqpoint{2.778765in}{1.864276in}}%
\pgfpathcurveto{\pgfqpoint{2.778765in}{1.856040in}}{\pgfqpoint{2.782037in}{1.848140in}}{\pgfqpoint{2.787861in}{1.842316in}}%
\pgfpathcurveto{\pgfqpoint{2.793685in}{1.836492in}}{\pgfqpoint{2.801585in}{1.833220in}}{\pgfqpoint{2.809822in}{1.833220in}}%
\pgfpathclose%
\pgfusepath{stroke,fill}%
\end{pgfscope}%
\begin{pgfscope}%
\pgfpathrectangle{\pgfqpoint{0.100000in}{0.212622in}}{\pgfqpoint{3.696000in}{3.696000in}}%
\pgfusepath{clip}%
\pgfsetbuttcap%
\pgfsetroundjoin%
\definecolor{currentfill}{rgb}{0.121569,0.466667,0.705882}%
\pgfsetfillcolor{currentfill}%
\pgfsetfillopacity{0.813858}%
\pgfsetlinewidth{1.003750pt}%
\definecolor{currentstroke}{rgb}{0.121569,0.466667,0.705882}%
\pgfsetstrokecolor{currentstroke}%
\pgfsetstrokeopacity{0.813858}%
\pgfsetdash{}{0pt}%
\pgfpathmoveto{\pgfqpoint{2.799799in}{1.836027in}}%
\pgfpathcurveto{\pgfqpoint{2.808035in}{1.836027in}}{\pgfqpoint{2.815935in}{1.839299in}}{\pgfqpoint{2.821759in}{1.845123in}}%
\pgfpathcurveto{\pgfqpoint{2.827583in}{1.850947in}}{\pgfqpoint{2.830855in}{1.858847in}}{\pgfqpoint{2.830855in}{1.867084in}}%
\pgfpathcurveto{\pgfqpoint{2.830855in}{1.875320in}}{\pgfqpoint{2.827583in}{1.883220in}}{\pgfqpoint{2.821759in}{1.889044in}}%
\pgfpathcurveto{\pgfqpoint{2.815935in}{1.894868in}}{\pgfqpoint{2.808035in}{1.898140in}}{\pgfqpoint{2.799799in}{1.898140in}}%
\pgfpathcurveto{\pgfqpoint{2.791563in}{1.898140in}}{\pgfqpoint{2.783663in}{1.894868in}}{\pgfqpoint{2.777839in}{1.889044in}}%
\pgfpathcurveto{\pgfqpoint{2.772015in}{1.883220in}}{\pgfqpoint{2.768742in}{1.875320in}}{\pgfqpoint{2.768742in}{1.867084in}}%
\pgfpathcurveto{\pgfqpoint{2.768742in}{1.858847in}}{\pgfqpoint{2.772015in}{1.850947in}}{\pgfqpoint{2.777839in}{1.845123in}}%
\pgfpathcurveto{\pgfqpoint{2.783663in}{1.839299in}}{\pgfqpoint{2.791563in}{1.836027in}}{\pgfqpoint{2.799799in}{1.836027in}}%
\pgfpathclose%
\pgfusepath{stroke,fill}%
\end{pgfscope}%
\begin{pgfscope}%
\pgfpathrectangle{\pgfqpoint{0.100000in}{0.212622in}}{\pgfqpoint{3.696000in}{3.696000in}}%
\pgfusepath{clip}%
\pgfsetbuttcap%
\pgfsetroundjoin%
\definecolor{currentfill}{rgb}{0.121569,0.466667,0.705882}%
\pgfsetfillcolor{currentfill}%
\pgfsetfillopacity{0.817320}%
\pgfsetlinewidth{1.003750pt}%
\definecolor{currentstroke}{rgb}{0.121569,0.466667,0.705882}%
\pgfsetstrokecolor{currentstroke}%
\pgfsetstrokeopacity{0.817320}%
\pgfsetdash{}{0pt}%
\pgfpathmoveto{\pgfqpoint{0.530113in}{2.580869in}}%
\pgfpathcurveto{\pgfqpoint{0.538349in}{2.580869in}}{\pgfqpoint{0.546249in}{2.584141in}}{\pgfqpoint{0.552073in}{2.589965in}}%
\pgfpathcurveto{\pgfqpoint{0.557897in}{2.595789in}}{\pgfqpoint{0.561169in}{2.603689in}}{\pgfqpoint{0.561169in}{2.611926in}}%
\pgfpathcurveto{\pgfqpoint{0.561169in}{2.620162in}}{\pgfqpoint{0.557897in}{2.628062in}}{\pgfqpoint{0.552073in}{2.633886in}}%
\pgfpathcurveto{\pgfqpoint{0.546249in}{2.639710in}}{\pgfqpoint{0.538349in}{2.642982in}}{\pgfqpoint{0.530113in}{2.642982in}}%
\pgfpathcurveto{\pgfqpoint{0.521876in}{2.642982in}}{\pgfqpoint{0.513976in}{2.639710in}}{\pgfqpoint{0.508152in}{2.633886in}}%
\pgfpathcurveto{\pgfqpoint{0.502328in}{2.628062in}}{\pgfqpoint{0.499056in}{2.620162in}}{\pgfqpoint{0.499056in}{2.611926in}}%
\pgfpathcurveto{\pgfqpoint{0.499056in}{2.603689in}}{\pgfqpoint{0.502328in}{2.595789in}}{\pgfqpoint{0.508152in}{2.589965in}}%
\pgfpathcurveto{\pgfqpoint{0.513976in}{2.584141in}}{\pgfqpoint{0.521876in}{2.580869in}}{\pgfqpoint{0.530113in}{2.580869in}}%
\pgfpathclose%
\pgfusepath{stroke,fill}%
\end{pgfscope}%
\begin{pgfscope}%
\pgfpathrectangle{\pgfqpoint{0.100000in}{0.212622in}}{\pgfqpoint{3.696000in}{3.696000in}}%
\pgfusepath{clip}%
\pgfsetbuttcap%
\pgfsetroundjoin%
\definecolor{currentfill}{rgb}{0.121569,0.466667,0.705882}%
\pgfsetfillcolor{currentfill}%
\pgfsetfillopacity{0.818024}%
\pgfsetlinewidth{1.003750pt}%
\definecolor{currentstroke}{rgb}{0.121569,0.466667,0.705882}%
\pgfsetstrokecolor{currentstroke}%
\pgfsetstrokeopacity{0.818024}%
\pgfsetdash{}{0pt}%
\pgfpathmoveto{\pgfqpoint{0.531780in}{2.580530in}}%
\pgfpathcurveto{\pgfqpoint{0.540016in}{2.580530in}}{\pgfqpoint{0.547916in}{2.583803in}}{\pgfqpoint{0.553740in}{2.589627in}}%
\pgfpathcurveto{\pgfqpoint{0.559564in}{2.595451in}}{\pgfqpoint{0.562837in}{2.603351in}}{\pgfqpoint{0.562837in}{2.611587in}}%
\pgfpathcurveto{\pgfqpoint{0.562837in}{2.619823in}}{\pgfqpoint{0.559564in}{2.627723in}}{\pgfqpoint{0.553740in}{2.633547in}}%
\pgfpathcurveto{\pgfqpoint{0.547916in}{2.639371in}}{\pgfqpoint{0.540016in}{2.642643in}}{\pgfqpoint{0.531780in}{2.642643in}}%
\pgfpathcurveto{\pgfqpoint{0.523544in}{2.642643in}}{\pgfqpoint{0.515644in}{2.639371in}}{\pgfqpoint{0.509820in}{2.633547in}}%
\pgfpathcurveto{\pgfqpoint{0.503996in}{2.627723in}}{\pgfqpoint{0.500724in}{2.619823in}}{\pgfqpoint{0.500724in}{2.611587in}}%
\pgfpathcurveto{\pgfqpoint{0.500724in}{2.603351in}}{\pgfqpoint{0.503996in}{2.595451in}}{\pgfqpoint{0.509820in}{2.589627in}}%
\pgfpathcurveto{\pgfqpoint{0.515644in}{2.583803in}}{\pgfqpoint{0.523544in}{2.580530in}}{\pgfqpoint{0.531780in}{2.580530in}}%
\pgfpathclose%
\pgfusepath{stroke,fill}%
\end{pgfscope}%
\begin{pgfscope}%
\pgfpathrectangle{\pgfqpoint{0.100000in}{0.212622in}}{\pgfqpoint{3.696000in}{3.696000in}}%
\pgfusepath{clip}%
\pgfsetbuttcap%
\pgfsetroundjoin%
\definecolor{currentfill}{rgb}{0.121569,0.466667,0.705882}%
\pgfsetfillcolor{currentfill}%
\pgfsetfillopacity{0.818554}%
\pgfsetlinewidth{1.003750pt}%
\definecolor{currentstroke}{rgb}{0.121569,0.466667,0.705882}%
\pgfsetstrokecolor{currentstroke}%
\pgfsetstrokeopacity{0.818554}%
\pgfsetdash{}{0pt}%
\pgfpathmoveto{\pgfqpoint{0.533088in}{2.580014in}}%
\pgfpathcurveto{\pgfqpoint{0.541324in}{2.580014in}}{\pgfqpoint{0.549224in}{2.583286in}}{\pgfqpoint{0.555048in}{2.589110in}}%
\pgfpathcurveto{\pgfqpoint{0.560872in}{2.594934in}}{\pgfqpoint{0.564144in}{2.602834in}}{\pgfqpoint{0.564144in}{2.611070in}}%
\pgfpathcurveto{\pgfqpoint{0.564144in}{2.619307in}}{\pgfqpoint{0.560872in}{2.627207in}}{\pgfqpoint{0.555048in}{2.633030in}}%
\pgfpathcurveto{\pgfqpoint{0.549224in}{2.638854in}}{\pgfqpoint{0.541324in}{2.642127in}}{\pgfqpoint{0.533088in}{2.642127in}}%
\pgfpathcurveto{\pgfqpoint{0.524851in}{2.642127in}}{\pgfqpoint{0.516951in}{2.638854in}}{\pgfqpoint{0.511127in}{2.633030in}}%
\pgfpathcurveto{\pgfqpoint{0.505303in}{2.627207in}}{\pgfqpoint{0.502031in}{2.619307in}}{\pgfqpoint{0.502031in}{2.611070in}}%
\pgfpathcurveto{\pgfqpoint{0.502031in}{2.602834in}}{\pgfqpoint{0.505303in}{2.594934in}}{\pgfqpoint{0.511127in}{2.589110in}}%
\pgfpathcurveto{\pgfqpoint{0.516951in}{2.583286in}}{\pgfqpoint{0.524851in}{2.580014in}}{\pgfqpoint{0.533088in}{2.580014in}}%
\pgfpathclose%
\pgfusepath{stroke,fill}%
\end{pgfscope}%
\begin{pgfscope}%
\pgfpathrectangle{\pgfqpoint{0.100000in}{0.212622in}}{\pgfqpoint{3.696000in}{3.696000in}}%
\pgfusepath{clip}%
\pgfsetbuttcap%
\pgfsetroundjoin%
\definecolor{currentfill}{rgb}{0.121569,0.466667,0.705882}%
\pgfsetfillcolor{currentfill}%
\pgfsetfillopacity{0.818837}%
\pgfsetlinewidth{1.003750pt}%
\definecolor{currentstroke}{rgb}{0.121569,0.466667,0.705882}%
\pgfsetstrokecolor{currentstroke}%
\pgfsetstrokeopacity{0.818837}%
\pgfsetdash{}{0pt}%
\pgfpathmoveto{\pgfqpoint{0.533877in}{2.579691in}}%
\pgfpathcurveto{\pgfqpoint{0.542113in}{2.579691in}}{\pgfqpoint{0.550014in}{2.582963in}}{\pgfqpoint{0.555837in}{2.588787in}}%
\pgfpathcurveto{\pgfqpoint{0.561661in}{2.594611in}}{\pgfqpoint{0.564934in}{2.602511in}}{\pgfqpoint{0.564934in}{2.610747in}}%
\pgfpathcurveto{\pgfqpoint{0.564934in}{2.618983in}}{\pgfqpoint{0.561661in}{2.626883in}}{\pgfqpoint{0.555837in}{2.632707in}}%
\pgfpathcurveto{\pgfqpoint{0.550014in}{2.638531in}}{\pgfqpoint{0.542113in}{2.641804in}}{\pgfqpoint{0.533877in}{2.641804in}}%
\pgfpathcurveto{\pgfqpoint{0.525641in}{2.641804in}}{\pgfqpoint{0.517741in}{2.638531in}}{\pgfqpoint{0.511917in}{2.632707in}}%
\pgfpathcurveto{\pgfqpoint{0.506093in}{2.626883in}}{\pgfqpoint{0.502821in}{2.618983in}}{\pgfqpoint{0.502821in}{2.610747in}}%
\pgfpathcurveto{\pgfqpoint{0.502821in}{2.602511in}}{\pgfqpoint{0.506093in}{2.594611in}}{\pgfqpoint{0.511917in}{2.588787in}}%
\pgfpathcurveto{\pgfqpoint{0.517741in}{2.582963in}}{\pgfqpoint{0.525641in}{2.579691in}}{\pgfqpoint{0.533877in}{2.579691in}}%
\pgfpathclose%
\pgfusepath{stroke,fill}%
\end{pgfscope}%
\begin{pgfscope}%
\pgfpathrectangle{\pgfqpoint{0.100000in}{0.212622in}}{\pgfqpoint{3.696000in}{3.696000in}}%
\pgfusepath{clip}%
\pgfsetbuttcap%
\pgfsetroundjoin%
\definecolor{currentfill}{rgb}{0.121569,0.466667,0.705882}%
\pgfsetfillcolor{currentfill}%
\pgfsetfillopacity{0.819274}%
\pgfsetlinewidth{1.003750pt}%
\definecolor{currentstroke}{rgb}{0.121569,0.466667,0.705882}%
\pgfsetstrokecolor{currentstroke}%
\pgfsetstrokeopacity{0.819274}%
\pgfsetdash{}{0pt}%
\pgfpathmoveto{\pgfqpoint{2.795067in}{1.836354in}}%
\pgfpathcurveto{\pgfqpoint{2.803304in}{1.836354in}}{\pgfqpoint{2.811204in}{1.839626in}}{\pgfqpoint{2.817028in}{1.845450in}}%
\pgfpathcurveto{\pgfqpoint{2.822851in}{1.851274in}}{\pgfqpoint{2.826124in}{1.859174in}}{\pgfqpoint{2.826124in}{1.867410in}}%
\pgfpathcurveto{\pgfqpoint{2.826124in}{1.875646in}}{\pgfqpoint{2.822851in}{1.883547in}}{\pgfqpoint{2.817028in}{1.889370in}}%
\pgfpathcurveto{\pgfqpoint{2.811204in}{1.895194in}}{\pgfqpoint{2.803304in}{1.898467in}}{\pgfqpoint{2.795067in}{1.898467in}}%
\pgfpathcurveto{\pgfqpoint{2.786831in}{1.898467in}}{\pgfqpoint{2.778931in}{1.895194in}}{\pgfqpoint{2.773107in}{1.889370in}}%
\pgfpathcurveto{\pgfqpoint{2.767283in}{1.883547in}}{\pgfqpoint{2.764011in}{1.875646in}}{\pgfqpoint{2.764011in}{1.867410in}}%
\pgfpathcurveto{\pgfqpoint{2.764011in}{1.859174in}}{\pgfqpoint{2.767283in}{1.851274in}}{\pgfqpoint{2.773107in}{1.845450in}}%
\pgfpathcurveto{\pgfqpoint{2.778931in}{1.839626in}}{\pgfqpoint{2.786831in}{1.836354in}}{\pgfqpoint{2.795067in}{1.836354in}}%
\pgfpathclose%
\pgfusepath{stroke,fill}%
\end{pgfscope}%
\begin{pgfscope}%
\pgfpathrectangle{\pgfqpoint{0.100000in}{0.212622in}}{\pgfqpoint{3.696000in}{3.696000in}}%
\pgfusepath{clip}%
\pgfsetbuttcap%
\pgfsetroundjoin%
\definecolor{currentfill}{rgb}{0.121569,0.466667,0.705882}%
\pgfsetfillcolor{currentfill}%
\pgfsetfillopacity{0.819428}%
\pgfsetlinewidth{1.003750pt}%
\definecolor{currentstroke}{rgb}{0.121569,0.466667,0.705882}%
\pgfsetstrokecolor{currentstroke}%
\pgfsetstrokeopacity{0.819428}%
\pgfsetdash{}{0pt}%
\pgfpathmoveto{\pgfqpoint{0.535145in}{2.579452in}}%
\pgfpathcurveto{\pgfqpoint{0.543382in}{2.579452in}}{\pgfqpoint{0.551282in}{2.582724in}}{\pgfqpoint{0.557106in}{2.588548in}}%
\pgfpathcurveto{\pgfqpoint{0.562930in}{2.594372in}}{\pgfqpoint{0.566202in}{2.602272in}}{\pgfqpoint{0.566202in}{2.610508in}}%
\pgfpathcurveto{\pgfqpoint{0.566202in}{2.618745in}}{\pgfqpoint{0.562930in}{2.626645in}}{\pgfqpoint{0.557106in}{2.632469in}}%
\pgfpathcurveto{\pgfqpoint{0.551282in}{2.638292in}}{\pgfqpoint{0.543382in}{2.641565in}}{\pgfqpoint{0.535145in}{2.641565in}}%
\pgfpathcurveto{\pgfqpoint{0.526909in}{2.641565in}}{\pgfqpoint{0.519009in}{2.638292in}}{\pgfqpoint{0.513185in}{2.632469in}}%
\pgfpathcurveto{\pgfqpoint{0.507361in}{2.626645in}}{\pgfqpoint{0.504089in}{2.618745in}}{\pgfqpoint{0.504089in}{2.610508in}}%
\pgfpathcurveto{\pgfqpoint{0.504089in}{2.602272in}}{\pgfqpoint{0.507361in}{2.594372in}}{\pgfqpoint{0.513185in}{2.588548in}}%
\pgfpathcurveto{\pgfqpoint{0.519009in}{2.582724in}}{\pgfqpoint{0.526909in}{2.579452in}}{\pgfqpoint{0.535145in}{2.579452in}}%
\pgfpathclose%
\pgfusepath{stroke,fill}%
\end{pgfscope}%
\begin{pgfscope}%
\pgfpathrectangle{\pgfqpoint{0.100000in}{0.212622in}}{\pgfqpoint{3.696000in}{3.696000in}}%
\pgfusepath{clip}%
\pgfsetbuttcap%
\pgfsetroundjoin%
\definecolor{currentfill}{rgb}{0.121569,0.466667,0.705882}%
\pgfsetfillcolor{currentfill}%
\pgfsetfillopacity{0.819695}%
\pgfsetlinewidth{1.003750pt}%
\definecolor{currentstroke}{rgb}{0.121569,0.466667,0.705882}%
\pgfsetstrokecolor{currentstroke}%
\pgfsetstrokeopacity{0.819695}%
\pgfsetdash{}{0pt}%
\pgfpathmoveto{\pgfqpoint{0.536311in}{2.579170in}}%
\pgfpathcurveto{\pgfqpoint{0.544547in}{2.579170in}}{\pgfqpoint{0.552447in}{2.582442in}}{\pgfqpoint{0.558271in}{2.588266in}}%
\pgfpathcurveto{\pgfqpoint{0.564095in}{2.594090in}}{\pgfqpoint{0.567367in}{2.601990in}}{\pgfqpoint{0.567367in}{2.610226in}}%
\pgfpathcurveto{\pgfqpoint{0.567367in}{2.618462in}}{\pgfqpoint{0.564095in}{2.626363in}}{\pgfqpoint{0.558271in}{2.632186in}}%
\pgfpathcurveto{\pgfqpoint{0.552447in}{2.638010in}}{\pgfqpoint{0.544547in}{2.641283in}}{\pgfqpoint{0.536311in}{2.641283in}}%
\pgfpathcurveto{\pgfqpoint{0.528075in}{2.641283in}}{\pgfqpoint{0.520175in}{2.638010in}}{\pgfqpoint{0.514351in}{2.632186in}}%
\pgfpathcurveto{\pgfqpoint{0.508527in}{2.626363in}}{\pgfqpoint{0.505254in}{2.618462in}}{\pgfqpoint{0.505254in}{2.610226in}}%
\pgfpathcurveto{\pgfqpoint{0.505254in}{2.601990in}}{\pgfqpoint{0.508527in}{2.594090in}}{\pgfqpoint{0.514351in}{2.588266in}}%
\pgfpathcurveto{\pgfqpoint{0.520175in}{2.582442in}}{\pgfqpoint{0.528075in}{2.579170in}}{\pgfqpoint{0.536311in}{2.579170in}}%
\pgfpathclose%
\pgfusepath{stroke,fill}%
\end{pgfscope}%
\begin{pgfscope}%
\pgfpathrectangle{\pgfqpoint{0.100000in}{0.212622in}}{\pgfqpoint{3.696000in}{3.696000in}}%
\pgfusepath{clip}%
\pgfsetbuttcap%
\pgfsetroundjoin%
\definecolor{currentfill}{rgb}{0.121569,0.466667,0.705882}%
\pgfsetfillcolor{currentfill}%
\pgfsetfillopacity{0.819966}%
\pgfsetlinewidth{1.003750pt}%
\definecolor{currentstroke}{rgb}{0.121569,0.466667,0.705882}%
\pgfsetstrokecolor{currentstroke}%
\pgfsetstrokeopacity{0.819966}%
\pgfsetdash{}{0pt}%
\pgfpathmoveto{\pgfqpoint{0.538843in}{2.578071in}}%
\pgfpathcurveto{\pgfqpoint{0.547079in}{2.578071in}}{\pgfqpoint{0.554979in}{2.581343in}}{\pgfqpoint{0.560803in}{2.587167in}}%
\pgfpathcurveto{\pgfqpoint{0.566627in}{2.592991in}}{\pgfqpoint{0.569900in}{2.600891in}}{\pgfqpoint{0.569900in}{2.609128in}}%
\pgfpathcurveto{\pgfqpoint{0.569900in}{2.617364in}}{\pgfqpoint{0.566627in}{2.625264in}}{\pgfqpoint{0.560803in}{2.631088in}}%
\pgfpathcurveto{\pgfqpoint{0.554979in}{2.636912in}}{\pgfqpoint{0.547079in}{2.640184in}}{\pgfqpoint{0.538843in}{2.640184in}}%
\pgfpathcurveto{\pgfqpoint{0.530607in}{2.640184in}}{\pgfqpoint{0.522707in}{2.636912in}}{\pgfqpoint{0.516883in}{2.631088in}}%
\pgfpathcurveto{\pgfqpoint{0.511059in}{2.625264in}}{\pgfqpoint{0.507787in}{2.617364in}}{\pgfqpoint{0.507787in}{2.609128in}}%
\pgfpathcurveto{\pgfqpoint{0.507787in}{2.600891in}}{\pgfqpoint{0.511059in}{2.592991in}}{\pgfqpoint{0.516883in}{2.587167in}}%
\pgfpathcurveto{\pgfqpoint{0.522707in}{2.581343in}}{\pgfqpoint{0.530607in}{2.578071in}}{\pgfqpoint{0.538843in}{2.578071in}}%
\pgfpathclose%
\pgfusepath{stroke,fill}%
\end{pgfscope}%
\begin{pgfscope}%
\pgfpathrectangle{\pgfqpoint{0.100000in}{0.212622in}}{\pgfqpoint{3.696000in}{3.696000in}}%
\pgfusepath{clip}%
\pgfsetbuttcap%
\pgfsetroundjoin%
\definecolor{currentfill}{rgb}{0.121569,0.466667,0.705882}%
\pgfsetfillcolor{currentfill}%
\pgfsetfillopacity{0.820093}%
\pgfsetlinewidth{1.003750pt}%
\definecolor{currentstroke}{rgb}{0.121569,0.466667,0.705882}%
\pgfsetstrokecolor{currentstroke}%
\pgfsetstrokeopacity{0.820093}%
\pgfsetdash{}{0pt}%
\pgfpathmoveto{\pgfqpoint{0.540140in}{2.577523in}}%
\pgfpathcurveto{\pgfqpoint{0.548377in}{2.577523in}}{\pgfqpoint{0.556277in}{2.580796in}}{\pgfqpoint{0.562100in}{2.586620in}}%
\pgfpathcurveto{\pgfqpoint{0.567924in}{2.592444in}}{\pgfqpoint{0.571197in}{2.600344in}}{\pgfqpoint{0.571197in}{2.608580in}}%
\pgfpathcurveto{\pgfqpoint{0.571197in}{2.616816in}}{\pgfqpoint{0.567924in}{2.624716in}}{\pgfqpoint{0.562100in}{2.630540in}}%
\pgfpathcurveto{\pgfqpoint{0.556277in}{2.636364in}}{\pgfqpoint{0.548377in}{2.639636in}}{\pgfqpoint{0.540140in}{2.639636in}}%
\pgfpathcurveto{\pgfqpoint{0.531904in}{2.639636in}}{\pgfqpoint{0.524004in}{2.636364in}}{\pgfqpoint{0.518180in}{2.630540in}}%
\pgfpathcurveto{\pgfqpoint{0.512356in}{2.624716in}}{\pgfqpoint{0.509084in}{2.616816in}}{\pgfqpoint{0.509084in}{2.608580in}}%
\pgfpathcurveto{\pgfqpoint{0.509084in}{2.600344in}}{\pgfqpoint{0.512356in}{2.592444in}}{\pgfqpoint{0.518180in}{2.586620in}}%
\pgfpathcurveto{\pgfqpoint{0.524004in}{2.580796in}}{\pgfqpoint{0.531904in}{2.577523in}}{\pgfqpoint{0.540140in}{2.577523in}}%
\pgfpathclose%
\pgfusepath{stroke,fill}%
\end{pgfscope}%
\begin{pgfscope}%
\pgfpathrectangle{\pgfqpoint{0.100000in}{0.212622in}}{\pgfqpoint{3.696000in}{3.696000in}}%
\pgfusepath{clip}%
\pgfsetbuttcap%
\pgfsetroundjoin%
\definecolor{currentfill}{rgb}{0.121569,0.466667,0.705882}%
\pgfsetfillcolor{currentfill}%
\pgfsetfillopacity{0.820340}%
\pgfsetlinewidth{1.003750pt}%
\definecolor{currentstroke}{rgb}{0.121569,0.466667,0.705882}%
\pgfsetstrokecolor{currentstroke}%
\pgfsetstrokeopacity{0.820340}%
\pgfsetdash{}{0pt}%
\pgfpathmoveto{\pgfqpoint{0.542513in}{2.576711in}}%
\pgfpathcurveto{\pgfqpoint{0.550749in}{2.576711in}}{\pgfqpoint{0.558649in}{2.579984in}}{\pgfqpoint{0.564473in}{2.585808in}}%
\pgfpathcurveto{\pgfqpoint{0.570297in}{2.591631in}}{\pgfqpoint{0.573570in}{2.599531in}}{\pgfqpoint{0.573570in}{2.607768in}}%
\pgfpathcurveto{\pgfqpoint{0.573570in}{2.616004in}}{\pgfqpoint{0.570297in}{2.623904in}}{\pgfqpoint{0.564473in}{2.629728in}}%
\pgfpathcurveto{\pgfqpoint{0.558649in}{2.635552in}}{\pgfqpoint{0.550749in}{2.638824in}}{\pgfqpoint{0.542513in}{2.638824in}}%
\pgfpathcurveto{\pgfqpoint{0.534277in}{2.638824in}}{\pgfqpoint{0.526377in}{2.635552in}}{\pgfqpoint{0.520553in}{2.629728in}}%
\pgfpathcurveto{\pgfqpoint{0.514729in}{2.623904in}}{\pgfqpoint{0.511457in}{2.616004in}}{\pgfqpoint{0.511457in}{2.607768in}}%
\pgfpathcurveto{\pgfqpoint{0.511457in}{2.599531in}}{\pgfqpoint{0.514729in}{2.591631in}}{\pgfqpoint{0.520553in}{2.585808in}}%
\pgfpathcurveto{\pgfqpoint{0.526377in}{2.579984in}}{\pgfqpoint{0.534277in}{2.576711in}}{\pgfqpoint{0.542513in}{2.576711in}}%
\pgfpathclose%
\pgfusepath{stroke,fill}%
\end{pgfscope}%
\begin{pgfscope}%
\pgfpathrectangle{\pgfqpoint{0.100000in}{0.212622in}}{\pgfqpoint{3.696000in}{3.696000in}}%
\pgfusepath{clip}%
\pgfsetbuttcap%
\pgfsetroundjoin%
\definecolor{currentfill}{rgb}{0.121569,0.466667,0.705882}%
\pgfsetfillcolor{currentfill}%
\pgfsetfillopacity{0.820646}%
\pgfsetlinewidth{1.003750pt}%
\definecolor{currentstroke}{rgb}{0.121569,0.466667,0.705882}%
\pgfsetstrokecolor{currentstroke}%
\pgfsetstrokeopacity{0.820646}%
\pgfsetdash{}{0pt}%
\pgfpathmoveto{\pgfqpoint{0.547064in}{2.575018in}}%
\pgfpathcurveto{\pgfqpoint{0.555301in}{2.575018in}}{\pgfqpoint{0.563201in}{2.578291in}}{\pgfqpoint{0.569025in}{2.584115in}}%
\pgfpathcurveto{\pgfqpoint{0.574849in}{2.589939in}}{\pgfqpoint{0.578121in}{2.597839in}}{\pgfqpoint{0.578121in}{2.606075in}}%
\pgfpathcurveto{\pgfqpoint{0.578121in}{2.614311in}}{\pgfqpoint{0.574849in}{2.622211in}}{\pgfqpoint{0.569025in}{2.628035in}}%
\pgfpathcurveto{\pgfqpoint{0.563201in}{2.633859in}}{\pgfqpoint{0.555301in}{2.637131in}}{\pgfqpoint{0.547064in}{2.637131in}}%
\pgfpathcurveto{\pgfqpoint{0.538828in}{2.637131in}}{\pgfqpoint{0.530928in}{2.633859in}}{\pgfqpoint{0.525104in}{2.628035in}}%
\pgfpathcurveto{\pgfqpoint{0.519280in}{2.622211in}}{\pgfqpoint{0.516008in}{2.614311in}}{\pgfqpoint{0.516008in}{2.606075in}}%
\pgfpathcurveto{\pgfqpoint{0.516008in}{2.597839in}}{\pgfqpoint{0.519280in}{2.589939in}}{\pgfqpoint{0.525104in}{2.584115in}}%
\pgfpathcurveto{\pgfqpoint{0.530928in}{2.578291in}}{\pgfqpoint{0.538828in}{2.575018in}}{\pgfqpoint{0.547064in}{2.575018in}}%
\pgfpathclose%
\pgfusepath{stroke,fill}%
\end{pgfscope}%
\begin{pgfscope}%
\pgfpathrectangle{\pgfqpoint{0.100000in}{0.212622in}}{\pgfqpoint{3.696000in}{3.696000in}}%
\pgfusepath{clip}%
\pgfsetbuttcap%
\pgfsetroundjoin%
\definecolor{currentfill}{rgb}{0.121569,0.466667,0.705882}%
\pgfsetfillcolor{currentfill}%
\pgfsetfillopacity{0.821467}%
\pgfsetlinewidth{1.003750pt}%
\definecolor{currentstroke}{rgb}{0.121569,0.466667,0.705882}%
\pgfsetstrokecolor{currentstroke}%
\pgfsetstrokeopacity{0.821467}%
\pgfsetdash{}{0pt}%
\pgfpathmoveto{\pgfqpoint{0.554858in}{2.572164in}}%
\pgfpathcurveto{\pgfqpoint{0.563094in}{2.572164in}}{\pgfqpoint{0.570994in}{2.575436in}}{\pgfqpoint{0.576818in}{2.581260in}}%
\pgfpathcurveto{\pgfqpoint{0.582642in}{2.587084in}}{\pgfqpoint{0.585914in}{2.594984in}}{\pgfqpoint{0.585914in}{2.603220in}}%
\pgfpathcurveto{\pgfqpoint{0.585914in}{2.611456in}}{\pgfqpoint{0.582642in}{2.619356in}}{\pgfqpoint{0.576818in}{2.625180in}}%
\pgfpathcurveto{\pgfqpoint{0.570994in}{2.631004in}}{\pgfqpoint{0.563094in}{2.634277in}}{\pgfqpoint{0.554858in}{2.634277in}}%
\pgfpathcurveto{\pgfqpoint{0.546622in}{2.634277in}}{\pgfqpoint{0.538722in}{2.631004in}}{\pgfqpoint{0.532898in}{2.625180in}}%
\pgfpathcurveto{\pgfqpoint{0.527074in}{2.619356in}}{\pgfqpoint{0.523801in}{2.611456in}}{\pgfqpoint{0.523801in}{2.603220in}}%
\pgfpathcurveto{\pgfqpoint{0.523801in}{2.594984in}}{\pgfqpoint{0.527074in}{2.587084in}}{\pgfqpoint{0.532898in}{2.581260in}}%
\pgfpathcurveto{\pgfqpoint{0.538722in}{2.575436in}}{\pgfqpoint{0.546622in}{2.572164in}}{\pgfqpoint{0.554858in}{2.572164in}}%
\pgfpathclose%
\pgfusepath{stroke,fill}%
\end{pgfscope}%
\begin{pgfscope}%
\pgfpathrectangle{\pgfqpoint{0.100000in}{0.212622in}}{\pgfqpoint{3.696000in}{3.696000in}}%
\pgfusepath{clip}%
\pgfsetbuttcap%
\pgfsetroundjoin%
\definecolor{currentfill}{rgb}{0.121569,0.466667,0.705882}%
\pgfsetfillcolor{currentfill}%
\pgfsetfillopacity{0.822022}%
\pgfsetlinewidth{1.003750pt}%
\definecolor{currentstroke}{rgb}{0.121569,0.466667,0.705882}%
\pgfsetstrokecolor{currentstroke}%
\pgfsetstrokeopacity{0.822022}%
\pgfsetdash{}{0pt}%
\pgfpathmoveto{\pgfqpoint{2.789644in}{1.837384in}}%
\pgfpathcurveto{\pgfqpoint{2.797881in}{1.837384in}}{\pgfqpoint{2.805781in}{1.840656in}}{\pgfqpoint{2.811605in}{1.846480in}}%
\pgfpathcurveto{\pgfqpoint{2.817429in}{1.852304in}}{\pgfqpoint{2.820701in}{1.860204in}}{\pgfqpoint{2.820701in}{1.868441in}}%
\pgfpathcurveto{\pgfqpoint{2.820701in}{1.876677in}}{\pgfqpoint{2.817429in}{1.884577in}}{\pgfqpoint{2.811605in}{1.890401in}}%
\pgfpathcurveto{\pgfqpoint{2.805781in}{1.896225in}}{\pgfqpoint{2.797881in}{1.899497in}}{\pgfqpoint{2.789644in}{1.899497in}}%
\pgfpathcurveto{\pgfqpoint{2.781408in}{1.899497in}}{\pgfqpoint{2.773508in}{1.896225in}}{\pgfqpoint{2.767684in}{1.890401in}}%
\pgfpathcurveto{\pgfqpoint{2.761860in}{1.884577in}}{\pgfqpoint{2.758588in}{1.876677in}}{\pgfqpoint{2.758588in}{1.868441in}}%
\pgfpathcurveto{\pgfqpoint{2.758588in}{1.860204in}}{\pgfqpoint{2.761860in}{1.852304in}}{\pgfqpoint{2.767684in}{1.846480in}}%
\pgfpathcurveto{\pgfqpoint{2.773508in}{1.840656in}}{\pgfqpoint{2.781408in}{1.837384in}}{\pgfqpoint{2.789644in}{1.837384in}}%
\pgfpathclose%
\pgfusepath{stroke,fill}%
\end{pgfscope}%
\begin{pgfscope}%
\pgfpathrectangle{\pgfqpoint{0.100000in}{0.212622in}}{\pgfqpoint{3.696000in}{3.696000in}}%
\pgfusepath{clip}%
\pgfsetbuttcap%
\pgfsetroundjoin%
\definecolor{currentfill}{rgb}{0.121569,0.466667,0.705882}%
\pgfsetfillcolor{currentfill}%
\pgfsetfillopacity{0.822723}%
\pgfsetlinewidth{1.003750pt}%
\definecolor{currentstroke}{rgb}{0.121569,0.466667,0.705882}%
\pgfsetstrokecolor{currentstroke}%
\pgfsetstrokeopacity{0.822723}%
\pgfsetdash{}{0pt}%
\pgfpathmoveto{\pgfqpoint{0.569641in}{2.567404in}}%
\pgfpathcurveto{\pgfqpoint{0.577877in}{2.567404in}}{\pgfqpoint{0.585777in}{2.570676in}}{\pgfqpoint{0.591601in}{2.576500in}}%
\pgfpathcurveto{\pgfqpoint{0.597425in}{2.582324in}}{\pgfqpoint{0.600697in}{2.590224in}}{\pgfqpoint{0.600697in}{2.598460in}}%
\pgfpathcurveto{\pgfqpoint{0.600697in}{2.606697in}}{\pgfqpoint{0.597425in}{2.614597in}}{\pgfqpoint{0.591601in}{2.620421in}}%
\pgfpathcurveto{\pgfqpoint{0.585777in}{2.626245in}}{\pgfqpoint{0.577877in}{2.629517in}}{\pgfqpoint{0.569641in}{2.629517in}}%
\pgfpathcurveto{\pgfqpoint{0.561404in}{2.629517in}}{\pgfqpoint{0.553504in}{2.626245in}}{\pgfqpoint{0.547680in}{2.620421in}}%
\pgfpathcurveto{\pgfqpoint{0.541856in}{2.614597in}}{\pgfqpoint{0.538584in}{2.606697in}}{\pgfqpoint{0.538584in}{2.598460in}}%
\pgfpathcurveto{\pgfqpoint{0.538584in}{2.590224in}}{\pgfqpoint{0.541856in}{2.582324in}}{\pgfqpoint{0.547680in}{2.576500in}}%
\pgfpathcurveto{\pgfqpoint{0.553504in}{2.570676in}}{\pgfqpoint{0.561404in}{2.567404in}}{\pgfqpoint{0.569641in}{2.567404in}}%
\pgfpathclose%
\pgfusepath{stroke,fill}%
\end{pgfscope}%
\begin{pgfscope}%
\pgfpathrectangle{\pgfqpoint{0.100000in}{0.212622in}}{\pgfqpoint{3.696000in}{3.696000in}}%
\pgfusepath{clip}%
\pgfsetbuttcap%
\pgfsetroundjoin%
\definecolor{currentfill}{rgb}{0.121569,0.466667,0.705882}%
\pgfsetfillcolor{currentfill}%
\pgfsetfillopacity{0.824499}%
\pgfsetlinewidth{1.003750pt}%
\definecolor{currentstroke}{rgb}{0.121569,0.466667,0.705882}%
\pgfsetstrokecolor{currentstroke}%
\pgfsetstrokeopacity{0.824499}%
\pgfsetdash{}{0pt}%
\pgfpathmoveto{\pgfqpoint{0.582912in}{2.563586in}}%
\pgfpathcurveto{\pgfqpoint{0.591149in}{2.563586in}}{\pgfqpoint{0.599049in}{2.566858in}}{\pgfqpoint{0.604873in}{2.572682in}}%
\pgfpathcurveto{\pgfqpoint{0.610697in}{2.578506in}}{\pgfqpoint{0.613969in}{2.586406in}}{\pgfqpoint{0.613969in}{2.594642in}}%
\pgfpathcurveto{\pgfqpoint{0.613969in}{2.602879in}}{\pgfqpoint{0.610697in}{2.610779in}}{\pgfqpoint{0.604873in}{2.616603in}}%
\pgfpathcurveto{\pgfqpoint{0.599049in}{2.622427in}}{\pgfqpoint{0.591149in}{2.625699in}}{\pgfqpoint{0.582912in}{2.625699in}}%
\pgfpathcurveto{\pgfqpoint{0.574676in}{2.625699in}}{\pgfqpoint{0.566776in}{2.622427in}}{\pgfqpoint{0.560952in}{2.616603in}}%
\pgfpathcurveto{\pgfqpoint{0.555128in}{2.610779in}}{\pgfqpoint{0.551856in}{2.602879in}}{\pgfqpoint{0.551856in}{2.594642in}}%
\pgfpathcurveto{\pgfqpoint{0.551856in}{2.586406in}}{\pgfqpoint{0.555128in}{2.578506in}}{\pgfqpoint{0.560952in}{2.572682in}}%
\pgfpathcurveto{\pgfqpoint{0.566776in}{2.566858in}}{\pgfqpoint{0.574676in}{2.563586in}}{\pgfqpoint{0.582912in}{2.563586in}}%
\pgfpathclose%
\pgfusepath{stroke,fill}%
\end{pgfscope}%
\begin{pgfscope}%
\pgfpathrectangle{\pgfqpoint{0.100000in}{0.212622in}}{\pgfqpoint{3.696000in}{3.696000in}}%
\pgfusepath{clip}%
\pgfsetbuttcap%
\pgfsetroundjoin%
\definecolor{currentfill}{rgb}{0.121569,0.466667,0.705882}%
\pgfsetfillcolor{currentfill}%
\pgfsetfillopacity{0.824759}%
\pgfsetlinewidth{1.003750pt}%
\definecolor{currentstroke}{rgb}{0.121569,0.466667,0.705882}%
\pgfsetstrokecolor{currentstroke}%
\pgfsetstrokeopacity{0.824759}%
\pgfsetdash{}{0pt}%
\pgfpathmoveto{\pgfqpoint{2.782379in}{1.839563in}}%
\pgfpathcurveto{\pgfqpoint{2.790615in}{1.839563in}}{\pgfqpoint{2.798515in}{1.842836in}}{\pgfqpoint{2.804339in}{1.848660in}}%
\pgfpathcurveto{\pgfqpoint{2.810163in}{1.854484in}}{\pgfqpoint{2.813435in}{1.862384in}}{\pgfqpoint{2.813435in}{1.870620in}}%
\pgfpathcurveto{\pgfqpoint{2.813435in}{1.878856in}}{\pgfqpoint{2.810163in}{1.886756in}}{\pgfqpoint{2.804339in}{1.892580in}}%
\pgfpathcurveto{\pgfqpoint{2.798515in}{1.898404in}}{\pgfqpoint{2.790615in}{1.901676in}}{\pgfqpoint{2.782379in}{1.901676in}}%
\pgfpathcurveto{\pgfqpoint{2.774142in}{1.901676in}}{\pgfqpoint{2.766242in}{1.898404in}}{\pgfqpoint{2.760418in}{1.892580in}}%
\pgfpathcurveto{\pgfqpoint{2.754594in}{1.886756in}}{\pgfqpoint{2.751322in}{1.878856in}}{\pgfqpoint{2.751322in}{1.870620in}}%
\pgfpathcurveto{\pgfqpoint{2.751322in}{1.862384in}}{\pgfqpoint{2.754594in}{1.854484in}}{\pgfqpoint{2.760418in}{1.848660in}}%
\pgfpathcurveto{\pgfqpoint{2.766242in}{1.842836in}}{\pgfqpoint{2.774142in}{1.839563in}}{\pgfqpoint{2.782379in}{1.839563in}}%
\pgfpathclose%
\pgfusepath{stroke,fill}%
\end{pgfscope}%
\begin{pgfscope}%
\pgfpathrectangle{\pgfqpoint{0.100000in}{0.212622in}}{\pgfqpoint{3.696000in}{3.696000in}}%
\pgfusepath{clip}%
\pgfsetbuttcap%
\pgfsetroundjoin%
\definecolor{currentfill}{rgb}{0.121569,0.466667,0.705882}%
\pgfsetfillcolor{currentfill}%
\pgfsetfillopacity{0.825788}%
\pgfsetlinewidth{1.003750pt}%
\definecolor{currentstroke}{rgb}{0.121569,0.466667,0.705882}%
\pgfsetstrokecolor{currentstroke}%
\pgfsetstrokeopacity{0.825788}%
\pgfsetdash{}{0pt}%
\pgfpathmoveto{\pgfqpoint{0.594556in}{2.559376in}}%
\pgfpathcurveto{\pgfqpoint{0.602792in}{2.559376in}}{\pgfqpoint{0.610692in}{2.562648in}}{\pgfqpoint{0.616516in}{2.568472in}}%
\pgfpathcurveto{\pgfqpoint{0.622340in}{2.574296in}}{\pgfqpoint{0.625612in}{2.582196in}}{\pgfqpoint{0.625612in}{2.590432in}}%
\pgfpathcurveto{\pgfqpoint{0.625612in}{2.598668in}}{\pgfqpoint{0.622340in}{2.606568in}}{\pgfqpoint{0.616516in}{2.612392in}}%
\pgfpathcurveto{\pgfqpoint{0.610692in}{2.618216in}}{\pgfqpoint{0.602792in}{2.621489in}}{\pgfqpoint{0.594556in}{2.621489in}}%
\pgfpathcurveto{\pgfqpoint{0.586319in}{2.621489in}}{\pgfqpoint{0.578419in}{2.618216in}}{\pgfqpoint{0.572595in}{2.612392in}}%
\pgfpathcurveto{\pgfqpoint{0.566771in}{2.606568in}}{\pgfqpoint{0.563499in}{2.598668in}}{\pgfqpoint{0.563499in}{2.590432in}}%
\pgfpathcurveto{\pgfqpoint{0.563499in}{2.582196in}}{\pgfqpoint{0.566771in}{2.574296in}}{\pgfqpoint{0.572595in}{2.568472in}}%
\pgfpathcurveto{\pgfqpoint{0.578419in}{2.562648in}}{\pgfqpoint{0.586319in}{2.559376in}}{\pgfqpoint{0.594556in}{2.559376in}}%
\pgfpathclose%
\pgfusepath{stroke,fill}%
\end{pgfscope}%
\begin{pgfscope}%
\pgfpathrectangle{\pgfqpoint{0.100000in}{0.212622in}}{\pgfqpoint{3.696000in}{3.696000in}}%
\pgfusepath{clip}%
\pgfsetbuttcap%
\pgfsetroundjoin%
\definecolor{currentfill}{rgb}{0.121569,0.466667,0.705882}%
\pgfsetfillcolor{currentfill}%
\pgfsetfillopacity{0.826975}%
\pgfsetlinewidth{1.003750pt}%
\definecolor{currentstroke}{rgb}{0.121569,0.466667,0.705882}%
\pgfsetstrokecolor{currentstroke}%
\pgfsetstrokeopacity{0.826975}%
\pgfsetdash{}{0pt}%
\pgfpathmoveto{\pgfqpoint{0.606017in}{2.556085in}}%
\pgfpathcurveto{\pgfqpoint{0.614254in}{2.556085in}}{\pgfqpoint{0.622154in}{2.559357in}}{\pgfqpoint{0.627978in}{2.565181in}}%
\pgfpathcurveto{\pgfqpoint{0.633802in}{2.571005in}}{\pgfqpoint{0.637074in}{2.578905in}}{\pgfqpoint{0.637074in}{2.587141in}}%
\pgfpathcurveto{\pgfqpoint{0.637074in}{2.595378in}}{\pgfqpoint{0.633802in}{2.603278in}}{\pgfqpoint{0.627978in}{2.609102in}}%
\pgfpathcurveto{\pgfqpoint{0.622154in}{2.614926in}}{\pgfqpoint{0.614254in}{2.618198in}}{\pgfqpoint{0.606017in}{2.618198in}}%
\pgfpathcurveto{\pgfqpoint{0.597781in}{2.618198in}}{\pgfqpoint{0.589881in}{2.614926in}}{\pgfqpoint{0.584057in}{2.609102in}}%
\pgfpathcurveto{\pgfqpoint{0.578233in}{2.603278in}}{\pgfqpoint{0.574961in}{2.595378in}}{\pgfqpoint{0.574961in}{2.587141in}}%
\pgfpathcurveto{\pgfqpoint{0.574961in}{2.578905in}}{\pgfqpoint{0.578233in}{2.571005in}}{\pgfqpoint{0.584057in}{2.565181in}}%
\pgfpathcurveto{\pgfqpoint{0.589881in}{2.559357in}}{\pgfqpoint{0.597781in}{2.556085in}}{\pgfqpoint{0.606017in}{2.556085in}}%
\pgfpathclose%
\pgfusepath{stroke,fill}%
\end{pgfscope}%
\begin{pgfscope}%
\pgfpathrectangle{\pgfqpoint{0.100000in}{0.212622in}}{\pgfqpoint{3.696000in}{3.696000in}}%
\pgfusepath{clip}%
\pgfsetbuttcap%
\pgfsetroundjoin%
\definecolor{currentfill}{rgb}{0.121569,0.466667,0.705882}%
\pgfsetfillcolor{currentfill}%
\pgfsetfillopacity{0.828025}%
\pgfsetlinewidth{1.003750pt}%
\definecolor{currentstroke}{rgb}{0.121569,0.466667,0.705882}%
\pgfsetstrokecolor{currentstroke}%
\pgfsetstrokeopacity{0.828025}%
\pgfsetdash{}{0pt}%
\pgfpathmoveto{\pgfqpoint{0.617214in}{2.552307in}}%
\pgfpathcurveto{\pgfqpoint{0.625451in}{2.552307in}}{\pgfqpoint{0.633351in}{2.555579in}}{\pgfqpoint{0.639175in}{2.561403in}}%
\pgfpathcurveto{\pgfqpoint{0.644998in}{2.567227in}}{\pgfqpoint{0.648271in}{2.575127in}}{\pgfqpoint{0.648271in}{2.583363in}}%
\pgfpathcurveto{\pgfqpoint{0.648271in}{2.591600in}}{\pgfqpoint{0.644998in}{2.599500in}}{\pgfqpoint{0.639175in}{2.605323in}}%
\pgfpathcurveto{\pgfqpoint{0.633351in}{2.611147in}}{\pgfqpoint{0.625451in}{2.614420in}}{\pgfqpoint{0.617214in}{2.614420in}}%
\pgfpathcurveto{\pgfqpoint{0.608978in}{2.614420in}}{\pgfqpoint{0.601078in}{2.611147in}}{\pgfqpoint{0.595254in}{2.605323in}}%
\pgfpathcurveto{\pgfqpoint{0.589430in}{2.599500in}}{\pgfqpoint{0.586158in}{2.591600in}}{\pgfqpoint{0.586158in}{2.583363in}}%
\pgfpathcurveto{\pgfqpoint{0.586158in}{2.575127in}}{\pgfqpoint{0.589430in}{2.567227in}}{\pgfqpoint{0.595254in}{2.561403in}}%
\pgfpathcurveto{\pgfqpoint{0.601078in}{2.555579in}}{\pgfqpoint{0.608978in}{2.552307in}}{\pgfqpoint{0.617214in}{2.552307in}}%
\pgfpathclose%
\pgfusepath{stroke,fill}%
\end{pgfscope}%
\begin{pgfscope}%
\pgfpathrectangle{\pgfqpoint{0.100000in}{0.212622in}}{\pgfqpoint{3.696000in}{3.696000in}}%
\pgfusepath{clip}%
\pgfsetbuttcap%
\pgfsetroundjoin%
\definecolor{currentfill}{rgb}{0.121569,0.466667,0.705882}%
\pgfsetfillcolor{currentfill}%
\pgfsetfillopacity{0.828870}%
\pgfsetlinewidth{1.003750pt}%
\definecolor{currentstroke}{rgb}{0.121569,0.466667,0.705882}%
\pgfsetstrokecolor{currentstroke}%
\pgfsetstrokeopacity{0.828870}%
\pgfsetdash{}{0pt}%
\pgfpathmoveto{\pgfqpoint{0.626515in}{2.546850in}}%
\pgfpathcurveto{\pgfqpoint{0.634752in}{2.546850in}}{\pgfqpoint{0.642652in}{2.550122in}}{\pgfqpoint{0.648476in}{2.555946in}}%
\pgfpathcurveto{\pgfqpoint{0.654300in}{2.561770in}}{\pgfqpoint{0.657572in}{2.569670in}}{\pgfqpoint{0.657572in}{2.577906in}}%
\pgfpathcurveto{\pgfqpoint{0.657572in}{2.586142in}}{\pgfqpoint{0.654300in}{2.594042in}}{\pgfqpoint{0.648476in}{2.599866in}}%
\pgfpathcurveto{\pgfqpoint{0.642652in}{2.605690in}}{\pgfqpoint{0.634752in}{2.608963in}}{\pgfqpoint{0.626515in}{2.608963in}}%
\pgfpathcurveto{\pgfqpoint{0.618279in}{2.608963in}}{\pgfqpoint{0.610379in}{2.605690in}}{\pgfqpoint{0.604555in}{2.599866in}}%
\pgfpathcurveto{\pgfqpoint{0.598731in}{2.594042in}}{\pgfqpoint{0.595459in}{2.586142in}}{\pgfqpoint{0.595459in}{2.577906in}}%
\pgfpathcurveto{\pgfqpoint{0.595459in}{2.569670in}}{\pgfqpoint{0.598731in}{2.561770in}}{\pgfqpoint{0.604555in}{2.555946in}}%
\pgfpathcurveto{\pgfqpoint{0.610379in}{2.550122in}}{\pgfqpoint{0.618279in}{2.546850in}}{\pgfqpoint{0.626515in}{2.546850in}}%
\pgfpathclose%
\pgfusepath{stroke,fill}%
\end{pgfscope}%
\begin{pgfscope}%
\pgfpathrectangle{\pgfqpoint{0.100000in}{0.212622in}}{\pgfqpoint{3.696000in}{3.696000in}}%
\pgfusepath{clip}%
\pgfsetbuttcap%
\pgfsetroundjoin%
\definecolor{currentfill}{rgb}{0.121569,0.466667,0.705882}%
\pgfsetfillcolor{currentfill}%
\pgfsetfillopacity{0.829059}%
\pgfsetlinewidth{1.003750pt}%
\definecolor{currentstroke}{rgb}{0.121569,0.466667,0.705882}%
\pgfsetstrokecolor{currentstroke}%
\pgfsetstrokeopacity{0.829059}%
\pgfsetdash{}{0pt}%
\pgfpathmoveto{\pgfqpoint{2.777926in}{1.839726in}}%
\pgfpathcurveto{\pgfqpoint{2.786163in}{1.839726in}}{\pgfqpoint{2.794063in}{1.842999in}}{\pgfqpoint{2.799886in}{1.848823in}}%
\pgfpathcurveto{\pgfqpoint{2.805710in}{1.854647in}}{\pgfqpoint{2.808983in}{1.862547in}}{\pgfqpoint{2.808983in}{1.870783in}}%
\pgfpathcurveto{\pgfqpoint{2.808983in}{1.879019in}}{\pgfqpoint{2.805710in}{1.886919in}}{\pgfqpoint{2.799886in}{1.892743in}}%
\pgfpathcurveto{\pgfqpoint{2.794063in}{1.898567in}}{\pgfqpoint{2.786163in}{1.901839in}}{\pgfqpoint{2.777926in}{1.901839in}}%
\pgfpathcurveto{\pgfqpoint{2.769690in}{1.901839in}}{\pgfqpoint{2.761790in}{1.898567in}}{\pgfqpoint{2.755966in}{1.892743in}}%
\pgfpathcurveto{\pgfqpoint{2.750142in}{1.886919in}}{\pgfqpoint{2.746870in}{1.879019in}}{\pgfqpoint{2.746870in}{1.870783in}}%
\pgfpathcurveto{\pgfqpoint{2.746870in}{1.862547in}}{\pgfqpoint{2.750142in}{1.854647in}}{\pgfqpoint{2.755966in}{1.848823in}}%
\pgfpathcurveto{\pgfqpoint{2.761790in}{1.842999in}}{\pgfqpoint{2.769690in}{1.839726in}}{\pgfqpoint{2.777926in}{1.839726in}}%
\pgfpathclose%
\pgfusepath{stroke,fill}%
\end{pgfscope}%
\begin{pgfscope}%
\pgfpathrectangle{\pgfqpoint{0.100000in}{0.212622in}}{\pgfqpoint{3.696000in}{3.696000in}}%
\pgfusepath{clip}%
\pgfsetbuttcap%
\pgfsetroundjoin%
\definecolor{currentfill}{rgb}{0.121569,0.466667,0.705882}%
\pgfsetfillcolor{currentfill}%
\pgfsetfillopacity{0.829678}%
\pgfsetlinewidth{1.003750pt}%
\definecolor{currentstroke}{rgb}{0.121569,0.466667,0.705882}%
\pgfsetstrokecolor{currentstroke}%
\pgfsetstrokeopacity{0.829678}%
\pgfsetdash{}{0pt}%
\pgfpathmoveto{\pgfqpoint{0.633985in}{2.545066in}}%
\pgfpathcurveto{\pgfqpoint{0.642221in}{2.545066in}}{\pgfqpoint{0.650121in}{2.548339in}}{\pgfqpoint{0.655945in}{2.554163in}}%
\pgfpathcurveto{\pgfqpoint{0.661769in}{2.559987in}}{\pgfqpoint{0.665041in}{2.567887in}}{\pgfqpoint{0.665041in}{2.576123in}}%
\pgfpathcurveto{\pgfqpoint{0.665041in}{2.584359in}}{\pgfqpoint{0.661769in}{2.592259in}}{\pgfqpoint{0.655945in}{2.598083in}}%
\pgfpathcurveto{\pgfqpoint{0.650121in}{2.603907in}}{\pgfqpoint{0.642221in}{2.607179in}}{\pgfqpoint{0.633985in}{2.607179in}}%
\pgfpathcurveto{\pgfqpoint{0.625748in}{2.607179in}}{\pgfqpoint{0.617848in}{2.603907in}}{\pgfqpoint{0.612025in}{2.598083in}}%
\pgfpathcurveto{\pgfqpoint{0.606201in}{2.592259in}}{\pgfqpoint{0.602928in}{2.584359in}}{\pgfqpoint{0.602928in}{2.576123in}}%
\pgfpathcurveto{\pgfqpoint{0.602928in}{2.567887in}}{\pgfqpoint{0.606201in}{2.559987in}}{\pgfqpoint{0.612025in}{2.554163in}}%
\pgfpathcurveto{\pgfqpoint{0.617848in}{2.548339in}}{\pgfqpoint{0.625748in}{2.545066in}}{\pgfqpoint{0.633985in}{2.545066in}}%
\pgfpathclose%
\pgfusepath{stroke,fill}%
\end{pgfscope}%
\begin{pgfscope}%
\pgfpathrectangle{\pgfqpoint{0.100000in}{0.212622in}}{\pgfqpoint{3.696000in}{3.696000in}}%
\pgfusepath{clip}%
\pgfsetbuttcap%
\pgfsetroundjoin%
\definecolor{currentfill}{rgb}{0.121569,0.466667,0.705882}%
\pgfsetfillcolor{currentfill}%
\pgfsetfillopacity{0.830285}%
\pgfsetlinewidth{1.003750pt}%
\definecolor{currentstroke}{rgb}{0.121569,0.466667,0.705882}%
\pgfsetstrokecolor{currentstroke}%
\pgfsetstrokeopacity{0.830285}%
\pgfsetdash{}{0pt}%
\pgfpathmoveto{\pgfqpoint{0.640098in}{2.542907in}}%
\pgfpathcurveto{\pgfqpoint{0.648334in}{2.542907in}}{\pgfqpoint{0.656234in}{2.546180in}}{\pgfqpoint{0.662058in}{2.552004in}}%
\pgfpathcurveto{\pgfqpoint{0.667882in}{2.557828in}}{\pgfqpoint{0.671154in}{2.565728in}}{\pgfqpoint{0.671154in}{2.573964in}}%
\pgfpathcurveto{\pgfqpoint{0.671154in}{2.582200in}}{\pgfqpoint{0.667882in}{2.590100in}}{\pgfqpoint{0.662058in}{2.595924in}}%
\pgfpathcurveto{\pgfqpoint{0.656234in}{2.601748in}}{\pgfqpoint{0.648334in}{2.605020in}}{\pgfqpoint{0.640098in}{2.605020in}}%
\pgfpathcurveto{\pgfqpoint{0.631861in}{2.605020in}}{\pgfqpoint{0.623961in}{2.601748in}}{\pgfqpoint{0.618137in}{2.595924in}}%
\pgfpathcurveto{\pgfqpoint{0.612313in}{2.590100in}}{\pgfqpoint{0.609041in}{2.582200in}}{\pgfqpoint{0.609041in}{2.573964in}}%
\pgfpathcurveto{\pgfqpoint{0.609041in}{2.565728in}}{\pgfqpoint{0.612313in}{2.557828in}}{\pgfqpoint{0.618137in}{2.552004in}}%
\pgfpathcurveto{\pgfqpoint{0.623961in}{2.546180in}}{\pgfqpoint{0.631861in}{2.542907in}}{\pgfqpoint{0.640098in}{2.542907in}}%
\pgfpathclose%
\pgfusepath{stroke,fill}%
\end{pgfscope}%
\begin{pgfscope}%
\pgfpathrectangle{\pgfqpoint{0.100000in}{0.212622in}}{\pgfqpoint{3.696000in}{3.696000in}}%
\pgfusepath{clip}%
\pgfsetbuttcap%
\pgfsetroundjoin%
\definecolor{currentfill}{rgb}{0.121569,0.466667,0.705882}%
\pgfsetfillcolor{currentfill}%
\pgfsetfillopacity{0.831073}%
\pgfsetlinewidth{1.003750pt}%
\definecolor{currentstroke}{rgb}{0.121569,0.466667,0.705882}%
\pgfsetstrokecolor{currentstroke}%
\pgfsetstrokeopacity{0.831073}%
\pgfsetdash{}{0pt}%
\pgfpathmoveto{\pgfqpoint{0.651644in}{2.538100in}}%
\pgfpathcurveto{\pgfqpoint{0.659880in}{2.538100in}}{\pgfqpoint{0.667780in}{2.541373in}}{\pgfqpoint{0.673604in}{2.547197in}}%
\pgfpathcurveto{\pgfqpoint{0.679428in}{2.553020in}}{\pgfqpoint{0.682700in}{2.560920in}}{\pgfqpoint{0.682700in}{2.569157in}}%
\pgfpathcurveto{\pgfqpoint{0.682700in}{2.577393in}}{\pgfqpoint{0.679428in}{2.585293in}}{\pgfqpoint{0.673604in}{2.591117in}}%
\pgfpathcurveto{\pgfqpoint{0.667780in}{2.596941in}}{\pgfqpoint{0.659880in}{2.600213in}}{\pgfqpoint{0.651644in}{2.600213in}}%
\pgfpathcurveto{\pgfqpoint{0.643407in}{2.600213in}}{\pgfqpoint{0.635507in}{2.596941in}}{\pgfqpoint{0.629683in}{2.591117in}}%
\pgfpathcurveto{\pgfqpoint{0.623859in}{2.585293in}}{\pgfqpoint{0.620587in}{2.577393in}}{\pgfqpoint{0.620587in}{2.569157in}}%
\pgfpathcurveto{\pgfqpoint{0.620587in}{2.560920in}}{\pgfqpoint{0.623859in}{2.553020in}}{\pgfqpoint{0.629683in}{2.547197in}}%
\pgfpathcurveto{\pgfqpoint{0.635507in}{2.541373in}}{\pgfqpoint{0.643407in}{2.538100in}}{\pgfqpoint{0.651644in}{2.538100in}}%
\pgfpathclose%
\pgfusepath{stroke,fill}%
\end{pgfscope}%
\begin{pgfscope}%
\pgfpathrectangle{\pgfqpoint{0.100000in}{0.212622in}}{\pgfqpoint{3.696000in}{3.696000in}}%
\pgfusepath{clip}%
\pgfsetbuttcap%
\pgfsetroundjoin%
\definecolor{currentfill}{rgb}{0.121569,0.466667,0.705882}%
\pgfsetfillcolor{currentfill}%
\pgfsetfillopacity{0.832090}%
\pgfsetlinewidth{1.003750pt}%
\definecolor{currentstroke}{rgb}{0.121569,0.466667,0.705882}%
\pgfsetstrokecolor{currentstroke}%
\pgfsetstrokeopacity{0.832090}%
\pgfsetdash{}{0pt}%
\pgfpathmoveto{\pgfqpoint{0.661011in}{2.535606in}}%
\pgfpathcurveto{\pgfqpoint{0.669248in}{2.535606in}}{\pgfqpoint{0.677148in}{2.538878in}}{\pgfqpoint{0.682972in}{2.544702in}}%
\pgfpathcurveto{\pgfqpoint{0.688796in}{2.550526in}}{\pgfqpoint{0.692068in}{2.558426in}}{\pgfqpoint{0.692068in}{2.566662in}}%
\pgfpathcurveto{\pgfqpoint{0.692068in}{2.574898in}}{\pgfqpoint{0.688796in}{2.582799in}}{\pgfqpoint{0.682972in}{2.588622in}}%
\pgfpathcurveto{\pgfqpoint{0.677148in}{2.594446in}}{\pgfqpoint{0.669248in}{2.597719in}}{\pgfqpoint{0.661011in}{2.597719in}}%
\pgfpathcurveto{\pgfqpoint{0.652775in}{2.597719in}}{\pgfqpoint{0.644875in}{2.594446in}}{\pgfqpoint{0.639051in}{2.588622in}}%
\pgfpathcurveto{\pgfqpoint{0.633227in}{2.582799in}}{\pgfqpoint{0.629955in}{2.574898in}}{\pgfqpoint{0.629955in}{2.566662in}}%
\pgfpathcurveto{\pgfqpoint{0.629955in}{2.558426in}}{\pgfqpoint{0.633227in}{2.550526in}}{\pgfqpoint{0.639051in}{2.544702in}}%
\pgfpathcurveto{\pgfqpoint{0.644875in}{2.538878in}}{\pgfqpoint{0.652775in}{2.535606in}}{\pgfqpoint{0.661011in}{2.535606in}}%
\pgfpathclose%
\pgfusepath{stroke,fill}%
\end{pgfscope}%
\begin{pgfscope}%
\pgfpathrectangle{\pgfqpoint{0.100000in}{0.212622in}}{\pgfqpoint{3.696000in}{3.696000in}}%
\pgfusepath{clip}%
\pgfsetbuttcap%
\pgfsetroundjoin%
\definecolor{currentfill}{rgb}{0.121569,0.466667,0.705882}%
\pgfsetfillcolor{currentfill}%
\pgfsetfillopacity{0.832850}%
\pgfsetlinewidth{1.003750pt}%
\definecolor{currentstroke}{rgb}{0.121569,0.466667,0.705882}%
\pgfsetstrokecolor{currentstroke}%
\pgfsetstrokeopacity{0.832850}%
\pgfsetdash{}{0pt}%
\pgfpathmoveto{\pgfqpoint{0.667970in}{2.533030in}}%
\pgfpathcurveto{\pgfqpoint{0.676206in}{2.533030in}}{\pgfqpoint{0.684106in}{2.536302in}}{\pgfqpoint{0.689930in}{2.542126in}}%
\pgfpathcurveto{\pgfqpoint{0.695754in}{2.547950in}}{\pgfqpoint{0.699026in}{2.555850in}}{\pgfqpoint{0.699026in}{2.564086in}}%
\pgfpathcurveto{\pgfqpoint{0.699026in}{2.572322in}}{\pgfqpoint{0.695754in}{2.580222in}}{\pgfqpoint{0.689930in}{2.586046in}}%
\pgfpathcurveto{\pgfqpoint{0.684106in}{2.591870in}}{\pgfqpoint{0.676206in}{2.595143in}}{\pgfqpoint{0.667970in}{2.595143in}}%
\pgfpathcurveto{\pgfqpoint{0.659734in}{2.595143in}}{\pgfqpoint{0.651833in}{2.591870in}}{\pgfqpoint{0.646010in}{2.586046in}}%
\pgfpathcurveto{\pgfqpoint{0.640186in}{2.580222in}}{\pgfqpoint{0.636913in}{2.572322in}}{\pgfqpoint{0.636913in}{2.564086in}}%
\pgfpathcurveto{\pgfqpoint{0.636913in}{2.555850in}}{\pgfqpoint{0.640186in}{2.547950in}}{\pgfqpoint{0.646010in}{2.542126in}}%
\pgfpathcurveto{\pgfqpoint{0.651833in}{2.536302in}}{\pgfqpoint{0.659734in}{2.533030in}}{\pgfqpoint{0.667970in}{2.533030in}}%
\pgfpathclose%
\pgfusepath{stroke,fill}%
\end{pgfscope}%
\begin{pgfscope}%
\pgfpathrectangle{\pgfqpoint{0.100000in}{0.212622in}}{\pgfqpoint{3.696000in}{3.696000in}}%
\pgfusepath{clip}%
\pgfsetbuttcap%
\pgfsetroundjoin%
\definecolor{currentfill}{rgb}{0.121569,0.466667,0.705882}%
\pgfsetfillcolor{currentfill}%
\pgfsetfillopacity{0.833240}%
\pgfsetlinewidth{1.003750pt}%
\definecolor{currentstroke}{rgb}{0.121569,0.466667,0.705882}%
\pgfsetstrokecolor{currentstroke}%
\pgfsetstrokeopacity{0.833240}%
\pgfsetdash{}{0pt}%
\pgfpathmoveto{\pgfqpoint{2.769883in}{1.840805in}}%
\pgfpathcurveto{\pgfqpoint{2.778119in}{1.840805in}}{\pgfqpoint{2.786019in}{1.844078in}}{\pgfqpoint{2.791843in}{1.849902in}}%
\pgfpathcurveto{\pgfqpoint{2.797667in}{1.855726in}}{\pgfqpoint{2.800939in}{1.863626in}}{\pgfqpoint{2.800939in}{1.871862in}}%
\pgfpathcurveto{\pgfqpoint{2.800939in}{1.880098in}}{\pgfqpoint{2.797667in}{1.887998in}}{\pgfqpoint{2.791843in}{1.893822in}}%
\pgfpathcurveto{\pgfqpoint{2.786019in}{1.899646in}}{\pgfqpoint{2.778119in}{1.902918in}}{\pgfqpoint{2.769883in}{1.902918in}}%
\pgfpathcurveto{\pgfqpoint{2.761646in}{1.902918in}}{\pgfqpoint{2.753746in}{1.899646in}}{\pgfqpoint{2.747922in}{1.893822in}}%
\pgfpathcurveto{\pgfqpoint{2.742099in}{1.887998in}}{\pgfqpoint{2.738826in}{1.880098in}}{\pgfqpoint{2.738826in}{1.871862in}}%
\pgfpathcurveto{\pgfqpoint{2.738826in}{1.863626in}}{\pgfqpoint{2.742099in}{1.855726in}}{\pgfqpoint{2.747922in}{1.849902in}}%
\pgfpathcurveto{\pgfqpoint{2.753746in}{1.844078in}}{\pgfqpoint{2.761646in}{1.840805in}}{\pgfqpoint{2.769883in}{1.840805in}}%
\pgfpathclose%
\pgfusepath{stroke,fill}%
\end{pgfscope}%
\begin{pgfscope}%
\pgfpathrectangle{\pgfqpoint{0.100000in}{0.212622in}}{\pgfqpoint{3.696000in}{3.696000in}}%
\pgfusepath{clip}%
\pgfsetbuttcap%
\pgfsetroundjoin%
\definecolor{currentfill}{rgb}{0.121569,0.466667,0.705882}%
\pgfsetfillcolor{currentfill}%
\pgfsetfillopacity{0.833900}%
\pgfsetlinewidth{1.003750pt}%
\definecolor{currentstroke}{rgb}{0.121569,0.466667,0.705882}%
\pgfsetstrokecolor{currentstroke}%
\pgfsetstrokeopacity{0.833900}%
\pgfsetdash{}{0pt}%
\pgfpathmoveto{\pgfqpoint{0.681152in}{2.527609in}}%
\pgfpathcurveto{\pgfqpoint{0.689388in}{2.527609in}}{\pgfqpoint{0.697288in}{2.530881in}}{\pgfqpoint{0.703112in}{2.536705in}}%
\pgfpathcurveto{\pgfqpoint{0.708936in}{2.542529in}}{\pgfqpoint{0.712208in}{2.550429in}}{\pgfqpoint{0.712208in}{2.558665in}}%
\pgfpathcurveto{\pgfqpoint{0.712208in}{2.566902in}}{\pgfqpoint{0.708936in}{2.574802in}}{\pgfqpoint{0.703112in}{2.580626in}}%
\pgfpathcurveto{\pgfqpoint{0.697288in}{2.586450in}}{\pgfqpoint{0.689388in}{2.589722in}}{\pgfqpoint{0.681152in}{2.589722in}}%
\pgfpathcurveto{\pgfqpoint{0.672915in}{2.589722in}}{\pgfqpoint{0.665015in}{2.586450in}}{\pgfqpoint{0.659191in}{2.580626in}}%
\pgfpathcurveto{\pgfqpoint{0.653367in}{2.574802in}}{\pgfqpoint{0.650095in}{2.566902in}}{\pgfqpoint{0.650095in}{2.558665in}}%
\pgfpathcurveto{\pgfqpoint{0.650095in}{2.550429in}}{\pgfqpoint{0.653367in}{2.542529in}}{\pgfqpoint{0.659191in}{2.536705in}}%
\pgfpathcurveto{\pgfqpoint{0.665015in}{2.530881in}}{\pgfqpoint{0.672915in}{2.527609in}}{\pgfqpoint{0.681152in}{2.527609in}}%
\pgfpathclose%
\pgfusepath{stroke,fill}%
\end{pgfscope}%
\begin{pgfscope}%
\pgfpathrectangle{\pgfqpoint{0.100000in}{0.212622in}}{\pgfqpoint{3.696000in}{3.696000in}}%
\pgfusepath{clip}%
\pgfsetbuttcap%
\pgfsetroundjoin%
\definecolor{currentfill}{rgb}{0.121569,0.466667,0.705882}%
\pgfsetfillcolor{currentfill}%
\pgfsetfillopacity{0.834694}%
\pgfsetlinewidth{1.003750pt}%
\definecolor{currentstroke}{rgb}{0.121569,0.466667,0.705882}%
\pgfsetstrokecolor{currentstroke}%
\pgfsetstrokeopacity{0.834694}%
\pgfsetdash{}{0pt}%
\pgfpathmoveto{\pgfqpoint{0.692602in}{2.524116in}}%
\pgfpathcurveto{\pgfqpoint{0.700838in}{2.524116in}}{\pgfqpoint{0.708738in}{2.527388in}}{\pgfqpoint{0.714562in}{2.533212in}}%
\pgfpathcurveto{\pgfqpoint{0.720386in}{2.539036in}}{\pgfqpoint{0.723658in}{2.546936in}}{\pgfqpoint{0.723658in}{2.555173in}}%
\pgfpathcurveto{\pgfqpoint{0.723658in}{2.563409in}}{\pgfqpoint{0.720386in}{2.571309in}}{\pgfqpoint{0.714562in}{2.577133in}}%
\pgfpathcurveto{\pgfqpoint{0.708738in}{2.582957in}}{\pgfqpoint{0.700838in}{2.586229in}}{\pgfqpoint{0.692602in}{2.586229in}}%
\pgfpathcurveto{\pgfqpoint{0.684365in}{2.586229in}}{\pgfqpoint{0.676465in}{2.582957in}}{\pgfqpoint{0.670641in}{2.577133in}}%
\pgfpathcurveto{\pgfqpoint{0.664817in}{2.571309in}}{\pgfqpoint{0.661545in}{2.563409in}}{\pgfqpoint{0.661545in}{2.555173in}}%
\pgfpathcurveto{\pgfqpoint{0.661545in}{2.546936in}}{\pgfqpoint{0.664817in}{2.539036in}}{\pgfqpoint{0.670641in}{2.533212in}}%
\pgfpathcurveto{\pgfqpoint{0.676465in}{2.527388in}}{\pgfqpoint{0.684365in}{2.524116in}}{\pgfqpoint{0.692602in}{2.524116in}}%
\pgfpathclose%
\pgfusepath{stroke,fill}%
\end{pgfscope}%
\begin{pgfscope}%
\pgfpathrectangle{\pgfqpoint{0.100000in}{0.212622in}}{\pgfqpoint{3.696000in}{3.696000in}}%
\pgfusepath{clip}%
\pgfsetbuttcap%
\pgfsetroundjoin%
\definecolor{currentfill}{rgb}{0.121569,0.466667,0.705882}%
\pgfsetfillcolor{currentfill}%
\pgfsetfillopacity{0.835484}%
\pgfsetlinewidth{1.003750pt}%
\definecolor{currentstroke}{rgb}{0.121569,0.466667,0.705882}%
\pgfsetstrokecolor{currentstroke}%
\pgfsetstrokeopacity{0.835484}%
\pgfsetdash{}{0pt}%
\pgfpathmoveto{\pgfqpoint{0.701619in}{2.520812in}}%
\pgfpathcurveto{\pgfqpoint{0.709855in}{2.520812in}}{\pgfqpoint{0.717755in}{2.524084in}}{\pgfqpoint{0.723579in}{2.529908in}}%
\pgfpathcurveto{\pgfqpoint{0.729403in}{2.535732in}}{\pgfqpoint{0.732675in}{2.543632in}}{\pgfqpoint{0.732675in}{2.551868in}}%
\pgfpathcurveto{\pgfqpoint{0.732675in}{2.560104in}}{\pgfqpoint{0.729403in}{2.568005in}}{\pgfqpoint{0.723579in}{2.573828in}}%
\pgfpathcurveto{\pgfqpoint{0.717755in}{2.579652in}}{\pgfqpoint{0.709855in}{2.582925in}}{\pgfqpoint{0.701619in}{2.582925in}}%
\pgfpathcurveto{\pgfqpoint{0.693382in}{2.582925in}}{\pgfqpoint{0.685482in}{2.579652in}}{\pgfqpoint{0.679658in}{2.573828in}}%
\pgfpathcurveto{\pgfqpoint{0.673834in}{2.568005in}}{\pgfqpoint{0.670562in}{2.560104in}}{\pgfqpoint{0.670562in}{2.551868in}}%
\pgfpathcurveto{\pgfqpoint{0.670562in}{2.543632in}}{\pgfqpoint{0.673834in}{2.535732in}}{\pgfqpoint{0.679658in}{2.529908in}}%
\pgfpathcurveto{\pgfqpoint{0.685482in}{2.524084in}}{\pgfqpoint{0.693382in}{2.520812in}}{\pgfqpoint{0.701619in}{2.520812in}}%
\pgfpathclose%
\pgfusepath{stroke,fill}%
\end{pgfscope}%
\begin{pgfscope}%
\pgfpathrectangle{\pgfqpoint{0.100000in}{0.212622in}}{\pgfqpoint{3.696000in}{3.696000in}}%
\pgfusepath{clip}%
\pgfsetbuttcap%
\pgfsetroundjoin%
\definecolor{currentfill}{rgb}{0.121569,0.466667,0.705882}%
\pgfsetfillcolor{currentfill}%
\pgfsetfillopacity{0.836391}%
\pgfsetlinewidth{1.003750pt}%
\definecolor{currentstroke}{rgb}{0.121569,0.466667,0.705882}%
\pgfsetstrokecolor{currentstroke}%
\pgfsetstrokeopacity{0.836391}%
\pgfsetdash{}{0pt}%
\pgfpathmoveto{\pgfqpoint{0.710052in}{2.518223in}}%
\pgfpathcurveto{\pgfqpoint{0.718288in}{2.518223in}}{\pgfqpoint{0.726188in}{2.521495in}}{\pgfqpoint{0.732012in}{2.527319in}}%
\pgfpathcurveto{\pgfqpoint{0.737836in}{2.533143in}}{\pgfqpoint{0.741108in}{2.541043in}}{\pgfqpoint{0.741108in}{2.549279in}}%
\pgfpathcurveto{\pgfqpoint{0.741108in}{2.557516in}}{\pgfqpoint{0.737836in}{2.565416in}}{\pgfqpoint{0.732012in}{2.571240in}}%
\pgfpathcurveto{\pgfqpoint{0.726188in}{2.577064in}}{\pgfqpoint{0.718288in}{2.580336in}}{\pgfqpoint{0.710052in}{2.580336in}}%
\pgfpathcurveto{\pgfqpoint{0.701815in}{2.580336in}}{\pgfqpoint{0.693915in}{2.577064in}}{\pgfqpoint{0.688091in}{2.571240in}}%
\pgfpathcurveto{\pgfqpoint{0.682268in}{2.565416in}}{\pgfqpoint{0.678995in}{2.557516in}}{\pgfqpoint{0.678995in}{2.549279in}}%
\pgfpathcurveto{\pgfqpoint{0.678995in}{2.541043in}}{\pgfqpoint{0.682268in}{2.533143in}}{\pgfqpoint{0.688091in}{2.527319in}}%
\pgfpathcurveto{\pgfqpoint{0.693915in}{2.521495in}}{\pgfqpoint{0.701815in}{2.518223in}}{\pgfqpoint{0.710052in}{2.518223in}}%
\pgfpathclose%
\pgfusepath{stroke,fill}%
\end{pgfscope}%
\begin{pgfscope}%
\pgfpathrectangle{\pgfqpoint{0.100000in}{0.212622in}}{\pgfqpoint{3.696000in}{3.696000in}}%
\pgfusepath{clip}%
\pgfsetbuttcap%
\pgfsetroundjoin%
\definecolor{currentfill}{rgb}{0.121569,0.466667,0.705882}%
\pgfsetfillcolor{currentfill}%
\pgfsetfillopacity{0.837426}%
\pgfsetlinewidth{1.003750pt}%
\definecolor{currentstroke}{rgb}{0.121569,0.466667,0.705882}%
\pgfsetstrokecolor{currentstroke}%
\pgfsetstrokeopacity{0.837426}%
\pgfsetdash{}{0pt}%
\pgfpathmoveto{\pgfqpoint{2.759258in}{1.843704in}}%
\pgfpathcurveto{\pgfqpoint{2.767494in}{1.843704in}}{\pgfqpoint{2.775394in}{1.846976in}}{\pgfqpoint{2.781218in}{1.852800in}}%
\pgfpathcurveto{\pgfqpoint{2.787042in}{1.858624in}}{\pgfqpoint{2.790314in}{1.866524in}}{\pgfqpoint{2.790314in}{1.874760in}}%
\pgfpathcurveto{\pgfqpoint{2.790314in}{1.882996in}}{\pgfqpoint{2.787042in}{1.890897in}}{\pgfqpoint{2.781218in}{1.896720in}}%
\pgfpathcurveto{\pgfqpoint{2.775394in}{1.902544in}}{\pgfqpoint{2.767494in}{1.905817in}}{\pgfqpoint{2.759258in}{1.905817in}}%
\pgfpathcurveto{\pgfqpoint{2.751021in}{1.905817in}}{\pgfqpoint{2.743121in}{1.902544in}}{\pgfqpoint{2.737297in}{1.896720in}}%
\pgfpathcurveto{\pgfqpoint{2.731473in}{1.890897in}}{\pgfqpoint{2.728201in}{1.882996in}}{\pgfqpoint{2.728201in}{1.874760in}}%
\pgfpathcurveto{\pgfqpoint{2.728201in}{1.866524in}}{\pgfqpoint{2.731473in}{1.858624in}}{\pgfqpoint{2.737297in}{1.852800in}}%
\pgfpathcurveto{\pgfqpoint{2.743121in}{1.846976in}}{\pgfqpoint{2.751021in}{1.843704in}}{\pgfqpoint{2.759258in}{1.843704in}}%
\pgfpathclose%
\pgfusepath{stroke,fill}%
\end{pgfscope}%
\begin{pgfscope}%
\pgfpathrectangle{\pgfqpoint{0.100000in}{0.212622in}}{\pgfqpoint{3.696000in}{3.696000in}}%
\pgfusepath{clip}%
\pgfsetbuttcap%
\pgfsetroundjoin%
\definecolor{currentfill}{rgb}{0.121569,0.466667,0.705882}%
\pgfsetfillcolor{currentfill}%
\pgfsetfillopacity{0.838396}%
\pgfsetlinewidth{1.003750pt}%
\definecolor{currentstroke}{rgb}{0.121569,0.466667,0.705882}%
\pgfsetstrokecolor{currentstroke}%
\pgfsetstrokeopacity{0.838396}%
\pgfsetdash{}{0pt}%
\pgfpathmoveto{\pgfqpoint{0.724952in}{2.514795in}}%
\pgfpathcurveto{\pgfqpoint{0.733188in}{2.514795in}}{\pgfqpoint{0.741088in}{2.518068in}}{\pgfqpoint{0.746912in}{2.523892in}}%
\pgfpathcurveto{\pgfqpoint{0.752736in}{2.529716in}}{\pgfqpoint{0.756009in}{2.537616in}}{\pgfqpoint{0.756009in}{2.545852in}}%
\pgfpathcurveto{\pgfqpoint{0.756009in}{2.554088in}}{\pgfqpoint{0.752736in}{2.561988in}}{\pgfqpoint{0.746912in}{2.567812in}}%
\pgfpathcurveto{\pgfqpoint{0.741088in}{2.573636in}}{\pgfqpoint{0.733188in}{2.576908in}}{\pgfqpoint{0.724952in}{2.576908in}}%
\pgfpathcurveto{\pgfqpoint{0.716716in}{2.576908in}}{\pgfqpoint{0.708816in}{2.573636in}}{\pgfqpoint{0.702992in}{2.567812in}}%
\pgfpathcurveto{\pgfqpoint{0.697168in}{2.561988in}}{\pgfqpoint{0.693896in}{2.554088in}}{\pgfqpoint{0.693896in}{2.545852in}}%
\pgfpathcurveto{\pgfqpoint{0.693896in}{2.537616in}}{\pgfqpoint{0.697168in}{2.529716in}}{\pgfqpoint{0.702992in}{2.523892in}}%
\pgfpathcurveto{\pgfqpoint{0.708816in}{2.518068in}}{\pgfqpoint{0.716716in}{2.514795in}}{\pgfqpoint{0.724952in}{2.514795in}}%
\pgfpathclose%
\pgfusepath{stroke,fill}%
\end{pgfscope}%
\begin{pgfscope}%
\pgfpathrectangle{\pgfqpoint{0.100000in}{0.212622in}}{\pgfqpoint{3.696000in}{3.696000in}}%
\pgfusepath{clip}%
\pgfsetbuttcap%
\pgfsetroundjoin%
\definecolor{currentfill}{rgb}{0.121569,0.466667,0.705882}%
\pgfsetfillcolor{currentfill}%
\pgfsetfillopacity{0.840263}%
\pgfsetlinewidth{1.003750pt}%
\definecolor{currentstroke}{rgb}{0.121569,0.466667,0.705882}%
\pgfsetstrokecolor{currentstroke}%
\pgfsetstrokeopacity{0.840263}%
\pgfsetdash{}{0pt}%
\pgfpathmoveto{\pgfqpoint{0.739097in}{2.509555in}}%
\pgfpathcurveto{\pgfqpoint{0.747333in}{2.509555in}}{\pgfqpoint{0.755233in}{2.512827in}}{\pgfqpoint{0.761057in}{2.518651in}}%
\pgfpathcurveto{\pgfqpoint{0.766881in}{2.524475in}}{\pgfqpoint{0.770154in}{2.532375in}}{\pgfqpoint{0.770154in}{2.540612in}}%
\pgfpathcurveto{\pgfqpoint{0.770154in}{2.548848in}}{\pgfqpoint{0.766881in}{2.556748in}}{\pgfqpoint{0.761057in}{2.562572in}}%
\pgfpathcurveto{\pgfqpoint{0.755233in}{2.568396in}}{\pgfqpoint{0.747333in}{2.571668in}}{\pgfqpoint{0.739097in}{2.571668in}}%
\pgfpathcurveto{\pgfqpoint{0.730861in}{2.571668in}}{\pgfqpoint{0.722961in}{2.568396in}}{\pgfqpoint{0.717137in}{2.562572in}}%
\pgfpathcurveto{\pgfqpoint{0.711313in}{2.556748in}}{\pgfqpoint{0.708041in}{2.548848in}}{\pgfqpoint{0.708041in}{2.540612in}}%
\pgfpathcurveto{\pgfqpoint{0.708041in}{2.532375in}}{\pgfqpoint{0.711313in}{2.524475in}}{\pgfqpoint{0.717137in}{2.518651in}}%
\pgfpathcurveto{\pgfqpoint{0.722961in}{2.512827in}}{\pgfqpoint{0.730861in}{2.509555in}}{\pgfqpoint{0.739097in}{2.509555in}}%
\pgfpathclose%
\pgfusepath{stroke,fill}%
\end{pgfscope}%
\begin{pgfscope}%
\pgfpathrectangle{\pgfqpoint{0.100000in}{0.212622in}}{\pgfqpoint{3.696000in}{3.696000in}}%
\pgfusepath{clip}%
\pgfsetbuttcap%
\pgfsetroundjoin%
\definecolor{currentfill}{rgb}{0.121569,0.466667,0.705882}%
\pgfsetfillcolor{currentfill}%
\pgfsetfillopacity{0.841799}%
\pgfsetlinewidth{1.003750pt}%
\definecolor{currentstroke}{rgb}{0.121569,0.466667,0.705882}%
\pgfsetstrokecolor{currentstroke}%
\pgfsetstrokeopacity{0.841799}%
\pgfsetdash{}{0pt}%
\pgfpathmoveto{\pgfqpoint{0.753596in}{2.504288in}}%
\pgfpathcurveto{\pgfqpoint{0.761832in}{2.504288in}}{\pgfqpoint{0.769732in}{2.507560in}}{\pgfqpoint{0.775556in}{2.513384in}}%
\pgfpathcurveto{\pgfqpoint{0.781380in}{2.519208in}}{\pgfqpoint{0.784653in}{2.527108in}}{\pgfqpoint{0.784653in}{2.535344in}}%
\pgfpathcurveto{\pgfqpoint{0.784653in}{2.543581in}}{\pgfqpoint{0.781380in}{2.551481in}}{\pgfqpoint{0.775556in}{2.557305in}}%
\pgfpathcurveto{\pgfqpoint{0.769732in}{2.563129in}}{\pgfqpoint{0.761832in}{2.566401in}}{\pgfqpoint{0.753596in}{2.566401in}}%
\pgfpathcurveto{\pgfqpoint{0.745360in}{2.566401in}}{\pgfqpoint{0.737460in}{2.563129in}}{\pgfqpoint{0.731636in}{2.557305in}}%
\pgfpathcurveto{\pgfqpoint{0.725812in}{2.551481in}}{\pgfqpoint{0.722540in}{2.543581in}}{\pgfqpoint{0.722540in}{2.535344in}}%
\pgfpathcurveto{\pgfqpoint{0.722540in}{2.527108in}}{\pgfqpoint{0.725812in}{2.519208in}}{\pgfqpoint{0.731636in}{2.513384in}}%
\pgfpathcurveto{\pgfqpoint{0.737460in}{2.507560in}}{\pgfqpoint{0.745360in}{2.504288in}}{\pgfqpoint{0.753596in}{2.504288in}}%
\pgfpathclose%
\pgfusepath{stroke,fill}%
\end{pgfscope}%
\begin{pgfscope}%
\pgfpathrectangle{\pgfqpoint{0.100000in}{0.212622in}}{\pgfqpoint{3.696000in}{3.696000in}}%
\pgfusepath{clip}%
\pgfsetbuttcap%
\pgfsetroundjoin%
\definecolor{currentfill}{rgb}{0.121569,0.466667,0.705882}%
\pgfsetfillcolor{currentfill}%
\pgfsetfillopacity{0.842925}%
\pgfsetlinewidth{1.003750pt}%
\definecolor{currentstroke}{rgb}{0.121569,0.466667,0.705882}%
\pgfsetstrokecolor{currentstroke}%
\pgfsetstrokeopacity{0.842925}%
\pgfsetdash{}{0pt}%
\pgfpathmoveto{\pgfqpoint{2.753982in}{1.843759in}}%
\pgfpathcurveto{\pgfqpoint{2.762218in}{1.843759in}}{\pgfqpoint{2.770118in}{1.847031in}}{\pgfqpoint{2.775942in}{1.852855in}}%
\pgfpathcurveto{\pgfqpoint{2.781766in}{1.858679in}}{\pgfqpoint{2.785039in}{1.866579in}}{\pgfqpoint{2.785039in}{1.874815in}}%
\pgfpathcurveto{\pgfqpoint{2.785039in}{1.883051in}}{\pgfqpoint{2.781766in}{1.890952in}}{\pgfqpoint{2.775942in}{1.896775in}}%
\pgfpathcurveto{\pgfqpoint{2.770118in}{1.902599in}}{\pgfqpoint{2.762218in}{1.905872in}}{\pgfqpoint{2.753982in}{1.905872in}}%
\pgfpathcurveto{\pgfqpoint{2.745746in}{1.905872in}}{\pgfqpoint{2.737846in}{1.902599in}}{\pgfqpoint{2.732022in}{1.896775in}}%
\pgfpathcurveto{\pgfqpoint{2.726198in}{1.890952in}}{\pgfqpoint{2.722926in}{1.883051in}}{\pgfqpoint{2.722926in}{1.874815in}}%
\pgfpathcurveto{\pgfqpoint{2.722926in}{1.866579in}}{\pgfqpoint{2.726198in}{1.858679in}}{\pgfqpoint{2.732022in}{1.852855in}}%
\pgfpathcurveto{\pgfqpoint{2.737846in}{1.847031in}}{\pgfqpoint{2.745746in}{1.843759in}}{\pgfqpoint{2.753982in}{1.843759in}}%
\pgfpathclose%
\pgfusepath{stroke,fill}%
\end{pgfscope}%
\begin{pgfscope}%
\pgfpathrectangle{\pgfqpoint{0.100000in}{0.212622in}}{\pgfqpoint{3.696000in}{3.696000in}}%
\pgfusepath{clip}%
\pgfsetbuttcap%
\pgfsetroundjoin%
\definecolor{currentfill}{rgb}{0.121569,0.466667,0.705882}%
\pgfsetfillcolor{currentfill}%
\pgfsetfillopacity{0.843344}%
\pgfsetlinewidth{1.003750pt}%
\definecolor{currentstroke}{rgb}{0.121569,0.466667,0.705882}%
\pgfsetstrokecolor{currentstroke}%
\pgfsetstrokeopacity{0.843344}%
\pgfsetdash{}{0pt}%
\pgfpathmoveto{\pgfqpoint{0.766236in}{2.500881in}}%
\pgfpathcurveto{\pgfqpoint{0.774472in}{2.500881in}}{\pgfqpoint{0.782372in}{2.504153in}}{\pgfqpoint{0.788196in}{2.509977in}}%
\pgfpathcurveto{\pgfqpoint{0.794020in}{2.515801in}}{\pgfqpoint{0.797293in}{2.523701in}}{\pgfqpoint{0.797293in}{2.531938in}}%
\pgfpathcurveto{\pgfqpoint{0.797293in}{2.540174in}}{\pgfqpoint{0.794020in}{2.548074in}}{\pgfqpoint{0.788196in}{2.553898in}}%
\pgfpathcurveto{\pgfqpoint{0.782372in}{2.559722in}}{\pgfqpoint{0.774472in}{2.562994in}}{\pgfqpoint{0.766236in}{2.562994in}}%
\pgfpathcurveto{\pgfqpoint{0.758000in}{2.562994in}}{\pgfqpoint{0.750100in}{2.559722in}}{\pgfqpoint{0.744276in}{2.553898in}}%
\pgfpathcurveto{\pgfqpoint{0.738452in}{2.548074in}}{\pgfqpoint{0.735180in}{2.540174in}}{\pgfqpoint{0.735180in}{2.531938in}}%
\pgfpathcurveto{\pgfqpoint{0.735180in}{2.523701in}}{\pgfqpoint{0.738452in}{2.515801in}}{\pgfqpoint{0.744276in}{2.509977in}}%
\pgfpathcurveto{\pgfqpoint{0.750100in}{2.504153in}}{\pgfqpoint{0.758000in}{2.500881in}}{\pgfqpoint{0.766236in}{2.500881in}}%
\pgfpathclose%
\pgfusepath{stroke,fill}%
\end{pgfscope}%
\begin{pgfscope}%
\pgfpathrectangle{\pgfqpoint{0.100000in}{0.212622in}}{\pgfqpoint{3.696000in}{3.696000in}}%
\pgfusepath{clip}%
\pgfsetbuttcap%
\pgfsetroundjoin%
\definecolor{currentfill}{rgb}{0.121569,0.466667,0.705882}%
\pgfsetfillcolor{currentfill}%
\pgfsetfillopacity{0.844954}%
\pgfsetlinewidth{1.003750pt}%
\definecolor{currentstroke}{rgb}{0.121569,0.466667,0.705882}%
\pgfsetstrokecolor{currentstroke}%
\pgfsetstrokeopacity{0.844954}%
\pgfsetdash{}{0pt}%
\pgfpathmoveto{\pgfqpoint{0.776540in}{2.498340in}}%
\pgfpathcurveto{\pgfqpoint{0.784777in}{2.498340in}}{\pgfqpoint{0.792677in}{2.501613in}}{\pgfqpoint{0.798501in}{2.507437in}}%
\pgfpathcurveto{\pgfqpoint{0.804325in}{2.513261in}}{\pgfqpoint{0.807597in}{2.521161in}}{\pgfqpoint{0.807597in}{2.529397in}}%
\pgfpathcurveto{\pgfqpoint{0.807597in}{2.537633in}}{\pgfqpoint{0.804325in}{2.545533in}}{\pgfqpoint{0.798501in}{2.551357in}}%
\pgfpathcurveto{\pgfqpoint{0.792677in}{2.557181in}}{\pgfqpoint{0.784777in}{2.560453in}}{\pgfqpoint{0.776540in}{2.560453in}}%
\pgfpathcurveto{\pgfqpoint{0.768304in}{2.560453in}}{\pgfqpoint{0.760404in}{2.557181in}}{\pgfqpoint{0.754580in}{2.551357in}}%
\pgfpathcurveto{\pgfqpoint{0.748756in}{2.545533in}}{\pgfqpoint{0.745484in}{2.537633in}}{\pgfqpoint{0.745484in}{2.529397in}}%
\pgfpathcurveto{\pgfqpoint{0.745484in}{2.521161in}}{\pgfqpoint{0.748756in}{2.513261in}}{\pgfqpoint{0.754580in}{2.507437in}}%
\pgfpathcurveto{\pgfqpoint{0.760404in}{2.501613in}}{\pgfqpoint{0.768304in}{2.498340in}}{\pgfqpoint{0.776540in}{2.498340in}}%
\pgfpathclose%
\pgfusepath{stroke,fill}%
\end{pgfscope}%
\begin{pgfscope}%
\pgfpathrectangle{\pgfqpoint{0.100000in}{0.212622in}}{\pgfqpoint{3.696000in}{3.696000in}}%
\pgfusepath{clip}%
\pgfsetbuttcap%
\pgfsetroundjoin%
\definecolor{currentfill}{rgb}{0.121569,0.466667,0.705882}%
\pgfsetfillcolor{currentfill}%
\pgfsetfillopacity{0.845770}%
\pgfsetlinewidth{1.003750pt}%
\definecolor{currentstroke}{rgb}{0.121569,0.466667,0.705882}%
\pgfsetstrokecolor{currentstroke}%
\pgfsetstrokeopacity{0.845770}%
\pgfsetdash{}{0pt}%
\pgfpathmoveto{\pgfqpoint{2.748341in}{1.845027in}}%
\pgfpathcurveto{\pgfqpoint{2.756577in}{1.845027in}}{\pgfqpoint{2.764477in}{1.848299in}}{\pgfqpoint{2.770301in}{1.854123in}}%
\pgfpathcurveto{\pgfqpoint{2.776125in}{1.859947in}}{\pgfqpoint{2.779398in}{1.867847in}}{\pgfqpoint{2.779398in}{1.876083in}}%
\pgfpathcurveto{\pgfqpoint{2.779398in}{1.884319in}}{\pgfqpoint{2.776125in}{1.892219in}}{\pgfqpoint{2.770301in}{1.898043in}}%
\pgfpathcurveto{\pgfqpoint{2.764477in}{1.903867in}}{\pgfqpoint{2.756577in}{1.907140in}}{\pgfqpoint{2.748341in}{1.907140in}}%
\pgfpathcurveto{\pgfqpoint{2.740105in}{1.907140in}}{\pgfqpoint{2.732205in}{1.903867in}}{\pgfqpoint{2.726381in}{1.898043in}}%
\pgfpathcurveto{\pgfqpoint{2.720557in}{1.892219in}}{\pgfqpoint{2.717285in}{1.884319in}}{\pgfqpoint{2.717285in}{1.876083in}}%
\pgfpathcurveto{\pgfqpoint{2.717285in}{1.867847in}}{\pgfqpoint{2.720557in}{1.859947in}}{\pgfqpoint{2.726381in}{1.854123in}}%
\pgfpathcurveto{\pgfqpoint{2.732205in}{1.848299in}}{\pgfqpoint{2.740105in}{1.845027in}}{\pgfqpoint{2.748341in}{1.845027in}}%
\pgfpathclose%
\pgfusepath{stroke,fill}%
\end{pgfscope}%
\begin{pgfscope}%
\pgfpathrectangle{\pgfqpoint{0.100000in}{0.212622in}}{\pgfqpoint{3.696000in}{3.696000in}}%
\pgfusepath{clip}%
\pgfsetbuttcap%
\pgfsetroundjoin%
\definecolor{currentfill}{rgb}{0.121569,0.466667,0.705882}%
\pgfsetfillcolor{currentfill}%
\pgfsetfillopacity{0.847755}%
\pgfsetlinewidth{1.003750pt}%
\definecolor{currentstroke}{rgb}{0.121569,0.466667,0.705882}%
\pgfsetstrokecolor{currentstroke}%
\pgfsetstrokeopacity{0.847755}%
\pgfsetdash{}{0pt}%
\pgfpathmoveto{\pgfqpoint{0.794943in}{2.491669in}}%
\pgfpathcurveto{\pgfqpoint{0.803180in}{2.491669in}}{\pgfqpoint{0.811080in}{2.494941in}}{\pgfqpoint{0.816904in}{2.500765in}}%
\pgfpathcurveto{\pgfqpoint{0.822727in}{2.506589in}}{\pgfqpoint{0.826000in}{2.514489in}}{\pgfqpoint{0.826000in}{2.522725in}}%
\pgfpathcurveto{\pgfqpoint{0.826000in}{2.530961in}}{\pgfqpoint{0.822727in}{2.538861in}}{\pgfqpoint{0.816904in}{2.544685in}}%
\pgfpathcurveto{\pgfqpoint{0.811080in}{2.550509in}}{\pgfqpoint{0.803180in}{2.553782in}}{\pgfqpoint{0.794943in}{2.553782in}}%
\pgfpathcurveto{\pgfqpoint{0.786707in}{2.553782in}}{\pgfqpoint{0.778807in}{2.550509in}}{\pgfqpoint{0.772983in}{2.544685in}}%
\pgfpathcurveto{\pgfqpoint{0.767159in}{2.538861in}}{\pgfqpoint{0.763887in}{2.530961in}}{\pgfqpoint{0.763887in}{2.522725in}}%
\pgfpathcurveto{\pgfqpoint{0.763887in}{2.514489in}}{\pgfqpoint{0.767159in}{2.506589in}}{\pgfqpoint{0.772983in}{2.500765in}}%
\pgfpathcurveto{\pgfqpoint{0.778807in}{2.494941in}}{\pgfqpoint{0.786707in}{2.491669in}}{\pgfqpoint{0.794943in}{2.491669in}}%
\pgfpathclose%
\pgfusepath{stroke,fill}%
\end{pgfscope}%
\begin{pgfscope}%
\pgfpathrectangle{\pgfqpoint{0.100000in}{0.212622in}}{\pgfqpoint{3.696000in}{3.696000in}}%
\pgfusepath{clip}%
\pgfsetbuttcap%
\pgfsetroundjoin%
\definecolor{currentfill}{rgb}{0.121569,0.466667,0.705882}%
\pgfsetfillcolor{currentfill}%
\pgfsetfillopacity{0.848677}%
\pgfsetlinewidth{1.003750pt}%
\definecolor{currentstroke}{rgb}{0.121569,0.466667,0.705882}%
\pgfsetstrokecolor{currentstroke}%
\pgfsetstrokeopacity{0.848677}%
\pgfsetdash{}{0pt}%
\pgfpathmoveto{\pgfqpoint{2.740943in}{1.847436in}}%
\pgfpathcurveto{\pgfqpoint{2.749179in}{1.847436in}}{\pgfqpoint{2.757079in}{1.850708in}}{\pgfqpoint{2.762903in}{1.856532in}}%
\pgfpathcurveto{\pgfqpoint{2.768727in}{1.862356in}}{\pgfqpoint{2.772000in}{1.870256in}}{\pgfqpoint{2.772000in}{1.878493in}}%
\pgfpathcurveto{\pgfqpoint{2.772000in}{1.886729in}}{\pgfqpoint{2.768727in}{1.894629in}}{\pgfqpoint{2.762903in}{1.900453in}}%
\pgfpathcurveto{\pgfqpoint{2.757079in}{1.906277in}}{\pgfqpoint{2.749179in}{1.909549in}}{\pgfqpoint{2.740943in}{1.909549in}}%
\pgfpathcurveto{\pgfqpoint{2.732707in}{1.909549in}}{\pgfqpoint{2.724807in}{1.906277in}}{\pgfqpoint{2.718983in}{1.900453in}}%
\pgfpathcurveto{\pgfqpoint{2.713159in}{1.894629in}}{\pgfqpoint{2.709887in}{1.886729in}}{\pgfqpoint{2.709887in}{1.878493in}}%
\pgfpathcurveto{\pgfqpoint{2.709887in}{1.870256in}}{\pgfqpoint{2.713159in}{1.862356in}}{\pgfqpoint{2.718983in}{1.856532in}}%
\pgfpathcurveto{\pgfqpoint{2.724807in}{1.850708in}}{\pgfqpoint{2.732707in}{1.847436in}}{\pgfqpoint{2.740943in}{1.847436in}}%
\pgfpathclose%
\pgfusepath{stroke,fill}%
\end{pgfscope}%
\begin{pgfscope}%
\pgfpathrectangle{\pgfqpoint{0.100000in}{0.212622in}}{\pgfqpoint{3.696000in}{3.696000in}}%
\pgfusepath{clip}%
\pgfsetbuttcap%
\pgfsetroundjoin%
\definecolor{currentfill}{rgb}{0.121569,0.466667,0.705882}%
\pgfsetfillcolor{currentfill}%
\pgfsetfillopacity{0.849234}%
\pgfsetlinewidth{1.003750pt}%
\definecolor{currentstroke}{rgb}{0.121569,0.466667,0.705882}%
\pgfsetstrokecolor{currentstroke}%
\pgfsetstrokeopacity{0.849234}%
\pgfsetdash{}{0pt}%
\pgfpathmoveto{\pgfqpoint{0.814351in}{2.485483in}}%
\pgfpathcurveto{\pgfqpoint{0.822587in}{2.485483in}}{\pgfqpoint{0.830487in}{2.488755in}}{\pgfqpoint{0.836311in}{2.494579in}}%
\pgfpathcurveto{\pgfqpoint{0.842135in}{2.500403in}}{\pgfqpoint{0.845407in}{2.508303in}}{\pgfqpoint{0.845407in}{2.516540in}}%
\pgfpathcurveto{\pgfqpoint{0.845407in}{2.524776in}}{\pgfqpoint{0.842135in}{2.532676in}}{\pgfqpoint{0.836311in}{2.538500in}}%
\pgfpathcurveto{\pgfqpoint{0.830487in}{2.544324in}}{\pgfqpoint{0.822587in}{2.547596in}}{\pgfqpoint{0.814351in}{2.547596in}}%
\pgfpathcurveto{\pgfqpoint{0.806114in}{2.547596in}}{\pgfqpoint{0.798214in}{2.544324in}}{\pgfqpoint{0.792390in}{2.538500in}}%
\pgfpathcurveto{\pgfqpoint{0.786566in}{2.532676in}}{\pgfqpoint{0.783294in}{2.524776in}}{\pgfqpoint{0.783294in}{2.516540in}}%
\pgfpathcurveto{\pgfqpoint{0.783294in}{2.508303in}}{\pgfqpoint{0.786566in}{2.500403in}}{\pgfqpoint{0.792390in}{2.494579in}}%
\pgfpathcurveto{\pgfqpoint{0.798214in}{2.488755in}}{\pgfqpoint{0.806114in}{2.485483in}}{\pgfqpoint{0.814351in}{2.485483in}}%
\pgfpathclose%
\pgfusepath{stroke,fill}%
\end{pgfscope}%
\begin{pgfscope}%
\pgfpathrectangle{\pgfqpoint{0.100000in}{0.212622in}}{\pgfqpoint{3.696000in}{3.696000in}}%
\pgfusepath{clip}%
\pgfsetbuttcap%
\pgfsetroundjoin%
\definecolor{currentfill}{rgb}{0.121569,0.466667,0.705882}%
\pgfsetfillcolor{currentfill}%
\pgfsetfillopacity{0.850978}%
\pgfsetlinewidth{1.003750pt}%
\definecolor{currentstroke}{rgb}{0.121569,0.466667,0.705882}%
\pgfsetstrokecolor{currentstroke}%
\pgfsetstrokeopacity{0.850978}%
\pgfsetdash{}{0pt}%
\pgfpathmoveto{\pgfqpoint{0.829935in}{2.479307in}}%
\pgfpathcurveto{\pgfqpoint{0.838171in}{2.479307in}}{\pgfqpoint{0.846071in}{2.482579in}}{\pgfqpoint{0.851895in}{2.488403in}}%
\pgfpathcurveto{\pgfqpoint{0.857719in}{2.494227in}}{\pgfqpoint{0.860991in}{2.502127in}}{\pgfqpoint{0.860991in}{2.510363in}}%
\pgfpathcurveto{\pgfqpoint{0.860991in}{2.518600in}}{\pgfqpoint{0.857719in}{2.526500in}}{\pgfqpoint{0.851895in}{2.532324in}}%
\pgfpathcurveto{\pgfqpoint{0.846071in}{2.538147in}}{\pgfqpoint{0.838171in}{2.541420in}}{\pgfqpoint{0.829935in}{2.541420in}}%
\pgfpathcurveto{\pgfqpoint{0.821698in}{2.541420in}}{\pgfqpoint{0.813798in}{2.538147in}}{\pgfqpoint{0.807974in}{2.532324in}}%
\pgfpathcurveto{\pgfqpoint{0.802151in}{2.526500in}}{\pgfqpoint{0.798878in}{2.518600in}}{\pgfqpoint{0.798878in}{2.510363in}}%
\pgfpathcurveto{\pgfqpoint{0.798878in}{2.502127in}}{\pgfqpoint{0.802151in}{2.494227in}}{\pgfqpoint{0.807974in}{2.488403in}}%
\pgfpathcurveto{\pgfqpoint{0.813798in}{2.482579in}}{\pgfqpoint{0.821698in}{2.479307in}}{\pgfqpoint{0.829935in}{2.479307in}}%
\pgfpathclose%
\pgfusepath{stroke,fill}%
\end{pgfscope}%
\begin{pgfscope}%
\pgfpathrectangle{\pgfqpoint{0.100000in}{0.212622in}}{\pgfqpoint{3.696000in}{3.696000in}}%
\pgfusepath{clip}%
\pgfsetbuttcap%
\pgfsetroundjoin%
\definecolor{currentfill}{rgb}{0.121569,0.466667,0.705882}%
\pgfsetfillcolor{currentfill}%
\pgfsetfillopacity{0.852933}%
\pgfsetlinewidth{1.003750pt}%
\definecolor{currentstroke}{rgb}{0.121569,0.466667,0.705882}%
\pgfsetstrokecolor{currentstroke}%
\pgfsetstrokeopacity{0.852933}%
\pgfsetdash{}{0pt}%
\pgfpathmoveto{\pgfqpoint{2.735259in}{1.847852in}}%
\pgfpathcurveto{\pgfqpoint{2.743495in}{1.847852in}}{\pgfqpoint{2.751395in}{1.851124in}}{\pgfqpoint{2.757219in}{1.856948in}}%
\pgfpathcurveto{\pgfqpoint{2.763043in}{1.862772in}}{\pgfqpoint{2.766315in}{1.870672in}}{\pgfqpoint{2.766315in}{1.878908in}}%
\pgfpathcurveto{\pgfqpoint{2.766315in}{1.887144in}}{\pgfqpoint{2.763043in}{1.895045in}}{\pgfqpoint{2.757219in}{1.900868in}}%
\pgfpathcurveto{\pgfqpoint{2.751395in}{1.906692in}}{\pgfqpoint{2.743495in}{1.909965in}}{\pgfqpoint{2.735259in}{1.909965in}}%
\pgfpathcurveto{\pgfqpoint{2.727022in}{1.909965in}}{\pgfqpoint{2.719122in}{1.906692in}}{\pgfqpoint{2.713298in}{1.900868in}}%
\pgfpathcurveto{\pgfqpoint{2.707474in}{1.895045in}}{\pgfqpoint{2.704202in}{1.887144in}}{\pgfqpoint{2.704202in}{1.878908in}}%
\pgfpathcurveto{\pgfqpoint{2.704202in}{1.870672in}}{\pgfqpoint{2.707474in}{1.862772in}}{\pgfqpoint{2.713298in}{1.856948in}}%
\pgfpathcurveto{\pgfqpoint{2.719122in}{1.851124in}}{\pgfqpoint{2.727022in}{1.847852in}}{\pgfqpoint{2.735259in}{1.847852in}}%
\pgfpathclose%
\pgfusepath{stroke,fill}%
\end{pgfscope}%
\begin{pgfscope}%
\pgfpathrectangle{\pgfqpoint{0.100000in}{0.212622in}}{\pgfqpoint{3.696000in}{3.696000in}}%
\pgfusepath{clip}%
\pgfsetbuttcap%
\pgfsetroundjoin%
\definecolor{currentfill}{rgb}{0.121569,0.466667,0.705882}%
\pgfsetfillcolor{currentfill}%
\pgfsetfillopacity{0.853286}%
\pgfsetlinewidth{1.003750pt}%
\definecolor{currentstroke}{rgb}{0.121569,0.466667,0.705882}%
\pgfsetstrokecolor{currentstroke}%
\pgfsetstrokeopacity{0.853286}%
\pgfsetdash{}{0pt}%
\pgfpathmoveto{\pgfqpoint{0.860373in}{2.469287in}}%
\pgfpathcurveto{\pgfqpoint{0.868609in}{2.469287in}}{\pgfqpoint{0.876510in}{2.472559in}}{\pgfqpoint{0.882333in}{2.478383in}}%
\pgfpathcurveto{\pgfqpoint{0.888157in}{2.484207in}}{\pgfqpoint{0.891430in}{2.492107in}}{\pgfqpoint{0.891430in}{2.500343in}}%
\pgfpathcurveto{\pgfqpoint{0.891430in}{2.508580in}}{\pgfqpoint{0.888157in}{2.516480in}}{\pgfqpoint{0.882333in}{2.522304in}}%
\pgfpathcurveto{\pgfqpoint{0.876510in}{2.528128in}}{\pgfqpoint{0.868609in}{2.531400in}}{\pgfqpoint{0.860373in}{2.531400in}}%
\pgfpathcurveto{\pgfqpoint{0.852137in}{2.531400in}}{\pgfqpoint{0.844237in}{2.528128in}}{\pgfqpoint{0.838413in}{2.522304in}}%
\pgfpathcurveto{\pgfqpoint{0.832589in}{2.516480in}}{\pgfqpoint{0.829317in}{2.508580in}}{\pgfqpoint{0.829317in}{2.500343in}}%
\pgfpathcurveto{\pgfqpoint{0.829317in}{2.492107in}}{\pgfqpoint{0.832589in}{2.484207in}}{\pgfqpoint{0.838413in}{2.478383in}}%
\pgfpathcurveto{\pgfqpoint{0.844237in}{2.472559in}}{\pgfqpoint{0.852137in}{2.469287in}}{\pgfqpoint{0.860373in}{2.469287in}}%
\pgfpathclose%
\pgfusepath{stroke,fill}%
\end{pgfscope}%
\begin{pgfscope}%
\pgfpathrectangle{\pgfqpoint{0.100000in}{0.212622in}}{\pgfqpoint{3.696000in}{3.696000in}}%
\pgfusepath{clip}%
\pgfsetbuttcap%
\pgfsetroundjoin%
\definecolor{currentfill}{rgb}{0.121569,0.466667,0.705882}%
\pgfsetfillcolor{currentfill}%
\pgfsetfillopacity{0.854817}%
\pgfsetlinewidth{1.003750pt}%
\definecolor{currentstroke}{rgb}{0.121569,0.466667,0.705882}%
\pgfsetstrokecolor{currentstroke}%
\pgfsetstrokeopacity{0.854817}%
\pgfsetdash{}{0pt}%
\pgfpathmoveto{\pgfqpoint{0.891290in}{2.457382in}}%
\pgfpathcurveto{\pgfqpoint{0.899526in}{2.457382in}}{\pgfqpoint{0.907426in}{2.460654in}}{\pgfqpoint{0.913250in}{2.466478in}}%
\pgfpathcurveto{\pgfqpoint{0.919074in}{2.472302in}}{\pgfqpoint{0.922346in}{2.480202in}}{\pgfqpoint{0.922346in}{2.488439in}}%
\pgfpathcurveto{\pgfqpoint{0.922346in}{2.496675in}}{\pgfqpoint{0.919074in}{2.504575in}}{\pgfqpoint{0.913250in}{2.510399in}}%
\pgfpathcurveto{\pgfqpoint{0.907426in}{2.516223in}}{\pgfqpoint{0.899526in}{2.519495in}}{\pgfqpoint{0.891290in}{2.519495in}}%
\pgfpathcurveto{\pgfqpoint{0.883053in}{2.519495in}}{\pgfqpoint{0.875153in}{2.516223in}}{\pgfqpoint{0.869329in}{2.510399in}}%
\pgfpathcurveto{\pgfqpoint{0.863505in}{2.504575in}}{\pgfqpoint{0.860233in}{2.496675in}}{\pgfqpoint{0.860233in}{2.488439in}}%
\pgfpathcurveto{\pgfqpoint{0.860233in}{2.480202in}}{\pgfqpoint{0.863505in}{2.472302in}}{\pgfqpoint{0.869329in}{2.466478in}}%
\pgfpathcurveto{\pgfqpoint{0.875153in}{2.460654in}}{\pgfqpoint{0.883053in}{2.457382in}}{\pgfqpoint{0.891290in}{2.457382in}}%
\pgfpathclose%
\pgfusepath{stroke,fill}%
\end{pgfscope}%
\begin{pgfscope}%
\pgfpathrectangle{\pgfqpoint{0.100000in}{0.212622in}}{\pgfqpoint{3.696000in}{3.696000in}}%
\pgfusepath{clip}%
\pgfsetbuttcap%
\pgfsetroundjoin%
\definecolor{currentfill}{rgb}{0.121569,0.466667,0.705882}%
\pgfsetfillcolor{currentfill}%
\pgfsetfillopacity{0.855062}%
\pgfsetlinewidth{1.003750pt}%
\definecolor{currentstroke}{rgb}{0.121569,0.466667,0.705882}%
\pgfsetstrokecolor{currentstroke}%
\pgfsetstrokeopacity{0.855062}%
\pgfsetdash{}{0pt}%
\pgfpathmoveto{\pgfqpoint{2.730181in}{1.848812in}}%
\pgfpathcurveto{\pgfqpoint{2.738418in}{1.848812in}}{\pgfqpoint{2.746318in}{1.852084in}}{\pgfqpoint{2.752142in}{1.857908in}}%
\pgfpathcurveto{\pgfqpoint{2.757965in}{1.863732in}}{\pgfqpoint{2.761238in}{1.871632in}}{\pgfqpoint{2.761238in}{1.879869in}}%
\pgfpathcurveto{\pgfqpoint{2.761238in}{1.888105in}}{\pgfqpoint{2.757965in}{1.896005in}}{\pgfqpoint{2.752142in}{1.901829in}}%
\pgfpathcurveto{\pgfqpoint{2.746318in}{1.907653in}}{\pgfqpoint{2.738418in}{1.910925in}}{\pgfqpoint{2.730181in}{1.910925in}}%
\pgfpathcurveto{\pgfqpoint{2.721945in}{1.910925in}}{\pgfqpoint{2.714045in}{1.907653in}}{\pgfqpoint{2.708221in}{1.901829in}}%
\pgfpathcurveto{\pgfqpoint{2.702397in}{1.896005in}}{\pgfqpoint{2.699125in}{1.888105in}}{\pgfqpoint{2.699125in}{1.879869in}}%
\pgfpathcurveto{\pgfqpoint{2.699125in}{1.871632in}}{\pgfqpoint{2.702397in}{1.863732in}}{\pgfqpoint{2.708221in}{1.857908in}}%
\pgfpathcurveto{\pgfqpoint{2.714045in}{1.852084in}}{\pgfqpoint{2.721945in}{1.848812in}}{\pgfqpoint{2.730181in}{1.848812in}}%
\pgfpathclose%
\pgfusepath{stroke,fill}%
\end{pgfscope}%
\begin{pgfscope}%
\pgfpathrectangle{\pgfqpoint{0.100000in}{0.212622in}}{\pgfqpoint{3.696000in}{3.696000in}}%
\pgfusepath{clip}%
\pgfsetbuttcap%
\pgfsetroundjoin%
\definecolor{currentfill}{rgb}{0.121569,0.466667,0.705882}%
\pgfsetfillcolor{currentfill}%
\pgfsetfillopacity{0.857408}%
\pgfsetlinewidth{1.003750pt}%
\definecolor{currentstroke}{rgb}{0.121569,0.466667,0.705882}%
\pgfsetstrokecolor{currentstroke}%
\pgfsetstrokeopacity{0.857408}%
\pgfsetdash{}{0pt}%
\pgfpathmoveto{\pgfqpoint{2.724312in}{1.850447in}}%
\pgfpathcurveto{\pgfqpoint{2.732549in}{1.850447in}}{\pgfqpoint{2.740449in}{1.853720in}}{\pgfqpoint{2.746273in}{1.859543in}}%
\pgfpathcurveto{\pgfqpoint{2.752097in}{1.865367in}}{\pgfqpoint{2.755369in}{1.873267in}}{\pgfqpoint{2.755369in}{1.881504in}}%
\pgfpathcurveto{\pgfqpoint{2.755369in}{1.889740in}}{\pgfqpoint{2.752097in}{1.897640in}}{\pgfqpoint{2.746273in}{1.903464in}}%
\pgfpathcurveto{\pgfqpoint{2.740449in}{1.909288in}}{\pgfqpoint{2.732549in}{1.912560in}}{\pgfqpoint{2.724312in}{1.912560in}}%
\pgfpathcurveto{\pgfqpoint{2.716076in}{1.912560in}}{\pgfqpoint{2.708176in}{1.909288in}}{\pgfqpoint{2.702352in}{1.903464in}}%
\pgfpathcurveto{\pgfqpoint{2.696528in}{1.897640in}}{\pgfqpoint{2.693256in}{1.889740in}}{\pgfqpoint{2.693256in}{1.881504in}}%
\pgfpathcurveto{\pgfqpoint{2.693256in}{1.873267in}}{\pgfqpoint{2.696528in}{1.865367in}}{\pgfqpoint{2.702352in}{1.859543in}}%
\pgfpathcurveto{\pgfqpoint{2.708176in}{1.853720in}}{\pgfqpoint{2.716076in}{1.850447in}}{\pgfqpoint{2.724312in}{1.850447in}}%
\pgfpathclose%
\pgfusepath{stroke,fill}%
\end{pgfscope}%
\begin{pgfscope}%
\pgfpathrectangle{\pgfqpoint{0.100000in}{0.212622in}}{\pgfqpoint{3.696000in}{3.696000in}}%
\pgfusepath{clip}%
\pgfsetbuttcap%
\pgfsetroundjoin%
\definecolor{currentfill}{rgb}{0.121569,0.466667,0.705882}%
\pgfsetfillcolor{currentfill}%
\pgfsetfillopacity{0.858027}%
\pgfsetlinewidth{1.003750pt}%
\definecolor{currentstroke}{rgb}{0.121569,0.466667,0.705882}%
\pgfsetstrokecolor{currentstroke}%
\pgfsetstrokeopacity{0.858027}%
\pgfsetdash{}{0pt}%
\pgfpathmoveto{\pgfqpoint{0.914068in}{2.443022in}}%
\pgfpathcurveto{\pgfqpoint{0.922304in}{2.443022in}}{\pgfqpoint{0.930204in}{2.446294in}}{\pgfqpoint{0.936028in}{2.452118in}}%
\pgfpathcurveto{\pgfqpoint{0.941852in}{2.457942in}}{\pgfqpoint{0.945124in}{2.465842in}}{\pgfqpoint{0.945124in}{2.474078in}}%
\pgfpathcurveto{\pgfqpoint{0.945124in}{2.482314in}}{\pgfqpoint{0.941852in}{2.490214in}}{\pgfqpoint{0.936028in}{2.496038in}}%
\pgfpathcurveto{\pgfqpoint{0.930204in}{2.501862in}}{\pgfqpoint{0.922304in}{2.505135in}}{\pgfqpoint{0.914068in}{2.505135in}}%
\pgfpathcurveto{\pgfqpoint{0.905831in}{2.505135in}}{\pgfqpoint{0.897931in}{2.501862in}}{\pgfqpoint{0.892107in}{2.496038in}}%
\pgfpathcurveto{\pgfqpoint{0.886283in}{2.490214in}}{\pgfqpoint{0.883011in}{2.482314in}}{\pgfqpoint{0.883011in}{2.474078in}}%
\pgfpathcurveto{\pgfqpoint{0.883011in}{2.465842in}}{\pgfqpoint{0.886283in}{2.457942in}}{\pgfqpoint{0.892107in}{2.452118in}}%
\pgfpathcurveto{\pgfqpoint{0.897931in}{2.446294in}}{\pgfqpoint{0.905831in}{2.443022in}}{\pgfqpoint{0.914068in}{2.443022in}}%
\pgfpathclose%
\pgfusepath{stroke,fill}%
\end{pgfscope}%
\begin{pgfscope}%
\pgfpathrectangle{\pgfqpoint{0.100000in}{0.212622in}}{\pgfqpoint{3.696000in}{3.696000in}}%
\pgfusepath{clip}%
\pgfsetbuttcap%
\pgfsetroundjoin%
\definecolor{currentfill}{rgb}{0.121569,0.466667,0.705882}%
\pgfsetfillcolor{currentfill}%
\pgfsetfillopacity{0.859926}%
\pgfsetlinewidth{1.003750pt}%
\definecolor{currentstroke}{rgb}{0.121569,0.466667,0.705882}%
\pgfsetstrokecolor{currentstroke}%
\pgfsetstrokeopacity{0.859926}%
\pgfsetdash{}{0pt}%
\pgfpathmoveto{\pgfqpoint{0.939396in}{2.435057in}}%
\pgfpathcurveto{\pgfqpoint{0.947632in}{2.435057in}}{\pgfqpoint{0.955532in}{2.438330in}}{\pgfqpoint{0.961356in}{2.444154in}}%
\pgfpathcurveto{\pgfqpoint{0.967180in}{2.449977in}}{\pgfqpoint{0.970452in}{2.457878in}}{\pgfqpoint{0.970452in}{2.466114in}}%
\pgfpathcurveto{\pgfqpoint{0.970452in}{2.474350in}}{\pgfqpoint{0.967180in}{2.482250in}}{\pgfqpoint{0.961356in}{2.488074in}}%
\pgfpathcurveto{\pgfqpoint{0.955532in}{2.493898in}}{\pgfqpoint{0.947632in}{2.497170in}}{\pgfqpoint{0.939396in}{2.497170in}}%
\pgfpathcurveto{\pgfqpoint{0.931159in}{2.497170in}}{\pgfqpoint{0.923259in}{2.493898in}}{\pgfqpoint{0.917435in}{2.488074in}}%
\pgfpathcurveto{\pgfqpoint{0.911611in}{2.482250in}}{\pgfqpoint{0.908339in}{2.474350in}}{\pgfqpoint{0.908339in}{2.466114in}}%
\pgfpathcurveto{\pgfqpoint{0.908339in}{2.457878in}}{\pgfqpoint{0.911611in}{2.449977in}}{\pgfqpoint{0.917435in}{2.444154in}}%
\pgfpathcurveto{\pgfqpoint{0.923259in}{2.438330in}}{\pgfqpoint{0.931159in}{2.435057in}}{\pgfqpoint{0.939396in}{2.435057in}}%
\pgfpathclose%
\pgfusepath{stroke,fill}%
\end{pgfscope}%
\begin{pgfscope}%
\pgfpathrectangle{\pgfqpoint{0.100000in}{0.212622in}}{\pgfqpoint{3.696000in}{3.696000in}}%
\pgfusepath{clip}%
\pgfsetbuttcap%
\pgfsetroundjoin%
\definecolor{currentfill}{rgb}{0.121569,0.466667,0.705882}%
\pgfsetfillcolor{currentfill}%
\pgfsetfillopacity{0.860871}%
\pgfsetlinewidth{1.003750pt}%
\definecolor{currentstroke}{rgb}{0.121569,0.466667,0.705882}%
\pgfsetstrokecolor{currentstroke}%
\pgfsetstrokeopacity{0.860871}%
\pgfsetdash{}{0pt}%
\pgfpathmoveto{\pgfqpoint{2.719011in}{1.851046in}}%
\pgfpathcurveto{\pgfqpoint{2.727247in}{1.851046in}}{\pgfqpoint{2.735147in}{1.854318in}}{\pgfqpoint{2.740971in}{1.860142in}}%
\pgfpathcurveto{\pgfqpoint{2.746795in}{1.865966in}}{\pgfqpoint{2.750068in}{1.873866in}}{\pgfqpoint{2.750068in}{1.882102in}}%
\pgfpathcurveto{\pgfqpoint{2.750068in}{1.890338in}}{\pgfqpoint{2.746795in}{1.898238in}}{\pgfqpoint{2.740971in}{1.904062in}}%
\pgfpathcurveto{\pgfqpoint{2.735147in}{1.909886in}}{\pgfqpoint{2.727247in}{1.913159in}}{\pgfqpoint{2.719011in}{1.913159in}}%
\pgfpathcurveto{\pgfqpoint{2.710775in}{1.913159in}}{\pgfqpoint{2.702875in}{1.909886in}}{\pgfqpoint{2.697051in}{1.904062in}}%
\pgfpathcurveto{\pgfqpoint{2.691227in}{1.898238in}}{\pgfqpoint{2.687955in}{1.890338in}}{\pgfqpoint{2.687955in}{1.882102in}}%
\pgfpathcurveto{\pgfqpoint{2.687955in}{1.873866in}}{\pgfqpoint{2.691227in}{1.865966in}}{\pgfqpoint{2.697051in}{1.860142in}}%
\pgfpathcurveto{\pgfqpoint{2.702875in}{1.854318in}}{\pgfqpoint{2.710775in}{1.851046in}}{\pgfqpoint{2.719011in}{1.851046in}}%
\pgfpathclose%
\pgfusepath{stroke,fill}%
\end{pgfscope}%
\begin{pgfscope}%
\pgfpathrectangle{\pgfqpoint{0.100000in}{0.212622in}}{\pgfqpoint{3.696000in}{3.696000in}}%
\pgfusepath{clip}%
\pgfsetbuttcap%
\pgfsetroundjoin%
\definecolor{currentfill}{rgb}{0.121569,0.466667,0.705882}%
\pgfsetfillcolor{currentfill}%
\pgfsetfillopacity{0.862028}%
\pgfsetlinewidth{1.003750pt}%
\definecolor{currentstroke}{rgb}{0.121569,0.466667,0.705882}%
\pgfsetstrokecolor{currentstroke}%
\pgfsetstrokeopacity{0.862028}%
\pgfsetdash{}{0pt}%
\pgfpathmoveto{\pgfqpoint{0.963944in}{2.426872in}}%
\pgfpathcurveto{\pgfqpoint{0.972181in}{2.426872in}}{\pgfqpoint{0.980081in}{2.430144in}}{\pgfqpoint{0.985905in}{2.435968in}}%
\pgfpathcurveto{\pgfqpoint{0.991728in}{2.441792in}}{\pgfqpoint{0.995001in}{2.449692in}}{\pgfqpoint{0.995001in}{2.457928in}}%
\pgfpathcurveto{\pgfqpoint{0.995001in}{2.466165in}}{\pgfqpoint{0.991728in}{2.474065in}}{\pgfqpoint{0.985905in}{2.479889in}}%
\pgfpathcurveto{\pgfqpoint{0.980081in}{2.485712in}}{\pgfqpoint{0.972181in}{2.488985in}}{\pgfqpoint{0.963944in}{2.488985in}}%
\pgfpathcurveto{\pgfqpoint{0.955708in}{2.488985in}}{\pgfqpoint{0.947808in}{2.485712in}}{\pgfqpoint{0.941984in}{2.479889in}}%
\pgfpathcurveto{\pgfqpoint{0.936160in}{2.474065in}}{\pgfqpoint{0.932888in}{2.466165in}}{\pgfqpoint{0.932888in}{2.457928in}}%
\pgfpathcurveto{\pgfqpoint{0.932888in}{2.449692in}}{\pgfqpoint{0.936160in}{2.441792in}}{\pgfqpoint{0.941984in}{2.435968in}}%
\pgfpathcurveto{\pgfqpoint{0.947808in}{2.430144in}}{\pgfqpoint{0.955708in}{2.426872in}}{\pgfqpoint{0.963944in}{2.426872in}}%
\pgfpathclose%
\pgfusepath{stroke,fill}%
\end{pgfscope}%
\begin{pgfscope}%
\pgfpathrectangle{\pgfqpoint{0.100000in}{0.212622in}}{\pgfqpoint{3.696000in}{3.696000in}}%
\pgfusepath{clip}%
\pgfsetbuttcap%
\pgfsetroundjoin%
\definecolor{currentfill}{rgb}{0.121569,0.466667,0.705882}%
\pgfsetfillcolor{currentfill}%
\pgfsetfillopacity{0.862556}%
\pgfsetlinewidth{1.003750pt}%
\definecolor{currentstroke}{rgb}{0.121569,0.466667,0.705882}%
\pgfsetstrokecolor{currentstroke}%
\pgfsetstrokeopacity{0.862556}%
\pgfsetdash{}{0pt}%
\pgfpathmoveto{\pgfqpoint{2.714347in}{1.852173in}}%
\pgfpathcurveto{\pgfqpoint{2.722583in}{1.852173in}}{\pgfqpoint{2.730483in}{1.855445in}}{\pgfqpoint{2.736307in}{1.861269in}}%
\pgfpathcurveto{\pgfqpoint{2.742131in}{1.867093in}}{\pgfqpoint{2.745403in}{1.874993in}}{\pgfqpoint{2.745403in}{1.883229in}}%
\pgfpathcurveto{\pgfqpoint{2.745403in}{1.891466in}}{\pgfqpoint{2.742131in}{1.899366in}}{\pgfqpoint{2.736307in}{1.905190in}}%
\pgfpathcurveto{\pgfqpoint{2.730483in}{1.911013in}}{\pgfqpoint{2.722583in}{1.914286in}}{\pgfqpoint{2.714347in}{1.914286in}}%
\pgfpathcurveto{\pgfqpoint{2.706111in}{1.914286in}}{\pgfqpoint{2.698211in}{1.911013in}}{\pgfqpoint{2.692387in}{1.905190in}}%
\pgfpathcurveto{\pgfqpoint{2.686563in}{1.899366in}}{\pgfqpoint{2.683290in}{1.891466in}}{\pgfqpoint{2.683290in}{1.883229in}}%
\pgfpathcurveto{\pgfqpoint{2.683290in}{1.874993in}}{\pgfqpoint{2.686563in}{1.867093in}}{\pgfqpoint{2.692387in}{1.861269in}}%
\pgfpathcurveto{\pgfqpoint{2.698211in}{1.855445in}}{\pgfqpoint{2.706111in}{1.852173in}}{\pgfqpoint{2.714347in}{1.852173in}}%
\pgfpathclose%
\pgfusepath{stroke,fill}%
\end{pgfscope}%
\begin{pgfscope}%
\pgfpathrectangle{\pgfqpoint{0.100000in}{0.212622in}}{\pgfqpoint{3.696000in}{3.696000in}}%
\pgfusepath{clip}%
\pgfsetbuttcap%
\pgfsetroundjoin%
\definecolor{currentfill}{rgb}{0.121569,0.466667,0.705882}%
\pgfsetfillcolor{currentfill}%
\pgfsetfillopacity{0.864041}%
\pgfsetlinewidth{1.003750pt}%
\definecolor{currentstroke}{rgb}{0.121569,0.466667,0.705882}%
\pgfsetstrokecolor{currentstroke}%
\pgfsetstrokeopacity{0.864041}%
\pgfsetdash{}{0pt}%
\pgfpathmoveto{\pgfqpoint{0.986320in}{2.414859in}}%
\pgfpathcurveto{\pgfqpoint{0.994557in}{2.414859in}}{\pgfqpoint{1.002457in}{2.418132in}}{\pgfqpoint{1.008281in}{2.423956in}}%
\pgfpathcurveto{\pgfqpoint{1.014105in}{2.429779in}}{\pgfqpoint{1.017377in}{2.437679in}}{\pgfqpoint{1.017377in}{2.445916in}}%
\pgfpathcurveto{\pgfqpoint{1.017377in}{2.454152in}}{\pgfqpoint{1.014105in}{2.462052in}}{\pgfqpoint{1.008281in}{2.467876in}}%
\pgfpathcurveto{\pgfqpoint{1.002457in}{2.473700in}}{\pgfqpoint{0.994557in}{2.476972in}}{\pgfqpoint{0.986320in}{2.476972in}}%
\pgfpathcurveto{\pgfqpoint{0.978084in}{2.476972in}}{\pgfqpoint{0.970184in}{2.473700in}}{\pgfqpoint{0.964360in}{2.467876in}}%
\pgfpathcurveto{\pgfqpoint{0.958536in}{2.462052in}}{\pgfqpoint{0.955264in}{2.454152in}}{\pgfqpoint{0.955264in}{2.445916in}}%
\pgfpathcurveto{\pgfqpoint{0.955264in}{2.437679in}}{\pgfqpoint{0.958536in}{2.429779in}}{\pgfqpoint{0.964360in}{2.423956in}}%
\pgfpathcurveto{\pgfqpoint{0.970184in}{2.418132in}}{\pgfqpoint{0.978084in}{2.414859in}}{\pgfqpoint{0.986320in}{2.414859in}}%
\pgfpathclose%
\pgfusepath{stroke,fill}%
\end{pgfscope}%
\begin{pgfscope}%
\pgfpathrectangle{\pgfqpoint{0.100000in}{0.212622in}}{\pgfqpoint{3.696000in}{3.696000in}}%
\pgfusepath{clip}%
\pgfsetbuttcap%
\pgfsetroundjoin%
\definecolor{currentfill}{rgb}{0.121569,0.466667,0.705882}%
\pgfsetfillcolor{currentfill}%
\pgfsetfillopacity{0.864396}%
\pgfsetlinewidth{1.003750pt}%
\definecolor{currentstroke}{rgb}{0.121569,0.466667,0.705882}%
\pgfsetstrokecolor{currentstroke}%
\pgfsetstrokeopacity{0.864396}%
\pgfsetdash{}{0pt}%
\pgfpathmoveto{\pgfqpoint{2.709374in}{1.853417in}}%
\pgfpathcurveto{\pgfqpoint{2.717610in}{1.853417in}}{\pgfqpoint{2.725510in}{1.856690in}}{\pgfqpoint{2.731334in}{1.862514in}}%
\pgfpathcurveto{\pgfqpoint{2.737158in}{1.868338in}}{\pgfqpoint{2.740430in}{1.876238in}}{\pgfqpoint{2.740430in}{1.884474in}}%
\pgfpathcurveto{\pgfqpoint{2.740430in}{1.892710in}}{\pgfqpoint{2.737158in}{1.900610in}}{\pgfqpoint{2.731334in}{1.906434in}}%
\pgfpathcurveto{\pgfqpoint{2.725510in}{1.912258in}}{\pgfqpoint{2.717610in}{1.915530in}}{\pgfqpoint{2.709374in}{1.915530in}}%
\pgfpathcurveto{\pgfqpoint{2.701137in}{1.915530in}}{\pgfqpoint{2.693237in}{1.912258in}}{\pgfqpoint{2.687413in}{1.906434in}}%
\pgfpathcurveto{\pgfqpoint{2.681590in}{1.900610in}}{\pgfqpoint{2.678317in}{1.892710in}}{\pgfqpoint{2.678317in}{1.884474in}}%
\pgfpathcurveto{\pgfqpoint{2.678317in}{1.876238in}}{\pgfqpoint{2.681590in}{1.868338in}}{\pgfqpoint{2.687413in}{1.862514in}}%
\pgfpathcurveto{\pgfqpoint{2.693237in}{1.856690in}}{\pgfqpoint{2.701137in}{1.853417in}}{\pgfqpoint{2.709374in}{1.853417in}}%
\pgfpathclose%
\pgfusepath{stroke,fill}%
\end{pgfscope}%
\begin{pgfscope}%
\pgfpathrectangle{\pgfqpoint{0.100000in}{0.212622in}}{\pgfqpoint{3.696000in}{3.696000in}}%
\pgfusepath{clip}%
\pgfsetbuttcap%
\pgfsetroundjoin%
\definecolor{currentfill}{rgb}{0.121569,0.466667,0.705882}%
\pgfsetfillcolor{currentfill}%
\pgfsetfillopacity{0.865718}%
\pgfsetlinewidth{1.003750pt}%
\definecolor{currentstroke}{rgb}{0.121569,0.466667,0.705882}%
\pgfsetstrokecolor{currentstroke}%
\pgfsetstrokeopacity{0.865718}%
\pgfsetdash{}{0pt}%
\pgfpathmoveto{\pgfqpoint{1.006933in}{2.408315in}}%
\pgfpathcurveto{\pgfqpoint{1.015170in}{2.408315in}}{\pgfqpoint{1.023070in}{2.411587in}}{\pgfqpoint{1.028894in}{2.417411in}}%
\pgfpathcurveto{\pgfqpoint{1.034717in}{2.423235in}}{\pgfqpoint{1.037990in}{2.431135in}}{\pgfqpoint{1.037990in}{2.439371in}}%
\pgfpathcurveto{\pgfqpoint{1.037990in}{2.447607in}}{\pgfqpoint{1.034717in}{2.455508in}}{\pgfqpoint{1.028894in}{2.461331in}}%
\pgfpathcurveto{\pgfqpoint{1.023070in}{2.467155in}}{\pgfqpoint{1.015170in}{2.470428in}}{\pgfqpoint{1.006933in}{2.470428in}}%
\pgfpathcurveto{\pgfqpoint{0.998697in}{2.470428in}}{\pgfqpoint{0.990797in}{2.467155in}}{\pgfqpoint{0.984973in}{2.461331in}}%
\pgfpathcurveto{\pgfqpoint{0.979149in}{2.455508in}}{\pgfqpoint{0.975877in}{2.447607in}}{\pgfqpoint{0.975877in}{2.439371in}}%
\pgfpathcurveto{\pgfqpoint{0.975877in}{2.431135in}}{\pgfqpoint{0.979149in}{2.423235in}}{\pgfqpoint{0.984973in}{2.417411in}}%
\pgfpathcurveto{\pgfqpoint{0.990797in}{2.411587in}}{\pgfqpoint{0.998697in}{2.408315in}}{\pgfqpoint{1.006933in}{2.408315in}}%
\pgfpathclose%
\pgfusepath{stroke,fill}%
\end{pgfscope}%
\begin{pgfscope}%
\pgfpathrectangle{\pgfqpoint{0.100000in}{0.212622in}}{\pgfqpoint{3.696000in}{3.696000in}}%
\pgfusepath{clip}%
\pgfsetbuttcap%
\pgfsetroundjoin%
\definecolor{currentfill}{rgb}{0.121569,0.466667,0.705882}%
\pgfsetfillcolor{currentfill}%
\pgfsetfillopacity{0.866942}%
\pgfsetlinewidth{1.003750pt}%
\definecolor{currentstroke}{rgb}{0.121569,0.466667,0.705882}%
\pgfsetstrokecolor{currentstroke}%
\pgfsetstrokeopacity{0.866942}%
\pgfsetdash{}{0pt}%
\pgfpathmoveto{\pgfqpoint{2.704029in}{1.854004in}}%
\pgfpathcurveto{\pgfqpoint{2.712265in}{1.854004in}}{\pgfqpoint{2.720165in}{1.857276in}}{\pgfqpoint{2.725989in}{1.863100in}}%
\pgfpathcurveto{\pgfqpoint{2.731813in}{1.868924in}}{\pgfqpoint{2.735085in}{1.876824in}}{\pgfqpoint{2.735085in}{1.885060in}}%
\pgfpathcurveto{\pgfqpoint{2.735085in}{1.893297in}}{\pgfqpoint{2.731813in}{1.901197in}}{\pgfqpoint{2.725989in}{1.907021in}}%
\pgfpathcurveto{\pgfqpoint{2.720165in}{1.912845in}}{\pgfqpoint{2.712265in}{1.916117in}}{\pgfqpoint{2.704029in}{1.916117in}}%
\pgfpathcurveto{\pgfqpoint{2.695792in}{1.916117in}}{\pgfqpoint{2.687892in}{1.912845in}}{\pgfqpoint{2.682068in}{1.907021in}}%
\pgfpathcurveto{\pgfqpoint{2.676244in}{1.901197in}}{\pgfqpoint{2.672972in}{1.893297in}}{\pgfqpoint{2.672972in}{1.885060in}}%
\pgfpathcurveto{\pgfqpoint{2.672972in}{1.876824in}}{\pgfqpoint{2.676244in}{1.868924in}}{\pgfqpoint{2.682068in}{1.863100in}}%
\pgfpathcurveto{\pgfqpoint{2.687892in}{1.857276in}}{\pgfqpoint{2.695792in}{1.854004in}}{\pgfqpoint{2.704029in}{1.854004in}}%
\pgfpathclose%
\pgfusepath{stroke,fill}%
\end{pgfscope}%
\begin{pgfscope}%
\pgfpathrectangle{\pgfqpoint{0.100000in}{0.212622in}}{\pgfqpoint{3.696000in}{3.696000in}}%
\pgfusepath{clip}%
\pgfsetbuttcap%
\pgfsetroundjoin%
\definecolor{currentfill}{rgb}{0.121569,0.466667,0.705882}%
\pgfsetfillcolor{currentfill}%
\pgfsetfillopacity{0.868123}%
\pgfsetlinewidth{1.003750pt}%
\definecolor{currentstroke}{rgb}{0.121569,0.466667,0.705882}%
\pgfsetstrokecolor{currentstroke}%
\pgfsetstrokeopacity{0.868123}%
\pgfsetdash{}{0pt}%
\pgfpathmoveto{\pgfqpoint{1.025707in}{2.402048in}}%
\pgfpathcurveto{\pgfqpoint{1.033943in}{2.402048in}}{\pgfqpoint{1.041843in}{2.405320in}}{\pgfqpoint{1.047667in}{2.411144in}}%
\pgfpathcurveto{\pgfqpoint{1.053491in}{2.416968in}}{\pgfqpoint{1.056763in}{2.424868in}}{\pgfqpoint{1.056763in}{2.433105in}}%
\pgfpathcurveto{\pgfqpoint{1.056763in}{2.441341in}}{\pgfqpoint{1.053491in}{2.449241in}}{\pgfqpoint{1.047667in}{2.455065in}}%
\pgfpathcurveto{\pgfqpoint{1.041843in}{2.460889in}}{\pgfqpoint{1.033943in}{2.464161in}}{\pgfqpoint{1.025707in}{2.464161in}}%
\pgfpathcurveto{\pgfqpoint{1.017471in}{2.464161in}}{\pgfqpoint{1.009570in}{2.460889in}}{\pgfqpoint{1.003747in}{2.455065in}}%
\pgfpathcurveto{\pgfqpoint{0.997923in}{2.449241in}}{\pgfqpoint{0.994650in}{2.441341in}}{\pgfqpoint{0.994650in}{2.433105in}}%
\pgfpathcurveto{\pgfqpoint{0.994650in}{2.424868in}}{\pgfqpoint{0.997923in}{2.416968in}}{\pgfqpoint{1.003747in}{2.411144in}}%
\pgfpathcurveto{\pgfqpoint{1.009570in}{2.405320in}}{\pgfqpoint{1.017471in}{2.402048in}}{\pgfqpoint{1.025707in}{2.402048in}}%
\pgfpathclose%
\pgfusepath{stroke,fill}%
\end{pgfscope}%
\begin{pgfscope}%
\pgfpathrectangle{\pgfqpoint{0.100000in}{0.212622in}}{\pgfqpoint{3.696000in}{3.696000in}}%
\pgfusepath{clip}%
\pgfsetbuttcap%
\pgfsetroundjoin%
\definecolor{currentfill}{rgb}{0.121569,0.466667,0.705882}%
\pgfsetfillcolor{currentfill}%
\pgfsetfillopacity{0.869567}%
\pgfsetlinewidth{1.003750pt}%
\definecolor{currentstroke}{rgb}{0.121569,0.466667,0.705882}%
\pgfsetstrokecolor{currentstroke}%
\pgfsetstrokeopacity{0.869567}%
\pgfsetdash{}{0pt}%
\pgfpathmoveto{\pgfqpoint{1.045522in}{2.394537in}}%
\pgfpathcurveto{\pgfqpoint{1.053759in}{2.394537in}}{\pgfqpoint{1.061659in}{2.397809in}}{\pgfqpoint{1.067483in}{2.403633in}}%
\pgfpathcurveto{\pgfqpoint{1.073307in}{2.409457in}}{\pgfqpoint{1.076579in}{2.417357in}}{\pgfqpoint{1.076579in}{2.425593in}}%
\pgfpathcurveto{\pgfqpoint{1.076579in}{2.433830in}}{\pgfqpoint{1.073307in}{2.441730in}}{\pgfqpoint{1.067483in}{2.447554in}}%
\pgfpathcurveto{\pgfqpoint{1.061659in}{2.453377in}}{\pgfqpoint{1.053759in}{2.456650in}}{\pgfqpoint{1.045522in}{2.456650in}}%
\pgfpathcurveto{\pgfqpoint{1.037286in}{2.456650in}}{\pgfqpoint{1.029386in}{2.453377in}}{\pgfqpoint{1.023562in}{2.447554in}}%
\pgfpathcurveto{\pgfqpoint{1.017738in}{2.441730in}}{\pgfqpoint{1.014466in}{2.433830in}}{\pgfqpoint{1.014466in}{2.425593in}}%
\pgfpathcurveto{\pgfqpoint{1.014466in}{2.417357in}}{\pgfqpoint{1.017738in}{2.409457in}}{\pgfqpoint{1.023562in}{2.403633in}}%
\pgfpathcurveto{\pgfqpoint{1.029386in}{2.397809in}}{\pgfqpoint{1.037286in}{2.394537in}}{\pgfqpoint{1.045522in}{2.394537in}}%
\pgfpathclose%
\pgfusepath{stroke,fill}%
\end{pgfscope}%
\begin{pgfscope}%
\pgfpathrectangle{\pgfqpoint{0.100000in}{0.212622in}}{\pgfqpoint{3.696000in}{3.696000in}}%
\pgfusepath{clip}%
\pgfsetbuttcap%
\pgfsetroundjoin%
\definecolor{currentfill}{rgb}{0.121569,0.466667,0.705882}%
\pgfsetfillcolor{currentfill}%
\pgfsetfillopacity{0.869661}%
\pgfsetlinewidth{1.003750pt}%
\definecolor{currentstroke}{rgb}{0.121569,0.466667,0.705882}%
\pgfsetstrokecolor{currentstroke}%
\pgfsetstrokeopacity{0.869661}%
\pgfsetdash{}{0pt}%
\pgfpathmoveto{\pgfqpoint{2.696426in}{1.855971in}}%
\pgfpathcurveto{\pgfqpoint{2.704662in}{1.855971in}}{\pgfqpoint{2.712562in}{1.859243in}}{\pgfqpoint{2.718386in}{1.865067in}}%
\pgfpathcurveto{\pgfqpoint{2.724210in}{1.870891in}}{\pgfqpoint{2.727482in}{1.878791in}}{\pgfqpoint{2.727482in}{1.887027in}}%
\pgfpathcurveto{\pgfqpoint{2.727482in}{1.895263in}}{\pgfqpoint{2.724210in}{1.903164in}}{\pgfqpoint{2.718386in}{1.908987in}}%
\pgfpathcurveto{\pgfqpoint{2.712562in}{1.914811in}}{\pgfqpoint{2.704662in}{1.918084in}}{\pgfqpoint{2.696426in}{1.918084in}}%
\pgfpathcurveto{\pgfqpoint{2.688189in}{1.918084in}}{\pgfqpoint{2.680289in}{1.914811in}}{\pgfqpoint{2.674465in}{1.908987in}}%
\pgfpathcurveto{\pgfqpoint{2.668641in}{1.903164in}}{\pgfqpoint{2.665369in}{1.895263in}}{\pgfqpoint{2.665369in}{1.887027in}}%
\pgfpathcurveto{\pgfqpoint{2.665369in}{1.878791in}}{\pgfqpoint{2.668641in}{1.870891in}}{\pgfqpoint{2.674465in}{1.865067in}}%
\pgfpathcurveto{\pgfqpoint{2.680289in}{1.859243in}}{\pgfqpoint{2.688189in}{1.855971in}}{\pgfqpoint{2.696426in}{1.855971in}}%
\pgfpathclose%
\pgfusepath{stroke,fill}%
\end{pgfscope}%
\begin{pgfscope}%
\pgfpathrectangle{\pgfqpoint{0.100000in}{0.212622in}}{\pgfqpoint{3.696000in}{3.696000in}}%
\pgfusepath{clip}%
\pgfsetbuttcap%
\pgfsetroundjoin%
\definecolor{currentfill}{rgb}{0.121569,0.466667,0.705882}%
\pgfsetfillcolor{currentfill}%
\pgfsetfillopacity{0.871092}%
\pgfsetlinewidth{1.003750pt}%
\definecolor{currentstroke}{rgb}{0.121569,0.466667,0.705882}%
\pgfsetstrokecolor{currentstroke}%
\pgfsetstrokeopacity{0.871092}%
\pgfsetdash{}{0pt}%
\pgfpathmoveto{\pgfqpoint{1.062844in}{2.389180in}}%
\pgfpathcurveto{\pgfqpoint{1.071081in}{2.389180in}}{\pgfqpoint{1.078981in}{2.392452in}}{\pgfqpoint{1.084805in}{2.398276in}}%
\pgfpathcurveto{\pgfqpoint{1.090629in}{2.404100in}}{\pgfqpoint{1.093901in}{2.412000in}}{\pgfqpoint{1.093901in}{2.420236in}}%
\pgfpathcurveto{\pgfqpoint{1.093901in}{2.428472in}}{\pgfqpoint{1.090629in}{2.436372in}}{\pgfqpoint{1.084805in}{2.442196in}}%
\pgfpathcurveto{\pgfqpoint{1.078981in}{2.448020in}}{\pgfqpoint{1.071081in}{2.451293in}}{\pgfqpoint{1.062844in}{2.451293in}}%
\pgfpathcurveto{\pgfqpoint{1.054608in}{2.451293in}}{\pgfqpoint{1.046708in}{2.448020in}}{\pgfqpoint{1.040884in}{2.442196in}}%
\pgfpathcurveto{\pgfqpoint{1.035060in}{2.436372in}}{\pgfqpoint{1.031788in}{2.428472in}}{\pgfqpoint{1.031788in}{2.420236in}}%
\pgfpathcurveto{\pgfqpoint{1.031788in}{2.412000in}}{\pgfqpoint{1.035060in}{2.404100in}}{\pgfqpoint{1.040884in}{2.398276in}}%
\pgfpathcurveto{\pgfqpoint{1.046708in}{2.392452in}}{\pgfqpoint{1.054608in}{2.389180in}}{\pgfqpoint{1.062844in}{2.389180in}}%
\pgfpathclose%
\pgfusepath{stroke,fill}%
\end{pgfscope}%
\begin{pgfscope}%
\pgfpathrectangle{\pgfqpoint{0.100000in}{0.212622in}}{\pgfqpoint{3.696000in}{3.696000in}}%
\pgfusepath{clip}%
\pgfsetbuttcap%
\pgfsetroundjoin%
\definecolor{currentfill}{rgb}{0.121569,0.466667,0.705882}%
\pgfsetfillcolor{currentfill}%
\pgfsetfillopacity{0.872589}%
\pgfsetlinewidth{1.003750pt}%
\definecolor{currentstroke}{rgb}{0.121569,0.466667,0.705882}%
\pgfsetstrokecolor{currentstroke}%
\pgfsetstrokeopacity{0.872589}%
\pgfsetdash{}{0pt}%
\pgfpathmoveto{\pgfqpoint{2.687943in}{1.858407in}}%
\pgfpathcurveto{\pgfqpoint{2.696180in}{1.858407in}}{\pgfqpoint{2.704080in}{1.861679in}}{\pgfqpoint{2.709904in}{1.867503in}}%
\pgfpathcurveto{\pgfqpoint{2.715728in}{1.873327in}}{\pgfqpoint{2.719000in}{1.881227in}}{\pgfqpoint{2.719000in}{1.889464in}}%
\pgfpathcurveto{\pgfqpoint{2.719000in}{1.897700in}}{\pgfqpoint{2.715728in}{1.905600in}}{\pgfqpoint{2.709904in}{1.911424in}}%
\pgfpathcurveto{\pgfqpoint{2.704080in}{1.917248in}}{\pgfqpoint{2.696180in}{1.920520in}}{\pgfqpoint{2.687943in}{1.920520in}}%
\pgfpathcurveto{\pgfqpoint{2.679707in}{1.920520in}}{\pgfqpoint{2.671807in}{1.917248in}}{\pgfqpoint{2.665983in}{1.911424in}}%
\pgfpathcurveto{\pgfqpoint{2.660159in}{1.905600in}}{\pgfqpoint{2.656887in}{1.897700in}}{\pgfqpoint{2.656887in}{1.889464in}}%
\pgfpathcurveto{\pgfqpoint{2.656887in}{1.881227in}}{\pgfqpoint{2.660159in}{1.873327in}}{\pgfqpoint{2.665983in}{1.867503in}}%
\pgfpathcurveto{\pgfqpoint{2.671807in}{1.861679in}}{\pgfqpoint{2.679707in}{1.858407in}}{\pgfqpoint{2.687943in}{1.858407in}}%
\pgfpathclose%
\pgfusepath{stroke,fill}%
\end{pgfscope}%
\begin{pgfscope}%
\pgfpathrectangle{\pgfqpoint{0.100000in}{0.212622in}}{\pgfqpoint{3.696000in}{3.696000in}}%
\pgfusepath{clip}%
\pgfsetbuttcap%
\pgfsetroundjoin%
\definecolor{currentfill}{rgb}{0.121569,0.466667,0.705882}%
\pgfsetfillcolor{currentfill}%
\pgfsetfillopacity{0.872682}%
\pgfsetlinewidth{1.003750pt}%
\definecolor{currentstroke}{rgb}{0.121569,0.466667,0.705882}%
\pgfsetstrokecolor{currentstroke}%
\pgfsetstrokeopacity{0.872682}%
\pgfsetdash{}{0pt}%
\pgfpathmoveto{\pgfqpoint{1.077496in}{2.383243in}}%
\pgfpathcurveto{\pgfqpoint{1.085732in}{2.383243in}}{\pgfqpoint{1.093632in}{2.386515in}}{\pgfqpoint{1.099456in}{2.392339in}}%
\pgfpathcurveto{\pgfqpoint{1.105280in}{2.398163in}}{\pgfqpoint{1.108552in}{2.406063in}}{\pgfqpoint{1.108552in}{2.414300in}}%
\pgfpathcurveto{\pgfqpoint{1.108552in}{2.422536in}}{\pgfqpoint{1.105280in}{2.430436in}}{\pgfqpoint{1.099456in}{2.436260in}}%
\pgfpathcurveto{\pgfqpoint{1.093632in}{2.442084in}}{\pgfqpoint{1.085732in}{2.445356in}}{\pgfqpoint{1.077496in}{2.445356in}}%
\pgfpathcurveto{\pgfqpoint{1.069259in}{2.445356in}}{\pgfqpoint{1.061359in}{2.442084in}}{\pgfqpoint{1.055535in}{2.436260in}}%
\pgfpathcurveto{\pgfqpoint{1.049712in}{2.430436in}}{\pgfqpoint{1.046439in}{2.422536in}}{\pgfqpoint{1.046439in}{2.414300in}}%
\pgfpathcurveto{\pgfqpoint{1.046439in}{2.406063in}}{\pgfqpoint{1.049712in}{2.398163in}}{\pgfqpoint{1.055535in}{2.392339in}}%
\pgfpathcurveto{\pgfqpoint{1.061359in}{2.386515in}}{\pgfqpoint{1.069259in}{2.383243in}}{\pgfqpoint{1.077496in}{2.383243in}}%
\pgfpathclose%
\pgfusepath{stroke,fill}%
\end{pgfscope}%
\begin{pgfscope}%
\pgfpathrectangle{\pgfqpoint{0.100000in}{0.212622in}}{\pgfqpoint{3.696000in}{3.696000in}}%
\pgfusepath{clip}%
\pgfsetbuttcap%
\pgfsetroundjoin%
\definecolor{currentfill}{rgb}{0.121569,0.466667,0.705882}%
\pgfsetfillcolor{currentfill}%
\pgfsetfillopacity{0.874973}%
\pgfsetlinewidth{1.003750pt}%
\definecolor{currentstroke}{rgb}{0.121569,0.466667,0.705882}%
\pgfsetstrokecolor{currentstroke}%
\pgfsetstrokeopacity{0.874973}%
\pgfsetdash{}{0pt}%
\pgfpathmoveto{\pgfqpoint{1.105529in}{2.372979in}}%
\pgfpathcurveto{\pgfqpoint{1.113765in}{2.372979in}}{\pgfqpoint{1.121665in}{2.376252in}}{\pgfqpoint{1.127489in}{2.382075in}}%
\pgfpathcurveto{\pgfqpoint{1.133313in}{2.387899in}}{\pgfqpoint{1.136585in}{2.395799in}}{\pgfqpoint{1.136585in}{2.404036in}}%
\pgfpathcurveto{\pgfqpoint{1.136585in}{2.412272in}}{\pgfqpoint{1.133313in}{2.420172in}}{\pgfqpoint{1.127489in}{2.425996in}}%
\pgfpathcurveto{\pgfqpoint{1.121665in}{2.431820in}}{\pgfqpoint{1.113765in}{2.435092in}}{\pgfqpoint{1.105529in}{2.435092in}}%
\pgfpathcurveto{\pgfqpoint{1.097292in}{2.435092in}}{\pgfqpoint{1.089392in}{2.431820in}}{\pgfqpoint{1.083568in}{2.425996in}}%
\pgfpathcurveto{\pgfqpoint{1.077744in}{2.420172in}}{\pgfqpoint{1.074472in}{2.412272in}}{\pgfqpoint{1.074472in}{2.404036in}}%
\pgfpathcurveto{\pgfqpoint{1.074472in}{2.395799in}}{\pgfqpoint{1.077744in}{2.387899in}}{\pgfqpoint{1.083568in}{2.382075in}}%
\pgfpathcurveto{\pgfqpoint{1.089392in}{2.376252in}}{\pgfqpoint{1.097292in}{2.372979in}}{\pgfqpoint{1.105529in}{2.372979in}}%
\pgfpathclose%
\pgfusepath{stroke,fill}%
\end{pgfscope}%
\begin{pgfscope}%
\pgfpathrectangle{\pgfqpoint{0.100000in}{0.212622in}}{\pgfqpoint{3.696000in}{3.696000in}}%
\pgfusepath{clip}%
\pgfsetbuttcap%
\pgfsetroundjoin%
\definecolor{currentfill}{rgb}{0.121569,0.466667,0.705882}%
\pgfsetfillcolor{currentfill}%
\pgfsetfillopacity{0.876559}%
\pgfsetlinewidth{1.003750pt}%
\definecolor{currentstroke}{rgb}{0.121569,0.466667,0.705882}%
\pgfsetstrokecolor{currentstroke}%
\pgfsetstrokeopacity{0.876559}%
\pgfsetdash{}{0pt}%
\pgfpathmoveto{\pgfqpoint{2.681345in}{1.859024in}}%
\pgfpathcurveto{\pgfqpoint{2.689582in}{1.859024in}}{\pgfqpoint{2.697482in}{1.862296in}}{\pgfqpoint{2.703306in}{1.868120in}}%
\pgfpathcurveto{\pgfqpoint{2.709130in}{1.873944in}}{\pgfqpoint{2.712402in}{1.881844in}}{\pgfqpoint{2.712402in}{1.890080in}}%
\pgfpathcurveto{\pgfqpoint{2.712402in}{1.898317in}}{\pgfqpoint{2.709130in}{1.906217in}}{\pgfqpoint{2.703306in}{1.912041in}}%
\pgfpathcurveto{\pgfqpoint{2.697482in}{1.917865in}}{\pgfqpoint{2.689582in}{1.921137in}}{\pgfqpoint{2.681345in}{1.921137in}}%
\pgfpathcurveto{\pgfqpoint{2.673109in}{1.921137in}}{\pgfqpoint{2.665209in}{1.917865in}}{\pgfqpoint{2.659385in}{1.912041in}}%
\pgfpathcurveto{\pgfqpoint{2.653561in}{1.906217in}}{\pgfqpoint{2.650289in}{1.898317in}}{\pgfqpoint{2.650289in}{1.890080in}}%
\pgfpathcurveto{\pgfqpoint{2.650289in}{1.881844in}}{\pgfqpoint{2.653561in}{1.873944in}}{\pgfqpoint{2.659385in}{1.868120in}}%
\pgfpathcurveto{\pgfqpoint{2.665209in}{1.862296in}}{\pgfqpoint{2.673109in}{1.859024in}}{\pgfqpoint{2.681345in}{1.859024in}}%
\pgfpathclose%
\pgfusepath{stroke,fill}%
\end{pgfscope}%
\begin{pgfscope}%
\pgfpathrectangle{\pgfqpoint{0.100000in}{0.212622in}}{\pgfqpoint{3.696000in}{3.696000in}}%
\pgfusepath{clip}%
\pgfsetbuttcap%
\pgfsetroundjoin%
\definecolor{currentfill}{rgb}{0.121569,0.466667,0.705882}%
\pgfsetfillcolor{currentfill}%
\pgfsetfillopacity{0.877704}%
\pgfsetlinewidth{1.003750pt}%
\definecolor{currentstroke}{rgb}{0.121569,0.466667,0.705882}%
\pgfsetstrokecolor{currentstroke}%
\pgfsetstrokeopacity{0.877704}%
\pgfsetdash{}{0pt}%
\pgfpathmoveto{\pgfqpoint{1.131086in}{2.364278in}}%
\pgfpathcurveto{\pgfqpoint{1.139322in}{2.364278in}}{\pgfqpoint{1.147222in}{2.367550in}}{\pgfqpoint{1.153046in}{2.373374in}}%
\pgfpathcurveto{\pgfqpoint{1.158870in}{2.379198in}}{\pgfqpoint{1.162142in}{2.387098in}}{\pgfqpoint{1.162142in}{2.395334in}}%
\pgfpathcurveto{\pgfqpoint{1.162142in}{2.403571in}}{\pgfqpoint{1.158870in}{2.411471in}}{\pgfqpoint{1.153046in}{2.417295in}}%
\pgfpathcurveto{\pgfqpoint{1.147222in}{2.423119in}}{\pgfqpoint{1.139322in}{2.426391in}}{\pgfqpoint{1.131086in}{2.426391in}}%
\pgfpathcurveto{\pgfqpoint{1.122850in}{2.426391in}}{\pgfqpoint{1.114949in}{2.423119in}}{\pgfqpoint{1.109126in}{2.417295in}}%
\pgfpathcurveto{\pgfqpoint{1.103302in}{2.411471in}}{\pgfqpoint{1.100029in}{2.403571in}}{\pgfqpoint{1.100029in}{2.395334in}}%
\pgfpathcurveto{\pgfqpoint{1.100029in}{2.387098in}}{\pgfqpoint{1.103302in}{2.379198in}}{\pgfqpoint{1.109126in}{2.373374in}}%
\pgfpathcurveto{\pgfqpoint{1.114949in}{2.367550in}}{\pgfqpoint{1.122850in}{2.364278in}}{\pgfqpoint{1.131086in}{2.364278in}}%
\pgfpathclose%
\pgfusepath{stroke,fill}%
\end{pgfscope}%
\begin{pgfscope}%
\pgfpathrectangle{\pgfqpoint{0.100000in}{0.212622in}}{\pgfqpoint{3.696000in}{3.696000in}}%
\pgfusepath{clip}%
\pgfsetbuttcap%
\pgfsetroundjoin%
\definecolor{currentfill}{rgb}{0.121569,0.466667,0.705882}%
\pgfsetfillcolor{currentfill}%
\pgfsetfillopacity{0.878659}%
\pgfsetlinewidth{1.003750pt}%
\definecolor{currentstroke}{rgb}{0.121569,0.466667,0.705882}%
\pgfsetstrokecolor{currentstroke}%
\pgfsetstrokeopacity{0.878659}%
\pgfsetdash{}{0pt}%
\pgfpathmoveto{\pgfqpoint{2.676965in}{1.859635in}}%
\pgfpathcurveto{\pgfqpoint{2.685201in}{1.859635in}}{\pgfqpoint{2.693101in}{1.862907in}}{\pgfqpoint{2.698925in}{1.868731in}}%
\pgfpathcurveto{\pgfqpoint{2.704749in}{1.874555in}}{\pgfqpoint{2.708021in}{1.882455in}}{\pgfqpoint{2.708021in}{1.890691in}}%
\pgfpathcurveto{\pgfqpoint{2.708021in}{1.898927in}}{\pgfqpoint{2.704749in}{1.906827in}}{\pgfqpoint{2.698925in}{1.912651in}}%
\pgfpathcurveto{\pgfqpoint{2.693101in}{1.918475in}}{\pgfqpoint{2.685201in}{1.921748in}}{\pgfqpoint{2.676965in}{1.921748in}}%
\pgfpathcurveto{\pgfqpoint{2.668728in}{1.921748in}}{\pgfqpoint{2.660828in}{1.918475in}}{\pgfqpoint{2.655005in}{1.912651in}}%
\pgfpathcurveto{\pgfqpoint{2.649181in}{1.906827in}}{\pgfqpoint{2.645908in}{1.898927in}}{\pgfqpoint{2.645908in}{1.890691in}}%
\pgfpathcurveto{\pgfqpoint{2.645908in}{1.882455in}}{\pgfqpoint{2.649181in}{1.874555in}}{\pgfqpoint{2.655005in}{1.868731in}}%
\pgfpathcurveto{\pgfqpoint{2.660828in}{1.862907in}}{\pgfqpoint{2.668728in}{1.859635in}}{\pgfqpoint{2.676965in}{1.859635in}}%
\pgfpathclose%
\pgfusepath{stroke,fill}%
\end{pgfscope}%
\begin{pgfscope}%
\pgfpathrectangle{\pgfqpoint{0.100000in}{0.212622in}}{\pgfqpoint{3.696000in}{3.696000in}}%
\pgfusepath{clip}%
\pgfsetbuttcap%
\pgfsetroundjoin%
\definecolor{currentfill}{rgb}{0.121569,0.466667,0.705882}%
\pgfsetfillcolor{currentfill}%
\pgfsetfillopacity{0.880032}%
\pgfsetlinewidth{1.003750pt}%
\definecolor{currentstroke}{rgb}{0.121569,0.466667,0.705882}%
\pgfsetstrokecolor{currentstroke}%
\pgfsetstrokeopacity{0.880032}%
\pgfsetdash{}{0pt}%
\pgfpathmoveto{\pgfqpoint{1.154808in}{2.355444in}}%
\pgfpathcurveto{\pgfqpoint{1.163044in}{2.355444in}}{\pgfqpoint{1.170944in}{2.358716in}}{\pgfqpoint{1.176768in}{2.364540in}}%
\pgfpathcurveto{\pgfqpoint{1.182592in}{2.370364in}}{\pgfqpoint{1.185864in}{2.378264in}}{\pgfqpoint{1.185864in}{2.386500in}}%
\pgfpathcurveto{\pgfqpoint{1.185864in}{2.394737in}}{\pgfqpoint{1.182592in}{2.402637in}}{\pgfqpoint{1.176768in}{2.408461in}}%
\pgfpathcurveto{\pgfqpoint{1.170944in}{2.414285in}}{\pgfqpoint{1.163044in}{2.417557in}}{\pgfqpoint{1.154808in}{2.417557in}}%
\pgfpathcurveto{\pgfqpoint{1.146571in}{2.417557in}}{\pgfqpoint{1.138671in}{2.414285in}}{\pgfqpoint{1.132847in}{2.408461in}}%
\pgfpathcurveto{\pgfqpoint{1.127023in}{2.402637in}}{\pgfqpoint{1.123751in}{2.394737in}}{\pgfqpoint{1.123751in}{2.386500in}}%
\pgfpathcurveto{\pgfqpoint{1.123751in}{2.378264in}}{\pgfqpoint{1.127023in}{2.370364in}}{\pgfqpoint{1.132847in}{2.364540in}}%
\pgfpathcurveto{\pgfqpoint{1.138671in}{2.358716in}}{\pgfqpoint{1.146571in}{2.355444in}}{\pgfqpoint{1.154808in}{2.355444in}}%
\pgfpathclose%
\pgfusepath{stroke,fill}%
\end{pgfscope}%
\begin{pgfscope}%
\pgfpathrectangle{\pgfqpoint{0.100000in}{0.212622in}}{\pgfqpoint{3.696000in}{3.696000in}}%
\pgfusepath{clip}%
\pgfsetbuttcap%
\pgfsetroundjoin%
\definecolor{currentfill}{rgb}{0.121569,0.466667,0.705882}%
\pgfsetfillcolor{currentfill}%
\pgfsetfillopacity{0.880767}%
\pgfsetlinewidth{1.003750pt}%
\definecolor{currentstroke}{rgb}{0.121569,0.466667,0.705882}%
\pgfsetstrokecolor{currentstroke}%
\pgfsetstrokeopacity{0.880767}%
\pgfsetdash{}{0pt}%
\pgfpathmoveto{\pgfqpoint{2.670619in}{1.861668in}}%
\pgfpathcurveto{\pgfqpoint{2.678856in}{1.861668in}}{\pgfqpoint{2.686756in}{1.864940in}}{\pgfqpoint{2.692579in}{1.870764in}}%
\pgfpathcurveto{\pgfqpoint{2.698403in}{1.876588in}}{\pgfqpoint{2.701676in}{1.884488in}}{\pgfqpoint{2.701676in}{1.892724in}}%
\pgfpathcurveto{\pgfqpoint{2.701676in}{1.900961in}}{\pgfqpoint{2.698403in}{1.908861in}}{\pgfqpoint{2.692579in}{1.914685in}}%
\pgfpathcurveto{\pgfqpoint{2.686756in}{1.920509in}}{\pgfqpoint{2.678856in}{1.923781in}}{\pgfqpoint{2.670619in}{1.923781in}}%
\pgfpathcurveto{\pgfqpoint{2.662383in}{1.923781in}}{\pgfqpoint{2.654483in}{1.920509in}}{\pgfqpoint{2.648659in}{1.914685in}}%
\pgfpathcurveto{\pgfqpoint{2.642835in}{1.908861in}}{\pgfqpoint{2.639563in}{1.900961in}}{\pgfqpoint{2.639563in}{1.892724in}}%
\pgfpathcurveto{\pgfqpoint{2.639563in}{1.884488in}}{\pgfqpoint{2.642835in}{1.876588in}}{\pgfqpoint{2.648659in}{1.870764in}}%
\pgfpathcurveto{\pgfqpoint{2.654483in}{1.864940in}}{\pgfqpoint{2.662383in}{1.861668in}}{\pgfqpoint{2.670619in}{1.861668in}}%
\pgfpathclose%
\pgfusepath{stroke,fill}%
\end{pgfscope}%
\begin{pgfscope}%
\pgfpathrectangle{\pgfqpoint{0.100000in}{0.212622in}}{\pgfqpoint{3.696000in}{3.696000in}}%
\pgfusepath{clip}%
\pgfsetbuttcap%
\pgfsetroundjoin%
\definecolor{currentfill}{rgb}{0.121569,0.466667,0.705882}%
\pgfsetfillcolor{currentfill}%
\pgfsetfillopacity{0.882177}%
\pgfsetlinewidth{1.003750pt}%
\definecolor{currentstroke}{rgb}{0.121569,0.466667,0.705882}%
\pgfsetstrokecolor{currentstroke}%
\pgfsetstrokeopacity{0.882177}%
\pgfsetdash{}{0pt}%
\pgfpathmoveto{\pgfqpoint{1.178934in}{2.348337in}}%
\pgfpathcurveto{\pgfqpoint{1.187171in}{2.348337in}}{\pgfqpoint{1.195071in}{2.351609in}}{\pgfqpoint{1.200895in}{2.357433in}}%
\pgfpathcurveto{\pgfqpoint{1.206718in}{2.363257in}}{\pgfqpoint{1.209991in}{2.371157in}}{\pgfqpoint{1.209991in}{2.379393in}}%
\pgfpathcurveto{\pgfqpoint{1.209991in}{2.387630in}}{\pgfqpoint{1.206718in}{2.395530in}}{\pgfqpoint{1.200895in}{2.401354in}}%
\pgfpathcurveto{\pgfqpoint{1.195071in}{2.407178in}}{\pgfqpoint{1.187171in}{2.410450in}}{\pgfqpoint{1.178934in}{2.410450in}}%
\pgfpathcurveto{\pgfqpoint{1.170698in}{2.410450in}}{\pgfqpoint{1.162798in}{2.407178in}}{\pgfqpoint{1.156974in}{2.401354in}}%
\pgfpathcurveto{\pgfqpoint{1.151150in}{2.395530in}}{\pgfqpoint{1.147878in}{2.387630in}}{\pgfqpoint{1.147878in}{2.379393in}}%
\pgfpathcurveto{\pgfqpoint{1.147878in}{2.371157in}}{\pgfqpoint{1.151150in}{2.363257in}}{\pgfqpoint{1.156974in}{2.357433in}}%
\pgfpathcurveto{\pgfqpoint{1.162798in}{2.351609in}}{\pgfqpoint{1.170698in}{2.348337in}}{\pgfqpoint{1.178934in}{2.348337in}}%
\pgfpathclose%
\pgfusepath{stroke,fill}%
\end{pgfscope}%
\begin{pgfscope}%
\pgfpathrectangle{\pgfqpoint{0.100000in}{0.212622in}}{\pgfqpoint{3.696000in}{3.696000in}}%
\pgfusepath{clip}%
\pgfsetbuttcap%
\pgfsetroundjoin%
\definecolor{currentfill}{rgb}{0.121569,0.466667,0.705882}%
\pgfsetfillcolor{currentfill}%
\pgfsetfillopacity{0.883805}%
\pgfsetlinewidth{1.003750pt}%
\definecolor{currentstroke}{rgb}{0.121569,0.466667,0.705882}%
\pgfsetstrokecolor{currentstroke}%
\pgfsetstrokeopacity{0.883805}%
\pgfsetdash{}{0pt}%
\pgfpathmoveto{\pgfqpoint{1.203270in}{2.340358in}}%
\pgfpathcurveto{\pgfqpoint{1.211506in}{2.340358in}}{\pgfqpoint{1.219406in}{2.343631in}}{\pgfqpoint{1.225230in}{2.349455in}}%
\pgfpathcurveto{\pgfqpoint{1.231054in}{2.355278in}}{\pgfqpoint{1.234326in}{2.363179in}}{\pgfqpoint{1.234326in}{2.371415in}}%
\pgfpathcurveto{\pgfqpoint{1.234326in}{2.379651in}}{\pgfqpoint{1.231054in}{2.387551in}}{\pgfqpoint{1.225230in}{2.393375in}}%
\pgfpathcurveto{\pgfqpoint{1.219406in}{2.399199in}}{\pgfqpoint{1.211506in}{2.402471in}}{\pgfqpoint{1.203270in}{2.402471in}}%
\pgfpathcurveto{\pgfqpoint{1.195034in}{2.402471in}}{\pgfqpoint{1.187134in}{2.399199in}}{\pgfqpoint{1.181310in}{2.393375in}}%
\pgfpathcurveto{\pgfqpoint{1.175486in}{2.387551in}}{\pgfqpoint{1.172213in}{2.379651in}}{\pgfqpoint{1.172213in}{2.371415in}}%
\pgfpathcurveto{\pgfqpoint{1.172213in}{2.363179in}}{\pgfqpoint{1.175486in}{2.355278in}}{\pgfqpoint{1.181310in}{2.349455in}}%
\pgfpathcurveto{\pgfqpoint{1.187134in}{2.343631in}}{\pgfqpoint{1.195034in}{2.340358in}}{\pgfqpoint{1.203270in}{2.340358in}}%
\pgfpathclose%
\pgfusepath{stroke,fill}%
\end{pgfscope}%
\begin{pgfscope}%
\pgfpathrectangle{\pgfqpoint{0.100000in}{0.212622in}}{\pgfqpoint{3.696000in}{3.696000in}}%
\pgfusepath{clip}%
\pgfsetbuttcap%
\pgfsetroundjoin%
\definecolor{currentfill}{rgb}{0.121569,0.466667,0.705882}%
\pgfsetfillcolor{currentfill}%
\pgfsetfillopacity{0.883897}%
\pgfsetlinewidth{1.003750pt}%
\definecolor{currentstroke}{rgb}{0.121569,0.466667,0.705882}%
\pgfsetstrokecolor{currentstroke}%
\pgfsetstrokeopacity{0.883897}%
\pgfsetdash{}{0pt}%
\pgfpathmoveto{\pgfqpoint{2.664549in}{1.862252in}}%
\pgfpathcurveto{\pgfqpoint{2.672786in}{1.862252in}}{\pgfqpoint{2.680686in}{1.865524in}}{\pgfqpoint{2.686510in}{1.871348in}}%
\pgfpathcurveto{\pgfqpoint{2.692334in}{1.877172in}}{\pgfqpoint{2.695606in}{1.885072in}}{\pgfqpoint{2.695606in}{1.893308in}}%
\pgfpathcurveto{\pgfqpoint{2.695606in}{1.901545in}}{\pgfqpoint{2.692334in}{1.909445in}}{\pgfqpoint{2.686510in}{1.915269in}}%
\pgfpathcurveto{\pgfqpoint{2.680686in}{1.921093in}}{\pgfqpoint{2.672786in}{1.924365in}}{\pgfqpoint{2.664549in}{1.924365in}}%
\pgfpathcurveto{\pgfqpoint{2.656313in}{1.924365in}}{\pgfqpoint{2.648413in}{1.921093in}}{\pgfqpoint{2.642589in}{1.915269in}}%
\pgfpathcurveto{\pgfqpoint{2.636765in}{1.909445in}}{\pgfqpoint{2.633493in}{1.901545in}}{\pgfqpoint{2.633493in}{1.893308in}}%
\pgfpathcurveto{\pgfqpoint{2.633493in}{1.885072in}}{\pgfqpoint{2.636765in}{1.877172in}}{\pgfqpoint{2.642589in}{1.871348in}}%
\pgfpathcurveto{\pgfqpoint{2.648413in}{1.865524in}}{\pgfqpoint{2.656313in}{1.862252in}}{\pgfqpoint{2.664549in}{1.862252in}}%
\pgfpathclose%
\pgfusepath{stroke,fill}%
\end{pgfscope}%
\begin{pgfscope}%
\pgfpathrectangle{\pgfqpoint{0.100000in}{0.212622in}}{\pgfqpoint{3.696000in}{3.696000in}}%
\pgfusepath{clip}%
\pgfsetbuttcap%
\pgfsetroundjoin%
\definecolor{currentfill}{rgb}{0.121569,0.466667,0.705882}%
\pgfsetfillcolor{currentfill}%
\pgfsetfillopacity{0.885491}%
\pgfsetlinewidth{1.003750pt}%
\definecolor{currentstroke}{rgb}{0.121569,0.466667,0.705882}%
\pgfsetstrokecolor{currentstroke}%
\pgfsetstrokeopacity{0.885491}%
\pgfsetdash{}{0pt}%
\pgfpathmoveto{\pgfqpoint{1.224339in}{2.331155in}}%
\pgfpathcurveto{\pgfqpoint{1.232575in}{2.331155in}}{\pgfqpoint{1.240475in}{2.334427in}}{\pgfqpoint{1.246299in}{2.340251in}}%
\pgfpathcurveto{\pgfqpoint{1.252123in}{2.346075in}}{\pgfqpoint{1.255396in}{2.353975in}}{\pgfqpoint{1.255396in}{2.362212in}}%
\pgfpathcurveto{\pgfqpoint{1.255396in}{2.370448in}}{\pgfqpoint{1.252123in}{2.378348in}}{\pgfqpoint{1.246299in}{2.384172in}}%
\pgfpathcurveto{\pgfqpoint{1.240475in}{2.389996in}}{\pgfqpoint{1.232575in}{2.393268in}}{\pgfqpoint{1.224339in}{2.393268in}}%
\pgfpathcurveto{\pgfqpoint{1.216103in}{2.393268in}}{\pgfqpoint{1.208203in}{2.389996in}}{\pgfqpoint{1.202379in}{2.384172in}}%
\pgfpathcurveto{\pgfqpoint{1.196555in}{2.378348in}}{\pgfqpoint{1.193283in}{2.370448in}}{\pgfqpoint{1.193283in}{2.362212in}}%
\pgfpathcurveto{\pgfqpoint{1.193283in}{2.353975in}}{\pgfqpoint{1.196555in}{2.346075in}}{\pgfqpoint{1.202379in}{2.340251in}}%
\pgfpathcurveto{\pgfqpoint{1.208203in}{2.334427in}}{\pgfqpoint{1.216103in}{2.331155in}}{\pgfqpoint{1.224339in}{2.331155in}}%
\pgfpathclose%
\pgfusepath{stroke,fill}%
\end{pgfscope}%
\begin{pgfscope}%
\pgfpathrectangle{\pgfqpoint{0.100000in}{0.212622in}}{\pgfqpoint{3.696000in}{3.696000in}}%
\pgfusepath{clip}%
\pgfsetbuttcap%
\pgfsetroundjoin%
\definecolor{currentfill}{rgb}{0.121569,0.466667,0.705882}%
\pgfsetfillcolor{currentfill}%
\pgfsetfillopacity{0.887313}%
\pgfsetlinewidth{1.003750pt}%
\definecolor{currentstroke}{rgb}{0.121569,0.466667,0.705882}%
\pgfsetstrokecolor{currentstroke}%
\pgfsetstrokeopacity{0.887313}%
\pgfsetdash{}{0pt}%
\pgfpathmoveto{\pgfqpoint{2.659622in}{1.862731in}}%
\pgfpathcurveto{\pgfqpoint{2.667858in}{1.862731in}}{\pgfqpoint{2.675758in}{1.866003in}}{\pgfqpoint{2.681582in}{1.871827in}}%
\pgfpathcurveto{\pgfqpoint{2.687406in}{1.877651in}}{\pgfqpoint{2.690678in}{1.885551in}}{\pgfqpoint{2.690678in}{1.893788in}}%
\pgfpathcurveto{\pgfqpoint{2.690678in}{1.902024in}}{\pgfqpoint{2.687406in}{1.909924in}}{\pgfqpoint{2.681582in}{1.915748in}}%
\pgfpathcurveto{\pgfqpoint{2.675758in}{1.921572in}}{\pgfqpoint{2.667858in}{1.924844in}}{\pgfqpoint{2.659622in}{1.924844in}}%
\pgfpathcurveto{\pgfqpoint{2.651385in}{1.924844in}}{\pgfqpoint{2.643485in}{1.921572in}}{\pgfqpoint{2.637661in}{1.915748in}}%
\pgfpathcurveto{\pgfqpoint{2.631837in}{1.909924in}}{\pgfqpoint{2.628565in}{1.902024in}}{\pgfqpoint{2.628565in}{1.893788in}}%
\pgfpathcurveto{\pgfqpoint{2.628565in}{1.885551in}}{\pgfqpoint{2.631837in}{1.877651in}}{\pgfqpoint{2.637661in}{1.871827in}}%
\pgfpathcurveto{\pgfqpoint{2.643485in}{1.866003in}}{\pgfqpoint{2.651385in}{1.862731in}}{\pgfqpoint{2.659622in}{1.862731in}}%
\pgfpathclose%
\pgfusepath{stroke,fill}%
\end{pgfscope}%
\begin{pgfscope}%
\pgfpathrectangle{\pgfqpoint{0.100000in}{0.212622in}}{\pgfqpoint{3.696000in}{3.696000in}}%
\pgfusepath{clip}%
\pgfsetbuttcap%
\pgfsetroundjoin%
\definecolor{currentfill}{rgb}{0.121569,0.466667,0.705882}%
\pgfsetfillcolor{currentfill}%
\pgfsetfillopacity{0.887813}%
\pgfsetlinewidth{1.003750pt}%
\definecolor{currentstroke}{rgb}{0.121569,0.466667,0.705882}%
\pgfsetstrokecolor{currentstroke}%
\pgfsetstrokeopacity{0.887813}%
\pgfsetdash{}{0pt}%
\pgfpathmoveto{\pgfqpoint{1.243662in}{2.326911in}}%
\pgfpathcurveto{\pgfqpoint{1.251898in}{2.326911in}}{\pgfqpoint{1.259799in}{2.330184in}}{\pgfqpoint{1.265622in}{2.336007in}}%
\pgfpathcurveto{\pgfqpoint{1.271446in}{2.341831in}}{\pgfqpoint{1.274719in}{2.349731in}}{\pgfqpoint{1.274719in}{2.357968in}}%
\pgfpathcurveto{\pgfqpoint{1.274719in}{2.366204in}}{\pgfqpoint{1.271446in}{2.374104in}}{\pgfqpoint{1.265622in}{2.379928in}}%
\pgfpathcurveto{\pgfqpoint{1.259799in}{2.385752in}}{\pgfqpoint{1.251898in}{2.389024in}}{\pgfqpoint{1.243662in}{2.389024in}}%
\pgfpathcurveto{\pgfqpoint{1.235426in}{2.389024in}}{\pgfqpoint{1.227526in}{2.385752in}}{\pgfqpoint{1.221702in}{2.379928in}}%
\pgfpathcurveto{\pgfqpoint{1.215878in}{2.374104in}}{\pgfqpoint{1.212606in}{2.366204in}}{\pgfqpoint{1.212606in}{2.357968in}}%
\pgfpathcurveto{\pgfqpoint{1.212606in}{2.349731in}}{\pgfqpoint{1.215878in}{2.341831in}}{\pgfqpoint{1.221702in}{2.336007in}}%
\pgfpathcurveto{\pgfqpoint{1.227526in}{2.330184in}}{\pgfqpoint{1.235426in}{2.326911in}}{\pgfqpoint{1.243662in}{2.326911in}}%
\pgfpathclose%
\pgfusepath{stroke,fill}%
\end{pgfscope}%
\begin{pgfscope}%
\pgfpathrectangle{\pgfqpoint{0.100000in}{0.212622in}}{\pgfqpoint{3.696000in}{3.696000in}}%
\pgfusepath{clip}%
\pgfsetbuttcap%
\pgfsetroundjoin%
\definecolor{currentfill}{rgb}{0.121569,0.466667,0.705882}%
\pgfsetfillcolor{currentfill}%
\pgfsetfillopacity{0.889000}%
\pgfsetlinewidth{1.003750pt}%
\definecolor{currentstroke}{rgb}{0.121569,0.466667,0.705882}%
\pgfsetstrokecolor{currentstroke}%
\pgfsetstrokeopacity{0.889000}%
\pgfsetdash{}{0pt}%
\pgfpathmoveto{\pgfqpoint{1.263887in}{2.319521in}}%
\pgfpathcurveto{\pgfqpoint{1.272123in}{2.319521in}}{\pgfqpoint{1.280023in}{2.322793in}}{\pgfqpoint{1.285847in}{2.328617in}}%
\pgfpathcurveto{\pgfqpoint{1.291671in}{2.334441in}}{\pgfqpoint{1.294943in}{2.342341in}}{\pgfqpoint{1.294943in}{2.350578in}}%
\pgfpathcurveto{\pgfqpoint{1.294943in}{2.358814in}}{\pgfqpoint{1.291671in}{2.366714in}}{\pgfqpoint{1.285847in}{2.372538in}}%
\pgfpathcurveto{\pgfqpoint{1.280023in}{2.378362in}}{\pgfqpoint{1.272123in}{2.381634in}}{\pgfqpoint{1.263887in}{2.381634in}}%
\pgfpathcurveto{\pgfqpoint{1.255651in}{2.381634in}}{\pgfqpoint{1.247751in}{2.378362in}}{\pgfqpoint{1.241927in}{2.372538in}}%
\pgfpathcurveto{\pgfqpoint{1.236103in}{2.366714in}}{\pgfqpoint{1.232830in}{2.358814in}}{\pgfqpoint{1.232830in}{2.350578in}}%
\pgfpathcurveto{\pgfqpoint{1.232830in}{2.342341in}}{\pgfqpoint{1.236103in}{2.334441in}}{\pgfqpoint{1.241927in}{2.328617in}}%
\pgfpathcurveto{\pgfqpoint{1.247751in}{2.322793in}}{\pgfqpoint{1.255651in}{2.319521in}}{\pgfqpoint{1.263887in}{2.319521in}}%
\pgfpathclose%
\pgfusepath{stroke,fill}%
\end{pgfscope}%
\begin{pgfscope}%
\pgfpathrectangle{\pgfqpoint{0.100000in}{0.212622in}}{\pgfqpoint{3.696000in}{3.696000in}}%
\pgfusepath{clip}%
\pgfsetbuttcap%
\pgfsetroundjoin%
\definecolor{currentfill}{rgb}{0.121569,0.466667,0.705882}%
\pgfsetfillcolor{currentfill}%
\pgfsetfillopacity{0.890197}%
\pgfsetlinewidth{1.003750pt}%
\definecolor{currentstroke}{rgb}{0.121569,0.466667,0.705882}%
\pgfsetstrokecolor{currentstroke}%
\pgfsetstrokeopacity{0.890197}%
\pgfsetdash{}{0pt}%
\pgfpathmoveto{\pgfqpoint{1.280356in}{2.310933in}}%
\pgfpathcurveto{\pgfqpoint{1.288592in}{2.310933in}}{\pgfqpoint{1.296492in}{2.314206in}}{\pgfqpoint{1.302316in}{2.320030in}}%
\pgfpathcurveto{\pgfqpoint{1.308140in}{2.325854in}}{\pgfqpoint{1.311413in}{2.333754in}}{\pgfqpoint{1.311413in}{2.341990in}}%
\pgfpathcurveto{\pgfqpoint{1.311413in}{2.350226in}}{\pgfqpoint{1.308140in}{2.358126in}}{\pgfqpoint{1.302316in}{2.363950in}}%
\pgfpathcurveto{\pgfqpoint{1.296492in}{2.369774in}}{\pgfqpoint{1.288592in}{2.373046in}}{\pgfqpoint{1.280356in}{2.373046in}}%
\pgfpathcurveto{\pgfqpoint{1.272120in}{2.373046in}}{\pgfqpoint{1.264220in}{2.369774in}}{\pgfqpoint{1.258396in}{2.363950in}}%
\pgfpathcurveto{\pgfqpoint{1.252572in}{2.358126in}}{\pgfqpoint{1.249300in}{2.350226in}}{\pgfqpoint{1.249300in}{2.341990in}}%
\pgfpathcurveto{\pgfqpoint{1.249300in}{2.333754in}}{\pgfqpoint{1.252572in}{2.325854in}}{\pgfqpoint{1.258396in}{2.320030in}}%
\pgfpathcurveto{\pgfqpoint{1.264220in}{2.314206in}}{\pgfqpoint{1.272120in}{2.310933in}}{\pgfqpoint{1.280356in}{2.310933in}}%
\pgfpathclose%
\pgfusepath{stroke,fill}%
\end{pgfscope}%
\begin{pgfscope}%
\pgfpathrectangle{\pgfqpoint{0.100000in}{0.212622in}}{\pgfqpoint{3.696000in}{3.696000in}}%
\pgfusepath{clip}%
\pgfsetbuttcap%
\pgfsetroundjoin%
\definecolor{currentfill}{rgb}{0.121569,0.466667,0.705882}%
\pgfsetfillcolor{currentfill}%
\pgfsetfillopacity{0.890710}%
\pgfsetlinewidth{1.003750pt}%
\definecolor{currentstroke}{rgb}{0.121569,0.466667,0.705882}%
\pgfsetstrokecolor{currentstroke}%
\pgfsetstrokeopacity{0.890710}%
\pgfsetdash{}{0pt}%
\pgfpathmoveto{\pgfqpoint{2.649800in}{1.865188in}}%
\pgfpathcurveto{\pgfqpoint{2.658037in}{1.865188in}}{\pgfqpoint{2.665937in}{1.868460in}}{\pgfqpoint{2.671760in}{1.874284in}}%
\pgfpathcurveto{\pgfqpoint{2.677584in}{1.880108in}}{\pgfqpoint{2.680857in}{1.888008in}}{\pgfqpoint{2.680857in}{1.896245in}}%
\pgfpathcurveto{\pgfqpoint{2.680857in}{1.904481in}}{\pgfqpoint{2.677584in}{1.912381in}}{\pgfqpoint{2.671760in}{1.918205in}}%
\pgfpathcurveto{\pgfqpoint{2.665937in}{1.924029in}}{\pgfqpoint{2.658037in}{1.927301in}}{\pgfqpoint{2.649800in}{1.927301in}}%
\pgfpathcurveto{\pgfqpoint{2.641564in}{1.927301in}}{\pgfqpoint{2.633664in}{1.924029in}}{\pgfqpoint{2.627840in}{1.918205in}}%
\pgfpathcurveto{\pgfqpoint{2.622016in}{1.912381in}}{\pgfqpoint{2.618744in}{1.904481in}}{\pgfqpoint{2.618744in}{1.896245in}}%
\pgfpathcurveto{\pgfqpoint{2.618744in}{1.888008in}}{\pgfqpoint{2.622016in}{1.880108in}}{\pgfqpoint{2.627840in}{1.874284in}}%
\pgfpathcurveto{\pgfqpoint{2.633664in}{1.868460in}}{\pgfqpoint{2.641564in}{1.865188in}}{\pgfqpoint{2.649800in}{1.865188in}}%
\pgfpathclose%
\pgfusepath{stroke,fill}%
\end{pgfscope}%
\begin{pgfscope}%
\pgfpathrectangle{\pgfqpoint{0.100000in}{0.212622in}}{\pgfqpoint{3.696000in}{3.696000in}}%
\pgfusepath{clip}%
\pgfsetbuttcap%
\pgfsetroundjoin%
\definecolor{currentfill}{rgb}{0.121569,0.466667,0.705882}%
\pgfsetfillcolor{currentfill}%
\pgfsetfillopacity{0.891814}%
\pgfsetlinewidth{1.003750pt}%
\definecolor{currentstroke}{rgb}{0.121569,0.466667,0.705882}%
\pgfsetstrokecolor{currentstroke}%
\pgfsetstrokeopacity{0.891814}%
\pgfsetdash{}{0pt}%
\pgfpathmoveto{\pgfqpoint{1.294607in}{2.307266in}}%
\pgfpathcurveto{\pgfqpoint{1.302843in}{2.307266in}}{\pgfqpoint{1.310743in}{2.310538in}}{\pgfqpoint{1.316567in}{2.316362in}}%
\pgfpathcurveto{\pgfqpoint{1.322391in}{2.322186in}}{\pgfqpoint{1.325663in}{2.330086in}}{\pgfqpoint{1.325663in}{2.338322in}}%
\pgfpathcurveto{\pgfqpoint{1.325663in}{2.346558in}}{\pgfqpoint{1.322391in}{2.354458in}}{\pgfqpoint{1.316567in}{2.360282in}}%
\pgfpathcurveto{\pgfqpoint{1.310743in}{2.366106in}}{\pgfqpoint{1.302843in}{2.369379in}}{\pgfqpoint{1.294607in}{2.369379in}}%
\pgfpathcurveto{\pgfqpoint{1.286370in}{2.369379in}}{\pgfqpoint{1.278470in}{2.366106in}}{\pgfqpoint{1.272646in}{2.360282in}}%
\pgfpathcurveto{\pgfqpoint{1.266822in}{2.354458in}}{\pgfqpoint{1.263550in}{2.346558in}}{\pgfqpoint{1.263550in}{2.338322in}}%
\pgfpathcurveto{\pgfqpoint{1.263550in}{2.330086in}}{\pgfqpoint{1.266822in}{2.322186in}}{\pgfqpoint{1.272646in}{2.316362in}}%
\pgfpathcurveto{\pgfqpoint{1.278470in}{2.310538in}}{\pgfqpoint{1.286370in}{2.307266in}}{\pgfqpoint{1.294607in}{2.307266in}}%
\pgfpathclose%
\pgfusepath{stroke,fill}%
\end{pgfscope}%
\begin{pgfscope}%
\pgfpathrectangle{\pgfqpoint{0.100000in}{0.212622in}}{\pgfqpoint{3.696000in}{3.696000in}}%
\pgfusepath{clip}%
\pgfsetbuttcap%
\pgfsetroundjoin%
\definecolor{currentfill}{rgb}{0.121569,0.466667,0.705882}%
\pgfsetfillcolor{currentfill}%
\pgfsetfillopacity{0.892620}%
\pgfsetlinewidth{1.003750pt}%
\definecolor{currentstroke}{rgb}{0.121569,0.466667,0.705882}%
\pgfsetstrokecolor{currentstroke}%
\pgfsetstrokeopacity{0.892620}%
\pgfsetdash{}{0pt}%
\pgfpathmoveto{\pgfqpoint{2.644603in}{1.866431in}}%
\pgfpathcurveto{\pgfqpoint{2.652840in}{1.866431in}}{\pgfqpoint{2.660740in}{1.869704in}}{\pgfqpoint{2.666564in}{1.875528in}}%
\pgfpathcurveto{\pgfqpoint{2.672387in}{1.881352in}}{\pgfqpoint{2.675660in}{1.889252in}}{\pgfqpoint{2.675660in}{1.897488in}}%
\pgfpathcurveto{\pgfqpoint{2.675660in}{1.905724in}}{\pgfqpoint{2.672387in}{1.913624in}}{\pgfqpoint{2.666564in}{1.919448in}}%
\pgfpathcurveto{\pgfqpoint{2.660740in}{1.925272in}}{\pgfqpoint{2.652840in}{1.928544in}}{\pgfqpoint{2.644603in}{1.928544in}}%
\pgfpathcurveto{\pgfqpoint{2.636367in}{1.928544in}}{\pgfqpoint{2.628467in}{1.925272in}}{\pgfqpoint{2.622643in}{1.919448in}}%
\pgfpathcurveto{\pgfqpoint{2.616819in}{1.913624in}}{\pgfqpoint{2.613547in}{1.905724in}}{\pgfqpoint{2.613547in}{1.897488in}}%
\pgfpathcurveto{\pgfqpoint{2.613547in}{1.889252in}}{\pgfqpoint{2.616819in}{1.881352in}}{\pgfqpoint{2.622643in}{1.875528in}}%
\pgfpathcurveto{\pgfqpoint{2.628467in}{1.869704in}}{\pgfqpoint{2.636367in}{1.866431in}}{\pgfqpoint{2.644603in}{1.866431in}}%
\pgfpathclose%
\pgfusepath{stroke,fill}%
\end{pgfscope}%
\begin{pgfscope}%
\pgfpathrectangle{\pgfqpoint{0.100000in}{0.212622in}}{\pgfqpoint{3.696000in}{3.696000in}}%
\pgfusepath{clip}%
\pgfsetbuttcap%
\pgfsetroundjoin%
\definecolor{currentfill}{rgb}{0.121569,0.466667,0.705882}%
\pgfsetfillcolor{currentfill}%
\pgfsetfillopacity{0.893107}%
\pgfsetlinewidth{1.003750pt}%
\definecolor{currentstroke}{rgb}{0.121569,0.466667,0.705882}%
\pgfsetstrokecolor{currentstroke}%
\pgfsetstrokeopacity{0.893107}%
\pgfsetdash{}{0pt}%
\pgfpathmoveto{\pgfqpoint{1.308687in}{2.302429in}}%
\pgfpathcurveto{\pgfqpoint{1.316924in}{2.302429in}}{\pgfqpoint{1.324824in}{2.305702in}}{\pgfqpoint{1.330647in}{2.311526in}}%
\pgfpathcurveto{\pgfqpoint{1.336471in}{2.317350in}}{\pgfqpoint{1.339744in}{2.325250in}}{\pgfqpoint{1.339744in}{2.333486in}}%
\pgfpathcurveto{\pgfqpoint{1.339744in}{2.341722in}}{\pgfqpoint{1.336471in}{2.349622in}}{\pgfqpoint{1.330647in}{2.355446in}}%
\pgfpathcurveto{\pgfqpoint{1.324824in}{2.361270in}}{\pgfqpoint{1.316924in}{2.364542in}}{\pgfqpoint{1.308687in}{2.364542in}}%
\pgfpathcurveto{\pgfqpoint{1.300451in}{2.364542in}}{\pgfqpoint{1.292551in}{2.361270in}}{\pgfqpoint{1.286727in}{2.355446in}}%
\pgfpathcurveto{\pgfqpoint{1.280903in}{2.349622in}}{\pgfqpoint{1.277631in}{2.341722in}}{\pgfqpoint{1.277631in}{2.333486in}}%
\pgfpathcurveto{\pgfqpoint{1.277631in}{2.325250in}}{\pgfqpoint{1.280903in}{2.317350in}}{\pgfqpoint{1.286727in}{2.311526in}}%
\pgfpathcurveto{\pgfqpoint{1.292551in}{2.305702in}}{\pgfqpoint{1.300451in}{2.302429in}}{\pgfqpoint{1.308687in}{2.302429in}}%
\pgfpathclose%
\pgfusepath{stroke,fill}%
\end{pgfscope}%
\begin{pgfscope}%
\pgfpathrectangle{\pgfqpoint{0.100000in}{0.212622in}}{\pgfqpoint{3.696000in}{3.696000in}}%
\pgfusepath{clip}%
\pgfsetbuttcap%
\pgfsetroundjoin%
\definecolor{currentfill}{rgb}{0.121569,0.466667,0.705882}%
\pgfsetfillcolor{currentfill}%
\pgfsetfillopacity{0.894130}%
\pgfsetlinewidth{1.003750pt}%
\definecolor{currentstroke}{rgb}{0.121569,0.466667,0.705882}%
\pgfsetstrokecolor{currentstroke}%
\pgfsetstrokeopacity{0.894130}%
\pgfsetdash{}{0pt}%
\pgfpathmoveto{\pgfqpoint{1.320368in}{2.295644in}}%
\pgfpathcurveto{\pgfqpoint{1.328604in}{2.295644in}}{\pgfqpoint{1.336504in}{2.298917in}}{\pgfqpoint{1.342328in}{2.304740in}}%
\pgfpathcurveto{\pgfqpoint{1.348152in}{2.310564in}}{\pgfqpoint{1.351424in}{2.318464in}}{\pgfqpoint{1.351424in}{2.326701in}}%
\pgfpathcurveto{\pgfqpoint{1.351424in}{2.334937in}}{\pgfqpoint{1.348152in}{2.342837in}}{\pgfqpoint{1.342328in}{2.348661in}}%
\pgfpathcurveto{\pgfqpoint{1.336504in}{2.354485in}}{\pgfqpoint{1.328604in}{2.357757in}}{\pgfqpoint{1.320368in}{2.357757in}}%
\pgfpathcurveto{\pgfqpoint{1.312131in}{2.357757in}}{\pgfqpoint{1.304231in}{2.354485in}}{\pgfqpoint{1.298407in}{2.348661in}}%
\pgfpathcurveto{\pgfqpoint{1.292583in}{2.342837in}}{\pgfqpoint{1.289311in}{2.334937in}}{\pgfqpoint{1.289311in}{2.326701in}}%
\pgfpathcurveto{\pgfqpoint{1.289311in}{2.318464in}}{\pgfqpoint{1.292583in}{2.310564in}}{\pgfqpoint{1.298407in}{2.304740in}}%
\pgfpathcurveto{\pgfqpoint{1.304231in}{2.298917in}}{\pgfqpoint{1.312131in}{2.295644in}}{\pgfqpoint{1.320368in}{2.295644in}}%
\pgfpathclose%
\pgfusepath{stroke,fill}%
\end{pgfscope}%
\begin{pgfscope}%
\pgfpathrectangle{\pgfqpoint{0.100000in}{0.212622in}}{\pgfqpoint{3.696000in}{3.696000in}}%
\pgfusepath{clip}%
\pgfsetbuttcap%
\pgfsetroundjoin%
\definecolor{currentfill}{rgb}{0.121569,0.466667,0.705882}%
\pgfsetfillcolor{currentfill}%
\pgfsetfillopacity{0.894890}%
\pgfsetlinewidth{1.003750pt}%
\definecolor{currentstroke}{rgb}{0.121569,0.466667,0.705882}%
\pgfsetstrokecolor{currentstroke}%
\pgfsetstrokeopacity{0.894890}%
\pgfsetdash{}{0pt}%
\pgfpathmoveto{\pgfqpoint{2.641108in}{1.866677in}}%
\pgfpathcurveto{\pgfqpoint{2.649344in}{1.866677in}}{\pgfqpoint{2.657244in}{1.869949in}}{\pgfqpoint{2.663068in}{1.875773in}}%
\pgfpathcurveto{\pgfqpoint{2.668892in}{1.881597in}}{\pgfqpoint{2.672165in}{1.889497in}}{\pgfqpoint{2.672165in}{1.897733in}}%
\pgfpathcurveto{\pgfqpoint{2.672165in}{1.905969in}}{\pgfqpoint{2.668892in}{1.913869in}}{\pgfqpoint{2.663068in}{1.919693in}}%
\pgfpathcurveto{\pgfqpoint{2.657244in}{1.925517in}}{\pgfqpoint{2.649344in}{1.928790in}}{\pgfqpoint{2.641108in}{1.928790in}}%
\pgfpathcurveto{\pgfqpoint{2.632872in}{1.928790in}}{\pgfqpoint{2.624972in}{1.925517in}}{\pgfqpoint{2.619148in}{1.919693in}}%
\pgfpathcurveto{\pgfqpoint{2.613324in}{1.913869in}}{\pgfqpoint{2.610052in}{1.905969in}}{\pgfqpoint{2.610052in}{1.897733in}}%
\pgfpathcurveto{\pgfqpoint{2.610052in}{1.889497in}}{\pgfqpoint{2.613324in}{1.881597in}}{\pgfqpoint{2.619148in}{1.875773in}}%
\pgfpathcurveto{\pgfqpoint{2.624972in}{1.869949in}}{\pgfqpoint{2.632872in}{1.866677in}}{\pgfqpoint{2.641108in}{1.866677in}}%
\pgfpathclose%
\pgfusepath{stroke,fill}%
\end{pgfscope}%
\begin{pgfscope}%
\pgfpathrectangle{\pgfqpoint{0.100000in}{0.212622in}}{\pgfqpoint{3.696000in}{3.696000in}}%
\pgfusepath{clip}%
\pgfsetbuttcap%
\pgfsetroundjoin%
\definecolor{currentfill}{rgb}{0.121569,0.466667,0.705882}%
\pgfsetfillcolor{currentfill}%
\pgfsetfillopacity{0.895116}%
\pgfsetlinewidth{1.003750pt}%
\definecolor{currentstroke}{rgb}{0.121569,0.466667,0.705882}%
\pgfsetstrokecolor{currentstroke}%
\pgfsetstrokeopacity{0.895116}%
\pgfsetdash{}{0pt}%
\pgfpathmoveto{\pgfqpoint{1.329717in}{2.293099in}}%
\pgfpathcurveto{\pgfqpoint{1.337953in}{2.293099in}}{\pgfqpoint{1.345853in}{2.296372in}}{\pgfqpoint{1.351677in}{2.302196in}}%
\pgfpathcurveto{\pgfqpoint{1.357501in}{2.308020in}}{\pgfqpoint{1.360774in}{2.315920in}}{\pgfqpoint{1.360774in}{2.324156in}}%
\pgfpathcurveto{\pgfqpoint{1.360774in}{2.332392in}}{\pgfqpoint{1.357501in}{2.340292in}}{\pgfqpoint{1.351677in}{2.346116in}}%
\pgfpathcurveto{\pgfqpoint{1.345853in}{2.351940in}}{\pgfqpoint{1.337953in}{2.355212in}}{\pgfqpoint{1.329717in}{2.355212in}}%
\pgfpathcurveto{\pgfqpoint{1.321481in}{2.355212in}}{\pgfqpoint{1.313581in}{2.351940in}}{\pgfqpoint{1.307757in}{2.346116in}}%
\pgfpathcurveto{\pgfqpoint{1.301933in}{2.340292in}}{\pgfqpoint{1.298661in}{2.332392in}}{\pgfqpoint{1.298661in}{2.324156in}}%
\pgfpathcurveto{\pgfqpoint{1.298661in}{2.315920in}}{\pgfqpoint{1.301933in}{2.308020in}}{\pgfqpoint{1.307757in}{2.302196in}}%
\pgfpathcurveto{\pgfqpoint{1.313581in}{2.296372in}}{\pgfqpoint{1.321481in}{2.293099in}}{\pgfqpoint{1.329717in}{2.293099in}}%
\pgfpathclose%
\pgfusepath{stroke,fill}%
\end{pgfscope}%
\begin{pgfscope}%
\pgfpathrectangle{\pgfqpoint{0.100000in}{0.212622in}}{\pgfqpoint{3.696000in}{3.696000in}}%
\pgfusepath{clip}%
\pgfsetbuttcap%
\pgfsetroundjoin%
\definecolor{currentfill}{rgb}{0.121569,0.466667,0.705882}%
\pgfsetfillcolor{currentfill}%
\pgfsetfillopacity{0.895980}%
\pgfsetlinewidth{1.003750pt}%
\definecolor{currentstroke}{rgb}{0.121569,0.466667,0.705882}%
\pgfsetstrokecolor{currentstroke}%
\pgfsetstrokeopacity{0.895980}%
\pgfsetdash{}{0pt}%
\pgfpathmoveto{\pgfqpoint{1.347874in}{2.286773in}}%
\pgfpathcurveto{\pgfqpoint{1.356110in}{2.286773in}}{\pgfqpoint{1.364010in}{2.290045in}}{\pgfqpoint{1.369834in}{2.295869in}}%
\pgfpathcurveto{\pgfqpoint{1.375658in}{2.301693in}}{\pgfqpoint{1.378931in}{2.309593in}}{\pgfqpoint{1.378931in}{2.317830in}}%
\pgfpathcurveto{\pgfqpoint{1.378931in}{2.326066in}}{\pgfqpoint{1.375658in}{2.333966in}}{\pgfqpoint{1.369834in}{2.339790in}}%
\pgfpathcurveto{\pgfqpoint{1.364010in}{2.345614in}}{\pgfqpoint{1.356110in}{2.348886in}}{\pgfqpoint{1.347874in}{2.348886in}}%
\pgfpathcurveto{\pgfqpoint{1.339638in}{2.348886in}}{\pgfqpoint{1.331738in}{2.345614in}}{\pgfqpoint{1.325914in}{2.339790in}}%
\pgfpathcurveto{\pgfqpoint{1.320090in}{2.333966in}}{\pgfqpoint{1.316818in}{2.326066in}}{\pgfqpoint{1.316818in}{2.317830in}}%
\pgfpathcurveto{\pgfqpoint{1.316818in}{2.309593in}}{\pgfqpoint{1.320090in}{2.301693in}}{\pgfqpoint{1.325914in}{2.295869in}}%
\pgfpathcurveto{\pgfqpoint{1.331738in}{2.290045in}}{\pgfqpoint{1.339638in}{2.286773in}}{\pgfqpoint{1.347874in}{2.286773in}}%
\pgfpathclose%
\pgfusepath{stroke,fill}%
\end{pgfscope}%
\begin{pgfscope}%
\pgfpathrectangle{\pgfqpoint{0.100000in}{0.212622in}}{\pgfqpoint{3.696000in}{3.696000in}}%
\pgfusepath{clip}%
\pgfsetbuttcap%
\pgfsetroundjoin%
\definecolor{currentfill}{rgb}{0.121569,0.466667,0.705882}%
\pgfsetfillcolor{currentfill}%
\pgfsetfillopacity{0.897125}%
\pgfsetlinewidth{1.003750pt}%
\definecolor{currentstroke}{rgb}{0.121569,0.466667,0.705882}%
\pgfsetstrokecolor{currentstroke}%
\pgfsetstrokeopacity{0.897125}%
\pgfsetdash{}{0pt}%
\pgfpathmoveto{\pgfqpoint{1.364870in}{2.279954in}}%
\pgfpathcurveto{\pgfqpoint{1.373106in}{2.279954in}}{\pgfqpoint{1.381006in}{2.283226in}}{\pgfqpoint{1.386830in}{2.289050in}}%
\pgfpathcurveto{\pgfqpoint{1.392654in}{2.294874in}}{\pgfqpoint{1.395926in}{2.302774in}}{\pgfqpoint{1.395926in}{2.311010in}}%
\pgfpathcurveto{\pgfqpoint{1.395926in}{2.319247in}}{\pgfqpoint{1.392654in}{2.327147in}}{\pgfqpoint{1.386830in}{2.332971in}}%
\pgfpathcurveto{\pgfqpoint{1.381006in}{2.338795in}}{\pgfqpoint{1.373106in}{2.342067in}}{\pgfqpoint{1.364870in}{2.342067in}}%
\pgfpathcurveto{\pgfqpoint{1.356633in}{2.342067in}}{\pgfqpoint{1.348733in}{2.338795in}}{\pgfqpoint{1.342910in}{2.332971in}}%
\pgfpathcurveto{\pgfqpoint{1.337086in}{2.327147in}}{\pgfqpoint{1.333813in}{2.319247in}}{\pgfqpoint{1.333813in}{2.311010in}}%
\pgfpathcurveto{\pgfqpoint{1.333813in}{2.302774in}}{\pgfqpoint{1.337086in}{2.294874in}}{\pgfqpoint{1.342910in}{2.289050in}}%
\pgfpathcurveto{\pgfqpoint{1.348733in}{2.283226in}}{\pgfqpoint{1.356633in}{2.279954in}}{\pgfqpoint{1.364870in}{2.279954in}}%
\pgfpathclose%
\pgfusepath{stroke,fill}%
\end{pgfscope}%
\begin{pgfscope}%
\pgfpathrectangle{\pgfqpoint{0.100000in}{0.212622in}}{\pgfqpoint{3.696000in}{3.696000in}}%
\pgfusepath{clip}%
\pgfsetbuttcap%
\pgfsetroundjoin%
\definecolor{currentfill}{rgb}{0.121569,0.466667,0.705882}%
\pgfsetfillcolor{currentfill}%
\pgfsetfillopacity{0.897487}%
\pgfsetlinewidth{1.003750pt}%
\definecolor{currentstroke}{rgb}{0.121569,0.466667,0.705882}%
\pgfsetstrokecolor{currentstroke}%
\pgfsetstrokeopacity{0.897487}%
\pgfsetdash{}{0pt}%
\pgfpathmoveto{\pgfqpoint{2.634819in}{1.867883in}}%
\pgfpathcurveto{\pgfqpoint{2.643055in}{1.867883in}}{\pgfqpoint{2.650955in}{1.871155in}}{\pgfqpoint{2.656779in}{1.876979in}}%
\pgfpathcurveto{\pgfqpoint{2.662603in}{1.882803in}}{\pgfqpoint{2.665876in}{1.890703in}}{\pgfqpoint{2.665876in}{1.898939in}}%
\pgfpathcurveto{\pgfqpoint{2.665876in}{1.907175in}}{\pgfqpoint{2.662603in}{1.915076in}}{\pgfqpoint{2.656779in}{1.920899in}}%
\pgfpathcurveto{\pgfqpoint{2.650955in}{1.926723in}}{\pgfqpoint{2.643055in}{1.929996in}}{\pgfqpoint{2.634819in}{1.929996in}}%
\pgfpathcurveto{\pgfqpoint{2.626583in}{1.929996in}}{\pgfqpoint{2.618683in}{1.926723in}}{\pgfqpoint{2.612859in}{1.920899in}}%
\pgfpathcurveto{\pgfqpoint{2.607035in}{1.915076in}}{\pgfqpoint{2.603763in}{1.907175in}}{\pgfqpoint{2.603763in}{1.898939in}}%
\pgfpathcurveto{\pgfqpoint{2.603763in}{1.890703in}}{\pgfqpoint{2.607035in}{1.882803in}}{\pgfqpoint{2.612859in}{1.876979in}}%
\pgfpathcurveto{\pgfqpoint{2.618683in}{1.871155in}}{\pgfqpoint{2.626583in}{1.867883in}}{\pgfqpoint{2.634819in}{1.867883in}}%
\pgfpathclose%
\pgfusepath{stroke,fill}%
\end{pgfscope}%
\begin{pgfscope}%
\pgfpathrectangle{\pgfqpoint{0.100000in}{0.212622in}}{\pgfqpoint{3.696000in}{3.696000in}}%
\pgfusepath{clip}%
\pgfsetbuttcap%
\pgfsetroundjoin%
\definecolor{currentfill}{rgb}{0.121569,0.466667,0.705882}%
\pgfsetfillcolor{currentfill}%
\pgfsetfillopacity{0.898037}%
\pgfsetlinewidth{1.003750pt}%
\definecolor{currentstroke}{rgb}{0.121569,0.466667,0.705882}%
\pgfsetstrokecolor{currentstroke}%
\pgfsetstrokeopacity{0.898037}%
\pgfsetdash{}{0pt}%
\pgfpathmoveto{\pgfqpoint{1.379545in}{2.275474in}}%
\pgfpathcurveto{\pgfqpoint{1.387781in}{2.275474in}}{\pgfqpoint{1.395682in}{2.278747in}}{\pgfqpoint{1.401505in}{2.284571in}}%
\pgfpathcurveto{\pgfqpoint{1.407329in}{2.290394in}}{\pgfqpoint{1.410602in}{2.298294in}}{\pgfqpoint{1.410602in}{2.306531in}}%
\pgfpathcurveto{\pgfqpoint{1.410602in}{2.314767in}}{\pgfqpoint{1.407329in}{2.322667in}}{\pgfqpoint{1.401505in}{2.328491in}}%
\pgfpathcurveto{\pgfqpoint{1.395682in}{2.334315in}}{\pgfqpoint{1.387781in}{2.337587in}}{\pgfqpoint{1.379545in}{2.337587in}}%
\pgfpathcurveto{\pgfqpoint{1.371309in}{2.337587in}}{\pgfqpoint{1.363409in}{2.334315in}}{\pgfqpoint{1.357585in}{2.328491in}}%
\pgfpathcurveto{\pgfqpoint{1.351761in}{2.322667in}}{\pgfqpoint{1.348489in}{2.314767in}}{\pgfqpoint{1.348489in}{2.306531in}}%
\pgfpathcurveto{\pgfqpoint{1.348489in}{2.298294in}}{\pgfqpoint{1.351761in}{2.290394in}}{\pgfqpoint{1.357585in}{2.284571in}}%
\pgfpathcurveto{\pgfqpoint{1.363409in}{2.278747in}}{\pgfqpoint{1.371309in}{2.275474in}}{\pgfqpoint{1.379545in}{2.275474in}}%
\pgfpathclose%
\pgfusepath{stroke,fill}%
\end{pgfscope}%
\begin{pgfscope}%
\pgfpathrectangle{\pgfqpoint{0.100000in}{0.212622in}}{\pgfqpoint{3.696000in}{3.696000in}}%
\pgfusepath{clip}%
\pgfsetbuttcap%
\pgfsetroundjoin%
\definecolor{currentfill}{rgb}{0.121569,0.466667,0.705882}%
\pgfsetfillcolor{currentfill}%
\pgfsetfillopacity{0.899673}%
\pgfsetlinewidth{1.003750pt}%
\definecolor{currentstroke}{rgb}{0.121569,0.466667,0.705882}%
\pgfsetstrokecolor{currentstroke}%
\pgfsetstrokeopacity{0.899673}%
\pgfsetdash{}{0pt}%
\pgfpathmoveto{\pgfqpoint{1.392141in}{2.273453in}}%
\pgfpathcurveto{\pgfqpoint{1.400378in}{2.273453in}}{\pgfqpoint{1.408278in}{2.276725in}}{\pgfqpoint{1.414102in}{2.282549in}}%
\pgfpathcurveto{\pgfqpoint{1.419926in}{2.288373in}}{\pgfqpoint{1.423198in}{2.296273in}}{\pgfqpoint{1.423198in}{2.304509in}}%
\pgfpathcurveto{\pgfqpoint{1.423198in}{2.312746in}}{\pgfqpoint{1.419926in}{2.320646in}}{\pgfqpoint{1.414102in}{2.326470in}}%
\pgfpathcurveto{\pgfqpoint{1.408278in}{2.332294in}}{\pgfqpoint{1.400378in}{2.335566in}}{\pgfqpoint{1.392141in}{2.335566in}}%
\pgfpathcurveto{\pgfqpoint{1.383905in}{2.335566in}}{\pgfqpoint{1.376005in}{2.332294in}}{\pgfqpoint{1.370181in}{2.326470in}}%
\pgfpathcurveto{\pgfqpoint{1.364357in}{2.320646in}}{\pgfqpoint{1.361085in}{2.312746in}}{\pgfqpoint{1.361085in}{2.304509in}}%
\pgfpathcurveto{\pgfqpoint{1.361085in}{2.296273in}}{\pgfqpoint{1.364357in}{2.288373in}}{\pgfqpoint{1.370181in}{2.282549in}}%
\pgfpathcurveto{\pgfqpoint{1.376005in}{2.276725in}}{\pgfqpoint{1.383905in}{2.273453in}}{\pgfqpoint{1.392141in}{2.273453in}}%
\pgfpathclose%
\pgfusepath{stroke,fill}%
\end{pgfscope}%
\begin{pgfscope}%
\pgfpathrectangle{\pgfqpoint{0.100000in}{0.212622in}}{\pgfqpoint{3.696000in}{3.696000in}}%
\pgfusepath{clip}%
\pgfsetbuttcap%
\pgfsetroundjoin%
\definecolor{currentfill}{rgb}{0.121569,0.466667,0.705882}%
\pgfsetfillcolor{currentfill}%
\pgfsetfillopacity{0.900111}%
\pgfsetlinewidth{1.003750pt}%
\definecolor{currentstroke}{rgb}{0.121569,0.466667,0.705882}%
\pgfsetstrokecolor{currentstroke}%
\pgfsetstrokeopacity{0.900111}%
\pgfsetdash{}{0pt}%
\pgfpathmoveto{\pgfqpoint{2.626097in}{1.870630in}}%
\pgfpathcurveto{\pgfqpoint{2.634334in}{1.870630in}}{\pgfqpoint{2.642234in}{1.873902in}}{\pgfqpoint{2.648058in}{1.879726in}}%
\pgfpathcurveto{\pgfqpoint{2.653882in}{1.885550in}}{\pgfqpoint{2.657154in}{1.893450in}}{\pgfqpoint{2.657154in}{1.901686in}}%
\pgfpathcurveto{\pgfqpoint{2.657154in}{1.909922in}}{\pgfqpoint{2.653882in}{1.917822in}}{\pgfqpoint{2.648058in}{1.923646in}}%
\pgfpathcurveto{\pgfqpoint{2.642234in}{1.929470in}}{\pgfqpoint{2.634334in}{1.932743in}}{\pgfqpoint{2.626097in}{1.932743in}}%
\pgfpathcurveto{\pgfqpoint{2.617861in}{1.932743in}}{\pgfqpoint{2.609961in}{1.929470in}}{\pgfqpoint{2.604137in}{1.923646in}}%
\pgfpathcurveto{\pgfqpoint{2.598313in}{1.917822in}}{\pgfqpoint{2.595041in}{1.909922in}}{\pgfqpoint{2.595041in}{1.901686in}}%
\pgfpathcurveto{\pgfqpoint{2.595041in}{1.893450in}}{\pgfqpoint{2.598313in}{1.885550in}}{\pgfqpoint{2.604137in}{1.879726in}}%
\pgfpathcurveto{\pgfqpoint{2.609961in}{1.873902in}}{\pgfqpoint{2.617861in}{1.870630in}}{\pgfqpoint{2.626097in}{1.870630in}}%
\pgfpathclose%
\pgfusepath{stroke,fill}%
\end{pgfscope}%
\begin{pgfscope}%
\pgfpathrectangle{\pgfqpoint{0.100000in}{0.212622in}}{\pgfqpoint{3.696000in}{3.696000in}}%
\pgfusepath{clip}%
\pgfsetbuttcap%
\pgfsetroundjoin%
\definecolor{currentfill}{rgb}{0.121569,0.466667,0.705882}%
\pgfsetfillcolor{currentfill}%
\pgfsetfillopacity{0.901106}%
\pgfsetlinewidth{1.003750pt}%
\definecolor{currentstroke}{rgb}{0.121569,0.466667,0.705882}%
\pgfsetstrokecolor{currentstroke}%
\pgfsetstrokeopacity{0.901106}%
\pgfsetdash{}{0pt}%
\pgfpathmoveto{\pgfqpoint{1.416905in}{2.266521in}}%
\pgfpathcurveto{\pgfqpoint{1.425141in}{2.266521in}}{\pgfqpoint{1.433041in}{2.269793in}}{\pgfqpoint{1.438865in}{2.275617in}}%
\pgfpathcurveto{\pgfqpoint{1.444689in}{2.281441in}}{\pgfqpoint{1.447961in}{2.289341in}}{\pgfqpoint{1.447961in}{2.297578in}}%
\pgfpathcurveto{\pgfqpoint{1.447961in}{2.305814in}}{\pgfqpoint{1.444689in}{2.313714in}}{\pgfqpoint{1.438865in}{2.319538in}}%
\pgfpathcurveto{\pgfqpoint{1.433041in}{2.325362in}}{\pgfqpoint{1.425141in}{2.328634in}}{\pgfqpoint{1.416905in}{2.328634in}}%
\pgfpathcurveto{\pgfqpoint{1.408668in}{2.328634in}}{\pgfqpoint{1.400768in}{2.325362in}}{\pgfqpoint{1.394944in}{2.319538in}}%
\pgfpathcurveto{\pgfqpoint{1.389120in}{2.313714in}}{\pgfqpoint{1.385848in}{2.305814in}}{\pgfqpoint{1.385848in}{2.297578in}}%
\pgfpathcurveto{\pgfqpoint{1.385848in}{2.289341in}}{\pgfqpoint{1.389120in}{2.281441in}}{\pgfqpoint{1.394944in}{2.275617in}}%
\pgfpathcurveto{\pgfqpoint{1.400768in}{2.269793in}}{\pgfqpoint{1.408668in}{2.266521in}}{\pgfqpoint{1.416905in}{2.266521in}}%
\pgfpathclose%
\pgfusepath{stroke,fill}%
\end{pgfscope}%
\begin{pgfscope}%
\pgfpathrectangle{\pgfqpoint{0.100000in}{0.212622in}}{\pgfqpoint{3.696000in}{3.696000in}}%
\pgfusepath{clip}%
\pgfsetbuttcap%
\pgfsetroundjoin%
\definecolor{currentfill}{rgb}{0.121569,0.466667,0.705882}%
\pgfsetfillcolor{currentfill}%
\pgfsetfillopacity{0.902444}%
\pgfsetlinewidth{1.003750pt}%
\definecolor{currentstroke}{rgb}{0.121569,0.466667,0.705882}%
\pgfsetstrokecolor{currentstroke}%
\pgfsetstrokeopacity{0.902444}%
\pgfsetdash{}{0pt}%
\pgfpathmoveto{\pgfqpoint{1.441097in}{2.257862in}}%
\pgfpathcurveto{\pgfqpoint{1.449334in}{2.257862in}}{\pgfqpoint{1.457234in}{2.261134in}}{\pgfqpoint{1.463058in}{2.266958in}}%
\pgfpathcurveto{\pgfqpoint{1.468882in}{2.272782in}}{\pgfqpoint{1.472154in}{2.280682in}}{\pgfqpoint{1.472154in}{2.288919in}}%
\pgfpathcurveto{\pgfqpoint{1.472154in}{2.297155in}}{\pgfqpoint{1.468882in}{2.305055in}}{\pgfqpoint{1.463058in}{2.310879in}}%
\pgfpathcurveto{\pgfqpoint{1.457234in}{2.316703in}}{\pgfqpoint{1.449334in}{2.319975in}}{\pgfqpoint{1.441097in}{2.319975in}}%
\pgfpathcurveto{\pgfqpoint{1.432861in}{2.319975in}}{\pgfqpoint{1.424961in}{2.316703in}}{\pgfqpoint{1.419137in}{2.310879in}}%
\pgfpathcurveto{\pgfqpoint{1.413313in}{2.305055in}}{\pgfqpoint{1.410041in}{2.297155in}}{\pgfqpoint{1.410041in}{2.288919in}}%
\pgfpathcurveto{\pgfqpoint{1.410041in}{2.280682in}}{\pgfqpoint{1.413313in}{2.272782in}}{\pgfqpoint{1.419137in}{2.266958in}}%
\pgfpathcurveto{\pgfqpoint{1.424961in}{2.261134in}}{\pgfqpoint{1.432861in}{2.257862in}}{\pgfqpoint{1.441097in}{2.257862in}}%
\pgfpathclose%
\pgfusepath{stroke,fill}%
\end{pgfscope}%
\begin{pgfscope}%
\pgfpathrectangle{\pgfqpoint{0.100000in}{0.212622in}}{\pgfqpoint{3.696000in}{3.696000in}}%
\pgfusepath{clip}%
\pgfsetbuttcap%
\pgfsetroundjoin%
\definecolor{currentfill}{rgb}{0.121569,0.466667,0.705882}%
\pgfsetfillcolor{currentfill}%
\pgfsetfillopacity{0.904122}%
\pgfsetlinewidth{1.003750pt}%
\definecolor{currentstroke}{rgb}{0.121569,0.466667,0.705882}%
\pgfsetstrokecolor{currentstroke}%
\pgfsetstrokeopacity{0.904122}%
\pgfsetdash{}{0pt}%
\pgfpathmoveto{\pgfqpoint{2.621597in}{1.870957in}}%
\pgfpathcurveto{\pgfqpoint{2.629833in}{1.870957in}}{\pgfqpoint{2.637733in}{1.874229in}}{\pgfqpoint{2.643557in}{1.880053in}}%
\pgfpathcurveto{\pgfqpoint{2.649381in}{1.885877in}}{\pgfqpoint{2.652653in}{1.893777in}}{\pgfqpoint{2.652653in}{1.902013in}}%
\pgfpathcurveto{\pgfqpoint{2.652653in}{1.910249in}}{\pgfqpoint{2.649381in}{1.918149in}}{\pgfqpoint{2.643557in}{1.923973in}}%
\pgfpathcurveto{\pgfqpoint{2.637733in}{1.929797in}}{\pgfqpoint{2.629833in}{1.933070in}}{\pgfqpoint{2.621597in}{1.933070in}}%
\pgfpathcurveto{\pgfqpoint{2.613361in}{1.933070in}}{\pgfqpoint{2.605461in}{1.929797in}}{\pgfqpoint{2.599637in}{1.923973in}}%
\pgfpathcurveto{\pgfqpoint{2.593813in}{1.918149in}}{\pgfqpoint{2.590540in}{1.910249in}}{\pgfqpoint{2.590540in}{1.902013in}}%
\pgfpathcurveto{\pgfqpoint{2.590540in}{1.893777in}}{\pgfqpoint{2.593813in}{1.885877in}}{\pgfqpoint{2.599637in}{1.880053in}}%
\pgfpathcurveto{\pgfqpoint{2.605461in}{1.874229in}}{\pgfqpoint{2.613361in}{1.870957in}}{\pgfqpoint{2.621597in}{1.870957in}}%
\pgfpathclose%
\pgfusepath{stroke,fill}%
\end{pgfscope}%
\begin{pgfscope}%
\pgfpathrectangle{\pgfqpoint{0.100000in}{0.212622in}}{\pgfqpoint{3.696000in}{3.696000in}}%
\pgfusepath{clip}%
\pgfsetbuttcap%
\pgfsetroundjoin%
\definecolor{currentfill}{rgb}{0.121569,0.466667,0.705882}%
\pgfsetfillcolor{currentfill}%
\pgfsetfillopacity{0.904295}%
\pgfsetlinewidth{1.003750pt}%
\definecolor{currentstroke}{rgb}{0.121569,0.466667,0.705882}%
\pgfsetstrokecolor{currentstroke}%
\pgfsetstrokeopacity{0.904295}%
\pgfsetdash{}{0pt}%
\pgfpathmoveto{\pgfqpoint{1.459420in}{2.247345in}}%
\pgfpathcurveto{\pgfqpoint{1.467656in}{2.247345in}}{\pgfqpoint{1.475556in}{2.250617in}}{\pgfqpoint{1.481380in}{2.256441in}}%
\pgfpathcurveto{\pgfqpoint{1.487204in}{2.262265in}}{\pgfqpoint{1.490477in}{2.270165in}}{\pgfqpoint{1.490477in}{2.278402in}}%
\pgfpathcurveto{\pgfqpoint{1.490477in}{2.286638in}}{\pgfqpoint{1.487204in}{2.294538in}}{\pgfqpoint{1.481380in}{2.300362in}}%
\pgfpathcurveto{\pgfqpoint{1.475556in}{2.306186in}}{\pgfqpoint{1.467656in}{2.309458in}}{\pgfqpoint{1.459420in}{2.309458in}}%
\pgfpathcurveto{\pgfqpoint{1.451184in}{2.309458in}}{\pgfqpoint{1.443284in}{2.306186in}}{\pgfqpoint{1.437460in}{2.300362in}}%
\pgfpathcurveto{\pgfqpoint{1.431636in}{2.294538in}}{\pgfqpoint{1.428364in}{2.286638in}}{\pgfqpoint{1.428364in}{2.278402in}}%
\pgfpathcurveto{\pgfqpoint{1.428364in}{2.270165in}}{\pgfqpoint{1.431636in}{2.262265in}}{\pgfqpoint{1.437460in}{2.256441in}}%
\pgfpathcurveto{\pgfqpoint{1.443284in}{2.250617in}}{\pgfqpoint{1.451184in}{2.247345in}}{\pgfqpoint{1.459420in}{2.247345in}}%
\pgfpathclose%
\pgfusepath{stroke,fill}%
\end{pgfscope}%
\begin{pgfscope}%
\pgfpathrectangle{\pgfqpoint{0.100000in}{0.212622in}}{\pgfqpoint{3.696000in}{3.696000in}}%
\pgfusepath{clip}%
\pgfsetbuttcap%
\pgfsetroundjoin%
\definecolor{currentfill}{rgb}{0.121569,0.466667,0.705882}%
\pgfsetfillcolor{currentfill}%
\pgfsetfillopacity{0.905322}%
\pgfsetlinewidth{1.003750pt}%
\definecolor{currentstroke}{rgb}{0.121569,0.466667,0.705882}%
\pgfsetstrokecolor{currentstroke}%
\pgfsetstrokeopacity{0.905322}%
\pgfsetdash{}{0pt}%
\pgfpathmoveto{\pgfqpoint{1.477950in}{2.241056in}}%
\pgfpathcurveto{\pgfqpoint{1.486186in}{2.241056in}}{\pgfqpoint{1.494087in}{2.244328in}}{\pgfqpoint{1.499910in}{2.250152in}}%
\pgfpathcurveto{\pgfqpoint{1.505734in}{2.255976in}}{\pgfqpoint{1.509007in}{2.263876in}}{\pgfqpoint{1.509007in}{2.272113in}}%
\pgfpathcurveto{\pgfqpoint{1.509007in}{2.280349in}}{\pgfqpoint{1.505734in}{2.288249in}}{\pgfqpoint{1.499910in}{2.294073in}}%
\pgfpathcurveto{\pgfqpoint{1.494087in}{2.299897in}}{\pgfqpoint{1.486186in}{2.303169in}}{\pgfqpoint{1.477950in}{2.303169in}}%
\pgfpathcurveto{\pgfqpoint{1.469714in}{2.303169in}}{\pgfqpoint{1.461814in}{2.299897in}}{\pgfqpoint{1.455990in}{2.294073in}}%
\pgfpathcurveto{\pgfqpoint{1.450166in}{2.288249in}}{\pgfqpoint{1.446894in}{2.280349in}}{\pgfqpoint{1.446894in}{2.272113in}}%
\pgfpathcurveto{\pgfqpoint{1.446894in}{2.263876in}}{\pgfqpoint{1.450166in}{2.255976in}}{\pgfqpoint{1.455990in}{2.250152in}}%
\pgfpathcurveto{\pgfqpoint{1.461814in}{2.244328in}}{\pgfqpoint{1.469714in}{2.241056in}}{\pgfqpoint{1.477950in}{2.241056in}}%
\pgfpathclose%
\pgfusepath{stroke,fill}%
\end{pgfscope}%
\begin{pgfscope}%
\pgfpathrectangle{\pgfqpoint{0.100000in}{0.212622in}}{\pgfqpoint{3.696000in}{3.696000in}}%
\pgfusepath{clip}%
\pgfsetbuttcap%
\pgfsetroundjoin%
\definecolor{currentfill}{rgb}{0.121569,0.466667,0.705882}%
\pgfsetfillcolor{currentfill}%
\pgfsetfillopacity{0.906163}%
\pgfsetlinewidth{1.003750pt}%
\definecolor{currentstroke}{rgb}{0.121569,0.466667,0.705882}%
\pgfsetstrokecolor{currentstroke}%
\pgfsetstrokeopacity{0.906163}%
\pgfsetdash{}{0pt}%
\pgfpathmoveto{\pgfqpoint{2.617583in}{1.871438in}}%
\pgfpathcurveto{\pgfqpoint{2.625819in}{1.871438in}}{\pgfqpoint{2.633719in}{1.874710in}}{\pgfqpoint{2.639543in}{1.880534in}}%
\pgfpathcurveto{\pgfqpoint{2.645367in}{1.886358in}}{\pgfqpoint{2.648639in}{1.894258in}}{\pgfqpoint{2.648639in}{1.902494in}}%
\pgfpathcurveto{\pgfqpoint{2.648639in}{1.910730in}}{\pgfqpoint{2.645367in}{1.918630in}}{\pgfqpoint{2.639543in}{1.924454in}}%
\pgfpathcurveto{\pgfqpoint{2.633719in}{1.930278in}}{\pgfqpoint{2.625819in}{1.933551in}}{\pgfqpoint{2.617583in}{1.933551in}}%
\pgfpathcurveto{\pgfqpoint{2.609346in}{1.933551in}}{\pgfqpoint{2.601446in}{1.930278in}}{\pgfqpoint{2.595622in}{1.924454in}}%
\pgfpathcurveto{\pgfqpoint{2.589798in}{1.918630in}}{\pgfqpoint{2.586526in}{1.910730in}}{\pgfqpoint{2.586526in}{1.902494in}}%
\pgfpathcurveto{\pgfqpoint{2.586526in}{1.894258in}}{\pgfqpoint{2.589798in}{1.886358in}}{\pgfqpoint{2.595622in}{1.880534in}}%
\pgfpathcurveto{\pgfqpoint{2.601446in}{1.874710in}}{\pgfqpoint{2.609346in}{1.871438in}}{\pgfqpoint{2.617583in}{1.871438in}}%
\pgfpathclose%
\pgfusepath{stroke,fill}%
\end{pgfscope}%
\begin{pgfscope}%
\pgfpathrectangle{\pgfqpoint{0.100000in}{0.212622in}}{\pgfqpoint{3.696000in}{3.696000in}}%
\pgfusepath{clip}%
\pgfsetbuttcap%
\pgfsetroundjoin%
\definecolor{currentfill}{rgb}{0.121569,0.466667,0.705882}%
\pgfsetfillcolor{currentfill}%
\pgfsetfillopacity{0.906187}%
\pgfsetlinewidth{1.003750pt}%
\definecolor{currentstroke}{rgb}{0.121569,0.466667,0.705882}%
\pgfsetstrokecolor{currentstroke}%
\pgfsetstrokeopacity{0.906187}%
\pgfsetdash{}{0pt}%
\pgfpathmoveto{\pgfqpoint{1.495962in}{2.234023in}}%
\pgfpathcurveto{\pgfqpoint{1.504198in}{2.234023in}}{\pgfqpoint{1.512098in}{2.237296in}}{\pgfqpoint{1.517922in}{2.243120in}}%
\pgfpathcurveto{\pgfqpoint{1.523746in}{2.248943in}}{\pgfqpoint{1.527018in}{2.256843in}}{\pgfqpoint{1.527018in}{2.265080in}}%
\pgfpathcurveto{\pgfqpoint{1.527018in}{2.273316in}}{\pgfqpoint{1.523746in}{2.281216in}}{\pgfqpoint{1.517922in}{2.287040in}}%
\pgfpathcurveto{\pgfqpoint{1.512098in}{2.292864in}}{\pgfqpoint{1.504198in}{2.296136in}}{\pgfqpoint{1.495962in}{2.296136in}}%
\pgfpathcurveto{\pgfqpoint{1.487726in}{2.296136in}}{\pgfqpoint{1.479826in}{2.292864in}}{\pgfqpoint{1.474002in}{2.287040in}}%
\pgfpathcurveto{\pgfqpoint{1.468178in}{2.281216in}}{\pgfqpoint{1.464905in}{2.273316in}}{\pgfqpoint{1.464905in}{2.265080in}}%
\pgfpathcurveto{\pgfqpoint{1.464905in}{2.256843in}}{\pgfqpoint{1.468178in}{2.248943in}}{\pgfqpoint{1.474002in}{2.243120in}}%
\pgfpathcurveto{\pgfqpoint{1.479826in}{2.237296in}}{\pgfqpoint{1.487726in}{2.234023in}}{\pgfqpoint{1.495962in}{2.234023in}}%
\pgfpathclose%
\pgfusepath{stroke,fill}%
\end{pgfscope}%
\begin{pgfscope}%
\pgfpathrectangle{\pgfqpoint{0.100000in}{0.212622in}}{\pgfqpoint{3.696000in}{3.696000in}}%
\pgfusepath{clip}%
\pgfsetbuttcap%
\pgfsetroundjoin%
\definecolor{currentfill}{rgb}{0.121569,0.466667,0.705882}%
\pgfsetfillcolor{currentfill}%
\pgfsetfillopacity{0.906967}%
\pgfsetlinewidth{1.003750pt}%
\definecolor{currentstroke}{rgb}{0.121569,0.466667,0.705882}%
\pgfsetstrokecolor{currentstroke}%
\pgfsetstrokeopacity{0.906967}%
\pgfsetdash{}{0pt}%
\pgfpathmoveto{\pgfqpoint{1.510734in}{2.227792in}}%
\pgfpathcurveto{\pgfqpoint{1.518971in}{2.227792in}}{\pgfqpoint{1.526871in}{2.231064in}}{\pgfqpoint{1.532695in}{2.236888in}}%
\pgfpathcurveto{\pgfqpoint{1.538518in}{2.242712in}}{\pgfqpoint{1.541791in}{2.250612in}}{\pgfqpoint{1.541791in}{2.258849in}}%
\pgfpathcurveto{\pgfqpoint{1.541791in}{2.267085in}}{\pgfqpoint{1.538518in}{2.274985in}}{\pgfqpoint{1.532695in}{2.280809in}}%
\pgfpathcurveto{\pgfqpoint{1.526871in}{2.286633in}}{\pgfqpoint{1.518971in}{2.289905in}}{\pgfqpoint{1.510734in}{2.289905in}}%
\pgfpathcurveto{\pgfqpoint{1.502498in}{2.289905in}}{\pgfqpoint{1.494598in}{2.286633in}}{\pgfqpoint{1.488774in}{2.280809in}}%
\pgfpathcurveto{\pgfqpoint{1.482950in}{2.274985in}}{\pgfqpoint{1.479678in}{2.267085in}}{\pgfqpoint{1.479678in}{2.258849in}}%
\pgfpathcurveto{\pgfqpoint{1.479678in}{2.250612in}}{\pgfqpoint{1.482950in}{2.242712in}}{\pgfqpoint{1.488774in}{2.236888in}}%
\pgfpathcurveto{\pgfqpoint{1.494598in}{2.231064in}}{\pgfqpoint{1.502498in}{2.227792in}}{\pgfqpoint{1.510734in}{2.227792in}}%
\pgfpathclose%
\pgfusepath{stroke,fill}%
\end{pgfscope}%
\begin{pgfscope}%
\pgfpathrectangle{\pgfqpoint{0.100000in}{0.212622in}}{\pgfqpoint{3.696000in}{3.696000in}}%
\pgfusepath{clip}%
\pgfsetbuttcap%
\pgfsetroundjoin%
\definecolor{currentfill}{rgb}{0.121569,0.466667,0.705882}%
\pgfsetfillcolor{currentfill}%
\pgfsetfillopacity{0.907850}%
\pgfsetlinewidth{1.003750pt}%
\definecolor{currentstroke}{rgb}{0.121569,0.466667,0.705882}%
\pgfsetstrokecolor{currentstroke}%
\pgfsetstrokeopacity{0.907850}%
\pgfsetdash{}{0pt}%
\pgfpathmoveto{\pgfqpoint{1.522965in}{2.224221in}}%
\pgfpathcurveto{\pgfqpoint{1.531201in}{2.224221in}}{\pgfqpoint{1.539101in}{2.227494in}}{\pgfqpoint{1.544925in}{2.233317in}}%
\pgfpathcurveto{\pgfqpoint{1.550749in}{2.239141in}}{\pgfqpoint{1.554022in}{2.247041in}}{\pgfqpoint{1.554022in}{2.255278in}}%
\pgfpathcurveto{\pgfqpoint{1.554022in}{2.263514in}}{\pgfqpoint{1.550749in}{2.271414in}}{\pgfqpoint{1.544925in}{2.277238in}}%
\pgfpathcurveto{\pgfqpoint{1.539101in}{2.283062in}}{\pgfqpoint{1.531201in}{2.286334in}}{\pgfqpoint{1.522965in}{2.286334in}}%
\pgfpathcurveto{\pgfqpoint{1.514729in}{2.286334in}}{\pgfqpoint{1.506829in}{2.283062in}}{\pgfqpoint{1.501005in}{2.277238in}}%
\pgfpathcurveto{\pgfqpoint{1.495181in}{2.271414in}}{\pgfqpoint{1.491909in}{2.263514in}}{\pgfqpoint{1.491909in}{2.255278in}}%
\pgfpathcurveto{\pgfqpoint{1.491909in}{2.247041in}}{\pgfqpoint{1.495181in}{2.239141in}}{\pgfqpoint{1.501005in}{2.233317in}}%
\pgfpathcurveto{\pgfqpoint{1.506829in}{2.227494in}}{\pgfqpoint{1.514729in}{2.224221in}}{\pgfqpoint{1.522965in}{2.224221in}}%
\pgfpathclose%
\pgfusepath{stroke,fill}%
\end{pgfscope}%
\begin{pgfscope}%
\pgfpathrectangle{\pgfqpoint{0.100000in}{0.212622in}}{\pgfqpoint{3.696000in}{3.696000in}}%
\pgfusepath{clip}%
\pgfsetbuttcap%
\pgfsetroundjoin%
\definecolor{currentfill}{rgb}{0.121569,0.466667,0.705882}%
\pgfsetfillcolor{currentfill}%
\pgfsetfillopacity{0.908292}%
\pgfsetlinewidth{1.003750pt}%
\definecolor{currentstroke}{rgb}{0.121569,0.466667,0.705882}%
\pgfsetstrokecolor{currentstroke}%
\pgfsetstrokeopacity{0.908292}%
\pgfsetdash{}{0pt}%
\pgfpathmoveto{\pgfqpoint{2.610910in}{1.873245in}}%
\pgfpathcurveto{\pgfqpoint{2.619147in}{1.873245in}}{\pgfqpoint{2.627047in}{1.876518in}}{\pgfqpoint{2.632871in}{1.882342in}}%
\pgfpathcurveto{\pgfqpoint{2.638694in}{1.888166in}}{\pgfqpoint{2.641967in}{1.896066in}}{\pgfqpoint{2.641967in}{1.904302in}}%
\pgfpathcurveto{\pgfqpoint{2.641967in}{1.912538in}}{\pgfqpoint{2.638694in}{1.920438in}}{\pgfqpoint{2.632871in}{1.926262in}}%
\pgfpathcurveto{\pgfqpoint{2.627047in}{1.932086in}}{\pgfqpoint{2.619147in}{1.935358in}}{\pgfqpoint{2.610910in}{1.935358in}}%
\pgfpathcurveto{\pgfqpoint{2.602674in}{1.935358in}}{\pgfqpoint{2.594774in}{1.932086in}}{\pgfqpoint{2.588950in}{1.926262in}}%
\pgfpathcurveto{\pgfqpoint{2.583126in}{1.920438in}}{\pgfqpoint{2.579854in}{1.912538in}}{\pgfqpoint{2.579854in}{1.904302in}}%
\pgfpathcurveto{\pgfqpoint{2.579854in}{1.896066in}}{\pgfqpoint{2.583126in}{1.888166in}}{\pgfqpoint{2.588950in}{1.882342in}}%
\pgfpathcurveto{\pgfqpoint{2.594774in}{1.876518in}}{\pgfqpoint{2.602674in}{1.873245in}}{\pgfqpoint{2.610910in}{1.873245in}}%
\pgfpathclose%
\pgfusepath{stroke,fill}%
\end{pgfscope}%
\begin{pgfscope}%
\pgfpathrectangle{\pgfqpoint{0.100000in}{0.212622in}}{\pgfqpoint{3.696000in}{3.696000in}}%
\pgfusepath{clip}%
\pgfsetbuttcap%
\pgfsetroundjoin%
\definecolor{currentfill}{rgb}{0.121569,0.466667,0.705882}%
\pgfsetfillcolor{currentfill}%
\pgfsetfillopacity{0.909408}%
\pgfsetlinewidth{1.003750pt}%
\definecolor{currentstroke}{rgb}{0.121569,0.466667,0.705882}%
\pgfsetstrokecolor{currentstroke}%
\pgfsetstrokeopacity{0.909408}%
\pgfsetdash{}{0pt}%
\pgfpathmoveto{\pgfqpoint{1.544840in}{2.215600in}}%
\pgfpathcurveto{\pgfqpoint{1.553076in}{2.215600in}}{\pgfqpoint{1.560976in}{2.218872in}}{\pgfqpoint{1.566800in}{2.224696in}}%
\pgfpathcurveto{\pgfqpoint{1.572624in}{2.230520in}}{\pgfqpoint{1.575896in}{2.238420in}}{\pgfqpoint{1.575896in}{2.246656in}}%
\pgfpathcurveto{\pgfqpoint{1.575896in}{2.254893in}}{\pgfqpoint{1.572624in}{2.262793in}}{\pgfqpoint{1.566800in}{2.268617in}}%
\pgfpathcurveto{\pgfqpoint{1.560976in}{2.274441in}}{\pgfqpoint{1.553076in}{2.277713in}}{\pgfqpoint{1.544840in}{2.277713in}}%
\pgfpathcurveto{\pgfqpoint{1.536603in}{2.277713in}}{\pgfqpoint{1.528703in}{2.274441in}}{\pgfqpoint{1.522880in}{2.268617in}}%
\pgfpathcurveto{\pgfqpoint{1.517056in}{2.262793in}}{\pgfqpoint{1.513783in}{2.254893in}}{\pgfqpoint{1.513783in}{2.246656in}}%
\pgfpathcurveto{\pgfqpoint{1.513783in}{2.238420in}}{\pgfqpoint{1.517056in}{2.230520in}}{\pgfqpoint{1.522880in}{2.224696in}}%
\pgfpathcurveto{\pgfqpoint{1.528703in}{2.218872in}}{\pgfqpoint{1.536603in}{2.215600in}}{\pgfqpoint{1.544840in}{2.215600in}}%
\pgfpathclose%
\pgfusepath{stroke,fill}%
\end{pgfscope}%
\begin{pgfscope}%
\pgfpathrectangle{\pgfqpoint{0.100000in}{0.212622in}}{\pgfqpoint{3.696000in}{3.696000in}}%
\pgfusepath{clip}%
\pgfsetbuttcap%
\pgfsetroundjoin%
\definecolor{currentfill}{rgb}{0.121569,0.466667,0.705882}%
\pgfsetfillcolor{currentfill}%
\pgfsetfillopacity{0.910843}%
\pgfsetlinewidth{1.003750pt}%
\definecolor{currentstroke}{rgb}{0.121569,0.466667,0.705882}%
\pgfsetstrokecolor{currentstroke}%
\pgfsetstrokeopacity{0.910843}%
\pgfsetdash{}{0pt}%
\pgfpathmoveto{\pgfqpoint{1.566303in}{2.207056in}}%
\pgfpathcurveto{\pgfqpoint{1.574540in}{2.207056in}}{\pgfqpoint{1.582440in}{2.210328in}}{\pgfqpoint{1.588264in}{2.216152in}}%
\pgfpathcurveto{\pgfqpoint{1.594088in}{2.221976in}}{\pgfqpoint{1.597360in}{2.229876in}}{\pgfqpoint{1.597360in}{2.238112in}}%
\pgfpathcurveto{\pgfqpoint{1.597360in}{2.246349in}}{\pgfqpoint{1.594088in}{2.254249in}}{\pgfqpoint{1.588264in}{2.260073in}}%
\pgfpathcurveto{\pgfqpoint{1.582440in}{2.265897in}}{\pgfqpoint{1.574540in}{2.269169in}}{\pgfqpoint{1.566303in}{2.269169in}}%
\pgfpathcurveto{\pgfqpoint{1.558067in}{2.269169in}}{\pgfqpoint{1.550167in}{2.265897in}}{\pgfqpoint{1.544343in}{2.260073in}}%
\pgfpathcurveto{\pgfqpoint{1.538519in}{2.254249in}}{\pgfqpoint{1.535247in}{2.246349in}}{\pgfqpoint{1.535247in}{2.238112in}}%
\pgfpathcurveto{\pgfqpoint{1.535247in}{2.229876in}}{\pgfqpoint{1.538519in}{2.221976in}}{\pgfqpoint{1.544343in}{2.216152in}}%
\pgfpathcurveto{\pgfqpoint{1.550167in}{2.210328in}}{\pgfqpoint{1.558067in}{2.207056in}}{\pgfqpoint{1.566303in}{2.207056in}}%
\pgfpathclose%
\pgfusepath{stroke,fill}%
\end{pgfscope}%
\begin{pgfscope}%
\pgfpathrectangle{\pgfqpoint{0.100000in}{0.212622in}}{\pgfqpoint{3.696000in}{3.696000in}}%
\pgfusepath{clip}%
\pgfsetbuttcap%
\pgfsetroundjoin%
\definecolor{currentfill}{rgb}{0.121569,0.466667,0.705882}%
\pgfsetfillcolor{currentfill}%
\pgfsetfillopacity{0.911510}%
\pgfsetlinewidth{1.003750pt}%
\definecolor{currentstroke}{rgb}{0.121569,0.466667,0.705882}%
\pgfsetstrokecolor{currentstroke}%
\pgfsetstrokeopacity{0.911510}%
\pgfsetdash{}{0pt}%
\pgfpathmoveto{\pgfqpoint{2.604360in}{1.873844in}}%
\pgfpathcurveto{\pgfqpoint{2.612597in}{1.873844in}}{\pgfqpoint{2.620497in}{1.877116in}}{\pgfqpoint{2.626321in}{1.882940in}}%
\pgfpathcurveto{\pgfqpoint{2.632144in}{1.888764in}}{\pgfqpoint{2.635417in}{1.896664in}}{\pgfqpoint{2.635417in}{1.904900in}}%
\pgfpathcurveto{\pgfqpoint{2.635417in}{1.913136in}}{\pgfqpoint{2.632144in}{1.921036in}}{\pgfqpoint{2.626321in}{1.926860in}}%
\pgfpathcurveto{\pgfqpoint{2.620497in}{1.932684in}}{\pgfqpoint{2.612597in}{1.935957in}}{\pgfqpoint{2.604360in}{1.935957in}}%
\pgfpathcurveto{\pgfqpoint{2.596124in}{1.935957in}}{\pgfqpoint{2.588224in}{1.932684in}}{\pgfqpoint{2.582400in}{1.926860in}}%
\pgfpathcurveto{\pgfqpoint{2.576576in}{1.921036in}}{\pgfqpoint{2.573304in}{1.913136in}}{\pgfqpoint{2.573304in}{1.904900in}}%
\pgfpathcurveto{\pgfqpoint{2.573304in}{1.896664in}}{\pgfqpoint{2.576576in}{1.888764in}}{\pgfqpoint{2.582400in}{1.882940in}}%
\pgfpathcurveto{\pgfqpoint{2.588224in}{1.877116in}}{\pgfqpoint{2.596124in}{1.873844in}}{\pgfqpoint{2.604360in}{1.873844in}}%
\pgfpathclose%
\pgfusepath{stroke,fill}%
\end{pgfscope}%
\begin{pgfscope}%
\pgfpathrectangle{\pgfqpoint{0.100000in}{0.212622in}}{\pgfqpoint{3.696000in}{3.696000in}}%
\pgfusepath{clip}%
\pgfsetbuttcap%
\pgfsetroundjoin%
\definecolor{currentfill}{rgb}{0.121569,0.466667,0.705882}%
\pgfsetfillcolor{currentfill}%
\pgfsetfillopacity{0.912234}%
\pgfsetlinewidth{1.003750pt}%
\definecolor{currentstroke}{rgb}{0.121569,0.466667,0.705882}%
\pgfsetstrokecolor{currentstroke}%
\pgfsetstrokeopacity{0.912234}%
\pgfsetdash{}{0pt}%
\pgfpathmoveto{\pgfqpoint{1.584615in}{2.201983in}}%
\pgfpathcurveto{\pgfqpoint{1.592851in}{2.201983in}}{\pgfqpoint{1.600751in}{2.205256in}}{\pgfqpoint{1.606575in}{2.211079in}}%
\pgfpathcurveto{\pgfqpoint{1.612399in}{2.216903in}}{\pgfqpoint{1.615672in}{2.224803in}}{\pgfqpoint{1.615672in}{2.233040in}}%
\pgfpathcurveto{\pgfqpoint{1.615672in}{2.241276in}}{\pgfqpoint{1.612399in}{2.249176in}}{\pgfqpoint{1.606575in}{2.255000in}}%
\pgfpathcurveto{\pgfqpoint{1.600751in}{2.260824in}}{\pgfqpoint{1.592851in}{2.264096in}}{\pgfqpoint{1.584615in}{2.264096in}}%
\pgfpathcurveto{\pgfqpoint{1.576379in}{2.264096in}}{\pgfqpoint{1.568479in}{2.260824in}}{\pgfqpoint{1.562655in}{2.255000in}}%
\pgfpathcurveto{\pgfqpoint{1.556831in}{2.249176in}}{\pgfqpoint{1.553559in}{2.241276in}}{\pgfqpoint{1.553559in}{2.233040in}}%
\pgfpathcurveto{\pgfqpoint{1.553559in}{2.224803in}}{\pgfqpoint{1.556831in}{2.216903in}}{\pgfqpoint{1.562655in}{2.211079in}}%
\pgfpathcurveto{\pgfqpoint{1.568479in}{2.205256in}}{\pgfqpoint{1.576379in}{2.201983in}}{\pgfqpoint{1.584615in}{2.201983in}}%
\pgfpathclose%
\pgfusepath{stroke,fill}%
\end{pgfscope}%
\begin{pgfscope}%
\pgfpathrectangle{\pgfqpoint{0.100000in}{0.212622in}}{\pgfqpoint{3.696000in}{3.696000in}}%
\pgfusepath{clip}%
\pgfsetbuttcap%
\pgfsetroundjoin%
\definecolor{currentfill}{rgb}{0.121569,0.466667,0.705882}%
\pgfsetfillcolor{currentfill}%
\pgfsetfillopacity{0.913412}%
\pgfsetlinewidth{1.003750pt}%
\definecolor{currentstroke}{rgb}{0.121569,0.466667,0.705882}%
\pgfsetstrokecolor{currentstroke}%
\pgfsetstrokeopacity{0.913412}%
\pgfsetdash{}{0pt}%
\pgfpathmoveto{\pgfqpoint{2.601635in}{1.874140in}}%
\pgfpathcurveto{\pgfqpoint{2.609872in}{1.874140in}}{\pgfqpoint{2.617772in}{1.877412in}}{\pgfqpoint{2.623596in}{1.883236in}}%
\pgfpathcurveto{\pgfqpoint{2.629420in}{1.889060in}}{\pgfqpoint{2.632692in}{1.896960in}}{\pgfqpoint{2.632692in}{1.905197in}}%
\pgfpathcurveto{\pgfqpoint{2.632692in}{1.913433in}}{\pgfqpoint{2.629420in}{1.921333in}}{\pgfqpoint{2.623596in}{1.927157in}}%
\pgfpathcurveto{\pgfqpoint{2.617772in}{1.932981in}}{\pgfqpoint{2.609872in}{1.936253in}}{\pgfqpoint{2.601635in}{1.936253in}}%
\pgfpathcurveto{\pgfqpoint{2.593399in}{1.936253in}}{\pgfqpoint{2.585499in}{1.932981in}}{\pgfqpoint{2.579675in}{1.927157in}}%
\pgfpathcurveto{\pgfqpoint{2.573851in}{1.921333in}}{\pgfqpoint{2.570579in}{1.913433in}}{\pgfqpoint{2.570579in}{1.905197in}}%
\pgfpathcurveto{\pgfqpoint{2.570579in}{1.896960in}}{\pgfqpoint{2.573851in}{1.889060in}}{\pgfqpoint{2.579675in}{1.883236in}}%
\pgfpathcurveto{\pgfqpoint{2.585499in}{1.877412in}}{\pgfqpoint{2.593399in}{1.874140in}}{\pgfqpoint{2.601635in}{1.874140in}}%
\pgfpathclose%
\pgfusepath{stroke,fill}%
\end{pgfscope}%
\begin{pgfscope}%
\pgfpathrectangle{\pgfqpoint{0.100000in}{0.212622in}}{\pgfqpoint{3.696000in}{3.696000in}}%
\pgfusepath{clip}%
\pgfsetbuttcap%
\pgfsetroundjoin%
\definecolor{currentfill}{rgb}{0.121569,0.466667,0.705882}%
\pgfsetfillcolor{currentfill}%
\pgfsetfillopacity{0.914117}%
\pgfsetlinewidth{1.003750pt}%
\definecolor{currentstroke}{rgb}{0.121569,0.466667,0.705882}%
\pgfsetstrokecolor{currentstroke}%
\pgfsetstrokeopacity{0.914117}%
\pgfsetdash{}{0pt}%
\pgfpathmoveto{\pgfqpoint{1.600914in}{2.196336in}}%
\pgfpathcurveto{\pgfqpoint{1.609150in}{2.196336in}}{\pgfqpoint{1.617050in}{2.199609in}}{\pgfqpoint{1.622874in}{2.205433in}}%
\pgfpathcurveto{\pgfqpoint{1.628698in}{2.211257in}}{\pgfqpoint{1.631970in}{2.219157in}}{\pgfqpoint{1.631970in}{2.227393in}}%
\pgfpathcurveto{\pgfqpoint{1.631970in}{2.235629in}}{\pgfqpoint{1.628698in}{2.243529in}}{\pgfqpoint{1.622874in}{2.249353in}}%
\pgfpathcurveto{\pgfqpoint{1.617050in}{2.255177in}}{\pgfqpoint{1.609150in}{2.258449in}}{\pgfqpoint{1.600914in}{2.258449in}}%
\pgfpathcurveto{\pgfqpoint{1.592678in}{2.258449in}}{\pgfqpoint{1.584778in}{2.255177in}}{\pgfqpoint{1.578954in}{2.249353in}}%
\pgfpathcurveto{\pgfqpoint{1.573130in}{2.243529in}}{\pgfqpoint{1.569857in}{2.235629in}}{\pgfqpoint{1.569857in}{2.227393in}}%
\pgfpathcurveto{\pgfqpoint{1.569857in}{2.219157in}}{\pgfqpoint{1.573130in}{2.211257in}}{\pgfqpoint{1.578954in}{2.205433in}}%
\pgfpathcurveto{\pgfqpoint{1.584778in}{2.199609in}}{\pgfqpoint{1.592678in}{2.196336in}}{\pgfqpoint{1.600914in}{2.196336in}}%
\pgfpathclose%
\pgfusepath{stroke,fill}%
\end{pgfscope}%
\begin{pgfscope}%
\pgfpathrectangle{\pgfqpoint{0.100000in}{0.212622in}}{\pgfqpoint{3.696000in}{3.696000in}}%
\pgfusepath{clip}%
\pgfsetbuttcap%
\pgfsetroundjoin%
\definecolor{currentfill}{rgb}{0.121569,0.466667,0.705882}%
\pgfsetfillcolor{currentfill}%
\pgfsetfillopacity{0.915374}%
\pgfsetlinewidth{1.003750pt}%
\definecolor{currentstroke}{rgb}{0.121569,0.466667,0.705882}%
\pgfsetstrokecolor{currentstroke}%
\pgfsetstrokeopacity{0.915374}%
\pgfsetdash{}{0pt}%
\pgfpathmoveto{\pgfqpoint{2.596294in}{1.875610in}}%
\pgfpathcurveto{\pgfqpoint{2.604530in}{1.875610in}}{\pgfqpoint{2.612430in}{1.878882in}}{\pgfqpoint{2.618254in}{1.884706in}}%
\pgfpathcurveto{\pgfqpoint{2.624078in}{1.890530in}}{\pgfqpoint{2.627350in}{1.898430in}}{\pgfqpoint{2.627350in}{1.906667in}}%
\pgfpathcurveto{\pgfqpoint{2.627350in}{1.914903in}}{\pgfqpoint{2.624078in}{1.922803in}}{\pgfqpoint{2.618254in}{1.928627in}}%
\pgfpathcurveto{\pgfqpoint{2.612430in}{1.934451in}}{\pgfqpoint{2.604530in}{1.937723in}}{\pgfqpoint{2.596294in}{1.937723in}}%
\pgfpathcurveto{\pgfqpoint{2.588057in}{1.937723in}}{\pgfqpoint{2.580157in}{1.934451in}}{\pgfqpoint{2.574333in}{1.928627in}}%
\pgfpathcurveto{\pgfqpoint{2.568510in}{1.922803in}}{\pgfqpoint{2.565237in}{1.914903in}}{\pgfqpoint{2.565237in}{1.906667in}}%
\pgfpathcurveto{\pgfqpoint{2.565237in}{1.898430in}}{\pgfqpoint{2.568510in}{1.890530in}}{\pgfqpoint{2.574333in}{1.884706in}}%
\pgfpathcurveto{\pgfqpoint{2.580157in}{1.878882in}}{\pgfqpoint{2.588057in}{1.875610in}}{\pgfqpoint{2.596294in}{1.875610in}}%
\pgfpathclose%
\pgfusepath{stroke,fill}%
\end{pgfscope}%
\begin{pgfscope}%
\pgfpathrectangle{\pgfqpoint{0.100000in}{0.212622in}}{\pgfqpoint{3.696000in}{3.696000in}}%
\pgfusepath{clip}%
\pgfsetbuttcap%
\pgfsetroundjoin%
\definecolor{currentfill}{rgb}{0.121569,0.466667,0.705882}%
\pgfsetfillcolor{currentfill}%
\pgfsetfillopacity{0.915429}%
\pgfsetlinewidth{1.003750pt}%
\definecolor{currentstroke}{rgb}{0.121569,0.466667,0.705882}%
\pgfsetstrokecolor{currentstroke}%
\pgfsetstrokeopacity{0.915429}%
\pgfsetdash{}{0pt}%
\pgfpathmoveto{\pgfqpoint{1.617504in}{2.190876in}}%
\pgfpathcurveto{\pgfqpoint{1.625740in}{2.190876in}}{\pgfqpoint{1.633640in}{2.194149in}}{\pgfqpoint{1.639464in}{2.199972in}}%
\pgfpathcurveto{\pgfqpoint{1.645288in}{2.205796in}}{\pgfqpoint{1.648560in}{2.213696in}}{\pgfqpoint{1.648560in}{2.221933in}}%
\pgfpathcurveto{\pgfqpoint{1.648560in}{2.230169in}}{\pgfqpoint{1.645288in}{2.238069in}}{\pgfqpoint{1.639464in}{2.243893in}}%
\pgfpathcurveto{\pgfqpoint{1.633640in}{2.249717in}}{\pgfqpoint{1.625740in}{2.252989in}}{\pgfqpoint{1.617504in}{2.252989in}}%
\pgfpathcurveto{\pgfqpoint{1.609268in}{2.252989in}}{\pgfqpoint{1.601368in}{2.249717in}}{\pgfqpoint{1.595544in}{2.243893in}}%
\pgfpathcurveto{\pgfqpoint{1.589720in}{2.238069in}}{\pgfqpoint{1.586447in}{2.230169in}}{\pgfqpoint{1.586447in}{2.221933in}}%
\pgfpathcurveto{\pgfqpoint{1.586447in}{2.213696in}}{\pgfqpoint{1.589720in}{2.205796in}}{\pgfqpoint{1.595544in}{2.199972in}}%
\pgfpathcurveto{\pgfqpoint{1.601368in}{2.194149in}}{\pgfqpoint{1.609268in}{2.190876in}}{\pgfqpoint{1.617504in}{2.190876in}}%
\pgfpathclose%
\pgfusepath{stroke,fill}%
\end{pgfscope}%
\begin{pgfscope}%
\pgfpathrectangle{\pgfqpoint{0.100000in}{0.212622in}}{\pgfqpoint{3.696000in}{3.696000in}}%
\pgfusepath{clip}%
\pgfsetbuttcap%
\pgfsetroundjoin%
\definecolor{currentfill}{rgb}{0.121569,0.466667,0.705882}%
\pgfsetfillcolor{currentfill}%
\pgfsetfillopacity{0.917237}%
\pgfsetlinewidth{1.003750pt}%
\definecolor{currentstroke}{rgb}{0.121569,0.466667,0.705882}%
\pgfsetstrokecolor{currentstroke}%
\pgfsetstrokeopacity{0.917237}%
\pgfsetdash{}{0pt}%
\pgfpathmoveto{\pgfqpoint{1.631876in}{2.187372in}}%
\pgfpathcurveto{\pgfqpoint{1.640112in}{2.187372in}}{\pgfqpoint{1.648012in}{2.190644in}}{\pgfqpoint{1.653836in}{2.196468in}}%
\pgfpathcurveto{\pgfqpoint{1.659660in}{2.202292in}}{\pgfqpoint{1.662933in}{2.210192in}}{\pgfqpoint{1.662933in}{2.218428in}}%
\pgfpathcurveto{\pgfqpoint{1.662933in}{2.226665in}}{\pgfqpoint{1.659660in}{2.234565in}}{\pgfqpoint{1.653836in}{2.240389in}}%
\pgfpathcurveto{\pgfqpoint{1.648012in}{2.246213in}}{\pgfqpoint{1.640112in}{2.249485in}}{\pgfqpoint{1.631876in}{2.249485in}}%
\pgfpathcurveto{\pgfqpoint{1.623640in}{2.249485in}}{\pgfqpoint{1.615740in}{2.246213in}}{\pgfqpoint{1.609916in}{2.240389in}}%
\pgfpathcurveto{\pgfqpoint{1.604092in}{2.234565in}}{\pgfqpoint{1.600820in}{2.226665in}}{\pgfqpoint{1.600820in}{2.218428in}}%
\pgfpathcurveto{\pgfqpoint{1.600820in}{2.210192in}}{\pgfqpoint{1.604092in}{2.202292in}}{\pgfqpoint{1.609916in}{2.196468in}}%
\pgfpathcurveto{\pgfqpoint{1.615740in}{2.190644in}}{\pgfqpoint{1.623640in}{2.187372in}}{\pgfqpoint{1.631876in}{2.187372in}}%
\pgfpathclose%
\pgfusepath{stroke,fill}%
\end{pgfscope}%
\begin{pgfscope}%
\pgfpathrectangle{\pgfqpoint{0.100000in}{0.212622in}}{\pgfqpoint{3.696000in}{3.696000in}}%
\pgfusepath{clip}%
\pgfsetbuttcap%
\pgfsetroundjoin%
\definecolor{currentfill}{rgb}{0.121569,0.466667,0.705882}%
\pgfsetfillcolor{currentfill}%
\pgfsetfillopacity{0.917477}%
\pgfsetlinewidth{1.003750pt}%
\definecolor{currentstroke}{rgb}{0.121569,0.466667,0.705882}%
\pgfsetstrokecolor{currentstroke}%
\pgfsetstrokeopacity{0.917477}%
\pgfsetdash{}{0pt}%
\pgfpathmoveto{\pgfqpoint{2.591208in}{1.876725in}}%
\pgfpathcurveto{\pgfqpoint{2.599444in}{1.876725in}}{\pgfqpoint{2.607344in}{1.879997in}}{\pgfqpoint{2.613168in}{1.885821in}}%
\pgfpathcurveto{\pgfqpoint{2.618992in}{1.891645in}}{\pgfqpoint{2.622265in}{1.899545in}}{\pgfqpoint{2.622265in}{1.907781in}}%
\pgfpathcurveto{\pgfqpoint{2.622265in}{1.916018in}}{\pgfqpoint{2.618992in}{1.923918in}}{\pgfqpoint{2.613168in}{1.929742in}}%
\pgfpathcurveto{\pgfqpoint{2.607344in}{1.935566in}}{\pgfqpoint{2.599444in}{1.938838in}}{\pgfqpoint{2.591208in}{1.938838in}}%
\pgfpathcurveto{\pgfqpoint{2.582972in}{1.938838in}}{\pgfqpoint{2.575072in}{1.935566in}}{\pgfqpoint{2.569248in}{1.929742in}}%
\pgfpathcurveto{\pgfqpoint{2.563424in}{1.923918in}}{\pgfqpoint{2.560152in}{1.916018in}}{\pgfqpoint{2.560152in}{1.907781in}}%
\pgfpathcurveto{\pgfqpoint{2.560152in}{1.899545in}}{\pgfqpoint{2.563424in}{1.891645in}}{\pgfqpoint{2.569248in}{1.885821in}}%
\pgfpathcurveto{\pgfqpoint{2.575072in}{1.879997in}}{\pgfqpoint{2.582972in}{1.876725in}}{\pgfqpoint{2.591208in}{1.876725in}}%
\pgfpathclose%
\pgfusepath{stroke,fill}%
\end{pgfscope}%
\begin{pgfscope}%
\pgfpathrectangle{\pgfqpoint{0.100000in}{0.212622in}}{\pgfqpoint{3.696000in}{3.696000in}}%
\pgfusepath{clip}%
\pgfsetbuttcap%
\pgfsetroundjoin%
\definecolor{currentfill}{rgb}{0.121569,0.466667,0.705882}%
\pgfsetfillcolor{currentfill}%
\pgfsetfillopacity{0.918598}%
\pgfsetlinewidth{1.003750pt}%
\definecolor{currentstroke}{rgb}{0.121569,0.466667,0.705882}%
\pgfsetstrokecolor{currentstroke}%
\pgfsetstrokeopacity{0.918598}%
\pgfsetdash{}{0pt}%
\pgfpathmoveto{\pgfqpoint{1.643810in}{2.181380in}}%
\pgfpathcurveto{\pgfqpoint{1.652046in}{2.181380in}}{\pgfqpoint{1.659946in}{2.184652in}}{\pgfqpoint{1.665770in}{2.190476in}}%
\pgfpathcurveto{\pgfqpoint{1.671594in}{2.196300in}}{\pgfqpoint{1.674866in}{2.204200in}}{\pgfqpoint{1.674866in}{2.212436in}}%
\pgfpathcurveto{\pgfqpoint{1.674866in}{2.220672in}}{\pgfqpoint{1.671594in}{2.228572in}}{\pgfqpoint{1.665770in}{2.234396in}}%
\pgfpathcurveto{\pgfqpoint{1.659946in}{2.240220in}}{\pgfqpoint{1.652046in}{2.243493in}}{\pgfqpoint{1.643810in}{2.243493in}}%
\pgfpathcurveto{\pgfqpoint{1.635573in}{2.243493in}}{\pgfqpoint{1.627673in}{2.240220in}}{\pgfqpoint{1.621849in}{2.234396in}}%
\pgfpathcurveto{\pgfqpoint{1.616025in}{2.228572in}}{\pgfqpoint{1.612753in}{2.220672in}}{\pgfqpoint{1.612753in}{2.212436in}}%
\pgfpathcurveto{\pgfqpoint{1.612753in}{2.204200in}}{\pgfqpoint{1.616025in}{2.196300in}}{\pgfqpoint{1.621849in}{2.190476in}}%
\pgfpathcurveto{\pgfqpoint{1.627673in}{2.184652in}}{\pgfqpoint{1.635573in}{2.181380in}}{\pgfqpoint{1.643810in}{2.181380in}}%
\pgfpathclose%
\pgfusepath{stroke,fill}%
\end{pgfscope}%
\begin{pgfscope}%
\pgfpathrectangle{\pgfqpoint{0.100000in}{0.212622in}}{\pgfqpoint{3.696000in}{3.696000in}}%
\pgfusepath{clip}%
\pgfsetbuttcap%
\pgfsetroundjoin%
\definecolor{currentfill}{rgb}{0.121569,0.466667,0.705882}%
\pgfsetfillcolor{currentfill}%
\pgfsetfillopacity{0.919826}%
\pgfsetlinewidth{1.003750pt}%
\definecolor{currentstroke}{rgb}{0.121569,0.466667,0.705882}%
\pgfsetstrokecolor{currentstroke}%
\pgfsetstrokeopacity{0.919826}%
\pgfsetdash{}{0pt}%
\pgfpathmoveto{\pgfqpoint{1.655474in}{2.177232in}}%
\pgfpathcurveto{\pgfqpoint{1.663710in}{2.177232in}}{\pgfqpoint{1.671610in}{2.180504in}}{\pgfqpoint{1.677434in}{2.186328in}}%
\pgfpathcurveto{\pgfqpoint{1.683258in}{2.192152in}}{\pgfqpoint{1.686530in}{2.200052in}}{\pgfqpoint{1.686530in}{2.208289in}}%
\pgfpathcurveto{\pgfqpoint{1.686530in}{2.216525in}}{\pgfqpoint{1.683258in}{2.224425in}}{\pgfqpoint{1.677434in}{2.230249in}}%
\pgfpathcurveto{\pgfqpoint{1.671610in}{2.236073in}}{\pgfqpoint{1.663710in}{2.239345in}}{\pgfqpoint{1.655474in}{2.239345in}}%
\pgfpathcurveto{\pgfqpoint{1.647238in}{2.239345in}}{\pgfqpoint{1.639338in}{2.236073in}}{\pgfqpoint{1.633514in}{2.230249in}}%
\pgfpathcurveto{\pgfqpoint{1.627690in}{2.224425in}}{\pgfqpoint{1.624417in}{2.216525in}}{\pgfqpoint{1.624417in}{2.208289in}}%
\pgfpathcurveto{\pgfqpoint{1.624417in}{2.200052in}}{\pgfqpoint{1.627690in}{2.192152in}}{\pgfqpoint{1.633514in}{2.186328in}}%
\pgfpathcurveto{\pgfqpoint{1.639338in}{2.180504in}}{\pgfqpoint{1.647238in}{2.177232in}}{\pgfqpoint{1.655474in}{2.177232in}}%
\pgfpathclose%
\pgfusepath{stroke,fill}%
\end{pgfscope}%
\begin{pgfscope}%
\pgfpathrectangle{\pgfqpoint{0.100000in}{0.212622in}}{\pgfqpoint{3.696000in}{3.696000in}}%
\pgfusepath{clip}%
\pgfsetbuttcap%
\pgfsetroundjoin%
\definecolor{currentfill}{rgb}{0.121569,0.466667,0.705882}%
\pgfsetfillcolor{currentfill}%
\pgfsetfillopacity{0.920170}%
\pgfsetlinewidth{1.003750pt}%
\definecolor{currentstroke}{rgb}{0.121569,0.466667,0.705882}%
\pgfsetstrokecolor{currentstroke}%
\pgfsetstrokeopacity{0.920170}%
\pgfsetdash{}{0pt}%
\pgfpathmoveto{\pgfqpoint{2.587840in}{1.877084in}}%
\pgfpathcurveto{\pgfqpoint{2.596076in}{1.877084in}}{\pgfqpoint{2.603976in}{1.880356in}}{\pgfqpoint{2.609800in}{1.886180in}}%
\pgfpathcurveto{\pgfqpoint{2.615624in}{1.892004in}}{\pgfqpoint{2.618896in}{1.899904in}}{\pgfqpoint{2.618896in}{1.908141in}}%
\pgfpathcurveto{\pgfqpoint{2.618896in}{1.916377in}}{\pgfqpoint{2.615624in}{1.924277in}}{\pgfqpoint{2.609800in}{1.930101in}}%
\pgfpathcurveto{\pgfqpoint{2.603976in}{1.935925in}}{\pgfqpoint{2.596076in}{1.939197in}}{\pgfqpoint{2.587840in}{1.939197in}}%
\pgfpathcurveto{\pgfqpoint{2.579603in}{1.939197in}}{\pgfqpoint{2.571703in}{1.935925in}}{\pgfqpoint{2.565879in}{1.930101in}}%
\pgfpathcurveto{\pgfqpoint{2.560055in}{1.924277in}}{\pgfqpoint{2.556783in}{1.916377in}}{\pgfqpoint{2.556783in}{1.908141in}}%
\pgfpathcurveto{\pgfqpoint{2.556783in}{1.899904in}}{\pgfqpoint{2.560055in}{1.892004in}}{\pgfqpoint{2.565879in}{1.886180in}}%
\pgfpathcurveto{\pgfqpoint{2.571703in}{1.880356in}}{\pgfqpoint{2.579603in}{1.877084in}}{\pgfqpoint{2.587840in}{1.877084in}}%
\pgfpathclose%
\pgfusepath{stroke,fill}%
\end{pgfscope}%
\begin{pgfscope}%
\pgfpathrectangle{\pgfqpoint{0.100000in}{0.212622in}}{\pgfqpoint{3.696000in}{3.696000in}}%
\pgfusepath{clip}%
\pgfsetbuttcap%
\pgfsetroundjoin%
\definecolor{currentfill}{rgb}{0.121569,0.466667,0.705882}%
\pgfsetfillcolor{currentfill}%
\pgfsetfillopacity{0.920665}%
\pgfsetlinewidth{1.003750pt}%
\definecolor{currentstroke}{rgb}{0.121569,0.466667,0.705882}%
\pgfsetstrokecolor{currentstroke}%
\pgfsetstrokeopacity{0.920665}%
\pgfsetdash{}{0pt}%
\pgfpathmoveto{\pgfqpoint{1.667197in}{2.172952in}}%
\pgfpathcurveto{\pgfqpoint{1.675433in}{2.172952in}}{\pgfqpoint{1.683333in}{2.176224in}}{\pgfqpoint{1.689157in}{2.182048in}}%
\pgfpathcurveto{\pgfqpoint{1.694981in}{2.187872in}}{\pgfqpoint{1.698253in}{2.195772in}}{\pgfqpoint{1.698253in}{2.204008in}}%
\pgfpathcurveto{\pgfqpoint{1.698253in}{2.212245in}}{\pgfqpoint{1.694981in}{2.220145in}}{\pgfqpoint{1.689157in}{2.225969in}}%
\pgfpathcurveto{\pgfqpoint{1.683333in}{2.231793in}}{\pgfqpoint{1.675433in}{2.235065in}}{\pgfqpoint{1.667197in}{2.235065in}}%
\pgfpathcurveto{\pgfqpoint{1.658961in}{2.235065in}}{\pgfqpoint{1.651061in}{2.231793in}}{\pgfqpoint{1.645237in}{2.225969in}}%
\pgfpathcurveto{\pgfqpoint{1.639413in}{2.220145in}}{\pgfqpoint{1.636140in}{2.212245in}}{\pgfqpoint{1.636140in}{2.204008in}}%
\pgfpathcurveto{\pgfqpoint{1.636140in}{2.195772in}}{\pgfqpoint{1.639413in}{2.187872in}}{\pgfqpoint{1.645237in}{2.182048in}}%
\pgfpathcurveto{\pgfqpoint{1.651061in}{2.176224in}}{\pgfqpoint{1.658961in}{2.172952in}}{\pgfqpoint{1.667197in}{2.172952in}}%
\pgfpathclose%
\pgfusepath{stroke,fill}%
\end{pgfscope}%
\begin{pgfscope}%
\pgfpathrectangle{\pgfqpoint{0.100000in}{0.212622in}}{\pgfqpoint{3.696000in}{3.696000in}}%
\pgfusepath{clip}%
\pgfsetbuttcap%
\pgfsetroundjoin%
\definecolor{currentfill}{rgb}{0.121569,0.466667,0.705882}%
\pgfsetfillcolor{currentfill}%
\pgfsetfillopacity{0.921884}%
\pgfsetlinewidth{1.003750pt}%
\definecolor{currentstroke}{rgb}{0.121569,0.466667,0.705882}%
\pgfsetstrokecolor{currentstroke}%
\pgfsetstrokeopacity{0.921884}%
\pgfsetdash{}{0pt}%
\pgfpathmoveto{\pgfqpoint{1.674131in}{2.167885in}}%
\pgfpathcurveto{\pgfqpoint{1.682367in}{2.167885in}}{\pgfqpoint{1.690267in}{2.171157in}}{\pgfqpoint{1.696091in}{2.176981in}}%
\pgfpathcurveto{\pgfqpoint{1.701915in}{2.182805in}}{\pgfqpoint{1.705187in}{2.190705in}}{\pgfqpoint{1.705187in}{2.198942in}}%
\pgfpathcurveto{\pgfqpoint{1.705187in}{2.207178in}}{\pgfqpoint{1.701915in}{2.215078in}}{\pgfqpoint{1.696091in}{2.220902in}}%
\pgfpathcurveto{\pgfqpoint{1.690267in}{2.226726in}}{\pgfqpoint{1.682367in}{2.229998in}}{\pgfqpoint{1.674131in}{2.229998in}}%
\pgfpathcurveto{\pgfqpoint{1.665894in}{2.229998in}}{\pgfqpoint{1.657994in}{2.226726in}}{\pgfqpoint{1.652170in}{2.220902in}}%
\pgfpathcurveto{\pgfqpoint{1.646346in}{2.215078in}}{\pgfqpoint{1.643074in}{2.207178in}}{\pgfqpoint{1.643074in}{2.198942in}}%
\pgfpathcurveto{\pgfqpoint{1.643074in}{2.190705in}}{\pgfqpoint{1.646346in}{2.182805in}}{\pgfqpoint{1.652170in}{2.176981in}}%
\pgfpathcurveto{\pgfqpoint{1.657994in}{2.171157in}}{\pgfqpoint{1.665894in}{2.167885in}}{\pgfqpoint{1.674131in}{2.167885in}}%
\pgfpathclose%
\pgfusepath{stroke,fill}%
\end{pgfscope}%
\begin{pgfscope}%
\pgfpathrectangle{\pgfqpoint{0.100000in}{0.212622in}}{\pgfqpoint{3.696000in}{3.696000in}}%
\pgfusepath{clip}%
\pgfsetbuttcap%
\pgfsetroundjoin%
\definecolor{currentfill}{rgb}{0.121569,0.466667,0.705882}%
\pgfsetfillcolor{currentfill}%
\pgfsetfillopacity{0.922510}%
\pgfsetlinewidth{1.003750pt}%
\definecolor{currentstroke}{rgb}{0.121569,0.466667,0.705882}%
\pgfsetstrokecolor{currentstroke}%
\pgfsetstrokeopacity{0.922510}%
\pgfsetdash{}{0pt}%
\pgfpathmoveto{\pgfqpoint{1.680513in}{2.166021in}}%
\pgfpathcurveto{\pgfqpoint{1.688750in}{2.166021in}}{\pgfqpoint{1.696650in}{2.169294in}}{\pgfqpoint{1.702474in}{2.175118in}}%
\pgfpathcurveto{\pgfqpoint{1.708297in}{2.180942in}}{\pgfqpoint{1.711570in}{2.188842in}}{\pgfqpoint{1.711570in}{2.197078in}}%
\pgfpathcurveto{\pgfqpoint{1.711570in}{2.205314in}}{\pgfqpoint{1.708297in}{2.213214in}}{\pgfqpoint{1.702474in}{2.219038in}}%
\pgfpathcurveto{\pgfqpoint{1.696650in}{2.224862in}}{\pgfqpoint{1.688750in}{2.228134in}}{\pgfqpoint{1.680513in}{2.228134in}}%
\pgfpathcurveto{\pgfqpoint{1.672277in}{2.228134in}}{\pgfqpoint{1.664377in}{2.224862in}}{\pgfqpoint{1.658553in}{2.219038in}}%
\pgfpathcurveto{\pgfqpoint{1.652729in}{2.213214in}}{\pgfqpoint{1.649457in}{2.205314in}}{\pgfqpoint{1.649457in}{2.197078in}}%
\pgfpathcurveto{\pgfqpoint{1.649457in}{2.188842in}}{\pgfqpoint{1.652729in}{2.180942in}}{\pgfqpoint{1.658553in}{2.175118in}}%
\pgfpathcurveto{\pgfqpoint{1.664377in}{2.169294in}}{\pgfqpoint{1.672277in}{2.166021in}}{\pgfqpoint{1.680513in}{2.166021in}}%
\pgfpathclose%
\pgfusepath{stroke,fill}%
\end{pgfscope}%
\begin{pgfscope}%
\pgfpathrectangle{\pgfqpoint{0.100000in}{0.212622in}}{\pgfqpoint{3.696000in}{3.696000in}}%
\pgfusepath{clip}%
\pgfsetbuttcap%
\pgfsetroundjoin%
\definecolor{currentfill}{rgb}{0.121569,0.466667,0.705882}%
\pgfsetfillcolor{currentfill}%
\pgfsetfillopacity{0.922855}%
\pgfsetlinewidth{1.003750pt}%
\definecolor{currentstroke}{rgb}{0.121569,0.466667,0.705882}%
\pgfsetstrokecolor{currentstroke}%
\pgfsetstrokeopacity{0.922855}%
\pgfsetdash{}{0pt}%
\pgfpathmoveto{\pgfqpoint{2.582006in}{1.878150in}}%
\pgfpathcurveto{\pgfqpoint{2.590243in}{1.878150in}}{\pgfqpoint{2.598143in}{1.881422in}}{\pgfqpoint{2.603967in}{1.887246in}}%
\pgfpathcurveto{\pgfqpoint{2.609791in}{1.893070in}}{\pgfqpoint{2.613063in}{1.900970in}}{\pgfqpoint{2.613063in}{1.909206in}}%
\pgfpathcurveto{\pgfqpoint{2.613063in}{1.917442in}}{\pgfqpoint{2.609791in}{1.925342in}}{\pgfqpoint{2.603967in}{1.931166in}}%
\pgfpathcurveto{\pgfqpoint{2.598143in}{1.936990in}}{\pgfqpoint{2.590243in}{1.940263in}}{\pgfqpoint{2.582006in}{1.940263in}}%
\pgfpathcurveto{\pgfqpoint{2.573770in}{1.940263in}}{\pgfqpoint{2.565870in}{1.936990in}}{\pgfqpoint{2.560046in}{1.931166in}}%
\pgfpathcurveto{\pgfqpoint{2.554222in}{1.925342in}}{\pgfqpoint{2.550950in}{1.917442in}}{\pgfqpoint{2.550950in}{1.909206in}}%
\pgfpathcurveto{\pgfqpoint{2.550950in}{1.900970in}}{\pgfqpoint{2.554222in}{1.893070in}}{\pgfqpoint{2.560046in}{1.887246in}}%
\pgfpathcurveto{\pgfqpoint{2.565870in}{1.881422in}}{\pgfqpoint{2.573770in}{1.878150in}}{\pgfqpoint{2.582006in}{1.878150in}}%
\pgfpathclose%
\pgfusepath{stroke,fill}%
\end{pgfscope}%
\begin{pgfscope}%
\pgfpathrectangle{\pgfqpoint{0.100000in}{0.212622in}}{\pgfqpoint{3.696000in}{3.696000in}}%
\pgfusepath{clip}%
\pgfsetbuttcap%
\pgfsetroundjoin%
\definecolor{currentfill}{rgb}{0.121569,0.466667,0.705882}%
\pgfsetfillcolor{currentfill}%
\pgfsetfillopacity{0.923603}%
\pgfsetlinewidth{1.003750pt}%
\definecolor{currentstroke}{rgb}{0.121569,0.466667,0.705882}%
\pgfsetstrokecolor{currentstroke}%
\pgfsetstrokeopacity{0.923603}%
\pgfsetdash{}{0pt}%
\pgfpathmoveto{\pgfqpoint{1.692054in}{2.162027in}}%
\pgfpathcurveto{\pgfqpoint{1.700290in}{2.162027in}}{\pgfqpoint{1.708190in}{2.165300in}}{\pgfqpoint{1.714014in}{2.171123in}}%
\pgfpathcurveto{\pgfqpoint{1.719838in}{2.176947in}}{\pgfqpoint{1.723110in}{2.184847in}}{\pgfqpoint{1.723110in}{2.193084in}}%
\pgfpathcurveto{\pgfqpoint{1.723110in}{2.201320in}}{\pgfqpoint{1.719838in}{2.209220in}}{\pgfqpoint{1.714014in}{2.215044in}}%
\pgfpathcurveto{\pgfqpoint{1.708190in}{2.220868in}}{\pgfqpoint{1.700290in}{2.224140in}}{\pgfqpoint{1.692054in}{2.224140in}}%
\pgfpathcurveto{\pgfqpoint{1.683818in}{2.224140in}}{\pgfqpoint{1.675918in}{2.220868in}}{\pgfqpoint{1.670094in}{2.215044in}}%
\pgfpathcurveto{\pgfqpoint{1.664270in}{2.209220in}}{\pgfqpoint{1.660997in}{2.201320in}}{\pgfqpoint{1.660997in}{2.193084in}}%
\pgfpathcurveto{\pgfqpoint{1.660997in}{2.184847in}}{\pgfqpoint{1.664270in}{2.176947in}}{\pgfqpoint{1.670094in}{2.171123in}}%
\pgfpathcurveto{\pgfqpoint{1.675918in}{2.165300in}}{\pgfqpoint{1.683818in}{2.162027in}}{\pgfqpoint{1.692054in}{2.162027in}}%
\pgfpathclose%
\pgfusepath{stroke,fill}%
\end{pgfscope}%
\begin{pgfscope}%
\pgfpathrectangle{\pgfqpoint{0.100000in}{0.212622in}}{\pgfqpoint{3.696000in}{3.696000in}}%
\pgfusepath{clip}%
\pgfsetbuttcap%
\pgfsetroundjoin%
\definecolor{currentfill}{rgb}{0.121569,0.466667,0.705882}%
\pgfsetfillcolor{currentfill}%
\pgfsetfillopacity{0.924259}%
\pgfsetlinewidth{1.003750pt}%
\definecolor{currentstroke}{rgb}{0.121569,0.466667,0.705882}%
\pgfsetstrokecolor{currentstroke}%
\pgfsetstrokeopacity{0.924259}%
\pgfsetdash{}{0pt}%
\pgfpathmoveto{\pgfqpoint{2.578204in}{1.879157in}}%
\pgfpathcurveto{\pgfqpoint{2.586440in}{1.879157in}}{\pgfqpoint{2.594340in}{1.882429in}}{\pgfqpoint{2.600164in}{1.888253in}}%
\pgfpathcurveto{\pgfqpoint{2.605988in}{1.894077in}}{\pgfqpoint{2.609260in}{1.901977in}}{\pgfqpoint{2.609260in}{1.910213in}}%
\pgfpathcurveto{\pgfqpoint{2.609260in}{1.918449in}}{\pgfqpoint{2.605988in}{1.926349in}}{\pgfqpoint{2.600164in}{1.932173in}}%
\pgfpathcurveto{\pgfqpoint{2.594340in}{1.937997in}}{\pgfqpoint{2.586440in}{1.941270in}}{\pgfqpoint{2.578204in}{1.941270in}}%
\pgfpathcurveto{\pgfqpoint{2.569968in}{1.941270in}}{\pgfqpoint{2.562068in}{1.937997in}}{\pgfqpoint{2.556244in}{1.932173in}}%
\pgfpathcurveto{\pgfqpoint{2.550420in}{1.926349in}}{\pgfqpoint{2.547147in}{1.918449in}}{\pgfqpoint{2.547147in}{1.910213in}}%
\pgfpathcurveto{\pgfqpoint{2.547147in}{1.901977in}}{\pgfqpoint{2.550420in}{1.894077in}}{\pgfqpoint{2.556244in}{1.888253in}}%
\pgfpathcurveto{\pgfqpoint{2.562068in}{1.882429in}}{\pgfqpoint{2.569968in}{1.879157in}}{\pgfqpoint{2.578204in}{1.879157in}}%
\pgfpathclose%
\pgfusepath{stroke,fill}%
\end{pgfscope}%
\begin{pgfscope}%
\pgfpathrectangle{\pgfqpoint{0.100000in}{0.212622in}}{\pgfqpoint{3.696000in}{3.696000in}}%
\pgfusepath{clip}%
\pgfsetbuttcap%
\pgfsetroundjoin%
\definecolor{currentfill}{rgb}{0.121569,0.466667,0.705882}%
\pgfsetfillcolor{currentfill}%
\pgfsetfillopacity{0.924652}%
\pgfsetlinewidth{1.003750pt}%
\definecolor{currentstroke}{rgb}{0.121569,0.466667,0.705882}%
\pgfsetstrokecolor{currentstroke}%
\pgfsetstrokeopacity{0.924652}%
\pgfsetdash{}{0pt}%
\pgfpathmoveto{\pgfqpoint{1.701268in}{2.156433in}}%
\pgfpathcurveto{\pgfqpoint{1.709504in}{2.156433in}}{\pgfqpoint{1.717404in}{2.159705in}}{\pgfqpoint{1.723228in}{2.165529in}}%
\pgfpathcurveto{\pgfqpoint{1.729052in}{2.171353in}}{\pgfqpoint{1.732324in}{2.179253in}}{\pgfqpoint{1.732324in}{2.187489in}}%
\pgfpathcurveto{\pgfqpoint{1.732324in}{2.195726in}}{\pgfqpoint{1.729052in}{2.203626in}}{\pgfqpoint{1.723228in}{2.209450in}}%
\pgfpathcurveto{\pgfqpoint{1.717404in}{2.215274in}}{\pgfqpoint{1.709504in}{2.218546in}}{\pgfqpoint{1.701268in}{2.218546in}}%
\pgfpathcurveto{\pgfqpoint{1.693031in}{2.218546in}}{\pgfqpoint{1.685131in}{2.215274in}}{\pgfqpoint{1.679307in}{2.209450in}}%
\pgfpathcurveto{\pgfqpoint{1.673483in}{2.203626in}}{\pgfqpoint{1.670211in}{2.195726in}}{\pgfqpoint{1.670211in}{2.187489in}}%
\pgfpathcurveto{\pgfqpoint{1.670211in}{2.179253in}}{\pgfqpoint{1.673483in}{2.171353in}}{\pgfqpoint{1.679307in}{2.165529in}}%
\pgfpathcurveto{\pgfqpoint{1.685131in}{2.159705in}}{\pgfqpoint{1.693031in}{2.156433in}}{\pgfqpoint{1.701268in}{2.156433in}}%
\pgfpathclose%
\pgfusepath{stroke,fill}%
\end{pgfscope}%
\begin{pgfscope}%
\pgfpathrectangle{\pgfqpoint{0.100000in}{0.212622in}}{\pgfqpoint{3.696000in}{3.696000in}}%
\pgfusepath{clip}%
\pgfsetbuttcap%
\pgfsetroundjoin%
\definecolor{currentfill}{rgb}{0.121569,0.466667,0.705882}%
\pgfsetfillcolor{currentfill}%
\pgfsetfillopacity{0.925402}%
\pgfsetlinewidth{1.003750pt}%
\definecolor{currentstroke}{rgb}{0.121569,0.466667,0.705882}%
\pgfsetstrokecolor{currentstroke}%
\pgfsetstrokeopacity{0.925402}%
\pgfsetdash{}{0pt}%
\pgfpathmoveto{\pgfqpoint{1.708336in}{2.154412in}}%
\pgfpathcurveto{\pgfqpoint{1.716572in}{2.154412in}}{\pgfqpoint{1.724472in}{2.157684in}}{\pgfqpoint{1.730296in}{2.163508in}}%
\pgfpathcurveto{\pgfqpoint{1.736120in}{2.169332in}}{\pgfqpoint{1.739392in}{2.177232in}}{\pgfqpoint{1.739392in}{2.185468in}}%
\pgfpathcurveto{\pgfqpoint{1.739392in}{2.193704in}}{\pgfqpoint{1.736120in}{2.201604in}}{\pgfqpoint{1.730296in}{2.207428in}}%
\pgfpathcurveto{\pgfqpoint{1.724472in}{2.213252in}}{\pgfqpoint{1.716572in}{2.216525in}}{\pgfqpoint{1.708336in}{2.216525in}}%
\pgfpathcurveto{\pgfqpoint{1.700100in}{2.216525in}}{\pgfqpoint{1.692200in}{2.213252in}}{\pgfqpoint{1.686376in}{2.207428in}}%
\pgfpathcurveto{\pgfqpoint{1.680552in}{2.201604in}}{\pgfqpoint{1.677279in}{2.193704in}}{\pgfqpoint{1.677279in}{2.185468in}}%
\pgfpathcurveto{\pgfqpoint{1.677279in}{2.177232in}}{\pgfqpoint{1.680552in}{2.169332in}}{\pgfqpoint{1.686376in}{2.163508in}}%
\pgfpathcurveto{\pgfqpoint{1.692200in}{2.157684in}}{\pgfqpoint{1.700100in}{2.154412in}}{\pgfqpoint{1.708336in}{2.154412in}}%
\pgfpathclose%
\pgfusepath{stroke,fill}%
\end{pgfscope}%
\begin{pgfscope}%
\pgfpathrectangle{\pgfqpoint{0.100000in}{0.212622in}}{\pgfqpoint{3.696000in}{3.696000in}}%
\pgfusepath{clip}%
\pgfsetbuttcap%
\pgfsetroundjoin%
\definecolor{currentfill}{rgb}{0.121569,0.466667,0.705882}%
\pgfsetfillcolor{currentfill}%
\pgfsetfillopacity{0.925970}%
\pgfsetlinewidth{1.003750pt}%
\definecolor{currentstroke}{rgb}{0.121569,0.466667,0.705882}%
\pgfsetstrokecolor{currentstroke}%
\pgfsetstrokeopacity{0.925970}%
\pgfsetdash{}{0pt}%
\pgfpathmoveto{\pgfqpoint{1.713796in}{2.152445in}}%
\pgfpathcurveto{\pgfqpoint{1.722032in}{2.152445in}}{\pgfqpoint{1.729932in}{2.155718in}}{\pgfqpoint{1.735756in}{2.161542in}}%
\pgfpathcurveto{\pgfqpoint{1.741580in}{2.167365in}}{\pgfqpoint{1.744852in}{2.175266in}}{\pgfqpoint{1.744852in}{2.183502in}}%
\pgfpathcurveto{\pgfqpoint{1.744852in}{2.191738in}}{\pgfqpoint{1.741580in}{2.199638in}}{\pgfqpoint{1.735756in}{2.205462in}}%
\pgfpathcurveto{\pgfqpoint{1.729932in}{2.211286in}}{\pgfqpoint{1.722032in}{2.214558in}}{\pgfqpoint{1.713796in}{2.214558in}}%
\pgfpathcurveto{\pgfqpoint{1.705559in}{2.214558in}}{\pgfqpoint{1.697659in}{2.211286in}}{\pgfqpoint{1.691835in}{2.205462in}}%
\pgfpathcurveto{\pgfqpoint{1.686011in}{2.199638in}}{\pgfqpoint{1.682739in}{2.191738in}}{\pgfqpoint{1.682739in}{2.183502in}}%
\pgfpathcurveto{\pgfqpoint{1.682739in}{2.175266in}}{\pgfqpoint{1.686011in}{2.167365in}}{\pgfqpoint{1.691835in}{2.161542in}}%
\pgfpathcurveto{\pgfqpoint{1.697659in}{2.155718in}}{\pgfqpoint{1.705559in}{2.152445in}}{\pgfqpoint{1.713796in}{2.152445in}}%
\pgfpathclose%
\pgfusepath{stroke,fill}%
\end{pgfscope}%
\begin{pgfscope}%
\pgfpathrectangle{\pgfqpoint{0.100000in}{0.212622in}}{\pgfqpoint{3.696000in}{3.696000in}}%
\pgfusepath{clip}%
\pgfsetbuttcap%
\pgfsetroundjoin%
\definecolor{currentfill}{rgb}{0.121569,0.466667,0.705882}%
\pgfsetfillcolor{currentfill}%
\pgfsetfillopacity{0.926638}%
\pgfsetlinewidth{1.003750pt}%
\definecolor{currentstroke}{rgb}{0.121569,0.466667,0.705882}%
\pgfsetstrokecolor{currentstroke}%
\pgfsetstrokeopacity{0.926638}%
\pgfsetdash{}{0pt}%
\pgfpathmoveto{\pgfqpoint{2.574696in}{1.879568in}}%
\pgfpathcurveto{\pgfqpoint{2.582933in}{1.879568in}}{\pgfqpoint{2.590833in}{1.882841in}}{\pgfqpoint{2.596657in}{1.888664in}}%
\pgfpathcurveto{\pgfqpoint{2.602481in}{1.894488in}}{\pgfqpoint{2.605753in}{1.902388in}}{\pgfqpoint{2.605753in}{1.910625in}}%
\pgfpathcurveto{\pgfqpoint{2.605753in}{1.918861in}}{\pgfqpoint{2.602481in}{1.926761in}}{\pgfqpoint{2.596657in}{1.932585in}}%
\pgfpathcurveto{\pgfqpoint{2.590833in}{1.938409in}}{\pgfqpoint{2.582933in}{1.941681in}}{\pgfqpoint{2.574696in}{1.941681in}}%
\pgfpathcurveto{\pgfqpoint{2.566460in}{1.941681in}}{\pgfqpoint{2.558560in}{1.938409in}}{\pgfqpoint{2.552736in}{1.932585in}}%
\pgfpathcurveto{\pgfqpoint{2.546912in}{1.926761in}}{\pgfqpoint{2.543640in}{1.918861in}}{\pgfqpoint{2.543640in}{1.910625in}}%
\pgfpathcurveto{\pgfqpoint{2.543640in}{1.902388in}}{\pgfqpoint{2.546912in}{1.894488in}}{\pgfqpoint{2.552736in}{1.888664in}}%
\pgfpathcurveto{\pgfqpoint{2.558560in}{1.882841in}}{\pgfqpoint{2.566460in}{1.879568in}}{\pgfqpoint{2.574696in}{1.879568in}}%
\pgfpathclose%
\pgfusepath{stroke,fill}%
\end{pgfscope}%
\begin{pgfscope}%
\pgfpathrectangle{\pgfqpoint{0.100000in}{0.212622in}}{\pgfqpoint{3.696000in}{3.696000in}}%
\pgfusepath{clip}%
\pgfsetbuttcap%
\pgfsetroundjoin%
\definecolor{currentfill}{rgb}{0.121569,0.466667,0.705882}%
\pgfsetfillcolor{currentfill}%
\pgfsetfillopacity{0.926919}%
\pgfsetlinewidth{1.003750pt}%
\definecolor{currentstroke}{rgb}{0.121569,0.466667,0.705882}%
\pgfsetstrokecolor{currentstroke}%
\pgfsetstrokeopacity{0.926919}%
\pgfsetdash{}{0pt}%
\pgfpathmoveto{\pgfqpoint{1.723746in}{2.148345in}}%
\pgfpathcurveto{\pgfqpoint{1.731982in}{2.148345in}}{\pgfqpoint{1.739882in}{2.151617in}}{\pgfqpoint{1.745706in}{2.157441in}}%
\pgfpathcurveto{\pgfqpoint{1.751530in}{2.163265in}}{\pgfqpoint{1.754803in}{2.171165in}}{\pgfqpoint{1.754803in}{2.179401in}}%
\pgfpathcurveto{\pgfqpoint{1.754803in}{2.187638in}}{\pgfqpoint{1.751530in}{2.195538in}}{\pgfqpoint{1.745706in}{2.201362in}}%
\pgfpathcurveto{\pgfqpoint{1.739882in}{2.207185in}}{\pgfqpoint{1.731982in}{2.210458in}}{\pgfqpoint{1.723746in}{2.210458in}}%
\pgfpathcurveto{\pgfqpoint{1.715510in}{2.210458in}}{\pgfqpoint{1.707610in}{2.207185in}}{\pgfqpoint{1.701786in}{2.201362in}}%
\pgfpathcurveto{\pgfqpoint{1.695962in}{2.195538in}}{\pgfqpoint{1.692690in}{2.187638in}}{\pgfqpoint{1.692690in}{2.179401in}}%
\pgfpathcurveto{\pgfqpoint{1.692690in}{2.171165in}}{\pgfqpoint{1.695962in}{2.163265in}}{\pgfqpoint{1.701786in}{2.157441in}}%
\pgfpathcurveto{\pgfqpoint{1.707610in}{2.151617in}}{\pgfqpoint{1.715510in}{2.148345in}}{\pgfqpoint{1.723746in}{2.148345in}}%
\pgfpathclose%
\pgfusepath{stroke,fill}%
\end{pgfscope}%
\begin{pgfscope}%
\pgfpathrectangle{\pgfqpoint{0.100000in}{0.212622in}}{\pgfqpoint{3.696000in}{3.696000in}}%
\pgfusepath{clip}%
\pgfsetbuttcap%
\pgfsetroundjoin%
\definecolor{currentfill}{rgb}{0.121569,0.466667,0.705882}%
\pgfsetfillcolor{currentfill}%
\pgfsetfillopacity{0.927988}%
\pgfsetlinewidth{1.003750pt}%
\definecolor{currentstroke}{rgb}{0.121569,0.466667,0.705882}%
\pgfsetstrokecolor{currentstroke}%
\pgfsetstrokeopacity{0.927988}%
\pgfsetdash{}{0pt}%
\pgfpathmoveto{\pgfqpoint{1.732479in}{2.146005in}}%
\pgfpathcurveto{\pgfqpoint{1.740715in}{2.146005in}}{\pgfqpoint{1.748615in}{2.149277in}}{\pgfqpoint{1.754439in}{2.155101in}}%
\pgfpathcurveto{\pgfqpoint{1.760263in}{2.160925in}}{\pgfqpoint{1.763535in}{2.168825in}}{\pgfqpoint{1.763535in}{2.177062in}}%
\pgfpathcurveto{\pgfqpoint{1.763535in}{2.185298in}}{\pgfqpoint{1.760263in}{2.193198in}}{\pgfqpoint{1.754439in}{2.199022in}}%
\pgfpathcurveto{\pgfqpoint{1.748615in}{2.204846in}}{\pgfqpoint{1.740715in}{2.208118in}}{\pgfqpoint{1.732479in}{2.208118in}}%
\pgfpathcurveto{\pgfqpoint{1.724242in}{2.208118in}}{\pgfqpoint{1.716342in}{2.204846in}}{\pgfqpoint{1.710518in}{2.199022in}}%
\pgfpathcurveto{\pgfqpoint{1.704694in}{2.193198in}}{\pgfqpoint{1.701422in}{2.185298in}}{\pgfqpoint{1.701422in}{2.177062in}}%
\pgfpathcurveto{\pgfqpoint{1.701422in}{2.168825in}}{\pgfqpoint{1.704694in}{2.160925in}}{\pgfqpoint{1.710518in}{2.155101in}}%
\pgfpathcurveto{\pgfqpoint{1.716342in}{2.149277in}}{\pgfqpoint{1.724242in}{2.146005in}}{\pgfqpoint{1.732479in}{2.146005in}}%
\pgfpathclose%
\pgfusepath{stroke,fill}%
\end{pgfscope}%
\begin{pgfscope}%
\pgfpathrectangle{\pgfqpoint{0.100000in}{0.212622in}}{\pgfqpoint{3.696000in}{3.696000in}}%
\pgfusepath{clip}%
\pgfsetbuttcap%
\pgfsetroundjoin%
\definecolor{currentfill}{rgb}{0.121569,0.466667,0.705882}%
\pgfsetfillcolor{currentfill}%
\pgfsetfillopacity{0.928449}%
\pgfsetlinewidth{1.003750pt}%
\definecolor{currentstroke}{rgb}{0.121569,0.466667,0.705882}%
\pgfsetstrokecolor{currentstroke}%
\pgfsetstrokeopacity{0.928449}%
\pgfsetdash{}{0pt}%
\pgfpathmoveto{\pgfqpoint{1.739472in}{2.145855in}}%
\pgfpathcurveto{\pgfqpoint{1.747708in}{2.145855in}}{\pgfqpoint{1.755608in}{2.149128in}}{\pgfqpoint{1.761432in}{2.154952in}}%
\pgfpathcurveto{\pgfqpoint{1.767256in}{2.160775in}}{\pgfqpoint{1.770529in}{2.168676in}}{\pgfqpoint{1.770529in}{2.176912in}}%
\pgfpathcurveto{\pgfqpoint{1.770529in}{2.185148in}}{\pgfqpoint{1.767256in}{2.193048in}}{\pgfqpoint{1.761432in}{2.198872in}}%
\pgfpathcurveto{\pgfqpoint{1.755608in}{2.204696in}}{\pgfqpoint{1.747708in}{2.207968in}}{\pgfqpoint{1.739472in}{2.207968in}}%
\pgfpathcurveto{\pgfqpoint{1.731236in}{2.207968in}}{\pgfqpoint{1.723336in}{2.204696in}}{\pgfqpoint{1.717512in}{2.198872in}}%
\pgfpathcurveto{\pgfqpoint{1.711688in}{2.193048in}}{\pgfqpoint{1.708416in}{2.185148in}}{\pgfqpoint{1.708416in}{2.176912in}}%
\pgfpathcurveto{\pgfqpoint{1.708416in}{2.168676in}}{\pgfqpoint{1.711688in}{2.160775in}}{\pgfqpoint{1.717512in}{2.154952in}}%
\pgfpathcurveto{\pgfqpoint{1.723336in}{2.149128in}}{\pgfqpoint{1.731236in}{2.145855in}}{\pgfqpoint{1.739472in}{2.145855in}}%
\pgfpathclose%
\pgfusepath{stroke,fill}%
\end{pgfscope}%
\begin{pgfscope}%
\pgfpathrectangle{\pgfqpoint{0.100000in}{0.212622in}}{\pgfqpoint{3.696000in}{3.696000in}}%
\pgfusepath{clip}%
\pgfsetbuttcap%
\pgfsetroundjoin%
\definecolor{currentfill}{rgb}{0.121569,0.466667,0.705882}%
\pgfsetfillcolor{currentfill}%
\pgfsetfillopacity{0.929125}%
\pgfsetlinewidth{1.003750pt}%
\definecolor{currentstroke}{rgb}{0.121569,0.466667,0.705882}%
\pgfsetstrokecolor{currentstroke}%
\pgfsetstrokeopacity{0.929125}%
\pgfsetdash{}{0pt}%
\pgfpathmoveto{\pgfqpoint{2.570619in}{1.879899in}}%
\pgfpathcurveto{\pgfqpoint{2.578855in}{1.879899in}}{\pgfqpoint{2.586755in}{1.883171in}}{\pgfqpoint{2.592579in}{1.888995in}}%
\pgfpathcurveto{\pgfqpoint{2.598403in}{1.894819in}}{\pgfqpoint{2.601676in}{1.902719in}}{\pgfqpoint{2.601676in}{1.910955in}}%
\pgfpathcurveto{\pgfqpoint{2.601676in}{1.919192in}}{\pgfqpoint{2.598403in}{1.927092in}}{\pgfqpoint{2.592579in}{1.932916in}}%
\pgfpathcurveto{\pgfqpoint{2.586755in}{1.938739in}}{\pgfqpoint{2.578855in}{1.942012in}}{\pgfqpoint{2.570619in}{1.942012in}}%
\pgfpathcurveto{\pgfqpoint{2.562383in}{1.942012in}}{\pgfqpoint{2.554483in}{1.938739in}}{\pgfqpoint{2.548659in}{1.932916in}}%
\pgfpathcurveto{\pgfqpoint{2.542835in}{1.927092in}}{\pgfqpoint{2.539563in}{1.919192in}}{\pgfqpoint{2.539563in}{1.910955in}}%
\pgfpathcurveto{\pgfqpoint{2.539563in}{1.902719in}}{\pgfqpoint{2.542835in}{1.894819in}}{\pgfqpoint{2.548659in}{1.888995in}}%
\pgfpathcurveto{\pgfqpoint{2.554483in}{1.883171in}}{\pgfqpoint{2.562383in}{1.879899in}}{\pgfqpoint{2.570619in}{1.879899in}}%
\pgfpathclose%
\pgfusepath{stroke,fill}%
\end{pgfscope}%
\begin{pgfscope}%
\pgfpathrectangle{\pgfqpoint{0.100000in}{0.212622in}}{\pgfqpoint{3.696000in}{3.696000in}}%
\pgfusepath{clip}%
\pgfsetbuttcap%
\pgfsetroundjoin%
\definecolor{currentfill}{rgb}{0.121569,0.466667,0.705882}%
\pgfsetfillcolor{currentfill}%
\pgfsetfillopacity{0.929467}%
\pgfsetlinewidth{1.003750pt}%
\definecolor{currentstroke}{rgb}{0.121569,0.466667,0.705882}%
\pgfsetstrokecolor{currentstroke}%
\pgfsetstrokeopacity{0.929467}%
\pgfsetdash{}{0pt}%
\pgfpathmoveto{\pgfqpoint{1.751119in}{2.141120in}}%
\pgfpathcurveto{\pgfqpoint{1.759355in}{2.141120in}}{\pgfqpoint{1.767255in}{2.144392in}}{\pgfqpoint{1.773079in}{2.150216in}}%
\pgfpathcurveto{\pgfqpoint{1.778903in}{2.156040in}}{\pgfqpoint{1.782175in}{2.163940in}}{\pgfqpoint{1.782175in}{2.172176in}}%
\pgfpathcurveto{\pgfqpoint{1.782175in}{2.180413in}}{\pgfqpoint{1.778903in}{2.188313in}}{\pgfqpoint{1.773079in}{2.194136in}}%
\pgfpathcurveto{\pgfqpoint{1.767255in}{2.199960in}}{\pgfqpoint{1.759355in}{2.203233in}}{\pgfqpoint{1.751119in}{2.203233in}}%
\pgfpathcurveto{\pgfqpoint{1.742882in}{2.203233in}}{\pgfqpoint{1.734982in}{2.199960in}}{\pgfqpoint{1.729158in}{2.194136in}}%
\pgfpathcurveto{\pgfqpoint{1.723334in}{2.188313in}}{\pgfqpoint{1.720062in}{2.180413in}}{\pgfqpoint{1.720062in}{2.172176in}}%
\pgfpathcurveto{\pgfqpoint{1.720062in}{2.163940in}}{\pgfqpoint{1.723334in}{2.156040in}}{\pgfqpoint{1.729158in}{2.150216in}}%
\pgfpathcurveto{\pgfqpoint{1.734982in}{2.144392in}}{\pgfqpoint{1.742882in}{2.141120in}}{\pgfqpoint{1.751119in}{2.141120in}}%
\pgfpathclose%
\pgfusepath{stroke,fill}%
\end{pgfscope}%
\begin{pgfscope}%
\pgfpathrectangle{\pgfqpoint{0.100000in}{0.212622in}}{\pgfqpoint{3.696000in}{3.696000in}}%
\pgfusepath{clip}%
\pgfsetbuttcap%
\pgfsetroundjoin%
\definecolor{currentfill}{rgb}{0.121569,0.466667,0.705882}%
\pgfsetfillcolor{currentfill}%
\pgfsetfillopacity{0.930757}%
\pgfsetlinewidth{1.003750pt}%
\definecolor{currentstroke}{rgb}{0.121569,0.466667,0.705882}%
\pgfsetstrokecolor{currentstroke}%
\pgfsetstrokeopacity{0.930757}%
\pgfsetdash{}{0pt}%
\pgfpathmoveto{\pgfqpoint{1.761390in}{2.137760in}}%
\pgfpathcurveto{\pgfqpoint{1.769626in}{2.137760in}}{\pgfqpoint{1.777526in}{2.141032in}}{\pgfqpoint{1.783350in}{2.146856in}}%
\pgfpathcurveto{\pgfqpoint{1.789174in}{2.152680in}}{\pgfqpoint{1.792446in}{2.160580in}}{\pgfqpoint{1.792446in}{2.168816in}}%
\pgfpathcurveto{\pgfqpoint{1.792446in}{2.177052in}}{\pgfqpoint{1.789174in}{2.184952in}}{\pgfqpoint{1.783350in}{2.190776in}}%
\pgfpathcurveto{\pgfqpoint{1.777526in}{2.196600in}}{\pgfqpoint{1.769626in}{2.199873in}}{\pgfqpoint{1.761390in}{2.199873in}}%
\pgfpathcurveto{\pgfqpoint{1.753154in}{2.199873in}}{\pgfqpoint{1.745254in}{2.196600in}}{\pgfqpoint{1.739430in}{2.190776in}}%
\pgfpathcurveto{\pgfqpoint{1.733606in}{2.184952in}}{\pgfqpoint{1.730333in}{2.177052in}}{\pgfqpoint{1.730333in}{2.168816in}}%
\pgfpathcurveto{\pgfqpoint{1.730333in}{2.160580in}}{\pgfqpoint{1.733606in}{2.152680in}}{\pgfqpoint{1.739430in}{2.146856in}}%
\pgfpathcurveto{\pgfqpoint{1.745254in}{2.141032in}}{\pgfqpoint{1.753154in}{2.137760in}}{\pgfqpoint{1.761390in}{2.137760in}}%
\pgfpathclose%
\pgfusepath{stroke,fill}%
\end{pgfscope}%
\begin{pgfscope}%
\pgfpathrectangle{\pgfqpoint{0.100000in}{0.212622in}}{\pgfqpoint{3.696000in}{3.696000in}}%
\pgfusepath{clip}%
\pgfsetbuttcap%
\pgfsetroundjoin%
\definecolor{currentfill}{rgb}{0.121569,0.466667,0.705882}%
\pgfsetfillcolor{currentfill}%
\pgfsetfillopacity{0.931553}%
\pgfsetlinewidth{1.003750pt}%
\definecolor{currentstroke}{rgb}{0.121569,0.466667,0.705882}%
\pgfsetstrokecolor{currentstroke}%
\pgfsetstrokeopacity{0.931553}%
\pgfsetdash{}{0pt}%
\pgfpathmoveto{\pgfqpoint{2.564774in}{1.881180in}}%
\pgfpathcurveto{\pgfqpoint{2.573010in}{1.881180in}}{\pgfqpoint{2.580910in}{1.884453in}}{\pgfqpoint{2.586734in}{1.890276in}}%
\pgfpathcurveto{\pgfqpoint{2.592558in}{1.896100in}}{\pgfqpoint{2.595830in}{1.904000in}}{\pgfqpoint{2.595830in}{1.912237in}}%
\pgfpathcurveto{\pgfqpoint{2.595830in}{1.920473in}}{\pgfqpoint{2.592558in}{1.928373in}}{\pgfqpoint{2.586734in}{1.934197in}}%
\pgfpathcurveto{\pgfqpoint{2.580910in}{1.940021in}}{\pgfqpoint{2.573010in}{1.943293in}}{\pgfqpoint{2.564774in}{1.943293in}}%
\pgfpathcurveto{\pgfqpoint{2.556538in}{1.943293in}}{\pgfqpoint{2.548638in}{1.940021in}}{\pgfqpoint{2.542814in}{1.934197in}}%
\pgfpathcurveto{\pgfqpoint{2.536990in}{1.928373in}}{\pgfqpoint{2.533717in}{1.920473in}}{\pgfqpoint{2.533717in}{1.912237in}}%
\pgfpathcurveto{\pgfqpoint{2.533717in}{1.904000in}}{\pgfqpoint{2.536990in}{1.896100in}}{\pgfqpoint{2.542814in}{1.890276in}}%
\pgfpathcurveto{\pgfqpoint{2.548638in}{1.884453in}}{\pgfqpoint{2.556538in}{1.881180in}}{\pgfqpoint{2.564774in}{1.881180in}}%
\pgfpathclose%
\pgfusepath{stroke,fill}%
\end{pgfscope}%
\begin{pgfscope}%
\pgfpathrectangle{\pgfqpoint{0.100000in}{0.212622in}}{\pgfqpoint{3.696000in}{3.696000in}}%
\pgfusepath{clip}%
\pgfsetbuttcap%
\pgfsetroundjoin%
\definecolor{currentfill}{rgb}{0.121569,0.466667,0.705882}%
\pgfsetfillcolor{currentfill}%
\pgfsetfillopacity{0.931712}%
\pgfsetlinewidth{1.003750pt}%
\definecolor{currentstroke}{rgb}{0.121569,0.466667,0.705882}%
\pgfsetstrokecolor{currentstroke}%
\pgfsetstrokeopacity{0.931712}%
\pgfsetdash{}{0pt}%
\pgfpathmoveto{\pgfqpoint{1.768062in}{2.133659in}}%
\pgfpathcurveto{\pgfqpoint{1.776299in}{2.133659in}}{\pgfqpoint{1.784199in}{2.136931in}}{\pgfqpoint{1.790023in}{2.142755in}}%
\pgfpathcurveto{\pgfqpoint{1.795847in}{2.148579in}}{\pgfqpoint{1.799119in}{2.156479in}}{\pgfqpoint{1.799119in}{2.164716in}}%
\pgfpathcurveto{\pgfqpoint{1.799119in}{2.172952in}}{\pgfqpoint{1.795847in}{2.180852in}}{\pgfqpoint{1.790023in}{2.186676in}}%
\pgfpathcurveto{\pgfqpoint{1.784199in}{2.192500in}}{\pgfqpoint{1.776299in}{2.195772in}}{\pgfqpoint{1.768062in}{2.195772in}}%
\pgfpathcurveto{\pgfqpoint{1.759826in}{2.195772in}}{\pgfqpoint{1.751926in}{2.192500in}}{\pgfqpoint{1.746102in}{2.186676in}}%
\pgfpathcurveto{\pgfqpoint{1.740278in}{2.180852in}}{\pgfqpoint{1.737006in}{2.172952in}}{\pgfqpoint{1.737006in}{2.164716in}}%
\pgfpathcurveto{\pgfqpoint{1.737006in}{2.156479in}}{\pgfqpoint{1.740278in}{2.148579in}}{\pgfqpoint{1.746102in}{2.142755in}}%
\pgfpathcurveto{\pgfqpoint{1.751926in}{2.136931in}}{\pgfqpoint{1.759826in}{2.133659in}}{\pgfqpoint{1.768062in}{2.133659in}}%
\pgfpathclose%
\pgfusepath{stroke,fill}%
\end{pgfscope}%
\begin{pgfscope}%
\pgfpathrectangle{\pgfqpoint{0.100000in}{0.212622in}}{\pgfqpoint{3.696000in}{3.696000in}}%
\pgfusepath{clip}%
\pgfsetbuttcap%
\pgfsetroundjoin%
\definecolor{currentfill}{rgb}{0.121569,0.466667,0.705882}%
\pgfsetfillcolor{currentfill}%
\pgfsetfillopacity{0.932383}%
\pgfsetlinewidth{1.003750pt}%
\definecolor{currentstroke}{rgb}{0.121569,0.466667,0.705882}%
\pgfsetstrokecolor{currentstroke}%
\pgfsetstrokeopacity{0.932383}%
\pgfsetdash{}{0pt}%
\pgfpathmoveto{\pgfqpoint{1.773688in}{2.131697in}}%
\pgfpathcurveto{\pgfqpoint{1.781924in}{2.131697in}}{\pgfqpoint{1.789824in}{2.134969in}}{\pgfqpoint{1.795648in}{2.140793in}}%
\pgfpathcurveto{\pgfqpoint{1.801472in}{2.146617in}}{\pgfqpoint{1.804744in}{2.154517in}}{\pgfqpoint{1.804744in}{2.162753in}}%
\pgfpathcurveto{\pgfqpoint{1.804744in}{2.170990in}}{\pgfqpoint{1.801472in}{2.178890in}}{\pgfqpoint{1.795648in}{2.184714in}}%
\pgfpathcurveto{\pgfqpoint{1.789824in}{2.190538in}}{\pgfqpoint{1.781924in}{2.193810in}}{\pgfqpoint{1.773688in}{2.193810in}}%
\pgfpathcurveto{\pgfqpoint{1.765451in}{2.193810in}}{\pgfqpoint{1.757551in}{2.190538in}}{\pgfqpoint{1.751727in}{2.184714in}}%
\pgfpathcurveto{\pgfqpoint{1.745903in}{2.178890in}}{\pgfqpoint{1.742631in}{2.170990in}}{\pgfqpoint{1.742631in}{2.162753in}}%
\pgfpathcurveto{\pgfqpoint{1.742631in}{2.154517in}}{\pgfqpoint{1.745903in}{2.146617in}}{\pgfqpoint{1.751727in}{2.140793in}}%
\pgfpathcurveto{\pgfqpoint{1.757551in}{2.134969in}}{\pgfqpoint{1.765451in}{2.131697in}}{\pgfqpoint{1.773688in}{2.131697in}}%
\pgfpathclose%
\pgfusepath{stroke,fill}%
\end{pgfscope}%
\begin{pgfscope}%
\pgfpathrectangle{\pgfqpoint{0.100000in}{0.212622in}}{\pgfqpoint{3.696000in}{3.696000in}}%
\pgfusepath{clip}%
\pgfsetbuttcap%
\pgfsetroundjoin%
\definecolor{currentfill}{rgb}{0.121569,0.466667,0.705882}%
\pgfsetfillcolor{currentfill}%
\pgfsetfillopacity{0.933391}%
\pgfsetlinewidth{1.003750pt}%
\definecolor{currentstroke}{rgb}{0.121569,0.466667,0.705882}%
\pgfsetstrokecolor{currentstroke}%
\pgfsetstrokeopacity{0.933391}%
\pgfsetdash{}{0pt}%
\pgfpathmoveto{\pgfqpoint{1.784358in}{2.128238in}}%
\pgfpathcurveto{\pgfqpoint{1.792594in}{2.128238in}}{\pgfqpoint{1.800494in}{2.131510in}}{\pgfqpoint{1.806318in}{2.137334in}}%
\pgfpathcurveto{\pgfqpoint{1.812142in}{2.143158in}}{\pgfqpoint{1.815415in}{2.151058in}}{\pgfqpoint{1.815415in}{2.159294in}}%
\pgfpathcurveto{\pgfqpoint{1.815415in}{2.167530in}}{\pgfqpoint{1.812142in}{2.175431in}}{\pgfqpoint{1.806318in}{2.181254in}}%
\pgfpathcurveto{\pgfqpoint{1.800494in}{2.187078in}}{\pgfqpoint{1.792594in}{2.190351in}}{\pgfqpoint{1.784358in}{2.190351in}}%
\pgfpathcurveto{\pgfqpoint{1.776122in}{2.190351in}}{\pgfqpoint{1.768222in}{2.187078in}}{\pgfqpoint{1.762398in}{2.181254in}}%
\pgfpathcurveto{\pgfqpoint{1.756574in}{2.175431in}}{\pgfqpoint{1.753302in}{2.167530in}}{\pgfqpoint{1.753302in}{2.159294in}}%
\pgfpathcurveto{\pgfqpoint{1.753302in}{2.151058in}}{\pgfqpoint{1.756574in}{2.143158in}}{\pgfqpoint{1.762398in}{2.137334in}}%
\pgfpathcurveto{\pgfqpoint{1.768222in}{2.131510in}}{\pgfqpoint{1.776122in}{2.128238in}}{\pgfqpoint{1.784358in}{2.128238in}}%
\pgfpathclose%
\pgfusepath{stroke,fill}%
\end{pgfscope}%
\begin{pgfscope}%
\pgfpathrectangle{\pgfqpoint{0.100000in}{0.212622in}}{\pgfqpoint{3.696000in}{3.696000in}}%
\pgfusepath{clip}%
\pgfsetbuttcap%
\pgfsetroundjoin%
\definecolor{currentfill}{rgb}{0.121569,0.466667,0.705882}%
\pgfsetfillcolor{currentfill}%
\pgfsetfillopacity{0.934210}%
\pgfsetlinewidth{1.003750pt}%
\definecolor{currentstroke}{rgb}{0.121569,0.466667,0.705882}%
\pgfsetstrokecolor{currentstroke}%
\pgfsetstrokeopacity{0.934210}%
\pgfsetdash{}{0pt}%
\pgfpathmoveto{\pgfqpoint{1.792906in}{2.123589in}}%
\pgfpathcurveto{\pgfqpoint{1.801142in}{2.123589in}}{\pgfqpoint{1.809042in}{2.126861in}}{\pgfqpoint{1.814866in}{2.132685in}}%
\pgfpathcurveto{\pgfqpoint{1.820690in}{2.138509in}}{\pgfqpoint{1.823962in}{2.146409in}}{\pgfqpoint{1.823962in}{2.154645in}}%
\pgfpathcurveto{\pgfqpoint{1.823962in}{2.162882in}}{\pgfqpoint{1.820690in}{2.170782in}}{\pgfqpoint{1.814866in}{2.176606in}}%
\pgfpathcurveto{\pgfqpoint{1.809042in}{2.182429in}}{\pgfqpoint{1.801142in}{2.185702in}}{\pgfqpoint{1.792906in}{2.185702in}}%
\pgfpathcurveto{\pgfqpoint{1.784669in}{2.185702in}}{\pgfqpoint{1.776769in}{2.182429in}}{\pgfqpoint{1.770945in}{2.176606in}}%
\pgfpathcurveto{\pgfqpoint{1.765121in}{2.170782in}}{\pgfqpoint{1.761849in}{2.162882in}}{\pgfqpoint{1.761849in}{2.154645in}}%
\pgfpathcurveto{\pgfqpoint{1.761849in}{2.146409in}}{\pgfqpoint{1.765121in}{2.138509in}}{\pgfqpoint{1.770945in}{2.132685in}}%
\pgfpathcurveto{\pgfqpoint{1.776769in}{2.126861in}}{\pgfqpoint{1.784669in}{2.123589in}}{\pgfqpoint{1.792906in}{2.123589in}}%
\pgfpathclose%
\pgfusepath{stroke,fill}%
\end{pgfscope}%
\begin{pgfscope}%
\pgfpathrectangle{\pgfqpoint{0.100000in}{0.212622in}}{\pgfqpoint{3.696000in}{3.696000in}}%
\pgfusepath{clip}%
\pgfsetbuttcap%
\pgfsetroundjoin%
\definecolor{currentfill}{rgb}{0.121569,0.466667,0.705882}%
\pgfsetfillcolor{currentfill}%
\pgfsetfillopacity{0.934774}%
\pgfsetlinewidth{1.003750pt}%
\definecolor{currentstroke}{rgb}{0.121569,0.466667,0.705882}%
\pgfsetstrokecolor{currentstroke}%
\pgfsetstrokeopacity{0.934774}%
\pgfsetdash{}{0pt}%
\pgfpathmoveto{\pgfqpoint{2.558384in}{1.881948in}}%
\pgfpathcurveto{\pgfqpoint{2.566620in}{1.881948in}}{\pgfqpoint{2.574520in}{1.885221in}}{\pgfqpoint{2.580344in}{1.891045in}}%
\pgfpathcurveto{\pgfqpoint{2.586168in}{1.896868in}}{\pgfqpoint{2.589440in}{1.904768in}}{\pgfqpoint{2.589440in}{1.913005in}}%
\pgfpathcurveto{\pgfqpoint{2.589440in}{1.921241in}}{\pgfqpoint{2.586168in}{1.929141in}}{\pgfqpoint{2.580344in}{1.934965in}}%
\pgfpathcurveto{\pgfqpoint{2.574520in}{1.940789in}}{\pgfqpoint{2.566620in}{1.944061in}}{\pgfqpoint{2.558384in}{1.944061in}}%
\pgfpathcurveto{\pgfqpoint{2.550148in}{1.944061in}}{\pgfqpoint{2.542248in}{1.940789in}}{\pgfqpoint{2.536424in}{1.934965in}}%
\pgfpathcurveto{\pgfqpoint{2.530600in}{1.929141in}}{\pgfqpoint{2.527327in}{1.921241in}}{\pgfqpoint{2.527327in}{1.913005in}}%
\pgfpathcurveto{\pgfqpoint{2.527327in}{1.904768in}}{\pgfqpoint{2.530600in}{1.896868in}}{\pgfqpoint{2.536424in}{1.891045in}}%
\pgfpathcurveto{\pgfqpoint{2.542248in}{1.885221in}}{\pgfqpoint{2.550148in}{1.881948in}}{\pgfqpoint{2.558384in}{1.881948in}}%
\pgfpathclose%
\pgfusepath{stroke,fill}%
\end{pgfscope}%
\begin{pgfscope}%
\pgfpathrectangle{\pgfqpoint{0.100000in}{0.212622in}}{\pgfqpoint{3.696000in}{3.696000in}}%
\pgfusepath{clip}%
\pgfsetbuttcap%
\pgfsetroundjoin%
\definecolor{currentfill}{rgb}{0.121569,0.466667,0.705882}%
\pgfsetfillcolor{currentfill}%
\pgfsetfillopacity{0.934923}%
\pgfsetlinewidth{1.003750pt}%
\definecolor{currentstroke}{rgb}{0.121569,0.466667,0.705882}%
\pgfsetstrokecolor{currentstroke}%
\pgfsetstrokeopacity{0.934923}%
\pgfsetdash{}{0pt}%
\pgfpathmoveto{\pgfqpoint{1.799494in}{2.121746in}}%
\pgfpathcurveto{\pgfqpoint{1.807730in}{2.121746in}}{\pgfqpoint{1.815630in}{2.125019in}}{\pgfqpoint{1.821454in}{2.130843in}}%
\pgfpathcurveto{\pgfqpoint{1.827278in}{2.136667in}}{\pgfqpoint{1.830550in}{2.144567in}}{\pgfqpoint{1.830550in}{2.152803in}}%
\pgfpathcurveto{\pgfqpoint{1.830550in}{2.161039in}}{\pgfqpoint{1.827278in}{2.168939in}}{\pgfqpoint{1.821454in}{2.174763in}}%
\pgfpathcurveto{\pgfqpoint{1.815630in}{2.180587in}}{\pgfqpoint{1.807730in}{2.183859in}}{\pgfqpoint{1.799494in}{2.183859in}}%
\pgfpathcurveto{\pgfqpoint{1.791257in}{2.183859in}}{\pgfqpoint{1.783357in}{2.180587in}}{\pgfqpoint{1.777533in}{2.174763in}}%
\pgfpathcurveto{\pgfqpoint{1.771709in}{2.168939in}}{\pgfqpoint{1.768437in}{2.161039in}}{\pgfqpoint{1.768437in}{2.152803in}}%
\pgfpathcurveto{\pgfqpoint{1.768437in}{2.144567in}}{\pgfqpoint{1.771709in}{2.136667in}}{\pgfqpoint{1.777533in}{2.130843in}}%
\pgfpathcurveto{\pgfqpoint{1.783357in}{2.125019in}}{\pgfqpoint{1.791257in}{2.121746in}}{\pgfqpoint{1.799494in}{2.121746in}}%
\pgfpathclose%
\pgfusepath{stroke,fill}%
\end{pgfscope}%
\begin{pgfscope}%
\pgfpathrectangle{\pgfqpoint{0.100000in}{0.212622in}}{\pgfqpoint{3.696000in}{3.696000in}}%
\pgfusepath{clip}%
\pgfsetbuttcap%
\pgfsetroundjoin%
\definecolor{currentfill}{rgb}{0.121569,0.466667,0.705882}%
\pgfsetfillcolor{currentfill}%
\pgfsetfillopacity{0.936153}%
\pgfsetlinewidth{1.003750pt}%
\definecolor{currentstroke}{rgb}{0.121569,0.466667,0.705882}%
\pgfsetstrokecolor{currentstroke}%
\pgfsetstrokeopacity{0.936153}%
\pgfsetdash{}{0pt}%
\pgfpathmoveto{\pgfqpoint{1.811405in}{2.117646in}}%
\pgfpathcurveto{\pgfqpoint{1.819641in}{2.117646in}}{\pgfqpoint{1.827541in}{2.120919in}}{\pgfqpoint{1.833365in}{2.126743in}}%
\pgfpathcurveto{\pgfqpoint{1.839189in}{2.132566in}}{\pgfqpoint{1.842461in}{2.140467in}}{\pgfqpoint{1.842461in}{2.148703in}}%
\pgfpathcurveto{\pgfqpoint{1.842461in}{2.156939in}}{\pgfqpoint{1.839189in}{2.164839in}}{\pgfqpoint{1.833365in}{2.170663in}}%
\pgfpathcurveto{\pgfqpoint{1.827541in}{2.176487in}}{\pgfqpoint{1.819641in}{2.179759in}}{\pgfqpoint{1.811405in}{2.179759in}}%
\pgfpathcurveto{\pgfqpoint{1.803168in}{2.179759in}}{\pgfqpoint{1.795268in}{2.176487in}}{\pgfqpoint{1.789444in}{2.170663in}}%
\pgfpathcurveto{\pgfqpoint{1.783620in}{2.164839in}}{\pgfqpoint{1.780348in}{2.156939in}}{\pgfqpoint{1.780348in}{2.148703in}}%
\pgfpathcurveto{\pgfqpoint{1.780348in}{2.140467in}}{\pgfqpoint{1.783620in}{2.132566in}}{\pgfqpoint{1.789444in}{2.126743in}}%
\pgfpathcurveto{\pgfqpoint{1.795268in}{2.120919in}}{\pgfqpoint{1.803168in}{2.117646in}}{\pgfqpoint{1.811405in}{2.117646in}}%
\pgfpathclose%
\pgfusepath{stroke,fill}%
\end{pgfscope}%
\begin{pgfscope}%
\pgfpathrectangle{\pgfqpoint{0.100000in}{0.212622in}}{\pgfqpoint{3.696000in}{3.696000in}}%
\pgfusepath{clip}%
\pgfsetbuttcap%
\pgfsetroundjoin%
\definecolor{currentfill}{rgb}{0.121569,0.466667,0.705882}%
\pgfsetfillcolor{currentfill}%
\pgfsetfillopacity{0.936668}%
\pgfsetlinewidth{1.003750pt}%
\definecolor{currentstroke}{rgb}{0.121569,0.466667,0.705882}%
\pgfsetstrokecolor{currentstroke}%
\pgfsetstrokeopacity{0.936668}%
\pgfsetdash{}{0pt}%
\pgfpathmoveto{\pgfqpoint{2.555980in}{1.882083in}}%
\pgfpathcurveto{\pgfqpoint{2.564216in}{1.882083in}}{\pgfqpoint{2.572116in}{1.885355in}}{\pgfqpoint{2.577940in}{1.891179in}}%
\pgfpathcurveto{\pgfqpoint{2.583764in}{1.897003in}}{\pgfqpoint{2.587036in}{1.904903in}}{\pgfqpoint{2.587036in}{1.913139in}}%
\pgfpathcurveto{\pgfqpoint{2.587036in}{1.921376in}}{\pgfqpoint{2.583764in}{1.929276in}}{\pgfqpoint{2.577940in}{1.935100in}}%
\pgfpathcurveto{\pgfqpoint{2.572116in}{1.940924in}}{\pgfqpoint{2.564216in}{1.944196in}}{\pgfqpoint{2.555980in}{1.944196in}}%
\pgfpathcurveto{\pgfqpoint{2.547743in}{1.944196in}}{\pgfqpoint{2.539843in}{1.940924in}}{\pgfqpoint{2.534019in}{1.935100in}}%
\pgfpathcurveto{\pgfqpoint{2.528195in}{1.929276in}}{\pgfqpoint{2.524923in}{1.921376in}}{\pgfqpoint{2.524923in}{1.913139in}}%
\pgfpathcurveto{\pgfqpoint{2.524923in}{1.904903in}}{\pgfqpoint{2.528195in}{1.897003in}}{\pgfqpoint{2.534019in}{1.891179in}}%
\pgfpathcurveto{\pgfqpoint{2.539843in}{1.885355in}}{\pgfqpoint{2.547743in}{1.882083in}}{\pgfqpoint{2.555980in}{1.882083in}}%
\pgfpathclose%
\pgfusepath{stroke,fill}%
\end{pgfscope}%
\begin{pgfscope}%
\pgfpathrectangle{\pgfqpoint{0.100000in}{0.212622in}}{\pgfqpoint{3.696000in}{3.696000in}}%
\pgfusepath{clip}%
\pgfsetbuttcap%
\pgfsetroundjoin%
\definecolor{currentfill}{rgb}{0.121569,0.466667,0.705882}%
\pgfsetfillcolor{currentfill}%
\pgfsetfillopacity{0.937083}%
\pgfsetlinewidth{1.003750pt}%
\definecolor{currentstroke}{rgb}{0.121569,0.466667,0.705882}%
\pgfsetstrokecolor{currentstroke}%
\pgfsetstrokeopacity{0.937083}%
\pgfsetdash{}{0pt}%
\pgfpathmoveto{\pgfqpoint{1.822488in}{2.112622in}}%
\pgfpathcurveto{\pgfqpoint{1.830724in}{2.112622in}}{\pgfqpoint{1.838624in}{2.115894in}}{\pgfqpoint{1.844448in}{2.121718in}}%
\pgfpathcurveto{\pgfqpoint{1.850272in}{2.127542in}}{\pgfqpoint{1.853545in}{2.135442in}}{\pgfqpoint{1.853545in}{2.143678in}}%
\pgfpathcurveto{\pgfqpoint{1.853545in}{2.151915in}}{\pgfqpoint{1.850272in}{2.159815in}}{\pgfqpoint{1.844448in}{2.165639in}}%
\pgfpathcurveto{\pgfqpoint{1.838624in}{2.171462in}}{\pgfqpoint{1.830724in}{2.174735in}}{\pgfqpoint{1.822488in}{2.174735in}}%
\pgfpathcurveto{\pgfqpoint{1.814252in}{2.174735in}}{\pgfqpoint{1.806352in}{2.171462in}}{\pgfqpoint{1.800528in}{2.165639in}}%
\pgfpathcurveto{\pgfqpoint{1.794704in}{2.159815in}}{\pgfqpoint{1.791432in}{2.151915in}}{\pgfqpoint{1.791432in}{2.143678in}}%
\pgfpathcurveto{\pgfqpoint{1.791432in}{2.135442in}}{\pgfqpoint{1.794704in}{2.127542in}}{\pgfqpoint{1.800528in}{2.121718in}}%
\pgfpathcurveto{\pgfqpoint{1.806352in}{2.115894in}}{\pgfqpoint{1.814252in}{2.112622in}}{\pgfqpoint{1.822488in}{2.112622in}}%
\pgfpathclose%
\pgfusepath{stroke,fill}%
\end{pgfscope}%
\begin{pgfscope}%
\pgfpathrectangle{\pgfqpoint{0.100000in}{0.212622in}}{\pgfqpoint{3.696000in}{3.696000in}}%
\pgfusepath{clip}%
\pgfsetbuttcap%
\pgfsetroundjoin%
\definecolor{currentfill}{rgb}{0.121569,0.466667,0.705882}%
\pgfsetfillcolor{currentfill}%
\pgfsetfillopacity{0.938241}%
\pgfsetlinewidth{1.003750pt}%
\definecolor{currentstroke}{rgb}{0.121569,0.466667,0.705882}%
\pgfsetstrokecolor{currentstroke}%
\pgfsetstrokeopacity{0.938241}%
\pgfsetdash{}{0pt}%
\pgfpathmoveto{\pgfqpoint{1.830100in}{2.111308in}}%
\pgfpathcurveto{\pgfqpoint{1.838336in}{2.111308in}}{\pgfqpoint{1.846236in}{2.114580in}}{\pgfqpoint{1.852060in}{2.120404in}}%
\pgfpathcurveto{\pgfqpoint{1.857884in}{2.126228in}}{\pgfqpoint{1.861156in}{2.134128in}}{\pgfqpoint{1.861156in}{2.142364in}}%
\pgfpathcurveto{\pgfqpoint{1.861156in}{2.150601in}}{\pgfqpoint{1.857884in}{2.158501in}}{\pgfqpoint{1.852060in}{2.164325in}}%
\pgfpathcurveto{\pgfqpoint{1.846236in}{2.170149in}}{\pgfqpoint{1.838336in}{2.173421in}}{\pgfqpoint{1.830100in}{2.173421in}}%
\pgfpathcurveto{\pgfqpoint{1.821863in}{2.173421in}}{\pgfqpoint{1.813963in}{2.170149in}}{\pgfqpoint{1.808139in}{2.164325in}}%
\pgfpathcurveto{\pgfqpoint{1.802315in}{2.158501in}}{\pgfqpoint{1.799043in}{2.150601in}}{\pgfqpoint{1.799043in}{2.142364in}}%
\pgfpathcurveto{\pgfqpoint{1.799043in}{2.134128in}}{\pgfqpoint{1.802315in}{2.126228in}}{\pgfqpoint{1.808139in}{2.120404in}}%
\pgfpathcurveto{\pgfqpoint{1.813963in}{2.114580in}}{\pgfqpoint{1.821863in}{2.111308in}}{\pgfqpoint{1.830100in}{2.111308in}}%
\pgfpathclose%
\pgfusepath{stroke,fill}%
\end{pgfscope}%
\begin{pgfscope}%
\pgfpathrectangle{\pgfqpoint{0.100000in}{0.212622in}}{\pgfqpoint{3.696000in}{3.696000in}}%
\pgfusepath{clip}%
\pgfsetbuttcap%
\pgfsetroundjoin%
\definecolor{currentfill}{rgb}{0.121569,0.466667,0.705882}%
\pgfsetfillcolor{currentfill}%
\pgfsetfillopacity{0.938813}%
\pgfsetlinewidth{1.003750pt}%
\definecolor{currentstroke}{rgb}{0.121569,0.466667,0.705882}%
\pgfsetstrokecolor{currentstroke}%
\pgfsetstrokeopacity{0.938813}%
\pgfsetdash{}{0pt}%
\pgfpathmoveto{\pgfqpoint{2.550876in}{1.883097in}}%
\pgfpathcurveto{\pgfqpoint{2.559113in}{1.883097in}}{\pgfqpoint{2.567013in}{1.886369in}}{\pgfqpoint{2.572837in}{1.892193in}}%
\pgfpathcurveto{\pgfqpoint{2.578661in}{1.898017in}}{\pgfqpoint{2.581933in}{1.905917in}}{\pgfqpoint{2.581933in}{1.914153in}}%
\pgfpathcurveto{\pgfqpoint{2.581933in}{1.922389in}}{\pgfqpoint{2.578661in}{1.930289in}}{\pgfqpoint{2.572837in}{1.936113in}}%
\pgfpathcurveto{\pgfqpoint{2.567013in}{1.941937in}}{\pgfqpoint{2.559113in}{1.945210in}}{\pgfqpoint{2.550876in}{1.945210in}}%
\pgfpathcurveto{\pgfqpoint{2.542640in}{1.945210in}}{\pgfqpoint{2.534740in}{1.941937in}}{\pgfqpoint{2.528916in}{1.936113in}}%
\pgfpathcurveto{\pgfqpoint{2.523092in}{1.930289in}}{\pgfqpoint{2.519820in}{1.922389in}}{\pgfqpoint{2.519820in}{1.914153in}}%
\pgfpathcurveto{\pgfqpoint{2.519820in}{1.905917in}}{\pgfqpoint{2.523092in}{1.898017in}}{\pgfqpoint{2.528916in}{1.892193in}}%
\pgfpathcurveto{\pgfqpoint{2.534740in}{1.886369in}}{\pgfqpoint{2.542640in}{1.883097in}}{\pgfqpoint{2.550876in}{1.883097in}}%
\pgfpathclose%
\pgfusepath{stroke,fill}%
\end{pgfscope}%
\begin{pgfscope}%
\pgfpathrectangle{\pgfqpoint{0.100000in}{0.212622in}}{\pgfqpoint{3.696000in}{3.696000in}}%
\pgfusepath{clip}%
\pgfsetbuttcap%
\pgfsetroundjoin%
\definecolor{currentfill}{rgb}{0.121569,0.466667,0.705882}%
\pgfsetfillcolor{currentfill}%
\pgfsetfillopacity{0.939203}%
\pgfsetlinewidth{1.003750pt}%
\definecolor{currentstroke}{rgb}{0.121569,0.466667,0.705882}%
\pgfsetstrokecolor{currentstroke}%
\pgfsetstrokeopacity{0.939203}%
\pgfsetdash{}{0pt}%
\pgfpathmoveto{\pgfqpoint{1.837404in}{2.109781in}}%
\pgfpathcurveto{\pgfqpoint{1.845640in}{2.109781in}}{\pgfqpoint{1.853540in}{2.113053in}}{\pgfqpoint{1.859364in}{2.118877in}}%
\pgfpathcurveto{\pgfqpoint{1.865188in}{2.124701in}}{\pgfqpoint{1.868460in}{2.132601in}}{\pgfqpoint{1.868460in}{2.140837in}}%
\pgfpathcurveto{\pgfqpoint{1.868460in}{2.149073in}}{\pgfqpoint{1.865188in}{2.156973in}}{\pgfqpoint{1.859364in}{2.162797in}}%
\pgfpathcurveto{\pgfqpoint{1.853540in}{2.168621in}}{\pgfqpoint{1.845640in}{2.171894in}}{\pgfqpoint{1.837404in}{2.171894in}}%
\pgfpathcurveto{\pgfqpoint{1.829167in}{2.171894in}}{\pgfqpoint{1.821267in}{2.168621in}}{\pgfqpoint{1.815443in}{2.162797in}}%
\pgfpathcurveto{\pgfqpoint{1.809619in}{2.156973in}}{\pgfqpoint{1.806347in}{2.149073in}}{\pgfqpoint{1.806347in}{2.140837in}}%
\pgfpathcurveto{\pgfqpoint{1.806347in}{2.132601in}}{\pgfqpoint{1.809619in}{2.124701in}}{\pgfqpoint{1.815443in}{2.118877in}}%
\pgfpathcurveto{\pgfqpoint{1.821267in}{2.113053in}}{\pgfqpoint{1.829167in}{2.109781in}}{\pgfqpoint{1.837404in}{2.109781in}}%
\pgfpathclose%
\pgfusepath{stroke,fill}%
\end{pgfscope}%
\begin{pgfscope}%
\pgfpathrectangle{\pgfqpoint{0.100000in}{0.212622in}}{\pgfqpoint{3.696000in}{3.696000in}}%
\pgfusepath{clip}%
\pgfsetbuttcap%
\pgfsetroundjoin%
\definecolor{currentfill}{rgb}{0.121569,0.466667,0.705882}%
\pgfsetfillcolor{currentfill}%
\pgfsetfillopacity{0.940844}%
\pgfsetlinewidth{1.003750pt}%
\definecolor{currentstroke}{rgb}{0.121569,0.466667,0.705882}%
\pgfsetstrokecolor{currentstroke}%
\pgfsetstrokeopacity{0.940844}%
\pgfsetdash{}{0pt}%
\pgfpathmoveto{\pgfqpoint{1.850363in}{2.105122in}}%
\pgfpathcurveto{\pgfqpoint{1.858600in}{2.105122in}}{\pgfqpoint{1.866500in}{2.108394in}}{\pgfqpoint{1.872324in}{2.114218in}}%
\pgfpathcurveto{\pgfqpoint{1.878148in}{2.120042in}}{\pgfqpoint{1.881420in}{2.127942in}}{\pgfqpoint{1.881420in}{2.136178in}}%
\pgfpathcurveto{\pgfqpoint{1.881420in}{2.144414in}}{\pgfqpoint{1.878148in}{2.152314in}}{\pgfqpoint{1.872324in}{2.158138in}}%
\pgfpathcurveto{\pgfqpoint{1.866500in}{2.163962in}}{\pgfqpoint{1.858600in}{2.167235in}}{\pgfqpoint{1.850363in}{2.167235in}}%
\pgfpathcurveto{\pgfqpoint{1.842127in}{2.167235in}}{\pgfqpoint{1.834227in}{2.163962in}}{\pgfqpoint{1.828403in}{2.158138in}}%
\pgfpathcurveto{\pgfqpoint{1.822579in}{2.152314in}}{\pgfqpoint{1.819307in}{2.144414in}}{\pgfqpoint{1.819307in}{2.136178in}}%
\pgfpathcurveto{\pgfqpoint{1.819307in}{2.127942in}}{\pgfqpoint{1.822579in}{2.120042in}}{\pgfqpoint{1.828403in}{2.114218in}}%
\pgfpathcurveto{\pgfqpoint{1.834227in}{2.108394in}}{\pgfqpoint{1.842127in}{2.105122in}}{\pgfqpoint{1.850363in}{2.105122in}}%
\pgfpathclose%
\pgfusepath{stroke,fill}%
\end{pgfscope}%
\begin{pgfscope}%
\pgfpathrectangle{\pgfqpoint{0.100000in}{0.212622in}}{\pgfqpoint{3.696000in}{3.696000in}}%
\pgfusepath{clip}%
\pgfsetbuttcap%
\pgfsetroundjoin%
\definecolor{currentfill}{rgb}{0.121569,0.466667,0.705882}%
\pgfsetfillcolor{currentfill}%
\pgfsetfillopacity{0.940987}%
\pgfsetlinewidth{1.003750pt}%
\definecolor{currentstroke}{rgb}{0.121569,0.466667,0.705882}%
\pgfsetstrokecolor{currentstroke}%
\pgfsetstrokeopacity{0.940987}%
\pgfsetdash{}{0pt}%
\pgfpathmoveto{\pgfqpoint{2.544293in}{1.885155in}}%
\pgfpathcurveto{\pgfqpoint{2.552530in}{1.885155in}}{\pgfqpoint{2.560430in}{1.888427in}}{\pgfqpoint{2.566254in}{1.894251in}}%
\pgfpathcurveto{\pgfqpoint{2.572078in}{1.900075in}}{\pgfqpoint{2.575350in}{1.907975in}}{\pgfqpoint{2.575350in}{1.916212in}}%
\pgfpathcurveto{\pgfqpoint{2.575350in}{1.924448in}}{\pgfqpoint{2.572078in}{1.932348in}}{\pgfqpoint{2.566254in}{1.938172in}}%
\pgfpathcurveto{\pgfqpoint{2.560430in}{1.943996in}}{\pgfqpoint{2.552530in}{1.947268in}}{\pgfqpoint{2.544293in}{1.947268in}}%
\pgfpathcurveto{\pgfqpoint{2.536057in}{1.947268in}}{\pgfqpoint{2.528157in}{1.943996in}}{\pgfqpoint{2.522333in}{1.938172in}}%
\pgfpathcurveto{\pgfqpoint{2.516509in}{1.932348in}}{\pgfqpoint{2.513237in}{1.924448in}}{\pgfqpoint{2.513237in}{1.916212in}}%
\pgfpathcurveto{\pgfqpoint{2.513237in}{1.907975in}}{\pgfqpoint{2.516509in}{1.900075in}}{\pgfqpoint{2.522333in}{1.894251in}}%
\pgfpathcurveto{\pgfqpoint{2.528157in}{1.888427in}}{\pgfqpoint{2.536057in}{1.885155in}}{\pgfqpoint{2.544293in}{1.885155in}}%
\pgfpathclose%
\pgfusepath{stroke,fill}%
\end{pgfscope}%
\begin{pgfscope}%
\pgfpathrectangle{\pgfqpoint{0.100000in}{0.212622in}}{\pgfqpoint{3.696000in}{3.696000in}}%
\pgfusepath{clip}%
\pgfsetbuttcap%
\pgfsetroundjoin%
\definecolor{currentfill}{rgb}{0.121569,0.466667,0.705882}%
\pgfsetfillcolor{currentfill}%
\pgfsetfillopacity{0.942268}%
\pgfsetlinewidth{1.003750pt}%
\definecolor{currentstroke}{rgb}{0.121569,0.466667,0.705882}%
\pgfsetstrokecolor{currentstroke}%
\pgfsetstrokeopacity{0.942268}%
\pgfsetdash{}{0pt}%
\pgfpathmoveto{\pgfqpoint{1.862921in}{2.102249in}}%
\pgfpathcurveto{\pgfqpoint{1.871157in}{2.102249in}}{\pgfqpoint{1.879057in}{2.105521in}}{\pgfqpoint{1.884881in}{2.111345in}}%
\pgfpathcurveto{\pgfqpoint{1.890705in}{2.117169in}}{\pgfqpoint{1.893977in}{2.125069in}}{\pgfqpoint{1.893977in}{2.133305in}}%
\pgfpathcurveto{\pgfqpoint{1.893977in}{2.141541in}}{\pgfqpoint{1.890705in}{2.149442in}}{\pgfqpoint{1.884881in}{2.155265in}}%
\pgfpathcurveto{\pgfqpoint{1.879057in}{2.161089in}}{\pgfqpoint{1.871157in}{2.164362in}}{\pgfqpoint{1.862921in}{2.164362in}}%
\pgfpathcurveto{\pgfqpoint{1.854685in}{2.164362in}}{\pgfqpoint{1.846785in}{2.161089in}}{\pgfqpoint{1.840961in}{2.155265in}}%
\pgfpathcurveto{\pgfqpoint{1.835137in}{2.149442in}}{\pgfqpoint{1.831864in}{2.141541in}}{\pgfqpoint{1.831864in}{2.133305in}}%
\pgfpathcurveto{\pgfqpoint{1.831864in}{2.125069in}}{\pgfqpoint{1.835137in}{2.117169in}}{\pgfqpoint{1.840961in}{2.111345in}}%
\pgfpathcurveto{\pgfqpoint{1.846785in}{2.105521in}}{\pgfqpoint{1.854685in}{2.102249in}}{\pgfqpoint{1.862921in}{2.102249in}}%
\pgfpathclose%
\pgfusepath{stroke,fill}%
\end{pgfscope}%
\begin{pgfscope}%
\pgfpathrectangle{\pgfqpoint{0.100000in}{0.212622in}}{\pgfqpoint{3.696000in}{3.696000in}}%
\pgfusepath{clip}%
\pgfsetbuttcap%
\pgfsetroundjoin%
\definecolor{currentfill}{rgb}{0.121569,0.466667,0.705882}%
\pgfsetfillcolor{currentfill}%
\pgfsetfillopacity{0.942647}%
\pgfsetlinewidth{1.003750pt}%
\definecolor{currentstroke}{rgb}{0.121569,0.466667,0.705882}%
\pgfsetstrokecolor{currentstroke}%
\pgfsetstrokeopacity{0.942647}%
\pgfsetdash{}{0pt}%
\pgfpathmoveto{\pgfqpoint{1.874765in}{2.103990in}}%
\pgfpathcurveto{\pgfqpoint{1.883002in}{2.103990in}}{\pgfqpoint{1.890902in}{2.107262in}}{\pgfqpoint{1.896726in}{2.113086in}}%
\pgfpathcurveto{\pgfqpoint{1.902550in}{2.118910in}}{\pgfqpoint{1.905822in}{2.126810in}}{\pgfqpoint{1.905822in}{2.135047in}}%
\pgfpathcurveto{\pgfqpoint{1.905822in}{2.143283in}}{\pgfqpoint{1.902550in}{2.151183in}}{\pgfqpoint{1.896726in}{2.157007in}}%
\pgfpathcurveto{\pgfqpoint{1.890902in}{2.162831in}}{\pgfqpoint{1.883002in}{2.166103in}}{\pgfqpoint{1.874765in}{2.166103in}}%
\pgfpathcurveto{\pgfqpoint{1.866529in}{2.166103in}}{\pgfqpoint{1.858629in}{2.162831in}}{\pgfqpoint{1.852805in}{2.157007in}}%
\pgfpathcurveto{\pgfqpoint{1.846981in}{2.151183in}}{\pgfqpoint{1.843709in}{2.143283in}}{\pgfqpoint{1.843709in}{2.135047in}}%
\pgfpathcurveto{\pgfqpoint{1.843709in}{2.126810in}}{\pgfqpoint{1.846981in}{2.118910in}}{\pgfqpoint{1.852805in}{2.113086in}}%
\pgfpathcurveto{\pgfqpoint{1.858629in}{2.107262in}}{\pgfqpoint{1.866529in}{2.103990in}}{\pgfqpoint{1.874765in}{2.103990in}}%
\pgfpathclose%
\pgfusepath{stroke,fill}%
\end{pgfscope}%
\begin{pgfscope}%
\pgfpathrectangle{\pgfqpoint{0.100000in}{0.212622in}}{\pgfqpoint{3.696000in}{3.696000in}}%
\pgfusepath{clip}%
\pgfsetbuttcap%
\pgfsetroundjoin%
\definecolor{currentfill}{rgb}{0.121569,0.466667,0.705882}%
\pgfsetfillcolor{currentfill}%
\pgfsetfillopacity{0.944213}%
\pgfsetlinewidth{1.003750pt}%
\definecolor{currentstroke}{rgb}{0.121569,0.466667,0.705882}%
\pgfsetstrokecolor{currentstroke}%
\pgfsetstrokeopacity{0.944213}%
\pgfsetdash{}{0pt}%
\pgfpathmoveto{\pgfqpoint{2.540615in}{1.885340in}}%
\pgfpathcurveto{\pgfqpoint{2.548851in}{1.885340in}}{\pgfqpoint{2.556751in}{1.888612in}}{\pgfqpoint{2.562575in}{1.894436in}}%
\pgfpathcurveto{\pgfqpoint{2.568399in}{1.900260in}}{\pgfqpoint{2.571671in}{1.908160in}}{\pgfqpoint{2.571671in}{1.916396in}}%
\pgfpathcurveto{\pgfqpoint{2.571671in}{1.924632in}}{\pgfqpoint{2.568399in}{1.932532in}}{\pgfqpoint{2.562575in}{1.938356in}}%
\pgfpathcurveto{\pgfqpoint{2.556751in}{1.944180in}}{\pgfqpoint{2.548851in}{1.947453in}}{\pgfqpoint{2.540615in}{1.947453in}}%
\pgfpathcurveto{\pgfqpoint{2.532378in}{1.947453in}}{\pgfqpoint{2.524478in}{1.944180in}}{\pgfqpoint{2.518654in}{1.938356in}}%
\pgfpathcurveto{\pgfqpoint{2.512830in}{1.932532in}}{\pgfqpoint{2.509558in}{1.924632in}}{\pgfqpoint{2.509558in}{1.916396in}}%
\pgfpathcurveto{\pgfqpoint{2.509558in}{1.908160in}}{\pgfqpoint{2.512830in}{1.900260in}}{\pgfqpoint{2.518654in}{1.894436in}}%
\pgfpathcurveto{\pgfqpoint{2.524478in}{1.888612in}}{\pgfqpoint{2.532378in}{1.885340in}}{\pgfqpoint{2.540615in}{1.885340in}}%
\pgfpathclose%
\pgfusepath{stroke,fill}%
\end{pgfscope}%
\begin{pgfscope}%
\pgfpathrectangle{\pgfqpoint{0.100000in}{0.212622in}}{\pgfqpoint{3.696000in}{3.696000in}}%
\pgfusepath{clip}%
\pgfsetbuttcap%
\pgfsetroundjoin%
\definecolor{currentfill}{rgb}{0.121569,0.466667,0.705882}%
\pgfsetfillcolor{currentfill}%
\pgfsetfillopacity{0.944757}%
\pgfsetlinewidth{1.003750pt}%
\definecolor{currentstroke}{rgb}{0.121569,0.466667,0.705882}%
\pgfsetstrokecolor{currentstroke}%
\pgfsetstrokeopacity{0.944757}%
\pgfsetdash{}{0pt}%
\pgfpathmoveto{\pgfqpoint{1.893622in}{2.098092in}}%
\pgfpathcurveto{\pgfqpoint{1.901858in}{2.098092in}}{\pgfqpoint{1.909758in}{2.101364in}}{\pgfqpoint{1.915582in}{2.107188in}}%
\pgfpathcurveto{\pgfqpoint{1.921406in}{2.113012in}}{\pgfqpoint{1.924678in}{2.120912in}}{\pgfqpoint{1.924678in}{2.129148in}}%
\pgfpathcurveto{\pgfqpoint{1.924678in}{2.137384in}}{\pgfqpoint{1.921406in}{2.145284in}}{\pgfqpoint{1.915582in}{2.151108in}}%
\pgfpathcurveto{\pgfqpoint{1.909758in}{2.156932in}}{\pgfqpoint{1.901858in}{2.160205in}}{\pgfqpoint{1.893622in}{2.160205in}}%
\pgfpathcurveto{\pgfqpoint{1.885386in}{2.160205in}}{\pgfqpoint{1.877485in}{2.156932in}}{\pgfqpoint{1.871662in}{2.151108in}}%
\pgfpathcurveto{\pgfqpoint{1.865838in}{2.145284in}}{\pgfqpoint{1.862565in}{2.137384in}}{\pgfqpoint{1.862565in}{2.129148in}}%
\pgfpathcurveto{\pgfqpoint{1.862565in}{2.120912in}}{\pgfqpoint{1.865838in}{2.113012in}}{\pgfqpoint{1.871662in}{2.107188in}}%
\pgfpathcurveto{\pgfqpoint{1.877485in}{2.101364in}}{\pgfqpoint{1.885386in}{2.098092in}}{\pgfqpoint{1.893622in}{2.098092in}}%
\pgfpathclose%
\pgfusepath{stroke,fill}%
\end{pgfscope}%
\begin{pgfscope}%
\pgfpathrectangle{\pgfqpoint{0.100000in}{0.212622in}}{\pgfqpoint{3.696000in}{3.696000in}}%
\pgfusepath{clip}%
\pgfsetbuttcap%
\pgfsetroundjoin%
\definecolor{currentfill}{rgb}{0.121569,0.466667,0.705882}%
\pgfsetfillcolor{currentfill}%
\pgfsetfillopacity{0.945881}%
\pgfsetlinewidth{1.003750pt}%
\definecolor{currentstroke}{rgb}{0.121569,0.466667,0.705882}%
\pgfsetstrokecolor{currentstroke}%
\pgfsetstrokeopacity{0.945881}%
\pgfsetdash{}{0pt}%
\pgfpathmoveto{\pgfqpoint{2.537355in}{1.885887in}}%
\pgfpathcurveto{\pgfqpoint{2.545591in}{1.885887in}}{\pgfqpoint{2.553491in}{1.889159in}}{\pgfqpoint{2.559315in}{1.894983in}}%
\pgfpathcurveto{\pgfqpoint{2.565139in}{1.900807in}}{\pgfqpoint{2.568411in}{1.908707in}}{\pgfqpoint{2.568411in}{1.916943in}}%
\pgfpathcurveto{\pgfqpoint{2.568411in}{1.925179in}}{\pgfqpoint{2.565139in}{1.933080in}}{\pgfqpoint{2.559315in}{1.938903in}}%
\pgfpathcurveto{\pgfqpoint{2.553491in}{1.944727in}}{\pgfqpoint{2.545591in}{1.948000in}}{\pgfqpoint{2.537355in}{1.948000in}}%
\pgfpathcurveto{\pgfqpoint{2.529119in}{1.948000in}}{\pgfqpoint{2.521219in}{1.944727in}}{\pgfqpoint{2.515395in}{1.938903in}}%
\pgfpathcurveto{\pgfqpoint{2.509571in}{1.933080in}}{\pgfqpoint{2.506298in}{1.925179in}}{\pgfqpoint{2.506298in}{1.916943in}}%
\pgfpathcurveto{\pgfqpoint{2.506298in}{1.908707in}}{\pgfqpoint{2.509571in}{1.900807in}}{\pgfqpoint{2.515395in}{1.894983in}}%
\pgfpathcurveto{\pgfqpoint{2.521219in}{1.889159in}}{\pgfqpoint{2.529119in}{1.885887in}}{\pgfqpoint{2.537355in}{1.885887in}}%
\pgfpathclose%
\pgfusepath{stroke,fill}%
\end{pgfscope}%
\begin{pgfscope}%
\pgfpathrectangle{\pgfqpoint{0.100000in}{0.212622in}}{\pgfqpoint{3.696000in}{3.696000in}}%
\pgfusepath{clip}%
\pgfsetbuttcap%
\pgfsetroundjoin%
\definecolor{currentfill}{rgb}{0.121569,0.466667,0.705882}%
\pgfsetfillcolor{currentfill}%
\pgfsetfillopacity{0.946736}%
\pgfsetlinewidth{1.003750pt}%
\definecolor{currentstroke}{rgb}{0.121569,0.466667,0.705882}%
\pgfsetstrokecolor{currentstroke}%
\pgfsetstrokeopacity{0.946736}%
\pgfsetdash{}{0pt}%
\pgfpathmoveto{\pgfqpoint{2.535043in}{1.886542in}}%
\pgfpathcurveto{\pgfqpoint{2.543279in}{1.886542in}}{\pgfqpoint{2.551179in}{1.889814in}}{\pgfqpoint{2.557003in}{1.895638in}}%
\pgfpathcurveto{\pgfqpoint{2.562827in}{1.901462in}}{\pgfqpoint{2.566099in}{1.909362in}}{\pgfqpoint{2.566099in}{1.917599in}}%
\pgfpathcurveto{\pgfqpoint{2.566099in}{1.925835in}}{\pgfqpoint{2.562827in}{1.933735in}}{\pgfqpoint{2.557003in}{1.939559in}}%
\pgfpathcurveto{\pgfqpoint{2.551179in}{1.945383in}}{\pgfqpoint{2.543279in}{1.948655in}}{\pgfqpoint{2.535043in}{1.948655in}}%
\pgfpathcurveto{\pgfqpoint{2.526806in}{1.948655in}}{\pgfqpoint{2.518906in}{1.945383in}}{\pgfqpoint{2.513082in}{1.939559in}}%
\pgfpathcurveto{\pgfqpoint{2.507258in}{1.933735in}}{\pgfqpoint{2.503986in}{1.925835in}}{\pgfqpoint{2.503986in}{1.917599in}}%
\pgfpathcurveto{\pgfqpoint{2.503986in}{1.909362in}}{\pgfqpoint{2.507258in}{1.901462in}}{\pgfqpoint{2.513082in}{1.895638in}}%
\pgfpathcurveto{\pgfqpoint{2.518906in}{1.889814in}}{\pgfqpoint{2.526806in}{1.886542in}}{\pgfqpoint{2.535043in}{1.886542in}}%
\pgfpathclose%
\pgfusepath{stroke,fill}%
\end{pgfscope}%
\begin{pgfscope}%
\pgfpathrectangle{\pgfqpoint{0.100000in}{0.212622in}}{\pgfqpoint{3.696000in}{3.696000in}}%
\pgfusepath{clip}%
\pgfsetbuttcap%
\pgfsetroundjoin%
\definecolor{currentfill}{rgb}{0.121569,0.466667,0.705882}%
\pgfsetfillcolor{currentfill}%
\pgfsetfillopacity{0.947974}%
\pgfsetlinewidth{1.003750pt}%
\definecolor{currentstroke}{rgb}{0.121569,0.466667,0.705882}%
\pgfsetstrokecolor{currentstroke}%
\pgfsetstrokeopacity{0.947974}%
\pgfsetdash{}{0pt}%
\pgfpathmoveto{\pgfqpoint{1.910357in}{2.095879in}}%
\pgfpathcurveto{\pgfqpoint{1.918593in}{2.095879in}}{\pgfqpoint{1.926493in}{2.099152in}}{\pgfqpoint{1.932317in}{2.104975in}}%
\pgfpathcurveto{\pgfqpoint{1.938141in}{2.110799in}}{\pgfqpoint{1.941414in}{2.118699in}}{\pgfqpoint{1.941414in}{2.126936in}}%
\pgfpathcurveto{\pgfqpoint{1.941414in}{2.135172in}}{\pgfqpoint{1.938141in}{2.143072in}}{\pgfqpoint{1.932317in}{2.148896in}}%
\pgfpathcurveto{\pgfqpoint{1.926493in}{2.154720in}}{\pgfqpoint{1.918593in}{2.157992in}}{\pgfqpoint{1.910357in}{2.157992in}}%
\pgfpathcurveto{\pgfqpoint{1.902121in}{2.157992in}}{\pgfqpoint{1.894221in}{2.154720in}}{\pgfqpoint{1.888397in}{2.148896in}}%
\pgfpathcurveto{\pgfqpoint{1.882573in}{2.143072in}}{\pgfqpoint{1.879301in}{2.135172in}}{\pgfqpoint{1.879301in}{2.126936in}}%
\pgfpathcurveto{\pgfqpoint{1.879301in}{2.118699in}}{\pgfqpoint{1.882573in}{2.110799in}}{\pgfqpoint{1.888397in}{2.104975in}}%
\pgfpathcurveto{\pgfqpoint{1.894221in}{2.099152in}}{\pgfqpoint{1.902121in}{2.095879in}}{\pgfqpoint{1.910357in}{2.095879in}}%
\pgfpathclose%
\pgfusepath{stroke,fill}%
\end{pgfscope}%
\begin{pgfscope}%
\pgfpathrectangle{\pgfqpoint{0.100000in}{0.212622in}}{\pgfqpoint{3.696000in}{3.696000in}}%
\pgfusepath{clip}%
\pgfsetbuttcap%
\pgfsetroundjoin%
\definecolor{currentfill}{rgb}{0.121569,0.466667,0.705882}%
\pgfsetfillcolor{currentfill}%
\pgfsetfillopacity{0.948295}%
\pgfsetlinewidth{1.003750pt}%
\definecolor{currentstroke}{rgb}{0.121569,0.466667,0.705882}%
\pgfsetstrokecolor{currentstroke}%
\pgfsetstrokeopacity{0.948295}%
\pgfsetdash{}{0pt}%
\pgfpathmoveto{\pgfqpoint{2.532393in}{1.886724in}}%
\pgfpathcurveto{\pgfqpoint{2.540630in}{1.886724in}}{\pgfqpoint{2.548530in}{1.889996in}}{\pgfqpoint{2.554354in}{1.895820in}}%
\pgfpathcurveto{\pgfqpoint{2.560177in}{1.901644in}}{\pgfqpoint{2.563450in}{1.909544in}}{\pgfqpoint{2.563450in}{1.917780in}}%
\pgfpathcurveto{\pgfqpoint{2.563450in}{1.926017in}}{\pgfqpoint{2.560177in}{1.933917in}}{\pgfqpoint{2.554354in}{1.939741in}}%
\pgfpathcurveto{\pgfqpoint{2.548530in}{1.945565in}}{\pgfqpoint{2.540630in}{1.948837in}}{\pgfqpoint{2.532393in}{1.948837in}}%
\pgfpathcurveto{\pgfqpoint{2.524157in}{1.948837in}}{\pgfqpoint{2.516257in}{1.945565in}}{\pgfqpoint{2.510433in}{1.939741in}}%
\pgfpathcurveto{\pgfqpoint{2.504609in}{1.933917in}}{\pgfqpoint{2.501337in}{1.926017in}}{\pgfqpoint{2.501337in}{1.917780in}}%
\pgfpathcurveto{\pgfqpoint{2.501337in}{1.909544in}}{\pgfqpoint{2.504609in}{1.901644in}}{\pgfqpoint{2.510433in}{1.895820in}}%
\pgfpathcurveto{\pgfqpoint{2.516257in}{1.889996in}}{\pgfqpoint{2.524157in}{1.886724in}}{\pgfqpoint{2.532393in}{1.886724in}}%
\pgfpathclose%
\pgfusepath{stroke,fill}%
\end{pgfscope}%
\begin{pgfscope}%
\pgfpathrectangle{\pgfqpoint{0.100000in}{0.212622in}}{\pgfqpoint{3.696000in}{3.696000in}}%
\pgfusepath{clip}%
\pgfsetbuttcap%
\pgfsetroundjoin%
\definecolor{currentfill}{rgb}{0.121569,0.466667,0.705882}%
\pgfsetfillcolor{currentfill}%
\pgfsetfillopacity{0.949202}%
\pgfsetlinewidth{1.003750pt}%
\definecolor{currentstroke}{rgb}{0.121569,0.466667,0.705882}%
\pgfsetstrokecolor{currentstroke}%
\pgfsetstrokeopacity{0.949202}%
\pgfsetdash{}{0pt}%
\pgfpathmoveto{\pgfqpoint{2.531276in}{1.886851in}}%
\pgfpathcurveto{\pgfqpoint{2.539512in}{1.886851in}}{\pgfqpoint{2.547412in}{1.890123in}}{\pgfqpoint{2.553236in}{1.895947in}}%
\pgfpathcurveto{\pgfqpoint{2.559060in}{1.901771in}}{\pgfqpoint{2.562332in}{1.909671in}}{\pgfqpoint{2.562332in}{1.917907in}}%
\pgfpathcurveto{\pgfqpoint{2.562332in}{1.926143in}}{\pgfqpoint{2.559060in}{1.934043in}}{\pgfqpoint{2.553236in}{1.939867in}}%
\pgfpathcurveto{\pgfqpoint{2.547412in}{1.945691in}}{\pgfqpoint{2.539512in}{1.948964in}}{\pgfqpoint{2.531276in}{1.948964in}}%
\pgfpathcurveto{\pgfqpoint{2.523040in}{1.948964in}}{\pgfqpoint{2.515140in}{1.945691in}}{\pgfqpoint{2.509316in}{1.939867in}}%
\pgfpathcurveto{\pgfqpoint{2.503492in}{1.934043in}}{\pgfqpoint{2.500219in}{1.926143in}}{\pgfqpoint{2.500219in}{1.917907in}}%
\pgfpathcurveto{\pgfqpoint{2.500219in}{1.909671in}}{\pgfqpoint{2.503492in}{1.901771in}}{\pgfqpoint{2.509316in}{1.895947in}}%
\pgfpathcurveto{\pgfqpoint{2.515140in}{1.890123in}}{\pgfqpoint{2.523040in}{1.886851in}}{\pgfqpoint{2.531276in}{1.886851in}}%
\pgfpathclose%
\pgfusepath{stroke,fill}%
\end{pgfscope}%
\begin{pgfscope}%
\pgfpathrectangle{\pgfqpoint{0.100000in}{0.212622in}}{\pgfqpoint{3.696000in}{3.696000in}}%
\pgfusepath{clip}%
\pgfsetbuttcap%
\pgfsetroundjoin%
\definecolor{currentfill}{rgb}{0.121569,0.466667,0.705882}%
\pgfsetfillcolor{currentfill}%
\pgfsetfillopacity{0.949423}%
\pgfsetlinewidth{1.003750pt}%
\definecolor{currentstroke}{rgb}{0.121569,0.466667,0.705882}%
\pgfsetstrokecolor{currentstroke}%
\pgfsetstrokeopacity{0.949423}%
\pgfsetdash{}{0pt}%
\pgfpathmoveto{\pgfqpoint{1.926940in}{2.089151in}}%
\pgfpathcurveto{\pgfqpoint{1.935176in}{2.089151in}}{\pgfqpoint{1.943076in}{2.092423in}}{\pgfqpoint{1.948900in}{2.098247in}}%
\pgfpathcurveto{\pgfqpoint{1.954724in}{2.104071in}}{\pgfqpoint{1.957996in}{2.111971in}}{\pgfqpoint{1.957996in}{2.120208in}}%
\pgfpathcurveto{\pgfqpoint{1.957996in}{2.128444in}}{\pgfqpoint{1.954724in}{2.136344in}}{\pgfqpoint{1.948900in}{2.142168in}}%
\pgfpathcurveto{\pgfqpoint{1.943076in}{2.147992in}}{\pgfqpoint{1.935176in}{2.151264in}}{\pgfqpoint{1.926940in}{2.151264in}}%
\pgfpathcurveto{\pgfqpoint{1.918704in}{2.151264in}}{\pgfqpoint{1.910804in}{2.147992in}}{\pgfqpoint{1.904980in}{2.142168in}}%
\pgfpathcurveto{\pgfqpoint{1.899156in}{2.136344in}}{\pgfqpoint{1.895883in}{2.128444in}}{\pgfqpoint{1.895883in}{2.120208in}}%
\pgfpathcurveto{\pgfqpoint{1.895883in}{2.111971in}}{\pgfqpoint{1.899156in}{2.104071in}}{\pgfqpoint{1.904980in}{2.098247in}}%
\pgfpathcurveto{\pgfqpoint{1.910804in}{2.092423in}}{\pgfqpoint{1.918704in}{2.089151in}}{\pgfqpoint{1.926940in}{2.089151in}}%
\pgfpathclose%
\pgfusepath{stroke,fill}%
\end{pgfscope}%
\begin{pgfscope}%
\pgfpathrectangle{\pgfqpoint{0.100000in}{0.212622in}}{\pgfqpoint{3.696000in}{3.696000in}}%
\pgfusepath{clip}%
\pgfsetbuttcap%
\pgfsetroundjoin%
\definecolor{currentfill}{rgb}{0.121569,0.466667,0.705882}%
\pgfsetfillcolor{currentfill}%
\pgfsetfillopacity{0.950472}%
\pgfsetlinewidth{1.003750pt}%
\definecolor{currentstroke}{rgb}{0.121569,0.466667,0.705882}%
\pgfsetstrokecolor{currentstroke}%
\pgfsetstrokeopacity{0.950472}%
\pgfsetdash{}{0pt}%
\pgfpathmoveto{\pgfqpoint{2.528395in}{1.887447in}}%
\pgfpathcurveto{\pgfqpoint{2.536632in}{1.887447in}}{\pgfqpoint{2.544532in}{1.890720in}}{\pgfqpoint{2.550356in}{1.896543in}}%
\pgfpathcurveto{\pgfqpoint{2.556180in}{1.902367in}}{\pgfqpoint{2.559452in}{1.910267in}}{\pgfqpoint{2.559452in}{1.918504in}}%
\pgfpathcurveto{\pgfqpoint{2.559452in}{1.926740in}}{\pgfqpoint{2.556180in}{1.934640in}}{\pgfqpoint{2.550356in}{1.940464in}}%
\pgfpathcurveto{\pgfqpoint{2.544532in}{1.946288in}}{\pgfqpoint{2.536632in}{1.949560in}}{\pgfqpoint{2.528395in}{1.949560in}}%
\pgfpathcurveto{\pgfqpoint{2.520159in}{1.949560in}}{\pgfqpoint{2.512259in}{1.946288in}}{\pgfqpoint{2.506435in}{1.940464in}}%
\pgfpathcurveto{\pgfqpoint{2.500611in}{1.934640in}}{\pgfqpoint{2.497339in}{1.926740in}}{\pgfqpoint{2.497339in}{1.918504in}}%
\pgfpathcurveto{\pgfqpoint{2.497339in}{1.910267in}}{\pgfqpoint{2.500611in}{1.902367in}}{\pgfqpoint{2.506435in}{1.896543in}}%
\pgfpathcurveto{\pgfqpoint{2.512259in}{1.890720in}}{\pgfqpoint{2.520159in}{1.887447in}}{\pgfqpoint{2.528395in}{1.887447in}}%
\pgfpathclose%
\pgfusepath{stroke,fill}%
\end{pgfscope}%
\begin{pgfscope}%
\pgfpathrectangle{\pgfqpoint{0.100000in}{0.212622in}}{\pgfqpoint{3.696000in}{3.696000in}}%
\pgfusepath{clip}%
\pgfsetbuttcap%
\pgfsetroundjoin%
\definecolor{currentfill}{rgb}{0.121569,0.466667,0.705882}%
\pgfsetfillcolor{currentfill}%
\pgfsetfillopacity{0.951141}%
\pgfsetlinewidth{1.003750pt}%
\definecolor{currentstroke}{rgb}{0.121569,0.466667,0.705882}%
\pgfsetstrokecolor{currentstroke}%
\pgfsetstrokeopacity{0.951141}%
\pgfsetdash{}{0pt}%
\pgfpathmoveto{\pgfqpoint{1.942859in}{2.084369in}}%
\pgfpathcurveto{\pgfqpoint{1.951096in}{2.084369in}}{\pgfqpoint{1.958996in}{2.087642in}}{\pgfqpoint{1.964820in}{2.093466in}}%
\pgfpathcurveto{\pgfqpoint{1.970644in}{2.099290in}}{\pgfqpoint{1.973916in}{2.107190in}}{\pgfqpoint{1.973916in}{2.115426in}}%
\pgfpathcurveto{\pgfqpoint{1.973916in}{2.123662in}}{\pgfqpoint{1.970644in}{2.131562in}}{\pgfqpoint{1.964820in}{2.137386in}}%
\pgfpathcurveto{\pgfqpoint{1.958996in}{2.143210in}}{\pgfqpoint{1.951096in}{2.146482in}}{\pgfqpoint{1.942859in}{2.146482in}}%
\pgfpathcurveto{\pgfqpoint{1.934623in}{2.146482in}}{\pgfqpoint{1.926723in}{2.143210in}}{\pgfqpoint{1.920899in}{2.137386in}}%
\pgfpathcurveto{\pgfqpoint{1.915075in}{2.131562in}}{\pgfqpoint{1.911803in}{2.123662in}}{\pgfqpoint{1.911803in}{2.115426in}}%
\pgfpathcurveto{\pgfqpoint{1.911803in}{2.107190in}}{\pgfqpoint{1.915075in}{2.099290in}}{\pgfqpoint{1.920899in}{2.093466in}}%
\pgfpathcurveto{\pgfqpoint{1.926723in}{2.087642in}}{\pgfqpoint{1.934623in}{2.084369in}}{\pgfqpoint{1.942859in}{2.084369in}}%
\pgfpathclose%
\pgfusepath{stroke,fill}%
\end{pgfscope}%
\begin{pgfscope}%
\pgfpathrectangle{\pgfqpoint{0.100000in}{0.212622in}}{\pgfqpoint{3.696000in}{3.696000in}}%
\pgfusepath{clip}%
\pgfsetbuttcap%
\pgfsetroundjoin%
\definecolor{currentfill}{rgb}{0.121569,0.466667,0.705882}%
\pgfsetfillcolor{currentfill}%
\pgfsetfillopacity{0.951832}%
\pgfsetlinewidth{1.003750pt}%
\definecolor{currentstroke}{rgb}{0.121569,0.466667,0.705882}%
\pgfsetstrokecolor{currentstroke}%
\pgfsetstrokeopacity{0.951832}%
\pgfsetdash{}{0pt}%
\pgfpathmoveto{\pgfqpoint{2.524743in}{1.888362in}}%
\pgfpathcurveto{\pgfqpoint{2.532979in}{1.888362in}}{\pgfqpoint{2.540879in}{1.891634in}}{\pgfqpoint{2.546703in}{1.897458in}}%
\pgfpathcurveto{\pgfqpoint{2.552527in}{1.903282in}}{\pgfqpoint{2.555799in}{1.911182in}}{\pgfqpoint{2.555799in}{1.919418in}}%
\pgfpathcurveto{\pgfqpoint{2.555799in}{1.927655in}}{\pgfqpoint{2.552527in}{1.935555in}}{\pgfqpoint{2.546703in}{1.941379in}}%
\pgfpathcurveto{\pgfqpoint{2.540879in}{1.947202in}}{\pgfqpoint{2.532979in}{1.950475in}}{\pgfqpoint{2.524743in}{1.950475in}}%
\pgfpathcurveto{\pgfqpoint{2.516506in}{1.950475in}}{\pgfqpoint{2.508606in}{1.947202in}}{\pgfqpoint{2.502782in}{1.941379in}}%
\pgfpathcurveto{\pgfqpoint{2.496958in}{1.935555in}}{\pgfqpoint{2.493686in}{1.927655in}}{\pgfqpoint{2.493686in}{1.919418in}}%
\pgfpathcurveto{\pgfqpoint{2.493686in}{1.911182in}}{\pgfqpoint{2.496958in}{1.903282in}}{\pgfqpoint{2.502782in}{1.897458in}}%
\pgfpathcurveto{\pgfqpoint{2.508606in}{1.891634in}}{\pgfqpoint{2.516506in}{1.888362in}}{\pgfqpoint{2.524743in}{1.888362in}}%
\pgfpathclose%
\pgfusepath{stroke,fill}%
\end{pgfscope}%
\begin{pgfscope}%
\pgfpathrectangle{\pgfqpoint{0.100000in}{0.212622in}}{\pgfqpoint{3.696000in}{3.696000in}}%
\pgfusepath{clip}%
\pgfsetbuttcap%
\pgfsetroundjoin%
\definecolor{currentfill}{rgb}{0.121569,0.466667,0.705882}%
\pgfsetfillcolor{currentfill}%
\pgfsetfillopacity{0.952818}%
\pgfsetlinewidth{1.003750pt}%
\definecolor{currentstroke}{rgb}{0.121569,0.466667,0.705882}%
\pgfsetstrokecolor{currentstroke}%
\pgfsetstrokeopacity{0.952818}%
\pgfsetdash{}{0pt}%
\pgfpathmoveto{\pgfqpoint{1.958461in}{2.079686in}}%
\pgfpathcurveto{\pgfqpoint{1.966697in}{2.079686in}}{\pgfqpoint{1.974598in}{2.082959in}}{\pgfqpoint{1.980421in}{2.088783in}}%
\pgfpathcurveto{\pgfqpoint{1.986245in}{2.094607in}}{\pgfqpoint{1.989518in}{2.102507in}}{\pgfqpoint{1.989518in}{2.110743in}}%
\pgfpathcurveto{\pgfqpoint{1.989518in}{2.118979in}}{\pgfqpoint{1.986245in}{2.126879in}}{\pgfqpoint{1.980421in}{2.132703in}}%
\pgfpathcurveto{\pgfqpoint{1.974598in}{2.138527in}}{\pgfqpoint{1.966697in}{2.141799in}}{\pgfqpoint{1.958461in}{2.141799in}}%
\pgfpathcurveto{\pgfqpoint{1.950225in}{2.141799in}}{\pgfqpoint{1.942325in}{2.138527in}}{\pgfqpoint{1.936501in}{2.132703in}}%
\pgfpathcurveto{\pgfqpoint{1.930677in}{2.126879in}}{\pgfqpoint{1.927405in}{2.118979in}}{\pgfqpoint{1.927405in}{2.110743in}}%
\pgfpathcurveto{\pgfqpoint{1.927405in}{2.102507in}}{\pgfqpoint{1.930677in}{2.094607in}}{\pgfqpoint{1.936501in}{2.088783in}}%
\pgfpathcurveto{\pgfqpoint{1.942325in}{2.082959in}}{\pgfqpoint{1.950225in}{2.079686in}}{\pgfqpoint{1.958461in}{2.079686in}}%
\pgfpathclose%
\pgfusepath{stroke,fill}%
\end{pgfscope}%
\begin{pgfscope}%
\pgfpathrectangle{\pgfqpoint{0.100000in}{0.212622in}}{\pgfqpoint{3.696000in}{3.696000in}}%
\pgfusepath{clip}%
\pgfsetbuttcap%
\pgfsetroundjoin%
\definecolor{currentfill}{rgb}{0.121569,0.466667,0.705882}%
\pgfsetfillcolor{currentfill}%
\pgfsetfillopacity{0.954078}%
\pgfsetlinewidth{1.003750pt}%
\definecolor{currentstroke}{rgb}{0.121569,0.466667,0.705882}%
\pgfsetstrokecolor{currentstroke}%
\pgfsetstrokeopacity{0.954078}%
\pgfsetdash{}{0pt}%
\pgfpathmoveto{\pgfqpoint{2.521745in}{1.888594in}}%
\pgfpathcurveto{\pgfqpoint{2.529981in}{1.888594in}}{\pgfqpoint{2.537881in}{1.891867in}}{\pgfqpoint{2.543705in}{1.897691in}}%
\pgfpathcurveto{\pgfqpoint{2.549529in}{1.903515in}}{\pgfqpoint{2.552801in}{1.911415in}}{\pgfqpoint{2.552801in}{1.919651in}}%
\pgfpathcurveto{\pgfqpoint{2.552801in}{1.927887in}}{\pgfqpoint{2.549529in}{1.935787in}}{\pgfqpoint{2.543705in}{1.941611in}}%
\pgfpathcurveto{\pgfqpoint{2.537881in}{1.947435in}}{\pgfqpoint{2.529981in}{1.950707in}}{\pgfqpoint{2.521745in}{1.950707in}}%
\pgfpathcurveto{\pgfqpoint{2.513508in}{1.950707in}}{\pgfqpoint{2.505608in}{1.947435in}}{\pgfqpoint{2.499784in}{1.941611in}}%
\pgfpathcurveto{\pgfqpoint{2.493960in}{1.935787in}}{\pgfqpoint{2.490688in}{1.927887in}}{\pgfqpoint{2.490688in}{1.919651in}}%
\pgfpathcurveto{\pgfqpoint{2.490688in}{1.911415in}}{\pgfqpoint{2.493960in}{1.903515in}}{\pgfqpoint{2.499784in}{1.897691in}}%
\pgfpathcurveto{\pgfqpoint{2.505608in}{1.891867in}}{\pgfqpoint{2.513508in}{1.888594in}}{\pgfqpoint{2.521745in}{1.888594in}}%
\pgfpathclose%
\pgfusepath{stroke,fill}%
\end{pgfscope}%
\begin{pgfscope}%
\pgfpathrectangle{\pgfqpoint{0.100000in}{0.212622in}}{\pgfqpoint{3.696000in}{3.696000in}}%
\pgfusepath{clip}%
\pgfsetbuttcap%
\pgfsetroundjoin%
\definecolor{currentfill}{rgb}{0.121569,0.466667,0.705882}%
\pgfsetfillcolor{currentfill}%
\pgfsetfillopacity{0.954233}%
\pgfsetlinewidth{1.003750pt}%
\definecolor{currentstroke}{rgb}{0.121569,0.466667,0.705882}%
\pgfsetstrokecolor{currentstroke}%
\pgfsetstrokeopacity{0.954233}%
\pgfsetdash{}{0pt}%
\pgfpathmoveto{\pgfqpoint{1.970514in}{2.073718in}}%
\pgfpathcurveto{\pgfqpoint{1.978750in}{2.073718in}}{\pgfqpoint{1.986650in}{2.076990in}}{\pgfqpoint{1.992474in}{2.082814in}}%
\pgfpathcurveto{\pgfqpoint{1.998298in}{2.088638in}}{\pgfqpoint{2.001570in}{2.096538in}}{\pgfqpoint{2.001570in}{2.104774in}}%
\pgfpathcurveto{\pgfqpoint{2.001570in}{2.113011in}}{\pgfqpoint{1.998298in}{2.120911in}}{\pgfqpoint{1.992474in}{2.126735in}}%
\pgfpathcurveto{\pgfqpoint{1.986650in}{2.132559in}}{\pgfqpoint{1.978750in}{2.135831in}}{\pgfqpoint{1.970514in}{2.135831in}}%
\pgfpathcurveto{\pgfqpoint{1.962277in}{2.135831in}}{\pgfqpoint{1.954377in}{2.132559in}}{\pgfqpoint{1.948553in}{2.126735in}}%
\pgfpathcurveto{\pgfqpoint{1.942729in}{2.120911in}}{\pgfqpoint{1.939457in}{2.113011in}}{\pgfqpoint{1.939457in}{2.104774in}}%
\pgfpathcurveto{\pgfqpoint{1.939457in}{2.096538in}}{\pgfqpoint{1.942729in}{2.088638in}}{\pgfqpoint{1.948553in}{2.082814in}}%
\pgfpathcurveto{\pgfqpoint{1.954377in}{2.076990in}}{\pgfqpoint{1.962277in}{2.073718in}}{\pgfqpoint{1.970514in}{2.073718in}}%
\pgfpathclose%
\pgfusepath{stroke,fill}%
\end{pgfscope}%
\begin{pgfscope}%
\pgfpathrectangle{\pgfqpoint{0.100000in}{0.212622in}}{\pgfqpoint{3.696000in}{3.696000in}}%
\pgfusepath{clip}%
\pgfsetbuttcap%
\pgfsetroundjoin%
\definecolor{currentfill}{rgb}{0.121569,0.466667,0.705882}%
\pgfsetfillcolor{currentfill}%
\pgfsetfillopacity{0.955193}%
\pgfsetlinewidth{1.003750pt}%
\definecolor{currentstroke}{rgb}{0.121569,0.466667,0.705882}%
\pgfsetstrokecolor{currentstroke}%
\pgfsetstrokeopacity{0.955193}%
\pgfsetdash{}{0pt}%
\pgfpathmoveto{\pgfqpoint{1.982555in}{2.069605in}}%
\pgfpathcurveto{\pgfqpoint{1.990791in}{2.069605in}}{\pgfqpoint{1.998691in}{2.072878in}}{\pgfqpoint{2.004515in}{2.078701in}}%
\pgfpathcurveto{\pgfqpoint{2.010339in}{2.084525in}}{\pgfqpoint{2.013612in}{2.092425in}}{\pgfqpoint{2.013612in}{2.100662in}}%
\pgfpathcurveto{\pgfqpoint{2.013612in}{2.108898in}}{\pgfqpoint{2.010339in}{2.116798in}}{\pgfqpoint{2.004515in}{2.122622in}}%
\pgfpathcurveto{\pgfqpoint{1.998691in}{2.128446in}}{\pgfqpoint{1.990791in}{2.131718in}}{\pgfqpoint{1.982555in}{2.131718in}}%
\pgfpathcurveto{\pgfqpoint{1.974319in}{2.131718in}}{\pgfqpoint{1.966419in}{2.128446in}}{\pgfqpoint{1.960595in}{2.122622in}}%
\pgfpathcurveto{\pgfqpoint{1.954771in}{2.116798in}}{\pgfqpoint{1.951499in}{2.108898in}}{\pgfqpoint{1.951499in}{2.100662in}}%
\pgfpathcurveto{\pgfqpoint{1.951499in}{2.092425in}}{\pgfqpoint{1.954771in}{2.084525in}}{\pgfqpoint{1.960595in}{2.078701in}}%
\pgfpathcurveto{\pgfqpoint{1.966419in}{2.072878in}}{\pgfqpoint{1.974319in}{2.069605in}}{\pgfqpoint{1.982555in}{2.069605in}}%
\pgfpathclose%
\pgfusepath{stroke,fill}%
\end{pgfscope}%
\begin{pgfscope}%
\pgfpathrectangle{\pgfqpoint{0.100000in}{0.212622in}}{\pgfqpoint{3.696000in}{3.696000in}}%
\pgfusepath{clip}%
\pgfsetbuttcap%
\pgfsetroundjoin%
\definecolor{currentfill}{rgb}{0.121569,0.466667,0.705882}%
\pgfsetfillcolor{currentfill}%
\pgfsetfillopacity{0.955281}%
\pgfsetlinewidth{1.003750pt}%
\definecolor{currentstroke}{rgb}{0.121569,0.466667,0.705882}%
\pgfsetstrokecolor{currentstroke}%
\pgfsetstrokeopacity{0.955281}%
\pgfsetdash{}{0pt}%
\pgfpathmoveto{\pgfqpoint{2.519816in}{1.888750in}}%
\pgfpathcurveto{\pgfqpoint{2.528052in}{1.888750in}}{\pgfqpoint{2.535952in}{1.892022in}}{\pgfqpoint{2.541776in}{1.897846in}}%
\pgfpathcurveto{\pgfqpoint{2.547600in}{1.903670in}}{\pgfqpoint{2.550872in}{1.911570in}}{\pgfqpoint{2.550872in}{1.919806in}}%
\pgfpathcurveto{\pgfqpoint{2.550872in}{1.928043in}}{\pgfqpoint{2.547600in}{1.935943in}}{\pgfqpoint{2.541776in}{1.941767in}}%
\pgfpathcurveto{\pgfqpoint{2.535952in}{1.947591in}}{\pgfqpoint{2.528052in}{1.950863in}}{\pgfqpoint{2.519816in}{1.950863in}}%
\pgfpathcurveto{\pgfqpoint{2.511580in}{1.950863in}}{\pgfqpoint{2.503680in}{1.947591in}}{\pgfqpoint{2.497856in}{1.941767in}}%
\pgfpathcurveto{\pgfqpoint{2.492032in}{1.935943in}}{\pgfqpoint{2.488759in}{1.928043in}}{\pgfqpoint{2.488759in}{1.919806in}}%
\pgfpathcurveto{\pgfqpoint{2.488759in}{1.911570in}}{\pgfqpoint{2.492032in}{1.903670in}}{\pgfqpoint{2.497856in}{1.897846in}}%
\pgfpathcurveto{\pgfqpoint{2.503680in}{1.892022in}}{\pgfqpoint{2.511580in}{1.888750in}}{\pgfqpoint{2.519816in}{1.888750in}}%
\pgfpathclose%
\pgfusepath{stroke,fill}%
\end{pgfscope}%
\begin{pgfscope}%
\pgfpathrectangle{\pgfqpoint{0.100000in}{0.212622in}}{\pgfqpoint{3.696000in}{3.696000in}}%
\pgfusepath{clip}%
\pgfsetbuttcap%
\pgfsetroundjoin%
\definecolor{currentfill}{rgb}{0.121569,0.466667,0.705882}%
\pgfsetfillcolor{currentfill}%
\pgfsetfillopacity{0.955886}%
\pgfsetlinewidth{1.003750pt}%
\definecolor{currentstroke}{rgb}{0.121569,0.466667,0.705882}%
\pgfsetstrokecolor{currentstroke}%
\pgfsetstrokeopacity{0.955886}%
\pgfsetdash{}{0pt}%
\pgfpathmoveto{\pgfqpoint{2.518222in}{1.889160in}}%
\pgfpathcurveto{\pgfqpoint{2.526458in}{1.889160in}}{\pgfqpoint{2.534358in}{1.892433in}}{\pgfqpoint{2.540182in}{1.898257in}}%
\pgfpathcurveto{\pgfqpoint{2.546006in}{1.904080in}}{\pgfqpoint{2.549278in}{1.911981in}}{\pgfqpoint{2.549278in}{1.920217in}}%
\pgfpathcurveto{\pgfqpoint{2.549278in}{1.928453in}}{\pgfqpoint{2.546006in}{1.936353in}}{\pgfqpoint{2.540182in}{1.942177in}}%
\pgfpathcurveto{\pgfqpoint{2.534358in}{1.948001in}}{\pgfqpoint{2.526458in}{1.951273in}}{\pgfqpoint{2.518222in}{1.951273in}}%
\pgfpathcurveto{\pgfqpoint{2.509986in}{1.951273in}}{\pgfqpoint{2.502086in}{1.948001in}}{\pgfqpoint{2.496262in}{1.942177in}}%
\pgfpathcurveto{\pgfqpoint{2.490438in}{1.936353in}}{\pgfqpoint{2.487165in}{1.928453in}}{\pgfqpoint{2.487165in}{1.920217in}}%
\pgfpathcurveto{\pgfqpoint{2.487165in}{1.911981in}}{\pgfqpoint{2.490438in}{1.904080in}}{\pgfqpoint{2.496262in}{1.898257in}}%
\pgfpathcurveto{\pgfqpoint{2.502086in}{1.892433in}}{\pgfqpoint{2.509986in}{1.889160in}}{\pgfqpoint{2.518222in}{1.889160in}}%
\pgfpathclose%
\pgfusepath{stroke,fill}%
\end{pgfscope}%
\begin{pgfscope}%
\pgfpathrectangle{\pgfqpoint{0.100000in}{0.212622in}}{\pgfqpoint{3.696000in}{3.696000in}}%
\pgfusepath{clip}%
\pgfsetbuttcap%
\pgfsetroundjoin%
\definecolor{currentfill}{rgb}{0.121569,0.466667,0.705882}%
\pgfsetfillcolor{currentfill}%
\pgfsetfillopacity{0.955934}%
\pgfsetlinewidth{1.003750pt}%
\definecolor{currentstroke}{rgb}{0.121569,0.466667,0.705882}%
\pgfsetstrokecolor{currentstroke}%
\pgfsetstrokeopacity{0.955934}%
\pgfsetdash{}{0pt}%
\pgfpathmoveto{\pgfqpoint{1.994442in}{2.065138in}}%
\pgfpathcurveto{\pgfqpoint{2.002678in}{2.065138in}}{\pgfqpoint{2.010578in}{2.068410in}}{\pgfqpoint{2.016402in}{2.074234in}}%
\pgfpathcurveto{\pgfqpoint{2.022226in}{2.080058in}}{\pgfqpoint{2.025498in}{2.087958in}}{\pgfqpoint{2.025498in}{2.096195in}}%
\pgfpathcurveto{\pgfqpoint{2.025498in}{2.104431in}}{\pgfqpoint{2.022226in}{2.112331in}}{\pgfqpoint{2.016402in}{2.118155in}}%
\pgfpathcurveto{\pgfqpoint{2.010578in}{2.123979in}}{\pgfqpoint{2.002678in}{2.127251in}}{\pgfqpoint{1.994442in}{2.127251in}}%
\pgfpathcurveto{\pgfqpoint{1.986206in}{2.127251in}}{\pgfqpoint{1.978306in}{2.123979in}}{\pgfqpoint{1.972482in}{2.118155in}}%
\pgfpathcurveto{\pgfqpoint{1.966658in}{2.112331in}}{\pgfqpoint{1.963385in}{2.104431in}}{\pgfqpoint{1.963385in}{2.096195in}}%
\pgfpathcurveto{\pgfqpoint{1.963385in}{2.087958in}}{\pgfqpoint{1.966658in}{2.080058in}}{\pgfqpoint{1.972482in}{2.074234in}}%
\pgfpathcurveto{\pgfqpoint{1.978306in}{2.068410in}}{\pgfqpoint{1.986206in}{2.065138in}}{\pgfqpoint{1.994442in}{2.065138in}}%
\pgfpathclose%
\pgfusepath{stroke,fill}%
\end{pgfscope}%
\begin{pgfscope}%
\pgfpathrectangle{\pgfqpoint{0.100000in}{0.212622in}}{\pgfqpoint{3.696000in}{3.696000in}}%
\pgfusepath{clip}%
\pgfsetbuttcap%
\pgfsetroundjoin%
\definecolor{currentfill}{rgb}{0.121569,0.466667,0.705882}%
\pgfsetfillcolor{currentfill}%
\pgfsetfillopacity{0.957044}%
\pgfsetlinewidth{1.003750pt}%
\definecolor{currentstroke}{rgb}{0.121569,0.466667,0.705882}%
\pgfsetstrokecolor{currentstroke}%
\pgfsetstrokeopacity{0.957044}%
\pgfsetdash{}{0pt}%
\pgfpathmoveto{\pgfqpoint{2.001760in}{2.060658in}}%
\pgfpathcurveto{\pgfqpoint{2.009996in}{2.060658in}}{\pgfqpoint{2.017896in}{2.063930in}}{\pgfqpoint{2.023720in}{2.069754in}}%
\pgfpathcurveto{\pgfqpoint{2.029544in}{2.075578in}}{\pgfqpoint{2.032816in}{2.083478in}}{\pgfqpoint{2.032816in}{2.091715in}}%
\pgfpathcurveto{\pgfqpoint{2.032816in}{2.099951in}}{\pgfqpoint{2.029544in}{2.107851in}}{\pgfqpoint{2.023720in}{2.113675in}}%
\pgfpathcurveto{\pgfqpoint{2.017896in}{2.119499in}}{\pgfqpoint{2.009996in}{2.122771in}}{\pgfqpoint{2.001760in}{2.122771in}}%
\pgfpathcurveto{\pgfqpoint{1.993523in}{2.122771in}}{\pgfqpoint{1.985623in}{2.119499in}}{\pgfqpoint{1.979799in}{2.113675in}}%
\pgfpathcurveto{\pgfqpoint{1.973976in}{2.107851in}}{\pgfqpoint{1.970703in}{2.099951in}}{\pgfqpoint{1.970703in}{2.091715in}}%
\pgfpathcurveto{\pgfqpoint{1.970703in}{2.083478in}}{\pgfqpoint{1.973976in}{2.075578in}}{\pgfqpoint{1.979799in}{2.069754in}}%
\pgfpathcurveto{\pgfqpoint{1.985623in}{2.063930in}}{\pgfqpoint{1.993523in}{2.060658in}}{\pgfqpoint{2.001760in}{2.060658in}}%
\pgfpathclose%
\pgfusepath{stroke,fill}%
\end{pgfscope}%
\begin{pgfscope}%
\pgfpathrectangle{\pgfqpoint{0.100000in}{0.212622in}}{\pgfqpoint{3.696000in}{3.696000in}}%
\pgfusepath{clip}%
\pgfsetbuttcap%
\pgfsetroundjoin%
\definecolor{currentfill}{rgb}{0.121569,0.466667,0.705882}%
\pgfsetfillcolor{currentfill}%
\pgfsetfillopacity{0.957112}%
\pgfsetlinewidth{1.003750pt}%
\definecolor{currentstroke}{rgb}{0.121569,0.466667,0.705882}%
\pgfsetstrokecolor{currentstroke}%
\pgfsetstrokeopacity{0.957112}%
\pgfsetdash{}{0pt}%
\pgfpathmoveto{\pgfqpoint{2.516001in}{1.889343in}}%
\pgfpathcurveto{\pgfqpoint{2.524237in}{1.889343in}}{\pgfqpoint{2.532137in}{1.892615in}}{\pgfqpoint{2.537961in}{1.898439in}}%
\pgfpathcurveto{\pgfqpoint{2.543785in}{1.904263in}}{\pgfqpoint{2.547057in}{1.912163in}}{\pgfqpoint{2.547057in}{1.920400in}}%
\pgfpathcurveto{\pgfqpoint{2.547057in}{1.928636in}}{\pgfqpoint{2.543785in}{1.936536in}}{\pgfqpoint{2.537961in}{1.942360in}}%
\pgfpathcurveto{\pgfqpoint{2.532137in}{1.948184in}}{\pgfqpoint{2.524237in}{1.951456in}}{\pgfqpoint{2.516001in}{1.951456in}}%
\pgfpathcurveto{\pgfqpoint{2.507764in}{1.951456in}}{\pgfqpoint{2.499864in}{1.948184in}}{\pgfqpoint{2.494040in}{1.942360in}}%
\pgfpathcurveto{\pgfqpoint{2.488217in}{1.936536in}}{\pgfqpoint{2.484944in}{1.928636in}}{\pgfqpoint{2.484944in}{1.920400in}}%
\pgfpathcurveto{\pgfqpoint{2.484944in}{1.912163in}}{\pgfqpoint{2.488217in}{1.904263in}}{\pgfqpoint{2.494040in}{1.898439in}}%
\pgfpathcurveto{\pgfqpoint{2.499864in}{1.892615in}}{\pgfqpoint{2.507764in}{1.889343in}}{\pgfqpoint{2.516001in}{1.889343in}}%
\pgfpathclose%
\pgfusepath{stroke,fill}%
\end{pgfscope}%
\begin{pgfscope}%
\pgfpathrectangle{\pgfqpoint{0.100000in}{0.212622in}}{\pgfqpoint{3.696000in}{3.696000in}}%
\pgfusepath{clip}%
\pgfsetbuttcap%
\pgfsetroundjoin%
\definecolor{currentfill}{rgb}{0.121569,0.466667,0.705882}%
\pgfsetfillcolor{currentfill}%
\pgfsetfillopacity{0.957793}%
\pgfsetlinewidth{1.003750pt}%
\definecolor{currentstroke}{rgb}{0.121569,0.466667,0.705882}%
\pgfsetstrokecolor{currentstroke}%
\pgfsetstrokeopacity{0.957793}%
\pgfsetdash{}{0pt}%
\pgfpathmoveto{\pgfqpoint{2.009780in}{2.058287in}}%
\pgfpathcurveto{\pgfqpoint{2.018017in}{2.058287in}}{\pgfqpoint{2.025917in}{2.061560in}}{\pgfqpoint{2.031741in}{2.067384in}}%
\pgfpathcurveto{\pgfqpoint{2.037565in}{2.073208in}}{\pgfqpoint{2.040837in}{2.081108in}}{\pgfqpoint{2.040837in}{2.089344in}}%
\pgfpathcurveto{\pgfqpoint{2.040837in}{2.097580in}}{\pgfqpoint{2.037565in}{2.105480in}}{\pgfqpoint{2.031741in}{2.111304in}}%
\pgfpathcurveto{\pgfqpoint{2.025917in}{2.117128in}}{\pgfqpoint{2.018017in}{2.120400in}}{\pgfqpoint{2.009780in}{2.120400in}}%
\pgfpathcurveto{\pgfqpoint{2.001544in}{2.120400in}}{\pgfqpoint{1.993644in}{2.117128in}}{\pgfqpoint{1.987820in}{2.111304in}}%
\pgfpathcurveto{\pgfqpoint{1.981996in}{2.105480in}}{\pgfqpoint{1.978724in}{2.097580in}}{\pgfqpoint{1.978724in}{2.089344in}}%
\pgfpathcurveto{\pgfqpoint{1.978724in}{2.081108in}}{\pgfqpoint{1.981996in}{2.073208in}}{\pgfqpoint{1.987820in}{2.067384in}}%
\pgfpathcurveto{\pgfqpoint{1.993644in}{2.061560in}}{\pgfqpoint{2.001544in}{2.058287in}}{\pgfqpoint{2.009780in}{2.058287in}}%
\pgfpathclose%
\pgfusepath{stroke,fill}%
\end{pgfscope}%
\begin{pgfscope}%
\pgfpathrectangle{\pgfqpoint{0.100000in}{0.212622in}}{\pgfqpoint{3.696000in}{3.696000in}}%
\pgfusepath{clip}%
\pgfsetbuttcap%
\pgfsetroundjoin%
\definecolor{currentfill}{rgb}{0.121569,0.466667,0.705882}%
\pgfsetfillcolor{currentfill}%
\pgfsetfillopacity{0.958679}%
\pgfsetlinewidth{1.003750pt}%
\definecolor{currentstroke}{rgb}{0.121569,0.466667,0.705882}%
\pgfsetstrokecolor{currentstroke}%
\pgfsetstrokeopacity{0.958679}%
\pgfsetdash{}{0pt}%
\pgfpathmoveto{\pgfqpoint{2.017158in}{2.056310in}}%
\pgfpathcurveto{\pgfqpoint{2.025394in}{2.056310in}}{\pgfqpoint{2.033294in}{2.059583in}}{\pgfqpoint{2.039118in}{2.065407in}}%
\pgfpathcurveto{\pgfqpoint{2.044942in}{2.071231in}}{\pgfqpoint{2.048214in}{2.079131in}}{\pgfqpoint{2.048214in}{2.087367in}}%
\pgfpathcurveto{\pgfqpoint{2.048214in}{2.095603in}}{\pgfqpoint{2.044942in}{2.103503in}}{\pgfqpoint{2.039118in}{2.109327in}}%
\pgfpathcurveto{\pgfqpoint{2.033294in}{2.115151in}}{\pgfqpoint{2.025394in}{2.118423in}}{\pgfqpoint{2.017158in}{2.118423in}}%
\pgfpathcurveto{\pgfqpoint{2.008921in}{2.118423in}}{\pgfqpoint{2.001021in}{2.115151in}}{\pgfqpoint{1.995197in}{2.109327in}}%
\pgfpathcurveto{\pgfqpoint{1.989373in}{2.103503in}}{\pgfqpoint{1.986101in}{2.095603in}}{\pgfqpoint{1.986101in}{2.087367in}}%
\pgfpathcurveto{\pgfqpoint{1.986101in}{2.079131in}}{\pgfqpoint{1.989373in}{2.071231in}}{\pgfqpoint{1.995197in}{2.065407in}}%
\pgfpathcurveto{\pgfqpoint{2.001021in}{2.059583in}}{\pgfqpoint{2.008921in}{2.056310in}}{\pgfqpoint{2.017158in}{2.056310in}}%
\pgfpathclose%
\pgfusepath{stroke,fill}%
\end{pgfscope}%
\begin{pgfscope}%
\pgfpathrectangle{\pgfqpoint{0.100000in}{0.212622in}}{\pgfqpoint{3.696000in}{3.696000in}}%
\pgfusepath{clip}%
\pgfsetbuttcap%
\pgfsetroundjoin%
\definecolor{currentfill}{rgb}{0.121569,0.466667,0.705882}%
\pgfsetfillcolor{currentfill}%
\pgfsetfillopacity{0.958689}%
\pgfsetlinewidth{1.003750pt}%
\definecolor{currentstroke}{rgb}{0.121569,0.466667,0.705882}%
\pgfsetstrokecolor{currentstroke}%
\pgfsetstrokeopacity{0.958689}%
\pgfsetdash{}{0pt}%
\pgfpathmoveto{\pgfqpoint{2.513945in}{1.889536in}}%
\pgfpathcurveto{\pgfqpoint{2.522181in}{1.889536in}}{\pgfqpoint{2.530081in}{1.892809in}}{\pgfqpoint{2.535905in}{1.898633in}}%
\pgfpathcurveto{\pgfqpoint{2.541729in}{1.904457in}}{\pgfqpoint{2.545002in}{1.912357in}}{\pgfqpoint{2.545002in}{1.920593in}}%
\pgfpathcurveto{\pgfqpoint{2.545002in}{1.928829in}}{\pgfqpoint{2.541729in}{1.936729in}}{\pgfqpoint{2.535905in}{1.942553in}}%
\pgfpathcurveto{\pgfqpoint{2.530081in}{1.948377in}}{\pgfqpoint{2.522181in}{1.951649in}}{\pgfqpoint{2.513945in}{1.951649in}}%
\pgfpathcurveto{\pgfqpoint{2.505709in}{1.951649in}}{\pgfqpoint{2.497809in}{1.948377in}}{\pgfqpoint{2.491985in}{1.942553in}}%
\pgfpathcurveto{\pgfqpoint{2.486161in}{1.936729in}}{\pgfqpoint{2.482889in}{1.928829in}}{\pgfqpoint{2.482889in}{1.920593in}}%
\pgfpathcurveto{\pgfqpoint{2.482889in}{1.912357in}}{\pgfqpoint{2.486161in}{1.904457in}}{\pgfqpoint{2.491985in}{1.898633in}}%
\pgfpathcurveto{\pgfqpoint{2.497809in}{1.892809in}}{\pgfqpoint{2.505709in}{1.889536in}}{\pgfqpoint{2.513945in}{1.889536in}}%
\pgfpathclose%
\pgfusepath{stroke,fill}%
\end{pgfscope}%
\begin{pgfscope}%
\pgfpathrectangle{\pgfqpoint{0.100000in}{0.212622in}}{\pgfqpoint{3.696000in}{3.696000in}}%
\pgfusepath{clip}%
\pgfsetbuttcap%
\pgfsetroundjoin%
\definecolor{currentfill}{rgb}{0.121569,0.466667,0.705882}%
\pgfsetfillcolor{currentfill}%
\pgfsetfillopacity{0.959341}%
\pgfsetlinewidth{1.003750pt}%
\definecolor{currentstroke}{rgb}{0.121569,0.466667,0.705882}%
\pgfsetstrokecolor{currentstroke}%
\pgfsetstrokeopacity{0.959341}%
\pgfsetdash{}{0pt}%
\pgfpathmoveto{\pgfqpoint{2.022028in}{2.055572in}}%
\pgfpathcurveto{\pgfqpoint{2.030265in}{2.055572in}}{\pgfqpoint{2.038165in}{2.058844in}}{\pgfqpoint{2.043989in}{2.064668in}}%
\pgfpathcurveto{\pgfqpoint{2.049812in}{2.070492in}}{\pgfqpoint{2.053085in}{2.078392in}}{\pgfqpoint{2.053085in}{2.086628in}}%
\pgfpathcurveto{\pgfqpoint{2.053085in}{2.094864in}}{\pgfqpoint{2.049812in}{2.102764in}}{\pgfqpoint{2.043989in}{2.108588in}}%
\pgfpathcurveto{\pgfqpoint{2.038165in}{2.114412in}}{\pgfqpoint{2.030265in}{2.117685in}}{\pgfqpoint{2.022028in}{2.117685in}}%
\pgfpathcurveto{\pgfqpoint{2.013792in}{2.117685in}}{\pgfqpoint{2.005892in}{2.114412in}}{\pgfqpoint{2.000068in}{2.108588in}}%
\pgfpathcurveto{\pgfqpoint{1.994244in}{2.102764in}}{\pgfqpoint{1.990972in}{2.094864in}}{\pgfqpoint{1.990972in}{2.086628in}}%
\pgfpathcurveto{\pgfqpoint{1.990972in}{2.078392in}}{\pgfqpoint{1.994244in}{2.070492in}}{\pgfqpoint{2.000068in}{2.064668in}}%
\pgfpathcurveto{\pgfqpoint{2.005892in}{2.058844in}}{\pgfqpoint{2.013792in}{2.055572in}}{\pgfqpoint{2.022028in}{2.055572in}}%
\pgfpathclose%
\pgfusepath{stroke,fill}%
\end{pgfscope}%
\begin{pgfscope}%
\pgfpathrectangle{\pgfqpoint{0.100000in}{0.212622in}}{\pgfqpoint{3.696000in}{3.696000in}}%
\pgfusepath{clip}%
\pgfsetbuttcap%
\pgfsetroundjoin%
\definecolor{currentfill}{rgb}{0.121569,0.466667,0.705882}%
\pgfsetfillcolor{currentfill}%
\pgfsetfillopacity{0.960331}%
\pgfsetlinewidth{1.003750pt}%
\definecolor{currentstroke}{rgb}{0.121569,0.466667,0.705882}%
\pgfsetstrokecolor{currentstroke}%
\pgfsetstrokeopacity{0.960331}%
\pgfsetdash{}{0pt}%
\pgfpathmoveto{\pgfqpoint{2.030942in}{2.052944in}}%
\pgfpathcurveto{\pgfqpoint{2.039179in}{2.052944in}}{\pgfqpoint{2.047079in}{2.056216in}}{\pgfqpoint{2.052903in}{2.062040in}}%
\pgfpathcurveto{\pgfqpoint{2.058727in}{2.067864in}}{\pgfqpoint{2.061999in}{2.075764in}}{\pgfqpoint{2.061999in}{2.084000in}}%
\pgfpathcurveto{\pgfqpoint{2.061999in}{2.092236in}}{\pgfqpoint{2.058727in}{2.100136in}}{\pgfqpoint{2.052903in}{2.105960in}}%
\pgfpathcurveto{\pgfqpoint{2.047079in}{2.111784in}}{\pgfqpoint{2.039179in}{2.115057in}}{\pgfqpoint{2.030942in}{2.115057in}}%
\pgfpathcurveto{\pgfqpoint{2.022706in}{2.115057in}}{\pgfqpoint{2.014806in}{2.111784in}}{\pgfqpoint{2.008982in}{2.105960in}}%
\pgfpathcurveto{\pgfqpoint{2.003158in}{2.100136in}}{\pgfqpoint{1.999886in}{2.092236in}}{\pgfqpoint{1.999886in}{2.084000in}}%
\pgfpathcurveto{\pgfqpoint{1.999886in}{2.075764in}}{\pgfqpoint{2.003158in}{2.067864in}}{\pgfqpoint{2.008982in}{2.062040in}}%
\pgfpathcurveto{\pgfqpoint{2.014806in}{2.056216in}}{\pgfqpoint{2.022706in}{2.052944in}}{\pgfqpoint{2.030942in}{2.052944in}}%
\pgfpathclose%
\pgfusepath{stroke,fill}%
\end{pgfscope}%
\begin{pgfscope}%
\pgfpathrectangle{\pgfqpoint{0.100000in}{0.212622in}}{\pgfqpoint{3.696000in}{3.696000in}}%
\pgfusepath{clip}%
\pgfsetbuttcap%
\pgfsetroundjoin%
\definecolor{currentfill}{rgb}{0.121569,0.466667,0.705882}%
\pgfsetfillcolor{currentfill}%
\pgfsetfillopacity{0.960488}%
\pgfsetlinewidth{1.003750pt}%
\definecolor{currentstroke}{rgb}{0.121569,0.466667,0.705882}%
\pgfsetstrokecolor{currentstroke}%
\pgfsetstrokeopacity{0.960488}%
\pgfsetdash{}{0pt}%
\pgfpathmoveto{\pgfqpoint{2.510054in}{1.890320in}}%
\pgfpathcurveto{\pgfqpoint{2.518290in}{1.890320in}}{\pgfqpoint{2.526190in}{1.893593in}}{\pgfqpoint{2.532014in}{1.899416in}}%
\pgfpathcurveto{\pgfqpoint{2.537838in}{1.905240in}}{\pgfqpoint{2.541110in}{1.913140in}}{\pgfqpoint{2.541110in}{1.921377in}}%
\pgfpathcurveto{\pgfqpoint{2.541110in}{1.929613in}}{\pgfqpoint{2.537838in}{1.937513in}}{\pgfqpoint{2.532014in}{1.943337in}}%
\pgfpathcurveto{\pgfqpoint{2.526190in}{1.949161in}}{\pgfqpoint{2.518290in}{1.952433in}}{\pgfqpoint{2.510054in}{1.952433in}}%
\pgfpathcurveto{\pgfqpoint{2.501817in}{1.952433in}}{\pgfqpoint{2.493917in}{1.949161in}}{\pgfqpoint{2.488093in}{1.943337in}}%
\pgfpathcurveto{\pgfqpoint{2.482269in}{1.937513in}}{\pgfqpoint{2.478997in}{1.929613in}}{\pgfqpoint{2.478997in}{1.921377in}}%
\pgfpathcurveto{\pgfqpoint{2.478997in}{1.913140in}}{\pgfqpoint{2.482269in}{1.905240in}}{\pgfqpoint{2.488093in}{1.899416in}}%
\pgfpathcurveto{\pgfqpoint{2.493917in}{1.893593in}}{\pgfqpoint{2.501817in}{1.890320in}}{\pgfqpoint{2.510054in}{1.890320in}}%
\pgfpathclose%
\pgfusepath{stroke,fill}%
\end{pgfscope}%
\begin{pgfscope}%
\pgfpathrectangle{\pgfqpoint{0.100000in}{0.212622in}}{\pgfqpoint{3.696000in}{3.696000in}}%
\pgfusepath{clip}%
\pgfsetbuttcap%
\pgfsetroundjoin%
\definecolor{currentfill}{rgb}{0.121569,0.466667,0.705882}%
\pgfsetfillcolor{currentfill}%
\pgfsetfillopacity{0.961231}%
\pgfsetlinewidth{1.003750pt}%
\definecolor{currentstroke}{rgb}{0.121569,0.466667,0.705882}%
\pgfsetstrokecolor{currentstroke}%
\pgfsetstrokeopacity{0.961231}%
\pgfsetdash{}{0pt}%
\pgfpathmoveto{\pgfqpoint{2.039305in}{2.050629in}}%
\pgfpathcurveto{\pgfqpoint{2.047541in}{2.050629in}}{\pgfqpoint{2.055441in}{2.053901in}}{\pgfqpoint{2.061265in}{2.059725in}}%
\pgfpathcurveto{\pgfqpoint{2.067089in}{2.065549in}}{\pgfqpoint{2.070361in}{2.073449in}}{\pgfqpoint{2.070361in}{2.081686in}}%
\pgfpathcurveto{\pgfqpoint{2.070361in}{2.089922in}}{\pgfqpoint{2.067089in}{2.097822in}}{\pgfqpoint{2.061265in}{2.103646in}}%
\pgfpathcurveto{\pgfqpoint{2.055441in}{2.109470in}}{\pgfqpoint{2.047541in}{2.112742in}}{\pgfqpoint{2.039305in}{2.112742in}}%
\pgfpathcurveto{\pgfqpoint{2.031068in}{2.112742in}}{\pgfqpoint{2.023168in}{2.109470in}}{\pgfqpoint{2.017344in}{2.103646in}}%
\pgfpathcurveto{\pgfqpoint{2.011520in}{2.097822in}}{\pgfqpoint{2.008248in}{2.089922in}}{\pgfqpoint{2.008248in}{2.081686in}}%
\pgfpathcurveto{\pgfqpoint{2.008248in}{2.073449in}}{\pgfqpoint{2.011520in}{2.065549in}}{\pgfqpoint{2.017344in}{2.059725in}}%
\pgfpathcurveto{\pgfqpoint{2.023168in}{2.053901in}}{\pgfqpoint{2.031068in}{2.050629in}}{\pgfqpoint{2.039305in}{2.050629in}}%
\pgfpathclose%
\pgfusepath{stroke,fill}%
\end{pgfscope}%
\begin{pgfscope}%
\pgfpathrectangle{\pgfqpoint{0.100000in}{0.212622in}}{\pgfqpoint{3.696000in}{3.696000in}}%
\pgfusepath{clip}%
\pgfsetbuttcap%
\pgfsetroundjoin%
\definecolor{currentfill}{rgb}{0.121569,0.466667,0.705882}%
\pgfsetfillcolor{currentfill}%
\pgfsetfillopacity{0.961861}%
\pgfsetlinewidth{1.003750pt}%
\definecolor{currentstroke}{rgb}{0.121569,0.466667,0.705882}%
\pgfsetstrokecolor{currentstroke}%
\pgfsetstrokeopacity{0.961861}%
\pgfsetdash{}{0pt}%
\pgfpathmoveto{\pgfqpoint{2.045456in}{2.048296in}}%
\pgfpathcurveto{\pgfqpoint{2.053693in}{2.048296in}}{\pgfqpoint{2.061593in}{2.051568in}}{\pgfqpoint{2.067417in}{2.057392in}}%
\pgfpathcurveto{\pgfqpoint{2.073241in}{2.063216in}}{\pgfqpoint{2.076513in}{2.071116in}}{\pgfqpoint{2.076513in}{2.079352in}}%
\pgfpathcurveto{\pgfqpoint{2.076513in}{2.087588in}}{\pgfqpoint{2.073241in}{2.095489in}}{\pgfqpoint{2.067417in}{2.101312in}}%
\pgfpathcurveto{\pgfqpoint{2.061593in}{2.107136in}}{\pgfqpoint{2.053693in}{2.110409in}}{\pgfqpoint{2.045456in}{2.110409in}}%
\pgfpathcurveto{\pgfqpoint{2.037220in}{2.110409in}}{\pgfqpoint{2.029320in}{2.107136in}}{\pgfqpoint{2.023496in}{2.101312in}}%
\pgfpathcurveto{\pgfqpoint{2.017672in}{2.095489in}}{\pgfqpoint{2.014400in}{2.087588in}}{\pgfqpoint{2.014400in}{2.079352in}}%
\pgfpathcurveto{\pgfqpoint{2.014400in}{2.071116in}}{\pgfqpoint{2.017672in}{2.063216in}}{\pgfqpoint{2.023496in}{2.057392in}}%
\pgfpathcurveto{\pgfqpoint{2.029320in}{2.051568in}}{\pgfqpoint{2.037220in}{2.048296in}}{\pgfqpoint{2.045456in}{2.048296in}}%
\pgfpathclose%
\pgfusepath{stroke,fill}%
\end{pgfscope}%
\begin{pgfscope}%
\pgfpathrectangle{\pgfqpoint{0.100000in}{0.212622in}}{\pgfqpoint{3.696000in}{3.696000in}}%
\pgfusepath{clip}%
\pgfsetbuttcap%
\pgfsetroundjoin%
\definecolor{currentfill}{rgb}{0.121569,0.466667,0.705882}%
\pgfsetfillcolor{currentfill}%
\pgfsetfillopacity{0.962285}%
\pgfsetlinewidth{1.003750pt}%
\definecolor{currentstroke}{rgb}{0.121569,0.466667,0.705882}%
\pgfsetstrokecolor{currentstroke}%
\pgfsetstrokeopacity{0.962285}%
\pgfsetdash{}{0pt}%
\pgfpathmoveto{\pgfqpoint{2.505011in}{1.891612in}}%
\pgfpathcurveto{\pgfqpoint{2.513248in}{1.891612in}}{\pgfqpoint{2.521148in}{1.894884in}}{\pgfqpoint{2.526972in}{1.900708in}}%
\pgfpathcurveto{\pgfqpoint{2.532796in}{1.906532in}}{\pgfqpoint{2.536068in}{1.914432in}}{\pgfqpoint{2.536068in}{1.922668in}}%
\pgfpathcurveto{\pgfqpoint{2.536068in}{1.930905in}}{\pgfqpoint{2.532796in}{1.938805in}}{\pgfqpoint{2.526972in}{1.944629in}}%
\pgfpathcurveto{\pgfqpoint{2.521148in}{1.950452in}}{\pgfqpoint{2.513248in}{1.953725in}}{\pgfqpoint{2.505011in}{1.953725in}}%
\pgfpathcurveto{\pgfqpoint{2.496775in}{1.953725in}}{\pgfqpoint{2.488875in}{1.950452in}}{\pgfqpoint{2.483051in}{1.944629in}}%
\pgfpathcurveto{\pgfqpoint{2.477227in}{1.938805in}}{\pgfqpoint{2.473955in}{1.930905in}}{\pgfqpoint{2.473955in}{1.922668in}}%
\pgfpathcurveto{\pgfqpoint{2.473955in}{1.914432in}}{\pgfqpoint{2.477227in}{1.906532in}}{\pgfqpoint{2.483051in}{1.900708in}}%
\pgfpathcurveto{\pgfqpoint{2.488875in}{1.894884in}}{\pgfqpoint{2.496775in}{1.891612in}}{\pgfqpoint{2.505011in}{1.891612in}}%
\pgfpathclose%
\pgfusepath{stroke,fill}%
\end{pgfscope}%
\begin{pgfscope}%
\pgfpathrectangle{\pgfqpoint{0.100000in}{0.212622in}}{\pgfqpoint{3.696000in}{3.696000in}}%
\pgfusepath{clip}%
\pgfsetbuttcap%
\pgfsetroundjoin%
\definecolor{currentfill}{rgb}{0.121569,0.466667,0.705882}%
\pgfsetfillcolor{currentfill}%
\pgfsetfillopacity{0.962966}%
\pgfsetlinewidth{1.003750pt}%
\definecolor{currentstroke}{rgb}{0.121569,0.466667,0.705882}%
\pgfsetstrokecolor{currentstroke}%
\pgfsetstrokeopacity{0.962966}%
\pgfsetdash{}{0pt}%
\pgfpathmoveto{\pgfqpoint{2.056857in}{2.044574in}}%
\pgfpathcurveto{\pgfqpoint{2.065094in}{2.044574in}}{\pgfqpoint{2.072994in}{2.047846in}}{\pgfqpoint{2.078818in}{2.053670in}}%
\pgfpathcurveto{\pgfqpoint{2.084641in}{2.059494in}}{\pgfqpoint{2.087914in}{2.067394in}}{\pgfqpoint{2.087914in}{2.075631in}}%
\pgfpathcurveto{\pgfqpoint{2.087914in}{2.083867in}}{\pgfqpoint{2.084641in}{2.091767in}}{\pgfqpoint{2.078818in}{2.097591in}}%
\pgfpathcurveto{\pgfqpoint{2.072994in}{2.103415in}}{\pgfqpoint{2.065094in}{2.106687in}}{\pgfqpoint{2.056857in}{2.106687in}}%
\pgfpathcurveto{\pgfqpoint{2.048621in}{2.106687in}}{\pgfqpoint{2.040721in}{2.103415in}}{\pgfqpoint{2.034897in}{2.097591in}}%
\pgfpathcurveto{\pgfqpoint{2.029073in}{2.091767in}}{\pgfqpoint{2.025801in}{2.083867in}}{\pgfqpoint{2.025801in}{2.075631in}}%
\pgfpathcurveto{\pgfqpoint{2.025801in}{2.067394in}}{\pgfqpoint{2.029073in}{2.059494in}}{\pgfqpoint{2.034897in}{2.053670in}}%
\pgfpathcurveto{\pgfqpoint{2.040721in}{2.047846in}}{\pgfqpoint{2.048621in}{2.044574in}}{\pgfqpoint{2.056857in}{2.044574in}}%
\pgfpathclose%
\pgfusepath{stroke,fill}%
\end{pgfscope}%
\begin{pgfscope}%
\pgfpathrectangle{\pgfqpoint{0.100000in}{0.212622in}}{\pgfqpoint{3.696000in}{3.696000in}}%
\pgfusepath{clip}%
\pgfsetbuttcap%
\pgfsetroundjoin%
\definecolor{currentfill}{rgb}{0.121569,0.466667,0.705882}%
\pgfsetfillcolor{currentfill}%
\pgfsetfillopacity{0.964040}%
\pgfsetlinewidth{1.003750pt}%
\definecolor{currentstroke}{rgb}{0.121569,0.466667,0.705882}%
\pgfsetstrokecolor{currentstroke}%
\pgfsetstrokeopacity{0.964040}%
\pgfsetdash{}{0pt}%
\pgfpathmoveto{\pgfqpoint{2.067552in}{2.041837in}}%
\pgfpathcurveto{\pgfqpoint{2.075788in}{2.041837in}}{\pgfqpoint{2.083688in}{2.045109in}}{\pgfqpoint{2.089512in}{2.050933in}}%
\pgfpathcurveto{\pgfqpoint{2.095336in}{2.056757in}}{\pgfqpoint{2.098608in}{2.064657in}}{\pgfqpoint{2.098608in}{2.072894in}}%
\pgfpathcurveto{\pgfqpoint{2.098608in}{2.081130in}}{\pgfqpoint{2.095336in}{2.089030in}}{\pgfqpoint{2.089512in}{2.094854in}}%
\pgfpathcurveto{\pgfqpoint{2.083688in}{2.100678in}}{\pgfqpoint{2.075788in}{2.103950in}}{\pgfqpoint{2.067552in}{2.103950in}}%
\pgfpathcurveto{\pgfqpoint{2.059316in}{2.103950in}}{\pgfqpoint{2.051416in}{2.100678in}}{\pgfqpoint{2.045592in}{2.094854in}}%
\pgfpathcurveto{\pgfqpoint{2.039768in}{2.089030in}}{\pgfqpoint{2.036495in}{2.081130in}}{\pgfqpoint{2.036495in}{2.072894in}}%
\pgfpathcurveto{\pgfqpoint{2.036495in}{2.064657in}}{\pgfqpoint{2.039768in}{2.056757in}}{\pgfqpoint{2.045592in}{2.050933in}}%
\pgfpathcurveto{\pgfqpoint{2.051416in}{2.045109in}}{\pgfqpoint{2.059316in}{2.041837in}}{\pgfqpoint{2.067552in}{2.041837in}}%
\pgfpathclose%
\pgfusepath{stroke,fill}%
\end{pgfscope}%
\begin{pgfscope}%
\pgfpathrectangle{\pgfqpoint{0.100000in}{0.212622in}}{\pgfqpoint{3.696000in}{3.696000in}}%
\pgfusepath{clip}%
\pgfsetbuttcap%
\pgfsetroundjoin%
\definecolor{currentfill}{rgb}{0.121569,0.466667,0.705882}%
\pgfsetfillcolor{currentfill}%
\pgfsetfillopacity{0.965030}%
\pgfsetlinewidth{1.003750pt}%
\definecolor{currentstroke}{rgb}{0.121569,0.466667,0.705882}%
\pgfsetstrokecolor{currentstroke}%
\pgfsetstrokeopacity{0.965030}%
\pgfsetdash{}{0pt}%
\pgfpathmoveto{\pgfqpoint{2.075786in}{2.038090in}}%
\pgfpathcurveto{\pgfqpoint{2.084022in}{2.038090in}}{\pgfqpoint{2.091922in}{2.041363in}}{\pgfqpoint{2.097746in}{2.047186in}}%
\pgfpathcurveto{\pgfqpoint{2.103570in}{2.053010in}}{\pgfqpoint{2.106842in}{2.060910in}}{\pgfqpoint{2.106842in}{2.069147in}}%
\pgfpathcurveto{\pgfqpoint{2.106842in}{2.077383in}}{\pgfqpoint{2.103570in}{2.085283in}}{\pgfqpoint{2.097746in}{2.091107in}}%
\pgfpathcurveto{\pgfqpoint{2.091922in}{2.096931in}}{\pgfqpoint{2.084022in}{2.100203in}}{\pgfqpoint{2.075786in}{2.100203in}}%
\pgfpathcurveto{\pgfqpoint{2.067549in}{2.100203in}}{\pgfqpoint{2.059649in}{2.096931in}}{\pgfqpoint{2.053825in}{2.091107in}}%
\pgfpathcurveto{\pgfqpoint{2.048001in}{2.085283in}}{\pgfqpoint{2.044729in}{2.077383in}}{\pgfqpoint{2.044729in}{2.069147in}}%
\pgfpathcurveto{\pgfqpoint{2.044729in}{2.060910in}}{\pgfqpoint{2.048001in}{2.053010in}}{\pgfqpoint{2.053825in}{2.047186in}}%
\pgfpathcurveto{\pgfqpoint{2.059649in}{2.041363in}}{\pgfqpoint{2.067549in}{2.038090in}}{\pgfqpoint{2.075786in}{2.038090in}}%
\pgfpathclose%
\pgfusepath{stroke,fill}%
\end{pgfscope}%
\begin{pgfscope}%
\pgfpathrectangle{\pgfqpoint{0.100000in}{0.212622in}}{\pgfqpoint{3.696000in}{3.696000in}}%
\pgfusepath{clip}%
\pgfsetbuttcap%
\pgfsetroundjoin%
\definecolor{currentfill}{rgb}{0.121569,0.466667,0.705882}%
\pgfsetfillcolor{currentfill}%
\pgfsetfillopacity{0.965211}%
\pgfsetlinewidth{1.003750pt}%
\definecolor{currentstroke}{rgb}{0.121569,0.466667,0.705882}%
\pgfsetstrokecolor{currentstroke}%
\pgfsetstrokeopacity{0.965211}%
\pgfsetdash{}{0pt}%
\pgfpathmoveto{\pgfqpoint{2.501266in}{1.892022in}}%
\pgfpathcurveto{\pgfqpoint{2.509502in}{1.892022in}}{\pgfqpoint{2.517402in}{1.895294in}}{\pgfqpoint{2.523226in}{1.901118in}}%
\pgfpathcurveto{\pgfqpoint{2.529050in}{1.906942in}}{\pgfqpoint{2.532322in}{1.914842in}}{\pgfqpoint{2.532322in}{1.923079in}}%
\pgfpathcurveto{\pgfqpoint{2.532322in}{1.931315in}}{\pgfqpoint{2.529050in}{1.939215in}}{\pgfqpoint{2.523226in}{1.945039in}}%
\pgfpathcurveto{\pgfqpoint{2.517402in}{1.950863in}}{\pgfqpoint{2.509502in}{1.954135in}}{\pgfqpoint{2.501266in}{1.954135in}}%
\pgfpathcurveto{\pgfqpoint{2.493029in}{1.954135in}}{\pgfqpoint{2.485129in}{1.950863in}}{\pgfqpoint{2.479305in}{1.945039in}}%
\pgfpathcurveto{\pgfqpoint{2.473481in}{1.939215in}}{\pgfqpoint{2.470209in}{1.931315in}}{\pgfqpoint{2.470209in}{1.923079in}}%
\pgfpathcurveto{\pgfqpoint{2.470209in}{1.914842in}}{\pgfqpoint{2.473481in}{1.906942in}}{\pgfqpoint{2.479305in}{1.901118in}}%
\pgfpathcurveto{\pgfqpoint{2.485129in}{1.895294in}}{\pgfqpoint{2.493029in}{1.892022in}}{\pgfqpoint{2.501266in}{1.892022in}}%
\pgfpathclose%
\pgfusepath{stroke,fill}%
\end{pgfscope}%
\begin{pgfscope}%
\pgfpathrectangle{\pgfqpoint{0.100000in}{0.212622in}}{\pgfqpoint{3.696000in}{3.696000in}}%
\pgfusepath{clip}%
\pgfsetbuttcap%
\pgfsetroundjoin%
\definecolor{currentfill}{rgb}{0.121569,0.466667,0.705882}%
\pgfsetfillcolor{currentfill}%
\pgfsetfillopacity{0.965458}%
\pgfsetlinewidth{1.003750pt}%
\definecolor{currentstroke}{rgb}{0.121569,0.466667,0.705882}%
\pgfsetstrokecolor{currentstroke}%
\pgfsetstrokeopacity{0.965458}%
\pgfsetdash{}{0pt}%
\pgfpathmoveto{\pgfqpoint{2.084499in}{2.034917in}}%
\pgfpathcurveto{\pgfqpoint{2.092735in}{2.034917in}}{\pgfqpoint{2.100635in}{2.038189in}}{\pgfqpoint{2.106459in}{2.044013in}}%
\pgfpathcurveto{\pgfqpoint{2.112283in}{2.049837in}}{\pgfqpoint{2.115555in}{2.057737in}}{\pgfqpoint{2.115555in}{2.065973in}}%
\pgfpathcurveto{\pgfqpoint{2.115555in}{2.074210in}}{\pgfqpoint{2.112283in}{2.082110in}}{\pgfqpoint{2.106459in}{2.087934in}}%
\pgfpathcurveto{\pgfqpoint{2.100635in}{2.093758in}}{\pgfqpoint{2.092735in}{2.097030in}}{\pgfqpoint{2.084499in}{2.097030in}}%
\pgfpathcurveto{\pgfqpoint{2.076262in}{2.097030in}}{\pgfqpoint{2.068362in}{2.093758in}}{\pgfqpoint{2.062538in}{2.087934in}}%
\pgfpathcurveto{\pgfqpoint{2.056714in}{2.082110in}}{\pgfqpoint{2.053442in}{2.074210in}}{\pgfqpoint{2.053442in}{2.065973in}}%
\pgfpathcurveto{\pgfqpoint{2.053442in}{2.057737in}}{\pgfqpoint{2.056714in}{2.049837in}}{\pgfqpoint{2.062538in}{2.044013in}}%
\pgfpathcurveto{\pgfqpoint{2.068362in}{2.038189in}}{\pgfqpoint{2.076262in}{2.034917in}}{\pgfqpoint{2.084499in}{2.034917in}}%
\pgfpathclose%
\pgfusepath{stroke,fill}%
\end{pgfscope}%
\begin{pgfscope}%
\pgfpathrectangle{\pgfqpoint{0.100000in}{0.212622in}}{\pgfqpoint{3.696000in}{3.696000in}}%
\pgfusepath{clip}%
\pgfsetbuttcap%
\pgfsetroundjoin%
\definecolor{currentfill}{rgb}{0.121569,0.466667,0.705882}%
\pgfsetfillcolor{currentfill}%
\pgfsetfillopacity{0.966786}%
\pgfsetlinewidth{1.003750pt}%
\definecolor{currentstroke}{rgb}{0.121569,0.466667,0.705882}%
\pgfsetstrokecolor{currentstroke}%
\pgfsetstrokeopacity{0.966786}%
\pgfsetdash{}{0pt}%
\pgfpathmoveto{\pgfqpoint{2.499188in}{1.892012in}}%
\pgfpathcurveto{\pgfqpoint{2.507424in}{1.892012in}}{\pgfqpoint{2.515324in}{1.895285in}}{\pgfqpoint{2.521148in}{1.901108in}}%
\pgfpathcurveto{\pgfqpoint{2.526972in}{1.906932in}}{\pgfqpoint{2.530244in}{1.914832in}}{\pgfqpoint{2.530244in}{1.923069in}}%
\pgfpathcurveto{\pgfqpoint{2.530244in}{1.931305in}}{\pgfqpoint{2.526972in}{1.939205in}}{\pgfqpoint{2.521148in}{1.945029in}}%
\pgfpathcurveto{\pgfqpoint{2.515324in}{1.950853in}}{\pgfqpoint{2.507424in}{1.954125in}}{\pgfqpoint{2.499188in}{1.954125in}}%
\pgfpathcurveto{\pgfqpoint{2.490951in}{1.954125in}}{\pgfqpoint{2.483051in}{1.950853in}}{\pgfqpoint{2.477227in}{1.945029in}}%
\pgfpathcurveto{\pgfqpoint{2.471404in}{1.939205in}}{\pgfqpoint{2.468131in}{1.931305in}}{\pgfqpoint{2.468131in}{1.923069in}}%
\pgfpathcurveto{\pgfqpoint{2.468131in}{1.914832in}}{\pgfqpoint{2.471404in}{1.906932in}}{\pgfqpoint{2.477227in}{1.901108in}}%
\pgfpathcurveto{\pgfqpoint{2.483051in}{1.895285in}}{\pgfqpoint{2.490951in}{1.892012in}}{\pgfqpoint{2.499188in}{1.892012in}}%
\pgfpathclose%
\pgfusepath{stroke,fill}%
\end{pgfscope}%
\begin{pgfscope}%
\pgfpathrectangle{\pgfqpoint{0.100000in}{0.212622in}}{\pgfqpoint{3.696000in}{3.696000in}}%
\pgfusepath{clip}%
\pgfsetbuttcap%
\pgfsetroundjoin%
\definecolor{currentfill}{rgb}{0.121569,0.466667,0.705882}%
\pgfsetfillcolor{currentfill}%
\pgfsetfillopacity{0.967186}%
\pgfsetlinewidth{1.003750pt}%
\definecolor{currentstroke}{rgb}{0.121569,0.466667,0.705882}%
\pgfsetstrokecolor{currentstroke}%
\pgfsetstrokeopacity{0.967186}%
\pgfsetdash{}{0pt}%
\pgfpathmoveto{\pgfqpoint{2.099233in}{2.030674in}}%
\pgfpathcurveto{\pgfqpoint{2.107469in}{2.030674in}}{\pgfqpoint{2.115369in}{2.033947in}}{\pgfqpoint{2.121193in}{2.039771in}}%
\pgfpathcurveto{\pgfqpoint{2.127017in}{2.045594in}}{\pgfqpoint{2.130289in}{2.053495in}}{\pgfqpoint{2.130289in}{2.061731in}}%
\pgfpathcurveto{\pgfqpoint{2.130289in}{2.069967in}}{\pgfqpoint{2.127017in}{2.077867in}}{\pgfqpoint{2.121193in}{2.083691in}}%
\pgfpathcurveto{\pgfqpoint{2.115369in}{2.089515in}}{\pgfqpoint{2.107469in}{2.092787in}}{\pgfqpoint{2.099233in}{2.092787in}}%
\pgfpathcurveto{\pgfqpoint{2.090996in}{2.092787in}}{\pgfqpoint{2.083096in}{2.089515in}}{\pgfqpoint{2.077272in}{2.083691in}}%
\pgfpathcurveto{\pgfqpoint{2.071449in}{2.077867in}}{\pgfqpoint{2.068176in}{2.069967in}}{\pgfqpoint{2.068176in}{2.061731in}}%
\pgfpathcurveto{\pgfqpoint{2.068176in}{2.053495in}}{\pgfqpoint{2.071449in}{2.045594in}}{\pgfqpoint{2.077272in}{2.039771in}}%
\pgfpathcurveto{\pgfqpoint{2.083096in}{2.033947in}}{\pgfqpoint{2.090996in}{2.030674in}}{\pgfqpoint{2.099233in}{2.030674in}}%
\pgfpathclose%
\pgfusepath{stroke,fill}%
\end{pgfscope}%
\begin{pgfscope}%
\pgfpathrectangle{\pgfqpoint{0.100000in}{0.212622in}}{\pgfqpoint{3.696000in}{3.696000in}}%
\pgfusepath{clip}%
\pgfsetbuttcap%
\pgfsetroundjoin%
\definecolor{currentfill}{rgb}{0.121569,0.466667,0.705882}%
\pgfsetfillcolor{currentfill}%
\pgfsetfillopacity{0.968500}%
\pgfsetlinewidth{1.003750pt}%
\definecolor{currentstroke}{rgb}{0.121569,0.466667,0.705882}%
\pgfsetstrokecolor{currentstroke}%
\pgfsetstrokeopacity{0.968500}%
\pgfsetdash{}{0pt}%
\pgfpathmoveto{\pgfqpoint{2.110432in}{2.022775in}}%
\pgfpathcurveto{\pgfqpoint{2.118668in}{2.022775in}}{\pgfqpoint{2.126568in}{2.026047in}}{\pgfqpoint{2.132392in}{2.031871in}}%
\pgfpathcurveto{\pgfqpoint{2.138216in}{2.037695in}}{\pgfqpoint{2.141488in}{2.045595in}}{\pgfqpoint{2.141488in}{2.053831in}}%
\pgfpathcurveto{\pgfqpoint{2.141488in}{2.062068in}}{\pgfqpoint{2.138216in}{2.069968in}}{\pgfqpoint{2.132392in}{2.075792in}}%
\pgfpathcurveto{\pgfqpoint{2.126568in}{2.081616in}}{\pgfqpoint{2.118668in}{2.084888in}}{\pgfqpoint{2.110432in}{2.084888in}}%
\pgfpathcurveto{\pgfqpoint{2.102195in}{2.084888in}}{\pgfqpoint{2.094295in}{2.081616in}}{\pgfqpoint{2.088471in}{2.075792in}}%
\pgfpathcurveto{\pgfqpoint{2.082647in}{2.069968in}}{\pgfqpoint{2.079375in}{2.062068in}}{\pgfqpoint{2.079375in}{2.053831in}}%
\pgfpathcurveto{\pgfqpoint{2.079375in}{2.045595in}}{\pgfqpoint{2.082647in}{2.037695in}}{\pgfqpoint{2.088471in}{2.031871in}}%
\pgfpathcurveto{\pgfqpoint{2.094295in}{2.026047in}}{\pgfqpoint{2.102195in}{2.022775in}}{\pgfqpoint{2.110432in}{2.022775in}}%
\pgfpathclose%
\pgfusepath{stroke,fill}%
\end{pgfscope}%
\begin{pgfscope}%
\pgfpathrectangle{\pgfqpoint{0.100000in}{0.212622in}}{\pgfqpoint{3.696000in}{3.696000in}}%
\pgfusepath{clip}%
\pgfsetbuttcap%
\pgfsetroundjoin%
\definecolor{currentfill}{rgb}{0.121569,0.466667,0.705882}%
\pgfsetfillcolor{currentfill}%
\pgfsetfillopacity{0.968740}%
\pgfsetlinewidth{1.003750pt}%
\definecolor{currentstroke}{rgb}{0.121569,0.466667,0.705882}%
\pgfsetstrokecolor{currentstroke}%
\pgfsetstrokeopacity{0.968740}%
\pgfsetdash{}{0pt}%
\pgfpathmoveto{\pgfqpoint{2.495660in}{1.892541in}}%
\pgfpathcurveto{\pgfqpoint{2.503897in}{1.892541in}}{\pgfqpoint{2.511797in}{1.895814in}}{\pgfqpoint{2.517621in}{1.901637in}}%
\pgfpathcurveto{\pgfqpoint{2.523445in}{1.907461in}}{\pgfqpoint{2.526717in}{1.915361in}}{\pgfqpoint{2.526717in}{1.923598in}}%
\pgfpathcurveto{\pgfqpoint{2.526717in}{1.931834in}}{\pgfqpoint{2.523445in}{1.939734in}}{\pgfqpoint{2.517621in}{1.945558in}}%
\pgfpathcurveto{\pgfqpoint{2.511797in}{1.951382in}}{\pgfqpoint{2.503897in}{1.954654in}}{\pgfqpoint{2.495660in}{1.954654in}}%
\pgfpathcurveto{\pgfqpoint{2.487424in}{1.954654in}}{\pgfqpoint{2.479524in}{1.951382in}}{\pgfqpoint{2.473700in}{1.945558in}}%
\pgfpathcurveto{\pgfqpoint{2.467876in}{1.939734in}}{\pgfqpoint{2.464604in}{1.931834in}}{\pgfqpoint{2.464604in}{1.923598in}}%
\pgfpathcurveto{\pgfqpoint{2.464604in}{1.915361in}}{\pgfqpoint{2.467876in}{1.907461in}}{\pgfqpoint{2.473700in}{1.901637in}}%
\pgfpathcurveto{\pgfqpoint{2.479524in}{1.895814in}}{\pgfqpoint{2.487424in}{1.892541in}}{\pgfqpoint{2.495660in}{1.892541in}}%
\pgfpathclose%
\pgfusepath{stroke,fill}%
\end{pgfscope}%
\begin{pgfscope}%
\pgfpathrectangle{\pgfqpoint{0.100000in}{0.212622in}}{\pgfqpoint{3.696000in}{3.696000in}}%
\pgfusepath{clip}%
\pgfsetbuttcap%
\pgfsetroundjoin%
\definecolor{currentfill}{rgb}{0.121569,0.466667,0.705882}%
\pgfsetfillcolor{currentfill}%
\pgfsetfillopacity{0.969307}%
\pgfsetlinewidth{1.003750pt}%
\definecolor{currentstroke}{rgb}{0.121569,0.466667,0.705882}%
\pgfsetstrokecolor{currentstroke}%
\pgfsetstrokeopacity{0.969307}%
\pgfsetdash{}{0pt}%
\pgfpathmoveto{\pgfqpoint{2.121529in}{2.018893in}}%
\pgfpathcurveto{\pgfqpoint{2.129765in}{2.018893in}}{\pgfqpoint{2.137665in}{2.022165in}}{\pgfqpoint{2.143489in}{2.027989in}}%
\pgfpathcurveto{\pgfqpoint{2.149313in}{2.033813in}}{\pgfqpoint{2.152585in}{2.041713in}}{\pgfqpoint{2.152585in}{2.049950in}}%
\pgfpathcurveto{\pgfqpoint{2.152585in}{2.058186in}}{\pgfqpoint{2.149313in}{2.066086in}}{\pgfqpoint{2.143489in}{2.071910in}}%
\pgfpathcurveto{\pgfqpoint{2.137665in}{2.077734in}}{\pgfqpoint{2.129765in}{2.081006in}}{\pgfqpoint{2.121529in}{2.081006in}}%
\pgfpathcurveto{\pgfqpoint{2.113292in}{2.081006in}}{\pgfqpoint{2.105392in}{2.077734in}}{\pgfqpoint{2.099568in}{2.071910in}}%
\pgfpathcurveto{\pgfqpoint{2.093744in}{2.066086in}}{\pgfqpoint{2.090472in}{2.058186in}}{\pgfqpoint{2.090472in}{2.049950in}}%
\pgfpathcurveto{\pgfqpoint{2.090472in}{2.041713in}}{\pgfqpoint{2.093744in}{2.033813in}}{\pgfqpoint{2.099568in}{2.027989in}}%
\pgfpathcurveto{\pgfqpoint{2.105392in}{2.022165in}}{\pgfqpoint{2.113292in}{2.018893in}}{\pgfqpoint{2.121529in}{2.018893in}}%
\pgfpathclose%
\pgfusepath{stroke,fill}%
\end{pgfscope}%
\begin{pgfscope}%
\pgfpathrectangle{\pgfqpoint{0.100000in}{0.212622in}}{\pgfqpoint{3.696000in}{3.696000in}}%
\pgfusepath{clip}%
\pgfsetbuttcap%
\pgfsetroundjoin%
\definecolor{currentfill}{rgb}{0.121569,0.466667,0.705882}%
\pgfsetfillcolor{currentfill}%
\pgfsetfillopacity{0.969661}%
\pgfsetlinewidth{1.003750pt}%
\definecolor{currentstroke}{rgb}{0.121569,0.466667,0.705882}%
\pgfsetstrokecolor{currentstroke}%
\pgfsetstrokeopacity{0.969661}%
\pgfsetdash{}{0pt}%
\pgfpathmoveto{\pgfqpoint{2.132725in}{2.015050in}}%
\pgfpathcurveto{\pgfqpoint{2.140961in}{2.015050in}}{\pgfqpoint{2.148861in}{2.018322in}}{\pgfqpoint{2.154685in}{2.024146in}}%
\pgfpathcurveto{\pgfqpoint{2.160509in}{2.029970in}}{\pgfqpoint{2.163782in}{2.037870in}}{\pgfqpoint{2.163782in}{2.046106in}}%
\pgfpathcurveto{\pgfqpoint{2.163782in}{2.054342in}}{\pgfqpoint{2.160509in}{2.062242in}}{\pgfqpoint{2.154685in}{2.068066in}}%
\pgfpathcurveto{\pgfqpoint{2.148861in}{2.073890in}}{\pgfqpoint{2.140961in}{2.077163in}}{\pgfqpoint{2.132725in}{2.077163in}}%
\pgfpathcurveto{\pgfqpoint{2.124489in}{2.077163in}}{\pgfqpoint{2.116589in}{2.073890in}}{\pgfqpoint{2.110765in}{2.068066in}}%
\pgfpathcurveto{\pgfqpoint{2.104941in}{2.062242in}}{\pgfqpoint{2.101669in}{2.054342in}}{\pgfqpoint{2.101669in}{2.046106in}}%
\pgfpathcurveto{\pgfqpoint{2.101669in}{2.037870in}}{\pgfqpoint{2.104941in}{2.029970in}}{\pgfqpoint{2.110765in}{2.024146in}}%
\pgfpathcurveto{\pgfqpoint{2.116589in}{2.018322in}}{\pgfqpoint{2.124489in}{2.015050in}}{\pgfqpoint{2.132725in}{2.015050in}}%
\pgfpathclose%
\pgfusepath{stroke,fill}%
\end{pgfscope}%
\begin{pgfscope}%
\pgfpathrectangle{\pgfqpoint{0.100000in}{0.212622in}}{\pgfqpoint{3.696000in}{3.696000in}}%
\pgfusepath{clip}%
\pgfsetbuttcap%
\pgfsetroundjoin%
\definecolor{currentfill}{rgb}{0.121569,0.466667,0.705882}%
\pgfsetfillcolor{currentfill}%
\pgfsetfillopacity{0.970297}%
\pgfsetlinewidth{1.003750pt}%
\definecolor{currentstroke}{rgb}{0.121569,0.466667,0.705882}%
\pgfsetstrokecolor{currentstroke}%
\pgfsetstrokeopacity{0.970297}%
\pgfsetdash{}{0pt}%
\pgfpathmoveto{\pgfqpoint{2.140784in}{2.010911in}}%
\pgfpathcurveto{\pgfqpoint{2.149021in}{2.010911in}}{\pgfqpoint{2.156921in}{2.014184in}}{\pgfqpoint{2.162744in}{2.020008in}}%
\pgfpathcurveto{\pgfqpoint{2.168568in}{2.025832in}}{\pgfqpoint{2.171841in}{2.033732in}}{\pgfqpoint{2.171841in}{2.041968in}}%
\pgfpathcurveto{\pgfqpoint{2.171841in}{2.050204in}}{\pgfqpoint{2.168568in}{2.058104in}}{\pgfqpoint{2.162744in}{2.063928in}}%
\pgfpathcurveto{\pgfqpoint{2.156921in}{2.069752in}}{\pgfqpoint{2.149021in}{2.073024in}}{\pgfqpoint{2.140784in}{2.073024in}}%
\pgfpathcurveto{\pgfqpoint{2.132548in}{2.073024in}}{\pgfqpoint{2.124648in}{2.069752in}}{\pgfqpoint{2.118824in}{2.063928in}}%
\pgfpathcurveto{\pgfqpoint{2.113000in}{2.058104in}}{\pgfqpoint{2.109728in}{2.050204in}}{\pgfqpoint{2.109728in}{2.041968in}}%
\pgfpathcurveto{\pgfqpoint{2.109728in}{2.033732in}}{\pgfqpoint{2.113000in}{2.025832in}}{\pgfqpoint{2.118824in}{2.020008in}}%
\pgfpathcurveto{\pgfqpoint{2.124648in}{2.014184in}}{\pgfqpoint{2.132548in}{2.010911in}}{\pgfqpoint{2.140784in}{2.010911in}}%
\pgfpathclose%
\pgfusepath{stroke,fill}%
\end{pgfscope}%
\begin{pgfscope}%
\pgfpathrectangle{\pgfqpoint{0.100000in}{0.212622in}}{\pgfqpoint{3.696000in}{3.696000in}}%
\pgfusepath{clip}%
\pgfsetbuttcap%
\pgfsetroundjoin%
\definecolor{currentfill}{rgb}{0.121569,0.466667,0.705882}%
\pgfsetfillcolor{currentfill}%
\pgfsetfillopacity{0.970698}%
\pgfsetlinewidth{1.003750pt}%
\definecolor{currentstroke}{rgb}{0.121569,0.466667,0.705882}%
\pgfsetstrokecolor{currentstroke}%
\pgfsetstrokeopacity{0.970698}%
\pgfsetdash{}{0pt}%
\pgfpathmoveto{\pgfqpoint{2.490510in}{1.893830in}}%
\pgfpathcurveto{\pgfqpoint{2.498746in}{1.893830in}}{\pgfqpoint{2.506646in}{1.897102in}}{\pgfqpoint{2.512470in}{1.902926in}}%
\pgfpathcurveto{\pgfqpoint{2.518294in}{1.908750in}}{\pgfqpoint{2.521567in}{1.916650in}}{\pgfqpoint{2.521567in}{1.924886in}}%
\pgfpathcurveto{\pgfqpoint{2.521567in}{1.933122in}}{\pgfqpoint{2.518294in}{1.941022in}}{\pgfqpoint{2.512470in}{1.946846in}}%
\pgfpathcurveto{\pgfqpoint{2.506646in}{1.952670in}}{\pgfqpoint{2.498746in}{1.955943in}}{\pgfqpoint{2.490510in}{1.955943in}}%
\pgfpathcurveto{\pgfqpoint{2.482274in}{1.955943in}}{\pgfqpoint{2.474374in}{1.952670in}}{\pgfqpoint{2.468550in}{1.946846in}}%
\pgfpathcurveto{\pgfqpoint{2.462726in}{1.941022in}}{\pgfqpoint{2.459454in}{1.933122in}}{\pgfqpoint{2.459454in}{1.924886in}}%
\pgfpathcurveto{\pgfqpoint{2.459454in}{1.916650in}}{\pgfqpoint{2.462726in}{1.908750in}}{\pgfqpoint{2.468550in}{1.902926in}}%
\pgfpathcurveto{\pgfqpoint{2.474374in}{1.897102in}}{\pgfqpoint{2.482274in}{1.893830in}}{\pgfqpoint{2.490510in}{1.893830in}}%
\pgfpathclose%
\pgfusepath{stroke,fill}%
\end{pgfscope}%
\begin{pgfscope}%
\pgfpathrectangle{\pgfqpoint{0.100000in}{0.212622in}}{\pgfqpoint{3.696000in}{3.696000in}}%
\pgfusepath{clip}%
\pgfsetbuttcap%
\pgfsetroundjoin%
\definecolor{currentfill}{rgb}{0.121569,0.466667,0.705882}%
\pgfsetfillcolor{currentfill}%
\pgfsetfillopacity{0.970886}%
\pgfsetlinewidth{1.003750pt}%
\definecolor{currentstroke}{rgb}{0.121569,0.466667,0.705882}%
\pgfsetstrokecolor{currentstroke}%
\pgfsetstrokeopacity{0.970886}%
\pgfsetdash{}{0pt}%
\pgfpathmoveto{\pgfqpoint{2.146763in}{2.009166in}}%
\pgfpathcurveto{\pgfqpoint{2.154999in}{2.009166in}}{\pgfqpoint{2.162899in}{2.012438in}}{\pgfqpoint{2.168723in}{2.018262in}}%
\pgfpathcurveto{\pgfqpoint{2.174547in}{2.024086in}}{\pgfqpoint{2.177819in}{2.031986in}}{\pgfqpoint{2.177819in}{2.040222in}}%
\pgfpathcurveto{\pgfqpoint{2.177819in}{2.048459in}}{\pgfqpoint{2.174547in}{2.056359in}}{\pgfqpoint{2.168723in}{2.062183in}}%
\pgfpathcurveto{\pgfqpoint{2.162899in}{2.068007in}}{\pgfqpoint{2.154999in}{2.071279in}}{\pgfqpoint{2.146763in}{2.071279in}}%
\pgfpathcurveto{\pgfqpoint{2.138526in}{2.071279in}}{\pgfqpoint{2.130626in}{2.068007in}}{\pgfqpoint{2.124802in}{2.062183in}}%
\pgfpathcurveto{\pgfqpoint{2.118978in}{2.056359in}}{\pgfqpoint{2.115706in}{2.048459in}}{\pgfqpoint{2.115706in}{2.040222in}}%
\pgfpathcurveto{\pgfqpoint{2.115706in}{2.031986in}}{\pgfqpoint{2.118978in}{2.024086in}}{\pgfqpoint{2.124802in}{2.018262in}}%
\pgfpathcurveto{\pgfqpoint{2.130626in}{2.012438in}}{\pgfqpoint{2.138526in}{2.009166in}}{\pgfqpoint{2.146763in}{2.009166in}}%
\pgfpathclose%
\pgfusepath{stroke,fill}%
\end{pgfscope}%
\begin{pgfscope}%
\pgfpathrectangle{\pgfqpoint{0.100000in}{0.212622in}}{\pgfqpoint{3.696000in}{3.696000in}}%
\pgfusepath{clip}%
\pgfsetbuttcap%
\pgfsetroundjoin%
\definecolor{currentfill}{rgb}{0.121569,0.466667,0.705882}%
\pgfsetfillcolor{currentfill}%
\pgfsetfillopacity{0.971834}%
\pgfsetlinewidth{1.003750pt}%
\definecolor{currentstroke}{rgb}{0.121569,0.466667,0.705882}%
\pgfsetstrokecolor{currentstroke}%
\pgfsetstrokeopacity{0.971834}%
\pgfsetdash{}{0pt}%
\pgfpathmoveto{\pgfqpoint{2.157685in}{2.005366in}}%
\pgfpathcurveto{\pgfqpoint{2.165921in}{2.005366in}}{\pgfqpoint{2.173821in}{2.008638in}}{\pgfqpoint{2.179645in}{2.014462in}}%
\pgfpathcurveto{\pgfqpoint{2.185469in}{2.020286in}}{\pgfqpoint{2.188742in}{2.028186in}}{\pgfqpoint{2.188742in}{2.036423in}}%
\pgfpathcurveto{\pgfqpoint{2.188742in}{2.044659in}}{\pgfqpoint{2.185469in}{2.052559in}}{\pgfqpoint{2.179645in}{2.058383in}}%
\pgfpathcurveto{\pgfqpoint{2.173821in}{2.064207in}}{\pgfqpoint{2.165921in}{2.067479in}}{\pgfqpoint{2.157685in}{2.067479in}}%
\pgfpathcurveto{\pgfqpoint{2.149449in}{2.067479in}}{\pgfqpoint{2.141549in}{2.064207in}}{\pgfqpoint{2.135725in}{2.058383in}}%
\pgfpathcurveto{\pgfqpoint{2.129901in}{2.052559in}}{\pgfqpoint{2.126629in}{2.044659in}}{\pgfqpoint{2.126629in}{2.036423in}}%
\pgfpathcurveto{\pgfqpoint{2.126629in}{2.028186in}}{\pgfqpoint{2.129901in}{2.020286in}}{\pgfqpoint{2.135725in}{2.014462in}}%
\pgfpathcurveto{\pgfqpoint{2.141549in}{2.008638in}}{\pgfqpoint{2.149449in}{2.005366in}}{\pgfqpoint{2.157685in}{2.005366in}}%
\pgfpathclose%
\pgfusepath{stroke,fill}%
\end{pgfscope}%
\begin{pgfscope}%
\pgfpathrectangle{\pgfqpoint{0.100000in}{0.212622in}}{\pgfqpoint{3.696000in}{3.696000in}}%
\pgfusepath{clip}%
\pgfsetbuttcap%
\pgfsetroundjoin%
\definecolor{currentfill}{rgb}{0.121569,0.466667,0.705882}%
\pgfsetfillcolor{currentfill}%
\pgfsetfillopacity{0.972459}%
\pgfsetlinewidth{1.003750pt}%
\definecolor{currentstroke}{rgb}{0.121569,0.466667,0.705882}%
\pgfsetstrokecolor{currentstroke}%
\pgfsetstrokeopacity{0.972459}%
\pgfsetdash{}{0pt}%
\pgfpathmoveto{\pgfqpoint{2.166291in}{2.000392in}}%
\pgfpathcurveto{\pgfqpoint{2.174527in}{2.000392in}}{\pgfqpoint{2.182427in}{2.003665in}}{\pgfqpoint{2.188251in}{2.009489in}}%
\pgfpathcurveto{\pgfqpoint{2.194075in}{2.015313in}}{\pgfqpoint{2.197347in}{2.023213in}}{\pgfqpoint{2.197347in}{2.031449in}}%
\pgfpathcurveto{\pgfqpoint{2.197347in}{2.039685in}}{\pgfqpoint{2.194075in}{2.047585in}}{\pgfqpoint{2.188251in}{2.053409in}}%
\pgfpathcurveto{\pgfqpoint{2.182427in}{2.059233in}}{\pgfqpoint{2.174527in}{2.062505in}}{\pgfqpoint{2.166291in}{2.062505in}}%
\pgfpathcurveto{\pgfqpoint{2.158054in}{2.062505in}}{\pgfqpoint{2.150154in}{2.059233in}}{\pgfqpoint{2.144330in}{2.053409in}}%
\pgfpathcurveto{\pgfqpoint{2.138506in}{2.047585in}}{\pgfqpoint{2.135234in}{2.039685in}}{\pgfqpoint{2.135234in}{2.031449in}}%
\pgfpathcurveto{\pgfqpoint{2.135234in}{2.023213in}}{\pgfqpoint{2.138506in}{2.015313in}}{\pgfqpoint{2.144330in}{2.009489in}}%
\pgfpathcurveto{\pgfqpoint{2.150154in}{2.003665in}}{\pgfqpoint{2.158054in}{2.000392in}}{\pgfqpoint{2.166291in}{2.000392in}}%
\pgfpathclose%
\pgfusepath{stroke,fill}%
\end{pgfscope}%
\begin{pgfscope}%
\pgfpathrectangle{\pgfqpoint{0.100000in}{0.212622in}}{\pgfqpoint{3.696000in}{3.696000in}}%
\pgfusepath{clip}%
\pgfsetbuttcap%
\pgfsetroundjoin%
\definecolor{currentfill}{rgb}{0.121569,0.466667,0.705882}%
\pgfsetfillcolor{currentfill}%
\pgfsetfillopacity{0.973119}%
\pgfsetlinewidth{1.003750pt}%
\definecolor{currentstroke}{rgb}{0.121569,0.466667,0.705882}%
\pgfsetstrokecolor{currentstroke}%
\pgfsetstrokeopacity{0.973119}%
\pgfsetdash{}{0pt}%
\pgfpathmoveto{\pgfqpoint{2.171683in}{1.998895in}}%
\pgfpathcurveto{\pgfqpoint{2.179919in}{1.998895in}}{\pgfqpoint{2.187819in}{2.002167in}}{\pgfqpoint{2.193643in}{2.007991in}}%
\pgfpathcurveto{\pgfqpoint{2.199467in}{2.013815in}}{\pgfqpoint{2.202739in}{2.021715in}}{\pgfqpoint{2.202739in}{2.029951in}}%
\pgfpathcurveto{\pgfqpoint{2.202739in}{2.038187in}}{\pgfqpoint{2.199467in}{2.046087in}}{\pgfqpoint{2.193643in}{2.051911in}}%
\pgfpathcurveto{\pgfqpoint{2.187819in}{2.057735in}}{\pgfqpoint{2.179919in}{2.061008in}}{\pgfqpoint{2.171683in}{2.061008in}}%
\pgfpathcurveto{\pgfqpoint{2.163447in}{2.061008in}}{\pgfqpoint{2.155547in}{2.057735in}}{\pgfqpoint{2.149723in}{2.051911in}}%
\pgfpathcurveto{\pgfqpoint{2.143899in}{2.046087in}}{\pgfqpoint{2.140626in}{2.038187in}}{\pgfqpoint{2.140626in}{2.029951in}}%
\pgfpathcurveto{\pgfqpoint{2.140626in}{2.021715in}}{\pgfqpoint{2.143899in}{2.013815in}}{\pgfqpoint{2.149723in}{2.007991in}}%
\pgfpathcurveto{\pgfqpoint{2.155547in}{2.002167in}}{\pgfqpoint{2.163447in}{1.998895in}}{\pgfqpoint{2.171683in}{1.998895in}}%
\pgfpathclose%
\pgfusepath{stroke,fill}%
\end{pgfscope}%
\begin{pgfscope}%
\pgfpathrectangle{\pgfqpoint{0.100000in}{0.212622in}}{\pgfqpoint{3.696000in}{3.696000in}}%
\pgfusepath{clip}%
\pgfsetbuttcap%
\pgfsetroundjoin%
\definecolor{currentfill}{rgb}{0.121569,0.466667,0.705882}%
\pgfsetfillcolor{currentfill}%
\pgfsetfillopacity{0.973708}%
\pgfsetlinewidth{1.003750pt}%
\definecolor{currentstroke}{rgb}{0.121569,0.466667,0.705882}%
\pgfsetstrokecolor{currentstroke}%
\pgfsetstrokeopacity{0.973708}%
\pgfsetdash{}{0pt}%
\pgfpathmoveto{\pgfqpoint{2.484815in}{1.894380in}}%
\pgfpathcurveto{\pgfqpoint{2.493051in}{1.894380in}}{\pgfqpoint{2.500951in}{1.897652in}}{\pgfqpoint{2.506775in}{1.903476in}}%
\pgfpathcurveto{\pgfqpoint{2.512599in}{1.909300in}}{\pgfqpoint{2.515871in}{1.917200in}}{\pgfqpoint{2.515871in}{1.925436in}}%
\pgfpathcurveto{\pgfqpoint{2.515871in}{1.933673in}}{\pgfqpoint{2.512599in}{1.941573in}}{\pgfqpoint{2.506775in}{1.947397in}}%
\pgfpathcurveto{\pgfqpoint{2.500951in}{1.953220in}}{\pgfqpoint{2.493051in}{1.956493in}}{\pgfqpoint{2.484815in}{1.956493in}}%
\pgfpathcurveto{\pgfqpoint{2.476579in}{1.956493in}}{\pgfqpoint{2.468679in}{1.953220in}}{\pgfqpoint{2.462855in}{1.947397in}}%
\pgfpathcurveto{\pgfqpoint{2.457031in}{1.941573in}}{\pgfqpoint{2.453758in}{1.933673in}}{\pgfqpoint{2.453758in}{1.925436in}}%
\pgfpathcurveto{\pgfqpoint{2.453758in}{1.917200in}}{\pgfqpoint{2.457031in}{1.909300in}}{\pgfqpoint{2.462855in}{1.903476in}}%
\pgfpathcurveto{\pgfqpoint{2.468679in}{1.897652in}}{\pgfqpoint{2.476579in}{1.894380in}}{\pgfqpoint{2.484815in}{1.894380in}}%
\pgfpathclose%
\pgfusepath{stroke,fill}%
\end{pgfscope}%
\begin{pgfscope}%
\pgfpathrectangle{\pgfqpoint{0.100000in}{0.212622in}}{\pgfqpoint{3.696000in}{3.696000in}}%
\pgfusepath{clip}%
\pgfsetbuttcap%
\pgfsetroundjoin%
\definecolor{currentfill}{rgb}{0.121569,0.466667,0.705882}%
\pgfsetfillcolor{currentfill}%
\pgfsetfillopacity{0.973896}%
\pgfsetlinewidth{1.003750pt}%
\definecolor{currentstroke}{rgb}{0.121569,0.466667,0.705882}%
\pgfsetstrokecolor{currentstroke}%
\pgfsetstrokeopacity{0.973896}%
\pgfsetdash{}{0pt}%
\pgfpathmoveto{\pgfqpoint{2.182057in}{1.995627in}}%
\pgfpathcurveto{\pgfqpoint{2.190293in}{1.995627in}}{\pgfqpoint{2.198193in}{1.998900in}}{\pgfqpoint{2.204017in}{2.004724in}}%
\pgfpathcurveto{\pgfqpoint{2.209841in}{2.010548in}}{\pgfqpoint{2.213113in}{2.018448in}}{\pgfqpoint{2.213113in}{2.026684in}}%
\pgfpathcurveto{\pgfqpoint{2.213113in}{2.034920in}}{\pgfqpoint{2.209841in}{2.042820in}}{\pgfqpoint{2.204017in}{2.048644in}}%
\pgfpathcurveto{\pgfqpoint{2.198193in}{2.054468in}}{\pgfqpoint{2.190293in}{2.057740in}}{\pgfqpoint{2.182057in}{2.057740in}}%
\pgfpathcurveto{\pgfqpoint{2.173820in}{2.057740in}}{\pgfqpoint{2.165920in}{2.054468in}}{\pgfqpoint{2.160096in}{2.048644in}}%
\pgfpathcurveto{\pgfqpoint{2.154272in}{2.042820in}}{\pgfqpoint{2.151000in}{2.034920in}}{\pgfqpoint{2.151000in}{2.026684in}}%
\pgfpathcurveto{\pgfqpoint{2.151000in}{2.018448in}}{\pgfqpoint{2.154272in}{2.010548in}}{\pgfqpoint{2.160096in}{2.004724in}}%
\pgfpathcurveto{\pgfqpoint{2.165920in}{1.998900in}}{\pgfqpoint{2.173820in}{1.995627in}}{\pgfqpoint{2.182057in}{1.995627in}}%
\pgfpathclose%
\pgfusepath{stroke,fill}%
\end{pgfscope}%
\begin{pgfscope}%
\pgfpathrectangle{\pgfqpoint{0.100000in}{0.212622in}}{\pgfqpoint{3.696000in}{3.696000in}}%
\pgfusepath{clip}%
\pgfsetbuttcap%
\pgfsetroundjoin%
\definecolor{currentfill}{rgb}{0.121569,0.466667,0.705882}%
\pgfsetfillcolor{currentfill}%
\pgfsetfillopacity{0.974595}%
\pgfsetlinewidth{1.003750pt}%
\definecolor{currentstroke}{rgb}{0.121569,0.466667,0.705882}%
\pgfsetstrokecolor{currentstroke}%
\pgfsetstrokeopacity{0.974595}%
\pgfsetdash{}{0pt}%
\pgfpathmoveto{\pgfqpoint{2.190279in}{1.992008in}}%
\pgfpathcurveto{\pgfqpoint{2.198515in}{1.992008in}}{\pgfqpoint{2.206415in}{1.995280in}}{\pgfqpoint{2.212239in}{2.001104in}}%
\pgfpathcurveto{\pgfqpoint{2.218063in}{2.006928in}}{\pgfqpoint{2.221335in}{2.014828in}}{\pgfqpoint{2.221335in}{2.023065in}}%
\pgfpathcurveto{\pgfqpoint{2.221335in}{2.031301in}}{\pgfqpoint{2.218063in}{2.039201in}}{\pgfqpoint{2.212239in}{2.045025in}}%
\pgfpathcurveto{\pgfqpoint{2.206415in}{2.050849in}}{\pgfqpoint{2.198515in}{2.054121in}}{\pgfqpoint{2.190279in}{2.054121in}}%
\pgfpathcurveto{\pgfqpoint{2.182043in}{2.054121in}}{\pgfqpoint{2.174143in}{2.050849in}}{\pgfqpoint{2.168319in}{2.045025in}}%
\pgfpathcurveto{\pgfqpoint{2.162495in}{2.039201in}}{\pgfqpoint{2.159222in}{2.031301in}}{\pgfqpoint{2.159222in}{2.023065in}}%
\pgfpathcurveto{\pgfqpoint{2.159222in}{2.014828in}}{\pgfqpoint{2.162495in}{2.006928in}}{\pgfqpoint{2.168319in}{2.001104in}}%
\pgfpathcurveto{\pgfqpoint{2.174143in}{1.995280in}}{\pgfqpoint{2.182043in}{1.992008in}}{\pgfqpoint{2.190279in}{1.992008in}}%
\pgfpathclose%
\pgfusepath{stroke,fill}%
\end{pgfscope}%
\begin{pgfscope}%
\pgfpathrectangle{\pgfqpoint{0.100000in}{0.212622in}}{\pgfqpoint{3.696000in}{3.696000in}}%
\pgfusepath{clip}%
\pgfsetbuttcap%
\pgfsetroundjoin%
\definecolor{currentfill}{rgb}{0.121569,0.466667,0.705882}%
\pgfsetfillcolor{currentfill}%
\pgfsetfillopacity{0.975311}%
\pgfsetlinewidth{1.003750pt}%
\definecolor{currentstroke}{rgb}{0.121569,0.466667,0.705882}%
\pgfsetstrokecolor{currentstroke}%
\pgfsetstrokeopacity{0.975311}%
\pgfsetdash{}{0pt}%
\pgfpathmoveto{\pgfqpoint{2.195510in}{1.991071in}}%
\pgfpathcurveto{\pgfqpoint{2.203747in}{1.991071in}}{\pgfqpoint{2.211647in}{1.994344in}}{\pgfqpoint{2.217471in}{2.000167in}}%
\pgfpathcurveto{\pgfqpoint{2.223294in}{2.005991in}}{\pgfqpoint{2.226567in}{2.013891in}}{\pgfqpoint{2.226567in}{2.022128in}}%
\pgfpathcurveto{\pgfqpoint{2.226567in}{2.030364in}}{\pgfqpoint{2.223294in}{2.038264in}}{\pgfqpoint{2.217471in}{2.044088in}}%
\pgfpathcurveto{\pgfqpoint{2.211647in}{2.049912in}}{\pgfqpoint{2.203747in}{2.053184in}}{\pgfqpoint{2.195510in}{2.053184in}}%
\pgfpathcurveto{\pgfqpoint{2.187274in}{2.053184in}}{\pgfqpoint{2.179374in}{2.049912in}}{\pgfqpoint{2.173550in}{2.044088in}}%
\pgfpathcurveto{\pgfqpoint{2.167726in}{2.038264in}}{\pgfqpoint{2.164454in}{2.030364in}}{\pgfqpoint{2.164454in}{2.022128in}}%
\pgfpathcurveto{\pgfqpoint{2.164454in}{2.013891in}}{\pgfqpoint{2.167726in}{2.005991in}}{\pgfqpoint{2.173550in}{2.000167in}}%
\pgfpathcurveto{\pgfqpoint{2.179374in}{1.994344in}}{\pgfqpoint{2.187274in}{1.991071in}}{\pgfqpoint{2.195510in}{1.991071in}}%
\pgfpathclose%
\pgfusepath{stroke,fill}%
\end{pgfscope}%
\begin{pgfscope}%
\pgfpathrectangle{\pgfqpoint{0.100000in}{0.212622in}}{\pgfqpoint{3.696000in}{3.696000in}}%
\pgfusepath{clip}%
\pgfsetbuttcap%
\pgfsetroundjoin%
\definecolor{currentfill}{rgb}{0.121569,0.466667,0.705882}%
\pgfsetfillcolor{currentfill}%
\pgfsetfillopacity{0.976218}%
\pgfsetlinewidth{1.003750pt}%
\definecolor{currentstroke}{rgb}{0.121569,0.466667,0.705882}%
\pgfsetstrokecolor{currentstroke}%
\pgfsetstrokeopacity{0.976218}%
\pgfsetdash{}{0pt}%
\pgfpathmoveto{\pgfqpoint{2.205305in}{1.987764in}}%
\pgfpathcurveto{\pgfqpoint{2.213541in}{1.987764in}}{\pgfqpoint{2.221441in}{1.991037in}}{\pgfqpoint{2.227265in}{1.996861in}}%
\pgfpathcurveto{\pgfqpoint{2.233089in}{2.002685in}}{\pgfqpoint{2.236361in}{2.010585in}}{\pgfqpoint{2.236361in}{2.018821in}}%
\pgfpathcurveto{\pgfqpoint{2.236361in}{2.027057in}}{\pgfqpoint{2.233089in}{2.034957in}}{\pgfqpoint{2.227265in}{2.040781in}}%
\pgfpathcurveto{\pgfqpoint{2.221441in}{2.046605in}}{\pgfqpoint{2.213541in}{2.049877in}}{\pgfqpoint{2.205305in}{2.049877in}}%
\pgfpathcurveto{\pgfqpoint{2.197068in}{2.049877in}}{\pgfqpoint{2.189168in}{2.046605in}}{\pgfqpoint{2.183344in}{2.040781in}}%
\pgfpathcurveto{\pgfqpoint{2.177521in}{2.034957in}}{\pgfqpoint{2.174248in}{2.027057in}}{\pgfqpoint{2.174248in}{2.018821in}}%
\pgfpathcurveto{\pgfqpoint{2.174248in}{2.010585in}}{\pgfqpoint{2.177521in}{2.002685in}}{\pgfqpoint{2.183344in}{1.996861in}}%
\pgfpathcurveto{\pgfqpoint{2.189168in}{1.991037in}}{\pgfqpoint{2.197068in}{1.987764in}}{\pgfqpoint{2.205305in}{1.987764in}}%
\pgfpathclose%
\pgfusepath{stroke,fill}%
\end{pgfscope}%
\begin{pgfscope}%
\pgfpathrectangle{\pgfqpoint{0.100000in}{0.212622in}}{\pgfqpoint{3.696000in}{3.696000in}}%
\pgfusepath{clip}%
\pgfsetbuttcap%
\pgfsetroundjoin%
\definecolor{currentfill}{rgb}{0.121569,0.466667,0.705882}%
\pgfsetfillcolor{currentfill}%
\pgfsetfillopacity{0.976809}%
\pgfsetlinewidth{1.003750pt}%
\definecolor{currentstroke}{rgb}{0.121569,0.466667,0.705882}%
\pgfsetstrokecolor{currentstroke}%
\pgfsetstrokeopacity{0.976809}%
\pgfsetdash{}{0pt}%
\pgfpathmoveto{\pgfqpoint{2.215018in}{1.984233in}}%
\pgfpathcurveto{\pgfqpoint{2.223254in}{1.984233in}}{\pgfqpoint{2.231154in}{1.987506in}}{\pgfqpoint{2.236978in}{1.993330in}}%
\pgfpathcurveto{\pgfqpoint{2.242802in}{1.999153in}}{\pgfqpoint{2.246074in}{2.007054in}}{\pgfqpoint{2.246074in}{2.015290in}}%
\pgfpathcurveto{\pgfqpoint{2.246074in}{2.023526in}}{\pgfqpoint{2.242802in}{2.031426in}}{\pgfqpoint{2.236978in}{2.037250in}}%
\pgfpathcurveto{\pgfqpoint{2.231154in}{2.043074in}}{\pgfqpoint{2.223254in}{2.046346in}}{\pgfqpoint{2.215018in}{2.046346in}}%
\pgfpathcurveto{\pgfqpoint{2.206781in}{2.046346in}}{\pgfqpoint{2.198881in}{2.043074in}}{\pgfqpoint{2.193057in}{2.037250in}}%
\pgfpathcurveto{\pgfqpoint{2.187233in}{2.031426in}}{\pgfqpoint{2.183961in}{2.023526in}}{\pgfqpoint{2.183961in}{2.015290in}}%
\pgfpathcurveto{\pgfqpoint{2.183961in}{2.007054in}}{\pgfqpoint{2.187233in}{1.999153in}}{\pgfqpoint{2.193057in}{1.993330in}}%
\pgfpathcurveto{\pgfqpoint{2.198881in}{1.987506in}}{\pgfqpoint{2.206781in}{1.984233in}}{\pgfqpoint{2.215018in}{1.984233in}}%
\pgfpathclose%
\pgfusepath{stroke,fill}%
\end{pgfscope}%
\begin{pgfscope}%
\pgfpathrectangle{\pgfqpoint{0.100000in}{0.212622in}}{\pgfqpoint{3.696000in}{3.696000in}}%
\pgfusepath{clip}%
\pgfsetbuttcap%
\pgfsetroundjoin%
\definecolor{currentfill}{rgb}{0.121569,0.466667,0.705882}%
\pgfsetfillcolor{currentfill}%
\pgfsetfillopacity{0.977064}%
\pgfsetlinewidth{1.003750pt}%
\definecolor{currentstroke}{rgb}{0.121569,0.466667,0.705882}%
\pgfsetstrokecolor{currentstroke}%
\pgfsetstrokeopacity{0.977064}%
\pgfsetdash{}{0pt}%
\pgfpathmoveto{\pgfqpoint{2.480812in}{1.894711in}}%
\pgfpathcurveto{\pgfqpoint{2.489048in}{1.894711in}}{\pgfqpoint{2.496948in}{1.897983in}}{\pgfqpoint{2.502772in}{1.903807in}}%
\pgfpathcurveto{\pgfqpoint{2.508596in}{1.909631in}}{\pgfqpoint{2.511868in}{1.917531in}}{\pgfqpoint{2.511868in}{1.925768in}}%
\pgfpathcurveto{\pgfqpoint{2.511868in}{1.934004in}}{\pgfqpoint{2.508596in}{1.941904in}}{\pgfqpoint{2.502772in}{1.947728in}}%
\pgfpathcurveto{\pgfqpoint{2.496948in}{1.953552in}}{\pgfqpoint{2.489048in}{1.956824in}}{\pgfqpoint{2.480812in}{1.956824in}}%
\pgfpathcurveto{\pgfqpoint{2.472576in}{1.956824in}}{\pgfqpoint{2.464675in}{1.953552in}}{\pgfqpoint{2.458852in}{1.947728in}}%
\pgfpathcurveto{\pgfqpoint{2.453028in}{1.941904in}}{\pgfqpoint{2.449755in}{1.934004in}}{\pgfqpoint{2.449755in}{1.925768in}}%
\pgfpathcurveto{\pgfqpoint{2.449755in}{1.917531in}}{\pgfqpoint{2.453028in}{1.909631in}}{\pgfqpoint{2.458852in}{1.903807in}}%
\pgfpathcurveto{\pgfqpoint{2.464675in}{1.897983in}}{\pgfqpoint{2.472576in}{1.894711in}}{\pgfqpoint{2.480812in}{1.894711in}}%
\pgfpathclose%
\pgfusepath{stroke,fill}%
\end{pgfscope}%
\begin{pgfscope}%
\pgfpathrectangle{\pgfqpoint{0.100000in}{0.212622in}}{\pgfqpoint{3.696000in}{3.696000in}}%
\pgfusepath{clip}%
\pgfsetbuttcap%
\pgfsetroundjoin%
\definecolor{currentfill}{rgb}{0.121569,0.466667,0.705882}%
\pgfsetfillcolor{currentfill}%
\pgfsetfillopacity{0.977538}%
\pgfsetlinewidth{1.003750pt}%
\definecolor{currentstroke}{rgb}{0.121569,0.466667,0.705882}%
\pgfsetstrokecolor{currentstroke}%
\pgfsetstrokeopacity{0.977538}%
\pgfsetdash{}{0pt}%
\pgfpathmoveto{\pgfqpoint{2.222606in}{1.982418in}}%
\pgfpathcurveto{\pgfqpoint{2.230842in}{1.982418in}}{\pgfqpoint{2.238742in}{1.985690in}}{\pgfqpoint{2.244566in}{1.991514in}}%
\pgfpathcurveto{\pgfqpoint{2.250390in}{1.997338in}}{\pgfqpoint{2.253662in}{2.005238in}}{\pgfqpoint{2.253662in}{2.013474in}}%
\pgfpathcurveto{\pgfqpoint{2.253662in}{2.021711in}}{\pgfqpoint{2.250390in}{2.029611in}}{\pgfqpoint{2.244566in}{2.035435in}}%
\pgfpathcurveto{\pgfqpoint{2.238742in}{2.041259in}}{\pgfqpoint{2.230842in}{2.044531in}}{\pgfqpoint{2.222606in}{2.044531in}}%
\pgfpathcurveto{\pgfqpoint{2.214370in}{2.044531in}}{\pgfqpoint{2.206469in}{2.041259in}}{\pgfqpoint{2.200646in}{2.035435in}}%
\pgfpathcurveto{\pgfqpoint{2.194822in}{2.029611in}}{\pgfqpoint{2.191549in}{2.021711in}}{\pgfqpoint{2.191549in}{2.013474in}}%
\pgfpathcurveto{\pgfqpoint{2.191549in}{2.005238in}}{\pgfqpoint{2.194822in}{1.997338in}}{\pgfqpoint{2.200646in}{1.991514in}}%
\pgfpathcurveto{\pgfqpoint{2.206469in}{1.985690in}}{\pgfqpoint{2.214370in}{1.982418in}}{\pgfqpoint{2.222606in}{1.982418in}}%
\pgfpathclose%
\pgfusepath{stroke,fill}%
\end{pgfscope}%
\begin{pgfscope}%
\pgfpathrectangle{\pgfqpoint{0.100000in}{0.212622in}}{\pgfqpoint{3.696000in}{3.696000in}}%
\pgfusepath{clip}%
\pgfsetbuttcap%
\pgfsetroundjoin%
\definecolor{currentfill}{rgb}{0.121569,0.466667,0.705882}%
\pgfsetfillcolor{currentfill}%
\pgfsetfillopacity{0.978269}%
\pgfsetlinewidth{1.003750pt}%
\definecolor{currentstroke}{rgb}{0.121569,0.466667,0.705882}%
\pgfsetstrokecolor{currentstroke}%
\pgfsetstrokeopacity{0.978269}%
\pgfsetdash{}{0pt}%
\pgfpathmoveto{\pgfqpoint{2.229616in}{1.981208in}}%
\pgfpathcurveto{\pgfqpoint{2.237853in}{1.981208in}}{\pgfqpoint{2.245753in}{1.984480in}}{\pgfqpoint{2.251576in}{1.990304in}}%
\pgfpathcurveto{\pgfqpoint{2.257400in}{1.996128in}}{\pgfqpoint{2.260673in}{2.004028in}}{\pgfqpoint{2.260673in}{2.012265in}}%
\pgfpathcurveto{\pgfqpoint{2.260673in}{2.020501in}}{\pgfqpoint{2.257400in}{2.028401in}}{\pgfqpoint{2.251576in}{2.034225in}}%
\pgfpathcurveto{\pgfqpoint{2.245753in}{2.040049in}}{\pgfqpoint{2.237853in}{2.043321in}}{\pgfqpoint{2.229616in}{2.043321in}}%
\pgfpathcurveto{\pgfqpoint{2.221380in}{2.043321in}}{\pgfqpoint{2.213480in}{2.040049in}}{\pgfqpoint{2.207656in}{2.034225in}}%
\pgfpathcurveto{\pgfqpoint{2.201832in}{2.028401in}}{\pgfqpoint{2.198560in}{2.020501in}}{\pgfqpoint{2.198560in}{2.012265in}}%
\pgfpathcurveto{\pgfqpoint{2.198560in}{2.004028in}}{\pgfqpoint{2.201832in}{1.996128in}}{\pgfqpoint{2.207656in}{1.990304in}}%
\pgfpathcurveto{\pgfqpoint{2.213480in}{1.984480in}}{\pgfqpoint{2.221380in}{1.981208in}}{\pgfqpoint{2.229616in}{1.981208in}}%
\pgfpathclose%
\pgfusepath{stroke,fill}%
\end{pgfscope}%
\begin{pgfscope}%
\pgfpathrectangle{\pgfqpoint{0.100000in}{0.212622in}}{\pgfqpoint{3.696000in}{3.696000in}}%
\pgfusepath{clip}%
\pgfsetbuttcap%
\pgfsetroundjoin%
\definecolor{currentfill}{rgb}{0.121569,0.466667,0.705882}%
\pgfsetfillcolor{currentfill}%
\pgfsetfillopacity{0.978899}%
\pgfsetlinewidth{1.003750pt}%
\definecolor{currentstroke}{rgb}{0.121569,0.466667,0.705882}%
\pgfsetstrokecolor{currentstroke}%
\pgfsetstrokeopacity{0.978899}%
\pgfsetdash{}{0pt}%
\pgfpathmoveto{\pgfqpoint{2.242841in}{1.976782in}}%
\pgfpathcurveto{\pgfqpoint{2.251078in}{1.976782in}}{\pgfqpoint{2.258978in}{1.980054in}}{\pgfqpoint{2.264802in}{1.985878in}}%
\pgfpathcurveto{\pgfqpoint{2.270626in}{1.991702in}}{\pgfqpoint{2.273898in}{1.999602in}}{\pgfqpoint{2.273898in}{2.007838in}}%
\pgfpathcurveto{\pgfqpoint{2.273898in}{2.016074in}}{\pgfqpoint{2.270626in}{2.023974in}}{\pgfqpoint{2.264802in}{2.029798in}}%
\pgfpathcurveto{\pgfqpoint{2.258978in}{2.035622in}}{\pgfqpoint{2.251078in}{2.038895in}}{\pgfqpoint{2.242841in}{2.038895in}}%
\pgfpathcurveto{\pgfqpoint{2.234605in}{2.038895in}}{\pgfqpoint{2.226705in}{2.035622in}}{\pgfqpoint{2.220881in}{2.029798in}}%
\pgfpathcurveto{\pgfqpoint{2.215057in}{2.023974in}}{\pgfqpoint{2.211785in}{2.016074in}}{\pgfqpoint{2.211785in}{2.007838in}}%
\pgfpathcurveto{\pgfqpoint{2.211785in}{1.999602in}}{\pgfqpoint{2.215057in}{1.991702in}}{\pgfqpoint{2.220881in}{1.985878in}}%
\pgfpathcurveto{\pgfqpoint{2.226705in}{1.980054in}}{\pgfqpoint{2.234605in}{1.976782in}}{\pgfqpoint{2.242841in}{1.976782in}}%
\pgfpathclose%
\pgfusepath{stroke,fill}%
\end{pgfscope}%
\begin{pgfscope}%
\pgfpathrectangle{\pgfqpoint{0.100000in}{0.212622in}}{\pgfqpoint{3.696000in}{3.696000in}}%
\pgfusepath{clip}%
\pgfsetbuttcap%
\pgfsetroundjoin%
\definecolor{currentfill}{rgb}{0.121569,0.466667,0.705882}%
\pgfsetfillcolor{currentfill}%
\pgfsetfillopacity{0.979630}%
\pgfsetlinewidth{1.003750pt}%
\definecolor{currentstroke}{rgb}{0.121569,0.466667,0.705882}%
\pgfsetstrokecolor{currentstroke}%
\pgfsetstrokeopacity{0.979630}%
\pgfsetdash{}{0pt}%
\pgfpathmoveto{\pgfqpoint{2.255176in}{1.973023in}}%
\pgfpathcurveto{\pgfqpoint{2.263413in}{1.973023in}}{\pgfqpoint{2.271313in}{1.976295in}}{\pgfqpoint{2.277137in}{1.982119in}}%
\pgfpathcurveto{\pgfqpoint{2.282961in}{1.987943in}}{\pgfqpoint{2.286233in}{1.995843in}}{\pgfqpoint{2.286233in}{2.004079in}}%
\pgfpathcurveto{\pgfqpoint{2.286233in}{2.012315in}}{\pgfqpoint{2.282961in}{2.020215in}}{\pgfqpoint{2.277137in}{2.026039in}}%
\pgfpathcurveto{\pgfqpoint{2.271313in}{2.031863in}}{\pgfqpoint{2.263413in}{2.035136in}}{\pgfqpoint{2.255176in}{2.035136in}}%
\pgfpathcurveto{\pgfqpoint{2.246940in}{2.035136in}}{\pgfqpoint{2.239040in}{2.031863in}}{\pgfqpoint{2.233216in}{2.026039in}}%
\pgfpathcurveto{\pgfqpoint{2.227392in}{2.020215in}}{\pgfqpoint{2.224120in}{2.012315in}}{\pgfqpoint{2.224120in}{2.004079in}}%
\pgfpathcurveto{\pgfqpoint{2.224120in}{1.995843in}}{\pgfqpoint{2.227392in}{1.987943in}}{\pgfqpoint{2.233216in}{1.982119in}}%
\pgfpathcurveto{\pgfqpoint{2.239040in}{1.976295in}}{\pgfqpoint{2.246940in}{1.973023in}}{\pgfqpoint{2.255176in}{1.973023in}}%
\pgfpathclose%
\pgfusepath{stroke,fill}%
\end{pgfscope}%
\begin{pgfscope}%
\pgfpathrectangle{\pgfqpoint{0.100000in}{0.212622in}}{\pgfqpoint{3.696000in}{3.696000in}}%
\pgfusepath{clip}%
\pgfsetbuttcap%
\pgfsetroundjoin%
\definecolor{currentfill}{rgb}{0.121569,0.466667,0.705882}%
\pgfsetfillcolor{currentfill}%
\pgfsetfillopacity{0.980468}%
\pgfsetlinewidth{1.003750pt}%
\definecolor{currentstroke}{rgb}{0.121569,0.466667,0.705882}%
\pgfsetstrokecolor{currentstroke}%
\pgfsetstrokeopacity{0.980468}%
\pgfsetdash{}{0pt}%
\pgfpathmoveto{\pgfqpoint{2.473212in}{1.896046in}}%
\pgfpathcurveto{\pgfqpoint{2.481448in}{1.896046in}}{\pgfqpoint{2.489348in}{1.899319in}}{\pgfqpoint{2.495172in}{1.905143in}}%
\pgfpathcurveto{\pgfqpoint{2.500996in}{1.910967in}}{\pgfqpoint{2.504268in}{1.918867in}}{\pgfqpoint{2.504268in}{1.927103in}}%
\pgfpathcurveto{\pgfqpoint{2.504268in}{1.935339in}}{\pgfqpoint{2.500996in}{1.943239in}}{\pgfqpoint{2.495172in}{1.949063in}}%
\pgfpathcurveto{\pgfqpoint{2.489348in}{1.954887in}}{\pgfqpoint{2.481448in}{1.958159in}}{\pgfqpoint{2.473212in}{1.958159in}}%
\pgfpathcurveto{\pgfqpoint{2.464975in}{1.958159in}}{\pgfqpoint{2.457075in}{1.954887in}}{\pgfqpoint{2.451251in}{1.949063in}}%
\pgfpathcurveto{\pgfqpoint{2.445428in}{1.943239in}}{\pgfqpoint{2.442155in}{1.935339in}}{\pgfqpoint{2.442155in}{1.927103in}}%
\pgfpathcurveto{\pgfqpoint{2.442155in}{1.918867in}}{\pgfqpoint{2.445428in}{1.910967in}}{\pgfqpoint{2.451251in}{1.905143in}}%
\pgfpathcurveto{\pgfqpoint{2.457075in}{1.899319in}}{\pgfqpoint{2.464975in}{1.896046in}}{\pgfqpoint{2.473212in}{1.896046in}}%
\pgfpathclose%
\pgfusepath{stroke,fill}%
\end{pgfscope}%
\begin{pgfscope}%
\pgfpathrectangle{\pgfqpoint{0.100000in}{0.212622in}}{\pgfqpoint{3.696000in}{3.696000in}}%
\pgfusepath{clip}%
\pgfsetbuttcap%
\pgfsetroundjoin%
\definecolor{currentfill}{rgb}{0.121569,0.466667,0.705882}%
\pgfsetfillcolor{currentfill}%
\pgfsetfillopacity{0.980665}%
\pgfsetlinewidth{1.003750pt}%
\definecolor{currentstroke}{rgb}{0.121569,0.466667,0.705882}%
\pgfsetstrokecolor{currentstroke}%
\pgfsetstrokeopacity{0.980665}%
\pgfsetdash{}{0pt}%
\pgfpathmoveto{\pgfqpoint{2.263413in}{1.966521in}}%
\pgfpathcurveto{\pgfqpoint{2.271649in}{1.966521in}}{\pgfqpoint{2.279549in}{1.969794in}}{\pgfqpoint{2.285373in}{1.975617in}}%
\pgfpathcurveto{\pgfqpoint{2.291197in}{1.981441in}}{\pgfqpoint{2.294470in}{1.989341in}}{\pgfqpoint{2.294470in}{1.997578in}}%
\pgfpathcurveto{\pgfqpoint{2.294470in}{2.005814in}}{\pgfqpoint{2.291197in}{2.013714in}}{\pgfqpoint{2.285373in}{2.019538in}}%
\pgfpathcurveto{\pgfqpoint{2.279549in}{2.025362in}}{\pgfqpoint{2.271649in}{2.028634in}}{\pgfqpoint{2.263413in}{2.028634in}}%
\pgfpathcurveto{\pgfqpoint{2.255177in}{2.028634in}}{\pgfqpoint{2.247277in}{2.025362in}}{\pgfqpoint{2.241453in}{2.019538in}}%
\pgfpathcurveto{\pgfqpoint{2.235629in}{2.013714in}}{\pgfqpoint{2.232357in}{2.005814in}}{\pgfqpoint{2.232357in}{1.997578in}}%
\pgfpathcurveto{\pgfqpoint{2.232357in}{1.989341in}}{\pgfqpoint{2.235629in}{1.981441in}}{\pgfqpoint{2.241453in}{1.975617in}}%
\pgfpathcurveto{\pgfqpoint{2.247277in}{1.969794in}}{\pgfqpoint{2.255177in}{1.966521in}}{\pgfqpoint{2.263413in}{1.966521in}}%
\pgfpathclose%
\pgfusepath{stroke,fill}%
\end{pgfscope}%
\begin{pgfscope}%
\pgfpathrectangle{\pgfqpoint{0.100000in}{0.212622in}}{\pgfqpoint{3.696000in}{3.696000in}}%
\pgfusepath{clip}%
\pgfsetbuttcap%
\pgfsetroundjoin%
\definecolor{currentfill}{rgb}{0.121569,0.466667,0.705882}%
\pgfsetfillcolor{currentfill}%
\pgfsetfillopacity{0.981417}%
\pgfsetlinewidth{1.003750pt}%
\definecolor{currentstroke}{rgb}{0.121569,0.466667,0.705882}%
\pgfsetstrokecolor{currentstroke}%
\pgfsetstrokeopacity{0.981417}%
\pgfsetdash{}{0pt}%
\pgfpathmoveto{\pgfqpoint{2.271590in}{1.964643in}}%
\pgfpathcurveto{\pgfqpoint{2.279826in}{1.964643in}}{\pgfqpoint{2.287726in}{1.967915in}}{\pgfqpoint{2.293550in}{1.973739in}}%
\pgfpathcurveto{\pgfqpoint{2.299374in}{1.979563in}}{\pgfqpoint{2.302646in}{1.987463in}}{\pgfqpoint{2.302646in}{1.995700in}}%
\pgfpathcurveto{\pgfqpoint{2.302646in}{2.003936in}}{\pgfqpoint{2.299374in}{2.011836in}}{\pgfqpoint{2.293550in}{2.017660in}}%
\pgfpathcurveto{\pgfqpoint{2.287726in}{2.023484in}}{\pgfqpoint{2.279826in}{2.026756in}}{\pgfqpoint{2.271590in}{2.026756in}}%
\pgfpathcurveto{\pgfqpoint{2.263354in}{2.026756in}}{\pgfqpoint{2.255454in}{2.023484in}}{\pgfqpoint{2.249630in}{2.017660in}}%
\pgfpathcurveto{\pgfqpoint{2.243806in}{2.011836in}}{\pgfqpoint{2.240533in}{2.003936in}}{\pgfqpoint{2.240533in}{1.995700in}}%
\pgfpathcurveto{\pgfqpoint{2.240533in}{1.987463in}}{\pgfqpoint{2.243806in}{1.979563in}}{\pgfqpoint{2.249630in}{1.973739in}}%
\pgfpathcurveto{\pgfqpoint{2.255454in}{1.967915in}}{\pgfqpoint{2.263354in}{1.964643in}}{\pgfqpoint{2.271590in}{1.964643in}}%
\pgfpathclose%
\pgfusepath{stroke,fill}%
\end{pgfscope}%
\begin{pgfscope}%
\pgfpathrectangle{\pgfqpoint{0.100000in}{0.212622in}}{\pgfqpoint{3.696000in}{3.696000in}}%
\pgfusepath{clip}%
\pgfsetbuttcap%
\pgfsetroundjoin%
\definecolor{currentfill}{rgb}{0.121569,0.466667,0.705882}%
\pgfsetfillcolor{currentfill}%
\pgfsetfillopacity{0.982364}%
\pgfsetlinewidth{1.003750pt}%
\definecolor{currentstroke}{rgb}{0.121569,0.466667,0.705882}%
\pgfsetstrokecolor{currentstroke}%
\pgfsetstrokeopacity{0.982364}%
\pgfsetdash{}{0pt}%
\pgfpathmoveto{\pgfqpoint{2.469275in}{1.896593in}}%
\pgfpathcurveto{\pgfqpoint{2.477511in}{1.896593in}}{\pgfqpoint{2.485411in}{1.899865in}}{\pgfqpoint{2.491235in}{1.905689in}}%
\pgfpathcurveto{\pgfqpoint{2.497059in}{1.911513in}}{\pgfqpoint{2.500331in}{1.919413in}}{\pgfqpoint{2.500331in}{1.927649in}}%
\pgfpathcurveto{\pgfqpoint{2.500331in}{1.935885in}}{\pgfqpoint{2.497059in}{1.943786in}}{\pgfqpoint{2.491235in}{1.949609in}}%
\pgfpathcurveto{\pgfqpoint{2.485411in}{1.955433in}}{\pgfqpoint{2.477511in}{1.958706in}}{\pgfqpoint{2.469275in}{1.958706in}}%
\pgfpathcurveto{\pgfqpoint{2.461038in}{1.958706in}}{\pgfqpoint{2.453138in}{1.955433in}}{\pgfqpoint{2.447314in}{1.949609in}}%
\pgfpathcurveto{\pgfqpoint{2.441490in}{1.943786in}}{\pgfqpoint{2.438218in}{1.935885in}}{\pgfqpoint{2.438218in}{1.927649in}}%
\pgfpathcurveto{\pgfqpoint{2.438218in}{1.919413in}}{\pgfqpoint{2.441490in}{1.911513in}}{\pgfqpoint{2.447314in}{1.905689in}}%
\pgfpathcurveto{\pgfqpoint{2.453138in}{1.899865in}}{\pgfqpoint{2.461038in}{1.896593in}}{\pgfqpoint{2.469275in}{1.896593in}}%
\pgfpathclose%
\pgfusepath{stroke,fill}%
\end{pgfscope}%
\begin{pgfscope}%
\pgfpathrectangle{\pgfqpoint{0.100000in}{0.212622in}}{\pgfqpoint{3.696000in}{3.696000in}}%
\pgfusepath{clip}%
\pgfsetbuttcap%
\pgfsetroundjoin%
\definecolor{currentfill}{rgb}{0.121569,0.466667,0.705882}%
\pgfsetfillcolor{currentfill}%
\pgfsetfillopacity{0.983041}%
\pgfsetlinewidth{1.003750pt}%
\definecolor{currentstroke}{rgb}{0.121569,0.466667,0.705882}%
\pgfsetstrokecolor{currentstroke}%
\pgfsetstrokeopacity{0.983041}%
\pgfsetdash{}{0pt}%
\pgfpathmoveto{\pgfqpoint{2.285882in}{1.960532in}}%
\pgfpathcurveto{\pgfqpoint{2.294118in}{1.960532in}}{\pgfqpoint{2.302018in}{1.963804in}}{\pgfqpoint{2.307842in}{1.969628in}}%
\pgfpathcurveto{\pgfqpoint{2.313666in}{1.975452in}}{\pgfqpoint{2.316938in}{1.983352in}}{\pgfqpoint{2.316938in}{1.991589in}}%
\pgfpathcurveto{\pgfqpoint{2.316938in}{1.999825in}}{\pgfqpoint{2.313666in}{2.007725in}}{\pgfqpoint{2.307842in}{2.013549in}}%
\pgfpathcurveto{\pgfqpoint{2.302018in}{2.019373in}}{\pgfqpoint{2.294118in}{2.022645in}}{\pgfqpoint{2.285882in}{2.022645in}}%
\pgfpathcurveto{\pgfqpoint{2.277646in}{2.022645in}}{\pgfqpoint{2.269745in}{2.019373in}}{\pgfqpoint{2.263922in}{2.013549in}}%
\pgfpathcurveto{\pgfqpoint{2.258098in}{2.007725in}}{\pgfqpoint{2.254825in}{1.999825in}}{\pgfqpoint{2.254825in}{1.991589in}}%
\pgfpathcurveto{\pgfqpoint{2.254825in}{1.983352in}}{\pgfqpoint{2.258098in}{1.975452in}}{\pgfqpoint{2.263922in}{1.969628in}}%
\pgfpathcurveto{\pgfqpoint{2.269745in}{1.963804in}}{\pgfqpoint{2.277646in}{1.960532in}}{\pgfqpoint{2.285882in}{1.960532in}}%
\pgfpathclose%
\pgfusepath{stroke,fill}%
\end{pgfscope}%
\begin{pgfscope}%
\pgfpathrectangle{\pgfqpoint{0.100000in}{0.212622in}}{\pgfqpoint{3.696000in}{3.696000in}}%
\pgfusepath{clip}%
\pgfsetbuttcap%
\pgfsetroundjoin%
\definecolor{currentfill}{rgb}{0.121569,0.466667,0.705882}%
\pgfsetfillcolor{currentfill}%
\pgfsetfillopacity{0.983961}%
\pgfsetlinewidth{1.003750pt}%
\definecolor{currentstroke}{rgb}{0.121569,0.466667,0.705882}%
\pgfsetstrokecolor{currentstroke}%
\pgfsetstrokeopacity{0.983961}%
\pgfsetdash{}{0pt}%
\pgfpathmoveto{\pgfqpoint{2.297696in}{1.953419in}}%
\pgfpathcurveto{\pgfqpoint{2.305932in}{1.953419in}}{\pgfqpoint{2.313832in}{1.956691in}}{\pgfqpoint{2.319656in}{1.962515in}}%
\pgfpathcurveto{\pgfqpoint{2.325480in}{1.968339in}}{\pgfqpoint{2.328752in}{1.976239in}}{\pgfqpoint{2.328752in}{1.984475in}}%
\pgfpathcurveto{\pgfqpoint{2.328752in}{1.992712in}}{\pgfqpoint{2.325480in}{2.000612in}}{\pgfqpoint{2.319656in}{2.006436in}}%
\pgfpathcurveto{\pgfqpoint{2.313832in}{2.012259in}}{\pgfqpoint{2.305932in}{2.015532in}}{\pgfqpoint{2.297696in}{2.015532in}}%
\pgfpathcurveto{\pgfqpoint{2.289459in}{2.015532in}}{\pgfqpoint{2.281559in}{2.012259in}}{\pgfqpoint{2.275735in}{2.006436in}}%
\pgfpathcurveto{\pgfqpoint{2.269911in}{2.000612in}}{\pgfqpoint{2.266639in}{1.992712in}}{\pgfqpoint{2.266639in}{1.984475in}}%
\pgfpathcurveto{\pgfqpoint{2.266639in}{1.976239in}}{\pgfqpoint{2.269911in}{1.968339in}}{\pgfqpoint{2.275735in}{1.962515in}}%
\pgfpathcurveto{\pgfqpoint{2.281559in}{1.956691in}}{\pgfqpoint{2.289459in}{1.953419in}}{\pgfqpoint{2.297696in}{1.953419in}}%
\pgfpathclose%
\pgfusepath{stroke,fill}%
\end{pgfscope}%
\begin{pgfscope}%
\pgfpathrectangle{\pgfqpoint{0.100000in}{0.212622in}}{\pgfqpoint{3.696000in}{3.696000in}}%
\pgfusepath{clip}%
\pgfsetbuttcap%
\pgfsetroundjoin%
\definecolor{currentfill}{rgb}{0.121569,0.466667,0.705882}%
\pgfsetfillcolor{currentfill}%
\pgfsetfillopacity{0.984476}%
\pgfsetlinewidth{1.003750pt}%
\definecolor{currentstroke}{rgb}{0.121569,0.466667,0.705882}%
\pgfsetstrokecolor{currentstroke}%
\pgfsetstrokeopacity{0.984476}%
\pgfsetdash{}{0pt}%
\pgfpathmoveto{\pgfqpoint{2.466198in}{1.896787in}}%
\pgfpathcurveto{\pgfqpoint{2.474434in}{1.896787in}}{\pgfqpoint{2.482334in}{1.900060in}}{\pgfqpoint{2.488158in}{1.905883in}}%
\pgfpathcurveto{\pgfqpoint{2.493982in}{1.911707in}}{\pgfqpoint{2.497255in}{1.919607in}}{\pgfqpoint{2.497255in}{1.927844in}}%
\pgfpathcurveto{\pgfqpoint{2.497255in}{1.936080in}}{\pgfqpoint{2.493982in}{1.943980in}}{\pgfqpoint{2.488158in}{1.949804in}}%
\pgfpathcurveto{\pgfqpoint{2.482334in}{1.955628in}}{\pgfqpoint{2.474434in}{1.958900in}}{\pgfqpoint{2.466198in}{1.958900in}}%
\pgfpathcurveto{\pgfqpoint{2.457962in}{1.958900in}}{\pgfqpoint{2.450062in}{1.955628in}}{\pgfqpoint{2.444238in}{1.949804in}}%
\pgfpathcurveto{\pgfqpoint{2.438414in}{1.943980in}}{\pgfqpoint{2.435142in}{1.936080in}}{\pgfqpoint{2.435142in}{1.927844in}}%
\pgfpathcurveto{\pgfqpoint{2.435142in}{1.919607in}}{\pgfqpoint{2.438414in}{1.911707in}}{\pgfqpoint{2.444238in}{1.905883in}}%
\pgfpathcurveto{\pgfqpoint{2.450062in}{1.900060in}}{\pgfqpoint{2.457962in}{1.896787in}}{\pgfqpoint{2.466198in}{1.896787in}}%
\pgfpathclose%
\pgfusepath{stroke,fill}%
\end{pgfscope}%
\begin{pgfscope}%
\pgfpathrectangle{\pgfqpoint{0.100000in}{0.212622in}}{\pgfqpoint{3.696000in}{3.696000in}}%
\pgfusepath{clip}%
\pgfsetbuttcap%
\pgfsetroundjoin%
\definecolor{currentfill}{rgb}{0.121569,0.466667,0.705882}%
\pgfsetfillcolor{currentfill}%
\pgfsetfillopacity{0.984872}%
\pgfsetlinewidth{1.003750pt}%
\definecolor{currentstroke}{rgb}{0.121569,0.466667,0.705882}%
\pgfsetstrokecolor{currentstroke}%
\pgfsetstrokeopacity{0.984872}%
\pgfsetdash{}{0pt}%
\pgfpathmoveto{\pgfqpoint{2.306893in}{1.951250in}}%
\pgfpathcurveto{\pgfqpoint{2.315130in}{1.951250in}}{\pgfqpoint{2.323030in}{1.954523in}}{\pgfqpoint{2.328854in}{1.960347in}}%
\pgfpathcurveto{\pgfqpoint{2.334678in}{1.966171in}}{\pgfqpoint{2.337950in}{1.974071in}}{\pgfqpoint{2.337950in}{1.982307in}}%
\pgfpathcurveto{\pgfqpoint{2.337950in}{1.990543in}}{\pgfqpoint{2.334678in}{1.998443in}}{\pgfqpoint{2.328854in}{2.004267in}}%
\pgfpathcurveto{\pgfqpoint{2.323030in}{2.010091in}}{\pgfqpoint{2.315130in}{2.013363in}}{\pgfqpoint{2.306893in}{2.013363in}}%
\pgfpathcurveto{\pgfqpoint{2.298657in}{2.013363in}}{\pgfqpoint{2.290757in}{2.010091in}}{\pgfqpoint{2.284933in}{2.004267in}}%
\pgfpathcurveto{\pgfqpoint{2.279109in}{1.998443in}}{\pgfqpoint{2.275837in}{1.990543in}}{\pgfqpoint{2.275837in}{1.982307in}}%
\pgfpathcurveto{\pgfqpoint{2.275837in}{1.974071in}}{\pgfqpoint{2.279109in}{1.966171in}}{\pgfqpoint{2.284933in}{1.960347in}}%
\pgfpathcurveto{\pgfqpoint{2.290757in}{1.954523in}}{\pgfqpoint{2.298657in}{1.951250in}}{\pgfqpoint{2.306893in}{1.951250in}}%
\pgfpathclose%
\pgfusepath{stroke,fill}%
\end{pgfscope}%
\begin{pgfscope}%
\pgfpathrectangle{\pgfqpoint{0.100000in}{0.212622in}}{\pgfqpoint{3.696000in}{3.696000in}}%
\pgfusepath{clip}%
\pgfsetbuttcap%
\pgfsetroundjoin%
\definecolor{currentfill}{rgb}{0.121569,0.466667,0.705882}%
\pgfsetfillcolor{currentfill}%
\pgfsetfillopacity{0.985615}%
\pgfsetlinewidth{1.003750pt}%
\definecolor{currentstroke}{rgb}{0.121569,0.466667,0.705882}%
\pgfsetstrokecolor{currentstroke}%
\pgfsetstrokeopacity{0.985615}%
\pgfsetdash{}{0pt}%
\pgfpathmoveto{\pgfqpoint{2.464280in}{1.896954in}}%
\pgfpathcurveto{\pgfqpoint{2.472516in}{1.896954in}}{\pgfqpoint{2.480416in}{1.900227in}}{\pgfqpoint{2.486240in}{1.906051in}}%
\pgfpathcurveto{\pgfqpoint{2.492064in}{1.911874in}}{\pgfqpoint{2.495336in}{1.919775in}}{\pgfqpoint{2.495336in}{1.928011in}}%
\pgfpathcurveto{\pgfqpoint{2.495336in}{1.936247in}}{\pgfqpoint{2.492064in}{1.944147in}}{\pgfqpoint{2.486240in}{1.949971in}}%
\pgfpathcurveto{\pgfqpoint{2.480416in}{1.955795in}}{\pgfqpoint{2.472516in}{1.959067in}}{\pgfqpoint{2.464280in}{1.959067in}}%
\pgfpathcurveto{\pgfqpoint{2.456043in}{1.959067in}}{\pgfqpoint{2.448143in}{1.955795in}}{\pgfqpoint{2.442320in}{1.949971in}}%
\pgfpathcurveto{\pgfqpoint{2.436496in}{1.944147in}}{\pgfqpoint{2.433223in}{1.936247in}}{\pgfqpoint{2.433223in}{1.928011in}}%
\pgfpathcurveto{\pgfqpoint{2.433223in}{1.919775in}}{\pgfqpoint{2.436496in}{1.911874in}}{\pgfqpoint{2.442320in}{1.906051in}}%
\pgfpathcurveto{\pgfqpoint{2.448143in}{1.900227in}}{\pgfqpoint{2.456043in}{1.896954in}}{\pgfqpoint{2.464280in}{1.896954in}}%
\pgfpathclose%
\pgfusepath{stroke,fill}%
\end{pgfscope}%
\begin{pgfscope}%
\pgfpathrectangle{\pgfqpoint{0.100000in}{0.212622in}}{\pgfqpoint{3.696000in}{3.696000in}}%
\pgfusepath{clip}%
\pgfsetbuttcap%
\pgfsetroundjoin%
\definecolor{currentfill}{rgb}{0.121569,0.466667,0.705882}%
\pgfsetfillcolor{currentfill}%
\pgfsetfillopacity{0.985690}%
\pgfsetlinewidth{1.003750pt}%
\definecolor{currentstroke}{rgb}{0.121569,0.466667,0.705882}%
\pgfsetstrokecolor{currentstroke}%
\pgfsetstrokeopacity{0.985690}%
\pgfsetdash{}{0pt}%
\pgfpathmoveto{\pgfqpoint{2.315767in}{1.948635in}}%
\pgfpathcurveto{\pgfqpoint{2.324003in}{1.948635in}}{\pgfqpoint{2.331903in}{1.951907in}}{\pgfqpoint{2.337727in}{1.957731in}}%
\pgfpathcurveto{\pgfqpoint{2.343551in}{1.963555in}}{\pgfqpoint{2.346823in}{1.971455in}}{\pgfqpoint{2.346823in}{1.979691in}}%
\pgfpathcurveto{\pgfqpoint{2.346823in}{1.987927in}}{\pgfqpoint{2.343551in}{1.995827in}}{\pgfqpoint{2.337727in}{2.001651in}}%
\pgfpathcurveto{\pgfqpoint{2.331903in}{2.007475in}}{\pgfqpoint{2.324003in}{2.010748in}}{\pgfqpoint{2.315767in}{2.010748in}}%
\pgfpathcurveto{\pgfqpoint{2.307530in}{2.010748in}}{\pgfqpoint{2.299630in}{2.007475in}}{\pgfqpoint{2.293806in}{2.001651in}}%
\pgfpathcurveto{\pgfqpoint{2.287982in}{1.995827in}}{\pgfqpoint{2.284710in}{1.987927in}}{\pgfqpoint{2.284710in}{1.979691in}}%
\pgfpathcurveto{\pgfqpoint{2.284710in}{1.971455in}}{\pgfqpoint{2.287982in}{1.963555in}}{\pgfqpoint{2.293806in}{1.957731in}}%
\pgfpathcurveto{\pgfqpoint{2.299630in}{1.951907in}}{\pgfqpoint{2.307530in}{1.948635in}}{\pgfqpoint{2.315767in}{1.948635in}}%
\pgfpathclose%
\pgfusepath{stroke,fill}%
\end{pgfscope}%
\begin{pgfscope}%
\pgfpathrectangle{\pgfqpoint{0.100000in}{0.212622in}}{\pgfqpoint{3.696000in}{3.696000in}}%
\pgfusepath{clip}%
\pgfsetbuttcap%
\pgfsetroundjoin%
\definecolor{currentfill}{rgb}{0.121569,0.466667,0.705882}%
\pgfsetfillcolor{currentfill}%
\pgfsetfillopacity{0.986237}%
\pgfsetlinewidth{1.003750pt}%
\definecolor{currentstroke}{rgb}{0.121569,0.466667,0.705882}%
\pgfsetstrokecolor{currentstroke}%
\pgfsetstrokeopacity{0.986237}%
\pgfsetdash{}{0pt}%
\pgfpathmoveto{\pgfqpoint{2.323757in}{1.945415in}}%
\pgfpathcurveto{\pgfqpoint{2.331993in}{1.945415in}}{\pgfqpoint{2.339893in}{1.948688in}}{\pgfqpoint{2.345717in}{1.954512in}}%
\pgfpathcurveto{\pgfqpoint{2.351541in}{1.960336in}}{\pgfqpoint{2.354813in}{1.968236in}}{\pgfqpoint{2.354813in}{1.976472in}}%
\pgfpathcurveto{\pgfqpoint{2.354813in}{1.984708in}}{\pgfqpoint{2.351541in}{1.992608in}}{\pgfqpoint{2.345717in}{1.998432in}}%
\pgfpathcurveto{\pgfqpoint{2.339893in}{2.004256in}}{\pgfqpoint{2.331993in}{2.007528in}}{\pgfqpoint{2.323757in}{2.007528in}}%
\pgfpathcurveto{\pgfqpoint{2.315521in}{2.007528in}}{\pgfqpoint{2.307621in}{2.004256in}}{\pgfqpoint{2.301797in}{1.998432in}}%
\pgfpathcurveto{\pgfqpoint{2.295973in}{1.992608in}}{\pgfqpoint{2.292700in}{1.984708in}}{\pgfqpoint{2.292700in}{1.976472in}}%
\pgfpathcurveto{\pgfqpoint{2.292700in}{1.968236in}}{\pgfqpoint{2.295973in}{1.960336in}}{\pgfqpoint{2.301797in}{1.954512in}}%
\pgfpathcurveto{\pgfqpoint{2.307621in}{1.948688in}}{\pgfqpoint{2.315521in}{1.945415in}}{\pgfqpoint{2.323757in}{1.945415in}}%
\pgfpathclose%
\pgfusepath{stroke,fill}%
\end{pgfscope}%
\begin{pgfscope}%
\pgfpathrectangle{\pgfqpoint{0.100000in}{0.212622in}}{\pgfqpoint{3.696000in}{3.696000in}}%
\pgfusepath{clip}%
\pgfsetbuttcap%
\pgfsetroundjoin%
\definecolor{currentfill}{rgb}{0.121569,0.466667,0.705882}%
\pgfsetfillcolor{currentfill}%
\pgfsetfillopacity{0.986238}%
\pgfsetlinewidth{1.003750pt}%
\definecolor{currentstroke}{rgb}{0.121569,0.466667,0.705882}%
\pgfsetstrokecolor{currentstroke}%
\pgfsetstrokeopacity{0.986238}%
\pgfsetdash{}{0pt}%
\pgfpathmoveto{\pgfqpoint{2.463180in}{1.897066in}}%
\pgfpathcurveto{\pgfqpoint{2.471416in}{1.897066in}}{\pgfqpoint{2.479317in}{1.900339in}}{\pgfqpoint{2.485140in}{1.906163in}}%
\pgfpathcurveto{\pgfqpoint{2.490964in}{1.911987in}}{\pgfqpoint{2.494237in}{1.919887in}}{\pgfqpoint{2.494237in}{1.928123in}}%
\pgfpathcurveto{\pgfqpoint{2.494237in}{1.936359in}}{\pgfqpoint{2.490964in}{1.944259in}}{\pgfqpoint{2.485140in}{1.950083in}}%
\pgfpathcurveto{\pgfqpoint{2.479317in}{1.955907in}}{\pgfqpoint{2.471416in}{1.959179in}}{\pgfqpoint{2.463180in}{1.959179in}}%
\pgfpathcurveto{\pgfqpoint{2.454944in}{1.959179in}}{\pgfqpoint{2.447044in}{1.955907in}}{\pgfqpoint{2.441220in}{1.950083in}}%
\pgfpathcurveto{\pgfqpoint{2.435396in}{1.944259in}}{\pgfqpoint{2.432124in}{1.936359in}}{\pgfqpoint{2.432124in}{1.928123in}}%
\pgfpathcurveto{\pgfqpoint{2.432124in}{1.919887in}}{\pgfqpoint{2.435396in}{1.911987in}}{\pgfqpoint{2.441220in}{1.906163in}}%
\pgfpathcurveto{\pgfqpoint{2.447044in}{1.900339in}}{\pgfqpoint{2.454944in}{1.897066in}}{\pgfqpoint{2.463180in}{1.897066in}}%
\pgfpathclose%
\pgfusepath{stroke,fill}%
\end{pgfscope}%
\begin{pgfscope}%
\pgfpathrectangle{\pgfqpoint{0.100000in}{0.212622in}}{\pgfqpoint{3.696000in}{3.696000in}}%
\pgfusepath{clip}%
\pgfsetbuttcap%
\pgfsetroundjoin%
\definecolor{currentfill}{rgb}{0.121569,0.466667,0.705882}%
\pgfsetfillcolor{currentfill}%
\pgfsetfillopacity{0.986783}%
\pgfsetlinewidth{1.003750pt}%
\definecolor{currentstroke}{rgb}{0.121569,0.466667,0.705882}%
\pgfsetstrokecolor{currentstroke}%
\pgfsetstrokeopacity{0.986783}%
\pgfsetdash{}{0pt}%
\pgfpathmoveto{\pgfqpoint{2.328605in}{1.944362in}}%
\pgfpathcurveto{\pgfqpoint{2.336842in}{1.944362in}}{\pgfqpoint{2.344742in}{1.947634in}}{\pgfqpoint{2.350566in}{1.953458in}}%
\pgfpathcurveto{\pgfqpoint{2.356390in}{1.959282in}}{\pgfqpoint{2.359662in}{1.967182in}}{\pgfqpoint{2.359662in}{1.975419in}}%
\pgfpathcurveto{\pgfqpoint{2.359662in}{1.983655in}}{\pgfqpoint{2.356390in}{1.991555in}}{\pgfqpoint{2.350566in}{1.997379in}}%
\pgfpathcurveto{\pgfqpoint{2.344742in}{2.003203in}}{\pgfqpoint{2.336842in}{2.006475in}}{\pgfqpoint{2.328605in}{2.006475in}}%
\pgfpathcurveto{\pgfqpoint{2.320369in}{2.006475in}}{\pgfqpoint{2.312469in}{2.003203in}}{\pgfqpoint{2.306645in}{1.997379in}}%
\pgfpathcurveto{\pgfqpoint{2.300821in}{1.991555in}}{\pgfqpoint{2.297549in}{1.983655in}}{\pgfqpoint{2.297549in}{1.975419in}}%
\pgfpathcurveto{\pgfqpoint{2.297549in}{1.967182in}}{\pgfqpoint{2.300821in}{1.959282in}}{\pgfqpoint{2.306645in}{1.953458in}}%
\pgfpathcurveto{\pgfqpoint{2.312469in}{1.947634in}}{\pgfqpoint{2.320369in}{1.944362in}}{\pgfqpoint{2.328605in}{1.944362in}}%
\pgfpathclose%
\pgfusepath{stroke,fill}%
\end{pgfscope}%
\begin{pgfscope}%
\pgfpathrectangle{\pgfqpoint{0.100000in}{0.212622in}}{\pgfqpoint{3.696000in}{3.696000in}}%
\pgfusepath{clip}%
\pgfsetbuttcap%
\pgfsetroundjoin%
\definecolor{currentfill}{rgb}{0.121569,0.466667,0.705882}%
\pgfsetfillcolor{currentfill}%
\pgfsetfillopacity{0.986968}%
\pgfsetlinewidth{1.003750pt}%
\definecolor{currentstroke}{rgb}{0.121569,0.466667,0.705882}%
\pgfsetstrokecolor{currentstroke}%
\pgfsetstrokeopacity{0.986968}%
\pgfsetdash{}{0pt}%
\pgfpathmoveto{\pgfqpoint{2.461909in}{1.897164in}}%
\pgfpathcurveto{\pgfqpoint{2.470145in}{1.897164in}}{\pgfqpoint{2.478045in}{1.900436in}}{\pgfqpoint{2.483869in}{1.906260in}}%
\pgfpathcurveto{\pgfqpoint{2.489693in}{1.912084in}}{\pgfqpoint{2.492966in}{1.919984in}}{\pgfqpoint{2.492966in}{1.928221in}}%
\pgfpathcurveto{\pgfqpoint{2.492966in}{1.936457in}}{\pgfqpoint{2.489693in}{1.944357in}}{\pgfqpoint{2.483869in}{1.950181in}}%
\pgfpathcurveto{\pgfqpoint{2.478045in}{1.956005in}}{\pgfqpoint{2.470145in}{1.959277in}}{\pgfqpoint{2.461909in}{1.959277in}}%
\pgfpathcurveto{\pgfqpoint{2.453673in}{1.959277in}}{\pgfqpoint{2.445773in}{1.956005in}}{\pgfqpoint{2.439949in}{1.950181in}}%
\pgfpathcurveto{\pgfqpoint{2.434125in}{1.944357in}}{\pgfqpoint{2.430853in}{1.936457in}}{\pgfqpoint{2.430853in}{1.928221in}}%
\pgfpathcurveto{\pgfqpoint{2.430853in}{1.919984in}}{\pgfqpoint{2.434125in}{1.912084in}}{\pgfqpoint{2.439949in}{1.906260in}}%
\pgfpathcurveto{\pgfqpoint{2.445773in}{1.900436in}}{\pgfqpoint{2.453673in}{1.897164in}}{\pgfqpoint{2.461909in}{1.897164in}}%
\pgfpathclose%
\pgfusepath{stroke,fill}%
\end{pgfscope}%
\begin{pgfscope}%
\pgfpathrectangle{\pgfqpoint{0.100000in}{0.212622in}}{\pgfqpoint{3.696000in}{3.696000in}}%
\pgfusepath{clip}%
\pgfsetbuttcap%
\pgfsetroundjoin%
\definecolor{currentfill}{rgb}{0.121569,0.466667,0.705882}%
\pgfsetfillcolor{currentfill}%
\pgfsetfillopacity{0.987129}%
\pgfsetlinewidth{1.003750pt}%
\definecolor{currentstroke}{rgb}{0.121569,0.466667,0.705882}%
\pgfsetstrokecolor{currentstroke}%
\pgfsetstrokeopacity{0.987129}%
\pgfsetdash{}{0pt}%
\pgfpathmoveto{\pgfqpoint{2.332987in}{1.943035in}}%
\pgfpathcurveto{\pgfqpoint{2.341223in}{1.943035in}}{\pgfqpoint{2.349123in}{1.946307in}}{\pgfqpoint{2.354947in}{1.952131in}}%
\pgfpathcurveto{\pgfqpoint{2.360771in}{1.957955in}}{\pgfqpoint{2.364043in}{1.965855in}}{\pgfqpoint{2.364043in}{1.974091in}}%
\pgfpathcurveto{\pgfqpoint{2.364043in}{1.982327in}}{\pgfqpoint{2.360771in}{1.990228in}}{\pgfqpoint{2.354947in}{1.996051in}}%
\pgfpathcurveto{\pgfqpoint{2.349123in}{2.001875in}}{\pgfqpoint{2.341223in}{2.005148in}}{\pgfqpoint{2.332987in}{2.005148in}}%
\pgfpathcurveto{\pgfqpoint{2.324751in}{2.005148in}}{\pgfqpoint{2.316850in}{2.001875in}}{\pgfqpoint{2.311027in}{1.996051in}}%
\pgfpathcurveto{\pgfqpoint{2.305203in}{1.990228in}}{\pgfqpoint{2.301930in}{1.982327in}}{\pgfqpoint{2.301930in}{1.974091in}}%
\pgfpathcurveto{\pgfqpoint{2.301930in}{1.965855in}}{\pgfqpoint{2.305203in}{1.957955in}}{\pgfqpoint{2.311027in}{1.952131in}}%
\pgfpathcurveto{\pgfqpoint{2.316850in}{1.946307in}}{\pgfqpoint{2.324751in}{1.943035in}}{\pgfqpoint{2.332987in}{1.943035in}}%
\pgfpathclose%
\pgfusepath{stroke,fill}%
\end{pgfscope}%
\begin{pgfscope}%
\pgfpathrectangle{\pgfqpoint{0.100000in}{0.212622in}}{\pgfqpoint{3.696000in}{3.696000in}}%
\pgfusepath{clip}%
\pgfsetbuttcap%
\pgfsetroundjoin%
\definecolor{currentfill}{rgb}{0.121569,0.466667,0.705882}%
\pgfsetfillcolor{currentfill}%
\pgfsetfillopacity{0.987348}%
\pgfsetlinewidth{1.003750pt}%
\definecolor{currentstroke}{rgb}{0.121569,0.466667,0.705882}%
\pgfsetstrokecolor{currentstroke}%
\pgfsetstrokeopacity{0.987348}%
\pgfsetdash{}{0pt}%
\pgfpathmoveto{\pgfqpoint{2.337109in}{1.941521in}}%
\pgfpathcurveto{\pgfqpoint{2.345346in}{1.941521in}}{\pgfqpoint{2.353246in}{1.944794in}}{\pgfqpoint{2.359070in}{1.950618in}}%
\pgfpathcurveto{\pgfqpoint{2.364894in}{1.956441in}}{\pgfqpoint{2.368166in}{1.964342in}}{\pgfqpoint{2.368166in}{1.972578in}}%
\pgfpathcurveto{\pgfqpoint{2.368166in}{1.980814in}}{\pgfqpoint{2.364894in}{1.988714in}}{\pgfqpoint{2.359070in}{1.994538in}}%
\pgfpathcurveto{\pgfqpoint{2.353246in}{2.000362in}}{\pgfqpoint{2.345346in}{2.003634in}}{\pgfqpoint{2.337109in}{2.003634in}}%
\pgfpathcurveto{\pgfqpoint{2.328873in}{2.003634in}}{\pgfqpoint{2.320973in}{2.000362in}}{\pgfqpoint{2.315149in}{1.994538in}}%
\pgfpathcurveto{\pgfqpoint{2.309325in}{1.988714in}}{\pgfqpoint{2.306053in}{1.980814in}}{\pgfqpoint{2.306053in}{1.972578in}}%
\pgfpathcurveto{\pgfqpoint{2.306053in}{1.964342in}}{\pgfqpoint{2.309325in}{1.956441in}}{\pgfqpoint{2.315149in}{1.950618in}}%
\pgfpathcurveto{\pgfqpoint{2.320973in}{1.944794in}}{\pgfqpoint{2.328873in}{1.941521in}}{\pgfqpoint{2.337109in}{1.941521in}}%
\pgfpathclose%
\pgfusepath{stroke,fill}%
\end{pgfscope}%
\begin{pgfscope}%
\pgfpathrectangle{\pgfqpoint{0.100000in}{0.212622in}}{\pgfqpoint{3.696000in}{3.696000in}}%
\pgfusepath{clip}%
\pgfsetbuttcap%
\pgfsetroundjoin%
\definecolor{currentfill}{rgb}{0.121569,0.466667,0.705882}%
\pgfsetfillcolor{currentfill}%
\pgfsetfillopacity{0.987577}%
\pgfsetlinewidth{1.003750pt}%
\definecolor{currentstroke}{rgb}{0.121569,0.466667,0.705882}%
\pgfsetstrokecolor{currentstroke}%
\pgfsetstrokeopacity{0.987577}%
\pgfsetdash{}{0pt}%
\pgfpathmoveto{\pgfqpoint{2.339702in}{1.940803in}}%
\pgfpathcurveto{\pgfqpoint{2.347939in}{1.940803in}}{\pgfqpoint{2.355839in}{1.944075in}}{\pgfqpoint{2.361663in}{1.949899in}}%
\pgfpathcurveto{\pgfqpoint{2.367487in}{1.955723in}}{\pgfqpoint{2.370759in}{1.963623in}}{\pgfqpoint{2.370759in}{1.971860in}}%
\pgfpathcurveto{\pgfqpoint{2.370759in}{1.980096in}}{\pgfqpoint{2.367487in}{1.987996in}}{\pgfqpoint{2.361663in}{1.993820in}}%
\pgfpathcurveto{\pgfqpoint{2.355839in}{1.999644in}}{\pgfqpoint{2.347939in}{2.002916in}}{\pgfqpoint{2.339702in}{2.002916in}}%
\pgfpathcurveto{\pgfqpoint{2.331466in}{2.002916in}}{\pgfqpoint{2.323566in}{1.999644in}}{\pgfqpoint{2.317742in}{1.993820in}}%
\pgfpathcurveto{\pgfqpoint{2.311918in}{1.987996in}}{\pgfqpoint{2.308646in}{1.980096in}}{\pgfqpoint{2.308646in}{1.971860in}}%
\pgfpathcurveto{\pgfqpoint{2.308646in}{1.963623in}}{\pgfqpoint{2.311918in}{1.955723in}}{\pgfqpoint{2.317742in}{1.949899in}}%
\pgfpathcurveto{\pgfqpoint{2.323566in}{1.944075in}}{\pgfqpoint{2.331466in}{1.940803in}}{\pgfqpoint{2.339702in}{1.940803in}}%
\pgfpathclose%
\pgfusepath{stroke,fill}%
\end{pgfscope}%
\begin{pgfscope}%
\pgfpathrectangle{\pgfqpoint{0.100000in}{0.212622in}}{\pgfqpoint{3.696000in}{3.696000in}}%
\pgfusepath{clip}%
\pgfsetbuttcap%
\pgfsetroundjoin%
\definecolor{currentfill}{rgb}{0.121569,0.466667,0.705882}%
\pgfsetfillcolor{currentfill}%
\pgfsetfillopacity{0.987815}%
\pgfsetlinewidth{1.003750pt}%
\definecolor{currentstroke}{rgb}{0.121569,0.466667,0.705882}%
\pgfsetstrokecolor{currentstroke}%
\pgfsetstrokeopacity{0.987815}%
\pgfsetdash{}{0pt}%
\pgfpathmoveto{\pgfqpoint{2.460245in}{1.897390in}}%
\pgfpathcurveto{\pgfqpoint{2.468481in}{1.897390in}}{\pgfqpoint{2.476381in}{1.900662in}}{\pgfqpoint{2.482205in}{1.906486in}}%
\pgfpathcurveto{\pgfqpoint{2.488029in}{1.912310in}}{\pgfqpoint{2.491302in}{1.920210in}}{\pgfqpoint{2.491302in}{1.928446in}}%
\pgfpathcurveto{\pgfqpoint{2.491302in}{1.936683in}}{\pgfqpoint{2.488029in}{1.944583in}}{\pgfqpoint{2.482205in}{1.950407in}}%
\pgfpathcurveto{\pgfqpoint{2.476381in}{1.956231in}}{\pgfqpoint{2.468481in}{1.959503in}}{\pgfqpoint{2.460245in}{1.959503in}}%
\pgfpathcurveto{\pgfqpoint{2.452009in}{1.959503in}}{\pgfqpoint{2.444109in}{1.956231in}}{\pgfqpoint{2.438285in}{1.950407in}}%
\pgfpathcurveto{\pgfqpoint{2.432461in}{1.944583in}}{\pgfqpoint{2.429189in}{1.936683in}}{\pgfqpoint{2.429189in}{1.928446in}}%
\pgfpathcurveto{\pgfqpoint{2.429189in}{1.920210in}}{\pgfqpoint{2.432461in}{1.912310in}}{\pgfqpoint{2.438285in}{1.906486in}}%
\pgfpathcurveto{\pgfqpoint{2.444109in}{1.900662in}}{\pgfqpoint{2.452009in}{1.897390in}}{\pgfqpoint{2.460245in}{1.897390in}}%
\pgfpathclose%
\pgfusepath{stroke,fill}%
\end{pgfscope}%
\begin{pgfscope}%
\pgfpathrectangle{\pgfqpoint{0.100000in}{0.212622in}}{\pgfqpoint{3.696000in}{3.696000in}}%
\pgfusepath{clip}%
\pgfsetbuttcap%
\pgfsetroundjoin%
\definecolor{currentfill}{rgb}{0.121569,0.466667,0.705882}%
\pgfsetfillcolor{currentfill}%
\pgfsetfillopacity{0.987994}%
\pgfsetlinewidth{1.003750pt}%
\definecolor{currentstroke}{rgb}{0.121569,0.466667,0.705882}%
\pgfsetstrokecolor{currentstroke}%
\pgfsetstrokeopacity{0.987994}%
\pgfsetdash{}{0pt}%
\pgfpathmoveto{\pgfqpoint{2.344505in}{1.939923in}}%
\pgfpathcurveto{\pgfqpoint{2.352741in}{1.939923in}}{\pgfqpoint{2.360641in}{1.943195in}}{\pgfqpoint{2.366465in}{1.949019in}}%
\pgfpathcurveto{\pgfqpoint{2.372289in}{1.954843in}}{\pgfqpoint{2.375561in}{1.962743in}}{\pgfqpoint{2.375561in}{1.970979in}}%
\pgfpathcurveto{\pgfqpoint{2.375561in}{1.979216in}}{\pgfqpoint{2.372289in}{1.987116in}}{\pgfqpoint{2.366465in}{1.992940in}}%
\pgfpathcurveto{\pgfqpoint{2.360641in}{1.998764in}}{\pgfqpoint{2.352741in}{2.002036in}}{\pgfqpoint{2.344505in}{2.002036in}}%
\pgfpathcurveto{\pgfqpoint{2.336269in}{2.002036in}}{\pgfqpoint{2.328369in}{1.998764in}}{\pgfqpoint{2.322545in}{1.992940in}}%
\pgfpathcurveto{\pgfqpoint{2.316721in}{1.987116in}}{\pgfqpoint{2.313448in}{1.979216in}}{\pgfqpoint{2.313448in}{1.970979in}}%
\pgfpathcurveto{\pgfqpoint{2.313448in}{1.962743in}}{\pgfqpoint{2.316721in}{1.954843in}}{\pgfqpoint{2.322545in}{1.949019in}}%
\pgfpathcurveto{\pgfqpoint{2.328369in}{1.943195in}}{\pgfqpoint{2.336269in}{1.939923in}}{\pgfqpoint{2.344505in}{1.939923in}}%
\pgfpathclose%
\pgfusepath{stroke,fill}%
\end{pgfscope}%
\begin{pgfscope}%
\pgfpathrectangle{\pgfqpoint{0.100000in}{0.212622in}}{\pgfqpoint{3.696000in}{3.696000in}}%
\pgfusepath{clip}%
\pgfsetbuttcap%
\pgfsetroundjoin%
\definecolor{currentfill}{rgb}{0.121569,0.466667,0.705882}%
\pgfsetfillcolor{currentfill}%
\pgfsetfillopacity{0.988253}%
\pgfsetlinewidth{1.003750pt}%
\definecolor{currentstroke}{rgb}{0.121569,0.466667,0.705882}%
\pgfsetstrokecolor{currentstroke}%
\pgfsetstrokeopacity{0.988253}%
\pgfsetdash{}{0pt}%
\pgfpathmoveto{\pgfqpoint{2.459099in}{1.897666in}}%
\pgfpathcurveto{\pgfqpoint{2.467336in}{1.897666in}}{\pgfqpoint{2.475236in}{1.900938in}}{\pgfqpoint{2.481060in}{1.906762in}}%
\pgfpathcurveto{\pgfqpoint{2.486884in}{1.912586in}}{\pgfqpoint{2.490156in}{1.920486in}}{\pgfqpoint{2.490156in}{1.928722in}}%
\pgfpathcurveto{\pgfqpoint{2.490156in}{1.936959in}}{\pgfqpoint{2.486884in}{1.944859in}}{\pgfqpoint{2.481060in}{1.950683in}}%
\pgfpathcurveto{\pgfqpoint{2.475236in}{1.956506in}}{\pgfqpoint{2.467336in}{1.959779in}}{\pgfqpoint{2.459099in}{1.959779in}}%
\pgfpathcurveto{\pgfqpoint{2.450863in}{1.959779in}}{\pgfqpoint{2.442963in}{1.956506in}}{\pgfqpoint{2.437139in}{1.950683in}}%
\pgfpathcurveto{\pgfqpoint{2.431315in}{1.944859in}}{\pgfqpoint{2.428043in}{1.936959in}}{\pgfqpoint{2.428043in}{1.928722in}}%
\pgfpathcurveto{\pgfqpoint{2.428043in}{1.920486in}}{\pgfqpoint{2.431315in}{1.912586in}}{\pgfqpoint{2.437139in}{1.906762in}}%
\pgfpathcurveto{\pgfqpoint{2.442963in}{1.900938in}}{\pgfqpoint{2.450863in}{1.897666in}}{\pgfqpoint{2.459099in}{1.897666in}}%
\pgfpathclose%
\pgfusepath{stroke,fill}%
\end{pgfscope}%
\begin{pgfscope}%
\pgfpathrectangle{\pgfqpoint{0.100000in}{0.212622in}}{\pgfqpoint{3.696000in}{3.696000in}}%
\pgfusepath{clip}%
\pgfsetbuttcap%
\pgfsetroundjoin%
\definecolor{currentfill}{rgb}{0.121569,0.466667,0.705882}%
\pgfsetfillcolor{currentfill}%
\pgfsetfillopacity{0.988725}%
\pgfsetlinewidth{1.003750pt}%
\definecolor{currentstroke}{rgb}{0.121569,0.466667,0.705882}%
\pgfsetstrokecolor{currentstroke}%
\pgfsetstrokeopacity{0.988725}%
\pgfsetdash{}{0pt}%
\pgfpathmoveto{\pgfqpoint{2.352984in}{1.936969in}}%
\pgfpathcurveto{\pgfqpoint{2.361221in}{1.936969in}}{\pgfqpoint{2.369121in}{1.940241in}}{\pgfqpoint{2.374945in}{1.946065in}}%
\pgfpathcurveto{\pgfqpoint{2.380769in}{1.951889in}}{\pgfqpoint{2.384041in}{1.959789in}}{\pgfqpoint{2.384041in}{1.968026in}}%
\pgfpathcurveto{\pgfqpoint{2.384041in}{1.976262in}}{\pgfqpoint{2.380769in}{1.984162in}}{\pgfqpoint{2.374945in}{1.989986in}}%
\pgfpathcurveto{\pgfqpoint{2.369121in}{1.995810in}}{\pgfqpoint{2.361221in}{1.999082in}}{\pgfqpoint{2.352984in}{1.999082in}}%
\pgfpathcurveto{\pgfqpoint{2.344748in}{1.999082in}}{\pgfqpoint{2.336848in}{1.995810in}}{\pgfqpoint{2.331024in}{1.989986in}}%
\pgfpathcurveto{\pgfqpoint{2.325200in}{1.984162in}}{\pgfqpoint{2.321928in}{1.976262in}}{\pgfqpoint{2.321928in}{1.968026in}}%
\pgfpathcurveto{\pgfqpoint{2.321928in}{1.959789in}}{\pgfqpoint{2.325200in}{1.951889in}}{\pgfqpoint{2.331024in}{1.946065in}}%
\pgfpathcurveto{\pgfqpoint{2.336848in}{1.940241in}}{\pgfqpoint{2.344748in}{1.936969in}}{\pgfqpoint{2.352984in}{1.936969in}}%
\pgfpathclose%
\pgfusepath{stroke,fill}%
\end{pgfscope}%
\begin{pgfscope}%
\pgfpathrectangle{\pgfqpoint{0.100000in}{0.212622in}}{\pgfqpoint{3.696000in}{3.696000in}}%
\pgfusepath{clip}%
\pgfsetbuttcap%
\pgfsetroundjoin%
\definecolor{currentfill}{rgb}{0.121569,0.466667,0.705882}%
\pgfsetfillcolor{currentfill}%
\pgfsetfillopacity{0.989442}%
\pgfsetlinewidth{1.003750pt}%
\definecolor{currentstroke}{rgb}{0.121569,0.466667,0.705882}%
\pgfsetstrokecolor{currentstroke}%
\pgfsetstrokeopacity{0.989442}%
\pgfsetdash{}{0pt}%
\pgfpathmoveto{\pgfqpoint{2.360279in}{1.935220in}}%
\pgfpathcurveto{\pgfqpoint{2.368516in}{1.935220in}}{\pgfqpoint{2.376416in}{1.938492in}}{\pgfqpoint{2.382240in}{1.944316in}}%
\pgfpathcurveto{\pgfqpoint{2.388064in}{1.950140in}}{\pgfqpoint{2.391336in}{1.958040in}}{\pgfqpoint{2.391336in}{1.966276in}}%
\pgfpathcurveto{\pgfqpoint{2.391336in}{1.974513in}}{\pgfqpoint{2.388064in}{1.982413in}}{\pgfqpoint{2.382240in}{1.988237in}}%
\pgfpathcurveto{\pgfqpoint{2.376416in}{1.994061in}}{\pgfqpoint{2.368516in}{1.997333in}}{\pgfqpoint{2.360279in}{1.997333in}}%
\pgfpathcurveto{\pgfqpoint{2.352043in}{1.997333in}}{\pgfqpoint{2.344143in}{1.994061in}}{\pgfqpoint{2.338319in}{1.988237in}}%
\pgfpathcurveto{\pgfqpoint{2.332495in}{1.982413in}}{\pgfqpoint{2.329223in}{1.974513in}}{\pgfqpoint{2.329223in}{1.966276in}}%
\pgfpathcurveto{\pgfqpoint{2.329223in}{1.958040in}}{\pgfqpoint{2.332495in}{1.950140in}}{\pgfqpoint{2.338319in}{1.944316in}}%
\pgfpathcurveto{\pgfqpoint{2.344143in}{1.938492in}}{\pgfqpoint{2.352043in}{1.935220in}}{\pgfqpoint{2.360279in}{1.935220in}}%
\pgfpathclose%
\pgfusepath{stroke,fill}%
\end{pgfscope}%
\begin{pgfscope}%
\pgfpathrectangle{\pgfqpoint{0.100000in}{0.212622in}}{\pgfqpoint{3.696000in}{3.696000in}}%
\pgfusepath{clip}%
\pgfsetbuttcap%
\pgfsetroundjoin%
\definecolor{currentfill}{rgb}{0.121569,0.466667,0.705882}%
\pgfsetfillcolor{currentfill}%
\pgfsetfillopacity{0.989565}%
\pgfsetlinewidth{1.003750pt}%
\definecolor{currentstroke}{rgb}{0.121569,0.466667,0.705882}%
\pgfsetstrokecolor{currentstroke}%
\pgfsetstrokeopacity{0.989565}%
\pgfsetdash{}{0pt}%
\pgfpathmoveto{\pgfqpoint{2.457078in}{1.897847in}}%
\pgfpathcurveto{\pgfqpoint{2.465315in}{1.897847in}}{\pgfqpoint{2.473215in}{1.901119in}}{\pgfqpoint{2.479039in}{1.906943in}}%
\pgfpathcurveto{\pgfqpoint{2.484863in}{1.912767in}}{\pgfqpoint{2.488135in}{1.920667in}}{\pgfqpoint{2.488135in}{1.928903in}}%
\pgfpathcurveto{\pgfqpoint{2.488135in}{1.937140in}}{\pgfqpoint{2.484863in}{1.945040in}}{\pgfqpoint{2.479039in}{1.950864in}}%
\pgfpathcurveto{\pgfqpoint{2.473215in}{1.956688in}}{\pgfqpoint{2.465315in}{1.959960in}}{\pgfqpoint{2.457078in}{1.959960in}}%
\pgfpathcurveto{\pgfqpoint{2.448842in}{1.959960in}}{\pgfqpoint{2.440942in}{1.956688in}}{\pgfqpoint{2.435118in}{1.950864in}}%
\pgfpathcurveto{\pgfqpoint{2.429294in}{1.945040in}}{\pgfqpoint{2.426022in}{1.937140in}}{\pgfqpoint{2.426022in}{1.928903in}}%
\pgfpathcurveto{\pgfqpoint{2.426022in}{1.920667in}}{\pgfqpoint{2.429294in}{1.912767in}}{\pgfqpoint{2.435118in}{1.906943in}}%
\pgfpathcurveto{\pgfqpoint{2.440942in}{1.901119in}}{\pgfqpoint{2.448842in}{1.897847in}}{\pgfqpoint{2.457078in}{1.897847in}}%
\pgfpathclose%
\pgfusepath{stroke,fill}%
\end{pgfscope}%
\begin{pgfscope}%
\pgfpathrectangle{\pgfqpoint{0.100000in}{0.212622in}}{\pgfqpoint{3.696000in}{3.696000in}}%
\pgfusepath{clip}%
\pgfsetbuttcap%
\pgfsetroundjoin%
\definecolor{currentfill}{rgb}{0.121569,0.466667,0.705882}%
\pgfsetfillcolor{currentfill}%
\pgfsetfillopacity{0.990388}%
\pgfsetlinewidth{1.003750pt}%
\definecolor{currentstroke}{rgb}{0.121569,0.466667,0.705882}%
\pgfsetstrokecolor{currentstroke}%
\pgfsetstrokeopacity{0.990388}%
\pgfsetdash{}{0pt}%
\pgfpathmoveto{\pgfqpoint{2.365795in}{1.932254in}}%
\pgfpathcurveto{\pgfqpoint{2.374031in}{1.932254in}}{\pgfqpoint{2.381932in}{1.935526in}}{\pgfqpoint{2.387755in}{1.941350in}}%
\pgfpathcurveto{\pgfqpoint{2.393579in}{1.947174in}}{\pgfqpoint{2.396852in}{1.955074in}}{\pgfqpoint{2.396852in}{1.963311in}}%
\pgfpathcurveto{\pgfqpoint{2.396852in}{1.971547in}}{\pgfqpoint{2.393579in}{1.979447in}}{\pgfqpoint{2.387755in}{1.985271in}}%
\pgfpathcurveto{\pgfqpoint{2.381932in}{1.991095in}}{\pgfqpoint{2.374031in}{1.994367in}}{\pgfqpoint{2.365795in}{1.994367in}}%
\pgfpathcurveto{\pgfqpoint{2.357559in}{1.994367in}}{\pgfqpoint{2.349659in}{1.991095in}}{\pgfqpoint{2.343835in}{1.985271in}}%
\pgfpathcurveto{\pgfqpoint{2.338011in}{1.979447in}}{\pgfqpoint{2.334739in}{1.971547in}}{\pgfqpoint{2.334739in}{1.963311in}}%
\pgfpathcurveto{\pgfqpoint{2.334739in}{1.955074in}}{\pgfqpoint{2.338011in}{1.947174in}}{\pgfqpoint{2.343835in}{1.941350in}}%
\pgfpathcurveto{\pgfqpoint{2.349659in}{1.935526in}}{\pgfqpoint{2.357559in}{1.932254in}}{\pgfqpoint{2.365795in}{1.932254in}}%
\pgfpathclose%
\pgfusepath{stroke,fill}%
\end{pgfscope}%
\begin{pgfscope}%
\pgfpathrectangle{\pgfqpoint{0.100000in}{0.212622in}}{\pgfqpoint{3.696000in}{3.696000in}}%
\pgfusepath{clip}%
\pgfsetbuttcap%
\pgfsetroundjoin%
\definecolor{currentfill}{rgb}{0.121569,0.466667,0.705882}%
\pgfsetfillcolor{currentfill}%
\pgfsetfillopacity{0.990931}%
\pgfsetlinewidth{1.003750pt}%
\definecolor{currentstroke}{rgb}{0.121569,0.466667,0.705882}%
\pgfsetstrokecolor{currentstroke}%
\pgfsetstrokeopacity{0.990931}%
\pgfsetdash{}{0pt}%
\pgfpathmoveto{\pgfqpoint{2.454315in}{1.898263in}}%
\pgfpathcurveto{\pgfqpoint{2.462551in}{1.898263in}}{\pgfqpoint{2.470451in}{1.901535in}}{\pgfqpoint{2.476275in}{1.907359in}}%
\pgfpathcurveto{\pgfqpoint{2.482099in}{1.913183in}}{\pgfqpoint{2.485371in}{1.921083in}}{\pgfqpoint{2.485371in}{1.929319in}}%
\pgfpathcurveto{\pgfqpoint{2.485371in}{1.937555in}}{\pgfqpoint{2.482099in}{1.945455in}}{\pgfqpoint{2.476275in}{1.951279in}}%
\pgfpathcurveto{\pgfqpoint{2.470451in}{1.957103in}}{\pgfqpoint{2.462551in}{1.960376in}}{\pgfqpoint{2.454315in}{1.960376in}}%
\pgfpathcurveto{\pgfqpoint{2.446078in}{1.960376in}}{\pgfqpoint{2.438178in}{1.957103in}}{\pgfqpoint{2.432354in}{1.951279in}}%
\pgfpathcurveto{\pgfqpoint{2.426530in}{1.945455in}}{\pgfqpoint{2.423258in}{1.937555in}}{\pgfqpoint{2.423258in}{1.929319in}}%
\pgfpathcurveto{\pgfqpoint{2.423258in}{1.921083in}}{\pgfqpoint{2.426530in}{1.913183in}}{\pgfqpoint{2.432354in}{1.907359in}}%
\pgfpathcurveto{\pgfqpoint{2.438178in}{1.901535in}}{\pgfqpoint{2.446078in}{1.898263in}}{\pgfqpoint{2.454315in}{1.898263in}}%
\pgfpathclose%
\pgfusepath{stroke,fill}%
\end{pgfscope}%
\begin{pgfscope}%
\pgfpathrectangle{\pgfqpoint{0.100000in}{0.212622in}}{\pgfqpoint{3.696000in}{3.696000in}}%
\pgfusepath{clip}%
\pgfsetbuttcap%
\pgfsetroundjoin%
\definecolor{currentfill}{rgb}{0.121569,0.466667,0.705882}%
\pgfsetfillcolor{currentfill}%
\pgfsetfillopacity{0.992288}%
\pgfsetlinewidth{1.003750pt}%
\definecolor{currentstroke}{rgb}{0.121569,0.466667,0.705882}%
\pgfsetstrokecolor{currentstroke}%
\pgfsetstrokeopacity{0.992288}%
\pgfsetdash{}{0pt}%
\pgfpathmoveto{\pgfqpoint{2.450771in}{1.898950in}}%
\pgfpathcurveto{\pgfqpoint{2.459007in}{1.898950in}}{\pgfqpoint{2.466907in}{1.902223in}}{\pgfqpoint{2.472731in}{1.908047in}}%
\pgfpathcurveto{\pgfqpoint{2.478555in}{1.913871in}}{\pgfqpoint{2.481827in}{1.921771in}}{\pgfqpoint{2.481827in}{1.930007in}}%
\pgfpathcurveto{\pgfqpoint{2.481827in}{1.938243in}}{\pgfqpoint{2.478555in}{1.946143in}}{\pgfqpoint{2.472731in}{1.951967in}}%
\pgfpathcurveto{\pgfqpoint{2.466907in}{1.957791in}}{\pgfqpoint{2.459007in}{1.961063in}}{\pgfqpoint{2.450771in}{1.961063in}}%
\pgfpathcurveto{\pgfqpoint{2.442534in}{1.961063in}}{\pgfqpoint{2.434634in}{1.957791in}}{\pgfqpoint{2.428810in}{1.951967in}}%
\pgfpathcurveto{\pgfqpoint{2.422986in}{1.946143in}}{\pgfqpoint{2.419714in}{1.938243in}}{\pgfqpoint{2.419714in}{1.930007in}}%
\pgfpathcurveto{\pgfqpoint{2.419714in}{1.921771in}}{\pgfqpoint{2.422986in}{1.913871in}}{\pgfqpoint{2.428810in}{1.908047in}}%
\pgfpathcurveto{\pgfqpoint{2.434634in}{1.902223in}}{\pgfqpoint{2.442534in}{1.898950in}}{\pgfqpoint{2.450771in}{1.898950in}}%
\pgfpathclose%
\pgfusepath{stroke,fill}%
\end{pgfscope}%
\begin{pgfscope}%
\pgfpathrectangle{\pgfqpoint{0.100000in}{0.212622in}}{\pgfqpoint{3.696000in}{3.696000in}}%
\pgfusepath{clip}%
\pgfsetbuttcap%
\pgfsetroundjoin%
\definecolor{currentfill}{rgb}{0.121569,0.466667,0.705882}%
\pgfsetfillcolor{currentfill}%
\pgfsetfillopacity{0.992807}%
\pgfsetlinewidth{1.003750pt}%
\definecolor{currentstroke}{rgb}{0.121569,0.466667,0.705882}%
\pgfsetstrokecolor{currentstroke}%
\pgfsetstrokeopacity{0.992807}%
\pgfsetdash{}{0pt}%
\pgfpathmoveto{\pgfqpoint{2.365506in}{1.922380in}}%
\pgfpathcurveto{\pgfqpoint{2.373742in}{1.922380in}}{\pgfqpoint{2.381642in}{1.925652in}}{\pgfqpoint{2.387466in}{1.931476in}}%
\pgfpathcurveto{\pgfqpoint{2.393290in}{1.937300in}}{\pgfqpoint{2.396562in}{1.945200in}}{\pgfqpoint{2.396562in}{1.953436in}}%
\pgfpathcurveto{\pgfqpoint{2.396562in}{1.961672in}}{\pgfqpoint{2.393290in}{1.969573in}}{\pgfqpoint{2.387466in}{1.975396in}}%
\pgfpathcurveto{\pgfqpoint{2.381642in}{1.981220in}}{\pgfqpoint{2.373742in}{1.984493in}}{\pgfqpoint{2.365506in}{1.984493in}}%
\pgfpathcurveto{\pgfqpoint{2.357269in}{1.984493in}}{\pgfqpoint{2.349369in}{1.981220in}}{\pgfqpoint{2.343545in}{1.975396in}}%
\pgfpathcurveto{\pgfqpoint{2.337721in}{1.969573in}}{\pgfqpoint{2.334449in}{1.961672in}}{\pgfqpoint{2.334449in}{1.953436in}}%
\pgfpathcurveto{\pgfqpoint{2.334449in}{1.945200in}}{\pgfqpoint{2.337721in}{1.937300in}}{\pgfqpoint{2.343545in}{1.931476in}}%
\pgfpathcurveto{\pgfqpoint{2.349369in}{1.925652in}}{\pgfqpoint{2.357269in}{1.922380in}}{\pgfqpoint{2.365506in}{1.922380in}}%
\pgfpathclose%
\pgfusepath{stroke,fill}%
\end{pgfscope}%
\begin{pgfscope}%
\pgfpathrectangle{\pgfqpoint{0.100000in}{0.212622in}}{\pgfqpoint{3.696000in}{3.696000in}}%
\pgfusepath{clip}%
\pgfsetbuttcap%
\pgfsetroundjoin%
\definecolor{currentfill}{rgb}{0.121569,0.466667,0.705882}%
\pgfsetfillcolor{currentfill}%
\pgfsetfillopacity{0.993887}%
\pgfsetlinewidth{1.003750pt}%
\definecolor{currentstroke}{rgb}{0.121569,0.466667,0.705882}%
\pgfsetstrokecolor{currentstroke}%
\pgfsetstrokeopacity{0.993887}%
\pgfsetdash{}{0pt}%
\pgfpathmoveto{\pgfqpoint{2.377091in}{1.918781in}}%
\pgfpathcurveto{\pgfqpoint{2.385327in}{1.918781in}}{\pgfqpoint{2.393227in}{1.922053in}}{\pgfqpoint{2.399051in}{1.927877in}}%
\pgfpathcurveto{\pgfqpoint{2.404875in}{1.933701in}}{\pgfqpoint{2.408148in}{1.941601in}}{\pgfqpoint{2.408148in}{1.949837in}}%
\pgfpathcurveto{\pgfqpoint{2.408148in}{1.958074in}}{\pgfqpoint{2.404875in}{1.965974in}}{\pgfqpoint{2.399051in}{1.971798in}}%
\pgfpathcurveto{\pgfqpoint{2.393227in}{1.977621in}}{\pgfqpoint{2.385327in}{1.980894in}}{\pgfqpoint{2.377091in}{1.980894in}}%
\pgfpathcurveto{\pgfqpoint{2.368855in}{1.980894in}}{\pgfqpoint{2.360955in}{1.977621in}}{\pgfqpoint{2.355131in}{1.971798in}}%
\pgfpathcurveto{\pgfqpoint{2.349307in}{1.965974in}}{\pgfqpoint{2.346035in}{1.958074in}}{\pgfqpoint{2.346035in}{1.949837in}}%
\pgfpathcurveto{\pgfqpoint{2.346035in}{1.941601in}}{\pgfqpoint{2.349307in}{1.933701in}}{\pgfqpoint{2.355131in}{1.927877in}}%
\pgfpathcurveto{\pgfqpoint{2.360955in}{1.922053in}}{\pgfqpoint{2.368855in}{1.918781in}}{\pgfqpoint{2.377091in}{1.918781in}}%
\pgfpathclose%
\pgfusepath{stroke,fill}%
\end{pgfscope}%
\begin{pgfscope}%
\pgfpathrectangle{\pgfqpoint{0.100000in}{0.212622in}}{\pgfqpoint{3.696000in}{3.696000in}}%
\pgfusepath{clip}%
\pgfsetbuttcap%
\pgfsetroundjoin%
\definecolor{currentfill}{rgb}{0.121569,0.466667,0.705882}%
\pgfsetfillcolor{currentfill}%
\pgfsetfillopacity{0.993934}%
\pgfsetlinewidth{1.003750pt}%
\definecolor{currentstroke}{rgb}{0.121569,0.466667,0.705882}%
\pgfsetstrokecolor{currentstroke}%
\pgfsetstrokeopacity{0.993934}%
\pgfsetdash{}{0pt}%
\pgfpathmoveto{\pgfqpoint{2.448054in}{1.899159in}}%
\pgfpathcurveto{\pgfqpoint{2.456290in}{1.899159in}}{\pgfqpoint{2.464190in}{1.902432in}}{\pgfqpoint{2.470014in}{1.908256in}}%
\pgfpathcurveto{\pgfqpoint{2.475838in}{1.914079in}}{\pgfqpoint{2.479111in}{1.921980in}}{\pgfqpoint{2.479111in}{1.930216in}}%
\pgfpathcurveto{\pgfqpoint{2.479111in}{1.938452in}}{\pgfqpoint{2.475838in}{1.946352in}}{\pgfqpoint{2.470014in}{1.952176in}}%
\pgfpathcurveto{\pgfqpoint{2.464190in}{1.958000in}}{\pgfqpoint{2.456290in}{1.961272in}}{\pgfqpoint{2.448054in}{1.961272in}}%
\pgfpathcurveto{\pgfqpoint{2.439818in}{1.961272in}}{\pgfqpoint{2.431918in}{1.958000in}}{\pgfqpoint{2.426094in}{1.952176in}}%
\pgfpathcurveto{\pgfqpoint{2.420270in}{1.946352in}}{\pgfqpoint{2.416998in}{1.938452in}}{\pgfqpoint{2.416998in}{1.930216in}}%
\pgfpathcurveto{\pgfqpoint{2.416998in}{1.921980in}}{\pgfqpoint{2.420270in}{1.914079in}}{\pgfqpoint{2.426094in}{1.908256in}}%
\pgfpathcurveto{\pgfqpoint{2.431918in}{1.902432in}}{\pgfqpoint{2.439818in}{1.899159in}}{\pgfqpoint{2.448054in}{1.899159in}}%
\pgfpathclose%
\pgfusepath{stroke,fill}%
\end{pgfscope}%
\begin{pgfscope}%
\pgfpathrectangle{\pgfqpoint{0.100000in}{0.212622in}}{\pgfqpoint{3.696000in}{3.696000in}}%
\pgfusepath{clip}%
\pgfsetbuttcap%
\pgfsetroundjoin%
\definecolor{currentfill}{rgb}{0.121569,0.466667,0.705882}%
\pgfsetfillcolor{currentfill}%
\pgfsetfillopacity{0.994641}%
\pgfsetlinewidth{1.003750pt}%
\definecolor{currentstroke}{rgb}{0.121569,0.466667,0.705882}%
\pgfsetstrokecolor{currentstroke}%
\pgfsetstrokeopacity{0.994641}%
\pgfsetdash{}{0pt}%
\pgfpathmoveto{\pgfqpoint{2.386668in}{1.914288in}}%
\pgfpathcurveto{\pgfqpoint{2.394905in}{1.914288in}}{\pgfqpoint{2.402805in}{1.917561in}}{\pgfqpoint{2.408629in}{1.923385in}}%
\pgfpathcurveto{\pgfqpoint{2.414453in}{1.929209in}}{\pgfqpoint{2.417725in}{1.937109in}}{\pgfqpoint{2.417725in}{1.945345in}}%
\pgfpathcurveto{\pgfqpoint{2.417725in}{1.953581in}}{\pgfqpoint{2.414453in}{1.961481in}}{\pgfqpoint{2.408629in}{1.967305in}}%
\pgfpathcurveto{\pgfqpoint{2.402805in}{1.973129in}}{\pgfqpoint{2.394905in}{1.976401in}}{\pgfqpoint{2.386668in}{1.976401in}}%
\pgfpathcurveto{\pgfqpoint{2.378432in}{1.976401in}}{\pgfqpoint{2.370532in}{1.973129in}}{\pgfqpoint{2.364708in}{1.967305in}}%
\pgfpathcurveto{\pgfqpoint{2.358884in}{1.961481in}}{\pgfqpoint{2.355612in}{1.953581in}}{\pgfqpoint{2.355612in}{1.945345in}}%
\pgfpathcurveto{\pgfqpoint{2.355612in}{1.937109in}}{\pgfqpoint{2.358884in}{1.929209in}}{\pgfqpoint{2.364708in}{1.923385in}}%
\pgfpathcurveto{\pgfqpoint{2.370532in}{1.917561in}}{\pgfqpoint{2.378432in}{1.914288in}}{\pgfqpoint{2.386668in}{1.914288in}}%
\pgfpathclose%
\pgfusepath{stroke,fill}%
\end{pgfscope}%
\begin{pgfscope}%
\pgfpathrectangle{\pgfqpoint{0.100000in}{0.212622in}}{\pgfqpoint{3.696000in}{3.696000in}}%
\pgfusepath{clip}%
\pgfsetbuttcap%
\pgfsetroundjoin%
\definecolor{currentfill}{rgb}{0.121569,0.466667,0.705882}%
\pgfsetfillcolor{currentfill}%
\pgfsetfillopacity{0.995703}%
\pgfsetlinewidth{1.003750pt}%
\definecolor{currentstroke}{rgb}{0.121569,0.466667,0.705882}%
\pgfsetstrokecolor{currentstroke}%
\pgfsetstrokeopacity{0.995703}%
\pgfsetdash{}{0pt}%
\pgfpathmoveto{\pgfqpoint{2.444803in}{1.899417in}}%
\pgfpathcurveto{\pgfqpoint{2.453039in}{1.899417in}}{\pgfqpoint{2.460939in}{1.902689in}}{\pgfqpoint{2.466763in}{1.908513in}}%
\pgfpathcurveto{\pgfqpoint{2.472587in}{1.914337in}}{\pgfqpoint{2.475860in}{1.922237in}}{\pgfqpoint{2.475860in}{1.930473in}}%
\pgfpathcurveto{\pgfqpoint{2.475860in}{1.938710in}}{\pgfqpoint{2.472587in}{1.946610in}}{\pgfqpoint{2.466763in}{1.952434in}}%
\pgfpathcurveto{\pgfqpoint{2.460939in}{1.958258in}}{\pgfqpoint{2.453039in}{1.961530in}}{\pgfqpoint{2.444803in}{1.961530in}}%
\pgfpathcurveto{\pgfqpoint{2.436567in}{1.961530in}}{\pgfqpoint{2.428667in}{1.958258in}}{\pgfqpoint{2.422843in}{1.952434in}}%
\pgfpathcurveto{\pgfqpoint{2.417019in}{1.946610in}}{\pgfqpoint{2.413747in}{1.938710in}}{\pgfqpoint{2.413747in}{1.930473in}}%
\pgfpathcurveto{\pgfqpoint{2.413747in}{1.922237in}}{\pgfqpoint{2.417019in}{1.914337in}}{\pgfqpoint{2.422843in}{1.908513in}}%
\pgfpathcurveto{\pgfqpoint{2.428667in}{1.902689in}}{\pgfqpoint{2.436567in}{1.899417in}}{\pgfqpoint{2.444803in}{1.899417in}}%
\pgfpathclose%
\pgfusepath{stroke,fill}%
\end{pgfscope}%
\begin{pgfscope}%
\pgfpathrectangle{\pgfqpoint{0.100000in}{0.212622in}}{\pgfqpoint{3.696000in}{3.696000in}}%
\pgfusepath{clip}%
\pgfsetbuttcap%
\pgfsetroundjoin%
\definecolor{currentfill}{rgb}{0.121569,0.466667,0.705882}%
\pgfsetfillcolor{currentfill}%
\pgfsetfillopacity{0.995704}%
\pgfsetlinewidth{1.003750pt}%
\definecolor{currentstroke}{rgb}{0.121569,0.466667,0.705882}%
\pgfsetstrokecolor{currentstroke}%
\pgfsetstrokeopacity{0.995704}%
\pgfsetdash{}{0pt}%
\pgfpathmoveto{\pgfqpoint{2.392994in}{1.913104in}}%
\pgfpathcurveto{\pgfqpoint{2.401230in}{1.913104in}}{\pgfqpoint{2.409130in}{1.916377in}}{\pgfqpoint{2.414954in}{1.922201in}}%
\pgfpathcurveto{\pgfqpoint{2.420778in}{1.928024in}}{\pgfqpoint{2.424051in}{1.935925in}}{\pgfqpoint{2.424051in}{1.944161in}}%
\pgfpathcurveto{\pgfqpoint{2.424051in}{1.952397in}}{\pgfqpoint{2.420778in}{1.960297in}}{\pgfqpoint{2.414954in}{1.966121in}}%
\pgfpathcurveto{\pgfqpoint{2.409130in}{1.971945in}}{\pgfqpoint{2.401230in}{1.975217in}}{\pgfqpoint{2.392994in}{1.975217in}}%
\pgfpathcurveto{\pgfqpoint{2.384758in}{1.975217in}}{\pgfqpoint{2.376858in}{1.971945in}}{\pgfqpoint{2.371034in}{1.966121in}}%
\pgfpathcurveto{\pgfqpoint{2.365210in}{1.960297in}}{\pgfqpoint{2.361938in}{1.952397in}}{\pgfqpoint{2.361938in}{1.944161in}}%
\pgfpathcurveto{\pgfqpoint{2.361938in}{1.935925in}}{\pgfqpoint{2.365210in}{1.928024in}}{\pgfqpoint{2.371034in}{1.922201in}}%
\pgfpathcurveto{\pgfqpoint{2.376858in}{1.916377in}}{\pgfqpoint{2.384758in}{1.913104in}}{\pgfqpoint{2.392994in}{1.913104in}}%
\pgfpathclose%
\pgfusepath{stroke,fill}%
\end{pgfscope}%
\begin{pgfscope}%
\pgfpathrectangle{\pgfqpoint{0.100000in}{0.212622in}}{\pgfqpoint{3.696000in}{3.696000in}}%
\pgfusepath{clip}%
\pgfsetbuttcap%
\pgfsetroundjoin%
\definecolor{currentfill}{rgb}{0.121569,0.466667,0.705882}%
\pgfsetfillcolor{currentfill}%
\pgfsetfillopacity{0.996351}%
\pgfsetlinewidth{1.003750pt}%
\definecolor{currentstroke}{rgb}{0.121569,0.466667,0.705882}%
\pgfsetstrokecolor{currentstroke}%
\pgfsetstrokeopacity{0.996351}%
\pgfsetdash{}{0pt}%
\pgfpathmoveto{\pgfqpoint{2.399186in}{1.912098in}}%
\pgfpathcurveto{\pgfqpoint{2.407422in}{1.912098in}}{\pgfqpoint{2.415322in}{1.915371in}}{\pgfqpoint{2.421146in}{1.921195in}}%
\pgfpathcurveto{\pgfqpoint{2.426970in}{1.927019in}}{\pgfqpoint{2.430242in}{1.934919in}}{\pgfqpoint{2.430242in}{1.943155in}}%
\pgfpathcurveto{\pgfqpoint{2.430242in}{1.951391in}}{\pgfqpoint{2.426970in}{1.959291in}}{\pgfqpoint{2.421146in}{1.965115in}}%
\pgfpathcurveto{\pgfqpoint{2.415322in}{1.970939in}}{\pgfqpoint{2.407422in}{1.974211in}}{\pgfqpoint{2.399186in}{1.974211in}}%
\pgfpathcurveto{\pgfqpoint{2.390949in}{1.974211in}}{\pgfqpoint{2.383049in}{1.970939in}}{\pgfqpoint{2.377225in}{1.965115in}}%
\pgfpathcurveto{\pgfqpoint{2.371401in}{1.959291in}}{\pgfqpoint{2.368129in}{1.951391in}}{\pgfqpoint{2.368129in}{1.943155in}}%
\pgfpathcurveto{\pgfqpoint{2.368129in}{1.934919in}}{\pgfqpoint{2.371401in}{1.927019in}}{\pgfqpoint{2.377225in}{1.921195in}}%
\pgfpathcurveto{\pgfqpoint{2.383049in}{1.915371in}}{\pgfqpoint{2.390949in}{1.912098in}}{\pgfqpoint{2.399186in}{1.912098in}}%
\pgfpathclose%
\pgfusepath{stroke,fill}%
\end{pgfscope}%
\begin{pgfscope}%
\pgfpathrectangle{\pgfqpoint{0.100000in}{0.212622in}}{\pgfqpoint{3.696000in}{3.696000in}}%
\pgfusepath{clip}%
\pgfsetbuttcap%
\pgfsetroundjoin%
\definecolor{currentfill}{rgb}{0.121569,0.466667,0.705882}%
\pgfsetfillcolor{currentfill}%
\pgfsetfillopacity{0.996658}%
\pgfsetlinewidth{1.003750pt}%
\definecolor{currentstroke}{rgb}{0.121569,0.466667,0.705882}%
\pgfsetstrokecolor{currentstroke}%
\pgfsetstrokeopacity{0.996658}%
\pgfsetdash{}{0pt}%
\pgfpathmoveto{\pgfqpoint{2.442872in}{1.899600in}}%
\pgfpathcurveto{\pgfqpoint{2.451108in}{1.899600in}}{\pgfqpoint{2.459008in}{1.902872in}}{\pgfqpoint{2.464832in}{1.908696in}}%
\pgfpathcurveto{\pgfqpoint{2.470656in}{1.914520in}}{\pgfqpoint{2.473928in}{1.922420in}}{\pgfqpoint{2.473928in}{1.930656in}}%
\pgfpathcurveto{\pgfqpoint{2.473928in}{1.938892in}}{\pgfqpoint{2.470656in}{1.946792in}}{\pgfqpoint{2.464832in}{1.952616in}}%
\pgfpathcurveto{\pgfqpoint{2.459008in}{1.958440in}}{\pgfqpoint{2.451108in}{1.961713in}}{\pgfqpoint{2.442872in}{1.961713in}}%
\pgfpathcurveto{\pgfqpoint{2.434635in}{1.961713in}}{\pgfqpoint{2.426735in}{1.958440in}}{\pgfqpoint{2.420911in}{1.952616in}}%
\pgfpathcurveto{\pgfqpoint{2.415087in}{1.946792in}}{\pgfqpoint{2.411815in}{1.938892in}}{\pgfqpoint{2.411815in}{1.930656in}}%
\pgfpathcurveto{\pgfqpoint{2.411815in}{1.922420in}}{\pgfqpoint{2.415087in}{1.914520in}}{\pgfqpoint{2.420911in}{1.908696in}}%
\pgfpathcurveto{\pgfqpoint{2.426735in}{1.902872in}}{\pgfqpoint{2.434635in}{1.899600in}}{\pgfqpoint{2.442872in}{1.899600in}}%
\pgfpathclose%
\pgfusepath{stroke,fill}%
\end{pgfscope}%
\begin{pgfscope}%
\pgfpathrectangle{\pgfqpoint{0.100000in}{0.212622in}}{\pgfqpoint{3.696000in}{3.696000in}}%
\pgfusepath{clip}%
\pgfsetbuttcap%
\pgfsetroundjoin%
\definecolor{currentfill}{rgb}{0.121569,0.466667,0.705882}%
\pgfsetfillcolor{currentfill}%
\pgfsetfillopacity{0.997012}%
\pgfsetlinewidth{1.003750pt}%
\definecolor{currentstroke}{rgb}{0.121569,0.466667,0.705882}%
\pgfsetstrokecolor{currentstroke}%
\pgfsetstrokeopacity{0.997012}%
\pgfsetdash{}{0pt}%
\pgfpathmoveto{\pgfqpoint{2.404719in}{1.910330in}}%
\pgfpathcurveto{\pgfqpoint{2.412956in}{1.910330in}}{\pgfqpoint{2.420856in}{1.913603in}}{\pgfqpoint{2.426680in}{1.919427in}}%
\pgfpathcurveto{\pgfqpoint{2.432503in}{1.925251in}}{\pgfqpoint{2.435776in}{1.933151in}}{\pgfqpoint{2.435776in}{1.941387in}}%
\pgfpathcurveto{\pgfqpoint{2.435776in}{1.949623in}}{\pgfqpoint{2.432503in}{1.957523in}}{\pgfqpoint{2.426680in}{1.963347in}}%
\pgfpathcurveto{\pgfqpoint{2.420856in}{1.969171in}}{\pgfqpoint{2.412956in}{1.972443in}}{\pgfqpoint{2.404719in}{1.972443in}}%
\pgfpathcurveto{\pgfqpoint{2.396483in}{1.972443in}}{\pgfqpoint{2.388583in}{1.969171in}}{\pgfqpoint{2.382759in}{1.963347in}}%
\pgfpathcurveto{\pgfqpoint{2.376935in}{1.957523in}}{\pgfqpoint{2.373663in}{1.949623in}}{\pgfqpoint{2.373663in}{1.941387in}}%
\pgfpathcurveto{\pgfqpoint{2.373663in}{1.933151in}}{\pgfqpoint{2.376935in}{1.925251in}}{\pgfqpoint{2.382759in}{1.919427in}}%
\pgfpathcurveto{\pgfqpoint{2.388583in}{1.913603in}}{\pgfqpoint{2.396483in}{1.910330in}}{\pgfqpoint{2.404719in}{1.910330in}}%
\pgfpathclose%
\pgfusepath{stroke,fill}%
\end{pgfscope}%
\begin{pgfscope}%
\pgfpathrectangle{\pgfqpoint{0.100000in}{0.212622in}}{\pgfqpoint{3.696000in}{3.696000in}}%
\pgfusepath{clip}%
\pgfsetbuttcap%
\pgfsetroundjoin%
\definecolor{currentfill}{rgb}{0.121569,0.466667,0.705882}%
\pgfsetfillcolor{currentfill}%
\pgfsetfillopacity{0.997731}%
\pgfsetlinewidth{1.003750pt}%
\definecolor{currentstroke}{rgb}{0.121569,0.466667,0.705882}%
\pgfsetstrokecolor{currentstroke}%
\pgfsetstrokeopacity{0.997731}%
\pgfsetdash{}{0pt}%
\pgfpathmoveto{\pgfqpoint{2.409450in}{1.908389in}}%
\pgfpathcurveto{\pgfqpoint{2.417686in}{1.908389in}}{\pgfqpoint{2.425586in}{1.911661in}}{\pgfqpoint{2.431410in}{1.917485in}}%
\pgfpathcurveto{\pgfqpoint{2.437234in}{1.923309in}}{\pgfqpoint{2.440506in}{1.931209in}}{\pgfqpoint{2.440506in}{1.939445in}}%
\pgfpathcurveto{\pgfqpoint{2.440506in}{1.947682in}}{\pgfqpoint{2.437234in}{1.955582in}}{\pgfqpoint{2.431410in}{1.961406in}}%
\pgfpathcurveto{\pgfqpoint{2.425586in}{1.967229in}}{\pgfqpoint{2.417686in}{1.970502in}}{\pgfqpoint{2.409450in}{1.970502in}}%
\pgfpathcurveto{\pgfqpoint{2.401213in}{1.970502in}}{\pgfqpoint{2.393313in}{1.967229in}}{\pgfqpoint{2.387489in}{1.961406in}}%
\pgfpathcurveto{\pgfqpoint{2.381665in}{1.955582in}}{\pgfqpoint{2.378393in}{1.947682in}}{\pgfqpoint{2.378393in}{1.939445in}}%
\pgfpathcurveto{\pgfqpoint{2.378393in}{1.931209in}}{\pgfqpoint{2.381665in}{1.923309in}}{\pgfqpoint{2.387489in}{1.917485in}}%
\pgfpathcurveto{\pgfqpoint{2.393313in}{1.911661in}}{\pgfqpoint{2.401213in}{1.908389in}}{\pgfqpoint{2.409450in}{1.908389in}}%
\pgfpathclose%
\pgfusepath{stroke,fill}%
\end{pgfscope}%
\begin{pgfscope}%
\pgfpathrectangle{\pgfqpoint{0.100000in}{0.212622in}}{\pgfqpoint{3.696000in}{3.696000in}}%
\pgfusepath{clip}%
\pgfsetbuttcap%
\pgfsetroundjoin%
\definecolor{currentfill}{rgb}{0.121569,0.466667,0.705882}%
\pgfsetfillcolor{currentfill}%
\pgfsetfillopacity{0.997749}%
\pgfsetlinewidth{1.003750pt}%
\definecolor{currentstroke}{rgb}{0.121569,0.466667,0.705882}%
\pgfsetstrokecolor{currentstroke}%
\pgfsetstrokeopacity{0.997749}%
\pgfsetdash{}{0pt}%
\pgfpathmoveto{\pgfqpoint{2.440812in}{1.899769in}}%
\pgfpathcurveto{\pgfqpoint{2.449048in}{1.899769in}}{\pgfqpoint{2.456948in}{1.903041in}}{\pgfqpoint{2.462772in}{1.908865in}}%
\pgfpathcurveto{\pgfqpoint{2.468596in}{1.914689in}}{\pgfqpoint{2.471868in}{1.922589in}}{\pgfqpoint{2.471868in}{1.930825in}}%
\pgfpathcurveto{\pgfqpoint{2.471868in}{1.939061in}}{\pgfqpoint{2.468596in}{1.946961in}}{\pgfqpoint{2.462772in}{1.952785in}}%
\pgfpathcurveto{\pgfqpoint{2.456948in}{1.958609in}}{\pgfqpoint{2.449048in}{1.961882in}}{\pgfqpoint{2.440812in}{1.961882in}}%
\pgfpathcurveto{\pgfqpoint{2.432576in}{1.961882in}}{\pgfqpoint{2.424676in}{1.958609in}}{\pgfqpoint{2.418852in}{1.952785in}}%
\pgfpathcurveto{\pgfqpoint{2.413028in}{1.946961in}}{\pgfqpoint{2.409755in}{1.939061in}}{\pgfqpoint{2.409755in}{1.930825in}}%
\pgfpathcurveto{\pgfqpoint{2.409755in}{1.922589in}}{\pgfqpoint{2.413028in}{1.914689in}}{\pgfqpoint{2.418852in}{1.908865in}}%
\pgfpathcurveto{\pgfqpoint{2.424676in}{1.903041in}}{\pgfqpoint{2.432576in}{1.899769in}}{\pgfqpoint{2.440812in}{1.899769in}}%
\pgfpathclose%
\pgfusepath{stroke,fill}%
\end{pgfscope}%
\begin{pgfscope}%
\pgfpathrectangle{\pgfqpoint{0.100000in}{0.212622in}}{\pgfqpoint{3.696000in}{3.696000in}}%
\pgfusepath{clip}%
\pgfsetbuttcap%
\pgfsetroundjoin%
\definecolor{currentfill}{rgb}{0.121569,0.466667,0.705882}%
\pgfsetfillcolor{currentfill}%
\pgfsetfillopacity{0.998044}%
\pgfsetlinewidth{1.003750pt}%
\definecolor{currentstroke}{rgb}{0.121569,0.466667,0.705882}%
\pgfsetstrokecolor{currentstroke}%
\pgfsetstrokeopacity{0.998044}%
\pgfsetdash{}{0pt}%
\pgfpathmoveto{\pgfqpoint{2.412207in}{1.907034in}}%
\pgfpathcurveto{\pgfqpoint{2.420443in}{1.907034in}}{\pgfqpoint{2.428343in}{1.910307in}}{\pgfqpoint{2.434167in}{1.916130in}}%
\pgfpathcurveto{\pgfqpoint{2.439991in}{1.921954in}}{\pgfqpoint{2.443263in}{1.929854in}}{\pgfqpoint{2.443263in}{1.938091in}}%
\pgfpathcurveto{\pgfqpoint{2.443263in}{1.946327in}}{\pgfqpoint{2.439991in}{1.954227in}}{\pgfqpoint{2.434167in}{1.960051in}}%
\pgfpathcurveto{\pgfqpoint{2.428343in}{1.965875in}}{\pgfqpoint{2.420443in}{1.969147in}}{\pgfqpoint{2.412207in}{1.969147in}}%
\pgfpathcurveto{\pgfqpoint{2.403970in}{1.969147in}}{\pgfqpoint{2.396070in}{1.965875in}}{\pgfqpoint{2.390246in}{1.960051in}}%
\pgfpathcurveto{\pgfqpoint{2.384422in}{1.954227in}}{\pgfqpoint{2.381150in}{1.946327in}}{\pgfqpoint{2.381150in}{1.938091in}}%
\pgfpathcurveto{\pgfqpoint{2.381150in}{1.929854in}}{\pgfqpoint{2.384422in}{1.921954in}}{\pgfqpoint{2.390246in}{1.916130in}}%
\pgfpathcurveto{\pgfqpoint{2.396070in}{1.910307in}}{\pgfqpoint{2.403970in}{1.907034in}}{\pgfqpoint{2.412207in}{1.907034in}}%
\pgfpathclose%
\pgfusepath{stroke,fill}%
\end{pgfscope}%
\begin{pgfscope}%
\pgfpathrectangle{\pgfqpoint{0.100000in}{0.212622in}}{\pgfqpoint{3.696000in}{3.696000in}}%
\pgfusepath{clip}%
\pgfsetbuttcap%
\pgfsetroundjoin%
\definecolor{currentfill}{rgb}{0.121569,0.466667,0.705882}%
\pgfsetfillcolor{currentfill}%
\pgfsetfillopacity{0.998299}%
\pgfsetlinewidth{1.003750pt}%
\definecolor{currentstroke}{rgb}{0.121569,0.466667,0.705882}%
\pgfsetstrokecolor{currentstroke}%
\pgfsetstrokeopacity{0.998299}%
\pgfsetdash{}{0pt}%
\pgfpathmoveto{\pgfqpoint{2.413427in}{1.906907in}}%
\pgfpathcurveto{\pgfqpoint{2.421663in}{1.906907in}}{\pgfqpoint{2.429563in}{1.910179in}}{\pgfqpoint{2.435387in}{1.916003in}}%
\pgfpathcurveto{\pgfqpoint{2.441211in}{1.921827in}}{\pgfqpoint{2.444484in}{1.929727in}}{\pgfqpoint{2.444484in}{1.937963in}}%
\pgfpathcurveto{\pgfqpoint{2.444484in}{1.946200in}}{\pgfqpoint{2.441211in}{1.954100in}}{\pgfqpoint{2.435387in}{1.959924in}}%
\pgfpathcurveto{\pgfqpoint{2.429563in}{1.965747in}}{\pgfqpoint{2.421663in}{1.969020in}}{\pgfqpoint{2.413427in}{1.969020in}}%
\pgfpathcurveto{\pgfqpoint{2.405191in}{1.969020in}}{\pgfqpoint{2.397291in}{1.965747in}}{\pgfqpoint{2.391467in}{1.959924in}}%
\pgfpathcurveto{\pgfqpoint{2.385643in}{1.954100in}}{\pgfqpoint{2.382371in}{1.946200in}}{\pgfqpoint{2.382371in}{1.937963in}}%
\pgfpathcurveto{\pgfqpoint{2.382371in}{1.929727in}}{\pgfqpoint{2.385643in}{1.921827in}}{\pgfqpoint{2.391467in}{1.916003in}}%
\pgfpathcurveto{\pgfqpoint{2.397291in}{1.910179in}}{\pgfqpoint{2.405191in}{1.906907in}}{\pgfqpoint{2.413427in}{1.906907in}}%
\pgfpathclose%
\pgfusepath{stroke,fill}%
\end{pgfscope}%
\begin{pgfscope}%
\pgfpathrectangle{\pgfqpoint{0.100000in}{0.212622in}}{\pgfqpoint{3.696000in}{3.696000in}}%
\pgfusepath{clip}%
\pgfsetbuttcap%
\pgfsetroundjoin%
\definecolor{currentfill}{rgb}{0.121569,0.466667,0.705882}%
\pgfsetfillcolor{currentfill}%
\pgfsetfillopacity{0.998327}%
\pgfsetlinewidth{1.003750pt}%
\definecolor{currentstroke}{rgb}{0.121569,0.466667,0.705882}%
\pgfsetstrokecolor{currentstroke}%
\pgfsetstrokeopacity{0.998327}%
\pgfsetdash{}{0pt}%
\pgfpathmoveto{\pgfqpoint{2.413691in}{1.906892in}}%
\pgfpathcurveto{\pgfqpoint{2.421927in}{1.906892in}}{\pgfqpoint{2.429827in}{1.910165in}}{\pgfqpoint{2.435651in}{1.915989in}}%
\pgfpathcurveto{\pgfqpoint{2.441475in}{1.921813in}}{\pgfqpoint{2.444747in}{1.929713in}}{\pgfqpoint{2.444747in}{1.937949in}}%
\pgfpathcurveto{\pgfqpoint{2.444747in}{1.946185in}}{\pgfqpoint{2.441475in}{1.954085in}}{\pgfqpoint{2.435651in}{1.959909in}}%
\pgfpathcurveto{\pgfqpoint{2.429827in}{1.965733in}}{\pgfqpoint{2.421927in}{1.969005in}}{\pgfqpoint{2.413691in}{1.969005in}}%
\pgfpathcurveto{\pgfqpoint{2.405454in}{1.969005in}}{\pgfqpoint{2.397554in}{1.965733in}}{\pgfqpoint{2.391730in}{1.959909in}}%
\pgfpathcurveto{\pgfqpoint{2.385906in}{1.954085in}}{\pgfqpoint{2.382634in}{1.946185in}}{\pgfqpoint{2.382634in}{1.937949in}}%
\pgfpathcurveto{\pgfqpoint{2.382634in}{1.929713in}}{\pgfqpoint{2.385906in}{1.921813in}}{\pgfqpoint{2.391730in}{1.915989in}}%
\pgfpathcurveto{\pgfqpoint{2.397554in}{1.910165in}}{\pgfqpoint{2.405454in}{1.906892in}}{\pgfqpoint{2.413691in}{1.906892in}}%
\pgfpathclose%
\pgfusepath{stroke,fill}%
\end{pgfscope}%
\begin{pgfscope}%
\pgfpathrectangle{\pgfqpoint{0.100000in}{0.212622in}}{\pgfqpoint{3.696000in}{3.696000in}}%
\pgfusepath{clip}%
\pgfsetbuttcap%
\pgfsetroundjoin%
\definecolor{currentfill}{rgb}{0.121569,0.466667,0.705882}%
\pgfsetfillcolor{currentfill}%
\pgfsetfillopacity{0.998343}%
\pgfsetlinewidth{1.003750pt}%
\definecolor{currentstroke}{rgb}{0.121569,0.466667,0.705882}%
\pgfsetstrokecolor{currentstroke}%
\pgfsetstrokeopacity{0.998343}%
\pgfsetdash{}{0pt}%
\pgfpathmoveto{\pgfqpoint{2.439650in}{1.899857in}}%
\pgfpathcurveto{\pgfqpoint{2.447886in}{1.899857in}}{\pgfqpoint{2.455786in}{1.903129in}}{\pgfqpoint{2.461610in}{1.908953in}}%
\pgfpathcurveto{\pgfqpoint{2.467434in}{1.914777in}}{\pgfqpoint{2.470706in}{1.922677in}}{\pgfqpoint{2.470706in}{1.930914in}}%
\pgfpathcurveto{\pgfqpoint{2.470706in}{1.939150in}}{\pgfqpoint{2.467434in}{1.947050in}}{\pgfqpoint{2.461610in}{1.952874in}}%
\pgfpathcurveto{\pgfqpoint{2.455786in}{1.958698in}}{\pgfqpoint{2.447886in}{1.961970in}}{\pgfqpoint{2.439650in}{1.961970in}}%
\pgfpathcurveto{\pgfqpoint{2.431413in}{1.961970in}}{\pgfqpoint{2.423513in}{1.958698in}}{\pgfqpoint{2.417689in}{1.952874in}}%
\pgfpathcurveto{\pgfqpoint{2.411865in}{1.947050in}}{\pgfqpoint{2.408593in}{1.939150in}}{\pgfqpoint{2.408593in}{1.930914in}}%
\pgfpathcurveto{\pgfqpoint{2.408593in}{1.922677in}}{\pgfqpoint{2.411865in}{1.914777in}}{\pgfqpoint{2.417689in}{1.908953in}}%
\pgfpathcurveto{\pgfqpoint{2.423513in}{1.903129in}}{\pgfqpoint{2.431413in}{1.899857in}}{\pgfqpoint{2.439650in}{1.899857in}}%
\pgfpathclose%
\pgfusepath{stroke,fill}%
\end{pgfscope}%
\begin{pgfscope}%
\pgfpathrectangle{\pgfqpoint{0.100000in}{0.212622in}}{\pgfqpoint{3.696000in}{3.696000in}}%
\pgfusepath{clip}%
\pgfsetbuttcap%
\pgfsetroundjoin%
\definecolor{currentfill}{rgb}{0.121569,0.466667,0.705882}%
\pgfsetfillcolor{currentfill}%
\pgfsetfillopacity{0.998397}%
\pgfsetlinewidth{1.003750pt}%
\definecolor{currentstroke}{rgb}{0.121569,0.466667,0.705882}%
\pgfsetstrokecolor{currentstroke}%
\pgfsetstrokeopacity{0.998397}%
\pgfsetdash{}{0pt}%
\pgfpathmoveto{\pgfqpoint{2.414117in}{1.906799in}}%
\pgfpathcurveto{\pgfqpoint{2.422353in}{1.906799in}}{\pgfqpoint{2.430253in}{1.910072in}}{\pgfqpoint{2.436077in}{1.915895in}}%
\pgfpathcurveto{\pgfqpoint{2.441901in}{1.921719in}}{\pgfqpoint{2.445173in}{1.929619in}}{\pgfqpoint{2.445173in}{1.937856in}}%
\pgfpathcurveto{\pgfqpoint{2.445173in}{1.946092in}}{\pgfqpoint{2.441901in}{1.953992in}}{\pgfqpoint{2.436077in}{1.959816in}}%
\pgfpathcurveto{\pgfqpoint{2.430253in}{1.965640in}}{\pgfqpoint{2.422353in}{1.968912in}}{\pgfqpoint{2.414117in}{1.968912in}}%
\pgfpathcurveto{\pgfqpoint{2.405881in}{1.968912in}}{\pgfqpoint{2.397981in}{1.965640in}}{\pgfqpoint{2.392157in}{1.959816in}}%
\pgfpathcurveto{\pgfqpoint{2.386333in}{1.953992in}}{\pgfqpoint{2.383060in}{1.946092in}}{\pgfqpoint{2.383060in}{1.937856in}}%
\pgfpathcurveto{\pgfqpoint{2.383060in}{1.929619in}}{\pgfqpoint{2.386333in}{1.921719in}}{\pgfqpoint{2.392157in}{1.915895in}}%
\pgfpathcurveto{\pgfqpoint{2.397981in}{1.910072in}}{\pgfqpoint{2.405881in}{1.906799in}}{\pgfqpoint{2.414117in}{1.906799in}}%
\pgfpathclose%
\pgfusepath{stroke,fill}%
\end{pgfscope}%
\begin{pgfscope}%
\pgfpathrectangle{\pgfqpoint{0.100000in}{0.212622in}}{\pgfqpoint{3.696000in}{3.696000in}}%
\pgfusepath{clip}%
\pgfsetbuttcap%
\pgfsetroundjoin%
\definecolor{currentfill}{rgb}{0.121569,0.466667,0.705882}%
\pgfsetfillcolor{currentfill}%
\pgfsetfillopacity{0.998419}%
\pgfsetlinewidth{1.003750pt}%
\definecolor{currentstroke}{rgb}{0.121569,0.466667,0.705882}%
\pgfsetstrokecolor{currentstroke}%
\pgfsetstrokeopacity{0.998419}%
\pgfsetdash{}{0pt}%
\pgfpathmoveto{\pgfqpoint{2.414214in}{1.906781in}}%
\pgfpathcurveto{\pgfqpoint{2.422450in}{1.906781in}}{\pgfqpoint{2.430350in}{1.910053in}}{\pgfqpoint{2.436174in}{1.915877in}}%
\pgfpathcurveto{\pgfqpoint{2.441998in}{1.921701in}}{\pgfqpoint{2.445270in}{1.929601in}}{\pgfqpoint{2.445270in}{1.937837in}}%
\pgfpathcurveto{\pgfqpoint{2.445270in}{1.946074in}}{\pgfqpoint{2.441998in}{1.953974in}}{\pgfqpoint{2.436174in}{1.959797in}}%
\pgfpathcurveto{\pgfqpoint{2.430350in}{1.965621in}}{\pgfqpoint{2.422450in}{1.968894in}}{\pgfqpoint{2.414214in}{1.968894in}}%
\pgfpathcurveto{\pgfqpoint{2.405978in}{1.968894in}}{\pgfqpoint{2.398078in}{1.965621in}}{\pgfqpoint{2.392254in}{1.959797in}}%
\pgfpathcurveto{\pgfqpoint{2.386430in}{1.953974in}}{\pgfqpoint{2.383157in}{1.946074in}}{\pgfqpoint{2.383157in}{1.937837in}}%
\pgfpathcurveto{\pgfqpoint{2.383157in}{1.929601in}}{\pgfqpoint{2.386430in}{1.921701in}}{\pgfqpoint{2.392254in}{1.915877in}}%
\pgfpathcurveto{\pgfqpoint{2.398078in}{1.910053in}}{\pgfqpoint{2.405978in}{1.906781in}}{\pgfqpoint{2.414214in}{1.906781in}}%
\pgfpathclose%
\pgfusepath{stroke,fill}%
\end{pgfscope}%
\begin{pgfscope}%
\pgfpathrectangle{\pgfqpoint{0.100000in}{0.212622in}}{\pgfqpoint{3.696000in}{3.696000in}}%
\pgfusepath{clip}%
\pgfsetbuttcap%
\pgfsetroundjoin%
\definecolor{currentfill}{rgb}{0.121569,0.466667,0.705882}%
\pgfsetfillcolor{currentfill}%
\pgfsetfillopacity{0.998452}%
\pgfsetlinewidth{1.003750pt}%
\definecolor{currentstroke}{rgb}{0.121569,0.466667,0.705882}%
\pgfsetstrokecolor{currentstroke}%
\pgfsetstrokeopacity{0.998452}%
\pgfsetdash{}{0pt}%
\pgfpathmoveto{\pgfqpoint{2.414392in}{1.906707in}}%
\pgfpathcurveto{\pgfqpoint{2.422628in}{1.906707in}}{\pgfqpoint{2.430528in}{1.909980in}}{\pgfqpoint{2.436352in}{1.915804in}}%
\pgfpathcurveto{\pgfqpoint{2.442176in}{1.921628in}}{\pgfqpoint{2.445448in}{1.929528in}}{\pgfqpoint{2.445448in}{1.937764in}}%
\pgfpathcurveto{\pgfqpoint{2.445448in}{1.946000in}}{\pgfqpoint{2.442176in}{1.953900in}}{\pgfqpoint{2.436352in}{1.959724in}}%
\pgfpathcurveto{\pgfqpoint{2.430528in}{1.965548in}}{\pgfqpoint{2.422628in}{1.968820in}}{\pgfqpoint{2.414392in}{1.968820in}}%
\pgfpathcurveto{\pgfqpoint{2.406155in}{1.968820in}}{\pgfqpoint{2.398255in}{1.965548in}}{\pgfqpoint{2.392431in}{1.959724in}}%
\pgfpathcurveto{\pgfqpoint{2.386608in}{1.953900in}}{\pgfqpoint{2.383335in}{1.946000in}}{\pgfqpoint{2.383335in}{1.937764in}}%
\pgfpathcurveto{\pgfqpoint{2.383335in}{1.929528in}}{\pgfqpoint{2.386608in}{1.921628in}}{\pgfqpoint{2.392431in}{1.915804in}}%
\pgfpathcurveto{\pgfqpoint{2.398255in}{1.909980in}}{\pgfqpoint{2.406155in}{1.906707in}}{\pgfqpoint{2.414392in}{1.906707in}}%
\pgfpathclose%
\pgfusepath{stroke,fill}%
\end{pgfscope}%
\begin{pgfscope}%
\pgfpathrectangle{\pgfqpoint{0.100000in}{0.212622in}}{\pgfqpoint{3.696000in}{3.696000in}}%
\pgfusepath{clip}%
\pgfsetbuttcap%
\pgfsetroundjoin%
\definecolor{currentfill}{rgb}{0.121569,0.466667,0.705882}%
\pgfsetfillcolor{currentfill}%
\pgfsetfillopacity{0.998506}%
\pgfsetlinewidth{1.003750pt}%
\definecolor{currentstroke}{rgb}{0.121569,0.466667,0.705882}%
\pgfsetstrokecolor{currentstroke}%
\pgfsetstrokeopacity{0.998506}%
\pgfsetdash{}{0pt}%
\pgfpathmoveto{\pgfqpoint{2.414730in}{1.906577in}}%
\pgfpathcurveto{\pgfqpoint{2.422967in}{1.906577in}}{\pgfqpoint{2.430867in}{1.909850in}}{\pgfqpoint{2.436691in}{1.915674in}}%
\pgfpathcurveto{\pgfqpoint{2.442515in}{1.921498in}}{\pgfqpoint{2.445787in}{1.929398in}}{\pgfqpoint{2.445787in}{1.937634in}}%
\pgfpathcurveto{\pgfqpoint{2.445787in}{1.945870in}}{\pgfqpoint{2.442515in}{1.953770in}}{\pgfqpoint{2.436691in}{1.959594in}}%
\pgfpathcurveto{\pgfqpoint{2.430867in}{1.965418in}}{\pgfqpoint{2.422967in}{1.968690in}}{\pgfqpoint{2.414730in}{1.968690in}}%
\pgfpathcurveto{\pgfqpoint{2.406494in}{1.968690in}}{\pgfqpoint{2.398594in}{1.965418in}}{\pgfqpoint{2.392770in}{1.959594in}}%
\pgfpathcurveto{\pgfqpoint{2.386946in}{1.953770in}}{\pgfqpoint{2.383674in}{1.945870in}}{\pgfqpoint{2.383674in}{1.937634in}}%
\pgfpathcurveto{\pgfqpoint{2.383674in}{1.929398in}}{\pgfqpoint{2.386946in}{1.921498in}}{\pgfqpoint{2.392770in}{1.915674in}}%
\pgfpathcurveto{\pgfqpoint{2.398594in}{1.909850in}}{\pgfqpoint{2.406494in}{1.906577in}}{\pgfqpoint{2.414730in}{1.906577in}}%
\pgfpathclose%
\pgfusepath{stroke,fill}%
\end{pgfscope}%
\begin{pgfscope}%
\pgfpathrectangle{\pgfqpoint{0.100000in}{0.212622in}}{\pgfqpoint{3.696000in}{3.696000in}}%
\pgfusepath{clip}%
\pgfsetbuttcap%
\pgfsetroundjoin%
\definecolor{currentfill}{rgb}{0.121569,0.466667,0.705882}%
\pgfsetfillcolor{currentfill}%
\pgfsetfillopacity{0.998622}%
\pgfsetlinewidth{1.003750pt}%
\definecolor{currentstroke}{rgb}{0.121569,0.466667,0.705882}%
\pgfsetstrokecolor{currentstroke}%
\pgfsetstrokeopacity{0.998622}%
\pgfsetdash{}{0pt}%
\pgfpathmoveto{\pgfqpoint{2.415332in}{1.906425in}}%
\pgfpathcurveto{\pgfqpoint{2.423568in}{1.906425in}}{\pgfqpoint{2.431468in}{1.909697in}}{\pgfqpoint{2.437292in}{1.915521in}}%
\pgfpathcurveto{\pgfqpoint{2.443116in}{1.921345in}}{\pgfqpoint{2.446388in}{1.929245in}}{\pgfqpoint{2.446388in}{1.937481in}}%
\pgfpathcurveto{\pgfqpoint{2.446388in}{1.945718in}}{\pgfqpoint{2.443116in}{1.953618in}}{\pgfqpoint{2.437292in}{1.959442in}}%
\pgfpathcurveto{\pgfqpoint{2.431468in}{1.965266in}}{\pgfqpoint{2.423568in}{1.968538in}}{\pgfqpoint{2.415332in}{1.968538in}}%
\pgfpathcurveto{\pgfqpoint{2.407096in}{1.968538in}}{\pgfqpoint{2.399195in}{1.965266in}}{\pgfqpoint{2.393372in}{1.959442in}}%
\pgfpathcurveto{\pgfqpoint{2.387548in}{1.953618in}}{\pgfqpoint{2.384275in}{1.945718in}}{\pgfqpoint{2.384275in}{1.937481in}}%
\pgfpathcurveto{\pgfqpoint{2.384275in}{1.929245in}}{\pgfqpoint{2.387548in}{1.921345in}}{\pgfqpoint{2.393372in}{1.915521in}}%
\pgfpathcurveto{\pgfqpoint{2.399195in}{1.909697in}}{\pgfqpoint{2.407096in}{1.906425in}}{\pgfqpoint{2.415332in}{1.906425in}}%
\pgfpathclose%
\pgfusepath{stroke,fill}%
\end{pgfscope}%
\begin{pgfscope}%
\pgfpathrectangle{\pgfqpoint{0.100000in}{0.212622in}}{\pgfqpoint{3.696000in}{3.696000in}}%
\pgfusepath{clip}%
\pgfsetbuttcap%
\pgfsetroundjoin%
\definecolor{currentfill}{rgb}{0.121569,0.466667,0.705882}%
\pgfsetfillcolor{currentfill}%
\pgfsetfillopacity{0.998671}%
\pgfsetlinewidth{1.003750pt}%
\definecolor{currentstroke}{rgb}{0.121569,0.466667,0.705882}%
\pgfsetstrokecolor{currentstroke}%
\pgfsetstrokeopacity{0.998671}%
\pgfsetdash{}{0pt}%
\pgfpathmoveto{\pgfqpoint{2.439010in}{1.899918in}}%
\pgfpathcurveto{\pgfqpoint{2.447246in}{1.899918in}}{\pgfqpoint{2.455146in}{1.903190in}}{\pgfqpoint{2.460970in}{1.909014in}}%
\pgfpathcurveto{\pgfqpoint{2.466794in}{1.914838in}}{\pgfqpoint{2.470066in}{1.922738in}}{\pgfqpoint{2.470066in}{1.930974in}}%
\pgfpathcurveto{\pgfqpoint{2.470066in}{1.939211in}}{\pgfqpoint{2.466794in}{1.947111in}}{\pgfqpoint{2.460970in}{1.952935in}}%
\pgfpathcurveto{\pgfqpoint{2.455146in}{1.958759in}}{\pgfqpoint{2.447246in}{1.962031in}}{\pgfqpoint{2.439010in}{1.962031in}}%
\pgfpathcurveto{\pgfqpoint{2.430774in}{1.962031in}}{\pgfqpoint{2.422874in}{1.958759in}}{\pgfqpoint{2.417050in}{1.952935in}}%
\pgfpathcurveto{\pgfqpoint{2.411226in}{1.947111in}}{\pgfqpoint{2.407953in}{1.939211in}}{\pgfqpoint{2.407953in}{1.930974in}}%
\pgfpathcurveto{\pgfqpoint{2.407953in}{1.922738in}}{\pgfqpoint{2.411226in}{1.914838in}}{\pgfqpoint{2.417050in}{1.909014in}}%
\pgfpathcurveto{\pgfqpoint{2.422874in}{1.903190in}}{\pgfqpoint{2.430774in}{1.899918in}}{\pgfqpoint{2.439010in}{1.899918in}}%
\pgfpathclose%
\pgfusepath{stroke,fill}%
\end{pgfscope}%
\begin{pgfscope}%
\pgfpathrectangle{\pgfqpoint{0.100000in}{0.212622in}}{\pgfqpoint{3.696000in}{3.696000in}}%
\pgfusepath{clip}%
\pgfsetbuttcap%
\pgfsetroundjoin%
\definecolor{currentfill}{rgb}{0.121569,0.466667,0.705882}%
\pgfsetfillcolor{currentfill}%
\pgfsetfillopacity{0.998762}%
\pgfsetlinewidth{1.003750pt}%
\definecolor{currentstroke}{rgb}{0.121569,0.466667,0.705882}%
\pgfsetstrokecolor{currentstroke}%
\pgfsetstrokeopacity{0.998762}%
\pgfsetdash{}{0pt}%
\pgfpathmoveto{\pgfqpoint{2.416539in}{1.906005in}}%
\pgfpathcurveto{\pgfqpoint{2.424775in}{1.906005in}}{\pgfqpoint{2.432675in}{1.909277in}}{\pgfqpoint{2.438499in}{1.915101in}}%
\pgfpathcurveto{\pgfqpoint{2.444323in}{1.920925in}}{\pgfqpoint{2.447595in}{1.928825in}}{\pgfqpoint{2.447595in}{1.937061in}}%
\pgfpathcurveto{\pgfqpoint{2.447595in}{1.945298in}}{\pgfqpoint{2.444323in}{1.953198in}}{\pgfqpoint{2.438499in}{1.959022in}}%
\pgfpathcurveto{\pgfqpoint{2.432675in}{1.964846in}}{\pgfqpoint{2.424775in}{1.968118in}}{\pgfqpoint{2.416539in}{1.968118in}}%
\pgfpathcurveto{\pgfqpoint{2.408302in}{1.968118in}}{\pgfqpoint{2.400402in}{1.964846in}}{\pgfqpoint{2.394578in}{1.959022in}}%
\pgfpathcurveto{\pgfqpoint{2.388754in}{1.953198in}}{\pgfqpoint{2.385482in}{1.945298in}}{\pgfqpoint{2.385482in}{1.937061in}}%
\pgfpathcurveto{\pgfqpoint{2.385482in}{1.928825in}}{\pgfqpoint{2.388754in}{1.920925in}}{\pgfqpoint{2.394578in}{1.915101in}}%
\pgfpathcurveto{\pgfqpoint{2.400402in}{1.909277in}}{\pgfqpoint{2.408302in}{1.906005in}}{\pgfqpoint{2.416539in}{1.906005in}}%
\pgfpathclose%
\pgfusepath{stroke,fill}%
\end{pgfscope}%
\begin{pgfscope}%
\pgfpathrectangle{\pgfqpoint{0.100000in}{0.212622in}}{\pgfqpoint{3.696000in}{3.696000in}}%
\pgfusepath{clip}%
\pgfsetbuttcap%
\pgfsetroundjoin%
\definecolor{currentfill}{rgb}{0.121569,0.466667,0.705882}%
\pgfsetfillcolor{currentfill}%
\pgfsetfillopacity{0.998853}%
\pgfsetlinewidth{1.003750pt}%
\definecolor{currentstroke}{rgb}{0.121569,0.466667,0.705882}%
\pgfsetstrokecolor{currentstroke}%
\pgfsetstrokeopacity{0.998853}%
\pgfsetdash{}{0pt}%
\pgfpathmoveto{\pgfqpoint{2.438660in}{1.899955in}}%
\pgfpathcurveto{\pgfqpoint{2.446896in}{1.899955in}}{\pgfqpoint{2.454796in}{1.903228in}}{\pgfqpoint{2.460620in}{1.909052in}}%
\pgfpathcurveto{\pgfqpoint{2.466444in}{1.914876in}}{\pgfqpoint{2.469716in}{1.922776in}}{\pgfqpoint{2.469716in}{1.931012in}}%
\pgfpathcurveto{\pgfqpoint{2.469716in}{1.939248in}}{\pgfqpoint{2.466444in}{1.947148in}}{\pgfqpoint{2.460620in}{1.952972in}}%
\pgfpathcurveto{\pgfqpoint{2.454796in}{1.958796in}}{\pgfqpoint{2.446896in}{1.962068in}}{\pgfqpoint{2.438660in}{1.962068in}}%
\pgfpathcurveto{\pgfqpoint{2.430424in}{1.962068in}}{\pgfqpoint{2.422524in}{1.958796in}}{\pgfqpoint{2.416700in}{1.952972in}}%
\pgfpathcurveto{\pgfqpoint{2.410876in}{1.947148in}}{\pgfqpoint{2.407603in}{1.939248in}}{\pgfqpoint{2.407603in}{1.931012in}}%
\pgfpathcurveto{\pgfqpoint{2.407603in}{1.922776in}}{\pgfqpoint{2.410876in}{1.914876in}}{\pgfqpoint{2.416700in}{1.909052in}}%
\pgfpathcurveto{\pgfqpoint{2.422524in}{1.903228in}}{\pgfqpoint{2.430424in}{1.899955in}}{\pgfqpoint{2.438660in}{1.899955in}}%
\pgfpathclose%
\pgfusepath{stroke,fill}%
\end{pgfscope}%
\begin{pgfscope}%
\pgfpathrectangle{\pgfqpoint{0.100000in}{0.212622in}}{\pgfqpoint{3.696000in}{3.696000in}}%
\pgfusepath{clip}%
\pgfsetbuttcap%
\pgfsetroundjoin%
\definecolor{currentfill}{rgb}{0.121569,0.466667,0.705882}%
\pgfsetfillcolor{currentfill}%
\pgfsetfillopacity{0.998952}%
\pgfsetlinewidth{1.003750pt}%
\definecolor{currentstroke}{rgb}{0.121569,0.466667,0.705882}%
\pgfsetstrokecolor{currentstroke}%
\pgfsetstrokeopacity{0.998952}%
\pgfsetdash{}{0pt}%
\pgfpathmoveto{\pgfqpoint{2.438469in}{1.899967in}}%
\pgfpathcurveto{\pgfqpoint{2.446706in}{1.899967in}}{\pgfqpoint{2.454606in}{1.903239in}}{\pgfqpoint{2.460430in}{1.909063in}}%
\pgfpathcurveto{\pgfqpoint{2.466254in}{1.914887in}}{\pgfqpoint{2.469526in}{1.922787in}}{\pgfqpoint{2.469526in}{1.931023in}}%
\pgfpathcurveto{\pgfqpoint{2.469526in}{1.939260in}}{\pgfqpoint{2.466254in}{1.947160in}}{\pgfqpoint{2.460430in}{1.952984in}}%
\pgfpathcurveto{\pgfqpoint{2.454606in}{1.958808in}}{\pgfqpoint{2.446706in}{1.962080in}}{\pgfqpoint{2.438469in}{1.962080in}}%
\pgfpathcurveto{\pgfqpoint{2.430233in}{1.962080in}}{\pgfqpoint{2.422333in}{1.958808in}}{\pgfqpoint{2.416509in}{1.952984in}}%
\pgfpathcurveto{\pgfqpoint{2.410685in}{1.947160in}}{\pgfqpoint{2.407413in}{1.939260in}}{\pgfqpoint{2.407413in}{1.931023in}}%
\pgfpathcurveto{\pgfqpoint{2.407413in}{1.922787in}}{\pgfqpoint{2.410685in}{1.914887in}}{\pgfqpoint{2.416509in}{1.909063in}}%
\pgfpathcurveto{\pgfqpoint{2.422333in}{1.903239in}}{\pgfqpoint{2.430233in}{1.899967in}}{\pgfqpoint{2.438469in}{1.899967in}}%
\pgfpathclose%
\pgfusepath{stroke,fill}%
\end{pgfscope}%
\begin{pgfscope}%
\pgfpathrectangle{\pgfqpoint{0.100000in}{0.212622in}}{\pgfqpoint{3.696000in}{3.696000in}}%
\pgfusepath{clip}%
\pgfsetbuttcap%
\pgfsetroundjoin%
\definecolor{currentfill}{rgb}{0.121569,0.466667,0.705882}%
\pgfsetfillcolor{currentfill}%
\pgfsetfillopacity{0.999004}%
\pgfsetlinewidth{1.003750pt}%
\definecolor{currentstroke}{rgb}{0.121569,0.466667,0.705882}%
\pgfsetstrokecolor{currentstroke}%
\pgfsetstrokeopacity{0.999004}%
\pgfsetdash{}{0pt}%
\pgfpathmoveto{\pgfqpoint{2.438346in}{1.899979in}}%
\pgfpathcurveto{\pgfqpoint{2.446582in}{1.899979in}}{\pgfqpoint{2.454482in}{1.903251in}}{\pgfqpoint{2.460306in}{1.909075in}}%
\pgfpathcurveto{\pgfqpoint{2.466130in}{1.914899in}}{\pgfqpoint{2.469403in}{1.922799in}}{\pgfqpoint{2.469403in}{1.931035in}}%
\pgfpathcurveto{\pgfqpoint{2.469403in}{1.939271in}}{\pgfqpoint{2.466130in}{1.947171in}}{\pgfqpoint{2.460306in}{1.952995in}}%
\pgfpathcurveto{\pgfqpoint{2.454482in}{1.958819in}}{\pgfqpoint{2.446582in}{1.962092in}}{\pgfqpoint{2.438346in}{1.962092in}}%
\pgfpathcurveto{\pgfqpoint{2.430110in}{1.962092in}}{\pgfqpoint{2.422210in}{1.958819in}}{\pgfqpoint{2.416386in}{1.952995in}}%
\pgfpathcurveto{\pgfqpoint{2.410562in}{1.947171in}}{\pgfqpoint{2.407290in}{1.939271in}}{\pgfqpoint{2.407290in}{1.931035in}}%
\pgfpathcurveto{\pgfqpoint{2.407290in}{1.922799in}}{\pgfqpoint{2.410562in}{1.914899in}}{\pgfqpoint{2.416386in}{1.909075in}}%
\pgfpathcurveto{\pgfqpoint{2.422210in}{1.903251in}}{\pgfqpoint{2.430110in}{1.899979in}}{\pgfqpoint{2.438346in}{1.899979in}}%
\pgfpathclose%
\pgfusepath{stroke,fill}%
\end{pgfscope}%
\begin{pgfscope}%
\pgfpathrectangle{\pgfqpoint{0.100000in}{0.212622in}}{\pgfqpoint{3.696000in}{3.696000in}}%
\pgfusepath{clip}%
\pgfsetbuttcap%
\pgfsetroundjoin%
\definecolor{currentfill}{rgb}{0.121569,0.466667,0.705882}%
\pgfsetfillcolor{currentfill}%
\pgfsetfillopacity{0.999088}%
\pgfsetlinewidth{1.003750pt}%
\definecolor{currentstroke}{rgb}{0.121569,0.466667,0.705882}%
\pgfsetstrokecolor{currentstroke}%
\pgfsetstrokeopacity{0.999088}%
\pgfsetdash{}{0pt}%
\pgfpathmoveto{\pgfqpoint{2.418645in}{1.905427in}}%
\pgfpathcurveto{\pgfqpoint{2.426881in}{1.905427in}}{\pgfqpoint{2.434781in}{1.908700in}}{\pgfqpoint{2.440605in}{1.914524in}}%
\pgfpathcurveto{\pgfqpoint{2.446429in}{1.920348in}}{\pgfqpoint{2.449701in}{1.928248in}}{\pgfqpoint{2.449701in}{1.936484in}}%
\pgfpathcurveto{\pgfqpoint{2.449701in}{1.944720in}}{\pgfqpoint{2.446429in}{1.952620in}}{\pgfqpoint{2.440605in}{1.958444in}}%
\pgfpathcurveto{\pgfqpoint{2.434781in}{1.964268in}}{\pgfqpoint{2.426881in}{1.967540in}}{\pgfqpoint{2.418645in}{1.967540in}}%
\pgfpathcurveto{\pgfqpoint{2.410409in}{1.967540in}}{\pgfqpoint{2.402508in}{1.964268in}}{\pgfqpoint{2.396685in}{1.958444in}}%
\pgfpathcurveto{\pgfqpoint{2.390861in}{1.952620in}}{\pgfqpoint{2.387588in}{1.944720in}}{\pgfqpoint{2.387588in}{1.936484in}}%
\pgfpathcurveto{\pgfqpoint{2.387588in}{1.928248in}}{\pgfqpoint{2.390861in}{1.920348in}}{\pgfqpoint{2.396685in}{1.914524in}}%
\pgfpathcurveto{\pgfqpoint{2.402508in}{1.908700in}}{\pgfqpoint{2.410409in}{1.905427in}}{\pgfqpoint{2.418645in}{1.905427in}}%
\pgfpathclose%
\pgfusepath{stroke,fill}%
\end{pgfscope}%
\begin{pgfscope}%
\pgfpathrectangle{\pgfqpoint{0.100000in}{0.212622in}}{\pgfqpoint{3.696000in}{3.696000in}}%
\pgfusepath{clip}%
\pgfsetbuttcap%
\pgfsetroundjoin%
\definecolor{currentfill}{rgb}{0.121569,0.466667,0.705882}%
\pgfsetfillcolor{currentfill}%
\pgfsetfillopacity{0.999157}%
\pgfsetlinewidth{1.003750pt}%
\definecolor{currentstroke}{rgb}{0.121569,0.466667,0.705882}%
\pgfsetstrokecolor{currentstroke}%
\pgfsetstrokeopacity{0.999157}%
\pgfsetdash{}{0pt}%
\pgfpathmoveto{\pgfqpoint{2.437909in}{1.900030in}}%
\pgfpathcurveto{\pgfqpoint{2.446145in}{1.900030in}}{\pgfqpoint{2.454045in}{1.903303in}}{\pgfqpoint{2.459869in}{1.909127in}}%
\pgfpathcurveto{\pgfqpoint{2.465693in}{1.914950in}}{\pgfqpoint{2.468965in}{1.922851in}}{\pgfqpoint{2.468965in}{1.931087in}}%
\pgfpathcurveto{\pgfqpoint{2.468965in}{1.939323in}}{\pgfqpoint{2.465693in}{1.947223in}}{\pgfqpoint{2.459869in}{1.953047in}}%
\pgfpathcurveto{\pgfqpoint{2.454045in}{1.958871in}}{\pgfqpoint{2.446145in}{1.962143in}}{\pgfqpoint{2.437909in}{1.962143in}}%
\pgfpathcurveto{\pgfqpoint{2.429673in}{1.962143in}}{\pgfqpoint{2.421773in}{1.958871in}}{\pgfqpoint{2.415949in}{1.953047in}}%
\pgfpathcurveto{\pgfqpoint{2.410125in}{1.947223in}}{\pgfqpoint{2.406852in}{1.939323in}}{\pgfqpoint{2.406852in}{1.931087in}}%
\pgfpathcurveto{\pgfqpoint{2.406852in}{1.922851in}}{\pgfqpoint{2.410125in}{1.914950in}}{\pgfqpoint{2.415949in}{1.909127in}}%
\pgfpathcurveto{\pgfqpoint{2.421773in}{1.903303in}}{\pgfqpoint{2.429673in}{1.900030in}}{\pgfqpoint{2.437909in}{1.900030in}}%
\pgfpathclose%
\pgfusepath{stroke,fill}%
\end{pgfscope}%
\begin{pgfscope}%
\pgfpathrectangle{\pgfqpoint{0.100000in}{0.212622in}}{\pgfqpoint{3.696000in}{3.696000in}}%
\pgfusepath{clip}%
\pgfsetbuttcap%
\pgfsetroundjoin%
\definecolor{currentfill}{rgb}{0.121569,0.466667,0.705882}%
\pgfsetfillcolor{currentfill}%
\pgfsetfillopacity{0.999375}%
\pgfsetlinewidth{1.003750pt}%
\definecolor{currentstroke}{rgb}{0.121569,0.466667,0.705882}%
\pgfsetstrokecolor{currentstroke}%
\pgfsetstrokeopacity{0.999375}%
\pgfsetdash{}{0pt}%
\pgfpathmoveto{\pgfqpoint{2.420386in}{1.904922in}}%
\pgfpathcurveto{\pgfqpoint{2.428622in}{1.904922in}}{\pgfqpoint{2.436522in}{1.908194in}}{\pgfqpoint{2.442346in}{1.914018in}}%
\pgfpathcurveto{\pgfqpoint{2.448170in}{1.919842in}}{\pgfqpoint{2.451442in}{1.927742in}}{\pgfqpoint{2.451442in}{1.935978in}}%
\pgfpathcurveto{\pgfqpoint{2.451442in}{1.944214in}}{\pgfqpoint{2.448170in}{1.952114in}}{\pgfqpoint{2.442346in}{1.957938in}}%
\pgfpathcurveto{\pgfqpoint{2.436522in}{1.963762in}}{\pgfqpoint{2.428622in}{1.967035in}}{\pgfqpoint{2.420386in}{1.967035in}}%
\pgfpathcurveto{\pgfqpoint{2.412149in}{1.967035in}}{\pgfqpoint{2.404249in}{1.963762in}}{\pgfqpoint{2.398425in}{1.957938in}}%
\pgfpathcurveto{\pgfqpoint{2.392601in}{1.952114in}}{\pgfqpoint{2.389329in}{1.944214in}}{\pgfqpoint{2.389329in}{1.935978in}}%
\pgfpathcurveto{\pgfqpoint{2.389329in}{1.927742in}}{\pgfqpoint{2.392601in}{1.919842in}}{\pgfqpoint{2.398425in}{1.914018in}}%
\pgfpathcurveto{\pgfqpoint{2.404249in}{1.908194in}}{\pgfqpoint{2.412149in}{1.904922in}}{\pgfqpoint{2.420386in}{1.904922in}}%
\pgfpathclose%
\pgfusepath{stroke,fill}%
\end{pgfscope}%
\begin{pgfscope}%
\pgfpathrectangle{\pgfqpoint{0.100000in}{0.212622in}}{\pgfqpoint{3.696000in}{3.696000in}}%
\pgfusepath{clip}%
\pgfsetbuttcap%
\pgfsetroundjoin%
\definecolor{currentfill}{rgb}{0.121569,0.466667,0.705882}%
\pgfsetfillcolor{currentfill}%
\pgfsetfillopacity{0.999398}%
\pgfsetlinewidth{1.003750pt}%
\definecolor{currentstroke}{rgb}{0.121569,0.466667,0.705882}%
\pgfsetstrokecolor{currentstroke}%
\pgfsetstrokeopacity{0.999398}%
\pgfsetdash{}{0pt}%
\pgfpathmoveto{\pgfqpoint{2.436964in}{1.900182in}}%
\pgfpathcurveto{\pgfqpoint{2.445200in}{1.900182in}}{\pgfqpoint{2.453101in}{1.903454in}}{\pgfqpoint{2.458924in}{1.909278in}}%
\pgfpathcurveto{\pgfqpoint{2.464748in}{1.915102in}}{\pgfqpoint{2.468021in}{1.923002in}}{\pgfqpoint{2.468021in}{1.931238in}}%
\pgfpathcurveto{\pgfqpoint{2.468021in}{1.939474in}}{\pgfqpoint{2.464748in}{1.947374in}}{\pgfqpoint{2.458924in}{1.953198in}}%
\pgfpathcurveto{\pgfqpoint{2.453101in}{1.959022in}}{\pgfqpoint{2.445200in}{1.962295in}}{\pgfqpoint{2.436964in}{1.962295in}}%
\pgfpathcurveto{\pgfqpoint{2.428728in}{1.962295in}}{\pgfqpoint{2.420828in}{1.959022in}}{\pgfqpoint{2.415004in}{1.953198in}}%
\pgfpathcurveto{\pgfqpoint{2.409180in}{1.947374in}}{\pgfqpoint{2.405908in}{1.939474in}}{\pgfqpoint{2.405908in}{1.931238in}}%
\pgfpathcurveto{\pgfqpoint{2.405908in}{1.923002in}}{\pgfqpoint{2.409180in}{1.915102in}}{\pgfqpoint{2.415004in}{1.909278in}}%
\pgfpathcurveto{\pgfqpoint{2.420828in}{1.903454in}}{\pgfqpoint{2.428728in}{1.900182in}}{\pgfqpoint{2.436964in}{1.900182in}}%
\pgfpathclose%
\pgfusepath{stroke,fill}%
\end{pgfscope}%
\begin{pgfscope}%
\pgfpathrectangle{\pgfqpoint{0.100000in}{0.212622in}}{\pgfqpoint{3.696000in}{3.696000in}}%
\pgfusepath{clip}%
\pgfsetbuttcap%
\pgfsetroundjoin%
\definecolor{currentfill}{rgb}{0.121569,0.466667,0.705882}%
\pgfsetfillcolor{currentfill}%
\pgfsetfillopacity{0.999581}%
\pgfsetlinewidth{1.003750pt}%
\definecolor{currentstroke}{rgb}{0.121569,0.466667,0.705882}%
\pgfsetstrokecolor{currentstroke}%
\pgfsetstrokeopacity{0.999581}%
\pgfsetdash{}{0pt}%
\pgfpathmoveto{\pgfqpoint{2.421670in}{1.904219in}}%
\pgfpathcurveto{\pgfqpoint{2.429907in}{1.904219in}}{\pgfqpoint{2.437807in}{1.907492in}}{\pgfqpoint{2.443631in}{1.913316in}}%
\pgfpathcurveto{\pgfqpoint{2.449455in}{1.919140in}}{\pgfqpoint{2.452727in}{1.927040in}}{\pgfqpoint{2.452727in}{1.935276in}}%
\pgfpathcurveto{\pgfqpoint{2.452727in}{1.943512in}}{\pgfqpoint{2.449455in}{1.951412in}}{\pgfqpoint{2.443631in}{1.957236in}}%
\pgfpathcurveto{\pgfqpoint{2.437807in}{1.963060in}}{\pgfqpoint{2.429907in}{1.966332in}}{\pgfqpoint{2.421670in}{1.966332in}}%
\pgfpathcurveto{\pgfqpoint{2.413434in}{1.966332in}}{\pgfqpoint{2.405534in}{1.963060in}}{\pgfqpoint{2.399710in}{1.957236in}}%
\pgfpathcurveto{\pgfqpoint{2.393886in}{1.951412in}}{\pgfqpoint{2.390614in}{1.943512in}}{\pgfqpoint{2.390614in}{1.935276in}}%
\pgfpathcurveto{\pgfqpoint{2.390614in}{1.927040in}}{\pgfqpoint{2.393886in}{1.919140in}}{\pgfqpoint{2.399710in}{1.913316in}}%
\pgfpathcurveto{\pgfqpoint{2.405534in}{1.907492in}}{\pgfqpoint{2.413434in}{1.904219in}}{\pgfqpoint{2.421670in}{1.904219in}}%
\pgfpathclose%
\pgfusepath{stroke,fill}%
\end{pgfscope}%
\begin{pgfscope}%
\pgfpathrectangle{\pgfqpoint{0.100000in}{0.212622in}}{\pgfqpoint{3.696000in}{3.696000in}}%
\pgfusepath{clip}%
\pgfsetbuttcap%
\pgfsetroundjoin%
\definecolor{currentfill}{rgb}{0.121569,0.466667,0.705882}%
\pgfsetfillcolor{currentfill}%
\pgfsetfillopacity{0.999732}%
\pgfsetlinewidth{1.003750pt}%
\definecolor{currentstroke}{rgb}{0.121569,0.466667,0.705882}%
\pgfsetstrokecolor{currentstroke}%
\pgfsetstrokeopacity{0.999732}%
\pgfsetdash{}{0pt}%
\pgfpathmoveto{\pgfqpoint{2.434947in}{1.900671in}}%
\pgfpathcurveto{\pgfqpoint{2.443183in}{1.900671in}}{\pgfqpoint{2.451083in}{1.903943in}}{\pgfqpoint{2.456907in}{1.909767in}}%
\pgfpathcurveto{\pgfqpoint{2.462731in}{1.915591in}}{\pgfqpoint{2.466003in}{1.923491in}}{\pgfqpoint{2.466003in}{1.931727in}}%
\pgfpathcurveto{\pgfqpoint{2.466003in}{1.939964in}}{\pgfqpoint{2.462731in}{1.947864in}}{\pgfqpoint{2.456907in}{1.953688in}}%
\pgfpathcurveto{\pgfqpoint{2.451083in}{1.959512in}}{\pgfqpoint{2.443183in}{1.962784in}}{\pgfqpoint{2.434947in}{1.962784in}}%
\pgfpathcurveto{\pgfqpoint{2.426711in}{1.962784in}}{\pgfqpoint{2.418811in}{1.959512in}}{\pgfqpoint{2.412987in}{1.953688in}}%
\pgfpathcurveto{\pgfqpoint{2.407163in}{1.947864in}}{\pgfqpoint{2.403890in}{1.939964in}}{\pgfqpoint{2.403890in}{1.931727in}}%
\pgfpathcurveto{\pgfqpoint{2.403890in}{1.923491in}}{\pgfqpoint{2.407163in}{1.915591in}}{\pgfqpoint{2.412987in}{1.909767in}}%
\pgfpathcurveto{\pgfqpoint{2.418811in}{1.903943in}}{\pgfqpoint{2.426711in}{1.900671in}}{\pgfqpoint{2.434947in}{1.900671in}}%
\pgfpathclose%
\pgfusepath{stroke,fill}%
\end{pgfscope}%
\begin{pgfscope}%
\pgfpathrectangle{\pgfqpoint{0.100000in}{0.212622in}}{\pgfqpoint{3.696000in}{3.696000in}}%
\pgfusepath{clip}%
\pgfsetbuttcap%
\pgfsetroundjoin%
\definecolor{currentfill}{rgb}{0.121569,0.466667,0.705882}%
\pgfsetfillcolor{currentfill}%
\pgfsetfillopacity{0.999794}%
\pgfsetlinewidth{1.003750pt}%
\definecolor{currentstroke}{rgb}{0.121569,0.466667,0.705882}%
\pgfsetstrokecolor{currentstroke}%
\pgfsetstrokeopacity{0.999794}%
\pgfsetdash{}{0pt}%
\pgfpathmoveto{\pgfqpoint{2.424420in}{1.903218in}}%
\pgfpathcurveto{\pgfqpoint{2.432656in}{1.903218in}}{\pgfqpoint{2.440556in}{1.906490in}}{\pgfqpoint{2.446380in}{1.912314in}}%
\pgfpathcurveto{\pgfqpoint{2.452204in}{1.918138in}}{\pgfqpoint{2.455476in}{1.926038in}}{\pgfqpoint{2.455476in}{1.934274in}}%
\pgfpathcurveto{\pgfqpoint{2.455476in}{1.942510in}}{\pgfqpoint{2.452204in}{1.950410in}}{\pgfqpoint{2.446380in}{1.956234in}}%
\pgfpathcurveto{\pgfqpoint{2.440556in}{1.962058in}}{\pgfqpoint{2.432656in}{1.965331in}}{\pgfqpoint{2.424420in}{1.965331in}}%
\pgfpathcurveto{\pgfqpoint{2.416184in}{1.965331in}}{\pgfqpoint{2.408284in}{1.962058in}}{\pgfqpoint{2.402460in}{1.956234in}}%
\pgfpathcurveto{\pgfqpoint{2.396636in}{1.950410in}}{\pgfqpoint{2.393363in}{1.942510in}}{\pgfqpoint{2.393363in}{1.934274in}}%
\pgfpathcurveto{\pgfqpoint{2.393363in}{1.926038in}}{\pgfqpoint{2.396636in}{1.918138in}}{\pgfqpoint{2.402460in}{1.912314in}}%
\pgfpathcurveto{\pgfqpoint{2.408284in}{1.906490in}}{\pgfqpoint{2.416184in}{1.903218in}}{\pgfqpoint{2.424420in}{1.903218in}}%
\pgfpathclose%
\pgfusepath{stroke,fill}%
\end{pgfscope}%
\begin{pgfscope}%
\pgfpathrectangle{\pgfqpoint{0.100000in}{0.212622in}}{\pgfqpoint{3.696000in}{3.696000in}}%
\pgfusepath{clip}%
\pgfsetbuttcap%
\pgfsetroundjoin%
\definecolor{currentfill}{rgb}{0.121569,0.466667,0.705882}%
\pgfsetfillcolor{currentfill}%
\pgfsetfillopacity{0.999934}%
\pgfsetlinewidth{1.003750pt}%
\definecolor{currentstroke}{rgb}{0.121569,0.466667,0.705882}%
\pgfsetstrokecolor{currentstroke}%
\pgfsetstrokeopacity{0.999934}%
\pgfsetdash{}{0pt}%
\pgfpathmoveto{\pgfqpoint{2.426943in}{1.902570in}}%
\pgfpathcurveto{\pgfqpoint{2.435179in}{1.902570in}}{\pgfqpoint{2.443079in}{1.905843in}}{\pgfqpoint{2.448903in}{1.911666in}}%
\pgfpathcurveto{\pgfqpoint{2.454727in}{1.917490in}}{\pgfqpoint{2.457999in}{1.925390in}}{\pgfqpoint{2.457999in}{1.933627in}}%
\pgfpathcurveto{\pgfqpoint{2.457999in}{1.941863in}}{\pgfqpoint{2.454727in}{1.949763in}}{\pgfqpoint{2.448903in}{1.955587in}}%
\pgfpathcurveto{\pgfqpoint{2.443079in}{1.961411in}}{\pgfqpoint{2.435179in}{1.964683in}}{\pgfqpoint{2.426943in}{1.964683in}}%
\pgfpathcurveto{\pgfqpoint{2.418706in}{1.964683in}}{\pgfqpoint{2.410806in}{1.961411in}}{\pgfqpoint{2.404982in}{1.955587in}}%
\pgfpathcurveto{\pgfqpoint{2.399158in}{1.949763in}}{\pgfqpoint{2.395886in}{1.941863in}}{\pgfqpoint{2.395886in}{1.933627in}}%
\pgfpathcurveto{\pgfqpoint{2.395886in}{1.925390in}}{\pgfqpoint{2.399158in}{1.917490in}}{\pgfqpoint{2.404982in}{1.911666in}}%
\pgfpathcurveto{\pgfqpoint{2.410806in}{1.905843in}}{\pgfqpoint{2.418706in}{1.902570in}}{\pgfqpoint{2.426943in}{1.902570in}}%
\pgfpathclose%
\pgfusepath{stroke,fill}%
\end{pgfscope}%
\begin{pgfscope}%
\pgfpathrectangle{\pgfqpoint{0.100000in}{0.212622in}}{\pgfqpoint{3.696000in}{3.696000in}}%
\pgfusepath{clip}%
\pgfsetbuttcap%
\pgfsetroundjoin%
\definecolor{currentfill}{rgb}{0.121569,0.466667,0.705882}%
\pgfsetfillcolor{currentfill}%
\pgfsetlinewidth{1.003750pt}%
\definecolor{currentstroke}{rgb}{0.121569,0.466667,0.705882}%
\pgfsetstrokecolor{currentstroke}%
\pgfsetdash{}{0pt}%
\pgfpathmoveto{\pgfqpoint{2.431677in}{1.901418in}}%
\pgfpathcurveto{\pgfqpoint{2.439913in}{1.901418in}}{\pgfqpoint{2.447813in}{1.904690in}}{\pgfqpoint{2.453637in}{1.910514in}}%
\pgfpathcurveto{\pgfqpoint{2.459461in}{1.916338in}}{\pgfqpoint{2.462733in}{1.924238in}}{\pgfqpoint{2.462733in}{1.932474in}}%
\pgfpathcurveto{\pgfqpoint{2.462733in}{1.940711in}}{\pgfqpoint{2.459461in}{1.948611in}}{\pgfqpoint{2.453637in}{1.954435in}}%
\pgfpathcurveto{\pgfqpoint{2.447813in}{1.960259in}}{\pgfqpoint{2.439913in}{1.963531in}}{\pgfqpoint{2.431677in}{1.963531in}}%
\pgfpathcurveto{\pgfqpoint{2.423441in}{1.963531in}}{\pgfqpoint{2.415541in}{1.960259in}}{\pgfqpoint{2.409717in}{1.954435in}}%
\pgfpathcurveto{\pgfqpoint{2.403893in}{1.948611in}}{\pgfqpoint{2.400620in}{1.940711in}}{\pgfqpoint{2.400620in}{1.932474in}}%
\pgfpathcurveto{\pgfqpoint{2.400620in}{1.924238in}}{\pgfqpoint{2.403893in}{1.916338in}}{\pgfqpoint{2.409717in}{1.910514in}}%
\pgfpathcurveto{\pgfqpoint{2.415541in}{1.904690in}}{\pgfqpoint{2.423441in}{1.901418in}}{\pgfqpoint{2.431677in}{1.901418in}}%
\pgfpathclose%
\pgfusepath{stroke,fill}%
\end{pgfscope}%
\begin{pgfscope}%
\definecolor{textcolor}{rgb}{0.000000,0.000000,0.000000}%
\pgfsetstrokecolor{textcolor}%
\pgfsetfillcolor{textcolor}%
\pgftext[x=1.948000in,y=3.991956in,,base]{\color{textcolor}\rmfamily\fontsize{12.000000}{14.400000}\selectfont OLEQ}%
\end{pgfscope}%
\begin{pgfscope}%
\pgfsetbuttcap%
\pgfsetmiterjoin%
\definecolor{currentfill}{rgb}{1.000000,1.000000,1.000000}%
\pgfsetfillcolor{currentfill}%
\pgfsetfillopacity{0.800000}%
\pgfsetlinewidth{1.003750pt}%
\definecolor{currentstroke}{rgb}{0.800000,0.800000,0.800000}%
\pgfsetstrokecolor{currentstroke}%
\pgfsetstrokeopacity{0.800000}%
\pgfsetdash{}{0pt}%
\pgfpathmoveto{\pgfqpoint{2.104889in}{3.410289in}}%
\pgfpathlineto{\pgfqpoint{3.698778in}{3.410289in}}%
\pgfpathquadraticcurveto{\pgfqpoint{3.726556in}{3.410289in}}{\pgfqpoint{3.726556in}{3.438067in}}%
\pgfpathlineto{\pgfqpoint{3.726556in}{3.811400in}}%
\pgfpathquadraticcurveto{\pgfqpoint{3.726556in}{3.839178in}}{\pgfqpoint{3.698778in}{3.839178in}}%
\pgfpathlineto{\pgfqpoint{2.104889in}{3.839178in}}%
\pgfpathquadraticcurveto{\pgfqpoint{2.077111in}{3.839178in}}{\pgfqpoint{2.077111in}{3.811400in}}%
\pgfpathlineto{\pgfqpoint{2.077111in}{3.438067in}}%
\pgfpathquadraticcurveto{\pgfqpoint{2.077111in}{3.410289in}}{\pgfqpoint{2.104889in}{3.410289in}}%
\pgfpathclose%
\pgfusepath{stroke,fill}%
\end{pgfscope}%
\begin{pgfscope}%
\pgfsetrectcap%
\pgfsetroundjoin%
\pgfsetlinewidth{1.505625pt}%
\definecolor{currentstroke}{rgb}{0.121569,0.466667,0.705882}%
\pgfsetstrokecolor{currentstroke}%
\pgfsetdash{}{0pt}%
\pgfpathmoveto{\pgfqpoint{2.132667in}{3.735011in}}%
\pgfpathlineto{\pgfqpoint{2.410444in}{3.735011in}}%
\pgfusepath{stroke}%
\end{pgfscope}%
\begin{pgfscope}%
\definecolor{textcolor}{rgb}{0.000000,0.000000,0.000000}%
\pgfsetstrokecolor{textcolor}%
\pgfsetfillcolor{textcolor}%
\pgftext[x=2.521555in,y=3.686400in,left,base]{\color{textcolor}\rmfamily\fontsize{10.000000}{12.000000}\selectfont Ground truth}%
\end{pgfscope}%
\begin{pgfscope}%
\pgfsetbuttcap%
\pgfsetroundjoin%
\definecolor{currentfill}{rgb}{0.121569,0.466667,0.705882}%
\pgfsetfillcolor{currentfill}%
\pgfsetlinewidth{1.003750pt}%
\definecolor{currentstroke}{rgb}{0.121569,0.466667,0.705882}%
\pgfsetstrokecolor{currentstroke}%
\pgfsetdash{}{0pt}%
\pgfsys@defobject{currentmarker}{\pgfqpoint{-0.031056in}{-0.031056in}}{\pgfqpoint{0.031056in}{0.031056in}}{%
\pgfpathmoveto{\pgfqpoint{0.000000in}{-0.031056in}}%
\pgfpathcurveto{\pgfqpoint{0.008236in}{-0.031056in}}{\pgfqpoint{0.016136in}{-0.027784in}}{\pgfqpoint{0.021960in}{-0.021960in}}%
\pgfpathcurveto{\pgfqpoint{0.027784in}{-0.016136in}}{\pgfqpoint{0.031056in}{-0.008236in}}{\pgfqpoint{0.031056in}{0.000000in}}%
\pgfpathcurveto{\pgfqpoint{0.031056in}{0.008236in}}{\pgfqpoint{0.027784in}{0.016136in}}{\pgfqpoint{0.021960in}{0.021960in}}%
\pgfpathcurveto{\pgfqpoint{0.016136in}{0.027784in}}{\pgfqpoint{0.008236in}{0.031056in}}{\pgfqpoint{0.000000in}{0.031056in}}%
\pgfpathcurveto{\pgfqpoint{-0.008236in}{0.031056in}}{\pgfqpoint{-0.016136in}{0.027784in}}{\pgfqpoint{-0.021960in}{0.021960in}}%
\pgfpathcurveto{\pgfqpoint{-0.027784in}{0.016136in}}{\pgfqpoint{-0.031056in}{0.008236in}}{\pgfqpoint{-0.031056in}{0.000000in}}%
\pgfpathcurveto{\pgfqpoint{-0.031056in}{-0.008236in}}{\pgfqpoint{-0.027784in}{-0.016136in}}{\pgfqpoint{-0.021960in}{-0.021960in}}%
\pgfpathcurveto{\pgfqpoint{-0.016136in}{-0.027784in}}{\pgfqpoint{-0.008236in}{-0.031056in}}{\pgfqpoint{0.000000in}{-0.031056in}}%
\pgfpathclose%
\pgfusepath{stroke,fill}%
}%
\begin{pgfscope}%
\pgfsys@transformshift{2.271555in}{3.529248in}%
\pgfsys@useobject{currentmarker}{}%
\end{pgfscope}%
\end{pgfscope}%
\begin{pgfscope}%
\definecolor{textcolor}{rgb}{0.000000,0.000000,0.000000}%
\pgfsetstrokecolor{textcolor}%
\pgfsetfillcolor{textcolor}%
\pgftext[x=2.521555in,y=3.492789in,left,base]{\color{textcolor}\rmfamily\fontsize{10.000000}{12.000000}\selectfont Estimated position}%
\end{pgfscope}%
\end{pgfpicture}%
\makeatother%
\endgroup%
}
% % %         \caption{INS Hardware}
% % %         \label{fig:square282D}
% % %     \end{subfigure}
% % %     \begin{subfigure}{0.49\textwidth}
% % %         \centering
% % %         \resizebox{1\linewidth}{!}{%% Creator: Matplotlib, PGF backend
%%
%% To include the figure in your LaTeX document, write
%%   \input{<filename>.pgf}
%%
%% Make sure the required packages are loaded in your preamble
%%   \usepackage{pgf}
%%
%% and, on pdftex
%%   \usepackage[utf8]{inputenc}\DeclareUnicodeCharacter{2212}{-}
%%
%% or, on luatex and xetex
%%   \usepackage{unicode-math}
%%
%% Figures using additional raster images can only be included by \input if
%% they are in the same directory as the main LaTeX file. For loading figures
%% from other directories you can use the `import` package
%%   \usepackage{import}
%%
%% and then include the figures with
%%   \import{<path to file>}{<filename>.pgf}
%%
%% Matplotlib used the following preamble
%%   \usepackage{fontspec}
%%
\begingroup%
\makeatletter%
\begin{pgfpicture}%
\pgfpathrectangle{\pgfpointorigin}{\pgfqpoint{4.342355in}{4.207622in}}%
\pgfusepath{use as bounding box, clip}%
\begin{pgfscope}%
\pgfsetbuttcap%
\pgfsetmiterjoin%
\definecolor{currentfill}{rgb}{1.000000,1.000000,1.000000}%
\pgfsetfillcolor{currentfill}%
\pgfsetlinewidth{0.000000pt}%
\definecolor{currentstroke}{rgb}{1.000000,1.000000,1.000000}%
\pgfsetstrokecolor{currentstroke}%
\pgfsetdash{}{0pt}%
\pgfpathmoveto{\pgfqpoint{0.000000in}{-0.000000in}}%
\pgfpathlineto{\pgfqpoint{4.342355in}{-0.000000in}}%
\pgfpathlineto{\pgfqpoint{4.342355in}{4.207622in}}%
\pgfpathlineto{\pgfqpoint{0.000000in}{4.207622in}}%
\pgfpathclose%
\pgfusepath{fill}%
\end{pgfscope}%
\begin{pgfscope}%
\pgfsetbuttcap%
\pgfsetmiterjoin%
\definecolor{currentfill}{rgb}{1.000000,1.000000,1.000000}%
\pgfsetfillcolor{currentfill}%
\pgfsetlinewidth{0.000000pt}%
\definecolor{currentstroke}{rgb}{0.000000,0.000000,0.000000}%
\pgfsetstrokecolor{currentstroke}%
\pgfsetstrokeopacity{0.000000}%
\pgfsetdash{}{0pt}%
\pgfpathmoveto{\pgfqpoint{0.100000in}{0.212622in}}%
\pgfpathlineto{\pgfqpoint{3.796000in}{0.212622in}}%
\pgfpathlineto{\pgfqpoint{3.796000in}{3.908622in}}%
\pgfpathlineto{\pgfqpoint{0.100000in}{3.908622in}}%
\pgfpathclose%
\pgfusepath{fill}%
\end{pgfscope}%
\begin{pgfscope}%
\pgfsetbuttcap%
\pgfsetmiterjoin%
\definecolor{currentfill}{rgb}{0.950000,0.950000,0.950000}%
\pgfsetfillcolor{currentfill}%
\pgfsetfillopacity{0.500000}%
\pgfsetlinewidth{1.003750pt}%
\definecolor{currentstroke}{rgb}{0.950000,0.950000,0.950000}%
\pgfsetstrokecolor{currentstroke}%
\pgfsetstrokeopacity{0.500000}%
\pgfsetdash{}{0pt}%
\pgfpathmoveto{\pgfqpoint{0.379073in}{1.123938in}}%
\pgfpathlineto{\pgfqpoint{1.599613in}{2.147018in}}%
\pgfpathlineto{\pgfqpoint{1.582647in}{3.622484in}}%
\pgfpathlineto{\pgfqpoint{0.303698in}{2.689165in}}%
\pgfusepath{stroke,fill}%
\end{pgfscope}%
\begin{pgfscope}%
\pgfsetbuttcap%
\pgfsetmiterjoin%
\definecolor{currentfill}{rgb}{0.900000,0.900000,0.900000}%
\pgfsetfillcolor{currentfill}%
\pgfsetfillopacity{0.500000}%
\pgfsetlinewidth{1.003750pt}%
\definecolor{currentstroke}{rgb}{0.900000,0.900000,0.900000}%
\pgfsetstrokecolor{currentstroke}%
\pgfsetstrokeopacity{0.500000}%
\pgfsetdash{}{0pt}%
\pgfpathmoveto{\pgfqpoint{1.599613in}{2.147018in}}%
\pgfpathlineto{\pgfqpoint{3.558144in}{1.577751in}}%
\pgfpathlineto{\pgfqpoint{3.628038in}{3.104037in}}%
\pgfpathlineto{\pgfqpoint{1.582647in}{3.622484in}}%
\pgfusepath{stroke,fill}%
\end{pgfscope}%
\begin{pgfscope}%
\pgfsetbuttcap%
\pgfsetmiterjoin%
\definecolor{currentfill}{rgb}{0.925000,0.925000,0.925000}%
\pgfsetfillcolor{currentfill}%
\pgfsetfillopacity{0.500000}%
\pgfsetlinewidth{1.003750pt}%
\definecolor{currentstroke}{rgb}{0.925000,0.925000,0.925000}%
\pgfsetstrokecolor{currentstroke}%
\pgfsetstrokeopacity{0.500000}%
\pgfsetdash{}{0pt}%
\pgfpathmoveto{\pgfqpoint{0.379073in}{1.123938in}}%
\pgfpathlineto{\pgfqpoint{2.455212in}{0.445871in}}%
\pgfpathlineto{\pgfqpoint{3.558144in}{1.577751in}}%
\pgfpathlineto{\pgfqpoint{1.599613in}{2.147018in}}%
\pgfusepath{stroke,fill}%
\end{pgfscope}%
\begin{pgfscope}%
\pgfsetrectcap%
\pgfsetroundjoin%
\pgfsetlinewidth{0.803000pt}%
\definecolor{currentstroke}{rgb}{0.000000,0.000000,0.000000}%
\pgfsetstrokecolor{currentstroke}%
\pgfsetdash{}{0pt}%
\pgfpathmoveto{\pgfqpoint{0.379073in}{1.123938in}}%
\pgfpathlineto{\pgfqpoint{2.455212in}{0.445871in}}%
\pgfusepath{stroke}%
\end{pgfscope}%
\begin{pgfscope}%
\definecolor{textcolor}{rgb}{0.000000,0.000000,0.000000}%
\pgfsetstrokecolor{textcolor}%
\pgfsetfillcolor{textcolor}%
\pgftext[x=0.730374in, y=0.408886in, left, base,rotate=341.912962]{\color{textcolor}\rmfamily\fontsize{10.000000}{12.000000}\selectfont Position X [\(\displaystyle m\)]}%
\end{pgfscope}%
\begin{pgfscope}%
\pgfsetbuttcap%
\pgfsetroundjoin%
\pgfsetlinewidth{0.803000pt}%
\definecolor{currentstroke}{rgb}{0.690196,0.690196,0.690196}%
\pgfsetstrokecolor{currentstroke}%
\pgfsetdash{}{0pt}%
\pgfpathmoveto{\pgfqpoint{0.579468in}{1.058489in}}%
\pgfpathlineto{\pgfqpoint{1.789411in}{2.091852in}}%
\pgfpathlineto{\pgfqpoint{1.780485in}{3.572338in}}%
\pgfusepath{stroke}%
\end{pgfscope}%
\begin{pgfscope}%
\pgfsetbuttcap%
\pgfsetroundjoin%
\pgfsetlinewidth{0.803000pt}%
\definecolor{currentstroke}{rgb}{0.690196,0.690196,0.690196}%
\pgfsetstrokecolor{currentstroke}%
\pgfsetdash{}{0pt}%
\pgfpathmoveto{\pgfqpoint{0.831446in}{0.976193in}}%
\pgfpathlineto{\pgfqpoint{2.027836in}{2.022551in}}%
\pgfpathlineto{\pgfqpoint{2.029123in}{3.509315in}}%
\pgfusepath{stroke}%
\end{pgfscope}%
\begin{pgfscope}%
\pgfsetbuttcap%
\pgfsetroundjoin%
\pgfsetlinewidth{0.803000pt}%
\definecolor{currentstroke}{rgb}{0.690196,0.690196,0.690196}%
\pgfsetstrokecolor{currentstroke}%
\pgfsetdash{}{0pt}%
\pgfpathmoveto{\pgfqpoint{1.086741in}{0.892813in}}%
\pgfpathlineto{\pgfqpoint{2.269137in}{1.952414in}}%
\pgfpathlineto{\pgfqpoint{2.280893in}{3.445499in}}%
\pgfusepath{stroke}%
\end{pgfscope}%
\begin{pgfscope}%
\pgfsetbuttcap%
\pgfsetroundjoin%
\pgfsetlinewidth{0.803000pt}%
\definecolor{currentstroke}{rgb}{0.690196,0.690196,0.690196}%
\pgfsetstrokecolor{currentstroke}%
\pgfsetdash{}{0pt}%
\pgfpathmoveto{\pgfqpoint{1.345418in}{0.808329in}}%
\pgfpathlineto{\pgfqpoint{2.513369in}{1.881426in}}%
\pgfpathlineto{\pgfqpoint{2.535852in}{3.380874in}}%
\pgfusepath{stroke}%
\end{pgfscope}%
\begin{pgfscope}%
\pgfsetbuttcap%
\pgfsetroundjoin%
\pgfsetlinewidth{0.803000pt}%
\definecolor{currentstroke}{rgb}{0.690196,0.690196,0.690196}%
\pgfsetstrokecolor{currentstroke}%
\pgfsetdash{}{0pt}%
\pgfpathmoveto{\pgfqpoint{1.607546in}{0.722718in}}%
\pgfpathlineto{\pgfqpoint{2.760585in}{1.809570in}}%
\pgfpathlineto{\pgfqpoint{2.794064in}{3.315425in}}%
\pgfusepath{stroke}%
\end{pgfscope}%
\begin{pgfscope}%
\pgfsetbuttcap%
\pgfsetroundjoin%
\pgfsetlinewidth{0.803000pt}%
\definecolor{currentstroke}{rgb}{0.690196,0.690196,0.690196}%
\pgfsetstrokecolor{currentstroke}%
\pgfsetdash{}{0pt}%
\pgfpathmoveto{\pgfqpoint{1.873194in}{0.635958in}}%
\pgfpathlineto{\pgfqpoint{3.010839in}{1.736831in}}%
\pgfpathlineto{\pgfqpoint{3.055589in}{3.249136in}}%
\pgfusepath{stroke}%
\end{pgfscope}%
\begin{pgfscope}%
\pgfsetbuttcap%
\pgfsetroundjoin%
\pgfsetlinewidth{0.803000pt}%
\definecolor{currentstroke}{rgb}{0.690196,0.690196,0.690196}%
\pgfsetstrokecolor{currentstroke}%
\pgfsetdash{}{0pt}%
\pgfpathmoveto{\pgfqpoint{2.142433in}{0.548024in}}%
\pgfpathlineto{\pgfqpoint{3.264188in}{1.663192in}}%
\pgfpathlineto{\pgfqpoint{3.320493in}{3.181990in}}%
\pgfusepath{stroke}%
\end{pgfscope}%
\begin{pgfscope}%
\pgfsetrectcap%
\pgfsetroundjoin%
\pgfsetlinewidth{0.803000pt}%
\definecolor{currentstroke}{rgb}{0.000000,0.000000,0.000000}%
\pgfsetstrokecolor{currentstroke}%
\pgfsetdash{}{0pt}%
\pgfpathmoveto{\pgfqpoint{0.590006in}{1.067489in}}%
\pgfpathlineto{\pgfqpoint{0.558347in}{1.040450in}}%
\pgfusepath{stroke}%
\end{pgfscope}%
\begin{pgfscope}%
\definecolor{textcolor}{rgb}{0.000000,0.000000,0.000000}%
\pgfsetstrokecolor{textcolor}%
\pgfsetfillcolor{textcolor}%
\pgftext[x=0.474972in,y=0.839769in,,top]{\color{textcolor}\rmfamily\fontsize{10.000000}{12.000000}\selectfont \(\displaystyle {0}\)}%
\end{pgfscope}%
\begin{pgfscope}%
\pgfsetrectcap%
\pgfsetroundjoin%
\pgfsetlinewidth{0.803000pt}%
\definecolor{currentstroke}{rgb}{0.000000,0.000000,0.000000}%
\pgfsetstrokecolor{currentstroke}%
\pgfsetdash{}{0pt}%
\pgfpathmoveto{\pgfqpoint{0.841871in}{0.985311in}}%
\pgfpathlineto{\pgfqpoint{0.810550in}{0.957917in}}%
\pgfusepath{stroke}%
\end{pgfscope}%
\begin{pgfscope}%
\definecolor{textcolor}{rgb}{0.000000,0.000000,0.000000}%
\pgfsetstrokecolor{textcolor}%
\pgfsetfillcolor{textcolor}%
\pgftext[x=0.727209in,y=0.755729in,,top]{\color{textcolor}\rmfamily\fontsize{10.000000}{12.000000}\selectfont \(\displaystyle {5}\)}%
\end{pgfscope}%
\begin{pgfscope}%
\pgfsetrectcap%
\pgfsetroundjoin%
\pgfsetlinewidth{0.803000pt}%
\definecolor{currentstroke}{rgb}{0.000000,0.000000,0.000000}%
\pgfsetstrokecolor{currentstroke}%
\pgfsetdash{}{0pt}%
\pgfpathmoveto{\pgfqpoint{1.097050in}{0.902052in}}%
\pgfpathlineto{\pgfqpoint{1.066078in}{0.874297in}}%
\pgfusepath{stroke}%
\end{pgfscope}%
\begin{pgfscope}%
\definecolor{textcolor}{rgb}{0.000000,0.000000,0.000000}%
\pgfsetstrokecolor{textcolor}%
\pgfsetfillcolor{textcolor}%
\pgftext[x=0.982779in,y=0.670578in,,top]{\color{textcolor}\rmfamily\fontsize{10.000000}{12.000000}\selectfont \(\displaystyle {10}\)}%
\end{pgfscope}%
\begin{pgfscope}%
\pgfsetrectcap%
\pgfsetroundjoin%
\pgfsetlinewidth{0.803000pt}%
\definecolor{currentstroke}{rgb}{0.000000,0.000000,0.000000}%
\pgfsetstrokecolor{currentstroke}%
\pgfsetdash{}{0pt}%
\pgfpathmoveto{\pgfqpoint{1.355607in}{0.817690in}}%
\pgfpathlineto{\pgfqpoint{1.324997in}{0.789566in}}%
\pgfusepath{stroke}%
\end{pgfscope}%
\begin{pgfscope}%
\definecolor{textcolor}{rgb}{0.000000,0.000000,0.000000}%
\pgfsetstrokecolor{textcolor}%
\pgfsetfillcolor{textcolor}%
\pgftext[x=1.241747in,y=0.584295in,,top]{\color{textcolor}\rmfamily\fontsize{10.000000}{12.000000}\selectfont \(\displaystyle {15}\)}%
\end{pgfscope}%
\begin{pgfscope}%
\pgfsetrectcap%
\pgfsetroundjoin%
\pgfsetlinewidth{0.803000pt}%
\definecolor{currentstroke}{rgb}{0.000000,0.000000,0.000000}%
\pgfsetstrokecolor{currentstroke}%
\pgfsetdash{}{0pt}%
\pgfpathmoveto{\pgfqpoint{1.617610in}{0.732205in}}%
\pgfpathlineto{\pgfqpoint{1.587374in}{0.703704in}}%
\pgfusepath{stroke}%
\end{pgfscope}%
\begin{pgfscope}%
\definecolor{textcolor}{rgb}{0.000000,0.000000,0.000000}%
\pgfsetstrokecolor{textcolor}%
\pgfsetfillcolor{textcolor}%
\pgftext[x=1.504184in,y=0.496857in,,top]{\color{textcolor}\rmfamily\fontsize{10.000000}{12.000000}\selectfont \(\displaystyle {20}\)}%
\end{pgfscope}%
\begin{pgfscope}%
\pgfsetrectcap%
\pgfsetroundjoin%
\pgfsetlinewidth{0.803000pt}%
\definecolor{currentstroke}{rgb}{0.000000,0.000000,0.000000}%
\pgfsetstrokecolor{currentstroke}%
\pgfsetdash{}{0pt}%
\pgfpathmoveto{\pgfqpoint{1.883129in}{0.645572in}}%
\pgfpathlineto{\pgfqpoint{1.853279in}{0.616687in}}%
\pgfusepath{stroke}%
\end{pgfscope}%
\begin{pgfscope}%
\definecolor{textcolor}{rgb}{0.000000,0.000000,0.000000}%
\pgfsetstrokecolor{textcolor}%
\pgfsetfillcolor{textcolor}%
\pgftext[x=1.770157in,y=0.408240in,,top]{\color{textcolor}\rmfamily\fontsize{10.000000}{12.000000}\selectfont \(\displaystyle {25}\)}%
\end{pgfscope}%
\begin{pgfscope}%
\pgfsetrectcap%
\pgfsetroundjoin%
\pgfsetlinewidth{0.803000pt}%
\definecolor{currentstroke}{rgb}{0.000000,0.000000,0.000000}%
\pgfsetstrokecolor{currentstroke}%
\pgfsetdash{}{0pt}%
\pgfpathmoveto{\pgfqpoint{2.152235in}{0.557769in}}%
\pgfpathlineto{\pgfqpoint{2.122785in}{0.528492in}}%
\pgfusepath{stroke}%
\end{pgfscope}%
\begin{pgfscope}%
\definecolor{textcolor}{rgb}{0.000000,0.000000,0.000000}%
\pgfsetstrokecolor{textcolor}%
\pgfsetfillcolor{textcolor}%
\pgftext[x=2.039741in,y=0.318421in,,top]{\color{textcolor}\rmfamily\fontsize{10.000000}{12.000000}\selectfont \(\displaystyle {30}\)}%
\end{pgfscope}%
\begin{pgfscope}%
\pgfsetrectcap%
\pgfsetroundjoin%
\pgfsetlinewidth{0.803000pt}%
\definecolor{currentstroke}{rgb}{0.000000,0.000000,0.000000}%
\pgfsetstrokecolor{currentstroke}%
\pgfsetdash{}{0pt}%
\pgfpathmoveto{\pgfqpoint{3.558144in}{1.577751in}}%
\pgfpathlineto{\pgfqpoint{2.455212in}{0.445871in}}%
\pgfusepath{stroke}%
\end{pgfscope}%
\begin{pgfscope}%
\definecolor{textcolor}{rgb}{0.000000,0.000000,0.000000}%
\pgfsetstrokecolor{textcolor}%
\pgfsetfillcolor{textcolor}%
\pgftext[x=3.120747in, y=0.305657in, left, base,rotate=45.742112]{\color{textcolor}\rmfamily\fontsize{10.000000}{12.000000}\selectfont Position Y [\(\displaystyle m\)]}%
\end{pgfscope}%
\begin{pgfscope}%
\pgfsetbuttcap%
\pgfsetroundjoin%
\pgfsetlinewidth{0.803000pt}%
\definecolor{currentstroke}{rgb}{0.690196,0.690196,0.690196}%
\pgfsetstrokecolor{currentstroke}%
\pgfsetdash{}{0pt}%
\pgfpathmoveto{\pgfqpoint{0.407438in}{2.764870in}}%
\pgfpathlineto{\pgfqpoint{0.477731in}{1.206634in}}%
\pgfpathlineto{\pgfqpoint{2.544726in}{0.537734in}}%
\pgfusepath{stroke}%
\end{pgfscope}%
\begin{pgfscope}%
\pgfsetbuttcap%
\pgfsetroundjoin%
\pgfsetlinewidth{0.803000pt}%
\definecolor{currentstroke}{rgb}{0.690196,0.690196,0.690196}%
\pgfsetstrokecolor{currentstroke}%
\pgfsetdash{}{0pt}%
\pgfpathmoveto{\pgfqpoint{0.586642in}{2.895645in}}%
\pgfpathlineto{\pgfqpoint{0.648298in}{1.349607in}}%
\pgfpathlineto{\pgfqpoint{2.699333in}{0.696399in}}%
\pgfusepath{stroke}%
\end{pgfscope}%
\begin{pgfscope}%
\pgfsetbuttcap%
\pgfsetroundjoin%
\pgfsetlinewidth{0.803000pt}%
\definecolor{currentstroke}{rgb}{0.690196,0.690196,0.690196}%
\pgfsetstrokecolor{currentstroke}%
\pgfsetdash{}{0pt}%
\pgfpathmoveto{\pgfqpoint{0.761590in}{3.023314in}}%
\pgfpathlineto{\pgfqpoint{0.814989in}{1.489331in}}%
\pgfpathlineto{\pgfqpoint{2.850243in}{0.851269in}}%
\pgfusepath{stroke}%
\end{pgfscope}%
\begin{pgfscope}%
\pgfsetbuttcap%
\pgfsetroundjoin%
\pgfsetlinewidth{0.803000pt}%
\definecolor{currentstroke}{rgb}{0.690196,0.690196,0.690196}%
\pgfsetstrokecolor{currentstroke}%
\pgfsetdash{}{0pt}%
\pgfpathmoveto{\pgfqpoint{0.932430in}{3.147985in}}%
\pgfpathlineto{\pgfqpoint{0.977934in}{1.625914in}}%
\pgfpathlineto{\pgfqpoint{2.997585in}{1.002479in}}%
\pgfusepath{stroke}%
\end{pgfscope}%
\begin{pgfscope}%
\pgfsetbuttcap%
\pgfsetroundjoin%
\pgfsetlinewidth{0.803000pt}%
\definecolor{currentstroke}{rgb}{0.690196,0.690196,0.690196}%
\pgfsetstrokecolor{currentstroke}%
\pgfsetdash{}{0pt}%
\pgfpathmoveto{\pgfqpoint{1.099307in}{3.269764in}}%
\pgfpathlineto{\pgfqpoint{1.137258in}{1.759463in}}%
\pgfpathlineto{\pgfqpoint{3.141485in}{1.150156in}}%
\pgfusepath{stroke}%
\end{pgfscope}%
\begin{pgfscope}%
\pgfsetbuttcap%
\pgfsetroundjoin%
\pgfsetlinewidth{0.803000pt}%
\definecolor{currentstroke}{rgb}{0.690196,0.690196,0.690196}%
\pgfsetstrokecolor{currentstroke}%
\pgfsetdash{}{0pt}%
\pgfpathmoveto{\pgfqpoint{1.262356in}{3.388750in}}%
\pgfpathlineto{\pgfqpoint{1.293080in}{1.890076in}}%
\pgfpathlineto{\pgfqpoint{3.282062in}{1.294422in}}%
\pgfusepath{stroke}%
\end{pgfscope}%
\begin{pgfscope}%
\pgfsetbuttcap%
\pgfsetroundjoin%
\pgfsetlinewidth{0.803000pt}%
\definecolor{currentstroke}{rgb}{0.690196,0.690196,0.690196}%
\pgfsetstrokecolor{currentstroke}%
\pgfsetdash{}{0pt}%
\pgfpathmoveto{\pgfqpoint{1.421708in}{3.505038in}}%
\pgfpathlineto{\pgfqpoint{1.445514in}{2.017849in}}%
\pgfpathlineto{\pgfqpoint{3.419430in}{1.435396in}}%
\pgfusepath{stroke}%
\end{pgfscope}%
\begin{pgfscope}%
\pgfsetrectcap%
\pgfsetroundjoin%
\pgfsetlinewidth{0.803000pt}%
\definecolor{currentstroke}{rgb}{0.000000,0.000000,0.000000}%
\pgfsetstrokecolor{currentstroke}%
\pgfsetdash{}{0pt}%
\pgfpathmoveto{\pgfqpoint{2.527308in}{0.543370in}}%
\pgfpathlineto{\pgfqpoint{2.579606in}{0.526446in}}%
\pgfusepath{stroke}%
\end{pgfscope}%
\begin{pgfscope}%
\definecolor{textcolor}{rgb}{0.000000,0.000000,0.000000}%
\pgfsetstrokecolor{textcolor}%
\pgfsetfillcolor{textcolor}%
\pgftext[x=2.723447in,y=0.351500in,,top]{\color{textcolor}\rmfamily\fontsize{10.000000}{12.000000}\selectfont \(\displaystyle {0}\)}%
\end{pgfscope}%
\begin{pgfscope}%
\pgfsetrectcap%
\pgfsetroundjoin%
\pgfsetlinewidth{0.803000pt}%
\definecolor{currentstroke}{rgb}{0.000000,0.000000,0.000000}%
\pgfsetstrokecolor{currentstroke}%
\pgfsetdash{}{0pt}%
\pgfpathmoveto{\pgfqpoint{2.682061in}{0.701900in}}%
\pgfpathlineto{\pgfqpoint{2.733923in}{0.685383in}}%
\pgfusepath{stroke}%
\end{pgfscope}%
\begin{pgfscope}%
\definecolor{textcolor}{rgb}{0.000000,0.000000,0.000000}%
\pgfsetstrokecolor{textcolor}%
\pgfsetfillcolor{textcolor}%
\pgftext[x=2.875981in,y=0.512515in,,top]{\color{textcolor}\rmfamily\fontsize{10.000000}{12.000000}\selectfont \(\displaystyle {5}\)}%
\end{pgfscope}%
\begin{pgfscope}%
\pgfsetrectcap%
\pgfsetroundjoin%
\pgfsetlinewidth{0.803000pt}%
\definecolor{currentstroke}{rgb}{0.000000,0.000000,0.000000}%
\pgfsetstrokecolor{currentstroke}%
\pgfsetdash{}{0pt}%
\pgfpathmoveto{\pgfqpoint{2.833113in}{0.856640in}}%
\pgfpathlineto{\pgfqpoint{2.884545in}{0.840516in}}%
\pgfusepath{stroke}%
\end{pgfscope}%
\begin{pgfscope}%
\definecolor{textcolor}{rgb}{0.000000,0.000000,0.000000}%
\pgfsetstrokecolor{textcolor}%
\pgfsetfillcolor{textcolor}%
\pgftext[x=3.024864in,y=0.669676in,,top]{\color{textcolor}\rmfamily\fontsize{10.000000}{12.000000}\selectfont \(\displaystyle {10}\)}%
\end{pgfscope}%
\begin{pgfscope}%
\pgfsetrectcap%
\pgfsetroundjoin%
\pgfsetlinewidth{0.803000pt}%
\definecolor{currentstroke}{rgb}{0.000000,0.000000,0.000000}%
\pgfsetstrokecolor{currentstroke}%
\pgfsetdash{}{0pt}%
\pgfpathmoveto{\pgfqpoint{2.980597in}{1.007723in}}%
\pgfpathlineto{\pgfqpoint{3.031604in}{0.991978in}}%
\pgfusepath{stroke}%
\end{pgfscope}%
\begin{pgfscope}%
\definecolor{textcolor}{rgb}{0.000000,0.000000,0.000000}%
\pgfsetstrokecolor{textcolor}%
\pgfsetfillcolor{textcolor}%
\pgftext[x=3.170227in,y=0.823120in,,top]{\color{textcolor}\rmfamily\fontsize{10.000000}{12.000000}\selectfont \(\displaystyle {15}\)}%
\end{pgfscope}%
\begin{pgfscope}%
\pgfsetrectcap%
\pgfsetroundjoin%
\pgfsetlinewidth{0.803000pt}%
\definecolor{currentstroke}{rgb}{0.000000,0.000000,0.000000}%
\pgfsetstrokecolor{currentstroke}%
\pgfsetdash{}{0pt}%
\pgfpathmoveto{\pgfqpoint{3.124636in}{1.155278in}}%
\pgfpathlineto{\pgfqpoint{3.175224in}{1.139899in}}%
\pgfusepath{stroke}%
\end{pgfscope}%
\begin{pgfscope}%
\definecolor{textcolor}{rgb}{0.000000,0.000000,0.000000}%
\pgfsetstrokecolor{textcolor}%
\pgfsetfillcolor{textcolor}%
\pgftext[x=3.312191in,y=0.972977in,,top]{\color{textcolor}\rmfamily\fontsize{10.000000}{12.000000}\selectfont \(\displaystyle {20}\)}%
\end{pgfscope}%
\begin{pgfscope}%
\pgfsetrectcap%
\pgfsetroundjoin%
\pgfsetlinewidth{0.803000pt}%
\definecolor{currentstroke}{rgb}{0.000000,0.000000,0.000000}%
\pgfsetstrokecolor{currentstroke}%
\pgfsetdash{}{0pt}%
\pgfpathmoveto{\pgfqpoint{3.265351in}{1.299427in}}%
\pgfpathlineto{\pgfqpoint{3.315526in}{1.284401in}}%
\pgfusepath{stroke}%
\end{pgfscope}%
\begin{pgfscope}%
\definecolor{textcolor}{rgb}{0.000000,0.000000,0.000000}%
\pgfsetstrokecolor{textcolor}%
\pgfsetfillcolor{textcolor}%
\pgftext[x=3.450875in,y=1.119372in,,top]{\color{textcolor}\rmfamily\fontsize{10.000000}{12.000000}\selectfont \(\displaystyle {25}\)}%
\end{pgfscope}%
\begin{pgfscope}%
\pgfsetrectcap%
\pgfsetroundjoin%
\pgfsetlinewidth{0.803000pt}%
\definecolor{currentstroke}{rgb}{0.000000,0.000000,0.000000}%
\pgfsetstrokecolor{currentstroke}%
\pgfsetdash{}{0pt}%
\pgfpathmoveto{\pgfqpoint{3.402854in}{1.440287in}}%
\pgfpathlineto{\pgfqpoint{3.452621in}{1.425602in}}%
\pgfusepath{stroke}%
\end{pgfscope}%
}
% % %         \caption{MPU-9250 Breakout}
% % %         \label{fig:square283D}
% % %     \end{subfigure}
% % %     \caption{Position estimation by the best performing algorithms in the 4-meter line experiment.}
% % %     \label{fig:square28}
% % % \end{figure}

% % % % \subsection{Squares}

% % % % % \begin{figure}[!h]
% % % % %     \centering
% % % % %     \begin{subfigure}{0.49\textwidth}
% % % % %         \centering
% % % % %         \resizebox{1\linewidth}{!}{%% Creator: Matplotlib, PGF backend
%%
%% To include the figure in your LaTeX document, write
%%   \input{<filename>.pgf}
%%
%% Make sure the required packages are loaded in your preamble
%%   \usepackage{pgf}
%%
%% and, on pdftex
%%   \usepackage[utf8]{inputenc}\DeclareUnicodeCharacter{2212}{-}
%%
%% or, on luatex and xetex
%%   \usepackage{unicode-math}
%%
%% Figures using additional raster images can only be included by \input if
%% they are in the same directory as the main LaTeX file. For loading figures
%% from other directories you can use the `import` package
%%   \usepackage{import}
%%
%% and then include the figures with
%%   \import{<path to file>}{<filename>.pgf}
%%
%% Matplotlib used the following preamble
%%   \usepackage{fontspec}
%%
\begingroup%
\makeatletter%
\begin{pgfpicture}%
\pgfpathrectangle{\pgfpointorigin}{\pgfqpoint{5.590556in}{4.311000in}}%
\pgfusepath{use as bounding box, clip}%
\begin{pgfscope}%
\pgfsetbuttcap%
\pgfsetmiterjoin%
\definecolor{currentfill}{rgb}{1.000000,1.000000,1.000000}%
\pgfsetfillcolor{currentfill}%
\pgfsetlinewidth{0.000000pt}%
\definecolor{currentstroke}{rgb}{1.000000,1.000000,1.000000}%
\pgfsetstrokecolor{currentstroke}%
\pgfsetdash{}{0pt}%
\pgfpathmoveto{\pgfqpoint{0.000000in}{0.000000in}}%
\pgfpathlineto{\pgfqpoint{5.590556in}{0.000000in}}%
\pgfpathlineto{\pgfqpoint{5.590556in}{4.311000in}}%
\pgfpathlineto{\pgfqpoint{0.000000in}{4.311000in}}%
\pgfpathclose%
\pgfusepath{fill}%
\end{pgfscope}%
\begin{pgfscope}%
\pgfsetbuttcap%
\pgfsetmiterjoin%
\definecolor{currentfill}{rgb}{1.000000,1.000000,1.000000}%
\pgfsetfillcolor{currentfill}%
\pgfsetlinewidth{0.000000pt}%
\definecolor{currentstroke}{rgb}{0.000000,0.000000,0.000000}%
\pgfsetstrokecolor{currentstroke}%
\pgfsetstrokeopacity{0.000000}%
\pgfsetdash{}{0pt}%
\pgfpathmoveto{\pgfqpoint{0.530556in}{0.515000in}}%
\pgfpathlineto{\pgfqpoint{5.490556in}{0.515000in}}%
\pgfpathlineto{\pgfqpoint{5.490556in}{4.211000in}}%
\pgfpathlineto{\pgfqpoint{0.530556in}{4.211000in}}%
\pgfpathclose%
\pgfusepath{fill}%
\end{pgfscope}%
\begin{pgfscope}%
\pgfpathrectangle{\pgfqpoint{0.530556in}{0.515000in}}{\pgfqpoint{4.960000in}{3.696000in}}%
\pgfusepath{clip}%
\pgfsetbuttcap%
\pgfsetroundjoin%
\definecolor{currentfill}{rgb}{0.121569,0.466667,0.705882}%
\pgfsetfillcolor{currentfill}%
\pgfsetlinewidth{1.003750pt}%
\definecolor{currentstroke}{rgb}{0.121569,0.466667,0.705882}%
\pgfsetstrokecolor{currentstroke}%
\pgfsetdash{}{0pt}%
\pgfsys@defobject{currentmarker}{\pgfqpoint{-0.041667in}{-0.041667in}}{\pgfqpoint{0.041667in}{0.041667in}}{%
\pgfpathmoveto{\pgfqpoint{0.000000in}{-0.041667in}}%
\pgfpathcurveto{\pgfqpoint{0.011050in}{-0.041667in}}{\pgfqpoint{0.021649in}{-0.037276in}}{\pgfqpoint{0.029463in}{-0.029463in}}%
\pgfpathcurveto{\pgfqpoint{0.037276in}{-0.021649in}}{\pgfqpoint{0.041667in}{-0.011050in}}{\pgfqpoint{0.041667in}{0.000000in}}%
\pgfpathcurveto{\pgfqpoint{0.041667in}{0.011050in}}{\pgfqpoint{0.037276in}{0.021649in}}{\pgfqpoint{0.029463in}{0.029463in}}%
\pgfpathcurveto{\pgfqpoint{0.021649in}{0.037276in}}{\pgfqpoint{0.011050in}{0.041667in}}{\pgfqpoint{0.000000in}{0.041667in}}%
\pgfpathcurveto{\pgfqpoint{-0.011050in}{0.041667in}}{\pgfqpoint{-0.021649in}{0.037276in}}{\pgfqpoint{-0.029463in}{0.029463in}}%
\pgfpathcurveto{\pgfqpoint{-0.037276in}{0.021649in}}{\pgfqpoint{-0.041667in}{0.011050in}}{\pgfqpoint{-0.041667in}{0.000000in}}%
\pgfpathcurveto{\pgfqpoint{-0.041667in}{-0.011050in}}{\pgfqpoint{-0.037276in}{-0.021649in}}{\pgfqpoint{-0.029463in}{-0.029463in}}%
\pgfpathcurveto{\pgfqpoint{-0.021649in}{-0.037276in}}{\pgfqpoint{-0.011050in}{-0.041667in}}{\pgfqpoint{0.000000in}{-0.041667in}}%
\pgfpathclose%
\pgfusepath{stroke,fill}%
}%
\begin{pgfscope}%
\pgfsys@transformshift{1.244158in}{0.884903in}%
\pgfsys@useobject{currentmarker}{}%
\end{pgfscope}%
\begin{pgfscope}%
\pgfsys@transformshift{1.244158in}{0.884903in}%
\pgfsys@useobject{currentmarker}{}%
\end{pgfscope}%
\begin{pgfscope}%
\pgfsys@transformshift{1.244158in}{0.884903in}%
\pgfsys@useobject{currentmarker}{}%
\end{pgfscope}%
\begin{pgfscope}%
\pgfsys@transformshift{1.244158in}{0.884903in}%
\pgfsys@useobject{currentmarker}{}%
\end{pgfscope}%
\begin{pgfscope}%
\pgfsys@transformshift{1.244158in}{0.884903in}%
\pgfsys@useobject{currentmarker}{}%
\end{pgfscope}%
\begin{pgfscope}%
\pgfsys@transformshift{1.244158in}{0.884903in}%
\pgfsys@useobject{currentmarker}{}%
\end{pgfscope}%
\begin{pgfscope}%
\pgfsys@transformshift{1.244158in}{0.884903in}%
\pgfsys@useobject{currentmarker}{}%
\end{pgfscope}%
\begin{pgfscope}%
\pgfsys@transformshift{1.244158in}{0.884903in}%
\pgfsys@useobject{currentmarker}{}%
\end{pgfscope}%
\begin{pgfscope}%
\pgfsys@transformshift{1.244158in}{0.884903in}%
\pgfsys@useobject{currentmarker}{}%
\end{pgfscope}%
\begin{pgfscope}%
\pgfsys@transformshift{1.244158in}{0.884903in}%
\pgfsys@useobject{currentmarker}{}%
\end{pgfscope}%
\begin{pgfscope}%
\pgfsys@transformshift{1.244158in}{0.884903in}%
\pgfsys@useobject{currentmarker}{}%
\end{pgfscope}%
\begin{pgfscope}%
\pgfsys@transformshift{1.244158in}{0.884903in}%
\pgfsys@useobject{currentmarker}{}%
\end{pgfscope}%
\begin{pgfscope}%
\pgfsys@transformshift{1.244158in}{0.884903in}%
\pgfsys@useobject{currentmarker}{}%
\end{pgfscope}%
\begin{pgfscope}%
\pgfsys@transformshift{1.244158in}{0.884903in}%
\pgfsys@useobject{currentmarker}{}%
\end{pgfscope}%
\begin{pgfscope}%
\pgfsys@transformshift{1.244158in}{0.884903in}%
\pgfsys@useobject{currentmarker}{}%
\end{pgfscope}%
\begin{pgfscope}%
\pgfsys@transformshift{1.244158in}{0.884903in}%
\pgfsys@useobject{currentmarker}{}%
\end{pgfscope}%
\begin{pgfscope}%
\pgfsys@transformshift{1.244158in}{0.884903in}%
\pgfsys@useobject{currentmarker}{}%
\end{pgfscope}%
\begin{pgfscope}%
\pgfsys@transformshift{1.244158in}{0.884903in}%
\pgfsys@useobject{currentmarker}{}%
\end{pgfscope}%
\begin{pgfscope}%
\pgfsys@transformshift{1.244158in}{0.884903in}%
\pgfsys@useobject{currentmarker}{}%
\end{pgfscope}%
\begin{pgfscope}%
\pgfsys@transformshift{1.244158in}{0.884903in}%
\pgfsys@useobject{currentmarker}{}%
\end{pgfscope}%
\begin{pgfscope}%
\pgfsys@transformshift{1.244158in}{0.884903in}%
\pgfsys@useobject{currentmarker}{}%
\end{pgfscope}%
\begin{pgfscope}%
\pgfsys@transformshift{1.244158in}{0.884903in}%
\pgfsys@useobject{currentmarker}{}%
\end{pgfscope}%
\begin{pgfscope}%
\pgfsys@transformshift{1.244158in}{0.884903in}%
\pgfsys@useobject{currentmarker}{}%
\end{pgfscope}%
\begin{pgfscope}%
\pgfsys@transformshift{1.244158in}{0.884903in}%
\pgfsys@useobject{currentmarker}{}%
\end{pgfscope}%
\begin{pgfscope}%
\pgfsys@transformshift{1.244158in}{0.884903in}%
\pgfsys@useobject{currentmarker}{}%
\end{pgfscope}%
\begin{pgfscope}%
\pgfsys@transformshift{1.244158in}{0.884903in}%
\pgfsys@useobject{currentmarker}{}%
\end{pgfscope}%
\begin{pgfscope}%
\pgfsys@transformshift{1.244158in}{0.884903in}%
\pgfsys@useobject{currentmarker}{}%
\end{pgfscope}%
\begin{pgfscope}%
\pgfsys@transformshift{1.244158in}{0.884903in}%
\pgfsys@useobject{currentmarker}{}%
\end{pgfscope}%
\begin{pgfscope}%
\pgfsys@transformshift{1.244158in}{0.884903in}%
\pgfsys@useobject{currentmarker}{}%
\end{pgfscope}%
\begin{pgfscope}%
\pgfsys@transformshift{1.244158in}{0.884903in}%
\pgfsys@useobject{currentmarker}{}%
\end{pgfscope}%
\begin{pgfscope}%
\pgfsys@transformshift{1.244158in}{0.884903in}%
\pgfsys@useobject{currentmarker}{}%
\end{pgfscope}%
\begin{pgfscope}%
\pgfsys@transformshift{1.244158in}{0.884903in}%
\pgfsys@useobject{currentmarker}{}%
\end{pgfscope}%
\begin{pgfscope}%
\pgfsys@transformshift{1.244158in}{0.884903in}%
\pgfsys@useobject{currentmarker}{}%
\end{pgfscope}%
\begin{pgfscope}%
\pgfsys@transformshift{1.244158in}{0.884903in}%
\pgfsys@useobject{currentmarker}{}%
\end{pgfscope}%
\begin{pgfscope}%
\pgfsys@transformshift{1.244158in}{0.884903in}%
\pgfsys@useobject{currentmarker}{}%
\end{pgfscope}%
\begin{pgfscope}%
\pgfsys@transformshift{1.244158in}{0.884903in}%
\pgfsys@useobject{currentmarker}{}%
\end{pgfscope}%
\begin{pgfscope}%
\pgfsys@transformshift{1.244158in}{0.884903in}%
\pgfsys@useobject{currentmarker}{}%
\end{pgfscope}%
\begin{pgfscope}%
\pgfsys@transformshift{1.244158in}{0.884903in}%
\pgfsys@useobject{currentmarker}{}%
\end{pgfscope}%
\begin{pgfscope}%
\pgfsys@transformshift{1.244158in}{0.884903in}%
\pgfsys@useobject{currentmarker}{}%
\end{pgfscope}%
\begin{pgfscope}%
\pgfsys@transformshift{1.244158in}{0.884903in}%
\pgfsys@useobject{currentmarker}{}%
\end{pgfscope}%
\begin{pgfscope}%
\pgfsys@transformshift{1.244158in}{0.884903in}%
\pgfsys@useobject{currentmarker}{}%
\end{pgfscope}%
\begin{pgfscope}%
\pgfsys@transformshift{1.244158in}{0.884903in}%
\pgfsys@useobject{currentmarker}{}%
\end{pgfscope}%
\begin{pgfscope}%
\pgfsys@transformshift{1.244158in}{0.884903in}%
\pgfsys@useobject{currentmarker}{}%
\end{pgfscope}%
\begin{pgfscope}%
\pgfsys@transformshift{1.244158in}{0.884903in}%
\pgfsys@useobject{currentmarker}{}%
\end{pgfscope}%
\begin{pgfscope}%
\pgfsys@transformshift{1.244158in}{0.884903in}%
\pgfsys@useobject{currentmarker}{}%
\end{pgfscope}%
\begin{pgfscope}%
\pgfsys@transformshift{1.244158in}{0.884903in}%
\pgfsys@useobject{currentmarker}{}%
\end{pgfscope}%
\begin{pgfscope}%
\pgfsys@transformshift{1.244158in}{0.884903in}%
\pgfsys@useobject{currentmarker}{}%
\end{pgfscope}%
\begin{pgfscope}%
\pgfsys@transformshift{1.244158in}{0.884903in}%
\pgfsys@useobject{currentmarker}{}%
\end{pgfscope}%
\begin{pgfscope}%
\pgfsys@transformshift{1.244158in}{0.884903in}%
\pgfsys@useobject{currentmarker}{}%
\end{pgfscope}%
\begin{pgfscope}%
\pgfsys@transformshift{1.244158in}{0.884903in}%
\pgfsys@useobject{currentmarker}{}%
\end{pgfscope}%
\begin{pgfscope}%
\pgfsys@transformshift{1.244158in}{0.884903in}%
\pgfsys@useobject{currentmarker}{}%
\end{pgfscope}%
\begin{pgfscope}%
\pgfsys@transformshift{1.244158in}{0.884903in}%
\pgfsys@useobject{currentmarker}{}%
\end{pgfscope}%
\begin{pgfscope}%
\pgfsys@transformshift{1.244158in}{0.884903in}%
\pgfsys@useobject{currentmarker}{}%
\end{pgfscope}%
\begin{pgfscope}%
\pgfsys@transformshift{1.244158in}{0.884903in}%
\pgfsys@useobject{currentmarker}{}%
\end{pgfscope}%
\begin{pgfscope}%
\pgfsys@transformshift{1.244158in}{0.884903in}%
\pgfsys@useobject{currentmarker}{}%
\end{pgfscope}%
\begin{pgfscope}%
\pgfsys@transformshift{1.244158in}{0.884903in}%
\pgfsys@useobject{currentmarker}{}%
\end{pgfscope}%
\begin{pgfscope}%
\pgfsys@transformshift{1.244158in}{0.884903in}%
\pgfsys@useobject{currentmarker}{}%
\end{pgfscope}%
\begin{pgfscope}%
\pgfsys@transformshift{1.245039in}{0.884753in}%
\pgfsys@useobject{currentmarker}{}%
\end{pgfscope}%
\begin{pgfscope}%
\pgfsys@transformshift{1.246738in}{0.885405in}%
\pgfsys@useobject{currentmarker}{}%
\end{pgfscope}%
\begin{pgfscope}%
\pgfsys@transformshift{1.248801in}{0.886470in}%
\pgfsys@useobject{currentmarker}{}%
\end{pgfscope}%
\begin{pgfscope}%
\pgfsys@transformshift{1.249227in}{0.887674in}%
\pgfsys@useobject{currentmarker}{}%
\end{pgfscope}%
\begin{pgfscope}%
\pgfsys@transformshift{1.249803in}{0.890722in}%
\pgfsys@useobject{currentmarker}{}%
\end{pgfscope}%
\begin{pgfscope}%
\pgfsys@transformshift{1.249230in}{0.895005in}%
\pgfsys@useobject{currentmarker}{}%
\end{pgfscope}%
\begin{pgfscope}%
\pgfsys@transformshift{1.248548in}{0.901400in}%
\pgfsys@useobject{currentmarker}{}%
\end{pgfscope}%
\begin{pgfscope}%
\pgfsys@transformshift{1.247337in}{0.909709in}%
\pgfsys@useobject{currentmarker}{}%
\end{pgfscope}%
\begin{pgfscope}%
\pgfsys@transformshift{1.246264in}{0.919982in}%
\pgfsys@useobject{currentmarker}{}%
\end{pgfscope}%
\begin{pgfscope}%
\pgfsys@transformshift{1.244473in}{0.931810in}%
\pgfsys@useobject{currentmarker}{}%
\end{pgfscope}%
\begin{pgfscope}%
\pgfsys@transformshift{1.245726in}{0.945440in}%
\pgfsys@useobject{currentmarker}{}%
\end{pgfscope}%
\begin{pgfscope}%
\pgfsys@transformshift{1.247132in}{0.960593in}%
\pgfsys@useobject{currentmarker}{}%
\end{pgfscope}%
\begin{pgfscope}%
\pgfsys@transformshift{1.248232in}{0.976474in}%
\pgfsys@useobject{currentmarker}{}%
\end{pgfscope}%
\begin{pgfscope}%
\pgfsys@transformshift{1.248929in}{0.992874in}%
\pgfsys@useobject{currentmarker}{}%
\end{pgfscope}%
\begin{pgfscope}%
\pgfsys@transformshift{1.249212in}{1.001898in}%
\pgfsys@useobject{currentmarker}{}%
\end{pgfscope}%
\begin{pgfscope}%
\pgfsys@transformshift{1.249679in}{1.006842in}%
\pgfsys@useobject{currentmarker}{}%
\end{pgfscope}%
\begin{pgfscope}%
\pgfsys@transformshift{1.249825in}{1.009569in}%
\pgfsys@useobject{currentmarker}{}%
\end{pgfscope}%
\begin{pgfscope}%
\pgfsys@transformshift{1.249916in}{1.011069in}%
\pgfsys@useobject{currentmarker}{}%
\end{pgfscope}%
\begin{pgfscope}%
\pgfsys@transformshift{1.249867in}{1.011893in}%
\pgfsys@useobject{currentmarker}{}%
\end{pgfscope}%
\begin{pgfscope}%
\pgfsys@transformshift{1.249878in}{1.012348in}%
\pgfsys@useobject{currentmarker}{}%
\end{pgfscope}%
\begin{pgfscope}%
\pgfsys@transformshift{1.249876in}{1.012598in}%
\pgfsys@useobject{currentmarker}{}%
\end{pgfscope}%
\begin{pgfscope}%
\pgfsys@transformshift{1.249872in}{1.012735in}%
\pgfsys@useobject{currentmarker}{}%
\end{pgfscope}%
\begin{pgfscope}%
\pgfsys@transformshift{1.249873in}{1.012811in}%
\pgfsys@useobject{currentmarker}{}%
\end{pgfscope}%
\begin{pgfscope}%
\pgfsys@transformshift{1.249871in}{1.012852in}%
\pgfsys@useobject{currentmarker}{}%
\end{pgfscope}%
\begin{pgfscope}%
\pgfsys@transformshift{1.249871in}{1.012875in}%
\pgfsys@useobject{currentmarker}{}%
\end{pgfscope}%
\begin{pgfscope}%
\pgfsys@transformshift{1.249871in}{1.012888in}%
\pgfsys@useobject{currentmarker}{}%
\end{pgfscope}%
\begin{pgfscope}%
\pgfsys@transformshift{1.249871in}{1.012894in}%
\pgfsys@useobject{currentmarker}{}%
\end{pgfscope}%
\begin{pgfscope}%
\pgfsys@transformshift{1.249871in}{1.012898in}%
\pgfsys@useobject{currentmarker}{}%
\end{pgfscope}%
\begin{pgfscope}%
\pgfsys@transformshift{1.249871in}{1.012900in}%
\pgfsys@useobject{currentmarker}{}%
\end{pgfscope}%
\begin{pgfscope}%
\pgfsys@transformshift{1.249871in}{1.012901in}%
\pgfsys@useobject{currentmarker}{}%
\end{pgfscope}%
\begin{pgfscope}%
\pgfsys@transformshift{1.249871in}{1.012902in}%
\pgfsys@useobject{currentmarker}{}%
\end{pgfscope}%
\begin{pgfscope}%
\pgfsys@transformshift{1.249871in}{1.012902in}%
\pgfsys@useobject{currentmarker}{}%
\end{pgfscope}%
\begin{pgfscope}%
\pgfsys@transformshift{1.249871in}{1.012903in}%
\pgfsys@useobject{currentmarker}{}%
\end{pgfscope}%
\begin{pgfscope}%
\pgfsys@transformshift{1.249871in}{1.012903in}%
\pgfsys@useobject{currentmarker}{}%
\end{pgfscope}%
\begin{pgfscope}%
\pgfsys@transformshift{1.249871in}{1.012903in}%
\pgfsys@useobject{currentmarker}{}%
\end{pgfscope}%
\begin{pgfscope}%
\pgfsys@transformshift{1.249871in}{1.012903in}%
\pgfsys@useobject{currentmarker}{}%
\end{pgfscope}%
\begin{pgfscope}%
\pgfsys@transformshift{1.249871in}{1.012903in}%
\pgfsys@useobject{currentmarker}{}%
\end{pgfscope}%
\begin{pgfscope}%
\pgfsys@transformshift{1.249871in}{1.012903in}%
\pgfsys@useobject{currentmarker}{}%
\end{pgfscope}%
\begin{pgfscope}%
\pgfsys@transformshift{1.249871in}{1.012903in}%
\pgfsys@useobject{currentmarker}{}%
\end{pgfscope}%
\begin{pgfscope}%
\pgfsys@transformshift{1.249871in}{1.012903in}%
\pgfsys@useobject{currentmarker}{}%
\end{pgfscope}%
\begin{pgfscope}%
\pgfsys@transformshift{1.249871in}{1.012903in}%
\pgfsys@useobject{currentmarker}{}%
\end{pgfscope}%
\begin{pgfscope}%
\pgfsys@transformshift{1.249871in}{1.012903in}%
\pgfsys@useobject{currentmarker}{}%
\end{pgfscope}%
\begin{pgfscope}%
\pgfsys@transformshift{1.249871in}{1.012903in}%
\pgfsys@useobject{currentmarker}{}%
\end{pgfscope}%
\begin{pgfscope}%
\pgfsys@transformshift{1.249871in}{1.012903in}%
\pgfsys@useobject{currentmarker}{}%
\end{pgfscope}%
\begin{pgfscope}%
\pgfsys@transformshift{1.249871in}{1.012903in}%
\pgfsys@useobject{currentmarker}{}%
\end{pgfscope}%
\begin{pgfscope}%
\pgfsys@transformshift{1.249871in}{1.012903in}%
\pgfsys@useobject{currentmarker}{}%
\end{pgfscope}%
\begin{pgfscope}%
\pgfsys@transformshift{1.249871in}{1.012903in}%
\pgfsys@useobject{currentmarker}{}%
\end{pgfscope}%
\begin{pgfscope}%
\pgfsys@transformshift{1.249871in}{1.012903in}%
\pgfsys@useobject{currentmarker}{}%
\end{pgfscope}%
\begin{pgfscope}%
\pgfsys@transformshift{1.249871in}{1.012903in}%
\pgfsys@useobject{currentmarker}{}%
\end{pgfscope}%
\begin{pgfscope}%
\pgfsys@transformshift{1.249871in}{1.012903in}%
\pgfsys@useobject{currentmarker}{}%
\end{pgfscope}%
\begin{pgfscope}%
\pgfsys@transformshift{1.249871in}{1.012903in}%
\pgfsys@useobject{currentmarker}{}%
\end{pgfscope}%
\begin{pgfscope}%
\pgfsys@transformshift{1.249871in}{1.012903in}%
\pgfsys@useobject{currentmarker}{}%
\end{pgfscope}%
\begin{pgfscope}%
\pgfsys@transformshift{1.249871in}{1.012903in}%
\pgfsys@useobject{currentmarker}{}%
\end{pgfscope}%
\begin{pgfscope}%
\pgfsys@transformshift{1.249871in}{1.012903in}%
\pgfsys@useobject{currentmarker}{}%
\end{pgfscope}%
\begin{pgfscope}%
\pgfsys@transformshift{1.249871in}{1.012903in}%
\pgfsys@useobject{currentmarker}{}%
\end{pgfscope}%
\begin{pgfscope}%
\pgfsys@transformshift{1.249871in}{1.012903in}%
\pgfsys@useobject{currentmarker}{}%
\end{pgfscope}%
\begin{pgfscope}%
\pgfsys@transformshift{1.249871in}{1.012903in}%
\pgfsys@useobject{currentmarker}{}%
\end{pgfscope}%
\begin{pgfscope}%
\pgfsys@transformshift{1.249871in}{1.012903in}%
\pgfsys@useobject{currentmarker}{}%
\end{pgfscope}%
\begin{pgfscope}%
\pgfsys@transformshift{1.249871in}{1.012903in}%
\pgfsys@useobject{currentmarker}{}%
\end{pgfscope}%
\begin{pgfscope}%
\pgfsys@transformshift{1.249871in}{1.012903in}%
\pgfsys@useobject{currentmarker}{}%
\end{pgfscope}%
\begin{pgfscope}%
\pgfsys@transformshift{1.249871in}{1.012903in}%
\pgfsys@useobject{currentmarker}{}%
\end{pgfscope}%
\begin{pgfscope}%
\pgfsys@transformshift{1.249871in}{1.012903in}%
\pgfsys@useobject{currentmarker}{}%
\end{pgfscope}%
\begin{pgfscope}%
\pgfsys@transformshift{1.249871in}{1.012903in}%
\pgfsys@useobject{currentmarker}{}%
\end{pgfscope}%
\begin{pgfscope}%
\pgfsys@transformshift{1.249871in}{1.012903in}%
\pgfsys@useobject{currentmarker}{}%
\end{pgfscope}%
\begin{pgfscope}%
\pgfsys@transformshift{1.249871in}{1.012903in}%
\pgfsys@useobject{currentmarker}{}%
\end{pgfscope}%
\begin{pgfscope}%
\pgfsys@transformshift{1.249871in}{1.012903in}%
\pgfsys@useobject{currentmarker}{}%
\end{pgfscope}%
\begin{pgfscope}%
\pgfsys@transformshift{1.249871in}{1.012903in}%
\pgfsys@useobject{currentmarker}{}%
\end{pgfscope}%
\begin{pgfscope}%
\pgfsys@transformshift{1.249871in}{1.012903in}%
\pgfsys@useobject{currentmarker}{}%
\end{pgfscope}%
\begin{pgfscope}%
\pgfsys@transformshift{1.249871in}{1.012903in}%
\pgfsys@useobject{currentmarker}{}%
\end{pgfscope}%
\begin{pgfscope}%
\pgfsys@transformshift{1.249871in}{1.012903in}%
\pgfsys@useobject{currentmarker}{}%
\end{pgfscope}%
\begin{pgfscope}%
\pgfsys@transformshift{1.249871in}{1.012903in}%
\pgfsys@useobject{currentmarker}{}%
\end{pgfscope}%
\begin{pgfscope}%
\pgfsys@transformshift{1.249871in}{1.012903in}%
\pgfsys@useobject{currentmarker}{}%
\end{pgfscope}%
\begin{pgfscope}%
\pgfsys@transformshift{1.249871in}{1.012903in}%
\pgfsys@useobject{currentmarker}{}%
\end{pgfscope}%
\begin{pgfscope}%
\pgfsys@transformshift{1.249871in}{1.012903in}%
\pgfsys@useobject{currentmarker}{}%
\end{pgfscope}%
\begin{pgfscope}%
\pgfsys@transformshift{1.249871in}{1.012903in}%
\pgfsys@useobject{currentmarker}{}%
\end{pgfscope}%
\begin{pgfscope}%
\pgfsys@transformshift{1.249871in}{1.012903in}%
\pgfsys@useobject{currentmarker}{}%
\end{pgfscope}%
\begin{pgfscope}%
\pgfsys@transformshift{1.249871in}{1.012903in}%
\pgfsys@useobject{currentmarker}{}%
\end{pgfscope}%
\begin{pgfscope}%
\pgfsys@transformshift{1.249871in}{1.012903in}%
\pgfsys@useobject{currentmarker}{}%
\end{pgfscope}%
\begin{pgfscope}%
\pgfsys@transformshift{1.249871in}{1.012903in}%
\pgfsys@useobject{currentmarker}{}%
\end{pgfscope}%
\begin{pgfscope}%
\pgfsys@transformshift{1.249871in}{1.012903in}%
\pgfsys@useobject{currentmarker}{}%
\end{pgfscope}%
\begin{pgfscope}%
\pgfsys@transformshift{1.249871in}{1.012903in}%
\pgfsys@useobject{currentmarker}{}%
\end{pgfscope}%
\begin{pgfscope}%
\pgfsys@transformshift{1.249871in}{1.012903in}%
\pgfsys@useobject{currentmarker}{}%
\end{pgfscope}%
\begin{pgfscope}%
\pgfsys@transformshift{1.249871in}{1.012903in}%
\pgfsys@useobject{currentmarker}{}%
\end{pgfscope}%
\begin{pgfscope}%
\pgfsys@transformshift{1.249871in}{1.012903in}%
\pgfsys@useobject{currentmarker}{}%
\end{pgfscope}%
\begin{pgfscope}%
\pgfsys@transformshift{1.249871in}{1.012903in}%
\pgfsys@useobject{currentmarker}{}%
\end{pgfscope}%
\begin{pgfscope}%
\pgfsys@transformshift{1.249871in}{1.012903in}%
\pgfsys@useobject{currentmarker}{}%
\end{pgfscope}%
\begin{pgfscope}%
\pgfsys@transformshift{1.249871in}{1.012903in}%
\pgfsys@useobject{currentmarker}{}%
\end{pgfscope}%
\begin{pgfscope}%
\pgfsys@transformshift{1.249871in}{1.012903in}%
\pgfsys@useobject{currentmarker}{}%
\end{pgfscope}%
\begin{pgfscope}%
\pgfsys@transformshift{1.249871in}{1.012903in}%
\pgfsys@useobject{currentmarker}{}%
\end{pgfscope}%
\begin{pgfscope}%
\pgfsys@transformshift{1.249871in}{1.012903in}%
\pgfsys@useobject{currentmarker}{}%
\end{pgfscope}%
\begin{pgfscope}%
\pgfsys@transformshift{1.249871in}{1.012903in}%
\pgfsys@useobject{currentmarker}{}%
\end{pgfscope}%
\begin{pgfscope}%
\pgfsys@transformshift{1.249871in}{1.012903in}%
\pgfsys@useobject{currentmarker}{}%
\end{pgfscope}%
\begin{pgfscope}%
\pgfsys@transformshift{1.249871in}{1.012903in}%
\pgfsys@useobject{currentmarker}{}%
\end{pgfscope}%
\begin{pgfscope}%
\pgfsys@transformshift{1.249871in}{1.012903in}%
\pgfsys@useobject{currentmarker}{}%
\end{pgfscope}%
\begin{pgfscope}%
\pgfsys@transformshift{1.249871in}{1.012903in}%
\pgfsys@useobject{currentmarker}{}%
\end{pgfscope}%
\begin{pgfscope}%
\pgfsys@transformshift{1.249871in}{1.012903in}%
\pgfsys@useobject{currentmarker}{}%
\end{pgfscope}%
\begin{pgfscope}%
\pgfsys@transformshift{1.249871in}{1.012903in}%
\pgfsys@useobject{currentmarker}{}%
\end{pgfscope}%
\begin{pgfscope}%
\pgfsys@transformshift{1.249871in}{1.012903in}%
\pgfsys@useobject{currentmarker}{}%
\end{pgfscope}%
\begin{pgfscope}%
\pgfsys@transformshift{1.249871in}{1.012903in}%
\pgfsys@useobject{currentmarker}{}%
\end{pgfscope}%
\begin{pgfscope}%
\pgfsys@transformshift{1.249871in}{1.012903in}%
\pgfsys@useobject{currentmarker}{}%
\end{pgfscope}%
\begin{pgfscope}%
\pgfsys@transformshift{1.249871in}{1.012903in}%
\pgfsys@useobject{currentmarker}{}%
\end{pgfscope}%
\begin{pgfscope}%
\pgfsys@transformshift{1.249871in}{1.012903in}%
\pgfsys@useobject{currentmarker}{}%
\end{pgfscope}%
\begin{pgfscope}%
\pgfsys@transformshift{1.249871in}{1.012903in}%
\pgfsys@useobject{currentmarker}{}%
\end{pgfscope}%
\begin{pgfscope}%
\pgfsys@transformshift{1.249871in}{1.012903in}%
\pgfsys@useobject{currentmarker}{}%
\end{pgfscope}%
\begin{pgfscope}%
\pgfsys@transformshift{1.249871in}{1.012903in}%
\pgfsys@useobject{currentmarker}{}%
\end{pgfscope}%
\begin{pgfscope}%
\pgfsys@transformshift{1.249871in}{1.012903in}%
\pgfsys@useobject{currentmarker}{}%
\end{pgfscope}%
\begin{pgfscope}%
\pgfsys@transformshift{1.249871in}{1.012903in}%
\pgfsys@useobject{currentmarker}{}%
\end{pgfscope}%
\begin{pgfscope}%
\pgfsys@transformshift{1.249871in}{1.012903in}%
\pgfsys@useobject{currentmarker}{}%
\end{pgfscope}%
\begin{pgfscope}%
\pgfsys@transformshift{1.249871in}{1.012903in}%
\pgfsys@useobject{currentmarker}{}%
\end{pgfscope}%
\begin{pgfscope}%
\pgfsys@transformshift{1.249871in}{1.012903in}%
\pgfsys@useobject{currentmarker}{}%
\end{pgfscope}%
\begin{pgfscope}%
\pgfsys@transformshift{1.249871in}{1.012903in}%
\pgfsys@useobject{currentmarker}{}%
\end{pgfscope}%
\begin{pgfscope}%
\pgfsys@transformshift{1.249871in}{1.012903in}%
\pgfsys@useobject{currentmarker}{}%
\end{pgfscope}%
\begin{pgfscope}%
\pgfsys@transformshift{1.249871in}{1.012903in}%
\pgfsys@useobject{currentmarker}{}%
\end{pgfscope}%
\begin{pgfscope}%
\pgfsys@transformshift{1.249871in}{1.012903in}%
\pgfsys@useobject{currentmarker}{}%
\end{pgfscope}%
\begin{pgfscope}%
\pgfsys@transformshift{1.249871in}{1.012903in}%
\pgfsys@useobject{currentmarker}{}%
\end{pgfscope}%
\begin{pgfscope}%
\pgfsys@transformshift{1.249871in}{1.012903in}%
\pgfsys@useobject{currentmarker}{}%
\end{pgfscope}%
\begin{pgfscope}%
\pgfsys@transformshift{1.249871in}{1.012903in}%
\pgfsys@useobject{currentmarker}{}%
\end{pgfscope}%
\begin{pgfscope}%
\pgfsys@transformshift{1.249871in}{1.012903in}%
\pgfsys@useobject{currentmarker}{}%
\end{pgfscope}%
\begin{pgfscope}%
\pgfsys@transformshift{1.249871in}{1.012903in}%
\pgfsys@useobject{currentmarker}{}%
\end{pgfscope}%
\begin{pgfscope}%
\pgfsys@transformshift{1.249871in}{1.012903in}%
\pgfsys@useobject{currentmarker}{}%
\end{pgfscope}%
\begin{pgfscope}%
\pgfsys@transformshift{1.249871in}{1.012903in}%
\pgfsys@useobject{currentmarker}{}%
\end{pgfscope}%
\begin{pgfscope}%
\pgfsys@transformshift{1.249871in}{1.012903in}%
\pgfsys@useobject{currentmarker}{}%
\end{pgfscope}%
\begin{pgfscope}%
\pgfsys@transformshift{1.249871in}{1.012903in}%
\pgfsys@useobject{currentmarker}{}%
\end{pgfscope}%
\begin{pgfscope}%
\pgfsys@transformshift{1.249871in}{1.012903in}%
\pgfsys@useobject{currentmarker}{}%
\end{pgfscope}%
\begin{pgfscope}%
\pgfsys@transformshift{1.249871in}{1.012903in}%
\pgfsys@useobject{currentmarker}{}%
\end{pgfscope}%
\begin{pgfscope}%
\pgfsys@transformshift{1.249871in}{1.012903in}%
\pgfsys@useobject{currentmarker}{}%
\end{pgfscope}%
\begin{pgfscope}%
\pgfsys@transformshift{1.249871in}{1.012903in}%
\pgfsys@useobject{currentmarker}{}%
\end{pgfscope}%
\begin{pgfscope}%
\pgfsys@transformshift{1.249871in}{1.012903in}%
\pgfsys@useobject{currentmarker}{}%
\end{pgfscope}%
\begin{pgfscope}%
\pgfsys@transformshift{1.249871in}{1.012903in}%
\pgfsys@useobject{currentmarker}{}%
\end{pgfscope}%
\begin{pgfscope}%
\pgfsys@transformshift{1.249871in}{1.012903in}%
\pgfsys@useobject{currentmarker}{}%
\end{pgfscope}%
\begin{pgfscope}%
\pgfsys@transformshift{1.249871in}{1.012903in}%
\pgfsys@useobject{currentmarker}{}%
\end{pgfscope}%
\begin{pgfscope}%
\pgfsys@transformshift{1.249871in}{1.012903in}%
\pgfsys@useobject{currentmarker}{}%
\end{pgfscope}%
\begin{pgfscope}%
\pgfsys@transformshift{1.249871in}{1.012903in}%
\pgfsys@useobject{currentmarker}{}%
\end{pgfscope}%
\begin{pgfscope}%
\pgfsys@transformshift{1.249645in}{1.013746in}%
\pgfsys@useobject{currentmarker}{}%
\end{pgfscope}%
\begin{pgfscope}%
\pgfsys@transformshift{1.250246in}{1.015899in}%
\pgfsys@useobject{currentmarker}{}%
\end{pgfscope}%
\begin{pgfscope}%
\pgfsys@transformshift{1.249659in}{1.018680in}%
\pgfsys@useobject{currentmarker}{}%
\end{pgfscope}%
\begin{pgfscope}%
\pgfsys@transformshift{1.250076in}{1.020187in}%
\pgfsys@useobject{currentmarker}{}%
\end{pgfscope}%
\begin{pgfscope}%
\pgfsys@transformshift{1.249449in}{1.022512in}%
\pgfsys@useobject{currentmarker}{}%
\end{pgfscope}%
\begin{pgfscope}%
\pgfsys@transformshift{1.250118in}{1.025680in}%
\pgfsys@useobject{currentmarker}{}%
\end{pgfscope}%
\begin{pgfscope}%
\pgfsys@transformshift{1.249254in}{1.029425in}%
\pgfsys@useobject{currentmarker}{}%
\end{pgfscope}%
\begin{pgfscope}%
\pgfsys@transformshift{1.249781in}{1.031472in}%
\pgfsys@useobject{currentmarker}{}%
\end{pgfscope}%
\begin{pgfscope}%
\pgfsys@transformshift{1.248786in}{1.035146in}%
\pgfsys@useobject{currentmarker}{}%
\end{pgfscope}%
\begin{pgfscope}%
\pgfsys@transformshift{1.249097in}{1.039428in}%
\pgfsys@useobject{currentmarker}{}%
\end{pgfscope}%
\begin{pgfscope}%
\pgfsys@transformshift{1.248737in}{1.041761in}%
\pgfsys@useobject{currentmarker}{}%
\end{pgfscope}%
\begin{pgfscope}%
\pgfsys@transformshift{1.248289in}{1.042981in}%
\pgfsys@useobject{currentmarker}{}%
\end{pgfscope}%
\begin{pgfscope}%
\pgfsys@transformshift{1.248585in}{1.045637in}%
\pgfsys@useobject{currentmarker}{}%
\end{pgfscope}%
\begin{pgfscope}%
\pgfsys@transformshift{1.247268in}{1.049565in}%
\pgfsys@useobject{currentmarker}{}%
\end{pgfscope}%
\begin{pgfscope}%
\pgfsys@transformshift{1.247292in}{1.054724in}%
\pgfsys@useobject{currentmarker}{}%
\end{pgfscope}%
\begin{pgfscope}%
\pgfsys@transformshift{1.247007in}{1.057546in}%
\pgfsys@useobject{currentmarker}{}%
\end{pgfscope}%
\begin{pgfscope}%
\pgfsys@transformshift{1.245919in}{1.060892in}%
\pgfsys@useobject{currentmarker}{}%
\end{pgfscope}%
\begin{pgfscope}%
\pgfsys@transformshift{1.246941in}{1.065859in}%
\pgfsys@useobject{currentmarker}{}%
\end{pgfscope}%
\begin{pgfscope}%
\pgfsys@transformshift{1.246257in}{1.068563in}%
\pgfsys@useobject{currentmarker}{}%
\end{pgfscope}%
\begin{pgfscope}%
\pgfsys@transformshift{1.246438in}{1.072001in}%
\pgfsys@useobject{currentmarker}{}%
\end{pgfscope}%
\begin{pgfscope}%
\pgfsys@transformshift{1.247092in}{1.075901in}%
\pgfsys@useobject{currentmarker}{}%
\end{pgfscope}%
\begin{pgfscope}%
\pgfsys@transformshift{1.247256in}{1.080404in}%
\pgfsys@useobject{currentmarker}{}%
\end{pgfscope}%
\begin{pgfscope}%
\pgfsys@transformshift{1.247367in}{1.082881in}%
\pgfsys@useobject{currentmarker}{}%
\end{pgfscope}%
\begin{pgfscope}%
\pgfsys@transformshift{1.247513in}{1.084236in}%
\pgfsys@useobject{currentmarker}{}%
\end{pgfscope}%
\begin{pgfscope}%
\pgfsys@transformshift{1.247289in}{1.087432in}%
\pgfsys@useobject{currentmarker}{}%
\end{pgfscope}%
\begin{pgfscope}%
\pgfsys@transformshift{1.247645in}{1.091104in}%
\pgfsys@useobject{currentmarker}{}%
\end{pgfscope}%
\begin{pgfscope}%
\pgfsys@transformshift{1.248353in}{1.095486in}%
\pgfsys@useobject{currentmarker}{}%
\end{pgfscope}%
\begin{pgfscope}%
\pgfsys@transformshift{1.248008in}{1.101414in}%
\pgfsys@useobject{currentmarker}{}%
\end{pgfscope}%
\begin{pgfscope}%
\pgfsys@transformshift{1.248168in}{1.104676in}%
\pgfsys@useobject{currentmarker}{}%
\end{pgfscope}%
\begin{pgfscope}%
\pgfsys@transformshift{1.248810in}{1.108381in}%
\pgfsys@useobject{currentmarker}{}%
\end{pgfscope}%
\begin{pgfscope}%
\pgfsys@transformshift{1.248284in}{1.113545in}%
\pgfsys@useobject{currentmarker}{}%
\end{pgfscope}%
\begin{pgfscope}%
\pgfsys@transformshift{1.248372in}{1.116400in}%
\pgfsys@useobject{currentmarker}{}%
\end{pgfscope}%
\begin{pgfscope}%
\pgfsys@transformshift{1.248633in}{1.117948in}%
\pgfsys@useobject{currentmarker}{}%
\end{pgfscope}%
\begin{pgfscope}%
\pgfsys@transformshift{1.248371in}{1.120810in}%
\pgfsys@useobject{currentmarker}{}%
\end{pgfscope}%
\begin{pgfscope}%
\pgfsys@transformshift{1.248242in}{1.122386in}%
\pgfsys@useobject{currentmarker}{}%
\end{pgfscope}%
\begin{pgfscope}%
\pgfsys@transformshift{1.249184in}{1.126152in}%
\pgfsys@useobject{currentmarker}{}%
\end{pgfscope}%
\begin{pgfscope}%
\pgfsys@transformshift{1.248934in}{1.128272in}%
\pgfsys@useobject{currentmarker}{}%
\end{pgfscope}%
\begin{pgfscope}%
\pgfsys@transformshift{1.248804in}{1.129439in}%
\pgfsys@useobject{currentmarker}{}%
\end{pgfscope}%
\begin{pgfscope}%
\pgfsys@transformshift{1.249524in}{1.132500in}%
\pgfsys@useobject{currentmarker}{}%
\end{pgfscope}%
\begin{pgfscope}%
\pgfsys@transformshift{1.249345in}{1.134220in}%
\pgfsys@useobject{currentmarker}{}%
\end{pgfscope}%
\begin{pgfscope}%
\pgfsys@transformshift{1.249338in}{1.135172in}%
\pgfsys@useobject{currentmarker}{}%
\end{pgfscope}%
\begin{pgfscope}%
\pgfsys@transformshift{1.249675in}{1.137092in}%
\pgfsys@useobject{currentmarker}{}%
\end{pgfscope}%
\begin{pgfscope}%
\pgfsys@transformshift{1.249607in}{1.139583in}%
\pgfsys@useobject{currentmarker}{}%
\end{pgfscope}%
\begin{pgfscope}%
\pgfsys@transformshift{1.249443in}{1.140944in}%
\pgfsys@useobject{currentmarker}{}%
\end{pgfscope}%
\begin{pgfscope}%
\pgfsys@transformshift{1.250084in}{1.143134in}%
\pgfsys@useobject{currentmarker}{}%
\end{pgfscope}%
\begin{pgfscope}%
\pgfsys@transformshift{1.249384in}{1.146889in}%
\pgfsys@useobject{currentmarker}{}%
\end{pgfscope}%
\begin{pgfscope}%
\pgfsys@transformshift{1.249489in}{1.148988in}%
\pgfsys@useobject{currentmarker}{}%
\end{pgfscope}%
\begin{pgfscope}%
\pgfsys@transformshift{1.249416in}{1.154235in}%
\pgfsys@useobject{currentmarker}{}%
\end{pgfscope}%
\begin{pgfscope}%
\pgfsys@transformshift{1.246870in}{1.161830in}%
\pgfsys@useobject{currentmarker}{}%
\end{pgfscope}%
\begin{pgfscope}%
\pgfsys@transformshift{1.247274in}{1.166218in}%
\pgfsys@useobject{currentmarker}{}%
\end{pgfscope}%
\begin{pgfscope}%
\pgfsys@transformshift{1.247537in}{1.171323in}%
\pgfsys@useobject{currentmarker}{}%
\end{pgfscope}%
\begin{pgfscope}%
\pgfsys@transformshift{1.246722in}{1.174013in}%
\pgfsys@useobject{currentmarker}{}%
\end{pgfscope}%
\begin{pgfscope}%
\pgfsys@transformshift{1.247584in}{1.178624in}%
\pgfsys@useobject{currentmarker}{}%
\end{pgfscope}%
\begin{pgfscope}%
\pgfsys@transformshift{1.246734in}{1.183991in}%
\pgfsys@useobject{currentmarker}{}%
\end{pgfscope}%
\begin{pgfscope}%
\pgfsys@transformshift{1.246480in}{1.186968in}%
\pgfsys@useobject{currentmarker}{}%
\end{pgfscope}%
\begin{pgfscope}%
\pgfsys@transformshift{1.246764in}{1.188587in}%
\pgfsys@useobject{currentmarker}{}%
\end{pgfscope}%
\begin{pgfscope}%
\pgfsys@transformshift{1.245786in}{1.191610in}%
\pgfsys@useobject{currentmarker}{}%
\end{pgfscope}%
\begin{pgfscope}%
\pgfsys@transformshift{1.245887in}{1.193355in}%
\pgfsys@useobject{currentmarker}{}%
\end{pgfscope}%
\begin{pgfscope}%
\pgfsys@transformshift{1.245504in}{1.196800in}%
\pgfsys@useobject{currentmarker}{}%
\end{pgfscope}%
\begin{pgfscope}%
\pgfsys@transformshift{1.244319in}{1.202993in}%
\pgfsys@useobject{currentmarker}{}%
\end{pgfscope}%
\begin{pgfscope}%
\pgfsys@transformshift{1.245258in}{1.209802in}%
\pgfsys@useobject{currentmarker}{}%
\end{pgfscope}%
\begin{pgfscope}%
\pgfsys@transformshift{1.249036in}{1.216434in}%
\pgfsys@useobject{currentmarker}{}%
\end{pgfscope}%
\begin{pgfscope}%
\pgfsys@transformshift{1.246576in}{1.225992in}%
\pgfsys@useobject{currentmarker}{}%
\end{pgfscope}%
\begin{pgfscope}%
\pgfsys@transformshift{1.245785in}{1.231362in}%
\pgfsys@useobject{currentmarker}{}%
\end{pgfscope}%
\begin{pgfscope}%
\pgfsys@transformshift{1.246798in}{1.234171in}%
\pgfsys@useobject{currentmarker}{}%
\end{pgfscope}%
\begin{pgfscope}%
\pgfsys@transformshift{1.246564in}{1.235796in}%
\pgfsys@useobject{currentmarker}{}%
\end{pgfscope}%
\begin{pgfscope}%
\pgfsys@transformshift{1.247483in}{1.239459in}%
\pgfsys@useobject{currentmarker}{}%
\end{pgfscope}%
\begin{pgfscope}%
\pgfsys@transformshift{1.247127in}{1.241505in}%
\pgfsys@useobject{currentmarker}{}%
\end{pgfscope}%
\begin{pgfscope}%
\pgfsys@transformshift{1.246979in}{1.242638in}%
\pgfsys@useobject{currentmarker}{}%
\end{pgfscope}%
\begin{pgfscope}%
\pgfsys@transformshift{1.247284in}{1.245487in}%
\pgfsys@useobject{currentmarker}{}%
\end{pgfscope}%
\begin{pgfscope}%
\pgfsys@transformshift{1.246219in}{1.248968in}%
\pgfsys@useobject{currentmarker}{}%
\end{pgfscope}%
\begin{pgfscope}%
\pgfsys@transformshift{1.246394in}{1.250962in}%
\pgfsys@useobject{currentmarker}{}%
\end{pgfscope}%
\begin{pgfscope}%
\pgfsys@transformshift{1.246592in}{1.252045in}%
\pgfsys@useobject{currentmarker}{}%
\end{pgfscope}%
\begin{pgfscope}%
\pgfsys@transformshift{1.246381in}{1.253669in}%
\pgfsys@useobject{currentmarker}{}%
\end{pgfscope}%
\begin{pgfscope}%
\pgfsys@transformshift{1.246170in}{1.255870in}%
\pgfsys@useobject{currentmarker}{}%
\end{pgfscope}%
\begin{pgfscope}%
\pgfsys@transformshift{1.246465in}{1.260102in}%
\pgfsys@useobject{currentmarker}{}%
\end{pgfscope}%
\begin{pgfscope}%
\pgfsys@transformshift{1.244667in}{1.265908in}%
\pgfsys@useobject{currentmarker}{}%
\end{pgfscope}%
\begin{pgfscope}%
\pgfsys@transformshift{1.244807in}{1.272360in}%
\pgfsys@useobject{currentmarker}{}%
\end{pgfscope}%
\begin{pgfscope}%
\pgfsys@transformshift{1.244236in}{1.282175in}%
\pgfsys@useobject{currentmarker}{}%
\end{pgfscope}%
\begin{pgfscope}%
\pgfsys@transformshift{1.246645in}{1.292927in}%
\pgfsys@useobject{currentmarker}{}%
\end{pgfscope}%
\begin{pgfscope}%
\pgfsys@transformshift{1.250360in}{1.304155in}%
\pgfsys@useobject{currentmarker}{}%
\end{pgfscope}%
\begin{pgfscope}%
\pgfsys@transformshift{1.244800in}{1.317140in}%
\pgfsys@useobject{currentmarker}{}%
\end{pgfscope}%
\begin{pgfscope}%
\pgfsys@transformshift{1.252342in}{1.329683in}%
\pgfsys@useobject{currentmarker}{}%
\end{pgfscope}%
\begin{pgfscope}%
\pgfsys@transformshift{1.252271in}{1.345990in}%
\pgfsys@useobject{currentmarker}{}%
\end{pgfscope}%
\begin{pgfscope}%
\pgfsys@transformshift{1.254560in}{1.363904in}%
\pgfsys@useobject{currentmarker}{}%
\end{pgfscope}%
\begin{pgfscope}%
\pgfsys@transformshift{1.256548in}{1.382396in}%
\pgfsys@useobject{currentmarker}{}%
\end{pgfscope}%
\begin{pgfscope}%
\pgfsys@transformshift{1.254125in}{1.401460in}%
\pgfsys@useobject{currentmarker}{}%
\end{pgfscope}%
\begin{pgfscope}%
\pgfsys@transformshift{1.256249in}{1.411813in}%
\pgfsys@useobject{currentmarker}{}%
\end{pgfscope}%
\begin{pgfscope}%
\pgfsys@transformshift{1.255861in}{1.417614in}%
\pgfsys@useobject{currentmarker}{}%
\end{pgfscope}%
\begin{pgfscope}%
\pgfsys@transformshift{1.256280in}{1.420783in}%
\pgfsys@useobject{currentmarker}{}%
\end{pgfscope}%
\begin{pgfscope}%
\pgfsys@transformshift{1.256138in}{1.422536in}%
\pgfsys@useobject{currentmarker}{}%
\end{pgfscope}%
\begin{pgfscope}%
\pgfsys@transformshift{1.256284in}{1.423492in}%
\pgfsys@useobject{currentmarker}{}%
\end{pgfscope}%
\begin{pgfscope}%
\pgfsys@transformshift{1.256250in}{1.424023in}%
\pgfsys@useobject{currentmarker}{}%
\end{pgfscope}%
\begin{pgfscope}%
\pgfsys@transformshift{1.256492in}{1.425073in}%
\pgfsys@useobject{currentmarker}{}%
\end{pgfscope}%
\begin{pgfscope}%
\pgfsys@transformshift{1.256568in}{1.426973in}%
\pgfsys@useobject{currentmarker}{}%
\end{pgfscope}%
\begin{pgfscope}%
\pgfsys@transformshift{1.256861in}{1.427978in}%
\pgfsys@useobject{currentmarker}{}%
\end{pgfscope}%
\begin{pgfscope}%
\pgfsys@transformshift{1.257066in}{1.429691in}%
\pgfsys@useobject{currentmarker}{}%
\end{pgfscope}%
\begin{pgfscope}%
\pgfsys@transformshift{1.257716in}{1.430383in}%
\pgfsys@useobject{currentmarker}{}%
\end{pgfscope}%
\begin{pgfscope}%
\pgfsys@transformshift{1.258552in}{1.431511in}%
\pgfsys@useobject{currentmarker}{}%
\end{pgfscope}%
\begin{pgfscope}%
\pgfsys@transformshift{1.261098in}{1.431996in}%
\pgfsys@useobject{currentmarker}{}%
\end{pgfscope}%
\begin{pgfscope}%
\pgfsys@transformshift{1.262478in}{1.432354in}%
\pgfsys@useobject{currentmarker}{}%
\end{pgfscope}%
\begin{pgfscope}%
\pgfsys@transformshift{1.263260in}{1.432288in}%
\pgfsys@useobject{currentmarker}{}%
\end{pgfscope}%
\begin{pgfscope}%
\pgfsys@transformshift{1.263691in}{1.432301in}%
\pgfsys@useobject{currentmarker}{}%
\end{pgfscope}%
\begin{pgfscope}%
\pgfsys@transformshift{1.264688in}{1.432294in}%
\pgfsys@useobject{currentmarker}{}%
\end{pgfscope}%
\begin{pgfscope}%
\pgfsys@transformshift{1.265237in}{1.432307in}%
\pgfsys@useobject{currentmarker}{}%
\end{pgfscope}%
\begin{pgfscope}%
\pgfsys@transformshift{1.265537in}{1.432336in}%
\pgfsys@useobject{currentmarker}{}%
\end{pgfscope}%
\begin{pgfscope}%
\pgfsys@transformshift{1.266432in}{1.432352in}%
\pgfsys@useobject{currentmarker}{}%
\end{pgfscope}%
\begin{pgfscope}%
\pgfsys@transformshift{1.266924in}{1.432358in}%
\pgfsys@useobject{currentmarker}{}%
\end{pgfscope}%
\begin{pgfscope}%
\pgfsys@transformshift{1.267870in}{1.432527in}%
\pgfsys@useobject{currentmarker}{}%
\end{pgfscope}%
\begin{pgfscope}%
\pgfsys@transformshift{1.268397in}{1.432529in}%
\pgfsys@useobject{currentmarker}{}%
\end{pgfscope}%
\begin{pgfscope}%
\pgfsys@transformshift{1.270207in}{1.432851in}%
\pgfsys@useobject{currentmarker}{}%
\end{pgfscope}%
\begin{pgfscope}%
\pgfsys@transformshift{1.274837in}{1.433984in}%
\pgfsys@useobject{currentmarker}{}%
\end{pgfscope}%
\begin{pgfscope}%
\pgfsys@transformshift{1.277453in}{1.434167in}%
\pgfsys@useobject{currentmarker}{}%
\end{pgfscope}%
\begin{pgfscope}%
\pgfsys@transformshift{1.281996in}{1.435140in}%
\pgfsys@useobject{currentmarker}{}%
\end{pgfscope}%
\begin{pgfscope}%
\pgfsys@transformshift{1.284490in}{1.435699in}%
\pgfsys@useobject{currentmarker}{}%
\end{pgfscope}%
\begin{pgfscope}%
\pgfsys@transformshift{1.288538in}{1.436053in}%
\pgfsys@useobject{currentmarker}{}%
\end{pgfscope}%
\begin{pgfscope}%
\pgfsys@transformshift{1.293721in}{1.436449in}%
\pgfsys@useobject{currentmarker}{}%
\end{pgfscope}%
\begin{pgfscope}%
\pgfsys@transformshift{1.299699in}{1.436336in}%
\pgfsys@useobject{currentmarker}{}%
\end{pgfscope}%
\begin{pgfscope}%
\pgfsys@transformshift{1.308984in}{1.438265in}%
\pgfsys@useobject{currentmarker}{}%
\end{pgfscope}%
\begin{pgfscope}%
\pgfsys@transformshift{1.318975in}{1.439859in}%
\pgfsys@useobject{currentmarker}{}%
\end{pgfscope}%
\begin{pgfscope}%
\pgfsys@transformshift{1.331663in}{1.440828in}%
\pgfsys@useobject{currentmarker}{}%
\end{pgfscope}%
\begin{pgfscope}%
\pgfsys@transformshift{1.344857in}{1.441507in}%
\pgfsys@useobject{currentmarker}{}%
\end{pgfscope}%
\begin{pgfscope}%
\pgfsys@transformshift{1.359935in}{1.443429in}%
\pgfsys@useobject{currentmarker}{}%
\end{pgfscope}%
\begin{pgfscope}%
\pgfsys@transformshift{1.375712in}{1.442928in}%
\pgfsys@useobject{currentmarker}{}%
\end{pgfscope}%
\begin{pgfscope}%
\pgfsys@transformshift{1.392108in}{1.445873in}%
\pgfsys@useobject{currentmarker}{}%
\end{pgfscope}%
\begin{pgfscope}%
\pgfsys@transformshift{1.409449in}{1.447493in}%
\pgfsys@useobject{currentmarker}{}%
\end{pgfscope}%
\begin{pgfscope}%
\pgfsys@transformshift{1.428661in}{1.447489in}%
\pgfsys@useobject{currentmarker}{}%
\end{pgfscope}%
\begin{pgfscope}%
\pgfsys@transformshift{1.439191in}{1.448371in}%
\pgfsys@useobject{currentmarker}{}%
\end{pgfscope}%
\begin{pgfscope}%
\pgfsys@transformshift{1.450249in}{1.448208in}%
\pgfsys@useobject{currentmarker}{}%
\end{pgfscope}%
\begin{pgfscope}%
\pgfsys@transformshift{1.456332in}{1.448279in}%
\pgfsys@useobject{currentmarker}{}%
\end{pgfscope}%
\begin{pgfscope}%
\pgfsys@transformshift{1.465012in}{1.447865in}%
\pgfsys@useobject{currentmarker}{}%
\end{pgfscope}%
\begin{pgfscope}%
\pgfsys@transformshift{1.474286in}{1.447168in}%
\pgfsys@useobject{currentmarker}{}%
\end{pgfscope}%
\begin{pgfscope}%
\pgfsys@transformshift{1.486009in}{1.446200in}%
\pgfsys@useobject{currentmarker}{}%
\end{pgfscope}%
\begin{pgfscope}%
\pgfsys@transformshift{1.492467in}{1.446581in}%
\pgfsys@useobject{currentmarker}{}%
\end{pgfscope}%
\begin{pgfscope}%
\pgfsys@transformshift{1.499809in}{1.446422in}%
\pgfsys@useobject{currentmarker}{}%
\end{pgfscope}%
\begin{pgfscope}%
\pgfsys@transformshift{1.503846in}{1.446506in}%
\pgfsys@useobject{currentmarker}{}%
\end{pgfscope}%
\begin{pgfscope}%
\pgfsys@transformshift{1.508458in}{1.445957in}%
\pgfsys@useobject{currentmarker}{}%
\end{pgfscope}%
\begin{pgfscope}%
\pgfsys@transformshift{1.514456in}{1.446156in}%
\pgfsys@useobject{currentmarker}{}%
\end{pgfscope}%
\begin{pgfscope}%
\pgfsys@transformshift{1.521763in}{1.445686in}%
\pgfsys@useobject{currentmarker}{}%
\end{pgfscope}%
\begin{pgfscope}%
\pgfsys@transformshift{1.530587in}{1.446294in}%
\pgfsys@useobject{currentmarker}{}%
\end{pgfscope}%
\begin{pgfscope}%
\pgfsys@transformshift{1.542175in}{1.446550in}%
\pgfsys@useobject{currentmarker}{}%
\end{pgfscope}%
\begin{pgfscope}%
\pgfsys@transformshift{1.548549in}{1.446459in}%
\pgfsys@useobject{currentmarker}{}%
\end{pgfscope}%
\begin{pgfscope}%
\pgfsys@transformshift{1.555484in}{1.446142in}%
\pgfsys@useobject{currentmarker}{}%
\end{pgfscope}%
\begin{pgfscope}%
\pgfsys@transformshift{1.565978in}{1.446290in}%
\pgfsys@useobject{currentmarker}{}%
\end{pgfscope}%
\begin{pgfscope}%
\pgfsys@transformshift{1.577102in}{1.446554in}%
\pgfsys@useobject{currentmarker}{}%
\end{pgfscope}%
\begin{pgfscope}%
\pgfsys@transformshift{1.589333in}{1.445925in}%
\pgfsys@useobject{currentmarker}{}%
\end{pgfscope}%
\begin{pgfscope}%
\pgfsys@transformshift{1.602662in}{1.446241in}%
\pgfsys@useobject{currentmarker}{}%
\end{pgfscope}%
\begin{pgfscope}%
\pgfsys@transformshift{1.609966in}{1.446899in}%
\pgfsys@useobject{currentmarker}{}%
\end{pgfscope}%
\begin{pgfscope}%
\pgfsys@transformshift{1.618332in}{1.447023in}%
\pgfsys@useobject{currentmarker}{}%
\end{pgfscope}%
\begin{pgfscope}%
\pgfsys@transformshift{1.622919in}{1.447400in}%
\pgfsys@useobject{currentmarker}{}%
\end{pgfscope}%
\begin{pgfscope}%
\pgfsys@transformshift{1.628955in}{1.446914in}%
\pgfsys@useobject{currentmarker}{}%
\end{pgfscope}%
\begin{pgfscope}%
\pgfsys@transformshift{1.635885in}{1.447736in}%
\pgfsys@useobject{currentmarker}{}%
\end{pgfscope}%
\begin{pgfscope}%
\pgfsys@transformshift{1.644486in}{1.447164in}%
\pgfsys@useobject{currentmarker}{}%
\end{pgfscope}%
\begin{pgfscope}%
\pgfsys@transformshift{1.649201in}{1.447651in}%
\pgfsys@useobject{currentmarker}{}%
\end{pgfscope}%
\begin{pgfscope}%
\pgfsys@transformshift{1.654481in}{1.447270in}%
\pgfsys@useobject{currentmarker}{}%
\end{pgfscope}%
\begin{pgfscope}%
\pgfsys@transformshift{1.660568in}{1.448016in}%
\pgfsys@useobject{currentmarker}{}%
\end{pgfscope}%
\begin{pgfscope}%
\pgfsys@transformshift{1.668480in}{1.447885in}%
\pgfsys@useobject{currentmarker}{}%
\end{pgfscope}%
\begin{pgfscope}%
\pgfsys@transformshift{1.672815in}{1.448277in}%
\pgfsys@useobject{currentmarker}{}%
\end{pgfscope}%
\begin{pgfscope}%
\pgfsys@transformshift{1.675202in}{1.448448in}%
\pgfsys@useobject{currentmarker}{}%
\end{pgfscope}%
\begin{pgfscope}%
\pgfsys@transformshift{1.678407in}{1.448507in}%
\pgfsys@useobject{currentmarker}{}%
\end{pgfscope}%
\begin{pgfscope}%
\pgfsys@transformshift{1.680170in}{1.448486in}%
\pgfsys@useobject{currentmarker}{}%
\end{pgfscope}%
\begin{pgfscope}%
\pgfsys@transformshift{1.683312in}{1.448822in}%
\pgfsys@useobject{currentmarker}{}%
\end{pgfscope}%
\begin{pgfscope}%
\pgfsys@transformshift{1.687606in}{1.448709in}%
\pgfsys@useobject{currentmarker}{}%
\end{pgfscope}%
\begin{pgfscope}%
\pgfsys@transformshift{1.689957in}{1.448933in}%
\pgfsys@useobject{currentmarker}{}%
\end{pgfscope}%
\begin{pgfscope}%
\pgfsys@transformshift{1.694023in}{1.448694in}%
\pgfsys@useobject{currentmarker}{}%
\end{pgfscope}%
\begin{pgfscope}%
\pgfsys@transformshift{1.696253in}{1.448914in}%
\pgfsys@useobject{currentmarker}{}%
\end{pgfscope}%
\begin{pgfscope}%
\pgfsys@transformshift{1.697483in}{1.448837in}%
\pgfsys@useobject{currentmarker}{}%
\end{pgfscope}%
\begin{pgfscope}%
\pgfsys@transformshift{1.700077in}{1.449357in}%
\pgfsys@useobject{currentmarker}{}%
\end{pgfscope}%
\begin{pgfscope}%
\pgfsys@transformshift{1.703267in}{1.449212in}%
\pgfsys@useobject{currentmarker}{}%
\end{pgfscope}%
\begin{pgfscope}%
\pgfsys@transformshift{1.707354in}{1.449632in}%
\pgfsys@useobject{currentmarker}{}%
\end{pgfscope}%
\begin{pgfscope}%
\pgfsys@transformshift{1.709611in}{1.449532in}%
\pgfsys@useobject{currentmarker}{}%
\end{pgfscope}%
\begin{pgfscope}%
\pgfsys@transformshift{1.714115in}{1.450128in}%
\pgfsys@useobject{currentmarker}{}%
\end{pgfscope}%
\begin{pgfscope}%
\pgfsys@transformshift{1.716606in}{1.449938in}%
\pgfsys@useobject{currentmarker}{}%
\end{pgfscope}%
\begin{pgfscope}%
\pgfsys@transformshift{1.720608in}{1.450359in}%
\pgfsys@useobject{currentmarker}{}%
\end{pgfscope}%
\begin{pgfscope}%
\pgfsys@transformshift{1.722809in}{1.450130in}%
\pgfsys@useobject{currentmarker}{}%
\end{pgfscope}%
\begin{pgfscope}%
\pgfsys@transformshift{1.725783in}{1.450291in}%
\pgfsys@useobject{currentmarker}{}%
\end{pgfscope}%
\begin{pgfscope}%
\pgfsys@transformshift{1.727417in}{1.450167in}%
\pgfsys@useobject{currentmarker}{}%
\end{pgfscope}%
\begin{pgfscope}%
\pgfsys@transformshift{1.728317in}{1.450165in}%
\pgfsys@useobject{currentmarker}{}%
\end{pgfscope}%
\begin{pgfscope}%
\pgfsys@transformshift{1.728771in}{1.449966in}%
\pgfsys@useobject{currentmarker}{}%
\end{pgfscope}%
\begin{pgfscope}%
\pgfsys@transformshift{1.729692in}{1.449304in}%
\pgfsys@useobject{currentmarker}{}%
\end{pgfscope}%
\begin{pgfscope}%
\pgfsys@transformshift{1.730683in}{1.447593in}%
\pgfsys@useobject{currentmarker}{}%
\end{pgfscope}%
\begin{pgfscope}%
\pgfsys@transformshift{1.731578in}{1.444522in}%
\pgfsys@useobject{currentmarker}{}%
\end{pgfscope}%
\begin{pgfscope}%
\pgfsys@transformshift{1.731960in}{1.439794in}%
\pgfsys@useobject{currentmarker}{}%
\end{pgfscope}%
\begin{pgfscope}%
\pgfsys@transformshift{1.732980in}{1.433837in}%
\pgfsys@useobject{currentmarker}{}%
\end{pgfscope}%
\begin{pgfscope}%
\pgfsys@transformshift{1.732247in}{1.427014in}%
\pgfsys@useobject{currentmarker}{}%
\end{pgfscope}%
\begin{pgfscope}%
\pgfsys@transformshift{1.732681in}{1.419362in}%
\pgfsys@useobject{currentmarker}{}%
\end{pgfscope}%
\begin{pgfscope}%
\pgfsys@transformshift{1.731305in}{1.411151in}%
\pgfsys@useobject{currentmarker}{}%
\end{pgfscope}%
\begin{pgfscope}%
\pgfsys@transformshift{1.730994in}{1.402373in}%
\pgfsys@useobject{currentmarker}{}%
\end{pgfscope}%
\begin{pgfscope}%
\pgfsys@transformshift{1.729658in}{1.393230in}%
\pgfsys@useobject{currentmarker}{}%
\end{pgfscope}%
\begin{pgfscope}%
\pgfsys@transformshift{1.727680in}{1.383654in}%
\pgfsys@useobject{currentmarker}{}%
\end{pgfscope}%
\begin{pgfscope}%
\pgfsys@transformshift{1.727291in}{1.372631in}%
\pgfsys@useobject{currentmarker}{}%
\end{pgfscope}%
\begin{pgfscope}%
\pgfsys@transformshift{1.727161in}{1.360972in}%
\pgfsys@useobject{currentmarker}{}%
\end{pgfscope}%
\begin{pgfscope}%
\pgfsys@transformshift{1.723314in}{1.347831in}%
\pgfsys@useobject{currentmarker}{}%
\end{pgfscope}%
\begin{pgfscope}%
\pgfsys@transformshift{1.723586in}{1.340304in}%
\pgfsys@useobject{currentmarker}{}%
\end{pgfscope}%
\begin{pgfscope}%
\pgfsys@transformshift{1.723926in}{1.336176in}%
\pgfsys@useobject{currentmarker}{}%
\end{pgfscope}%
\begin{pgfscope}%
\pgfsys@transformshift{1.722893in}{1.329743in}%
\pgfsys@useobject{currentmarker}{}%
\end{pgfscope}%
\begin{pgfscope}%
\pgfsys@transformshift{1.722215in}{1.326224in}%
\pgfsys@useobject{currentmarker}{}%
\end{pgfscope}%
\begin{pgfscope}%
\pgfsys@transformshift{1.722732in}{1.321286in}%
\pgfsys@useobject{currentmarker}{}%
\end{pgfscope}%
\begin{pgfscope}%
\pgfsys@transformshift{1.722398in}{1.318576in}%
\pgfsys@useobject{currentmarker}{}%
\end{pgfscope}%
\begin{pgfscope}%
\pgfsys@transformshift{1.722120in}{1.317100in}%
\pgfsys@useobject{currentmarker}{}%
\end{pgfscope}%
\begin{pgfscope}%
\pgfsys@transformshift{1.722266in}{1.314247in}%
\pgfsys@useobject{currentmarker}{}%
\end{pgfscope}%
\begin{pgfscope}%
\pgfsys@transformshift{1.722164in}{1.312680in}%
\pgfsys@useobject{currentmarker}{}%
\end{pgfscope}%
\begin{pgfscope}%
\pgfsys@transformshift{1.722038in}{1.311825in}%
\pgfsys@useobject{currentmarker}{}%
\end{pgfscope}%
\begin{pgfscope}%
\pgfsys@transformshift{1.722043in}{1.311350in}%
\pgfsys@useobject{currentmarker}{}%
\end{pgfscope}%
\begin{pgfscope}%
\pgfsys@transformshift{1.722086in}{1.311092in}%
\pgfsys@useobject{currentmarker}{}%
\end{pgfscope}%
\begin{pgfscope}%
\pgfsys@transformshift{1.721937in}{1.309773in}%
\pgfsys@useobject{currentmarker}{}%
\end{pgfscope}%
\begin{pgfscope}%
\pgfsys@transformshift{1.721744in}{1.307681in}%
\pgfsys@useobject{currentmarker}{}%
\end{pgfscope}%
\begin{pgfscope}%
\pgfsys@transformshift{1.721979in}{1.304215in}%
\pgfsys@useobject{currentmarker}{}%
\end{pgfscope}%
\begin{pgfscope}%
\pgfsys@transformshift{1.721984in}{1.302304in}%
\pgfsys@useobject{currentmarker}{}%
\end{pgfscope}%
\begin{pgfscope}%
\pgfsys@transformshift{1.721855in}{1.301261in}%
\pgfsys@useobject{currentmarker}{}%
\end{pgfscope}%
\begin{pgfscope}%
\pgfsys@transformshift{1.721847in}{1.300684in}%
\pgfsys@useobject{currentmarker}{}%
\end{pgfscope}%
\begin{pgfscope}%
\pgfsys@transformshift{1.721886in}{1.300368in}%
\pgfsys@useobject{currentmarker}{}%
\end{pgfscope}%
\begin{pgfscope}%
\pgfsys@transformshift{1.721754in}{1.299020in}%
\pgfsys@useobject{currentmarker}{}%
\end{pgfscope}%
\begin{pgfscope}%
\pgfsys@transformshift{1.721673in}{1.297119in}%
\pgfsys@useobject{currentmarker}{}%
\end{pgfscope}%
\begin{pgfscope}%
\pgfsys@transformshift{1.722045in}{1.293646in}%
\pgfsys@useobject{currentmarker}{}%
\end{pgfscope}%
\begin{pgfscope}%
\pgfsys@transformshift{1.721792in}{1.289621in}%
\pgfsys@useobject{currentmarker}{}%
\end{pgfscope}%
\begin{pgfscope}%
\pgfsys@transformshift{1.721469in}{1.287426in}%
\pgfsys@useobject{currentmarker}{}%
\end{pgfscope}%
\begin{pgfscope}%
\pgfsys@transformshift{1.721493in}{1.283555in}%
\pgfsys@useobject{currentmarker}{}%
\end{pgfscope}%
\begin{pgfscope}%
\pgfsys@transformshift{1.721678in}{1.281434in}%
\pgfsys@useobject{currentmarker}{}%
\end{pgfscope}%
\begin{pgfscope}%
\pgfsys@transformshift{1.721463in}{1.278067in}%
\pgfsys@useobject{currentmarker}{}%
\end{pgfscope}%
\begin{pgfscope}%
\pgfsys@transformshift{1.720781in}{1.274196in}%
\pgfsys@useobject{currentmarker}{}%
\end{pgfscope}%
\begin{pgfscope}%
\pgfsys@transformshift{1.721042in}{1.268737in}%
\pgfsys@useobject{currentmarker}{}%
\end{pgfscope}%
\begin{pgfscope}%
\pgfsys@transformshift{1.721247in}{1.265738in}%
\pgfsys@useobject{currentmarker}{}%
\end{pgfscope}%
\begin{pgfscope}%
\pgfsys@transformshift{1.720880in}{1.261724in}%
\pgfsys@useobject{currentmarker}{}%
\end{pgfscope}%
\begin{pgfscope}%
\pgfsys@transformshift{1.720676in}{1.259517in}%
\pgfsys@useobject{currentmarker}{}%
\end{pgfscope}%
\begin{pgfscope}%
\pgfsys@transformshift{1.721114in}{1.256323in}%
\pgfsys@useobject{currentmarker}{}%
\end{pgfscope}%
\begin{pgfscope}%
\pgfsys@transformshift{1.720794in}{1.252391in}%
\pgfsys@useobject{currentmarker}{}%
\end{pgfscope}%
\begin{pgfscope}%
\pgfsys@transformshift{1.720431in}{1.247990in}%
\pgfsys@useobject{currentmarker}{}%
\end{pgfscope}%
\begin{pgfscope}%
\pgfsys@transformshift{1.720845in}{1.242631in}%
\pgfsys@useobject{currentmarker}{}%
\end{pgfscope}%
\begin{pgfscope}%
\pgfsys@transformshift{1.721195in}{1.239696in}%
\pgfsys@useobject{currentmarker}{}%
\end{pgfscope}%
\begin{pgfscope}%
\pgfsys@transformshift{1.721128in}{1.235996in}%
\pgfsys@useobject{currentmarker}{}%
\end{pgfscope}%
\begin{pgfscope}%
\pgfsys@transformshift{1.720615in}{1.231723in}%
\pgfsys@useobject{currentmarker}{}%
\end{pgfscope}%
\begin{pgfscope}%
\pgfsys@transformshift{1.721733in}{1.225917in}%
\pgfsys@useobject{currentmarker}{}%
\end{pgfscope}%
\begin{pgfscope}%
\pgfsys@transformshift{1.721919in}{1.222670in}%
\pgfsys@useobject{currentmarker}{}%
\end{pgfscope}%
\begin{pgfscope}%
\pgfsys@transformshift{1.721711in}{1.220894in}%
\pgfsys@useobject{currentmarker}{}%
\end{pgfscope}%
\begin{pgfscope}%
\pgfsys@transformshift{1.721669in}{1.219911in}%
\pgfsys@useobject{currentmarker}{}%
\end{pgfscope}%
\begin{pgfscope}%
\pgfsys@transformshift{1.721749in}{1.219376in}%
\pgfsys@useobject{currentmarker}{}%
\end{pgfscope}%
\begin{pgfscope}%
\pgfsys@transformshift{1.721549in}{1.218013in}%
\pgfsys@useobject{currentmarker}{}%
\end{pgfscope}%
\begin{pgfscope}%
\pgfsys@transformshift{1.721427in}{1.216070in}%
\pgfsys@useobject{currentmarker}{}%
\end{pgfscope}%
\begin{pgfscope}%
\pgfsys@transformshift{1.721991in}{1.211917in}%
\pgfsys@useobject{currentmarker}{}%
\end{pgfscope}%
\begin{pgfscope}%
\pgfsys@transformshift{1.722065in}{1.206821in}%
\pgfsys@useobject{currentmarker}{}%
\end{pgfscope}%
\begin{pgfscope}%
\pgfsys@transformshift{1.721525in}{1.204071in}%
\pgfsys@useobject{currentmarker}{}%
\end{pgfscope}%
\begin{pgfscope}%
\pgfsys@transformshift{1.722095in}{1.199272in}%
\pgfsys@useobject{currentmarker}{}%
\end{pgfscope}%
\begin{pgfscope}%
\pgfsys@transformshift{1.722493in}{1.196644in}%
\pgfsys@useobject{currentmarker}{}%
\end{pgfscope}%
\begin{pgfscope}%
\pgfsys@transformshift{1.721632in}{1.192174in}%
\pgfsys@useobject{currentmarker}{}%
\end{pgfscope}%
\begin{pgfscope}%
\pgfsys@transformshift{1.721686in}{1.189670in}%
\pgfsys@useobject{currentmarker}{}%
\end{pgfscope}%
\begin{pgfscope}%
\pgfsys@transformshift{1.722064in}{1.185947in}%
\pgfsys@useobject{currentmarker}{}%
\end{pgfscope}%
\begin{pgfscope}%
\pgfsys@transformshift{1.722256in}{1.183897in}%
\pgfsys@useobject{currentmarker}{}%
\end{pgfscope}%
\begin{pgfscope}%
\pgfsys@transformshift{1.721985in}{1.182798in}%
\pgfsys@useobject{currentmarker}{}%
\end{pgfscope}%
\begin{pgfscope}%
\pgfsys@transformshift{1.722539in}{1.180009in}%
\pgfsys@useobject{currentmarker}{}%
\end{pgfscope}%
\begin{pgfscope}%
\pgfsys@transformshift{1.722592in}{1.178446in}%
\pgfsys@useobject{currentmarker}{}%
\end{pgfscope}%
\begin{pgfscope}%
\pgfsys@transformshift{1.722307in}{1.174814in}%
\pgfsys@useobject{currentmarker}{}%
\end{pgfscope}%
\begin{pgfscope}%
\pgfsys@transformshift{1.721658in}{1.170686in}%
\pgfsys@useobject{currentmarker}{}%
\end{pgfscope}%
\begin{pgfscope}%
\pgfsys@transformshift{1.722773in}{1.164562in}%
\pgfsys@useobject{currentmarker}{}%
\end{pgfscope}%
\begin{pgfscope}%
\pgfsys@transformshift{1.722626in}{1.161142in}%
\pgfsys@useobject{currentmarker}{}%
\end{pgfscope}%
\begin{pgfscope}%
\pgfsys@transformshift{1.722303in}{1.159287in}%
\pgfsys@useobject{currentmarker}{}%
\end{pgfscope}%
\begin{pgfscope}%
\pgfsys@transformshift{1.722256in}{1.158252in}%
\pgfsys@useobject{currentmarker}{}%
\end{pgfscope}%
\begin{pgfscope}%
\pgfsys@transformshift{1.722339in}{1.157689in}%
\pgfsys@useobject{currentmarker}{}%
\end{pgfscope}%
\begin{pgfscope}%
\pgfsys@transformshift{1.722279in}{1.156611in}%
\pgfsys@useobject{currentmarker}{}%
\end{pgfscope}%
\begin{pgfscope}%
\pgfsys@transformshift{1.722063in}{1.155027in}%
\pgfsys@useobject{currentmarker}{}%
\end{pgfscope}%
\begin{pgfscope}%
\pgfsys@transformshift{1.722575in}{1.151063in}%
\pgfsys@useobject{currentmarker}{}%
\end{pgfscope}%
\begin{pgfscope}%
\pgfsys@transformshift{1.722707in}{1.148869in}%
\pgfsys@useobject{currentmarker}{}%
\end{pgfscope}%
\begin{pgfscope}%
\pgfsys@transformshift{1.722643in}{1.145532in}%
\pgfsys@useobject{currentmarker}{}%
\end{pgfscope}%
\begin{pgfscope}%
\pgfsys@transformshift{1.722295in}{1.143729in}%
\pgfsys@useobject{currentmarker}{}%
\end{pgfscope}%
\begin{pgfscope}%
\pgfsys@transformshift{1.723107in}{1.139940in}%
\pgfsys@useobject{currentmarker}{}%
\end{pgfscope}%
\begin{pgfscope}%
\pgfsys@transformshift{1.723203in}{1.135468in}%
\pgfsys@useobject{currentmarker}{}%
\end{pgfscope}%
\begin{pgfscope}%
\pgfsys@transformshift{1.723083in}{1.129391in}%
\pgfsys@useobject{currentmarker}{}%
\end{pgfscope}%
\begin{pgfscope}%
\pgfsys@transformshift{1.722765in}{1.126063in}%
\pgfsys@useobject{currentmarker}{}%
\end{pgfscope}%
\begin{pgfscope}%
\pgfsys@transformshift{1.723515in}{1.121014in}%
\pgfsys@useobject{currentmarker}{}%
\end{pgfscope}%
\begin{pgfscope}%
\pgfsys@transformshift{1.723588in}{1.118207in}%
\pgfsys@useobject{currentmarker}{}%
\end{pgfscope}%
\begin{pgfscope}%
\pgfsys@transformshift{1.723357in}{1.114086in}%
\pgfsys@useobject{currentmarker}{}%
\end{pgfscope}%
\begin{pgfscope}%
\pgfsys@transformshift{1.723205in}{1.111821in}%
\pgfsys@useobject{currentmarker}{}%
\end{pgfscope}%
\begin{pgfscope}%
\pgfsys@transformshift{1.723700in}{1.108727in}%
\pgfsys@useobject{currentmarker}{}%
\end{pgfscope}%
\begin{pgfscope}%
\pgfsys@transformshift{1.723753in}{1.107005in}%
\pgfsys@useobject{currentmarker}{}%
\end{pgfscope}%
\begin{pgfscope}%
\pgfsys@transformshift{1.723644in}{1.104458in}%
\pgfsys@useobject{currentmarker}{}%
\end{pgfscope}%
\begin{pgfscope}%
\pgfsys@transformshift{1.723639in}{1.103057in}%
\pgfsys@useobject{currentmarker}{}%
\end{pgfscope}%
\begin{pgfscope}%
\pgfsys@transformshift{1.723796in}{1.102302in}%
\pgfsys@useobject{currentmarker}{}%
\end{pgfscope}%
\begin{pgfscope}%
\pgfsys@transformshift{1.723814in}{1.100933in}%
\pgfsys@useobject{currentmarker}{}%
\end{pgfscope}%
\begin{pgfscope}%
\pgfsys@transformshift{1.723773in}{1.100181in}%
\pgfsys@useobject{currentmarker}{}%
\end{pgfscope}%
\begin{pgfscope}%
\pgfsys@transformshift{1.723757in}{1.099768in}%
\pgfsys@useobject{currentmarker}{}%
\end{pgfscope}%
\begin{pgfscope}%
\pgfsys@transformshift{1.723790in}{1.099542in}%
\pgfsys@useobject{currentmarker}{}%
\end{pgfscope}%
\begin{pgfscope}%
\pgfsys@transformshift{1.723809in}{1.099418in}%
\pgfsys@useobject{currentmarker}{}%
\end{pgfscope}%
\begin{pgfscope}%
\pgfsys@transformshift{1.723704in}{1.098543in}%
\pgfsys@useobject{currentmarker}{}%
\end{pgfscope}%
\begin{pgfscope}%
\pgfsys@transformshift{1.723747in}{1.098061in}%
\pgfsys@useobject{currentmarker}{}%
\end{pgfscope}%
\begin{pgfscope}%
\pgfsys@transformshift{1.723784in}{1.097797in}%
\pgfsys@useobject{currentmarker}{}%
\end{pgfscope}%
\begin{pgfscope}%
\pgfsys@transformshift{1.723799in}{1.097651in}%
\pgfsys@useobject{currentmarker}{}%
\end{pgfscope}%
\begin{pgfscope}%
\pgfsys@transformshift{1.723688in}{1.096967in}%
\pgfsys@useobject{currentmarker}{}%
\end{pgfscope}%
\begin{pgfscope}%
\pgfsys@transformshift{1.724274in}{1.094650in}%
\pgfsys@useobject{currentmarker}{}%
\end{pgfscope}%
\begin{pgfscope}%
\pgfsys@transformshift{1.724470in}{1.093350in}%
\pgfsys@useobject{currentmarker}{}%
\end{pgfscope}%
\begin{pgfscope}%
\pgfsys@transformshift{1.724769in}{1.090374in}%
\pgfsys@useobject{currentmarker}{}%
\end{pgfscope}%
\begin{pgfscope}%
\pgfsys@transformshift{1.724530in}{1.088747in}%
\pgfsys@useobject{currentmarker}{}%
\end{pgfscope}%
\begin{pgfscope}%
\pgfsys@transformshift{1.725471in}{1.085364in}%
\pgfsys@useobject{currentmarker}{}%
\end{pgfscope}%
\begin{pgfscope}%
\pgfsys@transformshift{1.726114in}{1.081347in}%
\pgfsys@useobject{currentmarker}{}%
\end{pgfscope}%
\begin{pgfscope}%
\pgfsys@transformshift{1.726511in}{1.075339in}%
\pgfsys@useobject{currentmarker}{}%
\end{pgfscope}%
\begin{pgfscope}%
\pgfsys@transformshift{1.726227in}{1.072040in}%
\pgfsys@useobject{currentmarker}{}%
\end{pgfscope}%
\begin{pgfscope}%
\pgfsys@transformshift{1.727142in}{1.066904in}%
\pgfsys@useobject{currentmarker}{}%
\end{pgfscope}%
\begin{pgfscope}%
\pgfsys@transformshift{1.728091in}{1.061114in}%
\pgfsys@useobject{currentmarker}{}%
\end{pgfscope}%
\begin{pgfscope}%
\pgfsys@transformshift{1.727598in}{1.053733in}%
\pgfsys@useobject{currentmarker}{}%
\end{pgfscope}%
\begin{pgfscope}%
\pgfsys@transformshift{1.727507in}{1.049665in}%
\pgfsys@useobject{currentmarker}{}%
\end{pgfscope}%
\begin{pgfscope}%
\pgfsys@transformshift{1.728848in}{1.044701in}%
\pgfsys@useobject{currentmarker}{}%
\end{pgfscope}%
\begin{pgfscope}%
\pgfsys@transformshift{1.729250in}{1.039110in}%
\pgfsys@useobject{currentmarker}{}%
\end{pgfscope}%
\begin{pgfscope}%
\pgfsys@transformshift{1.729245in}{1.033020in}%
\pgfsys@useobject{currentmarker}{}%
\end{pgfscope}%
\begin{pgfscope}%
\pgfsys@transformshift{1.728342in}{1.026307in}%
\pgfsys@useobject{currentmarker}{}%
\end{pgfscope}%
\begin{pgfscope}%
\pgfsys@transformshift{1.729423in}{1.022742in}%
\pgfsys@useobject{currentmarker}{}%
\end{pgfscope}%
\begin{pgfscope}%
\pgfsys@transformshift{1.729800in}{1.018306in}%
\pgfsys@useobject{currentmarker}{}%
\end{pgfscope}%
\begin{pgfscope}%
\pgfsys@transformshift{1.730218in}{1.013268in}%
\pgfsys@useobject{currentmarker}{}%
\end{pgfscope}%
\begin{pgfscope}%
\pgfsys@transformshift{1.730198in}{1.010488in}%
\pgfsys@useobject{currentmarker}{}%
\end{pgfscope}%
\begin{pgfscope}%
\pgfsys@transformshift{1.731187in}{1.006799in}%
\pgfsys@useobject{currentmarker}{}%
\end{pgfscope}%
\begin{pgfscope}%
\pgfsys@transformshift{1.732061in}{1.002410in}%
\pgfsys@useobject{currentmarker}{}%
\end{pgfscope}%
\begin{pgfscope}%
\pgfsys@transformshift{1.732397in}{0.999971in}%
\pgfsys@useobject{currentmarker}{}%
\end{pgfscope}%
\begin{pgfscope}%
\pgfsys@transformshift{1.732349in}{0.998618in}%
\pgfsys@useobject{currentmarker}{}%
\end{pgfscope}%
\begin{pgfscope}%
\pgfsys@transformshift{1.732840in}{0.995889in}%
\pgfsys@useobject{currentmarker}{}%
\end{pgfscope}%
\begin{pgfscope}%
\pgfsys@transformshift{1.733100in}{0.994386in}%
\pgfsys@useobject{currentmarker}{}%
\end{pgfscope}%
\begin{pgfscope}%
\pgfsys@transformshift{1.733234in}{0.993558in}%
\pgfsys@useobject{currentmarker}{}%
\end{pgfscope}%
\begin{pgfscope}%
\pgfsys@transformshift{1.733155in}{0.992200in}%
\pgfsys@useobject{currentmarker}{}%
\end{pgfscope}%
\begin{pgfscope}%
\pgfsys@transformshift{1.733229in}{0.991456in}%
\pgfsys@useobject{currentmarker}{}%
\end{pgfscope}%
\begin{pgfscope}%
\pgfsys@transformshift{1.733291in}{0.991049in}%
\pgfsys@useobject{currentmarker}{}%
\end{pgfscope}%
\begin{pgfscope}%
\pgfsys@transformshift{1.733527in}{0.989882in}%
\pgfsys@useobject{currentmarker}{}%
\end{pgfscope}%
\begin{pgfscope}%
\pgfsys@transformshift{1.733576in}{0.988061in}%
\pgfsys@useobject{currentmarker}{}%
\end{pgfscope}%
\begin{pgfscope}%
\pgfsys@transformshift{1.733667in}{0.987064in}%
\pgfsys@useobject{currentmarker}{}%
\end{pgfscope}%
\begin{pgfscope}%
\pgfsys@transformshift{1.733992in}{0.984892in}%
\pgfsys@useobject{currentmarker}{}%
\end{pgfscope}%
\begin{pgfscope}%
\pgfsys@transformshift{1.734787in}{0.982230in}%
\pgfsys@useobject{currentmarker}{}%
\end{pgfscope}%
\begin{pgfscope}%
\pgfsys@transformshift{1.734968in}{0.980712in}%
\pgfsys@useobject{currentmarker}{}%
\end{pgfscope}%
\begin{pgfscope}%
\pgfsys@transformshift{1.734993in}{0.979872in}%
\pgfsys@useobject{currentmarker}{}%
\end{pgfscope}%
\begin{pgfscope}%
\pgfsys@transformshift{1.735038in}{0.979412in}%
\pgfsys@useobject{currentmarker}{}%
\end{pgfscope}%
\begin{pgfscope}%
\pgfsys@transformshift{1.735350in}{0.978321in}%
\pgfsys@useobject{currentmarker}{}%
\end{pgfscope}%
\begin{pgfscope}%
\pgfsys@transformshift{1.735702in}{0.976464in}%
\pgfsys@useobject{currentmarker}{}%
\end{pgfscope}%
\begin{pgfscope}%
\pgfsys@transformshift{1.735891in}{0.973924in}%
\pgfsys@useobject{currentmarker}{}%
\end{pgfscope}%
\begin{pgfscope}%
\pgfsys@transformshift{1.736165in}{0.970627in}%
\pgfsys@useobject{currentmarker}{}%
\end{pgfscope}%
\begin{pgfscope}%
\pgfsys@transformshift{1.736693in}{0.968886in}%
\pgfsys@useobject{currentmarker}{}%
\end{pgfscope}%
\begin{pgfscope}%
\pgfsys@transformshift{1.736934in}{0.967914in}%
\pgfsys@useobject{currentmarker}{}%
\end{pgfscope}%
\begin{pgfscope}%
\pgfsys@transformshift{1.737036in}{0.967373in}%
\pgfsys@useobject{currentmarker}{}%
\end{pgfscope}%
\begin{pgfscope}%
\pgfsys@transformshift{1.737001in}{0.966333in}%
\pgfsys@useobject{currentmarker}{}%
\end{pgfscope}%
\begin{pgfscope}%
\pgfsys@transformshift{1.737060in}{0.965763in}%
\pgfsys@useobject{currentmarker}{}%
\end{pgfscope}%
\begin{pgfscope}%
\pgfsys@transformshift{1.737272in}{0.964725in}%
\pgfsys@useobject{currentmarker}{}%
\end{pgfscope}%
\begin{pgfscope}%
\pgfsys@transformshift{1.737755in}{0.962824in}%
\pgfsys@useobject{currentmarker}{}%
\end{pgfscope}%
\begin{pgfscope}%
\pgfsys@transformshift{1.737755in}{0.961746in}%
\pgfsys@useobject{currentmarker}{}%
\end{pgfscope}%
\begin{pgfscope}%
\pgfsys@transformshift{1.738085in}{0.960003in}%
\pgfsys@useobject{currentmarker}{}%
\end{pgfscope}%
\begin{pgfscope}%
\pgfsys@transformshift{1.737981in}{0.957435in}%
\pgfsys@useobject{currentmarker}{}%
\end{pgfscope}%
\begin{pgfscope}%
\pgfsys@transformshift{1.738341in}{0.953894in}%
\pgfsys@useobject{currentmarker}{}%
\end{pgfscope}%
\begin{pgfscope}%
\pgfsys@transformshift{1.738043in}{0.951959in}%
\pgfsys@useobject{currentmarker}{}%
\end{pgfscope}%
\begin{pgfscope}%
\pgfsys@transformshift{1.737704in}{0.950938in}%
\pgfsys@useobject{currentmarker}{}%
\end{pgfscope}%
\begin{pgfscope}%
\pgfsys@transformshift{1.736498in}{0.949735in}%
\pgfsys@useobject{currentmarker}{}%
\end{pgfscope}%
\begin{pgfscope}%
\pgfsys@transformshift{1.734152in}{0.948245in}%
\pgfsys@useobject{currentmarker}{}%
\end{pgfscope}%
\begin{pgfscope}%
\pgfsys@transformshift{1.732648in}{0.947969in}%
\pgfsys@useobject{currentmarker}{}%
\end{pgfscope}%
\begin{pgfscope}%
\pgfsys@transformshift{1.730137in}{0.947651in}%
\pgfsys@useobject{currentmarker}{}%
\end{pgfscope}%
\begin{pgfscope}%
\pgfsys@transformshift{1.726136in}{0.948075in}%
\pgfsys@useobject{currentmarker}{}%
\end{pgfscope}%
\begin{pgfscope}%
\pgfsys@transformshift{1.721055in}{0.948636in}%
\pgfsys@useobject{currentmarker}{}%
\end{pgfscope}%
\begin{pgfscope}%
\pgfsys@transformshift{1.714778in}{0.949504in}%
\pgfsys@useobject{currentmarker}{}%
\end{pgfscope}%
\begin{pgfscope}%
\pgfsys@transformshift{1.707596in}{0.949177in}%
\pgfsys@useobject{currentmarker}{}%
\end{pgfscope}%
\begin{pgfscope}%
\pgfsys@transformshift{1.700056in}{0.951107in}%
\pgfsys@useobject{currentmarker}{}%
\end{pgfscope}%
\begin{pgfscope}%
\pgfsys@transformshift{1.691662in}{0.951673in}%
\pgfsys@useobject{currentmarker}{}%
\end{pgfscope}%
\begin{pgfscope}%
\pgfsys@transformshift{1.687146in}{0.952680in}%
\pgfsys@useobject{currentmarker}{}%
\end{pgfscope}%
\begin{pgfscope}%
\pgfsys@transformshift{1.684603in}{0.952775in}%
\pgfsys@useobject{currentmarker}{}%
\end{pgfscope}%
\begin{pgfscope}%
\pgfsys@transformshift{1.681343in}{0.953357in}%
\pgfsys@useobject{currentmarker}{}%
\end{pgfscope}%
\begin{pgfscope}%
\pgfsys@transformshift{1.679536in}{0.953132in}%
\pgfsys@useobject{currentmarker}{}%
\end{pgfscope}%
\begin{pgfscope}%
\pgfsys@transformshift{1.677138in}{0.953268in}%
\pgfsys@useobject{currentmarker}{}%
\end{pgfscope}%
\begin{pgfscope}%
\pgfsys@transformshift{1.672310in}{0.953220in}%
\pgfsys@useobject{currentmarker}{}%
\end{pgfscope}%
\begin{pgfscope}%
\pgfsys@transformshift{1.669666in}{0.953472in}%
\pgfsys@useobject{currentmarker}{}%
\end{pgfscope}%
\begin{pgfscope}%
\pgfsys@transformshift{1.665544in}{0.953700in}%
\pgfsys@useobject{currentmarker}{}%
\end{pgfscope}%
\begin{pgfscope}%
\pgfsys@transformshift{1.659916in}{0.953965in}%
\pgfsys@useobject{currentmarker}{}%
\end{pgfscope}%
\begin{pgfscope}%
\pgfsys@transformshift{1.656822in}{0.953797in}%
\pgfsys@useobject{currentmarker}{}%
\end{pgfscope}%
\begin{pgfscope}%
\pgfsys@transformshift{1.651684in}{0.953790in}%
\pgfsys@useobject{currentmarker}{}%
\end{pgfscope}%
\begin{pgfscope}%
\pgfsys@transformshift{1.645951in}{0.953765in}%
\pgfsys@useobject{currentmarker}{}%
\end{pgfscope}%
\begin{pgfscope}%
\pgfsys@transformshift{1.638096in}{0.953528in}%
\pgfsys@useobject{currentmarker}{}%
\end{pgfscope}%
\begin{pgfscope}%
\pgfsys@transformshift{1.627734in}{0.952571in}%
\pgfsys@useobject{currentmarker}{}%
\end{pgfscope}%
\begin{pgfscope}%
\pgfsys@transformshift{1.622014in}{0.952365in}%
\pgfsys@useobject{currentmarker}{}%
\end{pgfscope}%
\begin{pgfscope}%
\pgfsys@transformshift{1.613574in}{0.952051in}%
\pgfsys@useobject{currentmarker}{}%
\end{pgfscope}%
\begin{pgfscope}%
\pgfsys@transformshift{1.608933in}{0.951844in}%
\pgfsys@useobject{currentmarker}{}%
\end{pgfscope}%
\begin{pgfscope}%
\pgfsys@transformshift{1.602733in}{0.951082in}%
\pgfsys@useobject{currentmarker}{}%
\end{pgfscope}%
\begin{pgfscope}%
\pgfsys@transformshift{1.599300in}{0.950928in}%
\pgfsys@useobject{currentmarker}{}%
\end{pgfscope}%
\begin{pgfscope}%
\pgfsys@transformshift{1.594080in}{0.950885in}%
\pgfsys@useobject{currentmarker}{}%
\end{pgfscope}%
\begin{pgfscope}%
\pgfsys@transformshift{1.586279in}{0.949569in}%
\pgfsys@useobject{currentmarker}{}%
\end{pgfscope}%
\begin{pgfscope}%
\pgfsys@transformshift{1.581940in}{0.949248in}%
\pgfsys@useobject{currentmarker}{}%
\end{pgfscope}%
\begin{pgfscope}%
\pgfsys@transformshift{1.575208in}{0.949233in}%
\pgfsys@useobject{currentmarker}{}%
\end{pgfscope}%
\begin{pgfscope}%
\pgfsys@transformshift{1.571507in}{0.949170in}%
\pgfsys@useobject{currentmarker}{}%
\end{pgfscope}%
\begin{pgfscope}%
\pgfsys@transformshift{1.565967in}{0.948493in}%
\pgfsys@useobject{currentmarker}{}%
\end{pgfscope}%
\begin{pgfscope}%
\pgfsys@transformshift{1.562909in}{0.948232in}%
\pgfsys@useobject{currentmarker}{}%
\end{pgfscope}%
\begin{pgfscope}%
\pgfsys@transformshift{1.556864in}{0.947712in}%
\pgfsys@useobject{currentmarker}{}%
\end{pgfscope}%
\begin{pgfscope}%
\pgfsys@transformshift{1.549573in}{0.946663in}%
\pgfsys@useobject{currentmarker}{}%
\end{pgfscope}%
\begin{pgfscope}%
\pgfsys@transformshift{1.541235in}{0.946023in}%
\pgfsys@useobject{currentmarker}{}%
\end{pgfscope}%
\begin{pgfscope}%
\pgfsys@transformshift{1.531155in}{0.946018in}%
\pgfsys@useobject{currentmarker}{}%
\end{pgfscope}%
\begin{pgfscope}%
\pgfsys@transformshift{1.520081in}{0.945662in}%
\pgfsys@useobject{currentmarker}{}%
\end{pgfscope}%
\begin{pgfscope}%
\pgfsys@transformshift{1.506840in}{0.944631in}%
\pgfsys@useobject{currentmarker}{}%
\end{pgfscope}%
\begin{pgfscope}%
\pgfsys@transformshift{1.499552in}{0.944141in}%
\pgfsys@useobject{currentmarker}{}%
\end{pgfscope}%
\begin{pgfscope}%
\pgfsys@transformshift{1.490451in}{0.943085in}%
\pgfsys@useobject{currentmarker}{}%
\end{pgfscope}%
\begin{pgfscope}%
\pgfsys@transformshift{1.480106in}{0.942085in}%
\pgfsys@useobject{currentmarker}{}%
\end{pgfscope}%
\begin{pgfscope}%
\pgfsys@transformshift{1.469046in}{0.941007in}%
\pgfsys@useobject{currentmarker}{}%
\end{pgfscope}%
\begin{pgfscope}%
\pgfsys@transformshift{1.455454in}{0.939795in}%
\pgfsys@useobject{currentmarker}{}%
\end{pgfscope}%
\begin{pgfscope}%
\pgfsys@transformshift{1.447949in}{0.939770in}%
\pgfsys@useobject{currentmarker}{}%
\end{pgfscope}%
\begin{pgfscope}%
\pgfsys@transformshift{1.437838in}{0.938306in}%
\pgfsys@useobject{currentmarker}{}%
\end{pgfscope}%
\begin{pgfscope}%
\pgfsys@transformshift{1.432231in}{0.937945in}%
\pgfsys@useobject{currentmarker}{}%
\end{pgfscope}%
\begin{pgfscope}%
\pgfsys@transformshift{1.423706in}{0.937484in}%
\pgfsys@useobject{currentmarker}{}%
\end{pgfscope}%
\begin{pgfscope}%
\pgfsys@transformshift{1.414139in}{0.936513in}%
\pgfsys@useobject{currentmarker}{}%
\end{pgfscope}%
\begin{pgfscope}%
\pgfsys@transformshift{1.403821in}{0.935798in}%
\pgfsys@useobject{currentmarker}{}%
\end{pgfscope}%
\begin{pgfscope}%
\pgfsys@transformshift{1.392002in}{0.936524in}%
\pgfsys@useobject{currentmarker}{}%
\end{pgfscope}%
\begin{pgfscope}%
\pgfsys@transformshift{1.385500in}{0.936169in}%
\pgfsys@useobject{currentmarker}{}%
\end{pgfscope}%
\begin{pgfscope}%
\pgfsys@transformshift{1.376538in}{0.935646in}%
\pgfsys@useobject{currentmarker}{}%
\end{pgfscope}%
\begin{pgfscope}%
\pgfsys@transformshift{1.366832in}{0.935266in}%
\pgfsys@useobject{currentmarker}{}%
\end{pgfscope}%
\begin{pgfscope}%
\pgfsys@transformshift{1.355192in}{0.934713in}%
\pgfsys@useobject{currentmarker}{}%
\end{pgfscope}%
\begin{pgfscope}%
\pgfsys@transformshift{1.348783in}{0.934698in}%
\pgfsys@useobject{currentmarker}{}%
\end{pgfscope}%
\begin{pgfscope}%
\pgfsys@transformshift{1.340942in}{0.933750in}%
\pgfsys@useobject{currentmarker}{}%
\end{pgfscope}%
\begin{pgfscope}%
\pgfsys@transformshift{1.336598in}{0.933699in}%
\pgfsys@useobject{currentmarker}{}%
\end{pgfscope}%
\begin{pgfscope}%
\pgfsys@transformshift{1.330128in}{0.933526in}%
\pgfsys@useobject{currentmarker}{}%
\end{pgfscope}%
\begin{pgfscope}%
\pgfsys@transformshift{1.323065in}{0.933354in}%
\pgfsys@useobject{currentmarker}{}%
\end{pgfscope}%
\begin{pgfscope}%
\pgfsys@transformshift{1.314337in}{0.932901in}%
\pgfsys@useobject{currentmarker}{}%
\end{pgfscope}%
\begin{pgfscope}%
\pgfsys@transformshift{1.309531in}{0.932987in}%
\pgfsys@useobject{currentmarker}{}%
\end{pgfscope}%
\begin{pgfscope}%
\pgfsys@transformshift{1.304058in}{0.933057in}%
\pgfsys@useobject{currentmarker}{}%
\end{pgfscope}%
\begin{pgfscope}%
\pgfsys@transformshift{1.298069in}{0.933442in}%
\pgfsys@useobject{currentmarker}{}%
\end{pgfscope}%
\begin{pgfscope}%
\pgfsys@transformshift{1.294850in}{0.934171in}%
\pgfsys@useobject{currentmarker}{}%
\end{pgfscope}%
\begin{pgfscope}%
\pgfsys@transformshift{1.293245in}{0.935019in}%
\pgfsys@useobject{currentmarker}{}%
\end{pgfscope}%
\begin{pgfscope}%
\pgfsys@transformshift{1.291496in}{0.936932in}%
\pgfsys@useobject{currentmarker}{}%
\end{pgfscope}%
\begin{pgfscope}%
\pgfsys@transformshift{1.290488in}{0.937941in}%
\pgfsys@useobject{currentmarker}{}%
\end{pgfscope}%
\begin{pgfscope}%
\pgfsys@transformshift{1.290528in}{0.939965in}%
\pgfsys@useobject{currentmarker}{}%
\end{pgfscope}%
\begin{pgfscope}%
\pgfsys@transformshift{1.290522in}{0.942583in}%
\pgfsys@useobject{currentmarker}{}%
\end{pgfscope}%
\begin{pgfscope}%
\pgfsys@transformshift{1.290950in}{0.946324in}%
\pgfsys@useobject{currentmarker}{}%
\end{pgfscope}%
\begin{pgfscope}%
\pgfsys@transformshift{1.291622in}{0.951031in}%
\pgfsys@useobject{currentmarker}{}%
\end{pgfscope}%
\begin{pgfscope}%
\pgfsys@transformshift{1.292534in}{0.956827in}%
\pgfsys@useobject{currentmarker}{}%
\end{pgfscope}%
\begin{pgfscope}%
\pgfsys@transformshift{1.292883in}{0.963643in}%
\pgfsys@useobject{currentmarker}{}%
\end{pgfscope}%
\begin{pgfscope}%
\pgfsys@transformshift{1.293338in}{0.967368in}%
\pgfsys@useobject{currentmarker}{}%
\end{pgfscope}%
\begin{pgfscope}%
\pgfsys@transformshift{1.292324in}{0.971535in}%
\pgfsys@useobject{currentmarker}{}%
\end{pgfscope}%
\begin{pgfscope}%
\pgfsys@transformshift{1.294812in}{0.978915in}%
\pgfsys@useobject{currentmarker}{}%
\end{pgfscope}%
\begin{pgfscope}%
\pgfsys@transformshift{1.291569in}{0.989001in}%
\pgfsys@useobject{currentmarker}{}%
\end{pgfscope}%
\begin{pgfscope}%
\pgfsys@transformshift{1.291195in}{0.994816in}%
\pgfsys@useobject{currentmarker}{}%
\end{pgfscope}%
\begin{pgfscope}%
\pgfsys@transformshift{1.293223in}{1.003690in}%
\pgfsys@useobject{currentmarker}{}%
\end{pgfscope}%
\begin{pgfscope}%
\pgfsys@transformshift{1.289310in}{1.014408in}%
\pgfsys@useobject{currentmarker}{}%
\end{pgfscope}%
\begin{pgfscope}%
\pgfsys@transformshift{1.290004in}{1.026417in}%
\pgfsys@useobject{currentmarker}{}%
\end{pgfscope}%
\begin{pgfscope}%
\pgfsys@transformshift{1.293668in}{1.039683in}%
\pgfsys@useobject{currentmarker}{}%
\end{pgfscope}%
\begin{pgfscope}%
\pgfsys@transformshift{1.289090in}{1.055889in}%
\pgfsys@useobject{currentmarker}{}%
\end{pgfscope}%
\begin{pgfscope}%
\pgfsys@transformshift{1.288941in}{1.065149in}%
\pgfsys@useobject{currentmarker}{}%
\end{pgfscope}%
\begin{pgfscope}%
\pgfsys@transformshift{1.290632in}{1.077386in}%
\pgfsys@useobject{currentmarker}{}%
\end{pgfscope}%
\begin{pgfscope}%
\pgfsys@transformshift{1.287835in}{1.090662in}%
\pgfsys@useobject{currentmarker}{}%
\end{pgfscope}%
\begin{pgfscope}%
\pgfsys@transformshift{1.283836in}{1.104148in}%
\pgfsys@useobject{currentmarker}{}%
\end{pgfscope}%
\begin{pgfscope}%
\pgfsys@transformshift{1.280112in}{1.119594in}%
\pgfsys@useobject{currentmarker}{}%
\end{pgfscope}%
\begin{pgfscope}%
\pgfsys@transformshift{1.283713in}{1.135605in}%
\pgfsys@useobject{currentmarker}{}%
\end{pgfscope}%
\begin{pgfscope}%
\pgfsys@transformshift{1.277377in}{1.153528in}%
\pgfsys@useobject{currentmarker}{}%
\end{pgfscope}%
\begin{pgfscope}%
\pgfsys@transformshift{1.275880in}{1.173299in}%
\pgfsys@useobject{currentmarker}{}%
\end{pgfscope}%
\begin{pgfscope}%
\pgfsys@transformshift{1.278101in}{1.196543in}%
\pgfsys@useobject{currentmarker}{}%
\end{pgfscope}%
\begin{pgfscope}%
\pgfsys@transformshift{1.275878in}{1.220663in}%
\pgfsys@useobject{currentmarker}{}%
\end{pgfscope}%
\begin{pgfscope}%
\pgfsys@transformshift{1.267280in}{1.243868in}%
\pgfsys@useobject{currentmarker}{}%
\end{pgfscope}%
\begin{pgfscope}%
\pgfsys@transformshift{1.266822in}{1.269202in}%
\pgfsys@useobject{currentmarker}{}%
\end{pgfscope}%
\begin{pgfscope}%
\pgfsys@transformshift{1.272240in}{1.295098in}%
\pgfsys@useobject{currentmarker}{}%
\end{pgfscope}%
\begin{pgfscope}%
\pgfsys@transformshift{1.267617in}{1.322928in}%
\pgfsys@useobject{currentmarker}{}%
\end{pgfscope}%
\begin{pgfscope}%
\pgfsys@transformshift{1.263985in}{1.338014in}%
\pgfsys@useobject{currentmarker}{}%
\end{pgfscope}%
\begin{pgfscope}%
\pgfsys@transformshift{1.260232in}{1.354000in}%
\pgfsys@useobject{currentmarker}{}%
\end{pgfscope}%
\begin{pgfscope}%
\pgfsys@transformshift{1.265004in}{1.371422in}%
\pgfsys@useobject{currentmarker}{}%
\end{pgfscope}%
\begin{pgfscope}%
\pgfsys@transformshift{1.258631in}{1.392495in}%
\pgfsys@useobject{currentmarker}{}%
\end{pgfscope}%
\begin{pgfscope}%
\pgfsys@transformshift{1.256845in}{1.404471in}%
\pgfsys@useobject{currentmarker}{}%
\end{pgfscope}%
\begin{pgfscope}%
\pgfsys@transformshift{1.254336in}{1.419241in}%
\pgfsys@useobject{currentmarker}{}%
\end{pgfscope}%
\begin{pgfscope}%
\pgfsys@transformshift{1.255501in}{1.427398in}%
\pgfsys@useobject{currentmarker}{}%
\end{pgfscope}%
\begin{pgfscope}%
\pgfsys@transformshift{1.252223in}{1.437920in}%
\pgfsys@useobject{currentmarker}{}%
\end{pgfscope}%
\begin{pgfscope}%
\pgfsys@transformshift{1.252870in}{1.443947in}%
\pgfsys@useobject{currentmarker}{}%
\end{pgfscope}%
\begin{pgfscope}%
\pgfsys@transformshift{1.253750in}{1.451961in}%
\pgfsys@useobject{currentmarker}{}%
\end{pgfscope}%
\begin{pgfscope}%
\pgfsys@transformshift{1.253298in}{1.461006in}%
\pgfsys@useobject{currentmarker}{}%
\end{pgfscope}%
\begin{pgfscope}%
\pgfsys@transformshift{1.252043in}{1.465827in}%
\pgfsys@useobject{currentmarker}{}%
\end{pgfscope}%
\begin{pgfscope}%
\pgfsys@transformshift{1.252124in}{1.468565in}%
\pgfsys@useobject{currentmarker}{}%
\end{pgfscope}%
\begin{pgfscope}%
\pgfsys@transformshift{1.252724in}{1.473087in}%
\pgfsys@useobject{currentmarker}{}%
\end{pgfscope}%
\begin{pgfscope}%
\pgfsys@transformshift{1.251619in}{1.479495in}%
\pgfsys@useobject{currentmarker}{}%
\end{pgfscope}%
\begin{pgfscope}%
\pgfsys@transformshift{1.250649in}{1.482937in}%
\pgfsys@useobject{currentmarker}{}%
\end{pgfscope}%
\begin{pgfscope}%
\pgfsys@transformshift{1.250535in}{1.486997in}%
\pgfsys@useobject{currentmarker}{}%
\end{pgfscope}%
\begin{pgfscope}%
\pgfsys@transformshift{1.251631in}{1.492253in}%
\pgfsys@useobject{currentmarker}{}%
\end{pgfscope}%
\begin{pgfscope}%
\pgfsys@transformshift{1.248813in}{1.501161in}%
\pgfsys@useobject{currentmarker}{}%
\end{pgfscope}%
\begin{pgfscope}%
\pgfsys@transformshift{1.247934in}{1.506224in}%
\pgfsys@useobject{currentmarker}{}%
\end{pgfscope}%
\begin{pgfscope}%
\pgfsys@transformshift{1.246476in}{1.512556in}%
\pgfsys@useobject{currentmarker}{}%
\end{pgfscope}%
\begin{pgfscope}%
\pgfsys@transformshift{1.248453in}{1.519561in}%
\pgfsys@useobject{currentmarker}{}%
\end{pgfscope}%
\begin{pgfscope}%
\pgfsys@transformshift{1.245637in}{1.530022in}%
\pgfsys@useobject{currentmarker}{}%
\end{pgfscope}%
\begin{pgfscope}%
\pgfsys@transformshift{1.244823in}{1.535925in}%
\pgfsys@useobject{currentmarker}{}%
\end{pgfscope}%
\begin{pgfscope}%
\pgfsys@transformshift{1.244432in}{1.543417in}%
\pgfsys@useobject{currentmarker}{}%
\end{pgfscope}%
\begin{pgfscope}%
\pgfsys@transformshift{1.246193in}{1.551369in}%
\pgfsys@useobject{currentmarker}{}%
\end{pgfscope}%
\begin{pgfscope}%
\pgfsys@transformshift{1.246251in}{1.563047in}%
\pgfsys@useobject{currentmarker}{}%
\end{pgfscope}%
\begin{pgfscope}%
\pgfsys@transformshift{1.247997in}{1.575985in}%
\pgfsys@useobject{currentmarker}{}%
\end{pgfscope}%
\begin{pgfscope}%
\pgfsys@transformshift{1.251132in}{1.590771in}%
\pgfsys@useobject{currentmarker}{}%
\end{pgfscope}%
\begin{pgfscope}%
\pgfsys@transformshift{1.247214in}{1.606289in}%
\pgfsys@useobject{currentmarker}{}%
\end{pgfscope}%
\begin{pgfscope}%
\pgfsys@transformshift{1.253594in}{1.625306in}%
\pgfsys@useobject{currentmarker}{}%
\end{pgfscope}%
\begin{pgfscope}%
\pgfsys@transformshift{1.254548in}{1.636296in}%
\pgfsys@useobject{currentmarker}{}%
\end{pgfscope}%
\begin{pgfscope}%
\pgfsys@transformshift{1.257333in}{1.649227in}%
\pgfsys@useobject{currentmarker}{}%
\end{pgfscope}%
\begin{pgfscope}%
\pgfsys@transformshift{1.255700in}{1.656316in}%
\pgfsys@useobject{currentmarker}{}%
\end{pgfscope}%
\begin{pgfscope}%
\pgfsys@transformshift{1.258361in}{1.665815in}%
\pgfsys@useobject{currentmarker}{}%
\end{pgfscope}%
\begin{pgfscope}%
\pgfsys@transformshift{1.258968in}{1.671206in}%
\pgfsys@useobject{currentmarker}{}%
\end{pgfscope}%
\begin{pgfscope}%
\pgfsys@transformshift{1.259672in}{1.674106in}%
\pgfsys@useobject{currentmarker}{}%
\end{pgfscope}%
\begin{pgfscope}%
\pgfsys@transformshift{1.259161in}{1.677553in}%
\pgfsys@useobject{currentmarker}{}%
\end{pgfscope}%
\begin{pgfscope}%
\pgfsys@transformshift{1.261352in}{1.684108in}%
\pgfsys@useobject{currentmarker}{}%
\end{pgfscope}%
\begin{pgfscope}%
\pgfsys@transformshift{1.261821in}{1.687881in}%
\pgfsys@useobject{currentmarker}{}%
\end{pgfscope}%
\begin{pgfscope}%
\pgfsys@transformshift{1.262785in}{1.693687in}%
\pgfsys@useobject{currentmarker}{}%
\end{pgfscope}%
\begin{pgfscope}%
\pgfsys@transformshift{1.261696in}{1.700054in}%
\pgfsys@useobject{currentmarker}{}%
\end{pgfscope}%
\begin{pgfscope}%
\pgfsys@transformshift{1.264740in}{1.709484in}%
\pgfsys@useobject{currentmarker}{}%
\end{pgfscope}%
\begin{pgfscope}%
\pgfsys@transformshift{1.265434in}{1.714890in}%
\pgfsys@useobject{currentmarker}{}%
\end{pgfscope}%
\begin{pgfscope}%
\pgfsys@transformshift{1.265784in}{1.722379in}%
\pgfsys@useobject{currentmarker}{}%
\end{pgfscope}%
\begin{pgfscope}%
\pgfsys@transformshift{1.265618in}{1.726498in}%
\pgfsys@useobject{currentmarker}{}%
\end{pgfscope}%
\begin{pgfscope}%
\pgfsys@transformshift{1.265953in}{1.728741in}%
\pgfsys@useobject{currentmarker}{}%
\end{pgfscope}%
\begin{pgfscope}%
\pgfsys@transformshift{1.265586in}{1.731481in}%
\pgfsys@useobject{currentmarker}{}%
\end{pgfscope}%
\begin{pgfscope}%
\pgfsys@transformshift{1.264234in}{1.734640in}%
\pgfsys@useobject{currentmarker}{}%
\end{pgfscope}%
\begin{pgfscope}%
\pgfsys@transformshift{1.261027in}{1.737455in}%
\pgfsys@useobject{currentmarker}{}%
\end{pgfscope}%
\begin{pgfscope}%
\pgfsys@transformshift{1.256048in}{1.739616in}%
\pgfsys@useobject{currentmarker}{}%
\end{pgfscope}%
\begin{pgfscope}%
\pgfsys@transformshift{1.249644in}{1.740985in}%
\pgfsys@useobject{currentmarker}{}%
\end{pgfscope}%
\begin{pgfscope}%
\pgfsys@transformshift{1.245947in}{1.741255in}%
\pgfsys@useobject{currentmarker}{}%
\end{pgfscope}%
\begin{pgfscope}%
\pgfsys@transformshift{1.249939in}{1.741574in}%
\pgfsys@useobject{currentmarker}{}%
\end{pgfscope}%
\begin{pgfscope}%
\pgfsys@transformshift{1.252141in}{1.741619in}%
\pgfsys@useobject{currentmarker}{}%
\end{pgfscope}%
\begin{pgfscope}%
\pgfsys@transformshift{1.255081in}{1.741677in}%
\pgfsys@useobject{currentmarker}{}%
\end{pgfscope}%
\begin{pgfscope}%
\pgfsys@transformshift{1.259220in}{1.741576in}%
\pgfsys@useobject{currentmarker}{}%
\end{pgfscope}%
\begin{pgfscope}%
\pgfsys@transformshift{1.264255in}{1.741730in}%
\pgfsys@useobject{currentmarker}{}%
\end{pgfscope}%
\begin{pgfscope}%
\pgfsys@transformshift{1.267014in}{1.741476in}%
\pgfsys@useobject{currentmarker}{}%
\end{pgfscope}%
\begin{pgfscope}%
\pgfsys@transformshift{1.268526in}{1.741660in}%
\pgfsys@useobject{currentmarker}{}%
\end{pgfscope}%
\begin{pgfscope}%
\pgfsys@transformshift{1.273809in}{1.741344in}%
\pgfsys@useobject{currentmarker}{}%
\end{pgfscope}%
\begin{pgfscope}%
\pgfsys@transformshift{1.280156in}{1.742140in}%
\pgfsys@useobject{currentmarker}{}%
\end{pgfscope}%
\begin{pgfscope}%
\pgfsys@transformshift{1.287679in}{1.742052in}%
\pgfsys@useobject{currentmarker}{}%
\end{pgfscope}%
\begin{pgfscope}%
\pgfsys@transformshift{1.297025in}{1.743247in}%
\pgfsys@useobject{currentmarker}{}%
\end{pgfscope}%
\begin{pgfscope}%
\pgfsys@transformshift{1.306922in}{1.742161in}%
\pgfsys@useobject{currentmarker}{}%
\end{pgfscope}%
\begin{pgfscope}%
\pgfsys@transformshift{1.317320in}{1.742768in}%
\pgfsys@useobject{currentmarker}{}%
\end{pgfscope}%
\begin{pgfscope}%
\pgfsys@transformshift{1.329650in}{1.744477in}%
\pgfsys@useobject{currentmarker}{}%
\end{pgfscope}%
\begin{pgfscope}%
\pgfsys@transformshift{1.336495in}{1.744626in}%
\pgfsys@useobject{currentmarker}{}%
\end{pgfscope}%
\begin{pgfscope}%
\pgfsys@transformshift{1.340261in}{1.744635in}%
\pgfsys@useobject{currentmarker}{}%
\end{pgfscope}%
\begin{pgfscope}%
\pgfsys@transformshift{1.345989in}{1.744863in}%
\pgfsys@useobject{currentmarker}{}%
\end{pgfscope}%
\begin{pgfscope}%
\pgfsys@transformshift{1.354234in}{1.746145in}%
\pgfsys@useobject{currentmarker}{}%
\end{pgfscope}%
\begin{pgfscope}%
\pgfsys@transformshift{1.363374in}{1.746307in}%
\pgfsys@useobject{currentmarker}{}%
\end{pgfscope}%
\begin{pgfscope}%
\pgfsys@transformshift{1.368403in}{1.746317in}%
\pgfsys@useobject{currentmarker}{}%
\end{pgfscope}%
\begin{pgfscope}%
\pgfsys@transformshift{1.375593in}{1.746706in}%
\pgfsys@useobject{currentmarker}{}%
\end{pgfscope}%
\begin{pgfscope}%
\pgfsys@transformshift{1.384032in}{1.747745in}%
\pgfsys@useobject{currentmarker}{}%
\end{pgfscope}%
\begin{pgfscope}%
\pgfsys@transformshift{1.393145in}{1.748604in}%
\pgfsys@useobject{currentmarker}{}%
\end{pgfscope}%
\begin{pgfscope}%
\pgfsys@transformshift{1.402926in}{1.747968in}%
\pgfsys@useobject{currentmarker}{}%
\end{pgfscope}%
\begin{pgfscope}%
\pgfsys@transformshift{1.408313in}{1.747738in}%
\pgfsys@useobject{currentmarker}{}%
\end{pgfscope}%
\begin{pgfscope}%
\pgfsys@transformshift{1.415732in}{1.748418in}%
\pgfsys@useobject{currentmarker}{}%
\end{pgfscope}%
\begin{pgfscope}%
\pgfsys@transformshift{1.425121in}{1.748319in}%
\pgfsys@useobject{currentmarker}{}%
\end{pgfscope}%
\begin{pgfscope}%
\pgfsys@transformshift{1.435341in}{1.748311in}%
\pgfsys@useobject{currentmarker}{}%
\end{pgfscope}%
\begin{pgfscope}%
\pgfsys@transformshift{1.446389in}{1.744098in}%
\pgfsys@useobject{currentmarker}{}%
\end{pgfscope}%
\begin{pgfscope}%
\pgfsys@transformshift{1.460547in}{1.746329in}%
\pgfsys@useobject{currentmarker}{}%
\end{pgfscope}%
\begin{pgfscope}%
\pgfsys@transformshift{1.468422in}{1.746668in}%
\pgfsys@useobject{currentmarker}{}%
\end{pgfscope}%
\begin{pgfscope}%
\pgfsys@transformshift{1.476811in}{1.746796in}%
\pgfsys@useobject{currentmarker}{}%
\end{pgfscope}%
\begin{pgfscope}%
\pgfsys@transformshift{1.486090in}{1.746304in}%
\pgfsys@useobject{currentmarker}{}%
\end{pgfscope}%
\begin{pgfscope}%
\pgfsys@transformshift{1.496601in}{1.749136in}%
\pgfsys@useobject{currentmarker}{}%
\end{pgfscope}%
\begin{pgfscope}%
\pgfsys@transformshift{1.502580in}{1.749463in}%
\pgfsys@useobject{currentmarker}{}%
\end{pgfscope}%
\begin{pgfscope}%
\pgfsys@transformshift{1.511510in}{1.750712in}%
\pgfsys@useobject{currentmarker}{}%
\end{pgfscope}%
\begin{pgfscope}%
\pgfsys@transformshift{1.522544in}{1.751475in}%
\pgfsys@useobject{currentmarker}{}%
\end{pgfscope}%
\begin{pgfscope}%
\pgfsys@transformshift{1.528508in}{1.752672in}%
\pgfsys@useobject{currentmarker}{}%
\end{pgfscope}%
\begin{pgfscope}%
\pgfsys@transformshift{1.535238in}{1.754310in}%
\pgfsys@useobject{currentmarker}{}%
\end{pgfscope}%
\begin{pgfscope}%
\pgfsys@transformshift{1.543862in}{1.755610in}%
\pgfsys@useobject{currentmarker}{}%
\end{pgfscope}%
\begin{pgfscope}%
\pgfsys@transformshift{1.554114in}{1.755192in}%
\pgfsys@useobject{currentmarker}{}%
\end{pgfscope}%
\begin{pgfscope}%
\pgfsys@transformshift{1.565263in}{1.755106in}%
\pgfsys@useobject{currentmarker}{}%
\end{pgfscope}%
\begin{pgfscope}%
\pgfsys@transformshift{1.571386in}{1.755448in}%
\pgfsys@useobject{currentmarker}{}%
\end{pgfscope}%
\begin{pgfscope}%
\pgfsys@transformshift{1.578115in}{1.756493in}%
\pgfsys@useobject{currentmarker}{}%
\end{pgfscope}%
\begin{pgfscope}%
\pgfsys@transformshift{1.585765in}{1.756342in}%
\pgfsys@useobject{currentmarker}{}%
\end{pgfscope}%
\begin{pgfscope}%
\pgfsys@transformshift{1.595139in}{1.756878in}%
\pgfsys@useobject{currentmarker}{}%
\end{pgfscope}%
\begin{pgfscope}%
\pgfsys@transformshift{1.600266in}{1.757496in}%
\pgfsys@useobject{currentmarker}{}%
\end{pgfscope}%
\begin{pgfscope}%
\pgfsys@transformshift{1.605781in}{1.758769in}%
\pgfsys@useobject{currentmarker}{}%
\end{pgfscope}%
\begin{pgfscope}%
\pgfsys@transformshift{1.613972in}{1.759211in}%
\pgfsys@useobject{currentmarker}{}%
\end{pgfscope}%
\begin{pgfscope}%
\pgfsys@transformshift{1.618457in}{1.759705in}%
\pgfsys@useobject{currentmarker}{}%
\end{pgfscope}%
\begin{pgfscope}%
\pgfsys@transformshift{1.624863in}{1.761466in}%
\pgfsys@useobject{currentmarker}{}%
\end{pgfscope}%
\begin{pgfscope}%
\pgfsys@transformshift{1.628475in}{1.762016in}%
\pgfsys@useobject{currentmarker}{}%
\end{pgfscope}%
\begin{pgfscope}%
\pgfsys@transformshift{1.630426in}{1.762502in}%
\pgfsys@useobject{currentmarker}{}%
\end{pgfscope}%
\begin{pgfscope}%
\pgfsys@transformshift{1.631531in}{1.762521in}%
\pgfsys@useobject{currentmarker}{}%
\end{pgfscope}%
\begin{pgfscope}%
\pgfsys@transformshift{1.633447in}{1.762499in}%
\pgfsys@useobject{currentmarker}{}%
\end{pgfscope}%
\begin{pgfscope}%
\pgfsys@transformshift{1.638440in}{1.763877in}%
\pgfsys@useobject{currentmarker}{}%
\end{pgfscope}%
\begin{pgfscope}%
\pgfsys@transformshift{1.641196in}{1.764594in}%
\pgfsys@useobject{currentmarker}{}%
\end{pgfscope}%
\begin{pgfscope}%
\pgfsys@transformshift{1.646507in}{1.764752in}%
\pgfsys@useobject{currentmarker}{}%
\end{pgfscope}%
\begin{pgfscope}%
\pgfsys@transformshift{1.649429in}{1.764817in}%
\pgfsys@useobject{currentmarker}{}%
\end{pgfscope}%
\begin{pgfscope}%
\pgfsys@transformshift{1.654427in}{1.764608in}%
\pgfsys@useobject{currentmarker}{}%
\end{pgfscope}%
\begin{pgfscope}%
\pgfsys@transformshift{1.660498in}{1.765934in}%
\pgfsys@useobject{currentmarker}{}%
\end{pgfscope}%
\begin{pgfscope}%
\pgfsys@transformshift{1.663849in}{1.766611in}%
\pgfsys@useobject{currentmarker}{}%
\end{pgfscope}%
\begin{pgfscope}%
\pgfsys@transformshift{1.665705in}{1.766907in}%
\pgfsys@useobject{currentmarker}{}%
\end{pgfscope}%
\begin{pgfscope}%
\pgfsys@transformshift{1.668443in}{1.766847in}%
\pgfsys@useobject{currentmarker}{}%
\end{pgfscope}%
\begin{pgfscope}%
\pgfsys@transformshift{1.673330in}{1.766894in}%
\pgfsys@useobject{currentmarker}{}%
\end{pgfscope}%
\begin{pgfscope}%
\pgfsys@transformshift{1.680236in}{1.768736in}%
\pgfsys@useobject{currentmarker}{}%
\end{pgfscope}%
\begin{pgfscope}%
\pgfsys@transformshift{1.688166in}{1.770650in}%
\pgfsys@useobject{currentmarker}{}%
\end{pgfscope}%
\begin{pgfscope}%
\pgfsys@transformshift{1.699391in}{1.771596in}%
\pgfsys@useobject{currentmarker}{}%
\end{pgfscope}%
\begin{pgfscope}%
\pgfsys@transformshift{1.711866in}{1.770704in}%
\pgfsys@useobject{currentmarker}{}%
\end{pgfscope}%
\begin{pgfscope}%
\pgfsys@transformshift{1.725583in}{1.769020in}%
\pgfsys@useobject{currentmarker}{}%
\end{pgfscope}%
\begin{pgfscope}%
\pgfsys@transformshift{1.733045in}{1.770468in}%
\pgfsys@useobject{currentmarker}{}%
\end{pgfscope}%
\begin{pgfscope}%
\pgfsys@transformshift{1.741156in}{1.772298in}%
\pgfsys@useobject{currentmarker}{}%
\end{pgfscope}%
\begin{pgfscope}%
\pgfsys@transformshift{1.745703in}{1.772780in}%
\pgfsys@useobject{currentmarker}{}%
\end{pgfscope}%
\begin{pgfscope}%
\pgfsys@transformshift{1.752276in}{1.772458in}%
\pgfsys@useobject{currentmarker}{}%
\end{pgfscope}%
\begin{pgfscope}%
\pgfsys@transformshift{1.762176in}{1.772576in}%
\pgfsys@useobject{currentmarker}{}%
\end{pgfscope}%
\begin{pgfscope}%
\pgfsys@transformshift{1.775073in}{1.773452in}%
\pgfsys@useobject{currentmarker}{}%
\end{pgfscope}%
\begin{pgfscope}%
\pgfsys@transformshift{1.789856in}{1.775293in}%
\pgfsys@useobject{currentmarker}{}%
\end{pgfscope}%
\begin{pgfscope}%
\pgfsys@transformshift{1.807360in}{1.776090in}%
\pgfsys@useobject{currentmarker}{}%
\end{pgfscope}%
\begin{pgfscope}%
\pgfsys@transformshift{1.825444in}{1.775924in}%
\pgfsys@useobject{currentmarker}{}%
\end{pgfscope}%
\begin{pgfscope}%
\pgfsys@transformshift{1.835333in}{1.774847in}%
\pgfsys@useobject{currentmarker}{}%
\end{pgfscope}%
\begin{pgfscope}%
\pgfsys@transformshift{1.847021in}{1.776560in}%
\pgfsys@useobject{currentmarker}{}%
\end{pgfscope}%
\begin{pgfscope}%
\pgfsys@transformshift{1.859556in}{1.777142in}%
\pgfsys@useobject{currentmarker}{}%
\end{pgfscope}%
\begin{pgfscope}%
\pgfsys@transformshift{1.866433in}{1.777727in}%
\pgfsys@useobject{currentmarker}{}%
\end{pgfscope}%
\begin{pgfscope}%
\pgfsys@transformshift{1.874046in}{1.778207in}%
\pgfsys@useobject{currentmarker}{}%
\end{pgfscope}%
\begin{pgfscope}%
\pgfsys@transformshift{1.883573in}{1.779584in}%
\pgfsys@useobject{currentmarker}{}%
\end{pgfscope}%
\begin{pgfscope}%
\pgfsys@transformshift{1.888822in}{1.780272in}%
\pgfsys@useobject{currentmarker}{}%
\end{pgfscope}%
\begin{pgfscope}%
\pgfsys@transformshift{1.894365in}{1.781884in}%
\pgfsys@useobject{currentmarker}{}%
\end{pgfscope}%
\begin{pgfscope}%
\pgfsys@transformshift{1.897420in}{1.782746in}%
\pgfsys@useobject{currentmarker}{}%
\end{pgfscope}%
\begin{pgfscope}%
\pgfsys@transformshift{1.899160in}{1.782894in}%
\pgfsys@useobject{currentmarker}{}%
\end{pgfscope}%
\begin{pgfscope}%
\pgfsys@transformshift{1.901883in}{1.783115in}%
\pgfsys@useobject{currentmarker}{}%
\end{pgfscope}%
\begin{pgfscope}%
\pgfsys@transformshift{1.906285in}{1.783269in}%
\pgfsys@useobject{currentmarker}{}%
\end{pgfscope}%
\begin{pgfscope}%
\pgfsys@transformshift{1.912338in}{1.785674in}%
\pgfsys@useobject{currentmarker}{}%
\end{pgfscope}%
\begin{pgfscope}%
\pgfsys@transformshift{1.919651in}{1.787949in}%
\pgfsys@useobject{currentmarker}{}%
\end{pgfscope}%
\begin{pgfscope}%
\pgfsys@transformshift{1.929298in}{1.789365in}%
\pgfsys@useobject{currentmarker}{}%
\end{pgfscope}%
\begin{pgfscope}%
\pgfsys@transformshift{1.940599in}{1.789929in}%
\pgfsys@useobject{currentmarker}{}%
\end{pgfscope}%
\begin{pgfscope}%
\pgfsys@transformshift{1.946822in}{1.789893in}%
\pgfsys@useobject{currentmarker}{}%
\end{pgfscope}%
\begin{pgfscope}%
\pgfsys@transformshift{1.954860in}{1.791551in}%
\pgfsys@useobject{currentmarker}{}%
\end{pgfscope}%
\begin{pgfscope}%
\pgfsys@transformshift{1.964038in}{1.790686in}%
\pgfsys@useobject{currentmarker}{}%
\end{pgfscope}%
\begin{pgfscope}%
\pgfsys@transformshift{1.974716in}{1.792046in}%
\pgfsys@useobject{currentmarker}{}%
\end{pgfscope}%
\begin{pgfscope}%
\pgfsys@transformshift{1.986536in}{1.791931in}%
\pgfsys@useobject{currentmarker}{}%
\end{pgfscope}%
\begin{pgfscope}%
\pgfsys@transformshift{1.999421in}{1.792172in}%
\pgfsys@useobject{currentmarker}{}%
\end{pgfscope}%
\begin{pgfscope}%
\pgfsys@transformshift{2.014151in}{1.793141in}%
\pgfsys@useobject{currentmarker}{}%
\end{pgfscope}%
\begin{pgfscope}%
\pgfsys@transformshift{2.029640in}{1.792043in}%
\pgfsys@useobject{currentmarker}{}%
\end{pgfscope}%
\begin{pgfscope}%
\pgfsys@transformshift{2.037865in}{1.794341in}%
\pgfsys@useobject{currentmarker}{}%
\end{pgfscope}%
\begin{pgfscope}%
\pgfsys@transformshift{2.042558in}{1.794152in}%
\pgfsys@useobject{currentmarker}{}%
\end{pgfscope}%
\begin{pgfscope}%
\pgfsys@transformshift{2.048195in}{1.795074in}%
\pgfsys@useobject{currentmarker}{}%
\end{pgfscope}%
\begin{pgfscope}%
\pgfsys@transformshift{2.051337in}{1.795078in}%
\pgfsys@useobject{currentmarker}{}%
\end{pgfscope}%
\begin{pgfscope}%
\pgfsys@transformshift{2.055329in}{1.795564in}%
\pgfsys@useobject{currentmarker}{}%
\end{pgfscope}%
\begin{pgfscope}%
\pgfsys@transformshift{2.057540in}{1.795628in}%
\pgfsys@useobject{currentmarker}{}%
\end{pgfscope}%
\begin{pgfscope}%
\pgfsys@transformshift{2.061377in}{1.796298in}%
\pgfsys@useobject{currentmarker}{}%
\end{pgfscope}%
\begin{pgfscope}%
\pgfsys@transformshift{2.063517in}{1.796195in}%
\pgfsys@useobject{currentmarker}{}%
\end{pgfscope}%
\begin{pgfscope}%
\pgfsys@transformshift{2.066712in}{1.796454in}%
\pgfsys@useobject{currentmarker}{}%
\end{pgfscope}%
\begin{pgfscope}%
\pgfsys@transformshift{2.068418in}{1.796009in}%
\pgfsys@useobject{currentmarker}{}%
\end{pgfscope}%
\begin{pgfscope}%
\pgfsys@transformshift{2.071241in}{1.795342in}%
\pgfsys@useobject{currentmarker}{}%
\end{pgfscope}%
\begin{pgfscope}%
\pgfsys@transformshift{2.074913in}{1.794143in}%
\pgfsys@useobject{currentmarker}{}%
\end{pgfscope}%
\begin{pgfscope}%
\pgfsys@transformshift{2.078561in}{1.791282in}%
\pgfsys@useobject{currentmarker}{}%
\end{pgfscope}%
\begin{pgfscope}%
\pgfsys@transformshift{2.081423in}{1.784795in}%
\pgfsys@useobject{currentmarker}{}%
\end{pgfscope}%
\begin{pgfscope}%
\pgfsys@transformshift{2.085270in}{1.777631in}%
\pgfsys@useobject{currentmarker}{}%
\end{pgfscope}%
\begin{pgfscope}%
\pgfsys@transformshift{2.085784in}{1.768299in}%
\pgfsys@useobject{currentmarker}{}%
\end{pgfscope}%
\begin{pgfscope}%
\pgfsys@transformshift{2.088544in}{1.758295in}%
\pgfsys@useobject{currentmarker}{}%
\end{pgfscope}%
\begin{pgfscope}%
\pgfsys@transformshift{2.091229in}{1.746974in}%
\pgfsys@useobject{currentmarker}{}%
\end{pgfscope}%
\begin{pgfscope}%
\pgfsys@transformshift{2.093036in}{1.734571in}%
\pgfsys@useobject{currentmarker}{}%
\end{pgfscope}%
\begin{pgfscope}%
\pgfsys@transformshift{2.095515in}{1.728139in}%
\pgfsys@useobject{currentmarker}{}%
\end{pgfscope}%
\begin{pgfscope}%
\pgfsys@transformshift{2.094625in}{1.724454in}%
\pgfsys@useobject{currentmarker}{}%
\end{pgfscope}%
\begin{pgfscope}%
\pgfsys@transformshift{2.095105in}{1.722425in}%
\pgfsys@useobject{currentmarker}{}%
\end{pgfscope}%
\begin{pgfscope}%
\pgfsys@transformshift{2.094622in}{1.719560in}%
\pgfsys@useobject{currentmarker}{}%
\end{pgfscope}%
\begin{pgfscope}%
\pgfsys@transformshift{2.094061in}{1.716144in}%
\pgfsys@useobject{currentmarker}{}%
\end{pgfscope}%
\begin{pgfscope}%
\pgfsys@transformshift{2.094995in}{1.710676in}%
\pgfsys@useobject{currentmarker}{}%
\end{pgfscope}%
\begin{pgfscope}%
\pgfsys@transformshift{2.095759in}{1.704551in}%
\pgfsys@useobject{currentmarker}{}%
\end{pgfscope}%
\begin{pgfscope}%
\pgfsys@transformshift{2.094207in}{1.696702in}%
\pgfsys@useobject{currentmarker}{}%
\end{pgfscope}%
\begin{pgfscope}%
\pgfsys@transformshift{2.093967in}{1.692308in}%
\pgfsys@useobject{currentmarker}{}%
\end{pgfscope}%
\begin{pgfscope}%
\pgfsys@transformshift{2.094548in}{1.684835in}%
\pgfsys@useobject{currentmarker}{}%
\end{pgfscope}%
\begin{pgfscope}%
\pgfsys@transformshift{2.095126in}{1.680754in}%
\pgfsys@useobject{currentmarker}{}%
\end{pgfscope}%
\begin{pgfscope}%
\pgfsys@transformshift{2.094237in}{1.675148in}%
\pgfsys@useobject{currentmarker}{}%
\end{pgfscope}%
\begin{pgfscope}%
\pgfsys@transformshift{2.093896in}{1.672045in}%
\pgfsys@useobject{currentmarker}{}%
\end{pgfscope}%
\begin{pgfscope}%
\pgfsys@transformshift{2.093562in}{1.670361in}%
\pgfsys@useobject{currentmarker}{}%
\end{pgfscope}%
\begin{pgfscope}%
\pgfsys@transformshift{2.093914in}{1.668214in}%
\pgfsys@useobject{currentmarker}{}%
\end{pgfscope}%
\begin{pgfscope}%
\pgfsys@transformshift{2.093700in}{1.667037in}%
\pgfsys@useobject{currentmarker}{}%
\end{pgfscope}%
\begin{pgfscope}%
\pgfsys@transformshift{2.093655in}{1.666380in}%
\pgfsys@useobject{currentmarker}{}%
\end{pgfscope}%
\begin{pgfscope}%
\pgfsys@transformshift{2.093598in}{1.666023in}%
\pgfsys@useobject{currentmarker}{}%
\end{pgfscope}%
\begin{pgfscope}%
\pgfsys@transformshift{2.093768in}{1.665214in}%
\pgfsys@useobject{currentmarker}{}%
\end{pgfscope}%
\begin{pgfscope}%
\pgfsys@transformshift{2.093745in}{1.664760in}%
\pgfsys@useobject{currentmarker}{}%
\end{pgfscope}%
\begin{pgfscope}%
\pgfsys@transformshift{2.093594in}{1.663394in}%
\pgfsys@useobject{currentmarker}{}%
\end{pgfscope}%
\begin{pgfscope}%
\pgfsys@transformshift{2.093562in}{1.662638in}%
\pgfsys@useobject{currentmarker}{}%
\end{pgfscope}%
\begin{pgfscope}%
\pgfsys@transformshift{2.093773in}{1.659568in}%
\pgfsys@useobject{currentmarker}{}%
\end{pgfscope}%
\begin{pgfscope}%
\pgfsys@transformshift{2.094620in}{1.656083in}%
\pgfsys@useobject{currentmarker}{}%
\end{pgfscope}%
\begin{pgfscope}%
\pgfsys@transformshift{2.093654in}{1.649932in}%
\pgfsys@useobject{currentmarker}{}%
\end{pgfscope}%
\begin{pgfscope}%
\pgfsys@transformshift{2.093042in}{1.642358in}%
\pgfsys@useobject{currentmarker}{}%
\end{pgfscope}%
\begin{pgfscope}%
\pgfsys@transformshift{2.091828in}{1.633539in}%
\pgfsys@useobject{currentmarker}{}%
\end{pgfscope}%
\begin{pgfscope}%
\pgfsys@transformshift{2.093146in}{1.628823in}%
\pgfsys@useobject{currentmarker}{}%
\end{pgfscope}%
\begin{pgfscope}%
\pgfsys@transformshift{2.092288in}{1.623391in}%
\pgfsys@useobject{currentmarker}{}%
\end{pgfscope}%
\begin{pgfscope}%
\pgfsys@transformshift{2.091420in}{1.617400in}%
\pgfsys@useobject{currentmarker}{}%
\end{pgfscope}%
\begin{pgfscope}%
\pgfsys@transformshift{2.090990in}{1.610180in}%
\pgfsys@useobject{currentmarker}{}%
\end{pgfscope}%
\begin{pgfscope}%
\pgfsys@transformshift{2.092159in}{1.606378in}%
\pgfsys@useobject{currentmarker}{}%
\end{pgfscope}%
\begin{pgfscope}%
\pgfsys@transformshift{2.091997in}{1.601634in}%
\pgfsys@useobject{currentmarker}{}%
\end{pgfscope}%
\begin{pgfscope}%
\pgfsys@transformshift{2.091580in}{1.596441in}%
\pgfsys@useobject{currentmarker}{}%
\end{pgfscope}%
\begin{pgfscope}%
\pgfsys@transformshift{2.091066in}{1.593622in}%
\pgfsys@useobject{currentmarker}{}%
\end{pgfscope}%
\begin{pgfscope}%
\pgfsys@transformshift{2.092270in}{1.588201in}%
\pgfsys@useobject{currentmarker}{}%
\end{pgfscope}%
\begin{pgfscope}%
\pgfsys@transformshift{2.092499in}{1.585155in}%
\pgfsys@useobject{currentmarker}{}%
\end{pgfscope}%
\begin{pgfscope}%
\pgfsys@transformshift{2.092758in}{1.583495in}%
\pgfsys@useobject{currentmarker}{}%
\end{pgfscope}%
\begin{pgfscope}%
\pgfsys@transformshift{2.092582in}{1.582588in}%
\pgfsys@useobject{currentmarker}{}%
\end{pgfscope}%
\begin{pgfscope}%
\pgfsys@transformshift{2.092587in}{1.582080in}%
\pgfsys@useobject{currentmarker}{}%
\end{pgfscope}%
\begin{pgfscope}%
\pgfsys@transformshift{2.092663in}{1.581811in}%
\pgfsys@useobject{currentmarker}{}%
\end{pgfscope}%
\begin{pgfscope}%
\pgfsys@transformshift{2.092676in}{1.581658in}%
\pgfsys@useobject{currentmarker}{}%
\end{pgfscope}%
\begin{pgfscope}%
\pgfsys@transformshift{2.092743in}{1.580431in}%
\pgfsys@useobject{currentmarker}{}%
\end{pgfscope}%
\begin{pgfscope}%
\pgfsys@transformshift{2.092664in}{1.579760in}%
\pgfsys@useobject{currentmarker}{}%
\end{pgfscope}%
\begin{pgfscope}%
\pgfsys@transformshift{2.093715in}{1.576057in}%
\pgfsys@useobject{currentmarker}{}%
\end{pgfscope}%
\begin{pgfscope}%
\pgfsys@transformshift{2.094336in}{1.574032in}%
\pgfsys@useobject{currentmarker}{}%
\end{pgfscope}%
\begin{pgfscope}%
\pgfsys@transformshift{2.094383in}{1.570429in}%
\pgfsys@useobject{currentmarker}{}%
\end{pgfscope}%
\begin{pgfscope}%
\pgfsys@transformshift{2.094454in}{1.568448in}%
\pgfsys@useobject{currentmarker}{}%
\end{pgfscope}%
\begin{pgfscope}%
\pgfsys@transformshift{2.094398in}{1.565321in}%
\pgfsys@useobject{currentmarker}{}%
\end{pgfscope}%
\begin{pgfscope}%
\pgfsys@transformshift{2.095278in}{1.561729in}%
\pgfsys@useobject{currentmarker}{}%
\end{pgfscope}%
\begin{pgfscope}%
\pgfsys@transformshift{2.096266in}{1.557446in}%
\pgfsys@useobject{currentmarker}{}%
\end{pgfscope}%
\begin{pgfscope}%
\pgfsys@transformshift{2.095628in}{1.552458in}%
\pgfsys@useobject{currentmarker}{}%
\end{pgfscope}%
\begin{pgfscope}%
\pgfsys@transformshift{2.095604in}{1.549692in}%
\pgfsys@useobject{currentmarker}{}%
\end{pgfscope}%
\begin{pgfscope}%
\pgfsys@transformshift{2.096372in}{1.545715in}%
\pgfsys@useobject{currentmarker}{}%
\end{pgfscope}%
\begin{pgfscope}%
\pgfsys@transformshift{2.096926in}{1.541090in}%
\pgfsys@useobject{currentmarker}{}%
\end{pgfscope}%
\begin{pgfscope}%
\pgfsys@transformshift{2.097451in}{1.534563in}%
\pgfsys@useobject{currentmarker}{}%
\end{pgfscope}%
\begin{pgfscope}%
\pgfsys@transformshift{2.097121in}{1.530976in}%
\pgfsys@useobject{currentmarker}{}%
\end{pgfscope}%
\begin{pgfscope}%
\pgfsys@transformshift{2.098367in}{1.525713in}%
\pgfsys@useobject{currentmarker}{}%
\end{pgfscope}%
\begin{pgfscope}%
\pgfsys@transformshift{2.099603in}{1.519701in}%
\pgfsys@useobject{currentmarker}{}%
\end{pgfscope}%
\begin{pgfscope}%
\pgfsys@transformshift{2.100885in}{1.512541in}%
\pgfsys@useobject{currentmarker}{}%
\end{pgfscope}%
\begin{pgfscope}%
\pgfsys@transformshift{2.100646in}{1.508547in}%
\pgfsys@useobject{currentmarker}{}%
\end{pgfscope}%
\begin{pgfscope}%
\pgfsys@transformshift{2.101460in}{1.503612in}%
\pgfsys@useobject{currentmarker}{}%
\end{pgfscope}%
\begin{pgfscope}%
\pgfsys@transformshift{2.102111in}{1.500939in}%
\pgfsys@useobject{currentmarker}{}%
\end{pgfscope}%
\begin{pgfscope}%
\pgfsys@transformshift{2.102279in}{1.499435in}%
\pgfsys@useobject{currentmarker}{}%
\end{pgfscope}%
\begin{pgfscope}%
\pgfsys@transformshift{2.102195in}{1.498607in}%
\pgfsys@useobject{currentmarker}{}%
\end{pgfscope}%
\begin{pgfscope}%
\pgfsys@transformshift{2.102167in}{1.498150in}%
\pgfsys@useobject{currentmarker}{}%
\end{pgfscope}%
\begin{pgfscope}%
\pgfsys@transformshift{2.102594in}{1.496291in}%
\pgfsys@useobject{currentmarker}{}%
\end{pgfscope}%
\begin{pgfscope}%
\pgfsys@transformshift{2.102890in}{1.493889in}%
\pgfsys@useobject{currentmarker}{}%
\end{pgfscope}%
\begin{pgfscope}%
\pgfsys@transformshift{2.102780in}{1.489904in}%
\pgfsys@useobject{currentmarker}{}%
\end{pgfscope}%
\begin{pgfscope}%
\pgfsys@transformshift{2.102440in}{1.485349in}%
\pgfsys@useobject{currentmarker}{}%
\end{pgfscope}%
\begin{pgfscope}%
\pgfsys@transformshift{2.104250in}{1.478347in}%
\pgfsys@useobject{currentmarker}{}%
\end{pgfscope}%
\begin{pgfscope}%
\pgfsys@transformshift{2.105975in}{1.470788in}%
\pgfsys@useobject{currentmarker}{}%
\end{pgfscope}%
\begin{pgfscope}%
\pgfsys@transformshift{2.106690in}{1.461474in}%
\pgfsys@useobject{currentmarker}{}%
\end{pgfscope}%
\begin{pgfscope}%
\pgfsys@transformshift{2.106380in}{1.456346in}%
\pgfsys@useobject{currentmarker}{}%
\end{pgfscope}%
\begin{pgfscope}%
\pgfsys@transformshift{2.106179in}{1.448854in}%
\pgfsys@useobject{currentmarker}{}%
\end{pgfscope}%
\begin{pgfscope}%
\pgfsys@transformshift{2.107101in}{1.440341in}%
\pgfsys@useobject{currentmarker}{}%
\end{pgfscope}%
\begin{pgfscope}%
\pgfsys@transformshift{2.110988in}{1.431556in}%
\pgfsys@useobject{currentmarker}{}%
\end{pgfscope}%
\begin{pgfscope}%
\pgfsys@transformshift{2.111105in}{1.420108in}%
\pgfsys@useobject{currentmarker}{}%
\end{pgfscope}%
\begin{pgfscope}%
\pgfsys@transformshift{2.111932in}{1.407934in}%
\pgfsys@useobject{currentmarker}{}%
\end{pgfscope}%
\begin{pgfscope}%
\pgfsys@transformshift{2.111934in}{1.401223in}%
\pgfsys@useobject{currentmarker}{}%
\end{pgfscope}%
\begin{pgfscope}%
\pgfsys@transformshift{2.112974in}{1.397681in}%
\pgfsys@useobject{currentmarker}{}%
\end{pgfscope}%
\begin{pgfscope}%
\pgfsys@transformshift{2.113335in}{1.395684in}%
\pgfsys@useobject{currentmarker}{}%
\end{pgfscope}%
\begin{pgfscope}%
\pgfsys@transformshift{2.113248in}{1.393032in}%
\pgfsys@useobject{currentmarker}{}%
\end{pgfscope}%
\begin{pgfscope}%
\pgfsys@transformshift{2.113184in}{1.391573in}%
\pgfsys@useobject{currentmarker}{}%
\end{pgfscope}%
\begin{pgfscope}%
\pgfsys@transformshift{2.114142in}{1.387513in}%
\pgfsys@useobject{currentmarker}{}%
\end{pgfscope}%
\begin{pgfscope}%
\pgfsys@transformshift{2.115146in}{1.382926in}%
\pgfsys@useobject{currentmarker}{}%
\end{pgfscope}%
\begin{pgfscope}%
\pgfsys@transformshift{2.115603in}{1.380385in}%
\pgfsys@useobject{currentmarker}{}%
\end{pgfscope}%
\begin{pgfscope}%
\pgfsys@transformshift{2.115411in}{1.378978in}%
\pgfsys@useobject{currentmarker}{}%
\end{pgfscope}%
\begin{pgfscope}%
\pgfsys@transformshift{2.115491in}{1.377069in}%
\pgfsys@useobject{currentmarker}{}%
\end{pgfscope}%
\begin{pgfscope}%
\pgfsys@transformshift{2.116117in}{1.374267in}%
\pgfsys@useobject{currentmarker}{}%
\end{pgfscope}%
\begin{pgfscope}%
\pgfsys@transformshift{2.116211in}{1.370630in}%
\pgfsys@useobject{currentmarker}{}%
\end{pgfscope}%
\begin{pgfscope}%
\pgfsys@transformshift{2.115997in}{1.366374in}%
\pgfsys@useobject{currentmarker}{}%
\end{pgfscope}%
\begin{pgfscope}%
\pgfsys@transformshift{2.115889in}{1.364032in}%
\pgfsys@useobject{currentmarker}{}%
\end{pgfscope}%
\begin{pgfscope}%
\pgfsys@transformshift{2.117136in}{1.358910in}%
\pgfsys@useobject{currentmarker}{}%
\end{pgfscope}%
\begin{pgfscope}%
\pgfsys@transformshift{2.119273in}{1.353386in}%
\pgfsys@useobject{currentmarker}{}%
\end{pgfscope}%
\begin{pgfscope}%
\pgfsys@transformshift{2.119172in}{1.345267in}%
\pgfsys@useobject{currentmarker}{}%
\end{pgfscope}%
\begin{pgfscope}%
\pgfsys@transformshift{2.119561in}{1.340818in}%
\pgfsys@useobject{currentmarker}{}%
\end{pgfscope}%
\begin{pgfscope}%
\pgfsys@transformshift{2.119329in}{1.335597in}%
\pgfsys@useobject{currentmarker}{}%
\end{pgfscope}%
\begin{pgfscope}%
\pgfsys@transformshift{2.121869in}{1.329489in}%
\pgfsys@useobject{currentmarker}{}%
\end{pgfscope}%
\begin{pgfscope}%
\pgfsys@transformshift{2.121574in}{1.322314in}%
\pgfsys@useobject{currentmarker}{}%
\end{pgfscope}%
\begin{pgfscope}%
\pgfsys@transformshift{2.121375in}{1.318370in}%
\pgfsys@useobject{currentmarker}{}%
\end{pgfscope}%
\begin{pgfscope}%
\pgfsys@transformshift{2.121039in}{1.316224in}%
\pgfsys@useobject{currentmarker}{}%
\end{pgfscope}%
\begin{pgfscope}%
\pgfsys@transformshift{2.122350in}{1.312448in}%
\pgfsys@useobject{currentmarker}{}%
\end{pgfscope}%
\begin{pgfscope}%
\pgfsys@transformshift{2.122391in}{1.310251in}%
\pgfsys@useobject{currentmarker}{}%
\end{pgfscope}%
\begin{pgfscope}%
\pgfsys@transformshift{2.122732in}{1.306753in}%
\pgfsys@useobject{currentmarker}{}%
\end{pgfscope}%
\begin{pgfscope}%
\pgfsys@transformshift{2.122741in}{1.304820in}%
\pgfsys@useobject{currentmarker}{}%
\end{pgfscope}%
\begin{pgfscope}%
\pgfsys@transformshift{2.124037in}{1.301942in}%
\pgfsys@useobject{currentmarker}{}%
\end{pgfscope}%
\begin{pgfscope}%
\pgfsys@transformshift{2.124435in}{1.300252in}%
\pgfsys@useobject{currentmarker}{}%
\end{pgfscope}%
\begin{pgfscope}%
\pgfsys@transformshift{2.125142in}{1.297783in}%
\pgfsys@useobject{currentmarker}{}%
\end{pgfscope}%
\begin{pgfscope}%
\pgfsys@transformshift{2.125190in}{1.296371in}%
\pgfsys@useobject{currentmarker}{}%
\end{pgfscope}%
\begin{pgfscope}%
\pgfsys@transformshift{2.126183in}{1.293603in}%
\pgfsys@useobject{currentmarker}{}%
\end{pgfscope}%
\begin{pgfscope}%
\pgfsys@transformshift{2.127566in}{1.290310in}%
\pgfsys@useobject{currentmarker}{}%
\end{pgfscope}%
\begin{pgfscope}%
\pgfsys@transformshift{2.127901in}{1.284196in}%
\pgfsys@useobject{currentmarker}{}%
\end{pgfscope}%
\begin{pgfscope}%
\pgfsys@transformshift{2.128040in}{1.280831in}%
\pgfsys@useobject{currentmarker}{}%
\end{pgfscope}%
\begin{pgfscope}%
\pgfsys@transformshift{2.129216in}{1.275626in}%
\pgfsys@useobject{currentmarker}{}%
\end{pgfscope}%
\begin{pgfscope}%
\pgfsys@transformshift{2.131156in}{1.269940in}%
\pgfsys@useobject{currentmarker}{}%
\end{pgfscope}%
\begin{pgfscope}%
\pgfsys@transformshift{2.131863in}{1.266712in}%
\pgfsys@useobject{currentmarker}{}%
\end{pgfscope}%
\begin{pgfscope}%
\pgfsys@transformshift{2.131969in}{1.264897in}%
\pgfsys@useobject{currentmarker}{}%
\end{pgfscope}%
\begin{pgfscope}%
\pgfsys@transformshift{2.132011in}{1.263899in}%
\pgfsys@useobject{currentmarker}{}%
\end{pgfscope}%
\begin{pgfscope}%
\pgfsys@transformshift{2.132496in}{1.262482in}%
\pgfsys@useobject{currentmarker}{}%
\end{pgfscope}%
\begin{pgfscope}%
\pgfsys@transformshift{2.132684in}{1.261680in}%
\pgfsys@useobject{currentmarker}{}%
\end{pgfscope}%
\begin{pgfscope}%
\pgfsys@transformshift{2.132782in}{1.261238in}%
\pgfsys@useobject{currentmarker}{}%
\end{pgfscope}%
\begin{pgfscope}%
\pgfsys@transformshift{2.132778in}{1.260318in}%
\pgfsys@useobject{currentmarker}{}%
\end{pgfscope}%
\begin{pgfscope}%
\pgfsys@transformshift{2.132884in}{1.259823in}%
\pgfsys@useobject{currentmarker}{}%
\end{pgfscope}%
\begin{pgfscope}%
\pgfsys@transformshift{2.132967in}{1.259557in}%
\pgfsys@useobject{currentmarker}{}%
\end{pgfscope}%
\begin{pgfscope}%
\pgfsys@transformshift{2.133334in}{1.258493in}%
\pgfsys@useobject{currentmarker}{}%
\end{pgfscope}%
\begin{pgfscope}%
\pgfsys@transformshift{2.133374in}{1.256905in}%
\pgfsys@useobject{currentmarker}{}%
\end{pgfscope}%
\begin{pgfscope}%
\pgfsys@transformshift{2.133433in}{1.256033in}%
\pgfsys@useobject{currentmarker}{}%
\end{pgfscope}%
\begin{pgfscope}%
\pgfsys@transformshift{2.134091in}{1.253566in}%
\pgfsys@useobject{currentmarker}{}%
\end{pgfscope}%
\begin{pgfscope}%
\pgfsys@transformshift{2.135218in}{1.250339in}%
\pgfsys@useobject{currentmarker}{}%
\end{pgfscope}%
\begin{pgfscope}%
\pgfsys@transformshift{2.135728in}{1.245956in}%
\pgfsys@useobject{currentmarker}{}%
\end{pgfscope}%
\begin{pgfscope}%
\pgfsys@transformshift{2.135673in}{1.243529in}%
\pgfsys@useobject{currentmarker}{}%
\end{pgfscope}%
\begin{pgfscope}%
\pgfsys@transformshift{2.135752in}{1.242197in}%
\pgfsys@useobject{currentmarker}{}%
\end{pgfscope}%
\begin{pgfscope}%
\pgfsys@transformshift{2.135934in}{1.241486in}%
\pgfsys@useobject{currentmarker}{}%
\end{pgfscope}%
\begin{pgfscope}%
\pgfsys@transformshift{2.136024in}{1.241092in}%
\pgfsys@useobject{currentmarker}{}%
\end{pgfscope}%
\begin{pgfscope}%
\pgfsys@transformshift{2.136267in}{1.239586in}%
\pgfsys@useobject{currentmarker}{}%
\end{pgfscope}%
\begin{pgfscope}%
\pgfsys@transformshift{2.136208in}{1.238748in}%
\pgfsys@useobject{currentmarker}{}%
\end{pgfscope}%
\begin{pgfscope}%
\pgfsys@transformshift{2.136580in}{1.236512in}%
\pgfsys@useobject{currentmarker}{}%
\end{pgfscope}%
\begin{pgfscope}%
\pgfsys@transformshift{2.136815in}{1.235287in}%
\pgfsys@useobject{currentmarker}{}%
\end{pgfscope}%
\begin{pgfscope}%
\pgfsys@transformshift{2.137021in}{1.234633in}%
\pgfsys@useobject{currentmarker}{}%
\end{pgfscope}%
\begin{pgfscope}%
\pgfsys@transformshift{2.137025in}{1.234256in}%
\pgfsys@useobject{currentmarker}{}%
\end{pgfscope}%
\begin{pgfscope}%
\pgfsys@transformshift{2.137039in}{1.234049in}%
\pgfsys@useobject{currentmarker}{}%
\end{pgfscope}%
\begin{pgfscope}%
\pgfsys@transformshift{2.137225in}{1.233081in}%
\pgfsys@useobject{currentmarker}{}%
\end{pgfscope}%
\begin{pgfscope}%
\pgfsys@transformshift{2.137407in}{1.232570in}%
\pgfsys@useobject{currentmarker}{}%
\end{pgfscope}%
\begin{pgfscope}%
\pgfsys@transformshift{2.137709in}{1.231062in}%
\pgfsys@useobject{currentmarker}{}%
\end{pgfscope}%
\begin{pgfscope}%
\pgfsys@transformshift{2.137764in}{1.228604in}%
\pgfsys@useobject{currentmarker}{}%
\end{pgfscope}%
\begin{pgfscope}%
\pgfsys@transformshift{2.137574in}{1.225515in}%
\pgfsys@useobject{currentmarker}{}%
\end{pgfscope}%
\begin{pgfscope}%
\pgfsys@transformshift{2.138111in}{1.223900in}%
\pgfsys@useobject{currentmarker}{}%
\end{pgfscope}%
\begin{pgfscope}%
\pgfsys@transformshift{2.138597in}{1.221300in}%
\pgfsys@useobject{currentmarker}{}%
\end{pgfscope}%
\begin{pgfscope}%
\pgfsys@transformshift{2.139399in}{1.217295in}%
\pgfsys@useobject{currentmarker}{}%
\end{pgfscope}%
\begin{pgfscope}%
\pgfsys@transformshift{2.138879in}{1.212452in}%
\pgfsys@useobject{currentmarker}{}%
\end{pgfscope}%
\begin{pgfscope}%
\pgfsys@transformshift{2.139957in}{1.206976in}%
\pgfsys@useobject{currentmarker}{}%
\end{pgfscope}%
\begin{pgfscope}%
\pgfsys@transformshift{2.141855in}{1.200335in}%
\pgfsys@useobject{currentmarker}{}%
\end{pgfscope}%
\begin{pgfscope}%
\pgfsys@transformshift{2.143109in}{1.192812in}%
\pgfsys@useobject{currentmarker}{}%
\end{pgfscope}%
\begin{pgfscope}%
\pgfsys@transformshift{2.143348in}{1.188624in}%
\pgfsys@useobject{currentmarker}{}%
\end{pgfscope}%
\begin{pgfscope}%
\pgfsys@transformshift{2.142774in}{1.186389in}%
\pgfsys@useobject{currentmarker}{}%
\end{pgfscope}%
\begin{pgfscope}%
\pgfsys@transformshift{2.142606in}{1.183552in}%
\pgfsys@useobject{currentmarker}{}%
\end{pgfscope}%
\begin{pgfscope}%
\pgfsys@transformshift{2.142658in}{1.181989in}%
\pgfsys@useobject{currentmarker}{}%
\end{pgfscope}%
\begin{pgfscope}%
\pgfsys@transformshift{2.142758in}{1.181135in}%
\pgfsys@useobject{currentmarker}{}%
\end{pgfscope}%
\begin{pgfscope}%
\pgfsys@transformshift{2.142459in}{1.179794in}%
\pgfsys@useobject{currentmarker}{}%
\end{pgfscope}%
\begin{pgfscope}%
\pgfsys@transformshift{2.142451in}{1.179039in}%
\pgfsys@useobject{currentmarker}{}%
\end{pgfscope}%
\begin{pgfscope}%
\pgfsys@transformshift{2.142622in}{1.177805in}%
\pgfsys@useobject{currentmarker}{}%
\end{pgfscope}%
\begin{pgfscope}%
\pgfsys@transformshift{2.143125in}{1.176001in}%
\pgfsys@useobject{currentmarker}{}%
\end{pgfscope}%
\begin{pgfscope}%
\pgfsys@transformshift{2.142628in}{1.170989in}%
\pgfsys@useobject{currentmarker}{}%
\end{pgfscope}%
\begin{pgfscope}%
\pgfsys@transformshift{2.142040in}{1.164364in}%
\pgfsys@useobject{currentmarker}{}%
\end{pgfscope}%
\begin{pgfscope}%
\pgfsys@transformshift{2.141551in}{1.160738in}%
\pgfsys@useobject{currentmarker}{}%
\end{pgfscope}%
\begin{pgfscope}%
\pgfsys@transformshift{2.142694in}{1.155914in}%
\pgfsys@useobject{currentmarker}{}%
\end{pgfscope}%
\begin{pgfscope}%
\pgfsys@transformshift{2.143459in}{1.153297in}%
\pgfsys@useobject{currentmarker}{}%
\end{pgfscope}%
\begin{pgfscope}%
\pgfsys@transformshift{2.143338in}{1.148478in}%
\pgfsys@useobject{currentmarker}{}%
\end{pgfscope}%
\begin{pgfscope}%
\pgfsys@transformshift{2.143159in}{1.142746in}%
\pgfsys@useobject{currentmarker}{}%
\end{pgfscope}%
\begin{pgfscope}%
\pgfsys@transformshift{2.142055in}{1.136636in}%
\pgfsys@useobject{currentmarker}{}%
\end{pgfscope}%
\begin{pgfscope}%
\pgfsys@transformshift{2.143887in}{1.128397in}%
\pgfsys@useobject{currentmarker}{}%
\end{pgfscope}%
\begin{pgfscope}%
\pgfsys@transformshift{2.145661in}{1.118603in}%
\pgfsys@useobject{currentmarker}{}%
\end{pgfscope}%
\begin{pgfscope}%
\pgfsys@transformshift{2.146788in}{1.108142in}%
\pgfsys@useobject{currentmarker}{}%
\end{pgfscope}%
\begin{pgfscope}%
\pgfsys@transformshift{2.145799in}{1.102440in}%
\pgfsys@useobject{currentmarker}{}%
\end{pgfscope}%
\begin{pgfscope}%
\pgfsys@transformshift{2.145538in}{1.099268in}%
\pgfsys@useobject{currentmarker}{}%
\end{pgfscope}%
\begin{pgfscope}%
\pgfsys@transformshift{2.146035in}{1.094829in}%
\pgfsys@useobject{currentmarker}{}%
\end{pgfscope}%
\begin{pgfscope}%
\pgfsys@transformshift{2.147769in}{1.090148in}%
\pgfsys@useobject{currentmarker}{}%
\end{pgfscope}%
\begin{pgfscope}%
\pgfsys@transformshift{2.146504in}{1.083011in}%
\pgfsys@useobject{currentmarker}{}%
\end{pgfscope}%
\begin{pgfscope}%
\pgfsys@transformshift{2.146502in}{1.074923in}%
\pgfsys@useobject{currentmarker}{}%
\end{pgfscope}%
\begin{pgfscope}%
\pgfsys@transformshift{2.145868in}{1.070521in}%
\pgfsys@useobject{currentmarker}{}%
\end{pgfscope}%
\begin{pgfscope}%
\pgfsys@transformshift{2.147072in}{1.065319in}%
\pgfsys@useobject{currentmarker}{}%
\end{pgfscope}%
\begin{pgfscope}%
\pgfsys@transformshift{2.147456in}{1.059520in}%
\pgfsys@useobject{currentmarker}{}%
\end{pgfscope}%
\begin{pgfscope}%
\pgfsys@transformshift{2.147239in}{1.052970in}%
\pgfsys@useobject{currentmarker}{}%
\end{pgfscope}%
\begin{pgfscope}%
\pgfsys@transformshift{2.146878in}{1.049383in}%
\pgfsys@useobject{currentmarker}{}%
\end{pgfscope}%
\begin{pgfscope}%
\pgfsys@transformshift{2.146451in}{1.047447in}%
\pgfsys@useobject{currentmarker}{}%
\end{pgfscope}%
\begin{pgfscope}%
\pgfsys@transformshift{2.147118in}{1.043946in}%
\pgfsys@useobject{currentmarker}{}%
\end{pgfscope}%
\begin{pgfscope}%
\pgfsys@transformshift{2.147307in}{1.041995in}%
\pgfsys@useobject{currentmarker}{}%
\end{pgfscope}%
\begin{pgfscope}%
\pgfsys@transformshift{2.146840in}{1.038018in}%
\pgfsys@useobject{currentmarker}{}%
\end{pgfscope}%
\begin{pgfscope}%
\pgfsys@transformshift{2.146711in}{1.035819in}%
\pgfsys@useobject{currentmarker}{}%
\end{pgfscope}%
\begin{pgfscope}%
\pgfsys@transformshift{2.146427in}{1.034641in}%
\pgfsys@useobject{currentmarker}{}%
\end{pgfscope}%
\begin{pgfscope}%
\pgfsys@transformshift{2.146996in}{1.031656in}%
\pgfsys@useobject{currentmarker}{}%
\end{pgfscope}%
\begin{pgfscope}%
\pgfsys@transformshift{2.147838in}{1.028047in}%
\pgfsys@useobject{currentmarker}{}%
\end{pgfscope}%
\begin{pgfscope}%
\pgfsys@transformshift{2.147118in}{1.023933in}%
\pgfsys@useobject{currentmarker}{}%
\end{pgfscope}%
\begin{pgfscope}%
\pgfsys@transformshift{2.147076in}{1.019042in}%
\pgfsys@useobject{currentmarker}{}%
\end{pgfscope}%
\begin{pgfscope}%
\pgfsys@transformshift{2.146997in}{1.013075in}%
\pgfsys@useobject{currentmarker}{}%
\end{pgfscope}%
\begin{pgfscope}%
\pgfsys@transformshift{2.149152in}{1.005900in}%
\pgfsys@useobject{currentmarker}{}%
\end{pgfscope}%
\begin{pgfscope}%
\pgfsys@transformshift{2.151398in}{0.998102in}%
\pgfsys@useobject{currentmarker}{}%
\end{pgfscope}%
\begin{pgfscope}%
\pgfsys@transformshift{2.150746in}{0.989286in}%
\pgfsys@useobject{currentmarker}{}%
\end{pgfscope}%
\begin{pgfscope}%
\pgfsys@transformshift{2.150556in}{0.984428in}%
\pgfsys@useobject{currentmarker}{}%
\end{pgfscope}%
\begin{pgfscope}%
\pgfsys@transformshift{2.150906in}{0.978375in}%
\pgfsys@useobject{currentmarker}{}%
\end{pgfscope}%
\begin{pgfscope}%
\pgfsys@transformshift{2.151414in}{0.971242in}%
\pgfsys@useobject{currentmarker}{}%
\end{pgfscope}%
\begin{pgfscope}%
\pgfsys@transformshift{2.152492in}{0.967460in}%
\pgfsys@useobject{currentmarker}{}%
\end{pgfscope}%
\begin{pgfscope}%
\pgfsys@transformshift{2.152282in}{0.965307in}%
\pgfsys@useobject{currentmarker}{}%
\end{pgfscope}%
\begin{pgfscope}%
\pgfsys@transformshift{2.152358in}{0.964120in}%
\pgfsys@useobject{currentmarker}{}%
\end{pgfscope}%
\begin{pgfscope}%
\pgfsys@transformshift{2.152174in}{0.963492in}%
\pgfsys@useobject{currentmarker}{}%
\end{pgfscope}%
\begin{pgfscope}%
\pgfsys@transformshift{2.152013in}{0.963170in}%
\pgfsys@useobject{currentmarker}{}%
\end{pgfscope}%
\begin{pgfscope}%
\pgfsys@transformshift{2.150841in}{0.962433in}%
\pgfsys@useobject{currentmarker}{}%
\end{pgfscope}%
\begin{pgfscope}%
\pgfsys@transformshift{2.148183in}{0.961783in}%
\pgfsys@useobject{currentmarker}{}%
\end{pgfscope}%
\begin{pgfscope}%
\pgfsys@transformshift{2.146687in}{0.961615in}%
\pgfsys@useobject{currentmarker}{}%
\end{pgfscope}%
\begin{pgfscope}%
\pgfsys@transformshift{2.143862in}{0.961192in}%
\pgfsys@useobject{currentmarker}{}%
\end{pgfscope}%
\begin{pgfscope}%
\pgfsys@transformshift{2.140506in}{0.961339in}%
\pgfsys@useobject{currentmarker}{}%
\end{pgfscope}%
\begin{pgfscope}%
\pgfsys@transformshift{2.136430in}{0.960977in}%
\pgfsys@useobject{currentmarker}{}%
\end{pgfscope}%
\begin{pgfscope}%
\pgfsys@transformshift{2.129977in}{0.960680in}%
\pgfsys@useobject{currentmarker}{}%
\end{pgfscope}%
\begin{pgfscope}%
\pgfsys@transformshift{2.123157in}{0.959504in}%
\pgfsys@useobject{currentmarker}{}%
\end{pgfscope}%
\begin{pgfscope}%
\pgfsys@transformshift{2.115236in}{0.960350in}%
\pgfsys@useobject{currentmarker}{}%
\end{pgfscope}%
\begin{pgfscope}%
\pgfsys@transformshift{2.110877in}{0.959914in}%
\pgfsys@useobject{currentmarker}{}%
\end{pgfscope}%
\begin{pgfscope}%
\pgfsys@transformshift{2.108482in}{0.960177in}%
\pgfsys@useobject{currentmarker}{}%
\end{pgfscope}%
\begin{pgfscope}%
\pgfsys@transformshift{2.105368in}{0.959840in}%
\pgfsys@useobject{currentmarker}{}%
\end{pgfscope}%
\begin{pgfscope}%
\pgfsys@transformshift{2.103648in}{0.959943in}%
\pgfsys@useobject{currentmarker}{}%
\end{pgfscope}%
\begin{pgfscope}%
\pgfsys@transformshift{2.102701in}{0.959979in}%
\pgfsys@useobject{currentmarker}{}%
\end{pgfscope}%
\begin{pgfscope}%
\pgfsys@transformshift{2.102181in}{0.959993in}%
\pgfsys@useobject{currentmarker}{}%
\end{pgfscope}%
\begin{pgfscope}%
\pgfsys@transformshift{2.100372in}{0.959968in}%
\pgfsys@useobject{currentmarker}{}%
\end{pgfscope}%
\begin{pgfscope}%
\pgfsys@transformshift{2.099378in}{0.960017in}%
\pgfsys@useobject{currentmarker}{}%
\end{pgfscope}%
\begin{pgfscope}%
\pgfsys@transformshift{2.093643in}{0.959481in}%
\pgfsys@useobject{currentmarker}{}%
\end{pgfscope}%
\begin{pgfscope}%
\pgfsys@transformshift{2.086939in}{0.958776in}%
\pgfsys@useobject{currentmarker}{}%
\end{pgfscope}%
\begin{pgfscope}%
\pgfsys@transformshift{2.078728in}{0.957584in}%
\pgfsys@useobject{currentmarker}{}%
\end{pgfscope}%
\begin{pgfscope}%
\pgfsys@transformshift{2.074165in}{0.957627in}%
\pgfsys@useobject{currentmarker}{}%
\end{pgfscope}%
\begin{pgfscope}%
\pgfsys@transformshift{2.066158in}{0.958466in}%
\pgfsys@useobject{currentmarker}{}%
\end{pgfscope}%
\begin{pgfscope}%
\pgfsys@transformshift{2.056389in}{0.957985in}%
\pgfsys@useobject{currentmarker}{}%
\end{pgfscope}%
\begin{pgfscope}%
\pgfsys@transformshift{2.043086in}{0.956679in}%
\pgfsys@useobject{currentmarker}{}%
\end{pgfscope}%
\begin{pgfscope}%
\pgfsys@transformshift{2.027734in}{0.954220in}%
\pgfsys@useobject{currentmarker}{}%
\end{pgfscope}%
\begin{pgfscope}%
\pgfsys@transformshift{2.011668in}{0.953088in}%
\pgfsys@useobject{currentmarker}{}%
\end{pgfscope}%
\begin{pgfscope}%
\pgfsys@transformshift{1.990926in}{0.954314in}%
\pgfsys@useobject{currentmarker}{}%
\end{pgfscope}%
\begin{pgfscope}%
\pgfsys@transformshift{1.968785in}{0.954505in}%
\pgfsys@useobject{currentmarker}{}%
\end{pgfscope}%
\begin{pgfscope}%
\pgfsys@transformshift{1.943682in}{0.953720in}%
\pgfsys@useobject{currentmarker}{}%
\end{pgfscope}%
\begin{pgfscope}%
\pgfsys@transformshift{1.917203in}{0.948297in}%
\pgfsys@useobject{currentmarker}{}%
\end{pgfscope}%
\begin{pgfscope}%
\pgfsys@transformshift{1.889640in}{0.946826in}%
\pgfsys@useobject{currentmarker}{}%
\end{pgfscope}%
\begin{pgfscope}%
\pgfsys@transformshift{1.858215in}{0.950451in}%
\pgfsys@useobject{currentmarker}{}%
\end{pgfscope}%
\begin{pgfscope}%
\pgfsys@transformshift{1.826024in}{0.949524in}%
\pgfsys@useobject{currentmarker}{}%
\end{pgfscope}%
\begin{pgfscope}%
\pgfsys@transformshift{1.791998in}{0.944869in}%
\pgfsys@useobject{currentmarker}{}%
\end{pgfscope}%
\begin{pgfscope}%
\pgfsys@transformshift{1.755932in}{0.941612in}%
\pgfsys@useobject{currentmarker}{}%
\end{pgfscope}%
\begin{pgfscope}%
\pgfsys@transformshift{1.718974in}{0.940898in}%
\pgfsys@useobject{currentmarker}{}%
\end{pgfscope}%
\begin{pgfscope}%
\pgfsys@transformshift{1.677336in}{0.938182in}%
\pgfsys@useobject{currentmarker}{}%
\end{pgfscope}%
\begin{pgfscope}%
\pgfsys@transformshift{1.635186in}{0.934564in}%
\pgfsys@useobject{currentmarker}{}%
\end{pgfscope}%
\begin{pgfscope}%
\pgfsys@transformshift{1.589457in}{0.930190in}%
\pgfsys@useobject{currentmarker}{}%
\end{pgfscope}%
\begin{pgfscope}%
\pgfsys@transformshift{1.542641in}{0.925899in}%
\pgfsys@useobject{currentmarker}{}%
\end{pgfscope}%
\begin{pgfscope}%
\pgfsys@transformshift{1.492977in}{0.927226in}%
\pgfsys@useobject{currentmarker}{}%
\end{pgfscope}%
\begin{pgfscope}%
\pgfsys@transformshift{1.439513in}{0.927372in}%
\pgfsys@useobject{currentmarker}{}%
\end{pgfscope}%
\begin{pgfscope}%
\pgfsys@transformshift{1.410315in}{0.923890in}%
\pgfsys@useobject{currentmarker}{}%
\end{pgfscope}%
\begin{pgfscope}%
\pgfsys@transformshift{1.376490in}{0.921871in}%
\pgfsys@useobject{currentmarker}{}%
\end{pgfscope}%
\begin{pgfscope}%
\pgfsys@transformshift{1.341958in}{0.919188in}%
\pgfsys@useobject{currentmarker}{}%
\end{pgfscope}%
\begin{pgfscope}%
\pgfsys@transformshift{1.322940in}{0.920305in}%
\pgfsys@useobject{currentmarker}{}%
\end{pgfscope}%
\begin{pgfscope}%
\pgfsys@transformshift{1.303042in}{0.920276in}%
\pgfsys@useobject{currentmarker}{}%
\end{pgfscope}%
\begin{pgfscope}%
\pgfsys@transformshift{1.292243in}{0.922052in}%
\pgfsys@useobject{currentmarker}{}%
\end{pgfscope}%
\begin{pgfscope}%
\pgfsys@transformshift{1.286989in}{0.924989in}%
\pgfsys@useobject{currentmarker}{}%
\end{pgfscope}%
\begin{pgfscope}%
\pgfsys@transformshift{1.284291in}{0.926909in}%
\pgfsys@useobject{currentmarker}{}%
\end{pgfscope}%
\begin{pgfscope}%
\pgfsys@transformshift{1.283816in}{0.930987in}%
\pgfsys@useobject{currentmarker}{}%
\end{pgfscope}%
\begin{pgfscope}%
\pgfsys@transformshift{1.284118in}{0.937965in}%
\pgfsys@useobject{currentmarker}{}%
\end{pgfscope}%
\begin{pgfscope}%
\pgfsys@transformshift{1.286027in}{0.946600in}%
\pgfsys@useobject{currentmarker}{}%
\end{pgfscope}%
\begin{pgfscope}%
\pgfsys@transformshift{1.287595in}{0.956269in}%
\pgfsys@useobject{currentmarker}{}%
\end{pgfscope}%
\begin{pgfscope}%
\pgfsys@transformshift{1.289927in}{0.967336in}%
\pgfsys@useobject{currentmarker}{}%
\end{pgfscope}%
\begin{pgfscope}%
\pgfsys@transformshift{1.291334in}{0.979481in}%
\pgfsys@useobject{currentmarker}{}%
\end{pgfscope}%
\begin{pgfscope}%
\pgfsys@transformshift{1.293536in}{0.992794in}%
\pgfsys@useobject{currentmarker}{}%
\end{pgfscope}%
\begin{pgfscope}%
\pgfsys@transformshift{1.293682in}{1.000214in}%
\pgfsys@useobject{currentmarker}{}%
\end{pgfscope}%
\begin{pgfscope}%
\pgfsys@transformshift{1.294342in}{1.004242in}%
\pgfsys@useobject{currentmarker}{}%
\end{pgfscope}%
\begin{pgfscope}%
\pgfsys@transformshift{1.292874in}{1.010289in}%
\pgfsys@useobject{currentmarker}{}%
\end{pgfscope}%
\begin{pgfscope}%
\pgfsys@transformshift{1.292415in}{1.013681in}%
\pgfsys@useobject{currentmarker}{}%
\end{pgfscope}%
\begin{pgfscope}%
\pgfsys@transformshift{1.293407in}{1.017695in}%
\pgfsys@useobject{currentmarker}{}%
\end{pgfscope}%
\begin{pgfscope}%
\pgfsys@transformshift{1.290635in}{1.026164in}%
\pgfsys@useobject{currentmarker}{}%
\end{pgfscope}%
\begin{pgfscope}%
\pgfsys@transformshift{1.290765in}{1.031063in}%
\pgfsys@useobject{currentmarker}{}%
\end{pgfscope}%
\begin{pgfscope}%
\pgfsys@transformshift{1.292605in}{1.038051in}%
\pgfsys@useobject{currentmarker}{}%
\end{pgfscope}%
\begin{pgfscope}%
\pgfsys@transformshift{1.289487in}{1.049034in}%
\pgfsys@useobject{currentmarker}{}%
\end{pgfscope}%
\begin{pgfscope}%
\pgfsys@transformshift{1.286550in}{1.060965in}%
\pgfsys@useobject{currentmarker}{}%
\end{pgfscope}%
\begin{pgfscope}%
\pgfsys@transformshift{1.287847in}{1.078244in}%
\pgfsys@useobject{currentmarker}{}%
\end{pgfscope}%
\begin{pgfscope}%
\pgfsys@transformshift{1.284918in}{1.096074in}%
\pgfsys@useobject{currentmarker}{}%
\end{pgfscope}%
\begin{pgfscope}%
\pgfsys@transformshift{1.275048in}{1.114913in}%
\pgfsys@useobject{currentmarker}{}%
\end{pgfscope}%
\begin{pgfscope}%
\pgfsys@transformshift{1.274651in}{1.136730in}%
\pgfsys@useobject{currentmarker}{}%
\end{pgfscope}%
\begin{pgfscope}%
\pgfsys@transformshift{1.279124in}{1.158994in}%
\pgfsys@useobject{currentmarker}{}%
\end{pgfscope}%
\begin{pgfscope}%
\pgfsys@transformshift{1.270614in}{1.185160in}%
\pgfsys@useobject{currentmarker}{}%
\end{pgfscope}%
\begin{pgfscope}%
\pgfsys@transformshift{1.268523in}{1.213322in}%
\pgfsys@useobject{currentmarker}{}%
\end{pgfscope}%
\begin{pgfscope}%
\pgfsys@transformshift{1.269027in}{1.245518in}%
\pgfsys@useobject{currentmarker}{}%
\end{pgfscope}%
\begin{pgfscope}%
\pgfsys@transformshift{1.268216in}{1.278653in}%
\pgfsys@useobject{currentmarker}{}%
\end{pgfscope}%
\begin{pgfscope}%
\pgfsys@transformshift{1.255914in}{1.310088in}%
\pgfsys@useobject{currentmarker}{}%
\end{pgfscope}%
\begin{pgfscope}%
\pgfsys@transformshift{1.255277in}{1.328643in}%
\pgfsys@useobject{currentmarker}{}%
\end{pgfscope}%
\begin{pgfscope}%
\pgfsys@transformshift{1.258083in}{1.350170in}%
\pgfsys@useobject{currentmarker}{}%
\end{pgfscope}%
\begin{pgfscope}%
\pgfsys@transformshift{1.254817in}{1.372269in}%
\pgfsys@useobject{currentmarker}{}%
\end{pgfscope}%
\begin{pgfscope}%
\pgfsys@transformshift{1.250544in}{1.383789in}%
\pgfsys@useobject{currentmarker}{}%
\end{pgfscope}%
\begin{pgfscope}%
\pgfsys@transformshift{1.249510in}{1.390467in}%
\pgfsys@useobject{currentmarker}{}%
\end{pgfscope}%
\begin{pgfscope}%
\pgfsys@transformshift{1.251921in}{1.399796in}%
\pgfsys@useobject{currentmarker}{}%
\end{pgfscope}%
\begin{pgfscope}%
\pgfsys@transformshift{1.249555in}{1.410343in}%
\pgfsys@useobject{currentmarker}{}%
\end{pgfscope}%
\begin{pgfscope}%
\pgfsys@transformshift{1.248298in}{1.416154in}%
\pgfsys@useobject{currentmarker}{}%
\end{pgfscope}%
\begin{pgfscope}%
\pgfsys@transformshift{1.246849in}{1.423273in}%
\pgfsys@useobject{currentmarker}{}%
\end{pgfscope}%
\begin{pgfscope}%
\pgfsys@transformshift{1.249483in}{1.431380in}%
\pgfsys@useobject{currentmarker}{}%
\end{pgfscope}%
\begin{pgfscope}%
\pgfsys@transformshift{1.247074in}{1.443862in}%
\pgfsys@useobject{currentmarker}{}%
\end{pgfscope}%
\begin{pgfscope}%
\pgfsys@transformshift{1.247047in}{1.450854in}%
\pgfsys@useobject{currentmarker}{}%
\end{pgfscope}%
\begin{pgfscope}%
\pgfsys@transformshift{1.246354in}{1.461022in}%
\pgfsys@useobject{currentmarker}{}%
\end{pgfscope}%
\begin{pgfscope}%
\pgfsys@transformshift{1.250954in}{1.471831in}%
\pgfsys@useobject{currentmarker}{}%
\end{pgfscope}%
\begin{pgfscope}%
\pgfsys@transformshift{1.248709in}{1.487108in}%
\pgfsys@useobject{currentmarker}{}%
\end{pgfscope}%
\begin{pgfscope}%
\pgfsys@transformshift{1.247841in}{1.495556in}%
\pgfsys@useobject{currentmarker}{}%
\end{pgfscope}%
\begin{pgfscope}%
\pgfsys@transformshift{1.246270in}{1.505385in}%
\pgfsys@useobject{currentmarker}{}%
\end{pgfscope}%
\begin{pgfscope}%
\pgfsys@transformshift{1.249507in}{1.515966in}%
\pgfsys@useobject{currentmarker}{}%
\end{pgfscope}%
\begin{pgfscope}%
\pgfsys@transformshift{1.245949in}{1.531290in}%
\pgfsys@useobject{currentmarker}{}%
\end{pgfscope}%
\begin{pgfscope}%
\pgfsys@transformshift{1.246586in}{1.547269in}%
\pgfsys@useobject{currentmarker}{}%
\end{pgfscope}%
\begin{pgfscope}%
\pgfsys@transformshift{1.247285in}{1.566375in}%
\pgfsys@useobject{currentmarker}{}%
\end{pgfscope}%
\begin{pgfscope}%
\pgfsys@transformshift{1.244723in}{1.576574in}%
\pgfsys@useobject{currentmarker}{}%
\end{pgfscope}%
\begin{pgfscope}%
\pgfsys@transformshift{1.248752in}{1.590644in}%
\pgfsys@useobject{currentmarker}{}%
\end{pgfscope}%
\begin{pgfscope}%
\pgfsys@transformshift{1.251118in}{1.605783in}%
\pgfsys@useobject{currentmarker}{}%
\end{pgfscope}%
\begin{pgfscope}%
\pgfsys@transformshift{1.256968in}{1.623299in}%
\pgfsys@useobject{currentmarker}{}%
\end{pgfscope}%
\begin{pgfscope}%
\pgfsys@transformshift{1.251925in}{1.642101in}%
\pgfsys@useobject{currentmarker}{}%
\end{pgfscope}%
\begin{pgfscope}%
\pgfsys@transformshift{1.259428in}{1.664630in}%
\pgfsys@useobject{currentmarker}{}%
\end{pgfscope}%
\begin{pgfscope}%
\pgfsys@transformshift{1.262153in}{1.677403in}%
\pgfsys@useobject{currentmarker}{}%
\end{pgfscope}%
\begin{pgfscope}%
\pgfsys@transformshift{1.262960in}{1.693597in}%
\pgfsys@useobject{currentmarker}{}%
\end{pgfscope}%
\begin{pgfscope}%
\pgfsys@transformshift{1.261713in}{1.702428in}%
\pgfsys@useobject{currentmarker}{}%
\end{pgfscope}%
\begin{pgfscope}%
\pgfsys@transformshift{1.265590in}{1.715308in}%
\pgfsys@useobject{currentmarker}{}%
\end{pgfscope}%
\begin{pgfscope}%
\pgfsys@transformshift{1.266161in}{1.722684in}%
\pgfsys@useobject{currentmarker}{}%
\end{pgfscope}%
\begin{pgfscope}%
\pgfsys@transformshift{1.265807in}{1.733055in}%
\pgfsys@useobject{currentmarker}{}%
\end{pgfscope}%
\begin{pgfscope}%
\pgfsys@transformshift{1.264410in}{1.744092in}%
\pgfsys@useobject{currentmarker}{}%
\end{pgfscope}%
\begin{pgfscope}%
\pgfsys@transformshift{1.267744in}{1.758156in}%
\pgfsys@useobject{currentmarker}{}%
\end{pgfscope}%
\begin{pgfscope}%
\pgfsys@transformshift{1.268229in}{1.766091in}%
\pgfsys@useobject{currentmarker}{}%
\end{pgfscope}%
\begin{pgfscope}%
\pgfsys@transformshift{1.270479in}{1.776475in}%
\pgfsys@useobject{currentmarker}{}%
\end{pgfscope}%
\begin{pgfscope}%
\pgfsys@transformshift{1.269037in}{1.782139in}%
\pgfsys@useobject{currentmarker}{}%
\end{pgfscope}%
\begin{pgfscope}%
\pgfsys@transformshift{1.271304in}{1.792021in}%
\pgfsys@useobject{currentmarker}{}%
\end{pgfscope}%
\begin{pgfscope}%
\pgfsys@transformshift{1.274388in}{1.802865in}%
\pgfsys@useobject{currentmarker}{}%
\end{pgfscope}%
\begin{pgfscope}%
\pgfsys@transformshift{1.277192in}{1.814827in}%
\pgfsys@useobject{currentmarker}{}%
\end{pgfscope}%
\begin{pgfscope}%
\pgfsys@transformshift{1.273899in}{1.828563in}%
\pgfsys@useobject{currentmarker}{}%
\end{pgfscope}%
\begin{pgfscope}%
\pgfsys@transformshift{1.278308in}{1.844657in}%
\pgfsys@useobject{currentmarker}{}%
\end{pgfscope}%
\begin{pgfscope}%
\pgfsys@transformshift{1.284044in}{1.861793in}%
\pgfsys@useobject{currentmarker}{}%
\end{pgfscope}%
\begin{pgfscope}%
\pgfsys@transformshift{1.284902in}{1.871695in}%
\pgfsys@useobject{currentmarker}{}%
\end{pgfscope}%
\begin{pgfscope}%
\pgfsys@transformshift{1.282531in}{1.883327in}%
\pgfsys@useobject{currentmarker}{}%
\end{pgfscope}%
\begin{pgfscope}%
\pgfsys@transformshift{1.282975in}{1.889842in}%
\pgfsys@useobject{currentmarker}{}%
\end{pgfscope}%
\begin{pgfscope}%
\pgfsys@transformshift{1.284931in}{1.896955in}%
\pgfsys@useobject{currentmarker}{}%
\end{pgfscope}%
\begin{pgfscope}%
\pgfsys@transformshift{1.284807in}{1.901011in}%
\pgfsys@useobject{currentmarker}{}%
\end{pgfscope}%
\begin{pgfscope}%
\pgfsys@transformshift{1.285073in}{1.908181in}%
\pgfsys@useobject{currentmarker}{}%
\end{pgfscope}%
\begin{pgfscope}%
\pgfsys@transformshift{1.284058in}{1.915788in}%
\pgfsys@useobject{currentmarker}{}%
\end{pgfscope}%
\begin{pgfscope}%
\pgfsys@transformshift{1.287048in}{1.926678in}%
\pgfsys@useobject{currentmarker}{}%
\end{pgfscope}%
\begin{pgfscope}%
\pgfsys@transformshift{1.288425in}{1.932735in}%
\pgfsys@useobject{currentmarker}{}%
\end{pgfscope}%
\begin{pgfscope}%
\pgfsys@transformshift{1.289396in}{1.941280in}%
\pgfsys@useobject{currentmarker}{}%
\end{pgfscope}%
\begin{pgfscope}%
\pgfsys@transformshift{1.288624in}{1.945947in}%
\pgfsys@useobject{currentmarker}{}%
\end{pgfscope}%
\begin{pgfscope}%
\pgfsys@transformshift{1.291901in}{1.954471in}%
\pgfsys@useobject{currentmarker}{}%
\end{pgfscope}%
\begin{pgfscope}%
\pgfsys@transformshift{1.293982in}{1.963866in}%
\pgfsys@useobject{currentmarker}{}%
\end{pgfscope}%
\begin{pgfscope}%
\pgfsys@transformshift{1.295561in}{1.975793in}%
\pgfsys@useobject{currentmarker}{}%
\end{pgfscope}%
\begin{pgfscope}%
\pgfsys@transformshift{1.293204in}{1.988271in}%
\pgfsys@useobject{currentmarker}{}%
\end{pgfscope}%
\begin{pgfscope}%
\pgfsys@transformshift{1.297894in}{2.004335in}%
\pgfsys@useobject{currentmarker}{}%
\end{pgfscope}%
\begin{pgfscope}%
\pgfsys@transformshift{1.301502in}{2.021359in}%
\pgfsys@useobject{currentmarker}{}%
\end{pgfscope}%
\begin{pgfscope}%
\pgfsys@transformshift{1.307878in}{2.038488in}%
\pgfsys@useobject{currentmarker}{}%
\end{pgfscope}%
\begin{pgfscope}%
\pgfsys@transformshift{1.303849in}{2.058167in}%
\pgfsys@useobject{currentmarker}{}%
\end{pgfscope}%
\begin{pgfscope}%
\pgfsys@transformshift{1.309957in}{2.080792in}%
\pgfsys@useobject{currentmarker}{}%
\end{pgfscope}%
\begin{pgfscope}%
\pgfsys@transformshift{1.316504in}{2.103824in}%
\pgfsys@useobject{currentmarker}{}%
\end{pgfscope}%
\begin{pgfscope}%
\pgfsys@transformshift{1.317509in}{2.116956in}%
\pgfsys@useobject{currentmarker}{}%
\end{pgfscope}%
\begin{pgfscope}%
\pgfsys@transformshift{1.314785in}{2.123667in}%
\pgfsys@useobject{currentmarker}{}%
\end{pgfscope}%
\begin{pgfscope}%
\pgfsys@transformshift{1.308017in}{2.128453in}%
\pgfsys@useobject{currentmarker}{}%
\end{pgfscope}%
\begin{pgfscope}%
\pgfsys@transformshift{1.299294in}{2.132248in}%
\pgfsys@useobject{currentmarker}{}%
\end{pgfscope}%
\begin{pgfscope}%
\pgfsys@transformshift{1.289417in}{2.133739in}%
\pgfsys@useobject{currentmarker}{}%
\end{pgfscope}%
\begin{pgfscope}%
\pgfsys@transformshift{1.278410in}{2.133472in}%
\pgfsys@useobject{currentmarker}{}%
\end{pgfscope}%
\begin{pgfscope}%
\pgfsys@transformshift{1.266809in}{2.133110in}%
\pgfsys@useobject{currentmarker}{}%
\end{pgfscope}%
\begin{pgfscope}%
\pgfsys@transformshift{1.254504in}{2.131728in}%
\pgfsys@useobject{currentmarker}{}%
\end{pgfscope}%
\begin{pgfscope}%
\pgfsys@transformshift{1.247076in}{2.130810in}%
\pgfsys@useobject{currentmarker}{}%
\end{pgfscope}%
\begin{pgfscope}%
\pgfsys@transformshift{1.260836in}{2.129035in}%
\pgfsys@useobject{currentmarker}{}%
\end{pgfscope}%
\begin{pgfscope}%
\pgfsys@transformshift{1.268437in}{2.128371in}%
\pgfsys@useobject{currentmarker}{}%
\end{pgfscope}%
\begin{pgfscope}%
\pgfsys@transformshift{1.272599in}{2.127828in}%
\pgfsys@useobject{currentmarker}{}%
\end{pgfscope}%
\begin{pgfscope}%
\pgfsys@transformshift{1.277323in}{2.127670in}%
\pgfsys@useobject{currentmarker}{}%
\end{pgfscope}%
\begin{pgfscope}%
\pgfsys@transformshift{1.283477in}{2.126909in}%
\pgfsys@useobject{currentmarker}{}%
\end{pgfscope}%
\begin{pgfscope}%
\pgfsys@transformshift{1.291843in}{2.128373in}%
\pgfsys@useobject{currentmarker}{}%
\end{pgfscope}%
\begin{pgfscope}%
\pgfsys@transformshift{1.301670in}{2.128680in}%
\pgfsys@useobject{currentmarker}{}%
\end{pgfscope}%
\begin{pgfscope}%
\pgfsys@transformshift{1.314753in}{2.127444in}%
\pgfsys@useobject{currentmarker}{}%
\end{pgfscope}%
\begin{pgfscope}%
\pgfsys@transformshift{1.328450in}{2.125584in}%
\pgfsys@useobject{currentmarker}{}%
\end{pgfscope}%
\begin{pgfscope}%
\pgfsys@transformshift{1.336027in}{2.124965in}%
\pgfsys@useobject{currentmarker}{}%
\end{pgfscope}%
\begin{pgfscope}%
\pgfsys@transformshift{1.340199in}{2.124699in}%
\pgfsys@useobject{currentmarker}{}%
\end{pgfscope}%
\begin{pgfscope}%
\pgfsys@transformshift{1.346215in}{2.123696in}%
\pgfsys@useobject{currentmarker}{}%
\end{pgfscope}%
\begin{pgfscope}%
\pgfsys@transformshift{1.349570in}{2.123687in}%
\pgfsys@useobject{currentmarker}{}%
\end{pgfscope}%
\begin{pgfscope}%
\pgfsys@transformshift{1.354924in}{2.122337in}%
\pgfsys@useobject{currentmarker}{}%
\end{pgfscope}%
\begin{pgfscope}%
\pgfsys@transformshift{1.357960in}{2.122362in}%
\pgfsys@useobject{currentmarker}{}%
\end{pgfscope}%
\begin{pgfscope}%
\pgfsys@transformshift{1.363396in}{2.121598in}%
\pgfsys@useobject{currentmarker}{}%
\end{pgfscope}%
\begin{pgfscope}%
\pgfsys@transformshift{1.369462in}{2.121669in}%
\pgfsys@useobject{currentmarker}{}%
\end{pgfscope}%
\begin{pgfscope}%
\pgfsys@transformshift{1.378627in}{2.121501in}%
\pgfsys@useobject{currentmarker}{}%
\end{pgfscope}%
\begin{pgfscope}%
\pgfsys@transformshift{1.388334in}{2.120660in}%
\pgfsys@useobject{currentmarker}{}%
\end{pgfscope}%
\begin{pgfscope}%
\pgfsys@transformshift{1.393692in}{2.120517in}%
\pgfsys@useobject{currentmarker}{}%
\end{pgfscope}%
\begin{pgfscope}%
\pgfsys@transformshift{1.396612in}{2.120117in}%
\pgfsys@useobject{currentmarker}{}%
\end{pgfscope}%
\begin{pgfscope}%
\pgfsys@transformshift{1.400829in}{2.121023in}%
\pgfsys@useobject{currentmarker}{}%
\end{pgfscope}%
\begin{pgfscope}%
\pgfsys@transformshift{1.403191in}{2.120805in}%
\pgfsys@useobject{currentmarker}{}%
\end{pgfscope}%
\begin{pgfscope}%
\pgfsys@transformshift{1.406077in}{2.120962in}%
\pgfsys@useobject{currentmarker}{}%
\end{pgfscope}%
\begin{pgfscope}%
\pgfsys@transformshift{1.409416in}{2.120660in}%
\pgfsys@useobject{currentmarker}{}%
\end{pgfscope}%
\begin{pgfscope}%
\pgfsys@transformshift{1.414374in}{2.122216in}%
\pgfsys@useobject{currentmarker}{}%
\end{pgfscope}%
\begin{pgfscope}%
\pgfsys@transformshift{1.420098in}{2.121550in}%
\pgfsys@useobject{currentmarker}{}%
\end{pgfscope}%
\begin{pgfscope}%
\pgfsys@transformshift{1.427691in}{2.121295in}%
\pgfsys@useobject{currentmarker}{}%
\end{pgfscope}%
\begin{pgfscope}%
\pgfsys@transformshift{1.436820in}{2.119135in}%
\pgfsys@useobject{currentmarker}{}%
\end{pgfscope}%
\begin{pgfscope}%
\pgfsys@transformshift{1.441979in}{2.119102in}%
\pgfsys@useobject{currentmarker}{}%
\end{pgfscope}%
\begin{pgfscope}%
\pgfsys@transformshift{1.448803in}{2.118757in}%
\pgfsys@useobject{currentmarker}{}%
\end{pgfscope}%
\begin{pgfscope}%
\pgfsys@transformshift{1.456917in}{2.119806in}%
\pgfsys@useobject{currentmarker}{}%
\end{pgfscope}%
\begin{pgfscope}%
\pgfsys@transformshift{1.467193in}{2.120759in}%
\pgfsys@useobject{currentmarker}{}%
\end{pgfscope}%
\begin{pgfscope}%
\pgfsys@transformshift{1.478314in}{2.120988in}%
\pgfsys@useobject{currentmarker}{}%
\end{pgfscope}%
\begin{pgfscope}%
\pgfsys@transformshift{1.493718in}{2.119968in}%
\pgfsys@useobject{currentmarker}{}%
\end{pgfscope}%
\begin{pgfscope}%
\pgfsys@transformshift{1.510087in}{2.119348in}%
\pgfsys@useobject{currentmarker}{}%
\end{pgfscope}%
\begin{pgfscope}%
\pgfsys@transformshift{1.528763in}{2.119374in}%
\pgfsys@useobject{currentmarker}{}%
\end{pgfscope}%
\begin{pgfscope}%
\pgfsys@transformshift{1.547657in}{2.115991in}%
\pgfsys@useobject{currentmarker}{}%
\end{pgfscope}%
\begin{pgfscope}%
\pgfsys@transformshift{1.567756in}{2.115956in}%
\pgfsys@useobject{currentmarker}{}%
\end{pgfscope}%
\begin{pgfscope}%
\pgfsys@transformshift{1.592195in}{2.119506in}%
\pgfsys@useobject{currentmarker}{}%
\end{pgfscope}%
\begin{pgfscope}%
\pgfsys@transformshift{1.617706in}{2.118499in}%
\pgfsys@useobject{currentmarker}{}%
\end{pgfscope}%
\begin{pgfscope}%
\pgfsys@transformshift{1.644096in}{2.123379in}%
\pgfsys@useobject{currentmarker}{}%
\end{pgfscope}%
\begin{pgfscope}%
\pgfsys@transformshift{1.672882in}{2.123610in}%
\pgfsys@useobject{currentmarker}{}%
\end{pgfscope}%
\begin{pgfscope}%
\pgfsys@transformshift{1.702158in}{2.126993in}%
\pgfsys@useobject{currentmarker}{}%
\end{pgfscope}%
\begin{pgfscope}%
\pgfsys@transformshift{1.734346in}{2.127234in}%
\pgfsys@useobject{currentmarker}{}%
\end{pgfscope}%
\begin{pgfscope}%
\pgfsys@transformshift{1.767056in}{2.129031in}%
\pgfsys@useobject{currentmarker}{}%
\end{pgfscope}%
\begin{pgfscope}%
\pgfsys@transformshift{1.801723in}{2.125039in}%
\pgfsys@useobject{currentmarker}{}%
\end{pgfscope}%
\begin{pgfscope}%
\pgfsys@transformshift{1.820906in}{2.125692in}%
\pgfsys@useobject{currentmarker}{}%
\end{pgfscope}%
\begin{pgfscope}%
\pgfsys@transformshift{1.843172in}{2.125522in}%
\pgfsys@useobject{currentmarker}{}%
\end{pgfscope}%
\begin{pgfscope}%
\pgfsys@transformshift{1.865531in}{2.130409in}%
\pgfsys@useobject{currentmarker}{}%
\end{pgfscope}%
\begin{pgfscope}%
\pgfsys@transformshift{1.889884in}{2.133050in}%
\pgfsys@useobject{currentmarker}{}%
\end{pgfscope}%
\begin{pgfscope}%
\pgfsys@transformshift{1.903339in}{2.132360in}%
\pgfsys@useobject{currentmarker}{}%
\end{pgfscope}%
\begin{pgfscope}%
\pgfsys@transformshift{1.910734in}{2.131897in}%
\pgfsys@useobject{currentmarker}{}%
\end{pgfscope}%
\begin{pgfscope}%
\pgfsys@transformshift{1.920403in}{2.131859in}%
\pgfsys@useobject{currentmarker}{}%
\end{pgfscope}%
\begin{pgfscope}%
\pgfsys@transformshift{1.930922in}{2.130351in}%
\pgfsys@useobject{currentmarker}{}%
\end{pgfscope}%
\begin{pgfscope}%
\pgfsys@transformshift{1.942026in}{2.131100in}%
\pgfsys@useobject{currentmarker}{}%
\end{pgfscope}%
\begin{pgfscope}%
\pgfsys@transformshift{1.953907in}{2.129285in}%
\pgfsys@useobject{currentmarker}{}%
\end{pgfscope}%
\begin{pgfscope}%
\pgfsys@transformshift{1.966214in}{2.133675in}%
\pgfsys@useobject{currentmarker}{}%
\end{pgfscope}%
\begin{pgfscope}%
\pgfsys@transformshift{1.973401in}{2.133650in}%
\pgfsys@useobject{currentmarker}{}%
\end{pgfscope}%
\begin{pgfscope}%
\pgfsys@transformshift{1.983493in}{2.133035in}%
\pgfsys@useobject{currentmarker}{}%
\end{pgfscope}%
\begin{pgfscope}%
\pgfsys@transformshift{1.988916in}{2.131807in}%
\pgfsys@useobject{currentmarker}{}%
\end{pgfscope}%
\begin{pgfscope}%
\pgfsys@transformshift{1.996144in}{2.132838in}%
\pgfsys@useobject{currentmarker}{}%
\end{pgfscope}%
\begin{pgfscope}%
\pgfsys@transformshift{2.004246in}{2.132296in}%
\pgfsys@useobject{currentmarker}{}%
\end{pgfscope}%
\begin{pgfscope}%
\pgfsys@transformshift{2.016273in}{2.133231in}%
\pgfsys@useobject{currentmarker}{}%
\end{pgfscope}%
\begin{pgfscope}%
\pgfsys@transformshift{2.028987in}{2.132507in}%
\pgfsys@useobject{currentmarker}{}%
\end{pgfscope}%
\begin{pgfscope}%
\pgfsys@transformshift{2.043715in}{2.135498in}%
\pgfsys@useobject{currentmarker}{}%
\end{pgfscope}%
\begin{pgfscope}%
\pgfsys@transformshift{2.060069in}{2.134358in}%
\pgfsys@useobject{currentmarker}{}%
\end{pgfscope}%
\begin{pgfscope}%
\pgfsys@transformshift{2.077178in}{2.135341in}%
\pgfsys@useobject{currentmarker}{}%
\end{pgfscope}%
\begin{pgfscope}%
\pgfsys@transformshift{2.096607in}{2.135337in}%
\pgfsys@useobject{currentmarker}{}%
\end{pgfscope}%
\begin{pgfscope}%
\pgfsys@transformshift{2.119600in}{2.135003in}%
\pgfsys@useobject{currentmarker}{}%
\end{pgfscope}%
\begin{pgfscope}%
\pgfsys@transformshift{2.143775in}{2.136577in}%
\pgfsys@useobject{currentmarker}{}%
\end{pgfscope}%
\begin{pgfscope}%
\pgfsys@transformshift{2.170038in}{2.137350in}%
\pgfsys@useobject{currentmarker}{}%
\end{pgfscope}%
\begin{pgfscope}%
\pgfsys@transformshift{2.199431in}{2.139715in}%
\pgfsys@useobject{currentmarker}{}%
\end{pgfscope}%
\begin{pgfscope}%
\pgfsys@transformshift{2.228988in}{2.146374in}%
\pgfsys@useobject{currentmarker}{}%
\end{pgfscope}%
\begin{pgfscope}%
\pgfsys@transformshift{2.245487in}{2.148711in}%
\pgfsys@useobject{currentmarker}{}%
\end{pgfscope}%
\begin{pgfscope}%
\pgfsys@transformshift{2.263306in}{2.149524in}%
\pgfsys@useobject{currentmarker}{}%
\end{pgfscope}%
\begin{pgfscope}%
\pgfsys@transformshift{2.273075in}{2.150430in}%
\pgfsys@useobject{currentmarker}{}%
\end{pgfscope}%
\begin{pgfscope}%
\pgfsys@transformshift{2.285163in}{2.152714in}%
\pgfsys@useobject{currentmarker}{}%
\end{pgfscope}%
\begin{pgfscope}%
\pgfsys@transformshift{2.298264in}{2.153761in}%
\pgfsys@useobject{currentmarker}{}%
\end{pgfscope}%
\begin{pgfscope}%
\pgfsys@transformshift{2.312668in}{2.153192in}%
\pgfsys@useobject{currentmarker}{}%
\end{pgfscope}%
\begin{pgfscope}%
\pgfsys@transformshift{2.320596in}{2.153159in}%
\pgfsys@useobject{currentmarker}{}%
\end{pgfscope}%
\begin{pgfscope}%
\pgfsys@transformshift{2.330603in}{2.154898in}%
\pgfsys@useobject{currentmarker}{}%
\end{pgfscope}%
\begin{pgfscope}%
\pgfsys@transformshift{2.336179in}{2.155229in}%
\pgfsys@useobject{currentmarker}{}%
\end{pgfscope}%
\begin{pgfscope}%
\pgfsys@transformshift{2.339250in}{2.155306in}%
\pgfsys@useobject{currentmarker}{}%
\end{pgfscope}%
\begin{pgfscope}%
\pgfsys@transformshift{2.342804in}{2.155255in}%
\pgfsys@useobject{currentmarker}{}%
\end{pgfscope}%
\begin{pgfscope}%
\pgfsys@transformshift{2.347177in}{2.156414in}%
\pgfsys@useobject{currentmarker}{}%
\end{pgfscope}%
\begin{pgfscope}%
\pgfsys@transformshift{2.349660in}{2.156578in}%
\pgfsys@useobject{currentmarker}{}%
\end{pgfscope}%
\begin{pgfscope}%
\pgfsys@transformshift{2.351028in}{2.156565in}%
\pgfsys@useobject{currentmarker}{}%
\end{pgfscope}%
\begin{pgfscope}%
\pgfsys@transformshift{2.351781in}{2.156567in}%
\pgfsys@useobject{currentmarker}{}%
\end{pgfscope}%
\begin{pgfscope}%
\pgfsys@transformshift{2.353876in}{2.157198in}%
\pgfsys@useobject{currentmarker}{}%
\end{pgfscope}%
\begin{pgfscope}%
\pgfsys@transformshift{2.356970in}{2.157516in}%
\pgfsys@useobject{currentmarker}{}%
\end{pgfscope}%
\begin{pgfscope}%
\pgfsys@transformshift{2.362043in}{2.157508in}%
\pgfsys@useobject{currentmarker}{}%
\end{pgfscope}%
\begin{pgfscope}%
\pgfsys@transformshift{2.364829in}{2.157660in}%
\pgfsys@useobject{currentmarker}{}%
\end{pgfscope}%
\begin{pgfscope}%
\pgfsys@transformshift{2.369958in}{2.160389in}%
\pgfsys@useobject{currentmarker}{}%
\end{pgfscope}%
\begin{pgfscope}%
\pgfsys@transformshift{2.373144in}{2.160639in}%
\pgfsys@useobject{currentmarker}{}%
\end{pgfscope}%
\begin{pgfscope}%
\pgfsys@transformshift{2.378656in}{2.160648in}%
\pgfsys@useobject{currentmarker}{}%
\end{pgfscope}%
\begin{pgfscope}%
\pgfsys@transformshift{2.385036in}{2.159945in}%
\pgfsys@useobject{currentmarker}{}%
\end{pgfscope}%
\begin{pgfscope}%
\pgfsys@transformshift{2.388517in}{2.160531in}%
\pgfsys@useobject{currentmarker}{}%
\end{pgfscope}%
\begin{pgfscope}%
\pgfsys@transformshift{2.393267in}{2.160478in}%
\pgfsys@useobject{currentmarker}{}%
\end{pgfscope}%
\begin{pgfscope}%
\pgfsys@transformshift{2.398460in}{2.161350in}%
\pgfsys@useobject{currentmarker}{}%
\end{pgfscope}%
\begin{pgfscope}%
\pgfsys@transformshift{2.405054in}{2.162751in}%
\pgfsys@useobject{currentmarker}{}%
\end{pgfscope}%
\begin{pgfscope}%
\pgfsys@transformshift{2.412831in}{2.163524in}%
\pgfsys@useobject{currentmarker}{}%
\end{pgfscope}%
\begin{pgfscope}%
\pgfsys@transformshift{2.422117in}{2.164080in}%
\pgfsys@useobject{currentmarker}{}%
\end{pgfscope}%
\begin{pgfscope}%
\pgfsys@transformshift{2.427201in}{2.164660in}%
\pgfsys@useobject{currentmarker}{}%
\end{pgfscope}%
\begin{pgfscope}%
\pgfsys@transformshift{2.433102in}{2.165289in}%
\pgfsys@useobject{currentmarker}{}%
\end{pgfscope}%
\begin{pgfscope}%
\pgfsys@transformshift{2.436366in}{2.165368in}%
\pgfsys@useobject{currentmarker}{}%
\end{pgfscope}%
\begin{pgfscope}%
\pgfsys@transformshift{2.438121in}{2.165749in}%
\pgfsys@useobject{currentmarker}{}%
\end{pgfscope}%
\begin{pgfscope}%
\pgfsys@transformshift{2.440384in}{2.166116in}%
\pgfsys@useobject{currentmarker}{}%
\end{pgfscope}%
\begin{pgfscope}%
\pgfsys@transformshift{2.443392in}{2.166985in}%
\pgfsys@useobject{currentmarker}{}%
\end{pgfscope}%
\begin{pgfscope}%
\pgfsys@transformshift{2.448258in}{2.167286in}%
\pgfsys@useobject{currentmarker}{}%
\end{pgfscope}%
\begin{pgfscope}%
\pgfsys@transformshift{2.450937in}{2.167383in}%
\pgfsys@useobject{currentmarker}{}%
\end{pgfscope}%
\begin{pgfscope}%
\pgfsys@transformshift{2.454454in}{2.167041in}%
\pgfsys@useobject{currentmarker}{}%
\end{pgfscope}%
\begin{pgfscope}%
\pgfsys@transformshift{2.458618in}{2.166684in}%
\pgfsys@useobject{currentmarker}{}%
\end{pgfscope}%
\begin{pgfscope}%
\pgfsys@transformshift{2.460811in}{2.165995in}%
\pgfsys@useobject{currentmarker}{}%
\end{pgfscope}%
\begin{pgfscope}%
\pgfsys@transformshift{2.461928in}{2.165403in}%
\pgfsys@useobject{currentmarker}{}%
\end{pgfscope}%
\begin{pgfscope}%
\pgfsys@transformshift{2.463104in}{2.163254in}%
\pgfsys@useobject{currentmarker}{}%
\end{pgfscope}%
\begin{pgfscope}%
\pgfsys@transformshift{2.463962in}{2.162216in}%
\pgfsys@useobject{currentmarker}{}%
\end{pgfscope}%
\begin{pgfscope}%
\pgfsys@transformshift{2.464564in}{2.159807in}%
\pgfsys@useobject{currentmarker}{}%
\end{pgfscope}%
\begin{pgfscope}%
\pgfsys@transformshift{2.466283in}{2.156821in}%
\pgfsys@useobject{currentmarker}{}%
\end{pgfscope}%
\begin{pgfscope}%
\pgfsys@transformshift{2.466195in}{2.154928in}%
\pgfsys@useobject{currentmarker}{}%
\end{pgfscope}%
\begin{pgfscope}%
\pgfsys@transformshift{2.466840in}{2.151957in}%
\pgfsys@useobject{currentmarker}{}%
\end{pgfscope}%
\begin{pgfscope}%
\pgfsys@transformshift{2.467437in}{2.148295in}%
\pgfsys@useobject{currentmarker}{}%
\end{pgfscope}%
\begin{pgfscope}%
\pgfsys@transformshift{2.467952in}{2.146321in}%
\pgfsys@useobject{currentmarker}{}%
\end{pgfscope}%
\begin{pgfscope}%
\pgfsys@transformshift{2.468486in}{2.143163in}%
\pgfsys@useobject{currentmarker}{}%
\end{pgfscope}%
\begin{pgfscope}%
\pgfsys@transformshift{2.469151in}{2.139365in}%
\pgfsys@useobject{currentmarker}{}%
\end{pgfscope}%
\begin{pgfscope}%
\pgfsys@transformshift{2.469285in}{2.137249in}%
\pgfsys@useobject{currentmarker}{}%
\end{pgfscope}%
\begin{pgfscope}%
\pgfsys@transformshift{2.469300in}{2.136082in}%
\pgfsys@useobject{currentmarker}{}%
\end{pgfscope}%
\begin{pgfscope}%
\pgfsys@transformshift{2.470140in}{2.133777in}%
\pgfsys@useobject{currentmarker}{}%
\end{pgfscope}%
\begin{pgfscope}%
\pgfsys@transformshift{2.469929in}{2.130225in}%
\pgfsys@useobject{currentmarker}{}%
\end{pgfscope}%
\begin{pgfscope}%
\pgfsys@transformshift{2.470285in}{2.126128in}%
\pgfsys@useobject{currentmarker}{}%
\end{pgfscope}%
\begin{pgfscope}%
\pgfsys@transformshift{2.472569in}{2.119963in}%
\pgfsys@useobject{currentmarker}{}%
\end{pgfscope}%
\begin{pgfscope}%
\pgfsys@transformshift{2.472446in}{2.112684in}%
\pgfsys@useobject{currentmarker}{}%
\end{pgfscope}%
\begin{pgfscope}%
\pgfsys@transformshift{2.471759in}{2.108740in}%
\pgfsys@useobject{currentmarker}{}%
\end{pgfscope}%
\begin{pgfscope}%
\pgfsys@transformshift{2.473506in}{2.102347in}%
\pgfsys@useobject{currentmarker}{}%
\end{pgfscope}%
\begin{pgfscope}%
\pgfsys@transformshift{2.473465in}{2.098702in}%
\pgfsys@useobject{currentmarker}{}%
\end{pgfscope}%
\begin{pgfscope}%
\pgfsys@transformshift{2.472738in}{2.094027in}%
\pgfsys@useobject{currentmarker}{}%
\end{pgfscope}%
\begin{pgfscope}%
\pgfsys@transformshift{2.474731in}{2.087101in}%
\pgfsys@useobject{currentmarker}{}%
\end{pgfscope}%
\begin{pgfscope}%
\pgfsys@transformshift{2.476924in}{2.079160in}%
\pgfsys@useobject{currentmarker}{}%
\end{pgfscope}%
\begin{pgfscope}%
\pgfsys@transformshift{2.476818in}{2.067038in}%
\pgfsys@useobject{currentmarker}{}%
\end{pgfscope}%
\begin{pgfscope}%
\pgfsys@transformshift{2.479933in}{2.054672in}%
\pgfsys@useobject{currentmarker}{}%
\end{pgfscope}%
\begin{pgfscope}%
\pgfsys@transformshift{2.483863in}{2.041586in}%
\pgfsys@useobject{currentmarker}{}%
\end{pgfscope}%
\begin{pgfscope}%
\pgfsys@transformshift{2.486623in}{2.026154in}%
\pgfsys@useobject{currentmarker}{}%
\end{pgfscope}%
\begin{pgfscope}%
\pgfsys@transformshift{2.486872in}{2.017535in}%
\pgfsys@useobject{currentmarker}{}%
\end{pgfscope}%
\begin{pgfscope}%
\pgfsys@transformshift{2.490343in}{2.006691in}%
\pgfsys@useobject{currentmarker}{}%
\end{pgfscope}%
\begin{pgfscope}%
\pgfsys@transformshift{2.494003in}{1.995405in}%
\pgfsys@useobject{currentmarker}{}%
\end{pgfscope}%
\begin{pgfscope}%
\pgfsys@transformshift{2.495850in}{1.981396in}%
\pgfsys@useobject{currentmarker}{}%
\end{pgfscope}%
\begin{pgfscope}%
\pgfsys@transformshift{2.496102in}{1.973628in}%
\pgfsys@useobject{currentmarker}{}%
\end{pgfscope}%
\begin{pgfscope}%
\pgfsys@transformshift{2.499611in}{1.964707in}%
\pgfsys@useobject{currentmarker}{}%
\end{pgfscope}%
\begin{pgfscope}%
\pgfsys@transformshift{2.500563in}{1.954381in}%
\pgfsys@useobject{currentmarker}{}%
\end{pgfscope}%
\begin{pgfscope}%
\pgfsys@transformshift{2.500952in}{1.948692in}%
\pgfsys@useobject{currentmarker}{}%
\end{pgfscope}%
\begin{pgfscope}%
\pgfsys@transformshift{2.501874in}{1.941701in}%
\pgfsys@useobject{currentmarker}{}%
\end{pgfscope}%
\begin{pgfscope}%
\pgfsys@transformshift{2.503145in}{1.938037in}%
\pgfsys@useobject{currentmarker}{}%
\end{pgfscope}%
\begin{pgfscope}%
\pgfsys@transformshift{2.503209in}{1.932910in}%
\pgfsys@useobject{currentmarker}{}%
\end{pgfscope}%
\begin{pgfscope}%
\pgfsys@transformshift{2.503103in}{1.930092in}%
\pgfsys@useobject{currentmarker}{}%
\end{pgfscope}%
\begin{pgfscope}%
\pgfsys@transformshift{2.504412in}{1.924625in}%
\pgfsys@useobject{currentmarker}{}%
\end{pgfscope}%
\begin{pgfscope}%
\pgfsys@transformshift{2.505820in}{1.918697in}%
\pgfsys@useobject{currentmarker}{}%
\end{pgfscope}%
\begin{pgfscope}%
\pgfsys@transformshift{2.505439in}{1.909927in}%
\pgfsys@useobject{currentmarker}{}%
\end{pgfscope}%
\begin{pgfscope}%
\pgfsys@transformshift{2.506011in}{1.905133in}%
\pgfsys@useobject{currentmarker}{}%
\end{pgfscope}%
\begin{pgfscope}%
\pgfsys@transformshift{2.508757in}{1.899162in}%
\pgfsys@useobject{currentmarker}{}%
\end{pgfscope}%
\begin{pgfscope}%
\pgfsys@transformshift{2.509770in}{1.891640in}%
\pgfsys@useobject{currentmarker}{}%
\end{pgfscope}%
\begin{pgfscope}%
\pgfsys@transformshift{2.510546in}{1.887539in}%
\pgfsys@useobject{currentmarker}{}%
\end{pgfscope}%
\begin{pgfscope}%
\pgfsys@transformshift{2.512826in}{1.880105in}%
\pgfsys@useobject{currentmarker}{}%
\end{pgfscope}%
\begin{pgfscope}%
\pgfsys@transformshift{2.516907in}{1.872434in}%
\pgfsys@useobject{currentmarker}{}%
\end{pgfscope}%
\begin{pgfscope}%
\pgfsys@transformshift{2.517076in}{1.861332in}%
\pgfsys@useobject{currentmarker}{}%
\end{pgfscope}%
\begin{pgfscope}%
\pgfsys@transformshift{2.519585in}{1.849767in}%
\pgfsys@useobject{currentmarker}{}%
\end{pgfscope}%
\begin{pgfscope}%
\pgfsys@transformshift{2.524222in}{1.835461in}%
\pgfsys@useobject{currentmarker}{}%
\end{pgfscope}%
\begin{pgfscope}%
\pgfsys@transformshift{2.525864in}{1.827355in}%
\pgfsys@useobject{currentmarker}{}%
\end{pgfscope}%
\begin{pgfscope}%
\pgfsys@transformshift{2.525965in}{1.818151in}%
\pgfsys@useobject{currentmarker}{}%
\end{pgfscope}%
\begin{pgfscope}%
\pgfsys@transformshift{2.529972in}{1.806855in}%
\pgfsys@useobject{currentmarker}{}%
\end{pgfscope}%
\begin{pgfscope}%
\pgfsys@transformshift{2.532345in}{1.800705in}%
\pgfsys@useobject{currentmarker}{}%
\end{pgfscope}%
\begin{pgfscope}%
\pgfsys@transformshift{2.533023in}{1.791550in}%
\pgfsys@useobject{currentmarker}{}%
\end{pgfscope}%
\begin{pgfscope}%
\pgfsys@transformshift{2.532857in}{1.781582in}%
\pgfsys@useobject{currentmarker}{}%
\end{pgfscope}%
\begin{pgfscope}%
\pgfsys@transformshift{2.537492in}{1.770178in}%
\pgfsys@useobject{currentmarker}{}%
\end{pgfscope}%
\begin{pgfscope}%
\pgfsys@transformshift{2.538796in}{1.763533in}%
\pgfsys@useobject{currentmarker}{}%
\end{pgfscope}%
\begin{pgfscope}%
\pgfsys@transformshift{2.539302in}{1.754129in}%
\pgfsys@useobject{currentmarker}{}%
\end{pgfscope}%
\begin{pgfscope}%
\pgfsys@transformshift{2.541392in}{1.744007in}%
\pgfsys@useobject{currentmarker}{}%
\end{pgfscope}%
\begin{pgfscope}%
\pgfsys@transformshift{2.546867in}{1.733974in}%
\pgfsys@useobject{currentmarker}{}%
\end{pgfscope}%
\begin{pgfscope}%
\pgfsys@transformshift{2.547951in}{1.727782in}%
\pgfsys@useobject{currentmarker}{}%
\end{pgfscope}%
\begin{pgfscope}%
\pgfsys@transformshift{2.549662in}{1.720897in}%
\pgfsys@useobject{currentmarker}{}%
\end{pgfscope}%
\begin{pgfscope}%
\pgfsys@transformshift{2.551334in}{1.717371in}%
\pgfsys@useobject{currentmarker}{}%
\end{pgfscope}%
\begin{pgfscope}%
\pgfsys@transformshift{2.552808in}{1.712829in}%
\pgfsys@useobject{currentmarker}{}%
\end{pgfscope}%
\begin{pgfscope}%
\pgfsys@transformshift{2.555398in}{1.707972in}%
\pgfsys@useobject{currentmarker}{}%
\end{pgfscope}%
\begin{pgfscope}%
\pgfsys@transformshift{2.556209in}{1.701889in}%
\pgfsys@useobject{currentmarker}{}%
\end{pgfscope}%
\begin{pgfscope}%
\pgfsys@transformshift{2.558982in}{1.695517in}%
\pgfsys@useobject{currentmarker}{}%
\end{pgfscope}%
\begin{pgfscope}%
\pgfsys@transformshift{2.561057in}{1.688160in}%
\pgfsys@useobject{currentmarker}{}%
\end{pgfscope}%
\begin{pgfscope}%
\pgfsys@transformshift{2.562665in}{1.684275in}%
\pgfsys@useobject{currentmarker}{}%
\end{pgfscope}%
\begin{pgfscope}%
\pgfsys@transformshift{2.562823in}{1.679605in}%
\pgfsys@useobject{currentmarker}{}%
\end{pgfscope}%
\begin{pgfscope}%
\pgfsys@transformshift{2.564214in}{1.673571in}%
\pgfsys@useobject{currentmarker}{}%
\end{pgfscope}%
\begin{pgfscope}%
\pgfsys@transformshift{2.565006in}{1.670259in}%
\pgfsys@useobject{currentmarker}{}%
\end{pgfscope}%
\begin{pgfscope}%
\pgfsys@transformshift{2.566045in}{1.666152in}%
\pgfsys@useobject{currentmarker}{}%
\end{pgfscope}%
\begin{pgfscope}%
\pgfsys@transformshift{2.566157in}{1.663826in}%
\pgfsys@useobject{currentmarker}{}%
\end{pgfscope}%
\begin{pgfscope}%
\pgfsys@transformshift{2.566203in}{1.662545in}%
\pgfsys@useobject{currentmarker}{}%
\end{pgfscope}%
\begin{pgfscope}%
\pgfsys@transformshift{2.567485in}{1.658764in}%
\pgfsys@useobject{currentmarker}{}%
\end{pgfscope}%
\begin{pgfscope}%
\pgfsys@transformshift{2.567643in}{1.656573in}%
\pgfsys@useobject{currentmarker}{}%
\end{pgfscope}%
\begin{pgfscope}%
\pgfsys@transformshift{2.567245in}{1.652452in}%
\pgfsys@useobject{currentmarker}{}%
\end{pgfscope}%
\begin{pgfscope}%
\pgfsys@transformshift{2.567634in}{1.647586in}%
\pgfsys@useobject{currentmarker}{}%
\end{pgfscope}%
\begin{pgfscope}%
\pgfsys@transformshift{2.568085in}{1.644939in}%
\pgfsys@useobject{currentmarker}{}%
\end{pgfscope}%
\begin{pgfscope}%
\pgfsys@transformshift{2.568577in}{1.643547in}%
\pgfsys@useobject{currentmarker}{}%
\end{pgfscope}%
\begin{pgfscope}%
\pgfsys@transformshift{2.568264in}{1.639449in}%
\pgfsys@useobject{currentmarker}{}%
\end{pgfscope}%
\begin{pgfscope}%
\pgfsys@transformshift{2.568433in}{1.634817in}%
\pgfsys@useobject{currentmarker}{}%
\end{pgfscope}%
\begin{pgfscope}%
\pgfsys@transformshift{2.569965in}{1.629963in}%
\pgfsys@useobject{currentmarker}{}%
\end{pgfscope}%
\begin{pgfscope}%
\pgfsys@transformshift{2.570821in}{1.624040in}%
\pgfsys@useobject{currentmarker}{}%
\end{pgfscope}%
\begin{pgfscope}%
\pgfsys@transformshift{2.570435in}{1.617218in}%
\pgfsys@useobject{currentmarker}{}%
\end{pgfscope}%
\begin{pgfscope}%
\pgfsys@transformshift{2.570309in}{1.613462in}%
\pgfsys@useobject{currentmarker}{}%
\end{pgfscope}%
\begin{pgfscope}%
\pgfsys@transformshift{2.572318in}{1.607714in}%
\pgfsys@useobject{currentmarker}{}%
\end{pgfscope}%
\begin{pgfscope}%
\pgfsys@transformshift{2.573796in}{1.601133in}%
\pgfsys@useobject{currentmarker}{}%
\end{pgfscope}%
\begin{pgfscope}%
\pgfsys@transformshift{2.572989in}{1.593506in}%
\pgfsys@useobject{currentmarker}{}%
\end{pgfscope}%
\begin{pgfscope}%
\pgfsys@transformshift{2.572724in}{1.589296in}%
\pgfsys@useobject{currentmarker}{}%
\end{pgfscope}%
\begin{pgfscope}%
\pgfsys@transformshift{2.574661in}{1.583662in}%
\pgfsys@useobject{currentmarker}{}%
\end{pgfscope}%
\begin{pgfscope}%
\pgfsys@transformshift{2.576404in}{1.577415in}%
\pgfsys@useobject{currentmarker}{}%
\end{pgfscope}%
\begin{pgfscope}%
\pgfsys@transformshift{2.576953in}{1.568055in}%
\pgfsys@useobject{currentmarker}{}%
\end{pgfscope}%
\begin{pgfscope}%
\pgfsys@transformshift{2.577283in}{1.562908in}%
\pgfsys@useobject{currentmarker}{}%
\end{pgfscope}%
\begin{pgfscope}%
\pgfsys@transformshift{2.580046in}{1.555133in}%
\pgfsys@useobject{currentmarker}{}%
\end{pgfscope}%
\begin{pgfscope}%
\pgfsys@transformshift{2.580955in}{1.550687in}%
\pgfsys@useobject{currentmarker}{}%
\end{pgfscope}%
\begin{pgfscope}%
\pgfsys@transformshift{2.582127in}{1.543973in}%
\pgfsys@useobject{currentmarker}{}%
\end{pgfscope}%
\begin{pgfscope}%
\pgfsys@transformshift{2.582286in}{1.540227in}%
\pgfsys@useobject{currentmarker}{}%
\end{pgfscope}%
\begin{pgfscope}%
\pgfsys@transformshift{2.584457in}{1.534626in}%
\pgfsys@useobject{currentmarker}{}%
\end{pgfscope}%
\begin{pgfscope}%
\pgfsys@transformshift{2.584971in}{1.528011in}%
\pgfsys@useobject{currentmarker}{}%
\end{pgfscope}%
\begin{pgfscope}%
\pgfsys@transformshift{2.584903in}{1.517815in}%
\pgfsys@useobject{currentmarker}{}%
\end{pgfscope}%
\begin{pgfscope}%
\pgfsys@transformshift{2.583466in}{1.506209in}%
\pgfsys@useobject{currentmarker}{}%
\end{pgfscope}%
\begin{pgfscope}%
\pgfsys@transformshift{2.585397in}{1.493483in}%
\pgfsys@useobject{currentmarker}{}%
\end{pgfscope}%
\begin{pgfscope}%
\pgfsys@transformshift{2.587195in}{1.480046in}%
\pgfsys@useobject{currentmarker}{}%
\end{pgfscope}%
\begin{pgfscope}%
\pgfsys@transformshift{2.586219in}{1.464825in}%
\pgfsys@useobject{currentmarker}{}%
\end{pgfscope}%
\begin{pgfscope}%
\pgfsys@transformshift{2.585286in}{1.456488in}%
\pgfsys@useobject{currentmarker}{}%
\end{pgfscope}%
\begin{pgfscope}%
\pgfsys@transformshift{2.586883in}{1.446044in}%
\pgfsys@useobject{currentmarker}{}%
\end{pgfscope}%
\begin{pgfscope}%
\pgfsys@transformshift{2.588657in}{1.440511in}%
\pgfsys@useobject{currentmarker}{}%
\end{pgfscope}%
\begin{pgfscope}%
\pgfsys@transformshift{2.588170in}{1.433240in}%
\pgfsys@useobject{currentmarker}{}%
\end{pgfscope}%
\begin{pgfscope}%
\pgfsys@transformshift{2.586748in}{1.425373in}%
\pgfsys@useobject{currentmarker}{}%
\end{pgfscope}%
\begin{pgfscope}%
\pgfsys@transformshift{2.585426in}{1.417001in}%
\pgfsys@useobject{currentmarker}{}%
\end{pgfscope}%
\begin{pgfscope}%
\pgfsys@transformshift{2.587544in}{1.406690in}%
\pgfsys@useobject{currentmarker}{}%
\end{pgfscope}%
\begin{pgfscope}%
\pgfsys@transformshift{2.587259in}{1.395710in}%
\pgfsys@useobject{currentmarker}{}%
\end{pgfscope}%
\begin{pgfscope}%
\pgfsys@transformshift{2.585053in}{1.384338in}%
\pgfsys@useobject{currentmarker}{}%
\end{pgfscope}%
\begin{pgfscope}%
\pgfsys@transformshift{2.584857in}{1.372139in}%
\pgfsys@useobject{currentmarker}{}%
\end{pgfscope}%
\begin{pgfscope}%
\pgfsys@transformshift{2.586948in}{1.356358in}%
\pgfsys@useobject{currentmarker}{}%
\end{pgfscope}%
\begin{pgfscope}%
\pgfsys@transformshift{2.588371in}{1.347719in}%
\pgfsys@useobject{currentmarker}{}%
\end{pgfscope}%
\begin{pgfscope}%
\pgfsys@transformshift{2.586963in}{1.336938in}%
\pgfsys@useobject{currentmarker}{}%
\end{pgfscope}%
\begin{pgfscope}%
\pgfsys@transformshift{2.586327in}{1.330992in}%
\pgfsys@useobject{currentmarker}{}%
\end{pgfscope}%
\begin{pgfscope}%
\pgfsys@transformshift{2.585694in}{1.322199in}%
\pgfsys@useobject{currentmarker}{}%
\end{pgfscope}%
\begin{pgfscope}%
\pgfsys@transformshift{2.588550in}{1.312679in}%
\pgfsys@useobject{currentmarker}{}%
\end{pgfscope}%
\begin{pgfscope}%
\pgfsys@transformshift{2.586918in}{1.300849in}%
\pgfsys@useobject{currentmarker}{}%
\end{pgfscope}%
\begin{pgfscope}%
\pgfsys@transformshift{2.585049in}{1.288328in}%
\pgfsys@useobject{currentmarker}{}%
\end{pgfscope}%
\begin{pgfscope}%
\pgfsys@transformshift{2.583643in}{1.281508in}%
\pgfsys@useobject{currentmarker}{}%
\end{pgfscope}%
\begin{pgfscope}%
\pgfsys@transformshift{2.585863in}{1.272582in}%
\pgfsys@useobject{currentmarker}{}%
\end{pgfscope}%
\begin{pgfscope}%
\pgfsys@transformshift{2.586941in}{1.262691in}%
\pgfsys@useobject{currentmarker}{}%
\end{pgfscope}%
\begin{pgfscope}%
\pgfsys@transformshift{2.586426in}{1.251964in}%
\pgfsys@useobject{currentmarker}{}%
\end{pgfscope}%
\begin{pgfscope}%
\pgfsys@transformshift{2.585278in}{1.240524in}%
\pgfsys@useobject{currentmarker}{}%
\end{pgfscope}%
\begin{pgfscope}%
\pgfsys@transformshift{2.582170in}{1.226251in}%
\pgfsys@useobject{currentmarker}{}%
\end{pgfscope}%
\begin{pgfscope}%
\pgfsys@transformshift{2.585433in}{1.211431in}%
\pgfsys@useobject{currentmarker}{}%
\end{pgfscope}%
\begin{pgfscope}%
\pgfsys@transformshift{2.590245in}{1.196421in}%
\pgfsys@useobject{currentmarker}{}%
\end{pgfscope}%
\begin{pgfscope}%
\pgfsys@transformshift{2.588499in}{1.177585in}%
\pgfsys@useobject{currentmarker}{}%
\end{pgfscope}%
\begin{pgfscope}%
\pgfsys@transformshift{2.587022in}{1.158140in}%
\pgfsys@useobject{currentmarker}{}%
\end{pgfscope}%
\begin{pgfscope}%
\pgfsys@transformshift{2.583001in}{1.138227in}%
\pgfsys@useobject{currentmarker}{}%
\end{pgfscope}%
\begin{pgfscope}%
\pgfsys@transformshift{2.589231in}{1.117412in}%
\pgfsys@useobject{currentmarker}{}%
\end{pgfscope}%
\begin{pgfscope}%
\pgfsys@transformshift{2.586755in}{1.094583in}%
\pgfsys@useobject{currentmarker}{}%
\end{pgfscope}%
\begin{pgfscope}%
\pgfsys@transformshift{2.586003in}{1.081976in}%
\pgfsys@useobject{currentmarker}{}%
\end{pgfscope}%
\begin{pgfscope}%
\pgfsys@transformshift{2.585305in}{1.075064in}%
\pgfsys@useobject{currentmarker}{}%
\end{pgfscope}%
\begin{pgfscope}%
\pgfsys@transformshift{2.585623in}{1.066757in}%
\pgfsys@useobject{currentmarker}{}%
\end{pgfscope}%
\begin{pgfscope}%
\pgfsys@transformshift{2.587019in}{1.058084in}%
\pgfsys@useobject{currentmarker}{}%
\end{pgfscope}%
\begin{pgfscope}%
\pgfsys@transformshift{2.585540in}{1.047718in}%
\pgfsys@useobject{currentmarker}{}%
\end{pgfscope}%
\begin{pgfscope}%
\pgfsys@transformshift{2.585364in}{1.036736in}%
\pgfsys@useobject{currentmarker}{}%
\end{pgfscope}%
\begin{pgfscope}%
\pgfsys@transformshift{2.584079in}{1.030833in}%
\pgfsys@useobject{currentmarker}{}%
\end{pgfscope}%
\begin{pgfscope}%
\pgfsys@transformshift{2.584236in}{1.027514in}%
\pgfsys@useobject{currentmarker}{}%
\end{pgfscope}%
\begin{pgfscope}%
\pgfsys@transformshift{2.583474in}{1.023794in}%
\pgfsys@useobject{currentmarker}{}%
\end{pgfscope}%
\begin{pgfscope}%
\pgfsys@transformshift{2.584958in}{1.018490in}%
\pgfsys@useobject{currentmarker}{}%
\end{pgfscope}%
\begin{pgfscope}%
\pgfsys@transformshift{2.584721in}{1.012482in}%
\pgfsys@useobject{currentmarker}{}%
\end{pgfscope}%
\begin{pgfscope}%
\pgfsys@transformshift{2.583980in}{1.005004in}%
\pgfsys@useobject{currentmarker}{}%
\end{pgfscope}%
\begin{pgfscope}%
\pgfsys@transformshift{2.583366in}{1.000917in}%
\pgfsys@useobject{currentmarker}{}%
\end{pgfscope}%
\begin{pgfscope}%
\pgfsys@transformshift{2.583250in}{0.996239in}%
\pgfsys@useobject{currentmarker}{}%
\end{pgfscope}%
\begin{pgfscope}%
\pgfsys@transformshift{2.582932in}{0.991066in}%
\pgfsys@useobject{currentmarker}{}%
\end{pgfscope}%
\begin{pgfscope}%
\pgfsys@transformshift{2.582817in}{0.988219in}%
\pgfsys@useobject{currentmarker}{}%
\end{pgfscope}%
\begin{pgfscope}%
\pgfsys@transformshift{2.582892in}{0.986653in}%
\pgfsys@useobject{currentmarker}{}%
\end{pgfscope}%
\begin{pgfscope}%
\pgfsys@transformshift{2.583525in}{0.982919in}%
\pgfsys@useobject{currentmarker}{}%
\end{pgfscope}%
\begin{pgfscope}%
\pgfsys@transformshift{2.584139in}{0.978716in}%
\pgfsys@useobject{currentmarker}{}%
\end{pgfscope}%
\begin{pgfscope}%
\pgfsys@transformshift{2.583446in}{0.976485in}%
\pgfsys@useobject{currentmarker}{}%
\end{pgfscope}%
\begin{pgfscope}%
\pgfsys@transformshift{2.582604in}{0.975515in}%
\pgfsys@useobject{currentmarker}{}%
\end{pgfscope}%
\begin{pgfscope}%
\pgfsys@transformshift{2.580037in}{0.974618in}%
\pgfsys@useobject{currentmarker}{}%
\end{pgfscope}%
\begin{pgfscope}%
\pgfsys@transformshift{2.575434in}{0.973132in}%
\pgfsys@useobject{currentmarker}{}%
\end{pgfscope}%
\begin{pgfscope}%
\pgfsys@transformshift{2.569037in}{0.972548in}%
\pgfsys@useobject{currentmarker}{}%
\end{pgfscope}%
\begin{pgfscope}%
\pgfsys@transformshift{2.562114in}{0.969885in}%
\pgfsys@useobject{currentmarker}{}%
\end{pgfscope}%
\begin{pgfscope}%
\pgfsys@transformshift{2.553721in}{0.970777in}%
\pgfsys@useobject{currentmarker}{}%
\end{pgfscope}%
\begin{pgfscope}%
\pgfsys@transformshift{2.544745in}{0.969658in}%
\pgfsys@useobject{currentmarker}{}%
\end{pgfscope}%
\begin{pgfscope}%
\pgfsys@transformshift{2.534581in}{0.970645in}%
\pgfsys@useobject{currentmarker}{}%
\end{pgfscope}%
\begin{pgfscope}%
\pgfsys@transformshift{2.523807in}{0.970941in}%
\pgfsys@useobject{currentmarker}{}%
\end{pgfscope}%
\begin{pgfscope}%
\pgfsys@transformshift{2.512283in}{0.971898in}%
\pgfsys@useobject{currentmarker}{}%
\end{pgfscope}%
\begin{pgfscope}%
\pgfsys@transformshift{2.505943in}{0.972398in}%
\pgfsys@useobject{currentmarker}{}%
\end{pgfscope}%
\begin{pgfscope}%
\pgfsys@transformshift{2.502461in}{0.972732in}%
\pgfsys@useobject{currentmarker}{}%
\end{pgfscope}%
\begin{pgfscope}%
\pgfsys@transformshift{2.500544in}{0.972893in}%
\pgfsys@useobject{currentmarker}{}%
\end{pgfscope}%
\begin{pgfscope}%
\pgfsys@transformshift{2.499487in}{0.972944in}%
\pgfsys@useobject{currentmarker}{}%
\end{pgfscope}%
\begin{pgfscope}%
\pgfsys@transformshift{2.498909in}{0.973010in}%
\pgfsys@useobject{currentmarker}{}%
\end{pgfscope}%
\begin{pgfscope}%
\pgfsys@transformshift{2.495425in}{0.972731in}%
\pgfsys@useobject{currentmarker}{}%
\end{pgfscope}%
\begin{pgfscope}%
\pgfsys@transformshift{2.491249in}{0.972549in}%
\pgfsys@useobject{currentmarker}{}%
\end{pgfscope}%
\begin{pgfscope}%
\pgfsys@transformshift{2.485661in}{0.971809in}%
\pgfsys@useobject{currentmarker}{}%
\end{pgfscope}%
\begin{pgfscope}%
\pgfsys@transformshift{2.479330in}{0.972692in}%
\pgfsys@useobject{currentmarker}{}%
\end{pgfscope}%
\begin{pgfscope}%
\pgfsys@transformshift{2.468893in}{0.971939in}%
\pgfsys@useobject{currentmarker}{}%
\end{pgfscope}%
\begin{pgfscope}%
\pgfsys@transformshift{2.455434in}{0.971640in}%
\pgfsys@useobject{currentmarker}{}%
\end{pgfscope}%
\begin{pgfscope}%
\pgfsys@transformshift{2.439777in}{0.969676in}%
\pgfsys@useobject{currentmarker}{}%
\end{pgfscope}%
\begin{pgfscope}%
\pgfsys@transformshift{2.423395in}{0.970028in}%
\pgfsys@useobject{currentmarker}{}%
\end{pgfscope}%
\begin{pgfscope}%
\pgfsys@transformshift{2.403723in}{0.969264in}%
\pgfsys@useobject{currentmarker}{}%
\end{pgfscope}%
\begin{pgfscope}%
\pgfsys@transformshift{2.383272in}{0.968462in}%
\pgfsys@useobject{currentmarker}{}%
\end{pgfscope}%
\begin{pgfscope}%
\pgfsys@transformshift{2.360415in}{0.965860in}%
\pgfsys@useobject{currentmarker}{}%
\end{pgfscope}%
\begin{pgfscope}%
\pgfsys@transformshift{2.347783in}{0.965131in}%
\pgfsys@useobject{currentmarker}{}%
\end{pgfscope}%
\begin{pgfscope}%
\pgfsys@transformshift{2.331617in}{0.965843in}%
\pgfsys@useobject{currentmarker}{}%
\end{pgfscope}%
\begin{pgfscope}%
\pgfsys@transformshift{2.314275in}{0.964833in}%
\pgfsys@useobject{currentmarker}{}%
\end{pgfscope}%
\begin{pgfscope}%
\pgfsys@transformshift{2.294573in}{0.962072in}%
\pgfsys@useobject{currentmarker}{}%
\end{pgfscope}%
\begin{pgfscope}%
\pgfsys@transformshift{2.272646in}{0.959338in}%
\pgfsys@useobject{currentmarker}{}%
\end{pgfscope}%
\begin{pgfscope}%
\pgfsys@transformshift{2.249603in}{0.960320in}%
\pgfsys@useobject{currentmarker}{}%
\end{pgfscope}%
\begin{pgfscope}%
\pgfsys@transformshift{2.223591in}{0.959070in}%
\pgfsys@useobject{currentmarker}{}%
\end{pgfscope}%
\begin{pgfscope}%
\pgfsys@transformshift{2.196767in}{0.956803in}%
\pgfsys@useobject{currentmarker}{}%
\end{pgfscope}%
\begin{pgfscope}%
\pgfsys@transformshift{2.167266in}{0.956397in}%
\pgfsys@useobject{currentmarker}{}%
\end{pgfscope}%
\begin{pgfscope}%
\pgfsys@transformshift{2.151081in}{0.957570in}%
\pgfsys@useobject{currentmarker}{}%
\end{pgfscope}%
\begin{pgfscope}%
\pgfsys@transformshift{2.132113in}{0.956792in}%
\pgfsys@useobject{currentmarker}{}%
\end{pgfscope}%
\begin{pgfscope}%
\pgfsys@transformshift{2.111828in}{0.955196in}%
\pgfsys@useobject{currentmarker}{}%
\end{pgfscope}%
\begin{pgfscope}%
\pgfsys@transformshift{2.089545in}{0.952303in}%
\pgfsys@useobject{currentmarker}{}%
\end{pgfscope}%
\begin{pgfscope}%
\pgfsys@transformshift{2.065299in}{0.953823in}%
\pgfsys@useobject{currentmarker}{}%
\end{pgfscope}%
\begin{pgfscope}%
\pgfsys@transformshift{2.039914in}{0.950769in}%
\pgfsys@useobject{currentmarker}{}%
\end{pgfscope}%
\begin{pgfscope}%
\pgfsys@transformshift{2.012796in}{0.950743in}%
\pgfsys@useobject{currentmarker}{}%
\end{pgfscope}%
\begin{pgfscope}%
\pgfsys@transformshift{1.985249in}{0.944935in}%
\pgfsys@useobject{currentmarker}{}%
\end{pgfscope}%
\begin{pgfscope}%
\pgfsys@transformshift{1.956090in}{0.944536in}%
\pgfsys@useobject{currentmarker}{}%
\end{pgfscope}%
\begin{pgfscope}%
\pgfsys@transformshift{1.925545in}{0.944756in}%
\pgfsys@useobject{currentmarker}{}%
\end{pgfscope}%
\begin{pgfscope}%
\pgfsys@transformshift{1.892608in}{0.944697in}%
\pgfsys@useobject{currentmarker}{}%
\end{pgfscope}%
\begin{pgfscope}%
\pgfsys@transformshift{1.858104in}{0.941895in}%
\pgfsys@useobject{currentmarker}{}%
\end{pgfscope}%
\begin{pgfscope}%
\pgfsys@transformshift{1.822104in}{0.943732in}%
\pgfsys@useobject{currentmarker}{}%
\end{pgfscope}%
\begin{pgfscope}%
\pgfsys@transformshift{1.783839in}{0.943749in}%
\pgfsys@useobject{currentmarker}{}%
\end{pgfscope}%
\begin{pgfscope}%
\pgfsys@transformshift{1.744586in}{0.940652in}%
\pgfsys@useobject{currentmarker}{}%
\end{pgfscope}%
\begin{pgfscope}%
\pgfsys@transformshift{1.703047in}{0.933945in}%
\pgfsys@useobject{currentmarker}{}%
\end{pgfscope}%
\begin{pgfscope}%
\pgfsys@transformshift{1.658380in}{0.933101in}%
\pgfsys@useobject{currentmarker}{}%
\end{pgfscope}%
\begin{pgfscope}%
\pgfsys@transformshift{1.611017in}{0.935056in}%
\pgfsys@useobject{currentmarker}{}%
\end{pgfscope}%
\begin{pgfscope}%
\pgfsys@transformshift{1.562626in}{0.930276in}%
\pgfsys@useobject{currentmarker}{}%
\end{pgfscope}%
\begin{pgfscope}%
\pgfsys@transformshift{1.513138in}{0.926729in}%
\pgfsys@useobject{currentmarker}{}%
\end{pgfscope}%
\begin{pgfscope}%
\pgfsys@transformshift{1.460933in}{0.921745in}%
\pgfsys@useobject{currentmarker}{}%
\end{pgfscope}%
\begin{pgfscope}%
\pgfsys@transformshift{1.407937in}{0.922703in}%
\pgfsys@useobject{currentmarker}{}%
\end{pgfscope}%
\begin{pgfscope}%
\pgfsys@transformshift{1.355705in}{0.937050in}%
\pgfsys@useobject{currentmarker}{}%
\end{pgfscope}%
\begin{pgfscope}%
\pgfsys@transformshift{1.303858in}{0.961151in}%
\pgfsys@useobject{currentmarker}{}%
\end{pgfscope}%
\begin{pgfscope}%
\pgfsys@transformshift{1.263409in}{1.004115in}%
\pgfsys@useobject{currentmarker}{}%
\end{pgfscope}%
\begin{pgfscope}%
\pgfsys@transformshift{1.260641in}{1.036451in}%
\pgfsys@useobject{currentmarker}{}%
\end{pgfscope}%
\begin{pgfscope}%
\pgfsys@transformshift{1.260899in}{1.070463in}%
\pgfsys@useobject{currentmarker}{}%
\end{pgfscope}%
\begin{pgfscope}%
\pgfsys@transformshift{1.265291in}{1.107064in}%
\pgfsys@useobject{currentmarker}{}%
\end{pgfscope}%
\begin{pgfscope}%
\pgfsys@transformshift{1.274493in}{1.145106in}%
\pgfsys@useobject{currentmarker}{}%
\end{pgfscope}%
\begin{pgfscope}%
\pgfsys@transformshift{1.282024in}{1.184290in}%
\pgfsys@useobject{currentmarker}{}%
\end{pgfscope}%
\begin{pgfscope}%
\pgfsys@transformshift{1.286752in}{1.225269in}%
\pgfsys@useobject{currentmarker}{}%
\end{pgfscope}%
\begin{pgfscope}%
\pgfsys@transformshift{1.289061in}{1.267793in}%
\pgfsys@useobject{currentmarker}{}%
\end{pgfscope}%
\begin{pgfscope}%
\pgfsys@transformshift{1.291312in}{1.311949in}%
\pgfsys@useobject{currentmarker}{}%
\end{pgfscope}%
\begin{pgfscope}%
\pgfsys@transformshift{1.283729in}{1.335053in}%
\pgfsys@useobject{currentmarker}{}%
\end{pgfscope}%
\begin{pgfscope}%
\pgfsys@transformshift{1.288217in}{1.361371in}%
\pgfsys@useobject{currentmarker}{}%
\end{pgfscope}%
\begin{pgfscope}%
\pgfsys@transformshift{1.286175in}{1.388674in}%
\pgfsys@useobject{currentmarker}{}%
\end{pgfscope}%
\begin{pgfscope}%
\pgfsys@transformshift{1.276434in}{1.415072in}%
\pgfsys@useobject{currentmarker}{}%
\end{pgfscope}%
\begin{pgfscope}%
\pgfsys@transformshift{1.277226in}{1.430528in}%
\pgfsys@useobject{currentmarker}{}%
\end{pgfscope}%
\begin{pgfscope}%
\pgfsys@transformshift{1.280762in}{1.446980in}%
\pgfsys@useobject{currentmarker}{}%
\end{pgfscope}%
\begin{pgfscope}%
\pgfsys@transformshift{1.274124in}{1.467329in}%
\pgfsys@useobject{currentmarker}{}%
\end{pgfscope}%
\begin{pgfscope}%
\pgfsys@transformshift{1.274410in}{1.479097in}%
\pgfsys@useobject{currentmarker}{}%
\end{pgfscope}%
\begin{pgfscope}%
\pgfsys@transformshift{1.278724in}{1.494073in}%
\pgfsys@useobject{currentmarker}{}%
\end{pgfscope}%
\begin{pgfscope}%
\pgfsys@transformshift{1.275391in}{1.510248in}%
\pgfsys@useobject{currentmarker}{}%
\end{pgfscope}%
\begin{pgfscope}%
\pgfsys@transformshift{1.275204in}{1.519329in}%
\pgfsys@useobject{currentmarker}{}%
\end{pgfscope}%
\begin{pgfscope}%
\pgfsys@transformshift{1.273326in}{1.528996in}%
\pgfsys@useobject{currentmarker}{}%
\end{pgfscope}%
\begin{pgfscope}%
\pgfsys@transformshift{1.275605in}{1.539234in}%
\pgfsys@useobject{currentmarker}{}%
\end{pgfscope}%
\begin{pgfscope}%
\pgfsys@transformshift{1.270669in}{1.552326in}%
\pgfsys@useobject{currentmarker}{}%
\end{pgfscope}%
\begin{pgfscope}%
\pgfsys@transformshift{1.270571in}{1.560021in}%
\pgfsys@useobject{currentmarker}{}%
\end{pgfscope}%
\begin{pgfscope}%
\pgfsys@transformshift{1.267618in}{1.569333in}%
\pgfsys@useobject{currentmarker}{}%
\end{pgfscope}%
\begin{pgfscope}%
\pgfsys@transformshift{1.268785in}{1.574578in}%
\pgfsys@useobject{currentmarker}{}%
\end{pgfscope}%
\begin{pgfscope}%
\pgfsys@transformshift{1.266339in}{1.583068in}%
\pgfsys@useobject{currentmarker}{}%
\end{pgfscope}%
\begin{pgfscope}%
\pgfsys@transformshift{1.265792in}{1.587896in}%
\pgfsys@useobject{currentmarker}{}%
\end{pgfscope}%
\begin{pgfscope}%
\pgfsys@transformshift{1.267518in}{1.596225in}%
\pgfsys@useobject{currentmarker}{}%
\end{pgfscope}%
\begin{pgfscope}%
\pgfsys@transformshift{1.266728in}{1.605217in}%
\pgfsys@useobject{currentmarker}{}%
\end{pgfscope}%
\begin{pgfscope}%
\pgfsys@transformshift{1.263974in}{1.614512in}%
\pgfsys@useobject{currentmarker}{}%
\end{pgfscope}%
\begin{pgfscope}%
\pgfsys@transformshift{1.264110in}{1.619842in}%
\pgfsys@useobject{currentmarker}{}%
\end{pgfscope}%
\begin{pgfscope}%
\pgfsys@transformshift{1.264602in}{1.627500in}%
\pgfsys@useobject{currentmarker}{}%
\end{pgfscope}%
\begin{pgfscope}%
\pgfsys@transformshift{1.266340in}{1.635441in}%
\pgfsys@useobject{currentmarker}{}%
\end{pgfscope}%
\begin{pgfscope}%
\pgfsys@transformshift{1.263123in}{1.646799in}%
\pgfsys@useobject{currentmarker}{}%
\end{pgfscope}%
\begin{pgfscope}%
\pgfsys@transformshift{1.262891in}{1.653288in}%
\pgfsys@useobject{currentmarker}{}%
\end{pgfscope}%
\begin{pgfscope}%
\pgfsys@transformshift{1.264196in}{1.663568in}%
\pgfsys@useobject{currentmarker}{}%
\end{pgfscope}%
\begin{pgfscope}%
\pgfsys@transformshift{1.265370in}{1.674620in}%
\pgfsys@useobject{currentmarker}{}%
\end{pgfscope}%
\begin{pgfscope}%
\pgfsys@transformshift{1.262577in}{1.689185in}%
\pgfsys@useobject{currentmarker}{}%
\end{pgfscope}%
\begin{pgfscope}%
\pgfsys@transformshift{1.262783in}{1.697339in}%
\pgfsys@useobject{currentmarker}{}%
\end{pgfscope}%
\begin{pgfscope}%
\pgfsys@transformshift{1.260086in}{1.707440in}%
\pgfsys@useobject{currentmarker}{}%
\end{pgfscope}%
\begin{pgfscope}%
\pgfsys@transformshift{1.261330in}{1.713054in}%
\pgfsys@useobject{currentmarker}{}%
\end{pgfscope}%
\begin{pgfscope}%
\pgfsys@transformshift{1.259076in}{1.722036in}%
\pgfsys@useobject{currentmarker}{}%
\end{pgfscope}%
\begin{pgfscope}%
\pgfsys@transformshift{1.258062in}{1.727028in}%
\pgfsys@useobject{currentmarker}{}%
\end{pgfscope}%
\begin{pgfscope}%
\pgfsys@transformshift{1.256245in}{1.734570in}%
\pgfsys@useobject{currentmarker}{}%
\end{pgfscope}%
\begin{pgfscope}%
\pgfsys@transformshift{1.259246in}{1.744402in}%
\pgfsys@useobject{currentmarker}{}%
\end{pgfscope}%
\begin{pgfscope}%
\pgfsys@transformshift{1.256167in}{1.757808in}%
\pgfsys@useobject{currentmarker}{}%
\end{pgfscope}%
\begin{pgfscope}%
\pgfsys@transformshift{1.253218in}{1.771902in}%
\pgfsys@useobject{currentmarker}{}%
\end{pgfscope}%
\begin{pgfscope}%
\pgfsys@transformshift{1.251746in}{1.779683in}%
\pgfsys@useobject{currentmarker}{}%
\end{pgfscope}%
\begin{pgfscope}%
\pgfsys@transformshift{1.253695in}{1.789731in}%
\pgfsys@useobject{currentmarker}{}%
\end{pgfscope}%
\begin{pgfscope}%
\pgfsys@transformshift{1.253305in}{1.795347in}%
\pgfsys@useobject{currentmarker}{}%
\end{pgfscope}%
\begin{pgfscope}%
\pgfsys@transformshift{1.251557in}{1.802200in}%
\pgfsys@useobject{currentmarker}{}%
\end{pgfscope}%
\begin{pgfscope}%
\pgfsys@transformshift{1.251170in}{1.806071in}%
\pgfsys@useobject{currentmarker}{}%
\end{pgfscope}%
\begin{pgfscope}%
\pgfsys@transformshift{1.250714in}{1.813184in}%
\pgfsys@useobject{currentmarker}{}%
\end{pgfscope}%
\begin{pgfscope}%
\pgfsys@transformshift{1.251549in}{1.817014in}%
\pgfsys@useobject{currentmarker}{}%
\end{pgfscope}%
\begin{pgfscope}%
\pgfsys@transformshift{1.249275in}{1.825144in}%
\pgfsys@useobject{currentmarker}{}%
\end{pgfscope}%
\begin{pgfscope}%
\pgfsys@transformshift{1.249005in}{1.829780in}%
\pgfsys@useobject{currentmarker}{}%
\end{pgfscope}%
\begin{pgfscope}%
\pgfsys@transformshift{1.248086in}{1.837535in}%
\pgfsys@useobject{currentmarker}{}%
\end{pgfscope}%
\begin{pgfscope}%
\pgfsys@transformshift{1.249373in}{1.841633in}%
\pgfsys@useobject{currentmarker}{}%
\end{pgfscope}%
\begin{pgfscope}%
\pgfsys@transformshift{1.247369in}{1.850658in}%
\pgfsys@useobject{currentmarker}{}%
\end{pgfscope}%
\begin{pgfscope}%
\pgfsys@transformshift{1.247065in}{1.855733in}%
\pgfsys@useobject{currentmarker}{}%
\end{pgfscope}%
\begin{pgfscope}%
\pgfsys@transformshift{1.244464in}{1.864249in}%
\pgfsys@useobject{currentmarker}{}%
\end{pgfscope}%
\begin{pgfscope}%
\pgfsys@transformshift{1.247240in}{1.874733in}%
\pgfsys@useobject{currentmarker}{}%
\end{pgfscope}%
\begin{pgfscope}%
\pgfsys@transformshift{1.244686in}{1.889819in}%
\pgfsys@useobject{currentmarker}{}%
\end{pgfscope}%
\begin{pgfscope}%
\pgfsys@transformshift{1.247559in}{1.906367in}%
\pgfsys@useobject{currentmarker}{}%
\end{pgfscope}%
\begin{pgfscope}%
\pgfsys@transformshift{1.254101in}{1.922576in}%
\pgfsys@useobject{currentmarker}{}%
\end{pgfscope}%
\begin{pgfscope}%
\pgfsys@transformshift{1.250301in}{1.941555in}%
\pgfsys@useobject{currentmarker}{}%
\end{pgfscope}%
\begin{pgfscope}%
\pgfsys@transformshift{1.251529in}{1.961733in}%
\pgfsys@useobject{currentmarker}{}%
\end{pgfscope}%
\begin{pgfscope}%
\pgfsys@transformshift{1.256847in}{1.983122in}%
\pgfsys@useobject{currentmarker}{}%
\end{pgfscope}%
\begin{pgfscope}%
\pgfsys@transformshift{1.258195in}{1.995170in}%
\pgfsys@useobject{currentmarker}{}%
\end{pgfscope}%
\begin{pgfscope}%
\pgfsys@transformshift{1.257483in}{2.011395in}%
\pgfsys@useobject{currentmarker}{}%
\end{pgfscope}%
\begin{pgfscope}%
\pgfsys@transformshift{1.256142in}{2.020226in}%
\pgfsys@useobject{currentmarker}{}%
\end{pgfscope}%
\begin{pgfscope}%
\pgfsys@transformshift{1.259448in}{2.033461in}%
\pgfsys@useobject{currentmarker}{}%
\end{pgfscope}%
\begin{pgfscope}%
\pgfsys@transformshift{1.261784in}{2.048048in}%
\pgfsys@useobject{currentmarker}{}%
\end{pgfscope}%
\begin{pgfscope}%
\pgfsys@transformshift{1.266790in}{2.063787in}%
\pgfsys@useobject{currentmarker}{}%
\end{pgfscope}%
\begin{pgfscope}%
\pgfsys@transformshift{1.262446in}{2.081697in}%
\pgfsys@useobject{currentmarker}{}%
\end{pgfscope}%
\begin{pgfscope}%
\pgfsys@transformshift{1.266875in}{2.103359in}%
\pgfsys@useobject{currentmarker}{}%
\end{pgfscope}%
\begin{pgfscope}%
\pgfsys@transformshift{1.273233in}{2.125424in}%
\pgfsys@useobject{currentmarker}{}%
\end{pgfscope}%
\begin{pgfscope}%
\pgfsys@transformshift{1.276783in}{2.148650in}%
\pgfsys@useobject{currentmarker}{}%
\end{pgfscope}%
\begin{pgfscope}%
\pgfsys@transformshift{1.276752in}{2.173925in}%
\pgfsys@useobject{currentmarker}{}%
\end{pgfscope}%
\begin{pgfscope}%
\pgfsys@transformshift{1.273126in}{2.199452in}%
\pgfsys@useobject{currentmarker}{}%
\end{pgfscope}%
\begin{pgfscope}%
\pgfsys@transformshift{1.282180in}{2.229866in}%
\pgfsys@useobject{currentmarker}{}%
\end{pgfscope}%
\begin{pgfscope}%
\pgfsys@transformshift{1.290778in}{2.263069in}%
\pgfsys@useobject{currentmarker}{}%
\end{pgfscope}%
\begin{pgfscope}%
\pgfsys@transformshift{1.301421in}{2.297587in}%
\pgfsys@useobject{currentmarker}{}%
\end{pgfscope}%
\begin{pgfscope}%
\pgfsys@transformshift{1.294882in}{2.334610in}%
\pgfsys@useobject{currentmarker}{}%
\end{pgfscope}%
\begin{pgfscope}%
\pgfsys@transformshift{1.309632in}{2.373554in}%
\pgfsys@useobject{currentmarker}{}%
\end{pgfscope}%
\begin{pgfscope}%
\pgfsys@transformshift{1.326937in}{2.413815in}%
\pgfsys@useobject{currentmarker}{}%
\end{pgfscope}%
\begin{pgfscope}%
\pgfsys@transformshift{1.339302in}{2.456609in}%
\pgfsys@useobject{currentmarker}{}%
\end{pgfscope}%
\begin{pgfscope}%
\pgfsys@transformshift{1.334544in}{2.502648in}%
\pgfsys@useobject{currentmarker}{}%
\end{pgfscope}%
\begin{pgfscope}%
\pgfsys@transformshift{1.330003in}{2.551067in}%
\pgfsys@useobject{currentmarker}{}%
\end{pgfscope}%
\begin{pgfscope}%
\pgfsys@transformshift{1.329046in}{2.577798in}%
\pgfsys@useobject{currentmarker}{}%
\end{pgfscope}%
\begin{pgfscope}%
\pgfsys@transformshift{1.333278in}{2.591887in}%
\pgfsys@useobject{currentmarker}{}%
\end{pgfscope}%
\begin{pgfscope}%
\pgfsys@transformshift{1.334505in}{2.599885in}%
\pgfsys@useobject{currentmarker}{}%
\end{pgfscope}%
\begin{pgfscope}%
\pgfsys@transformshift{1.333832in}{2.611300in}%
\pgfsys@useobject{currentmarker}{}%
\end{pgfscope}%
\begin{pgfscope}%
\pgfsys@transformshift{1.333232in}{2.617560in}%
\pgfsys@useobject{currentmarker}{}%
\end{pgfscope}%
\begin{pgfscope}%
\pgfsys@transformshift{1.335710in}{2.627440in}%
\pgfsys@useobject{currentmarker}{}%
\end{pgfscope}%
\begin{pgfscope}%
\pgfsys@transformshift{1.336563in}{2.638231in}%
\pgfsys@useobject{currentmarker}{}%
\end{pgfscope}%
\begin{pgfscope}%
\pgfsys@transformshift{1.340070in}{2.650561in}%
\pgfsys@useobject{currentmarker}{}%
\end{pgfscope}%
\begin{pgfscope}%
\pgfsys@transformshift{1.336949in}{2.664967in}%
\pgfsys@useobject{currentmarker}{}%
\end{pgfscope}%
\begin{pgfscope}%
\pgfsys@transformshift{1.336699in}{2.673071in}%
\pgfsys@useobject{currentmarker}{}%
\end{pgfscope}%
\begin{pgfscope}%
\pgfsys@transformshift{1.337932in}{2.682734in}%
\pgfsys@useobject{currentmarker}{}%
\end{pgfscope}%
\begin{pgfscope}%
\pgfsys@transformshift{1.337884in}{2.688091in}%
\pgfsys@useobject{currentmarker}{}%
\end{pgfscope}%
\begin{pgfscope}%
\pgfsys@transformshift{1.338054in}{2.693995in}%
\pgfsys@useobject{currentmarker}{}%
\end{pgfscope}%
\begin{pgfscope}%
\pgfsys@transformshift{1.337724in}{2.700713in}%
\pgfsys@useobject{currentmarker}{}%
\end{pgfscope}%
\begin{pgfscope}%
\pgfsys@transformshift{1.333730in}{2.706789in}%
\pgfsys@useobject{currentmarker}{}%
\end{pgfscope}%
\begin{pgfscope}%
\pgfsys@transformshift{1.330174in}{2.708620in}%
\pgfsys@useobject{currentmarker}{}%
\end{pgfscope}%
\begin{pgfscope}%
\pgfsys@transformshift{1.322881in}{2.708712in}%
\pgfsys@useobject{currentmarker}{}%
\end{pgfscope}%
\begin{pgfscope}%
\pgfsys@transformshift{1.313114in}{2.708781in}%
\pgfsys@useobject{currentmarker}{}%
\end{pgfscope}%
\begin{pgfscope}%
\pgfsys@transformshift{1.307742in}{2.708766in}%
\pgfsys@useobject{currentmarker}{}%
\end{pgfscope}%
\begin{pgfscope}%
\pgfsys@transformshift{1.304800in}{2.708491in}%
\pgfsys@useobject{currentmarker}{}%
\end{pgfscope}%
\begin{pgfscope}%
\pgfsys@transformshift{1.303175in}{2.708446in}%
\pgfsys@useobject{currentmarker}{}%
\end{pgfscope}%
\begin{pgfscope}%
\pgfsys@transformshift{1.302283in}{2.708399in}%
\pgfsys@useobject{currentmarker}{}%
\end{pgfscope}%
\begin{pgfscope}%
\pgfsys@transformshift{1.300093in}{2.708501in}%
\pgfsys@useobject{currentmarker}{}%
\end{pgfscope}%
\begin{pgfscope}%
\pgfsys@transformshift{1.296588in}{2.708384in}%
\pgfsys@useobject{currentmarker}{}%
\end{pgfscope}%
\begin{pgfscope}%
\pgfsys@transformshift{1.291783in}{2.708150in}%
\pgfsys@useobject{currentmarker}{}%
\end{pgfscope}%
\begin{pgfscope}%
\pgfsys@transformshift{1.285399in}{2.708258in}%
\pgfsys@useobject{currentmarker}{}%
\end{pgfscope}%
\begin{pgfscope}%
\pgfsys@transformshift{1.281887in}{2.708256in}%
\pgfsys@useobject{currentmarker}{}%
\end{pgfscope}%
\begin{pgfscope}%
\pgfsys@transformshift{1.276857in}{2.707439in}%
\pgfsys@useobject{currentmarker}{}%
\end{pgfscope}%
\begin{pgfscope}%
\pgfsys@transformshift{1.270973in}{2.707124in}%
\pgfsys@useobject{currentmarker}{}%
\end{pgfscope}%
\begin{pgfscope}%
\pgfsys@transformshift{1.264554in}{2.708075in}%
\pgfsys@useobject{currentmarker}{}%
\end{pgfscope}%
\begin{pgfscope}%
\pgfsys@transformshift{1.257324in}{2.708279in}%
\pgfsys@useobject{currentmarker}{}%
\end{pgfscope}%
\begin{pgfscope}%
\pgfsys@transformshift{1.249159in}{2.707724in}%
\pgfsys@useobject{currentmarker}{}%
\end{pgfscope}%
\begin{pgfscope}%
\pgfsys@transformshift{1.248627in}{2.707245in}%
\pgfsys@useobject{currentmarker}{}%
\end{pgfscope}%
\begin{pgfscope}%
\pgfsys@transformshift{1.259977in}{2.706358in}%
\pgfsys@useobject{currentmarker}{}%
\end{pgfscope}%
\begin{pgfscope}%
\pgfsys@transformshift{1.273190in}{2.704350in}%
\pgfsys@useobject{currentmarker}{}%
\end{pgfscope}%
\begin{pgfscope}%
\pgfsys@transformshift{1.287184in}{2.703106in}%
\pgfsys@useobject{currentmarker}{}%
\end{pgfscope}%
\begin{pgfscope}%
\pgfsys@transformshift{1.294896in}{2.702627in}%
\pgfsys@useobject{currentmarker}{}%
\end{pgfscope}%
\begin{pgfscope}%
\pgfsys@transformshift{1.307090in}{2.702868in}%
\pgfsys@useobject{currentmarker}{}%
\end{pgfscope}%
\begin{pgfscope}%
\pgfsys@transformshift{1.320122in}{2.703698in}%
\pgfsys@useobject{currentmarker}{}%
\end{pgfscope}%
\begin{pgfscope}%
\pgfsys@transformshift{1.334853in}{2.702411in}%
\pgfsys@useobject{currentmarker}{}%
\end{pgfscope}%
\begin{pgfscope}%
\pgfsys@transformshift{1.350280in}{2.703244in}%
\pgfsys@useobject{currentmarker}{}%
\end{pgfscope}%
\begin{pgfscope}%
\pgfsys@transformshift{1.369990in}{2.701700in}%
\pgfsys@useobject{currentmarker}{}%
\end{pgfscope}%
\begin{pgfscope}%
\pgfsys@transformshift{1.390911in}{2.703714in}%
\pgfsys@useobject{currentmarker}{}%
\end{pgfscope}%
\begin{pgfscope}%
\pgfsys@transformshift{1.413935in}{2.701834in}%
\pgfsys@useobject{currentmarker}{}%
\end{pgfscope}%
\begin{pgfscope}%
\pgfsys@transformshift{1.437490in}{2.702684in}%
\pgfsys@useobject{currentmarker}{}%
\end{pgfscope}%
\begin{pgfscope}%
\pgfsys@transformshift{1.462358in}{2.699391in}%
\pgfsys@useobject{currentmarker}{}%
\end{pgfscope}%
\begin{pgfscope}%
\pgfsys@transformshift{1.488867in}{2.706382in}%
\pgfsys@useobject{currentmarker}{}%
\end{pgfscope}%
\begin{pgfscope}%
\pgfsys@transformshift{1.516680in}{2.708489in}%
\pgfsys@useobject{currentmarker}{}%
\end{pgfscope}%
\begin{pgfscope}%
\pgfsys@transformshift{1.532019in}{2.708791in}%
\pgfsys@useobject{currentmarker}{}%
\end{pgfscope}%
\begin{pgfscope}%
\pgfsys@transformshift{1.548433in}{2.708783in}%
\pgfsys@useobject{currentmarker}{}%
\end{pgfscope}%
\begin{pgfscope}%
\pgfsys@transformshift{1.566440in}{2.709017in}%
\pgfsys@useobject{currentmarker}{}%
\end{pgfscope}%
\begin{pgfscope}%
\pgfsys@transformshift{1.584735in}{2.712044in}%
\pgfsys@useobject{currentmarker}{}%
\end{pgfscope}%
\begin{pgfscope}%
\pgfsys@transformshift{1.606576in}{2.710559in}%
\pgfsys@useobject{currentmarker}{}%
\end{pgfscope}%
\begin{pgfscope}%
\pgfsys@transformshift{1.630842in}{2.709540in}%
\pgfsys@useobject{currentmarker}{}%
\end{pgfscope}%
\begin{pgfscope}%
\pgfsys@transformshift{1.655850in}{2.702765in}%
\pgfsys@useobject{currentmarker}{}%
\end{pgfscope}%
\begin{pgfscope}%
\pgfsys@transformshift{1.669946in}{2.704863in}%
\pgfsys@useobject{currentmarker}{}%
\end{pgfscope}%
\begin{pgfscope}%
\pgfsys@transformshift{1.684909in}{2.704812in}%
\pgfsys@useobject{currentmarker}{}%
\end{pgfscope}%
\begin{pgfscope}%
\pgfsys@transformshift{1.701827in}{2.704315in}%
\pgfsys@useobject{currentmarker}{}%
\end{pgfscope}%
\begin{pgfscope}%
\pgfsys@transformshift{1.719961in}{2.702798in}%
\pgfsys@useobject{currentmarker}{}%
\end{pgfscope}%
\begin{pgfscope}%
\pgfsys@transformshift{1.741828in}{2.700832in}%
\pgfsys@useobject{currentmarker}{}%
\end{pgfscope}%
\begin{pgfscope}%
\pgfsys@transformshift{1.766788in}{2.700336in}%
\pgfsys@useobject{currentmarker}{}%
\end{pgfscope}%
\begin{pgfscope}%
\pgfsys@transformshift{1.792278in}{2.700238in}%
\pgfsys@useobject{currentmarker}{}%
\end{pgfscope}%
\begin{pgfscope}%
\pgfsys@transformshift{1.806292in}{2.699847in}%
\pgfsys@useobject{currentmarker}{}%
\end{pgfscope}%
\begin{pgfscope}%
\pgfsys@transformshift{1.813976in}{2.700491in}%
\pgfsys@useobject{currentmarker}{}%
\end{pgfscope}%
\begin{pgfscope}%
\pgfsys@transformshift{1.823320in}{2.700305in}%
\pgfsys@useobject{currentmarker}{}%
\end{pgfscope}%
\begin{pgfscope}%
\pgfsys@transformshift{1.834477in}{2.702632in}%
\pgfsys@useobject{currentmarker}{}%
\end{pgfscope}%
\begin{pgfscope}%
\pgfsys@transformshift{1.848695in}{2.702714in}%
\pgfsys@useobject{currentmarker}{}%
\end{pgfscope}%
\begin{pgfscope}%
\pgfsys@transformshift{1.865062in}{2.703030in}%
\pgfsys@useobject{currentmarker}{}%
\end{pgfscope}%
\begin{pgfscope}%
\pgfsys@transformshift{1.874052in}{2.702553in}%
\pgfsys@useobject{currentmarker}{}%
\end{pgfscope}%
\begin{pgfscope}%
\pgfsys@transformshift{1.884793in}{2.703610in}%
\pgfsys@useobject{currentmarker}{}%
\end{pgfscope}%
\begin{pgfscope}%
\pgfsys@transformshift{1.896673in}{2.705335in}%
\pgfsys@useobject{currentmarker}{}%
\end{pgfscope}%
\begin{pgfscope}%
\pgfsys@transformshift{1.912164in}{2.705633in}%
\pgfsys@useobject{currentmarker}{}%
\end{pgfscope}%
\begin{pgfscope}%
\pgfsys@transformshift{1.929652in}{2.705350in}%
\pgfsys@useobject{currentmarker}{}%
\end{pgfscope}%
\begin{pgfscope}%
\pgfsys@transformshift{1.948025in}{2.706474in}%
\pgfsys@useobject{currentmarker}{}%
\end{pgfscope}%
\begin{pgfscope}%
\pgfsys@transformshift{1.967040in}{2.709820in}%
\pgfsys@useobject{currentmarker}{}%
\end{pgfscope}%
\begin{pgfscope}%
\pgfsys@transformshift{1.987718in}{2.713051in}%
\pgfsys@useobject{currentmarker}{}%
\end{pgfscope}%
\begin{pgfscope}%
\pgfsys@transformshift{1.999226in}{2.713315in}%
\pgfsys@useobject{currentmarker}{}%
\end{pgfscope}%
\begin{pgfscope}%
\pgfsys@transformshift{2.005504in}{2.712493in}%
\pgfsys@useobject{currentmarker}{}%
\end{pgfscope}%
\begin{pgfscope}%
\pgfsys@transformshift{2.013970in}{2.711529in}%
\pgfsys@useobject{currentmarker}{}%
\end{pgfscope}%
\begin{pgfscope}%
\pgfsys@transformshift{2.025162in}{2.712764in}%
\pgfsys@useobject{currentmarker}{}%
\end{pgfscope}%
\begin{pgfscope}%
\pgfsys@transformshift{2.038133in}{2.713743in}%
\pgfsys@useobject{currentmarker}{}%
\end{pgfscope}%
\begin{pgfscope}%
\pgfsys@transformshift{2.053863in}{2.713991in}%
\pgfsys@useobject{currentmarker}{}%
\end{pgfscope}%
\begin{pgfscope}%
\pgfsys@transformshift{2.072482in}{2.713224in}%
\pgfsys@useobject{currentmarker}{}%
\end{pgfscope}%
\begin{pgfscope}%
\pgfsys@transformshift{2.092037in}{2.712898in}%
\pgfsys@useobject{currentmarker}{}%
\end{pgfscope}%
\begin{pgfscope}%
\pgfsys@transformshift{2.102790in}{2.713186in}%
\pgfsys@useobject{currentmarker}{}%
\end{pgfscope}%
\begin{pgfscope}%
\pgfsys@transformshift{2.115144in}{2.714420in}%
\pgfsys@useobject{currentmarker}{}%
\end{pgfscope}%
\begin{pgfscope}%
\pgfsys@transformshift{2.121967in}{2.714696in}%
\pgfsys@useobject{currentmarker}{}%
\end{pgfscope}%
\begin{pgfscope}%
\pgfsys@transformshift{2.129781in}{2.714207in}%
\pgfsys@useobject{currentmarker}{}%
\end{pgfscope}%
\begin{pgfscope}%
\pgfsys@transformshift{2.134083in}{2.714003in}%
\pgfsys@useobject{currentmarker}{}%
\end{pgfscope}%
\begin{pgfscope}%
\pgfsys@transformshift{2.140168in}{2.714086in}%
\pgfsys@useobject{currentmarker}{}%
\end{pgfscope}%
\begin{pgfscope}%
\pgfsys@transformshift{2.147082in}{2.715292in}%
\pgfsys@useobject{currentmarker}{}%
\end{pgfscope}%
\begin{pgfscope}%
\pgfsys@transformshift{2.156068in}{2.715602in}%
\pgfsys@useobject{currentmarker}{}%
\end{pgfscope}%
\begin{pgfscope}%
\pgfsys@transformshift{2.166556in}{2.715922in}%
\pgfsys@useobject{currentmarker}{}%
\end{pgfscope}%
\begin{pgfscope}%
\pgfsys@transformshift{2.178524in}{2.715067in}%
\pgfsys@useobject{currentmarker}{}%
\end{pgfscope}%
\begin{pgfscope}%
\pgfsys@transformshift{2.185090in}{2.715734in}%
\pgfsys@useobject{currentmarker}{}%
\end{pgfscope}%
\begin{pgfscope}%
\pgfsys@transformshift{2.192144in}{2.716221in}%
\pgfsys@useobject{currentmarker}{}%
\end{pgfscope}%
\begin{pgfscope}%
\pgfsys@transformshift{2.200945in}{2.716542in}%
\pgfsys@useobject{currentmarker}{}%
\end{pgfscope}%
\begin{pgfscope}%
\pgfsys@transformshift{2.210721in}{2.715705in}%
\pgfsys@useobject{currentmarker}{}%
\end{pgfscope}%
\begin{pgfscope}%
\pgfsys@transformshift{2.221138in}{2.715123in}%
\pgfsys@useobject{currentmarker}{}%
\end{pgfscope}%
\begin{pgfscope}%
\pgfsys@transformshift{2.226851in}{2.714594in}%
\pgfsys@useobject{currentmarker}{}%
\end{pgfscope}%
\begin{pgfscope}%
\pgfsys@transformshift{2.234458in}{2.714990in}%
\pgfsys@useobject{currentmarker}{}%
\end{pgfscope}%
\begin{pgfscope}%
\pgfsys@transformshift{2.238628in}{2.714587in}%
\pgfsys@useobject{currentmarker}{}%
\end{pgfscope}%
\begin{pgfscope}%
\pgfsys@transformshift{2.243770in}{2.714392in}%
\pgfsys@useobject{currentmarker}{}%
\end{pgfscope}%
\begin{pgfscope}%
\pgfsys@transformshift{2.250283in}{2.713818in}%
\pgfsys@useobject{currentmarker}{}%
\end{pgfscope}%
\begin{pgfscope}%
\pgfsys@transformshift{2.258997in}{2.712554in}%
\pgfsys@useobject{currentmarker}{}%
\end{pgfscope}%
\begin{pgfscope}%
\pgfsys@transformshift{2.263825in}{2.712937in}%
\pgfsys@useobject{currentmarker}{}%
\end{pgfscope}%
\begin{pgfscope}%
\pgfsys@transformshift{2.271843in}{2.712437in}%
\pgfsys@useobject{currentmarker}{}%
\end{pgfscope}%
\begin{pgfscope}%
\pgfsys@transformshift{2.282620in}{2.712441in}%
\pgfsys@useobject{currentmarker}{}%
\end{pgfscope}%
\begin{pgfscope}%
\pgfsys@transformshift{2.294381in}{2.710885in}%
\pgfsys@useobject{currentmarker}{}%
\end{pgfscope}%
\begin{pgfscope}%
\pgfsys@transformshift{2.306848in}{2.711448in}%
\pgfsys@useobject{currentmarker}{}%
\end{pgfscope}%
\begin{pgfscope}%
\pgfsys@transformshift{2.313711in}{2.711473in}%
\pgfsys@useobject{currentmarker}{}%
\end{pgfscope}%
\begin{pgfscope}%
\pgfsys@transformshift{2.322525in}{2.711849in}%
\pgfsys@useobject{currentmarker}{}%
\end{pgfscope}%
\begin{pgfscope}%
\pgfsys@transformshift{2.331962in}{2.709986in}%
\pgfsys@useobject{currentmarker}{}%
\end{pgfscope}%
\begin{pgfscope}%
\pgfsys@transformshift{2.342793in}{2.709083in}%
\pgfsys@useobject{currentmarker}{}%
\end{pgfscope}%
\begin{pgfscope}%
\pgfsys@transformshift{2.354650in}{2.709493in}%
\pgfsys@useobject{currentmarker}{}%
\end{pgfscope}%
\begin{pgfscope}%
\pgfsys@transformshift{2.361175in}{2.709538in}%
\pgfsys@useobject{currentmarker}{}%
\end{pgfscope}%
\begin{pgfscope}%
\pgfsys@transformshift{2.369026in}{2.709347in}%
\pgfsys@useobject{currentmarker}{}%
\end{pgfscope}%
\begin{pgfscope}%
\pgfsys@transformshift{2.378079in}{2.709324in}%
\pgfsys@useobject{currentmarker}{}%
\end{pgfscope}%
\begin{pgfscope}%
\pgfsys@transformshift{2.383034in}{2.708839in}%
\pgfsys@useobject{currentmarker}{}%
\end{pgfscope}%
\begin{pgfscope}%
\pgfsys@transformshift{2.388874in}{2.709632in}%
\pgfsys@useobject{currentmarker}{}%
\end{pgfscope}%
\begin{pgfscope}%
\pgfsys@transformshift{2.396947in}{2.709128in}%
\pgfsys@useobject{currentmarker}{}%
\end{pgfscope}%
\begin{pgfscope}%
\pgfsys@transformshift{2.406822in}{2.709553in}%
\pgfsys@useobject{currentmarker}{}%
\end{pgfscope}%
\begin{pgfscope}%
\pgfsys@transformshift{2.418245in}{2.710775in}%
\pgfsys@useobject{currentmarker}{}%
\end{pgfscope}%
\begin{pgfscope}%
\pgfsys@transformshift{2.424503in}{2.711649in}%
\pgfsys@useobject{currentmarker}{}%
\end{pgfscope}%
\begin{pgfscope}%
\pgfsys@transformshift{2.432005in}{2.713469in}%
\pgfsys@useobject{currentmarker}{}%
\end{pgfscope}%
\begin{pgfscope}%
\pgfsys@transformshift{2.440316in}{2.713587in}%
\pgfsys@useobject{currentmarker}{}%
\end{pgfscope}%
\begin{pgfscope}%
\pgfsys@transformshift{2.449268in}{2.713297in}%
\pgfsys@useobject{currentmarker}{}%
\end{pgfscope}%
\begin{pgfscope}%
\pgfsys@transformshift{2.458818in}{2.713793in}%
\pgfsys@useobject{currentmarker}{}%
\end{pgfscope}%
\begin{pgfscope}%
\pgfsys@transformshift{2.469314in}{2.713012in}%
\pgfsys@useobject{currentmarker}{}%
\end{pgfscope}%
\begin{pgfscope}%
\pgfsys@transformshift{2.475016in}{2.714015in}%
\pgfsys@useobject{currentmarker}{}%
\end{pgfscope}%
\begin{pgfscope}%
\pgfsys@transformshift{2.478200in}{2.714027in}%
\pgfsys@useobject{currentmarker}{}%
\end{pgfscope}%
\begin{pgfscope}%
\pgfsys@transformshift{2.482963in}{2.714103in}%
\pgfsys@useobject{currentmarker}{}%
\end{pgfscope}%
\begin{pgfscope}%
\pgfsys@transformshift{2.489358in}{2.713482in}%
\pgfsys@useobject{currentmarker}{}%
\end{pgfscope}%
\begin{pgfscope}%
\pgfsys@transformshift{2.499161in}{2.713678in}%
\pgfsys@useobject{currentmarker}{}%
\end{pgfscope}%
\begin{pgfscope}%
\pgfsys@transformshift{2.511701in}{2.713396in}%
\pgfsys@useobject{currentmarker}{}%
\end{pgfscope}%
\begin{pgfscope}%
\pgfsys@transformshift{2.525714in}{2.715395in}%
\pgfsys@useobject{currentmarker}{}%
\end{pgfscope}%
\begin{pgfscope}%
\pgfsys@transformshift{2.541499in}{2.716012in}%
\pgfsys@useobject{currentmarker}{}%
\end{pgfscope}%
\begin{pgfscope}%
\pgfsys@transformshift{2.559208in}{2.716665in}%
\pgfsys@useobject{currentmarker}{}%
\end{pgfscope}%
\begin{pgfscope}%
\pgfsys@transformshift{2.578171in}{2.715882in}%
\pgfsys@useobject{currentmarker}{}%
\end{pgfscope}%
\begin{pgfscope}%
\pgfsys@transformshift{2.600873in}{2.720264in}%
\pgfsys@useobject{currentmarker}{}%
\end{pgfscope}%
\begin{pgfscope}%
\pgfsys@transformshift{2.624888in}{2.724378in}%
\pgfsys@useobject{currentmarker}{}%
\end{pgfscope}%
\begin{pgfscope}%
\pgfsys@transformshift{2.650576in}{2.727692in}%
\pgfsys@useobject{currentmarker}{}%
\end{pgfscope}%
\begin{pgfscope}%
\pgfsys@transformshift{2.677415in}{2.728038in}%
\pgfsys@useobject{currentmarker}{}%
\end{pgfscope}%
\begin{pgfscope}%
\pgfsys@transformshift{2.704773in}{2.729913in}%
\pgfsys@useobject{currentmarker}{}%
\end{pgfscope}%
\begin{pgfscope}%
\pgfsys@transformshift{2.735398in}{2.731471in}%
\pgfsys@useobject{currentmarker}{}%
\end{pgfscope}%
\begin{pgfscope}%
\pgfsys@transformshift{2.767578in}{2.735848in}%
\pgfsys@useobject{currentmarker}{}%
\end{pgfscope}%
\begin{pgfscope}%
\pgfsys@transformshift{2.801351in}{2.736370in}%
\pgfsys@useobject{currentmarker}{}%
\end{pgfscope}%
\begin{pgfscope}%
\pgfsys@transformshift{2.819909in}{2.737210in}%
\pgfsys@useobject{currentmarker}{}%
\end{pgfscope}%
\begin{pgfscope}%
\pgfsys@transformshift{2.839874in}{2.736134in}%
\pgfsys@useobject{currentmarker}{}%
\end{pgfscope}%
\begin{pgfscope}%
\pgfsys@transformshift{2.860853in}{2.738835in}%
\pgfsys@useobject{currentmarker}{}%
\end{pgfscope}%
\begin{pgfscope}%
\pgfsys@transformshift{2.872310in}{2.736810in}%
\pgfsys@useobject{currentmarker}{}%
\end{pgfscope}%
\begin{pgfscope}%
\pgfsys@transformshift{2.878400in}{2.738774in}%
\pgfsys@useobject{currentmarker}{}%
\end{pgfscope}%
\begin{pgfscope}%
\pgfsys@transformshift{2.881918in}{2.738692in}%
\pgfsys@useobject{currentmarker}{}%
\end{pgfscope}%
\begin{pgfscope}%
\pgfsys@transformshift{2.883806in}{2.739120in}%
\pgfsys@useobject{currentmarker}{}%
\end{pgfscope}%
\begin{pgfscope}%
\pgfsys@transformshift{2.886603in}{2.739033in}%
\pgfsys@useobject{currentmarker}{}%
\end{pgfscope}%
\begin{pgfscope}%
\pgfsys@transformshift{2.888138in}{2.739131in}%
\pgfsys@useobject{currentmarker}{}%
\end{pgfscope}%
\begin{pgfscope}%
\pgfsys@transformshift{2.890363in}{2.739122in}%
\pgfsys@useobject{currentmarker}{}%
\end{pgfscope}%
\begin{pgfscope}%
\pgfsys@transformshift{2.893623in}{2.739618in}%
\pgfsys@useobject{currentmarker}{}%
\end{pgfscope}%
\begin{pgfscope}%
\pgfsys@transformshift{2.897551in}{2.740064in}%
\pgfsys@useobject{currentmarker}{}%
\end{pgfscope}%
\begin{pgfscope}%
\pgfsys@transformshift{2.899702in}{2.740382in}%
\pgfsys@useobject{currentmarker}{}%
\end{pgfscope}%
\begin{pgfscope}%
\pgfsys@transformshift{2.900890in}{2.740252in}%
\pgfsys@useobject{currentmarker}{}%
\end{pgfscope}%
\begin{pgfscope}%
\pgfsys@transformshift{2.903870in}{2.740244in}%
\pgfsys@useobject{currentmarker}{}%
\end{pgfscope}%
\begin{pgfscope}%
\pgfsys@transformshift{2.908327in}{2.738852in}%
\pgfsys@useobject{currentmarker}{}%
\end{pgfscope}%
\begin{pgfscope}%
\pgfsys@transformshift{2.913491in}{2.737178in}%
\pgfsys@useobject{currentmarker}{}%
\end{pgfscope}%
\begin{pgfscope}%
\pgfsys@transformshift{2.917792in}{2.732799in}%
\pgfsys@useobject{currentmarker}{}%
\end{pgfscope}%
\begin{pgfscope}%
\pgfsys@transformshift{2.924079in}{2.728645in}%
\pgfsys@useobject{currentmarker}{}%
\end{pgfscope}%
\begin{pgfscope}%
\pgfsys@transformshift{2.927464in}{2.721315in}%
\pgfsys@useobject{currentmarker}{}%
\end{pgfscope}%
\begin{pgfscope}%
\pgfsys@transformshift{2.930122in}{2.717758in}%
\pgfsys@useobject{currentmarker}{}%
\end{pgfscope}%
\begin{pgfscope}%
\pgfsys@transformshift{2.930782in}{2.715406in}%
\pgfsys@useobject{currentmarker}{}%
\end{pgfscope}%
\begin{pgfscope}%
\pgfsys@transformshift{2.931139in}{2.712364in}%
\pgfsys@useobject{currentmarker}{}%
\end{pgfscope}%
\begin{pgfscope}%
\pgfsys@transformshift{2.930490in}{2.708391in}%
\pgfsys@useobject{currentmarker}{}%
\end{pgfscope}%
\begin{pgfscope}%
\pgfsys@transformshift{2.929717in}{2.703805in}%
\pgfsys@useobject{currentmarker}{}%
\end{pgfscope}%
\begin{pgfscope}%
\pgfsys@transformshift{2.928922in}{2.698747in}%
\pgfsys@useobject{currentmarker}{}%
\end{pgfscope}%
\begin{pgfscope}%
\pgfsys@transformshift{2.928574in}{2.695953in}%
\pgfsys@useobject{currentmarker}{}%
\end{pgfscope}%
\begin{pgfscope}%
\pgfsys@transformshift{2.928454in}{2.694408in}%
\pgfsys@useobject{currentmarker}{}%
\end{pgfscope}%
\begin{pgfscope}%
\pgfsys@transformshift{2.928295in}{2.693572in}%
\pgfsys@useobject{currentmarker}{}%
\end{pgfscope}%
\begin{pgfscope}%
\pgfsys@transformshift{2.927911in}{2.691986in}%
\pgfsys@useobject{currentmarker}{}%
\end{pgfscope}%
\begin{pgfscope}%
\pgfsys@transformshift{2.928344in}{2.688554in}%
\pgfsys@useobject{currentmarker}{}%
\end{pgfscope}%
\begin{pgfscope}%
\pgfsys@transformshift{2.928309in}{2.686652in}%
\pgfsys@useobject{currentmarker}{}%
\end{pgfscope}%
\begin{pgfscope}%
\pgfsys@transformshift{2.927639in}{2.681662in}%
\pgfsys@useobject{currentmarker}{}%
\end{pgfscope}%
\begin{pgfscope}%
\pgfsys@transformshift{2.926665in}{2.675936in}%
\pgfsys@useobject{currentmarker}{}%
\end{pgfscope}%
\begin{pgfscope}%
\pgfsys@transformshift{2.926754in}{2.668473in}%
\pgfsys@useobject{currentmarker}{}%
\end{pgfscope}%
\begin{pgfscope}%
\pgfsys@transformshift{2.927418in}{2.660055in}%
\pgfsys@useobject{currentmarker}{}%
\end{pgfscope}%
\begin{pgfscope}%
\pgfsys@transformshift{2.926025in}{2.650986in}%
\pgfsys@useobject{currentmarker}{}%
\end{pgfscope}%
\begin{pgfscope}%
\pgfsys@transformshift{2.924137in}{2.640511in}%
\pgfsys@useobject{currentmarker}{}%
\end{pgfscope}%
\begin{pgfscope}%
\pgfsys@transformshift{2.924314in}{2.634660in}%
\pgfsys@useobject{currentmarker}{}%
\end{pgfscope}%
\begin{pgfscope}%
\pgfsys@transformshift{2.924805in}{2.627579in}%
\pgfsys@useobject{currentmarker}{}%
\end{pgfscope}%
\begin{pgfscope}%
\pgfsys@transformshift{2.924324in}{2.623705in}%
\pgfsys@useobject{currentmarker}{}%
\end{pgfscope}%
\begin{pgfscope}%
\pgfsys@transformshift{2.923829in}{2.618863in}%
\pgfsys@useobject{currentmarker}{}%
\end{pgfscope}%
\begin{pgfscope}%
\pgfsys@transformshift{2.923872in}{2.613489in}%
\pgfsys@useobject{currentmarker}{}%
\end{pgfscope}%
\begin{pgfscope}%
\pgfsys@transformshift{2.924357in}{2.607234in}%
\pgfsys@useobject{currentmarker}{}%
\end{pgfscope}%
\begin{pgfscope}%
\pgfsys@transformshift{2.924726in}{2.600511in}%
\pgfsys@useobject{currentmarker}{}%
\end{pgfscope}%
\begin{pgfscope}%
\pgfsys@transformshift{2.923355in}{2.591747in}%
\pgfsys@useobject{currentmarker}{}%
\end{pgfscope}%
\begin{pgfscope}%
\pgfsys@transformshift{2.923034in}{2.586878in}%
\pgfsys@useobject{currentmarker}{}%
\end{pgfscope}%
\begin{pgfscope}%
\pgfsys@transformshift{2.922633in}{2.580944in}%
\pgfsys@useobject{currentmarker}{}%
\end{pgfscope}%
\begin{pgfscope}%
\pgfsys@transformshift{2.923031in}{2.577697in}%
\pgfsys@useobject{currentmarker}{}%
\end{pgfscope}%
\begin{pgfscope}%
\pgfsys@transformshift{2.922570in}{2.573116in}%
\pgfsys@useobject{currentmarker}{}%
\end{pgfscope}%
\begin{pgfscope}%
\pgfsys@transformshift{2.922336in}{2.570594in}%
\pgfsys@useobject{currentmarker}{}%
\end{pgfscope}%
\begin{pgfscope}%
\pgfsys@transformshift{2.922411in}{2.569204in}%
\pgfsys@useobject{currentmarker}{}%
\end{pgfscope}%
\begin{pgfscope}%
\pgfsys@transformshift{2.922604in}{2.566795in}%
\pgfsys@useobject{currentmarker}{}%
\end{pgfscope}%
\begin{pgfscope}%
\pgfsys@transformshift{2.922140in}{2.563369in}%
\pgfsys@useobject{currentmarker}{}%
\end{pgfscope}%
\begin{pgfscope}%
\pgfsys@transformshift{2.921539in}{2.559249in}%
\pgfsys@useobject{currentmarker}{}%
\end{pgfscope}%
\begin{pgfscope}%
\pgfsys@transformshift{2.921439in}{2.556962in}%
\pgfsys@useobject{currentmarker}{}%
\end{pgfscope}%
\begin{pgfscope}%
\pgfsys@transformshift{2.921107in}{2.552915in}%
\pgfsys@useobject{currentmarker}{}%
\end{pgfscope}%
\begin{pgfscope}%
\pgfsys@transformshift{2.921378in}{2.548272in}%
\pgfsys@useobject{currentmarker}{}%
\end{pgfscope}%
\begin{pgfscope}%
\pgfsys@transformshift{2.920360in}{2.542177in}%
\pgfsys@useobject{currentmarker}{}%
\end{pgfscope}%
\begin{pgfscope}%
\pgfsys@transformshift{2.920185in}{2.538783in}%
\pgfsys@useobject{currentmarker}{}%
\end{pgfscope}%
\begin{pgfscope}%
\pgfsys@transformshift{2.919871in}{2.534285in}%
\pgfsys@useobject{currentmarker}{}%
\end{pgfscope}%
\begin{pgfscope}%
\pgfsys@transformshift{2.921017in}{2.528794in}%
\pgfsys@useobject{currentmarker}{}%
\end{pgfscope}%
\begin{pgfscope}%
\pgfsys@transformshift{2.920617in}{2.525735in}%
\pgfsys@useobject{currentmarker}{}%
\end{pgfscope}%
\begin{pgfscope}%
\pgfsys@transformshift{2.920554in}{2.524039in}%
\pgfsys@useobject{currentmarker}{}%
\end{pgfscope}%
\begin{pgfscope}%
\pgfsys@transformshift{2.920449in}{2.523112in}%
\pgfsys@useobject{currentmarker}{}%
\end{pgfscope}%
\begin{pgfscope}%
\pgfsys@transformshift{2.921072in}{2.519635in}%
\pgfsys@useobject{currentmarker}{}%
\end{pgfscope}%
\begin{pgfscope}%
\pgfsys@transformshift{2.921327in}{2.515632in}%
\pgfsys@useobject{currentmarker}{}%
\end{pgfscope}%
\begin{pgfscope}%
\pgfsys@transformshift{2.921179in}{2.513430in}%
\pgfsys@useobject{currentmarker}{}%
\end{pgfscope}%
\begin{pgfscope}%
\pgfsys@transformshift{2.921326in}{2.512226in}%
\pgfsys@useobject{currentmarker}{}%
\end{pgfscope}%
\begin{pgfscope}%
\pgfsys@transformshift{2.921384in}{2.511561in}%
\pgfsys@useobject{currentmarker}{}%
\end{pgfscope}%
\begin{pgfscope}%
\pgfsys@transformshift{2.921352in}{2.511195in}%
\pgfsys@useobject{currentmarker}{}%
\end{pgfscope}%
\begin{pgfscope}%
\pgfsys@transformshift{2.921396in}{2.510998in}%
\pgfsys@useobject{currentmarker}{}%
\end{pgfscope}%
\begin{pgfscope}%
\pgfsys@transformshift{2.921437in}{2.510895in}%
\pgfsys@useobject{currentmarker}{}%
\end{pgfscope}%
\begin{pgfscope}%
\pgfsys@transformshift{2.921755in}{2.510015in}%
\pgfsys@useobject{currentmarker}{}%
\end{pgfscope}%
\begin{pgfscope}%
\pgfsys@transformshift{2.921350in}{2.507373in}%
\pgfsys@useobject{currentmarker}{}%
\end{pgfscope}%
\begin{pgfscope}%
\pgfsys@transformshift{2.921568in}{2.505918in}%
\pgfsys@useobject{currentmarker}{}%
\end{pgfscope}%
\begin{pgfscope}%
\pgfsys@transformshift{2.922136in}{2.501524in}%
\pgfsys@useobject{currentmarker}{}%
\end{pgfscope}%
\begin{pgfscope}%
\pgfsys@transformshift{2.922153in}{2.496193in}%
\pgfsys@useobject{currentmarker}{}%
\end{pgfscope}%
\begin{pgfscope}%
\pgfsys@transformshift{2.920299in}{2.488616in}%
\pgfsys@useobject{currentmarker}{}%
\end{pgfscope}%
\begin{pgfscope}%
\pgfsys@transformshift{2.920453in}{2.484329in}%
\pgfsys@useobject{currentmarker}{}%
\end{pgfscope}%
\begin{pgfscope}%
\pgfsys@transformshift{2.920609in}{2.479465in}%
\pgfsys@useobject{currentmarker}{}%
\end{pgfscope}%
\begin{pgfscope}%
\pgfsys@transformshift{2.920890in}{2.476803in}%
\pgfsys@useobject{currentmarker}{}%
\end{pgfscope}%
\begin{pgfscope}%
\pgfsys@transformshift{2.920076in}{2.470711in}%
\pgfsys@useobject{currentmarker}{}%
\end{pgfscope}%
\begin{pgfscope}%
\pgfsys@transformshift{2.918805in}{2.464160in}%
\pgfsys@useobject{currentmarker}{}%
\end{pgfscope}%
\begin{pgfscope}%
\pgfsys@transformshift{2.919511in}{2.455850in}%
\pgfsys@useobject{currentmarker}{}%
\end{pgfscope}%
\begin{pgfscope}%
\pgfsys@transformshift{2.920056in}{2.451296in}%
\pgfsys@useobject{currentmarker}{}%
\end{pgfscope}%
\begin{pgfscope}%
\pgfsys@transformshift{2.919111in}{2.444589in}%
\pgfsys@useobject{currentmarker}{}%
\end{pgfscope}%
\begin{pgfscope}%
\pgfsys@transformshift{2.918504in}{2.440914in}%
\pgfsys@useobject{currentmarker}{}%
\end{pgfscope}%
\begin{pgfscope}%
\pgfsys@transformshift{2.918675in}{2.438872in}%
\pgfsys@useobject{currentmarker}{}%
\end{pgfscope}%
\begin{pgfscope}%
\pgfsys@transformshift{2.918936in}{2.436123in}%
\pgfsys@useobject{currentmarker}{}%
\end{pgfscope}%
\begin{pgfscope}%
\pgfsys@transformshift{2.918944in}{2.434605in}%
\pgfsys@useobject{currentmarker}{}%
\end{pgfscope}%
\begin{pgfscope}%
\pgfsys@transformshift{2.918428in}{2.432223in}%
\pgfsys@useobject{currentmarker}{}%
\end{pgfscope}%
\begin{pgfscope}%
\pgfsys@transformshift{2.918567in}{2.430891in}%
\pgfsys@useobject{currentmarker}{}%
\end{pgfscope}%
\begin{pgfscope}%
\pgfsys@transformshift{2.918747in}{2.428329in}%
\pgfsys@useobject{currentmarker}{}%
\end{pgfscope}%
\begin{pgfscope}%
\pgfsys@transformshift{2.919201in}{2.424971in}%
\pgfsys@useobject{currentmarker}{}%
\end{pgfscope}%
\begin{pgfscope}%
\pgfsys@transformshift{2.918443in}{2.419152in}%
\pgfsys@useobject{currentmarker}{}%
\end{pgfscope}%
\begin{pgfscope}%
\pgfsys@transformshift{2.918226in}{2.415933in}%
\pgfsys@useobject{currentmarker}{}%
\end{pgfscope}%
\begin{pgfscope}%
\pgfsys@transformshift{2.918005in}{2.414171in}%
\pgfsys@useobject{currentmarker}{}%
\end{pgfscope}%
\begin{pgfscope}%
\pgfsys@transformshift{2.918231in}{2.411144in}%
\pgfsys@useobject{currentmarker}{}%
\end{pgfscope}%
\begin{pgfscope}%
\pgfsys@transformshift{2.918075in}{2.409481in}%
\pgfsys@useobject{currentmarker}{}%
\end{pgfscope}%
\begin{pgfscope}%
\pgfsys@transformshift{2.917900in}{2.405958in}%
\pgfsys@useobject{currentmarker}{}%
\end{pgfscope}%
\begin{pgfscope}%
\pgfsys@transformshift{2.917269in}{2.401952in}%
\pgfsys@useobject{currentmarker}{}%
\end{pgfscope}%
\begin{pgfscope}%
\pgfsys@transformshift{2.917224in}{2.399722in}%
\pgfsys@useobject{currentmarker}{}%
\end{pgfscope}%
\begin{pgfscope}%
\pgfsys@transformshift{2.917294in}{2.398498in}%
\pgfsys@useobject{currentmarker}{}%
\end{pgfscope}%
\begin{pgfscope}%
\pgfsys@transformshift{2.917229in}{2.397826in}%
\pgfsys@useobject{currentmarker}{}%
\end{pgfscope}%
\begin{pgfscope}%
\pgfsys@transformshift{2.917176in}{2.397459in}%
\pgfsys@useobject{currentmarker}{}%
\end{pgfscope}%
\begin{pgfscope}%
\pgfsys@transformshift{2.917105in}{2.396412in}%
\pgfsys@useobject{currentmarker}{}%
\end{pgfscope}%
\begin{pgfscope}%
\pgfsys@transformshift{2.917121in}{2.391965in}%
\pgfsys@useobject{currentmarker}{}%
\end{pgfscope}%
\begin{pgfscope}%
\pgfsys@transformshift{2.917264in}{2.389523in}%
\pgfsys@useobject{currentmarker}{}%
\end{pgfscope}%
\begin{pgfscope}%
\pgfsys@transformshift{2.916376in}{2.385130in}%
\pgfsys@useobject{currentmarker}{}%
\end{pgfscope}%
\begin{pgfscope}%
\pgfsys@transformshift{2.916235in}{2.382669in}%
\pgfsys@useobject{currentmarker}{}%
\end{pgfscope}%
\begin{pgfscope}%
\pgfsys@transformshift{2.915547in}{2.378957in}%
\pgfsys@useobject{currentmarker}{}%
\end{pgfscope}%
\begin{pgfscope}%
\pgfsys@transformshift{2.916318in}{2.373751in}%
\pgfsys@useobject{currentmarker}{}%
\end{pgfscope}%
\begin{pgfscope}%
\pgfsys@transformshift{2.915010in}{2.367485in}%
\pgfsys@useobject{currentmarker}{}%
\end{pgfscope}%
\begin{pgfscope}%
\pgfsys@transformshift{2.913825in}{2.360674in}%
\pgfsys@useobject{currentmarker}{}%
\end{pgfscope}%
\begin{pgfscope}%
\pgfsys@transformshift{2.913308in}{2.353317in}%
\pgfsys@useobject{currentmarker}{}%
\end{pgfscope}%
\begin{pgfscope}%
\pgfsys@transformshift{2.913480in}{2.343709in}%
\pgfsys@useobject{currentmarker}{}%
\end{pgfscope}%
\begin{pgfscope}%
\pgfsys@transformshift{2.913471in}{2.338424in}%
\pgfsys@useobject{currentmarker}{}%
\end{pgfscope}%
\begin{pgfscope}%
\pgfsys@transformshift{2.912304in}{2.332008in}%
\pgfsys@useobject{currentmarker}{}%
\end{pgfscope}%
\begin{pgfscope}%
\pgfsys@transformshift{2.911687in}{2.324776in}%
\pgfsys@useobject{currentmarker}{}%
\end{pgfscope}%
\begin{pgfscope}%
\pgfsys@transformshift{2.910668in}{2.315879in}%
\pgfsys@useobject{currentmarker}{}%
\end{pgfscope}%
\begin{pgfscope}%
\pgfsys@transformshift{2.911582in}{2.306142in}%
\pgfsys@useobject{currentmarker}{}%
\end{pgfscope}%
\begin{pgfscope}%
\pgfsys@transformshift{2.910568in}{2.300860in}%
\pgfsys@useobject{currentmarker}{}%
\end{pgfscope}%
\begin{pgfscope}%
\pgfsys@transformshift{2.909921in}{2.294570in}%
\pgfsys@useobject{currentmarker}{}%
\end{pgfscope}%
\begin{pgfscope}%
\pgfsys@transformshift{2.909272in}{2.287763in}%
\pgfsys@useobject{currentmarker}{}%
\end{pgfscope}%
\begin{pgfscope}%
\pgfsys@transformshift{2.908939in}{2.279306in}%
\pgfsys@useobject{currentmarker}{}%
\end{pgfscope}%
\begin{pgfscope}%
\pgfsys@transformshift{2.909283in}{2.274664in}%
\pgfsys@useobject{currentmarker}{}%
\end{pgfscope}%
\begin{pgfscope}%
\pgfsys@transformshift{2.908131in}{2.268207in}%
\pgfsys@useobject{currentmarker}{}%
\end{pgfscope}%
\begin{pgfscope}%
\pgfsys@transformshift{2.907522in}{2.260926in}%
\pgfsys@useobject{currentmarker}{}%
\end{pgfscope}%
\begin{pgfscope}%
\pgfsys@transformshift{2.907042in}{2.252441in}%
\pgfsys@useobject{currentmarker}{}%
\end{pgfscope}%
\begin{pgfscope}%
\pgfsys@transformshift{2.909873in}{2.243024in}%
\pgfsys@useobject{currentmarker}{}%
\end{pgfscope}%
\begin{pgfscope}%
\pgfsys@transformshift{2.907623in}{2.231456in}%
\pgfsys@useobject{currentmarker}{}%
\end{pgfscope}%
\begin{pgfscope}%
\pgfsys@transformshift{2.907558in}{2.224975in}%
\pgfsys@useobject{currentmarker}{}%
\end{pgfscope}%
\begin{pgfscope}%
\pgfsys@transformshift{2.907341in}{2.221417in}%
\pgfsys@useobject{currentmarker}{}%
\end{pgfscope}%
\begin{pgfscope}%
\pgfsys@transformshift{2.908410in}{2.216387in}%
\pgfsys@useobject{currentmarker}{}%
\end{pgfscope}%
\begin{pgfscope}%
\pgfsys@transformshift{2.908296in}{2.213562in}%
\pgfsys@useobject{currentmarker}{}%
\end{pgfscope}%
\begin{pgfscope}%
\pgfsys@transformshift{2.908173in}{2.210123in}%
\pgfsys@useobject{currentmarker}{}%
\end{pgfscope}%
\begin{pgfscope}%
\pgfsys@transformshift{2.908329in}{2.208237in}%
\pgfsys@useobject{currentmarker}{}%
\end{pgfscope}%
\begin{pgfscope}%
\pgfsys@transformshift{2.908896in}{2.205577in}%
\pgfsys@useobject{currentmarker}{}%
\end{pgfscope}%
\begin{pgfscope}%
\pgfsys@transformshift{2.909144in}{2.202261in}%
\pgfsys@useobject{currentmarker}{}%
\end{pgfscope}%
\begin{pgfscope}%
\pgfsys@transformshift{2.909333in}{2.198423in}%
\pgfsys@useobject{currentmarker}{}%
\end{pgfscope}%
\begin{pgfscope}%
\pgfsys@transformshift{2.909446in}{2.193973in}%
\pgfsys@useobject{currentmarker}{}%
\end{pgfscope}%
\begin{pgfscope}%
\pgfsys@transformshift{2.909518in}{2.188811in}%
\pgfsys@useobject{currentmarker}{}%
\end{pgfscope}%
\begin{pgfscope}%
\pgfsys@transformshift{2.910067in}{2.186025in}%
\pgfsys@useobject{currentmarker}{}%
\end{pgfscope}%
\begin{pgfscope}%
\pgfsys@transformshift{2.910047in}{2.184463in}%
\pgfsys@useobject{currentmarker}{}%
\end{pgfscope}%
\begin{pgfscope}%
\pgfsys@transformshift{2.910041in}{2.183604in}%
\pgfsys@useobject{currentmarker}{}%
\end{pgfscope}%
\begin{pgfscope}%
\pgfsys@transformshift{2.910043in}{2.183132in}%
\pgfsys@useobject{currentmarker}{}%
\end{pgfscope}%
\begin{pgfscope}%
\pgfsys@transformshift{2.910082in}{2.182875in}%
\pgfsys@useobject{currentmarker}{}%
\end{pgfscope}%
\begin{pgfscope}%
\pgfsys@transformshift{2.910081in}{2.182732in}%
\pgfsys@useobject{currentmarker}{}%
\end{pgfscope}%
\begin{pgfscope}%
\pgfsys@transformshift{2.910078in}{2.182653in}%
\pgfsys@useobject{currentmarker}{}%
\end{pgfscope}%
\begin{pgfscope}%
\pgfsys@transformshift{2.910076in}{2.182610in}%
\pgfsys@useobject{currentmarker}{}%
\end{pgfscope}%
\begin{pgfscope}%
\pgfsys@transformshift{2.910075in}{2.182586in}%
\pgfsys@useobject{currentmarker}{}%
\end{pgfscope}%
\begin{pgfscope}%
\pgfsys@transformshift{2.910077in}{2.182573in}%
\pgfsys@useobject{currentmarker}{}%
\end{pgfscope}%
\begin{pgfscope}%
\pgfsys@transformshift{2.909808in}{2.181131in}%
\pgfsys@useobject{currentmarker}{}%
\end{pgfscope}%
\begin{pgfscope}%
\pgfsys@transformshift{2.909715in}{2.179036in}%
\pgfsys@useobject{currentmarker}{}%
\end{pgfscope}%
\begin{pgfscope}%
\pgfsys@transformshift{2.909549in}{2.177895in}%
\pgfsys@useobject{currentmarker}{}%
\end{pgfscope}%
\begin{pgfscope}%
\pgfsys@transformshift{2.909916in}{2.175131in}%
\pgfsys@useobject{currentmarker}{}%
\end{pgfscope}%
\begin{pgfscope}%
\pgfsys@transformshift{2.909860in}{2.173599in}%
\pgfsys@useobject{currentmarker}{}%
\end{pgfscope}%
\begin{pgfscope}%
\pgfsys@transformshift{2.909635in}{2.170374in}%
\pgfsys@useobject{currentmarker}{}%
\end{pgfscope}%
\begin{pgfscope}%
\pgfsys@transformshift{2.909012in}{2.166697in}%
\pgfsys@useobject{currentmarker}{}%
\end{pgfscope}%
\begin{pgfscope}%
\pgfsys@transformshift{2.909602in}{2.162112in}%
\pgfsys@useobject{currentmarker}{}%
\end{pgfscope}%
\begin{pgfscope}%
\pgfsys@transformshift{2.911100in}{2.155683in}%
\pgfsys@useobject{currentmarker}{}%
\end{pgfscope}%
\begin{pgfscope}%
\pgfsys@transformshift{2.910587in}{2.148356in}%
\pgfsys@useobject{currentmarker}{}%
\end{pgfscope}%
\begin{pgfscope}%
\pgfsys@transformshift{2.909241in}{2.140185in}%
\pgfsys@useobject{currentmarker}{}%
\end{pgfscope}%
\begin{pgfscope}%
\pgfsys@transformshift{2.910749in}{2.131169in}%
\pgfsys@useobject{currentmarker}{}%
\end{pgfscope}%
\begin{pgfscope}%
\pgfsys@transformshift{2.913280in}{2.120233in}%
\pgfsys@useobject{currentmarker}{}%
\end{pgfscope}%
\begin{pgfscope}%
\pgfsys@transformshift{2.912312in}{2.107298in}%
\pgfsys@useobject{currentmarker}{}%
\end{pgfscope}%
\begin{pgfscope}%
\pgfsys@transformshift{2.911128in}{2.093107in}%
\pgfsys@useobject{currentmarker}{}%
\end{pgfscope}%
\begin{pgfscope}%
\pgfsys@transformshift{2.911205in}{2.085276in}%
\pgfsys@useobject{currentmarker}{}%
\end{pgfscope}%
\begin{pgfscope}%
\pgfsys@transformshift{2.912773in}{2.076050in}%
\pgfsys@useobject{currentmarker}{}%
\end{pgfscope}%
\begin{pgfscope}%
\pgfsys@transformshift{2.912527in}{2.070910in}%
\pgfsys@useobject{currentmarker}{}%
\end{pgfscope}%
\begin{pgfscope}%
\pgfsys@transformshift{2.912521in}{2.068079in}%
\pgfsys@useobject{currentmarker}{}%
\end{pgfscope}%
\begin{pgfscope}%
\pgfsys@transformshift{2.912698in}{2.066532in}%
\pgfsys@useobject{currentmarker}{}%
\end{pgfscope}%
\begin{pgfscope}%
\pgfsys@transformshift{2.913402in}{2.064249in}%
\pgfsys@useobject{currentmarker}{}%
\end{pgfscope}%
\begin{pgfscope}%
\pgfsys@transformshift{2.913326in}{2.062937in}%
\pgfsys@useobject{currentmarker}{}%
\end{pgfscope}%
\begin{pgfscope}%
\pgfsys@transformshift{2.913266in}{2.062217in}%
\pgfsys@useobject{currentmarker}{}%
\end{pgfscope}%
\begin{pgfscope}%
\pgfsys@transformshift{2.913304in}{2.061821in}%
\pgfsys@useobject{currentmarker}{}%
\end{pgfscope}%
\begin{pgfscope}%
\pgfsys@transformshift{2.913534in}{2.060613in}%
\pgfsys@useobject{currentmarker}{}%
\end{pgfscope}%
\begin{pgfscope}%
\pgfsys@transformshift{2.913628in}{2.058656in}%
\pgfsys@useobject{currentmarker}{}%
\end{pgfscope}%
\begin{pgfscope}%
\pgfsys@transformshift{2.913585in}{2.057579in}%
\pgfsys@useobject{currentmarker}{}%
\end{pgfscope}%
\begin{pgfscope}%
\pgfsys@transformshift{2.913817in}{2.056052in}%
\pgfsys@useobject{currentmarker}{}%
\end{pgfscope}%
\begin{pgfscope}%
\pgfsys@transformshift{2.914271in}{2.052951in}%
\pgfsys@useobject{currentmarker}{}%
\end{pgfscope}%
\begin{pgfscope}%
\pgfsys@transformshift{2.914427in}{2.051235in}%
\pgfsys@useobject{currentmarker}{}%
\end{pgfscope}%
\begin{pgfscope}%
\pgfsys@transformshift{2.914477in}{2.048896in}%
\pgfsys@useobject{currentmarker}{}%
\end{pgfscope}%
\begin{pgfscope}%
\pgfsys@transformshift{2.914558in}{2.047612in}%
\pgfsys@useobject{currentmarker}{}%
\end{pgfscope}%
\begin{pgfscope}%
\pgfsys@transformshift{2.914710in}{2.044602in}%
\pgfsys@useobject{currentmarker}{}%
\end{pgfscope}%
\begin{pgfscope}%
\pgfsys@transformshift{2.915003in}{2.041087in}%
\pgfsys@useobject{currentmarker}{}%
\end{pgfscope}%
\begin{pgfscope}%
\pgfsys@transformshift{2.914801in}{2.036507in}%
\pgfsys@useobject{currentmarker}{}%
\end{pgfscope}%
\begin{pgfscope}%
\pgfsys@transformshift{2.914830in}{2.033986in}%
\pgfsys@useobject{currentmarker}{}%
\end{pgfscope}%
\begin{pgfscope}%
\pgfsys@transformshift{2.915266in}{2.030696in}%
\pgfsys@useobject{currentmarker}{}%
\end{pgfscope}%
\begin{pgfscope}%
\pgfsys@transformshift{2.915644in}{2.026896in}%
\pgfsys@useobject{currentmarker}{}%
\end{pgfscope}%
\begin{pgfscope}%
\pgfsys@transformshift{2.915586in}{2.024797in}%
\pgfsys@useobject{currentmarker}{}%
\end{pgfscope}%
\begin{pgfscope}%
\pgfsys@transformshift{2.915935in}{2.022113in}%
\pgfsys@useobject{currentmarker}{}%
\end{pgfscope}%
\begin{pgfscope}%
\pgfsys@transformshift{2.916897in}{2.017477in}%
\pgfsys@useobject{currentmarker}{}%
\end{pgfscope}%
\begin{pgfscope}%
\pgfsys@transformshift{2.918189in}{2.012288in}%
\pgfsys@useobject{currentmarker}{}%
\end{pgfscope}%
\begin{pgfscope}%
\pgfsys@transformshift{2.917725in}{2.005141in}%
\pgfsys@useobject{currentmarker}{}%
\end{pgfscope}%
\begin{pgfscope}%
\pgfsys@transformshift{2.918047in}{1.997490in}%
\pgfsys@useobject{currentmarker}{}%
\end{pgfscope}%
\begin{pgfscope}%
\pgfsys@transformshift{2.919457in}{1.988128in}%
\pgfsys@useobject{currentmarker}{}%
\end{pgfscope}%
\begin{pgfscope}%
\pgfsys@transformshift{2.921708in}{1.978442in}%
\pgfsys@useobject{currentmarker}{}%
\end{pgfscope}%
\begin{pgfscope}%
\pgfsys@transformshift{2.921906in}{1.966028in}%
\pgfsys@useobject{currentmarker}{}%
\end{pgfscope}%
\begin{pgfscope}%
\pgfsys@transformshift{2.922422in}{1.953004in}%
\pgfsys@useobject{currentmarker}{}%
\end{pgfscope}%
\begin{pgfscope}%
\pgfsys@transformshift{2.923946in}{1.939055in}%
\pgfsys@useobject{currentmarker}{}%
\end{pgfscope}%
\begin{pgfscope}%
\pgfsys@transformshift{2.928736in}{1.925148in}%
\pgfsys@useobject{currentmarker}{}%
\end{pgfscope}%
\begin{pgfscope}%
\pgfsys@transformshift{2.929177in}{1.909103in}%
\pgfsys@useobject{currentmarker}{}%
\end{pgfscope}%
\begin{pgfscope}%
\pgfsys@transformshift{2.928641in}{1.900291in}%
\pgfsys@useobject{currentmarker}{}%
\end{pgfscope}%
\begin{pgfscope}%
\pgfsys@transformshift{2.929235in}{1.895473in}%
\pgfsys@useobject{currentmarker}{}%
\end{pgfscope}%
\begin{pgfscope}%
\pgfsys@transformshift{2.929491in}{1.889884in}%
\pgfsys@useobject{currentmarker}{}%
\end{pgfscope}%
\begin{pgfscope}%
\pgfsys@transformshift{2.931455in}{1.883991in}%
\pgfsys@useobject{currentmarker}{}%
\end{pgfscope}%
\begin{pgfscope}%
\pgfsys@transformshift{2.930943in}{1.875560in}%
\pgfsys@useobject{currentmarker}{}%
\end{pgfscope}%
\begin{pgfscope}%
\pgfsys@transformshift{2.930519in}{1.870933in}%
\pgfsys@useobject{currentmarker}{}%
\end{pgfscope}%
\begin{pgfscope}%
\pgfsys@transformshift{2.931395in}{1.865662in}%
\pgfsys@useobject{currentmarker}{}%
\end{pgfscope}%
\begin{pgfscope}%
\pgfsys@transformshift{2.933312in}{1.859139in}%
\pgfsys@useobject{currentmarker}{}%
\end{pgfscope}%
\begin{pgfscope}%
\pgfsys@transformshift{2.932405in}{1.849927in}%
\pgfsys@useobject{currentmarker}{}%
\end{pgfscope}%
\begin{pgfscope}%
\pgfsys@transformshift{2.930521in}{1.839591in}%
\pgfsys@useobject{currentmarker}{}%
\end{pgfscope}%
\begin{pgfscope}%
\pgfsys@transformshift{2.930826in}{1.828477in}%
\pgfsys@useobject{currentmarker}{}%
\end{pgfscope}%
\begin{pgfscope}%
\pgfsys@transformshift{2.935407in}{1.815721in}%
\pgfsys@useobject{currentmarker}{}%
\end{pgfscope}%
\begin{pgfscope}%
\pgfsys@transformshift{2.934962in}{1.808279in}%
\pgfsys@useobject{currentmarker}{}%
\end{pgfscope}%
\begin{pgfscope}%
\pgfsys@transformshift{2.934872in}{1.804180in}%
\pgfsys@useobject{currentmarker}{}%
\end{pgfscope}%
\begin{pgfscope}%
\pgfsys@transformshift{2.935216in}{1.799608in}%
\pgfsys@useobject{currentmarker}{}%
\end{pgfscope}%
\begin{pgfscope}%
\pgfsys@transformshift{2.936969in}{1.794035in}%
\pgfsys@useobject{currentmarker}{}%
\end{pgfscope}%
\begin{pgfscope}%
\pgfsys@transformshift{2.936585in}{1.787467in}%
\pgfsys@useobject{currentmarker}{}%
\end{pgfscope}%
\begin{pgfscope}%
\pgfsys@transformshift{2.936300in}{1.783860in}%
\pgfsys@useobject{currentmarker}{}%
\end{pgfscope}%
\begin{pgfscope}%
\pgfsys@transformshift{2.936558in}{1.781887in}%
\pgfsys@useobject{currentmarker}{}%
\end{pgfscope}%
\begin{pgfscope}%
\pgfsys@transformshift{2.937862in}{1.778321in}%
\pgfsys@useobject{currentmarker}{}%
\end{pgfscope}%
\begin{pgfscope}%
\pgfsys@transformshift{2.936938in}{1.772301in}%
\pgfsys@useobject{currentmarker}{}%
\end{pgfscope}%
\begin{pgfscope}%
\pgfsys@transformshift{2.935926in}{1.765120in}%
\pgfsys@useobject{currentmarker}{}%
\end{pgfscope}%
\begin{pgfscope}%
\pgfsys@transformshift{2.937007in}{1.755734in}%
\pgfsys@useobject{currentmarker}{}%
\end{pgfscope}%
\begin{pgfscope}%
\pgfsys@transformshift{2.940863in}{1.744450in}%
\pgfsys@useobject{currentmarker}{}%
\end{pgfscope}%
\begin{pgfscope}%
\pgfsys@transformshift{2.938390in}{1.730509in}%
\pgfsys@useobject{currentmarker}{}%
\end{pgfscope}%
\begin{pgfscope}%
\pgfsys@transformshift{2.936007in}{1.715650in}%
\pgfsys@useobject{currentmarker}{}%
\end{pgfscope}%
\begin{pgfscope}%
\pgfsys@transformshift{2.937822in}{1.699773in}%
\pgfsys@useobject{currentmarker}{}%
\end{pgfscope}%
\begin{pgfscope}%
\pgfsys@transformshift{2.942421in}{1.682316in}%
\pgfsys@useobject{currentmarker}{}%
\end{pgfscope}%
\begin{pgfscope}%
\pgfsys@transformshift{2.942568in}{1.661904in}%
\pgfsys@useobject{currentmarker}{}%
\end{pgfscope}%
\begin{pgfscope}%
\pgfsys@transformshift{2.941527in}{1.640876in}%
\pgfsys@useobject{currentmarker}{}%
\end{pgfscope}%
\begin{pgfscope}%
\pgfsys@transformshift{2.945744in}{1.618665in}%
\pgfsys@useobject{currentmarker}{}%
\end{pgfscope}%
\begin{pgfscope}%
\pgfsys@transformshift{2.949396in}{1.606779in}%
\pgfsys@useobject{currentmarker}{}%
\end{pgfscope}%
\begin{pgfscope}%
\pgfsys@transformshift{2.950180in}{1.593886in}%
\pgfsys@useobject{currentmarker}{}%
\end{pgfscope}%
\begin{pgfscope}%
\pgfsys@transformshift{2.950375in}{1.586784in}%
\pgfsys@useobject{currentmarker}{}%
\end{pgfscope}%
\begin{pgfscope}%
\pgfsys@transformshift{2.951640in}{1.578483in}%
\pgfsys@useobject{currentmarker}{}%
\end{pgfscope}%
\begin{pgfscope}%
\pgfsys@transformshift{2.953643in}{1.574321in}%
\pgfsys@useobject{currentmarker}{}%
\end{pgfscope}%
\begin{pgfscope}%
\pgfsys@transformshift{2.953757in}{1.571783in}%
\pgfsys@useobject{currentmarker}{}%
\end{pgfscope}%
\begin{pgfscope}%
\pgfsys@transformshift{2.953813in}{1.570387in}%
\pgfsys@useobject{currentmarker}{}%
\end{pgfscope}%
\begin{pgfscope}%
\pgfsys@transformshift{2.953850in}{1.569620in}%
\pgfsys@useobject{currentmarker}{}%
\end{pgfscope}%
\begin{pgfscope}%
\pgfsys@transformshift{2.953987in}{1.569220in}%
\pgfsys@useobject{currentmarker}{}%
\end{pgfscope}%
\begin{pgfscope}%
\pgfsys@transformshift{2.953912in}{1.567874in}%
\pgfsys@useobject{currentmarker}{}%
\end{pgfscope}%
\begin{pgfscope}%
\pgfsys@transformshift{2.953928in}{1.567132in}%
\pgfsys@useobject{currentmarker}{}%
\end{pgfscope}%
\begin{pgfscope}%
\pgfsys@transformshift{2.953961in}{1.566726in}%
\pgfsys@useobject{currentmarker}{}%
\end{pgfscope}%
\begin{pgfscope}%
\pgfsys@transformshift{2.954023in}{1.566510in}%
\pgfsys@useobject{currentmarker}{}%
\end{pgfscope}%
\begin{pgfscope}%
\pgfsys@transformshift{2.954015in}{1.566387in}%
\pgfsys@useobject{currentmarker}{}%
\end{pgfscope}%
\begin{pgfscope}%
\pgfsys@transformshift{2.954017in}{1.566319in}%
\pgfsys@useobject{currentmarker}{}%
\end{pgfscope}%
\begin{pgfscope}%
\pgfsys@transformshift{2.954016in}{1.566282in}%
\pgfsys@useobject{currentmarker}{}%
\end{pgfscope}%
\begin{pgfscope}%
\pgfsys@transformshift{2.954021in}{1.566262in}%
\pgfsys@useobject{currentmarker}{}%
\end{pgfscope}%
\begin{pgfscope}%
\pgfsys@transformshift{2.954020in}{1.566250in}%
\pgfsys@useobject{currentmarker}{}%
\end{pgfscope}%
\begin{pgfscope}%
\pgfsys@transformshift{2.954020in}{1.566244in}%
\pgfsys@useobject{currentmarker}{}%
\end{pgfscope}%
\begin{pgfscope}%
\pgfsys@transformshift{2.954020in}{1.566241in}%
\pgfsys@useobject{currentmarker}{}%
\end{pgfscope}%
\begin{pgfscope}%
\pgfsys@transformshift{2.954512in}{1.564105in}%
\pgfsys@useobject{currentmarker}{}%
\end{pgfscope}%
\begin{pgfscope}%
\pgfsys@transformshift{2.954984in}{1.561353in}%
\pgfsys@useobject{currentmarker}{}%
\end{pgfscope}%
\begin{pgfscope}%
\pgfsys@transformshift{2.954821in}{1.555934in}%
\pgfsys@useobject{currentmarker}{}%
\end{pgfscope}%
\begin{pgfscope}%
\pgfsys@transformshift{2.954609in}{1.552960in}%
\pgfsys@useobject{currentmarker}{}%
\end{pgfscope}%
\begin{pgfscope}%
\pgfsys@transformshift{2.955249in}{1.548427in}%
\pgfsys@useobject{currentmarker}{}%
\end{pgfscope}%
\begin{pgfscope}%
\pgfsys@transformshift{2.956370in}{1.543486in}%
\pgfsys@useobject{currentmarker}{}%
\end{pgfscope}%
\begin{pgfscope}%
\pgfsys@transformshift{2.956734in}{1.537398in}%
\pgfsys@useobject{currentmarker}{}%
\end{pgfscope}%
\begin{pgfscope}%
\pgfsys@transformshift{2.956758in}{1.534043in}%
\pgfsys@useobject{currentmarker}{}%
\end{pgfscope}%
\begin{pgfscope}%
\pgfsys@transformshift{2.956863in}{1.532201in}%
\pgfsys@useobject{currentmarker}{}%
\end{pgfscope}%
\begin{pgfscope}%
\pgfsys@transformshift{2.957443in}{1.527930in}%
\pgfsys@useobject{currentmarker}{}%
\end{pgfscope}%
\begin{pgfscope}%
\pgfsys@transformshift{2.957658in}{1.522961in}%
\pgfsys@useobject{currentmarker}{}%
\end{pgfscope}%
\begin{pgfscope}%
\pgfsys@transformshift{2.957425in}{1.517452in}%
\pgfsys@useobject{currentmarker}{}%
\end{pgfscope}%
\begin{pgfscope}%
\pgfsys@transformshift{2.957574in}{1.514423in}%
\pgfsys@useobject{currentmarker}{}%
\end{pgfscope}%
\begin{pgfscope}%
\pgfsys@transformshift{2.958157in}{1.508812in}%
\pgfsys@useobject{currentmarker}{}%
\end{pgfscope}%
\begin{pgfscope}%
\pgfsys@transformshift{2.958552in}{1.505735in}%
\pgfsys@useobject{currentmarker}{}%
\end{pgfscope}%
\begin{pgfscope}%
\pgfsys@transformshift{2.958466in}{1.501526in}%
\pgfsys@useobject{currentmarker}{}%
\end{pgfscope}%
\begin{pgfscope}%
\pgfsys@transformshift{2.958429in}{1.499211in}%
\pgfsys@useobject{currentmarker}{}%
\end{pgfscope}%
\begin{pgfscope}%
\pgfsys@transformshift{2.959380in}{1.494151in}%
\pgfsys@useobject{currentmarker}{}%
\end{pgfscope}%
\begin{pgfscope}%
\pgfsys@transformshift{2.959808in}{1.491352in}%
\pgfsys@useobject{currentmarker}{}%
\end{pgfscope}%
\begin{pgfscope}%
\pgfsys@transformshift{2.959675in}{1.487233in}%
\pgfsys@useobject{currentmarker}{}%
\end{pgfscope}%
\begin{pgfscope}%
\pgfsys@transformshift{2.959603in}{1.484967in}%
\pgfsys@useobject{currentmarker}{}%
\end{pgfscope}%
\begin{pgfscope}%
\pgfsys@transformshift{2.960206in}{1.481386in}%
\pgfsys@useobject{currentmarker}{}%
\end{pgfscope}%
\begin{pgfscope}%
\pgfsys@transformshift{2.961684in}{1.475538in}%
\pgfsys@useobject{currentmarker}{}%
\end{pgfscope}%
\begin{pgfscope}%
\pgfsys@transformshift{2.961181in}{1.468288in}%
\pgfsys@useobject{currentmarker}{}%
\end{pgfscope}%
\begin{pgfscope}%
\pgfsys@transformshift{2.960566in}{1.460363in}%
\pgfsys@useobject{currentmarker}{}%
\end{pgfscope}%
\begin{pgfscope}%
\pgfsys@transformshift{2.960894in}{1.451844in}%
\pgfsys@useobject{currentmarker}{}%
\end{pgfscope}%
\begin{pgfscope}%
\pgfsys@transformshift{2.961921in}{1.442246in}%
\pgfsys@useobject{currentmarker}{}%
\end{pgfscope}%
\begin{pgfscope}%
\pgfsys@transformshift{2.962102in}{1.436940in}%
\pgfsys@useobject{currentmarker}{}%
\end{pgfscope}%
\begin{pgfscope}%
\pgfsys@transformshift{2.961348in}{1.430932in}%
\pgfsys@useobject{currentmarker}{}%
\end{pgfscope}%
\begin{pgfscope}%
\pgfsys@transformshift{2.961780in}{1.427629in}%
\pgfsys@useobject{currentmarker}{}%
\end{pgfscope}%
\begin{pgfscope}%
\pgfsys@transformshift{2.961926in}{1.421315in}%
\pgfsys@useobject{currentmarker}{}%
\end{pgfscope}%
\begin{pgfscope}%
\pgfsys@transformshift{2.962422in}{1.417877in}%
\pgfsys@useobject{currentmarker}{}%
\end{pgfscope}%
\begin{pgfscope}%
\pgfsys@transformshift{2.962372in}{1.412638in}%
\pgfsys@useobject{currentmarker}{}%
\end{pgfscope}%
\begin{pgfscope}%
\pgfsys@transformshift{2.962597in}{1.409765in}%
\pgfsys@useobject{currentmarker}{}%
\end{pgfscope}%
\begin{pgfscope}%
\pgfsys@transformshift{2.963783in}{1.404637in}%
\pgfsys@useobject{currentmarker}{}%
\end{pgfscope}%
\begin{pgfscope}%
\pgfsys@transformshift{2.964139in}{1.401763in}%
\pgfsys@useobject{currentmarker}{}%
\end{pgfscope}%
\begin{pgfscope}%
\pgfsys@transformshift{2.964119in}{1.397691in}%
\pgfsys@useobject{currentmarker}{}%
\end{pgfscope}%
\begin{pgfscope}%
\pgfsys@transformshift{2.963592in}{1.393020in}%
\pgfsys@useobject{currentmarker}{}%
\end{pgfscope}%
\begin{pgfscope}%
\pgfsys@transformshift{2.965098in}{1.384989in}%
\pgfsys@useobject{currentmarker}{}%
\end{pgfscope}%
\begin{pgfscope}%
\pgfsys@transformshift{2.966961in}{1.376162in}%
\pgfsys@useobject{currentmarker}{}%
\end{pgfscope}%
\begin{pgfscope}%
\pgfsys@transformshift{2.965746in}{1.364684in}%
\pgfsys@useobject{currentmarker}{}%
\end{pgfscope}%
\begin{pgfscope}%
\pgfsys@transformshift{2.965650in}{1.358336in}%
\pgfsys@useobject{currentmarker}{}%
\end{pgfscope}%
\begin{pgfscope}%
\pgfsys@transformshift{2.966901in}{1.350341in}%
\pgfsys@useobject{currentmarker}{}%
\end{pgfscope}%
\begin{pgfscope}%
\pgfsys@transformshift{2.968696in}{1.340868in}%
\pgfsys@useobject{currentmarker}{}%
\end{pgfscope}%
\begin{pgfscope}%
\pgfsys@transformshift{2.968635in}{1.328272in}%
\pgfsys@useobject{currentmarker}{}%
\end{pgfscope}%
\begin{pgfscope}%
\pgfsys@transformshift{2.966216in}{1.314156in}%
\pgfsys@useobject{currentmarker}{}%
\end{pgfscope}%
\begin{pgfscope}%
\pgfsys@transformshift{2.970128in}{1.299538in}%
\pgfsys@useobject{currentmarker}{}%
\end{pgfscope}%
\begin{pgfscope}%
\pgfsys@transformshift{2.975476in}{1.282663in}%
\pgfsys@useobject{currentmarker}{}%
\end{pgfscope}%
\begin{pgfscope}%
\pgfsys@transformshift{2.975492in}{1.261841in}%
\pgfsys@useobject{currentmarker}{}%
\end{pgfscope}%
\begin{pgfscope}%
\pgfsys@transformshift{2.972387in}{1.240650in}%
\pgfsys@useobject{currentmarker}{}%
\end{pgfscope}%
\begin{pgfscope}%
\pgfsys@transformshift{2.977862in}{1.217861in}%
\pgfsys@useobject{currentmarker}{}%
\end{pgfscope}%
\begin{pgfscope}%
\pgfsys@transformshift{2.986372in}{1.193585in}%
\pgfsys@useobject{currentmarker}{}%
\end{pgfscope}%
\begin{pgfscope}%
\pgfsys@transformshift{2.985531in}{1.165137in}%
\pgfsys@useobject{currentmarker}{}%
\end{pgfscope}%
\begin{pgfscope}%
\pgfsys@transformshift{2.985737in}{1.149485in}%
\pgfsys@useobject{currentmarker}{}%
\end{pgfscope}%
\begin{pgfscope}%
\pgfsys@transformshift{2.985573in}{1.140877in}%
\pgfsys@useobject{currentmarker}{}%
\end{pgfscope}%
\begin{pgfscope}%
\pgfsys@transformshift{2.985724in}{1.136144in}%
\pgfsys@useobject{currentmarker}{}%
\end{pgfscope}%
\begin{pgfscope}%
\pgfsys@transformshift{2.986675in}{1.130867in}%
\pgfsys@useobject{currentmarker}{}%
\end{pgfscope}%
\begin{pgfscope}%
\pgfsys@transformshift{2.986810in}{1.127921in}%
\pgfsys@useobject{currentmarker}{}%
\end{pgfscope}%
\begin{pgfscope}%
\pgfsys@transformshift{2.986943in}{1.126304in}%
\pgfsys@useobject{currentmarker}{}%
\end{pgfscope}%
\begin{pgfscope}%
\pgfsys@transformshift{2.986482in}{1.123528in}%
\pgfsys@useobject{currentmarker}{}%
\end{pgfscope}%
\begin{pgfscope}%
\pgfsys@transformshift{2.986175in}{1.122011in}%
\pgfsys@useobject{currentmarker}{}%
\end{pgfscope}%
\begin{pgfscope}%
\pgfsys@transformshift{2.984727in}{1.120275in}%
\pgfsys@useobject{currentmarker}{}%
\end{pgfscope}%
\begin{pgfscope}%
\pgfsys@transformshift{2.981102in}{1.117292in}%
\pgfsys@useobject{currentmarker}{}%
\end{pgfscope}%
\begin{pgfscope}%
\pgfsys@transformshift{2.976080in}{1.115162in}%
\pgfsys@useobject{currentmarker}{}%
\end{pgfscope}%
\begin{pgfscope}%
\pgfsys@transformshift{2.969965in}{1.113661in}%
\pgfsys@useobject{currentmarker}{}%
\end{pgfscope}%
\begin{pgfscope}%
\pgfsys@transformshift{2.966502in}{1.113720in}%
\pgfsys@useobject{currentmarker}{}%
\end{pgfscope}%
\begin{pgfscope}%
\pgfsys@transformshift{2.964695in}{1.114323in}%
\pgfsys@useobject{currentmarker}{}%
\end{pgfscope}%
\begin{pgfscope}%
\pgfsys@transformshift{2.961861in}{1.115119in}%
\pgfsys@useobject{currentmarker}{}%
\end{pgfscope}%
\begin{pgfscope}%
\pgfsys@transformshift{2.958150in}{1.115650in}%
\pgfsys@useobject{currentmarker}{}%
\end{pgfscope}%
\begin{pgfscope}%
\pgfsys@transformshift{2.953546in}{1.116259in}%
\pgfsys@useobject{currentmarker}{}%
\end{pgfscope}%
\begin{pgfscope}%
\pgfsys@transformshift{2.951001in}{1.116465in}%
\pgfsys@useobject{currentmarker}{}%
\end{pgfscope}%
\begin{pgfscope}%
\pgfsys@transformshift{2.949596in}{1.116472in}%
\pgfsys@useobject{currentmarker}{}%
\end{pgfscope}%
\begin{pgfscope}%
\pgfsys@transformshift{2.948835in}{1.116604in}%
\pgfsys@useobject{currentmarker}{}%
\end{pgfscope}%
\begin{pgfscope}%
\pgfsys@transformshift{2.947442in}{1.116634in}%
\pgfsys@useobject{currentmarker}{}%
\end{pgfscope}%
\begin{pgfscope}%
\pgfsys@transformshift{2.944953in}{1.116220in}%
\pgfsys@useobject{currentmarker}{}%
\end{pgfscope}%
\begin{pgfscope}%
\pgfsys@transformshift{2.941910in}{1.116402in}%
\pgfsys@useobject{currentmarker}{}%
\end{pgfscope}%
\begin{pgfscope}%
\pgfsys@transformshift{2.933744in}{1.116216in}%
\pgfsys@useobject{currentmarker}{}%
\end{pgfscope}%
\begin{pgfscope}%
\pgfsys@transformshift{2.924206in}{1.116223in}%
\pgfsys@useobject{currentmarker}{}%
\end{pgfscope}%
\begin{pgfscope}%
\pgfsys@transformshift{2.912050in}{1.114733in}%
\pgfsys@useobject{currentmarker}{}%
\end{pgfscope}%
\begin{pgfscope}%
\pgfsys@transformshift{2.899054in}{1.114963in}%
\pgfsys@useobject{currentmarker}{}%
\end{pgfscope}%
\begin{pgfscope}%
\pgfsys@transformshift{2.882053in}{1.115400in}%
\pgfsys@useobject{currentmarker}{}%
\end{pgfscope}%
\begin{pgfscope}%
\pgfsys@transformshift{2.863038in}{1.114926in}%
\pgfsys@useobject{currentmarker}{}%
\end{pgfscope}%
\begin{pgfscope}%
\pgfsys@transformshift{2.842907in}{1.112554in}%
\pgfsys@useobject{currentmarker}{}%
\end{pgfscope}%
\begin{pgfscope}%
\pgfsys@transformshift{2.822047in}{1.112088in}%
\pgfsys@useobject{currentmarker}{}%
\end{pgfscope}%
\begin{pgfscope}%
\pgfsys@transformshift{2.795206in}{1.110589in}%
\pgfsys@useobject{currentmarker}{}%
\end{pgfscope}%
\begin{pgfscope}%
\pgfsys@transformshift{2.766343in}{1.106586in}%
\pgfsys@useobject{currentmarker}{}%
\end{pgfscope}%
\begin{pgfscope}%
\pgfsys@transformshift{2.736384in}{1.099969in}%
\pgfsys@useobject{currentmarker}{}%
\end{pgfscope}%
\begin{pgfscope}%
\pgfsys@transformshift{2.705082in}{1.099117in}%
\pgfsys@useobject{currentmarker}{}%
\end{pgfscope}%
\begin{pgfscope}%
\pgfsys@transformshift{2.668378in}{1.094871in}%
\pgfsys@useobject{currentmarker}{}%
\end{pgfscope}%
\begin{pgfscope}%
\pgfsys@transformshift{2.632537in}{1.081900in}%
\pgfsys@useobject{currentmarker}{}%
\end{pgfscope}%
\begin{pgfscope}%
\pgfsys@transformshift{2.593376in}{1.074484in}%
\pgfsys@useobject{currentmarker}{}%
\end{pgfscope}%
\begin{pgfscope}%
\pgfsys@transformshift{2.553075in}{1.073439in}%
\pgfsys@useobject{currentmarker}{}%
\end{pgfscope}%
\begin{pgfscope}%
\pgfsys@transformshift{2.507328in}{1.071726in}%
\pgfsys@useobject{currentmarker}{}%
\end{pgfscope}%
\begin{pgfscope}%
\pgfsys@transformshift{2.482176in}{1.070550in}%
\pgfsys@useobject{currentmarker}{}%
\end{pgfscope}%
\begin{pgfscope}%
\pgfsys@transformshift{2.453837in}{1.068060in}%
\pgfsys@useobject{currentmarker}{}%
\end{pgfscope}%
\begin{pgfscope}%
\pgfsys@transformshift{2.438220in}{1.067091in}%
\pgfsys@useobject{currentmarker}{}%
\end{pgfscope}%
\begin{pgfscope}%
\pgfsys@transformshift{2.420886in}{1.066525in}%
\pgfsys@useobject{currentmarker}{}%
\end{pgfscope}%
\begin{pgfscope}%
\pgfsys@transformshift{2.403039in}{1.066912in}%
\pgfsys@useobject{currentmarker}{}%
\end{pgfscope}%
\begin{pgfscope}%
\pgfsys@transformshift{2.384208in}{1.065472in}%
\pgfsys@useobject{currentmarker}{}%
\end{pgfscope}%
\begin{pgfscope}%
\pgfsys@transformshift{2.373822in}{1.065660in}%
\pgfsys@useobject{currentmarker}{}%
\end{pgfscope}%
\begin{pgfscope}%
\pgfsys@transformshift{2.360467in}{1.065394in}%
\pgfsys@useobject{currentmarker}{}%
\end{pgfscope}%
\begin{pgfscope}%
\pgfsys@transformshift{2.353123in}{1.065221in}%
\pgfsys@useobject{currentmarker}{}%
\end{pgfscope}%
\begin{pgfscope}%
\pgfsys@transformshift{2.345161in}{1.064050in}%
\pgfsys@useobject{currentmarker}{}%
\end{pgfscope}%
\begin{pgfscope}%
\pgfsys@transformshift{2.340741in}{1.063816in}%
\pgfsys@useobject{currentmarker}{}%
\end{pgfscope}%
\begin{pgfscope}%
\pgfsys@transformshift{2.333435in}{1.063786in}%
\pgfsys@useobject{currentmarker}{}%
\end{pgfscope}%
\begin{pgfscope}%
\pgfsys@transformshift{2.329418in}{1.063701in}%
\pgfsys@useobject{currentmarker}{}%
\end{pgfscope}%
\begin{pgfscope}%
\pgfsys@transformshift{2.323758in}{1.062547in}%
\pgfsys@useobject{currentmarker}{}%
\end{pgfscope}%
\begin{pgfscope}%
\pgfsys@transformshift{2.320582in}{1.062509in}%
\pgfsys@useobject{currentmarker}{}%
\end{pgfscope}%
\begin{pgfscope}%
\pgfsys@transformshift{2.315502in}{1.062318in}%
\pgfsys@useobject{currentmarker}{}%
\end{pgfscope}%
\begin{pgfscope}%
\pgfsys@transformshift{2.309701in}{1.062102in}%
\pgfsys@useobject{currentmarker}{}%
\end{pgfscope}%
\begin{pgfscope}%
\pgfsys@transformshift{2.302436in}{1.060873in}%
\pgfsys@useobject{currentmarker}{}%
\end{pgfscope}%
\begin{pgfscope}%
\pgfsys@transformshift{2.298388in}{1.060686in}%
\pgfsys@useobject{currentmarker}{}%
\end{pgfscope}%
\begin{pgfscope}%
\pgfsys@transformshift{2.292907in}{1.060511in}%
\pgfsys@useobject{currentmarker}{}%
\end{pgfscope}%
\begin{pgfscope}%
\pgfsys@transformshift{2.286879in}{1.059872in}%
\pgfsys@useobject{currentmarker}{}%
\end{pgfscope}%
\begin{pgfscope}%
\pgfsys@transformshift{2.279770in}{1.059882in}%
\pgfsys@useobject{currentmarker}{}%
\end{pgfscope}%
\begin{pgfscope}%
\pgfsys@transformshift{2.271148in}{1.058557in}%
\pgfsys@useobject{currentmarker}{}%
\end{pgfscope}%
\begin{pgfscope}%
\pgfsys@transformshift{2.266361in}{1.058234in}%
\pgfsys@useobject{currentmarker}{}%
\end{pgfscope}%
\begin{pgfscope}%
\pgfsys@transformshift{2.260255in}{1.057831in}%
\pgfsys@useobject{currentmarker}{}%
\end{pgfscope}%
\begin{pgfscope}%
\pgfsys@transformshift{2.253629in}{1.057766in}%
\pgfsys@useobject{currentmarker}{}%
\end{pgfscope}%
\begin{pgfscope}%
\pgfsys@transformshift{2.245521in}{1.056611in}%
\pgfsys@useobject{currentmarker}{}%
\end{pgfscope}%
\begin{pgfscope}%
\pgfsys@transformshift{2.241020in}{1.056428in}%
\pgfsys@useobject{currentmarker}{}%
\end{pgfscope}%
\begin{pgfscope}%
\pgfsys@transformshift{2.233838in}{1.056467in}%
\pgfsys@useobject{currentmarker}{}%
\end{pgfscope}%
\begin{pgfscope}%
\pgfsys@transformshift{2.229910in}{1.056054in}%
\pgfsys@useobject{currentmarker}{}%
\end{pgfscope}%
\begin{pgfscope}%
\pgfsys@transformshift{2.224609in}{1.055758in}%
\pgfsys@useobject{currentmarker}{}%
\end{pgfscope}%
\begin{pgfscope}%
\pgfsys@transformshift{2.221689in}{1.055723in}%
\pgfsys@useobject{currentmarker}{}%
\end{pgfscope}%
\begin{pgfscope}%
\pgfsys@transformshift{2.216983in}{1.055061in}%
\pgfsys@useobject{currentmarker}{}%
\end{pgfscope}%
\begin{pgfscope}%
\pgfsys@transformshift{2.211352in}{1.054370in}%
\pgfsys@useobject{currentmarker}{}%
\end{pgfscope}%
\begin{pgfscope}%
\pgfsys@transformshift{2.204872in}{1.053637in}%
\pgfsys@useobject{currentmarker}{}%
\end{pgfscope}%
\begin{pgfscope}%
\pgfsys@transformshift{2.201296in}{1.053359in}%
\pgfsys@useobject{currentmarker}{}%
\end{pgfscope}%
\begin{pgfscope}%
\pgfsys@transformshift{2.194258in}{1.053600in}%
\pgfsys@useobject{currentmarker}{}%
\end{pgfscope}%
\begin{pgfscope}%
\pgfsys@transformshift{2.190388in}{1.053429in}%
\pgfsys@useobject{currentmarker}{}%
\end{pgfscope}%
\begin{pgfscope}%
\pgfsys@transformshift{2.184835in}{1.052497in}%
\pgfsys@useobject{currentmarker}{}%
\end{pgfscope}%
\begin{pgfscope}%
\pgfsys@transformshift{2.181743in}{1.052306in}%
\pgfsys@useobject{currentmarker}{}%
\end{pgfscope}%
\begin{pgfscope}%
\pgfsys@transformshift{2.175467in}{1.052207in}%
\pgfsys@useobject{currentmarker}{}%
\end{pgfscope}%
\begin{pgfscope}%
\pgfsys@transformshift{2.172026in}{1.051935in}%
\pgfsys@useobject{currentmarker}{}%
\end{pgfscope}%
\begin{pgfscope}%
\pgfsys@transformshift{2.166236in}{1.051175in}%
\pgfsys@useobject{currentmarker}{}%
\end{pgfscope}%
\begin{pgfscope}%
\pgfsys@transformshift{2.159855in}{1.050733in}%
\pgfsys@useobject{currentmarker}{}%
\end{pgfscope}%
\begin{pgfscope}%
\pgfsys@transformshift{2.151500in}{1.050443in}%
\pgfsys@useobject{currentmarker}{}%
\end{pgfscope}%
\begin{pgfscope}%
\pgfsys@transformshift{2.142577in}{1.049331in}%
\pgfsys@useobject{currentmarker}{}%
\end{pgfscope}%
\begin{pgfscope}%
\pgfsys@transformshift{2.132090in}{1.046941in}%
\pgfsys@useobject{currentmarker}{}%
\end{pgfscope}%
\begin{pgfscope}%
\pgfsys@transformshift{2.126176in}{1.046805in}%
\pgfsys@useobject{currentmarker}{}%
\end{pgfscope}%
\begin{pgfscope}%
\pgfsys@transformshift{2.118284in}{1.045834in}%
\pgfsys@useobject{currentmarker}{}%
\end{pgfscope}%
\begin{pgfscope}%
\pgfsys@transformshift{2.113914in}{1.045682in}%
\pgfsys@useobject{currentmarker}{}%
\end{pgfscope}%
\begin{pgfscope}%
\pgfsys@transformshift{2.108219in}{1.044546in}%
\pgfsys@useobject{currentmarker}{}%
\end{pgfscope}%
\begin{pgfscope}%
\pgfsys@transformshift{2.105027in}{1.044637in}%
\pgfsys@useobject{currentmarker}{}%
\end{pgfscope}%
\begin{pgfscope}%
\pgfsys@transformshift{2.100044in}{1.044254in}%
\pgfsys@useobject{currentmarker}{}%
\end{pgfscope}%
\begin{pgfscope}%
\pgfsys@transformshift{2.097296in}{1.044247in}%
\pgfsys@useobject{currentmarker}{}%
\end{pgfscope}%
\begin{pgfscope}%
\pgfsys@transformshift{2.093552in}{1.043341in}%
\pgfsys@useobject{currentmarker}{}%
\end{pgfscope}%
\begin{pgfscope}%
\pgfsys@transformshift{2.091437in}{1.043211in}%
\pgfsys@useobject{currentmarker}{}%
\end{pgfscope}%
\begin{pgfscope}%
\pgfsys@transformshift{2.087577in}{1.042828in}%
\pgfsys@useobject{currentmarker}{}%
\end{pgfscope}%
\begin{pgfscope}%
\pgfsys@transformshift{2.083243in}{1.042517in}%
\pgfsys@useobject{currentmarker}{}%
\end{pgfscope}%
\begin{pgfscope}%
\pgfsys@transformshift{2.078073in}{1.041340in}%
\pgfsys@useobject{currentmarker}{}%
\end{pgfscope}%
\begin{pgfscope}%
\pgfsys@transformshift{2.071250in}{1.040341in}%
\pgfsys@useobject{currentmarker}{}%
\end{pgfscope}%
\begin{pgfscope}%
\pgfsys@transformshift{2.063472in}{1.040008in}%
\pgfsys@useobject{currentmarker}{}%
\end{pgfscope}%
\begin{pgfscope}%
\pgfsys@transformshift{2.053886in}{1.039676in}%
\pgfsys@useobject{currentmarker}{}%
\end{pgfscope}%
\begin{pgfscope}%
\pgfsys@transformshift{2.043615in}{1.038727in}%
\pgfsys@useobject{currentmarker}{}%
\end{pgfscope}%
\begin{pgfscope}%
\pgfsys@transformshift{2.030705in}{1.038179in}%
\pgfsys@useobject{currentmarker}{}%
\end{pgfscope}%
\begin{pgfscope}%
\pgfsys@transformshift{2.023632in}{1.037480in}%
\pgfsys@useobject{currentmarker}{}%
\end{pgfscope}%
\begin{pgfscope}%
\pgfsys@transformshift{2.015592in}{1.036806in}%
\pgfsys@useobject{currentmarker}{}%
\end{pgfscope}%
\begin{pgfscope}%
\pgfsys@transformshift{2.006970in}{1.035984in}%
\pgfsys@useobject{currentmarker}{}%
\end{pgfscope}%
\begin{pgfscope}%
\pgfsys@transformshift{1.996726in}{1.034569in}%
\pgfsys@useobject{currentmarker}{}%
\end{pgfscope}%
\begin{pgfscope}%
\pgfsys@transformshift{1.985934in}{1.033900in}%
\pgfsys@useobject{currentmarker}{}%
\end{pgfscope}%
\begin{pgfscope}%
\pgfsys@transformshift{1.973016in}{1.033135in}%
\pgfsys@useobject{currentmarker}{}%
\end{pgfscope}%
\begin{pgfscope}%
\pgfsys@transformshift{1.965946in}{1.032316in}%
\pgfsys@useobject{currentmarker}{}%
\end{pgfscope}%
\begin{pgfscope}%
\pgfsys@transformshift{1.956060in}{1.030600in}%
\pgfsys@useobject{currentmarker}{}%
\end{pgfscope}%
\begin{pgfscope}%
\pgfsys@transformshift{1.944763in}{1.029835in}%
\pgfsys@useobject{currentmarker}{}%
\end{pgfscope}%
\begin{pgfscope}%
\pgfsys@transformshift{1.930949in}{1.028777in}%
\pgfsys@useobject{currentmarker}{}%
\end{pgfscope}%
\begin{pgfscope}%
\pgfsys@transformshift{1.916619in}{1.028347in}%
\pgfsys@useobject{currentmarker}{}%
\end{pgfscope}%
\begin{pgfscope}%
\pgfsys@transformshift{1.901960in}{1.026365in}%
\pgfsys@useobject{currentmarker}{}%
\end{pgfscope}%
\begin{pgfscope}%
\pgfsys@transformshift{1.886681in}{1.025401in}%
\pgfsys@useobject{currentmarker}{}%
\end{pgfscope}%
\begin{pgfscope}%
\pgfsys@transformshift{1.868161in}{1.025525in}%
\pgfsys@useobject{currentmarker}{}%
\end{pgfscope}%
\begin{pgfscope}%
\pgfsys@transformshift{1.858022in}{1.024551in}%
\pgfsys@useobject{currentmarker}{}%
\end{pgfscope}%
\begin{pgfscope}%
\pgfsys@transformshift{1.845662in}{1.022908in}%
\pgfsys@useobject{currentmarker}{}%
\end{pgfscope}%
\begin{pgfscope}%
\pgfsys@transformshift{1.838810in}{1.022626in}%
\pgfsys@useobject{currentmarker}{}%
\end{pgfscope}%
\begin{pgfscope}%
\pgfsys@transformshift{1.830507in}{1.021690in}%
\pgfsys@useobject{currentmarker}{}%
\end{pgfscope}%
\begin{pgfscope}%
\pgfsys@transformshift{1.821228in}{1.020199in}%
\pgfsys@useobject{currentmarker}{}%
\end{pgfscope}%
\begin{pgfscope}%
\pgfsys@transformshift{1.809543in}{1.018085in}%
\pgfsys@useobject{currentmarker}{}%
\end{pgfscope}%
\begin{pgfscope}%
\pgfsys@transformshift{1.797062in}{1.017032in}%
\pgfsys@useobject{currentmarker}{}%
\end{pgfscope}%
\begin{pgfscope}%
\pgfsys@transformshift{1.782408in}{1.014986in}%
\pgfsys@useobject{currentmarker}{}%
\end{pgfscope}%
\begin{pgfscope}%
\pgfsys@transformshift{1.766957in}{1.011761in}%
\pgfsys@useobject{currentmarker}{}%
\end{pgfscope}%
\begin{pgfscope}%
\pgfsys@transformshift{1.749242in}{1.008416in}%
\pgfsys@useobject{currentmarker}{}%
\end{pgfscope}%
\begin{pgfscope}%
\pgfsys@transformshift{1.730626in}{1.007164in}%
\pgfsys@useobject{currentmarker}{}%
\end{pgfscope}%
\begin{pgfscope}%
\pgfsys@transformshift{1.709948in}{1.005627in}%
\pgfsys@useobject{currentmarker}{}%
\end{pgfscope}%
\begin{pgfscope}%
\pgfsys@transformshift{1.688763in}{1.002096in}%
\pgfsys@useobject{currentmarker}{}%
\end{pgfscope}%
\begin{pgfscope}%
\pgfsys@transformshift{1.665407in}{0.999343in}%
\pgfsys@useobject{currentmarker}{}%
\end{pgfscope}%
\begin{pgfscope}%
\pgfsys@transformshift{1.652498in}{0.998534in}%
\pgfsys@useobject{currentmarker}{}%
\end{pgfscope}%
\begin{pgfscope}%
\pgfsys@transformshift{1.638454in}{0.996722in}%
\pgfsys@useobject{currentmarker}{}%
\end{pgfscope}%
\begin{pgfscope}%
\pgfsys@transformshift{1.623687in}{0.993426in}%
\pgfsys@useobject{currentmarker}{}%
\end{pgfscope}%
\begin{pgfscope}%
\pgfsys@transformshift{1.607442in}{0.991036in}%
\pgfsys@useobject{currentmarker}{}%
\end{pgfscope}%
\begin{pgfscope}%
\pgfsys@transformshift{1.590226in}{0.990868in}%
\pgfsys@useobject{currentmarker}{}%
\end{pgfscope}%
\begin{pgfscope}%
\pgfsys@transformshift{1.571809in}{0.988659in}%
\pgfsys@useobject{currentmarker}{}%
\end{pgfscope}%
\begin{pgfscope}%
\pgfsys@transformshift{1.552171in}{0.987168in}%
\pgfsys@useobject{currentmarker}{}%
\end{pgfscope}%
\begin{pgfscope}%
\pgfsys@transformshift{1.531516in}{0.983932in}%
\pgfsys@useobject{currentmarker}{}%
\end{pgfscope}%
\begin{pgfscope}%
\pgfsys@transformshift{1.510141in}{0.983199in}%
\pgfsys@useobject{currentmarker}{}%
\end{pgfscope}%
\begin{pgfscope}%
\pgfsys@transformshift{1.487431in}{0.981669in}%
\pgfsys@useobject{currentmarker}{}%
\end{pgfscope}%
\begin{pgfscope}%
\pgfsys@transformshift{1.464046in}{0.981687in}%
\pgfsys@useobject{currentmarker}{}%
\end{pgfscope}%
\begin{pgfscope}%
\pgfsys@transformshift{1.439548in}{0.975844in}%
\pgfsys@useobject{currentmarker}{}%
\end{pgfscope}%
\begin{pgfscope}%
\pgfsys@transformshift{1.425700in}{0.975509in}%
\pgfsys@useobject{currentmarker}{}%
\end{pgfscope}%
\begin{pgfscope}%
\pgfsys@transformshift{1.410310in}{0.976155in}%
\pgfsys@useobject{currentmarker}{}%
\end{pgfscope}%
\begin{pgfscope}%
\pgfsys@transformshift{1.402028in}{0.977944in}%
\pgfsys@useobject{currentmarker}{}%
\end{pgfscope}%
\begin{pgfscope}%
\pgfsys@transformshift{1.398341in}{0.980794in}%
\pgfsys@useobject{currentmarker}{}%
\end{pgfscope}%
\begin{pgfscope}%
\pgfsys@transformshift{1.396077in}{0.987512in}%
\pgfsys@useobject{currentmarker}{}%
\end{pgfscope}%
\begin{pgfscope}%
\pgfsys@transformshift{1.396250in}{0.995539in}%
\pgfsys@useobject{currentmarker}{}%
\end{pgfscope}%
\begin{pgfscope}%
\pgfsys@transformshift{1.395137in}{1.004853in}%
\pgfsys@useobject{currentmarker}{}%
\end{pgfscope}%
\begin{pgfscope}%
\pgfsys@transformshift{1.396513in}{1.015699in}%
\pgfsys@useobject{currentmarker}{}%
\end{pgfscope}%
\begin{pgfscope}%
\pgfsys@transformshift{1.396476in}{1.027673in}%
\pgfsys@useobject{currentmarker}{}%
\end{pgfscope}%
\begin{pgfscope}%
\pgfsys@transformshift{1.396464in}{1.041085in}%
\pgfsys@useobject{currentmarker}{}%
\end{pgfscope}%
\begin{pgfscope}%
\pgfsys@transformshift{1.396840in}{1.048453in}%
\pgfsys@useobject{currentmarker}{}%
\end{pgfscope}%
\begin{pgfscope}%
\pgfsys@transformshift{1.396268in}{1.057544in}%
\pgfsys@useobject{currentmarker}{}%
\end{pgfscope}%
\begin{pgfscope}%
\pgfsys@transformshift{1.395411in}{1.067862in}%
\pgfsys@useobject{currentmarker}{}%
\end{pgfscope}%
\begin{pgfscope}%
\pgfsys@transformshift{1.393472in}{1.078675in}%
\pgfsys@useobject{currentmarker}{}%
\end{pgfscope}%
\begin{pgfscope}%
\pgfsys@transformshift{1.397328in}{1.092056in}%
\pgfsys@useobject{currentmarker}{}%
\end{pgfscope}%
\begin{pgfscope}%
\pgfsys@transformshift{1.396379in}{1.099655in}%
\pgfsys@useobject{currentmarker}{}%
\end{pgfscope}%
\begin{pgfscope}%
\pgfsys@transformshift{1.394977in}{1.107664in}%
\pgfsys@useobject{currentmarker}{}%
\end{pgfscope}%
\begin{pgfscope}%
\pgfsys@transformshift{1.398372in}{1.119273in}%
\pgfsys@useobject{currentmarker}{}%
\end{pgfscope}%
\begin{pgfscope}%
\pgfsys@transformshift{1.394757in}{1.132779in}%
\pgfsys@useobject{currentmarker}{}%
\end{pgfscope}%
\begin{pgfscope}%
\pgfsys@transformshift{1.393578in}{1.147447in}%
\pgfsys@useobject{currentmarker}{}%
\end{pgfscope}%
\begin{pgfscope}%
\pgfsys@transformshift{1.394229in}{1.167748in}%
\pgfsys@useobject{currentmarker}{}%
\end{pgfscope}%
\begin{pgfscope}%
\pgfsys@transformshift{1.397874in}{1.188326in}%
\pgfsys@useobject{currentmarker}{}%
\end{pgfscope}%
\begin{pgfscope}%
\pgfsys@transformshift{1.392326in}{1.209234in}%
\pgfsys@useobject{currentmarker}{}%
\end{pgfscope}%
\begin{pgfscope}%
\pgfsys@transformshift{1.392644in}{1.221127in}%
\pgfsys@useobject{currentmarker}{}%
\end{pgfscope}%
\begin{pgfscope}%
\pgfsys@transformshift{1.396073in}{1.233964in}%
\pgfsys@useobject{currentmarker}{}%
\end{pgfscope}%
\begin{pgfscope}%
\pgfsys@transformshift{1.394211in}{1.252246in}%
\pgfsys@useobject{currentmarker}{}%
\end{pgfscope}%
\begin{pgfscope}%
\pgfsys@transformshift{1.393650in}{1.271278in}%
\pgfsys@useobject{currentmarker}{}%
\end{pgfscope}%
\begin{pgfscope}%
\pgfsys@transformshift{1.393328in}{1.294803in}%
\pgfsys@useobject{currentmarker}{}%
\end{pgfscope}%
\begin{pgfscope}%
\pgfsys@transformshift{1.394428in}{1.307697in}%
\pgfsys@useobject{currentmarker}{}%
\end{pgfscope}%
\begin{pgfscope}%
\pgfsys@transformshift{1.389796in}{1.321852in}%
\pgfsys@useobject{currentmarker}{}%
\end{pgfscope}%
\begin{pgfscope}%
\pgfsys@transformshift{1.387420in}{1.339247in}%
\pgfsys@useobject{currentmarker}{}%
\end{pgfscope}%
\begin{pgfscope}%
\pgfsys@transformshift{1.391070in}{1.357685in}%
\pgfsys@useobject{currentmarker}{}%
\end{pgfscope}%
\begin{pgfscope}%
\pgfsys@transformshift{1.382763in}{1.379357in}%
\pgfsys@useobject{currentmarker}{}%
\end{pgfscope}%
\begin{pgfscope}%
\pgfsys@transformshift{1.380046in}{1.391830in}%
\pgfsys@useobject{currentmarker}{}%
\end{pgfscope}%
\begin{pgfscope}%
\pgfsys@transformshift{1.384208in}{1.408390in}%
\pgfsys@useobject{currentmarker}{}%
\end{pgfscope}%
\begin{pgfscope}%
\pgfsys@transformshift{1.380005in}{1.428540in}%
\pgfsys@useobject{currentmarker}{}%
\end{pgfscope}%
\begin{pgfscope}%
\pgfsys@transformshift{1.378766in}{1.449556in}%
\pgfsys@useobject{currentmarker}{}%
\end{pgfscope}%
\begin{pgfscope}%
\pgfsys@transformshift{1.380161in}{1.475397in}%
\pgfsys@useobject{currentmarker}{}%
\end{pgfscope}%
\begin{pgfscope}%
\pgfsys@transformshift{1.383007in}{1.501583in}%
\pgfsys@useobject{currentmarker}{}%
\end{pgfscope}%
\begin{pgfscope}%
\pgfsys@transformshift{1.375757in}{1.527666in}%
\pgfsys@useobject{currentmarker}{}%
\end{pgfscope}%
\begin{pgfscope}%
\pgfsys@transformshift{1.373804in}{1.542427in}%
\pgfsys@useobject{currentmarker}{}%
\end{pgfscope}%
\begin{pgfscope}%
\pgfsys@transformshift{1.378805in}{1.558238in}%
\pgfsys@useobject{currentmarker}{}%
\end{pgfscope}%
\begin{pgfscope}%
\pgfsys@transformshift{1.374190in}{1.579562in}%
\pgfsys@useobject{currentmarker}{}%
\end{pgfscope}%
\begin{pgfscope}%
\pgfsys@transformshift{1.375529in}{1.591486in}%
\pgfsys@useobject{currentmarker}{}%
\end{pgfscope}%
\begin{pgfscope}%
\pgfsys@transformshift{1.380043in}{1.607482in}%
\pgfsys@useobject{currentmarker}{}%
\end{pgfscope}%
\begin{pgfscope}%
\pgfsys@transformshift{1.379780in}{1.625471in}%
\pgfsys@useobject{currentmarker}{}%
\end{pgfscope}%
\begin{pgfscope}%
\pgfsys@transformshift{1.375280in}{1.643785in}%
\pgfsys@useobject{currentmarker}{}%
\end{pgfscope}%
\begin{pgfscope}%
\pgfsys@transformshift{1.375918in}{1.665784in}%
\pgfsys@useobject{currentmarker}{}%
\end{pgfscope}%
\begin{pgfscope}%
\pgfsys@transformshift{1.381574in}{1.688257in}%
\pgfsys@useobject{currentmarker}{}%
\end{pgfscope}%
\begin{pgfscope}%
\pgfsys@transformshift{1.376546in}{1.714166in}%
\pgfsys@useobject{currentmarker}{}%
\end{pgfscope}%
\begin{pgfscope}%
\pgfsys@transformshift{1.375745in}{1.728659in}%
\pgfsys@useobject{currentmarker}{}%
\end{pgfscope}%
\begin{pgfscope}%
\pgfsys@transformshift{1.380391in}{1.745881in}%
\pgfsys@useobject{currentmarker}{}%
\end{pgfscope}%
\begin{pgfscope}%
\pgfsys@transformshift{1.377328in}{1.766670in}%
\pgfsys@useobject{currentmarker}{}%
\end{pgfscope}%
\begin{pgfscope}%
\pgfsys@transformshift{1.375184in}{1.778026in}%
\pgfsys@useobject{currentmarker}{}%
\end{pgfscope}%
\begin{pgfscope}%
\pgfsys@transformshift{1.376712in}{1.793091in}%
\pgfsys@useobject{currentmarker}{}%
\end{pgfscope}%
\begin{pgfscope}%
\pgfsys@transformshift{1.377207in}{1.801404in}%
\pgfsys@useobject{currentmarker}{}%
\end{pgfscope}%
\begin{pgfscope}%
\pgfsys@transformshift{1.374124in}{1.812415in}%
\pgfsys@useobject{currentmarker}{}%
\end{pgfscope}%
\begin{pgfscope}%
\pgfsys@transformshift{1.372816in}{1.818566in}%
\pgfsys@useobject{currentmarker}{}%
\end{pgfscope}%
\begin{pgfscope}%
\pgfsys@transformshift{1.375147in}{1.828578in}%
\pgfsys@useobject{currentmarker}{}%
\end{pgfscope}%
\begin{pgfscope}%
\pgfsys@transformshift{1.374227in}{1.840085in}%
\pgfsys@useobject{currentmarker}{}%
\end{pgfscope}%
\begin{pgfscope}%
\pgfsys@transformshift{1.373136in}{1.846340in}%
\pgfsys@useobject{currentmarker}{}%
\end{pgfscope}%
\begin{pgfscope}%
\pgfsys@transformshift{1.372960in}{1.856062in}%
\pgfsys@useobject{currentmarker}{}%
\end{pgfscope}%
\begin{pgfscope}%
\pgfsys@transformshift{1.375508in}{1.865964in}%
\pgfsys@useobject{currentmarker}{}%
\end{pgfscope}%
\begin{pgfscope}%
\pgfsys@transformshift{1.372327in}{1.879295in}%
\pgfsys@useobject{currentmarker}{}%
\end{pgfscope}%
\begin{pgfscope}%
\pgfsys@transformshift{1.371587in}{1.886797in}%
\pgfsys@useobject{currentmarker}{}%
\end{pgfscope}%
\begin{pgfscope}%
\pgfsys@transformshift{1.374220in}{1.897015in}%
\pgfsys@useobject{currentmarker}{}%
\end{pgfscope}%
\begin{pgfscope}%
\pgfsys@transformshift{1.371277in}{1.911480in}%
\pgfsys@useobject{currentmarker}{}%
\end{pgfscope}%
\begin{pgfscope}%
\pgfsys@transformshift{1.368372in}{1.926447in}%
\pgfsys@useobject{currentmarker}{}%
\end{pgfscope}%
\begin{pgfscope}%
\pgfsys@transformshift{1.374884in}{1.946460in}%
\pgfsys@useobject{currentmarker}{}%
\end{pgfscope}%
\begin{pgfscope}%
\pgfsys@transformshift{1.372591in}{1.968555in}%
\pgfsys@useobject{currentmarker}{}%
\end{pgfscope}%
\begin{pgfscope}%
\pgfsys@transformshift{1.369257in}{1.980309in}%
\pgfsys@useobject{currentmarker}{}%
\end{pgfscope}%
\begin{pgfscope}%
\pgfsys@transformshift{1.369899in}{1.995982in}%
\pgfsys@useobject{currentmarker}{}%
\end{pgfscope}%
\begin{pgfscope}%
\pgfsys@transformshift{1.374934in}{2.013106in}%
\pgfsys@useobject{currentmarker}{}%
\end{pgfscope}%
\begin{pgfscope}%
\pgfsys@transformshift{1.369373in}{2.034824in}%
\pgfsys@useobject{currentmarker}{}%
\end{pgfscope}%
\begin{pgfscope}%
\pgfsys@transformshift{1.370474in}{2.047105in}%
\pgfsys@useobject{currentmarker}{}%
\end{pgfscope}%
\begin{pgfscope}%
\pgfsys@transformshift{1.374840in}{2.064321in}%
\pgfsys@useobject{currentmarker}{}%
\end{pgfscope}%
\begin{pgfscope}%
\pgfsys@transformshift{1.375718in}{2.083022in}%
\pgfsys@useobject{currentmarker}{}%
\end{pgfscope}%
\begin{pgfscope}%
\pgfsys@transformshift{1.371004in}{2.101679in}%
\pgfsys@useobject{currentmarker}{}%
\end{pgfscope}%
\begin{pgfscope}%
\pgfsys@transformshift{1.370064in}{2.122357in}%
\pgfsys@useobject{currentmarker}{}%
\end{pgfscope}%
\begin{pgfscope}%
\pgfsys@transformshift{1.373719in}{2.144057in}%
\pgfsys@useobject{currentmarker}{}%
\end{pgfscope}%
\begin{pgfscope}%
\pgfsys@transformshift{1.368271in}{2.170929in}%
\pgfsys@useobject{currentmarker}{}%
\end{pgfscope}%
\begin{pgfscope}%
\pgfsys@transformshift{1.367734in}{2.186000in}%
\pgfsys@useobject{currentmarker}{}%
\end{pgfscope}%
\begin{pgfscope}%
\pgfsys@transformshift{1.363396in}{2.203532in}%
\pgfsys@useobject{currentmarker}{}%
\end{pgfscope}%
\begin{pgfscope}%
\pgfsys@transformshift{1.366217in}{2.213056in}%
\pgfsys@useobject{currentmarker}{}%
\end{pgfscope}%
\begin{pgfscope}%
\pgfsys@transformshift{1.362780in}{2.228093in}%
\pgfsys@useobject{currentmarker}{}%
\end{pgfscope}%
\begin{pgfscope}%
\pgfsys@transformshift{1.362827in}{2.236576in}%
\pgfsys@useobject{currentmarker}{}%
\end{pgfscope}%
\begin{pgfscope}%
\pgfsys@transformshift{1.365293in}{2.249551in}%
\pgfsys@useobject{currentmarker}{}%
\end{pgfscope}%
\begin{pgfscope}%
\pgfsys@transformshift{1.365416in}{2.256814in}%
\pgfsys@useobject{currentmarker}{}%
\end{pgfscope}%
\begin{pgfscope}%
\pgfsys@transformshift{1.362957in}{2.265442in}%
\pgfsys@useobject{currentmarker}{}%
\end{pgfscope}%
\begin{pgfscope}%
\pgfsys@transformshift{1.363239in}{2.274874in}%
\pgfsys@useobject{currentmarker}{}%
\end{pgfscope}%
\begin{pgfscope}%
\pgfsys@transformshift{1.366751in}{2.285839in}%
\pgfsys@useobject{currentmarker}{}%
\end{pgfscope}%
\begin{pgfscope}%
\pgfsys@transformshift{1.365038in}{2.300051in}%
\pgfsys@useobject{currentmarker}{}%
\end{pgfscope}%
\begin{pgfscope}%
\pgfsys@transformshift{1.360832in}{2.315061in}%
\pgfsys@useobject{currentmarker}{}%
\end{pgfscope}%
\begin{pgfscope}%
\pgfsys@transformshift{1.362131in}{2.333017in}%
\pgfsys@useobject{currentmarker}{}%
\end{pgfscope}%
\begin{pgfscope}%
\pgfsys@transformshift{1.365608in}{2.351284in}%
\pgfsys@useobject{currentmarker}{}%
\end{pgfscope}%
\begin{pgfscope}%
\pgfsys@transformshift{1.360549in}{2.372266in}%
\pgfsys@useobject{currentmarker}{}%
\end{pgfscope}%
\begin{pgfscope}%
\pgfsys@transformshift{1.360570in}{2.384136in}%
\pgfsys@useobject{currentmarker}{}%
\end{pgfscope}%
\begin{pgfscope}%
\pgfsys@transformshift{1.362989in}{2.398576in}%
\pgfsys@useobject{currentmarker}{}%
\end{pgfscope}%
\begin{pgfscope}%
\pgfsys@transformshift{1.360868in}{2.416192in}%
\pgfsys@useobject{currentmarker}{}%
\end{pgfscope}%
\begin{pgfscope}%
\pgfsys@transformshift{1.356419in}{2.433979in}%
\pgfsys@useobject{currentmarker}{}%
\end{pgfscope}%
\begin{pgfscope}%
\pgfsys@transformshift{1.356195in}{2.455486in}%
\pgfsys@useobject{currentmarker}{}%
\end{pgfscope}%
\begin{pgfscope}%
\pgfsys@transformshift{1.357847in}{2.467199in}%
\pgfsys@useobject{currentmarker}{}%
\end{pgfscope}%
\begin{pgfscope}%
\pgfsys@transformshift{1.352662in}{2.483072in}%
\pgfsys@useobject{currentmarker}{}%
\end{pgfscope}%
\begin{pgfscope}%
\pgfsys@transformshift{1.352100in}{2.492239in}%
\pgfsys@useobject{currentmarker}{}%
\end{pgfscope}%
\begin{pgfscope}%
\pgfsys@transformshift{1.354940in}{2.505303in}%
\pgfsys@useobject{currentmarker}{}%
\end{pgfscope}%
\begin{pgfscope}%
\pgfsys@transformshift{1.353838in}{2.512573in}%
\pgfsys@useobject{currentmarker}{}%
\end{pgfscope}%
\begin{pgfscope}%
\pgfsys@transformshift{1.353096in}{2.516549in}%
\pgfsys@useobject{currentmarker}{}%
\end{pgfscope}%
\begin{pgfscope}%
\pgfsys@transformshift{1.351991in}{2.521463in}%
\pgfsys@useobject{currentmarker}{}%
\end{pgfscope}%
\begin{pgfscope}%
\pgfsys@transformshift{1.351468in}{2.530382in}%
\pgfsys@useobject{currentmarker}{}%
\end{pgfscope}%
\begin{pgfscope}%
\pgfsys@transformshift{1.352686in}{2.539723in}%
\pgfsys@useobject{currentmarker}{}%
\end{pgfscope}%
\begin{pgfscope}%
\pgfsys@transformshift{1.348897in}{2.552016in}%
\pgfsys@useobject{currentmarker}{}%
\end{pgfscope}%
\begin{pgfscope}%
\pgfsys@transformshift{1.348823in}{2.559091in}%
\pgfsys@useobject{currentmarker}{}%
\end{pgfscope}%
\begin{pgfscope}%
\pgfsys@transformshift{1.351197in}{2.568104in}%
\pgfsys@useobject{currentmarker}{}%
\end{pgfscope}%
\begin{pgfscope}%
\pgfsys@transformshift{1.349014in}{2.580602in}%
\pgfsys@useobject{currentmarker}{}%
\end{pgfscope}%
\begin{pgfscope}%
\pgfsys@transformshift{1.347389in}{2.587388in}%
\pgfsys@useobject{currentmarker}{}%
\end{pgfscope}%
\begin{pgfscope}%
\pgfsys@transformshift{1.347882in}{2.597905in}%
\pgfsys@useobject{currentmarker}{}%
\end{pgfscope}%
\begin{pgfscope}%
\pgfsys@transformshift{1.349257in}{2.609005in}%
\pgfsys@useobject{currentmarker}{}%
\end{pgfscope}%
\begin{pgfscope}%
\pgfsys@transformshift{1.345054in}{2.623272in}%
\pgfsys@useobject{currentmarker}{}%
\end{pgfscope}%
\begin{pgfscope}%
\pgfsys@transformshift{1.345000in}{2.631451in}%
\pgfsys@useobject{currentmarker}{}%
\end{pgfscope}%
\begin{pgfscope}%
\pgfsys@transformshift{1.347961in}{2.643143in}%
\pgfsys@useobject{currentmarker}{}%
\end{pgfscope}%
\begin{pgfscope}%
\pgfsys@transformshift{1.345828in}{2.655702in}%
\pgfsys@useobject{currentmarker}{}%
\end{pgfscope}%
\begin{pgfscope}%
\pgfsys@transformshift{1.342499in}{2.668646in}%
\pgfsys@useobject{currentmarker}{}%
\end{pgfscope}%
\begin{pgfscope}%
\pgfsys@transformshift{1.341404in}{2.683713in}%
\pgfsys@useobject{currentmarker}{}%
\end{pgfscope}%
\begin{pgfscope}%
\pgfsys@transformshift{1.342752in}{2.691911in}%
\pgfsys@useobject{currentmarker}{}%
\end{pgfscope}%
\begin{pgfscope}%
\pgfsys@transformshift{1.338473in}{2.703886in}%
\pgfsys@useobject{currentmarker}{}%
\end{pgfscope}%
\begin{pgfscope}%
\pgfsys@transformshift{1.337897in}{2.710856in}%
\pgfsys@useobject{currentmarker}{}%
\end{pgfscope}%
\begin{pgfscope}%
\pgfsys@transformshift{1.340732in}{2.721140in}%
\pgfsys@useobject{currentmarker}{}%
\end{pgfscope}%
\begin{pgfscope}%
\pgfsys@transformshift{1.339970in}{2.726958in}%
\pgfsys@useobject{currentmarker}{}%
\end{pgfscope}%
\begin{pgfscope}%
\pgfsys@transformshift{1.339116in}{2.730069in}%
\pgfsys@useobject{currentmarker}{}%
\end{pgfscope}%
\begin{pgfscope}%
\pgfsys@transformshift{1.338927in}{2.735778in}%
\pgfsys@useobject{currentmarker}{}%
\end{pgfscope}%
\begin{pgfscope}%
\pgfsys@transformshift{1.340098in}{2.741953in}%
\pgfsys@useobject{currentmarker}{}%
\end{pgfscope}%
\begin{pgfscope}%
\pgfsys@transformshift{1.336688in}{2.752359in}%
\pgfsys@useobject{currentmarker}{}%
\end{pgfscope}%
\begin{pgfscope}%
\pgfsys@transformshift{1.335820in}{2.763733in}%
\pgfsys@useobject{currentmarker}{}%
\end{pgfscope}%
\begin{pgfscope}%
\pgfsys@transformshift{1.339168in}{2.778984in}%
\pgfsys@useobject{currentmarker}{}%
\end{pgfscope}%
\begin{pgfscope}%
\pgfsys@transformshift{1.338395in}{2.787538in}%
\pgfsys@useobject{currentmarker}{}%
\end{pgfscope}%
\begin{pgfscope}%
\pgfsys@transformshift{1.337172in}{2.792100in}%
\pgfsys@useobject{currentmarker}{}%
\end{pgfscope}%
\begin{pgfscope}%
\pgfsys@transformshift{1.336212in}{2.798761in}%
\pgfsys@useobject{currentmarker}{}%
\end{pgfscope}%
\begin{pgfscope}%
\pgfsys@transformshift{1.338348in}{2.807704in}%
\pgfsys@useobject{currentmarker}{}%
\end{pgfscope}%
\begin{pgfscope}%
\pgfsys@transformshift{1.335694in}{2.820464in}%
\pgfsys@useobject{currentmarker}{}%
\end{pgfscope}%
\begin{pgfscope}%
\pgfsys@transformshift{1.334179in}{2.827471in}%
\pgfsys@useobject{currentmarker}{}%
\end{pgfscope}%
\begin{pgfscope}%
\pgfsys@transformshift{1.333998in}{2.838511in}%
\pgfsys@useobject{currentmarker}{}%
\end{pgfscope}%
\begin{pgfscope}%
\pgfsys@transformshift{1.335174in}{2.850609in}%
\pgfsys@useobject{currentmarker}{}%
\end{pgfscope}%
\begin{pgfscope}%
\pgfsys@transformshift{1.330937in}{2.866142in}%
\pgfsys@useobject{currentmarker}{}%
\end{pgfscope}%
\begin{pgfscope}%
\pgfsys@transformshift{1.330687in}{2.874993in}%
\pgfsys@useobject{currentmarker}{}%
\end{pgfscope}%
\begin{pgfscope}%
\pgfsys@transformshift{1.331530in}{2.887611in}%
\pgfsys@useobject{currentmarker}{}%
\end{pgfscope}%
\begin{pgfscope}%
\pgfsys@transformshift{1.332845in}{2.900809in}%
\pgfsys@useobject{currentmarker}{}%
\end{pgfscope}%
\begin{pgfscope}%
\pgfsys@transformshift{1.330870in}{2.915200in}%
\pgfsys@useobject{currentmarker}{}%
\end{pgfscope}%
\begin{pgfscope}%
\pgfsys@transformshift{1.331029in}{2.923187in}%
\pgfsys@useobject{currentmarker}{}%
\end{pgfscope}%
\begin{pgfscope}%
\pgfsys@transformshift{1.330657in}{2.927566in}%
\pgfsys@useobject{currentmarker}{}%
\end{pgfscope}%
\begin{pgfscope}%
\pgfsys@transformshift{1.330813in}{2.929977in}%
\pgfsys@useobject{currentmarker}{}%
\end{pgfscope}%
\begin{pgfscope}%
\pgfsys@transformshift{1.330883in}{2.932858in}%
\pgfsys@useobject{currentmarker}{}%
\end{pgfscope}%
\begin{pgfscope}%
\pgfsys@transformshift{1.332600in}{2.935794in}%
\pgfsys@useobject{currentmarker}{}%
\end{pgfscope}%
\begin{pgfscope}%
\pgfsys@transformshift{1.335664in}{2.938690in}%
\pgfsys@useobject{currentmarker}{}%
\end{pgfscope}%
\begin{pgfscope}%
\pgfsys@transformshift{1.340466in}{2.940277in}%
\pgfsys@useobject{currentmarker}{}%
\end{pgfscope}%
\begin{pgfscope}%
\pgfsys@transformshift{1.343128in}{2.941081in}%
\pgfsys@useobject{currentmarker}{}%
\end{pgfscope}%
\begin{pgfscope}%
\pgfsys@transformshift{1.346457in}{2.941564in}%
\pgfsys@useobject{currentmarker}{}%
\end{pgfscope}%
\begin{pgfscope}%
\pgfsys@transformshift{1.348258in}{2.941988in}%
\pgfsys@useobject{currentmarker}{}%
\end{pgfscope}%
\begin{pgfscope}%
\pgfsys@transformshift{1.350549in}{2.942306in}%
\pgfsys@useobject{currentmarker}{}%
\end{pgfscope}%
\begin{pgfscope}%
\pgfsys@transformshift{1.353891in}{2.942852in}%
\pgfsys@useobject{currentmarker}{}%
\end{pgfscope}%
\begin{pgfscope}%
\pgfsys@transformshift{1.358227in}{2.943150in}%
\pgfsys@useobject{currentmarker}{}%
\end{pgfscope}%
\begin{pgfscope}%
\pgfsys@transformshift{1.363309in}{2.943768in}%
\pgfsys@useobject{currentmarker}{}%
\end{pgfscope}%
\begin{pgfscope}%
\pgfsys@transformshift{1.368933in}{2.943881in}%
\pgfsys@useobject{currentmarker}{}%
\end{pgfscope}%
\begin{pgfscope}%
\pgfsys@transformshift{1.372016in}{2.944128in}%
\pgfsys@useobject{currentmarker}{}%
\end{pgfscope}%
\begin{pgfscope}%
\pgfsys@transformshift{1.373717in}{2.944189in}%
\pgfsys@useobject{currentmarker}{}%
\end{pgfscope}%
\begin{pgfscope}%
\pgfsys@transformshift{1.374652in}{2.944228in}%
\pgfsys@useobject{currentmarker}{}%
\end{pgfscope}%
\begin{pgfscope}%
\pgfsys@transformshift{1.375166in}{2.944245in}%
\pgfsys@useobject{currentmarker}{}%
\end{pgfscope}%
\begin{pgfscope}%
\pgfsys@transformshift{1.375449in}{2.944250in}%
\pgfsys@useobject{currentmarker}{}%
\end{pgfscope}%
\begin{pgfscope}%
\pgfsys@transformshift{1.375605in}{2.944244in}%
\pgfsys@useobject{currentmarker}{}%
\end{pgfscope}%
\begin{pgfscope}%
\pgfsys@transformshift{1.375689in}{2.944256in}%
\pgfsys@useobject{currentmarker}{}%
\end{pgfscope}%
\begin{pgfscope}%
\pgfsys@transformshift{1.375736in}{2.944259in}%
\pgfsys@useobject{currentmarker}{}%
\end{pgfscope}%
\begin{pgfscope}%
\pgfsys@transformshift{1.379217in}{2.944309in}%
\pgfsys@useobject{currentmarker}{}%
\end{pgfscope}%
\begin{pgfscope}%
\pgfsys@transformshift{1.383622in}{2.944291in}%
\pgfsys@useobject{currentmarker}{}%
\end{pgfscope}%
\begin{pgfscope}%
\pgfsys@transformshift{1.388701in}{2.944824in}%
\pgfsys@useobject{currentmarker}{}%
\end{pgfscope}%
\begin{pgfscope}%
\pgfsys@transformshift{1.395110in}{2.944546in}%
\pgfsys@useobject{currentmarker}{}%
\end{pgfscope}%
\begin{pgfscope}%
\pgfsys@transformshift{1.406107in}{2.944872in}%
\pgfsys@useobject{currentmarker}{}%
\end{pgfscope}%
\begin{pgfscope}%
\pgfsys@transformshift{1.417997in}{2.946275in}%
\pgfsys@useobject{currentmarker}{}%
\end{pgfscope}%
\begin{pgfscope}%
\pgfsys@transformshift{1.430636in}{2.946428in}%
\pgfsys@useobject{currentmarker}{}%
\end{pgfscope}%
\begin{pgfscope}%
\pgfsys@transformshift{1.445901in}{2.946659in}%
\pgfsys@useobject{currentmarker}{}%
\end{pgfscope}%
\begin{pgfscope}%
\pgfsys@transformshift{1.463105in}{2.946756in}%
\pgfsys@useobject{currentmarker}{}%
\end{pgfscope}%
\begin{pgfscope}%
\pgfsys@transformshift{1.481947in}{2.947436in}%
\pgfsys@useobject{currentmarker}{}%
\end{pgfscope}%
\begin{pgfscope}%
\pgfsys@transformshift{1.501735in}{2.947722in}%
\pgfsys@useobject{currentmarker}{}%
\end{pgfscope}%
\begin{pgfscope}%
\pgfsys@transformshift{1.524083in}{2.947372in}%
\pgfsys@useobject{currentmarker}{}%
\end{pgfscope}%
\begin{pgfscope}%
\pgfsys@transformshift{1.547282in}{2.946385in}%
\pgfsys@useobject{currentmarker}{}%
\end{pgfscope}%
\begin{pgfscope}%
\pgfsys@transformshift{1.570985in}{2.949454in}%
\pgfsys@useobject{currentmarker}{}%
\end{pgfscope}%
\begin{pgfscope}%
\pgfsys@transformshift{1.595554in}{2.947450in}%
\pgfsys@useobject{currentmarker}{}%
\end{pgfscope}%
\begin{pgfscope}%
\pgfsys@transformshift{1.621428in}{2.945990in}%
\pgfsys@useobject{currentmarker}{}%
\end{pgfscope}%
\begin{pgfscope}%
\pgfsys@transformshift{1.649567in}{2.944832in}%
\pgfsys@useobject{currentmarker}{}%
\end{pgfscope}%
\begin{pgfscope}%
\pgfsys@transformshift{1.665040in}{2.945548in}%
\pgfsys@useobject{currentmarker}{}%
\end{pgfscope}%
\begin{pgfscope}%
\pgfsys@transformshift{1.681614in}{2.945931in}%
\pgfsys@useobject{currentmarker}{}%
\end{pgfscope}%
\begin{pgfscope}%
\pgfsys@transformshift{1.698678in}{2.946858in}%
\pgfsys@useobject{currentmarker}{}%
\end{pgfscope}%
\begin{pgfscope}%
\pgfsys@transformshift{1.708074in}{2.947102in}%
\pgfsys@useobject{currentmarker}{}%
\end{pgfscope}%
\begin{pgfscope}%
\pgfsys@transformshift{1.718968in}{2.947561in}%
\pgfsys@useobject{currentmarker}{}%
\end{pgfscope}%
\begin{pgfscope}%
\pgfsys@transformshift{1.730698in}{2.949477in}%
\pgfsys@useobject{currentmarker}{}%
\end{pgfscope}%
\begin{pgfscope}%
\pgfsys@transformshift{1.743009in}{2.950442in}%
\pgfsys@useobject{currentmarker}{}%
\end{pgfscope}%
\begin{pgfscope}%
\pgfsys@transformshift{1.755876in}{2.950988in}%
\pgfsys@useobject{currentmarker}{}%
\end{pgfscope}%
\begin{pgfscope}%
\pgfsys@transformshift{1.769722in}{2.953383in}%
\pgfsys@useobject{currentmarker}{}%
\end{pgfscope}%
\begin{pgfscope}%
\pgfsys@transformshift{1.777435in}{2.953869in}%
\pgfsys@useobject{currentmarker}{}%
\end{pgfscope}%
\begin{pgfscope}%
\pgfsys@transformshift{1.785780in}{2.954550in}%
\pgfsys@useobject{currentmarker}{}%
\end{pgfscope}%
\begin{pgfscope}%
\pgfsys@transformshift{1.790385in}{2.954551in}%
\pgfsys@useobject{currentmarker}{}%
\end{pgfscope}%
\begin{pgfscope}%
\pgfsys@transformshift{1.792891in}{2.954915in}%
\pgfsys@useobject{currentmarker}{}%
\end{pgfscope}%
\begin{pgfscope}%
\pgfsys@transformshift{1.795917in}{2.955129in}%
\pgfsys@useobject{currentmarker}{}%
\end{pgfscope}%
\begin{pgfscope}%
\pgfsys@transformshift{1.797578in}{2.955298in}%
\pgfsys@useobject{currentmarker}{}%
\end{pgfscope}%
\begin{pgfscope}%
\pgfsys@transformshift{1.798495in}{2.955322in}%
\pgfsys@useobject{currentmarker}{}%
\end{pgfscope}%
\begin{pgfscope}%
\pgfsys@transformshift{1.798998in}{2.955366in}%
\pgfsys@useobject{currentmarker}{}%
\end{pgfscope}%
\begin{pgfscope}%
\pgfsys@transformshift{1.800259in}{2.955499in}%
\pgfsys@useobject{currentmarker}{}%
\end{pgfscope}%
\begin{pgfscope}%
\pgfsys@transformshift{1.802176in}{2.955588in}%
\pgfsys@useobject{currentmarker}{}%
\end{pgfscope}%
\begin{pgfscope}%
\pgfsys@transformshift{1.803231in}{2.955577in}%
\pgfsys@useobject{currentmarker}{}%
\end{pgfscope}%
\begin{pgfscope}%
\pgfsys@transformshift{1.805205in}{2.955685in}%
\pgfsys@useobject{currentmarker}{}%
\end{pgfscope}%
\begin{pgfscope}%
\pgfsys@transformshift{1.809557in}{2.956951in}%
\pgfsys@useobject{currentmarker}{}%
\end{pgfscope}%
\begin{pgfscope}%
\pgfsys@transformshift{1.815556in}{2.957235in}%
\pgfsys@useobject{currentmarker}{}%
\end{pgfscope}%
\begin{pgfscope}%
\pgfsys@transformshift{1.824272in}{2.957802in}%
\pgfsys@useobject{currentmarker}{}%
\end{pgfscope}%
\begin{pgfscope}%
\pgfsys@transformshift{1.834152in}{2.958007in}%
\pgfsys@useobject{currentmarker}{}%
\end{pgfscope}%
\begin{pgfscope}%
\pgfsys@transformshift{1.845242in}{2.960034in}%
\pgfsys@useobject{currentmarker}{}%
\end{pgfscope}%
\begin{pgfscope}%
\pgfsys@transformshift{1.851443in}{2.960078in}%
\pgfsys@useobject{currentmarker}{}%
\end{pgfscope}%
\begin{pgfscope}%
\pgfsys@transformshift{1.858236in}{2.960501in}%
\pgfsys@useobject{currentmarker}{}%
\end{pgfscope}%
\begin{pgfscope}%
\pgfsys@transformshift{1.866418in}{2.960648in}%
\pgfsys@useobject{currentmarker}{}%
\end{pgfscope}%
\begin{pgfscope}%
\pgfsys@transformshift{1.870871in}{2.961302in}%
\pgfsys@useobject{currentmarker}{}%
\end{pgfscope}%
\begin{pgfscope}%
\pgfsys@transformshift{1.873346in}{2.961318in}%
\pgfsys@useobject{currentmarker}{}%
\end{pgfscope}%
\begin{pgfscope}%
\pgfsys@transformshift{1.876975in}{2.961670in}%
\pgfsys@useobject{currentmarker}{}%
\end{pgfscope}%
\begin{pgfscope}%
\pgfsys@transformshift{1.881896in}{2.961507in}%
\pgfsys@useobject{currentmarker}{}%
\end{pgfscope}%
\begin{pgfscope}%
\pgfsys@transformshift{1.884555in}{2.962017in}%
\pgfsys@useobject{currentmarker}{}%
\end{pgfscope}%
\begin{pgfscope}%
\pgfsys@transformshift{1.888137in}{2.962114in}%
\pgfsys@useobject{currentmarker}{}%
\end{pgfscope}%
\begin{pgfscope}%
\pgfsys@transformshift{1.890100in}{2.962284in}%
\pgfsys@useobject{currentmarker}{}%
\end{pgfscope}%
\begin{pgfscope}%
\pgfsys@transformshift{1.891180in}{2.962370in}%
\pgfsys@useobject{currentmarker}{}%
\end{pgfscope}%
\begin{pgfscope}%
\pgfsys@transformshift{1.894363in}{2.962600in}%
\pgfsys@useobject{currentmarker}{}%
\end{pgfscope}%
\begin{pgfscope}%
\pgfsys@transformshift{1.900459in}{2.962920in}%
\pgfsys@useobject{currentmarker}{}%
\end{pgfscope}%
\begin{pgfscope}%
\pgfsys@transformshift{1.909322in}{2.964424in}%
\pgfsys@useobject{currentmarker}{}%
\end{pgfscope}%
\begin{pgfscope}%
\pgfsys@transformshift{1.920784in}{2.964527in}%
\pgfsys@useobject{currentmarker}{}%
\end{pgfscope}%
\begin{pgfscope}%
\pgfsys@transformshift{1.933830in}{2.965685in}%
\pgfsys@useobject{currentmarker}{}%
\end{pgfscope}%
\begin{pgfscope}%
\pgfsys@transformshift{1.941033in}{2.965791in}%
\pgfsys@useobject{currentmarker}{}%
\end{pgfscope}%
\begin{pgfscope}%
\pgfsys@transformshift{1.948912in}{2.966369in}%
\pgfsys@useobject{currentmarker}{}%
\end{pgfscope}%
\begin{pgfscope}%
\pgfsys@transformshift{1.958323in}{2.967230in}%
\pgfsys@useobject{currentmarker}{}%
\end{pgfscope}%
\begin{pgfscope}%
\pgfsys@transformshift{1.969226in}{2.967701in}%
\pgfsys@useobject{currentmarker}{}%
\end{pgfscope}%
\begin{pgfscope}%
\pgfsys@transformshift{1.981301in}{2.968151in}%
\pgfsys@useobject{currentmarker}{}%
\end{pgfscope}%
\begin{pgfscope}%
\pgfsys@transformshift{1.995566in}{2.968849in}%
\pgfsys@useobject{currentmarker}{}%
\end{pgfscope}%
\begin{pgfscope}%
\pgfsys@transformshift{2.003386in}{2.969588in}%
\pgfsys@useobject{currentmarker}{}%
\end{pgfscope}%
\begin{pgfscope}%
\pgfsys@transformshift{2.012824in}{2.971089in}%
\pgfsys@useobject{currentmarker}{}%
\end{pgfscope}%
\begin{pgfscope}%
\pgfsys@transformshift{2.018056in}{2.971597in}%
\pgfsys@useobject{currentmarker}{}%
\end{pgfscope}%
\begin{pgfscope}%
\pgfsys@transformshift{2.020943in}{2.971750in}%
\pgfsys@useobject{currentmarker}{}%
\end{pgfscope}%
\begin{pgfscope}%
\pgfsys@transformshift{2.022515in}{2.971990in}%
\pgfsys@useobject{currentmarker}{}%
\end{pgfscope}%
\begin{pgfscope}%
\pgfsys@transformshift{2.023380in}{2.972116in}%
\pgfsys@useobject{currentmarker}{}%
\end{pgfscope}%
\begin{pgfscope}%
\pgfsys@transformshift{2.026700in}{2.972415in}%
\pgfsys@useobject{currentmarker}{}%
\end{pgfscope}%
\begin{pgfscope}%
\pgfsys@transformshift{2.031720in}{2.973024in}%
\pgfsys@useobject{currentmarker}{}%
\end{pgfscope}%
\begin{pgfscope}%
\pgfsys@transformshift{2.037449in}{2.973670in}%
\pgfsys@useobject{currentmarker}{}%
\end{pgfscope}%
\begin{pgfscope}%
\pgfsys@transformshift{2.044705in}{2.974256in}%
\pgfsys@useobject{currentmarker}{}%
\end{pgfscope}%
\begin{pgfscope}%
\pgfsys@transformshift{2.052736in}{2.975414in}%
\pgfsys@useobject{currentmarker}{}%
\end{pgfscope}%
\begin{pgfscope}%
\pgfsys@transformshift{2.065725in}{2.976729in}%
\pgfsys@useobject{currentmarker}{}%
\end{pgfscope}%
\begin{pgfscope}%
\pgfsys@transformshift{2.081817in}{2.979834in}%
\pgfsys@useobject{currentmarker}{}%
\end{pgfscope}%
\begin{pgfscope}%
\pgfsys@transformshift{2.099570in}{2.981675in}%
\pgfsys@useobject{currentmarker}{}%
\end{pgfscope}%
\begin{pgfscope}%
\pgfsys@transformshift{2.118778in}{2.983798in}%
\pgfsys@useobject{currentmarker}{}%
\end{pgfscope}%
\begin{pgfscope}%
\pgfsys@transformshift{2.139330in}{2.985211in}%
\pgfsys@useobject{currentmarker}{}%
\end{pgfscope}%
\begin{pgfscope}%
\pgfsys@transformshift{2.162573in}{2.986661in}%
\pgfsys@useobject{currentmarker}{}%
\end{pgfscope}%
\begin{pgfscope}%
\pgfsys@transformshift{2.185999in}{2.990766in}%
\pgfsys@useobject{currentmarker}{}%
\end{pgfscope}%
\begin{pgfscope}%
\pgfsys@transformshift{2.198839in}{2.993262in}%
\pgfsys@useobject{currentmarker}{}%
\end{pgfscope}%
\begin{pgfscope}%
\pgfsys@transformshift{2.212966in}{2.994865in}%
\pgfsys@useobject{currentmarker}{}%
\end{pgfscope}%
\begin{pgfscope}%
\pgfsys@transformshift{2.228980in}{2.995613in}%
\pgfsys@useobject{currentmarker}{}%
\end{pgfscope}%
\begin{pgfscope}%
\pgfsys@transformshift{2.251235in}{3.002817in}%
\pgfsys@useobject{currentmarker}{}%
\end{pgfscope}%
\begin{pgfscope}%
\pgfsys@transformshift{2.275459in}{3.006150in}%
\pgfsys@useobject{currentmarker}{}%
\end{pgfscope}%
\begin{pgfscope}%
\pgfsys@transformshift{2.300322in}{3.010885in}%
\pgfsys@useobject{currentmarker}{}%
\end{pgfscope}%
\begin{pgfscope}%
\pgfsys@transformshift{2.326123in}{3.015234in}%
\pgfsys@useobject{currentmarker}{}%
\end{pgfscope}%
\begin{pgfscope}%
\pgfsys@transformshift{2.353117in}{3.018221in}%
\pgfsys@useobject{currentmarker}{}%
\end{pgfscope}%
\begin{pgfscope}%
\pgfsys@transformshift{2.382137in}{3.019987in}%
\pgfsys@useobject{currentmarker}{}%
\end{pgfscope}%
\begin{pgfscope}%
\pgfsys@transformshift{2.411940in}{3.020468in}%
\pgfsys@useobject{currentmarker}{}%
\end{pgfscope}%
\begin{pgfscope}%
\pgfsys@transformshift{2.443086in}{3.022527in}%
\pgfsys@useobject{currentmarker}{}%
\end{pgfscope}%
\begin{pgfscope}%
\pgfsys@transformshift{2.476257in}{3.023962in}%
\pgfsys@useobject{currentmarker}{}%
\end{pgfscope}%
\begin{pgfscope}%
\pgfsys@transformshift{2.512047in}{3.026376in}%
\pgfsys@useobject{currentmarker}{}%
\end{pgfscope}%
\begin{pgfscope}%
\pgfsys@transformshift{2.549715in}{3.027218in}%
\pgfsys@useobject{currentmarker}{}%
\end{pgfscope}%
\begin{pgfscope}%
\pgfsys@transformshift{2.587904in}{3.027921in}%
\pgfsys@useobject{currentmarker}{}%
\end{pgfscope}%
\begin{pgfscope}%
\pgfsys@transformshift{2.626832in}{3.029581in}%
\pgfsys@useobject{currentmarker}{}%
\end{pgfscope}%
\begin{pgfscope}%
\pgfsys@transformshift{2.667081in}{3.030899in}%
\pgfsys@useobject{currentmarker}{}%
\end{pgfscope}%
\begin{pgfscope}%
\pgfsys@transformshift{2.709128in}{3.033900in}%
\pgfsys@useobject{currentmarker}{}%
\end{pgfscope}%
\begin{pgfscope}%
\pgfsys@transformshift{2.752871in}{3.034430in}%
\pgfsys@useobject{currentmarker}{}%
\end{pgfscope}%
\begin{pgfscope}%
\pgfsys@transformshift{2.776911in}{3.035442in}%
\pgfsys@useobject{currentmarker}{}%
\end{pgfscope}%
\begin{pgfscope}%
\pgfsys@transformshift{2.803519in}{3.036755in}%
\pgfsys@useobject{currentmarker}{}%
\end{pgfscope}%
\begin{pgfscope}%
\pgfsys@transformshift{2.832021in}{3.037623in}%
\pgfsys@useobject{currentmarker}{}%
\end{pgfscope}%
\begin{pgfscope}%
\pgfsys@transformshift{2.861331in}{3.040731in}%
\pgfsys@useobject{currentmarker}{}%
\end{pgfscope}%
\begin{pgfscope}%
\pgfsys@transformshift{2.877537in}{3.041129in}%
\pgfsys@useobject{currentmarker}{}%
\end{pgfscope}%
\begin{pgfscope}%
\pgfsys@transformshift{2.895023in}{3.042024in}%
\pgfsys@useobject{currentmarker}{}%
\end{pgfscope}%
\begin{pgfscope}%
\pgfsys@transformshift{2.914574in}{3.042440in}%
\pgfsys@useobject{currentmarker}{}%
\end{pgfscope}%
\begin{pgfscope}%
\pgfsys@transformshift{2.925313in}{3.043038in}%
\pgfsys@useobject{currentmarker}{}%
\end{pgfscope}%
\begin{pgfscope}%
\pgfsys@transformshift{2.937625in}{3.043933in}%
\pgfsys@useobject{currentmarker}{}%
\end{pgfscope}%
\begin{pgfscope}%
\pgfsys@transformshift{2.952478in}{3.044898in}%
\pgfsys@useobject{currentmarker}{}%
\end{pgfscope}%
\begin{pgfscope}%
\pgfsys@transformshift{2.968456in}{3.044586in}%
\pgfsys@useobject{currentmarker}{}%
\end{pgfscope}%
\begin{pgfscope}%
\pgfsys@transformshift{2.985499in}{3.044181in}%
\pgfsys@useobject{currentmarker}{}%
\end{pgfscope}%
\begin{pgfscope}%
\pgfsys@transformshift{3.004017in}{3.047633in}%
\pgfsys@useobject{currentmarker}{}%
\end{pgfscope}%
\begin{pgfscope}%
\pgfsys@transformshift{3.014349in}{3.046868in}%
\pgfsys@useobject{currentmarker}{}%
\end{pgfscope}%
\begin{pgfscope}%
\pgfsys@transformshift{3.020044in}{3.047076in}%
\pgfsys@useobject{currentmarker}{}%
\end{pgfscope}%
\begin{pgfscope}%
\pgfsys@transformshift{3.026622in}{3.047137in}%
\pgfsys@useobject{currentmarker}{}%
\end{pgfscope}%
\begin{pgfscope}%
\pgfsys@transformshift{3.030236in}{3.047316in}%
\pgfsys@useobject{currentmarker}{}%
\end{pgfscope}%
\begin{pgfscope}%
\pgfsys@transformshift{3.035296in}{3.047211in}%
\pgfsys@useobject{currentmarker}{}%
\end{pgfscope}%
\begin{pgfscope}%
\pgfsys@transformshift{3.041891in}{3.047341in}%
\pgfsys@useobject{currentmarker}{}%
\end{pgfscope}%
\begin{pgfscope}%
\pgfsys@transformshift{3.050385in}{3.047720in}%
\pgfsys@useobject{currentmarker}{}%
\end{pgfscope}%
\begin{pgfscope}%
\pgfsys@transformshift{3.061933in}{3.048046in}%
\pgfsys@useobject{currentmarker}{}%
\end{pgfscope}%
\begin{pgfscope}%
\pgfsys@transformshift{3.074346in}{3.046976in}%
\pgfsys@useobject{currentmarker}{}%
\end{pgfscope}%
\begin{pgfscope}%
\pgfsys@transformshift{3.087657in}{3.049199in}%
\pgfsys@useobject{currentmarker}{}%
\end{pgfscope}%
\begin{pgfscope}%
\pgfsys@transformshift{3.102538in}{3.049222in}%
\pgfsys@useobject{currentmarker}{}%
\end{pgfscope}%
\begin{pgfscope}%
\pgfsys@transformshift{3.118369in}{3.049649in}%
\pgfsys@useobject{currentmarker}{}%
\end{pgfscope}%
\begin{pgfscope}%
\pgfsys@transformshift{3.134976in}{3.048813in}%
\pgfsys@useobject{currentmarker}{}%
\end{pgfscope}%
\begin{pgfscope}%
\pgfsys@transformshift{3.155933in}{3.048803in}%
\pgfsys@useobject{currentmarker}{}%
\end{pgfscope}%
\begin{pgfscope}%
\pgfsys@transformshift{3.179976in}{3.047084in}%
\pgfsys@useobject{currentmarker}{}%
\end{pgfscope}%
\begin{pgfscope}%
\pgfsys@transformshift{3.205872in}{3.048924in}%
\pgfsys@useobject{currentmarker}{}%
\end{pgfscope}%
\begin{pgfscope}%
\pgfsys@transformshift{3.220094in}{3.047645in}%
\pgfsys@useobject{currentmarker}{}%
\end{pgfscope}%
\begin{pgfscope}%
\pgfsys@transformshift{3.235148in}{3.047197in}%
\pgfsys@useobject{currentmarker}{}%
\end{pgfscope}%
\begin{pgfscope}%
\pgfsys@transformshift{3.252301in}{3.047527in}%
\pgfsys@useobject{currentmarker}{}%
\end{pgfscope}%
\begin{pgfscope}%
\pgfsys@transformshift{3.270466in}{3.047537in}%
\pgfsys@useobject{currentmarker}{}%
\end{pgfscope}%
\begin{pgfscope}%
\pgfsys@transformshift{3.291035in}{3.048021in}%
\pgfsys@useobject{currentmarker}{}%
\end{pgfscope}%
\begin{pgfscope}%
\pgfsys@transformshift{3.313576in}{3.046184in}%
\pgfsys@useobject{currentmarker}{}%
\end{pgfscope}%
\begin{pgfscope}%
\pgfsys@transformshift{3.338280in}{3.045567in}%
\pgfsys@useobject{currentmarker}{}%
\end{pgfscope}%
\begin{pgfscope}%
\pgfsys@transformshift{3.366096in}{3.045387in}%
\pgfsys@useobject{currentmarker}{}%
\end{pgfscope}%
\begin{pgfscope}%
\pgfsys@transformshift{3.396462in}{3.045240in}%
\pgfsys@useobject{currentmarker}{}%
\end{pgfscope}%
\begin{pgfscope}%
\pgfsys@transformshift{3.428194in}{3.045613in}%
\pgfsys@useobject{currentmarker}{}%
\end{pgfscope}%
\begin{pgfscope}%
\pgfsys@transformshift{3.461051in}{3.045553in}%
\pgfsys@useobject{currentmarker}{}%
\end{pgfscope}%
\begin{pgfscope}%
\pgfsys@transformshift{3.495425in}{3.048343in}%
\pgfsys@useobject{currentmarker}{}%
\end{pgfscope}%
\begin{pgfscope}%
\pgfsys@transformshift{3.530538in}{3.050775in}%
\pgfsys@useobject{currentmarker}{}%
\end{pgfscope}%
\begin{pgfscope}%
\pgfsys@transformshift{3.549668in}{3.047816in}%
\pgfsys@useobject{currentmarker}{}%
\end{pgfscope}%
\begin{pgfscope}%
\pgfsys@transformshift{3.568008in}{3.038196in}%
\pgfsys@useobject{currentmarker}{}%
\end{pgfscope}%
\begin{pgfscope}%
\pgfsys@transformshift{3.576515in}{3.030622in}%
\pgfsys@useobject{currentmarker}{}%
\end{pgfscope}%
\begin{pgfscope}%
\pgfsys@transformshift{3.578949in}{3.024849in}%
\pgfsys@useobject{currentmarker}{}%
\end{pgfscope}%
\begin{pgfscope}%
\pgfsys@transformshift{3.579224in}{3.018082in}%
\pgfsys@useobject{currentmarker}{}%
\end{pgfscope}%
\begin{pgfscope}%
\pgfsys@transformshift{3.578248in}{3.010190in}%
\pgfsys@useobject{currentmarker}{}%
\end{pgfscope}%
\begin{pgfscope}%
\pgfsys@transformshift{3.578694in}{3.001296in}%
\pgfsys@useobject{currentmarker}{}%
\end{pgfscope}%
\begin{pgfscope}%
\pgfsys@transformshift{3.578131in}{2.991794in}%
\pgfsys@useobject{currentmarker}{}%
\end{pgfscope}%
\begin{pgfscope}%
\pgfsys@transformshift{3.577455in}{2.981641in}%
\pgfsys@useobject{currentmarker}{}%
\end{pgfscope}%
\begin{pgfscope}%
\pgfsys@transformshift{3.579760in}{2.970996in}%
\pgfsys@useobject{currentmarker}{}%
\end{pgfscope}%
\begin{pgfscope}%
\pgfsys@transformshift{3.580687in}{2.958472in}%
\pgfsys@useobject{currentmarker}{}%
\end{pgfscope}%
\begin{pgfscope}%
\pgfsys@transformshift{3.578833in}{2.945258in}%
\pgfsys@useobject{currentmarker}{}%
\end{pgfscope}%
\begin{pgfscope}%
\pgfsys@transformshift{3.576402in}{2.938334in}%
\pgfsys@useobject{currentmarker}{}%
\end{pgfscope}%
\begin{pgfscope}%
\pgfsys@transformshift{3.576547in}{2.929109in}%
\pgfsys@useobject{currentmarker}{}%
\end{pgfscope}%
\begin{pgfscope}%
\pgfsys@transformshift{3.575428in}{2.924160in}%
\pgfsys@useobject{currentmarker}{}%
\end{pgfscope}%
\begin{pgfscope}%
\pgfsys@transformshift{3.573364in}{2.917237in}%
\pgfsys@useobject{currentmarker}{}%
\end{pgfscope}%
\begin{pgfscope}%
\pgfsys@transformshift{3.572133in}{2.908093in}%
\pgfsys@useobject{currentmarker}{}%
\end{pgfscope}%
\begin{pgfscope}%
\pgfsys@transformshift{3.572534in}{2.903035in}%
\pgfsys@useobject{currentmarker}{}%
\end{pgfscope}%
\begin{pgfscope}%
\pgfsys@transformshift{3.570881in}{2.895050in}%
\pgfsys@useobject{currentmarker}{}%
\end{pgfscope}%
\begin{pgfscope}%
\pgfsys@transformshift{3.571087in}{2.890570in}%
\pgfsys@useobject{currentmarker}{}%
\end{pgfscope}%
\begin{pgfscope}%
\pgfsys@transformshift{3.571945in}{2.885075in}%
\pgfsys@useobject{currentmarker}{}%
\end{pgfscope}%
\begin{pgfscope}%
\pgfsys@transformshift{3.571601in}{2.882035in}%
\pgfsys@useobject{currentmarker}{}%
\end{pgfscope}%
\begin{pgfscope}%
\pgfsys@transformshift{3.571360in}{2.880370in}%
\pgfsys@useobject{currentmarker}{}%
\end{pgfscope}%
\begin{pgfscope}%
\pgfsys@transformshift{3.571455in}{2.877427in}%
\pgfsys@useobject{currentmarker}{}%
\end{pgfscope}%
\begin{pgfscope}%
\pgfsys@transformshift{3.571714in}{2.873894in}%
\pgfsys@useobject{currentmarker}{}%
\end{pgfscope}%
\begin{pgfscope}%
\pgfsys@transformshift{3.571199in}{2.869490in}%
\pgfsys@useobject{currentmarker}{}%
\end{pgfscope}%
\begin{pgfscope}%
\pgfsys@transformshift{3.570523in}{2.863563in}%
\pgfsys@useobject{currentmarker}{}%
\end{pgfscope}%
\begin{pgfscope}%
\pgfsys@transformshift{3.570927in}{2.856989in}%
\pgfsys@useobject{currentmarker}{}%
\end{pgfscope}%
\begin{pgfscope}%
\pgfsys@transformshift{3.571450in}{2.853404in}%
\pgfsys@useobject{currentmarker}{}%
\end{pgfscope}%
\begin{pgfscope}%
\pgfsys@transformshift{3.570572in}{2.847938in}%
\pgfsys@useobject{currentmarker}{}%
\end{pgfscope}%
\begin{pgfscope}%
\pgfsys@transformshift{3.570427in}{2.844897in}%
\pgfsys@useobject{currentmarker}{}%
\end{pgfscope}%
\begin{pgfscope}%
\pgfsys@transformshift{3.569962in}{2.839787in}%
\pgfsys@useobject{currentmarker}{}%
\end{pgfscope}%
\begin{pgfscope}%
\pgfsys@transformshift{3.569999in}{2.836965in}%
\pgfsys@useobject{currentmarker}{}%
\end{pgfscope}%
\begin{pgfscope}%
\pgfsys@transformshift{3.569560in}{2.833082in}%
\pgfsys@useobject{currentmarker}{}%
\end{pgfscope}%
\begin{pgfscope}%
\pgfsys@transformshift{3.569550in}{2.828549in}%
\pgfsys@useobject{currentmarker}{}%
\end{pgfscope}%
\begin{pgfscope}%
\pgfsys@transformshift{3.569140in}{2.820933in}%
\pgfsys@useobject{currentmarker}{}%
\end{pgfscope}%
\begin{pgfscope}%
\pgfsys@transformshift{3.569388in}{2.816746in}%
\pgfsys@useobject{currentmarker}{}%
\end{pgfscope}%
\begin{pgfscope}%
\pgfsys@transformshift{3.568623in}{2.811814in}%
\pgfsys@useobject{currentmarker}{}%
\end{pgfscope}%
\begin{pgfscope}%
\pgfsys@transformshift{3.568402in}{2.809077in}%
\pgfsys@useobject{currentmarker}{}%
\end{pgfscope}%
\begin{pgfscope}%
\pgfsys@transformshift{3.568499in}{2.805012in}%
\pgfsys@useobject{currentmarker}{}%
\end{pgfscope}%
\begin{pgfscope}%
\pgfsys@transformshift{3.568284in}{2.802786in}%
\pgfsys@useobject{currentmarker}{}%
\end{pgfscope}%
\begin{pgfscope}%
\pgfsys@transformshift{3.567865in}{2.798808in}%
\pgfsys@useobject{currentmarker}{}%
\end{pgfscope}%
\begin{pgfscope}%
\pgfsys@transformshift{3.568195in}{2.794124in}%
\pgfsys@useobject{currentmarker}{}%
\end{pgfscope}%
\begin{pgfscope}%
\pgfsys@transformshift{3.569423in}{2.787967in}%
\pgfsys@useobject{currentmarker}{}%
\end{pgfscope}%
\begin{pgfscope}%
\pgfsys@transformshift{3.568626in}{2.779086in}%
\pgfsys@useobject{currentmarker}{}%
\end{pgfscope}%
\begin{pgfscope}%
\pgfsys@transformshift{3.567885in}{2.774238in}%
\pgfsys@useobject{currentmarker}{}%
\end{pgfscope}%
\begin{pgfscope}%
\pgfsys@transformshift{3.567626in}{2.767723in}%
\pgfsys@useobject{currentmarker}{}%
\end{pgfscope}%
\begin{pgfscope}%
\pgfsys@transformshift{3.567530in}{2.764138in}%
\pgfsys@useobject{currentmarker}{}%
\end{pgfscope}%
\begin{pgfscope}%
\pgfsys@transformshift{3.567290in}{2.759288in}%
\pgfsys@useobject{currentmarker}{}%
\end{pgfscope}%
\begin{pgfscope}%
\pgfsys@transformshift{3.567331in}{2.756617in}%
\pgfsys@useobject{currentmarker}{}%
\end{pgfscope}%
\begin{pgfscope}%
\pgfsys@transformshift{3.568066in}{2.751629in}%
\pgfsys@useobject{currentmarker}{}%
\end{pgfscope}%
\begin{pgfscope}%
\pgfsys@transformshift{3.567417in}{2.744415in}%
\pgfsys@useobject{currentmarker}{}%
\end{pgfscope}%
\begin{pgfscope}%
\pgfsys@transformshift{3.566919in}{2.740463in}%
\pgfsys@useobject{currentmarker}{}%
\end{pgfscope}%
\begin{pgfscope}%
\pgfsys@transformshift{3.566927in}{2.734843in}%
\pgfsys@useobject{currentmarker}{}%
\end{pgfscope}%
\begin{pgfscope}%
\pgfsys@transformshift{3.567322in}{2.731777in}%
\pgfsys@useobject{currentmarker}{}%
\end{pgfscope}%
\begin{pgfscope}%
\pgfsys@transformshift{3.566870in}{2.726047in}%
\pgfsys@useobject{currentmarker}{}%
\end{pgfscope}%
\begin{pgfscope}%
\pgfsys@transformshift{3.567182in}{2.719348in}%
\pgfsys@useobject{currentmarker}{}%
\end{pgfscope}%
\begin{pgfscope}%
\pgfsys@transformshift{3.568662in}{2.708663in}%
\pgfsys@useobject{currentmarker}{}%
\end{pgfscope}%
\begin{pgfscope}%
\pgfsys@transformshift{3.569507in}{2.702790in}%
\pgfsys@useobject{currentmarker}{}%
\end{pgfscope}%
\begin{pgfscope}%
\pgfsys@transformshift{3.568957in}{2.695412in}%
\pgfsys@useobject{currentmarker}{}%
\end{pgfscope}%
\begin{pgfscope}%
\pgfsys@transformshift{3.569848in}{2.687509in}%
\pgfsys@useobject{currentmarker}{}%
\end{pgfscope}%
\begin{pgfscope}%
\pgfsys@transformshift{3.571900in}{2.678434in}%
\pgfsys@useobject{currentmarker}{}%
\end{pgfscope}%
\begin{pgfscope}%
\pgfsys@transformshift{3.570850in}{2.665493in}%
\pgfsys@useobject{currentmarker}{}%
\end{pgfscope}%
\begin{pgfscope}%
\pgfsys@transformshift{3.570600in}{2.651234in}%
\pgfsys@useobject{currentmarker}{}%
\end{pgfscope}%
\begin{pgfscope}%
\pgfsys@transformshift{3.571577in}{2.636231in}%
\pgfsys@useobject{currentmarker}{}%
\end{pgfscope}%
\begin{pgfscope}%
\pgfsys@transformshift{3.574319in}{2.619591in}%
\pgfsys@useobject{currentmarker}{}%
\end{pgfscope}%
\begin{pgfscope}%
\pgfsys@transformshift{3.572513in}{2.598411in}%
\pgfsys@useobject{currentmarker}{}%
\end{pgfscope}%
\begin{pgfscope}%
\pgfsys@transformshift{3.573165in}{2.586738in}%
\pgfsys@useobject{currentmarker}{}%
\end{pgfscope}%
\begin{pgfscope}%
\pgfsys@transformshift{3.575581in}{2.573303in}%
\pgfsys@useobject{currentmarker}{}%
\end{pgfscope}%
\begin{pgfscope}%
\pgfsys@transformshift{3.576209in}{2.556615in}%
\pgfsys@useobject{currentmarker}{}%
\end{pgfscope}%
\begin{pgfscope}%
\pgfsys@transformshift{3.574505in}{2.539203in}%
\pgfsys@useobject{currentmarker}{}%
\end{pgfscope}%
\begin{pgfscope}%
\pgfsys@transformshift{3.574833in}{2.520080in}%
\pgfsys@useobject{currentmarker}{}%
\end{pgfscope}%
\begin{pgfscope}%
\pgfsys@transformshift{3.575541in}{2.509584in}%
\pgfsys@useobject{currentmarker}{}%
\end{pgfscope}%
\begin{pgfscope}%
\pgfsys@transformshift{3.574044in}{2.495320in}%
\pgfsys@useobject{currentmarker}{}%
\end{pgfscope}%
\begin{pgfscope}%
\pgfsys@transformshift{3.573884in}{2.487433in}%
\pgfsys@useobject{currentmarker}{}%
\end{pgfscope}%
\begin{pgfscope}%
\pgfsys@transformshift{3.575424in}{2.477800in}%
\pgfsys@useobject{currentmarker}{}%
\end{pgfscope}%
\begin{pgfscope}%
\pgfsys@transformshift{3.575200in}{2.464868in}%
\pgfsys@useobject{currentmarker}{}%
\end{pgfscope}%
\begin{pgfscope}%
\pgfsys@transformshift{3.573360in}{2.451581in}%
\pgfsys@useobject{currentmarker}{}%
\end{pgfscope}%
\begin{pgfscope}%
\pgfsys@transformshift{3.574665in}{2.436813in}%
\pgfsys@useobject{currentmarker}{}%
\end{pgfscope}%
\begin{pgfscope}%
\pgfsys@transformshift{3.578741in}{2.420947in}%
\pgfsys@useobject{currentmarker}{}%
\end{pgfscope}%
\begin{pgfscope}%
\pgfsys@transformshift{3.576464in}{2.401188in}%
\pgfsys@useobject{currentmarker}{}%
\end{pgfscope}%
\begin{pgfscope}%
\pgfsys@transformshift{3.575317in}{2.390308in}%
\pgfsys@useobject{currentmarker}{}%
\end{pgfscope}%
\begin{pgfscope}%
\pgfsys@transformshift{3.575021in}{2.376780in}%
\pgfsys@useobject{currentmarker}{}%
\end{pgfscope}%
\begin{pgfscope}%
\pgfsys@transformshift{3.576642in}{2.362839in}%
\pgfsys@useobject{currentmarker}{}%
\end{pgfscope}%
\begin{pgfscope}%
\pgfsys@transformshift{3.575013in}{2.345125in}%
\pgfsys@useobject{currentmarker}{}%
\end{pgfscope}%
\begin{pgfscope}%
\pgfsys@transformshift{3.577456in}{2.335650in}%
\pgfsys@useobject{currentmarker}{}%
\end{pgfscope}%
\begin{pgfscope}%
\pgfsys@transformshift{3.581319in}{2.324562in}%
\pgfsys@useobject{currentmarker}{}%
\end{pgfscope}%
\begin{pgfscope}%
\pgfsys@transformshift{3.579075in}{2.309009in}%
\pgfsys@useobject{currentmarker}{}%
\end{pgfscope}%
\begin{pgfscope}%
\pgfsys@transformshift{3.576919in}{2.291220in}%
\pgfsys@useobject{currentmarker}{}%
\end{pgfscope}%
\begin{pgfscope}%
\pgfsys@transformshift{3.577559in}{2.272349in}%
\pgfsys@useobject{currentmarker}{}%
\end{pgfscope}%
\begin{pgfscope}%
\pgfsys@transformshift{3.578205in}{2.261984in}%
\pgfsys@useobject{currentmarker}{}%
\end{pgfscope}%
\begin{pgfscope}%
\pgfsys@transformshift{3.576338in}{2.247233in}%
\pgfsys@useobject{currentmarker}{}%
\end{pgfscope}%
\begin{pgfscope}%
\pgfsys@transformshift{3.578902in}{2.239467in}%
\pgfsys@useobject{currentmarker}{}%
\end{pgfscope}%
\begin{pgfscope}%
\pgfsys@transformshift{3.582596in}{2.228550in}%
\pgfsys@useobject{currentmarker}{}%
\end{pgfscope}%
\begin{pgfscope}%
\pgfsys@transformshift{3.582676in}{2.213725in}%
\pgfsys@useobject{currentmarker}{}%
\end{pgfscope}%
\begin{pgfscope}%
\pgfsys@transformshift{3.580886in}{2.197050in}%
\pgfsys@useobject{currentmarker}{}%
\end{pgfscope}%
\begin{pgfscope}%
\pgfsys@transformshift{3.580137in}{2.176622in}%
\pgfsys@useobject{currentmarker}{}%
\end{pgfscope}%
\begin{pgfscope}%
\pgfsys@transformshift{3.580847in}{2.165402in}%
\pgfsys@useobject{currentmarker}{}%
\end{pgfscope}%
\begin{pgfscope}%
\pgfsys@transformshift{3.579975in}{2.152509in}%
\pgfsys@useobject{currentmarker}{}%
\end{pgfscope}%
\begin{pgfscope}%
\pgfsys@transformshift{3.582039in}{2.138577in}%
\pgfsys@useobject{currentmarker}{}%
\end{pgfscope}%
\begin{pgfscope}%
\pgfsys@transformshift{3.587632in}{2.122087in}%
\pgfsys@useobject{currentmarker}{}%
\end{pgfscope}%
\begin{pgfscope}%
\pgfsys@transformshift{3.586158in}{2.101516in}%
\pgfsys@useobject{currentmarker}{}%
\end{pgfscope}%
\begin{pgfscope}%
\pgfsys@transformshift{3.587094in}{2.080392in}%
\pgfsys@useobject{currentmarker}{}%
\end{pgfscope}%
\begin{pgfscope}%
\pgfsys@transformshift{3.588362in}{2.056305in}%
\pgfsys@useobject{currentmarker}{}%
\end{pgfscope}%
\begin{pgfscope}%
\pgfsys@transformshift{3.589991in}{2.043139in}%
\pgfsys@useobject{currentmarker}{}%
\end{pgfscope}%
\begin{pgfscope}%
\pgfsys@transformshift{3.588965in}{2.028783in}%
\pgfsys@useobject{currentmarker}{}%
\end{pgfscope}%
\begin{pgfscope}%
\pgfsys@transformshift{3.591944in}{2.013978in}%
\pgfsys@useobject{currentmarker}{}%
\end{pgfscope}%
\begin{pgfscope}%
\pgfsys@transformshift{3.595429in}{1.996550in}%
\pgfsys@useobject{currentmarker}{}%
\end{pgfscope}%
\begin{pgfscope}%
\pgfsys@transformshift{3.594204in}{1.974907in}%
\pgfsys@useobject{currentmarker}{}%
\end{pgfscope}%
\begin{pgfscope}%
\pgfsys@transformshift{3.593392in}{1.952446in}%
\pgfsys@useobject{currentmarker}{}%
\end{pgfscope}%
\begin{pgfscope}%
\pgfsys@transformshift{3.596735in}{1.928812in}%
\pgfsys@useobject{currentmarker}{}%
\end{pgfscope}%
\begin{pgfscope}%
\pgfsys@transformshift{3.600598in}{1.904505in}%
\pgfsys@useobject{currentmarker}{}%
\end{pgfscope}%
\begin{pgfscope}%
\pgfsys@transformshift{3.599018in}{1.891061in}%
\pgfsys@useobject{currentmarker}{}%
\end{pgfscope}%
\begin{pgfscope}%
\pgfsys@transformshift{3.600680in}{1.875778in}%
\pgfsys@useobject{currentmarker}{}%
\end{pgfscope}%
\begin{pgfscope}%
\pgfsys@transformshift{3.602150in}{1.856947in}%
\pgfsys@useobject{currentmarker}{}%
\end{pgfscope}%
\begin{pgfscope}%
\pgfsys@transformshift{3.601483in}{1.834480in}%
\pgfsys@useobject{currentmarker}{}%
\end{pgfscope}%
\begin{pgfscope}%
\pgfsys@transformshift{3.601627in}{1.822118in}%
\pgfsys@useobject{currentmarker}{}%
\end{pgfscope}%
\begin{pgfscope}%
\pgfsys@transformshift{3.604424in}{1.807740in}%
\pgfsys@useobject{currentmarker}{}%
\end{pgfscope}%
\begin{pgfscope}%
\pgfsys@transformshift{3.603167in}{1.789336in}%
\pgfsys@useobject{currentmarker}{}%
\end{pgfscope}%
\begin{pgfscope}%
\pgfsys@transformshift{3.599125in}{1.770410in}%
\pgfsys@useobject{currentmarker}{}%
\end{pgfscope}%
\begin{pgfscope}%
\pgfsys@transformshift{3.598368in}{1.750008in}%
\pgfsys@useobject{currentmarker}{}%
\end{pgfscope}%
\begin{pgfscope}%
\pgfsys@transformshift{3.600506in}{1.738984in}%
\pgfsys@useobject{currentmarker}{}%
\end{pgfscope}%
\begin{pgfscope}%
\pgfsys@transformshift{3.598971in}{1.726992in}%
\pgfsys@useobject{currentmarker}{}%
\end{pgfscope}%
\begin{pgfscope}%
\pgfsys@transformshift{3.599431in}{1.720359in}%
\pgfsys@useobject{currentmarker}{}%
\end{pgfscope}%
\begin{pgfscope}%
\pgfsys@transformshift{3.602232in}{1.711024in}%
\pgfsys@useobject{currentmarker}{}%
\end{pgfscope}%
\begin{pgfscope}%
\pgfsys@transformshift{3.600163in}{1.699344in}%
\pgfsys@useobject{currentmarker}{}%
\end{pgfscope}%
\begin{pgfscope}%
\pgfsys@transformshift{3.598083in}{1.685698in}%
\pgfsys@useobject{currentmarker}{}%
\end{pgfscope}%
\begin{pgfscope}%
\pgfsys@transformshift{3.602024in}{1.670058in}%
\pgfsys@useobject{currentmarker}{}%
\end{pgfscope}%
\begin{pgfscope}%
\pgfsys@transformshift{3.607883in}{1.651848in}%
\pgfsys@useobject{currentmarker}{}%
\end{pgfscope}%
\begin{pgfscope}%
\pgfsys@transformshift{3.603675in}{1.629105in}%
\pgfsys@useobject{currentmarker}{}%
\end{pgfscope}%
\begin{pgfscope}%
\pgfsys@transformshift{3.600500in}{1.604794in}%
\pgfsys@useobject{currentmarker}{}%
\end{pgfscope}%
\begin{pgfscope}%
\pgfsys@transformshift{3.601346in}{1.576972in}%
\pgfsys@useobject{currentmarker}{}%
\end{pgfscope}%
\begin{pgfscope}%
\pgfsys@transformshift{3.602734in}{1.561726in}%
\pgfsys@useobject{currentmarker}{}%
\end{pgfscope}%
\begin{pgfscope}%
\pgfsys@transformshift{3.601847in}{1.545182in}%
\pgfsys@useobject{currentmarker}{}%
\end{pgfscope}%
\begin{pgfscope}%
\pgfsys@transformshift{3.602346in}{1.536083in}%
\pgfsys@useobject{currentmarker}{}%
\end{pgfscope}%
\begin{pgfscope}%
\pgfsys@transformshift{3.604647in}{1.523498in}%
\pgfsys@useobject{currentmarker}{}%
\end{pgfscope}%
\begin{pgfscope}%
\pgfsys@transformshift{3.602034in}{1.508349in}%
\pgfsys@useobject{currentmarker}{}%
\end{pgfscope}%
\begin{pgfscope}%
\pgfsys@transformshift{3.599627in}{1.491578in}%
\pgfsys@useobject{currentmarker}{}%
\end{pgfscope}%
\begin{pgfscope}%
\pgfsys@transformshift{3.603010in}{1.473078in}%
\pgfsys@useobject{currentmarker}{}%
\end{pgfscope}%
\begin{pgfscope}%
\pgfsys@transformshift{3.607806in}{1.452170in}%
\pgfsys@useobject{currentmarker}{}%
\end{pgfscope}%
\begin{pgfscope}%
\pgfsys@transformshift{3.605649in}{1.426888in}%
\pgfsys@useobject{currentmarker}{}%
\end{pgfscope}%
\begin{pgfscope}%
\pgfsys@transformshift{3.607239in}{1.400852in}%
\pgfsys@useobject{currentmarker}{}%
\end{pgfscope}%
\begin{pgfscope}%
\pgfsys@transformshift{3.611975in}{1.371724in}%
\pgfsys@useobject{currentmarker}{}%
\end{pgfscope}%
\begin{pgfscope}%
\pgfsys@transformshift{3.612062in}{1.341268in}%
\pgfsys@useobject{currentmarker}{}%
\end{pgfscope}%
\begin{pgfscope}%
\pgfsys@transformshift{3.607463in}{1.310226in}%
\pgfsys@useobject{currentmarker}{}%
\end{pgfscope}%
\begin{pgfscope}%
\pgfsys@transformshift{3.611151in}{1.276926in}%
\pgfsys@useobject{currentmarker}{}%
\end{pgfscope}%
\begin{pgfscope}%
\pgfsys@transformshift{3.610721in}{1.242345in}%
\pgfsys@useobject{currentmarker}{}%
\end{pgfscope}%
\begin{pgfscope}%
\pgfsys@transformshift{3.615346in}{1.206285in}%
\pgfsys@useobject{currentmarker}{}%
\end{pgfscope}%
\begin{pgfscope}%
\pgfsys@transformshift{3.611733in}{1.164892in}%
\pgfsys@useobject{currentmarker}{}%
\end{pgfscope}%
\begin{pgfscope}%
\pgfsys@transformshift{3.604498in}{1.122095in}%
\pgfsys@useobject{currentmarker}{}%
\end{pgfscope}%
\begin{pgfscope}%
\pgfsys@transformshift{3.604147in}{1.077044in}%
\pgfsys@useobject{currentmarker}{}%
\end{pgfscope}%
\begin{pgfscope}%
\pgfsys@transformshift{3.599335in}{1.052737in}%
\pgfsys@useobject{currentmarker}{}%
\end{pgfscope}%
\begin{pgfscope}%
\pgfsys@transformshift{3.592685in}{1.040841in}%
\pgfsys@useobject{currentmarker}{}%
\end{pgfscope}%
\begin{pgfscope}%
\pgfsys@transformshift{3.579976in}{1.031929in}%
\pgfsys@useobject{currentmarker}{}%
\end{pgfscope}%
\begin{pgfscope}%
\pgfsys@transformshift{3.571910in}{1.029132in}%
\pgfsys@useobject{currentmarker}{}%
\end{pgfscope}%
\begin{pgfscope}%
\pgfsys@transformshift{3.561891in}{1.030518in}%
\pgfsys@useobject{currentmarker}{}%
\end{pgfscope}%
\begin{pgfscope}%
\pgfsys@transformshift{3.551208in}{1.030350in}%
\pgfsys@useobject{currentmarker}{}%
\end{pgfscope}%
\begin{pgfscope}%
\pgfsys@transformshift{3.545661in}{1.032289in}%
\pgfsys@useobject{currentmarker}{}%
\end{pgfscope}%
\begin{pgfscope}%
\pgfsys@transformshift{3.542463in}{1.032757in}%
\pgfsys@useobject{currentmarker}{}%
\end{pgfscope}%
\begin{pgfscope}%
\pgfsys@transformshift{3.538616in}{1.034226in}%
\pgfsys@useobject{currentmarker}{}%
\end{pgfscope}%
\begin{pgfscope}%
\pgfsys@transformshift{3.534065in}{1.035126in}%
\pgfsys@useobject{currentmarker}{}%
\end{pgfscope}%
\begin{pgfscope}%
\pgfsys@transformshift{3.528961in}{1.036523in}%
\pgfsys@useobject{currentmarker}{}%
\end{pgfscope}%
\begin{pgfscope}%
\pgfsys@transformshift{3.523209in}{1.038149in}%
\pgfsys@useobject{currentmarker}{}%
\end{pgfscope}%
\begin{pgfscope}%
\pgfsys@transformshift{3.516623in}{1.038192in}%
\pgfsys@useobject{currentmarker}{}%
\end{pgfscope}%
\begin{pgfscope}%
\pgfsys@transformshift{3.506195in}{1.039012in}%
\pgfsys@useobject{currentmarker}{}%
\end{pgfscope}%
\begin{pgfscope}%
\pgfsys@transformshift{3.495272in}{1.040817in}%
\pgfsys@useobject{currentmarker}{}%
\end{pgfscope}%
\begin{pgfscope}%
\pgfsys@transformshift{3.482187in}{1.039925in}%
\pgfsys@useobject{currentmarker}{}%
\end{pgfscope}%
\begin{pgfscope}%
\pgfsys@transformshift{3.467600in}{1.034263in}%
\pgfsys@useobject{currentmarker}{}%
\end{pgfscope}%
\begin{pgfscope}%
\pgfsys@transformshift{3.450584in}{1.033372in}%
\pgfsys@useobject{currentmarker}{}%
\end{pgfscope}%
\begin{pgfscope}%
\pgfsys@transformshift{3.427452in}{1.032471in}%
\pgfsys@useobject{currentmarker}{}%
\end{pgfscope}%
\begin{pgfscope}%
\pgfsys@transformshift{3.405059in}{1.024336in}%
\pgfsys@useobject{currentmarker}{}%
\end{pgfscope}%
\begin{pgfscope}%
\pgfsys@transformshift{3.378829in}{1.022984in}%
\pgfsys@useobject{currentmarker}{}%
\end{pgfscope}%
\begin{pgfscope}%
\pgfsys@transformshift{3.347884in}{1.021213in}%
\pgfsys@useobject{currentmarker}{}%
\end{pgfscope}%
\begin{pgfscope}%
\pgfsys@transformshift{3.314373in}{1.018685in}%
\pgfsys@useobject{currentmarker}{}%
\end{pgfscope}%
\begin{pgfscope}%
\pgfsys@transformshift{3.279065in}{1.008745in}%
\pgfsys@useobject{currentmarker}{}%
\end{pgfscope}%
\begin{pgfscope}%
\pgfsys@transformshift{3.240155in}{1.009391in}%
\pgfsys@useobject{currentmarker}{}%
\end{pgfscope}%
\begin{pgfscope}%
\pgfsys@transformshift{3.197064in}{1.009076in}%
\pgfsys@useobject{currentmarker}{}%
\end{pgfscope}%
\begin{pgfscope}%
\pgfsys@transformshift{3.153223in}{1.003494in}%
\pgfsys@useobject{currentmarker}{}%
\end{pgfscope}%
\begin{pgfscope}%
\pgfsys@transformshift{3.104724in}{0.997750in}%
\pgfsys@useobject{currentmarker}{}%
\end{pgfscope}%
\begin{pgfscope}%
\pgfsys@transformshift{3.055557in}{0.993754in}%
\pgfsys@useobject{currentmarker}{}%
\end{pgfscope}%
\begin{pgfscope}%
\pgfsys@transformshift{3.000769in}{0.996145in}%
\pgfsys@useobject{currentmarker}{}%
\end{pgfscope}%
\begin{pgfscope}%
\pgfsys@transformshift{2.970722in}{0.993517in}%
\pgfsys@useobject{currentmarker}{}%
\end{pgfscope}%
\begin{pgfscope}%
\pgfsys@transformshift{2.937767in}{0.985131in}%
\pgfsys@useobject{currentmarker}{}%
\end{pgfscope}%
\begin{pgfscope}%
\pgfsys@transformshift{2.919101in}{0.983961in}%
\pgfsys@useobject{currentmarker}{}%
\end{pgfscope}%
\begin{pgfscope}%
\pgfsys@transformshift{2.894530in}{0.984263in}%
\pgfsys@useobject{currentmarker}{}%
\end{pgfscope}%
\begin{pgfscope}%
\pgfsys@transformshift{2.869445in}{0.982036in}%
\pgfsys@useobject{currentmarker}{}%
\end{pgfscope}%
\begin{pgfscope}%
\pgfsys@transformshift{2.842310in}{0.976196in}%
\pgfsys@useobject{currentmarker}{}%
\end{pgfscope}%
\begin{pgfscope}%
\pgfsys@transformshift{2.813634in}{0.977199in}%
\pgfsys@useobject{currentmarker}{}%
\end{pgfscope}%
\begin{pgfscope}%
\pgfsys@transformshift{2.780670in}{0.974685in}%
\pgfsys@useobject{currentmarker}{}%
\end{pgfscope}%
\begin{pgfscope}%
\pgfsys@transformshift{2.762911in}{0.970783in}%
\pgfsys@useobject{currentmarker}{}%
\end{pgfscope}%
\begin{pgfscope}%
\pgfsys@transformshift{2.740591in}{0.969180in}%
\pgfsys@useobject{currentmarker}{}%
\end{pgfscope}%
\begin{pgfscope}%
\pgfsys@transformshift{2.716818in}{0.971285in}%
\pgfsys@useobject{currentmarker}{}%
\end{pgfscope}%
\begin{pgfscope}%
\pgfsys@transformshift{2.689741in}{0.968668in}%
\pgfsys@useobject{currentmarker}{}%
\end{pgfscope}%
\begin{pgfscope}%
\pgfsys@transformshift{2.661728in}{0.968244in}%
\pgfsys@useobject{currentmarker}{}%
\end{pgfscope}%
\begin{pgfscope}%
\pgfsys@transformshift{2.629942in}{0.965709in}%
\pgfsys@useobject{currentmarker}{}%
\end{pgfscope}%
\begin{pgfscope}%
\pgfsys@transformshift{2.612429in}{0.966655in}%
\pgfsys@useobject{currentmarker}{}%
\end{pgfscope}%
\begin{pgfscope}%
\pgfsys@transformshift{2.591065in}{0.965260in}%
\pgfsys@useobject{currentmarker}{}%
\end{pgfscope}%
\begin{pgfscope}%
\pgfsys@transformshift{2.569647in}{0.958362in}%
\pgfsys@useobject{currentmarker}{}%
\end{pgfscope}%
\begin{pgfscope}%
\pgfsys@transformshift{2.544877in}{0.957925in}%
\pgfsys@useobject{currentmarker}{}%
\end{pgfscope}%
\begin{pgfscope}%
\pgfsys@transformshift{2.517184in}{0.957037in}%
\pgfsys@useobject{currentmarker}{}%
\end{pgfscope}%
\begin{pgfscope}%
\pgfsys@transformshift{2.488332in}{0.950610in}%
\pgfsys@useobject{currentmarker}{}%
\end{pgfscope}%
\begin{pgfscope}%
\pgfsys@transformshift{2.458249in}{0.942347in}%
\pgfsys@useobject{currentmarker}{}%
\end{pgfscope}%
\begin{pgfscope}%
\pgfsys@transformshift{2.424906in}{0.937899in}%
\pgfsys@useobject{currentmarker}{}%
\end{pgfscope}%
\begin{pgfscope}%
\pgfsys@transformshift{2.386924in}{0.933874in}%
\pgfsys@useobject{currentmarker}{}%
\end{pgfscope}%
\begin{pgfscope}%
\pgfsys@transformshift{2.365953in}{0.932641in}%
\pgfsys@useobject{currentmarker}{}%
\end{pgfscope}%
\begin{pgfscope}%
\pgfsys@transformshift{2.343289in}{0.924153in}%
\pgfsys@useobject{currentmarker}{}%
\end{pgfscope}%
\begin{pgfscope}%
\pgfsys@transformshift{2.318510in}{0.920792in}%
\pgfsys@useobject{currentmarker}{}%
\end{pgfscope}%
\begin{pgfscope}%
\pgfsys@transformshift{2.288654in}{0.918733in}%
\pgfsys@useobject{currentmarker}{}%
\end{pgfscope}%
\begin{pgfscope}%
\pgfsys@transformshift{2.258351in}{0.915010in}%
\pgfsys@useobject{currentmarker}{}%
\end{pgfscope}%
\begin{pgfscope}%
\pgfsys@transformshift{2.226405in}{0.907144in}%
\pgfsys@useobject{currentmarker}{}%
\end{pgfscope}%
\begin{pgfscope}%
\pgfsys@transformshift{2.208318in}{0.906602in}%
\pgfsys@useobject{currentmarker}{}%
\end{pgfscope}%
\begin{pgfscope}%
\pgfsys@transformshift{2.185631in}{0.906772in}%
\pgfsys@useobject{currentmarker}{}%
\end{pgfscope}%
\begin{pgfscope}%
\pgfsys@transformshift{2.161766in}{0.904081in}%
\pgfsys@useobject{currentmarker}{}%
\end{pgfscope}%
\begin{pgfscope}%
\pgfsys@transformshift{2.136956in}{0.892633in}%
\pgfsys@useobject{currentmarker}{}%
\end{pgfscope}%
\begin{pgfscope}%
\pgfsys@transformshift{2.106457in}{0.890732in}%
\pgfsys@useobject{currentmarker}{}%
\end{pgfscope}%
\begin{pgfscope}%
\pgfsys@transformshift{2.074298in}{0.891304in}%
\pgfsys@useobject{currentmarker}{}%
\end{pgfscope}%
\begin{pgfscope}%
\pgfsys@transformshift{2.038306in}{0.889833in}%
\pgfsys@useobject{currentmarker}{}%
\end{pgfscope}%
\begin{pgfscope}%
\pgfsys@transformshift{2.002481in}{0.880949in}%
\pgfsys@useobject{currentmarker}{}%
\end{pgfscope}%
\begin{pgfscope}%
\pgfsys@transformshift{1.960913in}{0.877311in}%
\pgfsys@useobject{currentmarker}{}%
\end{pgfscope}%
\begin{pgfscope}%
\pgfsys@transformshift{1.918729in}{0.876292in}%
\pgfsys@useobject{currentmarker}{}%
\end{pgfscope}%
\begin{pgfscope}%
\pgfsys@transformshift{1.872058in}{0.870751in}%
\pgfsys@useobject{currentmarker}{}%
\end{pgfscope}%
\begin{pgfscope}%
\pgfsys@transformshift{1.847588in}{0.862417in}%
\pgfsys@useobject{currentmarker}{}%
\end{pgfscope}%
\begin{pgfscope}%
\pgfsys@transformshift{1.817791in}{0.859472in}%
\pgfsys@useobject{currentmarker}{}%
\end{pgfscope}%
\begin{pgfscope}%
\pgfsys@transformshift{1.801334in}{0.858854in}%
\pgfsys@useobject{currentmarker}{}%
\end{pgfscope}%
\begin{pgfscope}%
\pgfsys@transformshift{1.779991in}{0.857227in}%
\pgfsys@useobject{currentmarker}{}%
\end{pgfscope}%
\begin{pgfscope}%
\pgfsys@transformshift{1.768265in}{0.856164in}%
\pgfsys@useobject{currentmarker}{}%
\end{pgfscope}%
\begin{pgfscope}%
\pgfsys@transformshift{1.753639in}{0.857684in}%
\pgfsys@useobject{currentmarker}{}%
\end{pgfscope}%
\begin{pgfscope}%
\pgfsys@transformshift{1.745680in}{0.859126in}%
\pgfsys@useobject{currentmarker}{}%
\end{pgfscope}%
\begin{pgfscope}%
\pgfsys@transformshift{1.733528in}{0.861834in}%
\pgfsys@useobject{currentmarker}{}%
\end{pgfscope}%
\begin{pgfscope}%
\pgfsys@transformshift{1.726796in}{0.863087in}%
\pgfsys@useobject{currentmarker}{}%
\end{pgfscope}%
\begin{pgfscope}%
\pgfsys@transformshift{1.717815in}{0.864903in}%
\pgfsys@useobject{currentmarker}{}%
\end{pgfscope}%
\begin{pgfscope}%
\pgfsys@transformshift{1.712940in}{0.866179in}%
\pgfsys@useobject{currentmarker}{}%
\end{pgfscope}%
\begin{pgfscope}%
\pgfsys@transformshift{1.704575in}{0.867243in}%
\pgfsys@useobject{currentmarker}{}%
\end{pgfscope}%
\begin{pgfscope}%
\pgfsys@transformshift{1.700034in}{0.868183in}%
\pgfsys@useobject{currentmarker}{}%
\end{pgfscope}%
\begin{pgfscope}%
\pgfsys@transformshift{1.694216in}{0.868202in}%
\pgfsys@useobject{currentmarker}{}%
\end{pgfscope}%
\begin{pgfscope}%
\pgfsys@transformshift{1.687876in}{0.869778in}%
\pgfsys@useobject{currentmarker}{}%
\end{pgfscope}%
\begin{pgfscope}%
\pgfsys@transformshift{1.679639in}{0.870397in}%
\pgfsys@useobject{currentmarker}{}%
\end{pgfscope}%
\begin{pgfscope}%
\pgfsys@transformshift{1.675170in}{0.871208in}%
\pgfsys@useobject{currentmarker}{}%
\end{pgfscope}%
\begin{pgfscope}%
\pgfsys@transformshift{1.669884in}{0.870718in}%
\pgfsys@useobject{currentmarker}{}%
\end{pgfscope}%
\begin{pgfscope}%
\pgfsys@transformshift{1.664133in}{0.871704in}%
\pgfsys@useobject{currentmarker}{}%
\end{pgfscope}%
\begin{pgfscope}%
\pgfsys@transformshift{1.655030in}{0.872623in}%
\pgfsys@useobject{currentmarker}{}%
\end{pgfscope}%
\begin{pgfscope}%
\pgfsys@transformshift{1.644510in}{0.872607in}%
\pgfsys@useobject{currentmarker}{}%
\end{pgfscope}%
\begin{pgfscope}%
\pgfsys@transformshift{1.632153in}{0.872634in}%
\pgfsys@useobject{currentmarker}{}%
\end{pgfscope}%
\begin{pgfscope}%
\pgfsys@transformshift{1.617864in}{0.874650in}%
\pgfsys@useobject{currentmarker}{}%
\end{pgfscope}%
\begin{pgfscope}%
\pgfsys@transformshift{1.600601in}{0.874582in}%
\pgfsys@useobject{currentmarker}{}%
\end{pgfscope}%
\begin{pgfscope}%
\pgfsys@transformshift{1.591109in}{0.874340in}%
\pgfsys@useobject{currentmarker}{}%
\end{pgfscope}%
\begin{pgfscope}%
\pgfsys@transformshift{1.580001in}{0.874717in}%
\pgfsys@useobject{currentmarker}{}%
\end{pgfscope}%
\begin{pgfscope}%
\pgfsys@transformshift{1.567598in}{0.875656in}%
\pgfsys@useobject{currentmarker}{}%
\end{pgfscope}%
\begin{pgfscope}%
\pgfsys@transformshift{1.553633in}{0.875904in}%
\pgfsys@useobject{currentmarker}{}%
\end{pgfscope}%
\begin{pgfscope}%
\pgfsys@transformshift{1.545963in}{0.876338in}%
\pgfsys@useobject{currentmarker}{}%
\end{pgfscope}%
\begin{pgfscope}%
\pgfsys@transformshift{1.537147in}{0.877309in}%
\pgfsys@useobject{currentmarker}{}%
\end{pgfscope}%
\begin{pgfscope}%
\pgfsys@transformshift{1.532949in}{0.879792in}%
\pgfsys@useobject{currentmarker}{}%
\end{pgfscope}%
\begin{pgfscope}%
\pgfsys@transformshift{1.528103in}{0.882675in}%
\pgfsys@useobject{currentmarker}{}%
\end{pgfscope}%
\begin{pgfscope}%
\pgfsys@transformshift{1.525591in}{0.884492in}%
\pgfsys@useobject{currentmarker}{}%
\end{pgfscope}%
\begin{pgfscope}%
\pgfsys@transformshift{1.524823in}{0.886015in}%
\pgfsys@useobject{currentmarker}{}%
\end{pgfscope}%
\begin{pgfscope}%
\pgfsys@transformshift{1.523581in}{0.888239in}%
\pgfsys@useobject{currentmarker}{}%
\end{pgfscope}%
\begin{pgfscope}%
\pgfsys@transformshift{1.523592in}{0.889639in}%
\pgfsys@useobject{currentmarker}{}%
\end{pgfscope}%
\begin{pgfscope}%
\pgfsys@transformshift{1.523440in}{0.891811in}%
\pgfsys@useobject{currentmarker}{}%
\end{pgfscope}%
\begin{pgfscope}%
\pgfsys@transformshift{1.523575in}{0.895295in}%
\pgfsys@useobject{currentmarker}{}%
\end{pgfscope}%
\begin{pgfscope}%
\pgfsys@transformshift{1.523925in}{0.899613in}%
\pgfsys@useobject{currentmarker}{}%
\end{pgfscope}%
\begin{pgfscope}%
\pgfsys@transformshift{1.524169in}{0.904554in}%
\pgfsys@useobject{currentmarker}{}%
\end{pgfscope}%
\begin{pgfscope}%
\pgfsys@transformshift{1.524302in}{0.907272in}%
\pgfsys@useobject{currentmarker}{}%
\end{pgfscope}%
\begin{pgfscope}%
\pgfsys@transformshift{1.524404in}{0.908765in}%
\pgfsys@useobject{currentmarker}{}%
\end{pgfscope}%
\begin{pgfscope}%
\pgfsys@transformshift{1.524186in}{0.910804in}%
\pgfsys@useobject{currentmarker}{}%
\end{pgfscope}%
\begin{pgfscope}%
\pgfsys@transformshift{1.524026in}{0.913350in}%
\pgfsys@useobject{currentmarker}{}%
\end{pgfscope}%
\begin{pgfscope}%
\pgfsys@transformshift{1.523835in}{0.916492in}%
\pgfsys@useobject{currentmarker}{}%
\end{pgfscope}%
\begin{pgfscope}%
\pgfsys@transformshift{1.523858in}{0.918223in}%
\pgfsys@useobject{currentmarker}{}%
\end{pgfscope}%
\begin{pgfscope}%
\pgfsys@transformshift{1.523676in}{0.920594in}%
\pgfsys@useobject{currentmarker}{}%
\end{pgfscope}%
\begin{pgfscope}%
\pgfsys@transformshift{1.523737in}{0.921900in}%
\pgfsys@useobject{currentmarker}{}%
\end{pgfscope}%
\begin{pgfscope}%
\pgfsys@transformshift{1.523478in}{0.924112in}%
\pgfsys@useobject{currentmarker}{}%
\end{pgfscope}%
\begin{pgfscope}%
\pgfsys@transformshift{1.523569in}{0.926949in}%
\pgfsys@useobject{currentmarker}{}%
\end{pgfscope}%
\begin{pgfscope}%
\pgfsys@transformshift{1.523451in}{0.928506in}%
\pgfsys@useobject{currentmarker}{}%
\end{pgfscope}%
\begin{pgfscope}%
\pgfsys@transformshift{1.523487in}{0.929364in}%
\pgfsys@useobject{currentmarker}{}%
\end{pgfscope}%
\begin{pgfscope}%
\pgfsys@transformshift{1.523344in}{0.930712in}%
\pgfsys@useobject{currentmarker}{}%
\end{pgfscope}%
\begin{pgfscope}%
\pgfsys@transformshift{1.523459in}{0.932525in}%
\pgfsys@useobject{currentmarker}{}%
\end{pgfscope}%
\begin{pgfscope}%
\pgfsys@transformshift{1.523218in}{0.934884in}%
\pgfsys@useobject{currentmarker}{}%
\end{pgfscope}%
\begin{pgfscope}%
\pgfsys@transformshift{1.523292in}{0.936186in}%
\pgfsys@useobject{currentmarker}{}%
\end{pgfscope}%
\begin{pgfscope}%
\pgfsys@transformshift{1.523219in}{0.936899in}%
\pgfsys@useobject{currentmarker}{}%
\end{pgfscope}%
\begin{pgfscope}%
\pgfsys@transformshift{1.523250in}{0.937292in}%
\pgfsys@useobject{currentmarker}{}%
\end{pgfscope}%
\begin{pgfscope}%
\pgfsys@transformshift{1.523227in}{0.937508in}%
\pgfsys@useobject{currentmarker}{}%
\end{pgfscope}%
\begin{pgfscope}%
\pgfsys@transformshift{1.523236in}{0.937627in}%
\pgfsys@useobject{currentmarker}{}%
\end{pgfscope}%
\begin{pgfscope}%
\pgfsys@transformshift{1.523228in}{0.937692in}%
\pgfsys@useobject{currentmarker}{}%
\end{pgfscope}%
\begin{pgfscope}%
\pgfsys@transformshift{1.523275in}{0.938238in}%
\pgfsys@useobject{currentmarker}{}%
\end{pgfscope}%
\begin{pgfscope}%
\pgfsys@transformshift{1.523191in}{0.939270in}%
\pgfsys@useobject{currentmarker}{}%
\end{pgfscope}%
\begin{pgfscope}%
\pgfsys@transformshift{1.523231in}{0.939838in}%
\pgfsys@useobject{currentmarker}{}%
\end{pgfscope}%
\begin{pgfscope}%
\pgfsys@transformshift{1.523202in}{0.940150in}%
\pgfsys@useobject{currentmarker}{}%
\end{pgfscope}%
\begin{pgfscope}%
\pgfsys@transformshift{1.523213in}{0.940322in}%
\pgfsys@useobject{currentmarker}{}%
\end{pgfscope}%
\begin{pgfscope}%
\pgfsys@transformshift{1.523205in}{0.940416in}%
\pgfsys@useobject{currentmarker}{}%
\end{pgfscope}%
\begin{pgfscope}%
\pgfsys@transformshift{1.523208in}{0.940468in}%
\pgfsys@useobject{currentmarker}{}%
\end{pgfscope}%
\begin{pgfscope}%
\pgfsys@transformshift{1.523206in}{0.940497in}%
\pgfsys@useobject{currentmarker}{}%
\end{pgfscope}%
\begin{pgfscope}%
\pgfsys@transformshift{1.523207in}{0.940513in}%
\pgfsys@useobject{currentmarker}{}%
\end{pgfscope}%
\begin{pgfscope}%
\pgfsys@transformshift{1.523206in}{0.940521in}%
\pgfsys@useobject{currentmarker}{}%
\end{pgfscope}%
\begin{pgfscope}%
\pgfsys@transformshift{1.523206in}{0.940526in}%
\pgfsys@useobject{currentmarker}{}%
\end{pgfscope}%
\begin{pgfscope}%
\pgfsys@transformshift{1.523206in}{0.940529in}%
\pgfsys@useobject{currentmarker}{}%
\end{pgfscope}%
\begin{pgfscope}%
\pgfsys@transformshift{1.523206in}{0.940530in}%
\pgfsys@useobject{currentmarker}{}%
\end{pgfscope}%
\begin{pgfscope}%
\pgfsys@transformshift{1.523206in}{0.940531in}%
\pgfsys@useobject{currentmarker}{}%
\end{pgfscope}%
\begin{pgfscope}%
\pgfsys@transformshift{1.523206in}{0.940531in}%
\pgfsys@useobject{currentmarker}{}%
\end{pgfscope}%
\begin{pgfscope}%
\pgfsys@transformshift{1.523206in}{0.940531in}%
\pgfsys@useobject{currentmarker}{}%
\end{pgfscope}%
\begin{pgfscope}%
\pgfsys@transformshift{1.522952in}{0.941287in}%
\pgfsys@useobject{currentmarker}{}%
\end{pgfscope}%
\begin{pgfscope}%
\pgfsys@transformshift{1.523238in}{0.944973in}%
\pgfsys@useobject{currentmarker}{}%
\end{pgfscope}%
\begin{pgfscope}%
\pgfsys@transformshift{1.523533in}{0.949311in}%
\pgfsys@useobject{currentmarker}{}%
\end{pgfscope}%
\begin{pgfscope}%
\pgfsys@transformshift{1.521795in}{0.955312in}%
\pgfsys@useobject{currentmarker}{}%
\end{pgfscope}%
\begin{pgfscope}%
\pgfsys@transformshift{1.521101in}{0.962075in}%
\pgfsys@useobject{currentmarker}{}%
\end{pgfscope}%
\begin{pgfscope}%
\pgfsys@transformshift{1.521989in}{0.965708in}%
\pgfsys@useobject{currentmarker}{}%
\end{pgfscope}%
\begin{pgfscope}%
\pgfsys@transformshift{1.520132in}{0.972385in}%
\pgfsys@useobject{currentmarker}{}%
\end{pgfscope}%
\begin{pgfscope}%
\pgfsys@transformshift{1.520707in}{0.976153in}%
\pgfsys@useobject{currentmarker}{}%
\end{pgfscope}%
\begin{pgfscope}%
\pgfsys@transformshift{1.520826in}{0.978246in}%
\pgfsys@useobject{currentmarker}{}%
\end{pgfscope}%
\begin{pgfscope}%
\pgfsys@transformshift{1.520138in}{0.984317in}%
\pgfsys@useobject{currentmarker}{}%
\end{pgfscope}%
\begin{pgfscope}%
\pgfsys@transformshift{1.521140in}{0.990843in}%
\pgfsys@useobject{currentmarker}{}%
\end{pgfscope}%
\begin{pgfscope}%
\pgfsys@transformshift{1.523776in}{1.000735in}%
\pgfsys@useobject{currentmarker}{}%
\end{pgfscope}%
\begin{pgfscope}%
\pgfsys@transformshift{1.521775in}{1.013781in}%
\pgfsys@useobject{currentmarker}{}%
\end{pgfscope}%
\begin{pgfscope}%
\pgfsys@transformshift{1.523206in}{1.027429in}%
\pgfsys@useobject{currentmarker}{}%
\end{pgfscope}%
\begin{pgfscope}%
\pgfsys@transformshift{1.524164in}{1.044657in}%
\pgfsys@useobject{currentmarker}{}%
\end{pgfscope}%
\begin{pgfscope}%
\pgfsys@transformshift{1.526577in}{1.062352in}%
\pgfsys@useobject{currentmarker}{}%
\end{pgfscope}%
\begin{pgfscope}%
\pgfsys@transformshift{1.524583in}{1.071969in}%
\pgfsys@useobject{currentmarker}{}%
\end{pgfscope}%
\begin{pgfscope}%
\pgfsys@transformshift{1.525435in}{1.083147in}%
\pgfsys@useobject{currentmarker}{}%
\end{pgfscope}%
\begin{pgfscope}%
\pgfsys@transformshift{1.527681in}{1.094900in}%
\pgfsys@useobject{currentmarker}{}%
\end{pgfscope}%
\begin{pgfscope}%
\pgfsys@transformshift{1.524910in}{1.109677in}%
\pgfsys@useobject{currentmarker}{}%
\end{pgfscope}%
\begin{pgfscope}%
\pgfsys@transformshift{1.525457in}{1.125348in}%
\pgfsys@useobject{currentmarker}{}%
\end{pgfscope}%
\begin{pgfscope}%
\pgfsys@transformshift{1.530103in}{1.141468in}%
\pgfsys@useobject{currentmarker}{}%
\end{pgfscope}%
\begin{pgfscope}%
\pgfsys@transformshift{1.526662in}{1.163395in}%
\pgfsys@useobject{currentmarker}{}%
\end{pgfscope}%
\begin{pgfscope}%
\pgfsys@transformshift{1.525682in}{1.186751in}%
\pgfsys@useobject{currentmarker}{}%
\end{pgfscope}%
\begin{pgfscope}%
\pgfsys@transformshift{1.517118in}{1.211858in}%
\pgfsys@useobject{currentmarker}{}%
\end{pgfscope}%
\begin{pgfscope}%
\pgfsys@transformshift{1.521363in}{1.225818in}%
\pgfsys@useobject{currentmarker}{}%
\end{pgfscope}%
\begin{pgfscope}%
\pgfsys@transformshift{1.518630in}{1.245733in}%
\pgfsys@useobject{currentmarker}{}%
\end{pgfscope}%
\begin{pgfscope}%
\pgfsys@transformshift{1.518290in}{1.256784in}%
\pgfsys@useobject{currentmarker}{}%
\end{pgfscope}%
\begin{pgfscope}%
\pgfsys@transformshift{1.522022in}{1.270890in}%
\pgfsys@useobject{currentmarker}{}%
\end{pgfscope}%
\begin{pgfscope}%
\pgfsys@transformshift{1.521317in}{1.287224in}%
\pgfsys@useobject{currentmarker}{}%
\end{pgfscope}%
\begin{pgfscope}%
\pgfsys@transformshift{1.520731in}{1.296196in}%
\pgfsys@useobject{currentmarker}{}%
\end{pgfscope}%
\begin{pgfscope}%
\pgfsys@transformshift{1.520599in}{1.308367in}%
\pgfsys@useobject{currentmarker}{}%
\end{pgfscope}%
\begin{pgfscope}%
\pgfsys@transformshift{1.522345in}{1.314830in}%
\pgfsys@useobject{currentmarker}{}%
\end{pgfscope}%
\begin{pgfscope}%
\pgfsys@transformshift{1.520472in}{1.324082in}%
\pgfsys@useobject{currentmarker}{}%
\end{pgfscope}%
\begin{pgfscope}%
\pgfsys@transformshift{1.520398in}{1.329273in}%
\pgfsys@useobject{currentmarker}{}%
\end{pgfscope}%
\begin{pgfscope}%
\pgfsys@transformshift{1.523216in}{1.338701in}%
\pgfsys@useobject{currentmarker}{}%
\end{pgfscope}%
\begin{pgfscope}%
\pgfsys@transformshift{1.522178in}{1.351675in}%
\pgfsys@useobject{currentmarker}{}%
\end{pgfscope}%
\begin{pgfscope}%
\pgfsys@transformshift{1.523543in}{1.365201in}%
\pgfsys@useobject{currentmarker}{}%
\end{pgfscope}%
\begin{pgfscope}%
\pgfsys@transformshift{1.530648in}{1.382343in}%
\pgfsys@useobject{currentmarker}{}%
\end{pgfscope}%
\begin{pgfscope}%
\pgfsys@transformshift{1.530707in}{1.401606in}%
\pgfsys@useobject{currentmarker}{}%
\end{pgfscope}%
\begin{pgfscope}%
\pgfsys@transformshift{1.526143in}{1.420960in}%
\pgfsys@useobject{currentmarker}{}%
\end{pgfscope}%
\begin{pgfscope}%
\pgfsys@transformshift{1.534430in}{1.444777in}%
\pgfsys@useobject{currentmarker}{}%
\end{pgfscope}%
\begin{pgfscope}%
\pgfsys@transformshift{1.535922in}{1.458567in}%
\pgfsys@useobject{currentmarker}{}%
\end{pgfscope}%
\begin{pgfscope}%
\pgfsys@transformshift{1.535253in}{1.473015in}%
\pgfsys@useobject{currentmarker}{}%
\end{pgfscope}%
\begin{pgfscope}%
\pgfsys@transformshift{1.537358in}{1.490827in}%
\pgfsys@useobject{currentmarker}{}%
\end{pgfscope}%
\begin{pgfscope}%
\pgfsys@transformshift{1.540146in}{1.509060in}%
\pgfsys@useobject{currentmarker}{}%
\end{pgfscope}%
\begin{pgfscope}%
\pgfsys@transformshift{1.535713in}{1.529583in}%
\pgfsys@useobject{currentmarker}{}%
\end{pgfscope}%
\begin{pgfscope}%
\pgfsys@transformshift{1.536757in}{1.551250in}%
\pgfsys@useobject{currentmarker}{}%
\end{pgfscope}%
\begin{pgfscope}%
\pgfsys@transformshift{1.542287in}{1.573384in}%
\pgfsys@useobject{currentmarker}{}%
\end{pgfscope}%
\begin{pgfscope}%
\pgfsys@transformshift{1.538332in}{1.601627in}%
\pgfsys@useobject{currentmarker}{}%
\end{pgfscope}%
\begin{pgfscope}%
\pgfsys@transformshift{1.538085in}{1.617310in}%
\pgfsys@useobject{currentmarker}{}%
\end{pgfscope}%
\begin{pgfscope}%
\pgfsys@transformshift{1.540313in}{1.636189in}%
\pgfsys@useobject{currentmarker}{}%
\end{pgfscope}%
\begin{pgfscope}%
\pgfsys@transformshift{1.541929in}{1.646518in}%
\pgfsys@useobject{currentmarker}{}%
\end{pgfscope}%
\begin{pgfscope}%
\pgfsys@transformshift{1.539817in}{1.657724in}%
\pgfsys@useobject{currentmarker}{}%
\end{pgfscope}%
\begin{pgfscope}%
\pgfsys@transformshift{1.539689in}{1.670670in}%
\pgfsys@useobject{currentmarker}{}%
\end{pgfscope}%
\begin{pgfscope}%
\pgfsys@transformshift{1.544305in}{1.684995in}%
\pgfsys@useobject{currentmarker}{}%
\end{pgfscope}%
\begin{pgfscope}%
\pgfsys@transformshift{1.541653in}{1.702957in}%
\pgfsys@useobject{currentmarker}{}%
\end{pgfscope}%
\begin{pgfscope}%
\pgfsys@transformshift{1.542243in}{1.712926in}%
\pgfsys@useobject{currentmarker}{}%
\end{pgfscope}%
\begin{pgfscope}%
\pgfsys@transformshift{1.547187in}{1.726191in}%
\pgfsys@useobject{currentmarker}{}%
\end{pgfscope}%
\begin{pgfscope}%
\pgfsys@transformshift{1.548739in}{1.743707in}%
\pgfsys@useobject{currentmarker}{}%
\end{pgfscope}%
\begin{pgfscope}%
\pgfsys@transformshift{1.548790in}{1.753379in}%
\pgfsys@useobject{currentmarker}{}%
\end{pgfscope}%
\begin{pgfscope}%
\pgfsys@transformshift{1.551682in}{1.766840in}%
\pgfsys@useobject{currentmarker}{}%
\end{pgfscope}%
\begin{pgfscope}%
\pgfsys@transformshift{1.552582in}{1.781054in}%
\pgfsys@useobject{currentmarker}{}%
\end{pgfscope}%
\begin{pgfscope}%
\pgfsys@transformshift{1.549181in}{1.796571in}%
\pgfsys@useobject{currentmarker}{}%
\end{pgfscope}%
\begin{pgfscope}%
\pgfsys@transformshift{1.548603in}{1.805289in}%
\pgfsys@useobject{currentmarker}{}%
\end{pgfscope}%
\begin{pgfscope}%
\pgfsys@transformshift{1.550334in}{1.814502in}%
\pgfsys@useobject{currentmarker}{}%
\end{pgfscope}%
\begin{pgfscope}%
\pgfsys@transformshift{1.547513in}{1.829257in}%
\pgfsys@useobject{currentmarker}{}%
\end{pgfscope}%
\begin{pgfscope}%
\pgfsys@transformshift{1.548027in}{1.837503in}%
\pgfsys@useobject{currentmarker}{}%
\end{pgfscope}%
\begin{pgfscope}%
\pgfsys@transformshift{1.550113in}{1.849254in}%
\pgfsys@useobject{currentmarker}{}%
\end{pgfscope}%
\begin{pgfscope}%
\pgfsys@transformshift{1.546453in}{1.864547in}%
\pgfsys@useobject{currentmarker}{}%
\end{pgfscope}%
\begin{pgfscope}%
\pgfsys@transformshift{1.546585in}{1.873195in}%
\pgfsys@useobject{currentmarker}{}%
\end{pgfscope}%
\begin{pgfscope}%
\pgfsys@transformshift{1.549781in}{1.885480in}%
\pgfsys@useobject{currentmarker}{}%
\end{pgfscope}%
\begin{pgfscope}%
\pgfsys@transformshift{1.546225in}{1.898590in}%
\pgfsys@useobject{currentmarker}{}%
\end{pgfscope}%
\begin{pgfscope}%
\pgfsys@transformshift{1.544384in}{1.912934in}%
\pgfsys@useobject{currentmarker}{}%
\end{pgfscope}%
\begin{pgfscope}%
\pgfsys@transformshift{1.543419in}{1.930835in}%
\pgfsys@useobject{currentmarker}{}%
\end{pgfscope}%
\begin{pgfscope}%
\pgfsys@transformshift{1.545746in}{1.949350in}%
\pgfsys@useobject{currentmarker}{}%
\end{pgfscope}%
\begin{pgfscope}%
\pgfsys@transformshift{1.538105in}{1.969264in}%
\pgfsys@useobject{currentmarker}{}%
\end{pgfscope}%
\begin{pgfscope}%
\pgfsys@transformshift{1.537223in}{1.991289in}%
\pgfsys@useobject{currentmarker}{}%
\end{pgfscope}%
\begin{pgfscope}%
\pgfsys@transformshift{1.541728in}{2.014690in}%
\pgfsys@useobject{currentmarker}{}%
\end{pgfscope}%
\begin{pgfscope}%
\pgfsys@transformshift{1.532299in}{2.041963in}%
\pgfsys@useobject{currentmarker}{}%
\end{pgfscope}%
\begin{pgfscope}%
\pgfsys@transformshift{1.531909in}{2.057830in}%
\pgfsys@useobject{currentmarker}{}%
\end{pgfscope}%
\begin{pgfscope}%
\pgfsys@transformshift{1.534936in}{2.077138in}%
\pgfsys@useobject{currentmarker}{}%
\end{pgfscope}%
\begin{pgfscope}%
\pgfsys@transformshift{1.535609in}{2.097380in}%
\pgfsys@useobject{currentmarker}{}%
\end{pgfscope}%
\begin{pgfscope}%
\pgfsys@transformshift{1.529521in}{2.117293in}%
\pgfsys@useobject{currentmarker}{}%
\end{pgfscope}%
\begin{pgfscope}%
\pgfsys@transformshift{1.529829in}{2.140252in}%
\pgfsys@useobject{currentmarker}{}%
\end{pgfscope}%
\begin{pgfscope}%
\pgfsys@transformshift{1.536431in}{2.165121in}%
\pgfsys@useobject{currentmarker}{}%
\end{pgfscope}%
\begin{pgfscope}%
\pgfsys@transformshift{1.530385in}{2.193093in}%
\pgfsys@useobject{currentmarker}{}%
\end{pgfscope}%
\begin{pgfscope}%
\pgfsys@transformshift{1.529559in}{2.208811in}%
\pgfsys@useobject{currentmarker}{}%
\end{pgfscope}%
\begin{pgfscope}%
\pgfsys@transformshift{1.535245in}{2.228031in}%
\pgfsys@useobject{currentmarker}{}%
\end{pgfscope}%
\begin{pgfscope}%
\pgfsys@transformshift{1.528661in}{2.250083in}%
\pgfsys@useobject{currentmarker}{}%
\end{pgfscope}%
\begin{pgfscope}%
\pgfsys@transformshift{1.522735in}{2.273225in}%
\pgfsys@useobject{currentmarker}{}%
\end{pgfscope}%
\begin{pgfscope}%
\pgfsys@transformshift{1.526343in}{2.301727in}%
\pgfsys@useobject{currentmarker}{}%
\end{pgfscope}%
\begin{pgfscope}%
\pgfsys@transformshift{1.528139in}{2.331481in}%
\pgfsys@useobject{currentmarker}{}%
\end{pgfscope}%
\begin{pgfscope}%
\pgfsys@transformshift{1.518470in}{2.360401in}%
\pgfsys@useobject{currentmarker}{}%
\end{pgfscope}%
\begin{pgfscope}%
\pgfsys@transformshift{1.526448in}{2.393686in}%
\pgfsys@useobject{currentmarker}{}%
\end{pgfscope}%
\begin{pgfscope}%
\pgfsys@transformshift{1.525275in}{2.412475in}%
\pgfsys@useobject{currentmarker}{}%
\end{pgfscope}%
\begin{pgfscope}%
\pgfsys@transformshift{1.518295in}{2.432218in}%
\pgfsys@useobject{currentmarker}{}%
\end{pgfscope}%
\begin{pgfscope}%
\pgfsys@transformshift{1.518846in}{2.443722in}%
\pgfsys@useobject{currentmarker}{}%
\end{pgfscope}%
\begin{pgfscope}%
\pgfsys@transformshift{1.519626in}{2.456614in}%
\pgfsys@useobject{currentmarker}{}%
\end{pgfscope}%
\begin{pgfscope}%
\pgfsys@transformshift{1.513837in}{2.474148in}%
\pgfsys@useobject{currentmarker}{}%
\end{pgfscope}%
\begin{pgfscope}%
\pgfsys@transformshift{1.513413in}{2.484294in}%
\pgfsys@useobject{currentmarker}{}%
\end{pgfscope}%
\begin{pgfscope}%
\pgfsys@transformshift{1.518083in}{2.497140in}%
\pgfsys@useobject{currentmarker}{}%
\end{pgfscope}%
\begin{pgfscope}%
\pgfsys@transformshift{1.514406in}{2.515035in}%
\pgfsys@useobject{currentmarker}{}%
\end{pgfscope}%
\begin{pgfscope}%
\pgfsys@transformshift{1.513579in}{2.525048in}%
\pgfsys@useobject{currentmarker}{}%
\end{pgfscope}%
\begin{pgfscope}%
\pgfsys@transformshift{1.518471in}{2.538535in}%
\pgfsys@useobject{currentmarker}{}%
\end{pgfscope}%
\begin{pgfscope}%
\pgfsys@transformshift{1.514190in}{2.556442in}%
\pgfsys@useobject{currentmarker}{}%
\end{pgfscope}%
\begin{pgfscope}%
\pgfsys@transformshift{1.511943in}{2.575777in}%
\pgfsys@useobject{currentmarker}{}%
\end{pgfscope}%
\begin{pgfscope}%
\pgfsys@transformshift{1.512263in}{2.599146in}%
\pgfsys@useobject{currentmarker}{}%
\end{pgfscope}%
\begin{pgfscope}%
\pgfsys@transformshift{1.512286in}{2.612000in}%
\pgfsys@useobject{currentmarker}{}%
\end{pgfscope}%
\begin{pgfscope}%
\pgfsys@transformshift{1.510004in}{2.618692in}%
\pgfsys@useobject{currentmarker}{}%
\end{pgfscope}%
\begin{pgfscope}%
\pgfsys@transformshift{1.510650in}{2.628716in}%
\pgfsys@useobject{currentmarker}{}%
\end{pgfscope}%
\begin{pgfscope}%
\pgfsys@transformshift{1.514150in}{2.639287in}%
\pgfsys@useobject{currentmarker}{}%
\end{pgfscope}%
\begin{pgfscope}%
\pgfsys@transformshift{1.511762in}{2.653134in}%
\pgfsys@useobject{currentmarker}{}%
\end{pgfscope}%
\begin{pgfscope}%
\pgfsys@transformshift{1.511629in}{2.667660in}%
\pgfsys@useobject{currentmarker}{}%
\end{pgfscope}%
\begin{pgfscope}%
\pgfsys@transformshift{1.516003in}{2.685333in}%
\pgfsys@useobject{currentmarker}{}%
\end{pgfscope}%
\begin{pgfscope}%
\pgfsys@transformshift{1.512260in}{2.706632in}%
\pgfsys@useobject{currentmarker}{}%
\end{pgfscope}%
\begin{pgfscope}%
\pgfsys@transformshift{1.511411in}{2.718495in}%
\pgfsys@useobject{currentmarker}{}%
\end{pgfscope}%
\begin{pgfscope}%
\pgfsys@transformshift{1.515397in}{2.734678in}%
\pgfsys@useobject{currentmarker}{}%
\end{pgfscope}%
\begin{pgfscope}%
\pgfsys@transformshift{1.514783in}{2.743824in}%
\pgfsys@useobject{currentmarker}{}%
\end{pgfscope}%
\begin{pgfscope}%
\pgfsys@transformshift{1.513963in}{2.748799in}%
\pgfsys@useobject{currentmarker}{}%
\end{pgfscope}%
\begin{pgfscope}%
\pgfsys@transformshift{1.513519in}{2.756585in}%
\pgfsys@useobject{currentmarker}{}%
\end{pgfscope}%
\begin{pgfscope}%
\pgfsys@transformshift{1.514254in}{2.760811in}%
\pgfsys@useobject{currentmarker}{}%
\end{pgfscope}%
\begin{pgfscope}%
\pgfsys@transformshift{1.512827in}{2.768483in}%
\pgfsys@useobject{currentmarker}{}%
\end{pgfscope}%
\begin{pgfscope}%
\pgfsys@transformshift{1.513039in}{2.772770in}%
\pgfsys@useobject{currentmarker}{}%
\end{pgfscope}%
\begin{pgfscope}%
\pgfsys@transformshift{1.515750in}{2.781178in}%
\pgfsys@useobject{currentmarker}{}%
\end{pgfscope}%
\begin{pgfscope}%
\pgfsys@transformshift{1.514253in}{2.793107in}%
\pgfsys@useobject{currentmarker}{}%
\end{pgfscope}%
\begin{pgfscope}%
\pgfsys@transformshift{1.512460in}{2.799472in}%
\pgfsys@useobject{currentmarker}{}%
\end{pgfscope}%
\begin{pgfscope}%
\pgfsys@transformshift{1.515758in}{2.809946in}%
\pgfsys@useobject{currentmarker}{}%
\end{pgfscope}%
\begin{pgfscope}%
\pgfsys@transformshift{1.515793in}{2.821708in}%
\pgfsys@useobject{currentmarker}{}%
\end{pgfscope}%
\begin{pgfscope}%
\pgfsys@transformshift{1.515177in}{2.828148in}%
\pgfsys@useobject{currentmarker}{}%
\end{pgfscope}%
\begin{pgfscope}%
\pgfsys@transformshift{1.515283in}{2.837890in}%
\pgfsys@useobject{currentmarker}{}%
\end{pgfscope}%
\begin{pgfscope}%
\pgfsys@transformshift{1.516619in}{2.843079in}%
\pgfsys@useobject{currentmarker}{}%
\end{pgfscope}%
\begin{pgfscope}%
\pgfsys@transformshift{1.515380in}{2.851662in}%
\pgfsys@useobject{currentmarker}{}%
\end{pgfscope}%
\begin{pgfscope}%
\pgfsys@transformshift{1.514803in}{2.860904in}%
\pgfsys@useobject{currentmarker}{}%
\end{pgfscope}%
\begin{pgfscope}%
\pgfsys@transformshift{1.518441in}{2.871605in}%
\pgfsys@useobject{currentmarker}{}%
\end{pgfscope}%
\begin{pgfscope}%
\pgfsys@transformshift{1.516801in}{2.887827in}%
\pgfsys@useobject{currentmarker}{}%
\end{pgfscope}%
\begin{pgfscope}%
\pgfsys@transformshift{1.513697in}{2.904349in}%
\pgfsys@useobject{currentmarker}{}%
\end{pgfscope}%
\begin{pgfscope}%
\pgfsys@transformshift{1.520302in}{2.924171in}%
\pgfsys@useobject{currentmarker}{}%
\end{pgfscope}%
\begin{pgfscope}%
\pgfsys@transformshift{1.520502in}{2.946416in}%
\pgfsys@useobject{currentmarker}{}%
\end{pgfscope}%
\begin{pgfscope}%
\pgfsys@transformshift{1.518463in}{2.958480in}%
\pgfsys@useobject{currentmarker}{}%
\end{pgfscope}%
\begin{pgfscope}%
\pgfsys@transformshift{1.519214in}{2.973743in}%
\pgfsys@useobject{currentmarker}{}%
\end{pgfscope}%
\begin{pgfscope}%
\pgfsys@transformshift{1.520759in}{2.989614in}%
\pgfsys@useobject{currentmarker}{}%
\end{pgfscope}%
\begin{pgfscope}%
\pgfsys@transformshift{1.518949in}{3.007305in}%
\pgfsys@useobject{currentmarker}{}%
\end{pgfscope}%
\begin{pgfscope}%
\pgfsys@transformshift{1.519633in}{3.025660in}%
\pgfsys@useobject{currentmarker}{}%
\end{pgfscope}%
\begin{pgfscope}%
\pgfsys@transformshift{1.528149in}{3.044677in}%
\pgfsys@useobject{currentmarker}{}%
\end{pgfscope}%
\begin{pgfscope}%
\pgfsys@transformshift{1.527281in}{3.069079in}%
\pgfsys@useobject{currentmarker}{}%
\end{pgfscope}%
\begin{pgfscope}%
\pgfsys@transformshift{1.528214in}{3.082476in}%
\pgfsys@useobject{currentmarker}{}%
\end{pgfscope}%
\begin{pgfscope}%
\pgfsys@transformshift{1.535179in}{3.100275in}%
\pgfsys@useobject{currentmarker}{}%
\end{pgfscope}%
\begin{pgfscope}%
\pgfsys@transformshift{1.532379in}{3.120468in}%
\pgfsys@useobject{currentmarker}{}%
\end{pgfscope}%
\begin{pgfscope}%
\pgfsys@transformshift{1.530604in}{3.131539in}%
\pgfsys@useobject{currentmarker}{}%
\end{pgfscope}%
\begin{pgfscope}%
\pgfsys@transformshift{1.530278in}{3.146029in}%
\pgfsys@useobject{currentmarker}{}%
\end{pgfscope}%
\begin{pgfscope}%
\pgfsys@transformshift{1.531750in}{3.153864in}%
\pgfsys@useobject{currentmarker}{}%
\end{pgfscope}%
\begin{pgfscope}%
\pgfsys@transformshift{1.529620in}{3.165501in}%
\pgfsys@useobject{currentmarker}{}%
\end{pgfscope}%
\begin{pgfscope}%
\pgfsys@transformshift{1.529789in}{3.178122in}%
\pgfsys@useobject{currentmarker}{}%
\end{pgfscope}%
\begin{pgfscope}%
\pgfsys@transformshift{1.534788in}{3.194254in}%
\pgfsys@useobject{currentmarker}{}%
\end{pgfscope}%
\begin{pgfscope}%
\pgfsys@transformshift{1.533507in}{3.213227in}%
\pgfsys@useobject{currentmarker}{}%
\end{pgfscope}%
\begin{pgfscope}%
\pgfsys@transformshift{1.531794in}{3.232801in}%
\pgfsys@useobject{currentmarker}{}%
\end{pgfscope}%
\begin{pgfscope}%
\pgfsys@transformshift{1.535706in}{3.255486in}%
\pgfsys@useobject{currentmarker}{}%
\end{pgfscope}%
\begin{pgfscope}%
\pgfsys@transformshift{1.546105in}{3.277079in}%
\pgfsys@useobject{currentmarker}{}%
\end{pgfscope}%
\begin{pgfscope}%
\pgfsys@transformshift{1.555856in}{3.299974in}%
\pgfsys@useobject{currentmarker}{}%
\end{pgfscope}%
\begin{pgfscope}%
\pgfsys@transformshift{1.577063in}{3.314300in}%
\pgfsys@useobject{currentmarker}{}%
\end{pgfscope}%
\begin{pgfscope}%
\pgfsys@transformshift{1.599694in}{3.327981in}%
\pgfsys@useobject{currentmarker}{}%
\end{pgfscope}%
\begin{pgfscope}%
\pgfsys@transformshift{1.627220in}{3.332244in}%
\pgfsys@useobject{currentmarker}{}%
\end{pgfscope}%
\begin{pgfscope}%
\pgfsys@transformshift{1.655211in}{3.339614in}%
\pgfsys@useobject{currentmarker}{}%
\end{pgfscope}%
\begin{pgfscope}%
\pgfsys@transformshift{1.684817in}{3.338337in}%
\pgfsys@useobject{currentmarker}{}%
\end{pgfscope}%
\begin{pgfscope}%
\pgfsys@transformshift{1.700785in}{3.341603in}%
\pgfsys@useobject{currentmarker}{}%
\end{pgfscope}%
\begin{pgfscope}%
\pgfsys@transformshift{1.709749in}{3.341553in}%
\pgfsys@useobject{currentmarker}{}%
\end{pgfscope}%
\begin{pgfscope}%
\pgfsys@transformshift{1.719256in}{3.342401in}%
\pgfsys@useobject{currentmarker}{}%
\end{pgfscope}%
\begin{pgfscope}%
\pgfsys@transformshift{1.731188in}{3.340415in}%
\pgfsys@useobject{currentmarker}{}%
\end{pgfscope}%
\begin{pgfscope}%
\pgfsys@transformshift{1.744701in}{3.341290in}%
\pgfsys@useobject{currentmarker}{}%
\end{pgfscope}%
\begin{pgfscope}%
\pgfsys@transformshift{1.759186in}{3.339391in}%
\pgfsys@useobject{currentmarker}{}%
\end{pgfscope}%
\begin{pgfscope}%
\pgfsys@transformshift{1.767207in}{3.338925in}%
\pgfsys@useobject{currentmarker}{}%
\end{pgfscope}%
\begin{pgfscope}%
\pgfsys@transformshift{1.771619in}{3.338676in}%
\pgfsys@useobject{currentmarker}{}%
\end{pgfscope}%
\begin{pgfscope}%
\pgfsys@transformshift{1.778303in}{3.337700in}%
\pgfsys@useobject{currentmarker}{}%
\end{pgfscope}%
\begin{pgfscope}%
\pgfsys@transformshift{1.781995in}{3.337279in}%
\pgfsys@useobject{currentmarker}{}%
\end{pgfscope}%
\begin{pgfscope}%
\pgfsys@transformshift{1.784008in}{3.336928in}%
\pgfsys@useobject{currentmarker}{}%
\end{pgfscope}%
\begin{pgfscope}%
\pgfsys@transformshift{1.787056in}{3.336605in}%
\pgfsys@useobject{currentmarker}{}%
\end{pgfscope}%
\begin{pgfscope}%
\pgfsys@transformshift{1.790672in}{3.336711in}%
\pgfsys@useobject{currentmarker}{}%
\end{pgfscope}%
\begin{pgfscope}%
\pgfsys@transformshift{1.799518in}{3.336161in}%
\pgfsys@useobject{currentmarker}{}%
\end{pgfscope}%
\begin{pgfscope}%
\pgfsys@transformshift{1.810353in}{3.337143in}%
\pgfsys@useobject{currentmarker}{}%
\end{pgfscope}%
\begin{pgfscope}%
\pgfsys@transformshift{1.822031in}{3.335871in}%
\pgfsys@useobject{currentmarker}{}%
\end{pgfscope}%
\begin{pgfscope}%
\pgfsys@transformshift{1.828492in}{3.335808in}%
\pgfsys@useobject{currentmarker}{}%
\end{pgfscope}%
\begin{pgfscope}%
\pgfsys@transformshift{1.837087in}{3.338524in}%
\pgfsys@useobject{currentmarker}{}%
\end{pgfscope}%
\begin{pgfscope}%
\pgfsys@transformshift{1.842032in}{3.338869in}%
\pgfsys@useobject{currentmarker}{}%
\end{pgfscope}%
\begin{pgfscope}%
\pgfsys@transformshift{1.848102in}{3.338953in}%
\pgfsys@useobject{currentmarker}{}%
\end{pgfscope}%
\begin{pgfscope}%
\pgfsys@transformshift{1.855177in}{3.338529in}%
\pgfsys@useobject{currentmarker}{}%
\end{pgfscope}%
\begin{pgfscope}%
\pgfsys@transformshift{1.859033in}{3.339106in}%
\pgfsys@useobject{currentmarker}{}%
\end{pgfscope}%
\begin{pgfscope}%
\pgfsys@transformshift{1.861175in}{3.339187in}%
\pgfsys@useobject{currentmarker}{}%
\end{pgfscope}%
\begin{pgfscope}%
\pgfsys@transformshift{1.864715in}{3.338966in}%
\pgfsys@useobject{currentmarker}{}%
\end{pgfscope}%
\begin{pgfscope}%
\pgfsys@transformshift{1.866664in}{3.338866in}%
\pgfsys@useobject{currentmarker}{}%
\end{pgfscope}%
\begin{pgfscope}%
\pgfsys@transformshift{1.869722in}{3.338957in}%
\pgfsys@useobject{currentmarker}{}%
\end{pgfscope}%
\begin{pgfscope}%
\pgfsys@transformshift{1.871404in}{3.338997in}%
\pgfsys@useobject{currentmarker}{}%
\end{pgfscope}%
\begin{pgfscope}%
\pgfsys@transformshift{1.874760in}{3.338830in}%
\pgfsys@useobject{currentmarker}{}%
\end{pgfscope}%
\begin{pgfscope}%
\pgfsys@transformshift{1.876606in}{3.338736in}%
\pgfsys@useobject{currentmarker}{}%
\end{pgfscope}%
\begin{pgfscope}%
\pgfsys@transformshift{1.877602in}{3.338536in}%
\pgfsys@useobject{currentmarker}{}%
\end{pgfscope}%
\begin{pgfscope}%
\pgfsys@transformshift{1.880103in}{3.338436in}%
\pgfsys@useobject{currentmarker}{}%
\end{pgfscope}%
\begin{pgfscope}%
\pgfsys@transformshift{1.886606in}{3.337913in}%
\pgfsys@useobject{currentmarker}{}%
\end{pgfscope}%
\begin{pgfscope}%
\pgfsys@transformshift{1.893641in}{3.337873in}%
\pgfsys@useobject{currentmarker}{}%
\end{pgfscope}%
\begin{pgfscope}%
\pgfsys@transformshift{1.897431in}{3.337094in}%
\pgfsys@useobject{currentmarker}{}%
\end{pgfscope}%
\begin{pgfscope}%
\pgfsys@transformshift{1.899542in}{3.336826in}%
\pgfsys@useobject{currentmarker}{}%
\end{pgfscope}%
\begin{pgfscope}%
\pgfsys@transformshift{1.902127in}{3.336287in}%
\pgfsys@useobject{currentmarker}{}%
\end{pgfscope}%
\begin{pgfscope}%
\pgfsys@transformshift{1.907273in}{3.336220in}%
\pgfsys@useobject{currentmarker}{}%
\end{pgfscope}%
\begin{pgfscope}%
\pgfsys@transformshift{1.915526in}{3.336086in}%
\pgfsys@useobject{currentmarker}{}%
\end{pgfscope}%
\begin{pgfscope}%
\pgfsys@transformshift{1.925849in}{3.338031in}%
\pgfsys@useobject{currentmarker}{}%
\end{pgfscope}%
\begin{pgfscope}%
\pgfsys@transformshift{1.937387in}{3.337404in}%
\pgfsys@useobject{currentmarker}{}%
\end{pgfscope}%
\begin{pgfscope}%
\pgfsys@transformshift{1.949796in}{3.339791in}%
\pgfsys@useobject{currentmarker}{}%
\end{pgfscope}%
\begin{pgfscope}%
\pgfsys@transformshift{1.964680in}{3.342682in}%
\pgfsys@useobject{currentmarker}{}%
\end{pgfscope}%
\begin{pgfscope}%
\pgfsys@transformshift{1.981264in}{3.344712in}%
\pgfsys@useobject{currentmarker}{}%
\end{pgfscope}%
\begin{pgfscope}%
\pgfsys@transformshift{1.990442in}{3.344263in}%
\pgfsys@useobject{currentmarker}{}%
\end{pgfscope}%
\begin{pgfscope}%
\pgfsys@transformshift{1.995481in}{3.343872in}%
\pgfsys@useobject{currentmarker}{}%
\end{pgfscope}%
\begin{pgfscope}%
\pgfsys@transformshift{2.001385in}{3.343254in}%
\pgfsys@useobject{currentmarker}{}%
\end{pgfscope}%
\begin{pgfscope}%
\pgfsys@transformshift{2.004563in}{3.344001in}%
\pgfsys@useobject{currentmarker}{}%
\end{pgfscope}%
\begin{pgfscope}%
\pgfsys@transformshift{2.006349in}{3.343809in}%
\pgfsys@useobject{currentmarker}{}%
\end{pgfscope}%
\begin{pgfscope}%
\pgfsys@transformshift{2.008648in}{3.343803in}%
\pgfsys@useobject{currentmarker}{}%
\end{pgfscope}%
\begin{pgfscope}%
\pgfsys@transformshift{2.009896in}{3.343605in}%
\pgfsys@useobject{currentmarker}{}%
\end{pgfscope}%
\begin{pgfscope}%
\pgfsys@transformshift{2.011795in}{3.343683in}%
\pgfsys@useobject{currentmarker}{}%
\end{pgfscope}%
\begin{pgfscope}%
\pgfsys@transformshift{2.012838in}{3.343754in}%
\pgfsys@useobject{currentmarker}{}%
\end{pgfscope}%
\begin{pgfscope}%
\pgfsys@transformshift{2.014715in}{3.343530in}%
\pgfsys@useobject{currentmarker}{}%
\end{pgfscope}%
\begin{pgfscope}%
\pgfsys@transformshift{2.017071in}{3.343505in}%
\pgfsys@useobject{currentmarker}{}%
\end{pgfscope}%
\begin{pgfscope}%
\pgfsys@transformshift{2.022553in}{3.343258in}%
\pgfsys@useobject{currentmarker}{}%
\end{pgfscope}%
\begin{pgfscope}%
\pgfsys@transformshift{2.025555in}{3.343562in}%
\pgfsys@useobject{currentmarker}{}%
\end{pgfscope}%
\begin{pgfscope}%
\pgfsys@transformshift{2.027214in}{3.343508in}%
\pgfsys@useobject{currentmarker}{}%
\end{pgfscope}%
\begin{pgfscope}%
\pgfsys@transformshift{2.030368in}{3.343427in}%
\pgfsys@useobject{currentmarker}{}%
\end{pgfscope}%
\begin{pgfscope}%
\pgfsys@transformshift{2.034504in}{3.343374in}%
\pgfsys@useobject{currentmarker}{}%
\end{pgfscope}%
\begin{pgfscope}%
\pgfsys@transformshift{2.036772in}{3.343190in}%
\pgfsys@useobject{currentmarker}{}%
\end{pgfscope}%
\begin{pgfscope}%
\pgfsys@transformshift{2.037994in}{3.342919in}%
\pgfsys@useobject{currentmarker}{}%
\end{pgfscope}%
\begin{pgfscope}%
\pgfsys@transformshift{2.040252in}{3.342925in}%
\pgfsys@useobject{currentmarker}{}%
\end{pgfscope}%
\begin{pgfscope}%
\pgfsys@transformshift{2.043421in}{3.342820in}%
\pgfsys@useobject{currentmarker}{}%
\end{pgfscope}%
\begin{pgfscope}%
\pgfsys@transformshift{2.045164in}{3.342893in}%
\pgfsys@useobject{currentmarker}{}%
\end{pgfscope}%
\begin{pgfscope}%
\pgfsys@transformshift{2.046123in}{3.342880in}%
\pgfsys@useobject{currentmarker}{}%
\end{pgfscope}%
\begin{pgfscope}%
\pgfsys@transformshift{2.048487in}{3.342765in}%
\pgfsys@useobject{currentmarker}{}%
\end{pgfscope}%
\begin{pgfscope}%
\pgfsys@transformshift{2.051464in}{3.342562in}%
\pgfsys@useobject{currentmarker}{}%
\end{pgfscope}%
\begin{pgfscope}%
\pgfsys@transformshift{2.055960in}{3.341693in}%
\pgfsys@useobject{currentmarker}{}%
\end{pgfscope}%
\begin{pgfscope}%
\pgfsys@transformshift{2.062814in}{3.340918in}%
\pgfsys@useobject{currentmarker}{}%
\end{pgfscope}%
\begin{pgfscope}%
\pgfsys@transformshift{2.066585in}{3.340502in}%
\pgfsys@useobject{currentmarker}{}%
\end{pgfscope}%
\begin{pgfscope}%
\pgfsys@transformshift{2.071503in}{3.340189in}%
\pgfsys@useobject{currentmarker}{}%
\end{pgfscope}%
\begin{pgfscope}%
\pgfsys@transformshift{2.074194in}{3.339865in}%
\pgfsys@useobject{currentmarker}{}%
\end{pgfscope}%
\begin{pgfscope}%
\pgfsys@transformshift{2.078821in}{3.340275in}%
\pgfsys@useobject{currentmarker}{}%
\end{pgfscope}%
\begin{pgfscope}%
\pgfsys@transformshift{2.084171in}{3.338769in}%
\pgfsys@useobject{currentmarker}{}%
\end{pgfscope}%
\begin{pgfscope}%
\pgfsys@transformshift{2.090506in}{3.338813in}%
\pgfsys@useobject{currentmarker}{}%
\end{pgfscope}%
\begin{pgfscope}%
\pgfsys@transformshift{2.098365in}{3.339030in}%
\pgfsys@useobject{currentmarker}{}%
\end{pgfscope}%
\begin{pgfscope}%
\pgfsys@transformshift{2.107855in}{3.339872in}%
\pgfsys@useobject{currentmarker}{}%
\end{pgfscope}%
\begin{pgfscope}%
\pgfsys@transformshift{2.113074in}{3.339406in}%
\pgfsys@useobject{currentmarker}{}%
\end{pgfscope}%
\begin{pgfscope}%
\pgfsys@transformshift{2.118771in}{3.338815in}%
\pgfsys@useobject{currentmarker}{}%
\end{pgfscope}%
\begin{pgfscope}%
\pgfsys@transformshift{2.121916in}{3.338634in}%
\pgfsys@useobject{currentmarker}{}%
\end{pgfscope}%
\begin{pgfscope}%
\pgfsys@transformshift{2.126232in}{3.338679in}%
\pgfsys@useobject{currentmarker}{}%
\end{pgfscope}%
\begin{pgfscope}%
\pgfsys@transformshift{2.128548in}{3.338159in}%
\pgfsys@useobject{currentmarker}{}%
\end{pgfscope}%
\begin{pgfscope}%
\pgfsys@transformshift{2.129845in}{3.338011in}%
\pgfsys@useobject{currentmarker}{}%
\end{pgfscope}%
\begin{pgfscope}%
\pgfsys@transformshift{2.133080in}{3.338133in}%
\pgfsys@useobject{currentmarker}{}%
\end{pgfscope}%
\begin{pgfscope}%
\pgfsys@transformshift{2.137318in}{3.337327in}%
\pgfsys@useobject{currentmarker}{}%
\end{pgfscope}%
\begin{pgfscope}%
\pgfsys@transformshift{2.139654in}{3.336909in}%
\pgfsys@useobject{currentmarker}{}%
\end{pgfscope}%
\begin{pgfscope}%
\pgfsys@transformshift{2.142523in}{3.336444in}%
\pgfsys@useobject{currentmarker}{}%
\end{pgfscope}%
\begin{pgfscope}%
\pgfsys@transformshift{2.145977in}{3.336352in}%
\pgfsys@useobject{currentmarker}{}%
\end{pgfscope}%
\begin{pgfscope}%
\pgfsys@transformshift{2.147861in}{3.336099in}%
\pgfsys@useobject{currentmarker}{}%
\end{pgfscope}%
\begin{pgfscope}%
\pgfsys@transformshift{2.148894in}{3.335944in}%
\pgfsys@useobject{currentmarker}{}%
\end{pgfscope}%
\begin{pgfscope}%
\pgfsys@transformshift{2.150897in}{3.336130in}%
\pgfsys@useobject{currentmarker}{}%
\end{pgfscope}%
\begin{pgfscope}%
\pgfsys@transformshift{2.154600in}{3.336286in}%
\pgfsys@useobject{currentmarker}{}%
\end{pgfscope}%
\begin{pgfscope}%
\pgfsys@transformshift{2.156637in}{3.336329in}%
\pgfsys@useobject{currentmarker}{}%
\end{pgfscope}%
\begin{pgfscope}%
\pgfsys@transformshift{2.157735in}{3.336105in}%
\pgfsys@useobject{currentmarker}{}%
\end{pgfscope}%
\begin{pgfscope}%
\pgfsys@transformshift{2.158350in}{3.336051in}%
\pgfsys@useobject{currentmarker}{}%
\end{pgfscope}%
\begin{pgfscope}%
\pgfsys@transformshift{2.160507in}{3.335961in}%
\pgfsys@useobject{currentmarker}{}%
\end{pgfscope}%
\begin{pgfscope}%
\pgfsys@transformshift{2.161689in}{3.336075in}%
\pgfsys@useobject{currentmarker}{}%
\end{pgfscope}%
\begin{pgfscope}%
\pgfsys@transformshift{2.163877in}{3.335806in}%
\pgfsys@useobject{currentmarker}{}%
\end{pgfscope}%
\begin{pgfscope}%
\pgfsys@transformshift{2.165087in}{3.335888in}%
\pgfsys@useobject{currentmarker}{}%
\end{pgfscope}%
\begin{pgfscope}%
\pgfsys@transformshift{2.168230in}{3.336328in}%
\pgfsys@useobject{currentmarker}{}%
\end{pgfscope}%
\begin{pgfscope}%
\pgfsys@transformshift{2.169957in}{3.336079in}%
\pgfsys@useobject{currentmarker}{}%
\end{pgfscope}%
\begin{pgfscope}%
\pgfsys@transformshift{2.170913in}{3.335988in}%
\pgfsys@useobject{currentmarker}{}%
\end{pgfscope}%
\begin{pgfscope}%
\pgfsys@transformshift{2.172655in}{3.335968in}%
\pgfsys@useobject{currentmarker}{}%
\end{pgfscope}%
\begin{pgfscope}%
\pgfsys@transformshift{2.173614in}{3.335974in}%
\pgfsys@useobject{currentmarker}{}%
\end{pgfscope}%
\begin{pgfscope}%
\pgfsys@transformshift{2.175836in}{3.335458in}%
\pgfsys@useobject{currentmarker}{}%
\end{pgfscope}%
\begin{pgfscope}%
\pgfsys@transformshift{2.177079in}{3.335295in}%
\pgfsys@useobject{currentmarker}{}%
\end{pgfscope}%
\begin{pgfscope}%
\pgfsys@transformshift{2.177749in}{3.335128in}%
\pgfsys@useobject{currentmarker}{}%
\end{pgfscope}%
\begin{pgfscope}%
\pgfsys@transformshift{2.178122in}{3.335061in}%
\pgfsys@useobject{currentmarker}{}%
\end{pgfscope}%
\begin{pgfscope}%
\pgfsys@transformshift{2.178325in}{3.335012in}%
\pgfsys@useobject{currentmarker}{}%
\end{pgfscope}%
\begin{pgfscope}%
\pgfsys@transformshift{2.179195in}{3.334965in}%
\pgfsys@useobject{currentmarker}{}%
\end{pgfscope}%
\begin{pgfscope}%
\pgfsys@transformshift{2.179672in}{3.334918in}%
\pgfsys@useobject{currentmarker}{}%
\end{pgfscope}%
\begin{pgfscope}%
\pgfsys@transformshift{2.179934in}{3.334888in}%
\pgfsys@useobject{currentmarker}{}%
\end{pgfscope}%
\begin{pgfscope}%
\pgfsys@transformshift{2.180704in}{3.334844in}%
\pgfsys@useobject{currentmarker}{}%
\end{pgfscope}%
\begin{pgfscope}%
\pgfsys@transformshift{2.182424in}{3.334793in}%
\pgfsys@useobject{currentmarker}{}%
\end{pgfscope}%
\begin{pgfscope}%
\pgfsys@transformshift{2.184706in}{3.334351in}%
\pgfsys@useobject{currentmarker}{}%
\end{pgfscope}%
\begin{pgfscope}%
\pgfsys@transformshift{2.187732in}{3.333943in}%
\pgfsys@useobject{currentmarker}{}%
\end{pgfscope}%
\begin{pgfscope}%
\pgfsys@transformshift{2.191238in}{3.333635in}%
\pgfsys@useobject{currentmarker}{}%
\end{pgfscope}%
\begin{pgfscope}%
\pgfsys@transformshift{2.193172in}{3.333578in}%
\pgfsys@useobject{currentmarker}{}%
\end{pgfscope}%
\begin{pgfscope}%
\pgfsys@transformshift{2.196281in}{3.333130in}%
\pgfsys@useobject{currentmarker}{}%
\end{pgfscope}%
\begin{pgfscope}%
\pgfsys@transformshift{2.198001in}{3.332966in}%
\pgfsys@useobject{currentmarker}{}%
\end{pgfscope}%
\begin{pgfscope}%
\pgfsys@transformshift{2.198951in}{3.332966in}%
\pgfsys@useobject{currentmarker}{}%
\end{pgfscope}%
\begin{pgfscope}%
\pgfsys@transformshift{2.200856in}{3.333131in}%
\pgfsys@useobject{currentmarker}{}%
\end{pgfscope}%
\begin{pgfscope}%
\pgfsys@transformshift{2.201895in}{3.332963in}%
\pgfsys@useobject{currentmarker}{}%
\end{pgfscope}%
\begin{pgfscope}%
\pgfsys@transformshift{2.202473in}{3.332956in}%
\pgfsys@useobject{currentmarker}{}%
\end{pgfscope}%
\begin{pgfscope}%
\pgfsys@transformshift{2.202789in}{3.332921in}%
\pgfsys@useobject{currentmarker}{}%
\end{pgfscope}%
\begin{pgfscope}%
\pgfsys@transformshift{2.202963in}{3.332941in}%
\pgfsys@useobject{currentmarker}{}%
\end{pgfscope}%
\begin{pgfscope}%
\pgfsys@transformshift{2.203059in}{3.332932in}%
\pgfsys@useobject{currentmarker}{}%
\end{pgfscope}%
\begin{pgfscope}%
\pgfsys@transformshift{2.204759in}{3.332878in}%
\pgfsys@useobject{currentmarker}{}%
\end{pgfscope}%
\begin{pgfscope}%
\pgfsys@transformshift{2.208303in}{3.332738in}%
\pgfsys@useobject{currentmarker}{}%
\end{pgfscope}%
\begin{pgfscope}%
\pgfsys@transformshift{2.210252in}{3.332814in}%
\pgfsys@useobject{currentmarker}{}%
\end{pgfscope}%
\begin{pgfscope}%
\pgfsys@transformshift{2.211322in}{3.332731in}%
\pgfsys@useobject{currentmarker}{}%
\end{pgfscope}%
\begin{pgfscope}%
\pgfsys@transformshift{2.212977in}{3.332761in}%
\pgfsys@useobject{currentmarker}{}%
\end{pgfscope}%
\begin{pgfscope}%
\pgfsys@transformshift{2.215845in}{3.332793in}%
\pgfsys@useobject{currentmarker}{}%
\end{pgfscope}%
\begin{pgfscope}%
\pgfsys@transformshift{2.217422in}{3.332776in}%
\pgfsys@useobject{currentmarker}{}%
\end{pgfscope}%
\begin{pgfscope}%
\pgfsys@transformshift{2.218288in}{3.332731in}%
\pgfsys@useobject{currentmarker}{}%
\end{pgfscope}%
\begin{pgfscope}%
\pgfsys@transformshift{2.218764in}{3.332700in}%
\pgfsys@useobject{currentmarker}{}%
\end{pgfscope}%
\begin{pgfscope}%
\pgfsys@transformshift{2.221148in}{3.332623in}%
\pgfsys@useobject{currentmarker}{}%
\end{pgfscope}%
\begin{pgfscope}%
\pgfsys@transformshift{2.222458in}{3.332577in}%
\pgfsys@useobject{currentmarker}{}%
\end{pgfscope}%
\begin{pgfscope}%
\pgfsys@transformshift{2.223175in}{3.332493in}%
\pgfsys@useobject{currentmarker}{}%
\end{pgfscope}%
\begin{pgfscope}%
\pgfsys@transformshift{2.223570in}{3.332460in}%
\pgfsys@useobject{currentmarker}{}%
\end{pgfscope}%
\begin{pgfscope}%
\pgfsys@transformshift{2.223786in}{3.332431in}%
\pgfsys@useobject{currentmarker}{}%
\end{pgfscope}%
\begin{pgfscope}%
\pgfsys@transformshift{2.223906in}{3.332421in}%
\pgfsys@useobject{currentmarker}{}%
\end{pgfscope}%
\begin{pgfscope}%
\pgfsys@transformshift{2.223970in}{3.332405in}%
\pgfsys@useobject{currentmarker}{}%
\end{pgfscope}%
\begin{pgfscope}%
\pgfsys@transformshift{2.224645in}{3.332495in}%
\pgfsys@useobject{currentmarker}{}%
\end{pgfscope}%
\begin{pgfscope}%
\pgfsys@transformshift{2.225019in}{3.332470in}%
\pgfsys@useobject{currentmarker}{}%
\end{pgfscope}%
\begin{pgfscope}%
\pgfsys@transformshift{2.227227in}{3.331456in}%
\pgfsys@useobject{currentmarker}{}%
\end{pgfscope}%
\begin{pgfscope}%
\pgfsys@transformshift{2.228561in}{3.331374in}%
\pgfsys@useobject{currentmarker}{}%
\end{pgfscope}%
\begin{pgfscope}%
\pgfsys@transformshift{2.231898in}{3.331110in}%
\pgfsys@useobject{currentmarker}{}%
\end{pgfscope}%
\begin{pgfscope}%
\pgfsys@transformshift{2.237987in}{3.330987in}%
\pgfsys@useobject{currentmarker}{}%
\end{pgfscope}%
\begin{pgfscope}%
\pgfsys@transformshift{2.247621in}{3.330058in}%
\pgfsys@useobject{currentmarker}{}%
\end{pgfscope}%
\begin{pgfscope}%
\pgfsys@transformshift{2.258596in}{3.333559in}%
\pgfsys@useobject{currentmarker}{}%
\end{pgfscope}%
\begin{pgfscope}%
\pgfsys@transformshift{2.264920in}{3.333956in}%
\pgfsys@useobject{currentmarker}{}%
\end{pgfscope}%
\begin{pgfscope}%
\pgfsys@transformshift{2.277501in}{3.334454in}%
\pgfsys@useobject{currentmarker}{}%
\end{pgfscope}%
\begin{pgfscope}%
\pgfsys@transformshift{2.292868in}{3.331809in}%
\pgfsys@useobject{currentmarker}{}%
\end{pgfscope}%
\begin{pgfscope}%
\pgfsys@transformshift{2.309001in}{3.330195in}%
\pgfsys@useobject{currentmarker}{}%
\end{pgfscope}%
\begin{pgfscope}%
\pgfsys@transformshift{2.330145in}{3.324235in}%
\pgfsys@useobject{currentmarker}{}%
\end{pgfscope}%
\begin{pgfscope}%
\pgfsys@transformshift{2.353436in}{3.325724in}%
\pgfsys@useobject{currentmarker}{}%
\end{pgfscope}%
\begin{pgfscope}%
\pgfsys@transformshift{2.366235in}{3.326689in}%
\pgfsys@useobject{currentmarker}{}%
\end{pgfscope}%
\begin{pgfscope}%
\pgfsys@transformshift{2.383471in}{3.324434in}%
\pgfsys@useobject{currentmarker}{}%
\end{pgfscope}%
\begin{pgfscope}%
\pgfsys@transformshift{2.402965in}{3.322410in}%
\pgfsys@useobject{currentmarker}{}%
\end{pgfscope}%
\begin{pgfscope}%
\pgfsys@transformshift{2.426868in}{3.323444in}%
\pgfsys@useobject{currentmarker}{}%
\end{pgfscope}%
\begin{pgfscope}%
\pgfsys@transformshift{2.450070in}{3.315684in}%
\pgfsys@useobject{currentmarker}{}%
\end{pgfscope}%
\begin{pgfscope}%
\pgfsys@transformshift{2.473984in}{3.324922in}%
\pgfsys@useobject{currentmarker}{}%
\end{pgfscope}%
\begin{pgfscope}%
\pgfsys@transformshift{2.487864in}{3.327406in}%
\pgfsys@useobject{currentmarker}{}%
\end{pgfscope}%
\begin{pgfscope}%
\pgfsys@transformshift{2.505968in}{3.326655in}%
\pgfsys@useobject{currentmarker}{}%
\end{pgfscope}%
\begin{pgfscope}%
\pgfsys@transformshift{2.529826in}{3.323842in}%
\pgfsys@useobject{currentmarker}{}%
\end{pgfscope}%
\begin{pgfscope}%
\pgfsys@transformshift{2.553727in}{3.318433in}%
\pgfsys@useobject{currentmarker}{}%
\end{pgfscope}%
\begin{pgfscope}%
\pgfsys@transformshift{2.579054in}{3.323573in}%
\pgfsys@useobject{currentmarker}{}%
\end{pgfscope}%
\begin{pgfscope}%
\pgfsys@transformshift{2.606882in}{3.327607in}%
\pgfsys@useobject{currentmarker}{}%
\end{pgfscope}%
\begin{pgfscope}%
\pgfsys@transformshift{2.636772in}{3.338027in}%
\pgfsys@useobject{currentmarker}{}%
\end{pgfscope}%
\begin{pgfscope}%
\pgfsys@transformshift{2.653910in}{3.341091in}%
\pgfsys@useobject{currentmarker}{}%
\end{pgfscope}%
\begin{pgfscope}%
\pgfsys@transformshift{2.663405in}{3.342337in}%
\pgfsys@useobject{currentmarker}{}%
\end{pgfscope}%
\begin{pgfscope}%
\pgfsys@transformshift{2.668572in}{3.343352in}%
\pgfsys@useobject{currentmarker}{}%
\end{pgfscope}%
\begin{pgfscope}%
\pgfsys@transformshift{2.676243in}{3.341955in}%
\pgfsys@useobject{currentmarker}{}%
\end{pgfscope}%
\begin{pgfscope}%
\pgfsys@transformshift{2.686389in}{3.343898in}%
\pgfsys@useobject{currentmarker}{}%
\end{pgfscope}%
\begin{pgfscope}%
\pgfsys@transformshift{2.698928in}{3.341725in}%
\pgfsys@useobject{currentmarker}{}%
\end{pgfscope}%
\begin{pgfscope}%
\pgfsys@transformshift{2.705399in}{3.339057in}%
\pgfsys@useobject{currentmarker}{}%
\end{pgfscope}%
\begin{pgfscope}%
\pgfsys@transformshift{2.714014in}{3.339327in}%
\pgfsys@useobject{currentmarker}{}%
\end{pgfscope}%
\begin{pgfscope}%
\pgfsys@transformshift{2.718558in}{3.337977in}%
\pgfsys@useobject{currentmarker}{}%
\end{pgfscope}%
\begin{pgfscope}%
\pgfsys@transformshift{2.721110in}{3.337442in}%
\pgfsys@useobject{currentmarker}{}%
\end{pgfscope}%
\begin{pgfscope}%
\pgfsys@transformshift{2.730174in}{3.335832in}%
\pgfsys@useobject{currentmarker}{}%
\end{pgfscope}%
\begin{pgfscope}%
\pgfsys@transformshift{2.739973in}{3.338638in}%
\pgfsys@useobject{currentmarker}{}%
\end{pgfscope}%
\begin{pgfscope}%
\pgfsys@transformshift{2.750555in}{3.339879in}%
\pgfsys@useobject{currentmarker}{}%
\end{pgfscope}%
\begin{pgfscope}%
\pgfsys@transformshift{2.762902in}{3.337038in}%
\pgfsys@useobject{currentmarker}{}%
\end{pgfscope}%
\begin{pgfscope}%
\pgfsys@transformshift{2.777608in}{3.332841in}%
\pgfsys@useobject{currentmarker}{}%
\end{pgfscope}%
\begin{pgfscope}%
\pgfsys@transformshift{2.795472in}{3.328743in}%
\pgfsys@useobject{currentmarker}{}%
\end{pgfscope}%
\begin{pgfscope}%
\pgfsys@transformshift{2.805395in}{3.330521in}%
\pgfsys@useobject{currentmarker}{}%
\end{pgfscope}%
\begin{pgfscope}%
\pgfsys@transformshift{2.810894in}{3.331228in}%
\pgfsys@useobject{currentmarker}{}%
\end{pgfscope}%
\begin{pgfscope}%
\pgfsys@transformshift{2.818818in}{3.336414in}%
\pgfsys@useobject{currentmarker}{}%
\end{pgfscope}%
\begin{pgfscope}%
\pgfsys@transformshift{2.823559in}{3.334255in}%
\pgfsys@useobject{currentmarker}{}%
\end{pgfscope}%
\begin{pgfscope}%
\pgfsys@transformshift{2.829384in}{3.333904in}%
\pgfsys@useobject{currentmarker}{}%
\end{pgfscope}%
\begin{pgfscope}%
\pgfsys@transformshift{2.835990in}{3.334319in}%
\pgfsys@useobject{currentmarker}{}%
\end{pgfscope}%
\begin{pgfscope}%
\pgfsys@transformshift{2.844216in}{3.333973in}%
\pgfsys@useobject{currentmarker}{}%
\end{pgfscope}%
\begin{pgfscope}%
\pgfsys@transformshift{2.853839in}{3.336038in}%
\pgfsys@useobject{currentmarker}{}%
\end{pgfscope}%
\begin{pgfscope}%
\pgfsys@transformshift{2.867556in}{3.337588in}%
\pgfsys@useobject{currentmarker}{}%
\end{pgfscope}%
\begin{pgfscope}%
\pgfsys@transformshift{2.874977in}{3.335982in}%
\pgfsys@useobject{currentmarker}{}%
\end{pgfscope}%
\begin{pgfscope}%
\pgfsys@transformshift{2.879110in}{3.336578in}%
\pgfsys@useobject{currentmarker}{}%
\end{pgfscope}%
\begin{pgfscope}%
\pgfsys@transformshift{2.884422in}{3.335417in}%
\pgfsys@useobject{currentmarker}{}%
\end{pgfscope}%
\begin{pgfscope}%
\pgfsys@transformshift{2.892882in}{3.339427in}%
\pgfsys@useobject{currentmarker}{}%
\end{pgfscope}%
\begin{pgfscope}%
\pgfsys@transformshift{2.908090in}{3.344203in}%
\pgfsys@useobject{currentmarker}{}%
\end{pgfscope}%
\begin{pgfscope}%
\pgfsys@transformshift{2.925997in}{3.349267in}%
\pgfsys@useobject{currentmarker}{}%
\end{pgfscope}%
\begin{pgfscope}%
\pgfsys@transformshift{2.942968in}{3.364644in}%
\pgfsys@useobject{currentmarker}{}%
\end{pgfscope}%
\begin{pgfscope}%
\pgfsys@transformshift{2.968955in}{3.365270in}%
\pgfsys@useobject{currentmarker}{}%
\end{pgfscope}%
\begin{pgfscope}%
\pgfsys@transformshift{2.997865in}{3.362661in}%
\pgfsys@useobject{currentmarker}{}%
\end{pgfscope}%
\begin{pgfscope}%
\pgfsys@transformshift{3.035151in}{3.373634in}%
\pgfsys@useobject{currentmarker}{}%
\end{pgfscope}%
\begin{pgfscope}%
\pgfsys@transformshift{3.046629in}{3.355599in}%
\pgfsys@useobject{currentmarker}{}%
\end{pgfscope}%
\begin{pgfscope}%
\pgfsys@transformshift{3.068689in}{3.342325in}%
\pgfsys@useobject{currentmarker}{}%
\end{pgfscope}%
\begin{pgfscope}%
\pgfsys@transformshift{3.095247in}{3.352567in}%
\pgfsys@useobject{currentmarker}{}%
\end{pgfscope}%
\begin{pgfscope}%
\pgfsys@transformshift{3.128334in}{3.352791in}%
\pgfsys@useobject{currentmarker}{}%
\end{pgfscope}%
\begin{pgfscope}%
\pgfsys@transformshift{3.169740in}{3.355339in}%
\pgfsys@useobject{currentmarker}{}%
\end{pgfscope}%
\begin{pgfscope}%
\pgfsys@transformshift{3.212975in}{3.356329in}%
\pgfsys@useobject{currentmarker}{}%
\end{pgfscope}%
\begin{pgfscope}%
\pgfsys@transformshift{3.256819in}{3.351287in}%
\pgfsys@useobject{currentmarker}{}%
\end{pgfscope}%
\begin{pgfscope}%
\pgfsys@transformshift{3.304043in}{3.340448in}%
\pgfsys@useobject{currentmarker}{}%
\end{pgfscope}%
\begin{pgfscope}%
\pgfsys@transformshift{3.351505in}{3.325535in}%
\pgfsys@useobject{currentmarker}{}%
\end{pgfscope}%
\begin{pgfscope}%
\pgfsys@transformshift{3.399884in}{3.300482in}%
\pgfsys@useobject{currentmarker}{}%
\end{pgfscope}%
\begin{pgfscope}%
\pgfsys@transformshift{3.454386in}{3.287534in}%
\pgfsys@useobject{currentmarker}{}%
\end{pgfscope}%
\begin{pgfscope}%
\pgfsys@transformshift{3.506702in}{3.312483in}%
\pgfsys@useobject{currentmarker}{}%
\end{pgfscope}%
\begin{pgfscope}%
\pgfsys@transformshift{3.562363in}{3.332160in}%
\pgfsys@useobject{currentmarker}{}%
\end{pgfscope}%
\begin{pgfscope}%
\pgfsys@transformshift{3.618516in}{3.354763in}%
\pgfsys@useobject{currentmarker}{}%
\end{pgfscope}%
\begin{pgfscope}%
\pgfsys@transformshift{3.682195in}{3.351862in}%
\pgfsys@useobject{currentmarker}{}%
\end{pgfscope}%
\begin{pgfscope}%
\pgfsys@transformshift{3.746285in}{3.344104in}%
\pgfsys@useobject{currentmarker}{}%
\end{pgfscope}%
\begin{pgfscope}%
\pgfsys@transformshift{3.781695in}{3.341489in}%
\pgfsys@useobject{currentmarker}{}%
\end{pgfscope}%
\begin{pgfscope}%
\pgfsys@transformshift{3.818177in}{3.339847in}%
\pgfsys@useobject{currentmarker}{}%
\end{pgfscope}%
\begin{pgfscope}%
\pgfsys@transformshift{3.858122in}{3.341291in}%
\pgfsys@useobject{currentmarker}{}%
\end{pgfscope}%
\begin{pgfscope}%
\pgfsys@transformshift{3.879973in}{3.338882in}%
\pgfsys@useobject{currentmarker}{}%
\end{pgfscope}%
\begin{pgfscope}%
\pgfsys@transformshift{3.902030in}{3.334293in}%
\pgfsys@useobject{currentmarker}{}%
\end{pgfscope}%
\begin{pgfscope}%
\pgfsys@transformshift{3.918502in}{3.318250in}%
\pgfsys@useobject{currentmarker}{}%
\end{pgfscope}%
\begin{pgfscope}%
\pgfsys@transformshift{3.939147in}{3.307037in}%
\pgfsys@useobject{currentmarker}{}%
\end{pgfscope}%
\begin{pgfscope}%
\pgfsys@transformshift{3.956940in}{3.289206in}%
\pgfsys@useobject{currentmarker}{}%
\end{pgfscope}%
\begin{pgfscope}%
\pgfsys@transformshift{3.983218in}{3.283330in}%
\pgfsys@useobject{currentmarker}{}%
\end{pgfscope}%
\begin{pgfscope}%
\pgfsys@transformshift{4.007133in}{3.269239in}%
\pgfsys@useobject{currentmarker}{}%
\end{pgfscope}%
\begin{pgfscope}%
\pgfsys@transformshift{4.031267in}{3.253839in}%
\pgfsys@useobject{currentmarker}{}%
\end{pgfscope}%
\begin{pgfscope}%
\pgfsys@transformshift{4.041316in}{3.241717in}%
\pgfsys@useobject{currentmarker}{}%
\end{pgfscope}%
\begin{pgfscope}%
\pgfsys@transformshift{4.041118in}{3.224738in}%
\pgfsys@useobject{currentmarker}{}%
\end{pgfscope}%
\begin{pgfscope}%
\pgfsys@transformshift{4.045603in}{3.206328in}%
\pgfsys@useobject{currentmarker}{}%
\end{pgfscope}%
\begin{pgfscope}%
\pgfsys@transformshift{4.046986in}{3.186570in}%
\pgfsys@useobject{currentmarker}{}%
\end{pgfscope}%
\begin{pgfscope}%
\pgfsys@transformshift{4.053774in}{3.166406in}%
\pgfsys@useobject{currentmarker}{}%
\end{pgfscope}%
\begin{pgfscope}%
\pgfsys@transformshift{4.056947in}{3.144483in}%
\pgfsys@useobject{currentmarker}{}%
\end{pgfscope}%
\begin{pgfscope}%
\pgfsys@transformshift{4.060480in}{3.122106in}%
\pgfsys@useobject{currentmarker}{}%
\end{pgfscope}%
\begin{pgfscope}%
\pgfsys@transformshift{4.057293in}{3.097804in}%
\pgfsys@useobject{currentmarker}{}%
\end{pgfscope}%
\begin{pgfscope}%
\pgfsys@transformshift{4.056950in}{3.071934in}%
\pgfsys@useobject{currentmarker}{}%
\end{pgfscope}%
\begin{pgfscope}%
\pgfsys@transformshift{4.053655in}{3.045297in}%
\pgfsys@useobject{currentmarker}{}%
\end{pgfscope}%
\begin{pgfscope}%
\pgfsys@transformshift{4.055851in}{3.017478in}%
\pgfsys@useobject{currentmarker}{}%
\end{pgfscope}%
\begin{pgfscope}%
\pgfsys@transformshift{4.048660in}{2.989749in}%
\pgfsys@useobject{currentmarker}{}%
\end{pgfscope}%
\begin{pgfscope}%
\pgfsys@transformshift{4.053523in}{2.960997in}%
\pgfsys@useobject{currentmarker}{}%
\end{pgfscope}%
\begin{pgfscope}%
\pgfsys@transformshift{4.044212in}{2.930105in}%
\pgfsys@useobject{currentmarker}{}%
\end{pgfscope}%
\begin{pgfscope}%
\pgfsys@transformshift{4.045636in}{2.912416in}%
\pgfsys@useobject{currentmarker}{}%
\end{pgfscope}%
\begin{pgfscope}%
\pgfsys@transformshift{4.047197in}{2.893917in}%
\pgfsys@useobject{currentmarker}{}%
\end{pgfscope}%
\begin{pgfscope}%
\pgfsys@transformshift{4.042137in}{2.872722in}%
\pgfsys@useobject{currentmarker}{}%
\end{pgfscope}%
\begin{pgfscope}%
\pgfsys@transformshift{4.041487in}{2.860755in}%
\pgfsys@useobject{currentmarker}{}%
\end{pgfscope}%
\begin{pgfscope}%
\pgfsys@transformshift{4.041524in}{2.846279in}%
\pgfsys@useobject{currentmarker}{}%
\end{pgfscope}%
\begin{pgfscope}%
\pgfsys@transformshift{4.041367in}{2.838319in}%
\pgfsys@useobject{currentmarker}{}%
\end{pgfscope}%
\begin{pgfscope}%
\pgfsys@transformshift{4.038712in}{2.829437in}%
\pgfsys@useobject{currentmarker}{}%
\end{pgfscope}%
\begin{pgfscope}%
\pgfsys@transformshift{4.040746in}{2.816100in}%
\pgfsys@useobject{currentmarker}{}%
\end{pgfscope}%
\begin{pgfscope}%
\pgfsys@transformshift{4.045039in}{2.800689in}%
\pgfsys@useobject{currentmarker}{}%
\end{pgfscope}%
\begin{pgfscope}%
\pgfsys@transformshift{4.043659in}{2.782109in}%
\pgfsys@useobject{currentmarker}{}%
\end{pgfscope}%
\begin{pgfscope}%
\pgfsys@transformshift{4.037820in}{2.763711in}%
\pgfsys@useobject{currentmarker}{}%
\end{pgfscope}%
\begin{pgfscope}%
\pgfsys@transformshift{4.042461in}{2.738622in}%
\pgfsys@useobject{currentmarker}{}%
\end{pgfscope}%
\begin{pgfscope}%
\pgfsys@transformshift{4.043144in}{2.724606in}%
\pgfsys@useobject{currentmarker}{}%
\end{pgfscope}%
\begin{pgfscope}%
\pgfsys@transformshift{4.041903in}{2.710049in}%
\pgfsys@useobject{currentmarker}{}%
\end{pgfscope}%
\begin{pgfscope}%
\pgfsys@transformshift{4.038285in}{2.695105in}%
\pgfsys@useobject{currentmarker}{}%
\end{pgfscope}%
\begin{pgfscope}%
\pgfsys@transformshift{4.041855in}{2.676681in}%
\pgfsys@useobject{currentmarker}{}%
\end{pgfscope}%
\begin{pgfscope}%
\pgfsys@transformshift{4.041438in}{2.666367in}%
\pgfsys@useobject{currentmarker}{}%
\end{pgfscope}%
\begin{pgfscope}%
\pgfsys@transformshift{4.040449in}{2.653216in}%
\pgfsys@useobject{currentmarker}{}%
\end{pgfscope}%
\begin{pgfscope}%
\pgfsys@transformshift{4.040956in}{2.645981in}%
\pgfsys@useobject{currentmarker}{}%
\end{pgfscope}%
\begin{pgfscope}%
\pgfsys@transformshift{4.041978in}{2.642125in}%
\pgfsys@useobject{currentmarker}{}%
\end{pgfscope}%
\begin{pgfscope}%
\pgfsys@transformshift{4.041697in}{2.639949in}%
\pgfsys@useobject{currentmarker}{}%
\end{pgfscope}%
\begin{pgfscope}%
\pgfsys@transformshift{4.040527in}{2.635125in}%
\pgfsys@useobject{currentmarker}{}%
\end{pgfscope}%
\begin{pgfscope}%
\pgfsys@transformshift{4.041903in}{2.628293in}%
\pgfsys@useobject{currentmarker}{}%
\end{pgfscope}%
\begin{pgfscope}%
\pgfsys@transformshift{4.044892in}{2.620463in}%
\pgfsys@useobject{currentmarker}{}%
\end{pgfscope}%
\begin{pgfscope}%
\pgfsys@transformshift{4.043432in}{2.608893in}%
\pgfsys@useobject{currentmarker}{}%
\end{pgfscope}%
\begin{pgfscope}%
\pgfsys@transformshift{4.044380in}{2.595779in}%
\pgfsys@useobject{currentmarker}{}%
\end{pgfscope}%
\begin{pgfscope}%
\pgfsys@transformshift{4.048503in}{2.577626in}%
\pgfsys@useobject{currentmarker}{}%
\end{pgfscope}%
\begin{pgfscope}%
\pgfsys@transformshift{4.049588in}{2.567445in}%
\pgfsys@useobject{currentmarker}{}%
\end{pgfscope}%
\begin{pgfscope}%
\pgfsys@transformshift{4.048022in}{2.553932in}%
\pgfsys@useobject{currentmarker}{}%
\end{pgfscope}%
\begin{pgfscope}%
\pgfsys@transformshift{4.048365in}{2.539427in}%
\pgfsys@useobject{currentmarker}{}%
\end{pgfscope}%
\begin{pgfscope}%
\pgfsys@transformshift{4.052485in}{2.522127in}%
\pgfsys@useobject{currentmarker}{}%
\end{pgfscope}%
\begin{pgfscope}%
\pgfsys@transformshift{4.050845in}{2.503513in}%
\pgfsys@useobject{currentmarker}{}%
\end{pgfscope}%
\begin{pgfscope}%
\pgfsys@transformshift{4.044131in}{2.482494in}%
\pgfsys@useobject{currentmarker}{}%
\end{pgfscope}%
\begin{pgfscope}%
\pgfsys@transformshift{4.051820in}{2.457846in}%
\pgfsys@useobject{currentmarker}{}%
\end{pgfscope}%
\begin{pgfscope}%
\pgfsys@transformshift{4.057357in}{2.431658in}%
\pgfsys@useobject{currentmarker}{}%
\end{pgfscope}%
\begin{pgfscope}%
\pgfsys@transformshift{4.056684in}{2.401349in}%
\pgfsys@useobject{currentmarker}{}%
\end{pgfscope}%
\begin{pgfscope}%
\pgfsys@transformshift{4.054810in}{2.370598in}%
\pgfsys@useobject{currentmarker}{}%
\end{pgfscope}%
\begin{pgfscope}%
\pgfsys@transformshift{4.062216in}{2.337359in}%
\pgfsys@useobject{currentmarker}{}%
\end{pgfscope}%
\begin{pgfscope}%
\pgfsys@transformshift{4.067875in}{2.303119in}%
\pgfsys@useobject{currentmarker}{}%
\end{pgfscope}%
\begin{pgfscope}%
\pgfsys@transformshift{4.060013in}{2.265234in}%
\pgfsys@useobject{currentmarker}{}%
\end{pgfscope}%
\begin{pgfscope}%
\pgfsys@transformshift{4.067740in}{2.226582in}%
\pgfsys@useobject{currentmarker}{}%
\end{pgfscope}%
\begin{pgfscope}%
\pgfsys@transformshift{4.076764in}{2.187450in}%
\pgfsys@useobject{currentmarker}{}%
\end{pgfscope}%
\begin{pgfscope}%
\pgfsys@transformshift{4.080125in}{2.144176in}%
\pgfsys@useobject{currentmarker}{}%
\end{pgfscope}%
\begin{pgfscope}%
\pgfsys@transformshift{4.070849in}{2.098987in}%
\pgfsys@useobject{currentmarker}{}%
\end{pgfscope}%
\begin{pgfscope}%
\pgfsys@transformshift{4.087997in}{2.051953in}%
\pgfsys@useobject{currentmarker}{}%
\end{pgfscope}%
\begin{pgfscope}%
\pgfsys@transformshift{4.094354in}{2.001524in}%
\pgfsys@useobject{currentmarker}{}%
\end{pgfscope}%
\begin{pgfscope}%
\pgfsys@transformshift{4.084097in}{1.947333in}%
\pgfsys@useobject{currentmarker}{}%
\end{pgfscope}%
\begin{pgfscope}%
\pgfsys@transformshift{4.115654in}{1.900323in}%
\pgfsys@useobject{currentmarker}{}%
\end{pgfscope}%
\begin{pgfscope}%
\pgfsys@transformshift{4.144489in}{1.850919in}%
\pgfsys@useobject{currentmarker}{}%
\end{pgfscope}%
\begin{pgfscope}%
\pgfsys@transformshift{4.144188in}{1.792481in}%
\pgfsys@useobject{currentmarker}{}%
\end{pgfscope}%
\begin{pgfscope}%
\pgfsys@transformshift{4.136776in}{1.729914in}%
\pgfsys@useobject{currentmarker}{}%
\end{pgfscope}%
\begin{pgfscope}%
\pgfsys@transformshift{4.167857in}{1.672695in}%
\pgfsys@useobject{currentmarker}{}%
\end{pgfscope}%
\begin{pgfscope}%
\pgfsys@transformshift{4.200254in}{1.615528in}%
\pgfsys@useobject{currentmarker}{}%
\end{pgfscope}%
\begin{pgfscope}%
\pgfsys@transformshift{4.203671in}{1.545578in}%
\pgfsys@useobject{currentmarker}{}%
\end{pgfscope}%
\begin{pgfscope}%
\pgfsys@transformshift{4.198961in}{1.474352in}%
\pgfsys@useobject{currentmarker}{}%
\end{pgfscope}%
\begin{pgfscope}%
\pgfsys@transformshift{4.220374in}{1.404323in}%
\pgfsys@useobject{currentmarker}{}%
\end{pgfscope}%
\begin{pgfscope}%
\pgfsys@transformshift{4.224994in}{1.364313in}%
\pgfsys@useobject{currentmarker}{}%
\end{pgfscope}%
\begin{pgfscope}%
\pgfsys@transformshift{4.222162in}{1.319958in}%
\pgfsys@useobject{currentmarker}{}%
\end{pgfscope}%
\begin{pgfscope}%
\pgfsys@transformshift{4.236797in}{1.276615in}%
\pgfsys@useobject{currentmarker}{}%
\end{pgfscope}%
\begin{pgfscope}%
\pgfsys@transformshift{4.245558in}{1.253028in}%
\pgfsys@useobject{currentmarker}{}%
\end{pgfscope}%
\begin{pgfscope}%
\pgfsys@transformshift{4.254887in}{1.228641in}%
\pgfsys@useobject{currentmarker}{}%
\end{pgfscope}%
\begin{pgfscope}%
\pgfsys@transformshift{4.249539in}{1.199264in}%
\pgfsys@useobject{currentmarker}{}%
\end{pgfscope}%
\begin{pgfscope}%
\pgfsys@transformshift{4.263438in}{1.169059in}%
\pgfsys@useobject{currentmarker}{}%
\end{pgfscope}%
\begin{pgfscope}%
\pgfsys@transformshift{4.268117in}{1.151380in}%
\pgfsys@useobject{currentmarker}{}%
\end{pgfscope}%
\begin{pgfscope}%
\pgfsys@transformshift{4.271444in}{1.130851in}%
\pgfsys@useobject{currentmarker}{}%
\end{pgfscope}%
\begin{pgfscope}%
\pgfsys@transformshift{4.267581in}{1.108350in}%
\pgfsys@useobject{currentmarker}{}%
\end{pgfscope}%
\begin{pgfscope}%
\pgfsys@transformshift{4.278306in}{1.083202in}%
\pgfsys@useobject{currentmarker}{}%
\end{pgfscope}%
\begin{pgfscope}%
\pgfsys@transformshift{4.283879in}{1.055848in}%
\pgfsys@useobject{currentmarker}{}%
\end{pgfscope}%
\begin{pgfscope}%
\pgfsys@transformshift{4.281740in}{1.025400in}%
\pgfsys@useobject{currentmarker}{}%
\end{pgfscope}%
\begin{pgfscope}%
\pgfsys@transformshift{4.282283in}{1.008620in}%
\pgfsys@useobject{currentmarker}{}%
\end{pgfscope}%
\begin{pgfscope}%
\pgfsys@transformshift{4.290028in}{0.991861in}%
\pgfsys@useobject{currentmarker}{}%
\end{pgfscope}%
\begin{pgfscope}%
\pgfsys@transformshift{4.291531in}{0.972715in}%
\pgfsys@useobject{currentmarker}{}%
\end{pgfscope}%
\begin{pgfscope}%
\pgfsys@transformshift{4.289697in}{0.952214in}%
\pgfsys@useobject{currentmarker}{}%
\end{pgfscope}%
\begin{pgfscope}%
\pgfsys@transformshift{4.289682in}{0.940893in}%
\pgfsys@useobject{currentmarker}{}%
\end{pgfscope}%
\begin{pgfscope}%
\pgfsys@transformshift{4.292897in}{0.928700in}%
\pgfsys@useobject{currentmarker}{}%
\end{pgfscope}%
\begin{pgfscope}%
\pgfsys@transformshift{4.293309in}{0.921777in}%
\pgfsys@useobject{currentmarker}{}%
\end{pgfscope}%
\begin{pgfscope}%
\pgfsys@transformshift{4.292159in}{0.913008in}%
\pgfsys@useobject{currentmarker}{}%
\end{pgfscope}%
\begin{pgfscope}%
\pgfsys@transformshift{4.294940in}{0.903689in}%
\pgfsys@useobject{currentmarker}{}%
\end{pgfscope}%
\begin{pgfscope}%
\pgfsys@transformshift{4.299252in}{0.893223in}%
\pgfsys@useobject{currentmarker}{}%
\end{pgfscope}%
\begin{pgfscope}%
\pgfsys@transformshift{4.299326in}{0.886998in}%
\pgfsys@useobject{currentmarker}{}%
\end{pgfscope}%
\begin{pgfscope}%
\pgfsys@transformshift{4.298962in}{0.879175in}%
\pgfsys@useobject{currentmarker}{}%
\end{pgfscope}%
\begin{pgfscope}%
\pgfsys@transformshift{4.303238in}{0.868458in}%
\pgfsys@useobject{currentmarker}{}%
\end{pgfscope}%
\begin{pgfscope}%
\pgfsys@transformshift{4.307840in}{0.856597in}%
\pgfsys@useobject{currentmarker}{}%
\end{pgfscope}%
\begin{pgfscope}%
\pgfsys@transformshift{4.310980in}{0.843428in}%
\pgfsys@useobject{currentmarker}{}%
\end{pgfscope}%
\begin{pgfscope}%
\pgfsys@transformshift{4.308955in}{0.828524in}%
\pgfsys@useobject{currentmarker}{}%
\end{pgfscope}%
\begin{pgfscope}%
\pgfsys@transformshift{4.314078in}{0.811038in}%
\pgfsys@useobject{currentmarker}{}%
\end{pgfscope}%
\begin{pgfscope}%
\pgfsys@transformshift{4.321402in}{0.792573in}%
\pgfsys@useobject{currentmarker}{}%
\end{pgfscope}%
\begin{pgfscope}%
\pgfsys@transformshift{4.325748in}{0.771586in}%
\pgfsys@useobject{currentmarker}{}%
\end{pgfscope}%
\begin{pgfscope}%
\pgfsys@transformshift{4.327667in}{0.749501in}%
\pgfsys@useobject{currentmarker}{}%
\end{pgfscope}%
\begin{pgfscope}%
\pgfsys@transformshift{4.327987in}{0.737313in}%
\pgfsys@useobject{currentmarker}{}%
\end{pgfscope}%
\begin{pgfscope}%
\pgfsys@transformshift{4.327093in}{0.724569in}%
\pgfsys@useobject{currentmarker}{}%
\end{pgfscope}%
\begin{pgfscope}%
\pgfsys@transformshift{4.325336in}{0.711135in}%
\pgfsys@useobject{currentmarker}{}%
\end{pgfscope}%
\begin{pgfscope}%
\pgfsys@transformshift{4.321874in}{0.697423in}%
\pgfsys@useobject{currentmarker}{}%
\end{pgfscope}%
\begin{pgfscope}%
\pgfsys@transformshift{4.311114in}{0.686557in}%
\pgfsys@useobject{currentmarker}{}%
\end{pgfscope}%
\begin{pgfscope}%
\pgfsys@transformshift{4.294311in}{0.683179in}%
\pgfsys@useobject{currentmarker}{}%
\end{pgfscope}%
\begin{pgfscope}%
\pgfsys@transformshift{4.275235in}{0.683000in}%
\pgfsys@useobject{currentmarker}{}%
\end{pgfscope}%
\begin{pgfscope}%
\pgfsys@transformshift{4.256146in}{0.688526in}%
\pgfsys@useobject{currentmarker}{}%
\end{pgfscope}%
\begin{pgfscope}%
\pgfsys@transformshift{4.233982in}{0.691715in}%
\pgfsys@useobject{currentmarker}{}%
\end{pgfscope}%
\begin{pgfscope}%
\pgfsys@transformshift{4.212004in}{0.698993in}%
\pgfsys@useobject{currentmarker}{}%
\end{pgfscope}%
\begin{pgfscope}%
\pgfsys@transformshift{4.187077in}{0.701000in}%
\pgfsys@useobject{currentmarker}{}%
\end{pgfscope}%
\begin{pgfscope}%
\pgfsys@transformshift{4.162301in}{0.710854in}%
\pgfsys@useobject{currentmarker}{}%
\end{pgfscope}%
\begin{pgfscope}%
\pgfsys@transformshift{4.134133in}{0.709881in}%
\pgfsys@useobject{currentmarker}{}%
\end{pgfscope}%
\begin{pgfscope}%
\pgfsys@transformshift{4.105556in}{0.715571in}%
\pgfsys@useobject{currentmarker}{}%
\end{pgfscope}%
\begin{pgfscope}%
\pgfsys@transformshift{4.070458in}{0.719524in}%
\pgfsys@useobject{currentmarker}{}%
\end{pgfscope}%
\begin{pgfscope}%
\pgfsys@transformshift{4.033493in}{0.724567in}%
\pgfsys@useobject{currentmarker}{}%
\end{pgfscope}%
\begin{pgfscope}%
\pgfsys@transformshift{4.001769in}{0.752041in}%
\pgfsys@useobject{currentmarker}{}%
\end{pgfscope}%
\begin{pgfscope}%
\pgfsys@transformshift{3.958808in}{0.755417in}%
\pgfsys@useobject{currentmarker}{}%
\end{pgfscope}%
\begin{pgfscope}%
\pgfsys@transformshift{3.916191in}{0.767547in}%
\pgfsys@useobject{currentmarker}{}%
\end{pgfscope}%
\begin{pgfscope}%
\pgfsys@transformshift{3.867819in}{0.769496in}%
\pgfsys@useobject{currentmarker}{}%
\end{pgfscope}%
\begin{pgfscope}%
\pgfsys@transformshift{3.819138in}{0.773808in}%
\pgfsys@useobject{currentmarker}{}%
\end{pgfscope}%
\begin{pgfscope}%
\pgfsys@transformshift{3.766554in}{0.770606in}%
\pgfsys@useobject{currentmarker}{}%
\end{pgfscope}%
\begin{pgfscope}%
\pgfsys@transformshift{3.711650in}{0.777035in}%
\pgfsys@useobject{currentmarker}{}%
\end{pgfscope}%
\begin{pgfscope}%
\pgfsys@transformshift{3.655753in}{0.783973in}%
\pgfsys@useobject{currentmarker}{}%
\end{pgfscope}%
\begin{pgfscope}%
\pgfsys@transformshift{3.595629in}{0.786856in}%
\pgfsys@useobject{currentmarker}{}%
\end{pgfscope}%
\begin{pgfscope}%
\pgfsys@transformshift{3.535266in}{0.796944in}%
\pgfsys@useobject{currentmarker}{}%
\end{pgfscope}%
\begin{pgfscope}%
\pgfsys@transformshift{3.470353in}{0.807467in}%
\pgfsys@useobject{currentmarker}{}%
\end{pgfscope}%
\begin{pgfscope}%
\pgfsys@transformshift{3.404846in}{0.818899in}%
\pgfsys@useobject{currentmarker}{}%
\end{pgfscope}%
\begin{pgfscope}%
\pgfsys@transformshift{3.333084in}{0.823641in}%
\pgfsys@useobject{currentmarker}{}%
\end{pgfscope}%
\begin{pgfscope}%
\pgfsys@transformshift{3.293599in}{0.825984in}%
\pgfsys@useobject{currentmarker}{}%
\end{pgfscope}%
\begin{pgfscope}%
\pgfsys@transformshift{3.251228in}{0.840307in}%
\pgfsys@useobject{currentmarker}{}%
\end{pgfscope}%
\begin{pgfscope}%
\pgfsys@transformshift{3.205447in}{0.848110in}%
\pgfsys@useobject{currentmarker}{}%
\end{pgfscope}%
\begin{pgfscope}%
\pgfsys@transformshift{3.154488in}{0.850927in}%
\pgfsys@useobject{currentmarker}{}%
\end{pgfscope}%
\begin{pgfscope}%
\pgfsys@transformshift{3.102753in}{0.859821in}%
\pgfsys@useobject{currentmarker}{}%
\end{pgfscope}%
\begin{pgfscope}%
\pgfsys@transformshift{3.046412in}{0.868267in}%
\pgfsys@useobject{currentmarker}{}%
\end{pgfscope}%
\begin{pgfscope}%
\pgfsys@transformshift{2.989258in}{0.884181in}%
\pgfsys@useobject{currentmarker}{}%
\end{pgfscope}%
\begin{pgfscope}%
\pgfsys@transformshift{2.929048in}{0.891075in}%
\pgfsys@useobject{currentmarker}{}%
\end{pgfscope}%
\begin{pgfscope}%
\pgfsys@transformshift{2.863382in}{0.894493in}%
\pgfsys@useobject{currentmarker}{}%
\end{pgfscope}%
\begin{pgfscope}%
\pgfsys@transformshift{2.796599in}{0.893493in}%
\pgfsys@useobject{currentmarker}{}%
\end{pgfscope}%
\begin{pgfscope}%
\pgfsys@transformshift{2.727656in}{0.907216in}%
\pgfsys@useobject{currentmarker}{}%
\end{pgfscope}%
\begin{pgfscope}%
\pgfsys@transformshift{2.656627in}{0.912870in}%
\pgfsys@useobject{currentmarker}{}%
\end{pgfscope}%
\begin{pgfscope}%
\pgfsys@transformshift{2.580130in}{0.913546in}%
\pgfsys@useobject{currentmarker}{}%
\end{pgfscope}%
\begin{pgfscope}%
\pgfsys@transformshift{2.505288in}{0.936685in}%
\pgfsys@useobject{currentmarker}{}%
\end{pgfscope}%
\begin{pgfscope}%
\pgfsys@transformshift{2.424047in}{0.938082in}%
\pgfsys@useobject{currentmarker}{}%
\end{pgfscope}%
\begin{pgfscope}%
\pgfsys@transformshift{2.342048in}{0.945411in}%
\pgfsys@useobject{currentmarker}{}%
\end{pgfscope}%
\begin{pgfscope}%
\pgfsys@transformshift{2.254255in}{0.948616in}%
\pgfsys@useobject{currentmarker}{}%
\end{pgfscope}%
\begin{pgfscope}%
\pgfsys@transformshift{2.166401in}{0.963109in}%
\pgfsys@useobject{currentmarker}{}%
\end{pgfscope}%
\begin{pgfscope}%
\pgfsys@transformshift{2.073384in}{0.955317in}%
\pgfsys@useobject{currentmarker}{}%
\end{pgfscope}%
\begin{pgfscope}%
\pgfsys@transformshift{1.979188in}{0.959200in}%
\pgfsys@useobject{currentmarker}{}%
\end{pgfscope}%
\begin{pgfscope}%
\pgfsys@transformshift{1.878948in}{0.959214in}%
\pgfsys@useobject{currentmarker}{}%
\end{pgfscope}%
\begin{pgfscope}%
\pgfsys@transformshift{1.823998in}{0.963683in}%
\pgfsys@useobject{currentmarker}{}%
\end{pgfscope}%
\begin{pgfscope}%
\pgfsys@transformshift{1.767022in}{0.983691in}%
\pgfsys@useobject{currentmarker}{}%
\end{pgfscope}%
\begin{pgfscope}%
\pgfsys@transformshift{1.733940in}{0.986634in}%
\pgfsys@useobject{currentmarker}{}%
\end{pgfscope}%
\begin{pgfscope}%
\pgfsys@transformshift{1.694809in}{0.986296in}%
\pgfsys@useobject{currentmarker}{}%
\end{pgfscope}%
\begin{pgfscope}%
\pgfsys@transformshift{1.652361in}{0.988340in}%
\pgfsys@useobject{currentmarker}{}%
\end{pgfscope}%
\begin{pgfscope}%
\pgfsys@transformshift{1.608111in}{0.989775in}%
\pgfsys@useobject{currentmarker}{}%
\end{pgfscope}%
\begin{pgfscope}%
\pgfsys@transformshift{1.559384in}{0.997980in}%
\pgfsys@useobject{currentmarker}{}%
\end{pgfscope}%
\begin{pgfscope}%
\pgfsys@transformshift{1.509607in}{1.002345in}%
\pgfsys@useobject{currentmarker}{}%
\end{pgfscope}%
\begin{pgfscope}%
\pgfsys@transformshift{1.455648in}{0.999447in}%
\pgfsys@useobject{currentmarker}{}%
\end{pgfscope}%
\begin{pgfscope}%
\pgfsys@transformshift{1.400943in}{1.000410in}%
\pgfsys@useobject{currentmarker}{}%
\end{pgfscope}%
\begin{pgfscope}%
\pgfsys@transformshift{1.343498in}{0.995454in}%
\pgfsys@useobject{currentmarker}{}%
\end{pgfscope}%
\begin{pgfscope}%
\pgfsys@transformshift{1.285219in}{0.999084in}%
\pgfsys@useobject{currentmarker}{}%
\end{pgfscope}%
\begin{pgfscope}%
\pgfsys@transformshift{1.262209in}{1.001273in}%
\pgfsys@useobject{currentmarker}{}%
\end{pgfscope}%
\begin{pgfscope}%
\pgfsys@transformshift{1.302984in}{1.044860in}%
\pgfsys@useobject{currentmarker}{}%
\end{pgfscope}%
\begin{pgfscope}%
\pgfsys@transformshift{1.317309in}{1.105295in}%
\pgfsys@useobject{currentmarker}{}%
\end{pgfscope}%
\begin{pgfscope}%
\pgfsys@transformshift{1.315540in}{1.139409in}%
\pgfsys@useobject{currentmarker}{}%
\end{pgfscope}%
\begin{pgfscope}%
\pgfsys@transformshift{1.307730in}{1.174652in}%
\pgfsys@useobject{currentmarker}{}%
\end{pgfscope}%
\begin{pgfscope}%
\pgfsys@transformshift{1.309042in}{1.218440in}%
\pgfsys@useobject{currentmarker}{}%
\end{pgfscope}%
\begin{pgfscope}%
\pgfsys@transformshift{1.300254in}{1.262742in}%
\pgfsys@useobject{currentmarker}{}%
\end{pgfscope}%
\begin{pgfscope}%
\pgfsys@transformshift{1.293597in}{1.308372in}%
\pgfsys@useobject{currentmarker}{}%
\end{pgfscope}%
\begin{pgfscope}%
\pgfsys@transformshift{1.305440in}{1.354344in}%
\pgfsys@useobject{currentmarker}{}%
\end{pgfscope}%
\begin{pgfscope}%
\pgfsys@transformshift{1.300552in}{1.409055in}%
\pgfsys@useobject{currentmarker}{}%
\end{pgfscope}%
\begin{pgfscope}%
\pgfsys@transformshift{1.291934in}{1.438010in}%
\pgfsys@useobject{currentmarker}{}%
\end{pgfscope}%
\begin{pgfscope}%
\pgfsys@transformshift{1.292506in}{1.472279in}%
\pgfsys@useobject{currentmarker}{}%
\end{pgfscope}%
\begin{pgfscope}%
\pgfsys@transformshift{1.287255in}{1.508870in}%
\pgfsys@useobject{currentmarker}{}%
\end{pgfscope}%
\begin{pgfscope}%
\pgfsys@transformshift{1.282261in}{1.546481in}%
\pgfsys@useobject{currentmarker}{}%
\end{pgfscope}%
\begin{pgfscope}%
\pgfsys@transformshift{1.293226in}{1.587196in}%
\pgfsys@useobject{currentmarker}{}%
\end{pgfscope}%
\begin{pgfscope}%
\pgfsys@transformshift{1.292654in}{1.610380in}%
\pgfsys@useobject{currentmarker}{}%
\end{pgfscope}%
\begin{pgfscope}%
\pgfsys@transformshift{1.284258in}{1.632756in}%
\pgfsys@useobject{currentmarker}{}%
\end{pgfscope}%
\begin{pgfscope}%
\pgfsys@transformshift{1.293010in}{1.662699in}%
\pgfsys@useobject{currentmarker}{}%
\end{pgfscope}%
\begin{pgfscope}%
\pgfsys@transformshift{1.301124in}{1.696190in}%
\pgfsys@useobject{currentmarker}{}%
\end{pgfscope}%
\begin{pgfscope}%
\pgfsys@transformshift{1.293224in}{1.734078in}%
\pgfsys@useobject{currentmarker}{}%
\end{pgfscope}%
\begin{pgfscope}%
\pgfsys@transformshift{1.300635in}{1.776299in}%
\pgfsys@useobject{currentmarker}{}%
\end{pgfscope}%
\begin{pgfscope}%
\pgfsys@transformshift{1.311033in}{1.818915in}%
\pgfsys@useobject{currentmarker}{}%
\end{pgfscope}%
\begin{pgfscope}%
\pgfsys@transformshift{1.313475in}{1.868738in}%
\pgfsys@useobject{currentmarker}{}%
\end{pgfscope}%
\begin{pgfscope}%
\pgfsys@transformshift{1.298812in}{1.919449in}%
\pgfsys@useobject{currentmarker}{}%
\end{pgfscope}%
\begin{pgfscope}%
\pgfsys@transformshift{1.319142in}{1.974095in}%
\pgfsys@useobject{currentmarker}{}%
\end{pgfscope}%
\begin{pgfscope}%
\pgfsys@transformshift{1.335589in}{2.031309in}%
\pgfsys@useobject{currentmarker}{}%
\end{pgfscope}%
\begin{pgfscope}%
\pgfsys@transformshift{1.326248in}{2.096602in}%
\pgfsys@useobject{currentmarker}{}%
\end{pgfscope}%
\begin{pgfscope}%
\pgfsys@transformshift{1.330344in}{2.162951in}%
\pgfsys@useobject{currentmarker}{}%
\end{pgfscope}%
\begin{pgfscope}%
\pgfsys@transformshift{1.343317in}{2.197133in}%
\pgfsys@useobject{currentmarker}{}%
\end{pgfscope}%
\begin{pgfscope}%
\pgfsys@transformshift{1.339489in}{2.239120in}%
\pgfsys@useobject{currentmarker}{}%
\end{pgfscope}%
\begin{pgfscope}%
\pgfsys@transformshift{1.330608in}{2.283099in}%
\pgfsys@useobject{currentmarker}{}%
\end{pgfscope}%
\begin{pgfscope}%
\pgfsys@transformshift{1.342325in}{2.329360in}%
\pgfsys@useobject{currentmarker}{}%
\end{pgfscope}%
\begin{pgfscope}%
\pgfsys@transformshift{1.347101in}{2.355169in}%
\pgfsys@useobject{currentmarker}{}%
\end{pgfscope}%
\begin{pgfscope}%
\pgfsys@transformshift{1.340720in}{2.385224in}%
\pgfsys@useobject{currentmarker}{}%
\end{pgfscope}%
\begin{pgfscope}%
\pgfsys@transformshift{1.342829in}{2.416641in}%
\pgfsys@useobject{currentmarker}{}%
\end{pgfscope}%
\begin{pgfscope}%
\pgfsys@transformshift{1.345873in}{2.433689in}%
\pgfsys@useobject{currentmarker}{}%
\end{pgfscope}%
\begin{pgfscope}%
\pgfsys@transformshift{1.342820in}{2.455713in}%
\pgfsys@useobject{currentmarker}{}%
\end{pgfscope}%
\begin{pgfscope}%
\pgfsys@transformshift{1.340155in}{2.467648in}%
\pgfsys@useobject{currentmarker}{}%
\end{pgfscope}%
\begin{pgfscope}%
\pgfsys@transformshift{1.343820in}{2.483319in}%
\pgfsys@useobject{currentmarker}{}%
\end{pgfscope}%
\begin{pgfscope}%
\pgfsys@transformshift{1.341633in}{2.499851in}%
\pgfsys@useobject{currentmarker}{}%
\end{pgfscope}%
\begin{pgfscope}%
\pgfsys@transformshift{1.334256in}{2.516947in}%
\pgfsys@useobject{currentmarker}{}%
\end{pgfscope}%
\begin{pgfscope}%
\pgfsys@transformshift{1.338741in}{2.540963in}%
\pgfsys@useobject{currentmarker}{}%
\end{pgfscope}%
\begin{pgfscope}%
\pgfsys@transformshift{1.344867in}{2.565422in}%
\pgfsys@useobject{currentmarker}{}%
\end{pgfscope}%
\begin{pgfscope}%
\pgfsys@transformshift{1.336016in}{2.594217in}%
\pgfsys@useobject{currentmarker}{}%
\end{pgfscope}%
\begin{pgfscope}%
\pgfsys@transformshift{1.339257in}{2.625387in}%
\pgfsys@useobject{currentmarker}{}%
\end{pgfscope}%
\begin{pgfscope}%
\pgfsys@transformshift{1.344734in}{2.641730in}%
\pgfsys@useobject{currentmarker}{}%
\end{pgfscope}%
\begin{pgfscope}%
\pgfsys@transformshift{1.339335in}{2.662183in}%
\pgfsys@useobject{currentmarker}{}%
\end{pgfscope}%
\begin{pgfscope}%
\pgfsys@transformshift{1.345533in}{2.684791in}%
\pgfsys@useobject{currentmarker}{}%
\end{pgfscope}%
\begin{pgfscope}%
\pgfsys@transformshift{1.348070in}{2.697432in}%
\pgfsys@useobject{currentmarker}{}%
\end{pgfscope}%
\begin{pgfscope}%
\pgfsys@transformshift{1.346273in}{2.716066in}%
\pgfsys@useobject{currentmarker}{}%
\end{pgfscope}%
\begin{pgfscope}%
\pgfsys@transformshift{1.340372in}{2.737577in}%
\pgfsys@useobject{currentmarker}{}%
\end{pgfscope}%
\begin{pgfscope}%
\pgfsys@transformshift{1.349308in}{2.761759in}%
\pgfsys@useobject{currentmarker}{}%
\end{pgfscope}%
\begin{pgfscope}%
\pgfsys@transformshift{1.350777in}{2.775862in}%
\pgfsys@useobject{currentmarker}{}%
\end{pgfscope}%
\begin{pgfscope}%
\pgfsys@transformshift{1.344801in}{2.793808in}%
\pgfsys@useobject{currentmarker}{}%
\end{pgfscope}%
\begin{pgfscope}%
\pgfsys@transformshift{1.347273in}{2.813862in}%
\pgfsys@useobject{currentmarker}{}%
\end{pgfscope}%
\begin{pgfscope}%
\pgfsys@transformshift{1.352714in}{2.833956in}%
\pgfsys@useobject{currentmarker}{}%
\end{pgfscope}%
\begin{pgfscope}%
\pgfsys@transformshift{1.354610in}{2.857570in}%
\pgfsys@useobject{currentmarker}{}%
\end{pgfscope}%
\begin{pgfscope}%
\pgfsys@transformshift{1.345887in}{2.882212in}%
\pgfsys@useobject{currentmarker}{}%
\end{pgfscope}%
\begin{pgfscope}%
\pgfsys@transformshift{1.353347in}{2.912270in}%
\pgfsys@useobject{currentmarker}{}%
\end{pgfscope}%
\begin{pgfscope}%
\pgfsys@transformshift{1.356209in}{2.929061in}%
\pgfsys@useobject{currentmarker}{}%
\end{pgfscope}%
\begin{pgfscope}%
\pgfsys@transformshift{1.351196in}{2.950450in}%
\pgfsys@useobject{currentmarker}{}%
\end{pgfscope}%
\begin{pgfscope}%
\pgfsys@transformshift{1.351725in}{2.962521in}%
\pgfsys@useobject{currentmarker}{}%
\end{pgfscope}%
\begin{pgfscope}%
\pgfsys@transformshift{1.354219in}{2.968681in}%
\pgfsys@useobject{currentmarker}{}%
\end{pgfscope}%
\begin{pgfscope}%
\pgfsys@transformshift{1.353195in}{2.977821in}%
\pgfsys@useobject{currentmarker}{}%
\end{pgfscope}%
\begin{pgfscope}%
\pgfsys@transformshift{1.350355in}{2.987689in}%
\pgfsys@useobject{currentmarker}{}%
\end{pgfscope}%
\begin{pgfscope}%
\pgfsys@transformshift{1.354626in}{3.003358in}%
\pgfsys@useobject{currentmarker}{}%
\end{pgfscope}%
\begin{pgfscope}%
\pgfsys@transformshift{1.358518in}{3.019695in}%
\pgfsys@useobject{currentmarker}{}%
\end{pgfscope}%
\begin{pgfscope}%
\pgfsys@transformshift{1.352973in}{3.040734in}%
\pgfsys@useobject{currentmarker}{}%
\end{pgfscope}%
\begin{pgfscope}%
\pgfsys@transformshift{1.353148in}{3.063128in}%
\pgfsys@useobject{currentmarker}{}%
\end{pgfscope}%
\begin{pgfscope}%
\pgfsys@transformshift{1.363570in}{3.086350in}%
\pgfsys@useobject{currentmarker}{}%
\end{pgfscope}%
\begin{pgfscope}%
\pgfsys@transformshift{1.362750in}{3.112700in}%
\pgfsys@useobject{currentmarker}{}%
\end{pgfscope}%
\begin{pgfscope}%
\pgfsys@transformshift{1.356900in}{3.139658in}%
\pgfsys@useobject{currentmarker}{}%
\end{pgfscope}%
\begin{pgfscope}%
\pgfsys@transformshift{1.367525in}{3.171647in}%
\pgfsys@useobject{currentmarker}{}%
\end{pgfscope}%
\begin{pgfscope}%
\pgfsys@transformshift{1.371905in}{3.206361in}%
\pgfsys@useobject{currentmarker}{}%
\end{pgfscope}%
\begin{pgfscope}%
\pgfsys@transformshift{1.371109in}{3.244194in}%
\pgfsys@useobject{currentmarker}{}%
\end{pgfscope}%
\begin{pgfscope}%
\pgfsys@transformshift{1.368114in}{3.264790in}%
\pgfsys@useobject{currentmarker}{}%
\end{pgfscope}%
\begin{pgfscope}%
\pgfsys@transformshift{1.374961in}{3.289564in}%
\pgfsys@useobject{currentmarker}{}%
\end{pgfscope}%
\begin{pgfscope}%
\pgfsys@transformshift{1.372735in}{3.303525in}%
\pgfsys@useobject{currentmarker}{}%
\end{pgfscope}%
\begin{pgfscope}%
\pgfsys@transformshift{1.368284in}{3.318746in}%
\pgfsys@useobject{currentmarker}{}%
\end{pgfscope}%
\begin{pgfscope}%
\pgfsys@transformshift{1.372399in}{3.338870in}%
\pgfsys@useobject{currentmarker}{}%
\end{pgfscope}%
\begin{pgfscope}%
\pgfsys@transformshift{1.372278in}{3.359891in}%
\pgfsys@useobject{currentmarker}{}%
\end{pgfscope}%
\begin{pgfscope}%
\pgfsys@transformshift{1.369774in}{3.384666in}%
\pgfsys@useobject{currentmarker}{}%
\end{pgfscope}%
\begin{pgfscope}%
\pgfsys@transformshift{1.368031in}{3.398250in}%
\pgfsys@useobject{currentmarker}{}%
\end{pgfscope}%
\begin{pgfscope}%
\pgfsys@transformshift{1.372194in}{3.412045in}%
\pgfsys@useobject{currentmarker}{}%
\end{pgfscope}%
\begin{pgfscope}%
\pgfsys@transformshift{1.371468in}{3.419937in}%
\pgfsys@useobject{currentmarker}{}%
\end{pgfscope}%
\begin{pgfscope}%
\pgfsys@transformshift{1.368825in}{3.428246in}%
\pgfsys@useobject{currentmarker}{}%
\end{pgfscope}%
\begin{pgfscope}%
\pgfsys@transformshift{1.373781in}{3.443950in}%
\pgfsys@useobject{currentmarker}{}%
\end{pgfscope}%
\begin{pgfscope}%
\pgfsys@transformshift{1.376330in}{3.460807in}%
\pgfsys@useobject{currentmarker}{}%
\end{pgfscope}%
\begin{pgfscope}%
\pgfsys@transformshift{1.370670in}{3.480987in}%
\pgfsys@useobject{currentmarker}{}%
\end{pgfscope}%
\begin{pgfscope}%
\pgfsys@transformshift{1.374818in}{3.503775in}%
\pgfsys@useobject{currentmarker}{}%
\end{pgfscope}%
\begin{pgfscope}%
\pgfsys@transformshift{1.380195in}{3.515324in}%
\pgfsys@useobject{currentmarker}{}%
\end{pgfscope}%
\begin{pgfscope}%
\pgfsys@transformshift{1.376104in}{3.532906in}%
\pgfsys@useobject{currentmarker}{}%
\end{pgfscope}%
\begin{pgfscope}%
\pgfsys@transformshift{1.375336in}{3.551485in}%
\pgfsys@useobject{currentmarker}{}%
\end{pgfscope}%
\begin{pgfscope}%
\pgfsys@transformshift{1.385668in}{3.571903in}%
\pgfsys@useobject{currentmarker}{}%
\end{pgfscope}%
\begin{pgfscope}%
\pgfsys@transformshift{1.382080in}{3.595543in}%
\pgfsys@useobject{currentmarker}{}%
\end{pgfscope}%
\begin{pgfscope}%
\pgfsys@transformshift{1.374616in}{3.619499in}%
\pgfsys@useobject{currentmarker}{}%
\end{pgfscope}%
\begin{pgfscope}%
\pgfsys@transformshift{1.383940in}{3.650625in}%
\pgfsys@useobject{currentmarker}{}%
\end{pgfscope}%
\begin{pgfscope}%
\pgfsys@transformshift{1.389558in}{3.683498in}%
\pgfsys@useobject{currentmarker}{}%
\end{pgfscope}%
\begin{pgfscope}%
\pgfsys@transformshift{1.380672in}{3.720156in}%
\pgfsys@useobject{currentmarker}{}%
\end{pgfscope}%
\begin{pgfscope}%
\pgfsys@transformshift{1.380308in}{3.740899in}%
\pgfsys@useobject{currentmarker}{}%
\end{pgfscope}%
\begin{pgfscope}%
\pgfsys@transformshift{1.388355in}{3.762230in}%
\pgfsys@useobject{currentmarker}{}%
\end{pgfscope}%
\begin{pgfscope}%
\pgfsys@transformshift{1.384104in}{3.788907in}%
\pgfsys@useobject{currentmarker}{}%
\end{pgfscope}%
\begin{pgfscope}%
\pgfsys@transformshift{1.377835in}{3.816316in}%
\pgfsys@useobject{currentmarker}{}%
\end{pgfscope}%
\begin{pgfscope}%
\pgfsys@transformshift{1.386234in}{3.848020in}%
\pgfsys@useobject{currentmarker}{}%
\end{pgfscope}%
\begin{pgfscope}%
\pgfsys@transformshift{1.386803in}{3.866050in}%
\pgfsys@useobject{currentmarker}{}%
\end{pgfscope}%
\begin{pgfscope}%
\pgfsys@transformshift{1.379194in}{3.886512in}%
\pgfsys@useobject{currentmarker}{}%
\end{pgfscope}%
\begin{pgfscope}%
\pgfsys@transformshift{1.386116in}{3.910287in}%
\pgfsys@useobject{currentmarker}{}%
\end{pgfscope}%
\begin{pgfscope}%
\pgfsys@transformshift{1.393234in}{3.934872in}%
\pgfsys@useobject{currentmarker}{}%
\end{pgfscope}%
\begin{pgfscope}%
\pgfsys@transformshift{1.396721in}{3.948511in}%
\pgfsys@useobject{currentmarker}{}%
\end{pgfscope}%
\begin{pgfscope}%
\pgfsys@transformshift{1.397531in}{3.956211in}%
\pgfsys@useobject{currentmarker}{}%
\end{pgfscope}%
\begin{pgfscope}%
\pgfsys@transformshift{1.397714in}{3.960466in}%
\pgfsys@useobject{currentmarker}{}%
\end{pgfscope}%
\begin{pgfscope}%
\pgfsys@transformshift{1.397280in}{3.962767in}%
\pgfsys@useobject{currentmarker}{}%
\end{pgfscope}%
\begin{pgfscope}%
\pgfsys@transformshift{1.396717in}{3.963926in}%
\pgfsys@useobject{currentmarker}{}%
\end{pgfscope}%
\begin{pgfscope}%
\pgfsys@transformshift{1.395425in}{3.965875in}%
\pgfsys@useobject{currentmarker}{}%
\end{pgfscope}%
\begin{pgfscope}%
\pgfsys@transformshift{1.392673in}{3.967739in}%
\pgfsys@useobject{currentmarker}{}%
\end{pgfscope}%
\begin{pgfscope}%
\pgfsys@transformshift{1.391122in}{3.968706in}%
\pgfsys@useobject{currentmarker}{}%
\end{pgfscope}%
\begin{pgfscope}%
\pgfsys@transformshift{1.388681in}{3.969389in}%
\pgfsys@useobject{currentmarker}{}%
\end{pgfscope}%
\begin{pgfscope}%
\pgfsys@transformshift{1.385693in}{3.970320in}%
\pgfsys@useobject{currentmarker}{}%
\end{pgfscope}%
\begin{pgfscope}%
\pgfsys@transformshift{1.381543in}{3.970682in}%
\pgfsys@useobject{currentmarker}{}%
\end{pgfscope}%
\begin{pgfscope}%
\pgfsys@transformshift{1.376942in}{3.971511in}%
\pgfsys@useobject{currentmarker}{}%
\end{pgfscope}%
\begin{pgfscope}%
\pgfsys@transformshift{1.371907in}{3.970209in}%
\pgfsys@useobject{currentmarker}{}%
\end{pgfscope}%
\begin{pgfscope}%
\pgfsys@transformshift{1.363866in}{3.971527in}%
\pgfsys@useobject{currentmarker}{}%
\end{pgfscope}%
\begin{pgfscope}%
\pgfsys@transformshift{1.359408in}{3.971076in}%
\pgfsys@useobject{currentmarker}{}%
\end{pgfscope}%
\begin{pgfscope}%
\pgfsys@transformshift{1.353823in}{3.971761in}%
\pgfsys@useobject{currentmarker}{}%
\end{pgfscope}%
\begin{pgfscope}%
\pgfsys@transformshift{1.346992in}{3.971158in}%
\pgfsys@useobject{currentmarker}{}%
\end{pgfscope}%
\begin{pgfscope}%
\pgfsys@transformshift{1.343256in}{3.971680in}%
\pgfsys@useobject{currentmarker}{}%
\end{pgfscope}%
\begin{pgfscope}%
\pgfsys@transformshift{1.336430in}{3.971582in}%
\pgfsys@useobject{currentmarker}{}%
\end{pgfscope}%
\begin{pgfscope}%
\pgfsys@transformshift{1.322104in}{3.971247in}%
\pgfsys@useobject{currentmarker}{}%
\end{pgfscope}%
\begin{pgfscope}%
\pgfsys@transformshift{1.303867in}{3.974111in}%
\pgfsys@useobject{currentmarker}{}%
\end{pgfscope}%
\begin{pgfscope}%
\pgfsys@transformshift{1.293717in}{3.974366in}%
\pgfsys@useobject{currentmarker}{}%
\end{pgfscope}%
\begin{pgfscope}%
\pgfsys@transformshift{1.282150in}{3.976472in}%
\pgfsys@useobject{currentmarker}{}%
\end{pgfscope}%
\begin{pgfscope}%
\pgfsys@transformshift{1.267623in}{3.976911in}%
\pgfsys@useobject{currentmarker}{}%
\end{pgfscope}%
\begin{pgfscope}%
\pgfsys@transformshift{1.251804in}{3.978731in}%
\pgfsys@useobject{currentmarker}{}%
\end{pgfscope}%
\begin{pgfscope}%
\pgfsys@transformshift{1.254035in}{3.979210in}%
\pgfsys@useobject{currentmarker}{}%
\end{pgfscope}%
\begin{pgfscope}%
\pgfsys@transformshift{1.274994in}{3.979107in}%
\pgfsys@useobject{currentmarker}{}%
\end{pgfscope}%
\begin{pgfscope}%
\pgfsys@transformshift{1.301462in}{3.975377in}%
\pgfsys@useobject{currentmarker}{}%
\end{pgfscope}%
\begin{pgfscope}%
\pgfsys@transformshift{1.331300in}{3.981080in}%
\pgfsys@useobject{currentmarker}{}%
\end{pgfscope}%
\begin{pgfscope}%
\pgfsys@transformshift{1.362121in}{3.979530in}%
\pgfsys@useobject{currentmarker}{}%
\end{pgfscope}%
\begin{pgfscope}%
\pgfsys@transformshift{1.396497in}{3.981535in}%
\pgfsys@useobject{currentmarker}{}%
\end{pgfscope}%
\begin{pgfscope}%
\pgfsys@transformshift{1.435024in}{3.985441in}%
\pgfsys@useobject{currentmarker}{}%
\end{pgfscope}%
\begin{pgfscope}%
\pgfsys@transformshift{1.474497in}{3.989546in}%
\pgfsys@useobject{currentmarker}{}%
\end{pgfscope}%
\begin{pgfscope}%
\pgfsys@transformshift{1.517581in}{3.987235in}%
\pgfsys@useobject{currentmarker}{}%
\end{pgfscope}%
\begin{pgfscope}%
\pgfsys@transformshift{1.566715in}{3.991626in}%
\pgfsys@useobject{currentmarker}{}%
\end{pgfscope}%
\begin{pgfscope}%
\pgfsys@transformshift{1.616963in}{3.996460in}%
\pgfsys@useobject{currentmarker}{}%
\end{pgfscope}%
\begin{pgfscope}%
\pgfsys@transformshift{1.670850in}{4.002322in}%
\pgfsys@useobject{currentmarker}{}%
\end{pgfscope}%
\begin{pgfscope}%
\pgfsys@transformshift{1.729996in}{4.001404in}%
\pgfsys@useobject{currentmarker}{}%
\end{pgfscope}%
\begin{pgfscope}%
\pgfsys@transformshift{1.791113in}{4.007141in}%
\pgfsys@useobject{currentmarker}{}%
\end{pgfscope}%
\begin{pgfscope}%
\pgfsys@transformshift{1.853524in}{4.008851in}%
\pgfsys@useobject{currentmarker}{}%
\end{pgfscope}%
\begin{pgfscope}%
\pgfsys@transformshift{1.919046in}{4.006906in}%
\pgfsys@useobject{currentmarker}{}%
\end{pgfscope}%
\begin{pgfscope}%
\pgfsys@transformshift{1.990102in}{3.987200in}%
\pgfsys@useobject{currentmarker}{}%
\end{pgfscope}%
\begin{pgfscope}%
\pgfsys@transformshift{2.063446in}{4.011510in}%
\pgfsys@useobject{currentmarker}{}%
\end{pgfscope}%
\begin{pgfscope}%
\pgfsys@transformshift{2.141423in}{4.020244in}%
\pgfsys@useobject{currentmarker}{}%
\end{pgfscope}%
\begin{pgfscope}%
\pgfsys@transformshift{2.221030in}{4.036535in}%
\pgfsys@useobject{currentmarker}{}%
\end{pgfscope}%
\begin{pgfscope}%
\pgfsys@transformshift{2.304591in}{4.041848in}%
\pgfsys@useobject{currentmarker}{}%
\end{pgfscope}%
\begin{pgfscope}%
\pgfsys@transformshift{2.350625in}{4.040591in}%
\pgfsys@useobject{currentmarker}{}%
\end{pgfscope}%
\begin{pgfscope}%
\pgfsys@transformshift{2.375948in}{4.040068in}%
\pgfsys@useobject{currentmarker}{}%
\end{pgfscope}%
\begin{pgfscope}%
\pgfsys@transformshift{2.405282in}{4.042348in}%
\pgfsys@useobject{currentmarker}{}%
\end{pgfscope}%
\begin{pgfscope}%
\pgfsys@transformshift{2.437168in}{4.043000in}%
\pgfsys@useobject{currentmarker}{}%
\end{pgfscope}%
\begin{pgfscope}%
\pgfsys@transformshift{2.470627in}{4.039538in}%
\pgfsys@useobject{currentmarker}{}%
\end{pgfscope}%
\begin{pgfscope}%
\pgfsys@transformshift{2.504877in}{4.033691in}%
\pgfsys@useobject{currentmarker}{}%
\end{pgfscope}%
\begin{pgfscope}%
\pgfsys@transformshift{2.540203in}{4.031098in}%
\pgfsys@useobject{currentmarker}{}%
\end{pgfscope}%
\begin{pgfscope}%
\pgfsys@transformshift{2.577003in}{4.027361in}%
\pgfsys@useobject{currentmarker}{}%
\end{pgfscope}%
\begin{pgfscope}%
\pgfsys@transformshift{2.614824in}{4.028011in}%
\pgfsys@useobject{currentmarker}{}%
\end{pgfscope}%
\begin{pgfscope}%
\pgfsys@transformshift{2.653925in}{4.023428in}%
\pgfsys@useobject{currentmarker}{}%
\end{pgfscope}%
\begin{pgfscope}%
\pgfsys@transformshift{2.694025in}{4.006116in}%
\pgfsys@useobject{currentmarker}{}%
\end{pgfscope}%
\begin{pgfscope}%
\pgfsys@transformshift{2.741970in}{4.008515in}%
\pgfsys@useobject{currentmarker}{}%
\end{pgfscope}%
\begin{pgfscope}%
\pgfsys@transformshift{2.787517in}{3.990538in}%
\pgfsys@useobject{currentmarker}{}%
\end{pgfscope}%
\begin{pgfscope}%
\pgfsys@transformshift{2.833961in}{4.010157in}%
\pgfsys@useobject{currentmarker}{}%
\end{pgfscope}%
\begin{pgfscope}%
\pgfsys@transformshift{2.887570in}{4.000264in}%
\pgfsys@useobject{currentmarker}{}%
\end{pgfscope}%
\begin{pgfscope}%
\pgfsys@transformshift{2.943059in}{4.005563in}%
\pgfsys@useobject{currentmarker}{}%
\end{pgfscope}%
\begin{pgfscope}%
\pgfsys@transformshift{3.002342in}{4.009348in}%
\pgfsys@useobject{currentmarker}{}%
\end{pgfscope}%
\begin{pgfscope}%
\pgfsys@transformshift{3.064749in}{4.009328in}%
\pgfsys@useobject{currentmarker}{}%
\end{pgfscope}%
\begin{pgfscope}%
\pgfsys@transformshift{3.128666in}{4.008449in}%
\pgfsys@useobject{currentmarker}{}%
\end{pgfscope}%
\begin{pgfscope}%
\pgfsys@transformshift{3.196682in}{4.002066in}%
\pgfsys@useobject{currentmarker}{}%
\end{pgfscope}%
\begin{pgfscope}%
\pgfsys@transformshift{3.269852in}{3.984188in}%
\pgfsys@useobject{currentmarker}{}%
\end{pgfscope}%
\begin{pgfscope}%
\pgfsys@transformshift{3.347331in}{3.981686in}%
\pgfsys@useobject{currentmarker}{}%
\end{pgfscope}%
\begin{pgfscope}%
\pgfsys@transformshift{3.425783in}{3.980439in}%
\pgfsys@useobject{currentmarker}{}%
\end{pgfscope}%
\begin{pgfscope}%
\pgfsys@transformshift{3.468852in}{3.983132in}%
\pgfsys@useobject{currentmarker}{}%
\end{pgfscope}%
\begin{pgfscope}%
\pgfsys@transformshift{3.510071in}{3.999036in}%
\pgfsys@useobject{currentmarker}{}%
\end{pgfscope}%
\begin{pgfscope}%
\pgfsys@transformshift{3.555812in}{4.003488in}%
\pgfsys@useobject{currentmarker}{}%
\end{pgfscope}%
\begin{pgfscope}%
\pgfsys@transformshift{3.581006in}{4.001451in}%
\pgfsys@useobject{currentmarker}{}%
\end{pgfscope}%
\begin{pgfscope}%
\pgfsys@transformshift{3.610156in}{3.997794in}%
\pgfsys@useobject{currentmarker}{}%
\end{pgfscope}%
\begin{pgfscope}%
\pgfsys@transformshift{3.644710in}{3.988020in}%
\pgfsys@useobject{currentmarker}{}%
\end{pgfscope}%
\begin{pgfscope}%
\pgfsys@transformshift{3.683094in}{3.988391in}%
\pgfsys@useobject{currentmarker}{}%
\end{pgfscope}%
\begin{pgfscope}%
\pgfsys@transformshift{3.721617in}{3.994483in}%
\pgfsys@useobject{currentmarker}{}%
\end{pgfscope}%
\begin{pgfscope}%
\pgfsys@transformshift{3.762273in}{3.995280in}%
\pgfsys@useobject{currentmarker}{}%
\end{pgfscope}%
\begin{pgfscope}%
\pgfsys@transformshift{3.804233in}{4.001661in}%
\pgfsys@useobject{currentmarker}{}%
\end{pgfscope}%
\begin{pgfscope}%
\pgfsys@transformshift{3.847542in}{3.998243in}%
\pgfsys@useobject{currentmarker}{}%
\end{pgfscope}%
\begin{pgfscope}%
\pgfsys@transformshift{3.871353in}{4.000245in}%
\pgfsys@useobject{currentmarker}{}%
\end{pgfscope}%
\begin{pgfscope}%
\pgfsys@transformshift{3.897839in}{4.000359in}%
\pgfsys@useobject{currentmarker}{}%
\end{pgfscope}%
\begin{pgfscope}%
\pgfsys@transformshift{3.929191in}{3.992475in}%
\pgfsys@useobject{currentmarker}{}%
\end{pgfscope}%
\begin{pgfscope}%
\pgfsys@transformshift{3.962507in}{4.003331in}%
\pgfsys@useobject{currentmarker}{}%
\end{pgfscope}%
\begin{pgfscope}%
\pgfsys@transformshift{3.998683in}{3.998121in}%
\pgfsys@useobject{currentmarker}{}%
\end{pgfscope}%
\begin{pgfscope}%
\pgfsys@transformshift{4.037746in}{4.003300in}%
\pgfsys@useobject{currentmarker}{}%
\end{pgfscope}%
\begin{pgfscope}%
\pgfsys@transformshift{4.082887in}{4.010024in}%
\pgfsys@useobject{currentmarker}{}%
\end{pgfscope}%
\begin{pgfscope}%
\pgfsys@transformshift{4.132843in}{4.013381in}%
\pgfsys@useobject{currentmarker}{}%
\end{pgfscope}%
\begin{pgfscope}%
\pgfsys@transformshift{4.160350in}{4.012080in}%
\pgfsys@useobject{currentmarker}{}%
\end{pgfscope}%
\begin{pgfscope}%
\pgfsys@transformshift{4.192353in}{4.005075in}%
\pgfsys@useobject{currentmarker}{}%
\end{pgfscope}%
\begin{pgfscope}%
\pgfsys@transformshift{4.228679in}{3.998986in}%
\pgfsys@useobject{currentmarker}{}%
\end{pgfscope}%
\begin{pgfscope}%
\pgfsys@transformshift{4.248851in}{3.997120in}%
\pgfsys@useobject{currentmarker}{}%
\end{pgfscope}%
\begin{pgfscope}%
\pgfsys@transformshift{4.271111in}{3.992198in}%
\pgfsys@useobject{currentmarker}{}%
\end{pgfscope}%
\begin{pgfscope}%
\pgfsys@transformshift{4.283361in}{3.989525in}%
\pgfsys@useobject{currentmarker}{}%
\end{pgfscope}%
\begin{pgfscope}%
\pgfsys@transformshift{4.294530in}{3.981518in}%
\pgfsys@useobject{currentmarker}{}%
\end{pgfscope}%
\begin{pgfscope}%
\pgfsys@transformshift{4.299927in}{3.976226in}%
\pgfsys@useobject{currentmarker}{}%
\end{pgfscope}%
\begin{pgfscope}%
\pgfsys@transformshift{4.302597in}{3.968314in}%
\pgfsys@useobject{currentmarker}{}%
\end{pgfscope}%
\begin{pgfscope}%
\pgfsys@transformshift{4.305716in}{3.959593in}%
\pgfsys@useobject{currentmarker}{}%
\end{pgfscope}%
\begin{pgfscope}%
\pgfsys@transformshift{4.305522in}{3.949095in}%
\pgfsys@useobject{currentmarker}{}%
\end{pgfscope}%
\begin{pgfscope}%
\pgfsys@transformshift{4.306636in}{3.937293in}%
\pgfsys@useobject{currentmarker}{}%
\end{pgfscope}%
\begin{pgfscope}%
\pgfsys@transformshift{4.304527in}{3.925074in}%
\pgfsys@useobject{currentmarker}{}%
\end{pgfscope}%
\begin{pgfscope}%
\pgfsys@transformshift{4.300578in}{3.910558in}%
\pgfsys@useobject{currentmarker}{}%
\end{pgfscope}%
\begin{pgfscope}%
\pgfsys@transformshift{4.298874in}{3.894423in}%
\pgfsys@useobject{currentmarker}{}%
\end{pgfscope}%
\begin{pgfscope}%
\pgfsys@transformshift{4.296761in}{3.877768in}%
\pgfsys@useobject{currentmarker}{}%
\end{pgfscope}%
\begin{pgfscope}%
\pgfsys@transformshift{4.296832in}{3.868535in}%
\pgfsys@useobject{currentmarker}{}%
\end{pgfscope}%
\begin{pgfscope}%
\pgfsys@transformshift{4.297479in}{3.858564in}%
\pgfsys@useobject{currentmarker}{}%
\end{pgfscope}%
\begin{pgfscope}%
\pgfsys@transformshift{4.294598in}{3.848333in}%
\pgfsys@useobject{currentmarker}{}%
\end{pgfscope}%
\begin{pgfscope}%
\pgfsys@transformshift{4.297643in}{3.837207in}%
\pgfsys@useobject{currentmarker}{}%
\end{pgfscope}%
\begin{pgfscope}%
\pgfsys@transformshift{4.299092in}{3.831031in}%
\pgfsys@useobject{currentmarker}{}%
\end{pgfscope}%
\begin{pgfscope}%
\pgfsys@transformshift{4.295425in}{3.820531in}%
\pgfsys@useobject{currentmarker}{}%
\end{pgfscope}%
\begin{pgfscope}%
\pgfsys@transformshift{4.298468in}{3.815225in}%
\pgfsys@useobject{currentmarker}{}%
\end{pgfscope}%
\begin{pgfscope}%
\pgfsys@transformshift{4.302183in}{3.805978in}%
\pgfsys@useobject{currentmarker}{}%
\end{pgfscope}%
\begin{pgfscope}%
\pgfsys@transformshift{4.300634in}{3.792840in}%
\pgfsys@useobject{currentmarker}{}%
\end{pgfscope}%
\begin{pgfscope}%
\pgfsys@transformshift{4.298320in}{3.778441in}%
\pgfsys@useobject{currentmarker}{}%
\end{pgfscope}%
\begin{pgfscope}%
\pgfsys@transformshift{4.302082in}{3.763864in}%
\pgfsys@useobject{currentmarker}{}%
\end{pgfscope}%
\begin{pgfscope}%
\pgfsys@transformshift{4.306147in}{3.747595in}%
\pgfsys@useobject{currentmarker}{}%
\end{pgfscope}%
\begin{pgfscope}%
\pgfsys@transformshift{4.304920in}{3.725533in}%
\pgfsys@useobject{currentmarker}{}%
\end{pgfscope}%
\begin{pgfscope}%
\pgfsys@transformshift{4.303401in}{3.713475in}%
\pgfsys@useobject{currentmarker}{}%
\end{pgfscope}%
\begin{pgfscope}%
\pgfsys@transformshift{4.305461in}{3.707117in}%
\pgfsys@useobject{currentmarker}{}%
\end{pgfscope}%
\begin{pgfscope}%
\pgfsys@transformshift{4.307802in}{3.698811in}%
\pgfsys@useobject{currentmarker}{}%
\end{pgfscope}%
\begin{pgfscope}%
\pgfsys@transformshift{4.305046in}{3.684873in}%
\pgfsys@useobject{currentmarker}{}%
\end{pgfscope}%
\begin{pgfscope}%
\pgfsys@transformshift{4.307259in}{3.677379in}%
\pgfsys@useobject{currentmarker}{}%
\end{pgfscope}%
\begin{pgfscope}%
\pgfsys@transformshift{4.306905in}{3.673096in}%
\pgfsys@useobject{currentmarker}{}%
\end{pgfscope}%
\begin{pgfscope}%
\pgfsys@transformshift{4.307949in}{3.664423in}%
\pgfsys@useobject{currentmarker}{}%
\end{pgfscope}%
\begin{pgfscope}%
\pgfsys@transformshift{4.305642in}{3.653929in}%
\pgfsys@useobject{currentmarker}{}%
\end{pgfscope}%
\begin{pgfscope}%
\pgfsys@transformshift{4.309617in}{3.641855in}%
\pgfsys@useobject{currentmarker}{}%
\end{pgfscope}%
\begin{pgfscope}%
\pgfsys@transformshift{4.315683in}{3.626838in}%
\pgfsys@useobject{currentmarker}{}%
\end{pgfscope}%
\begin{pgfscope}%
\pgfsys@transformshift{4.314611in}{3.606127in}%
\pgfsys@useobject{currentmarker}{}%
\end{pgfscope}%
\begin{pgfscope}%
\pgfsys@transformshift{4.318686in}{3.595474in}%
\pgfsys@useobject{currentmarker}{}%
\end{pgfscope}%
\begin{pgfscope}%
\pgfsys@transformshift{4.325277in}{3.584263in}%
\pgfsys@useobject{currentmarker}{}%
\end{pgfscope}%
\begin{pgfscope}%
\pgfsys@transformshift{4.325406in}{3.565434in}%
\pgfsys@useobject{currentmarker}{}%
\end{pgfscope}%
\begin{pgfscope}%
\pgfsys@transformshift{4.327110in}{3.546176in}%
\pgfsys@useobject{currentmarker}{}%
\end{pgfscope}%
\begin{pgfscope}%
\pgfsys@transformshift{4.331295in}{3.525234in}%
\pgfsys@useobject{currentmarker}{}%
\end{pgfscope}%
\begin{pgfscope}%
\pgfsys@transformshift{4.340988in}{3.504983in}%
\pgfsys@useobject{currentmarker}{}%
\end{pgfscope}%
\begin{pgfscope}%
\pgfsys@transformshift{4.339445in}{3.476555in}%
\pgfsys@useobject{currentmarker}{}%
\end{pgfscope}%
\begin{pgfscope}%
\pgfsys@transformshift{4.351838in}{3.450221in}%
\pgfsys@useobject{currentmarker}{}%
\end{pgfscope}%
\begin{pgfscope}%
\pgfsys@transformshift{4.362074in}{3.421943in}%
\pgfsys@useobject{currentmarker}{}%
\end{pgfscope}%
\begin{pgfscope}%
\pgfsys@transformshift{4.364448in}{3.386033in}%
\pgfsys@useobject{currentmarker}{}%
\end{pgfscope}%
\begin{pgfscope}%
\pgfsys@transformshift{4.364910in}{3.347489in}%
\pgfsys@useobject{currentmarker}{}%
\end{pgfscope}%
\begin{pgfscope}%
\pgfsys@transformshift{4.377442in}{3.306136in}%
\pgfsys@useobject{currentmarker}{}%
\end{pgfscope}%
\begin{pgfscope}%
\pgfsys@transformshift{4.369690in}{3.261083in}%
\pgfsys@useobject{currentmarker}{}%
\end{pgfscope}%
\begin{pgfscope}%
\pgfsys@transformshift{4.368203in}{3.213280in}%
\pgfsys@useobject{currentmarker}{}%
\end{pgfscope}%
\begin{pgfscope}%
\pgfsys@transformshift{4.390153in}{3.168381in}%
\pgfsys@useobject{currentmarker}{}%
\end{pgfscope}%
\begin{pgfscope}%
\pgfsys@transformshift{4.409983in}{3.120964in}%
\pgfsys@useobject{currentmarker}{}%
\end{pgfscope}%
\begin{pgfscope}%
\pgfsys@transformshift{4.402282in}{3.066175in}%
\pgfsys@useobject{currentmarker}{}%
\end{pgfscope}%
\begin{pgfscope}%
\pgfsys@transformshift{4.419274in}{3.012522in}%
\pgfsys@useobject{currentmarker}{}%
\end{pgfscope}%
\begin{pgfscope}%
\pgfsys@transformshift{4.448654in}{2.962151in}%
\pgfsys@useobject{currentmarker}{}%
\end{pgfscope}%
\begin{pgfscope}%
\pgfsys@transformshift{4.436551in}{2.899705in}%
\pgfsys@useobject{currentmarker}{}%
\end{pgfscope}%
\begin{pgfscope}%
\pgfsys@transformshift{4.437409in}{2.832800in}%
\pgfsys@useobject{currentmarker}{}%
\end{pgfscope}%
\begin{pgfscope}%
\pgfsys@transformshift{4.452934in}{2.766789in}%
\pgfsys@useobject{currentmarker}{}%
\end{pgfscope}%
\begin{pgfscope}%
\pgfsys@transformshift{4.475363in}{2.700285in}%
\pgfsys@useobject{currentmarker}{}%
\end{pgfscope}%
\begin{pgfscope}%
\pgfsys@transformshift{4.462465in}{2.626340in}%
\pgfsys@useobject{currentmarker}{}%
\end{pgfscope}%
\begin{pgfscope}%
\pgfsys@transformshift{4.501316in}{2.561515in}%
\pgfsys@useobject{currentmarker}{}%
\end{pgfscope}%
\begin{pgfscope}%
\pgfsys@transformshift{4.535700in}{2.492384in}%
\pgfsys@useobject{currentmarker}{}%
\end{pgfscope}%
\begin{pgfscope}%
\pgfsys@transformshift{4.539784in}{2.410638in}%
\pgfsys@useobject{currentmarker}{}%
\end{pgfscope}%
\begin{pgfscope}%
\pgfsys@transformshift{4.515779in}{2.327072in}%
\pgfsys@useobject{currentmarker}{}%
\end{pgfscope}%
\begin{pgfscope}%
\pgfsys@transformshift{4.544751in}{2.289028in}%
\pgfsys@useobject{currentmarker}{}%
\end{pgfscope}%
\begin{pgfscope}%
\pgfsys@transformshift{4.576654in}{2.249194in}%
\pgfsys@useobject{currentmarker}{}%
\end{pgfscope}%
\begin{pgfscope}%
\pgfsys@transformshift{4.577734in}{2.193625in}%
\pgfsys@useobject{currentmarker}{}%
\end{pgfscope}%
\begin{pgfscope}%
\pgfsys@transformshift{4.585634in}{2.164094in}%
\pgfsys@useobject{currentmarker}{}%
\end{pgfscope}%
\begin{pgfscope}%
\pgfsys@transformshift{4.588015in}{2.129798in}%
\pgfsys@useobject{currentmarker}{}%
\end{pgfscope}%
\begin{pgfscope}%
\pgfsys@transformshift{4.590570in}{2.111063in}%
\pgfsys@useobject{currentmarker}{}%
\end{pgfscope}%
\begin{pgfscope}%
\pgfsys@transformshift{4.587957in}{2.086909in}%
\pgfsys@useobject{currentmarker}{}%
\end{pgfscope}%
\begin{pgfscope}%
\pgfsys@transformshift{4.600516in}{2.064989in}%
\pgfsys@useobject{currentmarker}{}%
\end{pgfscope}%
\begin{pgfscope}%
\pgfsys@transformshift{4.614228in}{2.042132in}%
\pgfsys@useobject{currentmarker}{}%
\end{pgfscope}%
\begin{pgfscope}%
\pgfsys@transformshift{4.605166in}{2.011366in}%
\pgfsys@useobject{currentmarker}{}%
\end{pgfscope}%
\begin{pgfscope}%
\pgfsys@transformshift{4.596346in}{1.977206in}%
\pgfsys@useobject{currentmarker}{}%
\end{pgfscope}%
\begin{pgfscope}%
\pgfsys@transformshift{4.609604in}{1.943484in}%
\pgfsys@useobject{currentmarker}{}%
\end{pgfscope}%
\begin{pgfscope}%
\pgfsys@transformshift{4.627173in}{1.910528in}%
\pgfsys@useobject{currentmarker}{}%
\end{pgfscope}%
\begin{pgfscope}%
\pgfsys@transformshift{4.624092in}{1.869767in}%
\pgfsys@useobject{currentmarker}{}%
\end{pgfscope}%
\begin{pgfscope}%
\pgfsys@transformshift{4.629431in}{1.847928in}%
\pgfsys@useobject{currentmarker}{}%
\end{pgfscope}%
\begin{pgfscope}%
\pgfsys@transformshift{4.641815in}{1.824579in}%
\pgfsys@useobject{currentmarker}{}%
\end{pgfscope}%
\begin{pgfscope}%
\pgfsys@transformshift{4.639703in}{1.795459in}%
\pgfsys@useobject{currentmarker}{}%
\end{pgfscope}%
\begin{pgfscope}%
\pgfsys@transformshift{4.639641in}{1.764181in}%
\pgfsys@useobject{currentmarker}{}%
\end{pgfscope}%
\begin{pgfscope}%
\pgfsys@transformshift{4.645788in}{1.725956in}%
\pgfsys@useobject{currentmarker}{}%
\end{pgfscope}%
\begin{pgfscope}%
\pgfsys@transformshift{4.648361in}{1.704819in}%
\pgfsys@useobject{currentmarker}{}%
\end{pgfscope}%
\begin{pgfscope}%
\pgfsys@transformshift{4.644517in}{1.679581in}%
\pgfsys@useobject{currentmarker}{}%
\end{pgfscope}%
\begin{pgfscope}%
\pgfsys@transformshift{4.661597in}{1.659936in}%
\pgfsys@useobject{currentmarker}{}%
\end{pgfscope}%
\begin{pgfscope}%
\pgfsys@transformshift{4.678050in}{1.636234in}%
\pgfsys@useobject{currentmarker}{}%
\end{pgfscope}%
\begin{pgfscope}%
\pgfsys@transformshift{4.684630in}{1.602585in}%
\pgfsys@useobject{currentmarker}{}%
\end{pgfscope}%
\begin{pgfscope}%
\pgfsys@transformshift{4.688669in}{1.584164in}%
\pgfsys@useobject{currentmarker}{}%
\end{pgfscope}%
\begin{pgfscope}%
\pgfsys@transformshift{4.698661in}{1.564025in}%
\pgfsys@useobject{currentmarker}{}%
\end{pgfscope}%
\begin{pgfscope}%
\pgfsys@transformshift{4.698104in}{1.539548in}%
\pgfsys@useobject{currentmarker}{}%
\end{pgfscope}%
\begin{pgfscope}%
\pgfsys@transformshift{4.694065in}{1.512417in}%
\pgfsys@useobject{currentmarker}{}%
\end{pgfscope}%
\begin{pgfscope}%
\pgfsys@transformshift{4.704736in}{1.484114in}%
\pgfsys@useobject{currentmarker}{}%
\end{pgfscope}%
\begin{pgfscope}%
\pgfsys@transformshift{4.718989in}{1.454967in}%
\pgfsys@useobject{currentmarker}{}%
\end{pgfscope}%
\begin{pgfscope}%
\pgfsys@transformshift{4.717694in}{1.419440in}%
\pgfsys@useobject{currentmarker}{}%
\end{pgfscope}%
\begin{pgfscope}%
\pgfsys@transformshift{4.724847in}{1.383798in}%
\pgfsys@useobject{currentmarker}{}%
\end{pgfscope}%
\begin{pgfscope}%
\pgfsys@transformshift{4.742165in}{1.350872in}%
\pgfsys@useobject{currentmarker}{}%
\end{pgfscope}%
\begin{pgfscope}%
\pgfsys@transformshift{4.742139in}{1.309581in}%
\pgfsys@useobject{currentmarker}{}%
\end{pgfscope}%
\begin{pgfscope}%
\pgfsys@transformshift{4.749171in}{1.267691in}%
\pgfsys@useobject{currentmarker}{}%
\end{pgfscope}%
\begin{pgfscope}%
\pgfsys@transformshift{4.755854in}{1.225133in}%
\pgfsys@useobject{currentmarker}{}%
\end{pgfscope}%
\begin{pgfscope}%
\pgfsys@transformshift{4.764697in}{1.182440in}%
\pgfsys@useobject{currentmarker}{}%
\end{pgfscope}%
\begin{pgfscope}%
\pgfsys@transformshift{4.767403in}{1.158614in}%
\pgfsys@useobject{currentmarker}{}%
\end{pgfscope}%
\begin{pgfscope}%
\pgfsys@transformshift{4.770547in}{1.134358in}%
\pgfsys@useobject{currentmarker}{}%
\end{pgfscope}%
\begin{pgfscope}%
\pgfsys@transformshift{4.774165in}{1.109661in}%
\pgfsys@useobject{currentmarker}{}%
\end{pgfscope}%
\begin{pgfscope}%
\pgfsys@transformshift{4.776954in}{1.084395in}%
\pgfsys@useobject{currentmarker}{}%
\end{pgfscope}%
\begin{pgfscope}%
\pgfsys@transformshift{4.772974in}{1.070992in}%
\pgfsys@useobject{currentmarker}{}%
\end{pgfscope}%
\begin{pgfscope}%
\pgfsys@transformshift{4.756023in}{1.060980in}%
\pgfsys@useobject{currentmarker}{}%
\end{pgfscope}%
\begin{pgfscope}%
\pgfsys@transformshift{4.735152in}{1.060709in}%
\pgfsys@useobject{currentmarker}{}%
\end{pgfscope}%
\begin{pgfscope}%
\pgfsys@transformshift{4.713402in}{1.060393in}%
\pgfsys@useobject{currentmarker}{}%
\end{pgfscope}%
\begin{pgfscope}%
\pgfsys@transformshift{4.688192in}{1.065166in}%
\pgfsys@useobject{currentmarker}{}%
\end{pgfscope}%
\begin{pgfscope}%
\pgfsys@transformshift{4.661750in}{1.067419in}%
\pgfsys@useobject{currentmarker}{}%
\end{pgfscope}%
\begin{pgfscope}%
\pgfsys@transformshift{4.647506in}{1.070606in}%
\pgfsys@useobject{currentmarker}{}%
\end{pgfscope}%
\begin{pgfscope}%
\pgfsys@transformshift{4.631810in}{1.070354in}%
\pgfsys@useobject{currentmarker}{}%
\end{pgfscope}%
\begin{pgfscope}%
\pgfsys@transformshift{4.616658in}{1.076218in}%
\pgfsys@useobject{currentmarker}{}%
\end{pgfscope}%
\begin{pgfscope}%
\pgfsys@transformshift{4.599534in}{1.079132in}%
\pgfsys@useobject{currentmarker}{}%
\end{pgfscope}%
\begin{pgfscope}%
\pgfsys@transformshift{4.582491in}{1.085944in}%
\pgfsys@useobject{currentmarker}{}%
\end{pgfscope}%
\begin{pgfscope}%
\pgfsys@transformshift{4.563514in}{1.088252in}%
\pgfsys@useobject{currentmarker}{}%
\end{pgfscope}%
\begin{pgfscope}%
\pgfsys@transformshift{4.554119in}{1.092972in}%
\pgfsys@useobject{currentmarker}{}%
\end{pgfscope}%
\begin{pgfscope}%
\pgfsys@transformshift{4.542698in}{1.093499in}%
\pgfsys@useobject{currentmarker}{}%
\end{pgfscope}%
\begin{pgfscope}%
\pgfsys@transformshift{4.531705in}{1.098734in}%
\pgfsys@useobject{currentmarker}{}%
\end{pgfscope}%
\begin{pgfscope}%
\pgfsys@transformshift{4.518513in}{1.100594in}%
\pgfsys@useobject{currentmarker}{}%
\end{pgfscope}%
\begin{pgfscope}%
\pgfsys@transformshift{4.511203in}{1.101097in}%
\pgfsys@useobject{currentmarker}{}%
\end{pgfscope}%
\begin{pgfscope}%
\pgfsys@transformshift{4.500966in}{1.098920in}%
\pgfsys@useobject{currentmarker}{}%
\end{pgfscope}%
\begin{pgfscope}%
\pgfsys@transformshift{4.495246in}{1.099565in}%
\pgfsys@useobject{currentmarker}{}%
\end{pgfscope}%
\begin{pgfscope}%
\pgfsys@transformshift{4.481552in}{1.099129in}%
\pgfsys@useobject{currentmarker}{}%
\end{pgfscope}%
\begin{pgfscope}%
\pgfsys@transformshift{4.466164in}{1.098030in}%
\pgfsys@useobject{currentmarker}{}%
\end{pgfscope}%
\begin{pgfscope}%
\pgfsys@transformshift{4.449387in}{1.094520in}%
\pgfsys@useobject{currentmarker}{}%
\end{pgfscope}%
\begin{pgfscope}%
\pgfsys@transformshift{4.430787in}{1.094203in}%
\pgfsys@useobject{currentmarker}{}%
\end{pgfscope}%
\begin{pgfscope}%
\pgfsys@transformshift{4.404852in}{1.092196in}%
\pgfsys@useobject{currentmarker}{}%
\end{pgfscope}%
\begin{pgfscope}%
\pgfsys@transformshift{4.390571in}{1.093051in}%
\pgfsys@useobject{currentmarker}{}%
\end{pgfscope}%
\begin{pgfscope}%
\pgfsys@transformshift{4.376243in}{1.085570in}%
\pgfsys@useobject{currentmarker}{}%
\end{pgfscope}%
\begin{pgfscope}%
\pgfsys@transformshift{4.367441in}{1.084322in}%
\pgfsys@useobject{currentmarker}{}%
\end{pgfscope}%
\begin{pgfscope}%
\pgfsys@transformshift{4.349905in}{1.089520in}%
\pgfsys@useobject{currentmarker}{}%
\end{pgfscope}%
\begin{pgfscope}%
\pgfsys@transformshift{4.328618in}{1.089753in}%
\pgfsys@useobject{currentmarker}{}%
\end{pgfscope}%
\begin{pgfscope}%
\pgfsys@transformshift{4.308049in}{1.080022in}%
\pgfsys@useobject{currentmarker}{}%
\end{pgfscope}%
\begin{pgfscope}%
\pgfsys@transformshift{4.284254in}{1.079554in}%
\pgfsys@useobject{currentmarker}{}%
\end{pgfscope}%
\begin{pgfscope}%
\pgfsys@transformshift{4.253722in}{1.082771in}%
\pgfsys@useobject{currentmarker}{}%
\end{pgfscope}%
\begin{pgfscope}%
\pgfsys@transformshift{4.220290in}{1.081029in}%
\pgfsys@useobject{currentmarker}{}%
\end{pgfscope}%
\begin{pgfscope}%
\pgfsys@transformshift{4.184289in}{1.070441in}%
\pgfsys@useobject{currentmarker}{}%
\end{pgfscope}%
\begin{pgfscope}%
\pgfsys@transformshift{4.144199in}{1.066159in}%
\pgfsys@useobject{currentmarker}{}%
\end{pgfscope}%
\begin{pgfscope}%
\pgfsys@transformshift{4.100687in}{1.064405in}%
\pgfsys@useobject{currentmarker}{}%
\end{pgfscope}%
\begin{pgfscope}%
\pgfsys@transformshift{4.054655in}{1.063496in}%
\pgfsys@useobject{currentmarker}{}%
\end{pgfscope}%
\begin{pgfscope}%
\pgfsys@transformshift{4.006934in}{1.047499in}%
\pgfsys@useobject{currentmarker}{}%
\end{pgfscope}%
\begin{pgfscope}%
\pgfsys@transformshift{3.953775in}{1.039550in}%
\pgfsys@useobject{currentmarker}{}%
\end{pgfscope}%
\begin{pgfscope}%
\pgfsys@transformshift{3.899543in}{1.033151in}%
\pgfsys@useobject{currentmarker}{}%
\end{pgfscope}%
\begin{pgfscope}%
\pgfsys@transformshift{3.844055in}{1.034965in}%
\pgfsys@useobject{currentmarker}{}%
\end{pgfscope}%
\begin{pgfscope}%
\pgfsys@transformshift{3.787418in}{1.031632in}%
\pgfsys@useobject{currentmarker}{}%
\end{pgfscope}%
\begin{pgfscope}%
\pgfsys@transformshift{3.731911in}{1.003711in}%
\pgfsys@useobject{currentmarker}{}%
\end{pgfscope}%
\begin{pgfscope}%
\pgfsys@transformshift{3.667140in}{0.986254in}%
\pgfsys@useobject{currentmarker}{}%
\end{pgfscope}%
\begin{pgfscope}%
\pgfsys@transformshift{3.599731in}{0.978951in}%
\pgfsys@useobject{currentmarker}{}%
\end{pgfscope}%
\begin{pgfscope}%
\pgfsys@transformshift{3.527835in}{0.973519in}%
\pgfsys@useobject{currentmarker}{}%
\end{pgfscope}%
\begin{pgfscope}%
\pgfsys@transformshift{3.452075in}{0.969251in}%
\pgfsys@useobject{currentmarker}{}%
\end{pgfscope}%
\begin{pgfscope}%
\pgfsys@transformshift{3.374985in}{0.949909in}%
\pgfsys@useobject{currentmarker}{}%
\end{pgfscope}%
\begin{pgfscope}%
\pgfsys@transformshift{3.294731in}{0.945013in}%
\pgfsys@useobject{currentmarker}{}%
\end{pgfscope}%
\begin{pgfscope}%
\pgfsys@transformshift{3.209567in}{0.935570in}%
\pgfsys@useobject{currentmarker}{}%
\end{pgfscope}%
\begin{pgfscope}%
\pgfsys@transformshift{3.126409in}{0.903425in}%
\pgfsys@useobject{currentmarker}{}%
\end{pgfscope}%
\begin{pgfscope}%
\pgfsys@transformshift{3.034941in}{0.899089in}%
\pgfsys@useobject{currentmarker}{}%
\end{pgfscope}%
\begin{pgfscope}%
\pgfsys@transformshift{2.942780in}{0.890204in}%
\pgfsys@useobject{currentmarker}{}%
\end{pgfscope}%
\begin{pgfscope}%
\pgfsys@transformshift{2.845619in}{0.892690in}%
\pgfsys@useobject{currentmarker}{}%
\end{pgfscope}%
\begin{pgfscope}%
\pgfsys@transformshift{2.753971in}{0.855043in}%
\pgfsys@useobject{currentmarker}{}%
\end{pgfscope}%
\begin{pgfscope}%
\pgfsys@transformshift{2.651847in}{0.850738in}%
\pgfsys@useobject{currentmarker}{}%
\end{pgfscope}%
\begin{pgfscope}%
\pgfsys@transformshift{2.547764in}{0.842944in}%
\pgfsys@useobject{currentmarker}{}%
\end{pgfscope}%
\begin{pgfscope}%
\pgfsys@transformshift{2.440766in}{0.835383in}%
\pgfsys@useobject{currentmarker}{}%
\end{pgfscope}%
\begin{pgfscope}%
\pgfsys@transformshift{2.335751in}{0.798622in}%
\pgfsys@useobject{currentmarker}{}%
\end{pgfscope}%
\begin{pgfscope}%
\pgfsys@transformshift{2.222003in}{0.789860in}%
\pgfsys@useobject{currentmarker}{}%
\end{pgfscope}%
\begin{pgfscope}%
\pgfsys@transformshift{2.102536in}{0.797856in}%
\pgfsys@useobject{currentmarker}{}%
\end{pgfscope}%
\begin{pgfscope}%
\pgfsys@transformshift{1.981797in}{0.801437in}%
\pgfsys@useobject{currentmarker}{}%
\end{pgfscope}%
\begin{pgfscope}%
\pgfsys@transformshift{1.861984in}{0.770587in}%
\pgfsys@useobject{currentmarker}{}%
\end{pgfscope}%
\begin{pgfscope}%
\pgfsys@transformshift{1.736395in}{0.778906in}%
\pgfsys@useobject{currentmarker}{}%
\end{pgfscope}%
\begin{pgfscope}%
\pgfsys@transformshift{1.606360in}{0.788880in}%
\pgfsys@useobject{currentmarker}{}%
\end{pgfscope}%
\begin{pgfscope}%
\pgfsys@transformshift{1.476250in}{0.809773in}%
\pgfsys@useobject{currentmarker}{}%
\end{pgfscope}%
\begin{pgfscope}%
\pgfsys@transformshift{1.344521in}{0.828357in}%
\pgfsys@useobject{currentmarker}{}%
\end{pgfscope}%
\begin{pgfscope}%
\pgfsys@transformshift{1.272667in}{0.842163in}%
\pgfsys@useobject{currentmarker}{}%
\end{pgfscope}%
\begin{pgfscope}%
\pgfsys@transformshift{1.290814in}{0.846316in}%
\pgfsys@useobject{currentmarker}{}%
\end{pgfscope}%
\begin{pgfscope}%
\pgfsys@transformshift{1.353341in}{0.890229in}%
\pgfsys@useobject{currentmarker}{}%
\end{pgfscope}%
\begin{pgfscope}%
\pgfsys@transformshift{1.424752in}{0.922581in}%
\pgfsys@useobject{currentmarker}{}%
\end{pgfscope}%
\begin{pgfscope}%
\pgfsys@transformshift{1.495226in}{0.961804in}%
\pgfsys@useobject{currentmarker}{}%
\end{pgfscope}%
\begin{pgfscope}%
\pgfsys@transformshift{1.525121in}{0.994578in}%
\pgfsys@useobject{currentmarker}{}%
\end{pgfscope}%
\begin{pgfscope}%
\pgfsys@transformshift{1.558665in}{1.024968in}%
\pgfsys@useobject{currentmarker}{}%
\end{pgfscope}%
\begin{pgfscope}%
\pgfsys@transformshift{1.593300in}{1.056347in}%
\pgfsys@useobject{currentmarker}{}%
\end{pgfscope}%
\end{pgfscope}%
\begin{pgfscope}%
\pgfsetbuttcap%
\pgfsetroundjoin%
\definecolor{currentfill}{rgb}{0.000000,0.000000,0.000000}%
\pgfsetfillcolor{currentfill}%
\pgfsetlinewidth{0.803000pt}%
\definecolor{currentstroke}{rgb}{0.000000,0.000000,0.000000}%
\pgfsetstrokecolor{currentstroke}%
\pgfsetdash{}{0pt}%
\pgfsys@defobject{currentmarker}{\pgfqpoint{0.000000in}{-0.048611in}}{\pgfqpoint{0.000000in}{0.000000in}}{%
\pgfpathmoveto{\pgfqpoint{0.000000in}{0.000000in}}%
\pgfpathlineto{\pgfqpoint{0.000000in}{-0.048611in}}%
\pgfusepath{stroke,fill}%
}%
\begin{pgfscope}%
\pgfsys@transformshift{1.244158in}{0.515000in}%
\pgfsys@useobject{currentmarker}{}%
\end{pgfscope}%
\end{pgfscope}%
\begin{pgfscope}%
\definecolor{textcolor}{rgb}{0.000000,0.000000,0.000000}%
\pgfsetstrokecolor{textcolor}%
\pgfsetfillcolor{textcolor}%
\pgftext[x=1.244158in,y=0.417777in,,top]{\color{textcolor}\rmfamily\fontsize{10.000000}{12.000000}\selectfont \(\displaystyle {0}\)}%
\end{pgfscope}%
\begin{pgfscope}%
\pgfsetbuttcap%
\pgfsetroundjoin%
\definecolor{currentfill}{rgb}{0.000000,0.000000,0.000000}%
\pgfsetfillcolor{currentfill}%
\pgfsetlinewidth{0.803000pt}%
\definecolor{currentstroke}{rgb}{0.000000,0.000000,0.000000}%
\pgfsetstrokecolor{currentstroke}%
\pgfsetdash{}{0pt}%
\pgfsys@defobject{currentmarker}{\pgfqpoint{0.000000in}{-0.048611in}}{\pgfqpoint{0.000000in}{0.000000in}}{%
\pgfpathmoveto{\pgfqpoint{0.000000in}{0.000000in}}%
\pgfpathlineto{\pgfqpoint{0.000000in}{-0.048611in}}%
\pgfusepath{stroke,fill}%
}%
\begin{pgfscope}%
\pgfsys@transformshift{2.151020in}{0.515000in}%
\pgfsys@useobject{currentmarker}{}%
\end{pgfscope}%
\end{pgfscope}%
\begin{pgfscope}%
\definecolor{textcolor}{rgb}{0.000000,0.000000,0.000000}%
\pgfsetstrokecolor{textcolor}%
\pgfsetfillcolor{textcolor}%
\pgftext[x=2.151020in,y=0.417777in,,top]{\color{textcolor}\rmfamily\fontsize{10.000000}{12.000000}\selectfont \(\displaystyle {10}\)}%
\end{pgfscope}%
\begin{pgfscope}%
\pgfsetbuttcap%
\pgfsetroundjoin%
\definecolor{currentfill}{rgb}{0.000000,0.000000,0.000000}%
\pgfsetfillcolor{currentfill}%
\pgfsetlinewidth{0.803000pt}%
\definecolor{currentstroke}{rgb}{0.000000,0.000000,0.000000}%
\pgfsetstrokecolor{currentstroke}%
\pgfsetdash{}{0pt}%
\pgfsys@defobject{currentmarker}{\pgfqpoint{0.000000in}{-0.048611in}}{\pgfqpoint{0.000000in}{0.000000in}}{%
\pgfpathmoveto{\pgfqpoint{0.000000in}{0.000000in}}%
\pgfpathlineto{\pgfqpoint{0.000000in}{-0.048611in}}%
\pgfusepath{stroke,fill}%
}%
\begin{pgfscope}%
\pgfsys@transformshift{3.057882in}{0.515000in}%
\pgfsys@useobject{currentmarker}{}%
\end{pgfscope}%
\end{pgfscope}%
\begin{pgfscope}%
\definecolor{textcolor}{rgb}{0.000000,0.000000,0.000000}%
\pgfsetstrokecolor{textcolor}%
\pgfsetfillcolor{textcolor}%
\pgftext[x=3.057882in,y=0.417777in,,top]{\color{textcolor}\rmfamily\fontsize{10.000000}{12.000000}\selectfont \(\displaystyle {20}\)}%
\end{pgfscope}%
\begin{pgfscope}%
\pgfsetbuttcap%
\pgfsetroundjoin%
\definecolor{currentfill}{rgb}{0.000000,0.000000,0.000000}%
\pgfsetfillcolor{currentfill}%
\pgfsetlinewidth{0.803000pt}%
\definecolor{currentstroke}{rgb}{0.000000,0.000000,0.000000}%
\pgfsetstrokecolor{currentstroke}%
\pgfsetdash{}{0pt}%
\pgfsys@defobject{currentmarker}{\pgfqpoint{0.000000in}{-0.048611in}}{\pgfqpoint{0.000000in}{0.000000in}}{%
\pgfpathmoveto{\pgfqpoint{0.000000in}{0.000000in}}%
\pgfpathlineto{\pgfqpoint{0.000000in}{-0.048611in}}%
\pgfusepath{stroke,fill}%
}%
\begin{pgfscope}%
\pgfsys@transformshift{3.964744in}{0.515000in}%
\pgfsys@useobject{currentmarker}{}%
\end{pgfscope}%
\end{pgfscope}%
\begin{pgfscope}%
\definecolor{textcolor}{rgb}{0.000000,0.000000,0.000000}%
\pgfsetstrokecolor{textcolor}%
\pgfsetfillcolor{textcolor}%
\pgftext[x=3.964744in,y=0.417777in,,top]{\color{textcolor}\rmfamily\fontsize{10.000000}{12.000000}\selectfont \(\displaystyle {30}\)}%
\end{pgfscope}%
\begin{pgfscope}%
\pgfsetbuttcap%
\pgfsetroundjoin%
\definecolor{currentfill}{rgb}{0.000000,0.000000,0.000000}%
\pgfsetfillcolor{currentfill}%
\pgfsetlinewidth{0.803000pt}%
\definecolor{currentstroke}{rgb}{0.000000,0.000000,0.000000}%
\pgfsetstrokecolor{currentstroke}%
\pgfsetdash{}{0pt}%
\pgfsys@defobject{currentmarker}{\pgfqpoint{0.000000in}{-0.048611in}}{\pgfqpoint{0.000000in}{0.000000in}}{%
\pgfpathmoveto{\pgfqpoint{0.000000in}{0.000000in}}%
\pgfpathlineto{\pgfqpoint{0.000000in}{-0.048611in}}%
\pgfusepath{stroke,fill}%
}%
\begin{pgfscope}%
\pgfsys@transformshift{4.871606in}{0.515000in}%
\pgfsys@useobject{currentmarker}{}%
\end{pgfscope}%
\end{pgfscope}%
\begin{pgfscope}%
\definecolor{textcolor}{rgb}{0.000000,0.000000,0.000000}%
\pgfsetstrokecolor{textcolor}%
\pgfsetfillcolor{textcolor}%
\pgftext[x=4.871606in,y=0.417777in,,top]{\color{textcolor}\rmfamily\fontsize{10.000000}{12.000000}\selectfont \(\displaystyle {40}\)}%
\end{pgfscope}%
\begin{pgfscope}%
\definecolor{textcolor}{rgb}{0.000000,0.000000,0.000000}%
\pgfsetstrokecolor{textcolor}%
\pgfsetfillcolor{textcolor}%
\pgftext[x=3.010556in,y=0.238889in,,top]{\color{textcolor}\rmfamily\fontsize{10.000000}{12.000000}\selectfont Position X [\(\displaystyle m\)]}%
\end{pgfscope}%
\begin{pgfscope}%
\pgfsetbuttcap%
\pgfsetroundjoin%
\definecolor{currentfill}{rgb}{0.000000,0.000000,0.000000}%
\pgfsetfillcolor{currentfill}%
\pgfsetlinewidth{0.803000pt}%
\definecolor{currentstroke}{rgb}{0.000000,0.000000,0.000000}%
\pgfsetstrokecolor{currentstroke}%
\pgfsetdash{}{0pt}%
\pgfsys@defobject{currentmarker}{\pgfqpoint{-0.048611in}{0.000000in}}{\pgfqpoint{-0.000000in}{0.000000in}}{%
\pgfpathmoveto{\pgfqpoint{-0.000000in}{0.000000in}}%
\pgfpathlineto{\pgfqpoint{-0.048611in}{0.000000in}}%
\pgfusepath{stroke,fill}%
}%
\begin{pgfscope}%
\pgfsys@transformshift{0.530556in}{0.884903in}%
\pgfsys@useobject{currentmarker}{}%
\end{pgfscope}%
\end{pgfscope}%
\begin{pgfscope}%
\definecolor{textcolor}{rgb}{0.000000,0.000000,0.000000}%
\pgfsetstrokecolor{textcolor}%
\pgfsetfillcolor{textcolor}%
\pgftext[x=0.363889in, y=0.836709in, left, base]{\color{textcolor}\rmfamily\fontsize{10.000000}{12.000000}\selectfont \(\displaystyle {0}\)}%
\end{pgfscope}%
\begin{pgfscope}%
\pgfsetbuttcap%
\pgfsetroundjoin%
\definecolor{currentfill}{rgb}{0.000000,0.000000,0.000000}%
\pgfsetfillcolor{currentfill}%
\pgfsetlinewidth{0.803000pt}%
\definecolor{currentstroke}{rgb}{0.000000,0.000000,0.000000}%
\pgfsetstrokecolor{currentstroke}%
\pgfsetdash{}{0pt}%
\pgfsys@defobject{currentmarker}{\pgfqpoint{-0.048611in}{0.000000in}}{\pgfqpoint{-0.000000in}{0.000000in}}{%
\pgfpathmoveto{\pgfqpoint{-0.000000in}{0.000000in}}%
\pgfpathlineto{\pgfqpoint{-0.048611in}{0.000000in}}%
\pgfusepath{stroke,fill}%
}%
\begin{pgfscope}%
\pgfsys@transformshift{0.530556in}{1.338334in}%
\pgfsys@useobject{currentmarker}{}%
\end{pgfscope}%
\end{pgfscope}%
\begin{pgfscope}%
\definecolor{textcolor}{rgb}{0.000000,0.000000,0.000000}%
\pgfsetstrokecolor{textcolor}%
\pgfsetfillcolor{textcolor}%
\pgftext[x=0.363889in, y=1.290140in, left, base]{\color{textcolor}\rmfamily\fontsize{10.000000}{12.000000}\selectfont \(\displaystyle {5}\)}%
\end{pgfscope}%
\begin{pgfscope}%
\pgfsetbuttcap%
\pgfsetroundjoin%
\definecolor{currentfill}{rgb}{0.000000,0.000000,0.000000}%
\pgfsetfillcolor{currentfill}%
\pgfsetlinewidth{0.803000pt}%
\definecolor{currentstroke}{rgb}{0.000000,0.000000,0.000000}%
\pgfsetstrokecolor{currentstroke}%
\pgfsetdash{}{0pt}%
\pgfsys@defobject{currentmarker}{\pgfqpoint{-0.048611in}{0.000000in}}{\pgfqpoint{-0.000000in}{0.000000in}}{%
\pgfpathmoveto{\pgfqpoint{-0.000000in}{0.000000in}}%
\pgfpathlineto{\pgfqpoint{-0.048611in}{0.000000in}}%
\pgfusepath{stroke,fill}%
}%
\begin{pgfscope}%
\pgfsys@transformshift{0.530556in}{1.791765in}%
\pgfsys@useobject{currentmarker}{}%
\end{pgfscope}%
\end{pgfscope}%
\begin{pgfscope}%
\definecolor{textcolor}{rgb}{0.000000,0.000000,0.000000}%
\pgfsetstrokecolor{textcolor}%
\pgfsetfillcolor{textcolor}%
\pgftext[x=0.294444in, y=1.743571in, left, base]{\color{textcolor}\rmfamily\fontsize{10.000000}{12.000000}\selectfont \(\displaystyle {10}\)}%
\end{pgfscope}%
\begin{pgfscope}%
\pgfsetbuttcap%
\pgfsetroundjoin%
\definecolor{currentfill}{rgb}{0.000000,0.000000,0.000000}%
\pgfsetfillcolor{currentfill}%
\pgfsetlinewidth{0.803000pt}%
\definecolor{currentstroke}{rgb}{0.000000,0.000000,0.000000}%
\pgfsetstrokecolor{currentstroke}%
\pgfsetdash{}{0pt}%
\pgfsys@defobject{currentmarker}{\pgfqpoint{-0.048611in}{0.000000in}}{\pgfqpoint{-0.000000in}{0.000000in}}{%
\pgfpathmoveto{\pgfqpoint{-0.000000in}{0.000000in}}%
\pgfpathlineto{\pgfqpoint{-0.048611in}{0.000000in}}%
\pgfusepath{stroke,fill}%
}%
\begin{pgfscope}%
\pgfsys@transformshift{0.530556in}{2.245196in}%
\pgfsys@useobject{currentmarker}{}%
\end{pgfscope}%
\end{pgfscope}%
\begin{pgfscope}%
\definecolor{textcolor}{rgb}{0.000000,0.000000,0.000000}%
\pgfsetstrokecolor{textcolor}%
\pgfsetfillcolor{textcolor}%
\pgftext[x=0.294444in, y=2.197002in, left, base]{\color{textcolor}\rmfamily\fontsize{10.000000}{12.000000}\selectfont \(\displaystyle {15}\)}%
\end{pgfscope}%
\begin{pgfscope}%
\pgfsetbuttcap%
\pgfsetroundjoin%
\definecolor{currentfill}{rgb}{0.000000,0.000000,0.000000}%
\pgfsetfillcolor{currentfill}%
\pgfsetlinewidth{0.803000pt}%
\definecolor{currentstroke}{rgb}{0.000000,0.000000,0.000000}%
\pgfsetstrokecolor{currentstroke}%
\pgfsetdash{}{0pt}%
\pgfsys@defobject{currentmarker}{\pgfqpoint{-0.048611in}{0.000000in}}{\pgfqpoint{-0.000000in}{0.000000in}}{%
\pgfpathmoveto{\pgfqpoint{-0.000000in}{0.000000in}}%
\pgfpathlineto{\pgfqpoint{-0.048611in}{0.000000in}}%
\pgfusepath{stroke,fill}%
}%
\begin{pgfscope}%
\pgfsys@transformshift{0.530556in}{2.698627in}%
\pgfsys@useobject{currentmarker}{}%
\end{pgfscope}%
\end{pgfscope}%
\begin{pgfscope}%
\definecolor{textcolor}{rgb}{0.000000,0.000000,0.000000}%
\pgfsetstrokecolor{textcolor}%
\pgfsetfillcolor{textcolor}%
\pgftext[x=0.294444in, y=2.650433in, left, base]{\color{textcolor}\rmfamily\fontsize{10.000000}{12.000000}\selectfont \(\displaystyle {20}\)}%
\end{pgfscope}%
\begin{pgfscope}%
\pgfsetbuttcap%
\pgfsetroundjoin%
\definecolor{currentfill}{rgb}{0.000000,0.000000,0.000000}%
\pgfsetfillcolor{currentfill}%
\pgfsetlinewidth{0.803000pt}%
\definecolor{currentstroke}{rgb}{0.000000,0.000000,0.000000}%
\pgfsetstrokecolor{currentstroke}%
\pgfsetdash{}{0pt}%
\pgfsys@defobject{currentmarker}{\pgfqpoint{-0.048611in}{0.000000in}}{\pgfqpoint{-0.000000in}{0.000000in}}{%
\pgfpathmoveto{\pgfqpoint{-0.000000in}{0.000000in}}%
\pgfpathlineto{\pgfqpoint{-0.048611in}{0.000000in}}%
\pgfusepath{stroke,fill}%
}%
\begin{pgfscope}%
\pgfsys@transformshift{0.530556in}{3.152059in}%
\pgfsys@useobject{currentmarker}{}%
\end{pgfscope}%
\end{pgfscope}%
\begin{pgfscope}%
\definecolor{textcolor}{rgb}{0.000000,0.000000,0.000000}%
\pgfsetstrokecolor{textcolor}%
\pgfsetfillcolor{textcolor}%
\pgftext[x=0.294444in, y=3.103864in, left, base]{\color{textcolor}\rmfamily\fontsize{10.000000}{12.000000}\selectfont \(\displaystyle {25}\)}%
\end{pgfscope}%
\begin{pgfscope}%
\pgfsetbuttcap%
\pgfsetroundjoin%
\definecolor{currentfill}{rgb}{0.000000,0.000000,0.000000}%
\pgfsetfillcolor{currentfill}%
\pgfsetlinewidth{0.803000pt}%
\definecolor{currentstroke}{rgb}{0.000000,0.000000,0.000000}%
\pgfsetstrokecolor{currentstroke}%
\pgfsetdash{}{0pt}%
\pgfsys@defobject{currentmarker}{\pgfqpoint{-0.048611in}{0.000000in}}{\pgfqpoint{-0.000000in}{0.000000in}}{%
\pgfpathmoveto{\pgfqpoint{-0.000000in}{0.000000in}}%
\pgfpathlineto{\pgfqpoint{-0.048611in}{0.000000in}}%
\pgfusepath{stroke,fill}%
}%
\begin{pgfscope}%
\pgfsys@transformshift{0.530556in}{3.605490in}%
\pgfsys@useobject{currentmarker}{}%
\end{pgfscope}%
\end{pgfscope}%
\begin{pgfscope}%
\definecolor{textcolor}{rgb}{0.000000,0.000000,0.000000}%
\pgfsetstrokecolor{textcolor}%
\pgfsetfillcolor{textcolor}%
\pgftext[x=0.294444in, y=3.557295in, left, base]{\color{textcolor}\rmfamily\fontsize{10.000000}{12.000000}\selectfont \(\displaystyle {30}\)}%
\end{pgfscope}%
\begin{pgfscope}%
\pgfsetbuttcap%
\pgfsetroundjoin%
\definecolor{currentfill}{rgb}{0.000000,0.000000,0.000000}%
\pgfsetfillcolor{currentfill}%
\pgfsetlinewidth{0.803000pt}%
\definecolor{currentstroke}{rgb}{0.000000,0.000000,0.000000}%
\pgfsetstrokecolor{currentstroke}%
\pgfsetdash{}{0pt}%
\pgfsys@defobject{currentmarker}{\pgfqpoint{-0.048611in}{0.000000in}}{\pgfqpoint{-0.000000in}{0.000000in}}{%
\pgfpathmoveto{\pgfqpoint{-0.000000in}{0.000000in}}%
\pgfpathlineto{\pgfqpoint{-0.048611in}{0.000000in}}%
\pgfusepath{stroke,fill}%
}%
\begin{pgfscope}%
\pgfsys@transformshift{0.530556in}{4.058921in}%
\pgfsys@useobject{currentmarker}{}%
\end{pgfscope}%
\end{pgfscope}%
\begin{pgfscope}%
\definecolor{textcolor}{rgb}{0.000000,0.000000,0.000000}%
\pgfsetstrokecolor{textcolor}%
\pgfsetfillcolor{textcolor}%
\pgftext[x=0.294444in, y=4.010726in, left, base]{\color{textcolor}\rmfamily\fontsize{10.000000}{12.000000}\selectfont \(\displaystyle {35}\)}%
\end{pgfscope}%
\begin{pgfscope}%
\definecolor{textcolor}{rgb}{0.000000,0.000000,0.000000}%
\pgfsetstrokecolor{textcolor}%
\pgfsetfillcolor{textcolor}%
\pgftext[x=0.238889in,y=2.363000in,,bottom,rotate=90.000000]{\color{textcolor}\rmfamily\fontsize{10.000000}{12.000000}\selectfont Position Y [\(\displaystyle m\)]}%
\end{pgfscope}%
\begin{pgfscope}%
\pgfpathrectangle{\pgfqpoint{0.530556in}{0.515000in}}{\pgfqpoint{4.960000in}{3.696000in}}%
\pgfusepath{clip}%
\pgfsetrectcap%
\pgfsetroundjoin%
\pgfsetlinewidth{1.505625pt}%
\definecolor{currentstroke}{rgb}{0.121569,0.466667,0.705882}%
\pgfsetstrokecolor{currentstroke}%
\pgfsetdash{}{0pt}%
\pgfpathmoveto{\pgfqpoint{1.244158in}{0.884903in}}%
\pgfpathlineto{\pgfqpoint{1.244158in}{1.247648in}}%
\pgfpathlineto{\pgfqpoint{1.606902in}{1.247648in}}%
\pgfpathlineto{\pgfqpoint{1.606902in}{0.884903in}}%
\pgfpathlineto{\pgfqpoint{1.244158in}{0.884903in}}%
\pgfpathlineto{\pgfqpoint{1.244158in}{1.610393in}}%
\pgfpathlineto{\pgfqpoint{1.969647in}{1.610393in}}%
\pgfpathlineto{\pgfqpoint{1.969647in}{0.884903in}}%
\pgfpathlineto{\pgfqpoint{1.244158in}{0.884903in}}%
\pgfpathlineto{\pgfqpoint{1.244158in}{1.973138in}}%
\pgfpathlineto{\pgfqpoint{2.332392in}{1.973138in}}%
\pgfpathlineto{\pgfqpoint{2.332392in}{0.884903in}}%
\pgfpathlineto{\pgfqpoint{1.244158in}{0.884903in}}%
\pgfpathlineto{\pgfqpoint{1.244158in}{2.335883in}}%
\pgfpathlineto{\pgfqpoint{2.695137in}{2.335883in}}%
\pgfpathlineto{\pgfqpoint{2.695137in}{0.884903in}}%
\pgfpathlineto{\pgfqpoint{1.244158in}{0.884903in}}%
\pgfpathlineto{\pgfqpoint{1.244158in}{2.698627in}}%
\pgfpathlineto{\pgfqpoint{3.057882in}{2.698627in}}%
\pgfpathlineto{\pgfqpoint{3.057882in}{0.884903in}}%
\pgfpathlineto{\pgfqpoint{1.244158in}{0.884903in}}%
\pgfpathlineto{\pgfqpoint{1.244158in}{3.061372in}}%
\pgfpathlineto{\pgfqpoint{3.420627in}{3.061372in}}%
\pgfpathlineto{\pgfqpoint{3.420627in}{0.884903in}}%
\pgfpathlineto{\pgfqpoint{1.244158in}{0.884903in}}%
\pgfpathlineto{\pgfqpoint{1.244158in}{3.424117in}}%
\pgfpathlineto{\pgfqpoint{3.783372in}{3.424117in}}%
\pgfpathlineto{\pgfqpoint{3.783372in}{0.884903in}}%
\pgfpathlineto{\pgfqpoint{1.244158in}{0.884903in}}%
\pgfusepath{stroke}%
\end{pgfscope}%
\begin{pgfscope}%
\pgfsetrectcap%
\pgfsetmiterjoin%
\pgfsetlinewidth{0.803000pt}%
\definecolor{currentstroke}{rgb}{0.000000,0.000000,0.000000}%
\pgfsetstrokecolor{currentstroke}%
\pgfsetdash{}{0pt}%
\pgfpathmoveto{\pgfqpoint{0.530556in}{0.515000in}}%
\pgfpathlineto{\pgfqpoint{0.530556in}{4.211000in}}%
\pgfusepath{stroke}%
\end{pgfscope}%
\begin{pgfscope}%
\pgfsetrectcap%
\pgfsetmiterjoin%
\pgfsetlinewidth{0.803000pt}%
\definecolor{currentstroke}{rgb}{0.000000,0.000000,0.000000}%
\pgfsetstrokecolor{currentstroke}%
\pgfsetdash{}{0pt}%
\pgfpathmoveto{\pgfqpoint{5.490556in}{0.515000in}}%
\pgfpathlineto{\pgfqpoint{5.490556in}{4.211000in}}%
\pgfusepath{stroke}%
\end{pgfscope}%
\begin{pgfscope}%
\pgfsetrectcap%
\pgfsetmiterjoin%
\pgfsetlinewidth{0.803000pt}%
\definecolor{currentstroke}{rgb}{0.000000,0.000000,0.000000}%
\pgfsetstrokecolor{currentstroke}%
\pgfsetdash{}{0pt}%
\pgfpathmoveto{\pgfqpoint{0.530556in}{0.515000in}}%
\pgfpathlineto{\pgfqpoint{5.490556in}{0.515000in}}%
\pgfusepath{stroke}%
\end{pgfscope}%
\begin{pgfscope}%
\pgfsetrectcap%
\pgfsetmiterjoin%
\pgfsetlinewidth{0.803000pt}%
\definecolor{currentstroke}{rgb}{0.000000,0.000000,0.000000}%
\pgfsetstrokecolor{currentstroke}%
\pgfsetdash{}{0pt}%
\pgfpathmoveto{\pgfqpoint{0.530556in}{4.211000in}}%
\pgfpathlineto{\pgfqpoint{5.490556in}{4.211000in}}%
\pgfusepath{stroke}%
\end{pgfscope}%
\begin{pgfscope}%
\pgfsetbuttcap%
\pgfsetmiterjoin%
\definecolor{currentfill}{rgb}{1.000000,1.000000,1.000000}%
\pgfsetfillcolor{currentfill}%
\pgfsetfillopacity{0.800000}%
\pgfsetlinewidth{1.003750pt}%
\definecolor{currentstroke}{rgb}{0.800000,0.800000,0.800000}%
\pgfsetstrokecolor{currentstroke}%
\pgfsetstrokeopacity{0.800000}%
\pgfsetdash{}{0pt}%
\pgfpathmoveto{\pgfqpoint{3.799444in}{2.148555in}}%
\pgfpathlineto{\pgfqpoint{5.393333in}{2.148555in}}%
\pgfpathquadraticcurveto{\pgfqpoint{5.421111in}{2.148555in}}{\pgfqpoint{5.421111in}{2.176333in}}%
\pgfpathlineto{\pgfqpoint{5.421111in}{2.549666in}}%
\pgfpathquadraticcurveto{\pgfqpoint{5.421111in}{2.577444in}}{\pgfqpoint{5.393333in}{2.577444in}}%
\pgfpathlineto{\pgfqpoint{3.799444in}{2.577444in}}%
\pgfpathquadraticcurveto{\pgfqpoint{3.771667in}{2.577444in}}{\pgfqpoint{3.771667in}{2.549666in}}%
\pgfpathlineto{\pgfqpoint{3.771667in}{2.176333in}}%
\pgfpathquadraticcurveto{\pgfqpoint{3.771667in}{2.148555in}}{\pgfqpoint{3.799444in}{2.148555in}}%
\pgfpathclose%
\pgfusepath{stroke,fill}%
\end{pgfscope}%
\begin{pgfscope}%
\pgfsetrectcap%
\pgfsetroundjoin%
\pgfsetlinewidth{1.505625pt}%
\definecolor{currentstroke}{rgb}{0.121569,0.466667,0.705882}%
\pgfsetstrokecolor{currentstroke}%
\pgfsetdash{}{0pt}%
\pgfpathmoveto{\pgfqpoint{3.827222in}{2.473277in}}%
\pgfpathlineto{\pgfqpoint{4.105000in}{2.473277in}}%
\pgfusepath{stroke}%
\end{pgfscope}%
\begin{pgfscope}%
\definecolor{textcolor}{rgb}{0.000000,0.000000,0.000000}%
\pgfsetstrokecolor{textcolor}%
\pgfsetfillcolor{textcolor}%
\pgftext[x=4.216111in,y=2.424666in,left,base]{\color{textcolor}\rmfamily\fontsize{10.000000}{12.000000}\selectfont Ground truth}%
\end{pgfscope}%
\begin{pgfscope}%
\pgfsetbuttcap%
\pgfsetroundjoin%
\definecolor{currentfill}{rgb}{0.121569,0.466667,0.705882}%
\pgfsetfillcolor{currentfill}%
\pgfsetlinewidth{1.003750pt}%
\definecolor{currentstroke}{rgb}{0.121569,0.466667,0.705882}%
\pgfsetstrokecolor{currentstroke}%
\pgfsetdash{}{0pt}%
\pgfsys@defobject{currentmarker}{\pgfqpoint{-0.041667in}{-0.041667in}}{\pgfqpoint{0.041667in}{0.041667in}}{%
\pgfpathmoveto{\pgfqpoint{0.000000in}{-0.041667in}}%
\pgfpathcurveto{\pgfqpoint{0.011050in}{-0.041667in}}{\pgfqpoint{0.021649in}{-0.037276in}}{\pgfqpoint{0.029463in}{-0.029463in}}%
\pgfpathcurveto{\pgfqpoint{0.037276in}{-0.021649in}}{\pgfqpoint{0.041667in}{-0.011050in}}{\pgfqpoint{0.041667in}{0.000000in}}%
\pgfpathcurveto{\pgfqpoint{0.041667in}{0.011050in}}{\pgfqpoint{0.037276in}{0.021649in}}{\pgfqpoint{0.029463in}{0.029463in}}%
\pgfpathcurveto{\pgfqpoint{0.021649in}{0.037276in}}{\pgfqpoint{0.011050in}{0.041667in}}{\pgfqpoint{0.000000in}{0.041667in}}%
\pgfpathcurveto{\pgfqpoint{-0.011050in}{0.041667in}}{\pgfqpoint{-0.021649in}{0.037276in}}{\pgfqpoint{-0.029463in}{0.029463in}}%
\pgfpathcurveto{\pgfqpoint{-0.037276in}{0.021649in}}{\pgfqpoint{-0.041667in}{0.011050in}}{\pgfqpoint{-0.041667in}{0.000000in}}%
\pgfpathcurveto{\pgfqpoint{-0.041667in}{-0.011050in}}{\pgfqpoint{-0.037276in}{-0.021649in}}{\pgfqpoint{-0.029463in}{-0.029463in}}%
\pgfpathcurveto{\pgfqpoint{-0.021649in}{-0.037276in}}{\pgfqpoint{-0.011050in}{-0.041667in}}{\pgfqpoint{0.000000in}{-0.041667in}}%
\pgfpathclose%
\pgfusepath{stroke,fill}%
}%
\begin{pgfscope}%
\pgfsys@transformshift{3.966111in}{2.267513in}%
\pgfsys@useobject{currentmarker}{}%
\end{pgfscope}%
\end{pgfscope}%
\begin{pgfscope}%
\definecolor{textcolor}{rgb}{0.000000,0.000000,0.000000}%
\pgfsetstrokecolor{textcolor}%
\pgfsetfillcolor{textcolor}%
\pgftext[x=4.216111in,y=2.231055in,left,base]{\color{textcolor}\rmfamily\fontsize{10.000000}{12.000000}\selectfont Estimated position}%
\end{pgfscope}%
\end{pgfpicture}%
\makeatother%
\endgroup%
}
% % % % %         \caption{INS Hardware}
% % % % %         \label{fig:sub1}
% % % % %     \end{subfigure}
% % % % %     \begin{subfigure}{0.49\textwidth}
% % % % %         \centering
% % % % %         \resizebox{1\linewidth}{!}{\input{plots/square/squares3D.pgf}}
% % % % %         \caption{MPU-9250 Breakout}
% % % % %         \label{fig:sub2}
% % % % %     \end{subfigure}
% % % % %     \caption{Position estimation by the best performing algorithms in the 4-meter line experiment.}
% % % % %     \label{fig:mag_calibration_output}
% % % % % \end{figure}

% % % % \begin{figure}[!h]
% % % %     \centering
% % % %     \begin{table}[H]
    \begin{center}
        \resizebox{1\linewidth}{!}{

            \begin{tabular}[t]{lcccc}
                \hline
                Algorithm        & Displacement Error[$m$] & Displacement Error[\%] & Turn Error[$m$] & Turn Error[\%] \\
                \hline
                AngularRate      & 32.91                   & 22.24                  & 51.46           & 34.77          \\
                \acrshort{aqua}  & 7.08                    & 4.78                   & 8.32            & 5.62           \\
                Complementary    & 11.63                   & 7.86                   & 11.34           & 7.66           \\
                Davenport        & 6.17                    & 4.17                   & 7.21            & 4.87           \\
                \acrshort{ekf}   & 7.29                    & 4.93                   & 8.33            & 5.63           \\
                \acrshort{famc}  & 31.83                   & 21.51                  & 38.19           & 25.81          \\
                \acrshort{flae}  & 7.61                    & 5.14                   & 9.16            & 6.19           \\
                Fourati          & 54.90                   & 37.09                  & 73.30           & 49.53          \\
                Madgwick         & 6.53                    & 4.42                   & 7.53            & 5.09           \\
                Mahony           & 6.18                    & 4.17                   & 7.18            & 4.85           \\
                \acrshort{oleq}  & 6.43                    & 4.35                   & 7.00            & 4.73           \\
                \acrshort{quest} & 12.57                   & 8.49                   & 25.22           & 17.04          \\
                \acrshort{roleq} & 6.53                    & 4.41                   & 7.06            & 4.77           \\
                \acrshort{saam}  & 6.11                    & 4.13                   & 7.23            & 4.88           \\
                Tilt             & 6.11                    & 4.13                   & 7.23            & 4.88           \\

                \hline
                Average          & 13.99                   & 9.45                   & 18.38           & 12.42
            \end{tabular}
        }
        \caption{Squares position estimation error (displacement and turn) of the sensor fusion algorithms. }
        \label{tab:squares}
    \end{center}
\end{table}
% % % % \end{figure}

% % % % \subsection{Spiral}

% % % % \begin{figure}[!h]
% % % %     \centering
% % % %     \begin{table}[H]
    \begin{center}
    \resizebox{1\linewidth}{!}{

        \begin{tabular}[t]{lcccc}
            \hline
            Algorithm                   & Displacement Error[$m$] & Displacement Error[\%]      & Turn Error[$m$]  & Turn Error[\%]             \\
            \hline 
            AngularRate            & 39.61  & 35.36 & 52.75 & 47.10              \\            AQUA            & 3.41  & 3.05 & 5.92 & 5.29              \\            Complementary            & 4.11  & 3.67 & 6.75 & 6.03              \\            Davenport            & 2.18  & 1.95 & 5.59 & 4.99              \\            EKF            & 2.62  & 2.34 & 5.35 & 4.78              \\            FAMC            & 39.10  & 34.91 & 48.27 & 43.09              \\            FLAE            & 3.29  & 2.94 & 5.07 & 4.53              \\            Fourati            & 42.44  & 37.89 & 54.37 & 48.55              \\            Madgwick            & 2.35  & 2.10 & 5.00 & 4.46              \\            Mahony            & 2.25  & 2.01 & 5.55 & 4.95              \\            OLEQ            & 2.41  & 2.16 & 5.81 & 5.19              \\            QUEST            & 17.84  & 15.92 & 36.06 & 32.19              \\            ROLEQ            & 2.48  & 2.21 & 5.65 & 5.04              \\            SAAM            & 2.26  & 2.02 & 5.54 & 4.95              \\            Tilt            & 2.26  & 2.02 & 5.54 & 4.95              \\
            \hline
            Average & 11.24 & 10.04 & 16.88 & 15.07
        \end{tabular}
        }
        \caption{Accelerometer Specifications. }
        \label{tab:accelerometer_specification}
    \end{center}
\end{table}
% % % % \end{figure}
